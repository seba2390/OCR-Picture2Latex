%% rnaastex.cls is the classfile used for Research Notes. It is derived
%% from aastex61.cls with a few tweaks to allow for the unique format required.
%% (10/15/17)
%%\documentclass{rnaastex}

%% Better is to use the "RNAAS" style option in AASTeX v6.2
%% (01/08/18)
%\documentclass[manuscript]{aastex62}
%\documentclass[RNAAS]{aastex62}
\documentclass[pdftex,twocolumn]{aastex62}
\usepackage{amssymb,amsmath}
\usepackage{color}

%% Define new commands here
\newcommand\latex{La\TeX}
\newcommand\aastex{AAS\TeX}

%% Tells LaTeX to search for image files in the 
%% current directory as well as in the figures/ folder.
\graphicspath{{./}{figures/}}

\def\degs{\ifmmode ^{\circ}\else$^{\circ}$\fi}
\def\amin{\ifmmode ^{\prime}\else$^{\prime}$\fi}
\def\asec{\ifmmode ^{\prime\prime}\else$^{\prime\prime}$\fi}
\def\h{$^{\rm h}$}
\def\m{$^{\rm m}$}
\def\s{$^{\rm s}$}

\newcommand{\rr}[1]{\textcolor{red}{(\bf #1)}}
\usepackage[normalem]{ulem}  % \sout{old text} for strikeout
\usepackage{color,soul}
\newcommand{\old}[1]{{\color[rgb]{0.7,0,0.7}\sout{#1}}}

\newcommand{\va}[1]{{\color[rgb]{1.0,0.0,0.0}VA: #1}}
\newcommand{\yus}[1]{{\color[rgb]{0.5,0.1,0.5}YS: #1}}
\newcommand{\gp}[1]{{\color{blue} #1}}
%\newcommand{\rmig}[1]{{\color[rgb]{0.1,0.6,0.3}DAZ: #1}}
%\newcommand{\vdag}{(v)^\dagger}

\usepackage{hyperref}
\usepackage{url}
\usepackage{natbib}

\begin{document}

\title{HUBBLE SPACE TELESCOPE OBSERVATIONS OF THE OLD PULSAR PSR J0108--1431\footnote{Based on observations made with the NASA/ESA {\sl Hubble Space Telescope}, obtained at the Space Telescope Science Institute, which is operated by the Association of Universities for Research in Astronomy, Inc., under NASA contract NAS 5-26555. These observations are associated with program \#14249.}
}

%% Note that the corresponding author command and emails has to come
%% before everything else. Also place all the emails in the \email
%% command instead of using multiple \email calls.
\correspondingauthor{Vadim Abramkin}
\email{vadab94@gmail.com}

\author{Vadim Abramkin}
\author{Yuriy Shibanov}
%\altaffiliation{Editor-in-Chief}
\affiliation{Ioffe Institute, Politekhnicheskaya 26, St.\ Petersburg, 194021, Russia}
%\author{Yuriy Shibanov}
%\altaffiliation{RNAAS Editor}
%\affiliation{Ioffe Istitute ...}
%% The \author command can take an optional ORCID.
%\author[0000-0002-0786-7307]{Third Author  }
\author{Roberto P.\ Mignani}
\affiliation{INAF -- Istituto di Astrofisica Spaziale e Fisica Cosmica Milano, via E. Bassini 15, I-20133, Milano, Italy}
\affiliation{Janusz Gil Institute of Astronomy, University of Zielona G\'{o}ra, ul Szafrana 2, 65-265, Zielona G\'{o}ra, Poland}
\author{George G.\ Pavlov}
\affiliation{The Pennsylvania State University, Department of Astronomy \& Astrophysics, 525 Davey Lab., University Park, PA 16802, USA}

%% Note that RNAAS manuscripts DO NOT have abstracts.
%% See the online documentation for the full list of available subject
%% keywords and the rules for their use.
\keywords{pulsars: individual (PSR\,J0108--1431) --- stars: neutron --- ultraviolet: stars}
%--- 
%miscellaneous --- HST observations --- optical-UV}

\begin{abstract}
We present results of optical-UV observations of the 200 Myr old rotation-powered radio 
pulsar J0108$-$1431 with the 
{\sl Hubble Space Telescope}.
We found a putative candidate for the far-UV (FUV) pulsar counterpart, with the flux density $f_\nu = %15\pm 5$ 
9.0\pm 3.2$~nJy at $\lambda = 1528$ \AA.
The pulsar was not detected, however,
at longer wavelengths,
with  $3\sigma$ upper limits of 52, 37, and 87 nJy 
at $\lambda = 4326$, 3355, and 2366 \AA, respectively. 
%We also reanalyzed an observation of the pulsar with the Very Large Telescope (VLT; published by R.\ Mignani and coauthors in 2003 and 2008) and estimated 
%$f_\nu = 44\pm 22$ and $30\pm 14$ in the $U$ and $B$ bands, respectively, and a $3\sigma$ upper limit of 36 nJy in the $V$ band.
Assuming that the pulsar counterpart was indeed detected in FUV, and the previously reported marginal $U$ and $B$ detections with the Very Large Telescope were real, 
%we found that 
the optical-UV spectrum of the
pulsar can be described by a power-law model
%, $f_\nu\propto \nu^\alpha$, with a slope $-0.7 \lesssim \alpha\lesssim 0.5$\gp{right?} 
with a nearly flat $f_\nu$ spectrum. Similar to younger pulsars detected in the optical, the slope of the nonthermal spectrum steepens in the X-ray range. The pulsar's luminosity in the 1500--6000 \AA\ wavelength range, $L \sim 1.2\times 10^{27} (d/210\,{\rm pc})^2$ erg s$^{-1}$, corresponds to a high %optical-UV 
efficiency of conversion of pulsar rotation energy loss rate $\dot {E}$ to the optical-UV radiation, $\eta = L/\dot{E} \sim 
%2.6
%3
%(2$
(1$--$6)\times 10^{-4}$, depending on somewhat uncertain values of distance and spectral slope.
%(d/210\,{\rm pc})^2$.
%\gp{correct efficiencies in the text}
The brightness temperature of the bulk neutron star surface does not exceed 59,000 K ($3\sigma$ upper bound), as seen by a distant observer. If we assume that the FUV flux is dominated by a thermal component,
then the surface temperature can be in the range of 27,000--55,000 K, requiring a heating mechanism to operate in old neutron stars.
%\gp{The abstract to be compressed.}

\end{abstract}

%% Start the main body of the article. If no sections in the 
%% research note leave the \section call blank to make the title.
%%%%%%%%%%%%%%%%%%%%%%%%%%%%%%%%%%%%%%%%%%%%%%%%%%%%%%%%%%%%%%%%%%%%%
\section{Introduction}
\label{intro}
%%%%%%%%%%%%%%%%%%%%%%%%%%%%%%%%%%%%%%%%%%%
%\gp{Please change the refs to bibtex ones.} 

Optical and ultraviolet (UV) observations of 
%$\ga 1$~Myr 
old  
rotation-powered 
pulsars (ages $\gtrsim 1$ Myr), 
supplemented by X-ray observations,
%are considered as
are  important to understand advanced stages of  thermal evolution  
of neutron stars (NSs) and study  non-thermal  
emission processes in their magnetospheres. 
So far, only a handful of such pulsars
%old radio pulsars 
have been detected in both X-rays and  optical-UV. 
These are 
%{\bf two ordinary pulsars,} 
the 3 Myr old PSR B1929+10 \citep{1996pavlov,2002mlcw} and 
%(Pavlov et al.\ 1996; Mignani et al.\ 2002), 
the 17 Myr old PSR B0950+08 \citep{1996pavlov,Pavlov2017}, 
%(Pavlov et al.\ 1996; 2017), 
%the 3.8 and 4.9 Gyr-old 
and two a few Gyr old  recycled  millisecond pulsars, PSR J2124$-$3358  \citep{2017rangelov} 
%(Rangelov et al.\ 2017) 
and PSR J0437$-$4715   \citep{2004karg,2012durant},    
%(Kargaltsev et al.\ 2004; Durant et al.\ 2012), 
all are identified in the UV-optical with the {\sl Hubble Space Telescope} ({\sl HST}). 
%{\bf \sout{The first millisecond pulsar}
%PSR J2144--3358 is an isolated NS while 
%\sout{the second one}
%PSR J0437--4715 is in a relatively wide binary system with an orbital period of 5.7~days and 
%a very cool, $\sim 4000$~K, white dwarf companion, which cannot contaminate the pulsar flux  in the UV.} 
%Mignani et al.\ (2019) gave an update on the UV identification of $\lesssim 1$ Myr-old pulsars.
%\gp{[I would remove the boldfaced text. No need to go into these details in the paper, it is only needed to educate the referee in the response.]}
For both PSR B1929+10 and PSR J2124$-$3358, the spectral data are insufficient to determine the nature of the optical-UV emission, whereas the others 
%feature a clear 
show a Rayleigh-Jeans (R-J) continuum, with an additional power-law (PL) component in PSR B0950+08. 
%Interestingly, in 
In both cases, the inferred temperatures of $\sim 10^5$ K, higher than predicted by NS cooling models 
\citep[e.g.,][]{2004yp}, 
%(e.g., Yakovlev \& Pethicj 2004), 
suggest that some re-heating mechanisms operate in the NS interior. A candidate optical counterpart to the 5 Myr old PSR B1133+16 was found with the Very Large Telescope (VLT), but the identification is still uncertain 
\citep{2008zhar,2013zm}.
%(Zharikov et al.\ 2008; Zharikov \& Mignani 2013).

Another old pulsar with a yet unconfirmed optical counterpart is PSR J0108$-$1431. %\citep{Mignani2008}.  
%(Mignani et al.\ 2008). 
This pulsar was discovered by \citet{1994tauris} %Tauris et al.\ (1994) 
%during 
in the Parkes Southern Pulsar Survey \citep{1996manch}.  
%(Manchester et al.\ 1996). 
Its spin period $P = 0.808$ s and period derivative
%\footnote{C
%(corrected for Shklovskii effect)
$\dot{P} = 
%7.44
6.51\times  10^{-17}$ s s$^{-1}$ 
(corrected for the Shklovskii effect), imply a rotational energy loss rate $\dot{E} = 5.1\times 10^{30}$ erg s$^{-1}$ 
and surface magnetic field $B_s = 2.3\times 10^{11}$ G. With the characteristic age of 
%166 
196 Myr, PSR J0108$-$1431 is one of the oldest non-recycled isolated radio pulsars known to date. 
%In particular, it 
It lies close to the so-called ``graveyard'' region in the pulsar $P$-$\dot{P}$ diagram, and it is among the faintest radio pulsars, with a 400 MHz luminosity of 0.391 mJy kpc$^2$ for a distance of $210^{+90}_{-50}$ pc, obtained from the Very Large Baseline Interferometer (VLBI) radio parallax \citep{Deller2009}, 
%(Deller et al.\ 2009), 
corrected for the Lutz-Kelker bias \citep{2012verbi}. 
%(Verbiest et al. 2012). 

In the first deep optical observation of the field of PSR J0108--1431 with the VLT,
\citet{Mignani2003} 
%Mignani et al.\ (2003) 
%performed 
noticed a faint brightness enhancement in the $U$ image within
%reported $3\sigma$ upper limits of $m_V\simeq 28$, $m_B\simeq 28.6$, and $m_U=26.4$  
%the first deep optical observations of PSR J0108$-$1431 with the VLT.
%and found a faint object ($m_U = 26.4 \pm 0.3$, $m_B = 27.9\pm 0.5$) compatible with 
 the error ellipse of the Australian Telescope Compact Array (ATCA) radio position, 
 %at the edge of the image of 
%an elliptical 
projected near the edge of an elongated background galaxy. However, they concluded that most likely it was not a real detection and reported only upper limits in the $V$, $B$ and $U$ filters.

The pulsar position measured by the {\sl Chandra X-ray Observatory} 
by \citet{2009pavlov}
% (Pavlov et al.\ 2009) 
 implied a significant proper motion, which, extrapolated to the epoch of the VLT observations, matched the position of the 
%object detected 
enhancement noticed by  \citet{Mignani2003}. It prompted \citet{Mignani2008}
%Mignani et al.\ (2008) 
to propose that the pulsar counterpart had probably been detected with the VLT, with magnitudes $U=26.4\pm 0.3$, $B=27.9\pm 0.5$, $V > 27.8$.
The improved VLBI proper motion,  
%($169 \pm 1.7$  
$170.0\pm 1.7$ mas yr$^{-1}$ \citep{Deller2009}, made the proposed identification 
%(Mignani et al.\ 2008) 
more robust, with a chance positional coincidence probability of $\sim 3\times 10^{-4}$  \citep{Mignani2011}.  
%(Mignani et al.\ 2011). 
The optical spectrum of the PSR J0108--1431 candidate counterpart is poorly defined, although the $U$ and $B$ fluxes are compatible with a $\sim 2.3 \times 10^5$ K R-J spectrum, for the NS radius of 13 km and 210 pc distance. 

The PSR J0108$-$1431 X-ray identification with {\sl Chandra} by 
\citet{2009pavlov} %{%Pavlov et al.\ 2009) 
has been confirmed by the detection of X-ray pulsations with {\sl XMM-Newton} \citep{Posselt2012}. 
%(Posselt et al. 2012). 
The X-ray spectrum is best fitted by a PL with a (fixed) photon index $\Gamma  = 2$ and a blackbody (BB) with temperature 
%$kT = 0.11^{+0.03}_{-0.01}$ keV 
1.28$^{+0.35}_{-0.12}\times 10^6$~K and effective radius 
%of the emitting area of
$R= 43^{+16}_{-9} d_{210}$ m, where $d_{210}$ is the pulsar distance in units of 210 pc, with a hydrogen column density $N_H=2.3^{+2.4}_{-2.2}\times 10^{20}$ cm$^{-2}$.

The proposed optical identification of PSR J0108--1431, however, has never been confirmed. Follow-up VLT observations in 2009 with about two times larger total exposures were not conclusive 
because of an almost twice worse seeing  of 0\farcs8 -- 1\farcs0 \citep{Mignani2011},  
as compared to  $\approx$0\farcs5 
in previous  observations in 2000 %non-optimal sky conditions 
 \citep{Mignani2003}.  The counterpart was not detected at the expected new position  of the pulsar accounting for its proper motion, while its  brightness limits,  $U\ga 26.5$, $B\ga 27.2$, were consistent with the tentative detection reported  by \citet{Mignani2008}. 
%(Mignani et al.\ 2011). 
%PSR J0108$−$1431 has never been re-observed in the optical since then2. Therefore, 
To verify the putative VLT identification, measure the optical-UV spectrum of this pulsar, and constrain the surface temperature of the very old NS, 
we carried out new observations with the {\sl HST}. We describe the {\sl HST} observations in Section 2 and astrometry of the {\sl HST} images in Section 3.
Photometry of the candidate pulsar counterpart in the {\sl HST} and VLT data is reported in Section 4.
In Section 5 we discuss spectral fits of the optical-UV data, compare them with the X-ray spectrum, and discuss constraints on the NS surface temperature. Conclusions from our analysis are presented in Section 6.
%to confirm the pulsar identification and characterize its spectrum in the optical-UV. This paper is divided as follows: observations and data analysis are described in Section 2, with the results presented and discussed in Sections 3 and 4, respectively.

%%%%%%%%%%%%%%%%%%%%%%%%%%%%%%%%%%%%%%%%%%%%%%%%%%%%%%%%%%%%%%%
\section{Observations}
\label{obs}
%%%%%%%%%%%%%%%%%%%%%%%%%%%%%%%%%%%%%%%%%%%%%%%%%%%%%%%%%%%%%%%%%%%%
%%%%%%%%%%%%%%%%%%%%%% Table 1 $$$$$$$$$$$$$$$$$$$$$$$$$$$$
\begin{deluxetable*}{cccrcc}[t]
\tablecaption{{\sl HST} observations of PSR J0108--1431  \label{tab:obs}}
\tablecolumns{6}
\tablenum{1}
\tablewidth{0pt}
\tablehead{
\colhead{Start time\tablenotemark{a}} &
\colhead{Instrument} &
\colhead{Filter} & 
\colhead{$\lambda_p$\tablenotemark{b} } &
\colhead{$\Delta\lambda$\tablenotemark{c} } &
\colhead{Exposure}\\
%\tablenotemark{c}} \\
\colhead{
%(yyyy-mm-dd hh:mm:ss)
} & \colhead{} &
\colhead{} &
\colhead{ (\AA)} &\colhead{ (\AA)} &\colhead{(s)} 
}
\startdata 
2016-08-08 15:00:47 & WFC3/UVIS & F225W & 2359 & 500 &2472 \\
2016-08-08 16:33:47 & WFC3/UVIS & F225W & 2359 & 500 &2460 \\
2016-08-08 18:09:13 & WFC3/UVIS & F336W & 3355 & 550 &2580 \\
2016-08-08 19:44:39 & WFC3/UVIS & F336W & 3355 & 550  &2580 \\
2016-08-08 21:20:04 & WFC3/UVIS & F438W & 4325 & 695 &2580 \\
2016-08-08 22:55:30 & WFC3/UVIS & F438W & 4325 & 695 &2580 \\
2016-08-11 22:32:15 & ACS/SBC & F140LP & 1528 & 294 &2800 \\ 
\enddata 
\tablecomments{Each line corresponds to one {\sl HST} orbit.}
\tablenotetext{a}{ UT start time corresponds to the start of first exposure for the orbit.}
%{Each line 
%is related to a specific 
%corresponds to one {\sl HST}  orbit, 
%and the start time 
%coincides with the start of  
%the first exposure in the orbit.}
\tablenotetext{b}{Pivot wavelength of the filter.}
\tablenotetext{c}{Bandwidth (FWHM) of the filter.}
%\tablenotetext{c}{Exposure time
%per the {\sl HST} orbit.}
\end{deluxetable*}
%%%%%%%%%%%%%%%%%%%%%%%%%%%%%%%%%%%%%%%%%
%%%%%%%%%%%%%%%%%%%%%%%%%%%%%%%%%%%%%%%5
%%%%%%%%%%%%%%%%%%%%%%%% Table 2new %%%%%%%%%%%%%%%%%%%%%%%%%%%%%%%
\begin{deluxetable*}{ccccccc}[t]
\tablecaption{{\sl Gaia} DR2 positions and p.m.\ components of the reference objects  
marked in Figure~\ref{fig:1}
\label{tab:ref-stars}}
\tablecolumns{7}
\tablenum{2}
%\tablewidth{Opt}
\tablehead{
\colhead{Object\tablenotemark} &
\colhead{RA (J2000)\tablenotemark{a}} &
\colhead{Decl (J2000)\tablenotemark{a}} & \colhead{RA-err} & \colhead{Decl-err} & 
\colhead{$\mu_{\alpha}$\tablenotemark{b}} & \colhead{$\mu_{\delta}$}\\
\colhead{} & \colhead{} & \colhead{} & \colhead{mas} & \colhead{mas} 
& \colhead{mas yr$^{-1}$} & \colhead{mas yr$^{-1}$}
}
\startdata
1 & 01\h08\m09\fs001 & $-$14\degs31\amin32\farcs121 & 0.70 & 0.55 &
+0.6 $\pm$ 1.9 & $-$5.3 $\pm$ 1.2\\
2 & 01\h08\m09\fs181 & $-$14\degs32\amin04\farcs753 & 2.0 & 1.6 &  
\nodata & \nodata \\
3 & 01\h08\m07\fs706 & $-$14\degs32\amin01\farcs488  & 0.15 & 0.12 &  
+5.09 $\pm$ 0.42 & $-$4.12 $\pm$ 0.26\\ 
4 &  01\h08\m15\fs742 & $-$14\degs32\amin33\farcs613 & 0.23 & 0.18 &  
$-$6.56 $\pm$ 0.65 & $-$30.61 $\pm$ 0.41\\
5 & 01\h08\m14\fs793 & $-$14\degs32\amin49\farcs463 & 0.30 & 0.23 &  
+46.72 $\pm$ 0.82 & $-$4.43 $\pm$ 0.52\\
\enddata 
\tablenotetext{a}{The {\sl Gaia} position epoch is 2015.5 (= MJD 57204).} 
\tablenotetext{b}{
$\mu_{\alpha}$ is defined as $\dot{\alpha}\,\cos\delta$, where $\alpha$ and $\delta$ are RA and Decl, respectively.}
\end{deluxetable*}
%%%%%%%%%%%%%%%%%%%%%%%%%%%%%%%%%%%%%%%%%%%%

The PSR J0108--1431 field was observed with the {\sl HST} in 2016 August 
(program \#14249, PI Mignani) using the Ultraviolet-Visible (UVIS) channel of the Wide Field 
Camera 3 (WFC3; 6 {\sl HST} orbits) 
and the Solar Blind Channel (SBC) of the Advanced Camera for Surveys (ACS; 2 {\sl HST} orbits). 
The WFC3/UVIS imaging was carried out with the F438W, F336W and F225W broad-band filters,
while the F140LP long-pass filter was used with the ACS/SBC. 
The log of the observations and the pivot wavelengths of the filters are presented in 
Table~\ref{tab:obs}. 
For each filter, the total integration time  was split 
into shorter exposures distributed over two {\sl HST} 
orbits. 
The WFC3/UVIS exposures were taken in the ACCUM mode, applying a four-point box dither pattern 
in each orbit. 
The UVIS2-C1K1C-CTE aperture was used to place the
pulsar close to a readout amplifier and minimize the CCD charge
transfer efficiency (CTE) losses, as advised in the WFC3
Instrument Handbook\footnote{
See \url{https://hst-docs.stsci.edu/display/WFC3IHB.}}. The data were reduced
and flux-calibrated through the CALWF3 pipeline, which 
also applies the image de-dithering, geometric distortion and CTE corrections,
cosmic-ray filtering, and stacking. For the ACS/SBC,  a single exposure  was taken in the ACCUM 
mode during each orbit. 
The data were processed through the CALACS  pipeline  including the flux-calibration and  
geometric distortion correction. The SBC images are not affected by
cosmic rays and CTE. 
The data for two {\sl HST} orbits obtained for each filter (see Table~\ref{tab:obs}) 
were combined to produce resulting images. However, the aperture door was closed during second ACS/SBC orbit
(2900 s) 
because the Fine Guidance Sensors failed to acquire the guide stars.
Since this orbit provided no science data,
only the first ASC/SBC orbit is included in Table~\ref{tab:obs}.  
  
%%%%%%%%%%%%%%%%%%%%%%%%%%%%%%%%%%%%%%%%%%%%%%%
\section{Astrometry}
\label{astrom}
%%%%%%%%%%%%%%%%%%%%%%%%%%%%%%%%%%%%%%%%%%%%%%%%%%%%%%%%%%%% 
Precise  astrometric referencing is crucial for searching the pulsar counterpart by its positional coincidence 
with the radio pulsar.  
We used the {\sl Gaia} DR2 Catalog \citep{Lindegren2018} and the IRAF tasks {\tt imcentroid} and {\tt ccmap }
to obtain astrometric solutions.   
Five catalogued objects fall within the UVIS field of view (FoV) of about $162\asec \times 162\asec$ %with the 
(pixel scale 
%0\farcs 0.0396 
of 39.6 mas was chosen in drizzling). %  \gp{Are you sure in this value? I thought that usual a round number, like 40 mas, is chosen in drizzling.}).
%, which is significantly larger than the $\approx 31\asec\times 34\asec$ SBC FoV with the pixel scale after  
%drizzling of 0\farcs025\footnote{For the UVIS and SBC, the original pixel scales are 
%about $40\times 40$  and $31\times 34$ mas, respectively.}. 
These objects are best detected in the F438W image  which we use
as a primary reference image for SBC astrometry
(SBC FoV $\approx 31 \asec \times 34 \asec$; pixel scale 25 mas after drizzling\footnote{For the UVIS and SBC, the original pixel scales are 
about $40\times 40$  and $31\times 34$ mas, respectively.}). 
The {\sl Gaia} objects are marked and numerated 
in this image shown 
in Figure~\ref{fig:1}; 
their coordinates and proper motion (p.m.) components $\mu_{\alpha}$ and $\mu_{\delta}$  
are listed in  
Table~\ref{tab:ref-stars}.
%Objects 1, 2, and 3 are located within $25''$ from the pulsar position. 
Object 2 looks like an elliptical galaxy and shows no p.m. 
%Two other objects are stars with p.m.\ of about 5 and 7 mas yr$^{-1}$.
%Objects 4 and 5, imaged on the detector chip that does not contain the pulsar,
%have relatively large p.m., about 31 and 47 mas yr$^{-1}$, respectively. 

We used the coordinates and p.m.\ values for the four stars in Table~\ref{tab:ref-stars} to calculate their coordinates at the epoch of the UVIS observations, MJD 57608, for further astrometry. To increase the number of reference objects, we also used the galaxy (object 2). Thanks to its regular (elliptical) shape, the uncertainty of its position in the F438W image is reasonably small, about $2.4$ mas.
%comparable to the typical {\sl Gaia} DR2 Catalog uncertainty of $\la 2$~mas,  including the objects with $G$ magnitudes as large as $\approx 21$\footnote{See the {\sl Gaia} data release documentation:  \url{https://gea.esac.esa.int/archive/documentation/GDR2}.}. 
%The centroiding uncertainties for the stellar objects  are  even smaller, 0.2--0.9 mas. 
The total (stars plus galaxy) root-mean-square ({\sl rms}) centroid radial uncertainty is 1.9 mas. According to Table~\ref{tab:ref-stars}, the {\tt rms} of the {\sl Gaia} radial uncertainty of the  reference objects is 1.3 mas. The p.m.\ corrections lead to an additional radial uncertainty of 1.3 mas.
%resulting in the total {\tt rms} of 1.8 mas at the epoch of the UVIS observations. 
%It is comparable with the centroiding  {\tt rms}. 
Performing the astrometric fit with the {\tt ccmap}, 
%which include the frame shift, rotation, and scale factor, 
we obtained formal {\tt rms} 
residuals of 0.7 mas for the right ascension (RA) 
and 2.3 mas for the declination (Decl). 
An additional uncertainty of about 2 mas is associated with the 
WFC3/UVIS geometric distortion correction \citep{Kozhurina-Platais2015}. 
As a result, combining all the uncertainties in quadrature,  we  got the final radial uncertainty of 4 mas (0.1 pixels) for astrometric referencing of the F438W image.
%%%%%%%%%%%%%%%%%%%%%%% Figure 1 %%%%%%%%%%%%%%%%%%%%%%%%%%%%%%%%
\begin{figure}[ht]
\begin{center}
\includegraphics[scale=0.263,angle=0]
{UVIS-Astrometry-8.jpeg} 
\caption{UVIS F438W image of the pulsar field. The image 
is binned by a factor of 8 (new pixel size is 0\farcs32) 
and smoothed with a 2 pixels Gaussian kernel. 
%The white stripe in the middle of the image corresponds to the gap between the two CCD chips. 
%The colorbar decodes  the image 
%indicates the range of intensity values in counts ks$^{-1}$ per pixel. 
Red numbered circles mark the five  objects from 
the {\sl Gaia} DR2 Catalog listed in Table~\ref{tab:ref-stars} and 
used for astrometry. The red box 
shows the SBC FoV presented in Figure~\ref{fig:2}. 
%a $10''\times 10''$ region where the pulsar is located.
%it is expanded in Figure 3d.  
\label{fig:1}}
\end{center}
\end{figure}
%%%%%%%%%%%%%%%%%%%%%%%%%%%%%%%%%%%%%%%%%%%%%%%%
%%%%%%%%%%%%%%%%%%%%%%%%%%%Fig 2%%%%%%%%%%%%%%%%%%%%%%%%%%
\begin{figure}[t!]
\begin{center}
\includegraphics[scale=0.29,angle=0]{SBC-Astrometry-20.jpeg}
\caption{SBC F140LP image of the pulsar field. The image is 
binned by a factor of 2 (new pixel size 0\farcs05) and 
smoothed with the 3 pixels Gaussian kernel. 
%Power scale was used for better contrast. 
The arrow shows the pulsar counterpart candidate.
White numbered circles mark four objects  which, together with a bright spiral galaxy seen edge-on north-west  
of the candidate,  were used to align the SBC image to the UVIS F438W image. Red contours  of 
the spiral galaxy 
are overlaid from the F438W image to demonstrate robustness of the alignment.
%The arrow show the pulsar counterpart candidate.  
%A bright spiral galaxy seen edge-on near the center was also used to check the alignment. 
%Only objects  2 and 3  are presented in the {\sl Gaia} DR2 catalog: the former is an elliptical galaxy and the latter is a star. 
%The white box shows the $10''\times 10''$ region where the pulsar is located.   
%it is expanded in Figure 3a. The colour-bar indicates the range of intensity 
%values in counts ks$^{-1}$ per pixel. 
\label{fig:2}} 
\end{center}
\end{figure} 
%%%%%%%%%%%%%%%%%%%%%%%%%%%%%%%%%%%%%%
%%%%%%%%%%%%%%%%%%
     
Using the same reference objects for 
the F336W image yields 
%the {\sl rms} fit residuals of $0.8$ mas for RA and $3.7$ mas for Decl, and   
the net radial astrometric uncertainty 
of 7 mas. 
For the F225W image, the {\sl rms} fit 
residuals are 23 mas 
for RA and 13 mas for Decl, 
and the net radial astrometric 
uncertainty is 49 mas. 
The degradation of the referencing accuracy, 
as compared 
to the F438W image, is caused by a significant 
brightness  decrease of the objects   
in the images.

%The SBC/F140LP image (Figure~\ref{fig:2})  
%contains only two of the 
%{\sl Gaia} sources 
%used  for the UVIS astrometry. 
%These are the galaxy (object 2), whose  position becomes much less certain in this image,
%and star 3, 
%which is much fainter in the FUV band. However, this image  also contains three sources with counterparts in the F438W image: a spiral galaxy
%seen edge-on (within the box in Figure~\ref{fig:1}) 
%and more compact  objects 6 and 7. 
%Comparison of their positions, especially 
%of the star 3, which is the only point source 
%in the F140LP FoV, with the well defined positions in the F438W  image,  reveals 
%offsets of 1\farcs32  in RA and 0\farcs39 in Decl.  
%We correct the WCS of the F140LP image to 
%align it to the reference F438W image.
%{\bf Accounting for the SBC positional uncertainty of star 3  
%\sout{calculated}
%\gp{estimated} as the full width \gp{at} half maximum (FWHM) of the SBC point spread function (PSF) of $\sim$20 original 
%pixels\footnote{\url{https://www.stsci.edu/itt/APT_help/ACS_Cycle21/c05_imaging7.html}} ($\sim$0\farcs6)
%\gp{I suspect the WFHM is much narrower! Your ref (in the footnote) is not good. Some info might be extracted from Avila \& Chiaberge 2016.} 
%divided by the star signal-to-noise ratio $\rm S/N \approx 3$,}
%we conservatively estimate  the  error of this alignment to be 0\farcs2 and consider this value as the resulting  radial astrometric uncertainty of the F140LP image.
 An accurate alignment of the F140LP and F438W images is not possible with a standard approach because 
three of the only four common objects (marked in Figure~\ref{fig:2}) are extended, with poorly defined positions in the F140LP image. 
The only common point source is star 3, which, however, becomes much fainter in the F140LP band. We used this star 
for an initial estimate of the shifts between the F140LP and F438W images, and found offsets of 1\farcs3  in R.A.\ and 0\farcs4 in Decl.
%\gp{If you want to use the (questionable) uncertainty estimate below, 0\farcs1, then better write the offsets as 1\farcs3 and 0\farcs4 (and even without that estimate such rounding is reasonable here).} 
Their uncertainties  are dominated by the  position error of $\approx$30 mas of star 3 in 
the F140LP image. It was conservatively        
estimated  \citep[see, e.g,][]{1995ApJSNeuschaefer}    %$\approx 0\farcs03$ 
 as FWHM of the SBC point spread function (PSF)   
of $\approx 0\farcs2$ \citep{Avila2016} divided by the signal-to-noise ratio, $S/N\approx 3$, of the star 
in the  image times $\sqrt{8\ln 2} = 2.35$. The {\tt imcentroid} task yields  a similar value.
%\footnote{\sout{See \url{https://pdfs.semanticscholar.org/356f/a45f97f9f22ca836ce35e8d0c1389204d984.pdf} \gp{The ref is actually in the reference list in a readable form}}}
%\gp{It would be better to refer to Avila \& Chiaberge 2016 in the text (no need for the footnote). More important, I think the FWHM is $\approx 0\farcs2$, not 0\farcs3. It follows from Table 2 in AC16, with account for the fact that  the half-maximum for a 2D Gaussian distribution corresponds to the radius at which the EEF = 1/2. For F140LP EEF = 0.51 at $r=0\farcs1$; `full width' is about $2r$, etc. divided by the signal-to-noise ratio $S/N\approx 3$ of star 3 in the F140LP image.} 
%\gp{That is, the uncertainty is 0\farcs06?} \yus{Yes.}
Then we corrected the WCS values in the header of the SBC fits file 
applying the measured offsets and used the extended objects, particularly the nearby spiral galaxy seen edge-on,
that shows a similar structure in both images, to check the shifts and reveal possible signatures of rotation between the two frames.  
A similar approach was applied by \cite{2002zhar} to align the UV and optical frames for PSR B0950$+$08. Overlaying the frames (see Figure~\ref{fig:2}), we found no signs of additional shifts or
%and the 
rotation within the  uncertainty of 0\farcs2 
%the above offsets rounded to 0\farcs1. %\old{conservative referencing  uncertainty of 0\farcs2.} 
%\yus{It might be useful to overlay the spiral galaxy contour from F438W (or from F438+F338) to F140 in Fig. 2?} 
%\gp{This 0\farcs2 just appeared out of the blue (from nowhere). Also ``rotation within the uncertainty of 0\farcs2'' does not make much sense to me.}
%\yus{Yes, actually, we have only one  semi-quantitative  estimate,  0\farcs1, confirmed by the overlay. I would save it as the final one,
%anyway the FUV source sits within the 
%0\farcs1 circle as well.}
and concluded that the SBC astrometric referencing is confident within this uncertainty.  

%%%%%%%% Table 3 %%%%%%%%
\begin{deluxetable*}{ccccc}[t!]
\tablecaption{Positions of PSR 0108$-$1431 at different epochs and 
its p.m.\ components  \label{tab:psr-positions}}
\tablecolumns{5}
\tablenum{3}
\tablewidth{0pt}
\tablehead{
\colhead{Epoch } &\colhead{RA (J2000)} &\colhead{Decl (J2000)} & \colhead{$\mu_{\alpha}$} & \colhead{$\mu_{\delta}$}\\ 
\colhead{MJD} & \colhead{} & \colhead{}  & \colhead{mas yr$^{-1}$} & \colhead{mas yr$^{-1}$}}
\startdata
51752 & 01\h08\m08\fs314(3) & $-$14\degs31\amin49\farcs207(37) & \nodata & \nodata \\
54100 & 01\h08\m08\fs34702(9) & $-$14\degs31\amin50\farcs187(1) & +75.1 $\pm$ 2.3 & $-$152.5 $\pm$ 1.7\\
57608 & 01\h08\m08\fs3967(15) & $-$14\degs31\amin51\farcs651(16) & \nodata & \nodata \\
57611 & 01\h08\m08\fs397(7) & $-$14\degs31\amin51\farcs65(10) & \nodata & \nodata \\
\enddata
\tablecomments{
The coordinates and proper motions of the radio pulsar in the second line are taken from Deller et al.\ (2009). 
They are used to calculate the coordinates at the epochs of the {\sl HST} UVIS and SBC observations (third 
and fourth lines, respectively) and VLT FORS1 observations (first line).
Hereafter, the numbers in brackets are uncertainties related to the last significant digits quoted. 
The uncertainties in the first, third and fourth lines include the pulsar p.m.\ propagation errors 
%while in the fourth line they also include 
and the astrometric reference  uncertainties of the corresponding images.} 
\end{deluxetable*}
%%%%%%%%%%%%%%%%%%%%%%%%%%
%%%%%%%% Fig 3 %%%%%%%%%%%
%\begin{figure*}[t!]
%\begin{minipage}[h]{0.5\linewidth}
%\center{\includegraphics[width=1\linewidth]{F140LP-cr2.jpeg}}
%\end{minipage}
%\hfill
%\begin{minipage}[h]{0.5\linewidth}
%\center{\includegraphics[width=1\linewidth]{F225W-cr2.jpeg}}
%\end{minipage}
%\hfill
%\begin{minipage}[h]{0.5\linewidth}
%\center{\includegraphics[width=1\linewidth]{F336W-cr2.jpeg}}
%\end{minipage}
%\hfill
%\begin{minipage}[h]{0.5\linewidth}
%\center{\includegraphics[width=1\linewidth]{F438W-cr2.jpeg}}
%\end{minipage}
%\caption{
%F140LP, F225W, F336W, and F438W unbinned and unsmoothed images of the pulsar vicinity. The color bars decode 
%the image intensity in counts~ks$^{-1}$ per pixel.  
%The white circles 
%are centered on the expected radio pulsar positions; their radii of 0\farcs2 
%correspond to the $1\sigma$ position uncertainty in the F140LP image. 
%The white arrows show 
%the direction of the pulsar p.m. A possible pulsar counterpart is detected within the 
%circle in the  F140LP image but not seen in the other images. To better resolve it, 
%a 2\asec$\times$2\asec region within the white box in the F140LP image is zoomed up in Figure~\ref{fig:4}. A 
%relatively bright galaxy is seen in all the bands northwest of the pulsar. In the F438W 
%image
%there are  two faint compact 
%field objects $\approx$1\farcs1 and 2\farcs5 southeast of the 
%pulsar's position, marked by A and B, respectively. The yellow cross shows 
%the pulsar position for VLT observation epoch (MJD 51752).
%\label{fig:3}}
%\end{figure*}
%%%%%%%%%%%%%%%%%%%%%%%%

The most accurate  radio  position and p.m.\ of the pulsar were obtained for the reference epoch of MJD 54100  
with the Very Long Baseline Interferometry  observations  using the Australian Long Baseline Array 
\citep{Deller2009}.  Using them, we calculated the pulsar coordinates 
for the epochs of the {\sl HST} and {\sl VLT} observations 
%(MJD 57608; 
(Table~\ref{tab:psr-positions}).
The uncertainties on the calculated position in the {\sl HST} images 
(third and fourth lines in 
Table~\ref{tab:psr-positions}) include the uncertainties of the 
%{\sl HST} 
F438W and F140LP astrometry   
%accounting for the most conservative ones  resulting from the F140LP image referencing,    
%in RA and DEC 
and the uncertainties 
due to propagation of the pulsar p.m.\ errors. 


%%%%%%%%%%%%%%%%%%%%%%%%%%%%%%%%%%%%%%%%%%%%%%%%%%%%%
\section{Possible pulsar counterpart}
%%%%%%%%%%%%%%%%%%%%%%%%%%%%%%%%%%%%%%%%%%%%%%%%%%%%%
%The {\sl HST} images of the pulsar field in all the used filters are presented in Figure~\ref{fig:3}. 
In the SBC/F140LP image (Figure~\ref{fig:2})   
%({\sl top-left panel} of Figure~\ref{fig:3}), within the dashed box region 
we found a faint point-like source with coordinates  
RA = 01\h08\m08\fs403(14) and  
Decl = $-$14\degs31\amin51\farcs66(20),    
 consistent with the expected  radio pulsar  
coordinates (see 4th row in Table~\ref{tab:psr-positions}).
A zoomed-in region around this source is shown in  Figure~\ref{fig:4}, which 
clearly demonstrates  that  the source is located 
within the circle  with the radius 
of 0\farcs2 corresponding to the 1$\sigma$ uncertainty of the pulsar radio position in this image. 

%To verify that the source is not a detector artifact, 
%we inspected the raw image where we found only one obvious artifact, with a count rate of 2.5 counts s$^{-1}$,
%at the detector coordinates $x=56$ pix, $y=282$ pix,
%far away from the pulsar position at $x=495$ pix, $y=446$ pix.
%This detector feature is known as a bright spot\footnote{See \url{ http://www.stsci.edu/hst/instrumentation/acs/performance/anomalies-and-artifacts}.}.  
%We also inspected the respective pulsar position on the detector  using other archival SBC observations of the same program 
%(the target was outside the FoV in those observations) carried out in 2015 with the same 
%filter. We did not find any detector artifacts around $x=495$ pix, $y=446$ pix.
%In the {\sl left panel} of Figure \ref{fig:5}, we present     
%a fragment of the F140LP frame binned 
%by $4\times4$ pixels, where the detected source is marked by the circle %({\sl left panel}),   
%and its spatial  profile is shown in the {\sl right panel}.  
%The profile looks like a stellar one.
%Star 3 (marked in Figures~\ref{fig:1} and \ref{fig:2}), certainly a real object,
%used for astrometry 
%in Section~\ref{astrom}, 
%has similar brightness and spatial profile in the SBC image. 
%This is an additional argument for the detected count excess to be a real source as well. 
%Below we will  consider it as a possible pulsar counterpart.    

%%%%%%%%%%%%%%%%%%%%%%%%%%%%%%%%%%%%%%%%%%%%%%%%%%%%%%%%%%%%%%%%%%%%%%%%%%%%%%%%%%%
\begin{deluxetable*}{ccccccc}[ht]
\tablecaption{SBC photometry of the pulsar counterpart candidate \label{tab:photometry-mod}}
%\tablecolumns{8}
\tablenum{4}
%\tablewidth{Opt}
\tablehead{
%\colhead{Image}
%\colhead{Filter} &
%\colhead{$t_{\rm exp}$} &
\colhead{$N_{t}$} & \colhead{$C_{\rm pos}$} %&\colhead{$N_{b}$} 
& \colhead{$\overline{C}_{\rm bgd}$} & \colhead{$\sigma_{\rm bgd}$}& \colhead{$N_{s}$} & \colhead{$C_{s}$} & 
%\colhead{${\cal P}_{\nu}$} & 
\colhead{$f_{\nu}$}\\
%\colhead{ } & \colhead{(ks)} & 
\colhead{(cts)} & \colhead{(cts/ks)} %& \colhead{(cts)} 
& (cts/ks) & \colhead{(cts/ks)} & 
%\colhead{(nJy ks/cts)} 
\colhead{(cts)} & \colhead{(cts/ks)} & \colhead{(nJy)}
}
\startdata
 %15.3 & 5.5 & 183.4 & \nodata & \nodata & $10.1\pm3.9$ & $5.8 \pm 2.2$ & $9.3 \pm 3.6$\\
 %43.2 & 15.4 & 183.4 & \nodata & \nodata & $19.7\pm6.8$ & $9.2 \pm 3.2$ & $14.7 \pm 5.1$\\
 %43.2 & 15.4 & \nodata & 8.7 & 1.2 & $18.8\pm 5.5$ & $8.7\pm 2.5$ & $14.0\pm 4.1$\\
 15.3 & 5.5 %& \nodata 
 & 2.0 & 0.7 & $9.8\pm 3.7$ & $5.6\pm 2.0$ & $9.0\pm 3.2$
\enddata
%\tablenotetext{a}{At exposure start.}
\tablecomments{
%First and second table lines correspond to different methods of background measurement -- from an annulus around the target and from 1000 circles randomly chosen in a larger area (see text for details). 
%The columns provide the following quantities:.......
$N_{t}$ is the total number of counts in the $A_s=0.126$ arcsec$^2$ source aperture, $C_{\rm pos}= N_t/t_{\rm exp}$ is the 
%source plus background 
count rate measured in the source aperture ($t_{\rm exp}=2.8$ ks is the exposure time), %$N_{b}$ is the number of background counts in the $A_b=4.43$ arcsec$^2$ annulus,
$\overline{C}_{\rm bgd}$ and $\sigma_{\rm bgd}$ are the mean and standard deviation of background measurements in 1000 circles with the 0\farcs2 radius in the $A_b=4.12$ arcsec$^2$ annulus,   $N_{s}$
%=N_{t}-N_b A_s/A_b$ 
is the net source count number in the source aperture,
%after background subtraction, 
its error is estimated as  %$[N_{t}+N_b (A_s/A_b)^2]^{1/2}$ in the first line and 
$[(\sigma_{\rm bgd} t_{\rm exp})^2 + N_s]^{1/2}$,  %in the second line, 
$C_{s}=N_s/t_{\rm exp}/\phi_E$ is the aperture-corrected net 
source count rate, $\phi_E\approx 0.63$ \citep{Avila2016} is a fraction of source counts in the $r=0\farcs2$ aperture, $f_\nu = C_s {\cal P}_\nu$ is the flux density at the pivot wavelength $\lambda_{\rm piv}=1528$ \AA, and
%corrected for the aperture,
${\cal P}_\nu = 1.61$ nJy ks cts$^{-1}$ 
 is 
the count-rate-to-flux conversion factor.
%(we took into account the recent correction of the SBC sensitivity by \citet{Avila2019}).
%of the count rate to 
%the flux density 
%at the pivot wavelength, $f_\nu = C_s{\cal P}_\nu$ 
%derived using the image header keywords PHOTFLAM and PHOTPLAM
%$f_{\nu}=C_{s}\cal{P}_{\nu}$ is the source flux density.}
%in its  entirety in the machine readable format.  A portion is
%shown here for guidance regarding its form and content.}
}
\end{deluxetable*}
%%%%%%%%%%%%%%%%%%%%%%%%%%%%%%%%%%%%%%%%%%%%%%%%%
%%%%%%%%%%%%%%%%%%%%%%%%%%%%%%%%%%%%%%%%%%%%%
%%%%%%%%%%%%%%%%%%%%%%%%%%%%%%%%% Fig 4 %%%%%%%%%%%%%%%%%%%%%%%%%%%%%%%%
\begin{figure}[b!]
\includegraphics[scale=0.318,angle=0]{140-0720-21.png}
\caption{2\asec$\times$2\asec region of the 
unbinned and unsmoothed
SBC F140LP image 
%containing 
around the pulsar. 
The color bar decodes the image intensity in counts per pixel.
The 0\farcs2 radius circle and the arrow mark the 
%pulsar 
$1 \sigma$ radio pulsar position uncertainty 
%in the image 
and the direction of its p.m. A faint point-like 
counterpart candidate to the pulsar is clearly visible within the circle.
%\gp{[[Make the arrow brighter/thicker, particularly the arrowhead.]]}
\label{fig:4}}
\end{figure}
%%%%%%%%%%%%%%%%%%%%%%%%%%%%%%%%%%%%%%%%%%%%%%%%%%%%%%%%%%%%%%%%%%%%%%


\subsection{Photometry of the FUV counterpart candidate}

%For the SBC/F140LP  photometry of the pulsar  counterpart candidate, we followed the approach 
%described by \cite{Pavlov2017}. 
To find an optimal source aperture for the SBC/F140LP photometry of the pulsar counterpart candidate,
we calculated the signal-to-noise ratio 
${\rm S/N}$ as a function of  radius of 
circular aperture centered at the brightest source pixel in the unbinned image using 
%Using 
the background extracted from the annulus around the same center with inner and outer radii of 20 %35  
and 50 %59 
pixels 
(area $A_b=4.12$ %$ $A_b=4.43
arcsec$^2$). 
We found the maximum S/N = 2.7 in the aperture 
of 8 pixels (0\farcs2) radius
%\footnote{\bf Note that this aperture radius is close the FWHM of a halo around a point source imaged with SBC; see Section 5.6 of the ACS Instrument Handbook; \url{https://www.stsci.edu/itt/APT_help/ACS_Cycle21/c05_imaging7.html}.}
(area $A_s=0.126$ arcsec$^2$). This aperture was selected as the optimal one. 
%provides a maximal ${\rm S/N} \approx 2.9$. The aperture area is $A_s=0.567$ arcsec$^2$. 
%The value of the maximal S/N slightly depends on 
%We used two methods to measure the background (see Table \ref{tab:photometry-mod}). In first method we extracted the background  from the annulus of 35 pix and 59 pix inner and outer radii, respectively 
%its area is 
%(area $A_b=4.43$ arcsec$^2$).
%, and the number of source counts in the source aperture and its uncertainty were estimated as $N_s=N_t-(A_s/A_b)N_b$ and $\delta N_s = [N_t+N_b(A_s/A_b)^2]^{1/2}$, where $N_t$ is the total number of counts
%in the source aperture, $N_b$ is the number of counts in the annulus. 
%In second method we measured the count rates in a set of 1000 circular background regions with 
%the same radius of 
%$r=17$ pix 
%(0/farcs425) as that of the source aperture; the circles were 
%randomly placed in the annulus with 25 and 65 pixels inner and outer radii, and found the mean and standard deviation, $\overline{C}_{\rm bgd}$ and $\sigma_{\rm bgd}$ of the background count rate.
%As seen from Figure \ref{fig:5}, where the image is binned by $4\times4$ pixels for a more clear   presentation, 
%this aperture  encapsulates a major fraction of the source flux.   
According to Table 2 in \citet{Avila2016}, 
it contains the fraction $\phi_E\approx 0.63$ of the total number of point source counts. 
%{\bf For  low ${\rm S/N}$ SBC point-like sources, optimal apertures were claimed to be $\la$10 pixels \citep[e.g.,][]{Pavlov2017,2017rangelov}. Our aperture is somewhat larger, which might  
%indicate the presence of an unresolved faint  diffusive structure around the pulsar. On the other hand, carefully inspecting the SBC/F140LP data for PSR B0950$+$08 we found that a similar aperture of about 15 pixels appears to be optimal for the low ${\rm S/N}$ UV counterpart of this pulsar, while smaller apertures leads to $\sim$10\% formal underestimation of the aperture corrected counterpart flux. It is also possible that provided SBC aperture corrections for small apertures are less certain than for larger ones. We thus selected 17 pixel aperture for our counterpart candidate as the most optimal assuming that it may contain a 10\% flux contribution from a possible nebula around it. }
%\sout{\citep{Avila2016}.}

%We used the background values found with the two methods to estimate the source count rate, the flux density at the pivot wavelength, and their errors (see the note to Table \ref{tab:photometry-mod}). The flux densities found with the two methods,
%$f_\nu = 14.7\pm 5.1$ and $14.0\pm 4.1$ nJy, coincide 
%within the uncertainties; they correspond to ${\rm S/N} = 2.9$ and 3.4, respectively. In the following estimates we will use the former value as more conservative.
 

%As seen from Figure \ref{fig:5},  
%the chosen aperture  encapsulates 
%a major fraction of the source flux.   
%According to the  SBC point spread function
%(PSF) presented by \citet{Avila2016}, this aperture contains $\phi_E\approx 0.77$ of the total number of point source counts.
%
%The aperture and annulus count numbers, $N_t$ and $N_b$, the aperture-corrected source count rate $C_s$,
%and the corresponding flux density $f_\nu$ at the pivot wavelength $\lambda_{\rm piv}=  1528$ \AA\
%are presented in the first line of Table~\ref{tab:photometry-mod}. 
%To convert the count rate to the flux density, we took into account the recent correction of SBC sensitivity, such that 
%the flux density at a given count rate is about 0.77 of the previously adopted value \citep{Avila2019}.
%We also used another method to measure the background contribution and the source flux.
Following \citet{Guillot2019}, we estimated the mean background $\overline{C}_{\rm bgd}$ and its standard deviation $\sigma_{\rm bgd}$
%$\overline{C}_{\rm bgd} =8.7$ cnts/ks and its standard deviation $\sigma_{\rm bgd}=1.2$ cnts/ks 
%[[why different numbers of significant digits? 1.19 $\to$ 1.2?]] 
for a set of 1000 circular background  
regions of size $r_{\rm extr}$=0\farcs2 (8 pixels, the same as of the source aperture), randomly 
placed in the annulus. % of 20 pix  
%\gp{[Isn't 17 pix too small (close to the the source center)?]}
%and 50 pix inner and outer radii
%respectively, 
%around the pulsar position. 
%The results are presented in  
%second line of 
%Table \ref{tab:photometry-mod}. %Table~\ref{tab:photometry2}. %Comparing them with those in Table~\ref{tab:photometry} we see that within errors both approaches  give consistent flux densities. Below we use a more conservative  flux density from  Table~\ref{tab:photometry}.
%The flux density $f_\nu = 9.0\pm 3.2$ nJy is virtually the same as $14.7\pm 5.1$ found with the first method.
%but S/N is slightly higher (3.4 vs.\ 2.9).
To convert the count rate to the flux density, we took into account the recent correction of the SBC sensitivity, such that 
the flux density at a given count rate is about 0.77 of the previously adopted value \citep{Avila2019}.
The results are presented in  
%second line of 
Table \ref{tab:photometry-mod}. 
%In the following we will use 
%a more conservative 
%flux density range rounded to $9\pm 3$~nJy.
%found by the first method (rounded to $9\pm 3$~nJy). 

%{\bf Possible contribution of the putative UV nebula around the pulsar lies within the rounded uncertainty.}    

%Subtracting 
%this $\overline{C}_{\rm bgd}$ 
%the mean background count rate from the measured count rate in the source aperture,
%$C_{\rm pos}=N_{\rm tot}/t_{\rm exp} =15.4$ cnts/ks, 
%we obtain $C_{\rm pos}-\overline{C}_{\rm bgd}=6.7$ cnts/ks,
%which corresponds to $N_s=18.8$ source counts in the source aperture.
%The uncertainty of $N_s$ is $[(\sigma_{\rm bgd}\times t_{\rm exp})^2 + N_s]^{1/2}
%((1.19\times 2.8)^2 + 18.2)^{1/2} 
%= 5.5$ cnts, which corresponds to the ${\rm S/N} = 3.4$, aperture-corrected $C_s=8.7\pm 2.6$ cnts/ks, and $f_\nu = C_s {\cal P}_\nu = 14.0\pm 4.1$ nJy.
%\gp{[[I am not sure the both methods should indeed be presented in the paper. And it is not clear why the first method ``deserves'' a separate table while the second one does not. Maybe to find a way to put both methods in one table? Or to retain only the second method because it is the same as used for the upper bound estimate below?]]}

As the putative pulsar counterpart %in the F140LP band 
is detected at only about $3\sigma$ level, we cannot rule out that the count 
rate excess at the pulsar position is caused by some fluctuation.
In this case, one can estimate the flux density upper bound following \citet{Kashyap2010} and \citet{Guillot2019}.
We define the upper bounds $C_{\rm ub}$ on the pulsar count rate as $C_{\rm ub}=C_{\rm pos}-\overline{C}_{\rm bgd}+n\sigma_{\rm bgd}$,
where 
$C_{\rm pos}$ is the measured count rate at the position of the pulsar, $\overline{C}_{\rm bgd}$ and $\sigma_{\rm bkg}$ are defined above (see also 
%the note to 
Table \ref{tab:photometry-mod}),
%are the mean 
%and standard deviation of the background count rate, and 
and $n$ determines the significance level of the upper bound\footnote{This 
definition of $C_{\rm ub}$ is applicable at $C_{\rm pos} \geq \overline{C}_{\rm bgd}$, which is fulfilled 
in our case.}.
%and the other terms are defined (and their values reported) in the %preceding paragraph. 
%We 
%choose $n=3$ and therefore 
%report 
The resulting 3$\sigma$ upper bounds %on the count rates and flux densities 
are 
%We obtain
%$C_{\rm pos}=15.4$ cts/ks,
%$\overline{C}_{\rm bgd}=8.9$ cts/ks,
%$\sigma_{\rm bgd}=1.19$ cts/ks,
$C_{\rm ub}=5.5$ cts/ks and  
$f_\nu^{\rm ub} = 14$ nJy.
%{\bf They are marginally consistent with the above  measurements  assuming the detection  of the real source.}
%under the hypothesis of detection of the real source   accounting for its  1$\sigma$ uncertainty.



%They are consistent with the measured source values accounting for their 1$\sigma$ uncertainties.  

%If we assume real detection and measure the background using a set of 1000 circles, 
%\gp{Okay. I think this exercise demonstrated stability of the result w.r.t. method of bgd measurement, and the result itself deserves presenting it in the paper (at the beginning of this section), together with (or even instead of?) the result obtained %with the more traditional method.}

%For a consistency check, we employed the recent version (ver.\ 28.1) of the Exposure Time Calculator (ETC) for the ACS/SBC. Using the 
%source flux density of 14 nJy, and $A_s$ and $t_{\rm exp}$ from Table~\ref{tab:photometry}, we got an expected net source count number $N_s\approx 18.3$   
%and S/N in the range of 1.8--4.2. 
%\gp{[[Why such a scatter of S/N? The different values correspond to which difference in the input parameters?]]} These values are consistent with the measured ones.
%with a mean value of $\approx 3$. These values are fully consistent with the measured ones (see Table~\ref{tab:photometry}).
%and S/N in the range of 1.8--4.3\yus{insert exact ETC value?}, with a mean value of $\approx 3$. These values are fully consistent with the measured ones (see Table~\ref{tab:photometry}). 

 





%%%%%%%%%%%%%%%%%%%%%%%%%%%%% former Table 4 %%%%%%%%%%%%%%%%%%%%%
%\begin{deluxetable*}{ccccccccc}[t!]
%\tablecaption{SBC photometry of the pulsar counterpart candidate.  \label{tab:photometry}}
%\tablecolumns{8}
%\tablenum{4}
%\tablewidth{Opt}
%\tablehead{
%\colhead{Image}
%\colhead{Filter} &
%\colhead{$t_{\rm exp}$} &
%\colhead{$N_{t}$} & \colhead{$N_{b}$} & \colhead{$N_{s}$} & $\phi_E$& \colhead{$C_{s}$} & \colhead{${\cal %P}_{\nu}$} & \colhead{$f_{\nu}$}\\
%\colhead{ } & \colhead{(ks)} & \colhead{(cts)} & \colhead{(cts)} & \colhead{(cts)} & (\%) & \colhead{(cts/ks)} & %\colhead{(nJy ks/cts)} & \colhead{(nJy)}
%}
%\startdata
%F140LP  & 2.8 & 43.2 & 183.4 & $19.7\pm6.8$ & 77 & $9.2 \pm 3.2$ & 1.61 & 14.7 $\pm$ 5.1
%\enddata
%\tablenotetext{a}{At exposure start.}
%\tablecomments{$t_{\rm exp}$ is the exposure time, 
%$N_{t}$ is the total number of counts in the $A_s=0.567$ arcsec$^2$ source aperture, $N_{b}$ is the background %count  number in the $A_b=4.43$ arcsec$^2$ annulus,   $N_{s}=N_{t}-N_b A_s/A_b$ is the net source count number 
%after background subtraction, its error is estimated as  $[N_{t}+N_b (A_s/A_b)^2]^{1/2}$, 
%$C_{s}$ is the aperture-corrected net 
%source count rate,  
%corrected for the aperture,
%$\cal{P}_{\nu}$ 
% is 
%the conversion factor of the count rate to 
%the flux density at the pivot wavelength, $f_\nu = C_s{\cal P}_\nu$ derived using the image header keywords %PHOTFLAM and PHOTPLAM.}  
%$f_{\nu}=C_{s}\cal{P}_{\nu}$ is the source flux density.}
%in its  entirety in the machine readable format.  A portion is
%shown here for guidance regarding its form and content.}
%\end{deluxetable*}
%%%%%%%%%%%%%%%%%%%%%%%%%%%%%%%%%%%%%%%%%%%%%%%%%


%%%%%%%%%%%%%%%%%%%%%%%%%%%%%%%%%%%%%%%%%%%%%%%%%%%%%%%%%%
%\begin{figure*}[t]
%\begin{minipage}[h]{0.4\linewidth}
%\includegraphics[scale=0.25,angle=0]{4x4small.jpeg}
%\end{minipage}
%\hfill
%\begin{minipage}[h]{0.58\linewidth}
%\includegraphics[scale=0.41,angle=0]{SBC-src-profile.eps}
%\end{minipage}
%\caption{
%{\sl Left:} $4\farcs0\times$4\farcs5 fragment of the SBC image binned by 4$\times$4 pixels 
%and then smoothed with the Gaussian kernel of 3 pix radius. The candidate counterpart is within the circle  
%showing the  aperture with the radius of 0\farcs425 used for photometry.  The bright source 
%in the upper-right corner is the fragment of the nearby galaxy seen in Figure~\ref{fig:3}. 
%The horizontal line indicates the image scale.    
%{\sl Right:} The radial brightness profile of the candidate  
%plotted using circular 
%annuli with equal areas centered  
%at the source brightness peak in the image  shown in the left panel. The horizontal axis 
%is in units of 25 mas pixels. 
%native SBC pixel units.
%THEY ARE ACTUALLY NOT "NATVW"..
%The vertical  dashed line shows  the  aperture radius.
%\label{fig:5}}
%\end{figure*}

%%%%%%%%%%%%%%%%%%%%%%%%%%%%% Table 6 -> 5 %%%%%%%%%%%%%%%%%%%%%%%%%%
\begin{deluxetable*}{cccccccccccc}[t]
\tablecaption{
%$3\sigma$ 
Upper bounds on count rates and mean flux densities
%upper bounds for PSR J0108$-$1431 
in the UVIS filters at the pulsar radio position, $\alpha$=01\h08\m08\fs397 and $\delta$=$-$14\degs31\amin51\farcs65, at the {\sl HST} observations epoch (MJD 57608)\label{tab:upperbounds}}
\tablecolumns{11}
\tablenum{5}
%\tablewidth{Opt}
\tablehead{
\colhead{Filter} &
\colhead{$\lambda_{\rm piv}$} &
\colhead{$t_{\rm exp}$} &
\colhead{$r_{\rm extr}$} & \colhead{$\phi_{E}$} & \colhead{$C_{\rm pos}$} & \colhead{$\overline{C}_{\rm bgd} \pm \sigma_{\rm bgd}$} & \colhead{$C_{\rm ub}$} & \colhead{${\cal P}_{\nu}$} & \colhead{$f_{\nu,3\sigma}^{\rm ub}$} & \colhead{$f_{\nu,1\sigma}^{\rm ub}$}\\
\colhead{} & \colhead{(\AA)} & \colhead{(s)} & \colhead{($''$)} & \colhead{(\%)} & \colhead{(cts/ks)} & \colhead{(cts/ks)} & \colhead{(cts/ks)} & \colhead{(nJy ks/cts)} & \colhead{(nJy)} & \colhead{(nJy)}
}
\startdata
%F140LP & 1528 & 2800 & 0.425 & 77 & 8.5 & 7.7 & 1.47 & 5.21 & 2.09 & 14.1\tablenotemark{a}\\
%F438W & 4326 & 5160 & 0.14 & 78 & 47.0 & 5.58 & 20.7 & 103.5 & 0.416 & 55 & 62\\
F438W & 4326 & 5160 & 0.14 & 81 & 47 & 
%16.2 $\pm$ 23.7
$16\pm 24$ & 
%101.9
102 & 0.416 & 52 & 28\\
%F336W & 3355 & 5160 & 0.12 & 67 & 26.8 & 8.65 & 14.9 & 62.8 & 0.470 & 44 & 51\\
F336W & 3355 & 5160 & 0.12 & 78 & %26.8 
27 & 
%13.0 $\pm$ 15.8
$13\pm 16$ & 
%61.2 
61 & 0.470 & 37 & 20\\
%F225W & 2366 & 4932 & 0.14 & 72 & 46.2 & 11.1 & 15.5 & 81.6 & 0.783 & 89 & 114\\
F225W & 2366 & 4932 & 0.14 & 74 & %46.2
46 & 
%16.2 $\pm$ 17.4
$16\pm 17$ & 
%82.2 
82 & 0.783 & 87 & 50\\
%F140LP & 1528 & 2800 & 0.425 & 77 & 8.5 & 7.7 & 1.47 & 5.21 & 1.61 & 11 & 14\\ 
%2.09 & 14.1\\
%\tablenotemark{a}\\
\enddata
%\tablenotetext{a}{9.9 nJy accounting for the SBC 30\% sensitivity drop.}
\tablecomments{
%$f_{\nu}^{\rm dr}$ is the de-reddened upper bound, 
%assuming $A_{V}=0.1$. 
 $C_{\rm ub}$ and $f_{\nu,3\sigma}^{\rm ub}$ are the $3 \sigma$ 
 upper bounds; $f_{\nu,1\sigma}^{\rm ub}$ is the $1 \sigma$ upper bound. 
 %\gp {[In the final version 
 %the count rates should be rounded to integer value of cnts/ks. %\yus{not sure, they are already integer in cts/s}]
 %} \gp{I would combine 7-th and 8-th columns in one column, 
 %E.g., $20\pm 19$, $11\pm 14$ and $12\pm 17$, for $\overline{C}_{\rm bgd}\pm \sigma_{\rm bgd}$.}
 }
 
%}
%\tablecomments{Table \ref{tab:astrometry} is published 
%in its entirety in the machine readable format.  A portion is
%shown here for guidance regarding its form and content.}
\end{deluxetable*}
%%%%%%%%%%%%%%%%%%%%%%%%%%%%%%%%%%%%%%%%%%%%%%%%%%%%%%%%%%
%%%%%%%%%%%%%%%%%%%%%%%%%%%%% Table 7 --> 6 %%%%%%%%%%%%%%%%%%%%%%%%%%
\begin{deluxetable*}{cccccccccc}[t!]
\tablecaption{$3\sigma$ upper bounds on count rate and mean flux density 
%upper bounds 
%for 
at the pulsar radio position, $\alpha$=01\h08\m08\fs314 and $\delta$=$-$14\degs31\amin49\farcs207, at the VLT observations epoch (MJD 51752)
%$\alpha$=01\h08\m08\fs314 and $\delta$=$-$14\degs31\amin49\farcs207 
%reported by \citet{Mignani2008} 
%$\alpha$=01\h08\m08\fs325 and $\delta$=$-$14\degs31\amin49\farcs47
\label{tab:upperbounds2}}
\tablecolumns{10}
\tablenum{6}
%\tablewidth{Opt}
\tablehead{
\colhead{Filter} &
\colhead{$\lambda$} &
\colhead{$t_{\rm exp}$} &
\colhead{$r_{\rm extr}$} & \colhead{$\phi_{E}$} & \colhead{$C_{\rm pos}$} & \colhead{$\overline{C}_{\rm bgd} \pm \sigma_{\rm bgd}$} & \colhead{$C_{\rm ub}$} & \colhead{${\cal P}_{\nu}$} & \colhead{$f_{\nu,3\sigma}^{\rm ub}$}\\
\colhead{} & \colhead{(\AA)} & \colhead{(s)} & \colhead{($''$)} & \colhead{(\%)} & \colhead{(cts/ks)} & \colhead{(cts/ks)}  & \colhead{(cts/ks)} & \colhead{(nJy ks/cts)} & \colhead{(nJy)}
}
\startdata
%F438W & 4326 & 5160 & 0.14 & 78 & 18.2 & 5.58 & 20.7 & 74.7 & 0.416 & 40\\
F438W & 4326 & 5160 & 0.14 & 81 & %13.1 
13 & 
%14.9 $\pm$ 16.6
$15\pm 17$ & 
%49.8 
50 & 0.416 & 26\\
%F336W & 3355 & 5160 & 0.12 & 67 & 3.5 & 8.65 & 14.9 & 44.7 & 0.470 & 31\\
F336W & 3355 & 5160 & 0.12 & 78 & 6.4 & 
%12.2 $\pm$ 13.2
$12 \pm 13$ & 
%39.6 
40 & 0.470 & 24\\
%F336W & 3355 & 5160 & 0.12 & 67 & 3.5 & 13.8 & 14.2 & 42.6 & 0.470 & 30\\
F225W & 2366 & 4932 & 0.14 & 74 & %24.7 
25 & 
%5.5 $\pm$ 13.1
$6\pm 13$ & 
%58.5 
59 & 0.783 & 62\\
%F225W & 2366 & 4932 & 0.14 & 72 & 26.9 & 11.1 & 15.5 & 62.3 & 0.783 & 68\\
%F140LP & 1528 & 2800 & 0.425 & 77 & 7.3 & 10.4 $\pm$ 1.5 & 4.5 & 1.61 & 9\\ 
F140LP & 1528 & 2800 & 0.2 & 63 & 0.9 & 2.1 $\pm$ 0.6 & 1.8 & 1.61 & 5\\
%2.09 & 14.1\\
%\tablenotemark{a}\\
\enddata
\tablecomments{For the F438W, F336W and F140LP bands, $C_{\rm pos}<\overline{C}_{\rm bgd}$ 
and the upper bounds 
%for the them 
are calculated as $C_{\rm ub}=3\sigma_{\rm bgd}$.
%\gp{See my remark to Table 5.}
} 
\end{deluxetable*}
%%%%%%%%%%%%%%%%%%%%%%%%%%%%%%%%%%%%%%%%%%%%%%%%%
\subsection{UVIS upper bounds and VLT observations of PSR J0108-1431 \label{UVIS+VLT}}

The counterpart candidate is not detected in the UVIS images.
%\footnote{{\bf We do not show F336W and F225W images as they are generally similar to that 
%of F438W presented in Figure~\ref{fig:1}.}}. 
%(Figure~\ref{fig:3}). %this is a blue object,
%\gp{The word `blue' may be misleading here -- we see something in FUV but nothing in the blue filter. Is there another way to make the same point? Maybe just remove ``implying ...''?}
%as expected for a pulsar. 
%In all the bands a relatively bright spiral galaxy seen edge-on is visible northwest of the pulsar.
%(Figure~\ref{fig:3}). 
%In the F438W image 
%there are also two faint objects, A and B, in the immediate pulsar vicinity, likely background galaxies ({\sl bottom-right panel} of Figure~\ref{fig:3}).  
%Both the A and B  objects and the spiral galaxy were previously detected  in a deep  $B$-band image obtained with the VLT FORS1 instrument in 2000 July 31 \citep{Mignani2003}.
%In the UVIS images we did not detect any source 
%at the calculated pulsar position (Figure ~\ref{fig:3}). 
%Therefore, we calculated bounds on the minimum detectable fluxes at the pulsar position 
%for each of the {\sl HST} filters. 
% Following \citet{Kashyap2010} and \citet{Guillot2019},
%We define 
Therefore, we calculated the upper bounds on the pulsar count rate and flux density  for each of the UVIS filters  using the same approach as for the F140LP filter.
%(see Table~\ref{tab:upperbounds}).
%(Table~\ref{tab:upperbounds}).
%$C_{\rm ub}$ on the pulsar count rate 
%as $C_{\rm ub}=C_{\rm pos}-\overline{C}_{\rm bgd}+n\sigma_{\rm bgd}$,
%\citep{Guillot2019}, 
%where $C_{\rm pos}$ is the measured count rate 
%at the position of the pulsar, $\overline{C}_{\rm bgd}$ 
%and $\sigma_{\rm bkg}$ are the mean 
%and  standard deviation of the background count rate, 
%and $n$ determines the significance of 
%the upper bound\footnote{This 
% definition of $C_{\rm ub}$ is applicable at 
% $C_{\rm pos} \geq \overline{C}_{\rm bgd}$, which is fulfilled 
%in our case.}. 
% We choose $n=3$ and therefore report 3$\sigma$ upper 
%bounds on the count rates and 
%flux densities.
The sizes of the extraction regions used to measure $C_{\rm pos}$ were chosen by identifying in each image 
the extraction radius that maximizes  $\rm S/N$ 
for a point source (star). 
%For instance, for the F438W image the optimal extraction radius is $r_{\rm extr}$ = 0\farcs14 (3.5 pixels). 
%It corresponds to the enclosed energy fraction $\phi_{E}=0.81$.
For each of the filters, we estimated $\overline{C}_{\rm bgd}$ and $\sigma_{\rm bgd}$ from a set of 1000 circular background  
regions of size $r_{\rm extr}$, randomly selected 
in the 
%half ring 
annulus of 10 pixels 
%\gp{[[Isn't 3 pix too small?]]}
and 30 pixels inner and outer radii 
%respectively, 
around the pulsar position. % where   object $A$ was excluded. 
%\gp{in how large area?}
%We obtained the sum of counts in each of these regions and calculate the mean
%$\overline{C}_{\rm bgd}$ and standard deviation $\sigma_{\rm bgd}$ needed to obtain 
These quantities, together with $C_{\rm ub}$ and the corresponding flux densities corrected for the finite extraction aperture, $f_\nu^{\rm ub} = C_{\rm ub} {\cal P}_\nu \phi_E^{-1}$, are presented in Table~\ref{tab:upperbounds} where we also show the $1\sigma$ flux density  upper bounds used in spectral fits (Section 5.2).


%Using the seeing-limited VLT FORS1 observations taken in 2000 July-August, \citet{Mignani2008} 
%reported a possible pulsar counterpart tentatively   detected in the $U$ and $B$ (but not $V$) bands 
%near the northern edge of
%in close vicinity to 
%the spiral galaxy at the brightness level of $U=26.4\pm 0.3$ and $B=27.9\pm 0.5$.
%\gp{%It would be useful to show that position in Fig 3, e.g. as a small yellow cross. 
%To my eye, the backward extrapolation of the psr trajectory goes far from that position; is it because of p.m. uncertainties? 
%Could p.m. uncertainties be also shown in Fig 3 - e.g. as two straight lines passing through the position at the reference epoch?} 
%The object was not detected in the VLT $V$ band, while its   
%Its coordinates, RA = 01\h08\m08\fs301 and Decl = $-14\degs31\amin49\farcs15$, were consistent with the radio pulsar position at the epoch 
%of the VLT observations presented  in Table~\ref{tab:psr-positions}. 
%and shown by the yellow cross in Figure~\ref{fig:3}. %obtained using the pulsar p.m.\ Weand backward extrapolation of its track. 
We also do not detect any object in any of the {\sl HST} bands  at the
position where \citet{Mignani2008} found a possible pulsar counterpart in the VLT $U$ and $B$  bands near the northern edge of the spiral galaxy. 
%The object was not detected in the VLT $V$ band, while its   coordinates, RA = 01\h08\m08\fs301 and Decl = $-14\degs31\amin49\farcs15$, were consistent with the radio pulsar position at the epoch of the VLT observations presented  in Table~\ref{tab:psr-positions}.%obtained using the pulsar p.m.\ and backward extrapolation of its track. 
%%%%%%%%%%%%%%%%%%%%%%%%%%%%%%%%%%%%%%%%%%% Fig 6
\begin{figure*}[t]
\begin{minipage}[h]{0.5\linewidth}
\includegraphics[scale=0.3,angle=0]{VLTUB-4.jpeg}
\end{minipage}
\begin{minipage}[h]{0.5\linewidth}
\includegraphics[scale=0.3,angle=0]{438-336-4.jpeg}
\end{minipage}
\caption{
%$\approx
12\farcs0$\times$11\farcs5\ vicinity of the pulsar as seen with the VLT in the $U$$+$$B$ band ({\sl left:}) and  with the {\sl HST}/UVIS in the $F438W$$+$$F336W$ band 
({\sl right:}). North is up and east is left. Smoothing with the Gaussian kernel %with the 
radius of 1 pixel is applied to the VLT image.  The sources are numerated using the nomenclature of \citet{Mignani2003}.   
%and marked by letters as in Figure~\ref{fig:3}.
Pulsar radio positions at the VLT epoch (2000) and the {\sl HST}  epoch (2016) are shown by  yellow and red crosses, respectively. %The shift of $\approx$2\farcs7 between them is due to the pulsar p.m. 
A faint source  is marginally detected in the VLT image near the 2000 position of the pulsar, while it is absent in the {\sl HST} image (for details see Sect.~\ref{UVIS+VLT})  
%\gp{The red symbols are not well seen on the grey bgd; wouldn't another color be better?}
\label{fig:6}}
\end{figure*}
%%%%%%%%%%%%%%%%%%%%%%%%%%%%%%%%%%%%%%%%%%%%%%%%%
For the {\sl HST}--VLT consistency check, we re-reduced  the VLT $UBV$  data %taken from the ESO-archive 
using  the recent version of the  
ESO recipe execution tool EsoRex 3.13.12\footnote{\url{https://www.eso.org/sci/software/cpl/esorex.html}}. To maximize the spatial resolution and better resolve the putative counterpart from the  galaxy, we selected six best-seeing 900 s  exposures of eight available in $B$, three 1800 s exposures of five available in $U$, and all twelve  %six 
600 s exposures %of twelve 
in $V$\footnote{For the VLT observation log, see \citet{Mignani2003}.}. This  resulted in an extremely good seeing on the stacked images of 0\farcs56 in the $U$ and $B$ bands and 0\farcs54 in the $V$ band. 
%A few obvious cosmic ray hits  retained in the published images were cleaned.     

The VLT astrometric solution was also revised using 11 relatively bright unsaturated {\sl Gaia} stars, 
with account for their p.m.\ shifts for  the epoch of the VLT observations.  This   
resulted in a 30 mas uncertainty  of the WCS  referencing of all images  in both coordinates, which is significantly better than the previous  astrometric %accuracy 
precision of 190 mas %0\farcs19 
based on the GSC-II catalog \citep{Mignani2008}. 
%All this 
The more precise astrometry  
%allowed us to confirm 
supports the assumption that the object detected with the VLT 
%counterpart candidate 
at a $2\sigma$ significance  in the $U$ and $B$ bands is the counterpart candidate.
%and its absence in the V band. 
%To better reveal the candidate, 
%To increase the detection significance,
%we concatenated   
Stacking the $U$ and $B$ images 
%\gp{[]Is ``stacking'' (or merging) okay here? Why should we weigh on exposures?]}
%weighting on their exposures. 
%This leads to a total exposure of $\approx$10.8 ks in the $U+B$ band %, in the wavelength range where two bands overlap,
%\gp{Sorry, I don't understand how the overlap (or a lack of it) is relevant} and %improves the source 
%and 
increases the detection significance to $3\sigma$.
%, 
%\sout{suggesting}
%which suggests that it is indeed a real object.  
The region of this image containing the pulsar  is shown in the left panel of Figure~\ref{fig:6}. 
For comparison, we also merged the {\sl HST}/UVIS images in the F336W and F438W bands.  
%which are the {\sl HST} analog of the  $U$  and $B$ bands. 
The respective region   
%The same 
%region   in the F336W+F438W band obtained with the HST/UVIS 
is  presented in the right panel of Figure~\ref{fig:6}. %in for comparison. 
%Besides the absence of
Not only the VLT source proposed  as the pulsar counterpart is not seen in the F336W+F438W image but also some other faint 
%weak 
%unrelated 
VLT objects, such as sources 3 and 8--11  %nominated by 
in \citet{Mignani2003},  are also either resolved only marginally or not visible.
%in the  F336W+F438W image. %  Merging  the  UVIS images %obtained in the  three filters 
%does not 
%help to reveal them, 
%although its total exposure %in the      F336W+F438W band
%is similar to that of $U+B$.  
%these faint objects. 
%This may be due to either the smaller %collecting 
%effective area %and 
%shorter  total exposure 
%of the {\sl HST} 
%and %effective 
%its filter transparencies    
%or time-variability  of some sources.
%caused  
% by the significantly  larger aperture of the VLT and the  longer total exposure of 7200 s. as compared with the HST.
At the same time, bright structures of the spiral galaxy are better resolved with the {\sl HST}.  

Photometric calibration of the VLT data was carried out using Landolt's standards PG1323--086, PG0231+051 and PG2331+055, observed at the same nights as the target; it resulted in the following  magnitude zero-points for the stacked images: $Z_U=25.08\pm0.04$,  $Z_B=27.55\pm0.05$ and $Z_V=28.00\pm0.01$,  
%
%$Z_U=25.076\pm0.034$,  $Z_B=27.553\pm0.048$ and $Z_V=28.00\pm0.01$, 
calculated for the fluxes in units of the CCD electron rate.  
%Using the mean atmosphere extinction coefficients $k_U=0.47\pm 0.01$,  $k_B=0.262\pm 0.001$ and $k_V=0.124\pm 0.007$ mag measured 
%for the VLT site at that epoch, 
%We obtained the following  magnitude zero-points for the stacked images: $Z_U=25.076\pm0.034$,  $Z_B=27.553\pm0.048$ and $Z_V=28.00\pm0.01$ calculated for the fluxes in units of the CCD electron rate. 
%The uncertainties include slight variations of the zero-points from night to night.   
%The results  are generally consistent with rough estimates  provided by the FORS1 quality control database for the  respective nights\footnote{See \url{http://archive.eso.org/qc1/qc1_cgi?action=qc1_browse_table&table=fors1_zp_night}.}. 
For photometry of the putative pulsar counterpart, we used apertures of 2.5 pixel in $U$ and 1.3 pixel  in $B$ (the pixel scale was 0\farcs2) which correspond to the enclosed energy fraction of 0.67 and 0.3, respectively, measured using 
bright unsaturated stars.
%in the images. 
Such small apertures were used 
%due to 
because of proximity of the bright spiral galaxy leading  to large systematic errors. We obtained  flux densities of the source of $44\pm22$ nJy   and $30\pm14$ nJy in the $U$ and $B$ bands, respectively, and a $3\sigma$ upper bound of 36 nJy in the $V$ band.  
%Within uncertainties 
 %they 
The fluxes are consistent, within the uncertainties,
with the results obtained by \cite{Mignani2008}. 
 %using the zero-points from the FORS1 quality control database.\va{??} 

To understand the nature of the putative VLT pulsar 
counterpart,  
%detected in the $U$ band at a $3\sigma$ level
%by \citet{Mignani2008},
we calculated the {\sl HST} upper bounds at its position in the VLT observations epoch
%marked by the yellow cross in Figures~\ref{fig:3} and \ref{fig:6} 
(Table~\ref{tab:upperbounds2}). 
%of the VLT counterpart candidate detected in the $U$ band \citep{Mignani2008}. The results are 
%and presented the results in Table~\ref{tab:upperbounds2}. 
%\gp{Isn't it strange/suspicious that all the three values of $C_{\rm pos}$ for the optical filters are lower at the VLT position than at the pulsar position at the HST observation epoch? The reason might be that a very faint pulsar counterpart is there (at the same position as the FUV candidate); it should be checked with the stacked UVIS image.
%[[Stacking did not yield a detection.]]}
%The flux density upper bounds are generally 
%deeper 
%lower at that position than at the pulsar position at the epoch of the {\sl HST} observations. This might be due to a lower brightness of the UVIS images around the position of the VLT pulsar counterpart candidate. 
%\gp{
%If this is true, it must be demonstrated by background measurements (
%Another reason might be that the bgd is lower around the `VLT pulsar position' [[indeed, the mean bgd is slightly lower, but the bgd variance in 2 of 3 filters is even higher at the `VLT position'; the upper bounds are lower at the VLT position]]
%}
The derived flux density 
$3\sigma$ upper bounds of 24 nJy in the F336W band and 26 
nJy in the F438W band 
are somewhat lower than the flux densities  in the VLT $U$ and $B$ bands, respectively, presented above.
%\footnote{They are also below those in Table~\ref{tab:upperbounds} mainly due 
%to the fact that local background near the spiral galaxy visible edge on is lower than in the pulsar vicinity  at the \textit{HST} observational %epoch.}. 
%which might be be caused  by the presence of unresolved objects 2, 3 and 8.
%and, $f_\nu = 44 \pm 22$ nJy 
%\gp{[should be remeasured]}
%at $\lambda\approx 3600$ \AA.  
%reported by \citet{Mignani2008}.
% fnu zeropint =1.79 10&^(-20) erg/sm2/s/Hz
%\gp{but the difference 49-31=18 nJy is just 1.5$\sigma$ of the VLT U ``measurement'', so these values do not look inconsistent? I am a bit confused, have to think, stopped here.} 
%On the one hand, 
This  implies that
%, most likely, 
the VLT object  has disappeared or moved out of its place 
by the epoch of the {\sl HST} observations, i.e., it is not a steady field object. 
%\gp{[This sentence might require a revision, depending on the remeasured U value.]}
%Meanwhile, 
On the other hand, our $3\sigma$ upper bounds of 37 nJy in the F336W band  and 52 nJy in the F438W band 
%derived for 
at the pulsar position at the {\sl HST} epoch (Table~\ref{tab:upperbounds}) are 
%consistent 
%with 
comparable to the $U$ and $B$ band flux densities\footnote{ Difference of the upper bounds 
at the two pulsar positions is due to different properties of the local backgrounds.}. 
%, which was actually also %derived  
%detected at a 3$\sigma$ significance.  
This suggests that the faint pulsar counterpart  
was indeed seen in  the $U$, $B$ and F140LP bands, 
and it 
could be seen in the {\sl HST} F336W and F438W bands 
if the exposures were 
just a factor of 1.5 longer. 

%At the same time, a hint of the VLT object was also observed in the $B$ band at the level of $f_\nu=30 \pm14$ nJy 
%\gp{[to be remeasured]}
%at $\lambda\approx 4400$ \AA\ %\va{$27^{+16}_{-10}$?} 
%\citep{Mignani2008}.
%Our $3\sigma$  upper bound in the 
%respective 
%F438W filter, $f_\nu < 24$ \gp{[45?]} nJy at $\lambda = 4326$ \AA,
%representing the {\sl HST} equivalent of the  $B$ band, is 
%apparently less 
%conclusive
%restrictive (40 nJy) to 
%state 
%does not allow us to confidently 
%conclude about the absence or presence of the $B$-band object at the {\sl HST} epoch. 

Thus, we cannot rule out the possibility that both the VLT ($U$ and $B$ bands) and {\sl HST} (F140LP band) detected the pulsar counterpart, but further deep observations are needed to prove it.




%\gp{This statement only makes sense if the VLT candidate's position was consistent with the current radio pulsar astrometry. It must be checked before finalizing this text.}

%It is hard to assume  
%that  the VLT and $HST$ in the $U$$B$ and FL140LP bands 
%detected the same object. It would have too unusual 
%colours either 
%for a pulsar or for a background ordinary star  with a high p.m. by 
%chance coinciding  with that of the pulsar.  
%Therefore the nature of the 
%VLT object remains unclear.     




%This upper bound exceeds by about $2\sigma$   
% the flux  density  $f_\nu =14.7\pm5.1$~nJy estimated above under the assumption that this is a real source.
 %%by about  
 %%2.1$\sigma$. 
 %At the same time a 1$\sigma$ upper bound of  
 %\approx$19.1 nJy is consistent with 
 %the derived flux within $\la1\sigma$ uncertainty. An apparent  difference is caused  by different algorithms of the background estimation resulting in  slightly different its count statistics. In the aperture photometry, it is  calculated as a mean count rate  per pixel in the 
 %annulus which than multiplied by the aperture area and results in ${C}_{\rm bgd}\approx8.4$ cts/ks. This is only slightly higher than $\overline{C}_{\rm bgd}$ calculated as a mean count rate number 
 %of a randomly selected  aperture areas around the pulsar position.         


%%%%%%%%%%%%%%%%%%%%%%%%%%%%% former Table 5 %%%%%%%%%%%%%%%%%%%%%
%\begin{deluxetable*}{ccccccccccc}[t!]
%\tablecaption{SBC photometry of the pulsar counterpart candidate using a different approach.  %\label{tab:photometry2}}
%\tablecolumns{8}
%%\tablenum{5}
%\tablewidth{Opt}
%\tablehead{
%%\colhead{Image}
%\colhead{Filter} &
%\colhead{$\lambda_{\rm piv}$} &
%\colhead{$t_{\rm exp}$} &
%\colhead{$r_{\rm extr}$} & \colhead{$\phi_{E}$} & \colhead{$C_{\rm pos}$} & \colhead{$\overline{C}_{bgd} \pm %\sigma_{\rm bgd}$} & \colhead{$N_{s}$} & \colhead{$C_{s}$} & \colhead{${\cal P}_{\nu}$} & \colhead{$f_{\nu}$}\\
%\colhead{ } & \colhead{(\AA)} & \colhead{(ks)} & \colhead{($''$)} & \colhead{(\%)} & \colhead{(cts/ks)} & %\colhead{(cts/ks)} & (cts) & \colhead{(cts/ks)} & \colhead{(nJy ks/cts)} & \colhead{(nJy)}
%}
%\startdata
%F140LP  & 1528 & 2800 & 0.425 & 77 & $15.4$ & $8.7 \pm 1.2$ & $18.8 \pm 5.5$ & $6.7 \pm 2.0$ & 1.61 & 14.0 $\pm$ 4.1
%\enddata
%\tablenotetext{a}{At exposure start.}
%\tablecomments{$t_{\rm exp}$ is the exposure time, 
%$N_{t}$ is the total number of counts in the $A_s=0.567$ arcsec$^2$ source aperture, 
%$C_{\rm pos}=N_{\rm tot}/t_{\rm exp}$ is the measured count rate in the source aperture,
%$\overline{C}_{\rm bgd} \pm \sigma_{\rm bgd}$ is the mean background and its standard deviation,
%$N_{s}$ is the number of source counts in the source aperture.
%The uncertainty of $N_s$ is $[(\sigma_{\rm bgd}\times t_{\rm exp})^2 + N_s]^{1/2}
%= 5.5$ cnts, which corresponds to the ${\rm S/N} = 3.4$.
%$C_{s}$ is the net source count rate,
%$\cal{P}_{\nu}$ is the conversion factor of the count rate to the flux density at the pivot wavelength, 
%$f_\nu$ is the aperture-corrected source flux density.}  
%\end{deluxetable*}
%%%%%%%%%%%%%%%%%%%%%%%%%%%%%%%%%%%%%%%%%%%%%%%%%





%%%%%%%%%%%%%%%%%%%%%%%%%%%%%%%%%%
\section{Discussion \label{disc}}
%%%%%%%%%%%%%%%%%%%%%%%%%%%%%%%%%%%%%
We likely detected the putative pulsar FUV counterpart in 
the F140LP band. Our photometric measurements
%together 
%combined with the ACS/SBC ETC predictions 
%including the flux density upper bound estimates,  
show that its brightness  is near the SBC detection threshold 
%by 
for the one-orbit {\sl HST} 
observation.
%s with this instrument. 
%Two or three 
Three {\sl HST} orbits would be 
%enough 
needed to confirm the FUV counterpart at a 
%4--5
$5\sigma$  significance. 
Nevertheless, the tentative detections with the {\sl HST} ACS/SBC and VLT FORS1 $U,B$ bands, combined 
%together 
with the 
%three 
upper bounds obtained in the UVIS 
(and VLT $V$) bands,   
%the VLT $UB$ tentative detection, and 
as well as the {\sl Chandra} and {\sl XMM-Newton} X-ray  data,  can provide interesting constraints on the optical-UV-X-ray spectral energy distribution (SED) of this old  pulsar.      
%To obtain this, we first have to 
%To constraint the SED, we first should  correct the observed flux densities  or their upper limits for the interstellar extinction. %absorption. 

%%%%%%%%%%%%%%%%%%%%%%%%%%%%% Table 8->7 %%%%%%%%%%%%%%%%%%%%%%%%%%
\begin{deluxetable*}{ccccccccccc}[t]
\tablecaption{
%Extinction  coefficients, 
De-reddened flux densities  
and $3\sigma$($1\sigma$)
%upper 
flux density  bounds of the pulsar counterpart (in nJy), and PL fit parameters,
%in all the bands where it was observed, 
%calculated 
for three values of 
%the color index 
$E(B-V)$.
%=[0.01, 0.02, 0.03]$
\label{tab:extinction}}
%\tablecolumns{8}
\tablenum{7}
%\tablewidth{Opt}
\tablehead{
 \colhead{$E(B-V)$} &
\colhead{$f_U$} &
\colhead{$f_B$} &
\colhead{$f_V$} & \colhead{$f_{\rm F140LP}$} & \colhead{$f_{\rm F225W}$} & \colhead{$f_{\rm F336W}$} & \colhead{$f_{\rm F438W}$} & \colhead{$\alpha$} & \colhead{$f_0$} }
\startdata
0.01 & $46 \pm 23$ & $31 \pm 14$ & $<37 (26)$ 
& $9.7\pm3.4$ & $<94 (54)$ & $<39 (21)$ & $< 54 (29)$ & $-0.71^{+0.74}_{-0.41}$ & $16.2^{+3.3}_{-7.1}$ \\
0.02 & $48 \pm 24$ & $33 \pm 16$ & $<38 (27)$
& $10.5 \pm 3.7$ & $<101 (58)$ & $<41 (22)$ & $<56 (30)$ & $-0.63^{+0.81}_{-0.45}$ & $16.7^{+3.5}_{-7.8}$ \\
0.03 & $50 \pm 25$ & $34 \pm 16$ & $<39 (27)$
& $11.3 \pm 4.0$ & $<109 (63)$ & $<43 (23)$ & $<58 (31)$ & $-0.60^{+0.77}_{-0.44}$ & $17.7^{+3.6}_{-8.2}$ \\
%\tablenotemark{a}\\
\enddata
\tablecomments{
%$A_{\lambda}$ 
Extinction values $A_\lambda$, used for de-reddening,  were calculated using using the extinction law from  \citet{Cardelli1989}.
HST upper bounds were calculated at the pulsar radio position, $\alpha$=01\h08\m08\fs397 and $\delta$=$-$14\degs31\amin51\farcs65, at the {\sl HST} observations epoch (MJD 57608).
%\gp{I added $\alpha$ and $f_0$ from the MCMC fits. Perphaps it would be better to write filter names as flux indices in tablehead, like $f_V$, $f_{\rm F140LP}$, etc.}
}
\end{deluxetable*}
%%%%%%%%%%%%%%%%%%%%%%%%%%%%%%%%%%%%

%%%%%%%%%%%%%%%%%%%%%%%%%%%%%%%%%%%%%%%%%%%%%%%%%
\subsection{Extinction towards PSR J0108--1431}
%%%%%%%%%%%%%%%%%%%%%%%%%%%%%%%%%%%%%%%%%%%%%%%
%The pulsar's  Galactic coordinates are $l= 140\fdg93$, $b=-76\fdg82$. 
%According to \citet{2011schlaf}, the total Galactic extinction toward J0108
%in this direction 
%is $A_V= 0.069$,
%\gp{no errors?}
%resulting in 
%corresponding to the 
%color excess  $E(B-V)=0.022$ (assuming $R_V=3.1$). The most recent Galactic 3D extinction map by \citet{Green2018} provides  a similar value, $E(B-V)=0.03 \pm 0.02$. The color excess grows with distance up to this value %with the  distance up to 
%at 430 pc and then remains  constant.
%Thus, 430 pc may indicate  the distance to the Galactic % gaseous \gp{Why gaseous? remove?} 
%disk edge in 
%this direction. 
%The distance to the pulsar based on the parallax measurements \citep{Deller2009}, corrected for the Lutz-Kelker bias \citep{2012verbi},  is $d=210^{+90}_{-50}$~pc. %\citep{2012verbi}.
%Within the uncertainty of this  distance 
According to the Galactic 3D extinction map by \citet{Green2018},
the color excess $E(B-V)$ varies between 0.00 and 0.04 within the uncertainty of the distance to the pulsar, %and gives a value of 
and $E(B-V)=0.02^{+0.02}_{-0.01}$ 
%\gp{why ``median''?}
for  $d=210$ pc.  
%\gp{I stopped here.}

The $E(B-V)$ value is correlated  %can be
%transformed  to the absorption 
with the %\sout{\bf X-ray absorbing} 
effective hydrogen column density $N_{\rm H}$
%the main parameter of the 
for X-ray photoelectric absorption models. 
Applying the empirical relation $N_{\rm H}=(0.7\pm0.1) \times 10^{22} E(B-V)$~cm$^{-2}$, obtained by \citet{Watson2011} for the Galaxy using observations of X-ray afterglows of a large number of $\gamma$-ray bursts, we 
%obtain 
expect $N_{\rm H}=1.4^{+1.8}_{-0.8}\times 10^{20}$~cm$^{-2}$.



Alternatively, $N_{\rm H}$
%the hydrogen column density 
can be estimated 
using the pulsar's dispersion measure, ${\rm DM} = 2.38$ pc\,cm$^{-3}$ and the correlation between  ${\rm DM}$ and $N_{\rm H}$ obtained by \citet{He2013}, 
 $N_{\rm H}= 0.30^{+0.13}_{-0.09} \times 10^{20} {\rm DM}$~cm$^{-2}$, which yields 
$N_{\rm H}=0.71^{+0.31}_{-0.21}\times 10^{20}$~cm$^{-2}$, in 
%accord 
agreement with the value derived from $E(B-V)$. 
This also overlaps with the $N_{\rm H}$ range of $(0.3-0.8)\times 10^{20}$~cm$^{-2}$ at $d=210$ pc obtained from 
the study of interstellar NaD absorption lines \citep{posselt2007,posselt2008}. 

Finally, the 
%rotation 
phase-integrated X-ray spectrum 
of the pulsar obtained with \textit{XMM-Newton} is 
most plausibly described by the absorbed power 
law (PL) plus blackbody (BB) model with $N_{\rm H} = 2.3^{+2.4}_{-2.3}\times 10^{20}$~cm$^{-2}$ \citep{Posselt2012,Arumugasamy2019}. The latter value is very uncertain but  consistent with the above estimates. 
%%%%%%%%%%%%%%%%%%%%%%% Figure 7 %%%%%%%%%%%%%%%%%%%%%%%
\begin{figure*}[t!]
\begin{minipage}[h]{0.6\linewidth}
\includegraphics[scale=0.23,angle=0]{Best002-7.PNG}
\end{minipage}
\hfill
\begin{minipage}[h]{0.4\linewidth}
\includegraphics[scale=0.25,angle=0]{HSTVLT002-11.jpg}
\end{minipage}
%\hfill
%\begin{minipage}[h]{0.5\linewidth}
%\includegraphics[scale=0.131,angle=0]{0108-002-1.PNG}
%\end{minipage}
%\hfill
%\begin{minipage}[h]{0.5\linewidth}
%\includegraphics[scale=0.131,angle=0]{0108-003-1.PNG}
%\end{minipage}
\caption{
{\sl Left:} 
Unabsorbed optical-UV SED  
of the pulsar counterpart candidate for $E(B-V)=0.02$. 
Red  and blue  error bars and downward arrows show  flux densities and $3\sigma$
upper bounds 
%(indicated by downward arrows)
measured with the {\sl HST} and VLT, respectively, with instruments and 
filters  indicated in the plot.
The dashed and solid lines correspond to the best fit of the data by the PL model and its 90\% credible  uncertainty, respectively.  
{\sl Right:} 2D and 1D marginal posterior 
probability distributions for the PL model parameters $\log f_0$ and $\alpha$ ($f_0$ in units of erg\,cm$^{-2}$\,s$^{-1}$\,Hz$^{-1}$).  
Vertical dashed lines in the 1D plots correspond 
to the 10\%, 25\%, 50\%, 75\%, and 90\% percentiles of the distributions. The contours in the 2D plot 
represent the  levels at 75\%, 50\%, and 25\% of the maximum probability value.
%\gp{[If it is not difficult, could you use italic fonts for $f$, $E(B-V$, and ``normal'' minus signs in $E(B-V)$ and $\alpha=-0.6$?]}
\label{fig:7}}
\end{figure*}
%%%%%%%%%%%%%%%%%%%%%%%%%%%%%%%%%%%%%%%%%%%%%%%%


All in all, we can accept  
$E(B-V)=0.01$--0.03 as the most probable color excess range for 
de-reddening of the optical-UV data and combining them consistently with the X-ray data. 
%obtained in X-rays. 
Using this color excess range and the extinction 
law from \citet{Cardelli1989},
 we can calculate the extinction 
%correction 
$A_{\lambda}$, and the  de-reddened source flux density  
or its upper bound for all the bands where the pulsar was observed. 
These quantities are  presented in Table~\ref{tab:extinction} and Figure~\ref{fig:7}.
%, {\bf  where we ignored the VLT $UB$ band upper bounds 
%of 2009 as they are consistent with detection in these bands of 2000 (see Section. 1). }  
%where 
%flux uncertainties include the measurement errors and 
%uncertainties of  $A_{\lambda}$  while upper bonds are calculated 
%using the maximal   $A_{\lambda}$  from its uncertainty range. 
%\gp{It is perhaps too much about $N_H$ in the above text, could be shortened unless it will be needed in further discussion.}

%%%%%%%%%%%%%%%%%%%%%%%%%%%%%%%%%%%%%%%%%%%%%%%%%%%%%%%%%%%%%%%%%%
\subsection{Multi-wavelength  spectrum of PSR J0108--1431}
%%%%%%%%%%%%%%%%%%%%%%%%%%%%%%%%%%%%%%%%%%%%%%%%%%%%%%%%%%%%5
%\subsubsection{Optical-UV spectral energy distribution}
%%%%%%%%%%%%%%%%%%%%%%%%%%%%%%%%%%%%%%%%%%%%%%%%%%%%%%%%%%%%%%%%%
Optical-UV emission from a rotation powered pulsar generally consists of two components, thermal and nonthermal. The nonthermal component, produced by relativistic particles in the pulsar magnetosphere, is usually described by a power-law (PL) model, $f_\nu \propto \nu^\alpha$, while the spectrum of the thermal component, emitted from the %neutron star 
NS surface, is close to a blackbody spectrum \citep{2011mignani}. In middle-aged %{\bf \sout{(10--100~kyr)}
(0.1--1~Myr) and moderately 
old (1--10~Myr) pulsars the nonthermal component dominates in the optical 
while the thermal component dominates in FUV 
\citep[e.g.,][]{2001kopts,2004zhar,2006shib,Kargaltsev2007, Pavlov2017},
%\gp{I would remove Zyuzin et al -- Vela is not a middle-aged pulsar (although shows some properties of it). You may add Koptsevich et al 2001 instead.}
%\sout{while the thermal component dominates in FUV} 
%\citep[e.g.,][]{Kargaltsev2007, Pavlov2017}.
%\gp{I see no point in separating the references. Please fix the parentheses...}
However, we know too little about optical-UV emissions of pulsars as old as PSR J0108--1431; the only other very old pulsar, J2144--3933 with the age of 300 Myr,
observed in the optical-UV, was not detected \citep{Guillot2019}. Therefore, we should explore various options and their connection with the results 
in X-rays where spectra of very old pulsars, including PSR J0108--1431,  
typically show only the PL component of the NS magnetosphere origin and a thermal component 
from small hot  spots  at the NS surface near magnetic pole regions  heated by relativistic particles generated 
in its magnetosphere. 
%\gp{The above paragraph may overlap with the Introduction, can be edited later}
%\gp{A few words about X-rays from old isolated pulsars ...?.}

%%%%%%%%%%%%%%%%%%%%%%%%%%%%%%%%%
\subsubsection{Possible power-law spectrum of the tentative pulsar counterpart}
%%%%%%%%%%%%%%%%%%%%%%%%%
   %%%%%%%%%%%%%%%%%%%%%%% Figure 8 %%%%%%%%%%%%%%%%%%%%%%%
\begin{figure*}[tbh!]
\begin{minipage}[h]{0.5\linewidth}
\includegraphics[scale=0.2,angle=0]{pn-mos1-opt-UV4-2.PNG}
\end{minipage}
\hfill
\begin{minipage}[h]{0.5\linewidth}
\includegraphics[scale=0.195,angle=0]{0108-X8.PNG}
\end{minipage}
\caption{
%\gp{Change left and right panels. Correct the caption.}
%{\sl Left:} 
Absorbed (left panel) and unabsorbed (right panel) multiwavelength spectra of the pulsar. The {\sl XMM-Newton} X-ray spectrum is described by a combination of a 
%power-law 
PL component with a photon index $\Gamma=2.35^{+0.35}_{-0.40}$ plus a 
%blackbody 
BB component with a temperature $kT=0.14^{+0.03}_{-0.02}$~keV, at a fixed hydrogen column density 
$N_{H}=1.4\times10^{20}$  cm$^{-2}$.
The dashed blue line in the left panel shows an extrapolation of the best-fit model spectrum to the optical-UV range, which strongly overshoots %while the dashed red line represents the best PL fit of 
the observed optical-UV data points  shown in this plot.  
%The best-fit X-ray model spectrum is shown by the black line, whereas its extrapolation in lower energies are shown by black dashed lines...
%{\sl Right:} Unabsorbed multiwavelength spectrum of the pulsar.
The red line in the right panel shows the best PL fit with $\alpha=-0.6$ to the optical-UV data points dereddened with $E(B-V)=0.02$, while black lines correspond to the 10\% and 90\% percentiles of the probability distribution 
of the optical-UV fit parameters. Blue dashed line shows the best-fit X-ray model spectrum, whereas black lines show its components and their uncertainties. 
%\gp{Would it be possible to have a better correspondence of line colors and types in the left and right panels?}
%\gp{[italic fonts....]}
\label{fig:8}}
\end{figure*}
%%%%%%%%%%%%%%%%%%%%%%%%%
Assuming both optical ($B$ and $U$) and FUV (F140LP) detections were real, we can fit the unabsorbed flux density points with a %single power law (PL): 
PL model: $f_\nu = f_0 (\nu/\nu_0)^\alpha$, where we choose the reference frequency $\nu_0 = 1\times 10^{15}$ Hz.
%(corresponding to $\lambda_0=3000$ \AA). 
%
%The optical-UV SEDs of ordinary rotation powered pulsars are mostly dominated by nonthermal emission created by relativistic particles in their magnetospheres \citep[e.g.,][]{Mignani2011}. It typically   described by the single  
%PL spectral model $F_{\nu} \propto \nu^{\alpha}$. Therefore,  assuming that we detected the pulsar in the $UB$ and F140LP bands, it is natural to try the PL model to fit its  SED.     
%If we assume VLT $U$ and $B$ detection to be real, the photometric data can be described by power  PL spectral model. 
Following the approach suggested by   \cite{Sawicki2012} and 
%later 
developed by  \cite{Drouart2018}, 
%in line with 
in addition to the three detected SED data points,
%in this fit, 
we also 
%handled  
included in the fits the four non-detections ($1\sigma$ upper bounds) 
%\gp{[why $3\sigma$?]}
in other filters presented in Table~\ref{tab:extinction}.  
%\yus{
Specifically, we used the  Python package {\tt Mr-Moose}  \citep{Drouart2018}, which allows data fitting in a Bayesian 
framework implementing the Markov chain Monte Carlo (MCMC) approach. 
To do that, we included FWHM of the {\sl HST} and VLT filters into the {\tt filters} directory of the {\tt Mr-Moose} distributive. 
For the MCMC convergence, we 
%typically 
utilized 
%about 
1000 walkers and 1000 steps.  
The fitting parameters were the spectral index $\alpha$ and the PL normalization $f_{0}$, which were
allowed to vary in wide ranges between $-1.7$  and 1.1 for $\alpha$ and between $-31.2$ and $-30.5$ 
    for  $\log f_0$ (where $f_0$ in units of erg\,cm$^{-2}$\,s$^{-1}$\,Hz$^{-1}$). 
    %$\nu_{0}=10^{15}$ Hz.  
%Minimization of  $\chi^2$, modified by adding a term accounting for the upper bounds, resulted in 
%a 
%flat 
%spectrum with the
Their best-fit values with uncertainties, corresponding to 10 and 90 percentiles of the parameter distribution obtained by cumulative integration of the posterior probability density function,
%derived with  the MCMC,  
are presented in Table~\ref{tab:extinction}.
%$\alpha \approx -0.274$ 
%and $f_{0} \approx 23$ nJy 
%for $E(B-V)=$ 0.01, $\alpha \approx -0.203$ for $E(B-V)=$ 0.02 and $\alpha \approx -0.198$ for $E(B-V)=$ 0.03.
%\gp{[Third digits in $\alpha$ are certainly meaningless, given the uncertainties.]}
%(reduced $\chi^2=0.97$ for 9 degrees of freedom).  
%It is 
Large uncertainties in the data 
result in large uncertainties of the PL parameters, and the parameter ranges  are almost insensitive to $E(B-V)$.
%though a marginal  steepening 
%of the model spectrum with $E(B-V)$ decrease can be noticed. 
As an example,  we show the fit results for $E(B-V)=0.02$ in Figure~\ref{fig:7}, where 
%where the {\sl left panel} 
%shows the de-reddened data points and the PL fit in the log-scale.
the right panel 
%\gp{Why do you use slanted fonts for left/right panels? Is it required by ApJ nowadays? I doubt. If not, please remove `sl' here and throughout the paper.} 
shows 1D and 2D marginalized posterior probability distributions of the parameters reflecting  the fit convergence and quality. %We checked, that  
The fit results %within uncertainties  
practically do not depend 
on the {\sl HST}/UVIS flux upper bounds, while they are  critically affected by the VLT upper bound in the $V$ band.   
We also tried to include into the fit  the VLT $U$ and $B$ 
%\gp{[I added +. Okay?]}
band upper bounds 
of 2009, which  are consistent with the detections in these bands of 2000 (see Sec.\ 1).   This led  to a marginal flattening of the best-fit PL. For instance, we obtained $\alpha \approx -0.57$ instead of $-0.63$ without these upper bounds, for $E(B-V)=0.02$. Such a small difference is insignificant accounting for the large error budget (see Table~\ref{tab:extinction}).


 The obtained constraints on the spectral index of the pulsar's nonthermal emission  are compatible with (broader than) 
 a typical range
 $-0.7\lesssim \alpha\lesssim 0.2$
 %$\alpha$ range of 0$-$-0.5  
 for other pulsars observed in the optical-UV range \citep[e.g.,][]{2011mignani,2019mignani}.  
The best-fit PL parameters correspond to the optical-UV flux 
 %$F({\rm 1500-6000\,\AA}) \sim 3\times 10^{-16}$ erg cm$^{-2}$ s$^{-1}$
 $F({\rm 1500-6000\,\AA}) = 2.3\times 10^{-16}$ erg cm$^{-2}$ s$^{-1}$ and luminosity 
 %$L({\rm 1500-6000\,\AA}) \sim 1.2\times 10^{26} d_{\rm 210\,pc}^2$ erg s$^{-1}$,
 $L({\rm 1500-6000\,\AA}) = 1.2\times 10^{27} d_{\rm 210\,pc}^2$ erg s$^{-1}$. Comparing the latter value with  the FUV luminosities 
 %\sout{list}
 of pulsars 
 %\sout{ever} 
 that have been detected 
 in this range  \citep{2019mignani},
 we see that J0108--1431 might be  the least luminous FUV pulsar.  
 %This is  the lowest luminosity for pulsars ever detected in  the FUV \citep{2019mignani}. 
 %and efficiency $\eta_{\rm opt-UV} = L({\rm 1500-6000\,\AA})/\dot{E} \sim 2.4\times 10^{-5} d_{\rm 210\,pc}^2$.
 On the other hand, the efficiency of 
 %transformation 
 conversion of pulsar rotation energy to optical-UV radiation is $\eta_{\rm opt-UV} = L({\rm 1500-6000\,\AA})/\dot{E} = 2.4\times 10^{-4} d_{\rm 210\,pc}^2$. 
 Accounting for the distance and fit uncertainties,
 %the latter value lies in a range of  (0.8--6.4)$\times 10^{-4}$ 
 %$\eta_{\rm opt-UV}$ 
 %the optical-UV 
 the efficiency could be anywhere in the range
 $0.7\lesssim \eta_{\rm opt-UV}/10^{-4} \lesssim 5.9$.
 %the efficiency value 
 %may be within a range of
 %%$\eta_{\rm opt-UV} \sim
 %(0.8--$6.4)\times 10^{-4}$. 
 Even for lower values from this range, 
 %making 
 PSR J0108--1431 is the most efficient nonthermal   emitter  
 among 
 %compared to other 
 pulsars %higher than typical efficiencies of younger  
  detected in the optical-UV. 
  For instance, based on the data obtained by \citet{Pavlov2017}, the 17 Myr old  PSR B0950$+$08 shows  about two orders of magnitude 
  %smaller 
  lower $\eta_{\rm opt-UV}$  of about  %$5.3
  $6\times 10^{-6}$ in the same range.
  %\gp{This is not a good example because Pavlov et al 2017 interpreted the FUV emission as thermal. }
%  \gp{Taking only the PL component with $\alpha=-1.17$, $f_0=47$ nJy, I got $F(1500-6000)=3.95\times 10^{-16}$, $L(1500-60000) = 3.3\times10^{26}$, $\eta=6\times 10^{-7}$. Is it wrong? 
 % That $\alpha$ is not realistic anyway.}\yus{yes,using your flux the luminosity is  %=3.95e-16*4.0*3.14195*(262.0*3.09e18)**2
%3.2536879430068E27 !}
  A typical optical efficiency range of pulsars is $10^{-7}$--$10^{-5}$, only the very young and much more 
  energetic Crab and B0540--69 pulsars have  efficiencies  comparable  to the 
  %presumed one 
  above estimate for J0108--1431 \citep{2006zhar,2010mpk,2015kir}.
 % \citep[e.g.][]{2010mpk, .  For the middle-aged pulsars B0656$+$14, Geminga, and B1055$-$52,  the efficiencies  are even lower, %and lie in a range of  
  %$\sim (0.1$--$1.0)\times10^{-6}$  \citep[e.g.][]{2010mpk}. Only very young and much more energetic Crab and B0540--69 pulsars have  efficiencies \citep{2006zhar,2015kir} comparable  to the 
  %presumed one 
  %above estimate for J0108--1431. 
  %\gp{I would omit the specific examples, just would provide refs, perhaps the range of efficiencies, e.g., $10^{-6}$--$10^{-5}$.}
  
  On the other hand, the pulsar is also a highly efficient nonthermal emitter in X-rays with 
  $\eta_{\rm X} \sim$ 0.003--0.006  \citep{Posselt2012}. %Here, the upper bound corresponds to 
  %inclusion of the thermal emission from pulsar polar caps heated by back particle current 
  %from its magnetosphere.   
 % \gp{[0.006 if thermal PC emission included]}
  This results in the ratio of the nonthermal optical-UV to the X-ray luminosity 
  $L({\rm 1500-6000\,\AA})/L_{X} \sim 0.02$--0.04,
  marginally compatible with a typical range of 0.001--0.01 
  for pulsars observed 
  in the optical-UV and X-rays \citep{2004zp,2006zhar}. 
%   This fact can be considered as an additional argument for realness of our optical and UV detections, and it
%  further supports identification of the source with an optical-UV counterpart of PSR J0108--1431. \va{???}
 % favour the real identification of J0106--1431 in the optical-UV.      
  %the spectral slope support  the pulsar magnetosphere  origin of the detected optical-UV emission. 
  % is likely to be dominated by the long-wavelength part of the nonthermal radiation from the pulsar which is also seen in X-rays. 
%  \gp{I don't understand this sentence.}
  %\gp{It is reasonable to include this paragraph in the paper, but maybe after the paragraphs about X-rays?}
  
  
 
 
% \gp{
 %[This values are for $\alpha=-0.14$, $f_0=20$ nJy; 
 %[it might be useful to crudely estimate uncertainties.]}
 %\va{[These values are for $\alpha=-0.2$, $f_0=20$ nJy]}
 % This efficiency is higher than typical values of ???-??? for younger pulsars ...
 % \yus{$L$ and $\eta$ values in literature  are typically for 4000--9000 \AA~range or for the  B-band, I would recalculate for 4000--9000 \AA.}
 % \gp{[1500 - 6000 \AA\ may be more logical because it is approximately the range in which we observed, and in which we assume the spectrum is a PL. If we want to compare efficiency values, not just orders of magnitude, we can introduce (maybe in addition to $\eta_{\rm opt-UV}$) $\eta_{\rm opt}= L({\rm 4000-9000\, \AA})/\dot{E} \sim 0.8\times 10^{-4}$] }
  

  
  %Given that, [it is poor English]
  It is interesting to compare the optical-UV spectrum of the pulsar candidate with the 
  X-ray 
  spectrum of PSR J0108--1431 obtained with \textit{XMM-Newton} \citep{Posselt2012, Arumugasamy2019}. 
  The latter presumably consists of a magnetospheric PL component 
  and a thermal component emitted from 
  hot polar caps. % at the surface of the neuron star 
  %\citep{Posselt2012, Arumugasamy2019}.
  Using the XSPEC tool (ver.\ 12.11.0)\footnote{See \url{https://heasarc.gsfc.nasa.gov/docs/xanadu/xspec}}, we fit
  %Fitting 
  the time-integrated spectra obtained  by %the EPIC instrument 
  {\sl XMM-Newton} EPIC-pn and MOS1 instruments in the 0.2--10 keV range  with the absorbed PL+BB model at fixed $N_{\rm H}=1.4\times 10^{20}$ cm$^{-2}$, corresponding to $E(B-V)=0.02$.
  %using the XSPEC  tool v 12.11.0 
  We obtained the photon index $\Gamma = 2.35^{+0.35}_{-0.40}$ (i.e., $\alpha_X = -\Gamma +1 = -1.35^{+0.35}_{-0.40} $), the PL normalization 
  $(1.4
  %^{+0.35}_{-0.35} 
  \pm 0.4)\times 10^{-6}$ ph cm$^{-2}$ s$^{-1}$ keV$^{-1}$ at $E=1$ keV, 
  the temperature $kT_{\rm BB} = 0.14^{+0.03}_{-0.02}$
  %0.14^{+0.03}_{-0.02}$ 
  keV 
 % \gp{[the number of digits after decimal point should be the same in the quantity and the error]}
  ($T_{\rm BB}=1.6^{+0.3}_{-0.2}$ MK),
  and 
  the BB radius % of the emitting area
  %\gp{[it is actually not the radius of an area but what I used to call the radius of equivalent sphere; to avoid explanations, I would simply call it the BB radius.]}
  $R_{\rm BB}= 22^{+23}_{-18} d_{210}$~m ($\chi_\nu^2 \approx 0.9$ for $\nu=20$ d.o.f). 
  The fit parameters are consistent, within the uncertainties (quoted above at a $1\sigma$ confidence level),
  with those 
  obtained by \citet{Posselt2012}, who %where the authors 
  fixed photon index (at $\Gamma=2.0$) instead of $N_H$. 
  %
  The obtained temperature is 
%  \sout{consistent with} 
  within the range of typical BB temperatures of hot polar caps of old pulsars, while the BB radius is 
 % \sout{by}
  a factor of 10 smaller than the ``canonical'' cap radius $R_{\rm pc} = R_{\rm NS} (2 \pi R_{\rm NS}/cP)^{1/2} \approx 238$~m for $P=0.808$~s, 
  assuming a 
  plausible intrinsic NS radius 
  %of an NS, 
  $R_{\rm NS}=13$~km.  
  %\sout{expected for this pulsar}. 
  %\gp{[give an equation.]} 
  %\sout{Similar   situation}
  A similar discrepancy between $R_{\rm BB}$ and $R_{\rm pc}$ is observed in other old pulsars \citep[see][and references therein]{2020geppert,Posselt2012}. 
  %\gp{May need editing.}
  %
  Figure \ref{fig:8}  
  shows the fit, for the absorbed and unabsorbed spectra, together with the PL fit to the optical-UV spectrum shown in Figure \ref{fig:7}. We see that the continuation of the X-ray model spectrum dominated by the PL component into the optical-UV lies well above the optical-UV detections and upper limits. %, while the X-ray points are within the uncertainties of the continuation of %the (rather uncertain) 
  %the optical-UV PL spectrum. 
  If we associate the optical-UV and X-ray PL components with the pulsar's magnetosphere emission, we can conclude that its spectrum steepens with increasing photon energy towards X-rays and  
  has a spectral break somewhere between the UV and soft X-rays,
  %optical through X-ray SED cannot  be described by a single PL model. Instead, the nonthermal spectral component %the nearly flat optical-UV spectrum 
  %steepens with increasing photon energy towards X-rays, 
  similar to other (younger) pulsars that were detected in both X-rays and the optical \citep[e.g.,][]{Kargaltsev2007}. 
  %Since the pulsar is rather faint in X-rays, the spectral parameters are very uncertain. 
 % According to Figure \ref{fig:8}, the spectral break for J0108--1431 
  %is expected near a photon energy $E_{\rm br}\approx 0.2$~keV ($4.5\times 10^{16}$~Hz). 
  %Using the XSPEC, we 
  %fit the optical-UV and X-ray data  simultaneously  with an absorbed  broken PL model with tied $E(B-V)$ and $N_H$ parameters and got $E_{\rm br}=(0.53 \pm 0.06)$~keV, with %$\Gamma$ 
  %spectral indices below and above  the break consistent with those obtained from the separate optical-UV and X-ray fits.
  %spectral fits of the optical-UV and X-ray data. 
  However,  better quality data in both ranges are needed to make a convincing multi-wavelength spectral analysis.  
  
  %Also, we estimated the multi-wavelength 
  %optical through X-ray  
  %efficiency of the pulsar in the 1 eV -- 10 keV range,  $\approx 1.5\times 10^{-2}$, i.e., about twice higher than 
  %in X-rays only. Such high efficiencies are observed  for pulsars only in $\gamma$-rays \citep{2013Abdo}. 
  %However, the spin-down luminosity of PSR J0108--1431, $\dot{E} = 5.1\times 10^{30}$ erg s$^{-1}$, 
  %is well below an empirical detection threshold of $\sim  10^{32}$ erg s$^{-1}$  for pulsars so far detected with {\textit Fermi}. 
  Based on the high optical through X-ray efficiency of J0108--1431, one can speculate 
  that for old pulsars the highest efficiency range 
  migrates from $\gamma$-rays towards lower photon energies. 
 
  
%\yus{I stopped here} 
%\gp{I edited this subsubsection slightly}
  
  

  
  %%%%%%%%%%%%%%%%%%%%%%%%%
  \subsubsection{Possibility of thermal emission in FUV and limits on surface temperature}
  
  %Although the possibly detected optical-UV emission of the pulsar is well described by the PL model, the flux uncertainties are so large that other models that include a thermal component can also fit the observed points.
  %Including thermal emission in the fits allows one to constrain the NS surface temperature, which was the main goal of these observations.
  The main goal of these observations was to constrain the NS surface temperature.
  First of all, we note that if the optical emission allegedly detected with the VLT  were thermal, then
  the FUV flux should 
  be much higher than either its presumably measured value or the upper bound, at any reasonable temperature and size of the emitting NS surface. 
 Therefore, we can rule out the temperature estimate by \citet{Mignani2008} derived from the assumption that the optical spectrum is a Rayleigh-Jeans part of thermal emission from the entire NS surface.

%\citet{Mignani2008} suggested that the VLT $UBV$ photometry of the possible pulsar counterpart can be interpreted as the Rayleigh-Jeans part of the thermal emission from the entire surface of the neutron star with the temperature of $(7-10)\times 10^4(D_{130}/R_{13})$ K, where $D_{130}$ and $R_{13}$ are the distance and NS radius in 130~pc and 13~km units, respectively. This interpretation is obviously ruled out now, if the $UB$ and F140LP photometric data points all represent the pulsar counterpart \gp{and the FUV detection is real}. The F140LP data point strongly declines from the suggested Rayleigh-Jeans law. This is true even if we consider this point as an upper limit, but not as the real detection.   

On the other hand, it is possible that the FUV emission is due, at least partly, to a thermal component. If the $U$ and $B$ detections 
are associated with the pulsar, 
%were real, 
then the relatively low upper limit on the $V$ flux does not allow a steep negative slope of the PL component, which leaves little room for the thermal component in F140LP.
If, however, the $U$ and $B$ detections are not associated with the pulsar
%were not real 
while the F140LP detection is,
%real while the $U$ and $B$ detections were not, 
then the F140LP flux  could be entirely thermal. 
As seen from Figure~\ref{fig:8}, it cannot come from  hot polar caps of the pulsar seen in X-rays, but can only come from a cooler bulk surface of the NS.
Assuming that the
spectrum of the NS surface emission 
%of the NS  
is described by the Planck function,
$B_\nu(T) = (2h\nu^3/c^2)[\exp(h\nu/kT)-1]^{-1}$, 
its brightness temperature can be estimated 
from the observed flux density, %equation
\begin{equation}
f_\nu = (R_\infty/d)^2 \pi B_\nu(T_\infty) 10^{-0.4 A_\nu}\,,
\label{eq1}
\end{equation}
where 
%$d=210 d_{210}$ pc,
$T_\infty=T/(1+z)$~K  and  $R_\infty=R(1+z)$~km  are the NS 
temperature and radius
%, respectively, 
as measured by a distant observer, $z=[1-2.953 %M_{NS}
(M/M_\odot)(1\,{\rm km}/R)]^{-1/2} - 1$
%R_{NS}(km)]^{-1/2}$ 
is the gravitational redshift.
%; for a most reasonable NS with the mass  $M_{NS}=1.4M_\odot$ and the circumferential radius $R_{NS}=13$~km $1+z=1.21$  and $R_\infty=15.8$~km. Adopting the most plausible  $E(B-V)=0.02$, Eq.(\ref{eq1}) transforms to 
For the F140LP filter ($\nu_{\rm piv} = 1.96\times 10^{15}$ Hz, $A_\nu = 8.15  E(B-V)$), 
%\gp{[I think it is better to provide an explicit equation for temperature]
%the bightness temperature can be estimated from the equation
we have
\begin{equation}
T_\infty = \frac{9.42\times 10^4\,{\rm K}}{\ln\left[1 + \frac{187\,{\rm nJy}}{f_{\rm F140LP}}\left(\frac{R_{15}}{d_{210}}\right)^2 10^{-3.26 E(B-V)}\right]}\,,
\end{equation}
where  
%$T_5=T_\infty/10^5\,{\rm K}$,
$R_{15}=R_\infty/15\,{\rm km}$, and $d_{210}= d/210\,{\rm pc}$.
%}

%Equation (\ref{eq1}) can be written as
%\begin{equation}
%f_{\rm F140LP}  = 
%\frac{187\,{\rm nJy}}{\exp(0.942/T_5)-1}\left(\frac{R_{15}}{d_{210}}\right)^2 10^{-3.26 E(B-V)}, 
%\label{eq2}
%\end{equation}
%which is suitable for  practical use;
%where  $T_5=T_\infty/10^5\,{\rm K}$, $R_{15}=R_\infty/15\,{\rm km}$, and $d_{210}= d/210\,{\rm pc}$. 
%%%%%%%%%%%%%%%%%%%%%%% Figure 9 %%%%%%%%%%%%%%%%%%%%%%%
\begin{figure*}[t]
\begin{minipage}[h]{0.5\linewidth}
\includegraphics[scale=0.2,angle=0]{0108-T-002-10.PNG}
\end{minipage}
\hfill
\begin{minipage}[h]{0.5\linewidth}
\includegraphics[scale=0.2,angle=0]{0108-XT-2.PNG}
\end{minipage}
\caption{
Limits on 
thermal emission from the NS surface.
{\sl Left:} 
Unabsorbed optical-UV blackbody spectrum assuming the F140LP flux is fully thermal, for
%Black lines show unabsorbed blackbody spectra emitted by the NS surface with 
%typical $R=13$ km, $M=1.4 M_{\odot}$ (
$R_{\infty}=15.8$ km, 
%at distance 
 $d=210$ pc, %\citep{Verbiest2012} 
$E(B-V)=0.02$, and surface temperature $T_\infty=3.1^{+0.3}_{-0.4}\times10^4$ K (the temperature uncertainties 
correspond to the flux uncertainties). %\gp{$^+$ is not well seen and $_-$ too short in the $3.7^{+0.4}_{-0.5}$. Is it possible to improve?}\gp{Now this minus looks long, while minus in $E(B-V)$ too short...}
{\sl Right:} Dereddened optical-UV thermal spectra extrapolated towards higher energies, and the PL+BB fit to the {\sl XMM-Newton} data (the same as shown in Figure \ref{fig:8}). The blackbody spectrum from the entire NS surface shown by  black lines 
%show blackbody spectra emitted by the NS surface with $R=10$ km, $M=1.4 M_{\odot}$ 
%($R_{\infty}=15.8$ km) 
%at the distance 
is plotted for $d=300$ pc, $R_\infty = 13.1$ km, %(corresponding to $R=10$ km at $M=1.4 M_\odot$),
$E(B-V)=0.03$,  $T_\infty = 4.8^{+0.7}_{-0.8}\times 10^4$ K. The red line corresponds to
a conservative upper limit of the NS surface temperature,
$T_\infty < 5.9\times 10^4$ K, when we consider the F140LP data point as an upper limit, for the same $d$, $R_\infty$ and $E(B-V)$.
%\gp{[italics...]}
\label{fig:9}}
\end{figure*}
%%%%%%%%%%%%%%%%%%%%%%%%%%%%%%%%%%%%%%%%%%%%%%%%



For plausible $d=210$ pc, $E(B-V)=0.02$, and $R_\infty = 15.8$ km (which correspond to $R=13$ km at $M=1.4 M_\odot$), the measured FUV flux density
$f_{\rm F140LP} = 9.0\pm 3.2$ nJy 
%\gp{Right?}
yields $T_\infty = 3.1^{+0.3}_{-0.4} \times 10^4$  K ($T=3.8^{+0.3}_{-0.5}\times10^4$ K). %\gp{added $T$ here and below, they can be removed later if not needed.}
%\gp{[$3.7^{+0.4}_{-0.5}$ ??]}
%(see Figure \ref{fig:9}, %left panel). 
%Using Eq.(\ref{eq2}) and the measured FUV flux density
%$f_{\rm F140LP} = 15\pm 5$ nJy we obtain  $T_\infty = 3.7^{+0.4}_{-0.3} \times 10^4$ K.
The 
%de-reddened 
optical-UV part of this thermal spectrum with its uncertainties   
%with its uncertainties 
is shown 
in the  left panel of Figure \ref{fig:9}.
In the  right panel of this figure, we show the dereddened thermal spectrum at an upper end of plausible distances, $d=300$ pc, and 
a lower end of plausible radii,  $R_\infty = 13.1$ km  (corresponding to $R=10$ km at $M=1.4 M_\odot$ \citep{LattPrak2016} for the 
upper bound of the color index, 
$E(B-V)=0.03$; these parameters, correspond to a higher temperature, $T_\infty = 4.8^{+0.7}_{-0.8} \times 10^4$ K ($T=6.3^{+0.8}_{-1.1} \times 10^4$ K).
The spectrum is extrapolated   
%An extrapolation of these spectra 
towards 
X-rays, where the unabsorbed PL+BB fit of the {\sl XMM-Newton} spectrum of the pulsar   
is also shown. %in the {\sl right panel} of this figure. 
%It is seen 
We see that emission from the NS surface 
%of the NS possible visible in the FUV is 
%too cool to 
with such a temperature would not be detectable in  the optical and X-rays.
If we assume that the F140LP detection was not real, then
the $3\sigma$ upper bound on the FUV flux density, $f_{\rm F140LP} < 14$ nJy, gives us an upper limit on the
temperature -- e.g., $T_\infty < 5.9\times 10^4$ K ($T<7.7\times 10^4$ K) at $d=300$ pc, $R_\infty = 13.1$ km, $E(B-V)=0.03$ (the corresponding thermal spectrum is shown by the red line in the right panel of Figure \ref{fig:9}). 


It is interesting to compare the obtained constraints on the surface temperature of the 
 196
%166 \gp{196 or 200 - corrected for Shklovskii effect} 
Myr old PSR J0108--1431 with those for other old pulsars. Based on the upper limit, $T_\infty \lesssim 6\times 10^4$ K, we can conclude that PSR J0108--1431 is colder than the 17 
%\gp{18 -- 3 digits too much}
Myr old PSR B0950+08, the only old ordinary pulsar whose thermal emission has been detected, with $T_\infty$ in the range of
%$T_\infty \approx
(1--$3)\times 10^5$ K \citep{Pavlov2017}. 

If the F140LP detection of thermal emission from PSR J0108--1431 was real, then 
%the estimated surface temperature of the 200 Myr old PSR J0108--1431, 
the plausible temperature range, 
$T_\infty \approx (2.7$--$5.5)\times 10^4$ K, is just slightly above the conservative 
upper limit, $T_\infty \lesssim 3.2\times 10^4$ K, for the 330 Myr old PSR J2144--3933 \citep{Guillot2019}\footnote{\citet{Guillot2019} present the upper limit $T\approx 4.2\times 10^4$ K %\gp{I would round some temperatures, both here and in the text} 
for 
unredshifted temperature assuming an NS with $R=10$~km and $M=1.4 M_\odot$.}. 
The highest temperature of this range is lower than the surface temperature, 
$T_\infty \approx (1.2-3.5)\times 10^5$~K,  
of the 7 Gyr old  nearest millisecond (recycled) pulsar 
(MSP) J0437--4715 \citep{2004karg,2012durant,2019gonzales}.
It means that either thermal evolution of NSs is not monotonous or it proceeded differently for J0108--1431 and J0437--4715, 
e.g., because  old ordinary and recycled pulsars 
have some different properties, including periods and their derivatives, magnetic field strengths and masses \citep{2015gonz}. 

Confirmation of the possible detection of thermal emission from J0108--1431 would mean that it is the coldest NS whose thermal emission has been detected, but 
%\sout{the still high temperature would}
its temperature is still high enough to support the idea that NSs do not just cool passively 
but some heating mechanisms strongly affect their thermal evolution. 
According to passive cooling scenarios, cooling of isolated NSs %born very hot at supernova explosions,
%begin to cool 
becomes exponentially fast
at ages of  a few  Myr after which  thermal emission from their surfaces %should soon become   invisible at any electromagnetic range
becomes undetectable \citep[e.g.,][]{2004yp}. 
However, this rapid cooling can be partly  compensated by a number of heating mechanisms.
One of them is the so-called rotochemical heating due to composition changes (such as the neutron beta decay) forced by density increase as the centrifugal force decreases in the course of NS spindown 
%deviations from beta equilibrium inside the star due to its contraction with the rotational energy loss  inducing 
%reactions that release heat \citep[so called rotochemical heating,]
%\citep[e.g.,][]{2010GR}. 
\citep{
1995reiseng,2005fr}.
This heating mechanism has a minor effect on surface temperatures of young NSs but its contribution can  dominate at ages $\gtrsim 10$ Myr. 
%Additional friction heating can also be provide by superfluid vortex creep in the matter inside the NS.
Another important mechanism is ``frictional heating'' caused by interaction of vortex lines of the faster rotating neutron superfluid with the slower rotating normal matter in the inner NS crust \citep{1984Alpar,1999ll}.


%According to these scenarios and 

According to the top panel of  Figure 5 of \citet{Guillot2019}, the
upper bound on the (unredshifted) surface temperature of PSR J0108--1431, $T < 8\times 10^4$ K,
is consistent with 
%lies 
%very close (
%slightly above 
the values predicted by the models of rotochemical heating by \citet{2010GR} for a 200 Myr old pulsar with the surface magnetic field of $2.4\times 10^{11}$ G and initial period at birth of 1 ms, assuming either modified Urca reactions or direct Urca reactions with additional frictional heating with excess angular momentum $J=1\times 10^{44}$ erg s (the predicted temperatures are very close to each other at these parameters). 
%The models become better constrained 
This is similar to the younger PSR B0950+08. 
However, if the FUV thermal emission was actually detected, then the observed temperature range $T\sim (3$--$7)\times 10^4$ K 
lies between  these predictions and the low temperature  boundary provided by the direct Urca models 
without frictional heating. This would mean that  frictional heating is less 
efficient in PSR J0108--1431 than in  PSR B0950+08.
%then the predicted temperatures for the three models are within the observed range,
%(unredshifted 
%temperature 
%$T\sim (3$--$7)\times 10^4$ K, similar to the younger PSR B0950+08, while the direct Urca models without frictional heating are 
%firmly 
%excluded. 
To obtain tighter constraints on heating mechanisms, one should re-observe PSR J0108--1431 with deeper exposures as well as observe more old pulsars in the optical-UV.


%\yus{[[of the Myr old pulsar with a magnetic field PSR J0108--1431 of a few  times $10^4$ could be likely provided by  
%rotochemical heating by modified Urca reactions without involving  direct Urca reactions and frictional mechanisms, if its initial period was about a few ms. This is in contrast to PSR B0950$+$08 where the frictional heating appears to be substantial to explain the data. ]]  
%}  

%The estimated surface temperature of PSR J0108--1431 is considerably 
%  lower than the plausible temperature range $1\leq T_5 \leq 3$ derived for the 17.5 Myr old ordinary PSR B0950$+$08 
%\citep{2017Pavlov}    and can be below the upper limit $ T_5 \leq 0.42$ obtained for  about twice older, $\sim 3\times 10^8$ yr,   ordinary pulsar J2124--3933 \citep{Guillot2019}.  For the most plausible parameters,  J0108--1431 is the coolest NS known. 
%On the other hand, its temperature lies well above the values predicted for its age by scenarios of  
%NSs passively cooling after supernova explosions where they were born. 

%As for  PSR B0950$+$08 \citep{2017Pavlov}, this requires some heating mechanisms working inside the star.   According to %\citet{2010GR}, this can be the rotochemical heating cased by chemical inbalance during pulsar spindown and  involving direct or modified Urca processes and  superfluid  vortex friction inside the NS ...



\section{Conclusions}

We observed the field of the nearby  
%166 
196 Myr 
old PSR J0108$-$1431 with the {\sl HST} in four optical-UV bands. 
We detected a point-like FUV source in the 
F140LP band at about 3$\sigma$ significance
level with coordinates coinciding with 
the position of the pulsar within the 1$\sigma$ uncertainty
of 0\farcs2. 
%accounting for its the known p.m.  
We consider this source as a possible FUV counterpart of 
PSR J0108$-$1431. Also, we placed upper limits on 
the flux densities of the pulsar in the F225W, 
F336W and F438W bands. 
Using more accurate astrometry, we confirmed the 
%likely 
$3\sigma$ detection of the optical source at the pulsar position in year 2000 in the VLT {\bf $U+B$} filters, 
%at the same significance level 
 and its upper limit in the $V$ filter,
 %flux
%density upper limit in the VLT $V$ band 
reported 
earlier by \citet{Mignani2008}.    

Assuming that the possibly detected F140LP, $U$ and $B$ emission comes from the pulsar counterpart,
%possible counterpart is real, 
we analyzed its 
multi-wavelength 
%time-integrated 
spectral energy distribution,
 including the 
archival X-ray data obtained with 
\textit{XMM-Newton}. We found that
the spectral flux density distribution can be described by a %nearly flat 
%spectral 
power-law model in the optical-UV part, suggesting its magnetospheric origin.
%emission of the pulsar has almost a flat 
%nonthermal spectrum of the pulsar magnetosphere 
%origin which 
The spectrum becomes steeper in X-rays, implying %suggesting  
a spectral break between the UV and X-ray ranges. 
%near a photon energy of 
%near 0.2--0.5 keV. 
Such 
%breaks are 
behavior is typical for pulsars observed in both ranges. 
The pulsar has a record high efficiency, $\eta\sim 10^{-2}$, of 
transformation of the 
%rotation 
%energy loss to the nonthermal 
spindown power to nonthermal (magnetosperic) radiation 
in the optical-UV through X-rays.
%$\sim 10^{2}$. 
%Such high efficiencies have been observed 
%in younger pulsars only in $\gamma$-rays. \gp{Is it true? Remove this sentence?}

In the FUV band, the pulsar emission 
%can 
might be dominated by thermal emission 
from the bulk of the NS surface.
%of the NS. 
If this is the case,
%We constrained its 
the NS surface temperature is
in the range of (3--6)$\times 10^4$~K,
as seen by a distant observer 
-- the lowest NS temperature ever measured.
%for  NSs. 
At the same time, it is much higher 
than predicted by scenarios of passive NS cooling at this NS age,
%and 
%can be explained by the rotochemical
which could be due to heating mechanisms operating in the NS interiors. A conservative consideration of the FUV data point as  an  upper bound yields the 3$\sigma$   upper limit on the NS  brightness temperature,   $T_\infty < 6\times 10^4$~K. 
%working inside the pulsar. 

%{\bf 4. We still cannot firmly exclude the possibility that
%neither FUV nor optical  detections of PSR J0108-1431 are  real.
%If this is the case, then the derived upper bound  
%on the flux density in the F140LP filter 
%rules out the previous surface temperature estimates, based on the VLT data only, and set  a new 
%constraint on the brightness temperature, $T_\infty < 7.3\times 10^4$~K, of the bulk surface of this NS. }
%\gp{[I am still not 100\% sure we need this paragraph, but I have no strong objections either -- up to you. Is the numeration of paragraphs really needed here?]}  

Detection of 
%the pulsar 
PSR J0108--1431 in the optical and 
%particularly in the FUV
UV bands at higher significance levels is needed to confirm the counterpart and study its properties.
%of  this  very interesting pulsar.  

%{\yus I stopped here}
%\gp{I edited the Conclusions slightly. I think we can stop here.}


%This is true if even we consider F140LP data point as an upper limit but no as real detection.
%

%I stopped here
%\gp{I made some corrections and editing, checked the numbers.}

%In this case the brightness temperature can be estimated 
%from the equation
%\gp{I'm not sure it is really needed here...}
%\begin{equation}
%f_\nu = (R_\infty/d)^2 \pi %B_\nu(T_\infty) 10^{-0.4 A_\nu} = 
%\frac{187\,{\rm nJy}}{\exp(0.942/T_5)-1}\left(\frac{R_{15}}{d_{210}}\right)^2
%\end{equation}
%where $d=210 d_{210}$ pc, $B_\nu(T_\infty) = (2h\nu^3/c^2)[\exp(h\nu/kT_\infty)-1]^{-1}$ is the Planck function, the temperature $T_\infty = 10^5 T_5$ K and the radius $R_\infty = 15 R_{13}$ are as measured by a distant observer. ...

%we can consider a  possibility that the VLT detected an unrelated object and 
%consider the thermal hypothesis based only on the F140LP data point.  (VLT upper limits can be deeper at this position? Did this make sense ? ...)



%Another assumption is that VLT object is unrelated to the pulsar, 


%%%%%%%%%%%%%%%%%%%%%%%

%\subsubsection{Comparison with thermal emission from other old pulsars. What we know about thermal evolution of old neutron stars.}

%\gp{This would not be a separate subsubsection?}




%% An example figure call using \includegraphics


%% An example table using AASTeX's deluxetable. Note that since
%% only one figure OR one table is allowed this is commented out.
%\begin{deluxetable}{ccl}
%\tablecaption{Example table some English and Greek letters\label{tab:1}}
%\tablehead{
%\colhead{Index number} & \colhead{English} & \colhead{Greek}
%}
%\startdata
%1 & a & alpha ($\alpha$) \\
%2 & b & beta ($\beta$) \\
%3 & c & gamma ($\gamma$) \\
%4 & d & delta ($\delta$) \\
%5 & e & epsilon ($\epsilon$) \\
%\enddata
%\tablecomments{Long tables should only show a short example with the full
%version as a machine readable table with the article.}
%\end{deluxetable} 


%items somewhere (I guess after the acknowledgements, pls check).

\acknowledgments
 We thank the referee 
for useful comments. %allowing us to improve the paper.}  
Support for \textit {HST}
program \#14249 was provided by NASA through a grant from
the Space Telescope Science Institute, which is operated by the
Association of Universities for Research in Astronomy, Inc.,
under NASA contract NAS 5-26555.
We thank Bettina Posselt for providing the reduced {\sl XMM-Newton} data. RPM is grateful to Denise Taylor (STScI) for support during the \textit{HST}  observations.
%This work has made use of data from the European Space Agency (ESA) mission {\it Gaia} 
%(\url{https://www.cosmos.esa.int/gaia}), processed by the {\it Gaia} Data Processing and Analysis Consortium 
%(DPAC; \url{https://www.cosmos.esa.int/web/gaia/dpac/consortium}).  Funding for the DPAC has been provided by national institutions, 
%in particular the institutions participating in the {\it Gaia} Multilateral Agreement. 
%The scientific results reported in this article are partly based on data obtained from  observations obtained with \textit{ XMM-Newton}, an ESA science mission with instruments and contributions directly funded by ESA Member States and the USA (NASA).
%\gp{Do we want to thank people, %\yus{(Bettina, if she is not co-author here)} other grants?}


\facilities{\textit{HST}/WFC3; \textit{HST}/ACS; VLT/FORS; \textit{XMM-Newton}; \textit{Gaia}}

\software{This research made use of the following softwares and packages: IRAF \citep{1986Tody, Tody1993}, XSPEC \citep{1996Arnaud}, EsoRex \citep{2015ESO}, Mr-Moose  \citep{Drouart2018}}

\bibliographystyle{aasjournal}
\bibliography{rnaas}

   


\end{document}
