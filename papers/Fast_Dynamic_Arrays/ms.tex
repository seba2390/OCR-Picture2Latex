%%% NO LIPICS

\documentclass{article}
%\usetikzlibrary{matrix}
%\usetikzlibrary{positioning}
\usepackage{amssymb}
\usepackage{amsmath}

\usepackage[utf8]{inputenc}
\usepackage[a4paper, total={426pt, 674pt}]{geometry}
\usepackage{hyperref}

\usepackage{graphicx}
\usepackage{subcaption}
\newtheorem{theorem}{Theorem}
\newtheorem{lemma}{Lemma}
\newtheorem{corollary}{Corollary}
\newtheorem{definition}{Definition}
\newtheorem{prop}{Proposition}
\newenvironment{proof}{\noindent\emph{Proof. }}

\usepackage{enumitem}
\bibliographystyle{plain}

\date{}
\author{Philip Bille\thanks{Supported by the Danish Research Council (DFF -- 4005-00267, DFF -- 1323-00178)}\\ \texttt{phbi@dtu.dk}
    \and Anders Roy Christiansen\thanks{Supported by the Danish Research Council (DFF -- 4005-00267)}\\ \texttt{miet@dtu.dk}
    \and Mikko Berggren Ettienne$^\dag$\\ \texttt{miet@dtu.dk}
\and Inge Li G{\o}rtz$^\dag$\\ \texttt{inge@dtu.dk}}

% LIPICS ONLY

%\documentclass[a4paper,UKenglish]{lipics-v2016}

%\bibliographystyle{plainurl}% the recommended bibstyle

%\author[1]{Philip Bille}
%%\thanks{Supported by the Danish Research Council (DFF -- 4005-00267, DFF -- 1323-00178)}
%\author[1]{Anders Roy Christiansen}
%%\thanks{Supported by the Danish Research Council (DFF -- 4005-00267)}
%\author[1]{Mikko Berggren Ettienne}
%%\thanks{Supported by the Danish Research Council (DFF -- 4005-00267)}
%\author[1]{Inge Li G{\o}rtz}
%\affil[1]{The Technical University of Denmark\\
%\texttt{\{phbi,aroy,miet,inge\}@dtu.dk}}
%\authorrunning{P. Bille, A.\,R. Christiansen, M.\,B. Ettienne and I.\,L. G{\o}rtz}

%\Copyright{Philip Bille, Anders Roy Christiansen, Mikko Berggren Ettienne and Inge Li G{\o}rtz}


%\subjclass{F.2.2 Nonnumerical Algorithms and Problems,  E.1 Data Structures}
%\keywords{Dynamic Arrays, Tiered Vectors}

%Editor-only macros:: begin (do not touch as author)%%%%%%%%%%%%%%%%%%%%%%%%%%%%%%%%%%
%\EventEditors{Kirk Pruhs and Christian Sohler}
%\EventNoEds{2}
%\EventLongTitle{25th Annual European Symposium on Algorithms (ESA 2017)}
%\EventShortTitle{ESA 2017}
%\EventAcronym{ESA}
%\EventYear{2017}
%\EventDate{September 4--6, 2017}
%\EventLocation{Vienna, Austria}
%\EventLogo{}
%\SeriesVolume{87}
%\ArticleNo{23}

%END LIPICS ONLY

\graphicspath{{./graphics/}}
\usepackage{tikz}

\newcommand{\ceil}[1]{\left\lceil{#1}\right\rceil}
\newcommand{\floor}[1]{\left\lfloor{#1}\right\rfloor}
\renewcommand{\angle}[1]{\langle{#1}\rangle}

\newcommand{\suc}{\ensuremath\mathrm{succ}}
\newcommand{\pos}{\ensuremath\mathrm{pos}}
\newcommand{\level}{\ensuremath\mathrm{level}}

\newcommand{\aaccess}{\ensuremath\mathsf{access}}
\newcommand{\ainsert}{\ensuremath\mathsf{insert}}
\newcommand{\adelete}{\ensuremath\mathsf{delete}}
\newcommand{\aupdate}{\ensuremath\mathsf{update}}
\newcommand{\ashift}{\ensuremath\mathsf{shift}}

\newcommand{\offset}{\ensuremath\mathrm{off}}
\newcommand{\capacity}{\ensuremath\mathrm{cap}}
\newcommand{\childcapacity}{\ensuremath\mathrm{ccap}}
\newcommand{\height}{\ensuremath\mathrm{height}}
\newcommand{\elements}{\ensuremath\mathrm{elems}}
\newcommand{\children}{\ensuremath\mathrm{children}}


\newcommand{\PP}{\ensuremath\mathrm{PP}}
\renewcommand{\SS}{\ensuremath\mathrm{SS}}
\newcommand{\Next}{\ensuremath\mathrm{Next}}

\newcommand{\str}{\ensuremath{S} }
\newcommand{\tree}{\ensuremath{T}}
\newcommand{\slp}{\ensuremath{\mathcal{S}} }
\newcommand{\kq}{\ensuremath{k_{\str, q}} }
\newcommand{\qg}{\ensuremath{G_q(\str)} }
\newcommand{\fca}{\ensuremath{\textsc{firstcolor}}}
\newcommand{\lca}{\ensuremath{\textsc{lastcolor}}}
\newcommand{\nca}{\ensuremath{\textsf{nca}}}
\newcommand{\labsuc}{\ensuremath{\textsc{ls}}}
\newcommand{\la}{\ensuremath{\textsc{la}}}
\newcommand{\leftc}{\ensuremath{\text{\textit{left}}}}
\newcommand{\rightc}{\ensuremath{\text{\textit{right}}}}
\newcommand{\band}{\ensuremath{\wedge}}
\newcommand{\bor}{\ensuremath{\vee}}
\newcommand{\bxor}{\ensuremath{\oplus}}
\newcommand{\hf}{\ensuremath{\mathcal{H}}}
\newcommand{\pat}{\ensuremath{P}}
\newcommand{\fingerprintq}{\ensuremath{\textsc{Fingerprint}}}
\newcommand{\lceq}{\ensuremath{\textsc{LCE}}}
\newcommand{\bm}{\ensuremath{\mathcal{B}}}
\newcommand{\chunk}{\ensuremath{X}}


\title{Fast Dynamic Arrays}



\begin{document}

\maketitle

\abstract{We present a highly optimized implementation
    of tiered vectors, a data structure
    for maintaining a sequence of $n$ elements
    supporting access in time $O(1)$
    and insertion and deletion in time $O(n^\epsilon)$ for $\epsilon > 0$
    while using $o(n)$ extra space.
    We consider several different implementation optimizations 
    in C++ and compare their performance to that of \text{vector} and
    \text{multiset}
    from the standard library on sequences with up to
    $10^8$ elements.
    Our fastest implementation uses
    much less space than multiset 
    while providing speedups of $40\times$ for
    access operations compared to multiset
    and speedups of $10.000\times$ compared
    to vector for insertion and deletion operations
    while being competitive
    with both data structures for all other
    operations.
}


\section{Introduction}

% \leavevmode
% \\
% \\
% \\
% \\
% \\
\section{Introduction}
\label{introduction}

AutoML is the process by which machine learning models are built automatically for a new dataset. Given a dataset, AutoML systems perform a search over valid data transformations and learners, along with hyper-parameter optimization for each learner~\cite{VolcanoML}. Choosing the transformations and learners over which to search is our focus.
A significant number of systems mine from prior runs of pipelines over a set of datasets to choose transformers and learners that are effective with different types of datasets (e.g. \cite{NEURIPS2018_b59a51a3}, \cite{10.14778/3415478.3415542}, \cite{autosklearn}). Thus, they build a database by actually running different pipelines with a diverse set of datasets to estimate the accuracy of potential pipelines. Hence, they can be used to effectively reduce the search space. A new dataset, based on a set of features (meta-features) is then matched to this database to find the most plausible candidates for both learner selection and hyper-parameter tuning. This process of choosing starting points in the search space is called meta-learning for the cold start problem.  

Other meta-learning approaches include mining existing data science code and their associated datasets to learn from human expertise. The AL~\cite{al} system mined existing Kaggle notebooks using dynamic analysis, i.e., actually running the scripts, and showed that such a system has promise.  However, this meta-learning approach does not scale because it is onerous to execute a large number of pipeline scripts on datasets, preprocessing datasets is never trivial, and older scripts cease to run at all as software evolves. It is not surprising that AL therefore performed dynamic analysis on just nine datasets.

Our system, {\sysname}, provides a scalable meta-learning approach to leverage human expertise, using static analysis to mine pipelines from large repositories of scripts. Static analysis has the advantage of scaling to thousands or millions of scripts \cite{graph4code} easily, but lacks the performance data gathered by dynamic analysis. The {\sysname} meta-learning approach guides the learning process by a scalable dataset similarity search, based on dataset embeddings, to find the most similar datasets and the semantics of ML pipelines applied on them.  Many existing systems, such as Auto-Sklearn \cite{autosklearn} and AL \cite{al}, compute a set of meta-features for each dataset. We developed a deep neural network model to generate embeddings at the granularity of a dataset, e.g., a table or CSV file, to capture similarity at the level of an entire dataset rather than relying on a set of meta-features.
 
Because we use static analysis to capture the semantics of the meta-learning process, we have no mechanism to choose the \textbf{best} pipeline from many seen pipelines, unlike the dynamic execution case where one can rely on runtime to choose the best performing pipeline.  Observing that pipelines are basically workflow graphs, we use graph generator neural models to succinctly capture the statically-observed pipelines for a single dataset. In {\sysname}, we formulate learner selection as a graph generation problem to predict optimized pipelines based on pipelines seen in actual notebooks.

%. This formulation enables {\sysname} for effective pruning of the AutoML search space to predict optimized pipelines based on pipelines seen in actual notebooks.}
%We note that increasingly, state-of-the-art performance in AutoML systems is being generated by more complex pipelines such as Directed Acyclic Graphs (DAGs) \cite{piper} rather than the linear pipelines used in earlier systems.  
 
{\sysname} does learner and transformation selection, and hence is a component of an AutoML systems. To evaluate this component, we integrated it into two existing AutoML systems, FLAML \cite{flaml} and Auto-Sklearn \cite{autosklearn}.  
% We evaluate each system with and without {\sysname}.  
We chose FLAML because it does not yet have any meta-learning component for the cold start problem and instead allows user selection of learners and transformers. The authors of FLAML explicitly pointed to the fact that FLAML might benefit from a meta-learning component and pointed to it as a possibility for future work. For FLAML, if mining historical pipelines provides an advantage, we should improve its performance. We also picked Auto-Sklearn as it does have a learner selection component based on meta-features, as described earlier~\cite{autosklearn2}. For Auto-Sklearn, we should at least match performance if our static mining of pipelines can match their extensive database. For context, we also compared {\sysname} with the recent VolcanoML~\cite{VolcanoML}, which provides an efficient decomposition and execution strategy for the AutoML search space. In contrast, {\sysname} prunes the search space using our meta-learning model to perform hyperparameter optimization only for the most promising candidates. 

The contributions of this paper are the following:
\begin{itemize}
    \item Section ~\ref{sec:mining} defines a scalable meta-learning approach based on representation learning of mined ML pipeline semantics and datasets for over 100 datasets and ~11K Python scripts.  
    \newline
    \item Sections~\ref{sec:kgpipGen} formulates AutoML pipeline generation as a graph generation problem. {\sysname} predicts efficiently an optimized ML pipeline for an unseen dataset based on our meta-learning model.  To the best of our knowledge, {\sysname} is the first approach to formulate  AutoML pipeline generation in such a way.
    \newline
    \item Section~\ref{sec:eval} presents a comprehensive evaluation using a large collection of 121 datasets from major AutoML benchmarks and Kaggle. Our experimental results show that {\sysname} outperforms all existing AutoML systems and achieves state-of-the-art results on the majority of these datasets. {\sysname} significantly improves the performance of both FLAML and Auto-Sklearn in classification and regression tasks. We also outperformed AL in 75 out of 77 datasets and VolcanoML in 75  out of 121 datasets, including 44 datasets used only by VolcanoML~\cite{VolcanoML}.  On average, {\sysname} achieves scores that are statistically better than the means of all other systems. 
\end{itemize}


%This approach does not need to apply cleaning or transformation methods to handle different variances among datasets. Moreover, we do not need to deal with complex analysis, such as dynamic code analysis. Thus, our approach proved to be scalable, as discussed in Sections~\ref{sec:mining}.

\section{Preliminaries}

%!TEX root = hopfwright.tex
%

In this section we systematically recast the Hopf bifurcation problem in Fourier space. 
We introduce appropriate scalings, sequence spaces of Fourier coefficients and convenient operators on these spaces. 
To study Equation~\eqref{eq:FourierSequenceEquation} we consider Fourier sequences $ \{a_k\}$ and fix a Banach space in which these sequences reside. It is indispensable for our analysis that this space have an algebraic structure. 
The Wiener algebra of absolutely summable Fourier series is a natural candidate, which we use with minor modifications. 
In numerical applications, weighted sequence spaces with algebraic and geometric decay have been used to great effect to study periodic solutions which are $C^k$ and analytic, respectively~\cite{lessard2010recent,hungria2016rigorous}. 
Although it follows from Lemma~\ref{l:analytic} that the Fourier coefficients of any solution decay exponentially, we choose to work in a space of less regularity. 
The reason is that by working in a space with less regularity, we are better able to connect our results with the global estimates in \cite{neumaier2014global}, see Theorem~\ref{thm:UniqunessNbd2}.


%
%
%\begin{remark}
%	Although it follows from Lemma~\ref{l:analytic} that the Fourier coefficients of any solution decay exponentially, we choose to work in a space of less regularity, namely summable Fourier coefficients. This allows us to draw SOME MORE INTERESTING CONCLUSION LATER.
%	EXPLAIN WHY WE CHOOSE A NORM WITH ALMOST NO DECAY!
%	% of s Periodic solutions to Wright's equation are known to be real analytic and so their  Fourier coefficients must decay geometrically [Nussbaum].
%	% We do not use such a strong result;  any periodic solution must be continuously differentiable, by which it follows that $ \sum | c_k| < \infty$.
%\end{remark}


\begin{remark}\label{r:a0}
There is considerable redundancy in Equation~\eqref{eq:FourierSequenceEquation}. First, since we are considering real-valued solutions $y$, we assume $\c_{-k}$ is the complex conjugate of $\c_k$. This symmetry implies it suffices to consider Equation~\eqref{eq:FourierSequenceEquation} for $k \geq 0$.
Second, we may effectively ignore the zeroth Fourier coefficient of any periodic solution \cite{jones1962existence}, since it is necessarily equal to $0$. 
%In \cite{jones1962existence}, it is shown that if $y \not\equiv -1$ is a periodic solution of~\eqref{eq:Wright} with frequency $\omega$, then $ \int_0^{2\pi/\omega} y(t) dt =0$. 
		The self contained argument is as follows. 
		As mentioned in the introduction, any periodic solution to Wright's equation must satisfy $ y(t) > -1$ for all $t$. 
	By dividing Equation~\eqref{eq:Wright} by $(1+y(t))$, which never vanishes, we obtain
	\[
	\frac{d}{dt} \log (1 + y(t)) = - \alpha y(t-1).
	\]  
	Integrating over one period $L$ we derive the condition 
	$0=\int_0^L y(t) dt $.
	Hence $a_0=0$ for any periodic solution. 
	It will be shown in Theorem~\ref{thm:FourierEquivalence1} that a related argument implies that we do not need to consider Equation~\eqref{eq:FourierSequenceEquation} for $k=0$.
\end{remark}

%%%
%%%
%%%\begin{remark}\label{r:c0} 
%%%In \cite{jones1962existence}, it is shown that if $y \not\equiv -1$ is a periodic solution of~\eqref{eq:Wright} with frequency $\omega$, then $ \int_0^{2\pi/\omega} y(t) dt =0$. 
%%%PERHAPS TOO MUCH DETAIL HERE. The self contained argument is as follows.
%%%If $y \not\equiv -1$ then $y(t) \neq -1$ for all $t$, since if $y(t_0)=-1$ for some $t_0 \in \R$ then $y'(t_0)=0$ by~\eqref{eq:Wright} and in fact by differentiating~\eqref{eq:Wright} repeatedly one obtains that all derivatives of $y$ vanish at $t_0$. Hence $y \equiv -1$ by Lemma~\ref{l:analytic}, a contradiction. Now divide~\eqref{eq:Wright} by $(1+y(t))$, which never vanishes, to obtain
%%%\[
%%%  \frac{d}{dt} \log |1 + y(t)| = - \alpha y(t-1).
%%%\]  
%%%Integrating over one period we obtain $\int_0^L y(t) dt =0$.
%%%\end{remark}



%Furthermore, the condition that $y(t)$ is real forces $\c_{-k} = \overline{\c}_{k}$.  
%
We define the spaces of absolutely summable Fourier series
\begin{alignat*}{1}
	\ell^1 &:= \left\{ \{ \c_k \}_{k \geq 1} : 
    \sum_{k \geq 1} | \c_k| < \infty  \right\} , \\
	\ell^1_\bi &:= \left\{ \{ \c_k \}_{k \in \Z} : 
    \sum_{k \in \Z} | \c_k| < \infty  \right\} .
\end{alignat*} 
We identify any semi-infinite sequence $ \{ \c_k \}_{k \geq 1} \in \ell^1$ with the bi-infinite sequence $ \{ \c_k \}_{k \in \Z} \in \ell^1_\bi$ via the conventions (see Remark~\ref{r:a0})
\begin{equation}
  \c_0=0 \qquad\text{ and }\qquad \c_{-k} = \c_{k}^*. 
\end{equation}
In other word, we identify $\ell^1$ with the set
\begin{equation*}
   \ell^1_\sym := \left\{ \c \in \ell^1_\bi : 
	\c_0=0,~\c_{-k}=\c_k^* \right\} .
\end{equation*}
On $\ell^1$ we introduce the norm
\begin{equation}\label{e:lnorm}
  \| \c \| = \| \c \|_{\ell^1} := 2 \sum_{k = 1}^\infty |\c_k|.
\end{equation}
The factor $2$ in this norm is chosen to have a Banach algebra estimate.
Indeed, for $\c, \tilde{\c} \in \ell^1 \cong \ell^1_\sym$ we define
the discrete convolution 
\[
\left[ \c * \tilde{\c} \right]_k = \sum_{\substack{k_1,k_2\in\Z\\ k_1 + k_2 = k}} \c_{k_1} \tilde{\c}_{k_2} .
\]
Although $[\c*\tilde{\c}]_0$ does not necessarily vanish, we have $\{\c*\tilde{\c}\}_{k \geq 1} \in \ell^1 $ and 
\begin{equation*}
	\| \c*\tilde{\c} \| \leq \| \c \| \cdot  \| \tilde{\c} \| 
	\qquad\text{for all } \c , \tilde{\c} \in \ell^1, 
\end{equation*}
hence $\ell^1$ with norm~\eqref{e:lnorm} is a Banach algebra.

By Lemma~\ref{l:analytic} it is clear that any periodic solution of~\eqref{eq:Wright} has a well-defined Fourier series $\c \in \ell^1_\bi$. 
The next theorem shows that in order to study periodic orbits to Wright's equation we only need to study Equation~\eqref{eq:FourierSequenceEquation} 
for $k \geq 1$. For convenience we introduce the notation 
\[
G(\alpha,\omega,\c)_k=
( i \omega k + \alpha e^{ - i \omega k}) \c_k + \alpha \sum_{k_1 + k_2 = k} e^{- i \omega k_1} \c_{k_1} \c_{k_2} \qquad \text{for } k \in \N.
\]
We note that we may interpret the trivial solution $y(t)\equiv 0$ as a periodic solution of arbitrary period.
\begin{theorem}
\label{thm:FourierEquivalence1}
Let $\alpha>0$ and $\omega>0$.
If $\c \in \ell^1 \cong \ell^1_{\sym}$ solves
$G(\alpha,\omega,\c)_k =0$  for all $k \geq 1$,
then $y(t)$ given by~\eqref{eq:FourierEquation} is a periodic solution of~\eqref{eq:Wright} with period~$2\pi/\omega$.
Vice versa, if $y(t)$ is a periodic solution of~\eqref{eq:Wright} with period~$2\pi/\omega$ then its Fourier coefficients $\c \in \ell^1_\bi$ lie in $\ell^1_\sym \cong \ell^1$ and solve $G(\alpha,\omega,\c)_k =0$ for all $k \geq 1$.
\end{theorem}

\begin{proof}	
	If $y(t)$ is a periodic solution of~\eqref{eq:Wright} then it is real analytic by Lemma~\ref{l:analytic}, hence its Fourier series $\c$ is well-defined and $\c \in \ell^1_{\sym}$ by Remark~\ref{r:a0}.
	Plugging the Fourier series~\eqref{eq:FourierEquation} into~\eqref{eq:Wright} one easily derives that $\c$ solves~\eqref{eq:FourierSequenceEquation} for all $k \geq 1$.

To prove the reverse implication, assume that $\c \in \ell^1_\sym$ solves
Equation~\eqref{eq:FourierSequenceEquation} for all $k \geq 1$. Since $\c_{-k}
= \c_k^*$, Equation \eqref{eq:FourierSequenceEquation} is also satisfied for
all $k \leq -1$. It follows from the Banach algebra property and
\eqref{eq:FourierSequenceEquation} that $\{k \c_k\}_{k \in \Z} \in \ell^1_\bi$,
hence $y$, given by~\eqref{eq:FourierEquation}, is continuously differentiable.
% (and by bootstrapping one infers that $\{k^m c_k \} \in \ell^1_\bi$, 
% hence $y \in C^m$ for any $m \geq 1$).
	Since~\eqref{eq:FourierSequenceEquation} is satisfied for all $k \in \Z \setminus \{0\}$ (but not necessarily for $k=0$) one may perform the inverse Fourier transform on~\eqref{eq:FourierSequenceEquation} to conclude that
	$y$ satisfies the delay equation 
\begin{equation}\label{eq:delaywithK}
   	y'(t) = - \alpha y(t-1) [ 1 + y(t)] + C
\end{equation}
	for some constant $C \in \R$. 
   Finally, to prove that $C=0$ we argue by contradiction.
   Suppose $C \neq 0$. Then $y(t) \neq -1$ for all $t$.
   Namely, at any point where $y(t_0) =-1$ one would have $y'(t_0) = C$
   which has fixed sign,   hence it would follow that $y$ is not periodic
   ($y$ would not be able to cross $-1$ in the opposite direction, 
   preventing $y$  from being periodic).  
  We may thus divide~\eqref{eq:delaywithK} through by $1 + y(t)$ and obtain 
\begin{equation*}
	\frac{d}{dt} \log | 1 + y(t) | = - \alpha y(t-1) + \frac{C}{1+y(t)} .
\end{equation*}
	By integrating both sides of the equation over one period $L$ and by using that $\c_0=0$, we 
	obtain
	\[
	 C \int_0^L \frac{1}{1+y(t)} dt =0.
	\]
	Since the integrand is either strictly negative or strictly positive, this implies that $C=0$. Hence~\eqref{eq:delaywithK} reduces to~\eqref{eq:Wright},
	and $y$ satisfies Wright's equation. 
\end{proof}






To efficiently study Equation~\eqref{eq:FourierSequenceEquation}, we introduce the following linear operators on $ \ell^1$:
\begin{alignat*}{1}
   [K \c ]_k &:= k^{-1} \c_k  , \\ 
   [ U_\omega \c ]_k &:= e^{-i k \omega} \c_k  .
\end{alignat*}
The map $K$ is a compact operator, and it has a densely defined inverse $K^{-1}$. The domain of $K^{-1}$ is denoted by
\[
  \ell^K := \{ \c \in \ell^1 : K^{-1} \c \in \ell^1 \}.  
\]
The map $U_{\omega}$ is a unitary operator on $\ell^1$, but
it is discontinuous in $\omega$. 
With this notation, Theorem~\ref{thm:FourierEquivalence1} implies that our problem of finding a SOPS to~\eqref{eq:Wright} is equivalent to finding an $\c \in \ell^1$ such that
\begin{equation}
\label{e:defG}
  G(\alpha,\omega,\c) :=
  \left( i \omega K^{-1} + \alpha U_\omega \right) \c + \alpha \left[U_\omega \, \c \right] * \c  = 0.
\end{equation}


%In order for the solutions of Equation \ref{eq:FHat} to be isolated we need to impose a phase condition. 
%If there is a sequence $ \{ c_k \} $ which satisfies  Equation \ref{eq:FHat}, then $ y( t + \tau) = \sum_{k \in \Z} c_k e^{ i k \omega (t + \tau)}$ satisfies Wright's equation at parameter $\alpha$. 
%Fix $ \tau = - Arg[c_1] / \omega$ so that $ c_1  e^{ i \omega \tau} $ is a nonnegative real number. 
%By Proposition \ref{thm:FourierEquivalence1} it follows that $\{ c'_k \} =  \{c_k e^{ i \omega k \tau }   \}$ is a solution to Equation \ref{eq:FHat}, and furthermore that $ c'_1 = \epsilon$ for some $ \epsilon \geq 0$. 


Periodic solutions are invariant under time translation: if $y(t)$ solves Wright's equation, then so does $ y(t+\tau)$ for any $\tau \in \R$. 
We remove this degeneracy by adding a phase condition. 
Without loss of generality, if $\c \in \ell^1$ solves Equation~\eqref{e:defG}, we may assume that $\c_1 = \epsilon$ for some 
\emph{real non-negative}~$\epsilon$:
\[
  \ell^1_{\epsilon} := \{\c \in \ell^1 : \c_1 = \epsilon \} 
  \qquad \text{where } \epsilon \in \R,  \epsilon \geq 0.
\]
In the rest of our analysis, we will split elements $\c \in \ell^1$ into two parts: $\c_1$ and $\{\c_{k}\}_{k \geq 2}$.  
We define the basis elements $\e_j \in \ell^1$ for $j=1,2,\dots$ as
\[
  [\e_j]_k = \begin{cases}
  1 & \text{if } k=j, \\
  0 & \text{if } k \neq j.
  \end{cases}
\]
We note that $\| \e_j \|=2$. 
Then we can decompose
% We define
% \[
%   \tilde{\epsilon} := (\epsilon,0,0,0,\dots) \in \ell^1
% \]
% and
% For clarity when referring to sequences $\{c_{k}\}_{k \geq 2}$, we make the following definition:
% \[
% \ell^1_0  := \{ \tc \in \ell^1 : \tc_1 = 0 \}.
% \]
% With the
any $\c \in \ell^1_\epsilon$ uniquely as
\begin{equation}\label{e:aepsc}
  \c= \epsilon \e_1 + \tc \qquad \text{with}\quad 
  \tc \in \ell^1_0 := \{ \tc \in \ell^1 : \tc_1 = 0 \}.
\end{equation}
We follow the classical approach in studying Hopf bifurcations and consider 
$\c_1 = \epsilon$ to be a parameter, and then find periodic solutions with Fourier modes in $\ell^1_{\epsilon}$.
This approach rewrites the function $G: \R^2 \times \ell^K \to \ell^1$ as a function $\tilde{F}_\epsilon : \R^2 \times \ell^K_0 \to \ell^1$, where 
we denote 
\[
\ell^K_0 := \ell^1_0 \cap \ell^K.
\]
% I AM ACTUALLY NOT SURE IF YOU WANT TO DEFINE THIS WITH RANGE IN $\ell^1$
% OR WITH DOMAIN IN $\ell^1_0$ ?? IT SEEMS TO DEPEND ON WHICH GLOBAL STATEMENT YOU WANT/NEED TO MAKE!?
\begin{definition}
We define the $\epsilon$-parameterized family of  functions $\tilde{F}_\epsilon: \R^2 \times \ell^K_0  \to \ell^1$ 
by 
\begin{equation}
\label{eq:fourieroperators}
\tilde{F}_{\epsilon}(\alpha,\omega, \tc) := 
\epsilon [i \omega + \alpha e^{-i \omega}] \e_1 + 
( i \omega K^{-1} + \alpha U_{\omega}) \tc + 
\epsilon^2 \alpha e^{-i \omega}  \e_2  +
\alpha \epsilon L_\omega \tc + 
\alpha  [ U_{\omega} \tc] * \tc ,
\end{equation}
where
$L_\omega : \ell^1_0 \to \ell^1$ is given by
\[
   L_{\omega} := \sigma^+( e^{- i \omega} I + U_{\omega}) + \sigma^-(e^{i \omega} I + U_{\omega}),
\]
with $I$ the identity and  $\sigma^\pm$ the shift operators on $\ell^1$:
\begin{alignat*}{2}
\left[ \sigma^- a \right]_k &:=  a_{k+1}  , \\
\left[ \sigma^+ a \right]_k &:=  a_{k-1}  &\qquad&\text{with the convention } \c_0=0.
\end{alignat*}
The operator $ L_\omega$ is discontinuous in $\omega$ and $ \| L_\omega \| \leq 4$. 
\end{definition} 

%The maps $ \sigma^{+}$ and $ \sigma^-$ are shift up and shift down operators respectively. 
We reformulate Theorem~\ref{thm:FourierEquivalence1}  in terms of the map  $\tilde{F}$. 
We note that it follows from Lemma~\ref{l:analytic} and 
%\marginpar{Reformulate}
%one's choice of  
Equation~\eqref{eq:FourierSequenceEquation}  
%or Equation ~\eqref{eq:fourieroperators},
that the Fourier coefficients of any periodic solution of~\eqref{eq:Wright} lie in $\ell^K$.
These observations are summarized in the following theorem.
\begin{theorem}
\label{thm:FourierEquivalence2}
	Let $ \epsilon \geq 0$,  $\tc \in \ell^K_0$, $\alpha>0$ and $ \omega >0$. 
	Define $y: \R\to \R$ as 
\begin{equation}\label{e:ytc}
	y(t) = 
	\epsilon \left( e^{i \omega t }  + e^{- i \omega t }\right) 
	+  \sum_{k = 2}^\infty   \tc_k e^{i \omega k t }  + \tc_k^* e^{- i \omega k t } .
\end{equation}
%	and suppose that $ y(t) > -1$. 
	Then $y(t)$ solves~\eqref{eq:Wright} if and only if $\tilde{F}_{\epsilon}( \alpha , \omega , \tc) = 0$. 
	Furthermore, up to time translation, any periodic solution of~\eqref{eq:Wright} with period $2\pi/\omega$ is described by a Fourier series of the form~\eqref{e:ytc} with $\epsilon \geq 0$ and $\tc \in \ell^K_0$.
\end{theorem}


%We note that for $\epsilon>0$ such solutions are truly periodic, while for $\epsilon=0$ a zero of $\tilde{F}_\epsilon$ may either correspond to a periodic solution or to the trivial solution $y(t) \equiv 0$. 



% \begin{proof}
%  By Proposition \ref{thm:FourierEquivalence1}, it suffices to show that $\tilde{F}(\alpha,\omega,c) =0$ is equivalent to Equation \ref{eq:FourierSequenceEquation} being satisfied for $k \geq 1$.
%  Since Equation \ref{eq:FourierSequenceEquation} is equivalent to Equation \ref{eq:FHat}, we expand  Equation \ref{eq:FHat} by writing $ \hat{c} = \hat{\epsilon } + c$  where $ \hat{\epsilon} := (\epsilon,0,0,\dots) \in \ell^1$ as below:
%  \begin{equation}
%  0=  \left( i \omega K^{-1} + \alpha U_\omega \right) (\hat{\epsilon}+ c) + \alpha \left[U_\omega \, (\hat{\epsilon}+ c) \right] * (\hat{\epsilon}+ c) \label{eq:Intial}
%  \end{equation}
%  The RHS of Equation \ref{eq:Intial} is $ \tilde{F}(\alpha,\omega,c)$, so the theorem is proved.
% \end{proof}



Since we want to analyze a Hopf bifurcation, we will want to solve $\tilde{F}_\epsilon = 0$ for small values of~$\epsilon$. 
However, at the bifurcation point, $ D \tilde{F}_0(\pp  ,\pp , 0)$ is not invertible.
In order for our asymptotic analysis to be non-degenerate,
we work with a rescaled version of the problem. To this end, for any $\epsilon >0$, we rescale both $\tc$ and $\tilde{F}$ as follows. Let $\tc = \epsilon c$ and 
\begin{equation}\label{e:changeofvariables}
  \tilde{F}_\epsilon (\alpha,\omega,\epsilon c) = \epsilon F_\epsilon (\alpha,\omega,c).
\end{equation}
For $\epsilon>0$ the problem then reduces to finding zeros of 
\begin{equation}
\label{eq:FDefinition}
	F_\epsilon(\alpha,\omega, c) := 
	[i \omega + \alpha e^{-i \omega}] \e_1 + 
	( i \omega K^{-1} + \alpha U_{\omega}) c + 
	\epsilon \alpha e^{-i \omega} \e_2  +
	\alpha \epsilon L_\omega c + 
	\alpha \epsilon [ U_{\omega} c] * c.
\end{equation}
We denote the triple $(\alpha,\omega,c) \in \R^2 \times \ell^1_0$ by $x$.
To pinpoint the components of $x$ we use the projection operators
\[
   \pi_\alpha x = \alpha, \quad \pi_\omega x = \omega, \quad 
  \pi_c x = c \qquad\text{for any } x=(\alpha,\omega,c).
\]

After the change of variables~\eqref{e:changeofvariables} we now have an invertible Jacobian $D F_0(\pp  ,\pp , 0)$ at the bifurcation point.
On the other hand, for $\epsilon=0$ the zero finding problems for $\tilde{F}_\epsilon$ and $F_\epsilon$ are not equivalent. 
However, it follows from the following lemma that any nontrivial periodic solution having $ \epsilon=0$ must have a relatively large size when $ \alpha $ and $ \omega $ are close to the bifurcation point. 

\begin{lemma}\label{lem:Cone}
	Fix $ \epsilon \geq 0$ and $\alpha,\omega >0$. 
	Let
	\[
	b_* :=  \frac{\omega}{\alpha} - \frac{1}{2} - \epsilon  \left(\frac{2}{3}+ \frac{1}{2}\sqrt{2 + 2 |\omega-\pp| } \right).
	\]
Assume that $b_*> \sqrt{2} \epsilon$. 
Define
% \begin{equation*}%\label{e:zstar}
% 	z^{\pm}_* :=b_* \pm \sqrt{(b_*)^2- \epsilon^2 } .
% \end{equation*}
% \note[J]{Proposed change to match Lemma E.4}
\begin{equation}\label{e:zstar}
z^{\pm}_* :=b_* \pm \sqrt{(b_*)^2- 2 \epsilon^2 } .
\end{equation}
If there exists a $\tc \in \ell^1_0$ such that $\tilde{F}_\epsilon(\alpha, \omega,\tc) = 0$, then \\
\mbox{}\quad\textup{(a)} either $ \|\tc\| \leq  z_*^-$ or $ \|\tc\| \geq z_*^+  $.\\
\mbox{}\quad\textup{(b)} 
$ \| K^{-1} \tc \| \leq (2\epsilon^2+ \|\tc\|^2) / b_*$. 
\end{lemma}
\begin{proof}
	The proof follows from Lemmas~\ref{lem:gamma} and~\ref{lem:thecone} in Appendix~\ref{appendix:aprioribounds}, combined with the observation that
$\frac{\omega}{\alpha} - \gamma \geq b_*$,
% \[
%   \frac{\omega}{\alpha} - \gamma \geq b_*
%  \qquad\text{for all }
% | \alpha - \pp| \leq r_\alpha \text{ and } 
%   | \omega - \pp| \leq r_\omega.
% \]
with $\gamma$ as defined in Lemma~\ref{lem:gamma}.
\end{proof}

\begin{remark}\label{r:smalleps}
We note that for $\alpha < 2\omega$
\begin{alignat*}{1}
z^+_* &\geq   \frac{2 \omega - \alpha}{\alpha} 
- \epsilon \left(4/3+\sqrt{2 + 2 |\omega-\pp| } \, \right) + \cO(\epsilon^2)
\\[1mm]
z^-_* & \leq   \cO(\epsilon^2)
\end{alignat*}
for small $\epsilon$. 
Hence Lemma~\ref{lem:Cone} implies that for values of $(\alpha,\omega)$ near $(\pp,\pp)$ any solution has either $\|\tc\|$ of order 1 or $\|\tc\| =  \cO(\epsilon^2)$. 
The asymptotically small term bounding $z_*^-$ is explicitly calculated in Lemma~\ref{lem:ZminusBound}. 
A related consequence is that for $\epsilon=0$ there are no nontrivial solutions 
of $\tilde{F}_0(\alpha,\omega,\tc)=0$ with 
$\| \tc \| < \frac{2 \omega - \alpha}{\alpha} $. 
\end{remark}

\begin{remark}\label{r:rhobound}
In Section~\ref{s:contraction} we will work on subsets of $\ell^K_0$ of the form
\[
  \ell_\rho := \{ c \in \ell^K_0 : \|K^{-1} c\| \leq \rho \} .
\]
Part (b) of Lemma~\ref{lem:Cone} will be used in Section~\ref{s:global} to guarantee that we are not missing any solutions by considering $\ell_\rho$ (for some specific choice of $\rho$) rather than the full space $\ell^K_0$.
In particular, we infer from Remark~\ref{r:smalleps} that  small solutions (meaning roughly that $\|\tc\| \to 0$ as $\epsilon \to 0$)
satisfy $\| K^{-1} \tc \| = \cO(\epsilon^2)$.
\end{remark}

The following theorem guarantees that near the bifurcation point the problem of finding all periodic solutions is equivalent to considering the rescaled problem $F_\epsilon(\alpha,\omega,c)=0$.
\begin{theorem}
\label{thm:FourierEquivalence3}
\textup{(a)} Let $ \epsilon > 0$,  $c \in \ell^K_0$, $\alpha>0$ and $ \omega >0$. 
	Define $y: \R\to \R$ as 
\begin{equation}\label{e:yc}
	y(t) = 
	\epsilon \left( e^{i \omega t }  + e^{- i \omega t }\right) 
	+ \epsilon  \sum_{k = 2}^\infty   c_k e^{i \omega k t }  + c_k^* e^{- i \omega k t } .
\end{equation}
%	and suppose that $ y(t) > -1$. 
	Then $y(t)$ solves~\eqref{eq:Wright} if and only if $F_{\epsilon}( \alpha , \omega , c) = 0$.\\
\textup{(b)}
Let $y(t) \not\equiv 0$ be a periodic solution of~\eqref{eq:Wright} of period $2\pi/\omega$
 with Fourier coefficients $\c$.
Suppose $\alpha < 2\omega$ and $\| \c \| < \frac{2 \omega - \alpha}{\alpha} $.
Then, up to time translation, $y(t)$ is described by a Fourier series of the form~\eqref{e:yc} with $\epsilon > 0$ and $c \in \ell^K_0$.
\end{theorem}

\begin{proof}
Part (a) follows directly from Theorem~\ref{thm:FourierEquivalence2} and the  change of variables~\eqref{e:changeofvariables}.
To prove part (b) we need to exclude the possibility that there is a nontrivial solution with $\epsilon=0$. The asserted bound on the ratio of $\alpha$ and $\omega$ guarantees, by Lemma~\ref{lem:Cone} (see also Remark~\ref{r:smalleps}), that indeed $\epsilon>0$ for any nontrivial solution. 
\end{proof}

We note that in practice (see Section~\ref{s:global}) a bound on $\| \c \|$ is derived from a bound on $y$ or $y'$ using Parseval's identity.

\begin{remark}\label{r:cone}
It follows from Theorem~\ref{thm:FourierEquivalence3} and Remark~\ref{r:smalleps} that for values of $(\alpha,\omega)$ near $(\pp,\pp)$ any reasonably bounded solution satisfies $\| c\| =  O(\epsilon)$ as well as $\|K^{-1} c \| = O(\epsilon)$ asymptotically (as $\epsilon \to 0$).
These bounds will be made explicit (and non-asymptotic) for specific choices of the parameters in Section~\ref{s:global}.
\end{remark}

% We are able to rule out such large amplitude solutions using global estimates such as those in \cite{neumaier2014global}.
% Hence, near the bifurcation point, the problem of describing periodic solutions of~\eqref{eq:Wright} reduces to studying the family of zeros finding problems $F_\epsilon=0$.





%Specifically, if a solution having $ \epsilon = 0$ does in fact correspond to a nontrivial periodic solution and $\alpha  < 2\omega $, then $ \| \tilde{c} \| > 2 \omega \alpha^{-1} -1$. 
%%PERHAPS THIS NEEDS A FORMULATION AS A THEOREM AS WELL?
%%IN OTHER WORDS: ARE WE SURE WE HAVE FOUND ALL ZEROS OF $\tilde{F}_0$, I.E. ALL SOLUTIONS WITH $\epsilon=0$ NEAR THE BIFURCATION POINT? AFTER RESCALING THESE ARE INVISIBLE?
%%THERE SHOULD BE A STATEMENT ABOUT THIS SOMEWHERE! EITHER HERE OR SOME





We finish this section by defining a curve of approximate zeros $\bx_\epsilon$ of $F_\epsilon$ 
(see \cite{chow1977integral,hassard1981theory}). 
%(see \cite{chow1977integral,morris1976perturbative,hassard1981theory}). 


\begin{definition}\label{def:xepsilon}
Let
\begin{alignat*}{1}
	\balpha_\epsilon &:= \pp + \tfrac{\epsilon^2}{5} ( \tfrac{3\pi}{2} -1)  \\
	\bomega_\epsilon &:= \pp -  \tfrac{\epsilon^2}{5} \\
	\bc_\epsilon 	 &:= \left(\tfrac{2 - i}{5}\right) \epsilon \,  \e_2 \,.
\end{alignat*}
We define the approximate solution 
$ \bx_\epsilon := \left( \balpha_\epsilon , \bomega_\epsilon  , \bc_\epsilon \right)$
for all $\epsilon \geq 0$.
\end{definition}

We leave it to the reader to verify that both 
 $F_\epsilon(\pp,\pp,\bc_{\epsilon})=\cO(\epsilon^2)$ and $F_\epsilon(\bx_\epsilon)=\cO(\epsilon^2)$.
%%%	
%%%	
%%%	}{Better like this?}
%%%\annote[J]{ $F_\epsilon(\bx_0)=\cO(\epsilon^2)$ and $F_\epsilon(\bx_\epsilon)=\cO(\epsilon^2)$.}{I think we'd still need the $ \bar{c}_\epsilon$ term in $\bar{x}_0$ to be of order $ \epsilon$.}
%%%\remove[JB]{We show in Proposition A.1
%%%%\ref{prop:ApproximateSolutionWorks} 
%%% that any $ x \in \R^2 \times \ell^1_0$ which is $ \cO(\epsilon^2)$ close to $ \bar{x}_\epsilon $ will yield the estimate $F_\epsilon(x) = \cO(\epsilon^2)$.
%%%Hence choosing $\{ \pp , \pp, \bar{c}_\epsilon\}$ as our approximate solution would also have been a natural choice for performing an $\cO(\epsilon^2)$ analysis and would have simplified several of our calculations.
%%%However,} 
%%%
We choose to use the more accurate approximation 
for the $ \alpha$ and $ \omega $ components to improve our final quantitative results. 














%
% Values for $ (\alpha, \omega,c)$ which approximately solve $\tilde{F}(\alpha,\omega,c) = 0$  are computed in  \cite{chow1977integral,morris1976perturbative,hassard1981theory} and are as follows:
%  \begin{eqnarray}
%  \tilde{\alpha}( \epsilon) &:=& \pi /2 + \tfrac{\epsilon^2}{5} ( \tfrac{3\pi}{2} -1) \nonumber \\
%  \tilde{\omega}( \epsilon) &:=& \pi /2 -  \tfrac{\epsilon^2}{5} \label{eq:ScaleApprox} \\
%  \tc(\epsilon) 	  &:=& \{ \left(\tfrac{2 - i}{5}\right)  \epsilon^2 , 0,0, \dots \} \nonumber
%  \end{eqnarray}
% In Appendix \ref{sec:OperatorNorms} we illustrate an alternative method for deriving this approximation.
%
%
%
%
% We want to solve $ \tilde{F}(\alpha , \omega, \hat{c}) =0$ for small values of $ \epsilon$.
% However $ D \tilde{F}(\alpha , \omega , c)$ is not invertible at $ ( \pp , \pp , 0)$ when $ \epsilon = 0$.
% In order for our asymptotic analysis to be non-degenerate, we need to make the change of variables $ c \mapsto \epsilon c$.
% Under this change of variables, we define the function $ F$ below so that $ \tilde{F}(\alpha , \omega , \epsilon c) =\epsilon  F( \alpha , \omega , c)$.
%
%
%
% \begin{definition}
% Construct an $\epsilon$-parameterized family of densely defined functions  $F : \R^2 \oplus \ell^1 / \C \to \ell^1$ by:
% \begin{equation}
% \label{eq:FDefinition}
% 	F(\alpha,\omega, c) :=
% 	[i \omega + \alpha e^{-i \omega}]_1 +
% 	( i \omega K^{-1} + \alpha U_{\omega}) c +
% 	[\epsilon \alpha e^{-i \omega}]_2  +
% 	\alpha \epsilon L_\omega c +
% 	\alpha \epsilon [ U_{\omega} c] * c.
% \end{equation}
% \end{definition}

%%
%%
%%\begin{corollary}
%%	\label{thm:FourierEquivalence3}
%%	Fix $ \epsilon > 0$, and $ c \in \ell^1 / \C $, and $ \omega >0$. Define $y: \R\to \R$ as 
%%	\[
%%	y(t) = 
%%	\epsilon \left( e^{i \omega t }  + e^{- i \omega t }\right) 
%%	+  \epsilon  \left( \sum_{k = 2}^\infty   c_k e^{i \omega k t }  + \overline{c}_k e^{- i \omega k t } \right) 
%%	\]
%%	and suppose that $ y(t) > -1$. 
%%	Then $y(t)$ solves Wright's equation at parameter $ \alpha > 0 $ if and only if $ F( \alpha , \omega , c) = 0$ at parameter $ \epsilon$. 
%%	
%%	
%%	
%%\end{corollary}
%%
%%
%%\begin{proof}
%%	Since $ \tilde{F}(\alpha,\omega, \epsilon c) = \epsilon F( \alpha , \omega , c)$, the result follows from Theorem \ref{thm:FourierEquivalence2}.
%%\end{proof}

% If we can find $(\alpha , \omega, c)$ for which $ F( \alpha , \omega,c)=0$ at parameter $\epsilon$, then $ \tilde{F}(\alpha ,\omega, c)=0$.
% By Theorem \ref{thm:FourierEquivalence2} this amounts to finding a periodic solution to Wright's equation.
% Lastly, because we have performed the change of variables $ c \mapsto \epsilon c$, we need to  apply this change of variables to our approximate solution as well.
%
% \begin{definition}
% 	Define the approximate solution $ x( \epsilon) = \left\{ \alpha(\epsilon ) , \omega ( \epsilon ) , c(\epsilon) \right\}$ as below,  where $c(\epsilon) = \{ c_2( \epsilon) , 0 ,0 , \dots\} $.
% 	We may also write $ x_\epsilon = x(\epsilon) $.
% 	\begin{eqnarray}
% 	\alpha( \epsilon) &:=& \pi /2 + \tfrac{\epsilon^2}{5} ( \tfrac{3\pi}{2} -1) \nonumber \\
% 	\omega( \epsilon) &:=& \pi /2 -  \tfrac{\epsilon^2}{5} \label{eq:Approx} \\
% 	c_2(\epsilon) 	  &:=& \left(\tfrac{2 - i}{5}\right) \epsilon \nonumber
% 	\end{eqnarray}
%
% \end{definition}


\section{Tiered Vectors}


In this section we will describe how the tiered vector data structure
from~\cite{Goodrich1999} works. 

\begin{figure}
	\includegraphics[width=\textwidth]{graphics/DSExample}
    \caption{An illustration of a tiered vector with $l = w = 3$. The elements are letters, and the tiered vector represents the sequence ABCDEFGHIJKLMNOPQRSTUVX. The elements in the leaves are the elements that are actually stored. The number above each node is its offset. The strings above an internal node $v$ with children $c_1, c_2, c_3$ is respectively $A(c_1) \cdot A(c_2) \cdot A(c_3)$ and $A(v)$, i.e.\ the elements $v$ represents before and after the circular shift. ? specifies an empty element.}
\label{fig:ds}
\end{figure}

%\paragraph{Data Structure} 
\subparagraph*{Data Structure} 
An $l$-tiered vector can be seen as a tree $T$ with root $r$, fixed
height $l - 1$ and out-degree $w$ for any $l \geq 2$.
A node $v \in T$ represents a sequence of elements $A(v)$ thus 
$A(r)$ is the sequence represented by the tiered vector. The capacity $\capacity(v)$ of a node $v$ is $w^{\height(v)+1}$. For a node $v$ with children $c_1, c_2, \ldots, c_w$, $A(v)$ is a circular shift of the
concatenation of the elements represented by its children, 
$A(c_1) \cdot A(c_2) \cdot \ldots \cdot A(c_w)$.
The circular shift is determined by an integer $\offset(v)
\in [\capacity(v)]$ that is explicitly stored for all nodes. Thus the sequence of
elements $A(v)$ of an internal node $v$ can be reconstructed by recursively
reconstructing the sequence for each of its children, concatenating these and
then circular shifting the sequence by $\offset(v)$. See Figure~\ref{fig:ds} for an illustration. A leaf $v$ of $T$
explicitly stores the sequence $A(v)$ in a circular array $\elements(v)$ with
size $w$ whereas internal nodes only store their respective offset.
 Call a node $v$ full if $|A(v)| = \capacity(v)$ and empty if $|A(v)| = 0$. In order to support fast $\aaccess$, for all nodes $v$ the elements of $A(v)$ are located in consecutive children of $v$ that
are all full, except the children containing the first
and last element of $A(v)$ which may be only partly full.

%\paragraph{Access \& Update}
\subparagraph*{Access \& Update}
To access an element $A(r)[i]$ at a given index $i$; one traverses a
path from the root down to a leaf in the tree. In each node the offset of the
node is added to the index to compensate for the cyclic shift, and the traversing is continued in the child corresponding to the newly calculated index. 
Finally when reaching a leaf, the desired element is
returned from the elements array of that leaf. The operation $\aaccess(v, i)$ returns the
element $A(v)[i]$ and is recursively computed as follows:

\begin{description} 
    \item[\quad v is internal:] Compute $i' = (i + \offset(v))
        \mod \capacity(v)$, let $v'$ be the $\lfloor i' / w \rfloor^{th}$ child of $v$ and return the element
    $\aaccess(v', i' \mod \capacity(v'))$. 
    
\item[\quad v is leaf:] Compute $i' = (i + \offset(v)) \mod w$
    and return the element $\elements(v)[i']$.
    \end{description}

The time complexity is $\Theta(l)$ as we visit all nodes on a root-to-leaf path in $T$. To navigate this path we must follow $l - 1$ child pointers, lookup $l$ offsets, and access the element itself. Therefore this requires $l - 1 + l + 1 = 2l$ memory probes.

The update operation is entirely similar to access, except
the element found is not returned but substituted with the new element. The
running time is therefore $\Theta(l)$ as well. For future use, let $\aupdate(v, i, e)$ be the operation that sets $A(v)[i] = e$ and returns the element that was
substituted. 

%\paragraph{Range Access}
\subparagraph*{Range Access}

Accessing a range of elements, can obviously be done by using the
$\aaccess$-operation multiple times, but this results in redundant traversing
of the tree, since consecutive elements of a leaf often
-- but not always due to circular shifts -- corresponds to consecutive elements of $A(r)$.
Let $\aaccess(v, i, m)$ report the
elements $A(v)[i \ldots i + m - 1]$ in order. The operation can recursively
be defined as:

\begin{description} \item[\quad v is internal:] 
    Let $i_l = (i + \offset(v)) \mod \capacity(v)$,
    and let $i_r = (i_l + m) \mod \capacity(v)$. The children of
    $v$ that contains the elements to be reported are in the range $[\lfloor i_l \cdot w / \capacity(v) \rfloor, \lfloor i_r \cdot w / \capacity(v) \rfloor] \mod w$,
    call these $c_l, c_{l+1},
    \ldots, c_r$. In order, call $\aaccess(c_l, i_l, \min(m, \capacity(c_l) -
    i_l))$, $\aaccess(c_i, 0, \capacity(c_i))$ for $c_i = c_{l+1}, \ldots,
    c_{r-1}$, and $\aaccess(c_r, e_{r-1}, 0, i_r \mod \capacity(c_r))$.
	
        \item[\quad v is leaf:] Report the elements $\elements(v)[i, i+m-1] \mod w$. \end{description}

The running time of this strategy is $O(lm)$, but saves a constant factor over the naive solution.


%\paragraph{Insert \& Delete}
\subparagraph*{Insert \& Delete}

Inserting an element in the end (or beginning) of the array can simply be
achieved using the $\aupdate$-operation. Thus the interesting part is fast insertion at an arbitrary position;
this is where we utilize the offsets.

Consider a node $v$, the key challenge is to shift a big chunk of elements $A(v)[i, i+m-1]$ one index right (or left) to $A(v)[i+1, i+m]$ to make room for a new element (without actually moving each element in the range). Look at the range of children $c_l, c_{l+1}, \ldots, c_r$ that covers the range of elements $A(v)[i, i+m-1]$ to be shifted. All elements in $c_{l+1}, \ldots, c_{r-1}$ must be shifted. These children are guaranteed to be full, so make a circular shift by decrementing each of their offsets by one. Afterwards take the element $A(c_{i-1})[0]$ and move it to $A(c_{i})[0]$ using the $\aupdate$ operation for $l < i \leq r$. In $c_l$ and $c_r$ only a subrange of the elements might need shifting, which we do recursively. In the base case of this recursion, namely when
$v$ is a leaf, shift the elements by actually moving the elements one-by-one in $\elements(v)$.

Formally we define the $\ashift(v, e, i, m)$ operation that (logically) shifts
all elements $A(v)[i, i+m-1]$ one place right to $A[i+1, i+m]$, sets $A[i] = e$ and returns the value that was previously on position $A[i+m]$ as:

\begin{description} \item[\quad v is internal:] Let $i_l = (i + \offset(v)) \mod
    \capacity(v)$, and let $i_r = (i_l + m) \mod \capacity(v)$. The children of
    $v$ that must be updated are in the range $[\lfloor i_l \cdot w / \capacity(v) \rfloor, \lfloor i_r \cdot w / \capacity(v) \rfloor] \mod w$ call these $c_l, c_{l+1}, \ldots, c_r$.
Let $e_l = \ashift(c_l, e, i_l, \min(m, \capacity(c_l) - i_l))$. Let $e_i =
\aupdate(c_i, size(c) - 1, e_{i-1})$ and set $\offset(c_i) = (\offset(c_i) - 1)
\mod \capacity(c)$ for $c_i = c_{l+1}, \ldots, c_{r-1}$. Finally call
$\ashift(c_r, e_{r-1}, 0, i_r \mod \capacity(c_r))$.
	
        \item[\quad v is leaf:] Let $e_o = \elements(v)[(i+m) \mod w]$. Move the
            elements $\elements(v)[i, (i+m-1) \mod w]$ to $\elements(v)[i+1,(i+m) \mod w]$, and set $\elements(v)[i] = e$. Return $e_o$.
    \end{description}

An insertion $\ainsert(i, e)$ can then be performed as $\ashift(root, e, i,
size(root) - i - 1)$. The running time of an insertion is $T(l) = 2T(l - 1) + w\cdot l \Rightarrow T(l) = O(2^l w)$.

%TODO: {Add illustration.}


A deletion of an element can basically be done as an inverted insertion, thus
deletion can be implemented using the $\ashift$-operation from before. A
$\adelete(i)$ can be performed as $\ashift(r, \bot, 0, i)$ followed by an
update of the root's offset to $(\offset(r) + 1) \mod \capacity(r)$.

%\paragraph{Space}
\subparagraph*{Space}

There are at most $O(w^{l-1})$ nodes in the tree and each takes up constant
space, thus the total space of the tree is $O(w^{l-1})$.
All leaves are either empty or full except the two leaves storing the first and
last element of the sequence which might contain less than $w$ elements.
Because the arrays of empty leaves are not allocated the space overhead of the arrays is $O(w)$.
Thus beyond the space required to store the $n$ elements themselves, tiered vectors
have a space overhead of $O(w^{l-1})$.

To obtain the desired bounds $w$ is maintained such that $w = \Theta(n^\epsilon)$ where $\epsilon = 1/l$ and $n$ is the number of elements in the tiered vector. This can be achieved by using global rebuilding to gradually increase/decrease the value of $w$ when elements are inserted/deleted without asymptotically changing the running times. We will not provide the details here. We sum up the original tiered vector data structure in the following theorem:

\begin{theorem}[\cite{Goodrich1999}] The original $l$-tiered vector solves the
    dynamic array problem for $l \geq 2$ using $\Theta(n^{1-1/l})$ extra space
    while supporting $\aaccess$ and $\aupdate$ in $\Theta(l)$ time and $2l$
    memory probes. The operations $\ainsert$ and $\adelete$ take $O(2^l n^{1/l})$ time.
    \label{thm:pointer}
\end{theorem}


\section{Improved Tiered Vectors}


In this paper, we consider several new variants of the tiered vector. This section considers the theoretical
properties of these approaches. In particular we are interested in the number of memory accesses that are required for the different memory layouts, since this turns out to have an effect on the experimental running time.
In Section~\ref{sec:experimental} we analyze the actual impact in practice through experiments.

\subsection{Implicit Tiered Vectors}

As the degree of all nodes is always fixed at some constant value $w$ (it may be
changed for all nodes when the tree is rebuilt due to a full root), it is possible to layout the
offsets and elements such that no pointers are necessary to navigate the
tree. Simply number all nodes from left-to-right level-by-level starting in the
root with number 0. Using this numbering scheme, we can store all
offsets of the nodes in a single array and similarly all the elements of the leaves in another array.

To access an element, we only have to lookup the offset for each
node on the root-to-leaf path which requires $l-1$ memory probes plus the final
element lookup, i.e.\ in total $l$ which is half as many as
the original tiered vector.
 The downside with this representation is that it must allocate the
two arrays in their entirety at the point of initialization (or when rebuilding). This results in a $\Theta(n)$ space overhead which is worse than the $\Theta(n^{1-\epsilon})$ space overhead from the original tiered vector.

\begin{theorem} The implicit $l$-tiered vector solves the dynamic array problem for $l \geq 2$
using $O(n)$ extra space while supporting $\aaccess$ and $\aupdate$ in
$O(l)$ time requiring $l$ memory probes. The operations $\ainsert$ and
$\adelete$ take $O(2^l n^{1/l})$ time.
\label{thm:implicit}
\end{theorem}

\subsection{Lazy Tiered Vectors}

We now combine the original and the implicit representation, to get both few memory probes and little space overhead. Instead of having a single array storing all
the elements of the leaves, we store for each leaf a pointer to a location with an array containing the leaf's elements. The array is lazily allocated in memory when elements are actually inserted into it.

The total size of the offset-array and the element pointers in the leaves is $O(n^{1-\epsilon})$. At most two leaves are only partially full, therefore the
total space is now again reduced to $O(n^{1-\epsilon})$. To navigate a root-to-leaf path, we now need to look at $l - 1$ offsets, follow a pointer from a leaf to its array and access the element in the array, giving a total of $l + 1$
memory accesses.

\begin{theorem}
        The lazy $l$-tiered vector solves the dynamic array problem for $l \geq 2$ using
        $\Theta(n^{1-1/l})$ extra space while supporting $\aaccess$ and
        $\aupdate$ in $\Theta(l)$ time requiring $l+1$ memory probes. The
        operations $\ainsert$ and $\adelete$ take $O(2^l n^{1/l})$
        time.
\label{thm:lazy}
\end{theorem}



\section{Implementation}

\section{Implementation: Ring Abstraction}
\label{sec:implement}
\subsection{Distributed \mbox{$G_t$} in QMC Solver}
\label{distributedG4}
Before introducing the communication phase of the ring abstraction layer,
it is important to understand how the authors distributed the large device array $G_t$ across MPI ranks.
%
Original $G_t$ was compared, and $G^d_t$ versions were distributed
(Figure~\ref{fig:compare_original_distributed_g4}). 


In the original $G_t$ implementation, the measurements---which were computed by matrix-matrix multiplication---are distributed statically and independently over the MPI ranks to avoid
inter-node communications. Each MPI rank keeps its partial copy of $G_{t,i}$ to accumulate 
measurements within a rank, where $i$ is the rank index. 
After all the measurements are finished, a reduction step is 
taken to accumulate $G_{t,i}$ across all MPI ranks into a final and complete
$G_t$ in the root MPI rank. The size of the $G_{t,i}$ in each rank is 
the same size as the final and complete $G_t$. 

With the distributed $G^d_t$ implementation, this large device array 
$G_t$ was evenly partitioned across all MPI ranks; each portion of it is local to each MPI rank.
%
Instead of keeping its partial copy of $G_t$, 
each rank now keeps an instance of $G^d_{t,i}$ to accumulate measurements 
of a portion or sub-slice of the final and complete $G_t$, where the notation
$d$ in $G^d_t$  refers to the distributed version, and $i$ means the $i$-th rank.
%
The $G^d_{t,i}$ size in each rank is 
reduced to $1/p$ of the size of the final and complete $G_t$, comparing the same configuration 
in original $G_t$ implementation, where $p$ is the number of MPI ranks used. 
%
For example, in Figure~\ref{fig:distributed_g4}, there are four ranks, and rank $i$
now only keeps $G^d_{t,i}$, which is one-fourth the size of the original $G_t$ array size.
% and contains values indexing from range of $[0, ..., N/4)$ in original $G_t$ array where $N$ is the 
% number of entries in  $G_t$  when viewed as a one-dimensional array.

To compute the final and complete $G^d_{t,i}$ for the distributed $G^d_t$ implementation, 
each rank must see every $G_{\sigma,i}$ from all ranks. 
%
In other words, each rank must broadcast the
locally generated $G_{\sigma,i}$ to the remainder of the other ranks at every measurement step. 
%
To efficiently perform this ``all-to-all'' broadcast, a ring abstraction layer was built (Section. \ref{section:ring_algorithm}), which circulates
all $G_{\sigma,i}$ across all ranks.

% over high-speed GPUs interconnect (GPUDirect RDMA) to facilitate the communication phase.

% \begin{figure}
% \centering
% \subfloat[Original $G_t$ implementation.]
%     {\includegraphics[width=\columnwidth]{original_g4.pdf}}\label{fig:original_g4}

% \subfloat[Distributed $G_t$ implementation.]
%     {\includegraphics[width=0.99\columnwidth]{distributed_g4.pdf} \label{fig:distributed_g4}}

% \caption{Comparison of the original $G_t$ vs. the distributed $G^d_t$ implementation. Each rank contains one GPU resource.}
% \label{fig:compare_original_distributed_g4} 
% \end{figure} 

\begin{figure}
\centering
     \begin{subfigure}[b]{\columnwidth}
         \centering
         \includegraphics[width=\textwidth]{images/original_g4.pdf}
         \caption{Original $G_t$ implementation.}
         \label{fig:original_g4}
     \end{subfigure}
     
    \begin{subfigure}[b]{\columnwidth}
         \centering
         \includegraphics[width=\textwidth]{images/distributed_g4.pdf}
         \caption{Distributed $G_t$ implementation.}
         \label{fig:distributed_g4}
     \end{subfigure}
     
\caption{Comparison of the original $G_t$ vs. the distributed $G^d_t$ implementation. Each rank contains one GPU resource.}
\label{fig:compare_original_distributed_g4}
\end{figure}

\subsection{Pipeline Ring Algorithm}
\label{section:ring_algorithm}
A pipeline ring algorithm was implemented that broadcasts the $G_{\sigma}$ 
array circularly during every measurement. 
%
The algorithm (Algorithm \ref{alg:ring_algorithm_code}) is 
visualized in Figure~\ref{fig:ring_algorithm_figure}.

\begin{algorithm}
\SetAlgoLined
    generateGSigma(gSigmaBuf)\; \label{lst:line:generateG2}
    updateG4(gSigmaBuf)\;       \label{lst:line:updateG4}
    %\texttt{\\}
    {$i\leftarrow 0$}\;         \label{lst:line:initStart}
    {$myRank \leftarrow worldRank$}\;          \label{lst:line:initRankId}
    {$ringSize \leftarrow mpiWorldSize$}\;      \label{lst:line:initRingSize}
    {$leftRank \leftarrow (myRank - 1 + ringSize) \: \% \: ringSize $}\;
    {$rightRank \leftarrow (myRank + 1 + ringSize) \: \% \: ringSize $}\;
    sendBuf.swap(gSigmaBuf)\;           \label{lst:line:initEnd}
    \While{$i< ringSize$}{
        MPI\_Irecv(recvBuf, source=leftRank, tag = recvTag, recvRequest)\; \label{lst:line:Irecv}
        MPI\_Isend(sendBuf, source=rightRank, tag = sendTag, sendRequest)\; \label{lst:line:Isend}
        MPI\_Wait(recvRequest)\;        \label{lst:line:recevBuffWait}
        
        updateG4(recvBuf)\;             \label{lst:line:updateG4_loop}
        
        MPI\_Wait(sendRequest)\;        \label{lst:line:sendBuffWait}
        
        sendBuf.swap(recvBuf)\;         \label{lst:line:swapBuff}
        i++\;
    }
\caption{Pipeline ring algorithm}
\label{alg:ring_algorithm_code}
\end{algorithm}

\begin{figure}
	\centering
	\includegraphics[width=\columnwidth, trim=0 5cm 0 0, clip]{images/ring_algorithm.pdf}
	\caption{Workflow of ring algorithm per iteration. }
	\label{fig:ring_algorithm_figure}
\end{figure}

At the start of every new measurement, a single-particle Green's function $G_{\sigma}$
 (Line~\ref{lst:line:generateG2}) is generated 
and then used to update $G^d_{t,i}$ (Line~\ref{lst:line:updateG4})
via the formula in Eq.~(\ref{eq:G4}).
%
% Different from original method that performs 
% full matrix-matrix multiplication (Equation~(\ref{eq:G4})), the current ring algorithm only performs partial
% matrix-matrix multiplication that contributes to $G^d_{t,i}$, a subslice of the final and complete $G_t$.
%
Between Lines \ref{lst:line:initStart} to \ref{lst:line:initEnd}, the algorithm 
initializes the indices
of left and right neighbors and prepares the sending message buffer from the
previously generated $G_{\sigma}$ buffer. 
%
The processes are organized as a ring so that the first and last rank are considered to be neighbors to each other. 
%
A \textit{swap} operation is used to avoid unnecessary memory copies for \textit{sendBuf} preparation.
%
A walker-accumulator thread allocates an additional \textit{recvBuf} buffer of the same size 
as \textit{gSigmaBuf} to hold incoming 
\textit{gSigmaBuf} buffer from \textit{leftRank}. 

The \textit{while} loop is the core part of the pipeline ring algorithm. 
%
For every iteration, each thread in a rank 
receives a $G_{\sigma}$ buffer from its left neighbor rank and sends a $G_{\sigma}$ buffer to its right neighbor rank. 
A synchronization step (Line~\ref{lst:line:recevBuffWait}) is performed
afterward to ensure that each rank receives a new buffer to update the 
local $G^d_{t,i}$ (Line~\ref{lst:line:updateG4_loop}). 
%
Another synchronization step
follows to ensure that all send requests are finalized 
(Line~\ref{lst:line:sendBuffWait}). Lastly, another \textit{swap} operation is used to exchange
content pointers between \textit{sendBuf} and \textit{recvBuf} to avoid unnecessary memory copy and prepare
for the next iteration of communication.
%
In the multi-threaded version (Section~\ref{subsec:multi-thread}), the thread of index, \textit{i}, only communicates with
	threads of index, \textit{i}, in neighbor ranks, and each thread allocates two buffers: \textit{sendBuff} and \textit{recvBuff}.

The \textit{while} loop will be terminated after $\mbox{\textit{ringSize}} - 1$ steps. By that time, 
each locally generated $G_{\sigma,i}$ will have traveled across all MPI ranks and
updated $G^d_{t,i}$ in all ranks. Eventually, each $G_{\sigma,i}$ reaches
to the left neighbor of its birth rank. For example, $G_{\sigma,0}$ generated from rank $0$ will end 
in last rank in the ring communicator.

Additionally, if the $G_t$ is too large to be stored in one node, 
it is optional to accumulate all $G^d_{t,i}$
at the end of all measurements. 
%
Instead, a parallel write into the file system could be taken.

\subsubsection{Sub-Ring Optimization.}

A sub-ring optimization strategy is further proposed to reduce message communication
times if the large device array $G_t$ can fit in fewer devices. 
%
The sub-ring algorithm is visualized in Figure~\ref{fig:subring_algorithm_figure}.

For the ring algorithm (Section~\ref{section:ring_algorithm}), the size of the ring communicator
(\textit{mpiWorldSize}) is set to the same size of the global \mbox{\texttt{MPI\_COMM\_WORLD}}, and thus the size of $G_t$ is equally 
distributed across all MPI ranks.

However, to complete the update to $G^d_{t,i}$ in one measurement, 
one $G_{\sigma,i}$
must travel \textit{mpiWorldSize} ranks. In total, 
there are \textit{mpiWorldSize} numbers of $G_{\sigma,i}$
being sent and received concurrently in one measurement 
in the global
\mbox{\texttt{MPI\_COMM\_WORLD}} 
communicator. If the size of $G^d_{t,i}$ is relatively small per rank, then this will cause high communication overhead.

If $G_t$ can be distributed and fitted in fewer devices, then a shorter travel distance is required 
for $G_{\sigma,i}$, thus reducing the communication overhead. One reduction
step was performed at the end of all measurements to accumulate $G^d_{t,s_i}$, 
where $s_i$ means $i$-th rank on the $s$-th sub-ring.

At the beginning of MPI initialization, the global \mbox{\texttt{MPI\_COMM\_WORLD}} was partitioned  into several new sub-ring communicators by using \mbox{\texttt{MPI\_Comm\_split}}. 
% where each new communicator represents conceptually a subring. 
The new
communicator information was passed to the DCA++ concurrency class by substituting the original global 
\mbox{\texttt{MPI\_COMM\_WORLD}} with this new communicator. Now, only a few minor modifications
are needed to transform the ring algorithm (Algorithm~\ref{alg:ring_algorithm_code})
to sub-ring Algorithm~\ref{alg:sub_ring_algorithm}. In Line~\ref{lst:line:initRankId}, \textit{myRank} is 
initialized to \textit{subRingRank} instead of \textit{worldRank}, where 
\textit{subRingRank} is the rank index in the local sub-ring communicator. 
%
In Line~\ref{lst:line:initRingSize}, \textit{ringSize} is initialized to \textit{subRingSize}
instead of \textit{mpiWorldSize}, where \textit{subRingSize} is the
size of the new communicator.
%
The general ring algorithm is a special case for the sub-ring algorithm because the
\textit{subRingSize} of the general ring algorithm is equal to \textit{mpiWorldSize}, and
there is only one sub-ring group throughout all MPI ranks.


\LinesNumberedHidden
\begin{algorithm}
    {$\mbox{\textit{myRank}} \leftarrow \mbox{\textit{subRingRank}}$}\;         
    {$\mbox{\textit{ringSize}} \leftarrow \mbox{\textit{subRingSize}}$}\;      
\caption{Modified ring algorithm to support sub-ring communication}
\label{alg:sub_ring_algorithm}
\end{algorithm}


\begin{figure}
	\centering
	\includegraphics[width=\columnwidth, trim=0 5cm 0 0, clip]{images/subring_alg.pdf}
	\caption{Workflow of sub-ring algorithm per iteration. Every consecutive $S$ rank forms a sub-ring communicator, 
	and no communication occurs between sub-ring communicators until all measurements are finished. Here, $S$ is the number of ranks in a sub-ring.}
	\label{fig:subring_algorithm_figure}
\end{figure}

\subsubsection{Multi-Threaded Ring Communication.}
\label{subsec:multi-thread}
To take advantage of the multi-threaded QMC model already in DCA++, 
multi-threaded ring communication support was further implemented in the ring algorithm.
%
Figure~\ref{fig:dca_overview} shows that in the original DCA++ method,
each walker-accumulator
thread in a rank is independent of each other, and all the threads in a 
rank synchronize only after all rank-local measurements are finished.
%
Moreover, during every measurement, each walker-accumulator thread
generates its own thread-private $G_{\sigma, i}$ to update $G_t$. 
%

The multi-threaded ring algorithm now allows concurrent message exchange so that threads of same rank-local thread index exchange their thread-private $G_{\sigma, i}$. 
%
Conceptually, there are $k$ parallel and independent rings, where $k$ 
is number of threads per rank, because threads of the same local thread ID
form a closed ring. 
%
For example, a thread of index $0$ in rank $0$ will send its $G_\sigma$ to 
the thread of index $0$ in rank $1$ and receive another $G_\sigma$ from thread index of $0$ 
from last rank in the ring algorithm.
%

The only changes in the ring algorithm are offsetting the tag values 
(\texttt{recvTag} and \texttt{sendTag}) by the thread index value. For example,
Lines~\ref{lst:line:Irecv} and ~\ref{lst:line:Isend} from 
Algorithm~\ref{alg:ring_algorithm_code} are modified to Algorithm~\ref{alg:multi_threaded_ring}.

\LinesNumberedHidden
\begin{algorithm}
        MPI\_Irecv(recvBuf, source=leftRank, tag = recvTag + threadId, recvRequest)\; 
        MPI\_Isend(sendBuf, source=rightRank, tag = sendTag + threadId, sendRequest)\;
\caption{Modified ring algorithm to support multi-threaded ring}
\label{alg:multi_threaded_ring}
\end{algorithm}

To efficiently send and receive $G_\sigma$, each thread
will allocate one additional \textit{recvBuff} to hold incoming 
\textit{gSigmaBuf} buffer from \textit{leftRank} and perform send/receive efficiently.
%
In the original DCA++ method, there are $k$ numbers of buffers of $G_\sigma$ 
size per rank, and in the multi-threaded ring method, there are $2k$
numbers of buffers of $G_\sigma$ size per rank, where $k$ is number of 
threads per rank.


\label{sec:experimental}

\section{Experiments}

\label{sec:experiments}
In this section we compare the tiered vector to some widely used C++ standard library containers. 
We also compare different variants of the tiered vector. 
We consider how the different representations of the data
structure listed in Section~\ref{sec:implementation}, 
and also how the height of tree and the capacity of the leaves affects the running time.
The following describes the test setup:

\subparagraph{Environment}

All experiments have been performed on a Intel Core i7-4770 CPU @ 3.40GHz with
32 GB RAM. The code has been compiled with GNU GCC version 5.4.0 with flags
``-O3''. The reported times are an average over 10 test runs.
 
 \subparagraph{Procedure}
%have been added to the data structure in
 
In all tests $10^8$ 32-bit integers 
are inserted in the data structure as a preliminary step
to simulate that it has already been
used\footnote{In order to minimize the overall running time of the experiments,
the elements were not added randomly, but we show this does not give our data
structure any benefits}.
For all the access and successor operations $10^9$ elements have been accessed
and the time reported is the average time per element.
For range access, 10.000 consecutive elements are accessed.
For insertion/deletion $10^6$ elements
have been (semi-)randomly\footnote{In order to not impact timing, a simple
access pattern has been used instead of a normal pseudo-random generator.}
added/deleted, though in the case of ``vector'' only 10.000 elements were
inserted/deleted to make the experiments terminate in reasonable time. 

\subsection{Comparison to C++ STL Data Structures}

In the following we have compared our best performing tiered vector (see the next sections) to the vector and
the multiset class from the C++ standard library.
The vector data structure directly supports the
operations of a dynamic array. The multiset class is implemented as a red-black
tree and is therefore interesting to compare with our data structure.
Unfortunately, multiset does not directly support the operations of a dynamic
array (in particular it has no notion of positions of elements). To simulate an
access operation we instead find the successor of an element in the multiset.
This requires a root-to-leaf traversal of the red-black tree, just as an access
operation in a dynamic array implemented as a red-black tree would. Insertion
is simulated as an insertion into the multiset, which again requires the same
computations as a dynamic array implemented as a red-black tree would.

Besides the random access, range access and insertion,
we have also tested the operations \textit{data dependent access},
insertion in the end, deletion, and \textit{successor} queries. In the
\textit{data dependent access} tests, the next index to lookup depends on the values of the prior
lookups. This ensures that the CPU cannot successfully pipeline
consecutive lookups, but must perform them in sequence. We test insertion in the end, since
this is a very common use case. Deletion is performed by deleting elements at
random positions. The $successor$ queries returns the successor of an element
and is not actually part of the
dynamic array problem, but is included since it is a commonly used operation on
a multiset in C++. It is simply implemented as a binary search over the elements in
both the vector and tiered vector tests where the elements are now inserted in sorted order. 

The results are summarized in Table~\ref{tab:test_comp} which shows that the vector performs slightly better than the tiered vector on all access and successor tests. As expected from the $\Theta(n)$ running time, it performs extremely poor on random insertion and deletion. For insertion in the end of the sequence, vector is also slightly faster than the tiered vector. The interesting part is that even though the tiered vector requires several extra memory lookups and computations, we have managed to get the running time down to less than the double of the vector for access, even less for data dependent access and only a few percent slowdown for range access. As discussed earlier,
this is most likely because the entire tree structure (without the elements)
fits within the CPU cache, and because the computations required has been minimized.

Comparing our tiered vector to multiset, we would expect access operations to be
faster since they run in $O(1)$ time compared to $O(\log n)$. On the other
hand, we would expect insertion/deletion to be significantly slower since it
runs in $O(n^{1/l})$ time compared to $O(\log n)$ (where $l = 4$ in these tests). We
see our expectations hold for the access operations where the tiered vector is faster by more than an order of magnitude.
In random insertions however,  the tiered vector is only $8\%$ slower -- even when operating on 100.000.000 elements. Both the tiered
vector and set requires $O(\log n)$ time for the successor operation. In our
experiments the tiered vector is 3 times faster for the successor operation.

Finally, we see that the memory usage of vector and tiered vector is almost identical.
This is expected since in both cases the space usage is dominated by the space taken by the actual elements.
The multiset uses more than 10 times as much space, so this is also a considerable drawback of the red-black tree behind this structure. 

To sum up, the tiered vectors performs better than multiset on all tests
but insertion, where it performs only slightly worse.

%\caption{Figures (a) through (e) show the performance of \textit{Tiered Arrays} (\protect\purple) compared
%to the \textit{set} (\protect\green) and \textit{vector} (\protect\blue) data structures from the C++ standard library.} \label{fig:animals}
\begin{table}
	\centering
	\begin{tabular}{|l|r|r|r|r|r|}
		\hline
		& \multicolumn{1}{l|}{\textit{tiered vector}} & \multicolumn{1}{l|}{\textit{set}} & \multicolumn{1}{l|}{\textit{set / tiered}} & \multicolumn{1}{l|}{\textit{vector}} & \multicolumn{1}{l|}{\textit{vector / tiered}} \\ \hline
		access     & $34.07$ ns                                  & $1432.05$ ns                      & 42.03                                      & $21.63$ ns                           & 0.63                                          \\ \hline
		dd-access    & $99.09$ ns                                  & $1436.67$ ns                      & 14.50                                      & $79.37$ ns                           & 0.80                                          \\ \hline
		range access   & $0.24$ ns                                   & $13.02$ ns                        & 53.53                                      & $0.23$ ns                            & 0.93                                          \\ \hline
		insert   & $1.79$ $\mu$s                               & $1.65$ $\mu$s                     & 0.92                                       & $21675.49$ $\mu$s                     & 12082.33                                      \\ \hline
		insertion in end     & $7.28$ ns                               & $242.90$ ns                     & 33.38                                       & $2.93$ ns                     & 0.40                                      \\ \hline
		successor & $0.55$ $\mu$s                               & $1.53$ $\mu$s                     & 2.75                                       & $0.36$ $\mu$s                        & 0.65                                          \\ \hline
		delete     & $1.92$ $\mu$s                               & $1.78$ $\mu$s                     & 0.93                                       & $21295.25$ $\mu$s                     & 11070.04                                      \\ \hline
		memory     & $408$ MB                               & $4802$ MB                     & 11.77                                       & $405$ MB                    & 0.99                                      \\ \hline
	\end{tabular}
	\caption{The table summarizes the performance of the implicit tiered vector
		compared to the performance of multiset and vector from the C++ standard library.\
		dd-access refers to data dependent access.}
\label{tab:test_comp}
\end{table}


\definecolor{cpurple}{RGB}{131,24,197}
\definecolor{cgreen}{RGB}{70,156,118}
\definecolor{cblue}{RGB}{11,178,228}
\definecolor{cdblue}{RGB}{11,112,173}
\definecolor{corange}{RGB}{219,162,55}
\definecolor{cyellow}{RGB}{238,228,98}
\definecolor{cred}{RGB}{110,55,38}
\newcommand{\purple}{\raisebox{2pt}{\tikz{\draw[cpurple,solid,line width=1.9pt](0,0) -- (3mm,0);}}}
\newcommand{\green}{\raisebox{2pt}{\tikz{\draw[cgreen,solid,line width=1.9pt](0,0) -- (3mm,0);}}}
\newcommand{\blue}{\raisebox{2pt}{\tikz{\draw[cblue,solid,line width=1.9pt](0,0) -- (3mm,0);}}}
\newcommand{\dblue}{\raisebox{2pt}{\tikz{\draw[cdblue,solid,line width=1.9pt](0,0) -- (3mm,0);}}}
\newcommand{\orange}{\raisebox{2pt}{\tikz{\draw[corange,solid,line width=1.9pt](0,0) -- (3mm,0);}}}
\newcommand{\yellow}{\raisebox{2pt}{\tikz{\draw[cyellow,solid,line width=1.9pt](0,0) -- (3mm,0);}}}
\newcommand{\red}{\raisebox{2pt}{\tikz{\draw[cred,solid,line width=1.9pt](0,0) -- (3mm,0);}}}


\begin{figure}[ht]
	\centering
	\begin{subfigure}[b]{0.3\textwidth}
		\includegraphics[width=\textwidth]{layout_test_get}
		\caption{\textit{access}}
	\end{subfigure}
	\begin{subfigure}[b]{0.3\textwidth}
		\includegraphics[width=\textwidth]{layout_test_random}
		\caption{\textit{insert}}
	\end{subfigure}
        \caption{Figures (a) and (b) show the performance of the
            \textit{original} (\protect\purple), \textit{optimized original}
            (\protect\green), \textit{lazy} (\protect\blue) \textit{packed
            lazy} (\protect\orange),
            \textit{implicit} (\protect\yellow)
            and \textit{packed implicit} (\protect\dblue) layouts.}
\label{fig:test_representation}
\end{figure}
\subsection{Tiered Vector Variants}

In this test we compare the performance
of the implementations listed in Section~\ref{sec:implementation} to that 
or the original data structure as described in~\ref{thm:pointer}.

%\paragraph{Optimized Original}
\subparagraph*{Optimized Original}
By co-locating the child offset and child pointer, the two memory lookups are at
adjacent memory locations. Due to the cache lines in modern processors,
the second memory lookup will then often be answered directly by the fast
L1-cache.
As can be seen on Figure~\ref{fig:test_representation}, this small change in the memory layout results in a significant improvement in performance for both access and insertion. In the latter case, the running time is more than halved.

%\paragraph{Lazy and Packed Lazy}
\subparagraph*{Lazy and Packed Lazy}

Figure~\ref{fig:test_representation} shows
how the fewer memory probes required by the
\textit{lazy} implementation in comparison to the \text{original}
and \text{optimized original} results in better performance.
Packing the offset and pointer in the leaves results in even better performance
for both access and insertion even though it requires a few extra instructions
to do the actual packing and unpacking.

%\paragraph{Implicit}
\subparagraph*{Implicit}
From Figure~\ref{fig:test_representation}, we see the implicit
data structure is the fastest.
This is as expected because it requires fewer
memory accesses than the other structures except
for the packed lazy which instead has a slight
computational overhead due to the packing and unpacking.

As shown in Theorem~\ref{thm:implicit} the implicit data structure has a
bigger memory overhead than the lazy data structure.
Therefore the packed lazy representation might be beneficial in some
settings.

%\paragraph{Packed Implicit}
\subparagraph*{Packed Implicit}

Packing the offsets array could lead to 
better cache performance due to the smaller memory footprint and therefore
yield better overall performance.
As can be seen on Figure~\ref{fig:test_representation},
the smaller memory footprint
did not improve the performance in practice.
The simple reason for this,
is that the strategy we used for packing the offsets required
extra computation. This clearly dominated the possible gain from the
hypothesized better cache performance. We tried a few strategies to minimize
the extra computations needed at the expense of slightly worse memory usage,
but none of these led to better results than when not packing the offsets at
all.

\subsection{Width Experiments}

\begin{figure}
	\centering
	\begin{subfigure}[b]{0.3\textwidth}
		\includegraphics[width=\textwidth]{width_test_get}
		\caption{\textit{access}}
	\end{subfigure}
	\begin{subfigure}[b]{0.3\textwidth}
		\includegraphics[width=\textwidth]{width_test_sum}
		\caption{\textit{range access}}
	\end{subfigure}
	\begin{subfigure}[b]{0.3\textwidth}
		\includegraphics[width=\textwidth]{width_test_random}
		\caption{\textit{insert}}
	\end{subfigure}
	\caption{Figures (a), (b) and (c) show the performance of the \textit{implicit} (\protect\purple) and
		the \textit{optimized original} tiered vector (\protect\green) for different tree widths.}
\label{fig:test_width}
\end{figure}

This experiment was performed to determine the best capacity ratio between the leaf nodes and the internal nodes.
The six different width configurations we have tested are: 32-32-32-4096, 32-32-64-2048, 32-64-64-1024, 64-64-64-512, 64-64-128-256, and 64-128-128-128.
All configurations have a constant height 4 and a capacity of approximately 130 mio.

We expect the performance of access operations to remain unchanged, since the
amount of work required only depends on the height of the tree,
and not the widths. We expect range access to perform better when the leaf size
is increased, since more elements will be located in consecutive memory
locations. For $insertion$ there is not a clearly expected behavior as the time
used to physically move elements in a leaf will increase with leaf size, but
then less operations on the internal nodes of the tree has to be performed.

On Figure~\ref{fig:test_width} we see access times are actually decreasing
slightly when leaves get bigger. This was not expected, but is most likely
due to small changes in the memory layout that results in slightly better cache
performance. The same is the case for range access, but this was expected. For
insertion, we see there is a tipping point. For our particular instance, the
best performance is achieved when the leaves have size 512.

%Based on this, we have performed the remaining tests with the 64-64-64-512 configuration (unless otherwise specified).

\subsection{Height Experiments}

\begin{figure}
	\centering
	\begin{subfigure}[b]{0.3\textwidth}
		\includegraphics[width=\textwidth]{height_get}
		\caption{\textit{access(i)}}
	\end{subfigure}
	\begin{subfigure}[b]{0.3\textwidth}
		\includegraphics[width=\textwidth]{height_sum}
		\caption{\textit{access(i, m)}}
	\end{subfigure}
	\begin{subfigure}[b]{0.3\textwidth}
		\includegraphics[width=\textwidth]{height_random}
		\caption{\textit{insert}}
	\end{subfigure}
	\caption{Figures (a),(b) and (c) show the performance of the \textit{implicit} (\protect\purple) and
		the \textit{optimized original} tiered vector (\protect\green) for different tree heights.}
\label{fig:test_height}
\end{figure}

In these tests we have studied how different heights affect the performance of
access and insertion operations. We have tested the configurations 8196-16384,
512-512-512, 64-64-64-512, 16-16-32-32-512, 8-8-16-16-16-512. All resulting in
the same capacity, but with heights in the range 2-6.

We expect the access operations to perform better for lower trees, since
the number of operations that must be performed is linear in the height. On the
other hand we expect insertion to perform significantly better with higher
trees, since its running time is $O(n^{1/l})$ where $l$ is the height plus one. 

On Figure~\ref{fig:test_height} we see the results follow our expectations. However, the access operations only perform slightly worse on higher trees.
This is most likely because all internal nodes fit within the L3-cache. Therefore the running time is dominated by the lookup of the element itself.
(It is highly unlikely that the element requested by an access 
to a random position would be among the small fraction of elements that
fit in the L3-cache).

Regarding insertion, we see significant improvements up until a height of 4. After that, increasing the height does not change the running time noticeably. This is most likely due to the hidden constant in $O(n^{1/l})$ increasing rapidly with the height.



\subsection{Configuration Experiments}

\begin{figure}
    \centering
    \begin{subfigure}[b]{0.3\textwidth}
        \includegraphics[width=\textwidth]{small_get}
        \caption{\textit{access}}
    \end{subfigure}
    \begin{subfigure}[b]{0.3\textwidth}
        \includegraphics[width=\textwidth]{small_sum}
        \caption{\textit{range access}}
    \end{subfigure}
    \begin{subfigure}[b]{0.3\textwidth}
        \includegraphics[width=\textwidth]{small_random}
        \caption{\textit{insert(i,x)}}
    \end{subfigure}
    \caption{Figures (a) and (b) show the performance of the
    \textit{base} (\protect\purple),
    \textit{rotated} (\protect\green), 
    \textit{non-aligned sizes} (\protect\blue),
    \textit{non-templated} (\protect\orange)
    layouts.}
\label{fig:test_minor}
\end{figure}

In these experiments, we test a few hypotheses about how different changes
impact the running time. The results are shown on
Figure~\ref{fig:test_minor}, the leftmost result (base) is
the implicit 64-64-64-512 configuration of the tiered vector 
to which we compare our hypotheses.
%our final and best

\textit{Rotated}: 
As already mentioned, the insertions performed as a
preliminary step to the tests are not done at random positions.
This means that all offsets are zero when our real operations
start. The purpose of this test is the ensure that
there are no significant performance gains in starting
from such a configuration which could otherwise
lead to misleading results.
To this end, we have randomized all
offsets (in a way such that the data structure is still valid, but the
order of elements change) after doing the preliminary insertions
but before timing the operations. As can be seen on
Figure~\ref{fig:test_minor}, the difference between this and the normal
procedure is insignificant, thus we find our approach gives a fair picture.


\textit{Non-Aligned Sizes}: In all our previous tests, we have ensured all
nodes had an out-degree that was a power of 2. This was chosen in order to let the
compiler simplify some calculations, i.e.\ replacing multiplication/division
instructions by shift/and instructions. As Figure~\ref{fig:test_minor} shows,
using sizes that are not powers of 2 results in significantly worse performance.
Besides showing that powers of 2 should always be used, this also indicates that not only
the number of memory accesses during an operation is critical for our
performance, but also the amount of computation we make.

\textit{Non-Templated}
The non-templated results 
in Figure~\ref{fig:test_representation} the
show that the change to templated recursion
has had a major impact on the running time. It should be noted that some
improvements have not been implemented in the non-templated version,
but it gives a good indication that this has been quite useful.


\section{Conclusion}

% \vspace{-0.5em}
\section{Conclusion}
% \vspace{-0.5em}
Recent advances in multimodal single-cell technology have enabled the simultaneous profiling of the transcriptome alongside other cellular modalities, leading to an increase in the availability of multimodal single-cell data. In this paper, we present \method{}, a multimodal transformer model for single-cell surface protein abundance from gene expression measurements. We combined the data with prior biological interaction knowledge from the STRING database into a richly connected heterogeneous graph and leveraged the transformer architectures to learn an accurate mapping between gene expression and surface protein abundance. Remarkably, \method{} achieves superior and more stable performance than other baselines on both 2021 and 2022 NeurIPS single-cell datasets.

\noindent\textbf{Future Work.}
% Our work is an extension of the model we implemented in the NeurIPS 2022 competition. 
Our framework of multimodal transformers with the cross-modality heterogeneous graph goes far beyond the specific downstream task of modality prediction, and there are lots of potentials to be further explored. Our graph contains three types of nodes. While the cell embeddings are used for predictions, the remaining protein embeddings and gene embeddings may be further interpreted for other tasks. The similarities between proteins may show data-specific protein-protein relationships, while the attention matrix of the gene transformer may help to identify marker genes of each cell type. Additionally, we may achieve gene interaction prediction using the attention mechanism.
% under adequate regulations. 
% We expect \method{} to be capable of much more than just modality prediction. Note that currently, we fuse information from different transformers with message-passing GNNs. 
To extend more on transformers, a potential next step is implementing cross-attention cross-modalities. Ideally, all three types of nodes, namely genes, proteins, and cells, would be jointly modeled using a large transformer that includes specific regulations for each modality. 

% insight of protein and gene embedding (diff task)

% all in one transformer

% \noindent\textbf{Limitations and future work}
% Despite the noticeable performance improvement by utilizing transformers with the cross-modality heterogeneous graph, there are still bottlenecks in the current settings. To begin with, we noticed that the performance variations of all methods are consistently higher in the ``CITE'' dataset compared to the ``GEX2ADT'' dataset. We hypothesized that the increased variability in ``CITE'' was due to both less number of training samples (43k vs. 66k cells) and a significantly more number of testing samples used (28k vs. 1k cells). One straightforward solution to alleviate the high variation issue is to include more training samples, which is not always possible given the training data availability. Nevertheless, publicly available single-cell datasets have been accumulated over the past decades and are still being collected on an ever-increasing scale. Taking advantage of these large-scale atlases is the key to a more stable and well-performing model, as some of the intra-cell variations could be common across different datasets. For example, reference-based methods are commonly used to identify the cell identity of a single cell, or cell-type compositions of a mixture of cells. (other examples for pretrained, e.g., scbert)


%\noindent\textbf{Future work.}
% Our work is an extension of the model we implemented in the NeurIPS 2022 competition. Now our framework of multimodal transformers with the cross-modality heterogeneous graph goes far beyond the specific downstream task of modality prediction, and there are lots of potentials to be further explored. Our graph contains three types of nodes. while the cell embeddings are used for predictions, the remaining protein embeddings and gene embeddings may be further interpreted for other tasks. The similarities between proteins may show data-specific protein-protein relationships, while the attention matrix of the gene transformer may help to identify marker genes of each cell type. Additionally, we may achieve gene interaction prediction using the attention mechanism under adequate regulations. We expect \method{} to be capable of much more than just modality prediction. Note that currently, we fuse information from different transformers with message-passing GNNs. To extend more on transformers, a potential next step is implementing cross-attention cross-modalities. Ideally, all three types of nodes, namely genes, proteins, and cells, would be jointly modeled using a large transformer that includes specific regulations for each modality. The self-attention within each modality would reconstruct the prior interaction network, while the cross-attention between modalities would be supervised by the data observations. Then, The attention matrix will provide insights into all the internal interactions and cross-relationships. With the linearized transformer, this idea would be both practical and versatile.

% \begin{acks}
% This research is supported by the National Science Foundation (NSF) and Johnson \& Johnson.
% \end{acks}

\bibliography{references}

\end{document}
