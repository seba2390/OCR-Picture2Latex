%%% NO LIPICS

\documentclass{article}
%\usetikzlibrary{matrix}
%\usetikzlibrary{positioning}
\usepackage{amssymb}
\usepackage{amsmath}

\usepackage[utf8]{inputenc}
\usepackage[a4paper, total={426pt, 674pt}]{geometry}
\usepackage{hyperref}

\usepackage{graphicx}
\usepackage{subcaption}
\newtheorem{theorem}{Theorem}
\newtheorem{lemma}{Lemma}
\newtheorem{corollary}{Corollary}
\newtheorem{definition}{Definition}
\newtheorem{prop}{Proposition}
\newenvironment{proof}{\noindent\emph{Proof. }}

\usepackage{enumitem}
\bibliographystyle{plain}

\date{}
\author{Philip Bille\thanks{Supported by the Danish Research Council (DFF -- 4005-00267, DFF -- 1323-00178)}\\ \texttt{phbi@dtu.dk}
    \and Anders Roy Christiansen\thanks{Supported by the Danish Research Council (DFF -- 4005-00267)}\\ \texttt{miet@dtu.dk}
    \and Mikko Berggren Ettienne$^\dag$\\ \texttt{miet@dtu.dk}
\and Inge Li G{\o}rtz$^\dag$\\ \texttt{inge@dtu.dk}}

% LIPICS ONLY

%\documentclass[a4paper,UKenglish]{lipics-v2016}

%\bibliographystyle{plainurl}% the recommended bibstyle

%\author[1]{Philip Bille}
%%\thanks{Supported by the Danish Research Council (DFF -- 4005-00267, DFF -- 1323-00178)}
%\author[1]{Anders Roy Christiansen}
%%\thanks{Supported by the Danish Research Council (DFF -- 4005-00267)}
%\author[1]{Mikko Berggren Ettienne}
%%\thanks{Supported by the Danish Research Council (DFF -- 4005-00267)}
%\author[1]{Inge Li G{\o}rtz}
%\affil[1]{The Technical University of Denmark\\
%\texttt{\{phbi,aroy,miet,inge\}@dtu.dk}}
%\authorrunning{P. Bille, A.\,R. Christiansen, M.\,B. Ettienne and I.\,L. G{\o}rtz}

%\Copyright{Philip Bille, Anders Roy Christiansen, Mikko Berggren Ettienne and Inge Li G{\o}rtz}


%\subjclass{F.2.2 Nonnumerical Algorithms and Problems,  E.1 Data Structures}
%\keywords{Dynamic Arrays, Tiered Vectors}

%Editor-only macros:: begin (do not touch as author)%%%%%%%%%%%%%%%%%%%%%%%%%%%%%%%%%%
%\EventEditors{Kirk Pruhs and Christian Sohler}
%\EventNoEds{2}
%\EventLongTitle{25th Annual European Symposium on Algorithms (ESA 2017)}
%\EventShortTitle{ESA 2017}
%\EventAcronym{ESA}
%\EventYear{2017}
%\EventDate{September 4--6, 2017}
%\EventLocation{Vienna, Austria}
%\EventLogo{}
%\SeriesVolume{87}
%\ArticleNo{23}

%END LIPICS ONLY

\graphicspath{{./graphics/}}
\usepackage{tikz}

\newcommand{\ceil}[1]{\left\lceil{#1}\right\rceil}
\newcommand{\floor}[1]{\left\lfloor{#1}\right\rfloor}
\renewcommand{\angle}[1]{\langle{#1}\rangle}

\newcommand{\suc}{\ensuremath\mathrm{succ}}
\newcommand{\pos}{\ensuremath\mathrm{pos}}
\newcommand{\level}{\ensuremath\mathrm{level}}

\newcommand{\aaccess}{\ensuremath\mathsf{access}}
\newcommand{\ainsert}{\ensuremath\mathsf{insert}}
\newcommand{\adelete}{\ensuremath\mathsf{delete}}
\newcommand{\aupdate}{\ensuremath\mathsf{update}}
\newcommand{\ashift}{\ensuremath\mathsf{shift}}

\newcommand{\offset}{\ensuremath\mathrm{off}}
\newcommand{\capacity}{\ensuremath\mathrm{cap}}
\newcommand{\childcapacity}{\ensuremath\mathrm{ccap}}
\newcommand{\height}{\ensuremath\mathrm{height}}
\newcommand{\elements}{\ensuremath\mathrm{elems}}
\newcommand{\children}{\ensuremath\mathrm{children}}


\newcommand{\PP}{\ensuremath\mathrm{PP}}
\renewcommand{\SS}{\ensuremath\mathrm{SS}}
\newcommand{\Next}{\ensuremath\mathrm{Next}}

\newcommand{\str}{\ensuremath{S} }
\newcommand{\tree}{\ensuremath{T}}
\newcommand{\slp}{\ensuremath{\mathcal{S}} }
\newcommand{\kq}{\ensuremath{k_{\str, q}} }
\newcommand{\qg}{\ensuremath{G_q(\str)} }
\newcommand{\fca}{\ensuremath{\textsc{firstcolor}}}
\newcommand{\lca}{\ensuremath{\textsc{lastcolor}}}
\newcommand{\nca}{\ensuremath{\textsf{nca}}}
\newcommand{\labsuc}{\ensuremath{\textsc{ls}}}
\newcommand{\la}{\ensuremath{\textsc{la}}}
\newcommand{\leftc}{\ensuremath{\text{\textit{left}}}}
\newcommand{\rightc}{\ensuremath{\text{\textit{right}}}}
\newcommand{\band}{\ensuremath{\wedge}}
\newcommand{\bor}{\ensuremath{\vee}}
\newcommand{\bxor}{\ensuremath{\oplus}}
\newcommand{\hf}{\ensuremath{\mathcal{H}}}
\newcommand{\pat}{\ensuremath{P}}
\newcommand{\fingerprintq}{\ensuremath{\textsc{Fingerprint}}}
\newcommand{\lceq}{\ensuremath{\textsc{LCE}}}
\newcommand{\bm}{\ensuremath{\mathcal{B}}}
\newcommand{\chunk}{\ensuremath{X}}


\title{Fast Dynamic Arrays}



\begin{document}

\maketitle

\abstract{We present a highly optimized implementation
    of tiered vectors, a data structure
    for maintaining a sequence of $n$ elements
    supporting access in time $O(1)$
    and insertion and deletion in time $O(n^\epsilon)$ for $\epsilon > 0$
    while using $o(n)$ extra space.
    We consider several different implementation optimizations 
    in C++ and compare their performance to that of \text{vector} and
    \text{multiset}
    from the standard library on sequences with up to
    $10^8$ elements.
    Our fastest implementation uses
    much less space than multiset 
    while providing speedups of $40\times$ for
    access operations compared to multiset
    and speedups of $10.000\times$ compared
    to vector for insertion and deletion operations
    while being competitive
    with both data structures for all other
    operations.
}


\section{Introduction}

\section{Introduction}  \label{sec:introduction}

\newcommand\inexpIntro[3]{#1?(#2,#3).}
\newcommand\rinexpIntro[3]{*#1?(#2,#3).}
\newcommand\outexpIntro[3]{#1!(#2,#3).}
\newcommand\outatomIntro[3]{#1!(#2,#3)}

We propose a fully automated method for proving termination of \(\pi\)-calculus processes.
Although there have been a lot of studies on termination analysis for the \(\pi\)-calculus
and related calculi~\cite{Deng06IC,Demangeon07,SangiorgiTermination,KobayashiHybrid,Yoshida04IC,DBLP:journals/jlp/DemangeonHS10,Venet98SAS}, most of them have been rather theoretical,
and there have been surprisingly little efforts in developing  fully automated termination
verification methods and tools based on them. To our knowledge,
Kobayashi's \typical{}~\cite{TyPiCal,KobayashiHybrid} is the only exception that
can prove termination of \(\pi\)-calculus processes (extended with natural numbers)
fully automatically, but its termination analysis is quite limited (see Section~\ref{sec:relatedwork}).

Our method is based on a reduction to termination analysis for sequential programs:
we translate a \(\pi\)-calculus process \(P\) to a sequential program \(S_P\), so that
if \(S_P\) is terminating, so is \(P\). The reduction allows us to use
powerful, mature methods and tools
for termination analysis of sequential programs~\cite{heizmann2016ultimate,freqterm,DBLP:conf/lics/PodelskiR04,Kuwahara2014Termination,DBLP:journals/cacm/CookPR11}.

The idea of the translation is to convert a chain of communications on replicated input
channels to a chain of recursive function calls of the target sequential program.
Let us consider the following Fibonacci process:
\begin{align*}
    & \rinexpIntro{\fib}{n}{r}
        \ifexp{n<2}{ \soutatom{r}{1} \\ &\quad}
                   { \nuexp{s_1} \nuexp{s_2} (\outatomIntro{\fib}{n-1}{s_1} \PAR \outatomIntro{\fib}{n-2}{s_2} \PAR \sinexp{s_1}{x}\sinexp{s_2}{y}\soutatom{r}{x+y}) \\}
    & \PAR \outatomIntro{\fib}{m}{r}
\end{align*}
Here, the process
$\rinexpIntro{\fib}{n}{r} \ldots$ is a function server that computes the \(n\)-th Fibonacci number
in parallel and returns the result to \(r\),
and $\outatom{\fib}{m}{r}$ sends a request for computing the \(m\)-th Fibonacci number;
those who are not familiar with the syntax of the \(\pi\)-calculus may wish to consult
Section~\ref{sec:targetlanguage} first.
To prove that the process above is terminating for any integer \(m\),
it suffices to show that there is no infinite chain of communications on $\fib$:
\[
    \fib(m,r) \to \fib(m_1,r_1) \to \fib(m_2,r_2) \to \cdots.
\]
We convert the process above to the following program:\footnote{The actual translation
  given later is a little more complex.}
\begin{verbatim}
 let rec fib(n) = if n<2 then () else (fib(n-1) [] fib(n-2)) in
 fib(m)
\end{verbatim}
Here, \texttt{[]} represents the non-deterministic choice.
Note that, although the calculation of Fibonacci numbers is not preserved,
for each chain of communications on \texttt{fib}, there is a corresponding
sequence of recursive calls:
\[
\mathtt{fib}(m) \to \mathtt{fib}(m_1) \to \mathtt{fib}(m_2) \to \cdots.
\]
Thus, the termination of the sequential program above implies the termination of
the original process.
As shown in the example above, (i) each communication on a replicated input channel
is converted to a function call, (ii) each communication on a non-replicated input
channel is just removed (or, in the actual translation, replaced by a call of
a trivial function defined by \(f(\seq{x})=(\,)\)), and (iii) parallel composition
is replaced by a non-deterministic choice.
We formalize the translation outlined above and prove its correctness.

The basic translation sketched above sometimes loses too much information.
For example, consider the following process:
\begin{align*}
    & \rinexpIntro{\pre}{n}{r} \soutatom{r}{n-1} \\
    & \PAR \rinexpIntro{f}{n}{r} \ifexp{n<0}{ \soutatom{r}{1} }
                                       { \nuexp{s} (\outatomIntro{\pre}{n}{s} \PAR \sinexp{s}{x}\outatomIntro{f}{x}{r}) } \\
    & \PAR \outatomIntro{f}{m}{r}
\end{align*}
The translation sketched above would yield:
\begin{verbatim}
  let pred(n) = n-1 in
  let rec f(n) = if n<0 then () else (pred(n) [] f(*)) in
  f(m)
\end{verbatim}
Here, \texttt{*} represents a non-deterministic integer: since we have removed
the input $\sinatom{s}{x}$, we do not have information about the value of \( x \).
As a result, the sequential program above is non-terminating, although the original
process is terminating.
To remedy this problem, we also refine the basic translation above by using a refinement
type system for the \(\pi\)-calculus. Using the refinement type system,
we can infer that the value of \(x\) in the original process is less than \(n\),
so that we can refine the definition of \texttt{f} to:
\begin{verbatim}
 let rec f(n) = ... else (pred(n) [] let x=* in assume(x<n);f(x))
\end{verbatim}
The target program is now terminating, from which
we can deduce that the original process is also terminating.
We have implemented an automated tool based on the refined translation above.

The contributions of this paper are summarized as follows.
\begin{itemize}
\item The formalization of the basic translation from the \(\pi\)-calculus
  (extended with integers) to sequential programs, and a proof of its correctness.
\item The formalization of a refined translation based on a refinement type system.
\item An implementation of the refined translation, including automated refinement type
  inference based on CHC solving, and experiments to evaluate the effectiveness of
  our method.
\end{itemize}

The rest of this paper is structured as follows.
Section~\ref{sec:targetlanguage} introduces the source and target languages
of our translation.
Section~\ref{sec:approach} 
formalizes the basic translation, and proves its correctness.
Section~\ref{sec:refinement} refines the basic translation by using a refinement type system.
Section~\ref{sec:implementation} reports an implementation and experiments.
Section~\ref{sec:relatedwork} discusses related work,
and Section~\ref{sec:conclusion} concludes the paper.


\section{Preliminaries}

\section{Preliminaries}\label{chpt:preliminiaries}
In this chapter we will introduce some of the mathematical background and notation needed for this thesis. In particular, we will shortly introduce the differential geometric description of spacetime in Section \ref{sec:spacetime_geometry} and give an introduction to the notion of global hyperbolicity and its connection to Green- and normally-hyperbolic operators in Section \ref{sec:global_hyperbolicity}. In a bit more detail, we will introduce the notion of differential forms and give explicit definitions, also in terms of an index based notation, in Section \ref{sec:differential_forms}. For completeness, in Section \ref{sec:cat-theory}, we present basic definitions of category theory. The reader familiar with these topics can safely skip this chapter and refer to it when interested in the chosen conventions.
%
%
%
%
%%%%%%
%%SPACTIME GEOMETRY
%%%%%
%
%
%
\subsection{Spacetime geometry}\label{sec:spacetime_geometry}
In GR, the universe is mathematically described as a four dimensional \emph{spacetime}, consisting of a smooth, four dimensional manifold \gls{M} (assumed to be Hausdorff, connected, oriented, time-oriented and para-compact) and a Lorentzian metric $g$. We will assume the signature of the Lorentzian metric $g$ to be $(-,+,+,+)$. The Levi-Civita connection on $(\M,g)$ is as usual denoted by \gls{nabla}.
Throughout this thesis, we treat spacetime as fixed, implementing a gravitational background determined classically by Einstein's field equations. Hence, we neglect any back-reaction of the fields on the metric, both in the quantum and the classical case. In that sense, we treat the fields as \emph{test fields}.\par
For the basic mathematical theory regarding Lorentzian manifolds, we refer to the literature: An introduction to the topic with an emphasis on the physical application in GR is for example given in \cite{wald_GR} and \cite{carroll_spacetime-and-gr}.
Here, we will shortly recap the notion of a tangent space and tangent bundle and generalize to the notion of a vector bundle which we will use in the general description of normally hyperbolic operators and differential forms.
In the following, we generalize the setting to an arbitrary smooth manifold $\N$ of dimension $N$ with either Lorentzian or Riemannian metric $k$.\par
%
%
A \emph{tangent vector} $v_x$ at point $x \in \N$ is a linear map $v_x : C^\infty(\N , \IR) \to \IR$ that obeys the Leibniz rule, that is, for $f,g \in C^\infty (\N,\IR)$ it holds $v_x(fg) = f(x)v_x(g) + v_x(f)g(x)$.
We define the \emph{tangent space} \gls{TxN} of $\N$ at $x$ as the real $N$-dimensional vector space of all tangent vectors at point $x$.
The disjoint union of all tangent spaces is called the \emph{tangent bundle} \gls{TN} of $\N$ and is itself a manifold of dimension $2N$. A \emph{vector field} is a map $v: \N \to T\N$ such that $v(x) \in T_x\N$.
The respective dual spaces, that is the space of all linear functionals, the \emph{co-tangent space} and the \emph{co-tangent bundle}, are denoted by \gls{TsxN} and \gls{TsN} respectively.\par
%
For Lorentzian manifolds, we call a tangent vector $v$ at $x \in \N$ \emph{timelike} if $k_{\mu \nu} v^\mu v^\nu < 0$, \emph{spacelike} if $k_{\mu \nu} v^\mu v^\nu > 0$ and \emph{null} (or lightlike) if $k_{\mu \nu} v^\mu v^\nu = 0$. At every point $x \in \N$, we define the set of all \emph{causal}, that is, either timelike or null, tangent vectors in the tangent space at $x$. This set is called the \emph{light cone} at $x$ and it is split up into two distinct parts, one that we call the future light cone, and one that we call the past light cone at $x$. Since we assume the manifold to be time orientable, there exists a smooth vector field $t$ that is timelike at every $x \in \N$. Given this time orientation, we identify the future (past) light cone with the set of tangent vectors $v \in T_x\N$ such that $k_{\mu\nu} v^\mu t^\nu < 0$ (respectively $> 0$). Therefore, a tangent vector $v$ at $x$ is called \emph{future directed} (past directed) if it lies in the future (past) light cone at $x$.\\
Accordingly, a curve $\gamma : I \to \N$ is called timelike (spacelike, null, causal, future or past directed) if its tangent vector $\dot{\gamma}$ is timelike (spacelike, null, causal, future or past directed) at every $x \in \N$.  For every point $x \in \N$ we define the \emph{causal future/past} \gls{causalfuturepast} of $x$ as the set of all points $q \in \N$ that can be reached by a future directed causal curve originating in $x$. For any subset $S \in \N$ we define $J^\pm (S) = \bigcup_{x \in S} J^\pm(x)$ and $J(S) = J^+(S) \cup J^- (S)$. Finally, the future/past domain of dependence $\gls{futurepastdomainofdependence}$ of a set $S \subset \N$ is the set of all points $x \in \N$ such that every inextendible causal curve through $x$ intersects $S$. The \emph{domain of dependence} \gls{domainofdependence} of $S$ is the union of the future and past domain of dependence of the set $S$.
For more details on the causal structure of spacetime we refer to for example \cite[Chapter 8]{wald_GR}.\par
%
%
%
The notion of tangent bundles can be generalized to the notion of a vector bundle. Instead of ``attaching'' the vector spaces $T_x \N$ to every point $x$ of the manifold, we allow for the occurrence of arbitrary vector spaces, called the fibres of the vector bundle. A vector bundle then consists of the base manifold, in our case $\N$, the total space and a map $\pi$ from the total space to the base manifold, that can be locally trivialized. At each point of the base manifold, the pre-image of $\pi$ is the fibre of the vector bundle. To be precise we define, following \cite{rudolph_schmidt}:
\begin{definition}[Vector bundle]
	A smooth \emph{vector bundle} over $\N$ is a tuple $\gls{vectorbundle} = (E,\N, \pi)$, where $E$ is a smooth manifold and $\pi : E \to \N$ is a smooth surjective map satisfying:
	\begin{enumerate}
		\item For every $x \in \N$, $\pi^{-1}(x)$ is a vector space, called the fibre of the bundle at point $x$.
		\item There exists a finite dimensional vector space $F$, an open covering $\left\{ U_\alpha\right\}_\alpha$ of $\N$ and a family of diffeomorphisms $\chi_\alpha : \pi^{-1}(U_\alpha) \to U_\alpha \times F$ such that for all $\alpha$ it holds $\chi_\alpha \comp \text{pr}_1 =  \restr{\pi}{\pi^{-1}(U_\alpha)}$ and for every $x \in \N$ the map $\text{pr}_2 \comp \restr{\chi_\alpha}{\pi^{-1}(x)} : \pi^{-1}(x) \to F$ is linear.
	\end{enumerate}
\end{definition}
Here, the maps $\text{pr}_1$ and $\text{pr}_2$ denote the projection onto the first respectively second component of an element in $U_\alpha \times F$. The properties graphically mean that \emph{locally}, the vector bundle ``looks like" the product of the base manifold with the fibre. The tuples $(U_\alpha, \chi_\alpha)$ are called \emph{local trivializations} of the vector bundle. Like for vector spaces, we can define the sum and product of vector bundles, by using the according vector space definitions on the fibres of the bundle.\par
Let $\mathfrak{X}, \mathfrak{Y}$ be vector bundles over $\N$ with fibres $X_x$ and $Y_x$ at $x \in \N$. We denote by \gls{whitneysum} the \emph{Whitney sum} of the two vector bundles - the vector bundle over $\N$ whose fibres are given by the direct sum $X_x \oplus Y_x$. Similarly, one obtains the local trivializations of the Whitney sum from the trivializations of $\mathfrak{X}, \mathfrak{Y}$ and direct sums.\par
Accordingly, let $\mathfrak{X}, \mathfrak{Y}$ be vector bundles over $\N$ and $\widetilde{\N}$, with fibres $X_x$ and $Y_{\tilde{x}}$ at $x \in \N$, $\tilde{x} \in \widetilde{\N}$ respectively. We denote by \gls{outerproductbundle} the \emph{outer product} of the two vector bundles - the vector bundle over $\N \times \widetilde{\N}$ whose fibres are given by the tensor products $X_x \otimes Y_x$. Similarly, one obtains the local trivializations of the outer product from the trivializations of $\mathfrak{X}, \mathfrak{Y}$ and tensor products. \par
%
Finally, we generalize the notion of vector fields:
\begin{definition}[Sections of vector bundles]
Let $\mathfrak{X}=(E,\N,\pi)$ be a vector bundle with fibres $X_x=\pi^{-1}(x)$ at $x \in \N$. A \emph{smooth section} of the vector bundle is a smooth map $\gamma : \N \to E$ such that $\gamma(x) \in X_x$ for all $x \in \N$. The \emph{vector space of smooth sections} of $\mathfrak{X}$ is denoted by \gls{gammax}, the one with compactly supported sections is as usual denoted by \gls{gammaxzero}.
\end{definition}
In this language, a vector field $v$ is just a smooth section of the tangent bundle of a manifold, $v \in \Gamma(T\N)$. One may therefore identify the physical notion of fields with smooth sections of vector bundles. This point of view will be used to define the notion of differential forms in Section \ref{sec:differential_forms}.\par
In this thesis, we usually are interested in complex valued functions (or sections in general). Therefore, we view all occurring vector bundles as complex, in the sense that we take two distinct copies of the vector bundle, one representing the real, one the imaginary part of the bundle. A section of that complex vector bundle is just a pair of two sections of the real vector bundle under consideration. From now, if not specified explicitly, we will view all vector bundles, including the tangent bundle $T\N$, as complex vector bundles. Accordingly, smooth sections of those bundles will in general be complex valued.
%
%
%
%
%
%
%
%
%%%%%%%
%%PARTIAL DIFFERENTIAL OPERATORS AND GLOBAL HYPERBOLICITY
%%%%%%%
%
%
%
\subsection{Partial differential operators and global hyperbolicity}\label{sec:global_hyperbolicity}
When dealing with field theories, whether classical or quantum, one is, of course, interested in the dynamics of the fields. These are usually described by some partial differential equation, often of second order. In the following, we give a short introduction to the theory of certain partial differential operators acting on smooth sections of a vector bundle over the spacetime $(\M,g)$.\par
%
As we have seen, these smooth sections are generalizations of the notion of a field.  In the following, let $\mathfrak{X}$ denote a vector bundle over the manifold $\M$ and let $P: \Gamma(\mathfrak{X}) \to \Gamma(\mathfrak{X})$ be a partial differential operator acting on smooth sections of the bundle. As in the case of flat spacetime, we are interested in basic questions regarding the differential equation $Pf = j$, for example: Can we formulate a (globally) well posed initial value problem? Does the differential equation possess (unique) solutions? To answer these questions, we will now restrict to the case where $P$ is linear and of second order, as it is often the case in physical applications. One can show that for a certain class of such operators, namely normally hyperbolic partial differential operators of second order, we can rigorously treat these questions.\par
Choosing local coordinates $x=(x_\mu)$ on $\M$ and a local trivialization of $\mathfrak{X}$, a linear partial differential operator of second order is called \emph{normally hyperbolic} if it takes the form
\begin{align}
	P = - \sum_{\mu,\nu} g^{\mu \nu} \partial_\mu \partial_\nu + \sum_{\alpha} A_\alpha (x) \partial_\alpha + B(x) \formspace,
\end{align}
where $A_\alpha$ and $B$ are matrix-valued coefficients depending smoothly on the coordinate $x$ (see. \cite[Chapter 1.5]{baer_ginoux_pfaeffle}). One can also formulate a coordinate independent definition in terms of the principal symbol, which we will not present here (see for example \cite[Section 1.5]{baer_ginoux_pfaeffle} ). \par
%
Normally hyperbolic operators possess unique fundamental solutions (see for example the fundamental solutions to the wave operator as noted in Lemma \ref{lem:fundamental_solution_wave_operator}). These fundamental solutions fulfill certain physically important properties, such as a finite propagation speed smaller than the speed of light. Furthermore, specifying the initial data on some space-like hypersurface $X \in  \M$ specifies a unique solution on the domain of dependence $D(X)$ of $X$. Due to these properties, one often calls normally hyperbolic operators just \emph{wave operators}. But to state a \emph{globally} well posed initial value problem for a wave equation, we need to restrict the class of spacetimes $\M$ under consideration to those that possess space-like hypersurfaces $X$ whose domain of dependence is all of the spacetime, $D(X) = \M$. This leads to the notion of \emph{globally hyperbolic} spacetimes:
\begin{definition}[Global Hyperbolicity]
	A spacetime $\M$ is called \emph{globally hyperbolic} if there exists a Cauchy surface $\gls{sigma}$ in $\M$.
\end{definition}
\noindent Here, a Cauchy surface is a space-like hypersurface $\Sigma \subset \M$ such that every inextendible causal curve $\gamma$ intersects $\Sigma$ exactly once. One can show that Cauchy surfaces fulfill the desired property mentioned above, that is,  $D(\Sigma) = \M$. Furthermore, one can show that any globally hyperbolic spacetime $\M$ is foliated by a one-parameter family $\left\{ \Sigma_t \right\}_t$ of Cauchy surfaces (see for example \cite[Theorem 8.3.14]{wald_GR}). \par
In physical applications, one often finds the dynamics of a theory to be described by wave operators. Most prominently, the Klein-Gordon operator $(\square + m^2)$ acting on scalar fields, or its generalization, the wave operator acting on differential forms introduced in Section \ref{sec:differential_forms}, is normally hyperbolic. But there are also important physical field theories that are not described by wave operators, such as the Proca field treated in this thesis. It turns out that the Proca operator (see Definition \ref{def:proca_operator}) is a so called \emph{Green-hyperbolic} operator. These are again partial differential operators $P$ of second order acting on smooth sections of some vector bundle, such that $P$ (and its dual $P'$) posses fundamental solutions. Obviously, normally hyperbolic operators are Green-hyperbolic, but the opposite is not true. One can generalize some results obtained by studying normally hyperbolic operators to Green-hyperbolic operators. An introduction to this topic is given in \cite{baer_green-hyperbolic}, where it is also shown that the Proca operator is Green-hyperbolic but not normally hyperbolic.\par
For our application, the notion of Green-hyperbolicity is not of vast importance, but it is worth mentioning that there exists a more detailed mathematical background on the treatment of such operators.
A very detailed description of normally hyperbolic operators on Lorentzian manifolds, including proofs of the above statements regarding the initial value problem and the existence of fundamental solutions, is given in \cite{baer_ginoux_pfaeffle}, also with an overview of quantization. A shorter introduction to the topic is for example treated in \cite{baer-ginoux_classical-and-quantum-fields}, also with a description of quantization.
%
%
%
%
%
%
%%%
%
%
%
%%
%%%%%%%%%
%%%DIFFERENTIAL FORMS
%%%%%%%%
%
%
%
\subsection{Differential forms}\label{sec:differential_forms}
%
%
Differential forms provide an elegant, coordinate independent description of calculus on smooth manifolds. In particular, they generalize the notion of line- and volume-integrals that are known from analysis. Differential forms play a remarkable role in physics, as one can argue that they indeed describe fundamental physical entities. As an example, instead of viewing a classical force as a vector, one can think of it, more closely related to experiments, as a differential one-form that assigns a scalar to a tangent vector of a curve. This scalar is the (infinitesimal) work associated with the force along the curve. Also, differential forms allow for an elegant geometric description of field theories, for example the Maxwell and Proca field theories that we encounter in this thesis. In Maxwell's classical theory of electromagnetism, instead of viewing the electric and magnetic field (which are conceptually just forces) as the fundamental physical entities, one introduces the \emph{vector potential}, a one-form, consisting of the scalar electric potential and the vector potential associated with the magnet field. Experiments like the Aharonov-Bohm experiment allow for an interpretation of the vector potential as the fundamental physical object, rather than the associated electromagnetic field. \\
Even more fundamentally, the two main theories of physics, General Relativity and the Standard Model of particle physics, are field theories. They are deeply connected to a geometric interpretation and can be elegantly described using differential forms. \par
%
%
Despite of all this, differential forms are usually not part of the standard curriculum of physicists. We shall therefore introduce the basic aspects and definitions regarding differential forms that are used in this thesis. For a more detailed introduction we refer to the literature: For example \cite[Chapter 2 and 4]{rudolph_schmidt} or \cite[Appendix B]{wald_GR} provide introductions to the topic.\par
%
%
In the following, let $\N$ denote a smooth $N$-dimensional manifold, assumed to be Hausdorff, connected, oriented and para-compact, with either Lorentzian or Riemannian metric $k$ and Levi-Civita connection $\nabla$. For a Lorentzian manifold we use the sign convention $(-,+,\dots,+)$ of the metric $k$. The number of negative eigenvalues of $k$ is denoted by $s$, so $s=0$ for a Riemannian manifold and, in our convention, $s=1$ for a Lorentzian manifold.
Later, we will specify to a four dimensional (globally hyperbolic) spacetime consisting of a four dimensional manifold $\M$ with Lorentzian metric $g$ and Cauchy surface $\Sigma$ with induced Riemannian metric $h$.
%
We define:
\begin{definition}[Differential form]
	Let $p\in \{0,1,\dots,N\}$. A \emph{differential form} $\omega$ of degree $p$, or $p$-form for short, on the manifold $\N$ is an anti-symmetric tensor field of rank $(0,p)$. That is, at every point $x \in \N$, $\omega_x$ is an anti-symmetric multi-linear map
	\begin{align}
	\omega_x : \underbrace{T_x \N \times T_x \N \times \cdots \times T_x \N}_{p\text{-times}} \to \IR \formspace.
	\end{align}
	We denote the vector space\footnote{Naturally, addition and scalar multiplication are defined point-wise.} of $p$-forms on $\N$ by $\gls{omegap}$, the space with compactly supported ones by \gls{omegapz}.
\end{definition}
As an example, a zero-form $f \in \Omega^0(\N)$ is just a $C^\infty$-function from $\N$ to $\IR$, hence we can identify $\Omega^0(\N) = C^\infty (\N, \IR)$. A one-form $A \in \Omega^1(\N)$ is nothing more than a co-vector field and in a physical context usually denoted in local coordinates by $A_\mu$. Note, that alternatively one can directly define a $p$-form as a smooth section of the $p$-th exterior product of the co-tangent bundle and hence identify $\Omega^p(\N) = \Gamma \big( \largewedge^k T^*\N\big)$. As mentioned in Section \ref{sec:spacetime_geometry}, we view the tangent bundle as a complex bundle. Therefore, the sections of that bundle will be complex valued functionals. In that fashion, we will usually view the spaces $\Omega^p(\N)$ as complex valued differential forms.\par
%
Next we define the basic operations, besides addition and scalar multiplication, that one can perform on differential forms.
%
\begin{definition}[Exterior product]
	Let $A \in \Omega^p(\N)$ be a $p$-form and  $B\in \Omega^q(\N)$ a $q$-form on $\N$. \\
	The \emph{exterior product} $\gls{wedge}:\Omega^p(\N) \times \Omega^q(\N) \to \Omega^{p+q} (\N)$ is defined by
	\begin{align}
	(A \wedge B)_{\mu_1\dots\mu_p \nu_1\dots\nu_q} = \frac{(p+q)!}{p!q!}\, A_{[\mu_1 \dots \mu_p} B_{\nu_1\dots\nu_q]} \formspace,
	\end{align}
	where the anti-symmetrization of a tensor $T$ is given through
	\begin{align}
	T_{[\mu_1\dots\mu_p]} = \frac{1}{p!} \sum\limits_{\sigma\in S_N }\textrm{sgn}(\sigma) T_{\sigma(\mu_1)\dots\sigma(\mu_p)} \formspace.
	\end{align}
\end{definition}
Here, $S_N$ denotes the symmetric group\footnote{Usually the symmetric group is defined as the set of permutations of $\{1,2,\dots,N\}$ but we chose the index to run over $\{0,1,\dots,N-1\}$, identifying the time component with zero rather then one.} of degree $N$, consisting of permutations of the set $\{0,1,\dots,N-1\}$.
With this notion of multiplication, point-wise addition and scalar multiplication, the space $\gls{omega} \coloneqq \bigoplus_{p = 0}^\infty \Omega^p(\N) = \bigoplus_{p = 0}^N \Omega^p(\N)$ becomes an algebra, usually called the Grassmann- or \emph{exterior algebra} of differential forms on $\N$. We have used that obviously $\Omega^k(\N) =0$ for $k >N$ due to the anti-symmetrization.\par
Furthermore, we find a notion of how to \emph{pullback} differential forms on manifolds to another manifold, for example the pullback of a differential form on the spacetime $\M$ to differential forms on its Cauchy surface $\Sigma$. Given a $C^\infty$-map $\psi: \widetilde{\N} \to \N$, where $\N, \widetilde{\N}$ are manifolds, we can naturally define the pullback of a function $f \in \Omega^0(\N)$ to a function $(\psi^* f) \in \Omega^0(\widetilde{\N})$ by composing $f$ with $\psi$:
\begin{align}
\psi^* f \coloneqq f \comp \psi \formspace.
\end{align}
\newpage
With the pullback of functions defined, we can define how to \emph{push forward}, or carry along, vector fields on $\widetilde{\N}$ to vector fields on $\N$: Let $f\in \Omega^0(\N)$ and $\tilde{v} \in \Gamma(T\widetilde{\N})$ and $\tilde{x} \in \widetilde{\N}$. Then
\begin{align}
(\psi_* \tilde{v})_{\psi(\tilde{x})} (f) \coloneqq \tilde{v}_{\tilde{x}}(\psi^* f)
\end{align}
defines the vector field $(\psi_* v) \in \Gamma(T\N)$. With these basic operations at hand, we can generalize to define the pullback of differential forms:
\begin{definition}[Pullback]\label{def:pullback}
	Let $\N, \widetilde{\N}$ be manifolds of dimension $N,\widetilde{N}$ respectively, and let $\psi: \widetilde{\N} \to \N$ be a smooth map. Then, $\psi$ defines an algebra homomorphism $\psi^* : \Omega(\N) \to  \Omega(\widetilde{\N})$,
	called the \emph{pullback} of differential forms. For $\omega \in \Omega^p(\N)$, $\tilde{x} \in \widetilde{\N}$ and $\tilde{v}_i \in T_x \widetilde{\N}$, $i=1,2,\dots,p$, it is defined by
	\begin{align}
	\left( \psi^* \omega \right)_{\tilde{x}}  (\tilde{v}_1,\tilde{v}_2,\dots,\tilde{v}_p) \coloneqq \omega_{\psi(\tilde{x})} (\psi_* \tilde{v}_1, \dots , \psi_* \tilde{v}_p) \formspace.
	\end{align}
\end{definition}
%
%
%
%
On the exterior algebra we find a duality, provided by the Hodge operator:
\begin{definition}[Hodge dual]
	The hodge star operator $\gls{hodge}: \Omega^p(\N) \to \Omega^{N-p}(\N)$ is defined through
	\begin{align}
	B \wedge *A = \frac{1}{p!} B^{\mu_1\dots\mu_p}A_{\mu_1\dots\mu_p} \dvolk \formspace,
	\end{align}
	which yields the coordinate representation
	\begin{align}
	(*A)_{\mu_{p+1}\dots\mu_N} = \frac{\detk}{p!} \, \epsilon_{\mu_1\dots\mu_N} A^{\mu_1\dots\mu_p} \formspace.
	\end{align}
\end{definition}
Here, \gls{levicivita} denotes the fully antisymmetric tensor of rank $N$ (Levi-Civita symbol) satisfying $\epsilon_{12,\dots,N} =1$ and the \emph{volume element} \gls{dvolk} is defined by
\begin{align}
\left( \gls{dvolk} \right)_{\alpha_1\dots\alpha_N} = \detk \, \epsilon_{\alpha_1\dots\alpha_N} \formspace.
\end{align}
In a sense, the volume element describes how the curvature of the manifold deforms a unit volume.
The duality follows from the important property of the Hodge operator as stated in the following lemma:
\begin{lemma}
	Let $*$ denote the Hodge star operator on the exterior algebra $\Omega(\N) $. It holds that
	\begin{align}
	** = (-1)^{s+p(N-p)} \, \mathbbm{1} \formspace,
	\end{align}
	which is trivially equivalent to $*^{-1} = (-1)^{s+p(N-p)} \, *$.
\end{lemma}
\begin{proof}
	Let $A \in \Omega^p(\N)$ be a $p$-form on $\N$. Then:
	\begin{align}
	(*{*A})_{\mu_1 \dots \mu_p}
	&= \frac{\detk \, \detk}{p! \, (N-p)!} \; \epsilon_{\alpha_{p+1}\dots\alpha_N \mu_1 \dots \mu_p}\;\epsilon^{\alpha_{1}\dots\alpha_N}\;A_{\alpha_1\dots\alpha_p} \notag\\
	&= (-1)^{p(N-p)} \frac{\detk \, \detk}{p! \, (N-p)!} \; \epsilon_{\alpha_{p+1}\dots\alpha_N \mu_1 \dots \mu_p}\;\epsilon^{\alpha_{p+1}\dots\alpha_{N}\alpha_1\dots\alpha_p}\;A_{\alpha_1\dots\alpha_p}  \notag\\
	&= (-1)^{s+p(N-p)} \delta\indices{^{[\alpha_{1}}_{\mu_{1}}}\, \dots \, \delta\indices{^{\alpha_p ] }_{\mu_p}} \;A_{\alpha_1\dots\alpha_p} \notag\\
	&=  (-1)^{s+p(N-p)}\;A_{\mu_1\dots\mu_p} \formspace
	\end{align}
	We have used Lemma \ref{lem:epsilon_contraction} and, in the last step, that the anti-symmetrization is absorbed by contraction because $A$ is antisymmetric.
\end{proof}
%
%
%
%
%
Furthermore, we can equip the exterior algebra with a differentiable structure, introducing the notion of the exterior derivative.
\begin{definition}[Exterior derivative]
	The \emph{exterior derivative} $\gls{d}:\Omega^p(\N) \to \Omega^{p+1} (\N)$ is defined by the following properties:
	\begin{enumerate}
		\item $d$ is linear
		\item $d$ obeys a graded Leibniz rule: Let $A \in \Omega^p(\N)$ and  $B\in \Omega^q(\N)$, then
		\begin{align}
		d(A \wedge B) = dA \wedge B + (-1)^p \, A \wedge dB
		\end{align}
		\item $d$ is nilpotent, that is,  $d^2 = 0$.
	\end{enumerate}
	In local coordinates, this is equivalent to the representation
	\begin{align}
	(dA)_{\mu \alpha_1\dots\alpha_p} = (p+1)\, \nabla_{[\mu}A_{\alpha_1\dots\alpha_p]} \formspace.
	\end{align}
\end{definition}
An important property of the exterior derivative is that it commutes (or rather intertwines its action) with pullbacks (see \cite[Proposition 4.1.7]{rudolph_schmidt}).
A $p$-form $\omega \in \Omega^p(\N)$ is called \emph{exact} if there is a $(p-1)$-form $\alpha \in \Omega^{p-1}(\N)$ such that $\omega = d\alpha$. We call $\omega$ \emph{closed} if $d \omega =0$. Accordingly, the space of closed $p$-forms is denoted by \gls{omegapd}, the space of exact ones by \gls{domegap}. As usual, the ones with compact support are denoted by a subscript zero. Note, that every exact form is closed, using that $d$ is by definition nilpotent, but the reverse is in general not true. It does hold, however, on certain manifolds with trivial topology, such as Minkowski spacetime. This is expressed in the so called Poincar\'e-Lemma (see for example \cite[Chapter 4]{bott_tu}) based on the study of de Rham cohomology.\par
%
Moreover, $N$-forms can naturally be integrated. Using local coordinates and a partition of unity, we define the integral of $N$-forms via the well known integration on $\IR^N$:
\begin{definition}[Integration on manifolds]
	Let $\left\{U_\alpha, \psi_\alpha\right\}_\alpha$ be an atlas of the manifold $\N$ and $\left\{\chi_\alpha\right\}_\alpha$ a partition of unity subordinate to the locally finite open cover $\left\{U_\alpha\right\}_\alpha$. Let $x^\mu_{(\alpha)}$ be a coordinate basis of $\psi$ on $U_\alpha$. For any $N$-form $\omega \in \Omega^N_0(\M)$ we define the integral
	\begin{align}
	\int\limits_{\N} \omega &\coloneqq \sum_{\alpha} \int\limits_{\psi_\alpha (U_\alpha)} w(x_{(\alpha)}^0,\dots,x_{(\alpha)}^1)\; dx_{(\alpha)}^0 \cdots dx_{(\alpha)}^{N-1} \formspace,
	\end{align}
	where $w$ are the components of $\omega$ in the coordinates $x_{(\alpha)}^\mu$, that is $\omega = w dx_{(\alpha)}^0 \wedge \cdots \wedge dx_{(\alpha)}^{N-1}$.
	This definition is independent of the choice of the atlas and the partition of unity (see \cite[Proposition 3.3]{bott_tu}).
\end{definition}
With integration at our disposal, we present an important theorem regarding the integration of exact differential forms:
\begin{theorem}[Stoke's Theorem]\label{thm:stokes}
	Let $\N$ be an oriented manifold of dimension $N$ and let its boundary $\partial \N$ be endowed with the induced orientation. Let $\gls{inclusionmap} : \partial \N \hookrightarrow \N$ be the inclusion operator.
	Let $\omega \in \Omega^{N-1}_0(\N)$ be a compactly supported $(N-1)$-form on $\N$. Then it holds
	\begin{align}
	\int\limits_\N d\omega = \int\limits_{\partial \N} i^*\omega \formspace.
	\end{align}
\end{theorem}
\begin{proof}
	A proof is given in most of the introductory literature on differential geometry (see for example \cite[Chapter 17, Theorem 2.1]{lang}).
	Note that one can equivalently formulate Stoke's theorem on a \emph{compact} manifold but for {arbitrary} (that is, in general not compactly supported) $(N-1)$-forms on the manifold (see for example \cite[Theorem 4.2.14]{rudolph_schmidt}). This will be of importance in later calculations.
\end{proof}
%
Furthermore, we can define a bilinear map on $\Omega^p(\N)$ using the integration of $N$-forms:
\begin{definition}
	Let $A,B \in \Omega^p(\N)$ such that their supports have a compact intersection. Define the bilinear map $\gls{innerprod} : \Omega^p(\N) \times \Omega^p(\N) \to \IC$ by
	\begin{align}
	\langle A, B \rangle_\N \coloneqq  \int_{\N } A \wedge * B = \int_{\N } A_{\mu_1 \dots \mu_p}B^{\mu_1 \dots \mu_p}\,\dvolk \formspace.
	\end{align}
\end{definition}
Since by definition $A \wedge * B$ is a compactly supported $N$-form, this is well defined. We may sometimes refer to $\langle \cdot , \cdot \rangle_\N$ as an inner product for simplicity, even though it is not positive definite.
%
%
%
%
%
Using the exterior derivative, we define the interior or co-derivative:
\begin{definition}[Interior derivative]
	The \emph{interior derivative} $\gls{delta} : \Omega^p(\N) \to \Omega^{p-1}(\N)$ is defined by
	\begin{align}
	\delta \coloneqq (-1)^{s+1+N(p-1)}\, {*{d*}} \formspace.
	\end{align}
	From the defining properties of $d$ and $*$ it follows $\delta^2 =0$.
\end{definition}
Here, $s$ again denotes the number of negative eigenvalues of the metric $k$ of $\N$. In accordance with our nomenclature, we call a $p$-form $\omega$ co-exact if there exists a $\alpha \in \Omega^{p+1}(\N)$ such that $\omega = \delta \alpha$ and co-closed if $\delta \omega = 0$. Accordingly, the spaces of co-closed and co-exact $p$-forms are denoted by \gls{omegapdelta} and \gls{deltaomegap} respectively.\par
Using the exterior and interior derivative we define the partial differential operator:
\begin{definition}[D'Alembert Operator]
	The d'Alembert (or Laplace - de Rham) operator $\gls{dalembert}: \Omega^p(\N) \to \Omega^{p}(\N)$ is defined by
	\begin{align}
	\square \coloneqq \delta d +d \delta \formspace.
	\end{align}
\end{definition}
By definition of the exterior and interior derivative, it is easy to show that $\square$ commutes with both $d$ and $\delta$:
\begin{align}
\square d &= (\delta d + d \delta )d \notag \\
&= d \delta d \notag \\
&= d (\delta d + d \delta) \formspace,
\end{align}
and analogously for $\delta$.
The d'Alembert operator, and its generalization to $(\square + m^2)$ for some constant $m > 0$, are important examples for a normally hyperbolic differential operators (see Section \ref{sec:global_hyperbolicity}) and we may therefore sometimes just refer to them as \emph{wave operators}.\par
The sign convention in the definition of the exterior derivative is chosen such that on any Lorentzian or Riemannian manifold the interior derivative is formally adjoint to the exterior derivative, that is,  for $A \in \Omega^{p}(\N)$ and $B \in \Omega^{p+1}(\N)$ it holds that
\begin{align}
\langle dA , B \rangle_{\N} = \langle A , \delta B \rangle_\N \formspace,
\end{align}
which leads to a representation in local coordinates of the Manifold given by:
\begin{align}
(\delta A)_{\mu_2\dots\mu_p} = - \nabla^{\mu_1}A_{\mu_1\dots\mu_p} \formspace.
\end{align}
To see that this is consistent, let $A \in \Omega^{p-1}(\N)$ and $B \in \Omega^{p}(\N)$ such that their supports have compact intersection.
We obtain, using Stoke's Theorem \ref{thm:stokes}:
\begin{align}
0 &= \int \limits_{\partial \N} i^* (A \wedge *B) \notag\\
&= \int \limits_{\N} d(A \wedge *B)  \notag\\
&= \int \limits_{\N} dA \wedge *B + (-1)^{p-1} A \wedge d{*B} \notag\\
&= \int \limits_{\N} dA \wedge *B + (-1)^{p-1} A \wedge *{*^{-1}}\underbrace{d{*B}}_{\textrm{is a } (N-p+1) \textrm{ form.}} \notag\\
&= \int \limits_{\N} dA \wedge *B + (-1)^{p-1}(-1)^{s+(N-p+1)(N-N+p-1)} A \wedge *{*d{*B}} \notag\\
&= \int \limits_{\N} dA \wedge *B + (-1)^{p+(1-p)(p-1)} A \wedge *\delta B \formspace.
\end{align}
It can easily be proven by induction that $\big(p+(1-p)(p-1)\big)$ is odd for any $p \in \IN$, which yields the result
\begin{align}
\langle dA , B \rangle_{\N} = \langle A , \delta B \rangle_\N \formspace.
\end{align}
The definitions stated above thus fulfill the requirement of formal adjointness of the exterior and interior derivate on an arbitrary Lorentzian or Riemannian manifold $\N$.
In local coordinates we use a partial integration to obtain
\begin{align}
\langle dA , B \rangle_\N &= \int \limits_{\N} dA \wedge * B \notag\\
%&= \int \limits_{\N} \frac{1}{p!} (dA)^{\alpha_1\dots\alpha_p}\,B_{\alpha_1 \dots \alpha_p} \, \dvolk \notag\\
&= \int \limits_{\N}  \frac{p}{p!} \nabla^{[\alpha_1}A^{\alpha_2\dots\alpha_p]}\,B_{\alpha_1 \dots \alpha_p} \, \dvolk \notag\\
&= \int \limits_{\N}  \frac{1}{(p-1)!} \nabla^{\alpha_1}A^{\alpha_2\dots\alpha_p}\,B_{\alpha_1 \dots \alpha_p} \, \dvolk \notag\\
&= - \int \limits_{\N}  \frac{1}{(p-1)!} A^{\alpha_2\dots\alpha_p}\, \nabla^{\alpha_1}B_{\alpha_1 \dots \alpha_p} \, \dvolk \notag\\
&= \langle A, \delta B \rangle_\N \formspace,
\end{align}
which yields
\begin{align}
-\nabla^{\alpha_1}B_{\alpha_1 \dots \alpha p} = (\delta B)_{\alpha_2 \dots \alpha_p}\formspace.
\end{align}
On the four dimensional spacetime $(\M,g)$ the definitions of the Hodge star operator and the interior derivative simplify, such that
\begin{align}
*_{(\M)}*_{(\M)} &= (-1)^{p+1} \mathbbm{1} \\
\delta_{(\M)} &= *_{(\M)}{d_{(\M)}*_{(\M)}} \formspace ,
\end{align}
holds on the spacetime $(\M,g)$ and
\begin{align}
*_{(\Sigma)}*_{(\Sigma)} &= \mathbbm{1} \\
\delta_{(\Sigma)} &= (-1)^p *_{(\Sigma)}{d_{(\Sigma)}*_{(\Sigma)}}
\end{align}
holds on  $(\Sigma,h)$. In the following we will drop the subscript ${(\M)}$, since we will perform all the calculations on a four dimensional spacetime, except when explicitly noted (for example with a subscript $(\Sigma)$).
%
%
%
%
%
%
%
%
%%%%%%
%%CATEGORY THEORY
%%%%%%
\subsection{Category theory}\label{sec:cat-theory}
The description of Quantum Field Theory on Curved Spacetimes (QFTCS) in the framework of \name{Brunetti}, \name{Fredenhagen} and \name{Verch} \cite{Brunetti_Fredenhagen_Verch} is based on category theory. In this thesis, we will not go into detail on those categorical aspects, however we will need some basic definitions to formulate the theory rigorously, that is namely the notion of a category and that of covariant functors, since, in the used framework, the generally covariant QFTCS is a functor.\par
Here, we present definitions given in \cite[Appendix A.1]{baer_ginoux_pfaeffle} and refer to the appropriate literature for details. We define:
\begin{definition}[Category]
	A \emph{category} $\mathsf{Cat}$ consists of the following:
	\begin{enumerate}
		\item a class $\mathsf{Obj}_\mathsf{Cat}$ whose members are called \emph{objects},
		\item a set $\mathsf{Mor}_\mathsf{Cat}(A,B)$, for any two objects $A,B \in \mathsf{Obj}_\mathsf{Cat}$, whose elements are called \emph{morphisms},
		\item for any three objects $A,B,C \in \mathsf{Obj}_\mathsf{Cat}$ there is a map
		\begin{align}
\mathsf{Mor}_\mathsf{Cat}(B,C) \times \mathsf{Mor}_\mathsf{Cat}(A,B) &\to \mathsf{Mor}_\mathsf{Cat}(A,C) \notag\\
(\psi,\phi) &\mapsto \psi \comp \phi
		\end{align}
		called the composition of morphisms subject to the relations:\vspace{4mm}
		\begin{enumerate}[label=(\arabic*)]
			\item for non equal pairs $(A,B)$, $(A',B')$ of objects, the sets $\mathsf{Mor}_\mathsf{Cat}(A,B)$ and $\mathsf{Mor}_\mathsf{Cat}(A',B')$ are disjoint,
			\item for every object $A$ there exists a morphism $\text{id}_A \in \mathsf{Mor}_\mathsf{Cat}(A,A)$ such that it holds for all objects $B$, morphisms $\psi \in \mathsf{Mor}_\mathsf{Cat}(B,A)$ and $\phi \in \mathsf{Mor}_\mathsf{Cat}(A,B)$
			\begin{align}
				\text{id}_A \comp \psi &= \psi \quad \text{and}\\
				\phi \comp \text{id}_A &= \phi \quad,
			\end{align}
			\item the composition law is associative, that is for an objects $A,B,C,D$ and any morphisms $\psi \in \mathsf{Mor}_\mathsf{Cat}(A,B)$, $\phi \in \mathsf{Mor}_\mathsf{Cat}(B,C)$ and $\chi \in \mathsf{Mor}_\mathsf{Cat}(C,D)$ it holds
			\begin{align}
				(\chi \comp \phi) \comp \psi = \chi \comp (\phi \comp \psi) \formspace.
			\end{align}
		\end{enumerate}
	\end{enumerate}
\end{definition}
%
%
%
\begin{definition}[Functor]
	Let $\mathsf{Cat1}$ and $\mathsf{Cat2}$ be categories. A \emph{covariant functor} $\mathscr{A}: \mathsf{Cat1} \to \mathsf{Cat2}$ consists of the map $\mathscr{A} : \mathsf{Obj}_\mathsf{Cat1} \to \mathsf{Obj}_\mathsf{Cat2}$ and maps $\mathscr{A}: \mathsf{Mor}_\mathsf{Cat1}(A,B) \to \mathsf{Mor}_\mathsf{Cat2}\big(\mathscr{A}(A),\mathscr{A}(B)\big)$ for any two objects $A,B \in \mathsf{Obj}_\mathsf{Cat1}$ such that
	\begin{enumerate}
		\item {the composition is preserved, that is for all objects $A,B,C \in \mathsf{Obj}_\mathsf{Cat1}$ and for any morphisms $\psi \in \mathsf{Mor}_\mathsf{Cat1}(A,B)$ and $\phi \in \mathsf{Mor}_\mathsf{Cat1}(B,C)$ it holds
		\begin{align}
			\mathscr{A}(\phi \comp \psi) = \mathscr{A}(\phi) \comp \mathscr{A}(\psi) \formspace,
		\end{align}}
		\item{
			$\mathscr{A}$ maps identities to identities, that is for any object $A \in \mathsf{Obj}_\mathsf{Cat1}$ it holds
			\begin{align}
				\mathscr{A}(\text{id}_\mathsf{A}) = \text{id}_{\mathscr{A}(A)} \formspace.
			\end{align}
			}
	\end{enumerate}
\end{definition}
%
%
%
%
%
%
%
%
%
%
%
%
%%%%%%
%%SIGN CONVENTIONS
%%%%%%
%
%
\subsection{Sign conventions}\label{sec:sign_conventions}
At certain points throughout this chapter we have had a freedom of choice regarding the signs of some entities, in particular the sign of the signature of the Lorentzian metric $g$ and that of the interior derivative $\delta$. Though at this stage the choice can be made arbitrarily, we want to make it in a way that in the end allows us to make certain physical interpretations on some parameters. More precisely, we want to interpret the parameter $m$ of the Klein-Gordon equation\footnote{or its generalization on $p$-forms} $(\square + m^2) f = 0$ for a zero-form $f \in \Omega^0(\M)$ as a mass in the physical sense. With the chosen sign convention for $\delta$ we find, using ${\delta}f = 0$:
\begin{align}
	\square f
	&= (\delta d + d \delta) f \notag\\
	&= \delta d f \notag\\
	&= - \nabla^\mu \nabla_\mu f \formspace.
\end{align}
In the following heuristic (local) argument we see
\begin{align}
	\square + m^2
	&= -\nabla^\mu \nabla_\mu + m^2 \notag\\
	&\sim \partial_t^2 + \sum_i \partial_i^2 + m^2\notag\\
	&\sim -E^2 + \abs{\vector{p}}^2 + m^2
\end{align}
which yields the correct relativistic relation of energy, momentum and mass according to $E^2 = \abs{\vector{p}}^2 + m^2$.
A similar calculation holds for the Klein-Gordon operator generalized to act on one-forms. If we had found a ``wrong'' relation between energy, momentum and mass, we would have had to adapt the chosen signs. Usually one chooses the sign of the metric and the interior derivative such that they are in some sense mathematically convenient (although one might disagree with another one's choice). We have made the choice of the metric, such that the Cauchy surfaces become Riemannian rather that ``anti-Riemannian'' (with an all minus signature), which seems more natural to some. Also, a lot of the used references on spacetime geometry (in particular the book by \name{Wald} \cite{wald_GR}) use this sign convention, which makes the application of certain formulas easier. As mentioned, the sign of the interior derivative was chosen such that it is formally adjoint to the exterior derivative (with respect the specified inner product) on all Lorentzian and Riemannian manifolds. It seemed convenient for the actual calculations to fix the sign regardless of the signature of the metric of the underlying manifold. One could equivalently have fixed the opposite sign, yielding the two derivatives to be skew-adjoint, which is also done in the literature. However, in the end, one has one freedom left to make the energy-momentum-mass relation work: that is the sign in front of the mass in the Klein-Gordon equation and all other wave equations accordingly. Hence, one regularly also finds the Klein-Gordon equation to be defined with a flipped sign of the mass term. But for our case, we want the mass $m$ in any wave equation to appear with a positive sign.
%
%


\section{Tiered Vectors}


In this section we will describe how the tiered vector data structure
from~\cite{Goodrich1999} works. 

\begin{figure}
	\includegraphics[width=\textwidth]{graphics/DSExample}
    \caption{An illustration of a tiered vector with $l = w = 3$. The elements are letters, and the tiered vector represents the sequence ABCDEFGHIJKLMNOPQRSTUVX. The elements in the leaves are the elements that are actually stored. The number above each node is its offset. The strings above an internal node $v$ with children $c_1, c_2, c_3$ is respectively $A(c_1) \cdot A(c_2) \cdot A(c_3)$ and $A(v)$, i.e.\ the elements $v$ represents before and after the circular shift. ? specifies an empty element.}
\label{fig:ds}
\end{figure}

%\paragraph{Data Structure} 
\subparagraph*{Data Structure} 
An $l$-tiered vector can be seen as a tree $T$ with root $r$, fixed
height $l - 1$ and out-degree $w$ for any $l \geq 2$.
A node $v \in T$ represents a sequence of elements $A(v)$ thus 
$A(r)$ is the sequence represented by the tiered vector. The capacity $\capacity(v)$ of a node $v$ is $w^{\height(v)+1}$. For a node $v$ with children $c_1, c_2, \ldots, c_w$, $A(v)$ is a circular shift of the
concatenation of the elements represented by its children, 
$A(c_1) \cdot A(c_2) \cdot \ldots \cdot A(c_w)$.
The circular shift is determined by an integer $\offset(v)
\in [\capacity(v)]$ that is explicitly stored for all nodes. Thus the sequence of
elements $A(v)$ of an internal node $v$ can be reconstructed by recursively
reconstructing the sequence for each of its children, concatenating these and
then circular shifting the sequence by $\offset(v)$. See Figure~\ref{fig:ds} for an illustration. A leaf $v$ of $T$
explicitly stores the sequence $A(v)$ in a circular array $\elements(v)$ with
size $w$ whereas internal nodes only store their respective offset.
 Call a node $v$ full if $|A(v)| = \capacity(v)$ and empty if $|A(v)| = 0$. In order to support fast $\aaccess$, for all nodes $v$ the elements of $A(v)$ are located in consecutive children of $v$ that
are all full, except the children containing the first
and last element of $A(v)$ which may be only partly full.

%\paragraph{Access \& Update}
\subparagraph*{Access \& Update}
To access an element $A(r)[i]$ at a given index $i$; one traverses a
path from the root down to a leaf in the tree. In each node the offset of the
node is added to the index to compensate for the cyclic shift, and the traversing is continued in the child corresponding to the newly calculated index. 
Finally when reaching a leaf, the desired element is
returned from the elements array of that leaf. The operation $\aaccess(v, i)$ returns the
element $A(v)[i]$ and is recursively computed as follows:

\begin{description} 
    \item[\quad v is internal:] Compute $i' = (i + \offset(v))
        \mod \capacity(v)$, let $v'$ be the $\lfloor i' / w \rfloor^{th}$ child of $v$ and return the element
    $\aaccess(v', i' \mod \capacity(v'))$. 
    
\item[\quad v is leaf:] Compute $i' = (i + \offset(v)) \mod w$
    and return the element $\elements(v)[i']$.
    \end{description}

The time complexity is $\Theta(l)$ as we visit all nodes on a root-to-leaf path in $T$. To navigate this path we must follow $l - 1$ child pointers, lookup $l$ offsets, and access the element itself. Therefore this requires $l - 1 + l + 1 = 2l$ memory probes.

The update operation is entirely similar to access, except
the element found is not returned but substituted with the new element. The
running time is therefore $\Theta(l)$ as well. For future use, let $\aupdate(v, i, e)$ be the operation that sets $A(v)[i] = e$ and returns the element that was
substituted. 

%\paragraph{Range Access}
\subparagraph*{Range Access}

Accessing a range of elements, can obviously be done by using the
$\aaccess$-operation multiple times, but this results in redundant traversing
of the tree, since consecutive elements of a leaf often
-- but not always due to circular shifts -- corresponds to consecutive elements of $A(r)$.
Let $\aaccess(v, i, m)$ report the
elements $A(v)[i \ldots i + m - 1]$ in order. The operation can recursively
be defined as:

\begin{description} \item[\quad v is internal:] 
    Let $i_l = (i + \offset(v)) \mod \capacity(v)$,
    and let $i_r = (i_l + m) \mod \capacity(v)$. The children of
    $v$ that contains the elements to be reported are in the range $[\lfloor i_l \cdot w / \capacity(v) \rfloor, \lfloor i_r \cdot w / \capacity(v) \rfloor] \mod w$,
    call these $c_l, c_{l+1},
    \ldots, c_r$. In order, call $\aaccess(c_l, i_l, \min(m, \capacity(c_l) -
    i_l))$, $\aaccess(c_i, 0, \capacity(c_i))$ for $c_i = c_{l+1}, \ldots,
    c_{r-1}$, and $\aaccess(c_r, e_{r-1}, 0, i_r \mod \capacity(c_r))$.
	
        \item[\quad v is leaf:] Report the elements $\elements(v)[i, i+m-1] \mod w$. \end{description}

The running time of this strategy is $O(lm)$, but saves a constant factor over the naive solution.


%\paragraph{Insert \& Delete}
\subparagraph*{Insert \& Delete}

Inserting an element in the end (or beginning) of the array can simply be
achieved using the $\aupdate$-operation. Thus the interesting part is fast insertion at an arbitrary position;
this is where we utilize the offsets.

Consider a node $v$, the key challenge is to shift a big chunk of elements $A(v)[i, i+m-1]$ one index right (or left) to $A(v)[i+1, i+m]$ to make room for a new element (without actually moving each element in the range). Look at the range of children $c_l, c_{l+1}, \ldots, c_r$ that covers the range of elements $A(v)[i, i+m-1]$ to be shifted. All elements in $c_{l+1}, \ldots, c_{r-1}$ must be shifted. These children are guaranteed to be full, so make a circular shift by decrementing each of their offsets by one. Afterwards take the element $A(c_{i-1})[0]$ and move it to $A(c_{i})[0]$ using the $\aupdate$ operation for $l < i \leq r$. In $c_l$ and $c_r$ only a subrange of the elements might need shifting, which we do recursively. In the base case of this recursion, namely when
$v$ is a leaf, shift the elements by actually moving the elements one-by-one in $\elements(v)$.

Formally we define the $\ashift(v, e, i, m)$ operation that (logically) shifts
all elements $A(v)[i, i+m-1]$ one place right to $A[i+1, i+m]$, sets $A[i] = e$ and returns the value that was previously on position $A[i+m]$ as:

\begin{description} \item[\quad v is internal:] Let $i_l = (i + \offset(v)) \mod
    \capacity(v)$, and let $i_r = (i_l + m) \mod \capacity(v)$. The children of
    $v$ that must be updated are in the range $[\lfloor i_l \cdot w / \capacity(v) \rfloor, \lfloor i_r \cdot w / \capacity(v) \rfloor] \mod w$ call these $c_l, c_{l+1}, \ldots, c_r$.
Let $e_l = \ashift(c_l, e, i_l, \min(m, \capacity(c_l) - i_l))$. Let $e_i =
\aupdate(c_i, size(c) - 1, e_{i-1})$ and set $\offset(c_i) = (\offset(c_i) - 1)
\mod \capacity(c)$ for $c_i = c_{l+1}, \ldots, c_{r-1}$. Finally call
$\ashift(c_r, e_{r-1}, 0, i_r \mod \capacity(c_r))$.
	
        \item[\quad v is leaf:] Let $e_o = \elements(v)[(i+m) \mod w]$. Move the
            elements $\elements(v)[i, (i+m-1) \mod w]$ to $\elements(v)[i+1,(i+m) \mod w]$, and set $\elements(v)[i] = e$. Return $e_o$.
    \end{description}

An insertion $\ainsert(i, e)$ can then be performed as $\ashift(root, e, i,
size(root) - i - 1)$. The running time of an insertion is $T(l) = 2T(l - 1) + w\cdot l \Rightarrow T(l) = O(2^l w)$.

%TODO: {Add illustration.}


A deletion of an element can basically be done as an inverted insertion, thus
deletion can be implemented using the $\ashift$-operation from before. A
$\adelete(i)$ can be performed as $\ashift(r, \bot, 0, i)$ followed by an
update of the root's offset to $(\offset(r) + 1) \mod \capacity(r)$.

%\paragraph{Space}
\subparagraph*{Space}

There are at most $O(w^{l-1})$ nodes in the tree and each takes up constant
space, thus the total space of the tree is $O(w^{l-1})$.
All leaves are either empty or full except the two leaves storing the first and
last element of the sequence which might contain less than $w$ elements.
Because the arrays of empty leaves are not allocated the space overhead of the arrays is $O(w)$.
Thus beyond the space required to store the $n$ elements themselves, tiered vectors
have a space overhead of $O(w^{l-1})$.

To obtain the desired bounds $w$ is maintained such that $w = \Theta(n^\epsilon)$ where $\epsilon = 1/l$ and $n$ is the number of elements in the tiered vector. This can be achieved by using global rebuilding to gradually increase/decrease the value of $w$ when elements are inserted/deleted without asymptotically changing the running times. We will not provide the details here. We sum up the original tiered vector data structure in the following theorem:

\begin{theorem}[\cite{Goodrich1999}] The original $l$-tiered vector solves the
    dynamic array problem for $l \geq 2$ using $\Theta(n^{1-1/l})$ extra space
    while supporting $\aaccess$ and $\aupdate$ in $\Theta(l)$ time and $2l$
    memory probes. The operations $\ainsert$ and $\adelete$ take $O(2^l n^{1/l})$ time.
    \label{thm:pointer}
\end{theorem}


\section{Improved Tiered Vectors}


In this paper, we consider several new variants of the tiered vector. This section considers the theoretical
properties of these approaches. In particular we are interested in the number of memory accesses that are required for the different memory layouts, since this turns out to have an effect on the experimental running time.
In Section~\ref{sec:experimental} we analyze the actual impact in practice through experiments.

\subsection{Implicit Tiered Vectors}

As the degree of all nodes is always fixed at some constant value $w$ (it may be
changed for all nodes when the tree is rebuilt due to a full root), it is possible to layout the
offsets and elements such that no pointers are necessary to navigate the
tree. Simply number all nodes from left-to-right level-by-level starting in the
root with number 0. Using this numbering scheme, we can store all
offsets of the nodes in a single array and similarly all the elements of the leaves in another array.

To access an element, we only have to lookup the offset for each
node on the root-to-leaf path which requires $l-1$ memory probes plus the final
element lookup, i.e.\ in total $l$ which is half as many as
the original tiered vector.
 The downside with this representation is that it must allocate the
two arrays in their entirety at the point of initialization (or when rebuilding). This results in a $\Theta(n)$ space overhead which is worse than the $\Theta(n^{1-\epsilon})$ space overhead from the original tiered vector.

\begin{theorem} The implicit $l$-tiered vector solves the dynamic array problem for $l \geq 2$
using $O(n)$ extra space while supporting $\aaccess$ and $\aupdate$ in
$O(l)$ time requiring $l$ memory probes. The operations $\ainsert$ and
$\adelete$ take $O(2^l n^{1/l})$ time.
\label{thm:implicit}
\end{theorem}

\subsection{Lazy Tiered Vectors}

We now combine the original and the implicit representation, to get both few memory probes and little space overhead. Instead of having a single array storing all
the elements of the leaves, we store for each leaf a pointer to a location with an array containing the leaf's elements. The array is lazily allocated in memory when elements are actually inserted into it.

The total size of the offset-array and the element pointers in the leaves is $O(n^{1-\epsilon})$. At most two leaves are only partially full, therefore the
total space is now again reduced to $O(n^{1-\epsilon})$. To navigate a root-to-leaf path, we now need to look at $l - 1$ offsets, follow a pointer from a leaf to its array and access the element in the array, giving a total of $l + 1$
memory accesses.

\begin{theorem}
        The lazy $l$-tiered vector solves the dynamic array problem for $l \geq 2$ using
        $\Theta(n^{1-1/l})$ extra space while supporting $\aaccess$ and
        $\aupdate$ in $\Theta(l)$ time requiring $l+1$ memory probes. The
        operations $\ainsert$ and $\adelete$ take $O(2^l n^{1/l})$
        time.
\label{thm:lazy}
\end{theorem}



\section{Implementation}

\newcommand{\anoise}{{\mathcal{AN}}}
\newcommand{\pnoise}{{\mathcal{PN}}}
\section{Stochastic Games for V-Formation}
\label{sec:sgv}

We describe the specialization of the stochastic-game verification problem to
V-formation.  In particular, we present the AMPC-based control strategy for reaching a V-formation, and the various attacker strategies against which we evaluate the resilience of our controller.

The MDP $\M$ for V-formation was presented in Section~\ref{sec:background}. The state variables of the MDP are the positions and velocities of the birds, and the control variables (defining the actions) are the accelerations and displacements. In the transition relation given in equation~(\ref{eq:v}), the attacker chooses the displacement $\vec{d}(t)$ it needs to manipulate the position of the birds,
whereas the controller chooses the acceleration $\vec{a}(t)$ to apply. Together, the pair $(\vec{a}(t),\vec{d}(t))$ defines the action that transforms one MDP state to another. We now define the controller's and attacker's strategies.

\subsection{Controller's Adaptive Strategies}

Given current state $(\vec{x}(t),\vec{v}(t))$, the controller's strategy $\sigma_C$ returns a probability distribution on the space of all possible accelerations (for all birds).  As mentioned above, this probability distribution is specified implicitly via a randomized algorithm that returns an actual acceleration (again for all birds).  This randomized algorithm is the AMPC algorithm, which inherits its randomization from the randomized PSO procedure it deploys.  

When the controller computes an acceleration, it assumes that the attacker does {\em{not}} introduce any disturbances; i.e., the controller uses the following model:
\vspace*{-4mm}\begin{eqnarray}
 \xv_i(t + 1) &=& \xv_i(t) + \vv_i(t+1) \qquad \forall~i\,{\in}\,\{1,\ldots,B\}, \nonumber \\
 \vv_i(t + 1) &=& \vv_i(t) + \va_i(t), \label{eq:noattack} %\\[-6mm]
\end{eqnarray}
where $\va(t)$ is the only control variable. Note that the controller chooses its next action $\va(t)$ based on the current configuration $(\xv(t),\vv(t))$ of the flock using MPC. The current configuration may have been influenced by the disturbance $\vec{d}(t-1)$ introduced by the attacker in the previous time step.  Hence, the current state need not be the state predicted by the controller when performing MPC in step $t-1$. Moreover, depending on the severity of the attacker action $\vec{d}(t-1)$, the AMPC procedure dynamically adapts its behavior, i.e.\ the choice of horizon $h$, in order to enable the controller to pick the best control action $\vec{a}(t)$ in response.

\subsection{Attacker's Strategies}

We are interested in evaluating the resilience of our V-formation controller when it is threatened by an attacker that can remove a certain number of birds from the flock, or manipulate a certain number of birds by taking control of their actuators (modeled by the displacement term in equation~(\ref{eq:trans})).
We assume that the attack lasts for a limited amount of time, after which the controller attempts to bring the system back into the good set of states. When there is no attack, the system behavior is the one given by equation~(\ref{eq:noattack}).

Note that there can be many different criteria for evaluating the success of an attack,  %(see Remark~\ref{remark:criteria})
but in our experiments, the controller is declared the winner if it can bring the flock to V-formation.
We consider three attack strategies (but see the future work discussion in Section~\ref{sec:conclusion}), each of which defines a V-formation game.

\vspace*{-0.5mm}\paragraph{\bf Remove Birds Game.}
In an RBG, the attacker selects a subset of $R$ birds, where $R\,{\ll}\,B$, and removes them from the flock.  The removal of bird $i$ from the flock at time $t\,{=}\,0$ can be simulated in our framework by allowing the attacker to set the displacement $\vd_i(0)$ for bird $i$ to $\infty$.  We assume that the flock is in a V-formation at time $t\,{=}\,0$.  
Thus, the goal of the controller is to bring the flock back into a V-formation consisting of $B\,{-}\,R$ birds.
%he controller needs to find the best adjustments in velocity $a_i$ for all remaining birds $i \in N - R$ during its turn. %$i \in N \wedge i \notin R$.
%Essentially, this results in a single-move game for the adversary. 
In an RBG, the attacker plays only one move.
When picking birds, the attacker is able to decide which birds will have the greatest negative impact on the flock's fitness when removed from the flock. Apart from seeing if the controller can bring the flock back to a V-formation, we also analyze the time it takes the controller to do so. 
%return to a v-formation for $R \leq \lceil\log(N)\rceil$ and 

% \todo[inline]{SAS: I would only suggest that the size R of the subset of
% birds removed from the flock (of size N) be such that R << N.  O/w I am
% not sure how interesting this game is.  Jesse has simulation results for
% R=1 and N=7.  Also, we should consider this game with and without process
% noise (PN), as Jesse has shown that the resiliency of the flock to remain
% in a V is highly dependent on the magnitude of PN.  It does very well with
% no PN or small PN, but resilience seems to degrade with increasing PN.}
%
%\begin{theorem}
%For any birds picked by the attacker, where $\left\vert{N - R}\right\vert \geq 3$, the planner can find 
%accelerations for each remaining bird in $N$ that will finally lead to a state $s^{*}$ such that cost 
%$J(s^{*})\{\leqslant}\,\varphi$.
%\end{theorem}

\vspace*{-0.5mm}\paragraph{\bf Random Displacement Game.}
In an RDG, the attacker chooses the displacement vector for a fixed number $R$ of birds uniformly from the space $[0,M]\times[0,2\pi]$. This means that the magnitude of the displacement vector is picked from the interval $[0,M]$, and the direction of the displacement vector is picked from the interval $[0,2\pi]$. We vary $M$ in our experiments. The $R$ birds that are picked in different steps are not necessarily the same, as the attacker makes this choice uniformly at random at runtime as well.
%In our second game, each player has control over all birds in the flock. The flock starts in a V-formation. However, both players have different goals and strategies. While the controller wants to keep the flock in a V-formation, the adversarial player tries to disrupt the V. Both players use the same planning approach but the controller tries to minimize the fitness function while the adversary tries to maximize the fitness in each step.
%In our second game, the adversarial player introduces malicious birds into the flock. These birds are controlled by the other player and hence can perturb the flock. To do so, the adversary adds small amounts of noise to this bird to distract the flock and disturb the v-formation. If this alone is not successful, the adversary can use a greater amount of noise to achieve the goal. However, this allows the controller to identify the adversary and henceforth ignore the malicious bird. 
The game starts from an initial V-formation. The attacker is allowed a fixed number of moves, say $20$, after which the displacement vector is identically $0$ for all birds.  The controller, which has been running in parallel with the attacker, is then tasked with moving the flock back to a V-formation, if necessary.
%
\vspace*{-0.5mm}\paragraph{\bf{AMPC Game.}}
An AMPC game is similar to an RDG except that the attacker does not use a uniform distribution to determine the displacement vector. The attacker is advanced and calculates the displacement (that will be the worst for the controller) using the AMPC procedure. See Figure~\ref{fig:ampc}.  In detail, the attacker applies AMPC, but assumes the controller applies zero acceleration. Thus, the attacker uses the following model of the flock dynamics:
\vspace*{-1mm}\begin{eqnarray}
 \xv_i(t + 1) &=& \xv_i(t) + \vv_i(t+1) + \vd_i(t) \qquad \forall~i\,{\in}\,\{1,\ldots,B\}, \nonumber \\
 \vv_i(t + 1) &=& \vv_i(t). \label{eq:attack} %\\[-6mm]
\end{eqnarray}
Note that the attacker is still allowed to have $\vd_i(t)$ be nonzero for a small number $R$ of birds. However, it can choose which $R$ birds it picks in each step.  It uses the AMPC procedure to simultaneously pick the $R$ birds and their displacements.
%Being a fair game, both players have the same capabilities. This means the controller as well as the adversary are able to use receding horizons to try to predict the best moves for their individual birds.

%\begin{theorem}
%
%\end{theorem}

%\paragraph{\bf Game 3.}%: Interior Lines}
% In our third game the adversary has only access to a specific subset of the birds. One could consider the attacker to add a set of malicious birds $M$ to the existing flock $N$.  Additionally we assume the controller is able to detect the attacker and hence the adversarial player needs to wait for the opportune moment to perform the actual attack. This means, the adversarial player can disrupt the V-formation slightly but only has one single move to interrupt and perturb the V-formation permanently. 
% \todo[inline] {Lukas: some important questions: the ATTACKER-ARES only controls the malicious birds and the CTL-ARES only the 'good' birds. however, does the CTL-ARES consider the malicious birds in its planning as 'good' birds? same for the ATTACKER-ARES. To me it would make sense, that the ATTACKER-ARES knows which ones are malicious birds and which ones are 'good' birds, but the CTL-ARES does not. So the CTL-ARES would consider ALL birds ($M \cup N$) but only controls the 'good' ones ($N$) -- i hope this makes any sense.}
%The third game is very similar to the second. However, when performing the final move, the attacker can decide whether it is more beneficial to introduce noise with a great magnitude to the flock or simply remove a specific number of birds from the flock. Again, we consider this a fair game where both players are able to use receding horizons do identify potential moves. Furthermore, we allow the adversary to remove up to $\log(N)$ birds from the flock.
%\subsection{Implementation: the Game is on}
%\label{sec:implementation}
%
%\todo[inline]{The following section would be the new implementation of our algorithm that deals with stochastic MDP and two-player games.}
%
% For this work, we extended the original \emph{deterministic Markov Decision Process} presented by Lukina et al.~\cite{lukina2016arxiv} to a \emph{classical MDP}~\cite{russellnorvig} by adding noise to the transition relation of the MDP. By doing so, we improved the original model and made it more realistic.
%
%We added and analyzed two different types of noises, processing noise ($\pnoise$) and actuator noise ($\anoise$). $\pnoise$ is applied to the position of each bird in our flock and changes the transition relation as follows
%\vspace*{-1mm}\begin{eqnarray*}
%\label{eq:pnoise_model}
% \xv_i(t + 1) &=& \xv_i(t) + \vv_i(t+1) + \pnoise %\label{eq:x_anoise},\\
% \vv_i(t + 1) &=& \vv_i(t) + \va_i(t) \label{eq:v_anoise},\\[-6mm]
%\end{eqnarray*}
%where $\pnoise \sim \mathcal{N}(0, \sigma^2)$. Here, $\sigma$ 

%In contrast, actuator noise is added to the acceleration action of the transition relation.
%\vspace*{-1mm}\begin{eqnarray*}
%\label{eq:model}
 %\xv_i(t + 1) &=& \xv_i(t) + \vv_i(t+1)\label{eq:x_anoise},\\
 %\vv_i(t + 1) &=& \vv_i(t) + \va_i(t) + \anoise\label{eq:v_anoise},\\[-6mm]
%\end{eqnarray*}

%\noindent where $\anoise \sim \mathcal{N}(0, \sigma^2)$. For our experiments we tried different $\sigma$, i.e. $\sigma = 0.05, 0.1, 0.2, 0.25$ and $0.3$.

%\begin{remark}\label{remark:criteria}
%Even though we use reaching V-formation as our success criterion (for the controller), we could have also used other criteria to decide if the attacker has been successful. For example, one could have used following criteria.
%
%\begin{itemize}
%\item \emph{Energy attack} is considered successful when a flock is not traveling in a V-formation for a certain amount of time. 
%
%\vspace*{1mm}\item \emph{Collisions} occur when two birds are in dangerous proximity from each other. This may happen through spoofing of existing birds or adversarial birds deliberately trying to lead to collisions with the others.
%
%\vspace*{1mm}\item \emph{Heading change} brings success, when the entire flock is diverged from its original direction (mission target) by a certain degree. 
%\end{itemize}
%\end{remark}

\begin{theorem}[AMPC resilience in a C-A game]
\label{thm:resilience}
Given a controller-attacker game, there is a finite maximum horizon $h_{\mathit{max}}$ and a finite maximum number of game-execution steps $m$ such that AMPC controller will win the controller-attacker game in $m$ steps with probability one.
\end{theorem}

\begin{proof}
Since the flock MDP (defined by Equation~6) is controllable, the PSO algorithm we use is fair, and the attack has a bounded duration, the proof of the theorem follows from Theorem~\ref{thm:ampc}. 
\end{proof}

\begin{remark}
While Theorem~\ref{thm:resilience} states that the controller is expected to win with probability one, we expect winning probability to be possibly lower than one in many cases because: (1)~the maximum horizon $h_{\mathit{max}}$ is fixed in advance, and so is (2) the maximum number of execution steps $m$; (3) the underlying PSO algorithm is also run with bounded number of particles and time.
\end{remark}


\label{sec:experimental}

\section{Experiments}

\label{sec:experiments}
In this section we compare the tiered vector to some widely used C++ standard library containers. 
We also compare different variants of the tiered vector. 
We consider how the different representations of the data
structure listed in Section~\ref{sec:implementation}, 
and also how the height of tree and the capacity of the leaves affects the running time.
The following describes the test setup:

\subparagraph{Environment}

All experiments have been performed on a Intel Core i7-4770 CPU @ 3.40GHz with
32 GB RAM. The code has been compiled with GNU GCC version 5.4.0 with flags
``-O3''. The reported times are an average over 10 test runs.
 
 \subparagraph{Procedure}
%have been added to the data structure in
 
In all tests $10^8$ 32-bit integers 
are inserted in the data structure as a preliminary step
to simulate that it has already been
used\footnote{In order to minimize the overall running time of the experiments,
the elements were not added randomly, but we show this does not give our data
structure any benefits}.
For all the access and successor operations $10^9$ elements have been accessed
and the time reported is the average time per element.
For range access, 10.000 consecutive elements are accessed.
For insertion/deletion $10^6$ elements
have been (semi-)randomly\footnote{In order to not impact timing, a simple
access pattern has been used instead of a normal pseudo-random generator.}
added/deleted, though in the case of ``vector'' only 10.000 elements were
inserted/deleted to make the experiments terminate in reasonable time. 

\subsection{Comparison to C++ STL Data Structures}

In the following we have compared our best performing tiered vector (see the next sections) to the vector and
the multiset class from the C++ standard library.
The vector data structure directly supports the
operations of a dynamic array. The multiset class is implemented as a red-black
tree and is therefore interesting to compare with our data structure.
Unfortunately, multiset does not directly support the operations of a dynamic
array (in particular it has no notion of positions of elements). To simulate an
access operation we instead find the successor of an element in the multiset.
This requires a root-to-leaf traversal of the red-black tree, just as an access
operation in a dynamic array implemented as a red-black tree would. Insertion
is simulated as an insertion into the multiset, which again requires the same
computations as a dynamic array implemented as a red-black tree would.

Besides the random access, range access and insertion,
we have also tested the operations \textit{data dependent access},
insertion in the end, deletion, and \textit{successor} queries. In the
\textit{data dependent access} tests, the next index to lookup depends on the values of the prior
lookups. This ensures that the CPU cannot successfully pipeline
consecutive lookups, but must perform them in sequence. We test insertion in the end, since
this is a very common use case. Deletion is performed by deleting elements at
random positions. The $successor$ queries returns the successor of an element
and is not actually part of the
dynamic array problem, but is included since it is a commonly used operation on
a multiset in C++. It is simply implemented as a binary search over the elements in
both the vector and tiered vector tests where the elements are now inserted in sorted order. 

The results are summarized in Table~\ref{tab:test_comp} which shows that the vector performs slightly better than the tiered vector on all access and successor tests. As expected from the $\Theta(n)$ running time, it performs extremely poor on random insertion and deletion. For insertion in the end of the sequence, vector is also slightly faster than the tiered vector. The interesting part is that even though the tiered vector requires several extra memory lookups and computations, we have managed to get the running time down to less than the double of the vector for access, even less for data dependent access and only a few percent slowdown for range access. As discussed earlier,
this is most likely because the entire tree structure (without the elements)
fits within the CPU cache, and because the computations required has been minimized.

Comparing our tiered vector to multiset, we would expect access operations to be
faster since they run in $O(1)$ time compared to $O(\log n)$. On the other
hand, we would expect insertion/deletion to be significantly slower since it
runs in $O(n^{1/l})$ time compared to $O(\log n)$ (where $l = 4$ in these tests). We
see our expectations hold for the access operations where the tiered vector is faster by more than an order of magnitude.
In random insertions however,  the tiered vector is only $8\%$ slower -- even when operating on 100.000.000 elements. Both the tiered
vector and set requires $O(\log n)$ time for the successor operation. In our
experiments the tiered vector is 3 times faster for the successor operation.

Finally, we see that the memory usage of vector and tiered vector is almost identical.
This is expected since in both cases the space usage is dominated by the space taken by the actual elements.
The multiset uses more than 10 times as much space, so this is also a considerable drawback of the red-black tree behind this structure. 

To sum up, the tiered vectors performs better than multiset on all tests
but insertion, where it performs only slightly worse.

%\caption{Figures (a) through (e) show the performance of \textit{Tiered Arrays} (\protect\purple) compared
%to the \textit{set} (\protect\green) and \textit{vector} (\protect\blue) data structures from the C++ standard library.} \label{fig:animals}
\begin{table}
	\centering
	\begin{tabular}{|l|r|r|r|r|r|}
		\hline
		& \multicolumn{1}{l|}{\textit{tiered vector}} & \multicolumn{1}{l|}{\textit{set}} & \multicolumn{1}{l|}{\textit{set / tiered}} & \multicolumn{1}{l|}{\textit{vector}} & \multicolumn{1}{l|}{\textit{vector / tiered}} \\ \hline
		access     & $34.07$ ns                                  & $1432.05$ ns                      & 42.03                                      & $21.63$ ns                           & 0.63                                          \\ \hline
		dd-access    & $99.09$ ns                                  & $1436.67$ ns                      & 14.50                                      & $79.37$ ns                           & 0.80                                          \\ \hline
		range access   & $0.24$ ns                                   & $13.02$ ns                        & 53.53                                      & $0.23$ ns                            & 0.93                                          \\ \hline
		insert   & $1.79$ $\mu$s                               & $1.65$ $\mu$s                     & 0.92                                       & $21675.49$ $\mu$s                     & 12082.33                                      \\ \hline
		insertion in end     & $7.28$ ns                               & $242.90$ ns                     & 33.38                                       & $2.93$ ns                     & 0.40                                      \\ \hline
		successor & $0.55$ $\mu$s                               & $1.53$ $\mu$s                     & 2.75                                       & $0.36$ $\mu$s                        & 0.65                                          \\ \hline
		delete     & $1.92$ $\mu$s                               & $1.78$ $\mu$s                     & 0.93                                       & $21295.25$ $\mu$s                     & 11070.04                                      \\ \hline
		memory     & $408$ MB                               & $4802$ MB                     & 11.77                                       & $405$ MB                    & 0.99                                      \\ \hline
	\end{tabular}
	\caption{The table summarizes the performance of the implicit tiered vector
		compared to the performance of multiset and vector from the C++ standard library.\
		dd-access refers to data dependent access.}
\label{tab:test_comp}
\end{table}


\definecolor{cpurple}{RGB}{131,24,197}
\definecolor{cgreen}{RGB}{70,156,118}
\definecolor{cblue}{RGB}{11,178,228}
\definecolor{cdblue}{RGB}{11,112,173}
\definecolor{corange}{RGB}{219,162,55}
\definecolor{cyellow}{RGB}{238,228,98}
\definecolor{cred}{RGB}{110,55,38}
\newcommand{\purple}{\raisebox{2pt}{\tikz{\draw[cpurple,solid,line width=1.9pt](0,0) -- (3mm,0);}}}
\newcommand{\green}{\raisebox{2pt}{\tikz{\draw[cgreen,solid,line width=1.9pt](0,0) -- (3mm,0);}}}
\newcommand{\blue}{\raisebox{2pt}{\tikz{\draw[cblue,solid,line width=1.9pt](0,0) -- (3mm,0);}}}
\newcommand{\dblue}{\raisebox{2pt}{\tikz{\draw[cdblue,solid,line width=1.9pt](0,0) -- (3mm,0);}}}
\newcommand{\orange}{\raisebox{2pt}{\tikz{\draw[corange,solid,line width=1.9pt](0,0) -- (3mm,0);}}}
\newcommand{\yellow}{\raisebox{2pt}{\tikz{\draw[cyellow,solid,line width=1.9pt](0,0) -- (3mm,0);}}}
\newcommand{\red}{\raisebox{2pt}{\tikz{\draw[cred,solid,line width=1.9pt](0,0) -- (3mm,0);}}}


\begin{figure}[ht]
	\centering
	\begin{subfigure}[b]{0.3\textwidth}
		\includegraphics[width=\textwidth]{layout_test_get}
		\caption{\textit{access}}
	\end{subfigure}
	\begin{subfigure}[b]{0.3\textwidth}
		\includegraphics[width=\textwidth]{layout_test_random}
		\caption{\textit{insert}}
	\end{subfigure}
        \caption{Figures (a) and (b) show the performance of the
            \textit{original} (\protect\purple), \textit{optimized original}
            (\protect\green), \textit{lazy} (\protect\blue) \textit{packed
            lazy} (\protect\orange),
            \textit{implicit} (\protect\yellow)
            and \textit{packed implicit} (\protect\dblue) layouts.}
\label{fig:test_representation}
\end{figure}
\subsection{Tiered Vector Variants}

In this test we compare the performance
of the implementations listed in Section~\ref{sec:implementation} to that 
or the original data structure as described in~\ref{thm:pointer}.

%\paragraph{Optimized Original}
\subparagraph*{Optimized Original}
By co-locating the child offset and child pointer, the two memory lookups are at
adjacent memory locations. Due to the cache lines in modern processors,
the second memory lookup will then often be answered directly by the fast
L1-cache.
As can be seen on Figure~\ref{fig:test_representation}, this small change in the memory layout results in a significant improvement in performance for both access and insertion. In the latter case, the running time is more than halved.

%\paragraph{Lazy and Packed Lazy}
\subparagraph*{Lazy and Packed Lazy}

Figure~\ref{fig:test_representation} shows
how the fewer memory probes required by the
\textit{lazy} implementation in comparison to the \text{original}
and \text{optimized original} results in better performance.
Packing the offset and pointer in the leaves results in even better performance
for both access and insertion even though it requires a few extra instructions
to do the actual packing and unpacking.

%\paragraph{Implicit}
\subparagraph*{Implicit}
From Figure~\ref{fig:test_representation}, we see the implicit
data structure is the fastest.
This is as expected because it requires fewer
memory accesses than the other structures except
for the packed lazy which instead has a slight
computational overhead due to the packing and unpacking.

As shown in Theorem~\ref{thm:implicit} the implicit data structure has a
bigger memory overhead than the lazy data structure.
Therefore the packed lazy representation might be beneficial in some
settings.

%\paragraph{Packed Implicit}
\subparagraph*{Packed Implicit}

Packing the offsets array could lead to 
better cache performance due to the smaller memory footprint and therefore
yield better overall performance.
As can be seen on Figure~\ref{fig:test_representation},
the smaller memory footprint
did not improve the performance in practice.
The simple reason for this,
is that the strategy we used for packing the offsets required
extra computation. This clearly dominated the possible gain from the
hypothesized better cache performance. We tried a few strategies to minimize
the extra computations needed at the expense of slightly worse memory usage,
but none of these led to better results than when not packing the offsets at
all.

\subsection{Width Experiments}

\begin{figure}
	\centering
	\begin{subfigure}[b]{0.3\textwidth}
		\includegraphics[width=\textwidth]{width_test_get}
		\caption{\textit{access}}
	\end{subfigure}
	\begin{subfigure}[b]{0.3\textwidth}
		\includegraphics[width=\textwidth]{width_test_sum}
		\caption{\textit{range access}}
	\end{subfigure}
	\begin{subfigure}[b]{0.3\textwidth}
		\includegraphics[width=\textwidth]{width_test_random}
		\caption{\textit{insert}}
	\end{subfigure}
	\caption{Figures (a), (b) and (c) show the performance of the \textit{implicit} (\protect\purple) and
		the \textit{optimized original} tiered vector (\protect\green) for different tree widths.}
\label{fig:test_width}
\end{figure}

This experiment was performed to determine the best capacity ratio between the leaf nodes and the internal nodes.
The six different width configurations we have tested are: 32-32-32-4096, 32-32-64-2048, 32-64-64-1024, 64-64-64-512, 64-64-128-256, and 64-128-128-128.
All configurations have a constant height 4 and a capacity of approximately 130 mio.

We expect the performance of access operations to remain unchanged, since the
amount of work required only depends on the height of the tree,
and not the widths. We expect range access to perform better when the leaf size
is increased, since more elements will be located in consecutive memory
locations. For $insertion$ there is not a clearly expected behavior as the time
used to physically move elements in a leaf will increase with leaf size, but
then less operations on the internal nodes of the tree has to be performed.

On Figure~\ref{fig:test_width} we see access times are actually decreasing
slightly when leaves get bigger. This was not expected, but is most likely
due to small changes in the memory layout that results in slightly better cache
performance. The same is the case for range access, but this was expected. For
insertion, we see there is a tipping point. For our particular instance, the
best performance is achieved when the leaves have size 512.

%Based on this, we have performed the remaining tests with the 64-64-64-512 configuration (unless otherwise specified).

\subsection{Height Experiments}

\begin{figure}
	\centering
	\begin{subfigure}[b]{0.3\textwidth}
		\includegraphics[width=\textwidth]{height_get}
		\caption{\textit{access(i)}}
	\end{subfigure}
	\begin{subfigure}[b]{0.3\textwidth}
		\includegraphics[width=\textwidth]{height_sum}
		\caption{\textit{access(i, m)}}
	\end{subfigure}
	\begin{subfigure}[b]{0.3\textwidth}
		\includegraphics[width=\textwidth]{height_random}
		\caption{\textit{insert}}
	\end{subfigure}
	\caption{Figures (a),(b) and (c) show the performance of the \textit{implicit} (\protect\purple) and
		the \textit{optimized original} tiered vector (\protect\green) for different tree heights.}
\label{fig:test_height}
\end{figure}

In these tests we have studied how different heights affect the performance of
access and insertion operations. We have tested the configurations 8196-16384,
512-512-512, 64-64-64-512, 16-16-32-32-512, 8-8-16-16-16-512. All resulting in
the same capacity, but with heights in the range 2-6.

We expect the access operations to perform better for lower trees, since
the number of operations that must be performed is linear in the height. On the
other hand we expect insertion to perform significantly better with higher
trees, since its running time is $O(n^{1/l})$ where $l$ is the height plus one. 

On Figure~\ref{fig:test_height} we see the results follow our expectations. However, the access operations only perform slightly worse on higher trees.
This is most likely because all internal nodes fit within the L3-cache. Therefore the running time is dominated by the lookup of the element itself.
(It is highly unlikely that the element requested by an access 
to a random position would be among the small fraction of elements that
fit in the L3-cache).

Regarding insertion, we see significant improvements up until a height of 4. After that, increasing the height does not change the running time noticeably. This is most likely due to the hidden constant in $O(n^{1/l})$ increasing rapidly with the height.



\subsection{Configuration Experiments}

\begin{figure}
    \centering
    \begin{subfigure}[b]{0.3\textwidth}
        \includegraphics[width=\textwidth]{small_get}
        \caption{\textit{access}}
    \end{subfigure}
    \begin{subfigure}[b]{0.3\textwidth}
        \includegraphics[width=\textwidth]{small_sum}
        \caption{\textit{range access}}
    \end{subfigure}
    \begin{subfigure}[b]{0.3\textwidth}
        \includegraphics[width=\textwidth]{small_random}
        \caption{\textit{insert(i,x)}}
    \end{subfigure}
    \caption{Figures (a) and (b) show the performance of the
    \textit{base} (\protect\purple),
    \textit{rotated} (\protect\green), 
    \textit{non-aligned sizes} (\protect\blue),
    \textit{non-templated} (\protect\orange)
    layouts.}
\label{fig:test_minor}
\end{figure}

In these experiments, we test a few hypotheses about how different changes
impact the running time. The results are shown on
Figure~\ref{fig:test_minor}, the leftmost result (base) is
the implicit 64-64-64-512 configuration of the tiered vector 
to which we compare our hypotheses.
%our final and best

\textit{Rotated}: 
As already mentioned, the insertions performed as a
preliminary step to the tests are not done at random positions.
This means that all offsets are zero when our real operations
start. The purpose of this test is the ensure that
there are no significant performance gains in starting
from such a configuration which could otherwise
lead to misleading results.
To this end, we have randomized all
offsets (in a way such that the data structure is still valid, but the
order of elements change) after doing the preliminary insertions
but before timing the operations. As can be seen on
Figure~\ref{fig:test_minor}, the difference between this and the normal
procedure is insignificant, thus we find our approach gives a fair picture.


\textit{Non-Aligned Sizes}: In all our previous tests, we have ensured all
nodes had an out-degree that was a power of 2. This was chosen in order to let the
compiler simplify some calculations, i.e.\ replacing multiplication/division
instructions by shift/and instructions. As Figure~\ref{fig:test_minor} shows,
using sizes that are not powers of 2 results in significantly worse performance.
Besides showing that powers of 2 should always be used, this also indicates that not only
the number of memory accesses during an operation is critical for our
performance, but also the amount of computation we make.

\textit{Non-Templated}
The non-templated results 
in Figure~\ref{fig:test_representation} the
show that the change to templated recursion
has had a major impact on the running time. It should be noted that some
improvements have not been implemented in the non-templated version,
but it gives a good indication that this has been quite useful.


\section{Conclusion}


\begin{comment}
\begin{figure}
\includegraphics[width=\linewidth]{figs/beyond_tss_lesion.pdf}
\caption[]{End-to-End runtime lesion study of the entire MNIST dataset and the FMA featurized music dataset. Each of DROP's contributions provides a runtime improvement.}
\label{fig:beyond_lesion}
\end{figure}
\end{comment}



\section{Conclusion}
\label{sec:conclusion}

Advanced data analytics techniques must scale to rising data volumes. 
DR techniques offer a powerful toolkit when processing these datasets, with PCA frequently outperforming popular techniques in exchange for high computational cost. 
In response, we propose DROP, a new dimensionality reduction optimizer. 
DROP combines progressive sampling, progress estimation, and online aggregation to identify high quality low dimensional bases via PCA without processing the entire dataset by balancing the runtime of downstream tasks and achieved dimensionality. 
Thus, DROP provides a first step in bridging the gap between quality and efficiency in end-to-end DR for downstream \red{analytics}. 

%We revisit canonical operators for time series dimensionality reduction and the measurement study of~\cite{keogh-study}, and show that PCA is more effective than popular alternatives in the data mining literature often by a margin of over $2\times$ on average on gold-standard time series benchmark data sets with respect to output data dimension. More surprisingly, we empirically demonstrate that a small number of samples are sufficient to accurately characterize directions of maximum variance and obtain a high-quality low-dimensional transformation.




\bibliography{references}

\end{document}
