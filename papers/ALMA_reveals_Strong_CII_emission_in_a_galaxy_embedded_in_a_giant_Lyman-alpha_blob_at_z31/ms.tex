%%
%% Beginning of file 'sample.tex'
%%
%% Modified 2005 December 5
%%
%% This is a sample manuscript marked up using the
%% AASTeX v5.x LaTeX 2e macros.

%% The first piece of markup in an AASTeX v5.x document
%% is the \documentclass command. LaTeX will ignore
%% any data that comes before this command.

%% The command below calls the preprint style
%% which will produce a one-column, single-spaced document.
%% Examples of commands for other substyles follow. Use
%% whichever is most appropriate for your purposes.
%%
%%\documentclass[12pt,preprint]{aastex}

%% manuscript produces a one-column, double-spaced document:

%\documentclass[manuscript]{aastex}
\documentclass[apj]{emulateapj}

%\usepackage{aas_macros}
\usepackage{natbib}
\usepackage{color}
\usepackage{mediabb}
\usepackage{amsmath,amssymb}
%\usepackage{lscape}
%\usepackage[dvipdfmx]{graphicx}
%\citestyle{apj}
%\usepackage{lscape}
%% preprint2 produces a double-column, single-spaced document:

%% \documentclass[preprint2]{aastex}

%% Sometimes a paper's abstract is too long to fit on the
%% title page in preparationrint2 mode. When that is the case,
%% use the longabstract style option.

%% \documentclass[preprint2,longabstract]{aastex}

%% If you want to create your own macros, you can do so
%% using \newcommand. Your macros should appear before
%% the \begin{document} command.
%%
%% If you are submitting to a journal that translates manuscripts
%% into SGML, you need to follow certain guidelines when preparing
%% your macros. See the AASTeX v5.x Author Guide
%% for information.

\newcommand{\vdag}{(v)^\dagger}
\newcommand{\myemail}{hideki.umehata@ouj.ac.jp}

%% You can insert a short comment on the title page using the command below.

\slugcomment{draft November 26 2016}

%% If you wish, you may supply running head information, although
%% this information may be modified by the editorial offices.
%% The left head contains a list of authors,
%% usually a maximum of three (otherwise use et al.).  The right
%% head is a modified title of up to roughly 44 characters.
%% Running heads will not print in the manuscript style.

\shorttitle{[\textsc{Cii}] from LAB1}
\shortauthors{Umehata et al.}

%% This is the end of the preamble.  Indicate the beginning of the
%% paper itself with \begin{document}.

\begin{document}

%% LaTeX will automatically break titles if they run longer than
%% one line. However, you may use \\ to force a line break if
%% you desire.

\title{
ALMA reveals Strong [CII]  emission in a galaxy embedded in a giant Ly$\alpha$ blob at $z=3.1$
}

%% Use \author, \affil, and the \and command to format
%% author and affiliation information.
%% Note that \email has replaced the old \authoremail command
%% from AASTeX v4.0. You can use \email to mark an email address
%% anywhere in the paper, not just in the front matter.
%% As in the title, use \\ to force line breaks.

\author{
Hideki Umehata\altaffilmark{1,2}, Yuichi Matsuda\altaffilmark{3,4}, Yoichi Tamura\altaffilmark{2}, Kotaro Kohno\altaffilmark{2,5}, Ian Smail\altaffilmark{6}, R.J. Ivison\altaffilmark{7,8}, Charles~C.~Steidel\altaffilmark{9}, Scott C. Chapman\altaffilmark{10}, James E. Geach\altaffilmark{11}, Matthew Hayes\altaffilmark{12}, Tohru Nagao\altaffilmark{13}, Yiping Ao\altaffilmark{3}, Ryohei Kawabe\altaffilmark{3,4,15}, Min S. Yun\altaffilmark{14}, Bunyo Hatsukade\altaffilmark{3}, Mariko Kubo\altaffilmark{3}, Yuta Kato\altaffilmark{3,15}, Tomoki Saito\altaffilmark{16}, Soh Ikarashi\altaffilmark{17}, Kouichiro~Nakanishi\altaffilmark{3,4}, Minju Lee\altaffilmark{3,15}, Takuma Izumi\altaffilmark{2}, Masao~Mori\altaffilmark{18}, Masami Ouchi\altaffilmark{19}
}
\affil{
$^1$ The Open University of Japan, 2-11 Wakaba, Mihama-ku, Chiba 261-8586, Japan; \textcolor{blue}{{\myemail}}
\\
$^2$ Institute of Astronomy, School of Science, The University of Tokyo, 2-21-1 Osawa, Mitaka, Tokyo 181-0015, Japan
\\
$^3$ National Astronomical Observatory of Japan, 2-21-1 Osawa, Mitaka, Tokyo 181-8588, Japan
\\
$^4$ Department of Astronomy, School of Science, SOKENDAI (The Graduate University for Advanced Studies), Osawa, Mitaka,
Tokyo 181-8588, Japan
\\
$^5$ Research Center for the Early Universe, The University of Tokyo, 7-3-1 Hongo, Bunkyo, Tokyo 113-0033
\\
$^6$ Centre for Extragalactic Astronomy, Department of Physics, Durham University, South Road, Durham, DH1 3LE, UK
\\
$^7$ European Southern Observatory, Karl-Schwarzschild-Str. 2, D-85748 Garching, Germany
\\
$^8$ Institute for Astronomy, University of Edinburgh, Royal Observatory, Blackford Hill, Edinburgh EH9 3HJ, UK
\\
$^{9}$ California Institute of Technology, MS 249-17, Pasadena, CA 91125, USA
\\
$^{10}$ Department of Physics and Atmospheric Science, Dalhousie University, Halifax, NS B3H
4R2, Canada
\\
$^{11}$ Centre for Astrophysics Research, Science \& Technology Research Institute, University of
Hertfordshire, Hatfield AL10 9AB, UK
\\
$^{12}$ Department of Astronomy, Oskar Klein Centre, Stockholm University, AlbaNova University
Centre, SE-106 91 Stockholm, Sweden
\\
$^{13}$ Research Center for Space and Cosmic Evolution, Ehime University, 2-5 Bunkyo-cho,
Matsuyama, Ehime 790-8577
\\
$^{14}$ Department of Astronomy, University of Massachusetts, Amherst, Massachusetts 01003
\\
$^{15}$ Department of Astronomy, Graduate school of Science, The University of Tokyo, 7-3-1
Hongo, Bunkyo-ku, Tokyo 133-0033
\\
$^{16}$ Nishi-Harima Astronomical Observatory, Centre for Astronomy,
University of Hyogo, 407-2 Nichigaichi, Sayo-cho, Sayo,
Hyogo 679-5313, Japan
\\
$^{17}$ Kapteyn Astronomical Institute, University of Groningen, P.O. Box 800, 9700AV Groningen, The Netherlands
\\
$^{18}$ Center for Computational Physics, University of Tsukuba, 1-1-1 Tennodai, Tsukuba, Ibaraki
305-8577
\\
$^{19}$ Institute for Cosmic Ray Research, University of Tokyo, 5-1-5 Kashiwa-no-Ha, Kashiwa City,
Chiba 277-8582
}

%\author{C. D. Biemesderfer\altaffilmark{4,5}}
%\affil{National Optical Astronomy Observatories, Tucson, AZ 85719}
%\email{aastex-help@aas.org}
%
%\and

%\author{R. J. Hanisch\altaffilmark{5}}
%\affil{Space Telescope Science Institute, Baltimore, MD 21218}

%% Notice that each of these authors has alternate affiliations, which
%% are identified by the \altaffilmark after each name.  Specify alternate
%% affiliation information with \altaffiltext, with one command per each
%% affiliation.

%% Mark off your abstract in the ``abstract'' environment. In the manuscript
%% style, abstract will output a Received/Accepted line after the
%% title and affiliation information. No date will appear since the author
%% does not have this information. The dates will be filled in by the
%% editorial office after submission.

\begin{abstract}
We report the result from observations conducted with the Atacama Large Millimeter/submillimeter Array (ALMA) to detect [\textsc{Cii}] 158~$\mu$m fine structure line emission from galaxies embedded in one of the most spectacular Ly$\alpha$ blobs (LABs) at $z=3.1$, SSA22-LAB1.
Of three dusty star-forming galaxies previously discovered by ALMA 860~$\mu$m dust continuum survey toward SSA22-LAB1, we detected the [\textsc{Cii}] line from one, LAB1-ALMA3 at $z$=3.0993$\pm$0.0004.
No line emission was detected, associated with the other ALMA continuum sources or from three 
rest-frame UV/optical selected $z_{\rm spec}\simeq3.1$ galaxies within the field of view.
For LAB1-ALMA3, we find relatively bright [\textsc{Cii}] emission compared to the infrared luminosity ($L_\textsc{[Cii]}$/$L_{\rm IR}\approx0.01$) and 
an extremely high [\textsc{Cii}] 158~$\mu$m and [\textsc{Nii}] 205~$\mu$m emission line ratio
($L_\textsc{[Cii]}$/$L_\textsc{[Nii]}>$55).
The relatively strong [\textsc{Cii}] emission may be caused by abundant photodissociation regions and sub-solar metallicity, or by
 shock heating.
The origin of the unusually strong [\textsc{Cii}] emission could be causally related to the location within the giant LAB, although the relationship between extended Ly$\alpha$ emission and ISM conditions of associated galaxies is yet to be understand.
 \end{abstract}

%% Keywords should appear after the \end{abstract} command. The uncommented
%% example has been keyed in ApJ style. See the instructions to authors
%% for the journal to which you are submitting your paper to determine
%% what keyword punctuation is appropriate.

\keywords{catalogs -- galaxies: high-redshift -- galaxies: starburst}

%% From the front matter, we move on to the body of the paper.
%% In the first two sections, notice the use of the natbib \citep
%% and \citet commands to identify citations.  The citations are
%% tied to the reference list via symbolic KEYs. The KEY corresponds
%% to the KEY in the \bibitem in the reference list below. We have
%% chosen the first three characters of the first author's name plus
%% the last two numeral of the year of publication as our KEY for
%% each reference.


%% Authors who wish to have the most important objects in their paper
%% linked in the electronic edition to a data center may do so by tagging
%% their objects with \objectname{} or \object{}.  Each macro takes the
%% object name as its required argument. The optional, square-bracket 
%% argument should be used in cases where the data center identification
%% differs from what is to be printed in the paper.  The text appearing 
%% in curly braces is what will appear in print in the published paper. 
%% If the object name is recognized by the data centers, it will be linked
%% in the electronic edition to the object data available at the data centers  
%%
%% Note that for sources with brackets in their names, e.g. [WEG2004] 14h-090,
%% the brackets must be escaped with backslashes when used in the first
%% square-bracket argument, for instance, \object[\[WEG2004\] 14h-090]{90}).
%%  Otherwise, LaTeX will issue an error. 

\section{Introduction}

\begin{figure*}
\epsscale{1.15}
\plotone{f1.eps}
\caption{
Images of SSA22-LAB1. Each field is $20^{\prime\prime}\times20^{\prime\prime}$ in size. The field of view of the ALMA band~8 observation is shown in each figure.
({\it left}) A pseudo color image with Subaru/Suprime-Cam $B$-, $NB497$-, and $V$-band where the strong Ly$\alpha$ emission falls in the green channel (\citealt{2004AJ....128..569M}). 
({\it middle}) {\it HST} STIS optical image as a finding chart. Contours show Ly$\alpha$ emission at levels of 4, 8, and 12~$\times10^{-18}$~erg~s$^{-1}$~cm$^{-2}$~arcsec$^{-2}$ (\citealt{2004AJ....128..569M}). We show the positions of three ALMA sources (ALMA1, ALMA2, and ALMA3; \citealt{2016ApJ...832...37G}; Y. Matsuda et al. in preparation) and other $z_{\rm spec}\approx3.1$ galaxies: one LBG (C11; \citealt{2003ApJ...592..728S}), and one {\it K}-selected galaxy (K1; \citealt{2015ApJ...799...38K}).
One faint [\textsc{Oiii}] emitter at $z=3.0968$ (S1; \citealt{2016ApJ...832...37G}) is also shown.
({\it right}) The non-primary-beam-corrected ALMA image at 860~$\mu$m (Y. Matsuda et al. in preparation).
%Three ALMA sources has been found toward LAB1 as labelled in the middle panel. 
}
\label{f1}
\end{figure*}

Investigating the physical and chemical properties of the interstellar medium (ISM) of dusty star-forming galaxies and/or high-redshift galaxies has been difficult, as typical UV/optical nebular lines are not useful due to heavy dust extinction and/or the lines are not accessible with conventional ground-based instruments.
Recently, the Atacama Large Millimeter/submillimeter Array (ALMA) has opened a new window, allowing us to exploit fine structure lines at rest-frame far-infrared (FIR) wavelengths to diagnose the ISM properties for these galaxy populations (e.g., \citealt{2012A&A...542L..34N}; \citealt{2014ApJ...782L..17D}; \citealt{2016Sci...352.1559I}).
%\citealt{2012MNRAS.427.1066S}
The [\textsc{Cii}]~158~$\mu$m ($^2P_{3/2}\rightarrow^2P_{1/2}$) is known to be the dominant coolant of the ISM and one of the brightest lines from star-forming galaxies in the FIR (e.g., \citealt{1996A&A...308..723I}).
While the [\textsc{Cii}] emission arises primarily from dense photodissociation regions (PDRs),  it is also observed in various regions/environments, including ionized regions, cool, diffuse interstellar gas, and shocked gas (e.g., \citealt{1991ApJ...373..423S}; \citealt{1993ApJ...407..579M}; \citealt{2011A&A...526A.149N}; \citealt{2013ApJ...777...66A}). 
%\textcolor{blue}{In addition} shocks can also contribute to the [\textsc{Cii}] emission (\citealt{2013ApJ...777...66A}).

% [CII] exploration for LABs Erb+11?
In order to characterize the [\textsc{Cii}] emission and investigate the nature of the ISM in star-forming galaxies at high redshift,
Ly$\alpha$ blobs (LABs) are a useful laboratory.
LABs are extended gaseous nebulae, preferentially found in regions of galaxy overdensities in the distant universe (e.g., \citealt{2000ApJ...532..170S}; \citealt{2004AJ....128..569M}; \citealt{2009ApJ...693.1579Y}).
A large number of LABs are associated with star-forming galaxies such as submillimeter galaxies (SMGs; e.g., \citealt{2005MNRAS.363.1398G}, \citeyear{2014ApJ...793...22G}; \citealt{2015ApJ...815L...8U}, \citeyear{2016arXiv161109857U}), distant red galaxies (DRGs; e.g., \citealt{2011ApJ...740L..31E}; \citealt{2012ApJ...750..116U}; \citealt{2013ApJ...778..170K}), and Lyman break galaxies (LBGs; e.g., \citealt{2004AJ....128..569M}).
Thus LABs are likely to be the sites of ongoing massive galaxy formation and assembly, and the extended gaseous structures around them are believed to be observational signs of large-scale gas flows (inflow/outflow) and their interactions as well as photoionization (e.g., \citealt{2000ApJ...532L..13T}; \citealt{2006Natur.440..644M}; \citealt{2009MNRAS.400.1109D}).
%Since it is hard to determine precise redshifts of dust-obscured galaxies in the UV/optical regime, FIR lines such as the [\textsc{Cii}]~158~$\mu$m are important to determine redshifts and diagnose the origin of LABs and their energetics.
%
% The most spectacular one, LAB1
SSA22-LAB1 (hereafter LAB1, \citealt{2000ApJ...532..170S}) is a giant LAB discovered in the $z=3.1$ SSA22 proto-cluster region and one of the most well-studied LABs (e.g., \citealt{2004ApJ...606...85C}; \citealt{2014ApJ...793...22G}; \citealt{2011Natur.476..304H}; \citealt{2015ApJ...799...38K}).
%Recent ALMA observations in band~7 (at $860~\mu$m)
% to search for dust continuum 
%unveiled three dusty star-forming galaxies seen in projection at the core, which may power the bright Ly$\alpha$ emission (\citealt{2016ApJ...832...37G}; Y. Matsuda et al. in preparation).
%Additionally, multiple LBGs and DRGs at $z_{\rm spec}\approx3.09$ are embedded in LAB1 (e.g., \citealt{2003ApJ...592..728S}; \citealt{2015ApJ...799...38K}).
The unique environment
%, together with abundant multi-wavelength data, and the richness of various associated galaxies 
makes LAB1 a useful laboratory for investigating the [\textsc{Cii}] emission from growing galaxies in the early universe.
Throughout the paper, we adopt a cosmology with 
$\Omega_{\rm m}=0.3, \Omega_\Lambda=0.7$, and H$_0$=70 km s$^{-1}$ Mpc$^{-1}$.


\section{Observations and data reduction}

We observed LAB1 with ALMA in band~8 as a part of an ALMA cycle-2 program (ID: 2013.1.00159.S; PI: Umehata), targeting the \textsc{[Cii]} 158~$\mu$m transition ($\nu_{\rm rest}=1900.537$~GHz, redshifted to 463.55~GHz or 647~$\mu$m, at $z=3.100$).
As shown in Fig.~\ref{f1}, the field of view (FoV) at $\sim$ 464 GHz is large enough to cover the majority of the Ly$\alpha$ emitting region ($d\sim13.5^{\prime\prime}$ or $\sim 100$ kpc at $z=3.1$ ) and contains three $860~\mu$m continuum ALMA sources: LAB1-ALMA1, LAB1-ALMA2, and LAB1-ALMA3 (hereafter ALMA1, ALMA2, and ALMA3, respectively; \citealt{2016ApJ...832...37G})\footnote{ALMA1, ALMA2, and ALMA3 correspond to SSA22-LAB01 ALMA b, SSA22-LAB01 ALMA a, and SSA22-LAB01 ALMA c in \citet{2016ApJ...832...37G}, respectively.}.
ALMA3 is spatially coincident with a DRG at $z_{\rm spec}=3.1$ (\citealt{2015ApJ...799...38K}).
While ALMA1 and ALMA2 do not have spectroscopic redshifts, their photometric redshifts and 
the low probability of chance association of ALMA sources suggest a physical association between the two ALMA sources and the giant Ly$\alpha$ nebula (\citealt{2012ApJ...750..116U}; Y. Matsuda et al. in preparation).
Three other galaxies at $z_{\rm spec}\simeq3.1$ (a LBG, a K-band~selected galaxy, and a [\textsc{Oiii}] emitter) are also located within the band~8 FoV (Fig.~\ref{f1}).

\begin{figure*}
\epsscale{1.16}
\plotone{f2.eps}
\caption{
Images of LAB1-ALMA3. The size of each map is $3^{\prime\prime}\times3^{\prime\prime}$.
%\textcolor{cyan}{Images are not corrected for primary beam attenuation for each panel.}
(a) The velocity-integrated map of the [\textsc{Cii}] emission. The background map is the ``tapered'' map (0$^{\prime\prime}$.53 FWHM, magenta contours), while we also show the ``full'' map (0$^{\prime\prime}$.27 FWHM, blue contours) for comparison. Contours start at $\pm2\sigma$, with steps of 1$\sigma$ for both.
(b) The velocity map of the  [\textsc{Cii}] emission, blanked at 2.5$\sigma$.
Velocities are relative to the [\textsc{Oiii}] peak (see also Fig. \ref{spectra}) and velocity contours are shown in steps of 80~km s$^{-1}$.
%, including emission above $2.5\sigma$ significance per 80~km s$^{-1}$ channel ($\sigma=3.5$ mJy beam$^{-1}$ at the phase center).
(c) The ``tapered'' band~7 continuum map (0$^{\prime\prime}$.55 FWHM), which presents rest-frame 210~$\mu$m continuum emission.
Contours are plotted from $\pm2\sigma$ in steps of 1$\sigma$.
%, where $\sigma= XX$ mJy beam$^{-1}$.
For comparison, we also show contours of the tapered [\textsc{Cii}] map presented in panel a.
(d) The $HST$ STIS optical image, compared to the [\textsc{Cii}] emission.
(e) The ``tapered'' [\textsc{Nii}] map. 
Contours are $\pm2\sigma$.
% ($\sigma= XX$ mJy beam$^{-1}$).
[\textsc{Cii}] emission is same as other panels.
}
\label{cii_map}
\end{figure*}

\begin{figure}
\epsscale{1.15}
\plotone{f3.eps}
\caption{
[\textsc{Cii}] spectrum of LAB1-ALMA3, integrated over a region of $d=1^{\prime\prime}$ in the tapered cube after correcting for the primary-beam response.
We also show the redshifts and errors determined from [\textsc{Cii}] (red lines) and [\textsc{Oiii}]/H$\beta$ (green lines) detections.
Velocities are relative to the [\textsc{Oiii}]/H$\beta$ redshift ($z=3.1000\pm0.0003$; \citealt{2015ApJ...799...38K}).
[\textsc{Cii}] emission from ALMA3 is detected at consistent redshift ($z=3.0993\pm0.0004$) with FWHM of $270\pm30$ km~s$^{-1}$.
The velocity range used to create the images in Fig.~\ref{cii_map} is indicated below the spectrum.
}
\label{spectra}
\end{figure}


Observations were carried out on 16 June 2015 using a spectral scan mode with the FDM correlator mode to cover the redshift range of the proto-cluster, $z=3.06-3.12$ (\citealt{2005ApJ...634L.125M}).
Among four planned spectral windows, only two were actually executed.
The incomplete observation resulted in frequency coverage of 461.03--462.78 GHz and 462.91--464.66 GHz ($z_\textsc{[Cii]}$=3.090--3.105, 3.107--3.122) after flagging the edge channels.
The array configuration was C34-5 and the baseline lengths were 21--784~m.
The on-source time was 4.5~minutes.
Ceres was observed for amplitude calibration, and 
the quasar J2148+0657 was utilized for bandpass and phase calibration.
The data were processed with the Common Astronomy Software Application ({\sc casa}) ver.~4.4.0 (\citealt{2007ASPC..376..127M}).
%% Imaging
The cube was first created with the natural weighting using the {\sc casa} task, {\sc clean}.
The resultant cube (hereafter ``full'' cube) has a typical synthesized beam FWHM of $0^{\prime\prime}.27\times0^{\prime\prime}.26$ (P.A. 46 deg).
We also created a ``tapered'' cube adopting the taper parameter, outertaper = 0.5 arcsec, which has a typical synthesized beam, $0^{\prime\prime}.53\times0^{\prime\prime}.52$ (P.A. $-70$ deg).
The typical rms level is $\approx$ 3.5 mJy beam$^{-1}$ at the phase center per 80 km s$^{-1}$ channel in the tapered cube. 
To search for band~8 continuum sources, we created a ``tapered'' continuum map at 463~GHz, using the line-free channels.
The ``dirty'' map has a rms level of 0.8~mJy beam$^{-1}$ at the phase center and none of the sources is found above 5$\sigma$. 

LAB1 has also been observed by ALMA in band~7.
One program (ID. 2013.1.00704.S; PI. Matsuda) covered the redshifted \textsc{[Nii]} 205~$\mu$m transition line ($\nu_{\rm rest}=1461.131$~GHz, redshifted to 356.37~GHz, at $z=3.100$) (Y. Matsuda et al. in preparation). 
The typical noise rms at $0^{\prime\prime}.55$ resolution, which is equivalent to the ``tapered'' cube in band~8, is $\approx$0.4 mJy beam$^{-1}$ at the phase center, per 80 km s$^{-1}$ channel. 
%(More detailed information will be presented in Y. Matsuda et al. in preparation).


\section{Results}

\subsection{\textsc{[Cii]} 158~$\mu$m in LAB1-ALMA3}

We detected \textsc{[Cii]} emission from one of the three dusty star-forming galaxies, ALMA3 (Fig.~\ref{cii_map} and Fig.~\ref{spectra}).
Fig.~\ref{spectra} shows the \textsc{[Cii]} spectrum.
A gaussian fit to the line has $z=3.0993\pm0.0004$ with FWHM $275\pm30$ km~s$^{-1}$.
\citet{2015ApJ...799...38K} reported a redshift of $z=3.1000\pm0.0003$ on the basis of H$\beta$ and \textsc{[Oiii]} $\lambda$5007 lines, and hence our measurement is  consistent (the velocity offset is within $\sim50$~km~s$^{-1}$ and the two measurements are consistent within errors).
Fig.~\ref{cii_map} shows the velocity-integrated \textsc{[Cii]} intensity and velocity maps, compared to the rest-frame 210~$\mu$m continuum (Y. Matsuda et al. in preparation; \citealt{2016ApJ...832...37G}),  $HST$ STIS optical image\footnote{The image has a pivot wavelength of 5733~\AA.} (\citealt{2003ApJ...599...92C}), and \textsc{[Nii]} image\footnote{We created the \textsc{[Nii]} image, integrated the cube over the same velocity range of the \textsc{[Cii]} map.}.
The \textsc{[Cii]} emission is spatially resolved as shown in Fig.~\ref{cii_map}a, while
the \textsc{[Cii]} emission has a modest signal to noise ratio and the various clumps seen are not significant.
The \textsc{[Cii]} velocity map (Fig. 2b) also shows complexity, which is not likely to be produced by a simple rotating disk.
The position of \textsc{[Cii]} emission is generally consistent with those of dust continuum and stellar emission\footnote{There might be a small offset, $\sim0^{\prime\prime}.2$, though the current data is insufficient to determine whether it is real.}.
%We cannot find significant \textsc{[Nii]} emission at the position and velocity range of \textsc{[Cii]} emission.

%\begin{landscape}
\begin{center}
\begin{deluxetable*}{ccccccccccc}
\tabletypesize{\scriptsize}
%\rotate
\tablecaption{[\textsc{Cii}] Line Parameters of galaxies in SSA22-LAB1}
\tablewidth{0pt}
\tablehead{
\colhead{Galaxy} &\colhead{RA} &\colhead{Dec} & \colhead{$z$} & \colhead{Type} & \colhead{Ref} & $I_\textsc{Cii}$ &  $L_\textsc{Cii}$ & $L_{\rm IR}$ \\
                             & (J2000)        & (J2000)              &            &                           &                              &   (Jy km$^{-1}$)     & (10$^{9} L_\sun$) & (10$^{11} L_\sun$)  \\                 
}
\startdata
LAB1-ALMA3 & 22:17:26.11 &   +00:12:32.4 & $3.0993\pm0.0004$  & [\textsc{Cii}] 158~$\mu$m                     & 1   & 16.8$\pm$2.1 & 5.7$\pm0.7$ & 5.8 \\
                       &  22:17:26.1  &  +00:12:32.3  & $3.1000\pm0.0003$  & [\textsc{Oiii}] $\lambda5007$, H$\beta$ & 2 &       ---        & --- & \\ 
LAB1-ALMA1 & 22:17:25.94  &  +00:12:36.6 & (3.1?) & photo-$z$ & --- & ($<2.3$) & ($<0.8$) & 3.5 \\
LAB1-ALMA2 & 22:17:26.01 &   +00:12:36.4 & (3.1?) &  photo-$z$ & --- & ($<2.3$) & ($<0.8$) & 4.0 \\
C11 (LBG)     & 22:17:25.7  &  +00:12:34.7 & $3.0999\pm0.0004$ & [\textsc{Oiii}] $\lambda5007$ & 3 &  $<3.0$ & $<1.0$ & --- \\
K1 (K-band~galaxy) & 22:17:25.7 &   +00:12:38.7 & $3.1007\pm0.0002$ & [\textsc{Oiii}] $\lambda5007$ & 2 & $<2.6$ & $<0.9$& ---\\
S1 ([\textsc{Oiii}] emitter) & 22:17:26.08  &  +00:12:34.2 &   3.0968 & [\textsc{Oiii}] $\lambda5007$ & 4 &$<2.2$ & $<0.7$ & ---
\enddata
%% Text for table notes should follow after the \enddata but before
%% the \end{deluxetable}. Make sure there is at least one \tablenotemark
%% in the table for each \tablenotetext.
\tablecomments{
[\textsc{Cii}] Line properties of three ALMA sources and three UV/optical selected galaxies.
Since ALMA1 and ALMA2 don't have $z_{\rm spec}$, we estimated rough upper limits using the cube for ALMA3, assuming same redshifts and velocity widths.
For C11, K1, and S1, we integrated the cube at the position in literatures over $300$~km s$^{-1}$ velocity range, and obtain 3$\sigma$ upper limits.
References are: 1. This work, 2. \citealt{2015ApJ...799...38K}, 3. \citealt{2013ApJ...767...48M}, and 4. \citealt{2016ApJ...832...37G}.
}
\end{deluxetable*}
\end{center}
%\end{landscape}

% [cii] properties
To describe the properties of \textsc{[Cii]} emission from the whole galaxy, we use the tapered map.
A two-dimensional elliptical Gaussian fit yields a deconvolved FWHM of ($0^{\prime\prime}.62\pm0^{\prime\prime}.11)\times(0^{\prime\prime}.55\pm0^{\prime\prime}.10$), which corresponds to $4.8\times4.3$~kpc$^2$.
For comparison, we similarly measured the size of the dusty starburst core using the band~7 continuum image at 0$^{\prime\prime}$.35 resolution.
The yielded size is ($0^{\prime\prime}.53\pm0^{\prime\prime}.14)\times(0^{\prime\prime}.40\pm0^{\prime\prime}.12$) ($4.1\times3.1$~kpc$^2$).
The measured integrated line flux is $I_\textsc{[Cii]}=16.8\pm2.1$~Jy km s$^{-1}$ and hence the line luminosity is $L_\textsc{[Cii]}=(5.7\pm0.7) \times10^{9} L_\sun$ (Table 1). 
The infrared (IR; 8-1000~$\mu$m) luminosity of ALMA3 is derived using an average SMG template from the ALESS survey (\citealt{2014MNRAS.438.1267S}) scaled to the 860~$\mu$m flux density, $S_{\rm 860 \mu m}=0.73\pm0.05$~mJy (\citealt{2016ApJ...832...37G}); $L_{\rm IR}\approx5.8\pm0.4\times10^{11} L_\odot$, so that $L_\textsc{[Cii]}/L_{\rm IR}\approx0.010\pm0.001$
(We note that the IR luminosity may have larger uncertainty.
\citet{2016ApJ...832...37G} estimated it in the range $L_{\rm IR}\approx(0.2-1.5)\times10^{12} L_\odot$ using varying templates.).
We also derived the dynamical mass of ALMA3, $M_{\rm dyn, vir}\sim 1.0\times10^{11} M_\odot$, using an isotropic virial estimator (e.g., \citealt{2010ApJ...724..233E}) on the basis of the line width and \textsc{[Cii]} size (major axis measured from the FWHM).

% [nii] properties
We also searched for \textsc{[Nii]} 205~$\mu$m emission from ALMA3, which resulted in non-detection (Fig.~\ref{cii_map}e).
Utilizing the \textsc{[Nii]} map at $0^{\prime\prime}.55$ resolution, we obtained a 3$\sigma$ (point-source) upper limit on its line intensity, $I_\textsc{[Nii]}<0.35$~Jy km s$^{-1}$ and 
thus $L_\textsc{[Nii]}<9.4\times10^{7} L_\sun$, and $L_\textsc{[Cii]}/L_\textsc{[Nii]}>61$.
The \textsc{[Nii]} upper limit can slightly be relaxed when the \textsc{[Nii]} 205~$\mu$m emission has larger extent compared to the size of the synthesized beam.
If we use the the other tapered \textsc{[Nii]} map at $0^{\prime\prime}.64$ resolution, which is comparable to the measured \textsc{[Cii]} size of ALMA3, we will have $I_\textsc{[Nii]}<0.39$~Jy km s$^{-1}$, $L_\textsc{[Nii]}<1.0\times10^{8} L_\sun$, and $L_\textsc{[Cii]}/L_\textsc{[Nii]}>55$, respectively.
In the following discussion, we adopt the latter conservatively.


\subsection{No \textsc{[Cii]} emission from the remaining LAB1 members}

Except for ALMA3, no emission line is found in the band~8 cube.
For ALMA1 and ALMA2, we just calculate a tentative upper limit of \textsc{[Cii]} emission, assuming that 
%the two galaxies are within LAB1 and 
the lines fall within our frequency coverage and the line widths are same as that of ALMA3.
%, which allows us to obtain upper limits on the \textsc{[Cii]} line emission for other LAB1-associated objects.
%The remaining two ALMA sources, ALMA1 and ALMA2, are suggested to reside within LAB1, although their redshifts have not yet been confirmed (\citealt{2016ApJ...832...37G}).
The IR luminosities of ALMA1 and ALMA2 are comparable to that of ALMA3 ($L_{\rm IR}\approx3.5\times10^{11}L_\odot$ and $L_{\rm IR}\approx4.0\times10^{11}L_\odot$, respectively)\footnote{
\citet{2016ApJ...832...37G} reported the sum of 860~$\mu$m flux density, $S_{\rm 860 \mu m}=0.95\pm0.04$~mJy.
We apportioned it between ALMA1 and ALMA2 according to their peak flux density at $0^{\prime\prime}.35$ resolution (Y. Matsuda et al. in preparation) and calculated IR luminosity in the same way for ALMA3.}.
%Here we just calculate a tentative upper limit of \textsc{[Cii]} emission, assuming that 
%the two galaxies are within LAB1 and 
%the lines fall within our frequency coverage {\bf and the line widths are same as that of ALMA3}.
Utilizing the intensity map for ALMA3, we obtained a 3$\sigma$ upper limit on their individual line intensity, $I_\textsc{[Cii]}<2.3$~Jy km s$^{-1}$, and line luminosity, $ L_\textsc{[Cii]}<0.8 \times10^{9} L_\sun$.
%Considering the sum of ALMA1 and ALMA2 as \citet{2016ApJ...832...37G}, the system may have $L_\textsc{[Cii]}/L_{\rm IR}<XX$.
Although this is just a crude estimate and $z_{\rm spec}$ information is essential for further discussion, 
our result suggests that the $L_\textsc{[Cii]}/L_{\rm IR}$ of ALMA1 and ALMA2 may be different from that of ALMA3.
We also evaluated 3$\sigma$ upper limits for the three rest-frame UV/optical galaxies with \textsc{[Oiii]} line detections, by integrating the cube over 300~km~s$^{-1}$ at the source position (Table 1).


\section{Discussion and Summary}

\begin{figure*}
\epsscale{1.15}
\plotone{f4.eps}
\caption{
({\it left}) \textsc{[Cii]}-IR luminosity ratio ($L_\textsc{[Cii]}/L_{\rm IR}$) as a function of IR luminosity ($L_{\rm IR}$). 
We show the measured ratio of LAB1-ALMA3 and the ``upper limit'' of LAB1-ALMA1 and ALMA2, 
assuming their redshifts lie within our \textsc{[Cii]} coverage (see text).
We also mark local IR-luminous galaxies (\citealt{2013ApJ...774...68D}), the SMGs at $z=3\sim6$ (\citealt{2014ApJ...796...84R} (R14); \citealt{2014ApJ...782L..17D}; \citealt{2014A&A...565A..59D}; \citealt{2014ApJ...783...59R}; \citealt{2015MNRAS.449.2883G} (G15)), LBGs at $z=5-6$ (R14; \citealt{2015Natur.522..455C} (C15)), and $z=1\sim2$ star-forming galaxies (including SMGs; \citealt{2010ApJ...724..957S} (S10); \citealt{2015ApJ...799...13B} (B15)).
ALMA3 shows high \textsc{[Cii]}-IR luminosity ratio, compared to other IR luminous galaxies with similar luminosity. Here we convert the IR luminosities in the literature, multiplying by the following factors: $L_{\rm 8-1000\mu m}/L_{\rm 42.5-122.5\mu m}=1.7$, $L_{\rm 8-1000\mu m}/L_{\rm 42.5-500\mu m}=1.3$.
({\it right}) \textsc{[Cii]} 158~$\mu$m-\textsc{[Nii]} 205~$\mu$m line luminosity ratio ($L_\textsc{[Cii]}/L_\textsc{[Nii]}$) as a function of IR luminosity ($L_{\rm IR}$). 
The ratio of  LAB1-ALMA3 is shown, compared with those of various galaxies at $z\sim5$ (\citealt{2014ApJ...783...59R}; \citealt{2014ApJ...782L..17D}; \citealt{2016A&A...586L...7B}; \citealt{2016arXiv160702520P}, and references therein) and local (U)LIRGs (\citealt{2016ApJ...819...69Z}; \citealt{2013ApJ...774...68D}).
ALMA3 shows one of the highest values seen to date, which indicates an enhanced \textsc{[Cii]} emission.}
\label{ratio}
\vspace{0.79cm}
\end{figure*}

%Our survey has unveiled one \textsc{[Cii]} luminous galaxy, ALMA3. 
%In this section, we discuss what is responsible for the \textsc{[Cii]} emission and what the discovery means.
One striking characteristic of ALMA3 is the high \textsc{[Cii]}--IR ratio seen in Fig.~\ref{ratio}.
While this ratio is known to decrease as IR luminosity increases (``\textsc{[Cii]} deficit'') for local and high-redshift IR luminous galaxies (e.g., \citealt{2013ApJ...774...68D}), ALMA3 shows approximately an order of magnitude higher ratio (Fig.~\ref{ratio}) at the same IR luminosity range
(This trend is independent of the uncertainties on $L_{\rm IR}$ described in \S 3.1. 
While the $L_\textsc{[Cii]}/L_{\rm IR}$ ratio may be $\sim\times$3 lower, the increased corresponding $L_{\rm IR}$ keeps the trend.).
The result implies different conditions responsible for \textsc{[Cii]} emission between ALMA3 and the majority of previously known IR luminous galaxies.
It has also been reported that some $z\sim1-2$ ULIRGs show $L_\textsc{[Cii]}/L_{\rm IR}$ ratios comparable to ALMA3, although they have slightly higher $L_{\rm IR}$ than ALMA3 (\citealt{2015ApJ...799...13B}).
One possible explanation for elevated \textsc{[Cii]}--IR ratios is that the galaxies host widely spread star formation, and the UV radiation field is therefore diluted, which make the \textsc{[Cii]} line a more efficient coolant (see e.g., \citealt{2015A&A...574A..14C}; \citealt{2015ApJ...799...13B}, and references therein).
The size of the dust continuum core in ALMA3 is 4.1~kpc, which
% 860~$\mu$m continuum in ALMA3 is $3.7$~kpc (FWHM, measured along the major axis; Y. Matsuda et al. in preparation), which}
% is $\approx0^{\prime\prime}.48$ (FWHM, $\approx3.7$~kpc; Y. Matsuda et al. in prepration)} 
is larger than a typical continuum size of bright SMGs at similar redshifts (2.4~kpc; \citealt{2015ApJ...799...81S}; see also \citealt{2015ApJ...810..133I}; \citealt{2016arXiv161109857U}). 
This supports that a relatively extended star-forming region in ALMA3 contributes the high \textsc{[Cii]}/IR ratio for ALMA3.
Gas accretion from the cosmic web is expected to accumulate a large amount of molecular gas necessary to fuel such widespread star formation (\citealt{2015ApJ...799...13B}).

We have another clue from the \textsc{[Cii]} 158~$\mu$m--\textsc{[Nii]} 205~$\mu$m line luminosity ratio, $L_\textsc{[Cii]}/L_\textsc{[Nii]}$.
ALMA3 shows one of the largest ratios ever reported (Fig.~\ref{ratio}).
The $L_\textsc{[Cii]}/L_\textsc{[Nii]}$ ratio has been utilized to diagnose the ISM conditions.
In particular, it is used to evaluate global trend on the fraction of \textsc{[Cii]} emission associated with ionized regions (i.e. \textsc{Hii} regions; e.g., \citealt{2006ApJ...652L.125O}; \citealt{2014ApJ...782L..17D};  \citealt{2016arXiv160702520P}), mainly because Nitrogen's ionization potential (14.5~eV) is higher than that of Hydrogen (13.6~eV) so that \textsc{[Nii]} arises only from ionized regions.
%In addition, critical densities of the two lines are similar and the ratio is insensitive to the {\bf electron} density in \textsc{Hii} regions of a fixed metallicity.
\citet{2016arXiv160702520P} reported the expected a line ratio $L_\textsc{[Cii]}/L_\textsc{[Nii]}\approx3.5$, for \textsc{Hii} regions with electron density of $\sim10-1000$~cm$^{-3}$.
If we adopt this estimate, it is expected that the contribution of ionized gas is only about $\sim$6\% and the vast majority of \textsc{[Cii]} emission arises from the surface of dense PDRs and/or other regions/environments.
The $L_\textsc{[Cii]}/L_\textsc{[Nii]}$ ratio is also sensitive to estimate gas metallicity (e.g., \citealt{2012A&A...542L..34N}; \citealt{2016A&A...586L...7B}; \citealt{2016arXiv160702520P}).
\citet{2012A&A...542L..34N} suggests that the line ratio increases as metallicity decreases, considering both PDRs and \textsc{Hii} regions in their model.
The measured ratio, $L_\textsc{[Cii]}/L_\textsc{[Nii]}>55$, favors sub-solar metallicity for the variety of densities and ionization parameters in their model. 
Gas accretion from the outside of ALMA3 may explain this relatively low metallicity.
%
It is suggested that Nitrogen may dominantly be in its doubly ionized state in high ionization conditions with lower dust shielding (e.g., \citealt{2016arXiv160702520P}).
This effect is unlikely to be significant in ALMA3 because it is detected in dust continuum.

Although it is not straightforward to identify the origin of \textsc{[Cii]} emission more, together with these clues, 
the properties and location of ALMA3 may support the importance of shock on the elevated \textsc{[Cii]} emission.
Recently some work has suggested that mechanical heating due to turbulence in shocks can contribute to \textsc{[Cii]} emission at high redshift (e.g., \citealt{2010ApJ...724..957S}; \citealt{2013A&A...550A.106L}; \citealt{2013ApJ...777...66A}; \citealt{2015ApJ...799...13B}).
For instance, \citet{2013ApJ...777...66A} reported that the resolved shocked regions of Stephan's Quintet have exceptionally high \textsc{[Cii]}--FIR ratio and they also suggest that this could be commonplace for high-redshift galaxies. 
%\citet{2015ApJ...799...13B} proposed that some of star-forming galaxies at $z~1-2$ can be 
\citet{2015ApJ...799...13B} suggested that a variety of shocks, originating from major-merger, intergalactic gas accretion, and stellar outflows, might contribute to the elevated \textsc{[Cii]} emission.
ALMA3 shows complicated rest-frame UV morphologies and \textsc{[Cii]} velocity structures (Fig.~\ref{cii_map}), which is suggestive of galaxy-galaxy interaction (dust obscuration may also contribute to it).
ALMA3 hosts intense star-formation activity, as the dust continuum detection shows, 
and appears to be a relatively evolved system with large stellar mass $M_*\approx10^{11} M_\odot$ 
(\citealt{2015ApJ...799...38K}) comparable to the derived dynamical mass (we need to recognize both estimates contain large uncertainties).
Therefore galactic outflow may interact with intergalactic gas stream (e.g., \citealt{2016arXiv160904405C}).
Thus shock heating might be a contributor of \textsc{[Cii]} emission from ALMA3.

%\section{Conclusion}
%In summary, 
%our ALMA band~8 survey has unveiled that one dusty star-forming galaxy at $z=3.1$, ALMA3, has relatively bright \textsc{[Cii]} 158~$\mu$m emission.
%One plausible scenario is 
%We show that ALMA3 hosts extended star formation and abundant PDRs, and its metallicity is relatively low, both of which may be caused by gas accretion from the cosmic web. We also suggest a possibility that shock heating may account for some portion of \textsc{[Cii]} emission, as suggested by some previous work. 

One key question is the role of environment, since ALMA3 is located within a giant LAB, SSA22-LAB1. 
LAB1 resides in a remarkable proto-cluster and is associated with a number of star-forming galaxies, which may reflect the abundant gas accretion from cosmic web. The overdensity of galaxies may lead a high frequency of galaxy-galaxy interaction. Therefore the unique environment might account for the relatively strong \textsc{[Cii]} line. On the other hand, if ALMA1 and ALMA2 are actually at redshifts similar to confirmed LAB1 members, the absence of detectable \textsc{[Cii]} would mean diversity of the ISM state within a LAB.
While we detected the \textsc{[Cii]} line from a massive, dusty star-forming galaxy,
much deeper observations of FIR lines like \textsc{[Cii]} and \textsc{[Nii]} toward a giant LAB at $z\sim3$, which allows us to assess the ISM state in UV/optical selected  galaxies (e.g., LBGs like C11 in LAB1), is highly expected.
Such surveys will give us an opportunity to estimate how the ISM in the galaxies evolve in biased regions in the early universe, through the comparison with other FIR line observations of galaxies in a biased region (e.g., AzTEC3 and LBG1 at $z=5.3$; e.g., \citealt{2014ApJ...796...84R}; \citealt{2016arXiv160702520P}) or galaxies in general environment in the same era.
 
%Deeper observations of giant LABs are needed to approach ...
%Spatially resolved...
%Further observations covering various properties and/or conditions will allow us to characterize the mechanism responsible for \textsc{[Cii]} emission and diagnose ISM phases more.


%We discovered a dusty star-forming galaxy with elevated \textsc{[Cii]} emission, embedded in a giant LAB at $z=3.1$.
%Further observations covering various properties and/or conditions (e.g., environments, morphologies, and stellar masses) will allow us to characterize the mechanism responsible for \textsc{[Cii]} emission and diagnose ISM phases more.

\acknowledgments
We greatly appreciate the anonymous referee for a helpful report.
HU is supported by the ALMA Japan Research Grant of NAOJ Chile Observatory, NAOJ-ALMA-0071, 0131, 140, and 0152. HU is supported by JSPS Grant-in-Aid for Research Activity Start-up (16H06713).
HU is thankful for the support from JSPS KAKENHI No 16H02166 (PI. Y. Taniguchi). 
YT is supported by JSPS KAKENHI No. 25102073.
RJI acknowledges support from ERC in the form of the Advanced Investigator Programme, 321302, COSMICISM.
IRS acknowledge support from STFC (ST/L00075X/1).
IRS acknowledge support from the ERC Advanced Investigator program
DUSTYGAL 321334, and a Royal Society/Wolfson Merit Award. 
MH acknowledges the support of the Swedish Research Council,
Vetenskapsr{\aa}det and the Swedish National Space Board (SNSB), and is Fellow of the Knut and Alice Wallenberg Foundation.
This paper makes use of the following ALMA data: ADS/JAO.ALMA\#2013.1.00159.S, ADS/JAO.ALMA\#2013.1.00704.S. ALMA is a partnership of ESO (representing its member states), NSF (USA) and NINS (Japan), together with NRC (Canada) and NSC and ASIAA (Taiwan) and KASI (Republic of Korea), in cooperation with the Republic of Chile. The Joint ALMA Observatory is operated by ESO, AUI/NRAO and NAOJ.


%% To help institutions obtain information on the effectiveness of their
%% telescopes, the AAS Journals has created a group of keywords for telescope
%% facilities. A common set of keywords will make these types of searches
%% significantly easier and more accurate. In addition, they will also be
%% useful in linking papers together which utilize the same telescopes
%% within the framework of the National Virtual Observatory.
%% See the AASTeX Web site at http://aastex.aas.org/
%% for information on obtaining the facility keywords.

%% After the acknowledgments section, use the following syntax and the
%% \facility{} macro to list the keywords of facilities used in the research
%% for the paper.  Each keyword will be checked against the master list during
%% copy editing.  Individual instruments or configurations can be provided 
%% in parentheses, after the keyword, but they will not be verified.

{\it Facilities:} \facility{ALMA}.

%% Appendix material should be preceded with a single \appendix command.
%% There should be a \section command for each appendix. Mark appendix
%% subsections with the same markup you use in the main body of the paper.

%% Each Appendix (indicated with \section) will be lettered A, B, C, etc.
%% The equation counter will reset when it encounters the \appendix
%% command and will number appendix equations (A1), (A2), etc.

%% The reference list follows the main body and any appendices.
%% Use LaTeX's thebibliography environment to mark up your reference list.
%% Note \begin{thebibliography} is followed by an empty set of
%% curly braces.  If you forget this, LaTeX will generate the error
%% "Perhaps a missing \item?".
%%
%% thebibliography produces citations in the text using \bibitem-\cite
%% cross-referencing. Each reference is preceded by a
%% \bibitem command that defines in curly braces the KEY that corresponds
%% to the KEY in the \cite commands (see the first section above).
%% Make sure that you provide a unique KEY for every \bibitem or else the
%% paper will not LaTeX. The square brackets should contain
%% the citation text that LaTeX will insert in
%% place of the \cite commands.

%% We have used macros to produce journal name abbreviations.
%% AASTeX provides a number of these for the more frequently-cited journals.
%% See the Author Guide for a list of them.

%% Note that the style of the \bibitem labels (in []) is slightly
%% different from previous examples.  The natbib system solves a host
%% of citation expression problems, but it is necessary to clearly
%% delimit the year from the author name used in the citation.
%% See the natbib documentation for more details and options.
\bibliographystyle{apj}
%\bibliographystyle{plainnat} % use this to have URLs listed in References
%\bibliography{../Reference/references} % Path to your References.bib file
%\bibliography{Umehata16_lab1_161121.bbl}
%\bibliography{Orion_paper_bib}
%\bibliography{ms.bbl}
\begin{thebibliography}{}
\expandafter\ifx\csname natexlab\endcsname\relax\def\natexlab#1{#1}\fi

\bibitem[{{Appleton} {et~al.}(2013){Appleton}, {Guillard}, {Boulanger},
  {Cluver}, {Ogle}, {Falgarone}, {Pineau des For{\^e}ts}, {O'Sullivan}, {Duc},
  {Gallagher}, {Gao}, {Jarrett}, {Konstantopoulos}, {Lisenfeld}, {Lord}, {Lu},
  {Peterson}, {Struck}, {Sturm}, {Tuffs}, {Valchanov}, {van der Werf}, \&
  {Xu}}]{2013ApJ...777...66A}
{Appleton}, P.~N., {Guillard}, P., {Boulanger}, F., {et~al.} 2013, \apj, 777,
  66

\bibitem[{{B{\'e}thermin} {et~al.}(2016){B{\'e}thermin}, {De Breuck},
  {Gullberg}, {Aravena}, {Bothwell}, {Chapman}, {Gonzalez}, {Greve}, {Litke},
  {Ma}, {Malkan}, {Marrone}, {Murphy}, {Spilker}, {Stark}, {Strandet},
  {Vieira}, {Wei{\ss}}, \& {Welikala}}]{2016A&A...586L...7B}
{B{\'e}thermin}, M., {De Breuck}, C., {Gullberg}, B., {et~al.} 2016, \aap, 586,
  L7

\bibitem[{{Brisbin} {et~al.}(2015){Brisbin}, {Ferkinhoff}, {Nikola},
  {Parshley}, {Stacey}, {Spoon}, {Hailey-Dunsheath}, \&
  {Verma}}]{2015ApJ...799...13B}
{Brisbin}, D., {Ferkinhoff}, C., {Nikola}, T., {et~al.} 2015, \apj, 799, 13

\bibitem[{{Capak} {et~al.}(2015){Capak}, {Carilli}, {Jones}, {Casey},
  {Riechers}, {Sheth}, {Carollo}, {Ilbert}, {Karim}, {Lefevre}, {Lilly},
  {Scoville}, {Smolcic}, \& {Yan}}]{2015Natur.522..455C}
{Capak}, P.~L., {Carilli}, C., {Jones}, G., {et~al.} 2015, \nat, 522, 455

\bibitem[{{Chapman} {et~al.}(2004){Chapman}, {Scott}, {Windhorst}, {Frayer},
  {Borys}, {Lewis}, \& {Ivison}}]{2004ApJ...606...85C}
{Chapman}, S.~C., {Scott}, D., {Windhorst}, R.~A., {et~al.} 2004, \apj, 606, 85

\bibitem[{{Chapman} {et~al.}(2003){Chapman}, {Windhorst}, {Odewahn}, {Yan}, \&
  {Conselice}}]{2003ApJ...599...92C}
{Chapman}, S.~C., {Windhorst}, R., {Odewahn}, S., {Yan}, H., \& {Conselice}, C.
  2003, \apj, 599, 92

\bibitem[{{Cicone} {et~al.}(2015){Cicone}, {Maiolino}, {Gallerani}, {Neri},
  {Ferrara}, {Sturm}, {Fiore}, {Piconcelli}, \&
  {Feruglio}}]{2015A&A...574A..14C}
{Cicone}, C., {Maiolino}, R., {Gallerani}, S., {et~al.} 2015, \aap, 574, A14

\bibitem[{{Cornuault} {et~al.}(2016){Cornuault}, {Lehnert}, {Boulanger}, \&
  {Guillard}}]{2016arXiv160904405C}
{Cornuault}, N., {Lehnert}, M., {Boulanger}, F., \& {Guillard}, P. 2016, ArXiv
  e-prints, arXiv:1609.04405

\bibitem[{{De Breuck} {et~al.}(2014){De Breuck}, {Williams}, {Swinbank},
  {Caselli}, {Coppin}, {Davis}, {Maiolino}, {Nagao}, {Smail}, {Walter},
  {Wei{\ss}}, \& {Zwaan}}]{2014A&A...565A..59D}
{De Breuck}, C., {Williams}, R.~J., {Swinbank}, M., {et~al.} 2014, \aap, 565,
  A59

\bibitem[{{Decarli} {et~al.}(2014){Decarli}, {Walter}, {Carilli}, {Bertoldi},
  {Cox}, {Ferkinhoff}, {Groves}, {Maiolino}, {Neri}, {Riechers}, \&
  {Weiss}}]{2014ApJ...782L..17D}
{Decarli}, R., {Walter}, F., {Carilli}, C., {et~al.} 2014, \apjl, 782, L17

\bibitem[{{D{\'{\i}}az-Santos} {et~al.}(2013){D{\'{\i}}az-Santos}, {Armus},
  {Charmandaris}, {Stierwalt}, {Murphy}, {Haan}, {Inami}, {Malhotra},
  {Meijerink}, {Stacey}, {Petric}, {Evans}, {Veilleux}, {van der Werf}, {Lord},
  {Lu}, {Howell}, {Appleton}, {Mazzarella}, {Surace}, {Xu}, {Schulz},
  {Sanders}, {Bridge}, {Chan}, {Frayer}, {Iwasawa}, {Melbourne}, \&
  {Sturm}}]{2013ApJ...774...68D}
{D{\'{\i}}az-Santos}, T., {Armus}, L., {Charmandaris}, V., {et~al.} 2013, \apj,
  774, 68

\bibitem[{{Dijkstra} \& {Loeb}(2009)}]{2009MNRAS.400.1109D}
{Dijkstra}, M., \& {Loeb}, A. 2009, \mnras, 400, 1109

\bibitem[{{Engel} {et~al.}(2010){Engel}, {Tacconi}, {Davies}, {Neri}, {Smail},
  {Chapman}, {Genzel}, {Cox}, {Greve}, {Ivison}, {Blain}, {Bertoldi}, \&
  {Omont}}]{2010ApJ...724..233E}
{Engel}, H., {Tacconi}, L.~J., {Davies}, R.~I., {et~al.} 2010, \apj, 724, 233

\bibitem[{{Erb} {et~al.}(2011){Erb}, {Bogosavljevi{\'c}}, \&
  {Steidel}}]{2011ApJ...740L..31E}
{Erb}, D.~K., {Bogosavljevi{\'c}}, M., \& {Steidel}, C.~C. 2011, \apjl, 740,
  L31

\bibitem[{{Geach} {et~al.}(2005){Geach}, {Matsuda}, {Smail}, {Chapman},
  {Yamada}, {Ivison}, {Hayashino}, {Ohta}, {Shioya}, \&
  {Taniguchi}}]{2005MNRAS.363.1398G}
{Geach}, J.~E., {Matsuda}, Y., {Smail}, I., {et~al.} 2005, \mnras, 363, 1398

\bibitem[{{Geach} {et~al.}(2014){Geach}, {Bower}, {Alexander}, {Blain},
  {Bremer}, {Chapin}, {Chapman}, {Clements}, {Coppin}, {Dunlop}, {Farrah},
  {Jenness}, {Koprowski}, {Micha{\l}owski}, {Robson}, {Scott}, {Smith},
  {Spaans}, {Swinbank}, \& {van der Werf}}]{2014ApJ...793...22G}
{Geach}, J.~E., {Bower}, R.~G., {Alexander}, D.~M., {et~al.} 2014, \apj, 793,
  22

\bibitem[{{Geach} {et~al.}(2016){Geach}, {Narayanan}, {Matsuda}, {Hayes},
  {Mas-Ribas}, {Dijkstra}, {Steidel}, {Chapman}, {Feldmann}, {Avison},
  {Agertz}, {Ao}, {Birkinshaw}, {Bremer}, {Clements}, {Dannerbauer}, {Farrah},
  {Harrison}, {Kubo}, {Micha{\l}owski}, {Scott}, {Smith}, {Spaans}, {Simpson},
  {Swinbank}, {Taniguchi}, {van der Werf}, {Verma}, \&
  {Yamada}}]{2016ApJ...832...37G}
{Geach}, J.~E., {Narayanan}, D., {Matsuda}, Y., {et~al.} 2016, \apj, 832, 37

\bibitem[{{Gullberg} {et~al.}(2015){Gullberg}, {De Breuck}, {Vieira},
  {Wei{\ss}}, {Aguirre}, {Aravena}, {B{\'e}thermin}, {Bradford}, {Bothwell},
  {Carlstrom}, {Chapman}, {Fassnacht}, {Gonzalez}, {Greve}, {Hezaveh},
  {Holzapfel}, {Husband}, {Ma}, {Malkan}, {Marrone}, {Menten}, {Murphy},
  {Reichardt}, {Spilker}, {Stark}, {Strandet}, \&
  {Welikala}}]{2015MNRAS.449.2883G}
{Gullberg}, B., {De Breuck}, C., {Vieira}, J.~D., {et~al.} 2015, \mnras, 449,
  2883

\bibitem[{{Hayes} {et~al.}(2011){Hayes}, {Scarlata}, \&
  {Siana}}]{2011Natur.476..304H}
{Hayes}, M., {Scarlata}, C., \& {Siana}, B. 2011, \nat, 476, 304

\bibitem[{{Ikarashi} {et~al.}(2015){Ikarashi}, {Ivison}, {Caputi}, {Aretxaga},
  {Dunlop}, {Hatsukade}, {Hughes}, {Iono}, {Izumi}, {Kawabe}, {Kohno}, {Lagos},
  {Motohara}, {Nakanishi}, {Ohta}, {Tamura}, {Umehata}, {Wilson}, {Yabe}, \&
  {Yun}}]{2015ApJ...810..133I}
{Ikarashi}, S., {Ivison}, R.~J., {Caputi}, K.~I., {et~al.} 2015, \apj, 810, 133

\bibitem[{{Inoue} {et~al.}(2016){Inoue}, {Tamura}, {Matsuo}, {Mawatari},
  {Shimizu}, {Shibuya}, {Ota}, {Yoshida}, {Zackrisson}, {Kashikawa}, {Kohno},
  {Umehata}, {Hatsukade}, {Iye}, {Matsuda}, {Okamoto}, \&
  {Yamaguchi}}]{2016Sci...352.1559I}
{Inoue}, A.~K., {Tamura}, Y., {Matsuo}, H., {et~al.} 2016, Science, 352, 1559

\bibitem[{{Israel} {et~al.}(1996){Israel}, {Bontekoe}, \&
  {Kester}}]{1996A&A...308..723I}
{Israel}, F.~P., {Bontekoe}, T.~R., \& {Kester}, D.~J.~M. 1996, \aap, 308, 723

\bibitem[{{Kubo} {et~al.}(2015){Kubo}, {Yamada}, {Ichikawa}, {Kajisawa},
  {Matsuda}, \& {Tanaka}}]{2015ApJ...799...38K}
{Kubo}, M., {Yamada}, T., {Ichikawa}, T., {et~al.} 2015, \apj, 799, 38

\bibitem[{{Kubo} {et~al.}(2013){Kubo}, {Uchimoto}, {Yamada}, {Kajisawa},
  {Ichikawa}, {Matsuda}, {Akiyama}, {Hayashino}, {Konishi}, {Nishimura},
  {Omata}, {Suzuki}, {Tanaka}, {Yoshikawa}, {Alexander}, {Fazio}, {Huang}, \&
  {Lehmer}}]{2013ApJ...778..170K}
{Kubo}, M., {Uchimoto}, Y.~K., {Yamada}, T., {et~al.} 2013, \apj, 778, 170

\bibitem[{{Lesaffre} {et~al.}(2013){Lesaffre}, {Pineau des For{\^e}ts},
  {Godard}, {Guillard}, {Boulanger}, \& {Falgarone}}]{2013A&A...550A.106L}
{Lesaffre}, P., {Pineau des For{\^e}ts}, G., {Godard}, B., {et~al.} 2013, \aap,
  550, A106

\bibitem[{{Madden} {et~al.}(1993){Madden}, {Geis}, {Genzel}, {Herrmann},
  {Jackson}, {Poglitsch}, {Stacey}, \& {Townes}}]{1993ApJ...407..579M}
{Madden}, S.~C., {Geis}, N., {Genzel}, R., {et~al.} 1993, \apj, 407, 579

\bibitem[{{Matsuda} {et~al.}(2004){Matsuda}, {Yamada}, {Hayashino}, {Tamura},
  {Yamauchi}, {Ajiki}, {Fujita}, {Murayama}, {Nagao}, {Ohta}, {Okamura},
  {Ouchi}, {Shimasaku}, {Shioya}, \& {Taniguchi}}]{2004AJ....128..569M}
{Matsuda}, Y., {Yamada}, T., {Hayashino}, T., {et~al.} 2004, \aj, 128, 569

\bibitem[{{Matsuda} {et~al.}(2005){Matsuda}, {Yamada}, {Hayashino}, {Tamura},
  {Yamauchi}, {Murayama}, {Nagao}, {Ohta}, {Okamura}, {Ouchi}, {Shimasaku},
  {Shioya}, \& {Taniguchi}}]{2005ApJ...634L.125M}
---. 2005, \apjl, 634, L125

\bibitem[{{McLinden} {et~al.}(2013){McLinden}, {Malhotra}, {Rhoads}, {Hibon},
  {Weijmans}, \& {Tilvi}}]{2013ApJ...767...48M}
{McLinden}, E.~M., {Malhotra}, S., {Rhoads}, J.~E., {et~al.} 2013, \apj, 767,
  48

\bibitem[{{McMullin} {et~al.}(2007){McMullin}, {Waters}, {Schiebel}, {Young},
  \& {Golap}}]{2007ASPC..376..127M}
{McMullin}, J.~P., {Waters}, B., {Schiebel}, D., {Young}, W., \& {Golap}, K.
  2007, in Astronomical Society of the Pacific Conference Series, Vol. 376,
  Astronomical Data Analysis Software and Systems XVI, ed. R.~A. {Shaw},
  F.~{Hill}, \& D.~J. {Bell}, 127

\bibitem[{{Mori} \& {Umemura}(2006)}]{2006Natur.440..644M}
{Mori}, M., \& {Umemura}, M. 2006, \nat, 440, 644

\bibitem[{{Nagao} {et~al.}(2012){Nagao}, {Maiolino}, {De Breuck}, {Caselli},
  {Hatsukade}, \& {Saigo}}]{2012A&A...542L..34N}
{Nagao}, T., {Maiolino}, R., {De Breuck}, C., {et~al.} 2012, \aap, 542, L34

\bibitem[{{Nagao} {et~al.}(2011){Nagao}, {Maiolino}, {Marconi}, \&
  {Matsuhara}}]{2011A&A...526A.149N}
{Nagao}, T., {Maiolino}, R., {Marconi}, A., \& {Matsuhara}, H. 2011, \aap, 526,
  A149

\bibitem[{{Oberst} {et~al.}(2006){Oberst}, {Parshley}, {Stacey}, {Nikola},
  {L{\"o}hr}, {Harnett}, {Tothill}, {Lane}, {Stark}, \&
  {Tucker}}]{2006ApJ...652L.125O}
{Oberst}, T.~E., {Parshley}, S.~C., {Stacey}, G.~J., {et~al.} 2006, \apjl, 652,
  L125

\bibitem[{{Pavesi} {et~al.}(2016){Pavesi}, {Riechers}, {Capak}, {Carilli},
  {Sharon}, {Stacey}, {Karim}, {Scoville}, \& {Smolcic}}]{2016arXiv160702520P}
{Pavesi}, R., {Riechers}, D.~A., {Capak}, P.~L., {et~al.} 2016, ArXiv e-prints,
  arXiv:1607.02520

\bibitem[{{Rawle} {et~al.}(2014){Rawle}, {Egami}, {Bussmann}, {Gurwell},
  {Ivison}, {Boone}, {Combes}, {Danielson}, {Rex}, {Richard}, {Smail},
  {Swinbank}, {Altieri}, {Blain}, {Clement}, {Dessauges-Zavadsky}, {Edge},
  {Fazio}, {Jones}, {Kneib}, {Omont}, {P{\'e}rez-Gonz{\'a}lez}, {Schaerer},
  {Valtchanov}, {van der Werf}, {Walth}, {Zamojski}, \&
  {Zemcov}}]{2014ApJ...783...59R}
{Rawle}, T.~D., {Egami}, E., {Bussmann}, R.~S., {et~al.} 2014, \apj, 783, 59

\bibitem[{{Riechers} {et~al.}(2014){Riechers}, {Carilli}, {Capak}, {Scoville},
  {Smol{\v c}i{\'c}}, {Schinnerer}, {Yun}, {Cox}, {Bertoldi}, {Karim}, \&
  {Yan}}]{2014ApJ...796...84R}
{Riechers}, D.~A., {Carilli}, C.~L., {Capak}, P.~L., {et~al.} 2014, \apj, 796,
  84

\bibitem[{{Simpson} {et~al.}(2015){Simpson}, {Smail}, {Swinbank}, {Almaini},
  {Blain}, {Bremer}, {Chapman}, {Chen}, {Conselice}, {Coppin}, {Danielson},
  {Dunlop}, {Edge}, {Farrah}, {Geach}, {Hartley}, {Ivison}, {Karim}, {Lani},
  {Ma}, {Meijerink}, {Micha{\l}owski}, {Mortlock}, {Scott}, {Simpson},
  {Spaans}, {Thomson}, {van Kampen}, \& {van der Werf}}]{2015ApJ...799...81S}
{Simpson}, J.~M., {Smail}, I., {Swinbank}, A.~M., {et~al.} 2015, \apj, 799, 81

\bibitem[{{Stacey} {et~al.}(1991){Stacey}, {Geis}, {Genzel}, {Lugten},
  {Poglitsch}, {Sternberg}, \& {Townes}}]{1991ApJ...373..423S}
{Stacey}, G.~J., {Geis}, N., {Genzel}, R., {et~al.} 1991, \apj, 373, 423

\bibitem[{{Stacey} {et~al.}(2010){Stacey}, {Hailey-Dunsheath}, {Ferkinhoff},
  {Nikola}, {Parshley}, {Benford}, {Staguhn}, \&
  {Fiolet}}]{2010ApJ...724..957S}
{Stacey}, G.~J., {Hailey-Dunsheath}, S., {Ferkinhoff}, C., {et~al.} 2010, \apj,
  724, 957

\bibitem[{{Steidel} {et~al.}(2000){Steidel}, {Adelberger}, {Shapley},
  {Pettini}, {Dickinson}, \& {Giavalisco}}]{2000ApJ...532..170S}
{Steidel}, C.~C., {Adelberger}, K.~L., {Shapley}, A.~E., {et~al.} 2000, \apj,
  532, 170

\bibitem[{{Steidel} {et~al.}(2003){Steidel}, {Adelberger}, {Shapley},
  {Pettini}, {Dickinson}, \& {Giavalisco}}]{2003ApJ...592..728S}
---. 2003, \apj, 592, 728

\bibitem[{{Swinbank} {et~al.}(2014){Swinbank}, {Simpson}, {Smail}, {Harrison},
  {Hodge}, {Karim}, {Walter}, {Alexander}, {Brandt}, {de Breuck}, {da Cunha},
  {Chapman}, {Coppin}, {Danielson}, {Dannerbauer}, {Decarli}, {Greve},
  {Ivison}, {Knudsen}, {Lagos}, {Schinnerer}, {Thomson}, {Wardlow}, {Wei{\ss}},
  \& {van der Werf}}]{2014MNRAS.438.1267S}
{Swinbank}, A.~M., {Simpson}, J.~M., {Smail}, I., {et~al.} 2014, \mnras, 438,
  1267

\bibitem[{{Taniguchi} \& {Shioya}(2000)}]{2000ApJ...532L..13T}
{Taniguchi}, Y., \& {Shioya}, Y. 2000, \apjl, 532, L13

\bibitem[{{Uchimoto} {et~al.}(2012){Uchimoto}, {Yamada}, {Kajisawa}, {Kubo},
  {Ichikawa}, {Matsuda}, {Akiyama}, {Hayashino}, {Konishi}, {Nishimura},
  {Omata}, {Suzuki}, {Tanaka}, {Tokoku}, \& {Yoshikawa}}]{2012ApJ...750..116U}
{Uchimoto}, Y.~K., {Yamada}, T., {Kajisawa}, M., {et~al.} 2012, \apj, 750, 116

\bibitem[{{Umehata} {et~al.}(2015){Umehata}, {Tamura}, {Kohno}, {Ivison},
  {Alexander}, {Geach}, {Hatsukade}, {Hughes}, {Ikarashi}, {Kato}, {Izumi},
  {Kawabe}, {Kubo}, {Lee}, {Lehmer}, {Makiya}, {Matsuda}, {Nakanishi}, {Saito},
  {Smail}, {Yamada}, {Yamaguchi}, \& {Yun}}]{2015ApJ...815L...8U}
{Umehata}, H., {Tamura}, Y., {Kohno}, K., {et~al.} 2015, \apjl, 815, L8

\bibitem[{{Umehata} {et~al.}(2016){Umehata}, {Tamura}, {Kohno}, {Ivison},
  {Smail}, {Hatsukade}, {Nakanishi}, {Kato}, {Ikarashi}, {Matsuda}, {Fujimoto},
  {Iono}, {Lee}, {Steidel}, {Saito}, {Alexander}, {Yun}, \&
  {Kubo}}]{2016arXiv161109857U}
---. 2016, ArXiv e-prints, arXiv:1611.09857

\bibitem[{{Yang} {et~al.}(2009){Yang}, {Zabludoff}, {Tremonti}, {Eisenstein},
  \& {Dav{\'e}}}]{2009ApJ...693.1579Y}
{Yang}, Y., {Zabludoff}, A., {Tremonti}, C., {Eisenstein}, D., \& {Dav{\'e}},
  R. 2009, \apj, 693, 1579

\bibitem[{{Zhao} {et~al.}(2016){Zhao}, {Lu}, {Xu}, {Gao}, {Lord},
  {Charmandaris}, {Diaz-Santos}, {Evans}, {Howell}, {Petric}, {van der Werf},
  \& {Sanders}}]{2016ApJ...819...69Z}
{Zhao}, Y., {Lu}, N., {Xu}, C.~K., {et~al.} 2016, \apj, 819, 69

\end{thebibliography}

%% Tables may also be prepared as separate files. See the accompanying
%% sample file table.tex for an example of an external table file.
%% To include an external file in your main document, use the \input
%% command. Uncomment the line below to include table.tex in this
%% sample file. (Note that you will need to comment out the \documentclass,
%% \begin{document}, and \end{document} commands from table.tex if you want
%% to include it in this document.)

%% \begin{table}[h]
\centering
\begin{tabular}{l|ll}
$k \setminus j$ & $2$ & $4$ \\
\hline
$2$ & $q^{-2}$ &  \\
$-2$ & $q^{2}$ & $1$ \\
\end{tabular}
\caption*{$3_1$}
\end{table}

%% The following command ends your manuscript. LaTeX will ignore any text
%% that appears after it.

%%Table



\end{document}

%%
%% End of file `sample.tex'.
