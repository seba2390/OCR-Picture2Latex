%done
\subsection{Image retrieval comparison} 
\label{image retrieval conparision}

            \begin{table*}[!t]
    \renewcommand{\thetable}{4}
    \caption{\emph{\textbf{Comparison of image retrieval results of DINO-Mix with other methods in difficult cases (Top-1).}The green and red boxes in the table represent image retrieval success and failure, respectively, and the yellow box represents that the image content should be correct but the localization distance exceeds the threshold s.}}
    \centering
    \begin{tabular}{c p{2cm} c p{2cm} c p{2cm} c p{2cm} c p{2cm} c p{2cm} c p{2cm}}
    \hline
    Category&Query&DINO-Mix&MixVPR&NetVLAD&ConvAP&CosPlace\\
    \hline
    Viewpoint Change&\includegraphics[width=0.1\textwidth]{Pics/q1.jpg}& \includegraphics[width=0.1\textwidth]{Pics/1-1.png}&\includegraphics[width=0.1\textwidth]{Pics/1-2.png}&\includegraphics[width=0.1\textwidth]{Pics/1-3.png}&\includegraphics[width=0.1\textwidth]{Pics/1-4.png}&\includegraphics[width=0.1\textwidth]{Pics/1-5.png} \\
     &\includegraphics[width=0.09\textwidth]{Pics/q2.jpg}& \includegraphics[width=0.1\textwidth]{Pics/2-1.png}&\includegraphics[width=0.1\textwidth]{Pics/2-2.png}&\includegraphics[width=0.1\textwidth]{Pics/2-3.png}&\includegraphics[width=0.1\textwidth]{Pics/2-4.png}&\includegraphics[width=0.1\textwidth]{Pics/2-2.png} \\
     \hline
     &\includegraphics[width=0.07\textwidth]{Pics/q3.jpg}& \includegraphics[width=0.1\textwidth]{Pics/3-1.png}&\includegraphics[width=0.1\textwidth]{Pics/3-2.png}&\includegraphics[width=0.1\textwidth]{Pics/3-3.png}&\includegraphics[width=0.1\textwidth]{Pics/3-4.png}&\includegraphics[width=0.1\textwidth]{Pics/3-5.png} \\
    Illumination Change&\includegraphics[width=0.07\textwidth]{Pics/q4.jpg}& \includegraphics[width=0.1\textwidth]{Pics/4-1.png}&\includegraphics[width=0.1\textwidth]{Pics/4-2.png}&\includegraphics[width=0.1\textwidth]{Pics/4-3.png}&\includegraphics[width=0.1\textwidth]{Pics/4-4.png}&\includegraphics[width=0.1\textwidth]{Pics/4-3.png} \\
    &\includegraphics[width=0.1\textwidth]{Pics/q5.jpg}& \includegraphics[width=0.1\textwidth]{Pics/5-1.png}&\includegraphics[width=0.1\textwidth]{Pics/5-2.png}&\includegraphics[width=0.1\textwidth]{Pics/5-3.png}&\includegraphics[width=0.1\textwidth]{Pics/5-4.png}&\includegraphics[width=0.1\textwidth]{Pics/5-5.png} \\
    &\includegraphics[width=0.08\textwidth]{Pics/q6.jpg}& \includegraphics[width=0.1\textwidth]{Pics/6-1.png}&\includegraphics[width=0.1\textwidth]{Pics/6-2.png}&\includegraphics[width=0.1\textwidth]{Pics/6-3.png}&\includegraphics[width=0.1\textwidth]{Pics/6-4.png}&\includegraphics[width=0.1\textwidth]{Pics/6-5.png} \\
    \hline
    Occlusions&\includegraphics[width=0.09\textwidth]{Pics/q7.jpg}& \includegraphics[width=0.1\textwidth]{Pics/7-1.png}&\includegraphics[width=0.1\textwidth]{Pics/7-2.png}&\includegraphics[width=0.1\textwidth]{Pics/7-3.png}&\includegraphics[width=0.1\textwidth]{Pics/7-4.png}&\includegraphics[width=0.1\textwidth]{Pics/7-5.png} \\
    
   &\includegraphics[width=0.09\textwidth]{Pics/q8.jpg}& \includegraphics[width=0.1\textwidth]{Pics/8-1.png}&\includegraphics[width=0.1\textwidth]{Pics/8-2.png}&\includegraphics[width=0.1\textwidth]{Pics/8-3.png}&\includegraphics[width=0.1\textwidth]{Pics/8-4.png}&\includegraphics[width=0.1\textwidth]{Pics/8-5.png} \\
    
    &\includegraphics[width=0.1\textwidth]{Pics/q9.jpg}& \includegraphics[width=0.1\textwidth]{Pics/9-1.png}&\includegraphics[width=0.1\textwidth]{Pics/9-2.png}&\includegraphics[width=0.1\textwidth]{Pics/9-3.png}&\includegraphics[width=0.1\textwidth]{Pics/9-4.png}&\includegraphics[width=0.1\textwidth]{Pics/9-5.png} \\
    \hline
    Season Change &\includegraphics[width=0.1\textwidth]{Pics/q10.jpg}& \includegraphics[width=0.1\textwidth]{Pics/10-1.png}&\includegraphics[width=0.1\textwidth]{Pics/10-2.png}&\includegraphics[width=0.1\textwidth]{Pics/10-3.png}&\includegraphics[width=0.1\textwidth]{Pics/10-4.png}&\includegraphics[width=0.1\textwidth]{Pics/10-5.png} \\
    
    &\includegraphics[width=0.1\textwidth]{Pics/q11.jpg}& \includegraphics[width=0.1\textwidth]{Pics/11-1.png}&\includegraphics[width=0.1\textwidth]{Pics/11-2.png}&\includegraphics[width=0.1\textwidth]{Pics/11-3.png}&\includegraphics[width=0.1\textwidth]{Pics/11-4.png}&\includegraphics[width=0.1\textwidth]{Pics/11-5.png} \\
    \hline
    \end{tabular}
    \label{tab:Comparison of image retrieval}
    \end{table*}
    
    %done
    In this study, we compared the performance of DINO-Mix with those of existing SOTA methods, including MixVPR, NetVLAD, ConvAP, and CosPlace, in image retrieval tasks. To demonstrate the robustness of DINO-Mix for VPR in complex environments, we selected several representative image retrieval cases from the Tokyo24/7, SF-XL-Testv1, and Nordland datasets. We presented four challenging scenarios: viewpoint changes, illumination changes, object occlusions, and seasonal variations. In cases where DINO-Mix succeeded, the other methods failed to accurately locate the query image, as displayed in Tab.\ref{tab:Comparison of image retrieval}.
    
    %done
    \textbf{Viewpoint Change:} Viewpoint changes encompass variations in the field angle and field range, posing challenges for image retrieval. Rows 1 and 2 in Tab.\ref{tab:Comparison of image retrieval} show examples of viewpoint changes in terms of field angle and field range. Notably, only DINO-Mix resists interference caused by viewpoint changes and retrieves the correct image, whereas the other methods retrieve similar buildings or scenes.

    %done
    \textbf{Illumination Change:} Illumination changes significantly affect image retrieval accuracy. Dim lighting conditions can blur textures in images, adversely affecting feature extraction and, consequently, image retrieval accuracy. Rows 3 and 4 in Tab.\ref{tab:Comparison of image retrieval} depict the image retrieval cases under dark conditions. Rows 5 and 6 present nighttime scenarios with artificial and natural light variations, respectively. DINO-Mix exhibits strong robustness against illumination changes, whereas the other methods suffer from the effects of lighting variations and fail to retrieve accurate results.

    \begin{table*}[!t]
    \renewcommand{\thetable}{5}
    \caption{\emph{\textbf{The attention map visualization of the query images.}}}
    \centering
    \begin{tabular}{ c p{2cm} c p{2cm} c p{2cm} c p{2cm} c p{2cm} c p{2cm}}
    \hline
    Query&DINO-Mix&MixVPR&NetVLAD&ConvAP&CosPlace\\
    \hline
 \includegraphics[width=0.1\textwidth]{Pics/q1.jpg}& \includegraphics[width=0.1\textwidth]{Pics/g1-1.jpg}&\includegraphics[width=0.1\textwidth]{Pics/g1-2.jpg}&\includegraphics[width=0.1\textwidth]{Pics/g1-3.jpg}&\includegraphics[width=0.1\textwidth]{Pics/g1-4.jpg}&\includegraphics[width=0.1\textwidth]{Pics/g1-5.jpg} \\
 \includegraphics[width=0.06\textwidth]{Pics/q2.jpg}& \includegraphics[width=0.1\textwidth]{Pics/g2-1.jpg}&\includegraphics[width=0.1\textwidth]{Pics/g2-2.jpg}&\includegraphics[width=0.1\textwidth]{Pics/g2-3.jpg}&\includegraphics[width=0.1\textwidth]{Pics/g2-4.jpg}&\includegraphics[width=0.1\textwidth]{Pics/g2-5.jpg} \\
 \includegraphics[width=0.073\textwidth]{Pics/q3.jpg}& \includegraphics[width=0.1\textwidth]{Pics/g3-1.jpg}&\includegraphics[width=0.1\textwidth]{Pics/g3-2.jpg}&\includegraphics[width=0.1\textwidth]{Pics/g3-3.jpg}&\includegraphics[width=0.1\textwidth]{Pics/g3-4.jpg}&\includegraphics[width=0.1\textwidth]{Pics/g3-5.jpg} \\
 \includegraphics[width=0.073\textwidth]{Pics/q4.jpg}& \includegraphics[width=0.1\textwidth]{Pics/g4-1.jpg}&\includegraphics[width=0.1\textwidth]{Pics/g4-2.jpg}&\includegraphics[width=0.1\textwidth]{Pics/g4-3.jpg}&\includegraphics[width=0.1\textwidth]{Pics/g4-4.jpg}&\includegraphics[width=0.1\textwidth]{Pics/g4-5.jpg} \\
 \includegraphics[width=0.1\textwidth]{Pics/q5.jpg}& \includegraphics[width=0.1\textwidth]{Pics/g5-1.jpg}&\includegraphics[width=0.1\textwidth]{Pics/g5-2.jpg}&\includegraphics[width=0.1\textwidth]{Pics/g5-3.jpg}&\includegraphics[width=0.1\textwidth]{Pics/g5-4.jpg}&\includegraphics[width=0.1\textwidth]{Pics/g5-5.jpg} \\
 \includegraphics[width=0.07\textwidth]{Pics/q6.jpg}& \includegraphics[width=0.1\textwidth]{Pics/g6-1.jpg}&\includegraphics[width=0.1\textwidth]{Pics/g6-2.jpg}&\includegraphics[width=0.1\textwidth]{Pics/g6-3.jpg}&\includegraphics[width=0.1\textwidth]{Pics/g6-4.jpg}&\includegraphics[width=0.1\textwidth]{Pics/g6-5.jpg} \\
 \includegraphics[width=0.07\textwidth]{Pics/q7.jpg}& \includegraphics[width=0.1\textwidth]{Pics/g7-1.jpg}&\includegraphics[width=0.1\textwidth]{Pics/g7-2.jpg}&\includegraphics[width=0.1\textwidth]{Pics/g7-3.jpg}&\includegraphics[width=0.1\textwidth]{Pics/g7-4.jpg}&\includegraphics[width=0.1\textwidth]{Pics/g7-5.jpg} \\
 \includegraphics[width=0.07\textwidth]{Pics/q8.jpg}& \includegraphics[width=0.1\textwidth]{Pics/g8-1.jpg}&\includegraphics[width=0.1\textwidth]{Pics/g8-2.jpg}&\includegraphics[width=0.1\textwidth]{Pics/g8-3.jpg}&\includegraphics[width=0.1\textwidth]{Pics/g8-4.jpg}&\includegraphics[width=0.1\textwidth]{Pics/g8-5.jpg} \\
 \includegraphics[width=0.1\textwidth]{Pics/q9.jpg}& \includegraphics[width=0.1\textwidth]{Pics/g9-1.jpg}&\includegraphics[width=0.1\textwidth]{Pics/g9-2.jpg}&\includegraphics[width=0.1\textwidth]{Pics/g9-3.jpg}&\includegraphics[width=0.1\textwidth]{Pics/g9-4.jpg}&\includegraphics[width=0.1\textwidth]{Pics/g9-5.jpg} \\
 \includegraphics[width=0.1\textwidth]{Pics/q10.jpg}& \includegraphics[width=0.1\textwidth]{Pics/g10-1.jpg}&\includegraphics[width=0.1\textwidth]{Pics/g10-2.jpg}&\includegraphics[width=0.1\textwidth]{Pics/g10-3.jpg}&\includegraphics[width=0.1\textwidth]{Pics/g10-4.jpg}&\includegraphics[width=0.1\textwidth]{Pics/g10-5.jpg} \\
 \includegraphics[width=0.1\textwidth]{Pics/q11.jpg}& \includegraphics[width=0.1\textwidth]{Pics/g11-1.jpg}&\includegraphics[width=0.1\textwidth]{Pics/g11-2.jpg}&\includegraphics[width=0.1\textwidth]{Pics/g11-3.jpg}&\includegraphics[width=0.1\textwidth]{Pics/g11-4.jpg}&\includegraphics[width=0.1\textwidth]{Pics/g11-5.jpg} \\
    \hline
    \end{tabular}
    \label{tab:attention map visualization}
    \end{table*}
    
    %done
    \textbf{Occlusion:} Image retrieval focuses primarily on objects, such as buildings, facilities, and natural landscapes. However, pedestrians, vehicles, and other objects can interfere with the semantic information in an image, posing challenges for image retrieval. As shown in rows 7 and 8 in Tab.\ref{tab:Comparison of image retrieval}, where a large number of pedestrians are present in the query images, and in row 9, where the influence of buildings is significant, these occlusions pose significant difficulties for image retrieval. MixVPR retrieved the correct content but exceeded the threshold $s$ ($25$ $m$) in the localization results. In contrast, DINO-Mix successfully extracted the correct features from the images and retrieved accurate results despite these challenges.
    
    %done
    \textbf{Season Change:} The appearance characteristics of a location undergo significant changes in different seasons, such as heavy snowfall in winter (as illustrated in row 10 of Tab.\ref{tab:Comparison of image retrieval}), and leaves falling from trees (row 11). These seasonal variations also have a profound impact on the image retrieval accuracy. Under such challenging circumstances, DINO-Mix overcame the drastic contrast caused by seasonal changes and achieved satisfactory results.

%
 \subsection{Attention map visualization} 
 \label{attention map visualization}


    
    % done
    To provide a more intuitive demonstration of the superiority of DINO-Mix over other VPR methods, we visualized their attention maps as presented in \ref{tab:attention map visualization}. The attention scores are represented by varying colors from blue to green to red, indicating low to high attention levels. Our analysis reveals that DINO-Mix can focus more on buildings, object contours, and textures, which are crucial factors for image retrieval. In contrast, it effectively excludes negative elements such as pedestrians, cars, and occlusions. This suggests that DINO-Mix has a greater ability to capture essential features and extract more robust image representations.