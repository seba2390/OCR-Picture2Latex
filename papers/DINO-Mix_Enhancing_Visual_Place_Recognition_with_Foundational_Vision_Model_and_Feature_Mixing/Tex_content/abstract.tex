% done
In the vast expanse of cyberspace, a plethora of publicly available images exist. Utilizing visual place recognition (VPR) technology to ascertain the geographical location of publicly available images is a pressing issue for real-world VPR applications. Although most current VPR methods achieve favorable results under ideal conditions, their performance in complex environments, characterized by lighting variations, seasonal changes, and occlusions caused by moving objects, is generally unsatisfactory. Therefore, obtaining efficient and robust image feature descriptors even in complex environments is a pressing issue in VPR applications. In this study, we utilize the DINOv2 model as the backbone network for trimming and fine-tuning to extract robust image features. We propose a novel VPR architecture called DINO-Mix, which combines a foundational vision model with feature aggregation. This architecture relies on the powerful image feature extraction capabilities of foundational vision models. We employ an MLP-Mixer-based mix module to aggregate image features, resulting in globally robust and generalizable descriptors that enable high-precision VPR. We experimentally demonstrate that the proposed DINO-Mix architecture significantly outperforms current state-of-the-art (SOTA) methods. In test sets having lighting variations, seasonal changes, and occlusions (Tokyo24/7, Nordland, SF-XL-Testv1), our proposed DINO-Mix architecture achieved Top-1 accuracy rates of $91.75\%$, $80.18\%$, and $82\%$, respectively. Compared with SOTA methods, our architecture exhibited an average accuracy improvement of $5.14\%$. To further evaluate the performance of DINO-Mix, we compared it with other SOTA methods using representative image retrieval case studies. Our analysis revealed that DINO-Mix outperforms its competitors in terms of VPR performance. Furthermore, we visualized the attention maps of DINO-Mix and other methods to provide a more intuitive understanding of their respective strengths. These visualizations serve as compelling evidence of the superiority of the DINO-Mix framework in this domain. Code is available at \url{https://github.com/GaoShuang98/DINO-Mix}.
