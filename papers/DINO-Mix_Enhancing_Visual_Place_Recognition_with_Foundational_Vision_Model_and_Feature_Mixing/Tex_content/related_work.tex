%
By retrieving the most similar image from the image database, the geographical location of the retrieved image can be used as the location of the target image~\cite{masone_survey_2021}. In recent years, numerous researchers have made significant contributions to the field of image retrieval for VPR. The features used in image retrieval can be broadly categorized into handcrafted and deep features. Zhang and Kosecka~\cite{zhang_image_2007} first extracted scale-invariant feature transform (SIFT) features from images to establish an image feature database. They performed a brute-force global search of the database and validated and ranked the top five candidate images using the random sample consensus (RANSAC)~\cite{martin_a_fischler_random_1981} algorithm. Finally, the geographical location of the target image was obtained by triangulating the top three images. Zamir and Shah~\cite{zamir_accurate_2010} extracted SIFT feature vectors from images to build a database and employed a nearest-neighbor tree search to improve the retrieval efficiency. Zamir and Shah~\cite{zamir_gps-tag_2014} further improved the nearest-neighbor matching technique by pruning outliers, and applied the generalized minimum clique problem (GMCP) in conjunction with approximate feature matching. This resulted in a $5\%$ improvement in the localization accuracy compared to their previous work~\cite{zamir_accurate_2010}. The advantages of using handcrafted features for VPR are their simplicity and strong interpretability. However, these methods tend to have high redundancy, require dimensionality reduction, are susceptible to environmental changes, and generally have low accuracy.

%
Deep features are extracted by neural networks with modules such as convolutional layers and attention mechanisms. These features often outperform handcrafted features owing to their strong expressive power, ability to freely define feature dimensions, and flexibility in designing neural network frameworks. Noh et al.~\cite{noh_large-scale_2017} proposed a deep local feature (DELF) descriptor and an attention mechanism for keypoint selection to identify semantic local features. Ng et al.~\cite{ng_solar_2020} introduced a global descriptor called Second-Order Loss and Attention for image Retrieval (SOLAR) that utilizes spatial attention and descriptor similarity to perform large-scale image retrieval using second-order information. Chu et al.~\cite{chu_grid_2020} constructed a CNN to extract dense features, embedded an attention module within the network to score features, and proposed a grid feature point selection (GFS) method to reduce the number of image features. Chu et al.~\cite{chu_street_2020} combined deep features with handcrafted features, extracted average pooling features from the intermediate layers of a CNN for retrieval on street-view datasets, and used SIFT to re-rank them. Yan~\cite{yan_hierarchical_2021} extracted hierarchical feature maps from CNNs and organically fused them for image feature representation. Chu et al.~\cite{chu_news_2022} employed a CNN with a HOW module~\cite{tolias_learning_2020} to extract local image features, aggregated them into a feature vector using VLAD, used the aggregated selective match kernel (ASMK), and estimated the geographical location of the query image using kernel density prediction (KDP).

%
To address environmental factors, Mishkin et al.~\cite{mishkin_place_2015} employed a BoW method with multiple detectors, descriptors, and adaptive thresholds. Relja et al.~\cite{relja_netvlad_2018} designed a trainable NetVLAD layer inspired by VLAD, which provides a pooling mechanism that can be integrated into other CNN structures. In addition, variants of NetVLAD have been proposed, such as CRN~\cite{kim_learned_2017}, SPE-VLAD~\cite{yu_spatial_2020}, MultiRes-NetVLAD~\cite{khaliq_multires-netvlad_2022}, SARE~\cite{liu_stochastic_2019}, and SFRS~\cite{ge_self-supervising_2020}. Ali-bey et al. proposed ConvAP~\cite{ali-bey_gsv-cities_2022}, which combines 1×1 convolutions with adaptive mean pooling to encode local features.
