
\section{Proofs}
\label{sec:proofs}
Our proof strategy is inspired by the paper \cite{doukhan2007probability}, which is based on \cite{bentkus:1980} (see also \cite{saulis:statulevicius:1991}). Before proceeding to the proof, we first introduce some  notation and definitions.

\subsection{Cumulants and central moments}
\label{sec:cumul-centr-moments}
We begin with the definitions of cumulants and central moments of a random vector, which play an essential role in the proofs. We present these notions in a general framework. Let $(\Omega,\mcf,\PP)$ be a probability space and $W=(W_1,\dots,W_n)$ be an $n$-dimensional random vector. For any  subset $I =(i_1 , \ldots , i_k) \in \{1,\ldots,n\}^k$, $W_I$ stands for the $k$-dimensional random variable $(W_{i_1},\ldots,W_{i_k})$. Note that it is possible that some of the indices $i_j$ coincides.

\paragraph{Cumulants} Recall that the characteristic function of random vector $W$ is defined for $u \in \rset^n$ as $\varphi_W(u) = \expeLigne{\rme^{\rmi \ps{u}{W}}}$. Let $\bnu= (\nu_1,\dots, \nu_n) \in \nset^n$. Assuming that $\PE[\prod_{i=1}^n |W_i|^{\nu_i}] < \infty$, define the mixed cumulant of $W$ as
\[
\Gamma^{(\bnu)}(W)=\left.\frac{1}{\rmi^{|\bnu|}} \frac{\partial^{|\bnu|}}{\partial u_{1}^{\nu_{1}} \ldots \partial u_{n}^{\nu_{n}}} \ln \varphi_W(u) \right|_{u=0} \eqsp,
\]
where $|\bnu| = \nu_1 + \dots + \nu_n$. If $\nu_1=\dots=\nu_n=1$, we simply write $\Gamma(W)= \Gamma^{(1,\dots,1)}(W)$.
Note that for any $n$-dimensional random vector $W$, collection of indices $I =(i_1,\ldots,i_k) \in \{1,\ldots,n\}^k$ and permutation $\upsigma : I \to I$,
\begin{equation}
\label{eq:inv_permuation_cumulant}
\Gamma(W_I) = \Gamma(W_{\upsigma(I)}) \eqsp.
\end{equation}
\paragraph{Centered moments} Assume that $\PE[|W_\ell|^n] < \infty$ for $\ell \in \{1,\dots,n\}$. Set $Z_{n+1} = 1$ and define $Z_{\ell}$, for $\ell= n, \dots, 2$ by the backward recursion
 $Z_{{\ell}} = W_{\ell} Z_{{\ell+1}} - \PE[ W_{\ell} Z_{{\ell+1}}]$.
The \emph{centered moment} of $W$ is then  defined by
\begin{equation}
\label{eq:def_centered_moment}
\PEC[W] = \PEC[W_1,\dots,W_n] = \PE[W_1 Z_{2}] \eqsp.
\end{equation}
Note that $\PEC[W]$ is a scalar. Moreover, contrary to the cumulants, the centered moment of $W$ is not invariant by permutation of its component.

Let $I =(i_1,\dots,i_k) \in \{1,\ldots,n\}^k$ be an ordered subset, satisfying $i_1 \leq \dots \leq i_{k}$. Then \cite[Lemma~3]{Statul1970} or \cite[Lemma~1.1]{saulis:statulevicius:1991} allows to express the cumulant $\Gamma(W_I)$  in terms of centered moments:
\begin{equation}
\label{eq:cumulantsviamoments}
\Gamma(W_I) = \sum_{r=1}^k (-1)^{r-1} \sum\nolimits_{\bigcup_{p=1}^r I_p = I} N_r(I_1, \ldots, I_r) \prod_{p=1}^r \PEC[W_{I_p} ]\eqsp.
\end{equation}
In the formula above $N_r(I_1,\dots,I_r)$ are non-negative integers defined in \cite[Appendix~2]{saulis:statulevicius:1991} and $\sum_{\bigcup_{p=1}^r I_p = I}$ denotes the summation over all the sets $ \{I_1,\ldots,I_r\}$  such that there exists a partition $J_1,\ldots,J_r$ of $\{1,\ldots,k\}$ satisfying for any $\ell \in \{1,\ldots,r\}$, $I_\ell = (i_{j_1},\ldots,i_{j_{n_\ell}})$ with $J_{\ell} = \{j_1,\ldots,j_{n_{\ell}} \}$, $j_1 < \ldots < j_{n_{\ell}}$. It is shown in  \cite[Eq. 4.43]{saulis:statulevicius:1991}
that
\begin{equation}
\label{eq:formula-summation}
\sum_{r=1}^k \sum\nolimits_{\bigcup_{p=1}^r I_p = I} N_r(I_1, \ldots, I_r)  = (k-1)! \eqsp.
\end{equation}
We now specify these definitions to Markov chain to derive  an expression for the $q$-th moment of the random variables $S_n$ defined by \eqref{eq:def_S_n_g} for an integer $n \geq 1$ and a measurable function $g$, integrable with respect to $\pi$. This expression is the cornerstone of our approach. Define
\begin{equation}
\label{eq:def_Y_ell}
\bar{g} = g - \pi(g) \eqsp, \quad   Y_{\ell} = \bar{g}(X_\ell) \eqsp,
\end{equation}
so that $S_n = \sum_{\ell =0}^{n-1} Y_{\ell}$.
Assume that for $\ell \in \{0, \ldots, n-1\}$, $\PE_{\pi}[\abs{Y_{\ell}}^{2q}] < \infty$  so that $\PE_{\pi}[|S_n|^{2q}] < \infty$. For $u \in \{1,\ldots,q\}$ and $(k_1, \ldots , k_u) \in \{1, \ldots, 2q \}^u$, let $\bk_u = (k_1, \ldots, k_u)$, $|\bk_u| = \sum_{p=1}^u k_p$ and $\bk_u! = \prod_{p=1}^u k_p!$. Then, by the Leonov-Shiraev formula \cite[Eq.~1.53]{saulis:statulevicius:1991}
\begin{equation}
\label{eq:leonov-shiraev}
    \PE_\pi[|S_n|^{2q}] = \sum_{u=1}^{2q} \frac{1}{u!} \sum_{|\bk_u| = 2q} \frac{(2q)!}{{\boldsymbol k_u}!} \prod_{p=1}^u \Gamma_{\pi, k_p}(S_n) \eqsp,
\end{equation}
where $\Gamma_{\pi, k}(S_n)$ denotes the $k$-th order cumulant of $S_n$ under the stationary probability $\PP_\pi$. Using \cite[Eq.~1.47]{saulis:statulevicius:1991} for any $k \in \{1,\ldots,2q\}$, we may express $\Gamma_{\pi,k}(S_n)$ as the sum of cumulants over all the $k$-tuple $(Y_{t_1},\dots,Y_{t_k})$ under $\PP_\pi$: $\Gamma_{\pi,k}(S_n)=  \sum_{0 \le t_1, \ldots, t_k \le n-1} \Gamma_{\pi}(Y_{t_1}, \ldots, Y_{t_k})$ and using \eqref{eq:inv_permuation_cumulant}, we get
\begin{equation}
  \label{eq:expression_gamma_pi_k_S_n}
\Gamma_{\pi,k}(S_n)= (k!) \sum_{0 \le t_1\leq  \cdots \leq t_k \le n-1} \Gamma_{\pi}(Y_{t_1}, \ldots, Y_{t_k}) \eqsp.
\end{equation}
Since for $i \in \{1,\ldots,n-1\}$, $\Gamma_{\pi}(Y_\ell) = \PE_{\pi}[Y_\ell] = 0$, terms in \eqref{eq:leonov-shiraev} corresponding to indices $u \geq q+1$ vanishes. Moreover, the only non-zero term in the sum corresponding to $u = q$ corresponds to the multi-index $\boldsymbol k_q = (2,\ldots,2)$. Thus, with the definition of $\momentGq[q]$, the expression \eqref{eq:leonov-shiraev} simplifies to
\begin{equation}
\label{eq:qmomentexpansion}
\PE_\pi[|S_n|^{2q}] = \momentGq[q] \{ \PVar[\pi](S_n) \}^q + \sum_{u=1}^{q-1} \frac{1}{u!} \sum_{|\bk_u| = 2q} \frac{(2q)!}{{\boldsymbol k_u}!} \prod_{p=1}^u \Gamma_{\pi, k_p}(S_n) \eqsp.
\end{equation}
Now we aim to bound $\PE_\pi[|S_n|^{2q}]$ based on the expression \eqref{eq:qmomentexpansion}. To do it, we first establish  bounds on centered moments of the random vectors $(Y_{t_1},\ldots,Y_{t_k})$ with $(t_1,\ldots,t_k) \in\{0,\ldots,n-1\}^k$, and then deduce bounds for the $k$-the order cumulants $\Gamma_{\pi,k}(S_n)$ using the relations \eqref{eq:cumulantsviamoments} and \eqref{eq:expression_gamma_pi_k_S_n}.
\par 
The first step of the proof relies on some lemmas on centered moments associated with the sequence of measurable functions $\{h_{\ell}\}_{\ell=1}^{k}$ satisfying $\pi(|h_{\ell}|^{k})<\infty$. We consider the centered moments for 
\[
(h_1(X_{t_1}), \ldots, h_k(X_{t_k})), \quad (t_1,\dots,t_k) \in \{0,\dots,n-1\}^{k}, \quad t_1 \leq \dots \leq t_k\eqsp.
\]
Define the sequence $\{Z^h_{\ell}\}_{\ell=2}^{k+1}$:
 \begin{equation}
 \label{eq:z_ell_definition}
 Z^h_{k+1} = 1\eqsp, \quad Z^h_{\ell} = h_\ell(X_{t_\ell}) Z^h_{\ell+1} - \PE_\pi[ h_\ell(X_{t_\ell}) Z^h_{\ell+1}]\eqsp,\quad  \ell= 2, \dots, k\eqsp.
 \end{equation}
 The dependence of the sequence $\{Z^h_\ell\}_{\ell=2}^{k+1}$ on $\{h_\ell \}_{\ell=1}^k$ and $\{t_\ell\}_{\ell=1}^k$ is implicit.
\begin{lemma}
\label{lem:centred_moments_markov_property}
Assume that $\MK$ has a unique invariant distribution $\pi$. Let $k \in \nset$ and $\{h_i \}_{i=1}^k$ be a family of real measurable functions on $\Xset$
such that $\pi(|h_i|^k) < \infty$. Then, for any $(t_1,\dots,t_{k}) \in \{0,\ldots,n-1\}^{k}$, $t_1 \leq \dots\leq t_{k}$, it holds
\begin{equation*}
\PEC_\pi[h_1(X_{t_1}), \ldots, h_{k}(X_{t_{k}})] = \PEC_\pi[h_1(X_{t_1}), \ldots , h_{k-1}(X_{t_{k-1}}) \tilde h_{k}(X_{t_{k-1}})] \eqsp,
\end{equation*}
where $\tilde h_{k}(x) = \MK^{t_{k} - t_{k-1}} h_{k}(x) - \pi(h_{k})$.
\end{lemma}
\begin{proof}
  Let $(t_1,\dots,t_{k}) \in \{0,\ldots,n-1\}^{k}$, $t_1 \leq \dots\leq t_{k}$.
Set $\mff_{k-1} = \sigma\{X_{t_1}, \ldots, X_{t_{k-1}}\}$. Using the definition \eqref{eq:def_centered_moment} and the tower property, we obtain
\begin{equation}
  \label{eq:def_centered_moment_lem}
     \PEC_{\pi}[h_1(X_{t_1}), \ldots, h_{k}(X_{t_{k}}) ] = \PE_\pi[h_1(X_{t_1}) Z^h_{2}] =\PE_\pi[h_1(X_{t_1}) \PE[Z^h_{2}|\mff_{k-1}]] \eqsp.
\end{equation}
It remains to establish that 
\[
\PE_\pi[h_1(X_{t_1}) \PE[Z^h_{2}|\mff_{k-1}]] = \PEC_\pi[h_1(X_{t_1}), \ldots , h_{k-1}(X_{t_{k-1}}) \tilde h_{k}(X_{t_{k-1}})]\eqsp.
\]
Taking the conditional expectation with respect to $\mff_{k-1}$ in the definition \eqref{eq:z_ell_definition}  %\eqref{eq:def_Z_i_center_moment}
of $\{Z^h_{\ell}\}_{\ell=2}^{k+1}$, we get for any $\ell \in \{1, \ldots, k-2\}$,
 setting  $\tilde Z^h_{\ell+1} = \CPE[\pi]{Z^h_{\ell+1}}{\mathfrak F_{k-1}}$,
\begin{equation}
\label{eq: tilde z_0}
    \tilde Z^h_\ell  = h_\ell(X_{t_\ell}) \tilde Z^h_{\ell+1} - \PE_\pi[ h_\ell(X_{t_\ell}) \tilde Z^h_{\ell+1}] \eqsp.
\end{equation}
For $\ell = k-1$ taking into account that $Z^h_{k+1} = 1$ and $\PE[h_{k}(X_{t_{k}}) - \pi(h_k) | \mathfrak F_{k-1}] = \tilde h_k(X_{t_{k-1}})$,
\begin{align}
\nonumber
\tilde Z^h_{k-1} = \PE[Z^h_{k-1}|\mff_{k-1}] &= h_{k-1}(X_{t_{k-1}}) \CPE[]{h_{k}(X_{t_{k}}) - \pi(h_k)}{\mathfrak F_{k-1}} \\
\nonumber
& \phantom{h_{k-1}(X_{t_{k-1}}) xxxxxx}
- \PE_\pi[ h_{k-1}(X_{t_{k-1}}) \CPE[]{h_{k}(X_{t_{k}}) - \pi(h_k)}{\mathfrak F_{k-1}}] \\
& = h_{k-1}(X_{t_{k-1}}) \tilde h_{k}(X_{t_{k-1}})  - \PE_\pi[ h_{k-1}(X_{t_{k-1}}) \tilde h_{k}(X_{t_{k-1}})]
\label{eq: tilde z_1}
\eqsp.
\end{align}
% We finally set
% \begin{equation}
% \label{eq: tilde z_2}
% \tilde Z_k = 1 \eqsp.
% \end{equation}
By \eqref{eq: tilde z_0}-\eqref{eq: tilde z_1} and the definition of centred moments \eqref{eq:def_centered_moment}, we get setting $\tilde{Z}^h_k=1$, that
\begin{equation*}
\PEC_\pi[h_1(X_{t_1}), \ldots , h_{k-1}(X_{t_{k-1}}) \tilde h_{k}(X_{t_{k-1}})] = \PE_\pi[h_1(X_{t_1}) \tilde Z^h_{2}] \eqsp.
\end{equation*}
Combining this result with \eqref{eq:def_centered_moment_lem} completes the proof. %Equations~\eqref{eq:def_centered_moment_lem} and \eqref{eq: tilde z and z} with $\ell=1$ implies the statement.
\end{proof}

\subsection{Upper-bounding cumulants from central moments}
\label{sec:cumulants_upper_bound_geom_ergodicity}
Throughout this section we assume that $\MK$ has a unique invariant distribution $\pi$ and fix some integer $q \geq 1$. Let $\lyapW: \Xset \to \coint{1,\infty}$ be a measurable function and $\setfunction_{\lyapW}$ be a set of
measurable functions. Finally, let $\normlike_{\lyapW}$ be a non-negative functional defined on $\setfunction_{\lyapW}$. The objective is to compute a bound for $\Gamma_{\pi, k}(S_n)$ for $k \leq 2q$, where $S_n= \sum_{\ell=1}^n \bar{g}(X_\ell)$ and $g \in \setfunction_{\lyapW}$.

Starting from \eqref{eq:expression_gamma_pi_k_S_n}, we establish a bound on $\Gamma_\pi(Y_I)$ setting $Y_I = (Y_{t_1},\ldots,Y_{t_k})$ with $ (t_1, \ldots, t_k)\in \{0,\ldots,n-1\}^{k}$, $t_1 \leq \dots\leq t_{k}$.
To this end, we use \eqref{eq:cumulantsviamoments} and the following assumption.
\begin{assumptionW}[$q,\lyapW,\normlike$]
\label{assum:central_moments_bound}
There exist $\arate,\ConstD \geq 0$ $\rate \in \coint{0,1}$ such that for any $k \in \{2,\dots,2q\}$, $k-$tuple $I = (t_1,\dots,t_k) \in \{0,\dots,n-1\}^{k}$, $t_1 \leq \dots \leq t_k$, and any family of measurable functions $\{h_{\ell}\}_{\ell=1}^{k} \subset \setfunction_{\lyapW}$
\begin{equation}
\label{eq:centred_moments_generic_assumption}
|\PEC_\pi[h_1(X_{t_1}), \ldots, h_k(X_{t_k})]| \leq \ConstD^{k} (k!)^{\arate}\left\{\prod_{j=1}^{k} \normlike_{\lyapW}(h_j)\right\}\rho_{q,\lyapW}^{\maxgap(I)}\eqsp,
\end{equation}
where $\maxgap(I) := \max_{j \in \{1,\dots,k-1\}}[t_{j+1}-t_{j}]$
\end{assumptionW}
The proof of the following results can be adapted from \cite{doukhan2007probability}. 
\begin{lemma}
\label{lem:cumulant_bounds_generic}
Assume \Cref{assum:central_moments_bound}($q,\lyapW,\normlike$). Then, for any  $k \in \{2,\ldots,2q\}$,  $I=(t_1,\dots,t_{k}) \in \{0,\ldots,n-1\}^{k}$, $t_1 \leq \dots\leq t_{k}$, and function $g \in \setfunction_{\lyapW}$, it holds that
\begin{equation}
  |\Gamma_{\pi}(Y_I)| \leq \ConstD^{k} (k!)^{\arate} \bigl\{\normlike_{\lyapW}(\bar{g})\bigr\}^{k} \sum_{r=1}^{k}\sum_{\bigcup_{\ell=1}^r I_{\ell} = I} N_r(I_1, \ldots, I_r)    \rho_{q,\lyapW}^{\sum_{\ell=1}^{r}\maxgap(I_{\ell})}  \eqsp.
\end{equation}
\end{lemma}
\begin{proof}
Eq.~\eqref{eq:cumulantsviamoments}
%%%% Don't need to repeat eqref{eq:cumulantsviamoments}
implies that
\begin{equation}
  \label{eq:1:lem:cumlants_generic}
|\Gamma_{\pi}(Y_I)| \leq \sum_{r=1}^k \sum\nolimits_{\bigcup_{\ell=1}^r I_{\ell} = I} N_r(I_1, \ldots, I_r) \prod_{\ell=1}^r \bigl|\PEC_{\pi}[Y_{I_{\ell}} ]\bigr|\eqsp.
\end{equation}
Using \cref{assum:central_moments_bound}($q,\lyapW,\normlike$), we obtain
$|\PEC_\pi[Y_{I_{\ell}}] | \leq  \ConstD^{\card{I_{\ell}}} (\card{I_{\ell}}!)^{\arate} \rate^{\maxgap(I_{\ell})}\{\normlike_{\lyapW}(\bar{g})\}^{\card{I_\ell}}$.
The proof is completed by plugging this bound in \eqref{eq:1:lem:cumlants_generic} and using $\prod_{\ell=1}^r \card{I_\ell}! \leq k!$.
\end{proof}

\begin{lemma}
\label{lem:cumulant_bounds_final_generic}
Assume \Cref{assum:central_moments_bound}$(q,\lyapW,\normlike)$. Then,  for any $k \in \{2,\dots,2q\}$,and function $g \in \setfunction_{\lyapW}$,
\begin{equation}
\label{eq: Gpik estimate}
 |\Gamma_{\pi, k}(S_n)| \leq  n \rate^{-1} \log^{1-k}\{1/\rate\} \ConstD^{k}  \bigl\{\normlike_{\lyapW}(\bar{g})\bigr\}^{k} (k!)^{3+\arate} \eqsp.
\end{equation}
\end{lemma}
\begin{proof}
By \eqref{eq:expression_gamma_pi_k_S_n}, we get $ \bigl|\Gamma_{\pi,k}(S_n)\bigr| \leq k! \sum_{0\leq t_1 \leq \cdots \leq t_k \leq n-1} \bigl|\Gamma_{\pi}(Y_{t_1}, \ldots, Y_{t_k})\bigr|$.
Denoting $I(n,k) = \{ (t_1, \ldots, t_k): 0 \le t_1 \le \ldots \le t_k \le n-1  \}$ and using \Cref{lem:cumulant_bounds_generic}, we get
\begin{equation}
\label{eq:bound_cumulant_S_n_generic}
%\begin{split}
|\Gamma_{\pi, k}(S_n)|
  \leq (k!)^{\arate + 1} \ConstD^k \bigl\{\normlike_{\lyapW}(\bar{g})\bigr\}^{k} 
 \sum_{I \in I(n,k)} \sum_{r=1}^k \sum_{\bigcup_{\ell=1}^r I_\ell = I} N_r(I_1, \ldots, I_r) \rate^{\sum_{\ell=1}^r \maxgap(I_{\ell})}
%\end{split}
\end{equation}
For any $m \in \{2,\dots,k\}$, $\gapindex \in \{0,\dots,m-1\}$, we set $I(n,k,m)=\bigcup_{\gapindex=0}^{n-1} I(n,k,m,\gapindex)$, where
\begin{equation}
  \label{eq:def_I_n_k_m_r_0_generic}
        I(n,k,m,\gapindex): = \{ (t_1, \ldots, t_k) \in I(n,k): t_{m} - t_{m-1} = \gapindex = \max_{i\in\{2,\ldots,k\}}(t_{i} - t_{i-1}) \} \eqsp.
\end{equation}
Then $I(n,k) = \bigcup_{m=2}^{k} I(n,k,m)$. Note that for any $I = (t_1,\ldots,t_k) \in I(n,k,m,\gapindex)$ and any partition $ \{I_1,\ldots,I_r\}$, $\bigcup_{\ell=1}^r I_\ell = I$, it holds that $\sum_{\ell=1}^{r}\maxgap(I_{\ell}) \geq \gapindex=\maxgap(I)$.
Using \eqref{eq:formula-summation}, we get
\begin{multline*}
\sum_{I \in I(n,k)} \sum_{r=1}^k \sum_{\bigcup_{\ell=1}^r I_\ell = I} N_r(I_1, \ldots, I_r) \rate^{\sum_{\ell=1}^r \maxgap(I_{\ell})} \leq k! \sum_{\gapindex=0}^{n-1} \rate^{\gapindex} \sum_{m=2}^k \card{I(n,k,m,\gapindex)}\eqsp.
\end{multline*}
Plugging this bound in \eqref{eq:bound_cumulant_S_n_generic} yields
\begin{equation}
\label{lem:cumulants:1:generic}
 |\Gamma_{\pi,k}(S_n)|
\le (k!)^{\arate + 2} \ConstD^k \bigl\{\normlike_{\lyapW}(\bar{g})\bigr\}^{k} \sum_{m=2}^{k}\sum_{\gapindex=0}^{n-1} \rate^{\gapindex} \card{I(n,k,m,\gapindex)} \eqsp.
\end{equation}
For any $(t_1,\ldots,t_k) \in I(n,k,m,\gapindex)$, using \eqref{eq:def_I_n_k_m_r_0_generic}, we get $t_{m-1} \in \{0,\dots,n-1-\gapindex\}$ and $\max_{i\neq m}\{t_{i}-t_{i-1}\} \leq \gapindex$.
Therefore, $\card{I(n,k,m,\gapindex)} \leq (n-1-\gapindex)(\gapindex+1)^{k-2}$. Combining \eqref{lem:cumulants:1:generic} and \eqref{eq:bound_cumulant_S_n_generic} completes the proof together with
\begin{align*}
\sum_{\gapindex=0}^{n-1} \rate^{\gapindex} (\gapindex+1)^{k-2} \leq \frac{1}{\rate} \int_{0}^n \rate^{s+1}(s+1)^{k-1} \rmd s  \leq \frac{1}{\rate}\biggl(\frac{1}{\log{1/\rate}}\biggr)^{k-1}(k-2)!\eqsp.
\end{align*}
\end{proof}


Based on \Cref{lem:cumulant_bounds_final_generic} and \eqref{eq:qmomentexpansion}, we can now establish a bound on $\PE_{\pi}[|S_n|^{2q}]$.

\begin{lemma}\label{th:generic_th_rosenthal}
Assume \Cref{assum:central_moments_bound}($q,\lyapW,\normlike$). Then, for any $g \in \setfunction_{\lyapW}$, it holds that
\begin{equation*}
\PE_\pi[|S_n|^{2q}] \leq \momentGq[q] \{\PVar[\pi](S_n)\}^q +  \ConstC^{2q}_{q,\lyapW} \ConstD^{2q}  \sum_{u=1}^{q-1} \ConstB_{\arate}(u,q) \biggl(\frac{n \log{(1/\rate)}}{\rate}\biggr)^{u}\eqsp,
\end{equation*}
where $\ConstC_{q,\lyapW}= \{ \log(1/\rate) \}^{-1}$ and $\ConstB_{\arate}(u,q)$ is defined in~\eqref{eq: B_u_q_def_new}.
\end{lemma}
\begin{proof}
Denote the second term in the right-hand side of~\eqref{eq:qmomentexpansion} by $R_{\pi, n}$, i.e.,
\begin{equation*}
    R_{\pi, n} = \sum_{u=1}^{q-1} \frac{1}{u!} \sum_{|\bk_u| = 2q} \frac{(2q)!}{{\boldsymbol k_u}!} \prod_{p=1}^u \Gamma_{\pi, k_p}(S_n) = \sum_{u=1}^{q-1} \frac{1}{u!} \sum_{\bk_u \in \scrE_{u,q}} \frac{(2q)!}{{\boldsymbol k_u}!} \prod_{p=1}^u \Gamma_{\pi, k_p}(S_n) \eqsp,
\end{equation*}
where we have used in the last equality $\Gamma_{\pi,1}(S_n) = 0$ and  $\scrE_{u,q} = \{\bk_u= (k_1,\ldots,k_u) \in \nset^u \, : \, \sum_{i=1}^u k_i = 2q\, ,\, \min_{i \in\iint{1}{u}} k_i \geq 2\}$. Applying \Cref{lem:cumulant_bounds_final_generic}, we get
\begin{equation*}
%\begin{split}
|R_{\pi,n}|  \leq  \ConstD^{2q} \bigl\{\normlike_{\lyapW}(\bar{g})\bigr\}^{2q} \sum_{u=1}^{q-1} n^{u} \frac{(2q)!}{u!}\biggl(\frac{1}{\rate}\biggr)^{u} \biggl(\frac{1}{\log{1/\rate}}\biggr)^{2q-u} \sum_{\boldsymbol{k}_u \in \scrE_{u,q}} \prod_{i=1}^u (k_i!)^{\arate+2}\eqsp,
\end{equation*}
which completes the proof by the definition of $\ConstB_{\arate}(u,q)$.
\end{proof}


\subsection{Proof of \Cref{th:rosenthal_V_q}}
\label{sec:proof-ros_v_q}
In this section, we show that assumptions \Cref{assG:kernelP_q} and \Cref{assG:kernelP_q_smallset} imply \Cref{assum:central_moments_bound}$(q,V^{1/(2q)},\Vnorm[V^{1/2q}]{\cdot})$ for any $q \in \nset$. Applying \Cref{lem:rate_UGE}, we get that for any $x \in \Xset$, $0 < \alpha  \leq 1$, and $n \in \nset$,
\begin{equation}
\label{eq:V-geometric-better-rate}
\Vnorm[V^\alpha]{\MK^n(x, \cdot) - \pi}
 \leq 2 \{\cmconstv \ratev^n \pi(V)   V(x) \}^{\alpha}
\eqsp.
\end{equation}

\begin{lemma}
\label{lem:centered_moments_Z_old}
Assume \Cref{assG:kernelP_q}, \Cref{assG:kernelP_q_smallset}, and let $s \in \nset$. Then for any $k \in \{1,\ldots,s\}$,  $(t_1,\dots,t_{k}) \in \{0,\ldots,n-1\}^{k}$, $t_1 \leq \dots\leq t_{k}$,   $(p_1,\dots,p_k) \in \nset^k$ satisfying  $p_i \geq 1$ for $i \in \{1,\dots,k\}$ and $\sum_{i=1}^k p_i \le s$, and functions $\{h_{\ell}\}_{\ell=1}^{k} \subset  \mrl_{V^{p_\ell/s}}$, we have 
\begin{multline}
\label{eq:centred_moments_one_Z_0}
  |\PEC_\pi[h_1(X_{t_1}), \ldots, h_k(X_{t_k})] |
   \\\leq 2^{k-1} \{\cmconstv \pi(V)\}^{\sum_{\ell=1}^k\sum_{j=\ell}^k p_{j}/s} \ratev^{\sum_{j=2}^{k}(t_j - t_{j-1})p_{j}/s} \prod_{\ell = 1}^k   \|h_\ell\|_{V^{p_{\ell}/s}} \eqsp.
\end{multline}
\end{lemma}
\begin{proof}
 The proof is based on induction on $k \in \{1,\dots,s\}$.  For $k = 1$, we get :
\begin{equation*}
|\PEC_\pi[h_1(X_{t_1})]| \leq \pi(V^{p_1/s}) \|h_1\|_{V^{p_{1}/s}} \overset{(a)}{\leq} (\cmconstv \pi(V))^{p_{1}/s} \|h_1\|_{V^{p_{1}/s}}\eqsp,
\end{equation*}
where in (a) we used Jensen's inequality and the fact that $\cmconstv \geq 1$.
Assume that~\eqref{eq:centred_moments_one_Z_0} holds for some $k \in \{1,\dots,s-1\}$. Let $t_1 \leq \dots\leq t_{k+1}$,   $(p_1,\dots,p_{k+1})\in \nset^{k+1}$ satisfying  $p_i \geq 1$ for $i \in \{1,\dots,k+1\}$ and $\sum_{i=1}^{k+1} p_i \le s$, and functions $\{h_{\ell}\}_{\ell=1}^{k+1} \in  \mrl_{V^{p_\ell/s}}^{k+1}$. By \Cref{lem:centred_moments_markov_property},
\begin{equation*}
   \PEC_\pi[h_1(X_{t_1}), \ldots, h_k(X_{t_k}), h_{k+1}(X_{t_{k+1}})] = \PEC_\pi[h_1(X_{t_1}), \ldots , h_k(X_{t_k}) \tilde h_{k+1}(X_{t_k})],
  \end{equation*}
  where $\tilde h_{k+1}(x) = \MK^{t_{k+1} - t_k} h_{k+1}(x) - \pi(h_{k+1})$.
  Since $h_{k+1} \in \mrl_{V^{p_{k+1}/s}}$, we apply~\eqref{eq:V-geometric-better-rate} with $\alpha = p_{k+1}/ s$. Thus,
  $$
  |\tilde h_{k+1}(x)| \leq 2(\cmconstv \pi(V))^{p_{k+1}/s} \ratev^{(t_{k+1} - t_k)p_{k+1}/s} V^{p_{k+1}/s}(x) \|h_{k+1}\|_{V^{p_{k+1}/s}}\eqsp.
  $$
  Hence, using that $h_{k} \in \mrl_{V^{p_{k}/s}}$, we obtain
\begin{equation*}
\|h_k \tilde h_{k+1}\|_{V^{(p_k+p_{k+1})/s}}  \leq
 2 \bigl(\cmconstv \pi(V)\bigr)^{p_{k+1}/s}\ratev^{(t_{k+1} - t_k)p_{k+1}/s}
 \|h_{k}\|_{V^{p_{k}/s}}
 \|h_{k+1}\|_{V^{p_{k+1}/s}} \eqsp.
\end{equation*}
Then, applying the induction hypothesis to $\bar{h}_i = h_i \in \mrl_{V^{\bar{p}_i/s}}$, $\bar{p}_i = p_i$, $i \in\{1,\ldots,k-1\}$, $\bar{h}_k = h_k \tilde h_{k+1} \in \mrl_{V^{\bar{p}_k/s}}$, $\bar{p}_k = p_k + p_{k+1}$ completes the proof.
\end{proof}
\begin{corollary}
\label{coro:centered_moments_V_class}
Assume \Cref{assG:kernelP_q}, \Cref{assG:kernelP_q_smallset}. Then for any $q \in \nset$, \Cref{assum:central_moments_bound}($q,V^{1/(2q)},\Vnorm[V^{1/2q}]{\cdot}$) is satisfied with $\ConstD[q,V^{1/(2q)}] = 2 \cmconstv \pi(V)$, $\arate[q,V^{1/(2q)}] = 0$, and $\rho_{q,V^{1/(2q)}} = \ratev^{1/2}$, where $\ratev$ is defined in \eqref{eq:V-geometric-coupling-general}.
%Then, for any $u,k \in \{1,\ldots,q\}$, $k$-tuple $I = (t_1,\dots,t_{k}) \in \{0,\ldots,n-1\}^{k}$, $t_1 \leq \dots\leq t_{k}$, functions $(h_1,\dots,h_k)$ such that $h_i \in \mrl_{V^{1/(2q)}}$ for $i \in \{1,\ldots,k\}$, it holds that
%\begin{equation}
%\label{eq:centred_moments_one_Z_new}
%  |\PEC_\pi[h_1(X_{t_1}), \ldots, h_k(X_{t_k})]|
%  \leq (2 c_m \pi(V))^{k} \rho^{\maxgap(I)/2} \prod_{\ell = 1}^k   \|h_\ell\|_{V^{1/(2q)}} \eqsp,
%\end{equation}
%where $\maxgap(I) := \max_{j \in \{2,\dots,k\}}[t_{j}-t_{j-1}]$. That is, the assumption \Cref{assum:central_moments_bound} holds with $\ConstD = 2 c_m \pi(V)$, $\alpha = 0$ and $\lyapW = V^{1/(2q)}$.
\end{corollary}
\begin{proof}
  Let $k \in \{1,\ldots,q\}$ and
 $(t_1,\dots,t_{k}) \in \{0,\ldots,n-1\}^{k}$, $t_1 \leq \dots\leq t_{k}$.
  Define $\maxind \in \{2,\ldots,k\}$ such that $t_{\maxind} - t_{\maxind-1} = \max_{j \in \{2,\ldots,k\}} [t_j - t_{j-1}]$. For $i \in \{1,\ldots,k\} \setminus \{\maxind\}$, we set $p_{i} = 1$, and put $p_{\maxind} = q$.
  Now we apply \Cref{lem:centered_moments_Z_old} with the mentioned choice of $(p_1,\dots, p_k)$ and $s = 2q$.  Note that $ h_i \in \mrl_{V^{p_i/(2q)}}$ for any $i \in \{1,\ldots,k\}$ and $\sum_{i=1}^k p_i \leq 2q$. Moreover, $\|h_{\maxind}\|_{V^{1/2}} \leq \|h_{\maxind}\|_{V^{1/(2q)}}$ since $q \geq 1$ and $V(x) > 1$. Therefore, the application of  \Cref{lem:centered_moments_Z_old} concludes the proof.
\end{proof}

\begin{proof}[Proof of \Cref{th:rosenthal_V_q}] The proof now follows from \Cref{th:generic_th_rosenthal} combined with \Cref{coro:centered_moments_V_class}.
\end{proof}

\subsection{Proof of \Cref{theo:changeofmeasure} and \Cref{theo:changeofmeasure-1}}
\label{sec:non-stationary-extension}
To go from an arbitrary initialization to the stationary case, we  use a distributional coupling argument (see \cite{thorisson1986maximal} and \cite[Chapter~19]{douc:moulines:priouret:soulier:2018}).
%For the sake of completeness, we recall below some important definitions and results.
Denote by $\QQ$ and $\QQ'$ two
probability measures on the canonical space  $(\Xset^\nset,\Xsigma^{\otimes \nset})$.
Fix $x^*\in \Xset$ and denote $\bar{\nset} = \nset \cup \{\infty\}$. For any $\Xset$-valued stochastic process $\Xcoupling=\sequence{\Xcoupling}[n][\nset]$ and any
$\bar \nset$-valued random variable $T$, define the $\Xset$-valued stochastic process $\shift_T \Xcoupling$
by $\shift_T \Xcoupling=\sequencen{\Xcoupling_{T+k}}[k \in \nset]$ on $\{T<\infty\}$ and $\shift_T \Xcoupling=(\x^*,x^*,
x^*,\ldots)$ on $\{T=\infty\}$.
Let $\Xcoupling=\sequence{\Xcoupling}[n][\nset]$, $\Xcoupling'=\sequence{\Xcoupling'}[n][\nset]$ be  $\Xset$-valued stochastic
processes and $T$, $T'$ be $\bar \nset$-valued random variables defined on the
probability space $(\Omega,\mcf,\PPcoupling)$.

We say that $\left\{(\Omega,\mcf,\PPcoupling,\Xcoupling,T,\Xcoupling',T') \right\}$ is a distributional coupling of $(\QQ,\QQ')$ if
\begin{description}
  \item[DC-1] for all $\msa \in \Xsigma^{\otimes \nset}$, $\PPcoupling(\Xcoupling \in \msa)=\QQ(\msa)$ and $\PPcoupling(\Xcoupling' \in \msa)=\QQ'(\msa)$,
  \item[DC-2] $(\shift_T \Xcoupling, T)$ and $(\shift_{T'} \Xcoupling', T')$ have the same distribution  under $\PPcoupling$.
\end{description}
The random variables $T$ and $T'$ are called the coupling times.  The distributional coupling is said to be \emph{successful} if $\PPcoupling(T<\infty)=1$.

For any measure $\mu$ on $(\Xset^\nset,\Xsigma^{\otimes \nset})$ and any
$\sigma$-field $\mcg \subset \Xsigma^{\otimes \nset}$, we denote by $\restric{\mu}{\mcg}$ the
restriction of the measure $\mu$ to $\mcg$. Moreover, for all $n \in \nset$, define the
$\sigma$-field $\mcg_n=\set{\shift_n^{-1}(\msa)}{\msa \in \Xsigma^{\otimes \nset}}$.
  A distributional coupling $(\Xcoupling,\Xcoupling')$ of $(\QQ,\QQ')$ with coupling times $(T,T')$ is maximal if for
  all $n\in\nset$,
$$
\tvnorm{\restric{\QQ}{\mcg_n}-\restric{\QQ'}{\mcg_n}}=2 \PPcoupling(T >n)\eqsp.
$$
By \cite[Theorem~19.3.9]{douc:moulines:priouret:soulier:2018}, for any two probabilities $\mu, \nu$  on $(\Xset,\Xsigma)$, we have $\tvnorm{\restric{\PP_\mu}{\mcg_n}-\restric{\PP_\nu}{\mcg_n}}$ $=\tvnorm{\mu \MK^n-\nu \MK^n}$  and
there exists a successful maximal distributional coupling of $(\PP_\mu,\PP_\nu)$ denoted by $\left\{(\Omega,\mcf,\PPcoupling[\mu,\nu],\Xcoupling,T,\Xcoupling',T') \right\}$. By \cite[Lemma~19.3.8]{douc:moulines:priouret:soulier:2018}, the distributional coupling $\PPcoupling[\mu,\nu]$ satisfies, for any nonnegative function $V$,
\begin{equation}
\label{eq:bound-coupling}
\PEcoupling[\mu,\nu][V(\check{X}_n) \indiacc{T > n}]= (\mu \MK^n - \nu \MK^n)^+ V \eqsp,
\quad \PEcoupling[\mu,\nu][V(\check{X}_n') \indiacc{T' > n}]= (\nu \MK^n - \mu \MK^n)^+ V \eqsp,
\end{equation}
where for any signed measure $\mu$ on $(\Xset,\Xsigma)$, $\mu^+$ denotes its positive part in the corresponding Jordan decomposition.
By construction,  $\PE_\xi[|S_n|^{2q}]= \PEcoupling[\xi,\pi][ | \sum_{k=0}^{n-1} g(\Xcoupling_k)|^{2q}]$ and $\PE_\pi[|S_n|^{2q}]= \PEcoupling[\xi,\pi][ | \sum_{k=0}^{n-1} g(\Xcoupling'_k) |^{2q}]$.
Denote $S_{T,n}= \sum_{k=0}^{n-1} |g(\Xcoupling_k)| \indiacc{T > k}$ and $S_{T',n}= \sum_{k=0}^{n-1} |g(\Xcoupling'_k) |\indiacc{T' > k}$.
\begin{lemma}
\label{lem:bound_sn_coupling_dist}
Assume \Cref{assG:kernelP_q}, \Cref{assG:kernelP_q_smallset}, and let $\xi$ be a probability measure on $(\Xset,\Xsigma)$. Then for any family of real measurable function $g$ on $\Xset$ it holds
\begin{enumerate}[label=(\alph*)]
\item   \label{lem:bound_sn_coupling_dist_1} for any $q \in\nsets$,
\begin{equation*}
\PE_\xi[|S_n|^{2q}]
\leq 2^{2q-1} \PE_\pi[|S_n|^{2q}]
+ 2^{4q-2} \PEcoupling[\xi,\pi]\Bigl[ S_{T',n}^{2q} \Bigr]
+ 2^{4q-2} \PEcoupling[\xi,\pi]\Bigl[S_{T,n}^{2q}  \Bigr] \eqsp.
\end{equation*}
\item \label{lem:bound_sn_coupling_dist_2} for any $t \geq 0$,
\begin{equation*}
\PP_{\xi}(|S_n| \geq t) \leq \PP_{\pi}(|S_n| \geq t/4) + \PPcoupling[\xi,\pi]( S_{T',n} \geq t/4) +
\PPcoupling[\xi,\pi]( S_{T,n} \geq t/2)\eqsp.
\end{equation*}
\end{enumerate}
\end{lemma}
\begin{proof}
Since
\begin{equation}
\label{eq:decomposition_distr_coupling}
\begin{split}
\sum_{k=0}^{n-1} g(\Xcoupling_k)
&= \sum_{k=0}^{n-1} g(\Xcoupling_k) \indiacc{T \geq n} + \sum_{k=0}^{n-1} g(\Xcoupling_k) \indiacc{T \leq n-1} \\
&= \sum_{k=0}^{n-1} g(\Xcoupling_k) \indiacc{T \geq n} + \sum_{k=0}^{T-1} g(\Xcoupling_k) \indiacc{T \leq n-1} +
\sum_{k=0}^{n-T-1} g(\shift_T \Xcoupling_{k}) \indiacc{T \leq n-1}\eqsp,
\end{split}
\end{equation}
we have
\begin{align}
\label{eq:coupling-1}
&\Bigl| \sum_{k=0}^{n-1} g(\Xcoupling_k) \Bigr|
\leq  S_{T,n} + \left| \sum_{k=0}^{n-T-1} g(\shift_T \Xcoupling_{k}) \indiacc{T \leq n-1}\right| \\
\label{eq:coupling-2}
&\Bigl| \sum_{k=0}^{n-T'-1} g(\shift_{T'} \Xcoupling'_{k}) \indiacc{T' \leq n-1} \Bigr|
\leq S_{T',n} + \left| \sum_{k=0}^{n-1} g(\Xcoupling'_k) \right|
\eqsp.
\end{align}
Now \ref{lem:bound_sn_coupling_dist_1} follows from Minkowski's inequality and (DC-1), (DC-2). Similarly, the proof of \ref{lem:bound_sn_coupling_dist_2} uses the same decomposition \eqref{eq:decomposition_distr_coupling}-\eqref{eq:coupling-2} and the union bound.
\end{proof}

\begin{lemma}
  \label{lem:prob_ineq_non_statio_v_norm}
  Assume \Cref{assG:kernelP_q},\Cref{assG:kernelP_q_smallset}, and let $\gamma \geq 0$, $\xi$ be a probability measure on $(\Xset,\Xsigma)$.   Then, for any real measurable function $g \in \mrl_{W^{\gamma}}$ on $\Xset$, $t \geq 0$,  it holds with  $\varpi_{\gamma} = 1/(1+\gamma)$, that
  \begin{align*}
    % \label{eq:1}
&    \PPcoupling[\xi,\pi](S_{T,n} \geq t) +     \PPcoupling[\xi,\pi](S_{T',n} \geq t) \\
    &\leq 2 \parenthese{\rme^{\log(\rho)t^{\varpi_{\gamma}}/(4\ConstM[n,W^{\gamma}]^{\varpi_{\gamma}}\varpi_{\gamma})}\ratev^{-1/2} +   \rme^{-(1+\gamma) t^{\varpi_{\gamma}}/(2\ConstM[n,W^{\gamma}]^{\varpi_{\gamma}}\gamma)} (1-\ratev)^{-1} }\cmconstv \{ \xi(V) + \pi(V) \} \eqsp.
  \end{align*}
\end{lemma}
\begin{proof}
    Without loss of generality, we can assume that  $ \| g \|_{W^{\gamma}} = 1$. We first assume that $\gamma >0$. Note that by Young's inequality for products, we have that for any $u_1,u_2 \in \rset_+$,
  \begin{equation}
    \label{eq:2}
    u_1^{\varpi_{\gamma}} u_2^{\varpi_{\gamma}} \leq \varpi_{\gamma} u_1 + (1-\varpi_{\gamma}) u_2^{\varpi_{\gamma}/(1-\varpi_{\gamma})} \eqsp.
  \end{equation}
  Then, we get since $\varpi_{\gamma}/(1-\varpi_{\gamma}) = 1/\gamma$ that
  \begin{align}
  S_{T,n}^{\varpi_{\gamma}} &\leq \varpi_{\gamma} (T \wedge n) + (1-\varpi_{\gamma}) \parentheseDeux{\frac{S_{T,n}}{T\wedge n}}^{\varpi_{\gamma}/(1-\varpi_{\gamma})} \\
%   &   \leq \varpi_{\gamma} \ConstM[n,W^{\gamma}](T \wedge n) + (1-\varpi_{\gamma})  \ConstM[n,W^{\gamma}]^{\varpi_{\gamma}} \max_{k \in \iint{0}{n-1}} \{|g_k(\Xcoupling_k)/\ConstM[n,W^{\gamma}]|^{\varpi_{\gamma}/(1-\varpi_{\gamma})} \indiacc{T > k} \} \\
        \label{eq:3}
    &   \leq \varpi_{\gamma}  (T \wedge n) + (1-\varpi_{\gamma}) \max_{k \in \iint{0}{n-1}}  \{\log V(\Xcoupling_k) \indiacc{T > k} \} \eqsp.
  \end{align}
  Similarly, it is easy to verify that \eqref{eq:3} holds for $\gamma =0$. Therefore, we obtain
  \begin{multline}
    \label{eq:bound_exp_distriub_coupling}
    \PPcoupling[\xi,\pi](S_{T,n} \geq t ) \leq
    \PPcoupling[\xi,\pi]\parenthese{ \varpi_{\gamma} T \geq t^{\varpi_{\gamma}}/2 } \\
    +         \PPcoupling[\xi,\pi]\parenthese{ (1-\varpi_{\gamma})  \max_{k \in \iint{0}{n-1}}  \{\log V(\Xcoupling_k) \indiacc{T > k} \} \geq t^{\varpi_{\gamma}}/2}\eqsp.
  \end{multline}
  We bound the two terms in the right-hand side separately. Setting $\lambda_\ratev = - \log(\ratev)/2$ and using that $\indiacc{T = k} = \indiacc{T > k-1} - \indiacc{T > k}, k \in \nset$ and $V(x) \geq 1, x \in \Xset$, we get
  \begin{align}
    \nonumber
    &\PPcoupling[\xi,\pi]\parenthese{ \varpi_{\gamma} T \geq  t^{\varpi_{\gamma}}/2} \leq
    \rme^{-\lambda_{\ratev}t^{\varpi_{\gamma}}/(2\varpi_{\gamma})}  \PEcoupling[\xi,\pi] \parentheseDeux{ \rme^{\lambda_{\ratev} T}} \\
    \nonumber
    &  \qquad \qquad    \leq  \rme^{-\lambda_{\ratev}t^{\varpi_{\gamma}}/(2\varpi_{\gamma})} \Bigl\{ 1+ (\ratev^{-1/2}-1)  \sum_{k=0}^{\plusinfty}  \ratev^{-k/2} \PEcoupling[\xi,\pi] \parentheseDeux{V(\Xcoupling_k)  \indiacc{T > k}} \Bigr\}\\
        \label{eq:bound_exp_distriub_coupling_1}
    & \qquad \qquad \txts  \overset{(a)}{\leq} \rme^{-\lambda_{\ratev}t^{\varpi_{\gamma}}/(2\varpi_{\gamma})}  \Bigl\{ 1+ (\ratev^{-1/2}-1) \sum_{k=0}^{\plusinfty}  \ratev^{-k/2}  \{(\xi \MK^k - \pi)^+(V) \} \Bigr\}\eqsp,
    %& \qquad \qquad \leq \rme^{-\lambda_{\rho}t^{\varpi_{\gamma}}/(2\varpi_{\gamma})}  \Bigl\{ 1+ (\rho^{-1/2}-1) \sum_{k=0}^{\plusinfty}  C \rho^{k/2}  \{\xi(V) + \pi(V)\} \Bigr\}\\
    %& \qquad \qquad \leq \rme^{-\lambda_{\rho}t^{\varpi_{\gamma}}/(2\varpi_{\gamma})}  \Bigl\{ 1+ (\rho^{-1/2}-1)  (1-\rho^{1/2})^{-1} C  \{\xi(V) + \pi(V)\} \Bigr\}\\
    %              & \qquad \qquad \leq \rme^{-\lambda_{\rho}t^{\varpi_{\gamma}}/(2\varpi_{\gamma})}  \Bigl\{ 1+ \rho^{-1/2} C  \{\xi(V) + \pi(V)\} \Bigr\}\eqsp.
  \end{align}
where (a) is due to \eqref{eq:bound-coupling}. We complete the bound using that $(\xi \MK^k - \pi)^+(V) \leq \cmconstv \{ \xi(V) + \pi(V) \} \ratev^k$ due to \eqref{eq:V-geometric-coupling-general}. Now, applying \eqref{eq:bound-coupling} again, we obtain
  \begin{align}
\nonumber
  &\PPcoupling[\xi,\pi]\parenthese{ (1-\varpi_{\gamma})  \max_{k \in \iint{0}{n-1}}  \{\log V(\Xcoupling_k) \indiacc{T > k} \} \geq t^{\varpi_{\gamma}}/2}\\
  \nonumber
& \qquad \leq \exp\parenthese{-\frac{t^{\varpi_{\gamma}}}{2(1-\varpi_{\gamma})}} \sum_{k=0}^{\plusinfty} \PEcoupling[\xi,\pi] \parentheseDeux{V(\Xcoupling_k)  \indiacc{T > k}} \\
& \qquad
  \leq  \exp\parenthese{-\frac{t^{\varpi_{\gamma}}}{2(1-\varpi_{\gamma})}} \sum_{k=0}^{\plusinfty}  \{(\xi \MK^k - \pi)^+(V) \} \eqsp,
%     & \qquad
%  \leq  \exp\parenthese{-\frac{ t^{\varpi_{\gamma}}}{2(1-\varpi_{\gamma})}} c\{\pi(V) + \xi(V)\} (1-\rho)^{-1} \eqsp.
% \eqsp.
\label{eq:bound_exp_distriub_coupling_2}
\end{align}
and we again complete the bounds using \eqref{eq:bound-coupling}. The statement now follows by combining \eqref{eq:bound_exp_distriub_coupling_1} and \eqref{eq:bound_exp_distriub_coupling_2} in
  \eqref{eq:bound_exp_distriub_coupling}. We proceed similarly for $\PPcoupling[\xi,\pi](S_{n,T'}  \geq t)$.
\end{proof}

\begin{proof}[Proof of \Cref{theo:changeofmeasure}]
By Minkowski's inequality, we get
\begin{align*}
\Bigl( \PEcoupling[\xi,\pi][   S_{T,n}^{2q}  ] \Bigr)^{1/2q}
\leq \Vnorm[V^{1/(2q)}]{g} \sum_{k=0}^{n-1} \left( \PEcoupling[\xi,\pi][V(\Xcoupling_k) \indiacc{T > k}] \right)^{1/2q}.
\end{align*}
Using \eqref{eq:bound-coupling}, we get
$ \PEcoupling[\xi,\pi][V(\Xcoupling_k) \indiacc{T > k}]  \leq
\Vnorm[V]{\pi - \xi \MK^k}^{1/2q}$. Note also that the same upper bound holds for $\PEcoupling[\xi,\pi][S_{T',n}^{2q}]$ by (DC-2). Now, combining \Cref{lem:bound_sn_coupling_dist}-\ref{lem:bound_sn_coupling_dist_1} with \eqref{eq:V-geometric-coupling-general}, we get 
\begin{equation*}
\PE_\xi\big[ \big|S_n \big|^{2q} \big] \leq 2^{2q-1} \PE_\pi\big[ \big|S_n \big|^{2q} \big]  + 2^{4q-1} \Vnorm[V^{1/(2q)}]{g}^{2q} \cmconstv \{ \xi(V) + \pi(V) \} (1 - \ratev^{1/2q})^{-2q}  \eqsp.
\end{equation*}
To conclude, we note that for $x \in (0,1)$, it holds that
\[
\frac{1}{\log{(1/x)}} = \frac{1}{\log(1 + (1-x)/x)} \geq \frac{x}{1-x}\,,
\]
and apply the above inequality with $x = \ratev^{1/2q}$.
\end{proof}


\begin{proof}[Proof of \Cref{theo:changeofmeasure-1}]
Using that $( \sum_{k=0}^{p-1} a_k )^{2q} \leq p^{2q-1} \sum_{k=0}^{p-1} a_k^{2q}$ for any $a_k \geq 0$, and
\begin{align*}
\PEcoupling[\xi,\pi]\Bigl[ S_{T,n}^{2q} \Bigr] &= \PEcoupling[\xi,\pi]\Bigl[ \bigl(\sum_{k=0}^{(n-1) \wedge (T-1)} |g(\Xcoupling_k)| \bigr)^{2q} \Bigr]
\leq \|g\|_{W^{\gamma}}^{2q} \sum_{k=0}^{n-1}  \PEcoupling[\xi,\pi][T^{2q-1} W(\Xcoupling_k)^{2 \gamma q} \indiacc{T > k}]\\
& \txts  \overset{(a)}{\leq} (1/2) \|g\|_{W^{\gamma}}^{2q} \PEcoupling[\xi,\pi][T^{4q-2}]  + (1/2) \|g\|_{W^{\gamma}}^{2q} (4 q \gamma/ \rme)^{4 q \gamma} \sum_{k=0}^{n-1}  \{(\xi \MK^k - \pi)^+(V) \}\eqsp.
\end{align*}
In (a) we used \eqref{eq:bound-coupling} combined with the bound $\sup_{x \in \Xset}W^{4\gamma q}(x) / V(x) \leq (4\gamma q/\rme)^{4\gamma q}$, which holds since $V(x) \geq \rme$. Since $\PEcoupling[\xi,\pi][T^{4q-2}] \leq  \rme^{-1} \sum_{k=0}^{\infty} (k+1)^{4q-2} \PEcoupling[\xi,\pi][V(\Xcoupling_k) \indiacc{T \geq k}]$, one more application of \eqref{eq:bound-coupling} yields
\begin{align*}
&\PEcoupling[\xi,\pi]\Bigl[ S_{T,n}^{2q} \Bigr]
+ \PEcoupling[\xi,\pi]\Bigl[ S_{T',n}^{2q} \Bigr]
\\
&\leq  \|g\|_{W^{\gamma}}^{2q} \rme^{-1} \sum_{k=0}^\infty (k+1)^{4q-2} \Vnorm[V]{\xi \MK^k - \pi} +
\|g\|_{W^{\gamma}}^{2q}  (4 q \gamma/ \rme)^{4 q \gamma} \sum_{k=0}^{\infty} \Vnorm[V]{\xi \MK^k - \pi} \\
&\txts  \overset{(b)}{\leq} \cmconstv \{ \xi(V) + \pi(V) \} \|g\|_{W^{\gamma}}^{2q} \left( \rme^{-1} \ratev^{-1} \{ \log(1/\ratev) \}^{1- 4 q} (4q-2) ! +  (4 q \gamma/ \rme)^{4 q \gamma} (1- \ratev)^{-1} \right)
\eqsp.
\end{align*}
In (b) we used \eqref{eq:V-geometric-coupling-general} together with an upper bound
\begin{equation*}
\sum_{k=0}^\infty (k+1)^{4q-2} \ratev^{k} \leq \ratev^{-1}\int_{0}^{+\infty}x^{4q-2}\ratev^{x}\,\rmd x = \ratev^{-1}(\log{1/\ratev})^{1-4q}(4q-2)!\eqsp.
\end{equation*}
The rest of the proof follows from \Cref{lem:bound_sn_coupling_dist}-\ref{lem:bound_sn_coupling_dist_1} combined with 
\[
\frac{1}{1-\ratev} \leq \frac{\{ \log(1/\ratev) \}^{-1}}{\ratev}\,.
\]
\end{proof}

\begin{proof}[Proof of \Cref{theo:prob_ineq_V_norm}]
  The proof follows from
  \Cref{lem:bound_sn_coupling_dist}-\ref{lem:bound_sn_coupling_dist_2}
  and \Cref{lem:prob_ineq_non_statio_v_norm}.
\end{proof}

\subsection{Proof of \Cref{th:rosenthal_log_V}}
\label{sec:proof-ros_log_V}
Similarly to \Cref{sec:proof-ros_v_q}, we show that, given $\gamma \geq 0$, assumptions \Cref{assG:kernelP_q} and \Cref{assG:kernelP_q_smallset} imply \Cref{assum:central_moments_bound}$(q,W^{\gamma},\Vnorm[W^\gamma]{\cdot})$ for any $q \in \nset$, where $W(x) = \log{V(x)}$. \Cref{assG:kernelP_q} implies that for all $x \in \Xset$ we have $W(x) \geq 1$.

\begin{lemma}
\label{coro:centered_moments_W_class_new}
Assume \Cref{assG:kernelP_q}, \Cref{assG:kernelP_q_smallset}. Then for any $q \in \nset$ and $\gamma \geq 0$, \Cref{assum:central_moments_bound}$(q,W^{\gamma},\Vnorm[W^\gamma]{\cdot})$ is satisfied with constants $\ConstD[q,W^{\gamma}] = 2^{1+\gamma}\gamma^{\gamma}\cmconstv\pi(V)$, $\arate[q,W^{\gamma}] = \gamma$, and $\rho_{q,W^{\gamma}} = \ratev^{1/2}$, where $\ratev$ is defined in \eqref{eq:V-geometric-coupling-general}.
\end{lemma}
\begin{proof}
  Let $k \in \{1,\ldots,q\}$ and
 $(t_1,\dots,t_{k}) \in \{0,\ldots,n-1\}^{k}$, $t_1 \leq \dots\leq t_{k}$.
Let $\maxind \in \{2,\ldots,k\}$ be an index of the largest gap in $(t_1,\dots,t_{k})$, that is, $t_{\maxind} - t_{\maxind-1} = \max_{j \in \{2,\ldots,k\}} [t_j - t_{j-1}]$. If such index $\maxind$ is not unique, we choose the largest one. For $i \in \{1,\ldots,k\} \setminus \{\maxind\}$, we set $p_{i} = 1$, and $p_{\maxind} = k$. Note that for $i \in \{1,\dots,k\}$, $h_i \in \mrl_{W^{\gamma}}$ implies $h_i \in \mrl_{V^{p_i/(2k)}}$.
Now we apply \Cref{lem:centered_moments_Z_old} with the mentioned choice of $(p_1,\dots, p_k)$ and $s = 2k$, and obtain
\begin{multline*}
|\PEC_\pi[h_1(X_{t_1}), \ldots, h_k(X_{t_k})] |
\\\leq 2^{k-1}  (\cmconstv \pi(V))^{\sum_{\ell=1}^k\sum_{j=\ell}^k p_{j}/(2k)} \ratev^{\sum_{j=2}^{k}(t_j - t_{j-1})p_{j}/(2k)} \prod_{\ell = 1}^k   \|h_\ell\|_{V^{1/(2k)}} \eqsp.
\end{multline*}
Here we used that $\|h_{\maxind}\|_{V^{1/2}} \leq \|h_{\maxind}\|_{V^{1/(2k)}}$ since $k \geq 1$ and $V(x) > 1$. To complete the proof it remains to note that $\sum_{i=1}^k p_i \leq 2k$ and
\begin{align*}
\|h_\ell\|_{V^{1/(2k)}}
= \sup_{x \in \Xset}\biggl\{\frac{|h(x)|}{V^{1/(2k)}(x)}\biggr\} \leq \sup_{x \in \Xset}\biggl\{\frac{|h(x)|}{W^{\gamma}(x)}\biggr\} \sup_{x \in \Xset}\biggl\{\frac{W^{\gamma}(x)}{V^{1/(2k)}(x)}\biggr\} \leq (2\gamma k / \rme)^{\gamma}\|h_\ell\|_{W^{\gamma}}\eqsp.
\end{align*}
Combining the previous inequalities,
\begin{align*}
|\PEC_\pi[h_1(X_{t_1}), \ldots, h_k(X_{t_k})] | \leq \bigl(2^{1+\gamma}\gamma^{\gamma}\cmconstv\pi(V)\rme^{-\gamma}\bigr)^{k}k^{\gamma k}\ratev^{\maxgap(I)/2} \prod_{\ell = 1}^k   \|h_\ell\|_{W^{\gamma}}\eqsp,
\end{align*}
which completes the proof together with the elementary inequality $k^{k} \leq k!\rme^{k}, k \in \nset$.
\end{proof}

\begin{proof}[Proof of \Cref{th:rosenthal_log_V}] The proof now follows from \Cref{coro:centered_moments_W_class_new} and \Cref{th:generic_th_rosenthal}.
\end{proof}


\subsection{Proof of \Cref{th:rosenthal_log_V_cor_2}}
\label{sec:proof_bernstein_bound}
We first prove the bound \eqref{eq:bernstein_mc}. \Cref{lem:cumulant_bounds_final_generic,coro:centered_moments_W_class_new} imply that for any $k \geq 3$,
\begin{equation*}
\begin{split}
|\Gamma_{\pi, k}(S_n)|
&\leq \ratev^{-1/2} 2^{k-1}\{\log(1/\ratev)\}^{1-k} \ConstD[q,W^{\gamma}]^{k} (k!)^{3+\gamma} n \|h_\ell\|_{W^{\gamma}}^{k}  \\
&\leq \biggl(\frac{k!}{2}\biggr)^{3+\gamma}\PVar[\pi](S_n) \biggl( \frac{n \ratev^{-1/2} \{\log(1/\ratev)\}^{-1} \ConstD[q,W^{\gamma}]^{2} \|\bar{g}\|_{W^{\gamma}}^{2}}{\PVar[\pi](S_n)} \vee 1\biggr) \biggl( \frac{2 \ConstD[q,W^{\gamma}] \|\bar{g}\|_{W^{\gamma}}}{\log(1/\ratev)} \biggr)^{k-2} \\
&\leq \biggl(\frac{k!}{2}\biggr)^{3+\gamma}\PVar[\pi](S_n) \ConstJ^{k-2} \eqsp,
\end{split}
\end{equation*}
where $\ConstD[q,W^{\gamma}] = 2^{1+\gamma}\gamma^{\gamma}\cmconstv\pi(V)$ and $\ConstJ$ is given in \eqref{eq:const_B_n_definition_main}. We conclude the statement using \cite[Lemma~2.1]{bentkus:1980} (see also \cite[Equation~(24)]{doukhan2007probability}). Next we show \eqref{eq:high_prob_bound_W_ergodic}. From the bound above we get for $t \geq 0$ that
\begin{equation*}
\PP_{\pi}(|S_n| \geq t) \leq 2\exp\biggl\{-\frac{t^2/2}{\PVar[\pi](S_n) + \ConstJ^{1/(3+\gamma)} t^{2-1/(3+\gamma)}}\biggr\}\eqsp.
\end{equation*}
Since for any $t$ it holds either $\PVar[\pi](S_n) + \ConstJ^{1/(3+\gamma)} t^{2-1/(3+\gamma)} \leq 2\PVar[\pi](S_n)$ or $\PVar[\pi](S_n) + \ConstJ^{1/(3+\gamma)} t^{2-1/(3+\gamma)} \leq 2\ConstJ^{1/(3+\gamma)} t^{2-1/(3+\gamma)}$,
the previous bound implies
\begin{align*}
\PP_{\pi}(|S_n| \geq t) \leq 2\exp\biggl\{-\frac{t^2}{4\PVar[\pi](S_n)}\biggr\} + 2\exp\biggl\{-\frac{t^{1/(3+\gamma)}}{4\ConstJ^{1/(3+\gamma)}}\biggr\} = T_1 + T_2\eqsp.
\end{align*}
To complete the proof we choose $t$ such that $T_1 \leq \delta/2$ and $T_2 \leq \delta/2$ for $\delta \in (0,1)$.


%\subsection{Proof of  \Cref{th:rosenthal_log_V_cor_1}}
%\label{sec:proof_moment_corollary_bentkus}
%To shorten notations let us introduce $a = \PVar[\pi](S_n)$, $b = \ConstB_n$, $\alpha = 1/(3\gamma+3)$. Then \Cref{th:rosenthal_log_V_cor_2} implies
%\begin{align*}
%\PE_{\pi}[|S_n|^{2q}]
%&= \int_{0}^{+\infty}\PP_{\pi}(|S_n|^{2q} \geq t)\,\rmd t = 2q\int_{0}^{+\infty}\PP_{\pi}(|S_n| \geq u)u^{2q-1}\,\rmd u \\
%&\leq 4q\int_{0}^{+\infty}u^{2q-1}\exp{\biggl\{-\frac{u^2/2}{a+b^{\alpha}u^{2-\alpha}}\biggr\}}\,\rmd u \\
%&\leq 4q\int_{0}^{+\infty}u^{2q-1}\exp{\biggl\{-\frac{u^2}{4a}\biggr\}}\,\rmd u + 4q\int_{0}^{+\infty}u^{2q-1}\exp{\biggl\{-\frac{u^{\alpha}}{4b^{\alpha}}\biggr\}}\,\rmd u\eqsp.
%\end{align*}
%Now we note that
%\begin{align*}
%4q\int_{0}^{+\infty}u^{2q-1}\exp{\biggl\{-\frac{u^2}{4a}\biggr\}}\,\rmd u = 2^{2q+1}q! a^{q}\eqsp,
%\end{align*}
%and
%\begin{align*}
%4q\int_{0}^{+\infty}u^{2q-1}\exp{\biggl\{-\frac{u^{\alpha}}{4b^{\alpha}}\biggr\}}\,\rmd u
%&= \alpha^{-1}q4^{2q/\alpha+1}b^{2q}\int_{0}^{+\infty}y^{2q/\alpha-1}\exp\{-y\}\,\rmd y \\
%&\leq 2^{4q/\alpha+1}b^{2q}\Gammabf(2q/\alpha + 1) \leq 2^{4q/\alpha+1}b^{2q}\rme^{-2q/\alpha + 1}(2q/\alpha)^{2q/\alpha + 1/2}\eqsp.
%\end{align*}
%The proof is concluded by noting that $(2q/\alpha)^{1/2}\rme^{-q/\alpha} \leq 1$.

\subsection{Weak Harris Theorem}
\label{sec:proof-ros_W_q}
\begin{proof}[Proof of Proposition~\ref{prop:wasser:geo}]
Set $\gamma = p/2q$. Note that $\MKK^m$ satisfies the geometric drift condition
\begin{equation}
\label{eq:geometric-drift-condition_m}
\MKK^m \bar{V} \leq \bar{\lambda}_m \bar{V} + b_m \indi{\CKset}\eqsp,
\end{equation}
where $\CKset$ is defined in \Cref{assG:kernelP_q_contractingset_m}. For $\delta  \geq 0$, set $\bar{V}_{\delta}=\bar{V}+\delta$ and
\begin{equation}
\label{eq:rho_tilde_def}
    \tilde{\rho}_{\gamma, \delta} = \sup_{(x,x') \in \Xset^2} \lrb{(1-\minorwas \indi{\CKset}(x,x'))^{1/2} \lr{ \frac{K^m\bar V^{\gamma}_\delta(x,x')}
    {\bar{V}^{\gamma}_\delta(x,x')}}^{1/2}} \eqsp ,
\end{equation}
  which is finite since $\MKK^m$ satisfies~\eqref{eq:geometric-drift-condition_m}. Furthermore, H\"older's inequality and \Cref{assG:kernelP_q_contractingset_m} yield
  \begin{align*}
    \MKK^m(\metricc^{1/2} \bar{V}_\delta^{\gamma/2})
    & \leq (\MKK^m\metricc)^{1/2} (\MKK^m\bar V^{\gamma}_\delta)^{1/2} \\
    & \leq \lrb{(1-\epsilon \indi{\CKset})^{1/2} \lr{{\MKK^m\bar V^{\gamma}_\delta}/{\bar{V}^{\gamma}_\delta}}^{1/2} }
      \metricc^{1/2} \bar{V}^{\gamma/2}_\delta \leq \tilde{\rho}_{\gamma, \delta} \metricc^{1/2} \bar{V}_\delta^{\gamma/2} \eqsp.
  \end{align*}
  Using $\bar{V} \leq \bar{V}_\delta$ and a straightforward induction, we obtain for any $k \geq 1$
  \begin{equation}
    \label{eq:wasser:geo:K_n}
    \MKK^{mk} (\metricc^{1/2} \bar V^{\gamma/2})\leq \MKK^{mk} (\metricc^{1/2} \bar V_\delta^{\gamma/2}) \\
    \leq \tilde{\rho}_{\gamma,\delta}^k \metricc^{1/2} \bar {V}_\delta^{\gamma/2}\eqsp.
  \end{equation}
  Let $n = k m + \ell$, $\ell \in \{0,\dots, m-1\}$. Note that the drift condition \Cref{assG:kernelP_q} together with Jensen's inequality imply
  \begin{align*}
  \MKK^\ell\bar V^{\gamma}_\delta \leq (\MKK^\ell\bar V_\delta)^{\gamma} \leq \biggl(\lambda^{\ell}\bar{V} + b(1-\lambda^\ell)/(1-\lambda)+\delta\biggr)^{\gamma} \leq (1 + b/(1-\lambda) + \delta)^{\gamma} \bar{V}^{\gamma}\eqsp.
  \end{align*}
  Combining this with H\"older's inequality yields
  \begin{equation*}
    \MKK^\ell(\metricc^{1/2} \bar{V}_\delta^{\gamma/2})
    \leq (\MKK^\ell\metricc)^{1/2} (\MKK^\ell\bar V^{\gamma}_\delta)^{1/2} \\
    \leq \boundmetric^{m/2} (1 + b/(1-\lambda) + \delta)^{\gamma/2}  \metricc^{1/2} \bar V^{\gamma/2} \eqsp.
  \end{equation*}
This inequality and \eqref{eq:wasser:geo:K_n}   imply
\begin{equation}
  \label{eq:prop:wasser:geo_1}
      \MKK^{km + \ell}(\metricc^{1/2} \bar{V}_\delta^{\gamma/2}) \leq \tilde{\rho}_{\gamma, \delta}^k \MKK^\ell(\metricc^{1/2}\bar {V}_\delta^{\gamma/2}) \le \tilde{\rho}_{\gamma,\delta}^k \boundmetric^{m/2} (1 + b/(1-\lambda) + \delta)^{\gamma/2}  \metricc^{1/2} \bar V_{\delta}^{\gamma/2} \eqsp.
    \end{equation}
We now provide a lower bound on $ \tilde{\rho}_{\gamma,\delta}$.
Applying~\eqref{eq:geometric-drift-condition_m}, we obtain
  \begin{equation}
    \label{eq:wasser:geo:third}
    \frac{\MKK^m\bar V^{\gamma}_\delta}{\bar V^{\gamma}_\delta} \leq \{ \varphi(\bar{V}) \}^{\gamma} \indi{\CKset} + \{\psi(\bar{V}) \}^{\gamma} \indi{\CKset^c} \eqsp,
  \end{equation}
  with $\varphi(v) = (\bar{\lambda}_m  v + b_m + \delta)/(v+\delta), \psi(v)= (\bar{\lambda}_m  v   + \delta)/(v+\delta)$. Since $\bar{V} \geq \indi{\CKset} +\bar{d}\indi{\CKset^c}$ and functions $\varphi$ and $\psi$ are decreasing on $[1;+\infty)$, we get
  \begin{align*}
    \frac{\MKK^m\bar V^{\gamma}_\delta}{\bar V^{\gamma}_\delta} \leq
      \left\{\frac{\bar{\lambda}_m+b_m+\delta}{1+\delta} \right\}^{\gamma} \indi{\CKset}
    + \left\{\frac{\bar{\lambda}_m + \bar{d}+\delta}{\bar{d}+\delta} \right\}^{\gamma}  \indi{\CKset^c}\eqsp.
  \end{align*}
  The previous inequality yields
  \begin{equation*}
    (1-\minorwas \indi{\CKset})^{1/2} \lr{{\MKK^m\bar{V}^{\gamma}_\delta}/{\bar{V}^{\gamma}_\delta}}^{1/2}
    \leq \lrb{(1-\minorwas)^{1/2} \lr{\frac{\bar \lambda_m+
          b_m+\delta}{1+\delta}}^{\gamma/2}}\indi{\CKset} + \lr{\frac{\bar \lambda_m \bar{d}+\delta}{\bar{d}
        +\delta}}^{\gamma/2} \indi{\CKset^c}\eqsp.
  \end{equation*}
Due to \eqref{eq:rho_tilde_def}, the previous inequality implies $\tilde{\rho}_{\gamma,\delta} \leq \rho_{\delta}^\gamma$, where
\begin{equation}
      \label{eq:def:rho_0}
      \rho_{\delta}=\lrb{(1-\minorwas)^{1/2} \lr{\frac{\bar \lambda_m+b_m+\delta}{1+\delta}}^{1/2}}
      \vee \lr{\frac{\bar \lambda_m \bar{d}+\delta}{\bar{d}+\delta}}^{1/2}\eqsp.
    \end{equation}
Now we choose $\delta = \deltawas$ defined in~\eqref{eq:delta_star_def}, and complete the proof setting $\ratewas = \rho_{\deltawas}$ in \eqref{eq:def:rho_0}.
\end{proof}

\begin{proof}[Proof of \Cref{cor:wasserstein-convergence}]
Using \cite[Corolllary~20.4.1]{douc:moulines:priouret:soulier:2018} and \eqref{eq: constraction} (with $p=2q$),  we get for any initial distribution $\xi, \xi'$ and $\gamma \in \couplingmeasure(\xi,\xi')$,
\begin{align}
  \nonumber
    \wasser[\cost]{\xi \MK^n}{\xi'\MK^n} &\leq \wasser[\cost^{1/2} \bar V^{1/2}]{\xi \MK^n}{\xi' \MK^n}
    \leq \int_{\Xset \times \Xset} \wasser[\cost^{1/2} \bar V^{1/2}]{\MK^n(x,\cdot)}{\MK^n(x',\cdot)}
      \gamma(\rmd x\rmd x')\eqsp, \\
    \nonumber
    &\leq \ratewas^{n}  \boundmetric^{m/2} \vartconstwas  \int_{\Xset \times \Xset} \gamma(\rmd x\rmd x') \bar{V}^{1/2}(x,x') \\
  \label{eq:cor:wasserstein-convergence_1}
    &\leq (1/\sqrt{2}) \ratewas^{n}   \boundmetric^{m/2} \vartconstwas \{ \xi(V^{1/2}) + \xi'(V^{1/2}) \} \eqsp.
\end{align}
We now show existence and uniqueness of $\pi$. Equation \eqref{eq:cor:wasserstein-convergence_1} implies that for some fixed $x_0 \in\Xset$
   \[
   \wasser[(\distance \wedge 1)^{\pcost}]{\MK^n(x_0,\cdot)}{\MK^{n+1}(x_0,\cdot)}  \leq (1/\sqrt{2}) \ratewas^{n}   \boundmetric^{m/2} \vartconstwas \{ V^{1/2}(x_0) + \MK V^{1/2}(x_0) \} \eqsp.
   \]
   Hence the sequence $\{\MK^n(x_0,\cdot)\}_{n=1}^\infty$ is a Cauchy sequence in the complete metric space of probability measure equipped with the Wasserstein distance $\wassersym[(\distance \wedge 1)^{\pcost}]$ (see \cite[Theorem~20.1.8]{douc:moulines:priouret:soulier:2018}). Therefore, there exists a probability measure $\pi$ on $(\Xset,\Xsigma)$ such that $\lim_{n \to \infty} \wasser[(\distance \wedge 1)^{\pcost}]{\MK^n(x_0,\cdot)}{\pi}=0$. It is easily seen that $\pi= \pi \MK$ (see e.g. \cite[Theorem~20.2.1]{douc:moulines:priouret:soulier:2018}), and the existence of invariant distribution is show. \Cref{assG:kernelP_q} implies for a stationary distribution $\pi$ that $\pi(V) \leq b / (1-\lambda)$ (see \cite[Lemma~14.1.10]{douc:moulines:priouret:soulier:2018}).
   \par
   Second, if $\pi$ and $\pi'$ are two invariant probability measures, \eqref{eq:cor:wasserstein-convergence_1} implies
$\wasser[\cost]{\pi}{\pi'}=0$. Hence, by \Cref{ass:cost_fun}, $\wassersym[(\distance \wedge 1)^{\pcost}](\pi,\pi')=0$ and finally $\pi=\pi'$. Now we complete the proof of \eqref{eq:wasser:geo:bound:pi} setting $\xi' = \pi$ in \eqref{eq:cor:wasserstein-convergence_1}.
\end{proof}




\subsection{Proof of \Cref{th:rosenthal_V_poly_wasserstein}}
\label{sec:proof:rosenthal_V_poly_wasserstein}
We preface the proof by technical lemmas.
\begin{lemma}
\label{lem: product of two funct}
Let $h_1 \in \Lclass_{\beta_1, \lyapW}$ and $h_2 \in \Lclass_{\beta_2, \lyapW}$ with $\beta_1, \beta_2 \in \rset_{+}$. Then 
\begin{equation*}
\Nnorm[\beta_1 + \beta_2, \lyapW]{h_1 h_2} \leq 2^{1+\beta_1 + \beta_2} \Nnorm[\beta_1, \lyapW]{h_1}  \Nnorm[\beta_2, \lyapW]{h_2}\eqsp.
\end{equation*}
\end{lemma}
\begin{proof}
Fix arbitrary $x, y \in \Xset$. Then
\begin{equation*}
\begin{split}
|h_1(x) h_2(x) - h_1(y) h_2(y)| &\leq |h_1(x) - h_1(y)||h_2(x)| + |h_2(x) - h_2(y)||h_1(y)| \\
& \leq \Nnorm[\beta_1, \lyapW]{h_1} \metricc^{1/2}(x,y) \bar \lyapW^{\beta_1}(x,y) \Nnorm[\beta_2, \lyapW]{h_2} \lyapW^{\beta_2}(x) \\
& + \Nnorm[\beta_2, \lyapW]{h_2} \metricc^{1/2}(x,y) \bar \lyapW^{\beta_2}(x,y) \Nnorm[\beta_1, \lyapW]{h_1} \lyapW^{\beta_1}(y)\eqsp.
\end{split}
\end{equation*}
Since for any $y \in \Xset$, $\lyapW(x) \leq 2\bar \lyapW(x,y)$ we get
\begin{equation*}
 |h_1(x) h_2(x) - h_1(y) h_2(y)| \le 2^{ 1 + \beta_1 + \beta_2} \Nnorm[\beta_1, \lyapW]{h_1}  \Nnorm[\beta_2, \lyapW]{h_2} \metricc^{1/2}(x,y) \bar \lyapW^{\beta_1+\beta_2}(x,y) \eqsp.
\end{equation*}
Similarly,
\begin{equation*}
|h_1(x) h_2(x)| \leq \Nnorm[\beta_1, \lyapW]{h_1}  \Nnorm[\beta_2, \lyapW]{h_2} \lyapW^{\beta_1 + \beta_2}(x)\eqsp.
\end{equation*}
The last two inequalities imply the statement.
\end{proof}

\begin{lemma}
\label{lem: P^n h class}
Assume \Cref{assG:kernelP_q} and \Cref{assG:kernelP_q_contractingset_m}. For any $p,q \in \nset$, $p \leq 2q$, $h \in \Lclass_{p/(4q),V}$ and $n \in \nset$, it holds that $\MK^n h - \pi(h) \in \Lclass_{p/(4q),V}$ with $\Nnorm[p/(4q), V]{\MK^n h} \leq \boundmetric^{m/2} \constlemqnpi^{p/(2q)}  \Nnorm[p/(4q),V]{h} \ratewas^{pn/(2q)}$, where the constant $\constlemqnpi$ is given by
\begin{equation}
\label{eq:kappa_alpha_def}
\constlemqnpi = \pi(V)^{1/2} \vartconstwas/\sqrt{2}\eqsp,
\end{equation}
and $\vartconstwas$ defined in \eqref{eq:def:rho}.
\end{lemma}
\begin{proof}
Applying \Cref{prop:wasser:geo} (equation \eqref{eq: constraction}), we get
\begin{align}
    |\MK^n h(x) - \MK^n h(x')| &\le \int \MKK^n(x,x', \rmd y \rmd y') |h(y) - h(y')| \nonumber\\
    & \leq \Nnorm[p/(4q), V]{h} \int \MKK^n(x,x'; \rmd y \rmd y') \metricc^{1/2}(y,y') \bar V^{p / (4q)}(y,y') \nonumber \\
    & \leq \Nnorm[p/(4q), V]{h} \boundmetric^{m/2} \vartconstwas^{p/(2q)} \ratewas^{pn/(2q)} \metricc^{1/2}(x,x') \bar{V}^{p/(4q)}(x,x')\eqsp. \label{eq:q_n_diff_bound}
\end{align}
Moreover, integrating \eqref{eq:q_n_diff_bound} w.r.t. $\pi$ and using that $\metricc(x,x') \leq 1$, we get
\begin{align}
| \MK^n h(x)  - \pi(h)|
&\leq \Nnorm[p/(4q), V]{h} \boundmetric^{m/2} \vartconstwas^{p/(2q)} \ratewas^{pn/(2q)} \int \bar{V}^{p/(4q)}(x,x') \pi(\rmd x') \nonumber \\
&\leq \boundmetric^{m/2} \vartconstwas^{p/(2q)} (\pi(V)/2)^{p/(4q)} \Nnorm[p/(4q), V]{h} \ratewas^{pn/(2q)} V^{p /(4q)}(x) \eqsp. \label{eq:q_n_pi_diff_bound}
\end{align}
In the last inequality we used that $V(x) + V(x') \leq V(x)V(x')$ since $V(x) \geq \rme$ for any $x \in \Xset$, and
\begin{align*}
\int \bar{V}^{p/(4q)}(x,x') \pi(\rmd x') = (V(x)/2)^{p/(4q)}\int V(x')^{p/(4q)} \pi(\rmd x') \leq (\pi(V)/2)^{p/(4q)}V^{p/(4q)}(x)\eqsp.
\end{align*}
Combining \eqref{eq:q_n_pi_diff_bound} and \eqref{eq:q_n_diff_bound} completes the proof.
\end{proof}

Based on \Cref{lem: product of two funct} and \Cref{lem: P^n h class}, we check Assumption \cref{assum:central_moments_bound}($q,\lyapW$) with the functional class $\Lclass_{1/(4q),V}$. The lemma below is an adaptation of \Cref{lem:centered_moments_Z_old}.
\begin{lemma}
\label{lem:centered_moments_Wasserstein_new}
Assume \Cref{assG:kernelP_q} and \Cref{assG:kernelP_q_contractingset_m}. Let $q \in \nset$. Then for any $k \in \{1,\ldots,2q\}$,  $(t_1,\dots,t_{k}) \in \{0,\ldots,n-1\}^{k}$, $t_1 \leq \dots\leq t_{k}$,   $(p_1,\dots,p_k) \in \nset^k$ satisfying  $p_i \geq 1$ for $i \in \{1,\dots,k\}$ and $\sum_{i=1}^k p_i \leq s$, and functions $\{h_{\ell}\}_{\ell=1}^{k}$ satisfying  $h_{i} \in  \Lclass_{p_i/(2s), V}$, $i \in \{1,\dots,k\}$, it holds
\begin{equation}
\label{eq:centred_moments_wasserstein_first_bound}
|\PEC_\pi[h_1(X_{t_1}), \ldots, h_k(X_{t_k})] |
\leq (2 \boundmetric^{m/2})^{k-1}  (2\constlemqnpi)^{\sum_{\ell=1}^k\sum_{j=\ell}^k p_{j}/s}
\ratewas^{\sum_{j=2}^{k}(t_j - t_{j-1})p_{j}/s} \prod_{\ell = 1}^k \Nnorm[p_{\ell}/(2s), V]{h_{\ell}} \eqsp,
\end{equation}
where $\constlemqnpi$ is defined in \eqref{eq:kappa_alpha_def}.
\end{lemma}
\begin{proof}
The proof is based on induction on $k \in \{1,\dots,2q\}$.  For $k = 1$, we get
\begin{equation*}
|\PEC_\pi[h_1(X_{t_1})]|
\leq \pi(V^{p_1/(2s)}) \|h_1\|_{V^{p_{1}/(2s)}}
\leq \{\pi(V)\}^{p_{1}/(2s)}
\Nnorm[p_1/(2s),V]{h_1}\eqsp,
\end{equation*}
where for the second inequality we used Jensen's inequality. Assume that~\eqref{eq:centred_moments_wasserstein_first_bound} holds for some $k \in \{1,\dots,2q-1\}$. Let $t_1 \leq \dots\leq t_{k+1}$,   $(p_1,\dots,p_{k+1})\in \nset^{k+1}$ satisfying  $p_i \geq 1$ for $i \in \{1,\dots,k+1\}$ and $\sum_{i=1}^{k+1} p_i \leq s$, and functions $h_{\ell} \in  \Lclass_{p_{\ell}/(2s),V}, \ell \in \{1,\dots,k+1\}$. Applying \Cref{lem:centred_moments_markov_property},
\begin{equation*}
\PEC_\pi[h_1(X_{t_1}), \ldots, h_k(X_{t_k}), h_{k+1}(X_{t_{k+1}})] = \PEC_\pi[h_1(X_{t_1}), \ldots , h_k(X_{t_k}) \tilde h_{k+1}(X_{t_k})]\eqsp,
\end{equation*}
where $\tilde h_{k+1}(x) = \MK^{t_{k+1} - t_k} h_{k+1}(x) - \pi(h_{k+1})$.
Since $h_{k+1} \in \Lclass_{p_{k+1}/(2s),V}$, we obtain using \Cref{lem: P^n h class} that $\tilde{h}_{k+1} \in \Lclass_{p_{k+1}/(2s),V}$ with
\begin{equation*}
\Nnorm[p_{k+1}/(2s),V]{\tilde h_{k+1}} \leq \boundmetric^{m/2} \constlemqnpi^{p_{k+1}/s}  \Nnorm[p_{k+1}/(2s), V]{h_{k+1}} \ratewas^{p_{k+1}(t_{k+1}-t_k)/s}\eqsp.
\end{equation*}
Hence, applying \Cref{lem: product of two funct} with $\lyapW = V$, $\beta_1 = p_{k}/(2s)$, and $\beta_2 = p_{k+1}/(2s)$
\begin{multline*}
\Nnorm[(p_k + p_{k+1})/(2s),V]{h_k \tilde h_{k+1}}
\leq 2^{1 + (p_k+p_{k+1})/(2s)} \boundmetric^{m/2} \constlemqnpi^{p_{k+1}/s} \ratewas^{p_{k+1}(t_{k+1}-t_k)/s} \\ \times \Nnorm[p_k/(2s), V]{h_k}  \Nnorm[p_{k+1}/(2s),V]{h_{k+1}} \eqsp.
\end{multline*}
Then, applying the induction hypothesis to $\bar{h}_i = h_i \in \Lclass_{\bar{p}_{i}/(2s),V}$, $\bar{p}_i = p_i$, $i \in\{1,\ldots,k-1\}$, $\bar{h}_k = h_k \tilde h_{k+1} \in \Lclass_{\bar{p}_{k}/(2s),V}$, $\bar{p}_k = p_k + p_{k+1}$ completes the proof.
\end{proof}

\begin{corollary}
\label{coro:centered_moments_wasserstein_new}
Assume \Cref{assG:kernelP_q} and \Cref{assG:kernelP_q_contractingset_m}. Then for any $q \in \nset$, \Cref{assum:central_moments_bound}($q,V,\Nnorm[1/(4q),V]{\cdot}$) is satisfied with $\ConstD[q,V] = 4 \boundmetric^{m/2}\constlemqnpi$, $\arate[q,V] = 0$ and $\rho_{q,V} = \ratewas^{1/2}$, where $\ratewas$ and $\constlemqnpi$ are defined in \eqref{eq:def:rho} and \eqref{eq:kappa_alpha_def}, respectively.
\end{corollary}
\begin{proof}
  Let $k \in \{1,\ldots,q\}$ and
 $(t_1,\dots,t_{k}) \in \{0,\ldots,n-1\}^{k}$, $t_1 \leq \dots\leq t_{k}$.
  Define $\maxind \in \{2,\ldots,k\}$ such that $t_{\maxind} - t_{\maxind-1} = \max_{j \in \{2,\ldots,k\}} [t_j - t_{j-1}]$. For $i \in \{1,\ldots,k\} \setminus \{\maxind\}$, we set $p_{i} = 1$, and put $p_{\maxind} = q$.
  Now we apply \Cref{lem:centered_moments_Wasserstein_new} with the mentioned $(p_1,\dots, p_k)$ and $s = 2q$.  Note that $ h_i \in \Lclass_{p_i/(4q), V}$ for any $i \in \{1,\ldots,k\}$ and $\sum_{i=1}^k p_i \leq 2q$. Moreover, $\Nnorm[1/4, V]{h_{\maxind}} \leq \Nnorm[1/(4q),V]{h_{\maxind}}$ since $q \geq 1$ and $V(x) > 1$. Therefore, the application of  \Cref{lem:centered_moments_Wasserstein_new} concludes the proof.
\end{proof}

\begin{proof}[Proof of \Cref{th:rosenthal_V_poly_wasserstein}] The proof now follows from \Cref{lem:centered_moments_Wasserstein_new} and \Cref{coro:centered_moments_wasserstein_new}.
\end{proof}


\subsection{Proof of \Cref{theo:changeofmeasure_wasser}}
\label{sec:proof-crefth_wass_change_mease}
Set $S_n' = \sum_{k=0}^{n-1} g_k(X_k')$. Using \Cref{assG:kernelP_q_contractingset_m}, we get
\begin{equation}
  \label{eq:sec:proof-crefth_wass_change_mease_1}
    \PE_\xi\big[ \big|S_n \big|^{2q} \big] = \PE_{\xi,\pi}^{\MKK}\big[ \big|S_n \big|^{2q} \big]   \leq 2^{2q-1} \PE_\pi\big[ \big|S_n \big|^{2q} \big]  +
     2^{2q-1} \PE_{\xi,\pi}^{\MKK}\big[ \big|S_n-S_n' \big|^{2q} \big]\eqsp,
  \end{equation}
where $\MKK$ is a kernel coupling. Using the Minkowski inequality and \Cref{prop:wasser:geo} completes the proof.

\subsection{Proof of \Cref{th:rosenthal_log_V_wasserstein}}
\label{sec:proof_ros_W_log}
\begin{lemma}
\label{lem:centered_moments_log_wasserstein}
Assume \Cref{assG:kernelP_q} and \Cref{assG:kernelP_q_contractingset_m}. Then for any $q \in \nset$ and $\gamma \geq 0$, \Cref{assum:central_moments_bound}($q,W^{\gamma},\Nnorm[1,W^\gamma]{\cdot}$) is satisfied with $\ConstDW[q,W^\gamma] = 2^{2+2\gamma}\gamma^{\gamma}\boundmetric^{m/2}\constlemqnpi$, $\arate[q,W^\gamma] = \gamma$ and $\rho_{q,W^{\gamma}} = \ratewas^{1/2}$, where $\ratewas$ and $\constlemqnpi$ are defined in \eqref{eq:def:rho} and \eqref{eq:kappa_alpha_def}, respectively.
\end{lemma}

\begin{proof}
Let $k \in \{1,\ldots,q\}$ and $I = (t_1,\dots,t_{k}) \in \{0,\ldots,n-1\}^{k}$, $t_1 \leq \dots\leq t_{k}$. Define $\maxind \in \{2,\ldots,k\}$ such that $t_{\maxind} - t_{\maxind-1} = \max_{j \in \{2,\ldots,k\}} [t_j - t_{j-1}]$. For $i \in \{1,\ldots,k\} \setminus \{\maxind\}$, we set $p_{i} = 1$, and put $p_{\maxind} = k$. Now we apply \Cref{lem:centered_moments_Wasserstein_new} with the mentioned $(p_1,\dots, p_k)$ and $s = 2k$. Proceeding as in \Cref{coro:centered_moments_wasserstein_new} with $\sum_{i=1}^k p_i \leq 2k$,
\begin{align*}
|\PEC_\pi[h_1(X_{t_1}), \ldots, h_k(X_{t_k})] |
\leq (4 \boundmetric^{m/2} \constlemqnpi)^{k} \ratewas^{\maxgap(I)/2} \prod_{\ell = 1}^k   \Nnorm[1/(4k), V]{h_{\ell}} \eqsp.
\end{align*}
To complete the proof it remains to note that for functions $h_{\ell} \in \Lclass_{1,W^{\gamma}}$ it holds
\begin{equation*}
\Nnorm[1/(4k), V]{h_{\ell}} \leq (4\gamma k/\rme)^{\gamma}\Nnorm[1, W^{\gamma}]{h_{\ell}}\eqsp.
\end{equation*}
\end{proof}

\begin{proof}[Proof of \Cref{th:rosenthal_log_V_wasserstein}] The proof now follows from \Cref{lem:centered_moments_Wasserstein_new} and \Cref{lem:centered_moments_log_wasserstein}.
\end{proof}
\subsection{Proof of \Cref{theo:changeofmeasure-1_wasser}}
\label{sec:proof-crefth-1_wass:theo:changeofmeasure-1_wasser}
Without loss of generality, we can assume that $\Nnorm[1, W^{\gamma}]{\bar{g}} = 1$. We set for any $x,x'\in\Xset$, $\bar{W}_{\gamma}(x,x') = 2^{-1}(W^{\gamma}(x)+ W^{\gamma}(x'))$.  Proceeding as in the proof of \Cref{theo:changeofmeasure_wasser}, \eqref{eq:sec:proof-crefth_wass_change_mease_1} holds and  we only need to bound $ \PE_{\xi,\pi}^{\MKK}\big[ \big|S_n-S_n' \big|^{2q} \big]$ with $S_n' = \sum_{k=0}^{n-1} g(X_k')$. Denote for any $k \in\{0,\ldots,n-1\}$, $c_k = \cost^{1/2}(X_k,X_k')$, $\Sigmabf_k = \sum_{l=0}^{k-1} c_l$ and $\Delta g_k = g(X_k)-g(X_k')$. First, we have
by Jensen inequality,
  \begin{align}
    \nonumber
    \big|S_n-S_n' \big|^{2q} &= \abs{\sum_{k=0}^{n-1} \{\Delta g_k\}}^{2q} \leq \Sigmabf_{n}^{2q-1} \defEns{\sum_{k=0}^{n-1} c_k \{\Delta g_k / c_k\}^{2q} }  \\
    \nonumber                             &\leq 2^{-1} \Sigmabf_{n}^{4q-2} \sum_{k=0}^{n-1} c_k  +  2^{-1} \sum_{k=0}^{n-1} c_k \{\Delta g_k / c_k\}^{4q}\\
                              \nonumber   & \leq 2^{-1} \Sigmabf_{n}^{4q-1}  +  2^{-1}\sum_{k=0}^{n-1} c_k\{ W^{4q \gamma}(X_k) + W^{4q\gamma}(X_k') \}\\
\label{eq:1:theo:changeofmeasure-1_wasser}                         & \leq 2^{-1} \Sigmabf_{n}^{4q-1}  +  \sqrt{2} \parenthese{8q \gamma/\rme}^{4q \gamma} \sum_{k=0}^{n-1} c_k\bar{V}^{1/2}(X_k,X_k') \eqsp,
  \end{align}
  where we used that  $(a+b)^{4q} \leq 2^{4q-1}\{a^{4q}+ b^{4q}\}$.
In addition by an easy induction on $\ell \in \{0,\ldots,n\}$, using that $\sup_{k \in \{0,\ldots,n-1\}} c_k \leq 1$, we have $ \Sigmabf_{\ell}^{4q-1} \leq \sum_{k=0}^{\ell-1} c_k (k+1)^{4q-1} \leq \sum_{k=0}^{\ell-1} c_k (k+1)^{4q-1} \bar{V}^{1/2}(X_k,X_k')$. Plugging this bound for $\ell=n$ in \eqref{eq:1:theo:changeofmeasure-1_wasser} and using \Cref{prop:wasser:geo} with $p=2q$ completes the proof.



\subsection{Proof of \Cref{th:rosenthal_log_V_cor_2_wasserstein}}
\label{sec:proof_bernstein_bound_wasserstein}
We proceed as in the proof of \Cref{th:rosenthal_log_V_cor_2}. Indeed, for any $k \geq 3$, \Cref{lem:centered_moments_log_wasserstein} implies
\begin{equation*}
\begin{split}
|\Gamma_{\pi, k}(S_n)|
&\leq \ratewas^{-1} 2^{k-1}\log^{1-k}\{1/\ratewas\} \ConstDW[q,W^{\gamma}]^{k} (k!)^{3+\gamma} n \Nnorm[1, W^{\gamma}]{\bar{g}}^{k} \\
&\leq \biggl(\frac{k!}{2}\biggr)^{3+\gamma}\PVar[\pi](S_n) \biggl( \frac{n \ratewas^{-1/2} \{\log(1/\ratewas)\}^{-1} \ConstDW[q,W^\gamma]^{2} \Nnorm[1, W^{\gamma}]{\bar{g}}^{2}}{\PVar[\pi](S_n)} \vee 1\biggr) \biggl( \frac{2 \ConstDW[q,W^\gamma] \Nnorm[1, W^{\gamma}]{\bar{g}}}{\log(1/\ratewas)} \biggr)^{k-2} \\
&\leq \biggl(\frac{k!}{2}\biggr)^{3+\gamma}\PVar[\pi](S_n)\,\ConstJW^{k-2}\eqsp,
\end{split}
\end{equation*}
with $\ConstDW[q,W^\gamma] = 2^{3/2+2\gamma}\gamma^{\gamma}\boundmetric^{m/2} \vartconstwas \{\pi(V)\}^{1/2}$ and $\ConstJW$ given in \eqref{eq:const_J_n_definition_main_was}. We conclude using \cite[Lemma~2.1]{bentkus:1980} (see also \cite[Equation~(24)]{doukhan2007probability}).

\subsection{Proof of \Cref{th:rosenthal_log_V_cor_2_wasserstein_non_statio}}
\label{sec:proof-crefth:r_th:rosenthal_log_V_cor_2_wasserstein_non_statio}
Without loss of generality, we can assume that $\Nnorm[1, W^{\gamma}]{\bar{g}} = 1$. Let $t \geq 0$, $g \in \Lclass_{1, W^\gamma}$ and  $\xi$ be a probability measure on $(\Xset,\Xsigma)$ satisfying $\xi(V^{1/2}) < \infty$.  First note that setting $S_n' = \sum_{k=0}^{n-1} g(X_k')$, we have using that $\MKK$ is a coupling kernel for $\MK$.
  \begin{equation}
    \label{eq:rosenthal_log_V_cor_2_wasserstein_non_statio_00}
  \PP_{\xi}(|S_n| \geq t) \leq \PP_{\pi}(|S_n| \geq t/2) + \PP_{\xi,\pi}^{\MKK}(|S_n-S_n'| \geq t/2)\eqsp,
\end{equation}
Set $c_k = \cost^{1/2}(X_k,X_k')$, $\Delta g_k = g(X_k)-g(X_k')$, $\Sigmabf^{(1/2)}_{\ell} = \sum_{\ell=0}^{k-1}c_{\ell}^{1/2}$ for $k \in\{0,\ldots,n-1\}$.
We distinguish the two cases $\gamma =0$ and $\gamma >0$.

First assume $\gamma >0$.
Then, we have setting $\varpi_{\gamma} = 1/(1+\gamma)$, using Young inequality with $1/\varpi_{\gamma} >1$ and since $\varpi_{\gamma}/(1+\varpi_{\gamma}) = 1/\gamma$ and $\bar{W}(x,x') = \{W(x) + W(x')\}/2$,
\begin{align}
  \nonumber
  |S_n-S_n'|^{\varpi_{\gamma}} &\leq \varpi_{\gamma} \Sigmabf^{(1/2)}_n  +(1-\varpi_{\gamma}) \defEns{\frac{1}{\Sigmabf^{(1/2)}_n} \sum_{k=0}^{n-1} \Delta g_k}^{1/\gamma} \\
  % \label{eq:rosenthal_log_V_cor_2_wasserstein_non_statio_1}
  \nonumber
                               & \leq \varpi_{\gamma} \Sigmabf^{(1/2)}_n  +(1-\varpi_{\gamma})\max_{k\in\{0,\ldots,n-1\}}\{ \Delta g_k/c_k^{1/2} \}^{1/\gamma} \\
    \label{eq:rosenthal_log_V_cor_2_wasserstein_non_statio_2}
   & \leq \varpi_{\gamma} \Sigmabf^{(1/2)}_n  +2 (1-\varpi_{\gamma}) \max_{k\in\{0,\ldots,n-1\}}\{ c_k^{1/(2\gamma)} [W(X_k)+ W(X_k')]\} \eqsp.
\end{align}
where we have used in \eqref{eq:rosenthal_log_V_cor_2_wasserstein_non_statio_2} that $(a+b)^u \leq 2^{(u-1)_+}(a^u+b^u)$ for $a,b \geq 1$, $u \geq 0$. It is easy to verify that \eqref{eq:rosenthal_log_V_cor_2_wasserstein_non_statio_2} still holds for $\gamma =0$.
Then, we get that for $\tilde{t} \geq 0$,
\begin{align}
   \label{eq:rosenthal_log_V_cor_2_wasserstein_non_statio_2_5}
  &\PP_{\xi,\pi}^{\MKK}(|S_n-S_n'| \geq \tilde{t}) \leq   \PP_{\xi,\pi}^{\MKK}(\varpi_{\gamma} \Sigmabf^{(\beta)}_n \geq \tilde{t}^{\varpi_{\gamma}}/2)\\
  \nonumber
&  \qquad \qquad \txts +   \PP_{\xi,\pi}^{\MKK}( (1-\varpi_{\gamma}) \max_{k\in\{0,\ldots,n-1\}}\{ c_k^{1/(2\gamma)} [W(X_k)+ W(X_k')]\} \geq \tilde{t}^{\varpi_{\gamma}}/4) \eqsp.
\end{align}
We now bound separately the two terms in the right hand side. Using that $\rme^{\uplambda_1 \sum_{k=0}^{n-1}a_k} \leq \uplambda_1 \sum_{k=0}^{n-1} a_k \rme^{\uplambda_1(k+1)} + 1$ for any $\{a_k\}_{k=0}^{n-1} \in\ccint{0,1}^n$ and Jensen's inequality, 
\begin{align*}
\txts  \PP_{\xi,\pi}^{\MKK}(\Sigmabf^{(1/2)}_n \geq \tilde{t}^{\varpi_{\gamma}}/(2\varpi_{\gamma})) &\leq \txts \rme^{- \uplambda_1 \tilde{t}^{\varpi_{\gamma}}/(2\varpi_{\gamma})}
                                                                                                \PE_{\xi,\pi}^{\MKK}[\rme^{\uplambda_1 \Sigmabf^{(1/2)}_n }] \\
  % &\txts\leq   \rme^{-\uplambda_1  t^{\varpi_{\gamma}}/(2\varpi_{\gamma})}  (1+\uplambda_1\PE_{\xi,\pi}^{\MKK}[ \sum_{k=0}^{n-1} c_k^{\beta} \rme^{\uplambda_1 (k+1)}]) \\
                                 \txts                                            & \txts
  \leq \rme^{-\uplambda_1  \tilde{t}^{\varpi_{\gamma}}/(2\varpi_{\gamma})} (1+ \uplambda_1 \sum_{k=0}^{n-1} \PE_{\xi,\pi}^{\MKK}[  c_k]^{1/2} \rme^{\uplambda_1 (k+1)}) \eqsp.
\end{align*}
Using \Cref{prop:wasser:geo} and since $c_k \leq c_k \bar{V}(X_k,X_k')$, we get with $\uplambda_1 = -\log(\ratewas)/4$ that
\begin{align}
  \nonumber
  &  \txts  \PP_{\xi,\pi}^{\MKK}(\Sigmabf^{(1/2)}_n \geq \tilde{t}^{\varpi_{\gamma}}/(2\varpi_{\gamma})) \\
  \label{eq:rosenthal_log_V_cor_2_wasserstein_non_statio_4}
  & \txts \qquad \leq \txts \rme^{\log(\ratewas) \tilde{t}^{\varpi_{\gamma}}/(8\varpi_{\gamma})}\defEns{1+(-\log(\ratewas)/4)\frac{[ \boundmetric^{m/2}  \vartconstwas \{\pi(V^{1/2}) + \xi(V^{1/2})\}]^{1/2}}{\ratewas^{1/4}(1-\ratewas^{1/4})}} \eqsp.
\end{align}


On the other hand,  we have for any $\uplambda_2 >0$,
\begin{align}
  \nonumber
 &\txts \PP_{\xi,\pi}^{\MKK}( (1-\varpi_{\gamma}) \max_{k\in\{0,\ldots,n-1\}}\{ c_k^{1/(2\gamma)} [W(X_k)+ W(X_k')]\} \geq \tilde{t}^{\varpi_{\gamma}}/4) \\
  \label{eq:rosenthal_log_V_cor_2_wasserstein_non_statio_3}
 &\qquad \qquad \txts \leq \exp\parenthese{-\frac{ \uplambda_2 \tilde{t}^{\varpi_{\gamma}}}{4(1-\varpi_{\gamma})}}  \PE_{\xi,\pi}^{\MKK}[\exp( \uplambda_2  \max_{k\in\{0,\ldots,n-1\}}A_k) ] \eqsp,
\end{align}
with $  A_k  =  c_k^{1/(2\gamma)} [W(X_k)+ W(X_k')]$.
Second, using $\rme^{u} -1 \leq u \rme^u$, and $\max_{k} c_k \leq 1$,
\begin{align*}
&  \PE_{\xi,\pi}^{\MKK}[\exp( \uplambda_2  \max_{k\in\{0,\ldots,n-1\}}A_k) ]-1 \leq \uplambda_2   \PE_{\xi,\pi}^{\MKK}[\max\limits_{k\in\{0,\ldots,n-1\}}A_k \exp( \uplambda_2 \max\limits_{k\in\{0,\ldots,n-1\}}A_k)] \\
  & \qquad\qquad\leq \uplambda_2 \sum_{k=0}^{n-1} \PE_{\xi,\pi}^{\MKK}[A_k \exp(\uplambda_2 A_k)] \\
  & \qquad\qquad \leq \uplambda_2 \sum_{k=0}^{n-1} \PE_{\xi,\pi}^{\MKK}[c_k^{1\wedge(2\gamma)^{-1}}[W(X_k)+ W(X_k')]\{V^{2\uplambda_2}(X_k) + V^{2\uplambda_2}(X_k')\}] \eqsp.
\end{align*}
Taking $\uplambda_2 = 8^{-1} \wedge (16 \gamma)^{-1}$, we obtain using Jensen inequality
\begin{align*}
  &  \PE_{\xi,\pi}^{\MKK}[\exp( \uplambda_2  \max_{k\in\{0,\ldots,n-1\}}A_k) ]-1  \leq (8^{-1} \wedge (16 \gamma)^{-1} ) \, 2\, \sup_{a \geq \rme} \{a^{4^{-1}\wedge (8\gamma)^{-1}}\log(a)\}\\
  & \qquad \qquad \times \sum_{k=0}^{n-1} \PE_{\xi,\pi}^{\MKK}[c_k^{1\wedge(2\gamma)^{-1}}\{V^{1/2(1\wedge(2\gamma)^{-1})}(X_k) + V^{1/2(1\wedge(2\gamma)^{-1})}(X_k')\}]\\
  & \leq (2^{-1} \wedge (4 \gamma)^{-1}) \sup_{a \geq \rme} \{a^{4^{-1}\wedge (8\gamma)^{-1}}\log(a)\} \sum_{k=0}^{n-1} \PE_{\xi,\pi}^{\MKK}[c_k\{V(X_k) + V(X_k')\}^{1/2}]^{(1\wedge 1/(2\gamma))} \eqsp.
\end{align*}
Using \Cref{prop:wasser:geo} and plugging the resulting bounds in \eqref{eq:rosenthal_log_V_cor_2_wasserstein_non_statio_00} completes the proof.


\subsection{Proof of \Cref{prop:pCN}}
\label{subsec:prop:pCN}
Let $\gausc = \muH(\ballH{0}{\tau})$. We use the following version of Fernique's theorem; see \citep[Theorem~2.8.5]{bogachev:1998}.
\begin{lemma}
\label{lem:bogachev:fernique}
Let $\muH$ be a centered Gaussian measure on  $(\msh,\mch)$. Then for any $\tau \in \rset_{+}$ such that $\gausc > 1/2$ and $\alphagaus = \log\{\gausc/(1-\gausc)\}/(24\tau^2)$ the following inequality holds
\begin{equation*}
\int_{\msh} \exp\bigl(\alphagaus \normH{x}^2\bigr) \rmd\muH(x) \leq \Constexpmoment_{\tau}\,,
\end{equation*}
where $\Constexpmoment_{\tau} = \gausc\bigl(\gausc/(1-\gausc)\bigr)^{1/24} + \gausc\bigl\{1 - (1/\gausc - 1)^{1 - (1+\sqrt{2})^2/6}\bigr\}^{-1}$. Moreover, for any $\betagaus \in \rset_{+}$ and $K \geq \betagaus/(2\alphagaus)$,
\begin{equation*}
\int_{\{\normH{y} \geq K\}}\exp\{\betagaus\normH{y}\}\rmd\muH(y) \leq \ConstC_{\alphagaus,\betagaus}\exp\{-\alphagaus K^2 + \betagaus K\}\eqsp,
\end{equation*}
where $\ConstC_{\tau,\betagaus} = \Constexpmoment_{\tau}\biggl(1 + \frac{\sqrt{\pi}\betagaus}{2\sqrt{\alphagaus}}\biggr)$.
\end{lemma}
\begin{proof}
The first part of the statement follows from \cite[Theorem~2.8.5]{bogachev:1998} by making the constants explicit. The second part of the statement follows from \cite[Proposition~$A.1.$]{hairer:stuart:vollmer:2012}.
\end{proof}

We first check the drift condition $\Cref{assG:kernelP_q}$. The proof essentially follows from \cite[Lemma~3.2]{hairer:stuart:vollmer:2012}, once again making  the constants explicit.
\begin{lemma}
Under the assumptions of \Cref{prop:pCN}, \Cref{assG:kernelP_q} holds with the constants
\begin{equation}
\label{eq:drift_constants_pcn}
\begin{split}
& \lambda = 1 - \muH(\ballH{0}{\constKone \rpCNconst^{a}})\bigl(1-
\exp{\bigl(-(1-\rhoH)\RpCN/2\bigr)}\bigr)\exp\left(\alphalpCN\right), \quad b = \constdriftpcnfirst \vee \constdriftpcnsecond, \\
& \constdriftpcnfirst = \Constexpmoment_{\tau}\exp\bigl\{\RpCN + (1-\rhoH^2)/(4\alphagaus)\bigr\}, \quad \constdriftpcnsecond = \ConstC_{\alphagaus,(1-\rhoH^2)^{1/2}}\exp\left\{g(\tstar) + (1-\rhoH^2)^{1/2}\constKone\right\}, \\
& g(t) = (\rhoH + (1-\rhoH^2)^{1/2} \constKone)t - \alphagaus \constKone^2 t^{2a}\eqsp, \quad
\tstar = \left(\frac{(1-\rhoH^2)^{1/2} \constKone + \rhoH}{2\alphagaus \constKone^2 a}\right)^{1/(2a-1)}, \\
& \constKone = \rpCNconst/(1-\rhoH^2)^{1/2}\eqsp, \quad \tau = \inf_{t \in \rset}\bigl\{\muH(\ballH{0}{t}) \geq 3/4\bigr\}, \\
& \Constexpmoment_{\tau} = (3/4)\left(3^{1/24} + \bigl\{1 - 3^{(1+\sqrt{2})^2/6 - 1}\bigr\}^{-1}\right)\eqsp, \quad \alphagaus = \log(3)/(24\tau^2)\eqsp.
\end{split}
\end{equation}
\end{lemma}
\begin{proof}
Let $V(x)= \exp(\normH{x})$ and $\prop(x,y) = \rhoH x + (1-\rhoH^2)^{1/2}y$. Note that it holds
\begin{equation}
\label{eq:MK_V_func_image}
\MK V(x) = \int_{\msh}\biggl(\exp\{\normH{x}\}(1-\alphaH(x,y)) + \exp\{\normH{\prop(x,y)}\}\alphaH(x,y)\biggr)\rmd \muH(y)\eqsp.
\end{equation}
Then for $x \in \ballH{0}{\RpCN}$, using $\normH{\prop(x,y)} \leq \normH{x} + (1-\rhoH^2)^{1/2}\normH{y}$, we get
\begin{align}
\MK V(x) &\leq \exp\{\normH{x}\} \int_{\msh} \exp\{(1-\rhoH^2)^{1/2}\normH{y}\} \rmd \muH(y) \nonumber \\
&\overset{(a)}{\leq} \exp\{\RpCN + (1-\rhoH^2)/(4\alphagaus)\} \int_{\msh} \exp\{\alphagaus \normH{y}^2\} \rmd \muH(y) \nonumber \\
&\overset{(b)}{\leq} \Constexpmoment_{\tau}\exp\bigl\{ \RpCN + (1-\rhoH^2)/(4\alphagaus)\bigr\} =: \constdriftpcnfirst \label{eq:const_b_1_pcn_def} \eqsp.
\end{align}
In the above, (a) is due to inequality $\exp\{\gamma t\} \leq \exp\{\gamma^2/(4\Delta) + \Delta t^2\}$, $t, \gamma, \Delta > 0$, and (b) is due to \Cref{lem:bogachev:fernique} applied with $\tau$ given in \eqref{eq:drift_constants_pcn}. Further, using \Cref{assum:d-small-set-pCN}, for $x \not\in \ballH{0}{\RpCN}$, it holds $\rpCNconst \normH{x}^a \leq (1-\rhoH)\normH{x}/2$. This implies
\begin{equation}
\label{eq:bound_V_ball}
\sup_{y \in \ballH{\rhoH x}{\rpCNconst \normH{x}^a}} V(y) \leq \smallconst V(x)\eqsp, \quad \smallconst = \exp{\bigl(-(1-\rhoH)\RpCN/2\bigr)}\eqsp.
\end{equation}
Let us now define $\eventA = \{y \in \msh | (1-\rhoH^2)^{1/2}\normH{y} \leq \rpCNconst\normH{x}^a\}$. Then for $x \not\in \ballH{0}{\RpCN}$ we use \eqref{eq:MK_V_func_image} and split integration over $\msh$ into integration over $\eventA$ and $\msh\setminus\eventA$. For $y \in \eventA, z(x,y) \in \ballH{\rhoH x}{\rpCNconst \normH{x}^a}$, thus \eqref{eq:bound_V_ball} implies
\begin{align*}
& \int_{\eventA}\biggl(\exp\{\normH{x}\}(1-\alphaH(x,y)) + \exp\{\normH{\prop(x,y)}\}\alphaH(x,y)\biggr)\rmd \muH(y) \\
&\leq \int_{\eventA}\biggl(\exp\{\normH{x}\}(1-\alphaH(x,y)) + \smallconst \exp\{\normH{x}\} \alphaH(x,y)\biggr)\rmd \muH(y) \\
&= \exp\{\normH{x}\}\bigl(\muH(\eventA) - (1 - \smallconst)\int_{\eventA}\alphaH(x,y)\rmd \muH(y)\bigr) \\
&\overset{(a)}{\leq}  \exp\{\normH{x}\}\muH(\eventA)\bigl(1 - (1 - \smallconst)\exp\left(\alphalpCN\right)\bigr)\eqsp.
\end{align*}
In the above, (a) is due to lower bound on $\alphaH(x,y)$, which follows from \Cref{assum:d-small-set-pCN}. Setting $\constKone = \rpCNconst/(1-\rhoH^2)^{1/2}$ and using \Cref{lem:bogachev:fernique} with $K = \constKone \normH{x}^a$ and $\betagaus = (1-\rhoH^2)^{1/2}$,
\begin{align*}
& \int_{\msh\setminus\eventA}\biggl(\exp\{\normH{x}\}(1-\alphaH(x,y)) + \exp\{\normH{\prop(x,y)}\}\alphaH(x,y)\biggr)\rmd \muH(y) \\
& \quad \leq (1 - \muH(\eventA))\exp\{\normH{x}\} + \exp\{\rhoH\normH{x}\}\int_{\msh\setminus\eventA}\exp\{\betagaus\normH{y}\} \rmd \muH(y) \\
& \quad \leq (1 - \muH(\eventA))\exp\{\normH{x}\} + \ConstC_{\tau,\betagaus}\exp\biggl\{\rhoH\normH{x} - \alphagaus \constKone^2 \normH{x}^{2a} + \betagaus \constKone \normH{x}^a\biggr\}\eqsp,
\end{align*}
where $\alphagaus$ is defined in \eqref{eq:drift_constants_pcn} and $\ConstC_{\tau,\betagaus}$ defined in \Cref{lem:bogachev:fernique}. Combining the above inequalities and \eqref{eq:MK_V_func_image}, for $x \not\in \ballH{0}{\RpCN}$,
\begin{equation}
 \label{eq:bound_outside_ball}
 \begin{split}
\MK V(x) &\leq \bigl(1 - \muH(A)(1-\smallconst)\exp\left(\alphalpCN\right)\bigr)V(x) \\
&+ \ConstC_{\alphagaus,\betagaus}\sup_{t \geq 0}\exp\biggl\{\rhoH t + \betagaus \constKone t^{a} - \alphagaus \constKone^2 t^{2a}\biggr\}\eqsp.
 \end{split}
\end{equation}
We complete the proof combining \eqref{eq:const_b_1_pcn_def}, \eqref{eq:bound_outside_ball}, and noting that $\muH(A) \geq \muH(\ballH{0}{\constKone \rpCNconst^{a}})$.
\end{proof}



\textbf{Proof of \Cref{ass:cost_fun} and \Cref{assG:kernelP_q_contractingset_m}.} It is easy to see that assumption \Cref{ass:cost_fun} is satisfied with $\cost(x,x') = 1 \wedge [\normH{x-x'}/\varepsilonH]$ and $\pcost=1$ as soon as $\varepsilonH \leq 1$. In our proof of \Cref{assG:kernelP_q_contractingset_m} we use the synchronous coupling suggested in \cite[Section~3.1.2]{hairer:stuart:vollmer:2012},
\begin{align*}
  X_{k+1} &= X_k \indiacc{U_{k+1} > \alphaH(X_k,Z_{k+1}) }+  \defEns{\rhoH X_k + (1-\rhoH^2)^{1/2} Z_{k+1}} \indiacc{U_{k+1} \leq \alphaH(X_k,Z_{k+1})}, \, X_0 = x \\
  X'_{k+1} &= X'_k \indiacc{U_{k+1} > \alphaH(X'_k,Z_{k+1}) }+  \defEns{\rhoH X'_k + (1-\rhoH^2)^{1/2} Z_{k+1}} \indiacc{U_{k+1} \leq \alphaH(X'_k,Z_{k+1})}, X'_0 = x'\eqsp.
\end{align*}
The associated coupling kernel is denoted $\MKK$. The first part of \Cref{assG:kernelP_q_contractingset_m} follows from the following lemma:
\begin{lemma}
\label{lem:pcn_contract_1step}
Let $\cost(x,x') = 1 \wedge [\normH{x-x'}/\varepsilonH]$ with
\begin{equation}
\label{eq:varepsilon_H_def}
\varepsilonH = \pcngamma/(2\Lippcn)\eqsp,
\end{equation}
where we have introduced
\begin{equation}
\label{eq:contact_small_pcn_const}
\begin{split}
&\pcngamma = \left(\lbprobpcn \muH\left(\ballH{0}{(1-\rhoH^2)^{-1/2}\RpCN}\right) \wedge \exp\left(\alphalpCN\right)\muH(\ballH{0}{\constKone \RpCN^a}) \right)(1-\rhoH)/2 \\
&\lbprobpcn = \exp\bigl\{-\sup_{y \in \ballH{0}{2\RpCN+1}}\potU(y) + \inf_{y \in \ballH{0}{2\RpCN+1}}\potU(y)\bigr\}\eqsp,
\end{split}
\end{equation}
where $\constKone$ is defined in \eqref{eq:drift_constants_pcn}. Then, for $x,x' \in \msh, \normH{x-x'} \geq \varepsilonH$, it holds
\begin{equation*}
\MKK \cost(x,x') \leq \cost(x,x')\eqsp.
\end{equation*}
Moreover, for $x,x' \in \msh, \normH{x-x'} < \varepsilonH$, it holds
\begin{equation*}
\MKK \cost(x,x') \leq (1- \pcngamma)\cost(x,x')\eqsp,
\end{equation*}
\end{lemma}
\begin{proof}
Let $\normH{x-x'} \geq \varepsilonH$ then $\cost(x,x') = 1$. The statement follows from $\MKK \cost(x,x') = \PE[\cost(X_1,X'_1)] \leq 1$. Consider the case $\cost(x,x') < 1$. Since $\varepsilonH \leq 1$ then either $x,x' \in \ballH{0}{\RpCN+1}$ or $x,x' \not \in \ballH{0}{\RpCN}$. We consider these cases separately. Let $x,x' \in \ballH{0}{\RpCN+1}$. Introduce the following events, $\eventA = \{(1-\rhoH^2)^{1/2}\normH{Z_1} \leq \RpCN\}$, $\eventB_1 = \{U_{1} \leq \alphaH(x,Z_{1}),U_{1} \leq \alphaH(x',Z_{1})\}$, $\eventB_2 = \{U_{1} > \alphaH(x,Z_{1}),U_{1} > \alphaH(x',Z_{1})\}$ and $\eventB_3 = \RandSpace \setminus (\eventB_1 \cup \eventB_2)$. Note that $\indiacc{\eventB_1}\cost(X_1,X'_1) = \rhoH\cost(x,x')$ and $\indiacc{\eventB_2}\cost(X_1,X'_1) = \cost(x,x')$. We get
\begin{align}
\MKK \cost(x,x')
&= \PE[\indiacc{\eventA \cap \eventB_1}\cost(X_1,X'_1)] + \PE[\indiacc{\eventA \cap \eventB_2}\cost(X_1,X'_1)] + \PE[\indiacc{\overline{\eventA} \cap (\eventB_1 \cup \eventB_2)}\cost(X_1,X'_1)] \nonumber \\
&\qquad + \PE[\indiacc{\eventB_3}\cost(X_1,X'_1)] \nonumber \\
&\overset{(a)}{\leq} \PP(\eventA)\bigl(\PP(\eventB_1|\eventA)\rhoH\cost(x,x') + \PP(\eventB_2|\eventA)\cost(x,x')\bigr) + (1-\PP(\eventA))\cost(x,x') \label{eq:Kc_bound} \\
&\qquad +\int_{\msh}\bigl|\alphaH(x,y) - \alphaH(x',y)\bigr| \rmd \muH(y) \nonumber\eqsp.
\end{align}
Here, (a) follows from the representation
\begin{align*}
\PE[\indiacc{\eventB_3}\cost(X_1,X'_1)]
&= \int_{\msh}\int_{0}^{1}\bigl[\cost(x,\rhoH x' + (1-\rhoH^2)^{1/2}y)\indiacc{\alphaH(x',y) \leq u \leq \alphaH(x,y)} \\
&\qquad +\cost(\rhoH x + (1-\rhoH^2)^{1/2}y,x')\indiacc{\alphaH(x,y) \leq u \leq \alphaH(x',y)}\bigr]\,\rmd u \rmd \muH(y)\eqsp.
\end{align*}
We use $\PP(\eventB_2|\eventA) \leq 1 - \PP(\eventB_1|\eventA)$ together with with $\PP(\eventB_1|\eventA) \geq \lbprobpcn$. The latter follows from \eqref{eq:accept_pcn} and definition of the set $\eventA$. Since $f(t) = 1 \wedge \exp\{t\}$ is $1$-Lipschitz we may use definition \eqref{eq:accept_pcn} and \Cref{assum:potU-lipshitz} to obtain
\begin{align*}
&\int_{\msh}\bigl|\alphaH(x,y) - \alphaH(x',y)\bigr| \rmd \muH(y) \leq \int_{\msh}\biggl| 1 \wedge \exp\parenthese{-\potU(\rhoH x+ (1-\rhoH^2)^{1/2} y) + \potU(x)} - \\
&\qquad 1 \wedge \exp\parenthese{-\potU(\rhoH x'+ (1-\rhoH^2)^{1/2} y) + \potU(x')}\biggr| \rmd \muH(y) \leq  2\Lippcn\normH{x-x'} \leq 2\varepsilonH\Lippcn\cost(x,x')\eqsp.
\end{align*}
Putting together the obtained inequalities, we arrive at an estimate of the form
\begin{align*}
\MKK \cost(x,x') \leq \bigl(1 - \lbprobpcn \muH\bigl(\ballH{0}{\RpCN}\bigr)(1-\rhoH) + 2 \varepsilonH \Lippcn\bigr)\cost(x,x') \leq (1- \pcngamma)\cost(x,x').
\end{align*}
The last inequality follows from the choice of $\varepsilonH$.
\par
Consider the case $x,x' \not \in \ballH{0}{\RpCN}$. Define $\eventC = \{\omega \in \RandSpace | (1-\rhoH^2)^{1/2}\normH{Z_1(\omega)} \leq \rpCNconst(\normH{x} \wedge \normH{x'})^{a}\}$. Repeating the argument \eqref{eq:Kc_bound} we get
\begin{align*}
\MKK \cost(x,x')
&\leq \PP(\eventC)\bigl(\PP(\eventB_1|\eventC)\rhoH\cost(x,x') + \PP(\eventB_2|\eventC)\cost(x,x')\bigr) + (1-\PP(\eventC))\cost(x,x') \\
&\qquad +\int_{\msh}\bigl|\alphaH(x,y) - \alphaH(x',y)\bigr| \rmd \muH(y) \nonumber\eqsp.
\end{align*}
To complete the proof it remains to note that $\PP(\eventB_2|\eventC) \leq 1 - \PP(\eventB_1|\eventC) \le \exp\left(\alphalpCN\right)$, where the last inequality follows from \Cref{assum:d-small-set-pCN}.
\end{proof}

Now we check the second part of \Cref{assG:kernelP_q_contractingset_m}.
\begin{lemma}
Under the assumptions of \Cref{prop:pCN}, it holds
\begin{equation}
\label{eq:K_m_bound}
\MKK^m \cost(x,x') \leq (1 - \minorwas \indi{\CKset}(x,x'))\cost(x,x')\eqsp,
\end{equation}
where $\CKset$, $\Rassumapcn$ and $m$ are defined in \Cref{prop:pCN},
\begin{equation}
\label{eq:epsilon_pcn_def}
\minorwas = \pcngamma \wedge \left(\lbprobpcn \muH(\ballH{0}{\Rassumapcn_{m}})\right)^{m}/2\eqsp, \quad \Rassumapcn_{m} = \frac{\Rassumapcn}{m(1-\rhoH^2)^{1/2}}\eqsp,
\end{equation}
and $\pcngamma$, $\lbprobpcn$ are defined in \eqref{eq:contact_small_pcn_const}.
\end{lemma}
\begin{proof}
Note that in case $(x,x') \not\in \CKset$ we can use \Cref{lem:pcn_contract_1step} which implies $\MKK^{m}\metricc(x,x') \leq \metricc(x,x')$ for any $m \in \nset$. Hence, \eqref{eq:K_m_bound} follows.
\par
Assume that $(x,x') \in \CKset$. Consider first the case $\metricc(x,x') = 1$, that is, $\normH{x-x'} \geq \varepsilonH$. Let $m \in \nset$ be a number to be chosen later and introduce $\eventB_{m} = \{ U_{k} \leq \alphaH(X_k,Z_{k}),U_{k} \leq \alphaH(X'_k,Z_{k}), \, k = 1,\dots,m\}$. Note that $\eventB_{m}$ is an event where first $m$ proposals were accepted both for sequence $(X_{k})_{k \in \nset}$ and $(X'_{k})_{k \in \nset}$. Then, using that $\cost(x,y) \leq \normH{x-y}/\varepsilonH$ and $\cost(x,y) \leq 1, \, x,y \in \msh$, we get
\begin{align}
\MKK^{m}\metricc(x,x')
&= \PE[\cost(X_{m},X'_{m})] = \PE[\cost(X_{m},X'_{m})\indiacc{\eventB_m}] + \PE[\cost(X_{m},X'_{m})\indiacc{\overline{\eventB}_m}] \nonumber \\
&\leq \PP(\eventB_m)\rhoH^{m}\normH{x-x'}/\varepsilonH + 1-\PP(\eventB_m)\eqsp. \label{eq:K_m_bound_intermediate}
\end{align}
We choose $m = \log(\varepsilonH/(4\Rassumapcn))/\log\rhoH$ and use $\metricc(x,x') = 1$. Then \eqref{eq:K_m_bound_intermediate} implies
\begin{equation}
\label{eq:MKK_m_step_bound_m_step_prob}
\MKK^{m}\metricc(x,x') \leq \bigl(1 - \PP(\eventB_{m})/2\bigr)\metricc(x,x')\eqsp.
\end{equation}
It remains to lower bound $\PP(\eventB_{m})$. Recall the definition \eqref{eq:epsilon_pcn_def} of $\Rassumapcn_{m}$. It follows
\begin{align*}
\PP(\eventB_{m}) &\geq \PP\bigl(\eventB_{m} | \cap_{k=1}^{m}\bigl\{\normH{Z_{k}} \leq \Rassumapcn_{m}\bigr\}\bigr)\PP(\normH{Z_1} \leq \Rassumapcn_{m})^{m} \\
&\geq \left(\lbprobpcn \muH(\ballH{0}{\Rassumapcn_{m}})\right)^{m}\eqsp,
\end{align*}
where $\lbprobpcn$ is defined in \eqref{eq:contact_small_pcn_const}. In case $\metricc(x,x') < 1$, \Cref{lem:pcn_contract_1step} implies
\begin{equation}
\label{eq:MKK_m_step_bound_contractive}
\MKK^{m}\metricc(x,x') \leq \MKK \metricc(x,x') \leq (1-\pcngamma)\metricc(x,x')\eqsp.
\end{equation}
Now the statement follows by combining \eqref{eq:MKK_m_step_bound_m_step_prob} and \eqref{eq:MKK_m_step_bound_contractive}.
\end{proof}

\subsection{Proof of \Cref{theo:SGD}}
\label{sec:proof_drift_sgd}
The proof of \Cref{ass:cost_fun} is immediate. We preface the proof of the drift condition \Cref{assG:kernelP_q} by the following instrumental lemma. Recall that $\kapf= \muf \Lf/ (\muf+\Lf)$.
\begin{lemma}
\label{lem:descent-SGD}
Assume \Cref{ass:sgd_field}. Then, for all $\gamma \in \ocint{0,1/(\muf+\Lf)}$ and $k \in \nset$,
\begin{equation*}
\norm{\theta_{k+1} - \thetas}^2 \leq (1 - \gamma \kapf) \norm{\theta_k - \thetas}^2
+ (\gamma / \kapf + 2 \gamma^2) \norm{\fieldH_{\theta_k}(\YSGD_{k+1}) - \nabla \objf(\theta_k)}^2 \eqsp.
\end{equation*}
\end{lemma}
\begin{proof}
Expanding the recurrence \eqref{eq:def_SGD} and using $\nabla \objf(\thetas) = 0$,
\begin{multline*}
\norm{\theta_{k+1}- \thetas}^2 \leq \| \theta_k - \thetas \|^2 - 2 \gamma \ps{\fieldH_{\theta_k}(\YSGD_{k+1}) - \nabla \objf(\theta_k)}{\theta_k - \thetas} - 2 \gamma \ps{\nabla \objf(\theta_k) - \nabla \objf(\thetas)}{\theta_k - \thetas} \\
+ 2 \gamma^2 \norm{\fieldH_{\theta_k}(\YSGD_{k+1}) - \nabla \objf(\theta_k)}^2 + 2 \gamma^2 \norm{\nabla \objf(\theta_k) - \nabla \objf(\thetas)}^2 \eqsp.
\end{multline*}
\cite[Theorem~2.1.12]{nesterov:2004} shows that, for all $(\theta,\theta') \in \rset^{2d}$,
\[
\ps{\nabla \objf(\theta)-\nabla \objf(\theta')}{\theta-\theta'} \geq \kapf \norm{\theta-\theta'}^{2}+\frac{1}{\muf+\Lf}\norm{\nabla \objf(\theta)-\nabla \objf(\theta')}^{2} \eqsp.
\]
Using that $\gamma \leq 1/(\muf+\Lf)$, and $|\ps{a}{b}| \leq (\varepsilon^2/2) \norm{a}^2 + 1/(2 \varepsilon^2) \norm{b}^2$ for $\varepsilon >0$ and $(a,b) \in \rset^{2d}$, we get
\[
\norm{\theta_{k+1}-\thetas}^2 \leq ( 1 - 2 \gamma \kapf + \gamma \varepsilon^2)
\norm{\theta_k - \thetas}^2 + (\gamma/ \varepsilon^2 + 2 \gamma^2) \norm{\fieldH_\theta(\YSGD_{k+1}) - \nabla \objf(\theta_k)}^2 \eqsp.
\]
We complete the proof by taking $\varepsilon^2 = \kapf$.
\end{proof}
Now we are ready to check \Cref{assG:kernelP_q}.

\begin{lemma}
Under the assumptions of \Cref{theo:SGD}, \Cref{assG:kernelP_q} holds with the constants $\lambda$ and $b$ given by 
\begin{equation}
\label{eq:ass_A1_PR_constants}
\begin{split}
\lambda &= \rme^{-\gamma\kapf/(2\tsgvarfac)}\eqsp, \quad \tsgvarfac = 2\sgvarfac(\rme+1)/(\rme-1)\eqsp, \\
b &= \gamma\bigl(1/\kapf + 2 \gamma + \kapf/(2\tsgvarfac)\bigr)\exp\left(2 + (2\tsgvarfac)^{-1} + (2\gamma \kapf+1)/\kapf^2\right)\eqsp.
\end{split}
\end{equation}
\end{lemma}
\begin{proof}
Under \Cref{ass:sgd_noise_exp_mom}, \Cref{subsec:proof:eq:bound_exp_sgd} implies for all $\theta \in \rset^{\dims}$
\begin{equation}
  \label{eq:bound_exp_sgd}
  \PE\parentheseDeux{\exp\parenthese{\normLigne{\fieldH_{\theta}(\YSGD) - \nabla \objf(\theta)}^2/\tsgvarfac}}
  \leq \rme \eqsp, \text{ with } \tsgvarfac = 2\sgvarfac(\rme+1)/(\rme-1) \eqsp.
\end{equation}
In particular, Jensen's inequality implies
  \begin{equation}
    \label{eq:bound_var_sgd}
    \PE[\normLigne{\fieldH_{\theta}(\YSGD) - \nabla \objf(\theta)}^2] \leq \tsgvarfac \eqsp.
  \end{equation}
Applying Jensen's inequality one more time to \eqref{eq:bound_exp_sgd}, and using $\gamma/\kapf + 2\gamma^2 \leq 1$,
\begin{equation*}
\PE\left[\exp\parenthese{(\gamma / \kapf + 2 \gamma^2) \norm{\fieldH_{\theta_k}(\YSGD_{k+1}) - \nabla \objf(\theta_k)}^2/\tsgvarfac}\right] \leq \rme^{\gamma / \kapf + 2 \gamma^2} \eqsp.
\end{equation*}
\Cref{lem:descent-SGD} implies that for any $\theta \in \rset^d$,
\begin{equation}
\label{eq:Q_gamma_image_bound}
\MKSGD_{\gamma} V_{1/\tsgvarfac}(\theta) \leq \exp\parenthese{-(\gamma \kapf/\tsgvarfac) \norm{\theta - \thetas}^2} \rme^{\gamma / \kapf + 2 \gamma^2} V_{1/\tsgvarfac}(\theta)   \eqsp.
\end{equation}
Now we consider the two cases separately. For $\theta \in\rset^d$ satisfying $\norm{\theta-\thetas}^2 \geq M_{\objf} = 1/2 + (2\gamma \kapf+1)\tsgvarfac/\kapf^2$, we have
\begin{equation}
\label{eq:7_drift_sgd_1}
\MKSGD_{\gamma} V_{1/\tsgvarfac}(\theta) \leq \rme^{-\gamma\kapf/(2\tsgvarfac)} V_{1/\tsgvarfac}(\theta) \eqsp.
\end{equation}
On the other hand, for $\theta \in \rset^d$ satisfying $\norm{\theta-\thetas}^2 \leq M_{\objf}$, we have using \eqref{eq:Q_gamma_image_bound} and $\rme^t -1 \leq t \rme^t$,
\begin{align}
\nonumber
\MKSGD_{\gamma} V_{1/\tsgvarfac}(\theta) &\leq \rme^{-\gamma\kapf/(2\tsgvarfac)} V_{1/\tsgvarfac}(\theta) + \{\rme^{\gamma / \kapf + 2 \gamma^2}-\rme^{-\gamma\kapf/(2\tsgvarfac)}\}V_{1/\tsgvarfac}(\theta) \\
\label{eq:8_drift_sgd_2}
& \leq \rme^{-\gamma\kapf/(2\tsgvarfac)} V_{1/\tsgvarfac}(\theta) + \gamma\bigl(1/\kapf + 2 \gamma + \kapf/(2\tsgvarfac)\bigr) \exp(2 + M_{\objf}/\tsgvarfac) \eqsp.
\end{align}
Combining \eqref{eq:7_drift_sgd_1} and \eqref{eq:8_drift_sgd_2} implies the statement.
\end{proof}

To check \Cref{assG:kernelP_q_contractingset_m}, we use the following synchronous coupling construction
\begin{equation*}
 \theta_{k+1} = \theta_k - \gamma  \fieldH_{\theta_k}(\YSGD_{k+1}) \quad \text{and}
 \quad  \theta'_{k+1} = \theta'_k - \gamma  \fieldH_{\theta_k'}(\YSGD_{k+1}) \eqsp,
\end{equation*}
\ie\ we use the same noise $Y_{k+1}$ at each iteration. We denote by $\MKKSGD_\gamma$ the associated coupling kernel.

%Checking assumption A3
\begin{lemma}
Under the assumptions of \Cref{theo:SGD}, it holds
\begin{equation*}
\MKKSGD_\gamma^m \cost(\theta,\theta') \leq (1 - \minorwas \indi{\CKset}(\theta,\theta'))\cost(\theta,\theta')\eqsp,
\end{equation*}
where $\CKset$, $\minorwas$, and $m$ are defined in \Cref{theo:SGD}.
\end{lemma}
\begin{proof}
We first note that \cite[Proposition 2]{dieuleveut2020bridging} implies that for any $\theta,\theta' \in \rset^{\dims}$, it holds
\begin{align*}
\MKKSGD_\gamma\norm{\theta - \theta'}^2 \leq \bigl(1 - \minorwas\bigr) \norm{\theta - \theta'}^2\eqsp,
\end{align*}
where $\minorwas = 2\muf \gamma (1-\gamma \Lf/2)$. Hence, in case $\theta,\theta' \in \rset^{\dims}$ with $\cost(\theta,\theta') < 1$ it holds
\begin{align*}
\MKKSGD_\gamma^{m}\cost(\theta,\theta') \leq \MKKSGD_\gamma^{m}\norm{\theta - \theta'}^2 \leq (1-\minorwas)\cost(\theta,\theta')
\end{align*}
for any $m \in \nset$. Consider now $\theta,\theta'$ with $\cost(\theta,\theta') = 1$. Then, with $m = \lceil \log(4\Rsgd^2)/\log(1/(1-\minorwas)) + 1 \rceil$,
\begin{align*}
\MKKSGD_\gamma^{m}\cost(\theta,\theta') \leq \MKKSGD_\gamma^{m}\norm{\theta - \theta'}^2 \leq (1-\minorwas)^{m}\cost(\theta,\theta') \leq 4\Rsgd^2(1-\minorwas)^{m} < (1-\varepsilon)\cost(\theta,\theta')
\end{align*}
\end{proof}

\subsection{Proof of \Cref{propo:bias}}
\label{sec:proof-crefpropo:bias}
  Let $\gamma \in \ooint{0,1/\Lf}$.
  Consider $\theta_0$ with distribution $\pi_{\gamma}$   and $\theta_1$ defined by \eqref{eq:def_SGD}. Then, by \Cref{theo:SGD}, $\theta_1$ has also distribution $\pi_{\gamma}$ which implies taking expectation in \eqref{eq:def_SGD} and rearranging terms that
  \begin{equation}
    \label{eq:4}
    0 = \int_{\rset^d} \nabla \objf(\theta) \rmd \pi_{\gamma}(\theta) \eqsp.
  \end{equation}
  In addition, \cite[]{nesterov:2004} implies that for any $\theta \in\rset^d$,
  \begin{equation*}
    \norm{\nabla \objf(\theta) - \nabla \objf(\thetas) - \nabla^2 \objf(\thetas)\{\theta - \thetas\}} \leq \LipHessianf\norm{\theta-\thetas}/2 \eqsp.
  \end{equation*}
  Plugging this result in \eqref{eq:4} and using $\nabla \objf(\thetas) =0$, we obtain that
  \begin{equation*}
    \label{eq:5}
    \norm{\int_{\rset^d} \nabla^2 \objf(\thetas)\{\theta - \thetas\} \rmd \pi_{\gamma}(\theta)} \leq
    (\LipHessianf/2)\int_{\rset^d} \norm{\theta-\thetas}^2  \rmd \pi_{\gamma}(\theta) \eqsp.
  \end{equation*}
Using   $\norm{\nabla^2 \objf(\thetas)\{\theta - \thetas\} } \geq \muf \norm{\theta-\thetas}$ for any $\theta \in\rset^d$  by \Cref{ass:sgd_field}, Jensen inequality and \Cref{theo:SGD}  complete the proof.

\section{Auxiliary proofs}
%definition of norm-subgaussian vector
We start this part with a technical lemma about subgaussian vectors.
\begin{lemma}
\label{subsec:proof:eq:bound_exp_sgd}
Let $Y \in \rset^d$ be a norm-subgaussian vector with variance factor $\sigma^2 < \infty$. Then
\[
\PE\left[\exp\parenthese{\normLigne{Y}^2/\sigmaconst^2}\right] \leq \rme
\]
with $\sigmaconst^2 = 2\sigma^2(\rme+1)/(\rme-1)$.
\end{lemma}
\begin{proof}
By the definition of norm-subgaussian vector (see \Cref{ass:sgd_noise_exp_mom}), it holds for $c > \sigma\sqrt{2}$ that
\begin{align*}
\PE\left[\exp\parenthese{\normLigne{Y}^2/c^2}\right]
&= \int_{0}^{+\infty}\PP\left(\exp\parenthese{\normLigne{Y}^2/c^2} \geq t\right)\rmd t \\
&\leq 1 + (2/c^2)\int_{0}^{+\infty}\PP\left(\normLigne{Y} \geq u\right)\exp\parenthese{u^2/c^2}u\rmd u \\
&\leq 1 + (4/c^2)\int_{0}^{+\infty}\exp\parenthese{-u^2\bigl(1/(2\sigma^2) - 1/c^2\bigr)}u\rmd u = 1 + \frac{4\sigma^2}{c^2-2\sigma^2}\,.
\end{align*}
Now the proof follows by letting $c^2 = \sigmaconst^2$.
\end{proof}
\begin{lemma}
\label{lem:rate_UGE}
Assume \Cref{assG:kernelP_q} and \Cref{assG:kernelP_q_smallset}. Then for any $\alpha \in (0,1]$, $x \in \Xset$, $n \in \nset$, it holds that
\begin{equation}
\label{eq:V-geometric-better-rate-appendix}
\Vnorm[V^\alpha]{\MK^n(x, \cdot) - \pi}
 \leq 2 \{\cmconstv \ratev^n \pi(V)   V(x) \}^{\alpha}
\eqsp.
\end{equation}
\end{lemma}
\begin{proof}
Let $\xi$ and $\xi'$ be arbitrary measures such that $\xi(V), \xi'(V) < \infty$. Then, using the definition of the $V$-norm and Jensen's inequality,
\begin{align*}
\Vnorm[V^\alpha]{\xi - \xi'} &= \int |\xi - \xi'|(\rmd x) V^{\alpha}(x) = 2\tvnorm{\xi - \xi'} \int \frac{|\xi - \xi'|(\rmd x)}{2\tvnorm{\xi - \xi'}} V^\alpha(x) \\
&\leq \{ 2\tvnorm{\xi - \xi'} \}^{1-\alpha} \left( \int |\xi- \xi'|(\rmd x) V(x) \right)^\alpha \eqsp.
\end{align*}
If we replace $\xi \leftarrow \delta_x P^n$ and $\xi' \leftarrow \pi$, we get 
\begin{equation}
\Vnorm[V^\alpha]{\delta_x P^n - \pi} \leq 
\{ 2\tvnorm{\delta_x P^n - \pi} \}^{1-\alpha} \left( \Vnorm[V]{\delta_x P^n - \pi} \right)^\alpha\eqsp.
\end{equation}
Now \eqref{eq:V-geometric-better-rate-appendix} follows from $\tvnorm{\delta_x P^n - \pi} \leq 1$ and equation~\eqref{eq:V-geometric-coupling-general}.
\end{proof}
\begin{lemma}
\label{lem:geom_ergodicity_variance_bound}
Assume \Cref{assG:kernelP_q}, \Cref{assG:kernelP_q_smallset}, and let $q \in \nsets$. Then for any $g \in \mrl_{V^{1/(2q)}}$ and $n \in \nsets$, it holds that 
\begin{equation}
\label{eq:var_bound}
\PVar[\pi](S_n) \leq 5 n c^{1/2} \ratev^{-1/2} \{\log{1/\ratev}\}^{-1} \pi(V)^{3/2} \Vnorm[V^{1/2}]{\bar{g}}^{2}\eqsp.
\end{equation}
\end{lemma}
\begin{proof}
Applying \Cref{lem:rate_UGE} with $\alpha = 1/2$, we get
\begin{align*}
\PVar[\pi](S_n) 
&= \PVar[\pi](\sum_{i=0}^{n-1}\bar{g}(X_i)) = n \PVar[\pi](g) + \sum_{i=0}^{n-2}\sum_{\ell=1}^{n-2-i}\PE_{\pi}[\bar{g}(X_i)\bar{g}(X_{i+\ell})] \\
&\leq n \PVar[\pi](g) + 2 \Vnorm[V^{1/2}]{\bar{g}} \sum_{i=0}^{n-2}\sum_{\ell=1}^{n-2-i}\PE_{\pi}\bigl[|g(X_{i})| \{c \ratev^{\ell} \pi(V) V(X_i)\}^{1/2}\bigr] \\
&\leq n \PVar[\pi](g) + 2 c^{1/2} \pi(V)^{3/2} \Vnorm[V^{1/2}]{\bar{g}}^{2} \sum_{i=0}^{n-2}\sum_{\ell=1}^{n-2-i} \ratev^{\ell/2} \\
&\leq 5 n c^{1/2} \ratev^{-1/2} \{\log{1/\ratev}\}^{-1} \pi(V)^{3/2} \Vnorm[V^{1/2}]{\bar{g}}^{2}\eqsp.
\end{align*}
To complete the proof it remains to note that $\Vnorm[V^{1/2}]{\bar{g}} \leq \Vnorm[V^{1/(2q)}]{\bar{g}}$.
\end{proof}
We end this section with a technical bound on the scaling of coefficients $\ConstB_{0}(u,q)$ defined in \eqref{eq: B_u_q_def_new}. 
\begin{lemma}
\label{lem:scale_B_gamma}
Let $\ConstB_{0}(u,q)$ be the coefficient defined in \eqref{eq: B_u_q_def_new}. Then for $u \in \{1,\ldots,q-1\}$ it holds that
\begin{equation}
\label{eq:non_unif_bound_B_u_q}
\ConstB_{0}(u,q) \leq c_{1} (2q)^{2q} (2q-u)^{2(2q-u)} \rme^{-(2q-u)}\eqsp,
\end{equation}
where $c_{1} = \rme^{2}\sqrt{2}$. Moreover, it holds that 
\[
\ConstB_{0}(u,q) \leq c_{1} q^{6q} 2^{7q} \rme^{-2q}\eqsp.
\]
\end{lemma}
\begin{proof}
Recall that due to \eqref{eq:bound-Balpha}, it holds that
\[
\ConstB_{0}(u,q)
\leq \frac{(2q)!}{u!} \binom{2q-u-1}{u-1} \bigl((2q-2u+2)!\bigr)^{2} 2^{2(u-1)}\eqsp.
\]
Now, using the bound of \cite[Theorem~2]{guo_gamma_func}, we use the following upper bound on the Gamma function, valid for $x \geq 0$:
\[
\Gamma(x+1) \leq (x+1)^{x+1/2}e^{-x}\,.
\]
Application of this bound together with $1 + x \leq \rme^{x}$ yield 
\begin{align*}
\ConstB_{0}(u,q) 
&\leq \frac{(2q)!}{u!} \frac{(2q-u-1)!}{(u-1)!} (2q-2u+2)! (2q-2u+2)^{2} 2^{2(u-1)} \\
&\leq (2q)^{2q} (2q-2u+2)^{2q-2u+2} (2q-u-1)^{2q-u-1} \rme^{-2q+2} 2^{u+1/2}\,.
\end{align*}
This yields the upper bound \eqref{eq:non_unif_bound_B_u_q}. 
The uniform upper bound can be obtained from the expression above letting $u = 1$.
\end{proof}

\begin{comment}
\section{Technical Lemmas}
\label{sec:technical-lemmas}
\begin{lemma}
\label{lem:P_n_log_V}
Assume \Cref{assG:kernelP_q}. Then, for any $\beta \geq 1$ and $n \in \nset$ it holds that
\begin{equation}
\label{eq:P_n_log_V}
\MK^{n}W^{\beta}(x) \leq \ConstC_1^{\beta} \beta^{\beta} W^{\beta}(x)\,,
\end{equation}
where
\begin{equation}
\label{eq:const_C_1_def}
\ConstC_1 = 2 + \log\bigl(1 + b/(1-\lambda)\bigr)\eqsp.
\end{equation}
\end{lemma}
\begin{proof}
Using Assumption \Cref{assG:kernelP_q}, we easily obtain by induction that, for any $x \in \Xset$,
\begin{align*}
\MK^{n}V(x) \leq \lambda^n V(x) + b \sum_{k=0}^{n-1}\lambda^k \leq \lambda^nV(x) + b/(1-\lambda)\eqsp.
\end{align*}
Hence, using Jensen's inequality and $V(x) \geq \rme$, we obtain
\begin{align*}
\MK^{n}\log^{\beta}V(x)
&\leq \MK^{n}\log^{\beta}(V(x) + \rme^{\beta-1}) \leq \log^{\beta}\bigl(\MK^{n} V(x) + \rme^{\beta-1}\bigr) \\
&\leq \log^{\beta}\bigl(\lambda^n V(x) + \rme^{\beta-1} + b/(1-\lambda)\bigr) \\
&\leq \log^{\beta}\bigl(\lambda^n V(x) + V(x)\{\rme^{\beta-1} + b/(1-\lambda)\}\bigr)\eqsp. \\
\end{align*}
Since $\lambda < 1$, and $1 + \rme^{\beta - 1} \leq \rme^{\beta}$ for $\beta \geq 1$, and $V(x) \geq \rme$, we get
\begin{align*}
\MK^{n}\log^{\beta}V(x) &\leq \log^{\beta}\biggl\{V(x)\bigl(\rme^{\beta} + b/(1-\lambda)\bigr)\biggr\} \\
&\leq \bigl[\log\{V(x)\} + \beta\bigl(1 + \log\{1 + b/(1-\lambda)\}\bigr)\bigr]^{\beta} \\
&\leq \ConstC_1^{\beta} \beta^{\beta} \log^{\beta}V(x)\eqsp,
\end{align*}
where the constant $\ConstC_1$ is given in \eqref{eq:const_C_1_def}.
\end{proof}
\end{comment}

\begin{comment}
\begin{proof}
We set $\xi= \delta_{x}\MK^n- \pi$. First we note that
\begin{equation}
\label{eq:norms_relation}
\Vnorm[\log^{\beta}V]{\xi} = 2^{\beta}\Vnorm[\log^{\beta}(V^{1/2})]{\xi}\eqsp.
\end{equation}
Equipped with this relation, we proceed with estimating $\Vnorm[\log^{\beta}(V^{1/2})]{\cdot}$. Using Jensen's inequality,
\begin{align}
\Vnorm[\log^{\beta}\{V^{1/2}\}]{\xi}
&\leq \Vnorm[\log^{\beta}\{V^{1/2} + \rme^{\beta - 1}\}]{\xi} = \int |\xi|(\rmd y) \log^{\beta}{\bigl\{V^{1/2}(y) + \rme^{\beta - 1}\bigr\}} \nonumber\\
&= \tvnorm{\xi} \int \frac{|\xi|(\rmd y)}{\tvnorm{\xi}} \log^{\beta}{\bigl\{V^{1/2}(y) + \rme^{\beta - 1}\bigr\}} \nonumber\\
&\leq \tvnorm{\xi} \log^{\beta}\biggl(\frac{\Vnorm[V^{1/2}]{\xi}}{\tvnorm{\xi}} + \rme^{\beta - 1}\biggr) \label{eq:jensen_inequality_bound}\eqsp.
\end{align}
In the last inequality we used that the function $z \mapsto \log^{\beta}\{z+\rme^{\beta-1}\}$ is convex for $z > 0$. Note that
\begin{align*}
\log\biggl(\frac{\Vnorm[V^{1/2}]{\xi}}{\tvnorm{\xi}} + \rme^{\beta - 1}\biggr) \leq \beta + \log\biggl(1 + \frac{\Vnorm[V^{1/2}]{\xi}}{\tvnorm{\xi}}\biggr)\eqsp.
\end{align*}
Since for  $x \geq 0$ and $\beta \geq 1$, $\log{(1+x)} \leq  \beta x^{1/\beta}$, we obtain
\begin{align*}
\log\biggl(\frac{\Vnorm[V^{1/2}]{\xi}}{\tvnorm{\xi}} + \rme^{\beta - 1}\biggr)
\leq \beta + \beta\biggl(\frac{\Vnorm[V^{1/2}]{\xi}}{\tvnorm{\xi}}\biggr)^{1/\beta}
%&\leq 2\gamma^2 + \biggl(\frac{\Vnorm[V^{1/2}]{\xi}}{\tvnorm{\xi}}\biggr)^{1/\gamma}\,.
\end{align*}
Combining this upper bound with \eqref{eq:jensen_inequality_bound},
\begin{align*}
\Vnorm[\log^{\beta}\{V^{1/2}\}]{\xi}
&\leq 2^{\beta - 1}\beta^{\beta}\bigl(\tvnorm{\xi} + \Vnorm[V^{1/2}]{\xi} \bigr) \leq 2^{\beta}\beta^{\beta}\Vnorm[V^{1/2}]{\xi} \\
&\leq 2^{\beta + 1}\beta^{\beta}\bigl(c_m \pi(V)\bigr)^{1/2} \rho^{n/2}V^{1/2}(x)\,,
\end{align*}
where the last inequality is due to \Cref{assG:kernelP_q}, and constants $\rho$, $c_m$ are given in \eqref{eq:bornes-v-geometric-coupling}. Now the statement follows from \eqref{eq:norms_relation}.
\end{proof}
\end{comment}


%\section{Numerical constants}
%\begin{table}[h]
%\begin{tabular}{c|c|c|c|}
%$B_0(u,q)$ & $q=2$ & $q=3$ & $q=4$\\
%$u=1$ & $(4!)^3$ & $(6!)^3$ & $(8!)^{4}$ \\
%$u=2$ & -- & $6!(8(4!)^2 + (3!)^4)/2$ & $8!(4!)^4/2 + 4(8!)(6!)^2 + (5!)^2(3!)^2(8!)$ \\
%\end{tabular}
%\end{table}

%%% Local Variables:
%%% mode: latex
%%% TeX-master: "main"
%%% End:
