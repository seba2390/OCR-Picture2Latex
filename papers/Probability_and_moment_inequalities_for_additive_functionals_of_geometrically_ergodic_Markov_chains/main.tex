%Version 2.1 April 2023
% See section 11 of the User Manual for version history
%
%%%%%%%%%%%%%%%%%%%%%%%%%%%%%%%%%%%%%%%%%%%%%%%%%%%%%%%%%%%%%%%%%%%%%%
%%                                                                 %%
%% Please do not use \input{...} to include other tex files.       %%
%% Submit your LaTeX manuscript as one .tex document.              %%
%%                                                                 %%
%% All additional figures and files should be attached             %%
%% separately and not embedded in the \TeX\ document itself.       %%
%%                                                                 %%
%%%%%%%%%%%%%%%%%%%%%%%%%%%%%%%%%%%%%%%%%%%%%%%%%%%%%%%%%%%%%%%%%%%%%

%%\documentclass[referee,sn-basic]{sn-jnl}% referee option is meant for double line spacing

%%=======================================================%%
%% to print line numbers in the margin use lineno option %%
%%=======================================================%%

%%\documentclass[lineno,sn-basic]{sn-jnl}% Basic Springer Nature Reference Style/Chemistry Reference Style

%%======================================================%%
%% to compile with pdflatex/xelatex use pdflatex option %%
%%======================================================%%

%%\documentclass[pdflatex,sn-basic]{sn-jnl}% Basic Springer Nature Reference Style/Chemistry Reference Style


%%Note: the following reference styles support Namedate and Numbered referencing. By default the style follows the most common style. To switch between the options you can add or remove “Numbered” in the optional parenthesis. 
%%The option is available for: sn-basic.bst, sn-vancouver.bst, sn-chicago.bst, sn-mathphys.bst. %  
 
%%\documentclass[sn-nature]{sn-jnl}% Style for submissions to Nature Portfolio journals
%%\documentclass[sn-basic]{sn-jnl}% Basic Springer Nature Reference Style/Chemistry Reference Style
\documentclass[sn-mathphys,Numbered]{sn-jnl}% Math and Physical Sciences Reference Style
%%\documentclass[sn-aps]{sn-jnl}% American Physical Society (APS) Reference Style
%%\documentclass[sn-vancouver,Numbered]{sn-jnl}% Vancouver Reference Style
%%\documentclass[sn-apa]{sn-jnl}% APA Reference Style 
%%\documentclass[sn-chicago]{sn-jnl}% Chicago-based Humanities Reference Style
%%\documentclass[default]{sn-jnl}% Default
%%\documentclass[default,iicol]{sn-jnl}% Default with double column layout

%%%% Standard Packages
%%<additional latex packages if required can be included here>
\usepackage{graphicx}%
\usepackage{multirow}%
\usepackage{amsmath,amssymb,amsfonts}%
\usepackage{amsthm}%
\usepackage{mathrsfs}%
\usepackage{xcolor}%
\usepackage{textcomp}%
\usepackage{manyfoot}%
\usepackage{booktabs}%
\usepackage{listings}%

%\documentclass[preprint,11pt]{elsarticle}
%\journal{Stochastic Processes and Their Applications}
%\usepackage[a4paper]{geometry}
%\usepackage{palatino}
%\usepackage[affil-it]{authblk}
%\newcommand\hmmax{0}
%\newcommand\bmmax{0}

%\usepackage[left=1in, right=1in, top=1in]{geometry}
\usepackage{hyperref}
\usepackage{mathrsfs}
%\usepackage{natbib}
\usepackage{nicefrac}
\usepackage{fancyhdr}
%\geometry{margin=1in}  %as per JSE instructions
\usepackage{setspace}
\usepackage{lastpage}
\usepackage{upgreek}
\usepackage{graphicx}	% Including figure files
\usepackage[lofdepth,lotdepth]{subfig}
\usepackage{amsmath,amsthm}	% Advanced maths commands
\usepackage{amssymb,dsfont,bbm}	% Extra maths symbols
\usepackage[ruled,vlined]{algorithm}  % algorithm
\usepackage{xcolor}  % colors
\usepackage{comment}
\usepackage{bm}
\usepackage{xargs}
\usepackage[shortlabels]{enumitem}
%bibliography style
%\bibliographystyle{elsarticle-num}
%\usepackage{numcompress}
%\bibliographystyle{model3-num-names}
%\bibliographystyle{elsarticle-harv}%\biboptions{authoryear}

\usepackage{aliascnt}
\usepackage{cleveref}
\usepackage[textwidth=3cm,textsize=footnotesize]{todonotes}

\newcommand{\eric}[1]{\todo[color=red!20]{{\bf EM:} #1}}
\newcommand{\alex}[1]{\todo[color=red!20]{{\bf AN:} #1}}
\newcommand{\alain}[1]{\todo[color=blue!20]{{\bf AD:} #1}}
\newcommand{\som}[1]{\todo[color=green!20]{{\bf SS:} #1}}

% see the list of further useful packages
% in the Reference Guide
\makeindex %used for the subject index
                       % please use the style xsvind.ist with
                       % your makeindex program


\newtheorem{lemma}{Lemma}
\newtheorem{proposition}{Proposition}
\newtheorem{theorem}{Theorem}
\newtheorem{definition}{Definition}
\newtheorem{corollary}{Corollary}
\newtheorem{remark}{Remark}

\newcommand{\argmin}{\mathop{^\rm argmin}}
\newcommand{\argmax}{\mathop{\rm argmax}}
\newcommand{\rank}{\mathop{\sf rank}}
\newcommand{\iprod}[2]{\langle #1, #2 \rangle}
\newcommand{\lmin}{\lambda_{\min}}
\newcommand{\norm}[1]{\left\|#1\right\|}
\newcommand{\mge}{\succeq}
\newcommand{\mi}{{-1}}
\newcommand{\R}{\mathbb{R}}\newcommand{\B}{\mathbb{B}}\newcommand{\Id}{\mathbf{I}}
\newcommand{\zerovec}{\mathbf{0}}
\newcommand{\onevec}{\mathbf{1}}
\newcommand{\defeq}{:=}
\newcommand{\secref}[1]{Section~\ref{#1}}
\newcommand{\lemref}[1]{Lemma~\ref{#1}}
\newcommand{\proref}[1]{Proposition~\ref{#1}}
\newcommand{\thmref}[1]{Theorem~\ref{#1}}
\newcommand{\assref}[1]{Assumption~\ref{#1}}
\newcommand{\eqnref}[1]{(\ref{#1})}
\newcommand{\algref}[1]{Algorithm~\ref{#1}}

\newcommand{\Prob}{\mathbb{P}}
\newcommand{\E}{\mathbb{E}}
\newcommand{\V}{\mathbb{V}}
\newcommand{\tr}[1]{\textrm{tr}\left(#1\right)}
%\newcommand{\dt}[1]{\textrm{det}\left(#1\right)}

\newcommand{\Evt}{\mathcal{E}}
\newcommand{\EvtG}{\Evt_G}
\newcommand{\EvtD}{\Evt_\Delta}
\newcommand{\EvtX}{\Evt_X}

\newcommand{\lihong}[1]{[[\textbf{LL:} #1]]}
\renewcommand{\lihong}[1]{}

\newcommand{\RN}[1]{%
  \textup{\uppercase\expandafter{\romannumeral#1}}%
}

\newtheorem{assumption}{Assumption}


\begin{document}

\title[Probability and moment inequalities for additive functionals of geometrically ergodic Markov chains]{Probability and moment inequalities for additive functionals of geometrically ergodic Markov chains}

%%=============================================================%%
%% Prefix	-> \pfx{Dr}
%% GivenName	-> \fnm{Joergen W.}
%% Particle	-> \spfx{van der} -> surname prefix
%% FamilyName	-> \sur{Ploeg}
%% Suffix	-> \sfx{IV}
%% NatureName	-> \tanm{Poet Laureate} -> Title after name
%% Degrees	-> \dgr{MSc, PhD}
%% \author*[1,2]{\pfx{Dr} \fnm{Joergen W.} \spfx{van der} \sur{Ploeg} \sfx{IV} \tanm{Poet Laureate} 
%%                 \dgr{MSc, PhD}}\email{iauthor@gmail.com}
%%=============================================================%%

\author[1]{\fnm{Alain} \sur{Durmus}}\email{alain.durmus@polytechnique.edu}

\author[1]{\fnm{Eric} \sur{Moulines}}\email{eric.moulines@polytechnique.edu}

\author[2]{\fnm{Alexey} \sur{Naumov}}\email{anaumov@hse.ru}

\author*[2]{\fnm{Sergey} \sur{Samsonov}}\email{svsamsonov@hse.ru}

\affil[1]{Ecole polytechnique, Paris, France}

\affil[2]{HSE University, Moscow, Russia}


%%==================================%%
%% sample for unstructured abstract %%
%%==================================%%

\abstract{In this paper, we establish moment and Bernstein-type inequalities for additive functionals of  geometrically ergodic Markov chains. These inequalities extend the corresponding inequalities for independent random variables. Our conditions cover Markov chains converging geometrically to the stationary distribution either in weighted total variation norm or in weighted Wasserstein distances. Our inequalities apply to unbounded functions and depend explicitly on constants appearing in the conditions that we consider.}

\keywords{concentration inequalities for Markov chains, cumulant expansion}

\pacs[MSC Classification]{60E15, 60J20, 65C40}

\maketitle

\section{Introduction}
\label{sec:introduction}
\section{Introduction}  \label{sec:introduction}

\newcommand\inexpIntro[3]{#1?(#2,#3).}
\newcommand\rinexpIntro[3]{*#1?(#2,#3).}
\newcommand\outexpIntro[3]{#1!(#2,#3).}
\newcommand\outatomIntro[3]{#1!(#2,#3)}

We propose a fully automated method for proving termination of \(\pi\)-calculus processes.
Although there have been a lot of studies on termination analysis for the \(\pi\)-calculus
and related calculi~\cite{Deng06IC,Demangeon07,SangiorgiTermination,KobayashiHybrid,Yoshida04IC,DBLP:journals/jlp/DemangeonHS10,Venet98SAS}, most of them have been rather theoretical,
and there have been surprisingly little efforts in developing  fully automated termination
verification methods and tools based on them. To our knowledge,
Kobayashi's \typical{}~\cite{TyPiCal,KobayashiHybrid} is the only exception that
can prove termination of \(\pi\)-calculus processes (extended with natural numbers)
fully automatically, but its termination analysis is quite limited (see Section~\ref{sec:relatedwork}).

Our method is based on a reduction to termination analysis for sequential programs:
we translate a \(\pi\)-calculus process \(P\) to a sequential program \(S_P\), so that
if \(S_P\) is terminating, so is \(P\). The reduction allows us to use
powerful, mature methods and tools
for termination analysis of sequential programs~\cite{heizmann2016ultimate,freqterm,DBLP:conf/lics/PodelskiR04,Kuwahara2014Termination,DBLP:journals/cacm/CookPR11}.

The idea of the translation is to convert a chain of communications on replicated input
channels to a chain of recursive function calls of the target sequential program.
Let us consider the following Fibonacci process:
\begin{align*}
    & \rinexpIntro{\fib}{n}{r}
        \ifexp{n<2}{ \soutatom{r}{1} \\ &\quad}
                   { \nuexp{s_1} \nuexp{s_2} (\outatomIntro{\fib}{n-1}{s_1} \PAR \outatomIntro{\fib}{n-2}{s_2} \PAR \sinexp{s_1}{x}\sinexp{s_2}{y}\soutatom{r}{x+y}) \\}
    & \PAR \outatomIntro{\fib}{m}{r}
\end{align*}
Here, the process
$\rinexpIntro{\fib}{n}{r} \ldots$ is a function server that computes the \(n\)-th Fibonacci number
in parallel and returns the result to \(r\),
and $\outatom{\fib}{m}{r}$ sends a request for computing the \(m\)-th Fibonacci number;
those who are not familiar with the syntax of the \(\pi\)-calculus may wish to consult
Section~\ref{sec:targetlanguage} first.
To prove that the process above is terminating for any integer \(m\),
it suffices to show that there is no infinite chain of communications on $\fib$:
\[
    \fib(m,r) \to \fib(m_1,r_1) \to \fib(m_2,r_2) \to \cdots.
\]
We convert the process above to the following program:\footnote{The actual translation
  given later is a little more complex.}
\begin{verbatim}
 let rec fib(n) = if n<2 then () else (fib(n-1) [] fib(n-2)) in
 fib(m)
\end{verbatim}
Here, \texttt{[]} represents the non-deterministic choice.
Note that, although the calculation of Fibonacci numbers is not preserved,
for each chain of communications on \texttt{fib}, there is a corresponding
sequence of recursive calls:
\[
\mathtt{fib}(m) \to \mathtt{fib}(m_1) \to \mathtt{fib}(m_2) \to \cdots.
\]
Thus, the termination of the sequential program above implies the termination of
the original process.
As shown in the example above, (i) each communication on a replicated input channel
is converted to a function call, (ii) each communication on a non-replicated input
channel is just removed (or, in the actual translation, replaced by a call of
a trivial function defined by \(f(\seq{x})=(\,)\)), and (iii) parallel composition
is replaced by a non-deterministic choice.
We formalize the translation outlined above and prove its correctness.

The basic translation sketched above sometimes loses too much information.
For example, consider the following process:
\begin{align*}
    & \rinexpIntro{\pre}{n}{r} \soutatom{r}{n-1} \\
    & \PAR \rinexpIntro{f}{n}{r} \ifexp{n<0}{ \soutatom{r}{1} }
                                       { \nuexp{s} (\outatomIntro{\pre}{n}{s} \PAR \sinexp{s}{x}\outatomIntro{f}{x}{r}) } \\
    & \PAR \outatomIntro{f}{m}{r}
\end{align*}
The translation sketched above would yield:
\begin{verbatim}
  let pred(n) = n-1 in
  let rec f(n) = if n<0 then () else (pred(n) [] f(*)) in
  f(m)
\end{verbatim}
Here, \texttt{*} represents a non-deterministic integer: since we have removed
the input $\sinatom{s}{x}$, we do not have information about the value of \( x \).
As a result, the sequential program above is non-terminating, although the original
process is terminating.
To remedy this problem, we also refine the basic translation above by using a refinement
type system for the \(\pi\)-calculus. Using the refinement type system,
we can infer that the value of \(x\) in the original process is less than \(n\),
so that we can refine the definition of \texttt{f} to:
\begin{verbatim}
 let rec f(n) = ... else (pred(n) [] let x=* in assume(x<n);f(x))
\end{verbatim}
The target program is now terminating, from which
we can deduce that the original process is also terminating.
We have implemented an automated tool based on the refined translation above.

The contributions of this paper are summarized as follows.
\begin{itemize}
\item The formalization of the basic translation from the \(\pi\)-calculus
  (extended with integers) to sequential programs, and a proof of its correctness.
\item The formalization of a refined translation based on a refinement type system.
\item An implementation of the refined translation, including automated refinement type
  inference based on CHC solving, and experiments to evaluate the effectiveness of
  our method.
\end{itemize}

The rest of this paper is structured as follows.
Section~\ref{sec:targetlanguage} introduces the source and target languages
of our translation.
Section~\ref{sec:approach} 
formalizes the basic translation, and proves its correctness.
Section~\ref{sec:refinement} refines the basic translation by using a refinement type system.
Section~\ref{sec:implementation} reports an implementation and experiments.
Section~\ref{sec:relatedwork} discusses related work,
and Section~\ref{sec:conclusion} concludes the paper.


\section{Main results}
\label{sec:main-results}
\subsection{Geometrically $V$-ergodic Markov chains}
\label{sec:geom-v-ergod}
First, we consider the case where the Markov kernel $\MK$ is $V$-uniformly geometrically ergodic. We impose the following assumptions on the Markov kernel $\MK$:
\begin{assumption}
\label{assG:kernelP_q}
There exist a measurable function $V: \Xset \to \coint{\rme,\infty}$, $\lambda \in (0,1)$, and $b \geq 0$ such that for any $x \in \Xset$, $\MK V(x) \leq \lambda V(x) + b$.
\end{assumption}
Note that, in contrast to the usual definition of Lyapunov functions in the Markov chain literature, we assume here that $V$ takes values in $\coint{\rme,\plusinfty}$, rather than in $\coint{1,\plusinfty}$. This choice allows us to avoid technical problems when we consider $W = \log
V$ later in this section.

\begin{assumption}
\label{assG:kernelP_q_smallset}
There are an integer $ m \geq 1$, $\epsilon \in (0,1)$, and $d \in \rset_{+}$, such that  the level set $\{x \in\Xset \, :\, V(x) \leq d\}$ is $(m,\epsilon)$-small and $\lambda+2b/(1+d)<1$. The quantities $\lambda$ and $b$ are defined in \Cref{assG:kernelP_q}.
\end{assumption}
The definition of $(m,\epsilon)$-small set can be found, e.g., in \cite[Definition~9.1.1.]{douc:moulines:priouret:soulier:2018}. In particular, it is known that the Markov kernel $\MK$ is known to be uniformly geometrically ergodic if and only if the entire space $\Xset$ is $(m,\epsilon)$-small, see \cite[Theorem~15.3.1]{douc:moulines:priouret:soulier:2018}. Under \Cref{assG:kernelP_q} and \Cref{assG:kernelP_q_smallset}, the Markov kernel $\MK$ admits a unique invariant probability measure $\pi$ satisfying $\pi(V) < \infty$. Moreover, \cite[Theorem~19.4.1]{douc:moulines:priouret:soulier:2018} implies that for any probability measure $\xi$ satisfying $\xi(V) < \infty$ and all $n \in \nset$,
\begin{align}
    \label{eq:V-geometric-coupling-general}
    \tvnorm{\xi \MK^n - \pi} \leq \Vnorm[V]{\xi \MK^n - \pi} \leq \cmconstv \{ \xi(V) + \pi(V) \} \ratev^n \eqsp,
  \end{align}
  where the constants $\ratev$ and $\cmconstv$ are given by
  \begin{equation}
    \label{eq:bornes-v-geometric-coupling}
      \begin{aligned}
        &\log \ratev = \frac{\log(1-\epsilon) \log\bar\lambda_m} { m\bigl(\log(1-\epsilon) +
          \log\bar\lambda_m-\log\bar{b}_m\bigr) } \eqsp ,\\
        &\bar\lambda_m = \lambda^m+2b_m/(1+d) \eqsp,
        \eqspp \bar{b}_m = \lambda^m b_m + d
        \eqsp, \eqspp b_m=b (1-\lambda^m) /(1-\lambda) \eqsp ,   \\
        &\cmconstv  = \ratev^{-m}\{\lambda^m+(1-\lambda^m)/(1-\lambda)\}\{1+\bar{b}_m/[(1-\epsilon)(1-\bar\lambda_m)]\} \eqsp .
  \end{aligned}
\end{equation}
Before proceeding with our main results, we introduce some additional quantities. For each $q \in \nset, u \in \{1, \dots, q-1\}$ and $\gamma \geq 0$, we introduce
\begin{equation}
\label{eq: B_u_q_def_new}
\ConstB_{\gamma}(u,q)
= \frac{(2q)!}{u!} \sum_{(k_1,\ldots,k_u) \in \scrE_{u,q}}   \prod_{i=1}^u (k_{i}!)^{\gamma + 2} \eqsp,
\end{equation}
where $\scrE_{u,q} = \{ (k_1,\ldots,k_u) \in \nset^u \, : \, \sum_{i=1}^u k_i = 2q\, ,\, k_i \geq 2\}$.
Note that the cardinality of $\scrE_{u,q}$ is $\binom{2q-u-1}{u-1}$ which implies and upper bound
\begin{equation}
\label{eq:bound-Balpha}
\ConstB_{\gamma}(u,q)
\leq \frac{(2q)!}{u!} \binom{2q-u-1}{u-1} \bigl((2q-2u+2)!\bigr)^{2+\gamma} 2^{(u-1)(2+\gamma)}\eqsp.
\end{equation}
We first establish a Rosenthal-type inequality for uniformly $V$-geometrically ergodic Markov chains. The leading term is the variance (under stationarity) scaled by the corresponding moment of the Gaussian distribution.
\begin{theorem}\label{th:rosenthal_V_q}
Assume \Cref{assG:kernelP_q}, \Cref{assG:kernelP_q_smallset}, and let $q \in \nset^*$. Then, for any function $g \in \mrl_{V^{1/(2q)}}$, 
\begin{equation}
\label{eq:main-rosenthal}
\PE_\pi[|S_n|^{2q}] \leq \momentGq[q] \{\PVar[\pi](S_n)\}^q + \Constmainros^{2q} \Vnorm[V^{1/(2q)}]{\bar{g}}^{2q}
\sum_{u=1}^{q-1} \frac{\ConstB_0(u,q) n^{u}}{\ratev^{u/2} \log^{2q-u}{(1/\ratev)}}\eqsp,
\end{equation}
where
\begin{equation}
\label{eq:constmainros}
\Constmainros = 2 \cmconstv \pi(V)\eqsp.
\end{equation}
\end{theorem}
\begin{proof}
The proof is postponed to \Cref{sec:proof-ros_v_q}.
\end{proof}
\begin{remark}
\label{remark:theo_1_scaling_mix_time}
Note that the bound \eqref{eq:main-rosenthal} scales homogeneously w.r.t. the factor $\{\log(1/\ratev)\}^{-1}$. Indeed, for any $g \in \mrl_{V^{1/(2q)}}$, we get applying \Cref{lem:geom_ergodicity_variance_bound}, that 
\[
\PVar[\pi](S_n) \leq 5 n c^{1/2} \ratev^{-1/2} \{\log{1/\ratev}\}^{-1} \pi(V)^{3/2} \Vnorm[V^{1/2}]{\bar{g}}^{2}\eqsp.
\]
Thus, applying \Cref{lem:scale_B_gamma} and dividing both parts of \eqref{eq:main-rosenthal} by $n^{2q}$, we obtain from \eqref{eq:main-rosenthal} that
\begin{equation}
\label{eq:main-rosenthal-simplified}
\PE_\pi\bigl[\bigl|\tfrac{S_n}{n}\bigr|^{2q}\bigr] \lesssim q^{q} \biggl(\frac{\ratev^{-1/2}}{n \log(1/\ratev)}\biggr)^{q} + q^{2q} \sum_{u=1}^{q-1} \biggl(\frac{\ratev^{-1/2} (2q-u)^{2}}{\rme n \log(1/\ratev)}\biggr)^{2q-u}\eqsp.
\end{equation}
Here $ a \lesssim b $ stands for $a \leq \operatorname{c} b$ where $\operatorname{c}$ prefactor, depending upon $c, \pi(V)$, and $\Vnorm[V^{1/(2q)}]{\bar{g}}$. That is, for $n = \kappa \ratev^{-1/2} / \log(1/\ratev)$, the r.h.s. of \eqref{eq:main-rosenthal-simplified} scales with $\kappa^{-1}$. 
\end{remark}
The above result can be extended using the construction of the exact distributional coupling for any initial distribution (see \Cref{sec:non-stationary-extension} for the necessary definitions). It is worth noting that in our approach it is not necessary to assume that the Markov chain is strongly aperiodic (unlike \cite{adamczak2015exponential}).
\begin{theorem}
\label{theo:changeofmeasure}
Assume \Cref{assG:kernelP_q}, \Cref{assG:kernelP_q_smallset} and let $q \in \nset^*$. Then, for any probability measure $\xi$ on $(\Xset,\Xsigma)$ satisfying $\xi(V) < \infty$ and
$g \in \mrl_{V^{1/(2q)}}$,
\begin{equation*}
\PE_\xi\big[ \big|S_n \big|^{2q} \big] \leq 2^{2q-1} \PE_\pi\big[ \big|S_n \big|^{2q} \big]  + 2^{6q-1} \Vnorm[V^{1/(2q)}]{\bar{g}}^{2q} \cmconstv \{ \xi(V) + \pi(V) \} \frac{q^{2q}}{\ratev (\log(1/\ratev))^{2q}}\eqsp.
\end{equation*}
\end{theorem}
\begin{proof}
  The proof is postponed to \Cref{sec:non-stationary-extension}.
\end{proof}

In the above results we fix $q \in \nsets$ and consider a function $g \in \mrl_{V^{1/(2q)}}$. Of course, with these assumptions we can only obtain moment bounds of order $2q$ or smaller and cannot control the exponential moments of $S_n$. In our next statement, we consider the case of the function $g \in \mrl_{W^{\gamma}}$, where $W = \log V$ and $\gamma \geq 0$. Note that when $\gamma = 0$, $\mrl_{W^{\gamma}}$ coincides with the set of bounded functions. In this case, in addition to the Rosenthal-type bound \eqref{eq:main-rosenthal}, we can formulate a counterpart of the Bernstein-type bound \eqref{eq:gene_Bernstein type bound}. We begin with the result that is a counterpart of \Cref{th:rosenthal_V_q}.
\begin{theorem}\label{th:rosenthal_log_V}
Assume \Cref{assG:kernelP_q}, \Cref{assG:kernelP_q_smallset} and let $\gamma \geq 0$, $q \in \nset^*$. Then for any $g \in \mrl_{W^\gamma}$, it holds
\begin{equation*}
\PE_{\pi}[|S_n|^{2q}] \leq \momentGq[q] \{\PVar[\pi](S_n)\}^q+  \Constmainros^{2q} (2\gamma)^{2\gamma q} \Vnorm[W^\gamma]{\bar{g}}^{2q} \sum_{u=1}^{q-1}  \frac{\ConstB_{\gamma}(u,q) n^{u}}{\ratev^{u/2} \log^{2q-u}{(1/\ratev)}}\eqsp,
\end{equation*}
where $\Constmainros$ is defined in \eqref{eq:constmainros}.
\end{theorem}
\begin{proof}
The proof is postponed to ~\Cref{sec:proof-ros_log_V}.
\end{proof}
Similar to \Cref{theo:changeofmeasure}, we provide a version of the above statement for any initial distribution.
\begin{theorem}
\label{theo:changeofmeasure-1}
Assume \Cref{assG:kernelP_q}, \Cref{assG:kernelP_q_smallset} and let $\gamma \geq 0$,  $q \in \nset^*$. Then for any probability measure $\xi$ on $(\Xset,\Xsigma)$, and any $g \in \mrl_{W^\gamma}$, it holds
\begin{equation}
     \PE_\xi\big[ \big|S_n \big|^{2q} \big] \leq 2^{2q-1} \PE_\pi\big[ \big|S_n \big|^{2q} \big]  + 2^{4q-2} \Vnorm[W^\gamma]{\bar{g}}^{2q} \cmconstv \{ \xi(V) + \pi(V) \} \operatorname{D}^{(1)}_{q,\gamma}\eqsp,
\end{equation}
where
\begin{equation}
\operatorname{D}^{(1)}_{q,\gamma}= \rme^{-1} \ratev^{-1} \{ \log(1/\ratev) \}^{1- 4 q} (4q-2) ! +  \ratev^{-1} \{ \log(1/\ratev) \}^{-1} (4 q \gamma/ \rme)^{4 q \gamma}\eqsp.
\end{equation}
\end{theorem}
\begin{proof}
  The proof is postponed to \Cref{sec:non-stationary-extension}.
\end{proof}
We can also obtain Bernstein-type bound. We start from the stationary case.
\begin{theorem}
\label{th:rosenthal_log_V_cor_2}
Assume \Cref{assG:kernelP_q}, \Cref{assG:kernelP_q_smallset} and let $\gamma \geq 0$. Then for any $g \in \mrl_{W^\gamma}$ and $t \geq 0$, it holds that 
\begin{equation}
\label{eq:bernstein_mc}
\PP_{\pi}(|S_n| \geq t) \leq 2\exp\biggl\{-\frac{t^2/2}{\PVar[\pi](S_n) + \ConstJ^{1/(\gamma+3)} t^{2-1/(\gamma+3)}}\biggr\}\eqsp.
\end{equation}
Moreover, for any $\delta > 0$ we get
\begin{equation}
\label{eq:high_prob_bound_W_ergodic}
\PP_{\pi}\biggl(|S_n| \geq 2\sqrt{\PVar[\pi](S_n)}\sqrt{\log(4/\delta)} + 4^{\gamma+3}\ConstJ\{\log(4/\delta)\}^{\gamma+3}\biggr) \leq \delta\eqsp.
\end{equation}
Here the constant $\ConstJ$ is given by
\begin{equation}
\label{eq:const_B_n_definition_main}
\ConstJ = \biggl( \frac{n \ratev^{-1/2} \{\log(1/\ratev)\}^{-1} \Constmainros^{2} \|\bar{g}\|_{W^{\gamma}}^{2}}{\PVar[\pi](S_n)} \vee 1\biggr) \frac{ 2^{1+3\gamma}\gamma^{3\gamma} \Constmainros \|\bar{g}\|_{W^{\gamma}}}{\log(1/\ratev)} \eqsp.
\end{equation}
\end{theorem}
\begin{proof}
The proof is postponed to ~\Cref{sec:proof_bernstein_bound}.
\end{proof}
Comparing \eqref{eq:bernstein_mc} with \eqref{eq: Bernstein type bound}, one can see that in the subexponential regime $t^{1/(\gamma+1)}$ is replaced by $t^{1/(\gamma+3)}$, as in \cite{doukhan2007probability}. This factor is caused by the dependence along the observations of the Markov chain. Similar to \eqref{eq: Bernstein type bound}, the constant $\ConstJ$ is not distribution-free, moreover, compared to \eqref{eq:gene_Bernstein type bound}, it is possible that $\ConstJ$ scales with $n$. However, the dependence in the distribution is fully explicit: one has to compare $\PVar[\pi](S_n)$ with $n \{\log(1/\ratev)\}^{-1} \Vnorm[W^\gamma]{\bar{g}}^{2}$.
\begin{remark}
\label{rem:lower_bound_variance}
For some particular instances of Markov chains, this comparison can be done explicitly. Indeed, let us choose a function $g$ with $\|\bar{g}\|_{W^\gamma} \leq 1$, $\pi(g^2) < \plusinfty$, and introduce for $\ell \in \nset$ the quantity
\[
\varsigma_\pi(g,\ell) = \PCov[\pi]( g(X_0),g(X_{\ell}))\eqsp.
\]
Under \Cref{assG:kernelP_q} and \Cref{assG:kernelP_q_smallset}, the Markov chain is$V$-uniformly geometrically ergodic, which implies $\sum_{\ell=-\infty}^\infty | \varsigma_\pi(g,\ell)| < \infty$.
Therefore, we can calculate the spectral density $f(g,\lambda)= (2\uppi)^{-1}\sum_{\ell = -\infty}^{\infty} \varsigma_\pi(g,\ell) \rme^{-\rmi \ell \lambda}$, for $\lambda \in [-\pi,+\pi]$. If we additionally assume that there is $f_{\min} \in \rset_{+}$ such that $f(g,\lambda) \geq f_{\min}$ for all $\lambda \in [-\pi,\pi]$, it is straightforward to show that $\PVar[\pi](S_n) \geq n f_{\min}$ and hence the constant
\[
\ConstJ \leq \biggl(\frac{\ratev^{-1/2} \{\log(1/\ratev)\}^{-1} \Constmainros^{2}}{f_{\min}} \vee 1\biggr) \frac{ 2^{1+3\gamma}\gamma^{3\gamma} \Constmainros}{\log(1/\ratev)}
\]
is independent of $n$.
\end{remark}
Finally, we provide a Bernstein-type bound for the case of an arbitrary initial distribution. For this proof we again use distributional coupling, but the reasoning is more complicated to obtain Weibulian dependence in the initial conditions.
\begin{comment}
\begin{corollary}
\label{th:rosenthal_log_V_cor_1}
Under assumptions of \Cref{th:rosenthal_log_V}, for any $q \geq 1$
\begin{equation}
\PE_{\pi}[|S_n|^{2q}] \leq 2^{2q+1}q!\{\PVar[\pi](S_n)\}^q + 2\rme \Constroslogmom^{q} q^{(6+6\gamma)q} \ConstB^{(6+6\gamma)q}\eqsp,
\end{equation}
where
\begin{equation}
\label{eq:const_K_2_def}
\Constroslogmom = 2^{18+18\gamma}\rme^{-(3\gamma+3)}(3\gamma+3)^{6\gamma+6}\eqsp.
\end{equation}
\end{corollary}
\begin{proof}
The proof is postponed to ~\Cref{sec:proof_moment_corollary_bentkus}.
\end{proof}
\end{comment}

\begin{theorem}
  \label{theo:prob_ineq_V_norm}
  Assume \Cref{assG:kernelP_q} and \Cref{assG:kernelP_q_smallset} and let $\gamma \geq 0$. Then, for any initial distribution $\xi$ on $(\Xset,\Xsigma)$, $g \in \mrl_{W^\gamma}$, and $t \geq 0$, it holds setting  $\varpi_{\gamma} = 1/(1+\gamma)$, that
  \begin{align*}
%    \label{eq:1}
&\PP_{\xi}(|S_n| \geq t) \leq    \PP_{\pi}(|S_n| \geq t/4) \\
    & \quad + \parenthese{\frac{\rme^{-\log(1/\rho)t^{\varpi_{\gamma}}/(4^{1+\varpi_{\gamma}}\|\bar{g}\|_{W^\gamma}^{\varpi_{\gamma}}\varpi_{\gamma})}}{\ratev^{1/2}} +   \frac{\rme^{-(1+\gamma) t^{\varpi_{\gamma}}/(2^{1+2\varpi_{\gamma}}\|\bar{g}\|_{W^\gamma}^{\varpi_{\gamma}}\gamma)}}{1-\ratev} }\cmconstv \{ \xi(V) + \pi(V) \}\eqsp.
  \end{align*}
\end{theorem}
\begin{proof}
  The proof is postponed to \Cref{sec:non-stationary-extension}.
\end{proof}
It is worth noting that the exponent of the terms reflecting the dependence in the initial conditions is $1/(1+\gamma)$, as in \cite[Theorem~5.1]{adamczak2015exponential}, but without the assumption of strong aperiodicity. Finally, unlike \cite[Theorem~5.1]{adamczak2015exponential}, the dependence in the initial condition appears as a multiplicative factor rather than in the exponential rate.
\subsection{Geometrically ergodic Markov chains with respect to Wasserstein semi-metric}
\label{sec:geom-ergod-mark}
Now we extend the results obtained in \Cref{sec:geom-v-ergod} to the case of Markov kernels that are geometrically contracting for a weighted Wasserstein (pseudo)distance. The advantage of this setting is that we do not have to assume that the Markov kernel is irreducible. This is a significant advantage for the study of stochastic algorithms (which is one of the goals of this paper), but also for Markov chains in infinite dimensions; see \cite{hairer2011asymptotic,hairer:stuart:vollmer:2012,butkovsky:veretennikov:2013} and \cite[Chapter~20]{douc:moulines:priouret:soulier:2018} and references therein.
In this section we assume that $(\Xset,\distance)$ is a complete separable metric space and denote by $\Xsigma$ its Borel $\sigma$-field. Let $\cost:\Xset\times\Xset\to\rplus$ satisfy the following condition.
\begin{assumptionC}
  \label{ass:cost_fun}
$\cost$ is a lower semicontinuous symmetric function such that $\cost(x,x')=0$ for $x=x'$. Also, there is $\pcost \in\nsets$ such that for any $x,x'\in\Xset$, $(\distance(x,x') \wedge 1)^{\pcost} \leq \cost(x,x') \leq 1$.
\end{assumptionC}
A function $\cost$ satisfying \Cref{ass:cost_fun} is called distance-like.
For two probability measures $\xi$ and
$\xi'$ on $(\Xset,\Xsigma)$, we say that a probability measure $\nu$ on $(\Xset^2,\Xsigma^{\otimes 2})$ is a coupling of $\xi$ and $\xi'$, if for each $\msa \in \Xsigma$, $\nu(\msa \times \Xset) = \xi(\msa)$ and $\nu( \Xset \times \msa) = \xi'(\msa)$.
Denote by
$\couplingmeasure(\xi,\xi')$  the set of couplings of $\xi$ and
$\xi'$ on $(\Xset,\Xsigma)$,  and define
\begin{align*}
  \wasser[\cost]{\xi}{\xi'} = \inf_{\nu \in \couplingmeasure(\xi,\xi')} \int_{\Xset\times\Xset} \cost(x,x')
  \nu(\rmd x\rmd x')  \eqsp .
\end{align*}
We say that $\MKK$ is a Markov coupling of $\MK$ if for all $(x,x') \in \Xset^2$ and $\msa \in \Xsigma$, $\MKK((x, x'), \msa \times \Xset) = \MK (x, \msa)$ and $\MKK((x,x'), \Xset \times \msa) = \MK(x',\msa)$.
If $\MKK$ is a kernel coupling of $\MK$, then for every $n \in \nset$, $\MKK^n$ is a kernel coupling of $\MK ^n$ and for every $\nu \in \couplingmeasure (\xi,\xi')$, $\nu \MKK^n$ is a coupling of $(\xi \MK ^n,\xi'\MK^n)$, which means
$ \wasser[\cost]{\xi \MK ^n}{\xi' \MK ^n} \leq \int_{\Xset\times\Xset} \MKK^n\cost(x,x') \nu(\rmd x\rmd x')$.
For any  probability measure $\nu$ on $(\Xset^2,\Xsigma^{\otimes 2})$, we denote by
$\PP_{\nu}^\MKK$ (respectively $\PE_{\nu}^\MKK$) the probability (respectively the
expectation) on the canonical space $((\Xset^2)^\nset,(\Xsigma^{\otimes 2})^{\otimes \nset})$  such that the canonical process $\sequenceDouble{X}{X'}[n][\nset]$ is a Markov chain with initial
probability $\nu$ and Markov kernel $\MKK$.  By convention, we set
$\PE_{x,x'}^{\MKK} = \PE_{\delta_{x,x'}}^{\MKK}$ for all $(x,x') \in \Xset^2$.
Consider the following assumption, which weakens the $\distance$-small set condition of \cite{hairer2011asymptotic} by allowing the contraction to occur in $m \in \nset^*$ steps:
\begin{assumption}
  \label{assG:kernelP_q_contractingset_m}
 There exist a kernel coupling $\MKK$ of $\MK$, $m \in \nset, \minorwas \in (0,1)$, $\boundmetric \geq 1$ such that  \begin{equation}
  \label{eq:assG:kernelP_q_contractingset_m}
\MKK \metricc(x,x') \leq \boundmetric \metricc(x,x') \eqsp, \qquad    \MKK^m \metricc(x,x') \leq (1 - \minorwas \indi{\CKset}(x,x'))\metricc(x,x') \eqsp,
\end{equation}
where $ \CKset = \{V \leq d\} \times \{V \leq d\}$ with $\lambda+2b/(1+d)<1$ where $\lambda$ and $b$ are given in \Cref{assG:kernelP_q}.
\end{assumption}
Define for $x,x'\in\Xset$, $\bar V(x,x') = \{V(x) + V(x')\}/2$, $\bar \lambda_m = \lambda^m+ 2 b_m/(1+d)$,
$b_m = b(1-\lambda^m)/(1-\lambda)$, and $\bar d = (d+1)/2$. Consider the equation with unknown $\delta \geq 0$,
\begin{equation}
\label{eq: delta def}
(1-\minorwas)\lr{\frac{\bar \lambda_m+ b_m+\delta}{1+\delta}} =  \frac{\bar \lambda_m \bar{d}+\delta}{\bar{d}+\delta}\eqsp.
\end{equation}
Since necessarily, $b \geq 1$, note that the left-hand side of this equation is  a decreasing function of $\delta$, while the right-hand side is an increasing function. Hence, \eqref{eq: delta def} has a unique positive root (denoted by $\rootwas$) if $(1-\minorwas)(\bar \lambda_m+ b_m) > \bar \lambda_m$, and we define
\begin{equation}
\label{eq:delta_star_def}
\deltawas =
\begin{cases}
\rootwas & \text{ if } (1-\minorwas)(\bar \lambda_m+ b_m) > \bar \lambda_m\eqsp, \\
0 & \text{otherwise}\eqsp.
\end{cases}
\end{equation}
We first note that the assumptions \Cref{assG:kernelP_q} and \Cref{assG:kernelP_q_contractingset_m} imply the existence and uniqueness of an invariant distribution $\pi$, and second, that for any initial $\xi$-distribution, the $\xi \MK ^n$-iterates geometrically converge  to the invariant distribution $\pi$- for the pseudodistance $\wassersym[\metricc^{1/2} \bar{V}^{1/2}]$. This result generalizes the weak Harris theorem of \cite{hairer2011asymptotic} (see also \cite[Theorem~20.4.5]{douc:moulines:priouret:soulier:2018}).
\begin{proposition}
\label{prop:wasser:geo}
Assume \Cref{assG:kernelP_q}, \Cref{assG:kernelP_q_contractingset_m} and \Cref{ass:cost_fun}, and let $q \in \nset^*$. Then for $(x,x')\in \Xset^2$, $p \le 2q$, and $n\in \nset$, $n \geq m$ it holds
\begin{equation}
\label{eq: constraction}
\PE_{x,x'}^{\MKK}[\metricc^{1/2}(X_n, X_n') \bar V^{p/(4q)}(X_n, X'_n)] \leq  \boundmetric^{m/2} \vartconstwas^{p/(2q)}  \metricc^{1/2}(x,x') \bar{V}^{p/(4q)}(x,x') \ratewas^{np/(2q)}  \eqsp,
\end{equation}
where
\begin{equation}
\label{eq:def:rho}
\ratewas  = \Bigl(\frac{\bar \lambda_m \bar{d}+\deltawas}{\bar{d}+\deltawas} \Bigr)^{1/(2m)} < 1 \eqsp,  \quad \vartconstwas = (1 + b/(1-\lambda) + \deltawas)^{1/2} / \ratewas^m \eqsp.
\end{equation}
\end{proposition}
\begin{corollary}
\label{cor:wasserstein-convergence}
Assume \Cref{assG:kernelP_q}, \Cref{assG:kernelP_q_contractingset_m}, and \Cref{ass:cost_fun}. Then $\MK$ admits a unique invariant probability measure $\pi$ satisfying $\pi(V) < \infty$. Moreover, for all
  initial distributions $\xi$ and $n\in \nset$,
  \begin{equation}
    \label{eq:wasser:geo:bound:pi}
    \wasser[\cost]{\xi \MK^n}{\pi} \leq \wasser[\cost^{1/2} \bar V^{1/2}]{\xi \MK^n}{\pi}
    \leq  (1/\sqrt{2}) \boundmetric^{m/2} \vartconstwas  \ratewas^{n}    \lrb{\xi(V^{1/2})+\pi(V^{1/2})} \eqsp.
  \end{equation}
\end{corollary}
\begin{proof}
The proof is postponed to \Cref{sec:proof-ros_W_q}.
\end{proof}
%definition of a new norm
For a measurable function  $\lyapW: \Xset \to \coint{1,\infty}$, set $\bar \lyapW(x,y) = (\lyapW(x) + \lyapW(y))/2$, and for  $\beta \in \rset_+$, define
\begin{equation*}
  \Nnorm[\beta, \lyapW]{f} = \max \bigg \{\sup_{\substack{x,x'\in\Xset\eqsp, \,\, x \neq x'}} \frac{|f(x) - f(x')|}{\metricc^{1/2}(x,x') \bar{\lyapW}^{\beta}(x,x')} , \, \sup_{x \in \Xset} \frac{|f(x)|}{\lyapW^{\beta}(x)} \bigg\}\eqsp,
\end{equation*}
and $\Lclass_{\beta,\lyapW} = \{f: \Xset \to \rset: \Nnorm[\beta,\lyapW]{f} < \infty\}$. The first main result of this section is a Rosenthal-type inequality for geometrically ergodic Markov chains in terms of the Wasserstein semi-metric. Again, the leading term is the stationary variance multiplied by the corresponding moment of a Gaussian random variable.
\begin{theorem}
\label{th:rosenthal_V_poly_wasserstein}
Assume \Cref{assG:kernelP_q}, \Cref{assG:kernelP_q_contractingset_m}, \Cref{ass:cost_fun}, and let $q \in\nsets$.
Then for any function $g \in \Lclass_{1/(4q), V}$,
\begin{equation}
\label{eq:wasser_scaling_main}
    \PE_\pi[|S_n|^{2q}] \leq \momentGq[q] \{\PVar[\pi](S_n)\}^q + \Constwasspoly^{2q} \Nnorm[1/(4q), V]{\bar{g}}^{2q} \sum_{u=1}^{q-1} \frac{\ConstB_0(u,q) n^{u}}{\ratewas^{u/2} \{\log(1/\ratewas)\}^{2q-u}}\eqsp,
\end{equation}
where $\ConstB_0(u,q)$ is defined in~\eqref{eq: B_u_q_def_new} and with $\vartconstwas$ in \eqref{eq:def:rho},
\begin{equation}
\label{eq:const_poly_class_wasserstein}
\Constwasspoly = 2\sqrt{2} \boundmetric^{m/2} \vartconstwas \{\pi(V)\}^{1/2}\eqsp.
\end{equation}
\end{theorem}
\begin{proof}
The proof is postponed to~\Cref{sec:proof:rosenthal_V_poly_wasserstein}.
\end{proof}

Note that the right-hand side of \eqref{eq:wasser_scaling_main} has the same homogeneous scaling with respect to the ratio $n/\log(1/\ratewas)$ as in the corresponding bound for the $V$-geometrically ergodic case \eqref{eq:main-rosenthal}. We can now extend this result to the non-stationary case in a similar way to \Cref{theo:changeofmeasure}.
We use here a coupling argument but unlike \Cref{theo:changeofmeasure} we do not use a distributional coupling but a coupling kernel together with the coupling inequality outlined in \Cref{prop:wasser:geo}.
\begin{theorem}
\label{theo:changeofmeasure_wasser}
Assume \Cref{assG:kernelP_q}, \Cref{assG:kernelP_q_contractingset_m}, \Cref{ass:cost_fun}, and let $q \in \nset^*$. Then, for any probability measure $\xi$ on $(\Xset,\Xsigma)$ satisfying $\xi(V^{1/2}) < \infty$ and $g \in \Lclass_{1/(4q), V}$, we get
\begin{equation*}
     \PE_\xi\big[ \big|S_n \big|^{2q} \big] \leq 2^{2q-1} \PE_\pi\big[ \big|S_n \big|^{2q} \big] + 2^{4q-1} \Nnorm[1/(4q), V]{\bar{g}}^{2q}\,\boundmetric^{m/2}  \vartconstwas \{\xi(V^{1/2}) + \pi(V^{1/2})\} \frac{q^{2q}}{\ratewas (\log(1/\ratewas))^{2q}}\eqsp.
\end{equation*}
\end{theorem}
\begin{proof}
  The proof is postponed to \Cref{sec:proof-crefth_wass_change_mease}.
\end{proof}

Finally, we provide a series of results where we replace the class $\Lclass_{1/(4q), V}$ by the class $\Lclass_{1, W^\gamma}$ for $\gamma \geq 0$. We first prove a Rosenthal-type inequality in the stationary case, which we then extend to the arbitrary inital distribution. Results below are the analogues of \Cref{th:rosenthal_log_V} and \Cref{theo:changeofmeasure-1}. The proof in the stationary case again involves an inequality on centered moments adapted to the weighted Wasserstein distance. The extension to the non-stationary case still requires a coupling inequality but more subtle than for \Cref{theo:changeofmeasure_wasser}.
\begin{theorem}\label{th:rosenthal_log_V_wasserstein}
  Assume \Cref{assG:kernelP_q}, \Cref{assG:kernelP_q_contractingset_m}, \Cref{ass:cost_fun}, let $\gamma \geq 0$, $q \in \nset^*$.  Then for any $g \in \Lclass_{1, W^\gamma}$, we get
\begin{equation*}
\PE_{\pi}[|S_n|^{2q}] \leq \momentGq[q] \{\PVar[\pi](S_n)\}^q+  \Constwasspoly^{2q} (2\gamma)^{2\gamma q} \Nnorm[1,W^\gamma]{\bar{g}}^{2q}\,\sum_{u=1}^{q-1} \frac{\ConstB_{\gamma}(u,q) n^{u}}{\ratewas^{u/2} \log^{2q-u}{(1/\ratewas)}}\eqsp,
\end{equation*}
where the constant $\Constwasspoly$ is defined in \eqref{eq:const_poly_class_wasserstein}.
\end{theorem}
\begin{proof}
The proof is postponed to~\Cref{sec:proof_ros_W_log}.
\end{proof}


\begin{theorem}
\label{theo:changeofmeasure-1_wasser}
Assume \Cref{assG:kernelP_q}, \Cref{assG:kernelP_q_contractingset_m}, \Cref{ass:cost_fun}. Then for any probability measure $\xi$ on $(\Xset,\Xsigma)$, $\gamma \geq 0$, $q \in \nsets$ and function $g \in \Lclass_{1, W^\gamma}$, it holds
\begin{equation*}
     \PE_\xi\big[ \big|S_n \big|^{2q} \big] \leq 2^{2q-1} \PE_\pi\big[ \big|S_n \big|^{2q} \big] + 2^{2q-1} \Nnorm[1,W^\gamma]{\bar{g}}^{2q} \operatorname{D}^{(2)}_{q,\gamma} \eqsp,
\end{equation*}
\begin{multline*}
  \operatorname{D}^{(2)}_{q,\gamma}=
  \boundmetric^{m/2}  \vartconstwas \{ \xi(V^{1/2}) + \pi(V^{1/2}) \} \ratewas^{-1} \defEns{ \bigl(\frac{2\sqrt{2}}{\log(1/\ratewas)}\bigr)^{4q} (4q-1)! +  \frac{(8q\gamma/\rme)^{4q\gamma}}{\log(1/\ratewas)}}\eqsp.
\end{multline*}
\end{theorem}
\begin{proof}
The proof is postponed to \Cref{sec:proof-crefth-1_wass:theo:changeofmeasure-1_wasser}.
\end{proof}
We conclude with a Bernstein-type inequality. The following results extend \Cref{th:rosenthal_log_V_cor_2}
and \Cref{theo:prob_ineq_V_norm}. The proof of \Cref{th:rosenthal_log_V_cor_2} is straightforward due to the centered moment inequality. The non-stationary extension \Cref{th:rosenthal_log_V_cor_2_wasserstein_non_statio} requires more effort to obtain the correct dependence in the initial conditions (which is the same as in the $V$-geometric-ergodic case).
\begin{theorem}
\label{th:rosenthal_log_V_cor_2_wasserstein}
Assume \Cref{assG:kernelP_q}, \Cref{assG:kernelP_q_contractingset_m}, \Cref{ass:cost_fun}. Then,  for any $\gamma \geq 0$, $g \in \Lclass_{1, W^\gamma}$, and  $t \geq 0$,
\begin{equation}
\PP_{\pi}(|S_n| \geq t) \leq 2\exp\biggl\{-\frac{t^2/2}{\PVar[\pi](S_n) + \ConstJW^{1/(\gamma+3)} t^{2-1/(\gamma+3)}}\biggr\}\eqsp,
\end{equation}
where $\ConstJW$ is given by
\begin{equation}
\label{eq:const_J_n_definition_main_was}
\ConstJW = \biggl( \frac{n \ratewas^{-1/2} \{\log(1/\ratewas)\}^{-1} \Constwasspoly^{2} (2\gamma)^{4\gamma} \Nnorm[1, W^{\gamma}]{\bar{g}}^{2}}{\PVar[\pi](S_n)} \vee 1\biggr) \frac{2 (2\gamma)^{2\gamma} \Constwasspoly \Nnorm[1, W^{\gamma}]{\bar{g}}}{\log(1/\ratewas)}\eqsp.
\end{equation}
\end{theorem}
\begin{proof}
The proof is postponed to ~\Cref{sec:proof_bernstein_bound_wasserstein}.
\end{proof}


\begin{theorem}
\label{th:rosenthal_log_V_cor_2_wasserstein_non_statio}
Assume \Cref{assG:kernelP_q}, \Cref{assG:kernelP_q_contractingset_m}, \Cref{ass:cost_fun}. Then,  for any probability measure $\xi$ on $(\Xset,\Xsigma)$ satisfying $\xi(V^{1/2}) < \infty$,  $\gamma \geq 0$,  function $g \in \Lclass_{1, W^\gamma}$, and  $t \geq 0$, it holds that
\begin{align*}
&\PP_{\xi}(|S_n| \geq t) \leq
  \PP_{\pi}(|S_n| \geq t/2) \\
&\quad   +  \exp\parenthese{-\frac{\log(1/\ratewas) t^{\varpi_{\gamma}}}{2^{3+\varpi_{\gamma}} \Nnorm[1,W^\gamma]{\bar{g}}^{\varpi_{\gamma}} \varpi_{\gamma}}}\defEns{1+(-\log(\ratewas)/4)\frac{[ \boundmetric^{m/2}  \vartconstwas \{\pi(V^{1/2}) + \xi(V^{1/2})\}]^{1/2}}{\ratewas^{1/4}(1-\ratewas^{1/4})}} \\
  &\quad   +\exp\parenthese{-\frac{ (1+\gamma)\upsilon_{\gamma} t^{\varpi_{\gamma}}}{2^{5+\varpi_{\gamma}} \Nnorm[1,W^\gamma]{\bar{g}}^{\varpi_{\gamma}} \gamma }}\defEns{1+ \upsilon_{\gamma} \sup_{a \geq \rme} \{a^{4^{-1}\upsilon_{\gamma}}\log(a)\} \frac{[\boundmetric^{m/2}  \vartconstwas \{\pi(V^{1/2}) + \xi(V^{1/2})\}]^{\upsilon_{\gamma}}}{1-\ratewas^{\upsilon_{\gamma}}}}\eqsp,
\end{align*}
where $\varpi_{\gamma} = 1/(1+\gamma)$ and $\upsilon_{\gamma} = 1\wedge(2\gamma)^{-1}$\,.
\end{theorem}
\begin{proof}

The proof is postponed to \Cref{sec:proof-crefth:r_th:rosenthal_log_V_cor_2_wasserstein_non_statio}.
\end{proof}
\subsection{Related works}
\label{sec:related-works}
Moment bounds and the concentration of the additive function of Markov chains have been studied in many papers using a wealth of different techniques; the list of papers below does not claim to be exhaustive, but rather provides a selection of existing results and related theoretical tools. \cite{dedecker2015subgaussian} used coupling techniques to obtain Azuma-Hoeffding type inequality (the variance parameter is not considered) for geometrically ergodic Markov chains and bounded functions $g$ \footnote{\cite{dedecker2015subgaussian} considered separately bounded functions, which is more general than additive functionals}; this result was extended to unbounded functions by \cite{wintenberger2017exponential} but with random normalization.
 In \cite{marton1996measure}, Hoeffding inequalities are derived using Marton coupling. \cite{samson2000concentration} extends Marton's information-theoretic approach to obtain Gaussian concentration results for uniformly ergodic Markov chains and $\Phi$-mixing processes.
Probability bounds for Markov kernels that are contractive with respect to a Wasserstein distance are presented in \cite{joulin:ollivier:2010}. However, additional conditions are needed involving quantities such as \textit{granularity} and \textit{local dimension}, which are difficult to evaluate in most applications.

% In addition, probability bounds are
Using Kato's perturbation theory on the spectrum of bounded operators on Hilbert space \cite{kato:2013}, \cite{lezaud:1998} establishes Chernoff-type bounds for Markov chains on general state spaces and bounded functions $g$. This work was followed by
\cite{paulin2015concentration,fan:jiang:sun:2018:hoeffding,fan:jiang:sun:2018:bernstein}, which establish Hoeffding and Berstein probability bounds using spectral methods for Markov chains and bounded functions $g$ under the assumption that $\MK$ admits a positive absolute spectral gap.
% \cite{paulin2015concentration} established Bernstein-type inequalities for additive functionals of Markov chains and bounded $\{g_\ell\}_{\ell=0}^{n-1}$; for non-reversible Markov chains, the Bernstein bound depends on a proxy for the variance and a pseudo-spectral gap which is difficult to evaluate for most examples.
Note also that geometric ergodicity assumptions (see \Cref{assG:kernelP_q} and \Cref{assG:kernelP_q_smallset}) do not necessarily imply the existence of a spectral gap (see \cite{kontoyiannis2012geometric}).
% , moreover, evaluation of the constants in the Bernstein-type inequalities \cite[Theorems~3.4-3.5]{paulin2015concentration} is intractable.

\cite{kontoyiannis2003spectral,kontoyiannis2005large} develop the theory of multiplicative regular Markov chains based on multiplicative drift conditions which strengthen the classical Foster-Lyapunov drift conditions.
These conditions, introduced by \cite{varadhan:1984}, play a key role in the study of large deviations of additive functions of Markov chains. Multiplicative drift conditions are generally difficult to verify; see the discussion in \cite[Section~3.1]{adamczak2015exponential}. The bounds reported in these papers are not quantitative: the bounds depend on the multiplicative Poisson equation, which amounts to solving an eigenvalue problem for an operator associated with $\MK$.

\cite{clemenccon2001moment,bertail2010sharp,Adamczak2008,adamczak2015exponential,bertail2018new,lemanczyk2020general}
use regenerative decompositions to obtain, among others, moment bounds and Bernstein inequalities under \Cref{assG:kernelP_q} and \Cref{assG:kernelP_q_smallset}.
%These methods apply to $\phi$-irreducible Markov chains (assuming equivalently the existence of an accessible small set).
These techniques are based on the Numellin splitting construction (see \cite{athreya1978new} and \cite{Nummelin1978AST}), which allows the sum $S_n$ to be decomposed into a random number of single-valued blocks of random length. The regenerative decomposition allows one to derive exponential inequalities for additive functionals of Markov chains from the concentration of a (random) sum of one-dependent random variables, at the cost of some very non-trivial technical work. \cite[Theorem~1]{adamczak2015exponential} provides a Bernstein-type inequality for a $V$-uniformly geometrically ergodic strongly aperiodic Markov chain and unbounded functions. \cite[Theorem~1]{lemanczyk2020general} extends the result to aperiodic Markov chains, but is restricted to bounded functions and does not give an explicit expression for constants.


%\cite{kloeckner2019effective} and the references therein.

Moment bounds and Bernstein-type inequalities have also been obtained under various conditions of weak dependence/mixing; see \cite{doukhan:louhichi:1999,doukhan2007probability,merlevede2011bernstein}. In general, these results are not directly comparable because the bounds depend on different types of weak dependence/mixing coefficients instead of drift conditions and local minorization/contraction conditions. However, the connections between weak dependence/mixing assumptions and $V$-geometric ergodicity are discussed in detail in \cite{adamczak2015exponential}. The results based on weak dependence / mixing methods are more appropriate for the stationary case. The extensions for the non-stationary case are less accurate than those given in our work (the way the bounds depend on the initial conditions). Finally, note that our proof for the stationary case is based on the argument developed by \cite{doukhan2007probability}, which we adapt to the Markov case. Compared to that work, we replace a covariance bound with an accurate bound for centered moments.


\section{SIMULATION RESULTS}
\label{sec:examples}
This section presents simulation results of the proposed method implemented on the unicycle model example.
Each semidefinite program was prepared using a custom software toolbox and the modeling tool YALMIP \cite{lofberg2004yalmip}.
The programs are run with commercial solver MOSEK on a machine with $1$ TB availabe memory. 

\subsection{FRS Computation}
We computed the FRS for a 3$^\text{rd}$ order Taylor-expanded Dubins car as the low-fidelity model $f_s$.
Trajectories produced by this model were tracked by the unicycle model from Equation \eqref{eq:big_dyn} as the high-fidelity model $f$.
The vehicle's representation as an initial distribution $X_0 \subset X_s$, was a rectangle of length $0.2$ [m] in $x$ and width $0.1$ [m] in $y$, at $0^\circ$ initial heading, and centered at $x=-0.75$ and $y=0$.
This is the same vehicle representation shown in all previous figures.

% The error function $g$, illustrated in Figure \ref{fig:error_dynamics}, was given by:
% \begin{equation}
% \label{eq:g_definition}
% g(t,x_s) = \begin{bmatrix}
% v_\text{err}\cdot(1 - \frac{1}{2}\theta^2)  \\
% v_\text{err}\cdot(\theta - \frac{1}{6}\theta^3) \\
% \dot{\theta}_\text{err}
% \end{bmatrix}
% \end{equation}
% where $v_\text{err} = (t-1)^2$ and $\dot{\theta}_\text{err} = (t-1)^4$.
We chose $\tau_\text{stop} = \tau_\text{plan} = 0.5$ [s], so $T = 1$ [s].
The stopping time can be seen in Figure \ref{fig:error_dynamics}. 
The FRS computation took 79 hours and used a maximum of 150 GB of memory 
%on a server with 1 TB of available memory and 18 processors each running at 1.2 GHz.

\subsection{Set Intersection and Trajectory Planning}

We used the precomputed FRS for safe trajectory planning in $1000$ simulated trials in MATLAB on the aforementioned machine.
For each trial, the vehicle began at the same initial location and heading, surrounded by $1-10$ randomized obstacles and a randomly-located goal to reach.
%If the planning time took more than $\tau_\text{plan}$, the simulation paused until the computation was complete. 
%In practice, if $\tau_\text{plan}$ was exceeded the vehicle could begin braking to ensure safety.
The vehicle's initial speed, and the desired speed to maintain for the duration of the trial, were randomly chosen between $0.25$ and $0.75$ [m/s].
% The trials ran in 12.7 hours.
% Prior to running these trials, several example trials were run on a laptop with a 2.3 GHz processor and 16 GB of RAM.
% The trials run on the server were individually no faster than running on the laptop, because the set intersection optimization is a single-core process that uses very little memory. 
% Therefore, the server did not provide any significant decrease in the implemented planning time.


Obstacles were represented as line segments between $0.1$ and $0.2$[m] in length, with random location and orientation.
The obstacles were always placed between the vehicle and the goal.
We checked for crashes conservatively for each trial, by inspecting if any obstacle was within a circle circumscribing the rectangular vehicle at any point of the vehicle's trajectory. 
Using this method, \emph{no crashes were detected in any trial}.
Out of all the trials, $82\%$ reached the goal, and $15\%$ performed an emergency braking maneuver (by setting $v_\text{des} = 0$). 
The remaining 3\% hit a simulation iteration limit.
Examples of the vehicle's path from a randomly-generated trial and from two constructed emergency braking cases are shown in Figure \ref{fig:example_trial}.


\begin{figure}
\centering
\includegraphics[width=1\columnwidth]{running_examples.pdf}
\caption{The top subplot shows an example result out of the $1000$ trials.
This trial used eight randomly-generated obstacles.
The vehicle begins on the left at $x = -0.75$ and reaches a randomly-generated goal near $(2.5, 0.5)$, plotted as a blue circle.
Every $\tau_\text{plan} = 0.5$[s], the vehicle replans its trajectory, shown by an asterisk plotted on the global trajectory in blue.
The bounding box of the vehicle at each planning step is shown as a grey rectangle. In the bottom-left subplot, an obstacle was constructed between the vehicle and the goal, forcing an emergency braking maneuver. In the bottom-right subplot, an obstacle was constructed with a hole that would allow the vehicle to pass, but the set intersection result is overly conservative, resulting in a braking maneuver.}
\label{fig:example_trial}
\end{figure}

Currently, our implementation cannot consistently achieve $\tau_\text{plan} = 0.5$ [s].
Consequently, instead of replanning and driving simultaneously, we pause time every 0.5 [s] of the simulation to guarantee that the vehicle can finish replanning.
In a physical implementation, if $\tau_\text{plan}$ is exceeded, then the vehicle must emergency brake; recall that a safe braking trajectory is always available.
As shown in Figure \ref{fig:planning_time_vs_Nobs}, $\tau_\text{plan}$ scales linearly with the number of obstacles.
%Methods for reducing the set intersection to meet $\tau_\text{plan}$ will be presented in future work.

\begin{figure}
\centering
\includegraphics[scale=0.45,trim={1cm 6cm 1cm 7cm},clip]{planning_time_vs_Nobs.pdf}
\caption{The mean set intersection time (top) and trajectory optimization time (bottom) versus the number of obstacles. Over the $1000$ trials, each number of obstacles from $1$ to $10$ was used for $100$ trials. Notice that set intersection takes up to $3$[s], and scales with the number of obstacles. On the other hand, the trajectory optimization takes around $80$ [ms] and has low correlation with number of obstacles.}
\label{fig:planning_time_vs_Nobs}
\end{figure}

% \begin{figure}
% \centering
% \includegraphics[scale=0.5,trim={1cm 8cm 1cm 8cm},clip]{example_trial_bluecar.pdf}
% \caption{An example result out of the 1000 trials.
% This trial used eight randomly-generated obstacles.
% The vehicle begins on the left at $x = -0.75$ and reaches a randomly-generated goal near $(2.5, 0.5)$, plotted as a blue circle.
% Every $\tau_\text{plan} = 0.5$ [s], the vehicle replans its trajectory, shown by an asterisk plotted on the global trajectory in blue.
% The bounding box of the vehicle at each planning step is shown as a grey rectangle.}
% \label{fig:example_trial}
% \end{figure}

% \begin{figure}
% \centering
% \includegraphics[scale=0.4,trim={1cm 7cm 1cm 7cm},clip]{example_emergency_brake.pdf}
% \caption{An example of a forced emergency braking situation. The vehicle cannot find a path to the desired location (plotted as a blue circle), so it brakes.}
% \label{fig:example_emergency_brake}
% \end{figure}

% \begin{figure}
% \centering
% \includegraphics[scale=0.4,trim={1cm 7cm 1cm 7cm},clip]{example_overly_conservative.pdf}
% \caption{An example of an unnecessary emergency braking situation. The vehicle cannot find a path to the desired location despite an obviously-safe path existing, because the FRS is overly conservative.}
% \label{fig:example_overly_conservative}
% \end{figure}

\section{Influence in Completely Bounded Block-multilinear Forms}
\label{sec:proof}
\newcommand{\blocks}{\mathrm{blocks}}
\newcommand{\lt}{\mathrm{left}}
\newcommand{\rt}{\mathrm{right}}



In this section we prove the non-commutative root-influence inequality (\thmref{thm:bh-intro}),  the special case of the Aaronson-Ambainis conjecture given in \thmref{thm:aa}, and also briefly mention how the simulation result in \corref{cor:sim} follows from \thmref{thm:aa} and the results in \cite{AA14}. We first need some preliminaries from free probability theory. 



\subsection{Low-degree Polynomials of Haar Random Unitaries}

As discussed in the proof overview, we require bounds on the operator norm (as well as normalized trace) of low-degree polynomials of random unitaries and these follow from known results in free probability theory. Here we explain these connections and also prove some auxillary lemmas needed for the proof of \thmref{thm:bh-intro} and \thmref{thm:aa}. 



Let $z_{\ui}$ denote the non-commutative monomial $z_{i_1} z_{i_2} \cdots z_{i_d}$ for a $d$-tuple $\ui  = (i_1, \ldots, i_d) \in [t]^d$ and let $p(z_1, \ldots, z_t)$ be a non-commutative polynomial in the variables $z_1, \ldots, z_t$. We are interested in computing the operator norm $\|\cdot\|_{\op}$ and the normalized trace  $\tr_N$ of the polynomial $p(z_1, \ldots, z_t)$ (or its higher moments) when substituting $N \times N$ Haar random unitaries for the variables $z_i$.

As explained previously, the theory of free probability gives us tools that allow us to compute  the above in the limit $N \to \infty$. In particular, Voiculescu \cite{V98} showed that the  (normalized) trace of polynomials in Haar random unitaries and their conjugates converges to the trace of the same polynomial evaluated on certain infinite-dimensional operators called \emph{Haar unitaries} that satisfy a non-commutative notion of independence called \emph{free independence}. This was strengthened by Collins and Male \cite{CM11} who showed that such convergence also holds for the operator norm. A short primer on free probability is given in \appref{sec:free}, but for now one can think of $\CA$ as a self-adjoint algebra of bounded linear operators on a Hilbert space and $\phi$ as a trace functional for such operators in the statement given below.


\begin{theorem}[\cite{V98, CM11}] \label{thm:voiculescu}
    Let $p(z_1, \ldots, z_{2t})$ be a non-commutative polynomial in $\BR\langle z_1, \ldots, z_{2t}\rangle$. If $U_1, \ldots, U_t$ are $N \times N$ Haar random unitaries, then almost surely,
    \begin{align*}
     \ \tr_N[p(U_1, \ldots, U_t, U^*_1, \ldots, U_t^*)] &~\xrightarrow[N \to \infty]{}~ \phi[p(u_1, \ldots, u_t, u^*_1, \ldots, u^*_t)],\\
    \  \|p(U_1, \ldots, U_t, U^*_1, \ldots, U_t^*)\|_{\op} &~\xrightarrow[N \to \infty]{}~ \| p(u_1, \ldots, u_t, u^*_1, \ldots, u^*_t)\|,
    \end{align*}
    where $u_1, \ldots, u_t$ are free Haar unitaries in a $C^*$-probability space $(\CA, \phi)$ and $\|\cdot\|$ is the norm for the underlying $C^*$-algebra.
\end{theorem}




Using the above result it suffices to consider free Haar unitaries in a $C^*$-probability space to compute the operator norm and trace of polynomials of random unitaries. For a non-commutative polynomial $p(z_1, \ldots, z_t) = \sum_{|\ui|\le d} c_{\ui}z_{\ui}$, denoting by $\|p\|_2 =  \left(\sum_{|\ui| \le d} |c_{\ui}|^2\right)^{1/2}$, one can show the following easily using techniques from free probability. 

\begin{lemma} \label{thm:trace}
    Let $p(z_1, \ldots, z_t) = \sum_{|\ui|\le d} c_{\ui}z_{\ui} $ be a non-commutative degree-$d$ polynomial in $\R\langle z_1, \ldots, z_t\rangle$ and $u_1, \ldots, u_t$ be free Haar unitaries in a $C^*$-probability space $(\CA, \phi)$. Then, 
     \[ \phi[p(u_1, \ldots, u_t) (p(u_1, \ldots, u_t))^*] =  \|p\|_2^2.\]
\end{lemma}

The above implies that $\tr_N[p(U_1, \ldots, U_t) (p(U_1, \ldots, U_t))^*]$ converges to $\|p\|_2^2$ almost surely as $N \to \infty$. We shall defer the proof of \lref{thm:trace} to \appref{sec:app}, but to aid our intuition we note here that since the  $U_i$'s are independent $N \times N$ Haar random unitaries, the expected value

\[ \BE\left[\tr_N[p(U_1, \ldots, U_t) (p(U_1, \ldots, U_t))^*\right] = \|p\|_2^2,\] 
{and from concentration of measure, it is natural to expect that it converges to the above value}. 


Similarly, to compute the operator norm of $p(U_1, \ldots, U_t)$ for Haar random unitaries one can instead study the norm of the polynomial evaluated on free Haar unitaries. Such bounds are easier to prove using the trace method since free independence imposes strong restrictions on the non-commutative moments. For instance, if $U_1$ and $U_2$ are independent $N \times N$ Haar random matrices, then $\BE[\tr_N(U_1U_2U^*_1U_2^*)]$ is non-zero (albeit quite small), while the corresponding trace evaluated on free Haar unitaries $u_1$ and $u_2$ is zero, that is $\phi(u_1u_2u^*_1u_2^*) = 0$. Thus, computing the trace $\phi[p(u_1,\ldots, u_t, u^*_1, \ldots, u_t^*)]$ reduces to handling the combinatorics of the patterns of $u_i$'s and $u_i^*$'s. 

In particular, we will rely on the following result that follows from the work of Kemp and Speicher \cite{KS05}  who consider the operator norm of homogeneous polynomials evaluated on free $R$-diagonal operators, a class that includes free Haar unitaries as well. We also remark that a bound where the right-hand side below is worse by a multiplicative $O(d^{1/2})$ factor also follows from the work of Haagerup\footnote{We note that Haagerup considered the more general case of polynomials in both $u_i$'s and $u^*_i$'s.}\cite{H78} who proved it in another context, predating even the introduction of free probability theory. 


\begin{theorem}[\cite{KS05}]
\label{thm:kemp-speicher}
    Let $p(z_1, \ldots, z_t) = \sum_{|\ui| = d} c_{\ui}z_{\ui} $ be a homogeneous non-commutative degree-$d$ polynomial in $\R\langle z_1, \ldots, z_t\rangle$ and $u_1, \ldots, u_t$ be free Haar unitaries in a $C^*$-probability space. Then, 
    \[ 
    \|p(u_1, \ldots, u_t)\| \le \sqrt{e(d+1)} \cdot \|p\|_2,
    \]
    where the left-hand side denotes the norm in the underlying $C^*$-algebra. 
\end{theorem}

For completeness, we  introduce the necessary free probability background and some combinatorial details in \appref{sec:app}, and we present the fairly short proof of \thmref{thm:kemp-speicher} (from \cite{KS05}) there in a self-contained way. We shall need to extend the above bound to non-homogeneous polynomials. Let $p(z_1, \ldots, z_t) = \sum_{|\ui| \le d} c_{\ui}z_{\ui}$ and  let $p_k(z_1, \ldots, z_t) = \sum_{|\ui| = k} c_{\ui}z_{\ui}$ denote the degree-$k$ homogeneous part of $p$. Writing $p_k = p_k(u_1, \ldots, u_t)$ for $0 \le k  \le d$ and $p = p(u_1, \ldots, u_t)$, it follows from the triangle inequality,  \thmref{thm:kemp-speicher}, and Cauchy-Schwarz, that
    \begin{align*}
        \ \|p\| &\le \sum_{k=0}^d \|p_k\| 
        \le 
        \sum_{k=0}^d\sqrt{e(k+1)}\|p_k\|_2
        \le
       \sqrt{e}\left(\sum_{k=0}^d (k+1)\right)^{1/2} \left(\sum_{k=0}^d  \|p_k\|^2_2\right)^{1/2} \leq \sqrt{e}(d+1)  \cdot\|p\|_2.
    \end{align*}
Thus, we essentially get the same bound as in the homogeneous case, at the expense of an additional $O(d^{1/2})$ factor.



Collecting all the above we have the following as a direct consequence:

\begin{theorem} \label{thm:op-norm}
    Let $p(z_1, \ldots, z_t) = \sum_{|\ui|\le d} c_{\ui}z_{\ui} $ be a non-commutative degree-$d$ polynomial in $\R\langle z_1, \ldots, z_t\rangle$ and $U_1, \ldots, U_t$ be independent $N \times N$ Haar random unitaries. Then, as $N \to \infty$, the following holds almost surely, 
    \[ \tr_N[p(U_1, \ldots, U_t) (p(U_1, \ldots, U_t))^*] =  \|p\|_2^2,\]
    and
    \[ \|p(U_1, \ldots, U_t)\|_{\op} \le \sqrt{e}(d+1)  \cdot \|p\|_2,\]
    Moreover, the factor $(d+1)$ in the operator norm bound can be improved to $\sqrt{d+1}$ if the polynomial is homogeneous.
\end{theorem}

Based on the above theorem, we prove the following key lemma which captures the polar decomposition strategy mentioned in the earlier proof overview (\secref{sec:bh}). This will serve as the key ingredient in the proof of \thmref{thm:aa} and \thmref{thm:bh-intro}. 

\begin{lemma}\label{lem:polar}
    Let $p$ be a non-commutative degree-$d$ polynomial in $\R\langle y_1, \ldots, y_m, z_1, \ldots, z_t\rangle$ given by
    \[ p(y_1, \ldots, y_m, z_1, \ldots, z_t) = \sum_{i=1}^m y_i q_i(z_1, \ldots, z_t) + q_0(z_1, \ldots, z_t).\]
    Then, for every $\delta > 0$, there exist an integer $N$ and $N \times N$ unitaries $V_1,\ldots, V_m, W_1, \ldots, W_t$ such that 
    \[ \|p(V_1, \ldots, V_m, W_1, \ldots, W_t)\|_{\op} \ge \frac{1}{\sqrt{e}(d+1)} \sum_{i=1}^m \|q_i\|_2 - \delta.\]
    Moreover, the factor in front can be improved to $(e(d+1))^{-1/2}$ if $p$ is homogeneous. 
\end{lemma}

\begin{proof}[Proof of \lref{lem:polar}]
     For an arbitrary integer $N$, let us pick independent $N \times N$ Haar random unitaries $W_1, \ldots, W_t$ which we substitute for the variables $z_1,\ldots,z_t$, respectively, and let $M_i = q_i(W_1, \ldots, W_t)$ be the corresponding random matrices. Then, for any tuple of matrices $V_1, \ldots, V_m$ that we could substitute for the variables $y_1, \ldots, y_m$, we have that 
    \[ 
    p(V_1, \ldots, V_m, W_1, \ldots, W_t) = \sum_{i=1}^m V_i M_i + M_0.
    \] 
     \thmref{thm:op-norm} and union bound imply that as $N \to \infty$, with probability $1$ all the following events simultaneously hold: 
    \begin{itemize}
        \item $\|M_i\|_{\op} \le \sqrt{e}(d+1) \cdot \|q_i\|_2$ for each $i$,
        \item $\tr_N(M^*_iM_i) = \|q_i\|_2^2$ for each $i$, where $\tr_N(M)$ is the normalized trace.
    \end{itemize}
   To show that the operator norm must be large, let us fix a sufficiently large $N$ and a choice of $N\times N$ unitaries $W_1, \ldots, W_t$ such that $M_i$ satisfies $\|M_i\|_{\op} \le \sqrt{e}(d+1) \cdot \|q_i\|_2 + \epsilon$ and $\tr_N(M^*_iM_i) \ge \|q_i\|_2^2 - \epsilon$ for each $0\le i\le m$, where $\epsilon$ can be made arbitrarily small by increasing $N$. For $0 \leq i \leq m$, let $M_i = U_i P_i$ be the left polar decomposition of $M_i$, where $U_i$ is a unitary matrix and $P_i$ is a positive semidefinite matrix.
   
   We select the tuple of unitary matrices $V_1, \ldots, V_m$ that we substitute for the variables $y_1, \ldots, y_m$ to be $V_i = U_0U^*_i$ for $i \in [m]$. With this we have that $\|p(V_1, \ldots, V_m, W_1, \ldots, W_t)\|_{\op}$ is at least
    \begin{align*}
         \Big\|M_0 + \sum_{i=1}^m V_iM_i\Big\|_{\op} & = \Big\|U_0 P_0 + \sum_{i=1}^m U_0 U_i^* U_iP_i \Big\|_{\op} \\
        \ & =  \Big\|U_0 P_0 + \sum_{i=1}^m U_0 P_i\Big\|_{\op}  = \Big\| P_0 + \sum_{i=1}^m  P_i\Big\|_{\op}\ge \tr_N\Big(P_0 + \sum_{i=1}^m P_i\Big) \ge \tr_N\Big(\sum_{i=1}^m P_i\Big),
    \end{align*}
    where the last equality follows since the operator norm is unitarily invariant and the last two inequalities follow from the positive semidefiniteness of the $P_i$'s.

    For every positive semidefinite matrix $P$, we have that $\tr_N(P) \ge {\tr_N(P^2)}/{\|P\|_{\op}}$. 
  
    Hence,
     \[ \|p(V_1, \ldots, V_m, W_1, \ldots, W_t)\|_{\op} \ge \sum_{i=1}^m \frac{\tr_N(P_i^2)}{\|P_i\|_{\op}}.\]
     By our choice of $M_i$, we have that $\tr_N(P_i^2) = \tr_N(M_i^* M_i) \ge \|q_i\|_2^2 - \eps$ and $\|P_i\|_{\op} = \|M_i\|_{\op} \le \sqrt{e}(d+1)\|q_i\|_2 + \eps$. Since $\eps$ can be made arbitrarily small by increasing $N$, it follows that 
      \[ \|p(V_1, \ldots, V_m, W_1, \ldots, W_t)\|_{\op} \ge \frac1{\sqrt{e}(d+1)} \sum_{i=1}^m \|q_i\|_2 - \delta ,\]
     for large enough $N$. The improved bound for the homogeneous case follows directly by plugging the bound of \thmref{thm:op-norm} into the above proof.
\end{proof}





\subsection{Non-commutative root-influence inequality}
\label{sec:bh-proof}


For clarity in the proofs below, we remind our  convention that all tuples or blocks are denoted with boldface fonts (e.g. $\BU_1$ or $\BA$), while a single element is denoted without boldface (e.g. $U_1(i)$ or $A_i$ or $A$). Before proceeding with the proof, we restate the statement for convenience.

\bh*





\begin{proof}[Proof of \thmref{thm:bh-intro}] 
Since $f$ is homogeneous, we can write
   \begin{align*}
    f(\x_1,\ldots, \x_d) &= \sum_{i_1, \ldots, i_d \in [n]} \hf_{i_1, \ldots, i_d} ~x_1(i_1)x_2(i_2)\cdots x_d({i_d}) \\
    \ & = \sum_{i=1}^n  x_1(i) \underbrace{\left(\sum_{i_2,\ldots, i_d \in [n]} \hf_{i_1, \ldots, i_d} ~x_2(i_2)\cdots x_d({i_d})\right)}_{\textstyle := f_i(\x_2,\ldots, \x_d)}.
\end{align*}
 In this case, it follows from \eqref{eqn:inf-tensor} that for each $i \in [n]$, we have 
 \begin{equation}\label{eqn:var}
     \ \Var[f_i] = \|f_i\|^2_2 = \inf_{1,i}(f) \text{ and }  \Var[f] = \sum_{i=1}^n \inf_{1,i}(f).
 \end{equation}

  Let us denote the corresponding non-commutative block-multilinear polynomials by $f(\BU_1, \ldots, \BU_d)$ and $f_i(\BU_2, \ldots,\BU_d)$ where $\BU_b = (U_b(1), \ldots, U_b(n))$ denotes the $b^\text{th}$ block of non-commutative variables. To show a lower bound on $\cbnorm{f}$ it suffices to exhibit a collection of square matrices $\{U_b(i)\}_{b\in [d], i \in [n]}$ with operator norm at most~1, such that $\|f(\BU_1, \ldots, \BU_d)\|_{\op}$ is large. 
  
%  

Applying \lref{lem:polar} for the homogeneous case (with $p = f$, $q_i=f_i$ for $i \in [n]$, and $q_0=0)$, it follows that for every $\delta > 0$ there exists an integer $N$ and a choice of tuples of $N \times N$ unitaries $\BU_1, \ldots, \BU_d$ such that  
      \[ \cbnorm{f} \ge \|f(\BU_1, \ldots, \BU_d)\|_{\op} \ge \frac1{\sqrt{e(d+1)}} \sum_{i\in [n]} \|f_i\|_2  -\delta \stackrel{\eqref{eqn:var}}{\ge}  \frac{1}{\sqrt{e(d+1)}} \left(\sum_{i=1}^n \sqrt{\Inf_{1,i}(f)} \right) -\delta.\]
Taking $\delta \to 0$, we get the statement of the lemma. The proof for the inequality when $b=d$ is the last block follows similarly by using the right polar decomposition.
\end{proof}

\subsection{Aaronson-Ambainis Conjecture for non-homogeneous forms}

In this section, we prove \thmref{thm:aa}, which requires handling non-homogeneous forms. The proof will be similar to the proof of \thmref{thm:bh-intro} but we will need to be careful about certain details. 

\begin{proof}[Proof of \thmref{thm:aa}]
Any block-multilinear polynomial $f(x_1, \ldots, x_d)$ can be written as 
\begin{align*}
    f(\x_1,\ldots, \x_d) &= \BE f + \sum_{b\in [d]} f_b(\x_b, \x_{b+1}, \ldots, \x_d),
\end{align*}
where $f_b$ consists of all monomials of $f$ that start with a variable in the $b^\text{th}$ block $\x_b$. Note that $f_b$ depends only on the variables in blocks $\x_b, \x_{b+1},\ldots, \x_d$. Moreover, it follows from \eqref{eqn:inf-tensor} that 
 \begin{equation}\label{eqn:var-general}
     \ \Var[f] = \sum_{b \in [d]} \|f_b\|_2^2 = \sum_{b \in [d]} \Var[f_b],
 \end{equation}
so there exists a block $\beta \in [d]$ such that $\Var[f_{\beta}] \ge \frac{1}{d}\Var[f]$. 

Since $f_{\beta}$ contributes a lot to the variance, it is natural to try to find an influential variable in the block $\x_{\beta}$. Towards this end,  we pull out the variables $x_{\beta}(i)$ and write
\begin{align*}
    f_{\beta}(\x_{\beta},\ldots, \x_d) &= \sum_{i\in [n]} x_{\beta}(i) f_{\beta,i}(\x_{\beta+1}, \ldots, \x_d),
\end{align*}
for block-multilinear polynomials $f_{\beta,i}(\x_{\beta+1}, \ldots, \x_d)$. Note that some of the $f_{\beta,i}$'s could be identically zero, so let us define $S$ to be the set of those $i$ such that $f_{\beta,i}$ is non-zero. We note that
\begin{align} \label{eqn:part-inf}
  \|f_{\beta,i}\|_2^2  =  \Inf_{\beta,i}(f_{\beta}) \le \Inf_{\beta,i}(f)  
\end{align}
which implies that
\begin{align}\label{eqn:var-main}
    \frac{1}{d} \Var[f] \le \Var[f_{\beta}] = \sum_{i \in S}\|f_{\beta,i}\|_2^2 = \sum_{i \in S} \Inf_{\beta,i}(f_{\beta}).
\end{align}
\begin{sloppypar}
Denote the corresponding non-commutative block-multilinear polynomials by $f(\BU_1, \ldots, \BU_d)$,  $f_b(\BU_{b}, \ldots,\BU_d)$, and $f_{\beta}(\BU_{\beta+1}, \ldots,\BU_d)$ where $\BU_b = (U_b(1), \ldots, U_b(n))$ denotes the $b^\text{th}$ block of non-commutative variables. To show a lower bound on $\cbnorm{f}$ it suffices to exhibit a collection of square matrices $\{U_b(i)\}_{b\in [d], i \in [n]}$ with operator norm at most~1 such that $\|f(\BU_1, \ldots, \BU_d)\|_{\op}$ is large.
\end{sloppypar}
  
 We set the matrices in blocks $\BU_1, \ldots, \BU_{\beta-1}$ to be zero (that is, the all-zero matrix $\BZ$). Note that with this choice all polynomials $f_b(\U_b, \ldots, \U_d)$ where $b < \beta$ vanish and the non-commutative polynomial becomes 
 \[ f(\BZ, \ldots, \BZ, \BU_{\beta}, \BU_{\beta+1}, \ldots, \BU_d) = \sum_{i\in S} U_{\beta}(i) f_{\beta,i}(\BU_{\beta+1}, \ldots, \BU_d) + \sum_{b=\beta+1}^d f_b(\BU_b, \BU_{b+1}, \ldots, \BU_d) + \Ef,\]
  which is a non-commutative polynomial of the form considered in \lref{lem:polar} (with $m = |S|$, $q_i = f_{\beta,i}$ and $q_0 = \sum_{b=\beta+1}^d f_b + \Ef$). Thus, by \lref{lem:polar} for every small $\delta>0$ there exists an integer $N$ and a choice of $N \times N$ matrices for the blocks $\BU_{\beta},\ldots, \BU_d$ such that 
        \begin{align*}
             \ \cbnorm{f} & \ge \|f(\BZ, \ldots, \BZ, \BU_{\beta}, \BU_{\beta+1}, \ldots, \BU_d)\|_{\op} & \\
             \  & \ge \frac1{\sqrt{e}(d+1)} \sum_{i\in S} \|f_{\beta,i}\|_2 -\delta  \stackrel{\eqref{eqn:part-inf}}{=}  \frac{1}{\sqrt{e}(d+1)} \left(\sum_{i \in S} \sqrt{\Inf_{\beta,i}(f_{\beta})} \right) -\delta & \\
             \ &\stackrel{\eqref{eqn:var-main}}{\ge}  \frac{1}{\sqrt{e}(d+1)} \left( \frac{\sum_{i \in S} \Inf_{\beta,i}(f_{\beta})}{\sqrt{\maxinf(f)}} \right) -\delta  \stackrel{\eqref{eqn:part-inf}}{\ge}  \frac{1}{\sqrt{e}(d+1)^{2}} \left( \frac{\Var[f]}{ \sqrt{\maxinf(f)}} \right) -\delta
        \end{align*}
        Taking $\delta \to 0$ and using the assumption that $\|f\|_{\cb} \le 1$, we obtain the statement of the theorem:
     \[
     1\geq \cbnorm{f} \ge \frac{1}{\sqrt{e}(d+1)^{2}} \cdot \frac{\Var[f]}{\sqrt{\maxinf(f)}} \implies \maxinf(f) \ge  \frac{(\Var[f])^2}{e(d+1)^4}. \qedhere
     \]
\end{proof}
 

     
     

\subsection{Approximating completely bounded forms with decision trees}



In this section, we briefly mention how to obtain \corref{cor:sim}.
Aaronson and Ambainis \cite[Theorem 3.3]{AA14} showed that querying the most influential variable reduces the variance of the function~$f$, and if that influence is lower bounded by a polynomial in $\Var[f]/d$, then after $\poly(d)$ queries (the exact quantitative dependence can be read off from their proof), the variance of the function becomes small enough so that it can be approximated almost-everywhere by its expectation.  Since the family of degree-$d$ block-multilinear forms with completely bounded norm at most one is closed under restrictions, one can apply \thmref{thm:aa} repeatedly. This gives us \corref{cor:sim}.

\section*{Declarations}

\textbf{Conflict of interest. } The authors have no competing interests to declare that are relevant to the content of this article.

\noindent\textbf{Data availability. } Data sharing is not applicable to this article as no datasets were generated or analyzed during the current study.

\newpage
\bibliography{biblio}
\end{document}



\bibliography{sn-bibliography}% common bib file
%% if required, the content of .bbl file can be included here once bbl is generated
%%\input sn-article.bbl


%\end{document}
