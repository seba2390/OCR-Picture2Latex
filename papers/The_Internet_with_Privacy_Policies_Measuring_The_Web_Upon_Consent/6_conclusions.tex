\section{Ethical considerations}
\label{sec:ethics}

During our measurements, we took care to avoid harming the crawled webpages. We contacted each website $5$ times in a span of two weeks and accessed a limited number of internal webpages each time. Considering that the target of our analysis were some of the most popular websites in Western countries, our belief is not to have caused an overload on the servers or any undesirable side effect. Moreover, since we did not interact with Third-Parties after accepting the privacy policies -- including displayed ads -- we consider not to have significantly altered the economic ecosystem of the crawled websites. We only used the standard HTTP and HTTPS ports for our measurements, carefully avoiding any type of port scanning procedures, and we used large timers to avoid creating any kind of congestion.

\section{Limitations}
\label{sec:limits}

Our work presents a few limitations, some of which could be addressed in future work.

First, \TOOL is designed to accept the privacy policy in the Consent Banner. It could be interesting to extend \TOOL to consider different keywords to choose the different options (e.g., to Opt-Out) on the Consent Banner and verify if websites correctly implement the end-user choice.

Second, the keyword list is manually compiled and static. We leave for future work the design of an automatic mechanism to enlarge and maintain the list. For instance, one can envision a community effort to enrich the list. It would also be interesting to consider some Natural Language Processing-based approaches to compile the keyword list automatically.

Third, currently, \TOOL uses a global list of keywords, regardless of the website's language. Although unlikely, a keyword may have a different meaning in another language, leading to false positives. A simple solution would be to add support for country- and language-specific lists of keywords.

Considering the results on the web tracking pervasiveness, we here focused on those based on tracking cookies, and we ignore advanced techniques for web tracking such as CNAME cloaking~\cite{dao2021}, a technique to embed Trackers as first-party domains, or device fingerprinting~\cite{rizzo2021unveiling}. Our results are thus an underestimation of the extensiveness. This problem is general and not specific for \TOOL.

Moreover, in our experiments, we set the browser language according to the country of each visited website. However, websites may customize their behaviour depending on the users' language, as some are already doing based on the user's location. \TOOL already allows configuring the content of \texttt{Accept-Language} header, making it possible to study this aspect in detail.

\section{Conclusions}
\label{sec:conclu}

In this paper, we demonstrated how the recent regulations had changed the Web landscape, challenging its automatic measurements through traditional Web crawlers. Websites now massively deploy Consent Banners to obtain visitors' consent for using tracking technologies and collecting personal data. As a result, webpages appear very different once users provide their consent. This has vast implications when measuring Web tracking, webpage characteristics, website performance, and any measurement based on Web crawling.

In this paper, we engineered \TOOL, a tool that automatically crawls websites accepting the privacy policy when a Consent Banner is found. We run it on many websites popular in Europe and worldwide. Our results highlighted how the observed picture of the Web varies when measured upon accepting privacy policies: Web Trackers and Third-Parties suddenly become more pervasive, websites more complex, and slower to load.

We release \TOOL as an open-source project, along with the dataset used throughout the paper. We based it on a set of keywords and, thus, has margins for improvement. We foster its use by the research community to contribute to it and extend our results. We also hope \TOOL will be included as part of the public projects that provide periodic Web measurements. Our goal is to keep developing \TOOL to enrich the keyword list, implement additional functionalities, adding the possibility to deny the privacy policies, a much more complex task. For this, we envision the design of more sophisticated approaches to manage Consent Banners, likely based on recent advances in Natural Language Processing and Machine Learning.
