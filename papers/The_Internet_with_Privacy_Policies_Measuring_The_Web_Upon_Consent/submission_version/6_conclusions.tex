\section{Ethical considerations}
\label{sec:ethics}

During our measurements, we took care to avoid harming the crawled webpages. We contacted each website $5$ times in a span of two weeks and accessed a limited number of internal webpages each time. Considering that the target of our analysis were some of the most popular websites of Western countries, our belief is not to have caused an overload on the servers or any undesirable side effect. Moreover, since we did not interact with Third-Parties after accepting the privacy policies -- included displayed ads -- we consider not to have significantly altered the economic ecosystem of the crawled websites. We only used the standard HTTP and HTTPS ports for our measurements, carefully avoiding any type of port scanning procedures, and we used large timers to avoid creating any kind of congestion.



\section{Conclusions}
\label{sec:conclu}

In this paper, we have demonstrated how the recent regulations have changed the Web scenario, challenging its automatic measurements through traditional Web crawlers. Websites now massively deploy Privacy Banners to obtain visitors' consent for using tracking technologies and collecting personal data. As a result, webpages appear very different once users provide their consent. This has vast implications on the understanding of Web tracking, on webpage characteristics, on performance measurement, and any other measurement based on Web crawling.

In this paper, we engineered \TOOL, a tool that automatically crawls websites accepting the privacy policy when a Privacy Banner is found. We run it on a large set of websites popular in Europe and worldwide. Our results highlighted how the picture of the Web varies when measured upon accepting privacy policies: Web Trackers and Third-Parties suddenly become more pervasive, websites more complex and slower to load.

We release \TOOL as an open-source project. We based it on a set of keywords and, thus, has margins for improvement. We foster its use by the research community to contribute to it and extend our results. We also hope \TOOL will be included as part of the public projects that provide periodic Web measurements. Our goal is to keep developing \TOOL to enrich the keyword list, implement additional functionalities, adding the possibility to deny the privacy policies, a much harder task. For this, we envision the design of more sophisticated approaches to manage Privacy Banners, likely based on recent advances in Natural Language Processing and Machine Learning.
