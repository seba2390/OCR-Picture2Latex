\section{Impact on Tracking}
\label{sec:tracking}

In this section, we characterize how the Web tracking ecosystem changes if observed with or without accepting the privacy policies. We break down results by Third-Party/Tracker, by country and website category. For this, we focus on the list of 100-top popular websites per country and category. Out of the $12\,277$ websites, \TOOL accepts the privacy policies on $57.3\%$ of them. The percentage is not uniform across countries and it is generally higher on European ($59-65\%$) and lower for US ($35\%$) websites - despite most keywords are in English. Differences are more pronounced across categories; \TOOL accepts privacy policies on $87\%$ of News websites, while only on $20\%$ of Adult portals. For the sake of completeness, the per-country and per-category acceptance rate is reported on the top-$x$ labels of Figure~\ref{fig:ca_country_one} and~\ref{fig:ca_category_one}, respectively. These figures are in line with the acceptance rates seen in Figures~\ref{fig:validation} and~\ref{fig:ca_vs_com}. Some manual random checking confirms that \TOOL does not accept the privacy policy for those websites that do not present any Privacy Banner or where the headless browser fails to visit. 

\subsection{Third-Party and Tracker Pervasiveness}

\begin{figure}
    \centering
    \includegraphics[width=0.6\columnwidth]{figures/cookieaccept_top_tp.pdf}
    \caption{Pervasiveness of the top-15 Third-Parties.}
    \label{fig:ca_prevasiveness_top}
\end{figure}

\begin{figure}
    \centering
    \includegraphics[width=0.5\columnwidth]{figures/cookieaccept_tracker_pervasiveness_log.pdf}
    \caption{Pervasiveness of the 342 identified Trackers.}
    \label{fig:ca_prevasiveness_all}

\end{figure}

We first study the pervasiveness of Third-Parties and Trackers and check how it varies when we measure it in a \BEFORE or \AFTER. We here focus on the $10\,542$ websites that are popular in the European countries according to the Similarweb ranks. Indeed, we aim at quantifying the impact of privacy policy acceptance on European websites. As such, we exclude those websites exclusively popular in the US.

We first detail the top-15 most pervasive Third-Parties in Figure~\ref{fig:ca_prevasiveness_top}. The GDPR mandates to obtain informed consent before starting to collect any personal data. As such, Third-Parties may be seen as possibly offending services if activated before accepting the privacy policy.\footnote{Here, we do not enter into the debate of what can be considered a Tracker.} With little surprise, the most pervasive Third-Party is \texttt{google-analytics.com}. It grows from $61\%$ to $74\%$ in popularity on the \AFTER. The growth is also sizeable for other Google services such as \texttt{googleadservices.com} and \texttt{googlesyndication.com}. Conversely, domains belonging to Content Delivery Networks, such as \texttt{cloudflare.com} and \texttt{cloudflare.net} do not increase their pervasiveness on the \AFTER, likely being not included in the mechanisms of Privacy Banners. Interestingly, only 3 out of the top-15 Third-Parties are Trackers -- i.e., present in our tracker list and setting a persistent cookie. \texttt{doubleclick.net} and \texttt{facebook.com} are the most popular ones, with pervasiveness growing from $41\%$ to $58\%$ and from $24\%$ to $39\%$ on the \AFTER, respectively. They are present in more than twice the number of websites than their first competitor (\texttt{quantserve.com}).

Focusing now on Trackers only, Figure~\ref{fig:ca_prevasiveness_all} shows their pervasiveness in our dataset. We count $342$ of them. Notice that the figure has log-log axes to better show the large variability of Tracker popularity. The red curve shows the pervasiveness on the \BEFORE, which is what a naive crawler would report. The blue curve shows how the figure changes on the \AFTER. The increase in pervasiveness is general and includes both popular and infrequent Trackers, reaching in a few cases one order of magnitude. On the \AFTER, the number of Trackers that are present on $1\%$ or more of websites grows from $40$ to $90$. Interestingly, if we sort the \BEFORE and \AFTER Trackers by their pervasiveness, the rank remains almost unchanged. The Spearman's rank correlation is $0.90$, indicating that the Tracker popularity order is approximately the same before and after the privacy policy acceptance. The difference is that their presence increases.

As it emerges from Figure~\ref{fig:ca_prevasiveness_all}, many Trackers are widespread even on the \BEFORE. This hints at a possibly wrong implementation of the GPDR regulation, which mandates to acquire first the visitor's explicit consent before activating any tracking mechanisms. To be precise, the presence of Trackers on the \BEFORE does not necessarily entail a violation of the law. A manual analysis displays that some Trackers install test cookies during the \BEFORE using a form similar to \texttt{test\_cookie = CheckForPermission}. These cookies are just a check for the possibility of installing profiling cookies upon the user's acceptance. It is thus possible that the \BEFORE pervasiveness of some Trackers includes cases in which only test cookies are actually used. Here we limit to observe that often Trackers set some (potentially) profiling cookies even on the \BEFORE.

In conclusion, these results show how different the picture is when collecting measurements with or without accepting the privacy policies. \TOOL enables the collection and analysis of what most users would be exposed to, thanks to its ability to handle the Privacy Banners and accept website privacy policies.



\subsection{Breakdown on Websites}

We now detail the impact of accepting privacy policies on the number of Trackers found in each website, breaking down our results by country and website category.

\subsubsection{Analysis by country}


\begin{figure*}
    \centering
    \includegraphics[width=0.185\textwidth]{figures/cookieaccept_overall_tracker_nb_fr.pdf}
    \includegraphics[width=0.15\textwidth]{figures/cookieaccept_overall_tracker_nb_de.pdf}
    \includegraphics[width=0.15\textwidth]{figures/cookieaccept_overall_tracker_nb_it.pdf}
    \includegraphics[width=0.15\textwidth]{figures/cookieaccept_overall_tracker_nb_es.pdf}
    \includegraphics[width=0.15\textwidth]{figures/cookieaccept_overall_tracker_nb_uk.pdf}
    \includegraphics[width=0.15\textwidth]{figures/cookieaccept_overall_tracker_nb_us.pdf}
	\caption{Trackers per website seen on the landing page. Websites are sorted by Tracker number on the \BEFORE.}
	\label{fig:ca_countries}
\end{figure*}

\begin{figure}
    \centering
    \begin{subfigure}[t]{0.495\columnwidth}
        \includegraphics[width=\columnwidth]{figures/cookieaccept_websites_with_trackers.pdf}
        \caption{Percentage of websites embedding Trackers.}
        \label{fig:ca_country_one}
    \end{subfigure}
    \begin{subfigure}[t]{0.495\columnwidth}
        \includegraphics[width=\columnwidth]{figures/cookieaccept_trackers_per_website.pdf}
        \caption{Average number of Trackers per website.}
        \label{fig:ca_country_avg}
    \end{subfigure}
	\caption{Trackers penetration and number 
	on websites during different phases of a browsing sessions.}
	\label{fig:ca_country}
\end{figure}


We first check if the number of Trackers found during the \BEFORE  differs in the \AFTER. Figure~\ref{fig:ca_countries} shows websites sorted in descending order by the number of contacted Trackers as measured in the \BEFORE (red curve). Focus now on the blue points. They report the number of Trackers in the \AFTER for the same website. The number tends to grow on the \AFTER, underlying again the need for tools such as \TOOL to accept the privacy policy and measure the footprint of Web tracking correctly. Some websites present a sizeable increase, with figures that grow by 50-70 Trackers. Curiously, some websites that already include Trackers in the \BEFORE include more Trackers in the \AFTER. This possibly hints at a wrong implementation of the Privacy Banner, which fails to hinder the presence of possibly offending Trackers. The increase is less remarkable for US-popular websites -- again, mainly due to the less widespread presence of Cookie Banners.

To better quantify Tracker presence, we show the fraction of websites containing at least one Tracker in Figure~\ref{fig:ca_country_one}. About $50\%$ of websites popular in European countries already include some Trackers on \BEFORE. This happens more frequently in the UK ($63\%$) and more occasionally in Germany ($44\%$). Again, notice that a website embedding a Tracker on the \BEFORE does not necessarily represent a violation of the GDPR, even if this can often be the case~\cite{trevisan20194}. Interestingly, in the US this figure is higher than in European countries. Recalling that in the US the probability of encountering a Privacy Banner is lower, this hints at a positive effect of the GDPR on popular European websites. The percentage of websites containing Trackers in the \AFTER grows for all European countries from a $+11\%$ increase in the UK to $+20\%$ for Germany. This increase is moderate ($+5\%$) in the US, given the lower fraction of those websites having a Privacy Banner. We complete this analysis by reporting how this fraction increases when performing \INTERNAL as recommended in~\cite{aqeel2020on}. We perform 5 \INTERNAL per website. Our results confirm this, with the chance to observe at least one Tracker that further grows by $5\%$-$10\%$ in \INTERNAL when compared to the \AFTER.

We next investigate the quantity of Trackers contacted while visiting websites in Figure~\ref{fig:ca_country_avg}, which shows the average number of Trackers contacted on the websites, separately by country. For all countries, the average amount of Trackers more than doubles on \AFTER, and performing \INTERNAL further increases this figure. In Italy, for instance, this figure grows by a factor of $4$ when comparing \BEFORE and \INTERNAL. As previously noted, the behavior of US-popular websites differs from the European: before acceptance, the number of Trackers is already higher than in popular European websites, while it is comparable after.\footnote{Recall that these measurements are taken from a European country.} This hints that popular websites in the United States may not have to deal with the European legislation, thus being less receptive to GDPR indications. On the opposite side, German-popular websites appear to be the most observant of the regulations, installing Trackers only upon accepting the privacy policies. Afterward, they reach levels comparable to the other countries. In summary, European websites use the same quantity of Trackers as US ones, although they are often contacted only after accepting the privacy policy.

We finally observe that the probabilistic nature of Web tracking and bidding mechanisms result in a different number of Trackers contacted at each visit. To obtain the most reliable measurements, we test each website $5$ times, each time visiting $5$ internal pages. We notice that measuring the fraction of websites containing at least one Tracker (as in Figure~\ref{fig:ca_country_one}) is moderately impacted by the number of tests. Indeed, when considering a single \AFTER per website, overall, we find $69.1\%$ of them containing one (or more) Trackers, which increases only to $70.0\%$ considering all $5$ tests. Similarly, the average number of Trackers (as in Figure~\ref{fig:ca_country_avg}), increases from $6.5$ to $7.8$.

\subsubsection{Analysis by category}

\begin{figure*}
    \centering
    \begin{subfigure}[t]{\textwidth}
        \includegraphics[width=\textwidth]{figures/cookieaccept_websites_with_trackers_category.pdf}
        \caption{Percentage of websites embedding Trackers.}
        \label{fig:ca_category_one}
    \end{subfigure}
    \begin{subfigure}[t]{\textwidth}
        \includegraphics[width=\textwidth]{figures/cookieaccept_trackers_per_website_category.pdf}
        \caption{Average number of Trackers per website.}
        \label{fig:ca_category_avg}
    \end{subfigure}
	\caption{Trackers penetration and number  on websites during different phases of a browsing session, separately by category.}
	\label{fig:ca_category}
\end{figure*}

We now break down the picture by category, showing the results in Figure~\ref{fig:ca_category}. As we previously pointed, we explicitly target websites from $25$ categories, each holding the top-$100$ websites for each of the considered countries.\footnote{We find a handful of websites belonging to more than one category.}

Figure~\ref{fig:ca_category_one} reports the percentage of websites of a given category that contain at least one Tracker. We sort categories from the highest to the lowest percentage of websites with Trackers in \BEFORE. For completeness, the top $x$-axis details the fraction of websites in such category where \TOOL accepts privacy policies. As before, there is a significant difference in the \BEFORE and \AFTER. An exception is the \textit{Adult} category, where the increase is marginal. This is likely due to the low number of websites with Privacy Banners ($20\%$) and confirms the peculiarity of the tracking ecosystem on Adult websites~\cite{vallina2019tales}. As observed in Figure~\ref{fig:ca_country_one}, considering \INTERNAL increases again the chance of encountering at least one Tracker.

Figure~\ref{fig:ca_category_avg} shows the average number of Trackers in websites, with categories sorted from the one with average highest to the one with average lowest number of Trackers in the \BEFORE. In the \AFTER and \INTERNAL, there is a large increase in the number of Trackers, confirming that most Trackers appear only after the user accepts the privacy policies and when visiting internal pages. Here, differences across categories are pronounced. Categories that heavily depend on advertisement-related incomes (such as \textit{News and Media}, \textit{Sports}, \textit{Games}, \textit{Arts and Entertainment}) tend to rely on a large number of Trackers to support more effective behavioral advertisements. This is noticeable already on the \BEFORE. For example, access to a \textit{News} website leads to contact $5.7$ Trackers on average. Here, \TOOL successfully accepts the privacy policies in $87\%$ of cases. Indeed, being \textit{News} websites very popular, they tend to correctly implement the privacy regulations, showing a well-configured Privacy Banner. Upon acceptance, suddenly, the number of Trackers becomes almost 6 times higher ($30.9$ for \textit{News}) and 9 times higher in \INTERNAL ($47.7$). Similar considerations hold for many other categories, e.g., for \emph{Sport}, \emph{Food and Drink} and \emph{Arts and Entertainment} the average number of Trackers more than triples.

These results well highlight the need for correctly handling the Privacy Banners to observe the extensiveness of Trackers. Without \TOOL, one would radically underestimate the footprint of the tracking and ads ecosystems on the Web. In a nutshell, thanks to \TOOL, we obtain the fundamentally different figure in the \AFTER and \INTERNAL.

The case of \textit{Adult} websites is worth a specific comment. \TOOL finds the Privacy Banner on only $20\%$ of them, and a manual check confirms that the large majority of them do not offer any Privacy Banner. In general, tracking is also scarce upon acceptance, as previously found by authors of~\cite{vallina2019tales}. They suggest that the specialized pornographic advertisement ecosystem may cause this behavior: usually, trackers and advertisers related to pornographic websites do not operate outside of them -- often evading tracker listing efforts.



\subsection{Visiting from Outside Europe}

\begin{figure}[t]
    \centering
    \includegraphics[width=0.5\columnwidth]{figures/cookieaccept_compare_countries.pdf}
    \caption{Trackers per websites when crawling from different countries.}
    \label{fig:ca_us}
\end{figure}

To complete the analysis, we run additional measurement campaigns using crawling servers in the Amazon AWS data centers located in the US (Ohio and California), Japan and Brazil. We target the same websites as before. We aim to check if websites behave differently based on the location of the visitors. Figure~\ref{fig:ca_us} summarize our findings. First, we notice how \TOOL accepted privacy policies on around $10\%$ fewer websites when run from outside Europe, as reported on top $x$-labels. Investigating further, we find $\approx 1\,150 - 1\,200$ websites for which \TOOL can accept the Privacy Banner when visited from Europe, but it fails when visited from not-EU countries. Checking the screenshot taken by \TOOL during the visit on a random subset of these websites, we confirm that no Privacy Banner is present. Thus, we conclude that some websites are starting to customize their Privacy Banners based on visitors' properties, such as their location.

This impacts the percentage of websites that embed Trackers on the \BEFORE. It grows from $49.7\%$ to $54-60\%$ when visiting from outside Europe. On the \AFTER, these differences smooth out, revealing how \TOOL helps obtain user-centric measurements regardless of the presence or absence of Privacy Banners on websites. As a final note, we do not observe any significant difference visiting the websites from Ohio or California, despite the CCPA.\footnote{This figure may require further investigation since we are measuring from Amazon AWS servers whose location may not be correctly handled by the CMPs.}



\section{Impact on Complexity and Performance on Top-100k Websites}
\label{sec:performance}

\begin{figure}[!t]
    \centering
    \begin{subfigure}[t]{0.495\columnwidth}
        \includegraphics[width=1.0\columnwidth]{figures/cookieaccept_tranco_rank_eu.pdf}
        \caption{Acceptance rate.}
        \label{fig:tranco_rank}
    \end{subfigure}
    \begin{subfigure}[t]{0.495\columnwidth}
        \includegraphics[width=1.0\columnwidth]{figures/cookieaccept_tranco_rank_eu_tp_nb.pdf}
        \caption{Average number of Third-Parties per website.}
        \label{fig:tranco_tp}
    \end{subfigure}    
     \caption{Acceptance rate and average Third-Parties per website over the top-100 k websites in Tranco list, computed every $5\,000$ websites in the rank.}
    \label{fig:tranco}   
\end{figure}

In this section, we measure the impact of accepting privacy policies on the webpage characteristics and loading performances. Trackers and Third-Party objects that the browser has to load and display upon consent could impact the amount of data to download and the rendering performance. We focus on the overall picture, mimicking what a general user would observe when browsing the internet. We thus do not restrict dataset on a per-country or per-category subset, and use the crawl on the top-$100\,000$ websites according to the Tranco worldwide list. 

For each website, we visit only the landing page, doing a \emph{Warm-up} visit to fill the browser cache, followed by a \BEFORE and \AFTER. We compare results on the latter two visits, considering only those websites for which \TOOL successfully accepted the privacy policy, which happens on $23$\% of websites. This is in line with the previous findings, as the Tranco list is a worldwide rank and includes (i) European websites in a language different from those for which we built the keyword list and (ii) websites based in non-European countries for which regulations do not apply. To give more insights, we detail the acceptance rate on the Tranco list in Figure~\ref{fig:tranco_rank}, computed over blocks of $5\,000$ websites sorted by their rank, totaling $100\,000$ on the $x$-axis. The solid red line reports the acceptance rate for websites popular in the 5 European countries we target. Websites belong to this set if (i) they appear in the Similarweb ranks for the 5 countries or (ii) the Top-Level Domain belongs to the 5 countries.\footnote{The Tranco list does not provide a per-country rank.} Out of these $6\,9178$ websites, \TOOL accepts the privacy policy on $3\,861$ ($55.8$\%), which is close to what we have obtained with the Similarweb ranks ($54.7\%$). This percentage does not change with website popularity. Conversely, for the remaining websites (blue dashed line), the acceptance rate is $32$\% for the top-ranked and then it settles around $20$\%, hinting that some globally popular websites tend to implement a Privacy Banner even if they are based outside Europe (as an effect of having in their audience also customers from Europe).

The high success rate for the 5 European countries reflects in a large increase of the number of Third-Parties from the \BEFORE to the \AFTER, as shown in Figure~\ref{fig:tranco_tp}. The solid red line highlights that these websites already include, on average, $11.1$ Third-Parties in the \BEFORE. In the \AFTER, the average grows to $17.3$. Differently, the increase for the non-EU websites is limited -- see the area between the blue solid and dashed lines. In the \BEFORE, Third-Parties are larger than for the 5 European countries if we compare the solid blue and red lines. This is likely due to the larger presence of non-EU websites, which do not have to implement a Privacy Banner. In the \AFTER (dashed blue line), the increase is moderate, not reaching the values of the 5 European countries (red dashed line), potentially because \TOOL misses many \emph{Accept-button} in non-supported languages. For the sake of completeness, in the Appendix, we report the same picture as in Figure~\ref{fig:tranco_tp} showing the number of Trackers instead of Third-Parties, providing similar insights.

\subsection{Impact on Page Objects and Size}

We focus on the webpage complexity in terms of  bytes and objects to download. To compare the results, we compute the ratio $R$ between the measurement on the \BEFORE and \AFTER, i.e., $R = x_{\textit{After}}/x_{\textit{Before}}$, where $x$ is the metric of interest. We show the results in Figure~\ref{fig:ca_perf_size}, separately for total downloaded bytes (solid red line) and objects (blue dashed line). The figure reports the distribution of the ratio over all websites. As expected, accepting the privacy policy increases the webpage size ($R>1$) by a sizeable factor. For instance, about $9$\% of websites download more than twice the objects, and about $5$\% of websites sees an increase of 3 times or more.

\begin{figure}[!t]
    \centering
    \begin{subfigure}[t]{0.495\columnwidth}
        \includegraphics[width=\textwidth]{figures/cookieaccept_performance_tranco.pdf}
        \caption{Page size (in number of bytes and objects) ratio.}
        \label{fig:ca_perf_size}
    \end{subfigure}
    \begin{subfigure}[t]{0.495\columnwidth}
        \includegraphics[width=\textwidth]{figures/cookieaccept_tp_nb_tranco.pdf}
        \caption{Number of Third Parties. Notice the log scales.}
        \label{fig:ca_perf_tp}
    \end{subfigure}
	\caption{Webpage characteristic before and upon consent to privacy policies (Tranco list).}
	\label{fig:ca_perf}
\end{figure}

Interestingly, we also observe some websites that are lighter in the \AFTER than in the \BEFORE. Investigating further, these cases are mostly due to the lack of additional content upon acceptance coupled with the saving of not loading the CMP objects on the \AFTER. This happens commonly on those websites that either add a Privacy Banner despite not using tracking mechanisms or those websites that load and contact Trackers and Third-Parties even before the user has accepted the privacy policies. While the former might be seen as an excess of caution, the latter cases are likely violating the privacy regulations.

To better characterize the differences, we quantify the number of Third-Parties seen in the \BEFORE and \AFTER. We show the Complementary Cumulative Distribution Function (CCDF) in Figure~\ref{fig:ca_perf_tp}. On median, websites rely on $12$ Third-Parties on the \BEFORE (solid red line). This figure grows to $17$ after (blue dashed line) on the \AFTER. The CCDF highlights the tail of the distribution where we observe those websites that rely on a very large number of Third-Parties: the percentage of websites with more than $50$ grows from $1.8\%$ to $9.2\%$, with $3.0\%$ including more than 75 Third-Parties upon acceptance. This growth in the number of Third-Parties is mostly due to an increase of Trackers and objects related to advertisements that gets loaded after accepting the privacy policy. In Appendix, we include the picture as above, plotting the number of Trackers instead of Third-Parties, leading to similar conclusions.

% \begin{figure}[!t]
%     \centering
%     \includegraphics[width=0.6\columnwidth]{figures/cookieaccept_onload_by_tracker_tranco.pdf}
%     \caption{OnLoad time of websites versus the increase of Third-Party number upon acceptance. The cardinality of each category is reported on the top axis.}
%     \label{fig:ca_onload}
% \end{figure}

\begin{figure}[!t]
    \centering
    \begin{subfigure}[t]{0.495\columnwidth}
        \includegraphics[width=\textwidth]{figures/cookieaccept_onload_by_tracker_tranco.pdf}
        \caption{Warm Browser Cache.}
        \label{fig:ca_onload_warm}
    \end{subfigure}
    \begin{subfigure}[t]{0.495\columnwidth}
        \includegraphics[width=\textwidth]{figures/cookieaccept_onload_by_tracker_tranco_cold_cache.pdf}
        \caption{Cold Browser Cache.}
        \label{fig:ca_onload_cold}
    \end{subfigure}
    \caption{OnLoad time of websites versus the increase of Third-Party number upon acceptance (Tranco list). The cardinality of each category is reported on the top axis of the left-most figure.}
     \label{fig:ca_onload}
\end{figure}

\subsection{Impact on Page Load Time}

The Third-Party domains appearing after acceptance are generally devoted to advertisements, analytics and Web tracking. % -- see in Figure~\ref{fig:ca_prevasiveness_top} the most pervasive. 
Contacting them has direct implications on the page load time and, indirectly, on the users' QoE~\cite{da2018narrowing}. We thus expect the growth of Third Parties and the increase in the number of objects to download to cause degradation on the page load time because the browser has to resolve the server name via DNS and contact more servers. For instance, this ultimately limits the advantages offered by new protocols like the stream multiplexing and the header compression offered by HTTP/2.

To gauge this, we dissect the webpage load time in Figure~\ref{fig:ca_onload}, comparing separately visits with a warm cache (Figure~\ref{fig:ca_onload_warm}) and with a cold cache (Figure~\ref{fig:ca_onload_cold}). We report the distributions of the \textit{onLoad} time for websites with similar number of additional Third-Parties that are loaded in the \AFTER. We use boxplots, where the boxes span from the first to the third quartile and whiskers from the $10^{th}$ to $90^{th}$ percentile. The central stroke represents the median. The number of websites in each set is detailed on the top the respective boxplot. As expected, the more Third-Parties are loaded upon acceptance, the larger the time needed to load the webpage and the larger its variability. Especially for the websites that add more than 10 Third-Parties, the distributions are remarkably different on the \BEFORE and \AFTER. Considering visits with cold browser cache (Figure~\ref{fig:ca_onload_warm}), those website with $20-50$ additional Third-Parties, the median \textit{onLoad} time passes from $0.91$ to $1.41$ seconds. The difference increases for the $632$ websites adding more than $50$ Third-Parties upon acceptance. Here, the median \textit{onLoad} time increases from $1.35$ to $3.38$ seconds, more than doubling.
Notice also the tail of $25\%$ of websites loading in more than $4.8$\,seconds, which happens in less than $2\%$ of cases during the \BEFORE. Similar considerations hold for visits with a cold browser cache (Figure~\ref{fig:ca_onload_cold}). In this case, \TOOL cleans the browser cache and the socket pool after each visit. As expected, with the clean cache, websites load generally more slowly -- compare values in Figures~\ref{fig:ca_onload_warm} and~\ref{fig:ca_onload_cold}. Those that do not add new Third-Parties tend to load slightly faster on the \AFTER, potentially due to the absence of the Privacy Banner or CMPs. Again, we observe that those adding several Third-Parties after acceptance have much higher \textit{onLoad} time on the \AFTER than on the \BEFORE:  The median \textit{onLoad} time increases from $1.8$ to $5.2$ seconds.

In summary, measuring the webpage load time of websites without considering the implications of accepting the privacy banners would result in a very biased measurement. Our experimental results highlight how \TOOL results fundamental to obtain a realistic picture of the Web measurements and testifies how actual users' experience cannot be measured without handling the Privacy Banners.

