\documentclass[acmsmall]{acmart}

\usepackage[hyphens]{url}
\usepackage{hyperref}

\usepackage[english]{babel}
\usepackage{blindtext}
\usepackage{xspace}
\usepackage{subcaption}
\settopmatter{printfolios=true}

%Comment commands
\newcommand{\nj}[1]{{\color{purple}{[nj: #1]}}}
\newcommand{\mt}[1]{{\color{red}{[mt: #1]}}}
\newcommand{\mm}[1]{{\color{blue}{[mm: #1]}}}
\newcommand{\lv}[1]{{\color{green}{[lv: #1]}}}
\newcommand{\TOOL}{{\emph{Priv-Accept}}\xspace}
\newcommand{\BEFORE}{{\emph{Before-Visit}}\xspace}
\newcommand{\AFTER}{{\emph{After-Visit}}\xspace}
\newcommand{\INTERNAL}{{\emph{Additional-Visits}}\xspace}

%Conference Info
\acmYear{2021}
\copyrightyear{2021}
\setcopyright{acmcopyright}
\acmJournal{TWEB}



\begin{document}
\title[The Internet with Privacy Policies]{The Internet with Privacy Policies: Measuring The Web Upon Consent}

%\author{Nikhil Jha, Martino Trevisan, Luca Vassio, Marco Mellia}
\author{Nikhil Jha}
\affiliation{
\institution{Politecnico di Torino}
\streetaddress{Corso Duca degli Abruzzi, 24}
\city{Torino}
\postcode{10129}
\country{Italy}}
\email{nikhil.jha@polito.it}

\author{Martino Trevisan}
\affiliation{
\institution{Politecnico di Torino}
\streetaddress{Corso Duca degli Abruzzi, 24}
\city{Torino}
\postcode{10129}
\country{Italy}}
\email{martino.trevisan@polito.it}

\author{Luca Vassio}
\affiliation{
\institution{Politecnico di Torino}
\streetaddress{Corso Duca degli Abruzzi, 24}
\city{Torino}
\postcode{10129}
\country{Italy}}
\email{luca.vassio@polito.it}

\author{Marco Mellia}
\affiliation{
\institution{Politecnico di Torino}
\streetaddress{Corso Duca degli Abruzzi, 24}
\city{Torino}
\postcode{10129}
\country{Italy}}
\email{marco.mellia@polito.it}

\begin{abstract}
To protect users' privacy, legislators have regulated the usage of tracking technologies, mandating the acquisition of users' consent before collecting data. Consequently, websites started showing more and more consent management modules -- i.e., Privacy Banners -- the visitors have to interact with to access the website content. They challenge the automatic collection of Web measurements, primarily to monitor the extensiveness of tracking technologies but also to measure Web performance in the wild. Privacy Banners, in fact, limit crawlers from observing the actual website content.

In this paper, we present a thorough measurement campaign focusing on popular websites in Europe and the US, visiting both landing and internal pages from different countries around the world. We engineer \TOOL, a Web crawler able to accept the privacy policies, as most users would do in practice. It lets us compare how webpages change before and after. Our results show that all measurements performed not dealing with the Privacy Banners offer a very biased and partial view of the Web. After accepting the privacy policies, web tracking is abundantly more pervasive, and webpages are larger and slower.
\end{abstract}

\begin{CCSXML}
<ccs2012>
   <concept>
       <concept_id>10003033.10003079.10011704</concept_id>
       <concept_desc>Networks~Network measurement</concept_desc>
       <concept_significance>500</concept_significance>
       </concept>
   <concept>
       <concept_id>10002978.10003029</concept_id>
       <concept_desc>Security and privacy~Human and societal aspects of security and privacy</concept_desc>
       <concept_significance>500</concept_significance>
       </concept>
 </ccs2012>
\end{CCSXML}

\ccsdesc[500]{Networks~Network measurement}
\ccsdesc[500]{Security and privacy~Human and societal aspects of security and privacy}

\ccsdesc[500]{Networks~Network measurement}

\keywords{Web Measurements, Crawling, Privacy Banner, GDPR}

\maketitle

\section{Introduction}

Scientific literature is most commonly available in the form of PDFs, which pose challenges for accessibility \citep{NielsenPDFStillUnfit, Bigham2016AnUT}. When researchers, students, and other individuals who are blind or low vision (BLV) interact with scientific PDFs through screen readers, the availability of document structure tags, labeled reading order, labeled headers, and image alt-text are necessary to facilitate these interactions. However, these features must be painstakingly added by authors using proprietary software tools, and as a result, are often missing from papers. Low vision or dyslexic readers who interact with PDFs through screen magnification or text-to-speech may also find the complexity of certain academic paper PDF formats challenging, e.g., non-linear layout can interrupt the flow of text in a magnifying tool. Inaccessible paper PDFs can lead to high cognitive overload, frustration, and abandonment of reading for BLV readers. 

Unfortunately, we find that the majority of scientific PDFs lack basic accessibility features. We estimate based on a sample of \numpdfs PDFs from multiple fields of study that only around \percaccessible of paper PDFs released in the last decade satisfy all of the aforementioned accessibility requirements. 
Accessibility challenges for academic PDFs are largely due to three factors: (1) the complexity of the PDF file format, which make it less amenable to certain accessibility features, (2) the dearth of tools, especially non-proprietary tools, for creating accessible PDFs, and (3) the dependency on volunteerism from the community with minimal support or enforcement \citep{Bigham2016AnUT}. The intent of the PDF file format is to support faithful visual representation of a document for printing, a goal that is inherently divergent from that of document representation for the purposes of accessibility. Though some professional organizations like the Association for Computing Machinery (ACM) have encouraged PDF accessibility through standards and writing guidelines,\footnote{\href{https://www.acm.org/publications/authors/submissions}{https://www.acm.org/publications/authors/submissions}} uptake among academic publishers and disciplines more broadly has been limited. 

While policy changes help, the fact remains that most academic PDFs produced today, and historically, are inaccessible, yet remain as the dominant way to read those papers. A long-range solution will necessitate buy-in from multiple stakeholders---publishers, authors, readers, technologists, granting agencies, and the like. But in the interim, there are technological solutions that can be offered as a sort of ``band-aid'' to the problem. We use this paper to offer an in-depth qualitative and quantitative description of the problem as it stands, and to introduce one such technological solution: the \scially system that automatically extracts semantic information from paper PDFs and re-renders this content in the form of an accessible HTML document. Though the process is imperfect and can introduce errors, we demonstrate the ability of the rendered HTMLs to reduce cognitive load and facilitate in-paper navigation and interactions for BLV users. 

The goals and contributions of this paper are three-fold:

\begin{enumerate}
    \item We characterize the state of academic-paper PDF accessibility by estimating the degree of adherence to accessibility criteria for papers published in the last decade (2010--2019), and describe correlations between year, field of study, PDF typesetting software, and PDF accessibility.
    \item We propose an automated approach for extracting the content of academic PDFs and displaying this content in a more accessible HTML document format. We build a prototype that re-renders 12 million PDFs in HTML, and describe the design decisions, features, and quality of the renders (assessed as faithfulness to the source PDF). We perform expert grading of the rendered HTML and report an error analysis. A demo of our system is available at \href{https://scia11y.org/}{scia11y.org}, which makes available 1.5M HTML renders of open access PDFs.
    \item We conduct an exploratory user study with \numusers BLV scholars to better understand the challenges they experience when reading academic papers and how our proposed tool might augment their current workflow. During the study, we ask users to interact with the prototype and offer feedback for its improvement. We perform open coding of interviews to identify existing reading challenges, coping mechanisms, as well as positive and negative responses to prototype features. We summarize the findings of this user study into a set of design recommendations.
\end{enumerate}

Our analysis reveals that PDF accessibility adherence is low across all fields of study. Of the five accessibility criteria we assess, only \percaccessible of the PDFs we assess demonstrate full compliance. Though compliance for several criteria seems to be increasing over time, author awareness and contribution to accessibility remains low, as Alt-text has the lowest compliance of the five criteria at between 5--10\% (Alt-text is the only criterion of the five that \textit{requires} author intervention in all cases using current tools). We also find that typesetting software is strongly associated with accessibility compliance, with LaTeX and publishing software like Arbortext APP producing low compliance PDFs, while Microsoft Word is generally associated with higher compliance.


\begin{figure}[t!]
    \centering
    \includegraphics[width=\textwidth]{figures/pipeline.png}
    \caption{A schematic for creating the \scially HTML render from a paper PDF. Starting with the raw two-column PDF on the left, S2ORC \citep{lo-wang-2020-s2orc} is used to extract title, authors, abstract, section headers, body text, and references. S2ORC also identifies links between inline citations and references to figures and table objects. DeepFigures \citep{Siegel2018ExtractingSF} is used to extract figures and tables, along with their captions. The output of these two models are merged with metadata from the Semantic Scholar API. Heuristics are used to construct a table of contents, to insert figures and tables in the appropriate places in the text, and to repair broken URLs. We add HTML headers as illustrated (header tags for sections, paragraph tags for body text, and figure tags for figures and tables); highlighted components (table of contents and links in references) are not in the PDF and novel navigational features that we introduce to the HTML render. An example HTML render of parts of a paper document is show to the right (actual render is single column, which is split here for presentation).}
    \label{fig:pipeline}
    \Description{A schematic diagram showing the components of the SciA11y pipeline. An image of a paper PDF is on the left. Red boxes on the PDF image highlight the text components from the paper, with an arrow pointing to a box that says "S2ORC extracts: title, authors, abstract, section headers, body paragraphs, and references." A blue box on the PDF image highlights a figure, with an arrow pointing to a box that says "DeepFigures extracts: figures, figure captions, tables, and table titles/captions." A box below "S2ORC extracts" and "DeepFigures extracts" says "Additional content: metadata from Semantic Scholar API, table of contents, figures and tables inserted at first mention, and links between references and text." Arrows from all three boxes point into a larger box that describes the SciA11y prototype, where HTML tags are inserted around various blocks of text extracted from the PDF. On the right of all this is a screen capture of an example HTML render, showing how the semantic content from the PDF is represented as a single-column HTML page for easy reading.}
\end{figure}

To offset the reading challenges of inaccessible papers for BLV researchers, we propose and test the \scially system for rendering academic PDFs into accessible HTML documents. As shown in Figure~\ref{fig:pipeline}, our prototype integrates several machine learning text and vision models to extract the structure and semantic content of papers. The content is represented as an HTML document with headings and links for navigation, figures and tables, as well as other novel features to assist in document structure understanding. Our evaluation of the \scially system identifies common classes of extraction problems, and finds that though many papers exhibit some extraction errors, the majority (55\%) have no major problems that impact readability, and another 32\% have only some problems that impact readability.

Through our user study, we identify numerous challenges faced by BLV users when reading paper PDFs, including some that affect the whole document or limit navigation, and many that affect the ability of the reader to understand text or various elements of a paper like math content or tables. Responses to \scially were positive; participants especially liked navigation features such as headings, the table of contents, and bidirectional links between inline citations and references. Of the extraction errors in \scially, missed or incorrectly extracted headings were the most problematic, as these impact the user's ability to navigate between sections and fully trust the system. All users reported being likely to use the system in the future. When asked how the system might be integrated into their workflow, one participant replied ``I think it would become the workflow.'' Another participant said, ``for unaccessible PDFs, this is life-changing.'' We condense these findings into a set of recommendations for designing and engineering accessible reading systems (Section~\ref{sec:designrecs}). Most importantly, documents should be structured to match a reader's mental model, objects should be properly tagged, and care should be taken to reduce the reader's cognitive load and increase trust in the system. Features that emulate the external memory that visual layout provides to sighted users can be especially beneficial.

This paper is organized as follows. Following a description of related work in Section \ref{sec:related_work}, we first provide a meta-scientific analysis of the current state of academic PDF accessibility in Section \ref{sec:sos}. In Section \ref{sec:pdf2html}, we document our pipeline for converting PDF to HTML and describe the \scially prototype for rendering papers. An evaluation of HTML render quality and faithfulness is provided in Section \ref{sec:evaluation}. Section \ref{sec:user_study} describes our user study and findings. 
We recognize that no PDF extraction system is perfect, and many open research challenges remain in improving these systems. However, based on our findings, we believe \scially can dramatically improve screen reader navigation of most papers compared to PDFs, and is well-positioned to assist BLV researchers with many of their most common reading use cases. Our hope is that a system such as \scially can improve BLV researcher access to the content of academic papers, and that these design recommendations can be leveraged by others to create better, more faithful, and ultimately more usable tools and systems for scholars in the BLV community.

\section{Background and related work}
\label{sec:history}

Content providers on the Web often monetize the content they offer by using advertisements. To increase their effectiveness, the so-called behavioral advertisement leverages users' interests to provide targeted ads. This is possible thanks to Web trackers, i.e., third-party services embedded in the webpages that gather users' browsing history. Trackers are nowadays largely present on websites and reach the majority of web users~\cite{metwalley2015online,pujol2015annoyed}. Trackers exploit cookies and advanced techniques to enable the collection of personal information~\cite{acar2014web,rizzo2021unveiling,papadogiannakis2021}.

%In parallel, many solutions exist for blocking trackers and ads, typically implemented via browser plugins (e.g., AdBlock Plus~\cite{adblock}, Ghostery~\cite{ghostery} or Disconnect~\cite{disconnect}). Their effectiveness has been testified in several studies~\cite{pujol2015annoyed, traverso2017benchmark, mazel2019comparison}, and none of them offers complete protection.

\begin{figure}[t]
    \centering
    \includegraphics[width=0.75\columnwidth]{figures/cookie-example-large.pdf}
    \caption{Example of Consent Banner on \texttt{dailymail.co.uk}. Only upon consent, trackers are contacted and ads displayed.}
    \label{fig:cookie_accept_example}
\end{figure}

\subsection{The Role of Legislators}

In this tangled picture, legislators started to regulate the ecosystem to avoid massive indiscriminate tracking that may threaten users' privacy. In 2013, the European Cookie Law~\cite{directive2009} entered into force, which mandates websites to ask for informed consent before using any profiling technology. Later, in May 2018, the General Data Protection Regulation (GDPR)~\cite{gpdr} entered into force in all European member states. It is an extensive regulation on privacy, aiming at protecting users' privacy by imposing strict rules when handling personal information. Unlike previous regulations, it sets severe fines and infringements that could result in a fine of up to €10 million, or 2\% of the firm's worldwide annual revenue, whichever amount is higher. Some websites have already been caught to present legal violations in their Consent Banner implementation~\cite{matte2020cookie} and a large fraction have been shown to use tracking technologies before user consent~\cite{trevisan20194, sanchez2019can}. In the US, the California Consumer Privacy Act (CCPA)~\cite{ccpa} enhances privacy rights and consumer protection for California residents by requiring businesses to give consumers notices about their privacy practices.

As a result, most of the websites now provide explicit Consent Banners~\cite{degeling2018we} and many adopt Consent Management toolsets~\cite{hills2020consent}, making the website content difficult to access until visitors accept the privacy policy. For example, Figure~\ref{fig:cookie_accept_example} shows the same news website homepage before and after accepting the privacy policy. Only upon pressing the ``Got it'' button, the website content is fully loaded and visible.
%In some cases, websites use more persistent and sophisticated forms of tracking in order to track users who denied their acceptance of cookies~\cite{papadogiannakis2021user}.



\begin{figure}[t]
    \centering
    \includegraphics[width=0.5\columnwidth]{figures/httparchive_websites_with_trackers_tld.pdf}
    \caption{Percentage of websites containing at least one tracker for five European Top-Level domains (from HTTPArchive). The black vertical line indicates the entry into force of the GDPR. Since then, the apparent pervasiveness of tracking decreased.}
    \label{fig:ha_websites_trackers}
\end{figure}



\subsection{The Effect of Consent Banners on Web Measurements}

Despite cases of misuse, the new regulations had a large impact on the web users and complicate the measurement of the tracking ecosystem. A simple Web crawler visiting the websites without accepting the privacy policies would offer a biased picture, with no tracker and no ad being loaded. Hu~\emph{et al.}~\cite{hu2019characterising} already found that the number of third-parties dropped by more than 10\% after GDPR when visiting websites automatically. Conversely, when using a dataset from 15 real users, they measure no significant reduction in long-term numbers of third-party cookies. Dabrowski~\emph{et al.}~\cite{dabrowski2019measuring} draw similar conclusions, finding an apparent decrease in the use of persistent cookies from 2016 to 2018. Sorensen~\emph{et al.}~\cite{sorensen2019before} testify a decreasing trend in the number of third parties during 2018. We quantify this phenomenon in Figure~\ref{fig:ha_websites_trackers}, using the HTTPArchive open dataset~\cite{httparchive}. The curators of this dataset maintain a list of top websites worldwide that they automatically visit using the Google Chrome browser from a US-based server to store a copy of each visited webpage. Using the tracker list detailed in Section~\ref{sec:metho}, we report the percentage of websites embedding one or more trackers for 5 European countries (simply using the Top-Level Domain to identify the country).\footnote{The Top-Level Domain can sometimes be an inaccurate proxy for a website's country. Here, our goal is only to provide a qualitative picture.} We restrict the analysis on those websites that exist for the whole six years-long periods ($9\,196$ website in total).

Figure~\ref{fig:ha_websites_trackers} could suggest that the introduction of the GDPR (the black vertical line in May 2018) results in an abrupt decrease in the number of tracker-embedding websites, a trend that continues up to the moment we write. However, as we will show, these measurements are an artifact due to the GDPR itself. Indeed, the Web crawler used by HTTPArchive can only capture the behavior of the websites as a ``first-time visitor'', before the user accepts any privacy policy. The crawler thus misses third-party trackers and ads.

Research papers that rely on crawling large portions of the Web for different reasons could be affected by the same bias in their measurements. For instance, this would challenge the automatic measurement of the Web ecosystem on privacy~\cite{acar2014web,falahrastegar2014rise,metwalley2015online,pujol2015annoyed,englehardt2016online,iordanou2018tracing,hu2019characterising,rizzo2021unveiling,vandrevu2019what,papadogiannakis2021,aqeel2020on} and counter-measurements~\cite{pujol2015annoyed, traverso2017benchmark, mazel2019comparison}. Moreover, this will also impact those works that rely on crawlers and headless browsers~\cite{avasarala2014selenium} to quantify the impact in the wild of new technologies like SPDY,  HTTP/2~\cite{wang2014speedy,de2015http,bocchi2016measuring,erman2015towards}, 4G/5G~\cite{alay2017experience,asrese2019measuring}, accelerating proxies~\cite{sivakumar2014parcel,wang2016speeding,ruamviboonsuk2017vroom}, or generic benchmark solutions~\cite{netravali2015mahimahi}. At last, even spiders and mirroring tools like Wayback Machine and HTTPArchive may be affected if the website allows the visitor to access its content only after accepting the privacy policy.

\subsection{Related Work and Tools}
\label{sec:related}
Vallina~\emph{et al.}~\cite{vallina2019tales} are the first to consider the impact of the Consent Banner presence. First, they instruct a custom OpenWPM crawler to identify specific Consent Banners, and then they manually verify the results. Unfortunately, they solely focus on the pornographic ecosystem, which they acknowledge to be rather different from the Web at large, and thus their work can hardly generalize.

Recently, authors of~\cite{aqeel2020on} demonstrated that it is fundamental to consider the complexity of the Web ecosystem and include internal pages in every measurement study. They find a number of recent works that neglect internal pages and, as such, might provide biased results. Yet, they ignore the complications due to Consent Banners. Here, we aim at providing an extensive and thorough study of their impact on the Web. Our goal is to enable the automatic study of webpage characteristics as visitors would experience, assuming that most of them accept the default privacy setting as offered by the Consent Banner. Indeed, it has been shown that most users tend to ignore privacy-related notices~\cite{vila2003we, grossklags2007empirical, coventry2016personality}. Considering GDPR Consent Banners, users tend to accept privacy policies when offered a default button via intrusive banners that nudge users ~\cite{CookieBenchmarkStudy,bauer2021you}, which is often the case~\cite{hausner2021dark} with websites presenting large pop-ups or wall-style banners that cover most of the webpage as seen in Figure~\ref{fig:cookie_accept_example}. 

For completeness, notice that cookies are among the simplest tracking mechanisms. Authors of~\cite{papadogiannakis2021} show how practices like cookie synchronization, cookie leaking, and other profiling techniques like canvas fingerprinting are common in today's Web. Similarly, authors of~\cite{jueckstock2021towards} show how the crawling context, in terms of vantage point and browser configuration, has a significant impact on the results. Our work is orthogonal to these to obtain automatic, realistic, reliable and user-centric measurements of the Web.

Focusing on automatic management of Consent Banners, some browser add-ons try to hide them by using a list of CSS selectors of known Consent Banners. The most popular add-ons of this kind are ``I don't care about cookies''~\cite{idontcare} and ``Remove Cookie Banners''~\cite{remove}. Unfortunately, hiding the Consent Banners has an unpredictable behavior, in some cases falling back to privacy policies acceptance, while, in other cases, triggering an opt-out choice. Other proposals, again in the form of browser add-ons, try to explicitly opt-in or opt-out to cookies. For example, ``Ninja Cookie''~\cite{ninja} approves only cookies strictly needed to proceed on the website. Conversely, Autoconsent~\cite{autoconsent} and Consent-O-Matic~\cite{consentomatic} use a set of predefined rules to either opt-in or opt-out to cookies, according to the user configuration. These two are the most similar solutions to \TOOL. However, they are based on a list of actions the browser automatically runs when finding a set of popular Consent Management Platforms (CMPs), limiting their effectiveness. In Section~\ref{sec:ca_vs_com}, we compare \TOOL with Consent-O-Matic -- the most mature tool -- showing that \TOOL offers a much higher coverage. Indeed, the diversity of the Web ecosystem, the presence of multiple languages and the fully customizable choice of Consent Banner buttons make the engineering of \TOOL not trivial.


\section{\TOOL design and testing}
\label{sec:metho}

We explicitly engineer \TOOL to fully automate the visit to websites and collect statistics. The key element of \TOOL is its ability to identify the presence of a Privacy Banner and automatically accept privacy policies. We aim at a practical and effective approach to accept privacy policies through the offered button. As previously said, most users will indeed be nudged in this direction, being the opt-out options often made cumbersome on purpose\cite{bauer2021you, hausner2021dark, CookieBenchmarkStudy}.

To illustrate \TOOL operation, consider again Figure~\ref{fig:cookie_accept_example}. A large Privacy Banner appears on the first-time-ever visit, and the user shall click on the ``Got it'' button to access the webpage content. \TOOL has to locate this button and click on it automatically. As a result, the website starts loading advertisements and contacting trackers in background. We refer to these two types of visits as \BEFORE and \AFTER in the remainder of the paper.

We implement \TOOL using the Selenium browser automation tool~\cite{avasarala2014selenium}, the de-facto standard for browser automation. We focus on Google Chrome, but we could easily extend it to other browsers.

Given a target URL, \TOOL carries out the following tasks:
\begin{enumerate}
    \item It navigates to the URL with a fresh browser profile, i.e., with an empty cache and cookie storage. This makes the visit the equivalent of a \BEFORE to the website.
    \item It inspects the Document Object Model (DOM) of the rendered webpage to find a possible \emph{Accept-button} in a Privacy Banner. For this, we match a list of keywords on the text of each node of the DOM. We identify an \emph{Accept-button} if we exactly match one of our keywords. For robustness, the match is case insensitive, and leading, trailing or repeated blank characters are removed.
    \item If \TOOL finds the \emph{Accept-button}, it tries to accept the default privacy policies by clicking on the corresponding DOM element (typically a \texttt{<button>}, \texttt{<href>} or \texttt{<span>} element).
    \item \TOOL then revisits the URL to collect statistics about the \AFTER experience.
\end{enumerate}

In the beginning, we built \TOOL to look for accept buttons through CSS selectors combined with keywords as done in~\cite{vallina2019tales} and popular add-ons. However, we soon observed that this methodology was too fragile as the use of selectors is strongly CMP-specific and highly customizable by webmasters. The keyword-based approach eases the generalization of the solution. Considering the complexity, \TOOL adds marginal overhead to the time required to visit a webpage. Only for very complex webpages, iterating through all DOM elements may require some time, but this is still much less than the time needed to load and render the webpage by the browser. 

During each visit, \TOOL stores metadata regarding the whole process in a JSON log file. It includes details on all HTTP transactions and installed cookies. Moreover, it optionally takes screenshots of the webpage during the various phases to allow manual verification.

\TOOL is highly customizable and offers the user various features. It lets the user customize the declared \texttt{User-Agent} and browser language (in the \texttt{Accept-Language} headers). Important to our analysis, it runs a:
\begin{itemize}
    \item \emph{Warm-up visit}: to populate the browser cache.
    \item \BEFORE: to collect statistics on the webpage before accepting the privacy policy, as a Naive Crawler would do.
    \item \AFTER: to collect statistics on the webpage as it appears after accepting the privacy policy (if an \emph{Accept-button} is found).
    \item \INTERNAL: to a number of webpages of the same website, randomly choosing among the internal links. This step runs regardless of the presence of the \emph{Accept-button}.
\end{itemize}

Among metadata \TOOL collects, we record the Page Load Time, or \emph{OnLoad} time, on all visits. It allows us to compare the performance with and without privacy acceptance. The \emph{OnLoad} time is a performance index often used as a proxy for Quality of Experience measurements~\cite{da2018narrowing}. We leave the measurements of more sophisticated QoE-related metrics such as the SpeedIndex~\cite{speedindex} as future work. Moreover, we neglect metrics that are not affected by the presence of a Privacy Banner, such as the Time-to-first-byte (TTFB). To avoid suffering the bias of the \AFTER that can only occur with a warm browser cache, we run a preliminary \emph{Warm-up visit}, then we perform another \BEFORE and take performance measurement only on the latter. This lets us fairly compare the \emph{OnLoad} on the two visits with hot cache in both cases. Alternatively, \TOOL can erase the HTTP cache and clean the socket pool upon each visit to measure webpage performance with a cold cache.

At last, to limit the impact of random delay due to webpage download and rendering, \TOOL uses quite conservative timeouts before eventually abort the visit. In detail, the DOM inspection starts 5 seconds after the \emph{OnLoad} event. While this clearly slows down the visit of multiple webpages, it maximizes the accept success rate.

To allow large-scale measurement campaigns, we containerize \TOOL using the Docker container engine~\cite{docker}. In the containerized version, we use Google Chrome version 89 in headless mode and force it to use a standard \texttt{User-Agent} instead of the pre-defined \texttt{ChromeHeadless}.

We offer \TOOL as open-source to foster its usage and allow the reproducibility of the results presented in this paper. For this, we also commit to releasing all the data we collected for this study.

\subsection{Keyword Selection and Validation}

\begin{figure}[t]
    \centering
    \includegraphics[width=0.5\columnwidth]{figures/cookieaccept_validation_third_round.pdf}
    \caption{Validation results of \TOOL over 200 randomly picked websites per country.}
    \label{fig:validation}
\end{figure}

\begin{figure}[t]
    \centering
    \includegraphics[width=0.5\columnwidth]{figures/cookieaccept_keywords_freq.pdf}
    \caption{Frequency of the \TOOL keywords, with some examples reported.}
    \label{fig:keywords}
\end{figure}

The core of \TOOL is the list of keywords to be matched against the webpage content to localize the clickable DOM element for accepting the privacy policy. We thoroughly build this list manually in an iterative way. To handle different languages, we build a list that includes keywords for each country we are interested in. For this work, we focus on 5 European countries, namely France, Germany, Italy, Spain, UK\footnote{Since January 2021 UK has enforced the UK GDPR - with practically identical requirements.}, plus the US -- which we use as an example of an extra-EU country. For each country, we pick the most popular websites according to the Similarweb lists~\cite{similarweb}, a website-ranking service analogous to Alexa.

\subsubsection{First Round - keyword extraction from top websites}

In the first round, for each of the $5$ countries, we consider the top-200 websites that have a Privacy Banner. We randomly choose half of these websites and manually visit them (from Europe) to extract the accept keyword. In total, we identify $186$ unique keywords. We next instruct \TOOL to visit the other half of websites and accept privacy policies. For those where it fails, we manually visit them and extract keywords. With this, we include $36$ new keywords, $222$ in total.

\subsubsection{Second Round - testing and keyword increase}

To evaluate the accuracy of \TOOL in the wild, we next consider 200 new random websites for each country from the Similarweb lists. We let \TOOL visit them and manually check the subset of websites for which \TOOL fails to accept the privacy policy. We depict the results in Figure~\ref{fig:validation}. \TOOL can accept the privacy policy in more than half of websites. In $6-14\%$ of cases, we find new keywords -- that we promptly add to our list. Interestingly, we find a non-negligible portion of websites ($26-30\%$) that do not present any Privacy Banner. At last, \TOOL fails to accept privacy in only $5-8\%$ of cases. Investigating further, this is due to non-standard behaviors of the webpage when accessed in headless mode. For instance, some websites present a CAPTCHA when they detect an automated visit; other websites return a blank webpage. This is a common problem for any crawler-based measurement study~\cite{vastel2020fp}. Note that cases of \emph{False Positives} -- i.e., \TOOL clicking on a wrong DOM element -- are possible, although we have not observed any during the development and testing phases. 

At the end of the keyword list building phases, we collect a total of $258$ keywords covering $6$ languages.\footnote{In Spain, some websites are in Catalan, rather than in Spanish.} The most frequent one is the simple ``Ok'' string. In Figure~\ref{fig:keywords} we show the cumulative distribution of keyword frequency on the whole set of Similarweb websites with some keyword examples. The top-50 keywords already cover $87\%$ of websites, while 100 are enough to cover $95\%$.  Interestingly, we find also complex strings like ``I'm fine with this'' or ``Alle auswählen, weiterlesen und unsere arbeit unterstützen''.\footnote{Which translates to ``Select all, keep reading and support our work''.}



\subsection{\TOOL vs. Consent-O-Matic}
\label{sec:ca_vs_com}

\begin{figure}[t]
    \centering
    \includegraphics[width=0.5\columnwidth]{figures/cookieaccept_consentomatic_random.pdf}
    \caption{Privacy policy acceptance rate of \TOOL and Consent-O-Matic on 100 websites per country.}
    \label{fig:ca_vs_com}
\end{figure}

We compare the effectiveness of \TOOL with Consent-O-Matic, the most mature browser plugin designed to offer/deny consent to privacy policies automatically. Unlike our tool, Consent-O-Matic exploits the presence of popular Consent Management Providers (CMP), services that take care of the management of users' choices on behalf of the website. At the time of writing, Consent-O-Matic allows managing Privacy Banners for 35 CMPs. To gauge its performance, we visit the top-100 most popular websites with a Privacy Banner for the 5 countries using a Chrome browser with the Consent-O-Matic plugin enabled. Consent-O-Matic accepts the privacy policies in less than 35\% of websites with Privacy Banner, and as little as 17\% and 20\% for websites in Italy and UK, respectively. Here \TOOL accepts the privacy policies on all websites by construction.

We then run a second experiment considering another set of 100 websites randomly picked from the Similarweb per country lists. We visit each website with \TOOL and a Consent-O-Matic-enabled browser. Figure~\ref{fig:ca_vs_com} summarizes the comparison. \TOOL can accept the privacy policies in more than 50\% of websites, more than twice the success rate of Consent-O-Matic. These results are in line with those of Figure~\ref{fig:validation}. The remaining websites may not have a Privacy Banner, fail to load, or use an unknown keyword. This testifies that the customization of Privacy Banners makes it difficult to engineer a generic and simple solution. The keyword-based strategy results more robust than the CMP-based approach (with similar complexity in curating the lists).



\subsection{Dataset and Tracker list}
\label{sec:dataset}

In the following, we use \TOOL to run several measurement campaigns. Most of our analyses, unless otherwise indicated, targets a large set of websites popular in Europe, using a test server located in a European university campus. We also use the US as a representative of an extra-EU country. For each of the $6$ countries, we consider the top 100 websites from 25 different categories - see Figure~\ref{fig:ca_category}. In total, we include $12\,277$ unique websites to visit (as the lists in different countries partially overlap).

We run \TOOL on the target websites using a single high-end server running 16 parallel instances to speed up the crawl. We instrument it to run a \emph{test sequence}, which consists in a \emph{Warm-up visit}, \BEFORE, and \AFTER to the landing page, followed by \INTERNAL to 5 randomly chosen internal pages -- previous studies indeed show that internal and landing pages have different properties~\cite{aqeel2020on}. For each website, we repeat the test sequence $5$ times, randomizing the order of websites to visit in each repetition.  Our main experimental campaign took place for two weeks on April 2021.

We run additional measurement campaigns to investigate specific aspects. First, we repeat the above experiments using servers located in the US, Brazil and Japan. We use Amazon Web Services to deploy on-demand servers on the desired availability zone. Our goal is to understand whether Privacy Banners appear or have a different impact depending on the visitor location. Second, we visit the top-100\,000 websites according to the Tranco list~\cite{pochat2018tranco} to offer a view on a larger number of websites. Unfortunately, the Tranco list does not offer a per-category and per-country rank. We test these websites twice, with and without clearing the browser cache, to compare webpage performance on the \BEFORE and on the \AFTER both with a warm and a cold cache.

To observe how the presence of trackers changes from the \BEFORE to the \AFTER, we rely on publicly-available lists provided by Whotracksme~\cite{whotracksme} (a tracking-related open-data provider), EasyPrivacy~\cite{easyprivacy} (one of the lists at the core of AdBlock tracker-blocking strategy) and AdGuard~\cite{adguard} (another ad-blocking tool). For robustness, we merge the three lists and consider as potential trackers those third-party domains that appear in at least two lists. In total, we obtain $1\,497$ domains that we consider tracking services.\footnote{In the following, we identify them with their \emph{second-level domain name} -- i.e., a hostname truncated after the second label. We handle the case of two-label country code TLDs such as \texttt{co.uk}.} We then record the presence of a tracker during a visit if the webpage embeds an object from a tracking domain, and the latter installs a cookie with a lifetime longer than one month~\cite{trevisan20194} -- commonly referred to as \textit{profiling cookie}. As such, we divide the HTTP transactions carried out during a visit in: 
\begin{itemize}
    \item First-Party: objects from the same domain of the target webpage.
    \item Third-Party: objects from a different domain than the target webpage.
    \item Trackers: objects from a Third-Party that is a tracking domain and sets a profiling cookie.
\end{itemize}

\subsection{Results on YCB-Video}

\begin{table}[b]
    \centering
\begin{tabular}{l|ccc}
    \toprule
                           &    ADD-S  &    ADD(-S) \\\midrule
DenseFusion (per-pixel)    &     91.2  &     82.9   \\ 
DenseFusion (iterative)    &     93.2  &     86.1   \\
CosyPose                   &     89.8  &     84.5   \\
PVN3D                      &     95.5  &     91.8   \\     
FFB6D                      &     96.6  &     92.7   \\     
ES6D                       &     93.6  &     89.0   \\   
SyMFM6D                    & \tb{96.8} & \tb{94.1}  \\ 
\bottomrule
\end{tabular}
    \caption{Single-view results on YCB-Video using the AUC metrics for ADD-S and \mbox{ADD(-S)}. The best results are printed in bold.}
    \label{tab_ycbv_sv}
\end{table}

\cref{tab_ycbv_sv} compares the single-view performance of our SyMFM6D network with all baseline methods using the AUC of ADD-S and \mbox{ADD(-S)} on YCB-Video. Please note that MV6D corresponds to PVN6D in the single-view scenario. The results show that our approach copes very well with the dynamic camera setup of YCB-Video while outperforming all methods significantly. On the symmetry-aware \mbox{ADD(-S)} AUC metric, SyMFM6D outperforms the current state-of-the-art FFB6D by even \SI{1.5}{\%}. 
Please note that unlike DenseFusion (iterative) and CosyPose, our approach does not perform computationally expensive post processing or iterative refinement procedures.


To examine the effect of our symmetry-aware training procedure, we provide an object-wise evaluation of the three best single-view methods on YCB-Video in \cref{fig_ycb_sv_objects}. Please note that in single-view mode, our model architecture is the same as FFB6D except for our novel symmetry-aware loss function. 
The results show that not only most symmetric objects (highlighted in bold) are estimated more accurate but also most non-symmetric objects.
This indicates that there is a synergy effect which improves the keypoint detection for non-symmetric objects due to an improvement of the keypoint detection for symmetric objects.

\begin{table}[tbp]
    \vspace{2mm}
    \centering
    \begin{tabular}{l|ccc}
        \toprule 
        Object class  		   &   PVN3D  &   FFB6D  &  SyMFM6D  \\\midrule
        Master chef can        &    80.5  &    80.6  &\tb{80.7} \\
        Cracker box            &    94.8  &    94.6  &\tb{94.9} \\
        Sugar box              &    96.3  &\tb{96.6} &\tb{96.6} \\
        Tomato soup can        &    88.5  &\tb{89.6} &    87.9  \\
        Mustard bottle         &    96.2  &    97.0  &\tb{97.8} \\
        Tuna fish can          &    89.3  &    88.9  &\tb{92.3} \\
        Pudding box            &\tb{95.7} &    94.6  &    93.3  \\
        Gelatin box            &    96.1  &\tb{96.9} &    96.1  \\
        Potted meat can        &    88.6  &    88.1  &\tb{90.0} \\
        Banana                 &    93.7  &    94.9  &\tb{95.2} \\
        Pitcher base           &    96.5  &    96.9  &\tb{97.5} \\
        Bleach cleanser        &    93.2  &\tb{94.8} &    93.9  \\
        \tb{Bowl}              &    90.2  &    96.3  &\tb{96.4} \\
        Mug                    &    95.4  &    94.2  &\tb{95.7} \\
        Power drill            &    95.1  &    95.9  &\tb{96.4} \\
        \tb{Wood block}        &    90.4  &    92.6  &\tb{95.2} \\
        Scissors               &    92.7  &    95.7  &\tb{95.8} \\
        Large marker           &\tb{91.8} &    89.1  &    90.0  \\
        \tb{Large clamp}       &    93.6  &    96.8  &\tb{96.9} \\
        \tb{Extra large clamp} &    88.4  &\tb{96.0} &    95.3  \\
        \tb{Foam brick}        &    96.8  &    97.3  &\tb{97.6} \\\midrule
        ALL                    &    91.8  &    92.7  &\tb{94.1} \\\bottomrule
    \end{tabular} 
	\caption{Single-view results on YCB-Video evaluated for each object class individually using the \mbox{ADD(-S)} AUC metric. Symmetric objects and the best results are printed in bold.}
	\label{fig_ycb_sv_objects}
\end{table}

\cref{fig_ycbv_sv} shows a visualization of three scenes of YCB-Video with 6D pose ground truth, predictions of FFB6D, and predictions of our SyMFM6D network using only the depicted view. It can be seen that both FFB6D and SyMFM6D estimate very accurate poses as the scenes of YCB-Video contain only a few objects and not many occlusions. However, SyMFM6D predicts even more accurate poses than FFB6D due to our proposed symmetry-aware training procedure.

\begin{figure*}[htbp]
        \vspace{2mm}
	\centering
	\begin{minipage}{0.24\textwidth}
		\centering
		\textbf{Original View}
	\end{minipage}%
	\begin{minipage}{0.24\textwidth}
		\centering
		\textbf{Ground Truth}
	\end{minipage}%
	\begin{minipage}{0.24\textwidth}
		\centering
		\textbf{FFB6D}
	\end{minipage}%
	\begin{minipage}{0.24\textwidth}
		\centering
		\textbf{SyMFM6D}
	\end{minipage}% 
	\setkeys{Gin}{width=0.24\linewidth}
	\includegraphics{figures/ycb_sv/0055_001042_rgb_1800.png}\,%
	\includegraphics{figures/ycb_sv/0055_001042_gt_1800.png}\,%
	\includegraphics{figures/ycb_sv/0055_001042_FFB6D_1view_1800.png}\,%
	\includegraphics{figures/ycb_sv/0055_001042_SyMV6D_16sym_1view.png}\,%
	\vspace{0.7mm}
	
	\includegraphics{figures/ycb_sv/0051_000636_rgb_864.png}\,%
	\includegraphics{figures/ycb_sv/0051_000636_gt_864.png}\,%
	\includegraphics{figures/ycb_sv/0051_000636_FFB6D_1view_864.png}\,%
	\includegraphics{figures/ycb_sv/0051_000636_SyMV6D_16sym_1view_864.png}\,%
	\vspace{0.7mm}
	
	\includegraphics{figures/ycb_sv/0058_000553_rgb_2461.png}\,%
	\includegraphics{figures/ycb_sv/0058_000553_gt_2461.png}\,%
	\includegraphics{figures/ycb_sv/0058_000553_FFB6D_1view_2461.png}\,%
	\includegraphics{figures/ycb_sv/0058_000553_SyMV6D_16sym_1view.png}\,%
	\caption{Comparison of 6D pose predictions on single frames of the YCB-Video dataset.}
	\label{fig_ycbv_sv}
\end{figure*}

\cref{tab_ycbv_mv} compares our multi-view results with all multi-view baseline methods on YCB-Video using three and five input views.
We see that our approach with disabled symmetry training procedure already outperforms all previous multi-view methods significantly. Enabling the symmetry awareness further improves the results slightly. However, 
using more views does not improve the accuracy as most views of YCB-Video are very similar in which case additional views do not provide beneficial information while the learning problem of fusing different views becomes slightly harder.

\begin{table}[!hbt]
    \tabcolsep=1.35mm
    \centering
\begin{tabular}{l|cc|cc}
    \toprule 
                  & \multicolumn{2}{c|}{ADD-S}
                                          & \multicolumn{2}{c}{ADD(-S)} \\
                  &  3 views  &  5 views  &  3 views  &  5 views  \\\midrule
CosyPose          &     92.3  &     93.4  &     87.7  &     88.8  \\
MV6D              &     91.2  &     91.1  &     85.6  &     84.0  \\
SyMFM6D (no sym)  &     95.2  &     95.2  &     91.5  &     91.4  \\
SyMFM6D           & \tb{95.4} & \tb{95.4} & \tb{91.7} & \tb{91.6} \\
\bottomrule
\end{tabular}
    \caption{Quantitative multi-view results on YCB-Video. The best results are printed in bold.}
    \label{tab_ycbv_mv}
\end{table}



\subsection{Results on MV-YCB FixCam, WiggleCam and SymMovCam}

We show the quantitative results on the datasets MV-YCB FixCam, MV-YCB WiggleCam, and MV-YCB SymMovCam in \cref{tab_fixCam_wiggleCam}. It includes a comparison with two modified CosyPose (CP) versions with and without known camera poses as presented by \cite{mv6d}.
Our SyMFM6D network yields the best results on all metrics on all three datasets. This shows that SyMFM6D copes very well with the strong occlusions in the datasets. The results on WiggleCam are just slightly worse than on FixCam which demonstrates that our approach is robust towards inaccurately known camera poses.

On the novel SymMovCam dataset, our method outperforms the baselines by a much larger margin than on FixCam and WiggleCam. This is due to the symmetric objects in the datasets on which the keypoint estimation of the baseline methods is inaccurate. The results also prove that our approach is robust to very dynamic camera setups where the cameras are mounted at varying positions.


\begin{table*}[h]
	\tabcolsep=1.0mm
	\centering
	\begin{tabular}{r|cccccc|cccccc|ccccc}
    \toprule
                           &     \multicolumn{6}{c|}{MV-YCB FixCam}                        &      \multicolumn{6}{c|}{MV-YCB WiggleCam}                    &      \multicolumn{5}{c}{MV-YCB SymMovCam}            \\
                           &  PVN3D   &   FFB6D  &   CP   &    CP    &   MV6D   &   Ours   &  PVN3D   & FFB6D    &   CP   &   CP     &  MV6D    &  Ours    &  PVN3D   & FFB6D    &   Ours   &    MV6D  &  Ours    \\ 
    Number of views        &   1      &     1    &   3    &    3     &   3      &    3     &    1     & 1        &   3    &   3      &   3      &    3     &    1     &    1     &     1    &     3    &    3     \\
    Known cam poses        &\checkmark&\checkmark&$\times$&\checkmark&\checkmark&\checkmark&\checkmark&\checkmark&$\times$&\checkmark&\checkmark&\checkmark&\checkmark&\checkmark&\checkmark&\checkmark&\checkmark\\
    \midrule                                                                                      
    ADD-S AUC              &  81.3    &   82.3   &  90.8  &   91.9   &  96.9    &\tb{97.3} &   80.8   &   81.9   & 90.0   &  91.3    &    96.2  &\tb{96.7} &   75.0   &   79.9   &   80.6   &   92.8   & \tb{94.2}\\
    ADD(-S) AUC            &  74.9    &   76.3   &  82.4  &   84.6   &  94.8    &\tb{95.6} &   74.0   &   75.5   & 81.0   &  83.4    &    93.0  &\tb{94.2} &   68.5   &   75.6   &   76.7   &   88.7   & \tb{91.6}\\
    ADD-S \textless   ~\SI{2}{cm} &  82.1    &   83.6   &  92.9  &   93.0   &  98.8    &\tb{98.9} &   82.0   &   83.4   & 92.3   &  92.6    &    98.7  &\tb{98.8} &   77.2   &   81.1   &   81.9   &   96.3   & \tb{96.6}\\
    ADD(-S) \textless ~\SI{2}{cm} &  73.0    &   74.8   &  80.6  &   82.4   &  96.5    &\tb{96.8} &   72.4   &   74.0   & 78.9   &  81.6    &\tb{96.0} &\tb{96.0} &   64.5   &   74.5   &   76.3   &   91.6   & \tb{93.6}\\
    \bottomrule
	\end{tabular}
	\caption{Quantitative results on the datasets MV-YCB FixCam (left), MV-YCB WiggleCam (middle), and MV-YCB SymMovCam (right). The baseline CosyPose (CP) uses PVN3D as backend network as described in \cite{mv6d}. The best results per dataset are printed in bold.}
	\label{tab_fixCam_wiggleCam}
\end{table*}



\subsection{Keypoint Visualization}

\cref{fig_ycbv_sv_keypoints} shows predicted keypoints of FFB6D and SyMFM6D in a YCB-Video scene. We additionally visualize the keypoint proposals of each object in individual colors.
The resulting predicted keypoints are white, the target keypoints are black. You can see that both FFB6D and SyMFM6D predict very accurate keypoints on all non-symmetric objects. However, FFB6D fails to predict accurate keypoints on the large clamp which has one discrete rotational symmetry. This shortcoming of FFB6D is also apparent on other symmetric objects. We believe that this is caused by the ambiguities of the object poses resulting in ambiguous target keypoints which results in averaging over the multiple solutions given by the symmetry. Therefore, the training loss is minimized when predicting keypoints on the symmetric axis rather than predicting them on the desired target locations. SyMFM6D in contrast overcomes this problem by our novel symmetry-aware training procedure as it can be seen in \cref{fig_ycbv_sv_keypoints_SyMFM6D}.

\begin{figure}[!tbh]
  \centering
\begin{subfigure}[b]{0.49\columnwidth}
  \includegraphics[width=1.0\columnwidth]{figures/0054_000204_FFB6D_1view_keypoints_cropped.jpg}
   \caption{FFB6D}
   \label{fig_ycbv_sv_keypoints_FFB6D}
\end{subfigure}
\begin{subfigure}[b]{0.49\columnwidth}
  \centering
  \includegraphics[width=1.0\columnwidth]{figures/0054_000204_SyMFM6D_16sym_1view_keypoints_BestSym_cropped.jpg}
   \caption{SyMFM6D}
   \label{fig_ycbv_sv_keypoints_SyMFM6D}
   \end{subfigure}
	\caption{Visualization of the predicted keypoints on single frames of the YCB-Video dataset.} 
   \label{fig_ycbv_sv_keypoints}
\end{figure}


\subsection{Implementation Details and Runtime}

We trained our network up to seven days on four NVIDIA Tesla V100 GPUs with \SI{32}{GB} of memory. 
The network architecture of our SyMFM6D approach has 3.5 million trainable parameters and requires about \SI{46}{ms} for processing a single RGB-D image on a single GPU. 
Mean shift clustering and least-squares fitting for computing a 6D pose require additional \SI{14}{ms} per object. 
Please visit our previously mentioned GitHub repository for code, datasets, and further details.

\section{Conclusions and Future Work}\label{Sec:con}
In this paper, we proposed a new method to translate human demonstration videos to commands using deep recurrent neural networks. We conducted experiments with the LSTM and GRU network using different visual feature representations. The experimental results showed that our purely neural sequence to sequence architecture outperformed current state-of-the-art methods by a fair margin. We also introduced a new large-scale videos to commands dataset that is suitable for deep learning methods. Finally, we combined our proposed method with the vision and planning module, and performed various manipulation tests on a real full-size humanoid robot.

Our robotic experiments so far are qualitative. We have focused on demonstrating how our approach can be used in a real robotic system to reduce the (tedious) programming when there are many manipulation tasks. Although using the learning approach to translate the demonstration videos to commands could help the robot understand human actions in a meaningful way, the imitation step is still challenging since it requires a robust vision, planning (and LfD) system. Currently, our framework relies solely on the vision system to plan the actions. This does not allow the robot to perform accurate tasks such as ``hammering" or ``cutting" which require precise skills. Therefore, an interesting problem is to combine our approach with LfD techniques to improve the robot manipulation capabilities.


\section*{Acknowledgment}
\addcontentsline{toc}{section}{Acknowledgment}
This work is supported by the European Union Seventh Framework Programme [FP7-ICT-2013-10] under grant agreement no 611832 (WALK-MAN). 

\junk
{

 . task such as  We can also combine our approach with LfD techniques to allow the robot to perform more   It is clear that we need to improve both three components in our framework (translation, vision, and planning) in order to 
 
 
achieve the human level in manipulation tasks.



the robot can understand the concept of human action using our translation module, however, the imitation step is very challenging since it requires a perfect vision and planning to achieve human level. 



We propose the idea of translating videos to commands and use these commands in robotic manipulation tasks. Our approach potentially can perform useful tasks without the need of tedious programming. We have demonstrated that our framework can be used together with other modules such as vision recognition system to complete complicated task. It is also straightforward to combine our framework with other LfD methods to perform tasks that require precise accuracy.

 Using the proposed method, we introduced  detect object affordances with CNN. We have demonstrated that the affordance detection results can be improved by using an object detector and dense CRF. Moreover, we introduced a challenging dataset that is suitable for real-world robotic applications. From the detected affordances, we presented a grasping method that is robust to noisy data. The effectiveness of our approach was demonstrated by performing different grasping experiments in cluttered scenes on the full-size humanoid robot WALK-MAN. We hope this opens up the door to practical solutions to the current limitations of real-world SLAM applications. It is worth restating that t...
 
 


Currently, our approach needs two separate networks to detect object affordances. This architecture can't be trained end-to-end as a single network. In future work, we aim to overcome this limitation by developing a new architecture that can detect the object identity and its affordances simultaneously. Another interesting problem is to extend our robotics experiments with more complicated scenarios.
}

\section*{Acknowledgments}
The research leading to these results has been funded by the European Union's Horizon 2020 research and innovation program under grant agreement No. 871370 (PIMCity project) and the SmartData@PoliTO center for Data Science technologies.

\bibliographystyle{ACM-Reference-Format}
\bibliography{reference}

\newpage

\section*{Appendix}
\label{sec:appendix}

For the sake of completeness, we here report the same analyses depicted in Figure~\ref{fig:tranco_tp} and Figure~\ref{fig:ca_perf_tp} showing the number of Trackers instead of the number of Third-Parties. The two pictures lead to similar conclusions.

\begin{figure}[!h]
    \centering
    \includegraphics[width=0.5\columnwidth]{figures/cookieaccept_tranco_rank_eu_tracker_nb.pdf}
    \caption{Average number of Trackers per website (Tranco list).}
    \label{fig:tranco_trackers}
\end{figure}


\begin{figure}[!h]
    \centering
    \includegraphics[width=0.5\textwidth]{figures/cookieaccept_tracker_nb_tranco.pdf}
    \caption{Number of Trackers (Tranco list). Notice the log scales.}
    \label{fig:ca_perf_tracker}
\end{figure}

\end{document}