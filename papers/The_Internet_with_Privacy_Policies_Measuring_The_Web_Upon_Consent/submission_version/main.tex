\documentclass[acmsmall]{acmart}

\usepackage[hyphens]{url}
\usepackage{hyperref}

\usepackage[english]{babel}
\usepackage{blindtext}
\usepackage{xspace}
\usepackage{subcaption}
\settopmatter{printfolios=true}

%Comment commands
\newcommand{\nj}[1]{{\color{purple}{[nj: #1]}}}
\newcommand{\mt}[1]{{\color{red}{[mt: #1]}}}
\newcommand{\mm}[1]{{\color{blue}{[mm: #1]}}}
\newcommand{\lv}[1]{{\color{green}{[lv: #1]}}}
\newcommand{\TOOL}{{\emph{Priv-Accept}}\xspace}
\newcommand{\BEFORE}{{\emph{Before-Visit}}\xspace}
\newcommand{\AFTER}{{\emph{After-Visit}}\xspace}
\newcommand{\INTERNAL}{{\emph{Additional-Visits}}\xspace}

%Conference Info
\acmYear{2021}
\copyrightyear{2021}
\setcopyright{acmcopyright}
\acmJournal{TWEB}



\begin{document}
\title[The Internet with Privacy Policies]{The Internet with Privacy Policies: Measuring The Web Upon Consent}

%\author{Nikhil Jha, Martino Trevisan, Luca Vassio, Marco Mellia}
\author{Nikhil Jha}
\affiliation{
\institution{Politecnico di Torino}
\streetaddress{Corso Duca degli Abruzzi, 24}
\city{Torino}
\postcode{10129}
\country{Italy}}
\email{nikhil.jha@polito.it}

\author{Martino Trevisan}
\affiliation{
\institution{Politecnico di Torino}
\streetaddress{Corso Duca degli Abruzzi, 24}
\city{Torino}
\postcode{10129}
\country{Italy}}
\email{martino.trevisan@polito.it}

\author{Luca Vassio}
\affiliation{
\institution{Politecnico di Torino}
\streetaddress{Corso Duca degli Abruzzi, 24}
\city{Torino}
\postcode{10129}
\country{Italy}}
\email{luca.vassio@polito.it}

\author{Marco Mellia}
\affiliation{
\institution{Politecnico di Torino}
\streetaddress{Corso Duca degli Abruzzi, 24}
\city{Torino}
\postcode{10129}
\country{Italy}}
\email{marco.mellia@polito.it}

\begin{abstract}
To protect users' privacy, legislators have regulated the usage of tracking technologies, mandating the acquisition of users' consent before collecting data. Consequently, websites started showing more and more consent management modules -- i.e., Privacy Banners -- the visitors have to interact with to access the website content. They challenge the automatic collection of Web measurements, primarily to monitor the extensiveness of tracking technologies but also to measure Web performance in the wild. Privacy Banners, in fact, limit crawlers from observing the actual website content.

In this paper, we present a thorough measurement campaign focusing on popular websites in Europe and the US, visiting both landing and internal pages from different countries around the world. We engineer \TOOL, a Web crawler able to accept the privacy policies, as most users would do in practice. It lets us compare how webpages change before and after. Our results show that all measurements performed not dealing with the Privacy Banners offer a very biased and partial view of the Web. After accepting the privacy policies, web tracking is abundantly more pervasive, and webpages are larger and slower.
\end{abstract}

\begin{CCSXML}
<ccs2012>
   <concept>
       <concept_id>10003033.10003079.10011704</concept_id>
       <concept_desc>Networks~Network measurement</concept_desc>
       <concept_significance>500</concept_significance>
       </concept>
   <concept>
       <concept_id>10002978.10003029</concept_id>
       <concept_desc>Security and privacy~Human and societal aspects of security and privacy</concept_desc>
       <concept_significance>500</concept_significance>
       </concept>
 </ccs2012>
\end{CCSXML}

\ccsdesc[500]{Networks~Network measurement}
\ccsdesc[500]{Security and privacy~Human and societal aspects of security and privacy}

\ccsdesc[500]{Networks~Network measurement}

\keywords{Web Measurements, Crawling, Privacy Banner, GDPR}

\maketitle

\begin{figure}[t]
\begin{center}
   \includegraphics[width=1.0\linewidth]{figures/nas_comp_v3}
\end{center}
   \vspace{-4mm}
   \caption{The comparison between NetAdaptV2 and related works. The number above a marker is the corresponding total search time measured on NVIDIA V100 GPUs.}
\label{fig:nas_comparison}
\end{figure}

\section{Introduction}
\label{sec:introduction}

Neural architecture search (NAS) applies machine learning to automatically discover deep neural networks (DNNs) with better performance (e.g., better accuracy-latency trade-offs) by sampling the search space, which is the union of all discoverable DNNs. The search time is one key metric for NAS algorithms, which accounts for three steps: 1) training a \emph{super-network}, whose weights are shared by all the DNNs in the search space and trained by minimizing the loss across them, 2) training and evaluating sampled DNNs (referred to as \emph{samples}), and 3) training the discovered DNN. Another important metric for NAS is whether it supports non-differentiable search metrics such as hardware metrics (e.g., latency and energy). Incorporating hardware metrics into NAS is the key to improving the performance of the discovered DNNs~\cite{eccv2018-netadapt, Tan2018MnasNetPN, cai2018proxylessnas, Chen2020MnasFPNLL, chamnet}.


There is usually a trade-off between the time spent for the three steps and the support of non-differentiable search metrics. For example, early reinforcement-learning-based NAS methods~\cite{zoph2017nasreinforcement, zoph2018nasnet, Tan2018MnasNetPN} suffer from the long time for training and evaluating samples. Using a super-network~\cite{yu2018slimmable, Yu_2019_ICCV, autoslim_arxiv, cai2020once, yu2020bignas, Bender2018UnderstandingAS, enas, tunas, Guo2020SPOS} solves this problem, but super-network training is typically time-consuming and becomes the new time bottleneck. The gradient-based methods~\cite{gordon2018morphnet, liu2018darts, wu2018fbnet, fbnetv2, cai2018proxylessnas, stamoulis2019singlepath, stamoulis2019singlepathautoml, Mei2020AtomNAS, Xu2020PC-DARTS} reduce the time for training a super-network and training and evaluating samples at the cost of sacrificing the support of non-differentiable search metrics. In summary, many existing works either have an unbalanced reduction in the time spent per step (i.e., optimizing some steps at the cost of a significant increase in the time for other steps), which still leads to a long \emph{total} search time, or are unable to support non-differentiable search metrics, which limits the performance of the discovered DNNs.

In this paper, we propose an efficient NAS algorithm, NetAdaptV2, to significantly reduce the \emph{total} search time by introducing three innovations to \emph{better balance} the reduction in the time spent per step while supporting non-differentiable search metrics:

\textbf{Channel-level bypass connections (mainly reduce the time for training and evaluating samples, Sec.~\ref{subsec:channel_level_bypass_connections})}: Early NAS works only search for DNNs with different numbers of filters (referred to as \emph{layer widths}). To improve the performance of the discovered DNN, more recent works search for DNNs with different numbers of layers (referred to as \emph{network depths}) in addition to different layer widths at the cost of training and evaluating more samples because network depths and layer widths are usually considered independently. In NetAdaptV2, we propose \emph{channel-level bypass connections} to merge network depth and layer width into a single search dimension, which requires only searching for layer width and hence reduces the number of samples.

\textbf{Ordered dropout (mainly reduces the time for training a super-network, Sec.~\ref{subsec:ordered_droput})}: We adopt the idea of super-network to reduce the time for training and evaluating samples. In previous works, \emph{each} DNN in the search space requires one forward-backward pass to train. As a result, training multiple DNNs in the search space requires multiple forward-backward passes, which results in a long training time. To address the problem, we propose \emph{ordered dropout} to jointly train multiple DNNs in a \emph{single} forward-backward pass, which decreases the required number of forward-backward passes for a given number of DNNs and hence the time for training a super-network.

\textbf{Multi-layer coordinate descent optimizer (mainly reduces the time for training and evaluating samples and supports non-differentiable search metrics, Sec.~\ref{subsec:optimizer}):} NetAdaptV1~\cite{eccv2018-netadapt} and MobileNetV3~\cite{Howard_2019_ICCV}, which utilizes NetAdaptV1, have demonstrated the effectiveness of the single-layer coordinate descent (SCD) optimizer~\cite{book2020sze} in discovering high-performance DNN architectures. The SCD optimizer supports both differentiable and non-differentiable search metrics and has only a few interpretable hyper-parameters that need to be tuned, such as the per-iteration resource reduction. However, there are two shortcomings of the SCD optimizer. First, it only considers one layer per optimization iteration. Failing to consider the joint effect of multiple layers may lead to a worse decision and hence sub-optimal performance. Second, the per-iteration resource reduction (e.g., latency reduction) is limited by the layer with the smallest resource consumption (e.g., latency). It may take a large number of iterations to search for a very deep network because the per-iteration resource reduction is relatively small compared with the network resource consumption. To address these shortcomings,  we propose the \emph{multi-layer coordinate descent (MCD) optimizer} that considers multiple layers per optimization iteration to improve performance while reducing search time and preserving the support of non-differentiable search metrics.

Fig.~\ref{fig:nas_comparison} (and Table~\ref{tab:nas_result}) compares NetAdaptV2 with related works. NetAdaptV2 can reduce the search time by up to $5.8\times$ and $2.4\times$ on ImageNet~\cite{imagenet_cvpr09} and NYU Depth V2~\cite{nyudepth} respectively and discover DNNs with better performance than state-of-the-art NAS works. Moreover, compared to NAS-discovered MobileNetV3~\cite{Howard_2019_ICCV}, the discovered DNN has $1.8\%$ higher accuracy with the same latency.


\section{Background and related work}
\label{sec:history}

Content providers on the Web (websites, social networks, etc.) often monetize the content they offer using advertisements. To increase their effectiveness, the so-called behavioral advertisement leverages users' interests to provide targeted ads. This is possible thanks to Web trackers, i.e., third-party services embedded in the webpages that gather users' browsing history. Trackers are nowadays largely present on websites and reach the majority of internauts~\cite{metwalley2015online,pujol2015annoyed}. Trackers exploit cookies and advanced techniques to enable the collection of personal information~\cite{acar2014web,rizzo2021unveiling,papadogiannakis2021user}.

%In parallel, many solutions exist for blocking trackers and ads, typically implemented via browser plugins (e.g., AdBlock Plus~\cite{adblock}, Ghostery~\cite{ghostery} or Disconnect~\cite{disconnect}). Their effectiveness has been testified in several studies~\cite{pujol2015annoyed, traverso2017benchmark, mazel2019comparison}, and none of them offers complete protection.

\begin{figure}[t]
    \centering
    \includegraphics[width=0.75\columnwidth]{figures/cookie-example-large.pdf}
    \caption{Example of Privacy Banner on \texttt{dailymail.co.uk}. Only upon consent, trackers are contacted and ads displayed.}
    \label{fig:cookie_accept_example}
\end{figure}

\subsection{The Role of Legislators}

In this tangled picture, legislators started to regulate the ecosystem to avoid massive indiscriminate tracking that may threaten users' privacy. The first attempt has been the European Cookie Law~\cite{directive2009} entered into force in 2013, which mandates websites to ask for informed consent before using any profiling technology. In May 2018, the General Data Protection Regulation (GDPR)~\cite{gpdr} entered into force in all European member states. It is an extensive regulation on privacy, aiming at protecting users' privacy by imposing strict rules when handling personal information. Unlike previous regulations, it sets severe fines and infringements that could result in a fine of up to €10 million, or 2\% of the firm's worldwide annual revenue, whichever amount is higher. Some websites have already been caught to present legal violations in their Cookie Banner implementation~\cite{matte2020cookie} and a large fraction have been shown to use tracking technologies before users consent~\cite{trevisan20194, sanchez2019can}. In the US, the California Consumer Privacy Act (CCPA)~\cite{ccpa} similarly enhances privacy rights and consumer protection for California residents by requiring businesses to give consumers notices about their privacy practices.

As a result, most of the websites now provide explicit Privacy Banners~\cite{degeling2018we} and many adopt Consent Management toolsets~\cite{hills2020consent}, making the website content difficult to access until visitors accept the privacy policy. For example, Figure~\ref{fig:cookie_accept_example} shows the same news website homepage before and after accepting the privacy policy. Only upon pressing the ``Got it'' button, the website content is fully visible.
%In some cases, websites use more persistent and sophisticated forms of tracking in order to track users who denied their acceptance of cookies~\cite{papadogiannakis2021user}.



\begin{figure}[t]
    \centering
    \includegraphics[width=0.5\columnwidth]{figures/httparchive_websites_with_trackers_tld.pdf}
    \caption{Percentage of websites containing at least one tracker (from HTTPArchive). The black vertical line indicates the entry into force of the GDPR.}
    \label{fig:ha_websites_trackers}
\end{figure}



\subsection{The Effect of Privacy Banners on Web Measurements}

Despite cases of misuse, the new regulations had a large impact on the internauts, and this complicates the measurement of the tracking ecosystem. A simple Web crawler visiting the websites without accepting the privacy policies would offer a biased picture, with no trackers and no ads being loaded.  Hu~\emph{et al.}~\cite{hu2019characterising} already found that the number of third-parties dropped by more than 10\% after GDPR when visiting websites automatically. Conversely, when using a dataset from 15 real users, they measure no significant reduction in long-term numbers of third-party cookies. Dabrowski~\emph{et al.}~\cite{dabrowski2019measuring} draw similar conclusions, finding an apparent decrease in the use of persistent cookies from 2016 to 2018. Sorensen~\emph{et al.}~\cite{sorensen2019before} testify a decreasing trend in the number of third parties during 2018. We quantify this phenomenon in Figure~\ref{fig:ha_websites_trackers}, using the HTTPArchive open dataset~\cite{httparchive}. The curators of this dataset maintain a list of top websites worldwide that they automatically visit using the Google Chrome browser from a US-based server to store a copy of each visited webpage. Using the tracker list detailed in Section~\ref{sec:metho}, we report the percentage of websites embedding one or more trackers for 5 European countries (simply using the Top-Level Domain to identify the country). We restrict the analysis on those websites that exist for the whole six years-long periods ($9\,196$ website in total).

Figure~\ref{fig:ha_websites_trackers} apparently shows that the introduction of the GDPR (the black vertical line in May 2018) results in an abrupt decrease in the number of tracker-embedding websites, a trend that continues up to the moment we write. However, as we will show, these measurements are an artifact due to the GDPR itself. Indeed, the Web crawler used by HTTPArchive can only capture the behavior of the websites as a ``first-time visitor'', before the user accepts any privacy policy. The crawler thus misses cookies, third-party trackers, and any personalized ads.

Research papers that rely on crawling large portions of the Web for different reasons could be affected by the same bias in their measurements. For instance, this would challenge the automatic measurement of the Web ecosystem on privacy~\cite{acar2014web,falahrastegar2014rise,metwalley2015online,pujol2015annoyed,englehardt2016online,iordanou2018tracing,hu2019characterising,rizzo2021unveiling,vandrevu2019what,papadogiannakis2021user,aqeel2020on} and counter-measurements~\cite{pujol2015annoyed, traverso2017benchmark, mazel2019comparison}. Moreover, this will also impact those works that rely on crawlers and headless browsers~\cite{avasarala2014selenium} to quantify the impact in the wild of new technologies like SPDY,  HTTP/2~\cite{wang2014speedy,de2015http,bocchi2016measuring,erman2015towards}, 4G/5G~\cite{alay2017experience,asrese2019measuring}, accelerating proxies~\cite{sivakumar2014parcel,wang2016speeding,ruamviboonsuk2017vroom}, or generic benchmark solutions~\cite{netravali2015mahimahi}. At last, even spiders and mirroring tools like HTTPArchive may be affected if the website allows the visitor to access its content only after accepting the privacy policy.



\subsection{Related Work and Tools}

Authors of~\cite{vallina2019tales} are the first to consider the impact of the Privacy Banner presence. First, they instruct a custom OpenWPM crawler to identify specific Cookie Banners, and then they manually verify the results. Unfortunately, they solely focus on the pornographic ecosystem, which they acknowledge to be rather different from the Web at large, and thus their work can hardly be generalized.

Recently, authors of~\cite{aqeel2020on} demonstrated that it is fundamental to consider the complexity of the Web ecosystem and include internal pages in every measurement study. They find a number of recent works that neglect internal pages and, as such, might provide biased results. Yet, they ignore the implications of Privacy Banners. Here, we aim at providing an extensive and thorough study of their impact on the Web. Our goal is to enable the study of webpage characteristics as visitors would experience, assuming that most of them accept the default privacy setting as offered by the Privacy Banner. Indeed, it has been shown that users tend to ignore privacy-related notices~\cite{vila2003we, grossklags2007empirical, coventry2016personality}. Considering GDPR Privacy Banners, users tend to accept privacy policies when offered a default button via intrusive banners that nudge users ~\cite{CookieBenchmarkStudy,bauer2021you}, which is often the case~\cite{hausner2021dark} with websites offering large pop-ups or wall-style banners that cover most of the webpage as seen in Figure~\ref{fig:cookie_accept_example}.

There exist solutions that aim at automatically managing Privacy Banners: some browser add-ons try to hide Privacy Banners using a list of CSS selectors of known Privacy Banners. The most popular add-ons of this kind are ``I don't care about cookies''~\cite{idontcare} and ``Remove Cookie Banners''~\cite{remove}. Unfortunately, hiding the Privacy Banners has an unpredictable behavior, in some cases falling back to privacy policies acceptance, while, in other cases, triggering an opt-out choice. Other proposals, again in the form of browser add-ons, try to explicitly opt-in or opt-out to cookies. For example, ``Ninja Cookie''~\cite{ninja} approves only cookies strictly needed to proceed on the website. Conversely, Autoconsent~\cite{autoconsent} and Consent-O-Matic~\cite{consentomatic} use a set of predefined rules to either opt-in or opt-out to cookies, according to the user configuration. These two are the most similar solutions to \TOOL, as they allow to automate the action of providing consent to privacy policies if used in combination with a crawler. However, they are based on a list of actions the browser has to automatically run when finding a set of popular Consent Management Providers (CMPs), limiting their effectiveness. In Section~\ref{sec:ca_vs_com}, we compare \TOOL with Consent-O-Matic -- the most mature tool -- showing that it accepts privacy policies on a much smaller portion of websites than \TOOL. Indeed, the diversity of the Web ecosystem, the presence of multiple languages and the fully customizable choice of cookie banner buttons make the engineering of \TOOL not trivial.


\section{\TOOL design and testing}
\label{sec:metho}

We explicitly engineer \TOOL to fully automate the visit to websites and collect statistics. The key element of \TOOL is its ability to identify the presence of a Consent Banner and automatically accept privacy policies. We aim at a practical and effective approach to accept privacy policies through the offered button. %As previously said, most users will indeed be nudged in this direction, being the opt-out options often made cumbersome on purpose\cite{bauer2021you, hausner2021dark, CookieBenchmarkStudy}.

To illustrate \TOOL operation, consider again Figure~\ref{fig:cookie_accept_example}. A large Consent Banner appears on the first visit, and the user shall click on the ``Got it'' button to access the webpage content. \TOOL has to locate this button and click on it automatically. As a result, the website starts loading advertisements and contacting trackers in the background. We refer to these two types of visits as \BEFORE and \AFTER in the remainder of the paper.

We implement \TOOL using the Selenium browser automation tool~\cite{avasarala2014selenium}, the de-facto standard for browser automation, using Google Chrome as browser. Given a target URL, \TOOL carries out the following tasks:
\begin{enumerate}
    \item It navigates to the URL with a fresh browser profile, i.e., with an empty cache and cookie storage. This makes the visit the equivalent of a \BEFORE to the website.
    \item It inspects the Document Object Model (DOM) of the rendered webpage to find a possible \emph{Accept-button} in a Consent Banner. For this, it matches a list of keywords on the text of each node of the DOM. We identify an \emph{Accept-button} if we exactly match any of these keywords. For robustness, we remove leading/trailing/repeated blank characters and the match is performed ignoring the case. We do not use stemming, lemmatization or other techniques for text processing given the specificity of the words to match and the need to support multiple languages.
    \item If \TOOL finds the \emph{Accept-button}, it clicks on the corresponding DOM element (typically a \texttt{<button>}, \texttt{<href>} or \texttt{<span>} element) to accept the privacy policy and logs the success acceptance.
%    \item \TOOL then revisits the URL to collect statistics about the \AFTER experience.
\end{enumerate}

In the beginning, we built \TOOL to look for accept buttons through CSS selectors combined with keywords as done in~\cite{vallina2019tales} and popular add-ons. However, we soon observed that this methodology was too fragile as the use of selectors is strongly CMP-specific and highly customizable by webmasters. The keyword-based approach eases the generalization of the solution. Considering complexity, \TOOL adds marginal overhead to the time required to visit a webpage. Only for very complex webpages, iterating through all DOM elements may require some time, but this is still less than the time needed to load and render the webpage by the browser. 

During each visit, \TOOL stores metadata regarding the whole process in a JSON log file. It includes details on all HTTP transactions and installed cookies. Moreover, it optionally takes screenshots of the webpage during the various phases to allow manual verification.

\TOOL is highly customizable and offers the user various features. It lets the user customize the declared \texttt{User-Agent} and browser language (in the \texttt{Accept-Language} headers). Important to our analysis, it can be configured to run a:
\begin{itemize}
    \item \emph{Warm-up visit}: to populate the browser cache.
    \item \BEFORE: to collect statistics on the webpage before accepting the privacy policy, as a Naive Crawler would do.
    \item \AFTER: to collect statistics on the webpage as it appears after accepting the privacy policy (if an \emph{Accept-button} is found).
    \item \INTERNAL: to a number of webpages of the same website, randomly choosing among the internal links.\footnote{We define internal links as those having the same Fully Qualified Domain Name as the visited website.} We visit internal pages both if \TOOL finds the \emph{Accept-button} and if it does not.
\end{itemize}

For each page visit, \TOOL collect several metadata. Considering QoE metrics, here we focus on the Page Load Time, or \emph{OnLoad} time~\cite{da2018narrowing}. It allows us to compare the webpage rendering performance with and without privacy policy acceptance. It is simpler and faster to compute than the SpeedIndex~\cite{speedindex}, allowing large scale measurements. Notice that we neglect metrics that are not affected by the presence of a Consent Banner, such as the Time-to-first-byte (TTFB). 

Notice that the \AFTER visit can only occur with a warm browser cache in real cases since the browser would have first to complete the \BEFORE visit.
To fairly compare a \BEFORE and \AFTER, in our experiments we run a preliminary \emph{Warm-up visit} before the \BEFORE to fill the browser cache. This lets us appreciate the eventual extra time to load additional components and fairly compare the \emph{OnLoad} on the two visits with the hot cache. Alternatively, \TOOL can erase the HTTP cache and clean the socket pool upon each visit to compare webpage performance with a cold cache. 

\TOOL follows possible redirects during the visits and cases when a script triggers a reload of the webpage. This allows us to manage cases in which the consent banner is hosted on a separate specific landing page than the actual website home page. At last, to limit the impact of random delay due to webpage download and rendering, \TOOL uses quite conservative timeouts before eventually abort the visit. In detail, the DOM inspection starts 5 seconds after the \emph{OnLoad} event. While this clearly slows down the visit of multiple webpages, it maximizes the success rate.

To allow large-scale measurement campaigns, we containerize \TOOL using the Docker container engine~\cite{docker}. In the containerized version, we use Google Chrome version 89 in headless mode and force it to use a standard \texttt{User-Agent} instead of the pre-defined \texttt{ChromeHeadless}.\footnote{The containerized version is available on Docker Hub as \emph{martino90/priv-accept}.}

\subsection{Keyword Selection and Validation}
\label{sec:keywords}

At the core of \TOOL there is the list of keywords to be matched against the webpage content to localize the clickable DOM element for accepting the privacy policy. We thoroughly build this list manually in an iterative way. To handle different languages, we build a list that includes keywords for each country we are interested in. For this work, we focus on 5 European countries, namely France, Germany, Italy, Spain, UK\footnote{In January 2021 UK has enforced the UK GDPR, with practically identical requirements.}, plus the US -- which we use as an example of a large, extra-EU country were privacy laws are in force. For each country, we pick the most popular websites according to the Similarweb lists~\cite{similarweb}, a website-ranking service analogous to Alexa.

\subsubsection{First Round - keyword extraction from top websites}

In the first round, for each of the $5$ countries, we consider the top-200 websites that have a Consent Banner. We randomly choose half of these websites and manually visit them (from Europe) to extract the accept keyword. In total, we visit $500$ websites and identify $186$ unique keywords. We next instruct \TOOL to visit the other half of websites and let it accept the privacy policy, if found. For those where it fails ($233$ cases), we manually visit them to check i) if they have a Consent Banner, and ii) eventually to extract new keywords. With this, we identify $36$ new keywords, $222$ in total. During these steps, we also check that the tool correctly accepts the policy.

\subsubsection{Second Round - testing and keyword increase}

\begin{figure}[t]
    \centering
    \includegraphics[width=0.5\columnwidth]{figures/cookieaccept_validation_third_round.pdf}
    \caption{Validation results of \TOOL over 200 randomly picked websites per country. Upon two rounds of keyword selection, \TOOL 92\%-95\% accurate. }
    \label{fig:validation}
\end{figure}

\begin{figure}[t]
    \centering
    \includegraphics[width=0.5\columnwidth]{figures/cookieaccept_keywords_freq_abs_alt.pdf}
    \caption{Frequency of the \TOOL keywords, with indication of the coverage at different points. The top-98 keywords already cover 95\% of websites.}
    \label{fig:keywords}
\end{figure}

To evaluate the accuracy of \TOOL in the wild, we next consider $200$ new random websites for each country from the Similarweb lists, $1000$ websites in total. We let \TOOL visit them and manually check the subset of $448$ websites for which \TOOL did not find (and accepted) a privacy policy. We depict the results in Figure~\ref{fig:validation}. \TOOL can accept the privacy policy in more than half of websites, independently from the language. In $6-14\%$ of cases, we find 36 new keywords -- that we promptly add to our list. Interestingly, we find a non-negligible portion of websites ($26-30\%$) that do not present any Consent Banner. At last, \TOOL fails to accept privacy in only $5-8\%$ of cases. Investigating further, this is due to some non-standard behavior of the webpage when accessed in headless mode. For instance, some websites present a CAPTCHA when they detect an automated visit; other websites return a blank webpage. This is a common problem for any crawler-based measurement study~\cite{vastel2020fp}. For completeness, cases of \emph{False Positives} -- i.e., \TOOL clicking on a wrong DOM element -- are possible, although we have not observed any in our manual validation tests. 

At the end of the keyword list building phases, we collect a total of $258 (186+36+36)$ keywords obtained by manually visiting $1181 (500+233+448)$ websites, covering 6 languages.\footnote{In Spain, some websites are in Catalan, rather than in Spanish.} In Figure~\ref{fig:keywords}, we show the distribution of keyword appearance frequency across the entire set of $12\,277$ Similarweb websites (see Section~\ref{sec:dataset} for details on this list). The most common keyword is the string ``Ok''. Red dots indicate the portion of websites covered by the top-$N$ keywords -- i.e., the coverage of the top-$N$ words. The top keywords are very common (note the logarithmic scale on the $y$-axis), with the top-$10$ that cover half of the websites. The top-$98$ keywords cover $95\%$ of the websites, while the remaining appear less than $10$ times each in the whole website set. Clearly, we expect the list of keywords to naturally grow as the tail of the Figure~\ref{fig:keywords} suggests. Notice indeed that more than 80 keywords have been found on a single website. Curiously, we find complex strings like ``I'm fine with this'' or ``Alle auswählen, weiterlesen und unsere arbeit unterstützen''.\footnote{Which translates to ``Select all, keep reading and support our work''.}



\subsection{\TOOL vs. Consent-O-Matic}
\label{sec:ca_vs_com}

\begin{figure}[t]
    \centering
    \includegraphics[width=0.5\columnwidth]{figures/cookieaccept_consentomatic_random.pdf}
    \caption{Privacy policy acceptance rate of \TOOL and Consent-O-Matic on 100 websites per country. \TOOL can find and accept Consent Banners on twice as many websites as Consent-O-Matic.}
    \label{fig:ca_vs_com}
\end{figure}

We compare the effectiveness of \TOOL with Consent-O-Matic, the most mature browser plugin designed to offer/deny consent to privacy policies automatically. Unlike our tool, Consent-O-Matic exploits the presence of popular Consent Management Platforms (CMP), services that take care of the management of users' choices on behalf of the website. At the time of writing, Consent-O-Matic allows managing Consent Banners for 35 CMPs. To gauge its performance, we visit the top-100 most popular websites with a Consent Banner for the 5 countries using a Chrome browser with the Consent-O-Matic plugin enabled. Consent-O-Matic accepts the privacy policies in less than 35\% of websites with Consent Banner, and as little as 17\% and 20\% for websites in Italy and UK, respectively. Here \TOOL accepts the privacy policies on all websites by construction.

We then run a second experiment considering another set of 100 websites randomly picked from the Similarweb per country lists. We visit each website with \TOOL and a Consent-O-Matic-enabled browser. Figure~\ref{fig:ca_vs_com} summarizes the comparison. \TOOL accepts the privacy policies in more than 50\% of websites, more than twice the success rate of Consent-O-Matic. These results are in line with those of Figure~\ref{fig:validation}. The remaining websites may not have a Consent Banner, fail to load, or use an unknown keyword. This testifies that the customization of Consent Banners makes it difficult to engineer a generic and simple solution. The keyword-based strategy results more robust than the CMP-based approach (with similar complexity in curating the lists).



\subsection{Dataset and Tracker list}
\label{sec:dataset}

In the following, we use \TOOL to check the impact of using \TOOL when doing large web measurement experiments. We targets a large set of websites popular in France, Germany, Italy, Spain and US, using a test server located in our university campus. For each of the $6$ countries, we use the Similarweb lists to select the top-100 websites from 24 different categories -- see Figure~\ref{fig:ca_category}. These are the top-level unique categories listed in the Similarweb page~\cite{similarwebcategories}. In total, we include $12\,277$ unique websites to visit (as the lists in different countries partially overlap). When visiting websites of a given country, we set the \texttt{Accept-Language} header to indicate the appropriate locale and country language. This behavior can be configured in the \TOOL configuration to allow further experimentation.

We run \TOOL on a single high-end server running 16 parallel instances to speed up the crawl. We instrument it to run a \emph{test sequence}, which consists in a \emph{Warm-up visit}, \BEFORE and \AFTER to the landing page, followed by \INTERNAL to 5 randomly chosen internal pages -- previous studies indeed show that internal and landing pages have different properties~\cite{aqeel2020on}. For each website, we repeat the test sequence $5$ times, randomizing the order of websites to visit in each repetition.  Our main experimental campaign took place for two weeks on April 2021.

We run additional measurement campaigns to investigate specific aspects. To understand whether Consent Banners appear or have a different impact depending on the visitor location, we repeat the above experiments using servers located in the US, Brazil and Japan. We use Amazon Web Services to deploy on-demand servers on the desired availability zone. Here, we aim to check if websites behave differently based on the location of the visitors. Since we are using cloud servers, targeted websites may wrongly recognise the test machines as not regular users and located them in a generic or wrong country. While we cannot check this, we verified that the two most popular commercial IP location databases (IP2Location\footnote{\url{https://www.ip2location.com/}} and MaxMind\footnote{\url{https://www.maxmind.com/}}) map the IP addresses of our crawlers to the correct country. 

To offer a view on a larger number of websites, we visit the top-100\,000 websites according to the Tranco list~\cite{pochat2018tranco}. Unfortunately, the Tranco list does not offer a per-category and per-country rank. We run two separate test sequences: with warm caches, doing (i) \emph{Warm-up visit}, (ii) \BEFORE, and (iii) \AFTER. And with cold caches, (i) \BEFORE, (ii) erase HTTP cache and clean socket pool and (iii) \AFTER. Following this procedure, we ensure a fair comparison between \BEFORE and \AFTER in the two scenarios. Recall that \TOOL allows one to generate any combination of test sequence with warm/cold cache.

To observe how the presence of trackers changes, we rely on publicly-available lists provided by Whotracksme~\cite{whotracksme} (a tracking-related open-data provider), EasyPrivacy~\cite{easyprivacy} (one of the lists at the core of AdBlock tracker-blocking strategy) and AdGuard~\cite{adguard} (a popular ad-blocking tool). For robustness, we merge the three lists and consider as a potential tracker any third-party domain that appear in at least two lists. In total, we obtain $1\,497$ domains that we consider tracking services.\footnote{In the following, we identify them with their \emph{second-level domain name} -- i.e., a hostname truncated after the second label. We handle the case of two-label country code TLDs such as \texttt{co.uk}.} We finally record the presence of a tracker during a visit if the webpage embeds an object from a tracking domain, and the latter installs a cookie with a lifetime longer than one month~\cite{trevisan20194} -- commonly referred to as \textit{profiling cookie}. As such, we divide the HTTP transactions carried out during a visit in: 
\begin{itemize}
    \item First-Party: objects from the same domain of the target webpage.
    \item Third-Party: objects from a different domain than the target webpage.
    \item Trackers: objects from a Third-Party that is a tracking domain and sets a profiling cookie.
\end{itemize}
\section{Methodology}
\label{sec:benchmark}

\subsection{Description of hardware and software}
\label{ssec:supercomp}
The benchmarks reported in this paper were performed on the Intel Xeon Phi systems provided by the Joint Laboratory for System Evaluation (JLSE) and the Theta supercomputer at the Argonne Leadership Computing Facility (ALCF) \cite{alcf}, which is a part of the U.S. Department of Energy (DOE) Office of Science (SC) Innovative and Novel Computational Impact on Theory and Experiment (INCITE) program \cite{incite}. Theta is a 10-petaflop Cray XC40 supercomputer consisting of 3,624 Intel Xeon Phi 7230 processors. Hardware details for the JLSE and Theta system are shown in \Cref{tab:hw}.

The Intel Xeon Phi processor used in this paper has 64 cores each equipped with L1 cache. Each core also has two Vector Processing Units, both of which need to be used to get peak performance. This is possible because the core can execute two instructions per cycle. In practical terms, this can be achieved by using two threads per core. Pairs of cores constitute a tile. Each tile has an L2 cache symmetrically shared by the core pair. The L2 caches between tiles are connected by a two dimensional mesh. The cores themselves operate at 1.3 GHz. Beyond the L1 and L2 cache structure, all the cores in the Intel Xeon Phi processor share 16 GBytes of MCDRAM (also known as High Bandwidth Memory) and 192 GBytes of DDR4. The bandwidth of MCDRAM is approximately 400 GBytes/sec while the bandwidth of DDR4 is approximately 100 GBytes/sec. 

\begin{table}
  \caption{Hardware and software specifications}
  \label{tab:hw}

  \begin{tabularx}{\columnwidth}{XX}
  \toprule
			\multicolumn{2}{c}{\textbf{\intelphi\ node characteristics}} \\
    \midrule 
    \intelphi\ models				&	7210 and 7230 (64~cores, 1.3~GHz, 
    									2,622 GFLOPs) \\
    Memory per node					&	16 GB MCDRAM, \newline 192 GB DDR4 RAM \\
    Compiler						&	Intel Parallel Studio XE 2016v3 \\
    \midrule
    		\multicolumn{2}{c}{\textbf{JLSE \iphi\ cluster (26.2 TFLOPS peak)}} \\
    \midrule
    \# of \intelphi\ nodes	&	10 \\
    Interconnect type				&	Intel Omni-Path\textsuperscript{TM} \\
    \midrule
    		\multicolumn{2}{c}{\textbf{Theta supercomputer (9.65~PFLOPS peak)}} \\
    \midrule
    \# of \intelphi\ nodes				&	3,624 \\
    Interconnect type				&	Aries interconnect with \newline Dragonfly topology \\
  \bottomrule
\end{tabularx}

\end{table}

\begin{table}
\begin{threeparttable}
  \caption{Chemical systems used in benchmarks and their size characteristics}
  \label{tab:chem}

  \begin{tabularx}{\columnwidth}{XYYYYY}

  \toprule

  \multirow{2}{*}{Name}	&	\multirow{2}{*}{\# atoms}	&	\multirow{2}{*}{\# BFs\tnote{a}}	&	\multicolumn{3}{c}{Memory footprint\tnote{b}, GB} \\
  \cmidrule(l){4-6}
  		& & &	{MPI\tnote{c}}	&	{Pr.F.\tnote{d}}	&	{Sh.F.\tnote{e}} \\
  \midrule
  	0.5~nm	&	44			&	660			&	7		&	0.13		&	0.03	\\
	1.0~nm	&	120			&	1800		&	48		&	1			&	0.2	\\
	1.5~nm	&	220			&	3300		&	160		&	3			&	0.8	\\
	2.0~nm	&	356			&	5340		&	417		&	8			&	2	\\
	5.0~nm	&	2016		&	30240		&	9869	&	257			&	52	\\
	\bottomrule
  \end{tabularx}

  \begin{tablenotes}
  	\item [a] BF -- basis function
    \item [b] Estimated using \crefrange{eqn:mem:mpi}{eqn:mem:shr}
 	\item [c] MPI-only SCF code
    \item [d] Private Fock SCF code
    \item [e] Shared Fock SCF code
  \end{tablenotes}
\end{threeparttable}
\end{table}

These two levels of memory can be configured in three different ways (or modes). The modes are referred to as Flat mode, Cache mode, and Hybrid mode. Flat mode treats the two levels of memory as separate entities. The Cache mode treats the MCDRAM as a direct mapped L3 cache to the DDR4 layer. Hybrid mode allows the user to use a fraction of MCDRM as L3 cache allocate the rest of the MCDRAM as part of the DDR4 memory.
In Flat mode, one may choose to run entirely in MCDRAM or entirely in DDR4. The "numactl" utility provides an easy mechanism to select which memory is used. It is also possible to choose the kind of memory used via the "memkind" API, though as expected this requires changes to the source code.

Beyond memory modes, the Intel Xeon Phi processor supports five cluster modes. The motivation for these modes can be understood in the following manner: to maintain cache coherency the Intel Xeon Phi processor employs a distributed tag directory (DTD). This is organized as a set of per-tile tag directories (TDs), which identify the state and the location on the chip of any cache line. For any memory address, the hardware can identify the TD responsible for that address. The most extreme case of a cache miss requires retrieving data from main memory (via a memory controller). It is therefore of interest to have the TD as close as possible to the memory controller. This leads to a concept of locality of the TD and the memory controllers.
It is in the developer's interest to maintain the locality of these messages to achieve the lowest latency and greatest bandwidth of communication with caches. Intel Xeon Phi supports all-to-all, quadrant/hemisphere and sub-NUMA cluster SNC-4/SNC-2 modes of cache operation.

For large problem sizes, different memory and clustering modes were observed to have little impact on the time to solution for the three versions of the GAMESS code. For this reason, we simply chose the mode most easily available to us. In other words, since the choice of mode made little difference in performance, our choice of Quad-Cache mode was ultimately driven by convenience (this being the default choice in our particular environment). Our comments here apply to large problem sizes, so for small problem sizes, the user will have to experiment to find the most suitable mode(s).


\subsection{Description of chemical systems}
\label{ssec:chemical}
For benchmarks, a system consisting of parallel series of graphene sheets was chosen. This system is of interest to researchers in the area of (micro)lubricants \cite{kawai2016superlubricity}. A physical depiction of the configuration is provided in \Cref{fig:graphene}. The graphene-sheet system is ideal for benchmarking, because the size of the system is easily manipulated. Various Fock matrix sizes can be targeted by adjusting the system size.

\begin{figure}
	\includegraphics[width=\columnwidth]{Figure2}
	\caption{Model system of a C$_{2016}$ graphene bilayer. In the text, we refer to this system as 5~nm.
    		 There are two layers with size 5~nm by 5~nm.
             Each graphene layer consists of 1,008 carbon atoms.}
    \label{fig:graphene}
\end{figure}

\subsection{Characteristics of datasets}
\label{ssec:datasets}
In all, five configurations of the graphene sheets system were studied. The datasets for the systems studied are labeled as follows: 0.5~nm, 1.0~nm, 1.5~nm, 2.0~nm, and 5.0~nm.  \Cref{tab:chem} lists size characteristics of these configurations. The same 6-31G(d) basis set (per atom) was used in all calculations. For N basis functions, the density, Fock, AO overlap, one-electron Fock matrices and the matrix of MO coefficients are N$\times$N in size. These are the main data structures of significant size. Therefore, the benchmarks performed in this work process matrices which range from 660$\times$660 to 30,240$\times$30,240. For each of the systems studied, \Cref{tab:chem} lists the memory requirements of the three versions of GAMESS HF code.
Denoting $N_{BF}$ as the number of basis functions, the following equations describe the asymptotic $(N_{BF}\to\infty)$ memory footprint for the studied HF algorithms:
\begin{subequations}
	\label{eqn:mem}
	\begin{align}
		M_{MPI} =& 5/2 \cdot N_{BF}^2 \cdot N_{MPI\_per\_node}, 				\label{eqn:mem:mpi} \\
		M_{PrF} =& (2+N_{threads}) \cdot N_{BF}^2 \cdot N_{MPI\_per\_node}, 	\label{eqn:mem:prv} \\
		M_{ShF} =& 7/2 \cdot N_{BF}^2 \cdot N_{MPI\_per\_node},					\label{eqn:mem:shr}
	\end{align}
\end{subequations}
where $M_{MPI}$, $M_{PrF}$, $M_{ShF}$ denote the memory footprint of MPI-only, private Fock, and shared Fock algorithms respectively; $N_{threads}$ denotes the number of threads per MPI process for the OpenMP code, and $N_{MPI\_per\_node}$ denotes the number of MPI processes per KNL node. For OpenMP runs $N_{MPI\_per\_node}=4$, while for MPI runs the number of MPI ranks was varied from 64 to 256.

If one compares columns MPI versus Pr.F and Sh.F. in \Cref{tab:chem}, you will see that the private Fock code has about a 50 times less footprint compared to the stock MPI code. For the shared Fock code, the difference is even more dramatic with a savings of about 200 times. The ideal difference is 256 times since we compare 256 MPI ranks in the stock MPI code where all data structures are replicated versus 1 MPI rank with 256 threads for the hybrid MPI/OpenMP codes. But we introduced additional replicated structures (see \Cref{fig:buffer}) and many relatively small data structures are replicated also in the MPI/OpenMP codes. This explains the difference between the ideal and observed footprints.

Each of the aforementioned datasets was used to benchmark three versions of the GAMESS code. The first version is the stock GAMESS MPI-only release that is freely available on the GAMESS website~\cite{gamesswebsite}. The second version is a hybrid MPI/OpenMP code, derived from the stock release. This version has a shared density matrix, but a thread-private Fock matrix. The third version of the code is in turn derived from the second version; it has shared density and Fock matrices. A key objective was to see how these incremental changes allow one to manage (i.e., reduce) the memory footprint of the original code while simultaneously driving higher performance.

\section{Results}
\label{sec:results}

\subsection{Single node performance}
\label{ssec:singlenode}
The second generation Intel Xeon Phi processor supports four hardware threads per physical core. Generally, more threads per core can help hide latencies inherent in an application. For example, when one thread is waiting for memory, another can use the processor. The out-of-order execution engine is beneficial in this regard as well. To manipulate the placement of processes and threads, the \verb|I_MPI_DOMAIN| and \verb|KMP_AFFINITY| environment variables were used. 
We examined the performance picture when one thread per core is utilized and when four threads per core are utilized. As expected, the benefit is highest for all versions of GAMESS for two threads (or processes) per core. For three and four threads per core, some gain is observed, albeit at a diminished level. \Cref{fig:afty} shows the scaling curves with respect to the number of hardware threads utilized observed by us.

\begin{figure}
	\includegraphics[width=\columnwidth]{Figure3}
	\caption{Performance dependence on OpenMP thread affinity type for the shared Fock version of the GAMESS code
    		 on a single \intelphireg\ processor using the 1.0 nm benchmark.
             All calculations are performed in quad-cache mode.
             Four MPI ranks were used in all cases.
             The number of threads per MPI rank was varied from 1 to 64.}
    \label{fig:afty}
\end{figure}

\begin{figure}
	\includegraphics[width=\columnwidth]{Figure4}
	\caption{Scalability with respect to the number of hardware threads of the original MPI code
    		and two OpenMP versions on a single \intelphireg\ processor using the 1.0~nm benchmark.}
    \label{fig:singlescaling}
\end{figure}

As a first test, single-node scalability was examined with respect to hardware threads of all three versions of GAMESS. For the MPI-only version of GAMESS, the number of ranks was varied from~4 to~256. For the hybrid versions of GAMESS, the number of ranks times the number of threads per rank is the number of hardware threads targeted. The larger memory requirements of the original MPI-only code restrict the computations to, at most, 128 hardware threads. In contrast, the two hybrid versions can easily utilize all 256 hardware threads available. Finally, in general terms, on cache based memory architectures, it is expected that larger memory footprints potentially lead to more cache capacity and cache line conflict effects. These effects can lead to diminished performance, and this is yet another motivation to look at a hybrid MPI+X approach.

The results of our single-node tests are plotted in \Cref{fig:singlescaling}. It is found that using the private Fock version leads to the best time to solution for the 1.0~nm dataset, for any number of hardware threads. This version of the code is much more memory-efficient than the stock version but, because the Fock matrix data structure is private, it has a much larger memory footprint than the shared Fock version of GAMESS. Nevertheless, because the Fock matrix is private, there is less thread contention than the shared Fock version.

It was mentioned in \Cref{ssec:omp} that shared Fock algorithm introduces additional overhead for thread synchronization. For small numbers of Intel Xeon Phi threads, this overhead is expected to be low. Therefore the shared Fock version is expected to be on par with the other versions. Eventually, as the overhead of the synchronization mechanisms begins to increase, the private Fock version of the code is found to dominate. In the end, the private Fock version outperforms stock GAMESS because of the reduced memory footprint, and outperforms the shared Fock version because of a lower synchronization overhead.
Therefore, on a single node, the private Fock version gives the best time-to-solution of the three codes, but the shared Fock version strikes a (better) balance between memory utilization and performance.

\begin{figure}
	\includegraphics[width=\columnwidth]{Figure5}
	\caption{Time to solution (x axis, time in seconds) for different clustering and memory modes.
    		 Left column displays the small chemical system -- 0.5~nm bilayer graphene and
             right column displays one of the largest molecules bilayer graphene -- 2.0~nm.}
    \label{fig:tts}
\end{figure}

Beyond this, one must consider the choice of memory mode and cluster mode of the Intel Xeon Phi processor. It should be noted that, depending on the compute and memory access patterns of a code, the choice of memory and cluster mode can be a potentially significant performance variable. The performance impact of different memory and cluster modes is examined for the 0.5~nm (small) and~2.0~nm (large) datasets. The results are shown in \Cref{fig:tts}. For both datasets, some variation in performance is apparent when different cluster modes and memory modes are used. The smaller dataset indicates more sensitivity to these variables than the larger dataset. Also, for both data sizes the private Fock version performs best in all cluster and memory modes tested. Also, except in the All-to-All cluster mode, the shared Fock version significantly outperforms the MPI-only stock version. In the All-to-All mode, the MPI-only version actually outperforms the shared Fock version for small datasets, and the two versions are close to parity for large datasets. In total, it is concluded that the quadrant-cache cluster-memory mode is best suited to the design of the GAMESS hybrid codes.

\subsection{Multi-node performance}
It is very important to note that the total number of MPI ranks for GAMESS is actually twice the number of compute ranks because of the DDI. The DDI layer was originally implemented to support one-sided communication using MPI-1. For GAMESS developers, the benefit of DDI is convenience in programming. The downside is that each MPI compute process is complemented by an MPI data server~(DDI) process, which clearly results in increased memory requirements. Because data structures are replicated on a rank-by-rank basis, the impact of DDI on memory requirements is particularly unfavorable to the original version of the GAMESS code. To alleviate some of the limitations of the original implementation, an implementation of DDI based on MPI-3 was developed \cite{pruitt2016private}. Indeed, by leveraging the ``native'' support of one-sided communication in MPI-3, the need for a DDI process alongside each MPI rank was eliminated. For all three versions of the code benchmarked here, no DDI processes were needed.

\begin{figure}
	\includegraphics[width=\columnwidth]{Figure6}
	\caption{Multi-node scalability of the Private Fock and the Shared Fock hybrid MPI-OpenMP
    		 and the MPI-only stock GAMESS codes on the Theta machine with the 2.0~nm dataset.
             The quad-cache cluster-memory mode was used for all data points.}
    \label{fig:2nm}
\end{figure}

\Cref{fig:2nm} shows the multi-node scalability of the MPI-only version of GAMESS versus the private Fock and the shared Fock hybrid versions. It is important to appreciate at the outset that the multi-node scalability of the original MPI-only version of GAMESS is already reasonable. For example, the code scales linearly to 256 Xeon Phi nodes, and it is really the memory footprint bottleneck that limits how well all the Xeon Phi cores on any given node can be used. This pressure is reduced in the private Fock version of the code, and it is essentially eliminated in the shared Fock version. Overall, for the 2~nm dataset, the shared Fock code runs about six times faster than stock GAMESS on 512 Xeon Phi processors. It resulted from the better load balance of the shared Fock algorithm that uses all four shell indices -- two are used in MPI and two are used in OpenMP workload distribution. The actual timings and efficiencies are listed in \Cref{tab:efficiency}.

\begin{table}
\begin{threeparttable}
  \caption{Parallel efficiency of the three different HF algorithms using 2.0~nm dataset}
  \label{tab:efficiency}
  \begin{tabularx}{\columnwidth}{XYYYYYY}

  \toprule
    	\multirow{2}{*}{\# Nodes}		&	\multicolumn{3}{c}{Time-to-solution, s} &
                            \multicolumn{3}{c}{Parallel efficiency, \%} \\
        \cmidrule(rl{0.75em}){2-4} \cmidrule(l){5-7}
  					&	{MPI\tnote{a}} &	{Pr.F.\tnote{a}} &	{Sh.F.\tnote{a}} &
                        {MPI\tnote{a}} &	{Pr.F.\tnote{a}} &	{Sh.F.\tnote{a}} \\

  	\midrule
		4	&	2661	&	1128	&	1318	&	100	&	100	&	100 \\
		16	&	685		&	288		&	332		&	97	&	98	&	99 \\
		64	&	195		&	78		&	85		&	85	&	90	&	97 \\
		128	&	118		&	49		&	43		&	70	&	72	&	96 \\
		256	&	85		&	44		&	23		&	49	&	40	&	90 \\
		512	&	82		&	44		&	13		&	25	&	20	&	79 \\
    \bottomrule
   \end{tabularx}

 	\begin{tablenotes}
 		\item [a] MPI-only SCF code
    	\item [b] Private Fock SCF code
    	\item [c] Shared Fock SCF code
 	\end{tablenotes}
\end{threeparttable}
\end{table}

\begin{figure}
	\includegraphics[width=\columnwidth]{Figure7}
	\caption{Scalability of the Shared Fock hybrid MPI-OpenMP version of GAMESS on the Theta machine
    		 for the 5.0~nm (i.e. large) dataset in quadrant cache mode on 3,000 \intelphireg\ processors.
             The results here are for 4~MPI ranks per node with 64~threads per rank,
             giving full saturation (in terms of hardware threads) on every \intelphireg\ node. For each point in the figure, we show the time in seconds.}
    \label{fig:5nm}
\end{figure}

\Cref{fig:5nm} shows the behavior of the shared Fock version of GAMESS for the 5~nm dataset. It is the largest dataset we could fit in memory on Theta. Since we run on 4~MPI ranks the memory footprint is approximately 208~GB per node. This figure shows good scaling of the code up to 3,000 Xeon Phi nodes, which is equal to 192,000 cores (64~cores per node).
\section{CONCLUSIONS}
In typical coverage planning tasks, with limited power supply, a single agent may have to stop the mission with a partially observed feature map. Given this background, this paper proposes a map predictive coverage planning method by leveraging the theory of \textit{Low-Rank Matrix Completion}~(LRMC). We model the urban terrain map as a maze-like map and carefully illustrate how maze-like environments fit well into the LRMC model. Then with intensive experiments, we demonstrate our proposed map predictive method significantly outperforms other coverage planning baseline methods in terms of feature map coverage convergence.

\section*{Acknowledgments}
The research leading to these results has been funded by the European Union's Horizon 2020 research and innovation program under grant agreement No. 871370 (PIMCity project) and the SmartData@PoliTO center for Data Science technologies.

\bibliographystyle{ACM-Reference-Format}
\bibliography{reference}

\newpage

\section*{Appendix}
\label{sec:appendix}

% \subsection*{Trackers on \BEFORE and \AFTER Separately for Websites with and without a Consent Banner }

% For the sake of completeness, we here show the number of Trackers per website and websites with at least one Tracker, similar to what is shown in Figure~\ref{fig:ca_country}. Unlike the previous figure, here we show separate bars for websites where \TOOL did or did not find a Consent Banner. Our goal is to show how Tracker number varies on the \BEFORE and \AFTER for those websites implementing the Consent Banner.

% The dark red bars show our measurements on the \BEFORE, only for the 57.3\% of websites where \TOOL \emph{found} a Consent Banner. The blue bars show numbers on \AFTER for the same websites. The figure allows quantifying the (sizeable) impact on tracking of accepting the Consent Banner on websites implementing one. We find that the tracking pervasiveness upon acceptance largely increases, leading to similar conclusions as in Figure~\ref{fig:ca_country}. For reference, the light red bars report the same measure for the 42.7\% of websites where \TOOL \emph{did not} find any Consent Banner.





% \begin{figure}[!h]
%     \centering
%     \begin{subfigure}[t]{0.495\columnwidth}
%         \includegraphics[width=\columnwidth]{figures/cookieaccept_websites_with_trackers_separate.pdf}
%         \caption{Percentage of websites embedding Trackers.}
%         \label{fig:ca_country_one_sep}
%     \end{subfigure}
%     \begin{subfigure}[t]{0.495\columnwidth}
%         \includegraphics[width=\columnwidth]{figures/cookieaccept_trackers_per_website_separate.pdf}
%         \caption{Average number of Trackers per website.}
%         \label{fig:ca_country_avg_sep}
%     \end{subfigure}
% 	\caption{Trackers penetration and number, 
% 	on websites during different phases of a browsing sessions (top 2\,500 websites per country). We separate websites with and without a Consent Banner. }
% 	\label{fig:ca_country_sep}
% \end{figure}

% \newpage

\subsection*{Impact of Repeated Visits on Tracking Measurements}

We here complement the analysis we carried out on the last paragraph of Section~\ref{sec:trackers_country}. Web tracking involves a number of mechanisms ( real-time bidding among all ) that result in the same page containing different Trackers on multiple visits. To obtain a reliable picture, we repeat each test 5 times. In Figure~\ref{fig:visit_nb}, we show how two macroscopic tracking measurements vary with different number of repetited visits for each website. The blue line in the figure shows the fraction of websites that contain at least one Tracker when measured with an increasing number of test repetitions. It is moderately affected by the number of tests, increasing from 69.1\% with a single repetition to 70.0\% with 5 repetitions. Similarly, the average number of Trackers, increases from $6.5$ to $7.8$.

\begin{figure}[!h]
    \centering
    \includegraphics[width=0.6\columnwidth]{figures/cookieaccept_visits_nb.pdf}
    \caption{Variation of tracker number with different numbers of repeated visits. Measurements have sizeable despite moderate variation when repeated.}
	\label{fig:visit_nb}
\end{figure}

\newpage

\subsection*{Trackers per Website (Tranco List)}

We here report the same analyses depicted in Figure~\ref{fig:tranco_tp} and Figure~\ref{fig:ca_perf_tp} showing the number of Trackers instead of the number of Third-Parties. The two pictures lead to similar conclusions.

\begin{figure}[!h]
    \centering
    \includegraphics[width=0.5\columnwidth]{figures/cookieaccept_tranco_rank_eu_tracker_nb.pdf}
    \caption{Average number of Trackers per website (Tranco list).}
    \label{fig:tranco_trackers}
\end{figure}


\begin{figure}[!h]
    \centering
    \includegraphics[width=0.5\textwidth]{figures/cookieaccept_tracker_nb_tranco.pdf}
    \caption{Distribution of the number of Trackers (Tranco list). Notice the log scales.}
    \label{fig:ca_perf_tracker}
\end{figure}

\end{document}