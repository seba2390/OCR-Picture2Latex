\newpage

\section*{Appendix}
\label{sec:appendix}

% \subsection*{Trackers on \BEFORE and \AFTER Separately for Websites with and without a Consent Banner }

% For the sake of completeness, we here show the number of Trackers per website and websites with at least one Tracker, similar to what is shown in Figure~\ref{fig:ca_country}. Unlike the previous figure, here we show separate bars for websites where \TOOL did or did not find a Consent Banner. Our goal is to show how Tracker number varies on the \BEFORE and \AFTER for those websites implementing the Consent Banner.

% The dark red bars show our measurements on the \BEFORE, only for the 57.3\% of websites where \TOOL \emph{found} a Consent Banner. The blue bars show numbers on \AFTER for the same websites. The figure allows quantifying the (sizeable) impact on tracking of accepting the Consent Banner on websites implementing one. We find that the tracking pervasiveness upon acceptance largely increases, leading to similar conclusions as in Figure~\ref{fig:ca_country}. For reference, the light red bars report the same measure for the 42.7\% of websites where \TOOL \emph{did not} find any Consent Banner.





% \begin{figure}[!h]
%     \centering
%     \begin{subfigure}[t]{0.495\columnwidth}
%         \includegraphics[width=\columnwidth]{figures/cookieaccept_websites_with_trackers_separate.pdf}
%         \caption{Percentage of websites embedding Trackers.}
%         \label{fig:ca_country_one_sep}
%     \end{subfigure}
%     \begin{subfigure}[t]{0.495\columnwidth}
%         \includegraphics[width=\columnwidth]{figures/cookieaccept_trackers_per_website_separate.pdf}
%         \caption{Average number of Trackers per website.}
%         \label{fig:ca_country_avg_sep}
%     \end{subfigure}
% 	\caption{Trackers penetration and number, 
% 	on websites during different phases of a browsing sessions (top 2\,500 websites per country). We separate websites with and without a Consent Banner. }
% 	\label{fig:ca_country_sep}
% \end{figure}

% \newpage

\subsection*{Impact of Repeated Visits on Tracking Measurements}

We here complement the analysis we carried out on the last paragraph of Section~\ref{sec:trackers_country}. Web tracking involves a number of mechanisms ( real-time bidding among all ) that result in the same page containing different Trackers on multiple visits. To obtain a reliable picture, we repeat each test 5 times. In Figure~\ref{fig:visit_nb}, we show how two macroscopic tracking measurements vary with different number of repetited visits for each website. The blue line in the figure shows the fraction of websites that contain at least one Tracker when measured with an increasing number of test repetitions. It is moderately affected by the number of tests, increasing from 69.1\% with a single repetition to 70.0\% with 5 repetitions. Similarly, the average number of Trackers, increases from $6.5$ to $7.8$.

\begin{figure}[!h]
    \centering
    \includegraphics[width=0.6\columnwidth]{figures/cookieaccept_visits_nb.pdf}
    \caption{Variation of tracker number with different numbers of repeated visits. Measurements have sizeable despite moderate variation when repeated.}
	\label{fig:visit_nb}
\end{figure}

\newpage

\subsection*{Trackers per Website (Tranco List)}

We here report the same analyses depicted in Figure~\ref{fig:tranco_tp} and Figure~\ref{fig:ca_perf_tp} showing the number of Trackers instead of the number of Third-Parties. The two pictures lead to similar conclusions.

\begin{figure}[!h]
    \centering
    \includegraphics[width=0.5\columnwidth]{figures/cookieaccept_tranco_rank_eu_tracker_nb.pdf}
    \caption{Average number of Trackers per website (Tranco list).}
    \label{fig:tranco_trackers}
\end{figure}


\begin{figure}[!h]
    \centering
    \includegraphics[width=0.5\textwidth]{figures/cookieaccept_tracker_nb_tranco.pdf}
    \caption{Distribution of the number of Trackers (Tranco list). Notice the log scales.}
    \label{fig:ca_perf_tracker}
\end{figure}