\documentclass[proof]{WileyASNA-v1}

\articletype{Original Article}%

\received{26 April 2016}
\revised{6 June 2016}
\accepted{6 June 2016}

\raggedbottom

\begin{document}

\title{Extremely inverted peaked spectrum radio sources} %\protect\thanks{This is an example for title footnote.}}

\author[1]{Mukul Mhaskey*}

\author[2]{Surajit Paul}

\author[3]{Gopal Krishna}

\authormark{MHASKEY \textsc{et al}}


\address[1]{\orgname{Th\"uringer Landessternwarte}, \orgaddress{\state{Sternwarte 5, D-07778 Tautenburg}, \country{Germany}}}

\address[2]{\orgdiv{Department of Physics}, \orgname{Savitribai Phule Pune University}, \orgaddress{\state{Ganeshkhind, Pune 411007}, \country{India}}}

\address[3]{\orgname{ UM-DAE Centre for Excellence in Basic Sciences (CEBS)}, \orgaddress{\state{University of Mumbai Campus, Vidya Nagari, Mumbai 400098}, \country{India}}}

\corres{*Mhaskey Mukul, Th\"uringer Landessternwarte, Tautenburg, Germany. \email{mukul@tls-tautenburg.de}}

%\presentaddress{This is sample for present address text this is sample for present address text}

\abstract{We report our ongoing search for extremely inverted spectrum compact radio galaxies, for which the defining feature in the radio spectrum is not the spectral peak, but instead the slope of the spectrum (alpha) in the high-opacity (i.e., lower frequency) part of the radio spectrum. Specifically, our focus is on the spectral regime with spectral index, $\alpha_{thick} > $+2.5. The motivation for our study is, firstly, extragalactic sources with such extreme spectral index are extremely rare, because of the unavailability of right combination of sensitivity and resolution over a range of low frequencies. The second reason is more physically motivated, since alpha = +2.5 is the maximum slope theoretically possible for a standard radio source emitting synchrotron radiation. Therefore such sources could be the test-bed for some already proposed alternative scenarios for synchrotron self-absorption (SSA), like the free-free absorption (FFA) highlighting the importance of jet-ISM interaction in the radio galaxy evolution.} 

%A detailed comparison of the two low frequency surveys TGSS-ADR1 (150 MHz) and WENSS/WISH (325 MHz) undertaken with the GMRT and Westerbork telescope respectively, led us to shortlist 35 candidates with $\alpha^{325 MHz}_{150 MHz} >$\,+2.5. We carried out quasi-simultaneous observations of all the candidate sources with the upgraded GMRT, at frequencies in the range 150 $-$ 500 MHz (for some sources, upto 900 MHz), to rule out variability, flux scale errors and other possible issues that would arise because the two surveys were conducted using two different telescope arrays at different resolutions and almost a decade apart from each other. Quasi-simultaneous observations done within a week help us ruling out any variability between the range of frequencies on which the spectral index is based. We report 6 sources clearly above $\alpha^{325 MHz}_{150 MHz} >$\,+2.5 and 10 sources have a spectral index above +2.0. As mentioned above these sources violate the SSA limit and it is shown that their spectra can be very well reproduced in terms of free-free absorption (FFA) occurring in an inhomogeneous external screen of thermal plasma, taking reasonable values for the input parameters.}

\keywords{AGN, jets, radio continuum, radio galaxies}

\jnlcitation{\cname{%
\author{Mukul Mhaskey}, 
\author{Surajit Paul}, and 
\author{Gopal-Krishna}} (\cyear{2021}), 
\ctitle{Extremely inverted peaked spectrum radio sources}, \cjournal{AN}, 
\cvol{2021;00:x--x}.}

%%\fundingInfo{Funding info text.}

\maketitle

%\footnotetext{\textbf{Abbreviations:} ANA, anti-nuclear antibodies; APC, antigen-presenting cells; IRF, interferon regulatory factor}


\section{Introduction}\label{sec1}
Gigahertz peaked spectrum (GPS) and compact steep spectrum (CSS) sources are compact radio-loud active galaxies with linear sizes of a few kilo-parsecs. They have a characteristic inverted radio-continuum spectrum which is a result of absorption of the synchrotron radiation at giga-hertz ($\nu_{peak} < 1.0$ GHz, GPS) and sub-GHz ($\nu_{peak} < 0.5$ GHz, CSS) frequencies, caused either by synchrotron self absorption (SSA) or free-free absorption (FFA), or a combination of both \citep[e.g.,][]{Odea1998}. GPS/CSS sources offer a unique opportunity to study the AGN-host galaxy feedback process because the radio emission is embedded within their host galaxies. Furthermore, there is evidence to suggest that these sources are associated with new or recent AGN activity, and thus are ideal candidates to study jet evolution and galaxy formation \citep{Odea1998, Odea2021}. This has however been contested due to the large numbers of GPS/CSS sources found in radio surveys and favours the alternative scenario in which the radio emission is associated with an old AGN activity and the emission cannot escape the host galaxy due to unusually dense environment. The absorption mechanisms that give rise to the inverted spectrum can resolve this ambiguity about the nature of the GPS/CSS sources since several physical factors in these sources govern the absorption mechanism. But these absorption mechanisms are still poorly understood because of lack of sufficient flux measurements to sample the spectrum %, especially in the optically thick part of the spectrum 
\citep{Odea1998, Callingham2017}.

In an attempt to understand the absorption mechanism in such sources and study the environmental interactions between AGN jets and the host galaxy ISM, we started a search program with the GMRT to look for inverted spectrum radio sources that have a spectral index, $\alpha^{\rm 325\, MHz}_{\rm 150\, MHz}$ $>+$2.5 ($S\propto\nu^{-\alpha}$). 
This is larger than the theoretically allowed spectral index limit for SSA in a perfectly homogeneous source which emits synchrotron radiation. So far, we have found 12 sources with $\alpha$ near $+$2.5 We have named such sources as `Extremely Inverted Spectrum Extragalactic Radio Sources' \citep[EISERS;][]{Mhaskey2019a,Mhaskey2019b, Mhaskey2020}.



%\begin{figure*}[t]
%\centerline{\includegraphics[width=342pt,height=9pc,draft]{empty}}
%\caption{This is the sample figure caption.\label{fig1}}
%\end{figure*}



\section{Sample}\label{sec3}
The two main radio surveys, viz., the TGSS-ADR1 (150 MHz) and WENSS (325 MHz), overlap a region of about 1.03$\pi$ steradian ($\sim$1/4$^{th}$ of the entire sky). The selected overlapping region contains 229420 radio sources in the WENSS catalogue covering the declination north of +28$^\circ$. Out of this extensive source list, we first extracted a subset of 35064 sources which belong to the morphological type \lq S\rq~(i.e., single, as per the WENSS catalogue) and are stronger than 150 mJy at 325 MHz. For each of these shortlisted WENSS sources, we then looked for a TGSS-ADR1 counterpart, within a search radius of 20 arcsec. We found counterparts for 33707 (out of 35064) of the WENSS sources. For 1357 WENSS sources no counterparts in TGSS-ADR1 were found. %The separation between the positions of sources from the WENSS and TGSS-ADR1 is found to have a median value of 2.10$\pm$0.02 arcsec, which is consistent with the quoted positional uncertainties for the relatively weak sources in the two surveys (rms $\sim$ 1 arcsec for WENSS and $\sim$ 2 arcsec for TGSS-ADR1). 
The spectral index, $\alpha$, between 150 and 325 MHz, for all the sources was calculated. For the WENSS sources without a counterpart (1357) in the TGSS-ADR1, a 5-$\sigma$ limit to the flux at 150 MHz was used to estimate the upper limits for the spectral index. Here $\sigma$ is the image rms for individual sources in TGSS-ADR1.    

The final list of EISERS candidates satisfy the following criteria: (i) structural type \lq S\rq~and flux density $>$ 150 mJy at 325 MHz, (ii) omission of sources found to lie within 10$^{\circ}$ of the galactic plane, or listed as H-II regions in the NASA Extragalactic Database (NED)\footnote{https://ned.ipac.caltech.edu/} (iii) a clean detection in the WENSS (325 MHz) and NVSS (1.4 GHz) (based on visual inspection of the respective radio images) and (iv) $\alpha^{325\;MHz}_{150\;MHz} >$ $+$2.75, in the case of sources detected in TGSS-ADR1 or $\alpha^{325\;MHz}_{150\;MHz} >$ $+$2.5, in the case of sources not-detected in TGSS-ADR1. A conservative threshold value of +2.75 was adopted for sources detected at 150 MHz in order to make an allowance for flux variability. This could be significant in the case of compact sources \citep{Bell2018,Chhetri2018}. The final list contains 29 EISERS candidates which were later observed quasi-simultaneous in band 2, 3 and 4 with the uGMRT.


%\begin{center}
%\begin{table*}[t]%
%\caption{This is sample table caption.\label{tab1}}
%\centering
%\begin{tabular*}{500pt}{@{\extracolsep\fill}lccD{.}{.}{3}c@{\extracolsep\fill}}
%\toprule
%&\multicolumn{2}{@{}c@{}}{\textbf{Spanned heading\tnote{1}}} & \multicolumn{2}{@{}c@{}}{\textbf{Spanned heading\tnote{2}}} \\\cmidrule{2-3}\cmidrule{4-5}
%\textbf{col1 head} & \textbf{col2 head}  & \textbf{col3 head}  & \multicolumn{1}{@{}l@{}}{\textbf{col4 head}}  & \textbf{col5 head}   \\
%\midrule
%col1 text & col2 text  & col3 text  & 12.34  & col5 text\tnote{1}   \\
%col1 text & col2 text  & col3 text  & 1.62  & col5 text\tnote{2}   \\
%col1 text & col2 text  & col3 text  & 51.809  & col5 text   \\
%\bottomrule
%\end{tabular*}
%\begin{tablenotes}%%[341pt]
%\item Source: Example for table source text.
%\item[1] Example for a first table footnote.
%\item[2] Example for a second table footnote.
%\end{tablenotes}
%\end{table*}
%\end{center}

%\begin{center}
%begin{table}[t]%
%\centering
%\caption{This is sample table caption.\label{tab2}}%
%\tabcolsep=0pt%
%\begin{tabular*}{20pc}{@{\extracolsep\fill}lcccc@{\extracolsep\fill}}
%%\toprule
%\%textbf{col1 head} & \textbf{col2 head}  & \textbf{col3 head}  & \textbf{col4 head}  & \textbf{col5 head} \\
%\midrule
%col1 text & col2 text  & col3 text  & col4 text  & col5 text\tnote{$\dagger$}   \\
%col1 text & col2 text  & col3 text  & col4 text  & col5 text   \\
%col1 text & col2 text  & col3 text  & col4 text  & col5 text\tnote{$\ddagger$}   \\
%\bottomrule
%\end{tabular*}
%\begin{tablenotes}
%\item Source: Example for table source text.
%\item[$\dagger$] Example for a first table footnote.
%\item[$\ddagger$] Example for a second table footnote.
%\end{tablenotes}
%\end{table}
%\end{center}



\section{Observations and data analysis}\label{sec4}
The need for quasi-simultaneous radio observations at frequencies below the spectral turnover is underscored by the fact that the two radio surveys (TGSS-ADR1 and WENSS) used for computing the spectral slopes of these compact radio sources had been made nearly a decade apart. The long time interval could then have introduced significant uncertainty due to flux variability expected from refractive interstellar scintillation at such low frequencies, e.g. \citep{Bell2018}. It may also be noted that the WENSS \citep{Rengelink1997} is known to be off the flux-density scales defined by Roger, Costain \& Bridle \citep[RCB,][]{Roger1973} and by \citet{Baars1977} by over 10\% \citep[see,][]{Hardcastle2016}. For some areas in the sky the TGSS-ADR1 reports systematically low fluxes, sometimes even by 40-50\% , also $\sim$5-10\% of the survey area has systematic flux deviations of $>$10\% \citep{Intema2017}. Therefore, the previous spectral index estimates could be substantially inaccurate due to the combined effect of measurement uncertainty and the calibration uncertainties of the WENSS and TGSS-ADR1 flux densities. This may account for the substantial differences found in some of the cases, between the WENSS flux density at 325 MHz and the present uGMRT measurements at the same frequency. For our sample of EISERS candidates, the present GMRT observations have yielded the data with the highest sensitivity and resolution currently available at 150 and 325 MHz. Moreover, the availability of two data points at well-spaced frequencies in their highly opaque spectral region raises the confidence in quantifying the steepness of the spectral turnover. 

The 29 EISERS candidates were observed with the recently upgraded GMRT \citep[`uGMRT',][]{Gupta2017}. The observations were performed in a snapshot mode, quasi-simultaneously at bands 2, 3 and 4, using the wide band receivers covering a frequency range from 150 to 900 MHz. The wide band of the uGMRT was divided into several smaller sub-bands and then imaged separately. The measured visibilities at 150 MHz and 325 MHz were processed using the Source Peeling and Atmospheric Modelling \citep[\textsc{spam}][]{Intema2014} package. \textsc{spam} is a semi-automated pipeline based on \textsc{aips}, \textsc{parseltongue} and \textsc{python}. It performs a series of iterative flagging and calibration sequences and the imaging involves direction dependent calibration. This package has been used for processing of the entire TGSS data at 150 MHz \citep{Intema2017}. Details of \textsc{spam} and its various routines are provided in \citet{Intema2017}. Thus several flux measurements between 150 and 900 MHz were obtained (see Figure 1). 

%\subsection{Radio Observations}
%The 15 EISERS candidates were observed with the recently upgraded GMRT \citep[`uGMRT',][]{Gupta2017}. The observations were performed in a snapshot mode, quasi-simultaneously at bands 2, 3 and 4, using the wide band receiver covering a frequency range from 150 to 900 MHz.
%(Table~\ref{table:obs-log}). The integration time was 8 sec at both 150 MHz and 325 MHz.
% and the bandwidths used are 16 MHz and 32 MHz, respectively. 
%One of the standard flux-density calibrators, 3C 286, 3C 48, and 3C 147, was observed at the start and at the end of each observing session at 150 MHz. At 325 MHz, the flux-density calibrator was observed only at the start of the session. Phase calibrator(s) were observed immediately before and after each snapshot on the target source. The average total on-target time depended on the frequency. Since the target sources have an inverted spectrum, they are stronger at 325 MHz than at 150 MHz. Total observing time for each target source was about 60 minutes at 150 MHz and about 15 minutes at 325 MHz. Further observational details are provided in the log (Table~\ref{table:obs-log}). Flux-densities at different frequencies are listed in 
%Table~\ref{table:spec-prop}, including those measured in the present uGMRT observations at 150 MHz and 325 MHz. The corresponding radio contour maps are provided in the Section 3.6.
%online material, except for the maps of the best two cases of EISERS found in this study, namely J1326$+$5712 and J1658$+$473. These are presented in Figures~\ref{fig:J1326} \&~\ref{fig:J1658}.  

%\vspace*{-5mm}
%\subsection{Analysis}\label{analysis}
%The measured visibilities at 150 MHz and 325 MHz were processed using the Source Peeling and Atmospheric Modelling \citep[\textsc{spam}][]{Intema2014} package. \textsc{spam} is a semi-automated pipeline based on \textsc{aips}, \textsc{parseltongue} and \textsc{python}. It performs a series of iterative flagging and calibration sequences and the imaging involves direction dependent calibration. This package has been used for processing of the entire TGSS data at 150 MHz \citep{Intema2017}. Details of \textsc{spam} and its various routines are provided in \citet{Intema2017}. 

%Equation~\ref{eq:error} defines the rms uncertainty of the measured flux-density of a source, the first term is the root mean square of the fitting error in the \textsc{aips} task \textsc{jmfit} and the second term is the systematic error component (10\%) which includes the error arising from the elevation dependent gains of the antennas \citep{Chandra2004}. 
%\begin{equation} \label{eq:error}
%\sqrt{ \rm{(map~rms)}^{2} + (10 \%~\rm{of~the~peak~flux})^{2}}
%\end{equation}

Out of the 29 candidates observed 12 have a very steep spectral index with $\alpha^{325\;MHz}_{150\;MHz} >$ $+$2.0. In Table 1 the GMRT flux densities found here at 150 and 325 MHz are listed together with the measured spectral index. This is the final list of confirmed EISERS. %Based on these, the radio spectra of the 16 sources are displayed in Figure~\ref{fig:chap4-spec_all}.

%\begin{landscape}
\begin{table}[h!]
\centering
%\small
%\addtolength{\tabcolsep}{-1pt}
%\vspace{-0.5cm}
\caption{Source name, flux densities and spectral indices (150-325 MHz, uGMRT) of the extremely inverted spectrum sources.}
\begin{tabular}{cccc}
\hline

\multicolumn{1}{c}{Source}& \multicolumn{2}{c}{uGMRT flux density (mJy)} & \multicolumn{1}{c}{Spectral index} \\
%(J2000)& uGMRT$^{\dagger}$ & uGMRT$^{\dagger}$ & FIRST$^{\dagger}$ & AG-VLBI$^{\dagger}$ &87GB$^{\dagger}$& CLASS$^{\dagger}$ & (150-325 MHz) \\
  &  150 MHz & 325 MHz & $\alpha_{\rm 150\,MHz}^{\rm 325\,MHz}$\\
\hline
%J0242$-$1649 & 06.4${\pm}$1.4&062.7${\pm}$06.3 &2.95${\pm}$0.3\\
J0304+7727 & 32.2${\pm}3.7$ &167.2${\pm}$16.7 &2.1${\pm}$0.2\\
J0614+6108 &29.1${\pm}$5.1 &146.9${\pm}$14.7 &2.1${\pm}$0.3\\
J0649+5947 &21.4${\pm}$4.3 &128.1${\pm}$12.8 &2.3${\pm}$0.3\\
%J0730+3304 & 07	30	41.47 & 33 04 56.4 & 6.9${\pm}$0.7$^{\dagger}$ &57${\pm}$5.7$^{\dagger}$ & $-$ & 54.4${\pm}$1.7$^{\star}$ & 2.65${\pm}$0.4\\
J0740+7129 &12.2${\pm}$2.8 &095.6${\pm}$09.6 &2.7${\pm}$0.3\\
%J0804+4725 & 08 04 37.5 & +47 25 48.5 & 15.9${\pm}$1.6$^{\dagger}$ &108${\pm}$10.9$^{\ddagger}$ & $-$ & 135.4${\pm}$4.1$^{\star}$ & 2.27${\pm}$0.8\\
J0847+5723 &29.1${\pm}$4.8 &180.9${\pm}$18.1 &2.4${\pm}$0.3\\
J0853+6722 &17.5${\pm}$3.1 &087.7${\pm}$08.8 &2.1${\pm}$0.3\\
J0858+7501 &25.4${\pm}$6.1 &162.2${\pm}$16.2 &2.4${\pm}$0.3\\
%J1003$-$2514 &09.4${\pm}$3.5 &047.9${\pm}$04.8 &2.11${\pm}$0.5\\
%J1031$-$2228 &22.8${\pm}$3.6 &107.1${\pm}$10.7 &2.00${\pm}$0.2\\
%J1209$-$2032 &$<$19.5 &149.4${\pm}$14.9 &$>$2.64\\
J1326+5712 &11.5${\pm}$1.4 &108.9${\pm}$10.9 &2.9${\pm}$0.2\\
J1536+8154 &35.1${\pm}$5.9 &198.9${\pm}$19.9 &2.2${\pm}$0.3\\ 
J1549+5038 & 44.3${\pm}$6.2&287.8${\pm}$28.8 &2.4${\pm}$0.2\\
J1658+4732 &15.4${\pm}$2.8 &163.4${\pm}$16.4 &3.0${\pm}$0.3\\
J2317+4738 &29.7${\pm}$3.7 &145.0${\pm}$14.5 &2.1${\pm}$0.2\\
\hline
\end{tabular}
\label{table:spec-prop}
%{$^{\dagger}$ LOFAR-DR2--\citet{Intema2017}; FIRST -- \citet{Helfand2015}; AG-VLBI -- Astrogeo VLBI FITS Image Database; 87GB -- \citet{Gregory1991}; CLASS -- \citet{Myers2003}; NED -- NASA Extragalactic Database.}\\
%$^{\dagger}$ Flux density measurement from the LoTSS-DR2\\
%$^{\dagger}$ Flux density measurement from the WENSS \citep{DeBreuck2002}\\
\end{table}

%\begin{sidewaystable}%[h]
%\caption{Sideways table caption. For decimal alignment refer column 4 to 9 in tabular* preamble.\label{tab3}}%
%\begin{tabular*}{\textheight}{@{\extracolsep\fill}lccD{.}{.}{4}D{.}{.}{4}D{.}{.}{4}D{.}{.}{4}D{.}{.}{4}D{.}{.}{4}@{\extracolsep\fill}}%
%\toprule
%  & \textbf{col2 head} & \textbf{col3 head} & \multicolumn{1}{c}{\textbf{10}} &\multicolumn{1}{c}{\textbf{20}} &\multicolumn{1}{c}{\textbf{30}} &\multicolumn{1}{c}{\textbf{10}} &\multicolumn{1}{c}{\textbf{20}} &\multicolumn{1}{c}{\textbf{30}} \\
%\midrule
%  &col2 text &col3 text &0.7568&1.0530&1.2642&0.9919&1.3541&1.6108 \\
%  & &col2 text &12.5701 &19.6603&25.6809&18.0689&28.4865&37.3011 \\
%3 &col2 text  & col3 text &0.7426&1.0393&1.2507&0.9095&1.2524&1.4958 \\
%  & &col3 text &12.8008&19.9620&26.0324&16.6347&26.0843&34.0765 \\
%  & col2 text & col3 text &0.7285&1.0257&1.2374&0.8195&1.1407&1.3691\tnote{*} \\
%  & & col3 text &13.0360&20.2690&26.3895&15.0812&23.4932&30.6060\tnote{\dagger} \\
%\bottomrule
%\end{tabular*}
%\begin{tablenotes}%%[\textheight]
%\item[*] First sideways table footnote. Sideways table footnote. Sideways table footnote. Sideways table footnote.
%\item[$\dagger$] Second sideways table footnote. Sideways table footnote. Sideways table footnote. Sideways table footnote.
%\end{tablenotes}
%\end{sidewaystable}

%\begin{sidewaysfigure}
%\centerline{\includegraphics[width=542pt,height=9pc,draft]{empty}}
%\caption{Sideways figure caption. Sideways figure caption. Sideways figure caption. Sideways figure caption. Sideways figure caption. Sideways figure caption.\label{fig3}}
%\end{sidewaysfigure}


\section{Discussion and Conclusion}\label{sec5}
%The work presented in this thesis is an attempt to find additional class of luminous compact radio sources, for which the defining feature in the radio spectrum is not the spectral peak, but instead the slope ($\alpha$) of the spectrum in the high-opacity (i.e., lower frequency) part of the radio spectrum. Specifically, our focus was on the the spectral regime with $\alpha_{thick}$ $>$ +2.5. Firstly, extragalactic sources with such extreme $\alpha$ had not yet been found, probably because of the unavailability of right combination of sensitivity and resolution over a range of low frequencies. The second reason was more physically motivated, since $\alpha$ = +2.5 is the maximum slope theoretically possible for a standard nonthermal radio source, believed to radiate incoherent synchrotron radiation from a relativistic particle population accelerated with a single power-law energy distribution \citep[e.g.,][]{Slish1963,Kellermann1969}. Thus, discovery of extragalactic radio sources with $\alpha_{thick}$ $>$ $\alpha_{c}$ = +2.5 would be the first step towards finding sources of non-standard synchrotron radiation. Note that such extreme spectra can be produced (over a limited frequency range) by an electron energy distribution which has a Maxwellian shape, or a delta function \citep{Rees1967}. An energy distribution in which the number of low-energy electrons is over represented relative to a power-law extrapolation from the distribution at higher energies, is another possible explanation \citep{deKool1989}. While these radical alternatives can exist, the conventional interpretation in terms of free-free-absorption (FFA) is always a viable candidate. However, until the first publication from our study in 2014, FFA effects (leading to $\alpha_{thick}$ $>$ +2.5) had been observed in a just few radio galaxies, and that too for only their parsec-scale nuclear radio jets. Prominent such examples are the well-known radio source 3C 345 \citep{Matveenko1990}, Centaurus A \citep{Jones1996,Tingay2001}, Cygnus A \citep{Krichbaum1998}, NGC 1275/Perseus A \citep{Levinson1995,Walker2000}, NGC 4261 \citep{Jones2001} and NGC 1052 \citep{Vermeulen2003, Kadler2004}. More recntly, \citet{Callingham2015} reported discovery of an \lq extreme GPS source\rq~PKS B0008$-$421, based on an exceptionally dense spectral sampling between 0.1 and 22 GHz, covering comprehensively both sides of the single spectral peak seen near 0.6 GHz. Strikingly, its spectral slope below the turnover frequency is seen to become as large as +2.4, closely approaching the SSA limit of $\alpha_{c}$ = +2.5.
The  main  objective of this project is to understand the jet-ISM environment  in  GPS/CSS  sources by  probing  the synchrotron radiation absorption medium and to distinguish between the SSA and FFA mechanisms. For this we concentrate our efforts on sources with extremely inverted radio spectra, since they are the sites of maximum synchrotron radiation opacity.
We conducted a first systematic search for the rare sub-class of inverted spectrum extragalactic radio sources which show an inverted (integrated) radio spectrum with a spectral index $\alpha >$ +2.0. %Such steeply inverted radio spectra cannot be explained with the standard synchrotron emission model of extragalactic radio sources, wherein the electrons are accelerated to a power-law energy distribution and thus radiating incoherent synchrotron emission. Therefore, such sources, if found, could be the test-bed for alternative scenarios for synchrotron self-absorption (SSA), involving free-free absorption (FFA). 

These sources seem to be particularly interesting from the viewpoint of violation of the theoretical SSA limit in compact extragalactic radio sources and may thus require alternative explanations. %The main objective of this project is to identify the absorption mechanism responsible for the ultra-steep spectral turnover ($\alpha$ $>$ + 2.5, for EISERS). 
One possibility is that the energy spectrum of the relativistic particles in EISERS deviates from the canonical shape (a power-law), to a Maxwellian or mono-energetic particle energy distribution \citep{Rees1967}.

\begin{figure}
%\vspace{-2.0cm}
%\hspace{-2.3cm}
\begin{center}
\includegraphics[trim={1.75cm 0cm 2.0cm 0.5cm},clip,width=0.45\textwidth]{J1326+5712_new.eps}
%\includegraphics[trim={1.5cm 0cm 2.0cm 0cm},clip,width=0.49\textwidth]{J0649+5947_new.eps}
%\includegraphics[trim={1.5cm 0cm 2.0cm 0cm},clip,width=0.49\textwidth]{J0740+7129_new.eps}
%\includegraphics[trim={1.5cm 0cm 2.0cm 0cm},clip,width=0.49\textwidth]{J0853+6722_new.eps}
\end{center}
%\vspace{-1.0cm}
\caption{\scriptsize{Radio spectrum of J1326+5712. The filled red circles represent the measurements from the uGMRT observations at the multiple narrow frequency sub-bands in band 2, band 3 and band 4. The shaded regions represents the range of frequencies for different bands observable by the uGMRT. The SSA and FFA absorption models are represented by the yellow and black curves respectively.}} \label{fig:spec_all} 
\end{figure}

An alternative explanation for EISERS invokes free-free absorption (FFA) \citep{Kellermann1966, Bicknell1997, Kuncic1998}. As mentioned above, EISERS violate the standard synchrotron picture. We have modelled the radio spectra of the confirmed EISERS in terms of an inhomeogeneous FFA screen external to the radio source, following the prescription of \citet{Bicknell2018}. Earlier, they had suggested that such a thermal screen can arise, as the bow shock produced by the advancing jet photo-ionises the thermal gas clouds filling the ambient (interstellar) space \citep{Bicknell1997}. A satisfactory fit to the radio spectrum was found for a majority of the confirmed EISERS. This is demonstrated for the source J1326+5712 ($\alpha^{325\;MHz}_{150\;MHz} =$ $+$2.91$\pm$0.2) in Figure 1, where the curve representing the the FFA model is a better fit than the curve for the SSA model. Further results are published in \citet{Mhaskey2019b}. The derived estimates of the mean electron density within the absorbing gas were found to match well with the densities predicted in the theoretical models for jet propagation in the ionised medium, by \citet{Begelman1996} and \citet{Bicknell1997}. Any evidence for high electron densities \citep[n$_{e}  >$100\,cm$^{-3}$,][]{Bicknell2018} in EISERS would be consistent with the FFA explanation. 

%It was noted in our recent paper, \citet{Mhaskey2019b}, that any evidence for high electron densities (n$_{e}$) in EISERS would be consistent with the FFA explanation. Potential tracers of n$_{e}$ are Rotation Measure (RM) or HI gas. Although the cool neutral hydrogen gas does not directly trace the ionised medium required for FFA, it acts as a good proxy for the over-density of ISM \citep{Pihlstrom2003, Morganti2018}. We have therefore, recently used uGMRT to make a sensitive search for the 21 cm HI absorption in both EISERS for which spectroscopic redshifts were available. A strong absorption line was detected is the spectrum of the EISERS J1209$-$2032 ($\alpha_{thick}$ $>$2.64) which is identified with a red galaxy at z = 0.404. Its HI column density of 34.8$\pm$2.9 $\times$10$^{20}$ cm$^{-2}$ is among the highest found for compact radio source and is consistent with the expectation from the FFA explanation for EISERS. The other EISERS J1549+5038 ($\alpha_{thick}$ $=$ 2.42$\pm$0.22), a z=2.174 quasars, did not show HI absorption, leading to an upper limit on the HI column density ($<$1.4$\times$10$^{20}$ cm$^{-2}$). The non detection is probably related to its very high UV luminosity which could have ionised most of the associated HI, as suggested by \citet{Curran2008}. Nonetheless, this quasar appears to have a high rotation measure of 1400$\pm$500 rad m$^{-2}$, which would be consistent with the FFA scenario for EISERS \citep{Mhaskey2020}.

Also, the simulations of relativistic jets interacting with a warm, inhomogeneous medium \citep{Mukherjee2016,Bicknell2018} show that free-free absorption can account for the $\sim$GHz peak frequencies and low frequency power-laws inferred from the radio observations. %These simulations utilize cloud densities and velocity dispersion in the range derived from optical observations. 
At early stages, the low frequency spectrum is steep but progressively flattens as a result of a broader distribution of optical depths. %This indicates that the steep low frequency spectra discovered in case of EISERS may be attributed to young sources, which later evolve into AGNs. 
Since EISERS have the steepest of slopes below the turnover in the optically thick regime, it can be assumed that these sources are very young AGNs and would eventually evolve into normal radio galaxies.

Our initial results indicate that the inhomogeneous-FFA model by \citep{Bicknell1997} is  typically preferred  over  single,  homogeneous SSA and FFA absorbing media.  To pin down the absorption mechanism with high statistical  confidence,  more  data  points  both  in  the  optically thick and thin part of the spectrum below and above the turnover frequency are crucial.  In order to  assess  the  impact  an  inhomogeneous  absorbing medium has on the absorbed spectrum, we are numerically building a new model which incorporates both SSA and FFA mechanisms and will compare them with our observations. 
%\backmatter

\section*{Acknowledgments}

We thank the staff of the GMRT who have made these observations possible. GMRT is run by the National Centre for Radio Astrophysics of the Tata Institute of Fundamental Research. This research has used NASA's Astrophysics Data System and NASA/IPAC Extragalactic Database (NED), Jet Propulsion Laboratory, California Institute of Technology under contract with National Aeronautics and Space Administration and VizieR catalogue access tool, CDS, Strasbourg, France. SP would like to thank DST INSPIRE Faculty Scheme (IF12/PH-44) for funding his research group. G-K acknowledges the Senior Scientist Fellowship of the Indian National Science Academy.

%This work was performed under the auspices of the \fundingAgency{National Nuclear Security Administration} of the US Department of Energy at Los Alamos National Laboratory under Contract No. \fundingNumber{DE-AC52-06NA25396} and supported. The authors acknowledge the partial support of the \fundingAgency{DOE Office of Science ASCR Program} under Contract No. \fundingNumber{SJS-AC52-06NA25968}. This work was partially supported by the \fundingAgency{Czech Technical University} grant \fundingNumber{SGS16/247/OHK4/3T/14}, the \fundingAgency{Czech Science Foundation} project \fundingNumber{14-21318S} and by the \fundingAgency{Czech Ministry of Education} project \fundingNumber{RVO 68407700}.

%\subsection*{Author contributions}
%This is an author contribution text. This is an author contribution text. This is an author contribution text.  

%\subsection*{Financial disclosure}
%None reported.

%\subsection*{Conflict of interest}
%The authors declare no potential conflict of interests.


%\section*{Supporting information}

%The following supporting information is available as part of the online article:

%\noindent
%\textbf{Figure S1.}
%{500{\uns}hPa geopotential anomalies for GC2C calculated against the ERA Interim reanalysis. The period is 1989--2008.}

%\noindent
%\textbf{Figure S2.}
%{The SST anomalies for GC2C calculated against the observations (OIsst).}


%\appendix

%\section{Section title of first appendix\label{app1}}




%\subsection{Subsection title of first appendix\label{app1.1a}}


%\subsubsection{Subsection title of first appendix\label{app1.1.1a}}

%\begin{center}
%\includegraphics[width=7pc,height=8pc,draft]{empty}
%\end{center}


%\section{Section title of second appendix\label{app2}}%



%== Figure 4 ==
%% Example for figure inside appendix
%\begin{figure}[t]
%\centerline{\includegraphics[height=10pc,width=78mm,draft]{empty}}
%\caption{This is an example for appendix figure.\label{fig5}}
%\end{figure}

%\subsection{Subsection title of second appendix\label{app2.1a}}



%\subsubsection{Subsection title of second appendix\label{app2.1.1a}}



%\begin{center}
%\begin{table}[t]%
%\centering
%\caption{This is an example of Appendix table showing food requirements of army, navy and airforce.\label{tab4}}%
%\begin{tabular*}{20pc}{@{\extracolsep\fill}lcc@{\extracolsep\fill}}%
%\toprule
%\textbf{col1 head} & \textbf{col2 head} & \textbf{col3 head} \\
%\midrule
%col1 text & col2 text & col3 text \\
%col1 text & col2 text & col3 text \\
%col1 text & col2 text & col3 text\\
%\bottomrule
%\end{tabular*}
%\end{table}
%\end{center}

%\nocite{*}% Show all bib entries - both cited and uncited; comment this line to view only cited bib entries;
\bibliography{Wiley-ASNA}%

\section*{Author Biography}
\begin{biography}{}{\textbf{Mukul Mhaskey} obtained his PhD from Savitribai Phule Pune University, Pune, India in Dec, 2020 and is currently a postdoctoral researcher at the Th\"uringer Landessternwarte, Tautenburg, Germany. His main researchinterests are multi-wavelength studies of radio-emitting active galactic nuclei to understand their evolution and the feedback with host galaxies.}
\end{biography}
\end{document}
