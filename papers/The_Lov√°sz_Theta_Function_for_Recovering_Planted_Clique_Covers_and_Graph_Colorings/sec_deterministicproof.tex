
\subsection{Proof of Result \ref{thm:result2}} 

\begin{theorem} \label{thm:abcd-exact-recovery}
Suppose $G$ satisfies the $\cscc$ property for some $c < \min \{ \frac{1}{4} (\frac{\min_l 1/|\ccs_l|}{\sum_l 1/|\ccs_l|})^2,\frac{1}{100} \}$.  Then $\ac$ is the unique optimal solution to \eqref{eq:lovasz_lambdamax}.
\end{theorem}

Following Proposition \ref{thm:Unequalsizecanonicalmatrix_spectrum}, the largest singular value of $\zz$ is $\sigma_{\max}(\zz) = \sum_l 1/|\ccs_l|$, while the smallest non-zero singular value of $\zz$ is $\sigma_{\ks}(\zz) = \min 1/|\ccs_l|$.  Hence an alternative way to express the lower bound in Theorem \ref{thm:abcd-exact-recovery} in terms of a condition number-type of quantity 
$$\frac{ \min_l 1/|\ccs_l| }{ \sum_l 1/|\ccs_l| }= \frac{ \sigma_{\ks}(\zz) }{ \sigma_{\max}(\zz) }.$$

\begin{proof}
As a reminder, our candidate dual certificate is $\zc = P_{\tilde{\mathcal{K}} \cap\tilde{\mathcal{L}}} (\zz)$ as defined earlier in \eqref{eq:euclideanprojection}.  By Theorem \ref{thm:extremepoint}, $\ac$ is an extreme point of the feasible region.  Thus, by  Theorem \ref{thm:constraintqualification}, we need to show that $\zc$ satisfies conditions \eqref{cond1}--\eqref{cond3}.  Conditions \eqref{eq:supportcondition} and \eqref{eq:complementaryslackness} are satisfied by construction.  As we noted in the proof of Theorem \ref{thm:disjointcliques_uniquerecovery}, conditions \eqref{eq:traceone} and   \eqref{eq:objvalks} are taken care of by scaling.  As such, it remains to show that $Z'$ is PSD and that ${\rm range}(Z')={\rm ker}(\xc)$.

%Then $Z_{\mathcal{C}} := P_{\mathcal{H}} (\zz)$ satisfies (i) $Z_{\mathcal{C}} \succeq 0$, (ii) \eqref{eq:complementaryslackness}, (iii) \eqref{eq:supportcondition}, and $\mathrm{rank}(Z_{\mathcal{C}}) = |V| - (k-1)$.

\paragraph{Simplification:}  We begin by showing that it suffices to prove the inequality $\sigma_{\max}(\zc - \zz) < \sigma_{\ks}(\zz)$ (as a reminder, $\sigma_{\ks}(\zz)$ is the smallest non-zero singular value of $\zz$).  To see why this is sufficient, let $\bv \in \mathbb{R}^{|\V|}$ and consider its direct sum decomposition with respect to $\mathcal{J}$; i.e., $\bv = \bv_{\mathcal{J}} + \bv_{\mathcal{J}^\perp}$.  Since $\zc \in \tilde{\mathcal{L}}$, we have by Lemma \ref{csexpnaded}, part (4) that   
$$
\bv^{T} \zc \bv = \bv^{T}_{\mathcal{J}} \zc \bv_{\mathcal{J}} = \bv^{T}_{\mathcal{J}} \zz \bv_{\mathcal{J}} + \bv^{T}_{\mathcal{J}} (\zc - \zz) \bv_{\mathcal{J}}.
$$
By Lemma \ref{thm:Unequalsizecanonicalmatrix_spectrum}, $\zz$ restricted on $\mathcal{J}$ is positive definite with smallest eigenvalue at least $\sigma_{\ks}(\zz)$, and hence 
$$
\bv^{T}_{\mathcal{J}} \zz \bv_{\mathcal{J}} \geq \sigma_{\ks}(\zz) \|\bv_{\mathcal{J}}\|_2^2.
$$
On the other hand, the inequality $\sigma_{\max}(\zc-\zz) < \sigma_{\ks}(\zz)$ implies
$$\bv^{T}_{\mathcal{J}} (\zc - \zz) \bv_{\mathcal{J}} <  \sigma_{\ks}(\zz) \|\bv_{\mathcal{J}}\|_2^2.$$
This means that $\bv^{T} \zc \bv \geq 0$ for all $\bv \in \mathbb{R}^{|\V|}$, which means that $\zc$ is PSD.  
  
Finally, we need to show that  ${\rm range}(Z')={\rm ker}(\xc)$. By definition, we have that $Z'\in \tilde{\mathcal{L}}$, so by Proposition \ref{csexpnaded}, we have that 
  $$ {\rm range}(Z')\subseteq \mathcal{J}={\rm ker}(\xc).$$
  For the converse inequality, we show that ${\rm ker}(Z')\subseteq \mathcal{J}^\perp$. For this, take $\bv\in {\rm ker}(Z')$. Let  $\bv = \bv_{\mathcal{J}} + \bv_{\mathcal{J}^\perp}$ and assume  that $\bv_{\mathcal{J}} \ne 0$. Then,  we would have
  $$0=\bv^{T} \zc \bv =\bv^{T}_{\mathcal{J}} \zc \bv_{\mathcal{J}}>0,$$
  leading to a contradiction. 
  
%  If $v_{\mathcal{J}} \ne 0$ it follows from \eqref{xsdcvdfv} that 
%  
%  Moreover, the inequality is strict whenever $v_{\mathcal{J}}$ is non-zero, and hence 
%$$\dim (\ker(\zc))=\dim \mathcal{J}^\perp=\dim {\rm ran}(X_\mathcal{C}).$$ 
%Finally, as 
%$${\rm ran}(X_\mathcal{C}) \subseteq {\rm ker}(\zc) $$
%it follows that  ${\rm ran}(X_\mathcal{C}) ={\rm  ker}(\zc).$

%the matrix $\zc$ is positive definite when restricted to the subspace $\mathcal{J}^\perp$, and thus has rank $|V|-(k-1)$.


\paragraph{Expressing  $\zz - \zc$ via the KKT conditions.} %First, we express the difference $\zz - \zc$ via the KKT conditions.  
The first-order optimality condition corresponding to \eqref{eq:euclideanprojection} gives 
\begin{equation} \label{eq:foc}
\zz - \zc = \tilde{L}  + \tilde{K},
\end{equation}
where $\tilde{L}$ lies in the normal cone of $\tilde{\mathcal{L}}$ at $Z'$ and $\tilde{K}$ lies in the normal cone of $\mathcal{K}$ at $Z'$. Here 
we used the fact  that the normal cone to an intersection is the sum of the normal cones, e.g. see \cite[Corollary 23.8.1]{rockafellar}.  Moreover, as  the normal cone to a subspace is the orthogonal subspace, we have that  $\tilde{L} \in {\tilde{\mathcal{L}}}^\perp$ and $\tilde{K}\in \tilde{\mathcal{K}}^\perp$. First, by projecting \eqref{eq:foc} onto $\tilde{\mathcal{L}}$ we have
\begin{equation} \label{eq:projection_v1}
\zz - \zc = P_{\tilde{\mathcal{L}}}(\tilde{K}).
\end{equation}
%and by projecting onto $\tilde{\tilde{\mathcal{L}}}^\perp$ we have
%$$
%0 = N_{\tilde{\mathcal{L}}} + P_{\tilde{\tilde{\mathcal{L}}}^\perp} (N_{\mathcal{K}}).
%$$
This follows as  $\zz \in \tilde{\mathcal{L}}$, $\zc \in \tilde{\mathcal{K}} \cap \tilde{\mathcal{L}}$.  Second, by projecting \eqref{eq:foc} onto the subspace $\tilde{\mathcal{K}}^\perp$ we have
%$$
%P_{\mathcal{K}} (Z) - \zc  = P_{\mathcal{K}}(N_{\tilde{\mathcal{L}}}).
%$$
%and
\begin{equation} \label{eq:projection_v2}
P_{\tilde{\mathcal{K}}^\perp} (\zz)  = P_{\tilde{\mathcal{K}}^\perp}(\tilde{L}) + \tilde{K}.
\end{equation}
Combining  \eqref{eq:projection_v1} and \eqref{eq:projection_v2}, we have
\begin{equation} \label{eq:zhatx_firstbound}
\| \zz - \zc \|_F = \| P_{\tilde{\mathcal{L}}}(\tilde{K}) \|_F \leq \| \tilde{K} \|_F =  \| P_{\tilde{\mathcal{K}}^\perp} (\zz) - P_{\tilde{\mathcal{K}}^\perp}(\tilde{L}) \|_F \leq \| P_{\tilde{\mathcal{K}}^\perp} (\zz) \|_F + \| P_{\tilde{\mathcal{K}}^\perp}(\tilde{L}) \|_F,
\end{equation}
where  the first equality follows from \eqref{eq:projection_v1}, and the last  inequality follows from  the triangle inequality.  
%{\color{red} By Lemma \ref{thm:incoherence_2} $\| P_{\tilde{\mathcal{K}}^\perp}(N_{\tilde{\mathcal{L}}}) \|_F \leq (2\sqrt{s/n})^{1/2} \|N_{\tilde{\mathcal{L}}}\|_F$.  }

We next proceed to bound the terms $\|P_{\tilde{\mathcal{K}}^\perp} (\zz) \|_F$ and $  \| P_{\tilde{\mathcal{K}}^\perp}(\tilde{L}) \|_F$.


\paragraph{ Bound  the term $  \| P_{\tilde{\mathcal{K}}^\perp} (\zz) \|_F $.} We show that

%[First bound]:  First, we bound the term $  \| P_{\tilde{\mathcal{K}}^\perp} (\zz) \|_F $.  Specifically, we show that
\begin{equation}\label{csvdgb}
\| P_{\tilde{\mathcal{K}}^\perp}(Z^*) \|_F \leq \sqrt{c} \left(\sum_l 1/|\ccs_l|\right).
\end{equation}
The $(i,j)$-th entry of the matrix $P_{\tilde{\mathcal{K}}^\perp}(Z)$ is equal to $Z_{i,j}$ if $(i,j) \in \mathcal{E}$, and is equal to zero otherwise.  Consider the block corresponding to $(\ccs_x,\ccs_y)$, where $x\neq y$.  Each non-zero entry is $1/(|\ccs_x||\ccs_y|)$ and there are at most $c |\ccs_x||\ccs_y|$ entries.  Hence the sum of squares of the entries in this block is at most $c / (|\ccs_x||\ccs_y|)$.  We sum over the all indices $x$ and $y$ to obtain $\| P_{\tilde{\mathcal{K}}^\perp}(Z) \|_F^2  \leq \sum_{x,y} c / (|\ccs_x||\ccs_y|) \leq c (\sum_l 1/|\ccs_l|)^2$, from which the result follows.  (In the first inequality, recall that the block-diagonal entries of $P_{\tilde{\mathcal{K}}^\perp}(Z)$ are zero and thus do not contribute to the sum.)
%\end{proof}

\paragraph{Bound the term  $ \| P_{\tilde{\mathcal{K}}^\perp}(\tilde{L}) \|_F$.}  Setting   
 $$\epsilon = \langle \tilde{L} / \|\tilde{L}\|_F, \tilde{K} / \|\tilde{K}\|_F \rangle,$$
   it follows  from \eqref{eq:foc} that  
$$\| \zz - \zc \|_F^2 = \| \tilde{L} \|_F^2 + \| \tilde{K} \|_F^2 + 2 \epsilon\| \tilde{L} \|_F  \| \tilde{K} \|_F .$$  By the AM-GM inequality we have that 
$$ 2\| \tilde{L} \|_F   \| \tilde{K} \|_F\ge -\| \tilde{L} \|_F^2 - \| \tilde{K} \|_F^2.$$
Consequently, we get  
$$\| \zz - \zc \|_F^2 \geq (1-|\epsilon|)(\| \tilde{L} \|_F^2 + \| \tilde{K} \|_F^2)$$ and since $|\epsilon|\le 1$ we have 
\begin{equation}\label{csdvdfb}
\| \tilde{L} \|_F^2 \le \| \tilde{L} \|_F^2 + \| \tilde{K} \|_F^2 \leq \left(\frac{1}{1-|\epsilon|}\right) \| \zz - \zc \|_F^2.
\end{equation}
Combining \eqref{csdvdfb} with Theorem \ref{thm:incoherence_1} $ (ii)$   we have
\begin{equation}\label{xsdvdfv}
\| P_{\tilde{\mathcal{K}}^\perp}(\tilde{L}) \|_F \leq (2\sqrt{c})^{1/2} \|\tilde{L}\|_F \leq { (2\sqrt{c})^{1/2} \over \sqrt{1-|\epsilon|}} \| \zz - \zc \|_F.
\end{equation}

\paragraph{Concluding the proof.} %We are now in a position to conclude the result.  
We have established in \eqref{eq:zhatx_firstbound} that 
$$ \| \zz - \zc \|_F  \leq \| P_{\tilde{\mathcal{K}}^\perp} (\zz) \|_F + \| P_{\tilde{\mathcal{K}}^\perp}(\tilde{L}) \|_F,$$
which combined with \eqref{csvdgb} and \eqref{xsdvdfv}  shows that
\begin{equation}\label{cxvdbdf}
\| \zz - \zc \|_F \leq \sqrt{c} (\sum_l 1/|\ccs_l|) + { (2\sqrt{c})^{1/2}\over \sqrt{1-|\epsilon|}} \| \zz - \zc \|_F.
\end{equation}
%Since $c\leq 1/23$, we have $(2\sqrt{c})^{1/2} \leq 2^{1/2} 23^{-1/4}$.  
By Theorem \ref{thm:incoherence_1} $(i)$ we have that 
$$|\epsilon| =|\langle \tilde{L} / \|\tilde{L}\|_F, \tilde{K} / \|\tilde{K}\|_F \rangle| \leq 2 \sqrt{c},$$ so for $c\le 1/100$ we get 
%and since $c \leq 1/23$, we have $1/\sqrt{1-|\epsilon|} \leq 1/(1-2 \sqrt{1/23})^{1/2}$.  Hence, 
$${ (2\sqrt{c})^{1/2}\over \sqrt{1-|\epsilon|}}  \leq { 1\over 2}.$$
 Then,   \eqref{cxvdbdf} implies  that 
$$\| \zz - \zc \|_F \leq 2 \sqrt{c} (\sum_l 1/|\ccs_l|).$$
 Finally, using that  the Frobenius norm is always greater than the spectral norm, we have
$$
\| \zz - \zc \|_2 \leq \| \zz - \zc \|_F \leq 2 \sqrt{c} (\sum_l 1/|\ccs_l|),
$$
which is strictly smaller than $\sigma_{\ks}(\zz)$ whenever $c < \frac{1}{4} ^2 (\sigma_{\ks}(\zz)/\sigma_{\max}(\zz))^{2} $.
\end{proof}

