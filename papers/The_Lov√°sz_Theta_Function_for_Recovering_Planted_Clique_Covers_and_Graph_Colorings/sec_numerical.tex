

\section{Future directions} \label{sec:numerical}

In this section, we conduct and discuss additional numerical experiments involving the Lov\'asz theta function $\lt(G)$.  Our objective is to report interesting behavior of the Lov\'asz theta function in the context of recovering clique covers that are not supported or apparent from our theoretical findings, and suggest future directions to investigate.

\subsection{Comparison with ILP}

In our first example, we compare how tight the Lov\'asz theta function is in comparison with the clique cover number.  To investigate this question, we generate a graph from the planted clique cover model for varying choices of $p \in [0,1]$.  We compute the clique cover number by solving the following Integer Linear Program (ILP)
%\begin{equation} \label{eq:cliquecover_ilp}
%\begin{aligned}
%\min ~~ & ~~ \sum_{c} y_c \\
%\mathrm{s.t.} ~~ & ~~ x_{i,c} + x_{j,c} \leq y_c \quad (i,j) \in \mathcal{E} \\
%& ~~ \sum_{c} x_{i,c} = 1 \quad \text{ for all } i \\
%& ~~ x_{i,c}, y_c \in \{0,1\}.
%\end{aligned}
%\end{equation}
%[Or this one?]
\begin{equation} \label{eq:cliquecover_ilp}
\begin{aligned}
\min ~~ & ~~ \sum_{c} y_c \\
\mathrm{s.t.} ~~ & ~~ x_{i,c} \leq y_c \quad \text{ for all } \quad i,c \\
& ~~ x_{i,c} + x_{j,c} \leq 1 \quad (i,j) \in \mathcal{E} \\
& ~~ \sum_{c} x_{i,c} = 1 \quad \text{ for all } \quad i \\
& ~~ x_{i,c}, y_c \in \{0,1\}.
\end{aligned}
\end{equation}
Here, the binary variable $x_{i,c}$ indicates whether vertex $i$ is contained in clique $c$, while the binary variable $y_c$ indicates whether clique $c$ is non-empty.  The constraint $x_{i,c} + x_{j,c} \leq 1$ ensures no adjacent pair of vertices are in the same clique, while the constraint $\sum_{c} x_{i,c} = 1$ ensures that every vertex belongs to a unique clique. % have a covering.

Our motivation for studying this question is as follows:  In our experiments in Section \ref{subsec:related worlk}, we generally take the clique covering number to be equal to $\ks$.  However, short of solving computing the clique covering number outright (say, via \eqref{eq:cliquecover_ilp}), there is no simple way of verifying this is indeed the case.  The concern increases as $p$ increases, as we may inadvertently lower the clique covering number when more edges are added.  As such, one objective of this experiment is to understand if our assumption that the clique covering number is well approximated by $\ks$ is sound.  

%The other objective is to understand how the Lov\'asz theta function differs from the clique covering number for varying levels of $p$.  As we increase $p$, we will generally observe that the Lov\'asz theta function decreases.  We wish to see if that is also because the clique covering number possibly decreases.

In the left sub-plot of Figure \ref{fig:ILP} we compare the clique covering number with the Lov\'asz theta function on a planted clique cover instance with $8$ cliques, each of size $8$, and with varying choices of parameter $p$.  Based on our results we notice that the deviation between these quantities is at most $\approx 2$, and is greatest at $p \approx 0.7$.  

In the right sub-plot of Figure \ref{fig:ILP} we compare the time taken for the ILP solver Gurobi \cite{gurobi} to compute \eqref{eq:cliquecover_ilp} with the SDP solver SDPT3 \cite{sdpt3:1,sdpt3:2} to compute \eqref{eq:lovasz_lambdamax}.  The time taken for SDP solver is approximately equal across all problem instances, as one would expect.  The time taken for the ILP solver is quite small for $p \leq 0.4$, but becomes substantially longer for $p=0.6$.  In the same plot we also track the number of simplex solves required by the ILP solver Gurobi -- the trend of this curve is largely similar to the previous curve.  These observations suggest that the most complex regime for computing the clique covering number for the Erd\H{o}s-R\'enyi noise model occurs around $p \approx 0.6$.  There are a number of explanations: First, our theoretical findings suggest that it is generally easier for the Lov\'asz theta function to discover minimal clique covers for small values of $p$, and these curves do also suggest that small values of $p$ correspond to `easier' instances.  On the other extreme, at $p = 1$, there is only one clique.  For values of $p = 1-\epsilon$, it may be that large cliques are easy to find, and hence ILP solvers are able to certify optimality relatively quickly.  It would be interesting to investigate the behavior of these curves for increasing problem sizes, though one obvious difficulty is that the ILP instances may become impractical to compute. 



\begin{figure}[H]
\centering
\includegraphics[width=0.48\textwidth]{images/ILP_objval.png}
\includegraphics[width=0.48\textwidth]{images/ILP_timetaken.png}
\caption{Comparison of the clique covering number with Lov\'asz theta for random clique cover model (left).  Comparison of time taken for ILP solver to compute clique covering number with time taken for SDP solver to compute Lov\'asz theta function (right). The number of simplex iterations taken by ILP solver is included as a reference.}
\label{fig:ILP}
\end{figure}




\subsection{Phase Transition}



In the second experiment, we investigate the probability that the Lov\'asz theta function correctly recovers the planted cliques as certain parameters are taken to $+\infty$.  Our objective is to see if a phase transition arises.

To investigate this problem, we consider an experimental set-up where all the cliques are of equal size.  We set the number of cliques $\ks$ to be equal to the size of each clique $n$, with the values $n = \ks$ ranging over $\{5,\ldots, 15\}$.  For each value of $n$ and $\ks$, we generate $10$ random graphs from the planted clique cover model with $p$ taking values from $\{ 0.00, 0.05, 0.10, \ldots, 1.00 \}$.  In Figure \ref{fig:phasetransition} we plot the success probabilities corresponding to each value of $p$.  We plot the curve corresponding to strong recovery in black, and the curve corresponding to weak recovery in red.

We make a number of observations.  First, we observe that the transition from success to failure becomes more abrupt as we increase the values of $n$ and $\ks$.  This observation holds for the both types of recovery conditions.  In particular, our results suggest that a phase transition occurs as these values are taken to $+\infty$, and we conjecture that this is indeed the case, as is typical with many phenomena in random graphs.  Second, and quite interestingly, the gap between both types of recovery {\em shrink} as the problem parameters $n$ and $\ks$ increase.  We conjecture that both curves {\em coincide} in the limit.

\begin{figure}[H]
\centering
\includegraphics[width=0.5\textwidth]{images/PhaseTransition.png}
\caption{Probability of correctly recovering a planted clique instance with increasing clique size and increasing number of cliques.  The black curves correspond to strong recovery while the red curves correspond to weak recovery.  The thickness of the lines increase with the problem parameters $n$ and $\ks$.}
\label{fig:phasetransition}
\end{figure}

%In this section, we are interested in the probability that Lov\'asz succeeds when the size of the graph goes to infinity. Here we take 20 equal-step mesh points for $p \in [0,1]$ and for each value of $p$, we iterate all steps starting from cliques generation for 10 times. Based on observation, we conclude that there exist a transition near $p=0.5$ when the size of graph goes to infinity.

% \begin{figure}[H]
% \centering
% \includegraphics[width=0.8\textwidth]{pic/ch-Numerical/equal_k_n10iter5p20.jpg}
% \caption{nk = 100}
% \label{fig:nk = 100}
% \end{figure}


%\begin{figure}[H]
%\centering
%\includegraphics[width=0.5\textwidth]{images/unequal_k10n10_20iter10p20.jpg}
%\caption{k = 10, n $\in$ [10, 20]}
%\label{fig:k =10, n from 10 to 20}
%\end{figure}

