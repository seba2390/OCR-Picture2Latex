%\tableofcontents

\section{Introduction}

Graph colorings and clique covers are central problems in combinatorial optimization.  Given an undirected graph $G = (\V,\E)$, the objective of the graph coloring problem is to partition the vertices into independent subsets; that is, all the vertices from the same subset are not adjacent.  The minimum number of  independent sets required is known as the {\em chromatic number} of $G$ and is denoted by $\chi(G)$.  The objective of the clique covering problem is to partition the vertices $\V$ such that each subset is a clique in $G$; that is, all pairs of vertices from the same set are adjacent.  The minimum number of cliques required is known as the {\em clique cover number} of $G$ and is denoted by~$\overline{\chi}(G)$.  

Graph colorings and clique covers are complementary notions -- the complement of an independent set is a clique.  In particular, a partition of $G$ into $k$ independent sets corresponds to a cover with $k$ cliques in the complementary graph $\overline{G}$.  Both clique covers and graph colorings have applications in various fields such as scheduling \cite{marx} and clustering.

The computational tasks of finding clique covers as well as graph colorings are not known to be tractable.  For instance, the problem of deciding whether a graph admits a coloring with $k$ colors is NP-complete; in fact, it is one of Karp's original 21~problems.   However, the notion of NP-hardness is rooted in a worst-case analysis perspective, which may not be representative of the instances that are typically encountered in practice.  As such, it is natural to move beyond the worst-case analysis \cite{rough} and study these problems in a suitably defined average-case instance \cite{Ban:16}.   
% Moving  beyond worst-case analysis~\cite{rough}  a natural question arises:  does  complexity  emerge   {\em only when it does not matter?}  \cite{linial} 
In fact, there are a number of prior works that study problems known to be hard in the worst-case sense, but admit polynomial-time solutions for average-case instances that align with the problem's inherent characteristics -- prominent examples include community detection \cite{ABH:16}, clustering \cite{ABCKVW:15,IMPV:15}, planted cliques and bicliques~\cite{AV:11,feige}, planted $k$-disjoint-cliques \cite{AV:14}, and graph bisection \cite{boppana}.


This is the viewpoint we adopt in this work.  We focus on the clique cover problem, which by complementation, is equivalent to graph coloring.  The central question we aim to address~is:

 \medskip
  \begin{center}
\parbox[c]{400pt}{\centering {\em  
Can the clique cover problem be efficiently solved for randomly generated instances where a clique cover structure naturally emerges?}}
\end{center}
 



We study the clique cover problem  in instances generated from the 
  {\em planted clique cover model}, which is  specified by a set of vertices $\V$, a corresponding partition of the vertices $\{\ccs_l\}_{l=1}^{\ks} \subset \V$, and a parameter $p \in [0,1]$.  We generate a graph from the planted clique model by creating cliques for every subset $\ccs_l$, $1 \leq l \leq \ks$, and by subsequently introducing edges between vertices $i,j$ belonging to different cliques, with probability $p$ independently across all edges; see Figure~\ref{cluster} for an example.   

\begin{figure}[h]\label{cluster}
	\centering
     \includegraphics[width=0.3\linewidth]{images/graph}
     \includegraphics[width=0.3\linewidth]{images/adjmat}
     \caption{A planted clique instance with $8$ cliques of size $8$ each, and with parameter $p=0.2$.}
\end{figure}

Graphs generated from the planted clique cover model have a natural clique cover structure obscured by noise. Our goal  is to understand whether  this latent structure    can be revealed using a convex relaxation of the  integer programming formulation of the clique cover number. 
Specifically, we focus on the Lov\'asz theta function of an undirected graph $G=(\V,\E)$ is given~by
\beq\tag{P} \label{eq:lovasz_lambdamax}
\lt(G) ~~ := ~~ \min_{t,A} \Big\{t : tI+A-J\succeq 0, \ A_{i,i}=0, \ A_{i,j}=0 \text{ for all } (i,j)\not \in \E \Big\},
\eeq
where $J$ is the matrix of all ones and $I$ is the identity matrix of size $|\V|$. 
We point out that the specific formulation \eqref{eq:lovasz_lambdamax} is simply one among a number of equivalent formulations, and we refer the interested reader to \cite{knuth} for a list of alternative definitions.  
 
 An important property of $\vartheta(G)$ -- often referred to as the ``sandwich theorem'' -- is that it is a lower bound to the clique cover number and an upper bound to the stability~{number \cite{theta}}
\beq \label{sand}
\alpha(G) \leq \lt (G) \leq \overline{\chi}(G).
\eeq
The Lov\'asz theta function is the solution of a semidefinite program (SDP), and hence computable in polynomial time \cite{NesNem:94,Ren:01} -- this stands in sharp contrast with the stability number and the chromatic number of a graph, both of which are NP-hard to compute.  In particular, the sandwich theorem has important consequences for {\em perfect} graphs -- these are graphs for which, in every induced subgraph of $G$, the chromatic number equals the size of the maximum clique.  For such graphs, one has $\omega(G) = \lt(G) = \chi(G)$, and thus the clique number and the chromatic number of any perfect graph can be computed efficiently in polynomial time.  One can also obtain the minimal clique covers and colorings for such graphs tractably -- see, e.g., \cite[Section 6.3.2]{monique}. 


\paragraph{Recovery of planted clique covers via $\vartheta(G)$.} The goal of this work is to show that the Lov\'asz theta recovers the planted clique cover model with high probability.  What does such a claim entail?  First, note that the latter inequality in \eqref{sand} -- namely $ \vartheta (G) \leq \overline{\chi}(G)$ -- relies on the fact that any clique cover corresponds to a feasible solution of the Lov\'asz theta formulation \eqref{eq:lovasz_lambdamax}.  
%  with an objective function value equal to the size of the clique cover. 
Concretely, given a graph $G=(\V,\E)$, let $\mathcal{C}=\{\mathcal{C}_l\}_{l=1}^{k}$ be any clique cover.  Consider the following matrix
\begin{equation} \label{eq:feasiblesolution}
X = kI + \Big(\sum_{l=1}^kk\bone_{\mathcal{C}_l}\bone_{\mathcal{C}_l}^T-kI \Big) - J = {1\over k}\sum_{l=1}^{k} (k\bone_{\mathcal{C}_l}-\be) (k\bone_{\mathcal{C}_l}-\be)^T.
\end{equation}
It is evident that $X \succeq 0$, and hence $A=\sum_{l=1}^kk\bone_{\mathcal{C}_l}\bone_{\mathcal{C}_l}^T-kI$ is a feasible solution to \eqref{eq:lovasz_lambdamax} with objective value $t=k$.  In what follows, we define $\ac$ and $\xc$ with respect to the planted clique covering $\{\ccs_l\}_{l=1}^{\ks}$
\begin{equation} \label{eq:feasiblesolution}
\ac := \sum_{l=1}^{\ks} \ks \bone_{\ccs_l}\bone_{\ccs_l}^T-\ks I \quad \text{ and } \quad  \xc  := \ks I + \ac - J = {1\over \ks}\sum_{l=1}^{\ks} (\ks\bone_{\ccs_l}-\be) (\ks\bone_{\ccs_l}-\be)^T.
\end{equation}
When we say that we wish to show that the Lov\'asz theta recovers the planted clique, we specifically mean to prove that the pair $(\ac,\ks)$ is the {\em unique} optimal solution to \eqref{eq:lovasz_lambdamax} for graphs generated according to the above planted clique cover model, with high probability.     

Our first main result is as follows.

\begin{result} \label{thm:result1}
Let $G$ be a random planted clique cover instance  defined on the cliques $\{ \ccs_l \}_{l=1}^{\ks}$, and where we introduce edges between cliques with probability $p < c:= \min \big\{ \frac{1}{4} (\frac{\min_l 1/|\ccs_l|}{\sum_l 1/|\ccs_l|})^2,{ 1\over 100 }\big\}$.  Then $(\ac,\ks)$ is the unique optimal solution to  the Lov\'asz theta formulation \eqref{eq:lovasz_lambdamax} with probability greater~than 
$$1 - \sum_{i=1}^{\ks} {|\ccs_i| \sum_{j \neq i}\exp(-2|\ccs_j|(c-p)^2}).
$$
\end{result}
The simplest type of graphs in the context of the clique cover problem  are disjoint unions of cliques.  An important intermediate step of our proof is to show that the Lov\'asz theta function succeeds at recovering clique covers for graphs that resemble disjoint union of cliques.  More concretely, we say that a graph $G$ with clique cover $\{ \ccs_l \}_{l=1}^{\ks}$ satisfies the $c$-sparse clique cover (c-SCC) property if, for every vertex $v$ and every clique $\ccs_l$ that does not contain $v$, one has

%The key intermediate step of our proof is to show that the Lov\'asz theta function succeeds at recovering clique covers for graphs that are disjoint unions of cliques, and also, graphs that ``resemble'' disjoint union of cliques. 
%   Delving deeper into the precise notion of resemblance, we say that a graph $G$ has the  $c$-sparse clique cover (c-SCC) property  if  its minimal  clique cover   $\{ \ccs_l \}_{l=1}^{\ks}$   satisfies:
% for every vertex $v$ and every clique  $\ccs_l$ that does not contain $v$, we have 
\begin{equation} \label{eq:neighborcondition}\tag{$c$-SCC}
|e(v, \ccs_l)| \leq c | \ccs_l|. %\tag{E1}
\end{equation}
%\end{definition}
Note that if the graph $G$ satisfies the $\cscc$ property for some $c<1$, then every vertex $v \in \V$ cannot form a clique with any other clique $\ccs_l$ for which it does not belong to.

\begin{result} \label{thm:result2}
% \label{thm:mainresult_deterministic}
%Let $G$ be a graph, and  $\{ \ccs_l \}_{l=1}^{\ks}$ be a clique cover for $G$.
%Let $G$ be a planted clique instance whereby the cliques are given by $\{ \ccs_l \}_{l=1}^{k}$.  
Suppose $G$ is a graph that satisfies the $c$-SCC property for some 
$c < \min \{ \frac{1}{4} (\frac{\min_l 1/|\ccs_l|}{\sum_l 1/|\ccs_l|})^2,\frac{1}{100} \}$.  Then $\xc$ is the unique optimal solution to the Lov\'asz theta formulation \eqref{eq:lovasz_lambdamax}.
\end{result}

%The proof of Result \ref{thm:result2} is given in Theorem \ref{thm:abcd-exact-recovery}. 
If the cliques $\{ \ccs_l \}_{l=1}^{\ks}$ are of equal size, then the above condition requires $c \lesssim (1/\ks)^2$.  An important observation from Result \ref{thm:result2} is that it reveals the types of graphs for which the Lov\'asz theta function is most effective at uncovering the underlying clique cover -- these are graphs for which the $c$-neighborly parameter is small.  The parameter $p$ in our model is related to the neighborly parameter $c$ in that small choices of $p$ will generate graphs with small neighborly parameter.  In our numerical experiments in Section \ref{sec:numerical}, we see that the parameter $p$ (and hence, by extension $c$) also reveal something about the difficulty of computing the clique covering number of a particular graph -- the `hard' instances (these are defined by those requiring a large number of simplex solves using a MILP solver) correspond to graphs generated using intermediate values of $p$ (say $p \in [ 0.6, 0.7]$).



%\subsection{Related work}
%
%There is a substantial literature tryiong to uncover the behavoor of convex optimization hola 
%
%There is a body of work that demonstrate the efficacy of certain tractable algorithms on hard optimization instances by studying the performance of these algorithms on suitably generated random instances.  One sub-thread is to show that the optimal solution of the (difficult) combinatorial optimization instance coincides with a cleverly chosen convex relaxation for random instance, often without requiring an intermediate processing step (such as rounding).  Examples of other problems that have been studied that have been examined in such a fashion include community detection \cite{ABH:16,HWX:15}, the closely related task of data clustering \cite{ABCKVW:15}, and sparse Principle Component Analysis (PCA) \cite{AW:09} -- to this end, we point out a short survey summarizing this philosophy \cite{Ban:16}.  Our work can be viewed as the latest member in the context of clique covering and correspondingly, graph coloring.  


%An orthogonal approach to quantifying the effectiveness of relaxations for hard optimization problems focuses on studying the efficacy of convex relaxations over distributions of input parameters that generate natural random instances of the original problem \cite{add}. 
%In this context, ``exact recovery'' refers to the convex relaxation's ability to accurately identify the optimal solution of the underlying optimization problem, typically unique, without any rounding, with a high probability.

%Our work also provides interesting insight about the inequality $\alpha(G) \leq \overline{\chi}(G)$, noted in \eqref{sand}.  These quantities coincide for perfect graphs, but little else is known beyond such graphs.  To this end, a notable result is there exists a polynomial-time coloring of 3-colorable graphs using $O(n^{1/4})$ colors. Furthermore, it has been shown that $\overline{\chi}(G)\le n^{1-c/\lt(G)}$ for some constant $c>0$, indicating that the Lovász theta provides an upper bound on the chromatic number. However, there exist graphs for which $\overline{\chi}(G)=n^\epsilon$ while $\lt(G)\le 3$, demonstrating that the Lovász theta can be significantly smaller than the chromatic number for certain instances. [DO NOT UNDERSTAND THIS PARAGRAPH].
%
%
%While the Lov\'asz theta function is sandwiched between $\alpha(G)$ and $\overline{\chi}(G)$, little else is known about gaps between the inequalities in \eqref{sand}.  To this end, the ``relax and round approach'', provides valuable insights into the relationship between the Lovász theta and coloring problems (e.g., see [CITE]). 
%

%\subsection{Notation}
%
%%In what follows, we generally use the font convention where $x$ denotes a scalar, $\bx$ denotes a vector, while $X$ denotes matrices.  
%{\color{blue} Given a pair of symmetric matrices $A,B$, we denote the trace inner product $\langle A, B \rangle = \mathrm{tr}(AB)$.
%}


