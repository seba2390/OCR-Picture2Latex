\documentclass[a4paper,11pt]{article}
\usepackage{amsfonts}
%\usepackage[dvipdfm]{graphicx}
\usepackage{bmpsize}
\usepackage{graphicx}
\usepackage{epstopdf}
\usepackage{amsthm}
\usepackage{amsmath}
\numberwithin{equation}{section}

%\usepackage{algorithmic}
\ifpdf
  \DeclareGraphicsExtensions{.eps,.pdf,.png,.jpg}
\else
  \DeclareGraphicsExtensions{.eps}
\fi


%\usepackage[latin1]{inputenc}
%\usepackage[english]{babel}
%\usepackage{amsmath}
%\usepackage{amssymb}
%\usepackage{ifpdf}
%%\ifpdf
%%  \usepackage[pdftex]{graphicx}
%  %\usepackage{epstopdf}
%%\else
%%  \usepackage[dvips]{graphicx}
%%\fi
%%\usepackage{subfigure}
%%\usepackage{sidecap}
%\usepackage[pdftitle={The PDF title},
%  pdfauthor={Your Name},
%  pdffitwindow=true,
%  breaklinks=true,
%  colorlinks=true,
%  urlcolor=blue,
%  linkcolor=red,
%  citecolor=red,  
%  anchorcolor=red]{hyperref}
%%\usepackage{pdfpages}
%%\usepackage{algorithm}
%%\usepackage{algorithmic}
%%\usepackage{natbib} % what is this???
%\usepackage{cases}
%\usepackage{booktabs}
%\usepackage{subfig}
%\usepackage{accents}

\usepackage{color,url}
\usepackage{enumitem}
%\usepackage{graphicx}

%\usepackage{graphicx,psfrag}
\usepackage{epic}
%\usepackage{eepic}
\usepackage{amssymb}
\usepackage{afterpage}
\usepackage{epsfig}
\usepackage{pstricks}
\usepackage{stmaryrd,url}
\usepackage{tikz,amsmath}
\usepackage{ulem}
\usepackage{algorithm,algpseudocode}
\usepackage{footmisc}
\usepackage{bm}
\usepackage{amsfonts}
\usepackage{mathrsfs}
%\usepackage[top=1.0in,bottom=1.0in,left=1.0in,right=1.0in]{geometry}
\usepackage[margin=2.5cm]{geometry}
\usepackage{hyperref}


\makeatletter
\newenvironment{breakablealgorithm}
{% \begin{breakablealgorithm}
	\begin{center}
		\refstepcounter{algorithm}% New algorithm
		\hrule height.8pt depth0pt \kern2pt% \@fs@pre for \@fs@ruled 
		\renewcommand{\caption}[2][\relax]{% Make a new \caption
			{\raggedright\textbf{\ALG@name~\thealgorithm} ##2\par}%
			\ifx\relax##1\relax % #1 is \relax
			\addcontentsline{loa}{algorithm}{\protect\numberline{\thealgorithm}##2}%
			\else % #1 is not \relax
			\addcontentsline{loa}{algorithm}{\protect\numberline{\thealgorithm}##1}%
			\fi
			\kern2pt\hrule\kern2pt
		}
	}{% \end{breakablealgorithm}
		\kern2pt\hrule\relax% \@fs@post for \@fs@ruled 
	\end{center}
}

%\setlength{\topmargin}{-15 mm}
%\setlength{\oddsidemargin}{0 mm}
%\setlength{\evensidemargin}{0 mm}
%\setlength{\textwidth}{160 mm}
%\setlength{\textheight}{225 mm}

\newcommand{\f}{\frac}
\newcommand{\p}{\partial}
\newcommand{\bs}{\boldsymbol}
\newcommand{\wt}{\widetilde}
%\newcommand\munderbar[1]{\underaccent{\widebar}{#1}}
\newcommand\ubar[1]{\underline{#1}}
%\newsiamremark{remark}{Remark}

\newtheorem{theorem}{Theorem}[section]
\newtheorem{lemma}[theorem]{Lemma}

\newcommand{\colorone}[1]{{\color{red} #1}}
\newcommand{\colortwo}[1]{{\color{blue} #1}}

% Sets running headers as well as PDF title and authors
%\headers{Finite difference methods for elastic wave equation}{L. Zhang, S. Wang and N. A. Petersson}
\title{Elastic wave propagation in curvilinear coordinates with mesh refinement interfaces by a fourth order finite difference method}


\begin{document}
\author{Lu Zhang\thanks{Department of Applied Physics and Applied Mathematics, Columbia University, New York, NY 10027, USA. \href{mailto:lz2784@columbia.edu}{Email: lz2784@columbia.edu}}\and Siyang Wang\thanks{Department of Mathematics and Mathematical Statistics, Umeå University, Umeå, Sweden. \href{mailto:siyang.wang@umu.se}{Email: siyang.wang@umu.se} Previous affiliation: Division of Applied Mathematics, UKK, M\"alardalen University, V\"aster\aa s, 
    Sweden. }\and N. Anders
  Petersson\thanks{Center for Applied Scientific Computing, Lawrence Livermore National Laboratory,
    Livermore, CA 94551, USA. \href{mailto:petersson1@llnl.gov}{Email: petersson1@llnl.gov}}}
\maketitle


\begin{abstract}
We develop a fourth order accurate finite difference method for the three dimensional elastic wave equation in isotropic media with the piecewise smooth material property. In our model, the material property can be discontinuous at curved interfaces. The governing equations are discretized in second order form on curvilinear meshes by using a fourth order finite difference operator satisfying a summation-by-parts property. The method is energy stable and high order accurate. The highlight is that mesh sizes can be chosen according to the velocity structure of the material so that computational efficiency is improved. At the mesh refinement interfaces with hanging nodes, physical interface conditions are imposed by using ghost points and interpolation.  With a fourth order predictor-corrector time integrator, the fully discrete scheme is energy conserving. Numerical experiments are presented to verify the fourth order convergence rate and the energy conserving property. 
\end{abstract}

%\begin{keywords}
\textbf{Keywords}: Elastic wave equations, Three space dimension,  Finite difference methods, Summation-by-parts, Non-conforming mesh refinement
%\end{keywords}

%\begin{AMS}
\textbf{AMS subject }:    65M06, 65M12
%\end{AMS}

% \leavevmode
% \\
% \\
% \\
% \\
% \\
\section{Introduction}
\label{introduction}

AutoML is the process by which machine learning models are built automatically for a new dataset. Given a dataset, AutoML systems perform a search over valid data transformations and learners, along with hyper-parameter optimization for each learner~\cite{VolcanoML}. Choosing the transformations and learners over which to search is our focus.
A significant number of systems mine from prior runs of pipelines over a set of datasets to choose transformers and learners that are effective with different types of datasets (e.g. \cite{NEURIPS2018_b59a51a3}, \cite{10.14778/3415478.3415542}, \cite{autosklearn}). Thus, they build a database by actually running different pipelines with a diverse set of datasets to estimate the accuracy of potential pipelines. Hence, they can be used to effectively reduce the search space. A new dataset, based on a set of features (meta-features) is then matched to this database to find the most plausible candidates for both learner selection and hyper-parameter tuning. This process of choosing starting points in the search space is called meta-learning for the cold start problem.  

Other meta-learning approaches include mining existing data science code and their associated datasets to learn from human expertise. The AL~\cite{al} system mined existing Kaggle notebooks using dynamic analysis, i.e., actually running the scripts, and showed that such a system has promise.  However, this meta-learning approach does not scale because it is onerous to execute a large number of pipeline scripts on datasets, preprocessing datasets is never trivial, and older scripts cease to run at all as software evolves. It is not surprising that AL therefore performed dynamic analysis on just nine datasets.

Our system, {\sysname}, provides a scalable meta-learning approach to leverage human expertise, using static analysis to mine pipelines from large repositories of scripts. Static analysis has the advantage of scaling to thousands or millions of scripts \cite{graph4code} easily, but lacks the performance data gathered by dynamic analysis. The {\sysname} meta-learning approach guides the learning process by a scalable dataset similarity search, based on dataset embeddings, to find the most similar datasets and the semantics of ML pipelines applied on them.  Many existing systems, such as Auto-Sklearn \cite{autosklearn} and AL \cite{al}, compute a set of meta-features for each dataset. We developed a deep neural network model to generate embeddings at the granularity of a dataset, e.g., a table or CSV file, to capture similarity at the level of an entire dataset rather than relying on a set of meta-features.
 
Because we use static analysis to capture the semantics of the meta-learning process, we have no mechanism to choose the \textbf{best} pipeline from many seen pipelines, unlike the dynamic execution case where one can rely on runtime to choose the best performing pipeline.  Observing that pipelines are basically workflow graphs, we use graph generator neural models to succinctly capture the statically-observed pipelines for a single dataset. In {\sysname}, we formulate learner selection as a graph generation problem to predict optimized pipelines based on pipelines seen in actual notebooks.

%. This formulation enables {\sysname} for effective pruning of the AutoML search space to predict optimized pipelines based on pipelines seen in actual notebooks.}
%We note that increasingly, state-of-the-art performance in AutoML systems is being generated by more complex pipelines such as Directed Acyclic Graphs (DAGs) \cite{piper} rather than the linear pipelines used in earlier systems.  
 
{\sysname} does learner and transformation selection, and hence is a component of an AutoML systems. To evaluate this component, we integrated it into two existing AutoML systems, FLAML \cite{flaml} and Auto-Sklearn \cite{autosklearn}.  
% We evaluate each system with and without {\sysname}.  
We chose FLAML because it does not yet have any meta-learning component for the cold start problem and instead allows user selection of learners and transformers. The authors of FLAML explicitly pointed to the fact that FLAML might benefit from a meta-learning component and pointed to it as a possibility for future work. For FLAML, if mining historical pipelines provides an advantage, we should improve its performance. We also picked Auto-Sklearn as it does have a learner selection component based on meta-features, as described earlier~\cite{autosklearn2}. For Auto-Sklearn, we should at least match performance if our static mining of pipelines can match their extensive database. For context, we also compared {\sysname} with the recent VolcanoML~\cite{VolcanoML}, which provides an efficient decomposition and execution strategy for the AutoML search space. In contrast, {\sysname} prunes the search space using our meta-learning model to perform hyperparameter optimization only for the most promising candidates. 

The contributions of this paper are the following:
\begin{itemize}
    \item Section ~\ref{sec:mining} defines a scalable meta-learning approach based on representation learning of mined ML pipeline semantics and datasets for over 100 datasets and ~11K Python scripts.  
    \newline
    \item Sections~\ref{sec:kgpipGen} formulates AutoML pipeline generation as a graph generation problem. {\sysname} predicts efficiently an optimized ML pipeline for an unseen dataset based on our meta-learning model.  To the best of our knowledge, {\sysname} is the first approach to formulate  AutoML pipeline generation in such a way.
    \newline
    \item Section~\ref{sec:eval} presents a comprehensive evaluation using a large collection of 121 datasets from major AutoML benchmarks and Kaggle. Our experimental results show that {\sysname} outperforms all existing AutoML systems and achieves state-of-the-art results on the majority of these datasets. {\sysname} significantly improves the performance of both FLAML and Auto-Sklearn in classification and regression tasks. We also outperformed AL in 75 out of 77 datasets and VolcanoML in 75  out of 121 datasets, including 44 datasets used only by VolcanoML~\cite{VolcanoML}.  On average, {\sysname} achieves scores that are statistically better than the means of all other systems. 
\end{itemize}


%This approach does not need to apply cleaning or transformation methods to handle different variances among datasets. Moreover, we do not need to deal with complex analysis, such as dynamic code analysis. Thus, our approach proved to be scalable, as discussed in Sections~\ref{sec:mining}.

%!TEX root = SISC_elastic_3d.tex
\section{The anisotropic elastic wave equation}
We consider the time dependent anisotropic elastic wave equation in a three dimensional domain ${\bf x}\in\Omega$, where ${\bf x} = (x^{(1)},x^{(2)},x^{(3)})^T$ are the Cartesian coordinates. The domain $\Omega$ is partitioned into two subdomains $\Omega^f$ and $\Omega^c$, with an interface $\Gamma = \Omega^f\cap\Omega^c$. The material property is assumed to be smooth in each subdomain, but may be discontinuous at the interface $\Gamma$. Without loss of generality, we may assume that the wave speed is slower in $\Omega^f$ than in $\Omega^c$, which motivates us to use a fine mesh in $\Omega^f$ and a coarse mesh in $\Omega^c$. We further assume that both $\Omega^f$ and $\Omega^c$ have six, possibly curved boundary faces. Denote ${\bf r} = (r^{(1)},r^{(2)},r^{(3)})^T$, the parameter coordinates, and  introduce smooth one-to-one mappings 
\begin{equation}\label{mapping}
{\bf x}= {\bf X}^{f}({\bf r}) :  [0,1]^3 \rightarrow \Omega^{f} \subset \mathbb{R}^3 \ \ \ \mbox{and} \ \ \ {\bf x} = {\bf X}^{c}({\bf r}) :  [0,1]^3 \rightarrow \Omega^{c} \subset \mathbb{R}^3.
\end{equation}
Let the inverse of the mappings in (\ref{mapping}) be ${\bf r} = {\bf R}^f({\bf x})$ with components ${\bf R}^f({\bf x}) = (R^{f,(1)}, R^{f,(2)}, R^{f,(3)})^T$ and ${\bf r} = {\bf R}^c({\bf x})$ with components ${\bf R}^c({\bf x}) = (R^{c,(1)}, R^{c,(2)}, R^{c,(3)})^T$, respectively. Note that we do not compute the components of the inverse mapping ${\bf R}^c$ and ${\bf R}^f$ in this paper, the definitions here are for the convenience of the rest of the contents.
 
We further assume that the interface $\Gamma$ corresponds to $r^{(3)} = 1$ for the coarse domain and $r^{(3)} = 0$ for the fine domain. Then the elastic wave equation in the coarse domain $\Omega^c$ in terms of the displacement vector ${\bf C} = {\bf C}({\bf r}, t)$ can be written in curvilinear coordinates as (see \cite{petersson2015wave})
\begin{align}\label{elastic_curvi}
	\rho^c\frac{\partial^2{\bf C}}{\partial^2 t} = \frac{1}{J^c}\left[\bar{\partial}_1(A_1^c\nabla_r{\bf C}) + \bar{\partial}_2(A_2^c\nabla_r{\bf C}) +\bar{\partial}_3(A_3^c\nabla_r{\bf C}) \right],\ \ \  {\bf r}\in[0,1]^3,\ \ \  t\geq0,
\end{align}
where $\rho^c$ is the density function in the coarse domain $\Omega^c$.  We define
\begin{align*} 
A_k^c\nabla_r{\bf C} = \sum_{j = 1}^3 N_{kj}^c\bar{\partial}_j {\bf C}, \ \ \ k = 1,2,3,
\end{align*}
with $\nabla_r  = (\bar{\partial}_1, \bar{\partial}_2, \bar{\partial}_3)^T$,  $\bar{\partial}_i =\frac{\partial}{\partial r^{(i)}}$, for $i = 1,2,3$ and
\begin{equation}\label{N_definition}
	N_{ij}^c = J^c\sum_{l,k = 1}^3\xi_{li}^cO_l^TZ^cO_k\xi_{kj}^c, \ \ i,j = 1,2,3,
\end{equation}
where, 
\[ O_{1}^T = \left(\begin{array}{cccccc}
1 & 0 & 0 &0 & 0 & 0\\
0 & 0 & 0 &0 & 0 & 1\\
0 & 0 & 0 &0 & 1 & 0\end{array}\right), \ \  O_{2}^T = \left(\begin{array}{cccccc}
0 & 0 & 0 &0 & 0 & 1\\
0 & 1 & 0 &0 & 0 & 0\\
0 & 0 & 0 &1 & 0 & 0\end{array}\right),  \ \ O_{3}^T = \left(\begin{array}{cccccc}
0 & 0 & 0 &0 & 1 & 0\\
0 & 0 & 0 &1 & 0 & 0\\
0 & 0 & 1 &0 & 0 & 0\end{array}\right),\]
$Z^c$ is a $6\times 6$ stiffness matrices which is symmetric and positive definite and $\xi_{kj} = \frac{\partial r^{(j)}}{\partial x^{(k)}}$. Further, Define $M^c_{lk} = O_l^TZ^cO_k$, then $M_{ii}^c$ are also symmetric positive definite and $M_{ij}^c = M_{ji}^T$. In particular, for the isotropic elastic wave equation, we have
\[ M_{11}^c = \left(\begin{array}{ccc}
2\mu^c+\lambda^c & 0 & 0\\
0 & \mu^c & 0\\
0 & 0 & \mu^c\end{array}\right),\ \ \  M_{12}^c = \left(\begin{array}{ccc}
0 & \lambda^c & 0\\
\mu^c & 0 & 0\\
0 & 0 & 0\end{array}\right), \]
\begin{equation}\label{M_definition}
M_{22}^c = \left(\begin{array}{ccc}
\mu^c & 0 & 0\\
0 & 2\mu^c+\lambda^c & 0\\
0 & 0 & \mu^c\end{array}\right),\ \ \ M_{13}^c = \left(\begin{array}{ccc}
0 & 0 & \lambda^c\\
0 & 0 & 0\\
\mu^c & 0 & 0\end{array}\right),
\end{equation}
\[\ M_{33}^c = \left(\begin{array}{ccc}
\mu^c & 0 & 0\\
0 & \mu^c & 0\\
0 & 0 & 2\mu^c+\lambda^c\end{array}\right),\ \ \ M_{23}^c = \left(\begin{array}{ccc}
0 & 0 & 0\\
0 & 0 & \lambda^c\\
0 & \mu^c & 0\end{array}\right),\]
\[ M_{31}^c = (M_{13}^c)^T, \ \ \  M_{32}^c =(M_{23}^c)^T, \ \ \ M_{21}^c =(M_{12}^c)^T.\]
Here, $\lambda^c$ and $\mu^c$ are the first and second Lam{\'{e}} parameters, respectively. 

From (\ref{N_definition}) we find that even in the isotropic case the matrices $N_{ij}^c$ are full. Hence, wave propagation in isotropic media has anisotropic properties in curvilinear coordinates. In both isotropic and anisotropic material, the matrices $N_{ii}^c$, $i = 1,2,3$, are symmetric positive definite and $N_{ij}^c=\big(N_{ji}^c\big)^T$, $i,j=1,2,3$. 

Last, $J^c$ is the Jacobian of the coordinate transformation with
\[J^c = \mbox{det} \left(\bar{\partial}_1 {\bf X}^c, \bar{\partial}_2 {\bf X}^c, \bar{\partial}_3 {\bf X}^c\right)\in (0,\infty).\] 
Denote the unit outward normal ${\bf n}_i^{c,\pm} = (n_i^{c,\pm,1},n_i^{c,\pm,2},n_i^{c,\pm,3})$, $i = 1,2,3$, for the boundaries of the subdomain $\Omega^c$, then
\begin{align}\label{outward_normal}
{\bf n}_i^{c,\pm}  = \pm \frac{\nabla_x R^{c,(i)}}{|\nabla_x R^{c,(i)}|}.
\end{align}
Here, $\nabla_x = (\partial_1, \partial_2, \partial_3)^T$, $\partial_i = \frac{\partial}{\partial
  x^{(i)}}$, $i = 1,2,3$. Here, $'+'$ corresponds to $r^{(i)} = 1$ and $'-'$ corresponds to $r^{(i)}
= 0$. The relation between covariant basis vectors $\bar{\partial}_i {\bf X}^c, i = 1,2,3$ and
contravariant basis vectors $\nabla_x R^{c,(i)}, i = 1,2,3$ can be found in \cite{petersson2015wave,
  thompson1985numerical}. The elastic wave equation in curvilinear coordinates for the fine domain
in terms of the displacement vector ${\bf F} = {\bf F}({\bf r}, t)$ is defined in the same way as in
the coarse domain. We have
\begin{align}\label{elastic_curvi_f}
	\rho^f\frac{\partial^2{\bf F}}{\partial^2 t} &= \frac{1}{J^f}\left[\bar{\partial}_1(A_1^f\nabla_r{\bf F}) + \bar{\partial}_2(A_2^f\nabla_r{\bf F}) + \bar{\partial}_3(A_3^f\nabla_r{\bf F}) \right], \ \ \  {\bf r}\in [0,1]^3,\ \ \  t\geq0.
\end{align}

At the interface $\Gamma$, suitable physical interface conditions are the continuity of the traction vectors and the continuity of the displacement vectors,
\begin{equation}\label{interface_cond}
\frac{A_3^c\nabla_r{\bf C}}{J^c\Lambda^c}= \frac{A_3^f\nabla_r{\bf F}}{J^f\Lambda^f}, \quad {\bf F} = {\bf C},
\end{equation}
where
\begin{equation}\label{lambda_cf}
\Lambda^c = \big|\nabla_x R^{c,(3)}\big| , \ \ \ \ \Lambda^f = \big|\nabla_x R^{f,(3)}\big|. 
\end{equation}
Together with suitable physical boundary conditions, the problem (\ref{elastic_curvi}, \ref{elastic_curvi_f}) is well-posed \cite{duru2014stable, petersson2015wave}.




%!TEX root = SISC_elastic_3d.tex
\section{The spatial discretization}

In this section, we describe the spatial discretization for the problem (\ref{elastic_curvi}, \ref{elastic_curvi_f}, \ref{interface_cond}) and start with the SBP operators for the first and second derivative.



%!TEX root = SISC_elastic_3d.tex
\subsection{SBP operators in $1$D}\label{sec_sbp_1d}
Consider a uniform discretization of the domain $x\in[0,1]$ with the grids,
\[\wt
	{\bf x} = [x_0,x_1,\cdots,x_n,x_{n+1}]^T,\ \  x_i = (i-1)h,\ \ i = 0,1,\cdots,n,n+1,\ \ h = 1/(n-1),\]
where $i = 1,n$ correspond to the grid points at the boundary, and $i = 0,n+1$ are ghost points outside of the physical domain. The  operator $D \approx \frac{\partial }{\partial x}$ is a first derivative SBP operator \cite{Kreiss1974,Strand1994} if 
\begin{equation}\label{first_sbp}
({\bf u}, D{\bf v})_h = -(D{\bf u},{\bf v})_h - u_1v_1 + u_nv_n,
\end{equation}
with a scalar product
\begin{equation}\label{inner_product}
({\bf u},{\bf v})_h = h\sum_{i = 1}^{n}\omega_iu_iv_i.
\end{equation}
Here, $0<\omega_i < \infty $ are the weights of scalar product. The SBP operator $D$ has a centered difference stencil at the grid points away from the boundary and the corresponding weights $\omega_i = 1$. To satisfy the SBP identity (\ref{first_sbp}), the coefficients in $D$ are  modified at a few points near the boundary and the corresponding weights $\omega_i \neq 1$. The operator $D$ does not use any ghost points. To discretize the elastic wave equation, we also need to approximate the second derivative with a variable coefficient $(\gamma(x)u_x)_x$. Here, the known function $\gamma(x)>0$ describes the property of the material. There are two different fourth order accurate SBP operators for the approximation of $(\gamma(x)u_x)_x$. The first one $\wt{G}(\gamma){\bf u} \approx (\gamma(x)u_x)_x $, derived by Sj\"ogreen and Petersson \cite{sjogreen2012fourth}, uses one ghost point outside each boundary, and satisfies the second derivative SBP identity,
\begin{equation}\label{sbp_2nd_1}
({\bf u}, \wt{G}(\gamma){\bf v})_h = -S_\gamma({\bf u},{\bf v})-u_1\gamma_1\wt{\bf b}_1{\bf v} + u_n\gamma_n\wt{\bf b}_n {\bf v}.
\end{equation}
Here, the symmetric positive semi-definite bilinear form $S_\gamma({\bf u},{\bf v}) = (D{\bf u},\gamma D{\bf v})_h + ({\bf u}, P(\gamma){\bf v})_{hr}$ does not use any ghost points, $(\cdot,\cdot)_{hr}$ is a standard discrete scalar $L^2$ inner product. The positive semi-definite operator $P(\gamma)$ is small for smooth grid functions but non-zero for odd-even modes, see \cite{petersson2015wave,sjogreen2012fourth} for details. The operators $\wt{\bf b}_1$ and $\wt{\bf b}_n$ take the form %Using the right boundary as an example, we have 
\begin{equation}\label{sbp_1st_1}
\wt{\bf b}_1 {\bf v} = \frac{1}{h}\sum_{i=0}^{4} \wt{d}_i v_i, \quad\wt{\bf b}_n {\bf v} = \frac{1}{h}\sum_{i=n-3}^{n+1} \wt{d}_i v_i.
\end{equation}
They are fourth order approximations of the first derivative $v_x$ on the left and right boundaries, respectively. We note that the notation $\wt{G}(\gamma){\bf v}$ implies that the operator $\wt{G}$ uses ${\bf v}$ on all grid points $\wt{\bf x}$, but $\wt{G}(\gamma){\bf v}$ only returns values on the grid ${\bf x}$ without ghost points. Therefore, when writing in matrix form, $\wt{G}$ is a rectangular matrix of size $n$ by $n+2$.

 In \cite{wang2018fourth}, a method was developed to convert the SBP operator $\wt{G}(\gamma)$ to another SBP operator $G(\gamma)$ which does not use any ghost point and satisfy
 \begin{equation}\label{sbp_2nd_2}
 ({\bf u}, G(\gamma){\bf v})_h = -S_\gamma({\bf u},{\bf v})-u_1\gamma_1{\bf b}_1{\bf v} + u_n\gamma_n{\bf b}_n{\bf v},
 \end{equation}
 where $S_\gamma(\cdot,\cdot)$ is symmetric positive semi-definite. 
 Here, ${\bf b}_1$ and ${\bf b}_n$ are also finite difference operators for the first derivative at the boundaries, and are constructed to fourth order accuracy. They take the form
 \begin{equation}\label{sbp_1st_2}
 {\bf b}_1 {\bf v} = \frac{1}{h}\sum_{i=1}^{5} d_i v_i,\quad {\bf b}_n {\bf v} = \frac{1}{h}\sum_{i=n-4}^{n} d_i v_i.
 \end{equation}
   In this case, ${G}(\gamma)$ is square in matrix form. We note that in  \cite{Mattsson2012}, Mattsson constructed a similar SBP operator with a third order approximation of the first derivative at the boundaries.  % We refer \cite{mattsson2004summation} for another SBP operator which is used to approximate $(\gamma(x)u_x)_x $ without using any ghost points.
 
For the second derivative SBP operators $\wt{G}(\gamma)$ in (\ref{sbp_2nd_1}) and $G(\gamma)$ in (\ref{sbp_2nd_2}), both of them use a fourth order five points centered difference stencil to approximate $(\gamma u_x)_x$ at the interior points away from the boundaries. For the first and the last six grid points close to the boundaries, the operators $G(\gamma)$ and $\wt{G}(\gamma)$ use second order accurate one-sided difference stencils. They are designed to satisfy (\ref{sbp_2nd_2}) and (\ref{sbp_2nd_1}), respectively.

In the following section, we use a combination of two SBP operators, $\wt{G}(\gamma)$ and $G(\gamma)$, to develop a multi-block finite difference discretization for the elastic wave equation. The first SBP operator is $\wt{G}(\gamma)$ with ghost point, and the second SBP operator $G(\gamma)$, converted from $\wt{G}(\gamma)$, does not use ghost point.


%!TEX root = SISC_elastic_3d.tex
\subsection{Semi-discretization of the elastic wave equation}\label{semi_discrete_form}

\begin{figure}[htbp]
	\centering
	%\includegraphics[width=0.6\textwidth,trim={0.4cm 0.7cm 0.8cm 1.4cm}, clip]{physical_discretization.eps}
\includegraphics[width=0.6\textwidth,trim={0.4cm 0.7cm 0.8cm 1.4cm}, clip]{physical_discretization-eps-converted-to}
	\caption{The sketch for the curvilinear mesh of the physical domain $\Omega$. The blue region is the spatial discretization of coarse subdomain $\Omega^c$ and the red region is the spatial discretization of the fine domain $\Omega^f$. Note that $x,y,z$ in the graph correspond to $x^{(1)}, x^{(2)}, x^{(3)}$ respectively. 
	 }\label{physical_discretization}
\end{figure}

In this section, we discretize the elastic wave equations (\ref{elastic_curvi}) and  (\ref{elastic_curvi_f}) with mesh refinement interface $\Gamma$. We assume the ratio of mesh sizes in the reference domains is $1:2$, that is the mesh sizes satisfy
\[h_1(n_1^h-1) = 1, \ \ \ h_2(n_2^h-1) = 1, \ \ \ h_3(n_3^h-1) = 1,\]
and
\[2h_1(n_1^{2h}-1) = 1, \ \ \ 2h_2(n_2^{2h}-1) = 1, \ \ \ 2h_3(n_3^{2h}-1) = 1.\]
 Other ratios can be treated analogously. Figure \ref{physical_discretization} gives an illustration of the discretization of a physical domain. This is an ideal mesh if the wave speed in $\Omega^f$ is half of the wave speed in $\Omega^c$.

In seismic wave simulation, far-field boundary conditions are often imposed in the $x^{(1)}$ and $x^{(2)}$ directions. Here, our focus is on the numerical treatment of the interface conditions (\ref{interface_cond}). Therefore, we assume periodic boundary conditions in $x^{(1)}$ and $x^{(2)}$ and ignore the boundaries in $x^{(3)}$. In Figure \ref{section_discretization}, we fix $x^{(2)} = 0$ and present the $x^{(1)}$-$x^{(3)}$ section of the domain $\Omega$ in both curvilinear space and parameter space.
\begin{figure}[htbp]
	\centering
	\includegraphics[width=0.45\textwidth,trim={1.0cm 2.0cm 1.0cm 1.8cm}, clip]{physical_section_discretization-eps-converted-to}
	\includegraphics[width=0.45\textwidth,trim={1.0cm 2.0cm 1cm 1.8cm}, clip]{parameter_section_discretization-eps-converted-to}
	%\caption{The sketch of spatial discretization of $x^{(1)}$-$x^{(3)}$ section with $x^{(2)} = 0$. From the left to the right are for physical domain and parameter space, respectively. The blue dots are the ghost points for the coarse domain $\Omega^c$.}\label{section_discretization}
\caption{The meshes in the physical domain (left) and parameter domain (right) of $x^{(1)}$-$x^{(3)}$ section with $x^{(2)} = 0$. The blue dots are the ghost points for the coarse domain $\Omega^c$.}\label{section_discretization}
\end{figure}
 To condense notations, we introduce the multi-index notations
\[{\bf i} = (i,j,k),\ \ {\bf r}_{\bf i} = (r^{(1)}_i,r^{(2)}_j,r^{(3)}_k),\ \ {\bf x}_{\bf i} = (x^{(1)}_i,x^{(2)}_j,x^{(3)}_k),\]
and group different sets of grid points according to
\begin{equation*}
\begin{aligned}
	I_{\Omega^c} &= \{i = 1,2,\cdots,n_1^{2h}, j = 1,2,\cdots,n_2^{2h}, k = 1,2,\cdots,n_3^{2h}\},\\
	I_{\Gamma^c} & = \{i = 1,2,\cdots,n_1^{2h}, j = 1,2,\cdots,n_2^{2h}, k = n_3^{2h}\},\\
	I_{\Omega^f} &= \{i = 1,2,\cdots,n_1^h, j = 1,2,\cdots,n_2^h, k = 1,2,\cdots,n_3^h\},\\
	I_{\Gamma^f} & = \{i = 1,2,\cdots,n_1^{h}, j = 1,2,\cdots,n_2^{h}, k = 1\}.
\end{aligned}	
\end{equation*}
The physical coordinates of the coarse grid points and fine grid points follow from the mappings ${\bf x}_{\bf i} = {\bf X}^c({\bf r}_{\bf i})$ and ${\bf x}_{\bf i} = {\bf X}^f({\bf r}_{\bf i})$, respectively. We denote a grid function by
\[{\bf u}_{\bf i} = {\bf u}_{i,j,k} = {\bf u}({\bf x}_{\bf i}),\]
where ${\bf u}$ can be either a scalar or vector. To distinguish between the continuous variables and the corresponding approximations on the grid, we use ${\bf c}_{\bf i}$ and ${\bf f}_{\bf i}$ to denote the grid functions for the approximations of ${\bf C}({\bf x}_{\bf i})$ and ${\bf F}({\bf x}_{\bf i})$, respectively. Let {\bf c} and {\bf f} be the vector representations of the grid functions ${\bf c_i}$ and ${\bf f_i}$ respectively. The elements of $\bf c$ and $\bf f$ are ordered in the following way:\\
	%\begin{description}
		a). for each grid point ${\bf x}_{\bf i}$, there is a $3\times 1$ vector, say ${\bf c}_{\bf i} = (c^{(1)}_{\bf i}, c^{(2)}_{\bf i}, c^{(3)}_{\bf i})^T$ and ${\bf f}_{\bf i} = (f^{(1)}_{\bf i}, f^{(2)}_{\bf i}, f^{(3)}_{\bf i})^T$;\\
		b). the grid points are ordered such that they first loop over $r^{(1)}$ direction ($i$), then $r^{(2)}$ direction ($j$), and finally $r^{(3)}$ direction ($k$) as  
		\[{\bf c} = [c^{(1)}_{1,1,1}, c^{(2)}_{1,1,1}, c^{(3)}_{1,1,1},c^{(1)}_{2,1,1}, c^{(2)}_{2,1,1}, c^{(3)}_{2,1,1},\cdots]^T, \quad {\bf f} = [f^{(1)}_{1,1,1}, f^{(2)}_{1,1,1}, f^{(3)}_{1,1,1},f^{(1)}_{2,1,1}, f^{(2)}_{2,1,1}, f^{(3)}_{2,1,1},\cdots]^T.\] 
		We note that $\bf c$ contains the ghost point values for $k = n_3^{2h}+1$, but ${\bf f}$ does not contain any ghost point values.
	%\end{description}
	
In the spatial discretization, we only use ghost points in the coarse domain and do not use ghost points in the fine domain. Comparing with the traditional approach of using ghost points in both domains, the system of linear equations at the interface becomes smaller and has a better structure. For the rest of the paper, the $\sim$ over an operator represents that the operator applies to a grid function with ghost points. We approximate the elastic wave equation (\ref{elastic_curvi}) in $\Omega^c$ by
%\begin{equation}\label{elastic_semi_c}
{%\rho}_{\bf i}^{c}\frac{d^2{{\bf c}_{\bf i}}}{dt^2} = \frac{1}{J^c_{\bf i}}\wt{\mathcal{L}}^{2h} {{\bf c}}_{\bf i},\quad {\bf i}\in I_{\Omega^c},\quad t>0.
%\end{equation}
\begin{equation}\label{elastic_semi_c}
\left(({\rho}^{2h}\otimes{\bf I})(J^{2h}\otimes {\bf I})\frac{d^2{{\bf c}}}{dt^2}\right)_{\bf i} = \wt{\mathcal{L}}^{2h}_{\bf i} {{\bf c}},\quad {\bf i}\in I_{\Omega^c},\quad t>0,
\end{equation}
where $\rho^{2h}$ and $J^{2h}$ are $n_1^{2h}n_2^{2h}n_3^{2h}\times n_1^{2h}n_2^{2h}n_3^{2h}$ diagonal matrices with the diagonal elements $\rho^{2h}_{\bf i} = \rho^c({\bf x}_{\bf i})$ and $J^{2h}_{\bf i} = J^c({\bf x}_{\bf i})$, ${\bf i}\in I_{\Omega^c}$; the matrix ${\bf I}$ is a $3\times 3$ identity matrix because the spatial dimension of the governing equation is $3$; finally, the discrete spatial operator is
\begin{equation}\label{L_operator}
\wt{\mathcal{L}}^{2h} {{\bf c}} = \left(\sum_{l=1}^2{Q}_l^{2h}({N}_{ll}^{2h}){\bf c}+\wt{{G}}_3^{2h}({N}_{33}^{2h}){{\bf c}}+\sum_{l=1}^3\sum_{m=1,m\neq l}^3{D}_l^{2h}({N}_{lm}^{2h}{D}_m^{2h}{\bf c})\right),
\end{equation}
which uses ghost points when computing $\wt{G}_3^{2h}(N^{2h}_{33}){\bf c}$. % The term $Q_l^{2h}(N_{ll}^{2h}){\bf c}$ approximates $\bar{\partial}_l(N_{ll}\bar{\partial}_l{\bf C})$ are shown in Appendix \ref{appendix_cdomain}; $\wt{G}_3^{2h}(N_{33}^{2h}){\bf c}$ uses ghost points to approximate $\bar{\partial}_3(N_{33}\bar{\partial}_3 {\bf C})$ given in Appendix \ref{appendix_cdomain}; and $D_{l}^{2h}(N_{lm}^{2h}D_m^{2h}{\bf c})$ approximates $\bar{\partial}_l(N_{lm}\bar{\partial}_m {\bf C})$ are presented in Appendix \ref{appendix_cdomain}.
In Appendix \ref{appendix_cdomain}, the terms $Q_l^{2h}(N_{ll}^{2h}){\bf c}$, $\wt{G}_3^{2h}(N_{33}^{2h}){\bf c}$ and $D_{l}^{2h}(N_{lm}^{2h}D_m^{2h}{\bf c})$ are presented, which approximate $\bar{\partial}_l(N_{ll}\bar{\partial}_l{\bf C})$, $\bar{\partial}_3(N_{33}\bar{\partial}_3 {\bf C})$ and $\bar{\partial}_l(N_{lm}\bar{\partial}_m {\bf C})$, respectively.

Next, we approximate the elastic wave equation (\ref{elastic_curvi_f}) on the fine grid points. For all fine grid points that are not located at the interface $\Gamma$, the semi-discretization  is
%\begin{equation}\label{elastic_semi_f}
{%\rho}_{\bf i}^{f}\frac{d^2{{\bf f}_{\bf i}}}{dt^2} = \frac{1}{J^f_{\bf i}}{\mathcal{L}}^{h} {{\bf f}}_{\bf i},\quad {\bf i}\in I_{\Omega^f}\backslash I_{{\Gamma^f}},\quad t>0.
%\end{equation}
\begin{equation}\label{elastic_semi_f}
\left(({\rho}^{h}\otimes{\bf I})(J^h\otimes{\bf I})\frac{d^2{{\bf f}}}{dt^2}\right)_{\bf i} = {\mathcal{L}}^{h}_{\bf i} {{\bf f}},\quad {\bf i}\in I_{\Omega^f}\backslash I_{{\Gamma^f}},\quad t>0.
\end{equation}
Here, $\rho^{h}$ and $J^{h}$ are $n_1^{h}n_2^{h}n_3^{h}\times n_1^{h}n_2^{h}n_3^{h}$ diagonal matrices with the diagonal elements $\rho^h_{\bf i} = \rho^f({\bf x}_{\bf i})$ and $J^h_{\bf i} = J^f({\bf x}_{\bf i})$, ${\bf i}\in I_{\Omega^f}$. And the discrete spatial operator is
\begin{equation}\label{Lf_operator}
{\mathcal{L}}^{h} {{\bf f}} = \left(\sum_{l=1}^2{Q}_l^{h}({N}_{ll}^h){\bf f}+{G}_3^{h}({N}_{33}^h){\bf f}+\sum_{l=1}^3\sum_{m=1,m\neq l}^3{D}_l^{h}({N}_{lm}^{h}{D}_m^{h}{\bf f})\right).
\end{equation}
Here, the term ${G}_3^{h}(N_{33}^{h}){\bf f}$ approximating $\bar{\partial}_3(N_{33}\bar{\partial}_3 {\bf F})$ without using any ghost points is presented in Appendix \ref{appendix_cdomain}; the terms $Q_l^{h}(N_{ll}^{h}){\bf f}$ and $D_{l}^{h}(N_{lm}^{2h}D_m^{h}{\bf f})$ are defined similar as those in (\ref{L_operator}) and are used to approximate $\bar{\partial}_l(N_{ll}\bar{\partial}_l{\bf F})$ and $\bar{\partial}_l(N_{lm}\bar{\partial}_m {\bf F})$, respectively. %Here, ${Q}_l^{h}({N}_{ll}^h){\bf f}$ and ${D}_l^{h}({N}_{lm}^{h}{D}_m^{h}{\bf f})$ are defined similar as those in (\ref{L_operator}), but with the grid points in $I_{\Omega^f}\backslash I_{{\Gamma^f}}$.

For the approximation at the interface $\Gamma$, we obtain the numerical solution using a scaled interpolation operator
\begin{equation}\label{continuous_sol}
{\bf f}_{\bf i} = {\mathcal{P}}_{\bf i}({\bf c}),\quad {\bf i}\in I_{\Gamma^f},
\end{equation}
which imposes the continuity of the solution at the interface $\Gamma$. 
For energy stability, the operator ${\mathcal{P}}$ must be of a specific form
%\[{\mathcal{P}} = (\mathcal{J}^h_\Gamma \bm{\Lambda}^h)^{-\frac{1}{2}}({\bf P}\otimes{\bf I})(\mathcal{J}^{2h}_\Gamma \bm{\Lambda}^{2h})^{\frac{1}{2}}.\]
\begin{equation}\label{scaleP}
{\mathcal{P}} = \left(({J}^h_\Gamma {\Lambda}^h)^{-\frac{1}{2}}{\bf P}({J}^{2h}_\Gamma {\Lambda}^{2h})^{\frac{1}{2}}\right)\otimes{\bf I}.
\end{equation}
Here, $J_{\Gamma}^h$ and $\Lambda^h$ are $n_1^{h}n_2^{h}\times n_1^{h}n_2^{h}$ diagonal matrices with diagonal elements $J_{\Gamma,{\bf i}}^h = J^f({\bf x}_{\bf i})$ and $\Lambda_{\bf i}^h = \Lambda^f({\bf x_i})$, ${\bf i}\in I_{\Gamma^f}$, with $\Lambda^f$ is given in (\ref{lambda_cf}). Similarly, ${J}_{\Gamma}^{2h}$ and ${{\Lambda}^{2h}}$ are $n_1^{2h}n_2^{2h}\times n_1^{2h}n_2^{2h}$ diagonal matrices with diagonal elements $J_{\Gamma,{\bf i}}^{2h} = J^c({\bf x}_{\bf i})$ and $\Lambda_{\bf i}^{2h} = \Lambda^c({\bf x_i})$, ${\bf i}\in I_{\Gamma^c}$, with $\Lambda^c$ is given in (\ref{lambda_cf}). Finally, ${\bf P}$ is an interpolation operator of size $n_1^hn_2^h\times n_1^{2h}n_2^{2h}$ for scalar grid functions at $\Gamma^c$. Since the spatial discretization is fourth order accurate, we also use a fourth order interpolation. With mesh refinement ratio  $1:2$, the stencils ${\bf P}$ have four cases as illustrated in  Figure \ref{interpolation}. Consequently, the scaled interpolation operator $\mathcal{P}$ is also fourth order accurate.

In the implementation of our scheme, we use \eqref{continuous_sol} to obtain the solution at the interface of the fine domain. However, in the energy analysis in  Sec.~\ref{sec_energy}, it is more convenient to use the equivalent form
%We note that \eqref{continuous_sol} is equivalent to 
%\begin{equation}\label{elastic_semi_f_i}
%{\rho}^f_{\bf i} \frac{d^2{\bf f}_{\bf i}}{dt^2} =
%\frac{1}{J^f_{\bf i}}(\mathcal{L}^h{\bf f}_{\bf i} + {\bm \eta}_{\bf i}), \quad {\bf i}\in I_{\Gamma^f}
%\end{equation}
\begin{equation}\label{elastic_semi_f_i}
\left(({\rho}^h\otimes{\bf I})(J^h\otimes{\bf I}) \frac{d^2{\bf f}}{dt^2} \right)_{\bf i}=
\mathcal{L}^h_{\bf i}{\bf f} + {\bm \eta}_{\bf i}, \quad {\bf i}\in I_{\Gamma^f}
\end{equation}
with 
%\begin{equation}\label{eta}
%{\bm \eta}_{\bf i} = {\rho}^f_{\bf i}J^f_{\bf i}{\mathcal{P}}\left(\frac{1}{\rho^c_{{\bf i}'}J^c_{{\bf i}'}}\wt{\mathcal{L}}^{2h} {\bf c}_{{\bf i}'}\right) - \mathcal{L}^{h}{\bf f}_{\bf i}, \quad {\bf i}\in I_{\Gamma^f},\quad {{\bf i}'}\in I_{\Gamma^c}.
%\end{equation}
\begin{equation}\label{eta}
{\bm \eta} = \left(({\rho}^hJ^h)\otimes{\bf I}\right){\mathcal{P}}\left(\left(({\rho^{2h}J^{2h}})\otimes{\bf I}\right)^{-1}\wt{\mathcal{L}}^{2h} {\bf c}\right) - \mathcal{L}^{h}{\bf f}.
\end{equation}
%which is useful in the energy analysis in the next section. 
The variable $\bm \eta$ in (\ref{eta}) is approximately zero with a second order truncation error, which is of the same order as the boundary stencil of the SBP operator. Hence,  $\bm \eta$ does not affect the order of truncation error in the spatial discretization. %Therefore, it does not affect the overall accuracy of the semi-discretization. 
For the simplicity of analysis, we introduce a general notation for the schemes (\ref{elastic_semi_f}) and (\ref{elastic_semi_f_i}) in the fine domain $\Omega^f$,
%\begin{align}\label{fine_scheme}
%{\rho}^f_{\bf i}\frac{d^2{\bf f}_{\bf i}}{dt^2} =\frac{1}{J^f_{\bf i}}\hat{\mathcal{L}}^h{\bf f}_{\bf i} = \left\{
%\begin{aligned}
%&\frac{1}{J^f_{\bf i}}(\mathcal{L}^h{\bf f}_{\bf i} +{\bm \eta}_{\bf i}), \quad {\bf i}\in I_{\Gamma^f}\\
%&\frac{1}{J^f_{\bf i}}\mathcal{L}^h{\bf f}_{\bf i},\quad\quad\quad {\bf i}\in I_{\Omega^f}\backslash I_{\Gamma^f} 
%\end{aligned}
%\right. \quad t > 0.
%\end{align}
\begin{align}\label{fine_scheme}
\left(({\rho}^h\otimes{\bf I})(J^h\otimes{\bf I})\frac{d^2{\bf f}}{dt^2}\right)_{\bf i} = \hat{\mathcal{L}}^h_{\bf i}{\bf f} = \left\{
\begin{aligned}
&\mathcal{L}^h_{\bf i}{\bf f} +{\bm \eta}_{\bf i}, \quad {\bf i}\in I_{\Gamma^f}\\
&\mathcal{L}^h_{\bf i}{\bf f},\quad\quad\quad {\bf i}\in I_{\Omega^f}\backslash I_{\Gamma^f} 
\end{aligned}
\right. \quad t > 0.
\end{align}

The following condition imposes continuity of traction at the interface, the first equation in (\ref{interface_cond}),
%\begin{equation}\label{continuous_traction}
%(\Lambda^{c}_{{\bf i}'}J_{{\bf i}'}^{c})^{-1}\wt{\mathcal{A}}_3^{2h}{\bf c}_{{\bf i}'}
%= {\mathcal{R}}\Big((\Lambda^f_{\bf i}{ J}^f_{\bf i})^{-1}(\mathcal{A}_3^h{\bf f}_{\bf i}-h_3\omega_1{\bm \eta}_{\bf i})\Big), \quad {\bf i}\in I_{\Gamma^f},\quad {{\bf i}'}\in I_{\Gamma^c}.
%\end{equation}
\begin{equation}\label{continuous_traction}
\left(\left((\Lambda^{2h}J_{\Gamma}^{2h})\otimes{\bf I}\right)^{-1}\wt{\mathcal{A}}_3^{2h}{\bf c}\right)_{\bf i}
= {\mathcal{R}}_{\bf i}\Big(\left((\Lambda^hJ_{\Gamma}^h)\otimes{\bf I}\right)^{-1}(\mathcal{A}_3^h{\bf f}-h_3\omega_1{\bm \eta})\Big), \quad {\bf i}\in I_{\Gamma^c}.
\end{equation}
Here,  $\left((\Lambda^{2h}J_{\Gamma}^{2h})\otimes{\bf I}\right)^{-1}\wt{\mathcal{A}}_3^{2h}{\bf c}$ and $\left((\Lambda^hJ_{\Gamma}^h)\otimes{\bf I}\right)^{-1}\mathcal{A}_3^h{\bf f}$ are approximations of the traction at the interface on the coarse grid and fine grid, respectively. The definitions of $\wt{\mathcal{A}}_3^{2h}{\bf c}$ and $\mathcal{A}_3^{h}{\bf f}$ are given in Appendix \ref{appendix_cdomain}. The operator $\mathcal{R}$ is a scaled restriction operator with the structure 
\begin{equation}\label{scaleR}
 {\mathcal{R}} =  \left(({J}^{2h}_\Gamma{\Lambda}^{2h})^{-\frac{1}{2}}{\bf R}({J}^{h}_\Gamma {\Lambda}^h)^{\frac{1}{2}}\right)\otimes {\bf I},
 \end{equation}
 where the stencils of ${\bf R}$  in Figure \ref{restriction} are determined by the compatibility condition ${\bf R}=\frac{1}{4}{\bf P}^T$. It is a restriction operator of size $n_1^{2h}n_2^{2h}\times n_1^hn_2^h$ for scalar grid functions at $\Gamma^f$.   Finally, $h_3\omega_1{\bm \eta}$ in \eqref{continuous_traction} is a term essential for stability, because in the stability analysis in the next section it cancels out ${\bm \eta}$ in the fine domain spatial discretization \eqref{fine_scheme}. The term is smaller than the truncation error of spatial discretization, so it does not affect the overall order of truncation error. Hence, (\ref{continuous_traction})  is a sufficiently accurate approximation for the continuity of traction at the interface.  
 As will be seen later, the compatibility condition, as well as the scaling of the interpolation and restriction operators, are important for energy stability \cite{Lundquist2018}. 
  We also remark that the condition \eqref{continuous_traction} determines the ghost points values in the coarse domain. 
%Here, $(\Lambda^{c}_{{\bf i}'}J_{{\bf i}'}^{c})^{-1}\wt{\mathcal{A}}_3^{2h}{\bf c}_{{\bf i}'}$ approximates the traction, $\frac{A_3^c\nabla_r{\bf C}}{J^c\left|\nabla_x R^{c,(3)}\right|}$, at the interface of the coarse domain with $\Lambda_{{\bf i}'}^c = |\nabla_x R^{c,(3)}({\bf x}_{{\bf i}'})|$ and
%\begin{equation}\label{hatAc}
%\wt{\mathcal{A}}_3^{2h}{\bf c} = {N}_{31}^{2h}{D}^{2h}_1{\bf c} + {N}_{32}^{2h}{D}^{2h}_2{\bf c} + {N}_{33}^{2h}\wt{\mathcal{D}}^{2h}_3{\bf c},
%\end{equation}
%where ${N}_{3l}^{2h}D_l^h{\bf c}$ approximates $N_{3l}\bar{\partial}_l{\bf C}$ and ${N}_{33}^{2h}\wt{\mathcal{D}}_3^{2h}{\bf c}$ approximates $N_{33}\bar{\partial}_3{\bf C}$ are given in Appendix \ref{appendix_cdomain}; $(\Lambda^f_{\bf i}{ J}^f_{\bf i})^{-1}\mathcal{A}_3^h{\bf f}_{\bf i}$ approximates the traction, $\frac{A_3^f\nabla_r{\bf F}}{J^f\left|\nabla_x R^{f,(3)}\right|}$, at the interface of the fine domain with $\Lambda_{\bf i}^f = |\nabla_x R^{f,(3)}({\bf x}_{\bf i})|$ and 
%\begin{equation}\label{hatAf}
%\mathcal{A}_3^h{\bf f} = {N}_{31}^{h}{D}^h_1{\bf f} + {N}_{32}^h{D}^h_2{\bf f} + {N}_{33}^h\mathcal{D}_3^h{\bf f},
%\end{equation}
%where ${N}_{3l}^{h}D_l^{h}{\bf f}$ approximating $N_{3l}\bar{\partial}_l{\bf F}$ have similar definitions as those in (\ref{hatAc}), but correspond to the grid points in $I_{\Gamma^f}$, ${N}_{33}^{2h}{\mathcal{D}}^{2h}_3{\bf f}$ approximating $N_{33}\bar{\partial}_3{\bf F}$ is given in Appendix \ref{appendix_cdomain}; the term $h_3\omega_1{\bm \eta}_{\bf i}$ originates from the fact that ghost points are only used in the coarse domain but not in the fine domain. The term is important for energy stability, as shown in Theorem \ref{thm1}. It is a penalty term on the order of the truncation error and thus (\ref{continuous_traction})  provides a valid way of enforcing the continuity of traction at the interface, $\omega_1$ is a weight in the inner product (\ref{inner_product}); $\mathcal{R}$ is a scaled restriction operator with the structure 
%\begin{equation}\label{scaleR}
% {\mathcal{R}} =  \left(({J}^{2h}_\Gamma{\Lambda}^{2h})^{-\frac{1}{2}}{\bf R}({J}^{h}_\Gamma {\Lambda}^h)^{\frac{1}{2}}\right)\otimes {\bf I},
% \end{equation}
% where ${\bf R}$ is a restriction operator of size $n_1^{2h}n_2^{2h}\times n_1^hn_2^h$ for scalar grid funcitons. It is determined by the compatibility condition ${\bf R}=\frac{1}{4}{\bf P}^T$, see the stencils in Figure \ref{restriction}. As will be seen later, the compatibility condition, as well as the scaling of the interpolation and restrictions, are essential for energy stability \cite{Lundquist2018}. We want to remark that the condition \eqref{continuous_traction} determines the ghost points values in the coarse domain. 

 
Let ${\bf u}$ and ${\bf v}$ be grid functions in the coarse domain $\Omega^c$. We define the discrete inner product at the interface by
\begin{equation}\label{scalar_product_discrete_interface_c}
\left<{\bf u}, {\bf v}\right>_{2h} = 4h_1h_2\sum_{i=1}^{n_1^{2h}}\sum_{j=1}^{n_2^{2h}}{  J}_{\Gamma,i,j,n_3^{2h}}^{2h}\Lambda_{i,j,n_3^{2h}}^{2h}({\bf u}_{i,j,n_3^{2h}}\cdot {\bf v}_{i,j,n_3^{2h}}).
\end{equation}
 Similarly, the discrete inner product at the interface for fine domain $\Omega^f$ is defined as
\begin{equation}\label{scalar_product_discrete_interface_f}
\left<{\bf u}, {\bf v}\right>_{h} = h_1h_2\sum_{i=1}^{n_1^{h}}\sum_{j=1}^{n_2^{h}}{J}_{\Gamma,i,j,1}^h\Lambda_{i,j,1}^h({\bf u}_{i,j,1}\cdot {\bf v}_{i,j,1})
\end{equation}
when $\bf u$ and $\bf v$ are grid functions in fine domain $\Omega^f$. Then we have the following lemma for the interpolation and restriction operators.
 
 \begin{lemma}\label{lemma1}
 	Let ${\bf c}$ and ${\bf f}$ be grid functions at the interface for coarse domain and fine domain, respectively. Then the interpolation operator $\mathcal{P}$ and the restriction operator $\mathcal{R}$ satisfy
 	\begin{equation}\label{pr_relation}
 	\left<\mathcal{P} {\bf c}, {\bf f}\right>_h = \left<{\bf c}, \mathcal{R}{\bf f}\right>_{2h}
 	\end{equation}
 	if the compatibility condition $\bm{R} = \frac{1}{4}\bm{P}^T$ holds. 
 \end{lemma}
 \begin{proof}
 	%Since 
 	%\begin{align*}
 	%{\mathcal{P}} &= (\mathcal{J}^h_\Gamma \bm{\Lambda}^h)^{-\frac{1}{2}}({\bf P}\otimes{\bf I})(\mathcal{J}^{2h}_\Gamma \bm{\Lambda}^{2h})^{\frac{1}{2}}\\
 	%& = ((J_{\Gamma}^h\otimes{\bf I})(\Lambda^h\otimes{\bf I}))^{-\frac{1}{2}}({\bf P}\otimes{\bf I})((J_{\Gamma}^{2h}\otimes{\bf I})(\Lambda^{2h}\otimes{\bf I}))^{\frac{1}{2}}\\
 	%& = (J_{\Gamma}^h\Lambda^h)^{-\frac{1}{2}}{\bf P}(J_{\Gamma}^{2h}\Lambda^{2h})^{\frac{1}{2}}\otimes{\bf I},
 	%\end{align*}
 	%and
 	%\begin{align*}
 	%{\mathcal{R}} &= (\mathcal{J}^{2h}_\Gamma \bm{\Lambda}^{2h})^{-\frac{1}{2}}({\bf R}\otimes{\bf I})(\mathcal{J}^{h}_\Gamma \bm{\Lambda}^{h})^{\frac{1}{2}}\\
 	%& = ((J_{\Gamma}^{2h}\otimes{\bf I})(\Lambda^{2h}\otimes{\bf I}))^{-\frac{1}{2}}({\bf R}\otimes{\bf I})((J_{\Gamma}^{h}\otimes{\bf I})(\Lambda^{h}\otimes{\bf I}))^{\frac{1}{2}}\\
 	%& = (J_{\Gamma}^{2h}\Lambda^{2h})^{-\frac{1}{2}}{\bf R}(J_{\Gamma}^{h}\Lambda^{h})^{\frac{1}{2}}\otimes{\bf I},
 	%\end{align*}
 	%then
 	From \eqref{scalar_product_discrete_interface_c}--\eqref{scalar_product_discrete_interface_f}, the definition of $\mathcal{P}$ in \eqref{scaleP} and $\mathcal{R}$ in \eqref{scaleR}, we obtain
 	\begin{align*}
 	\left<\mathcal{P}{\bf c}, {\bf f}\right>_h &= h_1h_2\left[\left((J_{\Gamma}^h\Lambda^h)^{\frac{1}{2}} {\bf P}(J_{\Gamma}^{2h}\Lambda^{2h})^{\frac{1}{2}}\otimes {\bf I}\right){\bf c}\right]^T{\bf f}\\
 	& = 4h_1h_2 {\bf c}^T  \left[\left((J_{\Gamma}^{2h}\Lambda^{2h})^{\frac{1}{2}}\frac{1}{4}{\bf P}^T(J_{\Gamma}^h\Lambda^h)^{\frac{1}{2}}\otimes {\bf I}\right) {\bf f}\right]\\
 	& =4h_1h_2 {\bf c}^T  \left[\left((J_{\Gamma}^{2h}\Lambda^{2h})^{\frac{1}{2}}{\bf R}(J_{\Gamma}^h\Lambda^h)^{\frac{1}{2}}\otimes {\bf I}\right) {\bf f}\right] = \left<{\bf c}, \mathcal{R}{\bf f}\right>_{2h}
 	\end{align*}
 \end{proof}
 

\begin{figure}%[htbp]
	\centering
	\includegraphics[width=0.24\textwidth,trim={1.8cm 0.8cm 1.4cm 1.2cm}, clip]{interpolation1-eps-converted-to}
	\includegraphics[width=0.24\textwidth,trim={1.8cm 0.8cm 1.4cm 1.2cm}, clip]{interpolation2-eps-converted-to}
	\includegraphics[width=0.24\textwidth,trim={1.8cm 0.8cm 1.4cm 1.2cm}, clip]{interpolation3-eps-converted-to}
	\includegraphics[width=0.24\textwidth,trim={1.8cm 0.8cm 1.4cm 1.2cm}, clip]{interpolation4-eps-converted-to}
	\caption{The sketch for the stencils of fourth order interpolation operator ${\bf P}$ in two dimensions with parameters $\gamma = -\frac{1}{16}$, $\eta = \frac{9}{16}$, $\mu = 1$, $\alpha = \frac{1}{256}$, $\beta = -\frac{9}{256}$ and $\theta = \frac{81}{256}$. }\label{interpolation}
\end{figure}
\begin{figure}[htbp]
	\centering
	\includegraphics[width=0.6\textwidth]{restriction-eps-converted-to}
	\caption{The sketch for the stencil of fourth order restriction operator ${\bf R}$ in two dimensions with parameters $\epsilon = \frac{1}{1024}$, $\nu = -\frac{9}{1024}$, $\phi = -\frac{16}{1024}$, $\delta = \frac{81}{1024}$, $\sigma = \frac{144}{1024}$, $\chi = \frac{256}{1024}$ and $\zeta = 0$.}\label{restriction}
\end{figure}






%!TEX root = SISC_elastic_3d.tex
\subsection{Energy estimate}\label{sec_energy}
In this section, we derive an energy estimate for the semi-discretization (\ref{elastic_semi_c}) and (\ref{fine_scheme}) in Sec.~\ref{semi_discrete_form}. Let ${\bf u}, {\bf v}$ be grid functions in the coarse domain $\Omega^c$ and define the three dimensional discrete scalar product in $\Omega^c$ as
\begin{equation}\label{scalar_product_inner}
({\bf v}, {\bf u})_{2h} = 8h_1h_2h_3\sum_{i=1}^{n_1^{2h}}\sum_{j=1}^{n_2^{2h}}\sum_{k=1}^{n_3^{2h}}\omega_k{J}^{2h}_{i,j,k}({\bf v}_{i,j,k}\cdot {\bf u}_{i,j,k}).
\end{equation}
 Similarly, define the three dimensional discrete scalar product in $\Omega^f$ as
\begin{equation}\label{scalar_product_inner_f}
({\bf v}, {\bf u})_{h} = h_1h_2h_3\sum_{i=1}^{n_1^{h}}\sum_{j=1}^{n_2^{h}}\sum_{k=1}^{n_3^{h}}\omega_k{J}^{h}_{i,j,k}({\bf v}_{i,j,k}\cdot {\bf u}_{i,j,k}),
\end{equation}
where $\bf u$ and $\bf v$ are grid functions in the fine domain $\Omega^f$. Now, we are ready to state the energy estimate of the proposed schemes in Section \ref{semi_discrete_form}. 
\begin{theorem}\label{thm1}
	The semi-discretization (\ref{elastic_semi_c}) and (\ref{fine_scheme}) is energy stable if the interface conditions \eqref{continuous_sol} and \eqref{continuous_traction} are satisfied.
\end{theorem}
\begin{proof}
	%Let $\rho^{h}, \rho^{2h}$ be diagonal matrices. Their diagonal elements are $\rho^f, \rho^c$ evaluated at grid points in $\Omega^f$ and $\Omega^c$, respectively; let $J^{h}, J^{2h}$ be diagonal matrices. Their diagonal elements are $J^f, J^c$ evaluated at grid points in $\Omega^f$ and $\Omega^c$, respectively.
		Forming the inner product between (\ref{elastic_semi_c}) and $8h_1h_2h_3\omega_k{\bf c}_t$, and summing over $i,j,k$, we have
	%\begin{equation}\label{coarse_simple}
	%({\bf c}_t, (\rho^{2h}\otimes {\bf I}){\bf c}_{tt})_{2h} = ({\bf c}_t,(J^{2h}\otimes {\bf I})^{-1}\wt{\mathcal{L}}^{2h}{\bf c})_{2h} = -\mathcal{S}_{2h}({\bf c}_t,{\bf c}) + B_{2h}({\bf c}_t,{{\bf c}}),
	%\end{equation}
	\begin{eqnarray}\label{coarse_simple}
	%&&8h_1h_2h_3\!\sum_{i = 1}^{n_1^{2h}}\sum_{j=1}^{n_2^{2h}}\sum_{k=1}^{n_3^{2h}}\!\omega_k{\bf c}_t\!\cdot\!\left(\!\!(\rho^{2h}\!\otimes{\bf I})(J^{2h}\!\otimes{\bf I})\frac{d^2{\bf c}}{dt^2}\!\right) \!\!=\! 8h_1h_2h_3\!\sum_{i = 1}^{n_1^{2h}}\sum_{j=1}^{n_2^{2h}}\sum_{k=1}^{n_3^{2h}}\!\omega_kJ^{2h}_{i,j,k}{\bf c}_t\!\cdot\!\left(\!\!(\rho^{2h}\otimes{\bf I})\frac{d^2{\bf c}}{dt^2}\!\right)\\
	%&=& 
	({\bf c}_t, (\rho^{2h}\otimes {\bf I}){\bf c}_{tt})_{2h} = ({\bf c}_t,(J^{2h}\otimes {\bf I})^{-1}\wt{\mathcal{L}}^{2h}{\bf c})_{2h} = -\mathcal{S}_{2h}({\bf c}_t,{\bf c}) + B_{2h}({\bf c}_t,{{\bf c}}),
	\end{eqnarray}
	where $\mathcal{S}_{2h}({\bf c}_t,{\bf c})$ is a symmetric and positive definite bilinear form given in Appendix \ref{appendix_bf}, the boundary term $B_{2h} ({\bf c}_t,{\bf c})$ is given by
	\begin{equation}\label{bounary_c1}
	B_{2h} ({\bf c}_t,{\bf c}) = 4h_1h_2\sum_{{\bf i}\in I_{\Gamma^c}}\frac{d{\bf c}_{\bf i}}{dt}\cdot (\wt{A}_3^{2h}{\bf c})_{\bf i}.
	\end{equation}
	Forming the inner product between (\ref{fine_scheme}) and $h_1h_2h_3\omega_k{\bf f}_t$, and summing over $i,j,k$, we obtain
	%\begin{equation}\label{fine_simple}
	%({\bf f}_t, (\rho^h\otimes {\bf I}){\bf f}_{tt})_h = ({\bf f}_t,(J^h\otimes {\bf I})^{-1}\hat{\mathcal{L}}^h{\bf f})_h = -\mathcal{S}_{h}({\bf f}_t,{\bf f}) + B_h({\bf f}_t,{\bf f}) 
	%+h_1h_2h_3\omega_1\sum_{i = 1}^{n_1^h}\sum_{j=1}^{n_2^h} \frac{d{\bf f}_{i,j,1}}{dt}\cdot{\bm \eta}_{i,j,1}.
	%\end{equation}
	\begin{eqnarray}\label{fine_simple}
	%&&h_1h_2h_3\sum_{i = 1}^{n_1^{h}}\sum_{j=1}^{n_2^{h}}\sum_{k=1}^{n_3^{h}}\omega_k{\bf f}_t\!\cdot\!\left(\!(\rho^{h}\otimes{\bf I})(J^{h}\otimes{\bf I})\frac{d^2{\bf f}}{dt^2}\right) \!=\! h_1h_2h_3\sum_{i = 1}^{n_1^{h}}\sum_{j=1}^{n_2^{h}}\sum_{k=1}^{n_3^{h}}\omega_kJ^{h}_{i,j,k}{\bf f}_t\cdot\left(\!(\rho^{h}\otimes{\bf I})\frac{d^2{\bf f}}{dt^2}\right)\\
	%&=&\! 
	({\bf f}_t, (\rho^{h}\otimes {\bf I}){\bf f}_{tt})_{h} \!=\! ({\bf f}_t,(J^{h}\otimes {\bf I})^{-1}\hat{\mathcal{L}}^{h}{\bf f})_{h} \!=\! -\mathcal{S}_{h}({\bf f}_t,{\bf f}) + B_{h}({\bf f}_t,{{\bf f}}) + h_1h_2h_3\omega_1\sum_{{\bf i}\in I_{\Gamma^f}} \frac{d{\bf f}_{\bf i}}{dt}\cdot{\bm \eta}_{\bf i}.
	%+h_1h_2h_3\omega_1\sum_{i = 1}^{n_1^h}\sum_{j=1}^{n_2^h} \frac{d{\bf f}_{i,j,1}}{dt}\cdot{\bm \eta}_{i,j,1}.\nonumber
	\end{eqnarray}
Here, $\mathcal{S}_h$ is also a symmetric and positive definite bilinear form given in Appendix \ref{appendix_bf}. The boundary term $B_h ({\bf f}_t,{\bf f})$ has the following form
	\begin{equation}\label{boundary_f1}
	B_h ({\bf f}_t,{\bf f}) = -h_1h_2\sum_{{\bf i}\in I_{\Gamma^f}}\frac{d{\bf f}_{\bf i}}{dt}\cdot(A_3^h {\bf f})_{\bf i}.
	\end{equation}
	 Adding \eqref{coarse_simple} and \eqref{fine_simple} together, we have
	\begin{multline}\label{semi_energy_1}
	\frac{d}{dt}\big[({\bf f}_t,(\rho^h\otimes {\bf I}) {\bf f}_t)_h + \mathcal{S}_{h}({\bf f},{\bf f}) + ({\bf c}_t,(\rho^{2h}\otimes {\bf I}) {\bf c}_t)_{2h} + \mathcal{S}_{2h}({\bf c},{\bf c}) \big]  = \\
	2B_{h}({\bf f}_t,{\bf f}) + 2B_{2h}({\bf c}_t,{\bf c}) +2h_1h_2h_3\omega_1\sum_{{\bf i}\in I_{\Gamma^f}} \frac{d{\bf f}_{\bf i}}{dt}\cdot{\bm \eta}_{\bf i}.
	%+ 2h_1h_2h_3\omega_1\sum_{i = 1}^{n_1^h}\sum_{j=1}^{n_2^h}\frac{d{\bf f}_{i,j,1}}{dt}\cdot{\bm \eta}_{i,j,1}.
	\end{multline}
	Substituting (\ref{boundary_f1}) and (\ref{bounary_c1}) into (\ref{semi_energy_1}) and combining the definitions of the scalar product at the interface (\ref{scalar_product_discrete_interface_c})--(\ref{scalar_product_discrete_interface_f}), the continuity of solution at the interface \eqref{continuous_sol} and Lemma \ref{lemma1}, we get
	\begin{align*}%\label{semi_energy_2}
	&\hspace{0.4cm}\frac{d}{dt}\left[({\bf f}_t,(\rho^h\otimes {\bf I}) {\bf f}_t)_{h} + \mathcal{S}_{h}({\bf f},{\bf f}) + ({\bf c}_t,(\rho^{2h}\otimes {\bf I}) {\bf c}_t)_{2h} + \mathcal{S}_{2h}({\bf c},{\bf c}) \right]   \\%\nonumber\\
	& = 2\left<{\bf f}_t,\big(({\Lambda}^{h}{J}^h_\Gamma\big)\otimes {\bf I})^{-1}(-\mathcal{A}_3^h{\bf f}+h_3\omega_1{\bm \eta})\right>_{h}+ 2\left<{\bf c}_t,\big(({\Lambda}^{2h}{J}^{2h}_\Gamma\big)\otimes{\bf I})^{-1}\wt{\mathcal{A}}_3^{2h}{\bf c}\right>_{2h}\\% \nonumber\\
	& = 2\left<{\mathcal{P}}{\bf c}_t,\big(({\Lambda}^{h}{J}^h_\Gamma)\otimes{\bf I}\big)^{-1}(-\mathcal{A}_3^h{\bf f}+h_3\omega_1{\bm \eta})\right>_{h}+ 2\left<{\bf c}_t, \big(({\Lambda}^{2h}{J}^{2h}_\Gamma)\otimes{\bf I}\big)^{-1}\wt{\mathcal{A}}_3^{2h}{\bf c}\right>_{2h}\\% \nonumber\\
	& = 2\left<{\bf c}_t,{\mathcal{R}}\Big(\big(({\Lambda}^{h}{J}^h_\Gamma)\otimes{\bf I}\big)^{-1}(-\mathcal{A}_3^h{\bf f}+h_3\omega_1{\bm \eta})\Big)\right>_{2h}+ 2\left<{\bf c}_t,\big(({\Lambda}^{2h}{J}^{2h}_\Gamma)\otimes{\bf I}\big)^{-1}\wt{\mathcal{A}}_3^{2h}{\bf c}\right>_{2h} = 0.
	\end{align*}
Note that the discrete energy for the semi-discretization \eqref{elastic_semi_c} and \eqref{fine_scheme} is given by $({\bf f}_t,(\rho^h\otimes {\bf I}) {\bf f}_t)_{h} + \mathcal{S}_{h}({\bf f},{\bf f}) + ({\bf c}_t,(\rho^{2h} \otimes {\bf I}){\bf c}_t)_{2h} + \mathcal{S}_{2h}({\bf c},{\bf c})$.
\end{proof}








 

%!TEX root = SISC_elastic_3d.tex
\section{The temporal discretization}
The equations are advanced in time with an explicit fourth order accurate predictor-corrector time integration method. Like all explicit time stepping methods, the time step must not exceed the CFL stability limit. By a similar analysis as in \cite{sjogreen2012fourth}, we require 
\begin{equation*}
\Delta_t\leq C_{\text{cfl}}\min\{h_1,h_2,h_3\}/\sqrt{\kappa_{\max}},
\end{equation*}
where %$C_{\text{cfl}} = 1.3$ and
$\kappa_{\text{max}}$ is the maximum eigenvalue of the matrices 
\[T_{\bf i}^{\{f,c\}} = \frac{1}{\rho^{\{f,c\}}({\bf r}_{\bf i})}\left(\begin{array}{ccc}
Tr(N_{11}^{\{f,c\}}({\bf r}_{\bf i})) &  Tr(N_{12}^{\{f,c\}}({\bf r}_{\bf i}))& Tr(N_{13}^{\{f,c\}}({\bf r}_{\bf i}))\\
Tr(N_{21}^{\{f,c\}}({\bf r}_{\bf i})) & Tr(N_{22}^{\{f,c\}}({\bf r}_{\bf i})) & Tr(N_{23}^{\{f,c\}}({\bf r}_{\bf i}))\\
Tr(N_{31}^{\{f,c\}}({\bf r}_{\bf i})) & Tr(N_{32}^{\{f,c\}}({\bf r}_{\bf i})) & Tr(N_{33}^{\{f,c\}}({\bf r}_{\bf i}))\end{array}\right), \]
and $Tr(N_{lm}^{\{f,c\}}({\bf r}_{\bf i}))$ represents the trace of $3\times3$ matrix $N_{lm}^{\{f,c\}}({\bf r}_{\bf i})$. Note that $\kappa_{\text{max}}$ is related to the material properties $\mu^{\{f,c\}}, \lambda^{\{f,c\}}$ and $\rho^{\{f,c\}}$. The notation $\{\cdot,\cdot\}$ represents the component-wise identities. We choose the Courant number $C_{\text{cfl}} = 1.3$, which has been shown to work well in practical problems \cite{petersson2015wave,sjogreen2012fourth}. The Courant number shall not be chosen too close to the stability limit so that noticeable reflections at mesh refinement interfaces can be avoided \cite{Collino2003}. In the following, we give detailed procedures about how we apply the fourth order time integrator to the semidiscretizations (\ref{elastic_semi_c}) and  (\ref{fine_scheme}). 

Let ${\bf c}^{n}$ and ${\bf f}^{n}$ denote the numerical approximations of ${\bf C}({\bf x},t_n), {\bf x}\in\Omega^c$ and ${\bf F}({\bf x},t_n), {\bf x}\in\Omega^f$, respectively. Here, $t_n = n\Delta_t, n = 0,1,\cdots$ and $\Delta_t > 0$ is a constant time step. We present the fourth order time integrator with predictor and corrector in  Algorithm \ref{first_alg}.
~\\
\begin{breakablealgorithm}
	\caption{Fourth order accurate time stepping for the semidiscretizations  (\ref{elastic_semi_c}) and  (\ref{fine_scheme}). }\label{first_alg}
	Given $\wt{{\bf c}}^{n}, \wt{{\bf c}}^{n-1}$ and ${\bf f}^{n}, {\bf f}^{n-1}$ that satisfy the discretized interface conditions.
	
	\begin{itemize}
		\item  {Compute the predictor at the interior grid points %for both fine and coarse domains
			\[{\bf c}^{*,n+1}_{\bf i} = 2{\bf c}^{n}_{\bf i} - {\bf c}^{n-1}_{\bf i} + \Delta_t^2\left((\rho^{2h}\otimes{\bf I})(J^{2h}\otimes{\bf I})\right)^{-1}{\wt{\mathcal{L}}}^{2h}_{\bf i} {\bf{c}}^{n},\quad {\bf i}\in I_{\Omega^c},\]
			\[{\bf f}^{*,n+1}_{\bf i} = 2{\bf f}^{n}_{\bf i} - {\bf f}^{n-1}_{\bf i} + \Delta_t^2\left((\rho^{h}\otimes{\bf I})(J^{h}\otimes{\bf I})\right)^{-1}\hat{\mathcal{L}}^{h}_{\bf i} {\bf{f}}^{n},\quad {\bf i}\in I_{\Omega^f}\backslash I_{\Gamma^f}.\]
		}
		\item{At the interface $\Gamma$, the values ${\bf f}^{*,n+1}_{\bf i}$  are computed by the continuity of solution 
			\begin{equation*}
			{\bf f}^{*,n+1}_{\bf i} = {\mathcal{P}}_{\bf i}({\bf c}^{*,n+1}),\quad {\bf i}\in I_{\Gamma^f}.
			\end{equation*}
		}
		\item{At the interface $\Gamma$, the ghost point values in $\wt{\bf c}^{*,n+1}$ are computed by solving the equation for the continuity of traction 
			%\begin{equation}\label{traction_gamma_pre}
			%({\Lambda}^{c}_{{\bf i}'}{J}_{{\bf i}'}^{c})^{-1}\wt{\mathcal{A}}_3^{2h}{\bf c}^{*,n+1}_{{\bf i}'}
			%= {\mathcal{R}}\left(({\Lambda}^{f}_{\bf i}{J}^f_{\bf i})^{-1}(\mathcal{A}_3^h{\bf f}^{*,n+1}_{\bf i}-h_3\omega_1{\bm \eta}^{\ast,n+1}_{\bf i})\right),\quad {\bf i}'\in I_{\Gamma^c},\quad {\bf i}\in I_{\Gamma^f}.
			%\end{equation}
				\begin{equation}\label{traction_gamma_pre}
			\left(\left((\Lambda^{2h}J^{2h}_{\Gamma})\otimes{\bf I}\right)^{-1}\wt{\mathcal{A}}_3^{2h}{\bf c}^{\star,n+1}\right)_{\bf i}
			= {\mathcal{R}}_{\bf i}\Big(\left((\Lambda^hJ_{\Gamma}^h)\otimes{\bf I}\right)^{-1}(\mathcal{A}_3^h{\bf f}^{\star,n+1}-h_3\omega_1{\bm \eta}^{\star,n+1})\Big), {\bf i}\in I_{\Gamma^c}.
			\end{equation}
		}
		\item{Evaluate the acceleration at all grid points 
			\begin{equation*}
			{\wt{\bf a}}_c^n= \frac{\wt{\bf c}^{*,n+1}-2\wt{\bf c}^{n}+\wt{\bf c}^{n-1}}{\Delta^2_t},\ \ \ \
			{{\bf a}}_f^{n} = \frac{{\bf f}^{*,n+1}-2{\bf f}^{n}+{\bf f}^{n-1}}{\Delta^2_t}.
			\end{equation*}
		}
		\item{Compute the corrector at the interior grid points
			\[{\bf c}^{n+1}_{\bf i} = {\bf c}^{*,n+1}_{\bf i} + \frac{\Delta_t^4}{12}\left((\rho^{2h}\otimes{\bf I})(J^{2h}\otimes{\bf I})\right)^{-1}\wt{\mathcal{L}}^{2h}_{\bf i}{\bf a}_c^{n},\quad {\bf i}\in I_{\Omega^c},\]
			\[{\bf f}^{n+1}_{\bf i} = {\bf f}^{*,n+1}_{\bf i} + \frac{\Delta_t^4}{12}\left((\rho^{h}\otimes{\bf I})(J^h\otimes{\bf I})\right)^{-1}\hat{\mathcal{L}}^{h}_{\bf i}{\bf a}_f^{n},\quad {\bf i}\in I_\Omega^f.\]
		}
		\item{At the interface $\Gamma$, the values ${\bf f}^{n+1}_{\bf i}$  are computed by the continuity of solution
			\begin{equation*}
			{\bf f}^{n+1}_{\bf i} = {\mathcal{P}}_{\bf i}({\bf c}^{n+1}), \quad {\bf i}\in I_{\Gamma^f}.
			\end{equation*}
		}
		\item{At the interface $\Gamma$, the ghost point values in $\wt{\bf c}^{n+1}$ are computed by solving the equation for the continuity of traction
			\begin{equation}\label{traction_gamma_corr}
			%({\Lambda}^{c}_{{\bf i}'}{J}_{{\bf i}'}^{c})^{-1}(\wt{\mathcal{A}}_3^{2h}{\bf c}^{n+1}_{{\bf i}'})
			%= {\mathcal{R}}\left(({\Lambda}^{f}_{\bf i}{ J}^f_{\bf i})^{-1}((\mathcal{A}_3^h{\bf f}^{n+1}_{\bf i})-h_3\omega_1{\bm \eta}^{n+1}_{\bf i})\right), \quad {{\bf i}'}\in I_{\Gamma^c}, \quad {\bf i}\in I_{\Gamma^f}.
			\left(\left((\Lambda^{2h}J^{2h}_{\Gamma})\otimes{\bf I}\right)^{-1}\wt{\mathcal{A}}_3^{2h}{\bf c}^{n+1}\right)_{\bf i}
			= {\mathcal{R}}_{\bf i}\Big(\left((\Lambda^hJ^h_{\Gamma})\otimes{\bf I}\right)^{-1}(\mathcal{A}_3^h{\bf f}^{n+1}-h_3\omega_1{\bm \eta}^{n+1})\Big), \quad {\bf i}\in I_{\Gamma^c}.
			\end{equation}
		}
	\end{itemize}
\end{breakablealgorithm}
~\\

In the Algorithm \ref{first_alg}, we need to solve the linear systems for the continuity of traction at the interface $\Gamma$ in both predictor step (\ref{traction_gamma_pre}) and corrector step (
\ref{traction_gamma_corr}). The linear system matrices of (\ref{traction_gamma_pre}) and (\ref{traction_gamma_corr}) are the same. Therefore, we only present how to solve (\ref{traction_gamma_pre}) in the predictor step.

There are $3n_1^{2h}n_2^{2h}$ unknowns and $3n_1^{2h}n_2^{2h}$ linear equations in (\ref{traction_gamma_pre}). For large problems in three dimensions, it is very memory inefficient to calculate the LU-factorization. Therefore, we use iterative methods to solve the linear system in (\ref{traction_gamma_pre}). In particular, we consider three different iterative methods: the block Jacobi iterative method, the conjugate gradient (CG) iterative method and the preconditioned conjugate gradient iterative method. The detailed methods and a comparison are given in Section \ref{iterative_section}.

%!TEX root = SISC_elastic_3d.tex
\section{Numerical Experiments}
We present four numerical experiments. 
 In Sec.~\ref{convergence_study}, we verify the order of the convergence of the proposed scheme (\ref{elastic_semi_c}, \ref{fine_scheme}, \ref{continuous_sol}, \ref{continuous_traction}).  In  Sec.~\ref{iterative_section}, we present three iterative methods for solving the linear systems (\ref{traction_gamma_pre}) and (\ref{traction_gamma_corr}). The efficiency of the iterative methods is investigated and a comparison with the LU-factorization method is conducted. Next, in Sec.~\ref{gaussian_source} we show that our schemes generate little reflection at the mesh refinement interface. Finally, the energy conservation property is verified in Sec.~\ref{conserved_energy} with heterogeneous and discontinuous material properties.

%!TEX root = SISC_elastic_3d.tex
\subsection{Verification of convergence rate}\label{convergence_study}
We use the method of the manufactured solution to verify the fourth order convergence rate of the proposed scheme. We choose the mapping of the coarse domain $\Omega^c$ as
\[ {\bf x} = {\bf X}^c({\bf r}) = \left(\begin{array}{c}
2\pi r^{(1)} \\
2\pi r^{(2)} \\
r^{(3)}\theta_i\big(r^{(1)},r^{(2)}\big) + (1-r^{(3)})\theta_b\big(r^{(1)},r^{(2)}\big) \end{array}\right), \]
where $0\leq r^{(1)}, r^{(2)}, r^{(3)}\leq 1$, $\theta_i$ represents the interface surface geometry,
\begin{equation}\label{iterface_geometry}
\theta_i\big(r^{(1)},r^{(2)}\big) = \pi+0.2\sin(4\pi r^{(1)})+0.2\cos(4\pi r^{(2)}),
\end{equation}
and $\theta_b$ is the bottom surface geometry,
\begin{equation*}\label{bottom_geometry}
\theta_b\big(r^{(1)},r^{(2)}\big) = 0.2\exp\left(-\frac{(r^{(1)}-0.6)^2}{0.04}\right)+0.2\exp\left(-\frac{(r^{(2)}-0.6)^2}{0.04}\right).
\end{equation*}
As for the fine domain $\Omega^f$, the mapping is chosen to be
\[ {\bf x} = {\bf X}^f({\bf r}) = \left(\begin{array}{c}
2\pi r^{(1)}\\
2\pi r^{(2)}\\
r^{(3)}\theta_t\big(r^{(1)},r^{(2)}\big) + (1-r^{(3)})\theta_i\big(r^{(1)},r^{(2)}\big) \end{array}\right), \]
where $0\leq r^{(1)}, r^{(2)}, r^{(3)}\leq 1$ and $\theta_t$ is the top surface geometry,
\begin{equation*}\label{top_geometry}
\theta_t\big(r^{(1)},r^{(2)}\big) = 2\pi+0.2\exp\left(-\frac{(r^{(1)}-0.5)^2}{0.04}\right)+0.2\exp\left(-\frac{(r^{(2)}-0.5)^2}{0.04}\right).
\end{equation*}
%and $\theta_i$ is the interface geometry which is given in (\ref{iterface_geometry}).
In the entire domain, we choose the density 
\begin{equation*}\label{density_function}
\rho(x^{(1)},x^{(2)},x^{(3)}) = 2 + \sin(x^{(1)}+0.3)\sin(x^{(2)}+0.3)\sin(x^{(3)}-0.2),
\end{equation*}
and material parameters $\mu, \lambda$ 
\begin{equation*}\label{mu_function}
\mu(x^{(1)},x^{(2)},x^{(3)}) = 3 + \sin(3x^{(1)}+0.1)\sin(3x^{(2)}+0.1)\sin(x^{(3)}),
\end{equation*}
and 
\begin{equation*}\label{lambda_function}
\lambda(x^{(1)},x^{(2)},x^{(3)})  = 21+ \cos(x^{(1)}+0.1)\cos(x^{(2)}+0.1)\sin^2(3x^{(3)}).
\end{equation*}
 In addition, we impose a boundary forcing on the top surface and Dirichlet boundary conditions for the other boundaries. The external forcing, top boundary forcing ${\bf g}$ and initial conditions are chosen such that the solutions for both fine domain ($\bf F$) and coarse domain ($\bf C$) are ${\bf F}(\cdot,t) = {\bf C}(\cdot,t) = {\bf u}(\cdot,t) = (u_1(\cdot,t),u_2(\cdot,t),u_3(\cdot,t))^T$ with
\begin{align*}
u_1(\cdot,t) &= \cos(x^{(1)}+0.3)\sin(x^{(2)}+0.3)\sin(x^{(3)}+0.2)\cos(t^2),\\
u_2(\cdot,t) &= \sin(x^{(1)}+0.3)\cos(x^{(2)}+0.3)\sin(x^{(3)}+0.2)\cos(t^2),\\
u_3(\cdot,t) &= \sin(x^{(1)}+0.2)\sin(x^{(2)}+0.2)\cos(x^{(3)}+0.2)\sin(t).
\end{align*}
For example, for the boundary forcing at the top surface, we have 
\begin{equation*}\label{traction_force}
{\bf g} = (g_1,g_2,g_3)^T = \sum_{i=1}^3\left(\sum_{j = 1}^3 M_{ij}^f\frac{\partial{\bf u}}{\partial x^{(j)}}\right) n^{f,+,i}_3,
\end{equation*}
where, $M_{ij}^f$ and $n^{f,+,i}_3$ are defined in (\ref{M_definition}) and (\ref{outward_normal}), respectively.

The problem is evolved until final time $T = 0.5$. In Table \ref{convergence_rate}, we use $L_2$ to represent the $L^2$ error in the entire domain $\Omega = \Omega^c\cup\Omega^f$. The notations $L_2^f$ and $L_2^c$ represent the $L^2$ error in the fine domain $\Omega^f$ and coarse domain $\Omega^c$, respectively. The convergence rates are shown in the parentheses in Table \ref{convergence_rate}. We observe that the convergence rate is fourth order for all cases. Even though the boundary accuracy of the SBP operator is only second order, the optimal convergence rate is fourth order. For a more detailed analysis of the convergence rate, we refer to \cite{Wang2017, Wang2018b}.  To solve the linear system for ghost point values, we use a block Jacobi iterative method. In the following section, we study two more iterative methods and compare them in terms of the condition number and the number of iterations.

\begin{table}[htb]
	\begin{center}
		\begin{tabular}{|c|c c c|}
			\hline
			$2h_1 = 2h_2 = 2h_3 = 2h$   & $L_2$ & $L_2^f$ & $L_2^c$  \\
			\hline
			$2\pi/24$ &2.2227e-03 ~~~~~~~~ & 8.0442e-04 ~~~~~~~~ & 2.0720e-03 ~~~~~~~~\\
			\hline
			$2\pi/48$ &1.4142e-04 (3.97) & 5.1478e-05 (3.97) & 1.3171e-04 (3.98)\\
			\hline 
			$2\pi/96$ &8.6166e-06 (4.04) & 3.0380e-06 (4.08) & 8.0632e-06 (4.03)\\
			\hline
		\end{tabular}
	\end{center}
	\caption{The $L^2$ error and corresponding convergence rates of the fourth order SBP method.}\label{convergence_rate}
\end{table} 


%!TEX root = SISC_elastic_3d.tex
\subsection{Iterative methods}\label{iterative_section}
In this section, we use the same example as in Sec.~\ref{convergence_study}. For the proposed scheme (\ref{elastic_semi_c}, \ref{fine_scheme}, \ref{continuous_sol}, \ref{continuous_traction}), we need to solve linear systems with $3n_1^{2h}n_2^{2h}$ unknown ghost point values on the coarse grid. At each time step, two linear systems with the same matrix are solved for the continuity of traction at the interface $\Gamma$. 

We investigate three iterative methods: the block Jacobi method, the conjugate gradient method  and the preconditioned conjugate gradient method. We note that the coefficient matrix of the linear system arising from the continuity of traction at interface $\Gamma$ is not symmetric for this test problem. However, our experiment shows that both the conjugate gradient method and the preconditioned conjugate gradient method converge.

For the problem proposed in Sec.~\ref{convergence_study}, the structure of the coefficient matrix of the linear system arising in (\ref{continuous_traction}) is shown in Figure \ref{Mass_matrix}, which is determined by the interpolation operator ${\mathcal{P}}$ and restriction operator ${\mathcal{R}}$. In this example, we use $n_1^{2h} = n_2^{2h}=13, n_3^{2h} = 7$. We choose the entries indicated by red color in Figure \ref{Mass_matrix} to be the block Jacobi matrix in the block Jacobi iterative method and the preconditioning matrix in the preconditioned conjugate gradient  method. The absolute error tolerance is set to be $10^{-7}$ for all three iterative methods and $h_1 = h_2 = h_3 = h$.

\begin{table}[htbp]
	\begin{center}
		\begin{tabular}{|c|c c c|}
			\hline
			$2h$   & ~~~~ CG ~~~~& Block Jacobi & Preconditioned CG  \\
			\hline
			$2\pi/24$ &37.78& 24.96& 4.01\\
			\hline
			$2\pi/48$ &38.61 & 25.38 & 2.87\\
			\hline 
			$2\pi/96$ &39.14 &25.43 & 2.25\\
			\hline
		\end{tabular}
	\end{center}
		\caption{The condition number of the matrices in the conjugate gradient method, the block Jacobi method and the preconditioned conjugate gradient method.}\label{condition_number}
\end{table} 
Table \ref{condition_number} shows the condition number of the original coefficient matrix, the block Jacobi matrix and the coefficient matrix after applying the preconditioning matrix. We observe that the condition number for preconditioned conjugate gradient method is smallest and is consistent with the results of iteration number for different iterative methods: there are around $44$ iterations for the conjugate gradient method, $13$ iterations for the block Jacobi method and $9$ iterations for the preconditioned conjugate gradient method.

In comparison, we have also performed an LU factorization for the linear system when the mesh size $2h = 2\pi/96$, and the computation takes 40.6 GB memory. In contrast, with the block Jacobi method, the peak memory usage is only 1.2 GB. For large-scale problems, the memory usage becomes infeasible for the LU factorization. 

\begin{figure}[H]
	\centering
	\includegraphics[width=0.45\textwidth,trim={0.6cm 1cm 1cm 1.2cm}, clip]{Mass_matrix-eps-converted-to}
	\includegraphics[width=0.45\textwidth,trim={0.6cm 1cm 1cm 1.2cm}, clip]{Mass_diagonal_matrix-eps-converted-to}
	\caption{The left panel is the structure of the coefficient matrix of the linear system (\ref{continuous_traction}).  The right panel shows a close-up of one diagonal block.}\label{Mass_matrix}
\end{figure}


%!TEX root = SISC_elastic_3d.tex
\subsection{Gaussian source}\label{gaussian_source}
In this section, we perform a numerical simulation with a Gaussian source at the top surface and verify that the curved mesh refinement interface does not generate any artifacts. 

We choose a flat top and bottom surface geometry 
\begin{equation*}
\theta_t\big(r^{(1)},r^{(2)}\big) = 1000,\quad \theta_b\big(r^{(1)},r^{(2)}\big) = 0,
\end{equation*}
respectively. The mesh refinement interface is parameterized by
\begin{equation}\label{interface_gausian}
\theta_i\big(r^{(1)},r^{(2)}\big) = 800+20\sin(4\pi r^{(1)})+20\cos(4\pi r^{(2)}),
\end{equation}
where $0\leq r^{(1)}, r^{(2)}, r^{(3)}\leq 1$. 
In addition, the mapping in the coarse domain $\Omega^c$ and fine domain $\Omega^f$ are given by 
\[ {\bf x} = {\bf X}^c({\bf r}) = \left(\begin{array}{c}
2000 r^{(1)}\\
2000 r^{(2)}\\
r^{(3)} \theta_i\big(r^{(1)}, r^{(2)}\big) + (1-r^{(3)}) \theta_b\big(r^{(1)},r^{(2)}\big) \end{array}\right) \]
and 
\[ {\bf x} = {\bf X}^f({\bf r}) = \left(\begin{array}{c}
2000 r^{(1)}\\
2000 r^{(2)}\\
r^{(3)}\theta_t\big(r^{(1)},r^{(2)}\big) + (1-r^{(3)})\theta_i\big(r^{(1)},r^{(2)}\big)\end{array}\right), \]
respectively. 
%For the coarse domain $\Omega^c$, the mapping is given by
%\[ {\bf x} = {\bf X}^c({\bf r}) = \left(\begin{array}{c}
%2000 r^{(1)}\\
%2000 r^{(2)}\\
%r^{(3)} \theta_i\big(r^{(1)}, r^{(2)}\big) + (1-r^{(3)}) \theta_b\big(r^{(1)},r^{(2)}\big) \end{array}\right). \]
%Here, $0\leq r^{(1)}, r^{(2)}, r^{(3)}\leq 1$, $\theta_i$ represents the interface surface geometry,
%\begin{equation}\label{interface_gausian}
%\theta_i\big(r^{(1)},r^{(2)}\big) = 800+20\sin(4\pi r^{(1)})+20\cos(4\pi r^{(2)}),
%\end{equation}
%and $\theta_b$ is the bottom surface geometry,
%\begin{equation*}
%\theta_b\big(r^{(1)},r^{(2)}\big) = 0.
%\end{equation*}
%As for the fine domain $\Omega^f$, the mapping is chosen to be
%\[ {\bf x} = {\bf X}^f({\bf r}) = \left(\begin{array}{c}
%2000 r^{(1)}\\
%2000 r^{(2)}\\
%r^{(3)}\theta_t\big(r^{(1)},r^{(2)}\big) + (1-r^{(3)})\theta_i\big(r^{(1)},r^{(2)}\big)\end{array}\right), \]
%where $0\leq r^{(1)}, r^{(2)}, r^{(3)}\leq 1$ and $\theta_t$ is the top surface geometry with
%\begin{equation*}
%\theta_t\big(r^{(1)},r^{(2)}\big) = 1000,
%\end{equation*}
%$\theta_i$ is the interface geometry which is given in (\ref{interface_gausian}). 
In the entire domain, we use the homogeneous material properties
%\begin{equation*}
%\rho(x^{(1)},x^{(2)},x^{(3)}) = 1.5\times 10^3,
%\end{equation*}
%and 
\begin{equation*}
\rho(x^{(1)},x^{(2)},x^{(3)}) = 1.5\times 10^3,\ \  \mu(x^{(1)},x^{(2)},x^{(3)}) = 1.5\times 10^9,\ \ 
\lambda(x^{(1)},x^{(2)},x^{(3)})  = 3\times 10^9.
\end{equation*}

At the top surface, the Gaussian source 
${\bf g} = (g_1,g_2,g_3)^T$ is imposed as the Dirichlet data with $g_1 = g_2 = 0$ and 
\[g_3 = 10^9 \text{exp}\left(-\left(\frac{t-4/44.2}{1/44.2}\right)^2\right)\text{exp}\left(-\left(\frac{x^{(1)}-1000}{12.5}\right)^2-\left(\frac{x^{(2)}-1000}{12.5}\right)^2\right).\]  
Homogeneous Dirichlet boundary conditions are imposed at other boundaries. Both the initial conditions and the external forcing are set to zero everywhere. % and the initial conditions are also set to be zero everywhere, ${\bf F}(\cdot,0) = {\bf C}(\cdot,0) = {\bf u}(\cdot,0) = {\bf 0}, {\bf u}(\cdot,t) = (u_1(\cdot,t), u_2(\cdot,t), u_3(\cdot,t))^T$.
For these material properties, the shear wave velocity is $c_s = \sqrt{\mu/\rho}=1000$. With the dominant wave frequency $f_0=44.2\sqrt{2}/(2\pi)\approx 10$, the corresponding wavelength $c_s/f_0$ is approximately 100.

In the numerical schemes, we consider three different meshes: Mesh 1 is the Cartesian mesh without any interface and $n_1 = n_2 = 201, n_3 = 101$ with $n_i$ denotes the number of grid points in the direction $x^{(i)}$. This corresponds to 10 grid points per wavelength and is considered as the  reference solution. Mesh 2 is the curvilinear mesh with a curved mesh refinement interface defined in \eqref{interface_gausian} and $n_1^{2h} = n_2^{2h} = 101, n_3^{2h} = 41$, $n_1^h = n_2^h = 201, n_3^h = 21$. The mesh size in $\Omega^f$ is approximately the same as the mesh size in the Cartesian mesh. As a result, the waves are resolved with 5 grid points per wavelength in $\Omega^c$. Mesh 3 is obtained by refining Mesh 2 in all three spatial directions. %The final simulation time is $T = 0.4$. %is a curvilinear mesh with the same curved mesh refinement interface and $n_1^{2h} = n_2^{2h} = 201, n_3^{2h} = 81$, $n_1^h = n_2^h = 401, n_3^h = 41$. We notice that the third mesh is a finer version of the second mesh. The final simulation time is $T = 0.4$.

\begin{figure}[htbp]
	\centering
	\includegraphics[width=0.49\textwidth,trim={0.05cm 0.1cm 0.55cm 0.45cm}, clip]{u1_t02_cartesian.png}
	\includegraphics[width=0.49\textwidth,trim={0.05cm 0.1cm 0.55cm 0.45cm}, clip]{u1_t04_cartesian.png}\\
	\includegraphics[width=0.49\textwidth,trim={0.05cm 0.1cm 0.55cm 0.45cm}, clip]{u1_t02_curvi.png}
	\includegraphics[width=0.49\textwidth,trim={0.05cm 0.1cm 0.55cm 0.45cm}, clip]{u1_t04_curvi.png}\\
	\includegraphics[width=0.49\textwidth,trim={0.05cm 0.1cm 0.55cm 0.45cm}, clip]{u1_t02_curvi_finer.png}
	\includegraphics[width=0.49\textwidth,trim={0.05cm 0.1cm 0.55cm 0.45cm}, clip]{u1_t04_curvi_finer.png}
\caption{The graphs for $u_1$. In the top, middle and bottom panel, we show numerical solutions at $t=0.2$ and $t=0.4$ computed with Mesh 1 (uniform Cartesian grid without any interface), Mesh 2 (curved interface) and Mesh 3 (a refinement of Mesh 2), respectively. The curved interfaces are marked with the red dash lines.}
\label{u1}
\end{figure}

%\begin{figure}[htbp]
%	\centering
%	\includegraphics[width=0.4\textwidth,trim={0 2.8cm 0 2.8cm}, clip]{u2_t02_cartesian.png}
%	\includegraphics[width=0.4\textwidth,trim={0 2.8cm 0 2.8cm}, clip]{u2_t02_curvi_mr.png}\\
%	\includegraphics[width=0.4\textwidth,trim={0 2.8cm 0 2.8cm}, clip]{u2_t04_cartesian.png}
%	\includegraphics[width=0.4\textwidth,trim={0 2.8cm 0 2.8cm}, clip]{u2_t04_curvi_mr.png}
%	\caption{The graph for $u_2$. From left to right are for Cartesian mesh without mesh refinement interface and curvilinear mesh with mesh refinement interface respectively. From top to bottom are for $t = 0.2$ and $t = 0.4$ respectively. Note that $x,z$ in the graph correspond to $x^{(1)}, x^{(3)}$ respectively.}\label{u2}
%\end{figure}

\begin{figure}[htbp]
	\centering
	\includegraphics[width=0.49\textwidth,trim={0.05cm 0.1cm 0.55cm 0.45cm}, clip]{u3_t02_cartesian.png}
	\includegraphics[width=0.49\textwidth,trim={0.05cm 0.1cm 0.55cm 0.45cm}, clip]{u3_t04_cartesian.png}\\
	\includegraphics[width=0.49\textwidth,trim={0.05cm 0.1cm 0.55cm 0.45cm}, clip]{u3_t02_curvi.png}
	\includegraphics[width=0.49\textwidth,trim={0.05cm 0.1cm 0.55cm 0.45cm}, clip]{u3_t04_curvi.png}\\
	\includegraphics[width=0.49\textwidth,trim={0.05cm 0.1cm 0.55cm 0.45cm}, clip]{u3_t02_curvi_finer.png}
	\includegraphics[width=0.49\textwidth,trim={0.05cm 0.1cm 0.55cm 0.45cm}, clip]{u3_t04_curvi_finer.png}
	\caption{The graphs for $u_3$. In the top, middle and bottom panel, we show numerical solutions at $t=0.2$ and $t=0.4$ computed with Mesh 1 (uniform Cartesian grid without any interface), Mesh 2 (curved interface) and Mesh 3 (a refinement of Mesh 2), respectively. The curved interfaces are marked with the red dash lines.}
\label{u3}
\end{figure}
In Figure \ref{u1}, we plot the component $u_1$ at $t=0.2$ and $t=0.4$.  Some artifacts are observed in the solution computed with the second mesh, which is due to the small number of grid points per wavelength in $\Omega^c$. The results become better when the finer curvilinear mesh is used. From Figure \ref{u3}, we observe that there is no obvious reflection at the mesh refinement interface for the component $u_3$, and we have a better result when a finer curvilinear mesh is used. The component $u_2$ is zero up to round-off error for both the Cartesian mesh and curvilinear meshes and is not presented here.

%!TEX root = SISC_elastic_3d.tex
\subsection{Energy conservation test}\label{conserved_energy}
To verify the energy conservation property of the scheme, we perform computation without external source term, but with a Gaussian initial data centered at the origin of the computational domain. The computational domain is chosen to be the same as in Sec.~\ref{convergence_study}. The material property is heterogeneous and discontinuous: for the fine domain $\Omega^f$, the density varies according to
\[\rho^f(x^{(1)},x^{(2)},x^{(3)}) = 3 + \sin(2x^{(1)}+0.3)\cos(x^{(2)}+0.3)\sin(2x^{(3)}-0.2),\] 
and material parameters satisfy
\[\mu^f(x^{(1)},x^{(2)},x^{(3)}) = 2 + \cos(3x^{(1)}+0.1)\sin(3x^{(2)}+0.1)\sin(x^{(3)})^2,\]
\[\lambda^f(x^{(1)},x^{(2)},x^{(3)}) = 15 + \cos(x^{(1)}+0.1)\sin(4x^{(2)}+0.1)\sin(3x^{(3)})^2;\]
for the coarse domain $\Omega^c$, the density varies according to
\[\rho^c(x^{(1)},x^{(2)},x^{(3)}) = 2 + \sin(x^{(1)}+0.3)\sin(x^{(2)}+0.3)\sin(2x^{(3)}-0.2),\] 
and material parameters satisfy
\[\mu^c(x^{(1)},x^{(2)},x^{(3)}) = 3 + \sin(3x^{(1)}+0.1)\sin(3x^{(2)}+0.1)\sin(x^{(3)}),\]
\[\lambda^c(x^{(1)},x^{(2)},x^{(3)}) = 21 + \cos(x^{(1)}+0.1)\cos(x^{(2)}+0.1)\sin(3x^{(3)})^2.\]
The initial Gaussian data is given by ${\bf C}(\cdot,0) = {\bf F}(\cdot,0) = {\bf u}(\cdot,0) = (u_1(\cdot,0),u_2(\cdot,0),u_3(\cdot,0))^T$ with
\begin{align*}
	u_1(\cdot,0) &= \mbox{exp}\left(-\frac{(x^{(1)}-\pi)^2}{0.1}\right)\mbox{exp}\left(-\frac{(x^{(2)}-\pi)^2}{0.1}\right)\mbox{exp}\left(-\frac{(x^{(3)}-\pi)^2}{0.1}\right),\\
	u_2(\cdot,0) &= \mbox{exp}\left(-\frac{(x^{(1)}-\pi)^2}{0.2}\right)\mbox{exp}\left(-\frac{(x^{(2)}-\pi)^2}{0.2}\right)\mbox{exp}\left(-\frac{(x^{(3)}-\pi)^2}{0.2}\right),\\
	u_3(\cdot,0) &= \mbox{exp}\left(-\frac{(x^{(1)}-\pi)^2}{0.1}\right)\mbox{exp}\left(-\frac{(x^{(2)}-\pi)^2}{0.2}\right)\mbox{exp}\left(-\frac{(x^{(3)}-\pi)^2}{0.2}\right).
\end{align*}
 The grid spacing in the parameter space for the coarse domain $\Omega^c$ is $2h_1 = 2h_2 = 2h_3 = \frac{\pi}{24}$ and for the fine domain $\Omega^f$ is $h_1 = h_2 = h_3 = \frac{\pi}{48}$, that is we have $25\times25\times13$ grid points in the coarse domain $\Omega^c$ and $49\times49\times25$ grid points in the fine domain $\Omega^f$. 

The semi-discrete energy is given by $({\bf f}_t,({\rho}^h\otimes{\bf I}){\bf f}_t)_h + \mathcal{S}_h({\bf f},{\bf f}) + ({\bf c}_t,({\rho}^{2h}\otimes{\bf I}){\bf c}_t)_{2h} + \mathcal{S}_{2h}({\bf c},{\bf c})$, see (\ref{semi_energy_1}). By using the same approach as for the isotropic elastic wave equation, see \cite{petersson2015wave,sjogreen2012fourth},  the expression for the fully discrete energy reads 
\begin{align*}
&E^{n+1/2} = \left|\left|(\rho^h\otimes {\bf I})^{\frac{1}{2}}\frac{{\bf f}^{n+1}-{\bf f}^n}{\Delta t}\right|\right|_h^2 \!+ S_h({\bf f}^{n+1},{\bf f}^n) - \frac{(\Delta t)^2}{12}\Big((J^h\otimes{\bf I})^{-1}\mathcal{L}^h{\bf f}^{n+1},(\rho^h\otimes{\bf I})^{-1}(J^h\otimes{\bf I})^{-1}\mathcal{L}^h{\bf f}^n\Big)_h\\
&+ \left|\left|(\rho^{2h}\otimes{\bf I})^{\frac{1}{2}}\frac{{\bf c}^{n+1}-{\bf c}^n}{\Delta t}\right|\right|_{2h}^2 \!\!\!+ S_{2h}({\bf c}^{n+1},{\bf c}^n) \!-\! \frac{(\Delta t)^2}{12}\Big(\!(J^{2h}\otimes{\bf I})^{-1}\wt{\mathcal{L}}^{2h}{\bf c}^{n+1}\!\!,(\rho^{2h}\otimes{\bf I})^{-1}(J^{2h}\otimes{\bf I})^{-1}\wt{\mathcal{L}}^{2h}{\bf c}^n\!\Big)_{2h}.
\end{align*}
We plot the relative change in the fully discrete energy, $(E^{n+1/2}-E^{1/2})/E^{1/2}$, as a function of time with $t\in[0,120]$ in Figure \ref{discrete_energy}. This corresponds to $6186$ time steps. Clearly, the fully discrete energy remains constant up to the round-off error.
\begin{figure}[htbp]
	\centering
	\includegraphics[width=0.6\textwidth,trim={0cm 0cm 0cm 0cm}, clip]{discrete_energy-eps-converted-to}
	\caption{The relative change in the fully discrete energy as a function of time. Here, $t = 120$ corresponds to $6186$ time steps.}\label{discrete_energy}
\end{figure}




%!TEX root = SISC_elastic_3d.tex
\subsection{LOH.1 model problem with layered material}
As the final numerical example, we consider the layer-over-halfspace benchmark problem LOH.1 \cite{Day2001}. The computational domain is taken to be $(x,y,z)\in[0,30000]^2\times[0,17000]$ with a free surface boundary conditions at $z=0$.   The problem is driven by a single point moment source defined as 
$g(t,t_0,\omega) \mathcal{M} \cdot \nabla\delta (\mathbf{x}-\mathbf{x_0})$, 
where the point source location is $\mathbf{x_0}= (15000, 15000, 2000)$  and the moment time function is
\[g(t,t_0,\omega) = \frac{\omega}{\sqrt{2\pi}}e^{-\omega^2(t - t_0)^2/2}, \ \ \ \omega = 16.6667,\ \ \ \ t_0 = 0.36.\]
In the 3-by-3 symmetric moment tensor $\mathcal{M}$, the only nonzero elements are $\mathcal{M}_{12}=\mathcal{M}_{21}=10^{18}$. The center frequency is ${\omega}/{(2\pi)}=2.65$ and the highest significant frequency is estimated to be $2.5{\omega}/{(2\pi)}=6.63$.

The LOH.1 model has a layered material property with a material discontinuity at $z=1000$, with the dynamic and mechanical parameters given in Table \ref{material_parameter}. In the top layer $z\in [0, 1000]$, both the compressional and shear velocity are lower than the rest of the domain. For computational efficiency, a smaller grid spacing shall be used in the top layer. 

\begin{table}[htbp]
	\begin{center}
		\begin{tabular}{c c c c c}
			\hline
			~   & Depth $[m]$& $V_p[m/s]$ & $V_s [m/s]$ & $\rho[Kg/m^3]$ \\
			\hline
			Layer&0--1000& 4000& 2000& 2600\\
			half-space &1000--17000 & 6000 & 3464& 2700\\
			\hline 
		\end{tabular}
	\end{center}
	\caption{Dynamic and mechanical parameters for the layer and the lower half-space of the layer over half-space test.}\label{material_parameter}
\end{table} 

We solve the LOH.1 model problem by using the open source code SW4, where our proposed method has been implemented. The solution is recorded in a receiver on the free surface at the point $(x, y, z) = (21000, 23000, 0)$. The time history of the vertical, transverse and radial velocities are shown in Figure \ref{loh1_100} with grid spacing $h = 100$ in the half-space and $h/2 = 50$ in the top layer. With the highest significant frequency 6.63 Hz, the smallest number of grid points per wavelength is only 5.22. Despite this, we observe  the numerical solutions agree well with the exact solution. In Figure \ref{loh1_50}, the solutions computed on a finer mesh with $h = 50$ in the half-space and $h/2 = 25$ in the top layer look identical to the exact solutions.
\begin{figure}[htbp]
	\centering
	\includegraphics[width=0.9\textwidth,trim={4cm 0.2cm 4cm 1.5cm}, clip]{loh1_h100.png}
	\caption{LOH.1: The radial (top), transverse (middle), and vertical (bottom) velocities time histories. Here the numerical solutions are plotted in blue ($h = 100$) and the semi-analytical solution is plotted in red.}\label{loh1_100}
\end{figure}

\begin{figure}[htbp]
	\centering
	\includegraphics[width=0.9\textwidth,trim={4cm 0.2cm 4cm 1.5cm}, clip]{loh1_h50.png}
	\caption{LOH.1: The radial (top), transverse (middle), and vertical (bottom) velocities time histories. Here the numerical solutions are plotted in blue ($h = 50$) and the semi-analytical solution is plotted in red.}\label{loh1_50}
\end{figure}

To test the performance of the new method, we record the quotient between the computational time of solving the linear system for the mesh refinement interface and of the time-stepping procedure in Table \ref{time}. We have run simulations on two different computer clusters. First, we use two nodes on the Rackham cluster with each node consisting of two 10-core Intel Xeon V4 CPUs and 128 GB memory. In the second simulation, we use three nodes on ManeFrame II (M2) with each node consisting of two 18-core Intel Xeon E5-2695 v4 CPUs and 256 GB memory. From Table \ref{time}, we observe that our new method (with ghost points from the coarse domain) needs much less time to solve the linear system for interface conditions compared with the old method in SW4 (with ghost points from both coarse and fine domains). 

\begin{table}[htbp]
	\begin{center}
		\begin{tabular}{|c|c|c|}
			\hline
			Machine   & new method & old method \\
			\hline
			Rackham & 4.02\% &  8.16\%\\
			\hline
			M2 &5.17\% & 8.87\%\\
			\hline 
		\end{tabular}
	\end{center}
		\caption{The quotient of the computational time of solving the linear system for the mesh refinement interface and of the time-stepping procedure.}\label{time}
\end{table} 

In addition, the proposed method implemented in SW4 has excellent parallel scalability. When running the same model problem with 4 nodes (80 cores) on the Rackham cluster, the computational time of the time stepping procedure is $51\%$ of that with 2 nodes. Further increasing to 8 nodes (160 cores), the computational time of the time stepping procedure is $52\%$ of that with 4 nodes.


% \vspace{-0.5em}
\section{Conclusion}
% \vspace{-0.5em}
Recent advances in multimodal single-cell technology have enabled the simultaneous profiling of the transcriptome alongside other cellular modalities, leading to an increase in the availability of multimodal single-cell data. In this paper, we present \method{}, a multimodal transformer model for single-cell surface protein abundance from gene expression measurements. We combined the data with prior biological interaction knowledge from the STRING database into a richly connected heterogeneous graph and leveraged the transformer architectures to learn an accurate mapping between gene expression and surface protein abundance. Remarkably, \method{} achieves superior and more stable performance than other baselines on both 2021 and 2022 NeurIPS single-cell datasets.

\noindent\textbf{Future Work.}
% Our work is an extension of the model we implemented in the NeurIPS 2022 competition. 
Our framework of multimodal transformers with the cross-modality heterogeneous graph goes far beyond the specific downstream task of modality prediction, and there are lots of potentials to be further explored. Our graph contains three types of nodes. While the cell embeddings are used for predictions, the remaining protein embeddings and gene embeddings may be further interpreted for other tasks. The similarities between proteins may show data-specific protein-protein relationships, while the attention matrix of the gene transformer may help to identify marker genes of each cell type. Additionally, we may achieve gene interaction prediction using the attention mechanism.
% under adequate regulations. 
% We expect \method{} to be capable of much more than just modality prediction. Note that currently, we fuse information from different transformers with message-passing GNNs. 
To extend more on transformers, a potential next step is implementing cross-attention cross-modalities. Ideally, all three types of nodes, namely genes, proteins, and cells, would be jointly modeled using a large transformer that includes specific regulations for each modality. 

% insight of protein and gene embedding (diff task)

% all in one transformer

% \noindent\textbf{Limitations and future work}
% Despite the noticeable performance improvement by utilizing transformers with the cross-modality heterogeneous graph, there are still bottlenecks in the current settings. To begin with, we noticed that the performance variations of all methods are consistently higher in the ``CITE'' dataset compared to the ``GEX2ADT'' dataset. We hypothesized that the increased variability in ``CITE'' was due to both less number of training samples (43k vs. 66k cells) and a significantly more number of testing samples used (28k vs. 1k cells). One straightforward solution to alleviate the high variation issue is to include more training samples, which is not always possible given the training data availability. Nevertheless, publicly available single-cell datasets have been accumulated over the past decades and are still being collected on an ever-increasing scale. Taking advantage of these large-scale atlases is the key to a more stable and well-performing model, as some of the intra-cell variations could be common across different datasets. For example, reference-based methods are commonly used to identify the cell identity of a single cell, or cell-type compositions of a mixture of cells. (other examples for pretrained, e.g., scbert)


%\noindent\textbf{Future work.}
% Our work is an extension of the model we implemented in the NeurIPS 2022 competition. Now our framework of multimodal transformers with the cross-modality heterogeneous graph goes far beyond the specific downstream task of modality prediction, and there are lots of potentials to be further explored. Our graph contains three types of nodes. while the cell embeddings are used for predictions, the remaining protein embeddings and gene embeddings may be further interpreted for other tasks. The similarities between proteins may show data-specific protein-protein relationships, while the attention matrix of the gene transformer may help to identify marker genes of each cell type. Additionally, we may achieve gene interaction prediction using the attention mechanism under adequate regulations. We expect \method{} to be capable of much more than just modality prediction. Note that currently, we fuse information from different transformers with message-passing GNNs. To extend more on transformers, a potential next step is implementing cross-attention cross-modalities. Ideally, all three types of nodes, namely genes, proteins, and cells, would be jointly modeled using a large transformer that includes specific regulations for each modality. The self-attention within each modality would reconstruct the prior interaction network, while the cross-attention between modalities would be supervised by the data observations. Then, The attention matrix will provide insights into all the internal interactions and cross-relationships. With the linearized transformer, this idea would be both practical and versatile.

% \begin{acks}
% This research is supported by the National Science Foundation (NSF) and Johnson \& Johnson.
% \end{acks}

\section{Acknowledgement}
\label{sec:conclusion}
This work is supported by National Natural Science Foundation of China (62076144), Shenzhen Key Laboratory of next generation interactive media innovative technology (ZDSYS20210623092001004),\\ Shenzhen Science and Technology Program (WDZC2022081614051\\5001) and Tencent AI Lab Rhino-Bird Focused Research Program (RBFR2022005).

\appendix

\chapter{Supplementary Material}
\label{appendix}

In this appendix, we present supplementary material for the techniques and
experiments presented in the main text.

\section{Baseline Results and Analysis for Informed Sampler}
\label{appendix:chap3}

Here, we give an in-depth
performance analysis of the various samplers and the effect of their
hyperparameters. We choose hyperparameters with the lowest PSRF value
after $10k$ iterations, for each sampler individually. If the
differences between PSRF are not significantly different among
multiple values, we choose the one that has the highest acceptance
rate.

\subsection{Experiment: Estimating Camera Extrinsics}
\label{appendix:chap3:room}

\subsubsection{Parameter Selection}
\paragraph{Metropolis Hastings (\MH)}

Figure~\ref{fig:exp1_MH} shows the median acceptance rates and PSRF
values corresponding to various proposal standard deviations of plain
\MH~sampling. Mixing gets better and the acceptance rate gets worse as
the standard deviation increases. The value $0.3$ is selected standard
deviation for this sampler.

\paragraph{Metropolis Hastings Within Gibbs (\MHWG)}

As mentioned in Section~\ref{sec:room}, the \MHWG~sampler with one-dimensional
updates did not converge for any value of proposal standard deviation.
This problem has high correlation of the camera parameters and is of
multi-modal nature, which this sampler has problems with.

\paragraph{Parallel Tempering (\PT)}

For \PT~sampling, we took the best performing \MH~sampler and used
different temperature chains to improve the mixing of the
sampler. Figure~\ref{fig:exp1_PT} shows the results corresponding to
different combination of temperature levels. The sampler with
temperature levels of $[1,3,27]$ performed best in terms of both
mixing and acceptance rate.

\paragraph{Effect of Mixture Coefficient in Informed Sampling (\MIXLMH)}

Figure~\ref{fig:exp1_alpha} shows the effect of mixture
coefficient ($\alpha$) on the informed sampling
\MIXLMH. Since there is no significant different in PSRF values for
$0 \le \alpha \le 0.7$, we chose $0.7$ due to its high acceptance
rate.


% \end{multicols}

\begin{figure}[h]
\centering
  \subfigure[MH]{%
    \includegraphics[width=.48\textwidth]{figures/supplementary/camPose_MH.pdf} \label{fig:exp1_MH}
  }
  \subfigure[PT]{%
    \includegraphics[width=.48\textwidth]{figures/supplementary/camPose_PT.pdf} \label{fig:exp1_PT}
  }
\\
  \subfigure[INF-MH]{%
    \includegraphics[width=.48\textwidth]{figures/supplementary/camPose_alpha.pdf} \label{fig:exp1_alpha}
  }
  \mycaption{Results of the `Estimating Camera Extrinsics' experiment}{PRSFs and Acceptance rates corresponding to (a) various standard deviations of \MH, (b) various temperature level combinations of \PT sampling and (c) various mixture coefficients of \MIXLMH sampling.}
\end{figure}



\begin{figure}[!t]
\centering
  \subfigure[\MH]{%
    \includegraphics[width=.48\textwidth]{figures/supplementary/occlusionExp_MH.pdf} \label{fig:exp2_MH}
  }
  \subfigure[\BMHWG]{%
    \includegraphics[width=.48\textwidth]{figures/supplementary/occlusionExp_BMHWG.pdf} \label{fig:exp2_BMHWG}
  }
\\
  \subfigure[\MHWG]{%
    \includegraphics[width=.48\textwidth]{figures/supplementary/occlusionExp_MHWG.pdf} \label{fig:exp2_MHWG}
  }
  \subfigure[\PT]{%
    \includegraphics[width=.48\textwidth]{figures/supplementary/occlusionExp_PT.pdf} \label{fig:exp2_PT}
  }
\\
  \subfigure[\INFBMHWG]{%
    \includegraphics[width=.5\textwidth]{figures/supplementary/occlusionExp_alpha.pdf} \label{fig:exp2_alpha}
  }
  \mycaption{Results of the `Occluding Tiles' experiment}{PRSF and
    Acceptance rates corresponding to various standard deviations of
    (a) \MH, (b) \BMHWG, (c) \MHWG, (d) various temperature level
    combinations of \PT~sampling and; (e) various mixture coefficients
    of our informed \INFBMHWG sampling.}
\end{figure}

%\onecolumn\newpage\twocolumn
\subsection{Experiment: Occluding Tiles}
\label{appendix:chap3:tiles}

\subsubsection{Parameter Selection}

\paragraph{Metropolis Hastings (\MH)}

Figure~\ref{fig:exp2_MH} shows the results of
\MH~sampling. Results show the poor convergence for all proposal
standard deviations and rapid decrease of AR with increasing standard
deviation. This is due to the high-dimensional nature of
the problem. We selected a standard deviation of $1.1$.

\paragraph{Blocked Metropolis Hastings Within Gibbs (\BMHWG)}

The results of \BMHWG are shown in Figure~\ref{fig:exp2_BMHWG}. In
this sampler we update only one block of tile variables (of dimension
four) in each sampling step. Results show much better performance
compared to plain \MH. The optimal proposal standard deviation for
this sampler is $0.7$.

\paragraph{Metropolis Hastings Within Gibbs (\MHWG)}

Figure~\ref{fig:exp2_MHWG} shows the result of \MHWG sampling. This
sampler is better than \BMHWG and converges much more quickly. Here
a standard deviation of $0.9$ is found to be best.

\paragraph{Parallel Tempering (\PT)}

Figure~\ref{fig:exp2_PT} shows the results of \PT sampling with various
temperature combinations. Results show no improvement in AR from plain
\MH sampling and again $[1,3,27]$ temperature levels are found to be optimal.

\paragraph{Effect of Mixture Coefficient in Informed Sampling (\INFBMHWG)}

Figure~\ref{fig:exp2_alpha} shows the effect of mixture
coefficient ($\alpha$) on the blocked informed sampling
\INFBMHWG. Since there is no significant different in PSRF values for
$0 \le \alpha \le 0.8$, we chose $0.8$ due to its high acceptance
rate.



\subsection{Experiment: Estimating Body Shape}
\label{appendix:chap3:body}

\subsubsection{Parameter Selection}
\paragraph{Metropolis Hastings (\MH)}

Figure~\ref{fig:exp3_MH} shows the result of \MH~sampling with various
proposal standard deviations. The value of $0.1$ is found to be
best.

\paragraph{Metropolis Hastings Within Gibbs (\MHWG)}

For \MHWG sampling we select $0.3$ proposal standard
deviation. Results are shown in Fig.~\ref{fig:exp3_MHWG}.


\paragraph{Parallel Tempering (\PT)}

As before, results in Fig.~\ref{fig:exp3_PT}, the temperature levels
were selected to be $[1,3,27]$ due its slightly higher AR.

\paragraph{Effect of Mixture Coefficient in Informed Sampling (\MIXLMH)}

Figure~\ref{fig:exp3_alpha} shows the effect of $\alpha$ on PSRF and
AR. Since there is no significant differences in PSRF values for $0 \le
\alpha \le 0.8$, we choose $0.8$.


\begin{figure}[t]
\centering
  \subfigure[\MH]{%
    \includegraphics[width=.48\textwidth]{figures/supplementary/bodyShape_MH.pdf} \label{fig:exp3_MH}
  }
  \subfigure[\MHWG]{%
    \includegraphics[width=.48\textwidth]{figures/supplementary/bodyShape_MHWG.pdf} \label{fig:exp3_MHWG}
  }
\\
  \subfigure[\PT]{%
    \includegraphics[width=.48\textwidth]{figures/supplementary/bodyShape_PT.pdf} \label{fig:exp3_PT}
  }
  \subfigure[\MIXLMH]{%
    \includegraphics[width=.48\textwidth]{figures/supplementary/bodyShape_alpha.pdf} \label{fig:exp3_alpha}
  }
\\
  \mycaption{Results of the `Body Shape Estimation' experiment}{PRSFs and
    Acceptance rates corresponding to various standard deviations of
    (a) \MH, (b) \MHWG; (c) various temperature level combinations
    of \PT sampling and; (d) various mixture coefficients of the
    informed \MIXLMH sampling.}
\end{figure}


\subsection{Results Overview}
Figure~\ref{fig:exp_summary} shows the summary results of the all the three
experimental studies related to informed sampler.
\begin{figure*}[h!]
\centering
  \subfigure[Results for: Estimating Camera Extrinsics]{%
    \includegraphics[width=0.9\textwidth]{figures/supplementary/camPose_ALL.pdf} \label{fig:exp1_all}
  }
  \subfigure[Results for: Occluding Tiles]{%
    \includegraphics[width=0.9\textwidth]{figures/supplementary/occlusionExp_ALL.pdf} \label{fig:exp2_all}
  }
  \subfigure[Results for: Estimating Body Shape]{%
    \includegraphics[width=0.9\textwidth]{figures/supplementary/bodyShape_ALL.pdf} \label{fig:exp3_all}
  }
  \label{fig:exp_summary}
  \mycaption{Summary of the statistics for the three experiments}{Shown are
    for several baseline methods and the informed samplers the
    acceptance rates (left), PSRFs (middle), and RMSE values
    (right). All results are median results over multiple test
    examples.}
\end{figure*}

\subsection{Additional Qualitative Results}

\subsubsection{Occluding Tiles}
In Figure~\ref{fig:exp2_visual_more} more qualitative results of the
occluding tiles experiment are shown. The informed sampling approach
(\INFBMHWG) is better than the best baseline (\MHWG). This still is a
very challenging problem since the parameters for occluded tiles are
flat over a large region. Some of the posterior variance of the
occluded tiles is already captured by the informed sampler.

\begin{figure*}[h!]
\begin{center}
\centerline{\includegraphics[width=0.95\textwidth]{figures/supplementary/occlusionExp_Visual.pdf}}
\mycaption{Additional qualitative results of the occluding tiles experiment}
  {From left to right: (a)
  Given image, (b) Ground truth tiles, (c) OpenCV heuristic and most probable estimates
  from 5000 samples obtained by (d) MHWG sampler (best baseline) and
  (e) our INF-BMHWG sampler. (f) Posterior expectation of the tiles
  boundaries obtained by INF-BMHWG sampling (First 2000 samples are
  discarded as burn-in).}
\label{fig:exp2_visual_more}
\end{center}
\end{figure*}

\subsubsection{Body Shape}
Figure~\ref{fig:exp3_bodyMeshes} shows some more results of 3D mesh
reconstruction using posterior samples obtained by our informed
sampling \MIXLMH.

\begin{figure*}[t]
\begin{center}
\centerline{\includegraphics[width=0.75\textwidth]{figures/supplementary/bodyMeshResults.pdf}}
\mycaption{Qualitative results for the body shape experiment}
  {Shown is the 3D mesh reconstruction results with first 1000 samples obtained
  using the \MIXLMH informed sampling method. (blue indicates small
  values and red indicates high values)}
\label{fig:exp3_bodyMeshes}
\end{center}
\end{figure*}

\clearpage



\section{Additional Results on the Face Problem with CMP}

Figure~\ref{fig:shading-qualitative-multiple-subjects-supp} shows inference results for reflectance maps, normal maps and lights for randomly chosen test images, and Fig.~\ref{fig:shading-qualitative-same-subject-supp} shows reflectance estimation results on multiple images of the same subject produced under different illumination conditions. CMP is able to produce estimates that are closer to the groundtruth across different subjects and illumination conditions.

\begin{figure*}[h]
  \begin{center}
  \centerline{\includegraphics[width=1.0\columnwidth]{figures/face_cmp_visual_results_supp.pdf}}
  \vspace{-1.2cm}
  \end{center}
	\mycaption{A visual comparison of inference results}{(a)~Observed images. (b)~Inferred reflectance maps. \textit{GT} is the photometric stereo groundtruth, \textit{BU} is the Biswas \etal (2009) reflectance estimate and \textit{Forest} is the consensus prediction. (c)~The variance of the inferred reflectance estimate produced by \MTD (normalized across rows).(d)~Visualization of inferred light directions. (e)~Inferred normal maps.}
	\label{fig:shading-qualitative-multiple-subjects-supp}
\end{figure*}


\begin{figure*}[h]
	\centering
	\setlength\fboxsep{0.2mm}
	\setlength\fboxrule{0pt}
	\begin{tikzpicture}

		\matrix at (0, 0) [matrix of nodes, nodes={anchor=east}, column sep=-0.05cm, row sep=-0.2cm]
		{
			\fbox{\includegraphics[width=1cm]{figures/sample_3_4_X.png}} &
			\fbox{\includegraphics[width=1cm]{figures/sample_3_4_GT.png}} &
			\fbox{\includegraphics[width=1cm]{figures/sample_3_4_BISWAS.png}}  &
			\fbox{\includegraphics[width=1cm]{figures/sample_3_4_VMP.png}}  &
			\fbox{\includegraphics[width=1cm]{figures/sample_3_4_FOREST.png}}  &
			\fbox{\includegraphics[width=1cm]{figures/sample_3_4_CMP.png}}  &
			\fbox{\includegraphics[width=1cm]{figures/sample_3_4_CMPVAR.png}}
			 \\

			\fbox{\includegraphics[width=1cm]{figures/sample_3_5_X.png}} &
			\fbox{\includegraphics[width=1cm]{figures/sample_3_5_GT.png}} &
			\fbox{\includegraphics[width=1cm]{figures/sample_3_5_BISWAS.png}}  &
			\fbox{\includegraphics[width=1cm]{figures/sample_3_5_VMP.png}}  &
			\fbox{\includegraphics[width=1cm]{figures/sample_3_5_FOREST.png}}  &
			\fbox{\includegraphics[width=1cm]{figures/sample_3_5_CMP.png}}  &
			\fbox{\includegraphics[width=1cm]{figures/sample_3_5_CMPVAR.png}}
			 \\

			\fbox{\includegraphics[width=1cm]{figures/sample_3_6_X.png}} &
			\fbox{\includegraphics[width=1cm]{figures/sample_3_6_GT.png}} &
			\fbox{\includegraphics[width=1cm]{figures/sample_3_6_BISWAS.png}}  &
			\fbox{\includegraphics[width=1cm]{figures/sample_3_6_VMP.png}}  &
			\fbox{\includegraphics[width=1cm]{figures/sample_3_6_FOREST.png}}  &
			\fbox{\includegraphics[width=1cm]{figures/sample_3_6_CMP.png}}  &
			\fbox{\includegraphics[width=1cm]{figures/sample_3_6_CMPVAR.png}}
			 \\
	     };

       \node at (-3.85, -2.0) {\small Observed};
       \node at (-2.55, -2.0) {\small `GT'};
       \node at (-1.27, -2.0) {\small BU};
       \node at (0.0, -2.0) {\small MP};
       \node at (1.27, -2.0) {\small Forest};
       \node at (2.55, -2.0) {\small \textbf{CMP}};
       \node at (3.85, -2.0) {\small Variance};

	\end{tikzpicture}
	\mycaption{Robustness to varying illumination}{Reflectance estimation on a subject images with varying illumination. Left to right: observed image, photometric stereo estimate (GT)
  which is used as a proxy for groundtruth, bottom-up estimate of \cite{Biswas2009}, VMP result, consensus forest estimate, CMP mean, and CMP variance.}
	\label{fig:shading-qualitative-same-subject-supp}
\end{figure*}

\clearpage

\section{Additional Material for Learning Sparse High Dimensional Filters}
\label{sec:appendix-bnn}

This part of supplementary material contains a more detailed overview of the permutohedral
lattice convolution in Section~\ref{sec:permconv}, more experiments in
Section~\ref{sec:addexps} and additional results with protocols for
the experiments presented in Chapter~\ref{chap:bnn} in Section~\ref{sec:addresults}.

\vspace{-0.2cm}
\subsection{General Permutohedral Convolutions}
\label{sec:permconv}

A core technical contribution of this work is the generalization of the Gaussian permutohedral lattice
convolution proposed in~\cite{adams2010fast} to the full non-separable case with the
ability to perform back-propagation. Although, conceptually, there are minor
differences between Gaussian and general parameterized filters, there are non-trivial practical
differences in terms of the algorithmic implementation. The Gauss filters belong to
the separable class and can thus be decomposed into multiple
sequential one dimensional convolutions. We are interested in the general filter
convolutions, which can not be decomposed. Thus, performing a general permutohedral
convolution at a lattice point requires the computation of the inner product with the
neighboring elements in all the directions in the high-dimensional space.

Here, we give more details of the implementation differences of separable
and non-separable filters. In the following, we will explain the scalar case first.
Recall, that the forward pass of general permutohedral convolution
involves 3 steps: \textit{splatting}, \textit{convolving} and \textit{slicing}.
We follow the same splatting and slicing strategies as in~\cite{adams2010fast}
since these operations do not depend on the filter kernel. The main difference
between our work and the existing implementation of~\cite{adams2010fast} is
the way that the convolution operation is executed. This proceeds by constructing
a \emph{blur neighbor} matrix $K$ that stores for every lattice point all
values of the lattice neighbors that are needed to compute the filter output.

\begin{figure}[t!]
  \centering
    \includegraphics[width=0.6\columnwidth]{figures/supplementary/lattice_construction}
  \mycaption{Illustration of 1D permutohedral lattice construction}
  {A $4\times 4$ $(x,y)$ grid lattice is projected onto the plane defined by the normal
  vector $(1,1)^{\top}$. This grid has $s+1=4$ and $d=2$ $(s+1)^{d}=4^2=16$ elements.
  In the projection, all points of the same color are projected onto the same points in the plane.
  The number of elements of the projected lattice is $t=(s+1)^d-s^d=4^2-3^2=7$, that is
  the $(4\times 4)$ grid minus the size of lattice that is $1$ smaller at each size, in this
  case a $(3\times 3)$ lattice (the upper right $(3\times 3)$ elements).
  }
\label{fig:latticeconstruction}
\end{figure}

The blur neighbor matrix is constructed by traversing through all the populated
lattice points and their neighboring elements.
% For efficiency, we do this matrix construction recursively with shared computations
% since $n^{th}$ neighbourhood elements are $1^{st}$ neighborhood elements of $n-1^{th}$ neighbourhood elements. \pg{do not understand}
This is done recursively to share computations. For any lattice point, the neighbors that are
$n$ hops away are the direct neighbors of the points that are $n-1$ hops away.
The size of a $d$ dimensional spatial filter with width $s+1$ is $(s+1)^{d}$ (\eg, a
$3\times 3$ filter, $s=2$ in $d=2$ has $3^2=9$ elements) and this size grows
exponentially in the number of dimensions $d$. The permutohedral lattice is constructed by
projecting a regular grid onto the plane spanned by the $d$ dimensional normal vector ${(1,\ldots,1)}^{\top}$. See
Fig.~\ref{fig:latticeconstruction} for an illustration of the 1D lattice construction.
Many corners of a grid filter are projected onto the same point, in total $t = {(s+1)}^{d} -
s^{d}$ elements remain in the permutohedral filter with $s$ neighborhood in $d-1$ dimensions.
If the lattice has $m$ populated elements, the
matrix $K$ has size $t\times m$. Note that, since the input signal is typically
sparse, only a few lattice corners are being populated in the \textit{slicing} step.
We use a hash-table to keep track of these points and traverse only through
the populated lattice points for this neighborhood matrix construction.

Once the blur neighbor matrix $K$ is constructed, we can perform the convolution
by the matrix vector multiplication
\begin{equation}
\ell' = BK,
\label{eq:conv}
\end{equation}
where $B$ is the $1 \times t$ filter kernel (whose values we will learn) and $\ell'\in\mathbb{R}^{1\times m}$
is the result of the filtering at the $m$ lattice points. In practice, we found that the
matrix $K$ is sometimes too large to fit into GPU memory and we divided the matrix $K$
into smaller pieces to compute Eq.~\ref{eq:conv} sequentially.

In the general multi-dimensional case, the signal $\ell$ is of $c$ dimensions. Then
the kernel $B$ is of size $c \times t$ and $K$ stores the $c$ dimensional vectors
accordingly. When the input and output points are different, we slice only the
input points and splat only at the output points.


\subsection{Additional Experiments}
\label{sec:addexps}
In this section, we discuss more use-cases for the learned bilateral filters, one
use-case of BNNs and two single filter applications for image and 3D mesh denoising.

\subsubsection{Recognition of subsampled MNIST}\label{sec:app_mnist}

One of the strengths of the proposed filter convolution is that it does not
require the input to lie on a regular grid. The only requirement is to define a distance
between features of the input signal.
We highlight this feature with the following experiment using the
classical MNIST ten class classification problem~\cite{lecun1998mnist}. We sample a
sparse set of $N$ points $(x,y)\in [0,1]\times [0,1]$
uniformly at random in the input image, use their interpolated values
as signal and the \emph{continuous} $(x,y)$ positions as features. This mimics
sub-sampling of a high-dimensional signal. To compare against a spatial convolution,
we interpolate the sparse set of values at the grid positions.

We take a reference implementation of LeNet~\cite{lecun1998gradient} that
is part of the Caffe project~\cite{jia2014caffe} and compare it
against the same architecture but replacing the first convolutional
layer with a bilateral convolution layer (BCL). The filter size
and numbers are adjusted to get a comparable number of parameters
($5\times 5$ for LeNet, $2$-neighborhood for BCL).

The results are shown in Table~\ref{tab:all-results}. We see that training
on the original MNIST data (column Original, LeNet vs. BNN) leads to a slight
decrease in performance of the BNN (99.03\%) compared to LeNet
(99.19\%). The BNN can be trained and evaluated on sparse
signals, and we resample the image as described above for $N=$ 100\%, 60\% and
20\% of the total number of pixels. The methods are also evaluated
on test images that are subsampled in the same way. Note that we can
train and test with different subsampling rates. We introduce an additional
bilinear interpolation layer for the LeNet architecture to train on the same
data. In essence, both models perform a spatial interpolation and thus we
expect them to yield a similar classification accuracy. Once the data is of
higher dimensions, the permutohedral convolution will be faster due to hashing
the sparse input points, as well as less memory demanding in comparison to
naive application of a spatial convolution with interpolated values.

\begin{table}[t]
  \begin{center}
    \footnotesize
    \centering
    \begin{tabular}[t]{lllll}
      \toprule
              &     & \multicolumn{3}{c}{Test Subsampling} \\
       Method  & Original & 100\% & 60\% & 20\%\\
      \midrule
       LeNet &  \textbf{0.9919} & 0.9660 & 0.9348 & \textbf{0.6434} \\
       BNN &  0.9903 & \textbf{0.9844} & \textbf{0.9534} & 0.5767 \\
      \hline
       LeNet 100\% & 0.9856 & 0.9809 & 0.9678 & \textbf{0.7386} \\
       BNN 100\% & \textbf{0.9900} & \textbf{0.9863} & \textbf{0.9699} & 0.6910 \\
      \hline
       LeNet 60\% & 0.9848 & 0.9821 & 0.9740 & 0.8151 \\
       BNN 60\% & \textbf{0.9885} & \textbf{0.9864} & \textbf{0.9771} & \textbf{0.8214}\\
      \hline
       LeNet 20\% & \textbf{0.9763} & \textbf{0.9754} & 0.9695 & 0.8928 \\
       BNN 20\% & 0.9728 & 0.9735 & \textbf{0.9701} & \textbf{0.9042}\\
      \bottomrule
    \end{tabular}
  \end{center}
\vspace{-.2cm}
\caption{Classification accuracy on MNIST. We compare the
    LeNet~\cite{lecun1998gradient} implementation that is part of
    Caffe~\cite{jia2014caffe} to the network with the first layer
    replaced by a bilateral convolution layer (BCL). Both are trained
    on the original image resolution (first two rows). Three more BNN
    and CNN models are trained with randomly subsampled images (100\%,
    60\% and 20\% of the pixels). An additional bilinear interpolation
    layer samples the input signal on a spatial grid for the CNN model.
  }
  \label{tab:all-results}
\vspace{-.5cm}
\end{table}

\subsubsection{Image Denoising}

The main application that inspired the development of the bilateral
filtering operation is image denoising~\cite{aurich1995non}, there
using a single Gaussian kernel. Our development allows to learn this
kernel function from data and we explore how to improve using a \emph{single}
but more general bilateral filter.

We use the Berkeley segmentation dataset
(BSDS500)~\cite{arbelaezi2011bsds500} as a test bed. The color
images in the dataset are converted to gray-scale,
and corrupted with Gaussian noise with a standard deviation of
$\frac {25} {255}$.

We compare the performance of four different filter models on a
denoising task.
The first baseline model (`Spatial' in Table \ref{tab:denoising}, $25$
weights) uses a single spatial filter with a kernel size of
$5$ and predicts the scalar gray-scale value at the center pixel. The next model
(`Gauss Bilateral') applies a bilateral \emph{Gaussian}
filter to the noisy input, using position and intensity features $\f=(x,y,v)^\top$.
The third setup (`Learned Bilateral', $65$ weights)
takes a Gauss kernel as initialization and
fits all filter weights on the train set to minimize the
mean squared error with respect to the clean images.
We run a combination
of spatial and permutohedral convolutions on spatial and bilateral
features (`Spatial + Bilateral (Learned)') to check for a complementary
performance of the two convolutions.

\label{sec:exp:denoising}
\begin{table}[!h]
\begin{center}
  \footnotesize
  \begin{tabular}[t]{lr}
    \toprule
    Method & PSNR \\
    \midrule
    Noisy Input & $20.17$ \\
    Spatial & $26.27$ \\
    Gauss Bilateral & $26.51$ \\
    Learned Bilateral & $26.58$ \\
    Spatial + Bilateral (Learned) & \textbf{$26.65$} \\
    \bottomrule
  \end{tabular}
\end{center}
\vspace{-0.5em}
\caption{PSNR results of a denoising task using the BSDS500
  dataset~\cite{arbelaezi2011bsds500}}
\vspace{-0.5em}
\label{tab:denoising}
\end{table}
\vspace{-0.2em}

The PSNR scores evaluated on full images of the test set are
shown in Table \ref{tab:denoising}. We find that an untrained bilateral
filter already performs better than a trained spatial convolution
($26.27$ to $26.51$). A learned convolution further improve the
performance slightly. We chose this simple one-kernel setup to
validate an advantage of the generalized bilateral filter. A competitive
denoising system would employ RGB color information and also
needs to be properly adjusted in network size. Multi-layer perceptrons
have obtained state-of-the-art denoising results~\cite{burger12cvpr}
and the permutohedral lattice layer can readily be used in such an
architecture, which is intended future work.

\subsection{Additional results}
\label{sec:addresults}

This section contains more qualitative results for the experiments presented in Chapter~\ref{chap:bnn}.

\begin{figure*}[th!]
  \centering
    \includegraphics[width=\columnwidth,trim={5cm 2.5cm 5cm 4.5cm},clip]{figures/supplementary/lattice_viz.pdf}
    \vspace{-0.7cm}
  \mycaption{Visualization of the Permutohedral Lattice}
  {Sample lattice visualizations for different feature spaces. All pixels falling in the same simplex cell are shown with
  the same color. $(x,y)$ features correspond to image pixel positions, and $(r,g,b) \in [0,255]$ correspond
  to the red, green and blue color values.}
\label{fig:latticeviz}
\end{figure*}

\subsubsection{Lattice Visualization}

Figure~\ref{fig:latticeviz} shows sample lattice visualizations for different feature spaces.

\newcolumntype{L}[1]{>{\raggedright\let\newline\\\arraybackslash\hspace{0pt}}b{#1}}
\newcolumntype{C}[1]{>{\centering\let\newline\\\arraybackslash\hspace{0pt}}b{#1}}
\newcolumntype{R}[1]{>{\raggedleft\let\newline\\\arraybackslash\hspace{0pt}}b{#1}}

\subsubsection{Color Upsampling}\label{sec:color_upsampling}
\label{sec:col_upsample_extra}

Some images of the upsampling for the Pascal
VOC12 dataset are shown in Fig.~\ref{fig:Colour_upsample_visuals}. It is
especially the low level image details that are better preserved with
a learned bilateral filter compared to the Gaussian case.

\begin{figure*}[t!]
  \centering
    \subfigure{%
   \raisebox{2.0em}{
    \includegraphics[width=.06\columnwidth]{figures/supplementary/2007_004969.jpg}
   }
  }
  \subfigure{%
    \includegraphics[width=.17\columnwidth]{figures/supplementary/2007_004969_gray.pdf}
  }
  \subfigure{%
    \includegraphics[width=.17\columnwidth]{figures/supplementary/2007_004969_gt.pdf}
  }
  \subfigure{%
    \includegraphics[width=.17\columnwidth]{figures/supplementary/2007_004969_bicubic.pdf}
  }
  \subfigure{%
    \includegraphics[width=.17\columnwidth]{figures/supplementary/2007_004969_gauss.pdf}
  }
  \subfigure{%
    \includegraphics[width=.17\columnwidth]{figures/supplementary/2007_004969_learnt.pdf}
  }\\
    \subfigure{%
   \raisebox{2.0em}{
    \includegraphics[width=.06\columnwidth]{figures/supplementary/2007_003106.jpg}
   }
  }
  \subfigure{%
    \includegraphics[width=.17\columnwidth]{figures/supplementary/2007_003106_gray.pdf}
  }
  \subfigure{%
    \includegraphics[width=.17\columnwidth]{figures/supplementary/2007_003106_gt.pdf}
  }
  \subfigure{%
    \includegraphics[width=.17\columnwidth]{figures/supplementary/2007_003106_bicubic.pdf}
  }
  \subfigure{%
    \includegraphics[width=.17\columnwidth]{figures/supplementary/2007_003106_gauss.pdf}
  }
  \subfigure{%
    \includegraphics[width=.17\columnwidth]{figures/supplementary/2007_003106_learnt.pdf}
  }\\
  \setcounter{subfigure}{0}
  \small{
  \subfigure[Inp.]{%
  \raisebox{2.0em}{
    \includegraphics[width=.06\columnwidth]{figures/supplementary/2007_006837.jpg}
   }
  }
  \subfigure[Guidance]{%
    \includegraphics[width=.17\columnwidth]{figures/supplementary/2007_006837_gray.pdf}
  }
   \subfigure[GT]{%
    \includegraphics[width=.17\columnwidth]{figures/supplementary/2007_006837_gt.pdf}
  }
  \subfigure[Bicubic]{%
    \includegraphics[width=.17\columnwidth]{figures/supplementary/2007_006837_bicubic.pdf}
  }
  \subfigure[Gauss-BF]{%
    \includegraphics[width=.17\columnwidth]{figures/supplementary/2007_006837_gauss.pdf}
  }
  \subfigure[Learned-BF]{%
    \includegraphics[width=.17\columnwidth]{figures/supplementary/2007_006837_learnt.pdf}
  }
  }
  \vspace{-0.5cm}
  \mycaption{Color Upsampling}{Color $8\times$ upsampling results
  using different methods, from left to right, (a)~Low-resolution input color image (Inp.),
  (b)~Gray scale guidance image, (c)~Ground-truth color image; Upsampled color images with
  (d)~Bicubic interpolation, (e) Gauss bilateral upsampling and, (f)~Learned bilateral
  updampgling (best viewed on screen).}

\label{fig:Colour_upsample_visuals}
\end{figure*}

\subsubsection{Depth Upsampling}
\label{sec:depth_upsample_extra}

Figure~\ref{fig:depth_upsample_visuals} presents some more qualitative results comparing bicubic interpolation, Gauss
bilateral and learned bilateral upsampling on NYU depth dataset image~\cite{silberman2012indoor}.

\subsubsection{Character Recognition}\label{sec:app_character}

 Figure~\ref{fig:nnrecognition} shows the schematic of different layers
 of the network architecture for LeNet-7~\cite{lecun1998mnist}
 and DeepCNet(5, 50)~\cite{ciresan2012multi,graham2014spatially}. For the BNN variants, the first layer filters are replaced
 with learned bilateral filters and are learned end-to-end.

\subsubsection{Semantic Segmentation}\label{sec:app_semantic_segmentation}
\label{sec:semantic_bnn_extra}

Some more visual results for semantic segmentation are shown in Figure~\ref{fig:semantic_visuals}.
These include the underlying DeepLab CNN\cite{chen2014semantic} result (DeepLab),
the 2 step mean-field result with Gaussian edge potentials (+2stepMF-GaussCRF)
and also corresponding results with learned edge potentials (+2stepMF-LearnedCRF).
In general, we observe that mean-field in learned CRF leads to slightly dilated
classification regions in comparison to using Gaussian CRF thereby filling-in the
false negative pixels and also correcting some mis-classified regions.

\begin{figure*}[t!]
  \centering
    \subfigure{%
   \raisebox{2.0em}{
    \includegraphics[width=.06\columnwidth]{figures/supplementary/2bicubic}
   }
  }
  \subfigure{%
    \includegraphics[width=.17\columnwidth]{figures/supplementary/2given_image}
  }
  \subfigure{%
    \includegraphics[width=.17\columnwidth]{figures/supplementary/2ground_truth}
  }
  \subfigure{%
    \includegraphics[width=.17\columnwidth]{figures/supplementary/2bicubic}
  }
  \subfigure{%
    \includegraphics[width=.17\columnwidth]{figures/supplementary/2gauss}
  }
  \subfigure{%
    \includegraphics[width=.17\columnwidth]{figures/supplementary/2learnt}
  }\\
    \subfigure{%
   \raisebox{2.0em}{
    \includegraphics[width=.06\columnwidth]{figures/supplementary/32bicubic}
   }
  }
  \subfigure{%
    \includegraphics[width=.17\columnwidth]{figures/supplementary/32given_image}
  }
  \subfigure{%
    \includegraphics[width=.17\columnwidth]{figures/supplementary/32ground_truth}
  }
  \subfigure{%
    \includegraphics[width=.17\columnwidth]{figures/supplementary/32bicubic}
  }
  \subfigure{%
    \includegraphics[width=.17\columnwidth]{figures/supplementary/32gauss}
  }
  \subfigure{%
    \includegraphics[width=.17\columnwidth]{figures/supplementary/32learnt}
  }\\
  \setcounter{subfigure}{0}
  \small{
  \subfigure[Inp.]{%
  \raisebox{2.0em}{
    \includegraphics[width=.06\columnwidth]{figures/supplementary/41bicubic}
   }
  }
  \subfigure[Guidance]{%
    \includegraphics[width=.17\columnwidth]{figures/supplementary/41given_image}
  }
   \subfigure[GT]{%
    \includegraphics[width=.17\columnwidth]{figures/supplementary/41ground_truth}
  }
  \subfigure[Bicubic]{%
    \includegraphics[width=.17\columnwidth]{figures/supplementary/41bicubic}
  }
  \subfigure[Gauss-BF]{%
    \includegraphics[width=.17\columnwidth]{figures/supplementary/41gauss}
  }
  \subfigure[Learned-BF]{%
    \includegraphics[width=.17\columnwidth]{figures/supplementary/41learnt}
  }
  }
  \mycaption{Depth Upsampling}{Depth $8\times$ upsampling results
  using different upsampling strategies, from left to right,
  (a)~Low-resolution input depth image (Inp.),
  (b)~High-resolution guidance image, (c)~Ground-truth depth; Upsampled depth images with
  (d)~Bicubic interpolation, (e) Gauss bilateral upsampling and, (f)~Learned bilateral
  updampgling (best viewed on screen).}

\label{fig:depth_upsample_visuals}
\end{figure*}

\subsubsection{Material Segmentation}\label{sec:app_material_segmentation}
\label{sec:material_bnn_extra}

In Fig.~\ref{fig:material_visuals-app2}, we present visual results comparing 2 step
mean-field inference with Gaussian and learned pairwise CRF potentials. In
general, we observe that the pixels belonging to dominant classes in the
training data are being more accurately classified with learned CRF. This leads to
a significant improvements in overall pixel accuracy. This also results
in a slight decrease of the accuracy from less frequent class pixels thereby
slightly reducing the average class accuracy with learning. We attribute this
to the type of annotation that is available for this dataset, which is not
for the entire image but for some segments in the image. We have very few
images of the infrequent classes to combat this behaviour during training.

\subsubsection{Experiment Protocols}
\label{sec:protocols}

Table~\ref{tbl:parameters} shows experiment protocols of different experiments.

 \begin{figure*}[t!]
  \centering
  \subfigure[LeNet-7]{
    \includegraphics[width=0.7\columnwidth]{figures/supplementary/lenet_cnn_network}
    }\\
    \subfigure[DeepCNet]{
    \includegraphics[width=\columnwidth]{figures/supplementary/deepcnet_cnn_network}
    }
  \mycaption{CNNs for Character Recognition}
  {Schematic of (top) LeNet-7~\cite{lecun1998mnist} and (bottom) DeepCNet(5,50)~\cite{ciresan2012multi,graham2014spatially} architectures used in Assamese
  character recognition experiments.}
\label{fig:nnrecognition}
\end{figure*}

\definecolor{voc_1}{RGB}{0, 0, 0}
\definecolor{voc_2}{RGB}{128, 0, 0}
\definecolor{voc_3}{RGB}{0, 128, 0}
\definecolor{voc_4}{RGB}{128, 128, 0}
\definecolor{voc_5}{RGB}{0, 0, 128}
\definecolor{voc_6}{RGB}{128, 0, 128}
\definecolor{voc_7}{RGB}{0, 128, 128}
\definecolor{voc_8}{RGB}{128, 128, 128}
\definecolor{voc_9}{RGB}{64, 0, 0}
\definecolor{voc_10}{RGB}{192, 0, 0}
\definecolor{voc_11}{RGB}{64, 128, 0}
\definecolor{voc_12}{RGB}{192, 128, 0}
\definecolor{voc_13}{RGB}{64, 0, 128}
\definecolor{voc_14}{RGB}{192, 0, 128}
\definecolor{voc_15}{RGB}{64, 128, 128}
\definecolor{voc_16}{RGB}{192, 128, 128}
\definecolor{voc_17}{RGB}{0, 64, 0}
\definecolor{voc_18}{RGB}{128, 64, 0}
\definecolor{voc_19}{RGB}{0, 192, 0}
\definecolor{voc_20}{RGB}{128, 192, 0}
\definecolor{voc_21}{RGB}{0, 64, 128}
\definecolor{voc_22}{RGB}{128, 64, 128}

\begin{figure*}[t]
  \centering
  \small{
  \fcolorbox{white}{voc_1}{\rule{0pt}{6pt}\rule{6pt}{0pt}} Background~~
  \fcolorbox{white}{voc_2}{\rule{0pt}{6pt}\rule{6pt}{0pt}} Aeroplane~~
  \fcolorbox{white}{voc_3}{\rule{0pt}{6pt}\rule{6pt}{0pt}} Bicycle~~
  \fcolorbox{white}{voc_4}{\rule{0pt}{6pt}\rule{6pt}{0pt}} Bird~~
  \fcolorbox{white}{voc_5}{\rule{0pt}{6pt}\rule{6pt}{0pt}} Boat~~
  \fcolorbox{white}{voc_6}{\rule{0pt}{6pt}\rule{6pt}{0pt}} Bottle~~
  \fcolorbox{white}{voc_7}{\rule{0pt}{6pt}\rule{6pt}{0pt}} Bus~~
  \fcolorbox{white}{voc_8}{\rule{0pt}{6pt}\rule{6pt}{0pt}} Car~~ \\
  \fcolorbox{white}{voc_9}{\rule{0pt}{6pt}\rule{6pt}{0pt}} Cat~~
  \fcolorbox{white}{voc_10}{\rule{0pt}{6pt}\rule{6pt}{0pt}} Chair~~
  \fcolorbox{white}{voc_11}{\rule{0pt}{6pt}\rule{6pt}{0pt}} Cow~~
  \fcolorbox{white}{voc_12}{\rule{0pt}{6pt}\rule{6pt}{0pt}} Dining Table~~
  \fcolorbox{white}{voc_13}{\rule{0pt}{6pt}\rule{6pt}{0pt}} Dog~~
  \fcolorbox{white}{voc_14}{\rule{0pt}{6pt}\rule{6pt}{0pt}} Horse~~
  \fcolorbox{white}{voc_15}{\rule{0pt}{6pt}\rule{6pt}{0pt}} Motorbike~~
  \fcolorbox{white}{voc_16}{\rule{0pt}{6pt}\rule{6pt}{0pt}} Person~~ \\
  \fcolorbox{white}{voc_17}{\rule{0pt}{6pt}\rule{6pt}{0pt}} Potted Plant~~
  \fcolorbox{white}{voc_18}{\rule{0pt}{6pt}\rule{6pt}{0pt}} Sheep~~
  \fcolorbox{white}{voc_19}{\rule{0pt}{6pt}\rule{6pt}{0pt}} Sofa~~
  \fcolorbox{white}{voc_20}{\rule{0pt}{6pt}\rule{6pt}{0pt}} Train~~
  \fcolorbox{white}{voc_21}{\rule{0pt}{6pt}\rule{6pt}{0pt}} TV monitor~~ \\
  }
  \subfigure{%
    \includegraphics[width=.18\columnwidth]{figures/supplementary/2007_001423_given.jpg}
  }
  \subfigure{%
    \includegraphics[width=.18\columnwidth]{figures/supplementary/2007_001423_gt.png}
  }
  \subfigure{%
    \includegraphics[width=.18\columnwidth]{figures/supplementary/2007_001423_cnn.png}
  }
  \subfigure{%
    \includegraphics[width=.18\columnwidth]{figures/supplementary/2007_001423_gauss.png}
  }
  \subfigure{%
    \includegraphics[width=.18\columnwidth]{figures/supplementary/2007_001423_learnt.png}
  }\\
  \subfigure{%
    \includegraphics[width=.18\columnwidth]{figures/supplementary/2007_001430_given.jpg}
  }
  \subfigure{%
    \includegraphics[width=.18\columnwidth]{figures/supplementary/2007_001430_gt.png}
  }
  \subfigure{%
    \includegraphics[width=.18\columnwidth]{figures/supplementary/2007_001430_cnn.png}
  }
  \subfigure{%
    \includegraphics[width=.18\columnwidth]{figures/supplementary/2007_001430_gauss.png}
  }
  \subfigure{%
    \includegraphics[width=.18\columnwidth]{figures/supplementary/2007_001430_learnt.png}
  }\\
    \subfigure{%
    \includegraphics[width=.18\columnwidth]{figures/supplementary/2007_007996_given.jpg}
  }
  \subfigure{%
    \includegraphics[width=.18\columnwidth]{figures/supplementary/2007_007996_gt.png}
  }
  \subfigure{%
    \includegraphics[width=.18\columnwidth]{figures/supplementary/2007_007996_cnn.png}
  }
  \subfigure{%
    \includegraphics[width=.18\columnwidth]{figures/supplementary/2007_007996_gauss.png}
  }
  \subfigure{%
    \includegraphics[width=.18\columnwidth]{figures/supplementary/2007_007996_learnt.png}
  }\\
   \subfigure{%
    \includegraphics[width=.18\columnwidth]{figures/supplementary/2010_002682_given.jpg}
  }
  \subfigure{%
    \includegraphics[width=.18\columnwidth]{figures/supplementary/2010_002682_gt.png}
  }
  \subfigure{%
    \includegraphics[width=.18\columnwidth]{figures/supplementary/2010_002682_cnn.png}
  }
  \subfigure{%
    \includegraphics[width=.18\columnwidth]{figures/supplementary/2010_002682_gauss.png}
  }
  \subfigure{%
    \includegraphics[width=.18\columnwidth]{figures/supplementary/2010_002682_learnt.png}
  }\\
     \subfigure{%
    \includegraphics[width=.18\columnwidth]{figures/supplementary/2010_004789_given.jpg}
  }
  \subfigure{%
    \includegraphics[width=.18\columnwidth]{figures/supplementary/2010_004789_gt.png}
  }
  \subfigure{%
    \includegraphics[width=.18\columnwidth]{figures/supplementary/2010_004789_cnn.png}
  }
  \subfigure{%
    \includegraphics[width=.18\columnwidth]{figures/supplementary/2010_004789_gauss.png}
  }
  \subfigure{%
    \includegraphics[width=.18\columnwidth]{figures/supplementary/2010_004789_learnt.png}
  }\\
       \subfigure{%
    \includegraphics[width=.18\columnwidth]{figures/supplementary/2007_001311_given.jpg}
  }
  \subfigure{%
    \includegraphics[width=.18\columnwidth]{figures/supplementary/2007_001311_gt.png}
  }
  \subfigure{%
    \includegraphics[width=.18\columnwidth]{figures/supplementary/2007_001311_cnn.png}
  }
  \subfigure{%
    \includegraphics[width=.18\columnwidth]{figures/supplementary/2007_001311_gauss.png}
  }
  \subfigure{%
    \includegraphics[width=.18\columnwidth]{figures/supplementary/2007_001311_learnt.png}
  }\\
  \setcounter{subfigure}{0}
  \subfigure[Input]{%
    \includegraphics[width=.18\columnwidth]{figures/supplementary/2010_003531_given.jpg}
  }
  \subfigure[Ground Truth]{%
    \includegraphics[width=.18\columnwidth]{figures/supplementary/2010_003531_gt.png}
  }
  \subfigure[DeepLab]{%
    \includegraphics[width=.18\columnwidth]{figures/supplementary/2010_003531_cnn.png}
  }
  \subfigure[+GaussCRF]{%
    \includegraphics[width=.18\columnwidth]{figures/supplementary/2010_003531_gauss.png}
  }
  \subfigure[+LearnedCRF]{%
    \includegraphics[width=.18\columnwidth]{figures/supplementary/2010_003531_learnt.png}
  }
  \vspace{-0.3cm}
  \mycaption{Semantic Segmentation}{Example results of semantic segmentation.
  (c)~depicts the unary results before application of MF, (d)~after two steps of MF with Gaussian edge CRF potentials, (e)~after
  two steps of MF with learned edge CRF potentials.}
    \label{fig:semantic_visuals}
\end{figure*}


\definecolor{minc_1}{HTML}{771111}
\definecolor{minc_2}{HTML}{CAC690}
\definecolor{minc_3}{HTML}{EEEEEE}
\definecolor{minc_4}{HTML}{7C8FA6}
\definecolor{minc_5}{HTML}{597D31}
\definecolor{minc_6}{HTML}{104410}
\definecolor{minc_7}{HTML}{BB819C}
\definecolor{minc_8}{HTML}{D0CE48}
\definecolor{minc_9}{HTML}{622745}
\definecolor{minc_10}{HTML}{666666}
\definecolor{minc_11}{HTML}{D54A31}
\definecolor{minc_12}{HTML}{101044}
\definecolor{minc_13}{HTML}{444126}
\definecolor{minc_14}{HTML}{75D646}
\definecolor{minc_15}{HTML}{DD4348}
\definecolor{minc_16}{HTML}{5C8577}
\definecolor{minc_17}{HTML}{C78472}
\definecolor{minc_18}{HTML}{75D6D0}
\definecolor{minc_19}{HTML}{5B4586}
\definecolor{minc_20}{HTML}{C04393}
\definecolor{minc_21}{HTML}{D69948}
\definecolor{minc_22}{HTML}{7370D8}
\definecolor{minc_23}{HTML}{7A3622}
\definecolor{minc_24}{HTML}{000000}

\begin{figure*}[t]
  \centering
  \small{
  \fcolorbox{white}{minc_1}{\rule{0pt}{6pt}\rule{6pt}{0pt}} Brick~~
  \fcolorbox{white}{minc_2}{\rule{0pt}{6pt}\rule{6pt}{0pt}} Carpet~~
  \fcolorbox{white}{minc_3}{\rule{0pt}{6pt}\rule{6pt}{0pt}} Ceramic~~
  \fcolorbox{white}{minc_4}{\rule{0pt}{6pt}\rule{6pt}{0pt}} Fabric~~
  \fcolorbox{white}{minc_5}{\rule{0pt}{6pt}\rule{6pt}{0pt}} Foliage~~
  \fcolorbox{white}{minc_6}{\rule{0pt}{6pt}\rule{6pt}{0pt}} Food~~
  \fcolorbox{white}{minc_7}{\rule{0pt}{6pt}\rule{6pt}{0pt}} Glass~~
  \fcolorbox{white}{minc_8}{\rule{0pt}{6pt}\rule{6pt}{0pt}} Hair~~ \\
  \fcolorbox{white}{minc_9}{\rule{0pt}{6pt}\rule{6pt}{0pt}} Leather~~
  \fcolorbox{white}{minc_10}{\rule{0pt}{6pt}\rule{6pt}{0pt}} Metal~~
  \fcolorbox{white}{minc_11}{\rule{0pt}{6pt}\rule{6pt}{0pt}} Mirror~~
  \fcolorbox{white}{minc_12}{\rule{0pt}{6pt}\rule{6pt}{0pt}} Other~~
  \fcolorbox{white}{minc_13}{\rule{0pt}{6pt}\rule{6pt}{0pt}} Painted~~
  \fcolorbox{white}{minc_14}{\rule{0pt}{6pt}\rule{6pt}{0pt}} Paper~~
  \fcolorbox{white}{minc_15}{\rule{0pt}{6pt}\rule{6pt}{0pt}} Plastic~~\\
  \fcolorbox{white}{minc_16}{\rule{0pt}{6pt}\rule{6pt}{0pt}} Polished Stone~~
  \fcolorbox{white}{minc_17}{\rule{0pt}{6pt}\rule{6pt}{0pt}} Skin~~
  \fcolorbox{white}{minc_18}{\rule{0pt}{6pt}\rule{6pt}{0pt}} Sky~~
  \fcolorbox{white}{minc_19}{\rule{0pt}{6pt}\rule{6pt}{0pt}} Stone~~
  \fcolorbox{white}{minc_20}{\rule{0pt}{6pt}\rule{6pt}{0pt}} Tile~~
  \fcolorbox{white}{minc_21}{\rule{0pt}{6pt}\rule{6pt}{0pt}} Wallpaper~~
  \fcolorbox{white}{minc_22}{\rule{0pt}{6pt}\rule{6pt}{0pt}} Water~~
  \fcolorbox{white}{minc_23}{\rule{0pt}{6pt}\rule{6pt}{0pt}} Wood~~ \\
  }
  \subfigure{%
    \includegraphics[width=.18\columnwidth]{figures/supplementary/000010868_given.jpg}
  }
  \subfigure{%
    \includegraphics[width=.18\columnwidth]{figures/supplementary/000010868_gt.png}
  }
  \subfigure{%
    \includegraphics[width=.18\columnwidth]{figures/supplementary/000010868_cnn.png}
  }
  \subfigure{%
    \includegraphics[width=.18\columnwidth]{figures/supplementary/000010868_gauss.png}
  }
  \subfigure{%
    \includegraphics[width=.18\columnwidth]{figures/supplementary/000010868_learnt.png}
  }\\[-2ex]
  \subfigure{%
    \includegraphics[width=.18\columnwidth]{figures/supplementary/000006011_given.jpg}
  }
  \subfigure{%
    \includegraphics[width=.18\columnwidth]{figures/supplementary/000006011_gt.png}
  }
  \subfigure{%
    \includegraphics[width=.18\columnwidth]{figures/supplementary/000006011_cnn.png}
  }
  \subfigure{%
    \includegraphics[width=.18\columnwidth]{figures/supplementary/000006011_gauss.png}
  }
  \subfigure{%
    \includegraphics[width=.18\columnwidth]{figures/supplementary/000006011_learnt.png}
  }\\[-2ex]
    \subfigure{%
    \includegraphics[width=.18\columnwidth]{figures/supplementary/000008553_given.jpg}
  }
  \subfigure{%
    \includegraphics[width=.18\columnwidth]{figures/supplementary/000008553_gt.png}
  }
  \subfigure{%
    \includegraphics[width=.18\columnwidth]{figures/supplementary/000008553_cnn.png}
  }
  \subfigure{%
    \includegraphics[width=.18\columnwidth]{figures/supplementary/000008553_gauss.png}
  }
  \subfigure{%
    \includegraphics[width=.18\columnwidth]{figures/supplementary/000008553_learnt.png}
  }\\[-2ex]
   \subfigure{%
    \includegraphics[width=.18\columnwidth]{figures/supplementary/000009188_given.jpg}
  }
  \subfigure{%
    \includegraphics[width=.18\columnwidth]{figures/supplementary/000009188_gt.png}
  }
  \subfigure{%
    \includegraphics[width=.18\columnwidth]{figures/supplementary/000009188_cnn.png}
  }
  \subfigure{%
    \includegraphics[width=.18\columnwidth]{figures/supplementary/000009188_gauss.png}
  }
  \subfigure{%
    \includegraphics[width=.18\columnwidth]{figures/supplementary/000009188_learnt.png}
  }\\[-2ex]
  \setcounter{subfigure}{0}
  \subfigure[Input]{%
    \includegraphics[width=.18\columnwidth]{figures/supplementary/000023570_given.jpg}
  }
  \subfigure[Ground Truth]{%
    \includegraphics[width=.18\columnwidth]{figures/supplementary/000023570_gt.png}
  }
  \subfigure[DeepLab]{%
    \includegraphics[width=.18\columnwidth]{figures/supplementary/000023570_cnn.png}
  }
  \subfigure[+GaussCRF]{%
    \includegraphics[width=.18\columnwidth]{figures/supplementary/000023570_gauss.png}
  }
  \subfigure[+LearnedCRF]{%
    \includegraphics[width=.18\columnwidth]{figures/supplementary/000023570_learnt.png}
  }
  \mycaption{Material Segmentation}{Example results of material segmentation.
  (c)~depicts the unary results before application of MF, (d)~after two steps of MF with Gaussian edge CRF potentials, (e)~after two steps of MF with learned edge CRF potentials.}
    \label{fig:material_visuals-app2}
\end{figure*}


\begin{table*}[h]
\tiny
  \centering
    \begin{tabular}{L{2.3cm} L{2.25cm} C{1.5cm} C{0.7cm} C{0.6cm} C{0.7cm} C{0.7cm} C{0.7cm} C{1.6cm} C{0.6cm} C{0.6cm} C{0.6cm}}
      \toprule
& & & & & \multicolumn{3}{c}{\textbf{Data Statistics}} & \multicolumn{4}{c}{\textbf{Training Protocol}} \\

\textbf{Experiment} & \textbf{Feature Types} & \textbf{Feature Scales} & \textbf{Filter Size} & \textbf{Filter Nbr.} & \textbf{Train}  & \textbf{Val.} & \textbf{Test} & \textbf{Loss Type} & \textbf{LR} & \textbf{Batch} & \textbf{Epochs} \\
      \midrule
      \multicolumn{2}{c}{\textbf{Single Bilateral Filter Applications}} & & & & & & & & & \\
      \textbf{2$\times$ Color Upsampling} & Position$_{1}$, Intensity (3D) & 0.13, 0.17 & 65 & 2 & 10581 & 1449 & 1456 & MSE & 1e-06 & 200 & 94.5\\
      \textbf{4$\times$ Color Upsampling} & Position$_{1}$, Intensity (3D) & 0.06, 0.17 & 65 & 2 & 10581 & 1449 & 1456 & MSE & 1e-06 & 200 & 94.5\\
      \textbf{8$\times$ Color Upsampling} & Position$_{1}$, Intensity (3D) & 0.03, 0.17 & 65 & 2 & 10581 & 1449 & 1456 & MSE & 1e-06 & 200 & 94.5\\
      \textbf{16$\times$ Color Upsampling} & Position$_{1}$, Intensity (3D) & 0.02, 0.17 & 65 & 2 & 10581 & 1449 & 1456 & MSE & 1e-06 & 200 & 94.5\\
      \textbf{Depth Upsampling} & Position$_{1}$, Color (5D) & 0.05, 0.02 & 665 & 2 & 795 & 100 & 654 & MSE & 1e-07 & 50 & 251.6\\
      \textbf{Mesh Denoising} & Isomap (4D) & 46.00 & 63 & 2 & 1000 & 200 & 500 & MSE & 100 & 10 & 100.0 \\
      \midrule
      \multicolumn{2}{c}{\textbf{DenseCRF Applications}} & & & & & & & & &\\
      \multicolumn{2}{l}{\textbf{Semantic Segmentation}} & & & & & & & & &\\
      \textbf{- 1step MF} & Position$_{1}$, Color (5D); Position$_{1}$ (2D) & 0.01, 0.34; 0.34  & 665; 19  & 2; 2 & 10581 & 1449 & 1456 & Logistic & 0.1 & 5 & 1.4 \\
      \textbf{- 2step MF} & Position$_{1}$, Color (5D); Position$_{1}$ (2D) & 0.01, 0.34; 0.34 & 665; 19 & 2; 2 & 10581 & 1449 & 1456 & Logistic & 0.1 & 5 & 1.4 \\
      \textbf{- \textit{loose} 2step MF} & Position$_{1}$, Color (5D); Position$_{1}$ (2D) & 0.01, 0.34; 0.34 & 665; 19 & 2; 2 &10581 & 1449 & 1456 & Logistic & 0.1 & 5 & +1.9  \\ \\
      \multicolumn{2}{l}{\textbf{Material Segmentation}} & & & & & & & & &\\
      \textbf{- 1step MF} & Position$_{2}$, Lab-Color (5D) & 5.00, 0.05, 0.30  & 665 & 2 & 928 & 150 & 1798 & Weighted Logistic & 1e-04 & 24 & 2.6 \\
      \textbf{- 2step MF} & Position$_{2}$, Lab-Color (5D) & 5.00, 0.05, 0.30 & 665 & 2 & 928 & 150 & 1798 & Weighted Logistic & 1e-04 & 12 & +0.7 \\
      \textbf{- \textit{loose} 2step MF} & Position$_{2}$, Lab-Color (5D) & 5.00, 0.05, 0.30 & 665 & 2 & 928 & 150 & 1798 & Weighted Logistic & 1e-04 & 12 & +0.2\\
      \midrule
      \multicolumn{2}{c}{\textbf{Neural Network Applications}} & & & & & & & & &\\
      \textbf{Tiles: CNN-9$\times$9} & - & - & 81 & 4 & 10000 & 1000 & 1000 & Logistic & 0.01 & 100 & 500.0 \\
      \textbf{Tiles: CNN-13$\times$13} & - & - & 169 & 6 & 10000 & 1000 & 1000 & Logistic & 0.01 & 100 & 500.0 \\
      \textbf{Tiles: CNN-17$\times$17} & - & - & 289 & 8 & 10000 & 1000 & 1000 & Logistic & 0.01 & 100 & 500.0 \\
      \textbf{Tiles: CNN-21$\times$21} & - & - & 441 & 10 & 10000 & 1000 & 1000 & Logistic & 0.01 & 100 & 500.0 \\
      \textbf{Tiles: BNN} & Position$_{1}$, Color (5D) & 0.05, 0.04 & 63 & 1 & 10000 & 1000 & 1000 & Logistic & 0.01 & 100 & 30.0 \\
      \textbf{LeNet} & - & - & 25 & 2 & 5490 & 1098 & 1647 & Logistic & 0.1 & 100 & 182.2 \\
      \textbf{Crop-LeNet} & - & - & 25 & 2 & 5490 & 1098 & 1647 & Logistic & 0.1 & 100 & 182.2 \\
      \textbf{BNN-LeNet} & Position$_{2}$ (2D) & 20.00 & 7 & 1 & 5490 & 1098 & 1647 & Logistic & 0.1 & 100 & 182.2 \\
      \textbf{DeepCNet} & - & - & 9 & 1 & 5490 & 1098 & 1647 & Logistic & 0.1 & 100 & 182.2 \\
      \textbf{Crop-DeepCNet} & - & - & 9 & 1 & 5490 & 1098 & 1647 & Logistic & 0.1 & 100 & 182.2 \\
      \textbf{BNN-DeepCNet} & Position$_{2}$ (2D) & 40.00  & 7 & 1 & 5490 & 1098 & 1647 & Logistic & 0.1 & 100 & 182.2 \\
      \bottomrule
      \\
    \end{tabular}
    \mycaption{Experiment Protocols} {Experiment protocols for the different experiments presented in this work. \textbf{Feature Types}:
    Feature spaces used for the bilateral convolutions. Position$_1$ corresponds to un-normalized pixel positions whereas Position$_2$ corresponds
    to pixel positions normalized to $[0,1]$ with respect to the given image. \textbf{Feature Scales}: Cross-validated scales for the features used.
     \textbf{Filter Size}: Number of elements in the filter that is being learned. \textbf{Filter Nbr.}: Half-width of the filter. \textbf{Train},
     \textbf{Val.} and \textbf{Test} corresponds to the number of train, validation and test images used in the experiment. \textbf{Loss Type}: Type
     of loss used for back-propagation. ``MSE'' corresponds to Euclidean mean squared error loss and ``Logistic'' corresponds to multinomial logistic
     loss. ``Weighted Logistic'' is the class-weighted multinomial logistic loss. We weighted the loss with inverse class probability for material
     segmentation task due to the small availability of training data with class imbalance. \textbf{LR}: Fixed learning rate used in stochastic gradient
     descent. \textbf{Batch}: Number of images used in one parameter update step. \textbf{Epochs}: Number of training epochs. In all the experiments,
     we used fixed momentum of 0.9 and weight decay of 0.0005 for stochastic gradient descent. ```Color Upsampling'' experiments in this Table corresponds
     to those performed on Pascal VOC12 dataset images. For all experiments using Pascal VOC12 images, we use extended
     training segmentation dataset available from~\cite{hariharan2011moredata}, and used standard validation and test splits
     from the main dataset~\cite{voc2012segmentation}.}
  \label{tbl:parameters}
\end{table*}

\clearpage

\section{Parameters and Additional Results for Video Propagation Networks}

In this Section, we present experiment protocols and additional qualitative results for experiments
on video object segmentation, semantic video segmentation and video color
propagation. Table~\ref{tbl:parameters_supp} shows the feature scales and other parameters used in different experiments.
Figures~\ref{fig:video_seg_pos_supp} show some qualitative results on video object segmentation
with some failure cases in Fig.~\ref{fig:video_seg_neg_supp}.
Figure~\ref{fig:semantic_visuals_supp} shows some qualitative results on semantic video segmentation and
Fig.~\ref{fig:color_visuals_supp} shows results on video color propagation.

\newcolumntype{L}[1]{>{\raggedright\let\newline\\\arraybackslash\hspace{0pt}}b{#1}}
\newcolumntype{C}[1]{>{\centering\let\newline\\\arraybackslash\hspace{0pt}}b{#1}}
\newcolumntype{R}[1]{>{\raggedleft\let\newline\\\arraybackslash\hspace{0pt}}b{#1}}

\begin{table*}[h]
\tiny
  \centering
    \begin{tabular}{L{3.0cm} L{2.4cm} L{2.8cm} L{2.8cm} C{0.5cm} C{1.0cm} L{1.2cm}}
      \toprule
\textbf{Experiment} & \textbf{Feature Type} & \textbf{Feature Scale-1, $\Lambda_a$} & \textbf{Feature Scale-2, $\Lambda_b$} & \textbf{$\alpha$} & \textbf{Input Frames} & \textbf{Loss Type} \\
      \midrule
      \textbf{Video Object Segmentation} & ($x,y,Y,Cb,Cr,t$) & (0.02,0.02,0.07,0.4,0.4,0.01) & (0.03,0.03,0.09,0.5,0.5,0.2) & 0.5 & 9 & Logistic\\
      \midrule
      \textbf{Semantic Video Segmentation} & & & & & \\
      \textbf{with CNN1~\cite{yu2015multi}-NoFlow} & ($x,y,R,G,B,t$) & (0.08,0.08,0.2,0.2,0.2,0.04) & (0.11,0.11,0.2,0.2,0.2,0.04) & 0.5 & 3 & Logistic \\
      \textbf{with CNN1~\cite{yu2015multi}-Flow} & ($x+u_x,y+u_y,R,G,B,t$) & (0.11,0.11,0.14,0.14,0.14,0.03) & (0.08,0.08,0.12,0.12,0.12,0.01) & 0.65 & 3 & Logistic\\
      \textbf{with CNN2~\cite{richter2016playing}-Flow} & ($x+u_x,y+u_y,R,G,B,t$) & (0.08,0.08,0.2,0.2,0.2,0.04) & (0.09,0.09,0.25,0.25,0.25,0.03) & 0.5 & 4 & Logistic\\
      \midrule
      \textbf{Video Color Propagation} & ($x,y,I,t$)  & (0.04,0.04,0.2,0.04) & No second kernel & 1 & 4 & MSE\\
      \bottomrule
      \\
    \end{tabular}
    \mycaption{Experiment Protocols} {Experiment protocols for the different experiments presented in this work. \textbf{Feature Types}:
    Feature spaces used for the bilateral convolutions, with position ($x,y$) and color
    ($R,G,B$ or $Y,Cb,Cr$) features $\in [0,255]$. $u_x$, $u_y$ denotes optical flow with respect
    to the present frame and $I$ denotes grayscale intensity.
    \textbf{Feature Scales ($\Lambda_a, \Lambda_b$)}: Cross-validated scales for the features used.
    \textbf{$\alpha$}: Exponential time decay for the input frames.
    \textbf{Input Frames}: Number of input frames for VPN.
    \textbf{Loss Type}: Type
     of loss used for back-propagation. ``MSE'' corresponds to Euclidean mean squared error loss and ``Logistic'' corresponds to multinomial logistic loss.}
  \label{tbl:parameters_supp}
\end{table*}

% \begin{figure}[th!]
% \begin{center}
%   \centerline{\includegraphics[width=\textwidth]{figures/video_seg_visuals_supp_small.pdf}}
%     \mycaption{Video Object Segmentation}
%     {Shown are the different frames in example videos with the corresponding
%     ground truth (GT) masks, predictions from BVS~\cite{marki2016bilateral},
%     OFL~\cite{tsaivideo}, VPN (VPN-Stage2) and VPN-DLab (VPN-DeepLab) models.}
%     \label{fig:video_seg_small_supp}
% \end{center}
% \vspace{-1.0cm}
% \end{figure}

\begin{figure}[th!]
\begin{center}
  \centerline{\includegraphics[width=0.7\textwidth]{figures/video_seg_visuals_supp_positive.pdf}}
    \mycaption{Video Object Segmentation}
    {Shown are the different frames in example videos with the corresponding
    ground truth (GT) masks, predictions from BVS~\cite{marki2016bilateral},
    OFL~\cite{tsaivideo}, VPN (VPN-Stage2) and VPN-DLab (VPN-DeepLab) models.}
    \label{fig:video_seg_pos_supp}
\end{center}
\vspace{-1.0cm}
\end{figure}

\begin{figure}[th!]
\begin{center}
  \centerline{\includegraphics[width=0.7\textwidth]{figures/video_seg_visuals_supp_negative.pdf}}
    \mycaption{Failure Cases for Video Object Segmentation}
    {Shown are the different frames in example videos with the corresponding
    ground truth (GT) masks, predictions from BVS~\cite{marki2016bilateral},
    OFL~\cite{tsaivideo}, VPN (VPN-Stage2) and VPN-DLab (VPN-DeepLab) models.}
    \label{fig:video_seg_neg_supp}
\end{center}
\vspace{-1.0cm}
\end{figure}

\begin{figure}[th!]
\begin{center}
  \centerline{\includegraphics[width=0.9\textwidth]{figures/supp_semantic_visual.pdf}}
    \mycaption{Semantic Video Segmentation}
    {Input video frames and the corresponding ground truth (GT)
    segmentation together with the predictions of CNN~\cite{yu2015multi} and with
    VPN-Flow.}
    \label{fig:semantic_visuals_supp}
\end{center}
\vspace{-0.7cm}
\end{figure}

\begin{figure}[th!]
\begin{center}
  \centerline{\includegraphics[width=\textwidth]{figures/colorization_visuals_supp.pdf}}
  \mycaption{Video Color Propagation}
  {Input grayscale video frames and corresponding ground-truth (GT) color images
  together with color predictions of Levin et al.~\cite{levin2004colorization} and VPN-Stage1 models.}
  \label{fig:color_visuals_supp}
\end{center}
\vspace{-0.7cm}
\end{figure}

\clearpage

\section{Additional Material for Bilateral Inception Networks}
\label{sec:binception-app}

In this section of the Appendix, we first discuss the use of approximate bilateral
filtering in BI modules (Sec.~\ref{sec:lattice}).
Later, we present some qualitative results using different models for the approach presented in
Chapter~\ref{chap:binception} (Sec.~\ref{sec:qualitative-app}).

\subsection{Approximate Bilateral Filtering}
\label{sec:lattice}

The bilateral inception module presented in Chapter~\ref{chap:binception} computes a matrix-vector
product between a Gaussian filter $K$ and a vector of activations $\bz_c$.
Bilateral filtering is an important operation and many algorithmic techniques have been
proposed to speed-up this operation~\cite{paris2006fast,adams2010fast,gastal2011domain}.
In the main paper we opted to implement what can be considered the
brute-force variant of explicitly constructing $K$ and then using BLAS to compute the
matrix-vector product. This resulted in a few millisecond operation.
The explicit way to compute is possible due to the
reduction to super-pixels, e.g., it would not work for DenseCRF variants
that operate on the full image resolution.

Here, we present experiments where we use the fast approximate bilateral filtering
algorithm of~\cite{adams2010fast}, which is also used in Chapter~\ref{chap:bnn}
for learning sparse high dimensional filters. This
choice allows for larger dimensions of matrix-vector multiplication. The reason for choosing
the explicit multiplication in Chapter~\ref{chap:binception} was that it was computationally faster.
For the small sizes of the involved matrices and vectors, the explicit computation is sufficient and we had no
GPU implementation of an approximate technique that matched this runtime. Also it
is conceptually easier and the gradient to the feature transformations ($\Lambda \mathbf{f}$) is
obtained using standard matrix calculus.

\subsubsection{Experiments}

We modified the existing segmentation architectures analogous to those in Chapter~\ref{chap:binception}.
The main difference is that, here, the inception modules use the lattice
approximation~\cite{adams2010fast} to compute the bilateral filtering.
Using the lattice approximation did not allow us to back-propagate through feature transformations ($\Lambda$)
and thus we used hand-specified feature scales as will be explained later.
Specifically, we take CNN architectures from the works
of~\cite{chen2014semantic,zheng2015conditional,bell2015minc} and insert the BI modules between
the spatial FC layers.
We use superpixels from~\cite{DollarICCV13edges}
for all the experiments with the lattice approximation. Experiments are
performed using Caffe neural network framework~\cite{jia2014caffe}.

\begin{table}
  \small
  \centering
  \begin{tabular}{p{5.5cm}>{\raggedright\arraybackslash}p{1.4cm}>{\centering\arraybackslash}p{2.2cm}}
    \toprule
		\textbf{Model} & \emph{IoU} & \emph{Runtime}(ms) \\
    \midrule

    %%%%%%%%%%%% Scores computed by us)%%%%%%%%%%%%
		\deeplablargefov & 68.9 & 145ms\\
    \midrule
    \bi{7}{2}-\bi{8}{10}& \textbf{73.8} & +600 \\
    \midrule
    \deeplablargefovcrf~\cite{chen2014semantic} & 72.7 & +830\\
    \deeplabmsclargefovcrf~\cite{chen2014semantic} & \textbf{73.6} & +880\\
    DeepLab-EdgeNet~\cite{chen2015semantic} & 71.7 & +30\\
    DeepLab-EdgeNet-CRF~\cite{chen2015semantic} & \textbf{73.6} & +860\\
  \bottomrule \\
  \end{tabular}
  \mycaption{Semantic Segmentation using the DeepLab model}
  {IoU scores on the Pascal VOC12 segmentation test dataset
  with different models and our modified inception model.
  Also shown are the corresponding runtimes in milliseconds. Runtimes
  also include superpixel computations (300 ms with Dollar superpixels~\cite{DollarICCV13edges})}
  \label{tab:largefovresults}
\end{table}

\paragraph{Semantic Segmentation}
The experiments in this section use the Pascal VOC12 segmentation dataset~\cite{voc2012segmentation} with 21 object classes and the images have a maximum resolution of 0.25 megapixels.
For all experiments on VOC12, we train using the extended training set of
10581 images collected by~\cite{hariharan2011moredata}.
We modified the \deeplab~network architecture of~\cite{chen2014semantic} and
the CRFasRNN architecture from~\cite{zheng2015conditional} which uses a CNN with
deconvolution layers followed by DenseCRF trained end-to-end.

\paragraph{DeepLab Model}\label{sec:deeplabmodel}
We experimented with the \bi{7}{2}-\bi{8}{10} inception model.
Results using the~\deeplab~model are summarized in Tab.~\ref{tab:largefovresults}.
Although we get similar improvements with inception modules as with the
explicit kernel computation, using lattice approximation is slower.

\begin{table}
  \small
  \centering
  \begin{tabular}{p{6.4cm}>{\raggedright\arraybackslash}p{1.8cm}>{\raggedright\arraybackslash}p{1.8cm}}
    \toprule
    \textbf{Model} & \emph{IoU (Val)} & \emph{IoU (Test)}\\
    \midrule
    %%%%%%%%%%%% Scores computed by us)%%%%%%%%%%%%
    CNN &  67.5 & - \\
    \deconv (CNN+Deconvolutions) & 69.8 & 72.0 \\
    \midrule
    \bi{3}{6}-\bi{4}{6}-\bi{7}{2}-\bi{8}{6}& 71.9 & - \\
    \bi{3}{6}-\bi{4}{6}-\bi{7}{2}-\bi{8}{6}-\gi{6}& 73.6 &  \href{http://host.robots.ox.ac.uk:8080/anonymous/VOTV5E.html}{\textbf{75.2}}\\
    \midrule
    \deconvcrf (CRF-RNN)~\cite{zheng2015conditional} & 73.0 & 74.7\\
    Context-CRF-RNN~\cite{yu2015multi} & ~~ - ~ & \textbf{75.3} \\
    \bottomrule \\
  \end{tabular}
  \mycaption{Semantic Segmentation using the CRFasRNN model}{IoU score corresponding to different models
  on Pascal VOC12 reduced validation / test segmentation dataset. The reduced validation set consists of 346 images
  as used in~\cite{zheng2015conditional} where we adapted the model from.}
  \label{tab:deconvresults-app}
\end{table}

\paragraph{CRFasRNN Model}\label{sec:deepinception}
We add BI modules after score-pool3, score-pool4, \fc{7} and \fc{8} $1\times1$ convolution layers
resulting in the \bi{3}{6}-\bi{4}{6}-\bi{7}{2}-\bi{8}{6}
model and also experimented with another variant where $BI_8$ is followed by another inception
module, G$(6)$, with 6 Gaussian kernels.
Note that here also we discarded both deconvolution and DenseCRF parts of the original model~\cite{zheng2015conditional}
and inserted the BI modules in the base CNN and found similar improvements compared to the inception modules with explicit
kernel computaion. See Tab.~\ref{tab:deconvresults-app} for results on the CRFasRNN model.

\paragraph{Material Segmentation}
Table~\ref{tab:mincresults-app} shows the results on the MINC dataset~\cite{bell2015minc}
obtained by modifying the AlexNet architecture with our inception modules. We observe
similar improvements as with explicit kernel construction.
For this model, we do not provide any learned setup due to very limited segment training
data. The weights to combine outputs in the bilateral inception layer are
found by validation on the validation set.

\begin{table}[t]
  \small
  \centering
  \begin{tabular}{p{3.5cm}>{\centering\arraybackslash}p{4.0cm}}
    \toprule
    \textbf{Model} & Class / Total accuracy\\
    \midrule

    %%%%%%%%%%%% Scores computed by us)%%%%%%%%%%%%
    AlexNet CNN & 55.3 / 58.9 \\
    \midrule
    \bi{7}{2}-\bi{8}{6}& 68.5 / 71.8 \\
    \bi{7}{2}-\bi{8}{6}-G$(6)$& 67.6 / 73.1 \\
    \midrule
    AlexNet-CRF & 65.5 / 71.0 \\
    \bottomrule \\
  \end{tabular}
  \mycaption{Material Segmentation using AlexNet}{Pixel accuracy of different models on
  the MINC material segmentation test dataset~\cite{bell2015minc}.}
  \label{tab:mincresults-app}
\end{table}

\paragraph{Scales of Bilateral Inception Modules}
\label{sec:scales}

Unlike the explicit kernel technique presented in the main text (Chapter~\ref{chap:binception}),
we didn't back-propagate through feature transformation ($\Lambda$)
using the approximate bilateral filter technique.
So, the feature scales are hand-specified and validated, which are as follows.
The optimal scale values for the \bi{7}{2}-\bi{8}{2} model are found by validation for the best performance which are
$\sigma_{xy}$ = (0.1, 0.1) for the spatial (XY) kernel and $\sigma_{rgbxy}$ = (0.1, 0.1, 0.1, 0.01, 0.01) for color and position (RGBXY)  kernel.
Next, as more kernels are added to \bi{8}{2}, we set scales to be $\alpha$*($\sigma_{xy}$, $\sigma_{rgbxy}$).
The value of $\alpha$ is chosen as  1, 0.5, 0.1, 0.05, 0.1, at uniform interval, for the \bi{8}{10} bilateral inception module.


\subsection{Qualitative Results}
\label{sec:qualitative-app}

In this section, we present more qualitative results obtained using the BI module with explicit
kernel computation technique presented in Chapter~\ref{chap:binception}. Results on the Pascal VOC12
dataset~\cite{voc2012segmentation} using the DeepLab-LargeFOV model are shown in Fig.~\ref{fig:semantic_visuals-app},
followed by the results on MINC dataset~\cite{bell2015minc}
in Fig.~\ref{fig:material_visuals-app} and on
Cityscapes dataset~\cite{Cordts2015Cvprw} in Fig.~\ref{fig:street_visuals-app}.


\definecolor{voc_1}{RGB}{0, 0, 0}
\definecolor{voc_2}{RGB}{128, 0, 0}
\definecolor{voc_3}{RGB}{0, 128, 0}
\definecolor{voc_4}{RGB}{128, 128, 0}
\definecolor{voc_5}{RGB}{0, 0, 128}
\definecolor{voc_6}{RGB}{128, 0, 128}
\definecolor{voc_7}{RGB}{0, 128, 128}
\definecolor{voc_8}{RGB}{128, 128, 128}
\definecolor{voc_9}{RGB}{64, 0, 0}
\definecolor{voc_10}{RGB}{192, 0, 0}
\definecolor{voc_11}{RGB}{64, 128, 0}
\definecolor{voc_12}{RGB}{192, 128, 0}
\definecolor{voc_13}{RGB}{64, 0, 128}
\definecolor{voc_14}{RGB}{192, 0, 128}
\definecolor{voc_15}{RGB}{64, 128, 128}
\definecolor{voc_16}{RGB}{192, 128, 128}
\definecolor{voc_17}{RGB}{0, 64, 0}
\definecolor{voc_18}{RGB}{128, 64, 0}
\definecolor{voc_19}{RGB}{0, 192, 0}
\definecolor{voc_20}{RGB}{128, 192, 0}
\definecolor{voc_21}{RGB}{0, 64, 128}
\definecolor{voc_22}{RGB}{128, 64, 128}

\begin{figure*}[!ht]
  \small
  \centering
  \fcolorbox{white}{voc_1}{\rule{0pt}{4pt}\rule{4pt}{0pt}} Background~~
  \fcolorbox{white}{voc_2}{\rule{0pt}{4pt}\rule{4pt}{0pt}} Aeroplane~~
  \fcolorbox{white}{voc_3}{\rule{0pt}{4pt}\rule{4pt}{0pt}} Bicycle~~
  \fcolorbox{white}{voc_4}{\rule{0pt}{4pt}\rule{4pt}{0pt}} Bird~~
  \fcolorbox{white}{voc_5}{\rule{0pt}{4pt}\rule{4pt}{0pt}} Boat~~
  \fcolorbox{white}{voc_6}{\rule{0pt}{4pt}\rule{4pt}{0pt}} Bottle~~
  \fcolorbox{white}{voc_7}{\rule{0pt}{4pt}\rule{4pt}{0pt}} Bus~~
  \fcolorbox{white}{voc_8}{\rule{0pt}{4pt}\rule{4pt}{0pt}} Car~~\\
  \fcolorbox{white}{voc_9}{\rule{0pt}{4pt}\rule{4pt}{0pt}} Cat~~
  \fcolorbox{white}{voc_10}{\rule{0pt}{4pt}\rule{4pt}{0pt}} Chair~~
  \fcolorbox{white}{voc_11}{\rule{0pt}{4pt}\rule{4pt}{0pt}} Cow~~
  \fcolorbox{white}{voc_12}{\rule{0pt}{4pt}\rule{4pt}{0pt}} Dining Table~~
  \fcolorbox{white}{voc_13}{\rule{0pt}{4pt}\rule{4pt}{0pt}} Dog~~
  \fcolorbox{white}{voc_14}{\rule{0pt}{4pt}\rule{4pt}{0pt}} Horse~~
  \fcolorbox{white}{voc_15}{\rule{0pt}{4pt}\rule{4pt}{0pt}} Motorbike~~
  \fcolorbox{white}{voc_16}{\rule{0pt}{4pt}\rule{4pt}{0pt}} Person~~\\
  \fcolorbox{white}{voc_17}{\rule{0pt}{4pt}\rule{4pt}{0pt}} Potted Plant~~
  \fcolorbox{white}{voc_18}{\rule{0pt}{4pt}\rule{4pt}{0pt}} Sheep~~
  \fcolorbox{white}{voc_19}{\rule{0pt}{4pt}\rule{4pt}{0pt}} Sofa~~
  \fcolorbox{white}{voc_20}{\rule{0pt}{4pt}\rule{4pt}{0pt}} Train~~
  \fcolorbox{white}{voc_21}{\rule{0pt}{4pt}\rule{4pt}{0pt}} TV monitor~~\\


  \subfigure{%
    \includegraphics[width=.15\columnwidth]{figures/supplementary/2008_001308_given.png}
  }
  \subfigure{%
    \includegraphics[width=.15\columnwidth]{figures/supplementary/2008_001308_sp.png}
  }
  \subfigure{%
    \includegraphics[width=.15\columnwidth]{figures/supplementary/2008_001308_gt.png}
  }
  \subfigure{%
    \includegraphics[width=.15\columnwidth]{figures/supplementary/2008_001308_cnn.png}
  }
  \subfigure{%
    \includegraphics[width=.15\columnwidth]{figures/supplementary/2008_001308_crf.png}
  }
  \subfigure{%
    \includegraphics[width=.15\columnwidth]{figures/supplementary/2008_001308_ours.png}
  }\\[-2ex]


  \subfigure{%
    \includegraphics[width=.15\columnwidth]{figures/supplementary/2008_001821_given.png}
  }
  \subfigure{%
    \includegraphics[width=.15\columnwidth]{figures/supplementary/2008_001821_sp.png}
  }
  \subfigure{%
    \includegraphics[width=.15\columnwidth]{figures/supplementary/2008_001821_gt.png}
  }
  \subfigure{%
    \includegraphics[width=.15\columnwidth]{figures/supplementary/2008_001821_cnn.png}
  }
  \subfigure{%
    \includegraphics[width=.15\columnwidth]{figures/supplementary/2008_001821_crf.png}
  }
  \subfigure{%
    \includegraphics[width=.15\columnwidth]{figures/supplementary/2008_001821_ours.png}
  }\\[-2ex]



  \subfigure{%
    \includegraphics[width=.15\columnwidth]{figures/supplementary/2008_004612_given.png}
  }
  \subfigure{%
    \includegraphics[width=.15\columnwidth]{figures/supplementary/2008_004612_sp.png}
  }
  \subfigure{%
    \includegraphics[width=.15\columnwidth]{figures/supplementary/2008_004612_gt.png}
  }
  \subfigure{%
    \includegraphics[width=.15\columnwidth]{figures/supplementary/2008_004612_cnn.png}
  }
  \subfigure{%
    \includegraphics[width=.15\columnwidth]{figures/supplementary/2008_004612_crf.png}
  }
  \subfigure{%
    \includegraphics[width=.15\columnwidth]{figures/supplementary/2008_004612_ours.png}
  }\\[-2ex]


  \subfigure{%
    \includegraphics[width=.15\columnwidth]{figures/supplementary/2009_001008_given.png}
  }
  \subfigure{%
    \includegraphics[width=.15\columnwidth]{figures/supplementary/2009_001008_sp.png}
  }
  \subfigure{%
    \includegraphics[width=.15\columnwidth]{figures/supplementary/2009_001008_gt.png}
  }
  \subfigure{%
    \includegraphics[width=.15\columnwidth]{figures/supplementary/2009_001008_cnn.png}
  }
  \subfigure{%
    \includegraphics[width=.15\columnwidth]{figures/supplementary/2009_001008_crf.png}
  }
  \subfigure{%
    \includegraphics[width=.15\columnwidth]{figures/supplementary/2009_001008_ours.png}
  }\\[-2ex]




  \subfigure{%
    \includegraphics[width=.15\columnwidth]{figures/supplementary/2009_004497_given.png}
  }
  \subfigure{%
    \includegraphics[width=.15\columnwidth]{figures/supplementary/2009_004497_sp.png}
  }
  \subfigure{%
    \includegraphics[width=.15\columnwidth]{figures/supplementary/2009_004497_gt.png}
  }
  \subfigure{%
    \includegraphics[width=.15\columnwidth]{figures/supplementary/2009_004497_cnn.png}
  }
  \subfigure{%
    \includegraphics[width=.15\columnwidth]{figures/supplementary/2009_004497_crf.png}
  }
  \subfigure{%
    \includegraphics[width=.15\columnwidth]{figures/supplementary/2009_004497_ours.png}
  }\\[-2ex]



  \setcounter{subfigure}{0}
  \subfigure[\scriptsize Input]{%
    \includegraphics[width=.15\columnwidth]{figures/supplementary/2010_001327_given.png}
  }
  \subfigure[\scriptsize Superpixels]{%
    \includegraphics[width=.15\columnwidth]{figures/supplementary/2010_001327_sp.png}
  }
  \subfigure[\scriptsize GT]{%
    \includegraphics[width=.15\columnwidth]{figures/supplementary/2010_001327_gt.png}
  }
  \subfigure[\scriptsize Deeplab]{%
    \includegraphics[width=.15\columnwidth]{figures/supplementary/2010_001327_cnn.png}
  }
  \subfigure[\scriptsize +DenseCRF]{%
    \includegraphics[width=.15\columnwidth]{figures/supplementary/2010_001327_crf.png}
  }
  \subfigure[\scriptsize Using BI]{%
    \includegraphics[width=.15\columnwidth]{figures/supplementary/2010_001327_ours.png}
  }
  \mycaption{Semantic Segmentation}{Example results of semantic segmentation
  on the Pascal VOC12 dataset.
  (d)~depicts the DeepLab CNN result, (e)~CNN + 10 steps of mean-field inference,
  (f~result obtained with bilateral inception (BI) modules (\bi{6}{2}+\bi{7}{6}) between \fc~layers.}
  \label{fig:semantic_visuals-app}
\end{figure*}


\definecolor{minc_1}{HTML}{771111}
\definecolor{minc_2}{HTML}{CAC690}
\definecolor{minc_3}{HTML}{EEEEEE}
\definecolor{minc_4}{HTML}{7C8FA6}
\definecolor{minc_5}{HTML}{597D31}
\definecolor{minc_6}{HTML}{104410}
\definecolor{minc_7}{HTML}{BB819C}
\definecolor{minc_8}{HTML}{D0CE48}
\definecolor{minc_9}{HTML}{622745}
\definecolor{minc_10}{HTML}{666666}
\definecolor{minc_11}{HTML}{D54A31}
\definecolor{minc_12}{HTML}{101044}
\definecolor{minc_13}{HTML}{444126}
\definecolor{minc_14}{HTML}{75D646}
\definecolor{minc_15}{HTML}{DD4348}
\definecolor{minc_16}{HTML}{5C8577}
\definecolor{minc_17}{HTML}{C78472}
\definecolor{minc_18}{HTML}{75D6D0}
\definecolor{minc_19}{HTML}{5B4586}
\definecolor{minc_20}{HTML}{C04393}
\definecolor{minc_21}{HTML}{D69948}
\definecolor{minc_22}{HTML}{7370D8}
\definecolor{minc_23}{HTML}{7A3622}
\definecolor{minc_24}{HTML}{000000}

\begin{figure*}[!ht]
  \small % scriptsize
  \centering
  \fcolorbox{white}{minc_1}{\rule{0pt}{4pt}\rule{4pt}{0pt}} Brick~~
  \fcolorbox{white}{minc_2}{\rule{0pt}{4pt}\rule{4pt}{0pt}} Carpet~~
  \fcolorbox{white}{minc_3}{\rule{0pt}{4pt}\rule{4pt}{0pt}} Ceramic~~
  \fcolorbox{white}{minc_4}{\rule{0pt}{4pt}\rule{4pt}{0pt}} Fabric~~
  \fcolorbox{white}{minc_5}{\rule{0pt}{4pt}\rule{4pt}{0pt}} Foliage~~
  \fcolorbox{white}{minc_6}{\rule{0pt}{4pt}\rule{4pt}{0pt}} Food~~
  \fcolorbox{white}{minc_7}{\rule{0pt}{4pt}\rule{4pt}{0pt}} Glass~~
  \fcolorbox{white}{minc_8}{\rule{0pt}{4pt}\rule{4pt}{0pt}} Hair~~\\
  \fcolorbox{white}{minc_9}{\rule{0pt}{4pt}\rule{4pt}{0pt}} Leather~~
  \fcolorbox{white}{minc_10}{\rule{0pt}{4pt}\rule{4pt}{0pt}} Metal~~
  \fcolorbox{white}{minc_11}{\rule{0pt}{4pt}\rule{4pt}{0pt}} Mirror~~
  \fcolorbox{white}{minc_12}{\rule{0pt}{4pt}\rule{4pt}{0pt}} Other~~
  \fcolorbox{white}{minc_13}{\rule{0pt}{4pt}\rule{4pt}{0pt}} Painted~~
  \fcolorbox{white}{minc_14}{\rule{0pt}{4pt}\rule{4pt}{0pt}} Paper~~
  \fcolorbox{white}{minc_15}{\rule{0pt}{4pt}\rule{4pt}{0pt}} Plastic~~\\
  \fcolorbox{white}{minc_16}{\rule{0pt}{4pt}\rule{4pt}{0pt}} Polished Stone~~
  \fcolorbox{white}{minc_17}{\rule{0pt}{4pt}\rule{4pt}{0pt}} Skin~~
  \fcolorbox{white}{minc_18}{\rule{0pt}{4pt}\rule{4pt}{0pt}} Sky~~
  \fcolorbox{white}{minc_19}{\rule{0pt}{4pt}\rule{4pt}{0pt}} Stone~~
  \fcolorbox{white}{minc_20}{\rule{0pt}{4pt}\rule{4pt}{0pt}} Tile~~
  \fcolorbox{white}{minc_21}{\rule{0pt}{4pt}\rule{4pt}{0pt}} Wallpaper~~
  \fcolorbox{white}{minc_22}{\rule{0pt}{4pt}\rule{4pt}{0pt}} Water~~
  \fcolorbox{white}{minc_23}{\rule{0pt}{4pt}\rule{4pt}{0pt}} Wood~~\\
  \subfigure{%
    \includegraphics[width=.15\columnwidth]{figures/supplementary/000008468_given.png}
  }
  \subfigure{%
    \includegraphics[width=.15\columnwidth]{figures/supplementary/000008468_sp.png}
  }
  \subfigure{%
    \includegraphics[width=.15\columnwidth]{figures/supplementary/000008468_gt.png}
  }
  \subfigure{%
    \includegraphics[width=.15\columnwidth]{figures/supplementary/000008468_cnn.png}
  }
  \subfigure{%
    \includegraphics[width=.15\columnwidth]{figures/supplementary/000008468_crf.png}
  }
  \subfigure{%
    \includegraphics[width=.15\columnwidth]{figures/supplementary/000008468_ours.png}
  }\\[-2ex]

  \subfigure{%
    \includegraphics[width=.15\columnwidth]{figures/supplementary/000009053_given.png}
  }
  \subfigure{%
    \includegraphics[width=.15\columnwidth]{figures/supplementary/000009053_sp.png}
  }
  \subfigure{%
    \includegraphics[width=.15\columnwidth]{figures/supplementary/000009053_gt.png}
  }
  \subfigure{%
    \includegraphics[width=.15\columnwidth]{figures/supplementary/000009053_cnn.png}
  }
  \subfigure{%
    \includegraphics[width=.15\columnwidth]{figures/supplementary/000009053_crf.png}
  }
  \subfigure{%
    \includegraphics[width=.15\columnwidth]{figures/supplementary/000009053_ours.png}
  }\\[-2ex]




  \subfigure{%
    \includegraphics[width=.15\columnwidth]{figures/supplementary/000014977_given.png}
  }
  \subfigure{%
    \includegraphics[width=.15\columnwidth]{figures/supplementary/000014977_sp.png}
  }
  \subfigure{%
    \includegraphics[width=.15\columnwidth]{figures/supplementary/000014977_gt.png}
  }
  \subfigure{%
    \includegraphics[width=.15\columnwidth]{figures/supplementary/000014977_cnn.png}
  }
  \subfigure{%
    \includegraphics[width=.15\columnwidth]{figures/supplementary/000014977_crf.png}
  }
  \subfigure{%
    \includegraphics[width=.15\columnwidth]{figures/supplementary/000014977_ours.png}
  }\\[-2ex]


  \subfigure{%
    \includegraphics[width=.15\columnwidth]{figures/supplementary/000022922_given.png}
  }
  \subfigure{%
    \includegraphics[width=.15\columnwidth]{figures/supplementary/000022922_sp.png}
  }
  \subfigure{%
    \includegraphics[width=.15\columnwidth]{figures/supplementary/000022922_gt.png}
  }
  \subfigure{%
    \includegraphics[width=.15\columnwidth]{figures/supplementary/000022922_cnn.png}
  }
  \subfigure{%
    \includegraphics[width=.15\columnwidth]{figures/supplementary/000022922_crf.png}
  }
  \subfigure{%
    \includegraphics[width=.15\columnwidth]{figures/supplementary/000022922_ours.png}
  }\\[-2ex]


  \subfigure{%
    \includegraphics[width=.15\columnwidth]{figures/supplementary/000025711_given.png}
  }
  \subfigure{%
    \includegraphics[width=.15\columnwidth]{figures/supplementary/000025711_sp.png}
  }
  \subfigure{%
    \includegraphics[width=.15\columnwidth]{figures/supplementary/000025711_gt.png}
  }
  \subfigure{%
    \includegraphics[width=.15\columnwidth]{figures/supplementary/000025711_cnn.png}
  }
  \subfigure{%
    \includegraphics[width=.15\columnwidth]{figures/supplementary/000025711_crf.png}
  }
  \subfigure{%
    \includegraphics[width=.15\columnwidth]{figures/supplementary/000025711_ours.png}
  }\\[-2ex]


  \subfigure{%
    \includegraphics[width=.15\columnwidth]{figures/supplementary/000034473_given.png}
  }
  \subfigure{%
    \includegraphics[width=.15\columnwidth]{figures/supplementary/000034473_sp.png}
  }
  \subfigure{%
    \includegraphics[width=.15\columnwidth]{figures/supplementary/000034473_gt.png}
  }
  \subfigure{%
    \includegraphics[width=.15\columnwidth]{figures/supplementary/000034473_cnn.png}
  }
  \subfigure{%
    \includegraphics[width=.15\columnwidth]{figures/supplementary/000034473_crf.png}
  }
  \subfigure{%
    \includegraphics[width=.15\columnwidth]{figures/supplementary/000034473_ours.png}
  }\\[-2ex]


  \subfigure{%
    \includegraphics[width=.15\columnwidth]{figures/supplementary/000035463_given.png}
  }
  \subfigure{%
    \includegraphics[width=.15\columnwidth]{figures/supplementary/000035463_sp.png}
  }
  \subfigure{%
    \includegraphics[width=.15\columnwidth]{figures/supplementary/000035463_gt.png}
  }
  \subfigure{%
    \includegraphics[width=.15\columnwidth]{figures/supplementary/000035463_cnn.png}
  }
  \subfigure{%
    \includegraphics[width=.15\columnwidth]{figures/supplementary/000035463_crf.png}
  }
  \subfigure{%
    \includegraphics[width=.15\columnwidth]{figures/supplementary/000035463_ours.png}
  }\\[-2ex]


  \setcounter{subfigure}{0}
  \subfigure[\scriptsize Input]{%
    \includegraphics[width=.15\columnwidth]{figures/supplementary/000035993_given.png}
  }
  \subfigure[\scriptsize Superpixels]{%
    \includegraphics[width=.15\columnwidth]{figures/supplementary/000035993_sp.png}
  }
  \subfigure[\scriptsize GT]{%
    \includegraphics[width=.15\columnwidth]{figures/supplementary/000035993_gt.png}
  }
  \subfigure[\scriptsize AlexNet]{%
    \includegraphics[width=.15\columnwidth]{figures/supplementary/000035993_cnn.png}
  }
  \subfigure[\scriptsize +DenseCRF]{%
    \includegraphics[width=.15\columnwidth]{figures/supplementary/000035993_crf.png}
  }
  \subfigure[\scriptsize Using BI]{%
    \includegraphics[width=.15\columnwidth]{figures/supplementary/000035993_ours.png}
  }
  \mycaption{Material Segmentation}{Example results of material segmentation.
  (d)~depicts the AlexNet CNN result, (e)~CNN + 10 steps of mean-field inference,
  (f)~result obtained with bilateral inception (BI) modules (\bi{7}{2}+\bi{8}{6}) between
  \fc~layers.}
\label{fig:material_visuals-app}
\end{figure*}


\definecolor{city_1}{RGB}{128, 64, 128}
\definecolor{city_2}{RGB}{244, 35, 232}
\definecolor{city_3}{RGB}{70, 70, 70}
\definecolor{city_4}{RGB}{102, 102, 156}
\definecolor{city_5}{RGB}{190, 153, 153}
\definecolor{city_6}{RGB}{153, 153, 153}
\definecolor{city_7}{RGB}{250, 170, 30}
\definecolor{city_8}{RGB}{220, 220, 0}
\definecolor{city_9}{RGB}{107, 142, 35}
\definecolor{city_10}{RGB}{152, 251, 152}
\definecolor{city_11}{RGB}{70, 130, 180}
\definecolor{city_12}{RGB}{220, 20, 60}
\definecolor{city_13}{RGB}{255, 0, 0}
\definecolor{city_14}{RGB}{0, 0, 142}
\definecolor{city_15}{RGB}{0, 0, 70}
\definecolor{city_16}{RGB}{0, 60, 100}
\definecolor{city_17}{RGB}{0, 80, 100}
\definecolor{city_18}{RGB}{0, 0, 230}
\definecolor{city_19}{RGB}{119, 11, 32}
\begin{figure*}[!ht]
  \small % scriptsize
  \centering


  \subfigure{%
    \includegraphics[width=.18\columnwidth]{figures/supplementary/frankfurt00000_016005_given.png}
  }
  \subfigure{%
    \includegraphics[width=.18\columnwidth]{figures/supplementary/frankfurt00000_016005_sp.png}
  }
  \subfigure{%
    \includegraphics[width=.18\columnwidth]{figures/supplementary/frankfurt00000_016005_gt.png}
  }
  \subfigure{%
    \includegraphics[width=.18\columnwidth]{figures/supplementary/frankfurt00000_016005_cnn.png}
  }
  \subfigure{%
    \includegraphics[width=.18\columnwidth]{figures/supplementary/frankfurt00000_016005_ours.png}
  }\\[-2ex]

  \subfigure{%
    \includegraphics[width=.18\columnwidth]{figures/supplementary/frankfurt00000_004617_given.png}
  }
  \subfigure{%
    \includegraphics[width=.18\columnwidth]{figures/supplementary/frankfurt00000_004617_sp.png}
  }
  \subfigure{%
    \includegraphics[width=.18\columnwidth]{figures/supplementary/frankfurt00000_004617_gt.png}
  }
  \subfigure{%
    \includegraphics[width=.18\columnwidth]{figures/supplementary/frankfurt00000_004617_cnn.png}
  }
  \subfigure{%
    \includegraphics[width=.18\columnwidth]{figures/supplementary/frankfurt00000_004617_ours.png}
  }\\[-2ex]

  \subfigure{%
    \includegraphics[width=.18\columnwidth]{figures/supplementary/frankfurt00000_020880_given.png}
  }
  \subfigure{%
    \includegraphics[width=.18\columnwidth]{figures/supplementary/frankfurt00000_020880_sp.png}
  }
  \subfigure{%
    \includegraphics[width=.18\columnwidth]{figures/supplementary/frankfurt00000_020880_gt.png}
  }
  \subfigure{%
    \includegraphics[width=.18\columnwidth]{figures/supplementary/frankfurt00000_020880_cnn.png}
  }
  \subfigure{%
    \includegraphics[width=.18\columnwidth]{figures/supplementary/frankfurt00000_020880_ours.png}
  }\\[-2ex]



  \subfigure{%
    \includegraphics[width=.18\columnwidth]{figures/supplementary/frankfurt00001_007285_given.png}
  }
  \subfigure{%
    \includegraphics[width=.18\columnwidth]{figures/supplementary/frankfurt00001_007285_sp.png}
  }
  \subfigure{%
    \includegraphics[width=.18\columnwidth]{figures/supplementary/frankfurt00001_007285_gt.png}
  }
  \subfigure{%
    \includegraphics[width=.18\columnwidth]{figures/supplementary/frankfurt00001_007285_cnn.png}
  }
  \subfigure{%
    \includegraphics[width=.18\columnwidth]{figures/supplementary/frankfurt00001_007285_ours.png}
  }\\[-2ex]


  \subfigure{%
    \includegraphics[width=.18\columnwidth]{figures/supplementary/frankfurt00001_059789_given.png}
  }
  \subfigure{%
    \includegraphics[width=.18\columnwidth]{figures/supplementary/frankfurt00001_059789_sp.png}
  }
  \subfigure{%
    \includegraphics[width=.18\columnwidth]{figures/supplementary/frankfurt00001_059789_gt.png}
  }
  \subfigure{%
    \includegraphics[width=.18\columnwidth]{figures/supplementary/frankfurt00001_059789_cnn.png}
  }
  \subfigure{%
    \includegraphics[width=.18\columnwidth]{figures/supplementary/frankfurt00001_059789_ours.png}
  }\\[-2ex]


  \subfigure{%
    \includegraphics[width=.18\columnwidth]{figures/supplementary/frankfurt00001_068208_given.png}
  }
  \subfigure{%
    \includegraphics[width=.18\columnwidth]{figures/supplementary/frankfurt00001_068208_sp.png}
  }
  \subfigure{%
    \includegraphics[width=.18\columnwidth]{figures/supplementary/frankfurt00001_068208_gt.png}
  }
  \subfigure{%
    \includegraphics[width=.18\columnwidth]{figures/supplementary/frankfurt00001_068208_cnn.png}
  }
  \subfigure{%
    \includegraphics[width=.18\columnwidth]{figures/supplementary/frankfurt00001_068208_ours.png}
  }\\[-2ex]

  \subfigure{%
    \includegraphics[width=.18\columnwidth]{figures/supplementary/frankfurt00001_082466_given.png}
  }
  \subfigure{%
    \includegraphics[width=.18\columnwidth]{figures/supplementary/frankfurt00001_082466_sp.png}
  }
  \subfigure{%
    \includegraphics[width=.18\columnwidth]{figures/supplementary/frankfurt00001_082466_gt.png}
  }
  \subfigure{%
    \includegraphics[width=.18\columnwidth]{figures/supplementary/frankfurt00001_082466_cnn.png}
  }
  \subfigure{%
    \includegraphics[width=.18\columnwidth]{figures/supplementary/frankfurt00001_082466_ours.png}
  }\\[-2ex]

  \subfigure{%
    \includegraphics[width=.18\columnwidth]{figures/supplementary/lindau00033_000019_given.png}
  }
  \subfigure{%
    \includegraphics[width=.18\columnwidth]{figures/supplementary/lindau00033_000019_sp.png}
  }
  \subfigure{%
    \includegraphics[width=.18\columnwidth]{figures/supplementary/lindau00033_000019_gt.png}
  }
  \subfigure{%
    \includegraphics[width=.18\columnwidth]{figures/supplementary/lindau00033_000019_cnn.png}
  }
  \subfigure{%
    \includegraphics[width=.18\columnwidth]{figures/supplementary/lindau00033_000019_ours.png}
  }\\[-2ex]

  \subfigure{%
    \includegraphics[width=.18\columnwidth]{figures/supplementary/lindau00052_000019_given.png}
  }
  \subfigure{%
    \includegraphics[width=.18\columnwidth]{figures/supplementary/lindau00052_000019_sp.png}
  }
  \subfigure{%
    \includegraphics[width=.18\columnwidth]{figures/supplementary/lindau00052_000019_gt.png}
  }
  \subfigure{%
    \includegraphics[width=.18\columnwidth]{figures/supplementary/lindau00052_000019_cnn.png}
  }
  \subfigure{%
    \includegraphics[width=.18\columnwidth]{figures/supplementary/lindau00052_000019_ours.png}
  }\\[-2ex]




  \subfigure{%
    \includegraphics[width=.18\columnwidth]{figures/supplementary/lindau00027_000019_given.png}
  }
  \subfigure{%
    \includegraphics[width=.18\columnwidth]{figures/supplementary/lindau00027_000019_sp.png}
  }
  \subfigure{%
    \includegraphics[width=.18\columnwidth]{figures/supplementary/lindau00027_000019_gt.png}
  }
  \subfigure{%
    \includegraphics[width=.18\columnwidth]{figures/supplementary/lindau00027_000019_cnn.png}
  }
  \subfigure{%
    \includegraphics[width=.18\columnwidth]{figures/supplementary/lindau00027_000019_ours.png}
  }\\[-2ex]



  \setcounter{subfigure}{0}
  \subfigure[\scriptsize Input]{%
    \includegraphics[width=.18\columnwidth]{figures/supplementary/lindau00029_000019_given.png}
  }
  \subfigure[\scriptsize Superpixels]{%
    \includegraphics[width=.18\columnwidth]{figures/supplementary/lindau00029_000019_sp.png}
  }
  \subfigure[\scriptsize GT]{%
    \includegraphics[width=.18\columnwidth]{figures/supplementary/lindau00029_000019_gt.png}
  }
  \subfigure[\scriptsize Deeplab]{%
    \includegraphics[width=.18\columnwidth]{figures/supplementary/lindau00029_000019_cnn.png}
  }
  \subfigure[\scriptsize Using BI]{%
    \includegraphics[width=.18\columnwidth]{figures/supplementary/lindau00029_000019_ours.png}
  }%\\[-2ex]

  \mycaption{Street Scene Segmentation}{Example results of street scene segmentation.
  (d)~depicts the DeepLab results, (e)~result obtained by adding bilateral inception (BI) modules (\bi{6}{2}+\bi{7}{6}) between \fc~layers.}
\label{fig:street_visuals-app}
\end{figure*}



\bibliography{lu}
\bibliographystyle{siamplain}


\end{document}
