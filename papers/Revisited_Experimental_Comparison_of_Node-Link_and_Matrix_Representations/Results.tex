\section{Results}

Our results are summarized in Fig.~\ref{fig:results}. By and large, they show that node-link diagrams were better for most types of connectivity tasks ($T1$, $T2$, $T4$, $T5$, $T9$, $T10$, $T13$) thereby invalidating both  H1 and H2. The fact that H1 does not hold is surprising given previous results. Performance on group tasks was generally comparable with the two visualizations, as hypothesized ($H3$), though we found that the AM was better for estimating the number of clusters rather than their interconnectivity. Finally, NL supported memorability tasks better (invalidating $H4$). In particular, NL users outperformed AM users when recalling previously used data ($T11$). 

\subsubsection{Data processing:}

We collected data from 557 individual participants distributed across task groups and conditions as shown in Fig.~\ref{tab:groups_and_users}. We removed a total of $28$ responses from participants who spent an average of $2$ seconds per task and had accuracy in the bottom $10$ percentile. We considered these likely to be random responses by participants attempting to game the study.


\begin{figure*}[t]
  \centering
  \includegraphics[width=0.44\linewidth]{images/table-image}
  \caption{Number of participants in each task group per condition and the number of valid submissions used after data cleaning.}
	\label{tab:groups_and_users}
\end{figure*}

%\begin{wrapfigure}{r}{.5\textwidth}
%\includegraphics[width=.44\textwidth]{images/table-image}
%\caption{Number of participants in each task group per condition and the number of valid submissions used after data cleaning.
%}
%\label{tab:groups_and_users}
%\end{wrapfigure}

To compute the accuracy of node selections ($T1$, $T2$, $T4$), we used the formula $Acc = (\|PS \cap TA\|)/\|TA\|\}$, where $PS$ is the participant's selection and $TA$ is the true answer. To compute answers for tasks involving numeric answers ($T6$, $T10$, $T13$) we used the formula $Acc=max(0,1-\|PA-TA|/|TA|)$, where $PA$ is the participant's answer and $TA$ is the true answer. For other tasks we gave a $1$ to correct answers, and a $0$ to incorrect answers. Since each task type was represented in the study by several repeats, we averaged the accuracies of a task's individual repeats into an accuracy for the task as a whole. 

\subsubsection{Statistical analysis:}

%For each task and condition 
%we tested normalcy with a Shapiro-Wilk test. I

If the data is normally distributed (determined via a Shapiro-Wilk test) we use a t-test analysis between conditions to determine if the observed differences are significant. Otherwise we use a Wilcoxon-Rank-Sum test. We indicate statistically significant differences and effect sizes in Fig.~\ref{fig:results}.

\begin{figure*}[t]
  \centering
  \includegraphics[width=0.95\linewidth]{images/GraphResults.png}
  \caption{Results: accuracy and time. Error bars show one standard error. Statistically significant results and effect sizes are also marked. Tasks $14$, $11$ had no time limits.}
	\label{fig:results}
\end{figure*}


