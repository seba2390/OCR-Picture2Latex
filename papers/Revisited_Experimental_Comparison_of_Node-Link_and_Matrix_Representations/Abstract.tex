\begin{abstract}

Visualizing network data is  applicable in domains such as biology, engineering, and social sciences. We report the results of a study comparing the effectiveness of the two primary techniques for showing network data: node-link diagrams and adjacency matrices. Specifically, an evaluation with a large number of online participants revealed statistically significant differences between the two visualizations.
%in terms of how fast or accurately participants solved a . 
Our work adds to existing research in several ways. First, we explore a broad spectrum of network tasks, many of which had not been previously evaluated. Second, our study uses a large dataset, typical of many real-life networks not explored by previous studies. Third, we leverage crowdsourcing to evaluate many tasks with many participants. 
%Our results thus broaden the understanding of how node-link diagrams and adjacency matrices can support specific tasks in specific types of datasets, and contribute towards a more general understanding of when to use each type of visualization in practice.


\end{abstract}
