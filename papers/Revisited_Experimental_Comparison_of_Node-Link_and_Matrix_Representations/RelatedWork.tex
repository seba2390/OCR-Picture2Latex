\section{Related Work}

Considerable effort has been expended on optimizing NL and AM visualizations to remove clutter, increase the saliency of visual patterns, and support data reading tasks~\cite{von2011visual}. NL, AM, and slight variations thereof have long been used in practice to support analyses of data in a broad range of domains, including proteomic data~\cite{shannon2003cytoscape,jourdan2003tool,barsky2007cerebral,jianu2014display}, brain connectivity data~\cite{alper2013weighted}, social-networks~\cite{viegas2004social}, and engineering~\cite{sedlmair2011cardiogram}.

Static visual encodings were augmented by interaction to support the exploration and analysis of large and intricate datasets typical of real-life applications. Interactive systems that visualize complex relational data use NL~\cite{auber2004tulip,bastian2009gephi,shannon2003cytoscape}, and AM~\cite{fekete2015reorder,behrisch2014visual,bach2014visualizing,blanch2015dendrogramix,rufiange2012treematrix,bezerianos2010geneaquilts,dinkla2012compressed,sheny2007path}. We reviewed such systems to determine common interactions and included them in our evaluated visualizations.
%elmqvist2008zame

While the two types of visualizations have been used broadly for a long time, studying how people parse them visually and %how people interact with them,
 which visualization method better supports specific tasks and datasets, is ongoing. For example, studies by Purchase {\it et al.}~\cite{purchase1997aesthetic,ware2002cognitive,purchase1996validating} consider how node-link layouts impact data readability, eye-tracking research by Huang {\it et al.} reveal visual patterns and measure the cognitive load associated with network exploration~\cite{huang2007using,huang2009measuring}. More recently Jianu {\it et al.} and Saket {\it et al.} consider the performance of node-link diagrams with overlaid group information~\cite{jianu2014display,saket2014node}.
 %while Holten {\it et al.} investigated the best way to draw directed edges in directed graphs~\cite{holten2009user}. 

Our work is one in a series of  studies that compare NL and AM representations. Ghoniem {\it et al.}~\cite{ghoniem2004comparison} evaluated the two approaches on seven connectivity and counting tasks, using interactive visualizations (e.g., node can be selected and highlighted). 
%such as estimating node and edge counts, finding the most connected node, determining if two nodes are connected or if they share a neighbor, and finding paths between them. 
%The two visualizations were interactive and users could select and highlight nodes. 
Synthetic graphs of three sizes (20, 50, 100 nodes) and three densities (0.2, 0.4, and 0.6) were used. The authors found that for small sparse graphs, NL was better in connectivity tasks, but that for large and dense graphs, AM outperformed NL for all tasks. Similarly, Keller {\it et al.}~\cite{keller2006matrices} evaluated six tasks on three real-life networks of varying small sizes ($8$, $22$, $50$) and  three densities (unspecified, 0.2, 0.5). Using both static and interactive variants of NL and AM,  Abuthawabeh {\it et al.} found that the participants were equally able to detect structure in graphs representing code dependencies~\cite{abuthawabeh2013finding}. 
%The authors investigated both static and interactive variants of these visualizations. 
Alper {\it et al.} found that in tasks involving the comparison of weighted graphs, matrices outperform node-link diagrams~\cite{alper2013weighted}. Finally, Christensen {\it et al.}~\cite{christensen2014understanding} evaluated matrix quilts in addition to NL and AM in a smaller scale study. 

Our study adds to what is already known in several ways. 
%above in two principle ways. 
First, we explore a significantly broader range of tasks than earlier studies. These were carefully selected to cover the graph task taxonomy of Lee {\it et al.}~\cite{lee2006task} and the general taxonomy of visualization tasks by Amar {\it et al.}~\cite{amar2005low}. We also considered the task taxonomies for simple graphs~\cite{lee2006task},  clustered graphs~\cite{saket2014group}, and more generally for visualization tasks~\cite{amar2005low,shneiderman1996eyes}, which have been found to be useful in guiding research and informing user study task choices~\cite{jianu2014display,saket2014node}. Second, our study uses a large real-world network, typical of many scale-free networks that arise in practical applications. 
Finally, unlike previous studies, we leverage crowdsourcing, via Amazon's Mechanical Turk, to evaluate many tasks with many participants. 

Note that Mechanical Turk provides access to a diverse participant population~\cite{mason2012conducting,kosara2010mechanical}, and is considered a valid platform for evaluation in general~\cite{paolacci2010running,kosara2010mechanical}, as well as specifically in the context of visualization studies~\cite{heer2010crowdsourcing}. 
%An increasing number of user studies in the 
Many recent visualization studies 
%are conducted using this mode of data collection
are crowdsourced~\cite{chapman2014visualizing,micallef2012assessing,jianu2014display,rodgers2015visualizing,borkin2013makes} and specific platforms for online evaluations are developed, including GraphUnit designed for online evaluation of network visualizations~\cite{okoe2015graphunit}.




