% This must be in the first 5 lines to tell arXiv to use pdfLaTeX, which is strongly recommended.
\pdfoutput=1
% In particular, the hyperref package requires pdfLaTeX in order to break URLs across lines.

\documentclass[11pt,a4paper]{article}

% Remove the "review" option to generate the final version.
\usepackage[acceptedWithA]{tacl2018v2}
% \usepackage[review]{tacl2018v2}
\usepackage{tikz}

% Standard package includes
\usepackage{times}
\usepackage{latexsym}
\usepackage{url}

% For proper rendering and hyphenation of words containing Latin characters (including in bib files)
\usepackage[T1]{fontenc}
% For Vietnamese characters
% \usepackage[T5]{fontenc}
% See https://www.latex-project.org/help/documentation/encguide.pdf for other character sets

% This assumes your files are encoded as UTF8
\usepackage[utf8]{inputenc}

% This is not strictly necessary, and may be commented out,
% but it will improve the layout of the manuscript,
% and will typically save some space.
\usepackage{microtype}

% If the title and author information does not fit in the area allocated, uncomment the following
%
\setlength\titlebox{6cm}
%
% and set <dim> to something 5cm or larger.

% \usepackage{xeCJK}
\usepackage{times,latexsym}
\usepackage{url}
\usepackage[T1]{fontenc}
\usepackage{booktabs} % To thicken table lines
\usepackage{adjustbox}
\usepackage{CJKutf8}
% \usepackage[usenames,dvipsnames]{color}
\usepackage{comment}

\newcommand*\circled[1]{\tikz[baseline=(char.base)]{
    \node[shape=circle,draw,inner sep=0.5pt] (char) {#1};}}
\newcommand{\abr}[1]{\textsc{#1}}
\newcommand{\camelabr}[2]{{\textsc{#1}}{\textsc{#2}}}
\newcommand{\shuowen}{\camelabr{Shuo}{Wen}}
\newcommand{\jiezi}{\camelabr{Jie}{Zi}}
\newcommand{\plm}[0]{{PLM}}
\newcommand{\cws}[0]{\abr{cws}}
\newcommand{\xiaozhi}[1]{{\color{red}Xiaozhi: #1}}
\newcommand{\sentp}[0]{\abr{sp}}
\newcommand{\ulm}[0]{\abr{ulm}}

\newcommand{\chenglei}[1]{
    \textcolor{magenta}{[Chenglei: #1]}
}
% \newcommand{\zzy}[1]{
%     \textcolor{purple!50!blue}{[zzy: #1]}
% }
% \newcommand{\cyf}[1]{
%     \textcolor{teal}{[cyf: #1]}
% }


\title{Sub-Character Tokenization for Chinese Pretrained Language Models}

\author{Chenglei Si$^{1,2*}$, Zhengyan Zhang$^{1*}$, Yingfa Chen$^{1*}$, Fanchao Qi$^{1}$,  \\
 \textbf{Xiaozhi Wang}$^{1}$, \textbf{Zhiyuan Liu}$^{1\dagger}$, \textbf{Yasheng Wang}$^3$, \textbf{Qun Liu}$^3$, \textbf{Maosong Sun}$^{1\dagger}$\\
  $^1$ NLP Group, DCST, IAI, BNRIST, Tsinghua University, Beijing, China \\
  \texttt{\{zy-z19,yingfa-c18,qfc17,wangxz20\}@mails.tsinghua.edu.cn} \\
  \texttt{\{liuzy,sms\}@tsinghua.edu.cn}\\
  $^2$ University of Maryland, College Park, MD, USA \\
  \texttt{clsi@terpmail.umd.edu} \\
        % $^2$Department of Computer Science and Technology, Tsinghua University, Beijing, China  \\
        %  $^3$Institute for Artificial Intelligence, Tsinghua University, Beijing, China \\ 
        %  $^4$Beijing National Research Center for Information Science and Technology, Beijing, China\\
  $^3$Huawei Noah's Ark Lab, Hong Kong, China \\
         \texttt{\{wangyasheng,qun.liu\}@huawei.com}\\
}

% Author information does not appear in the pdf unless the "acceptedWithA" option is given
% See tacl2018v2.sty for other ways to format author information
% \author{
%  Template Author\Thanks{The {\em actual} contributors to this instruction
%  document and corresponding template file are given in Section
%  \ref{sec:contributors}.} \\
%  Template Affiliation/Address Line 1 \\
%  Template Affiliation/Address Line 2 \\
%  Template Affiliation/Address Line 2 \\
%   {\sf template.email@sampledomain.com} \\
% }

\date{}

\begin{document}
\maketitle
\begin{CJK*}{UTF8}{gbsn}
\begin{abstract}
    In this paper, we explore the connection between secret key agreement and secure omniscience within the setting of the multiterminal source model with a wiretapper who has side information. While the secret key agreement problem considers the generation of a maximum-rate secret key through public discussion, the secure omniscience problem is concerned with communication protocols for omniscience that minimize the rate of information leakage to the wiretapper. The starting point of our work is a lower bound on the minimum leakage rate for omniscience, $\rl$, in terms of the wiretap secret key capacity, $\wskc$. Our interest is in identifying broad classes of sources for which this lower bound is met with equality, in which case we say that there is a duality between secure omniscience and secret key agreement. We show that this duality holds in the case of certain finite linear source (FLS) models, such as two-terminal FLS models and pairwise independent network models on trees with a linear wiretapper. Duality also holds for any FLS model in which $\wskc$ is achieved by a perfect linear secret key agreement scheme. We conjecture that the duality in fact holds unconditionally for any FLS model. On the negative side, we give an example of a (non-FLS) source model for which duality does not hold if we limit ourselves to communication-for-omniscience protocols with at most two (interactive) communications.  We also address the secure function computation problem and explore the connection between the minimum leakage rate for computing a function and the wiretap secret key capacity.
  
%   Finally, we demonstrate the usefulness of our lower bound on $\rl$ by using it to derive equivalent conditions for the positivity of $\wskc$ in the multiterminal model. This extends a recent result of Gohari, G\"{u}nl\"{u} and Kramer (2020) obtained for the two-user setting.
  
   
%   In this paper, we study the problem of secret key generation through an omniscience achieving communication that minimizes the 
%   leakage rate to a wiretapper who has side information in the setting of multiterminal source model.  We explore this problem by deriving a lower bound on the wiretap secret key capacity $\wskc$ in terms of the minimum leakage rate for omniscience, $\rl$. 
%   %The former quantity is defined to be the maximum secret key rate achievable, and the latter one is defined as the minimum possible leakage rate about the source through an omniscience scheme to a wiretapper. 
%   The main focus of our work is the characterization of the sources for which the lower bound holds with equality \textemdash it is referred to as a duality between secure omniscience and wiretap secret key agreement. For general source models, we show that duality need not hold if we limit to the communication protocols with at most two (interactive) communications. In the case when there is no restriction on the number of communications, whether the duality holds or not is still unknown. However, we resolve this question affirmatively for two-user finite linear sources (FLS) and pairwise independent networks (PIN) defined on trees, a subclass of FLS. Moreover, for these sources, we give a single-letter expression for $\wskc$. Furthermore, in the direction of proving the conjecture that duality holds for all FLS, we show that if $\wskc$ is achieved by a \emph{perfect} secret key agreement scheme for FLS then the duality must hold. All these results mount up the evidence in favor of the conjecture on FLS. Moreover, we demonstrate the usefulness of our lower bound on $\wskc$ in terms of $\rl$ by deriving some equivalent conditions on the positivity of secret key capacity for multiterminal source model. Our result indeed extends the work of Gohari, G\"{u}nl\"{u} and Kramer in two-user case.
\end{abstract}



\section{Introduction}

{\let\thefootnote\relax\footnotetext{$^*$ Equal contribution}}
{\let\thefootnote\relax\footnotetext{$^\dagger$ Corresponding authors}}

\begin{figure}[t]
\begin{center}
   \includegraphics[width=1.0\linewidth]{figures/nas_comp_v3}
\end{center}
   \vspace{-4mm}
   \caption{The comparison between NetAdaptV2 and related works. The number above a marker is the corresponding total search time measured on NVIDIA V100 GPUs.}
\label{fig:nas_comparison}
\end{figure}

\section{Introduction}
\label{sec:introduction}

Neural architecture search (NAS) applies machine learning to automatically discover deep neural networks (DNNs) with better performance (e.g., better accuracy-latency trade-offs) by sampling the search space, which is the union of all discoverable DNNs. The search time is one key metric for NAS algorithms, which accounts for three steps: 1) training a \emph{super-network}, whose weights are shared by all the DNNs in the search space and trained by minimizing the loss across them, 2) training and evaluating sampled DNNs (referred to as \emph{samples}), and 3) training the discovered DNN. Another important metric for NAS is whether it supports non-differentiable search metrics such as hardware metrics (e.g., latency and energy). Incorporating hardware metrics into NAS is the key to improving the performance of the discovered DNNs~\cite{eccv2018-netadapt, Tan2018MnasNetPN, cai2018proxylessnas, Chen2020MnasFPNLL, chamnet}.


There is usually a trade-off between the time spent for the three steps and the support of non-differentiable search metrics. For example, early reinforcement-learning-based NAS methods~\cite{zoph2017nasreinforcement, zoph2018nasnet, Tan2018MnasNetPN} suffer from the long time for training and evaluating samples. Using a super-network~\cite{yu2018slimmable, Yu_2019_ICCV, autoslim_arxiv, cai2020once, yu2020bignas, Bender2018UnderstandingAS, enas, tunas, Guo2020SPOS} solves this problem, but super-network training is typically time-consuming and becomes the new time bottleneck. The gradient-based methods~\cite{gordon2018morphnet, liu2018darts, wu2018fbnet, fbnetv2, cai2018proxylessnas, stamoulis2019singlepath, stamoulis2019singlepathautoml, Mei2020AtomNAS, Xu2020PC-DARTS} reduce the time for training a super-network and training and evaluating samples at the cost of sacrificing the support of non-differentiable search metrics. In summary, many existing works either have an unbalanced reduction in the time spent per step (i.e., optimizing some steps at the cost of a significant increase in the time for other steps), which still leads to a long \emph{total} search time, or are unable to support non-differentiable search metrics, which limits the performance of the discovered DNNs.

In this paper, we propose an efficient NAS algorithm, NetAdaptV2, to significantly reduce the \emph{total} search time by introducing three innovations to \emph{better balance} the reduction in the time spent per step while supporting non-differentiable search metrics:

\textbf{Channel-level bypass connections (mainly reduce the time for training and evaluating samples, Sec.~\ref{subsec:channel_level_bypass_connections})}: Early NAS works only search for DNNs with different numbers of filters (referred to as \emph{layer widths}). To improve the performance of the discovered DNN, more recent works search for DNNs with different numbers of layers (referred to as \emph{network depths}) in addition to different layer widths at the cost of training and evaluating more samples because network depths and layer widths are usually considered independently. In NetAdaptV2, we propose \emph{channel-level bypass connections} to merge network depth and layer width into a single search dimension, which requires only searching for layer width and hence reduces the number of samples.

\textbf{Ordered dropout (mainly reduces the time for training a super-network, Sec.~\ref{subsec:ordered_droput})}: We adopt the idea of super-network to reduce the time for training and evaluating samples. In previous works, \emph{each} DNN in the search space requires one forward-backward pass to train. As a result, training multiple DNNs in the search space requires multiple forward-backward passes, which results in a long training time. To address the problem, we propose \emph{ordered dropout} to jointly train multiple DNNs in a \emph{single} forward-backward pass, which decreases the required number of forward-backward passes for a given number of DNNs and hence the time for training a super-network.

\textbf{Multi-layer coordinate descent optimizer (mainly reduces the time for training and evaluating samples and supports non-differentiable search metrics, Sec.~\ref{subsec:optimizer}):} NetAdaptV1~\cite{eccv2018-netadapt} and MobileNetV3~\cite{Howard_2019_ICCV}, which utilizes NetAdaptV1, have demonstrated the effectiveness of the single-layer coordinate descent (SCD) optimizer~\cite{book2020sze} in discovering high-performance DNN architectures. The SCD optimizer supports both differentiable and non-differentiable search metrics and has only a few interpretable hyper-parameters that need to be tuned, such as the per-iteration resource reduction. However, there are two shortcomings of the SCD optimizer. First, it only considers one layer per optimization iteration. Failing to consider the joint effect of multiple layers may lead to a worse decision and hence sub-optimal performance. Second, the per-iteration resource reduction (e.g., latency reduction) is limited by the layer with the smallest resource consumption (e.g., latency). It may take a large number of iterations to search for a very deep network because the per-iteration resource reduction is relatively small compared with the network resource consumption. To address these shortcomings,  we propose the \emph{multi-layer coordinate descent (MCD) optimizer} that considers multiple layers per optimization iteration to improve performance while reducing search time and preserving the support of non-differentiable search metrics.

Fig.~\ref{fig:nas_comparison} (and Table~\ref{tab:nas_result}) compares NetAdaptV2 with related works. NetAdaptV2 can reduce the search time by up to $5.8\times$ and $2.4\times$ on ImageNet~\cite{imagenet_cvpr09} and NYU Depth V2~\cite{nyudepth} respectively and discover DNNs with better performance than state-of-the-art NAS works. Moreover, compared to NAS-discovered MobileNetV3~\cite{Howard_2019_ICCV}, the discovered DNN has $1.8\%$ higher accuracy with the same latency.




\section{Method}

\section{Proposed Method: SyMFM6D}

We propose a deep multi-directional fusion approach called SyMFM6D that estimates the 6D object poses of all objects in a cluttered scene based on multiple RGB-D images while considering object symmetries. 
In this section, we define the task of multi-view 6D object pose estimation and present our multi-view deep fusion architecture.

\begin{figure*}[tbh]
  \vspace{2mm}
  \centering
  \includegraphics[page=1, trim = 5mm 40mm 5mm 42mm, clip,  width=1.0\linewidth]{figures/SyMFM6D_architecture4_2.pdf}
   \caption{Network architecture of SyMFM6D which fuses $N$ RGB-D input images. Our method converts the $N$ depth images to a single point cloud which is processed by an encoder-decoder point cloud network. The $N$ RGB images are processed by an encoder-decoder CNN. Every hierarchy contains a point-to-pixel fusion module and a pixel-to-point fusion module for deep multi-directional multi-view fusion. We utilize three MLPs with four layers each to regress 3D keypoint offsets, center point offsets, and semantic labels based on the final features. The 6D object poses are computed as in \cite{pvn3d} based on mean shift clustering and least-squares fitting. We train our network by minimizing our proposed symmetry-aware multi-task loss function using precomputed object symmetries. $N_p$ is the number of points in the point cloud. $H$ and $W$ are height and width of the RGB images.}
   \label{fig_architecture}
   \vspace{-2mm}
\end{figure*}


6D object pose estimation describes the task of predicting a rigid transformation $\boldsymbol p = [\boldsymbol R |  \boldsymbol t] \in SE(3)$ which transforms the coordinates of an observed object from the object coordinate system into the camera coordinate system. This transformation is called 6D object pose because it is composed of a 3D rotation $\boldsymbol R \in SO(3)$ and a 3D translation $\boldsymbol t \in \mathbb{R}^3$. 
The designated aim of our approach is to jointly estimate the 6D poses of all objects in a given cluttered scene using multiple RGB-D images which depict the scene from multiple perspectives. We assume the 3D models of the objects and the camera poses to be known as proposed by \cite{mv6d}.



\subsection{Network Overview}

Our symmetry-aware multi-view network consists of three stages which are visualized in \cref{fig_architecture}. 
The first stage receives one or multiple RGB-D images and extracts visual features as well as geometric features which are fused to a joint representation of the scene. 
The second stage performs a detection of predefined 3D keypoints and an instance semantic segmentation.
Based on the keypoints and the information to which object the keypoints belong, we compute the 6D object poses with a least-squares fitting algorithm \cite{leastSquares} in the third stage.



\subsection{Multi-View Feature Extraction}

To efficiently predict keypoints and semantic labels, the first stage of our approach learns a compact representation of the given scene by extracting and merging features from all available RGB-D images in a deep multi-directional fusion manner. For that, we first separate the set of RGB images $\text{RGB}_1, ..., \text{RGB}_N$ from their corresponding depth images $\text{Dpt}_1$, ..., $\text{Dpt}_N$. The $N$ depth images are converted into point clouds, transformed into the coordinate system of the first camera, and merged to a single point cloud using the known camera poses as in \cite{mv6d}. 
Unlike \cite{mv6d}, we employ a point cloud network based on RandLA-Net \cite{hu2020randla} with an encoder-decoder architecture using skip connections.
The point cloud network learns geometric features from the fused point cloud and considers visual features from the multi-directional point-to-pixel fusion modules as described in \cref{sec_multi_view_fusion}.

The $N$ RGB images are independently processed by a CNN with encoder-decoder architecture using the same weights for all $N$ views. The CNN learns visual features while considering geometric features from the multi-directional pixel-to-point fusion modules. We followed \cite{ffb6d} and build the encoder upon a ResNet-34 \cite{resnet} pretrained on ImageNet~\cite{imagenet} and the decoder upon a PSPNet \cite{pspnet}. 

After the encoding and decoding procedures including several multi-view feature fusions, we collect the visual features from each view corresponding to the final geometric feature map and concatenate them. The output is a compact feature tensor containing the relevant information about the entire scene which is used for keypoint detection and instance semantic segmentation as described in \cref{sec_keypoint_detection_and_segmentation}.


\begin{figure*}[tbh]
  \vspace{2mm} 
  \centering  
\begin{subfigure}[b]{0.48\textwidth}
  \includegraphics[page=1, trim = 1mm 6mm 6mm 6mm, clip,  width=1.0\linewidth]{figures/p2r_8.pdf}
   \caption{Point-to-pixel fusion module.~~~~}
   \label{fig_pt2px_fusion}
\end{subfigure}
\begin{subfigure}[b]{0.48\textwidth}
  \centering  
  \includegraphics[page=1, trim = 1mm 6mm 6mm 6mm, clip,  width=1.0\linewidth]{figures/r2p_8.pdf}
   \caption{Pixel-to-point fusion module.~~~~~}
   \label{fig_px2pt_fusion}
   \end{subfigure}
      \caption{Overview of our proposed multi-directional multi-view fusion modules. They combine pixel-wise visual features and point-wise geometric features by exploiting the correspondence between pixels and points using the nearest neighbor algorithm. We compute the resulting features using multiple shared MLPs with a single layer and max-pooling.
      For simplification, we depict an example with $N=2$ views and $K_\text{i}=K_\text{p}=3$ nearest neighbors. The points of ellipsis (...) illustrate the generalization for an arbitrary number of views $N$. Please refer to \cite{ffb6d} for better understanding the basic operations.
      }
   \label{fig_fusion_modules}
   \vspace{-1mm}
\end{figure*}



\subsection{Multi-View Feature Fusion}
\label{sec_multi_view_fusion}
In order to efficiently fuse the visual and geometric features from multiple views, we extend the fusion modules of FFB6D~\cite{ffb6d} from bi-directional fusion to \emph{multi-directional fusion}. We present two types of multi-directional fusion modules which are illustrated in \cref{fig_fusion_modules}.
Both types of fusion modules take the pixel-wise visual feature maps and the point-wise geometric feature maps from each view, combine them, and compute a new feature map.
This process requires a correspondence between pixel-wise and point-wise features which we obtain by computing an XYZ map for each RGB feature map based on the depth data of each pixel using the camera intrinsic matrix as in \cite{ffb6d}. To deal with the changing dimensions at different layers, we use the centers of the convolutional kernels as new coordinates of the feature maps and resize the XYZ map to the same size using nearest interpolation as proposed in \cite{ffb6d}.

The \emph{point-to-pixel} fusion module in \cref{fig_pt2px_fusion} computes a 
fused feature map $\bb F_\text{f}$ based on the image features $\bb F_{\text{i}}(v)$ of all views $v \in \{1, \ldots, N\}$.
We collect the $K_\text{p}$ nearest point features $\bb F_{\text{p}_k}(v)$ with $k \in \{1, \ldots, K_\text{p}\}$ from the point cloud for each pixel-wise feature and each view independently by computing the nearest neighbors according to the Euclidean distance in the XYZ map. Subsequently, we process them by a shared MLP before aggregating them by max-pooling, i.e.,
\begin{align} 
    \widetilde{\bb F}_{\text{p}}(v) = \max_{k \in \{1, \ldots, K_\text{p}\}} 
    \Big( \text{MLP}_\text{p}(\bb F_{\text{p}_k}(v)) \Big).
    \label{eq_p2r}
\end{align}
Finally, we apply a second shared MLP to fuse all features $\bb F_\text{i}$ and 
$\widetilde{\bb F}_{\text{p}}$ as 
$\bb F_{\text{f}} = \text{MLP}_\text{fp}(\widetilde{\bb F}_{\text{p}} \oplus \bb F_\text{i})$ where $\oplus$ denotes the concatenate operation.


The \emph{pixel-to-point} fusion module in \cref{fig_px2pt_fusion} collects the $K_\text{i}$ nearest image features $\bb F_{\text{i}_k}(\textrm{i2v}(i_k))$ with $k\in\{1, ..., K_\text{i}\}$. $\textrm{i2v}(i_k)$ is a mapping that maps the index of an image feature to its corresponding view. This procedure is performed for each point feature vector $\bb F_\text{p}(n)$.
We aggregate the collected image features by max-pooling and apply a shared MLP, i.e.,
\begin{align}
    \widetilde{\bb F}_{\text{i}} = \text{MLP}_\text{i} 
    \left( \max_{k \in \{1, \ldots, K_\text{i}\}} 
    \Big( \bb F_{\text{i}_k}(\textrm{i2v}(i_k)) \Big)  
    \right).
    \label{eq_r2p}
\end{align}
One more shared MLP fuses the resulting image features $\widetilde{\bb F}_{\text{i}}$ with the point features $\bb F_\text{p}$ as 
$\bb F_{\text{f}} = \text{MLP}_\text{fi}(\widetilde{\bb F}_{\text{i}} \oplus \bb F_\text{p})$.




\subsection{Keypoint Detection and Segmentation}
\label{sec_keypoint_detection_and_segmentation}
The second stage of our SyMFM6D network contains modules for 3D keypoint detection and instance semantic segmentation following \cite{mv6d}. However, unlike \cite{mv6d}, we use the SIFT-FPS algorithm \cite{lowe1999sift} as proposed by FFB6D \cite{ffb6d} to define eight target keypoints for each object class. SIFT-FPS yields keypoints with salient features which are easier to detect.
Based on the extracted features, we apply two shared MLPs to estimate the translation offsets from each point of the fused point cloud to each target keypoint and to each object center.
We obtain the actual point proposals by adding the translation offsets to the respective points of the fused point cloud. 
Applying the mean shift clustering algorithm \cite{cheng1995meanshift} results in predictions for the keypoints and the object centers.
We employ one more shared MLP 
for estimating the object class of each point in the fused point cloud as in \cite{pvn3d}.



\subsection{6D Pose Computation via Least-Squares Fitting}

Following \cite{pvn3d}, we use the least-squares fitting algorithm \cite{leastSquares} to compute the 6D poses of all objects based on the estimated keypoints. As the $M$ estimated keypoints $\boldsymbol{\widehat{k}}_1, ..., \boldsymbol{\widehat{k}}_M$ are in the coordinate system of the first camera and the target keypoints $\boldsymbol k_1, ..., \boldsymbol k_M$ are in the object coordinate system, least-squares fitting calculates the rotation matrix $\boldsymbol R$ and the translation vector $\boldsymbol t$ of the 6D pose by minimizing the squared loss
\begin{equation}
    L_\text{Least-squares} = \sum_{i=1}^M \norm{\boldsymbol{\widehat{k}_i} - (\boldsymbol R \boldsymbol k_i + \boldsymbol t)}_2^2.
\end{equation}



\subsection{Symmetry-aware Keypoint Detection}

Most related work, including \cite{pvn3d, ffb6d}, and \cite{mv6d} does not specifically consider object symmetries. 
However, symmetries lead to ambiguities in the predicted keypoints as multiple 6D poses can have the same visual and geometric appearance. 
Therefore, we introduce a novel symmetry-aware training procedure for the 3D keypoint detection including a novel symmetry-aware objective function to make the network predicting either the original set of target keypoints for an object or a rotated version of the set corresponding to one object symmetry. Either way, we can still apply the least-squares fitting which efficiently computes an estimate of the target 6D pose or a rotated version corresponding to an object symmetry. To do so, we precompute the set $\boldsymbol{S}_I$ of all rotational symmetric transformations for the given object instance $I$ with a stochastic gradient
descent algorithm \cite{sgdr}.
Given the known mesh of an object and an initial estimate for the symmetry axis, we transform the object mesh along the symmetry axis estimate and optimize the symmetry axis iteratively by minimizing the ADD-S metric \cite{hinterstoisser2012model}.
Reflectional symmetries which can be represented as rotational symmetries are handled as rotational symmetries. 
Other reflectional symmetries are ignored, since the reflection cannot be expressed as an Euclidean transformation.
To consider continuous rotational symmetries, we discretize them into 16 discrete rotational symmetry transformations.

We extend the keypoints loss function of \cite{pvn3d} to become symmetry-aware such that it predicts the keypoints of the closest symmetric transformation, i.e. 
\begin{equation}
    L_\text{kp}(\mathcal{I}) = \frac{1}{N_I} 
    \min_{\boldsymbol{S} \in \boldsymbol{S}_I} 
    \sum_{i \in \mathcal{I}} \sum_{j=1}^M 
    \norm{\boldsymbol{x}_{ij} - \boldsymbol{S}\boldsymbol{\widehat{x}}_{ij}}_2, 
\label{eq_keypoint_loss}
\end{equation}
where $N_I$ is the number of points in the point cloud for object instance $I$, $M$ is the number of target keypoints per object, and $\mathcal{I}$ is the set of all point indices that belong to object instance $I$.  
The vector $\boldsymbol{\widehat{x}}_{ij}$ is the predicted keypoint offset for the $i$-th point and the $j$-th keypoint while $\boldsymbol{x}_{ij}$ is the corresponding ground truth. 



\subsection{Objective Function}

We train our network by minimizing the multi-task loss function
\begin{equation}
 \label{eq_total_loss}
    L_\text{multi-task} = \lambda_1 L_\text{kp} 
    + \lambda_2 L_\text{semantic}  
    +  \lambda_3 L_\text{cp},
\end{equation}
where $L_\text{kp}$ is our symmetry-aware keypoint loss from \cref{eq_keypoint_loss}.
$L_\text{cp}$ is an L1 loss for the center point prediction, $L_\text{semantic}$ is a Focal loss \cite{focalLoss} for the instance semantic segmentation, and $\lambda_1=2$, $\lambda_2=1$, and $\lambda_3=1$ are the weights for the individual loss functions as in \cite{ffb6d}.


\section{Experiment Setup}


\begin{figure*}
    \centering
    \includegraphics[width=1.0\linewidth]{Figures/imgs/tsne_motivation.pdf}
    \caption{$t$-SNE~\cite{tsne} visualizations of features learned from ER and \frameworkName on the test set of CIFAR-10.
    When learning new classes, ER suffers serious class confusion probably because shortcut learning. In contrast, \frameworkName significantly mitigates the forgetting.
    }
    \label{fig:tsne_motivation}
\end{figure*}
\begin{table*}[ht]
\small
\begin{center}
\resizebox{\linewidth}{!}{
\begin{tabular}{rrrrrrrrrrrr}
\shline
\multirow{2}{*}{Method}  & \multicolumn{3}{c}{CIFAR-10}   && \multicolumn{3}{c}{CIFAR-100}  && \multicolumn{3}{c}{TinyImageNet} \\ \cline{2-4}\cline{6-8}\cline{10-12}
       & $M=0.1k$   & $M=0.2k$   & $M=0.5k$     && $M=0.5k$     & $M=1k$     & $M=2k$     && $M=1k$      & $M=2k$ & $M=4k$   \\ \midrule
iCaRL~\cite{iCaRL}    & 31.0\std{$\pm$1.2} & 33.9\std{$\pm$0.9} & 42.0\std{$\pm$0.9} && 12.8\std{$\pm$0.4}  & 16.5\std{$\pm$0.4}  & 17.6\std{$\pm$0.5} && 5.0\std{$\pm$0.3}   & 6.6\std{$\pm$0.4} & 7.8\std{$\pm$0.4} \\ 
DER++~\cite{DER++}   & 31.5\std{$\pm$2.9} & 39.7\std{$\pm$2.7} & 50.9\std{$\pm$1.8} && 16.0\std{$\pm$0.6}  & 21.4\std{$\pm$0.9}  & 23.9\std{$\pm$1.0} && 3.7\std{$\pm$0.4} & 5.1\std{$\pm$0.8} & 6.8\std{$\pm$0.6} \\ 
PASS~\cite{protoAug}    & 33.7\std{$\pm$2.2} & 33.7\std{$\pm$2.2} & 33.7\std{$\pm$2.2} && 7.5\std{$\pm$0.7}  & 7.5\std{$\pm$0.7}  & 7.5\std{$\pm$0.7} && 0.5\std{$\pm$0.1}   & 0.5\std{$\pm$0.1} & 0.5\std{$\pm$0.1} \\ 
\hline
AGEM~\cite{AGEM}    & 17.7\std{$\pm$0.3} & 17.5\std{$\pm$0.3} & 17.5\std{$\pm$0.2} && 5.8\std{$\pm$0.1}  & 5.9\std{$\pm$0.1}  & 5.8\std{$\pm$0.1} && 0.8\std{$\pm$0.1}   & 0.8\std{$\pm$0.1} & 0.8\std{$\pm$0.1} \\ 
GSS~\cite{GSS}     & 18.4\std{$\pm$0.2} & 19.4\std{$\pm$0.7} & 25.2\std{$\pm$0.9} && 8.1\std{$\pm$0.2}  & 9.4\std{$\pm$0.5}  & 10.1\std{$\pm$0.8} && 1.1\std{$\pm$0.1}   & 1.5\std{$\pm$0.1} & 2.4\std{$\pm$0.4} \\ 
ER~\cite{ER}      & 19.4\std{$\pm$0.6} & 20.9\std{$\pm$0.9} & 26.0\std{$\pm$1.2} && 8.7\std{$\pm$0.3}  & 9.9\std{$\pm$0.5}  & 10.7\std{$\pm$0.8} && 1.2\std{$\pm$0.1}   & 1.5\std{$\pm$0.2} & 2.0\std{$\pm$0.2} \\ 
MIR~\cite{MIR}     & 20.7\std{$\pm$0.7} & 23.5\std{$\pm$0.8} & 29.9\std{$\pm$1.2} && 9.7\std{$\pm$0.3}  & 11.2\std{$\pm$0.4}  & 13.0\std{$\pm$0.7} && 1.4\std{$\pm$0.1}   & 1.9\std{$\pm$0.2} & 2.9\std{$\pm$0.3} \\ 
GDumb~\cite{GDumb}   & 23.3\std{$\pm$1.3} & 27.1\std{$\pm$0.7} & 34.0\std{$\pm$0.8} && 8.2\std{$\pm$0.2}  & 11.0\std{$\pm$0.4}  & 15.3\std{$\pm$0.3} && 4.6\std{$\pm$0.3}   & 6.6\std{$\pm$0.2} & 10.0\std{$\pm$0.3} \\ 
ASER~\cite{ASER}   & 20.0\std{$\pm$1.0} & 22.8\std{$\pm$0.6} & 31.6\std{$\pm$1.1} && 11.0\std{$\pm$0.3}  & 13.5\std{$\pm$0.3}  & 17.6\std{$\pm$0.4} && 2.2\std{$\pm$0.1}   & 4.2\std{$\pm$0.6} & 8.4\std{$\pm$0.7} \\ 
SCR~\cite{SCR}     & 40.2\std{$\pm$1.3} & 48.5\std{$\pm$1.5} & 59.1\std{$\pm$1.3} && 19.3\std{$\pm$0.6}  & 26.5\std{$\pm$0.5}  & 32.7\std{$\pm$0.3} && 8.9\std{$\pm$0.3}   & 14.7\std{$\pm$0.3} & 19.5\std{$\pm$0.3} \\ 
CoPE~\cite{online_pro_ema}  & 33.5\std{$\pm$3.2} & 37.3\std{$\pm$2.2} & 42.9\std{$\pm$3.5} && 11.6\std{$\pm$0.7}  & 14.6\std{$\pm$1.3}  & 16.8\std{$\pm$0.9} && 2.1\std{$\pm$0.3}   & 2.3\std{$\pm$0.4} & 2.5\std{$\pm$0.3} \\
DVC~\cite{DVC} & 35.2\std{$\pm$1.7}  & 41.6\std{$\pm$2.7} & 53.8\std{$\pm$2.2} &&  15.4\std{$\pm$0.7} & 20.3\std{$\pm$1.0} & 25.2\std{$\pm$1.6} && 4.9\std{$\pm$0.6} &  7.5\std{$\pm$0.5} & 10.9\std{$\pm$1.1} \\ 
OCM~\cite{OCM} & 47.5\std{$\pm$1.7}  & 59.6\std{$\pm$0.4} & 70.1\std{$\pm$1.5} && 19.7\std{$\pm$0.5} & 27.4\std{$\pm$0.3} & 34.4\std{$\pm$0.5} && 10.8\std{$\pm$0.4} & 15.4\std{$\pm$0.4} & 20.9\std{$\pm$0.7} \\ 
\hline
\frameworkName (\textbf{ours}) & \textbf{57.8}\std{$\pm$1.1} & \textbf{65.5}\std{$\pm$1.0} & \textbf{72.6}\std{$\pm$0.8} && \textbf{22.7}\std{$\pm$0.7} & \textbf{30.0}\std{$\pm$0.4} & \textbf{35.9}\std{$\pm$0.6} && \textbf{11.9}\std{$\pm$0.3} & \textbf{16.9}\std{$\pm$0.4} &  \textbf{22.1}\std{$\pm$0.4}
\\ 
\shline
\end{tabular}
}
\end{center}
\caption{Average Accuracy~(higher is better) on three benckmark datasets with different memory bank sizes $M$. All results are the average and standard deviation of 15 runs.}
\label{tab:acc}
\end{table*}

\section{Experiments}
\subsection{Experimental Setup}
\paragraph{Datasets.}
We use three image classification benchmark datasets, including \textbf{CIFAR-10}~\cite{cifar10_100}, \textbf{CIFAR-100}~\cite{cifar10_100}, and \textbf{TinyImageNet}~\cite{tinyImageNet}, to evaluate the performance of online CIL methods. 
Following~\cite{ASER, SCR, DVC}, we split CIFAR-10 into 5 disjoint tasks, where each task has 2 disjoint classes, 10,000 samples for training, and 2,000 samples for testing, and split CIFAR-100 into 10 disjoint tasks, where each task has 10 disjoint classes, 5,000 samples for training, and 1,000 samples for testing.
Following~\cite{OCM}, we split TinyImageNet into 100 disjoint tasks, where each task has 2 disjoint classes, 1,000 samples for training, and 100 samples for testing.
Note that the order of tasks is fixed in all experimental settings.

\paragraph{Baselines.}
We compare our \frameworkName with 13 baselines, including 10 replay-based online CL baselines: {AGEM}~\cite{AGEM}, {MIR}~\cite{MIR}, {GSS}~\cite{GSS}, {ER}~\cite{ER}, {GDumb}~\cite{GDumb}, {ASER}~\cite{ASER}, {SCR}~\cite{SCR}, {CoPE}~\cite{online_pro_ema}, {DVC}~\cite{DVC}, and {OCM}~\cite{OCM}; 3 offline CL baselines that use knowledge distillation by running them in one epoch: {iCaRL}~\cite{iCaRL}, {DER++}~\cite{DER++}, and PASS~\cite{protoAug}. Note that PASS is a non-exemplar method.

\paragraph{Evaluation metrics.}
We use Average Accuracy and Average Forgetting~\cite{ASER, DVC} to measure the performance of our framework in online CIL. Average Accuracy evaluates the accuracy of the test sets from all seen tasks, defined as $\text {Average Accuracy} =\frac{1}{T} \sum_{j=1}^T a_{T, j},$
where $a_{i, j}$ is the accuracy on task $j$ after the model is trained from task $1$ to $i$.
Average Forgetting represents how much the model forgets about each task after being trained on the final task, defined as
$\text { Average Forgetting } =\frac{1}{T-1} \sum_{j=1}^{T-1} f_{T, j}, 
\text { where } f_{i, j}=\max _{k \in\{1, \ldots, i-1\}} a_{k, j}-a_{i, j}.$

\paragraph{Implementation details.}
We use ResNet18~\cite{ResNet} as the backbone $f$ and a linear layer as the projection head $g$ like~\cite{SCR, OCM, Co2L}; the hidden dim in $g$ is set to 128 as~\cite{SimCLR}. We also employ a linear layer as the classifier $\varphi$. We train the model from scratch with Adam optimizer and an initial learning rate of $5\times10^{-4}$ for all datasets. The weight decay is set to $1.0\times10^{-4}$. Following~\cite{ASER, DVC}, we set the batch size $N$ as 10, and following~\cite{OCM} the replay batch size $m$ is set to 64. 
For CIFAR-10, we set the ratio of \dataaugname $\alpha = 0.25$. For CIFAR-100 and TinyImageNet, $\alpha$ is set to $0.1$. The temperature $\tau = 0.5$ and $\tau^{\prime} = 0.07$.
For baselines, we also use ResNet18 as their backbone and set the same batch size and replay batch size for fair comparisons.
We reproduce all baselines in the same environment with their source code and default settings; see Appendix~\ref{appendix:baselines} for implementation details about all baselines.
We report the average results across 15 runs for all experiments.



\paragraph{Data augmentation.}
Similar to data augmentations used in SimCLR~\cite{SimCLR}, we use resized-crop, horizontal-flip, and gray-scale as our data augmentations. For all baselines, we also use these augmentations. In addition, for DER++\cite{DER++}, SCR~\cite{SCR}, and DVC~\cite{DVC}, we follow their default settings and use their own extra data augmentations. OCM~\cite{OCM} uses extra rotation augmentations, which are also used in \frameworkName.


\subsection{Motivation Justification}
\label{pre_exp}
\paragraph{Shortcut learning in online CL.}
Shortcut learning is severe in online CL since the model cannot learn sufficient representative features due to the single-pass data stream. To intuitively demonstrate this issue,  
we conduct GradCAM++~\cite{Grad-cam++} on the training set of CIFAR-10 ($M=0.2k$) after the model is trained incrementally, as shown in Fig.~\ref{fig:heatmap}.
Each row in Fig.~\ref{fig:heatmap} represents a task with two classes.
We can observe that although ER and DVC predict the correct class, the models actually take shortcuts and focus on some object-unrelated features. 
An interesting phenomenon is that ER tends to take shortcuts in each task. For example, ER learns the sky on both the airplane class in task 1 (the first row) and the bird class in task 2 (the second row) . Thus, ER forgets almost all the knowledge of the old classes.  
DVC maximizes the mutual information between instances like contrastive learning~\cite{SimCLR, MoCo}, which only partially alleviates shortcut learning in online CL. 
In contrast, \frameworkName focuses on the representative features of the objects themselves. The results confirm that learning representative features is crucial against shortcut learning; see Appendix~\ref{appendix:more_visual} for more visual explanations.


\begin{table*}[htbp]
\small
\begin{center}
\resizebox{\linewidth}{!}{
\begin{tabular}{rrrrrrrrrrrr}
\shline
\multirow{2}{*}{Method}  & \multicolumn{3}{c}{CIFAR-10}   && \multicolumn{3}{c}{CIFAR-100}  && \multicolumn{3}{c}{TinyImageNet} \\ \cline{2-4}\cline{6-8}\cline{10-12}
       &  $M=0.1k$   &  $M=0.2k$   &  $M=0.5k$     &&  $M=0.5k$     &  $M=1k$     &  $M=2k$    &&  $M=1k$      &  $M=2k$ &  $M=4k$    \\ \midrule
iCaRL~\cite{iCaRL}    & 52.7\std{$\pm$1.0} & 49.3\std{$\pm$0.8} & 38.3\std{$\pm$0.9} && 16.5\std{$\pm$1.0}  & 11.2\std{$\pm$0.4}  & 10.4\std{$\pm$0.4} && 9.9\std{$\pm$0.5}   & 10.1\std{$\pm$0.5} & 9.7\std{$\pm$0.6} \\ 
DER++~\cite{DER++}   & 57.8\std{$\pm$4.1} & 46.7\std{$\pm$3.6} & 33.6\std{$\pm$3.5} && 41.0\std{$\pm$1.1} & 34.8\std{$\pm$1.1} & 33.2\std{$\pm$1.2} && 77.8\std{$\pm$1.0} & 74.9\std{$\pm$0.6} & 73.2\std{$\pm$0.8}  \\ 
PASS~\cite{protoAug}    & 21.2\std{$\pm$2.2} & 21.2\std{$\pm$2.2} & 21.2\std{$\pm$2.2} && 10.6\std{$\pm$0.9}  & 10.6\std{$\pm$0.9}  & 10.6\std{$\pm$0.9} && 27.0\std{$\pm$2.4}   & 27.0\std{$\pm$2.4} & 27.0\std{$\pm$2.4} \\ 
\hline
AGEM~\cite{AGEM}    & 64.8\std{$\pm$0.7} & 64.8\std{$\pm$0.7} & 64.5\std{$\pm$0.5} && 41.7\std{$\pm$0.8} & 41.8\std{$\pm$0.7} & 41.7\std{$\pm$0.6} && 73.9\std{$\pm$0.7} & 73.1\std{$\pm$0.7} & 72.9\std{$\pm$0.5} \\ 
GSS~\cite{GSS}     & 67.1\std{$\pm$0.6} & 65.8\std{$\pm$0.6} & 61.2\std{$\pm$1.2} && 48.7\std{$\pm$0.8} & 46.7\std{$\pm$1.3} & 44.7\std{$\pm$1.1} && 78.9\std{$\pm$0.7} & 77.0\std{$\pm$0.5} & 75.2\std{$\pm$0.7} \\ 
ER~\cite{ER}      & 64.7\std{$\pm$1.1} & 62.9\std{$\pm$1.0} & 57.5\std{$\pm$1.8} && 47.0\std{$\pm$1.0} & 46.4\std{$\pm$0.8} & 44.7\std{$\pm$1.5} && 79.1\std{$\pm$0.6} & 77.7\std{$\pm$0.6} & 76.3\std{$\pm$0.5} \\ 
MIR~\cite{MIR}     & 62.6\std{$\pm$1.0} & 58.5\std{$\pm$1.4} & 51.1\std{$\pm$1.1} && 45.7\std{$\pm$0.9} & 44.2\std{$\pm$1.3} & 42.3\std{$\pm$1.0} && 75.3\std{$\pm$0.9} & 71.5\std{$\pm$1.0} & 66.8\std{$\pm$0.8} \\ 
GDumb~\cite{GDumb}   & 28.5\std{$\pm$1.4} & 28.4\std{$\pm$1.0} & 28.1\std{$\pm$1.0} && 25.0\std{$\pm$0.4} & 23.2\std{$\pm$0.4} & 20.7\std{$\pm$0.3}  && 22.7\std{$\pm$0.3} & 18.4\std{$\pm$0.2} & 17.0\std{$\pm$0.2} \\
ASER~\cite{ASER}    & 64.8\std{$\pm$1.0} & 62.6\std{$\pm$1.1} & 53.2\std{$\pm$1.5} && 52.8\std{$\pm$0.8} & 50.4\std{$\pm$0.9} & 46.8\std{$\pm$0.7} && 78.9\std{$\pm$0.5} & 75.4\std{$\pm$0.7} & 68.2\std{$\pm$1.1} \\ 
SCR~\cite{SCR}     & 43.2\std{$\pm$1.5} & 35.5\std{$\pm$1.8} & 24.1\std{$\pm$1.0} && 29.3\std{$\pm$0.9} & 20.4\std{$\pm$0.6} & 11.5\std{$\pm$0.6} && 44.8\std{$\pm$0.6} & 26.8\std{$\pm$0.5} & 20.1\std{$\pm$0.4} \\ 
CoPE~\cite{online_pro_ema}  & 49.7\std{$\pm$1.6} & 45.7\std{$\pm$1.5} & 39.4\std{$\pm$1.8} && 25.6\std{$\pm$0.9}  & 17.8\std{$\pm$1.3}  & 14.4\std{$\pm$0.8} && 11.9\std{$\pm$0.6}   & 10.9\std{$\pm$0.4} & 9.7\std{$\pm$0.4} \\
DVC~\cite{DVC} & 40.2\std{$\pm$2.6} & 31.4\std{$\pm$4.1} & 21.2\std{$\pm$2.8} && 32.0\std{$\pm$0.9} & 32.7\std{$\pm$2.0} & 28.0\std{$\pm$2.2} && 59.8\std{$\pm$2.2} & 52.9\std{$\pm$1.3} & 45.1\std{$\pm$1.9} \\
OCM~\cite{OCM} & 35.5\std{$\pm$2.4} & 23.9\std{$\pm$1.4} & 13.5\std{$\pm$1.5} && 18.3\std{$\pm$0.9} & 15.2\std{$\pm$1.0} & 10.8\std{$\pm$0.6} && 23.6\std{$\pm$0.5} & 26.2\std{$\pm$0.5}  & 23.8\std{$\pm$1.0} \\ 
\hline
{\frameworkName} (\textbf{ours})   & 23.2\std{$\pm$1.3} & 17.6\std{$\pm$1.4} & 12.5\std{$\pm$0.7} && 
15.0\std{$\pm$0.8} & 10.4\std{$\pm$0.5} & 6.1\std{$\pm$0.6} && 21.3\std{$\pm$0.5} & 17.4\std{$\pm$0.4} & 16.8\std{$\pm$0.4} \\
\shline
\end{tabular}
}
\end{center}
\caption{Average Forgetting~(lower is better) on three benckmark datasets. All results are the average and standard deviation of 15 runs.}
\label{tab:forget}
\end{table*}

\begin{figure*}[htp]
  \centering
  \subfloat[Average incremental performance]{
    \includegraphics[width=0.55\linewidth]{Figures/imgs/incremental_step_acc.pdf}
    \label{fig:incrementalAcc}
  }
  \subfloat[Confusion matrix of OCM and \frameworkName]{
    \includegraphics[width=0.42\linewidth]{Figures/imgs/confusion_matrix.pdf}
    \label{fig:confusionMatrix}
  }
  \caption{Incremental accuracy on tasks observed so far and confusion matrix of accuracy (\%) in the {test set} of CIFAR-10.}
  \label{fig:incrementalAcc_confusionMatrix}
\end{figure*}



\paragraph{Class confusion in online CL.}
Fig.~\ref{fig:tsne_motivation} provides the $t$-SNE~\cite{tsne} visualization results for ER and \frameworkName on the test set of CIFAR-10 ($M=0.2k$). 
We can draw intuitive observations as follows. 
(1) There is serious class confusion in ER.
When the new task (task 2) arrives, features learned in task 1 are not discriminative for task 2, leading to class confusion and decreased performance in old classes.
(2) Shortcut learning may cause class confusion. For example, the performance of ER decreases more on airplanes compared to automobiles, probably because birds in the new task have more similar backgrounds to airplanes, as shown in Fig.~\ref{fig:heatmap}.
(3) \frameworkName achieves better discrimination both on task 1 and task 2. The results demonstrate that \frameworkName can maintain discrimination of all seen classes and significantly mitigate forgetting by 
combining the proposed \methodname and \dataaugname.






\subsection{Results and Analysis}
\label{result}
\paragraph{Performance of average accuracy.}
Table~\ref{tab:acc} presents the results of average accuracy with different memory bank sizes ($M$) on three benchmark datasets. Our \frameworkName consistently outperforms all baselines on three datasets.
Remarkably, the performance improvement of \frameworkName is more significant when the memory bank size is relatively small; this is critical for online CL with limited resources. For example, compared to the second-best method OCM, \frameworkName achieves about 10$\%$ and 6$\%$ improvement on CIFAR-10 when $M$ is 100 and 200, respectively. 
The results show that our \frameworkName can learn more representative and discriminative features with a limited memory bank.
Compared to baselines that use knowledge distillation (iCaRL, DER++, PASS, OCM), our \frameworkName achieves better performance by leveraging the feedback of online prototypes.  
Besides, \frameworkName significantly outperforms PASS and CoPE that also use prototypes, showing that online prototypes are more suitable for online CL. 


We find that the performance improvement tends to be gentle when $M$ increases.
The reason is that as $M$ increases, the samples in the memory bank become more diverse, and the model can extract sufficient information from massive samples to distinguish seen classes. 
In addition, many baselines perform poorly on CIFAR-100 and TinyImageNet due to a dramatic increase in the number of tasks. In contrast, \frameworkName still performs well and improves accuracy over the second best.



\paragraph{Performance of average forgetting.}
We report the Average Forgetting results of our \frameworkName and all baselines on three benchmark datasets in Table~\ref{tab:forget}. The results confirm that \frameworkName can effectively mitigate catastrophic forgetting. 
For CIFAR-10 and CIFAR-100, \frameworkName achieves the lowest average forgetting compared to all replay-based baselines. 
For TinyImageNet, our result is a little higher than iCaRL and CoPE but better than the latest methods DVC and OCM. 
The reason is that iCaRL uses a nearest class mean classifier, but we use softmax and FC layer during the test phase, and CoPE slowly updates prototypes with a high momentum.
However, as shown in Table~\ref{tab:acc}, \frameworkName provides more accurate classification results than iCaRL and CoPE. 
It is a fact that when the maximum accuracy of a task is small, the forgetting on this task is naturally rare, even if the model completely forgets what it learned.





\paragraph{Performance of each incremental step.}
We evaluate the average incremental performance~\cite{DER++, DVC} on CIFAR-10 ($M=0.1k$) and CIFAR-100 ($M=0.5k$), which indicates the accuracy over all seen tasks at each incremental step. 
Fig.~\ref{fig:incrementalAcc} shows that \frameworkName achieves better accuracy and effectively mitigates forgetting while the performance of most baselines degrades rapidly with the arrival of new classes.

\paragraph{Confusion matrices at the end of learning.}
We report the confusion matrices of our \frameworkName and the second-best method OCM, as shown in Fig.~\ref{fig:confusionMatrix}. 
After learning the last task (\ie, the last two classes), OCM forgets the knowledge of early tasks (classes 0 to 3). 
In contrast, \frameworkName performs relatively well in all classes, especially in the first task (classes 0 and 1), outperforming OCM by 27.8\% average improvements.
The results show that learning representative and discriminative features is crucial to mitigate catastrophic forgetting; see Appendix~\ref{appendix:extra_exp} for extra experimental results.  




\subsection{Ablation Studies}
\label{ablation}

\begin{table}[t]
\small
\begin{center}
\begin{tabular}{ccccc}
\shline
\multirow{2}{*}{{Method}} & {CIFAR-10}&{CIFAR-100} \\
& Acc $\uparrow$(Forget $\downarrow$) & Acc $\uparrow$(Forget $\downarrow$) \\ 
\midrule
baseline & 46.4\std{$\pm$1.2}(36.0\std{$\pm$}2.1) & 18.8\std{$\pm$0.8}(18.5\std{$\pm$}0.7) \\
w/o \methodname & 53.1\std{$\pm$1.4}(24.7\std{$\pm$2.0}) & 19.3\std{$\pm$0.7}(15.9\std{$\pm$0.9}) \\
w/o \dataaugname & 52.0\std{$\pm$1.5}(34.6\std{$\pm$2.4}) & 21.5\std{$\pm$0.5}(16.3\std{$\pm$0.8}) \\ 
\hline
w/o $\mathcal{L}^{\mathrm{new}}_{\mathrm{pro}}$ & 54.8\std{$\pm$1.2}(\textbf{22.1}\std{$\pm$3.0}) & 19.6\std{$\pm$0.8}(19.9\std{$\pm$0.7}) \\
w/o $\mathcal{L}^{\mathrm{seen}}_{\mathrm{pro}}$ & 55.7\std{$\pm$1.4}(25.5\std{$\pm$1.5}) & 20.1\std{$\pm$0.4}(16.2\std{$\pm$0.6}) \\ 
$\mathcal{L}^{\mathrm{seen}}_{\mathrm{pro}}$ w/o $\mathcal{C}^\mathrm{new}$ & 56.2\std{$\pm$1.2}(26.4\std{$\pm$2.3}) & 20.8\std{$\pm$0.6}(17.9\std{$\pm$0.7}) \\ 
\hline
{\frameworkName} (\textbf{ours}) & \textbf{57.8}\std{$\pm$1.1}(23.2\std{$\pm$1.3}) & \textbf{22.7}\std{$\pm$0.7}(\textbf{15.0}\std{$\pm$0.8}) \\ 
\shline 
\end{tabular}
\end{center}
\caption{Ablation studies on CIFAR-10 ($M=0.1k$) and CIFAR-100 ($M=0.5k$). 
``baseline'' means $\mathcal{L}_\mathrm{INS}+\mathcal{L}_\mathrm{CE}$.
``$\mathcal{L}^{\mathrm{seen}}_{\mathrm{pro}}$ w/o $\mathcal{C}^\mathrm{new}$'' means $\mathcal{L}^{\mathrm{seen}}_{\mathrm{pro}}$ do not consider new classes in current task.
}
\label{tab:ablation}
\end{table}

\paragraph{Effects of each component.} Table~\ref{tab:ablation} presents the ablation results of each component. Obviously, \methodname and \dataaugname can consistently improve the average accuracy of classification. 
We can observe that the effect of \methodname is more significant on more tasks while \dataaugname plays a crucial role when the memory bank size is limited. Moreover, when combining \methodname and \dataaugname, the performance is further improved, which indicates that both can benefit from each other. For example, \dataaugname boosts \methodname by about 6$\%$ improvements on CIFAR-10 ($M=0.1k$), and the performance of \dataaugname is improved by about 3$\%$ on CIFAR-100 ($M=0.5k$) by combining \methodname.


\paragraph{Equilibrium in \methodname.}
When learning new classes, the data of new classes is involved in both $\mathcal{L}^{\mathrm{new}}_{\mathrm{pro}}$ and $\mathcal{L}^{\mathrm{seen}}_{\mathrm{pro}}$ of \methodname, where $\mathcal{L}^{\mathrm{new}}_{\mathrm{pro}}$ only focuses on learning new knowledge while $\mathcal{L}^{\mathrm{seen}}_{\mathrm{pro}}$ tends to alleviate forgetting on seen classes.
To explore the best way of learning new classes, we consider three scenarios for \methodname in Table~\ref{tab:ablation}.
The results show that only learning new knowledge (w/o $\mathcal{L}^{\mathrm{seen}}_{\mathrm{pro}}$) or only consolidating the previous knowledge (w/o $\mathcal{L}^{\mathrm{new}}_{\mathrm{pro}}$) can significantly degrade the performance, which indicates that both are indispensable for online CL.
Furthermore, when $\mathcal{L}^{\mathrm{seen}}_{\mathrm{pro}}$ only considers old classes and ignores new classes ($\mathcal{L}^{\mathrm{seen}}_{\mathrm{pro}}$ w/o $\mathcal{C}^\mathrm{new}$), the performance also decreases. These results show that the equilibrium of all seen classes (\methodname) can achieve the best performance and is crucial for online CL.


\paragraph{Effects of \dataaugname.} 
To verify the advantage of \dataaugname, we compare it with the completely random mixup
in Table~\ref{tab:ablation_mixup}.
\begin{table}
\small
\begin{center}
\begin{tabular}{c|rrr}
\shline
\multicolumn{1}{c|}{Method}       & ${M=0.1k}$   & ${M=0.2k}$   & ${M=0.5k}$     \\ \hline
Random & 53.5\std{$\pm$2.7} & 62.9\std{$\pm$2.5} & 70.8\std{$\pm$2.2} \\
\dataaugname (\textbf{ours})  & \textbf{57.8}\std{$\pm$1.1} & \textbf{65.5}\std{$\pm$1.0} & \textbf{72.6}\std{$\pm$0.8} \\ 
\shline
\end{tabular}
\end{center}
\caption{Comparison of Random Mixup and \dataaugname on CIFAR-10. 
}
\label{tab:ablation_mixup}
\end{table}
\dataaugname outperforms random mixup in all three scenarios. Notably, \dataaugname works significantly when the memory bank size is small, which shows that the feedback can prevent class confusion due to a restricted memory bank; see Appendix~\ref{appendix:ablations} for extra ablation studies.



\subsection{Validation of Online Prototypes}
\label{prove_onlinePrototypes}
\begin{figure}
    \centering
    \includegraphics[width=1.0\linewidth]{Figures/imgs/cosine_similarity.pdf}
    \caption{The cosine similarity between online prototypes and prototypes of the entire memory bank.}
    \label{fig:cosine_similarity}
\end{figure}
Fig.~\ref{fig:cosine_similarity} shows the cosine similarity between online prototypes and global prototypes (prototypes of the entire memory bank) at each time step.
For the first mini-batch of each task, online prototypes are equal to global prototypes (similarity is 1, omitted in Fig.~\ref{fig:cosine_similarity}).
In the first task, online and global prototypes are updated synchronously with the model updates, resulting in high similarity. 
In subsequent tasks, the model initially learns inadequate features of new classes, causing online prototypes to be inconsistent with global prototypes and low similarity, which shows that accumulating early features as prototypes may be harmful to new tasks. However, the similarity will improve as the model learns, because the model gradually learns representative features of new classes.
Furthermore, the similarity on old classes is only slightly lower, showing that online prototypes are resistant to forgetting. 


\section{Experiment Results}

\section{Methodology}
\label{sec:benchmark}

\subsection{Description of hardware and software}
\label{ssec:supercomp}
The benchmarks reported in this paper were performed on the Intel Xeon Phi systems provided by the Joint Laboratory for System Evaluation (JLSE) and the Theta supercomputer at the Argonne Leadership Computing Facility (ALCF) \cite{alcf}, which is a part of the U.S. Department of Energy (DOE) Office of Science (SC) Innovative and Novel Computational Impact on Theory and Experiment (INCITE) program \cite{incite}. Theta is a 10-petaflop Cray XC40 supercomputer consisting of 3,624 Intel Xeon Phi 7230 processors. Hardware details for the JLSE and Theta system are shown in \Cref{tab:hw}.

The Intel Xeon Phi processor used in this paper has 64 cores each equipped with L1 cache. Each core also has two Vector Processing Units, both of which need to be used to get peak performance. This is possible because the core can execute two instructions per cycle. In practical terms, this can be achieved by using two threads per core. Pairs of cores constitute a tile. Each tile has an L2 cache symmetrically shared by the core pair. The L2 caches between tiles are connected by a two dimensional mesh. The cores themselves operate at 1.3 GHz. Beyond the L1 and L2 cache structure, all the cores in the Intel Xeon Phi processor share 16 GBytes of MCDRAM (also known as High Bandwidth Memory) and 192 GBytes of DDR4. The bandwidth of MCDRAM is approximately 400 GBytes/sec while the bandwidth of DDR4 is approximately 100 GBytes/sec. 

\begin{table}
  \caption{Hardware and software specifications}
  \label{tab:hw}

  \begin{tabularx}{\columnwidth}{XX}
  \toprule
			\multicolumn{2}{c}{\textbf{\intelphi\ node characteristics}} \\
    \midrule 
    \intelphi\ models				&	7210 and 7230 (64~cores, 1.3~GHz, 
    									2,622 GFLOPs) \\
    Memory per node					&	16 GB MCDRAM, \newline 192 GB DDR4 RAM \\
    Compiler						&	Intel Parallel Studio XE 2016v3 \\
    \midrule
    		\multicolumn{2}{c}{\textbf{JLSE \iphi\ cluster (26.2 TFLOPS peak)}} \\
    \midrule
    \# of \intelphi\ nodes	&	10 \\
    Interconnect type				&	Intel Omni-Path\textsuperscript{TM} \\
    \midrule
    		\multicolumn{2}{c}{\textbf{Theta supercomputer (9.65~PFLOPS peak)}} \\
    \midrule
    \# of \intelphi\ nodes				&	3,624 \\
    Interconnect type				&	Aries interconnect with \newline Dragonfly topology \\
  \bottomrule
\end{tabularx}

\end{table}

\begin{table}
\begin{threeparttable}
  \caption{Chemical systems used in benchmarks and their size characteristics}
  \label{tab:chem}

  \begin{tabularx}{\columnwidth}{XYYYYY}

  \toprule

  \multirow{2}{*}{Name}	&	\multirow{2}{*}{\# atoms}	&	\multirow{2}{*}{\# BFs\tnote{a}}	&	\multicolumn{3}{c}{Memory footprint\tnote{b}, GB} \\
  \cmidrule(l){4-6}
  		& & &	{MPI\tnote{c}}	&	{Pr.F.\tnote{d}}	&	{Sh.F.\tnote{e}} \\
  \midrule
  	0.5~nm	&	44			&	660			&	7		&	0.13		&	0.03	\\
	1.0~nm	&	120			&	1800		&	48		&	1			&	0.2	\\
	1.5~nm	&	220			&	3300		&	160		&	3			&	0.8	\\
	2.0~nm	&	356			&	5340		&	417		&	8			&	2	\\
	5.0~nm	&	2016		&	30240		&	9869	&	257			&	52	\\
	\bottomrule
  \end{tabularx}

  \begin{tablenotes}
  	\item [a] BF -- basis function
    \item [b] Estimated using \crefrange{eqn:mem:mpi}{eqn:mem:shr}
 	\item [c] MPI-only SCF code
    \item [d] Private Fock SCF code
    \item [e] Shared Fock SCF code
  \end{tablenotes}
\end{threeparttable}
\end{table}

These two levels of memory can be configured in three different ways (or modes). The modes are referred to as Flat mode, Cache mode, and Hybrid mode. Flat mode treats the two levels of memory as separate entities. The Cache mode treats the MCDRAM as a direct mapped L3 cache to the DDR4 layer. Hybrid mode allows the user to use a fraction of MCDRM as L3 cache allocate the rest of the MCDRAM as part of the DDR4 memory.
In Flat mode, one may choose to run entirely in MCDRAM or entirely in DDR4. The "numactl" utility provides an easy mechanism to select which memory is used. It is also possible to choose the kind of memory used via the "memkind" API, though as expected this requires changes to the source code.

Beyond memory modes, the Intel Xeon Phi processor supports five cluster modes. The motivation for these modes can be understood in the following manner: to maintain cache coherency the Intel Xeon Phi processor employs a distributed tag directory (DTD). This is organized as a set of per-tile tag directories (TDs), which identify the state and the location on the chip of any cache line. For any memory address, the hardware can identify the TD responsible for that address. The most extreme case of a cache miss requires retrieving data from main memory (via a memory controller). It is therefore of interest to have the TD as close as possible to the memory controller. This leads to a concept of locality of the TD and the memory controllers.
It is in the developer's interest to maintain the locality of these messages to achieve the lowest latency and greatest bandwidth of communication with caches. Intel Xeon Phi supports all-to-all, quadrant/hemisphere and sub-NUMA cluster SNC-4/SNC-2 modes of cache operation.

For large problem sizes, different memory and clustering modes were observed to have little impact on the time to solution for the three versions of the GAMESS code. For this reason, we simply chose the mode most easily available to us. In other words, since the choice of mode made little difference in performance, our choice of Quad-Cache mode was ultimately driven by convenience (this being the default choice in our particular environment). Our comments here apply to large problem sizes, so for small problem sizes, the user will have to experiment to find the most suitable mode(s).


\subsection{Description of chemical systems}
\label{ssec:chemical}
For benchmarks, a system consisting of parallel series of graphene sheets was chosen. This system is of interest to researchers in the area of (micro)lubricants \cite{kawai2016superlubricity}. A physical depiction of the configuration is provided in \Cref{fig:graphene}. The graphene-sheet system is ideal for benchmarking, because the size of the system is easily manipulated. Various Fock matrix sizes can be targeted by adjusting the system size.

\begin{figure}
	\includegraphics[width=\columnwidth]{Figure2}
	\caption{Model system of a C$_{2016}$ graphene bilayer. In the text, we refer to this system as 5~nm.
    		 There are two layers with size 5~nm by 5~nm.
             Each graphene layer consists of 1,008 carbon atoms.}
    \label{fig:graphene}
\end{figure}

\subsection{Characteristics of datasets}
\label{ssec:datasets}
In all, five configurations of the graphene sheets system were studied. The datasets for the systems studied are labeled as follows: 0.5~nm, 1.0~nm, 1.5~nm, 2.0~nm, and 5.0~nm.  \Cref{tab:chem} lists size characteristics of these configurations. The same 6-31G(d) basis set (per atom) was used in all calculations. For N basis functions, the density, Fock, AO overlap, one-electron Fock matrices and the matrix of MO coefficients are N$\times$N in size. These are the main data structures of significant size. Therefore, the benchmarks performed in this work process matrices which range from 660$\times$660 to 30,240$\times$30,240. For each of the systems studied, \Cref{tab:chem} lists the memory requirements of the three versions of GAMESS HF code.
Denoting $N_{BF}$ as the number of basis functions, the following equations describe the asymptotic $(N_{BF}\to\infty)$ memory footprint for the studied HF algorithms:
\begin{subequations}
	\label{eqn:mem}
	\begin{align}
		M_{MPI} =& 5/2 \cdot N_{BF}^2 \cdot N_{MPI\_per\_node}, 				\label{eqn:mem:mpi} \\
		M_{PrF} =& (2+N_{threads}) \cdot N_{BF}^2 \cdot N_{MPI\_per\_node}, 	\label{eqn:mem:prv} \\
		M_{ShF} =& 7/2 \cdot N_{BF}^2 \cdot N_{MPI\_per\_node},					\label{eqn:mem:shr}
	\end{align}
\end{subequations}
where $M_{MPI}$, $M_{PrF}$, $M_{ShF}$ denote the memory footprint of MPI-only, private Fock, and shared Fock algorithms respectively; $N_{threads}$ denotes the number of threads per MPI process for the OpenMP code, and $N_{MPI\_per\_node}$ denotes the number of MPI processes per KNL node. For OpenMP runs $N_{MPI\_per\_node}=4$, while for MPI runs the number of MPI ranks was varied from 64 to 256.

If one compares columns MPI versus Pr.F and Sh.F. in \Cref{tab:chem}, you will see that the private Fock code has about a 50 times less footprint compared to the stock MPI code. For the shared Fock code, the difference is even more dramatic with a savings of about 200 times. The ideal difference is 256 times since we compare 256 MPI ranks in the stock MPI code where all data structures are replicated versus 1 MPI rank with 256 threads for the hybrid MPI/OpenMP codes. But we introduced additional replicated structures (see \Cref{fig:buffer}) and many relatively small data structures are replicated also in the MPI/OpenMP codes. This explains the difference between the ideal and observed footprints.

Each of the aforementioned datasets was used to benchmark three versions of the GAMESS code. The first version is the stock GAMESS MPI-only release that is freely available on the GAMESS website~\cite{gamesswebsite}. The second version is a hybrid MPI/OpenMP code, derived from the stock release. This version has a shared density matrix, but a thread-private Fock matrix. The third version of the code is in turn derived from the second version; it has shared density and Fock matrices. A key objective was to see how these incremental changes allow one to manage (i.e., reduce) the memory footprint of the original code while simultaneously driving higher performance.

\section{Results}
\label{sec:results}

\subsection{Single node performance}
\label{ssec:singlenode}
The second generation Intel Xeon Phi processor supports four hardware threads per physical core. Generally, more threads per core can help hide latencies inherent in an application. For example, when one thread is waiting for memory, another can use the processor. The out-of-order execution engine is beneficial in this regard as well. To manipulate the placement of processes and threads, the \verb|I_MPI_DOMAIN| and \verb|KMP_AFFINITY| environment variables were used. 
We examined the performance picture when one thread per core is utilized and when four threads per core are utilized. As expected, the benefit is highest for all versions of GAMESS for two threads (or processes) per core. For three and four threads per core, some gain is observed, albeit at a diminished level. \Cref{fig:afty} shows the scaling curves with respect to the number of hardware threads utilized observed by us.

\begin{figure}
	\includegraphics[width=\columnwidth]{Figure3}
	\caption{Performance dependence on OpenMP thread affinity type for the shared Fock version of the GAMESS code
    		 on a single \intelphireg\ processor using the 1.0 nm benchmark.
             All calculations are performed in quad-cache mode.
             Four MPI ranks were used in all cases.
             The number of threads per MPI rank was varied from 1 to 64.}
    \label{fig:afty}
\end{figure}

\begin{figure}
	\includegraphics[width=\columnwidth]{Figure4}
	\caption{Scalability with respect to the number of hardware threads of the original MPI code
    		and two OpenMP versions on a single \intelphireg\ processor using the 1.0~nm benchmark.}
    \label{fig:singlescaling}
\end{figure}

As a first test, single-node scalability was examined with respect to hardware threads of all three versions of GAMESS. For the MPI-only version of GAMESS, the number of ranks was varied from~4 to~256. For the hybrid versions of GAMESS, the number of ranks times the number of threads per rank is the number of hardware threads targeted. The larger memory requirements of the original MPI-only code restrict the computations to, at most, 128 hardware threads. In contrast, the two hybrid versions can easily utilize all 256 hardware threads available. Finally, in general terms, on cache based memory architectures, it is expected that larger memory footprints potentially lead to more cache capacity and cache line conflict effects. These effects can lead to diminished performance, and this is yet another motivation to look at a hybrid MPI+X approach.

The results of our single-node tests are plotted in \Cref{fig:singlescaling}. It is found that using the private Fock version leads to the best time to solution for the 1.0~nm dataset, for any number of hardware threads. This version of the code is much more memory-efficient than the stock version but, because the Fock matrix data structure is private, it has a much larger memory footprint than the shared Fock version of GAMESS. Nevertheless, because the Fock matrix is private, there is less thread contention than the shared Fock version.

It was mentioned in \Cref{ssec:omp} that shared Fock algorithm introduces additional overhead for thread synchronization. For small numbers of Intel Xeon Phi threads, this overhead is expected to be low. Therefore the shared Fock version is expected to be on par with the other versions. Eventually, as the overhead of the synchronization mechanisms begins to increase, the private Fock version of the code is found to dominate. In the end, the private Fock version outperforms stock GAMESS because of the reduced memory footprint, and outperforms the shared Fock version because of a lower synchronization overhead.
Therefore, on a single node, the private Fock version gives the best time-to-solution of the three codes, but the shared Fock version strikes a (better) balance between memory utilization and performance.

\begin{figure}
	\includegraphics[width=\columnwidth]{Figure5}
	\caption{Time to solution (x axis, time in seconds) for different clustering and memory modes.
    		 Left column displays the small chemical system -- 0.5~nm bilayer graphene and
             right column displays one of the largest molecules bilayer graphene -- 2.0~nm.}
    \label{fig:tts}
\end{figure}

Beyond this, one must consider the choice of memory mode and cluster mode of the Intel Xeon Phi processor. It should be noted that, depending on the compute and memory access patterns of a code, the choice of memory and cluster mode can be a potentially significant performance variable. The performance impact of different memory and cluster modes is examined for the 0.5~nm (small) and~2.0~nm (large) datasets. The results are shown in \Cref{fig:tts}. For both datasets, some variation in performance is apparent when different cluster modes and memory modes are used. The smaller dataset indicates more sensitivity to these variables than the larger dataset. Also, for both data sizes the private Fock version performs best in all cluster and memory modes tested. Also, except in the All-to-All cluster mode, the shared Fock version significantly outperforms the MPI-only stock version. In the All-to-All mode, the MPI-only version actually outperforms the shared Fock version for small datasets, and the two versions are close to parity for large datasets. In total, it is concluded that the quadrant-cache cluster-memory mode is best suited to the design of the GAMESS hybrid codes.

\subsection{Multi-node performance}
It is very important to note that the total number of MPI ranks for GAMESS is actually twice the number of compute ranks because of the DDI. The DDI layer was originally implemented to support one-sided communication using MPI-1. For GAMESS developers, the benefit of DDI is convenience in programming. The downside is that each MPI compute process is complemented by an MPI data server~(DDI) process, which clearly results in increased memory requirements. Because data structures are replicated on a rank-by-rank basis, the impact of DDI on memory requirements is particularly unfavorable to the original version of the GAMESS code. To alleviate some of the limitations of the original implementation, an implementation of DDI based on MPI-3 was developed \cite{pruitt2016private}. Indeed, by leveraging the ``native'' support of one-sided communication in MPI-3, the need for a DDI process alongside each MPI rank was eliminated. For all three versions of the code benchmarked here, no DDI processes were needed.

\begin{figure}
	\includegraphics[width=\columnwidth]{Figure6}
	\caption{Multi-node scalability of the Private Fock and the Shared Fock hybrid MPI-OpenMP
    		 and the MPI-only stock GAMESS codes on the Theta machine with the 2.0~nm dataset.
             The quad-cache cluster-memory mode was used for all data points.}
    \label{fig:2nm}
\end{figure}

\Cref{fig:2nm} shows the multi-node scalability of the MPI-only version of GAMESS versus the private Fock and the shared Fock hybrid versions. It is important to appreciate at the outset that the multi-node scalability of the original MPI-only version of GAMESS is already reasonable. For example, the code scales linearly to 256 Xeon Phi nodes, and it is really the memory footprint bottleneck that limits how well all the Xeon Phi cores on any given node can be used. This pressure is reduced in the private Fock version of the code, and it is essentially eliminated in the shared Fock version. Overall, for the 2~nm dataset, the shared Fock code runs about six times faster than stock GAMESS on 512 Xeon Phi processors. It resulted from the better load balance of the shared Fock algorithm that uses all four shell indices -- two are used in MPI and two are used in OpenMP workload distribution. The actual timings and efficiencies are listed in \Cref{tab:efficiency}.

\begin{table}
\begin{threeparttable}
  \caption{Parallel efficiency of the three different HF algorithms using 2.0~nm dataset}
  \label{tab:efficiency}
  \begin{tabularx}{\columnwidth}{XYYYYYY}

  \toprule
    	\multirow{2}{*}{\# Nodes}		&	\multicolumn{3}{c}{Time-to-solution, s} &
                            \multicolumn{3}{c}{Parallel efficiency, \%} \\
        \cmidrule(rl{0.75em}){2-4} \cmidrule(l){5-7}
  					&	{MPI\tnote{a}} &	{Pr.F.\tnote{a}} &	{Sh.F.\tnote{a}} &
                        {MPI\tnote{a}} &	{Pr.F.\tnote{a}} &	{Sh.F.\tnote{a}} \\

  	\midrule
		4	&	2661	&	1128	&	1318	&	100	&	100	&	100 \\
		16	&	685		&	288		&	332		&	97	&	98	&	99 \\
		64	&	195		&	78		&	85		&	85	&	90	&	97 \\
		128	&	118		&	49		&	43		&	70	&	72	&	96 \\
		256	&	85		&	44		&	23		&	49	&	40	&	90 \\
		512	&	82		&	44		&	13		&	25	&	20	&	79 \\
    \bottomrule
   \end{tabularx}

 	\begin{tablenotes}
 		\item [a] MPI-only SCF code
    	\item [b] Private Fock SCF code
    	\item [c] Shared Fock SCF code
 	\end{tablenotes}
\end{threeparttable}
\end{table}

\begin{figure}
	\includegraphics[width=\columnwidth]{Figure7}
	\caption{Scalability of the Shared Fock hybrid MPI-OpenMP version of GAMESS on the Theta machine
    		 for the 5.0~nm (i.e. large) dataset in quadrant cache mode on 3,000 \intelphireg\ processors.
             The results here are for 4~MPI ranks per node with 64~threads per rank,
             giving full saturation (in terms of hardware threads) on every \intelphireg\ node. For each point in the figure, we show the time in seconds.}
    \label{fig:5nm}
\end{figure}

\Cref{fig:5nm} shows the behavior of the shared Fock version of GAMESS for the 5~nm dataset. It is the largest dataset we could fit in memory on Theta. Since we run on 4~MPI ranks the memory footprint is approximately 208~GB per node. This figure shows good scaling of the code up to 3,000 Xeon Phi nodes, which is equal to 192,000 cores (64~cores per node).

\section{Analysis}

\label{sec:analysis}

% \begin{table*}[t]
    \centering
    \small
    \addtolength{\tabcolsep}{-2pt}
    \setlength{\tabcolsep}{3.5pt}{
    \begin{tabular}{ lcccccccc }
    \toprule             & TNEWS & IFLY  & BQ    & WSC   & AFQMC & CSL   & OCNLI & AVG \\
    \hline
    SubChar-Pinyin       & 63.68 & 58.81 & 81.74 & 65.90 & 68.89 & 82.87 & 67.98 & 70.16 \\
    SubChar-Zhuyin       & 64.91 & 59.39 & 81.41 & 62.72 & 69.14 & 82.60 & 69.12 & 69.90 \\
    SubChar-Stroke       & 64.26 & 55.44 & 81.52 & 62.06 & 69.88 & 83.16 & 68.98 & 69.33 \\
    SubChar-Wubi         & 63.81 & 58.74 & 81.55 & 64.61 & 69.66 & 82.44 & 68.02 & 69.90 \\
    SubChar-Zhengma      & 63.86 & 59.51 & 81.59 & 63.27 & 70.47 & 82.91 & 69.03 & 70.09 \\
    SubChar-Cangjie      & 64.10 & 57.77 & 81.98 & 62.39 & 68.95 & 82.60 & 68.46 & 69.46 \\
    SubChar-Byte         & 63.58 & 59.55 & 81.65 & 63.60 & 68.60 & 82.66 & 67.93 & 69.65 \\
    SubChar-RandomIndex  & 64.11 & 59.16 & 81.64 & 63.93 & 68.53 & 82.86 & 69.39 & 69.95 \\
    \midrule
    SubChar-Pinyin (BPE) & 63.86 & 58.84 & 82.12 & 65.57 & 69.86 & 82.86 & 68.57 & 70.24 \\
    \bottomrule
    \end{tabular}}
    \caption{Results of SubChar tokenizers when using different encoding methods. The last row is a model with SubChar-Pinyin tokenizer using BPE as the subword tokenization algorithm, all previous rows are using unigram LM as the subword tokenization implementation. All models have 6-layers with the same hyper-parameters. The impact of different encoding methods on downstream performance is small, and the ULM and BPE versions of SubChar-Pinyin also achieve similar results.}
    \label{tab:encode_ablation}
    \end{table*}



\begin{figure}[t]
\centering
\includegraphics[width=0.47\textwidth]{figures/WubiBERT-Fig2.png}
\caption{Breakdown of different types of tokens in the vocabularies of various tokenizers. We observe the clear trend that in our SubChar tokenizers, a small fraction of sub-character tokens saves the space to store much more character combination tokens (e.g., words and phrases).}
\label{fig:composition}
\end{figure}


In this section, we conduct various analyses to better understand the working mechanisms of SubChar tokenization, including illustrations of the efficiency improvement and ablations on different components of our tokenization pipeline. 

\subsection{Vocabulary Composition}

We break down the vocabulary of each tokenizer into three different categories: sub-character tokens, character tokens, and character combination tokens (words and phrases). 
%
As shown in Figure~\ref{fig:composition}, character tokenizers only have character tokens, while sub-word tokenizers have a small percentage of combination tokens. The main reason for the relatively small number of combination tokens in sub-word tokenizers is that unlike how English words are composed with $26$ alphabet letters, there are thousands of unique Chinese characters, which take up a large proportion of the vocabulary in order to maintain coverage.

In contrast, SubChar tokenizers use a very small fraction of sub-character tokens to compose many complex Chinese characters, therefore saving up a large percentage of the vocabulary to store combination tokens. This brings the advantage of having more words and phrases in the tokenized outputs, thus shortening the sequence lengths, as elaborated in the next section.

\subsection{Efficiency Improvement}

\begin{table}[t]
    \centering
    \small
    \addtolength{\tabcolsep}{-4pt}
    \setlength{\tabcolsep}{1mm}{
    \begin{tabular}{ lcc }
    \toprule 
     & iFLYTEK
     & TNEWS
     \\
\hline
CharTokenizer & 289.0 & 22.0 \\
Sub-word & 255.2 & 20.1 \\
% SubChar-Stroke & 228.1 & 18.3 \\
SubChar-Wubi & 183.2  & 15.8 \\
% SubChar-Zhengma & 208.1 & 18.0 \\
% SubChar-Cangjie & 183.0 & 15.8 \\
% SubChar-Zhuyin & 203.6 & 17.6 \\
SubChar-Pinyin & 185.2  &  16.1 \\
SubChar-Pinyin-NoIndex & \textbf{175.4} & \textbf{15.2} \\
% \midrule
% SubChar-Pinyin (BPE) & 184.4 & 15.9 \\
% SubChar-Wubi-CWS & 212.7 & 17.3 \\
% SubChar-Pinyin-CWS & 209.1 & 17.0 \\
% SubChar-Byte & 207.2 & 16.5 \\
% SubChar-RandomIndex & 216.4 & 17.0 \\
    \bottomrule
    \end{tabular}}
    \caption{Comparison of average length of tokenized sequences with different tokenizers. 
    % The last row uses BPE and others use unigram LM for vocabulary construction.
    SubChar tokenizers produce much shorter tokenized sequences than the baselines. SubChar-Pinyin-NoIndex tokenizer achieves the most length reduction. BPE and Unigram LM counterparts achieve similar speedup improvement. }
    \label{tab:output_lengths}
    \end{table}

\begin{table}[t]
    \centering
    \small
    \addtolength{\tabcolsep}{-3pt}
    \setlength{\tabcolsep}{1mm}{
    \begin{tabular}{ lcc }
    \toprule 
                        & TNEWS     & iFLYTEK \\
    \hline
    CharTokenizer       & 100.0\%     & 100.0\% \\
    Sub-word            & 99.9\%    & 92.6\% \\
    SubChar-Wubi        & 87.0\%    & 69.6\% \\
    SubChar-Pinyin      & 83.8\%    & 70.4\% \\
    SubChar-Pinyin-NoIndex & \textbf{82.7\%} & \textbf{68.9\%} \\

    \bottomrule
    \end{tabular}}
    \caption{Finetuning time of models with different tokenizers. Numbers indicate time relative to the CharTokenizer baseline model. Models with SubChar tokenizers take much shorter time to finish finetuning. SubChar-Pinyin-NoIndex brings the most speedup. }
    \label{tab:finetune_time}
\end{table}

The direct consequence of having more character combinations in the vocabulary is that the tokenized sequences are shorter. Table~\ref{tab:output_lengths} shows the average sequence length by using different tokenizers on two downstream datasets. 
We observe that SubChar tokenizers can tokenize the inputs into much shorter sequences.
% We observe that different encoding methods do make a difference in terms of output length reduction. Encoding methods like Wubi, Cangjie, and Pinyin are particularly efficient. 
% Moreover, we find that the SubChar-Pinyin-NoIndex variant achieves even more output length reduction than the SubChar-Pinyin counterpart (40\% shorter on iFLYTEK and 31\% shorter on TNEWS compared to the CharTokenizer baseline). 
% This shows that SubChar-Pinyin-NoIndex is not only more robust, but also more efficient.

% During finetuning, the short sequence lengths allow us to reduce the maximum sequence length of each batch, leading to faster training.
Moreover, our SubChar tokenizers can speed up training for both pretraining and finetuning.
During finetuning, we can pack multiple sequences into one input sequence to reduce the computation waste introduced by sequence padding~\cite{kosec2021packing}, and shorter sequence lengths allow the sequences to be packed more densely, thus increasing the overall throughput.
% ~\footnote{For simplicity, we perform packing using the First-Fit Bin Packing algorithm.}

 \begin{figure}[t]
\centering
\includegraphics[width=0.45\textwidth]{figures/ifly_time.png}
\caption{Training curves on the iFLYTEK dataset with two different models. The y-axis indicates classification loss (cross-entropy), the x-axis indicates time (seconds). Our SubChar-Pinyin-NoIndex model gets a lower loss than the CharTokenizer baseline throughout training.}
\label{fig:ifly_loss_time}
\end{figure}

The large number of weights in powerful and complex neural networks makes them difficult to be deployed on embedded systems 
with limited hardware resources/ mobile systems. 
Large networks do not fit in on-chip storage  and are stored in external DRAM: thus they 
need to be  fetched every time for inference of each image, word, or speech sample, leading to 
large consumption of energy. For instance,  Han et al. ~\cite{HanMD15}
states that the energy cost per  fetch ranges from 5pJ for 32bit coefficients in on-chip SRAM to 640pJ for 32b coefficients in off-chip
LPDDR2 DRAM:  thus running a 1 billion connection neural network,  at 20Hz would require  12.8W just for DRAM access - 
this prohibits the inferencing on a  typical mobile device.
A similar issue arises if the inferencing is done on cloud: the complex and large DNNs need to be loaded before the inferencing can be done, thus requiring more
memory and cost.

To address this issue, there has been significant work to reduce the size of networks.  
The objective in the ideal scenario is to get a model of smaller size, with limited loss of accuracy in the prediction, and no sacrifice in the inference time.
Model compression can be  effected  through a combination of pruning, weight sharing and encoding of the connection weights.
In the pruning step,  the network is pruned  by removing the redundant connections of the network.
Next, the weights are quantized so that multiple connections share the same weight, thus only the codebook (effective weights) and the indices need to be stored. 
The codebook is generally of small size, and hence the indices can be represented with fewer bits than that required for the original weights.
Finally, some encoding (like Huffman coding) is done to take advantage of the biased distribution of effective weights to further reduce the model size.


Neural network pruning was pioneered even in the early development of neural networks (~\cite{Reed93}).
Anwar et  al.~\cite{AnwarHS17} and Molchanov et  al ~\cite{MolchanovTKAK16} employ structured pruning  at the level of feature maps and kernels. The advantage of this 
scheme of pruning is that the resultant connection matrix can be considered dense. However this is more suited for 
convolution layers. Song Han et  al.~\cite{HanMD15} have considered the weight based pruning where they remove all connections whose weight is lower than a threshold. 
Their pruning strategy (along with quantisation and Huffman encoding) was able to get the model size of AlexNet reduced  from 240MB to 6.9MB, 
and that of VGG-16 from 552MB to 11.3MB, without loss of accuracy on Imagenet dataset.
A lot of work in literature is available for weight sharing and quantisation as well. 
Half-precision networks (Amodei et al., ~\cite{AmodeiABCCCCCCD15}) cut sizes of neural networks in half. XNOR-Net (Rastegari et al., ~\cite{RastegariORF16}), 
DoReFa-Net (Zhou et al., ~\cite{ZhouNZWWZ16}) and network binarization (Courbariaux et al.~\cite{CourbariauxB16}; Lin et al.~\cite{LinCMB15}) use aggressively quantized weights, activations and gradients to further reduce computation during training, however, the extreme compression rate comes with a loss of accuracy. Hubara et al.~\cite{HubaraCSEB16} and Li  Liu~\cite{LiL16} propose ternary weight networks to trade off between model size and accuracy.
Zhu et  al.~\cite{ZhuHMD16}  propose trained ternary quantization which uses two full-precision scaling coefficients for each layer, where these coefficients are trainable parameters.
Gong. et  al.~\cite{GongLYB14} consider vector quantization methods for compressing the parameters of CNNs.
HashedNets~\cite{ChenWTWC15} uses hash function to randomly group connection weights into hash buckets, so that all connections within the same hash bucket share a single parameter value. 


In this work,  we consider the compression strategy as suggested by  Han et  al.~\cite{HanMD15, HanPTD15}. Since 
all connections with weights below a threshold are removed from the network, the pruned network is a 
sparse structure that is stored using compressed sparse row (CSR) or compressed sparse column (CSC) format.
The model is further compressed by storing the  index difference instead of the absolute position, and encoding this difference in 
$k$ bits for each layer: for an index difference larger than the bound,  zero padding is employed. In our work, we fix $k$ to 4.
Finally Huffman encoding is employed on both the weight clusters and the index differences to ensure that most common cluster centres and  index differences
are represented with fewer bits. It is observed that Huffman encoding adds to 25\% additional
compression to the  model compressed without the encoding.  

The real challenge with compressed models lies in processing them for inferencing.
As stated before, with pruning the matrix becomes sparse and the indexes become relative. 
With weight sharing,  a short (8-bit) index for each weight is stored. More indirection is added with Huffman encoding.
This causes complexity and inefficiency to process the model on CPUs and GPUs.
The trivial method of exploding the matrix back to dense and doing the computation is not a good choice because of the 
excessive memory usage and the running time. 
Previous work has considered hardware and software accelerators to facilitate computation on compressed models.
Han et  al.~\cite{HanLMPPHD16} has proposed EIE, an efficient inference engine, that performs
customized sparse matrix vector multiplication and handles weight sharing with no loss of efficiency. 
However this requires  specialized hardware to be implemented to act as the accelerator. On the software side,
Intel Math Kernel Library (MKL~\cite{mkl}) provides  sparse solver for matrix-matrix and matrix-vector multiplications,
however this does not incorporate relative indexing and Huffman encoding, which are necessary for the compressed models.

 
This paper makes the following contributions: 
1) To the best of our knowledge, we are the first to develop fast kernels for sparse matrix-vector and matrix-matrix multiplications, where the 
sparse matrix is indexed using relative indices, and encoded with Huffman encoding
2) We have performed a comprehensive evaluation of different schemes,  which has shown that the best choice of the scheme depends on the
sparsity and the encoding scheme.
3) We have performed experiments on popular deep learning networks like AlexNet, VGG-16, to show that our method achieves x factor boost in 
the inference time, while maintaining the memory constraints.





 
 
















% packing啥的对于长序列也能做,感觉不用提
Table~\ref{tab:finetune_time} shows the model finetuning time relative to the CharTokenizer baseline. We observe significant speedup by SubChar tokenizers, finishing in as little as $68.9\%$ time on iFLYTEK with the SubChar-Pinyin-NoIndex tokenizer.
In Figure~\ref{fig:ifly_loss_time}, we plot the training curves for the CharTokenizer baseline and the SubChar-Pinyin-NoIndex model on the iFLYTEK dataset, we observe that our SubChar-Pinyin-Noindex model indeed converges much faster and achieves lower training loss in the end.  


The speedup on pretraining is also significant. While the running speed differs on different machines, the compression brought by the shorter tokenized outputs is hardware-invariant. In Table~\ref{tab:compression}, we show the relative size (disk memory) of the tokenized pretraining corpus. We observe that SubChar tokenizers can tokenize the raw pretraining texts into shorter sequences than the baselines, thus resulting in a much smaller pretraining data (\textit{e.g.}, as much as $25.3\%$ smaller than that of the CharTokenizer baseline with SubChar-Pinyin-NoIndex). In turn, this can translate to much faster pretraining on any training infrastructure.



% On the other hand, we observe that \cws{} and non-linguistic encoding methods like Byte and RandomIndex are not as efficient as most of the linguistic encoding methods.


\begin{table}[t]
    \centering
    \small
    \addtolength{\tabcolsep}{-4pt}
    \setlength{\tabcolsep}{1mm}{
    \begin{tabular}{ lcc }
    \toprule 
     & iFLYTEK
     & TNEWS
     \\
\hline
\multicolumn{3}{l}{\textit{Vocab Size = 22675}} \\
% CharTokenizer & 289.0 & 22.0 \\
Sub-word & 255.2 & 20.1 \\
SubChar-Pinyin-NoIndex & \textbf{175.4} & \textbf{15.2} \\
\midrule
\multicolumn{3}{l}{\textit{Vocab Size = 40000}} \\
% CharTokenizer &  \\
Sub-word & 188.9 & 15.9 \\
SubChar-Pinyin-NoIndex & \textbf{166.1} & \textbf{14.4} \\
\midrule
\multicolumn{3}{l}{\textit{Vocab Size = 60000}} \\
% CharTokenizer &  \\
Sub-word & 176.2 & 14.9 \\
SubChar-Pinyin-NoIndex & \textbf{164.0} & \textbf{14.1} \\
    \bottomrule
    \end{tabular}}
    \caption{Comparison of average length of tokenized sequences with different tokenizers and different vocabulary sizes.}
    \label{tab:vocab_size_ablation}
    \end{table}


\begin{table*}[t]
    \centering
    \small
    \addtolength{\tabcolsep}{-2pt}
    \setlength{\tabcolsep}{3.5pt}{
    \begin{tabular}{ lcccccccc }
    \toprule             & TNEWS & IFLY  & BQ    & WSC   & AFQMC & CSL   & OCNLI & AVG \\
    \hline
    SubChar-Pinyin       & 63.68 & 58.81 & 81.74 & 65.90 & 68.89 & 82.87 & 67.98 & 70.16 \\
    SubChar-Zhuyin       & 64.91 & 59.39 & 81.41 & 62.72 & 69.14 & 82.60 & 69.12 & 69.90 \\
    SubChar-Stroke       & 64.26 & 55.44 & 81.52 & 62.06 & 69.88 & 83.16 & 68.98 & 69.33 \\
    SubChar-Wubi         & 63.81 & 58.74 & 81.55 & 64.61 & 69.66 & 82.44 & 68.02 & 69.90 \\
    SubChar-Zhengma      & 63.86 & 59.51 & 81.59 & 63.27 & 70.47 & 82.91 & 69.03 & 70.09 \\
    SubChar-Cangjie      & 64.10 & 57.77 & 81.98 & 62.39 & 68.95 & 82.60 & 68.46 & 69.46 \\
    SubChar-Byte         & 63.58 & 59.55 & 81.65 & 63.60 & 68.60 & 82.66 & 67.93 & 69.65 \\
    SubChar-RandomIndex  & 64.11 & 59.16 & 81.64 & 63.93 & 68.53 & 82.86 & 69.39 & 69.95 \\
    \midrule
    SubChar-Pinyin (BPE) & 63.86 & 58.84 & 82.12 & 65.57 & 69.86 & 82.86 & 68.57 & 70.24 \\
    \bottomrule
    \end{tabular}}
    \caption{Results of SubChar tokenizers when using different encoding methods. The last row is a model with SubChar-Pinyin tokenizer using BPE as the subword tokenization algorithm, all previous rows are using unigram LM as the subword tokenization implementation. All models have 6-layers with the same hyper-parameters. The impact of different encoding methods on downstream performance is small, and the ULM and BPE versions of SubChar-Pinyin also achieve similar results.}
    \label{tab:encode_ablation}
    \end{table*}




\subsection{Impact of Vocabulary Size}

Intuitively, when we increase the vocabulary size, there will also be more room to store combination tokens (\textit{e.g.}, words and phrases), leading to a decrease in tokenization length and thus better efficiency. Although we used the standard vocabulary size of 22675 in our previous experiments, to understand whether the efficiency benefits of SubChar tokenization wear off at larger vocabulary size, we perform an additional ablation on the impact of vocabulary size. 

As shown in Table~\ref{tab:vocab_size_ablation}, as we increase the vocabulary size, the efficiency advantage of SubChar tokenizers slightly diminishes. However, even at a very large vocab size of 60000, our SubChar-Pinyin tokenizer still tokenizes the inputs into significantly shorter sequences than the Sub-word baseline.  We thus conclude that the efficiency advantage of our SubChar tokenizers would hold in most practical cases where the vocabulary size is typically under 60000 (such as BERT and RoBERTa). 



\subsection{Impact of Pretraining Data Size}
\label{sec:pretraining-data}

To understand the impact of pretraining data size, we take the checkpoints of the 12-layer Transformer models pretrained on the 2.3G Baike corpus, and further pretrain them on a much larger corpus of 22.1GB text. This 22.1GB corpus is sampled from Chinese web text\footnote{\url{https://github.com/OpenBMB/CPM-Live}}, mainly consisting of books and web pages. We further pretrain for 8K steps with a maximum sequence length of 512. 

As shown in the bottom block of Table~\ref{tab:main_results}, further training on this larger corpus leads to small improvement on average performance (72.23 $\rightarrow$ 72.81 for CharTokenizer and 72.87 $\rightarrow$ 73.42 for SubChar-Pinyin), possibly because the original models trained on 2.3GB corpus are already close to being fully trained. More importantly, this result shows that even with pretraining on larger corpora, our proposed methods can still match or slightly outperform baselines on the downstream datasets. 

\subsection{Impact of Encoding Methods}
\label{sec:encoding}

As described in Section~\ref{sec:method}, 
we experiment with different types of encoding methods and compare their downstream performance to analyze the impact.

Our previous encoding methods are based on the hypothesis that linguistic information such as glyph or pronunciation provides useful inductive biases to the model. However, in the case where this hypothesis is not true, it is possible that non-linguistic encoding methods may work as well. To verify this, we add two encoding methods that do not consider any linguistic information: \textit{Byte Encoding} and \textit{Random Index Encoding}, for the purpose of ablation analysis.

In \textit{Byte Encoding}, we convert every character into its byte sequence, same as in ByT5~\cite{ByT5}. In cases where the byte sequence consists of multiple indices (each Chinese character has three byte indices), we concatenate them and append the character separation symbol as the encoding (\textit{e.g., 魑} $\rightarrow$ \textit{233\_173\_145\#}). 

In \textit{Random Index Encoding}, we map each character into a unique and randomly generated five-digit index and append the character separation symbol as the encoding ( \textit{e.g., 魑} $\rightarrow$ \textit{29146\#} ). 


We train SubChar tokenizers with all the different encoding methods and compare the corresponding BERT models using these tokenizers on downstream tasks. The results are presented in Table~\ref{tab:encode_ablation}. We observe that the differences between these different tokenizers are rather small in terms of the model performance on downstream datasets. Moreover, perhaps somewhat surprisingly, tokenizers with the non-linguistic encoding methods -- SubChar-Byte and SubChar-RandomIndex -- can also perform competitively despite the fact that they do not capture glyph or pronunciation information like the other tokenizers. 

These results suggest that linguistic encoding may not be necessary for SubChar tokenizers to achieve high performance on downstream tasks. However, the linguistic encoding methods can build more robust and efficient tokenizers as illustrated in previous sections.




% \subsection{Impact of Subword Tokenization Implementation}

% In previous experiments, we use the unigram LM method to construct the vocabulary based on the transliterated sequences. There are alternative implementations of the subword tokenization algorithm. In this section, we replace the unigram LM algorithm with byte-pair encoding (BPE) and analyze whether the specific choice of subword tokenization implementation matters for the overall performance. 

% We train a tokenizer on Pinyin transliterated corpus with BPE as the subword tokenization implementation. We pretrain a 6-layer transformer with this tokenizer and compare the performance with the unigram LM counterpart in Table~\ref{tab:encode_ablation} (comparing the first and last row of the results). 

% We further compare the vocabulary composition and tokenized sequence lengths between tokenizers using the unigram LM and BPE implementations, we share the detailed results in the appendix.


\subsection{Impact of Vocabulary Construction Algorithm}
\label{sec:bpe}

In previous experiments, we used the Unigram LM implementation in SentencePiece for vocabulary construction. We perform an additional ablation where we replace Unigram LM with Byte Pair Encoding (BPE) for vocabulary construction to train a pinyin-based tokenizer, while holding all other hyper-parameters constant. 

We compare the SubChar-Pinyin-BPE variant with the unigram LM (SubChar-Pinyin) tokenizer. We find that these two perform similarly. In terms of \textbf{efficiency}: SubChar-Pinyin-BPE tokenizes iFLYTEK to an average length of 184.4 and tokenizes TNEWS to an average length of 15.9. In comparison, SubChar-Pinyin tokenizes iFLYTEK to an average length of 185.2 and tokenizes TNEWS to an average length of 16.1. The vocabulary compositions of the two are also similar, where character combination takes up the majority of the space in the vocabulary for both BPE and unigram LM implementations. 
In terms of \textbf{performance}, we observe in Table~\ref{tab:encode_ablation} that the BPE implementation and the unigram LM implementation have little difference in downstream task performance.
Based on these results, we conclude that the choice of which vocabulary construction to use has a marginal impact on the tokenization efficiency and model performance. 

% \chenglei{TODO: Complete this part.}


\section{Related Work}
\section{Related work}
\label{sec:related_work}

Accessibility is an essential component of computing, which aims to make technology broadly accessible to as many users as possible, including those with differing sets of abilities. Improvements in usability and accessibility falls to the community, to better understand the needs of users with differing abilities, and to design technologies that play to this spectrum of abilities \citep{Wobbrock2011AbilityBasedDC}.
In computing, significant strides have been made to increase the accessibility of web content. For example, various versions of the Web Content Accessibility Guidelines (WCAG) \citep{Chisholm2001WebCA, Caldwell2008WebCA} and the in-progress working draft for WCAG 3.0,\footnote{\href{https://www.w3.org/TR/wcag-3.0/}{https://www.w3.org/TR/wcag-3.0/}} or standards such as ARIA from the W3C's Web Accessibility Initiative (WAI)\footnote{\href{https://www.w3.org/WAI/standards-guidelines/aria/}{https://www.w3.org/WAI/standards-guidelines/aria/}} have been released and used to guide web accessibility design and implementation. Similarly, positive steps have been made to improve the accessibility of user interfaces and user experience \citep{Peissner2012MyUIGA, Peissner2013UserCI, Thompson2014ImprovingTU, Bigham2014MakingTW}, as well as various types of media content \citep{Mirri2017TowardsAG, Nengroo2017AccessibleI, Gleason2020TwitterAA}. 

We take inspiration from accessibility design principles in our effort to make research publications more accessible to users who are blind and low vision. Blindness and low vision are some of the most common forms of disability, affecting an estimated 3--10\% of Americans depending on how visual impairment is defined \citep{CDCVisionLossBurden}. BLV researchers also make up a representative sample of researchers in the United States and worldwide. A recent Nature editorial pushes the scientific community to better support researchers with visual impairments \citep{NatureCareerColumn2020}, since existing tools and resources can be limited. There are many inherent accessibility challenges to performing research. In this paper, we engage with one of these challenges that affects all domains of study, accessing and reading the content of academic publications. 

BLV users interact with papers using screen readers, braille displays, text-to-speech, and other assistive tools. A WebAIM survey of screen reader users found that the vast majority (75.1\%) of respondents indicate that PDF documents are very or somewhat likely to pose significant accessibility issues.\footnote{\href{https://webaim.org/projects/screenreadersurvey8/}{https://webaim.org/projects/screenreadersurvey8/}} Most paper are published in PDF, which is inherently inaccessible, due in large part to its conflation of visual layout information with semantic content \citep{NielsenPDFStillUnfit, Bigham2016AnUT}. 
\citet{Bigham2016AnUT} describe the historical reasons we use PDF as the standard document format for scientific publications, as well as the barriers the format itself presents to accessibility. Prior work on scientific accessibility have made recommendations for how to make PDFs more accessible \cite{Rajkumar2020PDFAO, Darvishy2018PDFAT}, including greater awareness for what constitutes an accessible PDF and better tooling for generating accessible PDFs. Some work has focused on addressing components of paper accessibility, such as the correct way for screen readers to interpret and read mathematical equations \citep{Flores2010MathMLTA, Bates2010SpokenMU, Sorge2014TowardsMM, Mackowski2017MultimediaPF, Ahmetovic2018AxessibilityAL, Ferreira2004EnhancingTA, Sojka2013AccessibilityII}, describe charts and figures \citep{Elzer2008AccessibleBC, Engel2017TowardsAC, Engel2019SVGPlottAA}, automatically generate figure captions \citep{Chen2019NeuralCG, Qian2020AFS}, or automatically classify the content of figures \citep{Kim2018MultimodalDL}. Other work applicable to all types of PDF documents aims to improve automatic text and layout detection of scanned documents \cite{Nazemi2014PracticalSM} and extract table content \cite{Fan2015TableRD, Rastan2019TEXUSAU}. In this work, we focus on the issue of representing overall document structure, and navigation within that structure. Being able to quickly navigate the contents of a paper through skimming and scanning is an essential reading technique \citep{Maxwell1972SkimmingAS}, which is currently under-supported by PDF documents and PDF readers when reading these documents by screen reader. 

There also exists a variety of automatic and manual tools that assess and fix accessibility compliance issues in PDFs, including the Adobe Acrobat Pro Accessibility Checker\footnote{\href{https://www.adobe.com/accessibility/products/acrobat/using-acrobat-pro-accessibility-checker.html}{https://www.adobe.com/accessibility/products/acrobat/using-acrobat-pro-accessibility-checker.html}}, Common Look\footnote{\href{https://monsido.com/monsido-commonlook-partnership}{https://monsido.com/monsido-commonlook-partnership}}, ABBYY FineReader\footnote{\href{https://pdf.abbyy.com/}{https://pdf.abbyy.com/}}, PAVE\footnote{\href{https://pave-pdf.org/faq.html}{https://pave-pdf.org/faq.html}}, and PDFA Inspector\footnote{\href{https://github.com/pdfae/PDFAInspector}{https://github.com/pdfae/PDFAInspector}}. To our knowledge, PAVE and PDFA Inspector are the only non-proprietary, open-source tools for this purpose. Based on our experiences, however, all of these tools require some degree of human intervention to properly tag a scientific document, and tagging and fixing must be performed for each new version of a PDF, regardless of how minor the change may be.

Guidelines and policy changes have been introduced in the past decade to ameliorate some of the issues around scientific PDF accessibility. Some conferences, such as The ACM CHI Virtual Conference on Human Factors in Computing Systems (CHI) and The ACM SIGACCESS Conference on Computers and Accessibility (ASSETS), have released guidelines for creating accessible submissions.\footnote{See \href{http://chi2019.acm.org/authors/papers/guide-to-an-accessible-submission/}{http://chi2019.acm.org/authors/papers/guide-to-an-accessible-submission/} and \href{https://assets19.sigaccess.org/creating_accessible_pdfs.html}{https://assets19.sigaccess.org/creating\_accessible\_pdfs.html}} The ACM Digital Library\footnote{\href{https://dl.acm.org/}{https://dl.acm.org/}} provides some publications in HTML format, which is easier to make accessible than PDF~\cite{Graells2007EstudioDL}. \citet{Ribera2019PublishingAP} conducted a case study on DSAI 2016 (Software Development and Technologies for Enhancing Accessibility and Fighting Infoexclusion). The authors of DSAI were responsible for creating accessible proceedings and identified barriers to creating accessible proceedings, including lack of sufficient tooling and lack of awareness of accessibility. The authors recommended creating a new role in the organizing committee dedicated to accessible publishing. These policy changes have led to improvements in localized communities, but have not been widely adopted by all academic publishers and conference organizers.

Table~\ref{tab:prior_work} lists prior studies that have analyzed PDF accessibility of academic papers, and shows how our study compares. Prior work has primarily focused on papers published in Human-Computer Interaction and related fields, specific to certain publication venues, while our analysis tries to quantify paper accessibility more broadly.
\citet{Brady2015CreatingAP} quantified the accessibility of 1,811 papers from CHI 2010-2016, ASSETS 2014, and W4A, assessing the presence of document tags, headers, and language. They found that compliance improved over time as a response to conference organizers offering to make papers accessible as a service to any author upon request. \citet{Lazar2017MakingTF} conducted a study quantifying accessibility compliance at CHI from 2010 to 2016 as well as ASSETS 2015,
%\jb{Define acronyms in prev para}
confirming the results of \citet{Brady2015CreatingAP}. They found that across 5 accessibility criteria, the rate of compliance was less than 30\% for CHI papers in each of the 7 years that were studied. The study also analyzed papers from ASSETS 2015, an ACM conference explicitly focused on accessibility, and found that those papers had significantly higher rates of compliance, with over 90\% of the papers being tagged for correct reading order and no criteria having less than 50\% compliance. This finding indicates that community buy-in is an important contributor to paper accessibility.
\citet{Nganji2015ThePD} conducted a study of 200 PDFs of papers published in four disability studies journals, finding that accessibility compliance was between 15-30\% for the four journals analyzed, with some publishers having higher adherence than others. To date, no large scale analysis of scientific PDF accessibility has been conducted outside of disability studies and HCI, due in part to the challenge of scaling such an analysis. We believe such an analysis is useful for establishing a baseline and characterizing routes for future improvement. Consequently, as part of this work, we conduct an analysis of scientific PDF accessibility across various fields of study, and report our findings relative to prior work. 


\begin{table}[t!]
\small
    \centering
    \begin{tabularx}{\linewidth}{L{22mm}L{15mm}L{48mm}L{16mm}L{34mm}}
        \toprule
        \textbf{Prior work} & \textbf{PDFs analyzed} & \textbf{Venues} & \textbf{Year} & \textbf{Accessibility checker} \\
        \midrule
        \citet{Brady2015CreatingAP} & 1811 & CHI, ASSETS and W4A & 2011--2014 & PDFA Inspector \\ [0.5mm]
        \hline \\ [-2.5mm]
        \citet{Lazar2017MakingTF} & 465 + 32 & CHI and ASSETS & 2014--2015 & Adobe Acrobat Action Wizard \\ [0.5mm]
        \hline \\ [-2.5mm]
        \citet{Ribera2019PublishingAP} & 59 & DSAI & 2016 & Adobe PDF Accessibility Checker 2.0 \\ [0.5mm]
        \hline \\ [-2.5mm]
        \citet{Nganji2015ThePD} & 200 & \textit{Disability \& Society}, \textit{Journal of Developmental and Physical Disabilities}, \textit{Journal of Learning Disabilities}, and \textit{Research in Developmental Disabilities} & 2009--2013 & Adobe PDF Accessibility Checker 1.3 \\ [0.6mm]
        \hline \\ [-2.5mm]
        \textbf{\textit{Our analysis}} & \numpdfs & Venues across various fields of study & 2010--2019 & Adobe Acrobat Accessibility Plug-in Version 21.001.20145 \\
        \bottomrule
    \end{tabularx}
    \caption{Prior work has investigated PDF accessibility for papers published in specific venues such as CHI, ASSETS, W4A, DSAI, or various disability journals. Several of these works were conducted manually, and were limited to a small number of papers, while the more thorough analysis was conducted for CHI and ASSETS, two conference venues focused on accessibility and HCI. Our study expands on this prior work to investigate accessibility over \numpdfs PDFs sampled from across different fields of study.
    }
    % \Description{
    % Prior work, PDFs analyzed, Venues, Year, Accessibility checker 
    % Brady et al. [7], 1811, CHI, ASSETS and W4A, 2011--2014, PDFA Inspector 
    % Lazar et al. [23], 465 + 32, CHI and ASSETS, 2014--2015, Adobe Acrobat Action Wizard 
    % Ribera et al. [40], 59, DSAI, 2016, Adobe PDF Accessibility Checker 2.0 
    % Nganji [33], 200, Disability & Society, Journal of Developmental and Physical Disabilities, Journal of Learning Disabilities, and Research in Developmental Disabilities, 2009--2013, Adobe PDF Accessibility Checker 1.3
    % Our analysis, 11397, Venues across various fields of study, 2010--2019, Adobe Acrobat Accessibility Plug-in Version 21.001.20145 
    % }
    \label{tab:prior_work}
\end{table}

\section{Conclusion}

\section{Conclusion and Future Work}\label{sec:conclusion}
In this paper, we report an approach which adopts reinforcement learning algorithms to solve the problem of robustness-guided falsification of CPS systems. We implement our approach in a prototype tool and conduct preliminary evaluations with a widely adopted CPS system. The evaluation results show that our method can reduce the number of episodes to find the falsifying input. As a future work, we plan to extend the current work to explore more reinforcement learning algorithms and evaluate our methods on more CPS benchmarks. 


\section*{Limitations}

Our experiments are focused on natural language understanding tasks. We recognize that adapting our SubChar tokenization to language generation tasks might require additional efforts, for example, we may want to avoid cases of predicting sub-character tokens that do not form complete characters. Also, evaluating the robustness of language generation models on real-world input noises may require additional benchmarks beyond those used in this paper. We leave such exploration as an interesting direction for future work. 

Another limitation is that our method is designed specifically for the Chinese language. While we hypothesize that our method can also bring benefits to other languages with ideographic symbols, such as Kanji in Japanese, we leave such investigation to future work. 




\section*{Broader Impact}

We expect our work to have a positive impact on the society. Firstly, we addressed the practical problem of handling input with real-world noises. Such noisy settings are very common in real-life applications. Our method, along with the evaluation framework, can help make language technologies more robust and reliable in real-world applications, especially for Chinese users. Secondly, we addressed the efficiency concern of large language models by significantly reducing both and training and inference time. This not only reduces the latency of these models in real-world applications, more importantly, helps reduce the environmental costs of using these large language models, moving further towards Green AI. All of our code and models are released with proper documentation in order to better facilitate the adoption of our work in a wide range of research and industrial applications. 

\section*{Acknowledgement}

\section*{Acknowledgement}
We would like to express our special gratitude to Mark Tygert and Chuan Guo for the assistance with the privacy guarantee section. We also thank Edward Suh, Wenjie Xiong, Hsien-Hsin Sean Lee for sharing their wisdom with us. We are immensely grateful to Krishna Giri Narra and Caroline Tripple for their valuable feedbacks on the earlier version of this project.
This material is based upon work supported by the Defense Advanced Research Projects Agency (DARPA) under Contract No. HR001117C0053, HR001120C0088, Intel Private AI institute, and Facebook AI Research Award Numbers 2215031173 and 2215031183.

\bibliography{tacl2018}
\bibliographystyle{acl_natbib}

\clearpage

% \appendix 
% \section{Appendix}
% \label{appendix}

% \chapter{Supplementary Material}
\label{appendix}

In this appendix, we present supplementary material for the techniques and
experiments presented in the main text.

\section{Baseline Results and Analysis for Informed Sampler}
\label{appendix:chap3}

Here, we give an in-depth
performance analysis of the various samplers and the effect of their
hyperparameters. We choose hyperparameters with the lowest PSRF value
after $10k$ iterations, for each sampler individually. If the
differences between PSRF are not significantly different among
multiple values, we choose the one that has the highest acceptance
rate.

\subsection{Experiment: Estimating Camera Extrinsics}
\label{appendix:chap3:room}

\subsubsection{Parameter Selection}
\paragraph{Metropolis Hastings (\MH)}

Figure~\ref{fig:exp1_MH} shows the median acceptance rates and PSRF
values corresponding to various proposal standard deviations of plain
\MH~sampling. Mixing gets better and the acceptance rate gets worse as
the standard deviation increases. The value $0.3$ is selected standard
deviation for this sampler.

\paragraph{Metropolis Hastings Within Gibbs (\MHWG)}

As mentioned in Section~\ref{sec:room}, the \MHWG~sampler with one-dimensional
updates did not converge for any value of proposal standard deviation.
This problem has high correlation of the camera parameters and is of
multi-modal nature, which this sampler has problems with.

\paragraph{Parallel Tempering (\PT)}

For \PT~sampling, we took the best performing \MH~sampler and used
different temperature chains to improve the mixing of the
sampler. Figure~\ref{fig:exp1_PT} shows the results corresponding to
different combination of temperature levels. The sampler with
temperature levels of $[1,3,27]$ performed best in terms of both
mixing and acceptance rate.

\paragraph{Effect of Mixture Coefficient in Informed Sampling (\MIXLMH)}

Figure~\ref{fig:exp1_alpha} shows the effect of mixture
coefficient ($\alpha$) on the informed sampling
\MIXLMH. Since there is no significant different in PSRF values for
$0 \le \alpha \le 0.7$, we chose $0.7$ due to its high acceptance
rate.


% \end{multicols}

\begin{figure}[h]
\centering
  \subfigure[MH]{%
    \includegraphics[width=.48\textwidth]{figures/supplementary/camPose_MH.pdf} \label{fig:exp1_MH}
  }
  \subfigure[PT]{%
    \includegraphics[width=.48\textwidth]{figures/supplementary/camPose_PT.pdf} \label{fig:exp1_PT}
  }
\\
  \subfigure[INF-MH]{%
    \includegraphics[width=.48\textwidth]{figures/supplementary/camPose_alpha.pdf} \label{fig:exp1_alpha}
  }
  \mycaption{Results of the `Estimating Camera Extrinsics' experiment}{PRSFs and Acceptance rates corresponding to (a) various standard deviations of \MH, (b) various temperature level combinations of \PT sampling and (c) various mixture coefficients of \MIXLMH sampling.}
\end{figure}



\begin{figure}[!t]
\centering
  \subfigure[\MH]{%
    \includegraphics[width=.48\textwidth]{figures/supplementary/occlusionExp_MH.pdf} \label{fig:exp2_MH}
  }
  \subfigure[\BMHWG]{%
    \includegraphics[width=.48\textwidth]{figures/supplementary/occlusionExp_BMHWG.pdf} \label{fig:exp2_BMHWG}
  }
\\
  \subfigure[\MHWG]{%
    \includegraphics[width=.48\textwidth]{figures/supplementary/occlusionExp_MHWG.pdf} \label{fig:exp2_MHWG}
  }
  \subfigure[\PT]{%
    \includegraphics[width=.48\textwidth]{figures/supplementary/occlusionExp_PT.pdf} \label{fig:exp2_PT}
  }
\\
  \subfigure[\INFBMHWG]{%
    \includegraphics[width=.5\textwidth]{figures/supplementary/occlusionExp_alpha.pdf} \label{fig:exp2_alpha}
  }
  \mycaption{Results of the `Occluding Tiles' experiment}{PRSF and
    Acceptance rates corresponding to various standard deviations of
    (a) \MH, (b) \BMHWG, (c) \MHWG, (d) various temperature level
    combinations of \PT~sampling and; (e) various mixture coefficients
    of our informed \INFBMHWG sampling.}
\end{figure}

%\onecolumn\newpage\twocolumn
\subsection{Experiment: Occluding Tiles}
\label{appendix:chap3:tiles}

\subsubsection{Parameter Selection}

\paragraph{Metropolis Hastings (\MH)}

Figure~\ref{fig:exp2_MH} shows the results of
\MH~sampling. Results show the poor convergence for all proposal
standard deviations and rapid decrease of AR with increasing standard
deviation. This is due to the high-dimensional nature of
the problem. We selected a standard deviation of $1.1$.

\paragraph{Blocked Metropolis Hastings Within Gibbs (\BMHWG)}

The results of \BMHWG are shown in Figure~\ref{fig:exp2_BMHWG}. In
this sampler we update only one block of tile variables (of dimension
four) in each sampling step. Results show much better performance
compared to plain \MH. The optimal proposal standard deviation for
this sampler is $0.7$.

\paragraph{Metropolis Hastings Within Gibbs (\MHWG)}

Figure~\ref{fig:exp2_MHWG} shows the result of \MHWG sampling. This
sampler is better than \BMHWG and converges much more quickly. Here
a standard deviation of $0.9$ is found to be best.

\paragraph{Parallel Tempering (\PT)}

Figure~\ref{fig:exp2_PT} shows the results of \PT sampling with various
temperature combinations. Results show no improvement in AR from plain
\MH sampling and again $[1,3,27]$ temperature levels are found to be optimal.

\paragraph{Effect of Mixture Coefficient in Informed Sampling (\INFBMHWG)}

Figure~\ref{fig:exp2_alpha} shows the effect of mixture
coefficient ($\alpha$) on the blocked informed sampling
\INFBMHWG. Since there is no significant different in PSRF values for
$0 \le \alpha \le 0.8$, we chose $0.8$ due to its high acceptance
rate.



\subsection{Experiment: Estimating Body Shape}
\label{appendix:chap3:body}

\subsubsection{Parameter Selection}
\paragraph{Metropolis Hastings (\MH)}

Figure~\ref{fig:exp3_MH} shows the result of \MH~sampling with various
proposal standard deviations. The value of $0.1$ is found to be
best.

\paragraph{Metropolis Hastings Within Gibbs (\MHWG)}

For \MHWG sampling we select $0.3$ proposal standard
deviation. Results are shown in Fig.~\ref{fig:exp3_MHWG}.


\paragraph{Parallel Tempering (\PT)}

As before, results in Fig.~\ref{fig:exp3_PT}, the temperature levels
were selected to be $[1,3,27]$ due its slightly higher AR.

\paragraph{Effect of Mixture Coefficient in Informed Sampling (\MIXLMH)}

Figure~\ref{fig:exp3_alpha} shows the effect of $\alpha$ on PSRF and
AR. Since there is no significant differences in PSRF values for $0 \le
\alpha \le 0.8$, we choose $0.8$.


\begin{figure}[t]
\centering
  \subfigure[\MH]{%
    \includegraphics[width=.48\textwidth]{figures/supplementary/bodyShape_MH.pdf} \label{fig:exp3_MH}
  }
  \subfigure[\MHWG]{%
    \includegraphics[width=.48\textwidth]{figures/supplementary/bodyShape_MHWG.pdf} \label{fig:exp3_MHWG}
  }
\\
  \subfigure[\PT]{%
    \includegraphics[width=.48\textwidth]{figures/supplementary/bodyShape_PT.pdf} \label{fig:exp3_PT}
  }
  \subfigure[\MIXLMH]{%
    \includegraphics[width=.48\textwidth]{figures/supplementary/bodyShape_alpha.pdf} \label{fig:exp3_alpha}
  }
\\
  \mycaption{Results of the `Body Shape Estimation' experiment}{PRSFs and
    Acceptance rates corresponding to various standard deviations of
    (a) \MH, (b) \MHWG; (c) various temperature level combinations
    of \PT sampling and; (d) various mixture coefficients of the
    informed \MIXLMH sampling.}
\end{figure}


\subsection{Results Overview}
Figure~\ref{fig:exp_summary} shows the summary results of the all the three
experimental studies related to informed sampler.
\begin{figure*}[h!]
\centering
  \subfigure[Results for: Estimating Camera Extrinsics]{%
    \includegraphics[width=0.9\textwidth]{figures/supplementary/camPose_ALL.pdf} \label{fig:exp1_all}
  }
  \subfigure[Results for: Occluding Tiles]{%
    \includegraphics[width=0.9\textwidth]{figures/supplementary/occlusionExp_ALL.pdf} \label{fig:exp2_all}
  }
  \subfigure[Results for: Estimating Body Shape]{%
    \includegraphics[width=0.9\textwidth]{figures/supplementary/bodyShape_ALL.pdf} \label{fig:exp3_all}
  }
  \label{fig:exp_summary}
  \mycaption{Summary of the statistics for the three experiments}{Shown are
    for several baseline methods and the informed samplers the
    acceptance rates (left), PSRFs (middle), and RMSE values
    (right). All results are median results over multiple test
    examples.}
\end{figure*}

\subsection{Additional Qualitative Results}

\subsubsection{Occluding Tiles}
In Figure~\ref{fig:exp2_visual_more} more qualitative results of the
occluding tiles experiment are shown. The informed sampling approach
(\INFBMHWG) is better than the best baseline (\MHWG). This still is a
very challenging problem since the parameters for occluded tiles are
flat over a large region. Some of the posterior variance of the
occluded tiles is already captured by the informed sampler.

\begin{figure*}[h!]
\begin{center}
\centerline{\includegraphics[width=0.95\textwidth]{figures/supplementary/occlusionExp_Visual.pdf}}
\mycaption{Additional qualitative results of the occluding tiles experiment}
  {From left to right: (a)
  Given image, (b) Ground truth tiles, (c) OpenCV heuristic and most probable estimates
  from 5000 samples obtained by (d) MHWG sampler (best baseline) and
  (e) our INF-BMHWG sampler. (f) Posterior expectation of the tiles
  boundaries obtained by INF-BMHWG sampling (First 2000 samples are
  discarded as burn-in).}
\label{fig:exp2_visual_more}
\end{center}
\end{figure*}

\subsubsection{Body Shape}
Figure~\ref{fig:exp3_bodyMeshes} shows some more results of 3D mesh
reconstruction using posterior samples obtained by our informed
sampling \MIXLMH.

\begin{figure*}[t]
\begin{center}
\centerline{\includegraphics[width=0.75\textwidth]{figures/supplementary/bodyMeshResults.pdf}}
\mycaption{Qualitative results for the body shape experiment}
  {Shown is the 3D mesh reconstruction results with first 1000 samples obtained
  using the \MIXLMH informed sampling method. (blue indicates small
  values and red indicates high values)}
\label{fig:exp3_bodyMeshes}
\end{center}
\end{figure*}

\clearpage



\section{Additional Results on the Face Problem with CMP}

Figure~\ref{fig:shading-qualitative-multiple-subjects-supp} shows inference results for reflectance maps, normal maps and lights for randomly chosen test images, and Fig.~\ref{fig:shading-qualitative-same-subject-supp} shows reflectance estimation results on multiple images of the same subject produced under different illumination conditions. CMP is able to produce estimates that are closer to the groundtruth across different subjects and illumination conditions.

\begin{figure*}[h]
  \begin{center}
  \centerline{\includegraphics[width=1.0\columnwidth]{figures/face_cmp_visual_results_supp.pdf}}
  \vspace{-1.2cm}
  \end{center}
	\mycaption{A visual comparison of inference results}{(a)~Observed images. (b)~Inferred reflectance maps. \textit{GT} is the photometric stereo groundtruth, \textit{BU} is the Biswas \etal (2009) reflectance estimate and \textit{Forest} is the consensus prediction. (c)~The variance of the inferred reflectance estimate produced by \MTD (normalized across rows).(d)~Visualization of inferred light directions. (e)~Inferred normal maps.}
	\label{fig:shading-qualitative-multiple-subjects-supp}
\end{figure*}


\begin{figure*}[h]
	\centering
	\setlength\fboxsep{0.2mm}
	\setlength\fboxrule{0pt}
	\begin{tikzpicture}

		\matrix at (0, 0) [matrix of nodes, nodes={anchor=east}, column sep=-0.05cm, row sep=-0.2cm]
		{
			\fbox{\includegraphics[width=1cm]{figures/sample_3_4_X.png}} &
			\fbox{\includegraphics[width=1cm]{figures/sample_3_4_GT.png}} &
			\fbox{\includegraphics[width=1cm]{figures/sample_3_4_BISWAS.png}}  &
			\fbox{\includegraphics[width=1cm]{figures/sample_3_4_VMP.png}}  &
			\fbox{\includegraphics[width=1cm]{figures/sample_3_4_FOREST.png}}  &
			\fbox{\includegraphics[width=1cm]{figures/sample_3_4_CMP.png}}  &
			\fbox{\includegraphics[width=1cm]{figures/sample_3_4_CMPVAR.png}}
			 \\

			\fbox{\includegraphics[width=1cm]{figures/sample_3_5_X.png}} &
			\fbox{\includegraphics[width=1cm]{figures/sample_3_5_GT.png}} &
			\fbox{\includegraphics[width=1cm]{figures/sample_3_5_BISWAS.png}}  &
			\fbox{\includegraphics[width=1cm]{figures/sample_3_5_VMP.png}}  &
			\fbox{\includegraphics[width=1cm]{figures/sample_3_5_FOREST.png}}  &
			\fbox{\includegraphics[width=1cm]{figures/sample_3_5_CMP.png}}  &
			\fbox{\includegraphics[width=1cm]{figures/sample_3_5_CMPVAR.png}}
			 \\

			\fbox{\includegraphics[width=1cm]{figures/sample_3_6_X.png}} &
			\fbox{\includegraphics[width=1cm]{figures/sample_3_6_GT.png}} &
			\fbox{\includegraphics[width=1cm]{figures/sample_3_6_BISWAS.png}}  &
			\fbox{\includegraphics[width=1cm]{figures/sample_3_6_VMP.png}}  &
			\fbox{\includegraphics[width=1cm]{figures/sample_3_6_FOREST.png}}  &
			\fbox{\includegraphics[width=1cm]{figures/sample_3_6_CMP.png}}  &
			\fbox{\includegraphics[width=1cm]{figures/sample_3_6_CMPVAR.png}}
			 \\
	     };

       \node at (-3.85, -2.0) {\small Observed};
       \node at (-2.55, -2.0) {\small `GT'};
       \node at (-1.27, -2.0) {\small BU};
       \node at (0.0, -2.0) {\small MP};
       \node at (1.27, -2.0) {\small Forest};
       \node at (2.55, -2.0) {\small \textbf{CMP}};
       \node at (3.85, -2.0) {\small Variance};

	\end{tikzpicture}
	\mycaption{Robustness to varying illumination}{Reflectance estimation on a subject images with varying illumination. Left to right: observed image, photometric stereo estimate (GT)
  which is used as a proxy for groundtruth, bottom-up estimate of \cite{Biswas2009}, VMP result, consensus forest estimate, CMP mean, and CMP variance.}
	\label{fig:shading-qualitative-same-subject-supp}
\end{figure*}

\clearpage

\section{Additional Material for Learning Sparse High Dimensional Filters}
\label{sec:appendix-bnn}

This part of supplementary material contains a more detailed overview of the permutohedral
lattice convolution in Section~\ref{sec:permconv}, more experiments in
Section~\ref{sec:addexps} and additional results with protocols for
the experiments presented in Chapter~\ref{chap:bnn} in Section~\ref{sec:addresults}.

\vspace{-0.2cm}
\subsection{General Permutohedral Convolutions}
\label{sec:permconv}

A core technical contribution of this work is the generalization of the Gaussian permutohedral lattice
convolution proposed in~\cite{adams2010fast} to the full non-separable case with the
ability to perform back-propagation. Although, conceptually, there are minor
differences between Gaussian and general parameterized filters, there are non-trivial practical
differences in terms of the algorithmic implementation. The Gauss filters belong to
the separable class and can thus be decomposed into multiple
sequential one dimensional convolutions. We are interested in the general filter
convolutions, which can not be decomposed. Thus, performing a general permutohedral
convolution at a lattice point requires the computation of the inner product with the
neighboring elements in all the directions in the high-dimensional space.

Here, we give more details of the implementation differences of separable
and non-separable filters. In the following, we will explain the scalar case first.
Recall, that the forward pass of general permutohedral convolution
involves 3 steps: \textit{splatting}, \textit{convolving} and \textit{slicing}.
We follow the same splatting and slicing strategies as in~\cite{adams2010fast}
since these operations do not depend on the filter kernel. The main difference
between our work and the existing implementation of~\cite{adams2010fast} is
the way that the convolution operation is executed. This proceeds by constructing
a \emph{blur neighbor} matrix $K$ that stores for every lattice point all
values of the lattice neighbors that are needed to compute the filter output.

\begin{figure}[t!]
  \centering
    \includegraphics[width=0.6\columnwidth]{figures/supplementary/lattice_construction}
  \mycaption{Illustration of 1D permutohedral lattice construction}
  {A $4\times 4$ $(x,y)$ grid lattice is projected onto the plane defined by the normal
  vector $(1,1)^{\top}$. This grid has $s+1=4$ and $d=2$ $(s+1)^{d}=4^2=16$ elements.
  In the projection, all points of the same color are projected onto the same points in the plane.
  The number of elements of the projected lattice is $t=(s+1)^d-s^d=4^2-3^2=7$, that is
  the $(4\times 4)$ grid minus the size of lattice that is $1$ smaller at each size, in this
  case a $(3\times 3)$ lattice (the upper right $(3\times 3)$ elements).
  }
\label{fig:latticeconstruction}
\end{figure}

The blur neighbor matrix is constructed by traversing through all the populated
lattice points and their neighboring elements.
% For efficiency, we do this matrix construction recursively with shared computations
% since $n^{th}$ neighbourhood elements are $1^{st}$ neighborhood elements of $n-1^{th}$ neighbourhood elements. \pg{do not understand}
This is done recursively to share computations. For any lattice point, the neighbors that are
$n$ hops away are the direct neighbors of the points that are $n-1$ hops away.
The size of a $d$ dimensional spatial filter with width $s+1$ is $(s+1)^{d}$ (\eg, a
$3\times 3$ filter, $s=2$ in $d=2$ has $3^2=9$ elements) and this size grows
exponentially in the number of dimensions $d$. The permutohedral lattice is constructed by
projecting a regular grid onto the plane spanned by the $d$ dimensional normal vector ${(1,\ldots,1)}^{\top}$. See
Fig.~\ref{fig:latticeconstruction} for an illustration of the 1D lattice construction.
Many corners of a grid filter are projected onto the same point, in total $t = {(s+1)}^{d} -
s^{d}$ elements remain in the permutohedral filter with $s$ neighborhood in $d-1$ dimensions.
If the lattice has $m$ populated elements, the
matrix $K$ has size $t\times m$. Note that, since the input signal is typically
sparse, only a few lattice corners are being populated in the \textit{slicing} step.
We use a hash-table to keep track of these points and traverse only through
the populated lattice points for this neighborhood matrix construction.

Once the blur neighbor matrix $K$ is constructed, we can perform the convolution
by the matrix vector multiplication
\begin{equation}
\ell' = BK,
\label{eq:conv}
\end{equation}
where $B$ is the $1 \times t$ filter kernel (whose values we will learn) and $\ell'\in\mathbb{R}^{1\times m}$
is the result of the filtering at the $m$ lattice points. In practice, we found that the
matrix $K$ is sometimes too large to fit into GPU memory and we divided the matrix $K$
into smaller pieces to compute Eq.~\ref{eq:conv} sequentially.

In the general multi-dimensional case, the signal $\ell$ is of $c$ dimensions. Then
the kernel $B$ is of size $c \times t$ and $K$ stores the $c$ dimensional vectors
accordingly. When the input and output points are different, we slice only the
input points and splat only at the output points.


\subsection{Additional Experiments}
\label{sec:addexps}
In this section, we discuss more use-cases for the learned bilateral filters, one
use-case of BNNs and two single filter applications for image and 3D mesh denoising.

\subsubsection{Recognition of subsampled MNIST}\label{sec:app_mnist}

One of the strengths of the proposed filter convolution is that it does not
require the input to lie on a regular grid. The only requirement is to define a distance
between features of the input signal.
We highlight this feature with the following experiment using the
classical MNIST ten class classification problem~\cite{lecun1998mnist}. We sample a
sparse set of $N$ points $(x,y)\in [0,1]\times [0,1]$
uniformly at random in the input image, use their interpolated values
as signal and the \emph{continuous} $(x,y)$ positions as features. This mimics
sub-sampling of a high-dimensional signal. To compare against a spatial convolution,
we interpolate the sparse set of values at the grid positions.

We take a reference implementation of LeNet~\cite{lecun1998gradient} that
is part of the Caffe project~\cite{jia2014caffe} and compare it
against the same architecture but replacing the first convolutional
layer with a bilateral convolution layer (BCL). The filter size
and numbers are adjusted to get a comparable number of parameters
($5\times 5$ for LeNet, $2$-neighborhood for BCL).

The results are shown in Table~\ref{tab:all-results}. We see that training
on the original MNIST data (column Original, LeNet vs. BNN) leads to a slight
decrease in performance of the BNN (99.03\%) compared to LeNet
(99.19\%). The BNN can be trained and evaluated on sparse
signals, and we resample the image as described above for $N=$ 100\%, 60\% and
20\% of the total number of pixels. The methods are also evaluated
on test images that are subsampled in the same way. Note that we can
train and test with different subsampling rates. We introduce an additional
bilinear interpolation layer for the LeNet architecture to train on the same
data. In essence, both models perform a spatial interpolation and thus we
expect them to yield a similar classification accuracy. Once the data is of
higher dimensions, the permutohedral convolution will be faster due to hashing
the sparse input points, as well as less memory demanding in comparison to
naive application of a spatial convolution with interpolated values.

\begin{table}[t]
  \begin{center}
    \footnotesize
    \centering
    \begin{tabular}[t]{lllll}
      \toprule
              &     & \multicolumn{3}{c}{Test Subsampling} \\
       Method  & Original & 100\% & 60\% & 20\%\\
      \midrule
       LeNet &  \textbf{0.9919} & 0.9660 & 0.9348 & \textbf{0.6434} \\
       BNN &  0.9903 & \textbf{0.9844} & \textbf{0.9534} & 0.5767 \\
      \hline
       LeNet 100\% & 0.9856 & 0.9809 & 0.9678 & \textbf{0.7386} \\
       BNN 100\% & \textbf{0.9900} & \textbf{0.9863} & \textbf{0.9699} & 0.6910 \\
      \hline
       LeNet 60\% & 0.9848 & 0.9821 & 0.9740 & 0.8151 \\
       BNN 60\% & \textbf{0.9885} & \textbf{0.9864} & \textbf{0.9771} & \textbf{0.8214}\\
      \hline
       LeNet 20\% & \textbf{0.9763} & \textbf{0.9754} & 0.9695 & 0.8928 \\
       BNN 20\% & 0.9728 & 0.9735 & \textbf{0.9701} & \textbf{0.9042}\\
      \bottomrule
    \end{tabular}
  \end{center}
\vspace{-.2cm}
\caption{Classification accuracy on MNIST. We compare the
    LeNet~\cite{lecun1998gradient} implementation that is part of
    Caffe~\cite{jia2014caffe} to the network with the first layer
    replaced by a bilateral convolution layer (BCL). Both are trained
    on the original image resolution (first two rows). Three more BNN
    and CNN models are trained with randomly subsampled images (100\%,
    60\% and 20\% of the pixels). An additional bilinear interpolation
    layer samples the input signal on a spatial grid for the CNN model.
  }
  \label{tab:all-results}
\vspace{-.5cm}
\end{table}

\subsubsection{Image Denoising}

The main application that inspired the development of the bilateral
filtering operation is image denoising~\cite{aurich1995non}, there
using a single Gaussian kernel. Our development allows to learn this
kernel function from data and we explore how to improve using a \emph{single}
but more general bilateral filter.

We use the Berkeley segmentation dataset
(BSDS500)~\cite{arbelaezi2011bsds500} as a test bed. The color
images in the dataset are converted to gray-scale,
and corrupted with Gaussian noise with a standard deviation of
$\frac {25} {255}$.

We compare the performance of four different filter models on a
denoising task.
The first baseline model (`Spatial' in Table \ref{tab:denoising}, $25$
weights) uses a single spatial filter with a kernel size of
$5$ and predicts the scalar gray-scale value at the center pixel. The next model
(`Gauss Bilateral') applies a bilateral \emph{Gaussian}
filter to the noisy input, using position and intensity features $\f=(x,y,v)^\top$.
The third setup (`Learned Bilateral', $65$ weights)
takes a Gauss kernel as initialization and
fits all filter weights on the train set to minimize the
mean squared error with respect to the clean images.
We run a combination
of spatial and permutohedral convolutions on spatial and bilateral
features (`Spatial + Bilateral (Learned)') to check for a complementary
performance of the two convolutions.

\label{sec:exp:denoising}
\begin{table}[!h]
\begin{center}
  \footnotesize
  \begin{tabular}[t]{lr}
    \toprule
    Method & PSNR \\
    \midrule
    Noisy Input & $20.17$ \\
    Spatial & $26.27$ \\
    Gauss Bilateral & $26.51$ \\
    Learned Bilateral & $26.58$ \\
    Spatial + Bilateral (Learned) & \textbf{$26.65$} \\
    \bottomrule
  \end{tabular}
\end{center}
\vspace{-0.5em}
\caption{PSNR results of a denoising task using the BSDS500
  dataset~\cite{arbelaezi2011bsds500}}
\vspace{-0.5em}
\label{tab:denoising}
\end{table}
\vspace{-0.2em}

The PSNR scores evaluated on full images of the test set are
shown in Table \ref{tab:denoising}. We find that an untrained bilateral
filter already performs better than a trained spatial convolution
($26.27$ to $26.51$). A learned convolution further improve the
performance slightly. We chose this simple one-kernel setup to
validate an advantage of the generalized bilateral filter. A competitive
denoising system would employ RGB color information and also
needs to be properly adjusted in network size. Multi-layer perceptrons
have obtained state-of-the-art denoising results~\cite{burger12cvpr}
and the permutohedral lattice layer can readily be used in such an
architecture, which is intended future work.

\subsection{Additional results}
\label{sec:addresults}

This section contains more qualitative results for the experiments presented in Chapter~\ref{chap:bnn}.

\begin{figure*}[th!]
  \centering
    \includegraphics[width=\columnwidth,trim={5cm 2.5cm 5cm 4.5cm},clip]{figures/supplementary/lattice_viz.pdf}
    \vspace{-0.7cm}
  \mycaption{Visualization of the Permutohedral Lattice}
  {Sample lattice visualizations for different feature spaces. All pixels falling in the same simplex cell are shown with
  the same color. $(x,y)$ features correspond to image pixel positions, and $(r,g,b) \in [0,255]$ correspond
  to the red, green and blue color values.}
\label{fig:latticeviz}
\end{figure*}

\subsubsection{Lattice Visualization}

Figure~\ref{fig:latticeviz} shows sample lattice visualizations for different feature spaces.

\newcolumntype{L}[1]{>{\raggedright\let\newline\\\arraybackslash\hspace{0pt}}b{#1}}
\newcolumntype{C}[1]{>{\centering\let\newline\\\arraybackslash\hspace{0pt}}b{#1}}
\newcolumntype{R}[1]{>{\raggedleft\let\newline\\\arraybackslash\hspace{0pt}}b{#1}}

\subsubsection{Color Upsampling}\label{sec:color_upsampling}
\label{sec:col_upsample_extra}

Some images of the upsampling for the Pascal
VOC12 dataset are shown in Fig.~\ref{fig:Colour_upsample_visuals}. It is
especially the low level image details that are better preserved with
a learned bilateral filter compared to the Gaussian case.

\begin{figure*}[t!]
  \centering
    \subfigure{%
   \raisebox{2.0em}{
    \includegraphics[width=.06\columnwidth]{figures/supplementary/2007_004969.jpg}
   }
  }
  \subfigure{%
    \includegraphics[width=.17\columnwidth]{figures/supplementary/2007_004969_gray.pdf}
  }
  \subfigure{%
    \includegraphics[width=.17\columnwidth]{figures/supplementary/2007_004969_gt.pdf}
  }
  \subfigure{%
    \includegraphics[width=.17\columnwidth]{figures/supplementary/2007_004969_bicubic.pdf}
  }
  \subfigure{%
    \includegraphics[width=.17\columnwidth]{figures/supplementary/2007_004969_gauss.pdf}
  }
  \subfigure{%
    \includegraphics[width=.17\columnwidth]{figures/supplementary/2007_004969_learnt.pdf}
  }\\
    \subfigure{%
   \raisebox{2.0em}{
    \includegraphics[width=.06\columnwidth]{figures/supplementary/2007_003106.jpg}
   }
  }
  \subfigure{%
    \includegraphics[width=.17\columnwidth]{figures/supplementary/2007_003106_gray.pdf}
  }
  \subfigure{%
    \includegraphics[width=.17\columnwidth]{figures/supplementary/2007_003106_gt.pdf}
  }
  \subfigure{%
    \includegraphics[width=.17\columnwidth]{figures/supplementary/2007_003106_bicubic.pdf}
  }
  \subfigure{%
    \includegraphics[width=.17\columnwidth]{figures/supplementary/2007_003106_gauss.pdf}
  }
  \subfigure{%
    \includegraphics[width=.17\columnwidth]{figures/supplementary/2007_003106_learnt.pdf}
  }\\
  \setcounter{subfigure}{0}
  \small{
  \subfigure[Inp.]{%
  \raisebox{2.0em}{
    \includegraphics[width=.06\columnwidth]{figures/supplementary/2007_006837.jpg}
   }
  }
  \subfigure[Guidance]{%
    \includegraphics[width=.17\columnwidth]{figures/supplementary/2007_006837_gray.pdf}
  }
   \subfigure[GT]{%
    \includegraphics[width=.17\columnwidth]{figures/supplementary/2007_006837_gt.pdf}
  }
  \subfigure[Bicubic]{%
    \includegraphics[width=.17\columnwidth]{figures/supplementary/2007_006837_bicubic.pdf}
  }
  \subfigure[Gauss-BF]{%
    \includegraphics[width=.17\columnwidth]{figures/supplementary/2007_006837_gauss.pdf}
  }
  \subfigure[Learned-BF]{%
    \includegraphics[width=.17\columnwidth]{figures/supplementary/2007_006837_learnt.pdf}
  }
  }
  \vspace{-0.5cm}
  \mycaption{Color Upsampling}{Color $8\times$ upsampling results
  using different methods, from left to right, (a)~Low-resolution input color image (Inp.),
  (b)~Gray scale guidance image, (c)~Ground-truth color image; Upsampled color images with
  (d)~Bicubic interpolation, (e) Gauss bilateral upsampling and, (f)~Learned bilateral
  updampgling (best viewed on screen).}

\label{fig:Colour_upsample_visuals}
\end{figure*}

\subsubsection{Depth Upsampling}
\label{sec:depth_upsample_extra}

Figure~\ref{fig:depth_upsample_visuals} presents some more qualitative results comparing bicubic interpolation, Gauss
bilateral and learned bilateral upsampling on NYU depth dataset image~\cite{silberman2012indoor}.

\subsubsection{Character Recognition}\label{sec:app_character}

 Figure~\ref{fig:nnrecognition} shows the schematic of different layers
 of the network architecture for LeNet-7~\cite{lecun1998mnist}
 and DeepCNet(5, 50)~\cite{ciresan2012multi,graham2014spatially}. For the BNN variants, the first layer filters are replaced
 with learned bilateral filters and are learned end-to-end.

\subsubsection{Semantic Segmentation}\label{sec:app_semantic_segmentation}
\label{sec:semantic_bnn_extra}

Some more visual results for semantic segmentation are shown in Figure~\ref{fig:semantic_visuals}.
These include the underlying DeepLab CNN\cite{chen2014semantic} result (DeepLab),
the 2 step mean-field result with Gaussian edge potentials (+2stepMF-GaussCRF)
and also corresponding results with learned edge potentials (+2stepMF-LearnedCRF).
In general, we observe that mean-field in learned CRF leads to slightly dilated
classification regions in comparison to using Gaussian CRF thereby filling-in the
false negative pixels and also correcting some mis-classified regions.

\begin{figure*}[t!]
  \centering
    \subfigure{%
   \raisebox{2.0em}{
    \includegraphics[width=.06\columnwidth]{figures/supplementary/2bicubic}
   }
  }
  \subfigure{%
    \includegraphics[width=.17\columnwidth]{figures/supplementary/2given_image}
  }
  \subfigure{%
    \includegraphics[width=.17\columnwidth]{figures/supplementary/2ground_truth}
  }
  \subfigure{%
    \includegraphics[width=.17\columnwidth]{figures/supplementary/2bicubic}
  }
  \subfigure{%
    \includegraphics[width=.17\columnwidth]{figures/supplementary/2gauss}
  }
  \subfigure{%
    \includegraphics[width=.17\columnwidth]{figures/supplementary/2learnt}
  }\\
    \subfigure{%
   \raisebox{2.0em}{
    \includegraphics[width=.06\columnwidth]{figures/supplementary/32bicubic}
   }
  }
  \subfigure{%
    \includegraphics[width=.17\columnwidth]{figures/supplementary/32given_image}
  }
  \subfigure{%
    \includegraphics[width=.17\columnwidth]{figures/supplementary/32ground_truth}
  }
  \subfigure{%
    \includegraphics[width=.17\columnwidth]{figures/supplementary/32bicubic}
  }
  \subfigure{%
    \includegraphics[width=.17\columnwidth]{figures/supplementary/32gauss}
  }
  \subfigure{%
    \includegraphics[width=.17\columnwidth]{figures/supplementary/32learnt}
  }\\
  \setcounter{subfigure}{0}
  \small{
  \subfigure[Inp.]{%
  \raisebox{2.0em}{
    \includegraphics[width=.06\columnwidth]{figures/supplementary/41bicubic}
   }
  }
  \subfigure[Guidance]{%
    \includegraphics[width=.17\columnwidth]{figures/supplementary/41given_image}
  }
   \subfigure[GT]{%
    \includegraphics[width=.17\columnwidth]{figures/supplementary/41ground_truth}
  }
  \subfigure[Bicubic]{%
    \includegraphics[width=.17\columnwidth]{figures/supplementary/41bicubic}
  }
  \subfigure[Gauss-BF]{%
    \includegraphics[width=.17\columnwidth]{figures/supplementary/41gauss}
  }
  \subfigure[Learned-BF]{%
    \includegraphics[width=.17\columnwidth]{figures/supplementary/41learnt}
  }
  }
  \mycaption{Depth Upsampling}{Depth $8\times$ upsampling results
  using different upsampling strategies, from left to right,
  (a)~Low-resolution input depth image (Inp.),
  (b)~High-resolution guidance image, (c)~Ground-truth depth; Upsampled depth images with
  (d)~Bicubic interpolation, (e) Gauss bilateral upsampling and, (f)~Learned bilateral
  updampgling (best viewed on screen).}

\label{fig:depth_upsample_visuals}
\end{figure*}

\subsubsection{Material Segmentation}\label{sec:app_material_segmentation}
\label{sec:material_bnn_extra}

In Fig.~\ref{fig:material_visuals-app2}, we present visual results comparing 2 step
mean-field inference with Gaussian and learned pairwise CRF potentials. In
general, we observe that the pixels belonging to dominant classes in the
training data are being more accurately classified with learned CRF. This leads to
a significant improvements in overall pixel accuracy. This also results
in a slight decrease of the accuracy from less frequent class pixels thereby
slightly reducing the average class accuracy with learning. We attribute this
to the type of annotation that is available for this dataset, which is not
for the entire image but for some segments in the image. We have very few
images of the infrequent classes to combat this behaviour during training.

\subsubsection{Experiment Protocols}
\label{sec:protocols}

Table~\ref{tbl:parameters} shows experiment protocols of different experiments.

 \begin{figure*}[t!]
  \centering
  \subfigure[LeNet-7]{
    \includegraphics[width=0.7\columnwidth]{figures/supplementary/lenet_cnn_network}
    }\\
    \subfigure[DeepCNet]{
    \includegraphics[width=\columnwidth]{figures/supplementary/deepcnet_cnn_network}
    }
  \mycaption{CNNs for Character Recognition}
  {Schematic of (top) LeNet-7~\cite{lecun1998mnist} and (bottom) DeepCNet(5,50)~\cite{ciresan2012multi,graham2014spatially} architectures used in Assamese
  character recognition experiments.}
\label{fig:nnrecognition}
\end{figure*}

\definecolor{voc_1}{RGB}{0, 0, 0}
\definecolor{voc_2}{RGB}{128, 0, 0}
\definecolor{voc_3}{RGB}{0, 128, 0}
\definecolor{voc_4}{RGB}{128, 128, 0}
\definecolor{voc_5}{RGB}{0, 0, 128}
\definecolor{voc_6}{RGB}{128, 0, 128}
\definecolor{voc_7}{RGB}{0, 128, 128}
\definecolor{voc_8}{RGB}{128, 128, 128}
\definecolor{voc_9}{RGB}{64, 0, 0}
\definecolor{voc_10}{RGB}{192, 0, 0}
\definecolor{voc_11}{RGB}{64, 128, 0}
\definecolor{voc_12}{RGB}{192, 128, 0}
\definecolor{voc_13}{RGB}{64, 0, 128}
\definecolor{voc_14}{RGB}{192, 0, 128}
\definecolor{voc_15}{RGB}{64, 128, 128}
\definecolor{voc_16}{RGB}{192, 128, 128}
\definecolor{voc_17}{RGB}{0, 64, 0}
\definecolor{voc_18}{RGB}{128, 64, 0}
\definecolor{voc_19}{RGB}{0, 192, 0}
\definecolor{voc_20}{RGB}{128, 192, 0}
\definecolor{voc_21}{RGB}{0, 64, 128}
\definecolor{voc_22}{RGB}{128, 64, 128}

\begin{figure*}[t]
  \centering
  \small{
  \fcolorbox{white}{voc_1}{\rule{0pt}{6pt}\rule{6pt}{0pt}} Background~~
  \fcolorbox{white}{voc_2}{\rule{0pt}{6pt}\rule{6pt}{0pt}} Aeroplane~~
  \fcolorbox{white}{voc_3}{\rule{0pt}{6pt}\rule{6pt}{0pt}} Bicycle~~
  \fcolorbox{white}{voc_4}{\rule{0pt}{6pt}\rule{6pt}{0pt}} Bird~~
  \fcolorbox{white}{voc_5}{\rule{0pt}{6pt}\rule{6pt}{0pt}} Boat~~
  \fcolorbox{white}{voc_6}{\rule{0pt}{6pt}\rule{6pt}{0pt}} Bottle~~
  \fcolorbox{white}{voc_7}{\rule{0pt}{6pt}\rule{6pt}{0pt}} Bus~~
  \fcolorbox{white}{voc_8}{\rule{0pt}{6pt}\rule{6pt}{0pt}} Car~~ \\
  \fcolorbox{white}{voc_9}{\rule{0pt}{6pt}\rule{6pt}{0pt}} Cat~~
  \fcolorbox{white}{voc_10}{\rule{0pt}{6pt}\rule{6pt}{0pt}} Chair~~
  \fcolorbox{white}{voc_11}{\rule{0pt}{6pt}\rule{6pt}{0pt}} Cow~~
  \fcolorbox{white}{voc_12}{\rule{0pt}{6pt}\rule{6pt}{0pt}} Dining Table~~
  \fcolorbox{white}{voc_13}{\rule{0pt}{6pt}\rule{6pt}{0pt}} Dog~~
  \fcolorbox{white}{voc_14}{\rule{0pt}{6pt}\rule{6pt}{0pt}} Horse~~
  \fcolorbox{white}{voc_15}{\rule{0pt}{6pt}\rule{6pt}{0pt}} Motorbike~~
  \fcolorbox{white}{voc_16}{\rule{0pt}{6pt}\rule{6pt}{0pt}} Person~~ \\
  \fcolorbox{white}{voc_17}{\rule{0pt}{6pt}\rule{6pt}{0pt}} Potted Plant~~
  \fcolorbox{white}{voc_18}{\rule{0pt}{6pt}\rule{6pt}{0pt}} Sheep~~
  \fcolorbox{white}{voc_19}{\rule{0pt}{6pt}\rule{6pt}{0pt}} Sofa~~
  \fcolorbox{white}{voc_20}{\rule{0pt}{6pt}\rule{6pt}{0pt}} Train~~
  \fcolorbox{white}{voc_21}{\rule{0pt}{6pt}\rule{6pt}{0pt}} TV monitor~~ \\
  }
  \subfigure{%
    \includegraphics[width=.18\columnwidth]{figures/supplementary/2007_001423_given.jpg}
  }
  \subfigure{%
    \includegraphics[width=.18\columnwidth]{figures/supplementary/2007_001423_gt.png}
  }
  \subfigure{%
    \includegraphics[width=.18\columnwidth]{figures/supplementary/2007_001423_cnn.png}
  }
  \subfigure{%
    \includegraphics[width=.18\columnwidth]{figures/supplementary/2007_001423_gauss.png}
  }
  \subfigure{%
    \includegraphics[width=.18\columnwidth]{figures/supplementary/2007_001423_learnt.png}
  }\\
  \subfigure{%
    \includegraphics[width=.18\columnwidth]{figures/supplementary/2007_001430_given.jpg}
  }
  \subfigure{%
    \includegraphics[width=.18\columnwidth]{figures/supplementary/2007_001430_gt.png}
  }
  \subfigure{%
    \includegraphics[width=.18\columnwidth]{figures/supplementary/2007_001430_cnn.png}
  }
  \subfigure{%
    \includegraphics[width=.18\columnwidth]{figures/supplementary/2007_001430_gauss.png}
  }
  \subfigure{%
    \includegraphics[width=.18\columnwidth]{figures/supplementary/2007_001430_learnt.png}
  }\\
    \subfigure{%
    \includegraphics[width=.18\columnwidth]{figures/supplementary/2007_007996_given.jpg}
  }
  \subfigure{%
    \includegraphics[width=.18\columnwidth]{figures/supplementary/2007_007996_gt.png}
  }
  \subfigure{%
    \includegraphics[width=.18\columnwidth]{figures/supplementary/2007_007996_cnn.png}
  }
  \subfigure{%
    \includegraphics[width=.18\columnwidth]{figures/supplementary/2007_007996_gauss.png}
  }
  \subfigure{%
    \includegraphics[width=.18\columnwidth]{figures/supplementary/2007_007996_learnt.png}
  }\\
   \subfigure{%
    \includegraphics[width=.18\columnwidth]{figures/supplementary/2010_002682_given.jpg}
  }
  \subfigure{%
    \includegraphics[width=.18\columnwidth]{figures/supplementary/2010_002682_gt.png}
  }
  \subfigure{%
    \includegraphics[width=.18\columnwidth]{figures/supplementary/2010_002682_cnn.png}
  }
  \subfigure{%
    \includegraphics[width=.18\columnwidth]{figures/supplementary/2010_002682_gauss.png}
  }
  \subfigure{%
    \includegraphics[width=.18\columnwidth]{figures/supplementary/2010_002682_learnt.png}
  }\\
     \subfigure{%
    \includegraphics[width=.18\columnwidth]{figures/supplementary/2010_004789_given.jpg}
  }
  \subfigure{%
    \includegraphics[width=.18\columnwidth]{figures/supplementary/2010_004789_gt.png}
  }
  \subfigure{%
    \includegraphics[width=.18\columnwidth]{figures/supplementary/2010_004789_cnn.png}
  }
  \subfigure{%
    \includegraphics[width=.18\columnwidth]{figures/supplementary/2010_004789_gauss.png}
  }
  \subfigure{%
    \includegraphics[width=.18\columnwidth]{figures/supplementary/2010_004789_learnt.png}
  }\\
       \subfigure{%
    \includegraphics[width=.18\columnwidth]{figures/supplementary/2007_001311_given.jpg}
  }
  \subfigure{%
    \includegraphics[width=.18\columnwidth]{figures/supplementary/2007_001311_gt.png}
  }
  \subfigure{%
    \includegraphics[width=.18\columnwidth]{figures/supplementary/2007_001311_cnn.png}
  }
  \subfigure{%
    \includegraphics[width=.18\columnwidth]{figures/supplementary/2007_001311_gauss.png}
  }
  \subfigure{%
    \includegraphics[width=.18\columnwidth]{figures/supplementary/2007_001311_learnt.png}
  }\\
  \setcounter{subfigure}{0}
  \subfigure[Input]{%
    \includegraphics[width=.18\columnwidth]{figures/supplementary/2010_003531_given.jpg}
  }
  \subfigure[Ground Truth]{%
    \includegraphics[width=.18\columnwidth]{figures/supplementary/2010_003531_gt.png}
  }
  \subfigure[DeepLab]{%
    \includegraphics[width=.18\columnwidth]{figures/supplementary/2010_003531_cnn.png}
  }
  \subfigure[+GaussCRF]{%
    \includegraphics[width=.18\columnwidth]{figures/supplementary/2010_003531_gauss.png}
  }
  \subfigure[+LearnedCRF]{%
    \includegraphics[width=.18\columnwidth]{figures/supplementary/2010_003531_learnt.png}
  }
  \vspace{-0.3cm}
  \mycaption{Semantic Segmentation}{Example results of semantic segmentation.
  (c)~depicts the unary results before application of MF, (d)~after two steps of MF with Gaussian edge CRF potentials, (e)~after
  two steps of MF with learned edge CRF potentials.}
    \label{fig:semantic_visuals}
\end{figure*}


\definecolor{minc_1}{HTML}{771111}
\definecolor{minc_2}{HTML}{CAC690}
\definecolor{minc_3}{HTML}{EEEEEE}
\definecolor{minc_4}{HTML}{7C8FA6}
\definecolor{minc_5}{HTML}{597D31}
\definecolor{minc_6}{HTML}{104410}
\definecolor{minc_7}{HTML}{BB819C}
\definecolor{minc_8}{HTML}{D0CE48}
\definecolor{minc_9}{HTML}{622745}
\definecolor{minc_10}{HTML}{666666}
\definecolor{minc_11}{HTML}{D54A31}
\definecolor{minc_12}{HTML}{101044}
\definecolor{minc_13}{HTML}{444126}
\definecolor{minc_14}{HTML}{75D646}
\definecolor{minc_15}{HTML}{DD4348}
\definecolor{minc_16}{HTML}{5C8577}
\definecolor{minc_17}{HTML}{C78472}
\definecolor{minc_18}{HTML}{75D6D0}
\definecolor{minc_19}{HTML}{5B4586}
\definecolor{minc_20}{HTML}{C04393}
\definecolor{minc_21}{HTML}{D69948}
\definecolor{minc_22}{HTML}{7370D8}
\definecolor{minc_23}{HTML}{7A3622}
\definecolor{minc_24}{HTML}{000000}

\begin{figure*}[t]
  \centering
  \small{
  \fcolorbox{white}{minc_1}{\rule{0pt}{6pt}\rule{6pt}{0pt}} Brick~~
  \fcolorbox{white}{minc_2}{\rule{0pt}{6pt}\rule{6pt}{0pt}} Carpet~~
  \fcolorbox{white}{minc_3}{\rule{0pt}{6pt}\rule{6pt}{0pt}} Ceramic~~
  \fcolorbox{white}{minc_4}{\rule{0pt}{6pt}\rule{6pt}{0pt}} Fabric~~
  \fcolorbox{white}{minc_5}{\rule{0pt}{6pt}\rule{6pt}{0pt}} Foliage~~
  \fcolorbox{white}{minc_6}{\rule{0pt}{6pt}\rule{6pt}{0pt}} Food~~
  \fcolorbox{white}{minc_7}{\rule{0pt}{6pt}\rule{6pt}{0pt}} Glass~~
  \fcolorbox{white}{minc_8}{\rule{0pt}{6pt}\rule{6pt}{0pt}} Hair~~ \\
  \fcolorbox{white}{minc_9}{\rule{0pt}{6pt}\rule{6pt}{0pt}} Leather~~
  \fcolorbox{white}{minc_10}{\rule{0pt}{6pt}\rule{6pt}{0pt}} Metal~~
  \fcolorbox{white}{minc_11}{\rule{0pt}{6pt}\rule{6pt}{0pt}} Mirror~~
  \fcolorbox{white}{minc_12}{\rule{0pt}{6pt}\rule{6pt}{0pt}} Other~~
  \fcolorbox{white}{minc_13}{\rule{0pt}{6pt}\rule{6pt}{0pt}} Painted~~
  \fcolorbox{white}{minc_14}{\rule{0pt}{6pt}\rule{6pt}{0pt}} Paper~~
  \fcolorbox{white}{minc_15}{\rule{0pt}{6pt}\rule{6pt}{0pt}} Plastic~~\\
  \fcolorbox{white}{minc_16}{\rule{0pt}{6pt}\rule{6pt}{0pt}} Polished Stone~~
  \fcolorbox{white}{minc_17}{\rule{0pt}{6pt}\rule{6pt}{0pt}} Skin~~
  \fcolorbox{white}{minc_18}{\rule{0pt}{6pt}\rule{6pt}{0pt}} Sky~~
  \fcolorbox{white}{minc_19}{\rule{0pt}{6pt}\rule{6pt}{0pt}} Stone~~
  \fcolorbox{white}{minc_20}{\rule{0pt}{6pt}\rule{6pt}{0pt}} Tile~~
  \fcolorbox{white}{minc_21}{\rule{0pt}{6pt}\rule{6pt}{0pt}} Wallpaper~~
  \fcolorbox{white}{minc_22}{\rule{0pt}{6pt}\rule{6pt}{0pt}} Water~~
  \fcolorbox{white}{minc_23}{\rule{0pt}{6pt}\rule{6pt}{0pt}} Wood~~ \\
  }
  \subfigure{%
    \includegraphics[width=.18\columnwidth]{figures/supplementary/000010868_given.jpg}
  }
  \subfigure{%
    \includegraphics[width=.18\columnwidth]{figures/supplementary/000010868_gt.png}
  }
  \subfigure{%
    \includegraphics[width=.18\columnwidth]{figures/supplementary/000010868_cnn.png}
  }
  \subfigure{%
    \includegraphics[width=.18\columnwidth]{figures/supplementary/000010868_gauss.png}
  }
  \subfigure{%
    \includegraphics[width=.18\columnwidth]{figures/supplementary/000010868_learnt.png}
  }\\[-2ex]
  \subfigure{%
    \includegraphics[width=.18\columnwidth]{figures/supplementary/000006011_given.jpg}
  }
  \subfigure{%
    \includegraphics[width=.18\columnwidth]{figures/supplementary/000006011_gt.png}
  }
  \subfigure{%
    \includegraphics[width=.18\columnwidth]{figures/supplementary/000006011_cnn.png}
  }
  \subfigure{%
    \includegraphics[width=.18\columnwidth]{figures/supplementary/000006011_gauss.png}
  }
  \subfigure{%
    \includegraphics[width=.18\columnwidth]{figures/supplementary/000006011_learnt.png}
  }\\[-2ex]
    \subfigure{%
    \includegraphics[width=.18\columnwidth]{figures/supplementary/000008553_given.jpg}
  }
  \subfigure{%
    \includegraphics[width=.18\columnwidth]{figures/supplementary/000008553_gt.png}
  }
  \subfigure{%
    \includegraphics[width=.18\columnwidth]{figures/supplementary/000008553_cnn.png}
  }
  \subfigure{%
    \includegraphics[width=.18\columnwidth]{figures/supplementary/000008553_gauss.png}
  }
  \subfigure{%
    \includegraphics[width=.18\columnwidth]{figures/supplementary/000008553_learnt.png}
  }\\[-2ex]
   \subfigure{%
    \includegraphics[width=.18\columnwidth]{figures/supplementary/000009188_given.jpg}
  }
  \subfigure{%
    \includegraphics[width=.18\columnwidth]{figures/supplementary/000009188_gt.png}
  }
  \subfigure{%
    \includegraphics[width=.18\columnwidth]{figures/supplementary/000009188_cnn.png}
  }
  \subfigure{%
    \includegraphics[width=.18\columnwidth]{figures/supplementary/000009188_gauss.png}
  }
  \subfigure{%
    \includegraphics[width=.18\columnwidth]{figures/supplementary/000009188_learnt.png}
  }\\[-2ex]
  \setcounter{subfigure}{0}
  \subfigure[Input]{%
    \includegraphics[width=.18\columnwidth]{figures/supplementary/000023570_given.jpg}
  }
  \subfigure[Ground Truth]{%
    \includegraphics[width=.18\columnwidth]{figures/supplementary/000023570_gt.png}
  }
  \subfigure[DeepLab]{%
    \includegraphics[width=.18\columnwidth]{figures/supplementary/000023570_cnn.png}
  }
  \subfigure[+GaussCRF]{%
    \includegraphics[width=.18\columnwidth]{figures/supplementary/000023570_gauss.png}
  }
  \subfigure[+LearnedCRF]{%
    \includegraphics[width=.18\columnwidth]{figures/supplementary/000023570_learnt.png}
  }
  \mycaption{Material Segmentation}{Example results of material segmentation.
  (c)~depicts the unary results before application of MF, (d)~after two steps of MF with Gaussian edge CRF potentials, (e)~after two steps of MF with learned edge CRF potentials.}
    \label{fig:material_visuals-app2}
\end{figure*}


\begin{table*}[h]
\tiny
  \centering
    \begin{tabular}{L{2.3cm} L{2.25cm} C{1.5cm} C{0.7cm} C{0.6cm} C{0.7cm} C{0.7cm} C{0.7cm} C{1.6cm} C{0.6cm} C{0.6cm} C{0.6cm}}
      \toprule
& & & & & \multicolumn{3}{c}{\textbf{Data Statistics}} & \multicolumn{4}{c}{\textbf{Training Protocol}} \\

\textbf{Experiment} & \textbf{Feature Types} & \textbf{Feature Scales} & \textbf{Filter Size} & \textbf{Filter Nbr.} & \textbf{Train}  & \textbf{Val.} & \textbf{Test} & \textbf{Loss Type} & \textbf{LR} & \textbf{Batch} & \textbf{Epochs} \\
      \midrule
      \multicolumn{2}{c}{\textbf{Single Bilateral Filter Applications}} & & & & & & & & & \\
      \textbf{2$\times$ Color Upsampling} & Position$_{1}$, Intensity (3D) & 0.13, 0.17 & 65 & 2 & 10581 & 1449 & 1456 & MSE & 1e-06 & 200 & 94.5\\
      \textbf{4$\times$ Color Upsampling} & Position$_{1}$, Intensity (3D) & 0.06, 0.17 & 65 & 2 & 10581 & 1449 & 1456 & MSE & 1e-06 & 200 & 94.5\\
      \textbf{8$\times$ Color Upsampling} & Position$_{1}$, Intensity (3D) & 0.03, 0.17 & 65 & 2 & 10581 & 1449 & 1456 & MSE & 1e-06 & 200 & 94.5\\
      \textbf{16$\times$ Color Upsampling} & Position$_{1}$, Intensity (3D) & 0.02, 0.17 & 65 & 2 & 10581 & 1449 & 1456 & MSE & 1e-06 & 200 & 94.5\\
      \textbf{Depth Upsampling} & Position$_{1}$, Color (5D) & 0.05, 0.02 & 665 & 2 & 795 & 100 & 654 & MSE & 1e-07 & 50 & 251.6\\
      \textbf{Mesh Denoising} & Isomap (4D) & 46.00 & 63 & 2 & 1000 & 200 & 500 & MSE & 100 & 10 & 100.0 \\
      \midrule
      \multicolumn{2}{c}{\textbf{DenseCRF Applications}} & & & & & & & & &\\
      \multicolumn{2}{l}{\textbf{Semantic Segmentation}} & & & & & & & & &\\
      \textbf{- 1step MF} & Position$_{1}$, Color (5D); Position$_{1}$ (2D) & 0.01, 0.34; 0.34  & 665; 19  & 2; 2 & 10581 & 1449 & 1456 & Logistic & 0.1 & 5 & 1.4 \\
      \textbf{- 2step MF} & Position$_{1}$, Color (5D); Position$_{1}$ (2D) & 0.01, 0.34; 0.34 & 665; 19 & 2; 2 & 10581 & 1449 & 1456 & Logistic & 0.1 & 5 & 1.4 \\
      \textbf{- \textit{loose} 2step MF} & Position$_{1}$, Color (5D); Position$_{1}$ (2D) & 0.01, 0.34; 0.34 & 665; 19 & 2; 2 &10581 & 1449 & 1456 & Logistic & 0.1 & 5 & +1.9  \\ \\
      \multicolumn{2}{l}{\textbf{Material Segmentation}} & & & & & & & & &\\
      \textbf{- 1step MF} & Position$_{2}$, Lab-Color (5D) & 5.00, 0.05, 0.30  & 665 & 2 & 928 & 150 & 1798 & Weighted Logistic & 1e-04 & 24 & 2.6 \\
      \textbf{- 2step MF} & Position$_{2}$, Lab-Color (5D) & 5.00, 0.05, 0.30 & 665 & 2 & 928 & 150 & 1798 & Weighted Logistic & 1e-04 & 12 & +0.7 \\
      \textbf{- \textit{loose} 2step MF} & Position$_{2}$, Lab-Color (5D) & 5.00, 0.05, 0.30 & 665 & 2 & 928 & 150 & 1798 & Weighted Logistic & 1e-04 & 12 & +0.2\\
      \midrule
      \multicolumn{2}{c}{\textbf{Neural Network Applications}} & & & & & & & & &\\
      \textbf{Tiles: CNN-9$\times$9} & - & - & 81 & 4 & 10000 & 1000 & 1000 & Logistic & 0.01 & 100 & 500.0 \\
      \textbf{Tiles: CNN-13$\times$13} & - & - & 169 & 6 & 10000 & 1000 & 1000 & Logistic & 0.01 & 100 & 500.0 \\
      \textbf{Tiles: CNN-17$\times$17} & - & - & 289 & 8 & 10000 & 1000 & 1000 & Logistic & 0.01 & 100 & 500.0 \\
      \textbf{Tiles: CNN-21$\times$21} & - & - & 441 & 10 & 10000 & 1000 & 1000 & Logistic & 0.01 & 100 & 500.0 \\
      \textbf{Tiles: BNN} & Position$_{1}$, Color (5D) & 0.05, 0.04 & 63 & 1 & 10000 & 1000 & 1000 & Logistic & 0.01 & 100 & 30.0 \\
      \textbf{LeNet} & - & - & 25 & 2 & 5490 & 1098 & 1647 & Logistic & 0.1 & 100 & 182.2 \\
      \textbf{Crop-LeNet} & - & - & 25 & 2 & 5490 & 1098 & 1647 & Logistic & 0.1 & 100 & 182.2 \\
      \textbf{BNN-LeNet} & Position$_{2}$ (2D) & 20.00 & 7 & 1 & 5490 & 1098 & 1647 & Logistic & 0.1 & 100 & 182.2 \\
      \textbf{DeepCNet} & - & - & 9 & 1 & 5490 & 1098 & 1647 & Logistic & 0.1 & 100 & 182.2 \\
      \textbf{Crop-DeepCNet} & - & - & 9 & 1 & 5490 & 1098 & 1647 & Logistic & 0.1 & 100 & 182.2 \\
      \textbf{BNN-DeepCNet} & Position$_{2}$ (2D) & 40.00  & 7 & 1 & 5490 & 1098 & 1647 & Logistic & 0.1 & 100 & 182.2 \\
      \bottomrule
      \\
    \end{tabular}
    \mycaption{Experiment Protocols} {Experiment protocols for the different experiments presented in this work. \textbf{Feature Types}:
    Feature spaces used for the bilateral convolutions. Position$_1$ corresponds to un-normalized pixel positions whereas Position$_2$ corresponds
    to pixel positions normalized to $[0,1]$ with respect to the given image. \textbf{Feature Scales}: Cross-validated scales for the features used.
     \textbf{Filter Size}: Number of elements in the filter that is being learned. \textbf{Filter Nbr.}: Half-width of the filter. \textbf{Train},
     \textbf{Val.} and \textbf{Test} corresponds to the number of train, validation and test images used in the experiment. \textbf{Loss Type}: Type
     of loss used for back-propagation. ``MSE'' corresponds to Euclidean mean squared error loss and ``Logistic'' corresponds to multinomial logistic
     loss. ``Weighted Logistic'' is the class-weighted multinomial logistic loss. We weighted the loss with inverse class probability for material
     segmentation task due to the small availability of training data with class imbalance. \textbf{LR}: Fixed learning rate used in stochastic gradient
     descent. \textbf{Batch}: Number of images used in one parameter update step. \textbf{Epochs}: Number of training epochs. In all the experiments,
     we used fixed momentum of 0.9 and weight decay of 0.0005 for stochastic gradient descent. ```Color Upsampling'' experiments in this Table corresponds
     to those performed on Pascal VOC12 dataset images. For all experiments using Pascal VOC12 images, we use extended
     training segmentation dataset available from~\cite{hariharan2011moredata}, and used standard validation and test splits
     from the main dataset~\cite{voc2012segmentation}.}
  \label{tbl:parameters}
\end{table*}

\clearpage

\section{Parameters and Additional Results for Video Propagation Networks}

In this Section, we present experiment protocols and additional qualitative results for experiments
on video object segmentation, semantic video segmentation and video color
propagation. Table~\ref{tbl:parameters_supp} shows the feature scales and other parameters used in different experiments.
Figures~\ref{fig:video_seg_pos_supp} show some qualitative results on video object segmentation
with some failure cases in Fig.~\ref{fig:video_seg_neg_supp}.
Figure~\ref{fig:semantic_visuals_supp} shows some qualitative results on semantic video segmentation and
Fig.~\ref{fig:color_visuals_supp} shows results on video color propagation.

\newcolumntype{L}[1]{>{\raggedright\let\newline\\\arraybackslash\hspace{0pt}}b{#1}}
\newcolumntype{C}[1]{>{\centering\let\newline\\\arraybackslash\hspace{0pt}}b{#1}}
\newcolumntype{R}[1]{>{\raggedleft\let\newline\\\arraybackslash\hspace{0pt}}b{#1}}

\begin{table*}[h]
\tiny
  \centering
    \begin{tabular}{L{3.0cm} L{2.4cm} L{2.8cm} L{2.8cm} C{0.5cm} C{1.0cm} L{1.2cm}}
      \toprule
\textbf{Experiment} & \textbf{Feature Type} & \textbf{Feature Scale-1, $\Lambda_a$} & \textbf{Feature Scale-2, $\Lambda_b$} & \textbf{$\alpha$} & \textbf{Input Frames} & \textbf{Loss Type} \\
      \midrule
      \textbf{Video Object Segmentation} & ($x,y,Y,Cb,Cr,t$) & (0.02,0.02,0.07,0.4,0.4,0.01) & (0.03,0.03,0.09,0.5,0.5,0.2) & 0.5 & 9 & Logistic\\
      \midrule
      \textbf{Semantic Video Segmentation} & & & & & \\
      \textbf{with CNN1~\cite{yu2015multi}-NoFlow} & ($x,y,R,G,B,t$) & (0.08,0.08,0.2,0.2,0.2,0.04) & (0.11,0.11,0.2,0.2,0.2,0.04) & 0.5 & 3 & Logistic \\
      \textbf{with CNN1~\cite{yu2015multi}-Flow} & ($x+u_x,y+u_y,R,G,B,t$) & (0.11,0.11,0.14,0.14,0.14,0.03) & (0.08,0.08,0.12,0.12,0.12,0.01) & 0.65 & 3 & Logistic\\
      \textbf{with CNN2~\cite{richter2016playing}-Flow} & ($x+u_x,y+u_y,R,G,B,t$) & (0.08,0.08,0.2,0.2,0.2,0.04) & (0.09,0.09,0.25,0.25,0.25,0.03) & 0.5 & 4 & Logistic\\
      \midrule
      \textbf{Video Color Propagation} & ($x,y,I,t$)  & (0.04,0.04,0.2,0.04) & No second kernel & 1 & 4 & MSE\\
      \bottomrule
      \\
    \end{tabular}
    \mycaption{Experiment Protocols} {Experiment protocols for the different experiments presented in this work. \textbf{Feature Types}:
    Feature spaces used for the bilateral convolutions, with position ($x,y$) and color
    ($R,G,B$ or $Y,Cb,Cr$) features $\in [0,255]$. $u_x$, $u_y$ denotes optical flow with respect
    to the present frame and $I$ denotes grayscale intensity.
    \textbf{Feature Scales ($\Lambda_a, \Lambda_b$)}: Cross-validated scales for the features used.
    \textbf{$\alpha$}: Exponential time decay for the input frames.
    \textbf{Input Frames}: Number of input frames for VPN.
    \textbf{Loss Type}: Type
     of loss used for back-propagation. ``MSE'' corresponds to Euclidean mean squared error loss and ``Logistic'' corresponds to multinomial logistic loss.}
  \label{tbl:parameters_supp}
\end{table*}

% \begin{figure}[th!]
% \begin{center}
%   \centerline{\includegraphics[width=\textwidth]{figures/video_seg_visuals_supp_small.pdf}}
%     \mycaption{Video Object Segmentation}
%     {Shown are the different frames in example videos with the corresponding
%     ground truth (GT) masks, predictions from BVS~\cite{marki2016bilateral},
%     OFL~\cite{tsaivideo}, VPN (VPN-Stage2) and VPN-DLab (VPN-DeepLab) models.}
%     \label{fig:video_seg_small_supp}
% \end{center}
% \vspace{-1.0cm}
% \end{figure}

\begin{figure}[th!]
\begin{center}
  \centerline{\includegraphics[width=0.7\textwidth]{figures/video_seg_visuals_supp_positive.pdf}}
    \mycaption{Video Object Segmentation}
    {Shown are the different frames in example videos with the corresponding
    ground truth (GT) masks, predictions from BVS~\cite{marki2016bilateral},
    OFL~\cite{tsaivideo}, VPN (VPN-Stage2) and VPN-DLab (VPN-DeepLab) models.}
    \label{fig:video_seg_pos_supp}
\end{center}
\vspace{-1.0cm}
\end{figure}

\begin{figure}[th!]
\begin{center}
  \centerline{\includegraphics[width=0.7\textwidth]{figures/video_seg_visuals_supp_negative.pdf}}
    \mycaption{Failure Cases for Video Object Segmentation}
    {Shown are the different frames in example videos with the corresponding
    ground truth (GT) masks, predictions from BVS~\cite{marki2016bilateral},
    OFL~\cite{tsaivideo}, VPN (VPN-Stage2) and VPN-DLab (VPN-DeepLab) models.}
    \label{fig:video_seg_neg_supp}
\end{center}
\vspace{-1.0cm}
\end{figure}

\begin{figure}[th!]
\begin{center}
  \centerline{\includegraphics[width=0.9\textwidth]{figures/supp_semantic_visual.pdf}}
    \mycaption{Semantic Video Segmentation}
    {Input video frames and the corresponding ground truth (GT)
    segmentation together with the predictions of CNN~\cite{yu2015multi} and with
    VPN-Flow.}
    \label{fig:semantic_visuals_supp}
\end{center}
\vspace{-0.7cm}
\end{figure}

\begin{figure}[th!]
\begin{center}
  \centerline{\includegraphics[width=\textwidth]{figures/colorization_visuals_supp.pdf}}
  \mycaption{Video Color Propagation}
  {Input grayscale video frames and corresponding ground-truth (GT) color images
  together with color predictions of Levin et al.~\cite{levin2004colorization} and VPN-Stage1 models.}
  \label{fig:color_visuals_supp}
\end{center}
\vspace{-0.7cm}
\end{figure}

\clearpage

\section{Additional Material for Bilateral Inception Networks}
\label{sec:binception-app}

In this section of the Appendix, we first discuss the use of approximate bilateral
filtering in BI modules (Sec.~\ref{sec:lattice}).
Later, we present some qualitative results using different models for the approach presented in
Chapter~\ref{chap:binception} (Sec.~\ref{sec:qualitative-app}).

\subsection{Approximate Bilateral Filtering}
\label{sec:lattice}

The bilateral inception module presented in Chapter~\ref{chap:binception} computes a matrix-vector
product between a Gaussian filter $K$ and a vector of activations $\bz_c$.
Bilateral filtering is an important operation and many algorithmic techniques have been
proposed to speed-up this operation~\cite{paris2006fast,adams2010fast,gastal2011domain}.
In the main paper we opted to implement what can be considered the
brute-force variant of explicitly constructing $K$ and then using BLAS to compute the
matrix-vector product. This resulted in a few millisecond operation.
The explicit way to compute is possible due to the
reduction to super-pixels, e.g., it would not work for DenseCRF variants
that operate on the full image resolution.

Here, we present experiments where we use the fast approximate bilateral filtering
algorithm of~\cite{adams2010fast}, which is also used in Chapter~\ref{chap:bnn}
for learning sparse high dimensional filters. This
choice allows for larger dimensions of matrix-vector multiplication. The reason for choosing
the explicit multiplication in Chapter~\ref{chap:binception} was that it was computationally faster.
For the small sizes of the involved matrices and vectors, the explicit computation is sufficient and we had no
GPU implementation of an approximate technique that matched this runtime. Also it
is conceptually easier and the gradient to the feature transformations ($\Lambda \mathbf{f}$) is
obtained using standard matrix calculus.

\subsubsection{Experiments}

We modified the existing segmentation architectures analogous to those in Chapter~\ref{chap:binception}.
The main difference is that, here, the inception modules use the lattice
approximation~\cite{adams2010fast} to compute the bilateral filtering.
Using the lattice approximation did not allow us to back-propagate through feature transformations ($\Lambda$)
and thus we used hand-specified feature scales as will be explained later.
Specifically, we take CNN architectures from the works
of~\cite{chen2014semantic,zheng2015conditional,bell2015minc} and insert the BI modules between
the spatial FC layers.
We use superpixels from~\cite{DollarICCV13edges}
for all the experiments with the lattice approximation. Experiments are
performed using Caffe neural network framework~\cite{jia2014caffe}.

\begin{table}
  \small
  \centering
  \begin{tabular}{p{5.5cm}>{\raggedright\arraybackslash}p{1.4cm}>{\centering\arraybackslash}p{2.2cm}}
    \toprule
		\textbf{Model} & \emph{IoU} & \emph{Runtime}(ms) \\
    \midrule

    %%%%%%%%%%%% Scores computed by us)%%%%%%%%%%%%
		\deeplablargefov & 68.9 & 145ms\\
    \midrule
    \bi{7}{2}-\bi{8}{10}& \textbf{73.8} & +600 \\
    \midrule
    \deeplablargefovcrf~\cite{chen2014semantic} & 72.7 & +830\\
    \deeplabmsclargefovcrf~\cite{chen2014semantic} & \textbf{73.6} & +880\\
    DeepLab-EdgeNet~\cite{chen2015semantic} & 71.7 & +30\\
    DeepLab-EdgeNet-CRF~\cite{chen2015semantic} & \textbf{73.6} & +860\\
  \bottomrule \\
  \end{tabular}
  \mycaption{Semantic Segmentation using the DeepLab model}
  {IoU scores on the Pascal VOC12 segmentation test dataset
  with different models and our modified inception model.
  Also shown are the corresponding runtimes in milliseconds. Runtimes
  also include superpixel computations (300 ms with Dollar superpixels~\cite{DollarICCV13edges})}
  \label{tab:largefovresults}
\end{table}

\paragraph{Semantic Segmentation}
The experiments in this section use the Pascal VOC12 segmentation dataset~\cite{voc2012segmentation} with 21 object classes and the images have a maximum resolution of 0.25 megapixels.
For all experiments on VOC12, we train using the extended training set of
10581 images collected by~\cite{hariharan2011moredata}.
We modified the \deeplab~network architecture of~\cite{chen2014semantic} and
the CRFasRNN architecture from~\cite{zheng2015conditional} which uses a CNN with
deconvolution layers followed by DenseCRF trained end-to-end.

\paragraph{DeepLab Model}\label{sec:deeplabmodel}
We experimented with the \bi{7}{2}-\bi{8}{10} inception model.
Results using the~\deeplab~model are summarized in Tab.~\ref{tab:largefovresults}.
Although we get similar improvements with inception modules as with the
explicit kernel computation, using lattice approximation is slower.

\begin{table}
  \small
  \centering
  \begin{tabular}{p{6.4cm}>{\raggedright\arraybackslash}p{1.8cm}>{\raggedright\arraybackslash}p{1.8cm}}
    \toprule
    \textbf{Model} & \emph{IoU (Val)} & \emph{IoU (Test)}\\
    \midrule
    %%%%%%%%%%%% Scores computed by us)%%%%%%%%%%%%
    CNN &  67.5 & - \\
    \deconv (CNN+Deconvolutions) & 69.8 & 72.0 \\
    \midrule
    \bi{3}{6}-\bi{4}{6}-\bi{7}{2}-\bi{8}{6}& 71.9 & - \\
    \bi{3}{6}-\bi{4}{6}-\bi{7}{2}-\bi{8}{6}-\gi{6}& 73.6 &  \href{http://host.robots.ox.ac.uk:8080/anonymous/VOTV5E.html}{\textbf{75.2}}\\
    \midrule
    \deconvcrf (CRF-RNN)~\cite{zheng2015conditional} & 73.0 & 74.7\\
    Context-CRF-RNN~\cite{yu2015multi} & ~~ - ~ & \textbf{75.3} \\
    \bottomrule \\
  \end{tabular}
  \mycaption{Semantic Segmentation using the CRFasRNN model}{IoU score corresponding to different models
  on Pascal VOC12 reduced validation / test segmentation dataset. The reduced validation set consists of 346 images
  as used in~\cite{zheng2015conditional} where we adapted the model from.}
  \label{tab:deconvresults-app}
\end{table}

\paragraph{CRFasRNN Model}\label{sec:deepinception}
We add BI modules after score-pool3, score-pool4, \fc{7} and \fc{8} $1\times1$ convolution layers
resulting in the \bi{3}{6}-\bi{4}{6}-\bi{7}{2}-\bi{8}{6}
model and also experimented with another variant where $BI_8$ is followed by another inception
module, G$(6)$, with 6 Gaussian kernels.
Note that here also we discarded both deconvolution and DenseCRF parts of the original model~\cite{zheng2015conditional}
and inserted the BI modules in the base CNN and found similar improvements compared to the inception modules with explicit
kernel computaion. See Tab.~\ref{tab:deconvresults-app} for results on the CRFasRNN model.

\paragraph{Material Segmentation}
Table~\ref{tab:mincresults-app} shows the results on the MINC dataset~\cite{bell2015minc}
obtained by modifying the AlexNet architecture with our inception modules. We observe
similar improvements as with explicit kernel construction.
For this model, we do not provide any learned setup due to very limited segment training
data. The weights to combine outputs in the bilateral inception layer are
found by validation on the validation set.

\begin{table}[t]
  \small
  \centering
  \begin{tabular}{p{3.5cm}>{\centering\arraybackslash}p{4.0cm}}
    \toprule
    \textbf{Model} & Class / Total accuracy\\
    \midrule

    %%%%%%%%%%%% Scores computed by us)%%%%%%%%%%%%
    AlexNet CNN & 55.3 / 58.9 \\
    \midrule
    \bi{7}{2}-\bi{8}{6}& 68.5 / 71.8 \\
    \bi{7}{2}-\bi{8}{6}-G$(6)$& 67.6 / 73.1 \\
    \midrule
    AlexNet-CRF & 65.5 / 71.0 \\
    \bottomrule \\
  \end{tabular}
  \mycaption{Material Segmentation using AlexNet}{Pixel accuracy of different models on
  the MINC material segmentation test dataset~\cite{bell2015minc}.}
  \label{tab:mincresults-app}
\end{table}

\paragraph{Scales of Bilateral Inception Modules}
\label{sec:scales}

Unlike the explicit kernel technique presented in the main text (Chapter~\ref{chap:binception}),
we didn't back-propagate through feature transformation ($\Lambda$)
using the approximate bilateral filter technique.
So, the feature scales are hand-specified and validated, which are as follows.
The optimal scale values for the \bi{7}{2}-\bi{8}{2} model are found by validation for the best performance which are
$\sigma_{xy}$ = (0.1, 0.1) for the spatial (XY) kernel and $\sigma_{rgbxy}$ = (0.1, 0.1, 0.1, 0.01, 0.01) for color and position (RGBXY)  kernel.
Next, as more kernels are added to \bi{8}{2}, we set scales to be $\alpha$*($\sigma_{xy}$, $\sigma_{rgbxy}$).
The value of $\alpha$ is chosen as  1, 0.5, 0.1, 0.05, 0.1, at uniform interval, for the \bi{8}{10} bilateral inception module.


\subsection{Qualitative Results}
\label{sec:qualitative-app}

In this section, we present more qualitative results obtained using the BI module with explicit
kernel computation technique presented in Chapter~\ref{chap:binception}. Results on the Pascal VOC12
dataset~\cite{voc2012segmentation} using the DeepLab-LargeFOV model are shown in Fig.~\ref{fig:semantic_visuals-app},
followed by the results on MINC dataset~\cite{bell2015minc}
in Fig.~\ref{fig:material_visuals-app} and on
Cityscapes dataset~\cite{Cordts2015Cvprw} in Fig.~\ref{fig:street_visuals-app}.


\definecolor{voc_1}{RGB}{0, 0, 0}
\definecolor{voc_2}{RGB}{128, 0, 0}
\definecolor{voc_3}{RGB}{0, 128, 0}
\definecolor{voc_4}{RGB}{128, 128, 0}
\definecolor{voc_5}{RGB}{0, 0, 128}
\definecolor{voc_6}{RGB}{128, 0, 128}
\definecolor{voc_7}{RGB}{0, 128, 128}
\definecolor{voc_8}{RGB}{128, 128, 128}
\definecolor{voc_9}{RGB}{64, 0, 0}
\definecolor{voc_10}{RGB}{192, 0, 0}
\definecolor{voc_11}{RGB}{64, 128, 0}
\definecolor{voc_12}{RGB}{192, 128, 0}
\definecolor{voc_13}{RGB}{64, 0, 128}
\definecolor{voc_14}{RGB}{192, 0, 128}
\definecolor{voc_15}{RGB}{64, 128, 128}
\definecolor{voc_16}{RGB}{192, 128, 128}
\definecolor{voc_17}{RGB}{0, 64, 0}
\definecolor{voc_18}{RGB}{128, 64, 0}
\definecolor{voc_19}{RGB}{0, 192, 0}
\definecolor{voc_20}{RGB}{128, 192, 0}
\definecolor{voc_21}{RGB}{0, 64, 128}
\definecolor{voc_22}{RGB}{128, 64, 128}

\begin{figure*}[!ht]
  \small
  \centering
  \fcolorbox{white}{voc_1}{\rule{0pt}{4pt}\rule{4pt}{0pt}} Background~~
  \fcolorbox{white}{voc_2}{\rule{0pt}{4pt}\rule{4pt}{0pt}} Aeroplane~~
  \fcolorbox{white}{voc_3}{\rule{0pt}{4pt}\rule{4pt}{0pt}} Bicycle~~
  \fcolorbox{white}{voc_4}{\rule{0pt}{4pt}\rule{4pt}{0pt}} Bird~~
  \fcolorbox{white}{voc_5}{\rule{0pt}{4pt}\rule{4pt}{0pt}} Boat~~
  \fcolorbox{white}{voc_6}{\rule{0pt}{4pt}\rule{4pt}{0pt}} Bottle~~
  \fcolorbox{white}{voc_7}{\rule{0pt}{4pt}\rule{4pt}{0pt}} Bus~~
  \fcolorbox{white}{voc_8}{\rule{0pt}{4pt}\rule{4pt}{0pt}} Car~~\\
  \fcolorbox{white}{voc_9}{\rule{0pt}{4pt}\rule{4pt}{0pt}} Cat~~
  \fcolorbox{white}{voc_10}{\rule{0pt}{4pt}\rule{4pt}{0pt}} Chair~~
  \fcolorbox{white}{voc_11}{\rule{0pt}{4pt}\rule{4pt}{0pt}} Cow~~
  \fcolorbox{white}{voc_12}{\rule{0pt}{4pt}\rule{4pt}{0pt}} Dining Table~~
  \fcolorbox{white}{voc_13}{\rule{0pt}{4pt}\rule{4pt}{0pt}} Dog~~
  \fcolorbox{white}{voc_14}{\rule{0pt}{4pt}\rule{4pt}{0pt}} Horse~~
  \fcolorbox{white}{voc_15}{\rule{0pt}{4pt}\rule{4pt}{0pt}} Motorbike~~
  \fcolorbox{white}{voc_16}{\rule{0pt}{4pt}\rule{4pt}{0pt}} Person~~\\
  \fcolorbox{white}{voc_17}{\rule{0pt}{4pt}\rule{4pt}{0pt}} Potted Plant~~
  \fcolorbox{white}{voc_18}{\rule{0pt}{4pt}\rule{4pt}{0pt}} Sheep~~
  \fcolorbox{white}{voc_19}{\rule{0pt}{4pt}\rule{4pt}{0pt}} Sofa~~
  \fcolorbox{white}{voc_20}{\rule{0pt}{4pt}\rule{4pt}{0pt}} Train~~
  \fcolorbox{white}{voc_21}{\rule{0pt}{4pt}\rule{4pt}{0pt}} TV monitor~~\\


  \subfigure{%
    \includegraphics[width=.15\columnwidth]{figures/supplementary/2008_001308_given.png}
  }
  \subfigure{%
    \includegraphics[width=.15\columnwidth]{figures/supplementary/2008_001308_sp.png}
  }
  \subfigure{%
    \includegraphics[width=.15\columnwidth]{figures/supplementary/2008_001308_gt.png}
  }
  \subfigure{%
    \includegraphics[width=.15\columnwidth]{figures/supplementary/2008_001308_cnn.png}
  }
  \subfigure{%
    \includegraphics[width=.15\columnwidth]{figures/supplementary/2008_001308_crf.png}
  }
  \subfigure{%
    \includegraphics[width=.15\columnwidth]{figures/supplementary/2008_001308_ours.png}
  }\\[-2ex]


  \subfigure{%
    \includegraphics[width=.15\columnwidth]{figures/supplementary/2008_001821_given.png}
  }
  \subfigure{%
    \includegraphics[width=.15\columnwidth]{figures/supplementary/2008_001821_sp.png}
  }
  \subfigure{%
    \includegraphics[width=.15\columnwidth]{figures/supplementary/2008_001821_gt.png}
  }
  \subfigure{%
    \includegraphics[width=.15\columnwidth]{figures/supplementary/2008_001821_cnn.png}
  }
  \subfigure{%
    \includegraphics[width=.15\columnwidth]{figures/supplementary/2008_001821_crf.png}
  }
  \subfigure{%
    \includegraphics[width=.15\columnwidth]{figures/supplementary/2008_001821_ours.png}
  }\\[-2ex]



  \subfigure{%
    \includegraphics[width=.15\columnwidth]{figures/supplementary/2008_004612_given.png}
  }
  \subfigure{%
    \includegraphics[width=.15\columnwidth]{figures/supplementary/2008_004612_sp.png}
  }
  \subfigure{%
    \includegraphics[width=.15\columnwidth]{figures/supplementary/2008_004612_gt.png}
  }
  \subfigure{%
    \includegraphics[width=.15\columnwidth]{figures/supplementary/2008_004612_cnn.png}
  }
  \subfigure{%
    \includegraphics[width=.15\columnwidth]{figures/supplementary/2008_004612_crf.png}
  }
  \subfigure{%
    \includegraphics[width=.15\columnwidth]{figures/supplementary/2008_004612_ours.png}
  }\\[-2ex]


  \subfigure{%
    \includegraphics[width=.15\columnwidth]{figures/supplementary/2009_001008_given.png}
  }
  \subfigure{%
    \includegraphics[width=.15\columnwidth]{figures/supplementary/2009_001008_sp.png}
  }
  \subfigure{%
    \includegraphics[width=.15\columnwidth]{figures/supplementary/2009_001008_gt.png}
  }
  \subfigure{%
    \includegraphics[width=.15\columnwidth]{figures/supplementary/2009_001008_cnn.png}
  }
  \subfigure{%
    \includegraphics[width=.15\columnwidth]{figures/supplementary/2009_001008_crf.png}
  }
  \subfigure{%
    \includegraphics[width=.15\columnwidth]{figures/supplementary/2009_001008_ours.png}
  }\\[-2ex]




  \subfigure{%
    \includegraphics[width=.15\columnwidth]{figures/supplementary/2009_004497_given.png}
  }
  \subfigure{%
    \includegraphics[width=.15\columnwidth]{figures/supplementary/2009_004497_sp.png}
  }
  \subfigure{%
    \includegraphics[width=.15\columnwidth]{figures/supplementary/2009_004497_gt.png}
  }
  \subfigure{%
    \includegraphics[width=.15\columnwidth]{figures/supplementary/2009_004497_cnn.png}
  }
  \subfigure{%
    \includegraphics[width=.15\columnwidth]{figures/supplementary/2009_004497_crf.png}
  }
  \subfigure{%
    \includegraphics[width=.15\columnwidth]{figures/supplementary/2009_004497_ours.png}
  }\\[-2ex]



  \setcounter{subfigure}{0}
  \subfigure[\scriptsize Input]{%
    \includegraphics[width=.15\columnwidth]{figures/supplementary/2010_001327_given.png}
  }
  \subfigure[\scriptsize Superpixels]{%
    \includegraphics[width=.15\columnwidth]{figures/supplementary/2010_001327_sp.png}
  }
  \subfigure[\scriptsize GT]{%
    \includegraphics[width=.15\columnwidth]{figures/supplementary/2010_001327_gt.png}
  }
  \subfigure[\scriptsize Deeplab]{%
    \includegraphics[width=.15\columnwidth]{figures/supplementary/2010_001327_cnn.png}
  }
  \subfigure[\scriptsize +DenseCRF]{%
    \includegraphics[width=.15\columnwidth]{figures/supplementary/2010_001327_crf.png}
  }
  \subfigure[\scriptsize Using BI]{%
    \includegraphics[width=.15\columnwidth]{figures/supplementary/2010_001327_ours.png}
  }
  \mycaption{Semantic Segmentation}{Example results of semantic segmentation
  on the Pascal VOC12 dataset.
  (d)~depicts the DeepLab CNN result, (e)~CNN + 10 steps of mean-field inference,
  (f~result obtained with bilateral inception (BI) modules (\bi{6}{2}+\bi{7}{6}) between \fc~layers.}
  \label{fig:semantic_visuals-app}
\end{figure*}


\definecolor{minc_1}{HTML}{771111}
\definecolor{minc_2}{HTML}{CAC690}
\definecolor{minc_3}{HTML}{EEEEEE}
\definecolor{minc_4}{HTML}{7C8FA6}
\definecolor{minc_5}{HTML}{597D31}
\definecolor{minc_6}{HTML}{104410}
\definecolor{minc_7}{HTML}{BB819C}
\definecolor{minc_8}{HTML}{D0CE48}
\definecolor{minc_9}{HTML}{622745}
\definecolor{minc_10}{HTML}{666666}
\definecolor{minc_11}{HTML}{D54A31}
\definecolor{minc_12}{HTML}{101044}
\definecolor{minc_13}{HTML}{444126}
\definecolor{minc_14}{HTML}{75D646}
\definecolor{minc_15}{HTML}{DD4348}
\definecolor{minc_16}{HTML}{5C8577}
\definecolor{minc_17}{HTML}{C78472}
\definecolor{minc_18}{HTML}{75D6D0}
\definecolor{minc_19}{HTML}{5B4586}
\definecolor{minc_20}{HTML}{C04393}
\definecolor{minc_21}{HTML}{D69948}
\definecolor{minc_22}{HTML}{7370D8}
\definecolor{minc_23}{HTML}{7A3622}
\definecolor{minc_24}{HTML}{000000}

\begin{figure*}[!ht]
  \small % scriptsize
  \centering
  \fcolorbox{white}{minc_1}{\rule{0pt}{4pt}\rule{4pt}{0pt}} Brick~~
  \fcolorbox{white}{minc_2}{\rule{0pt}{4pt}\rule{4pt}{0pt}} Carpet~~
  \fcolorbox{white}{minc_3}{\rule{0pt}{4pt}\rule{4pt}{0pt}} Ceramic~~
  \fcolorbox{white}{minc_4}{\rule{0pt}{4pt}\rule{4pt}{0pt}} Fabric~~
  \fcolorbox{white}{minc_5}{\rule{0pt}{4pt}\rule{4pt}{0pt}} Foliage~~
  \fcolorbox{white}{minc_6}{\rule{0pt}{4pt}\rule{4pt}{0pt}} Food~~
  \fcolorbox{white}{minc_7}{\rule{0pt}{4pt}\rule{4pt}{0pt}} Glass~~
  \fcolorbox{white}{minc_8}{\rule{0pt}{4pt}\rule{4pt}{0pt}} Hair~~\\
  \fcolorbox{white}{minc_9}{\rule{0pt}{4pt}\rule{4pt}{0pt}} Leather~~
  \fcolorbox{white}{minc_10}{\rule{0pt}{4pt}\rule{4pt}{0pt}} Metal~~
  \fcolorbox{white}{minc_11}{\rule{0pt}{4pt}\rule{4pt}{0pt}} Mirror~~
  \fcolorbox{white}{minc_12}{\rule{0pt}{4pt}\rule{4pt}{0pt}} Other~~
  \fcolorbox{white}{minc_13}{\rule{0pt}{4pt}\rule{4pt}{0pt}} Painted~~
  \fcolorbox{white}{minc_14}{\rule{0pt}{4pt}\rule{4pt}{0pt}} Paper~~
  \fcolorbox{white}{minc_15}{\rule{0pt}{4pt}\rule{4pt}{0pt}} Plastic~~\\
  \fcolorbox{white}{minc_16}{\rule{0pt}{4pt}\rule{4pt}{0pt}} Polished Stone~~
  \fcolorbox{white}{minc_17}{\rule{0pt}{4pt}\rule{4pt}{0pt}} Skin~~
  \fcolorbox{white}{minc_18}{\rule{0pt}{4pt}\rule{4pt}{0pt}} Sky~~
  \fcolorbox{white}{minc_19}{\rule{0pt}{4pt}\rule{4pt}{0pt}} Stone~~
  \fcolorbox{white}{minc_20}{\rule{0pt}{4pt}\rule{4pt}{0pt}} Tile~~
  \fcolorbox{white}{minc_21}{\rule{0pt}{4pt}\rule{4pt}{0pt}} Wallpaper~~
  \fcolorbox{white}{minc_22}{\rule{0pt}{4pt}\rule{4pt}{0pt}} Water~~
  \fcolorbox{white}{minc_23}{\rule{0pt}{4pt}\rule{4pt}{0pt}} Wood~~\\
  \subfigure{%
    \includegraphics[width=.15\columnwidth]{figures/supplementary/000008468_given.png}
  }
  \subfigure{%
    \includegraphics[width=.15\columnwidth]{figures/supplementary/000008468_sp.png}
  }
  \subfigure{%
    \includegraphics[width=.15\columnwidth]{figures/supplementary/000008468_gt.png}
  }
  \subfigure{%
    \includegraphics[width=.15\columnwidth]{figures/supplementary/000008468_cnn.png}
  }
  \subfigure{%
    \includegraphics[width=.15\columnwidth]{figures/supplementary/000008468_crf.png}
  }
  \subfigure{%
    \includegraphics[width=.15\columnwidth]{figures/supplementary/000008468_ours.png}
  }\\[-2ex]

  \subfigure{%
    \includegraphics[width=.15\columnwidth]{figures/supplementary/000009053_given.png}
  }
  \subfigure{%
    \includegraphics[width=.15\columnwidth]{figures/supplementary/000009053_sp.png}
  }
  \subfigure{%
    \includegraphics[width=.15\columnwidth]{figures/supplementary/000009053_gt.png}
  }
  \subfigure{%
    \includegraphics[width=.15\columnwidth]{figures/supplementary/000009053_cnn.png}
  }
  \subfigure{%
    \includegraphics[width=.15\columnwidth]{figures/supplementary/000009053_crf.png}
  }
  \subfigure{%
    \includegraphics[width=.15\columnwidth]{figures/supplementary/000009053_ours.png}
  }\\[-2ex]




  \subfigure{%
    \includegraphics[width=.15\columnwidth]{figures/supplementary/000014977_given.png}
  }
  \subfigure{%
    \includegraphics[width=.15\columnwidth]{figures/supplementary/000014977_sp.png}
  }
  \subfigure{%
    \includegraphics[width=.15\columnwidth]{figures/supplementary/000014977_gt.png}
  }
  \subfigure{%
    \includegraphics[width=.15\columnwidth]{figures/supplementary/000014977_cnn.png}
  }
  \subfigure{%
    \includegraphics[width=.15\columnwidth]{figures/supplementary/000014977_crf.png}
  }
  \subfigure{%
    \includegraphics[width=.15\columnwidth]{figures/supplementary/000014977_ours.png}
  }\\[-2ex]


  \subfigure{%
    \includegraphics[width=.15\columnwidth]{figures/supplementary/000022922_given.png}
  }
  \subfigure{%
    \includegraphics[width=.15\columnwidth]{figures/supplementary/000022922_sp.png}
  }
  \subfigure{%
    \includegraphics[width=.15\columnwidth]{figures/supplementary/000022922_gt.png}
  }
  \subfigure{%
    \includegraphics[width=.15\columnwidth]{figures/supplementary/000022922_cnn.png}
  }
  \subfigure{%
    \includegraphics[width=.15\columnwidth]{figures/supplementary/000022922_crf.png}
  }
  \subfigure{%
    \includegraphics[width=.15\columnwidth]{figures/supplementary/000022922_ours.png}
  }\\[-2ex]


  \subfigure{%
    \includegraphics[width=.15\columnwidth]{figures/supplementary/000025711_given.png}
  }
  \subfigure{%
    \includegraphics[width=.15\columnwidth]{figures/supplementary/000025711_sp.png}
  }
  \subfigure{%
    \includegraphics[width=.15\columnwidth]{figures/supplementary/000025711_gt.png}
  }
  \subfigure{%
    \includegraphics[width=.15\columnwidth]{figures/supplementary/000025711_cnn.png}
  }
  \subfigure{%
    \includegraphics[width=.15\columnwidth]{figures/supplementary/000025711_crf.png}
  }
  \subfigure{%
    \includegraphics[width=.15\columnwidth]{figures/supplementary/000025711_ours.png}
  }\\[-2ex]


  \subfigure{%
    \includegraphics[width=.15\columnwidth]{figures/supplementary/000034473_given.png}
  }
  \subfigure{%
    \includegraphics[width=.15\columnwidth]{figures/supplementary/000034473_sp.png}
  }
  \subfigure{%
    \includegraphics[width=.15\columnwidth]{figures/supplementary/000034473_gt.png}
  }
  \subfigure{%
    \includegraphics[width=.15\columnwidth]{figures/supplementary/000034473_cnn.png}
  }
  \subfigure{%
    \includegraphics[width=.15\columnwidth]{figures/supplementary/000034473_crf.png}
  }
  \subfigure{%
    \includegraphics[width=.15\columnwidth]{figures/supplementary/000034473_ours.png}
  }\\[-2ex]


  \subfigure{%
    \includegraphics[width=.15\columnwidth]{figures/supplementary/000035463_given.png}
  }
  \subfigure{%
    \includegraphics[width=.15\columnwidth]{figures/supplementary/000035463_sp.png}
  }
  \subfigure{%
    \includegraphics[width=.15\columnwidth]{figures/supplementary/000035463_gt.png}
  }
  \subfigure{%
    \includegraphics[width=.15\columnwidth]{figures/supplementary/000035463_cnn.png}
  }
  \subfigure{%
    \includegraphics[width=.15\columnwidth]{figures/supplementary/000035463_crf.png}
  }
  \subfigure{%
    \includegraphics[width=.15\columnwidth]{figures/supplementary/000035463_ours.png}
  }\\[-2ex]


  \setcounter{subfigure}{0}
  \subfigure[\scriptsize Input]{%
    \includegraphics[width=.15\columnwidth]{figures/supplementary/000035993_given.png}
  }
  \subfigure[\scriptsize Superpixels]{%
    \includegraphics[width=.15\columnwidth]{figures/supplementary/000035993_sp.png}
  }
  \subfigure[\scriptsize GT]{%
    \includegraphics[width=.15\columnwidth]{figures/supplementary/000035993_gt.png}
  }
  \subfigure[\scriptsize AlexNet]{%
    \includegraphics[width=.15\columnwidth]{figures/supplementary/000035993_cnn.png}
  }
  \subfigure[\scriptsize +DenseCRF]{%
    \includegraphics[width=.15\columnwidth]{figures/supplementary/000035993_crf.png}
  }
  \subfigure[\scriptsize Using BI]{%
    \includegraphics[width=.15\columnwidth]{figures/supplementary/000035993_ours.png}
  }
  \mycaption{Material Segmentation}{Example results of material segmentation.
  (d)~depicts the AlexNet CNN result, (e)~CNN + 10 steps of mean-field inference,
  (f)~result obtained with bilateral inception (BI) modules (\bi{7}{2}+\bi{8}{6}) between
  \fc~layers.}
\label{fig:material_visuals-app}
\end{figure*}


\definecolor{city_1}{RGB}{128, 64, 128}
\definecolor{city_2}{RGB}{244, 35, 232}
\definecolor{city_3}{RGB}{70, 70, 70}
\definecolor{city_4}{RGB}{102, 102, 156}
\definecolor{city_5}{RGB}{190, 153, 153}
\definecolor{city_6}{RGB}{153, 153, 153}
\definecolor{city_7}{RGB}{250, 170, 30}
\definecolor{city_8}{RGB}{220, 220, 0}
\definecolor{city_9}{RGB}{107, 142, 35}
\definecolor{city_10}{RGB}{152, 251, 152}
\definecolor{city_11}{RGB}{70, 130, 180}
\definecolor{city_12}{RGB}{220, 20, 60}
\definecolor{city_13}{RGB}{255, 0, 0}
\definecolor{city_14}{RGB}{0, 0, 142}
\definecolor{city_15}{RGB}{0, 0, 70}
\definecolor{city_16}{RGB}{0, 60, 100}
\definecolor{city_17}{RGB}{0, 80, 100}
\definecolor{city_18}{RGB}{0, 0, 230}
\definecolor{city_19}{RGB}{119, 11, 32}
\begin{figure*}[!ht]
  \small % scriptsize
  \centering


  \subfigure{%
    \includegraphics[width=.18\columnwidth]{figures/supplementary/frankfurt00000_016005_given.png}
  }
  \subfigure{%
    \includegraphics[width=.18\columnwidth]{figures/supplementary/frankfurt00000_016005_sp.png}
  }
  \subfigure{%
    \includegraphics[width=.18\columnwidth]{figures/supplementary/frankfurt00000_016005_gt.png}
  }
  \subfigure{%
    \includegraphics[width=.18\columnwidth]{figures/supplementary/frankfurt00000_016005_cnn.png}
  }
  \subfigure{%
    \includegraphics[width=.18\columnwidth]{figures/supplementary/frankfurt00000_016005_ours.png}
  }\\[-2ex]

  \subfigure{%
    \includegraphics[width=.18\columnwidth]{figures/supplementary/frankfurt00000_004617_given.png}
  }
  \subfigure{%
    \includegraphics[width=.18\columnwidth]{figures/supplementary/frankfurt00000_004617_sp.png}
  }
  \subfigure{%
    \includegraphics[width=.18\columnwidth]{figures/supplementary/frankfurt00000_004617_gt.png}
  }
  \subfigure{%
    \includegraphics[width=.18\columnwidth]{figures/supplementary/frankfurt00000_004617_cnn.png}
  }
  \subfigure{%
    \includegraphics[width=.18\columnwidth]{figures/supplementary/frankfurt00000_004617_ours.png}
  }\\[-2ex]

  \subfigure{%
    \includegraphics[width=.18\columnwidth]{figures/supplementary/frankfurt00000_020880_given.png}
  }
  \subfigure{%
    \includegraphics[width=.18\columnwidth]{figures/supplementary/frankfurt00000_020880_sp.png}
  }
  \subfigure{%
    \includegraphics[width=.18\columnwidth]{figures/supplementary/frankfurt00000_020880_gt.png}
  }
  \subfigure{%
    \includegraphics[width=.18\columnwidth]{figures/supplementary/frankfurt00000_020880_cnn.png}
  }
  \subfigure{%
    \includegraphics[width=.18\columnwidth]{figures/supplementary/frankfurt00000_020880_ours.png}
  }\\[-2ex]



  \subfigure{%
    \includegraphics[width=.18\columnwidth]{figures/supplementary/frankfurt00001_007285_given.png}
  }
  \subfigure{%
    \includegraphics[width=.18\columnwidth]{figures/supplementary/frankfurt00001_007285_sp.png}
  }
  \subfigure{%
    \includegraphics[width=.18\columnwidth]{figures/supplementary/frankfurt00001_007285_gt.png}
  }
  \subfigure{%
    \includegraphics[width=.18\columnwidth]{figures/supplementary/frankfurt00001_007285_cnn.png}
  }
  \subfigure{%
    \includegraphics[width=.18\columnwidth]{figures/supplementary/frankfurt00001_007285_ours.png}
  }\\[-2ex]


  \subfigure{%
    \includegraphics[width=.18\columnwidth]{figures/supplementary/frankfurt00001_059789_given.png}
  }
  \subfigure{%
    \includegraphics[width=.18\columnwidth]{figures/supplementary/frankfurt00001_059789_sp.png}
  }
  \subfigure{%
    \includegraphics[width=.18\columnwidth]{figures/supplementary/frankfurt00001_059789_gt.png}
  }
  \subfigure{%
    \includegraphics[width=.18\columnwidth]{figures/supplementary/frankfurt00001_059789_cnn.png}
  }
  \subfigure{%
    \includegraphics[width=.18\columnwidth]{figures/supplementary/frankfurt00001_059789_ours.png}
  }\\[-2ex]


  \subfigure{%
    \includegraphics[width=.18\columnwidth]{figures/supplementary/frankfurt00001_068208_given.png}
  }
  \subfigure{%
    \includegraphics[width=.18\columnwidth]{figures/supplementary/frankfurt00001_068208_sp.png}
  }
  \subfigure{%
    \includegraphics[width=.18\columnwidth]{figures/supplementary/frankfurt00001_068208_gt.png}
  }
  \subfigure{%
    \includegraphics[width=.18\columnwidth]{figures/supplementary/frankfurt00001_068208_cnn.png}
  }
  \subfigure{%
    \includegraphics[width=.18\columnwidth]{figures/supplementary/frankfurt00001_068208_ours.png}
  }\\[-2ex]

  \subfigure{%
    \includegraphics[width=.18\columnwidth]{figures/supplementary/frankfurt00001_082466_given.png}
  }
  \subfigure{%
    \includegraphics[width=.18\columnwidth]{figures/supplementary/frankfurt00001_082466_sp.png}
  }
  \subfigure{%
    \includegraphics[width=.18\columnwidth]{figures/supplementary/frankfurt00001_082466_gt.png}
  }
  \subfigure{%
    \includegraphics[width=.18\columnwidth]{figures/supplementary/frankfurt00001_082466_cnn.png}
  }
  \subfigure{%
    \includegraphics[width=.18\columnwidth]{figures/supplementary/frankfurt00001_082466_ours.png}
  }\\[-2ex]

  \subfigure{%
    \includegraphics[width=.18\columnwidth]{figures/supplementary/lindau00033_000019_given.png}
  }
  \subfigure{%
    \includegraphics[width=.18\columnwidth]{figures/supplementary/lindau00033_000019_sp.png}
  }
  \subfigure{%
    \includegraphics[width=.18\columnwidth]{figures/supplementary/lindau00033_000019_gt.png}
  }
  \subfigure{%
    \includegraphics[width=.18\columnwidth]{figures/supplementary/lindau00033_000019_cnn.png}
  }
  \subfigure{%
    \includegraphics[width=.18\columnwidth]{figures/supplementary/lindau00033_000019_ours.png}
  }\\[-2ex]

  \subfigure{%
    \includegraphics[width=.18\columnwidth]{figures/supplementary/lindau00052_000019_given.png}
  }
  \subfigure{%
    \includegraphics[width=.18\columnwidth]{figures/supplementary/lindau00052_000019_sp.png}
  }
  \subfigure{%
    \includegraphics[width=.18\columnwidth]{figures/supplementary/lindau00052_000019_gt.png}
  }
  \subfigure{%
    \includegraphics[width=.18\columnwidth]{figures/supplementary/lindau00052_000019_cnn.png}
  }
  \subfigure{%
    \includegraphics[width=.18\columnwidth]{figures/supplementary/lindau00052_000019_ours.png}
  }\\[-2ex]




  \subfigure{%
    \includegraphics[width=.18\columnwidth]{figures/supplementary/lindau00027_000019_given.png}
  }
  \subfigure{%
    \includegraphics[width=.18\columnwidth]{figures/supplementary/lindau00027_000019_sp.png}
  }
  \subfigure{%
    \includegraphics[width=.18\columnwidth]{figures/supplementary/lindau00027_000019_gt.png}
  }
  \subfigure{%
    \includegraphics[width=.18\columnwidth]{figures/supplementary/lindau00027_000019_cnn.png}
  }
  \subfigure{%
    \includegraphics[width=.18\columnwidth]{figures/supplementary/lindau00027_000019_ours.png}
  }\\[-2ex]



  \setcounter{subfigure}{0}
  \subfigure[\scriptsize Input]{%
    \includegraphics[width=.18\columnwidth]{figures/supplementary/lindau00029_000019_given.png}
  }
  \subfigure[\scriptsize Superpixels]{%
    \includegraphics[width=.18\columnwidth]{figures/supplementary/lindau00029_000019_sp.png}
  }
  \subfigure[\scriptsize GT]{%
    \includegraphics[width=.18\columnwidth]{figures/supplementary/lindau00029_000019_gt.png}
  }
  \subfigure[\scriptsize Deeplab]{%
    \includegraphics[width=.18\columnwidth]{figures/supplementary/lindau00029_000019_cnn.png}
  }
  \subfigure[\scriptsize Using BI]{%
    \includegraphics[width=.18\columnwidth]{figures/supplementary/lindau00029_000019_ours.png}
  }%\\[-2ex]

  \mycaption{Street Scene Segmentation}{Example results of street scene segmentation.
  (d)~depicts the DeepLab results, (e)~result obtained by adding bilateral inception (BI) modules (\bi{6}{2}+\bi{7}{6}) between \fc~layers.}
\label{fig:street_visuals-app}
\end{figure*}

\end{CJK*}


\end{document}
