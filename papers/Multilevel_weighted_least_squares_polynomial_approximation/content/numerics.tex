\section{Numerical Experiments}
\label{sec:numerics}
To support our theoretical analysis, we performed numerical experiments on linear elliptic parametric PDEs of the form
%\subsection{Linear elliptic PDE}
%\label{ssec:numlin}
%We consider the PDE

\begin{equation}
\label{eq:pdenum}
\begin{aligned}
-\nabla \cdot (a(x,\psmi) \nabla \pde(x,\psmi))&=1&&\text{ in }U:=[-1,1]^D\\
\pde(x,\psmi)&=0&&\text{ on } \partial U,
\end{aligned}
\end{equation}
as in \Cref{sec:uq}.
%In the pre-asymptotic regime of our numerical experiments we observed convergence of the approximations at the rate $h^{-2.2}$ that the required computational work behaved like $h^{1.7}$ and . This corresponds to the values $\beta=2.2$, $\gamma=1.7$ for the parameters in \Cref{sec:nonadaptive}.
%\subsection{Non-smooth case}
We let
\begin{equation*}
a(x,\psmi)={1 + \|x\|_2^{r} + \|\psmi\|_2^{s}},\quad \psmi\in\domPS:=[-1,1]^d
\end{equation*}
for $r := 1$, $s := 3$, $D:=2$ and $d\in\{2,3,4,6\}$.
Our goal was to approximate the response surface
\begin{equation*}
\psmi\mapsto \rs(\psmi):=\QoI(\pde(\cdot,\psmi)):=0.5 \int_{U}\pde(\cdot,\psmi)\;dx
\end{equation*}
in $L^2(\Gamma)$.
%
%The results agree with the theory in \Cref{sec:nonadaptive}.
%Since the dependence of the coefficient $a_{\psmi}$ on $\psmi$ has a kink at $\psmi=0$, the response surface $\rs$ is only Lipschitz continuous. By \cite[Theorem 2]{BagbyBosLevenberg2002}) it can be approximated by polynomials of degree at most $k$ with accuracy $k^{-1}$.
%
%For $d\geq 1$ and $p\in\{2,\infty\}$, similar results hold with $\alpha<1$ and $\dvsp=\dim \vsp$, where $\vsp$ is a downward closed polynomial space (see e.g. \cite[Theorem 2]{BagbyBosLevenberg2002}).
The numerical scheme we used to solve \Cref{eq:pdenum} employs
centered finite difference approximations to the derivatives with a
constant mesh size, $h$. Such a numerical scheme converges
asymptotically at a rate of $\mathcal O(h^{2})$ in the $L^2$ norm
and requires a computational work of $\mathcal O(h^{-2})$, since the
PDE is two-dimensional. This corresponds to the values
$\beta_s = \beta_w = 2$ and $\gamma=2$ for the parameters in
Assumptions A2 and A3.
%%%%%
    % The
    % value of $\alpha$ in A1$(\infty)$ was fitted and found to be
    % $\alpha \approx 1$ for $d=1$ and $\alpha \approx 0.5$ for $d=2$.
    % \todo{Using what method}.  Using these values we can expect the
    % multilevel weighted least squares approximation to have a
    % complexity of $\mathcal{O}({\epsilon^{-1}})$ for $d=1$ and of
    % around $\mathcal{O}({\epsilon^{-2}})$ for $d=2$.
To estimate the projection error of our
estimate we evaluate the $L^2$ error norm using Monte Carlo sampling with $M=1000$ samples,
\begin{equation}\label{eq:l2-mc-error}
  \norm{f - S_L(f)}{L^2(\Gamma)}^2 \approx \frac{1}{M} \sum_{j=1}^M
  (f_{L+1}(\psmi_j) - S_L(f)(\psmi_j))^2.
\end{equation}
In our tests we employ both the nonadaptive and the adaptive
algorithms from Sections 4 and 5. As a basis for the nonadaptive
algorithm, we use total degree polynomial spaces
\(
  \vsp_{\dvsp} := \vspan\left\{ \leg_{\eta} : |\eta|_{1} \leq \dvsp \right\},
\)
where $\leg_\eta$ is a tensor product of Legendre
polynomials as in \Cref{sec:adaptive}.  We also compare the multilevel algorithm to the
straightforward, single-level approach, which for a given polynomial
approximation space $\vsp_{\dvsp}$ uses samples from a fixed PDE discretization
level that matches the accuracy of the polynomial best
approximation in $\vsp_{\dvsp}$. To find these matching PDE discretization levels, we consider the complexity curve of the single-level
method as the lower envelope of complexity curves with different
PDE discretization levels. Even though such a method is not practical, the
choice of discretization level for a given tolerance is always
optimal. The random points were sampled from the optimal
distribution as explained in \Cref{sec:optimal-sampling}.

Before presenting the numerical results, let us derive some a-priori
estimates of the complexity of the single-level and multilevel
projection methods.  From \Cref{pro:finite}, if
$a \in C^{r}(U)\otimes C^{s}(\domPS)$, then using finite elements
of order $r$ and mesh size $h$ would yield convergence in the space $F:=C^{s}(\domPS)$ with the values $\sc=\wc=r+1$ of the parameters in \Cref{sec:nonadaptive}, and  optimal solvers
would require the work $\mathcal{O}(h^{-\gamma})$,  $\gamma:=D$. Furthermore, since functions in
$C^{s}(\domPS)$ are approximable by polynomials of total degree less than or equal to $k$ at the rate $\mathcal{O}(k^{-s})$ in the supremum norm \cite{BagbyBosLevenberg2002}, we expect at least $\alpha=s$.
Even though our choice $a(x,\psmi) = 1 + \|  x\|_2^{r} + \| \psmi\|_2^{s}$
satisfies only $a \in C^{r-1, 1}(U)\otimes C^{s-1,1}(\domPS)$, we do not expect different rates than those derived above for $a\in C^{r}(U)\otimes C^{s}(\domPS)$. Finally, the dimension of total degree polynomial spaces $\vsp_{\dvsp}$ equals $\binom{\dvsp+d}{d}$ and asymptotically we have $\binom{\dvsp+d}{d}\lesssim \dvsp^d$.

Thus, we expect the complexity of the single-level method to be
$\mathcal{O}\left(\epsilon^{-\frac{D}{r+1} -
    \frac{d}{s}}\log(\epsilon^{-1})\right)$, while the complexity of
the multilevel method is of
$\mathcal{O}\left({\epsilon^{-\max\left({\frac{D}{r + 1},
          \frac{d}{s}}\right)}} \log(\epsilon^{-1})^{t}\right)$,
where
\[t =
  \begin{cases}
    1 & \frac{D}{r + 1} > \frac{d}{s},\\
    3 + \frac{D}{r+1} & \frac{D}{r + 1} = \frac{d}{s},\\
    2 & \frac{D}{r + 1} < \frac{d}{s}.
  \end{cases}
\]
Hence, for $r=1$ and $s=3$, the complexity of the single-level
method is
$\mathcal{O}\left(\epsilon^{-1 -
    \frac{d}{3}}\log(\epsilon^{-1})\right)$ and the complexity of the
multilevel method is
$\mathcal{O}\left({\epsilon^{-\max({1, \frac{d}{3}})}}
  \log(\epsilon^{-1})^{t}\right)$ where
\[t =
  \begin{cases}
    1, & d < 3,\\
    4, & d = 3,\\
    2, & d > 3.
  \end{cases}
\]

\Cref{fig:kink-work} shows the work estimate as defined in
\cref{eq:workdef} versus the $L^2$ error approximation in
\Cref{eq:l2-mc-error}. The results for the multilevel algorithm displayed there were obtained with the work parameter $\dimp:=d/2$, which we found describes the pre-asymptotic behavior of $\dim \vsp_{\dvsp}=\binom{\dvsp+d}{d}$ better than $\dimp:=d$. The theoretical rates
satisfactorily match the obtained numerical rates, which show an
improvement of the multilevel methods over the single-level method. Note
that the work estimate does \emph{not} include the cost of generating
points or the cost of assembling the projection matrix and computing
the projection. On the other hand, these costs are included in
\Cref{fig:kink-time}, which shows the total running time in seconds of the
three different methods. While these plots still show the same
complexity rates as \Cref{fig:kink-work} for all three methods for
sufficiently small errors, these plots also show the overhead of the
multilevel methods, especially as $d$ increases. The overhead of the
adaptive algorithm for the multilevel method is especially significant
and more work needs to be done to reduce it.

\setlength\figureheight{7.4cm}
\setlength\figurewidth{8.2cm}
\providecommand{\figlabel}{fig:}

\begin{figure}
	\centering
	\begin{subfigure}{0.49\textwidth}
      \renewcommand{\figlabel}{fig:work-est-vs-error-d2}
      % This file was created by matplotlib2tikz v0.6.10.
\begin{tikzpicture}

\begin{axis}[
xlabel={Max Error},
ylabel={Work Estimate},
xmin=1e-05, xmax=0.1,
ymin=10, ymax=10000000,
xmode=log,
ymode=log,
axis on top,
name=\figlabel,
width=\figurewidth,
height=\figureheight,
xtick={1e-06,1e-05,0.0001,0.001,0.01,0.1,1,10},
xticklabels={,$10^{-5}$,$10^{-4}$,$10^{-3}$,$10^{-2}$,$10^{-1}$,,},
ytick={1,10,100,1000,10000,100000,1000000,10000000,100000000},
yticklabels={,$10^{1}$,$10^{2}$,$10^{3}$,$10^{4}$,$10^{5}$,$10^{6}$,$10^{7}$,},
tick pos=both
]
\addplot [thick, black, opacity=0.4, dotted, mark=x, mark size=2, mark options={solid,fill opacity=0}, forget plot]
table {%
0.032164710232 69.1886323727
0.0323385638832 361.8947501008
0.051035633904 18.5754247591
};
\addplot [thick, black, opacity=0.4, dotted, mark=x, mark size=2, mark options={solid,fill opacity=0}, forget plot]
table {%
0.0215099418036 723.7895002021
0.0216397120011 4867.526024689
0.0217242797332 138.3772647455
0.0490792003595 37.1508495182
};
\addplot [thick, black, opacity=0.4, dotted, mark=x, mark size=2, mark options={solid,fill opacity=0}, forget plot]
table {%
0.0113282255642 1447.579000405
0.0114090379439 76914.110944276
0.0114164151911 9735.052049378
0.012577796492 276.754529491
0.0496575860038 74.3016990364
};
\addplot [thick, black, opacity=0.4, dotted, mark=x, mark size=2, mark options={solid,fill opacity=0}, forget plot]
table {%
0.00700020634185 2895.15800081
0.00701893900149 19470.104098755
0.00938574447436 553.509058982
0.0506262399478 148.603398073
};
\addplot [thick, black, opacity=0.4, dotted, mark=x, mark size=2, mark options={solid,fill opacity=0}, forget plot]
table {%
0.00339649414451 307656.44377673
0.00343469517396 38940.208197512
0.00358106706727 5790.316001612
0.00773531370794 1107.018117964
0.0517209057266 297.206796146
};
\addplot [thick, black, opacity=0.4, dotted, mark=x, mark size=2, mark options={solid,fill opacity=0}, forget plot]
table {%
0.0019931085648 615312.88755353
0.00206216284501 77880.41639506
0.0024145667152 11580.63200323
0.00747873832964 2214.036235931
0.0522163261965 594.413592291
};
\addplot [thick, black, opacity=0.4, dotted, mark=x, mark size=2, mark options={solid,fill opacity=0}, forget plot]
table {%
0.0009204159105695 1230625.77511105
0.00106861061652 155760.83279042
0.00177325026714 23161.26400645
0.00745826609476 4428.07247185
0.0526202481683 1188.82718458
};
\addplot [thick, black, opacity=0.4, dotted, mark=x, mark size=2, mark options={solid,fill opacity=0}, forget plot]
table {%
0.000493992319159 2461251.5502122
0.00073802650909 311521.6655798
0.00164834196358 46322.5280129
0.00749324983904 8856.14494371
0.05278678499 2377.65436916
};
\addplot [thick, black, opacity=0.4, dotted, mark=x, mark size=2, mark options={solid,fill opacity=0}, forget plot]
table {%
0.000233040511938 4922503.1004248
0.000598815200992 623043.3311597
0.00162332440566 92645.0560259
0.00752662028315 17712.28988747
0.0528903900769 4755.30873833
};
\addplot [thick, black, opacity=0.4, dotted, mark=x, mark size=2, mark options={solid,fill opacity=0}, forget plot]
table {%
0.0001165553314975 9845006.2008547
0.00056522288294 1246086.66232
0.0016256597444 185290.1120518
0.0075444453459 35424.5797748
0.0529371274145 9510.61747666
};
\addplot [thick, black, dash pattern=on 1pt off 3pt on 3pt off 3pt]
table {%
1e-05 4247662241.04439
1.09749876549306e-05 3497986977.08867
1.20450354025878e-05 2880431665.17057
1.32194114846603e-05 2371743206.54259
1.45082877849594e-05 1952755938.22796
1.59228279334109e-05 1607673978.99932
1.74752840000768e-05 1323479440.72066
1.91791026167249e-05 1089444452.74566
2.10490414451202e-05 896728807.049833
2.31012970008316e-05 738048216.724462
2.53536449397011e-05 607400808.327169
2.78255940220713e-05 499841636.722683
3.05385550883341e-05 411296799.815261
3.35160265093884e-05 338410206.292905
3.67837977182863e-05 278417266.910374
4.03701725859655e-05 229040784.107569
4.43062145758388e-05 188405143.173012
4.86260158006535e-05 154965591.484351
5.33669923120631e-05 127449955.976938
5.85702081805666e-05 104810613.857127
6.42807311728432e-05 86184914.9752186
7.05480231071865e-05 70862570.4671783
7.74263682681127e-05 58258783.0326291
8.49753435908644e-05 47892109.2486144
9.3260334688322e-05 39366221.6346528
0.000102353102189903 32354884.3931478
0.000112332403297803 26589577.3019887
0.000123284673944207 21849301.6303237
0.000135304777457981 17952183.8936959
0.000148496826225446 14748560.8208886
0.000162975083462064 12115284.5958094
0.000178864952905744 9951033.34513978
0.000196304065004027 8172449.68313259
0.000215443469003188 6710961.31467271
0.000236448941264541 5510163.40417582
0.000259502421139973 4523663.60391576
0.00028480358684358 3713308.09492341
0.000312571584968824 3047721.38134859
0.000343046928631492 2501104.43490921
0.000376493580679247 2052245.55481198
0.000413201240011534 1683706.35722152
0.000453487850812858 1381151.9394621
0.000497702356433211 1132799.72700641
0.000546227721768434 928966.011483401
0.000599484250318941 761692.894845177
0.000657933224657568 624441.408115537
0.000722080901838546 511839.087892472
0.000792482898353917 419472.36484652
0.000869749002617784 343715.824014196
0.000954548456661834 281591.801144955
0.00104761575278967 230654.935799257
0.00114975699539774 188897.254048057
0.00126185688306602 154670.137523375
0.00138488637139387 126620.180901583
0.00151991108295293 103636.471121581
0.00166810053720006 84807.2588949574
0.00183073828029537 69384.3529489508
0.00200923300256505 56753.863623767
0.00220513073990305 46412.166180794
0.00242012826479438 37946.154738533
0.00265608778294669 31017.0227732859
0.00291505306282518 25346.941888629
0.00319926713779738 20708.1222479161
0.00351119173421513 16913.8299395077
0.00385352859371053 13811.0121139179
0.0042292428743895 11274.2428841199
0.00464158883361278 9200.75409259546
0.00509413801481638 7506.35707842349
0.00559081018251223 6122.09613558111
0.00613590727341317 4991.50276527335
0.00673415065775082 4068.34318140864
0.00739072203352578 3314.77072690895
0.00811130830789687 2699.81063793624
0.00890215085445039 2198.11756094486
0.00977009957299226 1788.95688358251
0.0107226722201032 1455.36969602644
0.01176811952435 1183.48839254203
0.0129154966501488 961.97583202377
0.0141747416292681 781.565829749759
0.0155567614393047 634.686738629896
0.0170735264747069 515.153151493922
0.0187381742286039 417.913443582548
0.0205651230834865 338.843080883435
0.0225701971963392 274.575431198029
0.0247707635599171 222.3633014388
0.0271858824273294 179.965644661379
0.0298364724028334 145.554881410123
0.0327454916287773 117.641101272089
0.0359381366380463 95.0100842755018
0.0394420605943766 76.6726343707392
0.0432876128108306 61.8231704078335
0.047508101621028 49.805891606787
0.0521400828799969 40.0871391486319
0.0572236765935022 32.2328252190036
0.0628029144183425 25.8900054822274
0.068926121043497 20.7718386545672
0.075646332755463 16.6453142300011
0.0830217568131974 13.3212419476412
0.0911162756115489 10.6460887542957
0.1 8.49532448208878
};
\label{\figlabel-line1}
\addplot [thick, black, dashed]
table {%
1e-05 4976638.83475196
1.09749876549306e-05 4389724.41793334
1.20450354025878e-05 3870999.29773914
1.32194114846603e-05 3412649.84895434
1.45082877849594e-05 3007746.43371076
1.59228279334109e-05 2650144.56302362
1.74752840000768e-05 2334396.99003651
1.91791026167249e-05 2055675.53797336
2.10490414451202e-05 1809701.59565616
2.31012970008316e-05 1592684.32933987
2.53536449397011e-05 1401265.76303051
2.78255940220713e-05 1232471.97172266
3.05385550883341e-05 1083669.71430914
3.35160265093884e-05 952527.906343535
3.67837977182863e-05 836983.398328014
4.03701725859655e-05 735210.583604189
4.43062145758388e-05 645594.412004739
4.86260158006535e-05 566706.431857217
5.33669923120631e-05 497283.524325417
5.85702081805666e-05 436209.030970735
6.42807311728432e-05 382496.0082999
7.05480231071865e-05 335272.372369246
7.74263682681127e-05 293767.722625427
8.49753435908644e-05 257301.657423333
9.3260334688322e-05 225273.414381845
0.000102353102189903 197152.68719205
0.000112332403297803 172471.486926127
0.000123284673944207 150816.930527394
0.000135304777457981 131824.852188607
0.000148496826225446 115174.144920958
0.000162975083462064 100581.749936155
0.000178864952905744 87798.220647335
0.000196304065004027 76603.7962652827
0.000215443469003188 66804.9272351445
0.000236448941264541 58231.2012239875
0.000259502421139973 50732.6241191753
0.00028480358684358 44177.2156099519
0.000312571584968824 38448.8834697236
0.000343046928631492 33445.5446966009
0.000376493580679247 29077.4652603306
0.000413201240011534 25265.7933942708
0.000453487850812858 21941.2642056343
0.000497702356433211 19043.0558951114
0.000546227721768434 16517.7801131789
0.000599484250318941 14318.5909660502
0.000657933224657568 12404.3989470836
0.000722080901838546 10739.1776342485
0.000792482898353917 9291.35238293421
0.000869749002617784 8033.26147557669
0.000954548456661834 6940.68128267995
0.00104761575278967 5992.40795933928
0.00114975699539774 5169.88906113031
0.00126185688306602 4456.89922548041
0.00138488637139387 3839.25474030946
0.00151991108295293 3304.56242052167
0.00166810053720006 2841.99874348366
0.00183073828029537 2442.11566461068
0.00200923300256505 2096.66995042488
0.00220513073990305 1798.47323501001
0.00242012826479438 1541.26033205368
0.00265608778294669 1319.57362342661
0.00291505306282518 1128.66160075072
0.00319926713779738 964.389862429069
0.00351119173421513 823.16306850415
0.00385352859371053 701.856532452069
0.0042292428743895 597.756285251255
0.00464158883361278 508.506585129562
0.00509413801481638 432.063968372926
0.00559081018251223 366.657044317378
0.00613590727341317 310.751332780663
0.00673415065775082 263.018526170884
0.00739072203352578 222.309632624112
0.00811130830789687 187.631521910749
0.00890215085445039 158.126453522158
0.00977009957299226 133.054217199753
0.0107226722201032 111.776560991269
0.01176811952435 93.7436214154563
0.0129154966501488 78.4821051085996
0.0141747416292681 65.5850019643097
0.0155567614393047 54.7026367488334
0.0170735264747069 45.534889908531
0.0187381742286039 37.8244391651294
0.0205651230834865 31.3508918548465
0.0225701971963392 25.9256941073046
0.0247707635599171 21.3877171425766
0.0271858824273294 17.5994334223059
0.0298364724028334 14.4436063290594
0.0327454916287773 11.8204266483465
0.0359381366380463 9.64503755021114
0.0394420605943766 7.84539715335559
0.0432876128108306 6.36043422907291
0.047508101621028 5.13845827487981
0.0521400828799969 4.13579015562423
0.0572236765935022 3.31558385850608
0.0628029144183425 2.64681371322598
0.068926121043497 2.10340475573597
0.075646332755463 1.66348682224711
0.0830217568131974 1.30875550069652
0.0911162756115489 1.02392528506243
0.1 0.796262213560313
};
\label{\figlabel-line3}
\addplot [ultra thick, blue]
table {%
0.0001165553314975 9845006.2008547
0.000232761587034 4922503.1004248
0.000493564532628 2461251.5502122
0.00056522288294 1246086.66232
0.000598815200992 623043.3311597
0.00073802650909 311521.6655798
0.00106861061652 155760.83279042
0.00162332440566 92645.0560259
0.00164834196358 46322.5280129
0.00177325026714 23161.26400645
0.0024145667152 11580.63200323
0.00358106706727 5790.316001612
0.00699311153206 2895.15800081
0.00747873832964 2214.036235931
0.00773531370794 1107.018117964
0.00938574447436 553.509058982
0.012577796492 276.754529491
0.0217242797332 138.3772647455
0.032164710232 69.1886323727
0.0490792003595 37.1508495182
0.051035633904 18.5754247591
};
\label{\figlabel-line0}
\addplot [ultra thick, green!50.0!black]
table {%
2.77827474765e-05 1152962.3334012
5.75099002429e-05 493514.142529296
0.000112277003268 163790.047094071
0.000189863742175 81647.3554463046
0.000630736910547 39496.886904104
0.000919070948055 18531.561945883
0.00138045231564 8048.8994667681
0.00252201901249 3996.3954117929
0.00355017263387 1817.2503308473
0.00893802979483 727.6777903736
0.0112029278899 329.2445790003
0.0491479964041 55.7262742773
0.0493803165062 130.0279733137
0.051035633904 18.5754247591
};
\label{\figlabel-line2}
\addplot [ultra thick, red]
table {%
1.682905020824e-05 1465343.06887342
1.70492165828667e-05 1432620.32537575
1.70829609509e-05 1383076.82631634
1.727061807898e-05 1273207.1080593
1.736302509678e-05 1227807.87249397
1.7799389140975e-05 1193915.68002601
1.786538869205e-05 1168566.43935378
1.82194032139e-05 1148703.29900138
1.856508115925e-05 1125601.46645678
1.87717791392e-05 1051628.68486908
1.89474020089e-05 1018559.05929778
1.90675213093e-05 994661.455494854
2.3235792567e-05 978123.471502334
2.35003881724667e-05 934388.977361361
2.40564354213e-05 924006.324762458
2.48313255168e-05 915117.538082155
2.53234413125333e-05 891141.131022587
2.611400090278e-05 851067.527495727
3.02262617698333e-05 830184.235922585
3.07481399056e-05 820190.498320702
3.11307156634e-05 777015.878295156
3.124116675068e-05 795892.895903336
3.29682610306e-05 768823.878295156
3.30779291260438e-05 698274.344493606
3.33397167697556e-05 644783.705256492
3.38128293446e-05 618603.421923612
3.922086420675e-05 597321.30846969
4.040411529026e-05 570824.765072729
4.050032614895e-05 558264.134559534
4.11935211211143e-05 524204.698363391
4.21888423596e-05 519685.851445811
4.277386761995e-05 513156.533749525
4.3347078465e-05 508088.999329637
4.36207134424e-05 499195.685649433
4.39598908302e-05 486914.172601508
4.42634295686e-05 479882.103246832
4.47561574026e-05 473155.038197176
4.66864988026e-05 468710.644857017
4.8744441424e-05 464349.068050405
5.106364617055e-05 447177.667176095
5.19660353981e-05 434713.311142095
5.22337797112e-05 422764.509240635
5.25823878421e-05 411440.792733585
5.7551844229275e-05 400940.519333712
5.81883983372e-05 390411.028137782
6.52066751877778e-05 358549.069851966
6.55487280999769e-05 321430.248989756
6.61412936914e-05 317427.802184936
6.64338127461e-05 300217.624933135
6.69693097881e-05 309656.133737205
7.435644045065e-05 293348.44634553
7.44428594431e-05 290593.318500134
7.56537529264333e-05 282816.816548881
7.67297344697333e-05 275156.043416231
7.79030973681e-05 272896.619957441
7.8707419815e-05 269146.509283092
7.93225039713e-05 265504.187159583
7.96826933982e-05 261988.152482272
8.0190482274e-05 258624.619957441
9.93595104183e-05 246271.181276414
9.96115338976e-05 241587.29711738
0.0001013383994725 236741.990948784
0.000102775627007 234146.327799052
0.000104142115523 225513.976617019
0.0001055760732915 216715.787545298
0.0001271346248165 210047.524337203
0.000128170533344 207549.089936677
0.0001296658032005 199440.55132751
0.0001309227239555 186107.256498174
0.0001312628377554 171012.214179366
0.000132778786775 165747.468581401
0.000139017122603 162201.964579551
0.000140030996222 157482.710177516
0.000161864210675727 141828.699061826
0.00016361695023 140639.04542396
0.0001647859181715 136219.827603627
0.000166708480837 135073.20916862
0.000206367076476 130977.20916862
0.0002075592408526 122942.825624996
0.00020959627951575 107380.627505207
0.00021043711795 105622.610166555
0.000215839795109 104511.511831505
0.000218486581616 101739.351367437
0.000221478502515 97446.5011488516
0.000222819139759 94330.4121403406
0.000233317096064 92263.1641412656
0.000234430351084 89275.9636659096
0.0002373581483655 84286.8758920729
0.000238005333901286 79124.7694878455
0.0002403864397425 76146.7727827097
0.000243655907728 73514.3999837267
0.000261422112111 71741.6479828017
0.0002638372614615 69843.0855125755
0.000265363881059 68932.5049817041
0.000268249969741 68053.4963123738
0.0002713256933005 66688.8443381222
0.00034035304319725 62002.5557222732
0.000342829138345 58950.9109803232
0.000407766433344 51888.3308053483
0.0004099354345794 45132.4951725422
0.000421078262066 43721.2490566909
0.000422786674327 41222.0393344207
0.000479867118785 38163.1781691127
0.000483363282856 40083.1781691127
0.000497531146255286 35222.5131318682
0.000498172384842 34042.6995313612
0.00069860140118 33030.3936807582
0.000701836986693 31134.8810250614
0.0008877216977065 29671.3766903938
0.00090078833417725 27299.679203122
0.000906178962919857 22677.4389342099
0.000910468585032 21151.6165632459
0.000946374523019 18777.3132322953
0.0009803715917955 17601.2986852389
0.00101185457164667 15618.621837364
0.00103484144939 13823.2210769081
0.00103877776224 14481.3142766551
0.00110193295374 13129.8801513771
0.0011141376555 12539.9733511241
0.001160295898305 11423.6635603664
0.001260940236745 10880
0.00131180501709 10579.4937046998
0.00131695337261 10282.0802952397
0.00135348108296 9697.4556752799
0.00135851087529 9463.073758133
0.00145913469431 8948.7740166547
0.0016874048423 8033.5513429856
0.00169723382203 7751.1234106371
0.0017326181717 7253.5966595442
0.001849454865255 6591.4869830665
0.00185957730776667 5741.0989335429
0.00186588730191 5387.2327926977
0.00198582708167 4799.7801929393
0.00241314955181 4578.1861928242
0.00245187576688 4283.2327926977
0.00251679456459 4155.2327926977
0.00306595322722 3697.6214558629
0.00307740476944 3899.2327926977
0.00338655805440667 3186.9628631771
0.00340626702522 3010.0297927554
0.003545520928785 2734.709492761
0.00356141453672 2474.6195752042
0.003601981847475 2246.6121332869
0.004391029690795 2072.9364745392
0.00473438971969 1724.4076368528
0.00476834606068 2001.7576517077
0.00555048966464 1692.4076368528
0.00634597103084 1439.3311742023
0.0063816242189 1316.5958854162
0.0064201347306 1196.4665352114
0.00652480405925 506
0.00655479476002 684.8487414329
0.00891908090713 442
0.0128186847164 410
0.0217122569854667 130.56842503044
0.021862309007675 171.6992500145
0.0223981635367 102.86917501594
0.0233249075327 66
0.0236130554393 50
};
\label{\figlabel-line4}
\coordinate (legend) at (axis description cs:0.03,0.03);
\end{axis}

\matrix [matrix of nodes,
inner sep=1pt, row sep=1pt,cells={anchor=west},anchor={south west},at={(0.03,0.03)}, anchor=south west, draw=none, fill=none] at (legend) {
\ref{\figlabel-line0} {SL}\\
\ref{\figlabel-line1} {$\epsilon^{-2}\log(\epsilon^{-1})$}\\
\ref{\figlabel-line2} {ML}\\
\ref{\figlabel-line3} {$\epsilon^{-1}\log(\epsilon^{-1})^{4}$}\\
\ref{\figlabel-line4} {Adaptive ML}\\
};
\end{tikzpicture}
      \caption{$d=2$}
	\end{subfigure}
	\begin{subfigure}{0.5\textwidth}
      \renewcommand{\figlabel}{fig:work-est-vs-error-d3}
      % This file was created by matplotlib2tikz v0.6.10.
\begin{tikzpicture}

\begin{axis}[
xlabel={Max Error},
ylabel={Work Estimate},
xmin=1e-05, xmax=0.1,
ymin=10, ymax=10000000,
xmode=log,
ymode=log,
axis on top,
name=\figlabel,
width=\figurewidth,
height=\figureheight,
xtick={1e-06,1e-05,0.0001,0.001,0.01,0.1,1,10},
xticklabels={,$10^{-5}$,$10^{-4}$,$10^{-3}$,$10^{-2}$,$10^{-1}$,,},
ytick={1,10,100,1000,10000,100000,1000000,10000000,100000000},
yticklabels={,$10^{1}$,$10^{2}$,$10^{3}$,$10^{4}$,$10^{5}$,$10^{6}$,$10^{7}$,},
tick pos=both
]
\addplot [thick, black, opacity=0.4, dotted, mark=x, mark size=2, mark options={solid,fill opacity=0}, forget plot]
table {%
0.032164710232 69.1886323727
0.0323385638832 361.8947501008
0.051035633904 18.5754247591
};
\addplot [thick, black, opacity=0.4, dotted, mark=x, mark size=2, mark options={solid,fill opacity=0}, forget plot]
table {%
0.0215099418036 723.7895002021
0.0216397120011 4867.526024689
0.0217242797332 138.3772647455
0.0490792003595 37.1508495182
};
\addplot [thick, black, opacity=0.4, dotted, mark=x, mark size=2, mark options={solid,fill opacity=0}, forget plot]
table {%
0.0113282255642 1447.579000405
0.0114090379439 76914.110944276
0.0114164151911 9735.052049378
0.012577796492 276.754529491
0.0496575860038 74.3016990364
};
\addplot [thick, black, opacity=0.4, dotted, mark=x, mark size=2, mark options={solid,fill opacity=0}, forget plot]
table {%
0.00700020634185 2895.15800081
0.00701893900149 19470.104098755
0.00938574447436 553.509058982
0.0506262399478 148.603398073
};
\addplot [thick, black, opacity=0.4, dotted, mark=x, mark size=2, mark options={solid,fill opacity=0}, forget plot]
table {%
0.00339649414451 307656.44377673
0.00343469517396 38940.208197512
0.00358106706727 5790.316001612
0.00773531370794 1107.018117964
0.0517209057266 297.206796146
};
\addplot [thick, black, opacity=0.4, dotted, mark=x, mark size=2, mark options={solid,fill opacity=0}, forget plot]
table {%
0.0019931085648 615312.88755353
0.00206216284501 77880.41639506
0.0024145667152 11580.63200323
0.00747873832964 2214.036235931
0.0522163261965 594.413592291
};
\addplot [thick, black, opacity=0.4, dotted, mark=x, mark size=2, mark options={solid,fill opacity=0}, forget plot]
table {%
0.0009204159105695 1230625.77511105
0.00106861061652 155760.83279042
0.00177325026714 23161.26400645
0.00745826609476 4428.07247185
0.0526202481683 1188.82718458
};
\addplot [thick, black, opacity=0.4, dotted, mark=x, mark size=2, mark options={solid,fill opacity=0}, forget plot]
table {%
0.000493992319159 2461251.5502122
0.00073802650909 311521.6655798
0.00164834196358 46322.5280129
0.00749324983904 8856.14494371
0.05278678499 2377.65436916
};
\addplot [thick, black, opacity=0.4, dotted, mark=x, mark size=2, mark options={solid,fill opacity=0}, forget plot]
table {%
0.000233040511938 4922503.1004248
0.000598815200992 623043.3311597
0.00162332440566 92645.0560259
0.00752662028315 17712.28988747
0.0528903900769 4755.30873833
};
\addplot [thick, black, opacity=0.4, dotted, mark=x, mark size=2, mark options={solid,fill opacity=0}, forget plot]
table {%
0.0001165553314975 9845006.2008547
0.00056522288294 1246086.66232
0.0016256597444 185290.1120518
0.0075444453459 35424.5797748
0.0529371274145 9510.61747666
};
\addplot [thick, black, dash pattern=on 1pt off 3pt on 3pt off 3pt]
table {%
1e-05 4247662241.04439
1.09749876549306e-05 3497986977.08867
1.20450354025878e-05 2880431665.17057
1.32194114846603e-05 2371743206.54259
1.45082877849594e-05 1952755938.22796
1.59228279334109e-05 1607673978.99932
1.74752840000768e-05 1323479440.72066
1.91791026167249e-05 1089444452.74566
2.10490414451202e-05 896728807.049833
2.31012970008316e-05 738048216.724462
2.53536449397011e-05 607400808.327169
2.78255940220713e-05 499841636.722683
3.05385550883341e-05 411296799.815261
3.35160265093884e-05 338410206.292905
3.67837977182863e-05 278417266.910374
4.03701725859655e-05 229040784.107569
4.43062145758388e-05 188405143.173012
4.86260158006535e-05 154965591.484351
5.33669923120631e-05 127449955.976938
5.85702081805666e-05 104810613.857127
6.42807311728432e-05 86184914.9752186
7.05480231071865e-05 70862570.4671783
7.74263682681127e-05 58258783.0326291
8.49753435908644e-05 47892109.2486144
9.3260334688322e-05 39366221.6346528
0.000102353102189903 32354884.3931478
0.000112332403297803 26589577.3019887
0.000123284673944207 21849301.6303237
0.000135304777457981 17952183.8936959
0.000148496826225446 14748560.8208886
0.000162975083462064 12115284.5958094
0.000178864952905744 9951033.34513978
0.000196304065004027 8172449.68313259
0.000215443469003188 6710961.31467271
0.000236448941264541 5510163.40417582
0.000259502421139973 4523663.60391576
0.00028480358684358 3713308.09492341
0.000312571584968824 3047721.38134859
0.000343046928631492 2501104.43490921
0.000376493580679247 2052245.55481198
0.000413201240011534 1683706.35722152
0.000453487850812858 1381151.9394621
0.000497702356433211 1132799.72700641
0.000546227721768434 928966.011483401
0.000599484250318941 761692.894845177
0.000657933224657568 624441.408115537
0.000722080901838546 511839.087892472
0.000792482898353917 419472.36484652
0.000869749002617784 343715.824014196
0.000954548456661834 281591.801144955
0.00104761575278967 230654.935799257
0.00114975699539774 188897.254048057
0.00126185688306602 154670.137523375
0.00138488637139387 126620.180901583
0.00151991108295293 103636.471121581
0.00166810053720006 84807.2588949574
0.00183073828029537 69384.3529489508
0.00200923300256505 56753.863623767
0.00220513073990305 46412.166180794
0.00242012826479438 37946.154738533
0.00265608778294669 31017.0227732859
0.00291505306282518 25346.941888629
0.00319926713779738 20708.1222479161
0.00351119173421513 16913.8299395077
0.00385352859371053 13811.0121139179
0.0042292428743895 11274.2428841199
0.00464158883361278 9200.75409259546
0.00509413801481638 7506.35707842349
0.00559081018251223 6122.09613558111
0.00613590727341317 4991.50276527335
0.00673415065775082 4068.34318140864
0.00739072203352578 3314.77072690895
0.00811130830789687 2699.81063793624
0.00890215085445039 2198.11756094486
0.00977009957299226 1788.95688358251
0.0107226722201032 1455.36969602644
0.01176811952435 1183.48839254203
0.0129154966501488 961.97583202377
0.0141747416292681 781.565829749759
0.0155567614393047 634.686738629896
0.0170735264747069 515.153151493922
0.0187381742286039 417.913443582548
0.0205651230834865 338.843080883435
0.0225701971963392 274.575431198029
0.0247707635599171 222.3633014388
0.0271858824273294 179.965644661379
0.0298364724028334 145.554881410123
0.0327454916287773 117.641101272089
0.0359381366380463 95.0100842755018
0.0394420605943766 76.6726343707392
0.0432876128108306 61.8231704078335
0.047508101621028 49.805891606787
0.0521400828799969 40.0871391486319
0.0572236765935022 32.2328252190036
0.0628029144183425 25.8900054822274
0.068926121043497 20.7718386545672
0.075646332755463 16.6453142300011
0.0830217568131974 13.3212419476412
0.0911162756115489 10.6460887542957
0.1 8.49532448208878
};
\label{\figlabel-line1}
\addplot [thick, black, dashed]
table {%
1e-05 4976638.83475196
1.09749876549306e-05 4389724.41793334
1.20450354025878e-05 3870999.29773914
1.32194114846603e-05 3412649.84895434
1.45082877849594e-05 3007746.43371076
1.59228279334109e-05 2650144.56302362
1.74752840000768e-05 2334396.99003651
1.91791026167249e-05 2055675.53797336
2.10490414451202e-05 1809701.59565616
2.31012970008316e-05 1592684.32933987
2.53536449397011e-05 1401265.76303051
2.78255940220713e-05 1232471.97172266
3.05385550883341e-05 1083669.71430914
3.35160265093884e-05 952527.906343535
3.67837977182863e-05 836983.398328014
4.03701725859655e-05 735210.583604189
4.43062145758388e-05 645594.412004739
4.86260158006535e-05 566706.431857217
5.33669923120631e-05 497283.524325417
5.85702081805666e-05 436209.030970735
6.42807311728432e-05 382496.0082999
7.05480231071865e-05 335272.372369246
7.74263682681127e-05 293767.722625427
8.49753435908644e-05 257301.657423333
9.3260334688322e-05 225273.414381845
0.000102353102189903 197152.68719205
0.000112332403297803 172471.486926127
0.000123284673944207 150816.930527394
0.000135304777457981 131824.852188607
0.000148496826225446 115174.144920958
0.000162975083462064 100581.749936155
0.000178864952905744 87798.220647335
0.000196304065004027 76603.7962652827
0.000215443469003188 66804.9272351445
0.000236448941264541 58231.2012239875
0.000259502421139973 50732.6241191753
0.00028480358684358 44177.2156099519
0.000312571584968824 38448.8834697236
0.000343046928631492 33445.5446966009
0.000376493580679247 29077.4652603306
0.000413201240011534 25265.7933942708
0.000453487850812858 21941.2642056343
0.000497702356433211 19043.0558951114
0.000546227721768434 16517.7801131789
0.000599484250318941 14318.5909660502
0.000657933224657568 12404.3989470836
0.000722080901838546 10739.1776342485
0.000792482898353917 9291.35238293421
0.000869749002617784 8033.26147557669
0.000954548456661834 6940.68128267995
0.00104761575278967 5992.40795933928
0.00114975699539774 5169.88906113031
0.00126185688306602 4456.89922548041
0.00138488637139387 3839.25474030946
0.00151991108295293 3304.56242052167
0.00166810053720006 2841.99874348366
0.00183073828029537 2442.11566461068
0.00200923300256505 2096.66995042488
0.00220513073990305 1798.47323501001
0.00242012826479438 1541.26033205368
0.00265608778294669 1319.57362342661
0.00291505306282518 1128.66160075072
0.00319926713779738 964.389862429069
0.00351119173421513 823.16306850415
0.00385352859371053 701.856532452069
0.0042292428743895 597.756285251255
0.00464158883361278 508.506585129562
0.00509413801481638 432.063968372926
0.00559081018251223 366.657044317378
0.00613590727341317 310.751332780663
0.00673415065775082 263.018526170884
0.00739072203352578 222.309632624112
0.00811130830789687 187.631521910749
0.00890215085445039 158.126453522158
0.00977009957299226 133.054217199753
0.0107226722201032 111.776560991269
0.01176811952435 93.7436214154563
0.0129154966501488 78.4821051085996
0.0141747416292681 65.5850019643097
0.0155567614393047 54.7026367488334
0.0170735264747069 45.534889908531
0.0187381742286039 37.8244391651294
0.0205651230834865 31.3508918548465
0.0225701971963392 25.9256941073046
0.0247707635599171 21.3877171425766
0.0271858824273294 17.5994334223059
0.0298364724028334 14.4436063290594
0.0327454916287773 11.8204266483465
0.0359381366380463 9.64503755021114
0.0394420605943766 7.84539715335559
0.0432876128108306 6.36043422907291
0.047508101621028 5.13845827487981
0.0521400828799969 4.13579015562423
0.0572236765935022 3.31558385850608
0.0628029144183425 2.64681371322598
0.068926121043497 2.10340475573597
0.075646332755463 1.66348682224711
0.0830217568131974 1.30875550069652
0.0911162756115489 1.02392528506243
0.1 0.796262213560313
};
\label{\figlabel-line3}
\addplot [ultra thick, blue]
table {%
0.0001165553314975 9845006.2008547
0.000232761587034 4922503.1004248
0.000493564532628 2461251.5502122
0.00056522288294 1246086.66232
0.000598815200992 623043.3311597
0.00073802650909 311521.6655798
0.00106861061652 155760.83279042
0.00162332440566 92645.0560259
0.00164834196358 46322.5280129
0.00177325026714 23161.26400645
0.0024145667152 11580.63200323
0.00358106706727 5790.316001612
0.00699311153206 2895.15800081
0.00747873832964 2214.036235931
0.00773531370794 1107.018117964
0.00938574447436 553.509058982
0.012577796492 276.754529491
0.0217242797332 138.3772647455
0.032164710232 69.1886323727
0.0490792003595 37.1508495182
0.051035633904 18.5754247591
};
\label{\figlabel-line0}
\addplot [ultra thick, green!50.0!black]
table {%
2.77827474765e-05 1152962.3334012
5.75099002429e-05 493514.142529296
0.000112277003268 163790.047094071
0.000189863742175 81647.3554463046
0.000630736910547 39496.886904104
0.000919070948055 18531.561945883
0.00138045231564 8048.8994667681
0.00252201901249 3996.3954117929
0.00355017263387 1817.2503308473
0.00893802979483 727.6777903736
0.0112029278899 329.2445790003
0.0491479964041 55.7262742773
0.0493803165062 130.0279733137
0.051035633904 18.5754247591
};
\label{\figlabel-line2}
\addplot [ultra thick, red]
table {%
1.682905020824e-05 1465343.06887342
1.70492165828667e-05 1432620.32537575
1.70829609509e-05 1383076.82631634
1.727061807898e-05 1273207.1080593
1.736302509678e-05 1227807.87249397
1.7799389140975e-05 1193915.68002601
1.786538869205e-05 1168566.43935378
1.82194032139e-05 1148703.29900138
1.856508115925e-05 1125601.46645678
1.87717791392e-05 1051628.68486908
1.89474020089e-05 1018559.05929778
1.90675213093e-05 994661.455494854
2.3235792567e-05 978123.471502334
2.35003881724667e-05 934388.977361361
2.40564354213e-05 924006.324762458
2.48313255168e-05 915117.538082155
2.53234413125333e-05 891141.131022587
2.611400090278e-05 851067.527495727
3.02262617698333e-05 830184.235922585
3.07481399056e-05 820190.498320702
3.11307156634e-05 777015.878295156
3.124116675068e-05 795892.895903336
3.29682610306e-05 768823.878295156
3.30779291260438e-05 698274.344493606
3.33397167697556e-05 644783.705256492
3.38128293446e-05 618603.421923612
3.922086420675e-05 597321.30846969
4.040411529026e-05 570824.765072729
4.050032614895e-05 558264.134559534
4.11935211211143e-05 524204.698363391
4.21888423596e-05 519685.851445811
4.277386761995e-05 513156.533749525
4.3347078465e-05 508088.999329637
4.36207134424e-05 499195.685649433
4.39598908302e-05 486914.172601508
4.42634295686e-05 479882.103246832
4.47561574026e-05 473155.038197176
4.66864988026e-05 468710.644857017
4.8744441424e-05 464349.068050405
5.106364617055e-05 447177.667176095
5.19660353981e-05 434713.311142095
5.22337797112e-05 422764.509240635
5.25823878421e-05 411440.792733585
5.7551844229275e-05 400940.519333712
5.81883983372e-05 390411.028137782
6.52066751877778e-05 358549.069851966
6.55487280999769e-05 321430.248989756
6.61412936914e-05 317427.802184936
6.64338127461e-05 300217.624933135
6.69693097881e-05 309656.133737205
7.435644045065e-05 293348.44634553
7.44428594431e-05 290593.318500134
7.56537529264333e-05 282816.816548881
7.67297344697333e-05 275156.043416231
7.79030973681e-05 272896.619957441
7.8707419815e-05 269146.509283092
7.93225039713e-05 265504.187159583
7.96826933982e-05 261988.152482272
8.0190482274e-05 258624.619957441
9.93595104183e-05 246271.181276414
9.96115338976e-05 241587.29711738
0.0001013383994725 236741.990948784
0.000102775627007 234146.327799052
0.000104142115523 225513.976617019
0.0001055760732915 216715.787545298
0.0001271346248165 210047.524337203
0.000128170533344 207549.089936677
0.0001296658032005 199440.55132751
0.0001309227239555 186107.256498174
0.0001312628377554 171012.214179366
0.000132778786775 165747.468581401
0.000139017122603 162201.964579551
0.000140030996222 157482.710177516
0.000161864210675727 141828.699061826
0.00016361695023 140639.04542396
0.0001647859181715 136219.827603627
0.000166708480837 135073.20916862
0.000206367076476 130977.20916862
0.0002075592408526 122942.825624996
0.00020959627951575 107380.627505207
0.00021043711795 105622.610166555
0.000215839795109 104511.511831505
0.000218486581616 101739.351367437
0.000221478502515 97446.5011488516
0.000222819139759 94330.4121403406
0.000233317096064 92263.1641412656
0.000234430351084 89275.9636659096
0.0002373581483655 84286.8758920729
0.000238005333901286 79124.7694878455
0.0002403864397425 76146.7727827097
0.000243655907728 73514.3999837267
0.000261422112111 71741.6479828017
0.0002638372614615 69843.0855125755
0.000265363881059 68932.5049817041
0.000268249969741 68053.4963123738
0.0002713256933005 66688.8443381222
0.00034035304319725 62002.5557222732
0.000342829138345 58950.9109803232
0.000407766433344 51888.3308053483
0.0004099354345794 45132.4951725422
0.000421078262066 43721.2490566909
0.000422786674327 41222.0393344207
0.000479867118785 38163.1781691127
0.000483363282856 40083.1781691127
0.000497531146255286 35222.5131318682
0.000498172384842 34042.6995313612
0.00069860140118 33030.3936807582
0.000701836986693 31134.8810250614
0.0008877216977065 29671.3766903938
0.00090078833417725 27299.679203122
0.000906178962919857 22677.4389342099
0.000910468585032 21151.6165632459
0.000946374523019 18777.3132322953
0.0009803715917955 17601.2986852389
0.00101185457164667 15618.621837364
0.00103484144939 13823.2210769081
0.00103877776224 14481.3142766551
0.00110193295374 13129.8801513771
0.0011141376555 12539.9733511241
0.001160295898305 11423.6635603664
0.001260940236745 10880
0.00131180501709 10579.4937046998
0.00131695337261 10282.0802952397
0.00135348108296 9697.4556752799
0.00135851087529 9463.073758133
0.00145913469431 8948.7740166547
0.0016874048423 8033.5513429856
0.00169723382203 7751.1234106371
0.0017326181717 7253.5966595442
0.001849454865255 6591.4869830665
0.00185957730776667 5741.0989335429
0.00186588730191 5387.2327926977
0.00198582708167 4799.7801929393
0.00241314955181 4578.1861928242
0.00245187576688 4283.2327926977
0.00251679456459 4155.2327926977
0.00306595322722 3697.6214558629
0.00307740476944 3899.2327926977
0.00338655805440667 3186.9628631771
0.00340626702522 3010.0297927554
0.003545520928785 2734.709492761
0.00356141453672 2474.6195752042
0.003601981847475 2246.6121332869
0.004391029690795 2072.9364745392
0.00473438971969 1724.4076368528
0.00476834606068 2001.7576517077
0.00555048966464 1692.4076368528
0.00634597103084 1439.3311742023
0.0063816242189 1316.5958854162
0.0064201347306 1196.4665352114
0.00652480405925 506
0.00655479476002 684.8487414329
0.00891908090713 442
0.0128186847164 410
0.0217122569854667 130.56842503044
0.021862309007675 171.6992500145
0.0223981635367 102.86917501594
0.0233249075327 66
0.0236130554393 50
};
\label{\figlabel-line4}
\coordinate (legend) at (axis description cs:0.03,0.03);
\end{axis}

\matrix [matrix of nodes,
inner sep=1pt, row sep=1pt,cells={anchor=west},anchor={south west},at={(0.03,0.03)}, anchor=south west, draw=none, fill=none] at (legend) {
\ref{\figlabel-line0} {SL}\\
\ref{\figlabel-line1} {$\epsilon^{-2}\log(\epsilon^{-1})$}\\
\ref{\figlabel-line2} {ML}\\
\ref{\figlabel-line3} {$\epsilon^{-1}\log(\epsilon^{-1})^{4}$}\\
\ref{\figlabel-line4} {Adaptive ML}\\
};
\end{tikzpicture}
      \caption{$d=3$}
	\end{subfigure}
	\begin{subfigure}{0.49\textwidth}
      \renewcommand{\figlabel}{fig:work-est-vs-error-d4}
      % This file was created by matplotlib2tikz v0.6.10.
\begin{tikzpicture}

\begin{axis}[
xlabel={Max Error},
ylabel={Work Estimate},
xmin=1e-05, xmax=0.1,
ymin=10, ymax=10000000,
xmode=log,
ymode=log,
axis on top,
name=\figlabel,
width=\figurewidth,
height=\figureheight,
xtick={1e-06,1e-05,0.0001,0.001,0.01,0.1,1,10},
xticklabels={,$10^{-5}$,$10^{-4}$,$10^{-3}$,$10^{-2}$,$10^{-1}$,,},
ytick={1,10,100,1000,10000,100000,1000000,10000000,100000000},
yticklabels={,$10^{1}$,$10^{2}$,$10^{3}$,$10^{4}$,$10^{5}$,$10^{6}$,$10^{7}$,},
tick pos=both
]
\addplot [thick, black, opacity=0.4, dotted, mark=x, mark size=2, mark options={solid,fill opacity=0}, forget plot]
table {%
0.032164710232 69.1886323727
0.0323385638832 361.8947501008
0.051035633904 18.5754247591
};
\addplot [thick, black, opacity=0.4, dotted, mark=x, mark size=2, mark options={solid,fill opacity=0}, forget plot]
table {%
0.0215099418036 723.7895002021
0.0216397120011 4867.526024689
0.0217242797332 138.3772647455
0.0490792003595 37.1508495182
};
\addplot [thick, black, opacity=0.4, dotted, mark=x, mark size=2, mark options={solid,fill opacity=0}, forget plot]
table {%
0.0113282255642 1447.579000405
0.0114090379439 76914.110944276
0.0114164151911 9735.052049378
0.012577796492 276.754529491
0.0496575860038 74.3016990364
};
\addplot [thick, black, opacity=0.4, dotted, mark=x, mark size=2, mark options={solid,fill opacity=0}, forget plot]
table {%
0.00700020634185 2895.15800081
0.00701893900149 19470.104098755
0.00938574447436 553.509058982
0.0506262399478 148.603398073
};
\addplot [thick, black, opacity=0.4, dotted, mark=x, mark size=2, mark options={solid,fill opacity=0}, forget plot]
table {%
0.00339649414451 307656.44377673
0.00343469517396 38940.208197512
0.00358106706727 5790.316001612
0.00773531370794 1107.018117964
0.0517209057266 297.206796146
};
\addplot [thick, black, opacity=0.4, dotted, mark=x, mark size=2, mark options={solid,fill opacity=0}, forget plot]
table {%
0.0019931085648 615312.88755353
0.00206216284501 77880.41639506
0.0024145667152 11580.63200323
0.00747873832964 2214.036235931
0.0522163261965 594.413592291
};
\addplot [thick, black, opacity=0.4, dotted, mark=x, mark size=2, mark options={solid,fill opacity=0}, forget plot]
table {%
0.0009204159105695 1230625.77511105
0.00106861061652 155760.83279042
0.00177325026714 23161.26400645
0.00745826609476 4428.07247185
0.0526202481683 1188.82718458
};
\addplot [thick, black, opacity=0.4, dotted, mark=x, mark size=2, mark options={solid,fill opacity=0}, forget plot]
table {%
0.000493992319159 2461251.5502122
0.00073802650909 311521.6655798
0.00164834196358 46322.5280129
0.00749324983904 8856.14494371
0.05278678499 2377.65436916
};
\addplot [thick, black, opacity=0.4, dotted, mark=x, mark size=2, mark options={solid,fill opacity=0}, forget plot]
table {%
0.000233040511938 4922503.1004248
0.000598815200992 623043.3311597
0.00162332440566 92645.0560259
0.00752662028315 17712.28988747
0.0528903900769 4755.30873833
};
\addplot [thick, black, opacity=0.4, dotted, mark=x, mark size=2, mark options={solid,fill opacity=0}, forget plot]
table {%
0.0001165553314975 9845006.2008547
0.00056522288294 1246086.66232
0.0016256597444 185290.1120518
0.0075444453459 35424.5797748
0.0529371274145 9510.61747666
};
\addplot [thick, black, dash pattern=on 1pt off 3pt on 3pt off 3pt]
table {%
1e-05 4247662241.04439
1.09749876549306e-05 3497986977.08867
1.20450354025878e-05 2880431665.17057
1.32194114846603e-05 2371743206.54259
1.45082877849594e-05 1952755938.22796
1.59228279334109e-05 1607673978.99932
1.74752840000768e-05 1323479440.72066
1.91791026167249e-05 1089444452.74566
2.10490414451202e-05 896728807.049833
2.31012970008316e-05 738048216.724462
2.53536449397011e-05 607400808.327169
2.78255940220713e-05 499841636.722683
3.05385550883341e-05 411296799.815261
3.35160265093884e-05 338410206.292905
3.67837977182863e-05 278417266.910374
4.03701725859655e-05 229040784.107569
4.43062145758388e-05 188405143.173012
4.86260158006535e-05 154965591.484351
5.33669923120631e-05 127449955.976938
5.85702081805666e-05 104810613.857127
6.42807311728432e-05 86184914.9752186
7.05480231071865e-05 70862570.4671783
7.74263682681127e-05 58258783.0326291
8.49753435908644e-05 47892109.2486144
9.3260334688322e-05 39366221.6346528
0.000102353102189903 32354884.3931478
0.000112332403297803 26589577.3019887
0.000123284673944207 21849301.6303237
0.000135304777457981 17952183.8936959
0.000148496826225446 14748560.8208886
0.000162975083462064 12115284.5958094
0.000178864952905744 9951033.34513978
0.000196304065004027 8172449.68313259
0.000215443469003188 6710961.31467271
0.000236448941264541 5510163.40417582
0.000259502421139973 4523663.60391576
0.00028480358684358 3713308.09492341
0.000312571584968824 3047721.38134859
0.000343046928631492 2501104.43490921
0.000376493580679247 2052245.55481198
0.000413201240011534 1683706.35722152
0.000453487850812858 1381151.9394621
0.000497702356433211 1132799.72700641
0.000546227721768434 928966.011483401
0.000599484250318941 761692.894845177
0.000657933224657568 624441.408115537
0.000722080901838546 511839.087892472
0.000792482898353917 419472.36484652
0.000869749002617784 343715.824014196
0.000954548456661834 281591.801144955
0.00104761575278967 230654.935799257
0.00114975699539774 188897.254048057
0.00126185688306602 154670.137523375
0.00138488637139387 126620.180901583
0.00151991108295293 103636.471121581
0.00166810053720006 84807.2588949574
0.00183073828029537 69384.3529489508
0.00200923300256505 56753.863623767
0.00220513073990305 46412.166180794
0.00242012826479438 37946.154738533
0.00265608778294669 31017.0227732859
0.00291505306282518 25346.941888629
0.00319926713779738 20708.1222479161
0.00351119173421513 16913.8299395077
0.00385352859371053 13811.0121139179
0.0042292428743895 11274.2428841199
0.00464158883361278 9200.75409259546
0.00509413801481638 7506.35707842349
0.00559081018251223 6122.09613558111
0.00613590727341317 4991.50276527335
0.00673415065775082 4068.34318140864
0.00739072203352578 3314.77072690895
0.00811130830789687 2699.81063793624
0.00890215085445039 2198.11756094486
0.00977009957299226 1788.95688358251
0.0107226722201032 1455.36969602644
0.01176811952435 1183.48839254203
0.0129154966501488 961.97583202377
0.0141747416292681 781.565829749759
0.0155567614393047 634.686738629896
0.0170735264747069 515.153151493922
0.0187381742286039 417.913443582548
0.0205651230834865 338.843080883435
0.0225701971963392 274.575431198029
0.0247707635599171 222.3633014388
0.0271858824273294 179.965644661379
0.0298364724028334 145.554881410123
0.0327454916287773 117.641101272089
0.0359381366380463 95.0100842755018
0.0394420605943766 76.6726343707392
0.0432876128108306 61.8231704078335
0.047508101621028 49.805891606787
0.0521400828799969 40.0871391486319
0.0572236765935022 32.2328252190036
0.0628029144183425 25.8900054822274
0.068926121043497 20.7718386545672
0.075646332755463 16.6453142300011
0.0830217568131974 13.3212419476412
0.0911162756115489 10.6460887542957
0.1 8.49532448208878
};
\label{\figlabel-line1}
\addplot [thick, black, dashed]
table {%
1e-05 4976638.83475196
1.09749876549306e-05 4389724.41793334
1.20450354025878e-05 3870999.29773914
1.32194114846603e-05 3412649.84895434
1.45082877849594e-05 3007746.43371076
1.59228279334109e-05 2650144.56302362
1.74752840000768e-05 2334396.99003651
1.91791026167249e-05 2055675.53797336
2.10490414451202e-05 1809701.59565616
2.31012970008316e-05 1592684.32933987
2.53536449397011e-05 1401265.76303051
2.78255940220713e-05 1232471.97172266
3.05385550883341e-05 1083669.71430914
3.35160265093884e-05 952527.906343535
3.67837977182863e-05 836983.398328014
4.03701725859655e-05 735210.583604189
4.43062145758388e-05 645594.412004739
4.86260158006535e-05 566706.431857217
5.33669923120631e-05 497283.524325417
5.85702081805666e-05 436209.030970735
6.42807311728432e-05 382496.0082999
7.05480231071865e-05 335272.372369246
7.74263682681127e-05 293767.722625427
8.49753435908644e-05 257301.657423333
9.3260334688322e-05 225273.414381845
0.000102353102189903 197152.68719205
0.000112332403297803 172471.486926127
0.000123284673944207 150816.930527394
0.000135304777457981 131824.852188607
0.000148496826225446 115174.144920958
0.000162975083462064 100581.749936155
0.000178864952905744 87798.220647335
0.000196304065004027 76603.7962652827
0.000215443469003188 66804.9272351445
0.000236448941264541 58231.2012239875
0.000259502421139973 50732.6241191753
0.00028480358684358 44177.2156099519
0.000312571584968824 38448.8834697236
0.000343046928631492 33445.5446966009
0.000376493580679247 29077.4652603306
0.000413201240011534 25265.7933942708
0.000453487850812858 21941.2642056343
0.000497702356433211 19043.0558951114
0.000546227721768434 16517.7801131789
0.000599484250318941 14318.5909660502
0.000657933224657568 12404.3989470836
0.000722080901838546 10739.1776342485
0.000792482898353917 9291.35238293421
0.000869749002617784 8033.26147557669
0.000954548456661834 6940.68128267995
0.00104761575278967 5992.40795933928
0.00114975699539774 5169.88906113031
0.00126185688306602 4456.89922548041
0.00138488637139387 3839.25474030946
0.00151991108295293 3304.56242052167
0.00166810053720006 2841.99874348366
0.00183073828029537 2442.11566461068
0.00200923300256505 2096.66995042488
0.00220513073990305 1798.47323501001
0.00242012826479438 1541.26033205368
0.00265608778294669 1319.57362342661
0.00291505306282518 1128.66160075072
0.00319926713779738 964.389862429069
0.00351119173421513 823.16306850415
0.00385352859371053 701.856532452069
0.0042292428743895 597.756285251255
0.00464158883361278 508.506585129562
0.00509413801481638 432.063968372926
0.00559081018251223 366.657044317378
0.00613590727341317 310.751332780663
0.00673415065775082 263.018526170884
0.00739072203352578 222.309632624112
0.00811130830789687 187.631521910749
0.00890215085445039 158.126453522158
0.00977009957299226 133.054217199753
0.0107226722201032 111.776560991269
0.01176811952435 93.7436214154563
0.0129154966501488 78.4821051085996
0.0141747416292681 65.5850019643097
0.0155567614393047 54.7026367488334
0.0170735264747069 45.534889908531
0.0187381742286039 37.8244391651294
0.0205651230834865 31.3508918548465
0.0225701971963392 25.9256941073046
0.0247707635599171 21.3877171425766
0.0271858824273294 17.5994334223059
0.0298364724028334 14.4436063290594
0.0327454916287773 11.8204266483465
0.0359381366380463 9.64503755021114
0.0394420605943766 7.84539715335559
0.0432876128108306 6.36043422907291
0.047508101621028 5.13845827487981
0.0521400828799969 4.13579015562423
0.0572236765935022 3.31558385850608
0.0628029144183425 2.64681371322598
0.068926121043497 2.10340475573597
0.075646332755463 1.66348682224711
0.0830217568131974 1.30875550069652
0.0911162756115489 1.02392528506243
0.1 0.796262213560313
};
\label{\figlabel-line3}
\addplot [ultra thick, blue]
table {%
0.0001165553314975 9845006.2008547
0.000232761587034 4922503.1004248
0.000493564532628 2461251.5502122
0.00056522288294 1246086.66232
0.000598815200992 623043.3311597
0.00073802650909 311521.6655798
0.00106861061652 155760.83279042
0.00162332440566 92645.0560259
0.00164834196358 46322.5280129
0.00177325026714 23161.26400645
0.0024145667152 11580.63200323
0.00358106706727 5790.316001612
0.00699311153206 2895.15800081
0.00747873832964 2214.036235931
0.00773531370794 1107.018117964
0.00938574447436 553.509058982
0.012577796492 276.754529491
0.0217242797332 138.3772647455
0.032164710232 69.1886323727
0.0490792003595 37.1508495182
0.051035633904 18.5754247591
};
\label{\figlabel-line0}
\addplot [ultra thick, green!50.0!black]
table {%
2.77827474765e-05 1152962.3334012
5.75099002429e-05 493514.142529296
0.000112277003268 163790.047094071
0.000189863742175 81647.3554463046
0.000630736910547 39496.886904104
0.000919070948055 18531.561945883
0.00138045231564 8048.8994667681
0.00252201901249 3996.3954117929
0.00355017263387 1817.2503308473
0.00893802979483 727.6777903736
0.0112029278899 329.2445790003
0.0491479964041 55.7262742773
0.0493803165062 130.0279733137
0.051035633904 18.5754247591
};
\label{\figlabel-line2}
\addplot [ultra thick, red]
table {%
1.682905020824e-05 1465343.06887342
1.70492165828667e-05 1432620.32537575
1.70829609509e-05 1383076.82631634
1.727061807898e-05 1273207.1080593
1.736302509678e-05 1227807.87249397
1.7799389140975e-05 1193915.68002601
1.786538869205e-05 1168566.43935378
1.82194032139e-05 1148703.29900138
1.856508115925e-05 1125601.46645678
1.87717791392e-05 1051628.68486908
1.89474020089e-05 1018559.05929778
1.90675213093e-05 994661.455494854
2.3235792567e-05 978123.471502334
2.35003881724667e-05 934388.977361361
2.40564354213e-05 924006.324762458
2.48313255168e-05 915117.538082155
2.53234413125333e-05 891141.131022587
2.611400090278e-05 851067.527495727
3.02262617698333e-05 830184.235922585
3.07481399056e-05 820190.498320702
3.11307156634e-05 777015.878295156
3.124116675068e-05 795892.895903336
3.29682610306e-05 768823.878295156
3.30779291260438e-05 698274.344493606
3.33397167697556e-05 644783.705256492
3.38128293446e-05 618603.421923612
3.922086420675e-05 597321.30846969
4.040411529026e-05 570824.765072729
4.050032614895e-05 558264.134559534
4.11935211211143e-05 524204.698363391
4.21888423596e-05 519685.851445811
4.277386761995e-05 513156.533749525
4.3347078465e-05 508088.999329637
4.36207134424e-05 499195.685649433
4.39598908302e-05 486914.172601508
4.42634295686e-05 479882.103246832
4.47561574026e-05 473155.038197176
4.66864988026e-05 468710.644857017
4.8744441424e-05 464349.068050405
5.106364617055e-05 447177.667176095
5.19660353981e-05 434713.311142095
5.22337797112e-05 422764.509240635
5.25823878421e-05 411440.792733585
5.7551844229275e-05 400940.519333712
5.81883983372e-05 390411.028137782
6.52066751877778e-05 358549.069851966
6.55487280999769e-05 321430.248989756
6.61412936914e-05 317427.802184936
6.64338127461e-05 300217.624933135
6.69693097881e-05 309656.133737205
7.435644045065e-05 293348.44634553
7.44428594431e-05 290593.318500134
7.56537529264333e-05 282816.816548881
7.67297344697333e-05 275156.043416231
7.79030973681e-05 272896.619957441
7.8707419815e-05 269146.509283092
7.93225039713e-05 265504.187159583
7.96826933982e-05 261988.152482272
8.0190482274e-05 258624.619957441
9.93595104183e-05 246271.181276414
9.96115338976e-05 241587.29711738
0.0001013383994725 236741.990948784
0.000102775627007 234146.327799052
0.000104142115523 225513.976617019
0.0001055760732915 216715.787545298
0.0001271346248165 210047.524337203
0.000128170533344 207549.089936677
0.0001296658032005 199440.55132751
0.0001309227239555 186107.256498174
0.0001312628377554 171012.214179366
0.000132778786775 165747.468581401
0.000139017122603 162201.964579551
0.000140030996222 157482.710177516
0.000161864210675727 141828.699061826
0.00016361695023 140639.04542396
0.0001647859181715 136219.827603627
0.000166708480837 135073.20916862
0.000206367076476 130977.20916862
0.0002075592408526 122942.825624996
0.00020959627951575 107380.627505207
0.00021043711795 105622.610166555
0.000215839795109 104511.511831505
0.000218486581616 101739.351367437
0.000221478502515 97446.5011488516
0.000222819139759 94330.4121403406
0.000233317096064 92263.1641412656
0.000234430351084 89275.9636659096
0.0002373581483655 84286.8758920729
0.000238005333901286 79124.7694878455
0.0002403864397425 76146.7727827097
0.000243655907728 73514.3999837267
0.000261422112111 71741.6479828017
0.0002638372614615 69843.0855125755
0.000265363881059 68932.5049817041
0.000268249969741 68053.4963123738
0.0002713256933005 66688.8443381222
0.00034035304319725 62002.5557222732
0.000342829138345 58950.9109803232
0.000407766433344 51888.3308053483
0.0004099354345794 45132.4951725422
0.000421078262066 43721.2490566909
0.000422786674327 41222.0393344207
0.000479867118785 38163.1781691127
0.000483363282856 40083.1781691127
0.000497531146255286 35222.5131318682
0.000498172384842 34042.6995313612
0.00069860140118 33030.3936807582
0.000701836986693 31134.8810250614
0.0008877216977065 29671.3766903938
0.00090078833417725 27299.679203122
0.000906178962919857 22677.4389342099
0.000910468585032 21151.6165632459
0.000946374523019 18777.3132322953
0.0009803715917955 17601.2986852389
0.00101185457164667 15618.621837364
0.00103484144939 13823.2210769081
0.00103877776224 14481.3142766551
0.00110193295374 13129.8801513771
0.0011141376555 12539.9733511241
0.001160295898305 11423.6635603664
0.001260940236745 10880
0.00131180501709 10579.4937046998
0.00131695337261 10282.0802952397
0.00135348108296 9697.4556752799
0.00135851087529 9463.073758133
0.00145913469431 8948.7740166547
0.0016874048423 8033.5513429856
0.00169723382203 7751.1234106371
0.0017326181717 7253.5966595442
0.001849454865255 6591.4869830665
0.00185957730776667 5741.0989335429
0.00186588730191 5387.2327926977
0.00198582708167 4799.7801929393
0.00241314955181 4578.1861928242
0.00245187576688 4283.2327926977
0.00251679456459 4155.2327926977
0.00306595322722 3697.6214558629
0.00307740476944 3899.2327926977
0.00338655805440667 3186.9628631771
0.00340626702522 3010.0297927554
0.003545520928785 2734.709492761
0.00356141453672 2474.6195752042
0.003601981847475 2246.6121332869
0.004391029690795 2072.9364745392
0.00473438971969 1724.4076368528
0.00476834606068 2001.7576517077
0.00555048966464 1692.4076368528
0.00634597103084 1439.3311742023
0.0063816242189 1316.5958854162
0.0064201347306 1196.4665352114
0.00652480405925 506
0.00655479476002 684.8487414329
0.00891908090713 442
0.0128186847164 410
0.0217122569854667 130.56842503044
0.021862309007675 171.6992500145
0.0223981635367 102.86917501594
0.0233249075327 66
0.0236130554393 50
};
\label{\figlabel-line4}
\coordinate (legend) at (axis description cs:0.03,0.03);
\end{axis}

\matrix [matrix of nodes,
inner sep=1pt, row sep=1pt,cells={anchor=west},anchor={south west},at={(0.03,0.03)}, anchor=south west, draw=none, fill=none] at (legend) {
\ref{\figlabel-line0} {SL}\\
\ref{\figlabel-line1} {$\epsilon^{-2}\log(\epsilon^{-1})$}\\
\ref{\figlabel-line2} {ML}\\
\ref{\figlabel-line3} {$\epsilon^{-1}\log(\epsilon^{-1})^{4}$}\\
\ref{\figlabel-line4} {Adaptive ML}\\
};
\end{tikzpicture}
      \caption{$d=4$}
	\end{subfigure}
	\begin{subfigure}{0.5\textwidth}
      \renewcommand{\figlabel}{fig:work-est-vs-error-d6}
      % This file was created by matplotlib2tikz v0.6.10.
\begin{tikzpicture}

\begin{axis}[
xlabel={Max Error},
ylabel={Work Estimate},
xmin=1e-05, xmax=0.1,
ymin=10, ymax=10000000,
xmode=log,
ymode=log,
axis on top,
name=\figlabel,
width=\figurewidth,
height=\figureheight,
xtick={1e-06,1e-05,0.0001,0.001,0.01,0.1,1,10},
xticklabels={,$10^{-5}$,$10^{-4}$,$10^{-3}$,$10^{-2}$,$10^{-1}$,,},
ytick={1,10,100,1000,10000,100000,1000000,10000000,100000000},
yticklabels={,$10^{1}$,$10^{2}$,$10^{3}$,$10^{4}$,$10^{5}$,$10^{6}$,$10^{7}$,},
tick pos=both
]
\addplot [thick, black, opacity=0.4, dotted, mark=x, mark size=2, mark options={solid,fill opacity=0}, forget plot]
table {%
0.032164710232 69.1886323727
0.0323385638832 361.8947501008
0.051035633904 18.5754247591
};
\addplot [thick, black, opacity=0.4, dotted, mark=x, mark size=2, mark options={solid,fill opacity=0}, forget plot]
table {%
0.0215099418036 723.7895002021
0.0216397120011 4867.526024689
0.0217242797332 138.3772647455
0.0490792003595 37.1508495182
};
\addplot [thick, black, opacity=0.4, dotted, mark=x, mark size=2, mark options={solid,fill opacity=0}, forget plot]
table {%
0.0113282255642 1447.579000405
0.0114090379439 76914.110944276
0.0114164151911 9735.052049378
0.012577796492 276.754529491
0.0496575860038 74.3016990364
};
\addplot [thick, black, opacity=0.4, dotted, mark=x, mark size=2, mark options={solid,fill opacity=0}, forget plot]
table {%
0.00700020634185 2895.15800081
0.00701893900149 19470.104098755
0.00938574447436 553.509058982
0.0506262399478 148.603398073
};
\addplot [thick, black, opacity=0.4, dotted, mark=x, mark size=2, mark options={solid,fill opacity=0}, forget plot]
table {%
0.00339649414451 307656.44377673
0.00343469517396 38940.208197512
0.00358106706727 5790.316001612
0.00773531370794 1107.018117964
0.0517209057266 297.206796146
};
\addplot [thick, black, opacity=0.4, dotted, mark=x, mark size=2, mark options={solid,fill opacity=0}, forget plot]
table {%
0.0019931085648 615312.88755353
0.00206216284501 77880.41639506
0.0024145667152 11580.63200323
0.00747873832964 2214.036235931
0.0522163261965 594.413592291
};
\addplot [thick, black, opacity=0.4, dotted, mark=x, mark size=2, mark options={solid,fill opacity=0}, forget plot]
table {%
0.0009204159105695 1230625.77511105
0.00106861061652 155760.83279042
0.00177325026714 23161.26400645
0.00745826609476 4428.07247185
0.0526202481683 1188.82718458
};
\addplot [thick, black, opacity=0.4, dotted, mark=x, mark size=2, mark options={solid,fill opacity=0}, forget plot]
table {%
0.000493992319159 2461251.5502122
0.00073802650909 311521.6655798
0.00164834196358 46322.5280129
0.00749324983904 8856.14494371
0.05278678499 2377.65436916
};
\addplot [thick, black, opacity=0.4, dotted, mark=x, mark size=2, mark options={solid,fill opacity=0}, forget plot]
table {%
0.000233040511938 4922503.1004248
0.000598815200992 623043.3311597
0.00162332440566 92645.0560259
0.00752662028315 17712.28988747
0.0528903900769 4755.30873833
};
\addplot [thick, black, opacity=0.4, dotted, mark=x, mark size=2, mark options={solid,fill opacity=0}, forget plot]
table {%
0.0001165553314975 9845006.2008547
0.00056522288294 1246086.66232
0.0016256597444 185290.1120518
0.0075444453459 35424.5797748
0.0529371274145 9510.61747666
};
\addplot [thick, black, dash pattern=on 1pt off 3pt on 3pt off 3pt]
table {%
1e-05 4247662241.04439
1.09749876549306e-05 3497986977.08867
1.20450354025878e-05 2880431665.17057
1.32194114846603e-05 2371743206.54259
1.45082877849594e-05 1952755938.22796
1.59228279334109e-05 1607673978.99932
1.74752840000768e-05 1323479440.72066
1.91791026167249e-05 1089444452.74566
2.10490414451202e-05 896728807.049833
2.31012970008316e-05 738048216.724462
2.53536449397011e-05 607400808.327169
2.78255940220713e-05 499841636.722683
3.05385550883341e-05 411296799.815261
3.35160265093884e-05 338410206.292905
3.67837977182863e-05 278417266.910374
4.03701725859655e-05 229040784.107569
4.43062145758388e-05 188405143.173012
4.86260158006535e-05 154965591.484351
5.33669923120631e-05 127449955.976938
5.85702081805666e-05 104810613.857127
6.42807311728432e-05 86184914.9752186
7.05480231071865e-05 70862570.4671783
7.74263682681127e-05 58258783.0326291
8.49753435908644e-05 47892109.2486144
9.3260334688322e-05 39366221.6346528
0.000102353102189903 32354884.3931478
0.000112332403297803 26589577.3019887
0.000123284673944207 21849301.6303237
0.000135304777457981 17952183.8936959
0.000148496826225446 14748560.8208886
0.000162975083462064 12115284.5958094
0.000178864952905744 9951033.34513978
0.000196304065004027 8172449.68313259
0.000215443469003188 6710961.31467271
0.000236448941264541 5510163.40417582
0.000259502421139973 4523663.60391576
0.00028480358684358 3713308.09492341
0.000312571584968824 3047721.38134859
0.000343046928631492 2501104.43490921
0.000376493580679247 2052245.55481198
0.000413201240011534 1683706.35722152
0.000453487850812858 1381151.9394621
0.000497702356433211 1132799.72700641
0.000546227721768434 928966.011483401
0.000599484250318941 761692.894845177
0.000657933224657568 624441.408115537
0.000722080901838546 511839.087892472
0.000792482898353917 419472.36484652
0.000869749002617784 343715.824014196
0.000954548456661834 281591.801144955
0.00104761575278967 230654.935799257
0.00114975699539774 188897.254048057
0.00126185688306602 154670.137523375
0.00138488637139387 126620.180901583
0.00151991108295293 103636.471121581
0.00166810053720006 84807.2588949574
0.00183073828029537 69384.3529489508
0.00200923300256505 56753.863623767
0.00220513073990305 46412.166180794
0.00242012826479438 37946.154738533
0.00265608778294669 31017.0227732859
0.00291505306282518 25346.941888629
0.00319926713779738 20708.1222479161
0.00351119173421513 16913.8299395077
0.00385352859371053 13811.0121139179
0.0042292428743895 11274.2428841199
0.00464158883361278 9200.75409259546
0.00509413801481638 7506.35707842349
0.00559081018251223 6122.09613558111
0.00613590727341317 4991.50276527335
0.00673415065775082 4068.34318140864
0.00739072203352578 3314.77072690895
0.00811130830789687 2699.81063793624
0.00890215085445039 2198.11756094486
0.00977009957299226 1788.95688358251
0.0107226722201032 1455.36969602644
0.01176811952435 1183.48839254203
0.0129154966501488 961.97583202377
0.0141747416292681 781.565829749759
0.0155567614393047 634.686738629896
0.0170735264747069 515.153151493922
0.0187381742286039 417.913443582548
0.0205651230834865 338.843080883435
0.0225701971963392 274.575431198029
0.0247707635599171 222.3633014388
0.0271858824273294 179.965644661379
0.0298364724028334 145.554881410123
0.0327454916287773 117.641101272089
0.0359381366380463 95.0100842755018
0.0394420605943766 76.6726343707392
0.0432876128108306 61.8231704078335
0.047508101621028 49.805891606787
0.0521400828799969 40.0871391486319
0.0572236765935022 32.2328252190036
0.0628029144183425 25.8900054822274
0.068926121043497 20.7718386545672
0.075646332755463 16.6453142300011
0.0830217568131974 13.3212419476412
0.0911162756115489 10.6460887542957
0.1 8.49532448208878
};
\label{\figlabel-line1}
\addplot [thick, black, dashed]
table {%
1e-05 4976638.83475196
1.09749876549306e-05 4389724.41793334
1.20450354025878e-05 3870999.29773914
1.32194114846603e-05 3412649.84895434
1.45082877849594e-05 3007746.43371076
1.59228279334109e-05 2650144.56302362
1.74752840000768e-05 2334396.99003651
1.91791026167249e-05 2055675.53797336
2.10490414451202e-05 1809701.59565616
2.31012970008316e-05 1592684.32933987
2.53536449397011e-05 1401265.76303051
2.78255940220713e-05 1232471.97172266
3.05385550883341e-05 1083669.71430914
3.35160265093884e-05 952527.906343535
3.67837977182863e-05 836983.398328014
4.03701725859655e-05 735210.583604189
4.43062145758388e-05 645594.412004739
4.86260158006535e-05 566706.431857217
5.33669923120631e-05 497283.524325417
5.85702081805666e-05 436209.030970735
6.42807311728432e-05 382496.0082999
7.05480231071865e-05 335272.372369246
7.74263682681127e-05 293767.722625427
8.49753435908644e-05 257301.657423333
9.3260334688322e-05 225273.414381845
0.000102353102189903 197152.68719205
0.000112332403297803 172471.486926127
0.000123284673944207 150816.930527394
0.000135304777457981 131824.852188607
0.000148496826225446 115174.144920958
0.000162975083462064 100581.749936155
0.000178864952905744 87798.220647335
0.000196304065004027 76603.7962652827
0.000215443469003188 66804.9272351445
0.000236448941264541 58231.2012239875
0.000259502421139973 50732.6241191753
0.00028480358684358 44177.2156099519
0.000312571584968824 38448.8834697236
0.000343046928631492 33445.5446966009
0.000376493580679247 29077.4652603306
0.000413201240011534 25265.7933942708
0.000453487850812858 21941.2642056343
0.000497702356433211 19043.0558951114
0.000546227721768434 16517.7801131789
0.000599484250318941 14318.5909660502
0.000657933224657568 12404.3989470836
0.000722080901838546 10739.1776342485
0.000792482898353917 9291.35238293421
0.000869749002617784 8033.26147557669
0.000954548456661834 6940.68128267995
0.00104761575278967 5992.40795933928
0.00114975699539774 5169.88906113031
0.00126185688306602 4456.89922548041
0.00138488637139387 3839.25474030946
0.00151991108295293 3304.56242052167
0.00166810053720006 2841.99874348366
0.00183073828029537 2442.11566461068
0.00200923300256505 2096.66995042488
0.00220513073990305 1798.47323501001
0.00242012826479438 1541.26033205368
0.00265608778294669 1319.57362342661
0.00291505306282518 1128.66160075072
0.00319926713779738 964.389862429069
0.00351119173421513 823.16306850415
0.00385352859371053 701.856532452069
0.0042292428743895 597.756285251255
0.00464158883361278 508.506585129562
0.00509413801481638 432.063968372926
0.00559081018251223 366.657044317378
0.00613590727341317 310.751332780663
0.00673415065775082 263.018526170884
0.00739072203352578 222.309632624112
0.00811130830789687 187.631521910749
0.00890215085445039 158.126453522158
0.00977009957299226 133.054217199753
0.0107226722201032 111.776560991269
0.01176811952435 93.7436214154563
0.0129154966501488 78.4821051085996
0.0141747416292681 65.5850019643097
0.0155567614393047 54.7026367488334
0.0170735264747069 45.534889908531
0.0187381742286039 37.8244391651294
0.0205651230834865 31.3508918548465
0.0225701971963392 25.9256941073046
0.0247707635599171 21.3877171425766
0.0271858824273294 17.5994334223059
0.0298364724028334 14.4436063290594
0.0327454916287773 11.8204266483465
0.0359381366380463 9.64503755021114
0.0394420605943766 7.84539715335559
0.0432876128108306 6.36043422907291
0.047508101621028 5.13845827487981
0.0521400828799969 4.13579015562423
0.0572236765935022 3.31558385850608
0.0628029144183425 2.64681371322598
0.068926121043497 2.10340475573597
0.075646332755463 1.66348682224711
0.0830217568131974 1.30875550069652
0.0911162756115489 1.02392528506243
0.1 0.796262213560313
};
\label{\figlabel-line3}
\addplot [ultra thick, blue]
table {%
0.0001165553314975 9845006.2008547
0.000232761587034 4922503.1004248
0.000493564532628 2461251.5502122
0.00056522288294 1246086.66232
0.000598815200992 623043.3311597
0.00073802650909 311521.6655798
0.00106861061652 155760.83279042
0.00162332440566 92645.0560259
0.00164834196358 46322.5280129
0.00177325026714 23161.26400645
0.0024145667152 11580.63200323
0.00358106706727 5790.316001612
0.00699311153206 2895.15800081
0.00747873832964 2214.036235931
0.00773531370794 1107.018117964
0.00938574447436 553.509058982
0.012577796492 276.754529491
0.0217242797332 138.3772647455
0.032164710232 69.1886323727
0.0490792003595 37.1508495182
0.051035633904 18.5754247591
};
\label{\figlabel-line0}
\addplot [ultra thick, green!50.0!black]
table {%
2.77827474765e-05 1152962.3334012
5.75099002429e-05 493514.142529296
0.000112277003268 163790.047094071
0.000189863742175 81647.3554463046
0.000630736910547 39496.886904104
0.000919070948055 18531.561945883
0.00138045231564 8048.8994667681
0.00252201901249 3996.3954117929
0.00355017263387 1817.2503308473
0.00893802979483 727.6777903736
0.0112029278899 329.2445790003
0.0491479964041 55.7262742773
0.0493803165062 130.0279733137
0.051035633904 18.5754247591
};
\label{\figlabel-line2}
\addplot [ultra thick, red]
table {%
1.682905020824e-05 1465343.06887342
1.70492165828667e-05 1432620.32537575
1.70829609509e-05 1383076.82631634
1.727061807898e-05 1273207.1080593
1.736302509678e-05 1227807.87249397
1.7799389140975e-05 1193915.68002601
1.786538869205e-05 1168566.43935378
1.82194032139e-05 1148703.29900138
1.856508115925e-05 1125601.46645678
1.87717791392e-05 1051628.68486908
1.89474020089e-05 1018559.05929778
1.90675213093e-05 994661.455494854
2.3235792567e-05 978123.471502334
2.35003881724667e-05 934388.977361361
2.40564354213e-05 924006.324762458
2.48313255168e-05 915117.538082155
2.53234413125333e-05 891141.131022587
2.611400090278e-05 851067.527495727
3.02262617698333e-05 830184.235922585
3.07481399056e-05 820190.498320702
3.11307156634e-05 777015.878295156
3.124116675068e-05 795892.895903336
3.29682610306e-05 768823.878295156
3.30779291260438e-05 698274.344493606
3.33397167697556e-05 644783.705256492
3.38128293446e-05 618603.421923612
3.922086420675e-05 597321.30846969
4.040411529026e-05 570824.765072729
4.050032614895e-05 558264.134559534
4.11935211211143e-05 524204.698363391
4.21888423596e-05 519685.851445811
4.277386761995e-05 513156.533749525
4.3347078465e-05 508088.999329637
4.36207134424e-05 499195.685649433
4.39598908302e-05 486914.172601508
4.42634295686e-05 479882.103246832
4.47561574026e-05 473155.038197176
4.66864988026e-05 468710.644857017
4.8744441424e-05 464349.068050405
5.106364617055e-05 447177.667176095
5.19660353981e-05 434713.311142095
5.22337797112e-05 422764.509240635
5.25823878421e-05 411440.792733585
5.7551844229275e-05 400940.519333712
5.81883983372e-05 390411.028137782
6.52066751877778e-05 358549.069851966
6.55487280999769e-05 321430.248989756
6.61412936914e-05 317427.802184936
6.64338127461e-05 300217.624933135
6.69693097881e-05 309656.133737205
7.435644045065e-05 293348.44634553
7.44428594431e-05 290593.318500134
7.56537529264333e-05 282816.816548881
7.67297344697333e-05 275156.043416231
7.79030973681e-05 272896.619957441
7.8707419815e-05 269146.509283092
7.93225039713e-05 265504.187159583
7.96826933982e-05 261988.152482272
8.0190482274e-05 258624.619957441
9.93595104183e-05 246271.181276414
9.96115338976e-05 241587.29711738
0.0001013383994725 236741.990948784
0.000102775627007 234146.327799052
0.000104142115523 225513.976617019
0.0001055760732915 216715.787545298
0.0001271346248165 210047.524337203
0.000128170533344 207549.089936677
0.0001296658032005 199440.55132751
0.0001309227239555 186107.256498174
0.0001312628377554 171012.214179366
0.000132778786775 165747.468581401
0.000139017122603 162201.964579551
0.000140030996222 157482.710177516
0.000161864210675727 141828.699061826
0.00016361695023 140639.04542396
0.0001647859181715 136219.827603627
0.000166708480837 135073.20916862
0.000206367076476 130977.20916862
0.0002075592408526 122942.825624996
0.00020959627951575 107380.627505207
0.00021043711795 105622.610166555
0.000215839795109 104511.511831505
0.000218486581616 101739.351367437
0.000221478502515 97446.5011488516
0.000222819139759 94330.4121403406
0.000233317096064 92263.1641412656
0.000234430351084 89275.9636659096
0.0002373581483655 84286.8758920729
0.000238005333901286 79124.7694878455
0.0002403864397425 76146.7727827097
0.000243655907728 73514.3999837267
0.000261422112111 71741.6479828017
0.0002638372614615 69843.0855125755
0.000265363881059 68932.5049817041
0.000268249969741 68053.4963123738
0.0002713256933005 66688.8443381222
0.00034035304319725 62002.5557222732
0.000342829138345 58950.9109803232
0.000407766433344 51888.3308053483
0.0004099354345794 45132.4951725422
0.000421078262066 43721.2490566909
0.000422786674327 41222.0393344207
0.000479867118785 38163.1781691127
0.000483363282856 40083.1781691127
0.000497531146255286 35222.5131318682
0.000498172384842 34042.6995313612
0.00069860140118 33030.3936807582
0.000701836986693 31134.8810250614
0.0008877216977065 29671.3766903938
0.00090078833417725 27299.679203122
0.000906178962919857 22677.4389342099
0.000910468585032 21151.6165632459
0.000946374523019 18777.3132322953
0.0009803715917955 17601.2986852389
0.00101185457164667 15618.621837364
0.00103484144939 13823.2210769081
0.00103877776224 14481.3142766551
0.00110193295374 13129.8801513771
0.0011141376555 12539.9733511241
0.001160295898305 11423.6635603664
0.001260940236745 10880
0.00131180501709 10579.4937046998
0.00131695337261 10282.0802952397
0.00135348108296 9697.4556752799
0.00135851087529 9463.073758133
0.00145913469431 8948.7740166547
0.0016874048423 8033.5513429856
0.00169723382203 7751.1234106371
0.0017326181717 7253.5966595442
0.001849454865255 6591.4869830665
0.00185957730776667 5741.0989335429
0.00186588730191 5387.2327926977
0.00198582708167 4799.7801929393
0.00241314955181 4578.1861928242
0.00245187576688 4283.2327926977
0.00251679456459 4155.2327926977
0.00306595322722 3697.6214558629
0.00307740476944 3899.2327926977
0.00338655805440667 3186.9628631771
0.00340626702522 3010.0297927554
0.003545520928785 2734.709492761
0.00356141453672 2474.6195752042
0.003601981847475 2246.6121332869
0.004391029690795 2072.9364745392
0.00473438971969 1724.4076368528
0.00476834606068 2001.7576517077
0.00555048966464 1692.4076368528
0.00634597103084 1439.3311742023
0.0063816242189 1316.5958854162
0.0064201347306 1196.4665352114
0.00652480405925 506
0.00655479476002 684.8487414329
0.00891908090713 442
0.0128186847164 410
0.0217122569854667 130.56842503044
0.021862309007675 171.6992500145
0.0223981635367 102.86917501594
0.0233249075327 66
0.0236130554393 50
};
\label{\figlabel-line4}
\coordinate (legend) at (axis description cs:0.03,0.03);
\end{axis}

\matrix [matrix of nodes,
inner sep=1pt, row sep=1pt,cells={anchor=west},anchor={south west},at={(0.03,0.03)}, anchor=south west, draw=none, fill=none] at (legend) {
\ref{\figlabel-line0} {SL}\\
\ref{\figlabel-line1} {$\epsilon^{-2}\log(\epsilon^{-1})$}\\
\ref{\figlabel-line2} {ML}\\
\ref{\figlabel-line3} {$\epsilon^{-1}\log(\epsilon^{-1})^{4}$}\\
\ref{\figlabel-line4} {Adaptive ML}\\
};
\end{tikzpicture}
      \caption{$d=6$}
	\end{subfigure}
	\caption{$L^2([-1,1]^d)$-error, approximated using
      \Cref{eq:l2-mc-error} vs work estimate \Cref{eq:workdef} of
      single-level (SL), multilevel (ML) and adaptive ML (ML adaptive)
      methods for a linear elliptic PDE with non-smooth parameter
      dependence. The grey dotted lines are the complexity curves of
      different runs of the single-level, each with a different PDE
      discretization level. The single-level (SL) complexity curve is
      then the lower envelope of all single-level complexity
      curves. This figure shows the agreement of the numerical results
      with the theoretical rates.}
	\label{fig:kink-work}
  \end{figure}


  \begin{figure}
	\centering
    \begin{subfigure}{0.49\textwidth}
      \renewcommand{\figlabel}{fig:total-time-vs-error-d2}
      % This file was created by matplotlib2tikz v0.6.10.
\begin{tikzpicture}

\begin{axis}[
xlabel={Max Error},
ylabel={Time [s.]},
xmin=0.0001, xmax=0.1,
ymin=0.001, ymax=10000,
xmode=log,
ymode=log,
axis on top,
name=\figlabel,
width=\figurewidth,
height=\figureheight,
xtick={1e-05,0.0001,0.001,0.01,0.1,1,10},
xticklabels={,$10^{-4}$,$10^{-3}$,$10^{-2}$,$10^{-1}$,,},
ytick={0.0001,0.001,0.01,0.1,1,10,100,1000,10000,100000},
yticklabels={,$10^{-3}$,$10^{-2}$,$10^{-1}$,$10^{0}$,$10^{1}$,$10^{2}$,$10^{3}$,$10^{4}$,},
tick pos=both
]
\addplot [thick, black, opacity=0.4, dotted, mark=x, mark size=2, mark options={solid,fill opacity=0}, forget plot]
table {%
0.0219923706361 0.280572
0.0231485452804 0.032683
0.0473996127045 0.0073700000000001
};
\addplot [thick, black, opacity=0.4, dotted, mark=x, mark size=2, mark options={solid,fill opacity=0}, forget plot]
table {%
0.0146468171491 56.319003
0.0146825404421 0.281625
0.0171486181992 0.0369269999999999
0.0456595617549 0.00841200000000009
};
\addplot [thick, black, opacity=0.4, dotted, mark=x, mark size=2, mark options={solid,fill opacity=0}, forget plot]
table {%
0.00773597291035 71.243986
0.0079395211684 0.479064
0.0129739708107 0.0495500000000002
0.045076536669 0.00707199999999997
};
\addplot [thick, black, opacity=0.4, dotted, mark=x, mark size=2, mark options={solid,fill opacity=0}, forget plot]
table {%
0.00475918310523 52.994155
0.0051763040426 0.503328
0.0120103618342 0.0421529999999999
0.0451498063531 0.00845800000000008
};
\addplot [thick, black, opacity=0.4, dotted, mark=x, mark size=2, mark options={solid,fill opacity=0}, forget plot]
table {%
0.00232937038259 96.979911
0.00321933670459 1.083645
0.0117499134021 0.110281
0.045355625829 0.0184980000000006
};
\addplot [thick, black, opacity=0.4, dotted, mark=x, mark size=2, mark options={solid,fill opacity=0}, forget plot]
table {%
0.00139560009836 115.91719
0.00268287982911 1.983264
0.011787563253 0.200733
0.0454708567658 0.0245000000000002
};
\addplot [thick, black, opacity=0.4, dotted, mark=x, mark size=2, mark options={solid,fill opacity=0}, forget plot]
table {%
0.000712175432305 164.04346
0.00244887368137 4.321099
0.0118688266345 0.397337
0.0455720926749 0.0873309999999998
};
\addplot [thick, black, opacity=0.4, dotted, mark=x, mark size=2, mark options={solid,fill opacity=0}, forget plot]
table {%
0.000477321676295 285.061344
0.0024110537785 10.507983
0.0119135687612 0.877356000000001
0.0456154902222 0.140544
};
\addplot [thick, black, opacity=0.4, dotted, mark=x, mark size=2, mark options={solid,fill opacity=0}, forget plot]
table {%
0.0003730803792 640.047029
0.00240481031792 31.319652
0.0119444028253 2.397552
0.0456429401838 0.375838
};
\addplot [thick, black, opacity=0.4, dotted, mark=x, mark size=2, mark options={solid,fill opacity=0}, forget plot]
table {%
0.000346112906001 1759.361261
0.00240622839308 95.844415
0.011959031944 7.988817
0.0456554324994 1.096945
};
\addplot [thick, black, dash pattern=on 1pt off 3pt on 3pt off 3pt]
table {%
0.0001 423239.328877998
0.000107226722201032 340701.6921795
0.000114975699539774 274244.076807414
0.000123284673944207 220736.68873795
0.000132194114846603 177658.377646415
0.000141747416292681 142978.373492576
0.000151991108295294 115060.984459645
0.000162975083462064 92588.8015041979
0.000174752840000768 74500.818434034
0.000187381742286039 59942.5727609829
0.000200923300256505 48225.9739746426
0.000215443469003188 38796.9384804832
0.000231012970008316 31209.3153141528
0.000247707635599171 25103.8808769865
0.000265608778294669 20191.4180395954
0.00028480358684358 16239.0860817427
0.000305385550883342 13059.4419882983
0.000327454916287773 10501.5977879835
0.000351119173421514 8444.09869718282
0.000376493580679247 6789.18748632729
0.000403701725859656 5458.18548717894
0.000432876128108306 4387.7730419552
0.000464158883361278 3526.99440731665
0.000497702356433211 2834.84614103554
0.000533669923120631 2278.33540732314
0.000572236765935022 1830.91672053959
0.000613590727341318 1471.23343988438
0.000657933224657568 1182.1046627339
0.000705480231071865 949.709713000246
0.000756463327554629 762.93172439459
0.000811130830789688 612.829312925118
0.000869749002617784 492.211369825956
0.000932603346883221 395.294868644612
0.001 317.429496658499
0.00107226722201032 254.876075065581
0.00114975699539774 204.628272694764
0.00123284673944207 164.269163711963
0.00132194114846603 131.855827159449
0.00141747416292681 105.826512663797
0.00151991108295294 84.9259647202143
0.00162975083462064 68.1453579070896
0.00174752840000769 54.6739877217507
0.00187381742286038 43.8604190934022
0.00200923300256505 35.1812433093705
0.00215443469003189 28.2159552585332
0.00231012970008316 22.6267536027607
0.00247707635599171 18.1423004657214
0.00265608778294669 14.5446655369967
0.0028480358684358 11.658831033046
0.00305385550883342 9.34425590542035
0.00327454916287773 7.48809581404042
0.00351119173421513 5.99975433747202
0.00376493580679247 4.80650441509896
0.00403701725859656 3.84997012042086
0.00432876128108306 3.08329997542798
0.00464158883361278 2.46889608512166
0.00497702356433211 1.97658996989634
0.00533669923120632 1.58217736619662
0.00572236765935022 1.26624147027971
0.00613590727341317 1.01320793501471
0.00657933224657568 0.810586054446102
0.00705480231071864 0.648359515605939
0.00756463327554629 0.518497288423507
0.00811130830789687 0.414561005802286
0.00869749002617784 0.33138983315015
0.00932603346883221 0.26484756199189
0.01 0.211619664438999
0.0107226722201032 0.169050457951661
0.0114975699539774 0.135012468582112
0.0123284673944207 0.107801638685976
0.0132194114846603 0.0860532766724825
0.0141747416292681 0.068674651835017
0.0151991108295293 0.0547909449807835
0.0162975083462065 0.0437019143099813
0.0174752840000768 0.0348471570094675
0.0187381742286039 0.0277782654258214
0.0200923300256505 0.0221365126440983
0.0215443469003188 0.0176349720365833
0.0231012970008316 0.0140441918913688
0.0247707635599171 0.0111807200544562
0.0265608778294669 0.008897913034398
0.028480358684358 0.00707857598434937
0.0305385550883342 0.00562906982254237
0.0327454916287773 0.00447459384009732
0.0351119173421513 0.00355540997776119
0.0376493580679247 0.00282382134387065
0.0403701725859656 0.00224175475366278
0.0432876128108306 0.00177882690890076
0.0464158883361278 0.00141079776292666
0.0497702356433211 0.00111833379875714
0.0533669923120631 0.000886019325070111
0.0572236765935022 0.000701566220019842
0.0613590727341318 0.000555182430145048
0.0657933224657568 0.000439067446158306
0.0705480231071865 0.000347009318211629
0.0756463327554629 0.000274062852452426
0.0811130830789687 0.000216292698679454
0.0869749002617784 0.000170568296474342
0.093260334688322 0.000134400255339168
0.1 0.0001058098322195
};
\label{\figlabel-line1}
\addplot [thick, black, dashed]
table {%
0.0001 2958.06499642139
0.000107226722201032 2533.94031177136
0.000114975699539774 2170.37327630492
0.000123284673944207 1858.75045876355
0.000132194114846603 1591.67922140058
0.000141747416292681 1362.81526290182
0.000151991108295294 1166.71444949076
0.000162975083462064 998.705523258718
0.000174752840000768 854.780754519427
0.000187381742286039 731.502015994051
0.000200923300256505 625.920110191836
0.000215443469003188 535.505485474149
0.000231012970008316 458.088737871722
0.000247707635599171 391.809520703052
0.000265608778294669 335.072677525486
0.00028480358684358 286.5105803382
0.000305385550883342 244.950798032762
0.000327454916287773 209.388343110538
0.000351119173421514 178.961850459888
0.000376493580679247 152.933132922859
0.000403701725859656 130.669636556367
0.000432876128108306 111.629385693204
0.000464158883361278 95.3480656700718
0.000497702356433211 81.4279407355213
0.000533669923120631 69.5283473174526
0.000572236765935022 59.3575394960833
0.000613590727341318 50.6656950356365
0.000657933224657568 43.2389174002405
0.000705480231071865 36.8940924392158
0.000756463327554629 31.4744784094651
0.000811130830789688 26.8459251684521
0.000869749002617784 22.8936331161756
0.000932603346883221 19.5193751289983
0.001 16.6391156048703
0.00107226722201032 14.1809700799791
0.00114975699539774 12.083456897487
0.00123284673944207 10.2939992956943
0.00132194114846603 8.76764219552019
0.00141747416292681 7.46595304296641
0.00151991108295294 6.35608041927789
0.00162975083462064 5.40994787127154
0.00174752840000769 4.6035636239434
0.00187381742286038 3.91642959187773
0.00200923300256505 3.3310354695173
0.00215443469003189 2.83242570829301
0.00231012970008316 2.40782892843824
0.00247707635599171 2.04634080581864
0.00265608778294669 1.73865275432465
0.0028480358684358 1.47681982238756
0.00305385550883342 1.25406216382855
0.00327454916287773 1.06459525071853
0.00351119173421513 0.903484688263565
0.00376493580679247 0.766522085289607
0.00403701725859656 0.650118942720255
0.00432876128108306 0.551215958572727
0.00464158883361278 0.467205521783352
0.00497702356433211 0.395865487491264
0.00533669923120632 0.335302600875061
0.00572236765935022 0.283904171788411
0.00613590727341317 0.240296803884752
0.00657933224657568 0.203311154469702
0.00705480231071864 0.171951849099563
0.00756463327554629 0.145371801495314
0.00811130830789687 0.122850297699924
0.00869749002617784 0.103774296176057
0.00932603346883221 0.0876224749540735
0.01 0.0739516249105348
0.0107226722201032 0.0623850464263971
0.0114975699539774 0.0526026564481951
0.0123284673944207 0.0443325555605584
0.0132194114846603 0.0373438411075674
0.0141747416292681 0.0314404835591648
0.0151991108295293 0.0264561099657769
0.0162975083462065 0.022249561129367
0.0174752840000768 0.0187011085991373
0.0187381742286039 0.0157092342518651
0.0200923300256505 0.0131878894488149
0.0215443469003188 0.0110641629230196
0.0231012970008316 0.00927629694190239
0.0247707635599171 0.00777200016704239
0.0265608778294669 0.00650701321480119
0.028480358684358 0.00544388939630109
0.0305385550883342 0.00455095864359323
0.0327454916287773 0.00380144734826769
0.0351119173421513 0.00317273086687891
0.0376493580679247 0.00264569888500741
0.0403701725859656 0.00220421676461474
0.0432876128108306 0.00183466850116662
0.0464158883361278 0.00152556905072115
0.0497702356433211 0.00126723560659859
0.0533669923120631 0.00105150895634419
0.0572236765935022 0.000871517371760993
0.0613590727341318 0.000721476611401018
0.0657933224657568 0.000596520574746176
0.0705480231071865 0.00049255796488746
0.0756463327554629 0.000406151012749154
0.0811130830789687 0.000334412908672761
0.0869749002617784 0.000274921091729391
0.093260334688322 0.000225643976491221
0.1 0.000184879062276337
};
\label{\figlabel-line3}
\addplot [ultra thick, blue]
table {%
0.000346112906001 1759.361261
0.0003730803792 640.047029
0.000477321676295 285.061344
0.000712175432305 164.04346
0.00139560009836 115.91719
0.00232937038259 96.979911
0.00240481031792 31.319652
0.0024110537785 10.507983
0.00244887368137 4.321099
0.00268287982911 1.983264
0.00321933670459 1.083645
0.0051763040426 0.503328
0.0079395211684 0.479064
0.0117499134021 0.110281
0.0120103618342 0.0421529999999999
0.0171486181992 0.0369269999999999
0.0231485452804 0.032683
0.045076536669 0.00707199999999997
};
\label{\figlabel-line0}
\addplot [ultra thick, green!50.0!black]
table {%
0.000381566361572 161.411429
0.000451865796974 110.254917
0.000585790565792 36.67568
0.0024728679842 2.408645
0.00266604551364 1.18852
0.00311350959574 0.573986
0.0113418394376 0.204287
0.0115006079295 0.127699000000001
0.0124560343379 0.0566240000000005
0.0443133452182 0.024389
0.045512938072 0.0135700000000007
0.0473996127045 0.00540899999999978
};
\label{\figlabel-line2}
\addplot [ultra thick, red]
table {%
0.0001960709611192 737.808295999975
0.000196854020298 727.989599999988
0.00019804267732275 703.357109999984
0.000198603139171 692.055464000021
0.000199217701188 686.303994000003
0.000201205139752 666.920267000025
0.000202277559396667 658.144116999918
0.00020299445505 687.917508999979
0.000204172098731 616.318218999908
0.000205601422325667 668.136470999978
0.00020695197751575 650.707283999933
0.000207189973371 640.80303899992
0.000208255156965 640.700297999881
0.000209293310891 633.02301599995
0.000210497834136 626.100400999976
0.0002110656993055 615.800344999896
0.0002121289537415 598.275698999934
0.00021355474089 534.704004999994
0.0002144376853995 473.763580000073
0.000214863960154667 468.744607999966
0.0002162747768775 453.037967999991
0.000217025117762 404.578728000031
0.000218028211424 418.020339000018
0.000218802922344 416.438449999996
0.000219540432973 397.53161
0.000220985433509333 389.251668000006
0.000222027028759 384.714124000029
0.000223273544902 378.819726000025
0.000223641710546 369.962953999996
0.0002246833374775 360.460670999958
0.000226459466349 355.004198999997
0.0002278963871455 349.150304999982
0.000228448407873 345.759503999974
0.0002295748511238 328.664476999985
0.000230658385783 343.060018999947
0.000231432388934 325.868679999957
0.000232840166556667 316.403130999986
0.000233845033977 312.544023999975
0.000235061101871 309.696189999997
0.000236289876817667 275.996198999973
0.000237021541556 270.629156999968
0.000237641899166 264.188015999958
0.000238861588659333 258.673287999934
0.000240236713054667 251.762497999925
0.000241400224986 238.172312999961
0.0002423824746896 193.238509999997
0.000243209009407945 171.328636999989
0.000244510391002583 170.308438000012
0.000245448292627889 169.713555999993
0.0002514116945725 168.710344999979
0.000252312856879 165.042215999967
0.000253078487053067 160.130260999989
0.000254301324113 252.070107000016
0.000255685449247 259.965375000005
0.00025664855162 259.185443000016
0.000258033353835 252.979473000021
0.000259258378160625 156.902840000011
0.000259564185494 190.801435000003
0.000260665249915 158.368874000003
0.000262555296169 155.809023000001
0.000271821357645 156.199467000004
0.000273146157854 194.731583000009
0.0002745402101 204.407146000006
0.000275706888827 204.952800000008
0.000279031961344 211.663478000002
0.000281613775824333 176.205423999999
0.0002825831778344 143.320203000015
0.0002873768395395 125.370210000013
0.000290928390999 118.915649000011
0.000291518843047429 109.903805000014
0.000292507597082333 107.819865000021
0.000295409021421 106.52829200002
0.000296831862402 104.366019000013
0.000299747804516 97.8695310000253
0.00030051986085475 93.9857490000245
0.000302247549054 94.8908050000132
0.000303130251286 86.7264560000199
0.000304198980762 99.4044830000199
0.00031196363135225 82.4881840000194
0.000312784552774 82.4365290000147
0.000316057226580333 81.3968730000047
0.000322939430171 90.5538970000303
0.000325451023899 90.9088230000175
0.000328297896535 91.6993660000185
0.000329646255921 89.7327080000091
0.000331620727146667 87.6943980000138
0.000333927321716 88.7284220000152
0.000335208690935 84.826789000023
0.000337222825823 92.6448950000154
0.0003404374447565 89.2392550000164
0.0003417188045385 86.8321870000123
0.000343748703051 86.5258290000134
0.000346811421293 85.7166620000158
0.000347592665874 79.7095380000142
0.000349832932715333 73.6484740000138
0.000350657361504 76.5035010000179
0.000351790949572 68.4041140000159
0.000355189775868 76.8232230000181
0.0003567716952615 68.2209290000133
0.000359228010425 65.8621950000148
0.000362370974331 67.5965520000141
0.0003638470968518 73.0185990000078
0.0003654546503845 70.5382170000008
0.00036716586277 72.8126120000021
0.000368858490075 69.998957000004
0.0003703217690545 68.9518970000041
0.0003719916421635 61.0345159999993
0.00037341143468 69.1439410000008
0.000375017896589 59.392171
0.000376239087398 57.1620410000065
0.000380563621461 53.3604929999961
0.000383259838497 49.715608999994
0.0003853494346595 50.4747429999948
0.000386195906104 56.6000979999977
0.000396178762489 58.9059259999991
0.000397553316306 53.7578409999983
0.00040059381862 52.0624370000028
0.000403765067384 49.1649820000026
0.000406319199868 50.4132580000056
0.000415648338658 51.5216580000047
0.0004223096414522 50.2579040000042
0.000423775785566 47.2688190000095
0.000425732040548 45.5429130000098
0.0004282857708295 44.9886620000121
0.000430738152901 43.9581370000065
0.000432783000265 45.0125630000089
0.00043434641816 40.9138830000116
0.000437876509574 41.9253240000112
0.000439393788988333 38.1977680000105
0.000442367915847333 32.9127590000112
0.000443522961622 32.0322690000144
0.000445484308653529 31.1379250000105
0.00044702992839225 30.9396380000126
0.0004495685620275 29.7996620000118
0.000451297117691429 29.5381790000138
0.000453790036123 29.5791800000111
0.000455738948988 29.3170920000107
0.000457718227683 30.19855900001
0.000459946523051 28.6124760000117
0.000460720303909 28.5383980000093
0.0004780664734075 28.5216040000101
0.000504233355798 33.163295000009
0.0005060332322185 28.3121280000055
0.000509468836734 27.9349580000053
0.000509686034308 27.935909000004
0.000512997822149 28.0319760000057
0.000544860367978 27.9930950000062
0.000567859081628 27.8581090000062
0.000569800301941375 27.8265840000077
0.000596597141971 27.870156000005
0.000599072725923 28.9226660000077
0.0006276311424835 27.9553890000075
0.000699834225015286 28.0023080000074
0.0007016768976054 26.8475090000061
0.000704293847567667 27.5996560000036
0.000828017262612 26.2921910000051
0.000829293815150857 25.6969290000011
0.0008330600142377 25.9733319999987
0.000837256138046875 28.1570379999976
0.000840622521942857 26.4580189999975
0.000856196724845 27.4307909999971
0.0008590828447325 28.2941499999961
0.0009101524214365 26.8135270000012
0.000914848168306 28.0521579999982
0.0009181191978315 27.3343399999995
0.00095230470152 28.7968520000005
0.000954761222611 27.3446679999998
0.000959790925001 27.3936859999997
0.000981163360295928 14.2786110000002
0.000985524338298875 14.8479650000001
0.000991251030629584 22.6564560000015
0.000994346231230176 15.7006250000001
0.000998799281019742 21.4489660000015
0.00100165426563 16.0921910000004
0.00100749973353 15.8624130000005
0.00101138130559 15.6328850000003
0.00110563288523667 15.165496
0.0012457397417 13.2431530000002
0.00125521263354 15.0831399999999
0.00134898317914 13.4151099999999
0.00135611294534 15.4558780000002
0.00147141347477 13.1815700000005
0.00147854612149 12.9118660000004
0.00151228225734 3.34232000000002
0.00152762647136 3.24622100000004
0.00153169773782 13.2362430000003
0.00154298529365 13.0405530000004
0.0015526802791 12.8866380000003
0.00160323936772857 3.35218899999998
0.00160776268993437 9.75096099999995
0.00161595835644875 8.70159200000012
0.00162365803446333 7.32040800000015
0.00162906210730889 3.40064100000006
0.0016397705949975 3.53028800000006
0.00164430546038833 3.47047100000006
0.0016494392719275 3.77707500000005
0.00166219183754 3.33250100000006
0.001665512593 3.27130800000004
0.0016776084294 3.17432900000002
0.00169713595455 3.21848799999996
0.00172505790934 3.28989099999999
0.00173927036233 3.24402699999998
0.00175821376495 3.16073899999998
0.00185990376843 3.39204699999996
0.001874236136795 3.11851399999996
0.00200837974719 3.00519699999999
0.00202386821951 2.99724899999996
0.00204695500937 3.00066099999995
0.00205922574622 2.89798599999998
0.00207299838124 2.96865799999999
0.00207363653959 3.05860699999998
0.00208760631721667 2.80298499999997
0.00215259003478625 2.41263500000001
0.00217377843189 2.85465699999998
0.00252399607583 2.372277
0.00285229592348 2.35633000000002
0.00288848750212 2.28218900000003
0.00292826364476 2.29672400000001
0.003426858180955 2.51615700000003
0.00393458200465 2.138846
0.00398398449074 2.24905700000001
0.00398607940283 2.212697
0.00403096434561 2.15239300000001
0.0041822005356 2.13866999999999
0.004190465814158 1.910265
0.00421124559292667 1.858181
0.0042509789662 1.83105299999998
0.00427443175164 1.76151999999998
0.004282512831564 1.592591
0.00429752568439 1.647836
0.00432481144988 1.49686699999999
0.00434296669641667 1.38529099999998
0.00440247627817 1.36668799999999
0.00442786081377 1.40409499999999
0.00444601752833 1.302578
0.00447335183174 1.25027299999999
0.00450357449599 1.14247299999999
0.00451522277149667 0.412717999999999
0.00453942048376 0.401827999999999
0.00455911329184333 0.374944
0.004578781292512 0.343254
0.00459771037051 0.322570999999999
0.004622683267108 0.508585999999997
0.00463707588289111 0.554004999999997
0.0046610916286025 0.525983999999996
0.00551433624408 0.307455
0.00786218126462 0.307433
0.00800750761571 0.302148999999999
0.0085054816497 0.271561
0.00888182786324 0.254143999999999
0.00921071739139 0.24361
0.00944671915919 0.235196000000001
0.00974764103562 0.226171
0.00978738616827 0.215927000000001
0.0100046770699 0.206945
0.0102440683766 0.176523
0.01027439170435 0.163981999999999
0.0103644001146 0.186165000000001
0.0105621558774 0.154950999999999
0.0110994895487 0.146933999999999
0.0113329095418 0.132124999999999
0.0117176084402 0.130152
0.0118040737031 0.11099
0.012124501541 0.099470999999999
0.0122354578118 0.0884109999999998
0.0125027666171 0.0281459999999991
0.0125966541808 0.0852890000000004
0.012662267381 0.0588949999999997
0.0127585080144 0.036886999999999
0.0128663968753 0.064945
0.0129423751977 0.0687649999999995
0.0130108248144 0.0742459999999996
0.0160678013788 0.021407
};
\label{\figlabel-line4}
\coordinate (legend) at (axis description cs:0.03,0.03);
\end{axis}

\matrix [matrix of nodes,
inner sep=1pt, row sep=1pt,cells={anchor=west},anchor={south west},at={(0.03,0.03)}, anchor=south west, draw=none, fill=none] at (legend) {
\ref{\figlabel-line0} {SL}\\
\ref{\figlabel-line1} {$\epsilon^{-3}\log(\epsilon^{-1})$}\\
\ref{\figlabel-line2} {ML}\\
\ref{\figlabel-line3} {$\epsilon^{-2}\log(\epsilon^{-1})^{2}$}\\
\ref{\figlabel-line4} {Adaptive ML}\\
};
\end{tikzpicture}
      \caption{$d=2$}
	\end{subfigure}
	\begin{subfigure}{0.5\textwidth}
      \renewcommand{\figlabel}{fig:total-time-vs-error-d3}
      % This file was created by matplotlib2tikz v0.6.10.
\begin{tikzpicture}

\begin{axis}[
xlabel={Max Error},
ylabel={Time [s.]},
xmin=0.0001, xmax=0.1,
ymin=0.001, ymax=10000,
xmode=log,
ymode=log,
axis on top,
name=\figlabel,
width=\figurewidth,
height=\figureheight,
xtick={1e-05,0.0001,0.001,0.01,0.1,1,10},
xticklabels={,$10^{-4}$,$10^{-3}$,$10^{-2}$,$10^{-1}$,,},
ytick={0.0001,0.001,0.01,0.1,1,10,100,1000,10000,100000},
yticklabels={,$10^{-3}$,$10^{-2}$,$10^{-1}$,$10^{0}$,$10^{1}$,$10^{2}$,$10^{3}$,$10^{4}$,},
tick pos=both
]
\addplot [thick, black, opacity=0.4, dotted, mark=x, mark size=2, mark options={solid,fill opacity=0}, forget plot]
table {%
0.0219923706361 0.280572
0.0231485452804 0.032683
0.0473996127045 0.0073700000000001
};
\addplot [thick, black, opacity=0.4, dotted, mark=x, mark size=2, mark options={solid,fill opacity=0}, forget plot]
table {%
0.0146468171491 56.319003
0.0146825404421 0.281625
0.0171486181992 0.0369269999999999
0.0456595617549 0.00841200000000009
};
\addplot [thick, black, opacity=0.4, dotted, mark=x, mark size=2, mark options={solid,fill opacity=0}, forget plot]
table {%
0.00773597291035 71.243986
0.0079395211684 0.479064
0.0129739708107 0.0495500000000002
0.045076536669 0.00707199999999997
};
\addplot [thick, black, opacity=0.4, dotted, mark=x, mark size=2, mark options={solid,fill opacity=0}, forget plot]
table {%
0.00475918310523 52.994155
0.0051763040426 0.503328
0.0120103618342 0.0421529999999999
0.0451498063531 0.00845800000000008
};
\addplot [thick, black, opacity=0.4, dotted, mark=x, mark size=2, mark options={solid,fill opacity=0}, forget plot]
table {%
0.00232937038259 96.979911
0.00321933670459 1.083645
0.0117499134021 0.110281
0.045355625829 0.0184980000000006
};
\addplot [thick, black, opacity=0.4, dotted, mark=x, mark size=2, mark options={solid,fill opacity=0}, forget plot]
table {%
0.00139560009836 115.91719
0.00268287982911 1.983264
0.011787563253 0.200733
0.0454708567658 0.0245000000000002
};
\addplot [thick, black, opacity=0.4, dotted, mark=x, mark size=2, mark options={solid,fill opacity=0}, forget plot]
table {%
0.000712175432305 164.04346
0.00244887368137 4.321099
0.0118688266345 0.397337
0.0455720926749 0.0873309999999998
};
\addplot [thick, black, opacity=0.4, dotted, mark=x, mark size=2, mark options={solid,fill opacity=0}, forget plot]
table {%
0.000477321676295 285.061344
0.0024110537785 10.507983
0.0119135687612 0.877356000000001
0.0456154902222 0.140544
};
\addplot [thick, black, opacity=0.4, dotted, mark=x, mark size=2, mark options={solid,fill opacity=0}, forget plot]
table {%
0.0003730803792 640.047029
0.00240481031792 31.319652
0.0119444028253 2.397552
0.0456429401838 0.375838
};
\addplot [thick, black, opacity=0.4, dotted, mark=x, mark size=2, mark options={solid,fill opacity=0}, forget plot]
table {%
0.000346112906001 1759.361261
0.00240622839308 95.844415
0.011959031944 7.988817
0.0456554324994 1.096945
};
\addplot [thick, black, dash pattern=on 1pt off 3pt on 3pt off 3pt]
table {%
0.0001 423239.328877998
0.000107226722201032 340701.6921795
0.000114975699539774 274244.076807414
0.000123284673944207 220736.68873795
0.000132194114846603 177658.377646415
0.000141747416292681 142978.373492576
0.000151991108295294 115060.984459645
0.000162975083462064 92588.8015041979
0.000174752840000768 74500.818434034
0.000187381742286039 59942.5727609829
0.000200923300256505 48225.9739746426
0.000215443469003188 38796.9384804832
0.000231012970008316 31209.3153141528
0.000247707635599171 25103.8808769865
0.000265608778294669 20191.4180395954
0.00028480358684358 16239.0860817427
0.000305385550883342 13059.4419882983
0.000327454916287773 10501.5977879835
0.000351119173421514 8444.09869718282
0.000376493580679247 6789.18748632729
0.000403701725859656 5458.18548717894
0.000432876128108306 4387.7730419552
0.000464158883361278 3526.99440731665
0.000497702356433211 2834.84614103554
0.000533669923120631 2278.33540732314
0.000572236765935022 1830.91672053959
0.000613590727341318 1471.23343988438
0.000657933224657568 1182.1046627339
0.000705480231071865 949.709713000246
0.000756463327554629 762.93172439459
0.000811130830789688 612.829312925118
0.000869749002617784 492.211369825956
0.000932603346883221 395.294868644612
0.001 317.429496658499
0.00107226722201032 254.876075065581
0.00114975699539774 204.628272694764
0.00123284673944207 164.269163711963
0.00132194114846603 131.855827159449
0.00141747416292681 105.826512663797
0.00151991108295294 84.9259647202143
0.00162975083462064 68.1453579070896
0.00174752840000769 54.6739877217507
0.00187381742286038 43.8604190934022
0.00200923300256505 35.1812433093705
0.00215443469003189 28.2159552585332
0.00231012970008316 22.6267536027607
0.00247707635599171 18.1423004657214
0.00265608778294669 14.5446655369967
0.0028480358684358 11.658831033046
0.00305385550883342 9.34425590542035
0.00327454916287773 7.48809581404042
0.00351119173421513 5.99975433747202
0.00376493580679247 4.80650441509896
0.00403701725859656 3.84997012042086
0.00432876128108306 3.08329997542798
0.00464158883361278 2.46889608512166
0.00497702356433211 1.97658996989634
0.00533669923120632 1.58217736619662
0.00572236765935022 1.26624147027971
0.00613590727341317 1.01320793501471
0.00657933224657568 0.810586054446102
0.00705480231071864 0.648359515605939
0.00756463327554629 0.518497288423507
0.00811130830789687 0.414561005802286
0.00869749002617784 0.33138983315015
0.00932603346883221 0.26484756199189
0.01 0.211619664438999
0.0107226722201032 0.169050457951661
0.0114975699539774 0.135012468582112
0.0123284673944207 0.107801638685976
0.0132194114846603 0.0860532766724825
0.0141747416292681 0.068674651835017
0.0151991108295293 0.0547909449807835
0.0162975083462065 0.0437019143099813
0.0174752840000768 0.0348471570094675
0.0187381742286039 0.0277782654258214
0.0200923300256505 0.0221365126440983
0.0215443469003188 0.0176349720365833
0.0231012970008316 0.0140441918913688
0.0247707635599171 0.0111807200544562
0.0265608778294669 0.008897913034398
0.028480358684358 0.00707857598434937
0.0305385550883342 0.00562906982254237
0.0327454916287773 0.00447459384009732
0.0351119173421513 0.00355540997776119
0.0376493580679247 0.00282382134387065
0.0403701725859656 0.00224175475366278
0.0432876128108306 0.00177882690890076
0.0464158883361278 0.00141079776292666
0.0497702356433211 0.00111833379875714
0.0533669923120631 0.000886019325070111
0.0572236765935022 0.000701566220019842
0.0613590727341318 0.000555182430145048
0.0657933224657568 0.000439067446158306
0.0705480231071865 0.000347009318211629
0.0756463327554629 0.000274062852452426
0.0811130830789687 0.000216292698679454
0.0869749002617784 0.000170568296474342
0.093260334688322 0.000134400255339168
0.1 0.0001058098322195
};
\label{\figlabel-line1}
\addplot [thick, black, dashed]
table {%
0.0001 2958.06499642139
0.000107226722201032 2533.94031177136
0.000114975699539774 2170.37327630492
0.000123284673944207 1858.75045876355
0.000132194114846603 1591.67922140058
0.000141747416292681 1362.81526290182
0.000151991108295294 1166.71444949076
0.000162975083462064 998.705523258718
0.000174752840000768 854.780754519427
0.000187381742286039 731.502015994051
0.000200923300256505 625.920110191836
0.000215443469003188 535.505485474149
0.000231012970008316 458.088737871722
0.000247707635599171 391.809520703052
0.000265608778294669 335.072677525486
0.00028480358684358 286.5105803382
0.000305385550883342 244.950798032762
0.000327454916287773 209.388343110538
0.000351119173421514 178.961850459888
0.000376493580679247 152.933132922859
0.000403701725859656 130.669636556367
0.000432876128108306 111.629385693204
0.000464158883361278 95.3480656700718
0.000497702356433211 81.4279407355213
0.000533669923120631 69.5283473174526
0.000572236765935022 59.3575394960833
0.000613590727341318 50.6656950356365
0.000657933224657568 43.2389174002405
0.000705480231071865 36.8940924392158
0.000756463327554629 31.4744784094651
0.000811130830789688 26.8459251684521
0.000869749002617784 22.8936331161756
0.000932603346883221 19.5193751289983
0.001 16.6391156048703
0.00107226722201032 14.1809700799791
0.00114975699539774 12.083456897487
0.00123284673944207 10.2939992956943
0.00132194114846603 8.76764219552019
0.00141747416292681 7.46595304296641
0.00151991108295294 6.35608041927789
0.00162975083462064 5.40994787127154
0.00174752840000769 4.6035636239434
0.00187381742286038 3.91642959187773
0.00200923300256505 3.3310354695173
0.00215443469003189 2.83242570829301
0.00231012970008316 2.40782892843824
0.00247707635599171 2.04634080581864
0.00265608778294669 1.73865275432465
0.0028480358684358 1.47681982238756
0.00305385550883342 1.25406216382855
0.00327454916287773 1.06459525071853
0.00351119173421513 0.903484688263565
0.00376493580679247 0.766522085289607
0.00403701725859656 0.650118942720255
0.00432876128108306 0.551215958572727
0.00464158883361278 0.467205521783352
0.00497702356433211 0.395865487491264
0.00533669923120632 0.335302600875061
0.00572236765935022 0.283904171788411
0.00613590727341317 0.240296803884752
0.00657933224657568 0.203311154469702
0.00705480231071864 0.171951849099563
0.00756463327554629 0.145371801495314
0.00811130830789687 0.122850297699924
0.00869749002617784 0.103774296176057
0.00932603346883221 0.0876224749540735
0.01 0.0739516249105348
0.0107226722201032 0.0623850464263971
0.0114975699539774 0.0526026564481951
0.0123284673944207 0.0443325555605584
0.0132194114846603 0.0373438411075674
0.0141747416292681 0.0314404835591648
0.0151991108295293 0.0264561099657769
0.0162975083462065 0.022249561129367
0.0174752840000768 0.0187011085991373
0.0187381742286039 0.0157092342518651
0.0200923300256505 0.0131878894488149
0.0215443469003188 0.0110641629230196
0.0231012970008316 0.00927629694190239
0.0247707635599171 0.00777200016704239
0.0265608778294669 0.00650701321480119
0.028480358684358 0.00544388939630109
0.0305385550883342 0.00455095864359323
0.0327454916287773 0.00380144734826769
0.0351119173421513 0.00317273086687891
0.0376493580679247 0.00264569888500741
0.0403701725859656 0.00220421676461474
0.0432876128108306 0.00183466850116662
0.0464158883361278 0.00152556905072115
0.0497702356433211 0.00126723560659859
0.0533669923120631 0.00105150895634419
0.0572236765935022 0.000871517371760993
0.0613590727341318 0.000721476611401018
0.0657933224657568 0.000596520574746176
0.0705480231071865 0.00049255796488746
0.0756463327554629 0.000406151012749154
0.0811130830789687 0.000334412908672761
0.0869749002617784 0.000274921091729391
0.093260334688322 0.000225643976491221
0.1 0.000184879062276337
};
\label{\figlabel-line3}
\addplot [ultra thick, blue]
table {%
0.000346112906001 1759.361261
0.0003730803792 640.047029
0.000477321676295 285.061344
0.000712175432305 164.04346
0.00139560009836 115.91719
0.00232937038259 96.979911
0.00240481031792 31.319652
0.0024110537785 10.507983
0.00244887368137 4.321099
0.00268287982911 1.983264
0.00321933670459 1.083645
0.0051763040426 0.503328
0.0079395211684 0.479064
0.0117499134021 0.110281
0.0120103618342 0.0421529999999999
0.0171486181992 0.0369269999999999
0.0231485452804 0.032683
0.045076536669 0.00707199999999997
};
\label{\figlabel-line0}
\addplot [ultra thick, green!50.0!black]
table {%
0.000381566361572 161.411429
0.000451865796974 110.254917
0.000585790565792 36.67568
0.0024728679842 2.408645
0.00266604551364 1.18852
0.00311350959574 0.573986
0.0113418394376 0.204287
0.0115006079295 0.127699000000001
0.0124560343379 0.0566240000000005
0.0443133452182 0.024389
0.045512938072 0.0135700000000007
0.0473996127045 0.00540899999999978
};
\label{\figlabel-line2}
\addplot [ultra thick, red]
table {%
0.0001960709611192 737.808295999975
0.000196854020298 727.989599999988
0.00019804267732275 703.357109999984
0.000198603139171 692.055464000021
0.000199217701188 686.303994000003
0.000201205139752 666.920267000025
0.000202277559396667 658.144116999918
0.00020299445505 687.917508999979
0.000204172098731 616.318218999908
0.000205601422325667 668.136470999978
0.00020695197751575 650.707283999933
0.000207189973371 640.80303899992
0.000208255156965 640.700297999881
0.000209293310891 633.02301599995
0.000210497834136 626.100400999976
0.0002110656993055 615.800344999896
0.0002121289537415 598.275698999934
0.00021355474089 534.704004999994
0.0002144376853995 473.763580000073
0.000214863960154667 468.744607999966
0.0002162747768775 453.037967999991
0.000217025117762 404.578728000031
0.000218028211424 418.020339000018
0.000218802922344 416.438449999996
0.000219540432973 397.53161
0.000220985433509333 389.251668000006
0.000222027028759 384.714124000029
0.000223273544902 378.819726000025
0.000223641710546 369.962953999996
0.0002246833374775 360.460670999958
0.000226459466349 355.004198999997
0.0002278963871455 349.150304999982
0.000228448407873 345.759503999974
0.0002295748511238 328.664476999985
0.000230658385783 343.060018999947
0.000231432388934 325.868679999957
0.000232840166556667 316.403130999986
0.000233845033977 312.544023999975
0.000235061101871 309.696189999997
0.000236289876817667 275.996198999973
0.000237021541556 270.629156999968
0.000237641899166 264.188015999958
0.000238861588659333 258.673287999934
0.000240236713054667 251.762497999925
0.000241400224986 238.172312999961
0.0002423824746896 193.238509999997
0.000243209009407945 171.328636999989
0.000244510391002583 170.308438000012
0.000245448292627889 169.713555999993
0.0002514116945725 168.710344999979
0.000252312856879 165.042215999967
0.000253078487053067 160.130260999989
0.000254301324113 252.070107000016
0.000255685449247 259.965375000005
0.00025664855162 259.185443000016
0.000258033353835 252.979473000021
0.000259258378160625 156.902840000011
0.000259564185494 190.801435000003
0.000260665249915 158.368874000003
0.000262555296169 155.809023000001
0.000271821357645 156.199467000004
0.000273146157854 194.731583000009
0.0002745402101 204.407146000006
0.000275706888827 204.952800000008
0.000279031961344 211.663478000002
0.000281613775824333 176.205423999999
0.0002825831778344 143.320203000015
0.0002873768395395 125.370210000013
0.000290928390999 118.915649000011
0.000291518843047429 109.903805000014
0.000292507597082333 107.819865000021
0.000295409021421 106.52829200002
0.000296831862402 104.366019000013
0.000299747804516 97.8695310000253
0.00030051986085475 93.9857490000245
0.000302247549054 94.8908050000132
0.000303130251286 86.7264560000199
0.000304198980762 99.4044830000199
0.00031196363135225 82.4881840000194
0.000312784552774 82.4365290000147
0.000316057226580333 81.3968730000047
0.000322939430171 90.5538970000303
0.000325451023899 90.9088230000175
0.000328297896535 91.6993660000185
0.000329646255921 89.7327080000091
0.000331620727146667 87.6943980000138
0.000333927321716 88.7284220000152
0.000335208690935 84.826789000023
0.000337222825823 92.6448950000154
0.0003404374447565 89.2392550000164
0.0003417188045385 86.8321870000123
0.000343748703051 86.5258290000134
0.000346811421293 85.7166620000158
0.000347592665874 79.7095380000142
0.000349832932715333 73.6484740000138
0.000350657361504 76.5035010000179
0.000351790949572 68.4041140000159
0.000355189775868 76.8232230000181
0.0003567716952615 68.2209290000133
0.000359228010425 65.8621950000148
0.000362370974331 67.5965520000141
0.0003638470968518 73.0185990000078
0.0003654546503845 70.5382170000008
0.00036716586277 72.8126120000021
0.000368858490075 69.998957000004
0.0003703217690545 68.9518970000041
0.0003719916421635 61.0345159999993
0.00037341143468 69.1439410000008
0.000375017896589 59.392171
0.000376239087398 57.1620410000065
0.000380563621461 53.3604929999961
0.000383259838497 49.715608999994
0.0003853494346595 50.4747429999948
0.000386195906104 56.6000979999977
0.000396178762489 58.9059259999991
0.000397553316306 53.7578409999983
0.00040059381862 52.0624370000028
0.000403765067384 49.1649820000026
0.000406319199868 50.4132580000056
0.000415648338658 51.5216580000047
0.0004223096414522 50.2579040000042
0.000423775785566 47.2688190000095
0.000425732040548 45.5429130000098
0.0004282857708295 44.9886620000121
0.000430738152901 43.9581370000065
0.000432783000265 45.0125630000089
0.00043434641816 40.9138830000116
0.000437876509574 41.9253240000112
0.000439393788988333 38.1977680000105
0.000442367915847333 32.9127590000112
0.000443522961622 32.0322690000144
0.000445484308653529 31.1379250000105
0.00044702992839225 30.9396380000126
0.0004495685620275 29.7996620000118
0.000451297117691429 29.5381790000138
0.000453790036123 29.5791800000111
0.000455738948988 29.3170920000107
0.000457718227683 30.19855900001
0.000459946523051 28.6124760000117
0.000460720303909 28.5383980000093
0.0004780664734075 28.5216040000101
0.000504233355798 33.163295000009
0.0005060332322185 28.3121280000055
0.000509468836734 27.9349580000053
0.000509686034308 27.935909000004
0.000512997822149 28.0319760000057
0.000544860367978 27.9930950000062
0.000567859081628 27.8581090000062
0.000569800301941375 27.8265840000077
0.000596597141971 27.870156000005
0.000599072725923 28.9226660000077
0.0006276311424835 27.9553890000075
0.000699834225015286 28.0023080000074
0.0007016768976054 26.8475090000061
0.000704293847567667 27.5996560000036
0.000828017262612 26.2921910000051
0.000829293815150857 25.6969290000011
0.0008330600142377 25.9733319999987
0.000837256138046875 28.1570379999976
0.000840622521942857 26.4580189999975
0.000856196724845 27.4307909999971
0.0008590828447325 28.2941499999961
0.0009101524214365 26.8135270000012
0.000914848168306 28.0521579999982
0.0009181191978315 27.3343399999995
0.00095230470152 28.7968520000005
0.000954761222611 27.3446679999998
0.000959790925001 27.3936859999997
0.000981163360295928 14.2786110000002
0.000985524338298875 14.8479650000001
0.000991251030629584 22.6564560000015
0.000994346231230176 15.7006250000001
0.000998799281019742 21.4489660000015
0.00100165426563 16.0921910000004
0.00100749973353 15.8624130000005
0.00101138130559 15.6328850000003
0.00110563288523667 15.165496
0.0012457397417 13.2431530000002
0.00125521263354 15.0831399999999
0.00134898317914 13.4151099999999
0.00135611294534 15.4558780000002
0.00147141347477 13.1815700000005
0.00147854612149 12.9118660000004
0.00151228225734 3.34232000000002
0.00152762647136 3.24622100000004
0.00153169773782 13.2362430000003
0.00154298529365 13.0405530000004
0.0015526802791 12.8866380000003
0.00160323936772857 3.35218899999998
0.00160776268993437 9.75096099999995
0.00161595835644875 8.70159200000012
0.00162365803446333 7.32040800000015
0.00162906210730889 3.40064100000006
0.0016397705949975 3.53028800000006
0.00164430546038833 3.47047100000006
0.0016494392719275 3.77707500000005
0.00166219183754 3.33250100000006
0.001665512593 3.27130800000004
0.0016776084294 3.17432900000002
0.00169713595455 3.21848799999996
0.00172505790934 3.28989099999999
0.00173927036233 3.24402699999998
0.00175821376495 3.16073899999998
0.00185990376843 3.39204699999996
0.001874236136795 3.11851399999996
0.00200837974719 3.00519699999999
0.00202386821951 2.99724899999996
0.00204695500937 3.00066099999995
0.00205922574622 2.89798599999998
0.00207299838124 2.96865799999999
0.00207363653959 3.05860699999998
0.00208760631721667 2.80298499999997
0.00215259003478625 2.41263500000001
0.00217377843189 2.85465699999998
0.00252399607583 2.372277
0.00285229592348 2.35633000000002
0.00288848750212 2.28218900000003
0.00292826364476 2.29672400000001
0.003426858180955 2.51615700000003
0.00393458200465 2.138846
0.00398398449074 2.24905700000001
0.00398607940283 2.212697
0.00403096434561 2.15239300000001
0.0041822005356 2.13866999999999
0.004190465814158 1.910265
0.00421124559292667 1.858181
0.0042509789662 1.83105299999998
0.00427443175164 1.76151999999998
0.004282512831564 1.592591
0.00429752568439 1.647836
0.00432481144988 1.49686699999999
0.00434296669641667 1.38529099999998
0.00440247627817 1.36668799999999
0.00442786081377 1.40409499999999
0.00444601752833 1.302578
0.00447335183174 1.25027299999999
0.00450357449599 1.14247299999999
0.00451522277149667 0.412717999999999
0.00453942048376 0.401827999999999
0.00455911329184333 0.374944
0.004578781292512 0.343254
0.00459771037051 0.322570999999999
0.004622683267108 0.508585999999997
0.00463707588289111 0.554004999999997
0.0046610916286025 0.525983999999996
0.00551433624408 0.307455
0.00786218126462 0.307433
0.00800750761571 0.302148999999999
0.0085054816497 0.271561
0.00888182786324 0.254143999999999
0.00921071739139 0.24361
0.00944671915919 0.235196000000001
0.00974764103562 0.226171
0.00978738616827 0.215927000000001
0.0100046770699 0.206945
0.0102440683766 0.176523
0.01027439170435 0.163981999999999
0.0103644001146 0.186165000000001
0.0105621558774 0.154950999999999
0.0110994895487 0.146933999999999
0.0113329095418 0.132124999999999
0.0117176084402 0.130152
0.0118040737031 0.11099
0.012124501541 0.099470999999999
0.0122354578118 0.0884109999999998
0.0125027666171 0.0281459999999991
0.0125966541808 0.0852890000000004
0.012662267381 0.0588949999999997
0.0127585080144 0.036886999999999
0.0128663968753 0.064945
0.0129423751977 0.0687649999999995
0.0130108248144 0.0742459999999996
0.0160678013788 0.021407
};
\label{\figlabel-line4}
\coordinate (legend) at (axis description cs:0.03,0.03);
\end{axis}

\matrix [matrix of nodes,
inner sep=1pt, row sep=1pt,cells={anchor=west},anchor={south west},at={(0.03,0.03)}, anchor=south west, draw=none, fill=none] at (legend) {
\ref{\figlabel-line0} {SL}\\
\ref{\figlabel-line1} {$\epsilon^{-3}\log(\epsilon^{-1})$}\\
\ref{\figlabel-line2} {ML}\\
\ref{\figlabel-line3} {$\epsilon^{-2}\log(\epsilon^{-1})^{2}$}\\
\ref{\figlabel-line4} {Adaptive ML}\\
};
\end{tikzpicture}
      \caption{$d=3$}
	\end{subfigure}
	\begin{subfigure}{0.49\textwidth}
      \renewcommand{\figlabel}{fig:total-time-vs-error-d4}
      % This file was created by matplotlib2tikz v0.6.10.
\begin{tikzpicture}

\begin{axis}[
xlabel={Max Error},
ylabel={Time [s.]},
xmin=0.0001, xmax=0.1,
ymin=0.001, ymax=10000,
xmode=log,
ymode=log,
axis on top,
name=\figlabel,
width=\figurewidth,
height=\figureheight,
xtick={1e-05,0.0001,0.001,0.01,0.1,1,10},
xticklabels={,$10^{-4}$,$10^{-3}$,$10^{-2}$,$10^{-1}$,,},
ytick={0.0001,0.001,0.01,0.1,1,10,100,1000,10000,100000},
yticklabels={,$10^{-3}$,$10^{-2}$,$10^{-1}$,$10^{0}$,$10^{1}$,$10^{2}$,$10^{3}$,$10^{4}$,},
tick pos=both
]
\addplot [thick, black, opacity=0.4, dotted, mark=x, mark size=2, mark options={solid,fill opacity=0}, forget plot]
table {%
0.0219923706361 0.280572
0.0231485452804 0.032683
0.0473996127045 0.0073700000000001
};
\addplot [thick, black, opacity=0.4, dotted, mark=x, mark size=2, mark options={solid,fill opacity=0}, forget plot]
table {%
0.0146468171491 56.319003
0.0146825404421 0.281625
0.0171486181992 0.0369269999999999
0.0456595617549 0.00841200000000009
};
\addplot [thick, black, opacity=0.4, dotted, mark=x, mark size=2, mark options={solid,fill opacity=0}, forget plot]
table {%
0.00773597291035 71.243986
0.0079395211684 0.479064
0.0129739708107 0.0495500000000002
0.045076536669 0.00707199999999997
};
\addplot [thick, black, opacity=0.4, dotted, mark=x, mark size=2, mark options={solid,fill opacity=0}, forget plot]
table {%
0.00475918310523 52.994155
0.0051763040426 0.503328
0.0120103618342 0.0421529999999999
0.0451498063531 0.00845800000000008
};
\addplot [thick, black, opacity=0.4, dotted, mark=x, mark size=2, mark options={solid,fill opacity=0}, forget plot]
table {%
0.00232937038259 96.979911
0.00321933670459 1.083645
0.0117499134021 0.110281
0.045355625829 0.0184980000000006
};
\addplot [thick, black, opacity=0.4, dotted, mark=x, mark size=2, mark options={solid,fill opacity=0}, forget plot]
table {%
0.00139560009836 115.91719
0.00268287982911 1.983264
0.011787563253 0.200733
0.0454708567658 0.0245000000000002
};
\addplot [thick, black, opacity=0.4, dotted, mark=x, mark size=2, mark options={solid,fill opacity=0}, forget plot]
table {%
0.000712175432305 164.04346
0.00244887368137 4.321099
0.0118688266345 0.397337
0.0455720926749 0.0873309999999998
};
\addplot [thick, black, opacity=0.4, dotted, mark=x, mark size=2, mark options={solid,fill opacity=0}, forget plot]
table {%
0.000477321676295 285.061344
0.0024110537785 10.507983
0.0119135687612 0.877356000000001
0.0456154902222 0.140544
};
\addplot [thick, black, opacity=0.4, dotted, mark=x, mark size=2, mark options={solid,fill opacity=0}, forget plot]
table {%
0.0003730803792 640.047029
0.00240481031792 31.319652
0.0119444028253 2.397552
0.0456429401838 0.375838
};
\addplot [thick, black, opacity=0.4, dotted, mark=x, mark size=2, mark options={solid,fill opacity=0}, forget plot]
table {%
0.000346112906001 1759.361261
0.00240622839308 95.844415
0.011959031944 7.988817
0.0456554324994 1.096945
};
\addplot [thick, black, dash pattern=on 1pt off 3pt on 3pt off 3pt]
table {%
0.0001 423239.328877998
0.000107226722201032 340701.6921795
0.000114975699539774 274244.076807414
0.000123284673944207 220736.68873795
0.000132194114846603 177658.377646415
0.000141747416292681 142978.373492576
0.000151991108295294 115060.984459645
0.000162975083462064 92588.8015041979
0.000174752840000768 74500.818434034
0.000187381742286039 59942.5727609829
0.000200923300256505 48225.9739746426
0.000215443469003188 38796.9384804832
0.000231012970008316 31209.3153141528
0.000247707635599171 25103.8808769865
0.000265608778294669 20191.4180395954
0.00028480358684358 16239.0860817427
0.000305385550883342 13059.4419882983
0.000327454916287773 10501.5977879835
0.000351119173421514 8444.09869718282
0.000376493580679247 6789.18748632729
0.000403701725859656 5458.18548717894
0.000432876128108306 4387.7730419552
0.000464158883361278 3526.99440731665
0.000497702356433211 2834.84614103554
0.000533669923120631 2278.33540732314
0.000572236765935022 1830.91672053959
0.000613590727341318 1471.23343988438
0.000657933224657568 1182.1046627339
0.000705480231071865 949.709713000246
0.000756463327554629 762.93172439459
0.000811130830789688 612.829312925118
0.000869749002617784 492.211369825956
0.000932603346883221 395.294868644612
0.001 317.429496658499
0.00107226722201032 254.876075065581
0.00114975699539774 204.628272694764
0.00123284673944207 164.269163711963
0.00132194114846603 131.855827159449
0.00141747416292681 105.826512663797
0.00151991108295294 84.9259647202143
0.00162975083462064 68.1453579070896
0.00174752840000769 54.6739877217507
0.00187381742286038 43.8604190934022
0.00200923300256505 35.1812433093705
0.00215443469003189 28.2159552585332
0.00231012970008316 22.6267536027607
0.00247707635599171 18.1423004657214
0.00265608778294669 14.5446655369967
0.0028480358684358 11.658831033046
0.00305385550883342 9.34425590542035
0.00327454916287773 7.48809581404042
0.00351119173421513 5.99975433747202
0.00376493580679247 4.80650441509896
0.00403701725859656 3.84997012042086
0.00432876128108306 3.08329997542798
0.00464158883361278 2.46889608512166
0.00497702356433211 1.97658996989634
0.00533669923120632 1.58217736619662
0.00572236765935022 1.26624147027971
0.00613590727341317 1.01320793501471
0.00657933224657568 0.810586054446102
0.00705480231071864 0.648359515605939
0.00756463327554629 0.518497288423507
0.00811130830789687 0.414561005802286
0.00869749002617784 0.33138983315015
0.00932603346883221 0.26484756199189
0.01 0.211619664438999
0.0107226722201032 0.169050457951661
0.0114975699539774 0.135012468582112
0.0123284673944207 0.107801638685976
0.0132194114846603 0.0860532766724825
0.0141747416292681 0.068674651835017
0.0151991108295293 0.0547909449807835
0.0162975083462065 0.0437019143099813
0.0174752840000768 0.0348471570094675
0.0187381742286039 0.0277782654258214
0.0200923300256505 0.0221365126440983
0.0215443469003188 0.0176349720365833
0.0231012970008316 0.0140441918913688
0.0247707635599171 0.0111807200544562
0.0265608778294669 0.008897913034398
0.028480358684358 0.00707857598434937
0.0305385550883342 0.00562906982254237
0.0327454916287773 0.00447459384009732
0.0351119173421513 0.00355540997776119
0.0376493580679247 0.00282382134387065
0.0403701725859656 0.00224175475366278
0.0432876128108306 0.00177882690890076
0.0464158883361278 0.00141079776292666
0.0497702356433211 0.00111833379875714
0.0533669923120631 0.000886019325070111
0.0572236765935022 0.000701566220019842
0.0613590727341318 0.000555182430145048
0.0657933224657568 0.000439067446158306
0.0705480231071865 0.000347009318211629
0.0756463327554629 0.000274062852452426
0.0811130830789687 0.000216292698679454
0.0869749002617784 0.000170568296474342
0.093260334688322 0.000134400255339168
0.1 0.0001058098322195
};
\label{\figlabel-line1}
\addplot [thick, black, dashed]
table {%
0.0001 2958.06499642139
0.000107226722201032 2533.94031177136
0.000114975699539774 2170.37327630492
0.000123284673944207 1858.75045876355
0.000132194114846603 1591.67922140058
0.000141747416292681 1362.81526290182
0.000151991108295294 1166.71444949076
0.000162975083462064 998.705523258718
0.000174752840000768 854.780754519427
0.000187381742286039 731.502015994051
0.000200923300256505 625.920110191836
0.000215443469003188 535.505485474149
0.000231012970008316 458.088737871722
0.000247707635599171 391.809520703052
0.000265608778294669 335.072677525486
0.00028480358684358 286.5105803382
0.000305385550883342 244.950798032762
0.000327454916287773 209.388343110538
0.000351119173421514 178.961850459888
0.000376493580679247 152.933132922859
0.000403701725859656 130.669636556367
0.000432876128108306 111.629385693204
0.000464158883361278 95.3480656700718
0.000497702356433211 81.4279407355213
0.000533669923120631 69.5283473174526
0.000572236765935022 59.3575394960833
0.000613590727341318 50.6656950356365
0.000657933224657568 43.2389174002405
0.000705480231071865 36.8940924392158
0.000756463327554629 31.4744784094651
0.000811130830789688 26.8459251684521
0.000869749002617784 22.8936331161756
0.000932603346883221 19.5193751289983
0.001 16.6391156048703
0.00107226722201032 14.1809700799791
0.00114975699539774 12.083456897487
0.00123284673944207 10.2939992956943
0.00132194114846603 8.76764219552019
0.00141747416292681 7.46595304296641
0.00151991108295294 6.35608041927789
0.00162975083462064 5.40994787127154
0.00174752840000769 4.6035636239434
0.00187381742286038 3.91642959187773
0.00200923300256505 3.3310354695173
0.00215443469003189 2.83242570829301
0.00231012970008316 2.40782892843824
0.00247707635599171 2.04634080581864
0.00265608778294669 1.73865275432465
0.0028480358684358 1.47681982238756
0.00305385550883342 1.25406216382855
0.00327454916287773 1.06459525071853
0.00351119173421513 0.903484688263565
0.00376493580679247 0.766522085289607
0.00403701725859656 0.650118942720255
0.00432876128108306 0.551215958572727
0.00464158883361278 0.467205521783352
0.00497702356433211 0.395865487491264
0.00533669923120632 0.335302600875061
0.00572236765935022 0.283904171788411
0.00613590727341317 0.240296803884752
0.00657933224657568 0.203311154469702
0.00705480231071864 0.171951849099563
0.00756463327554629 0.145371801495314
0.00811130830789687 0.122850297699924
0.00869749002617784 0.103774296176057
0.00932603346883221 0.0876224749540735
0.01 0.0739516249105348
0.0107226722201032 0.0623850464263971
0.0114975699539774 0.0526026564481951
0.0123284673944207 0.0443325555605584
0.0132194114846603 0.0373438411075674
0.0141747416292681 0.0314404835591648
0.0151991108295293 0.0264561099657769
0.0162975083462065 0.022249561129367
0.0174752840000768 0.0187011085991373
0.0187381742286039 0.0157092342518651
0.0200923300256505 0.0131878894488149
0.0215443469003188 0.0110641629230196
0.0231012970008316 0.00927629694190239
0.0247707635599171 0.00777200016704239
0.0265608778294669 0.00650701321480119
0.028480358684358 0.00544388939630109
0.0305385550883342 0.00455095864359323
0.0327454916287773 0.00380144734826769
0.0351119173421513 0.00317273086687891
0.0376493580679247 0.00264569888500741
0.0403701725859656 0.00220421676461474
0.0432876128108306 0.00183466850116662
0.0464158883361278 0.00152556905072115
0.0497702356433211 0.00126723560659859
0.0533669923120631 0.00105150895634419
0.0572236765935022 0.000871517371760993
0.0613590727341318 0.000721476611401018
0.0657933224657568 0.000596520574746176
0.0705480231071865 0.00049255796488746
0.0756463327554629 0.000406151012749154
0.0811130830789687 0.000334412908672761
0.0869749002617784 0.000274921091729391
0.093260334688322 0.000225643976491221
0.1 0.000184879062276337
};
\label{\figlabel-line3}
\addplot [ultra thick, blue]
table {%
0.000346112906001 1759.361261
0.0003730803792 640.047029
0.000477321676295 285.061344
0.000712175432305 164.04346
0.00139560009836 115.91719
0.00232937038259 96.979911
0.00240481031792 31.319652
0.0024110537785 10.507983
0.00244887368137 4.321099
0.00268287982911 1.983264
0.00321933670459 1.083645
0.0051763040426 0.503328
0.0079395211684 0.479064
0.0117499134021 0.110281
0.0120103618342 0.0421529999999999
0.0171486181992 0.0369269999999999
0.0231485452804 0.032683
0.045076536669 0.00707199999999997
};
\label{\figlabel-line0}
\addplot [ultra thick, green!50.0!black]
table {%
0.000381566361572 161.411429
0.000451865796974 110.254917
0.000585790565792 36.67568
0.0024728679842 2.408645
0.00266604551364 1.18852
0.00311350959574 0.573986
0.0113418394376 0.204287
0.0115006079295 0.127699000000001
0.0124560343379 0.0566240000000005
0.0443133452182 0.024389
0.045512938072 0.0135700000000007
0.0473996127045 0.00540899999999978
};
\label{\figlabel-line2}
\addplot [ultra thick, red]
table {%
0.0001960709611192 737.808295999975
0.000196854020298 727.989599999988
0.00019804267732275 703.357109999984
0.000198603139171 692.055464000021
0.000199217701188 686.303994000003
0.000201205139752 666.920267000025
0.000202277559396667 658.144116999918
0.00020299445505 687.917508999979
0.000204172098731 616.318218999908
0.000205601422325667 668.136470999978
0.00020695197751575 650.707283999933
0.000207189973371 640.80303899992
0.000208255156965 640.700297999881
0.000209293310891 633.02301599995
0.000210497834136 626.100400999976
0.0002110656993055 615.800344999896
0.0002121289537415 598.275698999934
0.00021355474089 534.704004999994
0.0002144376853995 473.763580000073
0.000214863960154667 468.744607999966
0.0002162747768775 453.037967999991
0.000217025117762 404.578728000031
0.000218028211424 418.020339000018
0.000218802922344 416.438449999996
0.000219540432973 397.53161
0.000220985433509333 389.251668000006
0.000222027028759 384.714124000029
0.000223273544902 378.819726000025
0.000223641710546 369.962953999996
0.0002246833374775 360.460670999958
0.000226459466349 355.004198999997
0.0002278963871455 349.150304999982
0.000228448407873 345.759503999974
0.0002295748511238 328.664476999985
0.000230658385783 343.060018999947
0.000231432388934 325.868679999957
0.000232840166556667 316.403130999986
0.000233845033977 312.544023999975
0.000235061101871 309.696189999997
0.000236289876817667 275.996198999973
0.000237021541556 270.629156999968
0.000237641899166 264.188015999958
0.000238861588659333 258.673287999934
0.000240236713054667 251.762497999925
0.000241400224986 238.172312999961
0.0002423824746896 193.238509999997
0.000243209009407945 171.328636999989
0.000244510391002583 170.308438000012
0.000245448292627889 169.713555999993
0.0002514116945725 168.710344999979
0.000252312856879 165.042215999967
0.000253078487053067 160.130260999989
0.000254301324113 252.070107000016
0.000255685449247 259.965375000005
0.00025664855162 259.185443000016
0.000258033353835 252.979473000021
0.000259258378160625 156.902840000011
0.000259564185494 190.801435000003
0.000260665249915 158.368874000003
0.000262555296169 155.809023000001
0.000271821357645 156.199467000004
0.000273146157854 194.731583000009
0.0002745402101 204.407146000006
0.000275706888827 204.952800000008
0.000279031961344 211.663478000002
0.000281613775824333 176.205423999999
0.0002825831778344 143.320203000015
0.0002873768395395 125.370210000013
0.000290928390999 118.915649000011
0.000291518843047429 109.903805000014
0.000292507597082333 107.819865000021
0.000295409021421 106.52829200002
0.000296831862402 104.366019000013
0.000299747804516 97.8695310000253
0.00030051986085475 93.9857490000245
0.000302247549054 94.8908050000132
0.000303130251286 86.7264560000199
0.000304198980762 99.4044830000199
0.00031196363135225 82.4881840000194
0.000312784552774 82.4365290000147
0.000316057226580333 81.3968730000047
0.000322939430171 90.5538970000303
0.000325451023899 90.9088230000175
0.000328297896535 91.6993660000185
0.000329646255921 89.7327080000091
0.000331620727146667 87.6943980000138
0.000333927321716 88.7284220000152
0.000335208690935 84.826789000023
0.000337222825823 92.6448950000154
0.0003404374447565 89.2392550000164
0.0003417188045385 86.8321870000123
0.000343748703051 86.5258290000134
0.000346811421293 85.7166620000158
0.000347592665874 79.7095380000142
0.000349832932715333 73.6484740000138
0.000350657361504 76.5035010000179
0.000351790949572 68.4041140000159
0.000355189775868 76.8232230000181
0.0003567716952615 68.2209290000133
0.000359228010425 65.8621950000148
0.000362370974331 67.5965520000141
0.0003638470968518 73.0185990000078
0.0003654546503845 70.5382170000008
0.00036716586277 72.8126120000021
0.000368858490075 69.998957000004
0.0003703217690545 68.9518970000041
0.0003719916421635 61.0345159999993
0.00037341143468 69.1439410000008
0.000375017896589 59.392171
0.000376239087398 57.1620410000065
0.000380563621461 53.3604929999961
0.000383259838497 49.715608999994
0.0003853494346595 50.4747429999948
0.000386195906104 56.6000979999977
0.000396178762489 58.9059259999991
0.000397553316306 53.7578409999983
0.00040059381862 52.0624370000028
0.000403765067384 49.1649820000026
0.000406319199868 50.4132580000056
0.000415648338658 51.5216580000047
0.0004223096414522 50.2579040000042
0.000423775785566 47.2688190000095
0.000425732040548 45.5429130000098
0.0004282857708295 44.9886620000121
0.000430738152901 43.9581370000065
0.000432783000265 45.0125630000089
0.00043434641816 40.9138830000116
0.000437876509574 41.9253240000112
0.000439393788988333 38.1977680000105
0.000442367915847333 32.9127590000112
0.000443522961622 32.0322690000144
0.000445484308653529 31.1379250000105
0.00044702992839225 30.9396380000126
0.0004495685620275 29.7996620000118
0.000451297117691429 29.5381790000138
0.000453790036123 29.5791800000111
0.000455738948988 29.3170920000107
0.000457718227683 30.19855900001
0.000459946523051 28.6124760000117
0.000460720303909 28.5383980000093
0.0004780664734075 28.5216040000101
0.000504233355798 33.163295000009
0.0005060332322185 28.3121280000055
0.000509468836734 27.9349580000053
0.000509686034308 27.935909000004
0.000512997822149 28.0319760000057
0.000544860367978 27.9930950000062
0.000567859081628 27.8581090000062
0.000569800301941375 27.8265840000077
0.000596597141971 27.870156000005
0.000599072725923 28.9226660000077
0.0006276311424835 27.9553890000075
0.000699834225015286 28.0023080000074
0.0007016768976054 26.8475090000061
0.000704293847567667 27.5996560000036
0.000828017262612 26.2921910000051
0.000829293815150857 25.6969290000011
0.0008330600142377 25.9733319999987
0.000837256138046875 28.1570379999976
0.000840622521942857 26.4580189999975
0.000856196724845 27.4307909999971
0.0008590828447325 28.2941499999961
0.0009101524214365 26.8135270000012
0.000914848168306 28.0521579999982
0.0009181191978315 27.3343399999995
0.00095230470152 28.7968520000005
0.000954761222611 27.3446679999998
0.000959790925001 27.3936859999997
0.000981163360295928 14.2786110000002
0.000985524338298875 14.8479650000001
0.000991251030629584 22.6564560000015
0.000994346231230176 15.7006250000001
0.000998799281019742 21.4489660000015
0.00100165426563 16.0921910000004
0.00100749973353 15.8624130000005
0.00101138130559 15.6328850000003
0.00110563288523667 15.165496
0.0012457397417 13.2431530000002
0.00125521263354 15.0831399999999
0.00134898317914 13.4151099999999
0.00135611294534 15.4558780000002
0.00147141347477 13.1815700000005
0.00147854612149 12.9118660000004
0.00151228225734 3.34232000000002
0.00152762647136 3.24622100000004
0.00153169773782 13.2362430000003
0.00154298529365 13.0405530000004
0.0015526802791 12.8866380000003
0.00160323936772857 3.35218899999998
0.00160776268993437 9.75096099999995
0.00161595835644875 8.70159200000012
0.00162365803446333 7.32040800000015
0.00162906210730889 3.40064100000006
0.0016397705949975 3.53028800000006
0.00164430546038833 3.47047100000006
0.0016494392719275 3.77707500000005
0.00166219183754 3.33250100000006
0.001665512593 3.27130800000004
0.0016776084294 3.17432900000002
0.00169713595455 3.21848799999996
0.00172505790934 3.28989099999999
0.00173927036233 3.24402699999998
0.00175821376495 3.16073899999998
0.00185990376843 3.39204699999996
0.001874236136795 3.11851399999996
0.00200837974719 3.00519699999999
0.00202386821951 2.99724899999996
0.00204695500937 3.00066099999995
0.00205922574622 2.89798599999998
0.00207299838124 2.96865799999999
0.00207363653959 3.05860699999998
0.00208760631721667 2.80298499999997
0.00215259003478625 2.41263500000001
0.00217377843189 2.85465699999998
0.00252399607583 2.372277
0.00285229592348 2.35633000000002
0.00288848750212 2.28218900000003
0.00292826364476 2.29672400000001
0.003426858180955 2.51615700000003
0.00393458200465 2.138846
0.00398398449074 2.24905700000001
0.00398607940283 2.212697
0.00403096434561 2.15239300000001
0.0041822005356 2.13866999999999
0.004190465814158 1.910265
0.00421124559292667 1.858181
0.0042509789662 1.83105299999998
0.00427443175164 1.76151999999998
0.004282512831564 1.592591
0.00429752568439 1.647836
0.00432481144988 1.49686699999999
0.00434296669641667 1.38529099999998
0.00440247627817 1.36668799999999
0.00442786081377 1.40409499999999
0.00444601752833 1.302578
0.00447335183174 1.25027299999999
0.00450357449599 1.14247299999999
0.00451522277149667 0.412717999999999
0.00453942048376 0.401827999999999
0.00455911329184333 0.374944
0.004578781292512 0.343254
0.00459771037051 0.322570999999999
0.004622683267108 0.508585999999997
0.00463707588289111 0.554004999999997
0.0046610916286025 0.525983999999996
0.00551433624408 0.307455
0.00786218126462 0.307433
0.00800750761571 0.302148999999999
0.0085054816497 0.271561
0.00888182786324 0.254143999999999
0.00921071739139 0.24361
0.00944671915919 0.235196000000001
0.00974764103562 0.226171
0.00978738616827 0.215927000000001
0.0100046770699 0.206945
0.0102440683766 0.176523
0.01027439170435 0.163981999999999
0.0103644001146 0.186165000000001
0.0105621558774 0.154950999999999
0.0110994895487 0.146933999999999
0.0113329095418 0.132124999999999
0.0117176084402 0.130152
0.0118040737031 0.11099
0.012124501541 0.099470999999999
0.0122354578118 0.0884109999999998
0.0125027666171 0.0281459999999991
0.0125966541808 0.0852890000000004
0.012662267381 0.0588949999999997
0.0127585080144 0.036886999999999
0.0128663968753 0.064945
0.0129423751977 0.0687649999999995
0.0130108248144 0.0742459999999996
0.0160678013788 0.021407
};
\label{\figlabel-line4}
\coordinate (legend) at (axis description cs:0.03,0.03);
\end{axis}

\matrix [matrix of nodes,
inner sep=1pt, row sep=1pt,cells={anchor=west},anchor={south west},at={(0.03,0.03)}, anchor=south west, draw=none, fill=none] at (legend) {
\ref{\figlabel-line0} {SL}\\
\ref{\figlabel-line1} {$\epsilon^{-3}\log(\epsilon^{-1})$}\\
\ref{\figlabel-line2} {ML}\\
\ref{\figlabel-line3} {$\epsilon^{-2}\log(\epsilon^{-1})^{2}$}\\
\ref{\figlabel-line4} {Adaptive ML}\\
};
\end{tikzpicture}
      \caption{$d=4$}
	\end{subfigure}
	\begin{subfigure}{0.5\textwidth}
      \renewcommand{\figlabel}{fig:total-time-vs-error-d6}
      % This file was created by matplotlib2tikz v0.6.10.
\begin{tikzpicture}

\begin{axis}[
xlabel={Max Error},
ylabel={Time [s.]},
xmin=0.0001, xmax=0.1,
ymin=0.001, ymax=10000,
xmode=log,
ymode=log,
axis on top,
name=\figlabel,
width=\figurewidth,
height=\figureheight,
xtick={1e-05,0.0001,0.001,0.01,0.1,1,10},
xticklabels={,$10^{-4}$,$10^{-3}$,$10^{-2}$,$10^{-1}$,,},
ytick={0.0001,0.001,0.01,0.1,1,10,100,1000,10000,100000},
yticklabels={,$10^{-3}$,$10^{-2}$,$10^{-1}$,$10^{0}$,$10^{1}$,$10^{2}$,$10^{3}$,$10^{4}$,},
tick pos=both
]
\addplot [thick, black, opacity=0.4, dotted, mark=x, mark size=2, mark options={solid,fill opacity=0}, forget plot]
table {%
0.0219923706361 0.280572
0.0231485452804 0.032683
0.0473996127045 0.0073700000000001
};
\addplot [thick, black, opacity=0.4, dotted, mark=x, mark size=2, mark options={solid,fill opacity=0}, forget plot]
table {%
0.0146468171491 56.319003
0.0146825404421 0.281625
0.0171486181992 0.0369269999999999
0.0456595617549 0.00841200000000009
};
\addplot [thick, black, opacity=0.4, dotted, mark=x, mark size=2, mark options={solid,fill opacity=0}, forget plot]
table {%
0.00773597291035 71.243986
0.0079395211684 0.479064
0.0129739708107 0.0495500000000002
0.045076536669 0.00707199999999997
};
\addplot [thick, black, opacity=0.4, dotted, mark=x, mark size=2, mark options={solid,fill opacity=0}, forget plot]
table {%
0.00475918310523 52.994155
0.0051763040426 0.503328
0.0120103618342 0.0421529999999999
0.0451498063531 0.00845800000000008
};
\addplot [thick, black, opacity=0.4, dotted, mark=x, mark size=2, mark options={solid,fill opacity=0}, forget plot]
table {%
0.00232937038259 96.979911
0.00321933670459 1.083645
0.0117499134021 0.110281
0.045355625829 0.0184980000000006
};
\addplot [thick, black, opacity=0.4, dotted, mark=x, mark size=2, mark options={solid,fill opacity=0}, forget plot]
table {%
0.00139560009836 115.91719
0.00268287982911 1.983264
0.011787563253 0.200733
0.0454708567658 0.0245000000000002
};
\addplot [thick, black, opacity=0.4, dotted, mark=x, mark size=2, mark options={solid,fill opacity=0}, forget plot]
table {%
0.000712175432305 164.04346
0.00244887368137 4.321099
0.0118688266345 0.397337
0.0455720926749 0.0873309999999998
};
\addplot [thick, black, opacity=0.4, dotted, mark=x, mark size=2, mark options={solid,fill opacity=0}, forget plot]
table {%
0.000477321676295 285.061344
0.0024110537785 10.507983
0.0119135687612 0.877356000000001
0.0456154902222 0.140544
};
\addplot [thick, black, opacity=0.4, dotted, mark=x, mark size=2, mark options={solid,fill opacity=0}, forget plot]
table {%
0.0003730803792 640.047029
0.00240481031792 31.319652
0.0119444028253 2.397552
0.0456429401838 0.375838
};
\addplot [thick, black, opacity=0.4, dotted, mark=x, mark size=2, mark options={solid,fill opacity=0}, forget plot]
table {%
0.000346112906001 1759.361261
0.00240622839308 95.844415
0.011959031944 7.988817
0.0456554324994 1.096945
};
\addplot [thick, black, dash pattern=on 1pt off 3pt on 3pt off 3pt]
table {%
0.0001 423239.328877998
0.000107226722201032 340701.6921795
0.000114975699539774 274244.076807414
0.000123284673944207 220736.68873795
0.000132194114846603 177658.377646415
0.000141747416292681 142978.373492576
0.000151991108295294 115060.984459645
0.000162975083462064 92588.8015041979
0.000174752840000768 74500.818434034
0.000187381742286039 59942.5727609829
0.000200923300256505 48225.9739746426
0.000215443469003188 38796.9384804832
0.000231012970008316 31209.3153141528
0.000247707635599171 25103.8808769865
0.000265608778294669 20191.4180395954
0.00028480358684358 16239.0860817427
0.000305385550883342 13059.4419882983
0.000327454916287773 10501.5977879835
0.000351119173421514 8444.09869718282
0.000376493580679247 6789.18748632729
0.000403701725859656 5458.18548717894
0.000432876128108306 4387.7730419552
0.000464158883361278 3526.99440731665
0.000497702356433211 2834.84614103554
0.000533669923120631 2278.33540732314
0.000572236765935022 1830.91672053959
0.000613590727341318 1471.23343988438
0.000657933224657568 1182.1046627339
0.000705480231071865 949.709713000246
0.000756463327554629 762.93172439459
0.000811130830789688 612.829312925118
0.000869749002617784 492.211369825956
0.000932603346883221 395.294868644612
0.001 317.429496658499
0.00107226722201032 254.876075065581
0.00114975699539774 204.628272694764
0.00123284673944207 164.269163711963
0.00132194114846603 131.855827159449
0.00141747416292681 105.826512663797
0.00151991108295294 84.9259647202143
0.00162975083462064 68.1453579070896
0.00174752840000769 54.6739877217507
0.00187381742286038 43.8604190934022
0.00200923300256505 35.1812433093705
0.00215443469003189 28.2159552585332
0.00231012970008316 22.6267536027607
0.00247707635599171 18.1423004657214
0.00265608778294669 14.5446655369967
0.0028480358684358 11.658831033046
0.00305385550883342 9.34425590542035
0.00327454916287773 7.48809581404042
0.00351119173421513 5.99975433747202
0.00376493580679247 4.80650441509896
0.00403701725859656 3.84997012042086
0.00432876128108306 3.08329997542798
0.00464158883361278 2.46889608512166
0.00497702356433211 1.97658996989634
0.00533669923120632 1.58217736619662
0.00572236765935022 1.26624147027971
0.00613590727341317 1.01320793501471
0.00657933224657568 0.810586054446102
0.00705480231071864 0.648359515605939
0.00756463327554629 0.518497288423507
0.00811130830789687 0.414561005802286
0.00869749002617784 0.33138983315015
0.00932603346883221 0.26484756199189
0.01 0.211619664438999
0.0107226722201032 0.169050457951661
0.0114975699539774 0.135012468582112
0.0123284673944207 0.107801638685976
0.0132194114846603 0.0860532766724825
0.0141747416292681 0.068674651835017
0.0151991108295293 0.0547909449807835
0.0162975083462065 0.0437019143099813
0.0174752840000768 0.0348471570094675
0.0187381742286039 0.0277782654258214
0.0200923300256505 0.0221365126440983
0.0215443469003188 0.0176349720365833
0.0231012970008316 0.0140441918913688
0.0247707635599171 0.0111807200544562
0.0265608778294669 0.008897913034398
0.028480358684358 0.00707857598434937
0.0305385550883342 0.00562906982254237
0.0327454916287773 0.00447459384009732
0.0351119173421513 0.00355540997776119
0.0376493580679247 0.00282382134387065
0.0403701725859656 0.00224175475366278
0.0432876128108306 0.00177882690890076
0.0464158883361278 0.00141079776292666
0.0497702356433211 0.00111833379875714
0.0533669923120631 0.000886019325070111
0.0572236765935022 0.000701566220019842
0.0613590727341318 0.000555182430145048
0.0657933224657568 0.000439067446158306
0.0705480231071865 0.000347009318211629
0.0756463327554629 0.000274062852452426
0.0811130830789687 0.000216292698679454
0.0869749002617784 0.000170568296474342
0.093260334688322 0.000134400255339168
0.1 0.0001058098322195
};
\label{\figlabel-line1}
\addplot [thick, black, dashed]
table {%
0.0001 2958.06499642139
0.000107226722201032 2533.94031177136
0.000114975699539774 2170.37327630492
0.000123284673944207 1858.75045876355
0.000132194114846603 1591.67922140058
0.000141747416292681 1362.81526290182
0.000151991108295294 1166.71444949076
0.000162975083462064 998.705523258718
0.000174752840000768 854.780754519427
0.000187381742286039 731.502015994051
0.000200923300256505 625.920110191836
0.000215443469003188 535.505485474149
0.000231012970008316 458.088737871722
0.000247707635599171 391.809520703052
0.000265608778294669 335.072677525486
0.00028480358684358 286.5105803382
0.000305385550883342 244.950798032762
0.000327454916287773 209.388343110538
0.000351119173421514 178.961850459888
0.000376493580679247 152.933132922859
0.000403701725859656 130.669636556367
0.000432876128108306 111.629385693204
0.000464158883361278 95.3480656700718
0.000497702356433211 81.4279407355213
0.000533669923120631 69.5283473174526
0.000572236765935022 59.3575394960833
0.000613590727341318 50.6656950356365
0.000657933224657568 43.2389174002405
0.000705480231071865 36.8940924392158
0.000756463327554629 31.4744784094651
0.000811130830789688 26.8459251684521
0.000869749002617784 22.8936331161756
0.000932603346883221 19.5193751289983
0.001 16.6391156048703
0.00107226722201032 14.1809700799791
0.00114975699539774 12.083456897487
0.00123284673944207 10.2939992956943
0.00132194114846603 8.76764219552019
0.00141747416292681 7.46595304296641
0.00151991108295294 6.35608041927789
0.00162975083462064 5.40994787127154
0.00174752840000769 4.6035636239434
0.00187381742286038 3.91642959187773
0.00200923300256505 3.3310354695173
0.00215443469003189 2.83242570829301
0.00231012970008316 2.40782892843824
0.00247707635599171 2.04634080581864
0.00265608778294669 1.73865275432465
0.0028480358684358 1.47681982238756
0.00305385550883342 1.25406216382855
0.00327454916287773 1.06459525071853
0.00351119173421513 0.903484688263565
0.00376493580679247 0.766522085289607
0.00403701725859656 0.650118942720255
0.00432876128108306 0.551215958572727
0.00464158883361278 0.467205521783352
0.00497702356433211 0.395865487491264
0.00533669923120632 0.335302600875061
0.00572236765935022 0.283904171788411
0.00613590727341317 0.240296803884752
0.00657933224657568 0.203311154469702
0.00705480231071864 0.171951849099563
0.00756463327554629 0.145371801495314
0.00811130830789687 0.122850297699924
0.00869749002617784 0.103774296176057
0.00932603346883221 0.0876224749540735
0.01 0.0739516249105348
0.0107226722201032 0.0623850464263971
0.0114975699539774 0.0526026564481951
0.0123284673944207 0.0443325555605584
0.0132194114846603 0.0373438411075674
0.0141747416292681 0.0314404835591648
0.0151991108295293 0.0264561099657769
0.0162975083462065 0.022249561129367
0.0174752840000768 0.0187011085991373
0.0187381742286039 0.0157092342518651
0.0200923300256505 0.0131878894488149
0.0215443469003188 0.0110641629230196
0.0231012970008316 0.00927629694190239
0.0247707635599171 0.00777200016704239
0.0265608778294669 0.00650701321480119
0.028480358684358 0.00544388939630109
0.0305385550883342 0.00455095864359323
0.0327454916287773 0.00380144734826769
0.0351119173421513 0.00317273086687891
0.0376493580679247 0.00264569888500741
0.0403701725859656 0.00220421676461474
0.0432876128108306 0.00183466850116662
0.0464158883361278 0.00152556905072115
0.0497702356433211 0.00126723560659859
0.0533669923120631 0.00105150895634419
0.0572236765935022 0.000871517371760993
0.0613590727341318 0.000721476611401018
0.0657933224657568 0.000596520574746176
0.0705480231071865 0.00049255796488746
0.0756463327554629 0.000406151012749154
0.0811130830789687 0.000334412908672761
0.0869749002617784 0.000274921091729391
0.093260334688322 0.000225643976491221
0.1 0.000184879062276337
};
\label{\figlabel-line3}
\addplot [ultra thick, blue]
table {%
0.000346112906001 1759.361261
0.0003730803792 640.047029
0.000477321676295 285.061344
0.000712175432305 164.04346
0.00139560009836 115.91719
0.00232937038259 96.979911
0.00240481031792 31.319652
0.0024110537785 10.507983
0.00244887368137 4.321099
0.00268287982911 1.983264
0.00321933670459 1.083645
0.0051763040426 0.503328
0.0079395211684 0.479064
0.0117499134021 0.110281
0.0120103618342 0.0421529999999999
0.0171486181992 0.0369269999999999
0.0231485452804 0.032683
0.045076536669 0.00707199999999997
};
\label{\figlabel-line0}
\addplot [ultra thick, green!50.0!black]
table {%
0.000381566361572 161.411429
0.000451865796974 110.254917
0.000585790565792 36.67568
0.0024728679842 2.408645
0.00266604551364 1.18852
0.00311350959574 0.573986
0.0113418394376 0.204287
0.0115006079295 0.127699000000001
0.0124560343379 0.0566240000000005
0.0443133452182 0.024389
0.045512938072 0.0135700000000007
0.0473996127045 0.00540899999999978
};
\label{\figlabel-line2}
\addplot [ultra thick, red]
table {%
0.0001960709611192 737.808295999975
0.000196854020298 727.989599999988
0.00019804267732275 703.357109999984
0.000198603139171 692.055464000021
0.000199217701188 686.303994000003
0.000201205139752 666.920267000025
0.000202277559396667 658.144116999918
0.00020299445505 687.917508999979
0.000204172098731 616.318218999908
0.000205601422325667 668.136470999978
0.00020695197751575 650.707283999933
0.000207189973371 640.80303899992
0.000208255156965 640.700297999881
0.000209293310891 633.02301599995
0.000210497834136 626.100400999976
0.0002110656993055 615.800344999896
0.0002121289537415 598.275698999934
0.00021355474089 534.704004999994
0.0002144376853995 473.763580000073
0.000214863960154667 468.744607999966
0.0002162747768775 453.037967999991
0.000217025117762 404.578728000031
0.000218028211424 418.020339000018
0.000218802922344 416.438449999996
0.000219540432973 397.53161
0.000220985433509333 389.251668000006
0.000222027028759 384.714124000029
0.000223273544902 378.819726000025
0.000223641710546 369.962953999996
0.0002246833374775 360.460670999958
0.000226459466349 355.004198999997
0.0002278963871455 349.150304999982
0.000228448407873 345.759503999974
0.0002295748511238 328.664476999985
0.000230658385783 343.060018999947
0.000231432388934 325.868679999957
0.000232840166556667 316.403130999986
0.000233845033977 312.544023999975
0.000235061101871 309.696189999997
0.000236289876817667 275.996198999973
0.000237021541556 270.629156999968
0.000237641899166 264.188015999958
0.000238861588659333 258.673287999934
0.000240236713054667 251.762497999925
0.000241400224986 238.172312999961
0.0002423824746896 193.238509999997
0.000243209009407945 171.328636999989
0.000244510391002583 170.308438000012
0.000245448292627889 169.713555999993
0.0002514116945725 168.710344999979
0.000252312856879 165.042215999967
0.000253078487053067 160.130260999989
0.000254301324113 252.070107000016
0.000255685449247 259.965375000005
0.00025664855162 259.185443000016
0.000258033353835 252.979473000021
0.000259258378160625 156.902840000011
0.000259564185494 190.801435000003
0.000260665249915 158.368874000003
0.000262555296169 155.809023000001
0.000271821357645 156.199467000004
0.000273146157854 194.731583000009
0.0002745402101 204.407146000006
0.000275706888827 204.952800000008
0.000279031961344 211.663478000002
0.000281613775824333 176.205423999999
0.0002825831778344 143.320203000015
0.0002873768395395 125.370210000013
0.000290928390999 118.915649000011
0.000291518843047429 109.903805000014
0.000292507597082333 107.819865000021
0.000295409021421 106.52829200002
0.000296831862402 104.366019000013
0.000299747804516 97.8695310000253
0.00030051986085475 93.9857490000245
0.000302247549054 94.8908050000132
0.000303130251286 86.7264560000199
0.000304198980762 99.4044830000199
0.00031196363135225 82.4881840000194
0.000312784552774 82.4365290000147
0.000316057226580333 81.3968730000047
0.000322939430171 90.5538970000303
0.000325451023899 90.9088230000175
0.000328297896535 91.6993660000185
0.000329646255921 89.7327080000091
0.000331620727146667 87.6943980000138
0.000333927321716 88.7284220000152
0.000335208690935 84.826789000023
0.000337222825823 92.6448950000154
0.0003404374447565 89.2392550000164
0.0003417188045385 86.8321870000123
0.000343748703051 86.5258290000134
0.000346811421293 85.7166620000158
0.000347592665874 79.7095380000142
0.000349832932715333 73.6484740000138
0.000350657361504 76.5035010000179
0.000351790949572 68.4041140000159
0.000355189775868 76.8232230000181
0.0003567716952615 68.2209290000133
0.000359228010425 65.8621950000148
0.000362370974331 67.5965520000141
0.0003638470968518 73.0185990000078
0.0003654546503845 70.5382170000008
0.00036716586277 72.8126120000021
0.000368858490075 69.998957000004
0.0003703217690545 68.9518970000041
0.0003719916421635 61.0345159999993
0.00037341143468 69.1439410000008
0.000375017896589 59.392171
0.000376239087398 57.1620410000065
0.000380563621461 53.3604929999961
0.000383259838497 49.715608999994
0.0003853494346595 50.4747429999948
0.000386195906104 56.6000979999977
0.000396178762489 58.9059259999991
0.000397553316306 53.7578409999983
0.00040059381862 52.0624370000028
0.000403765067384 49.1649820000026
0.000406319199868 50.4132580000056
0.000415648338658 51.5216580000047
0.0004223096414522 50.2579040000042
0.000423775785566 47.2688190000095
0.000425732040548 45.5429130000098
0.0004282857708295 44.9886620000121
0.000430738152901 43.9581370000065
0.000432783000265 45.0125630000089
0.00043434641816 40.9138830000116
0.000437876509574 41.9253240000112
0.000439393788988333 38.1977680000105
0.000442367915847333 32.9127590000112
0.000443522961622 32.0322690000144
0.000445484308653529 31.1379250000105
0.00044702992839225 30.9396380000126
0.0004495685620275 29.7996620000118
0.000451297117691429 29.5381790000138
0.000453790036123 29.5791800000111
0.000455738948988 29.3170920000107
0.000457718227683 30.19855900001
0.000459946523051 28.6124760000117
0.000460720303909 28.5383980000093
0.0004780664734075 28.5216040000101
0.000504233355798 33.163295000009
0.0005060332322185 28.3121280000055
0.000509468836734 27.9349580000053
0.000509686034308 27.935909000004
0.000512997822149 28.0319760000057
0.000544860367978 27.9930950000062
0.000567859081628 27.8581090000062
0.000569800301941375 27.8265840000077
0.000596597141971 27.870156000005
0.000599072725923 28.9226660000077
0.0006276311424835 27.9553890000075
0.000699834225015286 28.0023080000074
0.0007016768976054 26.8475090000061
0.000704293847567667 27.5996560000036
0.000828017262612 26.2921910000051
0.000829293815150857 25.6969290000011
0.0008330600142377 25.9733319999987
0.000837256138046875 28.1570379999976
0.000840622521942857 26.4580189999975
0.000856196724845 27.4307909999971
0.0008590828447325 28.2941499999961
0.0009101524214365 26.8135270000012
0.000914848168306 28.0521579999982
0.0009181191978315 27.3343399999995
0.00095230470152 28.7968520000005
0.000954761222611 27.3446679999998
0.000959790925001 27.3936859999997
0.000981163360295928 14.2786110000002
0.000985524338298875 14.8479650000001
0.000991251030629584 22.6564560000015
0.000994346231230176 15.7006250000001
0.000998799281019742 21.4489660000015
0.00100165426563 16.0921910000004
0.00100749973353 15.8624130000005
0.00101138130559 15.6328850000003
0.00110563288523667 15.165496
0.0012457397417 13.2431530000002
0.00125521263354 15.0831399999999
0.00134898317914 13.4151099999999
0.00135611294534 15.4558780000002
0.00147141347477 13.1815700000005
0.00147854612149 12.9118660000004
0.00151228225734 3.34232000000002
0.00152762647136 3.24622100000004
0.00153169773782 13.2362430000003
0.00154298529365 13.0405530000004
0.0015526802791 12.8866380000003
0.00160323936772857 3.35218899999998
0.00160776268993437 9.75096099999995
0.00161595835644875 8.70159200000012
0.00162365803446333 7.32040800000015
0.00162906210730889 3.40064100000006
0.0016397705949975 3.53028800000006
0.00164430546038833 3.47047100000006
0.0016494392719275 3.77707500000005
0.00166219183754 3.33250100000006
0.001665512593 3.27130800000004
0.0016776084294 3.17432900000002
0.00169713595455 3.21848799999996
0.00172505790934 3.28989099999999
0.00173927036233 3.24402699999998
0.00175821376495 3.16073899999998
0.00185990376843 3.39204699999996
0.001874236136795 3.11851399999996
0.00200837974719 3.00519699999999
0.00202386821951 2.99724899999996
0.00204695500937 3.00066099999995
0.00205922574622 2.89798599999998
0.00207299838124 2.96865799999999
0.00207363653959 3.05860699999998
0.00208760631721667 2.80298499999997
0.00215259003478625 2.41263500000001
0.00217377843189 2.85465699999998
0.00252399607583 2.372277
0.00285229592348 2.35633000000002
0.00288848750212 2.28218900000003
0.00292826364476 2.29672400000001
0.003426858180955 2.51615700000003
0.00393458200465 2.138846
0.00398398449074 2.24905700000001
0.00398607940283 2.212697
0.00403096434561 2.15239300000001
0.0041822005356 2.13866999999999
0.004190465814158 1.910265
0.00421124559292667 1.858181
0.0042509789662 1.83105299999998
0.00427443175164 1.76151999999998
0.004282512831564 1.592591
0.00429752568439 1.647836
0.00432481144988 1.49686699999999
0.00434296669641667 1.38529099999998
0.00440247627817 1.36668799999999
0.00442786081377 1.40409499999999
0.00444601752833 1.302578
0.00447335183174 1.25027299999999
0.00450357449599 1.14247299999999
0.00451522277149667 0.412717999999999
0.00453942048376 0.401827999999999
0.00455911329184333 0.374944
0.004578781292512 0.343254
0.00459771037051 0.322570999999999
0.004622683267108 0.508585999999997
0.00463707588289111 0.554004999999997
0.0046610916286025 0.525983999999996
0.00551433624408 0.307455
0.00786218126462 0.307433
0.00800750761571 0.302148999999999
0.0085054816497 0.271561
0.00888182786324 0.254143999999999
0.00921071739139 0.24361
0.00944671915919 0.235196000000001
0.00974764103562 0.226171
0.00978738616827 0.215927000000001
0.0100046770699 0.206945
0.0102440683766 0.176523
0.01027439170435 0.163981999999999
0.0103644001146 0.186165000000001
0.0105621558774 0.154950999999999
0.0110994895487 0.146933999999999
0.0113329095418 0.132124999999999
0.0117176084402 0.130152
0.0118040737031 0.11099
0.012124501541 0.099470999999999
0.0122354578118 0.0884109999999998
0.0125027666171 0.0281459999999991
0.0125966541808 0.0852890000000004
0.012662267381 0.0588949999999997
0.0127585080144 0.036886999999999
0.0128663968753 0.064945
0.0129423751977 0.0687649999999995
0.0130108248144 0.0742459999999996
0.0160678013788 0.021407
};
\label{\figlabel-line4}
\coordinate (legend) at (axis description cs:0.03,0.03);
\end{axis}

\matrix [matrix of nodes,
inner sep=1pt, row sep=1pt,cells={anchor=west},anchor={south west},at={(0.03,0.03)}, anchor=south west, draw=none, fill=none] at (legend) {
\ref{\figlabel-line0} {SL}\\
\ref{\figlabel-line1} {$\epsilon^{-3}\log(\epsilon^{-1})$}\\
\ref{\figlabel-line2} {ML}\\
\ref{\figlabel-line3} {$\epsilon^{-2}\log(\epsilon^{-1})^{2}$}\\
\ref{\figlabel-line4} {Adaptive ML}\\
};
\end{tikzpicture}
      \caption{$d=6$}
	\end{subfigure}
	\caption{Similar to \Cref{fig:kink-work}, but showing the total
      running time of the methods instead of their work estimate. The
      discrepancy of the two figures is due to the overhead of
      sampling the points, assembling the projection matrix and
      computing the projection. Moreover, this plot shows the overhead
      of the adaptive algorithm compared to the non-adaptive one.}
	\label{fig:kink-time}
  \end{figure}


%   \begin{tcolorbox}
% 	Another possible example is:
% 	$a(x,\psmi):=|x-\psmi|^5$, with $\domPS:=U:=[0,1]^2$.
% 	In terms of, we then have
% 	\begin{equation*}
% 	a\in C^5(U\times \domPS)
% 	\end{equation*}
% 	(strictly speaking, the last derivative is not continuous, but lets ignore that).
% 	Using finite elements of order $5$ should thus give us
% 	\begin{align*}
% 	\sc=2\\
% 	\wc=4\\
% 	\alpha=4/d=2\\
% 	\end{align*}
% (the "strong" convergence in \Cref{pro:finite} happens in $C^{4}(\domPS)$ and such functions are approximable by polynomials of degree $k$ at rate $(k+1)^{-4}$. Again, divide by $d=2$ because space of polynomials of degree $k$ has dimension $k^2$

% 	Finally, optimal solvers should have $\gamma=2$.
% 	Since $\gamma/\sc=1>1/\alpha=1/2$ we expect
% 	\begin{align*}
% 	\rate=\theta \gamma/\sc+(1-\theta)1/\alpha=1/2+1/2*1/2=3/4.
% 	\end{align*}

% 	Replacing $5$ by $3$ should give you $\gamma/\sc=1/\alpha=1$.
% 	\end{tcolorbox}
% \subsection{Matern-like example}
% Our second example is the same as the one in
% \cite{Haji-AliNobileTamelliniEtAl2015}. More specifically, we let
% \begin{equation*}
% a_{\psmi}(x)=\exp\left( \sum_{\vec k \in \N^D} A_{\vec k}
% \sum_{\vec \ell \in \{0,1\}^D} \gamma_{\vec k,\vec \ell } \, \prod_{i=1}^D
% \left(\cos\left({\pi }  k_i  x_i \right)\right)^{\ell_i}
% \left(\sin\left({\pi }  k_i  x_i \right)\right)^{1-\ell_i} \right).
% \end{equation*}
% for $\gamma_{\vec k,\vec \ell } \in [-1,1]$ and
% \begin{equation}
% A_{\vec k} {= \left(\sqrt{3}\right)} 2^{\frac{|\vec k|_0}{2}}(1 + |\vec k|^2)^{-\frac{\nu+D/2}{2}},
% \end{equation}
% for some $\nu>0$. The summability of $a_{\psmi}$ is controlled by
% $\nu$. Namely, using the notation of \Cref{pro:UQ} we have that
% $(\|\psi_j\|_{L^{\infty}})_{j\in\N}\in \ell^p(\N)$ with
% $p = \left( \frac{\nu}{D} + \frac{1}{2} \right)^{-1}$ hence $\alpha$
% in Assumption A1($\infty$) is $\frac{\nu}{D} - \frac{1}{2}$. On the
% other hand, the solver we used employs a second order
% finite-difference method with step size $h$ along each dimension,
% which asymptotically converge at the rate $h^{2}$ in the $L^2$ norm
% and require the computational work $h^{-D}$, corresponding to the
% values $\beta=2$ and $\gamma=D$ for the parameters in Assumptions A2
% and A3.

% The quantity of interest in this example is:
% \begin{equation*}
% \QoI(\pde_{\gamma}):= \frac{10}{(\sigma\sqrt{2\pi})^D}
% \int_\mathscr{B} u_\gamma(x) \exp \left( -\frac{\|x-x_o\|_2^2}{2\sigma^2} \right) \;dx.
% \end{equation*}
% with $\sigma=0.2$ and location $x_o=0.3$ for $D=1$ and
% $x_0 =(0.3, 0.2,0.6)$ for $D=3$.

% \begin{tcolorbox}
% 	Consider $D=1$. In terms of \Cref{pro:UQ} we have:
% 	\begin{align*}
% 	r_{\max}&=\nu+1/2\\
% 	\delta&=1.
% 	\end{align*}
% Choosing $r:=1$ in \Cref{pro:UQ} shows that we can take
% 	\begin{align*}
% 	\sc&=1+1=2\\
% 	\alpha&=r_{\max}-1-1-\epsilon=\nu-3/2-\epsilon
% 	\end{align*}
% 	(meaning we can take any $\epsilon>0$. Below let's just assume we can take $\epsilon=0$.)
% 	Choosing $r:=\nu$ shows that we can take
% 	\begin{align*}
% 	\wc=1+\nu.
% 	\end{align*}
% Finally, optimal solver should have
% 	\begin{align*}
% 	\gamma=1.
% 	\end{align*}

% 	Let's for example say $\nu=3.5$. Then $\gamma/\sc=1/\alpha=1/2$ and we expect $\lambda=1/2$.


% %	\textbf{Mean square convergence}
% %If you consider mean square convergence instead, then $\alpha$ improves	by $1/2$ %and you expect (with $\theta=2/4.5=4/9$)
% %\begin{align*}
% %\lambda=\theta \gamma/\sc+(1-\theta)1/\alpha=4/9*1/2+5/9*1/2.5=4/9
% %\end{align*}
% \end{tcolorbox}

% %\begin{figure}[h]
% %	\centering
% %	\input{./figures/kl2/kl2.tex}
% %	\caption{Convergence of nonadaptive and adaptive multilevel algorithm for smooth infinite-dimensional problem.}
% %	\label{fig:kl}
% %\end{figure}

% %\subsection{Smooth case}
% %We let
% %\begin{equation*}
% %a_{\psmi}(x)=\exp\left(\sum_{j=1}^{\infty}\ps_j\psi_j(x)\right)
% %\end{equation*}
% %with
% %\begin{equation*}
% %\psi_j(x_1,x_2):=j^{-4}\phi_{\sigma_j(1)}(x_1)\phi_{\sigma_j(2)}(x_2),
% %\end{equation*}
% %\begin{equation*}
% %\phi_{n}(x):=\begin{cases}
% %	\sin(\frac{n}{2}\pi x)\quad&\text{if }n\text{ is even}\\
% %	\cos(\frac{n-1}{2}\pi x)\quad&\text{else}.
% %\end{cases}
% %\end{equation*}
% %and an enumeration $(\sigma_{j})_{j=1}^{\infty}$ of $\N^2$.
% %
% %
% %Our goal is to approximate the response surface
% %\begin{equation*}
% %\psmi\mapsto \rs(\psmi):=\QoI(\pde_{\psmi}),
% %\end{equation*}
% %with the quantity of interest
% %\begin{equation*}
% %\QoI(\pde):=\int_{[0,1]^2}u\;dx.
% %\end{equation*}
% %Since we are using piecewise linear finite elements as before, the work required to obtain an error of size $\epsilon>0$ grows at least like $\epsilon^{-1}$. \Cref{fig:kl} shows that both the adaptive and the non-adaptive algorithm are able to achieve this rate.
% %%\begin{figure}[h]
% %%	\centering
% %%	\input{./figures/kl/kl.tex}
% %%	\caption{Convergence of multilevel algorithm for smooth infinite-dimensional  problem.}
% %%	\label{fig:kl}
% %%\end{figure}
% %\begin{figure}[h]
% %	\centering
% %	\input{./figures/kl2/kl2.tex}
% %	\caption{Convergence of nonadaptive and adaptive multilevel algorithm for smooth infinite-dimensional problem.}
% %	\label{fig:kl}
% %\end{figure}


%%% Local Variables:
%%% mode: latex
%%% TeX-master: "../document"
%%% End:
