\section{Numerical Experiments}
\label{sec:numerics}
To support our theoretical analysis, we performed numerical experiments on linear elliptic parametric PDEs of the form
%\subsection{Linear elliptic PDE}
%\label{ssec:numlin}
%We consider the PDE

\begin{equation}
\label{eq:pdenum}
\begin{aligned}
-\nabla \cdot (a(x,\psmi) \nabla \pde(x,\psmi))&=1&&\text{ in }U:=[-1,1]^D\\
\pde(x,\psmi)&=0&&\text{ on } \partial U,
\end{aligned}
\end{equation}
as in \Cref{sec:uq}.
%In the pre-asymptotic regime of our numerical experiments we observed convergence of the approximations at the rate $h^{-2.2}$ that the required computational work behaved like $h^{1.7}$ and . This corresponds to the values $\beta=2.2$, $\gamma=1.7$ for the parameters in \Cref{sec:nonadaptive}.
%\subsection{Non-smooth case}
We let
\begin{equation*}
a(x,\psmi)={1 + \|x\|_2^{r} + \|\psmi\|_2^{s}},\quad \psmi\in\domPS:=[-1,1]^d
\end{equation*}
for $r := 1$, $s := 3$, $D:=2$ and $d\in\{2,3,4,6\}$.
Our goal was to approximate the response surface
\begin{equation*}
\psmi\mapsto \rs(\psmi):=\QoI(\pde(\cdot,\psmi)):=0.5 \int_{U}\pde(\cdot,\psmi)\;dx
\end{equation*}
in $L^2(\Gamma)$.
%
%The results agree with the theory in \Cref{sec:nonadaptive}.
%Since the dependence of the coefficient $a_{\psmi}$ on $\psmi$ has a kink at $\psmi=0$, the response surface $\rs$ is only Lipschitz continuous. By \cite[Theorem 2]{BagbyBosLevenberg2002}) it can be approximated by polynomials of degree at most $k$ with accuracy $k^{-1}$.
%
%For $d\geq 1$ and $p\in\{2,\infty\}$, similar results hold with $\alpha<1$ and $\dvsp=\dim \vsp$, where $\vsp$ is a downward closed polynomial space (see e.g. \cite[Theorem 2]{BagbyBosLevenberg2002}).
The numerical scheme we used to solve \Cref{eq:pdenum} employs
centered finite difference approximations to the derivatives with a
constant mesh size, $h$. Such a numerical scheme converges
asymptotically at a rate of $\mathcal O(h^{2})$ in the $L^2$ norm
and requires a computational work of $\mathcal O(h^{-2})$, since the
PDE is two-dimensional. This corresponds to the values
$\beta_s = \beta_w = 2$ and $\gamma=2$ for the parameters in
Assumptions A2 and A3.
%%%%%
    % The
    % value of $\alpha$ in A1$(\infty)$ was fitted and found to be
    % $\alpha \approx 1$ for $d=1$ and $\alpha \approx 0.5$ for $d=2$.
    % \todo{Using what method}.  Using these values we can expect the
    % multilevel weighted least squares approximation to have a
    % complexity of $\mathcal{O}({\epsilon^{-1}})$ for $d=1$ and of
    % around $\mathcal{O}({\epsilon^{-2}})$ for $d=2$.
To estimate the projection error of our
estimate we evaluate the $L^2$ error norm using Monte Carlo sampling with $M=1000$ samples,
\begin{equation}\label{eq:l2-mc-error}
  \norm{f - S_L(f)}{L^2(\Gamma)}^2 \approx \frac{1}{M} \sum_{j=1}^M
  (f_{L+1}(\psmi_j) - S_L(f)(\psmi_j))^2.
\end{equation}
In our tests we employ both the nonadaptive and the adaptive
algorithms from Sections 4 and 5. As a basis for the nonadaptive
algorithm, we use total degree polynomial spaces
\(
  \vsp_{\dvsp} := \vspan\left\{ \leg_{\eta} : |\eta|_{1} \leq \dvsp \right\},
\)
where $\leg_\eta$ is a tensor product of Legendre
polynomials as in \Cref{sec:adaptive}.  We also compare the multilevel algorithm to the
straightforward, single-level approach, which for a given polynomial
approximation space $\vsp_{\dvsp}$ uses samples from a fixed PDE discretization
level that matches the accuracy of the polynomial best
approximation in $\vsp_{\dvsp}$. To find these matching PDE discretization levels, we consider the complexity curve of the single-level
method as the lower envelope of complexity curves with different
PDE discretization levels. Even though such a method is not practical, the
choice of discretization level for a given tolerance is always
optimal. The random points were sampled from the optimal
distribution as explained in \Cref{sec:optimal-sampling}.

Before presenting the numerical results, let us derive some a-priori
estimates of the complexity of the single-level and multilevel
projection methods.  From \Cref{pro:finite}, if
$a \in C^{r}(U)\otimes C^{s}(\domPS)$, then using finite elements
of order $r$ and mesh size $h$ would yield convergence in the space $F:=C^{s}(\domPS)$ with the values $\sc=\wc=r+1$ of the parameters in \Cref{sec:nonadaptive}, and  optimal solvers
would require the work $\mathcal{O}(h^{-\gamma})$,  $\gamma:=D$. Furthermore, since functions in
$C^{s}(\domPS)$ are approximable by polynomials of total degree less than or equal to $k$ at the rate $\mathcal{O}(k^{-s})$ in the supremum norm \cite{BagbyBosLevenberg2002}, we expect at least $\alpha=s$.
Even though our choice $a(x,\psmi) = 1 + \|  x\|_2^{r} + \| \psmi\|_2^{s}$
satisfies only $a \in C^{r-1, 1}(U)\otimes C^{s-1,1}(\domPS)$, we do not expect different rates than those derived above for $a\in C^{r}(U)\otimes C^{s}(\domPS)$. Finally, the dimension of total degree polynomial spaces $\vsp_{\dvsp}$ equals $\binom{\dvsp+d}{d}$ and asymptotically we have $\binom{\dvsp+d}{d}\lesssim \dvsp^d$.

Thus, we expect the complexity of the single-level method to be
$\mathcal{O}\left(\epsilon^{-\frac{D}{r+1} -
    \frac{d}{s}}\log(\epsilon^{-1})\right)$, while the complexity of
the multilevel method is of
$\mathcal{O}\left({\epsilon^{-\max\left({\frac{D}{r + 1},
          \frac{d}{s}}\right)}} \log(\epsilon^{-1})^{t}\right)$,
where
\[t =
  \begin{cases}
    1 & \frac{D}{r + 1} > \frac{d}{s},\\
    3 + \frac{D}{r+1} & \frac{D}{r + 1} = \frac{d}{s},\\
    2 & \frac{D}{r + 1} < \frac{d}{s}.
  \end{cases}
\]
Hence, for $r=1$ and $s=3$, the complexity of the single-level
method is
$\mathcal{O}\left(\epsilon^{-1 -
    \frac{d}{3}}\log(\epsilon^{-1})\right)$ and the complexity of the
multilevel method is
$\mathcal{O}\left({\epsilon^{-\max({1, \frac{d}{3}})}}
  \log(\epsilon^{-1})^{t}\right)$ where
\[t =
  \begin{cases}
    1, & d < 3,\\
    4, & d = 3,\\
    2, & d > 3.
  \end{cases}
\]

\Cref{fig:kink-work} shows the work estimate as defined in
\cref{eq:workdef} versus the $L^2$ error approximation in
\Cref{eq:l2-mc-error}. The results for the multilevel algorithm displayed there were obtained with the work parameter $\dimp:=d/2$, which we found describes the pre-asymptotic behavior of $\dim \vsp_{\dvsp}=\binom{\dvsp+d}{d}$ better than $\dimp:=d$. The theoretical rates
satisfactorily match the obtained numerical rates, which show an
improvement of the multilevel methods over the single-level method. Note
that the work estimate does \emph{not} include the cost of generating
points or the cost of assembling the projection matrix and computing
the projection. On the other hand, these costs are included in
\Cref{fig:kink-time}, which shows the total running time in seconds of the
three different methods. While these plots still show the same
complexity rates as \Cref{fig:kink-work} for all three methods for
sufficiently small errors, these plots also show the overhead of the
multilevel methods, especially as $d$ increases. The overhead of the
adaptive algorithm for the multilevel method is especially significant
and more work needs to be done to reduce it.

\setlength\figureheight{7.4cm}
\setlength\figurewidth{8.2cm}
\providecommand{\figlabel}{fig:}

\begin{figure}
	\centering
	\begin{subfigure}{0.49\textwidth}
      \renewcommand{\figlabel}{fig:work-est-vs-error-d2}
      % This file was created by matplotlib2tikz v0.6.10.
\begin{tikzpicture}

\begin{axis}[
xlabel={Max Error},
ylabel={Work Estimate},
xmin=1e-05, xmax=0.1,
ymin=10, ymax=10000000,
xmode=log,
ymode=log,
axis on top,
name=\figlabel,
width=\figurewidth,
height=\figureheight,
xtick={1e-06,1e-05,0.0001,0.001,0.01,0.1,1,10},
xticklabels={,$10^{-5}$,$10^{-4}$,$10^{-3}$,$10^{-2}$,$10^{-1}$,,},
ytick={1,10,100,1000,10000,100000,1000000,10000000,100000000},
yticklabels={,$10^{1}$,$10^{2}$,$10^{3}$,$10^{4}$,$10^{5}$,$10^{6}$,$10^{7}$,},
tick pos=both
]
\addplot [thick, black, opacity=0.4, dotted, mark=x, mark size=2, mark options={solid,fill opacity=0}, forget plot]
table {%
0.0369623451546 120
0.0372614771795 33.6882590646
0.0544651671895 12
};
\addplot [thick, black, opacity=0.4, dotted, mark=x, mark size=2, mark options={solid,fill opacity=0}, forget plot]
table {%
0.0245481961676 994.2411520904
0.0247423094662 240
0.0251277204434 67.3765181294
0.0546298039869 24
};
\addplot [thick, black, opacity=0.4, dotted, mark=x, mark size=2, mark options={solid,fill opacity=0}, forget plot]
table {%
0.01294146219175 1988.482304176
0.0132315526855 480
0.01378949183 134.7530362588
0.0578140123296 48
};
\addplot [thick, black, opacity=0.4, dotted, mark=x, mark size=2, mark options={solid,fill opacity=0}, forget plot]
table {%
0.00794432726001667 3976.964608364
0.00835749102082 960
0.00910236800804 269.506072518
0.0599827769776 96
};
\addplot [thick, black, opacity=0.4, dotted, mark=x, mark size=2, mark options={solid,fill opacity=0}, forget plot]
table {%
0.00385477527667 7953.929216712
0.00456852438756 1920
0.0056613128853 539.012145036
0.0620678979094 192
};
\addplot [thick, black, opacity=0.4, dotted, mark=x, mark size=2, mark options={solid,fill opacity=0}, forget plot]
table {%
0.00226099166803 71156.37380646
0.00228054100492 15907.85843341
0.00329499256326 3840
0.00461446459391 1078.02429007
0.062948013559 384
};
\addplot [thick, black, opacity=0.4, dotted, mark=x, mark size=2, mark options={solid,fill opacity=0}, forget plot]
table {%
0.0010445467109675 142312.74761344
0.00109392185408 31815.71686694
0.0025720983848 7680
0.00405857663354 2156.04858014
0.0636452477121 768
};
\addplot [thick, black, opacity=0.4, dotted, mark=x, mark size=2, mark options={solid,fill opacity=0}, forget plot]
table {%
0.0005609622120305 284625.4952258
0.000653437109663 63631.4337338
0.00239813854189 15360
0.00392095687905 4312.09716028
0.0639281712417 1536
};
\addplot [thick, black, opacity=0.4, dotted, mark=x, mark size=2, mark options={solid,fill opacity=0}, forget plot]
table {%
0.000264467681055333 2623699.5408114
0.000266457400041 569250.9904515
0.000432004928221 127262.8674675
0.00233503884306 30720
0.00386412368807 8624.19432056
0.0641029550006 3072
};
\addplot [thick, black, opacity=0.4, dotted, mark=x, mark size=2, mark options={solid,fill opacity=0}, forget plot]
table {%
0.000132059976775 5247399.0816256
0.000134597669583 1138501.980904
0.000368612154141 254525.734935
0.00231851851758 61440
0.00384578519125 17248.38864112
0.0641815074831 6144
};
\addplot [thick, black, dash pattern=on 1pt off 3pt on 3pt off 3pt]
table {%
1e-05 193021887.814683
1.09749876549306e-05 163961840.496109
1.20450354025878e-05 139267630.459887
1.32194114846603e-05 118284622.624733
1.45082877849594e-05 100456167.171241
1.59228279334109e-05 85308957.1167019
1.74752840000768e-05 72440570.6818636
1.91791026167249e-05 61508872.9128733
2.10490414451202e-05 52222999.4640243
2.31012970008316e-05 44335686.6895027
2.53536449397011e-05 37636747.3038486
2.78255940220713e-05 31947520.7617652
3.05385550883341e-05 27116152.9535329
3.35160265093884e-05 23013581.4729542
3.67837977182863e-05 19530121.1527858
4.03701725859655e-05 16572560.2569017
4.43062145758388e-05 14061691.0767736
4.86260158006535e-05 11930210.0495089
5.33669923120631e-05 10120932.1913441
5.85702081805666e-05 8585272.87593681
6.42807311728432e-05 7281956.99540622
7.05480231071865e-05 6175921.50636993
7.74263682681127e-05 5237382.43861646
8.49753435908644e-05 4441041.76284724
9.3260334688322e-05 3765413.18875333
0.000102353102189903 3192249.09145162
0.000112332403297803 2706053.42462305
0.000123284673944207 2293667.74205591
0.000135304777457981 1943919.37485102
0.000148496826225446 1647322.4496375
0.000162975083462064 1395823.82663227
0.000178864952905744 1182587.22172474
0.000196304065004027 1001809.78502394
0.000215443469003188 848566.265886036
0.000236448941264541 718676.623829525
0.000259502421139973 608593.565075543
0.00028480358684358 515307.012003619
0.000312571584968824 436262.961442553
0.000343046928631492 369294.569210329
0.000376493580679247 312563.622706493
0.000413201240011534 264510.839178926
0.000453487850812858 223813.661794928
0.000497702356433211 189350.425020921
0.000546227721768434 160169.930310568
0.000599484250318941 135465.617187574
0.000657933224657568 114553.63728944
0.000722080901838546 96854.2430460833
0.000792482898353917 81875.9911535349
0.000869749002617784 69202.3362075097
0.000954548456661834 58480.2537744656
0.00104761575278967 49410.5864909792
0.00114975699539774 41739.8529350288
0.00126185688306602 35253.2982287533
0.00138488637139387 29768.998652038
0.00151991108295293 25132.8608545662
0.00166810053720006 21214.3803030242
0.00183073828029537 17903.0440294651
0.00200923300256505 15105.2801001109
0.00220513073990305 12741.8709632788
0.00242012826479438 10745.7603535303
0.00265608778294669 9060.19406048882
0.00291505306282518 7637.14389888693
0.00319926713779738 6435.97188251611
0.00351119173421513 5422.29811385303
0.00385352859371053 4567.04142745357
0.0042292428743895 3845.60651676445
0.00464158883361278 3237.19525659494
0.00509413801481638 2724.22331398912
0.00559081018251223 2291.82600944051
0.00613590727341317 1927.43982543428
0.00673415065775082 1620.44802571172
0.00739072203352578 1361.88060208895
0.00811130830789687 1144.16025340062
0.00890215085445039 960.887363333185
0.00977009957299226 806.658014658657
0.0107226722201032 676.909985635831
0.01176811952435 567.792444700022
0.0129154966501488 476.05571289057
0.0141747416292681 398.958017493127
0.0155567614393047 334.186630147957
0.0170735264747069 279.791180969527
0.0187381742286039 234.127277878778
0.0205651230834865 195.808846569854
0.0225701971963392 163.667849127434
0.0247707635599171 136.720244906646
0.0271858824273294 114.137231508356
0.0298364724028334 95.2209512999664
0.0327454916287773 79.3839739936263
0.0359381366380463 66.1319717340845
0.0394420605943766 55.0490928801436
0.0432876128108306 45.7856166573333
0.047508101621028 38.0475352103697
0.0521400828799969 31.5877640689064
0.0572236765935022 26.1987281656407
0.0628029144183425 21.7061095893008
0.068926121043497 17.9635762992856
0.075646332755463 14.8483389916287
0.0830217568131974 12.2574069654997
0.0911162756115489 10.1044338549612
0.1 8.31706102086792
};
\label{\figlabel-line1}
\addplot [thick, black, dashed]
table {%
1e-05 711740.79626965
1.09749876549306e-05 643271.206938027
1.20450354025878e-05 581349.818667477
1.32194114846603e-05 525353.549015187
1.45082877849594e-05 474718.342623386
1.59228279334109e-05 428933.601013713
1.74752840000768e-05 387537.136219963
1.91791026167249e-05 350110.599146875
2.10490414451202e-05 316275.338134456
2.31012970008316e-05 285688.647370952
2.53536449397011e-05 258040.368572739
2.78255940220713e-05 233049.812772363
3.05385550883341e-05 210462.972159392
3.35160265093884e-05 190049.994732534
3.67837977182863e-05 171602.897072463
4.03701725859655e-05 154933.492857557
4.43062145758388e-05 139871.51684148
4.86260158006535e-05 126262.92591228
5.33669923120631e-05 113968.360575814
5.85702081805666e-05 102861.751768325
6.42807311728432e-05 92829.0593189476
7.05480231071865e-05 83767.1296664276
7.74263682681127e-05 75582.6615977362
8.49753435908644e-05 68191.2698308178
9.3260334688322e-05 61516.6372195114
0.000102353102189903 55489.7472250006
0.000112332403297803 50048.1890833271
0.000123284673944207 45135.5288101089
0.000135304777457981 40700.7398285288
0.000148496826225446 36697.6875911242
0.000162975083462064 33084.6630955744
0.000178864952905744 29823.960674662
0.000196304065004027 26881.4958755365
0.000215443469003188 24226.4596375404
0.000236448941264541 21831.0053349864
0.000259502421139973 19669.96557487
0.00028480358684358 17720.5959327022
0.000312571584968824 15962.3430752914
0.000343046928631492 14376.6349599897
0.000376493580679247 12946.6910179643
0.000413201240011534 11657.3504266043
0.000453487850812858 10494.9167551272
0.000497702356433211 9447.01742957108
0.000546227721768434 8502.47661020885
0.000599484250318941 7651.20020745187
0.000657933224657568 6884.07188279991
0.000722080901838546 6192.85899053668
0.000792482898353917 5570.12751472013
0.000869749002617784 5009.16514554663
0.000954548456661834 4503.91172025189
0.00104761575278967 4048.89632714395
0.00114975699539774 3639.18043786411
0.00126185688306602 3270.30649319173
0.00138488637139387 2938.25142224092
0.00151991108295293 2639.38462427475
0.00166810053720006 2370.42998707287
0.00183073828029537 2128.43155626994
0.00200923300256505 1910.72250673345
0.00220513073990305 1714.8971002325
0.00242012826479438 1538.78534368935
0.00265608778294669 1380.43008950061
0.00291505306282518 1238.06634403335
0.00319926713779738 1110.10257268512
0.00351119173421513 995.103810067919
0.00385352859371053 891.776402133333
0.0042292428743895 798.954223579675
0.00464158883361278 715.586228837316
0.00509413801481638 640.72520846231
0.00559081018251223 573.517635016331
0.00613590727341317 513.194493594204
0.00673415065775082 459.063002189116
0.00739072203352578 410.499136160071
0.00811130830789687 366.940879276465
0.00890215085445039 327.882131242879
0.00977009957299226 292.867208327449
0.0107226722201032 261.485879796583
0.01176811952435 233.368888358261
0.0129154966501488 208.183907790674
0.0141747416292681 185.631895432429
0.0155567614393047 165.443801280123
0.0170735264747069 147.377599119601
0.0187381742286039 131.215608445705
0.0205651230834865 116.762078935315
0.0225701971963392 103.841011960201
0.0247707635599171 92.2941960872454
0.0271858824273294 81.9794357386748
0.0298364724028334 72.7689541966385
0.0327454916287773 64.547953955108
0.0359381366380463 57.2133190660633
0.0394420605943766 50.6724456129939
0.0432876128108306 44.8421877879096
0.047508101621028 39.6479082620407
0.0521400828799969 35.0226226375145
0.0572236765935022 30.9062287587497
0.0628029144183425 27.2448125581954
0.068926121043497 23.9900229205306
0.075646332755463 21.0985087808125
0.0830217568131974 18.5314123327906
0.0911162756115489 16.2539128204989
0.1 14.234815925393
};
\label{\figlabel-line3}
\addplot [ultra thick, blue]
table {%
0.000132059976775 5247399.0816256
0.000134597669583 1138501.980904
0.000266457400041 569250.9904515
0.000368612154141 254525.734935
0.000432004928221 127262.8674675
0.000653437109663 63631.4337338
0.00109392185408 31815.71686694
0.00228054100492 15907.85843341
0.00239813854189 15360
0.0025720983848 7680
0.00329499256326 3840
0.00405857663354 2156.04858014
0.00456852438756 1920
0.00461446459391 1078.02429007
0.0056613128853 539.012145036
0.00910236800804 269.506072518
0.01378949183 134.7530362588
0.0251277204434 67.3765181294
0.0372614771795 33.6882590646
0.0544651671895 12
};
\label{\figlabel-line0}
\addplot [ultra thick, green!50.0!black]
table {%
1.44639587272e-05 381616.272973868
3.78640122421e-05 167068.489783086
7.69552596262e-05 78409.8317259242
0.000181755811891 34080.502697318
0.000393103387445 15928.4330079298
0.000877573845432 6852.398163241
0.00140973705312 3177.6387935986
0.00329158299043 1340.2591087768
0.00451594669534 610.1295543882
0.00905869618308 245.064777194
0.0132448727133 105.6882590646
0.0533682590909 36
0.0544651671895 12
};
\label{\figlabel-line2}
\addplot [ultra thick, red]
table {%
1.3034378468e-05 383517.986738826
1.39050009416e-05 369609.163757975
1.48798437703e-05 335767.317465405
2.0074898237175e-05 267161.59028684
2.01188108224714e-05 245317.665178866
2.03324265740333e-05 309531.74886987
2.08081775932e-05 227937.783860046
2.54827506018e-05 219745.783860046
2.65519824786e-05 213297.747916658
2.72823539789e-05 206343.33642626
2.91569805826e-05 196547.462867188
3.701686162619e-05 152942.219742479
4.65300789151e-05 145851.211738779
4.69296625133e-05 135816.902038799
4.706852640005e-05 129790.125744773
4.7501381257825e-05 112688.763400133
5.088955632824e-05 101270.727143231
5.31329975684e-05 98046.7091715261
7.21409464253e-05 89854.7091715261
7.35018960717e-05 86377.5034263281
7.48626282901e-05 80226.7666864981
7.53449840032e-05 73763.3843938862
9.9333451235125e-05 65183.3788930986
9.99690003499e-05 58052.1552990853
0.000107236989272 56440.1463132343
0.000108754000751 54701.5434406319
0.0001126193183345 48864.6605372219
0.00014446327997 46816.6605372219
0.000196583722616 40318.7553358594
0.000197616639876333 36763.2998544486
0.0001989819586762 31104.3584289597
0.000199789998581 28595.7810039707
0.000212397094372 26823.0290030457
0.000219984164598 26017.024510123
0.000221974003399 25147.72307382
0.000227750879287 22229.2816221078
0.000230948709816 24401.7667869538
0.000235090033656 21205.2816221078
0.000327992842683 19157.2816221078
0.000329831117772333 16682.7716694665
0.00034922632046225 13951.6621560394
0.000368473400331 13065.2861555774
0.0003845962244095 11606.0654297174
0.000441431160307 11094.0654297174
0.0005753077757775 8350.0045625176
0.000646608734871 7279.6722060396
0.000653394667458 7906.8165622876
0.000660508158657 6736.5509148267
0.00071630841432 5928.8901096585
0.000724210191333 5400.3729236585
0.000726201391158 4623.4324289046
0.00101875768125 4111.4324289046
0.00102049864232 3797.8602507789
0.00138715636771 3304.7056050583
0.00139885001707 3576.2662506638
0.00148806368052 3176.7056050583
0.00149595611671667 2477.7220075841
0.00166427833724 2058.5596069494
0.00246312064836 1802.5596069494
0.00248308743525 1691.6317290194
0.00249925771548 1414.0893595726
0.00269048596292 1010.7259476494
0.00269649508992 1146.5062704521
0.00280521551337 946.7259476494
0.00282000929781 861.1535034983
0.003774509187635 727.3619589384
0.00463375049753 695.3619589384
0.00491080915654 594.61139732298
0.00577584791158 466.61139732298
0.005798797633845 305.47670006348
0.00886445468886 241.47670006348
0.013669101956 209.47670006348
0.024027090706 33.84962500724
0.0243881186303 49.84962500724
0.0245109480602 150.69047798802
0.02535514762155 83.79470570787
};
\label{\figlabel-line4}
\coordinate (legend) at (axis description cs:0.03,0.03);
\end{axis}

\matrix [matrix of nodes,
inner sep=1pt, row sep=1pt,cells={anchor=west},anchor={south west},at={(0.03,0.03)}, anchor=south west, draw=none, fill=none] at (legend) {
\ref{\figlabel-line0} {SL}\\
\ref{\figlabel-line1} {$\epsilon^{-\frac{ 5 }{ 3 }}\log(\epsilon^{-1})$}\\
\ref{\figlabel-line2} {ML}\\
\ref{\figlabel-line3} {$\epsilon^{-1}\log(\epsilon^{-1})$}\\
\ref{\figlabel-line4} {Adaptive ML}\\
};
\end{tikzpicture}
      \caption{$d=2$}
	\end{subfigure}
	\begin{subfigure}{0.5\textwidth}
      \renewcommand{\figlabel}{fig:work-est-vs-error-d3}
      % This file was created by matplotlib2tikz v0.6.10.
\begin{tikzpicture}

\begin{axis}[
xlabel={Max Error},
ylabel={Work Estimate},
xmin=1e-05, xmax=0.1,
ymin=10, ymax=10000000,
xmode=log,
ymode=log,
axis on top,
name=\figlabel,
width=\figurewidth,
height=\figureheight,
xtick={1e-06,1e-05,0.0001,0.001,0.01,0.1,1,10},
xticklabels={,$10^{-5}$,$10^{-4}$,$10^{-3}$,$10^{-2}$,$10^{-1}$,,},
ytick={1,10,100,1000,10000,100000,1000000,10000000,100000000},
yticklabels={,$10^{1}$,$10^{2}$,$10^{3}$,$10^{4}$,$10^{5}$,$10^{6}$,$10^{7}$,},
tick pos=both
]
\addplot [thick, black, opacity=0.4, dotted, mark=x, mark size=2, mark options={solid,fill opacity=0}, forget plot]
table {%
0.0369623451546 120
0.0372614771795 33.6882590646
0.0544651671895 12
};
\addplot [thick, black, opacity=0.4, dotted, mark=x, mark size=2, mark options={solid,fill opacity=0}, forget plot]
table {%
0.0245481961676 994.2411520904
0.0247423094662 240
0.0251277204434 67.3765181294
0.0546298039869 24
};
\addplot [thick, black, opacity=0.4, dotted, mark=x, mark size=2, mark options={solid,fill opacity=0}, forget plot]
table {%
0.01294146219175 1988.482304176
0.0132315526855 480
0.01378949183 134.7530362588
0.0578140123296 48
};
\addplot [thick, black, opacity=0.4, dotted, mark=x, mark size=2, mark options={solid,fill opacity=0}, forget plot]
table {%
0.00794432726001667 3976.964608364
0.00835749102082 960
0.00910236800804 269.506072518
0.0599827769776 96
};
\addplot [thick, black, opacity=0.4, dotted, mark=x, mark size=2, mark options={solid,fill opacity=0}, forget plot]
table {%
0.00385477527667 7953.929216712
0.00456852438756 1920
0.0056613128853 539.012145036
0.0620678979094 192
};
\addplot [thick, black, opacity=0.4, dotted, mark=x, mark size=2, mark options={solid,fill opacity=0}, forget plot]
table {%
0.00226099166803 71156.37380646
0.00228054100492 15907.85843341
0.00329499256326 3840
0.00461446459391 1078.02429007
0.062948013559 384
};
\addplot [thick, black, opacity=0.4, dotted, mark=x, mark size=2, mark options={solid,fill opacity=0}, forget plot]
table {%
0.0010445467109675 142312.74761344
0.00109392185408 31815.71686694
0.0025720983848 7680
0.00405857663354 2156.04858014
0.0636452477121 768
};
\addplot [thick, black, opacity=0.4, dotted, mark=x, mark size=2, mark options={solid,fill opacity=0}, forget plot]
table {%
0.0005609622120305 284625.4952258
0.000653437109663 63631.4337338
0.00239813854189 15360
0.00392095687905 4312.09716028
0.0639281712417 1536
};
\addplot [thick, black, opacity=0.4, dotted, mark=x, mark size=2, mark options={solid,fill opacity=0}, forget plot]
table {%
0.000264467681055333 2623699.5408114
0.000266457400041 569250.9904515
0.000432004928221 127262.8674675
0.00233503884306 30720
0.00386412368807 8624.19432056
0.0641029550006 3072
};
\addplot [thick, black, opacity=0.4, dotted, mark=x, mark size=2, mark options={solid,fill opacity=0}, forget plot]
table {%
0.000132059976775 5247399.0816256
0.000134597669583 1138501.980904
0.000368612154141 254525.734935
0.00231851851758 61440
0.00384578519125 17248.38864112
0.0641815074831 6144
};
\addplot [thick, black, dash pattern=on 1pt off 3pt on 3pt off 3pt]
table {%
1e-05 193021887.814683
1.09749876549306e-05 163961840.496109
1.20450354025878e-05 139267630.459887
1.32194114846603e-05 118284622.624733
1.45082877849594e-05 100456167.171241
1.59228279334109e-05 85308957.1167019
1.74752840000768e-05 72440570.6818636
1.91791026167249e-05 61508872.9128733
2.10490414451202e-05 52222999.4640243
2.31012970008316e-05 44335686.6895027
2.53536449397011e-05 37636747.3038486
2.78255940220713e-05 31947520.7617652
3.05385550883341e-05 27116152.9535329
3.35160265093884e-05 23013581.4729542
3.67837977182863e-05 19530121.1527858
4.03701725859655e-05 16572560.2569017
4.43062145758388e-05 14061691.0767736
4.86260158006535e-05 11930210.0495089
5.33669923120631e-05 10120932.1913441
5.85702081805666e-05 8585272.87593681
6.42807311728432e-05 7281956.99540622
7.05480231071865e-05 6175921.50636993
7.74263682681127e-05 5237382.43861646
8.49753435908644e-05 4441041.76284724
9.3260334688322e-05 3765413.18875333
0.000102353102189903 3192249.09145162
0.000112332403297803 2706053.42462305
0.000123284673944207 2293667.74205591
0.000135304777457981 1943919.37485102
0.000148496826225446 1647322.4496375
0.000162975083462064 1395823.82663227
0.000178864952905744 1182587.22172474
0.000196304065004027 1001809.78502394
0.000215443469003188 848566.265886036
0.000236448941264541 718676.623829525
0.000259502421139973 608593.565075543
0.00028480358684358 515307.012003619
0.000312571584968824 436262.961442553
0.000343046928631492 369294.569210329
0.000376493580679247 312563.622706493
0.000413201240011534 264510.839178926
0.000453487850812858 223813.661794928
0.000497702356433211 189350.425020921
0.000546227721768434 160169.930310568
0.000599484250318941 135465.617187574
0.000657933224657568 114553.63728944
0.000722080901838546 96854.2430460833
0.000792482898353917 81875.9911535349
0.000869749002617784 69202.3362075097
0.000954548456661834 58480.2537744656
0.00104761575278967 49410.5864909792
0.00114975699539774 41739.8529350288
0.00126185688306602 35253.2982287533
0.00138488637139387 29768.998652038
0.00151991108295293 25132.8608545662
0.00166810053720006 21214.3803030242
0.00183073828029537 17903.0440294651
0.00200923300256505 15105.2801001109
0.00220513073990305 12741.8709632788
0.00242012826479438 10745.7603535303
0.00265608778294669 9060.19406048882
0.00291505306282518 7637.14389888693
0.00319926713779738 6435.97188251611
0.00351119173421513 5422.29811385303
0.00385352859371053 4567.04142745357
0.0042292428743895 3845.60651676445
0.00464158883361278 3237.19525659494
0.00509413801481638 2724.22331398912
0.00559081018251223 2291.82600944051
0.00613590727341317 1927.43982543428
0.00673415065775082 1620.44802571172
0.00739072203352578 1361.88060208895
0.00811130830789687 1144.16025340062
0.00890215085445039 960.887363333185
0.00977009957299226 806.658014658657
0.0107226722201032 676.909985635831
0.01176811952435 567.792444700022
0.0129154966501488 476.05571289057
0.0141747416292681 398.958017493127
0.0155567614393047 334.186630147957
0.0170735264747069 279.791180969527
0.0187381742286039 234.127277878778
0.0205651230834865 195.808846569854
0.0225701971963392 163.667849127434
0.0247707635599171 136.720244906646
0.0271858824273294 114.137231508356
0.0298364724028334 95.2209512999664
0.0327454916287773 79.3839739936263
0.0359381366380463 66.1319717340845
0.0394420605943766 55.0490928801436
0.0432876128108306 45.7856166573333
0.047508101621028 38.0475352103697
0.0521400828799969 31.5877640689064
0.0572236765935022 26.1987281656407
0.0628029144183425 21.7061095893008
0.068926121043497 17.9635762992856
0.075646332755463 14.8483389916287
0.0830217568131974 12.2574069654997
0.0911162756115489 10.1044338549612
0.1 8.31706102086792
};
\label{\figlabel-line1}
\addplot [thick, black, dashed]
table {%
1e-05 711740.79626965
1.09749876549306e-05 643271.206938027
1.20450354025878e-05 581349.818667477
1.32194114846603e-05 525353.549015187
1.45082877849594e-05 474718.342623386
1.59228279334109e-05 428933.601013713
1.74752840000768e-05 387537.136219963
1.91791026167249e-05 350110.599146875
2.10490414451202e-05 316275.338134456
2.31012970008316e-05 285688.647370952
2.53536449397011e-05 258040.368572739
2.78255940220713e-05 233049.812772363
3.05385550883341e-05 210462.972159392
3.35160265093884e-05 190049.994732534
3.67837977182863e-05 171602.897072463
4.03701725859655e-05 154933.492857557
4.43062145758388e-05 139871.51684148
4.86260158006535e-05 126262.92591228
5.33669923120631e-05 113968.360575814
5.85702081805666e-05 102861.751768325
6.42807311728432e-05 92829.0593189476
7.05480231071865e-05 83767.1296664276
7.74263682681127e-05 75582.6615977362
8.49753435908644e-05 68191.2698308178
9.3260334688322e-05 61516.6372195114
0.000102353102189903 55489.7472250006
0.000112332403297803 50048.1890833271
0.000123284673944207 45135.5288101089
0.000135304777457981 40700.7398285288
0.000148496826225446 36697.6875911242
0.000162975083462064 33084.6630955744
0.000178864952905744 29823.960674662
0.000196304065004027 26881.4958755365
0.000215443469003188 24226.4596375404
0.000236448941264541 21831.0053349864
0.000259502421139973 19669.96557487
0.00028480358684358 17720.5959327022
0.000312571584968824 15962.3430752914
0.000343046928631492 14376.6349599897
0.000376493580679247 12946.6910179643
0.000413201240011534 11657.3504266043
0.000453487850812858 10494.9167551272
0.000497702356433211 9447.01742957108
0.000546227721768434 8502.47661020885
0.000599484250318941 7651.20020745187
0.000657933224657568 6884.07188279991
0.000722080901838546 6192.85899053668
0.000792482898353917 5570.12751472013
0.000869749002617784 5009.16514554663
0.000954548456661834 4503.91172025189
0.00104761575278967 4048.89632714395
0.00114975699539774 3639.18043786411
0.00126185688306602 3270.30649319173
0.00138488637139387 2938.25142224092
0.00151991108295293 2639.38462427475
0.00166810053720006 2370.42998707287
0.00183073828029537 2128.43155626994
0.00200923300256505 1910.72250673345
0.00220513073990305 1714.8971002325
0.00242012826479438 1538.78534368935
0.00265608778294669 1380.43008950061
0.00291505306282518 1238.06634403335
0.00319926713779738 1110.10257268512
0.00351119173421513 995.103810067919
0.00385352859371053 891.776402133333
0.0042292428743895 798.954223579675
0.00464158883361278 715.586228837316
0.00509413801481638 640.72520846231
0.00559081018251223 573.517635016331
0.00613590727341317 513.194493594204
0.00673415065775082 459.063002189116
0.00739072203352578 410.499136160071
0.00811130830789687 366.940879276465
0.00890215085445039 327.882131242879
0.00977009957299226 292.867208327449
0.0107226722201032 261.485879796583
0.01176811952435 233.368888358261
0.0129154966501488 208.183907790674
0.0141747416292681 185.631895432429
0.0155567614393047 165.443801280123
0.0170735264747069 147.377599119601
0.0187381742286039 131.215608445705
0.0205651230834865 116.762078935315
0.0225701971963392 103.841011960201
0.0247707635599171 92.2941960872454
0.0271858824273294 81.9794357386748
0.0298364724028334 72.7689541966385
0.0327454916287773 64.547953955108
0.0359381366380463 57.2133190660633
0.0394420605943766 50.6724456129939
0.0432876128108306 44.8421877879096
0.047508101621028 39.6479082620407
0.0521400828799969 35.0226226375145
0.0572236765935022 30.9062287587497
0.0628029144183425 27.2448125581954
0.068926121043497 23.9900229205306
0.075646332755463 21.0985087808125
0.0830217568131974 18.5314123327906
0.0911162756115489 16.2539128204989
0.1 14.234815925393
};
\label{\figlabel-line3}
\addplot [ultra thick, blue]
table {%
0.000132059976775 5247399.0816256
0.000134597669583 1138501.980904
0.000266457400041 569250.9904515
0.000368612154141 254525.734935
0.000432004928221 127262.8674675
0.000653437109663 63631.4337338
0.00109392185408 31815.71686694
0.00228054100492 15907.85843341
0.00239813854189 15360
0.0025720983848 7680
0.00329499256326 3840
0.00405857663354 2156.04858014
0.00456852438756 1920
0.00461446459391 1078.02429007
0.0056613128853 539.012145036
0.00910236800804 269.506072518
0.01378949183 134.7530362588
0.0251277204434 67.3765181294
0.0372614771795 33.6882590646
0.0544651671895 12
};
\label{\figlabel-line0}
\addplot [ultra thick, green!50.0!black]
table {%
1.44639587272e-05 381616.272973868
3.78640122421e-05 167068.489783086
7.69552596262e-05 78409.8317259242
0.000181755811891 34080.502697318
0.000393103387445 15928.4330079298
0.000877573845432 6852.398163241
0.00140973705312 3177.6387935986
0.00329158299043 1340.2591087768
0.00451594669534 610.1295543882
0.00905869618308 245.064777194
0.0132448727133 105.6882590646
0.0533682590909 36
0.0544651671895 12
};
\label{\figlabel-line2}
\addplot [ultra thick, red]
table {%
1.3034378468e-05 383517.986738826
1.39050009416e-05 369609.163757975
1.48798437703e-05 335767.317465405
2.0074898237175e-05 267161.59028684
2.01188108224714e-05 245317.665178866
2.03324265740333e-05 309531.74886987
2.08081775932e-05 227937.783860046
2.54827506018e-05 219745.783860046
2.65519824786e-05 213297.747916658
2.72823539789e-05 206343.33642626
2.91569805826e-05 196547.462867188
3.701686162619e-05 152942.219742479
4.65300789151e-05 145851.211738779
4.69296625133e-05 135816.902038799
4.706852640005e-05 129790.125744773
4.7501381257825e-05 112688.763400133
5.088955632824e-05 101270.727143231
5.31329975684e-05 98046.7091715261
7.21409464253e-05 89854.7091715261
7.35018960717e-05 86377.5034263281
7.48626282901e-05 80226.7666864981
7.53449840032e-05 73763.3843938862
9.9333451235125e-05 65183.3788930986
9.99690003499e-05 58052.1552990853
0.000107236989272 56440.1463132343
0.000108754000751 54701.5434406319
0.0001126193183345 48864.6605372219
0.00014446327997 46816.6605372219
0.000196583722616 40318.7553358594
0.000197616639876333 36763.2998544486
0.0001989819586762 31104.3584289597
0.000199789998581 28595.7810039707
0.000212397094372 26823.0290030457
0.000219984164598 26017.024510123
0.000221974003399 25147.72307382
0.000227750879287 22229.2816221078
0.000230948709816 24401.7667869538
0.000235090033656 21205.2816221078
0.000327992842683 19157.2816221078
0.000329831117772333 16682.7716694665
0.00034922632046225 13951.6621560394
0.000368473400331 13065.2861555774
0.0003845962244095 11606.0654297174
0.000441431160307 11094.0654297174
0.0005753077757775 8350.0045625176
0.000646608734871 7279.6722060396
0.000653394667458 7906.8165622876
0.000660508158657 6736.5509148267
0.00071630841432 5928.8901096585
0.000724210191333 5400.3729236585
0.000726201391158 4623.4324289046
0.00101875768125 4111.4324289046
0.00102049864232 3797.8602507789
0.00138715636771 3304.7056050583
0.00139885001707 3576.2662506638
0.00148806368052 3176.7056050583
0.00149595611671667 2477.7220075841
0.00166427833724 2058.5596069494
0.00246312064836 1802.5596069494
0.00248308743525 1691.6317290194
0.00249925771548 1414.0893595726
0.00269048596292 1010.7259476494
0.00269649508992 1146.5062704521
0.00280521551337 946.7259476494
0.00282000929781 861.1535034983
0.003774509187635 727.3619589384
0.00463375049753 695.3619589384
0.00491080915654 594.61139732298
0.00577584791158 466.61139732298
0.005798797633845 305.47670006348
0.00886445468886 241.47670006348
0.013669101956 209.47670006348
0.024027090706 33.84962500724
0.0243881186303 49.84962500724
0.0245109480602 150.69047798802
0.02535514762155 83.79470570787
};
\label{\figlabel-line4}
\coordinate (legend) at (axis description cs:0.03,0.03);
\end{axis}

\matrix [matrix of nodes,
inner sep=1pt, row sep=1pt,cells={anchor=west},anchor={south west},at={(0.03,0.03)}, anchor=south west, draw=none, fill=none] at (legend) {
\ref{\figlabel-line0} {SL}\\
\ref{\figlabel-line1} {$\epsilon^{-\frac{ 5 }{ 3 }}\log(\epsilon^{-1})$}\\
\ref{\figlabel-line2} {ML}\\
\ref{\figlabel-line3} {$\epsilon^{-1}\log(\epsilon^{-1})$}\\
\ref{\figlabel-line4} {Adaptive ML}\\
};
\end{tikzpicture}
      \caption{$d=3$}
	\end{subfigure}
	\begin{subfigure}{0.49\textwidth}
      \renewcommand{\figlabel}{fig:work-est-vs-error-d4}
      % This file was created by matplotlib2tikz v0.6.10.
\begin{tikzpicture}

\begin{axis}[
xlabel={Max Error},
ylabel={Work Estimate},
xmin=1e-05, xmax=0.1,
ymin=10, ymax=10000000,
xmode=log,
ymode=log,
axis on top,
name=\figlabel,
width=\figurewidth,
height=\figureheight,
xtick={1e-06,1e-05,0.0001,0.001,0.01,0.1,1,10},
xticklabels={,$10^{-5}$,$10^{-4}$,$10^{-3}$,$10^{-2}$,$10^{-1}$,,},
ytick={1,10,100,1000,10000,100000,1000000,10000000,100000000},
yticklabels={,$10^{1}$,$10^{2}$,$10^{3}$,$10^{4}$,$10^{5}$,$10^{6}$,$10^{7}$,},
tick pos=both
]
\addplot [thick, black, opacity=0.4, dotted, mark=x, mark size=2, mark options={solid,fill opacity=0}, forget plot]
table {%
0.0369623451546 120
0.0372614771795 33.6882590646
0.0544651671895 12
};
\addplot [thick, black, opacity=0.4, dotted, mark=x, mark size=2, mark options={solid,fill opacity=0}, forget plot]
table {%
0.0245481961676 994.2411520904
0.0247423094662 240
0.0251277204434 67.3765181294
0.0546298039869 24
};
\addplot [thick, black, opacity=0.4, dotted, mark=x, mark size=2, mark options={solid,fill opacity=0}, forget plot]
table {%
0.01294146219175 1988.482304176
0.0132315526855 480
0.01378949183 134.7530362588
0.0578140123296 48
};
\addplot [thick, black, opacity=0.4, dotted, mark=x, mark size=2, mark options={solid,fill opacity=0}, forget plot]
table {%
0.00794432726001667 3976.964608364
0.00835749102082 960
0.00910236800804 269.506072518
0.0599827769776 96
};
\addplot [thick, black, opacity=0.4, dotted, mark=x, mark size=2, mark options={solid,fill opacity=0}, forget plot]
table {%
0.00385477527667 7953.929216712
0.00456852438756 1920
0.0056613128853 539.012145036
0.0620678979094 192
};
\addplot [thick, black, opacity=0.4, dotted, mark=x, mark size=2, mark options={solid,fill opacity=0}, forget plot]
table {%
0.00226099166803 71156.37380646
0.00228054100492 15907.85843341
0.00329499256326 3840
0.00461446459391 1078.02429007
0.062948013559 384
};
\addplot [thick, black, opacity=0.4, dotted, mark=x, mark size=2, mark options={solid,fill opacity=0}, forget plot]
table {%
0.0010445467109675 142312.74761344
0.00109392185408 31815.71686694
0.0025720983848 7680
0.00405857663354 2156.04858014
0.0636452477121 768
};
\addplot [thick, black, opacity=0.4, dotted, mark=x, mark size=2, mark options={solid,fill opacity=0}, forget plot]
table {%
0.0005609622120305 284625.4952258
0.000653437109663 63631.4337338
0.00239813854189 15360
0.00392095687905 4312.09716028
0.0639281712417 1536
};
\addplot [thick, black, opacity=0.4, dotted, mark=x, mark size=2, mark options={solid,fill opacity=0}, forget plot]
table {%
0.000264467681055333 2623699.5408114
0.000266457400041 569250.9904515
0.000432004928221 127262.8674675
0.00233503884306 30720
0.00386412368807 8624.19432056
0.0641029550006 3072
};
\addplot [thick, black, opacity=0.4, dotted, mark=x, mark size=2, mark options={solid,fill opacity=0}, forget plot]
table {%
0.000132059976775 5247399.0816256
0.000134597669583 1138501.980904
0.000368612154141 254525.734935
0.00231851851758 61440
0.00384578519125 17248.38864112
0.0641815074831 6144
};
\addplot [thick, black, dash pattern=on 1pt off 3pt on 3pt off 3pt]
table {%
1e-05 193021887.814683
1.09749876549306e-05 163961840.496109
1.20450354025878e-05 139267630.459887
1.32194114846603e-05 118284622.624733
1.45082877849594e-05 100456167.171241
1.59228279334109e-05 85308957.1167019
1.74752840000768e-05 72440570.6818636
1.91791026167249e-05 61508872.9128733
2.10490414451202e-05 52222999.4640243
2.31012970008316e-05 44335686.6895027
2.53536449397011e-05 37636747.3038486
2.78255940220713e-05 31947520.7617652
3.05385550883341e-05 27116152.9535329
3.35160265093884e-05 23013581.4729542
3.67837977182863e-05 19530121.1527858
4.03701725859655e-05 16572560.2569017
4.43062145758388e-05 14061691.0767736
4.86260158006535e-05 11930210.0495089
5.33669923120631e-05 10120932.1913441
5.85702081805666e-05 8585272.87593681
6.42807311728432e-05 7281956.99540622
7.05480231071865e-05 6175921.50636993
7.74263682681127e-05 5237382.43861646
8.49753435908644e-05 4441041.76284724
9.3260334688322e-05 3765413.18875333
0.000102353102189903 3192249.09145162
0.000112332403297803 2706053.42462305
0.000123284673944207 2293667.74205591
0.000135304777457981 1943919.37485102
0.000148496826225446 1647322.4496375
0.000162975083462064 1395823.82663227
0.000178864952905744 1182587.22172474
0.000196304065004027 1001809.78502394
0.000215443469003188 848566.265886036
0.000236448941264541 718676.623829525
0.000259502421139973 608593.565075543
0.00028480358684358 515307.012003619
0.000312571584968824 436262.961442553
0.000343046928631492 369294.569210329
0.000376493580679247 312563.622706493
0.000413201240011534 264510.839178926
0.000453487850812858 223813.661794928
0.000497702356433211 189350.425020921
0.000546227721768434 160169.930310568
0.000599484250318941 135465.617187574
0.000657933224657568 114553.63728944
0.000722080901838546 96854.2430460833
0.000792482898353917 81875.9911535349
0.000869749002617784 69202.3362075097
0.000954548456661834 58480.2537744656
0.00104761575278967 49410.5864909792
0.00114975699539774 41739.8529350288
0.00126185688306602 35253.2982287533
0.00138488637139387 29768.998652038
0.00151991108295293 25132.8608545662
0.00166810053720006 21214.3803030242
0.00183073828029537 17903.0440294651
0.00200923300256505 15105.2801001109
0.00220513073990305 12741.8709632788
0.00242012826479438 10745.7603535303
0.00265608778294669 9060.19406048882
0.00291505306282518 7637.14389888693
0.00319926713779738 6435.97188251611
0.00351119173421513 5422.29811385303
0.00385352859371053 4567.04142745357
0.0042292428743895 3845.60651676445
0.00464158883361278 3237.19525659494
0.00509413801481638 2724.22331398912
0.00559081018251223 2291.82600944051
0.00613590727341317 1927.43982543428
0.00673415065775082 1620.44802571172
0.00739072203352578 1361.88060208895
0.00811130830789687 1144.16025340062
0.00890215085445039 960.887363333185
0.00977009957299226 806.658014658657
0.0107226722201032 676.909985635831
0.01176811952435 567.792444700022
0.0129154966501488 476.05571289057
0.0141747416292681 398.958017493127
0.0155567614393047 334.186630147957
0.0170735264747069 279.791180969527
0.0187381742286039 234.127277878778
0.0205651230834865 195.808846569854
0.0225701971963392 163.667849127434
0.0247707635599171 136.720244906646
0.0271858824273294 114.137231508356
0.0298364724028334 95.2209512999664
0.0327454916287773 79.3839739936263
0.0359381366380463 66.1319717340845
0.0394420605943766 55.0490928801436
0.0432876128108306 45.7856166573333
0.047508101621028 38.0475352103697
0.0521400828799969 31.5877640689064
0.0572236765935022 26.1987281656407
0.0628029144183425 21.7061095893008
0.068926121043497 17.9635762992856
0.075646332755463 14.8483389916287
0.0830217568131974 12.2574069654997
0.0911162756115489 10.1044338549612
0.1 8.31706102086792
};
\label{\figlabel-line1}
\addplot [thick, black, dashed]
table {%
1e-05 711740.79626965
1.09749876549306e-05 643271.206938027
1.20450354025878e-05 581349.818667477
1.32194114846603e-05 525353.549015187
1.45082877849594e-05 474718.342623386
1.59228279334109e-05 428933.601013713
1.74752840000768e-05 387537.136219963
1.91791026167249e-05 350110.599146875
2.10490414451202e-05 316275.338134456
2.31012970008316e-05 285688.647370952
2.53536449397011e-05 258040.368572739
2.78255940220713e-05 233049.812772363
3.05385550883341e-05 210462.972159392
3.35160265093884e-05 190049.994732534
3.67837977182863e-05 171602.897072463
4.03701725859655e-05 154933.492857557
4.43062145758388e-05 139871.51684148
4.86260158006535e-05 126262.92591228
5.33669923120631e-05 113968.360575814
5.85702081805666e-05 102861.751768325
6.42807311728432e-05 92829.0593189476
7.05480231071865e-05 83767.1296664276
7.74263682681127e-05 75582.6615977362
8.49753435908644e-05 68191.2698308178
9.3260334688322e-05 61516.6372195114
0.000102353102189903 55489.7472250006
0.000112332403297803 50048.1890833271
0.000123284673944207 45135.5288101089
0.000135304777457981 40700.7398285288
0.000148496826225446 36697.6875911242
0.000162975083462064 33084.6630955744
0.000178864952905744 29823.960674662
0.000196304065004027 26881.4958755365
0.000215443469003188 24226.4596375404
0.000236448941264541 21831.0053349864
0.000259502421139973 19669.96557487
0.00028480358684358 17720.5959327022
0.000312571584968824 15962.3430752914
0.000343046928631492 14376.6349599897
0.000376493580679247 12946.6910179643
0.000413201240011534 11657.3504266043
0.000453487850812858 10494.9167551272
0.000497702356433211 9447.01742957108
0.000546227721768434 8502.47661020885
0.000599484250318941 7651.20020745187
0.000657933224657568 6884.07188279991
0.000722080901838546 6192.85899053668
0.000792482898353917 5570.12751472013
0.000869749002617784 5009.16514554663
0.000954548456661834 4503.91172025189
0.00104761575278967 4048.89632714395
0.00114975699539774 3639.18043786411
0.00126185688306602 3270.30649319173
0.00138488637139387 2938.25142224092
0.00151991108295293 2639.38462427475
0.00166810053720006 2370.42998707287
0.00183073828029537 2128.43155626994
0.00200923300256505 1910.72250673345
0.00220513073990305 1714.8971002325
0.00242012826479438 1538.78534368935
0.00265608778294669 1380.43008950061
0.00291505306282518 1238.06634403335
0.00319926713779738 1110.10257268512
0.00351119173421513 995.103810067919
0.00385352859371053 891.776402133333
0.0042292428743895 798.954223579675
0.00464158883361278 715.586228837316
0.00509413801481638 640.72520846231
0.00559081018251223 573.517635016331
0.00613590727341317 513.194493594204
0.00673415065775082 459.063002189116
0.00739072203352578 410.499136160071
0.00811130830789687 366.940879276465
0.00890215085445039 327.882131242879
0.00977009957299226 292.867208327449
0.0107226722201032 261.485879796583
0.01176811952435 233.368888358261
0.0129154966501488 208.183907790674
0.0141747416292681 185.631895432429
0.0155567614393047 165.443801280123
0.0170735264747069 147.377599119601
0.0187381742286039 131.215608445705
0.0205651230834865 116.762078935315
0.0225701971963392 103.841011960201
0.0247707635599171 92.2941960872454
0.0271858824273294 81.9794357386748
0.0298364724028334 72.7689541966385
0.0327454916287773 64.547953955108
0.0359381366380463 57.2133190660633
0.0394420605943766 50.6724456129939
0.0432876128108306 44.8421877879096
0.047508101621028 39.6479082620407
0.0521400828799969 35.0226226375145
0.0572236765935022 30.9062287587497
0.0628029144183425 27.2448125581954
0.068926121043497 23.9900229205306
0.075646332755463 21.0985087808125
0.0830217568131974 18.5314123327906
0.0911162756115489 16.2539128204989
0.1 14.234815925393
};
\label{\figlabel-line3}
\addplot [ultra thick, blue]
table {%
0.000132059976775 5247399.0816256
0.000134597669583 1138501.980904
0.000266457400041 569250.9904515
0.000368612154141 254525.734935
0.000432004928221 127262.8674675
0.000653437109663 63631.4337338
0.00109392185408 31815.71686694
0.00228054100492 15907.85843341
0.00239813854189 15360
0.0025720983848 7680
0.00329499256326 3840
0.00405857663354 2156.04858014
0.00456852438756 1920
0.00461446459391 1078.02429007
0.0056613128853 539.012145036
0.00910236800804 269.506072518
0.01378949183 134.7530362588
0.0251277204434 67.3765181294
0.0372614771795 33.6882590646
0.0544651671895 12
};
\label{\figlabel-line0}
\addplot [ultra thick, green!50.0!black]
table {%
1.44639587272e-05 381616.272973868
3.78640122421e-05 167068.489783086
7.69552596262e-05 78409.8317259242
0.000181755811891 34080.502697318
0.000393103387445 15928.4330079298
0.000877573845432 6852.398163241
0.00140973705312 3177.6387935986
0.00329158299043 1340.2591087768
0.00451594669534 610.1295543882
0.00905869618308 245.064777194
0.0132448727133 105.6882590646
0.0533682590909 36
0.0544651671895 12
};
\label{\figlabel-line2}
\addplot [ultra thick, red]
table {%
1.3034378468e-05 383517.986738826
1.39050009416e-05 369609.163757975
1.48798437703e-05 335767.317465405
2.0074898237175e-05 267161.59028684
2.01188108224714e-05 245317.665178866
2.03324265740333e-05 309531.74886987
2.08081775932e-05 227937.783860046
2.54827506018e-05 219745.783860046
2.65519824786e-05 213297.747916658
2.72823539789e-05 206343.33642626
2.91569805826e-05 196547.462867188
3.701686162619e-05 152942.219742479
4.65300789151e-05 145851.211738779
4.69296625133e-05 135816.902038799
4.706852640005e-05 129790.125744773
4.7501381257825e-05 112688.763400133
5.088955632824e-05 101270.727143231
5.31329975684e-05 98046.7091715261
7.21409464253e-05 89854.7091715261
7.35018960717e-05 86377.5034263281
7.48626282901e-05 80226.7666864981
7.53449840032e-05 73763.3843938862
9.9333451235125e-05 65183.3788930986
9.99690003499e-05 58052.1552990853
0.000107236989272 56440.1463132343
0.000108754000751 54701.5434406319
0.0001126193183345 48864.6605372219
0.00014446327997 46816.6605372219
0.000196583722616 40318.7553358594
0.000197616639876333 36763.2998544486
0.0001989819586762 31104.3584289597
0.000199789998581 28595.7810039707
0.000212397094372 26823.0290030457
0.000219984164598 26017.024510123
0.000221974003399 25147.72307382
0.000227750879287 22229.2816221078
0.000230948709816 24401.7667869538
0.000235090033656 21205.2816221078
0.000327992842683 19157.2816221078
0.000329831117772333 16682.7716694665
0.00034922632046225 13951.6621560394
0.000368473400331 13065.2861555774
0.0003845962244095 11606.0654297174
0.000441431160307 11094.0654297174
0.0005753077757775 8350.0045625176
0.000646608734871 7279.6722060396
0.000653394667458 7906.8165622876
0.000660508158657 6736.5509148267
0.00071630841432 5928.8901096585
0.000724210191333 5400.3729236585
0.000726201391158 4623.4324289046
0.00101875768125 4111.4324289046
0.00102049864232 3797.8602507789
0.00138715636771 3304.7056050583
0.00139885001707 3576.2662506638
0.00148806368052 3176.7056050583
0.00149595611671667 2477.7220075841
0.00166427833724 2058.5596069494
0.00246312064836 1802.5596069494
0.00248308743525 1691.6317290194
0.00249925771548 1414.0893595726
0.00269048596292 1010.7259476494
0.00269649508992 1146.5062704521
0.00280521551337 946.7259476494
0.00282000929781 861.1535034983
0.003774509187635 727.3619589384
0.00463375049753 695.3619589384
0.00491080915654 594.61139732298
0.00577584791158 466.61139732298
0.005798797633845 305.47670006348
0.00886445468886 241.47670006348
0.013669101956 209.47670006348
0.024027090706 33.84962500724
0.0243881186303 49.84962500724
0.0245109480602 150.69047798802
0.02535514762155 83.79470570787
};
\label{\figlabel-line4}
\coordinate (legend) at (axis description cs:0.03,0.03);
\end{axis}

\matrix [matrix of nodes,
inner sep=1pt, row sep=1pt,cells={anchor=west},anchor={south west},at={(0.03,0.03)}, anchor=south west, draw=none, fill=none] at (legend) {
\ref{\figlabel-line0} {SL}\\
\ref{\figlabel-line1} {$\epsilon^{-\frac{ 5 }{ 3 }}\log(\epsilon^{-1})$}\\
\ref{\figlabel-line2} {ML}\\
\ref{\figlabel-line3} {$\epsilon^{-1}\log(\epsilon^{-1})$}\\
\ref{\figlabel-line4} {Adaptive ML}\\
};
\end{tikzpicture}
      \caption{$d=4$}
	\end{subfigure}
	\begin{subfigure}{0.5\textwidth}
      \renewcommand{\figlabel}{fig:work-est-vs-error-d6}
      % This file was created by matplotlib2tikz v0.6.10.
\begin{tikzpicture}

\begin{axis}[
xlabel={Max Error},
ylabel={Work Estimate},
xmin=1e-05, xmax=0.1,
ymin=10, ymax=10000000,
xmode=log,
ymode=log,
axis on top,
name=\figlabel,
width=\figurewidth,
height=\figureheight,
xtick={1e-06,1e-05,0.0001,0.001,0.01,0.1,1,10},
xticklabels={,$10^{-5}$,$10^{-4}$,$10^{-3}$,$10^{-2}$,$10^{-1}$,,},
ytick={1,10,100,1000,10000,100000,1000000,10000000,100000000},
yticklabels={,$10^{1}$,$10^{2}$,$10^{3}$,$10^{4}$,$10^{5}$,$10^{6}$,$10^{7}$,},
tick pos=both
]
\addplot [thick, black, opacity=0.4, dotted, mark=x, mark size=2, mark options={solid,fill opacity=0}, forget plot]
table {%
0.0369623451546 120
0.0372614771795 33.6882590646
0.0544651671895 12
};
\addplot [thick, black, opacity=0.4, dotted, mark=x, mark size=2, mark options={solid,fill opacity=0}, forget plot]
table {%
0.0245481961676 994.2411520904
0.0247423094662 240
0.0251277204434 67.3765181294
0.0546298039869 24
};
\addplot [thick, black, opacity=0.4, dotted, mark=x, mark size=2, mark options={solid,fill opacity=0}, forget plot]
table {%
0.01294146219175 1988.482304176
0.0132315526855 480
0.01378949183 134.7530362588
0.0578140123296 48
};
\addplot [thick, black, opacity=0.4, dotted, mark=x, mark size=2, mark options={solid,fill opacity=0}, forget plot]
table {%
0.00794432726001667 3976.964608364
0.00835749102082 960
0.00910236800804 269.506072518
0.0599827769776 96
};
\addplot [thick, black, opacity=0.4, dotted, mark=x, mark size=2, mark options={solid,fill opacity=0}, forget plot]
table {%
0.00385477527667 7953.929216712
0.00456852438756 1920
0.0056613128853 539.012145036
0.0620678979094 192
};
\addplot [thick, black, opacity=0.4, dotted, mark=x, mark size=2, mark options={solid,fill opacity=0}, forget plot]
table {%
0.00226099166803 71156.37380646
0.00228054100492 15907.85843341
0.00329499256326 3840
0.00461446459391 1078.02429007
0.062948013559 384
};
\addplot [thick, black, opacity=0.4, dotted, mark=x, mark size=2, mark options={solid,fill opacity=0}, forget plot]
table {%
0.0010445467109675 142312.74761344
0.00109392185408 31815.71686694
0.0025720983848 7680
0.00405857663354 2156.04858014
0.0636452477121 768
};
\addplot [thick, black, opacity=0.4, dotted, mark=x, mark size=2, mark options={solid,fill opacity=0}, forget plot]
table {%
0.0005609622120305 284625.4952258
0.000653437109663 63631.4337338
0.00239813854189 15360
0.00392095687905 4312.09716028
0.0639281712417 1536
};
\addplot [thick, black, opacity=0.4, dotted, mark=x, mark size=2, mark options={solid,fill opacity=0}, forget plot]
table {%
0.000264467681055333 2623699.5408114
0.000266457400041 569250.9904515
0.000432004928221 127262.8674675
0.00233503884306 30720
0.00386412368807 8624.19432056
0.0641029550006 3072
};
\addplot [thick, black, opacity=0.4, dotted, mark=x, mark size=2, mark options={solid,fill opacity=0}, forget plot]
table {%
0.000132059976775 5247399.0816256
0.000134597669583 1138501.980904
0.000368612154141 254525.734935
0.00231851851758 61440
0.00384578519125 17248.38864112
0.0641815074831 6144
};
\addplot [thick, black, dash pattern=on 1pt off 3pt on 3pt off 3pt]
table {%
1e-05 193021887.814683
1.09749876549306e-05 163961840.496109
1.20450354025878e-05 139267630.459887
1.32194114846603e-05 118284622.624733
1.45082877849594e-05 100456167.171241
1.59228279334109e-05 85308957.1167019
1.74752840000768e-05 72440570.6818636
1.91791026167249e-05 61508872.9128733
2.10490414451202e-05 52222999.4640243
2.31012970008316e-05 44335686.6895027
2.53536449397011e-05 37636747.3038486
2.78255940220713e-05 31947520.7617652
3.05385550883341e-05 27116152.9535329
3.35160265093884e-05 23013581.4729542
3.67837977182863e-05 19530121.1527858
4.03701725859655e-05 16572560.2569017
4.43062145758388e-05 14061691.0767736
4.86260158006535e-05 11930210.0495089
5.33669923120631e-05 10120932.1913441
5.85702081805666e-05 8585272.87593681
6.42807311728432e-05 7281956.99540622
7.05480231071865e-05 6175921.50636993
7.74263682681127e-05 5237382.43861646
8.49753435908644e-05 4441041.76284724
9.3260334688322e-05 3765413.18875333
0.000102353102189903 3192249.09145162
0.000112332403297803 2706053.42462305
0.000123284673944207 2293667.74205591
0.000135304777457981 1943919.37485102
0.000148496826225446 1647322.4496375
0.000162975083462064 1395823.82663227
0.000178864952905744 1182587.22172474
0.000196304065004027 1001809.78502394
0.000215443469003188 848566.265886036
0.000236448941264541 718676.623829525
0.000259502421139973 608593.565075543
0.00028480358684358 515307.012003619
0.000312571584968824 436262.961442553
0.000343046928631492 369294.569210329
0.000376493580679247 312563.622706493
0.000413201240011534 264510.839178926
0.000453487850812858 223813.661794928
0.000497702356433211 189350.425020921
0.000546227721768434 160169.930310568
0.000599484250318941 135465.617187574
0.000657933224657568 114553.63728944
0.000722080901838546 96854.2430460833
0.000792482898353917 81875.9911535349
0.000869749002617784 69202.3362075097
0.000954548456661834 58480.2537744656
0.00104761575278967 49410.5864909792
0.00114975699539774 41739.8529350288
0.00126185688306602 35253.2982287533
0.00138488637139387 29768.998652038
0.00151991108295293 25132.8608545662
0.00166810053720006 21214.3803030242
0.00183073828029537 17903.0440294651
0.00200923300256505 15105.2801001109
0.00220513073990305 12741.8709632788
0.00242012826479438 10745.7603535303
0.00265608778294669 9060.19406048882
0.00291505306282518 7637.14389888693
0.00319926713779738 6435.97188251611
0.00351119173421513 5422.29811385303
0.00385352859371053 4567.04142745357
0.0042292428743895 3845.60651676445
0.00464158883361278 3237.19525659494
0.00509413801481638 2724.22331398912
0.00559081018251223 2291.82600944051
0.00613590727341317 1927.43982543428
0.00673415065775082 1620.44802571172
0.00739072203352578 1361.88060208895
0.00811130830789687 1144.16025340062
0.00890215085445039 960.887363333185
0.00977009957299226 806.658014658657
0.0107226722201032 676.909985635831
0.01176811952435 567.792444700022
0.0129154966501488 476.05571289057
0.0141747416292681 398.958017493127
0.0155567614393047 334.186630147957
0.0170735264747069 279.791180969527
0.0187381742286039 234.127277878778
0.0205651230834865 195.808846569854
0.0225701971963392 163.667849127434
0.0247707635599171 136.720244906646
0.0271858824273294 114.137231508356
0.0298364724028334 95.2209512999664
0.0327454916287773 79.3839739936263
0.0359381366380463 66.1319717340845
0.0394420605943766 55.0490928801436
0.0432876128108306 45.7856166573333
0.047508101621028 38.0475352103697
0.0521400828799969 31.5877640689064
0.0572236765935022 26.1987281656407
0.0628029144183425 21.7061095893008
0.068926121043497 17.9635762992856
0.075646332755463 14.8483389916287
0.0830217568131974 12.2574069654997
0.0911162756115489 10.1044338549612
0.1 8.31706102086792
};
\label{\figlabel-line1}
\addplot [thick, black, dashed]
table {%
1e-05 711740.79626965
1.09749876549306e-05 643271.206938027
1.20450354025878e-05 581349.818667477
1.32194114846603e-05 525353.549015187
1.45082877849594e-05 474718.342623386
1.59228279334109e-05 428933.601013713
1.74752840000768e-05 387537.136219963
1.91791026167249e-05 350110.599146875
2.10490414451202e-05 316275.338134456
2.31012970008316e-05 285688.647370952
2.53536449397011e-05 258040.368572739
2.78255940220713e-05 233049.812772363
3.05385550883341e-05 210462.972159392
3.35160265093884e-05 190049.994732534
3.67837977182863e-05 171602.897072463
4.03701725859655e-05 154933.492857557
4.43062145758388e-05 139871.51684148
4.86260158006535e-05 126262.92591228
5.33669923120631e-05 113968.360575814
5.85702081805666e-05 102861.751768325
6.42807311728432e-05 92829.0593189476
7.05480231071865e-05 83767.1296664276
7.74263682681127e-05 75582.6615977362
8.49753435908644e-05 68191.2698308178
9.3260334688322e-05 61516.6372195114
0.000102353102189903 55489.7472250006
0.000112332403297803 50048.1890833271
0.000123284673944207 45135.5288101089
0.000135304777457981 40700.7398285288
0.000148496826225446 36697.6875911242
0.000162975083462064 33084.6630955744
0.000178864952905744 29823.960674662
0.000196304065004027 26881.4958755365
0.000215443469003188 24226.4596375404
0.000236448941264541 21831.0053349864
0.000259502421139973 19669.96557487
0.00028480358684358 17720.5959327022
0.000312571584968824 15962.3430752914
0.000343046928631492 14376.6349599897
0.000376493580679247 12946.6910179643
0.000413201240011534 11657.3504266043
0.000453487850812858 10494.9167551272
0.000497702356433211 9447.01742957108
0.000546227721768434 8502.47661020885
0.000599484250318941 7651.20020745187
0.000657933224657568 6884.07188279991
0.000722080901838546 6192.85899053668
0.000792482898353917 5570.12751472013
0.000869749002617784 5009.16514554663
0.000954548456661834 4503.91172025189
0.00104761575278967 4048.89632714395
0.00114975699539774 3639.18043786411
0.00126185688306602 3270.30649319173
0.00138488637139387 2938.25142224092
0.00151991108295293 2639.38462427475
0.00166810053720006 2370.42998707287
0.00183073828029537 2128.43155626994
0.00200923300256505 1910.72250673345
0.00220513073990305 1714.8971002325
0.00242012826479438 1538.78534368935
0.00265608778294669 1380.43008950061
0.00291505306282518 1238.06634403335
0.00319926713779738 1110.10257268512
0.00351119173421513 995.103810067919
0.00385352859371053 891.776402133333
0.0042292428743895 798.954223579675
0.00464158883361278 715.586228837316
0.00509413801481638 640.72520846231
0.00559081018251223 573.517635016331
0.00613590727341317 513.194493594204
0.00673415065775082 459.063002189116
0.00739072203352578 410.499136160071
0.00811130830789687 366.940879276465
0.00890215085445039 327.882131242879
0.00977009957299226 292.867208327449
0.0107226722201032 261.485879796583
0.01176811952435 233.368888358261
0.0129154966501488 208.183907790674
0.0141747416292681 185.631895432429
0.0155567614393047 165.443801280123
0.0170735264747069 147.377599119601
0.0187381742286039 131.215608445705
0.0205651230834865 116.762078935315
0.0225701971963392 103.841011960201
0.0247707635599171 92.2941960872454
0.0271858824273294 81.9794357386748
0.0298364724028334 72.7689541966385
0.0327454916287773 64.547953955108
0.0359381366380463 57.2133190660633
0.0394420605943766 50.6724456129939
0.0432876128108306 44.8421877879096
0.047508101621028 39.6479082620407
0.0521400828799969 35.0226226375145
0.0572236765935022 30.9062287587497
0.0628029144183425 27.2448125581954
0.068926121043497 23.9900229205306
0.075646332755463 21.0985087808125
0.0830217568131974 18.5314123327906
0.0911162756115489 16.2539128204989
0.1 14.234815925393
};
\label{\figlabel-line3}
\addplot [ultra thick, blue]
table {%
0.000132059976775 5247399.0816256
0.000134597669583 1138501.980904
0.000266457400041 569250.9904515
0.000368612154141 254525.734935
0.000432004928221 127262.8674675
0.000653437109663 63631.4337338
0.00109392185408 31815.71686694
0.00228054100492 15907.85843341
0.00239813854189 15360
0.0025720983848 7680
0.00329499256326 3840
0.00405857663354 2156.04858014
0.00456852438756 1920
0.00461446459391 1078.02429007
0.0056613128853 539.012145036
0.00910236800804 269.506072518
0.01378949183 134.7530362588
0.0251277204434 67.3765181294
0.0372614771795 33.6882590646
0.0544651671895 12
};
\label{\figlabel-line0}
\addplot [ultra thick, green!50.0!black]
table {%
1.44639587272e-05 381616.272973868
3.78640122421e-05 167068.489783086
7.69552596262e-05 78409.8317259242
0.000181755811891 34080.502697318
0.000393103387445 15928.4330079298
0.000877573845432 6852.398163241
0.00140973705312 3177.6387935986
0.00329158299043 1340.2591087768
0.00451594669534 610.1295543882
0.00905869618308 245.064777194
0.0132448727133 105.6882590646
0.0533682590909 36
0.0544651671895 12
};
\label{\figlabel-line2}
\addplot [ultra thick, red]
table {%
1.3034378468e-05 383517.986738826
1.39050009416e-05 369609.163757975
1.48798437703e-05 335767.317465405
2.0074898237175e-05 267161.59028684
2.01188108224714e-05 245317.665178866
2.03324265740333e-05 309531.74886987
2.08081775932e-05 227937.783860046
2.54827506018e-05 219745.783860046
2.65519824786e-05 213297.747916658
2.72823539789e-05 206343.33642626
2.91569805826e-05 196547.462867188
3.701686162619e-05 152942.219742479
4.65300789151e-05 145851.211738779
4.69296625133e-05 135816.902038799
4.706852640005e-05 129790.125744773
4.7501381257825e-05 112688.763400133
5.088955632824e-05 101270.727143231
5.31329975684e-05 98046.7091715261
7.21409464253e-05 89854.7091715261
7.35018960717e-05 86377.5034263281
7.48626282901e-05 80226.7666864981
7.53449840032e-05 73763.3843938862
9.9333451235125e-05 65183.3788930986
9.99690003499e-05 58052.1552990853
0.000107236989272 56440.1463132343
0.000108754000751 54701.5434406319
0.0001126193183345 48864.6605372219
0.00014446327997 46816.6605372219
0.000196583722616 40318.7553358594
0.000197616639876333 36763.2998544486
0.0001989819586762 31104.3584289597
0.000199789998581 28595.7810039707
0.000212397094372 26823.0290030457
0.000219984164598 26017.024510123
0.000221974003399 25147.72307382
0.000227750879287 22229.2816221078
0.000230948709816 24401.7667869538
0.000235090033656 21205.2816221078
0.000327992842683 19157.2816221078
0.000329831117772333 16682.7716694665
0.00034922632046225 13951.6621560394
0.000368473400331 13065.2861555774
0.0003845962244095 11606.0654297174
0.000441431160307 11094.0654297174
0.0005753077757775 8350.0045625176
0.000646608734871 7279.6722060396
0.000653394667458 7906.8165622876
0.000660508158657 6736.5509148267
0.00071630841432 5928.8901096585
0.000724210191333 5400.3729236585
0.000726201391158 4623.4324289046
0.00101875768125 4111.4324289046
0.00102049864232 3797.8602507789
0.00138715636771 3304.7056050583
0.00139885001707 3576.2662506638
0.00148806368052 3176.7056050583
0.00149595611671667 2477.7220075841
0.00166427833724 2058.5596069494
0.00246312064836 1802.5596069494
0.00248308743525 1691.6317290194
0.00249925771548 1414.0893595726
0.00269048596292 1010.7259476494
0.00269649508992 1146.5062704521
0.00280521551337 946.7259476494
0.00282000929781 861.1535034983
0.003774509187635 727.3619589384
0.00463375049753 695.3619589384
0.00491080915654 594.61139732298
0.00577584791158 466.61139732298
0.005798797633845 305.47670006348
0.00886445468886 241.47670006348
0.013669101956 209.47670006348
0.024027090706 33.84962500724
0.0243881186303 49.84962500724
0.0245109480602 150.69047798802
0.02535514762155 83.79470570787
};
\label{\figlabel-line4}
\coordinate (legend) at (axis description cs:0.03,0.03);
\end{axis}

\matrix [matrix of nodes,
inner sep=1pt, row sep=1pt,cells={anchor=west},anchor={south west},at={(0.03,0.03)}, anchor=south west, draw=none, fill=none] at (legend) {
\ref{\figlabel-line0} {SL}\\
\ref{\figlabel-line1} {$\epsilon^{-\frac{ 5 }{ 3 }}\log(\epsilon^{-1})$}\\
\ref{\figlabel-line2} {ML}\\
\ref{\figlabel-line3} {$\epsilon^{-1}\log(\epsilon^{-1})$}\\
\ref{\figlabel-line4} {Adaptive ML}\\
};
\end{tikzpicture}
      \caption{$d=6$}
	\end{subfigure}
	\caption{$L^2([-1,1]^d)$-error, approximated using
      \Cref{eq:l2-mc-error} vs work estimate \Cref{eq:workdef} of
      single-level (SL), multilevel (ML) and adaptive ML (ML adaptive)
      methods for a linear elliptic PDE with non-smooth parameter
      dependence. The grey dotted lines are the complexity curves of
      different runs of the single-level, each with a different PDE
      discretization level. The single-level (SL) complexity curve is
      then the lower envelope of all single-level complexity
      curves. This figure shows the agreement of the numerical results
      with the theoretical rates.}
	\label{fig:kink-work}
  \end{figure}


  \begin{figure}
	\centering
    \begin{subfigure}{0.49\textwidth}
      \renewcommand{\figlabel}{fig:total-time-vs-error-d2}
      % This file was created by matplotlib2tikz v0.6.10.
\begin{tikzpicture}

\begin{axis}[
xlabel={Max Error},
ylabel={Time [s.]},
xmin=1e-05, xmax=0.1,
ymin=0.001, ymax=10000,
xmode=log,
ymode=log,
axis on top,
name=\figlabel,
width=\figurewidth,
height=\figureheight,
xtick={1e-06,1e-05,0.0001,0.001,0.01,0.1,1,10},
xticklabels={,$10^{-5}$,$10^{-4}$,$10^{-3}$,$10^{-2}$,$10^{-1}$,,},
ytick={0.0001,0.001,0.01,0.1,1,10,100,1000,10000,100000},
yticklabels={,$10^{-3}$,$10^{-2}$,$10^{-1}$,$10^{0}$,$10^{1}$,$10^{2}$,$10^{3}$,$10^{4}$,},
tick pos=both
]
\addplot [thick, black, opacity=0.4, dotted, mark=x, mark size=2, mark options={solid,fill opacity=0}, forget plot]
table {%
0.028850722164 0.0717530000000004
0.0295967561139 0.0129580000000005
0.0564662247388 0.00757999999999992
};
\addplot [thick, black, opacity=0.4, dotted, mark=x, mark size=2, mark options={solid,fill opacity=0}, forget plot]
table {%
0.0192019313784 1.015519
0.0192637300731 0.0846990000000001
0.0208200459224 0.0201250000000006
0.0533431109427 0.00798600000000005
};
\addplot [thick, black, opacity=0.4, dotted, mark=x, mark size=2, mark options={solid,fill opacity=0}, forget plot]
table {%
0.0100960086376 108.389575
0.0101660326209 1.187
0.0102601901521 0.12245
0.0136774813885 0.0226859999999998
0.0518649941625 0.00599800000000017
};
\addplot [thick, black, opacity=0.4, dotted, mark=x, mark size=2, mark options={solid,fill opacity=0}, forget plot]
table {%
0.0061981501287 111.328325
0.00628025704927 1.853047
0.00641119183914 0.170275
0.0114773994563 0.0342469999999999
0.0516988407492 0.0126769999999996
};
\addplot [thick, black, opacity=0.4, dotted, mark=x, mark size=2, mark options={solid,fill opacity=0}, forget plot]
table {%
0.00300619987375 197.410262
0.00311908110281 3.240387
0.00333851386423 0.305088
0.0104501445112 0.0420390000000002
0.0517785643764 0.0175519999999998
};
\addplot [thick, black, opacity=0.4, dotted, mark=x, mark size=2, mark options={solid,fill opacity=0}, forget plot]
table {%
0.00176426575202 215.522559
0.00191623362252 4.546133
0.0022329793161 0.462441
0.0102950449777 0.0629820000000001
0.0518619939716 0.0178560000000001
};
\addplot [thick, black, opacity=0.4, dotted, mark=x, mark size=2, mark options={solid,fill opacity=0}, forget plot]
table {%
0.000814952629474 284.020738
0.00105902604125 11.989475
0.00153439349992 1.27499
0.0102773465982 0.208521
0.0519454543224 0.0416829999999999
};
\addplot [thick, black, opacity=0.4, dotted, mark=x, mark size=2, mark options={solid,fill opacity=0}, forget plot]
table {%
0.000437595506412 426.204314
0.000779824570117 26.455463
0.00134457874406 2.938779
0.010294830759 0.464164
0.0519833560028 0.155849
};
\addplot [thick, black, opacity=0.4, dotted, mark=x, mark size=2, mark options={solid,fill opacity=0}, forget plot]
table {%
0.000206737057054 1132.274854
0.000659063252098 90.08939
0.00127031301023 9.029039
0.0103123898405 1.183068
0.0520078815272 0.331731
};
\addplot [thick, black, opacity=0.4, dotted, mark=x, mark size=2, mark options={solid,fill opacity=0}, forget plot]
table {%
0.00010382158587 3142.399468
0.000625460792001 213.572087
0.00124953797103 22.288365
0.01032191961 2.977998
0.052019174254 0.758581
};
\addplot [thick, black, dash pattern=on 1pt off 3pt on 3pt off 3pt]
table {%
1e-05 2406598.96952747
1.09749876549306e-05 1921338.67711945
1.20450354025878e-05 1533823.19182859
1.32194114846603e-05 1224383.23797227
1.45082877849594e-05 977303.971845199
1.59228279334109e-05 780030.646077453
1.74752840000768e-05 622533.703742001
1.91791026167249e-05 496801.328026675
2.10490414451202e-05 396433.881134684
2.31012970008316e-05 316319.790638917
2.53536449397011e-05 252376.539620611
2.78255940220713e-05 201343.694036271
3.05385550883341e-05 160617.521240128
3.35160265093884e-05 128118.848920966
3.67837977182863e-05 102187.489060944
4.03701725859655e-05 81497.8910113284
4.43062145758388e-05 64991.7586891578
4.86260158006535e-05 51824.2230431855
5.33669923120631e-05 41320.8453554799
5.85702081805666e-05 32943.2740564871
6.42807311728432e-05 26261.8150617143
7.05480231071865e-05 20933.5251990324
7.74263682681127e-05 16684.7176881857
8.49753435908644e-05 13296.9919325413
9.3260334688322e-05 10596.0783406417
0.000102353102189903 8442.93150911752
0.000112332403297803 6726.61906238674
0.000123284673944207 5358.64450917561
0.000135304777457981 4268.41523866852
0.000148496826225446 3399.62491514806
0.000162975083462064 2707.36597706256
0.000178864952905744 2155.82505271971
0.000196304065004027 1716.44374709966
0.000215443469003188 1366.45093264749
0.000236448941264541 1087.69159017695
0.000259502421139973 865.692352250223
0.00028480358684358 688.915966192035
0.000312571584968824 548.166529022746
0.000343046928631492 436.115040880359
0.000376493580679247 346.920967515424
0.000413201240011534 275.930408228267
0.000453487850812858 219.435382469175
0.000497702356433211 174.481875389191
0.000546227721768434 138.716778996568
0.000599484250318941 110.265858313298
0.000657933224657568 87.6364625324969
0.000722080901838546 69.6399707123585
0.000792482898353917 55.3299747437565
0.000869749002617784 43.9530108947778
0.000954548456661834 34.9092964490731
0.00104761575278967 27.721442779213
0.00114975699539774 22.0095269500272
0.00126185688306602 17.4712316424527
0.00138488637139387 13.8660246003165
0.00151991108295293 11.0025573213533
0.00166810053720006 8.7286290282092
0.00183073828029537 6.92319459641526
0.00200923300256505 5.49000089309973
0.00220513073990305 4.35252032626792
0.00242012826479438 3.44991765584216
0.00265608778294669 2.73383973416691
0.00291505306282518 2.16586058584291
0.00319926713779738 1.71544830689882
0.00351119173421513 1.3583474183444
0.00385352859371053 1.07529195052255
0.0042292428743895 0.850981780204163
0.00464158883361278 0.673268483563903
0.00509413801481638 0.53250791600166
0.00559081018251223 0.421045451169121
0.00613590727341317 0.332806758564462
0.00673415065775082 0.262972532072022
0.00739072203352578 0.207719988159774
0.00811130830789687 0.164017461129536
0.00890215085445039 0.129461216387524
0.00977009957299226 0.102145826649196
0.0107226722201032 0.0805612262476732
0.01176811952435 0.063510967658543
0.0129154966501488 0.0500473255875186
0.0141747416292681 0.0394197861263907
0.0155567614393047 0.0310341682740019
0.0170735264747069 0.024420189743834
0.0187381742286039 0.0192057380651279
0.0205651230834865 0.0150964651222539
0.0225701971963392 0.0118596072515296
0.0247707635599171 0.0093111587803203
0.0271858824273294 0.00730570635411625
0.0298364724028334 0.00572837402808977
0.0327454916287773 0.00448844244060772
0.0359381366380463 0.00351429543719209
0.0394420605943766 0.00274941904810059
0.0432876128108306 0.00214923453846778
0.047508101621028 0.00167859236728383
0.0521400828799969 0.00130978971252197
0.0572236765935022 0.00102100265456328
0.0628029144183425 0.000795046677334134
0.068926121043497 0.000618397053474035
0.075646332755463 0.000480414885877113
0.0830217568131974 0.000372735845589451
0.0911162756115489 0.000288787581126161
0.1 0.000223408858078854
};
\label{\figlabel-line1}
\addplot [thick, black, dashed]
table {%
1e-05 7264.78478695498
1.09749876549306e-05 6313.98121875807
1.20450354025878e-05 5486.88894378568
1.32194114846603e-05 4767.49729428287
1.45082877849594e-05 4141.85763083549
1.59228279334109e-05 3597.81895263679
1.74752840000768e-05 3124.79726899365
1.91791026167249e-05 2713.57443713104
2.10490414451202e-05 2356.12271583912
2.31012970008316e-05 2045.4517601894
2.53536449397011e-05 1775.47519809111
2.78255940220713e-05 1540.89429244757
3.05385550883341e-05 1337.0965097312
3.35160265093884e-05 1160.06709271839
3.67837977182863e-05 1006.31197697758
4.03701725859655e-05 872.790601912026
4.43062145758388e-05 756.857351594127
4.86260158006535e-05 656.210521673294
5.33669923120631e-05 568.847849250989
5.85702081805666e-05 493.027765378793
6.42807311728432e-05 427.235637006573
7.05480231071865e-05 370.154358761464
7.74263682681127e-05 320.638736598729
8.49753435908644e-05 277.693176639614
9.3260334688322e-05 240.452254715402
0.000102353102189903 208.163796421057
0.000112332403297803 180.174144851154
0.000123284673944207 155.915334522645
0.000135304777457981 134.893926050297
0.000148496826225446 116.681287600892
0.000162975083462064 100.905136596407
0.000178864952905744 87.2421790750957
0.000196304065004027 75.4117049988015
0.000215443469003188 65.1700160043223
0.000236448941264541 56.3055779763491
0.000259502421139973 48.6348046663053
0.00028480358684358 41.9983906545351
0.000312571584968824 36.2581224789987
0.000343046928631492 31.294105929456
0.000376493580679247 27.0023555044587
0.000413201240011534 23.2926989997873
0.000453487850812858 20.0869562725815
0.000497702356433211 17.3173565199053
0.000546227721768434 14.9251630238539
0.000599484250318941 12.8594783347844
0.000657933224657568 11.076206365945
0.000722080901838546 9.5371509230975
0.000792482898353917 8.20923284956955
0.000869749002617784 7.06381028108468
0.000954548456661834 6.07608851972727
0.00104761575278967 5.22460779095448
0.00114975699539774 4.4907986751575
0.00126185688306602 3.85859633512129
0.00138488637139387 3.3141058183041
0.00151991108295293 2.84531172036009
0.00166810053720006 2.44182637311673
0.00183073828029537 2.09467148316381
0.00200923300256505 1.7960888110171
0.00220513073990305 1.53937605829974
0.00242012826479438 1.31874463270582
0.00265608778294669 1.12919639739891
0.00291505306282518 0.966416891432253
0.00319926713779738 0.826682838137407
0.00351119173421513 0.70678204564733
0.00385352859371053 0.603944053397958
0.0042292428743895 0.515780095467815
0.00464158883361278 0.440231140214973
0.00509413801481638 0.375522929552934
0.00559081018251223 0.32012708359111
0.00613590727341317 0.272727460053991
0.00673415065775082 0.232191065324891
0.00739072203352578 0.197542907258246
0.00811130830789687 0.167944260915935
0.00890215085445039 0.142673888715176
0.00977009957299226 0.121111817526605
0.0107226722201032 0.102725328247241
0.01176811952435 0.0870568593522661
0.0129154966501488 0.0737135658217345
0.0141747416292681 0.0623583094435444
0.0155567614393047 0.0527018865077252
0.0170735264747069 0.0444963249339515
0.0187381742286039 0.0375291054397457
0.0205651230834865 0.0316181809177924
0.0225701971963392 0.026607685144065
0.0247707635599171 0.0223642366288739
0.0271858824273294 0.0187737561500543
0.0298364724028334 0.0157387275315083
0.0327454916287773 0.0131758407771668
0.0359381366380463 0.0110139649363512
0.0394420605943766 0.00919240523184468
0.0432876128108306 0.00765940517459433
0.047508101621028 0.00637085974720528
0.0521400828799969 0.00528921037361429
0.0572236765935022 0.00438249640118246
0.0628029144183425 0.0036235412877351
0.068926121043497 0.00298925468250947
0.075646332755463 0.00246003417958829
0.0830217568131974 0.00201925275985916
0.0911162756115489 0.00165281987023173
0.1 0.00134880575782921
};
\label{\figlabel-line3}
\addplot [ultra thick, blue]
table {%
0.00010382158587 3142.399468
0.000206737057054 1132.274854
0.000437595506412 426.204314
0.000625460792001 213.572087
0.000659063252098 90.08939
0.000779824570117 26.455463
0.00105902604125 11.989475
0.00127031301023 9.029039
0.00134457874406 2.938779
0.00153439349992 1.27499
0.0022329793161 0.462441
0.00333851386423 0.305088
0.00641119183914 0.170275
0.0102601901521 0.12245
0.0102950449777 0.0629820000000001
0.0104501445112 0.0420390000000002
0.0114773994563 0.0342469999999999
0.0136774813885 0.0226859999999998
0.0208200459224 0.0201250000000006
0.0295967561139 0.0129580000000005
0.05178191745585 0.00599800000000017
};
\label{\figlabel-line0}
\addplot [ultra thick, green!50.0!black]
table {%
4.38378183888e-05 1119.346201
8.33039967375e-05 252.305227
0.000171950944015 114.749576
0.000660061750766 4.707762
0.000842738934826 2.070529
0.00167966598481 0.694652000000001
0.00241062258745 0.322607000000001
0.00351338479303 0.166611
0.0107445668526 0.0751949999999995
0.0123733619109 0.049795
0.0509587050092 0.0226240000000002
0.0527863355072 0.0125219999999999
0.0564662247388 0.00512100000000015
};
\label{\figlabel-line2}
\addplot [ultra thick, red]
table {%
5.166463392456e-05 826.398530999836
5.20135367082e-05 878.90814299986
5.7706200994e-05 827.170985999704
5.7980304607975e-05 820.869913999804
6.3070808781975e-05 753.698467999869
6.3629830322e-05 588.175911999865
6.38128119674727e-05 432.484100999825
6.42126193414e-05 758.882064999806
6.45657540918667e-05 387.604139999833
6.514230915655e-05 386.593793999816
6.54609337485e-05 401.404986999825
6.58753575629e-05 384.943788999849
6.64041904747e-05 382.971643999849
6.67000993935e-05 383.920847999904
6.712063362655e-05 382.747271999889
6.79076474708e-05 383.487253999859
6.82759345052e-05 383.675616999867
6.894180324015e-05 382.353180999909
6.92918998478e-05 381.12649799993
6.94118091888e-05 382.237210999903
7.0035513574575e-05 380.735857999903
7.05590631134e-05 380.667598999903
7.08431043756e-05 382.125140999907
7.12845008058333e-05 380.182598999924
7.18697233814e-05 381.215477999945
7.19891405109e-05 379.148866999943
7.26023650848e-05 379.390931999978
7.28651290182e-05 378.869885999969
7.346570295132e-05 377.370247000004
7.38800139614333e-05 376.599408999949
7.48168366828e-05 376.547468999958
7.535829793975e-05 376.599718999981
7.65595305171e-05 377.331090999976
7.934722410826e-05 374.775916000011
8.01180722695e-05 375.684652999936
8.05925289719e-05 374.470680999974
8.11911794664e-05 374.308478999968
8.189152365385e-05 373.705968999963
8.30494103435e-05 373.785219999952
8.3664163077475e-05 288.175898999995
8.38409439523692e-05 289.55830099998
9.45226777655e-05 291.194267000022
9.4886613247625e-05 286.922363000006
9.55524797338e-05 287.584774000007
9.62390114112e-05 284.787907999996
9.69429892956e-05 284.18074400002
9.770873976985e-05 277.256797000014
9.89797120225e-05 279.904126000006
0.000100280593018 277.338860000005
0.000101609151162 276.329955000006
0.000102602044138 276.524563
0.0001042739187785 260.416412000025
0.000105180947027 256.782162000023
0.000106800784959833 244.911297000014
0.0001068196360515 250.471306000019
0.000107892517542 323.791522000022
0.000108445123697 334.727740000021
0.000108923735235 328.013089000005
0.000116201299295 317.233230000032
0.00011718427798 292.276740000022
0.000118660442074 245.308726000008
0.000120080097512 225.771907000009
0.0001210951463305 226.378756000007
0.000121814529012 226.084565000004
0.000122153364157 227.858690999999
0.00013177976245613 149.506512000008
0.000132260211363 320.09499500001
0.000137386962308 155.152416000004
0.000138286429285 146.549664999996
0.000138981426191 148.192632999994
0.0001411174197545 143.219898000005
0.000141416285912 150.109073999987
0.000142950775949 119.467531
0.000143179899814333 135.09347500001
0.000147086520859333 126.858175000008
0.000150913561057 119.500042000001
0.000153682507651 92.6062800000011
0.000161356526235929 73.1216200000047
0.000161854015827 71.6499030000034
0.000162846205595833 71.5008670000036
0.000163483286634 75.2484290000068
0.00016588218708 90.3982730000055
0.000168702923257 87.7668630000051
0.000169870306769 82.9553930000014
0.000171040267601 89.4829820000034
0.000171988023008 75.7033550000035
0.000173328237249 75.0868370000028
0.000190654625461 70.6777530000016
0.000192903627601 73.6512320000041
0.000210698079811429 66.733198000002
0.000211145831562 74.6587880000032
0.0002157782068618 63.2033020000024
0.000216288453707 52.942345000002
0.00022262982363975 46.4967970000018
0.000223019972331333 42.0522770000023
0.00022647550308425 35.0239979999999
0.000231244310738 32.1493600000006
0.000249840669419 31.6651210000009
0.000252717155200533 25.4877569999991
0.000253756249862571 26.2434699999988
0.000256495873754 27.7433419999989
0.000259896649859 25.975036999999
0.000338490328946 24.2035179999997
0.000346592552494333 21.1664079999992
0.000352909739106 20.4827069999991
0.0003542643043605 20.1793229999999
0.000360989619306 19.8305499999997
0.000368701837839 17.3960699999998
0.00037408188956575 16.4671289999996
0.000377758986374 18.4431349999995
0.000416320841375 16.7853329999994
0.00041874780832375 16.9077479999993
0.000422186624728 16.3957599999993
0.0004412406621815 16.0592769999996
0.000448583492883 16.0533549999996
0.000457289197606 15.3812059999994
0.000461717299636 11.3074759999997
0.000463479573614278 11.1560669999997
0.000505691831200555 9.25924499999978
0.0005080297727942 8.51466199999981
0.000517407575865 8.81639199999988
0.000524953307809 8.15773099999977
0.00053291639749475 7.84321600000013
0.000535546653815 7.28106600000004
0.00054133953001 7.20829100000003
0.000541624773118 7.20280400000003
0.000553936667545 6.90091100000008
0.00056297374972 6.28981800000006
0.0005723394315855 6.03530300000003
0.00058063449875 5.52174100000005
0.0005885053909565 5.37156000000002
0.00059960466074 5.24924700000006
0.000608391303682 4.95299400000002
0.0006282243834955 4.77806600000001
0.000637596038142 4.70177199999998
0.000657579434886 4.58105700000001
0.000660368057151 4.51938100000003
0.000662094272679 4.32068600000004
0.000670635922083 4.15837500000007
0.000680752413720667 3.88673000000001
0.00068465449003525 3.89182900000003
0.00068640633154 4.07189100000002
0.000697662462052 3.815286
0.000699097717867 3.74950300000007
0.000709099572131812 3.15652700000002
0.00074009013215 3.04272799999997
0.000757035519925 2.95229099999996
0.000766205202986 2.83692799999998
0.000850270263834 2.74942799999994
0.000867051475432 2.58168499999994
0.000870609530594 2.56189099999991
0.000876805291243 2.49612499999993
0.000885652875215 2.4798919999999
0.000893747462195 2.4372459999999
0.000945793988961 2.36667099999989
0.0009499959849865 2.33570799999989
0.00100439878711571 2.14582799999993
0.00101110823161333 1.98031299999996
0.00101322395762333 1.98765799999994
0.00104314506775 1.90685299999995
0.00105873637318 1.84919399999995
0.00108560765242 1.84169799999994
0.00116815474592 1.85952799999996
0.00118025535796 1.76753299999995
0.00119616523312333 1.63917999999996
0.00121545264835 1.57684799999997
0.00123631120291 1.53011899999996
0.00125651350121 1.52264899999996
0.00127178305999 1.43212699999998
0.00128681520997 1.43459399999997
0.00129831519478 1.35728399999998
0.001337910213155 1.26210699999998
0.00134100728547 1.23644799999998
0.00136116993381 1.12556699999999
0.00150759960148 1.03524699999998
0.0015171688806 1.02876399999999
0.001526964476075 0.963382999999981
0.00165447259295 0.953832999999982
0.0018427350117 0.916516999999984
0.002123076005635 0.868691999999986
0.00212878627448 0.843082999999988
0.00239883444475 0.825886999999995
0.00305150512756 0.79892699999998
0.00307912091924 0.766789999999981
0.00309098586283 0.767190999999989
0.00330844289854 0.740345999999987
0.0033214185031575 0.681819999999995
0.00362578735408 0.652520999999995
0.003640959907864 0.54978999999999
0.00366091114612 0.544129999999994
0.00369424566286 0.643961999999995
0.00437704018641 0.513579999999992
0.004721115804445 0.487772999999992
0.00497970201738 0.446329999999989
0.00507997784354 0.424871999999992
0.00536616648322 0.449137999999993
0.00551412912657 0.133118999999996
0.005540666360772 0.122411
0.0055714440701825 0.112843999999998
0.00709047350622 0.105037999999998
0.0108101508011 0.0985769999999995
0.0191364893705 0.0944070000000001
0.0192290086527 0.0886989999999999
0.0194264722851 0.0806519999999991
0.0195581412143 0.0754479999999995
0.0199436140142 0.0698729999999996
0.0201293567057 0.0619239999999994
0.0205159086314 0.0569559999999993
0.02070218203545 0.0450849999999994
0.0208391967506 0.0278259999999997
0.0210360630119 0.036546
0.0212382346776 0.0225230000000001
0.0218290756358 0.0180039999999999
0.0223567165325 0.0149869999999999
};
\label{\figlabel-line4}
\coordinate (legend) at (axis description cs:0.03,0.03);
\end{axis}

\matrix [matrix of nodes,
inner sep=1pt, row sep=1pt,cells={anchor=west},anchor={south west},at={(0.03,0.03)}, anchor=south west, draw=none, fill=none] at (legend) {
\ref{\figlabel-line0} {SL}\\
\ref{\figlabel-line1} {$\epsilon^{-\frac{ 7 }{ 3 }}\log(\epsilon^{-1})$}\\
\ref{\figlabel-line2} {ML}\\
\ref{\figlabel-line3} {$\epsilon^{-\frac{ 4 }{ 3 }}\log(\epsilon^{-1})^{2}$}\\
\ref{\figlabel-line4} {Adaptive ML}\\
};
\end{tikzpicture}
      \caption{$d=2$}
	\end{subfigure}
	\begin{subfigure}{0.5\textwidth}
      \renewcommand{\figlabel}{fig:total-time-vs-error-d3}
      % This file was created by matplotlib2tikz v0.6.10.
\begin{tikzpicture}

\begin{axis}[
xlabel={Max Error},
ylabel={Time [s.]},
xmin=1e-05, xmax=0.1,
ymin=0.001, ymax=10000,
xmode=log,
ymode=log,
axis on top,
name=\figlabel,
width=\figurewidth,
height=\figureheight,
xtick={1e-06,1e-05,0.0001,0.001,0.01,0.1,1,10},
xticklabels={,$10^{-5}$,$10^{-4}$,$10^{-3}$,$10^{-2}$,$10^{-1}$,,},
ytick={0.0001,0.001,0.01,0.1,1,10,100,1000,10000,100000},
yticklabels={,$10^{-3}$,$10^{-2}$,$10^{-1}$,$10^{0}$,$10^{1}$,$10^{2}$,$10^{3}$,$10^{4}$,},
tick pos=both
]
\addplot [thick, black, opacity=0.4, dotted, mark=x, mark size=2, mark options={solid,fill opacity=0}, forget plot]
table {%
0.028850722164 0.0717530000000004
0.0295967561139 0.0129580000000005
0.0564662247388 0.00757999999999992
};
\addplot [thick, black, opacity=0.4, dotted, mark=x, mark size=2, mark options={solid,fill opacity=0}, forget plot]
table {%
0.0192019313784 1.015519
0.0192637300731 0.0846990000000001
0.0208200459224 0.0201250000000006
0.0533431109427 0.00798600000000005
};
\addplot [thick, black, opacity=0.4, dotted, mark=x, mark size=2, mark options={solid,fill opacity=0}, forget plot]
table {%
0.0100960086376 108.389575
0.0101660326209 1.187
0.0102601901521 0.12245
0.0136774813885 0.0226859999999998
0.0518649941625 0.00599800000000017
};
\addplot [thick, black, opacity=0.4, dotted, mark=x, mark size=2, mark options={solid,fill opacity=0}, forget plot]
table {%
0.0061981501287 111.328325
0.00628025704927 1.853047
0.00641119183914 0.170275
0.0114773994563 0.0342469999999999
0.0516988407492 0.0126769999999996
};
\addplot [thick, black, opacity=0.4, dotted, mark=x, mark size=2, mark options={solid,fill opacity=0}, forget plot]
table {%
0.00300619987375 197.410262
0.00311908110281 3.240387
0.00333851386423 0.305088
0.0104501445112 0.0420390000000002
0.0517785643764 0.0175519999999998
};
\addplot [thick, black, opacity=0.4, dotted, mark=x, mark size=2, mark options={solid,fill opacity=0}, forget plot]
table {%
0.00176426575202 215.522559
0.00191623362252 4.546133
0.0022329793161 0.462441
0.0102950449777 0.0629820000000001
0.0518619939716 0.0178560000000001
};
\addplot [thick, black, opacity=0.4, dotted, mark=x, mark size=2, mark options={solid,fill opacity=0}, forget plot]
table {%
0.000814952629474 284.020738
0.00105902604125 11.989475
0.00153439349992 1.27499
0.0102773465982 0.208521
0.0519454543224 0.0416829999999999
};
\addplot [thick, black, opacity=0.4, dotted, mark=x, mark size=2, mark options={solid,fill opacity=0}, forget plot]
table {%
0.000437595506412 426.204314
0.000779824570117 26.455463
0.00134457874406 2.938779
0.010294830759 0.464164
0.0519833560028 0.155849
};
\addplot [thick, black, opacity=0.4, dotted, mark=x, mark size=2, mark options={solid,fill opacity=0}, forget plot]
table {%
0.000206737057054 1132.274854
0.000659063252098 90.08939
0.00127031301023 9.029039
0.0103123898405 1.183068
0.0520078815272 0.331731
};
\addplot [thick, black, opacity=0.4, dotted, mark=x, mark size=2, mark options={solid,fill opacity=0}, forget plot]
table {%
0.00010382158587 3142.399468
0.000625460792001 213.572087
0.00124953797103 22.288365
0.01032191961 2.977998
0.052019174254 0.758581
};
\addplot [thick, black, dash pattern=on 1pt off 3pt on 3pt off 3pt]
table {%
1e-05 2406598.96952747
1.09749876549306e-05 1921338.67711945
1.20450354025878e-05 1533823.19182859
1.32194114846603e-05 1224383.23797227
1.45082877849594e-05 977303.971845199
1.59228279334109e-05 780030.646077453
1.74752840000768e-05 622533.703742001
1.91791026167249e-05 496801.328026675
2.10490414451202e-05 396433.881134684
2.31012970008316e-05 316319.790638917
2.53536449397011e-05 252376.539620611
2.78255940220713e-05 201343.694036271
3.05385550883341e-05 160617.521240128
3.35160265093884e-05 128118.848920966
3.67837977182863e-05 102187.489060944
4.03701725859655e-05 81497.8910113284
4.43062145758388e-05 64991.7586891578
4.86260158006535e-05 51824.2230431855
5.33669923120631e-05 41320.8453554799
5.85702081805666e-05 32943.2740564871
6.42807311728432e-05 26261.8150617143
7.05480231071865e-05 20933.5251990324
7.74263682681127e-05 16684.7176881857
8.49753435908644e-05 13296.9919325413
9.3260334688322e-05 10596.0783406417
0.000102353102189903 8442.93150911752
0.000112332403297803 6726.61906238674
0.000123284673944207 5358.64450917561
0.000135304777457981 4268.41523866852
0.000148496826225446 3399.62491514806
0.000162975083462064 2707.36597706256
0.000178864952905744 2155.82505271971
0.000196304065004027 1716.44374709966
0.000215443469003188 1366.45093264749
0.000236448941264541 1087.69159017695
0.000259502421139973 865.692352250223
0.00028480358684358 688.915966192035
0.000312571584968824 548.166529022746
0.000343046928631492 436.115040880359
0.000376493580679247 346.920967515424
0.000413201240011534 275.930408228267
0.000453487850812858 219.435382469175
0.000497702356433211 174.481875389191
0.000546227721768434 138.716778996568
0.000599484250318941 110.265858313298
0.000657933224657568 87.6364625324969
0.000722080901838546 69.6399707123585
0.000792482898353917 55.3299747437565
0.000869749002617784 43.9530108947778
0.000954548456661834 34.9092964490731
0.00104761575278967 27.721442779213
0.00114975699539774 22.0095269500272
0.00126185688306602 17.4712316424527
0.00138488637139387 13.8660246003165
0.00151991108295293 11.0025573213533
0.00166810053720006 8.7286290282092
0.00183073828029537 6.92319459641526
0.00200923300256505 5.49000089309973
0.00220513073990305 4.35252032626792
0.00242012826479438 3.44991765584216
0.00265608778294669 2.73383973416691
0.00291505306282518 2.16586058584291
0.00319926713779738 1.71544830689882
0.00351119173421513 1.3583474183444
0.00385352859371053 1.07529195052255
0.0042292428743895 0.850981780204163
0.00464158883361278 0.673268483563903
0.00509413801481638 0.53250791600166
0.00559081018251223 0.421045451169121
0.00613590727341317 0.332806758564462
0.00673415065775082 0.262972532072022
0.00739072203352578 0.207719988159774
0.00811130830789687 0.164017461129536
0.00890215085445039 0.129461216387524
0.00977009957299226 0.102145826649196
0.0107226722201032 0.0805612262476732
0.01176811952435 0.063510967658543
0.0129154966501488 0.0500473255875186
0.0141747416292681 0.0394197861263907
0.0155567614393047 0.0310341682740019
0.0170735264747069 0.024420189743834
0.0187381742286039 0.0192057380651279
0.0205651230834865 0.0150964651222539
0.0225701971963392 0.0118596072515296
0.0247707635599171 0.0093111587803203
0.0271858824273294 0.00730570635411625
0.0298364724028334 0.00572837402808977
0.0327454916287773 0.00448844244060772
0.0359381366380463 0.00351429543719209
0.0394420605943766 0.00274941904810059
0.0432876128108306 0.00214923453846778
0.047508101621028 0.00167859236728383
0.0521400828799969 0.00130978971252197
0.0572236765935022 0.00102100265456328
0.0628029144183425 0.000795046677334134
0.068926121043497 0.000618397053474035
0.075646332755463 0.000480414885877113
0.0830217568131974 0.000372735845589451
0.0911162756115489 0.000288787581126161
0.1 0.000223408858078854
};
\label{\figlabel-line1}
\addplot [thick, black, dashed]
table {%
1e-05 7264.78478695498
1.09749876549306e-05 6313.98121875807
1.20450354025878e-05 5486.88894378568
1.32194114846603e-05 4767.49729428287
1.45082877849594e-05 4141.85763083549
1.59228279334109e-05 3597.81895263679
1.74752840000768e-05 3124.79726899365
1.91791026167249e-05 2713.57443713104
2.10490414451202e-05 2356.12271583912
2.31012970008316e-05 2045.4517601894
2.53536449397011e-05 1775.47519809111
2.78255940220713e-05 1540.89429244757
3.05385550883341e-05 1337.0965097312
3.35160265093884e-05 1160.06709271839
3.67837977182863e-05 1006.31197697758
4.03701725859655e-05 872.790601912026
4.43062145758388e-05 756.857351594127
4.86260158006535e-05 656.210521673294
5.33669923120631e-05 568.847849250989
5.85702081805666e-05 493.027765378793
6.42807311728432e-05 427.235637006573
7.05480231071865e-05 370.154358761464
7.74263682681127e-05 320.638736598729
8.49753435908644e-05 277.693176639614
9.3260334688322e-05 240.452254715402
0.000102353102189903 208.163796421057
0.000112332403297803 180.174144851154
0.000123284673944207 155.915334522645
0.000135304777457981 134.893926050297
0.000148496826225446 116.681287600892
0.000162975083462064 100.905136596407
0.000178864952905744 87.2421790750957
0.000196304065004027 75.4117049988015
0.000215443469003188 65.1700160043223
0.000236448941264541 56.3055779763491
0.000259502421139973 48.6348046663053
0.00028480358684358 41.9983906545351
0.000312571584968824 36.2581224789987
0.000343046928631492 31.294105929456
0.000376493580679247 27.0023555044587
0.000413201240011534 23.2926989997873
0.000453487850812858 20.0869562725815
0.000497702356433211 17.3173565199053
0.000546227721768434 14.9251630238539
0.000599484250318941 12.8594783347844
0.000657933224657568 11.076206365945
0.000722080901838546 9.5371509230975
0.000792482898353917 8.20923284956955
0.000869749002617784 7.06381028108468
0.000954548456661834 6.07608851972727
0.00104761575278967 5.22460779095448
0.00114975699539774 4.4907986751575
0.00126185688306602 3.85859633512129
0.00138488637139387 3.3141058183041
0.00151991108295293 2.84531172036009
0.00166810053720006 2.44182637311673
0.00183073828029537 2.09467148316381
0.00200923300256505 1.7960888110171
0.00220513073990305 1.53937605829974
0.00242012826479438 1.31874463270582
0.00265608778294669 1.12919639739891
0.00291505306282518 0.966416891432253
0.00319926713779738 0.826682838137407
0.00351119173421513 0.70678204564733
0.00385352859371053 0.603944053397958
0.0042292428743895 0.515780095467815
0.00464158883361278 0.440231140214973
0.00509413801481638 0.375522929552934
0.00559081018251223 0.32012708359111
0.00613590727341317 0.272727460053991
0.00673415065775082 0.232191065324891
0.00739072203352578 0.197542907258246
0.00811130830789687 0.167944260915935
0.00890215085445039 0.142673888715176
0.00977009957299226 0.121111817526605
0.0107226722201032 0.102725328247241
0.01176811952435 0.0870568593522661
0.0129154966501488 0.0737135658217345
0.0141747416292681 0.0623583094435444
0.0155567614393047 0.0527018865077252
0.0170735264747069 0.0444963249339515
0.0187381742286039 0.0375291054397457
0.0205651230834865 0.0316181809177924
0.0225701971963392 0.026607685144065
0.0247707635599171 0.0223642366288739
0.0271858824273294 0.0187737561500543
0.0298364724028334 0.0157387275315083
0.0327454916287773 0.0131758407771668
0.0359381366380463 0.0110139649363512
0.0394420605943766 0.00919240523184468
0.0432876128108306 0.00765940517459433
0.047508101621028 0.00637085974720528
0.0521400828799969 0.00528921037361429
0.0572236765935022 0.00438249640118246
0.0628029144183425 0.0036235412877351
0.068926121043497 0.00298925468250947
0.075646332755463 0.00246003417958829
0.0830217568131974 0.00201925275985916
0.0911162756115489 0.00165281987023173
0.1 0.00134880575782921
};
\label{\figlabel-line3}
\addplot [ultra thick, blue]
table {%
0.00010382158587 3142.399468
0.000206737057054 1132.274854
0.000437595506412 426.204314
0.000625460792001 213.572087
0.000659063252098 90.08939
0.000779824570117 26.455463
0.00105902604125 11.989475
0.00127031301023 9.029039
0.00134457874406 2.938779
0.00153439349992 1.27499
0.0022329793161 0.462441
0.00333851386423 0.305088
0.00641119183914 0.170275
0.0102601901521 0.12245
0.0102950449777 0.0629820000000001
0.0104501445112 0.0420390000000002
0.0114773994563 0.0342469999999999
0.0136774813885 0.0226859999999998
0.0208200459224 0.0201250000000006
0.0295967561139 0.0129580000000005
0.05178191745585 0.00599800000000017
};
\label{\figlabel-line0}
\addplot [ultra thick, green!50.0!black]
table {%
4.38378183888e-05 1119.346201
8.33039967375e-05 252.305227
0.000171950944015 114.749576
0.000660061750766 4.707762
0.000842738934826 2.070529
0.00167966598481 0.694652000000001
0.00241062258745 0.322607000000001
0.00351338479303 0.166611
0.0107445668526 0.0751949999999995
0.0123733619109 0.049795
0.0509587050092 0.0226240000000002
0.0527863355072 0.0125219999999999
0.0564662247388 0.00512100000000015
};
\label{\figlabel-line2}
\addplot [ultra thick, red]
table {%
5.166463392456e-05 826.398530999836
5.20135367082e-05 878.90814299986
5.7706200994e-05 827.170985999704
5.7980304607975e-05 820.869913999804
6.3070808781975e-05 753.698467999869
6.3629830322e-05 588.175911999865
6.38128119674727e-05 432.484100999825
6.42126193414e-05 758.882064999806
6.45657540918667e-05 387.604139999833
6.514230915655e-05 386.593793999816
6.54609337485e-05 401.404986999825
6.58753575629e-05 384.943788999849
6.64041904747e-05 382.971643999849
6.67000993935e-05 383.920847999904
6.712063362655e-05 382.747271999889
6.79076474708e-05 383.487253999859
6.82759345052e-05 383.675616999867
6.894180324015e-05 382.353180999909
6.92918998478e-05 381.12649799993
6.94118091888e-05 382.237210999903
7.0035513574575e-05 380.735857999903
7.05590631134e-05 380.667598999903
7.08431043756e-05 382.125140999907
7.12845008058333e-05 380.182598999924
7.18697233814e-05 381.215477999945
7.19891405109e-05 379.148866999943
7.26023650848e-05 379.390931999978
7.28651290182e-05 378.869885999969
7.346570295132e-05 377.370247000004
7.38800139614333e-05 376.599408999949
7.48168366828e-05 376.547468999958
7.535829793975e-05 376.599718999981
7.65595305171e-05 377.331090999976
7.934722410826e-05 374.775916000011
8.01180722695e-05 375.684652999936
8.05925289719e-05 374.470680999974
8.11911794664e-05 374.308478999968
8.189152365385e-05 373.705968999963
8.30494103435e-05 373.785219999952
8.3664163077475e-05 288.175898999995
8.38409439523692e-05 289.55830099998
9.45226777655e-05 291.194267000022
9.4886613247625e-05 286.922363000006
9.55524797338e-05 287.584774000007
9.62390114112e-05 284.787907999996
9.69429892956e-05 284.18074400002
9.770873976985e-05 277.256797000014
9.89797120225e-05 279.904126000006
0.000100280593018 277.338860000005
0.000101609151162 276.329955000006
0.000102602044138 276.524563
0.0001042739187785 260.416412000025
0.000105180947027 256.782162000023
0.000106800784959833 244.911297000014
0.0001068196360515 250.471306000019
0.000107892517542 323.791522000022
0.000108445123697 334.727740000021
0.000108923735235 328.013089000005
0.000116201299295 317.233230000032
0.00011718427798 292.276740000022
0.000118660442074 245.308726000008
0.000120080097512 225.771907000009
0.0001210951463305 226.378756000007
0.000121814529012 226.084565000004
0.000122153364157 227.858690999999
0.00013177976245613 149.506512000008
0.000132260211363 320.09499500001
0.000137386962308 155.152416000004
0.000138286429285 146.549664999996
0.000138981426191 148.192632999994
0.0001411174197545 143.219898000005
0.000141416285912 150.109073999987
0.000142950775949 119.467531
0.000143179899814333 135.09347500001
0.000147086520859333 126.858175000008
0.000150913561057 119.500042000001
0.000153682507651 92.6062800000011
0.000161356526235929 73.1216200000047
0.000161854015827 71.6499030000034
0.000162846205595833 71.5008670000036
0.000163483286634 75.2484290000068
0.00016588218708 90.3982730000055
0.000168702923257 87.7668630000051
0.000169870306769 82.9553930000014
0.000171040267601 89.4829820000034
0.000171988023008 75.7033550000035
0.000173328237249 75.0868370000028
0.000190654625461 70.6777530000016
0.000192903627601 73.6512320000041
0.000210698079811429 66.733198000002
0.000211145831562 74.6587880000032
0.0002157782068618 63.2033020000024
0.000216288453707 52.942345000002
0.00022262982363975 46.4967970000018
0.000223019972331333 42.0522770000023
0.00022647550308425 35.0239979999999
0.000231244310738 32.1493600000006
0.000249840669419 31.6651210000009
0.000252717155200533 25.4877569999991
0.000253756249862571 26.2434699999988
0.000256495873754 27.7433419999989
0.000259896649859 25.975036999999
0.000338490328946 24.2035179999997
0.000346592552494333 21.1664079999992
0.000352909739106 20.4827069999991
0.0003542643043605 20.1793229999999
0.000360989619306 19.8305499999997
0.000368701837839 17.3960699999998
0.00037408188956575 16.4671289999996
0.000377758986374 18.4431349999995
0.000416320841375 16.7853329999994
0.00041874780832375 16.9077479999993
0.000422186624728 16.3957599999993
0.0004412406621815 16.0592769999996
0.000448583492883 16.0533549999996
0.000457289197606 15.3812059999994
0.000461717299636 11.3074759999997
0.000463479573614278 11.1560669999997
0.000505691831200555 9.25924499999978
0.0005080297727942 8.51466199999981
0.000517407575865 8.81639199999988
0.000524953307809 8.15773099999977
0.00053291639749475 7.84321600000013
0.000535546653815 7.28106600000004
0.00054133953001 7.20829100000003
0.000541624773118 7.20280400000003
0.000553936667545 6.90091100000008
0.00056297374972 6.28981800000006
0.0005723394315855 6.03530300000003
0.00058063449875 5.52174100000005
0.0005885053909565 5.37156000000002
0.00059960466074 5.24924700000006
0.000608391303682 4.95299400000002
0.0006282243834955 4.77806600000001
0.000637596038142 4.70177199999998
0.000657579434886 4.58105700000001
0.000660368057151 4.51938100000003
0.000662094272679 4.32068600000004
0.000670635922083 4.15837500000007
0.000680752413720667 3.88673000000001
0.00068465449003525 3.89182900000003
0.00068640633154 4.07189100000002
0.000697662462052 3.815286
0.000699097717867 3.74950300000007
0.000709099572131812 3.15652700000002
0.00074009013215 3.04272799999997
0.000757035519925 2.95229099999996
0.000766205202986 2.83692799999998
0.000850270263834 2.74942799999994
0.000867051475432 2.58168499999994
0.000870609530594 2.56189099999991
0.000876805291243 2.49612499999993
0.000885652875215 2.4798919999999
0.000893747462195 2.4372459999999
0.000945793988961 2.36667099999989
0.0009499959849865 2.33570799999989
0.00100439878711571 2.14582799999993
0.00101110823161333 1.98031299999996
0.00101322395762333 1.98765799999994
0.00104314506775 1.90685299999995
0.00105873637318 1.84919399999995
0.00108560765242 1.84169799999994
0.00116815474592 1.85952799999996
0.00118025535796 1.76753299999995
0.00119616523312333 1.63917999999996
0.00121545264835 1.57684799999997
0.00123631120291 1.53011899999996
0.00125651350121 1.52264899999996
0.00127178305999 1.43212699999998
0.00128681520997 1.43459399999997
0.00129831519478 1.35728399999998
0.001337910213155 1.26210699999998
0.00134100728547 1.23644799999998
0.00136116993381 1.12556699999999
0.00150759960148 1.03524699999998
0.0015171688806 1.02876399999999
0.001526964476075 0.963382999999981
0.00165447259295 0.953832999999982
0.0018427350117 0.916516999999984
0.002123076005635 0.868691999999986
0.00212878627448 0.843082999999988
0.00239883444475 0.825886999999995
0.00305150512756 0.79892699999998
0.00307912091924 0.766789999999981
0.00309098586283 0.767190999999989
0.00330844289854 0.740345999999987
0.0033214185031575 0.681819999999995
0.00362578735408 0.652520999999995
0.003640959907864 0.54978999999999
0.00366091114612 0.544129999999994
0.00369424566286 0.643961999999995
0.00437704018641 0.513579999999992
0.004721115804445 0.487772999999992
0.00497970201738 0.446329999999989
0.00507997784354 0.424871999999992
0.00536616648322 0.449137999999993
0.00551412912657 0.133118999999996
0.005540666360772 0.122411
0.0055714440701825 0.112843999999998
0.00709047350622 0.105037999999998
0.0108101508011 0.0985769999999995
0.0191364893705 0.0944070000000001
0.0192290086527 0.0886989999999999
0.0194264722851 0.0806519999999991
0.0195581412143 0.0754479999999995
0.0199436140142 0.0698729999999996
0.0201293567057 0.0619239999999994
0.0205159086314 0.0569559999999993
0.02070218203545 0.0450849999999994
0.0208391967506 0.0278259999999997
0.0210360630119 0.036546
0.0212382346776 0.0225230000000001
0.0218290756358 0.0180039999999999
0.0223567165325 0.0149869999999999
};
\label{\figlabel-line4}
\coordinate (legend) at (axis description cs:0.03,0.03);
\end{axis}

\matrix [matrix of nodes,
inner sep=1pt, row sep=1pt,cells={anchor=west},anchor={south west},at={(0.03,0.03)}, anchor=south west, draw=none, fill=none] at (legend) {
\ref{\figlabel-line0} {SL}\\
\ref{\figlabel-line1} {$\epsilon^{-\frac{ 7 }{ 3 }}\log(\epsilon^{-1})$}\\
\ref{\figlabel-line2} {ML}\\
\ref{\figlabel-line3} {$\epsilon^{-\frac{ 4 }{ 3 }}\log(\epsilon^{-1})^{2}$}\\
\ref{\figlabel-line4} {Adaptive ML}\\
};
\end{tikzpicture}
      \caption{$d=3$}
	\end{subfigure}
	\begin{subfigure}{0.49\textwidth}
      \renewcommand{\figlabel}{fig:total-time-vs-error-d4}
      % This file was created by matplotlib2tikz v0.6.10.
\begin{tikzpicture}

\begin{axis}[
xlabel={Max Error},
ylabel={Time [s.]},
xmin=1e-05, xmax=0.1,
ymin=0.001, ymax=10000,
xmode=log,
ymode=log,
axis on top,
name=\figlabel,
width=\figurewidth,
height=\figureheight,
xtick={1e-06,1e-05,0.0001,0.001,0.01,0.1,1,10},
xticklabels={,$10^{-5}$,$10^{-4}$,$10^{-3}$,$10^{-2}$,$10^{-1}$,,},
ytick={0.0001,0.001,0.01,0.1,1,10,100,1000,10000,100000},
yticklabels={,$10^{-3}$,$10^{-2}$,$10^{-1}$,$10^{0}$,$10^{1}$,$10^{2}$,$10^{3}$,$10^{4}$,},
tick pos=both
]
\addplot [thick, black, opacity=0.4, dotted, mark=x, mark size=2, mark options={solid,fill opacity=0}, forget plot]
table {%
0.028850722164 0.0717530000000004
0.0295967561139 0.0129580000000005
0.0564662247388 0.00757999999999992
};
\addplot [thick, black, opacity=0.4, dotted, mark=x, mark size=2, mark options={solid,fill opacity=0}, forget plot]
table {%
0.0192019313784 1.015519
0.0192637300731 0.0846990000000001
0.0208200459224 0.0201250000000006
0.0533431109427 0.00798600000000005
};
\addplot [thick, black, opacity=0.4, dotted, mark=x, mark size=2, mark options={solid,fill opacity=0}, forget plot]
table {%
0.0100960086376 108.389575
0.0101660326209 1.187
0.0102601901521 0.12245
0.0136774813885 0.0226859999999998
0.0518649941625 0.00599800000000017
};
\addplot [thick, black, opacity=0.4, dotted, mark=x, mark size=2, mark options={solid,fill opacity=0}, forget plot]
table {%
0.0061981501287 111.328325
0.00628025704927 1.853047
0.00641119183914 0.170275
0.0114773994563 0.0342469999999999
0.0516988407492 0.0126769999999996
};
\addplot [thick, black, opacity=0.4, dotted, mark=x, mark size=2, mark options={solid,fill opacity=0}, forget plot]
table {%
0.00300619987375 197.410262
0.00311908110281 3.240387
0.00333851386423 0.305088
0.0104501445112 0.0420390000000002
0.0517785643764 0.0175519999999998
};
\addplot [thick, black, opacity=0.4, dotted, mark=x, mark size=2, mark options={solid,fill opacity=0}, forget plot]
table {%
0.00176426575202 215.522559
0.00191623362252 4.546133
0.0022329793161 0.462441
0.0102950449777 0.0629820000000001
0.0518619939716 0.0178560000000001
};
\addplot [thick, black, opacity=0.4, dotted, mark=x, mark size=2, mark options={solid,fill opacity=0}, forget plot]
table {%
0.000814952629474 284.020738
0.00105902604125 11.989475
0.00153439349992 1.27499
0.0102773465982 0.208521
0.0519454543224 0.0416829999999999
};
\addplot [thick, black, opacity=0.4, dotted, mark=x, mark size=2, mark options={solid,fill opacity=0}, forget plot]
table {%
0.000437595506412 426.204314
0.000779824570117 26.455463
0.00134457874406 2.938779
0.010294830759 0.464164
0.0519833560028 0.155849
};
\addplot [thick, black, opacity=0.4, dotted, mark=x, mark size=2, mark options={solid,fill opacity=0}, forget plot]
table {%
0.000206737057054 1132.274854
0.000659063252098 90.08939
0.00127031301023 9.029039
0.0103123898405 1.183068
0.0520078815272 0.331731
};
\addplot [thick, black, opacity=0.4, dotted, mark=x, mark size=2, mark options={solid,fill opacity=0}, forget plot]
table {%
0.00010382158587 3142.399468
0.000625460792001 213.572087
0.00124953797103 22.288365
0.01032191961 2.977998
0.052019174254 0.758581
};
\addplot [thick, black, dash pattern=on 1pt off 3pt on 3pt off 3pt]
table {%
1e-05 2406598.96952747
1.09749876549306e-05 1921338.67711945
1.20450354025878e-05 1533823.19182859
1.32194114846603e-05 1224383.23797227
1.45082877849594e-05 977303.971845199
1.59228279334109e-05 780030.646077453
1.74752840000768e-05 622533.703742001
1.91791026167249e-05 496801.328026675
2.10490414451202e-05 396433.881134684
2.31012970008316e-05 316319.790638917
2.53536449397011e-05 252376.539620611
2.78255940220713e-05 201343.694036271
3.05385550883341e-05 160617.521240128
3.35160265093884e-05 128118.848920966
3.67837977182863e-05 102187.489060944
4.03701725859655e-05 81497.8910113284
4.43062145758388e-05 64991.7586891578
4.86260158006535e-05 51824.2230431855
5.33669923120631e-05 41320.8453554799
5.85702081805666e-05 32943.2740564871
6.42807311728432e-05 26261.8150617143
7.05480231071865e-05 20933.5251990324
7.74263682681127e-05 16684.7176881857
8.49753435908644e-05 13296.9919325413
9.3260334688322e-05 10596.0783406417
0.000102353102189903 8442.93150911752
0.000112332403297803 6726.61906238674
0.000123284673944207 5358.64450917561
0.000135304777457981 4268.41523866852
0.000148496826225446 3399.62491514806
0.000162975083462064 2707.36597706256
0.000178864952905744 2155.82505271971
0.000196304065004027 1716.44374709966
0.000215443469003188 1366.45093264749
0.000236448941264541 1087.69159017695
0.000259502421139973 865.692352250223
0.00028480358684358 688.915966192035
0.000312571584968824 548.166529022746
0.000343046928631492 436.115040880359
0.000376493580679247 346.920967515424
0.000413201240011534 275.930408228267
0.000453487850812858 219.435382469175
0.000497702356433211 174.481875389191
0.000546227721768434 138.716778996568
0.000599484250318941 110.265858313298
0.000657933224657568 87.6364625324969
0.000722080901838546 69.6399707123585
0.000792482898353917 55.3299747437565
0.000869749002617784 43.9530108947778
0.000954548456661834 34.9092964490731
0.00104761575278967 27.721442779213
0.00114975699539774 22.0095269500272
0.00126185688306602 17.4712316424527
0.00138488637139387 13.8660246003165
0.00151991108295293 11.0025573213533
0.00166810053720006 8.7286290282092
0.00183073828029537 6.92319459641526
0.00200923300256505 5.49000089309973
0.00220513073990305 4.35252032626792
0.00242012826479438 3.44991765584216
0.00265608778294669 2.73383973416691
0.00291505306282518 2.16586058584291
0.00319926713779738 1.71544830689882
0.00351119173421513 1.3583474183444
0.00385352859371053 1.07529195052255
0.0042292428743895 0.850981780204163
0.00464158883361278 0.673268483563903
0.00509413801481638 0.53250791600166
0.00559081018251223 0.421045451169121
0.00613590727341317 0.332806758564462
0.00673415065775082 0.262972532072022
0.00739072203352578 0.207719988159774
0.00811130830789687 0.164017461129536
0.00890215085445039 0.129461216387524
0.00977009957299226 0.102145826649196
0.0107226722201032 0.0805612262476732
0.01176811952435 0.063510967658543
0.0129154966501488 0.0500473255875186
0.0141747416292681 0.0394197861263907
0.0155567614393047 0.0310341682740019
0.0170735264747069 0.024420189743834
0.0187381742286039 0.0192057380651279
0.0205651230834865 0.0150964651222539
0.0225701971963392 0.0118596072515296
0.0247707635599171 0.0093111587803203
0.0271858824273294 0.00730570635411625
0.0298364724028334 0.00572837402808977
0.0327454916287773 0.00448844244060772
0.0359381366380463 0.00351429543719209
0.0394420605943766 0.00274941904810059
0.0432876128108306 0.00214923453846778
0.047508101621028 0.00167859236728383
0.0521400828799969 0.00130978971252197
0.0572236765935022 0.00102100265456328
0.0628029144183425 0.000795046677334134
0.068926121043497 0.000618397053474035
0.075646332755463 0.000480414885877113
0.0830217568131974 0.000372735845589451
0.0911162756115489 0.000288787581126161
0.1 0.000223408858078854
};
\label{\figlabel-line1}
\addplot [thick, black, dashed]
table {%
1e-05 7264.78478695498
1.09749876549306e-05 6313.98121875807
1.20450354025878e-05 5486.88894378568
1.32194114846603e-05 4767.49729428287
1.45082877849594e-05 4141.85763083549
1.59228279334109e-05 3597.81895263679
1.74752840000768e-05 3124.79726899365
1.91791026167249e-05 2713.57443713104
2.10490414451202e-05 2356.12271583912
2.31012970008316e-05 2045.4517601894
2.53536449397011e-05 1775.47519809111
2.78255940220713e-05 1540.89429244757
3.05385550883341e-05 1337.0965097312
3.35160265093884e-05 1160.06709271839
3.67837977182863e-05 1006.31197697758
4.03701725859655e-05 872.790601912026
4.43062145758388e-05 756.857351594127
4.86260158006535e-05 656.210521673294
5.33669923120631e-05 568.847849250989
5.85702081805666e-05 493.027765378793
6.42807311728432e-05 427.235637006573
7.05480231071865e-05 370.154358761464
7.74263682681127e-05 320.638736598729
8.49753435908644e-05 277.693176639614
9.3260334688322e-05 240.452254715402
0.000102353102189903 208.163796421057
0.000112332403297803 180.174144851154
0.000123284673944207 155.915334522645
0.000135304777457981 134.893926050297
0.000148496826225446 116.681287600892
0.000162975083462064 100.905136596407
0.000178864952905744 87.2421790750957
0.000196304065004027 75.4117049988015
0.000215443469003188 65.1700160043223
0.000236448941264541 56.3055779763491
0.000259502421139973 48.6348046663053
0.00028480358684358 41.9983906545351
0.000312571584968824 36.2581224789987
0.000343046928631492 31.294105929456
0.000376493580679247 27.0023555044587
0.000413201240011534 23.2926989997873
0.000453487850812858 20.0869562725815
0.000497702356433211 17.3173565199053
0.000546227721768434 14.9251630238539
0.000599484250318941 12.8594783347844
0.000657933224657568 11.076206365945
0.000722080901838546 9.5371509230975
0.000792482898353917 8.20923284956955
0.000869749002617784 7.06381028108468
0.000954548456661834 6.07608851972727
0.00104761575278967 5.22460779095448
0.00114975699539774 4.4907986751575
0.00126185688306602 3.85859633512129
0.00138488637139387 3.3141058183041
0.00151991108295293 2.84531172036009
0.00166810053720006 2.44182637311673
0.00183073828029537 2.09467148316381
0.00200923300256505 1.7960888110171
0.00220513073990305 1.53937605829974
0.00242012826479438 1.31874463270582
0.00265608778294669 1.12919639739891
0.00291505306282518 0.966416891432253
0.00319926713779738 0.826682838137407
0.00351119173421513 0.70678204564733
0.00385352859371053 0.603944053397958
0.0042292428743895 0.515780095467815
0.00464158883361278 0.440231140214973
0.00509413801481638 0.375522929552934
0.00559081018251223 0.32012708359111
0.00613590727341317 0.272727460053991
0.00673415065775082 0.232191065324891
0.00739072203352578 0.197542907258246
0.00811130830789687 0.167944260915935
0.00890215085445039 0.142673888715176
0.00977009957299226 0.121111817526605
0.0107226722201032 0.102725328247241
0.01176811952435 0.0870568593522661
0.0129154966501488 0.0737135658217345
0.0141747416292681 0.0623583094435444
0.0155567614393047 0.0527018865077252
0.0170735264747069 0.0444963249339515
0.0187381742286039 0.0375291054397457
0.0205651230834865 0.0316181809177924
0.0225701971963392 0.026607685144065
0.0247707635599171 0.0223642366288739
0.0271858824273294 0.0187737561500543
0.0298364724028334 0.0157387275315083
0.0327454916287773 0.0131758407771668
0.0359381366380463 0.0110139649363512
0.0394420605943766 0.00919240523184468
0.0432876128108306 0.00765940517459433
0.047508101621028 0.00637085974720528
0.0521400828799969 0.00528921037361429
0.0572236765935022 0.00438249640118246
0.0628029144183425 0.0036235412877351
0.068926121043497 0.00298925468250947
0.075646332755463 0.00246003417958829
0.0830217568131974 0.00201925275985916
0.0911162756115489 0.00165281987023173
0.1 0.00134880575782921
};
\label{\figlabel-line3}
\addplot [ultra thick, blue]
table {%
0.00010382158587 3142.399468
0.000206737057054 1132.274854
0.000437595506412 426.204314
0.000625460792001 213.572087
0.000659063252098 90.08939
0.000779824570117 26.455463
0.00105902604125 11.989475
0.00127031301023 9.029039
0.00134457874406 2.938779
0.00153439349992 1.27499
0.0022329793161 0.462441
0.00333851386423 0.305088
0.00641119183914 0.170275
0.0102601901521 0.12245
0.0102950449777 0.0629820000000001
0.0104501445112 0.0420390000000002
0.0114773994563 0.0342469999999999
0.0136774813885 0.0226859999999998
0.0208200459224 0.0201250000000006
0.0295967561139 0.0129580000000005
0.05178191745585 0.00599800000000017
};
\label{\figlabel-line0}
\addplot [ultra thick, green!50.0!black]
table {%
4.38378183888e-05 1119.346201
8.33039967375e-05 252.305227
0.000171950944015 114.749576
0.000660061750766 4.707762
0.000842738934826 2.070529
0.00167966598481 0.694652000000001
0.00241062258745 0.322607000000001
0.00351338479303 0.166611
0.0107445668526 0.0751949999999995
0.0123733619109 0.049795
0.0509587050092 0.0226240000000002
0.0527863355072 0.0125219999999999
0.0564662247388 0.00512100000000015
};
\label{\figlabel-line2}
\addplot [ultra thick, red]
table {%
5.166463392456e-05 826.398530999836
5.20135367082e-05 878.90814299986
5.7706200994e-05 827.170985999704
5.7980304607975e-05 820.869913999804
6.3070808781975e-05 753.698467999869
6.3629830322e-05 588.175911999865
6.38128119674727e-05 432.484100999825
6.42126193414e-05 758.882064999806
6.45657540918667e-05 387.604139999833
6.514230915655e-05 386.593793999816
6.54609337485e-05 401.404986999825
6.58753575629e-05 384.943788999849
6.64041904747e-05 382.971643999849
6.67000993935e-05 383.920847999904
6.712063362655e-05 382.747271999889
6.79076474708e-05 383.487253999859
6.82759345052e-05 383.675616999867
6.894180324015e-05 382.353180999909
6.92918998478e-05 381.12649799993
6.94118091888e-05 382.237210999903
7.0035513574575e-05 380.735857999903
7.05590631134e-05 380.667598999903
7.08431043756e-05 382.125140999907
7.12845008058333e-05 380.182598999924
7.18697233814e-05 381.215477999945
7.19891405109e-05 379.148866999943
7.26023650848e-05 379.390931999978
7.28651290182e-05 378.869885999969
7.346570295132e-05 377.370247000004
7.38800139614333e-05 376.599408999949
7.48168366828e-05 376.547468999958
7.535829793975e-05 376.599718999981
7.65595305171e-05 377.331090999976
7.934722410826e-05 374.775916000011
8.01180722695e-05 375.684652999936
8.05925289719e-05 374.470680999974
8.11911794664e-05 374.308478999968
8.189152365385e-05 373.705968999963
8.30494103435e-05 373.785219999952
8.3664163077475e-05 288.175898999995
8.38409439523692e-05 289.55830099998
9.45226777655e-05 291.194267000022
9.4886613247625e-05 286.922363000006
9.55524797338e-05 287.584774000007
9.62390114112e-05 284.787907999996
9.69429892956e-05 284.18074400002
9.770873976985e-05 277.256797000014
9.89797120225e-05 279.904126000006
0.000100280593018 277.338860000005
0.000101609151162 276.329955000006
0.000102602044138 276.524563
0.0001042739187785 260.416412000025
0.000105180947027 256.782162000023
0.000106800784959833 244.911297000014
0.0001068196360515 250.471306000019
0.000107892517542 323.791522000022
0.000108445123697 334.727740000021
0.000108923735235 328.013089000005
0.000116201299295 317.233230000032
0.00011718427798 292.276740000022
0.000118660442074 245.308726000008
0.000120080097512 225.771907000009
0.0001210951463305 226.378756000007
0.000121814529012 226.084565000004
0.000122153364157 227.858690999999
0.00013177976245613 149.506512000008
0.000132260211363 320.09499500001
0.000137386962308 155.152416000004
0.000138286429285 146.549664999996
0.000138981426191 148.192632999994
0.0001411174197545 143.219898000005
0.000141416285912 150.109073999987
0.000142950775949 119.467531
0.000143179899814333 135.09347500001
0.000147086520859333 126.858175000008
0.000150913561057 119.500042000001
0.000153682507651 92.6062800000011
0.000161356526235929 73.1216200000047
0.000161854015827 71.6499030000034
0.000162846205595833 71.5008670000036
0.000163483286634 75.2484290000068
0.00016588218708 90.3982730000055
0.000168702923257 87.7668630000051
0.000169870306769 82.9553930000014
0.000171040267601 89.4829820000034
0.000171988023008 75.7033550000035
0.000173328237249 75.0868370000028
0.000190654625461 70.6777530000016
0.000192903627601 73.6512320000041
0.000210698079811429 66.733198000002
0.000211145831562 74.6587880000032
0.0002157782068618 63.2033020000024
0.000216288453707 52.942345000002
0.00022262982363975 46.4967970000018
0.000223019972331333 42.0522770000023
0.00022647550308425 35.0239979999999
0.000231244310738 32.1493600000006
0.000249840669419 31.6651210000009
0.000252717155200533 25.4877569999991
0.000253756249862571 26.2434699999988
0.000256495873754 27.7433419999989
0.000259896649859 25.975036999999
0.000338490328946 24.2035179999997
0.000346592552494333 21.1664079999992
0.000352909739106 20.4827069999991
0.0003542643043605 20.1793229999999
0.000360989619306 19.8305499999997
0.000368701837839 17.3960699999998
0.00037408188956575 16.4671289999996
0.000377758986374 18.4431349999995
0.000416320841375 16.7853329999994
0.00041874780832375 16.9077479999993
0.000422186624728 16.3957599999993
0.0004412406621815 16.0592769999996
0.000448583492883 16.0533549999996
0.000457289197606 15.3812059999994
0.000461717299636 11.3074759999997
0.000463479573614278 11.1560669999997
0.000505691831200555 9.25924499999978
0.0005080297727942 8.51466199999981
0.000517407575865 8.81639199999988
0.000524953307809 8.15773099999977
0.00053291639749475 7.84321600000013
0.000535546653815 7.28106600000004
0.00054133953001 7.20829100000003
0.000541624773118 7.20280400000003
0.000553936667545 6.90091100000008
0.00056297374972 6.28981800000006
0.0005723394315855 6.03530300000003
0.00058063449875 5.52174100000005
0.0005885053909565 5.37156000000002
0.00059960466074 5.24924700000006
0.000608391303682 4.95299400000002
0.0006282243834955 4.77806600000001
0.000637596038142 4.70177199999998
0.000657579434886 4.58105700000001
0.000660368057151 4.51938100000003
0.000662094272679 4.32068600000004
0.000670635922083 4.15837500000007
0.000680752413720667 3.88673000000001
0.00068465449003525 3.89182900000003
0.00068640633154 4.07189100000002
0.000697662462052 3.815286
0.000699097717867 3.74950300000007
0.000709099572131812 3.15652700000002
0.00074009013215 3.04272799999997
0.000757035519925 2.95229099999996
0.000766205202986 2.83692799999998
0.000850270263834 2.74942799999994
0.000867051475432 2.58168499999994
0.000870609530594 2.56189099999991
0.000876805291243 2.49612499999993
0.000885652875215 2.4798919999999
0.000893747462195 2.4372459999999
0.000945793988961 2.36667099999989
0.0009499959849865 2.33570799999989
0.00100439878711571 2.14582799999993
0.00101110823161333 1.98031299999996
0.00101322395762333 1.98765799999994
0.00104314506775 1.90685299999995
0.00105873637318 1.84919399999995
0.00108560765242 1.84169799999994
0.00116815474592 1.85952799999996
0.00118025535796 1.76753299999995
0.00119616523312333 1.63917999999996
0.00121545264835 1.57684799999997
0.00123631120291 1.53011899999996
0.00125651350121 1.52264899999996
0.00127178305999 1.43212699999998
0.00128681520997 1.43459399999997
0.00129831519478 1.35728399999998
0.001337910213155 1.26210699999998
0.00134100728547 1.23644799999998
0.00136116993381 1.12556699999999
0.00150759960148 1.03524699999998
0.0015171688806 1.02876399999999
0.001526964476075 0.963382999999981
0.00165447259295 0.953832999999982
0.0018427350117 0.916516999999984
0.002123076005635 0.868691999999986
0.00212878627448 0.843082999999988
0.00239883444475 0.825886999999995
0.00305150512756 0.79892699999998
0.00307912091924 0.766789999999981
0.00309098586283 0.767190999999989
0.00330844289854 0.740345999999987
0.0033214185031575 0.681819999999995
0.00362578735408 0.652520999999995
0.003640959907864 0.54978999999999
0.00366091114612 0.544129999999994
0.00369424566286 0.643961999999995
0.00437704018641 0.513579999999992
0.004721115804445 0.487772999999992
0.00497970201738 0.446329999999989
0.00507997784354 0.424871999999992
0.00536616648322 0.449137999999993
0.00551412912657 0.133118999999996
0.005540666360772 0.122411
0.0055714440701825 0.112843999999998
0.00709047350622 0.105037999999998
0.0108101508011 0.0985769999999995
0.0191364893705 0.0944070000000001
0.0192290086527 0.0886989999999999
0.0194264722851 0.0806519999999991
0.0195581412143 0.0754479999999995
0.0199436140142 0.0698729999999996
0.0201293567057 0.0619239999999994
0.0205159086314 0.0569559999999993
0.02070218203545 0.0450849999999994
0.0208391967506 0.0278259999999997
0.0210360630119 0.036546
0.0212382346776 0.0225230000000001
0.0218290756358 0.0180039999999999
0.0223567165325 0.0149869999999999
};
\label{\figlabel-line4}
\coordinate (legend) at (axis description cs:0.03,0.03);
\end{axis}

\matrix [matrix of nodes,
inner sep=1pt, row sep=1pt,cells={anchor=west},anchor={south west},at={(0.03,0.03)}, anchor=south west, draw=none, fill=none] at (legend) {
\ref{\figlabel-line0} {SL}\\
\ref{\figlabel-line1} {$\epsilon^{-\frac{ 7 }{ 3 }}\log(\epsilon^{-1})$}\\
\ref{\figlabel-line2} {ML}\\
\ref{\figlabel-line3} {$\epsilon^{-\frac{ 4 }{ 3 }}\log(\epsilon^{-1})^{2}$}\\
\ref{\figlabel-line4} {Adaptive ML}\\
};
\end{tikzpicture}
      \caption{$d=4$}
	\end{subfigure}
	\begin{subfigure}{0.5\textwidth}
      \renewcommand{\figlabel}{fig:total-time-vs-error-d6}
      % This file was created by matplotlib2tikz v0.6.10.
\begin{tikzpicture}

\begin{axis}[
xlabel={Max Error},
ylabel={Time [s.]},
xmin=1e-05, xmax=0.1,
ymin=0.001, ymax=10000,
xmode=log,
ymode=log,
axis on top,
name=\figlabel,
width=\figurewidth,
height=\figureheight,
xtick={1e-06,1e-05,0.0001,0.001,0.01,0.1,1,10},
xticklabels={,$10^{-5}$,$10^{-4}$,$10^{-3}$,$10^{-2}$,$10^{-1}$,,},
ytick={0.0001,0.001,0.01,0.1,1,10,100,1000,10000,100000},
yticklabels={,$10^{-3}$,$10^{-2}$,$10^{-1}$,$10^{0}$,$10^{1}$,$10^{2}$,$10^{3}$,$10^{4}$,},
tick pos=both
]
\addplot [thick, black, opacity=0.4, dotted, mark=x, mark size=2, mark options={solid,fill opacity=0}, forget plot]
table {%
0.028850722164 0.0717530000000004
0.0295967561139 0.0129580000000005
0.0564662247388 0.00757999999999992
};
\addplot [thick, black, opacity=0.4, dotted, mark=x, mark size=2, mark options={solid,fill opacity=0}, forget plot]
table {%
0.0192019313784 1.015519
0.0192637300731 0.0846990000000001
0.0208200459224 0.0201250000000006
0.0533431109427 0.00798600000000005
};
\addplot [thick, black, opacity=0.4, dotted, mark=x, mark size=2, mark options={solid,fill opacity=0}, forget plot]
table {%
0.0100960086376 108.389575
0.0101660326209 1.187
0.0102601901521 0.12245
0.0136774813885 0.0226859999999998
0.0518649941625 0.00599800000000017
};
\addplot [thick, black, opacity=0.4, dotted, mark=x, mark size=2, mark options={solid,fill opacity=0}, forget plot]
table {%
0.0061981501287 111.328325
0.00628025704927 1.853047
0.00641119183914 0.170275
0.0114773994563 0.0342469999999999
0.0516988407492 0.0126769999999996
};
\addplot [thick, black, opacity=0.4, dotted, mark=x, mark size=2, mark options={solid,fill opacity=0}, forget plot]
table {%
0.00300619987375 197.410262
0.00311908110281 3.240387
0.00333851386423 0.305088
0.0104501445112 0.0420390000000002
0.0517785643764 0.0175519999999998
};
\addplot [thick, black, opacity=0.4, dotted, mark=x, mark size=2, mark options={solid,fill opacity=0}, forget plot]
table {%
0.00176426575202 215.522559
0.00191623362252 4.546133
0.0022329793161 0.462441
0.0102950449777 0.0629820000000001
0.0518619939716 0.0178560000000001
};
\addplot [thick, black, opacity=0.4, dotted, mark=x, mark size=2, mark options={solid,fill opacity=0}, forget plot]
table {%
0.000814952629474 284.020738
0.00105902604125 11.989475
0.00153439349992 1.27499
0.0102773465982 0.208521
0.0519454543224 0.0416829999999999
};
\addplot [thick, black, opacity=0.4, dotted, mark=x, mark size=2, mark options={solid,fill opacity=0}, forget plot]
table {%
0.000437595506412 426.204314
0.000779824570117 26.455463
0.00134457874406 2.938779
0.010294830759 0.464164
0.0519833560028 0.155849
};
\addplot [thick, black, opacity=0.4, dotted, mark=x, mark size=2, mark options={solid,fill opacity=0}, forget plot]
table {%
0.000206737057054 1132.274854
0.000659063252098 90.08939
0.00127031301023 9.029039
0.0103123898405 1.183068
0.0520078815272 0.331731
};
\addplot [thick, black, opacity=0.4, dotted, mark=x, mark size=2, mark options={solid,fill opacity=0}, forget plot]
table {%
0.00010382158587 3142.399468
0.000625460792001 213.572087
0.00124953797103 22.288365
0.01032191961 2.977998
0.052019174254 0.758581
};
\addplot [thick, black, dash pattern=on 1pt off 3pt on 3pt off 3pt]
table {%
1e-05 2406598.96952747
1.09749876549306e-05 1921338.67711945
1.20450354025878e-05 1533823.19182859
1.32194114846603e-05 1224383.23797227
1.45082877849594e-05 977303.971845199
1.59228279334109e-05 780030.646077453
1.74752840000768e-05 622533.703742001
1.91791026167249e-05 496801.328026675
2.10490414451202e-05 396433.881134684
2.31012970008316e-05 316319.790638917
2.53536449397011e-05 252376.539620611
2.78255940220713e-05 201343.694036271
3.05385550883341e-05 160617.521240128
3.35160265093884e-05 128118.848920966
3.67837977182863e-05 102187.489060944
4.03701725859655e-05 81497.8910113284
4.43062145758388e-05 64991.7586891578
4.86260158006535e-05 51824.2230431855
5.33669923120631e-05 41320.8453554799
5.85702081805666e-05 32943.2740564871
6.42807311728432e-05 26261.8150617143
7.05480231071865e-05 20933.5251990324
7.74263682681127e-05 16684.7176881857
8.49753435908644e-05 13296.9919325413
9.3260334688322e-05 10596.0783406417
0.000102353102189903 8442.93150911752
0.000112332403297803 6726.61906238674
0.000123284673944207 5358.64450917561
0.000135304777457981 4268.41523866852
0.000148496826225446 3399.62491514806
0.000162975083462064 2707.36597706256
0.000178864952905744 2155.82505271971
0.000196304065004027 1716.44374709966
0.000215443469003188 1366.45093264749
0.000236448941264541 1087.69159017695
0.000259502421139973 865.692352250223
0.00028480358684358 688.915966192035
0.000312571584968824 548.166529022746
0.000343046928631492 436.115040880359
0.000376493580679247 346.920967515424
0.000413201240011534 275.930408228267
0.000453487850812858 219.435382469175
0.000497702356433211 174.481875389191
0.000546227721768434 138.716778996568
0.000599484250318941 110.265858313298
0.000657933224657568 87.6364625324969
0.000722080901838546 69.6399707123585
0.000792482898353917 55.3299747437565
0.000869749002617784 43.9530108947778
0.000954548456661834 34.9092964490731
0.00104761575278967 27.721442779213
0.00114975699539774 22.0095269500272
0.00126185688306602 17.4712316424527
0.00138488637139387 13.8660246003165
0.00151991108295293 11.0025573213533
0.00166810053720006 8.7286290282092
0.00183073828029537 6.92319459641526
0.00200923300256505 5.49000089309973
0.00220513073990305 4.35252032626792
0.00242012826479438 3.44991765584216
0.00265608778294669 2.73383973416691
0.00291505306282518 2.16586058584291
0.00319926713779738 1.71544830689882
0.00351119173421513 1.3583474183444
0.00385352859371053 1.07529195052255
0.0042292428743895 0.850981780204163
0.00464158883361278 0.673268483563903
0.00509413801481638 0.53250791600166
0.00559081018251223 0.421045451169121
0.00613590727341317 0.332806758564462
0.00673415065775082 0.262972532072022
0.00739072203352578 0.207719988159774
0.00811130830789687 0.164017461129536
0.00890215085445039 0.129461216387524
0.00977009957299226 0.102145826649196
0.0107226722201032 0.0805612262476732
0.01176811952435 0.063510967658543
0.0129154966501488 0.0500473255875186
0.0141747416292681 0.0394197861263907
0.0155567614393047 0.0310341682740019
0.0170735264747069 0.024420189743834
0.0187381742286039 0.0192057380651279
0.0205651230834865 0.0150964651222539
0.0225701971963392 0.0118596072515296
0.0247707635599171 0.0093111587803203
0.0271858824273294 0.00730570635411625
0.0298364724028334 0.00572837402808977
0.0327454916287773 0.00448844244060772
0.0359381366380463 0.00351429543719209
0.0394420605943766 0.00274941904810059
0.0432876128108306 0.00214923453846778
0.047508101621028 0.00167859236728383
0.0521400828799969 0.00130978971252197
0.0572236765935022 0.00102100265456328
0.0628029144183425 0.000795046677334134
0.068926121043497 0.000618397053474035
0.075646332755463 0.000480414885877113
0.0830217568131974 0.000372735845589451
0.0911162756115489 0.000288787581126161
0.1 0.000223408858078854
};
\label{\figlabel-line1}
\addplot [thick, black, dashed]
table {%
1e-05 7264.78478695498
1.09749876549306e-05 6313.98121875807
1.20450354025878e-05 5486.88894378568
1.32194114846603e-05 4767.49729428287
1.45082877849594e-05 4141.85763083549
1.59228279334109e-05 3597.81895263679
1.74752840000768e-05 3124.79726899365
1.91791026167249e-05 2713.57443713104
2.10490414451202e-05 2356.12271583912
2.31012970008316e-05 2045.4517601894
2.53536449397011e-05 1775.47519809111
2.78255940220713e-05 1540.89429244757
3.05385550883341e-05 1337.0965097312
3.35160265093884e-05 1160.06709271839
3.67837977182863e-05 1006.31197697758
4.03701725859655e-05 872.790601912026
4.43062145758388e-05 756.857351594127
4.86260158006535e-05 656.210521673294
5.33669923120631e-05 568.847849250989
5.85702081805666e-05 493.027765378793
6.42807311728432e-05 427.235637006573
7.05480231071865e-05 370.154358761464
7.74263682681127e-05 320.638736598729
8.49753435908644e-05 277.693176639614
9.3260334688322e-05 240.452254715402
0.000102353102189903 208.163796421057
0.000112332403297803 180.174144851154
0.000123284673944207 155.915334522645
0.000135304777457981 134.893926050297
0.000148496826225446 116.681287600892
0.000162975083462064 100.905136596407
0.000178864952905744 87.2421790750957
0.000196304065004027 75.4117049988015
0.000215443469003188 65.1700160043223
0.000236448941264541 56.3055779763491
0.000259502421139973 48.6348046663053
0.00028480358684358 41.9983906545351
0.000312571584968824 36.2581224789987
0.000343046928631492 31.294105929456
0.000376493580679247 27.0023555044587
0.000413201240011534 23.2926989997873
0.000453487850812858 20.0869562725815
0.000497702356433211 17.3173565199053
0.000546227721768434 14.9251630238539
0.000599484250318941 12.8594783347844
0.000657933224657568 11.076206365945
0.000722080901838546 9.5371509230975
0.000792482898353917 8.20923284956955
0.000869749002617784 7.06381028108468
0.000954548456661834 6.07608851972727
0.00104761575278967 5.22460779095448
0.00114975699539774 4.4907986751575
0.00126185688306602 3.85859633512129
0.00138488637139387 3.3141058183041
0.00151991108295293 2.84531172036009
0.00166810053720006 2.44182637311673
0.00183073828029537 2.09467148316381
0.00200923300256505 1.7960888110171
0.00220513073990305 1.53937605829974
0.00242012826479438 1.31874463270582
0.00265608778294669 1.12919639739891
0.00291505306282518 0.966416891432253
0.00319926713779738 0.826682838137407
0.00351119173421513 0.70678204564733
0.00385352859371053 0.603944053397958
0.0042292428743895 0.515780095467815
0.00464158883361278 0.440231140214973
0.00509413801481638 0.375522929552934
0.00559081018251223 0.32012708359111
0.00613590727341317 0.272727460053991
0.00673415065775082 0.232191065324891
0.00739072203352578 0.197542907258246
0.00811130830789687 0.167944260915935
0.00890215085445039 0.142673888715176
0.00977009957299226 0.121111817526605
0.0107226722201032 0.102725328247241
0.01176811952435 0.0870568593522661
0.0129154966501488 0.0737135658217345
0.0141747416292681 0.0623583094435444
0.0155567614393047 0.0527018865077252
0.0170735264747069 0.0444963249339515
0.0187381742286039 0.0375291054397457
0.0205651230834865 0.0316181809177924
0.0225701971963392 0.026607685144065
0.0247707635599171 0.0223642366288739
0.0271858824273294 0.0187737561500543
0.0298364724028334 0.0157387275315083
0.0327454916287773 0.0131758407771668
0.0359381366380463 0.0110139649363512
0.0394420605943766 0.00919240523184468
0.0432876128108306 0.00765940517459433
0.047508101621028 0.00637085974720528
0.0521400828799969 0.00528921037361429
0.0572236765935022 0.00438249640118246
0.0628029144183425 0.0036235412877351
0.068926121043497 0.00298925468250947
0.075646332755463 0.00246003417958829
0.0830217568131974 0.00201925275985916
0.0911162756115489 0.00165281987023173
0.1 0.00134880575782921
};
\label{\figlabel-line3}
\addplot [ultra thick, blue]
table {%
0.00010382158587 3142.399468
0.000206737057054 1132.274854
0.000437595506412 426.204314
0.000625460792001 213.572087
0.000659063252098 90.08939
0.000779824570117 26.455463
0.00105902604125 11.989475
0.00127031301023 9.029039
0.00134457874406 2.938779
0.00153439349992 1.27499
0.0022329793161 0.462441
0.00333851386423 0.305088
0.00641119183914 0.170275
0.0102601901521 0.12245
0.0102950449777 0.0629820000000001
0.0104501445112 0.0420390000000002
0.0114773994563 0.0342469999999999
0.0136774813885 0.0226859999999998
0.0208200459224 0.0201250000000006
0.0295967561139 0.0129580000000005
0.05178191745585 0.00599800000000017
};
\label{\figlabel-line0}
\addplot [ultra thick, green!50.0!black]
table {%
4.38378183888e-05 1119.346201
8.33039967375e-05 252.305227
0.000171950944015 114.749576
0.000660061750766 4.707762
0.000842738934826 2.070529
0.00167966598481 0.694652000000001
0.00241062258745 0.322607000000001
0.00351338479303 0.166611
0.0107445668526 0.0751949999999995
0.0123733619109 0.049795
0.0509587050092 0.0226240000000002
0.0527863355072 0.0125219999999999
0.0564662247388 0.00512100000000015
};
\label{\figlabel-line2}
\addplot [ultra thick, red]
table {%
5.166463392456e-05 826.398530999836
5.20135367082e-05 878.90814299986
5.7706200994e-05 827.170985999704
5.7980304607975e-05 820.869913999804
6.3070808781975e-05 753.698467999869
6.3629830322e-05 588.175911999865
6.38128119674727e-05 432.484100999825
6.42126193414e-05 758.882064999806
6.45657540918667e-05 387.604139999833
6.514230915655e-05 386.593793999816
6.54609337485e-05 401.404986999825
6.58753575629e-05 384.943788999849
6.64041904747e-05 382.971643999849
6.67000993935e-05 383.920847999904
6.712063362655e-05 382.747271999889
6.79076474708e-05 383.487253999859
6.82759345052e-05 383.675616999867
6.894180324015e-05 382.353180999909
6.92918998478e-05 381.12649799993
6.94118091888e-05 382.237210999903
7.0035513574575e-05 380.735857999903
7.05590631134e-05 380.667598999903
7.08431043756e-05 382.125140999907
7.12845008058333e-05 380.182598999924
7.18697233814e-05 381.215477999945
7.19891405109e-05 379.148866999943
7.26023650848e-05 379.390931999978
7.28651290182e-05 378.869885999969
7.346570295132e-05 377.370247000004
7.38800139614333e-05 376.599408999949
7.48168366828e-05 376.547468999958
7.535829793975e-05 376.599718999981
7.65595305171e-05 377.331090999976
7.934722410826e-05 374.775916000011
8.01180722695e-05 375.684652999936
8.05925289719e-05 374.470680999974
8.11911794664e-05 374.308478999968
8.189152365385e-05 373.705968999963
8.30494103435e-05 373.785219999952
8.3664163077475e-05 288.175898999995
8.38409439523692e-05 289.55830099998
9.45226777655e-05 291.194267000022
9.4886613247625e-05 286.922363000006
9.55524797338e-05 287.584774000007
9.62390114112e-05 284.787907999996
9.69429892956e-05 284.18074400002
9.770873976985e-05 277.256797000014
9.89797120225e-05 279.904126000006
0.000100280593018 277.338860000005
0.000101609151162 276.329955000006
0.000102602044138 276.524563
0.0001042739187785 260.416412000025
0.000105180947027 256.782162000023
0.000106800784959833 244.911297000014
0.0001068196360515 250.471306000019
0.000107892517542 323.791522000022
0.000108445123697 334.727740000021
0.000108923735235 328.013089000005
0.000116201299295 317.233230000032
0.00011718427798 292.276740000022
0.000118660442074 245.308726000008
0.000120080097512 225.771907000009
0.0001210951463305 226.378756000007
0.000121814529012 226.084565000004
0.000122153364157 227.858690999999
0.00013177976245613 149.506512000008
0.000132260211363 320.09499500001
0.000137386962308 155.152416000004
0.000138286429285 146.549664999996
0.000138981426191 148.192632999994
0.0001411174197545 143.219898000005
0.000141416285912 150.109073999987
0.000142950775949 119.467531
0.000143179899814333 135.09347500001
0.000147086520859333 126.858175000008
0.000150913561057 119.500042000001
0.000153682507651 92.6062800000011
0.000161356526235929 73.1216200000047
0.000161854015827 71.6499030000034
0.000162846205595833 71.5008670000036
0.000163483286634 75.2484290000068
0.00016588218708 90.3982730000055
0.000168702923257 87.7668630000051
0.000169870306769 82.9553930000014
0.000171040267601 89.4829820000034
0.000171988023008 75.7033550000035
0.000173328237249 75.0868370000028
0.000190654625461 70.6777530000016
0.000192903627601 73.6512320000041
0.000210698079811429 66.733198000002
0.000211145831562 74.6587880000032
0.0002157782068618 63.2033020000024
0.000216288453707 52.942345000002
0.00022262982363975 46.4967970000018
0.000223019972331333 42.0522770000023
0.00022647550308425 35.0239979999999
0.000231244310738 32.1493600000006
0.000249840669419 31.6651210000009
0.000252717155200533 25.4877569999991
0.000253756249862571 26.2434699999988
0.000256495873754 27.7433419999989
0.000259896649859 25.975036999999
0.000338490328946 24.2035179999997
0.000346592552494333 21.1664079999992
0.000352909739106 20.4827069999991
0.0003542643043605 20.1793229999999
0.000360989619306 19.8305499999997
0.000368701837839 17.3960699999998
0.00037408188956575 16.4671289999996
0.000377758986374 18.4431349999995
0.000416320841375 16.7853329999994
0.00041874780832375 16.9077479999993
0.000422186624728 16.3957599999993
0.0004412406621815 16.0592769999996
0.000448583492883 16.0533549999996
0.000457289197606 15.3812059999994
0.000461717299636 11.3074759999997
0.000463479573614278 11.1560669999997
0.000505691831200555 9.25924499999978
0.0005080297727942 8.51466199999981
0.000517407575865 8.81639199999988
0.000524953307809 8.15773099999977
0.00053291639749475 7.84321600000013
0.000535546653815 7.28106600000004
0.00054133953001 7.20829100000003
0.000541624773118 7.20280400000003
0.000553936667545 6.90091100000008
0.00056297374972 6.28981800000006
0.0005723394315855 6.03530300000003
0.00058063449875 5.52174100000005
0.0005885053909565 5.37156000000002
0.00059960466074 5.24924700000006
0.000608391303682 4.95299400000002
0.0006282243834955 4.77806600000001
0.000637596038142 4.70177199999998
0.000657579434886 4.58105700000001
0.000660368057151 4.51938100000003
0.000662094272679 4.32068600000004
0.000670635922083 4.15837500000007
0.000680752413720667 3.88673000000001
0.00068465449003525 3.89182900000003
0.00068640633154 4.07189100000002
0.000697662462052 3.815286
0.000699097717867 3.74950300000007
0.000709099572131812 3.15652700000002
0.00074009013215 3.04272799999997
0.000757035519925 2.95229099999996
0.000766205202986 2.83692799999998
0.000850270263834 2.74942799999994
0.000867051475432 2.58168499999994
0.000870609530594 2.56189099999991
0.000876805291243 2.49612499999993
0.000885652875215 2.4798919999999
0.000893747462195 2.4372459999999
0.000945793988961 2.36667099999989
0.0009499959849865 2.33570799999989
0.00100439878711571 2.14582799999993
0.00101110823161333 1.98031299999996
0.00101322395762333 1.98765799999994
0.00104314506775 1.90685299999995
0.00105873637318 1.84919399999995
0.00108560765242 1.84169799999994
0.00116815474592 1.85952799999996
0.00118025535796 1.76753299999995
0.00119616523312333 1.63917999999996
0.00121545264835 1.57684799999997
0.00123631120291 1.53011899999996
0.00125651350121 1.52264899999996
0.00127178305999 1.43212699999998
0.00128681520997 1.43459399999997
0.00129831519478 1.35728399999998
0.001337910213155 1.26210699999998
0.00134100728547 1.23644799999998
0.00136116993381 1.12556699999999
0.00150759960148 1.03524699999998
0.0015171688806 1.02876399999999
0.001526964476075 0.963382999999981
0.00165447259295 0.953832999999982
0.0018427350117 0.916516999999984
0.002123076005635 0.868691999999986
0.00212878627448 0.843082999999988
0.00239883444475 0.825886999999995
0.00305150512756 0.79892699999998
0.00307912091924 0.766789999999981
0.00309098586283 0.767190999999989
0.00330844289854 0.740345999999987
0.0033214185031575 0.681819999999995
0.00362578735408 0.652520999999995
0.003640959907864 0.54978999999999
0.00366091114612 0.544129999999994
0.00369424566286 0.643961999999995
0.00437704018641 0.513579999999992
0.004721115804445 0.487772999999992
0.00497970201738 0.446329999999989
0.00507997784354 0.424871999999992
0.00536616648322 0.449137999999993
0.00551412912657 0.133118999999996
0.005540666360772 0.122411
0.0055714440701825 0.112843999999998
0.00709047350622 0.105037999999998
0.0108101508011 0.0985769999999995
0.0191364893705 0.0944070000000001
0.0192290086527 0.0886989999999999
0.0194264722851 0.0806519999999991
0.0195581412143 0.0754479999999995
0.0199436140142 0.0698729999999996
0.0201293567057 0.0619239999999994
0.0205159086314 0.0569559999999993
0.02070218203545 0.0450849999999994
0.0208391967506 0.0278259999999997
0.0210360630119 0.036546
0.0212382346776 0.0225230000000001
0.0218290756358 0.0180039999999999
0.0223567165325 0.0149869999999999
};
\label{\figlabel-line4}
\coordinate (legend) at (axis description cs:0.03,0.03);
\end{axis}

\matrix [matrix of nodes,
inner sep=1pt, row sep=1pt,cells={anchor=west},anchor={south west},at={(0.03,0.03)}, anchor=south west, draw=none, fill=none] at (legend) {
\ref{\figlabel-line0} {SL}\\
\ref{\figlabel-line1} {$\epsilon^{-\frac{ 7 }{ 3 }}\log(\epsilon^{-1})$}\\
\ref{\figlabel-line2} {ML}\\
\ref{\figlabel-line3} {$\epsilon^{-\frac{ 4 }{ 3 }}\log(\epsilon^{-1})^{2}$}\\
\ref{\figlabel-line4} {Adaptive ML}\\
};
\end{tikzpicture}
      \caption{$d=6$}
	\end{subfigure}
	\caption{Similar to \Cref{fig:kink-work}, but showing the total
      running time of the methods instead of their work estimate. The
      discrepancy of the two figures is due to the overhead of
      sampling the points, assembling the projection matrix and
      computing the projection. Moreover, this plot shows the overhead
      of the adaptive algorithm compared to the non-adaptive one.}
	\label{fig:kink-time}
  \end{figure}


%   \begin{tcolorbox}
% 	Another possible example is:
% 	$a(x,\psmi):=|x-\psmi|^5$, with $\domPS:=U:=[0,1]^2$.
% 	In terms of, we then have
% 	\begin{equation*}
% 	a\in C^5(U\times \domPS)
% 	\end{equation*}
% 	(strictly speaking, the last derivative is not continuous, but lets ignore that).
% 	Using finite elements of order $5$ should thus give us
% 	\begin{align*}
% 	\sc=2\\
% 	\wc=4\\
% 	\alpha=4/d=2\\
% 	\end{align*}
% (the "strong" convergence in \Cref{pro:finite} happens in $C^{4}(\domPS)$ and such functions are approximable by polynomials of degree $k$ at rate $(k+1)^{-4}$. Again, divide by $d=2$ because space of polynomials of degree $k$ has dimension $k^2$

% 	Finally, optimal solvers should have $\gamma=2$.
% 	Since $\gamma/\sc=1>1/\alpha=1/2$ we expect
% 	\begin{align*}
% 	\rate=\theta \gamma/\sc+(1-\theta)1/\alpha=1/2+1/2*1/2=3/4.
% 	\end{align*}

% 	Replacing $5$ by $3$ should give you $\gamma/\sc=1/\alpha=1$.
% 	\end{tcolorbox}
% \subsection{Matern-like example}
% Our second example is the same as the one in
% \cite{Haji-AliNobileTamelliniEtAl2015}. More specifically, we let
% \begin{equation*}
% a_{\psmi}(x)=\exp\left( \sum_{\vec k \in \N^D} A_{\vec k}
% \sum_{\vec \ell \in \{0,1\}^D} \gamma_{\vec k,\vec \ell } \, \prod_{i=1}^D
% \left(\cos\left({\pi }  k_i  x_i \right)\right)^{\ell_i}
% \left(\sin\left({\pi }  k_i  x_i \right)\right)^{1-\ell_i} \right).
% \end{equation*}
% for $\gamma_{\vec k,\vec \ell } \in [-1,1]$ and
% \begin{equation}
% A_{\vec k} {= \left(\sqrt{3}\right)} 2^{\frac{|\vec k|_0}{2}}(1 + |\vec k|^2)^{-\frac{\nu+D/2}{2}},
% \end{equation}
% for some $\nu>0$. The summability of $a_{\psmi}$ is controlled by
% $\nu$. Namely, using the notation of \Cref{pro:UQ} we have that
% $(\|\psi_j\|_{L^{\infty}})_{j\in\N}\in \ell^p(\N)$ with
% $p = \left( \frac{\nu}{D} + \frac{1}{2} \right)^{-1}$ hence $\alpha$
% in Assumption A1($\infty$) is $\frac{\nu}{D} - \frac{1}{2}$. On the
% other hand, the solver we used employs a second order
% finite-difference method with step size $h$ along each dimension,
% which asymptotically converge at the rate $h^{2}$ in the $L^2$ norm
% and require the computational work $h^{-D}$, corresponding to the
% values $\beta=2$ and $\gamma=D$ for the parameters in Assumptions A2
% and A3.

% The quantity of interest in this example is:
% \begin{equation*}
% \QoI(\pde_{\gamma}):= \frac{10}{(\sigma\sqrt{2\pi})^D}
% \int_\mathscr{B} u_\gamma(x) \exp \left( -\frac{\|x-x_o\|_2^2}{2\sigma^2} \right) \;dx.
% \end{equation*}
% with $\sigma=0.2$ and location $x_o=0.3$ for $D=1$ and
% $x_0 =(0.3, 0.2,0.6)$ for $D=3$.

% \begin{tcolorbox}
% 	Consider $D=1$. In terms of \Cref{pro:UQ} we have:
% 	\begin{align*}
% 	r_{\max}&=\nu+1/2\\
% 	\delta&=1.
% 	\end{align*}
% Choosing $r:=1$ in \Cref{pro:UQ} shows that we can take
% 	\begin{align*}
% 	\sc&=1+1=2\\
% 	\alpha&=r_{\max}-1-1-\epsilon=\nu-3/2-\epsilon
% 	\end{align*}
% 	(meaning we can take any $\epsilon>0$. Below let's just assume we can take $\epsilon=0$.)
% 	Choosing $r:=\nu$ shows that we can take
% 	\begin{align*}
% 	\wc=1+\nu.
% 	\end{align*}
% Finally, optimal solver should have
% 	\begin{align*}
% 	\gamma=1.
% 	\end{align*}

% 	Let's for example say $\nu=3.5$. Then $\gamma/\sc=1/\alpha=1/2$ and we expect $\lambda=1/2$.


% %	\textbf{Mean square convergence}
% %If you consider mean square convergence instead, then $\alpha$ improves	by $1/2$ %and you expect (with $\theta=2/4.5=4/9$)
% %\begin{align*}
% %\lambda=\theta \gamma/\sc+(1-\theta)1/\alpha=4/9*1/2+5/9*1/2.5=4/9
% %\end{align*}
% \end{tcolorbox}

% %\begin{figure}[h]
% %	\centering
% %	\input{./figures/kl2/kl2.tex}
% %	\caption{Convergence of nonadaptive and adaptive multilevel algorithm for smooth infinite-dimensional problem.}
% %	\label{fig:kl}
% %\end{figure}

% %\subsection{Smooth case}
% %We let
% %\begin{equation*}
% %a_{\psmi}(x)=\exp\left(\sum_{j=1}^{\infty}\ps_j\psi_j(x)\right)
% %\end{equation*}
% %with
% %\begin{equation*}
% %\psi_j(x_1,x_2):=j^{-4}\phi_{\sigma_j(1)}(x_1)\phi_{\sigma_j(2)}(x_2),
% %\end{equation*}
% %\begin{equation*}
% %\phi_{n}(x):=\begin{cases}
% %	\sin(\frac{n}{2}\pi x)\quad&\text{if }n\text{ is even}\\
% %	\cos(\frac{n-1}{2}\pi x)\quad&\text{else}.
% %\end{cases}
% %\end{equation*}
% %and an enumeration $(\sigma_{j})_{j=1}^{\infty}$ of $\N^2$.
% %
% %
% %Our goal is to approximate the response surface
% %\begin{equation*}
% %\psmi\mapsto \rs(\psmi):=\QoI(\pde_{\psmi}),
% %\end{equation*}
% %with the quantity of interest
% %\begin{equation*}
% %\QoI(\pde):=\int_{[0,1]^2}u\;dx.
% %\end{equation*}
% %Since we are using piecewise linear finite elements as before, the work required to obtain an error of size $\epsilon>0$ grows at least like $\epsilon^{-1}$. \Cref{fig:kl} shows that both the adaptive and the non-adaptive algorithm are able to achieve this rate.
% %%\begin{figure}[h]
% %%	\centering
% %%	\input{./figures/kl/kl.tex}
% %%	\caption{Convergence of multilevel algorithm for smooth infinite-dimensional  problem.}
% %%	\label{fig:kl}
% %%\end{figure}
% %\begin{figure}[h]
% %	\centering
% %	\input{./figures/kl2/kl2.tex}
% %	\caption{Convergence of nonadaptive and adaptive multilevel algorithm for smooth infinite-dimensional problem.}
% %	\label{fig:kl}
% %\end{figure}


%%% Local Variables:
%%% mode: latex
%%% TeX-master: "../document"
%%% End:
