
%\Cref{pro:stability} shows that if the optimal densities $\optimaldensity_{l}$ associated to a sequence $(\vsp_l)_{l\in\N}$ of polynomial spaces converge to a limit $\optimaldensity_{\infty}$ in an appropriate sense, then we may use samples from $\optimaldensity_{\infty}\measure$ for all $l$ without changing the asymptotic behavior of the approximation error. Next, we study conditions under which such a limit exists. Furthermore, even when no limit exists, we derive upper and lower bounds on $\optimaldensity$ that are independent of the polynomial space. A lower bound on $\optimaldensity$ allows to get rid of the weight $\sqrt{\weight(\psmi)}=1/\sqrt{\optimaldensity}$ in the definition of $e_{\vsp,\infty}(\rs)$ in \Cref{eq:bestlinf}, thus making it easier to verify bounds on this quantity in practice. An upper bound on $\optimaldensity$ allows for the application of Markov Chain methods to obtain samples from $\optimaldensity\measure$.

%Recall from \Cref{eq:optimalmeasure} that the optimal density associated to a polynomial space $\vsp$ and a measure $\measure$ is the mean square function of a $L^2(\measure)$-orthonormal basis $(L_j)_{j=1}^{\dvsp}$ of $\vsp$,
\begin{equation*}
\optimaldensity(\psmi):=\frac{1}{\dvsp}\sum_{j=1}^{\dvsp}|L_j(\psmi)|^2.
\end{equation*}
 The theory of orthogonal univariate polynomials shows that for $\domPS:=[0,1]$ the optimal sampling measures $\samplingmeasure_{k}=\optimaldensity_{k}\measure$ associated to spaces $\polka([0,1])$ of polynomials of degree less than or equal to $k$ converge to the arcsine distribution with Lebesgue density
\begin{equation}
\label{eq:arcsine}
\arcsine_{1}(\psmi)=\frac{1}{\pi \sqrt{\psmi(1-\psmi)}}, \quad\psmi\in[0,1]
\end{equation}
as $k\to\infty$, under rather permissive assumptions on $\measure$ \cite{}.
In other words, the density $\optimaldensity$ converges to $\frac{d\lambda}{d\measure}\arcsine_{1}$, where $\lambda$ is the Lebesgue measure and $\frac{d\lambda}{d\measure}$ the Radon-Nikodym derivative. 

By \Cref{pro:stability} we may thus use samples from $\arcsine_{1}\lambda$ for any choice of $\measure$ and any $k\in\N$ without changing the asymptotic behavior of the approximation error in \Cref{thm:dpls}. Conveniently, samples from $\arcsine_{1}$ can be computed as $\sin(X)$, for a uniform random variable $X$ on $[-\pi/2,\pi/2]$.

In the multivariate setting, if the space of polynomials $\vsp$ under consideration is a tensor product space $\vsp=\bigotimes_{j=1}^{d}\polka([0,1])$ and if $\measure$ is a product measure, then the associated optimal sampling measure $\samplingmeasure=\optimaldensity\measure$ is the product of the univariate ones and converges to the $d$-fold tensor product of the arcsine distribution,
 \begin{equation*}
 \arcsine_{d}(\psmi):=\prod_{j=1}^{d}\frac{1}{\pi\sqrt{\psmi(1-\psmi)}},\quad \psmi\in[0,1]^d
 \end{equation*}
 as $k\to\infty$. However, this does not hold in general. For example, for the space $\vsp_{k,\psmi_1}\oplus \vsp_{k,\psmi_2}=\vspan\{1,\psmi_1,\psmi_2,\psmi_1^2,\psmi_2^2,\dots,\psmi_1^k,\psmi_2^k\}$ of polynomials on $[0,1]^2$ equipped with the Lebesgue measure we have
\begin{equation*}
\begin{split}
\optimaldensity_k(\psmi)&=\frac{1}{2k}\left(\sum_{j=1}^{k}|L_j(\psmi_1)|^2+\sum_{j=1}^{k}|L_j(\psmi_2)|^2\right)\\
&\to \frac{1}{2\pi\sqrt{\psmi_1(1-\psmi_1)}}+\frac{1}{2\pi\sqrt{\psmi_2(1-\psmi_2)}}
\end{split}
\end{equation*}
as $k\to\infty$. The limit density in this case is still bounded by the $d$-dimensional arcsine distribution, which is a special case of the following proposition.
%This shows that we cannot simply replace the optimal measure $\density_{*}\measure$ by the $d$-fold tensor product of the arcsine distribution in \Cref{eq:arcsine}.
 %For example, if  To the best of our knowledge, little is known about the behavior of $\samplingmeasure_{\vsp}=\density_{\vsp}\measure$ for general spaces $\vsp$ and measures $\measure$. In any case, we do 






%To extend \Cref{alg:MCMC} to measures on $\R^d$, the arcsine distribution $\arcsine$ needs to be replaced by one supported on all of $\R$. For example, it is known that if $\measure$ is Gaussian, then the sampling measure $\density_m\measure$ concentrates on an interval of size $[-a_m,a_m]$ with $a_m\approx m^{1/2}$ \cite{}.