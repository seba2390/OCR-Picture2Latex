\section{Application to parametric PDE}
\label{sec:uq}
%In this section, we discuss the applicability of \Cref{thm:main,thm:main2} to problems in uncertainty quantification.

We assume in this section that $\pde(\cdot,\psmi)$ is the solution of some partial differential equation (PDE) with parameters $\psmi\in\domPS\subset\R^\dps$ and that we are interested in the \emph{response surface}
\begin{equation*}
\psmi\mapsto \rs_{\infty}(\psmi):=\QoI(\pde(\cdot,\psmi))\in\R,
\end{equation*}
where $\QoI(\pde(\cdot,\psmi))$ is a real-valued quantity of interest, such as a point evaluation, a spatial average, or a maximum.  
In most situations, we cannot evaluate $\rs_{\infty}(\psmi)$ exactly, as this would require an analytic solution of the PDE. Instead, we have to work with discretized solutions $\pde_{n}(\cdot,\psmi)$ for each $\psmi$, which yield approximate response surfaces
\begin{equation*}
\begin{split}
\rs_{n}\colon &\domPS\to\R \\
&\psmi\mapsto Q(\pde_{n}(\cdot,\psmi)).
\end{split}
\end{equation*}
For example, if we employ finite element discretizations with maximal element diameter $h:=n^{-1}$, then the work required for evaluations of $\rs_{n}$ grows like $h^{-\gamma}=n^{\gamma}$ for some $\gamma>0$. To apply the multilevel method of \Cref{sec:nonadaptive}, we need to verify the remaining Assumptions A1 and A2 from there. 

%consider parametric partial differential equations of the form
%\begin{equation}
%\label{eq:PDE}
%\begin{split}
%\PDE_y(u)=\rs_y
%\end{split}
%\end{equation}
%where both operator  $\PDE_y$ and right-hand side $\rs_y$ may depend on a %parameter $y\in\domPS\subset\mathbb{R}^d$ with $d\in\N\cup\{\infty\}$.%
% Assuming that there is a unique solution $u_y:=\PDE_y^{-1}(\rs_y)$ for each $y\in\domPS$, our goal is to approximate the dependence of a scalar, possibly nonlinear, quantity of interest $\qoi(u_y)$ on the parameter $y$.
As a motivating example, we consider a linear elliptic second order PDE, which has been extensively studied in recent years \cite{harbrecht2013multilevel,ChkifaCohenSchwab2015,CohenDevoreSchwab2011,BabuskaTemponeZouraris2004},
\begin{equation}
\label{eq:UQex}
\begin{aligned}
-\nabla \cdot (a(x,\psmi) \nabla \pde(x,\psmi))&=\rhs(x)&\text{ in }U\subset\R^{\dpde}\\
\pde(x,\psmi)&=0&\text{ on } \partial U,
\end{aligned}
\end{equation}
with $a:U\times \domPS\to\R$ and $\domPS:=[0,1]^\dps$. 


%If $\rhs\in H^{-1}(U)$ and
%\begin{equation}
%\inf_{x\in U}a(x,\psmi)>0\quad \forall \psmi\in\domPS,
%\end{equation}
%then the Lax-Milgram theorem guarantees existence of a unique solution $\pde(\cdot,\psmi)\in H_0^1(U)$ for all $\psmi\in\domPS$. In order to obtain convergence of 
\begin{pro}
	\label{pro:finite}
	For any $n\in\N$, let $\pde_{n}$ be finite element approximations of order $r\geq 1$ and maximal element diameter $h:=(n+1)^{-1}$, and let $\rs_{n}(\psmi):=Q(\pde_n(\cdot,\psmi))$.
	Assume that $g$ and $U$ are sufficiently smooth, that 
	\begin{equation}
		\inf_{x\in U,\psmi\in\domPS}a(x,\psmi)>0,
	\end{equation}
	and that $Q$ is a continuous linear functional on $L^2(U)$.
	
	
	
	\begin{enumerate}[(i)]
		\item If $a\in C^{r}(U\times\domPS)$ for some $r\geq 1$, then  %$\norm{\partial_{x}^{\mathbf{r}}\partial_{\psmi}^{\mathbf{s}}a}{C^0(U\times \domPS)}<\infty$ for all $|\mathbf{r}|_{1}+|\mathbf{s}|_{1}\leq r$, then 
		\begin{equation*}
		\norm{\rs_{\infty}-\rs_{n}}{L^2(\domPS)}\lesssim h^{r+1}
		\end{equation*}
		and
		\begin{equation*}
		\norm{\rs_{\infty}-\rs_{n}}{C^{r-1}(\domPS)}\lesssim h^{2}.
		\end{equation*}
		\item If for some $r,s\geq 1$ we have
		\begin{equation}
		\label{eq:tensor}
		\begin{split}
		a\in C^{r}(U)\otimes C^{s}(\domPS):=\{a\colon U\times \domPS\to\R :  \norm{\partial_{x}^{\mathbf{r}}\partial_{\psmi}^{\mathbf{s}}a}{C^0(U\times\domPS)}<\infty\;\forall\;|\mathbf{r}|_{1}\leq r,|\mathbf{s}|_{1}\leq s\}
		,
		\end{split}
		\end{equation} then 
		\begin{equation*}
		\norm{\rs_{\infty}-\rs_{n}}{C^{s}(\domPS)}\lesssim h^{r+1}.
		\end{equation*}
		
	%	\item If there exists $s\geq $ such that $\norm{\partial_{x}^{\mu}\partial_{\psmi}^{\nu}a}{C^0(U\times \domPS)}<\infty$ for all $|\mu|_{1}\leq r,|\nu|_{1}\leq s$ for some $s\geq 1$, then 
	%	\begin{equation*}
	%	\norm{\rs_{\infty}-\rs_{n}}{C^{s}_{\mix}(\domPS)}\lesssim h^{r+1}.
	%	\end{equation*}
	\end{enumerate}
\end{pro}
\begin{proof}
	In both cases, the standard theory of second order elliptic differential equations shows that $\psmi\mapsto \pde(\cdot,\psmi)$ is well defined as a map from $\domPS$ into $H^{r+1}(U)$, with 
	\begin{equation*}
	\norm{\pde}{L^\infty(\domPS;H^{r+1}(U))}<\infty.
	\end{equation*}
	Next, we observe that the derivatives $\partial_{\psmi_{j}}\pde(\cdot,\psmi)$, $j\in \{1,\dots,d\}$ satisfy PDEs with the same operator as in \Cref{eq:UQex} but with new right-hand sides
	\begin{equation*}
	\tilde{\rhs}(x):=\nabla\cdot(\partial_{\psmi_j}a(x,\psmi)\nabla \pde(x,\psmi)).
	\end{equation*}
	The regularity of this right-hand side now depends on the assumptions on the coefficient $a$.
	In case (i) we have $\partial_{\psmi_j}a(\cdot,\psmi)\in C^{r-1}(U)$ and thus $\tilde{\rhs}\in H^{r-2}(U)$. Therefore, $\partial_{\psmi_{j}}\pde(\cdot,\psmi)\in H^{r}(U)$ for each $\psmi\in\domPS$ and, moreover, we have the uniform estimate
		\begin{equation*}
		\norm{\partial_{\psmi_j}\pde}{L^\infty(\domPS;H^{r}(U))}<\infty.
		\end{equation*}
		In case (ii) we have $\partial_{\psmi_j}a(\cdot,\psmi)\in C^{r}(U)$ and thus $\tilde{\rhs}\in H^{r-1}(U)$. Therefore, $\partial_{\psmi_{j}}\pde(\cdot,\psmi)\in H^{r+1}(U)$ for each $\psmi\in\domPS$ and, moreover, we have the uniform estimate
		\begin{equation*}
		\norm{\partial_{\psmi_j}\pde}{L^\infty(\domPS;H^{r+1}(U))}<\infty.
		\end{equation*}
		Repeatedly applying these arguments yields
			\begin{equation*}
			\norm{\pde}{C^{r-1}(\domPS;H^2(U))}<\infty,
			\end{equation*}
			and
			\begin{equation*}
			\norm{\pde}{C^{s}(\domPS;H^{r+1}(U))}<\infty,
			\end{equation*}
%			or
%			\begin{equation*}
%			\norm{\pde}{C^{s}_{\mix}(\domPS;H^{r+1}(U))}<\infty,
%			\end{equation*}
		in cases (i) and (ii), respectively. We may now conclude by using standard finite-element theory. In case (i), we have
		\begin{equation*}
		\begin{split}
		\norm{\rs_{\infty}-\rs_{n}}{L^2(\domPS)}&\leq \norm{Q}{}\norm{\pde-\pde_{n}}{L^2(\domPS;L^2(U))}\\
		&\lesssim h^{r+1}\norm{\pde}{L^2(\domPS;H^{r+1}(U))}
		\end{split}
		\end{equation*}
		and
		\begin{equation*}
		\begin{split}
		\norm{\rs_{\infty}-\rs_{n}}{C^{r-1}(\domPS)}&\lesssim \norm{\pde-\pde_{n}}{C^{r-1}(\domPS;L^2(U))}\\
		&\lesssim h^{2}\norm{\pde}{C^{r-1}(\domPS;H^{2}(U))},
		\end{split}
		\end{equation*} 
	whereas in case (ii), we have
	\begin{equation*}
	\begin{split}
	\norm{\rs_{\infty}-\rs_{n}}{C^{s}(\domPS)}&\lesssim \norm{\pde-\pde_{n}}{C^{s}(\domPS;L^2(U))}\\
	&\lesssim h^{r+1}\norm{\pde}{C^{s}(\domPS;H^{r+1}(U))},
	\end{split}
	\end{equation*}
\end{proof}
\begin{rem}
In case (i) of the previous proposition, differentiating with respect to $\psmi$ reduces the number of available derivatives in $x$, which are required for convergence of the finite element method. Thus, the convergence in $L^2(\domPS)$ is faster than that in $C^{r-1}(\domPS)$. Case (ii), on the other hand, describes the so-called \emph{mixed smoothness} of the coefficient in $x$ and $\psmi$, meaning that differentiating in $\psmi$ does not affect the differentiability with respect to $x$.  
\end{rem}
 % Roughly speaking, it can be shown that for \Cref{eq:UQex} the response surface $\rs_{\infty}$ is as regular with respect to $\psmi$ as the diffusion coefficient $a_{\psmi}$. For example, if $a_{\psmi}$ is $s$ times differentiable with respect to $\psmi$ and the functional $\QoI$ is $s$ times differentiable with respect to $u$ (both in a suitable sense), then $\rs_{\infty}$ is $s$ times differentiable as well. % and lies in a Sobolev or a Hölder space of functions on $\domPS$ depending on the specifics of the problem and on the integrability of the derivatives $\partial^{s}_{\psmi}a_{\psmi}$. 
If the coefficients depend analytically on $\psmi$, then the same holds for $\rs_{\infty}$, which can be exploited to obtain algebraic polynomial approximability rates of $\rs_{\infty}$ even in the case of infinite-dimensional parameters \cite{ChkifaCohenSchwab2015,Haji-AliNobileTamelliniEtAl2015}, as shown below.

%However, to obtain strong convergence, we need not only know that $\rs_{\infty}\in \F$  for some space $\F$ with good polynomial approximability properties, but that 
%\begin{equation*}
%\norm{\rs_{\infty}-\rs_{n}}{\F}\lesssim n^{-\sc}
%\end{equation*}
%for some $\sc>0$. Estimates of this type have been established for example in \cite{harbrecht2013multilevel,Haji-AliNobileTamelliniEtAl2015}. An example of such an estimate is provided in \Cref{pro:UQ} below. 

%The underlying idea of this example is that the strong convergence rate $\sc=1$ can usually be obtained in any space $\F$ such that $\rs_{\infty}\in \F$. 

\begin{pro}
\label{pro:UQ}
Let $\domPS:=[-1,1]^{\infty}$.
Assume that $Q$ is a linear and continuous functional on $L^2(U)$, that $0<\inf_{x,\psmi}  a(x,\psmi)\leq \sup_{x,\psmi}  a(x,\psmi)<\infty$, and that
	\begin{align*}
	a(x,\psmi)&=\bar{a}(x)+\sum_{j=0}^{\infty}\ps_j\psi_j(x),\\
		a(x,\psmi)&=\bar{a}(x)+\left(\sum_{j=0}^{\infty}\ps_j\psi_j(x)\right)^2,\quad \\
	\intertext{or}
		a(x,\psmi)&=\exp\left(\sum_{j=0}^{\infty}\ps_j\psi_j(x)\right). 
	\end{align*}
	If there exists $r_{\max}>1$ such that
	\begin{equation*}
	 \|\psi_j\|_{C^{r}(U)}\lesssim (j+1)^{-(r_{\max}+1 - r)}\quad\forall j\in\N, \; 0\leq r< r_{\max},
	 \end{equation*} 
	 	 then, for any $r\in \N$ with $1\leq r< r_{\max}$, finite element approximations with maximal element diameter $h:=(n+1)^{-1}$ achieve
\begin{equation*}
\norm{\rs_{\infty}-\rs_{\n}}{L^{\infty}(\domPS)}\leq C h^{r+1}
\end{equation*}
with a constant $C$ independent of $n$.
Furthermore, for any such $r$, there is a sequence $(\vsp_\dvsp)_{\dvsp\in\N}$ of downward closed polynomial spaces with $\dim \vsp_\dvsp=m$ such that finite element approximations with order $r$ and maximal diameter $h:=(n+1)^{-1}$ achieve 
\begin{equation*}
e_{\vsp_\dvsp,1,\infty}(\rs_{\infty}-\rs_\n)\leq C(\dvsp+1)^{-\uqsummability}h^{r+1} \quad \forall\, 0<\uqsummability<r_{\max}- r
\end{equation*}
with a constant $C$ independent of $n$ and $m$.
%and
%\begin{equation*}
%e_{\vsp_\dvsp,2}(\rs_{\infty}-\rs_\n)\leq C(\dvsp+1)^{-\uqsummability}h^{r+1},\quad \forall \uqsummability<r_{\max}-\delta r-\frac{1}{2}
%\end{equation*}
\end{pro}
\begin{proof}
	
	It was shown in \cite[Theorem 4.1 \& Section 5]{ChkifaCohenSchwab2015} that for each $0\leq r<r_{\max}$ there exists a set $\domPS_{r}\subset \C^{\infty}$, $\domPS\subset \domPS_{r}$ such that
	$\norm{a}{L^{\infty}(\domPS_{r};C^{r}(U))}<\infty$ and such that $\psmi\mapsto \pde(\cdot,\psmi)$ may be extended to a complex differentiable map from $\domPS_{r}$ into $H^{1+r}(U)$ with
		\begin{equation}
		\label{eq:complexbounds}
		\begin{split}
	\norm{\pde}{L^{\infty}(\domPS_{r};H^{1+r}(U))}<\infty
		\end{split}
		\end{equation}
	
For a detailed description of the sets $\domPS_{r}$ we refer to \cite{ChkifaCohenSchwab2015}. For our purposes it suffices to know that the better the summability of $(\norm{\psi_j}{C^{r}(U)})_{j\in\N}$, the larger $\domPS_{r}$ can be chosen; and the larger $\domPS_{r}$ the better the polynomial approximability properties of complex differentiable maps defined on $\domPS_{r}$.
In particular, the results of \cite[Section 2]{ChkifaCohenSchwab2015}, show that when  restricted to the smaller set $\domPS$ such maps may be approximated at algebraic convergence rates within downward closed polynomial subspaces. More specifically, \cite[Equation (2.27)]{ChkifaCohenSchwab2015} shows that if a function $e$ is complex differentiable on $\domPS_{r}$, then for any $\dvsp\in\N$ there exists a downward closed polynomial subspace $\vsp_{\dvsp}$ such that
	\begin{equation*}
	\begin{split}
	\inf_{\tilde{\p}\in \vsp_{\dvsp}\otimes L^2(U)}\norm{e-\tilde{\p}}{L^\infty(\domPS;L^2(U))}\lesssim (\dvsp+1)^{-\alpha} \norm{e}{L^\infty(\domPS_{r};L^2(U))}
	\end{split}
	\end{equation*}
	for all $\alpha<r_{\max}- r$. 
 Applying this estimate with $e:=\pde-\pde_{n}$ shows
	\begin{equation*}
	\begin{split}
	\inf_{\p\in\vsp_{\dvsp}}\norm{(\rs_{\infty}-\rs_{\n})-\p}{L^\infty(\domPS)}&\leq \norm{Q}{} \inf_{\tilde{\p}\in \vsp_{\dvsp}\otimes L^2(U)}\norm{(\pde-\pde_n)-\tilde{\p}}{L^\infty(\domPS;L^2(U))}\\
\lesssim & (\dvsp+1)^{-\alpha} \norm{\pde-\pde_{n}}{L^\infty(\domPS_{r};L^2(U))}.
	\end{split}
	\end{equation*}

By standard finite element analysis we finally obtain
\begin{equation*}
\norm{\pde-\pde_{n}}{L^{\infty}(\domPS_{r};L^2(U))}\leq C h^{r+1}\norm{\pde}{L^{\infty}(\domPS_{r};H^{r+1}(U))}.
\end{equation*}
with  $C=C\big(	\norm{a}{L^{\infty}(\domPS_{r};C^{r}(U))}\big)<\infty$. Combining the previous two estimates with \Cref{eq:complexbounds} concludes the proof.
\end{proof}

\begin{rem}
	Similar results can also be shown for PDEs of parabolic type and for some nonlinear PDEs \cite{ChkifaCohenSchwab2015}.
\end{rem}