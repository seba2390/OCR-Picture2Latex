%!TEX root = main.tex
\section{Main tool}\label{sec:simple}

In this section we state and prove our first main result, Theorem~\ref{thm:simple}.
It is simpler both in the formulation and in the proof from the more involved
Theorem~\ref{thm:advanced}, but sufficient to show applications in Section \ref{sec:applications}.

The statement of Theorem~\ref{thm:simple} may seem at first glance a bit artificially involved
thus before stating it we explain the intuition behind this research direction.
Recently there is a big progress in showing lower complexity bounds for the reachability problem
in VASSes. On the intuitive level proving a new lower bound often boils down to finding
a new source of hardness in VASSes or in other words designing a family of VASSes
which is involved in some sense. In particular these involved VASSes should have
a run from some source configuration to some target configuration,
but each such run should be long. For some reason VASS families proposed in a number of recent constructions
share many similarities. The first one is that the runs from the source to the target usually have some
very specific shape and there is exactly one shortest run. Along this run usually some very restrictive
invariants are kept and any deviation from them results in the impossibility of reaching the target configuration.
To our best knowledge the same schema reappears in all the known lower bound
results, in particular in the recent ones~\cite{DBLP:journals/jacm/CzerwinskiLLLM21,DBLP:conf/concur/Czerwinski0LLM20,DBLP:conf/focs/Leroux21,DBLP:conf/focs/CzerwinskiO21,DBLP:journals/corr/abs-2203-04243}.
In many of these constructions the following invariant is kept: product of some two counters
$\vr{x}$ and $\vr{y}$ equals the other counter $\vr{z}$ where the counter $\vr{x}$ is bounded,
while $\vr{y}$ and $\vr{z}$ are unbounded. In other words the ratio $\vr{y} / \vr{z}$ is fixed and bounded
at some point at the run. The reason behind this phenomenon is the use of so called
multiplication triples technique introduced in~\cite{DBLP:conf/stoc/CzerwinskiLLLM19}.
Theorem~\ref{thm:simple} tries to describe this situation more precisely and state that in some way for VASSes
following this popular line of design we might expect to have troubles with avoiding big separators.
Because assumptions of Theorem~\ref{thm:simple} are actually very strong and restrictive
we formulate also Theorem~\ref{thm:advanced} in Section~\ref{sec:advanced} which tries
to take a bit more flexible approach for making the above described phenomena more precise.
Notice that in Theorem~\ref{thm:simple} we describe VASSes with no run from $s$ to $t$, but
there is a close resemblance to the above described setting as there is a run from $s + e_2$ to $t$.

\begin{theorem}\label{thm:simple}
Let $V$ be a $d$-VASS, $s, t \in Q \times \N^d$ be two its configurations, $q \in Q$ be a state
and a line $\alpha = a + \N \Delta$ for vectors $a, \Delta \in \N^d$ be such that
\begin{enumerate}[(1)]
  \item $\Delta = (\Delta[1], \Delta[2], 0^{d-2}) \in \N^d$,
  \item there is no run from $s$ to $t$,
  \item $q(\alpha) \subseteq \post_V(s)$,
  \item $q(\alpha + \N_{+} e_2) \subseteq \pre_V(t)$.
\end{enumerate}
Then each separator for $(V, s, t)$ contains a period $r \cdot \Delta$ for some $r \in \Q$.
\end{theorem}

\begin{proof}
Consider an arbitrary separator $S$ for $(V, s, t)$. Let $S_q = \{v \mid q(v) \in S, v \in \N^d\}$
be its part devoted to the state $q$ and let $S_q = \bigcup_{i \in I} L_i$,
where $L_i$ are linear sets. As the line $\alpha$ contains infinitely many points
and $\alpha \subseteq S_q$ then some of the linear sets $L_i$ have to contain infinitely many points of $\alpha$.
Denote this $L_i$ by $L$, let $L = b + \N p_1 + \ldots + \N p_k$.
We know that for arbitrarily big $n$ we have $a + n\Delta = b + n_1 p_1 + \ldots + n_k p_k$ for some $n_i \in \N$.
Wlog. we can assume that $n_i > 0$, we just do not write the periods $p_i$ with coefficient $n_i = 0$.
Notice first that for each coordinate $j \in [3,d]$ we have $(a + n\Delta)[j] \leq \norm(a)$.
Therefore for $p_i$ such that $p_i[j] > 0$ for some $j \in [3,d]$ we have $n_i \leq \norm(a)$.
Let $P_0$ be the set of periods $p_i$ such that $p_i[j] = 0$ for all $j \in [3,d]$ and
$P_{\neq 0}$ be the set of the other periods $p_i$.
We thus have
\[
(a - b - \sum_{p_i \in P_{\neq 0}} n_i p_i) + n\Delta = \sum_{p_i \in P_0} n_i p_i,
\]
where the sum $v = (a - b - \sum_{p_i \in P_{\neq 0}} n_i p_i)$ has a norm bounded by
$B = \norm(a) + \norm(b) + k \cdot \norm(a) \cdot \max_{p_i \in P_{\neq 0}} \norm(p_i)$,
which is independent of $n$. If we restrict the equation to the first two coordinates of the considered vectors we have
\begin{equation}\label{eq:pzero}
(v[1], v[2]) + n (\Delta[1], \Delta[2]) = \sum_{p_i \in P_0} n_i (p_i[1], p_i[2]).
\end{equation}
We aim at showing that one of $p_i$ is equal to $r \cdot \Delta$ for some $r \in \Q$. Recall that for each $j \in [3,d]$
we have $\Delta[j] = p_i[j] = 0$, so it is enough to show that $(p_i[1], p_i[2]) = r \cdot (\Delta[1], \Delta[2])$.

We first show the following claim.

\begin{claim}\label{cl:ratio}
For each period $p \in P_0$ we have $\Delta[1] \cdot p[2] \leq \Delta[2] \cdot p[1]$.
\end{claim}

\begin{proof}
The intuitive meaning of $\Delta[1] \cdot p[2] \leq \Delta[2] \cdot p[1]$ is that $\frac{\Delta[1]}{\Delta[2]} \leq \frac{p[1]}{p[2]}$,
however we cannot write the fraction $\frac{p[1]}{p[2]}$ as it might happen that $p[2] =  0$.
Assume towards a contradiction that for some $p \in P_0$ we have $\Delta[1] \cdot p[2] > \Delta[2] \cdot p[1]$.
Let $\delta = \Delta[1] \cdot p[2] - \Delta[2] \cdot p[1] > 0$. Recall that $a + n\Delta \in L$ for some $n \in \N$,
therefore also $a + n\Delta + \Delta[1] \cdot p \in L$ as $p \in P_0$ is a period of $L$.
Then however
\begin{align*}
\Delta[1] \cdot p & = \Delta[1]  \cdot (p[1], p[2], 0^{d-2}) = (\Delta[1] \cdot p[1], \Delta[1] \cdot p[2], 0^{d-2}) \\
& = (\Delta[1] \cdot p[1], \Delta[2] \cdot p[1] + \delta, 0^{d-2}) = p[1] \cdot \Delta + \delta \cdot e_2 \in \N \Delta + \N_{+} e_2.
\end{align*}
Therefore $a + n\Delta + \Delta[1] \cdot p \in b + \N \Delta + \N_{+} e_2$ and it means that $q(a + n\Delta + \Delta[1] \cdot p) \in \pre_V(t)$.
However we know that $q(a + n\Delta + \Delta[1] \cdot p) \in q(S_q) \subseteq S$ and therefore separator $S$ nonempty intersects 
the set $\pre_V(t)$. This is a contradiction with the definition of separator.
\end{proof}

In order to show that $p = r \cdot \Delta$ for some $p \in P_0$ it is sufficient to show that $\Delta[1] \cdot p[2] = \Delta[2] \cdot p[1]$.
Assume towards a contradiction that for all $p \in P_0$ we have $\Delta[1] \cdot p[2] \neq \Delta[2] \cdot p[1]$.
By Claim~\ref{cl:ratio} we know that actually for all $p \in P_0$ we have that $\Delta[1] \cdot p[2] < \Delta[2] \cdot p[1]$.
Thus $p[1] > 0$ and we can equivalently write that for all $p \in P_0$ we have that $\frac{\Delta[2]}{\Delta[1]} > \frac{p[2]}{p[1]}$.
Let $F$ be the maximal value of $\frac{p[2]}{p[1]}$ for $p \in P_0$, clearly $\frac{\Delta[2]}{\Delta[1]} > F$,
so
\begin{equation}\label{eq:f}
\Delta[2] > F \cdot \Delta[1].
\end{equation}

By~\eqref{eq:pzero} we know that
\[
\frac{(v+n\Delta)[2]}{(v+n\Delta)[1]} \leq F,
\]
as $v+n\Delta$ is a positive linear combination of vectors $p \in P_0$ and for each $p \in P_0$
we have $\frac{p[2]}{p[1]} \leq F$.
Therefore it holds
\[
v[2] + n\Delta[2] \leq F(v[1] + n\Delta[1])
\]
and equivalently
\[
n(\Delta[2] - F\Delta[1]) \leq Fv[1] - v[2].
\]
By~\eqref{eq:f} we have that $\Delta[2] - F\Delta[1] > 0$ therefore
\[
n \leq \frac{Fv[1] - v[2]}{\Delta[2] - F\Delta[1]}.
\]
This is in contradiction with the assumption that $n$ can be arbitrarily big
and finishes the proof.
\end{proof}

\begin{remark}
In Theorem~\ref{thm:simple} instead of separator for $(V, s, t)$ one can consider a separator for $(V', s, t)$,
where $V'$ is obtained from $V$ by adding a loop in state $q$ with the effect of decreasing the second counter.
Indeed, it is easy to observe that all the points 1-4 in the theorem statement remain true after substitution of $V$ by $V'$.
Such a version of Theorem~\ref{thm:simple} is a bit more convenient for some of the applications.
\end{remark}
