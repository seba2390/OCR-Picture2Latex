%!TEX root = main.tex
\section{Applications}\label{sec:applications}
In this section we show how Theorem~\ref{thm:simple} can be used to obtain lower bounds
on the separator size. In Section~\ref{ssec:4vass} we prove that using a construction from~\cite{DBLP:conf/concur/Czerwinski0LLM20}
one can obtain a $4$-VASS with separators of at least doubly-exponential size. 
In Section~\ref{ssec:tower} we show that there exist VASSes with separators of arbitrary high elementary size
of a special shape. Existence of separators of arbitrary high elementary size is an easy consequence
of \tower-hardness of VASS reachability problem, we provide in Theorem~\ref{thm:tower-sep}
a concrete instance of such a separator. It is not a big contribution, but the aim of proving Theorem~\ref{thm:tower-sep}
is rather to illustrate that our techniques can be quite easily applied to many existing VASS examples.
%Finally in Section~\ref{ssec:further} we comment about possible application of Theorem~\ref{thm:advanced}.

\subsection{Doubly-exponential separator in a $4$-VASS}\label{ssec:4vass}
The aim of this section is to show the following theorem.

\begin{theorem}\label{thm:4vass-sep}
There exists a family of binary $4$-VASSes $(V_n)_{n \in \N}$ of size $\Oo(n^3)$
such that for some configurations $s_n, t_n$ of $V_n$ with $\norm(s_n), \norm(t_n) \leq 1$
such that $s_n$ does not reach $t_n$ the smallest
separator for $(V_n, s_n, t_n)$ is of doubly-exponential size wrt. $n$.
\end{theorem}

The rest of this section is devoted to the proof of Theorem~\ref{thm:4vass-sep}.
Our construction is based on the construction of a family of binary $4$-VASSes $(U_n)_{n \in \N}$
with shortest run of doubly-exponential length,
which is described in Section 5 in~\cite{DBLP:conf/concur/Czerwinski0LLM20}.
The picture below illustrates VASS $U_n$, it is taken from~\cite{DBLP:conf/concur/Czerwinski0LLM20} and modified a bit.
The proof follows a very natural line and is not a big challenge.
It boils down to performing a small modification to $4$-VASSes from~\cite{DBLP:conf/concur/Czerwinski0LLM20}
in order to ensure conditions of Theorem~\ref{thm:simple} and then checking that indeed these
conditions are satisfied. We sketch here the idea behind the construction of the mentioned family of $4$-VASSes
only into such an extend that we can explain the proof of Theorem~\ref{thm:4vass-sep},
for details we refer to~\cite{DBLP:conf/concur/Czerwinski0LLM20}.


\begin{tikzpicture}[->,>=stealth',shorten >=1pt,auto,node distance=2.4cm,semithick]%, xscale=0.7]

\node(sp) {$\cdot$};
%\node(spp) [right of=sp] {};
\node(s) [above of = sp] {$\cdot$};

\node (pk) [right of = sp] {$p_k$};
%\node (ppk) [right of=pk] {};
\node (qk) [above of = pk] {$q_k$};

\node (pk1) [right = 2.5cm of pk] {$p_{k-1}$};
%\node (ppk1) [right of=pk1] {};
\node (qk1) [above of=pk1] {$q_{k-1}$};

\node (dotst) [right = 0.5cm of pk1] {$\cdots$};
\node (dots) [right = 0.1cm of dotst] {};

\node (p1) [right = 1.2cm of dots] {$\ p_1\ $};
\node (q1) [above of=p1] {$\ q_1\ $};

\node (e) [right = 1.2cm of p1] {$\ p_0\ $};

\path[->]
(s) edge[->] node[left] {\small $(1,1,0,0)$} (sp)
(sp) edge[->, in=310, out=220, loop] node[below] {\small $(1,1,0,0)$}(sp)
%
(sp) edge[->] node[above] {\small $(0,0,0, 2^k)$} (pk)
(pk) edge[->, in=310, out=220, loop] node[below] {\small $(0,-1,1,0)$}(pk)
(qk) edge[->, in=130, out=40, loop] node[above] {\small $(0, a_k,-b_k,0)$}(qk)
(pk) edge[->, bend left=25]  (qk)
(qk) edge[->, bend left=25] node[right] {\small $(0,0,0,-1)$} (pk)
%
(pk) edge[->] node[above] {\small $(0,0,0, 2^{k-1})$} (pk1)
(pk1) edge[->, in=310, out=220, loop] node[below] {\small $(0,-1,1,0)$}(pk1)
(qk1) edge[->, in=130, out=40,loop] node[above] {\small $(0, a_{k-1},-b_{k-1},0)$}(qk1)
(pk1) edge[->, bend left=25] (qk1)
(qk1) edge[->, bend left=25] node[right] {\small $(0,0,0,-1)$} (pk1)
%
(dots) edge[->] node[above] {\small $(0,0,0, 2^{1})$} (p1)
(p1) edge[->, in=310, out=220,loop] node[below] {\small $(0,-1,1,0)$}(p1)
(q1) edge[->, in=130, out=40, loop] node[above] {\small $(0, a_{1},-b_{1},0)$}(q1)
(p1) edge[->, bend left=25]  (q1)
(q1) edge[->, bend left=25] node[right] {\small $(0,0,0,-1)$} (p1)
%
(p1) edge[->] node[left] {\small } (e)
(e) edge[->, in=310, out=220, loop] node[below] {\small $(-a,-b,0,0)$}(e)
;
\end{tikzpicture}


The construction relies on the following lemma (Lemma 12 in~\cite{DBLP:conf/concur/Czerwinski0LLM20}).

\begin{lemma}\label{lem:fractions}
For each $n \geq 1$ there are $n$ rational numbers
\[
1 < f_1 < \ldots < f_n = 1 + \frac{1}{4^n}
\]
of description size bounded by $4^{n^2+n}$, such that the description size of $f$
defined as
\begin{equation}\label{eq:fractions}
f = (f_n)^{2^n} \cdot \ldots \cdot (f_2)^{2^2} \cdot (f_1)^{2^1}
\end{equation}
is bounded by $4^{2(n^2+n)}$.
\end{lemma}

The VASSes $U_n$ are constructed as follows.
Valuation of the four counters $(\vr{x}_1, \vr{x}_2, \vr{x}_3, \vr{x}_4)$ in the distinguished \emph{initial} state is initially $0^4$,
the run is \emph{accepting} if it finishes in the distinguished \emph{final} state also with valuation $0^4$.
Each run of $U_n$ consists of $n+2$ phases: the \emph{initial phase},
$n$ phases corresponding to fractions $f_n, f_{n-1}, \ldots, f_2, f_1$, respectively and the \emph{final phase}.
In each run after the initial phase the counter valuation is equal to $(N, N, 0, 0)$ for some nondeterministically guessed value $N \in \N$.
In every accepting run the phase corresponding to the fraction $f_i$ results only in multiplying
the second counter $\vr{x}_2$ by a value $f_i^{2^i}$. Therefore in such an accepting run
counter valuation after phases corresponding to fractions $f_n, \ldots, f_i$ is the following
\[
(\vr{x}_1, \vr{x}_2, \vr{x}_3, \vr{x}_4) = (N, N \cdot f_n^{2^n} \cdot \ldots \cdot f_i^{2^i}, 0, 0).
\]
In particular after all the $n$ phases corresponding to fractions the second counter is equal to
\[
N \cdot f_n^{2^n} \cdot \ldots \cdot f_1^{2^1} = N \cdot f,
\]
where the equality follows from Lemma~\ref{lem:fractions}.
Let $p_i$ be the state of $U_n$ after the initial phase, $n-i$ phases corresponding to fractions $f_n, \ldots, f_{i+1}$
and just before the phase corresponding to fraction $f_i$ (in the case when $i > 0$).
It is important that due to Claim 15 in~\cite{DBLP:conf/concur/Czerwinski0LLM20} any reachable configuration
of the form $p_i(\vr{x}_1, \vr{x}_2, \vr{x}_3, \vr{x}_4)$ satisfies $\vr{x}_2 \leq f_n^{2^n} \cdot \ldots \cdot f_i^{2^i} \cdot \vr{x}_1$.
In particular for any reachable configuration $p_0(\vr{x}_1, \vr{x}_2, \vr{x}_3, \vr{x}_4)$ we have that
$\vr{x}_2 \leq f \cdot \vr{x}_1$, as $f = f_n^{2^n} \cdot \ldots \cdot f_1^{2^1}$.
Let $f = \frac{a}{b}$. In the final phase the transition with the effect $(-b, -a, 0, 0)$ is applied in a loop. One can easily see that in order
to reach counter values $0^4$ after the final phase we need to have in the state $p_0$ values satisfying
the equality $\vr{x}_2 = f \cdot \vr{x}_1$. Similarly any configuration $q_{n-1}(\vr{x}_1, \vr{x}_2, \vr{x}_3, \vr{x}_4)$
on an accepting run needs to satisfy $\vr{x}_2 = f_n^{2^n} \cdot \vr{x}_1$. Let $f_i = \frac{a_i}{b_i}$ for all $i \in [1,n]$.
Then in any accepting run $\vr{x}_1$ needs to be divisible by $b_n^{2^n}$, which is a doubly-exponential number wrt. $n$. This forces
any accepting run of $U_n$ to be doubly-exponential.

We slightly modify $4$-VASSes $U_n$ in order to obtain $4$-VASSes $V_n$, which fulfil conditions of Theorem~\ref{thm:simple}.
VASS $V_n$ is obtained from $U_n$ by adding at the end of $U_n$ two instructions:
1) decrease of $\vr{x}_2$ by $1$; and then 2) a loop, which decreases $\vr{x}_2$ by an arbitrary nonnegative number.
We first show that $V_n$ indeed satisfies conditions of Theorem~\ref{thm:simple} and then we argue how this finishes
the proof of Theorem~\ref{thm:4vass-sep}.
Let $q_\inp$ be the initial state of VASS $U_n$, $q_\out$ be its final state and $q_\last$ be the state after the final phase,
but before applying the above mentioned decrements of $\vr{x}_2$.
We set $s_n = q_\inp(0^4)$ and $t_n = q_\out(0^4)$. Let $N = \prod_{i=1}^n b_i^{2^i}$.
We set $\Delta = (N, N \cdot f_n^{2^n}, 0, 0) \in \N^4$ and fix state $q = q_{n-1}$.
We claim that VASS $V_n$ together with two configurations $s_n$ and $t_n$,
state $q$ and vector $\Delta$ satisfies conditions of Theorem~\ref{thm:simple}.

Condition (1) is satisfied trivially as $\Delta = (N, N \cdot f_n^{2^n}, 0, 0)$.
In order to see that (2) is satisfied recall that if $V_n$ reaches $q_0(\vr{x}_1, \vr{x}_2, \vr{x}_3, \vr{x}_4)$ then
we have that $\vr{x}_2 \leq \frac{a}{b} \vr{x}_1$. In the final phase we subtract in the loop vector $(b, a, 0, 0)$,
but it does not change this inequality. Finally subtracting at least $1$ at $\vr{x}_2$ implies that any
reachable configuration $q_\out(\vr{x}_1, \vr{x}_2, \vr{x}_3, \vr{x}_4)$ satisfies $\vr{x}_2 + 1 \leq \frac{a}{b} \cdot \vr{x}_1$.
This is not true for $\vr{x}_1 = \vr{x}_2 = 0$, which shows that condition (2) is indeed satisfied.
Observe now that for each $k \in \N$ configuration $p_n(kN, kN, 0, 0)$ after the initial phase is reachable from $s_n$.
Thus also configuration $p_{n-1}(kN, kN\cdot f_n^{2^n}, 0, 0)$ is reachable from $s_n$ if we multiply the second counter by $f_n^{2^n}$,
which proves condition (3).
Observe also that starting in a configuration $p_{n-1}(kN, kN\cdot f_n^{2^n}, 0, 0)$ one can reach valuation $q_\last(0^4)$ before
the decrements of $\vr{x}_2$. Therefore after adding $e_2 = (0, 1, 0, 0)$ to both sides
we get that $p_{n-1}(kN, kN\cdot f_n^{2^n}+1, 0, 0) \reaches q_\last(0,1,0,0) \reaches t_n$ for any $k \in \N$.
Because of an option of decreasing $\vr{x}_2$ many times in $q_\last$ we get that 
$p_{n-1}(kN, kN\cdot f_n^{2^n} + \ell, 0, 0) \reaches t_n$ for any $k, \ell \geq 1$, which equivalent to the condition (4).

Therefore by Theorem~\ref{thm:simple} we have that each separator for $(V_n, s_n, t_n)$
contains a period $\Delta \cdot r \in \N^4$ for some $r \in \Q$. Let $\Delta \cdot r = (\vr{x}_1, \vr{x}_2, 0, 0) \in \N^4$.
We know that $\vr{x}_2 = \vr{x}_1 \cdot \Big(\frac{a_n}{b_n}\Big)^{2^n}$, so in order for $\vr{x}_2 \in \N$
we need to have $b_n^{2^n} \mid \vr{x}_1$. This implies that $\vr{x}_1$ is doubly-exponential with respect to $n$.
Therefore size of the period $\Delta \cdot r$ and thus size of the separator is doubly-exponential with respect to $n$.

\subsection{Tower size separators}\label{ssec:tower}
In this section we show the following result.

\begin{theorem}\label{thm:tower-sep}
There exists a family of VASSes $(V_n)_{n \in \N}$ of size polynomial wrt. $n$
such that for some configurations $s_n, t_n$ of $V_n$ with $\norm(s_n) = \norm(t_n) = 0$ the following is true:
\begin{itemize}
  \item $s_n \reaches t_n$, but the shortest run is $n$-fold exponential,
  \item $s_n \nreaches t_n + e$ for some elementary vector $e$ and each separator
  for $(V_n, s_n, t_n + e)$ is of at least $n$-fold exponential size.
\end{itemize}
\end{theorem}

The rest of this section is dedicated to prove Theorem~\ref{thm:tower-sep}.
Construction of VASSes $V_n$ is based on constructions in~\cite{DBLP:conf/stoc/CzerwinskiLLLM19},
but we do not follow exactly~\cite{DBLP:conf/stoc/CzerwinskiLLLM19} in order to avoid
some technicalities and simplify the construction.
As the construction~\cite{DBLP:conf/stoc/CzerwinskiLLLM19} is pretty involved we decided
only to sketch the intuition behind the constructed VASSes and use many of its properties
without a detailed explanation. The main goal of this section is to provide an intuition why Theorem~\ref{thm:simple}
is indeed applicable to that case, for details of the construction we refer to the original paper~\cite{DBLP:conf/stoc/CzerwinskiLLLM19}.
Essentially speaking Theorem~\ref{thm:simple} is applicable to VASSes in which the configurations on the accepting run
are distinguished from the others by keeping some specific ratio of counter values. The bigger the description
size of the ratio the bigger the separator. In VASSes from~\cite{DBLP:conf/stoc/CzerwinskiLLLM19} the description size
is $n$-fold exponential, which implies the lower bound on the size of separators.

We say that a counter is $B$-bounded if its value is upper bounded by $B$ along the whole run.
In~\cite{DBLP:conf/stoc/CzerwinskiLLLM19} counters are assured to be $B$-bounded (for various values of $B$)
in the following way. In order to guarantee that counter $\vr{x}$ initially equal to $0$ is $B$-bounded
we introduce another counter $\vr{\bar{x}}$ which is initialised to $B$ and an invariant
$\vr{a} + \vr{\bar{a}} = B$ is kept throughout the whole run.
The construction of~\cite{DBLP:conf/stoc/CzerwinskiLLLM19} strongly relies on the fact that having
a triple $(\vr{x}, \vr{y}, \vr{z}) = (B, C, BC)$ one can simulate $C / 2$ zero-tests for a $B$-bounded counter.
One zero-test for counter $\vr{a}$ is realised as follows. We decrease $\vr{y}$ by $2$. Then we enter two loops, the first
one with the effect $(-1, 1, -1)$ on counters $(\vr{a}, \vr{\bar{a}}, \vr{z})$
and the second one with the effect $(1, -1, -1)$ on the same counters. It is easy to see that for a $B$-bounded
counter $\vr{a}$ the maximal possible decrease on $\vr{z}$ after these two loops is equal to $2B$ and it can
only be realised if before and after the loops we have $(\vr{a}, \vr{\bar{a}}) = (0, B)$.

Let $3!^n$ be the number $3$ followed by $n$ applications of the factorial function.
For example $3!^0 = 3$, $3!^1 = 6$ and $3!^2 = 720$.
Roughly speaking VASS in~\cite{DBLP:conf/stoc/CzerwinskiLLLM19} consists of a sequence of $n$ gadgets,
such that in every accepting run they compute triples of the form $(3!^i, k_i, 3!^i \cdot k_i)$
for some nondeterministically guessed values $k_i \in \N$.
The first gadget $\B$ computes triple $(3, k_1, 3 \cdot k_1)$ in a very easy way: after increasing its counters by $(3, 0, 0)$
it fires a nondeterministically guessed number $k_1$ of times a loop with the effect equal to $(0, 1, 3)$.
Next we have a sequence of $n-1$ gadgets $\F$, the $i$-th one inputing a triple $(3!^i, k_i, 3!^i \cdot k_i)$
and outputting a triple $(3!^{i+1}, k_{i+1}, 3!^{i+1} \cdot k_{i+1})$ on some other set of three counters.
It is important to mention that correct value of the output triple requires that after the run values of the input triple are all zero.
The last triple $(3!^n, k_n, 3!^i \cdot k_n)$ is used in~\cite{DBLP:conf/stoc/CzerwinskiLLLM19}
to simulate $3!^n$-bounded counters of a counter automaton.
Here however we modify this construction in order to obtain VASSes $V_n$,
which fulfil the conditions of Theorem~\ref{thm:tower-sep}.
As all the triples use different counters the presented VASS has at least dimension $3n$.
In the original construction of~\cite{DBLP:conf/stoc/CzerwinskiLLLM19}
some of the counters were actually reused in order to decrease the dimension.
Here we allow for a wasteful use of counters, this however do not change the idea of the construction.

The family of VASSes $V_n$ is defined as follows.
We distinguish an initial state $q_\inp$ of $V_n$ and a final state $q_\out$ of $V_n$.
For $i \in [1,n]$ let $q_i$ be the state after the $i$-th gadget. From the above description we get that
in state $q_i$ valuation of some three counters is equal to $(3!^i, k_i, 3!^i \cdot k_i)$. Let us denote
these counters $(\vr{x}_i, \vr{y}_i, \vr{z}_i)$. Thus in $q_n$ we reach counter values
$(\vr{x}_n, \vr{y}_n, \vr{z}_n) = (3!^n, k_n, 3!^n \cdot k_n)$ for some $k_n \geq 0$.
Then after state $q_n$ we define a state $q_\decr$ in which we decrease values of $\vr{y}_n$ and $\vr{z}_n$
by applying some nonzero number of zero-tests for $3!^n$-bounded counters.
This operation can be seen as a loop decreasing counters $(\vr{x}_n, \vr{y}_n, \vr{z}_n)$ by $(0, 1, 3!^n)$,
but of course subtracting $(0, 1, 3!^n)$ is not realised by a single transition, but by some smaller sub-gadget of our VASS.
Then we go to a state $q_\out$ in which we can decrease in a loop the counter $\vr{x}_n$
and as well the counter $\vr{y}_n$.
We define $s_n = q_\inp(0^d)$ and $t_n = q_\out(0^d)$ for an appropriate dimension $d$.
We set a vector $e$ to be zero on all the coordinates beside $\vr{y}_n$ on which it is set to be one.
We claim that $V_n$ with configurations $s_n$, $t_n$ and an elementary vector $e$ satisfy
the conditions of Theorem~\ref{thm:tower-sep}. It is easy to see
(assuming all the above remarks about the construction of~\cite{DBLP:conf/stoc/CzerwinskiLLLM19})
that all the accepting runs need to traverse through a configuration $q_n(3!^n, k, 3!^n \cdot k)$ for some $k \geq 1$,
which implies that all the accepting runs have at least $n$-fold exponential length.
It therefore remains to show the second point of Theorem~\ref{thm:tower-sep},
we apply Theorem~\ref{thm:simple} for that purpose.

As counters $\vr{x}_n$, $\vr{y}_n$ and $\vr{z}_n$ are important let us assume wlog. that they correspond to the first three
coordinates in our notation, respectively.
Let $\Delta = (0, 1, 3!^n, 0^{d-3}) \in \N^d$, namely $\Delta[i] = 0$ for all $i \not\in \{\vr{y}_n, \vr{z}_n\}$, $\Delta[\vr{y}_n] = 1$
and $\Delta[\vr{z}_n] = 3!^n$.  Properties of $V_n$ can be summarised in the following claim, which can be
derived from~\cite{DBLP:conf/stoc/CzerwinskiLLLM19}.

\begin{claim}\label{cl:tower}
If $s_n \reaches q_\decr(x, y, z, 0^{d-3})$ then $x = 3!^n$ and $y \leq 3!^n z$.
Moreover for any $y \in \N$ we have $s_n \reaches q_\decr(3!^n, y, 3!^n \cdot y, 0^{d-3})$.
\end{claim}

Now we aim at showing that VASS $V_n$ together with configurations $s_n$, $t_n+e$,
vector $\Delta$ and state $q_\decr$ fulfils conditions of Theorem~\ref{thm:simple}.
It is immediate to see that condition (1) is satisfied as only second and third coordinates in $\Delta$ are nonzero
(we allow for reordering the coordinates in Theorem~\ref{thm:simple} without loss of generality).
In order to show (2) we rely on Claim~\ref{cl:tower}. We have $t_n + e = q_\out(0, 1, 0, 0^{d-3})$,
thus if $s_n \reaches t_n + e$ we need to have $s_n \reaches q_\decr(x, y, 0, 0^{d-3}) \reaches t_n + e$
for some $y > 0$. This is however a contradiction with Claim~\ref{cl:tower}, as then $y \cdot 3!^n > 0$.
By Claim~\ref{cl:tower} we also immediately derive condition (3). To show condition (4) notice
that because of the loop in $q_\decr$ of effect $(0, -1, -3!^n)$ on counters $(\vr{x}_n, \vr{y}_n, \vr{z}_n)$
we have
\[
q_\decr(0, k+\ell, k \cdot 3!^n, 0^{d-3}) \reaches q_\decr(0, \ell, 0^{d-2}) \reaches q_\out(0, \ell-1, 0^{d-2})
\reaches q_\out(0, 1, 0^{d-2}) = t_n + e 
\]
for any $k, \ell \geq 1$. This shows that indeed Theorem~\ref{thm:simple} can be applied to $V_n$.
Thus each separator for $(V_n, s_n, t_n + e)$ contains a period of a form $(r, r \cdot 3!^n, 0, 0^{d-3}) \in \N$.
As $r \in \N$ the number $r \cdot 3!^n$ is $n$-fold exponential and thus the size of any separator for $(V_n, s_n, t_n+e)$
is $n$-fold exponential, which finishes the proof of Theorem~\ref{thm:tower-sep}.

