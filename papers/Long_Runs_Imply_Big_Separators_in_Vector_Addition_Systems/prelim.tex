%!TEX root = main.tex
\section{Preliminaries}\label{sec:prelim}

\subparagraph*{Basic notions}
We denote by $\N$ the set of nonnegative integers and by $\N_{+}$ the set of positive integers.
For $a, b \in \N$ by $[a, b]$ we denote the set $\{a, a+1, \ldots, b-1, b\}$.
For a set $S$ we write $|S|$ to denote its size, i.e. the number of its elements.
For two sets $A, B$ we define $A + B = \{a+b \mid a \in A, b \in A\}$ and $AB = \{a \cdot b \mid a \in A, b \in B\}$.
In that context we often simplify the notation and write $x$ instead of the singleton set $\{x\}$,
for example $x + \N y$ denotes the set $\{x\} + \N \{y\}$.
The \emph{description size} of an irreducible fraction $\frac{p}{q}$ is $\max(|p|, |q|)$, for $r \in \Q$
its description size is the description size of its irreducible form.

For a $d$-dimensional vector $x = (x_1, \ldots, x_d) \in \R^d$ and index $i \in [1,d]$ we write $x[i]$ for $x_i$.
For $S \subseteq [1,d]$ we write $\proj_S(x)$ to denote the $|S|$-dimensional vector
obtained from $x$ by removing all the coordinates outside $S$.
The \emph{norm} of a vector $x \in \N^d$ is $\norm(x) = \max_{i \in [1,d]} |x[i]|$.
For $i \in [1,d]$ the elementary vector $e_i$
is the unique vector such that $e_i[j] = 0$ for $j \neq i$ and $e_i[i] = 1$.
By $0^d \in \N^d$ we denote the $d$-dimensional vector with all the coordinates being zero.

\subparagraph*{Vector Addition Systems}
A $d$-dimensional Vector Addition System with States (shortly $d$-VASS or just a VASS) consists of finite set of \emph{states} $Q$
and finite set of \emph{transitions} $T \subseteq Q \times \Z^d \times Q$. 
\emph{Configuration} of a $d$-VASS $V = (Q, T)$ is a pair $(q, v) \in Q \times \N^d$, we often write $q(v)$ instead of $(q, v)$.
For a configuration $c = q(v)$ we write $\state(c) = q$.
For a set of vectors $S \subseteq \N^d$ and state $q \in Q$ we write $q(S) = \{q(v) \mid v \in S\}$.
Transition $t = (p, u, q)$ can be fired in configuration $(r, v) \in Q \times \N^d$ if $p = r$ and $u+v \in \N^d$.
Then we write $p(v) \trans{t} q(u+v)$. The triple $(p(u), t, q(u+v))$ is called an \emph{anchored transition}.
A \emph{run} is a sequence of anchored transitions $\rho = (c_1, u_1, c_2), \ldots, (c_n, u_n, c_{n+1})$.
Such a run $\rho$ is then a run \emph{from} configuration $c_1$ \emph{to} configuration $c_{n+1}$
and \emph{traverses} through configurations $c_i$ for $i \in [2,n]$.
We also say that $\rho$ is from $\state(c_1)$ to $\state(c_{n+1})$.
If there is a run from configuration $c$ to configuration $c'$ we also say that $c'$ is \emph{reachable} from $c$
or $c$ \emph{reaches} $c'$ and write $c \reaches c'$. Otherwise we write $c \nreaches c'$.
The configuration $c$ is the \emph{source} of $\rho$ while configuration $c'$ is the \emph{target} of $\rho$,
we write $\src(\rho) = c$ and $\trg(\rho) = c'$.
For a configuration $c \in Q \times \N^d$ we denote $\post_V(c) = \{c' \mid c \reaches c'\}$
the set of all the configurations reachable from $c$ and $\pre_V(c) = \{c' \mid c' \reaches c \}$
the set of all the configurations which reach $c$.
The \emph{reachability problem for VASSes} given a VASS $V$ and two its configurations $s$ and $t$
asks whether $s$ reaches $t$ in $V$.
For a VASS $V$ by its \emph{size} we denote the total number of bits needed to represent its states and transitions.
A VASS is said to be \emph{binary} if numbers in its transitions are encoded in binary.
Effect of a transition $(c, u, c') \in Q \times \Z^d \times Q$ is the vector $u \in \N^d$.
We extend this notion naturally to anchored transitions and runs, effect of the run $\rho = (c_1, u_1, c_2), \ldots, (c_n, u_n, c_{n+1})$
is equal to $u_1 + \ldots + u_n$.
Vector Addition Systems (VASes) are VASSes with just one state or in other words VASSes without states.
It is well known and simple to show that the reachability problems for VASes and for VASSes are polynomially interreducible.
In this work we focus wlog. on the reachability problem for VASSes.

\subparagraph*{Semilinear sets}
For any vectors $b, v_1, \ldots, v_k \in \N^d$ the set $L = b + \N v_1 + \ldots + \N v_k$ is called a \emph{linear} set.
Then vector $b$ is the \emph{base} of $L$ and vectors $v_1, \ldots, v_k$ are \emph{periods} of $L$.
Set of vectors is \emph{semilinear} if it is a finite union of linear sets.
Set of VASS configurations $S \subseteq Q \times \N^d$ is \emph{semilinear} if it is a finite union of sets of the
form $q_i(S_i) \subseteq Q \times \N^d$, where $S_i$ are semilinear as sets of vectors.
The \emph{size} of a representation of a linear set is the sum of norms of its base and periods.
The \emph{size} of a representation of a semilinear set $\bigcup_i L_i$ is the sum of sizes of representations of the sets $L_i$.
The size of a semilinear set is the size of its smallest representation.

For a $d$-VASS $V = (Q, T)$ and two of its configurations $s, t \in Q \times \N^d$ a set $S \subseteq Q \times \N^d$
of configurations is a \emph{separator for $(V, s, t)$} if it fulfils the following conditions:
1) $s \in S$, 2) $t \not\in S$, 3) $S$ is invariant under transitions of $V$, namely for any $c \in S$ such that $c \trans{t} c'$
for some $t \in T$ we also have $c' \in S$. In our work we usually do not exploit by condition 3) by itself,
but use the facts which are implied by all the conditions 1), 2) and 3) together: $\post(s) \subseteq S$ and $\pre(t) \cap S = \emptyset$.


\subparagraph*{Well quasi-order on runs}
We say that an order $(X, \preceq)$ is a well-quasi order (wqo) if in every infinite sequence $x_1, x_2, \ldots$ of elements of $X$
there is a \emph{domination}, i.e. there exist $i < j$ such that $x_i \preceq x_j$. 

Fix a $d$-VASS $V = (Q, T)$.
We define here a very useful order on runs, which turns out to be a wqo (a weaker version was originally introduced in~\cite{DBLP:journals/tcs/Jancar90}).
For two configurations $p(u), q(v) \in Q \times \N^d$ we write $p(u) \preceq q(v)$ if $p = q$ and $u[i] \leq v[i]$ for each $i \in [1,d]$.
For two anchored transitions in $(c_1, t, c_2), (c'_1, t', c'_2)$ we write $(c_1, t, c_2) \preceq (c'_1, t', c'_2)$
if $t = t'$, $c_1 \preceq c'_1$ and $c_2 \preceq c'_2$ (notice that the last condition is actually implied by the previous two).
For two runs $\rho = m_1 \ldots m_k$ and $\rho' = m'_1 \ldots m'_\ell$, where all $m_i$ for $i \in [1,k]$
and $m'_i$ for $i \in [1,\ell]$ are anchored transitions we write
$\rho \unlhd \rho'$ if and there exists a sequence of indices $i_1 < i_2 < \ldots < i_{k-1} < i_k = \ell$ such that
for each $j \in [1,k]$ we have $m_j \preceq m'_{i_j}$. Notice there that setting $i_k = \ell$ implies that $\trg(\rho) \preceq \trg(\rho')$.
All the subsequent considerations can be analogously applied in the case when we demand $\trg(\rho) = \trg(\rho')$
and $i_1 = 1$, but $i_k$ does not necessarily equal $\ell$, which enforces that $\src(\rho) \preceq \src(\rho')$.
The following claim is a folklore, for a proof see Proposition 19 in~\cite{DBLP:conf/stacs/ClementeCLP17}.

\begin{claim}
Order $\unlhd$ is a wqo on runs with the same source.
\end{claim}

The order $\unlhd$ has a nice property that runs bigger than a fixed one are additive in a certain sense.
The following claim is also a folklore, for a proof see Lemma 23 in~\cite{DBLP:journals/corr/ClementeCLP16}
(arXiv version of~\cite{DBLP:conf/stacs/ClementeCLP17}).

\begin{claim}\label{cl:adding-runs}
Let $\rho$, $\rho_1$ and $\rho_2$ be runs of VASS $V$ with $\src(\rho) = \src(\rho_1) = \src(\rho_2)$
such that $\rho \unlhd \rho_1, \rho_2$,
$\trg(\rho_1) = \trg(\rho) + \delta_1$ and $\trg(\rho_2) = \trg(\rho) + \delta_2$ for some $\delta_1, \delta_2 \in \N^d$.
Then there exist a run $\rho'$ such that $\src(\rho) = \src(\rho')$, $\rho \unlhd \rho'$
and $\trg(\rho') = \trg(\rho) + \delta_1 + \delta_2$.
\end{claim}

The following corollary can be easily shown by induction on $n$.

\begin{corollary}\label{corr:pumping}
Let $\rho \unlhd \rho'$ be runs of VASS $V$ such that
$\src(\rho') = \src(\rho)$ and $\trg(\rho') = \trg(\rho) + \delta$ for some $\delta \in \N^d$.
Then for any $n \in \N$ there exists a run $\rho_n$ of $V$ such that $\src(\rho_n) = \src(\rho)$
and $\trg(\rho_n) = \trg(\rho) + n \delta$.
\end{corollary}

The above notions will be useful in the proof of Theorem~\ref{thm:advanced}.
