%!TEX root = main.tex
\section{Missing proof}

\begin{proof}[Proof of Claim~\ref{cl:zero-run}]
We prove the claim by induction on $d$. For $d = 1$ clearly $u = v$ and there is nothing to show.
Let us assume that the claim holds for $d-1$, our aim is to prove it for $d$.
Adding a vector $y \in \zero(\lin)$ to $x_{i-1}$ in order to obtain $x_i = x_{i-1} + y \in \N^d$ we call a \emph{step}.
Clearly it is enough find a sequence of steps from $u$ to $v$.
The plan is to apply first such a sequence of steps from $u$ to some $u'$ such that $u'[i] = v[i]$ for some $i \in [1,d]$
and then show by induction assumption that a sequence of steps from $u'$ to $v$ exists as well.

For a subset of coordinates $I \subseteq [1,d]$ and $x \in \N^d$ by $\lin_I(x)$ we denote $\sum_{i \in I} n_i x[i]$.
As $\lin(v) \geq d \cdot M^3$ there exists some $j \in [1,d]$ such that $\lin_{[1,d] \setminus \{j\}}(v) \geq (d-1) M^3$.
Assume wlog. that $j = d$, so
\begin{equation}\label{eq:linv}
\lin_{[1,d-1]}(v) \geq (d-1) M^3.
\end{equation}
We first aim to reach $u''$ such that $u''[d] - v[d] \geq M^2$. Clearly until for some $i \neq d$
we have $u[i] \geq M$ we can apply the step $n_i e_d - n_d e_i$ to $u$ and increase value of $u[d]$. We continue this
until we reach some $u''$ with $\sum_{i=1}^{d-1} u''[i] < (d-1) M$.
This is indeed possible as $\sum_{i=1}^{d-1} u''[i] \geq (d-1) M$ implies that for some $i \neq d$ we have $u''[i] \geq M$. 
Then we have that $\lin_{[1,d-1]}(u'') < (d-1) M^2$ as $M = \max_{i \in [1,d]} n_i$, so $n_d \cdot u''[d] > dM^3 - (d-1)M^2$.
By~\eqref{eq:linv} we have $n_d \cdot v[d] \leq dM^3 - (d-1) M^3 = M^3$.
Therefore $n_d (u''[d] - v[d]) > (d-1) (M^3 - M^2)$ and thus $u''[d] - v[d] \geq (d-1) (M^2 - M)$.
By Claim~\ref{cl:gcd} we have that $u''[d] - v[d] = \sum_{i=1}^{d-1} n_i b_i$ for some $b_i \in \N$
(notice that here we use the fact that $\lin$ is reduced and therefore $\gcd(n_1, \ldots, n_{d-1}) = 1$
and divides $u''[d] - v[d]$). Therefore in order to obtain $u'$ such that $u'[d] = v[d]$ we apply for each $i \in [1,d-1]$
exactly $b_i$ number of times the step $n_d e_i - n_i e_d$ to $u''$.
Notice that all the other coordinates beside the $d$-th one increase, so these steps lead
to vectors with nonnegative coordinates. As $\lin_{i \in [1,d-1]}(u') = \lin_{i \in [1,d-1]}(v) \geq (d-1) M^3$ we apply
the induction assumption to show that indeed starting from $u'$ one can reach $v$ by a sequence of steps. This finishes the proof.
\end{proof}



\begin{proof}[Proof of Claim~\ref{cl:ratio2}]
Similarly as in Claim~\ref{cl:ratio} we intuitively mean to show that for all $p \in P$ we have
$\frac{\lin_1(\Delta)}{\lin_2(\Delta)} \leq \frac{\lin_1(p)}{\lin_2(p)}$, but this is not a formally correct
statement as it may be that $\lin_2(p) = 0$.
Assume towards a contradiction that there is a period $p$ such
that $\lin_1(\Delta) \cdot \lin_2(p) > \lin_2(\Delta) \cdot \lin_1(p)$.
Clearly $a \in L$, therefore also $a + mp \in L$ for any $m \in \N$.
We aim at showing that $a + mp \reaches a + n\Delta + e_i$ in $V'$ for some $m, n \in \N$.
This would lead to a contradiction as we know that $a + n\Delta \reaches t$, so also
$a + n\Delta + e_i \reaches t+e_i$. Therefore we would have that $a + mp \in L$ and
also $a + mp \reaches a + n\Delta + e_i \reaches t + e_i$, which is a contradiction
with the definition of the separator.

Let $\lin_2(x_1, \ldots, x_d) = \sum_{i=1}^d n_{i,2} x_i$ and let $M$ be the maximal coefficient in $\lin_2$,
namely $M = \max_{i \in [1,d]} n_{i,2}$. We set $m = dM^3 \cdot \lin_1(\Delta)$ and $n = dM^3 \cdot \lin_1(p)$.
Observe now that $\lin_1(a + mp) = \lin_1(a + n\Delta + e_i)$.
Indeed
\begin{align*}
\lin_1(a + mp) & = \lin_1(v) + m \cdot \lin_1(p) = \lin_1(v) + dM^3 \cdot \lin_1(\Delta) \cdot \lin_1(p) \\
& = \lin_1(v) + n \cdot \lin_1(\Delta) = \lin_1(a + n\Delta + e_i),
\end{align*}
as $i \not\in \supp(\lin_1)$, so $\lin_1(e_i) = 0$.
Let $p = p_1 + p_2 + p_{\trash}$ and $\Delta = \Delta_1 + \Delta_2 + \Delta_{\trash}$,
where $p_1, \Delta_1$ are positive only on $\supp(\lin_1)$,
$p_2, \Delta_2$ are positive only on $\supp(\lin_2)$
and $p_\trash, \Delta_\trash$ are positive only outside $\supp(\lin_1) \, \cup \, \supp(\lin_2)$.
First notice that thanks to transitions in $V'$, which can freely modify coordinates outside $\supp(\lin_1) \, \cup \, \supp(\lin_2)$
we can assume wlog. that $p_\trash = \Delta_\trash = 0$.
Therefore by Claim~\ref{cl:zero-run} and because $\lin_1$ is reduced there is a run in $V'$ from $a + mp = a + mp_1 + mp_2$
to $a + n \Delta_1 + mp_2$, as $\lin_1(a + mp_1) = \lin_1(a + n \Delta_1) \geq dM^3 \geq |\supp(\lin_1)| \cdot M^3$.
We claim now that
\begin{equation}\label{eq:needed}
\lin_2(mp_2) = \lin_2(n \Delta_2 + e_i + v_\trash) \geq dM^3 \geq |\supp(\lin_2)| \cdot M^3
\end{equation}
for some $v_\trash$ positive only on $\supp(\lin_2)$.
This would finalise the argument as then
\[
a + n \Delta_1 + mp_2 \reaches a + n \Delta_1 + n \Delta_2 + e_i + v_\trash \reaches a + n \Delta_1 + n \Delta_2 + e_i = a + n \Delta + e_i,
\]
where the first relation $\reaches$ follows from Claim~\ref{cl:zero-run} and fact that $\lin_2$ is reduced
and the second one follows from existence of transitions in $V'$ which arbitrarily decrease any coordinate in $\supp(\lin_2)$.

Recall first that $\lin_1(\Delta) \cdot \lin_2(p) > \lin_2(\Delta) \cdot \lin_1(p)$, so 
\[
\lin_1(\Delta) \cdot \lin_2(p) - \lin_2(\Delta) \cdot \lin_1(p) \geq 1
\]
and that we set $m = dM^3 \cdot \lin_1(\Delta)$ and $n = dM^3 \cdot \lin_1(p)$.
Therefore
\[
\lin_2(m p_2) - \lin_2(n \Delta_2) = dM^3 \cdot \lin_1(\Delta) \cdot \lin_2(p) - dM^3 \cdot \lin_2(\Delta) \cdot \lin_1(p) \geq dM^3.
\]
So $\lin_2(m p_2) - \lin_2(n \Delta_2 + e_i) \geq dM^3 - M \geq d(M^2 - M)$. Thus by Claim~\ref{cl:gcd}
there exist some $b_1, \ldots, b_d \in \N$ such that $\sum_{j=1}^d n_{j,2} b_j = \lin_2(m p_2) - \lin_2(n \Delta_2 + e_i)$.
Hence $\lin_2(m p_2) = \lin_2(n \Delta_2 + e_i + \sum_{j=1}^n b_j e_j) \geq dM^3$
and we can set $v_\trash =  \sum_{j=1}^n b_j e_j$ thus satisfying~\eqref{eq:needed}
and finishing the proof of the claim.
\end{proof}