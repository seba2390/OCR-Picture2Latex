%!TEX root = main.tex
\section{Introduction}
The complexity of the reachability problem for Vector Addition Systems (VASes) was a challenging and natural problem
for a few decades of research. The first result about its complexity
was \expspace-hardness by Lipton in 1976~\cite{Lipton76}.
Decidability was shown by Mayr in 1981~\cite{DBLP:conf/stoc/Mayr81}.
Later the construction of Mayr was further simplified and presented in a bit different light
by Kosaraju~\cite{DBLP:conf/stoc/Kosaraju82} and Lambert~\cite{DBLP:journals/tcs/Lambert92}.
This construction is currently known under the name KLM decomposition after the three main inventors.
The first complexity upper bound was obtained by Leroux and Schmitz as recent as in 2016~\cite{DBLP:conf/lics/LerouxS15},
where the reachability problem was shown to be solvable in cubic-Ackermann time. A few years later
the same authors have shown that the problem can be solved in Ackermann time and actually in primitive
recursive time, when the dimension is fixed~\cite{DBLP:conf/lics/LerouxS19}.
At the same time a lower bound of \tower-hardness was established for the reachability
problem~\cite{DBLP:conf/stoc/CzerwinskiLLLM19,DBLP:journals/jacm/CzerwinskiLLLM21}. Last year two independent
papers have shown \ackermann-hardness of the reachability
problem~\cite{DBLP:conf/focs/Leroux21, DBLP:conf/focs/CzerwinskiO21}
thus settling the complexity of the problem to be \ackermann-complete.

However, despite this recent huge progress we still lack much understanding about the
reachability problem in VASSes. The most striking example is the problem for dimension three.
The best known lower bound for the reachability problem for 3-VASSes
is \pspace-hardness inherited from the hardness result for 2-VASSes~\cite{DBLP:conf/lics/BlondinFGHM15},
while the best known upper bound is much bigger then \tower. Concretely speaking the problem
can be solved in time $\F_7$~\cite{DBLP:conf/lics/LerouxS19}, where $\F_\alpha$ is the hierarchy
of fast-growing complexity classes, see~\cite{DBLP:journals/toct/Schmitz16} (recall that $\tower = \F_3$).
Similarly for other low dimensional VASSes the situation is unclear.
Currently the smallest dimension $d$ for which an \expspace-hardness
is known for $d$-VASSes is $d = 6$ and for which a \tower-hardness is known is $d = 8$~\cite{DBLP:journals/corr/abs-2203-04243}.
Therefore for any $d \in [3,5]$ the problem can be in \pspace and is not known to be elementary
and for $d \in [6,7]$ the problem can be elementary.
For some of those dimensions one can hope to get better hardness results in the future,
but we conjecture that for $3$-VASSes and $4$-VASSes the reachability problem actually is elementary
and the challenge is to find this algorithm.
This complexity gap is just one witness of our lack of understanding of the structure of VASSes.
One can hope that in the future we will be able to design efficient algorithms for the reachability problem
even for VASSes in high dimensions under the condition that they belong to some subclass, for example
they avoid some hard patterns. This can be quite important from a practical point of view. Understanding
for which classes efficient (say \pspace) algorithms are possible may be thus another future goal.
Therefore we think that the quest for better understanding the reachability problem is still valid,
even though the complexity of the reachability problem for general VASSes is settled.

A common and very often successful approach to the reachability problem is proving a short run property,
namely showing that in order to decide reachability it suffices to inspect only runs of some bounded length.
This technique was exploited a lot in the area of VASSes.
Rackoff proved his \expspace upper bound on the complexity of the coverability problem~\cite{DBLP:journals/tcs/Rackoff78}
using this approach: he has shown that if there exists a covering run, then there exists also a covering run
of at most doubly-exponential length.
Recently there was a lot of research about low dimensional VASSes and the technique of bounded run length was also
used there. In~\cite{DBLP:conf/lics/BlondinFGHM15} authors established complexity of the reachability problem
for 2-VASSes to \pspace-completeness by proving that in order to decide reachability it suffices to inspect runs
of at most exponential length. Similarly in the next paper in this line of research~\cite{DBLP:conf/lics/EnglertLT16}
it was shown that in 2-VASSes with unary transition representation it suffices to consider runs of polynomial length,
which proves \nl-completeness of the reachability problem.

This approach however meets a subtle obstacle when one tries to prove some upper bound on the complexity
of the reachability problem in say $d$-VASSes.
In order to show a bounded length property it is natural to try to unpump long runs in some way. Unpumping however
can be very tricky when the run is close to some of the axes, as only a small modification of the run may cause some counters to
become negative. A common approach to that problem is to modify a run when all its counters are high and therefore local run
modifications cannot cause any problem. Such an approach is used for example in 2-VASSes~\cite{DBLP:conf/lics/BlondinFGHM15}
and in the KLM decomposition~\cite{DBLP:conf/stoc/Kosaraju82, DBLP:journals/tcs/Lambert92, DBLP:conf/stoc/Mayr81}.
However, not all the runs may have a configuration with all counter values high. Therefore it is very convenient to have
a cycle, which increases all the counters simultaneously. Observe that existence of such a cycle implies that
the set of reachable configurations is infinite. Thus using this approach is challenging in the case when the reachability set is finite.
For finite set in turn it is hard to design algorithm substantially faster than the size of the reachability set.
It is well known that in $d$-VASSes finite reachability sets can be as large as $F_{d-1}(n)$, where
$n$ is the VASS size and $F_\alpha$ is the hierarchy of very fast growing functions, see~\cite{DBLP:journals/toct/Schmitz16}.
Therefore designing an algorithm breaking $F_{d-1}(n)$ time for the reachability problem in $d$-VASSes
(if such exist) may need some other approach. Breaking the barrier of the size of finite reachability set is possible in general,
but probably in many cases very challenging. To the best of our knowledge the only nontrivial algorithm breaking it 
is the one in~\cite{DBLP:conf/lics/EnglertLT16} doing a sophisticated analysis of possible behaviours of runs in 2-VASSes.
This motivates a search for other techniques, which may be more suitable for designing fast algorithms.

Leroux in his work~\cite{DBLP:conf/lics/Leroux09} provided an algorithm for the reachability problem,
which follows a completely different direction. He has shown that if there is no run from a configuration $s \in \N^d$
to a configuration $t \in \N^d$ in a VAS $V$ then there exists a semilinear set $S$, called a \emph{separator}, which
(1) contains $s$, (2) does not contain $t$ and (3) is an \emph{inductive invariant}, namely if $v \in S$ then also $v+t \in S$
for any transition $t$ of $V$ such that $v+t \in \N^d$. Notice that existence of a separator clearly implies non-reachability between
$s$ and $t$, so by Leroux's work non-reachability and existence of separator are equivalent. Then the following simple algorithm
decides whether $t$ is reachable from $s$: run two semi-procedures, one looks for possible runs between $s$ and $t$, longer and longer,
another one looks for possible separators between $s$ and $t$, bigger and bigger. Clearly either there exists a run or there exists
a separator, so at some point the algorithm will find it and terminate. From this perspective one can view runs and separators as dual
objects. If bounding the length of a run is nontrivial, maybe bounding the size of a separator can be another promising approach.
Notice that proving for example an $F_d(n)$ upper bound on the size of separators would provide an algorithm solving
the reachability problem which works in time at most exponential $F_d(n)$, which is still $F_d(\Oo(n))$ for $d \geq 3$.

\subparagraph*{Our contribution}
Our main contribution are two theorems stating that in VASSes fulfilling certain rather natural conditions if there are only long runs between
some of its two configurations then in a small modification of these VASSes there are only big separators for some other two configurations.
We designed Theorem~\ref{thm:simple} to have relatively simple statement, but also to be sufficiently strong for our applications.
Theorem~\ref{thm:advanced} needs more sophisticated notions and more advanced tools to be proven, but it has potentially a broader
spectrum of applications. 

Additionally we have shown that two nontrivial constructions of involved VASSes,
namely the 4-VASS from~\cite{DBLP:conf/concur/Czerwinski0LLM20} and VASS used in the \tower-hardness
construction from~\cite{DBLP:conf/stoc/CzerwinskiLLLM19} fulfil the conditions proposed by us in Theorem~\ref{thm:simple}.
This indicates that for each VASS subclass $\F$ (for example $3$-VASSes) either (1) in order to prove better upper complexity bound
for the reachability problem in $\F$ one should focus more on proving the short run property than on proving the small separator property or
(2) proving small separator property is somehow possible for $\F$.
However, in the latter case the mentioned VASS has to be constructed by the use of rather different techniques than currently known, as it needs to violate conditions of Theorems~\ref{thm:simple}~and~\ref{thm:advanced}.

We have not considered VASSes occurring in the most recent papers
proving the \ackermann-hardness~\cite{DBLP:conf/focs/Leroux21,DBLP:conf/focs/CzerwinskiO21},
but it seems to us that these techniques are promising
at least with respect to VASSes occurring in~\cite{DBLP:conf/focs/CzerwinskiO21}.

\subparagraph*{Organisation of the paper}
In Section~\ref{sec:prelim} we introduce preliminary notions and recall standard facts. Then in Section~\ref{sec:simple}
we state and prove our first main result, Theorem~\ref{thm:simple}.
In Section~\ref{sec:applications} we provide two applications of Theorem~\ref{thm:simple},
in Section~\ref{ssec:4vass} to the $4$-VASS described in~\cite{DBLP:conf/concur/Czerwinski0LLM20}
and in Section~\ref{ssec:tower} to VASSes occurring in the paper~\cite{DBLP:conf/stoc/CzerwinskiLLLM19} proving the \tower-hardness
of the reachability problem.
Finally, in Section~\ref{sec:advanced} we prove our second main result, Theorem~\ref{thm:advanced}.




