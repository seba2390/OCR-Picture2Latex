\documentclass[11pt]{amsart}
\usepackage{preamble}
\usepackage{hyperref}
%\usepackage{cleverref}


\begin{document}


\title[A greedy chip-firing game]{A greedy chip-firing game}
\author[Rupert Li]{Rupert Li}
\address{Massachusetts Institute of Technology}
\email{\href{mailto:rupertli@mit.edu}{{\tt rupertli@mit.edu}}}
\author[James Propp]{James Propp}
\address{University of Massachusetts Lowell, Department of Mathematical Sciences}
\email{\href{mailto:jamespropp@gmail.com}{{\tt jamespropp@gmail.com}}}
\date{\today}
\keywords{chip-firing, recurrence, stationary distribution}

\begin{abstract}
We introduce a deterministic analogue of Markov chains 
that we call the hunger game.  
Like rotor-routing, the hunger game
deterministically mimics the behavior of both recurrent Markov chains 
and absorbing Markov chains.
In the case of recurrent Markov chains with finitely many states, 
hunger game simulation concentrates around the stationary distribution
with discrepancy falling off like $N^{-1}$,
where $N$ is the number of simulation steps;
in the case of absorbing Markov chains with finitely many states, 
hunger game simulation also exhibits concentration 
for hitting measures and expected hitting times
with discrepancy falling off like $N^{-1}$
rather than $N^{-1/2}$.
When transition probabilities in a finite Markov chain are rational,
the game is eventually periodic;
the period seems to be the same for all initial configurations
and the basin of attraction appears to tile the configuration space 
(the set of hunger vectors) by translation,
but we have not proved this.
\end{abstract}

\maketitle

\import{}{introduction}
\import{}{preliminaries}
\import{}{boundedness}
\import{}{termination}
\import{}{stationary}
\import{}{hitting}
\import{}{recurrence}
\import{}{conclusion}

\section*{Acknowledgments}
The authors thank Grant Barkley, Darij Grinberg,
Sam Hopkins, Lionel Levine, Alex Postnikov, and Tom Roby
for various helpful suggestions made during the course of this research.
We are grateful to the anonymous referees who made many suggestions that improved this article.

\bibliographystyle{plain}
\bibliography{ref}

\end{document}
