\documentclass[12pt]{amsart}
\usepackage{preamble}

\begin{document}


\title[Response to Reviewers' Comments]{Response to Reviewers' Comments for 
``A greedy chip-firing game"}
\author[Rupert Li and James Propp]{Rupert Li and James Propp}

\date{\today}

\maketitle

Thank you for the helpful comments and feedback.
The vast majority of the suggested edits have been made, and for the cases 
where we do not exactly follow the feedback or we have other notes to make, 
we provide a brief commentary below.
We also made other small corrections when we found them (most notably being
standardizing our matrix multiplication of measures 
with transition matrices as $vP$).

\section{Report A}
All suggestions were implemented.

In response to the second remark, additional clarification that the chip 
addition operators $E_i$ are well-defined assuming a finite absorbing Markov 
chain was added to the beginning paragraph of Section 4.

The third remark points out an inconsistency in our definition of the rerouted
Markov chain.  We originally assumed $v \not\in U$, so while the rerouted Markov
chain can be defined for absorbing state $v \in U$, as observed in the referee
report it is a trivial chain with one state, so this case distracts the main 
focus of Section 6.  As a result, Figure 9(c) was removed from the paper for 
sake of clarity and simplicity.

\section{Report B}
All suggestions were implemented, with slight modifications listed below.

In response to the feedback regarding the first paragraph of page 3, part of the Preliminaries section has been restructured to more explicitly and thoroughly discuss the various deterministic analogues: chip-firing, rotor-routing, and the hunger game.
The original first paragraph of page 3 has essentially been expanded into what is now the entirety of page 3, as well as the first third of page 4, including examples of all three analogues being applied to the same example Markov chain.

Various edits have been made to Remark 3.3 on page 8.
In particular, the argument has been simplified to not involve corank.

Regarding Remark 4.3, it has been decided that the original reference [11] 
will not be published, and thus all mention of it in this paper has been removed.

Combined with the comments from report A, we cleaned up the section on hitting
probabilities to be consistent in our assumption that $v\not\in U$.

Additional clarification was added to the definition of the rerouted Markov 
chain.

A quick example was added to the introduction of absorption times.

The stationary probabilities were provided in Example 7.1, and additional 
clarification was provided for Figure 10.
In particular, this clarification references a specific point in a 
violet region, hopefully resolving any misunderstanding about what 
the axes mean, and which color is violet.
Figure 10 was also updated to have larger tick labels for sake of readability.

We think our revision of Section 2 addresses the second referee’s concerns about Section 8. However, we are open to making further changes if the referee feels this would be helpful.

\end{document}
