\section{Recurrent states and the basin of attraction}\label{section: recurrence}

In this section we restrict our attention to irreducible finite Markov chains
with no absorbing state.
Additionally we assume that all transition probabilities are rational.

% For an initial hunger vector $\h$ 
% on a finite rational Markov chain, 
% \cref{lemma: hunger bounded} tells us that hunger stays uniformly bounded, 
% so for each state $i$, the set of values $\{\h^{(t)}_i \mid t \geq 1\}$ 
% is bounded on both sides.
% If the least common denominator of the transition probabilities is $d$, 
% then for any $t \geq 1$, we have $\h^{(t)}_i$ must differ from $\h^{(0)}_i$ 
% by a multiple of $\frac{1}{d}$, i.e., $d(\h^{(t)}_i-\h^{(0)}_i) \in \Z$.
% This implies $\{\h^{(t)}_i \mid t \geq 1\}$ is a finite set, for each $i$.
% Hence $\{\h^{(t)} \mid t \geq 1\}$ is also a finite set, 
From the argument in \cref{remark: finite orbit}, 
all hunger vectors $\h\in\R^n$ are pre-periodic (i.e.,
eventually enter a cycle).
We say that a vector that is part a cycle
(i.e., returns to itself after a finite number of steps) is \textbf{recurrent},
and we define the \textbf{basin of attraction} to be 
the set of recurrent vectors on the hyperplane 
$Z=\{\h\mid \h\cdot \mathbf{1} = 0\}$ 
(the hyperplane of hunger vectors with total hunger 0).
The basin of attraction, being the set of recurrent vectors, 
is thus in some sense analogous to the critical group $\Z^n/\Z^n\Delta$ 
in the context of chip-firing.

\begin{example}\label{example: basin of attraction 1}
Consider the finite rational irreducible Markov chain with hunger matrix
\[ H = \begin{bmatrix} -0.2 & 0.2 & 0 \\ 0.2 & -0.6 & 0.4 \\ 0.6 & 0.4 & -1 \end{bmatrix},\]
whose unique stationary distribution is $\left(\frac{11}{18},\frac{5}{18},\frac{2}{18}\right)$.
Its basin of attraction is given in \cref{fig: basin of attraction 1}.
The basin is partitioned into four colors based on the firing order, up to cyclic shift.
Red corresponds to the cyclic firing order 231121121311211211, i.e., firing state 2, then state 3, then state 1, and so on.
Green corresponds to 312112112311211211, cyan corresponds to 231112112311211211, and violet corresponds to 312112112311121121.
As the visit vector for each of these 18-step cycles must satisfy $\v H = \mathbf{0}$, i.e., be a multiple of the stationary distribution, we find each cyclic firing order consists of state 1 firing eleven times, state 2 firing five times, and state 3 firing two times.
\begin{figure}[htb]
    \centering
    \includegraphics[width=0.5\textwidth]{basin_partition_test1.png}
    \caption{The basin of attraction for a finite rational irreducible Markov chain with three states.
    The basin lies in the hyperplane $x + y + z = 0$, so the third coordinate is omitted in order to plot it on the $x,y$-plane.
    For example, the point $(0.8,-0.1)$ in the rightmost violet region corresponds to $\h = (0.8,-0.1,-0.7)$.
    The basin is partitioned into four colors based on the order of firings; for example, the vectors in the violet portions of the basin all follow the same sequence of firings, up to cyclic shift, i.e., choosing a starting index for the periodic firing sequence.
    So, the four colors correspond to four distinct cyclic firing orders.}
    \label{fig: basin of attraction 1}
\end{figure}
\end{example}

\begin{remark}\label{remark: convex partition basin}
Numerous observations can be made from \cref{example: basin of attraction 1} 
and \cref{fig: basin of attraction 1}.
Firstly, the partitioning based on cyclic firing order appears to yield 
discrete congruent pieces; in \cref{fig: basin of attraction 1}, 
the red and violet colors are split into congruent parallelograms, 
and the cyan and green colors are each split into congruent triangles.
Each color corresponds to a cyclic firing order, and if this cycle has period $p$, 
then the set of vectors of this particular color 
can be partitioned into $p$ sets based on the firing order of their cycle, 
no longer up to cyclic shift.
Each of these sub-pieces are mapped to each other in a cycle 
under the hunger game process, 
so clearly the sub-pieces of a given color are all congruent.

While evidently the entire basin of attraction need not be convex, 
each sub-piece of the basin of attraction, e.g., 
the parallelograms and triangles of \cref{fig: basin of attraction 1}, is convex; 
this holds for all finite rational Markov chains.
To see why this is the case, notice that each sub-piece 
corresponds to a unique firing order that returns a hunger vector to itself.
The set of vectors that will fire to a given state next under the hunger game process 
is given by the intersection of various linear inequalities, 
namely the inequalities that ensure this state had the highest hunger.
Hence, the set of vectors that follow the firing order corresponding to a given sub-piece 
is given by the intersection of various linear inequalities, 
one set of inequalities for each firing step of the hunger game process.
As the intersection of linear inequalities is convex, we find each sub-piece is convex.
\end{remark}

Intuitively, the basin of attraction should be near the origin, 
for the hunger game naturally attempts to equilibrate the hunger vector 
so that each state has approximately the same hunger.
The following result supports this intuition by demonstrating that the origin 
is always in the basin of attraction for any finite rational irreducible Markov chain.

\begin{proposition}\label{proposition: 0 hunger recurrent}
For any finite rational irreducible Markov chain, 
the zero vector $\mathbf{0}$ is a recurrent hunger state.
Moreover, the number of steps needed to return to $\mathbf{0}$ 
is the least common denominator of the stationary probabilities of the Markov chain.
\end{proposition}
\begin{proof}
Let $d$ be the least common denominator 
of the transition probabilities of the Markov chain.
Viewing the Markov chain as a weighted digraph $G$
with each vertex having outgoing weights summing to 1,
we can multiply all weights by $d$ so that all weights are now positive integers, 
and convert weighted edges into multiedges to obtain a directed multigraph $G'$
in which every vertex has outdegree $d$.

A finite irreducible Markov chain with rational transition probabilities
has a unique stationary distribution $\ppi$, all of whose entries are rational.
Let $n$ be the least common denominator of $\ppi$, so that $n\ppi$
is the unique multiple of $\ppi$ 
that has nonnegative, mutually coprime integer entries, and let $\p = n\ppi$.
Since the sum of the entries of $\ppi$ is 1, the sum of the entries of $\p$ is $n$. 
As $\ppi H=\mathbf{0}$, we have $\p H=\mathbf{0}$.

Starting at $\mathbf{0}$, consider the first vertex $v$ that fires more than $\p_v$ times, 
and consider the hunger state just before this vertex fires for the $(\p_v+1)$-th time.
Let the firing vector at this step be $\x$, so that $\x_u$ is the number of times 
state $u$ received the chip, and thus $\x H$ is the current hunger state.
Because $\x_v = \p_v$ and 
because for all $u \neq v$ state $u$ has fired at most $\p_u$ times, 
$\w := \p - \x$ satisfies $\w_v = 0$ and $\w_u \geq 0$ for all $u \neq v$.
Firing vertices $u\neq v$ can only (weakly) increase the hunger of $v$, 
so $(\w H)_v \geq 0$.
To prove that last assertion more formally, recall that $\w_v = 0$, 
so we can write $\w=\sum_{u \neq v} c_u \e^{(u)}$ with $c_u \geq 0$ for all $u$,
where $\e^{(u)}$ is the elementary basis vector with a 1 at index $u$ and 0 otherwise.
Then we have
\begin{align*}
    (\w H)_v
    &= \sum_{u \neq v} c_u (\e^{(u)} H)_v
    = \sum_{u \neq v} c_u H_{uv}
    = \sum_{u \neq v} c_u P_{uv}
    \geq 0,
\end{align*}
as claimed. This yields that 
\begin{align*}
    (\x H)_v
    % &= ((\p - \w)H)_v
    = (\p H - \w H)_v
    = -(\w H)_v
    \leq 0.
\end{align*}
Yet with a hunger state of $\x H$, vertex $v$ received the chip, 
meaning it has the highest hunger.
This means every state has hunger at most $(\x H)_v \leq 0$, so total hunger is at most 0.
However, total hunger is invariant under the hunger game process;
since the initial hunger vector was $\mathbf{0}$ with total hunger 0, 
total hunger must still be equal to 0.
This equality case requires that $\x H = \mathbf{0}$, 
implying $\x$ is a multiple of $\ppi$. 
Since $\x_v = \p_v$ and $\p = n\ppi$, we deduce $\x = \p$.

Thus the hunger state is $\x H=\mathbf{0}$, so $\mathbf{0}$ is recurrent.
Moreover, the number of steps needed to return to $\mathbf{0}$ 
is the sum of the entries of $\x=\p=n\ppi$, which is $n$, as claimed.
\end{proof}

From empirical observations, we conjecture that 
the periods of all cycles in the hunger game for a given chain are equal, 
and specifically are equal to the $n$ 
that was shown in \cref{proposition: 0 hunger recurrent}
to be the period of the hunger vector $\mathbf{0}$.

\begin{conjecture}\label{conjecture: period stationary lcd}
The period of every cycle in the hunger game 
for a given finite rational irreducible Markov chain 
is the least common denominator of the stationary probabilities of the Markov chain.
\end{conjecture}

Visually, \cref{conjecture: period stationary lcd} 
applied to the Markov chain from \cref{example: basin of attraction 1} 
corresponds to the observation that each color in \cref{fig: basin of attraction 1} 
consists of the same number of congruent pieces, namely 18.

The basin of attraction in \cref{fig: basin of attraction 1} tiles the hyperplane $Z$, as illustrated in \cref{fig: basin tiling}; 
in particular, one can observe that its concave boundaries fit complementarily 
with the opposite side of the basin of attraction.
The following conjecture formalizes this observation 
and poses it for general Markov chains.
\begin{figure}[htbp]
    \centering
    \includegraphics[width=0.6\textwidth]{example_tiling.png}
    \caption{The basin of attraction from \cref{example: basin of attraction 1} tiles the (hyper)plane $Z$.}
    \label{fig: basin tiling}
\end{figure}

\begin{conjecture}\label{conjecture: basin tiles hyperplane}
For a finite rational irreducible Markov chain, 
the basin of attraction tiles the hyperplane $Z$ by translation, 
and the translation vectors that relate each tile with each other 
forms an $(n-1)$-dimensional sublattice in $Z$ of $\R^n$ 
with basis vectors $H_1-H_2,H_2-H_3,\dots,H_{n-1}-H_n$.
\end{conjecture}

In fact, \cref{conjecture: basin tiles hyperplane} 
implies \cref{conjecture: period stationary lcd}, as seen in the following proposition.

\begin{proposition}\label{proposition: tiling conjecture implies periodicity conjecture}
For a finite rational irreducible Markov chain, 
if the basin of attraction tiles the hyperplane $Z$ 
with lattice of translation vectors generated by $H_1-H_2,\dots,H_{n-1}-H_n$, 
then the period of any cycle under the hunger game process is 
the least common denominator of the stationary probabilities of the Markov chain.
\end{proposition}
\begin{proof}
By using \cref{proposition: 0 hunger recurrent}, it suffices to show that 
the tiling condition implies that all periods in the basin of attraction are the same.
Suppose the lattice in the statement of the proposition is $L$.
We say that two points in $Z$ are equivalent mod $L$ if they differ by 
an element of $L$, and we write the set of equivalence classes as $Z/L$.
Now view the hunger game process, which naturally acts upon $Z$, 
as acting instead on $Z/L$, where the tiling condition implies that 
the basin of attraction can serve as a set of coset representatives for $Z/L$.
Initially this might seem like nonsense since the choice of which state to fire
depends on inequalities relating the elements of the hunger vector 
and these inequalities get washed out when we mod out by $L$,
but all rows of $H$ are equivalent mod $L$, 
i.e., each row of $H$ has the same image under the projection map $Z\to Z/L$,
so the choice of which state gets fired is moot;
we may as well suppose that the $H_1$ coset
is added at each stage, as if state 1 were firing repeatedly.
Using the basin of attraction as a set of coset representatives 
and observing that a vector inside the basin 
must stay within the basin under the hunger game process, 
we find the hunger game process on $Z/L$ is equivalent to 
the original hunger game on $Z$ for any vector in the basin.
If the zero vector $\0$ has period $p$, then $pH_1=0$ in $Z/L$.
Then let $q$ be the minimum of the set of positive integers $q'$ 
such that $q'H_1=0$ in $Z/L$.
After $q$ steps of the hunger game process, 
every vector in the basin thus returns to itself, 
so we find $p=q$ and every vector in the basin of attraction must have period $p$, 
which completes the proof.
\end{proof}

We provide a partial result towards \cref{conjecture: basin tiles hyperplane}, 
proving that the basin of attraction translated under the stated lattice 
covers the hyperplane $Z$; 
to prove \cref{conjecture: basin tiles hyperplane} 
and thus \cref{conjecture: period stationary lcd}, 
it remains to show this covering has no overlap.

\begin{proposition}\label{proposition: basin covers hyperplane}
For a finite rational irreducible Markov chain, 
let $L\subset Z$ be the sublattice in $Z$ of $\R^n$ 
with basis vectors $H_1-H_2,\dots,H_{n-1}-H_n$.
Then for any vector $\h\in Z$, there exists a $\u\in L$ 
such that $\h-\u$ lies in the basin of attraction.
\end{proposition}
\begin{proof}
As $\h$ is pre-periodic, after some finite number of steps 
it will reach some vector in the basin, say after $t_0$ steps.
Additionally, every vector in the basin 
stays in the basin under the hunger game process, 
so for all integers $t \geq t_0$, 
applying the hunger game process $t$ times to $\h$ yields a vector in the basin.
Let $p$ be the least common denominator of the stationary probabilities 
of the unique stationary distribution $\ppi$ of our Markov chain, 
and fix $t$ to be the minimum nonnegative integer $\geq t_0$ such that 
$t$ is a multiple of $p$, say $kp$ for positive integer $k$.
Then applying the hunger game process $t$ times to $\h$ 
yields a vector $\x \in Z$ in the basin of attraction, 
say after firing state $i$ a total of $\v_i$ times for each $i$.
Constructing vector $\v$ from these values, 
as $\x$ was reached after firing $\h$ a total of $t$ times, 
we have $\x=\h+\v H$ and $\v\cdot\1=t=kp$.
Notice that $p\ppi$ is the primitive integer vector 
in the direction of the stationary distribution, so we have $p\ppi H= \0$.
Hence $\x = \h + (\v - kp\ppi)H$, where $(\v-kp\ppi)\cdot \1 = 0$.
As $L=\{\w H\mid\w\cdot\1=0\}$, 
letting $\u=(kp\ppi-\v)H\in L$ yields $\x = \h - \u$, as desired.
\end{proof}
