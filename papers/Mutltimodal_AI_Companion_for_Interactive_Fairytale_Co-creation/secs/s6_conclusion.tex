\section{Conclusion}

In this paper, we presented an intelligent visualization interface mainly focusing on fairy tales. Fairy tales, different from casual languages, have unique characters and relationships, which would generally lead direct text-to-image visualization methods to fail. Based on this point, we introduce the doodler graph as an intermediate representation. We fuse the rule-based language model with a learning-based generation model into a unified visualization system to allow for more flexibility during generation. With our fused framework, AI.R Taletorium is able to visualize complex fairy tale scenes with stylized characters and rare relationships and automatically update the visualization along with story generation. 

However, the visualization quality is still constraint by Sketch-RNN generation quality, which could generates unrecognizable characters and making the visualization hard to understand. Re-training the sketcher with Transformer based architecture could help in improving generation quality. In future work, we will re-design the RNN sketcher to improve the doodling quality and further improve the generation efficiency to support real-time mobile inference.