
\section{AI.R TALETORIUM SYSTEM}

\begin{figure*}[t]
\begin{center}
   \includegraphics[width=1.0\linewidth]{figs/pnw_net.pdf}
\end{center}
\caption{Character-centric story generation. We combine plan-and-write\cite{Yao} story generation framework with character features to compose each story fragment. We apply a dynamic generation strategy and plan for each fragment on the fly based on the previous story fragment and current character status. Such a character-centric generation scheme thus supports user co-creation with AI in storytelling by modifying characters.}
\label{fig:pnw_net}
\end{figure*}

AI.R Taletorium is a character-centric multi-modal AI fairy tale telling system aims at connecting users with different background and physical needs into a unified fairy-tale co-creation process with AI. The system consists of two intelligent components:
\begin{itemize}
    \item Character-centric AI fairy tale generator
    \item Intelligent doodling-based fairy tale visualizer. 
\end{itemize}
The system interacts with the users in the following ways:
\begin{itemize}
    \item Collecting users’ facial characteristics and selecting characters for the storyline based on predefined matching criteria.
    \item Interpreting users’ doodlers, they are added to the AI generated fairy tale illustration and creatively implementing it in the storyline.
\end{itemize}

With AI.R Taletorium, we aim to intensify User-AI interaction in the joint act of creation by using users’ personal properties (face features and doodlers) and challenge artificial intelligence creativity to the highest level of imagination.

% \subsection{AI.R Tentorium Character-Centric Fairy Tale Design Approach}

% AI.R Taletorium is an character-centric multi-modal AI fairy tale telling system aims at connecting users with different background and physical needs into an unified fairy-tale co-creation process with AI. In AI.R Taletorium, we focus on three design problems:

% We formulate the system input/output as follows:

% \begin{itemize}
%     \item How to overcome the language or understanding barrier between different user groups?
%     \item How to better immerse users to the fairy tale telling system instead of just being an audience?
%     \item How to design and support user-AI interaction under the context of co-creation.
% \end{itemize}

% Towards these problems, we propose with AI.R Taletorium the following solutions:

% \begin{itemize}
%     \item Apart from an AI fairy tale generator, we embed AI.R Taletorium with an intelligent sketch board to visualize the generated fairy tale in real-time with doodlers, which is an universal communication tool accepted by majority user groups.
    
%     \item Apart from an AI fairy tale generator, we embed AI.R Taletorium with an intelligent sketch board to visualize the generated fairy tale in real-time with doodlers, which is an universal communication tool accepted by majority user groups.
    
%     \item As both story generator and visualizer are character-centric, we encourage users to join the fairy tale creation by adding or removing doodlers (i.e. characters) from the sketch board, and AI.R Taletorium will automatically update the fairy tale based on user sketch.
% \end{itemize}

% We prototype the system interfaces and interactions in Fig.~\ref{fig:sys_proto}. In summary, the whole system is composed with two intelligent components, a \textit{character-centric AI fairy tale generator} and an \textit{intelligent doodling-based fairy tale visualizer}. Our system interacts with user in two ways. Firstly, we initialize the fairy tale with user face features; Secondly, we encourage user to add\/remove doodlers from the sketch board during visualization and automatically update the story according to user sketches. In the following, we’ll first formulate the character-based fairy tale generator, then we’ll focus on the visualization part and explain in details how we visualize the generated story and integrate user sketch to control story generation. 

\subsection{AI.R Tentorium Character-Centric Fairy Tale Design Approach}
We formulate the system input\/output as follows:

\textbf{Input:} A character set $\mathcal{C}=\{c_0,c_1,..., c_{n-1}\}$, a title $\mathcal{T}=\{t_0,...,$ $t_{k-1}\}$ that defines the main topic of the fairy tale and a storyline size $m$.

\textbf{Intermediate output:} We map characters onto the character embedding space as character feature vector $\mathcal{F}=\{f_0,f_1,...,f_n\}$ as in \cite{Liu2020}. At each generation step $j\in[1,...,m]$, we generate storyline keywords\footnote{We allow $k_j^i$ to be empty, which means the character is not participating the current step.} $\mathcal{K}_j=\{k_j^0,k_j^1,...,k_j^n\}$ for each character according to previous story fragment to define character actions at current step.

\textbf{Output:} A story $S=\{s_0,s_1,...,s_{m-1}\}$ composed with $m$ story fragments, where $s_j=Merge(l_j^0,l_j^1,...,l_j^n)$ is merged from $n\leq\|\mathcal{C}\|$ sentences\footnote{We define $l_j^i=\emptyset$ when $k_j^i=\emptyset$, which means some characters might not be involved from the current generation step, and current fragment size n could be less than number of characters $\|\mathcal{C}\|$.} generated from each character’s storyline keyword at current step.

We follow a recurrent generation framework\cite{Yao} to generate the whole story (Fig.~\ref{fig:pnw_net}). Each fragment is generated based on previous fragment and current status $X_j=(F,K_j)$. We initialize the first fragment as the given title:

\begin{equation}
s_{j+1}=(s_j,X_{j+1}),\quad s_0=\mathcal{T}={t_0,t_1,...,t_k}      
\end{equation}

As shown in Fig.\ref{fig:sys_proto}, we initialize the characters by mapping user facial attributes onto 24 predefined fairy tale characters according to \textit{Psychological $\&$ Cultural Stereotypical} rules.
