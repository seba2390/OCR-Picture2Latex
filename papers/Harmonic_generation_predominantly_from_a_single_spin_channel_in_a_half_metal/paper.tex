%\documentclass[checkout,showpacs,psfig,aps,prl,twocolumn]{revtex4}
%\documentclass[checkout,showpacs,psfig,twocolumn,aps,pra]{revtex4}
\documentclass[checkin,showpacs,psfig,aps,pra]{revtex4}

\def\bull{\vrule height .9ex width .8ex depth -.1ex}
\newcommand{\ba}{\begin{eqnarray}}
\newcommand{\ea}{\end{eqnarray}}
\newcommand{\de}{$\Delta E$ }

\usepackage{graphicx,color}

\newcommand{\xx}{$\gamma^{(3)}$}
\newcommand{\oso}{oscillator strength }
\newcommand{\w}{$\omega$}
\newcommand{\bu}{$B_u$}
\newcommand{\ce}{$\rm C_{60}$}
\newcommand{\cm}{$\rm C_{60}$ }

\newcommand{\Cr}{$\rm CrO_{2}$ }

\newcommand{\Cre}{$\rm CrO_{2}$}

\newcommand{\ham}{\left |\frac {\partial H}{\partial \vec{r}}\right |}

\newcommand{\hp}{hyperpolarizability~}
\newcommand{\hpe}{hyperpolarizability}
\newcommand{\gm}{$\gamma~$}
\newcommand{\gme}{$\gamma$}
\newcommand{\gx}{$\gamma_{max}~$}

\newcommand{\agl}{$\rm A_g(1)$~}
\newcommand{\agle}{$\rm A_g(1)$}
\newcommand{\agz}{$\rm A_g(2)$~}
\newcommand{\agze}{$\rm A_g(2)$}
\newcommand{\ag}{$\rm A_g$~}
\newcommand{\age}{$\rm A_g$}


\newcommand{\hgl}{$\rm H_g(1)$~}
\newcommand{\hgz}{$\rm H_g(2)$~}
\newcommand{\hg}{$\rm H_g$ }
\newcommand{\hge}{$\rm H_g$}

\newcommand{\kag}{$K_{\rm A_g}$}

\newcommand{\kagl}{$K_{\rm A_g(1)}$~}
\newcommand{\kagle}{$K_{\rm A_g(1)}$}
\newcommand{\kagz}{$K_{\rm A_g(2)}$~}
\newcommand{\kagze}{$K_{\rm A_g(2)}$}

\newcommand{\khge}{$K_{\rm H_g(1)}$}
\newcommand{\khgl}{$K_{\rm H_g(1)}$~}
\newcommand{\khgz}{$K_{\rm H_g(2)}$~}
\newcommand{\khgle}{$K_{\rm H_g(1)}$}
\newcommand{\khgze}{$K_{\rm H_g(2)}$}

\newcommand{\khg}{$K_{\rm H_g}$~}


\newcommand{\uni}{$U_{\rm Ni}~$}
\newcommand{\ux}{$U_{\rm X}~$}

\newcommand{\gee}{$\gamma$}
\newcommand{\go}{\gamma_0}
\newcommand{\U}{\uparrow}
\newcommand{\D}{\downarrow}
\newcommand{\R}{\rightarrow}
\newcommand{\z}{\leftarrow}

\newcommand{\xam}{$|\gamma|~$}
\newcommand{\xae}{$|\gamma|$}

\newcommand{\dg}{$^{\circ}$ }
\newcommand{\dge}{$^{\circ}$}
\newcommand{\na}{$\rm NaV_2O_5$ }
\newcommand{\nae}{$\rm NaV_2O_5$}
\newcommand{\dxy}{$d_{xy}$ }
\newcommand{\dxys}{$d_{xy}^*$ }
\newcommand{\be}{\begin{equation}}
\newcommand{\ee}{\end{equation}}
\newcommand{\la}{\langle}
\newcommand{\no}{\noindent}
\newcommand{\ra}{\rangle}

\newcommand{\clr}{}

\newcommand{\et}{{\it et al. }}
\newcommand{\ete}{{\it et al.}}
\def\oa{$\rm O_1$}
\def\ob{$\rm O_2$}
\def\oc{$\rm O_3$}
\def\cpc{Comp.\ Phys.\ Commun.\ }
\def\cpl{Chem.\ Phys.\ Lett.\ }
\def\jpsj{J. Phys. Soc. Jpn.\ }



\newcommand{\bn}{\begin{enumerate}}
\newcommand{\en}{\end{enumerate}}



\newcommand{\raw}{\rightarrow}
\newcommand{\wo}{\omega}

\newcommand{\figone}
{
\vspace{5cm}
\centerline {FIGURE 1, {Zhang and George, PRL }}
\newpage}

\newcommand{\figtwo}
{
\vspace{5cm}
\centerline {FIGURE 2, {Zhang and George, PRL }}
\newpage}


\newcommand{\figthree}
{
\vspace{5cm}
\centerline {FIGURE 3, {Zhang and George, PRL }}
\newpage}


\newcommand{\al}{\eta}


\newcommand{\ws}{\omega_{\sigma}}

\newcommand{\xawm}{$|\gamma(-3\wo;\wo,\wo,\wo)|~$}
\newcommand{\xawe}{$|\gamma (-3\wo;\wo,\wo,\wo)|$}

\newcommand{\xazm}{$|\gamma(0)|~$}
\newcommand{\xaze}{$|\gamma(0)|$}
\newcommand{\mgm}{MgB$_2$ }
\newcommand{\mge}{MgB$_2$}



\begin{document}

\title{Harmonic generation {predominantly} from a single spin channel
  in a half metal}


\author{G. P. Zhang$^*$} \affiliation{Department of Physics, Indiana State
 University, Terre Haute, Indiana 47809, USA}

\author{Y. H. Bai}
\affiliation{Office of Information Technology, Indiana State
  University, Terre Haute, Indiana 47809, USA}




\date{\today}




\begin{abstract}
{Harmonic generation in atoms and molecules has reshaped our
  understanding of ultrafast phenomena beyond the traditional
  nonlinear optics and has launched attosecond physics. Harmonics from
  solids represents a new frontier, where both majority and minority
  spin channels contribute to harmonics.}  This is true even in a
ferromagnet whose electronic states are equally available to optical
excitation.  Here, we demonstrate that harmonics can be generated
{mostly} from a single spin channel in half metallic chromium
dioxide. {An energy gap in the minority channel greatly reduces
  the harmonic generation}, so harmonics predominantly emit from the
majority channel, with a small contribution from the minority
channel. However, this is only possible when the incident photon
energy is well below the energy gap in the minority channel, so all
the transitions in the minority channel are virtual. The onset of the
photon energy is determined by the transition energy between the
dipole-allowed transition between the O-$2p$ and Cr-$3d$
states. Harmonics {mainly} from a single spin channel can be
detected, regardless of laser field strength, as far as the photon
energy is below the minority band energy gap. This prediction should
be tested experimentally.
\end{abstract}

\pacs{42.65.Ky, 78.66.Tr}




\maketitle

\section{Introduction}

Rapid developments of high harmonic generation (HHG) in atoms and
molecules \cite{mcpherson1987,ferray1988} have opened new frontiers in
material research and ultrafast science.  Starting from 1990s,
interest has shifted from atoms, molecules to nanostructure and actual
solids
\cite{farkas1992,vonderlinde1995,prl05,ganeev2009a,ghimire2011,luu2015,garg2016,nd}.
     {\clr Solids are much more complex, and they demand more from
       HHG. For instance, can HHG directly probe band structures in a
       solid \cite{vampa2015a,vampa2015b,prb20}?}  Developing harmonics
       into a tool to explore properties in topological materials such
       as Bi$_2$Se$_3$ \cite{jia2019,baykusheva2021}, MnBi$_2$Te$_4$
       \cite{jia2020b}, MoS$_2$ \cite{jia2020a} and spin-orbit coupled
       systems \cite{lysne2020} is particularly important, given that
       vast majority of devices are solids.  Up to now, most of
     materials investigated so far are nonmagnetic and most of them
     are insulators and semiconductors
     \cite{ghimire2011,garg2016,you2017}.  In ferromagnets, both
     majority and minority spin channels contribute to harmonic
     signals \cite{nc18}, so it is difficult to disentangle their
     respective contributions cleanly \cite{takayoshi2019}.  In
     antiferromagnets, one encounters the same difficulty where two
     sublattices with opposite spin orientations both contribute to
     harmonic signals \cite{Tancogne-Dejean2018}. So far, pure spin
     polarized harmonics in solids have not materialized, but are in
     huge demand in materials where spin dynamics is important. In
     laser-induced ultrafast demagnetization \cite{eric,ourreview},
     one needs to know how differently two spin channels evolve with
     time.  THz emission was reported in Cr(30 $\rm\AA$)/Ni(42
     $\rm\AA$)/Cr(70 $\rm \AA$) films \cite{eric2004}, and more
     recently in Co/ZnO/Pt and Co/Cu/Pt multilayers \cite{li2018} and
     MgO/Fe/MgO thin films \cite{zhang2020}. THz emission only allows
     one to access magnetization on a picosecond time scale.  Being
     able to detect the spin signal from harmonics beyond THz emission
     potentially provides a new way to investigate spin wave dynamics
     \cite{zhang2002,muller2009} on a few hundred femtoseconds.


In this paper, we demonstrate harmonic generation {mostly} from a
single spin channel in half metallic \Cr \cite{schwarz1986,coey}.  \Cr
has no energy gap in the majority spin channel, but has a gap in the
minority one.  The minority state is away from the Fermi level and the
spin-orbit coupling is weak, leading to nearly 100\% spin polarization
\cite{soulen1998}.  These interesting properties manifest themselves
in harmonic generation. The energy gap in the minority band limits the
efficiency of harmonic generation, but the effect on each order of
harmonics is different. The fundamental harmonic, the first harmonic,
is weakly affected, and has no strong dependence on the incident laser
photon energy. A much stronger effect is found in higher orders of
harmonics. With the laser parameters used, we find that upon
excitation of a 0.4-eV pulse, the fifth harmonic from the majority
channel is 60 times stronger than that from the minority signal,
making it a nearly pure single-spin channel harmonic. This difference
disappears when a higher photon energy is used. Such a dramatic effect
is closely related to the dipole-allowed transitions between the
oxygen $2p$ states and chromium $3d$ state, where the strongest
transition is slightly above 2 eV. It is generally believed that
harmonic generation does not have the spin selectivity.  Our finding
here demonstrates that optical harmonics can be a powerful tool to
access spin information in half-metals, where harmonics are {\clr
  largely} emitted from a single spin channel.


{The rest of the paper is arranged as follows. In Sec. II, we present
our theoretical formalism. The results and discussions are in Sec.
III. We conclude this paper in Sec. IV. }

\section{Theoretical formalism}

\newcommand{\ik}{i{\bf k}}
\newcommand{\jk}{j{\bf k}}

\newcommand{\rr}{{\bf r}}

We employ the first-principles density functional theory and couple it
to the time-dependent Liouville equation. This formalism represents an
important alternative to the time-dependent density functional theory
and rigorously respects the Pauli exclusion principles \cite{jpcm16}.
Here in brief, we solve the the spin-polarized Kohn-Sham equation
\cite{wien2k,blaha2020,np09,prb09}, \be \left
     [-\frac{\hbar^2\nabla^2}{2m_e}+V_{ne}+V_{ee}+V_{xc} \right
     ]\psi_{\ik}^\sigma(\rr)=E_{\ik}^\sigma \psi_{\ik}^\sigma
     (\rr), \label{ks} \ee where $\sigma$ is the spin index, $m_e$ is
     the electron mass, the terms on the left-hand side represent the
     kinetic energy, nuclear-electron attraction, electron-electron
     Coulomb repulsion and exchange correlation (generalized gradient
     approximation \cite{pbe}), respectively.
     $\psi_{\ik}^\sigma(\rr)$ is the Bloch wavefunction of band $i$
     for spin $\sigma$ at crystal momentum ${\bf k}$, and
     $E_{\ik}^\sigma$ is the band energy. {Our two spin channels
       are not mixed because of weak spin-orbit coupling
       \cite{soulen1998}. This allows us to cleanly investigate the
       contributions from these two spin channels}.  We use the
     full-potential augmented plane wave method as implemented in the
     Wien2k code \cite{wien2k}, where within the Muffin-tin sphere
     atomic basis functions are used and in the interstitial region a
     plane wave is used. These two basis functions are matched both in
     value and slope at the Muffin-tin boundary.



To investigate high harmonic generation, we solve the Liouville
equation in real time. Different from prior studies \cite{prb19}, we
have to solve the equation for each spin channel $\sigma$, \be
\frac{\partial \rho_{\bf k}^\sigma}{\partial t}= \frac{1}{i\hbar}
     [H_0^\sigma+H_I^\sigma,\rho_{\bf k}^\sigma] -\frac{ \rho_{\bf
         k}^\sigma- \rho_{\bf k}^\sigma(0)}{T_{1(2)}}, \label{decay}
     \ee so the computational cost doubles.  Here $\rho_{\bf
       k}^\sigma$ is the time-dependent density matrix with spin
     $\sigma$ at the ${\bf k}$ point and $\rho_{\bf k}^\sigma(0)$ is
     its initial value. {The second term on the right side
       describes the decay of the density matrix to its initial value
       \cite{shen}.  For a diagonal density matrix element $\rho_{{\bf
           k};ii}^\sigma$, the second term becomes $-[\rho_{{\bf
             k};ii}^\sigma - \rho_{{\bf k};ii}^\sigma(0)]/T_1 $, where
       $i$ is the band index and $T_1$ is called the longitudinal
       relaxation time. For an off-diagonal density matrix element
       $\rho_{{\bf k};i\ne j}^\sigma$, the second term is $-\rho_{{\bf
           k};i\ne j}^\sigma /T_2 $, where $i$ and $j$ are the band
       indices and $T_2$ is called the transverse relaxation time.
       Note that the initial off-diagonal matrix elements are
       zero. $T_1$ and $T_2$ are different from those in NMR where
       $T_1$ is used for the $z$ component of magnetization $M_z$ and
       $T_2$ is used for the $x$ and $y$ components \cite{kittel}.}
     $H_0$ is the field-free system Hamiltonian.  $H_I^\sigma$
     represents the interaction between the laser field and system, $
     H_I^\sigma=-\sum_{{\bf k};i,j} \rho_{{\bf k};i,j}^\sigma {\bf
       P}_{{\bf k};j,i}^\sigma \cdot {\bf A}(t)$.  {${\bf
         P}_{{\bf k};ij}^\sigma$ is the momentum matrix element
       between bands $i$ and $j$ at the ${\bf k}$ point, and is
       calculated within the Wien2k code, but these matrix elements
       must be converted properly before use (see the details in
       Ref. \cite{ourbook})}.  ${\bf A}(t)$ is the vector potential, $
     {\bf A}(t)= A_0 \sin^2(t/\tau) (\cos(\omega t) \hat{x} \pm
     \sin(\omega t) \hat{y}), $ where $A_0$ is the amplitude of the
     vector potential in units of Vfs$/\rm\AA$ \cite{nc18}, $\hat{x}$
     and $\hat{y}$ are the unit vectors along the $x$ and $y$ axes,
     respectively, $t$ is the time, $\omega$ is the carrier frequency,
     + and $-$ refer to the left ($\sigma^-$) and right ($\sigma^+$)
     circularly polarized light within the $xy$ plane, respectively.
     $\tau$ is the laser pulse duration.  We choose the duration
     $\tau$ to be 64 cycles of laser period \cite{prl05}. This is
     sufficient to resolve harmonics at different orders. If the laser
     polarization is within the $yz$ plane, those unit vectors must be
     changed properly. 

 


\section{Results and discussions}



 \Cr crystallizes in the rutile structure, with the space group $\rm
 P4_2/mnm$.  The Cr atoms form a body-centered tetragonal lattice.  If
 we orient the structure in such a way that the Cr atom is at the
 center of the oxygen distorted octahedron, then two apex oxygen atoms
 $O_1$ and $O_2$ are placed along the vertical direction. Four
 equatorial oxygen atoms $O_3$,..., and $O_6$, whose Wyckoff positions
 are $(u,u,0)$, $(u,u,1)$, $(1-u,1-u,0)$, and $(1-u,1-u,1)$, are
 placed in the horizontal plane. The distance between the body center
 Cr atom and $O_1$ and $O_2$ is shorter, $d_a=1.8962 \rm \AA$, and
 that between the body center Cr atom and $O_3$,..., and $O_6$ is
 longer, $d_e=1.9087 \rm \AA$, both of which are computed from
 $d_a=\sqrt{2}ua$ and $d_e=\sqrt{2(\frac{1}{2}-u)^2a^2+(c/2)^2}$
 \cite{sorantin1992}. There are two distances between neighboring
 equatorial O atoms, $c$ and $\sqrt{2}(2u-1)a$, and normally they
 differ, so one does not have an ideal octahedron.



 Our ${\bf k}$ mesh is $29\times 29 \times 45$, with 2760 points in the
 irreducible Brillouin zone.  {The product of planewave cutoff $K_{\rm
     max}$ and Muffin-tin radius $R_{\rm MT}$, $R_{\rm MT}K_{\rm
     max}$, determines the accuracy of our calculation.  Our $R_{\rm
     MT}K_{\rm max}$ is 9, which is more than enough for our
   purpose. $R_{\rm MT}(Cr)$ is 1.89 a.u. and $R_{\rm MT}(O)$ is 1.67
   a.u.  We optimize the internal coordinates of atoms, and we compare
   our results with the experimental ones in Table \ref{tab}.} We see
 that the agreement with the experiment is excellent.  Our theoretical
 spin moment for formula unit is 2.0 $\mu_B$, which agrees with the
 experimental result of 2.0 $\mu_B$ nearly perfectly and prior
 theoretical calculations \cite{schwarz1986,sorantin1992}.

Figure \ref{pdos}(a) displays the partial density of states (PDOS) for
the majority (positive value) and minority (negative value) spins.
The solid line and dotted line denote the Cr-$3d$ and O-$2p$ majority
states, respectively. The vertical dashed line represents the Fermi
energy which is set at 0. One sees that both Cr-$3d$ and O-$2p$ states
cross the Fermi level, which is why \Cr is a metal. Each cell has 28
valence majority electrons and 24 minority electrons.  The Fermi
energy nearly cuts at the local minimum of Cr-$3d$ states, stabilizing
\Cr \cite{sorantin1992}.  The minority PDOS is quite different (see
the solid and dashed lines on the negative axis). At the Fermi level,
there is no DOS, with a band gap of 1.43 eV \cite{schwarz1986}. These
features are fully consistent with prior studies
\cite{schwarz1986,sorantin1992,mazin1999}.  What is important for
harmonic generation is that the valence band is dominated by the
O-$2p$ state, while the conduction band is dominated by the Cr-$3d$
state. This constitutes an ideal case for dipole-allowed transitions.


We find there is no major difference between left and right circularly
polarized light for our system, so in the following we use $\sigma^+$,
with the photon energy 0.4 eV and field amplitude $A_0=0.03$ Vfs/$\rm
\AA$. 
%, i. e. {$1.162 \times 10^9\rm W/cm^2$}
Our interested observable is the expectation value of the momentum
operator, $\langle {\bf P}^\sigma(t) \rangle =\langle \sum_{{\bf
    k};i,j} \rho_{{\bf k};i,j}^\sigma {\bf P}_{{\bf
    k};j,i}^\sigma\rangle, $ where the summation is over the Brillouin
zone and band states. In our calculation, we include all the states
from Cr-$3s$ states (energy at $-4.89$ Ry for spin up and $-4.73$ Ry
for spin down), Cr-$3p$ state (energy at $-2.79$ Ry for spin up and
$-2.64$ Ry for spin down), O-$2s$ (-0.95 Ry), all the way up to 4.95
Ry, with 10 Ry energy window width covering all the possible channels
for excitation. To compute harmonic spectra, we Fourier transform
$\langle {\bf P}^\sigma(t) \rangle$ into the frequency domain
\cite{nc18,prl05,jia2020a}, \be {\bf P}(\Omega)=\int_{-\infty}^{\infty} {\bf
  P}(t) {\rm e}^{i\Omega t} {\cal W}(t) dt, \ee where ${\cal W}(t)$ is
the window function \cite{nc18}. 


 Figure \ref{fig1}(b) shows harmonic signals from the majority (top)
 and minority (bottom) spin channels, respectively.  We see that all
 the harmonics appear at odd orders because our system has an
 inversion symmetry. The majority spin channel has a stronger harmonic
 signal than the minority spin channel, and it has a broader peak.
 This difference is expected from the PDOS. Recall that the minority
 band has an energy gap, so with the photon energy of 0.4 eV, the
 excitation is virtual.  The majority band has no such gap, so
 electrons in the valence band can successively excite into conduction
 bands and then emit harmonics. 

We notice a crucial difference when we compare the 3rd harmonic signal
with the 5th harmonic signal from the majority and minority spin
channels. The difference in the 5th harmonic between the majority and
minority spin channels is much larger.  A very weak 5th harmonic in
the minority channel is interesting because it potentially provides an
opportunity to generate harmonics from a single spin channel. We note
that in a nonmagnetic material \cite{jia2019}, there is no difference
in harmonics between spin up and spin down channels. In ferromagnets,
this difference is also small because each channel is able to generate
harmonics with a small difference in strength
\cite{nc18,jia2020a,jia2020b}.  A challenge in ferromagnetic metals is
that the electron deexcitation is extremely fast \cite{nc18}, so it is
difficult to detect those harmonics. CrO$_2$ is very unique. It is
known that laser-induced ultrafast demagnetization lasts very long
\cite{zhang2006}, partly because its spin-orbit coupling is very
weak. This prolonged process is advantageous to detect harmonic
signals.


Next we investigate how laser parameters affect harmonic generation in
CrO$_2$. We increase the photon energy to 0.8 eV, while keeping the
rest of laser parameters unchanged. Figure \ref{fig2}(b) shows that
the 5th harmonic in the minority channel at 0.8 eV is slightly smaller
than that in the majority channel, very different from that at 0.4 eV
(see Fig. \ref{fig2}(a)). Energetically, when the photon energy
$\hbar\omega$ is 0.4 eV, $5\hbar\omega=2$ eV is slightly above the gap
energy of 1.43 eV. Because a transition from the O-$2p$ to Cr-$3d$
does not occur immediately at this band gap energy, one is still in
the virtual excitation regime. This situation is important because not
all half metals have this property. For instance, in another
half-metal, Mn$_2$RuGa does not have this unique feature, so its
harmonic generation from the majority channel is similar to that in
the minority channel. With $\hbar\omega= 0.8$ eV, $5\hbar\omega=4$ eV
is well into the real excitation regime, so the band gap suppression
is no longer at work. This is further verified when we increase the
incident photon energy to 1.6 eV. Figure \ref{fig2}(c) shows that
harmonic strength in the minority and majority channels is almost
exactly the same, so harmonics are generated from both spin
channels. Figure \ref{fig2}(d) compares the harmonic signal ratio of
the majority harmonic signal to the minority harmonic signal as a
function of incident photon energy. We see that in general higher
order harmonics have a much stronger variation. The ratio is larger
when the incident photon energy is small. At 0.4 eV, the fifth
harmonic has a ratio of 67 with $A_0=\rm 0.03Vfs/\AA$.  This photon
energy window predicted here is critical to future experiments.


To show that our prediction is not specific to a particular laser
field strength, we choose several vector potential amplitudes. We
increase the amplitude $A_0$ from 0.01 to 0.05 $\rm Vfs/\AA$ and
include both the third and fifth harmonics. Figure \ref{fig3}(a)
displays the third harmonic signal as a function of $A_0$ for spin up
(filled circles) and spin down (filled boxes) channels.  The photon
energy is 0.4 eV and the duration is also 64 cycles of the laser
period as above. We see that both signals increase superlinearly. We
fit the signal to a power function and find that the signal scales
with $A_0$ as $A_0^{2.37}$.  According to the traditional nonlinear
optics \cite{shen}, the polarization ${\cal P}$ is proportional to the
response function $R(t)$ multiplied by the external field $E(t)$
\cite{mukamel}, and for the third harmonic ${\cal P}^{(3)}\propto
R^{(3)}(t) E(t)^3$. If $R^{(3)}(t)$ is significantly modulated by band
states, the scaling exponent differs from 3. This shows that real
excitations among band states contribute to the harmonic signal, which
is exactly what one expects from the density of states in the spin
majority channel (Fig. \ref{fig1}(a)). The spin minority channel is
different. We see that the signal scales as $A_0^{2.99}$ (filled boxes
in Fig. \ref{fig3}(a)), with the exponent close to 3. This again shows
that the excitation is virtual, as expected. 


Different excitations seen above also leave a crucial hallmark on the
power spectrum as well. Figure \ref{fig3}(b) shows the third harmonic
signal as a function of $A_0$, from 0.01 to 0.05 $\rm Vfs/\AA$,
denoted respectively by the circles, boxes, diamonds, up triangles and
left triangles. We see that the peaks for the spin up channel are not
entirely symmetric with respect to the 3rd order. This is an
indication that harmonics are generated through real states
(Fig. \ref{fig1}).  In the spin down channel, the harmonic peak is
highly symmetric (see Fig. \ref{fig3}(c)). In other words, the
harmonic asymmetry is a hallmark for transitions that involve real
band states, which is crucial for electronic structure detection in
solids.

 We can further demonstrate this in the fifth harmonic. Figure
 \ref{fig3}(d) shows the 5th harmonic signal as a function of $A_0$
 for the majority spin (circles) and minority spin (boxes)
 channels. As we increase $A_0$, the signal scales as $A_0^{4.32}$,
 indicative of real transitions among band states.  To reveal further
 details in harmonics, we also plot the harmonic spectrum. Figure
 \ref{fig3}(e) shows that the harmonic peak is not symmetric with
 respect to 5.  All the harmonic signals are multiplied by 100 for an
 easy view.  In the spin down channel, we note that the nominal 5th
 peak does not appear at 5 (Fig. \ref{fig3}(f)), where harmonic
 signals are multiplied by 1000 for an easy view.  This is because the
 main excitation occurs above 2 eV, which is fully consistent with our
 observation in the photon energy dependence in Fig. \ref{fig2}.  In
 comparison with the majority signal, the minority signal is extremely
 weak even at 0.05 $\rm Vfs/\AA$. Quantitatively, we find the ratio
 between the majority and minority 5th order signals is 46, or about
 2\%.


{To establish our results on a solid ground, we carry out several
  additional tests.  We first check our harmonic signal convergence
  with the Brillouin zone sampling by increasing the number of $k$
  points from $18\times 18 \times 28$, $19\times 19 \times 30$, to
  $29\times 29 \times 45$, and we find the results are well converged.
  For all the results in this paper, we use a ${\bf k}$ mesh of
  $29\times 29 \times 45$, with 2760 points in the irreducible
  Brillouin zone, or 10830 in the full Brillouin zone.  Secondly, we
  examine the influence of relaxation times $T_1$ and $T_2$ on the
  harmonic signal.  $T_2$ is normally shorter than $T_1$, but their
  precise values are unknown, so to this end, we take $T_1=200$ fs and
  $T_2=100$ fs.  {\clr Extremely smaller $T_1$ and $T_2$ were used
    before in ZnO \cite{vampa2014}.}  Figures \ref{fig4}(a)-(d) show
  the results for the same laser parameters used in
  Fig. \ref{fig1}(a), but with $T_1=400$ fs and $T_2=200$ fs.  Their
  influence on our harmonic signal is very weak since our laser pulse
  duration is on the order of 200 fs.  When we increase $T_1$ and
  $T_2$ from 200 and 100 fs to 400 and 200 fs, the fifth order
  harmonic changes from $0.36\times 10^{-2}$ to $0.50\times 10^{-2}$
  for the majority channel.  The change is even smaller for the
  minority band, and it only changes from $0.55\times 10^{-4}$ to
  $0.52\times 10^{-4}$ for the minority channel in the same unit. This
  huge difference between the spin majority and minority channels
  remains, regardless of the $x$ (Figs.  \ref{fig4}(a) and \ref{fig4}
  (c)) or $y$ (Figs. \ref{fig4}(b) and (d) or Figs. \ref{fig4}(e) and
  \ref{fig4}(f)) or $z$ component (Figs. \ref{fig4}(f) and
  \ref{fig4}(h)).  Since \Cr is tetragonal, we wonder whether the
  laser polarization in the $yz$ plane ($\sigma_{yz}$), instead of the
  $xy$ plane ($\sigma_{xy}$), makes a difference. Figures
  \ref{fig4}(e)-(h) show our results with the same relaxation times as
  Figs.  \ref{fig4}(a)-(d).  We find that for the majority band, the
  $y$ component maximum of the 5th harmonic signal is similar, but the
  line shape is slightly different. Under $\sigma_{yz}$ excitation,
  the peak appears more symmetric (see Fig. \ref{fig4}(e)) and has no
  structure in it, while under $\sigma_{xy}$ excitation, the peak has
  more structures. This difference is due to the different band states
  involved during harmonic generation. This potentially provides
  another interesting direction for experimental investigation as a
  function of laser polarization angle, in a similar way as done in
  MgO \cite{you2017}. For the minority band, the difference is much
  smaller (see Figs. \ref{fig4}(g) and \ref{fig4}(f)). Importantly,
  the ratio of the majority 5th harmonic to that of the minority
  remains the same.}  This concludes that the fifth order harmonic
emission is mainly from the majority channel.  This paves the way to
future experimental detection of harmonics from a single spin channel.
We believe that the results found here are interesting and should
motivate future experimental and theoretical investigations.


\section{conclusion}

We have investigated harmonic generation from half-metallic \Cre. This
nearly 100\% spin polarized material shows peculiar features that are
not found in other materials. The majority spin channel has no energy
gap just like a metal, but in the minority channel has 1.4-eV energy
gap. Upon laser excitation, this leads to a significant difference in
harmonic generation in these two channels. This difference depends on
the harmonic order and photon energy $\hbar\omega$ used. The higher
the order is, the larger the effect is. At $\hbar\omega=0.4$ eV and
the fifth harmonic, we find that the ratio of the majority channel
signal to the minority signal exceeds 60. This makes the 5th harmonic
as if it is emitted {nearly} purely from a single spin
channel. Microscopically, in the minority channel, transitions are
mainly from the O-$2p$ states to Cr-$3d$ states, so the onset of
single spin channel harmonics starts at 2 eV. This explains why the
fifth order shows such a strong effect. In general, it is known that
harmonics are less sensitive to material properties. Here, we show
that when a material has a peculiar difference in spin channels,
harmonics can be very powerful.  Our prediction, if confirmed
experimentally, potentially represents a new direction for high
harmonic generation for solids
\cite{ghimire2011,luu2015,garg2016,nd,neufeld2019}.


\acknowledgments
This work was solely supported by the U.S. Department of Energy under
Contract No. DE-FG02-06ER46304. Part of the work was done on Indiana
State University's high performance Quantum and Obsidian clusters.
The research used resources of the National Energy Research Scientific
Computing Center, which is supported by the Office of Science of the
U.S. Department of Energy under Contract No. DE-AC02-05CH11231.



$^*$guo-ping.zhang@outlook.com. ~~  https://orcid.org/0000-0002-1792-2701



\begin{thebibliography}{99}
%
\bibliographystyle{./IEEEtran}
\bibliography{./IEEEabrv,./IEEEexample}

@article{kousha2,
	title={Robust Privacy-Utility Tradeoffs Under Differential Privacy and Hamming Distortion},
	author={Kousha Kalantari and Lalitha Sankar and Anand D. Sarwate},
	journal={IEEE Transactions on Information Forensics and Security},
	volume={13},
	number={11},
	pages={2816--2830},
	year={2018},
	publisher={IEEE}
}



@article{DBLP:ITjournal,
  author    = {Naanin Takbiri and
               Amir Houmansadr and
               Dennis Goeckel and
               Hossein Pishro-Nik},
  title     = {Matching Anonymized and Obfuscated Time Series to Users' Profiles
},
  journal   = {CoRR},
  volume    = {abs/1710.00197},
  year      = {2017},
  url       = {https://arxiv.org/abs/1710.00197},
  archivePrefix = {arXiv},
  eprint    = {1710.00197},
  timestamp = {Sat, 30 Sep 2017 13:03:19 GMT},
}


@inproceedings{KeConferance,
   title={Bayesian Time Series Matching and Privacy},
    author={Ke Le and 	Hossein Pishro-Nik and Dennis Goeckel},
	booktitle={51th Asilomar Conference on Signals, Systems and Computers},
   year={2017},
  	address={Pacific Grove, CA}
   }	

@article{DBLP:journals/corr/LiaoSCT17,
  author    = {Jiachun Liao and
               Lalitha Sankar and
               Fl{\'{a}}vio du Pin Calmon and
               Vincent Yan Fu Tan},
  title     = {Hypothesis Testing under Maximal Leakage Privacy Constraints},
  journal   = {CoRR},
  volume    = {abs/1701.07099},
  year      = {2017},
  url       = {http://arxiv.org/abs/1701.07099},
  timestamp = {Wed, 07 Jun 2017 14:40:47 +0200},
  biburl    = {http://dblp.uni-trier.de/rec/bib/journals/corr/LiaoSCT17},
  bibsource = {dblp computer science bibliography, http://dblp.org}
}


@inproceedings{sankar,
    title={On information-theoretic privacy with general distortion cost functions},
    author={Kalantari, Kousha and Sankar, Lalitha and  Kosut, Oliver},
    booktitle={2017 IEEE International Symposium on Information Theory (ISIT)},
    year={2017},
      	address={Aachen, Germany},

    organization={IEEE}
    
  }	


@article{hyposankar,
  author    = {Jiachun Liao and
               Lalitha Sankar and
               Vincent Y. F. Tan and
               Fl{\'{a}}vio du Pin Calmon},
  title     = {Hypothesis Testing in the High Privacy Limit},
  journal   = {CoRR},
  volume    = {abs/1607.00533},
  year      = {2016},
  url       = {http://arxiv.org/abs/1607.00533},
  timestamp = {Wed, 07 Jun 2017 14:41:19 +0200},
  biburl    = {http://dblp.uni-trier.de/rec/bib/journals/corr/LiaoSTC16},
  bibsource = {dblp computer science bibliography, http://dblp.org}
}

@article{battery15,
  author    = {Simon Li and
               Ashish Khisti and
               Aditya Mahajan},
  title     = {Privacy-Optimal Strategies for Smart Metering Systems with a Rechargeable
               Battery},
  journal   = {CoRR},
  volume    = {abs/1510.07170},
  year      = {2015},
  url       = {http://arxiv.org/abs/1510.07170},
  timestamp = {Wed, 07 Jun 2017 14:40:53 +0200},
  biburl    = {http://dblp.uni-trier.de/rec/bib/journals/corr/LiKM15},
  bibsource = {dblp computer science bibliography, http://dblp.org}
}



@inproceedings{geo2013,
   title={Geo-indistinguishability: differential privacy for location-based systems},
    author={Miguel E. Andres and Nicolas E. Bordenabe and Konstantinos Chatzikokolakis and Catuscia Palamidessi},
	booktitle={Proceedings of the 2013 ACM SIGSAC conference on Computer and communications security},

   year={2013},
   pages={901--914},
  	address={New York, NY}
   }	

@article{Yeb17,
  author    = {Min Ye and
               Alexander Barg},
  title     = {Optimal Schemes for Discrete Distribution Estimation under Locally
               Differential Privacy},
  journal   = {CoRR},
  volume    = {abs/1702.00610},
  year      = {2017},
  url       = {http://arxiv.org/abs/1702.00610},
  timestamp = {Wed, 07 Jun 2017 14:41:08 +0200},
  biburl    = {http://dblp.uni-trier.de/rec/bib/journals/corr/YeB17},
  bibsource = {dblp computer science bibliography, http://dblp.org}
}

@inproceedings{info2012,
    title={Information-Theoretic Foundations of Differential Privacy},
    author={Mir J. Darakhshan},
    booktitle={International Symposium on Foundations and Practice of Security},
    year={2012},
    organization={Springer}
  }	

@inproceedings{diff2017,
    title={Dynamic Differential Location Privacy with Personalized Error Bounds},
    author={Lei Yu and Ling Liu and Calton Pu},
    booktitle={The Network and Distributed System Security Symposium},
    year={2017}
  }	

@inproceedings{ciss2017,
    title={Fundamental Limits of Location Privacy using Anonymization},
    author={N. Takbiri and A. Houmansadr and D.L. Goeckel and H. Pishro-Nik},
    booktitle={Annual Conference on Information Science and Systems (CISS)},
    year={2017},
    organization={IEEE}
  }	

@inproceedings{sit2017,
    title={Limits of Location Privacy under Anonymization and Obfuscation},
    author={Nazanin Takbiri and Amir Houmansadr and Dennis L. Goeckel and Hossein Pishro-Nik},
    booktitle={International Symposium on Information Theory (ISIT)},
    year={2017},
    organization={IEEE}
  }	



@article{tifs2016,
	Author = {Z. Montazeri and A. Houmansadr and H. Pishro-Nik},
	Journal = {IEEE Transaction on Information Forensics and Security, to appear},
	Publisher = {IEEE},
	Title = {{Achieving Perfect Location Privacy in Wireless Devices Using Anonymization}},
	Year = {2017}}
	
	@article{matching,
  title={Asymptotically Optimal Matching of Multiple Sequences to Source Distributions and Training Sequences},
  author={Jayakrishnan Unnikrishnan},
  journal={IEEE Transactions on Information Theory},
  volume={61},
  number={1},
  pages={452-468},
  year={2015},
  publisher={IEEE}
}

@article{Naini2016,
	Author = {F. Naini and J. Unnikrishnan and P. Thiran and M. Vetterli},
	Journal = {IEEE Transactions on Information Forensics and Security},
	Publisher = {IEEE},
	Title = {Where You Are Is Who You Are: User Identification by Matching Statistics},
	 volume={11},
    number={2},
     pages={358--372},
    Year = {2016}
}



@inproceedings{montazeri2016defining,
    title={Defining perfect location privacy using anonymization},
    author={Montazeri, Zarrin and Houmansadr, Amir and Pishro-Nik, Hossein},
    booktitle={2016 Annual Conference on Information Science and Systems (CISS)},
    pages={204--209},
    year={2016},
    organization={IEEE}
  }	
	
	
@inproceedings{Mont1610Achieving,
  title={Achieving Perfect Location Privacy in Markov Models Using Anonymization},
  author={Montazeri, Zarrin and Houmansadr, Amir and H.Pishro-Nik},
  booktitle="2016 International Symposium on Information Theory and its Applications
  (ISITA2016)",
  address="Monterey, USA",
  days=30,
  month="oct",
  year=2016,
}

@misc{uber-stats,
title = {{By The Numbers 24 Amazing Uber Statistics}},
author = {Craig Smith},
note = {\url{http://expandedramblings.com/index.php/uber-statistics/}},
year=2015,
month= "September"
}


@misc{GMaps-users,
title = {{55\% of U.S. iOS users with Google Maps use it weekly}},
author = {Mike Dano},
note = {\url{http://www.fiercemobileit.com/story/55-us-ios-users-google-maps-use-it-weekly/2013-08-27}},
year=2013
}

@misc{Google-stats,
title = {{Statistics and facts about Google}},
note = {\url{http://www.statista.com/topics/1001/google/}},
}

@misc{yelp-stats,
title = {{By The Numbers: 45 Amazing Yelp Statistics}},
author = {Craig Smith},
note = {\url{http://expandedramblings.com/index.php/yelp-statistics/}},
year=2015,
month= "May"
}


@misc{Uber-hacker,
title = {{Is Uber's rider database a sitting duck for hackers?}},
author = {Craig Timberg},
note = {\url{https://www.washingtonpost.com/news/the-switch/wp/2014/12/01/is-ubers-rider-database-a-sitting-duck-for-hackers/}},
year=2014,
month= "December"
}

@misc{Uber-godview,
title = {{``God View'': Uber Investigates Its Top New York Executive For Privacy Violations}},
note = {\url{https://www.washingtonpost.com/news/the-switch/wp/2014/12/01/is-ubers-rider-database-a-sitting-duck-for-hackers/}},
year=2014,
month= "November"
}


@misc{Uber-breach-statement,
title = {{Uber Statement}},
note = {\url{http://newsroom.uber.com/2015/02/uber-statement/}},
year=2015,
month= "February"
}



@inproceedings{zhou2007privacy,
  title={Privacy-preserving detection of sybil attacks in vehicular ad hoc networks},
  author={Zhou, Tong and Choudhury, Romit Roy and Ning, Peng and Chakrabarty, Krishnendu},
  booktitle={Mobile and Ubiquitous Systems: Networking \& Services, 2007. MobiQuitous 2007. Fourth Annual International Conference on},
  pages={1--8},
  year={2007},
  organization={IEEE}
}

@article{chang2012footprint,
  title={Footprint: Detecting sybil attacks in urban vehicular networks},
  author={Chang, Shan and Qi, Yong and Zhu, Hongzi and Zhao, Jizhong and Shen, Xuemin Sherman},
  journal={Parallel and Distributed Systems, IEEE Transactions on},
  volume={23},
  number={6},
  pages={1103--1114},
  year={2012},
  publisher={IEEE}
}

@inproceedings{shokri2012protecting,
	Author = {Shokri, Reza and Theodorakopoulos, George and Troncoso, Carmela and Hubaux, Jean-Pierre and Le Boudec, Jean-Yves},
	Booktitle = {Proceedings of the 2012 ACM conference on Computer and communications security},
	Organization = {ACM},
	Pages = {617--627},
	Title = {Protecting location privacy: optimal strategy against localization attacks},
	Year = {2012}}


@inproceedings{gruteser2003anonymous,
	Author = {Gruteser, Marco and Grunwald, Dirk},
	Booktitle = {Proceedings of the 1st international conference on Mobile systems, applications and services},
	Organization = {ACM},
	Pages = {31--42},
	Title = {Anonymous usage of location-based services through spatial and temporal cloaking},
	Year = {2003}}



@inproceedings{hoh2007preserving,
	Author = {Hoh, Baik and Gruteser, Marco and Xiong, Hui and Alrabady, Ansaf},
	Booktitle = {Proceedings of the 14th ACM conference on Computer and communications security},
	Organization = {ACM},
	Pages = {161--171},
	Title = {Preserving privacy in gps traces via uncertainty-aware path cloaking},
	Year = {2007}}


@inproceedings{shokri2011quantifying,
	Author = {Shokri, Reza and Theodorakopoulos, George and Le Boudec, Jean-Yves and Hubaux, Jean-Pierre},
	Booktitle = {Security and Privacy (SP), 2011 IEEE Symposium on},
	Organization = {IEEE},
	Pages = {247--262},
	Title = {Quantifying location privacy},
	Year = {2011}}


@inproceedings{bordenabe2014optimal,
	Author = {Bordenabe, Nicol{\'a}s E and Chatzikokolakis, Konstantinos and Palamidessi, Catuscia},
	Booktitle = {Proceedings of the 2014 ACM SIGSAC Conference on Computer and Communications Security},
	Organization = {ACM},
	Pages = {251--262},
	Title = {Optimal geo-indistinguishable mechanisms for location privacy},
	Year = {2014}}




@inproceedings{freudiger2009non,
	Author = {Freudiger, Julien and Manshaei, Mohammad Hossein and Hubaux, Jean-Pierre and Parkes, David C},
	Booktitle = {Proceedings of the 16th ACM conference on Computer and communications security},
	Organization = {ACM},
	Pages = {324--337},
	Title = {On non-cooperative location privacy: a game-theoretic analysis},
	Year = {2009}}

@inproceedings{ma2009location,
	Author = {Ma, Zhendong and Kargl, Frank and Weber, Michael},
	Booktitle = {Sarnoff Symposium, 2009. SARNOFF'09. IEEE},
	Organization = {IEEE},
	Pages = {1--6},
	Title = {A location privacy metric for v2x communication systems},
	Year = {2009}}

@article{1corser2016evaluating,
  title={Evaluating Location Privacy in Vehicular Communications and Applications},
  author={Corser, George P and Fu, Huirong and Banihani, Abdelnasser},
  journal={IEEE Transactions on Intelligent Transportation Systems},
  volume={17},
  number={9},
  pages={2658-2667},
  year={2016},
  publisher={IEEE}
}
@article{2zhang2016designing,
  title={On Designing Satisfaction-Ratio-Aware Truthful Incentive Mechanisms for k-Anonymity Location Privacy},
  author={Zhang, Yuan and Tong, Wei and Zhong, Sheng},
  journal={IEEE Transactions on Information Forensics and Security},
  volume={11},
  number={11},
  pages={2528--2541},
  year={2016},
  publisher={IEEE}
}

@article{11dewri2014exploiting,
  title={Exploiting service similarity for privacy in location-based search queries},
  author={Dewri, Rinku and Thurimella, Ramakrisha},
  journal={IEEE Transactions on Parallel and Distributed Systems},
  volume={25},
  number={2},
  pages={374--383},
  year={2014},
  publisher={IEEE}
}

@inproceedings{gedik2005location,
	Author = {Gedik, Bu{\u{g}}ra and Liu, Ling},
	Booktitle = {Distributed Computing Systems, 2005. ICDCS 2005. Proceedings. 25th IEEE International Conference on},
	Organization = {IEEE},
	Pages = {620--629},
	Title = {Location privacy in mobile systems: A personalized anonymization model},
	Year = {2005}}

@inproceedings{zhong2009distributed,
	Author = {Zhong, Ge and Hengartner, Urs},
	Booktitle = {Pervasive Computing and Communications, 2009. PerCom 2009. IEEE International Conference on},
	Organization = {IEEE},
	Pages = {1--10},
	Title = {A distributed k-anonymity protocol for location privacy},
	Year = {2009}}

@inproceedings{mokbel2006new,
	Author = {Mokbel, Mohamed F and Chow, Chi-Yin and Aref, Walid G},
	Booktitle = {Proceedings of the 32nd international conference on Very large data bases},
	Organization = {VLDB Endowment},
	Pages = {763--774},
	Title = {The new Casper: query processing for location services without compromising privacy},
	Year = {2006}}

@article{kalnis2007preventing,
	Author = {Kalnis, Panos and Ghinita, Gabriel and Mouratidis, Kyriakos and Papadias, Dimitris},
	Journal = {Knowledge and Data Engineering, IEEE Transactions on},
	Number = {12},
	Pages = {1719--1733},
	Publisher = {IEEE},
	Title = {Preventing location-based identity inference in anonymous spatial queries},
	Volume = {19},
	Year = {2007}}


@article{sweeney2002k,
	Author = {Sweeney, Latanya},
	Journal = {International Journal of Uncertainty, Fuzziness and Knowledge-Based Systems},
	Number = {05},
	Pages = {557--570},
	Publisher = {World Scientific},
	Title = {k-anonymity: A model for protecting privacy},
	Volume = {10},
	Year = {2002}}

@article{sweeney2002achieving,
	Author = {Sweeney, Latanya},
	Journal = {International Journal of Uncertainty, Fuzziness and Knowledge-Based Systems},
	Number = {05},
	Pages = {571-588},
	Publisher = {World Scientific},
	Title = {Achieving k-anonymity privacy protection using generalization and suppression},
	Volume = {10},
	Year = {2002}}

@inproceedings{liu2013game,
	Author = {Liu, Xinxin and Liu, Kaikai and Guo, Linke and Li, Xiaolin and Fang, Yuguang},
	Booktitle = {INFOCOM, 2013 Proceedings IEEE},
	Organization = {IEEE},
	Pages = {2985--2993},
	Title = {A game-theoretic approach for achieving k-anonymity in location based services},
	Year = {2013}}

@inproceedings{hoh2005protecting,
	Author = {Hoh, Baik and Gruteser, Marco},
	Booktitle = {Security and Privacy for Emerging Areas in Communications Networks, 2005. SecureComm 2005. First International Conference on},
	Organization = {IEEE},
	Pages = {194--205},
	Title = {Protecting location privacy through path confusion},
	Year = {2005}}

@article{beresford2003location,
	Author = {Beresford, Alastair R and Stajano, Frank},
	Journal = {IEEE Pervasive computing},
	Number = {1},
	Pages = {46--55},
	Publisher = {IEEE},
	Title = {Location privacy in pervasive computing},
	Year = {2003}}


@inproceedings{palanisamy2011mobimix,
	Author = {Palanisamy, Balaji and Liu, Ling},
	Booktitle = {Data Engineering (ICDE), 2011 IEEE 27th International Conference on},
	Organization = {IEEE},
	Pages = {494--505},
	Title = {Mobimix: Protecting location privacy with mix-zones over road networks},
	Year = {2011}}
@inproceedings{freudiger2009optimal,
	Author = {Freudiger, Julien and Shokri, Reza and Hubaux, Jean-Pierre},
	Booktitle = {Privacy enhancing technologies},
	Organization = {Springer},
	Pages = {216--234},
	Title = {On the optimal placement of mix zones},
	Year = {2009}}

@article{manshaei2013game,
	Author = {Manshaei, Mohammad Hossein and Zhu, Quanyan and Alpcan, Tansu and Bac{\c{s}}ar, Tamer and Hubaux, Jean-Pierre},
	Journal = {ACM Computing Surveys (CSUR)},
	Number = {3},
	Pages = {25},
	Publisher = {ACM},
	Title = {Game theory meets network security and privacy},
	Volume = {45},
	Year = {2013}}

@article{19freudiger2013non,
  title={Non-cooperative location privacy},
  author={Freudiger, Julien and Manshaei, Mohammad Hossein and Hubaux, Jean-Pierre and Parkes, David C},
  journal={IEEE Transactions on Dependable and Secure Computing},
  volume={10},
  number={2},
  pages={84--98},
  year={2013},
  publisher={IEEE}
}


@article{paulet2014privacy,
	Author = {Paulet, Russell and Kaosar, Md Golam and Yi, Xun and Bertino, Elisa},
	Journal = {Knowledge and Data Engineering, IEEE Transactions on},
	Number = {5},
	Pages = {1200--1210},
	Publisher = {IEEE},
	Title = {Privacy-preserving and content-protecting location based queries},
	Volume = {26},
	Year = {2014}}

@article{khoshgozaran2011location,
	Author = {Khoshgozaran, Ali and Shahabi, Cyrus and Shirani-Mehr, Houtan},
	Journal = {Knowledge and Information Systems},
	Number = {3},
	Pages = {435--465},
	Publisher = {Springer},
	Title = {Location privacy: going beyond K-anonymity, cloaking and anonymizers},
	Volume = {26},
	Year = {2011}}

@article{18shokri2014hiding,
  title={Hiding in the mobile crowd: Locationprivacy through collaboration},
  author={Shokri, Reza and Theodorakopoulos, George and Papadimitratos, Panos and Kazemi, Ehsan and Hubaux, Jean-Pierre},
  journal={IEEE transactions on dependable and secure computing},
  volume={11},
  number={3},
  pages={266--279},
  year={2014},
  publisher={IEEE}
}

@article{8zurbaran2015near,
  title={Near-Rand: Noise-based Location Obfuscation Based on Random Neighboring Points},
  author={Zurbaran, Mayra Alejandra and Avila, Karen and Wightman, Pedro and Fernandez, Michael},
  journal={IEEE Latin America Transactions},
  volume={13},
  number={11},
  pages={3661--3667},
  year={2015},
  publisher={IEEE}
}

@inproceedings{hoh2007preserving,
	Author = {Hoh, Baik and Gruteser, Marco and Xiong, Hui and Alrabady, Ansaf},
	Booktitle = {Proceedings of the 14th ACM conference on Computer and communications security},
	Organization = {ACM},
	Pages = {161--171},
	Title = {Preserving privacy in gps traces via uncertainty-aware path cloaking},
	Year = {2007}}

@inproceedings{ho2011differential,
	Author = {Ho, Shen-Shyang and Ruan, Shuhua},
	Booktitle = {Proceedings of the 4th ACM SIGSPATIAL International Workshop on Security and Privacy in GIS and LBS},
	Organization = {ACM},
	Pages = {17--24},
	Title = {Differential privacy for location pattern mining},
	Year = {2011}}


@article{12hwang2014novel,
  title={A novel time-obfuscated algorithm for trajectory privacy protection},
  author={Hwang, Ren-Hung and Hsueh, Yu-Ling and Chung, Hao-Wei},
  journal={IEEE Transactions on Services Computing},
  volume={7},
  number={2},
  pages={126--139},
  year={2014},
  publisher={IEEE}
  }

  @article{16haghnegahdar2014privacy,
  title={Privacy Risks in Publishing Mobile Device Trajectories},
  author={Haghnegahdar, Alireza and Khabbazian, Majid and Bhargava, Vijay K},
  journal={IEEE Wireless Communications Letters},
  volume={3},
  number={3},
  pages={241--244},
  year={2014},
  publisher={IEEE}
}

@article{20gao2013trpf,
  title={TrPF: A trajectory privacy-preserving framework for participatory sensing},
  author={Gao, Sheng and Ma, Jianfeng and Shi, Weisong and Zhan, Guoxing and Sun, Cong},
  journal={IEEE Transactions on Information Forensics and Security},
  volume={8},
  number={6},
  pages={874--887},
  year={2013},
  publisher={IEEE}
}

@article{21ma2013privacy,
  title={Privacy vulnerability of published anonymous mobility traces},
  author={Ma, Chris YT and Yau, David KY and Yip, Nung Kwan and Rao, Nageswara SV},
  journal={IEEE/ACM Transactions on Networking},
  volume={21},
  number={3},
  pages={720--733},
  year={2013},
  publisher={IEEE}
}

@article{6li2016privacy,
  title={Privacy Leakage of Location Sharing in Mobile Social Networks: Attacks and Defense},
  author={Li, Huaxin and Zhu, Haojin and Du, Suguo and Liang, Xiaohui and Shen, Xuemin},
  journal={IEEE Transactions on Dependable and Secure Computing},
  year={2016},
  volume={PP},
  number={99},
  publisher={IEEE}
}


@article{14zhang2014privacy,
  title={Privacy quantification model based on the Bayes conditional risk in Location-Based Services},
  author={Zhang, Xuejun and Gui, Xiaolin and Tian, Feng and Yu, Si and An, Jian},
  journal={Tsinghua Science and Technology},
  volume={19},
  number={5},
  pages={452--462},
  year={2014},
  publisher={TUP}
}

@article{4olteanu2016quantifying,
  title={Quantifying Interdependent Privacy Risks with Location Data},
  author={Olteanu, Alexandra-Mihaela and Huguenin, K{\'e}vin and Shokri, Reza and Humbert, Mathias and Hubaux, Jean-Pierre},
  journal={IEEE Transactions on Mobile Computing},
  year={2016},
  volume={PP},
  number={99},
  pages={1-1},
  publisher={IEEE}
}















@misc{Leberknight2010,
	Author = {Leberknight, C. and Chiang, M. and Poor, H. and Wong, F.},
	Howpublished = {\url{http://www.princeton.edu/~chiangm/anticensorship.pdf}},
	Title = {{A Taxonomy of Internet Censorship and Anti-censorship}},
	Year = {2010}}

@techreport{ultrasurf-analysis,
	Author = {Appelbaum, Jacob},
	Institution = {The Tor Project},
	Title = {{Technical analysis of the Ultrasurf proxying software}},
	Url = {http://scholar.google.com/scholar?hl=en\&btnG=Search\&q=intitle:Technical+analysis+of+the+Ultrasurf+proxying+software\#0},
	Year = {2012},
	Bdsk-Url-1 = {http://scholar.google.com/scholar?hl=en%5C&btnG=Search%5C&q=intitle:Technical+analysis+of+the+Ultrasurf+proxying+software%5C#0}}

@misc{gifc:07,
	Howpublished = {\url{http://www.internetfreedom.org/archive/Defeat\_Internet\_Censorship\_White\_Paper.pdf}},
	Key = {defeatcensorship},
	Publisher = {Global Internet Freedom Consortium (GIFC)},
	Title = {{Defeat Internet Censorship: Overview of Advanced Technologies and Products}},
	Type = {White Paper},
	Year = {2007}}

@article{pan2011survey,
	Author = {Pan, J. and Paul, S. and Jain, R.},
	Journal = {Communications Magazine, IEEE},
	Number = {7},
	Pages = {26--36},
	Publisher = {IEEE},
	Title = {{A Survey of the Research on Future Internet Architectures}},
	Volume = {49},
	Year = {2011}}

@misc{nsf-fia,
	Howpublished = {\url{http://www.nets-fia.net/}},
	Key = {FIA},
	Title = {{NSF Future Internet Architecture Project}}}

@misc{NDN,
	Howpublished = {\url{http://www.named- data.net}},
	Key = {NDN},
	Title = {{Named Data Networking Project}}}

@inproceedings{MobilityFirst,
	Author = {Seskar, I. and Nagaraja, K. and Nelson, S. and Raychaudhuri, D.},
	Booktitle = {Asian Internet Engineering Conference},
	Title = {{Mobilityfirst Future internet Architecture Project}},
	Year = {2011}}

@incollection{NEBULA,
	Author = {Anderson, T. and Birman, K. and Broberg, R. and Caesar, M. and Comer, D. and Cotton, C. and Freedman, M.~J. and Haeberlen, A. and Ives, Z.~G. and Krishnamurthy, A. and others},
	Booktitle = {The Future Internet},
	Pages = {16--26},
	Publisher = {Springer},
	Title = {{The NEBULA Future Internet Architecture}},
	Year = {2013}}

@inproceedings{XIA,
	Author = {Anand, A. and Dogar, F. and Han, D. and Li, B. and Lim, H. and Machado, M. and Wu, W. and Akella, A. and Andersen, D.~G. and Byers, J.~W. and others},
	Booktitle = {ACM Workshop on Hot Topics in Networks},
	Pages = {2},
	Title = {{XIA: An Architecture for an Evolvable and Trustworthy Internet}},
	Year = {2011}}

@inproceedings{ChoiceNet,
	Author = {Rouskas, G.~N. and Baldine, I. and Calvert, K.~L. and Dutta, R. and Griffioen, J. and Nagurney, A. and Wolf, T.},
	Booktitle = {ONDM},
	Title = {{ChoiceNet: Network Innovation Through Choice}},
	Year = {2013}}

@misc{nsf-find,
	Howpublished = {http://www.nets-find.net/},
	Title = {{NSF NeTS FIND Initiative}}}

@article{traid,
	Author = {Cheriton, D.~R. and Gritter, M.},
	Title = {{TRIAD: A New Next-Generation Internet Architecture}},
	Year = {2000}}

@inproceedings{dona,
	Author = {Koponen, T. and Chawla, M. and Chun, B-G. and Ermolinskiy, A. and Kim, K.~H. and Shenker, S. and Stoica, I.},
	Booktitle = {ACM SIGCOMM Computer Communication Review},
	Number = {4},
	Organization = {ACM},
	Pages = {181--192},
	Title = {{A Data-Oriented (and Beyond) Network Architecture}},
	Volume = {37},
	Year = {2007}}

@misc{ultrasurf,
	Howpublished = {\url{http://www.ultrareach.com}},
	Key = {ultrasurf},
	Title = {{Ultrasurf}}}

@misc{tor-bridge,
	Author = {Dingledine, R. and Mathewson, N.},
	Howpublished = {\url{https://svn.torproject.org/svn/projects/design-paper/blocking.html}},
	Title = {{Design of a Blocking-Resistant Anonymity System}}}

@inproceedings{McLachlanH09,
	Author = {J. McLachlan and N. Hopper},
	Booktitle = {WPES},
	Title = {{On the Risks of Serving Whenever You Surf: Vulnerabilities in Tor's Blocking Resistance Design}},
	Year = {2009}}

@inproceedings{mahdian2010,
	Author = {Mahdian, M.},
	Booktitle = {{Fun with Algorithms}},
	Title = {{Fighting Censorship with Algorithms}},
	Year = {2010}}

@inproceedings{McCoy2011,
	Author = {McCoy, D. and Morales, J.~A. and Levchenko, K.},
	Booktitle = {FC},
	Title = {{Proximax: A Measurement Based System for Proxies Dissemination}},
	Year = {2011}}

@inproceedings{Sovran2008,
	Author = {Sovran, Y. and Libonati, A. and Li, J.},
	Booktitle = {IPTPS},
	Title = {{Pass it on: Social Networks Stymie Censors}},
	Year = {2008}}

@inproceedings{rbridge,
	Author = {Wang, Q. and Lin, Zi and Borisov, N. and Hopper, N.},
	Booktitle = {{NDSS}},
	Title = {{rBridge: User Reputation based Tor Bridge Distribution with Privacy Preservation}},
	Year = {2013}}

@inproceedings{telex,
	Author = {Wustrow, E. and Wolchok, S. and Goldberg, I. and Halderman, J.},
	Booktitle = {{USENIX Security}},
	Title = {{Telex: Anticensorship in the Network Infrastructure}},
	Year = {2011}}

@inproceedings{cirripede,
	Author = {Houmansadr, A. and Nguyen, G. and Caesar, M. and Borisov, N.},
	Booktitle = {CCS},
	Title = {{Cirripede: Circumvention Infrastructure Using Router Redirection with Plausible Deniability}},
	Year = {2011}}

@inproceedings{decoyrouting,
	Author = {Karlin, J. and Ellard, D. and Jackson, A. and Jones, C. and Lauer, G. and Mankins, D. and Strayer, W.},
	Booktitle = {{FOCI}},
	Title = {{Decoy Routing: Toward Unblockable Internet Communication}},
	Year = {2011}}

@inproceedings{routing-around-decoys,
	Author = {M.~Schuchard and J.~Geddes and C.~Thompson and N.~Hopper},
	Booktitle = {{CCS}},
	Title = {{Routing Around Decoys}},
	Year = {2012}}

@inproceedings{parrot,
	Author = {A. Houmansadr and C. Brubaker and V. Shmatikov},
	Booktitle = {IEEE S\&P},
	Title = {{The Parrot is Dead: Observing Unobservable Network Communications}},
	Year = {2013}}

@misc{knock,
	Author = {T. Wilde},
	Howpublished = {\url{https://blog.torproject.org/blog/knock-knock-knockin-bridges-doors}},
	Title = {{Knock Knock Knockin' on Bridges' Doors}},
	Year = {2012}}

@inproceedings{china-tor,
	Author = {Winter, P. and Lindskog, S.},
	Booktitle = {{FOCI}},
	Title = {{How the Great Firewall of China Is Blocking Tor}},
	Year = {2012}}

@misc{discover-bridge,
	Howpublished = {\url{https://blog.torproject.org/blog/research-problems-ten-ways-discover-tor-bridges}},
	Key = {tenways},
	Title = {{Ten Ways to Discover Tor Bridges}}}

@inproceedings{freewave,
	Author = {A.~Houmansadr and T.~Riedl and N.~Borisov and A.~Singer},
	Booktitle = {{NDSS}},
	Title = {{I Want My Voice to Be Heard: IP over Voice-over-IP for Unobservable Censorship Circumvention}},
	Year = 2013}

@inproceedings{censorspoofer,
	Author = {Q. Wang and X. Gong and G. Nguyen and A. Houmansadr and N. Borisov},
	Booktitle = {CCS},
	Title = {{CensorSpoofer: Asymmetric Communication Using IP Spoofing for Censorship-Resistant Web Browsing}},
	Year = {2012}}

@inproceedings{skypemorph,
	Author = {H. Moghaddam and B. Li and M. Derakhshani and I. Goldberg},
	Booktitle = {CCS},
	Title = {{SkypeMorph: Protocol Obfuscation for Tor Bridges}},
	Year = {2012}}

@inproceedings{stegotorus,
	Author = {Weinberg, Z. and Wang, J. and Yegneswaran, V. and Briesemeister, L. and Cheung, S. and Wang, F. and Boneh, D.},
	Booktitle = {CCS},
	Title = {{StegoTorus: A Camouflage Proxy for the Tor Anonymity System}},
	Year = {2012}}

@techreport{dust,
	Author = {{B.~Wiley}},
	Howpublished = {\url{http://blanu.net/ Dust.pdf}},
	Institution = {School of Information, University of Texas at Austin},
	Title = {{Dust: A Blocking-Resistant Internet Transport Protocol}},
	Year = {2011}}

@inproceedings{FTE,
	Author = {K.~Dyer and S.~Coull and T.~Ristenpart and T.~Shrimpton},
	Booktitle = {CCS},
	Title = {{Protocol Misidentification Made Easy with Format-Transforming Encryption}},
	Year = {2013}}

@inproceedings{fp,
	Author = {Fifield, D. and Hardison, N. and Ellithrope, J. and Stark, E. and Dingledine, R. and Boneh, D. and Porras, P.},
	Booktitle = {PETS},
	Title = {{Evading Censorship with Browser-Based Proxies}},
	Year = {2012}}

@misc{obfsproxy,
	Howpublished = {\url{https://www.torproject.org/projects/obfsproxy.html.en}},
	Key = {obfsproxy},
	Publisher = {The Tor Project},
	Title = {{A Simple Obfuscating Proxy}}}

@inproceedings{Tor-instead-of-IP,
	Author = {Liu, V. and Han, S. and Krishnamurthy, A. and Anderson, T.},
	Booktitle = {HotNets},
	Title = {{Tor instead of IP}},
	Year = {2011}}

@misc{roger-slides,
	Howpublished = {\url{https://svn.torproject.org/svn/projects/presentations/slides-28c3.pdf}},
	Key = {torblocking},
	Title = {{How Governments Have Tried to Block Tor}}}

@inproceedings{infranet,
	Author = {Feamster, N. and Balazinska, M. and Harfst, G. and Balakrishnan, H. and Karger, D.},
	Booktitle = {USENIX Security},
	Title = {{Infranet: Circumventing Web Censorship and Surveillance}},
	Year = {2002}}

@inproceedings{collage,
	Author = {S.~Burnett and N.~Feamster and S.~Vempala},
	Booktitle = {USENIX Security},
	Title = {{Chipping Away at Censorship Firewalls with User-Generated Content}},
	Year = {2010}}

@article{anonymizer,
	Author = {Boyan, J.},
	Journal = {Computer-Mediated Communication Magazine},
	Month = sep,
	Number = {9},
	Title = {{The Anonymizer: Protecting User Privacy on the Web}},
	Volume = {4},
	Year = {1997}}

@article{schulze2009internet,
	Author = {Schulze, H. and Mochalski, K.},
	Journal = {IPOQUE Report},
	Pages = {351--362},
	Title = {Internet Study 2008/2009},
	Volume = {37},
	Year = {2009}}

@inproceedings{cya-ccs13,
	Author = {J.~Geddes and M.~Schuchard and N.~Hopper},
	Booktitle = {{CCS}},
	Title = {{Cover Your ACKs: Pitfalls of Covert Channel Censorship Circumvention}},
	Year = {2013}}

@inproceedings{andana,
	Author = {DiBenedetto, S. and Gasti, P. and Tsudik, G. and Uzun, E.},
	Booktitle = {{NDSS}},
	Title = {{ANDaNA: Anonymous Named Data Networking Application}},
	Year = {2012}}

@inproceedings{darkly,
	Author = {Jana, S. and Narayanan, A. and Shmatikov, V.},
	Booktitle = {IEEE S\&P},
	Title = {{A Scanner Darkly: Protecting User Privacy From Perceptual Applications}},
	Year = {2013}}

@inproceedings{NS08,
	Author = {A.~Narayanan and V.~Shmatikov},
	Booktitle = {IEEE S\&P},
	Title = {Robust de-anonymization of large sparse datasets},
	Year = {2008}}

@inproceedings{NS09,
	Author = {A.~Narayanan and V.~Shmatikov},
	Booktitle = {IEEE S\&P},
	Title = {De-anonymizing social networks},
	Year = {2009}}

@inproceedings{memento,
	Author = {Jana, S. and Shmatikov, V.},
	Booktitle = {IEEE S\&P},
	Title = {{Memento: Learning secrets from process footprints}},
	Year = {2012}}

@misc{plugtor,
	Howpublished = {\url{https://www.torproject.org/docs/pluggable-transports.html.en}},
	Key = {PluggableTransports},
	Publisher = {The Tor Project},
	Title = {{Tor: Pluggable transports}}}

@misc{psiphon,
	Author = {J.~Jia and P.~Smith},
	Howpublished = {\url{http://www.cdf.toronto.edu/~csc494h/reports/2004-fall/psiphon_ae.html}},
	Title = {{Psiphon: Analysis and Estimation}},
	Year = 2004}

@misc{china-github,
	Howpublished = {\url{http://mobile.informationweek.com/80269/show/72e30386728f45f56b343ddfd0fdb119/}},
	Key = {github},
	Title = {{China's GitHub Censorship Dilemma}}}

@inproceedings{txbox,
	Author = {Jana, S. and Porter, D. and Shmatikov, V.},
	Booktitle = {IEEE S\&P},
	Title = {{TxBox: Building Secure, Efficient Sandboxes with System Transactions}},
	Year = {2011}}

@inproceedings{airavat,
	Author = {I. Roy and S. Setty and A. Kilzer and V. Shmatikov and E. Witchel},
	Booktitle = {NSDI},
	Title = {{Airavat: Security and Privacy for MapReduce}},
	Year = {2010}}

@inproceedings{osdi12,
	Author = {A. Dunn and M. Lee and S. Jana and S. Kim and M. Silberstein and Y. Xu and V. Shmatikov and E. Witchel},
	Booktitle = {OSDI},
	Title = {{Eternal Sunshine of the Spotless Machine: Protecting Privacy with Ephemeral Channels}},
	Year = {2012}}

@inproceedings{ymal,
	Author = {J. Calandrino and A. Kilzer and A. Narayanan and E. Felten and V. Shmatikov},
	Booktitle = {IEEE S\&P},
	Title = {{``You Might Also Like:'' Privacy Risks of Collaborative Filtering}},
	Year = {2011}}

@inproceedings{srivastava11,
	Author = {V. Srivastava and M. Bond and K. McKinley and V. Shmatikov},
	Booktitle = {PLDI},
	Title = {{A Security Policy Oracle: Detecting Security Holes Using Multiple API Implementations}},
	Year = {2011}}

@inproceedings{chen-oakland10,
	Author = {Chen, S. and Wang, R. and Wang, X. and Zhang, K.},
	Booktitle = {IEEE S\&P},
	Title = {{Side-Channel Leaks in Web Applications: A Reality Today, a Challenge Tomorrow}},
	Year = {2010}}

@book{kerck,
	Author = {Kerckhoffs, A.},
	Publisher = {University Microfilms},
	Title = {{La cryptographie militaire}},
	Year = {1978}}

@inproceedings{foci11,
	Author = {J. Karlin and D. Ellard and A.~Jackson and C.~ Jones and G. Lauer and D. Mankins and W.~T.~Strayer},
	Booktitle = {FOCI},
	Title = {{Decoy Routing: Toward Unblockable Internet Communication}},
	Year = 2011}

@inproceedings{sun02,
	Author = {Sun, Q. and Simon, D.~R. and Wang, Y. and Russell, W. and Padmanabhan, V. and Qiu, L.},
	Booktitle = {IEEE S\&P},
	Title = {{Statistical Identification of Encrypted Web Browsing Traffic}},
	Year = {2002}}

@inproceedings{danezis,
	Author = {Murdoch, S.~J. and Danezis, G.},
	Booktitle = {IEEE S\&P},
	Title = {{Low-Cost Traffic Analysis of Tor}},
	Year = {2005}}

@inproceedings{pakicensorship,
	Author = {Z.~Nabi},
	Booktitle = {FOCI},
	Title = {The Anatomy of {Web} Censorship in {Pakistan}},
	Year = {2013}}

@inproceedings{irancensorship,
	Author = {S.~Aryan and H.~Aryan and A.~Halderman},
	Booktitle = {FOCI},
	Title = {Internet Censorship in {Iran}: {A} First Look},
	Year = {2013}}

@inproceedings{ford10efficient,
	Author = {Amittai Aviram and Shu-Chun Weng and Sen Hu and Bryan Ford},
	Booktitle = {\bibconf[9th]{OSDI}{USENIX Symposium on Operating Systems Design and Implementation}},
	Location = {Vancouver, BC, Canada},
	Month = oct,
	Title = {Efficient System-Enforced Deterministic Parallelism},
	Year = 2010}

@inproceedings{ford10determinating,
	Author = {Amittai Aviram and Sen Hu and Bryan Ford and Ramakrishna Gummadi},
	Booktitle = {\bibconf{CCSW}{ACM Cloud Computing Security Workshop}},
	Location = {Chicago, IL},
	Month = oct,
	Title = {Determinating Timing Channels in Compute Clouds},
	Year = 2010}

@inproceedings{ford12plugging,
	Author = {Bryan Ford},
	Booktitle = {\bibconf[4th]{HotCloud}{USENIX Workshop on Hot Topics in Cloud Computing}},
	Location = {Boston, MA},
	Month = jun,
	Title = {Plugging Side-Channel Leaks with Timing Information Flow Control},
	Year = 2012}

@inproceedings{ford12icebergs,
	Author = {Bryan Ford},
	Booktitle = {\bibconf[4th]{HotCloud}{USENIX Workshop on Hot Topics in Cloud Computing}},
	Location = {Boston, MA},
	Month = jun,
	Title = {Icebergs in the Clouds: the {\em Other} Risks of Cloud Computing},
	Year = 2012}

@misc{mullenize,
	Author = {Washington Post},
	Howpublished = {\url{http://apps.washingtonpost.com/g/page/world/gchq-report-on-mullenize-program-to-stain-anonymous-electronic-traffic/502/}},
	Month = {oct},
	Title = {{GCHQ} report on {`MULLENIZE'} program to `stain' anonymous electronic traffic},
	Year = {2013}}

@inproceedings{shue13street,
	Author = {Craig A. Shue and Nathanael Paul and Curtis R. Taylor},
	Booktitle = {\bibbrev[7th]{WOOT}{USENIX Workshop on Offensive Technologies}},
	Month = aug,
	Title = {From an {IP} Address to a Street Address: Using Wireless Signals to Locate a Target},
	Year = 2013}

@inproceedings{knockel11three,
	Author = {Jeffrey Knockel and Jedidiah R. Crandall and Jared Saia},
	Booktitle = {\bibbrev{FOCI}{USENIX Workshop on Free and Open Communications on the Internet}},
	Location = {San Francisco, CA},
	Month = aug,
	Year = 2011}

@misc{rfc4960,
	Author = {R. {Stewart, ed.}},
	Month = sep,
	Note = {RFC 4960},
	Title = {Stream Control Transmission Protocol},
	Year = 2007}

@inproceedings{ford07structured,
	Author = {Bryan Ford},
	Booktitle = {\bibbrev{SIGCOMM}{ACM SIGCOMM}},
	Location = {Kyoto, Japan},
	Month = aug,
	Title = {Structured Streams: a New Transport Abstraction},
	Year = {2007}}

@misc{spdy,
	Author = {Google, Inc.},
	Note = {\url{http://www.chromium.org/spdy/spdy-whitepaper}},
	Title = {{SPDY}: An Experimental Protocol For a Faster {Web}}}

@misc{quic,
	Author = {Jim Roskind},
	Month = jun,
	Note = {\url{http://blog.chromium.org/2013/06/experimenting-with-quic.html}},
	Title = {Experimenting with {QUIC}},
	Year = 2013}

@misc{podjarny12not,
	Author = {G.~Podjarny},
	Month = jun,
	Note = {\url{http://www.guypo.com/technical/not-as-spdy-as-you-thought/}},
	Title = {{Not as SPDY as You Thought}},
	Year = 2012}

@inproceedings{cor,
	Author = {Jones, N.~A. and Arye, M. and Cesareo, J. and Freedman, M.~J.},
	Booktitle = {FOCI},
	Title = {{Hiding Amongst the Clouds: A Proposal for Cloud-based Onion Routing}},
	Year = {2011}}

@misc{torcloud,
	Howpublished = {\url{https://cloud.torproject.org/}},
	Key = {tor cloud},
	Title = {{The Tor Cloud Project}}}

@inproceedings{scramblesuit,
	Author = {Philipp Winter and Tobias Pulls and Juergen Fuss},
	Booktitle = {WPES},
	Title = {{ScrambleSuit: A Polymorphic Network Protocol to Circumvent Censorship}},
	Year = 2013}

@article{savage2000practical,
	Author = {Savage, S. and Wetherall, D. and Karlin, A. and Anderson, T.},
	Journal = {ACM SIGCOMM Computer Communication Review},
	Number = {4},
	Pages = {295--306},
	Publisher = {ACM},
	Title = {Practical network support for IP traceback},
	Volume = {30},
	Year = {2000}}

@inproceedings{ooni,
	Author = {Filast, A. and Appelbaum, J.},
	Booktitle = {{FOCI}},
	Title = {{OONI : Open Observatory of Network Interference}},
	Year = {2012}}

@misc{caida-rank,
	Howpublished = {\url{http://as-rank.caida.org/}},
	Key = {caida rank},
	Title = {{AS Rank: AS Ranking}}}

@inproceedings{usersrouted-ccs13,
	Author = {A.~Johnson and C.~Wacek and R.~Jansen and M.~Sherr and P.~Syverson},
	Booktitle = {CCS},
	Title = {{Users Get Routed: Traffic Correlation on Tor by Realistic Adversaries}},
	Year = {2013}}

@inproceedings{edman2009awareness,
	Author = {Edman, M. and Syverson, P.},
	Booktitle = {{CCS}},
	Title = {{AS-awareness in Tor path selection}},
	Year = {2009}}

@inproceedings{DecoyCosts,
	Author = {A.~Houmansadr and E.~L.~Wong and V.~Shmatikov},
	Booktitle = {NDSS},
	Title = {{No Direction Home: The True Cost of Routing Around Decoys}},
	Year = {2014}}

@article{cordon,
	Author = {Elahi, T. and Goldberg, I.},
	Journal = {University of Waterloo CACR},
	Title = {{CORDON--A Taxonomy of Internet Censorship Resistance Strategies}},
	Volume = {33},
	Year = {2012}}

@inproceedings{privex,
	Author = {T.~Elahi and G.~Danezis and I.~Goldberg	},
	Booktitle = {{CCS}},
	Title = {{AS-awareness in Tor path selection}},
	Year = {2014}}

@inproceedings{changeGuards,
	Author = {T.~Elahi and K.~Bauer and M.~AlSabah and R.~Dingledine and I.~Goldberg},
	Booktitle = {{WPES}},
	Title = {{ Changing of the Guards: Framework for Understanding and Improving Entry Guard Selection in Tor}},
	Year = {2012}}

@article{RAINBOW:Journal,
	Author = {A.~Houmansadr and N.~Kiyavash and N.~Borisov},
	Journal = {IEEE/ACM Transactions on Networking},
	Title = {{Non-Blind Watermarking of Network Flows}},
	Year = 2014}

@inproceedings{info-tod,
	Author = {A.~Houmansadr and S.~Gorantla and T.~Coleman and N.~Kiyavash and and N.~Borisov},
	Booktitle = {{CCS (poster session)}},
	Title = {{On the Channel Capacity of Network Flow Watermarking}},
	Year = {2009}}

@inproceedings{johnson2014game,
	Author = {Johnson, B. and Laszka, A. and Grossklags, J. and Vasek, M. and Moore, T.},
	Booktitle = {Workshop on Bitcoin Research},
	Title = {{Game-theoretic Analysis of DDoS Attacks Against Bitcoin Mining Pools}},
	Year = {2014}}

@incollection{laszka2013mitigation,
	Author = {Laszka, A. and Johnson, B. and Grossklags, J.},
	Booktitle = {Decision and Game Theory for Security},
	Pages = {175--191},
	Publisher = {Springer},
	Title = {{Mitigation of Targeted and Non-targeted Covert Attacks as a Timing Game}},
	Year = {2013}}

@inproceedings{schottle2013game,
	Author = {Schottle, P. and Laszka, A. and Johnson, B. and Grossklags, J. and Bohme, R.},
	Booktitle = {EUSIPCO},
	Title = {{A Game-theoretic Analysis of Content-adaptive Steganography with Independent Embedding}},
	Year = {2013}}

@inproceedings{CloudTransport,
	Author = {C.~Brubaker and A.~Houmansadr and V.~Shmatikov},
	Booktitle = {PETS},
	Title = {{CloudTransport: Using Cloud Storage for Censorship-Resistant Networking}},
	Year = {2014}}

@inproceedings{sweet,
	Author = {W.~Zhou and A.~Houmansadr and M.~Caesar and N.~Borisov},
	Booktitle = {HotPETs},
	Title = {{SWEET: Serving the Web by Exploiting Email Tunnels}},
	Year = {2013}}

@inproceedings{ahsan2002practical,
	Author = {Ahsan, K. and Kundur, D.},
	Booktitle = {Workshop on Multimedia Security},
	Title = {{Practical data hiding in TCP/IP}},
	Year = {2002}}

@incollection{danezis2011covert,
	Author = {Danezis, G.},
	Booktitle = {Security Protocols XVI},
	Pages = {198--214},
	Publisher = {Springer},
	Title = {{Covert Communications Despite Traffic Data Retention}},
	Year = {2011}}

@inproceedings{liu2009hide,
	Author = {Liu, Y. and Ghosal, D. and Armknecht, F. and Sadeghi, A.-R. and Schulz, S. and Katzenbeisser, S.},
	Booktitle = {ESORICS},
	Title = {{Hide and Seek in Time---Robust Covert Timing Channels}},
	Year = {2009}}

@misc{image-watermark-fing,
	Author = {Jonathan Bailey},
	Howpublished = {\url{https://www.plagiarismtoday.com/2009/12/02/image-detection-watermarking-vs-fingerprinting/}},
	Title = {{Image Detection: Watermarking vs. Fingerprinting}},
	Year = {2009}}

@inproceedings{Servetto98,
	Author = {S. D. Servetto and C. I. Podilchuk and K. Ramchandran},
	Booktitle = {Int. Conf. Image Processing},
	Title = {Capacity issues in digital image watermarking},
	Year = {1998}}

@inproceedings{Chen01,
	Author = {B. Chen and G.W.Wornell},
	Booktitle = {IEEE Trans. Inform. Theory},
	Pages = {1423--1443},
	Title = {Quantization index modulation: A class of provably good methods for digital watermarking and information embedding},
	Year = {2001}}

@inproceedings{Karakos00,
	Author = {D. Karakos and A. Papamarcou},
	Booktitle = {IEEE Int. Symp. Information Theory},
	Pages = {47},
	Title = {Relationship between quantization and distribution rates of digitally watermarked data},
	Year = {2000}}

@inproceedings{Sullivan98,
	Author = {J. A. OSullivan and P. Moulin and J. M. Ettinger},
	Booktitle = {IEEE Int. Symp. Information Theory},
	Pages = {297},
	Title = {Information theoretic analysis of steganography},
	Year = {1998}}

@inproceedings{Merhav00,
	Author = {N. Merhav},
	Booktitle = {IEEE Trans. Inform. Theory},
	Pages = {420--430},
	Title = {On random coding error exponents of watermarking systems},
	Year = {2000}}

@inproceedings{Somekh01,
	Author = {A. Somekh-Baruch and N. Merhav},
	Booktitle = {IEEE Int. Symp. Information Theory},
	Pages = {7},
	Title = {On the error exponent and capacity games of private watermarking systems},
	Year = {2001}}

@inproceedings{Steinberg01,
	Author = {Y. Steinberg and N. Merhav},
	Booktitle = {IEEE Trans. Inform. Theory},
	Pages = {1410--1422},
	Title = {Identification in the presence of side information with application to watermarking},
	Year = {2001}}

@article{Moulin03,
	Author = {P. Moulin and J.A. O'Sullivan},
	Journal = {IEEE Trans. Info. Theory},
	Number = {3},
	Title = {Information-theoretic analysis of information hiding},
	Volume = 49,
	Year = 2003}

@article{Gelfand80,
	Author = {S.I.~Gelfand and M.S.~Pinsker},
	Journal = {Problems of Control and Information Theory},
	Number = {1},
	Pages = {19-31},
	Title = {{Coding for channel with random parameters}},
	Url = {citeseer.ist.psu.edu/anantharam96bits.html},
	Volume = {9},
	Year = {1980},
	Bdsk-Url-1 = {citeseer.ist.psu.edu/anantharam96bits.html}}

@book{Wolfowitz78,
	Author = {J. Wolfowitz},
	Edition = {3rd},
	Location = {New York},
	Publisher = {Springer-Verlag},
	Title = {Coding Theorems of Information Theory},
	Year = 1978}

@article{caire99,
	Author = {G. Caire and S. Shamai},
	Journal = {IEEE Transactions on Information Theory},
	Number = {6},
	Pages = {2007--2019},
	Title = {On the Capacity of Some Channels with Channel State Information},
	Volume = {45},
	Year = {1999}}

@inproceedings{wright2007language,
	Author = {Wright, Charles V and Ballard, Lucas and Monrose, Fabian and Masson, Gerald M},
	Booktitle = {USENIX Security},
	Title = {{Language identification of encrypted VoIP traffic: Alejandra y Roberto or Alice and Bob?}},
	Year = {2007}}

@inproceedings{backes2010speaker,
	Author = {Backes, Michael and Doychev, Goran and D{\"u}rmuth, Markus and K{\"o}pf, Boris},
	Booktitle = {{European Symposium on Research in Computer Security (ESORICS)}},
	Pages = {508--523},
	Publisher = {Springer},
	Title = {{Speaker Recognition in Encrypted Voice Streams}},
	Year = {2010}}

@phdthesis{lu2009traffic,
	Author = {Lu, Yuanchao},
	School = {Cleveland State University},
	Title = {{On Traffic Analysis Attacks to Encrypted VoIP Calls}},
	Year = {2009}}

@inproceedings{wright2008spot,
	Author = {Wright, Charles V and Ballard, Lucas and Coull, Scott E and Monrose, Fabian and Masson, Gerald M},
	Booktitle = {IEEE Symposium on Security and Privacy},
	Pages = {35--49},
	Title = {Spot me if you can: Uncovering spoken phrases in encrypted VoIP conversations},
	Year = {2008}}

@inproceedings{white2011phonotactic,
	Author = {White, Andrew M and Matthews, Austin R and Snow, Kevin Z and Monrose, Fabian},
	Booktitle = {IEEE Symposium on Security and Privacy},
	Pages = {3--18},
	Title = {Phonotactic reconstruction of encrypted VoIP conversations: Hookt on fon-iks},
	Year = {2011}}

@inproceedings{fancy,
	Author = {Houmansadr, Amir and Borisov, Nikita},
	Booktitle = {Privacy Enhancing Technologies},
	Organization = {Springer},
	Pages = {205--224},
	Title = {The Need for Flow Fingerprints to Link Correlated Network Flows},
	Year = {2013}}

@article{botmosaic,
	Author = {Amir Houmansadr and Nikita Borisov},
	Doi = {10.1016/j.jss.2012.11.005},
	Issn = {0164-1212},
	Journal = {Journal of Systems and Software},
	Keywords = {Network security},
	Number = {3},
	Pages = {707 - 715},
	Title = {BotMosaic: Collaborative network watermark for the detection of IRC-based botnets},
	Url = {http://www.sciencedirect.com/science/article/pii/S0164121212003068},
	Volume = {86},
	Year = {2013},
	Bdsk-Url-1 = {http://www.sciencedirect.com/science/article/pii/S0164121212003068},
	Bdsk-Url-2 = {http://dx.doi.org/10.1016/j.jss.2012.11.005}}

@inproceedings{ramsbrock2008first,
	Author = {Ramsbrock, Daniel and Wang, Xinyuan and Jiang, Xuxian},
	Booktitle = {Recent Advances in Intrusion Detection},
	Organization = {Springer},
	Pages = {59--77},
	Title = {A first step towards live botmaster traceback},
	Year = {2008}}

@inproceedings{potdar2005survey,
	Author = {Potdar, Vidyasagar M and Han, Song and Chang, Elizabeth},
	Booktitle = {Industrial Informatics, 2005. INDIN'05. 2005 3rd IEEE International Conference on},
	Organization = {IEEE},
	Pages = {709--716},
	Title = {A survey of digital image watermarking techniques},
	Year = {2005}}

@book{cole2003hiding,
	Author = {Cole, Eric and Krutz, Ronald D},
	Publisher = {John Wiley \& Sons, Inc.},
	Title = {Hiding in plain sight: Steganography and the art of covert communication},
	Year = {2003}}

@incollection{akaike1998information,
	Author = {Akaike, Hirotogu},
	Booktitle = {Selected Papers of Hirotugu Akaike},
	Pages = {199--213},
	Publisher = {Springer},
	Title = {Information theory and an extension of the maximum likelihood principle},
	Year = {1998}}

@misc{central-command-hack,
	Author = {Everett Rosenfeld},
	Howpublished = {\url{http://www.cnbc.com/id/102330338}},
	Title = {{FBI investigating Central Command Twitter hack}},
	Year = {2015}}

@misc{sony-psp-ddos,
	Howpublished = {\url{http://n4g.com/news/1644853/sony-and-microsoft-cant-do-much-ddos-attacks-explained}},
	Key = {sony},
	Month = {December},
	Title = {{Sony and Microsoft cant do much -- DDoS attacks explained}},
	Year = {2014}}

@misc{sony-hack,
	Author = {David Bloom},
	Howpublished = {\url{http://goo.gl/MwR4A7}},
	Title = {{Online Game Networks Hacked, Sony Unit President Threatened}},
	Year = {2014}}

@misc{home-depot,
	Author = {Dune Lawrence},
	Howpublished = {\url{http://www.businessweek.com/articles/2014-09-02/home-depots-credit-card-breach-looks-just-like-the-target-hack}},
	Title = {{Home Depot's Suspected Breach Looks Just Like the Target Hack}},
	Year = {2014}}

@misc{target,
	Author = {Julio Ojeda-Zapata},
	Howpublished = {\url{http://www.mercurynews.com/business/ci_24765398/how-did-hackers-pull-off-target-data-heist}},
	Title = {{Target hack: How did they do it?}},
	Year = {2014}}


@article{probabilitycourse,
	Author = {H. Pishro-Nik},
	note = {\url{http://www.probabilitycourse.com}},
	Title = {Introduction to probability, statistics, and random processes},
    Year = {2014}}

@article{our-isit-location,
	Author = {Z. Montazeri and A. Houmansadr and H. Pishro-Nik},
	Journal = {To be submitted to IEEE ISIT},
	Title = {Location Privacy for Multi-State Networks},
	Year = {2016}}






@article{kafsi2013entropy,
	Author = {Kafsi, Mohamed and Grossglauser, Matthias and Thiran, Patrick},
	Journal = {Information Theory, IEEE Transactions on},
	Number = {9},
	Pages = {5577--5583},
	Publisher = {IEEE},
	Title = {The entropy of conditional Markov trajectories},
	Volume = {59},
	Year = {2013}}

@inproceedings{gruteser2003anonymous,
	Author = {Gruteser, Marco and Grunwald, Dirk},
	Booktitle = {Proceedings of the 1st international conference on Mobile systems, applications and services},
	Organization = {ACM},
	Pages = {31--42},
	Title = {Anonymous usage of location-based services through spatial and temporal cloaking},
	Year = {2003}}

@inproceedings{husted2010mobile,
	Author = {Husted, Nathaniel and Myers, Steven},
	Booktitle = {Proceedings of the 17th ACM conference on Computer and communications security},
	Organization = {ACM},
	Pages = {85--96},
	Title = {Mobile location tracking in metro areas: malnets and others},
	Year = {2010}}

@inproceedings{li2009tradeoff,
	Author = {Li, Tiancheng and Li, Ninghui},
	Booktitle = {Proceedings of the 15th ACM SIGKDD international conference on Knowledge discovery and data mining},
	Organization = {ACM},
	Pages = {517--526},
	Title = {On the tradeoff between privacy and utility in data publishing},
	Year = {2009}}





@incollection{humbert2010tracking,
	Author = {Humbert, Mathias and Manshaei, Mohammad Hossein and Freudiger, Julien and Hubaux, Jean-Pierre},
	Booktitle = {Decision and Game Theory for Security},
	Pages = {38--57},
	Publisher = {Springer},
	Title = {Tracking games in mobile networks},
	Year = {2010}}


@article{palamidessi2006probabilistic,
	Author = {Palamidessi, Catuscia},
	Journal = {Electronic Notes in Theoretical Computer Science},
	Pages = {33--42},
	Publisher = {Elsevier},
	Title = {Probabilistic and nondeterministic aspects of anonymity},
	Volume = {155},
	Year = {2006}}



@inproceedings{freudiger2007mix,
	Author = {Freudiger, Julien and Raya, Maxim and F{\'e}legyh{\'a}zi, M{\'a}rk and Papadimitratos, Panos and Hubaux, Jean-Pierre},
	Booktitle = {CM Workshop on Wireless Networking for Intelligent Transportation Systems (WiN-ITS)},
	Title = {Mix-zones for location privacy in vehicular networks},
	Year = {2007}}



@inproceedings{niu2014achieving,
	Author = {Niu, Ben and Li, Qinghua and Zhu, Xiaoyan and Cao, Guohong and Li, Hui},
	Booktitle = {INFOCOM, 2014 Proceedings IEEE},
	Organization = {IEEE},
	Pages = {754--762},
	Title = {Achieving k-anonymity in privacy-aware location-based services},
	Year = {2014}}



@inproceedings{kido2005protection,
	Author = {Kido, Hidetoshi and Yanagisawa, Yutaka and Satoh, Tetsuji},
	Booktitle = {Data Engineering Workshops, 2005. 21st International Conference on},
	Organization = {IEEE},
	Pages = {1248--1248},
	Title = {Protection of location privacy using dummies for location-based services},
	Year = {2005}}

@inproceedings{gedik2005location,
	Author = {Gedik, Bu{\u{g}}ra and Liu, Ling},
	Booktitle = {Distributed Computing Systems, 2005. ICDCS 2005. Proceedings. 25th IEEE International Conference on},
	Organization = {IEEE},
	Pages = {620--629},
	Title = {Location privacy in mobile systems: A personalized anonymization model},
	Year = {2005}}


@incollection{duckham2005formal,
	Author = {Duckham, Matt and Kulik, Lars},
	Booktitle = {Pervasive computing},
	Pages = {152--170},
	Publisher = {Springer},
	Title = {A formal model of obfuscation and negotiation for location privacy},
	Year = {2005}}

@inproceedings{kido2005anonymous,
	Author = {Kido, Hidetoshi and Yanagisawa, Yutaka and Satoh, Tetsuji},
	Booktitle = {Pervasive Services, 2005. ICPS'05. Proceedings. International Conference on},
	Organization = {IEEE},
	Pages = {88--97},
	Title = {An anonymous communication technique using dummies for location-based services},
	Year = {2005}}

@incollection{duckham2006spatiotemporal,
	Author = {Duckham, Matt and Kulik, Lars and Birtley, Athol},
	Booktitle = {Geographic Information Science},
	Pages = {47--64},
	Publisher = {Springer},
	Title = {A spatiotemporal model of strategies and counter strategies for location privacy protection},
	Year = {2006}}

@inproceedings{shankar2009privately,
	Author = {Shankar, Pravin and Ganapathy, Vinod and Iftode, Liviu},
	Booktitle = {Proceedings of the 11th international conference on Ubiquitous computing},
	Organization = {ACM},
	Pages = {31--40},
	Title = {Privately querying location-based services with SybilQuery},
	Year = {2009}}

@inproceedings{chow2009faking,
	Author = {Chow, Richard and Golle, Philippe},
	Booktitle = {Proceedings of the 8th ACM workshop on Privacy in the electronic society},
	Organization = {ACM},
	Pages = {105--108},
	Title = {Faking contextual data for fun, profit, and privacy},
	Year = {2009}}

@incollection{xue2009location,
	Author = {Xue, Mingqiang and Kalnis, Panos and Pung, Hung Keng},
	Booktitle = {Location and Context Awareness},
	Pages = {70--87},
	Publisher = {Springer},
	Title = {Location diversity: Enhanced privacy protection in location based services},
	Year = {2009}}

@article{wernke2014classification,
	Author = {Wernke, Marius and Skvortsov, Pavel and D{\"u}rr, Frank and Rothermel, Kurt},
	Journal = {Personal and Ubiquitous Computing},
	Number = {1},
	Pages = {163--175},
	Publisher = {Springer-Verlag},
	Title = {A classification of location privacy attacks and approaches},
	Volume = {18},
	Year = {2014}}

@misc{cai2015cloaking,
	Author = {Cai, Y. and Xu, G.},
	Month = jan # {~1},
	Note = {US Patent App. 14/472,462},
	Publisher = {Google Patents},
	Title = {Cloaking with footprints to provide location privacy protection in location-based services},
	Url = {https://www.google.com/patents/US20150007341},
	Year = {2015},
	Bdsk-Url-1 = {https://www.google.com/patents/US20150007341}}

@article{gedik2008protecting,
	Author = {Gedik, Bu{\u{g}}ra and Liu, Ling},
	Journal = {Mobile Computing, IEEE Transactions on},
	Number = {1},
	Pages = {1--18},
	Publisher = {IEEE},
	Title = {Protecting location privacy with personalized k-anonymity: Architecture and algorithms},
	Volume = {7},
	Year = {2008}}

@article{kalnis2006preserving,
	Author = {Kalnis, Panos and Ghinita, Gabriel and Mouratidis, Kyriakos and Papadias, Dimitris},
	Publisher = {TRB6/06},
	Title = {Preserving anonymity in location based services},
	Year = {2006}}



@article{terrovitis2011privacy,
	Author = {Terrovitis, Manolis},
	Journal = {ACM SIGKDD Explorations Newsletter},
	Number = {1},
	Pages = {6--18},
	Publisher = {ACM},
	Title = {Privacy preservation in the dissemination of location data},
	Volume = {13},
	Year = {2011}}

@article{shin2012privacy,
	Author = {Shin, Kang G and Ju, Xiaoen and Chen, Zhigang and Hu, Xin},
	Journal = {Wireless Communications, IEEE},
	Number = {1},
	Pages = {30--39},
	Publisher = {IEEE},
	Title = {Privacy protection for users of location-based services},
	Volume = {19},
	Year = {2012}}



@incollection{chatzikokolakis2015geo,
	Author = {Chatzikokolakis, Konstantinos and Palamidessi, Catuscia and Stronati, Marco},
	Booktitle = {Distributed Computing and Internet Technology},
	Pages = {49--72},
	Publisher = {Springer},
	Title = {Geo-indistinguishability: A Principled Approach to Location Privacy},
	Year = {2015}}

@inproceedings{ngo2015location,
	Author = {Ngo, Hoa and Kim, Jong},
	Booktitle = {Computer Security Foundations Symposium (CSF), 2015 IEEE 28th},
	Organization = {IEEE},
	Pages = {63--74},
	Title = {Location Privacy via Differential Private Perturbation of Cloaking Area},
	Year = {2015}}


@inproceedings{um2010advanced,
	Author = {Um, Jung-Ho and Kim, Hee-Dae and Chang, Jae-Woo},
	Booktitle = {Social Computing (SocialCom), 2010 IEEE Second International Conference on},
	Organization = {IEEE},
	Pages = {1093--1098},
	Title = {An advanced cloaking algorithm using Hilbert curves for anonymous location based service},
	Year = {2010}}

@inproceedings{bamba2008supporting,
	Author = {Bamba, Bhuvan and Liu, Ling and Pesti, Peter and Wang, Ting},
	Booktitle = {Proceedings of the 17th international conference on World Wide Web},
	Organization = {ACM},
	Pages = {237--246},
	Title = {Supporting anonymous location queries in mobile environments with privacygrid},
	Year = {2008}}

@inproceedings{zhangwei2010distributed,
	Author = {Zhangwei, Huang and Mingjun, Xin},
	Booktitle = {Networks Security Wireless Communications and Trusted Computing (NSWCTC), 2010 Second International Conference on},
	Organization = {IEEE},
	Pages = {468--471},
	Title = {A distributed spatial cloaking protocol for location privacy},
	Volume = {2},
	Year = {2010}}

@article{chow2011spatial,
	Author = {Chow, Chi-Yin and Mokbel, Mohamed F and Liu, Xuan},
	Journal = {GeoInformatica},
	Number = {2},
	Pages = {351--380},
	Publisher = {Springer},
	Title = {Spatial cloaking for anonymous location-based services in mobile peer-to-peer environments},
	Volume = {15},
	Year = {2011}}

@inproceedings{lu2008pad,
	Author = {Lu, Hua and Jensen, Christian S and Yiu, Man Lung},
	Booktitle = {Proceedings of the Seventh ACM International Workshop on Data Engineering for Wireless and Mobile Access},
	Organization = {ACM},
	Pages = {16--23},
	Title = {Pad: privacy-area aware, dummy-based location privacy in mobile services},
	Year = {2008}}

@incollection{khoshgozaran2007blind,
	Author = {Khoshgozaran, Ali and Shahabi, Cyrus},
	Booktitle = {Advances in Spatial and Temporal Databases},
	Pages = {239--257},
	Publisher = {Springer},
	Title = {Blind evaluation of nearest neighbor queries using space transformation to preserve location privacy},
	Year = {2007}}

@inproceedings{ghinita2008private,
	Author = {Ghinita, Gabriel and Kalnis, Panos and Khoshgozaran, Ali and Shahabi, Cyrus and Tan, Kian-Lee},
	Booktitle = {Proceedings of the 2008 ACM SIGMOD international conference on Management of data},
	Organization = {ACM},
	Pages = {121--132},
	Title = {Private queries in location based services: anonymizers are not necessary},
	Year = {2008}}



@article{nguyen2013differential,
	Author = {Nguyen, Hiep H and Kim, Jong and Kim, Yoonho},
	Journal = {Journal of Computing Science and Engineering},
	Number = {3},
	Pages = {177--186},
	Title = {Differential privacy in practice},
	Volume = {7},
	Year = {2013}}

@inproceedings{lee2012differential,
	Author = {Lee, Jaewoo and Clifton, Chris},
	Booktitle = {Proceedings of the 18th ACM SIGKDD international conference on Knowledge discovery and data mining},
	Organization = {ACM},
	Pages = {1041--1049},
	Title = {Differential identifiability},
	Year = {2012}}

@inproceedings{andres2013geo,
	Author = {Andr{\'e}s, Miguel E and Bordenabe, Nicol{\'a}s E and Chatzikokolakis, Konstantinos and Palamidessi, Catuscia},
	Booktitle = {Proceedings of the 2013 ACM SIGSAC conference on Computer \& communications security},
	Organization = {ACM},
	Pages = {901--914},
	Title = {Geo-indistinguishability: Differential privacy for location-based systems},
	Year = {2013}}

@inproceedings{machanavajjhala2008privacy,
	Author = {Machanavajjhala, Ashwin and Kifer, Daniel and Abowd, John and Gehrke, Johannes and Vilhuber, Lars},
	Booktitle = {Data Engineering, 2008. ICDE 2008. IEEE 24th International Conference on},
	Organization = {IEEE},
	Pages = {277--286},
	Title = {Privacy: Theory meets practice on the map},
	Year = {2008}}

@article{dewri2013local,
	Author = {Dewri, Rinku},
	Journal = {Mobile Computing, IEEE Transactions on},
	Number = {12},
	Pages = {2360--2372},
	Publisher = {IEEE},
	Title = {Local differential perturbations: Location privacy under approximate knowledge attackers},
	Volume = {12},
	Year = {2013}}

@inproceedings{chatzikokolakis2013broadening,
	Author = {Chatzikokolakis, Konstantinos and Andr{\'e}s, Miguel E and Bordenabe, Nicol{\'a}s Emilio and Palamidessi, Catuscia},
	Booktitle = {Privacy Enhancing Technologies},
	Organization = {Springer},
	Pages = {82--102},
	Title = {Broadening the Scope of Differential Privacy Using Metrics.},
	Year = {2013}}



@inproceedings{cheng2006preserving,
	Author = {Cheng, Reynold and Zhang, Yu and Bertino, Elisa and Prabhakar, Sunil},
	Booktitle = {Privacy Enhancing Technologies},
	Organization = {Springer},
	Pages = {393--412},
	Title = {Preserving user location privacy in mobile data management infrastructures},
	Year = {2006}}




@article{krumm2009survey,
	Author = {Krumm, John},
	Journal = {Personal and Ubiquitous Computing},
	Number = {6},
	Pages = {391--399},
	Publisher = {Springer},
	Title = {A survey of computational location privacy},
	Volume = {13},
	Year = {2009}}

@article{Rakhshan2015letter,
	Author = {Rakhshan, Ali and Pishro-Nik, Hossein},
	Journal = {IEEE Wireless Communications Letter},
	Publisher = {IEEE},
	Title = {A Stochastic Geometry Model for Customized Vehicular Communication},
	Year = {2015, submitted}}

@article{Rakhshan2015Journal,
	Author = {Rakhshan, Ali and Pishro-Nik, Hossein},
	Journal = {IEEE Transactions on Wireless Communications},
	Publisher = {IEEE},
	Title = {Improving Safety on Highways by Customizing Vehicular Ad Hoc Networks},
	Year = {2015, submitted}}

@inproceedings{Rakhshan2015Cogsima,
	Author = {Rakhshan, Ali and Pishro-Nik, Hossein},
	Booktitle = {IEEE International Multi-Disciplinary Conference on Cognitive Methods in Situation Awareness and Decision Support},
	Organization = {IEEE},
	Title = {A New Approach to Customization of Accident Warning Systems to Individual Drivers},
	Year = {2015}}

@inproceedings{Rakhshan2015CISS,
	Author = {Rakhshan, Ali and Pishro-Nik, Hossein and Nekoui, Mohammad},
	Booktitle = {Conference on Information Sciences and Systems},
	Organization = {IEEE},
	Pages = {1--6},
	Title = {Driver-based adaptation of Vehicular Ad Hoc Networks for design of active safety systems},
	Year = {2015}}

@inproceedings{Rakhshan2014IV,
	Author = {Rakhshan, Ali and Pishro-Nik, Hossein and Ray, Evan},
	Booktitle = {Intelligent Vehicles Symposium},
	Organization = {IEEE},
	Pages = {1181--1186},
	Title = {Real-time estimation of the distribution of brake response times for an individual driver using Vehicular Ad Hoc Network.},
	Year = {2014}}

@inproceedings{Rakhshan2013Globecom,
	Author = {Rakhshan, Ali and Pishro-Nik, Hossein and Fisher, Donald and Nekoui, Mohammad},
	Booktitle = {IEEE Global Communications Conference},
	Organization = {IEEE},
	Pages = {1333--1337},
	Title = {Tuning collision warning algorithms to individual drivers for design of active safety systems.},
	Year = {2013}}

@article{Nekoui2012Journal,
	Author = {Nekoui, Mohammad and Pishro-Nik, Hossein},
	Journal = {IEEE Transactions on Wireless Communications},
	Number = {8},
	Pages = {2895--2905},
	Publisher = {IEEE},
	Title = {Throughput Scaling laws for Vehicular Ad Hoc Networks},
	Volume = {11},
	Year = {2012}}

@article{Nekoui2013Journal,
	Author = {Nekoui, Mohammad and Pishro-Nik, Hossein},
	Journal = {Journal on Selected Areas in Communications, Special Issue on Emerging Technologies in Communications},
	Number = {9},
	Pages = {491--503},
	Publisher = {IEEE},
	Title = {Analytic Design of Active Safety Systems for Vehicular Ad hoc Networks},
	Volume = {31},
	Year = {2013}}

@article{Nekoui2011Journal,
	Author = {Nekoui, Mohammad and Pishro-Nik, Hossein and Ni, Daiheng},
	Journal = {International Journal of Vehicular Technology},
	Pages = {1--11},
	Publisher = {Hindawi Publishing Corporation},
	Title = {Analytic Design of Active Safety Systems for Vehicular Ad hoc Networks},
	Volume = {2011},
	Year = {2011}}

@inproceedings{Nekoui2011MOBICOM,
	Author = {Nekoui, Mohammad and Pishro-Nik, Hossein},
	Booktitle = {MOBICOM workshop on VehiculAr InterNETworking},
	Organization = {ACM},
	Title = {Analytic Design of Active Vehicular Safety Systems in Sparse Traffic},
	Year = {2011}}

@inproceedings{Nekoui2011VTC,
	Author = {Nekoui, Mohammad and Pishro-Nik, Hossein},
	Booktitle = {VTC-Fall},
	Organization = {IEEE},
	Title = {Analytical Design of Inter-vehicular Communications for Collision Avoidance},
	Year = {2011}}

@inproceedings{Bovee2011VTC,
	Author = {Bovee, Ben Louis and Nekoui, Mohammad and Pishro-Nik, Hossein},
	Booktitle = {VTC-Fall},
	Organization = {IEEE},
	Title = {Evaluation of the Universal Geocast Scheme For VANETs},
	Year = {2011}}

@inproceedings{Nekoui2010MOBICOM,
	Author = {Nekoui, Mohammad and Pishro-Nik, Hossein},
	Booktitle = {MOBICOM},
	Organization = {ACM},
	Title = {Fundamental Tradeoffs in Vehicular Ad Hoc Networks},
	Year = {2010}}

@inproceedings{Nekoui2010IVCS,
	Author = {Nekoui, Mohammad and Pishro-Nik, Hossein},
	Booktitle = {IVCS},
	Organization = {IEEE},
	Title = {A Universal Geocast Scheme for Vehicular Ad Hoc Networks},
	Year = {2010}}

@inproceedings{Nekoui2009ITW,
	Author = {Nekoui, Mohammad and Pishro-Nik, Hossein},
	Booktitle = {IEEE Communications Society Conference on Sensor, Mesh and Ad Hoc Communications and Networks Workshops},
	Organization = {IEEE},
	Pages = {1--3},
	Title = {A Geometrical Analysis of Obstructed Wireless Networks},
	Year = {2009}}

@article{Eslami2013Journal,
	Author = {Eslami, Ali and Nekoui, Mohammad and Pishro-Nik, Hossein and Fekri, Faramarz},
	Journal = {ACM Transactions on Sensor Networks},
	Number = {4},
	Pages = {51},
	Publisher = {ACM},
	Title = {Results on finite wireless sensor networks: Connectivity and coverage},
	Volume = {9},
	Year = {2013}}

@article{shokri2014optimal,
	  title={Optimal user-centric data obfuscation},
 	 author={Shokri, Reza},
 	 journal={arXiv preprint arXiv:1402.3426},
 	 year={2014}
	}
@article{chatzikokolakis2015location,
  title={Location privacy via geo-indistinguishability},
  author={Chatzikokolakis, Konstantinos and Palamidessi, Catuscia and Stronati, Marco},
  journal={ACM SIGLOG News},
  volume={2},
  number={3},
  pages={46--69},
  year={2015},
  publisher={ACM}

}
@inproceedings{shokri2011quantifying2,
  title={Quantifying location privacy: the case of sporadic location exposure},
  author={Shokri, Reza and Theodorakopoulos, George and Danezis, George and Hubaux, Jean-Pierre and Le Boudec, Jean-Yves},
  booktitle={Privacy Enhancing Technologies},
  pages={57--76},
  year={2011},
  organization={Springer}
}

@inproceedings{calmon2015fundamental,
  title={Fundamental limits of perfect privacy},
  author={Calmon, Flavio P and Makhdoumi, Ali and M{\'e}dard, Muriel},
  booktitle={Information Theory (ISIT), 2015 IEEE International Symposium on},
  pages={1796--1800},
  year={2015},
  organization={IEEE}
}

@inproceedings{salamatian2013hide,
  title={How to hide the elephant-or the donkey-in the room: Practical privacy against statistical inference for large data.},
  author={Salamatian, Salman and Zhang, Amy and du Pin Calmon, Flavio and Bhamidipati, Sandilya and Fawaz, Nadia and Kveton, Branislav and Oliveira, Pedro and Taft, Nina},
  booktitle={GlobalSIP},
  pages={269--272},
  year={2013}
}

@article{sankar2013utility,
  title={Utility-privacy tradeoffs in databases: An information-theoretic approach},
  author={Sankar, Lalitha and Rajagopalan, S Raj and Poor, H Vincent},
  journal={Information Forensics and Security, IEEE Transactions on},
  volume={8},
  number={6},
  pages={838--852},
  year={2013},
  publisher={IEEE}
}
@inproceedings{ghinita2007prive,
  title={PRIVE: anonymous location-based queries in distributed mobile systems},
  author={Ghinita, Gabriel and Kalnis, Panos and Skiadopoulos, Spiros},
  booktitle={Proceedings of the 16th international conference on World Wide Web},
  pages={371--380},
  year={2007},
  organization={ACM}
}

@article{beresford2004mix,
  title={Mix zones: User privacy in location-aware services},
  author={Beresford, Alastair R and Stajano, Frank},
  year={2004},
  publisher={IEEE}
}


@article{csiszar1996almost,
  title={Almost independence and secrecy capacity},
  author={Csisz{\'a}r, Imre},
  journal={Problemy Peredachi Informatsii},
  volume={32},
  number={1},
  pages={48--57},
  year={1996},
  publisher={Russian Academy of Sciences, Branch of Informatics, Computer Equipment and Automatization}
}

@article{yamamoto1983source,
  title={A source coding problem for sources with additional outputs to keep secret from the receiver or wiretappers (corresp.)},
  author={Yamamoto, Hirosuke},
  journal={IEEE Transactions on Information Theory},
  volume={29},
  number={6},
  pages={918--923},
  year={1983},
  publisher={IEEE}
}



  @inproceedings{golle2009anonymity,
  title={On the anonymity of home/work location pairs},
  author={Golle, Philippe and Partridge, Kurt},
  booktitle={International Conference on Pervasive Computing},
  pages={390--397},
  year={2009},
  organization={Springer}
}

@inproceedings{zang2011anonymization,
  title={Anonymization of location data does not work: A large-scale measurement study},
  author={Zang, Hui and Bolot, Jean},
  booktitle={Proceedings of the 17th annual international conference on Mobile computing and networking},
  pages={145--156},
  year={2011},
  organization={ACM}
}
@article{wang2015privacy,
  title={Privacy-preserving collaborative spectrum sensing with multiple service providers},
  author={Wang, Wei and Zhang, Qian},
  journal={IEEE Transactions on Wireless Communications},
  volume={14},
  number={2},
  pages={1011--1019},
  year={2015},
  publisher={IEEE}
}

@article{wang2015toward,
  title={Toward long-term quality of protection in mobile networks: a context-aware perspective},
  author={Wang, Wei and Zhang, Qian},
  journal={IEEE Wireless Communications},
  volume={22},
  number={4},
  pages={34--40},
  year={2015},
  publisher={IEEE}
}

@inproceedings{niu2015enhancing,
  title={Enhancing privacy through caching in location-based services},
  author={Niu, Ben and Li, Qinghua and Zhu, Xiaoyan and Cao, Guohong and Li, Hui},
  booktitle={2015 IEEE Conference on Computer Communications (INFOCOM)},
  pages={1017--1025},
  year={2015},
  organization={IEEE}
}

%% This BibTeX bibliography file was created using BibDesk.
%% http://bibdesk.sourceforge.net/

%% Created for Zarrin Montazeri at 2015-11-09 18:45:31 -0500


%% Saved with string encoding Unicode (UTF-8)



%%%%%%%%%%%%%%IOT%%%%%%%%%%%%%%%%%%%%%%%%%%%%%%%%%%%%%%%%%%%%%%%%%%%


@article{osma2015,
	title={Impact of Time-to-Collision Information on Driving Behavior in Connected Vehicle Environments Using A Driving Simulator Test Bed},
	journal{Journal of Traffic and Logistics Engineering},
	author={Osama A. Osman, Julius Codjoe, and Sherif Ishak},
	volume={3},
	number={1},
	pages={18--24},
	year={2015}
}


@article{charisma2010,
	title={Dynamic Latent Plan Models},
	author={Charisma F. Choudhurya, Moshe Ben-Akivab and Maya Abou-Zeid},
	journal={Journal of Choice Modelling},
	volume={3},
	number={2},
	pages={50--70},
	year={2010},
	publisher={Elsvier}
}


@misc{noble2014,
	author = {A. M. Noble, Shane B. McLaughlin, Zachary R. Doerzaph and Thomas A. Dingus},
	title = {Crowd-sourced Connected-vehicle Warning Algorithm using Naturalistic Driving Data},
	howpublished = {Downloaded from \url{http://hdl.handle.net/10919/53978}},
	
	month = August,
	year = 2014
}


@phdthesis{charisma2007,
	title    = {Modeling Driving Decisions with Latent Plans},
	school   = {Massachusetts Institute of Technology },
	author   = {Charisma Farheen Choudhury},
	year     = {2007}, %other attributes omitted
}


@article{chrysler2015,
	title={Cost of Warning of Unseen Threats:Unintended Consequences of Connected Vehicle Alerts},
	author={S. T. Chrysler, J. M. Cooper and D. C. Marshall},
	journal={Transportation Research Record: Journal of the Transportation Research Board},
	volume={2518},
	pages={79--85},
	year={2015},
}





@article{FTC2015,
	title={Internet of Things: Privacy and Security in a Connected World},
	author={FTC Staff Report},
	year={2015}
}



%% Saved with string encoding Unicode (UTF-8)
@inproceedings{1zhou2014security,
	title={Security/privacy of wearable fitness tracking {I}o{T} devices},
	author={Zhou, Wei and Piramuthu, Selwyn},
	booktitle={Information Systems and Technologies (CISTI), 2014 9th Iberian Conference on},
	pages={1--5},
	year={2014},
	organization={IEEE}
}


@inproceedings{3ukil2014iot,
	title={{I}o{T}-privacy: To be private or not to be private},
	author={Arijit Ukil and Soma Bandyopadhyay and Arpan Pal},
	booktitle={Computer Communications Workshops (INFOCOM WKSHPS), IEEE Conference on},
	pages={123--124},
	year={2014},
	organization={IEEE}
}

@inproceedings{4Hosseinzadeh2014,
	title={Security in the Internet of Things through obfuscation and diversification},
	author={Hosseinzadeh, Shohreh and Rauti, Sampsa and Hyrynsalmi, Sami and Leppanen, Ville},
	booktitle={Computing, Communication and Security (ICCCS), IEEE Conference on},
	pages={123--124},
	year={2015},
	organization={IEEE}
}
@article{4arias2015privacy,
	title={Privacy and security in internet of things and wearable devices},
	author={Arias, Orlando and Wurm, Jacob and Hoang, Khoa and Jin, Yier},
	journal={IEEE Transactions on Multi-Scale Computing Systems},
	volume={1},
	number={2},
	pages={99--109},
	year={2015},
	publisher={IEEE}
}
@inproceedings{5ullah2016novel,
	title={A novel model for preserving Location Privacy in Internet of Things},
	author={Ullah, Ikram and Shah, Munam Ali},
	booktitle={Automation and Computing (ICAC), 2016 22nd International Conference on},
	pages={542--547},
	year={2016},
	organization={IEEE}
}
@inproceedings{6sathishkumar2016enhanced,
	title={Enhanced location privacy algorithm for wireless sensor network in Internet of Things},
	author={Sathishkumar, J and Patel, Dhiren R},
	booktitle={Internet of Things and Applications (IOTA), International Conference on},
	pages={208--212},
	year={2016},
	organization={IEEE}
}
@inproceedings{7zhou2012preserving,
	title={Preserving sensor location privacy in internet of things},
	author={Zhou, Liming and Wen, Qiaoyan and Zhang, Hua},
	booktitle={Computational and Information Sciences (ICCIS), 2012 Fourth International Conference on},
	pages={856--859},
	year={2012},
	organization={IEEE}
}

@inproceedings{8ukil2015privacy,
	title={Privacy for {I}o{T}: Involuntary privacy enablement for smart energy systems},
	author={Ukil, Arijit and Bandyopadhyay, Soma and Pal, Arpan},
	booktitle={Communications (ICC), 2015 IEEE International Conference on},
	pages={536--541},
	year={2015},
	organization={IEEE}
}

@inproceedings{9dalipi2016security,
	title={Security and Privacy Considerations for {I}o{T} Application on Smart Grids: Survey and Research Challenges},
	author={Dalipi, Fisnik and Yayilgan, Sule Yildirim},
	booktitle={Future Internet of Things and Cloud Workshops (FiCloudW), IEEE International Conference on},
	pages={63--68},
	year={2016},
	organization={IEEE}
}
@inproceedings{10harris2016security,
	title={Security and Privacy in Public {I}o{T} Spaces},
	author={Harris, Albert F and Sundaram, Hari and Kravets, Robin},
	booktitle={Computer Communication and Networks (ICCCN), 2016 25th International Conference on},
	pages={1--8},
	year={2016},
	organization={IEEE}
}

@inproceedings{11al2015security,
	title={Security and privacy framework for ubiquitous healthcare {I}o{T} devices},
	author={Al Alkeem, Ebrahim and Yeun, Chan Yeob and Zemerly, M Jamal},
	booktitle={Internet Technology and Secured Transactions (ICITST), 2015 10th International Conference for},
	pages={70--75},
	year={2015},
	organization={IEEE}
}
@inproceedings{12sivaraman2015network,
	title={Network-level security and privacy control for smart-home {I}o{T} devices},
	author={Sivaraman, Vijay and Gharakheili, Hassan Habibi and Vishwanath, Arun and Boreli, Roksana and Mehani, Olivier},
	booktitle={Wireless and Mobile Computing, Networking and Communications (WiMob), 2015 IEEE 11th International Conference on},
	pages={163--167},
	year={2015},
	organization={IEEE}
}

@inproceedings{13srinivasan2016privacy,
	title={Privacy conscious architecture for improving emergency response in smart cities},
	author={Srinivasan, Ramya and Mohan, Apurva and Srinivasan, Priyanka},
	booktitle={Smart City Security and Privacy Workshop (SCSP-W), 2016},
	pages={1--5},
	year={2016},
	organization={IEEE}
}
@inproceedings{14sadeghi2015security,
	title={Security and privacy challenges in industrial internet of things},
	author={Sadeghi, Ahmad-Reza and Wachsmann, Christian and Waidner, Michael},
	booktitle={Design Automation Conference (DAC), 2015 52nd ACM/EDAC/IEEE},
	pages={1--6},
	year={2015},
	organization={IEEE}
}
@inproceedings{15otgonbayar2016toward,
	title={Toward Anonymizing {I}o{T} Data Streams via Partitioning},
	author={Otgonbayar, Ankhbayar and Pervez, Zeeshan and Dahal, Keshav},
	booktitle={Mobile Ad Hoc and Sensor Systems (MASS), 2016 IEEE 13th International Conference on},
	pages={331--336},
	year={2016},
	organization={IEEE}
}
@inproceedings{16rutledge2016privacy,
	title={Privacy Impacts of {I}o{T} Devices: A SmartTV Case Study},
	author={Rutledge, Richard L and Massey, Aaron K and Ant{\'o}n, Annie I},
	booktitle={Requirements Engineering Conference Workshops (REW), IEEE International},
	pages={261--270},
	year={2016},
	organization={IEEE}
}

@inproceedings{17andrea2015internet,
	title={Internet of Things: Security vulnerabilities and challenges},
	author={Andrea, Ioannis and Chrysostomou, Chrysostomos and Hadjichristofi, George},
	booktitle={Computers and Communication (ISCC), 2015 IEEE Symposium on},
	pages={180--187},
	year={2015},
	organization={IEEE}
}






























%%%%%%%%%%%%%%%%%%%%%%%%%%%%%%%%%%%%%%%%%%%%%%%%%%%%%%%%%%%


@misc{epfl-mobility-20090224,
	author = {Michal Piorkowski and Natasa Sarafijanovic-Djukic and Matthias Grossglauser},
	title = {{CRAWDAD} dataset epfl/mobility (v. 2009-02-24)},
	howpublished = {Downloaded from \url{http://crawdad.org/epfl/mobility/20090224}},
	doi = {10.15783/C7J010},
	month = feb,
	year = 2009
}

@misc{roma-taxi-20140717,
	author = {Lorenzo Bracciale and Marco Bonola and Pierpaolo Loreti and Giuseppe Bianchi and Raul Amici and Antonello Rabuffi},
	title = {{CRAWDAD} dataset roma/taxi (v. 2014-07-17)},
	howpublished = {Downloaded from \url{http://crawdad.org/roma/taxi/20140717}},
	doi = {10.15783/C7QC7M},
	month = jul,
	year = 2014
}

@misc{rice-ad_hoc_city-20030911,
	author = {Jorjeta G. Jetcheva and Yih-Chun Hu and Santashil PalChaudhuri and Amit Kumar Saha and David B. Johnson},
	title = {{CRAWDAD} dataset rice/ad\_hoc\_city (v. 2003-09-11)},
	howpublished = {Downloaded from \url{http://crawdad.org/rice/ad_hoc_city/20030911}},
	doi = {10.15783/C73K5B},
	month = sep,
	year = 2003
}

@misc{china:2012,
	author = {Microsoft Research Asia},
	title = {GeoLife GPS Trajectories},
	year = {2012},
	howpublished= {\url{https://www.microsoft.com/en-us/download/details.aspx?id=52367}},
}


@misc{china:2011,
	ALTauthor = {Microsoft Research Asia)},
	ALTeditor = {},
	title = {GeoLife GPS Trajectories,
	year = {2012},
	url = {https://www.microsoft.com/en-us/download/details.aspx?id=52367},
	}
	
	
	@misc{longversion,
	author = {N. Takbiri and A. Houmansadr and D.L. Goeckel and H. Pishro-Nik},
	title = {{Limits of Location Privacy under Anonymization and Obfuscation}},
	howpublished = "\url{https://dl.dropboxusercontent.com/u/49263048/ISIT_2017-2.pdf}",
	year = 2017,
	month= "January",
	note = "[Online; accessed 22-Jan-2017]"
	}
	
	
	
	@article{matching,
	title={Asymptotically Optimal Matching of Multiple Sequences to Source Distributions and Training Sequences},
	author={Jayakrishnan Unnikrishnan},
	journal={ IEEE Transactions on Information Theory},
	volume={61},
	number={1},
	pages={452-468},
	year={2015},
	publisher={IEEE}
	}
	
	
	@article{Naini2016,
	Author = {F. Naini and J. Unnikrishnan and P. Thiran and M. Vetterli},
	Journal = {IEEE Transactions on Information Forensics and Security},
	Publisher = {IEEE},
	Title = {Where You Are Is Who You Are: User Identification by Matching Statistics},
	volume={11},
	number={2},
	pages={358--372},
	Year = {2016}
	}
	
	
	
	@inproceedings{holowczak2015cachebrowser,
	title={{CacheBrowser: Bypassing Chinese Censorship without Proxies Using Cached Content}},
	author={Holowczak, John and Houmansadr, Amir},
	booktitle={Proceedings of the 22nd ACM SIGSAC Conference on Computer and Communications Security},
	pages={70--83},
	year={2015},
	organization={ACM}
	}
	@misc{cb-website,
	Howpublished = {\url{https://cachebrowser.net/#/}},
	Title = {{CacheBrowser}},
	key={cachebrowser}
	}
	
	@inproceedings{GameOfDecoys,
	title={{GAME OF DECOYS: Optimal Decoy Routing Through Game Theory}},
	author={Milad Nasr and Amir Houmansadr},
	booktitle={The $23^{rd}$ ACM Conference on Computer and Communications Security (CCS)},
	year={2016}
	}
	
	@inproceedings{CDNReaper,
	title={{Practical Censorship Evasion Leveraging Content Delivery Networks}},
	author={Hadi Zolfaghari and Amir Houmansadr},
	booktitle={The $23^{rd}$ ACM Conference on Computer and Communications Security (CCS)},
	year={2016}
	}
	
	@misc{Leberknight2010,
	Author = {Leberknight, C. and Chiang, M. and Poor, H. and Wong, F.},
	Howpublished = {\url{http://www.princeton.edu/~chiangm/anticensorship.pdf}},
	Title = {{A Taxonomy of Internet Censorship and Anti-censorship}},
	Year = {2010}}
	
	@techreport{ultrasurf-analysis,
	Author = {Appelbaum, Jacob},
	Institution = {The Tor Project},
	Title = {{Technical analysis of the Ultrasurf proxying software}},
	Url = {http://scholar.google.com/scholar?hl=en\&btnG=Search\&q=intitle:Technical+analysis+of+the+Ultrasurf+proxying+software\#0},
	Year = {2012},
	Bdsk-Url-1 = {http://scholar.google.com/scholar?hl=en%5C&btnG=Search%5C&q=intitle:Technical+analysis+of+the+Ultrasurf+proxying+software%5C#0}}
	
	@misc{gifc:07,
	Howpublished = {\url{http://www.internetfreedom.org/archive/Defeat\_Internet\_Censorship\_White\_Paper.pdf}},
	Key = {defeatcensorship},
	Publisher = {Global Internet Freedom Consortium (GIFC)},
	Title = {{Defeat Internet Censorship: Overview of Advanced Technologies and Products}},
	Type = {White Paper},
	Year = {2007}}
	
	@article{pan2011survey,
	Author = {Pan, J. and Paul, S. and Jain, R.},
	Journal = {Communications Magazine, IEEE},
	Number = {7},
	Pages = {26--36},
	Publisher = {IEEE},
	Title = {{A Survey of the Research on Future Internet Architectures}},
	Volume = {49},
	Year = {2011}}
	
	@misc{nsf-fia,
	Howpublished = {\url{http://www.nets-fia.net/}},
	Key = {FIA},
	Title = {{NSF Future Internet Architecture Project}}}
	
	@misc{NDN,
	Howpublished = {\url{http://www.named- data.net}},
	Key = {NDN},
	Title = {{Named Data Networking Project}}}
	
	@inproceedings{MobilityFirst,
	Author = {Seskar, I. and Nagaraja, K. and Nelson, S. and Raychaudhuri, D.},
	Booktitle = {Asian Internet Engineering Conference},
	Title = {{Mobilityfirst Future internet Architecture Project}},
	Year = {2011}}
	
	@incollection{NEBULA,
	Author = {Anderson, T. and Birman, K. and Broberg, R. and Caesar, M. and Comer, D. and Cotton, C. and Freedman, M.~J. and Haeberlen, A. and Ives, Z.~G. and Krishnamurthy, A. and others},
	Booktitle = {The Future Internet},
	Pages = {16--26},
	Publisher = {Springer},
	Title = {{The NEBULA Future Internet Architecture}},
	Year = {2013}}
	
	@inproceedings{XIA,
	Author = {Anand, A. and Dogar, F. and Han, D. and Li, B. and Lim, H. and Machado, M. and Wu, W. and Akella, A. and Andersen, D.~G. and Byers, J.~W. and others},
	Booktitle = {ACM Workshop on Hot Topics in Networks},
	Pages = {2},
	Title = {{XIA: An Architecture for an Evolvable and Trustworthy Internet}},
	Year = {2011}}
	
	@inproceedings{ChoiceNet,
	Author = {Rouskas, G.~N. and Baldine, I. and Calvert, K.~L. and Dutta, R. and Griffioen, J. and Nagurney, A. and Wolf, T.},
	Booktitle = {ONDM},
	Title = {{ChoiceNet: Network Innovation Through Choice}},
	Year = {2013}}
	
	@misc{nsf-find,
	Howpublished = {http://www.nets-find.net/},
	Title = {{NSF NeTS FIND Initiative}}}
	
	@article{traid,
	Author = {Cheriton, D.~R. and Gritter, M.},
	Title = {{TRIAD: A New Next-Generation Internet Architecture}},
	Year = {2000}}
	
	@inproceedings{dona,
	Author = {Koponen, T. and Chawla, M. and Chun, B-G. and Ermolinskiy, A. and Kim, K.~H. and Shenker, S. and Stoica, I.},
	Booktitle = {ACM SIGCOMM Computer Communication Review},
	Number = {4},
	Organization = {ACM},
	Pages = {181--192},
	Title = {{A Data-Oriented (and Beyond) Network Architecture}},
	Volume = {37},
	Year = {2007}}
	
	@misc{ultrasurf,
	Howpublished = {\url{http://www.ultrareach.com}},
	Key = {ultrasurf},
	Title = {{Ultrasurf}}}
	
	@misc{tor-bridge,
	Author = {Dingledine, R. and Mathewson, N.},
	Howpublished = {\url{https://svn.torproject.org/svn/projects/design-paper/blocking.html}},
	Title = {{Design of a Blocking-Resistant Anonymity System}}}
	
	@inproceedings{McLachlanH09,
	Author = {J. McLachlan and N. Hopper},
	Booktitle = {WPES},
	Title = {{On the Risks of Serving Whenever You Surf: Vulnerabilities in Tor's Blocking Resistance Design}},
	Year = {2009}}
	
	@inproceedings{mahdian2010,
	Author = {Mahdian, M.},
	Booktitle = {{Fun with Algorithms}},
	Title = {{Fighting Censorship with Algorithms}},
	Year = {2010}}
	
	@inproceedings{McCoy2011,
	Author = {McCoy, D. and Morales, J.~A. and Levchenko, K.},
	Booktitle = {FC},
	Title = {{Proximax: A Measurement Based System for Proxies Dissemination}},
	Year = {2011}}
	
	@inproceedings{Sovran2008,
	Author = {Sovran, Y. and Libonati, A. and Li, J.},
	Booktitle = {IPTPS},
	Title = {{Pass it on: Social Networks Stymie Censors}},
	Year = {2008}}
	
	@inproceedings{rbridge,
	Author = {Wang, Q. and Lin, Zi and Borisov, N. and Hopper, N.},
	Booktitle = {{NDSS}},
	Title = {{rBridge: User Reputation based Tor Bridge Distribution with Privacy Preservation}},
	Year = {2013}}
	
	@inproceedings{telex,
	Author = {Wustrow, E. and Wolchok, S. and Goldberg, I. and Halderman, J.},
	Booktitle = {{USENIX Security}},
	Title = {{Telex: Anticensorship in the Network Infrastructure}},
	Year = {2011}}
	
	@inproceedings{cirripede,
	Author = {Houmansadr, A. and Nguyen, G. and Caesar, M. and Borisov, N.},
	Booktitle = {CCS},
	Title = {{Cirripede: Circumvention Infrastructure Using Router Redirection with Plausible Deniability}},
	Year = {2011}}
	
	@inproceedings{decoyrouting,
	Author = {Karlin, J. and Ellard, D. and Jackson, A. and Jones, C. and Lauer, G. and Mankins, D. and Strayer, W.},
	Booktitle = {{FOCI}},
	Title = {{Decoy Routing: Toward Unblockable Internet Communication}},
	Year = {2011}}
	
	@inproceedings{routing-around-decoys,
	Author = {M.~Schuchard and J.~Geddes and C.~Thompson and N.~Hopper},
	Booktitle = {{CCS}},
	Title = {{Routing Around Decoys}},
	Year = {2012}}
	
	@inproceedings{parrot,
	Author = {A. Houmansadr and C. Brubaker and V. Shmatikov},
	Booktitle = {IEEE S\&P},
	Title = {{The Parrot is Dead: Observing Unobservable Network Communications}},
	Year = {2013}}
	
	@misc{knock,
	Author = {T. Wilde},
	Howpublished = {\url{https://blog.torproject.org/blog/knock-knock-knockin-bridges-doors}},
	Title = {{Knock Knock Knockin' on Bridges' Doors}},
	Year = {2012}}
	
	@inproceedings{china-tor,
	Author = {Winter, P. and Lindskog, S.},
	Booktitle = {{FOCI}},
	Title = {{How the Great Firewall of China Is Blocking Tor}},
	Year = {2012}}
	
	@misc{discover-bridge,
	Howpublished = {\url{https://blog.torproject.org/blog/research-problems-ten-ways-discover-tor-bridges}},
	Key = {tenways},
	Title = {{Ten Ways to Discover Tor Bridges}}}
	
	@inproceedings{freewave,
	Author = {A.~Houmansadr and T.~Riedl and N.~Borisov and A.~Singer},
	Booktitle = {{NDSS}},
	Title = {{I Want My Voice to Be Heard: IP over Voice-over-IP for Unobservable Censorship Circumvention}},
	Year = 2013}
	
	@inproceedings{censorspoofer,
	Author = {Q. Wang and X. Gong and G. Nguyen and A. Houmansadr and N. Borisov},
	Booktitle = {CCS},
	Title = {{CensorSpoofer: Asymmetric Communication Using IP Spoofing for Censorship-Resistant Web Browsing}},
	Year = {2012}}
	
	@inproceedings{skypemorph,
	Author = {H. Moghaddam and B. Li and M. Derakhshani and I. Goldberg},
	Booktitle = {CCS},
	Title = {{SkypeMorph: Protocol Obfuscation for Tor Bridges}},
	Year = {2012}}
	
	@inproceedings{stegotorus,
	Author = {Weinberg, Z. and Wang, J. and Yegneswaran, V. and Briesemeister, L. and Cheung, S. and Wang, F. and Boneh, D.},
	Booktitle = {CCS},
	Title = {{StegoTorus: A Camouflage Proxy for the Tor Anonymity System}},
	Year = {2012}}
	
	@techreport{dust,
	Author = {{B.~Wiley}},
	Howpublished = {\url{http://blanu.net/ Dust.pdf}},
	Institution = {School of Information, University of Texas at Austin},
	Title = {{Dust: A Blocking-Resistant Internet Transport Protocol}},
	Year = {2011}}
	
	@inproceedings{FTE,
	Author = {K.~Dyer and S.~Coull and T.~Ristenpart and T.~Shrimpton},
	Booktitle = {CCS},
	Title = {{Protocol Misidentification Made Easy with Format-Transforming Encryption}},
	Year = {2013}}
	
	@inproceedings{fp,
	Author = {Fifield, D. and Hardison, N. and Ellithrope, J. and Stark, E. and Dingledine, R. and Boneh, D. and Porras, P.},
	Booktitle = {PETS},
	Title = {{Evading Censorship with Browser-Based Proxies}},
	Year = {2012}}
	
	@misc{obfsproxy,
	Howpublished = {\url{https://www.torproject.org/projects/obfsproxy.html.en}},
	Key = {obfsproxy},
	Publisher = {The Tor Project},
	Title = {{A Simple Obfuscating Proxy}}}
	
	@inproceedings{Tor-instead-of-IP,
	Author = {Liu, V. and Han, S. and Krishnamurthy, A. and Anderson, T.},
	Booktitle = {HotNets},
	Title = {{Tor instead of IP}},
	Year = {2011}}
	
	@misc{roger-slides,
	Howpublished = {\url{https://svn.torproject.org/svn/projects/presentations/slides-28c3.pdf}},
	Key = {torblocking},
	Title = {{How Governments Have Tried to Block Tor}}}
	
	@inproceedings{infranet,
	Author = {Feamster, N. and Balazinska, M. and Harfst, G. and Balakrishnan, H. and Karger, D.},
	Booktitle = {USENIX Security},
	Title = {{Infranet: Circumventing Web Censorship and Surveillance}},
	Year = {2002}}
	
	@inproceedings{collage,
	Author = {S.~Burnett and N.~Feamster and S.~Vempala},
	Booktitle = {USENIX Security},
	Title = {{Chipping Away at Censorship Firewalls with User-Generated Content}},
	Year = {2010}}
	
	@article{anonymizer,
	Author = {Boyan, J.},
	Journal = {Computer-Mediated Communication Magazine},
	Month = sep,
	Number = {9},
	Title = {{The Anonymizer: Protecting User Privacy on the Web}},
	Volume = {4},
	Year = {1997}}
	
	@article{schulze2009internet,
	Author = {Schulze, H. and Mochalski, K.},
	Journal = {IPOQUE Report},
	Pages = {351--362},
	Title = {Internet Study 2008/2009},
	Volume = {37},
	Year = {2009}}
	
	@inproceedings{cya-ccs13,
	Author = {J.~Geddes and M.~Schuchard and N.~Hopper},
	Booktitle = {{CCS}},
	Title = {{Cover Your ACKs: Pitfalls of Covert Channel Censorship Circumvention}},
	Year = {2013}}
	
	@inproceedings{andana,
	Author = {DiBenedetto, S. and Gasti, P. and Tsudik, G. and Uzun, E.},
	Booktitle = {{NDSS}},
	Title = {{ANDaNA: Anonymous Named Data Networking Application}},
	Year = {2012}}
	
	@inproceedings{darkly,
	Author = {Jana, S. and Narayanan, A. and Shmatikov, V.},
	Booktitle = {IEEE S\&P},
	Title = {{A Scanner Darkly: Protecting User Privacy From Perceptual Applications}},
	Year = {2013}}
	
	@inproceedings{NS08,
	Author = {A.~Narayanan and V.~Shmatikov},
	Booktitle = {IEEE S\&P},
	Title = {Robust de-anonymization of large sparse datasets},
	Year = {2008}}
	
	@inproceedings{NS09,
	Author = {A.~Narayanan and V.~Shmatikov},
	Booktitle = {IEEE S\&P},
	Title = {De-anonymizing social networks},
	Year = {2009}}
	
	@inproceedings{memento,
	Author = {Jana, S. and Shmatikov, V.},
	Booktitle = {IEEE S\&P},
	Title = {{Memento: Learning secrets from process footprints}},
	Year = {2012}}
	
	@misc{plugtor,
	Howpublished = {\url{https://www.torproject.org/docs/pluggable-transports.html.en}},
	Key = {PluggableTransports},
	Publisher = {The Tor Project},
	Title = {{Tor: Pluggable transports}}}
	
	@misc{psiphon,
	Author = {J.~Jia and P.~Smith},
	Howpublished = {\url{http://www.cdf.toronto.edu/~csc494h/reports/2004-fall/psiphon_ae.html}},
	Title = {{Psiphon: Analysis and Estimation}},
	Year = 2004}
	
	@misc{china-github,
	Howpublished = {\url{http://mobile.informationweek.com/80269/show/72e30386728f45f56b343ddfd0fdb119/}},
	Key = {github},
	Title = {{China's GitHub Censorship Dilemma}}}
	
	@inproceedings{txbox,
	Author = {Jana, S. and Porter, D. and Shmatikov, V.},
	Booktitle = {IEEE S\&P},
	Title = {{TxBox: Building Secure, Efficient Sandboxes with System Transactions}},
	Year = {2011}}
	
	@inproceedings{airavat,
	Author = {I. Roy and S. Setty and A. Kilzer and V. Shmatikov and E. Witchel},
	Booktitle = {NSDI},
	Title = {{Airavat: Security and Privacy for MapReduce}},
	Year = {2010}}
	
	@inproceedings{osdi12,
	Author = {A. Dunn and M. Lee and S. Jana and S. Kim and M. Silberstein and Y. Xu and V. Shmatikov and E. Witchel},
	Booktitle = {OSDI},
	Title = {{Eternal Sunshine of the Spotless Machine: Protecting Privacy with Ephemeral Channels}},
	Year = {2012}}
	
	@inproceedings{ymal,
	Author = {J. Calandrino and A. Kilzer and A. Narayanan and E. Felten and V. Shmatikov},
	Booktitle = {IEEE S\&P},
	Title = {{``You Might Also Like:'' Privacy Risks of Collaborative Filtering}},
	Year = {2011}}
	
	@inproceedings{srivastava11,
	Author = {V. Srivastava and M. Bond and K. McKinley and V. Shmatikov},
	Booktitle = {PLDI},
	Title = {{A Security Policy Oracle: Detecting Security Holes Using Multiple API Implementations}},
	Year = {2011}}
	
	@inproceedings{chen-oakland10,
	Author = {Chen, S. and Wang, R. and Wang, X. and Zhang, K.},
	Booktitle = {IEEE S\&P},
	Title = {{Side-Channel Leaks in Web Applications: A Reality Today, a Challenge Tomorrow}},
	Year = {2010}}
	
	@book{kerck,
	Author = {Kerckhoffs, A.},
	Publisher = {University Microfilms},
	Title = {{La cryptographie militaire}},
	Year = {1978}}
	
	@inproceedings{foci11,
	Author = {J. Karlin and D. Ellard and A.~Jackson and C.~ Jones and G. Lauer and D. Mankins and W.~T.~Strayer},
	Booktitle = {FOCI},
	Title = {{Decoy Routing: Toward Unblockable Internet Communication}},
	Year = 2011}
	
	@inproceedings{sun02,
	Author = {Sun, Q. and Simon, D.~R. and Wang, Y. and Russell, W. and Padmanabhan, V. and Qiu, L.},
	Booktitle = {IEEE S\&P},
	Title = {{Statistical Identification of Encrypted Web Browsing Traffic}},
	Year = {2002}}
	
	@inproceedings{danezis,
	Author = {Murdoch, S.~J. and Danezis, G.},
	Booktitle = {IEEE S\&P},
	Title = {{Low-Cost Traffic Analysis of Tor}},
	Year = {2005}}
	
	@inproceedings{pakicensorship,
	Author = {Z.~Nabi},
	Booktitle = {FOCI},
	Title = {The Anatomy of {Web} Censorship in {Pakistan}},
	Year = {2013}}
	
	@inproceedings{irancensorship,
	Author = {S.~Aryan and H.~Aryan and A.~Halderman},
	Booktitle = {FOCI},
	Title = {Internet Censorship in {Iran}: {A} First Look},
	Year = {2013}}
	
	@inproceedings{ford10efficient,
	Author = {Amittai Aviram and Shu-Chun Weng and Sen Hu and Bryan Ford},
	Booktitle = {\bibconf[9th]{OSDI}{USENIX Symposium on Operating Systems Design and Implementation}},
	Location = {Vancouver, BC, Canada},
	Month = oct,
	Title = {Efficient System-Enforced Deterministic Parallelism},
	Year = 2010}
	
	@inproceedings{ford10determinating,
	Author = {Amittai Aviram and Sen Hu and Bryan Ford and Ramakrishna Gummadi},
	Booktitle = {\bibconf{CCSW}{ACM Cloud Computing Security Workshop}},
	Location = {Chicago, IL},
	Month = oct,
	Title = {Determinating Timing Channels in Compute Clouds},
	Year = 2010}
	
	@inproceedings{ford12plugging,
	Author = {Bryan Ford},
	Booktitle = {\bibconf[4th]{HotCloud}{USENIX Workshop on Hot Topics in Cloud Computing}},
	Location = {Boston, MA},
	Month = jun,
	Title = {Plugging Side-Channel Leaks with Timing Information Flow Control},
	Year = 2012}
	
	@inproceedings{ford12icebergs,
	Author = {Bryan Ford},
	Booktitle = {\bibconf[4th]{HotCloud}{USENIX Workshop on Hot Topics in Cloud Computing}},
	Location = {Boston, MA},
	Month = jun,
	Title = {Icebergs in the Clouds: the {\em Other} Risks of Cloud Computing},
	Year = 2012}
	
	@misc{mullenize,
	Author = {Washington Post},
	Howpublished = {\url{http://apps.washingtonpost.com/g/page/world/gchq-report-on-mullenize-program-to-stain-anonymous-electronic-traffic/502/}},
	Month = {oct},
	Title = {{GCHQ} report on {`MULLENIZE'} program to `stain' anonymous electronic traffic},
	Year = {2013}}
	
	@inproceedings{shue13street,
	Author = {Craig A. Shue and Nathanael Paul and Curtis R. Taylor},
	Booktitle = {\bibbrev[7th]{WOOT}{USENIX Workshop on Offensive Technologies}},
	Month = aug,
	Title = {From an {IP} Address to a Street Address: Using Wireless Signals to Locate a Target},
	Year = 2013}
	
	@inproceedings{knockel11three,
	Author = {Jeffrey Knockel and Jedidiah R. Crandall and Jared Saia},
	Booktitle = {\bibbrev{FOCI}{USENIX Workshop on Free and Open Communications on the Internet}},
	Location = {San Francisco, CA},
	Month = aug,
	Year = 2011}
	
	@misc{rfc4960,
	Author = {R. {Stewart, ed.}},
	Month = sep,
	Note = {RFC 4960},
	Title = {Stream Control Transmission Protocol},
	Year = 2007}
	
	@inproceedings{ford07structured,
	Author = {Bryan Ford},
	Booktitle = {\bibbrev{SIGCOMM}{ACM SIGCOMM}},
	Location = {Kyoto, Japan},
	Month = aug,
	Title = {Structured Streams: a New Transport Abstraction},
	Year = {2007}}
	
	@misc{spdy,
	Author = {Google, Inc.},
	Note = {\url{http://www.chromium.org/spdy/spdy-whitepaper}},
	Title = {{SPDY}: An Experimental Protocol For a Faster {Web}}}
	
	@misc{quic,
	Author = {Jim Roskind},
	Month = jun,
	Note = {\url{http://blog.chromium.org/2013/06/experimenting-with-quic.html}},
	Title = {Experimenting with {QUIC}},
	Year = 2013}
	
	@misc{podjarny12not,
	Author = {G.~Podjarny},
	Month = jun,
	Note = {\url{http://www.guypo.com/technical/not-as-spdy-as-you-thought/}},
	Title = {{Not as SPDY as You Thought}},
	Year = 2012}
	
	@inproceedings{cor,
	Author = {Jones, N.~A. and Arye, M. and Cesareo, J. and Freedman, M.~J.},
	Booktitle = {FOCI},
	Title = {{Hiding Amongst the Clouds: A Proposal for Cloud-based Onion Routing}},
	Year = {2011}}
	
	@misc{torcloud,
	Howpublished = {\url{https://cloud.torproject.org/}},
	Key = {tor cloud},
	Title = {{The Tor Cloud Project}}}
	
	@inproceedings{scramblesuit,
	Author = {Philipp Winter and Tobias Pulls and Juergen Fuss},
	Booktitle = {WPES},
	Title = {{ScrambleSuit: A Polymorphic Network Protocol to Circumvent Censorship}},
	Year = 2013}
	
	@article{savage2000practical,
	Author = {Savage, S. and Wetherall, D. and Karlin, A. and Anderson, T.},
	Journal = {ACM SIGCOMM Computer Communication Review},
	Number = {4},
	Pages = {295--306},
	Publisher = {ACM},
	Title = {Practical network support for IP traceback},
	Volume = {30},
	Year = {2000}}
	
	@inproceedings{ooni,
	Author = {Filast, A. and Appelbaum, J.},
	Booktitle = {{FOCI}},
	Title = {{OONI : Open Observatory of Network Interference}},
	Year = {2012}}
	
	@misc{caida-rank,
	Howpublished = {\url{http://as-rank.caida.org/}},
	Key = {caida rank},
	Title = {{AS Rank: AS Ranking}}}
	
	@inproceedings{usersrouted-ccs13,
	Author = {A.~Johnson and C.~Wacek and R.~Jansen and M.~Sherr and P.~Syverson},
	Booktitle = {CCS},
	Title = {{Users Get Routed: Traffic Correlation on Tor by Realistic Adversaries}},
	Year = {2013}}
	
	@inproceedings{edman2009awareness,
	Author = {Edman, M. and Syverson, P.},
	Booktitle = {{CCS}},
	Title = {{AS-awareness in Tor path selection}},
	Year = {2009}}
	
	@inproceedings{DecoyCosts,
	Author = {A.~Houmansadr and E.~L.~Wong and V.~Shmatikov},
	Booktitle = {NDSS},
	Title = {{No Direction Home: The True Cost of Routing Around Decoys}},
	Year = {2014}}
	
	@article{cordon,
	Author = {Elahi, T. and Goldberg, I.},
	Journal = {University of Waterloo CACR},
	Title = {{CORDON--A Taxonomy of Internet Censorship Resistance Strategies}},
	Volume = {33},
	Year = {2012}}
	
	@inproceedings{privex,
	Author = {T.~Elahi and G.~Danezis and I.~Goldberg	},
	Booktitle = {{CCS}},
	Title = {{AS-awareness in Tor path selection}},
	Year = {2014}}
	
	@inproceedings{changeGuards,
	Author = {T.~Elahi and K.~Bauer and M.~AlSabah and R.~Dingledine and I.~Goldberg},
	Booktitle = {{WPES}},
	Title = {{ Changing of the Guards: Framework for Understanding and Improving Entry Guard Selection in Tor}},
	Year = {2012}}
	
	@article{RAINBOW:Journal,
	Author = {A.~Houmansadr and N.~Kiyavash and N.~Borisov},
	Journal = {IEEE/ACM Transactions on Networking},
	Title = {{Non-Blind Watermarking of Network Flows}},
	Year = 2014}
	
	@inproceedings{info-tod,
	Author = {A.~Houmansadr and S.~Gorantla and T.~Coleman and N.~Kiyavash and and N.~Borisov},
	Booktitle = {{CCS (poster session)}},
	Title = {{On the Channel Capacity of Network Flow Watermarking}},
	Year = {2009}}
	
	@inproceedings{johnson2014game,
	Author = {Johnson, B. and Laszka, A. and Grossklags, J. and Vasek, M. and Moore, T.},
	Booktitle = {Workshop on Bitcoin Research},
	Title = {{Game-theoretic Analysis of DDoS Attacks Against Bitcoin Mining Pools}},
	Year = {2014}}
	
	@incollection{laszka2013mitigation,
	Author = {Laszka, A. and Johnson, B. and Grossklags, J.},
	Booktitle = {Decision and Game Theory for Security},
	Pages = {175--191},
	Publisher = {Springer},
	Title = {{Mitigation of Targeted and Non-targeted Covert Attacks as a Timing Game}},
	Year = {2013}}
	
	@inproceedings{schottle2013game,
	Author = {Schottle, P. and Laszka, A. and Johnson, B. and Grossklags, J. and Bohme, R.},
	Booktitle = {EUSIPCO},
	Title = {{A Game-theoretic Analysis of Content-adaptive Steganography with Independent Embedding}},
	Year = {2013}}
	
	@inproceedings{CloudTransport,
	Author = {C.~Brubaker and A.~Houmansadr and V.~Shmatikov},
	Booktitle = {PETS},
	Title = {{CloudTransport: Using Cloud Storage for Censorship-Resistant Networking}},
	Year = {2014}}
	
	@inproceedings{sweet,
	Author = {W.~Zhou and A.~Houmansadr and M.~Caesar and N.~Borisov},
	Booktitle = {HotPETs},
	Title = {{SWEET: Serving the Web by Exploiting Email Tunnels}},
	Year = {2013}}
	
	@inproceedings{ahsan2002practical,
	Author = {Ahsan, K. and Kundur, D.},
	Booktitle = {Workshop on Multimedia Security},
	Title = {{Practical data hiding in TCP/IP}},
	Year = {2002}}
	
	@incollection{danezis2011covert,
	Author = {Danezis, G.},
	Booktitle = {Security Protocols XVI},
	Pages = {198--214},
	Publisher = {Springer},
	Title = {{Covert Communications Despite Traffic Data Retention}},
	Year = {2011}}
	
	@inproceedings{liu2009hide,
	Author = {Liu, Y. and Ghosal, D. and Armknecht, F. and Sadeghi, A.-R. and Schulz, S. and Katzenbeisser, S.},
	Booktitle = {ESORICS},
	Title = {{Hide and Seek in Time---Robust Covert Timing Channels}},
	Year = {2009}}
	
	@misc{image-watermark-fing,
	Author = {Jonathan Bailey},
	Howpublished = {\url{https://www.plagiarismtoday.com/2009/12/02/image-detection-watermarking-vs-fingerprinting/}},
	Title = {{Image Detection: Watermarking vs. Fingerprinting}},
	Year = {2009}}
	
	@inproceedings{Servetto98,
	Author = {S. D. Servetto and C. I. Podilchuk and K. Ramchandran},
	Booktitle = {Int. Conf. Image Processing},
	Title = {Capacity issues in digital image watermarking},
	Year = {1998}}
	
	@inproceedings{Chen01,
	Author = {B. Chen and G.W.Wornell},
	Booktitle = {IEEE Trans. Inform. Theory},
	Pages = {1423--1443},
	Title = {Quantization index modulation: A class of provably good methods for digital watermarking and information embedding},
	Year = {2001}}
	
	@inproceedings{Karakos00,
	Author = {D. Karakos and A. Papamarcou},
	Booktitle = {IEEE Int. Symp. Information Theory},
	Pages = {47},
	Title = {Relationship between quantization and distribution rates of digitally watermarked data},
	Year = {2000}}
	
	@inproceedings{Sullivan98,
	Author = {J. A. OSullivan and P. Moulin and J. M. Ettinger},
	Booktitle = {IEEE Int. Symp. Information Theory},
	Pages = {297},
	Title = {Information theoretic analysis of steganography},
	Year = {1998}}
	
	@inproceedings{Merhav00,
	Author = {N. Merhav},
	Booktitle = {IEEE Trans. Inform. Theory},
	Pages = {420--430},
	Title = {On random coding error exponents of watermarking systems},
	Year = {2000}}
	
	@inproceedings{Somekh01,
	Author = {A. Somekh-Baruch and N. Merhav},
	Booktitle = {IEEE Int. Symp. Information Theory},
	Pages = {7},
	Title = {On the error exponent and capacity games of private watermarking systems},
	Year = {2001}}
	
	@inproceedings{Steinberg01,
	Author = {Y. Steinberg and N. Merhav},
	Booktitle = {IEEE Trans. Inform. Theory},
	Pages = {1410--1422},
	Title = {Identification in the presence of side information with application to watermarking},
	Year = {2001}}
	
	@article{Moulin03,
	Author = {P. Moulin and J.A. O'Sullivan},
	Journal = {IEEE Trans. Info. Theory},
	Number = {3},
	Title = {Information-theoretic analysis of information hiding},
	Volume = 49,
	Year = 2003}
	
	@article{Gelfand80,
	Author = {S.I.~Gelfand and M.S.~Pinsker},
	Journal = {Problems of Control and Information Theory},
	Number = {1},
	Pages = {19-31},
	Title = {{Coding for channel with random parameters}},
	Url = {citeseer.ist.psu.edu/anantharam96bits.html},
	Volume = {9},
	Year = {1980},
	Bdsk-Url-1 = {citeseer.ist.psu.edu/anantharam96bits.html}}
	
	@book{Wolfowitz78,
	Author = {J. Wolfowitz},
	Edition = {3rd},
	Location = {New York},
	Publisher = {Springer-Verlag},
	Title = {Coding Theorems of Information Theory},
	Year = 1978}
	
	@article{caire99,
	Author = {G. Caire and S. Shamai},
	Journal = {IEEE Transactions on Information Theory},
	Number = {6},
	Pages = {2007--2019},
	Title = {On the Capacity of Some Channels with Channel State Information},
	Volume = {45},
	Year = {1999}}
	
	@inproceedings{wright2007language,
	Author = {Wright, Charles V and Ballard, Lucas and Monrose, Fabian and Masson, Gerald M},
	Booktitle = {USENIX Security},
	Title = {{Language identification of encrypted VoIP traffic: Alejandra y Roberto or Alice and Bob?}},
	Year = {2007}}
	
	@inproceedings{backes2010speaker,
	Author = {Backes, Michael and Doychev, Goran and D{\"u}rmuth, Markus and K{\"o}pf, Boris},
	Booktitle = {{European Symposium on Research in Computer Security (ESORICS)}},
	Pages = {508--523},
	Publisher = {Springer},
	Title = {{Speaker Recognition in Encrypted Voice Streams}},
	Year = {2010}}
	
	@phdthesis{lu2009traffic,
	Author = {Lu, Yuanchao},
	School = {Cleveland State University},
	Title = {{On Traffic Analysis Attacks to Encrypted VoIP Calls}},
	Year = {2009}}
	
	@inproceedings{wright2008spot,
	Author = {Wright, Charles V and Ballard, Lucas and Coull, Scott E and Monrose, Fabian and Masson, Gerald M},
	Booktitle = {IEEE Symposium on Security and Privacy},
	Pages = {35--49},
	Title = {Spot me if you can: Uncovering spoken phrases in encrypted VoIP conversations},
	Year = {2008}}
	
	@inproceedings{white2011phonotactic,
	Author = {White, Andrew M and Matthews, Austin R and Snow, Kevin Z and Monrose, Fabian},
	Booktitle = {IEEE Symposium on Security and Privacy},
	Pages = {3--18},
	Title = {Phonotactic reconstruction of encrypted VoIP conversations: Hookt on fon-iks},
	Year = {2011}}
	
	@inproceedings{fancy,
	Author = {Houmansadr, Amir and Borisov, Nikita},
	Booktitle = {Privacy Enhancing Technologies},
	Organization = {Springer},
	Pages = {205--224},
	Title = {The Need for Flow Fingerprints to Link Correlated Network Flows},
	Year = {2013}}
	
	@article{botmosaic,
	Author = {Amir Houmansadr and Nikita Borisov},
	Doi = {10.1016/j.jss.2012.11.005},
	Issn = {0164-1212},
	Journal = {Journal of Systems and Software},
	Keywords = {Network security},
	Number = {3},
	Pages = {707 - 715},
	Title = {BotMosaic: Collaborative network watermark for the detection of IRC-based botnets},
	Url = {http://www.sciencedirect.com/science/article/pii/S0164121212003068},
	Volume = {86},
	Year = {2013},
	Bdsk-Url-1 = {http://www.sciencedirect.com/science/article/pii/S0164121212003068},
	Bdsk-Url-2 = {http://dx.doi.org/10.1016/j.jss.2012.11.005}}
	
	@inproceedings{ramsbrock2008first,
	Author = {Ramsbrock, Daniel and Wang, Xinyuan and Jiang, Xuxian},
	Booktitle = {Recent Advances in Intrusion Detection},
	Organization = {Springer},
	Pages = {59--77},
	Title = {A first step towards live botmaster traceback},
	Year = {2008}}
	
	@inproceedings{potdar2005survey,
	Author = {Potdar, Vidyasagar M and Han, Song and Chang, Elizabeth},
	Booktitle = {Industrial Informatics, 2005. INDIN'05. 2005 3rd IEEE International Conference on},
	Organization = {IEEE},
	Pages = {709--716},
	Title = {A survey of digital image watermarking techniques},
	Year = {2005}}
	
	@book{cole2003hiding,
	Author = {Cole, Eric and Krutz, Ronald D},
	Publisher = {John Wiley \& Sons, Inc.},
	Title = {Hiding in plain sight: Steganography and the art of covert communication},
	Year = {2003}}
	
	@incollection{akaike1998information,
	Author = {Akaike, Hirotogu},
	Booktitle = {Selected Papers of Hirotugu Akaike},
	Pages = {199--213},
	Publisher = {Springer},
	Title = {Information theory and an extension of the maximum likelihood principle},
	Year = {1998}}
	
	@misc{central-command-hack,
	Author = {Everett Rosenfeld},
	Howpublished = {\url{http://www.cnbc.com/id/102330338}},
	Title = {{FBI investigating Central Command Twitter hack}},
	Year = {2015}}
	
	@misc{sony-psp-ddos,
	Howpublished = {\url{http://n4g.com/news/1644853/sony-and-microsoft-cant-do-much-ddos-attacks-explained}},
	Key = {sony},
	Month = {December},
	Title = {{Sony and Microsoft cant do much -- DDoS attacks explained}},
	Year = {2014}}
	
	@misc{sony-hack,
	Author = {David Bloom},
	Howpublished = {\url{http://goo.gl/MwR4A7}},
	Title = {{Online Game Networks Hacked, Sony Unit President Threatened}},
	Year = {2014}}
	
	@misc{home-depot,
	Author = {Dune Lawrence},
	Howpublished = {\url{http://www.businessweek.com/articles/2014-09-02/home-depots-credit-card-breach-looks-just-like-the-target-hack}},
	Title = {{Home Depot's Suspected Breach Looks Just Like the Target Hack}},
	Year = {2014}}
	
	@misc{target,
	Author = {Julio Ojeda-Zapata},
	Howpublished = {\url{http://www.mercurynews.com/business/ci_24765398/how-did-hackers-pull-off-target-data-heist}},
	Title = {{Target hack: How did they do it?}},
	Year = {2014}}
	
	
	@article{probabilitycourse,
	Author = {H. Pishro-Nik},
	note = {\url{http://www.probabilitycourse.com}},
	Title = {Introduction to probability, statistics, and random processes},
	Year = {2014}}
	
	
	
	@inproceedings{shokri2011quantifying,
	Author = {Shokri, Reza and Theodorakopoulos, George and Le Boudec, Jean-Yves and Hubaux, Jean-Pierre},
	Booktitle = {Security and Privacy (SP), 2011 IEEE Symposium on},
	Organization = {IEEE},
	Pages = {247--262},
	Title = {Quantifying location privacy},
	Year = {2011}}
	
	@inproceedings{hoh2007preserving,
	Author = {Hoh, Baik and Gruteser, Marco and Xiong, Hui and Alrabady, Ansaf},
	Booktitle = {Proceedings of the 14th ACM conference on Computer and communications security},
	Organization = {ACM},
	Pages = {161--171},
	Title = {Preserving privacy in gps traces via uncertainty-aware path cloaking},
	Year = {2007}}
	
	
	
	@article{kafsi2013entropy,
	Author = {Kafsi, Mohamed and Grossglauser, Matthias and Thiran, Patrick},
	Journal = {Information Theory, IEEE Transactions on},
	Number = {9},
	Pages = {5577--5583},
	Publisher = {IEEE},
	Title = {The entropy of conditional Markov trajectories},
	Volume = {59},
	Year = {2013}}
	
	@inproceedings{gruteser2003anonymous,
	Author = {Gruteser, Marco and Grunwald, Dirk},
	Booktitle = {Proceedings of the 1st international conference on Mobile systems, applications and services},
	Organization = {ACM},
	Pages = {31--42},
	Title = {Anonymous usage of location-based services through spatial and temporal cloaking},
	Year = {2003}}
	
	@inproceedings{husted2010mobile,
	Author = {Husted, Nathaniel and Myers, Steven},
	Booktitle = {Proceedings of the 17th ACM conference on Computer and communications security},
	Organization = {ACM},
	Pages = {85--96},
	Title = {Mobile location tracking in metro areas: malnets and others},
	Year = {2010}}
	
	@inproceedings{li2009tradeoff,
	Author = {Li, Tiancheng and Li, Ninghui},
	Booktitle = {Proceedings of the 15th ACM SIGKDD international conference on Knowledge discovery and data mining},
	Organization = {ACM},
	Pages = {517--526},
	Title = {On the tradeoff between privacy and utility in data publishing},
	Year = {2009}}
	
	@inproceedings{ma2009location,
	Author = {Ma, Zhendong and Kargl, Frank and Weber, Michael},
	Booktitle = {Sarnoff Symposium, 2009. SARNOFF'09. IEEE},
	Organization = {IEEE},
	Pages = {1--6},
	Title = {A location privacy metric for v2x communication systems},
	Year = {2009}}
	
	@inproceedings{shokri2012protecting,
	Author = {Shokri, Reza and Theodorakopoulos, George and Troncoso, Carmela and Hubaux, Jean-Pierre and Le Boudec, Jean-Yves},
	Booktitle = {Proceedings of the 2012 ACM conference on Computer and communications security},
	Organization = {ACM},
	Pages = {617--627},
	Title = {Protecting location privacy: optimal strategy against localization attacks},
	Year = {2012}}
	
	@inproceedings{freudiger2009non,
	Author = {Freudiger, Julien and Manshaei, Mohammad Hossein and Hubaux, Jean-Pierre and Parkes, David C},
	Booktitle = {Proceedings of the 16th ACM conference on Computer and communications security},
	Organization = {ACM},
	Pages = {324--337},
	Title = {On non-cooperative location privacy: a game-theoretic analysis},
	Year = {2009}}
	
	@incollection{humbert2010tracking,
	Author = {Humbert, Mathias and Manshaei, Mohammad Hossein and Freudiger, Julien and Hubaux, Jean-Pierre},
	Booktitle = {Decision and Game Theory for Security},
	Pages = {38--57},
	Publisher = {Springer},
	Title = {Tracking games in mobile networks},
	Year = {2010}}
	
	@article{manshaei2013game,
	Author = {Manshaei, Mohammad Hossein and Zhu, Quanyan and Alpcan, Tansu and Bac{\c{s}}ar, Tamer and Hubaux, Jean-Pierre},
	Journal = {ACM Computing Surveys (CSUR)},
	Number = {3},
	Pages = {25},
	Publisher = {ACM},
	Title = {Game theory meets network security and privacy},
	Volume = {45},
	Year = {2013}}
	
	@article{palamidessi2006probabilistic,
	Author = {Palamidessi, Catuscia},
	Journal = {Electronic Notes in Theoretical Computer Science},
	Pages = {33--42},
	Publisher = {Elsevier},
	Title = {Probabilistic and nondeterministic aspects of anonymity},
	Volume = {155},
	Year = {2006}}
	
	@inproceedings{mokbel2006new,
	Author = {Mokbel, Mohamed F and Chow, Chi-Yin and Aref, Walid G},
	Booktitle = {Proceedings of the 32nd international conference on Very large data bases},
	Organization = {VLDB Endowment},
	Pages = {763--774},
	Title = {The new Casper: query processing for location services without compromising privacy},
	Year = {2006}}
	
	@article{kalnis2007preventing,
	Author = {Kalnis, Panos and Ghinita, Gabriel and Mouratidis, Kyriakos and Papadias, Dimitris},
	Journal = {Knowledge and Data Engineering, IEEE Transactions on},
	Number = {12},
	Pages = {1719--1733},
	Publisher = {IEEE},
	Title = {Preventing location-based identity inference in anonymous spatial queries},
	Volume = {19},
	Year = {2007}}
	
	@article{freudiger2007mix,
	title={Mix-zones for location privacy in vehicular networks},
	author={Freudiger, Julien and Raya, Maxim and F{\'e}legyh{\'a}zi, M{\'a}rk and Papadimitratos, Panos and Hubaux, Jean-Pierre},
	year={2007}
	}
	@article{sweeney2002k,
	Author = {Sweeney, Latanya},
	Journal = {International Journal of Uncertainty, Fuzziness and Knowledge-Based Systems},
	Number = {05},
	Pages = {557--570},
	Publisher = {World Scientific},
	Title = {k-anonymity: A model for protecting privacy},
	Volume = {10},
	Year = {2002}}
	
	@article{sweeney2002achieving,
	Author = {Sweeney, Latanya},
	Journal = {International Journal of Uncertainty, Fuzziness and Knowledge-Based Systems},
	Number = {05},
	Pages = {571--588},
	Publisher = {World Scientific},
	Title = {Achieving k-anonymity privacy protection using generalization and suppression},
	Volume = {10},
	Year = {2002}}
	
	@inproceedings{niu2014achieving,
	Author = {Niu, Ben and Li, Qinghua and Zhu, Xiaoyan and Cao, Guohong and Li, Hui},
	Booktitle = {INFOCOM, 2014 Proceedings IEEE},
	Organization = {IEEE},
	Pages = {754--762},
	Title = {Achieving k-anonymity in privacy-aware location-based services},
	Year = {2014}}
	
	@inproceedings{liu2013game,
	Author = {Liu, Xinxin and Liu, Kaikai and Guo, Linke and Li, Xiaolin and Fang, Yuguang},
	Booktitle = {INFOCOM, 2013 Proceedings IEEE},
	Organization = {IEEE},
	Pages = {2985--2993},
	Title = {A game-theoretic approach for achieving k-anonymity in location based services},
	Year = {2013}}
	
	@inproceedings{kido2005protection,
	Author = {Kido, Hidetoshi and Yanagisawa, Yutaka and Satoh, Tetsuji},
	Booktitle = {Data Engineering Workshops, 2005. 21st International Conference on},
	Organization = {IEEE},
	Pages = {1248--1248},
	Title = {Protection of location privacy using dummies for location-based services},
	Year = {2005}}
	
	@inproceedings{gedik2005location,
	Author = {Gedik, Bu{\u{g}}ra and Liu, Ling},
	Booktitle = {Distributed Computing Systems, 2005. ICDCS 2005. Proceedings. 25th IEEE International Conference on},
	Organization = {IEEE},
	Pages = {620--629},
	Title = {Location privacy in mobile systems: A personalized anonymization model},
	Year = {2005}}
	
	@inproceedings{bordenabe2014optimal,
	Author = {Bordenabe, Nicol{\'a}s E and Chatzikokolakis, Konstantinos and Palamidessi, Catuscia},
	Booktitle = {Proceedings of the 2014 ACM SIGSAC Conference on Computer and Communications Security},
	Organization = {ACM},
	Pages = {251--262},
	Title = {Optimal geo-indistinguishable mechanisms for location privacy},
	Year = {2014}}
	
	@incollection{duckham2005formal,
	Author = {Duckham, Matt and Kulik, Lars},
	Booktitle = {Pervasive computing},
	Pages = {152--170},
	Publisher = {Springer},
	Title = {A formal model of obfuscation and negotiation for location privacy},
	Year = {2005}}
	
	@inproceedings{kido2005anonymous,
	Author = {Kido, Hidetoshi and Yanagisawa, Yutaka and Satoh, Tetsuji},
	Booktitle = {Pervasive Services, 2005. ICPS'05. Proceedings. International Conference on},
	Organization = {IEEE},
	Pages = {88--97},
	Title = {An anonymous communication technique using dummies for location-based services},
	Year = {2005}}
	
	@incollection{duckham2006spatiotemporal,
	Author = {Duckham, Matt and Kulik, Lars and Birtley, Athol},
	Booktitle = {Geographic Information Science},
	Pages = {47--64},
	Publisher = {Springer},
	Title = {A spatiotemporal model of strategies and counter strategies for location privacy protection},
	Year = {2006}}
	
	@inproceedings{shankar2009privately,
	Author = {Shankar, Pravin and Ganapathy, Vinod and Iftode, Liviu},
	Booktitle = {Proceedings of the 11th international conference on Ubiquitous computing},
	Organization = {ACM},
	Pages = {31--40},
	Title = {Privately querying location-based services with SybilQuery},
	Year = {2009}}
	
	@inproceedings{chow2009faking,
	Author = {Chow, Richard and Golle, Philippe},
	Booktitle = {Proceedings of the 8th ACM workshop on Privacy in the electronic society},
	Organization = {ACM},
	Pages = {105--108},
	Title = {Faking contextual data for fun, profit, and privacy},
	Year = {2009}}
	
	@incollection{xue2009location,
	Author = {Xue, Mingqiang and Kalnis, Panos and Pung, Hung Keng},
	Booktitle = {Location and Context Awareness},
	Pages = {70--87},
	Publisher = {Springer},
	Title = {Location diversity: Enhanced privacy protection in location based services},
	Year = {2009}}
	
	@article{wernke2014classification,
	Author = {Wernke, Marius and Skvortsov, Pavel and D{\"u}rr, Frank and Rothermel, Kurt},
	Journal = {Personal and Ubiquitous Computing},
	Number = {1},
	Pages = {163--175},
	Publisher = {Springer-Verlag},
	Title = {A classification of location privacy attacks and approaches},
	Volume = {18},
	Year = {2014}}
	
	@misc{cai2015cloaking,
	Author = {Cai, Y. and Xu, G.},
	Month = jan # {~1},
	Note = {US Patent App. 14/472,462},
	Publisher = {Google Patents},
	Title = {Cloaking with footprints to provide location privacy protection in location-based services},
	Url = {https://www.google.com/patents/US20150007341},
	Year = {2015},
	Bdsk-Url-1 = {https://www.google.com/patents/US20150007341}}
	
	@article{gedik2008protecting,
	Author = {Gedik, Bu{\u{g}}ra and Liu, Ling},
	Journal = {Mobile Computing, IEEE Transactions on},
	Number = {1},
	Pages = {1--18},
	Publisher = {IEEE},
	Title = {Protecting location privacy with personalized k-anonymity: Architecture and algorithms},
	Volume = {7},
	Year = {2008}}
	
	@article{kalnis2006preserving,
	Author = {Kalnis, Panos and Ghinita, Gabriel and Mouratidis, Kyriakos and Papadias, Dimitris},
	Publisher = {TRB6/06},
	Title = {Preserving anonymity in location based services},
	Year = {2006}}
	
	@inproceedings{hoh2005protecting,
	Author = {Hoh, Baik and Gruteser, Marco},
	Booktitle = {Security and Privacy for Emerging Areas in Communications Networks, 2005. SecureComm 2005. First International Conference on},
	Organization = {IEEE},
	Pages = {194--205},
	Title = {Protecting location privacy through path confusion},
	Year = {2005}}
	
	@article{terrovitis2011privacy,
	Author = {Terrovitis, Manolis},
	Journal = {ACM SIGKDD Explorations Newsletter},
	Number = {1},
	Pages = {6--18},
	Publisher = {ACM},
	Title = {Privacy preservation in the dissemination of location data},
	Volume = {13},
	Year = {2011}}
	
	@article{shin2012privacy,
	Author = {Shin, Kang G and Ju, Xiaoen and Chen, Zhigang and Hu, Xin},
	Journal = {Wireless Communications, IEEE},
	Number = {1},
	Pages = {30--39},
	Publisher = {IEEE},
	Title = {Privacy protection for users of location-based services},
	Volume = {19},
	Year = {2012}}
	
	@article{khoshgozaran2011location,
	Author = {Khoshgozaran, Ali and Shahabi, Cyrus and Shirani-Mehr, Houtan},
	Journal = {Knowledge and Information Systems},
	Number = {3},
	Pages = {435--465},
	Publisher = {Springer},
	Title = {Location privacy: going beyond K-anonymity, cloaking and anonymizers},
	Volume = {26},
	Year = {2011}}
	
	@incollection{chatzikokolakis2015geo,
	Author = {Chatzikokolakis, Konstantinos and Palamidessi, Catuscia and Stronati, Marco},
	Booktitle = {Distributed Computing and Internet Technology},
	Pages = {49--72},
	Publisher = {Springer},
	Title = {Geo-indistinguishability: A Principled Approach to Location Privacy},
	Year = {2015}}
	
	@inproceedings{ngo2015location,
	Author = {Ngo, Hoa and Kim, Jong},
	Booktitle = {Computer Security Foundations Symposium (CSF), 2015 IEEE 28th},
	Organization = {IEEE},
	Pages = {63--74},
	Title = {Location Privacy via Differential Private Perturbation of Cloaking Area},
	Year = {2015}}
	
	@inproceedings{palanisamy2011mobimix,
	Author = {Palanisamy, Balaji and Liu, Ling},
	Booktitle = {Data Engineering (ICDE), 2011 IEEE 27th International Conference on},
	Organization = {IEEE},
	Pages = {494--505},
	Title = {Mobimix: Protecting location privacy with mix-zones over road networks},
	Year = {2011}}
	
	@inproceedings{um2010advanced,
	Author = {Um, Jung-Ho and Kim, Hee-Dae and Chang, Jae-Woo},
	Booktitle = {Social Computing (SocialCom), 2010 IEEE Second International Conference on},
	Organization = {IEEE},
	Pages = {1093--1098},
	Title = {An advanced cloaking algorithm using Hilbert curves for anonymous location based service},
	Year = {2010}}
	
	@inproceedings{bamba2008supporting,
	Author = {Bamba, Bhuvan and Liu, Ling and Pesti, Peter and Wang, Ting},
	Booktitle = {Proceedings of the 17th international conference on World Wide Web},
	Organization = {ACM},
	Pages = {237--246},
	Title = {Supporting anonymous location queries in mobile environments with privacygrid},
	Year = {2008}}
	
	@inproceedings{zhangwei2010distributed,
	Author = {Zhangwei, Huang and Mingjun, Xin},
	Booktitle = {Networks Security Wireless Communications and Trusted Computing (NSWCTC), 2010 Second International Conference on},
	Organization = {IEEE},
	Pages = {468--471},
	Title = {A distributed spatial cloaking protocol for location privacy},
	Volume = {2},
	Year = {2010}}
	
	@article{chow2011spatial,
	Author = {Chow, Chi-Yin and Mokbel, Mohamed F and Liu, Xuan},
	Journal = {GeoInformatica},
	Number = {2},
	Pages = {351--380},
	Publisher = {Springer},
	Title = {Spatial cloaking for anonymous location-based services in mobile peer-to-peer environments},
	Volume = {15},
	Year = {2011}}
	
	@inproceedings{lu2008pad,
	Author = {Lu, Hua and Jensen, Christian S and Yiu, Man Lung},
	Booktitle = {Proceedings of the Seventh ACM International Workshop on Data Engineering for Wireless and Mobile Access},
	Organization = {ACM},
	Pages = {16--23},
	Title = {Pad: privacy-area aware, dummy-based location privacy in mobile services},
	Year = {2008}}
	
	@incollection{khoshgozaran2007blind,
	Author = {Khoshgozaran, Ali and Shahabi, Cyrus},
	Booktitle = {Advances in Spatial and Temporal Databases},
	Pages = {239--257},
	Publisher = {Springer},
	Title = {Blind evaluation of nearest neighbor queries using space transformation to preserve location privacy},
	Year = {2007}}
	
	@inproceedings{ghinita2008private,
	Author = {Ghinita, Gabriel and Kalnis, Panos and Khoshgozaran, Ali and Shahabi, Cyrus and Tan, Kian-Lee},
	Booktitle = {Proceedings of the 2008 ACM SIGMOD international conference on Management of data},
	Organization = {ACM},
	Pages = {121--132},
	Title = {Private queries in location based services: anonymizers are not necessary},
	Year = {2008}}
	
	@article{paulet2014privacy,
	Author = {Paulet, Russell and Kaosar, Md Golam and Yi, Xun and Bertino, Elisa},
	Journal = {Knowledge and Data Engineering, IEEE Transactions on},
	Number = {5},
	Pages = {1200--1210},
	Publisher = {IEEE},
	Title = {Privacy-preserving and content-protecting location based queries},
	Volume = {26},
	Year = {2014}}
	
	@article{nguyen2013differential,
	Author = {Nguyen, Hiep H and Kim, Jong and Kim, Yoonho},
	Journal = {Journal of Computing Science and Engineering},
	Number = {3},
	Pages = {177--186},
	Title = {Differential privacy in practice},
	Volume = {7},
	Year = {2013}}
	
	@inproceedings{lee2012differential,
	Author = {Lee, Jaewoo and Clifton, Chris},
	Booktitle = {Proceedings of the 18th ACM SIGKDD international conference on Knowledge discovery and data mining},
	Organization = {ACM},
	Pages = {1041--1049},
	Title = {Differential identifiability},
	Year = {2012}}
	
	@inproceedings{andres2013geo,
	Author = {Andr{\'e}s, Miguel E and Bordenabe, Nicol{\'a}s E and Chatzikokolakis, Konstantinos and Palamidessi, Catuscia},
	Booktitle = {Proceedings of the 2013 ACM SIGSAC conference on Computer \& communications security},
	Organization = {ACM},
	Pages = {901--914},
	Title = {Geo-indistinguishability: Differential privacy for location-based systems},
	Year = {2013}}
	
	@inproceedings{machanavajjhala2008privacy,
	Author = {Machanavajjhala, Ashwin and Kifer, Daniel and Abowd, John and Gehrke, Johannes and Vilhuber, Lars},
	Booktitle = {Data Engineering, 2008. ICDE 2008. IEEE 24th International Conference on},
	Organization = {IEEE},
	Pages = {277--286},
	Title = {Privacy: Theory meets practice on the map},
	Year = {2008}}
	
	@article{dewri2013local,
	Author = {Dewri, Rinku},
	Journal = {Mobile Computing, IEEE Transactions on},
	Number = {12},
	Pages = {2360--2372},
	Publisher = {IEEE},
	Title = {Local differential perturbations: Location privacy under approximate knowledge attackers},
	Volume = {12},
	Year = {2013}}
	
	@inproceedings{chatzikokolakis2013broadening,
	Author = {Chatzikokolakis, Konstantinos and Andr{\'e}s, Miguel E and Bordenabe, Nicol{\'a}s Emilio and Palamidessi, Catuscia},
	Booktitle = {Privacy Enhancing Technologies},
	Organization = {Springer},
	Pages = {82--102},
	Title = {Broadening the Scope of Differential Privacy Using Metrics.},
	Year = {2013}}
	
	@inproceedings{zhong2009distributed,
	Author = {Zhong, Ge and Hengartner, Urs},
	Booktitle = {Pervasive Computing and Communications, 2009. PerCom 2009. IEEE International Conference on},
	Organization = {IEEE},
	Pages = {1--10},
	Title = {A distributed k-anonymity protocol for location privacy},
	Year = {2009}}
	
	@inproceedings{ho2011differential,
	Author = {Ho, Shen-Shyang and Ruan, Shuhua},
	Booktitle = {Proceedings of the 4th ACM SIGSPATIAL International Workshop on Security and Privacy in GIS and LBS},
	Organization = {ACM},
	Pages = {17--24},
	Title = {Differential privacy for location pattern mining},
	Year = {2011}}
	
	@inproceedings{cheng2006preserving,
	Author = {Cheng, Reynold and Zhang, Yu and Bertino, Elisa and Prabhakar, Sunil},
	Booktitle = {Privacy Enhancing Technologies},
	Organization = {Springer},
	Pages = {393--412},
	Title = {Preserving user location privacy in mobile data management infrastructures},
	Year = {2006}}
	
	@article{beresford2003location,
	Author = {Beresford, Alastair R and Stajano, Frank},
	Journal = {IEEE Pervasive computing},
	Number = {1},
	Pages = {46--55},
	Publisher = {IEEE},
	Title = {Location privacy in pervasive computing},
	Year = {2003}}
	
	@inproceedings{freudiger2009optimal,
	Author = {Freudiger, Julien and Shokri, Reza and Hubaux, Jean-Pierre},
	Booktitle = {Privacy enhancing technologies},
	Organization = {Springer},
	Pages = {216--234},
	Title = {On the optimal placement of mix zones},
	Year = {2009}}
	
	@article{krumm2009survey,
	Author = {Krumm, John},
	Journal = {Personal and Ubiquitous Computing},
	Number = {6},
	Pages = {391--399},
	Publisher = {Springer},
	Title = {A survey of computational location privacy},
	Volume = {13},
	Year = {2009}}
	
	@article{Rakhshan2016letter,
	Author = {Rakhshan, Ali and Pishro-Nik, Hossein},
	Journal = {IEEE Wireless Communications Letter},
	Publisher = {IEEE},
	Title = {Interference Models for Vehicular Ad Hoc Networks},
	Year = {2016, submitted}}
	
	@article{Rakhshan2015Journal,
	Author = {Rakhshan, Ali and Pishro-Nik, Hossein},
	Journal = {IEEE Transactions on Wireless Communications},
	Publisher = {IEEE},
	Title = {Improving Safety on Highways by Customizing Vehicular Ad Hoc Networks},
	Year = {to appear, 2017}}
	
	@inproceedings{Rakhshan2015Cogsima,
	Author = {Rakhshan, Ali and Pishro-Nik, Hossein},
	Booktitle = {IEEE International Multi-Disciplinary Conference on Cognitive Methods in Situation Awareness and Decision Support},
	Organization = {IEEE},
	Title = {A New Approach to Customization of Accident Warning Systems to Individual Drivers},
	Year = {2015}}
	
	@inproceedings{Rakhshan2015CISS,
	Author = {Rakhshan, Ali and Pishro-Nik, Hossein and Nekoui, Mohammad},
	Booktitle = {Conference on Information Sciences and Systems},
	Organization = {IEEE},
	Pages = {1--6},
	Title = {Driver-based adaptation of Vehicular Ad Hoc Networks for design of active safety systems},
	Year = {2015}}
	
	@inproceedings{Rakhshan2014IV,
	Author = {Rakhshan, Ali and Pishro-Nik, Hossein and Ray, Evan},
	Booktitle = {Intelligent Vehicles Symposium},
	Organization = {IEEE},
	Pages = {1181--1186},
	Title = {Real-time estimation of the distribution of brake response times for an individual driver using Vehicular Ad Hoc Network.},
	Year = {2014}}
	
	@inproceedings{Rakhshan2013Globecom,
	Author = {Rakhshan, Ali and Pishro-Nik, Hossein and Fisher, Donald and Nekoui, Mohammad},
	Booktitle = {IEEE Global Communications Conference},
	Organization = {IEEE},
	Pages = {1333--1337},
	Title = {Tuning collision warning algorithms to individual drivers for design of active safety systems.},
	Year = {2013}}
	
	@article{Nekoui2012Journal,
	Author = {Nekoui, Mohammad and Pishro-Nik, Hossein},
	Journal = {IEEE Transactions on Wireless Communications},
	Number = {8},
	Pages = {2895--2905},
	Publisher = {IEEE},
	Title = {Throughput Scaling laws for Vehicular Ad Hoc Networks},
	Volume = {11},
	Year = {2012}}
	
	
	
	
	
	
	
	
	
	@article{Nekoui2011Journal,
	Author = {Nekoui, Mohammad and Pishro-Nik, Hossein and Ni, Daiheng},
	Journal = {International Journal of Vehicular Technology},
	Pages = {1--11},
	Publisher = {Hindawi Publishing Corporation},
	Title = {Analytic Design of Active Safety Systems for Vehicular Ad hoc Networks},
	Volume = {2011},
	Year = {2011}}
	
	
	
	
	
	
	@article{shokri2014optimal,
	title={Optimal user-centric data obfuscation},
	author={Shokri, Reza},
	journal={arXiv preprint arXiv:1402.3426},
	year={2014}
	}
	@article{chatzikokolakis2015location,
	title={Location privacy via geo-indistinguishability},
	author={Chatzikokolakis, Konstantinos and Palamidessi, Catuscia and Stronati, Marco},
	journal={ACM SIGLOG News},
	volume={2},
	number={3},
	pages={46--69},
	year={2015},
	publisher={ACM}
	
	}
	@inproceedings{shokri2011quantifying2,
	title={Quantifying location privacy: the case of sporadic location exposure},
	author={Shokri, Reza and Theodorakopoulos, George and Danezis, George and Hubaux, Jean-Pierre and Le Boudec, Jean-Yves},
	booktitle={Privacy Enhancing Technologies},
	pages={57--76},
	year={2011},
	organization={Springer}
	}
	
	
	@inproceedings{Mont1603:Defining,
	AUTHOR="Zarrin Montazeri and Amir Houmansadr and Hossein Pishro-Nik",
	TITLE="Defining Perfect Location Privacy Using Anonymization",
	BOOKTITLE="2016 Annual Conference on Information Science and Systems (CISS) (CISS
	2016)",
	ADDRESS="Princeton, USA",
	DAYS=16,
	MONTH=mar,
	YEAR=2016,
	KEYWORDS="Information Theoretic Privacy; location-based services; Location Privacy;
	Information Theory",
	ABSTRACT="The popularity of mobile devices and location-based services (LBS) have
	created great concerns regarding the location privacy of users of such
	devices and services. Anonymization is a common technique that is often
	being used to protect the location privacy of LBS users. In this paper, we
	provide a general information theoretic definition for location privacy. In
	particular, we define perfect location privacy. We show that under certain
	conditions, perfect privacy is achieved if the pseudonyms of users is
	changed after o(N^(2/r?1)) observations by the adversary, where N is the
	number of users and r is the number of sub-regions or locations.
	"
	}
	@article{our-isita-location,
	Author = {Zarrin Montazeri and Amir Houmansadr and Hossein Pishro-Nik},
	Journal = {IEEE International Symposium on Information Theory and Its Applications (ISITA)},
	Title = {Achieving Perfect Location Privacy in Markov Models Using Anonymization},
	Year = {2016}
	}
	@article{our-TIFS,
	Author = {Zarrin Montazeri and Hossein Pishro-Nik and Amir Houmansadr},
	Journal = {IEEE Transactions on Information Forensics and Security, under revison},
	Title = {Perfect Location Privacy Using Anonymization in Mobile Networks},
	Year = {2016},
	note={Available on arxiv.org}
	}
	
	
	
	@techreport{sampigethaya2005caravan,
	title={CARAVAN: Providing location privacy for VANET},
	author={Sampigethaya, Krishna and Huang, Leping and Li, Mingyan and Poovendran, Radha and Matsuura, Kanta and Sezaki, Kaoru},
	year={2005},
	institution={DTIC Document}
	}
	@incollection{buttyan2007effectiveness,
	title={On the effectiveness of changing pseudonyms to provide location privacy in VANETs},
	author={Butty{\'a}n, Levente and Holczer, Tam{\'a}s and Vajda, Istv{\'a}n},
	booktitle={Security and Privacy in Ad-hoc and Sensor Networks},
	pages={129--141},
	year={2007},
	publisher={Springer}
	}
	@article{sampigethaya2007amoeba,
	title={AMOEBA: Robust location privacy scheme for VANET},
	author={Sampigethaya, Krishna and Li, Mingyan and Huang, Leping and Poovendran, Radha},
	journal={Selected Areas in communications, IEEE Journal on},
	volume={25},
	number={8},
	pages={1569--1589},
	year={2007},
	publisher={IEEE}
	}
	
	@article{lu2012pseudonym,
	title={Pseudonym changing at social spots: An effective strategy for location privacy in vanets},
	author={Lu, Rongxing and Li, Xiaodong and Luan, Tom H and Liang, Xiaohui and Shen, Xuemin},
	journal={Vehicular Technology, IEEE Transactions on},
	volume={61},
	number={1},
	pages={86--96},
	year={2012},
	publisher={IEEE}
	}
	@inproceedings{lu2010sacrificing,
	title={Sacrificing the plum tree for the peach tree: A socialspot tactic for protecting receiver-location privacy in VANET},
	author={Lu, Rongxing and Lin, Xiaodong and Liang, Xiaohui and Shen, Xuemin},
	booktitle={Global Telecommunications Conference (GLOBECOM 2010), 2010 IEEE},
	pages={1--5},
	year={2010},
	organization={IEEE}
	}
	@inproceedings{lin2011stap,
	title={STAP: A social-tier-assisted packet forwarding protocol for achieving receiver-location privacy preservation in VANETs},
	author={Lin, Xiaodong and Lu, Rongxing and Liang, Xiaohui and Shen, Xuemin Sherman},
	booktitle={INFOCOM, 2011 Proceedings IEEE},
	pages={2147--2155},
	year={2011},
	organization={IEEE}
	}
	@inproceedings{gerlach2007privacy,
	title={Privacy in VANETs using changing pseudonyms-ideal and real},
	author={Gerlach, Matthias and Guttler, Felix},
	booktitle={Vehicular Technology Conference, 2007. VTC2007-Spring. IEEE 65th},
	pages={2521--2525},
	year={2007},
	organization={IEEE}
	}
	@inproceedings{el2002security,
	title={Security issues in a future vehicular network},
	author={El Zarki, Magda and Mehrotra, Sharad and Tsudik, Gene and Venkatasubramanian, Nalini},
	booktitle={European Wireless},
	volume={2},
	year={2002}
	}
	
	@article{hubaux2004security,
	title={The security and privacy of smart vehicles},
	author={Hubaux, Jean-Pierre and Capkun, Srdjan and Luo, Jun},
	journal={IEEE Security \& Privacy Magazine},
	volume={2},
	number={LCA-ARTICLE-2004-007},
	pages={49--55},
	year={2004}
	}
	
	
	
	@inproceedings{duri2002framework,
	title={Framework for security and privacy in automotive telematics},
	author={Duri, Sastry and Gruteser, Marco and Liu, Xuan and Moskowitz, Paul and Perez, Ronald and Singh, Moninder and Tang, Jung-Mu},
	booktitle={Proceedings of the 2nd international workshop on Mobile commerce},
	pages={25--32},
	year={2002},
	organization={ACM}
	}
	@misc{NS-3,
	Howpublished = {\url{https://www.nsnam.org/}}},
}
@misc{testbed,
	Howpublished = {\url{http://www.its.dot.gov/testbed/PDF/SE-MI-Resource-Guide-9-3-1.pdf}}},
@misc{NGSIM,
	Howpublished = {\url{http://ops.fhwa.dot.gov/trafficanalysistools/ngsim.htm}},
}

@misc{National-a2013,
	Author = {National Highway Traffic Safety Administration},
	Howpublished = {\url{http://ops.fhwa.dot.gov/trafficanalysistools/ngsim.htm}},
	Title = {2013 Motor Vehicle Crashes: Overview. Traffic Safety Factors},
	Year = {2013}
}

@inproceedings{karnadi2007rapid,
	title={Rapid generation of realistic mobility models for VANET},
	author={Karnadi, Feliz Kristianto and Mo, Zhi Hai and Lan, Kun-chan},
	booktitle={Wireless Communications and Networking Conference, 2007. WCNC 2007. IEEE},
	pages={2506--2511},
	year={2007},
	organization={IEEE}
}
@inproceedings{saha2004modeling,
	title={Modeling mobility for vehicular ad-hoc networks},
	author={Saha, Amit Kumar and Johnson, David B},
	booktitle={Proceedings of the 1st ACM international workshop on Vehicular ad hoc networks},
	pages={91--92},
	year={2004},
	organization={ACM}
}
@inproceedings{lee2006modeling,
	title={Modeling steady-state and transient behaviors of user mobility: formulation, analysis, and application},
	author={Lee, Jong-Kwon and Hou, Jennifer C},
	booktitle={Proceedings of the 7th ACM international symposium on Mobile ad hoc networking and computing},
	pages={85--96},
	year={2006},
	organization={ACM}
}
@inproceedings{yoon2006building,
	title={Building realistic mobility models from coarse-grained traces},
	author={Yoon, Jungkeun and Noble, Brian D and Liu, Mingyan and Kim, Minkyong},
	booktitle={Proceedings of the 4th international conference on Mobile systems, applications and services},
	pages={177--190},
	year={2006},
	organization={ACM}
}

@inproceedings{choffnes2005integrated,
	title={An integrated mobility and traffic model for vehicular wireless networks},
	author={Choffnes, David R and Bustamante, Fabi{\'a}n E},
	booktitle={Proceedings of the 2nd ACM international workshop on Vehicular ad hoc networks},
	pages={69--78},
	year={2005},
	organization={ACM}
}

@inproceedings{Qian2008Globecom,
	title={CA Secure VANET MAC Protocol for DSRC Applications},
	author={Yi, Q. and Lu, K. and Moyeri, N.{\'a}n E},
	booktitle={Proceedings of IEEE GLOBECOM 2008},
	pages={1--5},
	year={2008},
	organization={IEEE}
}





@inproceedings{naumov2006evaluation,
	title={An evaluation of inter-vehicle ad hoc networks based on realistic vehicular traces},
	author={Naumov, Valery and Baumann, Rainer and Gross, Thomas},
	booktitle={Proceedings of the 7th ACM international symposium on Mobile ad hoc networking and computing},
	pages={108--119},
	year={2006},
	organization={ACM}
}
@article{sommer2008progressing,
	title={Progressing toward realistic mobility models in VANET simulations},
	author={Sommer, Christoph and Dressler, Falko},
	journal={Communications Magazine, IEEE},
	volume={46},
	number={11},
	pages={132--137},
	year={2008},
	publisher={IEEE}
}




@inproceedings{mahajan2006urban,
	title={Urban mobility models for vanets},
	author={Mahajan, Atulya and Potnis, Niranjan and Gopalan, Kartik and Wang, Andy},
	booktitle={2nd IEEE International Workshop on Next Generation Wireless Networks},
	volume={33},
	year={2006}
}

@inproceedings{Rakhshan2016packet,
	title={Packet success probability derivation in a vehicular ad hoc network for a highway scenario},
	author={Rakhshan, Ali and Pishro-Nik, Hossein},
	booktitle={2016 Annual Conference on Information Science and Systems (CISS)},
	pages={210--215},
	year={2016},
	organization={IEEE}
}

@inproceedings{Rakhshan2016CISS,
	Author = {Rakhshan, Ali and Pishro-Nik, Hossein},
	Booktitle = {Conference on Information Sciences and Systems},
	Organization = {IEEE},
	Pages = {210--215},
	Title = {Packet Success Probability Derivation in a Vehicular Ad Hoc Network for a Highway Scenario},
	Year = {2016}}

@article{Nekoui2013Journal,
	Author = {Nekoui, Mohammad and Pishro-Nik, Hossein},
	Journal = {Journal on Selected Areas in Communications, Special Issue on Emerging Technologies in Communications},
	Number = {9},
	Pages = {491--503},
	Publisher = {IEEE},
	Title = {Analytic Design of Active Safety Systems for Vehicular Ad hoc Networks},
	Volume = {31},
	Year = {2013}}


@inproceedings{Nekoui2011MOBICOM,
	Author = {Nekoui, Mohammad and Pishro-Nik, Hossein},
	Booktitle = {MOBICOM workshop on VehiculAr InterNETworking},
	Organization = {ACM},
	Title = {Analytic Design of Active Vehicular Safety Systems in Sparse Traffic},
	Year = {2011}}

@inproceedings{Nekoui2011VTC,
	Author = {Nekoui, Mohammad and Pishro-Nik, Hossein},
	Booktitle = {VTC-Fall},
	Organization = {IEEE},
	Title = {Analytical Design of Inter-vehicular Communications for Collision Avoidance},
	Year = {2011}}

@inproceedings{Bovee2011VTC,
	Author = {Bovee, Ben Louis and Nekoui, Mohammad and Pishro-Nik, Hossein},
	Booktitle = {VTC-Fall},
	Organization = {IEEE},
	Title = {Evaluation of the Universal Geocast Scheme For VANETs},
	Year = {2011}}

@inproceedings{Nekoui2010MOBICOM,
	Author = {Nekoui, Mohammad and Pishro-Nik, Hossein},
	Booktitle = {MOBICOM},
	Organization = {ACM},
	Title = {Fundamental Tradeoffs in Vehicular Ad Hoc Networks},
	Year = {2010}}

@inproceedings{Nekoui2010IVCS,
	Author = {Nekoui, Mohammad and Pishro-Nik, Hossein},
	Booktitle = {IVCS},
	Organization = {IEEE},
	Title = {A Universal Geocast Scheme for Vehicular Ad Hoc Networks},
	Year = {2010}}

@inproceedings{Nekoui2009ITW,
	Author = {Nekoui, Mohammad and Pishro-Nik, Hossein},
	Booktitle = {IEEE Communications Society Conference on Sensor, Mesh and Ad Hoc Communications and Networks Workshops},
	Organization = {IEEE},
	Pages = {1--3},
	Title = {A Geometrical Analysis of Obstructed Wireless Networks},
	Year = {2009}}

@article{Eslami2013Journal,
	Author = {Eslami, Ali and Nekoui, Mohammad and Pishro-Nik, Hossein and Fekri, Faramarz},
	Journal = {ACM Transactions on Sensor Networks},
	Number = {4},
	Pages = {51},
	Publisher = {ACM},
	Title = {Results on finite wireless sensor networks: Connectivity and coverage},
	Volume = {9},
	Year = {2013}}


@article{Jiafu2014Journal,
	Author = {Jiafu, W. and Zhang, D. and Zhao, S. and Yang, L. and Lloret, J.},
	Journal = {Communications Magazine},
	Number = {8},
	Pages = {106-113},
	Publisher = {IEEE},
	Title = {Context-aware vehicular cyber-physical systems with cloud support: architecture, challenges, and solutions},
	Volume = {52},
	Year = {2014}}

@inproceedings{Haas2010ACM,
	Author = {Haas, J.J. and Hu, Y.},
	Booktitle = {international workshop on VehiculAr InterNETworking},
	Organization = {ACM},
	Title = {Communication requirements for crash avoidance.},
	Year = {2010}}

@inproceedings{Yi2008GLOBECOM,
	Author = {Yi, Q. and Lu, K. and Moayeri, N.},
	Booktitle = {GLOBECOM},
	Organization = {IEEE},
	Title = {CA Secure VANET MAC Protocol for DSRC Applications.},
	Year = {2008}}

@inproceedings{Mughal2010ITSim,
	Author = {Mughal, B.M. and Wagan, A. and Hasbullah, H.},
	Booktitle = {International Symposium on Information Technology (ITSim)},
	Organization = {IEEE},
	Title = {Efficient congestion control in VANET for safety messaging.},
	Year = {2010}}

@article{Chang2011Journal,
	Author = {Chang, Y. and Lee, C. and Copeland, J.},
	Journal = {Selected Areas in Communications},
	Pages = {236 –249},
	Publisher = {IEEE},
	Title = {Goodput enhancement of VANETs in noisy CSMA/CA channels},
	Volume = {29},
	Year = {2011}}

@article{Garcia-Costa2011Journal,
	Author = {Garcia-Costa, C. and Egea-Lopez, E. and Tomas-Gabarron, J. B. and Garcia-Haro, J. and Haas, Z. J.},
	Journal = {Transactions on Intelligent Transportation Systems},
	Pages = {1 –16},
	Publisher = {IEEE},
	Title = {A stochastic model for chain collisions of vehicles equipped with vehicular communications},
	Volume = {99},
	Year = {2011}}

@article{Carbaugh2011Journal,
	Author = {Carbaugh, J. and Godbole,  D. N. and Sengupta, R. and Garcia-Haro, J. and Haas, Z. J.},
	Publisher = {Transportation Research Part C (Emerging Technologies)},
	Title = {Safety and capacity analysis of automated and manual highway systems},
	Year = {1997}}

@article{Goh2004Journal,
	Author = {Goh, P. and Wong, Y.},
	Publisher = {Appl Health Econ Health Policy},
	Title = {Driver perception response time during the signal change interval},
	Year = {2004}}

@article{Chang1985Journal,
	Author = {Chang, M.S. and Santiago, A.J.},
	Pages = {20-30},
	Publisher = {Transportation Research Record},
	Title = {Timing traffic signal changes based on driver behavior},
	Volume = {1027},
	Year = {1985}}

@article{Green2000Journal,
	Author = {Green, M.},
	Pages = {195-216},
	Publisher = {Transportation Human Factors},
	Title = {How long does it take to stop? Methodological analysis of driver perception-brake times.},
	Year = {2000}}

@article{Koppa2005,
	Author = {Koppa, R.J.},
	Pages = {195-216},
	Publisher = {http://www.fhwa.dot.gov/publications/},
	Title = {Human Factors},
	Year = {2005}}

@inproceedings{Maxwell2010ETC,
	Author = {Maxwell, A. and Wood, K.},
	Booktitle = {Europian Transport Conference},
	Organization = {http://www.etcproceedings.org/paper/review-of-traffic-signals-on-high-speed-roads},
	Title = {Review of Traffic Signals on High Speed Road},
	Year = {2010}}

@article{Wortman1983,
	Author = {Wortman, R.H. and Matthias, J.S.},
	Publisher = {Arizona Department of Transportation},
	Title = {An Evaluation of Driver Behavior at Signalized Intersections},
	Year = {1983}}
@inproceedings{Zhang2007IASTED,
	Author = {Zhang, X. and Bham, G.H.},
	Booktitle = {18th IASTED International Conference: modeling and simulation},
	Title = {Estimation of driver reaction time from detailed vehicle trajectory data.},
	Year = {2007}}


@inproceedings{bai2003important,
	title={IMPORTANT: A framework to systematically analyze the Impact of Mobility on Performance of RouTing protocols for Adhoc NeTworks},
	author={Bai, Fan and Sadagopan, Narayanan and Helmy, Ahmed},
	booktitle={INFOCOM 2003. Twenty-second annual joint conference of the IEEE computer and communications. IEEE societies},
	volume={2},
	pages={825--835},
	year={2003},
	organization={IEEE}
}


@inproceedings{abedi2008enhancing,
	title={Enhancing AODV routing protocol using mobility parameters in VANET},
	author={Abedi, Omid and Fathy, Mahmood and Taghiloo, Jamshid},
	booktitle={Computer Systems and Applications, 2008. AICCSA 2008. IEEE/ACS International Conference on},
	pages={229--235},
	year={2008},
	organization={IEEE}
}


@article{AlSultan2013Journal,
	Author = {Al-Sultan, Saif and Al-Bayatti, Ali H. and Zedan, Hussien},
	Journal = {IEEE Transactions on Vehicular Technology},
	Number = {9},
	Pages = {4264-4275},
	Publisher = {IEEE},
	Title = {Context Aware Driver Behaviour Detection System in Intelligent Transportation Systems},
	Volume = {62},
	Year = {2013}}






@article{Leow2008ITS,
	Author = {Leow, Woei Ling and Ni, Daiheng and Pishro-Nik, Hossein},
	Journal = {IEEE Transactions on Intelligent Transportation Systems},
	Number = {2},
	Pages = {369--374},
	Publisher = {IEEE},
	Title = {A Sampling Theorem Approach to Traffic Sensor Optimization},
	Volume = {9},
	Year = {2008}}



@article{REU2007,
	Author = {D. Ni and H. Pishro-Nik and R. Prasad and M. R. Kanjee and H. Zhu and T. Nguyen},
	Journal = {in 14th World Congress on Intelligent Transport Systems},
	Title = {Development of a prototype intersection collision avoidance system under VII},
	Year = {2007}}




@inproceedings{salamatian2013hide,
	title={How to hide the elephant-or the donkey-in the room: Practical privacy against statistical inference for large data.},
	author={Salamatian, Salman and Zhang, Amy and du Pin Calmon, Flavio and Bhamidipati, Sandilya and Fawaz, Nadia and Kveton, Branislav and Oliveira, Pedro and Taft, Nina},
	booktitle={GlobalSIP},
	pages={269--272},
	year={2013}
}

@article{sankar2013utility,
	title={Utility-privacy tradeoffs in databases: An information-theoretic approach},
	author={Sankar, Lalitha and Rajagopalan, S Raj and Poor, H Vincent},
	journal={Information Forensics and Security, IEEE Transactions on},
	volume={8},
	number={6},
	pages={838--852},
	year={2013},
	publisher={IEEE}
}
@inproceedings{ghinita2007prive,
	title={PRIVE: anonymous location-based queries in distributed mobile systems},
	author={Ghinita, Gabriel and Kalnis, Panos and Skiadopoulos, Spiros},
	booktitle={Proceedings of the 16th international conference on World Wide Web},
	pages={371--380},
	year={2007},
	organization={ACM}
}

@article{beresford2004mix,
	title={Mix zones: User privacy in location-aware services},
	author={Beresford, Alastair R and Stajano, Frank},
	year={2004},
	publisher={IEEE}
}

%@inproceedings{Mont1610Achieving,
	%  title={Achieving Perfect Location Privacy in Markov Models Using Anonymization},
	%  author={Montazeri, Zarrin and Houmansadr, Amir and H.Pishro-Nik},
	%  booktitle="2016 International Symposium on Information Theory and its Applications
	%  (ISITA2016)",
	%  address="Monterey, USA",
	%  days=30,
	%  month=oct,
	%  year=2016,
	%}

@article{csiszar1996almost,
	title={Almost independence and secrecy capacity},
	author={Csisz{\'a}r, Imre},
	journal={Problemy Peredachi Informatsii},
	volume={32},
	number={1},
	pages={48--57},
	year={1996},
	publisher={Russian Academy of Sciences, Branch of Informatics, Computer Equipment and Automatization}
}

@article{yamamoto1983source,
	title={A source coding problem for sources with additional outputs to keep secret from the receiver or wiretappers (corresp.)},
	author={Yamamoto, Hirosuke},
	journal={IEEE Transactions on Information Theory},
	volume={29},
	number={6},
	pages={918--923},
	year={1983},
	publisher={IEEE}
}


@inproceedings{calmon2015fundamental,
	title={Fundamental limits of perfect privacy},
	author={Calmon, Flavio P and Makhdoumi, Ali and M{\'e}dard, Muriel},
	booktitle={Information Theory (ISIT), 2015 IEEE International Symposium on},
	pages={1796--1800},
	year={2015},
	organization={IEEE}
}



@inproceedings{Lehman1999Large-Sample-Theory,
	title={Elements of Large Sample Theory},
	author={E. L. Lehman},
	organization={Springer},
	year={1999}
}


@inproceedings{Ferguson1999Large-Sample-Theory,
	title={A Course in Large Sample Theory},
	author={Thomas S. Ferguson},
	organization={CRC Press},
	year={1996}
}



@inproceedings{Dembo1999Large-Deviations,
	title={Large Deviation Techniques and Applications, Second Edition},
	author={A. Dembo and O. Zeitouni},
	organization={Springer},
	year={1998}
}


%%%%%%%%%%%%%%%%%%%%%%%%%%%%%%%%%%%%%%%%%%%%%%%%
Hossein's Coding Journals
%%%%%%%%%%%%%%%%%%%%%%

@ARTICLE{myoptics,
	AUTHOR =       "H. Pishro-Nik and N. Rahnavard and J. Ha and F. Fekri and A. Adibi ",
	TITLE =        "Low-density parity-check codes for volume holographic memory systems",
	JOURNAL =      " Appl. Opt.",
	YEAR =         "2003",
	volume =       "42",
	pages =        "861-870  "
}






@ARTICLE{myit,
	AUTHOR =       "H. Pishro-Nik and F. Fekri  ",
	TITLE =        "On Decoding of Low-Density Parity-Check Codes on the Binary Erasure Channel",
	JOURNAL =      "IEEE Trans. Inform. Theory",
	YEAR =         "2004",
	volume =       "50",
	pages =        "439--454"
}




@ARTICLE{myitpuncture,
	AUTHOR =       "H. Pishro-Nik and F. Fekri  ",
	TITLE =        "Results on Punctured Low-Density Parity-Check Codes and Improved Iterative Decoding Techniques",
	JOURNAL =      "IEEE Trans. on Inform. Theory",
	YEAR =         "2007",
	volume =       "53",
	number=        "2",
	pages =        "599--614",
	month= "February"
}




@ARTICLE{myitlinmimdist,
	AUTHOR =       "H. Pishro-Nik and F. Fekri",
	TITLE =        "Performance of Low-Density Parity-Check Codes With Linear Minimum Distance",
	JOURNAL =         "IEEE Trans. Inform. Theory ",
	YEAR =         "2006",
	volume =       "52",
	number="1",
	pages =        "292 --300"
}






@ARTICLE{myitnonuni,
	AUTHOR =       "H. Pishro-Nik and N. Rahnavard and F. Fekri  ",
	TITLE =        "Non-uniform Error Correction Using Low-Density Parity-Check Codes",
	JOURNAL =      "IEEE Trans. Inform. Theory",
	YEAR =         "2005",
	volume =       "51",
	number=  "7",
	pages =        "2702--2714"
}





@article{eslamitcomhybrid10,
	author = {A. Eslami and S. Vangala and H. Pishro-Nik},
	title = {Hybrid channel codes for highly efficient FSO/RF communication systems},
	journal = {IEEE Transactions on Communications},
	volume = {58},
	number = {10},
	year = {2010},
	pages = {2926--2938},
}


@article{eslamitcompolar13,
	author = {A. Eslami and H. Pishro-Nik},
	title = {On Finite-Length Performance of Polar Codes: Stopping Sets, Error Floor, and Concatenated Design},
	journal = {IEEE Transactions on Communications},
	volume = {61},
	number = {13},
	year = {2013},
	pages = {919--929},
}



@article{saeeditcom11,
	author = {H. Saeedi and H. Pishro-Nik and  A. H. Banihashemi},
	title = {Successive maximization for the systematic design of universally capacity approaching rate-compatible
	sequences of LDPC code ensembles over binary-input output-symmetric memoryless channels},
	journal = {IEEE Transactions on Communications},
	year = {2011},
	volume={59},
	number = {7}
}


@article{rahnavard07,
	author = {Rahnavard, N. and Pishro-Nik, H. and Fekri, F.},
	title = {Unequal Error Protection Using Partially Regular LDPC Codes},
	journal = {IEEE Transactions on Communications},
	year = {2007},
	volume = {55},
	number = {3},
	pages = {387 -- 391}
}


@article{hosseinira04,
	author = {H. Pishro-Nik and F. Fekri},
	title = {Irregular repeat-accumulate codes for volume holographic memory systems},
	journal = {Journal of Applied Optics},
	year = {2004},
	volume = {43},
	number = {27},
	pages = {5222--5227},
}


@article{azadeh2015Ephemeralkey,
	author = {A. Sheikholeslami and D. Goeckel and H. Pishro-Nik},
	title = {Jamming Based on an Ephemeral Key to Obtain Everlasting Security in Wireless Environments},
	journal = {IEEE Transactions on Wireless Communications},
	year = {2015},
	volume = {14},
	number = {11},
	pages = {6072--6081},
}


@article{azadeh2014Everlasting,
	author = {A. Sheikholeslami and D. Goeckel and H. Pishro-Nik},
	title = {Everlasting secrecy in disadvantaged wireless environments against sophisticated eavesdroppers},
	journal = {48th Asilomar Conference on Signals, Systems and Computers},
	year = {2014},
	pages = {1994--1998},
}


@article{azadeh2013ISIT,
	author = {A. Sheikholeslami and D. Goeckel and H. Pishro-Nik},
	title = {Artificial intersymbol interference (ISI) to exploit receiver imperfections for secrecy},
	journal = {IEEE International Symposium on Information Theory (ISIT)},
	year = {2013},
}


@article{azadeh2013Jsac,
	author = {A. Sheikholeslami and D. Goeckel and H. Pishro-Nik},
	title = {Jamming Based on an Ephemeral Key to Obtain Everlasting Security in Wireless Environments},
	journal = {IEEE Journal on Selected Areas in Communications},
	year = {2013},
	volume = {31},
	number = {9},
	pages = {1828--1839},
}


@article{azadeh2012Allerton,
	author = {A. Sheikholeslami and D. Goeckel and H. Pishro-Nik},
	title = {Exploiting the non-commutativity of nonlinear operators for information-theoretic security in disadvantaged wireless environments},
	journal = {50th Annual Allerton Conference on Communication, Control, and Computing},
	year = {2012},
	pages = {233--240},
}


@article{azadeh2012Infocom,
	author = {A. Sheikholeslami and D. Goeckel and H. Pishro-Nik},
	title = {Jamming Based on an Ephemeral Key to Obtain Everlasting Security in Wireless Environments},
	journal = {IEEE INFOCOM},
	year = {2012},
	pages = {1179--1187},
}

@article{1corser2016evaluating,
	title={Evaluating Location Privacy in Vehicular Communications and Applications},
	author={Corser, George P and Fu, Huirong and Banihani, Abdelnasser},
	journal={IEEE Transactions on Intelligent Transportation Systems},
	volume={17},
	number={9},
	pages={2658-2667},
	year={2016},
	publisher={IEEE}
}
@article{2zhang2016designing,
	title={On Designing Satisfaction-Ratio-Aware Truthful Incentive Mechanisms for k-Anonymity Location Privacy},
	author={Zhang, Yuan and Tong, Wei and Zhong, Sheng},
	journal={IEEE Transactions on Information Forensics and Security},
	volume={11},
	number={11},
	pages={2528--2541},
	year={2016},
	publisher={IEEE}
}
@article{3li2016privacy,
	title={Privacy-preserving Location Proof for Securing Large-scale Database-driven Cognitive Radio Networks},
	author={Li, Yi and Zhou, Lu and Zhu, Haojin and Sun, Limin},
	journal={IEEE Internet of Things Journal},
	volume={3},
	number={4},
	pages={563-571},
	year={2016},
	publisher={IEEE}
}
@article{4olteanu2016quantifying,
	title={Quantifying Interdependent Privacy Risks with Location Data},
	author={Olteanu, Alexandra-Mihaela and Huguenin, K{\'e}vin and Shokri, Reza and Humbert, Mathias and Hubaux, Jean-Pierre},
	journal={IEEE Transactions on Mobile Computing},
	year={2016},
	volume={PP},
	number={99},
	pages={1-1},
	publisher={IEEE}
}
@article{5yi2016practical,
	title={Practical Approximate k Nearest Neighbor Queries with Location and Query Privacy},
	author={Yi, Xun and Paulet, Russell and Bertino, Elisa and Varadharajan, Vijay},
	journal={IEEE Transactions on Knowledge and Data Engineering},
	volume={28},
	number={6},
	pages={1546--1559},
	year={2016},
	publisher={IEEE}
}
@article{6li2016privacy,
	title={Privacy Leakage of Location Sharing in Mobile Social Networks: Attacks and Defense},
	author={Li, Huaxin and Zhu, Haojin and Du, Suguo and Liang, Xiaohui and Shen, Xuemin},
	journal={IEEE Transactions on Dependable and Secure Computing},
	year={2016},
	volume={PP},
	number={99},
	publisher={IEEE}
}

@article{7murakami2016localization,
	title={Localization Attacks Using Matrix and Tensor Factorization},
	author={Murakami, Takao and Watanabe, Hajime},
	journal={IEEE Transactions on Information Forensics and Security},
	volume={11},
	number={8},
	pages={1647--1660},
	year={2016},
	publisher={IEEE}
}
@article{8zurbaran2015near,
	title={Near-Rand: Noise-based Location Obfuscation Based on Random Neighboring Points},
	author={Zurbaran, Mayra Alejandra and Avila, Karen and Wightman, Pedro and Fernandez, Michael},
	journal={IEEE Latin America Transactions},
	volume={13},
	number={11},
	pages={3661--3667},
	year={2015},
	publisher={IEEE}
}

@article{9tan2014anti,
	title={An anti-tracking source-location privacy protection protocol in wsns based on path extension},
	author={Tan, Wei and Xu, Ke and Wang, Dan},
	journal={IEEE Internet of Things Journal},
	volume={1},
	number={5},
	pages={461--471},
	year={2014},
	publisher={IEEE}
}

@article{10peng2014enhanced,
	title={Enhanced Location Privacy Preserving Scheme in Location-Based Services},
	author={Peng, Tao and Liu, Qin and Wang, Guojun},
	journal={IEEE Systems Journal},
	year={2014},
	volume={PP},
	number={99},
	pages={1-12},
	publisher={IEEE}
}
@article{11dewri2014exploiting,
	title={Exploiting service similarity for privacy in location-based search queries},
	author={Dewri, Rinku and Thurimella, Ramakrisha},
	journal={IEEE Transactions on Parallel and Distributed Systems},
	volume={25},
	number={2},
	pages={374--383},
	year={2014},
	publisher={IEEE}
}

@article{12hwang2014novel,
	title={A novel time-obfuscated algorithm for trajectory privacy protection},
	author={Hwang, Ren-Hung and Hsueh, Yu-Ling and Chung, Hao-Wei},
	journal={IEEE Transactions on Services Computing},
	volume={7},
	number={2},
	pages={126--139},
	year={2014},
	publisher={IEEE}
}
@article{13puttaswamy2014preserving,
	title={Preserving location privacy in geosocial applications},
	author={Puttaswamy, Krishna PN and Wang, Shiyuan and Steinbauer, Troy and Agrawal, Divyakant and El Abbadi, Amr and Kruegel, Christopher and Zhao, Ben Y},
	journal={IEEE Transactions on Mobile Computing},
	volume={13},
	number={1},
	pages={159--173},
	year={2014},
	publisher={IEEE}
}

@article{14zhang2014privacy,
	title={Privacy quantification model based on the Bayes conditional risk in Location-Based Services},
	author={Zhang, Xuejun and Gui, Xiaolin and Tian, Feng and Yu, Si and An, Jian},
	journal={Tsinghua Science and Technology},
	volume={19},
	number={5},
	pages={452--462},
	year={2014},
	publisher={TUP}
}

@article{15bilogrevic2014privacy,
	title={Privacy-preserving optimal meeting location determination on mobile devices},
	author={Bilogrevic, Igor and Jadliwala, Murtuza and Joneja, Vishal and Kalkan, K{\"u}bra and Hubaux, Jean-Pierre and Aad, Imad},
	journal={IEEE transactions on information forensics and security},
	volume={9},
	number={7},
	pages={1141--1156},
	year={2014},
	publisher={IEEE}
}
@article{16haghnegahdar2014privacy,
	title={Privacy Risks in Publishing Mobile Device Trajectories},
	author={Haghnegahdar, Alireza and Khabbazian, Majid and Bhargava, Vijay K},
	journal={IEEE Wireless Communications Letters},
	volume={3},
	number={3},
	pages={241--244},
	year={2014},
	publisher={IEEE}
}
@article{17malandrino2014verification,
	title={Verification and inference of positions in vehicular networks through anonymous beaconing},
	author={Malandrino, Francesco and Borgiattino, Carlo and Casetti, Claudio and Chiasserini, Carla-Fabiana and Fiore, Marco and Sadao, Roberto},
	journal={IEEE Transactions on Mobile Computing},
	volume={13},
	number={10},
	pages={2415--2428},
	year={2014},
	publisher={IEEE}
}
@article{18shokri2014hiding,
	title={Hiding in the mobile crowd: Locationprivacy through collaboration},
	author={Shokri, Reza and Theodorakopoulos, George and Papadimitratos, Panos and Kazemi, Ehsan and Hubaux, Jean-Pierre},
	journal={IEEE transactions on dependable and secure computing},
	volume={11},
	number={3},
	pages={266--279},
	year={2014},
	publisher={IEEE}
}
@article{19freudiger2013non,
	title={Non-cooperative location privacy},
	author={Freudiger, Julien and Manshaei, Mohammad Hossein and Hubaux, Jean-Pierre and Parkes, David C},
	journal={IEEE Transactions on Dependable and Secure Computing},
	volume={10},
	number={2},
	pages={84--98},
	year={2013},
	publisher={IEEE}
}
@article{20gao2013trpf,
	title={TrPF: A trajectory privacy-preserving framework for participatory sensing},
	author={Gao, Sheng and Ma, Jianfeng and Shi, Weisong and Zhan, Guoxing and Sun, Cong},
	journal={IEEE Transactions on Information Forensics and Security},
	volume={8},
	number={6},
	pages={874--887},
	year={2013},
	publisher={IEEE}
}
@article{21ma2013privacy,
	title={Privacy vulnerability of published anonymous mobility traces},
	author={Ma, Chris YT and Yau, David KY and Yip, Nung Kwan and Rao, Nageswara SV},
	journal={IEEE/ACM Transactions on Networking},
	volume={21},
	number={3},
	pages={720--733},
	year={2013},
	publisher={IEEE}
}
@article{22niu2013pseudo,
	title={Pseudo-Location Updating System for privacy-preserving location-based services},
	author={Niu, Ben and Zhu, Xiaoyan and Chi, Haotian and Li, Hui},
	journal={China Communications},
	volume={10},
	number={9},
	pages={1--12},
	year={2013},
	publisher={IEEE}
}
@article{23dewri2013local,
	title={Local differential perturbations: Location privacy under approximate knowledge attackers},
	author={Dewri, Rinku},
	journal={IEEE Transactions on Mobile Computing},
	volume={12},
	number={12},
	pages={2360--2372},
	year={2013},
	publisher={IEEE}
}
@inproceedings{24kanoria2012tractable,
	title={Tractable bayesian social learning on trees},
	author={Kanoria, Yashodhan and Tamuz, Omer},
	booktitle={Information Theory Proceedings (ISIT), 2012 IEEE International Symposium on},
	pages={2721--2725},
	year={2012},
	organization={IEEE}
}
@inproceedings{25farias2005universal,
	title={A universal scheme for learning},
	author={Farias, Vivek F and Moallemi, Ciamac C and Van Roy, Benjamin and Weissman, Tsachy},
	booktitle={Proceedings. International Symposium on Information Theory, 2005. ISIT 2005.},
	pages={1158--1162},
	year={2005},
	organization={IEEE}
}
@inproceedings{26misra2013unsupervised,
	title={Unsupervised learning and universal communication},
	author={Misra, Vinith and Weissman, Tsachy},
	booktitle={Information Theory Proceedings (ISIT), 2013 IEEE International Symposium on},
	pages={261--265},
	year={2013},
	organization={IEEE}
}
@inproceedings{27ryabko2013time,
	title={Time-series information and learning},
	author={Ryabko, Daniil},
	booktitle={Information Theory Proceedings (ISIT), 2013 IEEE International Symposium on},
	pages={1392--1395},
	year={2013},
	organization={IEEE}
}
@inproceedings{28krzakala2013phase,
	title={Phase diagram and approximate message passing for blind calibration and dictionary learning},
	author={Krzakala, Florent and M{\'e}zard, Marc and Zdeborov{\'a}, Lenka},
	booktitle={Information Theory Proceedings (ISIT), 2013 IEEE International Symposium on},
	pages={659--663},
	year={2013},
	organization={IEEE}
}
@inproceedings{29sakata2013sample,
	title={Sample complexity of Bayesian optimal dictionary learning},
	author={Sakata, Ayaka and Kabashima, Yoshiyuki},
	booktitle={Information Theory Proceedings (ISIT), 2013 IEEE International Symposium on},
	pages={669--673},
	year={2013},
	organization={IEEE}
}
@inproceedings{30predd2004consistency,
	title={Consistency in a model for distributed learning with specialists},
	author={Predd, Joel B and Kulkarni, Sanjeev R and Poor, H Vincent},
	booktitle={IEEE International Symposium on Information Theory},
	year={2004},
	organization={IEEE}
}
@inproceedings{31nokleby2016rate,
	title={Rate-Distortion Bounds on Bayes Risk in Supervised Learning},
	author={Nokleby, Matthew and Beirami, Ahmad and Calderbank, Robert},
	booktitle={2016 IEEE International Symposium on Information Theory (ISIT)},
	pages={2099-2103},
	year={2016},
	organization={IEEE}
}

@inproceedings{32le2016imperfect,
	title={Are imperfect reviews helpful in social learning?},
	author={Le, Tho Ngoc and Subramanian, Vijay G and Berry, Randall A},
	booktitle={Information Theory (ISIT), 2016 IEEE International Symposium on},
	pages={2089--2093},
	year={2016},
	organization={IEEE}
}
@inproceedings{33gadde2016active,
	title={Active Learning for Community Detection in Stochastic Block Models},
	author={Gadde, Akshay and Gad, Eyal En and Avestimehr, Salman and Ortega, Antonio},
	booktitle={2016 IEEE International Symposium on Information Theory (ISIT)},
	pages={1889-1893},
	year={2016}
}
@inproceedings{34shakeri2016minimax,
	title={Minimax Lower Bounds for Kronecker-Structured Dictionary Learning},
	author={Shakeri, Zahra and Bajwa, Waheed U and Sarwate, Anand D},
	booktitle={2016 IEEE International Symposium on Information Theory (ISIT)},
	pages={1148-1152},
	year={2016}
}
@article{35lee2015speeding,
	title={Speeding up distributed machine learning using codes},
	author={Lee, Kangwook and Lam, Maximilian and Pedarsani, Ramtin and Papailiopoulos, Dimitris and Ramchandran, Kannan},
	booktitle={2016 IEEE International Symposium on Information Theory (ISIT)},
	pages={1143-1147},
	year={2016}
}
@article{36oneto2016statistical,
	title={Statistical Learning Theory and ELM for Big Social Data Analysis},
	author={Oneto, Luca and Bisio, Federica and Cambria, Erik and Anguita, Davide},
	journal={ieee CompUTATionAl inTelliGenCe mAGAzine},
	volume={11},
	number={3},
	pages={45--55},
	year={2016},
	publisher={IEEE}
}
@article{37lin2015probabilistic,
	title={Probabilistic approach to modeling and parameter learning of indirect drive robots from incomplete data},
	author={Lin, Chung-Yen and Tomizuka, Masayoshi},
	journal={IEEE/ASME Transactions on Mechatronics},
	volume={20},
	number={3},
	pages={1036--1045},
	year={2015},
	publisher={IEEE}
}
@article{38wang2016towards,
	title={Towards Bayesian Deep Learning: A Framework and Some Existing Methods},
	author={Wang, Hao and Yeung, Dit-Yan},
	journal={IEEE Transactions on Knowledge and Data Engineering},
	volume={PP},
	number={99},
	year={2016},
	publisher={IEEE}
}


%%%%%Informationtheoreticsecurity%%%%%%%%%%%%%%%%%%%%%%%




@inproceedings{Bloch2011PhysicalSecBook,
	title={Physical-Layer Security},
	author={M. Bloch and J. Barros},
	organization={Cambridge University Press},
	year={2011}
}



@inproceedings{Liang2009InfoSecBook,
	title={Information Theoretic Security},
	author={Y. Liang and H. V. Poor and S. Shamai (Shitz)},
	organization={Now Publishers Inc.},
	year={2009}
}


@inproceedings{Zhou2013PhysicalSecBook,
	title={Physical Layer Security in Wireless Communications},
	author={ X. Zhou and L. Song and Y. Zhang},
	organization={CRC Press},
	year={2013}
}

@article{Ni2012IEA,
	Author = {D. Ni and H. Liu and W. Ding and  Y. Xie and H. Wang and H. Pishro-Nik and Q. Yu},
	Journal = {IEA/AIE},
	Title = {Cyber-Physical Integration to Connect Vehicles for Transformed Transportation Safety and Efficiency},
	Year = {2012}}



@inproceedings{Ni2012Inproceedings,
	Author = {D. Ni, H. Liu, Y. Xie, W. Ding, H. Wang, H. Pishro-Nik, Q. Yu and M. Ferreira},
	Booktitle = {Spring Simulation Multiconference},
	Date-Added = {2016-09-04 14:18:42 +0000},
	Date-Modified = {2016-09-06 16:22:14 +0000},
	Title = {Virtual Lab of Connected Vehicle Technology},
	Year = {2012}}

@inproceedings{Ni2012Inproceedings,
	Author = {D. Ni, H. Liu, W. Ding, Y. Xie, H. Wang, H. Pishro-Nik and Q. Yu,},
	Booktitle = {IEA/AIE},
	Date-Added = {2016-09-04 09:11:02 +0000},
	Date-Modified = {2016-09-06 14:46:53 +0000},
	Title = {Cyber-Physical Integration to Connect Vehicles for Transformed Transportation Safety and Efficiency},
	Year = {2012}}


@article{Nekoui_IJIPT_2009,
	Author = {M. Nekoui and D. Ni and H. Pishro-Nik and R. Prasad and M. Kanjee and H. Zhu and T. Nguyen},
	Journal = {International Journal of Internet Protocol Technology (IJIPT)},
	Number = {3},
	Pages = {},
	Publisher = {},
	Title = {Development of a VII-Enabled Prototype Intersection Collision Warning System},
	Volume = {4},
	Year = {2009}}


@inproceedings{Pishro_Ganz_Ni,
	Author = {H. Pishro-Nik, A. Ganz, and Daiheng Ni},
	Booktitle = {Forty-Fifth Annual Allerton Conference on Communication, Control, and Computing. Allerton House, Monticello, IL},
	Date-Added = {},
	Date-Modified = {},
	Number = {},
	Pages = {},
	Title = {The capacity of vehicular ad hoc networks},
	Volume = {},
	Year = {September 26-28, 2007}}

@inproceedings{Leow_Pishro_Ni_1,
	Author = {W. L. Leow, H. Pishro-Nik and Daiheng Ni},
	Booktitle = {IEEE Global Telecommunications Conference, Washington, D.C.},
	Date-Added = {},
	Date-Modified = {},
	Number = {},
	Pages = {},
	Title = {Delay and Energy Tradeoff in Multi-state Wireless Sensor Networks},
	Volume = {},
	Year = {November 26-30, 2007}}


@misc{UMass-Trans,
	title = {{UMass Transportation Center}},
	note = {\url{http://www.umasstransportationcenter.org/}},
}


@inproceedings{Haenggi2013book,
	title={Stochastic geometry for wireless networks},
	author={M. Haenggi},
	organization={Cambridge Uinversity Press},
	year={2013}
}


%%%%%%%%%%%%%%%%personalization%%%%%%%%%%%%%%%%%%%%%%%%%%%%%%%%%%

@article{osma2015,
	title={Impact of Time-to-Collision Information on Driving Behavior in Connected Vehicle Environments Using A Driving Simulator Test Bed},
	journal{Journal of Traffic and Logistics Engineering},
	author={Osama A. Osman, Julius Codjoe, and Sherif Ishak},
	volume={3},
	number={1},
	pages={18--24},
	year={2015}
}


@article{charisma2010,
	title={Dynamic Latent Plan Models},
	author={Charisma F. Choudhurya, Moshe Ben-Akivab and Maya Abou-Zeid},
	journal={Journal of Choice Modelling},
	volume={3},
	number={2},
	pages={50--70},
	year={2010},
	publisher={Elsvier}
}


@misc{noble2014,
	author = {A. M. Noble, Shane B. McLaughlin, Zachary R. Doerzaph and Thomas A. Dingus},
	title = {Crowd-sourced Connected-vehicle Warning Algorithm using Naturalistic Driving Data},
	howpublished = {Downloaded from \url{http://hdl.handle.net/10919/53978}},
	
	month = August,
	year = 2014
}


@phdthesis{charisma2007,
	title    = {Modeling Driving Decisions with Latent Plans},
	school   = {Massachusetts Institute of Technology },
	author   = {Charisma Farheen Choudhury},
	year     = {2007}, %other attributes omitted
}


@article{chrysler2015,
	title={Cost of Warning of Unseen Threats:Unintended Consequences of Connected Vehicle Alerts},
	author={S. T. Chrysler, J. M. Cooper and D. C. Marshall},
	journal={Transportation Research Record: Journal of the Transportation Research Board},
	volume={2518},
	pages={79--85},
	year={2015},
}

@misc{nsf_cps,
	title = {Cyber-Physical Systems (CPS) PROGRAM SOLICITATION NSF 17-529},
	howpublished = {Downloaded from \url{https://www.nsf.gov/publications/pub_summ.jsp?WT.z_pims_id=503286&ods_key=nsf17529}},
}



%%%%%%%%%%%%%%IOT%%%%%%%%%%%%%%%%%%%%%%%%%%%%%%%%%%%%%%%%%%%%%%%%%%%




@article{FTC2015,
	title={Internet of Things: Privacy and Security in a Connected World},
	author={FTC Staff Report},
	year={2015}
}



%% Saved with string encoding Unicode (UTF-8)
@inproceedings{1zhou2014security,
	title={Security/privacy of wearable fitness tracking {I}o{T} devices},
	author={Zhou, Wei and Piramuthu, Selwyn},
	booktitle={Information Systems and Technologies (CISTI), 2014 9th Iberian Conference on},
	pages={1--5},
	year={2014},
	organization={IEEE}
}

@article{2nia2016comprehensive,
	title={A comprehensive study of security of internet-of-things},
	author={Mohsenia, Arsalan and Jha, Niraj K},
	journal={IEEE Transactions on Emerging Topics in Computing},
	volum={5},
	number={4},
	pages={586--602},
	year={2017},
	publisher={IEEE}
}

@inproceedings{3ukil2014iot,
	title={{I}o{T}-privacy: To be private or not to be private},
	author={Ukil, Arijit and Bandyopadhyay, Soma and Pal, Arpan},
	booktitle={Computer Communications Workshops (INFOCOM WKSHPS), IEEE Conference on},
	pages={123--124},
	year={2014},
	organization={IEEE}
}

@article{4arias2015privacy,
	title={Privacy and security in internet of things and wearable devices},
	author={Arias, Orlando and Wurm, Jacob and Hoang, Khoa and Jin, Yier},
	journal={IEEE Transactions on Multi-Scale Computing Systems},
	volume={1},
	number={2},
	pages={99--109},
	year={2015},
	publisher={IEEE}
}
@inproceedings{5ullah2016novel,
	title={A novel model for preserving Location Privacy in Internet of Things},
	author={Ullah, Ikram and Shah, Munam Ali},
	booktitle={Automation and Computing (ICAC), 2016 22nd International Conference on},
	pages={542--547},
	year={2016},
	organization={IEEE}
}
@inproceedings{6sathishkumar2016enhanced,
	title={Enhanced location privacy algorithm for wireless sensor network in Internet of Things},
	author={Sathishkumar, J and Patel, Dhiren R},
	booktitle={Internet of Things and Applications (IOTA), International Conference on},
	pages={208--212},
	year={2016},
	organization={IEEE}
}
@inproceedings{7zhou2012preserving,
	title={Preserving sensor location privacy in internet of things},
	author={Zhou, Liming and Wen, Qiaoyan and Zhang, Hua},
	booktitle={Computational and Information Sciences (ICCIS), 2012 Fourth International Conference on},
	pages={856--859},
	year={2012},
	organization={IEEE}
}

@inproceedings{8ukil2015privacy,
	title={Privacy for {I}o{T}: Involuntary privacy enablement for smart energy systems},
	author={Ukil, Arijit and Bandyopadhyay, Soma and Pal, Arpan},
	booktitle={Communications (ICC), 2015 IEEE International Conference on},
	pages={536--541},
	year={2015},
	organization={IEEE}
}

@inproceedings{9dalipi2016security,
	title={Security and Privacy Considerations for {I}o{T} Application on Smart Grids: Survey and Research Challenges},
	author={Dalipi, Fisnik and Yayilgan, Sule Yildirim},
	booktitle={Future Internet of Things and Cloud Workshops (FiCloudW), IEEE International Conference on},
	pages={63--68},
	year={2016},
	organization={IEEE}
}
@inproceedings{10harris2016security,
	title={Security and Privacy in Public {I}o{T} Spaces},
	author={Harris, Albert F and Sundaram, Hari and Kravets, Robin},
	booktitle={Computer Communication and Networks (ICCCN), 2016 25th International Conference on},
	pages={1--8},
	year={2016},
	organization={IEEE}
}

@inproceedings{11al2015security,
	title={Security and privacy framework for ubiquitous healthcare {I}o{T} devices},
	author={Al Alkeem, Ebrahim and Yeun, Chan Yeob and Zemerly, M Jamal},
	booktitle={Internet Technology and Secured Transactions (ICITST), 2015 10th International Conference for},
	pages={70--75},
	year={2015},
	organization={IEEE}
}
@inproceedings{12sivaraman2015network,
	title={Network-level security and privacy control for smart-home {I}o{T} devices},
	author={Sivaraman, Vijay and Gharakheili, Hassan Habibi and Vishwanath, Arun and Boreli, Roksana and Mehani, Olivier},
	booktitle={Wireless and Mobile Computing, Networking and Communications (WiMob), 2015 IEEE 11th International Conference on},
	pages={163--167},
	year={2015},
	organization={IEEE}
}

@inproceedings{13srinivasan2016privacy,
	title={Privacy conscious architecture for improving emergency response in smart cities},
	author={Srinivasan, Ramya and Mohan, Apurva and Srinivasan, Priyanka},
	booktitle={Smart City Security and Privacy Workshop (SCSP-W), 2016},
	pages={1--5},
	year={2016},
	organization={IEEE}
}
@inproceedings{14sadeghi2015security,
	title={Security and privacy challenges in industrial internet of things},
	author={Sadeghi, Ahmad-Reza and Wachsmann, Christian and Waidner, Michael},
	booktitle={Design Automation Conference (DAC), 2015 52nd ACM/EDAC/IEEE},
	pages={1--6},
	year={2015},
	organization={IEEE}
}
@inproceedings{15otgonbayar2016toward,
	title={Toward Anonymizing {I}o{T} Data Streams via Partitioning},
	author={Otgonbayar, Ankhbayar and Pervez, Zeeshan and Dahal, Keshav},
	booktitle={Mobile Ad Hoc and Sensor Systems (MASS), 2016 IEEE 13th International Conference on},
	pages={331--336},
	year={2016},
	organization={IEEE}
}
@inproceedings{16rutledge2016privacy,
	title={Privacy Impacts of {I}o{T} Devices: A SmartTV Case Study},
	author={Rutledge, Richard L and Massey, Aaron K and Ant{\'o}n, Annie I},
	booktitle={Requirements Engineering Conference Workshops (REW), IEEE International},
	pages={261--270},
	year={2016},
	organization={IEEE}
}

@inproceedings{17andrea2015internet,
	title={Internet of Things: Security vulnerabilities and challenges},
	author={Andrea, Ioannis and Chrysostomou, Chrysostomos and Hadjichristofi, George},
	booktitle={Computers and Communication (ISCC), 2015 IEEE Symposium on},
	pages={180--187},
	year={2015},
	organization={IEEE}
}






























%%%%%%%%%%%%%%%%%%%%%%%%%%%%%%%%%%%%%%%%%%%%%%%%%%%%%%%%%%%


@misc{epfl-mobility-20090224,
	author = {Michal Piorkowski and Natasa Sarafijanovic-Djukic and Matthias Grossglauser},
	title = {{CRAWDAD} dataset epfl/mobility (v. 2009-02-24)},
	howpublished = {Downloaded from \url{http://crawdad.org/epfl/mobility/20090224}},
	doi = {10.15783/C7J010},
	month = feb,
	year = 2009
}

@misc{roma-taxi-20140717,
	author = {Lorenzo Bracciale and Marco Bonola and Pierpaolo Loreti and Giuseppe Bianchi and Raul Amici and Antonello Rabuffi},
	title = {{CRAWDAD} dataset roma/taxi (v. 2014-07-17)},
	howpublished = {Downloaded from \url{http://crawdad.org/roma/taxi/20140717}},
	doi = {10.15783/C7QC7M},
	month = jul,
	year = 2014
}

@misc{rice-ad_hoc_city-20030911,
	author = {Jorjeta G. Jetcheva and Yih-Chun Hu and Santashil PalChaudhuri and Amit Kumar Saha and David B. Johnson},
	title = {{CRAWDAD} dataset rice/ad\_hoc\_city (v. 2003-09-11)},
	howpublished = {Downloaded from \url{http://crawdad.org/rice/ad_hoc_city/20030911}},
	doi = {10.15783/C73K5B},
	month = sep,
	year = 2003
}

@misc{china:2012,
	author = {Microsoft Research Asia},
	title = {GeoLife GPS Trajectories},
	year = {2012},
	howpublished= {\url{https://www.microsoft.com/en-us/download/details.aspx?id=52367}},
}


@misc{china:2011,
	ALTauthor = {Microsoft Research Asia)},
	ALTeditor = {},
	title = {GeoLife GPS Trajectories,
	year = {2012},
	url = {https://www.microsoft.com/en-us/download/details.aspx?id=52367},
	}
	
	
	@misc{longversion,
	author = {N. Takbiri and A. Houmansadr and D.L. Goeckel and H. Pishro-Nik},
	title = {{Limits of Location Privacy under Anonymization and Obfuscation}},
	howpublished = "\url{http://www.ecs.umass.edu/ece/pishro/Papers/ISIT_2017-2.pdf}",
	year = 2017,
	month= "January",
	note = "Summarized version submitted to IEEE ISIT 2017"
	}
	
	@misc{isit_ke,
	author = {K. Li and D. Goeckel and H. Pishro-Nik},
	title = {{Bayesian Time Series Matching and Privacy}},
	note = "submitted to IEEE ISIT 2017"
	}
	
	@article{matching,
	title={Asymptotically Optimal Matching of Multiple Sequences to Source Distributions and Training Sequences},
	author={Jayakrishnan Unnikrishnan},
	journal={ IEEE Transactions on Information Theory},
	volume={61},
	number={1},
	pages={452-468},
	year={2015},
	publisher={IEEE}
	}
	
	
	@article{Naini2016,
	Author = {F. Naini and J. Unnikrishnan and P. Thiran and M. Vetterli},
	Journal = {IEEE Transactions on Information Forensics and Security},
	Publisher = {IEEE},
	Title = {Where You Are Is Who You Are: User Identification by Matching Statistics},
	volume={11},
	number={2},
	pages={358--372},
	Year = {2016}
	}
	
	
	
	@inproceedings{holowczak2015cachebrowser,
	title={{CacheBrowser: Bypassing Chinese Censorship without Proxies Using Cached Content}},
	author={Holowczak, John and Houmansadr, Amir},
	booktitle={Proceedings of the 22nd ACM SIGSAC Conference on Computer and Communications Security},
	pages={70--83},
	year={2015},
	organization={ACM}
	}
	@misc{cb-website,
	Howpublished = {\url{https://cachebrowser.net/#/}},
	Title = {{CacheBrowser}},
	key={cachebrowser}
	}
	
	@inproceedings{GameOfDecoys,
	title={{GAME OF DECOYS: Optimal Decoy Routing Through Game Theory}},
	author={Milad Nasr and Amir Houmansadr},
	booktitle={The $23^{rd}$ ACM Conference on Computer and Communications Security (CCS)},
	year={2016}
	}
	
	@inproceedings{CDNReaper,
	title={{Practical Censorship Evasion Leveraging Content Delivery Networks}},
	author={Hadi Zolfaghari and Amir Houmansadr},
	booktitle={The $23^{rd}$ ACM Conference on Computer and Communications Security (CCS)},
	year={2016}
	}
	
	@misc{Leberknight2010,
	Author = {Leberknight, C. and Chiang, M. and Poor, H. and Wong, F.},
	Howpublished = {\url{http://www.princeton.edu/~chiangm/anticensorship.pdf}},
	Title = {{A Taxonomy of Internet Censorship and Anti-censorship}},
	Year = {2010}}
	
	@techreport{ultrasurf-analysis,
	Author = {Appelbaum, Jacob},
	Institution = {The Tor Project},
	Title = {{Technical analysis of the Ultrasurf proxying software}},
	Url = {http://scholar.google.com/scholar?hl=en\&btnG=Search\&q=intitle:Technical+analysis+of+the+Ultrasurf+proxying+software\#0},
	Year = {2012},
	Bdsk-Url-1 = {http://scholar.google.com/scholar?hl=en%5C&btnG=Search%5C&q=intitle:Technical+analysis+of+the+Ultrasurf+proxying+software%5C#0}}
	
	@misc{gifc:07,
	Howpublished = {\url{http://www.internetfreedom.org/archive/Defeat\_Internet\_Censorship\_White\_Paper.pdf}},
	Key = {defeatcensorship},
	Publisher = {Global Internet Freedom Consortium (GIFC)},
	Title = {{Defeat Internet Censorship: Overview of Advanced Technologies and Products}},
	Type = {White Paper},
	Year = {2007}}
	
	@article{pan2011survey,
	Author = {Pan, J. and Paul, S. and Jain, R.},
	Journal = {Communications Magazine, IEEE},
	Number = {7},
	Pages = {26--36},
	Publisher = {IEEE},
	Title = {{A Survey of the Research on Future Internet Architectures}},
	Volume = {49},
	Year = {2011}}
	
	@misc{nsf-fia,
	Howpublished = {\url{http://www.nets-fia.net/}},
	Key = {FIA},
	Title = {{NSF Future Internet Architecture Project}}}
	
	@misc{NDN,
	Howpublished = {\url{http://www.named- data.net}},
	Key = {NDN},
	Title = {{Named Data Networking Project}}}
	
	@inproceedings{MobilityFirst,
	Author = {Seskar, I. and Nagaraja, K. and Nelson, S. and Raychaudhuri, D.},
	Booktitle = {Asian Internet Engineering Conference},
	Title = {{Mobilityfirst Future internet Architecture Project}},
	Year = {2011}}
	
	@incollection{NEBULA,
	Author = {Anderson, T. and Birman, K. and Broberg, R. and Caesar, M. and Comer, D. and Cotton, C. and Freedman, M.~J. and Haeberlen, A. and Ives, Z.~G. and Krishnamurthy, A. and others},
	Booktitle = {The Future Internet},
	Pages = {16--26},
	Publisher = {Springer},
	Title = {{The NEBULA Future Internet Architecture}},
	Year = {2013}}
	
	@inproceedings{XIA,
	Author = {Anand, A. and Dogar, F. and Han, D. and Li, B. and Lim, H. and Machado, M. and Wu, W. and Akella, A. and Andersen, D.~G. and Byers, J.~W. and others},
	Booktitle = {ACM Workshop on Hot Topics in Networks},
	Pages = {2},
	Title = {{XIA: An Architecture for an Evolvable and Trustworthy Internet}},
	Year = {2011}}
	
	@inproceedings{ChoiceNet,
	Author = {Rouskas, G.~N. and Baldine, I. and Calvert, K.~L. and Dutta, R. and Griffioen, J. and Nagurney, A. and Wolf, T.},
	Booktitle = {ONDM},
	Title = {{ChoiceNet: Network Innovation Through Choice}},
	Year = {2013}}
	
	@misc{nsf-find,
	Howpublished = {http://www.nets-find.net/},
	Title = {{NSF NeTS FIND Initiative}}}
	
	@article{traid,
	Author = {Cheriton, D.~R. and Gritter, M.},
	Title = {{TRIAD: A New Next-Generation Internet Architecture}},
	Year = {2000}}
	
	@inproceedings{dona,
	Author = {Koponen, T. and Chawla, M. and Chun, B-G. and Ermolinskiy, A. and Kim, K.~H. and Shenker, S. and Stoica, I.},
	Booktitle = {ACM SIGCOMM Computer Communication Review},
	Number = {4},
	Organization = {ACM},
	Pages = {181--192},
	Title = {{A Data-Oriented (and Beyond) Network Architecture}},
	Volume = {37},
	Year = {2007}}
	
	@misc{ultrasurf,
	Howpublished = {\url{http://www.ultrareach.com}},
	Key = {ultrasurf},
	Title = {{Ultrasurf}}}
	
	@misc{tor-bridge,
	Author = {Dingledine, R. and Mathewson, N.},
	Howpublished = {\url{https://svn.torproject.org/svn/projects/design-paper/blocking.html}},
	Title = {{Design of a Blocking-Resistant Anonymity System}}}
	
	@inproceedings{McLachlanH09,
	Author = {J. McLachlan and N. Hopper},
	Booktitle = {WPES},
	Title = {{On the Risks of Serving Whenever You Surf: Vulnerabilities in Tor's Blocking Resistance Design}},
	Year = {2009}}
	
	@inproceedings{mahdian2010,
	Author = {Mahdian, M.},
	Booktitle = {{Fun with Algorithms}},
	Title = {{Fighting Censorship with Algorithms}},
	Year = {2010}}
	
	@inproceedings{McCoy2011,
	Author = {McCoy, D. and Morales, J.~A. and Levchenko, K.},
	Booktitle = {FC},
	Title = {{Proximax: A Measurement Based System for Proxies Dissemination}},
	Year = {2011}}
	
	@inproceedings{Sovran2008,
	Author = {Sovran, Y. and Libonati, A. and Li, J.},
	Booktitle = {IPTPS},
	Title = {{Pass it on: Social Networks Stymie Censors}},
	Year = {2008}}
	
	@inproceedings{rbridge,
	Author = {Wang, Q. and Lin, Zi and Borisov, N. and Hopper, N.},
	Booktitle = {{NDSS}},
	Title = {{rBridge: User Reputation based Tor Bridge Distribution with Privacy Preservation}},
	Year = {2013}}
	
	@inproceedings{telex,
	Author = {Wustrow, E. and Wolchok, S. and Goldberg, I. and Halderman, J.},
	Booktitle = {{USENIX Security}},
	Title = {{Telex: Anticensorship in the Network Infrastructure}},
	Year = {2011}}
	
	@inproceedings{cirripede,
	Author = {Houmansadr, A. and Nguyen, G. and Caesar, M. and Borisov, N.},
	Booktitle = {CCS},
	Title = {{Cirripede: Circumvention Infrastructure Using Router Redirection with Plausible Deniability}},
	Year = {2011}}
	
	@inproceedings{decoyrouting,
	Author = {Karlin, J. and Ellard, D. and Jackson, A. and Jones, C. and Lauer, G. and Mankins, D. and Strayer, W.},
	Booktitle = {{FOCI}},
	Title = {{Decoy Routing: Toward Unblockable Internet Communication}},
	Year = {2011}}
	
	@inproceedings{routing-around-decoys,
	Author = {M.~Schuchard and J.~Geddes and C.~Thompson and N.~Hopper},
	Booktitle = {{CCS}},
	Title = {{Routing Around Decoys}},
	Year = {2012}}
	
	@inproceedings{parrot,
	Author = {A. Houmansadr and C. Brubaker and V. Shmatikov},
	Booktitle = {IEEE S\&P},
	Title = {{The Parrot is Dead: Observing Unobservable Network Communications}},
	Year = {2013}}
	
	@misc{knock,
	Author = {T. Wilde},
	Howpublished = {\url{https://blog.torproject.org/blog/knock-knock-knockin-bridges-doors}},
	Title = {{Knock Knock Knockin' on Bridges' Doors}},
	Year = {2012}}
	
	@inproceedings{china-tor,
	Author = {Winter, P. and Lindskog, S.},
	Booktitle = {{FOCI}},
	Title = {{How the Great Firewall of China Is Blocking Tor}},
	Year = {2012}}
	
	@misc{discover-bridge,
	Howpublished = {\url{https://blog.torproject.org/blog/research-problems-ten-ways-discover-tor-bridges}},
	Key = {tenways},
	Title = {{Ten Ways to Discover Tor Bridges}}}
	
	@inproceedings{freewave,
	Author = {A.~Houmansadr and T.~Riedl and N.~Borisov and A.~Singer},
	Booktitle = {{NDSS}},
	Title = {{I Want My Voice to Be Heard: IP over Voice-over-IP for Unobservable Censorship Circumvention}},
	Year = 2013}
	
	@inproceedings{censorspoofer,
	Author = {Q. Wang and X. Gong and G. Nguyen and A. Houmansadr and N. Borisov},
	Booktitle = {CCS},
	Title = {{CensorSpoofer: Asymmetric Communication Using IP Spoofing for Censorship-Resistant Web Browsing}},
	Year = {2012}}
	
	@inproceedings{skypemorph,
	Author = {H. Moghaddam and B. Li and M. Derakhshani and I. Goldberg},
	Booktitle = {CCS},
	Title = {{SkypeMorph: Protocol Obfuscation for Tor Bridges}},
	Year = {2012}}
	
	@inproceedings{stegotorus,
	Author = {Weinberg, Z. and Wang, J. and Yegneswaran, V. and Briesemeister, L. and Cheung, S. and Wang, F. and Boneh, D.},
	Booktitle = {CCS},
	Title = {{StegoTorus: A Camouflage Proxy for the Tor Anonymity System}},
	Year = {2012}}
	
	@techreport{dust,
	Author = {{B.~Wiley}},
	Howpublished = {\url{http://blanu.net/ Dust.pdf}},
	Institution = {School of Information, University of Texas at Austin},
	Title = {{Dust: A Blocking-Resistant Internet Transport Protocol}},
	Year = {2011}}
	
	@inproceedings{FTE,
	Author = {K.~Dyer and S.~Coull and T.~Ristenpart and T.~Shrimpton},
	Booktitle = {CCS},
	Title = {{Protocol Misidentification Made Easy with Format-Transforming Encryption}},
	Year = {2013}}
	
	@inproceedings{fp,
	Author = {Fifield, D. and Hardison, N. and Ellithrope, J. and Stark, E. and Dingledine, R. and Boneh, D. and Porras, P.},
	Booktitle = {PETS},
	Title = {{Evading Censorship with Browser-Based Proxies}},
	Year = {2012}}
	
	@misc{obfsproxy,
	Howpublished = {\url{https://www.torproject.org/projects/obfsproxy.html.en}},
	Key = {obfsproxy},
	Publisher = {The Tor Project},
	Title = {{A Simple Obfuscating Proxy}}}
	
	@inproceedings{Tor-instead-of-IP,
	Author = {Liu, V. and Han, S. and Krishnamurthy, A. and Anderson, T.},
	Booktitle = {HotNets},
	Title = {{Tor instead of IP}},
	Year = {2011}}
	
	@misc{roger-slides,
	Howpublished = {\url{https://svn.torproject.org/svn/projects/presentations/slides-28c3.pdf}},
	Key = {torblocking},
	Title = {{How Governments Have Tried to Block Tor}}}
	
	@inproceedings{infranet,
	Author = {Feamster, N. and Balazinska, M. and Harfst, G. and Balakrishnan, H. and Karger, D.},
	Booktitle = {USENIX Security},
	Title = {{Infranet: Circumventing Web Censorship and Surveillance}},
	Year = {2002}}
	
	@inproceedings{collage,
	Author = {S.~Burnett and N.~Feamster and S.~Vempala},
	Booktitle = {USENIX Security},
	Title = {{Chipping Away at Censorship Firewalls with User-Generated Content}},
	Year = {2010}}
	
	@article{anonymizer,
	Author = {Boyan, J.},
	Journal = {Computer-Mediated Communication Magazine},
	Month = sep,
	Number = {9},
	Title = {{The Anonymizer: Protecting User Privacy on the Web}},
	Volume = {4},
	Year = {1997}}
	
	@article{schulze2009internet,
	Author = {Schulze, H. and Mochalski, K.},
	Journal = {IPOQUE Report},
	Pages = {351--362},
	Title = {Internet Study 2008/2009},
	Volume = {37},
	Year = {2009}}
	
	@inproceedings{cya-ccs13,
	Author = {J.~Geddes and M.~Schuchard and N.~Hopper},
	Booktitle = {{CCS}},
	Title = {{Cover Your ACKs: Pitfalls of Covert Channel Censorship Circumvention}},
	Year = {2013}}
	
	@inproceedings{andana,
	Author = {DiBenedetto, S. and Gasti, P. and Tsudik, G. and Uzun, E.},
	Booktitle = {{NDSS}},
	Title = {{ANDaNA: Anonymous Named Data Networking Application}},
	Year = {2012}}
	
	@inproceedings{darkly,
	Author = {Jana, S. and Narayanan, A. and Shmatikov, V.},
	Booktitle = {IEEE S\&P},
	Title = {{A Scanner Darkly: Protecting User Privacy From Perceptual Applications}},
	Year = {2013}}
	
	@inproceedings{NS08,
	Author = {A.~Narayanan and V.~Shmatikov},
	Booktitle = {IEEE S\&P},
	Title = {Robust de-anonymization of large sparse datasets},
	Year = {2008}}
	
	@inproceedings{NS09,
	Author = {Arvind Narayanan and Vitaly Shmatikov},
	Booktitle = {IEEE S\&P},
	Title = {De-anonymizing Social Networks},
	Year = {2009}}
	
	@inproceedings{memento,
	Author = {Jana, S. and Shmatikov, V.},
	Booktitle = {IEEE S\&P},
	Title = {{Memento: Learning secrets from process footprints}},
	Year = {2012}}
	
	@misc{plugtor,
	Howpublished = {\url{https://www.torproject.org/docs/pluggable-transports.html.en}},
	Key = {PluggableTransports},
	Publisher = {The Tor Project},
	Title = {{Tor: Pluggable transports}}}
	
	@misc{psiphon,
	Author = {J.~Jia and P.~Smith},
	Howpublished = {\url{http://www.cdf.toronto.edu/~csc494h/reports/2004-fall/psiphon_ae.html}},
	Title = {{Psiphon: Analysis and Estimation}},
	Year = 2004}
	
	@misc{china-github,
	Howpublished = {\url{http://mobile.informationweek.com/80269/show/72e30386728f45f56b343ddfd0fdb119/}},
	Key = {github},
	Title = {{China's GitHub Censorship Dilemma}}}
	
	@inproceedings{txbox,
	Author = {Jana, S. and Porter, D. and Shmatikov, V.},
	Booktitle = {IEEE S\&P},
	Title = {{TxBox: Building Secure, Efficient Sandboxes with System Transactions}},
	Year = {2011}}
	
	@inproceedings{airavat,
	Author = {I. Roy and S. Setty and A. Kilzer and V. Shmatikov and E. Witchel},
	Booktitle = {NSDI},
	Title = {{Airavat: Security and Privacy for MapReduce}},
	Year = {2010}}
	
	@inproceedings{osdi12,
	Author = {A. Dunn and M. Lee and S. Jana and S. Kim and M. Silberstein and Y. Xu and V. Shmatikov and E. Witchel},
	Booktitle = {OSDI},
	Title = {{Eternal Sunshine of the Spotless Machine: Protecting Privacy with Ephemeral Channels}},
	Year = {2012}}
	
	@inproceedings{ymal,
	Author = {J. Calandrino and A. Kilzer and A. Narayanan and E. Felten and V. Shmatikov},
	Booktitle = {IEEE S\&P},
	Title = {{``You Might Also Like:'' Privacy Risks of Collaborative Filtering}},
	Year = {2011}}
	
	@inproceedings{srivastava11,
	Author = {V. Srivastava and M. Bond and K. McKinley and V. Shmatikov},
	Booktitle = {PLDI},
	Title = {{A Security Policy Oracle: Detecting Security Holes Using Multiple API Implementations}},
	Year = {2011}}
	
	@inproceedings{chen-oakland10,
	Author = {Chen, S. and Wang, R. and Wang, X. and Zhang, K.},
	Booktitle = {IEEE S\&P},
	Title = {{Side-Channel Leaks in Web Applications: A Reality Today, a Challenge Tomorrow}},
	Year = {2010}}
	
	@book{kerck,
	Author = {Kerckhoffs, A.},
	Publisher = {University Microfilms},
	Title = {{La cryptographie militaire}},
	Year = {1978}}
	
	@inproceedings{foci11,
	Author = {J. Karlin and D. Ellard and A.~Jackson and C.~ Jones and G. Lauer and D. Mankins and W.~T.~Strayer},
	Booktitle = {FOCI},
	Title = {{Decoy Routing: Toward Unblockable Internet Communication}},
	Year = 2011}
	
	@inproceedings{sun02,
	Author = {Sun, Q. and Simon, D.~R. and Wang, Y. and Russell, W. and Padmanabhan, V. and Qiu, L.},
	Booktitle = {IEEE S\&P},
	Title = {{Statistical Identification of Encrypted Web Browsing Traffic}},
	Year = {2002}}
	
	@inproceedings{danezis,
	Author = {Murdoch, S.~J. and Danezis, G.},
	Booktitle = {IEEE S\&P},
	Title = {{Low-Cost Traffic Analysis of Tor}},
	Year = {2005}}
	
	@inproceedings{pakicensorship,
	Author = {Z.~Nabi},
	Booktitle = {FOCI},
	Title = {The Anatomy of {Web} Censorship in {Pakistan}},
	Year = {2013}}
	
	@inproceedings{irancensorship,
	Author = {S.~Aryan and H.~Aryan and A.~Halderman},
	Booktitle = {FOCI},
	Title = {Internet Censorship in {Iran}: {A} First Look},
	Year = {2013}}
	
	@inproceedings{ford10efficient,
	Author = {Amittai Aviram and Shu-Chun Weng and Sen Hu and Bryan Ford},
	Booktitle = {\bibconf[9th]{OSDI}{USENIX Symposium on Operating Systems Design and Implementation}},
	Location = {Vancouver, BC, Canada},
	Month = oct,
	Title = {Efficient System-Enforced Deterministic Parallelism},
	Year = 2010}
	
	@inproceedings{ford10determinating,
	Author = {Amittai Aviram and Sen Hu and Bryan Ford and Ramakrishna Gummadi},
	Booktitle = {\bibconf{CCSW}{ACM Cloud Computing Security Workshop}},
	Location = {Chicago, IL},
	Month = oct,
	Title = {Determinating Timing Channels in Compute Clouds},
	Year = 2010}
	
	@inproceedings{ford12plugging,
	Author = {Bryan Ford},
	Booktitle = {\bibconf[4th]{HotCloud}{USENIX Workshop on Hot Topics in Cloud Computing}},
	Location = {Boston, MA},
	Month = jun,
	Title = {Plugging Side-Channel Leaks with Timing Information Flow Control},
	Year = 2012}
	
	@inproceedings{ford12icebergs,
	Author = {Bryan Ford},
	Booktitle = {\bibconf[4th]{HotCloud}{USENIX Workshop on Hot Topics in Cloud Computing}},
	Location = {Boston, MA},
	Month = jun,
	Title = {Icebergs in the Clouds: the {\em Other} Risks of Cloud Computing},
	Year = 2012}
	
	@misc{mullenize,
	Author = {Washington Post},
	Howpublished = {\url{http://apps.washingtonpost.com/g/page/world/gchq-report-on-mullenize-program-to-stain-anonymous-electronic-traffic/502/}},
	Month = {oct},
	Title = {{GCHQ} report on {`MULLENIZE'} program to `stain' anonymous electronic traffic},
	Year = {2013}}
	
	@inproceedings{shue13street,
	Author = {Craig A. Shue and Nathanael Paul and Curtis R. Taylor},
	Booktitle = {\bibbrev[7th]{WOOT}{USENIX Workshop on Offensive Technologies}},
	Month = aug,
	Title = {From an {IP} Address to a Street Address: Using Wireless Signals to Locate a Target},
	Year = 2013}
	
	@inproceedings{knockel11three,
	Author = {Jeffrey Knockel and Jedidiah R. Crandall and Jared Saia},
	Booktitle = {\bibbrev{FOCI}{USENIX Workshop on Free and Open Communications on the Internet}},
	Location = {San Francisco, CA},
	Month = aug,
	Year = 2011}
	
	@misc{rfc4960,
	Author = {R. {Stewart, ed.}},
	Month = sep,
	Note = {RFC 4960},
	Title = {Stream Control Transmission Protocol},
	Year = 2007}
	
	@inproceedings{ford07structured,
	Author = {Bryan Ford},
	Booktitle = {\bibbrev{SIGCOMM}{ACM SIGCOMM}},
	Location = {Kyoto, Japan},
	Month = aug,
	Title = {Structured Streams: a New Transport Abstraction},
	Year = {2007}}
	
	@misc{spdy,
	Author = {Google, Inc.},
	Note = {\url{http://www.chromium.org/spdy/spdy-whitepaper}},
	Title = {{SPDY}: An Experimental Protocol For a Faster {Web}}}
	
	@misc{quic,
	Author = {Jim Roskind},
	Month = jun,
	Note = {\url{http://blog.chromium.org/2013/06/experimenting-with-quic.html}},
	Title = {Experimenting with {QUIC}},
	Year = 2013}
	
	@misc{podjarny12not,
	Author = {G.~Podjarny},
	Month = jun,
	Note = {\url{http://www.guypo.com/technical/not-as-spdy-as-you-thought/}},
	Title = {{Not as SPDY as You Thought}},
	Year = 2012}
	
	@inproceedings{cor,
	Author = {Jones, N.~A. and Arye, M. and Cesareo, J. and Freedman, M.~J.},
	Booktitle = {FOCI},
	Title = {{Hiding Amongst the Clouds: A Proposal for Cloud-based Onion Routing}},
	Year = {2011}}
	
	@misc{torcloud,
	Howpublished = {\url{https://cloud.torproject.org/}},
	Key = {tor cloud},
	Title = {{The Tor Cloud Project}}}
	
	@inproceedings{scramblesuit,
	Author = {Philipp Winter and Tobias Pulls and Juergen Fuss},
	Booktitle = {WPES},
	Title = {{ScrambleSuit: A Polymorphic Network Protocol to Circumvent Censorship}},
	Year = 2013}
	
	@article{savage2000practical,
	Author = {Savage, S. and Wetherall, D. and Karlin, A. and Anderson, T.},
	Journal = {ACM SIGCOMM Computer Communication Review},
	Number = {4},
	Pages = {295--306},
	Publisher = {ACM},
	Title = {Practical network support for IP traceback},
	Volume = {30},
	Year = {2000}}
	
	@inproceedings{ooni,
	Author = {Filast, A. and Appelbaum, J.},
	Booktitle = {{FOCI}},
	Title = {{OONI : Open Observatory of Network Interference}},
	Year = {2012}}
	
	@misc{caida-rank,
	Howpublished = {\url{http://as-rank.caida.org/}},
	Key = {caida rank},
	Title = {{AS Rank: AS Ranking}}}
	
	@inproceedings{usersrouted-ccs13,
	Author = {A.~Johnson and C.~Wacek and R.~Jansen and M.~Sherr and P.~Syverson},
	Booktitle = {CCS},
	Title = {{Users Get Routed: Traffic Correlation on Tor by Realistic Adversaries}},
	Year = {2013}}
	
	@inproceedings{edman2009awareness,
	Author = {Edman, M. and Syverson, P.},
	Booktitle = {{CCS}},
	Title = {{AS-awareness in Tor path selection}},
	Year = {2009}}
	
	@inproceedings{DecoyCosts,
	Author = {A.~Houmansadr and E.~L.~Wong and V.~Shmatikov},
	Booktitle = {NDSS},
	Title = {{No Direction Home: The True Cost of Routing Around Decoys}},
	Year = {2014}}
	
	@article{cordon,
	Author = {Elahi, T. and Goldberg, I.},
	Journal = {University of Waterloo CACR},
	Title = {{CORDON--A Taxonomy of Internet Censorship Resistance Strategies}},
	Volume = {33},
	Year = {2012}}
	
	@inproceedings{privex,
	Author = {T.~Elahi and G.~Danezis and I.~Goldberg	},
	Booktitle = {{CCS}},
	Title = {{AS-awareness in Tor path selection}},
	Year = {2014}}
	
	@inproceedings{changeGuards,
	Author = {T.~Elahi and K.~Bauer and M.~AlSabah and R.~Dingledine and I.~Goldberg},
	Booktitle = {{WPES}},
	Title = {{ Changing of the Guards: Framework for Understanding and Improving Entry Guard Selection in Tor}},
	Year = {2012}}
	
	@article{RAINBOW:Journal,
	Author = {A.~Houmansadr and N.~Kiyavash and N.~Borisov},
	Journal = {IEEE/ACM Transactions on Networking},
	Title = {{Non-Blind Watermarking of Network Flows}},
	Year = 2014}
	
	@inproceedings{info-tod,
	Author = {A.~Houmansadr and S.~Gorantla and T.~Coleman and N.~Kiyavash and and N.~Borisov},
	Booktitle = {{CCS (poster session)}},
	Title = {{On the Channel Capacity of Network Flow Watermarking}},
	Year = {2009}}
	
	@inproceedings{johnson2014game,
	Author = {Johnson, B. and Laszka, A. and Grossklags, J. and Vasek, M. and Moore, T.},
	Booktitle = {Workshop on Bitcoin Research},
	Title = {{Game-theoretic Analysis of DDoS Attacks Against Bitcoin Mining Pools}},
	Year = {2014}}
	
	@incollection{laszka2013mitigation,
	Author = {Laszka, A. and Johnson, B. and Grossklags, J.},
	Booktitle = {Decision and Game Theory for Security},
	Pages = {175--191},
	Publisher = {Springer},
	Title = {{Mitigation of Targeted and Non-targeted Covert Attacks as a Timing Game}},
	Year = {2013}}
	
	@inproceedings{schottle2013game,
	Author = {Schottle, P. and Laszka, A. and Johnson, B. and Grossklags, J. and Bohme, R.},
	Booktitle = {EUSIPCO},
	Title = {{A Game-theoretic Analysis of Content-adaptive Steganography with Independent Embedding}},
	Year = {2013}}
	
	@inproceedings{CloudTransport,
	Author = {C.~Brubaker and A.~Houmansadr and V.~Shmatikov},
	Booktitle = {PETS},
	Title = {{CloudTransport: Using Cloud Storage for Censorship-Resistant Networking}},
	Year = {2014}}
	
	@inproceedings{sweet,
	Author = {W.~Zhou and A.~Houmansadr and M.~Caesar and N.~Borisov},
	Booktitle = {HotPETs},
	Title = {{SWEET: Serving the Web by Exploiting Email Tunnels}},
	Year = {2013}}
	
	@inproceedings{ahsan2002practical,
	Author = {Ahsan, K. and Kundur, D.},
	Booktitle = {Workshop on Multimedia Security},
	Title = {{Practical data hiding in TCP/IP}},
	Year = {2002}}
	
	@incollection{danezis2011covert,
	Author = {Danezis, G.},
	Booktitle = {Security Protocols XVI},
	Pages = {198--214},
	Publisher = {Springer},
	Title = {{Covert Communications Despite Traffic Data Retention}},
	Year = {2011}}
	
	@inproceedings{liu2009hide,
	Author = {Liu, Y. and Ghosal, D. and Armknecht, F. and Sadeghi, A.-R. and Schulz, S. and Katzenbeisser, S.},
	Booktitle = {ESORICS},
	Title = {{Hide and Seek in Time---Robust Covert Timing Channels}},
	Year = {2009}}
	
	@misc{image-watermark-fing,
	Author = {Jonathan Bailey},
	Howpublished = {\url{https://www.plagiarismtoday.com/2009/12/02/image-detection-watermarking-vs-fingerprinting/}},
	Title = {{Image Detection: Watermarking vs. Fingerprinting}},
	Year = {2009}}
	
	@inproceedings{Servetto98,
	Author = {S. D. Servetto and C. I. Podilchuk and K. Ramchandran},
	Booktitle = {Int. Conf. Image Processing},
	Title = {Capacity issues in digital image watermarking},
	Year = {1998}}
	
	@inproceedings{Chen01,
	Author = {B. Chen and G.W.Wornell},
	Booktitle = {IEEE Trans. Inform. Theory},
	Pages = {1423--1443},
	Title = {Quantization index modulation: A class of provably good methods for digital watermarking and information embedding},
	Year = {2001}}
	
	@inproceedings{Karakos00,
	Author = {D. Karakos and A. Papamarcou},
	Booktitle = {IEEE Int. Symp. Information Theory},
	Pages = {47},
	Title = {Relationship between quantization and distribution rates of digitally watermarked data},
	Year = {2000}}
	
	@inproceedings{Sullivan98,
	Author = {J. A. OSullivan and P. Moulin and J. M. Ettinger},
	Booktitle = {IEEE Int. Symp. Information Theory},
	Pages = {297},
	Title = {Information theoretic analysis of steganography},
	Year = {1998}}
	
	@inproceedings{Merhav00,
	Author = {N. Merhav},
	Booktitle = {IEEE Trans. Inform. Theory},
	Pages = {420--430},
	Title = {On random coding error exponents of watermarking systems},
	Year = {2000}}
	
	@inproceedings{Somekh01,
	Author = {A. Somekh-Baruch and N. Merhav},
	Booktitle = {IEEE Int. Symp. Information Theory},
	Pages = {7},
	Title = {On the error exponent and capacity games of private watermarking systems},
	Year = {2001}}
	
	@inproceedings{Steinberg01,
	Author = {Y. Steinberg and N. Merhav},
	Booktitle = {IEEE Trans. Inform. Theory},
	Pages = {1410--1422},
	Title = {Identification in the presence of side information with application to watermarking},
	Year = {2001}}
	
	@article{Moulin03,
	Author = {P. Moulin and J.A. O'Sullivan},
	Journal = {IEEE Trans. Info. Theory},
	Number = {3},
	Title = {Information-theoretic analysis of information hiding},
	Volume = 49,
	Year = 2003}
	
	@article{Gelfand80,
	Author = {S.I.~Gelfand and M.S.~Pinsker},
	Journal = {Problems of Control and Information Theory},
	Number = {1},
	Pages = {19-31},
	Title = {{Coding for channel with random parameters}},
	Url = {citeseer.ist.psu.edu/anantharam96bits.html},
	Volume = {9},
	Year = {1980},
	Bdsk-Url-1 = {citeseer.ist.psu.edu/anantharam96bits.html}}
	
	@book{Wolfowitz78,
	Author = {J. Wolfowitz},
	Edition = {3rd},
	Location = {New York},
	Publisher = {Springer-Verlag},
	Title = {Coding Theorems of Information Theory},
	Year = 1978}
	
	@article{caire99,
	Author = {G. Caire and S. Shamai},
	Journal = {IEEE Transactions on Information Theory},
	Number = {6},
	Pages = {2007--2019},
	Title = {On the Capacity of Some Channels with Channel State Information},
	Volume = {45},
	Year = {1999}}
	
	@inproceedings{wright2007language,
	Author = {Wright, Charles V and Ballard, Lucas and Monrose, Fabian and Masson, Gerald M},
	Booktitle = {USENIX Security},
	Title = {{Language identification of encrypted VoIP traffic: Alejandra y Roberto or Alice and Bob?}},
	Year = {2007}}
	
	@inproceedings{backes2010speaker,
	Author = {Backes, Michael and Doychev, Goran and D{\"u}rmuth, Markus and K{\"o}pf, Boris},
	Booktitle = {{European Symposium on Research in Computer Security (ESORICS)}},
	Pages = {508--523},
	Publisher = {Springer},
	Title = {{Speaker Recognition in Encrypted Voice Streams}},
	Year = {2010}}
	
	@phdthesis{lu2009traffic,
	Author = {Lu, Yuanchao},
	School = {Cleveland State University},
	Title = {{On Traffic Analysis Attacks to Encrypted VoIP Calls}},
	Year = {2009}}
	
	@inproceedings{wright2008spot,
	Author = {Wright, Charles V and Ballard, Lucas and Coull, Scott E and Monrose, Fabian and Masson, Gerald M},
	Booktitle = {IEEE Symposium on Security and Privacy},
	Pages = {35--49},
	Title = {Spot me if you can: Uncovering spoken phrases in encrypted VoIP conversations},
	Year = {2008}}
	
	@inproceedings{white2011phonotactic,
	Author = {White, Andrew M and Matthews, Austin R and Snow, Kevin Z and Monrose, Fabian},
	Booktitle = {IEEE Symposium on Security and Privacy},
	Pages = {3--18},
	Title = {Phonotactic reconstruction of encrypted VoIP conversations: Hookt on fon-iks},
	Year = {2011}}
	
	@inproceedings{fancy,
	Author = {Houmansadr, Amir and Borisov, Nikita},
	Booktitle = {Privacy Enhancing Technologies},
	Organization = {Springer},
	Pages = {205--224},
	Title = {The Need for Flow Fingerprints to Link Correlated Network Flows},
	Year = {2013}}
	
	@article{botmosaic,
	Author = {Amir Houmansadr and Nikita Borisov},
	Doi = {10.1016/j.jss.2012.11.005},
	Issn = {0164-1212},
	Journal = {Journal of Systems and Software},
	Keywords = {Network security},
	Number = {3},
	Pages = {707 - 715},
	Title = {BotMosaic: Collaborative network watermark for the detection of IRC-based botnets},
	Url = {http://www.sciencedirect.com/science/article/pii/S0164121212003068},
	Volume = {86},
	Year = {2013},
	Bdsk-Url-1 = {http://www.sciencedirect.com/science/article/pii/S0164121212003068},
	Bdsk-Url-2 = {http://dx.doi.org/10.1016/j.jss.2012.11.005}}
	
	@inproceedings{ramsbrock2008first,
	Author = {Ramsbrock, Daniel and Wang, Xinyuan and Jiang, Xuxian},
	Booktitle = {Recent Advances in Intrusion Detection},
	Organization = {Springer},
	Pages = {59--77},
	Title = {A first step towards live botmaster traceback},
	Year = {2008}}
	
	@inproceedings{potdar2005survey,
	Author = {Potdar, Vidyasagar M and Han, Song and Chang, Elizabeth},
	Booktitle = {Industrial Informatics, 2005. INDIN'05. 2005 3rd IEEE International Conference on},
	Organization = {IEEE},
	Pages = {709--716},
	Title = {A survey of digital image watermarking techniques},
	Year = {2005}}
	
	@book{cole2003hiding,
	Author = {Cole, Eric and Krutz, Ronald D},
	Publisher = {John Wiley \& Sons, Inc.},
	Title = {Hiding in plain sight: Steganography and the art of covert communication},
	Year = {2003}}
	
	@incollection{akaike1998information,
	Author = {Akaike, Hirotogu},
	Booktitle = {Selected Papers of Hirotugu Akaike},
	Pages = {199--213},
	Publisher = {Springer},
	Title = {Information theory and an extension of the maximum likelihood principle},
	Year = {1998}}
	
	@misc{central-command-hack,
	Author = {Everett Rosenfeld},
	Howpublished = {\url{http://www.cnbc.com/id/102330338}},
	Title = {{FBI investigating Central Command Twitter hack}},
	Year = {2015}}
	
	@misc{sony-psp-ddos,
	Howpublished = {\url{http://n4g.com/news/1644853/sony-and-microsoft-cant-do-much-ddos-attacks-explained}},
	Key = {sony},
	Month = {December},
	Title = {{Sony and Microsoft cant do much -- DDoS attacks explained}},
	Year = {2014}}
	
	@misc{sony-hack,
	Author = {David Bloom},
	Howpublished = {\url{http://goo.gl/MwR4A7}},
	Title = {{Online Game Networks Hacked, Sony Unit President Threatened}},
	Year = {2014}}
	
	@misc{home-depot,
	Author = {Dune Lawrence},
	Howpublished = {\url{http://www.businessweek.com/articles/2014-09-02/home-depots-credit-card-breach-looks-just-like-the-target-hack}},
	Title = {{Home Depot's Suspected Breach Looks Just Like the Target Hack}},
	Year = {2014}}
	
	@misc{target,
	Author = {Julio Ojeda-Zapata},
	Howpublished = {\url{http://www.mercurynews.com/business/ci_24765398/how-did-hackers-pull-off-target-data-heist}},
	Title = {{Target hack: How did they do it?}},
	Year = {2014}}
	
	
	@article{probabilitycourse,
	Author = {H. Pishro-Nik},
	note = {\url{http://www.probabilitycourse.com}},
	Title = {Introduction to probability, statistics, and random processes},
	Year = {2014}}
	
	
	
	@inproceedings{shokri2011quantifying,
	Author = {Shokri, Reza and Theodorakopoulos, George and Le Boudec, Jean-Yves and Hubaux, Jean-Pierre},
	Booktitle = {Security and Privacy (SP), 2011 IEEE Symposium on},
	Organization = {IEEE},
	Pages = {247--262},
	Title = {Quantifying location privacy},
	Year = {2011}}
	
	@inproceedings{hoh2007preserving,
	Author = {Hoh, Baik and Gruteser, Marco and Xiong, Hui and Alrabady, Ansaf},
	Booktitle = {Proceedings of the 14th ACM conference on Computer and communications security},
	Organization = {ACM},
	Pages = {161--171},
	Title = {Preserving privacy in gps traces via uncertainty-aware path cloaking},
	Year = {2007}}
	
	
	
	@article{kafsi2013entropy,
	Author = {Kafsi, Mohamed and Grossglauser, Matthias and Thiran, Patrick},
	Journal = {Information Theory, IEEE Transactions on},
	Number = {9},
	Pages = {5577--5583},
	Publisher = {IEEE},
	Title = {The entropy of conditional Markov trajectories},
	Volume = {59},
	Year = {2013}}
	
	@inproceedings{gruteser2003anonymous,
	Author = {Gruteser, Marco and Grunwald, Dirk},
	Booktitle = {Proceedings of the 1st international conference on Mobile systems, applications and services},
	Organization = {ACM},
	Pages = {31--42},
	Title = {Anonymous usage of location-based services through spatial and temporal cloaking},
	Year = {2003}}
	
	@inproceedings{husted2010mobile,
	Author = {Husted, Nathaniel and Myers, Steven},
	Booktitle = {Proceedings of the 17th ACM conference on Computer and communications security},
	Organization = {ACM},
	Pages = {85--96},
	Title = {Mobile location tracking in metro areas: malnets and others},
	Year = {2010}}
	
	@inproceedings{li2009tradeoff,
	Author = {Li, Tiancheng and Li, Ninghui},
	Booktitle = {Proceedings of the 15th ACM SIGKDD international conference on Knowledge discovery and data mining},
	Organization = {ACM},
	Pages = {517--526},
	Title = {On the tradeoff between privacy and utility in data publishing},
	Year = {2009}}
	
	@inproceedings{ma2009location,
	Author = {Ma, Zhendong and Kargl, Frank and Weber, Michael},
	Booktitle = {Sarnoff Symposium, 2009. SARNOFF'09. IEEE},
	Organization = {IEEE},
	Pages = {1--6},
	Title = {A location privacy metric for v2x communication systems},
	Year = {2009}}
	
	@inproceedings{shokri2012protecting,
	Author = {Shokri, Reza and Theodorakopoulos, George and Troncoso, Carmela and Hubaux, Jean-Pierre and Le Boudec, Jean-Yves},
	Booktitle = {Proceedings of the 2012 ACM conference on Computer and communications security},
	Organization = {ACM},
	Pages = {617--627},
	Title = {Protecting location privacy: optimal strategy against localization attacks},
	Year = {2012}}
	
	@inproceedings{freudiger2009non,
	Author = {Freudiger, Julien and Manshaei, Mohammad Hossein and Hubaux, Jean-Pierre and Parkes, David C},
	Booktitle = {Proceedings of the 16th ACM conference on Computer and communications security},
	Organization = {ACM},
	Pages = {324--337},
	Title = {On non-cooperative location privacy: a game-theoretic analysis},
	Year = {2009}}
	
	@incollection{humbert2010tracking,
	Author = {Humbert, Mathias and Manshaei, Mohammad Hossein and Freudiger, Julien and Hubaux, Jean-Pierre},
	Booktitle = {Decision and Game Theory for Security},
	Pages = {38--57},
	Publisher = {Springer},
	Title = {Tracking games in mobile networks},
	Year = {2010}}
	
	@article{manshaei2013game,
	Author = {Manshaei, Mohammad Hossein and Zhu, Quanyan and Alpcan, Tansu and Bac{\c{s}}ar, Tamer and Hubaux, Jean-Pierre},
	Journal = {ACM Computing Surveys (CSUR)},
	Number = {3},
	Pages = {25},
	Publisher = {ACM},
	Title = {Game theory meets network security and privacy},
	Volume = {45},
	Year = {2013}}
	
	@article{palamidessi2006probabilistic,
	Author = {Palamidessi, Catuscia},
	Journal = {Electronic Notes in Theoretical Computer Science},
	Pages = {33--42},
	Publisher = {Elsevier},
	Title = {Probabilistic and nondeterministic aspects of anonymity},
	Volume = {155},
	Year = {2006}}
	
	@inproceedings{mokbel2006new,
	Author = {Mokbel, Mohamed F and Chow, Chi-Yin and Aref, Walid G},
	Booktitle = {Proceedings of the 32nd international conference on Very large data bases},
	Organization = {VLDB Endowment},
	Pages = {763--774},
	Title = {The new Casper: query processing for location services without compromising privacy},
	Year = {2006}}
	
	@article{kalnis2007preventing,
	Author = {Kalnis, Panos and Ghinita, Gabriel and Mouratidis, Kyriakos and Papadias, Dimitris},
	Journal = {Knowledge and Data Engineering, IEEE Transactions on},
	Number = {12},
	Pages = {1719--1733},
	Publisher = {IEEE},
	Title = {Preventing location-based identity inference in anonymous spatial queries},
	Volume = {19},
	Year = {2007}}
	
	@article{freudiger2007mix,
	title={Mix-zones for location privacy in vehicular networks},
	author={Freudiger, Julien and Raya, Maxim and F{\'e}legyh{\'a}zi, M{\'a}rk and Papadimitratos, Panos and Hubaux, Jean-Pierre},
	year={2007}
	}
	@article{sweeney2002k,
	Author = {Sweeney, Latanya},
	Journal = {International Journal of Uncertainty, Fuzziness and Knowledge-Based Systems},
	Number = {05},
	Pages = {557--570},
	Publisher = {World Scientific},
	Title = {k-anonymity: A model for protecting privacy},
	Volume = {10},
	Year = {2002}}
	
	@article{sweeney2002achieving,
	Author = {Sweeney, Latanya},
	Journal = {International Journal of Uncertainty, Fuzziness and Knowledge-Based Systems},
	Number = {05},
	Pages = {571--588},
	Publisher = {World Scientific},
	Title = {Achieving k-anonymity privacy protection using generalization and suppression},
	Volume = {10},
	Year = {2002}}
	
	@inproceedings{niu2014achieving,
	Author = {Niu, Ben and Li, Qinghua and Zhu, Xiaoyan and Cao, Guohong and Li, Hui},
	Booktitle = {INFOCOM, 2014 Proceedings IEEE},
	Organization = {IEEE},
	Pages = {754--762},
	Title = {Achieving k-anonymity in privacy-aware location-based services},
	Year = {2014}}
	
	@inproceedings{liu2013game,
	Author = {Liu, Xinxin and Liu, Kaikai and Guo, Linke and Li, Xiaolin and Fang, Yuguang},
	Booktitle = {INFOCOM, 2013 Proceedings IEEE},
	Organization = {IEEE},
	Pages = {2985--2993},
	Title = {A game-theoretic approach for achieving k-anonymity in location based services},
	Year = {2013}}
	
	@inproceedings{kido2005protection,
	Author = {Kido, Hidetoshi and Yanagisawa, Yutaka and Satoh, Tetsuji},
	Booktitle = {Data Engineering Workshops, 2005. 21st International Conference on},
	Organization = {IEEE},
	Pages = {1248--1248},
	Title = {Protection of location privacy using dummies for location-based services},
	Year = {2005}}
	
	@inproceedings{gedik2005location,
	Author = {Gedik, Bu{\u{g}}ra and Liu, Ling},
	Booktitle = {Distributed Computing Systems, 2005. ICDCS 2005. Proceedings. 25th IEEE International Conference on},
	Organization = {IEEE},
	Pages = {620--629},
	Title = {Location privacy in mobile systems: A personalized anonymization model},
	Year = {2005}}
	
	@inproceedings{bordenabe2014optimal,
	Author = {Bordenabe, Nicol{\'a}s E and Chatzikokolakis, Konstantinos and Palamidessi, Catuscia},
	Booktitle = {Proceedings of the 2014 ACM SIGSAC Conference on Computer and Communications Security},
	Organization = {ACM},
	Pages = {251--262},
	Title = {Optimal geo-indistinguishable mechanisms for location privacy},
	Year = {2014}}
	
	@incollection{duckham2005formal,
	Author = {Duckham, Matt and Kulik, Lars},
	Booktitle = {Pervasive computing},
	Pages = {152--170},
	Publisher = {Springer},
	Title = {A formal model of obfuscation and negotiation for location privacy},
	Year = {2005}}
	
	@inproceedings{kido2005anonymous,
	Author = {Kido, Hidetoshi and Yanagisawa, Yutaka and Satoh, Tetsuji},
	Booktitle = {Pervasive Services, 2005. ICPS'05. Proceedings. International Conference on},
	Organization = {IEEE},
	Pages = {88--97},
	Title = {An anonymous communication technique using dummies for location-based services},
	Year = {2005}}
	
	@incollection{duckham2006spatiotemporal,
	Author = {Duckham, Matt and Kulik, Lars and Birtley, Athol},
	Booktitle = {Geographic Information Science},
	Pages = {47--64},
	Publisher = {Springer},
	Title = {A spatiotemporal model of strategies and counter strategies for location privacy protection},
	Year = {2006}}
	
	@inproceedings{shankar2009privately,
	Author = {Shankar, Pravin and Ganapathy, Vinod and Iftode, Liviu},
	Booktitle = {Proceedings of the 11th international conference on Ubiquitous computing},
	Organization = {ACM},
	Pages = {31--40},
	Title = {Privately querying location-based services with SybilQuery},
	Year = {2009}}
	
	@inproceedings{chow2009faking,
	Author = {Chow, Richard and Golle, Philippe},
	Booktitle = {Proceedings of the 8th ACM workshop on Privacy in the electronic society},
	Organization = {ACM},
	Pages = {105--108},
	Title = {Faking contextual data for fun, profit, and privacy},
	Year = {2009}}
	
	@incollection{xue2009location,
	Author = {Xue, Mingqiang and Kalnis, Panos and Pung, Hung Keng},
	Booktitle = {Location and Context Awareness},
	Pages = {70--87},
	Publisher = {Springer},
	Title = {Location diversity: Enhanced privacy protection in location based services},
	Year = {2009}}
	
	@article{wernke2014classification,
	Author = {Wernke, Marius and Skvortsov, Pavel and D{\"u}rr, Frank and Rothermel, Kurt},
	Journal = {Personal and Ubiquitous Computing},
	Number = {1},
	Pages = {163--175},
	Publisher = {Springer-Verlag},
	Title = {A classification of location privacy attacks and approaches},
	Volume = {18},
	Year = {2014}}
	
	@misc{cai2015cloaking,
	Author = {Cai, Y. and Xu, G.},
	Month = jan # {~1},
	Note = {US Patent App. 14/472,462},
	Publisher = {Google Patents},
	Title = {Cloaking with footprints to provide location privacy protection in location-based services},
	Url = {https://www.google.com/patents/US20150007341},
	Year = {2015},
	Bdsk-Url-1 = {https://www.google.com/patents/US20150007341}}
	
	@article{gedik2008protecting,
	Author = {Gedik, Bu{\u{g}}ra and Liu, Ling},
	Journal = {Mobile Computing, IEEE Transactions on},
	Number = {1},
	Pages = {1--18},
	Publisher = {IEEE},
	Title = {Protecting location privacy with personalized k-anonymity: Architecture and algorithms},
	Volume = {7},
	Year = {2008}}
	
	@article{kalnis2006preserving,
	Author = {Kalnis, Panos and Ghinita, Gabriel and Mouratidis, Kyriakos and Papadias, Dimitris},
	Publisher = {TRB6/06},
	Title = {Preserving anonymity in location based services},
	Year = {2006}}
	
	@inproceedings{hoh2005protecting,
	Author = {Hoh, Baik and Gruteser, Marco},
	Booktitle = {Security and Privacy for Emerging Areas in Communications Networks, 2005. SecureComm 2005. First International Conference on},
	Organization = {IEEE},
	Pages = {194--205},
	Title = {Protecting location privacy through path confusion},
	Year = {2005}}
	
	@article{terrovitis2011privacy,
	Author = {Terrovitis, Manolis},
	Journal = {ACM SIGKDD Explorations Newsletter},
	Number = {1},
	Pages = {6--18},
	Publisher = {ACM},
	Title = {Privacy preservation in the dissemination of location data},
	Volume = {13},
	Year = {2011}}
	
	@article{shin2012privacy,
	Author = {Shin, Kang G and Ju, Xiaoen and Chen, Zhigang and Hu, Xin},
	Journal = {Wireless Communications, IEEE},
	Number = {1},
	Pages = {30--39},
	Publisher = {IEEE},
	Title = {Privacy protection for users of location-based services},
	Volume = {19},
	Year = {2012}}
	
	@article{khoshgozaran2011location,
	Author = {Khoshgozaran, Ali and Shahabi, Cyrus and Shirani-Mehr, Houtan},
	Journal = {Knowledge and Information Systems},
	Number = {3},
	Pages = {435--465},
	Publisher = {Springer},
	Title = {Location privacy: going beyond K-anonymity, cloaking and anonymizers},
	Volume = {26},
	Year = {2011}}
	
	@incollection{chatzikokolakis2015geo,
	Author = {Chatzikokolakis, Konstantinos and Palamidessi, Catuscia and Stronati, Marco},
	Booktitle = {Distributed Computing and Internet Technology},
	Pages = {49--72},
	Publisher = {Springer},
	Title = {Geo-indistinguishability: A Principled Approach to Location Privacy},
	Year = {2015}}
	
	@inproceedings{ngo2015location,
	Author = {Ngo, Hoa and Kim, Jong},
	Booktitle = {Computer Security Foundations Symposium (CSF), 2015 IEEE 28th},
	Organization = {IEEE},
	Pages = {63--74},
	Title = {Location Privacy via Differential Private Perturbation of Cloaking Area},
	Year = {2015}}
	
	@inproceedings{palanisamy2011mobimix,
	Author = {Palanisamy, Balaji and Liu, Ling},
	Booktitle = {Data Engineering (ICDE), 2011 IEEE 27th International Conference on},
	Organization = {IEEE},
	Pages = {494--505},
	Title = {Mobimix: Protecting location privacy with mix-zones over road networks},
	Year = {2011}}
	
	@inproceedings{um2010advanced,
	Author = {Um, Jung-Ho and Kim, Hee-Dae and Chang, Jae-Woo},
	Booktitle = {Social Computing (SocialCom), 2010 IEEE Second International Conference on},
	Organization = {IEEE},
	Pages = {1093--1098},
	Title = {An advanced cloaking algorithm using Hilbert curves for anonymous location based service},
	Year = {2010}}
	
	@inproceedings{bamba2008supporting,
	Author = {Bamba, Bhuvan and Liu, Ling and Pesti, Peter and Wang, Ting},
	Booktitle = {Proceedings of the 17th international conference on World Wide Web},
	Organization = {ACM},
	Pages = {237--246},
	Title = {Supporting anonymous location queries in mobile environments with privacygrid},
	Year = {2008}}
	
	@inproceedings{zhangwei2010distributed,
	Author = {Zhangwei, Huang and Mingjun, Xin},
	Booktitle = {Networks Security Wireless Communications and Trusted Computing (NSWCTC), 2010 Second International Conference on},
	Organization = {IEEE},
	Pages = {468--471},
	Title = {A distributed spatial cloaking protocol for location privacy},
	Volume = {2},
	Year = {2010}}
	
	@article{chow2011spatial,
	Author = {Chow, Chi-Yin and Mokbel, Mohamed F and Liu, Xuan},
	Journal = {GeoInformatica},
	Number = {2},
	Pages = {351--380},
	Publisher = {Springer},
	Title = {Spatial cloaking for anonymous location-based services in mobile peer-to-peer environments},
	Volume = {15},
	Year = {2011}}
	
	@inproceedings{lu2008pad,
	Author = {Lu, Hua and Jensen, Christian S and Yiu, Man Lung},
	Booktitle = {Proceedings of the Seventh ACM International Workshop on Data Engineering for Wireless and Mobile Access},
	Organization = {ACM},
	Pages = {16--23},
	Title = {Pad: privacy-area aware, dummy-based location privacy in mobile services},
	Year = {2008}}
	
	@incollection{khoshgozaran2007blind,
	Author = {Khoshgozaran, Ali and Shahabi, Cyrus},
	Booktitle = {Advances in Spatial and Temporal Databases},
	Pages = {239--257},
	Publisher = {Springer},
	Title = {Blind evaluation of nearest neighbor queries using space transformation to preserve location privacy},
	Year = {2007}}
	
	@inproceedings{ghinita2008private,
	Author = {Ghinita, Gabriel and Kalnis, Panos and Khoshgozaran, Ali and Shahabi, Cyrus and Tan, Kian-Lee},
	Booktitle = {Proceedings of the 2008 ACM SIGMOD international conference on Management of data},
	Organization = {ACM},
	Pages = {121--132},
	Title = {Private queries in location based services: anonymizers are not necessary},
	Year = {2008}}
	
	@article{paulet2014privacy,
	Author = {Paulet, Russell and Kaosar, Md Golam and Yi, Xun and Bertino, Elisa},
	Journal = {Knowledge and Data Engineering, IEEE Transactions on},
	Number = {5},
	Pages = {1200--1210},
	Publisher = {IEEE},
	Title = {Privacy-preserving and content-protecting location based queries},
	Volume = {26},
	Year = {2014}}
	
	@article{nguyen2013differential,
	Author = {Nguyen, Hiep H and Kim, Jong and Kim, Yoonho},
	Journal = {Journal of Computing Science and Engineering},
	Number = {3},
	Pages = {177--186},
	Title = {Differential privacy in practice},
	Volume = {7},
	Year = {2013}}
	
	@inproceedings{lee2012differential,
	Author = {Lee, Jaewoo and Clifton, Chris},
	Booktitle = {Proceedings of the 18th ACM SIGKDD international conference on Knowledge discovery and data mining},
	Organization = {ACM},
	Pages = {1041--1049},
	Title = {Differential identifiability},
	Year = {2012}}
	
	@inproceedings{andres2013geo,
	Author = {Andr{\'e}s, Miguel E and Bordenabe, Nicol{\'a}s E and Chatzikokolakis, Konstantinos and Palamidessi, Catuscia},
	Booktitle = {Proceedings of the 2013 ACM SIGSAC conference on Computer \& communications security},
	Organization = {ACM},
	Pages = {901--914},
	Title = {Geo-indistinguishability: Differential privacy for location-based systems},
	Year = {2013}}
	
	@inproceedings{machanavajjhala2008privacy,
	Author = {Machanavajjhala, Ashwin and Kifer, Daniel and Abowd, John and Gehrke, Johannes and Vilhuber, Lars},
	Booktitle = {Data Engineering, 2008. ICDE 2008. IEEE 24th International Conference on},
	Organization = {IEEE},
	Pages = {277--286},
	Title = {Privacy: Theory meets practice on the map},
	Year = {2008}}
	
	@article{dewri2013local,
	Author = {Dewri, Rinku},
	Journal = {Mobile Computing, IEEE Transactions on},
	Number = {12},
	Pages = {2360--2372},
	Publisher = {IEEE},
	Title = {Local differential perturbations: Location privacy under approximate knowledge attackers},
	Volume = {12},
	Year = {2013}}
	
	@inproceedings{chatzikokolakis2013broadening,
	Author = {Chatzikokolakis, Konstantinos and Andr{\'e}s, Miguel E and Bordenabe, Nicol{\'a}s Emilio and Palamidessi, Catuscia},
	Booktitle = {Privacy Enhancing Technologies},
	Organization = {Springer},
	Pages = {82--102},
	Title = {Broadening the Scope of Differential Privacy Using Metrics.},
	Year = {2013}}
	
	@inproceedings{zhong2009distributed,
	Author = {Zhong, Ge and Hengartner, Urs},
	Booktitle = {Pervasive Computing and Communications, 2009. PerCom 2009. IEEE International Conference on},
	Organization = {IEEE},
	Pages = {1--10},
	Title = {A distributed k-anonymity protocol for location privacy},
	Year = {2009}}
	
	@inproceedings{ho2011differential,
	Author = {Ho, Shen-Shyang and Ruan, Shuhua},
	Booktitle = {Proceedings of the 4th ACM SIGSPATIAL International Workshop on Security and Privacy in GIS and LBS},
	Organization = {ACM},
	Pages = {17--24},
	Title = {Differential privacy for location pattern mining},
	Year = {2011}}
	
	@inproceedings{cheng2006preserving,
	Author = {Cheng, Reynold and Zhang, Yu and Bertino, Elisa and Prabhakar, Sunil},
	Booktitle = {Privacy Enhancing Technologies},
	Organization = {Springer},
	Pages = {393--412},
	Title = {Preserving user location privacy in mobile data management infrastructures},
	Year = {2006}}
	
	@article{beresford2003location,
	Author = {Beresford, Alastair R and Stajano, Frank},
	Journal = {IEEE Pervasive computing},
	Number = {1},
	Pages = {46--55},
	Publisher = {IEEE},
	Title = {Location privacy in pervasive computing},
	Year = {2003}}
	
	@inproceedings{freudiger2009optimal,
	Author = {Freudiger, Julien and Shokri, Reza and Hubaux, Jean-Pierre},
	Booktitle = {Privacy enhancing technologies},
	Organization = {Springer},
	Pages = {216--234},
	Title = {On the optimal placement of mix zones},
	Year = {2009}}
	
	@article{krumm2009survey,
	Author = {Krumm, John},
	Journal = {Personal and Ubiquitous Computing},
	Number = {6},
	Pages = {391--399},
	Publisher = {Springer},
	Title = {A survey of computational location privacy},
	Volume = {13},
	Year = {2009}}
	
	@article{Rakhshan2016letter,
	Author = {Rakhshan, Ali and Pishro-Nik, Hossein},
	Journal = {IEEE Wireless Communications Letter},
	Publisher = {IEEE},
	Title = {Interference Models for Vehicular Ad Hoc Networks},
	Year = {2016, submitted}}
	
	@article{Rakhshan2015Journal,
	Author = {Rakhshan, Ali and Pishro-Nik, Hossein},
	Journal = {IEEE Transactions on Wireless Communications},
	Publisher = {IEEE},
	Title = {Improving Safety on Highways by Customizing Vehicular Ad Hoc Networks},
	Year = {to appear, 2017}}
	
	@inproceedings{Rakhshan2015Cogsima,
	Author = {Rakhshan, Ali and Pishro-Nik, Hossein},
	Booktitle = {IEEE International Multi-Disciplinary Conference on Cognitive Methods in Situation Awareness and Decision Support},
	Organization = {IEEE},
	Title = {A New Approach to Customization of Accident Warning Systems to Individual Drivers},
	Year = {2015}}
	
	@inproceedings{Rakhshan2015CISS,
	Author = {Rakhshan, Ali and Pishro-Nik, Hossein and Nekoui, Mohammad},
	Booktitle = {Conference on Information Sciences and Systems},
	Organization = {IEEE},
	Pages = {1--6},
	Title = {Driver-based adaptation of Vehicular Ad Hoc Networks for design of active safety systems},
	Year = {2015}}
	
	@inproceedings{Rakhshan2014IV,
	Author = {Rakhshan, Ali and Pishro-Nik, Hossein and Ray, Evan},
	Booktitle = {Intelligent Vehicles Symposium},
	Organization = {IEEE},
	Pages = {1181--1186},
	Title = {Real-time estimation of the distribution of brake response times for an individual driver using Vehicular Ad Hoc Network.},
	Year = {2014}}
	
	@inproceedings{Rakhshan2013Globecom,
	Author = {Rakhshan, Ali and Pishro-Nik, Hossein and Fisher, Donald and Nekoui, Mohammad},
	Booktitle = {IEEE Global Communications Conference},
	Organization = {IEEE},
	Pages = {1333--1337},
	Title = {Tuning collision warning algorithms to individual drivers for design of active safety systems.},
	Year = {2013}}
	
	@article{Nekoui2012Journal,
	Author = {Nekoui, Mohammad and Pishro-Nik, Hossein},
	Journal = {IEEE Transactions on Wireless Communications},
	Number = {8},
	Pages = {2895--2905},
	Publisher = {IEEE},
	Title = {Throughput Scaling laws for Vehicular Ad Hoc Networks},
	Volume = {11},
	Year = {2012}}
	
	
	
	
	
	
	
	
	
	@article{Nekoui2011Journal,
	Author = {Nekoui, Mohammad and Pishro-Nik, Hossein and Ni, Daiheng},
	Journal = {International Journal of Vehicular Technology},
	Pages = {1--11},
	Publisher = {Hindawi Publishing Corporation},
	Title = {Analytic Design of Active Safety Systems for Vehicular Ad hoc Networks},
	Volume = {2011},
	Year = {2011}}
	
	
	
	
	
	
	@article{shokri2014optimal,
	title={Optimal user-centric data obfuscation},
	author={Shokri, Reza},
	journal={arXiv preprint arXiv:1402.3426},
	year={2014}
	}
	@article{chatzikokolakis2015location,
	title={Location privacy via geo-indistinguishability},
	author={Chatzikokolakis, Konstantinos and Palamidessi, Catuscia and Stronati, Marco},
	journal={ACM SIGLOG News},
	volume={2},
	number={3},
	pages={46--69},
	year={2015},
	publisher={ACM}
	
	}
	@inproceedings{shokri2011quantifying2,
	title={Quantifying location privacy: the case of sporadic location exposure},
	author={Shokri, Reza and Theodorakopoulos, George and Danezis, George and Hubaux, Jean-Pierre and Le Boudec, Jean-Yves},
	booktitle={Privacy Enhancing Technologies},
	pages={57--76},
	year={2011},
	organization={Springer}
	}
	
	
	@inproceedings{Mont1603:Defining,
	AUTHOR="Zarrin Montazeri and Amir Houmansadr and Hossein Pishro-Nik",
	TITLE="Defining Perfect Location Privacy Using Anonymization",
	BOOKTITLE="2016 Annual Conference on Information Science and Systems (CISS) (CISS
	2016)",
	ADDRESS="Princeton, USA",
	DAYS=16,
	MONTH=mar,
	YEAR=2016,
	KEYWORDS="Information Theoretic Privacy; location-based services; Location Privacy;
	Information Theory",
	ABSTRACT="The popularity of mobile devices and location-based services (LBS) have
	created great concerns regarding the location privacy of users of such
	devices and services. Anonymization is a common technique that is often
	being used to protect the location privacy of LBS users. In this paper, we
	provide a general information theoretic definition for location privacy. In
	particular, we define perfect location privacy. We show that under certain
	conditions, perfect privacy is achieved if the pseudonyms of users is
	changed after o(N^(2/r?1)) observations by the adversary, where N is the
	number of users and r is the number of sub-regions or locations.
	"
	}
	@article{our-isita-location,
	Author = {Zarrin Montazeri and Amir Houmansadr and Hossein Pishro-Nik},
	Journal = {IEEE International Symposium on Information Theory and Its Applications (ISITA)},
	Title = {Achieving Perfect Location Privacy in Markov Models Using Anonymization},
	Year = {2016}
	}
	@article{our-TIFS,
	Author = {Zarrin Montazeri and Hossein Pishro-Nik and Amir Houmansadr},
	Journal = {IEEE Transactions on Information Forensics and Security, accepted with mandatory minor revisions},
	Title = {Perfect Location Privacy Using Anonymization in Mobile Networks},
	Year = {2017},
	note={Available on arxiv.org}
	}
	
	
	
	@techreport{sampigethaya2005caravan,
	title={CARAVAN: Providing location privacy for VANET},
	author={Sampigethaya, Krishna and Huang, Leping and Li, Mingyan and Poovendran, Radha and Matsuura, Kanta and Sezaki, Kaoru},
	year={2005},
	institution={DTIC Document}
	}
	@incollection{buttyan2007effectiveness,
	title={On the effectiveness of changing pseudonyms to provide location privacy in VANETs},
	author={Butty{\'a}n, Levente and Holczer, Tam{\'a}s and Vajda, Istv{\'a}n},
	booktitle={Security and Privacy in Ad-hoc and Sensor Networks},
	pages={129--141},
	year={2007},
	publisher={Springer}
	}
	@article{sampigethaya2007amoeba,
	title={AMOEBA: Robust location privacy scheme for VANET},
	author={Sampigethaya, Krishna and Li, Mingyan and Huang, Leping and Poovendran, Radha},
	journal={Selected Areas in communications, IEEE Journal on},
	volume={25},
	number={8},
	pages={1569--1589},
	year={2007},
	publisher={IEEE}
	}
	
	@article{lu2012pseudonym,
	title={Pseudonym changing at social spots: An effective strategy for location privacy in vanets},
	author={Lu, Rongxing and Li, Xiaodong and Luan, Tom H and Liang, Xiaohui and Shen, Xuemin},
	journal={Vehicular Technology, IEEE Transactions on},
	volume={61},
	number={1},
	pages={86--96},
	year={2012},
	publisher={IEEE}
	}
	@inproceedings{lu2010sacrificing,
	title={Sacrificing the plum tree for the peach tree: A socialspot tactic for protecting receiver-location privacy in VANET},
	author={Lu, Rongxing and Lin, Xiaodong and Liang, Xiaohui and Shen, Xuemin},
	booktitle={Global Telecommunications Conference (GLOBECOM 2010), 2010 IEEE},
	pages={1--5},
	year={2010},
	organization={IEEE}
	}
	@inproceedings{lin2011stap,
	title={STAP: A social-tier-assisted packet forwarding protocol for achieving receiver-location privacy preservation in VANETs},
	author={Lin, Xiaodong and Lu, Rongxing and Liang, Xiaohui and Shen, Xuemin Sherman},
	booktitle={INFOCOM, 2011 Proceedings IEEE},
	pages={2147--2155},
	year={2011},
	organization={IEEE}
	}
	@inproceedings{gerlach2007privacy,
	title={Privacy in VANETs using changing pseudonyms-ideal and real},
	author={Gerlach, Matthias and Guttler, Felix},
	booktitle={Vehicular Technology Conference, 2007. VTC2007-Spring. IEEE 65th},
	pages={2521--2525},
	year={2007},
	organization={IEEE}
	}
	@inproceedings{el2002security,
	title={Security issues in a future vehicular network},
	author={El Zarki, Magda and Mehrotra, Sharad and Tsudik, Gene and Venkatasubramanian, Nalini},
	booktitle={European Wireless},
	volume={2},
	year={2002}
	}
	
	@article{hubaux2004security,
	title={The security and privacy of smart vehicles},
	author={Hubaux, Jean-Pierre and Capkun, Srdjan and Luo, Jun},
	journal={IEEE Security \& Privacy Magazine},
	volume={2},
	number={LCA-ARTICLE-2004-007},
	pages={49--55},
	year={2004}
	}
	
	
	
	@inproceedings{duri2002framework,
	title={Framework for security and privacy in automotive telematics},
	author={Duri, Sastry and Gruteser, Marco and Liu, Xuan and Moskowitz, Paul and Perez, Ronald and Singh, Moninder and Tang, Jung-Mu},
	booktitle={Proceedings of the 2nd international workshop on Mobile commerce},
	pages={25--32},
	year={2002},
	organization={ACM}
	}
	@misc{NS-3,
	Howpublished = {\url{https://www.nsnam.org/}}},
}
@misc{testbed,
	Howpublished = {\url{http://www.its.dot.gov/testbed/PDF/SE-MI-Resource-Guide-9-3-1.pdf}}},
@misc{NGSIM,
	Howpublished = {\url{http://ops.fhwa.dot.gov/trafficanalysistools/ngsim.htm}},
}

@misc{National-a2013,
	Author = {National Highway Traffic Safety Administration},
	Howpublished = {\url{http://ops.fhwa.dot.gov/trafficanalysistools/ngsim.htm}},
	Title = {2013 Motor Vehicle Crashes: Overview. Traffic Safety Factors},
	Year = {2013}
}

@inproceedings{karnadi2007rapid,
	title={Rapid generation of realistic mobility models for VANET},
	author={Karnadi, Feliz Kristianto and Mo, Zhi Hai and Lan, Kun-chan},
	booktitle={Wireless Communications and Networking Conference, 2007. WCNC 2007. IEEE},
	pages={2506--2511},
	year={2007},
	organization={IEEE}
}
@inproceedings{saha2004modeling,
	title={Modeling mobility for vehicular ad-hoc networks},
	author={Saha, Amit Kumar and Johnson, David B},
	booktitle={Proceedings of the 1st ACM international workshop on Vehicular ad hoc networks},
	pages={91--92},
	year={2004},
	organization={ACM}
}
@inproceedings{lee2006modeling,
	title={Modeling steady-state and transient behaviors of user mobility: formulation, analysis, and application},
	author={Lee, Jong-Kwon and Hou, Jennifer C},
	booktitle={Proceedings of the 7th ACM international symposium on Mobile ad hoc networking and computing},
	pages={85--96},
	year={2006},
	organization={ACM}
}
@inproceedings{yoon2006building,
	title={Building realistic mobility models from coarse-grained traces},
	author={Yoon, Jungkeun and Noble, Brian D and Liu, Mingyan and Kim, Minkyong},
	booktitle={Proceedings of the 4th international conference on Mobile systems, applications and services},
	pages={177--190},
	year={2006},
	organization={ACM}
}

@inproceedings{choffnes2005integrated,
	title={An integrated mobility and traffic model for vehicular wireless networks},
	author={Choffnes, David R and Bustamante, Fabi{\'a}n E},
	booktitle={Proceedings of the 2nd ACM international workshop on Vehicular ad hoc networks},
	pages={69--78},
	year={2005},
	organization={ACM}
}

@inproceedings{Qian2008Globecom,
	title={CA Secure VANET MAC Protocol for DSRC Applications},
	author={Yi, Q. and Lu, K. and Moyeri, N.{\'a}n E},
	booktitle={Proceedings of IEEE GLOBECOM 2008},
	pages={1--5},
	year={2008},
	organization={IEEE}
}





@inproceedings{naumov2006evaluation,
	title={An evaluation of inter-vehicle ad hoc networks based on realistic vehicular traces},
	author={Naumov, Valery and Baumann, Rainer and Gross, Thomas},
	booktitle={Proceedings of the 7th ACM international symposium on Mobile ad hoc networking and computing},
	pages={108--119},
	year={2006},
	organization={ACM}
}
@article{sommer2008progressing,
	title={Progressing toward realistic mobility models in VANET simulations},
	author={Sommer, Christoph and Dressler, Falko},
	journal={Communications Magazine, IEEE},
	volume={46},
	number={11},
	pages={132--137},
	year={2008},
	publisher={IEEE}
}




@inproceedings{mahajan2006urban,
	title={Urban mobility models for vanets},
	author={Mahajan, Atulya and Potnis, Niranjan and Gopalan, Kartik and Wang, Andy},
	booktitle={2nd IEEE International Workshop on Next Generation Wireless Networks},
	volume={33},
	year={2006}
}

@inproceedings{Rakhshan2016packet,
	title={Packet success probability derivation in a vehicular ad hoc network for a highway scenario},
	author={Rakhshan, Ali and Pishro-Nik, Hossein},
	booktitle={2016 Annual Conference on Information Science and Systems (CISS)},
	pages={210--215},
	year={2016},
	organization={IEEE}
}

@inproceedings{Rakhshan2016CISS,
	Author = {Rakhshan, Ali and Pishro-Nik, Hossein},
	Booktitle = {Conference on Information Sciences and Systems},
	Organization = {IEEE},
	Pages = {210--215},
	Title = {Packet Success Probability Derivation in a Vehicular Ad Hoc Network for a Highway Scenario},
	Year = {2016}}

@article{Nekoui2013Journal,
	Author = {Nekoui, Mohammad and Pishro-Nik, Hossein},
	Journal = {Journal on Selected Areas in Communications, Special Issue on Emerging Technologies in Communications},
	Number = {9},
	Pages = {491--503},
	Publisher = {IEEE},
	Title = {Analytic Design of Active Safety Systems for Vehicular Ad hoc Networks},
	Volume = {31},
	Year = {2013}}


@inproceedings{Nekoui2011MOBICOM,
	Author = {Nekoui, Mohammad and Pishro-Nik, Hossein},
	Booktitle = {MOBICOM workshop on VehiculAr InterNETworking},
	Organization = {ACM},
	Title = {Analytic Design of Active Vehicular Safety Systems in Sparse Traffic},
	Year = {2011}}

@inproceedings{Nekoui2011VTC,
	Author = {Nekoui, Mohammad and Pishro-Nik, Hossein},
	Booktitle = {VTC-Fall},
	Organization = {IEEE},
	Title = {Analytical Design of Inter-vehicular Communications for Collision Avoidance},
	Year = {2011}}

@inproceedings{Bovee2011VTC,
	Author = {Bovee, Ben Louis and Nekoui, Mohammad and Pishro-Nik, Hossein},
	Booktitle = {VTC-Fall},
	Organization = {IEEE},
	Title = {Evaluation of the Universal Geocast Scheme For VANETs},
	Year = {2011}}

@inproceedings{Nekoui2010MOBICOM,
	Author = {Nekoui, Mohammad and Pishro-Nik, Hossein},
	Booktitle = {MOBICOM},
	Organization = {ACM},
	Title = {Fundamental Tradeoffs in Vehicular Ad Hoc Networks},
	Year = {2010}}

@inproceedings{Nekoui2010IVCS,
	Author = {Nekoui, Mohammad and Pishro-Nik, Hossein},
	Booktitle = {IVCS},
	Organization = {IEEE},
	Title = {A Universal Geocast Scheme for Vehicular Ad Hoc Networks},
	Year = {2010}}

@inproceedings{Nekoui2009ITW,
	Author = {Nekoui, Mohammad and Pishro-Nik, Hossein},
	Booktitle = {IEEE Communications Society Conference on Sensor, Mesh and Ad Hoc Communications and Networks Workshops},
	Organization = {IEEE},
	Pages = {1--3},
	Title = {A Geometrical Analysis of Obstructed Wireless Networks},
	Year = {2009}}

@article{Eslami2013Journal,
	Author = {Eslami, Ali and Nekoui, Mohammad and Pishro-Nik, Hossein and Fekri, Faramarz},
	Journal = {ACM Transactions on Sensor Networks},
	Number = {4},
	Pages = {51},
	Publisher = {ACM},
	Title = {Results on finite wireless sensor networks: Connectivity and coverage},
	Volume = {9},
	Year = {2013}}


@article{Jiafu2014Journal,
	Author = {Jiafu, W. and Zhang, D. and Zhao, S. and Yang, L. and Lloret, J.},
	Journal = {Communications Magazine},
	Number = {8},
	Pages = {106-113},
	Publisher = {IEEE},
	Title = {Context-aware vehicular cyber-physical systems with cloud support: architecture, challenges, and solutions},
	Volume = {52},
	Year = {2014}}

@inproceedings{Haas2010ACM,
	Author = {Haas, J.J. and Hu, Y.},
	Booktitle = {international workshop on VehiculAr InterNETworking},
	Organization = {ACM},
	Title = {Communication requirements for crash avoidance.},
	Year = {2010}}

@inproceedings{Yi2008GLOBECOM,
	Author = {Yi, Q. and Lu, K. and Moayeri, N.},
	Booktitle = {GLOBECOM},
	Organization = {IEEE},
	Title = {CA Secure VANET MAC Protocol for DSRC Applications.},
	Year = {2008}}

@inproceedings{Mughal2010ITSim,
	Author = {Mughal, B.M. and Wagan, A. and Hasbullah, H.},
	Booktitle = {International Symposium on Information Technology (ITSim)},
	Organization = {IEEE},
	Title = {Efficient congestion control in VANET for safety messaging.},
	Year = {2010}}

@article{Chang2011Journal,
	Author = {Chang, Y. and Lee, C. and Copeland, J.},
	Journal = {Selected Areas in Communications},
	Pages = {236 –249},
	Publisher = {IEEE},
	Title = {Goodput enhancement of VANETs in noisy CSMA/CA channels},
	Volume = {29},
	Year = {2011}}

@article{Garcia-Costa2011Journal,
	Author = {Garcia-Costa, C. and Egea-Lopez, E. and Tomas-Gabarron, J. B. and Garcia-Haro, J. and Haas, Z. J.},
	Journal = {Transactions on Intelligent Transportation Systems},
	Pages = {1 –16},
	Publisher = {IEEE},
	Title = {A stochastic model for chain collisions of vehicles equipped with vehicular communications},
	Volume = {99},
	Year = {2011}}

@article{Carbaugh2011Journal,
	Author = {Carbaugh, J. and Godbole,  D. N. and Sengupta, R. and Garcia-Haro, J. and Haas, Z. J.},
	Publisher = {Transportation Research Part C (Emerging Technologies)},
	Title = {Safety and capacity analysis of automated and manual highway systems},
	Year = {1997}}

@article{Goh2004Journal,
	Author = {Goh, P. and Wong, Y.},
	Publisher = {Appl Health Econ Health Policy},
	Title = {Driver perception response time during the signal change interval},
	Year = {2004}}

@article{Chang1985Journal,
	Author = {Chang, M.S. and Santiago, A.J.},
	Pages = {20-30},
	Publisher = {Transportation Research Record},
	Title = {Timing traffic signal changes based on driver behavior},
	Volume = {1027},
	Year = {1985}}

@article{Green2000Journal,
	Author = {Green, M.},
	Pages = {195-216},
	Publisher = {Transportation Human Factors},
	Title = {How long does it take to stop? Methodological analysis of driver perception-brake times.},
	Year = {2000}}

@article{Koppa2005,
	Author = {Koppa, R.J.},
	Pages = {195-216},
	Publisher = {http://www.fhwa.dot.gov/publications/},
	Title = {Human Factors},
	Year = {2005}}

@inproceedings{Maxwell2010ETC,
	Author = {Maxwell, A. and Wood, K.},
	Booktitle = {Europian Transport Conference},
	Organization = {http://www.etcproceedings.org/paper/review-of-traffic-signals-on-high-speed-roads},
	Title = {Review of Traffic Signals on High Speed Road},
	Year = {2010}}

@article{Wortman1983,
	Author = {Wortman, R.H. and Matthias, J.S.},
	Publisher = {Arizona Department of Transportation},
	Title = {An Evaluation of Driver Behavior at Signalized Intersections},
	Year = {1983}}
@inproceedings{Zhang2007IASTED,
	Author = {Zhang, X. and Bham, G.H.},
	Booktitle = {18th IASTED International Conference: modeling and simulation},
	Title = {Estimation of driver reaction time from detailed vehicle trajectory data.},
	Year = {2007}}


@inproceedings{bai2003important,
	title={IMPORTANT: A framework to systematically analyze the Impact of Mobility on Performance of RouTing protocols for Adhoc NeTworks},
	author={Bai, Fan and Sadagopan, Narayanan and Helmy, Ahmed},
	booktitle={INFOCOM 2003. Twenty-second annual joint conference of the IEEE computer and communications. IEEE societies},
	volume={2},
	pages={825--835},
	year={2003},
	organization={IEEE}
}


@inproceedings{abedi2008enhancing,
	title={Enhancing AODV routing protocol using mobility parameters in VANET},
	author={Abedi, Omid and Fathy, Mahmood and Taghiloo, Jamshid},
	booktitle={Computer Systems and Applications, 2008. AICCSA 2008. IEEE/ACS International Conference on},
	pages={229--235},
	year={2008},
	organization={IEEE}
}


@article{AlSultan2013Journal,
	Author = {Al-Sultan, Saif and Al-Bayatti, Ali H. and Zedan, Hussien},
	Journal = {IEEE Transactions on Vehicular Technology},
	Number = {9},
	Pages = {4264-4275},
	Publisher = {IEEE},
	Title = {Context Aware Driver Behaviour Detection System in Intelligent Transportation Systems},
	Volume = {62},
	Year = {2013}}






@article{Leow2008ITS,
	Author = {Leow, Woei Ling and Ni, Daiheng and Pishro-Nik, Hossein},
	Journal = {IEEE Transactions on Intelligent Transportation Systems},
	Number = {2},
	Pages = {369--374},
	Publisher = {IEEE},
	Title = {A Sampling Theorem Approach to Traffic Sensor Optimization},
	Volume = {9},
	Year = {2008}}



@article{REU2007,
	Author = {D. Ni and H. Pishro-Nik and R. Prasad and M. R. Kanjee and H. Zhu and T. Nguyen},
	Journal = {in 14th World Congress on Intelligent Transport Systems},
	Title = {Development of a prototype intersection collision avoidance system under VII},
	Year = {2007}}




@inproceedings{salamatian2013hide,
	title={How to hide the elephant-or the donkey-in the room: Practical privacy against statistical inference for large data.},
	author={Salamatian, Salman and Zhang, Amy and du Pin Calmon, Flavio and Bhamidipati, Sandilya and Fawaz, Nadia and Kveton, Branislav and Oliveira, Pedro and Taft, Nina},
	booktitle={GlobalSIP},
	pages={269--272},
	year={2013}
}

@article{sankar2013utility,
	title={Utility-privacy tradeoffs in databases: An information-theoretic approach},
	author={Sankar, Lalitha and Rajagopalan, S Raj and Poor, H Vincent},
	journal={Information Forensics and Security, IEEE Transactions on},
	volume={8},
	number={6},
	pages={838--852},
	year={2013},
	publisher={IEEE}
}
@inproceedings{ghinita2007prive,
	title={PRIVE: anonymous location-based queries in distributed mobile systems},
	author={Ghinita, Gabriel and Kalnis, Panos and Skiadopoulos, Spiros},
	booktitle={Proceedings of the 16th international conference on World Wide Web},
	pages={371--380},
	year={2007},
	organization={ACM}
}

@article{beresford2004mix,
	title={Mix zones: User privacy in location-aware services},
	author={Beresford, Alastair R and Stajano, Frank},
	year={2004},
	publisher={IEEE}
}

%@inproceedings{Mont1610Achieving,
	%  title={Achieving Perfect Location Privacy in Markov Models Using Anonymization},
	%  author={Montazeri, Zarrin and Houmansadr, Amir and H.Pishro-Nik},
	%  booktitle="2016 International Symposium on Information Theory and its Applications
	%  (ISITA2016)",
	%  address="Monterey, USA",
	%  days=30,
	%  month=oct,
	%  year=2016,
	%}

@article{csiszar1996almost,
	title={Almost independence and secrecy capacity},
	author={Csisz{\'a}r, Imre},
	journal={Problemy Peredachi Informatsii},
	volume={32},
	number={1},
	pages={48--57},
	year={1996},
	publisher={Russian Academy of Sciences, Branch of Informatics, Computer Equipment and Automatization}
}

@article{yamamoto1983source,
	title={A source coding problem for sources with additional outputs to keep secret from the receiver or wiretappers (corresp.)},
	author={Yamamoto, Hirosuke},
	journal={IEEE Transactions on Information Theory},
	volume={29},
	number={6},
	pages={918--923},
	year={1983},
	publisher={IEEE}
}


@inproceedings{calmon2015fundamental,
	title={Fundamental limits of perfect privacy},
	author={Calmon, Flavio P and Makhdoumi, Ali and M{\'e}dard, Muriel},
	booktitle={Information Theory (ISIT), 2015 IEEE International Symposium on},
	pages={1796--1800},
	year={2015},
	organization={IEEE}
}



@inproceedings{Lehman1999Large-Sample-Theory,
	title={Elements of Large Sample Theory},
	author={E. L. Lehman},
	organization={Springer},
	year={1999}
}


@inproceedings{Ferguson1999Large-Sample-Theory,
	title={A Course in Large Sample Theory},
	author={Thomas S. Ferguson},
	organization={CRC Press},
	year={1996}
}



@inproceedings{Dembo1999Large-Deviations,
	title={Large Deviation Techniques and Applications, Second Edition},
	author={A. Dembo and O. Zeitouni},
	organization={Springer},
	year={1998}
}


%%%%%%%%%%%%%%%%%%%%%%%%%%%%%%%%%%%%%%%%%%%%%%%%
Hossein's Coding Journals
%%%%%%%%%%%%%%%%%%%%%%

@ARTICLE{myoptics,
	AUTHOR =       "H. Pishro-Nik and N. Rahnavard and J. Ha and F. Fekri and A. Adibi ",
	TITLE =        "Low-density parity-check codes for volume holographic memory systems",
	JOURNAL =      " Appl. Opt.",
	YEAR =         "2003",
	volume =       "42",
	pages =        "861-870  "
}






@ARTICLE{myit,
	AUTHOR =       "H. Pishro-Nik and F. Fekri  ",
	TITLE =        "On Decoding of Low-Density Parity-Check Codes on the Binary Erasure Channel",
	JOURNAL =      "IEEE Trans. Inform. Theory",
	YEAR =         "2004",
	volume =       "50",
	pages =        "439--454"
}




@ARTICLE{myitpuncture,
	AUTHOR =       "H. Pishro-Nik and F. Fekri  ",
	TITLE =        "Results on Punctured Low-Density Parity-Check Codes and Improved Iterative Decoding Techniques",
	JOURNAL =      "IEEE Trans. on Inform. Theory",
	YEAR =         "2007",
	volume =       "53",
	number=        "2",
	pages =        "599--614",
	month= "February"
}




@ARTICLE{myitlinmimdist,
	AUTHOR =       "H. Pishro-Nik and F. Fekri",
	TITLE =        "Performance of Low-Density Parity-Check Codes With Linear Minimum Distance",
	JOURNAL =         "IEEE Trans. Inform. Theory ",
	YEAR =         "2006",
	volume =       "52",
	number="1",
	pages =        "292 --300"
}






@ARTICLE{myitnonuni,
	AUTHOR =       "H. Pishro-Nik and N. Rahnavard and F. Fekri  ",
	TITLE =        "Non-uniform Error Correction Using Low-Density Parity-Check Codes",
	JOURNAL =      "IEEE Trans. Inform. Theory",
	YEAR =         "2005",
	volume =       "51",
	number=  "7",
	pages =        "2702--2714"
}





@article{eslamitcomhybrid10,
	author = {A. Eslami and S. Vangala and H. Pishro-Nik},
	title = {Hybrid channel codes for highly efficient FSO/RF communication systems},
	journal = {IEEE Transactions on Communications},
	volume = {58},
	number = {10},
	year = {2010},
	pages = {2926--2938},
}


@article{eslamitcompolar13,
	author = {A. Eslami and H. Pishro-Nik},
	title = {On Finite-Length Performance of Polar Codes: Stopping Sets, Error Floor, and Concatenated Design},
	journal = {IEEE Transactions on Communications},
	volume = {61},
	number = {13},
	year = {2013},
	pages = {919--929},
}



@article{saeeditcom11,
	author = {H. Saeedi and H. Pishro-Nik and  A. H. Banihashemi},
	title = {Successive maximization for the systematic design of universally capacity approaching rate-compatible
	sequences of LDPC code ensembles over binary-input output-symmetric memoryless channels},
	journal = {IEEE Transactions on Communications},
	year = {2011},
	volume={59},
	number = {7}
}


@article{rahnavard07,
	author = {Rahnavard, N. and Pishro-Nik, H. and Fekri, F.},
	title = {Unequal Error Protection Using Partially Regular LDPC Codes},
	journal = {IEEE Transactions on Communications},
	year = {2007},
	volume = {55},
	number = {3},
	pages = {387 -- 391}
}


@article{hosseinira04,
	author = {H. Pishro-Nik and F. Fekri},
	title = {Irregular repeat-accumulate codes for volume holographic memory systems},
	journal = {Journal of Applied Optics},
	year = {2004},
	volume = {43},
	number = {27},
	pages = {5222--5227},
}


@article{azadeh2015Ephemeralkey,
	author = {A. Sheikholeslami and D. Goeckel and H. Pishro-Nik},
	title = {Jamming Based on an Ephemeral Key to Obtain Everlasting Security in Wireless Environments},
	journal = {IEEE Transactions on Wireless Communications},
	year = {2015},
	volume = {14},
	number = {11},
	pages = {6072--6081},
}


@article{azadeh2014Everlasting,
	author = {A. Sheikholeslami and D. Goeckel and H. Pishro-Nik},
	title = {Everlasting secrecy in disadvantaged wireless environments against sophisticated eavesdroppers},
	journal = {48th Asilomar Conference on Signals, Systems and Computers},
	year = {2014},
	pages = {1994--1998},
}


@article{azadeh2013ISIT,
	author = {A. Sheikholeslami and D. Goeckel and H. Pishro-Nik},
	title = {Artificial intersymbol interference (ISI) to exploit receiver imperfections for secrecy},
	journal = {IEEE International Symposium on Information Theory (ISIT)},
	year = {2013},
}


@article{azadeh2013Jsac,
	author = {A. Sheikholeslami and D. Goeckel and H. Pishro-Nik},
	title = {Jamming Based on an Ephemeral Key to Obtain Everlasting Security in Wireless Environments},
	journal = {IEEE Journal on Selected Areas in Communications},
	year = {2013},
	volume = {31},
	number = {9},
	pages = {1828--1839},
}


@article{azadeh2012Allerton,
	author = {A. Sheikholeslami and D. Goeckel and H. Pishro-Nik},
	title = {Exploiting the non-commutativity of nonlinear operators for information-theoretic security in disadvantaged wireless environments},
	journal = {50th Annual Allerton Conference on Communication, Control, and Computing},
	year = {2012},
	pages = {233--240},
}


@article{azadeh2012Infocom,
	author = {A. Sheikholeslami and D. Goeckel and H. Pishro-Nik},
	title = {Jamming Based on an Ephemeral Key to Obtain Everlasting Security in Wireless Environments},
	journal = {IEEE INFOCOM},
	year = {2012},
	pages = {1179--1187},
}

@article{1corser2016evaluating,
	title={Evaluating Location Privacy in Vehicular Communications and Applications},
	author={Corser, George P and Fu, Huirong and Banihani, Abdelnasser},
	journal={IEEE Transactions on Intelligent Transportation Systems},
	volume={17},
	number={9},
	pages={2658-2667},
	year={2016},
	publisher={IEEE}
}
@article{2zhang2016designing,
	title={On Designing Satisfaction-Ratio-Aware Truthful Incentive Mechanisms for k-Anonymity Location Privacy},
	author={Zhang, Yuan and Tong, Wei and Zhong, Sheng},
	journal={IEEE Transactions on Information Forensics and Security},
	volume={11},
	number={11},
	pages={2528--2541},
	year={2016},
	publisher={IEEE}
}
@article{3li2016privacy,
	title={Privacy-preserving Location Proof for Securing Large-scale Database-driven Cognitive Radio Networks},
	author={Li, Yi and Zhou, Lu and Zhu, Haojin and Sun, Limin},
	journal={IEEE Internet of Things Journal},
	volume={3},
	number={4},
	pages={563-571},
	year={2016},
	publisher={IEEE}
}
@article{4olteanu2016quantifying,
	title={Quantifying Interdependent Privacy Risks with Location Data},
	author={Olteanu, Alexandra-Mihaela and Huguenin, K{\'e}vin and Shokri, Reza and Humbert, Mathias and Hubaux, Jean-Pierre},
	journal={IEEE Transactions on Mobile Computing},
	year={2016},
	volume={PP},
	number={99},
	pages={1-1},
	publisher={IEEE}
}
@article{5yi2016practical,
	title={Practical Approximate k Nearest Neighbor Queries with Location and Query Privacy},
	author={Yi, Xun and Paulet, Russell and Bertino, Elisa and Varadharajan, Vijay},
	journal={IEEE Transactions on Knowledge and Data Engineering},
	volume={28},
	number={6},
	pages={1546--1559},
	year={2016},
	publisher={IEEE}
}
@article{6li2016privacy,
	title={Privacy Leakage of Location Sharing in Mobile Social Networks: Attacks and Defense},
	author={Li, Huaxin and Zhu, Haojin and Du, Suguo and Liang, Xiaohui and Shen, Xuemin},
	journal={IEEE Transactions on Dependable and Secure Computing},
	year={2016},
	volume={PP},
	number={99},
	publisher={IEEE}
}

@article{7murakami2016localization,
	title={Localization Attacks Using Matrix and Tensor Factorization},
	author={Murakami, Takao and Watanabe, Hajime},
	journal={IEEE Transactions on Information Forensics and Security},
	volume={11},
	number={8},
	pages={1647--1660},
	year={2016},
	publisher={IEEE}
}
@article{8zurbaran2015near,
	title={Near-Rand: Noise-based Location Obfuscation Based on Random Neighboring Points},
	author={Zurbaran, Mayra Alejandra and Avila, Karen and Wightman, Pedro and Fernandez, Michael},
	journal={IEEE Latin America Transactions},
	volume={13},
	number={11},
	pages={3661--3667},
	year={2015},
	publisher={IEEE}
}

@article{9tan2014anti,
	title={An anti-tracking source-location privacy protection protocol in wsns based on path extension},
	author={Tan, Wei and Xu, Ke and Wang, Dan},
	journal={IEEE Internet of Things Journal},
	volume={1},
	number={5},
	pages={461--471},
	year={2014},
	publisher={IEEE}
}

@article{10peng2014enhanced,
	title={Enhanced Location Privacy Preserving Scheme in Location-Based Services},
	author={Peng, Tao and Liu, Qin and Wang, Guojun},
	journal={IEEE Systems Journal},
	year={2014},
	volume={PP},
	number={99},
	pages={1-12},
	publisher={IEEE}
}
@article{11dewri2014exploiting,
	title={Exploiting service similarity for privacy in location-based search queries},
	author={Dewri, Rinku and Thurimella, Ramakrisha},
	journal={IEEE Transactions on Parallel and Distributed Systems},
	volume={25},
	number={2},
	pages={374--383},
	year={2014},
	publisher={IEEE}
}

@article{12hwang2014novel,
	title={A novel time-obfuscated algorithm for trajectory privacy protection},
	author={Hwang, Ren-Hung and Hsueh, Yu-Ling and Chung, Hao-Wei},
	journal={IEEE Transactions on Services Computing},
	volume={7},
	number={2},
	pages={126--139},
	year={2014},
	publisher={IEEE}
}
@article{13puttaswamy2014preserving,
	title={Preserving location privacy in geosocial applications},
	author={Puttaswamy, Krishna PN and Wang, Shiyuan and Steinbauer, Troy and Agrawal, Divyakant and El Abbadi, Amr and Kruegel, Christopher and Zhao, Ben Y},
	journal={IEEE Transactions on Mobile Computing},
	volume={13},
	number={1},
	pages={159--173},
	year={2014},
	publisher={IEEE}
}

@article{14zhang2014privacy,
	title={Privacy quantification model based on the Bayes conditional risk in Location-Based Services},
	author={Zhang, Xuejun and Gui, Xiaolin and Tian, Feng and Yu, Si and An, Jian},
	journal={Tsinghua Science and Technology},
	volume={19},
	number={5},
	pages={452--462},
	year={2014},
	publisher={TUP}
}

@article{15bilogrevic2014privacy,
	title={Privacy-preserving optimal meeting location determination on mobile devices},
	author={Bilogrevic, Igor and Jadliwala, Murtuza and Joneja, Vishal and Kalkan, K{\"u}bra and Hubaux, Jean-Pierre and Aad, Imad},
	journal={IEEE transactions on information forensics and security},
	volume={9},
	number={7},
	pages={1141--1156},
	year={2014},
	publisher={IEEE}
}
@article{16haghnegahdar2014privacy,
	title={Privacy Risks in Publishing Mobile Device Trajectories},
	author={Haghnegahdar, Alireza and Khabbazian, Majid and Bhargava, Vijay K},
	journal={IEEE Wireless Communications Letters},
	volume={3},
	number={3},
	pages={241--244},
	year={2014},
	publisher={IEEE}
}
@article{17malandrino2014verification,
	title={Verification and inference of positions in vehicular networks through anonymous beaconing},
	author={Malandrino, Francesco and Borgiattino, Carlo and Casetti, Claudio and Chiasserini, Carla-Fabiana and Fiore, Marco and Sadao, Roberto},
	journal={IEEE Transactions on Mobile Computing},
	volume={13},
	number={10},
	pages={2415--2428},
	year={2014},
	publisher={IEEE}
}
@article{18shokri2014hiding,
	title={Hiding in the mobile crowd: Locationprivacy through collaboration},
	author={Shokri, Reza and Theodorakopoulos, George and Papadimitratos, Panos and Kazemi, Ehsan and Hubaux, Jean-Pierre},
	journal={IEEE transactions on dependable and secure computing},
	volume={11},
	number={3},
	pages={266--279},
	year={2014},
	publisher={IEEE}
}
@article{19freudiger2013non,
	title={Non-cooperative location privacy},
	author={Freudiger, Julien and Manshaei, Mohammad Hossein and Hubaux, Jean-Pierre and Parkes, David C},
	journal={IEEE Transactions on Dependable and Secure Computing},
	volume={10},
	number={2},
	pages={84--98},
	year={2013},
	publisher={IEEE}
}
@article{20gao2013trpf,
	title={TrPF: A trajectory privacy-preserving framework for participatory sensing},
	author={Gao, Sheng and Ma, Jianfeng and Shi, Weisong and Zhan, Guoxing and Sun, Cong},
	journal={IEEE Transactions on Information Forensics and Security},
	volume={8},
	number={6},
	pages={874--887},
	year={2013},
	publisher={IEEE}
}
@article{21ma2013privacy,
	title={Privacy vulnerability of published anonymous mobility traces},
	author={Ma, Chris YT and Yau, David KY and Yip, Nung Kwan and Rao, Nageswara SV},
	journal={IEEE/ACM Transactions on Networking},
	volume={21},
	number={3},
	pages={720--733},
	year={2013},
	publisher={IEEE}
}
@article{22niu2013pseudo,
	title={Pseudo-Location Updating System for privacy-preserving location-based services},
	author={Niu, Ben and Zhu, Xiaoyan and Chi, Haotian and Li, Hui},
	journal={China Communications},
	volume={10},
	number={9},
	pages={1--12},
	year={2013},
	publisher={IEEE}
}
@article{23dewri2013local,
	title={Local differential perturbations: Location privacy under approximate knowledge attackers},
	author={Dewri, Rinku},
	journal={IEEE Transactions on Mobile Computing},
	volume={12},
	number={12},
	pages={2360--2372},
	year={2013},
	publisher={IEEE}
}
@inproceedings{24kanoria2012tractable,
	title={Tractable bayesian social learning on trees},
	author={Kanoria, Yashodhan and Tamuz, Omer},
	booktitle={Information Theory Proceedings (ISIT), 2012 IEEE International Symposium on},
	pages={2721--2725},
	year={2012},
	organization={IEEE}
}
@inproceedings{25farias2005universal,
	title={A universal scheme for learning},
	author={Farias, Vivek F and Moallemi, Ciamac C and Van Roy, Benjamin and Weissman, Tsachy},
	booktitle={Proceedings. International Symposium on Information Theory, 2005. ISIT 2005.},
	pages={1158--1162},
	year={2005},
	organization={IEEE}
}
@inproceedings{26misra2013unsupervised,
	title={Unsupervised learning and universal communication},
	author={Misra, Vinith and Weissman, Tsachy},
	booktitle={Information Theory Proceedings (ISIT), 2013 IEEE International Symposium on},
	pages={261--265},
	year={2013},
	organization={IEEE}
}
@inproceedings{27ryabko2013time,
	title={Time-series information and learning},
	author={Ryabko, Daniil},
	booktitle={Information Theory Proceedings (ISIT), 2013 IEEE International Symposium on},
	pages={1392--1395},
	year={2013},
	organization={IEEE}
}
@inproceedings{28krzakala2013phase,
	title={Phase diagram and approximate message passing for blind calibration and dictionary learning},
	author={Krzakala, Florent and M{\'e}zard, Marc and Zdeborov{\'a}, Lenka},
	booktitle={Information Theory Proceedings (ISIT), 2013 IEEE International Symposium on},
	pages={659--663},
	year={2013},
	organization={IEEE}
}
@inproceedings{29sakata2013sample,
	title={Sample complexity of Bayesian optimal dictionary learning},
	author={Sakata, Ayaka and Kabashima, Yoshiyuki},
	booktitle={Information Theory Proceedings (ISIT), 2013 IEEE International Symposium on},
	pages={669--673},
	year={2013},
	organization={IEEE}
}
@inproceedings{30predd2004consistency,
	title={Consistency in a model for distributed learning with specialists},
	author={Predd, Joel B and Kulkarni, Sanjeev R and Poor, H Vincent},
	booktitle={IEEE International Symposium on Information Theory},
	year={2004},
	organization={IEEE}
}
@inproceedings{31nokleby2016rate,
	title={Rate-Distortion Bounds on Bayes Risk in Supervised Learning},
	author={Nokleby, Matthew and Beirami, Ahmad and Calderbank, Robert},
	booktitle={2016 IEEE International Symposium on Information Theory (ISIT)},
	pages={2099-2103},
	year={2016},
	organization={IEEE}
}

@inproceedings{32le2016imperfect,
	title={Are imperfect reviews helpful in social learning?},
	author={Le, Tho Ngoc and Subramanian, Vijay G and Berry, Randall A},
	booktitle={Information Theory (ISIT), 2016 IEEE International Symposium on},
	pages={2089--2093},
	year={2016},
	organization={IEEE}
}
@inproceedings{33gadde2016active,
	title={Active Learning for Community Detection in Stochastic Block Models},
	author={Gadde, Akshay and Gad, Eyal En and Avestimehr, Salman and Ortega, Antonio},
	booktitle={2016 IEEE International Symposium on Information Theory (ISIT)},
	pages={1889-1893},
	year={2016}
}
@inproceedings{34shakeri2016minimax,
	title={Minimax Lower Bounds for Kronecker-Structured Dictionary Learning},
	author={Shakeri, Zahra and Bajwa, Waheed U and Sarwate, Anand D},
	booktitle={2016 IEEE International Symposium on Information Theory (ISIT)},
	pages={1148-1152},
	year={2016}
}
@article{35lee2015speeding,
	title={Speeding up distributed machine learning using codes},
	author={Lee, Kangwook and Lam, Maximilian and Pedarsani, Ramtin and Papailiopoulos, Dimitris and Ramchandran, Kannan},
	booktitle={2016 IEEE International Symposium on Information Theory (ISIT)},
	pages={1143-1147},
	year={2016}
}
@article{36oneto2016statistical,
	title={Statistical Learning Theory and ELM for Big Social Data Analysis},
	author={Oneto, Luca and Bisio, Federica and Cambria, Erik and Anguita, Davide},
	journal={ieee CompUTATionAl inTelliGenCe mAGAzine},
	volume={11},
	number={3},
	pages={45--55},
	year={2016},
	publisher={IEEE}
}
@article{37lin2015probabilistic,
	title={Probabilistic approach to modeling and parameter learning of indirect drive robots from incomplete data},
	author={Lin, Chung-Yen and Tomizuka, Masayoshi},
	journal={IEEE/ASME Transactions on Mechatronics},
	volume={20},
	number={3},
	pages={1036--1045},
	year={2015},
	publisher={IEEE}
}
@article{38wang2016towards,
	title={Towards Bayesian Deep Learning: A Framework and Some Existing Methods},
	author={Wang, Hao and Yeung, Dit-Yan},
	journal={IEEE Transactions on Knowledge and Data Engineering},
	volume={PP},
	number={99},
	year={2016},
	publisher={IEEE}
}


%%%%%Informationtheoreticsecurity%%%%%%%%%%%%%%%%%%%%%%%




@inproceedings{Bloch2011PhysicalSecBook,
	title={Physical-Layer Security},
	author={M. Bloch and J. Barros},
	organization={Cambridge University Press},
	year={2011}
}



@inproceedings{Liang2009InfoSecBook,
	title={Information Theoretic Security},
	author={Y. Liang and H. V. Poor and S. Shamai (Shitz)},
	organization={Now Publishers Inc.},
	year={2009}
}


@inproceedings{Zhou2013PhysicalSecBook,
	title={Physical Layer Security in Wireless Communications},
	author={ X. Zhou and L. Song and Y. Zhang},
	organization={CRC Press},
	year={2013}
}

@article{Ni2012IEA,
	Author = {D. Ni and H. Liu and W. Ding and  Y. Xie and H. Wang and H. Pishro-Nik and Q. Yu},
	Journal = {IEA/AIE},
	Title = {Cyber-Physical Integration to Connect Vehicles for Transformed Transportation Safety and Efficiency},
	Year = {2012}}



@inproceedings{Ni2012Inproceedings,
	Author = {D. Ni, H. Liu, Y. Xie, W. Ding, H. Wang, H. Pishro-Nik, Q. Yu and M. Ferreira},
	Booktitle = {Spring Simulation Multiconference},
	Date-Added = {2016-09-04 14:18:42 +0000},
	Date-Modified = {2016-09-06 16:22:14 +0000},
	Title = {Virtual Lab of Connected Vehicle Technology},
	Year = {2012}}

@inproceedings{Ni2012Inproceedings,
	Author = {D. Ni, H. Liu, W. Ding, Y. Xie, H. Wang, H. Pishro-Nik and Q. Yu,},
	Booktitle = {IEA/AIE},
	Date-Added = {2016-09-04 09:11:02 +0000},
	Date-Modified = {2016-09-06 14:46:53 +0000},
	Title = {Cyber-Physical Integration to Connect Vehicles for Transformed Transportation Safety and Efficiency},
	Year = {2012}}


@article{Nekoui_IJIPT_2009,
	Author = {M. Nekoui and D. Ni and H. Pishro-Nik and R. Prasad and M. Kanjee and H. Zhu and T. Nguyen},
	Journal = {International Journal of Internet Protocol Technology (IJIPT)},
	Number = {3},
	Pages = {},
	Publisher = {},
	Title = {Development of a VII-Enabled Prototype Intersection Collision Warning System},
	Volume = {4},
	Year = {2009}}


@inproceedings{Pishro_Ganz_Ni,
	Author = {H. Pishro-Nik, A. Ganz, and Daiheng Ni},
	Booktitle = {Forty-Fifth Annual Allerton Conference on Communication, Control, and Computing. Allerton House, Monticello, IL},
	Date-Added = {},
	Date-Modified = {},
	Number = {},
	Pages = {},
	Title = {The capacity of vehicular ad hoc networks},
	Volume = {},
	Year = {September 26-28, 2007}}

@inproceedings{Leow_Pishro_Ni_1,
	Author = {W. L. Leow, H. Pishro-Nik and Daiheng Ni},
	Booktitle = {IEEE Global Telecommunications Conference, Washington, D.C.},
	Date-Added = {},
	Date-Modified = {},
	Number = {},
	Pages = {},
	Title = {Delay and Energy Tradeoff in Multi-state Wireless Sensor Networks},
	Volume = {},
	Year = {November 26-30, 2007}}


@misc{UMass-Trans,
	title = {{UMass Transportation Center}},
	note = {\url{http://www.umasstransportationcenter.org/}},
}


@inproceedings{Haenggi2013book,
	title={Stochastic geometry for wireless networks},
	author={M. Haenggi},
	organization={Cambridge Uinversity Press},
	year={2013}
}


%% This BibTeX bibliography file was created using BibDesk.
%% http://bibdesk.sourceforge.net/

%% Created for Zarrin Montazeri at 2015-11-09 18:45:31 -0500


%% Saved with string encoding Unicode (UTF-8)


%%%%%%%%%%%%%%%%personalization%%%%%%%%%%%%%%%%%%%%%%%%%%%%%%%%%%

@article{osma2015,
	title={Impact of Time-to-Collision Information on Driving Behavior in Connected Vehicle Environments Using A Driving Simulator Test Bed},
	journal{Journal of Traffic and Logistics Engineering},
	author={Osama A. Osman, Julius Codjoe, and Sherif Ishak},
	volume={3},
	number={1},
	 pages={18--24},
	year={2015}
}


@article{charisma2010,
	title={Dynamic Latent Plan Models},
	author={Charisma F. Choudhurya, Moshe Ben-Akivab and Maya Abou-Zeid},
	journal={Journal of Choice Modelling},
	volume={3},
	number={2},
	pages={50--70},
	year={2010},
	publisher={Elsvier}
}


@misc{noble2014,
	author = {A. M. Noble, Shane B. McLaughlin, Zachary R. Doerzaph and Thomas A. Dingus},
	title = {Crowd-sourced Connected-vehicle Warning Algorithm using Naturalistic Driving Data},
	howpublished = {Downloaded from \url{http://hdl.handle.net/10919/53978}},

	month = August,
	year = 2014
}


@phdthesis{charisma2007,
	title    = {Modeling Driving Decisions with Latent Plans},
	school   = {Massachusetts Institute of Technology },
	author   = {Charisma Farheen Choudhury},
	year     = {2007}, %other attributes omitted
}


@article{chrysler2015,
	title={Cost of Warning of Unseen Threats:Unintended Consequences of Connected Vehicle Alerts},
	author={S. T. Chrysler, J. M. Cooper and D. C. Marshall},
	journal={Transportation Research Record: Journal of the Transportation Research Board},
	volume={2518},
	pages={79--85},
	year={2015},
}

@misc{nsf_cps,
        title = {Cyber-Physical Systems (CPS) PROGRAM SOLICITATION NSF 17-529},
        howpublished = {Downloaded from \url{https://www.nsf.gov/publications/pub_summ.jsp?WT.z_pims_id=503286&ods_key=nsf17529}},
}



%%%%%%%%%%%%%%IOT%%%%%%%%%%%%%%%%%%%%%%%%%%%%%%%%%%%%%%%%%%%%%%%%%%%




@article{FTC2015,
  title={Internet of Things: Privacy and Security in a Connected World},
  author={FTC Staff Report},
  year={2015}
}



@article{0Quest2016,
  title={The Quest for Privacy in the Internet of Things},
  author={ Pawani Porambag and Mika Ylianttila and Corinna Schmitt and  Pardeep Kumar and  Andrei Gurtov and  Athanasios V. Vasilakos},
  journal={IEEE Cloud Computing},
  volum={3},
  number={2},
  year={2016},
  publisher={IEEE}
}

%% Saved with string encoding Unicode (UTF-8)
@inproceedings{1zhou2014security,
  title={Security/privacy of wearable fitness tracking {I}o{T} devices},
  author={Zhou, Wei and Piramuthu, Selwyn},
  booktitle={Information Systems and Technologies (CISTI), 2014 9th Iberian Conference on},
  pages={1--5},
  year={2014},
  organization={IEEE}
}


@inproceedings{3ukil2014iot,
  title={{I}o{T}-privacy: To be private or not to be private},
  author={Ukil, Arijit and Bandyopadhyay, Soma and Pal, Arpan},
  booktitle={Computer Communications Workshops (INFOCOM WKSHPS), IEEE Conference on},
  pages={123--124},
  year={2014},
  organization={IEEE}
}


@article{4arias2015privacy,
  title={Privacy and security in internet of things and wearable devices},
  author={Arias, Orlando and Wurm, Jacob and Hoang, Khoa and Jin, Yier},
  journal={IEEE Transactions on Multi-Scale Computing Systems},
  volume={1},
  number={2},
  pages={99--109},
  year={2015},
  publisher={IEEE}
}
@inproceedings{5ullah2016novel,
  title={A novel model for preserving Location Privacy in Internet of Things},
  author={Ullah, Ikram and Shah, Munam Ali},
  booktitle={Automation and Computing (ICAC), 2016 22nd International Conference on},
  pages={542--547},
  year={2016},
  organization={IEEE}
}
@inproceedings{6sathishkumar2016enhanced,
  title={Enhanced location privacy algorithm for wireless sensor network in Internet of Things},
  author={Sathishkumar, J and Patel, Dhiren R},
  booktitle={Internet of Things and Applications (IOTA), International Conference on},
  pages={208--212},
  year={2016},
  organization={IEEE}
}
@inproceedings{7zhou2012preserving,
  title={Preserving sensor location privacy in internet of things},
  author={Zhou, Liming and Wen, Qiaoyan and Zhang, Hua},
  booktitle={Computational and Information Sciences (ICCIS), 2012 Fourth International Conference on},
  pages={856--859},
  year={2012},
  organization={IEEE}
}

@inproceedings{8ukil2015privacy,
  title={Privacy for {I}o{T}: Involuntary privacy enablement for smart energy systems},
  author={Ukil, Arijit and Bandyopadhyay, Soma and Pal, Arpan},
  booktitle={Communications (ICC), 2015 IEEE International Conference on},
  pages={536--541},
  year={2015},
  organization={IEEE}
}

@inproceedings{9dalipi2016security,
  title={Security and Privacy Considerations for {I}o{T} Application on Smart Grids: Survey and Research Challenges},
  author={Dalipi, Fisnik and Yayilgan, Sule Yildirim},
  booktitle={Future Internet of Things and Cloud Workshops (FiCloudW), IEEE International Conference on},
  pages={63--68},
  year={2016},
  organization={IEEE}
}
@inproceedings{10harris2016security,
  title={Security and Privacy in Public {I}o{T} Spaces},
  author={Harris, Albert F and Sundaram, Hari and Kravets, Robin},
  booktitle={Computer Communication and Networks (ICCCN), 2016 25th International Conference on},
  pages={1--8},
  year={2016},
  organization={IEEE}
}

@inproceedings{11al2015security,
  title={Security and privacy framework for ubiquitous healthcare {I}o{T} devices},
  author={Al Alkeem, Ebrahim and Yeun, Chan Yeob and Zemerly, M Jamal},
  booktitle={Internet Technology and Secured Transactions (ICITST), 2015 10th International Conference for},
  pages={70--75},
  year={2015},
  organization={IEEE}
}
@inproceedings{12sivaraman2015network,
  title={Network-level security and privacy control for smart-home {I}o{T} devices},
  author={Sivaraman, Vijay and Gharakheili, Hassan Habibi and Vishwanath, Arun and Boreli, Roksana and Mehani, Olivier},
  booktitle={Wireless and Mobile Computing, Networking and Communications (WiMob), 2015 IEEE 11th International Conference on},
  pages={163--167},
  year={2015},
  organization={IEEE}
}

@inproceedings{13srinivasan2016privacy,
  title={Privacy conscious architecture for improving emergency response in smart cities},
  author={Srinivasan, Ramya and Mohan, Apurva and Srinivasan, Priyanka},
  booktitle={Smart City Security and Privacy Workshop (SCSP-W), 2016},
  pages={1--5},
  year={2016},
  organization={IEEE}
}
@inproceedings{14sadeghi2015security,
  title={Security and privacy challenges in industrial internet of things},
  author={Sadeghi, Ahmad-Reza and Wachsmann, Christian and Waidner, Michael},
  booktitle={Design Automation Conference (DAC), 2015 52nd ACM/EDAC/IEEE},
  pages={1--6},
  year={2015},
  organization={IEEE}
}
@inproceedings{15otgonbayar2016toward,
  title={Toward Anonymizing {I}o{T} Data Streams via Partitioning},
  author={Otgonbayar, Ankhbayar and Pervez, Zeeshan and Dahal, Keshav},
  booktitle={Mobile Ad Hoc and Sensor Systems (MASS), 2016 IEEE 13th International Conference on},
  pages={331--336},
  year={2016},
  organization={IEEE}
}
@inproceedings{16rutledge2016privacy,
  title={Privacy Impacts of {I}o{T} Devices: A SmartTV Case Study},
  author={Rutledge, Richard L and Massey, Aaron K and Ant{\'o}n, Annie I},
  booktitle={Requirements Engineering Conference Workshops (REW), IEEE International},
  pages={261--270},
  year={2016},
  organization={IEEE}
}

@inproceedings{17andrea2015internet,
  title={Internet of Things: Security vulnerabilities and challenges},
  author={Andrea, Ioannis and Chrysostomou, Chrysostomos and Hadjichristofi, George},
  booktitle={Computers and Communication (ISCC), 2015 IEEE Symposium on},
  pages={180--187},
  year={2015},
  organization={IEEE}
}






























%%%%%%%%%%%%%%%%%%%%%%%%%%%%%%%%%%%%%%%%%%%%%%%%%%%%%%%%%%%


@misc{epfl-mobility-20090224,
    author = {Michal Piorkowski and Natasa Sarafijanovic-Djukic and Matthias Grossglauser},
    title = {{CRAWDAD} dataset epfl/mobility (v. 2009-02-24)},
    howpublished = {Downloaded from \url{http://crawdad.org/epfl/mobility/20090224}},
    doi = {10.15783/C7J010},
    month = feb,
    year = 2009
}

@misc{roma-taxi-20140717,
    author = {Lorenzo Bracciale and Marco Bonola and Pierpaolo Loreti and Giuseppe Bianchi and Raul Amici and Antonello Rabuffi},
    title = {{CRAWDAD} dataset roma/taxi (v. 2014-07-17)},
    howpublished = {Downloaded from \url{http://crawdad.org/roma/taxi/20140717}},
    doi = {10.15783/C7QC7M},
    month = jul,
    year = 2014
}

@misc{rice-ad_hoc_city-20030911,
    author = {Jorjeta G. Jetcheva and Yih-Chun Hu and Santashil PalChaudhuri and Amit Kumar Saha and David B. Johnson},
    title = {{CRAWDAD} dataset rice/ad\_hoc\_city (v. 2003-09-11)},
    howpublished = {Downloaded from \url{http://crawdad.org/rice/ad_hoc_city/20030911}},
    doi = {10.15783/C73K5B},
    month = sep,
    year = 2003
}

@misc{china:2012,
author = {Microsoft Research Asia},
title = {GeoLife GPS Trajectories},
year = {2012},
howpublished= {\url{https://www.microsoft.com/en-us/download/details.aspx?id=52367}},
}


@misc{china:2011,
ALTauthor = {Microsoft Research Asia)},
ALTeditor = {},
title = {GeoLife GPS Trajectories,
year = {2012},
url = {https://www.microsoft.com/en-us/download/details.aspx?id=52367},
}


@misc{longversion,
  author = {N. Takbiri and A. Houmansadr and D.L. Goeckel and H. Pishro-Nik},
  title = {{Limits of Location Privacy under Anonymization and Obfuscation}},
  howpublished = "\url{http://www.ecs.umass.edu/ece/pishro/Papers/ISIT_2017-2.pdf}",
  year = 2017,
month= "January",
  note = "Summarized version submitted to IEEE ISIT 2017"
}

@misc{isit_ke,
  author = {K. Li and D. Goeckel and H. Pishro-Nik},
  title = {{Bayesian Time Series Matching and Privacy}},
  note = "submitted to IEEE ISIT 2017"
}

@article{matching,
  title={Asymptotically Optimal Matching of Multiple Sequences to Source Distributions and Training Sequences},
  author={Jayakrishnan Unnikrishnan},
  journal={ IEEE Transactions on Information Theory},
  volume={61},
  number={1},
  pages={452-468},
  year={2015},
  publisher={IEEE}
}


@article{Naini2016,
	Author = {F. Naini and J. Unnikrishnan and P. Thiran and M. Vetterli},
	Journal = {IEEE Transactions on Information Forensics and Security},
	Publisher = {IEEE},
	Title = {Where You Are Is Who You Are: User Identification by Matching Statistics},
	 volume={11},
    number={2},
     pages={358--372},
    Year = {2016}
}



@inproceedings{holowczak2015cachebrowser,
  title={{CacheBrowser: Bypassing Chinese Censorship without Proxies Using Cached Content}},
  author={Holowczak, John and Houmansadr, Amir},
  booktitle={Proceedings of the 22nd ACM SIGSAC Conference on Computer and Communications Security},
  pages={70--83},
  year={2015},
  organization={ACM}
}
@misc{cb-website,
	Howpublished = {\url{https://cachebrowser.net/#/}},
	Title = {{CacheBrowser}},
	key={cachebrowser}
}

@inproceedings{GameOfDecoys,
 title={{GAME OF DECOYS: Optimal Decoy Routing Through Game Theory}},
 author={Milad Nasr and Amir Houmansadr},
 booktitle={The $23^{rd}$ ACM Conference on Computer and Communications Security (CCS)},
 year={2016}
}

@inproceedings{CDNReaper,
 title={{Practical Censorship Evasion Leveraging Content Delivery Networks}},
 author={Hadi Zolfaghari and Amir Houmansadr},
 booktitle={The $23^{rd}$ ACM Conference on Computer and Communications Security (CCS)},
 year={2016}
}

@misc{Leberknight2010,
	Author = {Leberknight, C. and Chiang, M. and Poor, H. and Wong, F.},
	Howpublished = {\url{http://www.princeton.edu/~chiangm/anticensorship.pdf}},
	Title = {{A Taxonomy of Internet Censorship and Anti-censorship}},
	Year = {2010}}

@techreport{ultrasurf-analysis,
	Author = {Appelbaum, Jacob},
	Institution = {The Tor Project},
	Title = {{Technical analysis of the Ultrasurf proxying software}},
	Url = {http://scholar.google.com/scholar?hl=en\&btnG=Search\&q=intitle:Technical+analysis+of+the+Ultrasurf+proxying+software\#0},
	Year = {2012},
	Bdsk-Url-1 = {http://scholar.google.com/scholar?hl=en%5C&btnG=Search%5C&q=intitle:Technical+analysis+of+the+Ultrasurf+proxying+software%5C#0}}

@misc{gifc:07,
	Howpublished = {\url{http://www.internetfreedom.org/archive/Defeat\_Internet\_Censorship\_White\_Paper.pdf}},
	Key = {defeatcensorship},
	Publisher = {Global Internet Freedom Consortium (GIFC)},
	Title = {{Defeat Internet Censorship: Overview of Advanced Technologies and Products}},
	Type = {White Paper},
	Year = {2007}}

@article{pan2011survey,
	Author = {Pan, J. and Paul, S. and Jain, R.},
	Journal = {Communications Magazine, IEEE},
	Number = {7},
	Pages = {26--36},
	Publisher = {IEEE},
	Title = {{A Survey of the Research on Future Internet Architectures}},
	Volume = {49},
	Year = {2011}}

@misc{nsf-fia,
	Howpublished = {\url{http://www.nets-fia.net/}},
	Key = {FIA},
	Title = {{NSF Future Internet Architecture Project}}}

@misc{NDN,
	Howpublished = {\url{http://www.named- data.net}},
	Key = {NDN},
	Title = {{Named Data Networking Project}}}

@inproceedings{MobilityFirst,
	Author = {Seskar, I. and Nagaraja, K. and Nelson, S. and Raychaudhuri, D.},
	Booktitle = {Asian Internet Engineering Conference},
	Title = {{Mobilityfirst Future internet Architecture Project}},
	Year = {2011}}

@incollection{NEBULA,
	Author = {Anderson, T. and Birman, K. and Broberg, R. and Caesar, M. and Comer, D. and Cotton, C. and Freedman, M.~J. and Haeberlen, A. and Ives, Z.~G. and Krishnamurthy, A. and others},
	Booktitle = {The Future Internet},
	Pages = {16--26},
	Publisher = {Springer},
	Title = {{The NEBULA Future Internet Architecture}},
	Year = {2013}}

@inproceedings{XIA,
	Author = {Anand, A. and Dogar, F. and Han, D. and Li, B. and Lim, H. and Machado, M. and Wu, W. and Akella, A. and Andersen, D.~G. and Byers, J.~W. and others},
	Booktitle = {ACM Workshop on Hot Topics in Networks},
	Pages = {2},
	Title = {{XIA: An Architecture for an Evolvable and Trustworthy Internet}},
	Year = {2011}}

@inproceedings{ChoiceNet,
	Author = {Rouskas, G.~N. and Baldine, I. and Calvert, K.~L. and Dutta, R. and Griffioen, J. and Nagurney, A. and Wolf, T.},
	Booktitle = {ONDM},
	Title = {{ChoiceNet: Network Innovation Through Choice}},
	Year = {2013}}

@misc{nsf-find,
	Howpublished = {http://www.nets-find.net/},
	Title = {{NSF NeTS FIND Initiative}}}

@article{traid,
	Author = {Cheriton, D.~R. and Gritter, M.},
	Title = {{TRIAD: A New Next-Generation Internet Architecture}},
	Year = {2000}}

@inproceedings{dona,
	Author = {Koponen, T. and Chawla, M. and Chun, B-G. and Ermolinskiy, A. and Kim, K.~H. and Shenker, S. and Stoica, I.},
	Booktitle = {ACM SIGCOMM Computer Communication Review},
	Number = {4},
	Organization = {ACM},
	Pages = {181--192},
	Title = {{A Data-Oriented (and Beyond) Network Architecture}},
	Volume = {37},
	Year = {2007}}

@misc{ultrasurf,
	Howpublished = {\url{http://www.ultrareach.com}},
	Key = {ultrasurf},
	Title = {{Ultrasurf}}}

@misc{tor-bridge,
	Author = {Dingledine, R. and Mathewson, N.},
	Howpublished = {\url{https://svn.torproject.org/svn/projects/design-paper/blocking.html}},
	Title = {{Design of a Blocking-Resistant Anonymity System}}}

@inproceedings{McLachlanH09,
	Author = {J. McLachlan and N. Hopper},
	Booktitle = {WPES},
	Title = {{On the Risks of Serving Whenever You Surf: Vulnerabilities in Tor's Blocking Resistance Design}},
	Year = {2009}}

@inproceedings{mahdian2010,
	Author = {Mahdian, M.},
	Booktitle = {{Fun with Algorithms}},
	Title = {{Fighting Censorship with Algorithms}},
	Year = {2010}}

@inproceedings{McCoy2011,
	Author = {McCoy, D. and Morales, J.~A. and Levchenko, K.},
	Booktitle = {FC},
	Title = {{Proximax: A Measurement Based System for Proxies Dissemination}},
	Year = {2011}}

@inproceedings{Sovran2008,
	Author = {Sovran, Y. and Libonati, A. and Li, J.},
	Booktitle = {IPTPS},
	Title = {{Pass it on: Social Networks Stymie Censors}},
	Year = {2008}}

@inproceedings{rbridge,
	Author = {Wang, Q. and Lin, Zi and Borisov, N. and Hopper, N.},
	Booktitle = {{NDSS}},
	Title = {{rBridge: User Reputation based Tor Bridge Distribution with Privacy Preservation}},
	Year = {2013}}

@inproceedings{telex,
	Author = {Wustrow, E. and Wolchok, S. and Goldberg, I. and Halderman, J.},
	Booktitle = {{USENIX Security}},
	Title = {{Telex: Anticensorship in the Network Infrastructure}},
	Year = {2011}}

@inproceedings{cirripede,
	Author = {Houmansadr, A. and Nguyen, G. and Caesar, M. and Borisov, N.},
	Booktitle = {CCS},
	Title = {{Cirripede: Circumvention Infrastructure Using Router Redirection with Plausible Deniability}},
	Year = {2011}}

@inproceedings{decoyrouting,
	Author = {Karlin, J. and Ellard, D. and Jackson, A. and Jones, C. and Lauer, G. and Mankins, D. and Strayer, W.},
	Booktitle = {{FOCI}},
	Title = {{Decoy Routing: Toward Unblockable Internet Communication}},
	Year = {2011}}

@inproceedings{routing-around-decoys,
	Author = {M.~Schuchard and J.~Geddes and C.~Thompson and N.~Hopper},
	Booktitle = {{CCS}},
	Title = {{Routing Around Decoys}},
	Year = {2012}}

@inproceedings{parrot,
	Author = {A. Houmansadr and C. Brubaker and V. Shmatikov},
	Booktitle = {IEEE S\&P},
	Title = {{The Parrot is Dead: Observing Unobservable Network Communications}},
	Year = {2013}}

@misc{knock,
	Author = {T. Wilde},
	Howpublished = {\url{https://blog.torproject.org/blog/knock-knock-knockin-bridges-doors}},
	Title = {{Knock Knock Knockin' on Bridges' Doors}},
	Year = {2012}}

@inproceedings{china-tor,
	Author = {Winter, P. and Lindskog, S.},
	Booktitle = {{FOCI}},
	Title = {{How the Great Firewall of China Is Blocking Tor}},
	Year = {2012}}

@misc{discover-bridge,
	Howpublished = {\url{https://blog.torproject.org/blog/research-problems-ten-ways-discover-tor-bridges}},
	Key = {tenways},
	Title = {{Ten Ways to Discover Tor Bridges}}}

@inproceedings{freewave,
	Author = {A.~Houmansadr and T.~Riedl and N.~Borisov and A.~Singer},
	Booktitle = {{NDSS}},
	Title = {{I Want My Voice to Be Heard: IP over Voice-over-IP for Unobservable Censorship Circumvention}},
	Year = 2013}

@inproceedings{censorspoofer,
	Author = {Q. Wang and X. Gong and G. Nguyen and A. Houmansadr and N. Borisov},
	Booktitle = {CCS},
	Title = {{CensorSpoofer: Asymmetric Communication Using IP Spoofing for Censorship-Resistant Web Browsing}},
	Year = {2012}}

@inproceedings{skypemorph,
	Author = {H. Moghaddam and B. Li and M. Derakhshani and I. Goldberg},
	Booktitle = {CCS},
	Title = {{SkypeMorph: Protocol Obfuscation for Tor Bridges}},
	Year = {2012}}

@inproceedings{stegotorus,
	Author = {Weinberg, Z. and Wang, J. and Yegneswaran, V. and Briesemeister, L. and Cheung, S. and Wang, F. and Boneh, D.},
	Booktitle = {CCS},
	Title = {{StegoTorus: A Camouflage Proxy for the Tor Anonymity System}},
	Year = {2012}}

@techreport{dust,
	Author = {{B.~Wiley}},
	Howpublished = {\url{http://blanu.net/ Dust.pdf}},
	Institution = {School of Information, University of Texas at Austin},
	Title = {{Dust: A Blocking-Resistant Internet Transport Protocol}},
	Year = {2011}}

@inproceedings{FTE,
	Author = {K.~Dyer and S.~Coull and T.~Ristenpart and T.~Shrimpton},
	Booktitle = {CCS},
	Title = {{Protocol Misidentification Made Easy with Format-Transforming Encryption}},
	Year = {2013}}

@inproceedings{fp,
	Author = {Fifield, D. and Hardison, N. and Ellithrope, J. and Stark, E. and Dingledine, R. and Boneh, D. and Porras, P.},
	Booktitle = {PETS},
	Title = {{Evading Censorship with Browser-Based Proxies}},
	Year = {2012}}

@misc{obfsproxy,
	Howpublished = {\url{https://www.torproject.org/projects/obfsproxy.html.en}},
	Key = {obfsproxy},
	Publisher = {The Tor Project},
	Title = {{A Simple Obfuscating Proxy}}}

@inproceedings{Tor-instead-of-IP,
	Author = {Liu, V. and Han, S. and Krishnamurthy, A. and Anderson, T.},
	Booktitle = {HotNets},
	Title = {{Tor instead of IP}},
	Year = {2011}}

@misc{roger-slides,
	Howpublished = {\url{https://svn.torproject.org/svn/projects/presentations/slides-28c3.pdf}},
	Key = {torblocking},
	Title = {{How Governments Have Tried to Block Tor}}}

@inproceedings{infranet,
	Author = {Feamster, N. and Balazinska, M. and Harfst, G. and Balakrishnan, H. and Karger, D.},
	Booktitle = {USENIX Security},
	Title = {{Infranet: Circumventing Web Censorship and Surveillance}},
	Year = {2002}}

@inproceedings{collage,
	Author = {S.~Burnett and N.~Feamster and S.~Vempala},
	Booktitle = {USENIX Security},
	Title = {{Chipping Away at Censorship Firewalls with User-Generated Content}},
	Year = {2010}}

@article{anonymizer,
	Author = {Boyan, J.},
	Journal = {Computer-Mediated Communication Magazine},
	Month = sep,
	Number = {9},
	Title = {{The Anonymizer: Protecting User Privacy on the Web}},
	Volume = {4},
	Year = {1997}}

@article{schulze2009internet,
	Author = {Schulze, H. and Mochalski, K.},
	Journal = {IPOQUE Report},
	Pages = {351--362},
	Title = {Internet Study 2008/2009},
	Volume = {37},
	Year = {2009}}

@inproceedings{cya-ccs13,
	Author = {J.~Geddes and M.~Schuchard and N.~Hopper},
	Booktitle = {{CCS}},
	Title = {{Cover Your ACKs: Pitfalls of Covert Channel Censorship Circumvention}},
	Year = {2013}}

@inproceedings{andana,
	Author = {DiBenedetto, S. and Gasti, P. and Tsudik, G. and Uzun, E.},
	Booktitle = {{NDSS}},
	Title = {{ANDaNA: Anonymous Named Data Networking Application}},
	Year = {2012}}

@inproceedings{darkly,
	Author = {Jana, S. and Narayanan, A. and Shmatikov, V.},
	Booktitle = {IEEE S\&P},
	Title = {{A Scanner Darkly: Protecting User Privacy From Perceptual Applications}},
	Year = {2013}}

@inproceedings{NS08,
	Author = {A.~Narayanan and V.~Shmatikov},
	Booktitle = {IEEE S\&P},
	Title = {Robust de-anonymization of large sparse datasets},
	Year = {2008}}

@inproceedings{NS09,
	Author = {A.~Narayanan and V.~Shmatikov},
	Booktitle = {IEEE S\&P},
	Title = {De-anonymizing social networks},
	Year = {2009}}

@inproceedings{memento,
	Author = {Jana, S. and Shmatikov, V.},
	Booktitle = {IEEE S\&P},
	Title = {{Memento: Learning secrets from process footprints}},
	Year = {2012}}

@misc{plugtor,
	Howpublished = {\url{https://www.torproject.org/docs/pluggable-transports.html.en}},
	Key = {PluggableTransports},
	Publisher = {The Tor Project},
	Title = {{Tor: Pluggable transports}}}

@misc{psiphon,
	Author = {J.~Jia and P.~Smith},
	Howpublished = {\url{http://www.cdf.toronto.edu/~csc494h/reports/2004-fall/psiphon_ae.html}},
	Title = {{Psiphon: Analysis and Estimation}},
	Year = 2004}

@misc{china-github,
	Howpublished = {\url{http://mobile.informationweek.com/80269/show/72e30386728f45f56b343ddfd0fdb119/}},
	Key = {github},
	Title = {{China's GitHub Censorship Dilemma}}}

@inproceedings{txbox,
	Author = {Jana, S. and Porter, D. and Shmatikov, V.},
	Booktitle = {IEEE S\&P},
	Title = {{TxBox: Building Secure, Efficient Sandboxes with System Transactions}},
	Year = {2011}}

@inproceedings{airavat,
	Author = {I. Roy and S. Setty and A. Kilzer and V. Shmatikov and E. Witchel},
	Booktitle = {NSDI},
	Title = {{Airavat: Security and Privacy for MapReduce}},
	Year = {2010}}

@inproceedings{osdi12,
	Author = {A. Dunn and M. Lee and S. Jana and S. Kim and M. Silberstein and Y. Xu and V. Shmatikov and E. Witchel},
	Booktitle = {OSDI},
	Title = {{Eternal Sunshine of the Spotless Machine: Protecting Privacy with Ephemeral Channels}},
	Year = {2012}}

@inproceedings{ymal,
	Author = {J. Calandrino and A. Kilzer and A. Narayanan and E. Felten and V. Shmatikov},
	Booktitle = {IEEE S\&P},
	Title = {{``You Might Also Like:'' Privacy Risks of Collaborative Filtering}},
	Year = {2011}}

@inproceedings{srivastava11,
	Author = {V. Srivastava and M. Bond and K. McKinley and V. Shmatikov},
	Booktitle = {PLDI},
	Title = {{A Security Policy Oracle: Detecting Security Holes Using Multiple API Implementations}},
	Year = {2011}}

@inproceedings{chen-oakland10,
	Author = {Chen, S. and Wang, R. and Wang, X. and Zhang, K.},
	Booktitle = {IEEE S\&P},
	Title = {{Side-Channel Leaks in Web Applications: A Reality Today, a Challenge Tomorrow}},
	Year = {2010}}

@book{kerck,
	Author = {Kerckhoffs, A.},
	Publisher = {University Microfilms},
	Title = {{La cryptographie militaire}},
	Year = {1978}}

@inproceedings{foci11,
	Author = {J. Karlin and D. Ellard and A.~Jackson and C.~ Jones and G. Lauer and D. Mankins and W.~T.~Strayer},
	Booktitle = {FOCI},
	Title = {{Decoy Routing: Toward Unblockable Internet Communication}},
	Year = 2011}

@inproceedings{sun02,
	Author = {Sun, Q. and Simon, D.~R. and Wang, Y. and Russell, W. and Padmanabhan, V. and Qiu, L.},
	Booktitle = {IEEE S\&P},
	Title = {{Statistical Identification of Encrypted Web Browsing Traffic}},
	Year = {2002}}

@inproceedings{danezis,
	Author = {Murdoch, S.~J. and Danezis, G.},
	Booktitle = {IEEE S\&P},
	Title = {{Low-Cost Traffic Analysis of Tor}},
	Year = {2005}}

@inproceedings{pakicensorship,
	Author = {Z.~Nabi},
	Booktitle = {FOCI},
	Title = {The Anatomy of {Web} Censorship in {Pakistan}},
	Year = {2013}}

@inproceedings{irancensorship,
	Author = {S.~Aryan and H.~Aryan and A.~Halderman},
	Booktitle = {FOCI},
	Title = {Internet Censorship in {Iran}: {A} First Look},
	Year = {2013}}

@inproceedings{ford10efficient,
	Author = {Amittai Aviram and Shu-Chun Weng and Sen Hu and Bryan Ford},
	Booktitle = {\bibconf[9th]{OSDI}{USENIX Symposium on Operating Systems Design and Implementation}},
	Location = {Vancouver, BC, Canada},
	Month = oct,
	Title = {Efficient System-Enforced Deterministic Parallelism},
	Year = 2010}

@inproceedings{ford10determinating,
	Author = {Amittai Aviram and Sen Hu and Bryan Ford and Ramakrishna Gummadi},
	Booktitle = {\bibconf{CCSW}{ACM Cloud Computing Security Workshop}},
	Location = {Chicago, IL},
	Month = oct,
	Title = {Determinating Timing Channels in Compute Clouds},
	Year = 2010}

@inproceedings{ford12plugging,
	Author = {Bryan Ford},
	Booktitle = {\bibconf[4th]{HotCloud}{USENIX Workshop on Hot Topics in Cloud Computing}},
	Location = {Boston, MA},
	Month = jun,
	Title = {Plugging Side-Channel Leaks with Timing Information Flow Control},
	Year = 2012}

@inproceedings{ford12icebergs,
	Author = {Bryan Ford},
	Booktitle = {\bibconf[4th]{HotCloud}{USENIX Workshop on Hot Topics in Cloud Computing}},
	Location = {Boston, MA},
	Month = jun,
	Title = {Icebergs in the Clouds: the {\em Other} Risks of Cloud Computing},
	Year = 2012}

@misc{mullenize,
	Author = {Washington Post},
	Howpublished = {\url{http://apps.washingtonpost.com/g/page/world/gchq-report-on-mullenize-program-to-stain-anonymous-electronic-traffic/502/}},
	Month = {oct},
	Title = {{GCHQ} report on {`MULLENIZE'} program to `stain' anonymous electronic traffic},
	Year = {2013}}

@inproceedings{shue13street,
	Author = {Craig A. Shue and Nathanael Paul and Curtis R. Taylor},
	Booktitle = {\bibbrev[7th]{WOOT}{USENIX Workshop on Offensive Technologies}},
	Month = aug,
	Title = {From an {IP} Address to a Street Address: Using Wireless Signals to Locate a Target},
	Year = 2013}

@inproceedings{knockel11three,
	Author = {Jeffrey Knockel and Jedidiah R. Crandall and Jared Saia},
	Booktitle = {\bibbrev{FOCI}{USENIX Workshop on Free and Open Communications on the Internet}},
	Location = {San Francisco, CA},
	Month = aug,
	Year = 2011}

@misc{rfc4960,
	Author = {R. {Stewart, ed.}},
	Month = sep,
	Note = {RFC 4960},
	Title = {Stream Control Transmission Protocol},
	Year = 2007}

@inproceedings{ford07structured,
	Author = {Bryan Ford},
	Booktitle = {\bibbrev{SIGCOMM}{ACM SIGCOMM}},
	Location = {Kyoto, Japan},
	Month = aug,
	Title = {Structured Streams: a New Transport Abstraction},
	Year = {2007}}

@misc{spdy,
	Author = {Google, Inc.},
	Note = {\url{http://www.chromium.org/spdy/spdy-whitepaper}},
	Title = {{SPDY}: An Experimental Protocol For a Faster {Web}}}

@misc{quic,
	Author = {Jim Roskind},
	Month = jun,
	Note = {\url{http://blog.chromium.org/2013/06/experimenting-with-quic.html}},
	Title = {Experimenting with {QUIC}},
	Year = 2013}

@misc{podjarny12not,
	Author = {G.~Podjarny},
	Month = jun,
	Note = {\url{http://www.guypo.com/technical/not-as-spdy-as-you-thought/}},
	Title = {{Not as SPDY as You Thought}},
	Year = 2012}

@inproceedings{cor,
	Author = {Jones, N.~A. and Arye, M. and Cesareo, J. and Freedman, M.~J.},
	Booktitle = {FOCI},
	Title = {{Hiding Amongst the Clouds: A Proposal for Cloud-based Onion Routing}},
	Year = {2011}}

@misc{torcloud,
	Howpublished = {\url{https://cloud.torproject.org/}},
	Key = {tor cloud},
	Title = {{The Tor Cloud Project}}}

@inproceedings{scramblesuit,
	Author = {Philipp Winter and Tobias Pulls and Juergen Fuss},
	Booktitle = {WPES},
	Title = {{ScrambleSuit: A Polymorphic Network Protocol to Circumvent Censorship}},
	Year = 2013}

@article{savage2000practical,
	Author = {Savage, S. and Wetherall, D. and Karlin, A. and Anderson, T.},
	Journal = {ACM SIGCOMM Computer Communication Review},
	Number = {4},
	Pages = {295--306},
	Publisher = {ACM},
	Title = {Practical network support for IP traceback},
	Volume = {30},
	Year = {2000}}

@inproceedings{ooni,
	Author = {Filast, A. and Appelbaum, J.},
	Booktitle = {{FOCI}},
	Title = {{OONI : Open Observatory of Network Interference}},
	Year = {2012}}

@misc{caida-rank,
	Howpublished = {\url{http://as-rank.caida.org/}},
	Key = {caida rank},
	Title = {{AS Rank: AS Ranking}}}

@inproceedings{usersrouted-ccs13,
	Author = {A.~Johnson and C.~Wacek and R.~Jansen and M.~Sherr and P.~Syverson},
	Booktitle = {CCS},
	Title = {{Users Get Routed: Traffic Correlation on Tor by Realistic Adversaries}},
	Year = {2013}}

@inproceedings{edman2009awareness,
	Author = {Edman, M. and Syverson, P.},
	Booktitle = {{CCS}},
	Title = {{AS-awareness in Tor path selection}},
	Year = {2009}}

@inproceedings{DecoyCosts,
	Author = {A.~Houmansadr and E.~L.~Wong and V.~Shmatikov},
	Booktitle = {NDSS},
	Title = {{No Direction Home: The True Cost of Routing Around Decoys}},
	Year = {2014}}

@article{cordon,
	Author = {Elahi, T. and Goldberg, I.},
	Journal = {University of Waterloo CACR},
	Title = {{CORDON--A Taxonomy of Internet Censorship Resistance Strategies}},
	Volume = {33},
	Year = {2012}}

@inproceedings{privex,
	Author = {T.~Elahi and G.~Danezis and I.~Goldberg	},
	Booktitle = {{CCS}},
	Title = {{AS-awareness in Tor path selection}},
	Year = {2014}}

@inproceedings{changeGuards,
	Author = {T.~Elahi and K.~Bauer and M.~AlSabah and R.~Dingledine and I.~Goldberg},
	Booktitle = {{WPES}},
	Title = {{ Changing of the Guards: Framework for Understanding and Improving Entry Guard Selection in Tor}},
	Year = {2012}}

@article{RAINBOW:Journal,
	Author = {A.~Houmansadr and N.~Kiyavash and N.~Borisov},
	Journal = {IEEE/ACM Transactions on Networking},
	Title = {{Non-Blind Watermarking of Network Flows}},
	Year = 2014}

@inproceedings{info-tod,
	Author = {A.~Houmansadr and S.~Gorantla and T.~Coleman and N.~Kiyavash and and N.~Borisov},
	Booktitle = {{CCS (poster session)}},
	Title = {{On the Channel Capacity of Network Flow Watermarking}},
	Year = {2009}}

@inproceedings{johnson2014game,
	Author = {Johnson, B. and Laszka, A. and Grossklags, J. and Vasek, M. and Moore, T.},
	Booktitle = {Workshop on Bitcoin Research},
	Title = {{Game-theoretic Analysis of DDoS Attacks Against Bitcoin Mining Pools}},
	Year = {2014}}

@incollection{laszka2013mitigation,
	Author = {Laszka, A. and Johnson, B. and Grossklags, J.},
	Booktitle = {Decision and Game Theory for Security},
	Pages = {175--191},
	Publisher = {Springer},
	Title = {{Mitigation of Targeted and Non-targeted Covert Attacks as a Timing Game}},
	Year = {2013}}

@inproceedings{schottle2013game,
	Author = {Schottle, P. and Laszka, A. and Johnson, B. and Grossklags, J. and Bohme, R.},
	Booktitle = {EUSIPCO},
	Title = {{A Game-theoretic Analysis of Content-adaptive Steganography with Independent Embedding}},
	Year = {2013}}

@inproceedings{CloudTransport,
	Author = {C.~Brubaker and A.~Houmansadr and V.~Shmatikov},
	Booktitle = {PETS},
	Title = {{CloudTransport: Using Cloud Storage for Censorship-Resistant Networking}},
	Year = {2014}}

@inproceedings{sweet,
	Author = {W.~Zhou and A.~Houmansadr and M.~Caesar and N.~Borisov},
	Booktitle = {HotPETs},
	Title = {{SWEET: Serving the Web by Exploiting Email Tunnels}},
	Year = {2013}}

@inproceedings{ahsan2002practical,
	Author = {Ahsan, K. and Kundur, D.},
	Booktitle = {Workshop on Multimedia Security},
	Title = {{Practical data hiding in TCP/IP}},
	Year = {2002}}

@incollection{danezis2011covert,
	Author = {Danezis, G.},
	Booktitle = {Security Protocols XVI},
	Pages = {198--214},
	Publisher = {Springer},
	Title = {{Covert Communications Despite Traffic Data Retention}},
	Year = {2011}}

@inproceedings{liu2009hide,
	Author = {Liu, Y. and Ghosal, D. and Armknecht, F. and Sadeghi, A.-R. and Schulz, S. and Katzenbeisser, S.},
	Booktitle = {ESORICS},
	Title = {{Hide and Seek in Time---Robust Covert Timing Channels}},
	Year = {2009}}

@misc{image-watermark-fing,
	Author = {Jonathan Bailey},
	Howpublished = {\url{https://www.plagiarismtoday.com/2009/12/02/image-detection-watermarking-vs-fingerprinting/}},
	Title = {{Image Detection: Watermarking vs. Fingerprinting}},
	Year = {2009}}

@inproceedings{Servetto98,
	Author = {S. D. Servetto and C. I. Podilchuk and K. Ramchandran},
	Booktitle = {Int. Conf. Image Processing},
	Title = {Capacity issues in digital image watermarking},
	Year = {1998}}

@inproceedings{Chen01,
	Author = {B. Chen and G.W.Wornell},
	Booktitle = {IEEE Trans. Inform. Theory},
	Pages = {1423--1443},
	Title = {Quantization index modulation: A class of provably good methods for digital watermarking and information embedding},
	Year = {2001}}

@inproceedings{Karakos00,
	Author = {D. Karakos and A. Papamarcou},
	Booktitle = {IEEE Int. Symp. Information Theory},
	Pages = {47},
	Title = {Relationship between quantization and distribution rates of digitally watermarked data},
	Year = {2000}}

@inproceedings{Sullivan98,
	Author = {J. A. OSullivan and P. Moulin and J. M. Ettinger},
	Booktitle = {IEEE Int. Symp. Information Theory},
	Pages = {297},
	Title = {Information theoretic analysis of steganography},
	Year = {1998}}

@inproceedings{Merhav00,
	Author = {N. Merhav},
	Booktitle = {IEEE Trans. Inform. Theory},
	Pages = {420--430},
	Title = {On random coding error exponents of watermarking systems},
	Year = {2000}}

@inproceedings{Somekh01,
	Author = {A. Somekh-Baruch and N. Merhav},
	Booktitle = {IEEE Int. Symp. Information Theory},
	Pages = {7},
	Title = {On the error exponent and capacity games of private watermarking systems},
	Year = {2001}}

@inproceedings{Steinberg01,
	Author = {Y. Steinberg and N. Merhav},
	Booktitle = {IEEE Trans. Inform. Theory},
	Pages = {1410--1422},
	Title = {Identification in the presence of side information with application to watermarking},
	Year = {2001}}

@article{Moulin03,
	Author = {P. Moulin and J.A. O'Sullivan},
	Journal = {IEEE Trans. Info. Theory},
	Number = {3},
	Title = {Information-theoretic analysis of information hiding},
	Volume = 49,
	Year = 2003}

@article{Gelfand80,
	Author = {S.I.~Gelfand and M.S.~Pinsker},
	Journal = {Problems of Control and Information Theory},
	Number = {1},
	Pages = {19-31},
	Title = {{Coding for channel with random parameters}},
	Url = {citeseer.ist.psu.edu/anantharam96bits.html},
	Volume = {9},
	Year = {1980},
	Bdsk-Url-1 = {citeseer.ist.psu.edu/anantharam96bits.html}}

@book{Wolfowitz78,
	Author = {J. Wolfowitz},
	Edition = {3rd},
	Location = {New York},
	Publisher = {Springer-Verlag},
	Title = {Coding Theorems of Information Theory},
	Year = 1978}

@article{caire99,
	Author = {G. Caire and S. Shamai},
	Journal = {IEEE Transactions on Information Theory},
	Number = {6},
	Pages = {2007--2019},
	Title = {On the Capacity of Some Channels with Channel State Information},
	Volume = {45},
	Year = {1999}}

@inproceedings{wright2007language,
	Author = {Wright, Charles V and Ballard, Lucas and Monrose, Fabian and Masson, Gerald M},
	Booktitle = {USENIX Security},
	Title = {{Language identification of encrypted VoIP traffic: Alejandra y Roberto or Alice and Bob?}},
	Year = {2007}}

@inproceedings{backes2010speaker,
	Author = {Backes, Michael and Doychev, Goran and D{\"u}rmuth, Markus and K{\"o}pf, Boris},
	Booktitle = {{European Symposium on Research in Computer Security (ESORICS)}},
	Pages = {508--523},
	Publisher = {Springer},
	Title = {{Speaker Recognition in Encrypted Voice Streams}},
	Year = {2010}}

@phdthesis{lu2009traffic,
	Author = {Lu, Yuanchao},
	School = {Cleveland State University},
	Title = {{On Traffic Analysis Attacks to Encrypted VoIP Calls}},
	Year = {2009}}

@inproceedings{wright2008spot,
	Author = {Wright, Charles V and Ballard, Lucas and Coull, Scott E and Monrose, Fabian and Masson, Gerald M},
	Booktitle = {IEEE Symposium on Security and Privacy},
	Pages = {35--49},
	Title = {Spot me if you can: Uncovering spoken phrases in encrypted VoIP conversations},
	Year = {2008}}

@inproceedings{white2011phonotactic,
	Author = {White, Andrew M and Matthews, Austin R and Snow, Kevin Z and Monrose, Fabian},
	Booktitle = {IEEE Symposium on Security and Privacy},
	Pages = {3--18},
	Title = {Phonotactic reconstruction of encrypted VoIP conversations: Hookt on fon-iks},
	Year = {2011}}

@inproceedings{fancy,
	Author = {Houmansadr, Amir and Borisov, Nikita},
	Booktitle = {Privacy Enhancing Technologies},
	Organization = {Springer},
	Pages = {205--224},
	Title = {The Need for Flow Fingerprints to Link Correlated Network Flows},
	Year = {2013}}

@article{botmosaic,
	Author = {Amir Houmansadr and Nikita Borisov},
	Doi = {10.1016/j.jss.2012.11.005},
	Issn = {0164-1212},
	Journal = {Journal of Systems and Software},
	Keywords = {Network security},
	Number = {3},
	Pages = {707 - 715},
	Title = {BotMosaic: Collaborative network watermark for the detection of IRC-based botnets},
	Url = {http://www.sciencedirect.com/science/article/pii/S0164121212003068},
	Volume = {86},
	Year = {2013},
	Bdsk-Url-1 = {http://www.sciencedirect.com/science/article/pii/S0164121212003068},
	Bdsk-Url-2 = {http://dx.doi.org/10.1016/j.jss.2012.11.005}}

@inproceedings{ramsbrock2008first,
	Author = {Ramsbrock, Daniel and Wang, Xinyuan and Jiang, Xuxian},
	Booktitle = {Recent Advances in Intrusion Detection},
	Organization = {Springer},
	Pages = {59--77},
	Title = {A first step towards live botmaster traceback},
	Year = {2008}}

@inproceedings{potdar2005survey,
	Author = {Potdar, Vidyasagar M and Han, Song and Chang, Elizabeth},
	Booktitle = {Industrial Informatics, 2005. INDIN'05. 2005 3rd IEEE International Conference on},
	Organization = {IEEE},
	Pages = {709--716},
	Title = {A survey of digital image watermarking techniques},
	Year = {2005}}

@book{cole2003hiding,
	Author = {Cole, Eric and Krutz, Ronald D},
	Publisher = {John Wiley \& Sons, Inc.},
	Title = {Hiding in plain sight: Steganography and the art of covert communication},
	Year = {2003}}

@incollection{akaike1998information,
	Author = {Akaike, Hirotogu},
	Booktitle = {Selected Papers of Hirotugu Akaike},
	Pages = {199--213},
	Publisher = {Springer},
	Title = {Information theory and an extension of the maximum likelihood principle},
	Year = {1998}}

@misc{central-command-hack,
	Author = {Everett Rosenfeld},
	Howpublished = {\url{http://www.cnbc.com/id/102330338}},
	Title = {{FBI investigating Central Command Twitter hack}},
	Year = {2015}}

@misc{sony-psp-ddos,
	Howpublished = {\url{http://n4g.com/news/1644853/sony-and-microsoft-cant-do-much-ddos-attacks-explained}},
	Key = {sony},
	Month = {December},
	Title = {{Sony and Microsoft cant do much -- DDoS attacks explained}},
	Year = {2014}}

@misc{sony-hack,
	Author = {David Bloom},
	Howpublished = {\url{http://goo.gl/MwR4A7}},
	Title = {{Online Game Networks Hacked, Sony Unit President Threatened}},
	Year = {2014}}

@misc{home-depot,
	Author = {Dune Lawrence},
	Howpublished = {\url{http://www.businessweek.com/articles/2014-09-02/home-depots-credit-card-breach-looks-just-like-the-target-hack}},
	Title = {{Home Depot's Suspected Breach Looks Just Like the Target Hack}},
	Year = {2014}}

@misc{target,
	Author = {Julio Ojeda-Zapata},
	Howpublished = {\url{http://www.mercurynews.com/business/ci_24765398/how-did-hackers-pull-off-target-data-heist}},
	Title = {{Target hack: How did they do it?}},
	Year = {2014}}


@article{probabilitycourse,
	Author = {H. Pishro-Nik},
	note = {\url{http://www.probabilitycourse.com}},
	Title = {Introduction to probability, statistics, and random processes},
    Year = {2014}}



@inproceedings{shokri2011quantifying,
	Author = {Shokri, Reza and Theodorakopoulos, George and Le Boudec, Jean-Yves and Hubaux, Jean-Pierre},
	Booktitle = {Security and Privacy (SP), 2011 IEEE Symposium on},
	Organization = {IEEE},
	Pages = {247--262},
	Title = {Quantifying location privacy},
	Year = {2011}}

@inproceedings{hoh2007preserving,
	Author = {Hoh, Baik and Gruteser, Marco and Xiong, Hui and Alrabady, Ansaf},
	Booktitle = {Proceedings of the 14th ACM conference on Computer and communications security},
	Organization = {ACM},
	Pages = {161--171},
	Title = {Preserving privacy in gps traces via uncertainty-aware path cloaking},
	Year = {2007}}



@article{kafsi2013entropy,
	Author = {Kafsi, Mohamed and Grossglauser, Matthias and Thiran, Patrick},
	Journal = {Information Theory, IEEE Transactions on},
	Number = {9},
	Pages = {5577--5583},
	Publisher = {IEEE},
	Title = {The entropy of conditional Markov trajectories},
	Volume = {59},
	Year = {2013}}

@inproceedings{gruteser2003anonymous,
	Author = {Gruteser, Marco and Grunwald, Dirk},
	Booktitle = {Proceedings of the 1st international conference on Mobile systems, applications and services},
	Organization = {ACM},
	Pages = {31--42},
	Title = {Anonymous usage of location-based services through spatial and temporal cloaking},
	Year = {2003}}

@inproceedings{husted2010mobile,
	Author = {Husted, Nathaniel and Myers, Steven},
	Booktitle = {Proceedings of the 17th ACM conference on Computer and communications security},
	Organization = {ACM},
	Pages = {85--96},
	Title = {Mobile location tracking in metro areas: malnets and others},
	Year = {2010}}

@inproceedings{li2009tradeoff,
	Author = {Li, Tiancheng and Li, Ninghui},
	Booktitle = {Proceedings of the 15th ACM SIGKDD international conference on Knowledge discovery and data mining},
	Organization = {ACM},
	Pages = {517--526},
	Title = {On the tradeoff between privacy and utility in data publishing},
	Year = {2009}}

@inproceedings{ma2009location,
	Author = {Ma, Zhendong and Kargl, Frank and Weber, Michael},
	Booktitle = {Sarnoff Symposium, 2009. SARNOFF'09. IEEE},
	Organization = {IEEE},
	Pages = {1--6},
	Title = {A location privacy metric for v2x communication systems},
	Year = {2009}}

@inproceedings{shokri2012protecting,
	Author = {Shokri, Reza and Theodorakopoulos, George and Troncoso, Carmela and Hubaux, Jean-Pierre and Le Boudec, Jean-Yves},
	Booktitle = {Proceedings of the 2012 ACM conference on Computer and communications security},
	Organization = {ACM},
	Pages = {617--627},
	Title = {Protecting location privacy: optimal strategy against localization attacks},
	Year = {2012}}

@inproceedings{freudiger2009non,
	Author = {Freudiger, Julien and Manshaei, Mohammad Hossein and Hubaux, Jean-Pierre and Parkes, David C},
	Booktitle = {Proceedings of the 16th ACM conference on Computer and communications security},
	Organization = {ACM},
	Pages = {324--337},
	Title = {On non-cooperative location privacy: a game-theoretic analysis},
	Year = {2009}}

@incollection{humbert2010tracking,
	Author = {Humbert, Mathias and Manshaei, Mohammad Hossein and Freudiger, Julien and Hubaux, Jean-Pierre},
	Booktitle = {Decision and Game Theory for Security},
	Pages = {38--57},
	Publisher = {Springer},
	Title = {Tracking games in mobile networks},
	Year = {2010}}

@article{manshaei2013game,
	Author = {Manshaei, Mohammad Hossein and Zhu, Quanyan and Alpcan, Tansu and Bac{\c{s}}ar, Tamer and Hubaux, Jean-Pierre},
	Journal = {ACM Computing Surveys (CSUR)},
	Number = {3},
	Pages = {25},
	Publisher = {ACM},
	Title = {Game theory meets network security and privacy},
	Volume = {45},
	Year = {2013}}

@article{palamidessi2006probabilistic,
	Author = {Palamidessi, Catuscia},
	Journal = {Electronic Notes in Theoretical Computer Science},
	Pages = {33--42},
	Publisher = {Elsevier},
	Title = {Probabilistic and nondeterministic aspects of anonymity},
	Volume = {155},
	Year = {2006}}

@inproceedings{mokbel2006new,
	Author = {Mokbel, Mohamed F and Chow, Chi-Yin and Aref, Walid G},
	Booktitle = {Proceedings of the 32nd international conference on Very large data bases},
	Organization = {VLDB Endowment},
	Pages = {763--774},
	Title = {The new Casper: query processing for location services without compromising privacy},
	Year = {2006}}

@article{kalnis2007preventing,
	Author = {Kalnis, Panos and Ghinita, Gabriel and Mouratidis, Kyriakos and Papadias, Dimitris},
	Journal = {Knowledge and Data Engineering, IEEE Transactions on},
	Number = {12},
	Pages = {1719--1733},
	Publisher = {IEEE},
	Title = {Preventing location-based identity inference in anonymous spatial queries},
	Volume = {19},
	Year = {2007}}
	
@article{freudiger2007mix,
  title={Mix-zones for location privacy in vehicular networks},
  author={Freudiger, Julien and Raya, Maxim and F{\'e}legyh{\'a}zi, M{\'a}rk and Papadimitratos, Panos and Hubaux, Jean-Pierre},
  year={2007}
}
@article{sweeney2002k,
	Author = {Sweeney, Latanya},
	Journal = {International Journal of Uncertainty, Fuzziness and Knowledge-Based Systems},
	Number = {05},
	Pages = {557--570},
	Publisher = {World Scientific},
	Title = {k-anonymity: A model for protecting privacy},
	Volume = {10},
	Year = {2002}}

@article{sweeney2002achieving,
	Author = {Sweeney, Latanya},
	Journal = {International Journal of Uncertainty, Fuzziness and Knowledge-Based Systems},
	Number = {05},
	Pages = {571--588},
	Publisher = {World Scientific},
	Title = {Achieving k-anonymity privacy protection using generalization and suppression},
	Volume = {10},
	Year = {2002}}

@inproceedings{niu2014achieving,
	Author = {Niu, Ben and Li, Qinghua and Zhu, Xiaoyan and Cao, Guohong and Li, Hui},
	Booktitle = {INFOCOM, 2014 Proceedings IEEE},
	Organization = {IEEE},
	Pages = {754--762},
	Title = {Achieving k-anonymity in privacy-aware location-based services},
	Year = {2014}}

@inproceedings{liu2013game,
	Author = {Liu, Xinxin and Liu, Kaikai and Guo, Linke and Li, Xiaolin and Fang, Yuguang},
	Booktitle = {INFOCOM, 2013 Proceedings IEEE},
	Organization = {IEEE},
	Pages = {2985--2993},
	Title = {A game-theoretic approach for achieving k-anonymity in location based services},
	Year = {2013}}

@inproceedings{kido2005protection,
	Author = {Kido, Hidetoshi and Yanagisawa, Yutaka and Satoh, Tetsuji},
	Booktitle = {Data Engineering Workshops, 2005. 21st International Conference on},
	Organization = {IEEE},
	Pages = {1248--1248},
	Title = {Protection of location privacy using dummies for location-based services},
	Year = {2005}}

@inproceedings{gedik2005location,
	Author = {Gedik, Bu{\u{g}}ra and Liu, Ling},
	Booktitle = {Distributed Computing Systems, 2005. ICDCS 2005. Proceedings. 25th IEEE International Conference on},
	Organization = {IEEE},
	Pages = {620--629},
	Title = {Location privacy in mobile systems: A personalized anonymization model},
	Year = {2005}}

@inproceedings{bordenabe2014optimal,
	Author = {Bordenabe, Nicol{\'a}s E and Chatzikokolakis, Konstantinos and Palamidessi, Catuscia},
	Booktitle = {Proceedings of the 2014 ACM SIGSAC Conference on Computer and Communications Security},
	Organization = {ACM},
	Pages = {251--262},
	Title = {Optimal geo-indistinguishable mechanisms for location privacy},
	Year = {2014}}

@incollection{duckham2005formal,
	Author = {Duckham, Matt and Kulik, Lars},
	Booktitle = {Pervasive computing},
	Pages = {152--170},
	Publisher = {Springer},
	Title = {A formal model of obfuscation and negotiation for location privacy},
	Year = {2005}}

@inproceedings{kido2005anonymous,
	Author = {Kido, Hidetoshi and Yanagisawa, Yutaka and Satoh, Tetsuji},
	Booktitle = {Pervasive Services, 2005. ICPS'05. Proceedings. International Conference on},
	Organization = {IEEE},
	Pages = {88--97},
	Title = {An anonymous communication technique using dummies for location-based services},
	Year = {2005}}

@incollection{duckham2006spatiotemporal,
	Author = {Duckham, Matt and Kulik, Lars and Birtley, Athol},
	Booktitle = {Geographic Information Science},
	Pages = {47--64},
	Publisher = {Springer},
	Title = {A spatiotemporal model of strategies and counter strategies for location privacy protection},
	Year = {2006}}

@inproceedings{shankar2009privately,
	Author = {Shankar, Pravin and Ganapathy, Vinod and Iftode, Liviu},
	Booktitle = {Proceedings of the 11th international conference on Ubiquitous computing},
	Organization = {ACM},
	Pages = {31--40},
	Title = {Privately querying location-based services with SybilQuery},
	Year = {2009}}

@inproceedings{chow2009faking,
	Author = {Chow, Richard and Golle, Philippe},
	Booktitle = {Proceedings of the 8th ACM workshop on Privacy in the electronic society},
	Organization = {ACM},
	Pages = {105--108},
	Title = {Faking contextual data for fun, profit, and privacy},
	Year = {2009}}

@incollection{xue2009location,
	Author = {Xue, Mingqiang and Kalnis, Panos and Pung, Hung Keng},
	Booktitle = {Location and Context Awareness},
	Pages = {70--87},
	Publisher = {Springer},
	Title = {Location diversity: Enhanced privacy protection in location based services},
	Year = {2009}}

@article{wernke2014classification,
	Author = {Wernke, Marius and Skvortsov, Pavel and D{\"u}rr, Frank and Rothermel, Kurt},
	Journal = {Personal and Ubiquitous Computing},
	Number = {1},
	Pages = {163--175},
	Publisher = {Springer-Verlag},
	Title = {A classification of location privacy attacks and approaches},
	Volume = {18},
	Year = {2014}}

@misc{cai2015cloaking,
	Author = {Cai, Y. and Xu, G.},
	Month = jan # {~1},
	Note = {US Patent App. 14/472,462},
	Publisher = {Google Patents},
	Title = {Cloaking with footprints to provide location privacy protection in location-based services},
	Url = {https://www.google.com/patents/US20150007341},
	Year = {2015},
	Bdsk-Url-1 = {https://www.google.com/patents/US20150007341}}

@article{gedik2008protecting,
	Author = {Gedik, Bu{\u{g}}ra and Liu, Ling},
	Journal = {Mobile Computing, IEEE Transactions on},
	Number = {1},
	Pages = {1--18},
	Publisher = {IEEE},
	Title = {Protecting location privacy with personalized k-anonymity: Architecture and algorithms},
	Volume = {7},
	Year = {2008}}

@article{kalnis2006preserving,
	Author = {Kalnis, Panos and Ghinita, Gabriel and Mouratidis, Kyriakos and Papadias, Dimitris},
	Publisher = {TRB6/06},
	Title = {Preserving anonymity in location based services},
	Year = {2006}}

@inproceedings{hoh2005protecting,
	Author = {Hoh, Baik and Gruteser, Marco},
	Booktitle = {Security and Privacy for Emerging Areas in Communications Networks, 2005. SecureComm 2005. First International Conference on},
	Organization = {IEEE},
	Pages = {194--205},
	Title = {Protecting location privacy through path confusion},
	Year = {2005}}

@article{terrovitis2011privacy,
	Author = {Terrovitis, Manolis},
	Journal = {ACM SIGKDD Explorations Newsletter},
	Number = {1},
	Pages = {6--18},
	Publisher = {ACM},
	Title = {Privacy preservation in the dissemination of location data},
	Volume = {13},
	Year = {2011}}

@article{shin2012privacy,
	Author = {Shin, Kang G and Ju, Xiaoen and Chen, Zhigang and Hu, Xin},
	Journal = {Wireless Communications, IEEE},
	Number = {1},
	Pages = {30--39},
	Publisher = {IEEE},
	Title = {Privacy protection for users of location-based services},
	Volume = {19},
	Year = {2012}}

@article{khoshgozaran2011location,
	Author = {Khoshgozaran, Ali and Shahabi, Cyrus and Shirani-Mehr, Houtan},
	Journal = {Knowledge and Information Systems},
	Number = {3},
	Pages = {435--465},
	Publisher = {Springer},
	Title = {Location privacy: going beyond K-anonymity, cloaking and anonymizers},
	Volume = {26},
	Year = {2011}}

@incollection{chatzikokolakis2015geo,
	Author = {Chatzikokolakis, Konstantinos and Palamidessi, Catuscia and Stronati, Marco},
	Booktitle = {Distributed Computing and Internet Technology},
	Pages = {49--72},
	Publisher = {Springer},
	Title = {Geo-indistinguishability: A Principled Approach to Location Privacy},
	Year = {2015}}

@inproceedings{ngo2015location,
	Author = {Ngo, Hoa and Kim, Jong},
	Booktitle = {Computer Security Foundations Symposium (CSF), 2015 IEEE 28th},
	Organization = {IEEE},
	Pages = {63--74},
	Title = {Location Privacy via Differential Private Perturbation of Cloaking Area},
	Year = {2015}}

@inproceedings{palanisamy2011mobimix,
	Author = {Palanisamy, Balaji and Liu, Ling},
	Booktitle = {Data Engineering (ICDE), 2011 IEEE 27th International Conference on},
	Organization = {IEEE},
	Pages = {494--505},
	Title = {Mobimix: Protecting location privacy with mix-zones over road networks},
	Year = {2011}}

@inproceedings{um2010advanced,
	Author = {Um, Jung-Ho and Kim, Hee-Dae and Chang, Jae-Woo},
	Booktitle = {Social Computing (SocialCom), 2010 IEEE Second International Conference on},
	Organization = {IEEE},
	Pages = {1093--1098},
	Title = {An advanced cloaking algorithm using Hilbert curves for anonymous location based service},
	Year = {2010}}

@inproceedings{bamba2008supporting,
	Author = {Bamba, Bhuvan and Liu, Ling and Pesti, Peter and Wang, Ting},
	Booktitle = {Proceedings of the 17th international conference on World Wide Web},
	Organization = {ACM},
	Pages = {237--246},
	Title = {Supporting anonymous location queries in mobile environments with privacygrid},
	Year = {2008}}

@inproceedings{zhangwei2010distributed,
	Author = {Zhangwei, Huang and Mingjun, Xin},
	Booktitle = {Networks Security Wireless Communications and Trusted Computing (NSWCTC), 2010 Second International Conference on},
	Organization = {IEEE},
	Pages = {468--471},
	Title = {A distributed spatial cloaking protocol for location privacy},
	Volume = {2},
	Year = {2010}}

@article{chow2011spatial,
	Author = {Chow, Chi-Yin and Mokbel, Mohamed F and Liu, Xuan},
	Journal = {GeoInformatica},
	Number = {2},
	Pages = {351--380},
	Publisher = {Springer},
	Title = {Spatial cloaking for anonymous location-based services in mobile peer-to-peer environments},
	Volume = {15},
	Year = {2011}}

@inproceedings{lu2008pad,
	Author = {Lu, Hua and Jensen, Christian S and Yiu, Man Lung},
	Booktitle = {Proceedings of the Seventh ACM International Workshop on Data Engineering for Wireless and Mobile Access},
	Organization = {ACM},
	Pages = {16--23},
	Title = {Pad: privacy-area aware, dummy-based location privacy in mobile services},
	Year = {2008}}

@incollection{khoshgozaran2007blind,
	Author = {Khoshgozaran, Ali and Shahabi, Cyrus},
	Booktitle = {Advances in Spatial and Temporal Databases},
	Pages = {239--257},
	Publisher = {Springer},
	Title = {Blind evaluation of nearest neighbor queries using space transformation to preserve location privacy},
	Year = {2007}}

@inproceedings{ghinita2008private,
	Author = {Ghinita, Gabriel and Kalnis, Panos and Khoshgozaran, Ali and Shahabi, Cyrus and Tan, Kian-Lee},
	Booktitle = {Proceedings of the 2008 ACM SIGMOD international conference on Management of data},
	Organization = {ACM},
	Pages = {121--132},
	Title = {Private queries in location based services: anonymizers are not necessary},
	Year = {2008}}

@article{paulet2014privacy,
	Author = {Paulet, Russell and Kaosar, Md Golam and Yi, Xun and Bertino, Elisa},
	Journal = {Knowledge and Data Engineering, IEEE Transactions on},
	Number = {5},
	Pages = {1200--1210},
	Publisher = {IEEE},
	Title = {Privacy-preserving and content-protecting location based queries},
	Volume = {26},
	Year = {2014}}

@article{nguyen2013differential,
	Author = {Nguyen, Hiep H and Kim, Jong and Kim, Yoonho},
	Journal = {Journal of Computing Science and Engineering},
	Number = {3},
	Pages = {177--186},
	Title = {Differential privacy in practice},
	Volume = {7},
	Year = {2013}}

@inproceedings{lee2012differential,
	Author = {Lee, Jaewoo and Clifton, Chris},
	Booktitle = {Proceedings of the 18th ACM SIGKDD international conference on Knowledge discovery and data mining},
	Organization = {ACM},
	Pages = {1041--1049},
	Title = {Differential identifiability},
	Year = {2012}}

@inproceedings{andres2013geo,
	Author = {Andr{\'e}s, Miguel E and Bordenabe, Nicol{\'a}s E and Chatzikokolakis, Konstantinos and Palamidessi, Catuscia},
	Booktitle = {Proceedings of the 2013 ACM SIGSAC conference on Computer \& communications security},
	Organization = {ACM},
	Pages = {901--914},
	Title = {Geo-indistinguishability: Differential privacy for location-based systems},
	Year = {2013}}

@inproceedings{machanavajjhala2008privacy,
	Author = {Machanavajjhala, Ashwin and Kifer, Daniel and Abowd, John and Gehrke, Johannes and Vilhuber, Lars},
	Booktitle = {Data Engineering, 2008. ICDE 2008. IEEE 24th International Conference on},
	Organization = {IEEE},
	Pages = {277--286},
	Title = {Privacy: Theory meets practice on the map},
	Year = {2008}}

@article{dewri2013local,
	Author = {Dewri, Rinku},
	Journal = {Mobile Computing, IEEE Transactions on},
	Number = {12},
	Pages = {2360--2372},
	Publisher = {IEEE},
	Title = {Local differential perturbations: Location privacy under approximate knowledge attackers},
	Volume = {12},
	Year = {2013}}

@inproceedings{chatzikokolakis2013broadening,
	Author = {Chatzikokolakis, Konstantinos and Andr{\'e}s, Miguel E and Bordenabe, Nicol{\'a}s Emilio and Palamidessi, Catuscia},
	Booktitle = {Privacy Enhancing Technologies},
	Organization = {Springer},
	Pages = {82--102},
	Title = {Broadening the Scope of Differential Privacy Using Metrics.},
	Year = {2013}}

@inproceedings{zhong2009distributed,
	Author = {Zhong, Ge and Hengartner, Urs},
	Booktitle = {Pervasive Computing and Communications, 2009. PerCom 2009. IEEE International Conference on},
	Organization = {IEEE},
	Pages = {1--10},
	Title = {A distributed k-anonymity protocol for location privacy},
	Year = {2009}}

@inproceedings{ho2011differential,
	Author = {Ho, Shen-Shyang and Ruan, Shuhua},
	Booktitle = {Proceedings of the 4th ACM SIGSPATIAL International Workshop on Security and Privacy in GIS and LBS},
	Organization = {ACM},
	Pages = {17--24},
	Title = {Differential privacy for location pattern mining},
	Year = {2011}}

@inproceedings{cheng2006preserving,
	Author = {Cheng, Reynold and Zhang, Yu and Bertino, Elisa and Prabhakar, Sunil},
	Booktitle = {Privacy Enhancing Technologies},
	Organization = {Springer},
	Pages = {393--412},
	Title = {Preserving user location privacy in mobile data management infrastructures},
	Year = {2006}}

@article{beresford2003location,
	Author = {Beresford, Alastair R and Stajano, Frank},
	Journal = {IEEE Pervasive computing},
	Number = {1},
	Pages = {46--55},
	Publisher = {IEEE},
	Title = {Location privacy in pervasive computing},
	Year = {2003}}

@inproceedings{freudiger2009optimal,
	Author = {Freudiger, Julien and Shokri, Reza and Hubaux, Jean-Pierre},
	Booktitle = {Privacy enhancing technologies},
	Organization = {Springer},
	Pages = {216--234},
	Title = {On the optimal placement of mix zones},
	Year = {2009}}

@article{krumm2009survey,
	Author = {Krumm, John},
	Journal = {Personal and Ubiquitous Computing},
	Number = {6},
	Pages = {391--399},
	Publisher = {Springer},
	Title = {A survey of computational location privacy},
	Volume = {13},
	Year = {2009}}

@article{Rakhshan2016letter,
	Author = {Rakhshan, Ali and Pishro-Nik, Hossein},
	Journal = {IEEE Wireless Communications Letter},
	Publisher = {IEEE},
	Title = {Interference Models for Vehicular Ad Hoc Networks},
	Year = {2016, submitted}}

@article{Rakhshan2015Journal,
	Author = {Rakhshan, Ali and Pishro-Nik, Hossein},
	Journal = {IEEE Transactions on Wireless Communications},
	Publisher = {IEEE},
	Title = {Improving Safety on Highways by Customizing Vehicular Ad Hoc Networks},
	Year = {to appear, 2017}}

@inproceedings{Rakhshan2015Cogsima,
	Author = {Rakhshan, Ali and Pishro-Nik, Hossein},
	Booktitle = {IEEE International Multi-Disciplinary Conference on Cognitive Methods in Situation Awareness and Decision Support},
	Organization = {IEEE},
	Title = {A New Approach to Customization of Accident Warning Systems to Individual Drivers},
	Year = {2015}}

@inproceedings{Rakhshan2015CISS,
	Author = {Rakhshan, Ali and Pishro-Nik, Hossein and Nekoui, Mohammad},
	Booktitle = {Conference on Information Sciences and Systems},
	Organization = {IEEE},
	Pages = {1--6},
	Title = {Driver-based adaptation of Vehicular Ad Hoc Networks for design of active safety systems},
	Year = {2015}}

@inproceedings{Rakhshan2014IV,
	Author = {Rakhshan, Ali and Pishro-Nik, Hossein and Ray, Evan},
	Booktitle = {Intelligent Vehicles Symposium},
	Organization = {IEEE},
	Pages = {1181--1186},
	Title = {Real-time estimation of the distribution of brake response times for an individual driver using Vehicular Ad Hoc Network.},
	Year = {2014}}

@inproceedings{Rakhshan2013Globecom,
	Author = {Rakhshan, Ali and Pishro-Nik, Hossein and Fisher, Donald and Nekoui, Mohammad},
	Booktitle = {IEEE Global Communications Conference},
	Organization = {IEEE},
	Pages = {1333--1337},
	Title = {Tuning collision warning algorithms to individual drivers for design of active safety systems.},
	Year = {2013}}

@article{Nekoui2012Journal,
	Author = {Nekoui, Mohammad and Pishro-Nik, Hossein},
	Journal = {IEEE Transactions on Wireless Communications},
	Number = {8},
	Pages = {2895--2905},
	Publisher = {IEEE},
	Title = {Throughput Scaling laws for Vehicular Ad Hoc Networks},
	Volume = {11},
	Year = {2012}}









@article{Nekoui2011Journal,
	Author = {Nekoui, Mohammad and Pishro-Nik, Hossein and Ni, Daiheng},
	Journal = {International Journal of Vehicular Technology},
	Pages = {1--11},
	Publisher = {Hindawi Publishing Corporation},
	Title = {Analytic Design of Active Safety Systems for Vehicular Ad hoc Networks},
	Volume = {2011},
	Year = {2011}}





	
@article{shokri2014optimal,
	  title={Optimal user-centric data obfuscation},
 	 author={Shokri, Reza},
 	 journal={arXiv preprint arXiv:1402.3426},
 	 year={2014}
	}
@article{chatzikokolakis2015location,
  title={Location privacy via geo-indistinguishability},
  author={Chatzikokolakis, Konstantinos and Palamidessi, Catuscia and Stronati, Marco},
  journal={ACM SIGLOG News},
  volume={2},
  number={3},
  pages={46--69},
  year={2015},
  publisher={ACM}

}
@inproceedings{shokri2011quantifying2,
  title={Quantifying location privacy: the case of sporadic location exposure},
  author={Shokri, Reza and Theodorakopoulos, George and Danezis, George and Hubaux, Jean-Pierre and Le Boudec, Jean-Yves},
  booktitle={Privacy Enhancing Technologies},
  pages={57--76},
  year={2011},
  organization={Springer}
}


@inproceedings{Mont1603:Defining,
AUTHOR="Zarrin Montazeri and Amir Houmansadr and Hossein Pishro-Nik",
TITLE="Defining Perfect Location Privacy Using Anonymization",
BOOKTITLE="2016 Annual Conference on Information Science and Systems (CISS) (CISS
2016)",
ADDRESS="Princeton, USA",
DAYS=16,
MONTH=mar,
YEAR=2016,
KEYWORDS="Information Theoretic Privacy; location-based services; Location Privacy;
Information Theory",
ABSTRACT="The popularity of mobile devices and location-based services (LBS) have
created great concerns regarding the location privacy of users of such
devices and services. Anonymization is a common technique that is often
being used to protect the location privacy of LBS users. In this paper, we
provide a general information theoretic definition for location privacy. In
particular, we define perfect location privacy. We show that under certain
conditions, perfect privacy is achieved if the pseudonyms of users is
changed after o(N^(2/r?1)) observations by the adversary, where N is the
number of users and r is the number of sub-regions or locations.
"
}
@article{our-isita-location,
	Author = {Zarrin Montazeri and Amir Houmansadr and Hossein Pishro-Nik},
	Journal = {IEEE International Symposium on Information Theory and Its Applications (ISITA)},
	Title = {Achieving Perfect Location Privacy in Markov Models Using Anonymization},
	Year = {2016}
	}
@article{our-TIFS,
	Author = {Zarrin Montazeri and Hossein Pishro-Nik and Amir Houmansadr},
	Journal = {IEEE Transactions on Information Forensics and Security, accepted with mandatory minor revisions},
	Title = {Perfect Location Privacy Using Anonymization in Mobile Networks},
	Year = {2017},
    note={Available on arxiv.org}
	}



@techreport{sampigethaya2005caravan,
  title={CARAVAN: Providing location privacy for VANET},
  author={Sampigethaya, Krishna and Huang, Leping and Li, Mingyan and Poovendran, Radha and Matsuura, Kanta and Sezaki, Kaoru},
  year={2005},
  institution={DTIC Document}
}
@incollection{buttyan2007effectiveness,
  title={On the effectiveness of changing pseudonyms to provide location privacy in VANETs},
  author={Butty{\'a}n, Levente and Holczer, Tam{\'a}s and Vajda, Istv{\'a}n},
  booktitle={Security and Privacy in Ad-hoc and Sensor Networks},
  pages={129--141},
  year={2007},
  publisher={Springer}
}
@article{sampigethaya2007amoeba,
  title={AMOEBA: Robust location privacy scheme for VANET},
  author={Sampigethaya, Krishna and Li, Mingyan and Huang, Leping and Poovendran, Radha},
  journal={Selected Areas in communications, IEEE Journal on},
  volume={25},
  number={8},
  pages={1569--1589},
  year={2007},
  publisher={IEEE}
}

@article{lu2012pseudonym,
  title={Pseudonym changing at social spots: An effective strategy for location privacy in vanets},
  author={Lu, Rongxing and Li, Xiaodong and Luan, Tom H and Liang, Xiaohui and Shen, Xuemin},
  journal={Vehicular Technology, IEEE Transactions on},
  volume={61},
  number={1},
  pages={86--96},
  year={2012},
  publisher={IEEE}
}
@inproceedings{lu2010sacrificing,
  title={Sacrificing the plum tree for the peach tree: A socialspot tactic for protecting receiver-location privacy in VANET},
  author={Lu, Rongxing and Lin, Xiaodong and Liang, Xiaohui and Shen, Xuemin},
  booktitle={Global Telecommunications Conference (GLOBECOM 2010), 2010 IEEE},
  pages={1--5},
  year={2010},
  organization={IEEE}
}
@inproceedings{lin2011stap,
  title={STAP: A social-tier-assisted packet forwarding protocol for achieving receiver-location privacy preservation in VANETs},
  author={Lin, Xiaodong and Lu, Rongxing and Liang, Xiaohui and Shen, Xuemin Sherman},
  booktitle={INFOCOM, 2011 Proceedings IEEE},
  pages={2147--2155},
  year={2011},
  organization={IEEE}
}
@inproceedings{gerlach2007privacy,
  title={Privacy in VANETs using changing pseudonyms-ideal and real},
  author={Gerlach, Matthias and Guttler, Felix},
  booktitle={Vehicular Technology Conference, 2007. VTC2007-Spring. IEEE 65th},
  pages={2521--2525},
  year={2007},
  organization={IEEE}
}
@inproceedings{el2002security,
  title={Security issues in a future vehicular network},
  author={El Zarki, Magda and Mehrotra, Sharad and Tsudik, Gene and Venkatasubramanian, Nalini},
  booktitle={European Wireless},
  volume={2},
  year={2002}
}

@article{hubaux2004security,
  title={The security and privacy of smart vehicles},
  author={Hubaux, Jean-Pierre and Capkun, Srdjan and Luo, Jun},
  journal={IEEE Security \& Privacy Magazine},
  volume={2},
  number={LCA-ARTICLE-2004-007},
  pages={49--55},
  year={2004}
}



@inproceedings{duri2002framework,
  title={Framework for security and privacy in automotive telematics},
  author={Duri, Sastry and Gruteser, Marco and Liu, Xuan and Moskowitz, Paul and Perez, Ronald and Singh, Moninder and Tang, Jung-Mu},
  booktitle={Proceedings of the 2nd international workshop on Mobile commerce},
  pages={25--32},
  year={2002},
  organization={ACM}
}
@misc{NS-3,
	Howpublished = {\url{https://www.nsnam.org/}}},
}
@misc{testbed,
	Howpublished = {\url{http://www.its.dot.gov/testbed/PDF/SE-MI-Resource-Guide-9-3-1.pdf}}},
@misc{NGSIM,
	Howpublished = {\url{http://ops.fhwa.dot.gov/trafficanalysistools/ngsim.htm}},
	}

@misc{National-a2013,
	Author = {National Highway Traffic Safety Administration},
	Howpublished = {\url{http://ops.fhwa.dot.gov/trafficanalysistools/ngsim.htm}},
	Title = {2013 Motor Vehicle Crashes: Overview. Traffic Safety Factors},
	Year = {2013}
	}

	@inproceedings{karnadi2007rapid,
	  title={Rapid generation of realistic mobility models for VANET},
	  author={Karnadi, Feliz Kristianto and Mo, Zhi Hai and Lan, Kun-chan},
	  booktitle={Wireless Communications and Networking Conference, 2007. WCNC 2007. IEEE},
	  pages={2506--2511},
	  year={2007},
	  organization={IEEE}
	}
	@inproceedings{saha2004modeling,
  title={Modeling mobility for vehicular ad-hoc networks},
  author={Saha, Amit Kumar and Johnson, David B},
  booktitle={Proceedings of the 1st ACM international workshop on Vehicular ad hoc networks},
  pages={91--92},
  year={2004},
  organization={ACM}
}
@inproceedings{lee2006modeling,
  title={Modeling steady-state and transient behaviors of user mobility: formulation, analysis, and application},
  author={Lee, Jong-Kwon and Hou, Jennifer C},
  booktitle={Proceedings of the 7th ACM international symposium on Mobile ad hoc networking and computing},
  pages={85--96},
  year={2006},
  organization={ACM}
}
@inproceedings{yoon2006building,
  title={Building realistic mobility models from coarse-grained traces},
  author={Yoon, Jungkeun and Noble, Brian D and Liu, Mingyan and Kim, Minkyong},
  booktitle={Proceedings of the 4th international conference on Mobile systems, applications and services},
  pages={177--190},
  year={2006},
  organization={ACM}
}

@inproceedings{choffnes2005integrated,
	title={An integrated mobility and traffic model for vehicular wireless networks},
	author={Choffnes, David R and Bustamante, Fabi{\'a}n E},
	booktitle={Proceedings of the 2nd ACM international workshop on Vehicular ad hoc networks},
	pages={69--78},
	year={2005},
	organization={ACM}
}

@inproceedings{Qian2008Globecom,
	title={CA Secure VANET MAC Protocol for DSRC Applications},
	author={Yi, Q. and Lu, K. and Moyeri, N.{\'a}n E},
	booktitle={Proceedings of IEEE GLOBECOM 2008},
	pages={1--5},
	year={2008},
	organization={IEEE}
}





	@inproceedings{naumov2006evaluation,
  title={An evaluation of inter-vehicle ad hoc networks based on realistic vehicular traces},
  author={Naumov, Valery and Baumann, Rainer and Gross, Thomas},
  booktitle={Proceedings of the 7th ACM international symposium on Mobile ad hoc networking and computing},
  pages={108--119},
  year={2006},
  organization={ACM}
}
	@article{sommer2008progressing,
  title={Progressing toward realistic mobility models in VANET simulations},
  author={Sommer, Christoph and Dressler, Falko},
  journal={Communications Magazine, IEEE},
  volume={46},
  number={11},
  pages={132--137},
  year={2008},
  publisher={IEEE}
}




	@inproceedings{mahajan2006urban,
  title={Urban mobility models for vanets},
  author={Mahajan, Atulya and Potnis, Niranjan and Gopalan, Kartik and Wang, Andy},
  booktitle={2nd IEEE International Workshop on Next Generation Wireless Networks},
  volume={33},
  year={2006}
}

@inproceedings{Rakhshan2016packet,
  title={Packet success probability derivation in a vehicular ad hoc network for a highway scenario},
  author={Rakhshan, Ali and Pishro-Nik, Hossein},
  booktitle={2016 Annual Conference on Information Science and Systems (CISS)},
  pages={210--215},
  year={2016},
  organization={IEEE}
}

@inproceedings{Rakhshan2016CISS,
	Author = {Rakhshan, Ali and Pishro-Nik, Hossein},
	Booktitle = {Conference on Information Sciences and Systems},
	Organization = {IEEE},
	Pages = {210--215},
	Title = {Packet Success Probability Derivation in a Vehicular Ad Hoc Network for a Highway Scenario},
	Year = {2016}}

@article{Nekoui2013Journal,
	Author = {Nekoui, Mohammad and Pishro-Nik, Hossein},
	Journal = {Journal on Selected Areas in Communications, Special Issue on Emerging Technologies in Communications},
	Number = {9},
	Pages = {491--503},
	Publisher = {IEEE},
	Title = {Analytic Design of Active Safety Systems for Vehicular Ad hoc Networks},
	Volume = {31},
	Year = {2013}}


@inproceedings{Nekoui2011MOBICOM,
	Author = {Nekoui, Mohammad and Pishro-Nik, Hossein},
	Booktitle = {MOBICOM workshop on VehiculAr InterNETworking},
	Organization = {ACM},
	Title = {Analytic Design of Active Vehicular Safety Systems in Sparse Traffic},
	Year = {2011}}

@inproceedings{Nekoui2011VTC,
	Author = {Nekoui, Mohammad and Pishro-Nik, Hossein},
	Booktitle = {VTC-Fall},
	Organization = {IEEE},
	Title = {Analytical Design of Inter-vehicular Communications for Collision Avoidance},
	Year = {2011}}

@inproceedings{Bovee2011VTC,
	Author = {Bovee, Ben Louis and Nekoui, Mohammad and Pishro-Nik, Hossein},
	Booktitle = {VTC-Fall},
	Organization = {IEEE},
	Title = {Evaluation of the Universal Geocast Scheme For VANETs},
	Year = {2011}}

@inproceedings{Nekoui2010MOBICOM,
	Author = {Nekoui, Mohammad and Pishro-Nik, Hossein},
	Booktitle = {MOBICOM},
	Organization = {ACM},
	Title = {Fundamental Tradeoffs in Vehicular Ad Hoc Networks},
	Year = {2010}}

@inproceedings{Nekoui2010IVCS,
	Author = {Nekoui, Mohammad and Pishro-Nik, Hossein},
	Booktitle = {IVCS},
	Organization = {IEEE},
	Title = {A Universal Geocast Scheme for Vehicular Ad Hoc Networks},
	Year = {2010}}

@inproceedings{Nekoui2009ITW,
	Author = {Nekoui, Mohammad and Pishro-Nik, Hossein},
	Booktitle = {IEEE Communications Society Conference on Sensor, Mesh and Ad Hoc Communications and Networks Workshops},
	Organization = {IEEE},
	Pages = {1--3},
	Title = {A Geometrical Analysis of Obstructed Wireless Networks},
	Year = {2009}}

@article{Eslami2013Journal,
	Author = {Eslami, Ali and Nekoui, Mohammad and Pishro-Nik, Hossein and Fekri, Faramarz},
	Journal = {ACM Transactions on Sensor Networks},
	Number = {4},
	Pages = {51},
	Publisher = {ACM},
	Title = {Results on finite wireless sensor networks: Connectivity and coverage},
	Volume = {9},
	Year = {2013}}


@article{Jiafu2014Journal,
	Author = {Jiafu, W. and Zhang, D. and Zhao, S. and Yang, L. and Lloret, J.},
	Journal = {Communications Magazine},
	Number = {8},
	Pages = {106-113},
	Publisher = {IEEE},
	Title = {Context-aware vehicular cyber-physical systems with cloud support: architecture, challenges, and solutions},
	Volume = {52},
	Year = {2014}}

@inproceedings{Haas2010ACM,
	Author = {Haas, J.J. and Hu, Y.},
	Booktitle = {international workshop on VehiculAr InterNETworking},
	Organization = {ACM},
	Title = {Communication requirements for crash avoidance.},
	Year = {2010}}

@inproceedings{Yi2008GLOBECOM,
	Author = {Yi, Q. and Lu, K. and Moayeri, N.},
	Booktitle = {GLOBECOM},
	Organization = {IEEE},
	Title = {CA Secure VANET MAC Protocol for DSRC Applications.},
	Year = {2008}}

@inproceedings{Mughal2010ITSim,
	Author = {Mughal, B.M. and Wagan, A. and Hasbullah, H.},
	Booktitle = {International Symposium on Information Technology (ITSim)},
	Organization = {IEEE},
	Title = {Efficient congestion control in VANET for safety messaging.},
	Year = {2010}}

@article{Chang2011Journal,
	Author = {Chang, Y. and Lee, C. and Copeland, J.},
	Journal = {Selected Areas in Communications},
	Pages = {236 –249},
	Publisher = {IEEE},
	Title = {Goodput enhancement of VANETs in noisy CSMA/CA channels},
	Volume = {29},
	Year = {2011}}

@article{Garcia-Costa2011Journal,
	Author = {Garcia-Costa, C. and Egea-Lopez, E. and Tomas-Gabarron, J. B. and Garcia-Haro, J. and Haas, Z. J.},
	Journal = {Transactions on Intelligent Transportation Systems},
	Pages = {1 –16},
	Publisher = {IEEE},
	Title = {A stochastic model for chain collisions of vehicles equipped with vehicular communications},
	Volume = {99},
	Year = {2011}}

@article{Carbaugh2011Journal,
	Author = {Carbaugh, J. and Godbole,  D. N. and Sengupta, R. and Garcia-Haro, J. and Haas, Z. J.},
	Publisher = {Transportation Research Part C (Emerging Technologies)},
	Title = {Safety and capacity analysis of automated and manual highway systems},
	Year = {1997}}

@article{Goh2004Journal,
	Author = {Goh, P. and Wong, Y.},
	Publisher = {Appl Health Econ Health Policy},
	Title = {Driver perception response time during the signal change interval},
	Year = {2004}}

@article{Chang1985Journal,
	Author = {Chang, M.S. and Santiago, A.J.},
	Pages = {20-30},
	Publisher = {Transportation Research Record},
	Title = {Timing traffic signal changes based on driver behavior},
	Volume = {1027},
	Year = {1985}}

@article{Green2000Journal,
	Author = {Green, M.},
	Pages = {195-216},
	Publisher = {Transportation Human Factors},
	Title = {How long does it take to stop? Methodological analysis of driver perception-brake times.},
	Year = {2000}}

@article{Koppa2005,
	Author = {Koppa, R.J.},
	Pages = {195-216},
	Publisher = {http://www.fhwa.dot.gov/publications/},
	Title = {Human Factors},
	Year = {2005}}

@inproceedings{Maxwell2010ETC,
	Author = {Maxwell, A. and Wood, K.},
	Booktitle = {Europian Transport Conference},
	Organization = {http://www.etcproceedings.org/paper/review-of-traffic-signals-on-high-speed-roads},
	Title = {Review of Traffic Signals on High Speed Road},
	Year = {2010}}

@article{Wortman1983,
	Author = {Wortman, R.H. and Matthias, J.S.},
	Publisher = {Arizona Department of Transportation},
	Title = {An Evaluation of Driver Behavior at Signalized Intersections},
	Year = {1983}}
@inproceedings{Zhang2007IASTED,
	Author = {Zhang, X. and Bham, G.H.},
	Booktitle = {18th IASTED International Conference: modeling and simulation},
	Title = {Estimation of driver reaction time from detailed vehicle trajectory data.},
	Year = {2007}}


@inproceedings{bai2003important,
  title={IMPORTANT: A framework to systematically analyze the Impact of Mobility on Performance of RouTing protocols for Adhoc NeTworks},
  author={Bai, Fan and Sadagopan, Narayanan and Helmy, Ahmed},
  booktitle={INFOCOM 2003. Twenty-second annual joint conference of the IEEE computer and communications. IEEE societies},
  volume={2},
  pages={825--835},
  year={2003},
  organization={IEEE}
}


@inproceedings{abedi2008enhancing,
	  title={Enhancing AODV routing protocol using mobility parameters in VANET},
	  author={Abedi, Omid and Fathy, Mahmood and Taghiloo, Jamshid},
	  booktitle={Computer Systems and Applications, 2008. AICCSA 2008. IEEE/ACS International Conference on},
	  pages={229--235},
	  year={2008},
	  organization={IEEE}
	}


@article{AlSultan2013Journal,
	Author = {Al-Sultan, Saif and Al-Bayatti, Ali H. and Zedan, Hussien},
	Journal = {IEEE Transactions on Vehicular Technology},
	Number = {9},
	Pages = {4264-4275},
	Publisher = {IEEE},
	Title = {Context Aware Driver Behaviour Detection System in Intelligent Transportation Systems},
	Volume = {62},
	Year = {2013}}






@article{Leow2008ITS,
	Author = {Leow, Woei Ling and Ni, Daiheng and Pishro-Nik, Hossein},
	Journal = {IEEE Transactions on Intelligent Transportation Systems},
	Number = {2},
	Pages = {369--374},
	Publisher = {IEEE},
	Title = {A Sampling Theorem Approach to Traffic Sensor Optimization},
	Volume = {9},
	Year = {2008}}



@article{REU2007,
	Author = {D. Ni and H. Pishro-Nik and R. Prasad and M. R. Kanjee and H. Zhu and T. Nguyen},
	Journal = {in 14th World Congress on Intelligent Transport Systems},
	Title = {Development of a prototype intersection collision avoidance system under VII},
	Year = {2007}}




@inproceedings{salamatian2013hide,
  title={How to hide the elephant-or the donkey-in the room: Practical privacy against statistical inference for large data.},
  author={Salamatian, Salman and Zhang, Amy and du Pin Calmon, Flavio and Bhamidipati, Sandilya and Fawaz, Nadia and Kveton, Branislav and Oliveira, Pedro and Taft, Nina},
  booktitle={GlobalSIP},
  pages={269--272},
  year={2013}
}

@article{sankar2013utility,
  title={Utility-privacy tradeoffs in databases: An information-theoretic approach},
  author={Sankar, Lalitha and Rajagopalan, S Raj and Poor, H Vincent},
  journal={Information Forensics and Security, IEEE Transactions on},
  volume={8},
  number={6},
  pages={838--852},
  year={2013},
  publisher={IEEE}
}
@inproceedings{ghinita2007prive,
  title={PRIVE: anonymous location-based queries in distributed mobile systems},
  author={Ghinita, Gabriel and Kalnis, Panos and Skiadopoulos, Spiros},
  booktitle={Proceedings of the 16th international conference on World Wide Web},
  pages={371--380},
  year={2007},
  organization={ACM}
}

@article{beresford2004mix,
  title={Mix zones: User privacy in location-aware services},
  author={Beresford, Alastair R and Stajano, Frank},
  year={2004},
  publisher={IEEE}
}

%@inproceedings{Mont1610Achieving,
%  title={Achieving Perfect Location Privacy in Markov Models Using Anonymization},
%  author={Montazeri, Zarrin and Houmansadr, Amir and H.Pishro-Nik},
%  booktitle="2016 International Symposium on Information Theory and its Applications
%  (ISITA2016)",
%  address="Monterey, USA",
%  days=30,
%  month=oct,
%  year=2016,
%}

@article{csiszar1996almost,
  title={Almost independence and secrecy capacity},
  author={Csisz{\'a}r, Imre},
  journal={Problemy Peredachi Informatsii},
  volume={32},
  number={1},
  pages={48--57},
  year={1996},
  publisher={Russian Academy of Sciences, Branch of Informatics, Computer Equipment and Automatization}
}

@article{yamamoto1983source,
  title={A source coding problem for sources with additional outputs to keep secret from the receiver or wiretappers (corresp.)},
  author={Yamamoto, Hirosuke},
  journal={IEEE Transactions on Information Theory},
  volume={29},
  number={6},
  pages={918--923},
  year={1983},
  publisher={IEEE}
}


@inproceedings{calmon2015fundamental,
  title={Fundamental limits of perfect privacy},
  author={Calmon, Flavio P and Makhdoumi, Ali and M{\'e}dard, Muriel},
  booktitle={Information Theory (ISIT), 2015 IEEE International Symposium on},
  pages={1796--1800},
  year={2015},
  organization={IEEE}
}



@inproceedings{Lehman1999Large-Sample-Theory,
	title={Elements of Large Sample Theory},
	author={E. L. Lehman},
	organization={Springer},
	year={1999}
}


@inproceedings{Ferguson1999Large-Sample-Theory,
	title={A Course in Large Sample Theory},
	author={Thomas S. Ferguson},
	organization={CRC Press},
	year={1996}
}



@inproceedings{Dembo1999Large-Deviations,
	title={Large Deviation Techniques and Applications, Second Edition},
	author={A. Dembo and O. Zeitouni},
	organization={Springer},
	year={1998}
}


%%%%%%%%%%%%%%%%%%%%%%%%%%%%%%%%%%%%%%%%%%%%%%%%
Hossein's Coding Journals
%%%%%%%%%%%%%%%%%%%%%%

@ARTICLE{myoptics,
  AUTHOR =       "H. Pishro-Nik and N. Rahnavard and J. Ha and F. Fekri and A. Adibi ",
  TITLE =        "Low-density parity-check codes for volume holographic memory systems",
  JOURNAL =      " Appl. Opt.",
  YEAR =         "2003",
  volume =       "42",
  pages =        "861-870  "
 }






@ARTICLE{myit,
  AUTHOR =       "H. Pishro-Nik and F. Fekri  ",
  TITLE =        "On Decoding of Low-Density Parity-Check Codes on the Binary Erasure Channel",
  JOURNAL =      "IEEE Trans. Inform. Theory",
  YEAR =         "2004",
  volume =       "50",
  pages =        "439--454"
  }




@ARTICLE{myitpuncture,
  AUTHOR =       "H. Pishro-Nik and F. Fekri  ",
  TITLE =        "Results on Punctured Low-Density Parity-Check Codes and Improved Iterative Decoding Techniques",
  JOURNAL =      "IEEE Trans. on Inform. Theory",
  YEAR =         "2007",
  volume =       "53",
  number=        "2",
  pages =        "599--614",
  month= "February"
  }




@ARTICLE{myitlinmimdist,
  AUTHOR =       "H. Pishro-Nik and F. Fekri",
  TITLE =        "Performance of Low-Density Parity-Check Codes With Linear Minimum Distance",
  JOURNAL =         "IEEE Trans. Inform. Theory ",
  YEAR =         "2006",
  volume =       "52",
  number="1",
  pages =        "292 --300"
  }






@ARTICLE{myitnonuni,
  AUTHOR =       "H. Pishro-Nik and N. Rahnavard and F. Fekri  ",
  TITLE =        "Non-uniform Error Correction Using Low-Density Parity-Check Codes",
  JOURNAL =      "IEEE Trans. Inform. Theory",
  YEAR =         "2005",
  volume =       "51",
  number=  "7",
  pages =        "2702--2714"
 }





@article{eslamitcomhybrid10,
 author = {A. Eslami and S. Vangala and H. Pishro-Nik},
 title = {Hybrid channel codes for highly efficient FSO/RF communication systems},
 journal = {IEEE Transactions on Communications},
 volume = {58},
 number = {10},
 year = {2010},
 pages = {2926--2938},
 }


@article{eslamitcompolar13,
 author = {A. Eslami and H. Pishro-Nik},
 title = {On Finite-Length Performance of Polar Codes: Stopping Sets, Error Floor, and Concatenated Design},
 journal = {IEEE Transactions on Communications},
 volume = {61},
 number = {13},
 year = {2013},
 pages = {919--929},
 }



 @article{saeeditcom11,
 author = {H. Saeedi and H. Pishro-Nik and  A. H. Banihashemi},
 title = {Successive maximization for the systematic design of universally capacity approaching rate-compatible
 sequences of LDPC code ensembles over binary-input output-symmetric memoryless channels},
 journal = {IEEE Transactions on Communications},
 year = {2011},
 volume={59},
 number = {7}
 }


@article{rahnavard07,
 author = {Rahnavard, N. and Pishro-Nik, H. and Fekri, F.},
 title = {Unequal Error Protection Using Partially Regular LDPC Codes},
 journal = {IEEE Transactions on Communications},
 year = {2007},
 volume = {55},
 number = {3},
 pages = {387 -- 391}
 }


 @article{hosseinira04,
 author = {H. Pishro-Nik and F. Fekri},
 title = {Irregular repeat-accumulate codes for volume holographic memory systems},
 journal = {Journal of Applied Optics},
 year = {2004},
 volume = {43},
 number = {27},
 pages = {5222--5227},
 }


@article{azadeh2015Ephemeralkey,
 author = {A. Sheikholeslami and D. Goeckel and H. Pishro-Nik},
 title = {Jamming Based on an Ephemeral Key to Obtain Everlasting Security in Wireless Environments},
 journal = {IEEE Transactions on Wireless Communications},
 year = {2015},
 volume = {14},
 number = {11},
 pages = {6072--6081},
}


@article{azadeh2014Everlasting,
 author = {A. Sheikholeslami and D. Goeckel and H. Pishro-Nik},
 title = {Everlasting secrecy in disadvantaged wireless environments against sophisticated eavesdroppers},
 journal = {48th Asilomar Conference on Signals, Systems and Computers},
 year = {2014},
 pages = {1994--1998},
}


@article{azadeh2013ISIT,
 author = {A. Sheikholeslami and D. Goeckel and H. Pishro-Nik},
 title = {Artificial intersymbol interference (ISI) to exploit receiver imperfections for secrecy},
 journal = {IEEE International Symposium on Information Theory (ISIT)},
 year = {2013},
}


@article{azadeh2013Jsac,
 author = {A. Sheikholeslami and D. Goeckel and H. Pishro-Nik},
 title = {Jamming Based on an Ephemeral Key to Obtain Everlasting Security in Wireless Environments},
 journal = {IEEE Journal on Selected Areas in Communications},
 year = {2013},
 volume = {31},
 number = {9},
 pages = {1828--1839},
}


@article{azadeh2012Allerton,
 author = {A. Sheikholeslami and D. Goeckel and H. Pishro-Nik},
 title = {Exploiting the non-commutativity of nonlinear operators for information-theoretic security in disadvantaged wireless environments},
 journal = {50th Annual Allerton Conference on Communication, Control, and Computing},
 year = {2012},
 pages = {233--240},
}


@article{azadeh2012Infocom,
 author = {A. Sheikholeslami and D. Goeckel and H. Pishro-Nik},
 title = {Jamming Based on an Ephemeral Key to Obtain Everlasting Security in Wireless Environments},
 journal = {IEEE INFOCOM},
 year = {2012},
 pages = {1179--1187},
}

@article{1corser2016evaluating,
  title={Evaluating Location Privacy in Vehicular Communications and Applications},
  author={Corser, George P and Fu, Huirong and Banihani, Abdelnasser},
  journal={IEEE Transactions on Intelligent Transportation Systems},
  volume={17},
  number={9},
  pages={2658-2667},
  year={2016},
  publisher={IEEE}
}
@article{2zhang2016designing,
  title={On Designing Satisfaction-Ratio-Aware Truthful Incentive Mechanisms for k-Anonymity Location Privacy},
  author={Zhang, Yuan and Tong, Wei and Zhong, Sheng},
  journal={IEEE Transactions on Information Forensics and Security},
  volume={11},
  number={11},
  pages={2528--2541},
  year={2016},
  publisher={IEEE}
}
@article{3li2016privacy,
  title={Privacy-preserving Location Proof for Securing Large-scale Database-driven Cognitive Radio Networks},
  author={Li, Yi and Zhou, Lu and Zhu, Haojin and Sun, Limin},
  journal={IEEE Internet of Things Journal},
  volume={3},
  number={4},
  pages={563-571},
  year={2016},
  publisher={IEEE}
}
@article{4olteanu2016quantifying,
  title={Quantifying Interdependent Privacy Risks with Location Data},
  author={Olteanu, Alexandra-Mihaela and Huguenin, K{\'e}vin and Shokri, Reza and Humbert, Mathias and Hubaux, Jean-Pierre},
  journal={IEEE Transactions on Mobile Computing},
  year={2016},
  volume={PP},
  number={99},
  pages={1-1},
  publisher={IEEE}
}
@article{5yi2016practical,
  title={Practical Approximate k Nearest Neighbor Queries with Location and Query Privacy},
  author={Yi, Xun and Paulet, Russell and Bertino, Elisa and Varadharajan, Vijay},
  journal={IEEE Transactions on Knowledge and Data Engineering},
  volume={28},
  number={6},
  pages={1546--1559},
  year={2016},
  publisher={IEEE}
}
@article{6li2016privacy,
  title={Privacy Leakage of Location Sharing in Mobile Social Networks: Attacks and Defense},
  author={Li, Huaxin and Zhu, Haojin and Du, Suguo and Liang, Xiaohui and Shen, Xuemin},
  journal={IEEE Transactions on Dependable and Secure Computing},
  year={2016},
  volume={PP},
  number={99},
  publisher={IEEE}
}

@article{7murakami2016localization,
  title={Localization Attacks Using Matrix and Tensor Factorization},
  author={Murakami, Takao and Watanabe, Hajime},
  journal={IEEE Transactions on Information Forensics and Security},
  volume={11},
  number={8},
  pages={1647--1660},
  year={2016},
  publisher={IEEE}
}
@article{8zurbaran2015near,
  title={Near-Rand: Noise-based Location Obfuscation Based on Random Neighboring Points},
  author={Zurbaran, Mayra Alejandra and Avila, Karen and Wightman, Pedro and Fernandez, Michael},
  journal={IEEE Latin America Transactions},
  volume={13},
  number={11},
  pages={3661--3667},
  year={2015},
  publisher={IEEE}
}

@article{9tan2014anti,
  title={An anti-tracking source-location privacy protection protocol in wsns based on path extension},
  author={Tan, Wei and Xu, Ke and Wang, Dan},
  journal={IEEE Internet of Things Journal},
  volume={1},
  number={5},
  pages={461--471},
  year={2014},
  publisher={IEEE}
}

@article{10peng2014enhanced,
  title={Enhanced Location Privacy Preserving Scheme in Location-Based Services},
  author={Peng, Tao and Liu, Qin and Wang, Guojun},
  journal={IEEE Systems Journal},
  year={2014},
  volume={PP},
  number={99},
  pages={1-12},
  publisher={IEEE}
}
@article{11dewri2014exploiting,
  title={Exploiting service similarity for privacy in location-based search queries},
  author={Dewri, Rinku and Thurimella, Ramakrisha},
  journal={IEEE Transactions on Parallel and Distributed Systems},
  volume={25},
  number={2},
  pages={374--383},
  year={2014},
  publisher={IEEE}
}

@article{12hwang2014novel,
  title={A novel time-obfuscated algorithm for trajectory privacy protection},
  author={Hwang, Ren-Hung and Hsueh, Yu-Ling and Chung, Hao-Wei},
  journal={IEEE Transactions on Services Computing},
  volume={7},
  number={2},
  pages={126--139},
  year={2014},
  publisher={IEEE}
}
@article{13puttaswamy2014preserving,
  title={Preserving location privacy in geosocial applications},
  author={Puttaswamy, Krishna PN and Wang, Shiyuan and Steinbauer, Troy and Agrawal, Divyakant and El Abbadi, Amr and Kruegel, Christopher and Zhao, Ben Y},
  journal={IEEE Transactions on Mobile Computing},
  volume={13},
  number={1},
  pages={159--173},
  year={2014},
  publisher={IEEE}
}

@article{14zhang2014privacy,
  title={Privacy quantification model based on the Bayes conditional risk in Location-Based Services},
  author={Zhang, Xuejun and Gui, Xiaolin and Tian, Feng and Yu, Si and An, Jian},
  journal={Tsinghua Science and Technology},
  volume={19},
  number={5},
  pages={452--462},
  year={2014},
  publisher={TUP}
}

@article{15bilogrevic2014privacy,
  title={Privacy-preserving optimal meeting location determination on mobile devices},
  author={Bilogrevic, Igor and Jadliwala, Murtuza and Joneja, Vishal and Kalkan, K{\"u}bra and Hubaux, Jean-Pierre and Aad, Imad},
  journal={IEEE transactions on information forensics and security},
  volume={9},
  number={7},
  pages={1141--1156},
  year={2014},
  publisher={IEEE}
}
@article{16haghnegahdar2014privacy,
  title={Privacy Risks in Publishing Mobile Device Trajectories},
  author={Haghnegahdar, Alireza and Khabbazian, Majid and Bhargava, Vijay K},
  journal={IEEE Wireless Communications Letters},
  volume={3},
  number={3},
  pages={241--244},
  year={2014},
  publisher={IEEE}
}
@article{17malandrino2014verification,
  title={Verification and inference of positions in vehicular networks through anonymous beaconing},
  author={Malandrino, Francesco and Borgiattino, Carlo and Casetti, Claudio and Chiasserini, Carla-Fabiana and Fiore, Marco and Sadao, Roberto},
  journal={IEEE Transactions on Mobile Computing},
  volume={13},
  number={10},
  pages={2415--2428},
  year={2014},
  publisher={IEEE}
}
@article{18shokri2014hiding,
  title={Hiding in the mobile crowd: Locationprivacy through collaboration},
  author={Shokri, Reza and Theodorakopoulos, George and Papadimitratos, Panos and Kazemi, Ehsan and Hubaux, Jean-Pierre},
  journal={IEEE transactions on dependable and secure computing},
  volume={11},
  number={3},
  pages={266--279},
  year={2014},
  publisher={IEEE}
}
@article{19freudiger2013non,
  title={Non-cooperative location privacy},
  author={Freudiger, Julien and Manshaei, Mohammad Hossein and Hubaux, Jean-Pierre and Parkes, David C},
  journal={IEEE Transactions on Dependable and Secure Computing},
  volume={10},
  number={2},
  pages={84--98},
  year={2013},
  publisher={IEEE}
}
@article{20gao2013trpf,
  title={TrPF: A trajectory privacy-preserving framework for participatory sensing},
  author={Gao, Sheng and Ma, Jianfeng and Shi, Weisong and Zhan, Guoxing and Sun, Cong},
  journal={IEEE Transactions on Information Forensics and Security},
  volume={8},
  number={6},
  pages={874--887},
  year={2013},
  publisher={IEEE}
}
@article{21ma2013privacy,
  title={Privacy vulnerability of published anonymous mobility traces},
  author={Ma, Chris YT and Yau, David KY and Yip, Nung Kwan and Rao, Nageswara SV},
  journal={IEEE/ACM Transactions on Networking},
  volume={21},
  number={3},
  pages={720--733},
  year={2013},
  publisher={IEEE}
}
@article{22niu2013pseudo,
  title={Pseudo-Location Updating System for privacy-preserving location-based services},
  author={Niu, Ben and Zhu, Xiaoyan and Chi, Haotian and Li, Hui},
  journal={China Communications},
  volume={10},
  number={9},
  pages={1--12},
  year={2013},
  publisher={IEEE}
}
@article{23dewri2013local,
  title={Local differential perturbations: Location privacy under approximate knowledge attackers},
  author={Dewri, Rinku},
  journal={IEEE Transactions on Mobile Computing},
  volume={12},
  number={12},
  pages={2360--2372},
  year={2013},
  publisher={IEEE}
}
@inproceedings{24kanoria2012tractable,
  title={Tractable bayesian social learning on trees},
  author={Kanoria, Yashodhan and Tamuz, Omer},
  booktitle={Information Theory Proceedings (ISIT), 2012 IEEE International Symposium on},
  pages={2721--2725},
  year={2012},
  organization={IEEE}
}
@inproceedings{25farias2005universal,
  title={A universal scheme for learning},
  author={Farias, Vivek F and Moallemi, Ciamac C and Van Roy, Benjamin and Weissman, Tsachy},
  booktitle={Proceedings. International Symposium on Information Theory, 2005. ISIT 2005.},
  pages={1158--1162},
  year={2005},
  organization={IEEE}
}
@inproceedings{26misra2013unsupervised,
  title={Unsupervised learning and universal communication},
  author={Misra, Vinith and Weissman, Tsachy},
  booktitle={Information Theory Proceedings (ISIT), 2013 IEEE International Symposium on},
  pages={261--265},
  year={2013},
  organization={IEEE}
}
@inproceedings{27ryabko2013time,
  title={Time-series information and learning},
  author={Ryabko, Daniil},
  booktitle={Information Theory Proceedings (ISIT), 2013 IEEE International Symposium on},
  pages={1392--1395},
  year={2013},
  organization={IEEE}
}
@inproceedings{28krzakala2013phase,
  title={Phase diagram and approximate message passing for blind calibration and dictionary learning},
  author={Krzakala, Florent and M{\'e}zard, Marc and Zdeborov{\'a}, Lenka},
  booktitle={Information Theory Proceedings (ISIT), 2013 IEEE International Symposium on},
  pages={659--663},
  year={2013},
  organization={IEEE}
}
@inproceedings{29sakata2013sample,
  title={Sample complexity of Bayesian optimal dictionary learning},
  author={Sakata, Ayaka and Kabashima, Yoshiyuki},
  booktitle={Information Theory Proceedings (ISIT), 2013 IEEE International Symposium on},
  pages={669--673},
  year={2013},
  organization={IEEE}
}
@inproceedings{30predd2004consistency,
  title={Consistency in a model for distributed learning with specialists},
  author={Predd, Joel B and Kulkarni, Sanjeev R and Poor, H Vincent},
  booktitle={IEEE International Symposium on Information Theory},
  year={2004},
organization={IEEE}
}
@inproceedings{31nokleby2016rate,
  title={Rate-Distortion Bounds on Bayes Risk in Supervised Learning},
  author={Nokleby, Matthew and Beirami, Ahmad and Calderbank, Robert},
  booktitle={2016 IEEE International Symposium on Information Theory (ISIT)},
pages={2099-2103},
  year={2016},
organization={IEEE}
}

@inproceedings{32le2016imperfect,
  title={Are imperfect reviews helpful in social learning?},
  author={Le, Tho Ngoc and Subramanian, Vijay G and Berry, Randall A},
  booktitle={Information Theory (ISIT), 2016 IEEE International Symposium on},
  pages={2089--2093},
  year={2016},
  organization={IEEE}
}
@inproceedings{33gadde2016active,
  title={Active Learning for Community Detection in Stochastic Block Models},
  author={Gadde, Akshay and Gad, Eyal En and Avestimehr, Salman and Ortega, Antonio},
  booktitle={2016 IEEE International Symposium on Information Theory (ISIT)},
  pages={1889-1893},
  year={2016}
}
@inproceedings{34shakeri2016minimax,
  title={Minimax Lower Bounds for Kronecker-Structured Dictionary Learning},
  author={Shakeri, Zahra and Bajwa, Waheed U and Sarwate, Anand D},
  booktitle={2016 IEEE International Symposium on Information Theory (ISIT)},
  pages={1148-1152},
  year={2016}
}
@article{35lee2015speeding,
  title={Speeding up distributed machine learning using codes},
  author={Lee, Kangwook and Lam, Maximilian and Pedarsani, Ramtin and Papailiopoulos, Dimitris and Ramchandran, Kannan},
  booktitle={2016 IEEE International Symposium on Information Theory (ISIT)},
  pages={1143-1147},
  year={2016}
}
@article{36oneto2016statistical,
  title={Statistical Learning Theory and ELM for Big Social Data Analysis},
  author={Oneto, Luca and Bisio, Federica and Cambria, Erik and Anguita, Davide},
  journal={ieee CompUTATionAl inTelliGenCe mAGAzine},
  volume={11},
  number={3},
  pages={45--55},
  year={2016},
  publisher={IEEE}
}
@article{37lin2015probabilistic,
  title={Probabilistic approach to modeling and parameter learning of indirect drive robots from incomplete data},
  author={Lin, Chung-Yen and Tomizuka, Masayoshi},
  journal={IEEE/ASME Transactions on Mechatronics},
  volume={20},
  number={3},
  pages={1036--1045},
  year={2015},
  publisher={IEEE}
}
@article{38wang2016towards,
  title={Towards Bayesian Deep Learning: A Framework and Some Existing Methods},
  author={Wang, Hao and Yeung, Dit-Yan},
  journal={IEEE Transactions on Knowledge and Data Engineering},
  volume={PP},
  number={99},
  year={2016},
  publisher={IEEE}
}


%%%%%Informationtheoreticsecurity%%%%%%%%%%%%%%%%%%%%%%%




@inproceedings{Bloch2011PhysicalSecBook,
	title={Physical-Layer Security},
	author={M. Bloch and J. Barros},
	organization={Cambridge University Press},
	year={2011}
}



@inproceedings{Liang2009InfoSecBook,
	title={Information Theoretic Security},
	author={Y. Liang and H. V. Poor and S. Shamai (Shitz)},
	organization={Now Publishers Inc.},
	year={2009}
}


@inproceedings{Zhou2013PhysicalSecBook,
	title={Physical Layer Security in Wireless Communications},
	author={ X. Zhou and L. Song and Y. Zhang},
	organization={CRC Press},
	year={2013}
}

@article{Ni2012IEA,
	Author = {D. Ni and H. Liu and W. Ding and  Y. Xie and H. Wang and H. Pishro-Nik and Q. Yu},
	Journal = {IEA/AIE},
	Title = {Cyber-Physical Integration to Connect Vehicles for Transformed Transportation Safety and Efficiency},
	Year = {2012}}



@inproceedings{Ni2012Inproceedings,
	Author = {D. Ni, H. Liu, Y. Xie, W. Ding, H. Wang, H. Pishro-Nik, Q. Yu and M. Ferreira},
	Booktitle = {Spring Simulation Multiconference},
	Date-Added = {2016-09-04 14:18:42 +0000},
	Date-Modified = {2016-09-06 16:22:14 +0000},
	Title = {Virtual Lab of Connected Vehicle Technology},
	Year = {2012}}

@inproceedings{Ni2012Inproceedings,
	Author = {D. Ni, H. Liu, W. Ding, Y. Xie, H. Wang, H. Pishro-Nik and Q. Yu,},
	Booktitle = {IEA/AIE},
	Date-Added = {2016-09-04 09:11:02 +0000},
	Date-Modified = {2016-09-06 14:46:53 +0000},
	Title = {Cyber-Physical Integration to Connect Vehicles for Transformed Transportation Safety and Efficiency},
	Year = {2012}}


@article{Nekoui_IJIPT_2009,
	Author = {M. Nekoui and D. Ni and H. Pishro-Nik and R. Prasad and M. Kanjee and H. Zhu and T. Nguyen},
	Journal = {International Journal of Internet Protocol Technology (IJIPT)},
	Number = {3},
	Pages = {},
	Publisher = {},
	Title = {Development of a VII-Enabled Prototype Intersection Collision Warning System},
	Volume = {4},
	Year = {2009}}


@inproceedings{Pishro_Ganz_Ni,
	Author = {H. Pishro-Nik, A. Ganz, and Daiheng Ni},
	Booktitle = {Forty-Fifth Annual Allerton Conference on Communication, Control, and Computing. Allerton House, Monticello, IL},
	Date-Added = {},
	Date-Modified = {},
	Number = {},
	Pages = {},
	Title = {The capacity of vehicular ad hoc networks},
	Volume = {},
	Year = {September 26-28, 2007}}

@inproceedings{Leow_Pishro_Ni_1,
	Author = {W. L. Leow, H. Pishro-Nik and Daiheng Ni},
	Booktitle = {IEEE Global Telecommunications Conference, Washington, D.C.},
	Date-Added = {},
	Date-Modified = {},
	Number = {},
	Pages = {},
	Title = {Delay and Energy Tradeoff in Multi-state Wireless Sensor Networks},
	Volume = {},
	Year = {November 26-30, 2007}}


@misc{UMass-Trans,
title = {{UMass Transportation Center}},
note = {\url{http://www.umasstransportationcenter.org/}},
}


@inproceedings{Haenggi2013book,
	title={Stochastic geometry for wireless networks},
	author={M. Haenggi},
	organization={Cambridge Uinversity Press},
	year={2013}
}


\bibitem{mcpherson1987}A. McPherson, G. Gibson, H. Jara, U. Johann,
T. S. Luk, I. A. McIntyre, K. Boyer, and C. K. Rhodes, {Studies of
multiphoton production of vacuum-ultraviolet radiation in the rare
gases}, J. Opt. Soc. Am. B {\bf 4}, 595 (1987),

\bibitem{ferray1988}M. Ferray, A. L'Huillier, X. F.  Li, L. A. Lompre,
G. Mainfray, and C. Manus, {Multiple-harmonic conversion of 1064 nm
radiation in rare gases},  J. Phys. B: At. Mol. Opt. Phys. {\bf 21},
L31 (1988).

\bibitem{farkas1992} Gy. Farkas, Cs. T\'{o}th, S. D. Moustaizis,
N. A. Papadogiannis, and C. Fotakis, {Observation of
multiple-harmonic radiation induced from a gold surface by
picosecond neodymium-doped yttrium aluminum garnet laser pulses},
Phys. Rev. A {\bf 46}, R3605 (1992).

\bibitem{vonderlinde1995} D. von der Linde, T. Engers, G. Jenke,
P. Agostini, G. Grillon, E. Nibbering, A. Mysyrowicz, and
A. Antonetti, {Generation of high-order harmonics from solid
surfaces by intense femtosecond laser pulses}, Phys. Rev. A {\bf    52}, R25 (1995).

\bibitem{prl05} G. P. Zhang, {Optical high harmonic generations in
$\rm C_{60}$}, \prl {\bf 95}, 047401 (2005).

\bibitem{ganeev2009a} R. Ganeev, L. Bom, J. Abdul-Hadi, M. Wong,
J. Brichta, V. Bhardwaj, and T. Ozaki, {Higher-order harmonic
generation from fullerene by means of the plasma harmonic method},
Phys. Rev. Lett. {\bf 102}, 013903 (2009).

\bibitem{ghimire2011} S. Ghimire, E. Sistrunk, P. Agostini,
L. F. DiMauro, and D. A. Reis, {Observation of high-order harmonic
generation in a bulk crystal}, Nat. Phys. {\bf 7}, 138
(2011).

\bibitem{luu2015}T. T. Luu, M. Garg, S. Yu. Kruchinin, A. Moulet,
M. Th. Hassan, and E. Goulielmakis, {Extreme ultraviolet
high-harmonic spectroscopy of solids}, Nature {\bf 521}, 498 (2015).

\bibitem{garg2016}M. Garg, M. Zhan, T. T. Luu, H. Lakhotia,
T. Klostermann, A. Guggenmos, and E. Goulielmakis, {Multi-petahertz
electronic metrology}, Nature {\bf 538}, 359 (2016).

\bibitem{nd} G. Ndabashimiye, S. Ghimire, M. Wu, D.  A. Browne,
K. J. Schafer, M.  B. Gaarde, and D. A. Reis, {Solid-state harmonics
beyond the atomic limit}, Nature {\bf 534}, 520 (2016).

\bibitem{vampa2015a}G. Vampa, T. J. Hammond, N. Thire, B. E. Schmidt,
F. Legare, C. R. McDonald, T. Brabec, D. D. Klug, and P. B. Corkum,
{All-optical reconstruction of crystal band structure},
Phys. Rev. Lett.  {\bf 115}, 193603 (2015).

\bibitem{vampa2015b} G. Vampa, C. R. McDonald, G. Orlando, P. B. Corkum
and T. Brabec, {Semiclassical analysis of high harmonic generation
in bulk crystals}, Phys. Rev. B {\bf 91}, 064302 (2015).

\bibitem{prb20} G. P. Zhang and Y. H. Bai, {High-order harmonic
generation in solid C$_{60}$}, Phys. Rev. B {\bf 101}, 081412(R)
(2020).

\bibitem{jia2019} L. Jia, Z. Zhang, D. Z. Yang, M. S. Si, G. P. Zhang,
and Y. S. Liu, {High harmonic generation in magnetically doped
topological insulators}, Phys. Rev. B {\bf 100}, 125144 (2019).

\bibitem{baykusheva2021}D. Baykusheva, A. Chacon, D. Kim, D. E. Kim,
D.  A. Reis, and S. Ghimire, {Strong-field physics in
three-dimensional topological insulators}, Phys. Rev. A {\bf 103},
023101 (2021).

\bibitem{jia2020b} L. Jia , Z. Zhang, D. Z. Yang, M. S. Si, and
G. P. Zhang, {Probing magnetic configuration-mediated topological
phases via high harmonic generation in MnBi$_2$Te$_4$},
Phys. Rev. B {\bf 102}, 174314 (2020).

\bibitem{jia2020a} L. Jia, Z. Zhang, D. Z. Yang, Y. Liu, M. S. Si,
G. P. Zhang, and Y. S. Liu, {Optical high-order harmonic generation
as a structural characterization tool}, Phys. Rev. B {\bf 101},
144304 (2020).

\bibitem{lysne2020}M. Lysne, Y. Murakami, M. Sch\"uler, and P. Werner,
{High-harmonic generation in spin-orbit coupled systems},
Phys. Rev. B {\bf 102}, 081121(R) (2020).

\bibitem{you2017} Y. S. You, D. A. Reis and S. Ghimire, {Anisotropic
high-harmonic generation in bulk crystals}, Nature Phys. {\bf 13}, 345
(2017).

\bibitem{nc18}G. P. Zhang, M. S. Si, M. Murakami, Y. H. Bai, and  T.
F. George, {Generating high-order optical and spin harmonics
from ferromagnetic monolayers}, Nat. Commun. {\bf 9}, 3031 (2018).

\bibitem{takayoshi2019}S. Takayoshi, Y. Murakami, and P. Werner,
{High-harmonic generation in quantum spin systems}, Phys. Rev. B
{\bf 99}, 184303, (2019).

\bibitem{Tancogne-Dejean2018} N. Tancogne-Dejean, M. A. Sentef, and
A. Rubio, {Ultrafast modification of Hubbard $U$ in a strongly
correlated material: {\it ab initio} high-harmonic generation in
NiO},  Phys. Rev. Lett. {\bf 121}, 097402 (2018).

\bibitem{eric} E. Beaurepaire, J. C. Merle, A. Daunois, and J.-Y. Bigot,
{Ultrafast spin dynamics in ferromagnetic nickel},
Phys.  Rev. Lett. {\bf 76}, 4250 (1996).

\bibitem{ourreview}G. P. Zhang, W. H\"ubner, E.  Beaurepaire, and
J.-Y. Bigot, Laser-induced ultrafast demagnetization: Femtomagnetism,
A new frontier? Topics Appl. Phys.  {\bf 83}, 245 (2002).

\bibitem{eric2004} E. Beaurepaire, G. M. Turner, S. M. Harrel,
M. C. Beard, J.-Y. Bigot, and C. A. Schmuttenmaer, Coherent
terahertz emission from ferromagnetic films excited by femtosecond
laser pulses, Appl. Phys. Lett. {\bf 84}, 3465 (2004).

\bibitem{li2018}G. Li, R. V. Mikhaylovskiy, K. A. Grishunin,
J. D. Costa, Th. Rasing, and A. V. Kimel, {Laser induced THz
emission from femtosecond photocurrents in {Co/ZnO/Pt} and {Co/Cu/Pt}
multilayers},  J. Phys. D: Appl. Phys., {\bf 51}, 134001 (2018).

\bibitem{zhang2020}W. Zhang, P. Maldonado, Z. Jin,
T. S. Seifert, J. Arabski, G. Schmerber, E.
Beaurepaire, M. Bonn, T. Kampfrath, P. M. Oppeneer, and
D. Turchinovich, {Ultrafast terahertz magnetometry},
Nat. Commun. {\bf 11}, 4247 (2020).

\bibitem{zhang2002}Q. Zhang, A. V. Nurmikko, A. Anguelouch, G. Xiao and
A. Gupta, {Coherent magnetization rotation and phase control by
ultrashort optical pulses in CrO$_2$ thin films},
Phys. Rev. Lett. {\bf 89}, 177402 (2002).

\bibitem{muller2009}G. M. M\"uller, J. Walowski, M.  Djordjevic,
G.-X. Miao, A. Gupta, A. V. Ramos, K. Gehrke, V. Moshnyaga,
K. Samwer, J. Schmalhorst, A. Thomas, A. H\"utten, G.  Reiss,
J. S. Moodera, and M. M\"unzenberg, {Spin polarization in
half-metals probed by femtosecond spin excitation},
Nat. Mater. {\bf 8}, 56 (2009).

\bibitem{schwarz1986}K.-H.  Schwarz, {CrO$_2$ predicted as a
half-metallic ferromagnet}, J. Phys. F: Met. Phys. {\bf 16}, L211
(1986).

\bibitem{coey}J. M. D. Coey, {Magnetism and Magnetic Materials},
Cambridge University Press (2010).

\bibitem{soulen1998} R. J. Soulen Jr., J. M. Byers, M. S. Osofsky,
B. Nadgorny, T. Ambrose, S. F. Cheng, P. R. Broussard, C. T. Tanaka,
J. Nowak, J. S. Moodera, A. Barry, and J. M. D. Coey, {Measuring the
spin polarization of a metal with a superconducting point
contact}, Science {\bf 282}, 85 (1998).

\bibitem{jpcm16}G. P. Zhang, Y. H. Bai, and T. F. George, {Ultrafast
reduction of exchange splitting in ferromagnetic nickel}, J. Phys.:
Condens. Matt. {\bf 28}, 236004 (2016).

\bibitem{wien2k} P. Blaha, K. Schwarz, G. K. H. Madsen, D. Kvasnicka,
and J. Luitz, WIEN2k, An Augmented Plane Wave + Local Orbitals
Program for Calculating Crystal Properties (Karlheinz Schwarz,
Techn. Universit\"at Wien, Austria, 2001).

\bibitem{blaha2020}P. Blaha, K. Schwarz, F. Tran, R. Laskowski,
G. K. H. Madsen and L. D. Marks, {WIEN2k: An APW+lo program for
calculating the properties o f solids}, J. Chem. Phys. {\bf 152},
074101 (2020).

\bibitem{np09}G. P. Zhang, W. H\"ubner, G. Lefkidis, Y. Bai, and
T. F. George, {Paradigm of the time-resolved magneto-optical Kerr
effect for femtosecond magnetism}, {Nat. Phys.} {\bf 5}, 499
(2009).

\bibitem{prb09} G. P. Zhang, Y. H. Bai, and T. F. George, {Energy- and
crystal momentum-resolved study of laser-induced femtosecond
magnetism}, Phys. Rev. B {\bf 80}, 214415 (2009).

\bibitem{pbe}J. P. Perdew, K. Burke, and M. Ernzerhof, {Generalized
gradient approximation made simple}, Phys. Rev. Lett. {\bf 77}, 3865
(1996).

\bibitem{prb19} G. P. Zhang and Y. H. Bai, {Magic high-order harmonics
from a quasi-one-dimensional hexagonal solid}, Phys. Rev. B {\bf
99}, 094313 (2019).

\bibitem{shen}Y. R. Shen, {\it The Principles of Nonlinear Optics},
John Wiley \& Sons, Inc., Hoboken, New Jersey (2003).

\bibitem{kittel}C. Kittel, {\it Introduction to Solid State Physics},
7th Ed. John Wiley \& Sons, Inc., New York (1996).

\bibitem{ourbook}G. P. Zhang, G. Lefkidis, M. Murakami, W.  H\"ubner
and T. F. George, {\it Introduction to Ultrafast Phenomena: From
Femtosecond Magnetism to High-Harmonic Generation}, CRC Press
(2020).

\bibitem{sorantin1992}P. I. Sorantin and K. Schwarz, {Chemical bonding
in rutile-type compounds}, Inorg. Chem {\bf 31}, 567 (1992).

\bibitem{mazin1999}I. I. Mazin, D. J. Singh, and C.  Ambrosch-Draxl,
{Transport, optical, and electronic properties of the half-metal
CrO$_2$},  Phys. Rev. B {\bf 59}, 411 (1999).

\bibitem{zhang2006}Q. Zhang, A. V. Nurmikko, G. X. Miao, G. Xiao, and
A. Gupta, {Ultrafast spin-dynamics in half-metallic CrO$_2$ thin
films}, Phys. Rev. B {\bf 74}, 064414 (2006).

\bibitem{mukamel}S. Mukamel, {\it Principles of Nonlinear Optical
Spectroscopy}, Oxford University Press, New York (1995).

\bibitem{vampa2014}G. Vampa, C. R. McDonald, G. Orlando, D. D. Klug,
P. B. Corkum and T. Brabec, {Theoretical analysis of high-harmonic
generation in solids}, Phys. Rev. Lett. {\bf 113}, 073901 (2014).

\bibitem{neufeld2019}O. Neufeld, D. Podolsky, and O. Cohen, {Floquet
group theory and its application to selection rules in harmonic
generation}, Nat. Comm. {\bf 10}, 405 (2019).

\bibitem{cloud1962}W. H. Cloud, D. S. Schreiber, and K. R. Babcock,
{X-ray and magnetic studies of CrO$_2$ single crystals},
J. Appl. Phys. {\bf 33}, 1193 (1962).

\bibitem{singh2009}G. P. Singh, S. Ram, J. Eckert, and H.-J. Fecht,
{Synthesis and morphological stability in CrO$_2$ single crystals of a
half-metallic ferromagnetic compound}, J. Phys.: Conf. Ser. {\bf
144}, 012110 (2009).

\bibitem{pathak2009}M. Pathak, H. Sims, K.  B. Chetry, D. Mazumdar,
P. R. LeClair, G.  J. Mankey, W. H. Butler, and A. Gupta,
{Robust room-temperature magnetism of (110) CrO$_2$ thin films},
Phys. Rev. B {\bf 80}, 212405 (2009).

\bibitem{thamer1956} B. J. Thamer, R. M. Douglass and E. Staritzky,
{The thermal decomposition of aqueous chromic acid and some
properties of the resulting solid phases}, J. Am. Chem. Soc. {\bf
79}, 547 (1957).

\end{thebibliography}

%\newpage
\clearpage



\begin{table}
\caption{Optimized structural parameters of \Cre.  Cr
  atoms are at $(2a)$ positions, (0,0,0) and
  $(\frac{1}{2},\frac{1}{2},\frac{1}{2})$, and O atoms at $(4f)$
  positions, ($u$,$u$,0), ($\bar{u}$,$\bar{u}$,0), ($\frac{1}{2}+u,
  \frac{1}2-u, \frac{1}2$), and
  $(\frac{1}{2}-u,\frac{1}{2}+u,\frac{1}{2})$.  Our theoretical values
  are listed under ``Theory'', with the experimental ones in
  parenthesis.  Our theoretical $u$ is 0.303, in good agreement with
  the experimental $u=0.301\pm 0.004$ \cite{cloud1962}.  The
  experiment lattice constants are $a=4.4218~\rm\AA$ and
  $c=2.9182~\rm\AA$ \cite{cloud1962}, $a=4.423~\rm \AA$ and
  $c=2.918~\rm \AA$ \cite{singh2009}, $a=4.421~\rm\AA$ and
  $c=2.916~\rm\AA$ \cite{pathak2009}, and $a=4.421~\rm\AA$ and
  $c=2.916~\rm\AA$ \cite{thamer1956}.  We adopt the experimental
  lattice constant from Cloud \et \cite{cloud1962}.  }
\begin{tabular}{c|c|c|c|c|c|c}
\hline
\hline
Atom & \multicolumn{2}{c|} {$x$} &  \multicolumn{2}{c|}{$y$}& \multicolumn{2}{c} {$z$} \\
\hline
      &~~Theory~~  &Experiment     &~~Theory~~&Experiment &~~Theory~~ &Experiment\\
Cr$_1$ &0 &(0)  & 0  &(0) &0 &(0)\\
Cr$_2$ & $\frac{1}2$ &($\frac{1}2$) & $\frac{1}{2}$ &($\frac{1}2$) & $\frac{1}{2}$
& ($\frac{1}2$) \\
O$_1$ &0.303 &(0.301) &0.303 &(0.301) &0& (0) \\
\hline\hline
\end{tabular}
\label{tab}
\end{table}



\begin{figure}





%\includegraphics[angle=0,width=0.3\columnwidth]{/home/gpzhang/doe/paper/hhg/cro2/qn/cro2a/cro2a.eps}

%\includegraphics[angle=0,width=0.3\columnwidth]{cro2a.eps}


%\includegraphics[angle=0,width=0.8\columnwidth]{/home/gpzhang/doe/paper/hhg/cro2/obsi/cro2dos1/Results/pdos.eps}
%\includegraphics[angle=0,width=0.8\columnwidth]{/home/gpzhang/doe/paper/hhg/cro2/qn/cro2a/pdos.eps}
\includegraphics[angle=0,width=1\columnwidth]{pdos.eps}
\caption{ (a) The partial density of states (PDOS) for Cr-$3d$ and
  O-$2p$ states for the majority (thick solid and dotted lines on the
  positive axis) and minority spins (solid and dashed lines on the
  negative axis). The vertical dashed line denotes the Fermi energy.
  (b) {Logarithmic scale of harmonics from the majority and
    minority spin channels.} Here we use the laser duration of 64
  cycles of laser period. The photon energy is 0.4 eV and the laser
  field amplitude $A$ is 0.03 $\rm Vfs/\AA$. A circularly polarized
  pulse is used.  }
\label{fig1}
\label{pdos}
\end{figure}



%Add the real time evolution of the majority and minority spins. 




\begin{figure}
%\includegraphics[angle=0,width=0.8\columnwidth]{/home/gpzhang/doe/paper/hhg/cro2/cori/cro2sp2O1/60fs.0.4ev.rc/up/time.eps}
\includegraphics[angle=0,width=0.8\columnwidth]{energy.eps}
\caption{ Dependence of harmonics on photon energies $\hbar\omega$ (a)
  0.4 eV, (b) 0.8 eV and (c) 1.6 eV. We choose the vector potential
  $A_0=0.03 \rm Vfs/\AA$ and the duration of the laser pulse is 64
  cycles of laser period. Harmonics from the minority channel are
  plotted on the negative axis.  (a) At 0.4 eV, the majority and
  minority spin channels produce a similar signal strength at the 1st
  and 3rd orders, but differ a lot at the 5th order, where the
  majority dominates over the minority spin channel. This presents an
  opportunity to generate harmonic mainly from a single spin
  channel. As we increase photon energy to 0.8 eV (b), the minority
  5th harmonic increases its signal strength, so it starts to compete
  with that from the majority channel. This trend continues as we
  increase the photon energy to 1.6 eV (c), where the majority and
  minority channels have a similar strength.  (d) Ratio of the
  majority harmonic signal to the minority harmonic signal as a
  function of photon energy for the first (circles), third (squares)
  and fifth (diamonds) harmonics. At 0.4 eV, the ratio for the fifth
  harmonic reaches 67.  }
\label{fig2}
\end{figure}


\begin{figure}
%\includegraphics[angle=0,width=0.8\columnwidth]{/home/gpzhang/doe/paper/hhg/cro2/qn/compare.eps}
\includegraphics[angle=0,width=0.8\columnwidth]{intensity.eps}
\caption{Dependence of harmonic signals on the laser field amplitude
  $A_0$.  The photon energy is 0.4 eV, and pulse duration is 64 cycles
  of the laser period. Here we use a dense $k$ mesh of $29\times 29
  \times 45$. {(a) Third-order harmonics as a function of $A_0$
    for spin up (filled circles) and spin down (filled boxes), which
    are fitted to a power of $A_0^{2.37}$ and $A_0^{2.99}$.  (b) Third
    harmonic from the spin up channel. The circles, boxes, diamonds,
    upper triangles and left triangles denote $A_0=0.01$, 0.02, 0.03,
    0.04 and 0.05 $\rm Vfs/\AA$, respectively.  (c) Third harmonic
    from the spin down channel. All the legends have the same meaning
    as those in (b).  (d) Fifth-order harmonics as a function of $A_0$
    for spin up (filled circles) and spin down (filled boxes). The
    spin up signal can be fitted to a power of $A_0^{4.32}$.  (e) The
    fifth harmonic from the spin up channel. Here all the signals are
    multiplied by 100 for an easy view. All the legends have the same
    meaning as those in (b).  (f) Fifth harmonic from the spin down
    channel.  Here all the signals are multiplied by 1000. All the
    legends have the same meaning as those in (b).} }
\label{fig3}
\end{figure}




\begin{figure}
%\includegraphics[angle=0,width=0.8\columnwidth]{/home/gpzhang/doe/paper/hhg/cro2/qn/compare.eps}
%\includegraphics[angle=0,width=0.8\columnwidth]{/home/gpzhang/doe/paper/hhg/cro2/cori/cro2B/intensity/compare.eps}
\includegraphics[angle=0,width=0.8\columnwidth]{compare.eps}
\caption{{Influence of relaxation times and laser polarization on
    harmonics.  (a)-(d) use a different set of relaxation times,
    $T_1=400$ and $T_2=200$ fs with laser polarization within the $xy$
    plane.  (a) and (c) show the majority and minority harmonic
    signals along the $x$ direction respectively, which should be
    compared with those in Fig. \ref{fig1}(b) with $T_1=200$ and
    $T_2=100$ fs.  (b) and (d) show the majority and minority harmonic
    signals along the $y$ direction, respectively. (e)-(h) use the
    laser polarization in the $yz$ plane with $T_1=400$ and $T_2=200$
    fs.  (e) and (g) show the majority and minority harmonic signals
    along the $y$ direction, respectively.  (f) and (h) show the
    majority and minority harmonic signals along the $z$ direction,
    respectively.  For all the figures, the photon energy is 0.4 eV,
    the pulse duration is 64 cycles of the laser period and the laser
    field amplitude $A_0$ is 0.03 $\rm Vfs/\AA$.  } }
\label{fig4}
\end{figure}



\end{document}

