\documentclass[11pt,a4paper]{article}
\usepackage{acl2020}
\usepackage{times}
\usepackage{graphicx}
\usepackage{wrapfig}
\usepackage{shortcuts}
% \usepackage{times}
\usepackage{dsfont}
\usepackage{latexsym}
\usepackage[linesnumbered,algoruled,boxed,noend]{algorithm2e}
\usepackage{amssymb}
\usepackage{amsmath}
\usepackage{url} 
\SetKwInOut{Parameter}{parameter}
\usepackage{enumitem}
\usepackage{mathtools}
\usepackage{cleveref}
\usepackage{multirow}
\usepackage{hhline}
\usepackage[export]{adjustbox}
\usepackage{booktabs,array}
% \usepackage[ruled,noend]{algorithm2e}
\usepackage{amsmath, bm}
\usepackage{color, colortbl}
\usepackage{subfig}
\SetKwInput{KwInput}{Input}
\SetKwInput{KwOutput}{Output}
\newcolumntype{P}[1]{>{\centering\arraybackslash}p{#1}}
\usepackage{booktabs,makecell,tabularx} 
\setitemize{noitemsep,topsep=0pt,parsep=0pt,partopsep=0pt}
\let\oldnl\nl% Store \nl in \oldnl
\definecolor{Gray}{gray}{0.85}
\definecolor{LightCyan}{rgb}{0.88,1,1}
\definecolor{FigCsqaOrange}{RGB}{236, 215, 192}
\definecolor{FigRsBlue}{RGB}{191, 207, 255}
\newcommand{\nonl}{\renewcommand{\nl}{\let\nl\oldnl}}
\newcommand{\comm}[1]{{\nonl{\scriptsize{{\color{gray}{/*~#1}~*/}}}}}
\usepackage{pifont}% http://ctan.org/pkg/pifont
\newcommand{\cmark}{\ding{51}}%
\newcommand{\xmark}{\ding{55}}%
\aclfinalcopy % Uncomment for camera-ready version, but NOT for submission.
\def\aclpaperid{1126}
\usepackage{epigraph}

% \epigraphsize{\small}% Default
\setlength\epigraphwidth{7cm}
\setlength\epigraphrule{0pt}

\usepackage{etoolbox}
\patchcmd{\epigraph}{\@epitext{#1}}{\itshape\@epitext{#1}}{}{}

\newlength{\Width}%
\newlength{\DepthReference}
\settodepth{\DepthReference}{g}
\newlength{\HeightReference}
\settoheight{\HeightReference}{T}
\newcommand{\MyColorBox}[2][red]%
{%
    \settowidth{\Width}{#2}%
    %\setlength{\fboxsep}{0pt}%
    \colorbox{#1}%
    {%      
        \raisebox{-\DepthReference}%
        {%
                \parbox[b][\HeightReference+\DepthReference][c]{\Width}{\centering#2}%
        }%
    }%
}
\setlength{\fboxsep}{1pt}

\renewcommand{\baselinestretch}{1} 
\newcommand{\yl}[1]{\textcolor{purple}{Yuchen: #1}} % William's comment
\begin{document}

\title{\vspace*{-0.5in}
{{\small \hfill ACL-IJCNLP 2021 Findings}\\
\vspace*{.25in}}
RiddleSense: Reasoning about Riddle Questions \\ Featuring Linguistic Creativity and Commonsense Knowledge}


\author{
Bill Yuchen Lin \quad Ziyi Wu\quad Yichi Yang\quad  Dong-Ho Lee\quad Xiang Ren\\
\texttt{\{yuchen.lin,ziyiwu,yichiyan,dongho.lee,xiangren\}@usc.edu}\\
Department of Computer Science and Information Sciences Institute,  \\ University of Southern California\\
}

% For research notes, remove the comment character in the line below.
% \researchnote

\maketitle



\begin{abstract}
\begin{abstract}

Visual perception tasks often require vast amounts of labelled data, including 3D poses and image space segmentation masks. The process of creating such training data sets can prove difficult or time-intensive to scale up to efficacy for general use. Consider the task of pose estimation for rigid objects. Deep neural network based approaches have shown good performance when trained on large, public datasets. However, adapting these networks for other novel objects, or fine-tuning existing models for different environments, requires significant time investment to generate newly labelled instances. Towards this end, we propose ProgressLabeller as a method for more efficiently generating large amounts of 6D pose training data from color images sequences for custom scenes in a scalable manner. ProgressLabeller is intended to also support transparent or translucent objects, for which the previous methods based on depth dense reconstruction will fail.
We demonstrate the effectiveness of ProgressLabeller by rapidly create a dataset of over 1M samples with which we fine-tune a state-of-the-art pose estimation network in order to markedly improve the downstream robotic grasp success rates. Progresslabeller is open-source at \href{https://github.com/huijieZH/ProgressLabeller}{https://github.com/huijieZH/ProgressLabeller}

\end{abstract}
\end{abstract}

\section{Introduction}\label{sec:intro}
\epigraph{\normalsize ``\textit{ \textbf{The essence of a riddle is to express true facts under impossible combinations.}}"}{\normalsize--- \textit{Aristotle}, \textit{Poetics} (350 BCE)\vspace{0pt}}

\noindent
A \textit{riddle} is a puzzling question about {concepts} in our everyday life.
% , and we which one needs common sense to reason about.
For example, a riddle might ask ``\textit{My life can be measured in hours. I serve by being devoured. Thin, I am quick. Fat, I am slow. Wind is my foe. What am I?}''~
The correct answer ``\textit{candle},'' is reached by considering a collection of \textit{commonsense knowledge}:
{a candle can be lit and burns for a few hours; a candle's life depends upon its diameter; wind can extinguish candles, etc.}
\begin{figure}[t]
	\centering 
	\includegraphics[width=1\linewidth]{riddle_intro_final.pdf}
	\caption{ 
    The top example is a trivial commonsense question from the CommonsenseQA~\cite{Talmor2018CommonsenseQAAQ} dataset. 
    The two bottom examples are from our proposed \textsc{RiddleSense} challenge.
    The right-bottom question is a descriptive riddle that implies multiple commonsense facts about \textit{candle}, and it needs understanding of figurative language such as metaphor;
    The left-bottom one additionally needs counterfactual reasoning ability to address the \textit{`but-no'} cues. 
    These riddle-style commonsense questions  require NLU systems to have higher-order reasoning skills with the understanding of creative language use.
	}
	\label{fig:intro} 
\end{figure}

It is believed that the \textit{riddle} is one of the earliest forms of oral literature,
which can be seen as a formulation of thoughts about common sense, a mode of association between everyday concepts, and a metaphor as higher-order use of natural language~\cite{hirsch2014poet}.
Aristotle stated in his \textit{Rhetoric} (335-330 BCE) that good riddles generally provide satisfactory metaphors for rethinking common concepts in our daily life.
He also pointed out in the \textit{Poetics} (350 BCE): ``the essence of a riddle is to express true facts under impossible combinations,'' which suggests that solving riddles is a nontrivial  reasoning task.

Answering riddles is indeed a challenging cognitive process as it requires \textit{complex} {commonsense reasoning skills}.
% which we refer to \textit{higher-order commonsense reasoning}. 
% A successful riddle-solving model should be able to reason with \textit{multiple pieces} of commonsense facts, as 
A riddle can describe \textit{multiple pieces} of commonsense knowledge with \textit{figurative} devices such as metaphor and personification (e.g., ``wind is my \underline{foe} $\xrightarrow[]{}$ \textit{extinguish}'').
% , as shown by the examples in Figure~\ref{fig:intro}.
%%%
Moreover, \textit{counterfactual thinking} is also necessary for answering many riddles such as ``\textit{what can you hold in your left hand \underline{but not} in your right hand? $\xrightarrow[]{}$ your right elbow.}''
These riddles with \textit{`but-no'} cues require that models use counterfactual reasoning ability to consider possible solutions for situations or objects that are seemingly impossible at face value.
This \textit{reporting bias}~\cite{gordon2013reporting} makes riddles a more difficult type of commonsense question for pretrained language models to learn and reason.
% In addition, the model needs to associate commonsense knowledge with the creative use of language in descriptions, which may have figurative devices such as metaphor and personification (e.g., ``wind is my \underline{foe} $\xrightarrow[]{}$ \textit{extinguish}''). 
%For instance, one needs to know that devour
% Thus, a riddle here can be seen as a complex commonsense question that tests higher-order reasoning ability with creativity.
In contrast, \textit{superficial} commonsense questions such as ``\textit{What home entertainment equipment requires cable?}'' in  CommonsenseQA~\cite{Talmor2018CommonsenseQAAQ} are more straightforward and explicitly stated.
We illustrate this comparison in Figure~\ref{fig:intro}.


In this paper,
we introduce the \textsc{RiddleSense} challenge 
to study the task of answering riddle-style commonsense questions\footnote{We use ``riddle'' and ``riddle-style commonsense question'' interchangeably in this paper.} requiring \textit{creativity}, \textit{counterfactual thinking} and \textit{complex commonsense reasoning}.
\textsc{RiddleSense} is presented as a \textit{multiple-choice question answering} task where a model selects one of five answer choices to a given riddle question as its predicted answer, as shown in Fig.~\ref{fig:intro}.
We construct the dataset by first crawling from several free websites featuring large collections of human-written riddles and then aggregating, verifying, and correcting these examples using a combination of human rating and NLP tools to create a dataset consisting of 5.7k high-quality examples.
Finally, we use \textit{Amazon Mechanical Turk} to crowdsource quality distractors to create a challenging benchmark.
We show that our riddle questions are more challenging than {CommonsenseQA} by analyzing graph-based statistics over ConceptNet~\cite{Speer2017ConceptNet5A}, a large knowledge graph for common sense reasoning.

% The distractors for the training data are automatically generated from ConceptNet and language models while the distractors for the dev and the test sets are crowd-sourced from Amazon Mechanical Turk (AMT).
% Through data analysis based on graph connectivity, 




Recent studies have demonstrated that
 fine-tuning large pretrained language models, such as {BERT}~\cite{Devlin2019}, RoBERTa, and ALBERT~\cite{Lan2020ALBERT}, can achieve strong results on current commonsense reasoning benchmarks.
Developed on top of these language models, graph-based language reasoning models such as KagNet~\cite{kagnet-emnlp19} and MHGRN~\cite{feng2020scalable} show superior performance. 
Most recently, UnifiedQA~\cite{khashabi2020unifiedqa} proposes to unify different QA tasks and train a text-to-text model for learning from all of them, which achieves state-of-the-art performance on many commonsense benchmarks.

To provide a comprehensive benchmarking analysis, we systematically compare the above methods.
Our experiments reveal that while humans achieve 91.33\% accuracy on \textsc{riddlesense}, the best language models can only achieve 68.80\% accuracy, suggesting that there is still much room for improvement in the field of solutions to complex commonsense reasoning questions with language models.
% We also provide error analysis to better understand the limitation of current methods.
We believe the proposed \textsc{RiddleSense} challenge suggests productive future directions for machine commonsense reasoning as well as the understanding of higher-order and creative use of natural language.


% (previous state-of-the-art on \texttt{CommonsenseQA} (56.7\%)).
% However, there still exists a large gap between performance of said baselines and human performance.
% we show that the questions in RiddleSense is significantly more challenging, in terms of the length of the paths from question concepts and answer concepts.


%Apart from that, current pre-trained language models (e.g., BERT~\cite{}, RoBERTa~\cite{}, etc.) and commonsense-reasoning models (e.g., KagNet~\cite{}), can be easily adapted to work for this format with minimal modifications. 


%Note that these auto-generated distractors may be still easy for , which could diminish the testing ability of the dataset.
%We design an ader filtering method to get rid of the false negative   and control the task difficulty. 
% To strengthen the task, we propose an adversarial cross-filtering method to remove the distractors that ineffectively mislead the selected base models.
% Finally, we use human efforts to inspect the distractors and remove false negative ones, to make sure that all distractors either does not make sense or much less plausible than the correct answers.
%Introducing these fine-tuned models is inspired by the adversarial filtering algorithms~\cite{}, which can effectively reduce the  bias inside datasets for creating a more reliable benchmark.  



%Those distractors are explicitly annotated by human experts such that they are close to the meaning of 
%The main idea is to use multiple trainable generative models for learning to generate answers in a cross-validation style. 
%The wrong predictions
%Simply put, for every step, we use a large subset of the riddles and their current options ot learn multiple models for answering the remaining riddles via generation.
%After each step, we consolidate the 


% In the distantly supervised learning, we use the definition of concepts (i.e., glossary) of \textit{Wiktionary}\footnote{\url{https://www.wiktionary.org/}} to create riddles with answers as training data. 
% In the transfer learning setting, we aim to test the transferability of models across relevant datasets, such as CommonsenseQA~\cite{Talmor2018CommonsenseQAAQ}.

% We believe the \textsc{RiddleSense} task can benefit multiple communities in natural language processing. 
% First, the commonsense reasoning community can use \textsc{RiddleSense} as a new space to evaluate their reasoning models. The \textsc{RiddleSense} focuses on more complex and creative commonsense questions, which will encourage them to propose more higher-order commonsense reasoning models. 
% Second, \textsc{RiddleSense} is an NLU 
% task similar to those in the GLUE~\cite{wang2018glue} and SuperGLUE~\cite{wang2019superglue} leaderboard that can serve as a benchmark for testing various pre-trained language models.
% Last but not the least, as our task shares the similar format with many open-domain question answering tasks like \textit{Natural Questions}~\cite{kwiatkowski2019natural}, researchers in QA area may be also interested in \textsc{RiddleSense}. 






% \section{Problem Formulation}\label{sec:problem}
% \input{2_problem.tex}


% \section{Task Formulation}
% \label{sec:problem}
% \input{2_problem.tex}
 
% The \textsc{RiddleSense} is a $K$-way multi-choice question answering task. Given a question (i.e., a riddle) $q$, there are $K$ different choices $\{c_1, \dots, c_K\}$, where only one of them is the correct choice and the others are distractors.
% A model needs to rank all choices and select the best one.
% The evaluation metric will be the accuracy.
% Using external knowledge sources via distantly supervised learning and/or transfer learning is allowed and encouraged for solving \textsc{RiddleSense}.


% The correct answer can be either a shorter concept or phrase or a longer explanation to the situation in the riddle.

% \begin{itemize}
% 	\item An example for a \textbf{short-answer riddle}: ``\textit{It has five fingers. It's not a part of any animal. What is it?}'' {Choices:  ``A hand'' (\xmark); ``A nail'' (\xmark), ``A glove.'' (\cmark) ; ...}
% 	\item An example for a \textbf{long-answer riddle}: ``\textit{Two people are born at the same moment, but they don't have the same birthdays. How could this be?}'' {Choice: ``They are born in different hospitals.''(\xmark); ``They are born in different time zones.''(\cmark); ``They are twins.''(\xmark); ...}. 
	
% \end{itemize}


%There are two major types of riddles: 
%\begin{itemize}
%	\item \textbf{concept-based riddles} usually have a single concept as answers, often ending with ``What am I?'' or ``What is it?''
%	\item \textbf{explanation-based riddles} are about an 
%\end{itemize}
 


 

\section{Proposed Method: SyMFM6D}

We propose a deep multi-directional fusion approach called SyMFM6D that estimates the 6D object poses of all objects in a cluttered scene based on multiple RGB-D images while considering object symmetries. 
In this section, we define the task of multi-view 6D object pose estimation and present our multi-view deep fusion architecture.

\begin{figure*}[tbh]
  \vspace{2mm}
  \centering
  \includegraphics[page=1, trim = 5mm 40mm 5mm 42mm, clip,  width=1.0\linewidth]{figures/SyMFM6D_architecture4_2.pdf}
   \caption{Network architecture of SyMFM6D which fuses $N$ RGB-D input images. Our method converts the $N$ depth images to a single point cloud which is processed by an encoder-decoder point cloud network. The $N$ RGB images are processed by an encoder-decoder CNN. Every hierarchy contains a point-to-pixel fusion module and a pixel-to-point fusion module for deep multi-directional multi-view fusion. We utilize three MLPs with four layers each to regress 3D keypoint offsets, center point offsets, and semantic labels based on the final features. The 6D object poses are computed as in \cite{pvn3d} based on mean shift clustering and least-squares fitting. We train our network by minimizing our proposed symmetry-aware multi-task loss function using precomputed object symmetries. $N_p$ is the number of points in the point cloud. $H$ and $W$ are height and width of the RGB images.}
   \label{fig_architecture}
   \vspace{-2mm}
\end{figure*}


6D object pose estimation describes the task of predicting a rigid transformation $\boldsymbol p = [\boldsymbol R |  \boldsymbol t] \in SE(3)$ which transforms the coordinates of an observed object from the object coordinate system into the camera coordinate system. This transformation is called 6D object pose because it is composed of a 3D rotation $\boldsymbol R \in SO(3)$ and a 3D translation $\boldsymbol t \in \mathbb{R}^3$. 
The designated aim of our approach is to jointly estimate the 6D poses of all objects in a given cluttered scene using multiple RGB-D images which depict the scene from multiple perspectives. We assume the 3D models of the objects and the camera poses to be known as proposed by \cite{mv6d}.



\subsection{Network Overview}

Our symmetry-aware multi-view network consists of three stages which are visualized in \cref{fig_architecture}. 
The first stage receives one or multiple RGB-D images and extracts visual features as well as geometric features which are fused to a joint representation of the scene. 
The second stage performs a detection of predefined 3D keypoints and an instance semantic segmentation.
Based on the keypoints and the information to which object the keypoints belong, we compute the 6D object poses with a least-squares fitting algorithm \cite{leastSquares} in the third stage.



\subsection{Multi-View Feature Extraction}

To efficiently predict keypoints and semantic labels, the first stage of our approach learns a compact representation of the given scene by extracting and merging features from all available RGB-D images in a deep multi-directional fusion manner. For that, we first separate the set of RGB images $\text{RGB}_1, ..., \text{RGB}_N$ from their corresponding depth images $\text{Dpt}_1$, ..., $\text{Dpt}_N$. The $N$ depth images are converted into point clouds, transformed into the coordinate system of the first camera, and merged to a single point cloud using the known camera poses as in \cite{mv6d}. 
Unlike \cite{mv6d}, we employ a point cloud network based on RandLA-Net \cite{hu2020randla} with an encoder-decoder architecture using skip connections.
The point cloud network learns geometric features from the fused point cloud and considers visual features from the multi-directional point-to-pixel fusion modules as described in \cref{sec_multi_view_fusion}.

The $N$ RGB images are independently processed by a CNN with encoder-decoder architecture using the same weights for all $N$ views. The CNN learns visual features while considering geometric features from the multi-directional pixel-to-point fusion modules. We followed \cite{ffb6d} and build the encoder upon a ResNet-34 \cite{resnet} pretrained on ImageNet~\cite{imagenet} and the decoder upon a PSPNet \cite{pspnet}. 

After the encoding and decoding procedures including several multi-view feature fusions, we collect the visual features from each view corresponding to the final geometric feature map and concatenate them. The output is a compact feature tensor containing the relevant information about the entire scene which is used for keypoint detection and instance semantic segmentation as described in \cref{sec_keypoint_detection_and_segmentation}.


\begin{figure*}[tbh]
  \vspace{2mm} 
  \centering  
\begin{subfigure}[b]{0.48\textwidth}
  \includegraphics[page=1, trim = 1mm 6mm 6mm 6mm, clip,  width=1.0\linewidth]{figures/p2r_8.pdf}
   \caption{Point-to-pixel fusion module.~~~~}
   \label{fig_pt2px_fusion}
\end{subfigure}
\begin{subfigure}[b]{0.48\textwidth}
  \centering  
  \includegraphics[page=1, trim = 1mm 6mm 6mm 6mm, clip,  width=1.0\linewidth]{figures/r2p_8.pdf}
   \caption{Pixel-to-point fusion module.~~~~~}
   \label{fig_px2pt_fusion}
   \end{subfigure}
      \caption{Overview of our proposed multi-directional multi-view fusion modules. They combine pixel-wise visual features and point-wise geometric features by exploiting the correspondence between pixels and points using the nearest neighbor algorithm. We compute the resulting features using multiple shared MLPs with a single layer and max-pooling.
      For simplification, we depict an example with $N=2$ views and $K_\text{i}=K_\text{p}=3$ nearest neighbors. The points of ellipsis (...) illustrate the generalization for an arbitrary number of views $N$. Please refer to \cite{ffb6d} for better understanding the basic operations.
      }
   \label{fig_fusion_modules}
   \vspace{-1mm}
\end{figure*}



\subsection{Multi-View Feature Fusion}
\label{sec_multi_view_fusion}
In order to efficiently fuse the visual and geometric features from multiple views, we extend the fusion modules of FFB6D~\cite{ffb6d} from bi-directional fusion to \emph{multi-directional fusion}. We present two types of multi-directional fusion modules which are illustrated in \cref{fig_fusion_modules}.
Both types of fusion modules take the pixel-wise visual feature maps and the point-wise geometric feature maps from each view, combine them, and compute a new feature map.
This process requires a correspondence between pixel-wise and point-wise features which we obtain by computing an XYZ map for each RGB feature map based on the depth data of each pixel using the camera intrinsic matrix as in \cite{ffb6d}. To deal with the changing dimensions at different layers, we use the centers of the convolutional kernels as new coordinates of the feature maps and resize the XYZ map to the same size using nearest interpolation as proposed in \cite{ffb6d}.

The \emph{point-to-pixel} fusion module in \cref{fig_pt2px_fusion} computes a 
fused feature map $\bb F_\text{f}$ based on the image features $\bb F_{\text{i}}(v)$ of all views $v \in \{1, \ldots, N\}$.
We collect the $K_\text{p}$ nearest point features $\bb F_{\text{p}_k}(v)$ with $k \in \{1, \ldots, K_\text{p}\}$ from the point cloud for each pixel-wise feature and each view independently by computing the nearest neighbors according to the Euclidean distance in the XYZ map. Subsequently, we process them by a shared MLP before aggregating them by max-pooling, i.e.,
\begin{align} 
    \widetilde{\bb F}_{\text{p}}(v) = \max_{k \in \{1, \ldots, K_\text{p}\}} 
    \Big( \text{MLP}_\text{p}(\bb F_{\text{p}_k}(v)) \Big).
    \label{eq_p2r}
\end{align}
Finally, we apply a second shared MLP to fuse all features $\bb F_\text{i}$ and 
$\widetilde{\bb F}_{\text{p}}$ as 
$\bb F_{\text{f}} = \text{MLP}_\text{fp}(\widetilde{\bb F}_{\text{p}} \oplus \bb F_\text{i})$ where $\oplus$ denotes the concatenate operation.


The \emph{pixel-to-point} fusion module in \cref{fig_px2pt_fusion} collects the $K_\text{i}$ nearest image features $\bb F_{\text{i}_k}(\textrm{i2v}(i_k))$ with $k\in\{1, ..., K_\text{i}\}$. $\textrm{i2v}(i_k)$ is a mapping that maps the index of an image feature to its corresponding view. This procedure is performed for each point feature vector $\bb F_\text{p}(n)$.
We aggregate the collected image features by max-pooling and apply a shared MLP, i.e.,
\begin{align}
    \widetilde{\bb F}_{\text{i}} = \text{MLP}_\text{i} 
    \left( \max_{k \in \{1, \ldots, K_\text{i}\}} 
    \Big( \bb F_{\text{i}_k}(\textrm{i2v}(i_k)) \Big)  
    \right).
    \label{eq_r2p}
\end{align}
One more shared MLP fuses the resulting image features $\widetilde{\bb F}_{\text{i}}$ with the point features $\bb F_\text{p}$ as 
$\bb F_{\text{f}} = \text{MLP}_\text{fi}(\widetilde{\bb F}_{\text{i}} \oplus \bb F_\text{p})$.




\subsection{Keypoint Detection and Segmentation}
\label{sec_keypoint_detection_and_segmentation}
The second stage of our SyMFM6D network contains modules for 3D keypoint detection and instance semantic segmentation following \cite{mv6d}. However, unlike \cite{mv6d}, we use the SIFT-FPS algorithm \cite{lowe1999sift} as proposed by FFB6D \cite{ffb6d} to define eight target keypoints for each object class. SIFT-FPS yields keypoints with salient features which are easier to detect.
Based on the extracted features, we apply two shared MLPs to estimate the translation offsets from each point of the fused point cloud to each target keypoint and to each object center.
We obtain the actual point proposals by adding the translation offsets to the respective points of the fused point cloud. 
Applying the mean shift clustering algorithm \cite{cheng1995meanshift} results in predictions for the keypoints and the object centers.
We employ one more shared MLP 
for estimating the object class of each point in the fused point cloud as in \cite{pvn3d}.



\subsection{6D Pose Computation via Least-Squares Fitting}

Following \cite{pvn3d}, we use the least-squares fitting algorithm \cite{leastSquares} to compute the 6D poses of all objects based on the estimated keypoints. As the $M$ estimated keypoints $\boldsymbol{\widehat{k}}_1, ..., \boldsymbol{\widehat{k}}_M$ are in the coordinate system of the first camera and the target keypoints $\boldsymbol k_1, ..., \boldsymbol k_M$ are in the object coordinate system, least-squares fitting calculates the rotation matrix $\boldsymbol R$ and the translation vector $\boldsymbol t$ of the 6D pose by minimizing the squared loss
\begin{equation}
    L_\text{Least-squares} = \sum_{i=1}^M \norm{\boldsymbol{\widehat{k}_i} - (\boldsymbol R \boldsymbol k_i + \boldsymbol t)}_2^2.
\end{equation}



\subsection{Symmetry-aware Keypoint Detection}

Most related work, including \cite{pvn3d, ffb6d}, and \cite{mv6d} does not specifically consider object symmetries. 
However, symmetries lead to ambiguities in the predicted keypoints as multiple 6D poses can have the same visual and geometric appearance. 
Therefore, we introduce a novel symmetry-aware training procedure for the 3D keypoint detection including a novel symmetry-aware objective function to make the network predicting either the original set of target keypoints for an object or a rotated version of the set corresponding to one object symmetry. Either way, we can still apply the least-squares fitting which efficiently computes an estimate of the target 6D pose or a rotated version corresponding to an object symmetry. To do so, we precompute the set $\boldsymbol{S}_I$ of all rotational symmetric transformations for the given object instance $I$ with a stochastic gradient
descent algorithm \cite{sgdr}.
Given the known mesh of an object and an initial estimate for the symmetry axis, we transform the object mesh along the symmetry axis estimate and optimize the symmetry axis iteratively by minimizing the ADD-S metric \cite{hinterstoisser2012model}.
Reflectional symmetries which can be represented as rotational symmetries are handled as rotational symmetries. 
Other reflectional symmetries are ignored, since the reflection cannot be expressed as an Euclidean transformation.
To consider continuous rotational symmetries, we discretize them into 16 discrete rotational symmetry transformations.

We extend the keypoints loss function of \cite{pvn3d} to become symmetry-aware such that it predicts the keypoints of the closest symmetric transformation, i.e. 
\begin{equation}
    L_\text{kp}(\mathcal{I}) = \frac{1}{N_I} 
    \min_{\boldsymbol{S} \in \boldsymbol{S}_I} 
    \sum_{i \in \mathcal{I}} \sum_{j=1}^M 
    \norm{\boldsymbol{x}_{ij} - \boldsymbol{S}\boldsymbol{\widehat{x}}_{ij}}_2, 
\label{eq_keypoint_loss}
\end{equation}
where $N_I$ is the number of points in the point cloud for object instance $I$, $M$ is the number of target keypoints per object, and $\mathcal{I}$ is the set of all point indices that belong to object instance $I$.  
The vector $\boldsymbol{\widehat{x}}_{ij}$ is the predicted keypoint offset for the $i$-th point and the $j$-th keypoint while $\boldsymbol{x}_{ij}$ is the corresponding ground truth. 



\subsection{Objective Function}

We train our network by minimizing the multi-task loss function
\begin{equation}
 \label{eq_total_loss}
    L_\text{multi-task} = \lambda_1 L_\text{kp} 
    + \lambda_2 L_\text{semantic}  
    +  \lambda_3 L_\text{cp},
\end{equation}
where $L_\text{kp}$ is our symmetry-aware keypoint loss from \cref{eq_keypoint_loss}.
$L_\text{cp}$ is an L1 loss for the center point prediction, $L_\text{semantic}$ is a Focal loss \cite{focalLoss} for the instance semantic segmentation, and $\lambda_1=2$, $\lambda_2=1$, and $\lambda_3=1$ are the weights for the individual loss functions as in \cite{ffb6d}.
 



\begin{figure}[th!]
	\centering
% 	\vspace{-2em}
	\includegraphics[width=1\linewidth]{methods.pdf}
	\caption{Three types of baseline methods: 1) fine-tuning pre-trained LMs, 2) incorporating graph-based reasoner, 3) fine-tuning a unified text-to-text LM. }
	\label{fig:models} 
\end{figure}

\section{Experiments}\label{sec:exp}
\section{EXPERIMENTS} \label{Sec:exp}
\subsection{Dataset}

Recently, the task of describing video using natural language has gradually received more interest in the computer vision community. Eventually, many video description datasets have been released~\cite{Xu16_MSR_Dataset}. However, these datasets only provide general descriptions of the video and there is no detailed understanding of the action. The captions are also written using natural language sentences which can not be used directly in robotic applications. Motivated by these limitations, we introduce a new \textit{video to command} (IIT-V2C) dataset which focuses on \textit{fine-grained} action understanding~\cite{lea2016learning}. Our goal is to create a new large scale dataset that provides fine-grained understanding of human actions in a grammar-free format. This is more suitable for robotic applications and can be used with deep learning methods.

\textbf{Video annotation} 
Since our main purpose is to develop a framework that can be used by real robots for manipulation tasks, we use only videos that contain human actions. To this end, the raw videos in the Breakfast dataset~\cite{Kuehne14_BF_Dataset} are best suited to our purpose since they were originally designed for activity recognition. We only reuse the raw videos from this dataset and manually segment each video into short clips in a fine granularity level. Each short clip is then annotated with a \textit{command sentence} that describes the current human action.

\textbf{Dataset statistics} 
In particular, we reuse $419$ videos from the Breakfast dataset. The dataset contains $52$ unique participants performing cooking tasks in different kitchens. We segment each video (approximately $2-3$ minutes long) into around $10-50$ short clips (approximately $1-15$ seconds long), resulting in $11,000$ unique short videos. Each short video has a single command sentence that describes human actions. We use $70\%$ of the dataset for training and the remaining $30\%$ for testing. Although our new-form dataset is characterized by its grammar-free property for the convenience in robotic applications, it can easily be adapted to classical video captioning task by adding the full natural sentences as the new groundtruth for each video.


\subsection{Evaluation Metric, Baseline, and Implementation}
\textbf{Evaluation Metric} We report the experimental results using the standard metrics in the captioning task~\cite{Xu16_MSR_Dataset}: BLEU, METEOR, ROUGE-L, and CIDEr. This makes our results directly comparable with the recent state-of-the-art methods in the video captioning field.

\textbf{Baseline} We compare our results with two recent methods in the video captioning field: S2VT~\cite{Venugopalan2016} and SGC~\cite{Ramanishka2017cvpr}. The authors of S2VT used LSTM in the encoder-decoder architecture, while the inputs are from the features of RGB images (extracted by VGG16) and optical flow images (extracted by AlexNet). SGC also used LSTM with encoder-decoder architecture, however, this work integrated a saliency guided method as the attention mechanism, while the features are from Inception\_v3. We use the code provided by the authors of the associated papers for the fair comparison.


\begin{figure*}
\centering
\footnotesize
 %\stackunder[5pt]{\includegraphics[width=0.49\linewidth, height=0.09\linewidth]{figures/4_exp/result.pdf}}
  \stackunder[2pt]{\includegraphics[width=0.49\linewidth, height=0.09\linewidth]{figures/4_exp/P13_pancake_82.jpg}}  				
  				  {\tableCaption {righthand carry spatula} {righthand carry spatula} {lefthand reach stove} {lefthand reach pan}}
  				  \vspace{2ex} 
  \hspace{0.25cm}%
  \stackunder[2pt]{\includegraphics[width=0.49\linewidth, height=0.09\linewidth]{figures/4_exp/P38_salad_2.jpg}}  
  				  {\tableCaption {righthand cut fruit} {righthand cut fruit} {righthand cut fruit} {righthand cut fruit}}
  				  \vspace{2ex}
 %\stackunder[5pt]{\includegraphics[width=0.49\linewidth, height=0.09\linewidth]{figures/4_exp/result.pdf}} 
  \stackunder[2pt]{\includegraphics[width=0.49\linewidth, height=0.09\linewidth]{figures/4_exp/p07_pancake_6.jpg}}  
    			  {\tableCaption {righthand crack egg} {righthand carry egg} {lefthand reach spatula} {righthand carry egg}}
  				  \vspace{0ex}
  \hspace{0.25cm}%
 %\stackunder[5pt]{\includegraphics[width=0.49\linewidth, height=0.09\linewidth]{figures/4_exp/result.pdf}}  
  \stackunder[2pt]{\includegraphics[width=0.49\linewidth, height=0.09\linewidth]{figures/4_exp/P43_milk_15.jpg}}    
				  {\tableCaption {righthand stir milk} {righthand hold teabag} {righthand place kettle} {righthand take cacao}}
  				  \vspace{0ex}
  
  %\hspace{-0.25cm}%
 %\hspace{-0.25cm}%
\vspace{1ex}
\caption{Example of translation results of the S2VT, SGC and our LSTM\_Inception\_v3 network on the IIT-V2C dataset.}

\label{Fig:main_result} 
\end{figure*}


\textbf{Implementation} We use $512$ hidden units in both LSTM and GRU in our implementation. The first hidden state of LSTM/GRU is initialized uniformly in $[-0.1, 0.1]$. We set the number of frames for each input video at $30$. Sequentially, we consider each command has maximum $30$ words. If there are not enough $30$ frames/words in the input video/command, we pad the mean frame (from ImageNet dataset)/empty word at the end of the list until it reaches $30$.  During training, we only accumulate the softmax losses of the real words to the total loss, while the losses from the empty words are ignored. We train all the networks for $150$ epochs using Adam optimizer with a learning rate of $0.0001$. The batch size is empirically set to $16$. The training time for each network is around $3$ hours on a NVIDA Titan X GPU. 


\subsection{Results}

%//////////////////////////////////////////////
\begin{table}[!ht]
\centering\ra{1.4}
\caption{Performance on IIT-V2C Dataset}
\renewcommand\tabcolsep{2.5pt}
\label{tb_result_v2c}
\hspace{2ex}

\begin{tabular}{@{}rcccccccc@{}}
\toprule 					 &
\ssmall Bleu\_1  & 
\ssmall Bleu\_2  & 
%\multirow{1}{*}[2.5pt]{\scriptsize DeepLab~\cite{Chen2016_deeplab}} & 
\ssmall Bleu\_3  &
\ssmall Bleu\_4  & 
%\mu\ltirow{1}{*}[2.5pt]{\scriptsize BB-CNN} & 
\ssmall METEOR  &
\ssmall ROUGE\_L &
\ssmall CIDEr \\


\midrule
S2VT~\cite{Venugopalan2016} 				& 0.383   & 0.265   & 0.201	& 0.159	& 0.183    & 0.382   & 1.431     \\
SGC~\cite{Ramanishka2017cvpr}			& 0.370   & 0.256   & 0.198	& 0.161	& 0.179    & 0.371   & 1.422     \\
\cline{1-8}
LSTM\_VGG16					& 0.372   & 0.255  			 & 0.193	& 0.159	& 0.180    & 0.375   & 1.395     \\
GRU\_VGG16 					& 0.350   & 0.233 			 & 0.173	& 0.137	& 0.168    & 0.351   & 1.255     \\
LSTM\_Inception\_v3				& \textbf{0.400}  			 & \textbf{0.286}   & 0.221	& 0.178	& \textbf{0.194}    & \textbf{0.402}   & \textbf{1.594}     \\
GRU\_Inception\_v3 				& 0.391   & 0.281  			 & \textbf{0.222}	& \textbf{0.188}	& 0.190    & 0.398   & 1.588     \\
LSTM\_ResNet50 				& 0.398   & 0.279            & 0.215	& 0.174	& 0.193    & 0.398   & 1.550     \\
GRU\_ResNet50 				& 0.398   & 0.284   & 0.220	& 0.183	& 0.193    & 0.399   & 1.567     \\
\bottomrule
\end{tabular}
\end{table}

Table~\ref{tb_result_v2c} summarizes the captioning results on the IIT-V2C dataset. Overall, the LSTM network that uses visual features from Inception\_v3 (LSTM\_Inception\_v3) achieves the highest performance, winning on the Blue\_1, Blue\_2, METEOR, ROUGE\_L, and CIDEr metrics. Our LSTM\_Inception\_v3 also outperforms S2VT and SGC in all metrics by a fair margin. We also notice that both the LSTM\_ResNet50 and GRU\_ResNet50 networks give competitive results in comparison with the LSTM\_Inception\_v3 network. Overall, we observe that the architectures that use LSTM give slightly better results than those using GRU. However, this difference is not significant when the ResNet50 features are used to train the models (LSTM\_ResNet50 and GRU\_ResNet50 results are a tie).


From the experiments, we notice that there are two main factors that affect the results of this problem: the network architecture and the input visual features. Since the IIT-V2C dataset contains mainly the fine-grained human actions in a limited environment (i.e., the kitchen), the SGC architecture that used saliency guide as the attention mechanism does not perform well as in the normal video captioning task. On the other hand, the visual features strongly affect the final results. Our experiments show that the ResNet50 and Inception\_v3 features significantly outperform the VGG16 features in both LSTM and GRU networks. Since the visual features are not re-trained in the sequence to sequence model, in practice it is crucial to choose the state-of-the-art CNN as the feature extractor for the best performance.


Fig.~\ref{Fig:main_result} shows some examples of the generated commands by our LSTM\_Inception\_v3, S2VT, and SGC models on the test videos of the IIT-V2C dataset. These qualitative results show that our LSTM\_Inception\_v3 gives good predictions in many cases, while S2VT and SGC results are more variable. In addition to the good predictions that are identical with the groundtruth, we note that many other generated commands are relevant. Due to the nature of the IIT-V2C dataset, most of the videos are short and contain fine-grained human manipulation actions, while the groundtruth commands are also very short. This makes the problem of translating videos to commands is more challenging than the normal video captioning task since the network has to rely on the minimal information to predict the output command.  


\subsection{Robotic Applications}

%
%\begin{figure*}
%\centering
%\footnotesize
% %\stackunder[5pt]{\includegraphics[width=0.49\linewidth, height=0.09\linewidth]{figures/4_exp/result.pdf}}
%  \stackunder[2pt]{\includegraphics[width=0.49\linewidth, height=0.14\linewidth]{figures/5_robot/human_pick_hammer_cut/human_pick_all.jpg}}  				
%  				  {Human instruction}
%  				  \vspace{2ex} 
%  \hspace{0.25cm}%
%  \stackunder[2pt]{\includegraphics[width=0.49\linewidth, height=0.14\linewidth]{figures/5_robot/pick_place_hammer_cut/pick_all.jpg}}  
%  				  {Robot execution}
%  				  \vspace{2ex}
% %\stackunder[5pt]{\includegraphics[width=0.49\linewidth, height=0.09\linewidth]{figures/4_exp/result.pdf}} 
%  \stackunder[2pt]{\includegraphics[width=0.49\linewidth, height=0.14\linewidth]{figures/5_robot/human_pour_bottle_cut/human_pour_all.jpg}}  
%    			  {Human instruction}
%  				  \vspace{0ex}
%  \hspace{0.25cm}%
% %\stackunder[5pt]{\includegraphics[width=0.49\linewidth, height=0.09\linewidth]{figures/4_exp/result.pdf}}  
%  \stackunder[2pt]{\includegraphics[width=0.49\linewidth, height=0.14\linewidth]{figures/5_robot/pour_bottle_cut/pour_all.jpg}}    
%				  {Robot execution}
%  				  \vspace{0ex}
%%  \stackunder[2pt]{\includegraphics[width=0.49\linewidth, height=0.13\linewidth]{figures/5_robot/pick_place_hammer_cut/pick_all.jpg}}    
%%				  {\tableCaption {righthand stir milk} {righthand hold teabag} {righthand place kettle} {righthand take cacao}}
%%  				  \vspace{0ex}
%%  \stackunder[2pt]{\includegraphics[width=0.49\linewidth, height=0.13\linewidth]{figures/5_robot/pick_place_hammer_cut/pick_all.jpg}}    
%%				  {\tableCaption {righthand stir milk} {righthand hold teabag} {righthand place kettle} {righthand take cacao}}
%%  				  \vspace{0ex}
%    
%  %\hspace{0.35cm}%
% %\hspace{-0.25cm}%
%\vspace{3ex}
%\caption{ROBOT Example Results. TBD.}
%\label{Fig:robot_imitation} 
%\end{figure*}


\begin{figure*}[ht]
  \centering
 \subfigure[Pick and place task]{\label{fig_resize_map_a}\includegraphics[width=0.99\linewidth, height=0.16\linewidth]{figures/5_robot/pick_all_in_one.pdf}}
 \subfigure[Pouring task]{\label{fig_resize_map_b}\includegraphics[width=0.99\linewidth, height=0.16\linewidth]{figures/5_robot/pour_all_in_one.pdf}}
    
     
 \vspace{2.0ex}
 \caption{Example of manipulation tasks performed by WALK-MAN using our proposed framework. \textbf{(a)} Pick and place task. \textbf{(b)} Pouring task. The frames from human instruction videos are on the left side, while the robot performs actions on the right side. We notice that there are two sub-tasks (i.e., two commands) in these tasks: grasping the object and manipulating it. More illustrations can be found in the supplemental video.}
 \label{Fig:robot_imitation}
\end{figure*}



Given the proposed translation module, we build a robotic framework that allows the robot to perform various manipulation tasks by just ``\textit{watching}" the input video. Our goal in this work is similar to~\cite{Yang2015}, however, we propose to keep the video understanding separately from the vision system. In this way, the robot can learn to understand the task and execute it independently. This makes the proposed approach more practical since it does not require a dataset that has both the caption and the object (or grasping) location. It is also important to note that our goals differ from LfD since we only focus on finding a general way to let the robot execute different manipulation actions, while the trajectory in each action is assumed to be known.

In particular, for each task presented by a video, the translation module will generate an output command sentence. Based on this command, the robot uses its vision system to find relevant objects and plan the actions. Experiments are conducted using the humanoid WALK-MAN~\cite{Niko2016_full}. The robot is controlled using the XBotCore software architecture~\cite{muratore2017xbotcore}, while the OpenSoT library~\cite{Rocchi15} is used to plan full-body motion. The relevant objects and their affordances are detected using AffordanceNet framework~\cite{AffordanceNet17}. For simplicity, we only use objects in the IIT-Aff dataset~\cite{Nguyen2017_Aff} in the demonstration videos so the robot can recognize them. Using this setup, the robot can successfully perform various manipulation tasks by closing the loop: understanding the human demonstration from the video using the proposed method, finding the relevant objects and grasping poses~\cite{Nguyen2017_Aff}, and planning for each action~\cite{Rocchi15}.


Fig.~\ref{Fig:robot_imitation} shows some manipulation tasks performed by WALK-MAN using our proposed framework. For a simple task such as ``righthand grasp bottle", the robot can effectively repeat the human action through the command. Since the output of our translation module is in grammar-free format, we can directly map each word in the command sentence to the real robot command. In this way, we avoid using other modules as in~\cite{Tellex2011} to parse the natural command into the one that uses in the real robot. The visual system also plays an important role in our framework since it provides the object location and the target frames (e.g., grasping frame, ending frame) for the robot to plan the actions. Using our approach, the robot can also complete long manipulation tasks by stacking a list of demonstration videos in order for the translation module. Note that, for the long manipulation tasks, we assume that the ending state of one task will be the starting state of the next task. Overall, WALK-MAN successfully performs various manipulation tasks such as grasping, pick and place, or pouring. The experimental video and our IIT-V2C dataset can be found at the following link:
\vspace{1ex}
\centerline{\url{https://sites.google.com/site/video2command/}}






\junk{
 Furthermore, we can also  We can also stack several videos to create a chain of commands. 
 
Using our approach, the robot can perform different tasks based on the instructions from human demonstration videos.

  can understand the human instructions from the video via the proposed translation module, while the visual information can be solved effectively with the recent advances in deep learning~\cite{Nguyen2017_Aff}.
  
  
, and we already have many separated datasets for video captioning~\cite{Xu16_MSR_Dataset} and affordance detection~\cite{Nguyen2017_Aff}


 using all standard metrics
 
 We train each LSTM/GRU network using the features from VGG16, Inception\_v3 and ResNet50, respectively.
 
 method for video captioning and the recent method that used encoder-decoder scheme with saliency guided~\cite{Ramanishka2017cvpr} (denoted as SGC) as the attention mechanism.


   \textbf{Grasping}, \textbf{Pick and Place} In this experiment, we address how the robot executes a new scenario, i.e. ``pick up an object and place it into another object". Our experiment involves two affordances \texttt{grasp} and \texttt{contain}. Using our approach, the robot can grasp an object from its \texttt{grasp} affordance and bring it to a new location that belongs to the \texttt{contain} affordance of another object. Furthermore, we note that with the aid of our semantic understanding framework, the robot can recognize the target objects while ignoring the irrelevant ones. \textbf{Pouring} Similar to the pick and place experiment, we use two affordances \texttt{grasp} and \texttt{contain} and the associated objects in our dataset. The goal is to pour the liquid from an object to the \texttt{contain} region of another object. Our semantic perception framework gives the robot detail understanding about the target objects, their affordances as well as the relative coordinates for the movement. We notice that besides the basic actions such as grasp, raise, etc., we predefined the pouring action to help the robot complete the task. 
   
 we use the vision system from~\cite{Nguyen2017_Aff} to provide object location, its affordances and grasping frame for the robot, while trajectory is generated by the OpenSoT library~\cite{Rocchi15}. For simplicity, we only use the objects that the IIT-Aff~\cite{Nguyen2017_Aff} dataset so the robot can recognize them. Since there are a huge gap between the number of words in the IIT-V2C dataset  and the the number of objects that the robot can recoginze in the  in the  . videoFor each To create the command for  n our work, instead of using full natural command sentences, we propose to use human demonstrations from videos as the input. The translation module is then used to interpret the video to commands in grammar-free format that the robot can follow. Combined with the vision and planning system, our approach allows the robot to perform manipulation tasks by just ``watching" the input video. Fig.~\ref{Fig:robot_apps} shows a full description of framework in our robotic applications. 


We validate our framework using the WALK-MAN full-size humanoid~\cite{Niko2016_full}. The robot is controlled in real-time using the XBotCore software architecture~\cite{muratore2017xbotcore}, while the OpenSoT library~\cite{Rocchi15} is used to plan full body motion. The relevant objects and their affordances are detected using the framework in~\cite{Nguyen2017_Aff}. To allow the real-time performance, the control and planning system run a control pc, while the vision system runs on a vision pc with a NVIDIA Titan X GPU.


For the safety of the robot, we define some key action such as "reach", "grasp". Each word in the output command will be compared with these basic action using the similarity in word2vector. 

Fig.~\ref{Fig:robot_apps} shows an overview of our robotic application.


To compare the origi- nality in generation, we compute the Levenshtein distance of the predicted sentences with those in the training set. From Table 3, for the MSVD corpus, 42.9 of the predic- tions are identical to some training sentence, and another 38.3 can be obtained by inserting, deleting or substituting one word from some sentence in the training corpus. We note that many of the descriptions generated are relevant. TODO: challenging, failures case.


 We follow the standard procedure in the captioning tasks to evaluate our results.
 
 
 In particular, we use the code from COCO evaluation server~\cite{Chen_COCO_Evaluation} that implements several metrics:

 the baseline network architecture remains challenging to modify, most of the recent work used different models such as attention mechanism~\cite{•}, saliency 



GoogLeNet [32, 12] to extract the frame-level features in our experiment. All the videos’ lengths are kept to 200 frames. For a video with more than 200 frames, we drop the extra frames. For a video with- out enough frames, we pad zero frames. These are com- mon approaches to ensure all the videos have the same length [38, 43]. Feature Extractor: CNN. We consider the input video as a sequence of frames and encode each frame using a CNN. This process extracts the meaningful features from the input images at every time step. These features are then fed to the LSTM network as inputs. In particular, we use three most popular CNNs: VGG16, GoogleNet, and ResNet-50 as our feature extractors. For the VGG16 and ResNet-50, we remove the last classification layer and TODO: describe net without the last layer.
\\




The MSR-VTT dataset is characterized by the unique properties including the large scale clip-sentence pairs, comprehensive video categories, diverse video content and descriptions, as well as multimodal audio and video stream- s. We



Current datasets for video to text mostly focus on specific fine-grained domains. For example, YouCook [5], TACoS [25, 28] and TACoS Multi-level [26] are mainly de- signed for cooking behavior. MSR-VTT focuses on general videos in our life, while MPII-MD [27] and M-VAD [32] on movie domain. Although MSVD [3] contains general web videos which may cover different categories, the very limited size (1,970) is far from representativeness. To col- lect representative videos, we obtain the top 257 represen- tative queries from a commercial video search engine, cor- responding to 20 categories


Since our goal is to collect short video clips that each can be described with one single sentence in our current version


. TOFIX: Similar to the treatment of frame features, we embed words to a lower 500 dimensional space by applying a linear transformation to the input data and learning its parameters via back propa- gation. The embedded word vector concatenated with the output (ht) of the first LSTM layer forms the input to the second LSTM layer (marked green in Figure 2). When considering the output of the LSTM we apply a softmax over the complete vocabulary as in Equation 5.




}  



\section{Related Work}\label{sec:rel_work}

\subsection*{Benchmarking Machine Common Sense}
 
The prior works on building commonsense reasoning benchmarks touch different aspects of commonsense reasoning:
SWAG~\cite{Zellers2018SWAGAL}, HellaSWAG~\cite{Zellers2019HellaSwagCA}, CODAH~\cite{Chen2019CODAHAA}, aNLI~\cite{bhagavatula2019abductive} for situation-based reasoning;
Physical IQA~\cite{bisk2020piqa} on physical knowledge;
Social IQA~\cite{sap-etal-2019-social} on social psychology knowledge;
LocatedNearRE~\cite{Xu2018AutomaticEO} on mining spatial commonsense knowledge;
DoQ~\cite{elazar2019large} and NumerSense~\cite{lin2020birds} on numerical common sense;
CommonGen~\cite{lin2019commongen} for generative commonsense reasoning, and many others;
OpenCSR~\cite{lin2021opencsr} and ProtoQA~\cite{Boratko2020ProtoQAAQ} aim to test commonsense reasoning ability in an open-ended setting.
% However, none of them requires higher-order reasoning ability where understanding creativity use of language is a must.

CommonsenseQA~\cite{Talmor2018CommonsenseQAAQ} has the same format as our proposed \textsc{RiddleSense}, and both target general commonsense knowledge via multiple-choice question answering.
However, CSQA focuses more on straightforward questions where the description of the answer concept is easy to understand and retrieval over ConceptNet, while RS makes use of riddle questions to test higher-order commonsense reasoning ability.
More detailed comparisions between them are in Section~\ref{sec:dataana}, which shows that the unique challenges of the RiddleSense on multiple dimensions.

\subsection*{Commonsense Reasoning Methods}
Our experiments cover three major types of commonsense reasoning methods that are popular in many benchmarks: fine-tuning pretrained LMs~\cite{Devlin2019, Liu2019RoBERTaAR, Lan2020ALBERT}, graph-based reasoning with external KGs~\cite{kagnet-emnlp19, feng2020scalable}, and fine-tuning unified text-to-text QA models~\cite{khashabi2020unifiedqa}.
Apart from ConceptNet, 
There are also some methods~\cite{lv2019graph, xu2020fusing}
using additional knowledge resources such as Wikipedia and Wiktionary.
A few recent methods also aim to generate relevant triples via language generation models so that the context graph is more beneficial for reasoning~\cite{wang2020connecting, yan2020learning}.
Our experiments in this paper aim to compare the most typical and popular methods which have open-source implementations,
which we believe are beneficial for understanding the limitation of these methods in higher-order commonsense reasoning --- \textsc{RiddleSense}.

\subsection*{Computational Creativity and NLP}
Creativity has been seen as a central property of the human use of natural language~\cite{mcdonald1994creative}.
% As NLP seeks to process all kinds of human language, 
Text should not be always taken at face value, however, higher-order use of language and figurative devices such as metaphor can communicate richer meanings and needs deeper reading and more complicated reasoning skills~\cite{veale2011creative}.
Recent works on processing language with creative use focus on metaphor detection~\cite{gao2018neural}, pun generation~\cite{He2019PunGW, Luo2019PunGANGA}, creative story generation, and humor detection~\cite{Weller2019HumorDA, Weller2020TheRD}, sarcasm generation~\cite{chakrabarty2020r}, etc. 

Riddling, as a way to use creative descriptions to query a common concept, are relatively underexplored.
Previous works~\cite{tan2016solving, oliveira2018exploring} focus on the generation of riddles in specific languages and usually rely on language-specific features (e.g., decomposing a Chinese character into multiple smaller pieces).
There is few datasets or public resources for studying riddles as a reasoning task, to the best of our knowledge. 
The proposed \textsc{RiddleSense} is among the very first works connecting commonsense reasoning and computational creative, and provides a large dataset to train and evaluate models for answering riddle questions.


% \smallskip
% \noindent
% \textbf{CommonsenseQA vs} \textbf{RiddleSense}.
% The most similar work to ours is CommonsenseQA, which is also a multi-choice question answering task, and targets general commonsense knowledge like \textsc{RiddleSense} does.
% % They sample a pair of concepts ($c_q, c_a$) that are directly connected in ConceptNet, i.e., there exists such a triple ($c_q, r, c_a$). Then they ask human annotators to compose a question that mentions $c_q$, to which the answer is exactly $c_a$.
% %They control the difficulty of the task by restricting that 
% % Therefore, the composed questions in their datasets are mostly one-hop questions, and about 50\% of them are about the \texttt{AtLocation} relation.
% % The current state-of-the-art models already achieves very high performance, and the scope 
% %Also, many distractors in can also make sense. 
% In contrast,
% \textsc{RiddleSense} does not rely on the single-hop connections in ConceptNet but use natural existing riddles about everyday concepts and situations.
% The connections between question concepts and answer concepts are much more distant than CommonsenseQA, yielding a higher-order commonsense reasoning problem.
% Another unique feature of \textsc{RiddleSense} is the creativity in the riddle questions, where understanding metaphor and other rhetoric methods may be helpful.







\section{Conclusion}\label{sec:conclusion}
 We propose a novel commonsense reasoning challenge, \textsc{RiddleSense}, which requires complex commonsense skills for reasoning about creative and counterfactual questions, coming with a large multiple-choice QA dataset.  
 We systematically evaluate recent commonsense reasoning methods over the proposed \textsc{RiddleSense} dataset, and find that the best model is still far behind human performance, suggesting that there is still much space for commonsense reasoning methods to improve.
 We hope \textsc{RiddleSense} can serve as a benchmark dataset for future research targeting complex commonsense reasoning and computational creativity.


\section*{Acknowledgements}
This research is supported in part by the Office of the Director of National Intelligence (ODNI), Intelligence Advanced Research Projects Activity (IARPA), via Contract No. 2019-19051600007, the DARPA MCS program under Contract No. N660011924033 with the United States Office Of Naval Research, the Defense Advanced Research Projects Agency with award W911NF-19-20271, and NSF SMA 18-29268. The views and conclusions contained herein are those of the authors and should not be interpreted as necessarily representing the official policies, either expressed or implied, of ODNI, IARPA, or the U.S. Government. We would like to thank all the collaborators in USC INK research lab and the reviewers for their constructive feedback on the work.

\section*{{Ethical Considerations}}

\paragraph{Copyright of Riddles.} 
The RiddleSense dataset is consistent with the terms of use of the fan websites and the intellectual property and privacy rights of the original sources.
All of our riddles and answers are from fan websites that can be accessed freely.
The website owners state that we may print and download material from the sites solely for \textit{non-commercial use} provided that we agree not to change or delete any copyright or proprietary notices from the materials.
Therefore, 
in addition to the dataset itself, we also provide the according copyright statements of every website and an URL link to the original page for each riddle. 
The dataset users must sign an informed consent form that they will only use our dataset for \textit{research purposes} before they can access the both the riddles and our annotations.


% \paragraph{Crowd-workers.} This work presents a new dataset for addressing a new problem, riddle-style common-sense reasoning.  
% The wrong choices within the dataset were produced by filtering questions and using crowd-workers to annotate riddle-style common-sense questions by suggesting additional distractors. 
% Most of the questions are about common knowledge about our physical world.
% \textit{None of the questions involve sensitive personal opinions or involve personally identifiable information. }
% We study posted tasks to be completed by crowd-workers instead of crowd workers themselves, and we do not retrieve any identifiable private information about a human subject.
% All annotators were fairly compensated by the Amazon Mechanical Turk platform solely based on the quantity and quality of their submissions.


% We used 
% \yl{yuchen can you add compensation and IRB details}


% \smallskip
% \noindent
% \textbf{Data bias.} Like most crowdsourced data, and in particular most common-sense data, these crowdsourced answers are inherently subject to bias: for example, a question like ``'' might be answered very differently by people from different backgrounds and cultures.  
% % The prior multiple-choice CSR datasets which our datasets are built on are strongly biased culturally, as they include a single correct answer and a small number of distractor answers, while our new datasets include many answers considered correct by several annotators.  
% However, this potential bias (or reduction in bias) has not been systematically measured in this work.
 


% \smallskip
% \noindent
% \textbf{Sustainability.} 
% For most of the experiments,
% we use the virtual compute engines on Google Cloud Platform, which ``is committed to purchasing enough renewable energy to match consumption for all of their operations globally.''\footnote{\url{https://cloud.google.com/sustainability}}
% With such virtual machine instances, we are able to use the resources only when we have jobs to run, thus avoiding unnecessary waste.






% \appendix
%
%\section{Resources}
%The dataset will be collected mainly from the web.
%We have already found some websites such as \texttt{riddles.com}, \texttt{riddles.tips}, \texttt{riddles.fyi}, \texttt{brainzilla.com}, and etc.
%They are all \textbf{free and publicly available. }
%
%\section{Risks}
%There are two major risks:
%\begin{itemize}
%	\item The web-crawled data is not enough. If the riddles are not enough, then we have to use them as the test data only, and argue that the training data will be the distant supervision from the Wiktionary glossary, or transferred data from other datasets. The key motivation of the paper won't change, though.
%	\item The generated distractors are too weak. We need to design better methods for generating distractors. The worst case is that we may need to add  human-annotated distractors (one for each test riddle) by ourselves to make sure the data is challenging enough. 
%\end{itemize}
%
%
%
%\section{Mid-term Checklist}
%\begin{enumerate}
%	\item An finalized version of the crawled, cleaned set of riddles. 
%	\item A first version of the distractors and the experimental pipeline for running the full evaluation.
%	\item The data analysis pipeline for reporting and visualizing the data (i.e., the connectivity between question concepts and answer concepts). 
%\end{enumerate}
%


\bibliography{riddleqa_rebiber} 
\bibliographystyle{acl_natbib}

\end{document}






