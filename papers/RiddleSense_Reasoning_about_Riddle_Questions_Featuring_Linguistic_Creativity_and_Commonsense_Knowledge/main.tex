\documentclass[11pt,a4paper]{article}
\usepackage{acl2020}
\usepackage{times}
\usepackage{graphicx}
\usepackage{wrapfig}
\usepackage{shortcuts}
% \usepackage{times}
\usepackage{dsfont}
\usepackage{latexsym}
\usepackage[linesnumbered,algoruled,boxed,noend]{algorithm2e}
\usepackage{amssymb}
\usepackage{amsmath}
\usepackage{url} 
\SetKwInOut{Parameter}{parameter}
\usepackage{enumitem}
\usepackage{mathtools}
\usepackage{cleveref}
\usepackage{multirow}
\usepackage{hhline}
\usepackage[export]{adjustbox}
\usepackage{booktabs,array}
% \usepackage[ruled,noend]{algorithm2e}
\usepackage{amsmath, bm}
\usepackage{color, colortbl}
\usepackage{subfig}
\SetKwInput{KwInput}{Input}
\SetKwInput{KwOutput}{Output}
\newcolumntype{P}[1]{>{\centering\arraybackslash}p{#1}}
\usepackage{booktabs,makecell,tabularx} 
\setitemize{noitemsep,topsep=0pt,parsep=0pt,partopsep=0pt}
\let\oldnl\nl% Store \nl in \oldnl
\definecolor{Gray}{gray}{0.85}
\definecolor{LightCyan}{rgb}{0.88,1,1}
\definecolor{FigCsqaOrange}{RGB}{236, 215, 192}
\definecolor{FigRsBlue}{RGB}{191, 207, 255}
\newcommand{\nonl}{\renewcommand{\nl}{\let\nl\oldnl}}
\newcommand{\comm}[1]{{\nonl{\scriptsize{{\color{gray}{/*~#1}~*/}}}}}
\usepackage{pifont}% http://ctan.org/pkg/pifont
\newcommand{\cmark}{\ding{51}}%
\newcommand{\xmark}{\ding{55}}%
\aclfinalcopy % Uncomment for camera-ready version, but NOT for submission.
\def\aclpaperid{1126}
\usepackage{epigraph}

% \epigraphsize{\small}% Default
\setlength\epigraphwidth{7cm}
\setlength\epigraphrule{0pt}

\usepackage{etoolbox}
\patchcmd{\epigraph}{\@epitext{#1}}{\itshape\@epitext{#1}}{}{}

\newlength{\Width}%
\newlength{\DepthReference}
\settodepth{\DepthReference}{g}
\newlength{\HeightReference}
\settoheight{\HeightReference}{T}
\newcommand{\MyColorBox}[2][red]%
{%
    \settowidth{\Width}{#2}%
    %\setlength{\fboxsep}{0pt}%
    \colorbox{#1}%
    {%      
        \raisebox{-\DepthReference}%
        {%
                \parbox[b][\HeightReference+\DepthReference][c]{\Width}{\centering#2}%
        }%
    }%
}
\setlength{\fboxsep}{1pt}

\renewcommand{\baselinestretch}{1} 
\newcommand{\yl}[1]{\textcolor{purple}{Yuchen: #1}} % William's comment
\begin{document}

\title{\vspace*{-0.5in}
{{\small \hfill ACL-IJCNLP 2021 Findings}\\
\vspace*{.25in}}
RiddleSense: Reasoning about Riddle Questions \\ Featuring Linguistic Creativity and Commonsense Knowledge}


\author{
Bill Yuchen Lin \quad Ziyi Wu\quad Yichi Yang\quad  Dong-Ho Lee\quad Xiang Ren\\
\texttt{\{yuchen.lin,ziyiwu,yichiyan,dongho.lee,xiangren\}@usc.edu}\\
Department of Computer Science and Information Sciences Institute,  \\ University of Southern California\\
}

% For research notes, remove the comment character in the line below.
% \researchnote

\maketitle



\begin{abstract}
\begin{abstract}
%\medskip
%\centering \textcolor{red}{Write the abstract last}
Silicon-compatible short- and mid-wave infrared emitters are highly sought-after for on-chip monolithic integration of electronic and photonic circuits to serve a myriad of applications in sensing and communication. To address this longstanding challenge, GeSn semiconductors have been proposed as versatile building blocks for silicon-integrated optoelectronic devices. In this regard, this work demonstrates light-emitting diodes (LEDs) consisting of a vertical PIN double heterostructure  p-Ge$_{0.94}$Sn$_{0.06}$/i-Ge$_{0.91}$Sn$_{0.09}$/n-Ge$_{0.95}$Sn$_{0.05}$ grown epitaxially on a silicon wafer using germanium interlayer and multiple GeSn buffer layers. The emission from these GeSn LEDs at variable diameters in the 40-120 $\mu$m range is investigated under both DC and AC operation modes. The fabricated LEDs exhibit a room temperature emission in the extended short-wave range centered around 2.5 $\mu$m under an injected current density as low as 45 A/cm$^2$.  By comparing the photoluminescence and electroluminescence signals, it is demonstrated that the LED emission wavelength is not affected by the device fabrication process or heating during the LED operation. Moreover, the measured optical power was found to increase monotonically as the duty cycle increases indicating that the DC operation yields the highest achievable optical power. The LED emission profile and bandwidth are also presented and discussed. 
\end{abstract}
\end{abstract}

\section{Introduction}\label{sec:intro}
\section{Introduction}

Many problems in econometrics, statistics, causal inference, and finance involve linear functionals of unknown functions:
\begin{equation}
\theta(g)=\E[m(Z; g)]
\end{equation}
where $Z$ denotes a random vector, and $g: \mcX\to \R$ is a function in some space $ \mcG$. A continuous linear functional that is mean square continuous with respect to $\ell_2$ norm can be written in a more benign and useful manner. Formally, for a given linear functional $\theta(\cdot)$, there exists a function $a_0$ such that for any $g\in \mcG$:\footnote{For simplicity of exposition, throughout the paper we consider scalar-valued functions $g$. All our results naturally extend to vector-valued functions $g$, and estimate a vector valued Riesz representer that satisfies that $\theta(g)=\E[a(X)'g(X)]$.}
\begin{equation}
    \theta(g) = \E[a_0(X)\, g(X)]
\end{equation}
This result is known as the Riesz representation theorem, and the function $a_0$ is the Riesz representer of the linear functional. Evaluation of a linear functional $\theta(g)$ can be achieved by simply taking the inner product between $a_0$ and $g$.

Knowing the Riesz representation of a linear functional is a critical building block in a variety of learning problems. For instance, in semi-parametric models, $g_0$ is an unknown regression function and $\theta(g_0)$ is a causal or structural parameter of interest. The Riesz representer $a_0$ of the functional $\theta(\cdot)$ can be used to debias the plug-in estimator and construct semi-parametrically efficient estimators of the parameter $\theta(g_0)$. In asset pricing applications, the Riesz representer corresponds to the stochastic discount factor, which is of primary interest when pricing financial derivatives.

Irrespective of the downstream application, the goal of this paper is to derive an estimator for the Riesz representer of any linear functional, when given access to $n$ samples of the random vector $Z$ and a target function space $\mcA$ that can well approximate the function $a_0$. We propose and analyze an estimator $\hat{a}$, with small mean-squared-error. Formally, with probability (w.p.) $1-\zeta$:
\begin{equation}
    \|\hat{a}-a_0\|_2 = \sqrt{\E\left[\left(\hat{a}(X) - a_0(X)\right)^2\right]} \leq \epsilon_{n,\zeta}
\end{equation}

We consider estimation of the Riesz representer within some function space $\mcA$ and propose an adversarial estimator based on regularized variants of the following min-max criterion:
\begin{equation}
    \hat{a} = \argmin_{a\in \mcA} \max_{f\in \mcF} \frac{1}{n}\sum_{i=1}^n \left(m(Z_i;f) - a(X_i)\cdot f(X_i) - f(X_i)^2\right)
\end{equation}
We derive oracle inequalities for this estimator as a function of the localized Rademacher complexity of the function space $\mcA$ and the approximation error $\epsilon = \min_{a\in \mcA} \|a-a_0\|_{2}$.

We show that as long as the function class $\mcF$ contains the star-hull of differences of functions in $\mcA$, i.e. $\mcF:= \{r(a-a'): a, a'\in \mcA, r\in [0, 1]\}$, then the estimation rate of the adversarial estimator achieves w.p. $1-\zeta$:
\begin{equation}
    \|\hat{a} - a_0\|_2 = O\left(\epsilon + \delta_n + \sqrt{\frac{\log(1/\zeta)}{n}}\right)
\end{equation}
where $\delta_n$ is the critical radius of the function classes $\mcF$ and $m\circ \mcF=\{Z\to m(Z; f): f\in \mcF\}$. The critical radius of a function class is a widely used quantity in statistical learning theory that allows one to argue fast estimation rates that are nearly optimal. For instance, for parametric function classes, the critical radius is of order $n^{-1/2}$, leading to fast parametric rates (as compared to $n^{-1/4}$ which would be achievable via looser uniform deviation bounds).

Moreover, the critical radius has been analyzed and derived for a variety of function spaces of interest, such as neural networks, high-dimensional linear functions, reproducing kernel Hilbert spaces, and VC-subgraph classes. Thus our general theorem allows us to appeal to these characterizations and provide oracle rates for a family of Riesz representer estimators. Prior work on estimating Riesz representers only considered particular high-dimensional parametric classes and derived specialized estimators for the function space of interest. Our adversarial estimator provides a single approach that tackles generic function spaces in a uniform manner.

We also examine the computational aspect of our estimator. We provide examples of how estimation can be achieved in a computationally efficient manner for several function spaces of interest.

Finally, we show how our estimator can be used in the context of estimating causal or structural parameters in semi-parametric models. Specifically, our mean square rate for the Riesz representer is sufficiently fast to achieve semi-parametric efficiency and asymptotic normality of the causal or structural parameter.

\subsection{Applications: Causal Inference and Asset Pricing}\label{sec:intro_examples}

This learning problem arises in two important domains for economic research: causal inference and asset pricing.

\paragraph{Automated De-biasing of Causal Estimates.} In causal inference, a variety of treatment effects and policy effects can be formulated as functionals--i.e., scalar summaries--of an underlying regression \cite{chernozhukov2016locally}. Formally, the causal parameter $\theta_0=\theta(g_0)=\mathbb{E}[m(Z;g_0)]$ is a functional $\theta(\cdot)$ of the nuisance parameter $g_0(x):=\mathbb{E}[Y|X=x]$. In this paper, we consider a variety of treatment and policy effects including
\begin{enumerate}
    \item Average treatment effect (ATE): $\theta_0=\mathbb{E}[g_0(1,W)-g_0(0,W)]$, where $X=(D,W)$ consists of treatment and covariates.
    \item Average policy effect: $\theta_0=\int g_0(x)d\mu(x)$ where $\mu(x)=F_1(x)-F_0(x)$
    \item Policy effect from transporting covariates: $\theta_0=\mathbb{E}[g_0(t(X))-g_0(X)]$
    \item Cross effect: $\theta_0=\mathbb{E}[Dg_0(0,W)]$, where $X=(D,W)$ consists of treatment and covariates.
    \item Regression decomposition: $\mathbb{E}[Y|D=1]-\mathbb{E}[Y|D=0]=\theta_0^{response}+\theta_0^{composition}$
    where
    \begin{align}
        \theta_0^{response}&=\mathbb{E}[g_0(1,W)|D=1]-\mathbb{E}[g_0(0,W)|D=1] \\
        \theta_0^{composition}&=\mathbb{E}[g_0(0,W)|D=1]-\mathbb{E}[g_0(0,W)|D=0]
    \end{align}
    \item Average treatment on the treated (ATT): $\theta_0=\mathbb{E}[g_0(1,W)|D=1]-\mathbb{E}[g_0(0,W)|D=1]$, where $X=(D,W)$ consists of treatment and covariates.
    \item Local average treatment effect (LATE): $\theta_0=\frac{\mathbb{E}[g_0(1,W)-g_0(0,W)]}{\mathbb{E}[h_0(1,W)-h_0(0,W)]}$, where $X=(V,W)$ consists of instrument and covariates and $h_0(x):=\mathbb{E}[D|X=x]$ is a second regression.
\end{enumerate}
More generally, our results extend to parameters defined implicitly by $0=\mathbb{E}[m(Z;g_0;\theta_0)]$, such as partially linear regression and partially linear instrumental variable regression.

    If the regression $g_0$ is learned by a regularized estimator $\hat{g}$, then estimation of the causal parameter $\theta_0$  by a plug-in estimator $\mathbb{E}_n[m(Z;\hat{g})]$ is badly biased. The solution is to use a de-biased formulation of the causal parameter instead: $\theta_0=\mathbb{E}[m(Z;g_0)+a_0(X)\{Y-g_0(X)\}]$. Observe that $a_0$ arises in the bias correction term. We re-visit this example in Section~\ref{sec:debiasing}.

%

\paragraph{Fundamental Asset Pricing Equation.} In asset pricing, a variety of financial models deliver the same fundamental asset pricing equation. This equation is of both theoretical and practical interest. Theoretically, it elucidates why asset prices or returns are what they are. Practically, it can be used to identify trading opportunities when assets are mis-priced. The asset pricing equation follows from two weak assumptions: free portfolio formation, and the law of one price.  In Appendix~\ref{sec:finance}, we review the derivation for a general audience.\footnote{The same asset pricing equation can be derived from either a model of complete markets for contingent claims, or a model of investor utility maximization. Free portfolio formation is a weaker assumption on market structure than the existence of complete markets for contingent claims. The law of one price is a weaker assumption on preference structure than investor utility maximization. We present these additional derivations in Appendix~\ref{sec:finance}.}

Formally, the fundamental asset pricing equation is $p_{t,i}=\mathbb{E}_t[m_{t+1}x_{t+1,i}]$ where $p_{t,i}$ is the price of asset $i$ at time $t$, $x_{t+1,i}$ is payoff of asset $i$ at time $t+1$, and $m_{t+1}$ is the market-wide stochastic discount factor (SDF) at time $t+1$.\footnote{The SDF has many additional names: marginal rate of substitution, state price density, and pricing kernel. Each name corresponds to a different derivation of the asset pricing equation, starting from different first principles.} The expectation is conditional on information $(I_t,I_{t,i})$ known at time $t$:  $I_t$ are macroeconomic conditioning variables that are not asset specific, e.g. inflation rates and market return; $I_{t,i}$ are asset-specific characteristics, e.g. the size or book-to-market ratio of firm $i$ at time $t$. The asset pricing equation encompasses stocks, bonds, and options. We clarify its many instantiations below, where $d_{t+1}$ is dividend, $C$ is the call price, $S_T$ is the stock price at expiration, $K$ is the strike price. 

\begin{table}[H]
       \centering
       \begin{tabular}{|c||c|c|}
        \hline 
            Asset & Price $p_t$ & Payoff $x_{t+1}$ \\
             \hline 
            \hline
            Stock &$p_t$& $p_{t+1}+d_{t+1}$ \\
              Bond &$p_t$&$1$\\
             Option &$C$&$\max\{S_T-K,0\}$ \\
             \hline 
            Return & $1$& $R_{t+1}$ \\
            Excess return &0&$R^e_{t+1}$ \\
            \hline 
       \end{tabular}
       \caption{Generality of asset pricing equation}
       \label{tab:my_label}
   \end{table}
 
 The fundamental asset pricing equation can also be parametrized in terms of returns. If an investor pays one dollar for an asset $i$ today, the gross rate of return $R_{t+1,i}$ is how many dollars the investor receives tomorrow; formally, the price is $p_{t,i}=1$ and the payoff is $x_{t+1,i}=R_{t+1,i}$ by definition. Next consider what happens when an investor borrows a dollar today at the interest rate $R_{t+1}^f$ and buys an asset $i$ that gives the gross rate of return $R_{t+1,i}$ tomorrow. From the perspective of the investor, who paid nothing out-of-pocket, the price is $p_{t,i}=0$ while the payoff is the excess rate of return $R_{t+1,i}^e:=R_{t+1,i}-R_{t+1}^f$, leading to the asset pricing equation: $0=\mathbb{E}_t[m_{t+1}R^e_{t+1,i}]$.
 
 
 Following \cite{chen2019deep}, we focus on the latter excess return parametrization of the asset pricing equation. Taking expectations yields the unconditional moment restriction
$$
0=\mathbb{E}[m_{t+1}R^e_{t+1,i}z(I_t,I_{t,i})]=\mathbb{E}[\mathbb{E}[m_{t+1}|R^e_{t+1,i},I_t,I_{t,i}]R^e_{t+1,i}z(I_t,I_{t,i})],\quad \forall z(\cdot)
$$
Our framework nests this final expression. Specifically,
$$
\theta(g)=0,\quad g(R^e_{t+1,i},I_t,I_{t,i})=R^e_{t+1,i}z(I_t,I_{t,i}),\quad a_0(R^e_{t+1,i},I_t,I_{t,i})=\mathbb{E}[m_{t+1}|R^e_{t+1,i},I_t,I_{t,i}]
$$
By estimating $a_0$, which is the projection of the SDF onto excess returns and other available information, one can pin down the price of any hypothetical asset. 

%
%
%
%

\subsection{Related Work}

\textbf{Classical Semi-parametric Statistics.} Classical semi-parametric statistical theory studies the asymptotic properties of statistical quantities that are functionals of a density or a regression over a low-dimensional domain \cite{levit1976efficiency,hasminskii1979nonparametric,ibragimov1981statistical,pfanzagl1982lecture,klaassen1987consistent,robinson1988root,van1991differentiable,bickel1993efficient,newey1994asymptotic,robins1995semiparametric,vaart,bickel1988estimating,newey1998undersmoothing,ai2003efficient,newey2004twicing,ai2007estimation,tsiatis2007semiparametric,kosorok2007introduction,ai2012semiparametric}. Any continuous linear functional has a Riesz representer. In this classical theory, the Riesz representer appears in the influence function and therefore in the asymptotic variance of semi-parametric estimators \cite{newey1994asymptotic}. We depart from classical theory by considering the high-dimensional setting.

\textbf{De-biased Machine Learning and Targeted Maximum Likelihood.} Because the Riesz representer appears in the asymptotic variance of semi-parametric estimators, it can be incorporated into estimation to ensure semi-parametric efficiency. In practice, this can be achieved by introducing a de-biasing term into the estimating equation \cite{hasminskii1979nonparametric,bickel1988estimating,zhang2014confidence,belloni2011inference,belloni2014inference,belloni2014uniform,belloni2014pivotal,javanmard2014confidence,javanmard2014hypothesis,javanmard2018debiasing,van2014asymptotically,ning2017general,chernozhukov2015valid,neykov2018unified,ren2015asymptotic,jankova2015confidence,jankova2016confidence,jankova2018semiparametric,bradic2017uniform,zhu2017breaking,zhu2018linear}. In doubly robust estimating equations for regression functionals, the de-biasing term is the product between the Riesz representer and the regression residual \cite{robins1995analysis,robins1995semiparametric,van2006targeted,van2011targeted,luedtke2016statistical,toth2016tmle}. The more general principle at play is Neyman orthogonality: the learning problem for the functional of interest becomes orthogonal to the learning problems for both the regression and the Riesz representer \cite{neyman1959,neyman1979c,vaart,robins2008higher,zheng2010asymptotic,belloni2014uniform,belloni2014pivotal,chernozhukov2016locally,belloni2017program,chernozhukov2018double,foster2019orthogonal}.

De-biased machine learning and targeted maximum likelihood combine the algorithmic insight of doubly-robust moment functions with the algorithmic insight of sample splitting \cite{bickel1982adaptive,schick1986asymptotically,klaassen1987consistent,vaart,robins2008higher}.  In doing so, these frameworks facilitate a general analysis of residuals such that the target functional is $\sqrt{n}$-consistent under minimal assumptions on the estimators used for the regression and Riesz representer \cite{scharfstein1999adjusting,rubin2005general,rubin2006extending,van2006targeted,zheng2010asymptotic,van2011targeted,diaz2013targeted,van2014targeted,kennedy2017nonparametric,kennedy2020optimal}. In particular, any machine learning estimators are permitted that satisfy $\sqrt{n}\|\hat{g}-g_0\|_2\cdot\|\hat{a}-a_0\|_2\rightarrow 0$ \cite{chernozhukov2018double,chernozhukov2016locally}.

The Riesz representer may be a difficult object to estimate. Even for simple regression functionals such as policy effects, its closed form involves ratios of densities. In restricted models, where the regression is known to belong to a certain function class, there is the further difficulty of projecting the Riesz representer accordingly. A recent literature explores the possibility of directly estimating the Riesz representer, without estimating its components or even knowing its functional form \cite{robins2007comment,newey2018cross,athey2018approximate,chernozhukov2018global,chernozhukov2018learning,hirshberg2018debiased,hirshberg2019augmented,singh2019biased,rothenhausler2019incremental}. A crucial insight, on which we build, is that the Riesz representer is directly identified from data. 

\cite{hirshberg2019augmented} observe that to debias an average moment, it is sufficient to estimate an empirical analogue of the Riesz representer that approximately satisfies the Riesz representer moment equation on the $n$ samples. They propose a parametric min-max criterion to estimate $n$ parameters corresponding to the $n$ evaluations of the empirical Riesz representer. Unlike \cite{hirshberg2019augmented}, we provide a guarantee on learning the true Riesz representer, we approximate the Riesz representer within non-parametric function spaces, and our result therefore has broader application beyond causal inference. Importantly, \cite{hirshberg2019augmented} require that the same sample used to estimate the $n$ parameters is used in final stage estimation of the causal parameter. As such, the analysis requires that the regression function $g$ lies in a Donsker class--a restriction that precludes many machine learning estimators. By contrast, our adversarial estimator provides fast estimation rates with respect to the true Reisz representer and hence can be used in combination with cross-fitting and sample splitting to eliminate the Donsker assumption.


\textbf{Adversarial Estimation.} Riesz representation theorem can be viewed as a continuum of unconditional moment restrictions. The non-parametric instrumental variable problem, based on a conditional moment restriction, also implies a continuum of unconditional moment restrictions \cite{newey2003instrumental,hall2005nonparametric,blundell2007semi,chen2009efficient,darolles2011nonparametric,chen2012estimation,chen2015sieve,chen2018optimal}. A central insight of this work is that the min-max approach for conditional moment models may be adapted to the problem of learning the Riesz representer. In a min-max approach, the continuum of unconditional moment restrictions is enforced adversarially over a set of test functions \cite{goodfellow2014generative,arjovsky2017wasserstein,dikkala2020minimax}. 

The fundamental advantage of the min-max approach is its unified analysis over arbitrary function classes. In particular, via local Rademacher analysis, one can derive an abstract bound that encompasses sparse linear models, neural networks, and RKHS methods \cite{koltchinskii2000rademacher,bartlett2005local}. As such, the min-max approach is actually a family of algorithms adaptive to a variety of data settings with a unified guarantee \cite{negahban2012,lecue2017regularization,Lecue2018}. 

\textbf{Machine Learning Approaches to Causal Inference and Asset Pricing.} By pursuing a min-max approach, our work relates to previous work that incorporates a variety of machine learning methods into causal inference. Much work on de-biased machine learning focuses on sparse and approximately sparse models \cite{chernozhukov2018global,chernozhukov2018learning,chernozhukov2018plug}. A neural network estimator with mean square rate has been successfully used to learn the nuisance regression in semiparametric estimation \cite{chen1999improved,farrell2018deep} and to learn the structural function in nonparametric instrumental variable regression \cite{deepiv,bennett2019deep,dikkala2020minimax}. A more recent literature incorporates RKHS methods into causal inference due to their convenient closed form solutions and strong performance on smooth designs \cite{nie2017quasi,singh2019kernel,muandet2019dual,singh2020kernel,muandet2020kernel}.

Finally, our works provides a theoretical foundation for a growing literature that incorporates machine learning into asset pricing. We follow the asset pricing literature in framing the problem of learning a stochastic discount factor as the problem of learning a Riesz representer \cite{hansen1997assessing}. Specifically, we propose a deep min-max approach based on free portfolio formation and the law of one price \cite{bansal1993no,chen2019deep}. This approach differs from deep learning approaches that predict asset prices via nonparametric regression \cite{messmer2017deep,feng2018deep,gu2020autoencoder,bianchi2020bond}. Unlike previous work, we prove mean square rates for the stochastic discount factor, and we prove $\sqrt{n}$-consistency and semiparametric efficiency for expected asset prices.

% \section{Problem Formulation}\label{sec:problem}
% \input{2_problem.tex}


% \section{Task Formulation}
% \label{sec:problem}
% \input{2_problem.tex}
 
% The \textsc{RiddleSense} is a $K$-way multi-choice question answering task. Given a question (i.e., a riddle) $q$, there are $K$ different choices $\{c_1, \dots, c_K\}$, where only one of them is the correct choice and the others are distractors.
% A model needs to rank all choices and select the best one.
% The evaluation metric will be the accuracy.
% Using external knowledge sources via distantly supervised learning and/or transfer learning is allowed and encouraged for solving \textsc{RiddleSense}.


% The correct answer can be either a shorter concept or phrase or a longer explanation to the situation in the riddle.

% \begin{itemize}
% 	\item An example for a \textbf{short-answer riddle}: ``\textit{It has five fingers. It's not a part of any animal. What is it?}'' {Choices:  ``A hand'' (\xmark); ``A nail'' (\xmark), ``A glove.'' (\cmark) ; ...}
% 	\item An example for a \textbf{long-answer riddle}: ``\textit{Two people are born at the same moment, but they don't have the same birthdays. How could this be?}'' {Choice: ``They are born in different hospitals.''(\xmark); ``They are born in different time zones.''(\cmark); ``They are twins.''(\xmark); ...}. 
	
% \end{itemize}


%There are two major types of riddles: 
%\begin{itemize}
%	\item \textbf{concept-based riddles} usually have a single concept as answers, often ending with ``What am I?'' or ``What is it?''
%	\item \textbf{explanation-based riddles} are about an 
%\end{itemize}
 


 

\section{Method}
Fig.~\ref{fig:framework} presents the illustration of the proposed \frameworkName.
In this section,  
we start by providing the problem definition of online CIL. Then, we describe the definition of the online prototype, the proposed online prototype equilibrium, and the proposed adaptive prototypical feedback. Finally, we propose an online prototype learning framework.

\subsection{Problem Definition}
Formally, online CIL considers a continuous sequence of tasks from a single-pass data stream $\mathfrak{D}=\left\{\mathcal{D}_1, \ldots, \mathcal{D}_T \right\} $, where $\mathcal{D}_t = \left\{ x_{i}, y_{i} \right\} ^{N_t}_{i=1} $ is the dataset of task $t$, and $T$ is the total number of tasks. Dataset $\mathcal{D}_t$ contains $N_t$ labeled samples, $y_{i}$ is the class label of sample $x_{i}$ and $y_{i} \in \mathcal{C}_t$, where $\mathcal{C}_t$ is the class set of task $t$ and the class sets of different tasks are disjoint. 
For replay-based methods, a memory bank is used to store a small subset of seen data, and we also maintain a memory bank $\mathcal{M}$ in our method.
At each time step of task $t$, the model receives a mini-batch data $X \cup X^\mathrm{b}$ for training, where $X$ and $X^\mathrm{b}$ are drawn from the i.i.d distribution $\mathcal{D}_t$ and the memory bank $\mathcal{M}$, respectively. 
Moreover, we adopt the single-head evaluation setup~\cite{EWC}, where a unified classifier must choose labels from all seen classes at inference due to unavailable task identifiers. 
The goal of online CIL is to train a unified model on data seen only once while predicting well on both new and old classes.

\subsection{Online Prototype Definition}
Prior to introducing the online prototypes, we first present the network architecture of our \frameworkName. Suppose that the model consists of three components: an encoder network $f$, a projection head $g$, and a classifier $\varphi$. Each sample $x$ in incoming data $X$ (a mini-batch data from new classes) is mapped to a projected vectorial embedding (representation) $\mathbf{z}$ by encoder $f$ and projector $g$:
\begin{align}
\label{eq:cal_z}
    \mathbf{z} = g(f(\operatorname{aug}(x);\theta_f);\theta_g),
\end{align}
where $\operatorname{aug}$ represents the data augmentation operation, $\theta_f$ and $\theta_g$ represent the parameters of $f$ and $g$, respectively, and $\mathbf{z}$ is $\ell_2$-normalized. 
Similar to Eq.~\eqref{eq:cal_z}, we use $\mathbf{z}^\mathrm{b}$ to denote the representation of replay data $X^\mathrm{b}$ (a mini-batch data from seen classes in the memory bank). 

At each time step of task $t$, the online prototype of each class is defined as the mean representation in a mini-batch:
\begin{align}
\label{eq:cal_p}
    \mathbf{p}_i = \frac{1}{n_i}\sum\nolimits_j\mathbf{z}_j\cdot \mathbbm{1}\{y_j = i\},
\end{align}
where $n_i$ is the number of samples for class $i$ in a mini-batch, and $\mathbbm{1}$ is the indicator function. 
We can get a set of $K$ online prototypes  in $X$, $\mathcal{P} = \left\{ \mathbf{p}_{i} \right\} ^{K}_{i=1}$, and a set of $K^\mathrm{b}$ online prototypes in $X^\mathrm{b}$, $\mathcal{P}^\mathrm{b} = \left\{ \mathbf{p}_i^\mathrm{b} \right\} ^{K^\mathrm{b}}_{i=1}$.
Note that $K = |\mathcal{P}| \leq |\mathcal{C}_t|$ and $K^\mathrm{b} = |\mathcal{P}^\mathrm{b}| \leq \sum_{i=1}^{t}|\mathcal{C}_i| $, where $|\cdot|$ denotes the cardinal number.



\subsection{Online Prototype Equilibrium}
The introduced online prototypes can provide representative features and avoid class-unrelated information.  
These characteristics are exactly the key to counteracting shortcut learning in online CL.
Besides, maintaining the discrimination among seen classes is also essential to mitigate catastrophic forgetting.
Based on these, we attempt to learn representative features of each class by pulling online prototypes $\mathcal{P}$ and their augmented views $\widehat{\mathcal{P}}$ closer in the embedding space, and learn discriminative features between classes by pushing online prototypes of different classes away, formally defined as a contrastive loss:
\begin{align}
\label{eq:proto_infoNCE}
    \ell(\mathcal{P},\widehat{\mathcal{P}})\!=\!
    % \frac{-1}{K}
    \frac{-1}{|\mathcal{P}|}\sum_{i=1}^{|\mathcal{P}|}\!\log\! 
    \tfrac
    {\exp \big(\tfrac{{\mathbf{p}_i^\mathrm{T} \widehat{\mathbf{p}}_i}}{\tau}\big)}
    {
    \sum\limits_{j} \exp \big(\tfrac{{\mathbf{p}_i^\mathrm{T} \widehat{\mathbf{p}}_j}}{\tau}\big)
    +\!
    \sum\limits_{\substack{j \neq i}} \exp \big(\tfrac{{\mathbf{p}_i^\mathrm{T} \mathbf{p}_j}}{\tau}\big) 
    },
\end{align}
where 
$\tau$ is the temperature hyper-parameter, 
$\mathcal{P}$ and $\widehat{\mathcal{P}}$ are $\ell_2$-normalized. To compute the contrastive loss across all positive pairs in both $(\mathcal{P}, \widehat{\mathcal{P}})$ and $(\widehat{\mathcal{P}}, \mathcal{P})$, we define $\mathcal{L}_{\mathrm{pro}}$ as the final contrastive loss over online prototypes:
\begin{align}
    \mathcal{L}_{\mathrm{pro}}(\mathcal{P},\widehat{\mathcal{P}}) = 
    \frac{1}{2}
    \left[\ell(\mathcal{P}, \widehat{\mathcal{P}}) + \ell(\widehat{\mathcal{P}}, \mathcal{P})\right].
\end{align}



Considering the learning of new classes and the consolidation of learned knowledge simultaneously in online CL, we propose Online Prototype Equilibrium (\methodname) to 
learn representative and discriminative features on both new and seen classes by employing $\mathcal{L}_{\mathrm{pro}}$:
\begin{equation}
    \begin{aligned}
    \mathcal{L}_{\mathrm{\methodname}}
    &=
    \mathcal{L}^{\mathrm{new}}_{\mathrm{pro}}(\mathcal{P},\widehat{\mathcal{P}}) + \mathcal{L}^{\mathrm{seen}}_{\mathrm{pro}}(\mathcal{P}^\mathrm{b},\widehat{\mathcal{P}}^\mathrm{b}),
    \end{aligned}
\end{equation}
where
$\mathcal{L}^{\mathrm{new}}_{\mathrm{pro}}$ focuses on learning knowledge from \emph{new} classes, and $\mathcal{L}^{\mathrm{seen}}_{\mathrm{pro}}$ is dedicated to preserving learned knowledge of all \emph{seen} classes.
\textit{This process is similar to a zero-sum game, 
and \methodname aims to achieve the equilibrium to play a win-win game.}
Concretely,
as the model learns, the knowledge of new classes is gained and added to the prototypes over the memory bank $\mathcal{M}$, causing $\mathcal{L}^{\mathrm{seen}}_{\mathrm{pro}}$ gradually changes to the equilibrium that separates all seen classes well, including new ones. 
This variation is crucial to mitigate forgetting and is consistent with the goal of CIL.



\subsection{Adaptive Prototypical Feedback} 
Although \methodname can bring an overall equilibrium, it tends to treat each class \emph{equally}. 
In fact, the degree of confusion varies among classes, 
and the model should focus purposefully on confused classes to consolidate learned knowledge. 
To this end, we propose Adaptive Prototypical Feedback (\dataaugname) with the feedback of online prototypes to sense the classes that are prone to be misclassified and then enhance their decision boundaries.
 
For each class pair in the memory bank $\mathcal{M}$, \dataaugname calculates the distances between online prototypes of all seen classes from the previous time step, showing the class confusion status by these distances. The closer the two prototypes are, the easier to be misclassified. 
Based on this analysis, 
our idea is to enhance the boundaries for those classes. Therefore, we convert the prototype distance matrix to a probability distribution $P$ over the classes via a symmetric Gaussian kernel, defined as follows:
\begin{align}
\label{eq:cal_d}
    P_{i, j} \propto \exp (-\| \mathbf{p}_i^\mathrm{b} - \mathbf{p}_j^\mathrm{b} \|_2^2),
\end{align}
where $i,j \in \{ 1, \ldots, |\mathcal{P}^\mathrm{b}| \}$ and $i \neq j$. 
Then, 
all probabilities are normalized to a probability mass function that sums to one.
\dataaugname returns probabilities to $\mathcal{M}$ for guiding the next sampling process and enhancing decision boundaries of easily misclassified classes. 


Our adaptive prototypical feedback is implemented as a sampling-based mixup. Specifically, 
\dataaugname adaptively selects more samples from easily misclassified classes in $\mathcal{M}$ for mixup~\cite{Mixup} according to the probability distribution $P$. 
Considering not over-penalizing the equilibrium of current online prototypes, we introduce a two-stage sampling strategy for replay data $X^\mathrm{b}$ of size $m$. 
First, we select $n_{\mathrm{\dataaugname}}$ samples  
with $P$, and a larger $P_{a,b}$ means more sampling from classes $a$ and $b$. Here, $n_{\mathrm{\dataaugname}} = \alpha \cdot m$, and $\alpha$ is the ratio of \dataaugname.
Second, the remaining $m-n_{\mathrm{\dataaugname}}$ samples are uniformly randomly selected from the entire memory bank to avoid the model only focusing on easily misclassified classes and disrupting the established equilibrium. 




\subsection{Overall Framework of \frameworkName}
The overall structure of \frameworkName is shown in Fig.~\ref{fig:framework}. \frameworkName comprises two key components based on proposed online prototypes: Online Prototype Equilibrium (\methodname) and Adaptive Prototypical Feedback (\dataaugname). 
With the two components, 
the model can learn representative features against shortcut learning, and 
all seen classes maintain discriminative when learning new classes. 
However, classes may not be compact, because the online prototypes cannot cover full instance-level information.
To further achieve intra-class compactness, 
we employ supervised contrastive learning~\cite{SupCL} to learn instance-wise representations:
\begin{equation}
\begin{aligned}
    \mathcal{L}_{\mathrm{INS}}
    &=
    \sum_{i=1}^{2N} \frac{-1}{\left|I_i\right|} \sum_{j \in I_i} \log \frac{\exp \left(\mathrm{sim}(\mathbf{z}_i, \mathbf{z}_j) / \tau^{\prime}\right)}{\sum\limits_{k \neq i} \exp \left(\mathrm{sim}(\mathbf{z}_i, \mathbf{z}_k) / \tau^{\prime}\right)}
    \\
    &+
    \sum_{i=1}^{2m} \frac{-1}{\left|I_i^{\mathrm{b}}\right|} \sum_{j \in I_i^{\mathrm{b}}} \log \frac{\exp (\mathrm{sim}(\mathbf{z}_i^{\mathrm{b}}, \mathbf{z}_j^{\mathrm{b}}) / \tau^{\prime})}{\sum\limits_{k \neq i} \exp \left(\mathrm{sim}(\mathbf{z}_i^{\mathrm{b}}, \mathbf{z}_k^{\mathrm{b}}) / \tau^{\prime}\right)},
\end{aligned}
\end{equation}
where $I_i=\left\{j \in\{1, \ldots, 2 N\} \mid j \neq i, y_j=y_i\right\}$ and $I_i^\mathrm{{b}}=\left\{j \in\{1, \ldots, 2m\} \mid j \neq i, y_j^\mathrm{b}=y_i^\mathrm{b}\right\}$ are the set of positive samples indexes to $\mathbf{z}_i$ and $\mathbf{z}_i^\mathrm{{b}}$, respectively. $y_i^\mathrm{b}$ is the class label of input $x_i^\mathrm{b}$ from $X^\mathrm{b}$. $N$ is the batch size of $X$. $\tau^{\prime}$ is the temperature hyperparameter.
The similarity function $\mathrm{sim}$ is computed in the same way as Eq.~(9) in OCM~\cite{OCM}.

Thus, the total loss of our \frameworkName framework is given as:
\begin{align}
    \mathcal{L}_{\mathrm{\frameworkName}}=\mathcal{L}_{\mathrm{\methodname}} + \mathcal{L}_{\mathrm{INS}} + \mathcal{L}_{\mathrm{CE}},
\end{align}
where $\mathcal{L}_{\mathrm{CE}} = \mathrm{CE}(y^\mathrm{b}, \varphi(f(\operatorname{aug}(x^\mathrm{b}))))$ is the cross-entropy loss; see Appendix~\ref{appendix:algorithm} for detailed training algorithms.

Following other replay-based methods~\cite{ER, SCR, OCM}, we update the memory bank in each time step by uniformly randomly selecting samples from $X$ to push into $\mathcal{M}$ and, if $\mathcal{M}$ is full, pulling an equal number of samples out of $\mathcal{M}$.
 



\begin{figure}[th!]
	\centering
% 	\vspace{-2em}
	\includegraphics[width=1\linewidth]{methods.pdf}
	\caption{Three types of baseline methods: 1) fine-tuning pre-trained LMs, 2) incorporating graph-based reasoner, 3) fine-tuning a unified text-to-text LM. }
	\label{fig:models} 
\end{figure}

\section{Experiments}\label{sec:exp}
\section{EXPERIMENTS} \label{Sec:exp}
\subsection{Dataset}

Recently, the task of describing video using natural language has gradually received more interest in the computer vision community. Eventually, many video description datasets have been released~\cite{Xu16_MSR_Dataset}. However, these datasets only provide general descriptions of the video and there is no detailed understanding of the action. The captions are also written using natural language sentences which can not be used directly in robotic applications. Motivated by these limitations, we introduce a new \textit{video to command} (IIT-V2C) dataset which focuses on \textit{fine-grained} action understanding~\cite{lea2016learning}. Our goal is to create a new large scale dataset that provides fine-grained understanding of human actions in a grammar-free format. This is more suitable for robotic applications and can be used with deep learning methods.

\textbf{Video annotation} 
Since our main purpose is to develop a framework that can be used by real robots for manipulation tasks, we use only videos that contain human actions. To this end, the raw videos in the Breakfast dataset~\cite{Kuehne14_BF_Dataset} are best suited to our purpose since they were originally designed for activity recognition. We only reuse the raw videos from this dataset and manually segment each video into short clips in a fine granularity level. Each short clip is then annotated with a \textit{command sentence} that describes the current human action.

\textbf{Dataset statistics} 
In particular, we reuse $419$ videos from the Breakfast dataset. The dataset contains $52$ unique participants performing cooking tasks in different kitchens. We segment each video (approximately $2-3$ minutes long) into around $10-50$ short clips (approximately $1-15$ seconds long), resulting in $11,000$ unique short videos. Each short video has a single command sentence that describes human actions. We use $70\%$ of the dataset for training and the remaining $30\%$ for testing. Although our new-form dataset is characterized by its grammar-free property for the convenience in robotic applications, it can easily be adapted to classical video captioning task by adding the full natural sentences as the new groundtruth for each video.


\subsection{Evaluation Metric, Baseline, and Implementation}
\textbf{Evaluation Metric} We report the experimental results using the standard metrics in the captioning task~\cite{Xu16_MSR_Dataset}: BLEU, METEOR, ROUGE-L, and CIDEr. This makes our results directly comparable with the recent state-of-the-art methods in the video captioning field.

\textbf{Baseline} We compare our results with two recent methods in the video captioning field: S2VT~\cite{Venugopalan2016} and SGC~\cite{Ramanishka2017cvpr}. The authors of S2VT used LSTM in the encoder-decoder architecture, while the inputs are from the features of RGB images (extracted by VGG16) and optical flow images (extracted by AlexNet). SGC also used LSTM with encoder-decoder architecture, however, this work integrated a saliency guided method as the attention mechanism, while the features are from Inception\_v3. We use the code provided by the authors of the associated papers for the fair comparison.


\begin{figure*}
\centering
\footnotesize
 %\stackunder[5pt]{\includegraphics[width=0.49\linewidth, height=0.09\linewidth]{figures/4_exp/result.pdf}}
  \stackunder[2pt]{\includegraphics[width=0.49\linewidth, height=0.09\linewidth]{figures/4_exp/P13_pancake_82.jpg}}  				
  				  {\tableCaption {righthand carry spatula} {righthand carry spatula} {lefthand reach stove} {lefthand reach pan}}
  				  \vspace{2ex} 
  \hspace{0.25cm}%
  \stackunder[2pt]{\includegraphics[width=0.49\linewidth, height=0.09\linewidth]{figures/4_exp/P38_salad_2.jpg}}  
  				  {\tableCaption {righthand cut fruit} {righthand cut fruit} {righthand cut fruit} {righthand cut fruit}}
  				  \vspace{2ex}
 %\stackunder[5pt]{\includegraphics[width=0.49\linewidth, height=0.09\linewidth]{figures/4_exp/result.pdf}} 
  \stackunder[2pt]{\includegraphics[width=0.49\linewidth, height=0.09\linewidth]{figures/4_exp/p07_pancake_6.jpg}}  
    			  {\tableCaption {righthand crack egg} {righthand carry egg} {lefthand reach spatula} {righthand carry egg}}
  				  \vspace{0ex}
  \hspace{0.25cm}%
 %\stackunder[5pt]{\includegraphics[width=0.49\linewidth, height=0.09\linewidth]{figures/4_exp/result.pdf}}  
  \stackunder[2pt]{\includegraphics[width=0.49\linewidth, height=0.09\linewidth]{figures/4_exp/P43_milk_15.jpg}}    
				  {\tableCaption {righthand stir milk} {righthand hold teabag} {righthand place kettle} {righthand take cacao}}
  				  \vspace{0ex}
  
  %\hspace{-0.25cm}%
 %\hspace{-0.25cm}%
\vspace{1ex}
\caption{Example of translation results of the S2VT, SGC and our LSTM\_Inception\_v3 network on the IIT-V2C dataset.}

\label{Fig:main_result} 
\end{figure*}


\textbf{Implementation} We use $512$ hidden units in both LSTM and GRU in our implementation. The first hidden state of LSTM/GRU is initialized uniformly in $[-0.1, 0.1]$. We set the number of frames for each input video at $30$. Sequentially, we consider each command has maximum $30$ words. If there are not enough $30$ frames/words in the input video/command, we pad the mean frame (from ImageNet dataset)/empty word at the end of the list until it reaches $30$.  During training, we only accumulate the softmax losses of the real words to the total loss, while the losses from the empty words are ignored. We train all the networks for $150$ epochs using Adam optimizer with a learning rate of $0.0001$. The batch size is empirically set to $16$. The training time for each network is around $3$ hours on a NVIDA Titan X GPU. 


\subsection{Results}

%//////////////////////////////////////////////
\begin{table}[!ht]
\centering\ra{1.4}
\caption{Performance on IIT-V2C Dataset}
\renewcommand\tabcolsep{2.5pt}
\label{tb_result_v2c}
\hspace{2ex}

\begin{tabular}{@{}rcccccccc@{}}
\toprule 					 &
\ssmall Bleu\_1  & 
\ssmall Bleu\_2  & 
%\multirow{1}{*}[2.5pt]{\scriptsize DeepLab~\cite{Chen2016_deeplab}} & 
\ssmall Bleu\_3  &
\ssmall Bleu\_4  & 
%\mu\ltirow{1}{*}[2.5pt]{\scriptsize BB-CNN} & 
\ssmall METEOR  &
\ssmall ROUGE\_L &
\ssmall CIDEr \\


\midrule
S2VT~\cite{Venugopalan2016} 				& 0.383   & 0.265   & 0.201	& 0.159	& 0.183    & 0.382   & 1.431     \\
SGC~\cite{Ramanishka2017cvpr}			& 0.370   & 0.256   & 0.198	& 0.161	& 0.179    & 0.371   & 1.422     \\
\cline{1-8}
LSTM\_VGG16					& 0.372   & 0.255  			 & 0.193	& 0.159	& 0.180    & 0.375   & 1.395     \\
GRU\_VGG16 					& 0.350   & 0.233 			 & 0.173	& 0.137	& 0.168    & 0.351   & 1.255     \\
LSTM\_Inception\_v3				& \textbf{0.400}  			 & \textbf{0.286}   & 0.221	& 0.178	& \textbf{0.194}    & \textbf{0.402}   & \textbf{1.594}     \\
GRU\_Inception\_v3 				& 0.391   & 0.281  			 & \textbf{0.222}	& \textbf{0.188}	& 0.190    & 0.398   & 1.588     \\
LSTM\_ResNet50 				& 0.398   & 0.279            & 0.215	& 0.174	& 0.193    & 0.398   & 1.550     \\
GRU\_ResNet50 				& 0.398   & 0.284   & 0.220	& 0.183	& 0.193    & 0.399   & 1.567     \\
\bottomrule
\end{tabular}
\end{table}

Table~\ref{tb_result_v2c} summarizes the captioning results on the IIT-V2C dataset. Overall, the LSTM network that uses visual features from Inception\_v3 (LSTM\_Inception\_v3) achieves the highest performance, winning on the Blue\_1, Blue\_2, METEOR, ROUGE\_L, and CIDEr metrics. Our LSTM\_Inception\_v3 also outperforms S2VT and SGC in all metrics by a fair margin. We also notice that both the LSTM\_ResNet50 and GRU\_ResNet50 networks give competitive results in comparison with the LSTM\_Inception\_v3 network. Overall, we observe that the architectures that use LSTM give slightly better results than those using GRU. However, this difference is not significant when the ResNet50 features are used to train the models (LSTM\_ResNet50 and GRU\_ResNet50 results are a tie).


From the experiments, we notice that there are two main factors that affect the results of this problem: the network architecture and the input visual features. Since the IIT-V2C dataset contains mainly the fine-grained human actions in a limited environment (i.e., the kitchen), the SGC architecture that used saliency guide as the attention mechanism does not perform well as in the normal video captioning task. On the other hand, the visual features strongly affect the final results. Our experiments show that the ResNet50 and Inception\_v3 features significantly outperform the VGG16 features in both LSTM and GRU networks. Since the visual features are not re-trained in the sequence to sequence model, in practice it is crucial to choose the state-of-the-art CNN as the feature extractor for the best performance.


Fig.~\ref{Fig:main_result} shows some examples of the generated commands by our LSTM\_Inception\_v3, S2VT, and SGC models on the test videos of the IIT-V2C dataset. These qualitative results show that our LSTM\_Inception\_v3 gives good predictions in many cases, while S2VT and SGC results are more variable. In addition to the good predictions that are identical with the groundtruth, we note that many other generated commands are relevant. Due to the nature of the IIT-V2C dataset, most of the videos are short and contain fine-grained human manipulation actions, while the groundtruth commands are also very short. This makes the problem of translating videos to commands is more challenging than the normal video captioning task since the network has to rely on the minimal information to predict the output command.  


\subsection{Robotic Applications}

%
%\begin{figure*}
%\centering
%\footnotesize
% %\stackunder[5pt]{\includegraphics[width=0.49\linewidth, height=0.09\linewidth]{figures/4_exp/result.pdf}}
%  \stackunder[2pt]{\includegraphics[width=0.49\linewidth, height=0.14\linewidth]{figures/5_robot/human_pick_hammer_cut/human_pick_all.jpg}}  				
%  				  {Human instruction}
%  				  \vspace{2ex} 
%  \hspace{0.25cm}%
%  \stackunder[2pt]{\includegraphics[width=0.49\linewidth, height=0.14\linewidth]{figures/5_robot/pick_place_hammer_cut/pick_all.jpg}}  
%  				  {Robot execution}
%  				  \vspace{2ex}
% %\stackunder[5pt]{\includegraphics[width=0.49\linewidth, height=0.09\linewidth]{figures/4_exp/result.pdf}} 
%  \stackunder[2pt]{\includegraphics[width=0.49\linewidth, height=0.14\linewidth]{figures/5_robot/human_pour_bottle_cut/human_pour_all.jpg}}  
%    			  {Human instruction}
%  				  \vspace{0ex}
%  \hspace{0.25cm}%
% %\stackunder[5pt]{\includegraphics[width=0.49\linewidth, height=0.09\linewidth]{figures/4_exp/result.pdf}}  
%  \stackunder[2pt]{\includegraphics[width=0.49\linewidth, height=0.14\linewidth]{figures/5_robot/pour_bottle_cut/pour_all.jpg}}    
%				  {Robot execution}
%  				  \vspace{0ex}
%%  \stackunder[2pt]{\includegraphics[width=0.49\linewidth, height=0.13\linewidth]{figures/5_robot/pick_place_hammer_cut/pick_all.jpg}}    
%%				  {\tableCaption {righthand stir milk} {righthand hold teabag} {righthand place kettle} {righthand take cacao}}
%%  				  \vspace{0ex}
%%  \stackunder[2pt]{\includegraphics[width=0.49\linewidth, height=0.13\linewidth]{figures/5_robot/pick_place_hammer_cut/pick_all.jpg}}    
%%				  {\tableCaption {righthand stir milk} {righthand hold teabag} {righthand place kettle} {righthand take cacao}}
%%  				  \vspace{0ex}
%    
%  %\hspace{0.35cm}%
% %\hspace{-0.25cm}%
%\vspace{3ex}
%\caption{ROBOT Example Results. TBD.}
%\label{Fig:robot_imitation} 
%\end{figure*}


\begin{figure*}[ht]
  \centering
 \subfigure[Pick and place task]{\label{fig_resize_map_a}\includegraphics[width=0.99\linewidth, height=0.16\linewidth]{figures/5_robot/pick_all_in_one.pdf}}
 \subfigure[Pouring task]{\label{fig_resize_map_b}\includegraphics[width=0.99\linewidth, height=0.16\linewidth]{figures/5_robot/pour_all_in_one.pdf}}
    
     
 \vspace{2.0ex}
 \caption{Example of manipulation tasks performed by WALK-MAN using our proposed framework. \textbf{(a)} Pick and place task. \textbf{(b)} Pouring task. The frames from human instruction videos are on the left side, while the robot performs actions on the right side. We notice that there are two sub-tasks (i.e., two commands) in these tasks: grasping the object and manipulating it. More illustrations can be found in the supplemental video.}
 \label{Fig:robot_imitation}
\end{figure*}



Given the proposed translation module, we build a robotic framework that allows the robot to perform various manipulation tasks by just ``\textit{watching}" the input video. Our goal in this work is similar to~\cite{Yang2015}, however, we propose to keep the video understanding separately from the vision system. In this way, the robot can learn to understand the task and execute it independently. This makes the proposed approach more practical since it does not require a dataset that has both the caption and the object (or grasping) location. It is also important to note that our goals differ from LfD since we only focus on finding a general way to let the robot execute different manipulation actions, while the trajectory in each action is assumed to be known.

In particular, for each task presented by a video, the translation module will generate an output command sentence. Based on this command, the robot uses its vision system to find relevant objects and plan the actions. Experiments are conducted using the humanoid WALK-MAN~\cite{Niko2016_full}. The robot is controlled using the XBotCore software architecture~\cite{muratore2017xbotcore}, while the OpenSoT library~\cite{Rocchi15} is used to plan full-body motion. The relevant objects and their affordances are detected using AffordanceNet framework~\cite{AffordanceNet17}. For simplicity, we only use objects in the IIT-Aff dataset~\cite{Nguyen2017_Aff} in the demonstration videos so the robot can recognize them. Using this setup, the robot can successfully perform various manipulation tasks by closing the loop: understanding the human demonstration from the video using the proposed method, finding the relevant objects and grasping poses~\cite{Nguyen2017_Aff}, and planning for each action~\cite{Rocchi15}.


Fig.~\ref{Fig:robot_imitation} shows some manipulation tasks performed by WALK-MAN using our proposed framework. For a simple task such as ``righthand grasp bottle", the robot can effectively repeat the human action through the command. Since the output of our translation module is in grammar-free format, we can directly map each word in the command sentence to the real robot command. In this way, we avoid using other modules as in~\cite{Tellex2011} to parse the natural command into the one that uses in the real robot. The visual system also plays an important role in our framework since it provides the object location and the target frames (e.g., grasping frame, ending frame) for the robot to plan the actions. Using our approach, the robot can also complete long manipulation tasks by stacking a list of demonstration videos in order for the translation module. Note that, for the long manipulation tasks, we assume that the ending state of one task will be the starting state of the next task. Overall, WALK-MAN successfully performs various manipulation tasks such as grasping, pick and place, or pouring. The experimental video and our IIT-V2C dataset can be found at the following link:
\vspace{1ex}
\centerline{\url{https://sites.google.com/site/video2command/}}






\junk{
 Furthermore, we can also  We can also stack several videos to create a chain of commands. 
 
Using our approach, the robot can perform different tasks based on the instructions from human demonstration videos.

  can understand the human instructions from the video via the proposed translation module, while the visual information can be solved effectively with the recent advances in deep learning~\cite{Nguyen2017_Aff}.
  
  
, and we already have many separated datasets for video captioning~\cite{Xu16_MSR_Dataset} and affordance detection~\cite{Nguyen2017_Aff}


 using all standard metrics
 
 We train each LSTM/GRU network using the features from VGG16, Inception\_v3 and ResNet50, respectively.
 
 method for video captioning and the recent method that used encoder-decoder scheme with saliency guided~\cite{Ramanishka2017cvpr} (denoted as SGC) as the attention mechanism.


   \textbf{Grasping}, \textbf{Pick and Place} In this experiment, we address how the robot executes a new scenario, i.e. ``pick up an object and place it into another object". Our experiment involves two affordances \texttt{grasp} and \texttt{contain}. Using our approach, the robot can grasp an object from its \texttt{grasp} affordance and bring it to a new location that belongs to the \texttt{contain} affordance of another object. Furthermore, we note that with the aid of our semantic understanding framework, the robot can recognize the target objects while ignoring the irrelevant ones. \textbf{Pouring} Similar to the pick and place experiment, we use two affordances \texttt{grasp} and \texttt{contain} and the associated objects in our dataset. The goal is to pour the liquid from an object to the \texttt{contain} region of another object. Our semantic perception framework gives the robot detail understanding about the target objects, their affordances as well as the relative coordinates for the movement. We notice that besides the basic actions such as grasp, raise, etc., we predefined the pouring action to help the robot complete the task. 
   
 we use the vision system from~\cite{Nguyen2017_Aff} to provide object location, its affordances and grasping frame for the robot, while trajectory is generated by the OpenSoT library~\cite{Rocchi15}. For simplicity, we only use the objects that the IIT-Aff~\cite{Nguyen2017_Aff} dataset so the robot can recognize them. Since there are a huge gap between the number of words in the IIT-V2C dataset  and the the number of objects that the robot can recoginze in the  in the  . videoFor each To create the command for  n our work, instead of using full natural command sentences, we propose to use human demonstrations from videos as the input. The translation module is then used to interpret the video to commands in grammar-free format that the robot can follow. Combined with the vision and planning system, our approach allows the robot to perform manipulation tasks by just ``watching" the input video. Fig.~\ref{Fig:robot_apps} shows a full description of framework in our robotic applications. 


We validate our framework using the WALK-MAN full-size humanoid~\cite{Niko2016_full}. The robot is controlled in real-time using the XBotCore software architecture~\cite{muratore2017xbotcore}, while the OpenSoT library~\cite{Rocchi15} is used to plan full body motion. The relevant objects and their affordances are detected using the framework in~\cite{Nguyen2017_Aff}. To allow the real-time performance, the control and planning system run a control pc, while the vision system runs on a vision pc with a NVIDIA Titan X GPU.


For the safety of the robot, we define some key action such as "reach", "grasp". Each word in the output command will be compared with these basic action using the similarity in word2vector. 

Fig.~\ref{Fig:robot_apps} shows an overview of our robotic application.


To compare the origi- nality in generation, we compute the Levenshtein distance of the predicted sentences with those in the training set. From Table 3, for the MSVD corpus, 42.9 of the predic- tions are identical to some training sentence, and another 38.3 can be obtained by inserting, deleting or substituting one word from some sentence in the training corpus. We note that many of the descriptions generated are relevant. TODO: challenging, failures case.


 We follow the standard procedure in the captioning tasks to evaluate our results.
 
 
 In particular, we use the code from COCO evaluation server~\cite{Chen_COCO_Evaluation} that implements several metrics:

 the baseline network architecture remains challenging to modify, most of the recent work used different models such as attention mechanism~\cite{•}, saliency 



GoogLeNet [32, 12] to extract the frame-level features in our experiment. All the videos’ lengths are kept to 200 frames. For a video with more than 200 frames, we drop the extra frames. For a video with- out enough frames, we pad zero frames. These are com- mon approaches to ensure all the videos have the same length [38, 43]. Feature Extractor: CNN. We consider the input video as a sequence of frames and encode each frame using a CNN. This process extracts the meaningful features from the input images at every time step. These features are then fed to the LSTM network as inputs. In particular, we use three most popular CNNs: VGG16, GoogleNet, and ResNet-50 as our feature extractors. For the VGG16 and ResNet-50, we remove the last classification layer and TODO: describe net without the last layer.
\\




The MSR-VTT dataset is characterized by the unique properties including the large scale clip-sentence pairs, comprehensive video categories, diverse video content and descriptions, as well as multimodal audio and video stream- s. We



Current datasets for video to text mostly focus on specific fine-grained domains. For example, YouCook [5], TACoS [25, 28] and TACoS Multi-level [26] are mainly de- signed for cooking behavior. MSR-VTT focuses on general videos in our life, while MPII-MD [27] and M-VAD [32] on movie domain. Although MSVD [3] contains general web videos which may cover different categories, the very limited size (1,970) is far from representativeness. To col- lect representative videos, we obtain the top 257 represen- tative queries from a commercial video search engine, cor- responding to 20 categories


Since our goal is to collect short video clips that each can be described with one single sentence in our current version


. TOFIX: Similar to the treatment of frame features, we embed words to a lower 500 dimensional space by applying a linear transformation to the input data and learning its parameters via back propa- gation. The embedded word vector concatenated with the output (ht) of the first LSTM layer forms the input to the second LSTM layer (marked green in Figure 2). When considering the output of the LSTM we apply a softmax over the complete vocabulary as in Equation 5.




}  



\section{Related Work}\label{sec:rel_work}
\section{Related Work}
A subfield of NLG analyzes the role of human evaluations, including discussions of the tradeoffs of human and automatic evaluations \citep{belz-reiter-2006-comparing, hashimoto-etal-2019-unifying}.
There are critiques and recommendations for different aspects of human evaluations, like the evaluation design \citep{novikova-etal-2018-rankme, santhanam-shaikh-2019-towards}, question framing \citep{schoch-etal-2020-problem}, and evaluation measures like agreement \citep{amidei-etal-2018-rethinking}, as well as analyses of past NLG papers' human evaluations \citep{vanderlee_journal, howcroft-etal-2020-twenty}.
Additionally, crowdsourcing literature has work on effectively using platforms like Amazon Mechanical Turk \citep[e.g.,][]{florian_crowdsourcing,oppenheimer_crowdsourcing,weld_crowdsourcing,mitra_crowdsourcing}.
In this work, we focus on the role evaluator training can play for producing better accuracy at distinguishing human- and machine-generated text, though other quality control methods are worth exploring.

Previous work has asked evaluators to distinguish between human- and machine-authored text. For example, \citet{ippolito-etal-2020-automatic} found that trained evaluators were able to detect open-ended GPT2-L-generated text 71.4\% of the time, \citet{garbacea-etal-2019-judge} reported that individual evaluators guessed correctly 66.6\% of the time when evaluating product reviews, and \citet{gpt3} found evaluators could guess GPT3-davinci-generated news articles' source with 52\% accuracy, though these results are not directly comparable to ours due to differences in the evaluation setup, data, and participants.

Finally, our findings that untrained evaluators are not well equipped to detect machine-generated text point to the importance of researching the safe deployment of NLG systems. \citet{gehrmann-etal-2019-gltr} proposed visualization techniques to help readers detect generated text, and work like \citet{grover_zellers}, \citet{ippolito-etal-2020-automatic}, and \citet{uchendu-etal-2020-authorship} investigated large language models' ability to detect generated text.





\section{Conclusion}\label{sec:conclusion}
 We propose a novel commonsense reasoning challenge, \textsc{RiddleSense}, which requires complex commonsense skills for reasoning about creative and counterfactual questions, coming with a large multiple-choice QA dataset.  
 We systematically evaluate recent commonsense reasoning methods over the proposed \textsc{RiddleSense} dataset, and find that the best model is still far behind human performance, suggesting that there is still much space for commonsense reasoning methods to improve.
 We hope \textsc{RiddleSense} can serve as a benchmark dataset for future research targeting complex commonsense reasoning and computational creativity.


\section*{Acknowledgements}
This research is supported in part by the Office of the Director of National Intelligence (ODNI), Intelligence Advanced Research Projects Activity (IARPA), via Contract No. 2019-19051600007, the DARPA MCS program under Contract No. N660011924033 with the United States Office Of Naval Research, the Defense Advanced Research Projects Agency with award W911NF-19-20271, and NSF SMA 18-29268. The views and conclusions contained herein are those of the authors and should not be interpreted as necessarily representing the official policies, either expressed or implied, of ODNI, IARPA, or the U.S. Government. We would like to thank all the collaborators in USC INK research lab and the reviewers for their constructive feedback on the work.

\section*{{Ethical Considerations}}

\paragraph{Copyright of Riddles.} 
The RiddleSense dataset is consistent with the terms of use of the fan websites and the intellectual property and privacy rights of the original sources.
All of our riddles and answers are from fan websites that can be accessed freely.
The website owners state that we may print and download material from the sites solely for \textit{non-commercial use} provided that we agree not to change or delete any copyright or proprietary notices from the materials.
Therefore, 
in addition to the dataset itself, we also provide the according copyright statements of every website and an URL link to the original page for each riddle. 
The dataset users must sign an informed consent form that they will only use our dataset for \textit{research purposes} before they can access the both the riddles and our annotations.


% \paragraph{Crowd-workers.} This work presents a new dataset for addressing a new problem, riddle-style common-sense reasoning.  
% The wrong choices within the dataset were produced by filtering questions and using crowd-workers to annotate riddle-style common-sense questions by suggesting additional distractors. 
% Most of the questions are about common knowledge about our physical world.
% \textit{None of the questions involve sensitive personal opinions or involve personally identifiable information. }
% We study posted tasks to be completed by crowd-workers instead of crowd workers themselves, and we do not retrieve any identifiable private information about a human subject.
% All annotators were fairly compensated by the Amazon Mechanical Turk platform solely based on the quantity and quality of their submissions.


% We used 
% \yl{yuchen can you add compensation and IRB details}


% \smallskip
% \noindent
% \textbf{Data bias.} Like most crowdsourced data, and in particular most common-sense data, these crowdsourced answers are inherently subject to bias: for example, a question like ``'' might be answered very differently by people from different backgrounds and cultures.  
% % The prior multiple-choice CSR datasets which our datasets are built on are strongly biased culturally, as they include a single correct answer and a small number of distractor answers, while our new datasets include many answers considered correct by several annotators.  
% However, this potential bias (or reduction in bias) has not been systematically measured in this work.
 


% \smallskip
% \noindent
% \textbf{Sustainability.} 
% For most of the experiments,
% we use the virtual compute engines on Google Cloud Platform, which ``is committed to purchasing enough renewable energy to match consumption for all of their operations globally.''\footnote{\url{https://cloud.google.com/sustainability}}
% With such virtual machine instances, we are able to use the resources only when we have jobs to run, thus avoiding unnecessary waste.






% \appendix
%
%\section{Resources}
%The dataset will be collected mainly from the web.
%We have already found some websites such as \texttt{riddles.com}, \texttt{riddles.tips}, \texttt{riddles.fyi}, \texttt{brainzilla.com}, and etc.
%They are all \textbf{free and publicly available. }
%
%\section{Risks}
%There are two major risks:
%\begin{itemize}
%	\item The web-crawled data is not enough. If the riddles are not enough, then we have to use them as the test data only, and argue that the training data will be the distant supervision from the Wiktionary glossary, or transferred data from other datasets. The key motivation of the paper won't change, though.
%	\item The generated distractors are too weak. We need to design better methods for generating distractors. The worst case is that we may need to add  human-annotated distractors (one for each test riddle) by ourselves to make sure the data is challenging enough. 
%\end{itemize}
%
%
%
%\section{Mid-term Checklist}
%\begin{enumerate}
%	\item An finalized version of the crawled, cleaned set of riddles. 
%	\item A first version of the distractors and the experimental pipeline for running the full evaluation.
%	\item The data analysis pipeline for reporting and visualizing the data (i.e., the connectivity between question concepts and answer concepts). 
%\end{enumerate}
%


\bibliography{riddleqa_rebiber} 
\bibliographystyle{acl_natbib}

\end{document}






