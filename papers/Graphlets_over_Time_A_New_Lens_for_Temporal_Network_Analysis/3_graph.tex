\begin{table*}[t]
\begin{center}
\vspace{-2mm}
\caption{\label{tab:graphlet_evolution} 
Ratios of graphlets over time. The colors in the plots are matched with the colors of the graphlets in Figure~\ref{fig:graphlet_and_role}, and the evolution ratio means the fraction of edges added to graphs.
The evolution patterns in real-world graphs vary depending on domains (Observation~\ref{obs:graphlet_evolve}), and they are clearly distinguished from the evolution patterns in randomized graphs (Observation~\ref{obs:graphlet_evolve:random}). 
}
%The graphlet evolution pattern of $\SG$ and $\SGR$. The colors are matched with those in Figure~\ref{fig:graphlet_and_role} (a) to indicate the types of graphlets. The evolution patterns of $\SG$ differs from those in $\SGR$ and they also vary depending on the domain of the graphs.}
%\vspace{-3mm}
\scalebox{0.95}{
\begin{tabular}{c|ccc|ccc}
    \toprule
    & \multicolumn{3}{c|}{Temporal graph $\SG$} & \multicolumn{3}{c}{Randomized graph $\SGR$} \\
    \hline
    \parbox[t]{2mm}{\multirow{8}{*}{\rotatebox[origin=c]{90}{\ \ \ \ \ Citation}}} &  
        \raisebox{-.9\totalheight}{\includegraphics[width=0.1475\textwidth]{FIG/graphlet_distribution/hepph.pdf}} &
        \raisebox{-.9\totalheight}{\includegraphics[width=0.1475\textwidth]{FIG/graphlet_distribution/hepth.pdf}} & 
        \raisebox{-.9\totalheight}{\includegraphics[width=0.1475\textwidth]{FIG/graphlet_distribution/patent.pdf}} & 
        \raisebox{-.9\totalheight}{\includegraphics[width=0.1475\textwidth]{FIG/graphlet_distribution/hepph-random.pdf}} & \raisebox{-.9\totalheight}{\includegraphics[width=0.1475\textwidth]{FIG/graphlet_distribution/hepth-random.pdf}} &
        \raisebox{-.9\totalheight}{\includegraphics[width=0.1475\textwidth]{FIG/graphlet_distribution/patent-random.pdf}} \\
    \hline
    \parbox[t]{2mm}{\multirow{8}{*}{\rotatebox[origin=c]{90}{\ \ \ \ \ Email/Message}}} &  
        \raisebox{-.9\totalheight}{\includegraphics[width=0.1475\textwidth]{FIG/graphlet_distribution/email-eu.pdf}} & \raisebox{-.9\totalheight}{\includegraphics[width=0.1475\textwidth]{FIG/graphlet_distribution/enron.pdf}} & 
        \raisebox{-.9\totalheight}{\includegraphics[width=0.1475\textwidth]{FIG/graphlet_distribution/college_msg.pdf}} & 
        \raisebox{-.9\totalheight}{\includegraphics[width=0.1475\textwidth]{FIG/graphlet_distribution/email-eu-random.pdf}} & \raisebox{-.9\totalheight}{\includegraphics[width=0.1475\textwidth]{FIG/graphlet_distribution/enron-random.pdf}} &
        \raisebox{-.9\totalheight}{\includegraphics[width=0.1475\textwidth]{FIG/graphlet_distribution/college_msg-random.pdf}} \\
    \hline
    \parbox[t]{2mm}{\multirow{8}{*}{\rotatebox[origin=c]{90}{\ \ \ \ Online Q/A}}} &  
        \raisebox{-.9\totalheight}{\includegraphics[width=0.1475\textwidth]{FIG/graphlet_distribution/mathoverflow.pdf}} & \raisebox{-.9\totalheight}{\includegraphics[width=0.1475\textwidth]{FIG/graphlet_distribution/askubuntu.pdf}} & 
        \raisebox{-.9\totalheight}{\includegraphics[width=0.1475\textwidth]{FIG/graphlet_distribution/stackoverflow.pdf}} & 
        \raisebox{-.9\totalheight}{\includegraphics[width=0.1475\textwidth]{FIG/graphlet_distribution/mathoverflow-random.pdf}} & \raisebox{-.9\totalheight}{\includegraphics[width=0.1475\textwidth]{FIG/graphlet_distribution/askubuntu-random.pdf}} &
        \raisebox{-.9\totalheight}{\includegraphics[width=0.1475\textwidth]{FIG/graphlet_distribution/stackoverflow-random.pdf}} \\
%& \hspace{0.4cm} Math & \hspace{0.4cm} Ask & \hspace{0.4cm} Stack & \hspace{0.4cm} Math & \hspace{0.4cm} Ask & \hspace{0.4cm} Stack \\
    \bottomrule
\end{tabular}}
\end{center}
\end{table*}


\begin{algorithm}[t]
\small
	\caption{Counting the Instances of Each Graphlet in a Temporal Graph \label{alg:track_graphlet}} 
	\SetKwInOut{Input}{Input}
	\SetKwInOut{Output}{Output}
	\SetKwProg{Procedure}{Procedure}{}{}
	\SetKwFunction{update}{UPDATE}
	
	\Input{Temporal Graph $\SG=(\SV, \SE, \ST)$}
	\Output{The count of the instances of each graphlet in $\SG$}
	\vspace{3pt}
	
	%\texttt{\color{blue}/* Initialization */}\\
	Initialize the count of the instances of each graphlet to zero \\
	Initialize $\SE$ to an empty set \\
    \vspace{3pt}
    
	\For{\upshape\textbf{each edge }$e_i=u \rightarrow v$ in arrival order}{
	    
	    $\SN \leftarrow $ union of the neighbors of $u$ and the neighbors of $v$ (except for $u$ and $v$) \label{alg:track_graphlet:line:neighbors}
	    
	    \For{\upshape\textbf{each} $w\in \SN$}{
	       % \update{$u$, $v$, $w$} \\
	        \If{$u$, $v$ \textbf{and} $w$ form a graphlet instance}{ \label{alg:track_graphlet:line:if_instance}
	         decrement the count of the graphlet of the instance formed by $u$, $v$ and $w$ \label{alg:track_graphlet:line:decrease}
	         \\
            
            %  \If{ \upshape{is\_inverse} \upshape\textbf{is} True}{
            %      decrease the count of the instance of graphlet prev by $1$ \\
            %      increase the count of the instance of graphlet next by $1$ \\
            %  }
            }
         }
         
         add $u\rightarrow v$ to $\SE$
         
         \For{\upshape\textbf{each} $w\in \SN$}{
              increment the count of the graphlet of the instance formed by $u$, $v$ and $w$ \label{alg:track_graphlet:line:increase}
	         \\
         }
	    \vspace{3pt}
	    
	}
	
	\textbf{return} count of the instances of each graphlet instances \\
\end{algorithm}

\section{Graph Level Analysis}\label{section:graph}

In this section, we study the evolution of local structures in real-world graphs.
We examine the dynamics in the distribution of graphlet instances and transitions between graphlets.
%We first observe the distributions of graphlet instances over time in the graphs. After that, we study patterns of transitions between graphlets.

\subsection{Global Level 1. Graphlets Over Time}
\label{section:graph:time}
We track how the ratio of the instances of each graphlet changes as the considered real-world graphs evolve over time. Our tracking algorithm, which is described in Algorithm~\ref{alg:track_graphlet}, is adapted from StreaM \cite{schiller2015stream}, which maintains the counts of the instances of the $4$-node undirected graphlets in a fully dynamic graph stream, where edges are not just added but also deleted over time. The time complexity of Algorithm~\ref{alg:track_graphlet} is $\Theta(\Sigma_{v\in\SV}(d(v))^2)$, as proven in Theorem~\ref{thm:time:track}.
It should be noticed that, by Lemma~\ref{lem:time:optimality}, the time complexity is $\Theta($the number of instances of all graphlets in the last snapshot$)$, which is the optimal time complexity achievable by any algorithm that counts graphlet instances by enumerating them.

\begin{theorem} \label{thm:time:track} 
The time complexity of Algorithm \ref{alg:track_graphlet} is $\Theta(\Sigma_{v\in\SV}(d(v))^2)$.
%Given a temporal graph $\SG=(\SV, \SE, \ST)$, Algorithm~\ref{alg:track_graphlet} takes $\Theta(\Sigma_{v\in\SV}(d(v))^2)$ times for counting the number of instances of all graphlets in $\SG$.
\end{theorem}
\begin{proof}
Since the number of nodes forming each graphlet instance is a constant, finding the graphlet corresponding to a given instance and updating the corresponding count (lines \ref{alg:track_graphlet:line:if_instance}-\ref{alg:track_graphlet:line:decrease} and \ref{alg:track_graphlet:line:increase}) take $O(1)$ time. Thus, the time complexity of processing each incoming edge $e_i=u\rightarrow v$ is that of computing the union of the neighbors of $u$ and $v$ (line~\ref{alg:track_graphlet:line:neighbors}), which is $\Theta(\DTUi+\DTVi)$.
Hence, the total complexity is $\Theta(\sum_{e_i=u \rightarrow v \in E} (\DTUi+\DTVi) = \Theta(\sum_{v\in \SV}(d(v))^2)$.
\end{proof}
\begin{lemma} \label{lem:time:optimality} 
The number of instances of all graphlets in a snapshot $\SGT$ is $\Theta(\Sigma_{v\in\SVT}(\DTV)^2)$.
\end{lemma}
\begin{proof}
Given a snapshot $\SGT=(\SVT, \SET)$, for each node $v\in \SVT$, if we count the instances of all graphlets that consist of $v$ and its two neighbors, then the count of such instances is $\Theta((\DTV)^2)$ for each node $v$, and since $\DTV\geq 1$ for every node $v$, the total count $C$ is $\Theta(\Sigma_{v\in\SVT}(\DTV)^2)$. 

\smallsection{Lower Bound:} Since each graphlet instance, which consists of three nodes, is counted at most three times, $C$ is at most three times the number of instances of all graphlets in $\SGT$.
In other words, the number of instances of all graphlets is at least $1/3$ of $C$, and thus it is $\Omega(\Sigma_{v\in\SVT}(\DTV)^2)$.

\smallsection{Upper Bound:} In each graphlet instance, there exists at least one center node, who composes the graphlet together with its neighbors. Thus, each instance is counted at least once, and thus $C$ is at least the number of instances of all graphlets in $\SGT$.
In other words, the number of instances of all graphlets is at most $C$, and thus it is $O(\Sigma_{v\in\SVT}(\DTV)^2)$.
\end{proof}

As seen in Table~\ref{tab:graphlet_evolution}, the dynamics of the ratios depend on the domains of the graphs, as summarized in Observation~\ref{obs:graphlet_evolve}.

\noindent\fbox{%
        \parbox{\columnwidth}{%
        \vspace{-2mm}
        \begin{observation} \label{obs:graphlet_evolve}
            The dynamics in the distributions of graphlet instances in graphs from the same domain share some commonalities.
            \begin{itemize}
                \item Instances of graphlet 4 are more dominant in the citation graphs than other graphs. 
                \item Graphlets with many edges (e.g., graphlets 8, 12, and 13) account for a larger fraction in email/message networks than in other networks. 
                \item The fraction of graphlet 1 increases over time only in the online Q/A graphs.
            \end{itemize}
        \end{observation}
        \vspace{-2mm}
        }%
    }
%\vspace{0.5mm}



\noindent However, the dynamics are not exactly the same within domains. For example, while graphlets 1, 2, and 4 are dominant compared to other graphlets in all citation graphs, the ratios among them vary greatly in different graphs.

We also notice a consistent difference between the dynamics in real-world graphs and those in randomized graphs (see Section~\ref{sec:prelim:concept}), as summarized in Observation~\ref{obs:graphlet_evolve:random}.

\vspace{0.5mm}
\noindent\fbox{%
        \parbox{0.98\columnwidth}{%
        \vspace{-2mm}
        \begin{observation} \label{obs:graphlet_evolve:random}
            The ratios of graphlet instances change more linearly in randomized graphs than in real-world graphs.
        \end{observation}
        \vspace{-2mm}
        }%
    }
\vspace{0.5mm}


\begin{table}[t]
\caption{\label{tab:linearity} 
The non-linearity of the ratios of graphlet instances over time in real-world graphs and randomized graphs. 
We describe in Section~\ref{section:graph:time} how the non-linearity is measured.
The lower the non-linearity is, the more linear the change of the ratio of the corresponding graphlet instances is.
Note that
the ratios of graphlet instances change more linearly in randomized graphs than in real-world graphs.
}
\resizebox{\columnwidth}{!}{
\begin{tabular}{|c|ccc|ccc|ccc|}
    \hline
    	  Dataset & HepPh & HepTh & Patent & EU & Enron & College & Math & Ask & Stack \\
    	  \hline
    	  real      &  0.0027   & 0.0080    & 0.0093 & 0.0107 & 0.0042 & 0.0095 & 0.0028 & 0.0038 & 0.0047 \\
    	  random    &  0.0003   & 0.0011    & 0.0000 & 0.0081 & 0.0017 & 0.0058 & 0.0007 & 0.0005 & 0.0001 \\
    \hline
\end{tabular}}
\end{table}
\noindent In order to numerically support this observation, we measure the non-linearity \cite{kroll1993theoretical, hsieh2008statistical} of the ratios of graphlet instances over time.
Specifically, we fit a linear regression model and a non-linear polynomial regression model to each time series in Table~\ref{tab:graphlet_evolution}, and then we measure the average absolute difference between the predicted values of the two models as the non-linearity of the time series.\footnote{We use the linearity test implemented in Analyse-it (Ver. 5.65) and select a cubic model as the non-linearity polynomial model, as suggested in the program. For computational efficiency, we measure the absolute difference at 1,000 evolution ratios sampled uniformly at equal intervals.}
Lastly, we average the non-linearity of all time-series from each graph and report the results in Table~\ref{tab:linearity}.
Note that non-linearity is significantly higher in real-world graphs than in corresponding randomized graphs. 
That is, the ratios of graphlet instances change more linearly in randomized graphs than in real-world graphs.
%We fit a linear regression model and a cubic polynomial regression model to the ratio of each graphlet and the non-linearity is measured as the difference between the predicted value of the two models. 

\begin{table*}[ht]
\vspace{-2mm}
\caption{\label{tab:gtg} Using graphlet transition graphs (GTGs) and characteristic profiles (CPs) from GTGs, we can accurately characterize the dynamics of local structures in real-world graphs.
The colors of edges in GTGs indicate their normalized weights.
Note that GTGs and CPs are particularly similar in real-world graphs from the same domains (Observation~\ref{obs:transition:domain}).}
\resizebox{\textwidth}{!}{
\begin{tabular}{c|ccc|c}
    \toprule
    	      & \multicolumn{3}{c|}{Graphlet transition graphs (GTGs)} & Characteristic profiles (CPs) \\
    \hline
    \parbox[t]{2mm}{\multirow{6}{*}{\rotatebox[origin=c]{90}{Citation}}} &  
        \raisebox{-.9\totalheight}{\includegraphics[width=0.2\textwidth]{FIG/transition_graph/hepph.pdf}} &
        \raisebox{-.9\totalheight}{\includegraphics[width=0.2\textwidth]{FIG/transition_graph/hepth.pdf}} & 
        \raisebox{-.9\totalheight}{\includegraphics[width=0.2\textwidth]{FIG/transition_graph/patent.pdf}} & 
        \raisebox{-.9\totalheight}{\includegraphics[width=0.3\textwidth]{FIG/transition_graph/significance_citation.pdf}} \\
    \hline
    \parbox[t]{2mm}{\multirow{6}{*}{\rotatebox[origin=c]{90}{Email/Message}}} &  
        \raisebox{-.9\totalheight}{\includegraphics[width=0.2\textwidth]{FIG/transition_graph/email-eu.pdf}} &
        \raisebox{-.9\totalheight}{\includegraphics[width=0.2\textwidth]{FIG/transition_graph/enron.pdf}} & 
        \raisebox{-.9\totalheight}{\includegraphics[width=0.2\textwidth]{FIG/transition_graph/college_msg.pdf}} & 
        \raisebox{-.9\totalheight}{\includegraphics[width=0.3\textwidth]{FIG/transition_graph/significance_email.pdf}} \\
    \hline
    \parbox[t]{2mm}{\multirow{6}{*}{\rotatebox[origin=c]{90}{Online Q/A}}} &  
        \raisebox{-.9\totalheight}{\includegraphics[width=0.2\textwidth]{FIG/transition_graph/mathoverflow.pdf}} &
        \raisebox{-.9\totalheight}{\includegraphics[width=0.2\textwidth]{FIG/transition_graph/askubuntu.pdf}} & 
        \raisebox{-.9\totalheight}{\includegraphics[width=0.2\textwidth]{FIG/transition_graph/stackoverflow.pdf}} & 
        \raisebox{-.9\totalheight}{\includegraphics[width=0.3\textwidth]{FIG/transition_graph/significance_qna.pdf}} \\
    \bottomrule
\end{tabular}}
\end{table*}



\subsection{Global Level 2. Graphlet Transitions}
\label{section:graph:transition}

In a temporal graph, an instance of a graphlet may transition to an instance of another graphlet due to new edges added to it.
In this subsection, we examine the counts of such transitions between graphlets to characterize the local dynamics in temporal graphs and also to make comparisons between them.

\smallsection{Graphlet Transition Graph:}
We define \textit{graphlet transition graphs} (GTGs) to encode transitions between graphlets.

\noindent\fbox{%
    \parbox{0.98\columnwidth}{%
    \vspace{-2mm}
    \begin{definition}[Graphlet transition graph] \label{defn:transition}
       A \textit{graphlet transition graph} (GTG) $G=(V,E,W)$ of a temporal graph $\SG$ is a static directed weighted graph where the nodes are graphlets and each edge indicates that the source graphlet is transformed into the destination graphlet by an edge added to $\SG$.
        The weight of edges is the number of occurrences of the corresponding transitions. 
        We use $W=\{w_1, \cdots, w_{|E|}\}$ to denote the edge weights.
    \end{definition}
    \vspace{-2mm}
    }%
 }
\vspace{0.5mm}
\\
\noindent Since we focus on the 13 graphlets in Figure~\ref{fig:graphlet_and_role}(a), a GTG consists of the 28 types of transitions between these graphlets.
In Table~\ref{tab:gtg}, we visualize the GTGs from the real-world graphs.
Algorithm~\ref{alg:count_graphlet} describes the computation of the edge weights of 
 a GTG. In a nutshell, for each edge in arrival order, we count the transitions caused by it.
%The Algorithm~\ref{alg:count_graphlet} describes how we compute the counts. 
Its time complexity is formalized in Theorem~\ref{thm:time:transition}.

\begin{theorem} \label{thm:time:transition} 
The time complexity of Algorithm~\ref{alg:count_graphlet} is $\Theta(\Sigma_{v\in\SV}(d(v))^2)$ = $\Theta($the number of instances of all graphlets in the last snapshot$)$.
%Given a temporal graph $\SG=(\SV, \SE, \ST)$, Algorithm~\ref{alg:track_graphlet} takes $\Theta(\Sigma_{v\in\SV}(d(v))^2)$ times for counting the number of instances of all graphlets in $\SG$.
\end{theorem}
\begin{proof}
We can prove the complexity of $\Theta(\Sigma_{v\in\SV}(d(v))^2)$ similarly to Theorem~\ref{thm:time:track}, and by Lemma~\ref{lem:time:optimality}, it is $\Theta($the number of instances of all graphlets in the last snapshot$)$.
\end{proof}


\begin{algorithm}[t]
\small
	\caption{Computing the Edge Weights of Graphlet Transition Graphs \label{alg:count_graphlet}} 
	\SetKwInOut{Input}{Input}
	\SetKwInOut{Output}{Output}
	\SetKwProg{Procedure}{Procedure}{}{}
	\SetKwFunction{update}{UPDATE}
	
	\Input{Temporal graph $\SG = (\SV, \SE, \ST)$}
	\Output{Edge weights of the graphlet transition graph of $\SG$}
	
	%$\ST\leftarrow \varnothing $ \\
	%$\SD\leftarrow \varnothing $ \\
	
	Initialize all edge weights to zero \\
	Initialize $\SE$ to an empty set \\
%	Let $\SW$ as an array with size of 28. \\
	\For{\upshape\textbf{each }edge $e_i=u\rightarrow v$ in arrival order}{
	   % let directed edge $e_i$  
        \For{\upshape\textbf{each} $w_1$ $\in$ neighbors$(u)$ {$\setminus$ $\{v\}$} }{
            \update{$u, v, w_1$}
        }	
        \For{\upshape\textbf{each} {$w_2$ $\in$ neighbors$(v)$ $\setminus\{$neighbors$(u)$ $\cup$ $u\}$}}{
            \update{$u, v, w_2$}
        }
        add $u\rightarrow v$ to $\SE$
	}
	
	\textbf{return} the edge weights
	
    \Procedure{\update{$u, v, w$}}{ 
        \If{$u$, $v$, and $w$ form a graphlet instance}{
            prev $\leftarrow$ graphlet of the instance $(u, v, w)$ without $u\rightarrow v$ \\
            next $\leftarrow$ graphlet of the instance $(u, v, w)$ with $u\rightarrow v$ \\
            $i$ $\leftarrow$  index of the graphlet transition from prev to next \\
            increase the weight of the edge $i$ (i.e., $w_{i}$) by $1$
%            $\SD$[prev] $\leftarrow \SD$[prev] - 1 \\
        }
    }
\end{algorithm}

\smallsection{Characteristic Profile (CP):} 
We characterize the evolution of local structure in a graph $\SG$ using the significance of edge weights in its GTG $G=(V,E,W)$. 
In order to measure the significance, we follow the steps in \cite{milo2004superfamilies} for measuring the significance of each graphlet itself.
To this end, we construct the graphlet transition graph $\tilde{G}$ of a randomized graph $\SGR$.
Then, we measure the significance $SP_i$ of each edge weight $w_i$ in $G$ as follows:
\begin{equation}\label{eq1}
    SP_i := \frac{w_i - \tilde{w}_i}{w_i+\tilde{w}_i + \epsilon},
\end{equation}
where $\tilde{w}_i$ is the corresponding edge weight in $\tilde{G}$, and $\epsilon$ is a constant, which we fix to $4$. For $\tilde{w}_i$, we generate 50 instances of randomized graphs and we use the average edge weights in them. Lastly, we normalize each significance as follows:
\begin{equation}\label{eq2}
    CP_{i} := {SP_{i}}/{\sqrt{\Sigma_{i=1}^{|E|} SP_{i}^2}}.
\end{equation} 
We characterize the evolution of local structures in $\SG$ using the vector of the normalized significances (i.e., $[CP_1,\cdots,CP_{|E|}]$), which we call  \textit{characteristic profile (CP)}.
%We characterize the evolution of local structures in a graph $\SG$ using the vector of the normalized significances (i.e., $[CP_1,\cdots,CP_{28}]$), which we call the \textit{characteristic profile (CP)}.

%\begin{equation}\label{eq2}
%    CP_{i} := {SP_{i}}/{\sqrt{\Sigma_{i=1}^{28} SP_{i}^2}}.
%\end{equation} 

%We characterize the evolution of local structures in a graph $\SG$ using the vector of the normalized significances (i.e., $[CP_1,\cdots,CP_{28}]$), which we call the \textit{characteristic profile (CP)}.

\begin{figure}[t]
    \vspace{-3mm}
     \centering
     \begin{subfigure}{0.155\textwidth}
         \includegraphics[width=\textwidth]{FIG/signal/node_signal_2.pdf}
         \caption{$d_\theta = 2$}
     \end{subfigure}
     \begin{subfigure}{0.155\textwidth}
        \includegraphics[width=\textwidth]{FIG/signal/node_signal_4.pdf}
        \caption{$d_\theta = 4$}
     \end{subfigure}
     \begin{subfigure}{0.155\textwidth}
        \includegraphics[width=\textwidth]{FIG/signal/node_signal_8.pdf}
        \caption{$d_\theta = 8$}
     \end{subfigure}
     \caption{\label{fig:trend_signal} 
     \label{fig:signal_degree}
     Example signals from the local structures of nodes regarding their future importance. The ratios of some node roles (e.g., node roles 2 and 4) at nodes monotonically increase with respect to the future in-degrees of the nodes. The ratios are rescaled so that their maximum values are the same.}
  %   The example for local structural signals of nodes about their degree centrality. The higher the in-degree of predicted node $d_\theta$, the higher the number of signals.\red{TODO}}
\end{figure}

\smallsection{Comparison between CPs:}
We plot the CPs of the considered real-world graphs in Table~\ref{tab:gtg}, and high levels of similarity are observed within domains.
We numerically measure the similarity between CPs using the Pearson correlation coefficients, and the results are shown in Figure~\ref{fig:domain_correlation}(a).
The correlation coefficients are particularly high between graphs from the same domain, and specifically the domains can be classified with $97.2\%$ accuracy if we use the best threshold of the correlation coefficient ($0.58$).
The results demonstrate that CPs accurately characterize the evolution of local structures.
Our observations are summarized in Observation~\ref{obs:transition:domain}.

% \begin{algorithm}[t]
% \small
% 	\caption{Computing the Edge Weights of Graphlet Transition Graphs \label{alg:count_graphlet}} 
% 	\SetKwInOut{Input}{Input}
% 	\SetKwInOut{Output}{Output}
% 	\SetKwProg{Procedure}{Procedure}{}{}
% 	\SetKwFunction{update}{UPDATE}
	
% 	\Input{Temporal graph $\SG = (\SV, \SE, \ST)$}
% 	\Output{Edge weights of the graphlet transition graph of $\SG$}
	
% 	%$\ST\leftarrow \varnothing $ \\
% 	%$\SD\leftarrow \varnothing $ \\
	
% 	Initialize all edge weights to zero. \\
% 	Initialize $\SE$ to an empty set. \\
% %	Let $\SW$ as an array with size of 28. \\
% 	\For{\upshape\textbf{each }edge $e_i=u\rightarrow v$ in arrival order}{
% 	   % let directed edge $e_i$  
%         \For{\upshape\textbf{each } $w_1$ $\in$ neighbors$(u)$ {$\setminus$ $\{v\}$} }{
%             \update{$u, v, w_1$}
%         }	
%         \For{\upshape\textbf{each } {$w_2$ $\in$ neighbors$(v)$ $\setminus\{$neighbors$(u)$ $\cup$ $u\}$}}{
%             \update{$u, v, w_2$}
%         }
%         add $u\rightarrow v$ to $\SE$
% 	}
	
% 	\textbf{return} the edge weights
	
%     \Procedure{\update{$u, v, w$}}{ 
%         \If{$u$, $v$, and $w$ form a graphlet instance}{
%             prev $\leftarrow$ graphlet of the instance $(u, v, w)$ without $u\rightarrow v$ \\
%             next $\leftarrow$ graphlet of the instance $(u, v, w)$ with $u\rightarrow v$ \\
%             $i$ $\leftarrow$  index of the graphlet transition from prev to next \\
%             increase the weight of the edge $i$ (i.e., $w_{i}$) by $1$
% %            $\SD$[prev] $\leftarrow \SD$[prev] - 1 \\
%         }
%     }
% \end{algorithm}


\vspace{0.5mm}
\noindent\fbox{%
        \parbox{0.98\columnwidth}{%
        \vspace{-2mm}
        \begin{observation} \label{obs:transition:domain}
            The evolution patterns of local structures are similar
            in real-world graphs from the same domains.
        \end{observation}
        \vspace{-2mm}
        }%
    }

%with the best thresholds of similarity, the classification accuracy is $97.2\%$

%We follow \cite{milo2004superfamilies} for the above steps 

%We call the signifiance of each edge weight (i.e., $[\SP_1,\cdots, \SP_2]$) a 
%Table~\ref{tab:transition_significance} shows the significances from the considered real-world graphs, and their significance profile. The temporal graphs within the same domain share similar significance profile vectors. 

%$\SW_{rand}[i]$ is the average weight of same edge in the graphlet transition graphs obtained from $\SGR$ and we fix $\epsilon$ to 4. Table~\ref{tab:transition_significance} shows graphlet transition graphs of $\SG$ and their significance profile. The temporal graphs within the same domain share similar significance profile vectors. 

% \smallsection{:} 

% We 
% To compare the similarity of temporal graphs, we normalize the significance of all edges and concatenate them. Since the number of edges in $G$ is twenty-eight, we obtain characteristic profile (CP) as follow:





\smallsection{Comparison with Other Methods:} We evaluate three other graph characterization methods, as we evaluate ours in the right above paragraph.
In Figure~\ref{fig:domain_correlation}(b), we provide the correlation coefficients between the CPs obtained from the count of the instances of each graphlet~\cite{milo2004superfamilies}. 
%We calculate the classification accuracy as described in the above paragraph.
Note that the email/message graphs (blue) and the online Q/A graphs (green) are not distinguished clearly. Numerically, with the best threshold of correlation coefficient ($0.95$), the classification accuracy is $83.3\%$.

We also compute the similarity between the considered real-world graphs using Graphlet-orbit Transition (GoT) \cite{aparicio2018graphlet} and Orbit Temporal Agreement (OTA) \cite{aparicio2018graphlet}, which are also based on transitions between graphlets (see Section~\ref{section:relwork} for details). 
Our way of characterization has the following major advantages over them:
\begin{itemize}
    \item \textbf{(1) Speed:} Empirically, GoT and OTA are up to $10\times$ slower than our method, as shown in Appendix~\ref{sec:appendix:compare_got_ota}.
The time complexity of them is proportional to the sum of the counts of graphlet instances in all used snapshots, while the time complexity of Algorithm~\ref{alg:count_graphlet} is proportional only the to the count of graphlet instances in the last snapshot (Theorem~\ref{thm:time:transition}).
    \item \textbf{(2) Space Efficiency:} GoT and OTA run out of memory in the two largest graphs (Patent and Stackoverflow), as shown in Appendix~\ref{sec:appendix:compare_got_ota}, while our method does not. They need to store all graphlet instances in each considered snapshot for comparison with those in the next snapshot, while Algorithm~\ref{alg:count_graphlet} maintains only the latest snapshot without having to store graphlet instances.
    \item \textbf{(3) Characterization Accuracy:} The best classification accuracies computed using the considered real-world graphs (except for Patent and Stackoverflow for which GoT and OTA run out of memory) are $81.0\%$ (GoT) and $85.7\%$ (OTA), which is lower than our classification accuracy ($97.2\%$). Detailed results are given in Appendix~\ref{sec:appendix:compare_got_ota}. Note that GoT and OTA approximate the counts of transitions between graphlets based on a small number of snapshots, while Algorithm~\ref{alg:count_graphlet} exactly counts the transitions.
\end{itemize}
% \red{
%By Lemma~\ref{lem:time:optimality}, counting the instances of graphlets in a snapshot $\SGT$ takes at least $\Omega(\Sigma_{v\in\SVT}(\DTV)^2)$ times. To count the graphlet instances in each snapshot, they have to repeat the counting process with the number of snapshots used in the algorithm. 
% \textbf{(2) Lossless graphlet transition information:} For the above reason, GoT and OTA consider only the transitions between a few snapshots. However, in real-world temporal graphs, the future edges are more likely to be attached at the recently arrived edges rather than older edges, which is called a temporal locality~\cite{lee2020temporal}. Therefore, one-by-one transition information can be lost in GoT and OTA, which is successfully preserved in the proposed method. 

In summary, \textbf{our way of characterizing temporal graphs using GTGs distinguishes the domains of temporal graphs most accurately with the accuracy of $97.2\%$.} The accuracies of the other methods are $83.3\%$, $81.0\%$, and $85.7\%$.

