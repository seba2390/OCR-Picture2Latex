\section{Basic Concepts, Notations, and Data}\label{section:prelim}

In this section, we first introduce some basic concepts and notations. Then, we describe the nine datasets used in this paper.







\subsection{Basic Concepts and Notations}
\label{sec:prelim:concept}

\smallsection{Temporal Graph:} 
%A \textit{graph} $G = (V, E)$ consists of a set of nodes $V = (v_1, v_2, ..., v_{|V|})$ and a set of directed edges $E = (\tilde{e}_1, \tilde{e}_2, ..., \tilde{e}_{|E|})$.
A \textit{temporal graph} $\SG = (\SV, \SE, \ST)$ consists of a set of nodes $\SV$, a set of directed edges $\SE:=\{e_1,\cdots,e_{|\SE|}\}$, and a multiset of edge arrival times $\ST:=[t_1,\cdots, t_{|\SE|}]$.
For each directed edge $e_i\in \SE$,  we use $t_i \in \ST$ to denote the arrival time of $e_i$.
We use $u \rightarrow v$ to denote a directed edge from a node $u$ to a node $v$, and the nodes $u$ and $v$ are \textit{adjacent} if $u \rightarrow v$ or $v \rightarrow u$ exists.
From now on, we will use the term \textit{edge} to indicate a directed edge when there is no ambiguity.

\begin{table}[t]
%    \vspace{-1mm}
	\centering
	\caption{\label{tab:notations} Table of symbols.}
% \vspace{-2mm}
    %\scalebox{0.95}{
    \resizebox{\columnwidth}{!}{
	\begin{tabular}{ r | l }
		\toprule
		{\bf Notation} & {\bf Definition} \\
		\midrule
	    $\SG = (\SV, \SE, \ST)$ & temporal graph with nodes $\SV$, edges $\SE$, and times $\ST$  \\
	    $\SGT=(\SVT,\SET)$ & snapshot of $\SG$ at time $t$ \\
	    \midrule
	    $\SGR= (\SV, \SE, \STR)$ & a temporal graph randomized from $\SG$ \\
	    $\SGRT=(\SVRT,\SERT)$ & snapshot of $\SGR$ at time $t$ \\
	    \midrule
%	    $\SGTk$ & count of graphlet $k$ instances in $\SGT$ \\
	    $\SNTi(v)$ & count of node role $i$ at a node $v$ in $\SGT$ \\
	    %$\SETj(e)$ & count of edge role $j$ at an edge $e$ in $\SGT$ \\
	    %$G_t = (V_t, E_t)$ & snapshot of the temporal graph $T$ at time $t$ \\
	   % \midrule
	   % $\SM$, $\SN$, $\SI$ & network motifs, node roles, and edge roles of size three.\\
	   %% $\SM_i$, $\SN_i$, $\SI_i$ & motif $i$, node role $i$, and edge role $i$ \\
	   % $\SM_i$, $\SN_i$, $\SI_i$ & $i$-th motif, node role, and edge role \\
	   % $\SM\{i_1, i_2, ...,  i_k\}$ & set of motifs $\SM_{i_1}, \SM_{i_2}, ..., \SM_{i_k}$  \\
	    
	   % \midrule
	   % $\SD$ & evolution pattern of the distributions of network motifs over time\\
	   % $\SN_t[v]$ & number of instances $\SN$ in node $v$ at the snapshot $G_t$\\
	   % $\SI_t[e]$ & number of instances $\SI$ in edge $e$ at the snapshot $G_t$\\
		\bottomrule
	\end{tabular}}
\end{table}

%We assume that edges are ordered chronologically, i.e., if $i<j$, then $t_i \leq t_j$. 

\smallsection{Randomized Graph:} 
A \textit{randomized graph} $\SGR= (\SV, \SE, \STR)$ of $\SG=(\SV,\SE, \ST)$ is obtained by assigning arrival times in $\ST$ to edges in $\SE$ uniformly at random in a one-to-one manner. For each edge $e_i\in \SE$, we use $\tilde{t}_i\in \STR$ to denote the arrival time assigned to it. % by randomization.

\smallsection{Snapshot:}
%For each edge $e_i\in \SE$, we use $src(e_i)\in \SV$ and $dst(e_i)\in \SV$ to denote its source node and destination node, respectively.
We define the \textit{snapshot at time $t$} of $\SG=(\SV, \SE, \ST)$ as $\SGT=(\SVT,\SET)$ where $\SET:=\{e_i\in \SE:t_i \leq t\}$ and $\SVT\subseteq \SV$ is the endpoints of any edge in $\SET$.
That is, $\SGT$ consists of the nodes and edges arriving at time $t$ or earlier. Similarly, the snapshot at time $t$ of $\SGR= (\SV, \SE, \STR)$ is 
$\SGRT=(\SVRT,\SERT)$ where $\SERT:=$ $\{e_i\in \SE:\tilde{t}_i \leq t\}$ and $\SVRT$ is the endpoints of any edge in $\SERT$. 
We define the \textit{neighbors} of a node $v\in\SVT$ in a snapshot $\SGT$ as the nodes adjacent to $v$ in $\SGT$.  %i.e. $\{u\in\SVT|\exists u\rightarrow v \text{ or } \exists v\rightarrow u\}$. 
We define the \textit{degree} of a node $v\in\SVT$ in a snapshot $\SGT$, which is denoted by $\DTV$, as the number of directed edges whose endpoints include $v$ in $\SGT$.
We simply use $d(v)$ to denote the degree of the node $v$ in the last snapshot $\SG^{(t_{|\SE|})}$.

\smallsection{Induced Subgraphs:} 
%Since the term network motifs are used to represent different types of subgraphs in many studies, we first define it. 
A subgraph of a snapshot $\SGT=(\SVT,\SET)$ is \textit{induced} if and only if it consists of a subset of $\SVT$ and all of the edges connecting pairs of the nodes in the subset. 
Two subgraphs $\SH$ and $\SHP$ are \textit{isomorphic} if there exists a one-to-one mapping $f$ between the nodes of both graphs such that there exists an edge from a node $u$ to a node $v$ in $\SH$ if and only if there exists an edge from the node $f(u)$ to the node $f(v)$ in $\SHP$.

\smallsection{Graphlets:}
A \textit{graphlet} is the set of induced subgraphs that are isomorphic to each other.
%\textit{Graphlets} are sets of isomorphic subgraphs with a predefined number of nodes.
In this paper, we limit our attention to the $13$ graphlets consisting of three connected nodes. %as shown in Figure~\ref{fig:graphlet_and_role}(a). %, due to computational cost of tracking the counts of disconnected ones.
An induced subgraph is called an \textit{instance} of graphlet $k$ if it is isomorphic to the $k$-th graph in Figure~\ref{fig:graphlet_and_role}(a).
%We denote by $\SGTk$ the count of graphlet $k$ instances in a snapshot $\SGT$.

%We denote them in a snapshot $\SGT$ by $\SSTone$, $\SSTtwo$, $\cdots$, $\SSTthirteen$, and each $\SSTi$ corresponds to the set of induced subgraphs of $\SGT$ that are isomorphic to the $i$-th graph in Figure~\ref{fig:graphlet_and_role}(a).
%Similarly, we use $\SSRTi$ to denote each $i$-th graphlet in $\SGRT$.

%\red{KJ: I am here}

\smallsection{Node Roles:} 
Consider an induced subgraph $\SH$ with a node set $\SV'$. 
An \textit{automorphism} of $\SH$ is an isomorphism between $\SH$ and itself. i.e., an automorphism of $\SH$ is a one-to-one mapping between nodes of $\SH$ such that there exists an edge from a node $u$ to a node $v$ in $\SH$ if and only if there exists an edge from the node corresponding to $u$ to the node corresponding to $v$ in $\SH$.  
%The set of automorphisms of $\SH$ forms a \textit{group} under the operation of composition, denoted by $Aut(\SH)$. 
If denoting the set of automorphisms of $\SH$ by $Aut(\SH)$,
the \textit{automorphism orbit} of a node $u\in \SV'$ is the set $\{y \in \SV' : \exists g \in Aut(\SH) \text{ s.t. } y = g(u)\}$ of nodes \cite{prvzulj2007biological}.
Formally, \textit{node roles} are node automorphism orbits, and roughly, they are sets of symmetric positions of nodes within graphlets.
Figure~\ref{fig:graphlet_and_role}(b) (see the positions of black nodes) shows all $30$ node roles in the $13$ graphlets that we consider.
We say a node $v$ ``takes'' node role $i$ in a graphlet instance if there exists an isomorphism of the graphlet instance and the $i$-th graph in Figure~\ref{fig:graphlet_and_role}(b) that maps $v$ to the black node in the graph.
%We say a node $v$ ``takes'' node role $i$ in a graphlet instance if there exists an isomorphism of the graphlet and the $i$-th graph in Figure~\ref{fig:graphlet_and_role}(b) that maps $v$ to the black node in the graph.
We define the \textit{count of node role $i$ at a node $v$} as the number of graphlet instances where $v$ takes $i$, and $\SNTi(v)$ denotes the count at a snapshot $\SGT$.

\begin{table}[t]
    %    \vspace{-2mm}
	\centering
	\caption{\label{tab:data} Summary of nine real-world temporal graphs used throughout this paper.} 
    %    \vspace{-2mm}
	%\scalebox{0.95}{
     \resizebox{\columnwidth}{!}{
		\begin{tabular}{c | c | c | c | c }
			\toprule
			{\bf Domain }& {\bf Dataset} & {$|V|$} & {\bf$|E_T|$} & {\bf Period}\\
			\midrule
			\multirow{3}{*}{Citation}
			& \ul{\bf{\smash{HepPh}}}    & $34,565$      & $346,849$     & 9 years \\ 
			& \ul{\bf{\smash{HepTh}}}    & $18,477$      & $136,190$     & 10 years \\
			& \ul{\bf{Patent}}           & $3,774,362$   & $16,512,782$  & 25 years \\
			\midrule
			\multirow{3}{*}{Email/Message}
			& \ul{\bf{Enron}}               & $55,655$      & $209,203$     & 24 years \\
		    & \ul{\bf{EU}}                  & $986$         & $24,929$      & 1.5 years \\
		    & \ul{\bf{\smash{College}}}   & $1,899$       & $20,296$      & 0.5 years \\
		    \midrule
		    \multirow{3}{*}{Online Q/A}
		    & \ul{\bf{Ask}}ubuntu             & $159,316$     & $262,106$     & 6 years \\
		    & \ul{\bf{Math}}overflow          & $24,818$      & $90,489$      & 7 years \\
		    & \ul{\bf{Stack}}overflow         & $2,601,977$   & $16,266,395$  & 8 years \\
			\bottomrule
		\end{tabular}}
\end{table}

%We use $\SNT[v]$ & number of instances $\SN$ in node $v$ at the snapshot $G_t$\\

% \begin{algorithm}[ht]
% \small
% 	\caption{Counting the Instances of Each Graphlet in a Temporal Graph \label{alg:track_graphlet}} 
% 	\SetKwInOut{Input}{Input}
% 	\SetKwInOut{Output}{Output}
% 	\SetKwProg{Procedure}{Procedure}{}{}
% 	\SetKwFunction{update}{UPDATE}
	
% 	\Input{Temporal Graph $\SG=(\SV, \SE, \ST)$}
% 	\Output{The count of the instances of each graphlet in $\SG$.}
% 	\vspace{3pt}
	
% 	%\texttt{\color{blue}/* Initialization */}\\
% 	Initialize the count of the instances of each graphlet to zero. \\
% 	Initialize $\SE$ to an empty set.\\
%     \vspace{3pt}
    
% 	\For{\upshape\textbf{each edge }$e_i=u \rightarrow v$ in arrival order}{
	    
% 	    $\SN \leftarrow $ union of the neighbors of $u$ and the neighbors of $v$ (except for $u$ and $v$) \label{alg:track_graphlet:line:neighbors}
	    
% 	    \For{\upshape\textbf{each} $w\in \SN$}{
% 	       % \update{$u$, $v$, $w$} \\
% 	        \If{$u$, $v$ \textbf{and} $w$ form a graphlet instance}{ \label{alg:track_graphlet:line:if_instance}
% 	         decrement the count of the graphlet of the instance formed by $u$, $v$ and $w$ \label{alg:track_graphlet:line:decrease}
% 	         \\
            
%             %  \If{ \upshape{is\_inverse} \upshape\textbf{is} True}{
%             %      decrease the count of the instance of graphlet prev by $1$ \\
%             %      increase the count of the instance of graphlet next by $1$ \\
%             %  }
%             }
%          }
         
%          add $u\rightarrow v$ to $\SE$
         
%          \For{\upshape\textbf{each} $w\in \SN$}{
%               increment the count of the graphlet of the instance formed by $u$, $v$ and $w$ \label{alg:track_graphlet:line:increase}
% 	         \\
%          }
% 	    \vspace{3pt}
	    
% 	}
	
% 	\textbf{return} count of the instances of each graphlet instances.\\
% \end{algorithm}

\smallsection{Edge Roles:} 
Consider an induced subgraph $\SH$ with an edge set $\SE'$.
Based on the concepts defined above, we define the \textit{edge role} of an edge $u\rightarrow v$ is the set $\{x\rightarrow y \in \SE' : \exists g \in Aut(\SH) \text{ s.t. } x = g(u) \wedge y = g(v)\}$ of edges.
Roughly, edge roles are the sets of symmetric positions of edges within graphlets.
Figure~\ref{fig:graphlet_and_role}(c) (see the positions of edges from a red node to a blue node) shows all $30$ edge roles in the $13$ considered graphlets.
We say an edge $u\rightarrow v$ ``takes'' edge role $j$ in a graphlet instance if there exists an isomorphism of the graphlet instance and the $j$-th graph in Figure~\ref{fig:graphlet_and_role}(c) that maps $u$ and $v$ to the red node and the blue node, respectively, in the graph.
We define the \textit{count of edge role $j$ at an edge $e$} as the number of graphlet instances where $e$ takes $j$.
%We use $\SETj(e)$ to denote the count in a snapshot $\SGT$.

\subsection{Datasets}\label{section:datasets}
Throughout this paper, we use the nine real-world temporal graphs from the three domains, which are summarized in Table~\ref{tab:data}. 

\smallsection{Citation Graphs:} 
Each node is a paper or a patent. Each directed edge from a node $u$ to a node $v$ means that $u$ cites $v$. 

\smallsection{Email/Message Graphs:}
Each node is a user. Each directed edge from a node $u$ to a node $v$ indicates that $u$ sends $v$ emails (messages). 

\smallsection{Online Q/A Graphs:}
Each node is a user. Each directed edge from a node $u$ to a node $v$ means that $u$ answers $v$'s questions.