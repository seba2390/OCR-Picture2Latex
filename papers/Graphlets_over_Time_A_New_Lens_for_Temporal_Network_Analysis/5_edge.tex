\begin{table}[t]
 	\vspace{-2mm}
	\centering
	\caption{\label{tab:pred-node-overall-degree} 
    Average F1-score, accuracy, and AUROC on the task of predicting future node importance depending on $d_\theta$ (i.e., in-degree of nodes when their input features are extracted). 
	The overall performance improves with respect to $d_\theta$ in most cases. That is,
	as nodes have more neighbors, their future importance can be predicted more accurately.
 Detailed results on each dataset can be found in Appendix~\ref{sec:appendix:node:details}.
	}
	\resizebox{\columnwidth}{!}{
	%\scalebox{0.95}{
		\begin{tabular}{|c|c|c|c|c|c|c|}
		\hline
			Target & \multicolumn{3}{c|}{Degree} & \multicolumn{3}{c|} {Betweenness} \\ 
			\hline
			Measure & F1-score & Accuracy & AUROC & F1-score & Accuracy & AUROC \\
			\hline
			ALL $(d_\theta=2)$ & 0.59 & 0.74 & 0.78 & 0.65 & 0.85 & 0.86 \\
			ALL $(d_\theta=4)$ & 0.69 & 0.78 & 0.79 & 0.73 & 0.83 & 0.87 \\
			ALL $(d_\theta=8)$ & 0.80 & 0.81 & 0.86 & 0.82 & 0.85 & 0.90 \\
			\hline
			\hline
			Target & \multicolumn{3}{c|}{Closeness} & \multicolumn{3}{c|} {PageRank} \\ 
			\hline
			Measure & F1-score & Accuracy & AUROC & F1-score & Accuracy & AUROC \\
			\hline
			ALL $(d_\theta=2)$ & 0.69 & 0.83 & 0.88 & 0.55 & 0.75 & 0.79  \\
			ALL $(d_\theta=4)$ & 0.78 & 0.83 & 0.89 & 0.73 & 0.77 & 0.80 \\
			ALL $(d_\theta=8)$ & 0.86 & 0.88 & 0.92 & 0.85 & 0.85 & 0.83 \\
			\hline
		\end{tabular}
	}
\end{table}

\begin{figure}[t]
    \vspace{-3mm}
     \centering
     \includegraphics[width=0.45\textwidth]{FIG/signal/edge-betweenness.pdf} \\
     \vspace{-2mm}
     \caption{\label{fig:edge_signal}
     The Spearman's rank correlation coefficient between edge role ratios 
     (when endpoints have in-degree $4$ in total, i.e., $d_\theta$ = 4)
     and future edge centralities.
     The darker a cell is, the larger the absolute value of the corresponding coefficient is. Note that the absolute values of many coefficients are significantly greater than $0$, while they tend to be smaller than those in Figure~\ref{fig:node_signals}.
     % Edge roles whose ratios at edges in their early stage (specifically, edges whose endpoints have in-degree $4$ in total) monotonically increase (colored red) or decrease (colored blue) with respect to future edge importance.
     % Such edge roles are rare, compared to node roles with similar properties (see Figure~\ref{fig:node_signals}).
     }
     
%     The signal of local structural patterns of edges. Red color represents monotonic increasing patterns and blue color represents monotonic decreasing patterns when $d_\theta$=4.}
\end{figure}

\section{Edge Level Analysis}\label{section:edge}

In this section, we investigate the signal of local structures of each edge regarding their future centrality, and based on the signal, we predict the future importance of edges.  

We generally follow the procedures in Section~\ref{section:node}, except for the following differences: (a) we examine the ratios of edge roles at each edge $u\rightarrow v$ when the sum of the in-degrees of $u$ and $v$ becomes $d_\theta$, (b) we use edge betweenness centrality~\cite{freeman1977set} as the importance measure, (c) we formulate the problem of predicting future edge importance as described in Problem~\ref{problem:edge}, (d)
we extract feature sets \textbf{Local-ER} and \textbf{Global-ER} using the (relative) counts of edge roles at edges as we extract \textbf{Local-NR} and \textbf{Global-NR}, and (e) we union \textbf{Global-ER} and \textbf{Global-Basic} for $\textbf{ALL}$.

\vspace{0.5mm}

\noindent\fbox{%
        \parbox{0.98\columnwidth}{%
        \vspace{-2mm}
        \begin{problem}[Edge Centrality Prediction] \label{problem:edge}
        \noindent\begin{itemize}
            \item \textbf{Given:} the snapshot $\SGTe$ of the input graph when the sum of the in-degrees of the endpoints of each edge first reaches $d_\theta$,
            \item \textbf{Predict:} whether the centrality of each edge belongs to the top $20\%$ in the last snapshot of $\SG$.
        \end{itemize}
        \end{problem}
        \vspace{-2mm}
        }%
    }

\vspace{0.5mm}

From Figure~\ref{fig:edge_signal}, Table~\ref{tab:signal_degree}, and Table~\ref{tab:pred-edge-overall}, we draw the following observations.
%\red{From the results in Figure~\ref{fig:edge_signal}, Table~\ref{tab:num_signals}, and Table~\ref{tab:pred-edge-overall}, we draw the following observations.}


%drawing the following observations.

\noindent\fbox{%
        \parbox{0.98\columnwidth}{%
        \vspace{-2mm}
        \begin{observation}\label{obs:edge_signal}
        In real-world graphs, the signals from the local structures of edges in their early stage regarding their future importance are weaker, compared to the signals that from the local structures of nodes (see Figure~\ref{fig:edge_signal}).
        \end{observation}
        
        \begin{observation}\label{obs:edge_stronger}
        However, the signals become stronger as the edges are better connected, leading to better prediction performance (see Tables~\ref{tab:signal_degree} and~\ref{tab:pred-edge-overall}).
        \end{observation}
        
        \begin{observation}\label{obs:edge_pred}
           The features from edge roles (\textbf{Local-ER} and \textbf{Global-ER}) are more informative than simple global statistics (\textbf{Global-Basic}) for future importance prediction (see Table~\ref{tab:pred-edge-overall}).
        \end{observation}
        \vspace{-2mm}
        }%
     }
% \smallsection{Signal of Initial Edge:} 
% We also investigate local structure signals of edges at their early stage. Although those signals are not significantly strong than those of node roles, we find below observations.





\begin{table}[t]
	\vspace{-3mm}
	\centering
	\caption{\label{tab:pred-edge-overall} 
    F1-score, accuracy, and AUROC on the task of predicting future edge importance averaged over the 7 considered real-world graphs.
	Using edge role-based features (\textbf{Local-ER} and \textbf{Global-ER}) yields better performance than using \textbf{Global-Basic} in most settings. 
	The overall performance improves with respect to $d_\theta$.
	That is, as edges are better connected, their future importance is predicted more accurately.
     Detailed results on each dataset can be found in Appendix~\ref{sec:appendix:edge:details}
	}
	%Second, Global-NR has a little higher performance than Global-NPP, and it also has higher than Global-basic. Lastly, ALL has the highest predictive performance which indicates the proposed feature is complementary to other features. \red{todo} }   
	\scalebox{0.90}{
		\begin{tabular}{|c|c|c|c|}
		\hline
			Target & \multicolumn{3}{c|}{Edge betweenness}  \\ \hline
            Measure & F1-Score & Accuracy   &  AUROC \\ \hline
            Local-ER $(d_\theta = 2)$       & 0.45 & 0.78 & 0.76 \\
            Global-ER $(d_\theta = 2)$      & 0.47 & 0.81 & 0.78 \\
            Global-Basic $(d_\theta = 2)$   & 0.42 & 0.79 & 0.73 \\
            ALL $(d_\theta = 2)$            & 0.50 & 0.80 & 0.75 \\
			\hline
			ALL $(d_\theta = 2)$            & 0.50 & 0.80 & 0.75 \\
			ALL $(d_\theta = 4)$            & 0.53 & 0.82 & 0.84 \\
			ALL $(d_\theta = 8)$            & 0.52 & 0.85 & 0.85 \\
			\hline
		\end{tabular}
	}
\end{table}
