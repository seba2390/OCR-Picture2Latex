\begin{figure}[t]
    %\vspace{-1mm}
    \begin{subfigure}{0.24\textwidth}
        \includegraphics[width=\textwidth]{FIG/transition_graph/motif_transition_graph.pdf}
        \caption{Ours}
    \end{subfigure} \\
    \begin{subfigure}{0.22\textwidth}
        \includegraphics[width=\textwidth]{FIG/transition_graph/graphlet_orbit_transition.pdf}
        \caption{GoT}
    \end{subfigure}
    \begin{subfigure}{0.22\textwidth}
        \includegraphics[width=\textwidth]{FIG/transition_graph/orbit_temporal_agreement.pdf}
        \caption{OTA}
    \end{subfigure}
    \caption{\label{fig:got_sim}Similarity matrices from ours, GoT, and OTA.
    The domains of graphs (distinguished by colors) are classified more accurately by ours than by GoT or OTA.}
%    graphlet-orbit transition (GoT), and orbit temporal agreement (OTA)}
\end{figure}

\appendix
\section{Comparison with GoT and OTA}
\label{sec:appendix:compare_got_ota}
% Include only the SI item label in the paragraph heading. Use the \nameref{label} command to cite SI items in the text.
We provide additional details regarding the comparison between our characterization method based on graphlet transition graphs (GTGs) and regarding the comparison with Graphlet-orbit Transition (GoT) and Orbit Temporal Agreement (OTA).

\smallsection{Detailed Setting:} Our experiments were conducted on a desktop with a 3.8 GHz AMD Ryzen 3900x CPU and 128GB memory. We implemented our characterization method based on GTGs in Java, and we used the official implementations for GoT and OTA provided by the authors, which were implemented in C++. In each dataset, we used $12$ snapshots with the same intervals for GoT and OTA. 

\smallsection{Output Similar Matrix:} Figure~\ref{fig:got_sim} shows the output similar matrices from our characterization method based on GTGs, GoT, and OTA.
GoT and OTA run out of memory in the two largest datasets (Patent and Stackoverflow). %GoT fail to distinguish the Enron graph (blue) is much similar with online Q/A graphs (green) and OTA 
Both GoT and OTA fail to distinguish email/message graphs (blue) and online Q/A graphs (green) clearly. Numerically, with the best thresholds of similarity, the classification accuracies are $81.0\%$ (GoT) and $85.7\%$ (OTA), while the accuracy is $97.2\%$ in ours.



%In summary, \textbf{our way of characterizing temporal graphs using GTGs distinguishes the domains of temporal graphs most accurately with accuracy $97.2\%$.} The accuracies of the other methods are $83.3\%$, $81.0\%$, and $85.7\%$.


% \begin{figure}[t]
%     \centering
%         \hspace{-2mm}
% 	\subfigure[Ours]{
% 	\includegraphics[width=0.153\textwidth]{FIG/appendix/motif_transition_graph.pdf}
% 	} 
%         \hspace{-2mm}
% 	\subfigure[GoT]{
% 		\includegraphics[width=0.148\textwidth]{FIG/appendix/graphlet_orbit_transition.pdf}
% 	}
%         \hspace{-2mm}
%  	\subfigure[OTA]{
% 		\includegraphics[width=0.148\textwidth]{FIG/appendix/orbit_temporal_agreement.pdf}
% 	} \\
%         \vspace{-1mm}
% 	\caption{\label{fig:got_sim} Similarity matrices from ours, GoT, and OTA.
%     High similarity between graphs from the same domain is most clear in the matrix from ours.}
% \end{figure}

\begin{figure}[t]
%    \vspace{-2mm}
    \includegraphics[width=0.9\columnwidth]{FIG/appendix/figure10.pdf}
    \vspace{-2mm}
    \caption{\label{fig:got_time} 
    Running times of ours, GoT and OTA. Ours is consistently and significantly faster than both competitors, which run out of memory in the two largest datasets.}
    %Running times of ours, GoT and OTA. Ours is consistently and significantly faster than GoT and OTA. GoT and OTA run out of memory in the two largest datasets.}
\end{figure}

\smallsection{Speed Comparison:} As seen in Figure~\ref{fig:got_time}, ours is faster than GoT and OTA in all the graphs. Specifically, ours is $6.68\times$ faster than the others on average. 

\section{Feature Important Analysis}
\label{sec:appendix:feature_importance}
We measure the importance of each feature in the set \textbf{ALL} (see Section~\ref{subsection:node_prediction} of the main paper) using the \textit{Gini importance}~\cite{loh2011classification}, and we report the top five important features in Table~\ref{tab:feature_importance}.
