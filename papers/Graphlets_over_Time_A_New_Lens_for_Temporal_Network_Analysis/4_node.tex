

\section{Node Level Analysis} \label{section:node}
In this section, we study how local structures around nodes are related to their future importance.
Then, we enhance the predictability of future node centrality using the relations.


\subsection{Patterns}
We characterize the local structures of nodes using node roles and examine their relation to the nodes' future centrality.
%We investigate that the local structural signals of nodes in their early stage regarding their future centrality.

\begin{table}[t]
\vspace{-3mm}
\caption{\label{tab:signal_degree}
The absolute value of the Spearman's rank correlation coefficients between node role ratios and future centralities (averaged over all node roles and all datasets for each centrality measure) and each value of the threshold $d_\theta$.
As the number of node neighbors increases (i.e., $d_\theta$ increases), the local-structural signals about future centralities become stronger (i.e., the absolute values increase).
%Specifically, as $d_\theta$ increases, the ratios of more node (or edge) roles monotonically increase (INC) or decrease (DEC) with respect to future centrality.
}
\resizebox{\columnwidth}{!}{
    \begin{tabular}{|c|c|c|c|c|c|}
        \hline
        $d_\theta$ & Degree & Betweenness & Closeness & PageRank & Edge Betweenness \\
        \hline
        2        & 0.640  & 0.697       & 0.682     & 0.663    & 0.546            \\
        4        & 0.721  & 0.723       & 0.712     & 0.704    & 0.558            \\
        8        & 0.816  & 0.793       & 0.759     & 0.701    & 0.599            \\
        \hline
    \end{tabular}}
\end{table}

\smallsection{Local Structures of Nodes:}
Given a temporal graph $\SG$, we characterize the local structure of each node $v$ in their early stage by measuring the ratio of each node role at $v$ in the snapshot at time $t$ when the in-degree of $v$ first reaches a threshold $d_\theta$.
That is, each node $v$ is represented as a $30$-dimensional vector whose $i$-th is $\SNTi(v)/(\sum_{j=1}^{30}\SNTj(v))$  (see Section~\ref{sec:prelim:concept} for $\SNTi(v)$).

%the ratio of node role $i$ at $v$ in the snapshot.

\smallsection{Future Importance of Nodes:}
Given a temporal graph $\SG$, as future importance of each node, we measure its in-degree, node betweenness centrality~\cite{freeman1977set}, closeness centrality~\cite{bavelas1950communication}, and PageRank~\cite{page1999PageRank} in the last snapshot of $\SG$.
Based on each centrality measure, we divide the nodes in $\SG$ into six groups (Group 1: top 50-100\%, Group 2: top 30-50\%, Group 3: top 10-30\%, Group 4: top 5-10\%, Group 5: top 1-5\%, and Group 6: top 0-1\%).


\begin{figure*}[t]
    \vspace{-3mm}
    \centering
		\includegraphics[width=0.4\textwidth]{FIG/signal/degree.pdf}
		\includegraphics[width=0.4\textwidth]{FIG/signal/between.pdf} \\
            \vspace{-2mm}
		\includegraphics[width=0.4\textwidth]{FIG/signal/closeness.pdf}
            \includegraphics[width=0.4\textwidth]{FIG/signal/PageRank.pdf} \\
            \vspace{-3mm}
	\caption{\label{fig:node_signals} 
     The Spearman's rank correlation coefficient  between node role ratios (when nodes have in-degree four, i.e., $d_\theta$ = 4)
     and future node centralities.
     The darker a cell is, the larger the absolute value of the corresponding coefficient is. Note that the absolute values of most coefficients are significantly greater than $0$.
     }
\end{figure*}

\smallsection{Finding Signals:}
For each group, we average the ratio vectors of the nodes in the group. 
Figure~\ref{fig:signal_degree} shows some averaged ratios when in-degree is used as the centrality measure. Note that the ratios of node roles 2 and 4 monotonically grow as future centrality increases, regardless of $d_\theta$ values. That is, the ratios of node roles 2 and 4 give a consistent signal regarding the nodes' future in-degree.

In Figure~\ref{fig:node_signals}, we report the Spearman's rank correlation coefficient~\cite{zwillinger1999crc} between each averaged ratio and the future centralities of nodes (specifically, the above group numbers between 1 and 6).
%The coefficient value is especially large when node betweenness is used as the centrality measure. 
We also report in Table~\ref{tab:signal_degree} the absolute value of the coefficients (averaged over all node roles and all datasets) for each centrality measure and each value of the threshold $d_\theta$. Note that the average values are significantly greater than $0$ and specifically around $0.7$; and they
increase as $d_\theta$ increases, as summarized in Observation~\ref{obs:node:signal}.
%For each group, we average the ratio vectors of the nodes in the group.
%Figure~\ref{fig:signal_degree} shows some averaged ratios when in-degree is used as the centrality measure.
%
%Notably, the number of such roles with consistent signals increase as $d_\theta$ increases, as also shown in Table~\ref{tab:num_signals}.
%Our findings are summarized in Observation~\ref{obs:node:signal}.


\noindent\fbox{%
        \parbox{0.98\columnwidth}{
        \vspace{-2mm}
        \begin{observation}\label{obs:node:signal}
        In real-world graphs,
           the local structures of nodes in their early stage provide a signal regarding their future importance. The signals become stronger as nodes have more neighbors.
        \end{observation}
        % \begin{observation}\label{obs:node:signal:degree}
        %     Signals from the local structurals of nodes in their early stage are especially strong regarding their future betweenness, compared to other centrality measures.
        % \end{observation}
        \vspace{-2mm}
        }
     }
\vspace{0.5mm}


% Figure~\ref{fig:node_signals} shows all node roles whose ratios monotonically increase or decrease with respect to future centrality.
% The number of such roles is especially large when in-degree is used as the centrality measure. That is, Observation~\ref{obs:node:signal:degree} is made.
% \vspace{0.5mm}
% \noindent\fbox{%
%         \parbox{0.98\columnwidth}{
%         \vspace{-2mm}
%         \begin{observation}\label{obs:node:signal:degree}
%             Signals from the local structurals of nodes in their early stage are especially strong regarding their future in-degree, compared to other centrality measures.
%         \end{observation}
%         \vspace{-2mm}
%         }
%      }
% \vspace{0.5mm}

% \begin{figure}[t]
%     \centering
% 	\subfigure[$d_\theta = 2$]{
% 		\includegraphics[width=0.14\textwidth]{FIG/signal/node_signal_2.pdf}
% 	} 
% 	\subfigure[$d_\theta = 4$]{
% 		\includegraphics[width=0.14\textwidth]{FIG/signal/node_signal_4.pdf}
% 	} 
%  	\subfigure[$d_\theta = 8$]{
% 		\includegraphics[width=0.14\textwidth]{FIG/signal/node_signal_8.pdf}
% 	} \\
%         \vspace{-1mm}
% 	\caption{\label{fig:signal_degree} Example signals from the local structures of nodes regarding their future importance. The ratios of some node roles (e.g., node roles 2 and 4) at nodes monotonically increase with respect to the future in-degrees of the nodes. The ratios are rescaled so that their maximum values are the same. }
% \end{figure}




%an example of local structural signals of nodes where each color indicates the type of node roles. 
%The average ratio of node roles shows monotonic ascending trends regard as nodes' future centrality. The example also shows that the number of the signals increases as $d_\theta$ increases.



%Each node belongs to certain graphlets and its local-structural role can be represented as ratio of node roles it take in.
%Even though the nodes are in early stage, we find that their future centrality is highly related with local structural role.
%For local structures, we use the roles of each node when its in-degree reaches a certain value $d_\theta$.
%For future importances, we use  centrality. We divide the nodes into 6 groups for each centrality so that they have similar centrality values. Then, we calculate the average distribution of node roles when nodes' in-degree becomes $d_\theta$. 



\subsection{Prediction} \label{subsection:node_prediction}

Based on the observations above, we predict the future centrality of nodes using the counts of their roles at them in their early stage. 

% In this subsection, we predict the nodes' future centrality using their roles. They enhance the accuracy of simpler local structural features called Node Prominence Profile (NPP) \cite{yang2014predicting}. Also, we show the higher $d_\theta$ of the predicted node is, the higher accuracy nodes have. Moreover, they are complementary to global features: the number of nodes and edges.

\smallsection{Problem Formulation:} 
We formulate the prediction problem as a classification problem, as described in Problem~\ref{problem:node}.

\vspace{0.5mm}

\noindent\fbox{%
        \parbox{0.98\columnwidth}{%
        \vspace{-2mm}
        \begin{problem}[Node Centrality Prediction] \label{problem:node}
        \noindent\begin{itemize}
            \item \textbf{Given:} the snapshot $\SGTv$ of the input graph when the in-degree of a node $v$ first reaches $d_\theta$,
            \item \textbf{Predict:} whether the centrality of the node $v$ belongs to the top $20\%$ in the last snapshot of $\SG$.
        \end{itemize}
        \end{problem}
        \vspace{-2mm}
        }%
    }
\vspace{0.5mm}

\noindent As the centrality measure, we use in-degree, betweenness centrality, closeness centrality, and PageRank.
As $d_\theta$, we use $2$, $4$, or $8$. 


\smallsection{Input Features:} 
For each node $v$, we consider the snapshot $\SGT$ of the input graph $\SG$ when the in-degree of $v$ first reaches $d_\theta$. That is, $t=\tv$ and $\SGT=\SGTv$.
Then, we extract the following sets of input features for $v$: 
\bit
    \item \textbf{Local-NR:} The count of each node role at $v$ in $\SGT$. That is, $[\SNTone(v),$ $\SNTtwo(v), \cdots, \SNTthirty(v)]$ (see Section~\ref{sec:prelim:concept} for $\SNTi(v)$).
    \item \textbf{Local-NPP \cite{yang2014predicting}:} In $\SGT$, we compute (1) the count of triangles at $v$, (2) the count of wedges centered at $v$, (3) the count of wedges ended at $v$.
    \item \textbf{Global-Basic:} Counts of nodes and edges in the snapshot.
    \item \textbf{Global-NR:} We compute the $30$-dimensional vector whose $i$-th entry $\SNTi(v)/(\sum_{j=1}^{30}\SNTj(v))$
    is the ratio of each node role at $v$ in $\SGT$.
    Then, we standardize (i.e., compute the $z$-score of) the role ratio vector using the mean and standard deviation from the role ratio vectors (in $\SGT$) of all nodes with degree $d_\theta$ in $\SGT$. The features in Local-NR are also included.
    \item \textbf{Global-NPP~\cite{yang2014predicting}:} In $\SGT$, we compute (1) the number of edges not incident to $v$ and (2) the number of non-adjacent node pairs where one is a neighbor of $v$ and the other is neither a neighbor of $v$ nor $v$ itself.
%    product of the degree of $v$ and the number of disconnected noes. (b) the product of 
    The features in Local-NPP are also included.
    \item \textbf{ALL:}  All of \textbf{Global-NR}, \textbf{Global-NPP}, and \textbf{Global-Basic}.
    %from the the ratio vector of $v$ and we divide the result by the standard deviation. 
    %The combined features of Local-NR and their global information. We add the relative ratio of the node roles in each node using the z-score method. Suppose that the in-degree of node $v$ is $d_\theta$ in $\SGT$. Let $\SVT_\theta$ be the set of nodes whose degree is also $d_\theta$ in $\SGT$ and let $\tilde{m}_i^{(t)}(v)$ be ratio of node role $i$ at a node $v$ in $\SGT$. We calculate the mean $\mu_i$ and standard deviation $\sigma_i$ of $\tilde{m}_i^{(t)}(v)$ for every node role $i$ of nodes in $\SVT_\theta$. Then, we add the global feature Global-NR $i$ for every node role $i$ calculated as below:
        % (ratio(instance) - mu) / (sigma + epsilon)
    % \begin{equation*}
    %     \text{Global-NR } i := \frac{\tilde{m}_i^{(t)}(v) - \mu_i}{1+\sigma_i}
    % \end{equation*} 
%    The count of node role instances described in Figure~\ref{fig:graphlet_and_role}(c).
\eit

Note that we categorize the above sets into global and local depending on whether global information in $\SGT$ (i.e., the number of all nodes in $\SGT$) is used or only local information at $v$ is used.

\smallsection{Prediction Method:}  
As the classifier, we use the \textit{random forest} model from the Scikit-learn library. The model has 30 decision trees with a maximum depth of 10.


\smallsection{Evaluation Method:} 
We use $80\%$ of the nodes for training and the remaining $20\%$ for testing. 
We evaluate the predictive performance in terms of F1-score, accuracy, and \textit{Area Under the ROC curve} (AUROC). A higher value indicates better prediction performance.

%Prediction performance (AUROC) on future node centrality prediction

\smallsection{Result:}
%The predictive performances are summarized in Tables~\ref{tab:pred-node-feature} and \ref{tab:pred-node-degree}. 
Table~\ref{tab:pred-node-overall} shows the predictive performance from each set of input features when $d_\theta=2$, and Table~\ref{tab:pred-node-overall-degree} shows how the performance depends on the in-degree threshold $d_\theta$.
In the tables, we report the mean of each prediction performance over $10$ runs in the 7 datasets in Section~\ref{section:datasets} except for the two largest ones (i.e., Patent and Stackoverflow).
From the results, we draw the following observations.

\vspace{0.5mm}

\noindent\fbox{%
        \parbox{0.98\columnwidth}{%
        \vspace{-2mm}
        \begin{observation} \label{obs:node:pred_nr_npp}
            Among local features, 
            the counts of node roles at each node (\textbf{Local-NR}) are more informative than (\textbf{Local-NPP}) for future importance prediction.
        \end{observation}
        \begin{observation}\label{obs:node:pred_all}
            The considered sets of features are complementary to each other. Using them all (\textbf{ALL}) leads to the best predictive performance in most cases.
        \end{observation}
        \begin{observation}\label{obs:node:degree}
            As nodes have more neighbors, their future importance can be predicted more accurately.
        \end{observation}
        \vspace{-2mm}
        }%
    }

%The prediction performances with standard deviations in each graph can be founded in Appendix~\ref{sec:appendix:pred_details}.


\begin{table}
% 	\vspace{-2mm}
	\centering
	\caption{\label{tab:pred-node-overall} 
    F1-score, accuracy, and AUROC on the task of predicting future node importance when $d_\theta=2$ averaged over the 7 considered real-world graphs.
	Among local features, using \textbf{Local-NR} yields better performance than using \textbf{Local-NPP} in all settings. 
	Using  \textbf{ALL} leads to the best performance in most cases, indicating that the considered sets of features are complementary to each other. Detailed results on each dataset can be found in Appendix~\ref{sec:appendix:node:details}.
	}
	%Second, Global-NR has a little higher performance than Global-NPP, and it also has higher than Global-basic. Lastly, ALL has the highest predictive performance which indicates the proposed feature is complementary to other features. \red{todo} }   
	%\scalebox{0.95}{
	\resizebox{\columnwidth}{!}{
		\begin{tabular}{|c|c|c|c|c|c|c|}
		\hline
			Target & \multicolumn{3}{c|}{Degree} & \multicolumn{3}{c|} {Betweenness}  \\ 
			\hline
			Measure & F1-score & Accuracy & AUROC & F1-score & Accuracy & AUROC \\
			\hline
			Local-NR     & 0.39 & 0.69 & 0.68 & 0.59 & 0.83 & 0.82 \\
			Local-NPP    & 0.38 & 0.68 & 0.64 & 0.58 & 0.81 & 0.79 \\
			\hline
			Global-NR    & \bf{0.57} & \bf{0.74} & \bf{0.78} & 0.64 & 0.84 & 0.85 \\
			Global-NPP   & \bf{0.57} & 0.73      & 0.77      & 0.64 & 0.84 & 0.85 \\
			Global-Basic & 0.50      & 0.72      & 0.73      & 0.24 & 0.73 & 0.67 \\
			\hline
			ALL          & \bf{0.57}      & \bf{0.74} & \bf{0.78} & \bf{0.65} & \bf{0.85} & \bf{0.86}\\
			\hline
			\hline
			Target & \multicolumn{3}{c|}{Closeness} & \multicolumn{3}{c|} {PageRank}  \\ 
			\hline
			Measure & F1-score & Accuracy & AUROC & F1-score & Accuracy & AUROC \\
			\hline
			Local-NR     & 0.51 & 0.76 & 0.78 & 0.42 & 0.73 & 0.73  \\
			Local-NPP    & 0.43 & 0.70 & 0.69 & 0.37 & 0.69 & 0.67 \\
			\hline
			Global-NR    & 0.68 & 0.82 & 0.87 & 0.54 & \bf{0.75} & \bf{0.79} \\
			Global-NPP   & 0.66 & 0.80 & 0.85 & 0.54 & 0.74 & 0.78 \\
			Global-Basic & 0.59 & 0.75 & 0.79 & 0.47 & 0.71 & 0.74 \\
			\hline
			ALL          & \bf{0.69} & \bf{0.83} & \bf{0.88} & \bf{0.56} & \bf{0.75} & \bf{0.79} \\
			\hline
		\end{tabular}
	}
\end{table}

\smallsection{Feature Importance:} Additionally, we measure the importance of each feature in the set \textbf{ALL} using \textit{Gini-importance}~\cite{loh2011classification}, and we report the top five important features in Table~\ref{tab:feature_importance} in Appendix~\ref{sec:appendix:feature_importance}.

\vspace{0.5mm}

\noindent\fbox{%
        \parbox{0.98\columnwidth}{%
        \vspace{-2mm}
        \begin{observation}
            Strong predictors vary depending on centrality measures. For example, for betweenness centrality, the counts of node roles as bridges (i.e., \textbf{Local NR-4} and \textbf{Global NR-4}) are strong.
        \end{observation}
        
        \vspace{-2mm}
        }%
    }