%%
%% This is file `sample-authordraft.tex',
%% generated with the docstrip utility.
%%
%% The original source files were:
%%s
%% samples.dtx  (with options: `authordraft')
%% 
%% IMPORTANT NOTICE:
%% 
%% For the copyright see the source file.
%% 
%% Any modified versions of this file must be renamed
%% with new filenames distinct from sample-authordraft.tex.
%% 
%% For distribution of the original source see the terms
%% for copying and modification in the file samples.dtx.
%% 
%% This generated file may be distributed as long as the
%% original source files, as listed above, are part of the
%% same distribution. (The sources need not necessarily be
%% in the same archive or directory.)
%%
%% The first command in your LaTeX source must be the \documentclass command.
%\documentclass[sigconf, anonymous, review]{acmart}
\documentclass[sigconf]{acmart}
%% NOTE that a single column version may be required for 
%% submission and peer review. This can be done by changing
%% the \doucmentclass[...]{acmart} in this template to 
%% \documentclass[manuscript,screen,review]{acmart}
%% 
%% To ensure 100% compatibility, please check the white list of
%% approved LaTeX packages to be used with the Master Article Template at
%% https://www.acm.org/publications/taps/whitelist-of-latex-packages 
%% before creating your document. The white list page provides 
%% information on how to submit additional LaTeX packages for 
%% review and adoption.
%% Fonts used in the template cannot be substituted; margin 
%% adjustments are not allowed.
%%
%% \BibTeX command to typeset BibTeX logo in the docs
\AtBeginDocument{%
  \providecommand\BibTeX{{%
    \normalfont B\kern-0.5em{\scshape i\kern-0.25em b}\kern-0.8em\TeX}}}

    
%% NOTE that a single column version may be required for 
%% submission and peer review. This can be done by changing
%% the \doucmentclass[...]{acmart} in this template to 
%% \documentclass[manuscript,screen,review]{acmart}
%% 
%% To ensure 100% compatibility, please check the white list of
%% approved LaTeX packages to be used with the Master Article Template at
%% https://www.acm.org/publications/taps/whitelist-of-latex-packages 
%% before creating your document. The white list page provides 
%% information on how to submit additional LaTeX packages for 
%% review and adoption.
%% Fonts used in the template cannot be substituted; margin 
%% adjustments are not allowed.
%%
%% \BibTeX command to typeset BibTeX logo in the docs
\usepackage{multirow}
\usepackage{caption}
%\usepackage{subcaption}
\usepackage[textfont=normalfont]{subcaption}
\usepackage{enumitem}
\usepackage[export]{adjustbox}
\setlist[itemize]{leftmargin=*}
\usepackage{color}
%\usepackage{colortbl}
\definecolor{bc}{RGB}{0,0,255}
\definecolor{rc}{RGB}{255, 0, 0}
\usepackage[ruled, vlined, linesnumbered]{algorithm2e} %preamble
\usepackage{algpseudocode}
\usepackage{amsmath}
\let\Bbbk\relax
\usepackage{amssymb}
%\usepackage{booktabs}
\usepackage{amsthm}
\usepackage{changepage}
\usepackage{tabularx}
\usepackage{url}

%% Rights management information.  This information is sent to you
%% when you complete the rights form.  These commands have SAMPLE
%% values in them; it is your responsibility as an author to replace
%% the commands and values with those provided to you when you
%% complete the rights form.
%\setcopyright{acmcopyright}
%\copyrightyear{2022}
%\acmYear{2022}
%\acmDOI{10.1145/1122445.1122456}

%% These commands are for a PROCEEDINGS abstract or paper.
%\acmConference[WWW '22]{the Web Conference 2022}{April 25--29, 2022}{Online}
%\acmBooktitle{KDD '21: ACM SIGKDD Conference on Knowledge Discovery and Data Mining,
%  August 14--18, 2021, Virtual}
%\acmPrice{15.00}
%\acmISBN{978-1-4503-XXXX-X/18/06}

%% \newtheorem{observation}{Observation}
%% \newtheorem{conjecture}{Conjecture}
%\newtheorem{problem}{Problem}
%% \newtheorem{algo}{Algorithm}
%\newtheorem{definition}{Definition}
%\newtheorem{lemma}{Lemma}
%\newtheorem{theorem}{Theorem}
%% \newtheorem{question}{Question}
%%\newtheorem{example}{Example}
%%\newtheorem{answer}{Answer}
%%\newtheorem{proof}{Proof}
%\newtheorem{axiom}{Axiom}
%\newtheorem{property}{Property}
%\newtheorem{invariant}{Invariant}

% \newcommand{\beq}{\begin{equation}}
%	\newcommand{\eeq}{\end{equation}}
% \newcommand{\bit}{\begin{itemize*}}
%	\newcommand{\eit}{\end{itemize*}}
\newcommand{\goal}[1]{ {\noindent {$\Rightarrow$} \em {#1} } }
% \newcommand{\hide}[1]{}
%\newcommand{\comment}[1]{ {\footnotesize {#1} } }
%\newtheorem{lemma}{Lemma}
%\newtheorem{proposition}{Proposition}
%\newtheorem{theorem}{Theorem}
\newtheorem{defn}{Definition}
%\newtheorem{algo}{Algorithm}
\newtheorem{observation}{Observation}
\newtheorem{pattern}{Pattern}

%\renewcommand{\algorithmicforall}{\textbf{for each}}
%\renewcommand{\algorithmicrequire}{\textbf{Input:}}
%\renewcommand{\algorithmicensure}{\textbf{Output:}}
%\renewcommand{\qedsymbol}{\rule{0.7em}{0.7em}}

\newcommand{\hide}[1]{}
\newcommand{\notice}[1]{{\textsf{\textcolor{green}{{\em [#1]}}}}}
\newcommand{\reminder}[1]{{\textsf{\textcolor{red}{[#1]}}}}
\newcommand{\vectornorm}[1]{\left|\left|#1\right|\right|}

\newcommand{\mkclean}{
    \renewcommand{\reminder}{\hide}
}

\renewcommand\labelenumi{(\theenumi)}


\newcommand{\BR}{\mathbb{R}}
\newcommand{\BN}{\mathbb{N}}

\newcommand{\TX}{\tensor{X}}
\newcommand{\TY}{\tensor{Y}}
\newcommand{\TR}{\tensor{R}}
\newcommand{\TB}{\tensor{B}}
\newcommand{\TS}{\tensor{S}}
\newcommand{\TT}{\tensor{T}}

\newcommand{\HH}{\tensor{H}}
\newcommand{\VS}[1]{V^{(#1)}}
\newcommand{\VE}[1]{v^{(#1)}}

\newcommand{\cint}{\textit{C}}
%\newcommand{\fint}{\textit{I}}
\newcommand{\cintlong}{Closing Interval\xspace}
\newcommand{\fintlong}{Total Interval\xspace}

\newcommand{\NZX}{\Omega}
\newcommand{\NZY}{\Psi}
\newcommand{\SG}{\mathcal{G}}
\newcommand{\SV}{\mathcal{V}}
\newcommand{\SM}{\mathcal{M}}
\newcommand{\SH}{\mathcal{H}}
\newcommand{\SHP}{\mathcal{H'}}
\newcommand{\SN}{\mathcal{N}}
\newcommand{\SE}{\mathcal{E}}
\newcommand{\SSS}{\mathcal{S}}
\newcommand{\SW}{\mathcal{W}}
\newcommand{\SR}{\mathcal{R}}
\newcommand{\SI}{\mathcal{I}}
\newcommand{\tv}{t_{v,d_\theta}}
\newcommand{\te}{t_{e,d_\theta}}
\newcommand{\SGTv}{\mathcal{G}^{(t_{v,d_\theta})}}
\newcommand{\SGTe}{\mathcal{G}^{(t_{e,d_\theta})}}
\newcommand{\SGT}{\mathcal{G}^{(t)}}
\newcommand{\SVT}{\mathcal{V}^{(t)}}
\newcommand{\ET}{e^{(t)}}
\newcommand{\ES}{e^{(s)}}
\newcommand{\PT}{p^{(t)}}
\newcommand{\PTM}{p^{(t-1)}}
\newcommand{\SET}{\mathcal{E}^{(t)}}
\newcommand{\SDT}{\mathcal{D}^{(t)}}
\newcommand{\SAT}{\mathcal{A}^{(t)}}
\newcommand{\SER}{\mathcal{\tilde{E}}}

\newcommand{\SGR}{\mathcal{\tilde{G}}}
\newcommand{\SGRT}{\mathcal{\tilde{G}}^{(t)}}
\newcommand{\SERT}{\mathcal{\tilde{E}}^{(t)}}
\newcommand{\SVRT}{\mathcal{\tilde{V}}^{(t)}}

\newcommand{\SSTone}{\mathcal{S}^{(t)}_1}
\newcommand{\SSTtwo}{\mathcal{S}^{(t)}_2}
\newcommand{\SSTthirteen}{\mathcal{S}^{(t)}_{13}}
\newcommand{\SSTi}{\mathcal{S}^{(t)}_i}
\newcommand{\SSRTi}{\mathcal{\tilde{S}}^{(t)}_i}

\newcommand{\SGTk}{n^{(t)}_k}
\newcommand{\SNTi}{m^{(t)}_i}
\newcommand{\SNTone}{m^{(t)}_1}
\newcommand{\SNTtwo}{m^{(t)}_2}
\newcommand{\SNTthirty}{m^{(t)}_{30}}
\newcommand{\SNTj}{m^{(t)}_j}
%\newcommand{\SETj}{l^{(t)}_j}
\newcommand{\SNTvone}{m^{(\tv)}_1}
\newcommand{\SNTvtwo}{m^{(\tv)}_2}
\newcommand{\SNTvthirty}{m^{(\tv)}_{30}}
\newcommand{\SNTvj}{m^{(\tv)}_j}
\newcommand{\SETvj}{l^{(\tv)}_j}

%\newcommand{\SER}{\mathcal{\tilde{E}}}
\newcommand{\STR}{\mathcal{\tilde{T}}}
%%%%%% Add some command by dj %%%%%%
\newcommand{\DT}{\Delta^{(t)}}
\newcommand{\dt}{\delta^{(t)}}
\newcommand{\SD}{\mathcal{D}}
\newcommand{\SA}{\mathcal{A}}
\newcommand{\SB}{\mathcal{B}}
%\newcommand{\SER}{\SE_{\SR}}
\newcommand{\SERTM}{\SE^{(t-1)}_{\SR}}
\newcommand{\Euv}{\{u,v\}}
\newcommand{\Evw}{\{v,w\}}
\newcommand{\Ewu}{\{w,u\}}
\newcommand{\Exy}{\{x,y\}}
\newcommand{\Tuv}{t_{uv}}
\newcommand{\Tvw}{t_{vw}}
\newcommand{\Twu}{t_{wu}}
\newcommand{\Euvdelta}{(\Euv,\delta)}
\newcommand{\Euvplus}{(\Euv,+)}
\newcommand{\Euvminus}{(\Euv,-)}
\newcommand{\Egen}{e^{(i)}_{uvw}}
\newcommand{\Emax}{e^{(3)}_{uvw}}
\newcommand{\Emed}{e^{(2)}_{uvw}}
\newcommand{\Emin}{e^{(1)}_{uvw}}
\newcommand{\tprobadv}{q_{uvw}^{(t)}}
\newcommand{\uvinRt}{\Euv\in\SR^{(t)}}
\newcommand{\uvinRtm}{\Euv\in\SR^{(t-1)}}
\newcommand{\xyinRtm}{\Exy\in\SR^{(t-1)}}
\newcommand{\vwinRt}{\Evw\in\SR^{(t)}}
\newcommand{\wuinRt}{\Ewu\in\SR^{(t)}}
\newcommand{\oneuv}{\mathds{1}(\uvinRtm)}
\newcommand{\onexy}{\mathds{1}(\Exy\in\SR^{(t-1)})}
\newcommand{\zrtm}{z_{\SR}^{(t-1)}}
\newcommand{\yrtm}{y_{\SR}^{(t-1)}}
\newcommand{\dtm}{d^{(t-1)}}
\newcommand{\mtm}{m^{(t-1)}}
\newcommand{\msm}{m^{(s-1)}}

\newcommand{\addedt}{\mathcal{A}}
\newcommand{\deletedt}{\mathcal{D}}



\newcommand{\triple}{\{u,v,w\}}
\newcommand{\triplet}{\{u,v,w\}^{(t)}}
\newcommand{\triples}{\{u,v,w\}^{(s)}}
\newcommand{\type}{type_{uvw}}
\newcommand{\tprob}{p_{uvw}^{(t)}}
\newcommand{\tcount}{x_{uvw}}
\newcommand{\tcountt}{x_{uvw}^{(t)}}
\newcommand{\tcounty}{y_{uvw}}
\newcommand{\tcounts}{x_{uvw}^{(s)}}
\newcommand{\tcountys}{y_{uvw}^{(s)}}

\newcommand{\tgen}{t^{(i)}_{uvw}}
\newcommand{\tmax}{t^{(3)}_{uvw}}
\newcommand{\tmed}{t^{(2)}_{uvw}}
\newcommand{\tmin}{t^{(1)}_{uvw}}

\newcommand{\SGTM}{\mathcal{G}^{(t-1)}}
\newcommand{\SVTM}{\mathcal{V}^{(t-1)}}
\newcommand{\SETM}{\mathcal{E}^{(t-1)}}
\newcommand{\SESM}{\mathcal{E}^{(s-1)}}

\newcommand{\ST}{\mathcal{T}}
\newcommand{\STT}{\mathcal{T}^{(t)}}
\newcommand{\STTM}{\mathcal{T}^{(t-1)}}
\newcommand{\CT}{c^{(t)}}
\newcommand{\SGH}{\hat{\mathcal{G}}}
\newcommand{\SVH}{\hat{\mathcal{V}}}
\newcommand{\SEH}{\hat{\mathcal{E}}}

% journal
\newcommand{\nbb}{n_{b}}
\newcommand{\nbbt}{n_{b}^{(t)}}
\newcommand{\nggt}{n_{g}^{(t)}}
\newcommand{\nbbtm}{n_{b}^{(t-1)}}
\newcommand{\nggtm}{n_{g}^{(t-1)}}

% from cf: shorthands - also they make tighter lists
\renewcommand{\labelitemi}{$\bullet$}
\newcommand{\bit}{\begin{itemize}}
\newcommand{\eit}{\end{itemize}}
\newcommand{\ben}{\begin{enumerate*}}
\newcommand{\een}{\end{enumerate*}}

%\newcommand{\bit}{\begin{compactitem}}
%\newcommand{\eit}{\end{compactitem}}
%\newcommand{\ben}{\begin{compactenum}}
%\newcommand{\een}{\end{compactenum}}


\newcommand{\QED}{ \hfill {\bf QED}}
\newcommand{\method}{\textsc{WRS}\xspace}
\newcommand{\simple}{\textsc{WRS\textsubscript{INS}}\xspace}
\newcommand{\adv}{\textsc{WRS\textsubscript{DEL}}\xspace}
%\newcommand{\simple}{\textsc{WRS\textsubscript{SIMPLE}}\xspace}


\newcommand{\doulion}{\textsc{Doulion}\xspace}
\newcommand{\mascot}{\textsc{Mascot}\xspace}
\newcommand{\triestbase}{\textsc{Tri{\`e}st}\xspace}
\newcommand{\triest}{\textsc{Triest-IMPR}\xspace}
\newcommand{\triestimpr}{\textsc{Tri{\`e}st\textsubscript{IMPR}}\xspace}
\newcommand{\triestfd}{\textsc{Tri{\`e}st\textsubscript{FD}}\xspace}
\newcommand{\thinkd}{\textsc{ThinkD}\xspace}
\newcommand{\thinkdfast}{\textsc{ThinkD\textsubscript{FAST}}\xspace}
\newcommand{\thinkdacc}{\textsc{ThinkD\textsubscript{ACC}}\xspace}
\newcommand{\esd}{\textsc{ESD}\xspace}

\newcommand{\methodS}{\textsc{DenseStream}\xspace}
\newcommand{\methodA}{\textsc{DenseAlert}\xspace}
\newcommand{\dcube}{\textsc{D-Cube}\xspace}
\newcommand{\mzoom}{\textsc{M-Zoom}\xspace}
\newcommand{\cross}{\textsc{CrossSpot}\xspace}
\newcommand{\eigenspokes}{\textsc{EigenSpokes}\xspace}
\newcommand{\spoken}{\textsc{SpokEn}\xspace}
\newcommand{\fBox}{\textsc{FBox}\xspace}
\newcommand{\netprobe}{\textsc{NetProbe}\xspace}
\newcommand{\trustrank}{\textsc{TrustRank}\xspace}
\newcommand{\fraudeagle}{\textsc{FraudEagle}\xspace}
\newcommand{\copycatch}{\textsc{CopyCatch}\xspace}
\newcommand{\oddball}{\textsc{OddBall}\xspace}
\newcommand{\fraudar}{\textsc{Fraudar}\xspace}
\newcommand{\cube}{Cube}
\newcommand{\order}{D-ordering\xspace}
\newcommand{\aslice}{\TT_{(n,i_{n})}}
\newcommand{\aindex}{(i_{1},...,i_{N})}
\newcommand{\avalue}{t_{i_{1}...i_{N}}}
\newcommand{\increment}{(\aindex,\delta, +)}
\newcommand{\decrement}{(\aindex,\delta, -)}
\newcommand{\cmax}{c_{max}}


\newcommand{\aset}{Q}
\newcommand{\amem}{q}
\newcommand{\amemtwo}{r}
\newcommand{\asubset}{\aset_{\pi,\amem}}
\newcommand{\asubtensor}{\TT(\asubset)}
\newcommand{\asubsetmax}{S_{max}}
%\newcommand{\asubsetmax}{\aset_{\pi}^{max}}
\newcommand{\asimplesubtensor}{\TT(S)}
\newcommand{\asubtensormax}{\TT(\asubsetmax)}
\newcommand{\pifunnoarg}{\pi(\cdot)}
\newcommand{\dfunnoarg}{d_{\pi}(\cdot)}
\newcommand{\cfunnoarg}{c_{\pi}(\cdot)}
\newcommand{\pifun}[1]{\pi(#1)}
\newcommand{\pifunnew}[1]{\pi'(#1)}
\newcommand{\piinvfun}[1]{\pi^{-1}(#1)}
\newcommand{\dfun}[1]{d(#1)}
\newcommand{\dpifun}[1]{d_{\pi}(#1)}
\newcommand{\cpifun}[1]{c_{\pi}(#1)}
\newcommand{\adegree}{\dpifun{\amem}}
\newcommand{\acore}{\cpifun{\amem}}
\newcommand{\asum}[1]{sum(#1)}
\newcommand{\annz}[1]{nnz(#1)}
\newcommand{\asimpledegree}{\dfun{\asimplesubtensor, \amem}}

\newcommand{\massc}[1]{mass(#1)}
\newcommand{\sizec}[1]{size(#1)} 
\newcommand{\volumec}[1]{volume(#1)}
\newcommand{\mass}[1]{M_{#1}}
\newcommand{\size}[1]{S_{#1}}
\newcommand{\volume}[1]{V_{#1}}
\newcommand{\avgval}[1]{\bar{\omega}(#1)}
%\newcommand{\size}{Size}
%\newcommand{\volume}{Vol}
%\newcommand{\mass}{Mass}
\newcommand{\density}[1]{\rho(#1)}
\newcommand{\densityopt}{\rho_{opt}}
\newcommand{\densitynoarg}{\rho}
%\newcommand{\densityarinoarg}{\rho_{ari}}
%\newcommand{\densitygeonoarg}{\rho_{geo}}
%\newcommand{\densitysuspnoarg}{\rho_{susp}}
%\newcommand{\densityno}[2]{\rho(#1,#2)}
%\newcommand{\densityari}[2]{\rho_{ari}(#1,#2)}
%\newcommand{\densitygeo}[2]{\rho_{geo}(#1,#2)}
%\newcommand{\densitysusp}[2]{\rho_{susp}(#1,#2)}
\newcommand{\minsize}{S_{min}}
\newcommand{\maxsize}{S_{max}}
\newtheorem{problem}{Problem}
%\DeclareMathOperator*{\argmax}{arg\,max}
%\DeclareMathOperator*{\argmin}{arg\,min}

\newcommand{\cmark}{\textcolor{OliveGreen}{\ding{51}}}%
%\newcommand{\cmark}{\ding{51}}%
\newcommand{\lmark}{$\triangle$\xspace}%
\newcommand{\xmark}{\textcolor{OrangeRed}{\ding{55}}}%
\newcommand{\dmark}{-}%

%\newcommand{\douli}{\textsc{Douli}}
\newcommand{\mapreduce}{\textsc{MapReduce}\xspace}
\newcommand{\hadoop}{\textsc{Hadoop}\xspace}
%\newcommand{\nodeiterator}{\textsc{NodeIterator}\xspace}
%\newcommand{\edgeiterator}{\textsc{EdgeIterator}\xspace}
%\newcommand{\doulinodeiterator}{\textsc{DouliNodeIterator}\xspace}
%\newcommand{\eigen}{\textsc{EigenTriangle}\xspace}

\newcommand{\bluecolor}{\textcolor{blue}}
\newcommand{\redcolor}{\textcolor{red}}
\newcommand{\change}{\textcolor{blue}}
\newcommand{\todo}{\redcolor{TODO: }}
\newcommand{\kijung}[1]{\redcolor{[Kijung: #1]}}
%\newcommand\red[1]{\textcolor{red}{#1}}

%\renewcommand{\algorithmicrequire}{\textbf{Input:}}
%\renewcommand{\algorithmicensure}{\textbf{Output:}}

\newcommand\Wtilde{\stackrel{\sim}{\smash{\mathcal{W}}\rule{0pt}{1.1ex}}}

\newcommand{\ul}{\underline}
\newcommand{\bc}{\cellcolor{bc}}
\newcommand{\rc}{\cellcolor{rc}}
%\newcommand{\f}[1]{\ul{\bf{#1}}}
\newcommand{\f}[1]{\textbf{#1}}
\newcommand{\s}{\underline}

\newcommand{\dongjin}[1]{\bluecolor{[Dongjin: #1]}\xspace}

\newcommand{\here}{\redcolor{Deukryeol is here}}
\newcommand{\blue}[1]{\textcolor{blue}{#1}}
\newcommand\red[1]{\textcolor{red}{#1}}
\newtheorem{definition}{Definition}
\newtheorem{theorem}{Theorem}
\newtheorem{lemma}{Lemma}
\newcommand{\DTV}{d^{(t)}(v)}
\newcommand{\DTU}{d^{(t)}(u)}
\newcommand{\DTVi}{d^{(t_i-1)}(v)}
\newcommand{\DTUi}{d^{(t_i-1)}(u)}

%%
%% Submission ID.
%% Use this when submitting an article to a sponsored event. You'll
%% receive a unique submission ID from the organizers
%% of the event, and this ID should be used as the parameter to this command.
%%\acmSubmissionID{123-A56-BU3}

%%
%% The majority of ACM publications use numbered citations and
%% references.  The command \citestyle{authoryear} switches to the
%% "author year" style.
%%
%% If you are preparing content for an event
%% sponsored by ACM SIGGRAPH, you must use the "author year" style of
%% citations and references.
%% Uncommenting
%% the next command will enable that style.
%%\citestyle{acmauthoryear}

%\setlength{\textfloatsep}{0.05cm}
%\setlength{\dbltextfloatsep}{0.05cm}
\setlength{\abovecaptionskip}{0.25cm}
\setlength{\skip\footins}{0.05cm}

%%
%% end of the preamble, start of the body of the document source.

\begin{document}
\newcommand{\smallsection}[1]{{\vspace{0.05in} \noindent {\bf{\underline{\smash{#1}}}}}}

%%
%% The "title" command has an optional parameter,
%% allowing the author to define a "short title" to be used in page headers.
%\title{How Do Network Motifs Change over Time?}
\title{\huge Graphlets over Time: A New Lens for Temporal Network Analysis}

%%
%% The "author" command and its associated commands are used to define
%% the authors and their affiliations.
%% Of note is the shared affiliation of the first two authors, and the
%% "authornote" and "authornotemark" commands
%% used to denote shared contribution to the research.

\settopmatter{authorsperrow=4}
\author{Deukryeol Yoon}
\affiliation{%
  \institution{KAIST AI}
  \city{Seoul}
  \country{South Korea}
}
\email{deukryeol.yoon@kaist.ac.kr}

\author{Dongjin Lee}
\affiliation{%
  \institution{KAIST EE}
  \city{Daejeon}
  \country{South Korea}
}
\email{dongjin.lee@kaist.ac.kr}

\author{Minyoung Choe}
\affiliation{%
  \institution{KAIST AI}
  \city{Seoul}
  \country{South Korea}
}
\email{minyoung.choe@kaist.ac.kr}

\author{Kijung Shin}
\affiliation{%
  \institution{KAIST AI \& EE}
  \city{Seoul}
  \country{South Korea}
}
\email{kijungs@kaist.ac.kr}
%\author{Anonymous Author(s)}

%\ccsdesc[300]{Information systems~Data mining}
%\ccsdesc[100]{Networks~Network reliability}

%%
%% Keywords. The author(s) should pick words that accurately describe
%% the work being presented. Separate the keywords with commas.
%\keywords{Temporal Graph Analysis, Network Motifs, Graphlets}


%%
%% By default, the full list of authors will be used in the page
%% headers. Often, this list is too long, and will overlap
%% other information printed in the page headers. This command allows
%% the author to define a more concise list
%% of authors' names for this purpose.
%%\renewcommand{\shortauthors}{, et al.}

%%
%% The abstract is a short summary of the work to be presented in the
%% article.


\setlength{\textfloatsep}{0.12cm}
\setlength{\dbltextfloatsep}{0.12cm}
\setlength{\abovecaptionskip}{0.12cm}
\setlength{\skip\footins}{0.12cm}

\settopmatter{printacmref=false} 
\settopmatter{printfolios=false}
\fancyhead{}
    
%%
%% This command processes the author and affiliation and title
%% information and builds the first part of the formatted document.
\begin{abstract}
    \begin{abstract}

Visual perception tasks often require vast amounts of labelled data, including 3D poses and image space segmentation masks. The process of creating such training data sets can prove difficult or time-intensive to scale up to efficacy for general use. Consider the task of pose estimation for rigid objects. Deep neural network based approaches have shown good performance when trained on large, public datasets. However, adapting these networks for other novel objects, or fine-tuning existing models for different environments, requires significant time investment to generate newly labelled instances. Towards this end, we propose ProgressLabeller as a method for more efficiently generating large amounts of 6D pose training data from color images sequences for custom scenes in a scalable manner. ProgressLabeller is intended to also support transparent or translucent objects, for which the previous methods based on depth dense reconstruction will fail.
We demonstrate the effectiveness of ProgressLabeller by rapidly create a dataset of over 1M samples with which we fine-tune a state-of-the-art pose estimation network in order to markedly improve the downstream robotic grasp success rates. Progresslabeller is open-source at \href{https://github.com/huijieZH/ProgressLabeller}{https://github.com/huijieZH/ProgressLabeller}

\end{abstract}
\end{abstract}

\maketitle


\epigraph{\normalsize ``\textit{ \textbf{The essence of a riddle is to express true facts under impossible combinations.}}"}{\normalsize--- \textit{Aristotle}, \textit{Poetics} (350 BCE)\vspace{0pt}}

\noindent
A \textit{riddle} is a puzzling question about {concepts} in our everyday life.
% , and we which one needs common sense to reason about.
For example, a riddle might ask ``\textit{My life can be measured in hours. I serve by being devoured. Thin, I am quick. Fat, I am slow. Wind is my foe. What am I?}''~
The correct answer ``\textit{candle},'' is reached by considering a collection of \textit{commonsense knowledge}:
{a candle can be lit and burns for a few hours; a candle's life depends upon its diameter; wind can extinguish candles, etc.}
\begin{figure}[t]
	\centering 
	\includegraphics[width=1\linewidth]{riddle_intro_final.pdf}
	\caption{ 
    The top example is a trivial commonsense question from the CommonsenseQA~\cite{Talmor2018CommonsenseQAAQ} dataset. 
    The two bottom examples are from our proposed \textsc{RiddleSense} challenge.
    The right-bottom question is a descriptive riddle that implies multiple commonsense facts about \textit{candle}, and it needs understanding of figurative language such as metaphor;
    The left-bottom one additionally needs counterfactual reasoning ability to address the \textit{`but-no'} cues. 
    These riddle-style commonsense questions  require NLU systems to have higher-order reasoning skills with the understanding of creative language use.
	}
	\label{fig:intro} 
\end{figure}

It is believed that the \textit{riddle} is one of the earliest forms of oral literature,
which can be seen as a formulation of thoughts about common sense, a mode of association between everyday concepts, and a metaphor as higher-order use of natural language~\cite{hirsch2014poet}.
Aristotle stated in his \textit{Rhetoric} (335-330 BCE) that good riddles generally provide satisfactory metaphors for rethinking common concepts in our daily life.
He also pointed out in the \textit{Poetics} (350 BCE): ``the essence of a riddle is to express true facts under impossible combinations,'' which suggests that solving riddles is a nontrivial  reasoning task.

Answering riddles is indeed a challenging cognitive process as it requires \textit{complex} {commonsense reasoning skills}.
% which we refer to \textit{higher-order commonsense reasoning}. 
% A successful riddle-solving model should be able to reason with \textit{multiple pieces} of commonsense facts, as 
A riddle can describe \textit{multiple pieces} of commonsense knowledge with \textit{figurative} devices such as metaphor and personification (e.g., ``wind is my \underline{foe} $\xrightarrow[]{}$ \textit{extinguish}'').
% , as shown by the examples in Figure~\ref{fig:intro}.
%%%
Moreover, \textit{counterfactual thinking} is also necessary for answering many riddles such as ``\textit{what can you hold in your left hand \underline{but not} in your right hand? $\xrightarrow[]{}$ your right elbow.}''
These riddles with \textit{`but-no'} cues require that models use counterfactual reasoning ability to consider possible solutions for situations or objects that are seemingly impossible at face value.
This \textit{reporting bias}~\cite{gordon2013reporting} makes riddles a more difficult type of commonsense question for pretrained language models to learn and reason.
% In addition, the model needs to associate commonsense knowledge with the creative use of language in descriptions, which may have figurative devices such as metaphor and personification (e.g., ``wind is my \underline{foe} $\xrightarrow[]{}$ \textit{extinguish}''). 
%For instance, one needs to know that devour
% Thus, a riddle here can be seen as a complex commonsense question that tests higher-order reasoning ability with creativity.
In contrast, \textit{superficial} commonsense questions such as ``\textit{What home entertainment equipment requires cable?}'' in  CommonsenseQA~\cite{Talmor2018CommonsenseQAAQ} are more straightforward and explicitly stated.
We illustrate this comparison in Figure~\ref{fig:intro}.


In this paper,
we introduce the \textsc{RiddleSense} challenge 
to study the task of answering riddle-style commonsense questions\footnote{We use ``riddle'' and ``riddle-style commonsense question'' interchangeably in this paper.} requiring \textit{creativity}, \textit{counterfactual thinking} and \textit{complex commonsense reasoning}.
\textsc{RiddleSense} is presented as a \textit{multiple-choice question answering} task where a model selects one of five answer choices to a given riddle question as its predicted answer, as shown in Fig.~\ref{fig:intro}.
We construct the dataset by first crawling from several free websites featuring large collections of human-written riddles and then aggregating, verifying, and correcting these examples using a combination of human rating and NLP tools to create a dataset consisting of 5.7k high-quality examples.
Finally, we use \textit{Amazon Mechanical Turk} to crowdsource quality distractors to create a challenging benchmark.
We show that our riddle questions are more challenging than {CommonsenseQA} by analyzing graph-based statistics over ConceptNet~\cite{Speer2017ConceptNet5A}, a large knowledge graph for common sense reasoning.

% The distractors for the training data are automatically generated from ConceptNet and language models while the distractors for the dev and the test sets are crowd-sourced from Amazon Mechanical Turk (AMT).
% Through data analysis based on graph connectivity, 




Recent studies have demonstrated that
 fine-tuning large pretrained language models, such as {BERT}~\cite{Devlin2019}, RoBERTa, and ALBERT~\cite{Lan2020ALBERT}, can achieve strong results on current commonsense reasoning benchmarks.
Developed on top of these language models, graph-based language reasoning models such as KagNet~\cite{kagnet-emnlp19} and MHGRN~\cite{feng2020scalable} show superior performance. 
Most recently, UnifiedQA~\cite{khashabi2020unifiedqa} proposes to unify different QA tasks and train a text-to-text model for learning from all of them, which achieves state-of-the-art performance on many commonsense benchmarks.

To provide a comprehensive benchmarking analysis, we systematically compare the above methods.
Our experiments reveal that while humans achieve 91.33\% accuracy on \textsc{riddlesense}, the best language models can only achieve 68.80\% accuracy, suggesting that there is still much room for improvement in the field of solutions to complex commonsense reasoning questions with language models.
% We also provide error analysis to better understand the limitation of current methods.
We believe the proposed \textsc{RiddleSense} challenge suggests productive future directions for machine commonsense reasoning as well as the understanding of higher-order and creative use of natural language.


% (previous state-of-the-art on \texttt{CommonsenseQA} (56.7\%)).
% However, there still exists a large gap between performance of said baselines and human performance.
% we show that the questions in RiddleSense is significantly more challenging, in terms of the length of the paths from question concepts and answer concepts.


%Apart from that, current pre-trained language models (e.g., BERT~\cite{}, RoBERTa~\cite{}, etc.) and commonsense-reasoning models (e.g., KagNet~\cite{}), can be easily adapted to work for this format with minimal modifications. 


%Note that these auto-generated distractors may be still easy for , which could diminish the testing ability of the dataset.
%We design an ader filtering method to get rid of the false negative   and control the task difficulty. 
% To strengthen the task, we propose an adversarial cross-filtering method to remove the distractors that ineffectively mislead the selected base models.
% Finally, we use human efforts to inspect the distractors and remove false negative ones, to make sure that all distractors either does not make sense or much less plausible than the correct answers.
%Introducing these fine-tuned models is inspired by the adversarial filtering algorithms~\cite{}, which can effectively reduce the  bias inside datasets for creating a more reliable benchmark.  



%Those distractors are explicitly annotated by human experts such that they are close to the meaning of 
%The main idea is to use multiple trainable generative models for learning to generate answers in a cross-validation style. 
%The wrong predictions
%Simply put, for every step, we use a large subset of the riddles and their current options ot learn multiple models for answering the remaining riddles via generation.
%After each step, we consolidate the 


% In the distantly supervised learning, we use the definition of concepts (i.e., glossary) of \textit{Wiktionary}\footnote{\url{https://www.wiktionary.org/}} to create riddles with answers as training data. 
% In the transfer learning setting, we aim to test the transferability of models across relevant datasets, such as CommonsenseQA~\cite{Talmor2018CommonsenseQAAQ}.

% We believe the \textsc{RiddleSense} task can benefit multiple communities in natural language processing. 
% First, the commonsense reasoning community can use \textsc{RiddleSense} as a new space to evaluate their reasoning models. The \textsc{RiddleSense} focuses on more complex and creative commonsense questions, which will encourage them to propose more higher-order commonsense reasoning models. 
% Second, \textsc{RiddleSense} is an NLU 
% task similar to those in the GLUE~\cite{wang2018glue} and SuperGLUE~\cite{wang2019superglue} leaderboard that can serve as a benchmark for testing various pre-trained language models.
% Last but not the least, as our task shares the similar format with many open-domain question answering tasks like \textit{Natural Questions}~\cite{kwiatkowski2019natural}, researchers in QA area may be also interested in \textsc{RiddleSense}. 





\section{Basic Concepts, Notations, and Data}\label{section:prelim}

In this section, we first introduce some basic concepts and notations. Then, we describe the nine datasets used in this paper.







\subsection{Basic Concepts and Notations}
\label{sec:prelim:concept}

\smallsection{Temporal Graph:} 
%A \textit{graph} $G = (V, E)$ consists of a set of nodes $V = (v_1, v_2, ..., v_{|V|})$ and a set of directed edges $E = (\tilde{e}_1, \tilde{e}_2, ..., \tilde{e}_{|E|})$.
A \textit{temporal graph} $\SG = (\SV, \SE, \ST)$ consists of a set of nodes $\SV$, a set of directed edges $\SE:=\{e_1,\cdots,e_{|\SE|}\}$, and a multiset of edge arrival times $\ST:=[t_1,\cdots, t_{|\SE|}]$.
For each directed edge $e_i\in \SE$,  we use $t_i \in \ST$ to denote the arrival time of $e_i$.
We use $u \rightarrow v$ to denote a directed edge from a node $u$ to a node $v$, and the nodes $u$ and $v$ are \textit{adjacent} if $u \rightarrow v$ or $v \rightarrow u$ exists.
From now on, we will use the term \textit{edge} to indicate a directed edge when there is no ambiguity.

\begin{table}[t]
%    \vspace{-1mm}
	\centering
	\caption{\label{tab:notations} Table of symbols.}
% \vspace{-2mm}
    %\scalebox{0.95}{
    \resizebox{\columnwidth}{!}{
	\begin{tabular}{ r | l }
		\toprule
		{\bf Notation} & {\bf Definition} \\
		\midrule
	    $\SG = (\SV, \SE, \ST)$ & temporal graph with nodes $\SV$, edges $\SE$, and times $\ST$  \\
	    $\SGT=(\SVT,\SET)$ & snapshot of $\SG$ at time $t$ \\
	    \midrule
	    $\SGR= (\SV, \SE, \STR)$ & a temporal graph randomized from $\SG$ \\
	    $\SGRT=(\SVRT,\SERT)$ & snapshot of $\SGR$ at time $t$ \\
	    \midrule
%	    $\SGTk$ & count of graphlet $k$ instances in $\SGT$ \\
	    $\SNTi(v)$ & count of node role $i$ at a node $v$ in $\SGT$ \\
	    %$\SETj(e)$ & count of edge role $j$ at an edge $e$ in $\SGT$ \\
	    %$G_t = (V_t, E_t)$ & snapshot of the temporal graph $T$ at time $t$ \\
	   % \midrule
	   % $\SM$, $\SN$, $\SI$ & network motifs, node roles, and edge roles of size three.\\
	   %% $\SM_i$, $\SN_i$, $\SI_i$ & motif $i$, node role $i$, and edge role $i$ \\
	   % $\SM_i$, $\SN_i$, $\SI_i$ & $i$-th motif, node role, and edge role \\
	   % $\SM\{i_1, i_2, ...,  i_k\}$ & set of motifs $\SM_{i_1}, \SM_{i_2}, ..., \SM_{i_k}$  \\
	    
	   % \midrule
	   % $\SD$ & evolution pattern of the distributions of network motifs over time\\
	   % $\SN_t[v]$ & number of instances $\SN$ in node $v$ at the snapshot $G_t$\\
	   % $\SI_t[e]$ & number of instances $\SI$ in edge $e$ at the snapshot $G_t$\\
		\bottomrule
	\end{tabular}}
\end{table}

%We assume that edges are ordered chronologically, i.e., if $i<j$, then $t_i \leq t_j$. 

\smallsection{Randomized Graph:} 
A \textit{randomized graph} $\SGR= (\SV, \SE, \STR)$ of $\SG=(\SV,\SE, \ST)$ is obtained by assigning arrival times in $\ST$ to edges in $\SE$ uniformly at random in a one-to-one manner. For each edge $e_i\in \SE$, we use $\tilde{t}_i\in \STR$ to denote the arrival time assigned to it. % by randomization.

\smallsection{Snapshot:}
%For each edge $e_i\in \SE$, we use $src(e_i)\in \SV$ and $dst(e_i)\in \SV$ to denote its source node and destination node, respectively.
We define the \textit{snapshot at time $t$} of $\SG=(\SV, \SE, \ST)$ as $\SGT=(\SVT,\SET)$ where $\SET:=\{e_i\in \SE:t_i \leq t\}$ and $\SVT\subseteq \SV$ is the endpoints of any edge in $\SET$.
That is, $\SGT$ consists of the nodes and edges arriving at time $t$ or earlier. Similarly, the snapshot at time $t$ of $\SGR= (\SV, \SE, \STR)$ is 
$\SGRT=(\SVRT,\SERT)$ where $\SERT:=$ $\{e_i\in \SE:\tilde{t}_i \leq t\}$ and $\SVRT$ is the endpoints of any edge in $\SERT$. 
We define the \textit{neighbors} of a node $v\in\SVT$ in a snapshot $\SGT$ as the nodes adjacent to $v$ in $\SGT$.  %i.e. $\{u\in\SVT|\exists u\rightarrow v \text{ or } \exists v\rightarrow u\}$. 
We define the \textit{degree} of a node $v\in\SVT$ in a snapshot $\SGT$, which is denoted by $\DTV$, as the number of directed edges whose endpoints include $v$ in $\SGT$.
We simply use $d(v)$ to denote the degree of the node $v$ in the last snapshot $\SG^{(t_{|\SE|})}$.

\smallsection{Induced Subgraphs:} 
%Since the term network motifs are used to represent different types of subgraphs in many studies, we first define it. 
A subgraph of a snapshot $\SGT=(\SVT,\SET)$ is \textit{induced} if and only if it consists of a subset of $\SVT$ and all of the edges connecting pairs of the nodes in the subset. 
Two subgraphs $\SH$ and $\SHP$ are \textit{isomorphic} if there exists a one-to-one mapping $f$ between the nodes of both graphs such that there exists an edge from a node $u$ to a node $v$ in $\SH$ if and only if there exists an edge from the node $f(u)$ to the node $f(v)$ in $\SHP$.

\smallsection{Graphlets:}
A \textit{graphlet} is the set of induced subgraphs that are isomorphic to each other.
%\textit{Graphlets} are sets of isomorphic subgraphs with a predefined number of nodes.
In this paper, we limit our attention to the $13$ graphlets consisting of three connected nodes. %as shown in Figure~\ref{fig:graphlet_and_role}(a). %, due to computational cost of tracking the counts of disconnected ones.
An induced subgraph is called an \textit{instance} of graphlet $k$ if it is isomorphic to the $k$-th graph in Figure~\ref{fig:graphlet_and_role}(a).
%We denote by $\SGTk$ the count of graphlet $k$ instances in a snapshot $\SGT$.

%We denote them in a snapshot $\SGT$ by $\SSTone$, $\SSTtwo$, $\cdots$, $\SSTthirteen$, and each $\SSTi$ corresponds to the set of induced subgraphs of $\SGT$ that are isomorphic to the $i$-th graph in Figure~\ref{fig:graphlet_and_role}(a).
%Similarly, we use $\SSRTi$ to denote each $i$-th graphlet in $\SGRT$.

%\red{KJ: I am here}

\smallsection{Node Roles:} 
Consider an induced subgraph $\SH$ with a node set $\SV'$. 
An \textit{automorphism} of $\SH$ is an isomorphism between $\SH$ and itself. i.e., an automorphism of $\SH$ is a one-to-one mapping between nodes of $\SH$ such that there exists an edge from a node $u$ to a node $v$ in $\SH$ if and only if there exists an edge from the node corresponding to $u$ to the node corresponding to $v$ in $\SH$.  
%The set of automorphisms of $\SH$ forms a \textit{group} under the operation of composition, denoted by $Aut(\SH)$. 
If denoting the set of automorphisms of $\SH$ by $Aut(\SH)$,
the \textit{automorphism orbit} of a node $u\in \SV'$ is the set $\{y \in \SV' : \exists g \in Aut(\SH) \text{ s.t. } y = g(u)\}$ of nodes \cite{prvzulj2007biological}.
Formally, \textit{node roles} are node automorphism orbits, and roughly, they are sets of symmetric positions of nodes within graphlets.
Figure~\ref{fig:graphlet_and_role}(b) (see the positions of black nodes) shows all $30$ node roles in the $13$ graphlets that we consider.
We say a node $v$ ``takes'' node role $i$ in a graphlet instance if there exists an isomorphism of the graphlet instance and the $i$-th graph in Figure~\ref{fig:graphlet_and_role}(b) that maps $v$ to the black node in the graph.
%We say a node $v$ ``takes'' node role $i$ in a graphlet instance if there exists an isomorphism of the graphlet and the $i$-th graph in Figure~\ref{fig:graphlet_and_role}(b) that maps $v$ to the black node in the graph.
We define the \textit{count of node role $i$ at a node $v$} as the number of graphlet instances where $v$ takes $i$, and $\SNTi(v)$ denotes the count at a snapshot $\SGT$.

\begin{table}[t]
    %    \vspace{-2mm}
	\centering
	\caption{\label{tab:data} Summary of nine real-world temporal graphs used throughout this paper.} 
    %    \vspace{-2mm}
	%\scalebox{0.95}{
     \resizebox{\columnwidth}{!}{
		\begin{tabular}{c | c | c | c | c }
			\toprule
			{\bf Domain }& {\bf Dataset} & {$|V|$} & {\bf$|E_T|$} & {\bf Period}\\
			\midrule
			\multirow{3}{*}{Citation}
			& \ul{\bf{\smash{HepPh}}}    & $34,565$      & $346,849$     & 9 years \\ 
			& \ul{\bf{\smash{HepTh}}}    & $18,477$      & $136,190$     & 10 years \\
			& \ul{\bf{Patent}}           & $3,774,362$   & $16,512,782$  & 25 years \\
			\midrule
			\multirow{3}{*}{Email/Message}
			& \ul{\bf{Enron}}               & $55,655$      & $209,203$     & 24 years \\
		    & \ul{\bf{EU}}                  & $986$         & $24,929$      & 1.5 years \\
		    & \ul{\bf{\smash{College}}}   & $1,899$       & $20,296$      & 0.5 years \\
		    \midrule
		    \multirow{3}{*}{Online Q/A}
		    & \ul{\bf{Ask}}ubuntu             & $159,316$     & $262,106$     & 6 years \\
		    & \ul{\bf{Math}}overflow          & $24,818$      & $90,489$      & 7 years \\
		    & \ul{\bf{Stack}}overflow         & $2,601,977$   & $16,266,395$  & 8 years \\
			\bottomrule
		\end{tabular}}
\end{table}

%We use $\SNT[v]$ & number of instances $\SN$ in node $v$ at the snapshot $G_t$\\

% \begin{algorithm}[ht]
% \small
% 	\caption{Counting the Instances of Each Graphlet in a Temporal Graph \label{alg:track_graphlet}} 
% 	\SetKwInOut{Input}{Input}
% 	\SetKwInOut{Output}{Output}
% 	\SetKwProg{Procedure}{Procedure}{}{}
% 	\SetKwFunction{update}{UPDATE}
	
% 	\Input{Temporal Graph $\SG=(\SV, \SE, \ST)$}
% 	\Output{The count of the instances of each graphlet in $\SG$.}
% 	\vspace{3pt}
	
% 	%\texttt{\color{blue}/* Initialization */}\\
% 	Initialize the count of the instances of each graphlet to zero. \\
% 	Initialize $\SE$ to an empty set.\\
%     \vspace{3pt}
    
% 	\For{\upshape\textbf{each edge }$e_i=u \rightarrow v$ in arrival order}{
	    
% 	    $\SN \leftarrow $ union of the neighbors of $u$ and the neighbors of $v$ (except for $u$ and $v$) \label{alg:track_graphlet:line:neighbors}
	    
% 	    \For{\upshape\textbf{each} $w\in \SN$}{
% 	       % \update{$u$, $v$, $w$} \\
% 	        \If{$u$, $v$ \textbf{and} $w$ form a graphlet instance}{ \label{alg:track_graphlet:line:if_instance}
% 	         decrement the count of the graphlet of the instance formed by $u$, $v$ and $w$ \label{alg:track_graphlet:line:decrease}
% 	         \\
            
%             %  \If{ \upshape{is\_inverse} \upshape\textbf{is} True}{
%             %      decrease the count of the instance of graphlet prev by $1$ \\
%             %      increase the count of the instance of graphlet next by $1$ \\
%             %  }
%             }
%          }
         
%          add $u\rightarrow v$ to $\SE$
         
%          \For{\upshape\textbf{each} $w\in \SN$}{
%               increment the count of the graphlet of the instance formed by $u$, $v$ and $w$ \label{alg:track_graphlet:line:increase}
% 	         \\
%          }
% 	    \vspace{3pt}
	    
% 	}
	
% 	\textbf{return} count of the instances of each graphlet instances.\\
% \end{algorithm}

\smallsection{Edge Roles:} 
Consider an induced subgraph $\SH$ with an edge set $\SE'$.
Based on the concepts defined above, we define the \textit{edge role} of an edge $u\rightarrow v$ is the set $\{x\rightarrow y \in \SE' : \exists g \in Aut(\SH) \text{ s.t. } x = g(u) \wedge y = g(v)\}$ of edges.
Roughly, edge roles are the sets of symmetric positions of edges within graphlets.
Figure~\ref{fig:graphlet_and_role}(c) (see the positions of edges from a red node to a blue node) shows all $30$ edge roles in the $13$ considered graphlets.
We say an edge $u\rightarrow v$ ``takes'' edge role $j$ in a graphlet instance if there exists an isomorphism of the graphlet instance and the $j$-th graph in Figure~\ref{fig:graphlet_and_role}(c) that maps $u$ and $v$ to the red node and the blue node, respectively, in the graph.
We define the \textit{count of edge role $j$ at an edge $e$} as the number of graphlet instances where $e$ takes $j$.
%We use $\SETj(e)$ to denote the count in a snapshot $\SGT$.

\subsection{Datasets}\label{section:datasets}
Throughout this paper, we use the nine real-world temporal graphs from the three domains, which are summarized in Table~\ref{tab:data}. 

\smallsection{Citation Graphs:} 
Each node is a paper or a patent. Each directed edge from a node $u$ to a node $v$ means that $u$ cites $v$. 

\smallsection{Email/Message Graphs:}
Each node is a user. Each directed edge from a node $u$ to a node $v$ indicates that $u$ sends $v$ emails (messages). 

\smallsection{Online Q/A Graphs:}
Each node is a user. Each directed edge from a node $u$ to a node $v$ means that $u$ answers $v$'s questions.
\begin{table*}[t]
\begin{center}
\vspace{-2mm}
\caption{\label{tab:graphlet_evolution} 
Ratios of graphlets over time. The colors in the plots are matched with the colors of the graphlets in Figure~\ref{fig:graphlet_and_role}, and the evolution ratio means the fraction of edges added to graphs.
The evolution patterns in real-world graphs vary depending on domains (Observation~\ref{obs:graphlet_evolve}), and they are clearly distinguished from the evolution patterns in randomized graphs (Observation~\ref{obs:graphlet_evolve:random}). 
}
%The graphlet evolution pattern of $\SG$ and $\SGR$. The colors are matched with those in Figure~\ref{fig:graphlet_and_role} (a) to indicate the types of graphlets. The evolution patterns of $\SG$ differs from those in $\SGR$ and they also vary depending on the domain of the graphs.}
%\vspace{-3mm}
\scalebox{0.95}{
\begin{tabular}{c|ccc|ccc}
    \toprule
    & \multicolumn{3}{c|}{Temporal graph $\SG$} & \multicolumn{3}{c}{Randomized graph $\SGR$} \\
    \hline
    \parbox[t]{2mm}{\multirow{8}{*}{\rotatebox[origin=c]{90}{\ \ \ \ \ Citation}}} &  
        \raisebox{-.9\totalheight}{\includegraphics[width=0.1475\textwidth]{FIG/graphlet_distribution/hepph.pdf}} &
        \raisebox{-.9\totalheight}{\includegraphics[width=0.1475\textwidth]{FIG/graphlet_distribution/hepth.pdf}} & 
        \raisebox{-.9\totalheight}{\includegraphics[width=0.1475\textwidth]{FIG/graphlet_distribution/patent.pdf}} & 
        \raisebox{-.9\totalheight}{\includegraphics[width=0.1475\textwidth]{FIG/graphlet_distribution/hepph-random.pdf}} & \raisebox{-.9\totalheight}{\includegraphics[width=0.1475\textwidth]{FIG/graphlet_distribution/hepth-random.pdf}} &
        \raisebox{-.9\totalheight}{\includegraphics[width=0.1475\textwidth]{FIG/graphlet_distribution/patent-random.pdf}} \\
    \hline
    \parbox[t]{2mm}{\multirow{8}{*}{\rotatebox[origin=c]{90}{\ \ \ \ \ Email/Message}}} &  
        \raisebox{-.9\totalheight}{\includegraphics[width=0.1475\textwidth]{FIG/graphlet_distribution/email-eu.pdf}} & \raisebox{-.9\totalheight}{\includegraphics[width=0.1475\textwidth]{FIG/graphlet_distribution/enron.pdf}} & 
        \raisebox{-.9\totalheight}{\includegraphics[width=0.1475\textwidth]{FIG/graphlet_distribution/college_msg.pdf}} & 
        \raisebox{-.9\totalheight}{\includegraphics[width=0.1475\textwidth]{FIG/graphlet_distribution/email-eu-random.pdf}} & \raisebox{-.9\totalheight}{\includegraphics[width=0.1475\textwidth]{FIG/graphlet_distribution/enron-random.pdf}} &
        \raisebox{-.9\totalheight}{\includegraphics[width=0.1475\textwidth]{FIG/graphlet_distribution/college_msg-random.pdf}} \\
    \hline
    \parbox[t]{2mm}{\multirow{8}{*}{\rotatebox[origin=c]{90}{\ \ \ \ Online Q/A}}} &  
        \raisebox{-.9\totalheight}{\includegraphics[width=0.1475\textwidth]{FIG/graphlet_distribution/mathoverflow.pdf}} & \raisebox{-.9\totalheight}{\includegraphics[width=0.1475\textwidth]{FIG/graphlet_distribution/askubuntu.pdf}} & 
        \raisebox{-.9\totalheight}{\includegraphics[width=0.1475\textwidth]{FIG/graphlet_distribution/stackoverflow.pdf}} & 
        \raisebox{-.9\totalheight}{\includegraphics[width=0.1475\textwidth]{FIG/graphlet_distribution/mathoverflow-random.pdf}} & \raisebox{-.9\totalheight}{\includegraphics[width=0.1475\textwidth]{FIG/graphlet_distribution/askubuntu-random.pdf}} &
        \raisebox{-.9\totalheight}{\includegraphics[width=0.1475\textwidth]{FIG/graphlet_distribution/stackoverflow-random.pdf}} \\
%& \hspace{0.4cm} Math & \hspace{0.4cm} Ask & \hspace{0.4cm} Stack & \hspace{0.4cm} Math & \hspace{0.4cm} Ask & \hspace{0.4cm} Stack \\
    \bottomrule
\end{tabular}}
\end{center}
\end{table*}


\begin{algorithm}[t]
\small
	\caption{Counting the Instances of Each Graphlet in a Temporal Graph \label{alg:track_graphlet}} 
	\SetKwInOut{Input}{Input}
	\SetKwInOut{Output}{Output}
	\SetKwProg{Procedure}{Procedure}{}{}
	\SetKwFunction{update}{UPDATE}
	
	\Input{Temporal Graph $\SG=(\SV, \SE, \ST)$}
	\Output{The count of the instances of each graphlet in $\SG$}
	\vspace{3pt}
	
	%\texttt{\color{blue}/* Initialization */}\\
	Initialize the count of the instances of each graphlet to zero \\
	Initialize $\SE$ to an empty set \\
    \vspace{3pt}
    
	\For{\upshape\textbf{each edge }$e_i=u \rightarrow v$ in arrival order}{
	    
	    $\SN \leftarrow $ union of the neighbors of $u$ and the neighbors of $v$ (except for $u$ and $v$) \label{alg:track_graphlet:line:neighbors}
	    
	    \For{\upshape\textbf{each} $w\in \SN$}{
	       % \update{$u$, $v$, $w$} \\
	        \If{$u$, $v$ \textbf{and} $w$ form a graphlet instance}{ \label{alg:track_graphlet:line:if_instance}
	         decrement the count of the graphlet of the instance formed by $u$, $v$ and $w$ \label{alg:track_graphlet:line:decrease}
	         \\
            
            %  \If{ \upshape{is\_inverse} \upshape\textbf{is} True}{
            %      decrease the count of the instance of graphlet prev by $1$ \\
            %      increase the count of the instance of graphlet next by $1$ \\
            %  }
            }
         }
         
         add $u\rightarrow v$ to $\SE$
         
         \For{\upshape\textbf{each} $w\in \SN$}{
              increment the count of the graphlet of the instance formed by $u$, $v$ and $w$ \label{alg:track_graphlet:line:increase}
	         \\
         }
	    \vspace{3pt}
	    
	}
	
	\textbf{return} count of the instances of each graphlet instances \\
\end{algorithm}

\section{Graph Level Analysis}\label{section:graph}

In this section, we study the evolution of local structures in real-world graphs.
We examine the dynamics in the distribution of graphlet instances and transitions between graphlets.
%We first observe the distributions of graphlet instances over time in the graphs. After that, we study patterns of transitions between graphlets.

\subsection{Global Level 1. Graphlets Over Time}
\label{section:graph:time}
We track how the ratio of the instances of each graphlet changes as the considered real-world graphs evolve over time. Our tracking algorithm, which is described in Algorithm~\ref{alg:track_graphlet}, is adapted from StreaM \cite{schiller2015stream}, which maintains the counts of the instances of the $4$-node undirected graphlets in a fully dynamic graph stream, where edges are not just added but also deleted over time. The time complexity of Algorithm~\ref{alg:track_graphlet} is $\Theta(\Sigma_{v\in\SV}(d(v))^2)$, as proven in Theorem~\ref{thm:time:track}.
It should be noticed that, by Lemma~\ref{lem:time:optimality}, the time complexity is $\Theta($the number of instances of all graphlets in the last snapshot$)$, which is the optimal time complexity achievable by any algorithm that counts graphlet instances by enumerating them.

\begin{theorem} \label{thm:time:track} 
The time complexity of Algorithm \ref{alg:track_graphlet} is $\Theta(\Sigma_{v\in\SV}(d(v))^2)$.
%Given a temporal graph $\SG=(\SV, \SE, \ST)$, Algorithm~\ref{alg:track_graphlet} takes $\Theta(\Sigma_{v\in\SV}(d(v))^2)$ times for counting the number of instances of all graphlets in $\SG$.
\end{theorem}
\begin{proof}
Since the number of nodes forming each graphlet instance is a constant, finding the graphlet corresponding to a given instance and updating the corresponding count (lines \ref{alg:track_graphlet:line:if_instance}-\ref{alg:track_graphlet:line:decrease} and \ref{alg:track_graphlet:line:increase}) take $O(1)$ time. Thus, the time complexity of processing each incoming edge $e_i=u\rightarrow v$ is that of computing the union of the neighbors of $u$ and $v$ (line~\ref{alg:track_graphlet:line:neighbors}), which is $\Theta(\DTUi+\DTVi)$.
Hence, the total complexity is $\Theta(\sum_{e_i=u \rightarrow v \in E} (\DTUi+\DTVi) = \Theta(\sum_{v\in \SV}(d(v))^2)$.
\end{proof}
\begin{lemma} \label{lem:time:optimality} 
The number of instances of all graphlets in a snapshot $\SGT$ is $\Theta(\Sigma_{v\in\SVT}(\DTV)^2)$.
\end{lemma}
\begin{proof}
Given a snapshot $\SGT=(\SVT, \SET)$, for each node $v\in \SVT$, if we count the instances of all graphlets that consist of $v$ and its two neighbors, then the count of such instances is $\Theta((\DTV)^2)$ for each node $v$, and since $\DTV\geq 1$ for every node $v$, the total count $C$ is $\Theta(\Sigma_{v\in\SVT}(\DTV)^2)$. 

\smallsection{Lower Bound:} Since each graphlet instance, which consists of three nodes, is counted at most three times, $C$ is at most three times the number of instances of all graphlets in $\SGT$.
In other words, the number of instances of all graphlets is at least $1/3$ of $C$, and thus it is $\Omega(\Sigma_{v\in\SVT}(\DTV)^2)$.

\smallsection{Upper Bound:} In each graphlet instance, there exists at least one center node, who composes the graphlet together with its neighbors. Thus, each instance is counted at least once, and thus $C$ is at least the number of instances of all graphlets in $\SGT$.
In other words, the number of instances of all graphlets is at most $C$, and thus it is $O(\Sigma_{v\in\SVT}(\DTV)^2)$.
\end{proof}

As seen in Table~\ref{tab:graphlet_evolution}, the dynamics of the ratios depend on the domains of the graphs, as summarized in Observation~\ref{obs:graphlet_evolve}.

\noindent\fbox{%
        \parbox{\columnwidth}{%
        \vspace{-2mm}
        \begin{observation} \label{obs:graphlet_evolve}
            The dynamics in the distributions of graphlet instances in graphs from the same domain share some commonalities.
            \begin{itemize}
                \item Instances of graphlet 4 are more dominant in the citation graphs than other graphs. 
                \item Graphlets with many edges (e.g., graphlets 8, 12, and 13) account for a larger fraction in email/message networks than in other networks. 
                \item The fraction of graphlet 1 increases over time only in the online Q/A graphs.
            \end{itemize}
        \end{observation}
        \vspace{-2mm}
        }%
    }
%\vspace{0.5mm}



\noindent However, the dynamics are not exactly the same within domains. For example, while graphlets 1, 2, and 4 are dominant compared to other graphlets in all citation graphs, the ratios among them vary greatly in different graphs.

We also notice a consistent difference between the dynamics in real-world graphs and those in randomized graphs (see Section~\ref{sec:prelim:concept}), as summarized in Observation~\ref{obs:graphlet_evolve:random}.

\vspace{0.5mm}
\noindent\fbox{%
        \parbox{0.98\columnwidth}{%
        \vspace{-2mm}
        \begin{observation} \label{obs:graphlet_evolve:random}
            The ratios of graphlet instances change more linearly in randomized graphs than in real-world graphs.
        \end{observation}
        \vspace{-2mm}
        }%
    }
\vspace{0.5mm}


\begin{table}[t]
\caption{\label{tab:linearity} 
The non-linearity of the ratios of graphlet instances over time in real-world graphs and randomized graphs. 
We describe in Section~\ref{section:graph:time} how the non-linearity is measured.
The lower the non-linearity is, the more linear the change of the ratio of the corresponding graphlet instances is.
Note that
the ratios of graphlet instances change more linearly in randomized graphs than in real-world graphs.
}
\resizebox{\columnwidth}{!}{
\begin{tabular}{|c|ccc|ccc|ccc|}
    \hline
    	  Dataset & HepPh & HepTh & Patent & EU & Enron & College & Math & Ask & Stack \\
    	  \hline
    	  real      &  0.0027   & 0.0080    & 0.0093 & 0.0107 & 0.0042 & 0.0095 & 0.0028 & 0.0038 & 0.0047 \\
    	  random    &  0.0003   & 0.0011    & 0.0000 & 0.0081 & 0.0017 & 0.0058 & 0.0007 & 0.0005 & 0.0001 \\
    \hline
\end{tabular}}
\end{table}
\noindent In order to numerically support this observation, we measure the non-linearity \cite{kroll1993theoretical, hsieh2008statistical} of the ratios of graphlet instances over time.
Specifically, we fit a linear regression model and a non-linear polynomial regression model to each time series in Table~\ref{tab:graphlet_evolution}, and then we measure the average absolute difference between the predicted values of the two models as the non-linearity of the time series.\footnote{We use the linearity test implemented in Analyse-it (Ver. 5.65) and select a cubic model as the non-linearity polynomial model, as suggested in the program. For computational efficiency, we measure the absolute difference at 1,000 evolution ratios sampled uniformly at equal intervals.}
Lastly, we average the non-linearity of all time-series from each graph and report the results in Table~\ref{tab:linearity}.
Note that non-linearity is significantly higher in real-world graphs than in corresponding randomized graphs. 
That is, the ratios of graphlet instances change more linearly in randomized graphs than in real-world graphs.
%We fit a linear regression model and a cubic polynomial regression model to the ratio of each graphlet and the non-linearity is measured as the difference between the predicted value of the two models. 

\begin{table*}[ht]
\vspace{-2mm}
\caption{\label{tab:gtg} Using graphlet transition graphs (GTGs) and characteristic profiles (CPs) from GTGs, we can accurately characterize the dynamics of local structures in real-world graphs.
The colors of edges in GTGs indicate their normalized weights.
Note that GTGs and CPs are particularly similar in real-world graphs from the same domains (Observation~\ref{obs:transition:domain}).}
\resizebox{\textwidth}{!}{
\begin{tabular}{c|ccc|c}
    \toprule
    	      & \multicolumn{3}{c|}{Graphlet transition graphs (GTGs)} & Characteristic profiles (CPs) \\
    \hline
    \parbox[t]{2mm}{\multirow{6}{*}{\rotatebox[origin=c]{90}{Citation}}} &  
        \raisebox{-.9\totalheight}{\includegraphics[width=0.2\textwidth]{FIG/transition_graph/hepph.pdf}} &
        \raisebox{-.9\totalheight}{\includegraphics[width=0.2\textwidth]{FIG/transition_graph/hepth.pdf}} & 
        \raisebox{-.9\totalheight}{\includegraphics[width=0.2\textwidth]{FIG/transition_graph/patent.pdf}} & 
        \raisebox{-.9\totalheight}{\includegraphics[width=0.3\textwidth]{FIG/transition_graph/significance_citation.pdf}} \\
    \hline
    \parbox[t]{2mm}{\multirow{6}{*}{\rotatebox[origin=c]{90}{Email/Message}}} &  
        \raisebox{-.9\totalheight}{\includegraphics[width=0.2\textwidth]{FIG/transition_graph/email-eu.pdf}} &
        \raisebox{-.9\totalheight}{\includegraphics[width=0.2\textwidth]{FIG/transition_graph/enron.pdf}} & 
        \raisebox{-.9\totalheight}{\includegraphics[width=0.2\textwidth]{FIG/transition_graph/college_msg.pdf}} & 
        \raisebox{-.9\totalheight}{\includegraphics[width=0.3\textwidth]{FIG/transition_graph/significance_email.pdf}} \\
    \hline
    \parbox[t]{2mm}{\multirow{6}{*}{\rotatebox[origin=c]{90}{Online Q/A}}} &  
        \raisebox{-.9\totalheight}{\includegraphics[width=0.2\textwidth]{FIG/transition_graph/mathoverflow.pdf}} &
        \raisebox{-.9\totalheight}{\includegraphics[width=0.2\textwidth]{FIG/transition_graph/askubuntu.pdf}} & 
        \raisebox{-.9\totalheight}{\includegraphics[width=0.2\textwidth]{FIG/transition_graph/stackoverflow.pdf}} & 
        \raisebox{-.9\totalheight}{\includegraphics[width=0.3\textwidth]{FIG/transition_graph/significance_qna.pdf}} \\
    \bottomrule
\end{tabular}}
\end{table*}



\subsection{Global Level 2. Graphlet Transitions}
\label{section:graph:transition}

In a temporal graph, an instance of a graphlet may transition to an instance of another graphlet due to new edges added to it.
In this subsection, we examine the counts of such transitions between graphlets to characterize the local dynamics in temporal graphs and also to make comparisons between them.

\smallsection{Graphlet Transition Graph:}
We define \textit{graphlet transition graphs} (GTGs) to encode transitions between graphlets.

\noindent\fbox{%
    \parbox{0.98\columnwidth}{%
    \vspace{-2mm}
    \begin{definition}[Graphlet transition graph] \label{defn:transition}
       A \textit{graphlet transition graph} (GTG) $G=(V,E,W)$ of a temporal graph $\SG$ is a static directed weighted graph where the nodes are graphlets and each edge indicates that the source graphlet is transformed into the destination graphlet by an edge added to $\SG$.
        The weight of edges is the number of occurrences of the corresponding transitions. 
        We use $W=\{w_1, \cdots, w_{|E|}\}$ to denote the edge weights.
    \end{definition}
    \vspace{-2mm}
    }%
 }
\vspace{0.5mm}
\\
\noindent Since we focus on the 13 graphlets in Figure~\ref{fig:graphlet_and_role}(a), a GTG consists of the 28 types of transitions between these graphlets.
In Table~\ref{tab:gtg}, we visualize the GTGs from the real-world graphs.
Algorithm~\ref{alg:count_graphlet} describes the computation of the edge weights of 
 a GTG. In a nutshell, for each edge in arrival order, we count the transitions caused by it.
%The Algorithm~\ref{alg:count_graphlet} describes how we compute the counts. 
Its time complexity is formalized in Theorem~\ref{thm:time:transition}.

\begin{theorem} \label{thm:time:transition} 
The time complexity of Algorithm~\ref{alg:count_graphlet} is $\Theta(\Sigma_{v\in\SV}(d(v))^2)$ = $\Theta($the number of instances of all graphlets in the last snapshot$)$.
%Given a temporal graph $\SG=(\SV, \SE, \ST)$, Algorithm~\ref{alg:track_graphlet} takes $\Theta(\Sigma_{v\in\SV}(d(v))^2)$ times for counting the number of instances of all graphlets in $\SG$.
\end{theorem}
\begin{proof}
We can prove the complexity of $\Theta(\Sigma_{v\in\SV}(d(v))^2)$ similarly to Theorem~\ref{thm:time:track}, and by Lemma~\ref{lem:time:optimality}, it is $\Theta($the number of instances of all graphlets in the last snapshot$)$.
\end{proof}


\begin{algorithm}[t]
\small
	\caption{Computing the Edge Weights of Graphlet Transition Graphs \label{alg:count_graphlet}} 
	\SetKwInOut{Input}{Input}
	\SetKwInOut{Output}{Output}
	\SetKwProg{Procedure}{Procedure}{}{}
	\SetKwFunction{update}{UPDATE}
	
	\Input{Temporal graph $\SG = (\SV, \SE, \ST)$}
	\Output{Edge weights of the graphlet transition graph of $\SG$}
	
	%$\ST\leftarrow \varnothing $ \\
	%$\SD\leftarrow \varnothing $ \\
	
	Initialize all edge weights to zero \\
	Initialize $\SE$ to an empty set \\
%	Let $\SW$ as an array with size of 28. \\
	\For{\upshape\textbf{each }edge $e_i=u\rightarrow v$ in arrival order}{
	   % let directed edge $e_i$  
        \For{\upshape\textbf{each} $w_1$ $\in$ neighbors$(u)$ {$\setminus$ $\{v\}$} }{
            \update{$u, v, w_1$}
        }	
        \For{\upshape\textbf{each} {$w_2$ $\in$ neighbors$(v)$ $\setminus\{$neighbors$(u)$ $\cup$ $u\}$}}{
            \update{$u, v, w_2$}
        }
        add $u\rightarrow v$ to $\SE$
	}
	
	\textbf{return} the edge weights
	
    \Procedure{\update{$u, v, w$}}{ 
        \If{$u$, $v$, and $w$ form a graphlet instance}{
            prev $\leftarrow$ graphlet of the instance $(u, v, w)$ without $u\rightarrow v$ \\
            next $\leftarrow$ graphlet of the instance $(u, v, w)$ with $u\rightarrow v$ \\
            $i$ $\leftarrow$  index of the graphlet transition from prev to next \\
            increase the weight of the edge $i$ (i.e., $w_{i}$) by $1$
%            $\SD$[prev] $\leftarrow \SD$[prev] - 1 \\
        }
    }
\end{algorithm}

\smallsection{Characteristic Profile (CP):} 
We characterize the evolution of local structure in a graph $\SG$ using the significance of edge weights in its GTG $G=(V,E,W)$. 
In order to measure the significance, we follow the steps in \cite{milo2004superfamilies} for measuring the significance of each graphlet itself.
To this end, we construct the graphlet transition graph $\tilde{G}$ of a randomized graph $\SGR$.
Then, we measure the significance $SP_i$ of each edge weight $w_i$ in $G$ as follows:
\begin{equation}\label{eq1}
    SP_i := \frac{w_i - \tilde{w}_i}{w_i+\tilde{w}_i + \epsilon},
\end{equation}
where $\tilde{w}_i$ is the corresponding edge weight in $\tilde{G}$, and $\epsilon$ is a constant, which we fix to $4$. For $\tilde{w}_i$, we generate 50 instances of randomized graphs and we use the average edge weights in them. Lastly, we normalize each significance as follows:
\begin{equation}\label{eq2}
    CP_{i} := {SP_{i}}/{\sqrt{\Sigma_{i=1}^{|E|} SP_{i}^2}}.
\end{equation} 
We characterize the evolution of local structures in $\SG$ using the vector of the normalized significances (i.e., $[CP_1,\cdots,CP_{|E|}]$), which we call  \textit{characteristic profile (CP)}.
%We characterize the evolution of local structures in a graph $\SG$ using the vector of the normalized significances (i.e., $[CP_1,\cdots,CP_{28}]$), which we call the \textit{characteristic profile (CP)}.

%\begin{equation}\label{eq2}
%    CP_{i} := {SP_{i}}/{\sqrt{\Sigma_{i=1}^{28} SP_{i}^2}}.
%\end{equation} 

%We characterize the evolution of local structures in a graph $\SG$ using the vector of the normalized significances (i.e., $[CP_1,\cdots,CP_{28}]$), which we call the \textit{characteristic profile (CP)}.

\begin{figure}[t]
    \vspace{-3mm}
     \centering
     \begin{subfigure}{0.155\textwidth}
         \includegraphics[width=\textwidth]{FIG/signal/node_signal_2.pdf}
         \caption{$d_\theta = 2$}
     \end{subfigure}
     \begin{subfigure}{0.155\textwidth}
        \includegraphics[width=\textwidth]{FIG/signal/node_signal_4.pdf}
        \caption{$d_\theta = 4$}
     \end{subfigure}
     \begin{subfigure}{0.155\textwidth}
        \includegraphics[width=\textwidth]{FIG/signal/node_signal_8.pdf}
        \caption{$d_\theta = 8$}
     \end{subfigure}
     \caption{\label{fig:trend_signal} 
     \label{fig:signal_degree}
     Example signals from the local structures of nodes regarding their future importance. The ratios of some node roles (e.g., node roles 2 and 4) at nodes monotonically increase with respect to the future in-degrees of the nodes. The ratios are rescaled so that their maximum values are the same.}
  %   The example for local structural signals of nodes about their degree centrality. The higher the in-degree of predicted node $d_\theta$, the higher the number of signals.\red{TODO}}
\end{figure}

\smallsection{Comparison between CPs:}
We plot the CPs of the considered real-world graphs in Table~\ref{tab:gtg}, and high levels of similarity are observed within domains.
We numerically measure the similarity between CPs using the Pearson correlation coefficients, and the results are shown in Figure~\ref{fig:domain_correlation}(a).
The correlation coefficients are particularly high between graphs from the same domain, and specifically the domains can be classified with $97.2\%$ accuracy if we use the best threshold of the correlation coefficient ($0.58$).
The results demonstrate that CPs accurately characterize the evolution of local structures.
Our observations are summarized in Observation~\ref{obs:transition:domain}.

% \begin{algorithm}[t]
% \small
% 	\caption{Computing the Edge Weights of Graphlet Transition Graphs \label{alg:count_graphlet}} 
% 	\SetKwInOut{Input}{Input}
% 	\SetKwInOut{Output}{Output}
% 	\SetKwProg{Procedure}{Procedure}{}{}
% 	\SetKwFunction{update}{UPDATE}
	
% 	\Input{Temporal graph $\SG = (\SV, \SE, \ST)$}
% 	\Output{Edge weights of the graphlet transition graph of $\SG$}
	
% 	%$\ST\leftarrow \varnothing $ \\
% 	%$\SD\leftarrow \varnothing $ \\
	
% 	Initialize all edge weights to zero. \\
% 	Initialize $\SE$ to an empty set. \\
% %	Let $\SW$ as an array with size of 28. \\
% 	\For{\upshape\textbf{each }edge $e_i=u\rightarrow v$ in arrival order}{
% 	   % let directed edge $e_i$  
%         \For{\upshape\textbf{each } $w_1$ $\in$ neighbors$(u)$ {$\setminus$ $\{v\}$} }{
%             \update{$u, v, w_1$}
%         }	
%         \For{\upshape\textbf{each } {$w_2$ $\in$ neighbors$(v)$ $\setminus\{$neighbors$(u)$ $\cup$ $u\}$}}{
%             \update{$u, v, w_2$}
%         }
%         add $u\rightarrow v$ to $\SE$
% 	}
	
% 	\textbf{return} the edge weights
	
%     \Procedure{\update{$u, v, w$}}{ 
%         \If{$u$, $v$, and $w$ form a graphlet instance}{
%             prev $\leftarrow$ graphlet of the instance $(u, v, w)$ without $u\rightarrow v$ \\
%             next $\leftarrow$ graphlet of the instance $(u, v, w)$ with $u\rightarrow v$ \\
%             $i$ $\leftarrow$  index of the graphlet transition from prev to next \\
%             increase the weight of the edge $i$ (i.e., $w_{i}$) by $1$
% %            $\SD$[prev] $\leftarrow \SD$[prev] - 1 \\
%         }
%     }
% \end{algorithm}


\vspace{0.5mm}
\noindent\fbox{%
        \parbox{0.98\columnwidth}{%
        \vspace{-2mm}
        \begin{observation} \label{obs:transition:domain}
            The evolution patterns of local structures are similar
            in real-world graphs from the same domains.
        \end{observation}
        \vspace{-2mm}
        }%
    }

%with the best thresholds of similarity, the classification accuracy is $97.2\%$

%We follow \cite{milo2004superfamilies} for the above steps 

%We call the signifiance of each edge weight (i.e., $[\SP_1,\cdots, \SP_2]$) a 
%Table~\ref{tab:transition_significance} shows the significances from the considered real-world graphs, and their significance profile. The temporal graphs within the same domain share similar significance profile vectors. 

%$\SW_{rand}[i]$ is the average weight of same edge in the graphlet transition graphs obtained from $\SGR$ and we fix $\epsilon$ to 4. Table~\ref{tab:transition_significance} shows graphlet transition graphs of $\SG$ and their significance profile. The temporal graphs within the same domain share similar significance profile vectors. 

% \smallsection{:} 

% We 
% To compare the similarity of temporal graphs, we normalize the significance of all edges and concatenate them. Since the number of edges in $G$ is twenty-eight, we obtain characteristic profile (CP) as follow:





\smallsection{Comparison with Other Methods:} We evaluate three other graph characterization methods, as we evaluate ours in the right above paragraph.
In Figure~\ref{fig:domain_correlation}(b), we provide the correlation coefficients between the CPs obtained from the count of the instances of each graphlet~\cite{milo2004superfamilies}. 
%We calculate the classification accuracy as described in the above paragraph.
Note that the email/message graphs (blue) and the online Q/A graphs (green) are not distinguished clearly. Numerically, with the best threshold of correlation coefficient ($0.95$), the classification accuracy is $83.3\%$.

We also compute the similarity between the considered real-world graphs using Graphlet-orbit Transition (GoT) \cite{aparicio2018graphlet} and Orbit Temporal Agreement (OTA) \cite{aparicio2018graphlet}, which are also based on transitions between graphlets (see Section~\ref{section:relwork} for details). 
Our way of characterization has the following major advantages over them:
\begin{itemize}
    \item \textbf{(1) Speed:} Empirically, GoT and OTA are up to $10\times$ slower than our method, as shown in Appendix~\ref{sec:appendix:compare_got_ota}.
The time complexity of them is proportional to the sum of the counts of graphlet instances in all used snapshots, while the time complexity of Algorithm~\ref{alg:count_graphlet} is proportional only the to the count of graphlet instances in the last snapshot (Theorem~\ref{thm:time:transition}).
    \item \textbf{(2) Space Efficiency:} GoT and OTA run out of memory in the two largest graphs (Patent and Stackoverflow), as shown in Appendix~\ref{sec:appendix:compare_got_ota}, while our method does not. They need to store all graphlet instances in each considered snapshot for comparison with those in the next snapshot, while Algorithm~\ref{alg:count_graphlet} maintains only the latest snapshot without having to store graphlet instances.
    \item \textbf{(3) Characterization Accuracy:} The best classification accuracies computed using the considered real-world graphs (except for Patent and Stackoverflow for which GoT and OTA run out of memory) are $81.0\%$ (GoT) and $85.7\%$ (OTA), which is lower than our classification accuracy ($97.2\%$). Detailed results are given in Appendix~\ref{sec:appendix:compare_got_ota}. Note that GoT and OTA approximate the counts of transitions between graphlets based on a small number of snapshots, while Algorithm~\ref{alg:count_graphlet} exactly counts the transitions.
\end{itemize}
% \red{
%By Lemma~\ref{lem:time:optimality}, counting the instances of graphlets in a snapshot $\SGT$ takes at least $\Omega(\Sigma_{v\in\SVT}(\DTV)^2)$ times. To count the graphlet instances in each snapshot, they have to repeat the counting process with the number of snapshots used in the algorithm. 
% \textbf{(2) Lossless graphlet transition information:} For the above reason, GoT and OTA consider only the transitions between a few snapshots. However, in real-world temporal graphs, the future edges are more likely to be attached at the recently arrived edges rather than older edges, which is called a temporal locality~\cite{lee2020temporal}. Therefore, one-by-one transition information can be lost in GoT and OTA, which is successfully preserved in the proposed method. 

In summary, \textbf{our way of characterizing temporal graphs using GTGs distinguishes the domains of temporal graphs most accurately with the accuracy of $97.2\%$.} The accuracies of the other methods are $83.3\%$, $81.0\%$, and $85.7\%$.




\section{Node Level Analysis} \label{section:node}
In this section, we study how local structures around nodes are related to their future importance.
Then, we enhance the predictability of future node centrality using the relations.


\subsection{Patterns}
We characterize the local structures of nodes using node roles and examine their relation to the nodes' future centrality.
%We investigate that the local structural signals of nodes in their early stage regarding their future centrality.

\begin{table}[t]
\vspace{-3mm}
\caption{\label{tab:signal_degree}
The absolute value of the Spearman's rank correlation coefficients between node role ratios and future centralities (averaged over all node roles and all datasets for each centrality measure) and each value of the threshold $d_\theta$.
As the number of node neighbors increases (i.e., $d_\theta$ increases), the local-structural signals about future centralities become stronger (i.e., the absolute values increase).
%Specifically, as $d_\theta$ increases, the ratios of more node (or edge) roles monotonically increase (INC) or decrease (DEC) with respect to future centrality.
}
\resizebox{\columnwidth}{!}{
    \begin{tabular}{|c|c|c|c|c|c|}
        \hline
        $d_\theta$ & Degree & Betweenness & Closeness & PageRank & Edge Betweenness \\
        \hline
        2        & 0.640  & 0.697       & 0.682     & 0.663    & 0.546            \\
        4        & 0.721  & 0.723       & 0.712     & 0.704    & 0.558            \\
        8        & 0.816  & 0.793       & 0.759     & 0.701    & 0.599            \\
        \hline
    \end{tabular}}
\end{table}

\smallsection{Local Structures of Nodes:}
Given a temporal graph $\SG$, we characterize the local structure of each node $v$ in their early stage by measuring the ratio of each node role at $v$ in the snapshot at time $t$ when the in-degree of $v$ first reaches a threshold $d_\theta$.
That is, each node $v$ is represented as a $30$-dimensional vector whose $i$-th is $\SNTi(v)/(\sum_{j=1}^{30}\SNTj(v))$  (see Section~\ref{sec:prelim:concept} for $\SNTi(v)$).

%the ratio of node role $i$ at $v$ in the snapshot.

\smallsection{Future Importance of Nodes:}
Given a temporal graph $\SG$, as future importance of each node, we measure its in-degree, node betweenness centrality~\cite{freeman1977set}, closeness centrality~\cite{bavelas1950communication}, and PageRank~\cite{page1999PageRank} in the last snapshot of $\SG$.
Based on each centrality measure, we divide the nodes in $\SG$ into six groups (Group 1: top 50-100\%, Group 2: top 30-50\%, Group 3: top 10-30\%, Group 4: top 5-10\%, Group 5: top 1-5\%, and Group 6: top 0-1\%).


\begin{figure*}[t]
    \vspace{-3mm}
    \centering
		\includegraphics[width=0.4\textwidth]{FIG/signal/degree.pdf}
		\includegraphics[width=0.4\textwidth]{FIG/signal/between.pdf} \\
            \vspace{-2mm}
		\includegraphics[width=0.4\textwidth]{FIG/signal/closeness.pdf}
            \includegraphics[width=0.4\textwidth]{FIG/signal/PageRank.pdf} \\
            \vspace{-3mm}
	\caption{\label{fig:node_signals} 
     The Spearman's rank correlation coefficient  between node role ratios (when nodes have in-degree four, i.e., $d_\theta$ = 4)
     and future node centralities.
     The darker a cell is, the larger the absolute value of the corresponding coefficient is. Note that the absolute values of most coefficients are significantly greater than $0$.
     }
\end{figure*}

\smallsection{Finding Signals:}
For each group, we average the ratio vectors of the nodes in the group. 
Figure~\ref{fig:signal_degree} shows some averaged ratios when in-degree is used as the centrality measure. Note that the ratios of node roles 2 and 4 monotonically grow as future centrality increases, regardless of $d_\theta$ values. That is, the ratios of node roles 2 and 4 give a consistent signal regarding the nodes' future in-degree.

In Figure~\ref{fig:node_signals}, we report the Spearman's rank correlation coefficient~\cite{zwillinger1999crc} between each averaged ratio and the future centralities of nodes (specifically, the above group numbers between 1 and 6).
%The coefficient value is especially large when node betweenness is used as the centrality measure. 
We also report in Table~\ref{tab:signal_degree} the absolute value of the coefficients (averaged over all node roles and all datasets) for each centrality measure and each value of the threshold $d_\theta$. Note that the average values are significantly greater than $0$ and specifically around $0.7$; and they
increase as $d_\theta$ increases, as summarized in Observation~\ref{obs:node:signal}.
%For each group, we average the ratio vectors of the nodes in the group.
%Figure~\ref{fig:signal_degree} shows some averaged ratios when in-degree is used as the centrality measure.
%
%Notably, the number of such roles with consistent signals increase as $d_\theta$ increases, as also shown in Table~\ref{tab:num_signals}.
%Our findings are summarized in Observation~\ref{obs:node:signal}.


\noindent\fbox{%
        \parbox{0.98\columnwidth}{
        \vspace{-2mm}
        \begin{observation}\label{obs:node:signal}
        In real-world graphs,
           the local structures of nodes in their early stage provide a signal regarding their future importance. The signals become stronger as nodes have more neighbors.
        \end{observation}
        % \begin{observation}\label{obs:node:signal:degree}
        %     Signals from the local structurals of nodes in their early stage are especially strong regarding their future betweenness, compared to other centrality measures.
        % \end{observation}
        \vspace{-2mm}
        }
     }
\vspace{0.5mm}


% Figure~\ref{fig:node_signals} shows all node roles whose ratios monotonically increase or decrease with respect to future centrality.
% The number of such roles is especially large when in-degree is used as the centrality measure. That is, Observation~\ref{obs:node:signal:degree} is made.
% \vspace{0.5mm}
% \noindent\fbox{%
%         \parbox{0.98\columnwidth}{
%         \vspace{-2mm}
%         \begin{observation}\label{obs:node:signal:degree}
%             Signals from the local structurals of nodes in their early stage are especially strong regarding their future in-degree, compared to other centrality measures.
%         \end{observation}
%         \vspace{-2mm}
%         }
%      }
% \vspace{0.5mm}

% \begin{figure}[t]
%     \centering
% 	\subfigure[$d_\theta = 2$]{
% 		\includegraphics[width=0.14\textwidth]{FIG/signal/node_signal_2.pdf}
% 	} 
% 	\subfigure[$d_\theta = 4$]{
% 		\includegraphics[width=0.14\textwidth]{FIG/signal/node_signal_4.pdf}
% 	} 
%  	\subfigure[$d_\theta = 8$]{
% 		\includegraphics[width=0.14\textwidth]{FIG/signal/node_signal_8.pdf}
% 	} \\
%         \vspace{-1mm}
% 	\caption{\label{fig:signal_degree} Example signals from the local structures of nodes regarding their future importance. The ratios of some node roles (e.g., node roles 2 and 4) at nodes monotonically increase with respect to the future in-degrees of the nodes. The ratios are rescaled so that their maximum values are the same. }
% \end{figure}




%an example of local structural signals of nodes where each color indicates the type of node roles. 
%The average ratio of node roles shows monotonic ascending trends regard as nodes' future centrality. The example also shows that the number of the signals increases as $d_\theta$ increases.



%Each node belongs to certain graphlets and its local-structural role can be represented as ratio of node roles it take in.
%Even though the nodes are in early stage, we find that their future centrality is highly related with local structural role.
%For local structures, we use the roles of each node when its in-degree reaches a certain value $d_\theta$.
%For future importances, we use  centrality. We divide the nodes into 6 groups for each centrality so that they have similar centrality values. Then, we calculate the average distribution of node roles when nodes' in-degree becomes $d_\theta$. 



\subsection{Prediction} \label{subsection:node_prediction}

Based on the observations above, we predict the future centrality of nodes using the counts of their roles at them in their early stage. 

% In this subsection, we predict the nodes' future centrality using their roles. They enhance the accuracy of simpler local structural features called Node Prominence Profile (NPP) \cite{yang2014predicting}. Also, we show the higher $d_\theta$ of the predicted node is, the higher accuracy nodes have. Moreover, they are complementary to global features: the number of nodes and edges.

\smallsection{Problem Formulation:} 
We formulate the prediction problem as a classification problem, as described in Problem~\ref{problem:node}.

\vspace{0.5mm}

\noindent\fbox{%
        \parbox{0.98\columnwidth}{%
        \vspace{-2mm}
        \begin{problem}[Node Centrality Prediction] \label{problem:node}
        \noindent\begin{itemize}
            \item \textbf{Given:} the snapshot $\SGTv$ of the input graph when the in-degree of a node $v$ first reaches $d_\theta$,
            \item \textbf{Predict:} whether the centrality of the node $v$ belongs to the top $20\%$ in the last snapshot of $\SG$.
        \end{itemize}
        \end{problem}
        \vspace{-2mm}
        }%
    }
\vspace{0.5mm}

\noindent As the centrality measure, we use in-degree, betweenness centrality, closeness centrality, and PageRank.
As $d_\theta$, we use $2$, $4$, or $8$. 


\smallsection{Input Features:} 
For each node $v$, we consider the snapshot $\SGT$ of the input graph $\SG$ when the in-degree of $v$ first reaches $d_\theta$. That is, $t=\tv$ and $\SGT=\SGTv$.
Then, we extract the following sets of input features for $v$: 
\bit
    \item \textbf{Local-NR:} The count of each node role at $v$ in $\SGT$. That is, $[\SNTone(v),$ $\SNTtwo(v), \cdots, \SNTthirty(v)]$ (see Section~\ref{sec:prelim:concept} for $\SNTi(v)$).
    \item \textbf{Local-NPP \cite{yang2014predicting}:} In $\SGT$, we compute (1) the count of triangles at $v$, (2) the count of wedges centered at $v$, (3) the count of wedges ended at $v$.
    \item \textbf{Global-Basic:} Counts of nodes and edges in the snapshot.
    \item \textbf{Global-NR:} We compute the $30$-dimensional vector whose $i$-th entry $\SNTi(v)/(\sum_{j=1}^{30}\SNTj(v))$
    is the ratio of each node role at $v$ in $\SGT$.
    Then, we standardize (i.e., compute the $z$-score of) the role ratio vector using the mean and standard deviation from the role ratio vectors (in $\SGT$) of all nodes with degree $d_\theta$ in $\SGT$. The features in Local-NR are also included.
    \item \textbf{Global-NPP~\cite{yang2014predicting}:} In $\SGT$, we compute (1) the number of edges not incident to $v$ and (2) the number of non-adjacent node pairs where one is a neighbor of $v$ and the other is neither a neighbor of $v$ nor $v$ itself.
%    product of the degree of $v$ and the number of disconnected noes. (b) the product of 
    The features in Local-NPP are also included.
    \item \textbf{ALL:}  All of \textbf{Global-NR}, \textbf{Global-NPP}, and \textbf{Global-Basic}.
    %from the the ratio vector of $v$ and we divide the result by the standard deviation. 
    %The combined features of Local-NR and their global information. We add the relative ratio of the node roles in each node using the z-score method. Suppose that the in-degree of node $v$ is $d_\theta$ in $\SGT$. Let $\SVT_\theta$ be the set of nodes whose degree is also $d_\theta$ in $\SGT$ and let $\tilde{m}_i^{(t)}(v)$ be ratio of node role $i$ at a node $v$ in $\SGT$. We calculate the mean $\mu_i$ and standard deviation $\sigma_i$ of $\tilde{m}_i^{(t)}(v)$ for every node role $i$ of nodes in $\SVT_\theta$. Then, we add the global feature Global-NR $i$ for every node role $i$ calculated as below:
        % (ratio(instance) - mu) / (sigma + epsilon)
    % \begin{equation*}
    %     \text{Global-NR } i := \frac{\tilde{m}_i^{(t)}(v) - \mu_i}{1+\sigma_i}
    % \end{equation*} 
%    The count of node role instances described in Figure~\ref{fig:graphlet_and_role}(c).
\eit

Note that we categorize the above sets into global and local depending on whether global information in $\SGT$ (i.e., the number of all nodes in $\SGT$) is used or only local information at $v$ is used.

\smallsection{Prediction Method:}  
As the classifier, we use the \textit{random forest} model from the Scikit-learn library. The model has 30 decision trees with a maximum depth of 10.


\smallsection{Evaluation Method:} 
We use $80\%$ of the nodes for training and the remaining $20\%$ for testing. 
We evaluate the predictive performance in terms of F1-score, accuracy, and \textit{Area Under the ROC curve} (AUROC). A higher value indicates better prediction performance.

%Prediction performance (AUROC) on future node centrality prediction

\smallsection{Result:}
%The predictive performances are summarized in Tables~\ref{tab:pred-node-feature} and \ref{tab:pred-node-degree}. 
Table~\ref{tab:pred-node-overall} shows the predictive performance from each set of input features when $d_\theta=2$, and Table~\ref{tab:pred-node-overall-degree} shows how the performance depends on the in-degree threshold $d_\theta$.
In the tables, we report the mean of each prediction performance over $10$ runs in the 7 datasets in Section~\ref{section:datasets} except for the two largest ones (i.e., Patent and Stackoverflow).
From the results, we draw the following observations.

\vspace{0.5mm}

\noindent\fbox{%
        \parbox{0.98\columnwidth}{%
        \vspace{-2mm}
        \begin{observation} \label{obs:node:pred_nr_npp}
            Among local features, 
            the counts of node roles at each node (\textbf{Local-NR}) are more informative than (\textbf{Local-NPP}) for future importance prediction.
        \end{observation}
        \begin{observation}\label{obs:node:pred_all}
            The considered sets of features are complementary to each other. Using them all (\textbf{ALL}) leads to the best predictive performance in most cases.
        \end{observation}
        \begin{observation}\label{obs:node:degree}
            As nodes have more neighbors, their future importance can be predicted more accurately.
        \end{observation}
        \vspace{-2mm}
        }%
    }

%The prediction performances with standard deviations in each graph can be founded in Appendix~\ref{sec:appendix:pred_details}.


\begin{table}
% 	\vspace{-2mm}
	\centering
	\caption{\label{tab:pred-node-overall} 
    F1-score, accuracy, and AUROC on the task of predicting future node importance when $d_\theta=2$ averaged over the 7 considered real-world graphs.
	Among local features, using \textbf{Local-NR} yields better performance than using \textbf{Local-NPP} in all settings. 
	Using  \textbf{ALL} leads to the best performance in most cases, indicating that the considered sets of features are complementary to each other. Detailed results on each dataset can be found in Appendix~\ref{sec:appendix:node:details}.
	}
	%Second, Global-NR has a little higher performance than Global-NPP, and it also has higher than Global-basic. Lastly, ALL has the highest predictive performance which indicates the proposed feature is complementary to other features. \red{todo} }   
	%\scalebox{0.95}{
	\resizebox{\columnwidth}{!}{
		\begin{tabular}{|c|c|c|c|c|c|c|}
		\hline
			Target & \multicolumn{3}{c|}{Degree} & \multicolumn{3}{c|} {Betweenness}  \\ 
			\hline
			Measure & F1-score & Accuracy & AUROC & F1-score & Accuracy & AUROC \\
			\hline
			Local-NR     & 0.39 & 0.69 & 0.68 & 0.59 & 0.83 & 0.82 \\
			Local-NPP    & 0.38 & 0.68 & 0.64 & 0.58 & 0.81 & 0.79 \\
			\hline
			Global-NR    & \bf{0.57} & \bf{0.74} & \bf{0.78} & 0.64 & 0.84 & 0.85 \\
			Global-NPP   & \bf{0.57} & 0.73      & 0.77      & 0.64 & 0.84 & 0.85 \\
			Global-Basic & 0.50      & 0.72      & 0.73      & 0.24 & 0.73 & 0.67 \\
			\hline
			ALL          & \bf{0.57}      & \bf{0.74} & \bf{0.78} & \bf{0.65} & \bf{0.85} & \bf{0.86}\\
			\hline
			\hline
			Target & \multicolumn{3}{c|}{Closeness} & \multicolumn{3}{c|} {PageRank}  \\ 
			\hline
			Measure & F1-score & Accuracy & AUROC & F1-score & Accuracy & AUROC \\
			\hline
			Local-NR     & 0.51 & 0.76 & 0.78 & 0.42 & 0.73 & 0.73  \\
			Local-NPP    & 0.43 & 0.70 & 0.69 & 0.37 & 0.69 & 0.67 \\
			\hline
			Global-NR    & 0.68 & 0.82 & 0.87 & 0.54 & \bf{0.75} & \bf{0.79} \\
			Global-NPP   & 0.66 & 0.80 & 0.85 & 0.54 & 0.74 & 0.78 \\
			Global-Basic & 0.59 & 0.75 & 0.79 & 0.47 & 0.71 & 0.74 \\
			\hline
			ALL          & \bf{0.69} & \bf{0.83} & \bf{0.88} & \bf{0.56} & \bf{0.75} & \bf{0.79} \\
			\hline
		\end{tabular}
	}
\end{table}

\smallsection{Feature Importance:} Additionally, we measure the importance of each feature in the set \textbf{ALL} using \textit{Gini-importance}~\cite{loh2011classification}, and we report the top five important features in Table~\ref{tab:feature_importance} in Appendix~\ref{sec:appendix:feature_importance}.

\vspace{0.5mm}

\noindent\fbox{%
        \parbox{0.98\columnwidth}{%
        \vspace{-2mm}
        \begin{observation}
            Strong predictors vary depending on centrality measures. For example, for betweenness centrality, the counts of node roles as bridges (i.e., \textbf{Local NR-4} and \textbf{Global NR-4}) are strong.
        \end{observation}
        
        \vspace{-2mm}
        }%
    }
\begin{table}[t]
 	\vspace{-2mm}
	\centering
	\caption{\label{tab:pred-node-overall-degree} 
    Average F1-score, accuracy, and AUROC on the task of predicting future node importance depending on $d_\theta$ (i.e., in-degree of nodes when their input features are extracted). 
	The overall performance improves with respect to $d_\theta$ in most cases. That is,
	as nodes have more neighbors, their future importance can be predicted more accurately.
 Detailed results on each dataset can be found in Appendix~\ref{sec:appendix:node:details}.
	}
	\resizebox{\columnwidth}{!}{
	%\scalebox{0.95}{
		\begin{tabular}{|c|c|c|c|c|c|c|}
		\hline
			Target & \multicolumn{3}{c|}{Degree} & \multicolumn{3}{c|} {Betweenness} \\ 
			\hline
			Measure & F1-score & Accuracy & AUROC & F1-score & Accuracy & AUROC \\
			\hline
			ALL $(d_\theta=2)$ & 0.59 & 0.74 & 0.78 & 0.65 & 0.85 & 0.86 \\
			ALL $(d_\theta=4)$ & 0.69 & 0.78 & 0.79 & 0.73 & 0.83 & 0.87 \\
			ALL $(d_\theta=8)$ & 0.80 & 0.81 & 0.86 & 0.82 & 0.85 & 0.90 \\
			\hline
			\hline
			Target & \multicolumn{3}{c|}{Closeness} & \multicolumn{3}{c|} {PageRank} \\ 
			\hline
			Measure & F1-score & Accuracy & AUROC & F1-score & Accuracy & AUROC \\
			\hline
			ALL $(d_\theta=2)$ & 0.69 & 0.83 & 0.88 & 0.55 & 0.75 & 0.79  \\
			ALL $(d_\theta=4)$ & 0.78 & 0.83 & 0.89 & 0.73 & 0.77 & 0.80 \\
			ALL $(d_\theta=8)$ & 0.86 & 0.88 & 0.92 & 0.85 & 0.85 & 0.83 \\
			\hline
		\end{tabular}
	}
\end{table}

\begin{figure}[t]
    \vspace{-3mm}
     \centering
     \includegraphics[width=0.45\textwidth]{FIG/signal/edge-betweenness.pdf} \\
     \vspace{-2mm}
     \caption{\label{fig:edge_signal}
     The Spearman's rank correlation coefficient between edge role ratios 
     (when endpoints have in-degree $4$ in total, i.e., $d_\theta$ = 4)
     and future edge centralities.
     The darker a cell is, the larger the absolute value of the corresponding coefficient is. Note that the absolute values of many coefficients are significantly greater than $0$, while they tend to be smaller than those in Figure~\ref{fig:node_signals}.
     % Edge roles whose ratios at edges in their early stage (specifically, edges whose endpoints have in-degree $4$ in total) monotonically increase (colored red) or decrease (colored blue) with respect to future edge importance.
     % Such edge roles are rare, compared to node roles with similar properties (see Figure~\ref{fig:node_signals}).
     }
     
%     The signal of local structural patterns of edges. Red color represents monotonic increasing patterns and blue color represents monotonic decreasing patterns when $d_\theta$=4.}
\end{figure}

\section{Edge Level Analysis}\label{section:edge}

In this section, we investigate the signal of local structures of each edge regarding their future centrality, and based on the signal, we predict the future importance of edges.  

We generally follow the procedures in Section~\ref{section:node}, except for the following differences: (a) we examine the ratios of edge roles at each edge $u\rightarrow v$ when the sum of the in-degrees of $u$ and $v$ becomes $d_\theta$, (b) we use edge betweenness centrality~\cite{freeman1977set} as the importance measure, (c) we formulate the problem of predicting future edge importance as described in Problem~\ref{problem:edge}, (d)
we extract feature sets \textbf{Local-ER} and \textbf{Global-ER} using the (relative) counts of edge roles at edges as we extract \textbf{Local-NR} and \textbf{Global-NR}, and (e) we union \textbf{Global-ER} and \textbf{Global-Basic} for $\textbf{ALL}$.

\vspace{0.5mm}

\noindent\fbox{%
        \parbox{0.98\columnwidth}{%
        \vspace{-2mm}
        \begin{problem}[Edge Centrality Prediction] \label{problem:edge}
        \noindent\begin{itemize}
            \item \textbf{Given:} the snapshot $\SGTe$ of the input graph when the sum of the in-degrees of the endpoints of each edge first reaches $d_\theta$,
            \item \textbf{Predict:} whether the centrality of each edge belongs to the top $20\%$ in the last snapshot of $\SG$.
        \end{itemize}
        \end{problem}
        \vspace{-2mm}
        }%
    }

\vspace{0.5mm}

From Figure~\ref{fig:edge_signal}, Table~\ref{tab:signal_degree}, and Table~\ref{tab:pred-edge-overall}, we draw the following observations.
%\red{From the results in Figure~\ref{fig:edge_signal}, Table~\ref{tab:num_signals}, and Table~\ref{tab:pred-edge-overall}, we draw the following observations.}


%drawing the following observations.

\noindent\fbox{%
        \parbox{0.98\columnwidth}{%
        \vspace{-2mm}
        \begin{observation}\label{obs:edge_signal}
        In real-world graphs, the signals from the local structures of edges in their early stage regarding their future importance are weaker, compared to the signals that from the local structures of nodes (see Figure~\ref{fig:edge_signal}).
        \end{observation}
        
        \begin{observation}\label{obs:edge_stronger}
        However, the signals become stronger as the edges are better connected, leading to better prediction performance (see Tables~\ref{tab:signal_degree} and~\ref{tab:pred-edge-overall}).
        \end{observation}
        
        \begin{observation}\label{obs:edge_pred}
           The features from edge roles (\textbf{Local-ER} and \textbf{Global-ER}) are more informative than simple global statistics (\textbf{Global-Basic}) for future importance prediction (see Table~\ref{tab:pred-edge-overall}).
        \end{observation}
        \vspace{-2mm}
        }%
     }
% \smallsection{Signal of Initial Edge:} 
% We also investigate local structure signals of edges at their early stage. Although those signals are not significantly strong than those of node roles, we find below observations.





\begin{table}[t]
	\vspace{-3mm}
	\centering
	\caption{\label{tab:pred-edge-overall} 
    F1-score, accuracy, and AUROC on the task of predicting future edge importance averaged over the 7 considered real-world graphs.
	Using edge role-based features (\textbf{Local-ER} and \textbf{Global-ER}) yields better performance than using \textbf{Global-Basic} in most settings. 
	The overall performance improves with respect to $d_\theta$.
	That is, as edges are better connected, their future importance is predicted more accurately.
     Detailed results on each dataset can be found in Appendix~\ref{sec:appendix:edge:details}
	}
	%Second, Global-NR has a little higher performance than Global-NPP, and it also has higher than Global-basic. Lastly, ALL has the highest predictive performance which indicates the proposed feature is complementary to other features. \red{todo} }   
	\scalebox{0.90}{
		\begin{tabular}{|c|c|c|c|}
		\hline
			Target & \multicolumn{3}{c|}{Edge betweenness}  \\ \hline
            Measure & F1-Score & Accuracy   &  AUROC \\ \hline
            Local-ER $(d_\theta = 2)$       & 0.45 & 0.78 & 0.76 \\
            Global-ER $(d_\theta = 2)$      & 0.47 & 0.81 & 0.78 \\
            Global-Basic $(d_\theta = 2)$   & 0.42 & 0.79 & 0.73 \\
            ALL $(d_\theta = 2)$            & 0.50 & 0.80 & 0.75 \\
			\hline
			ALL $(d_\theta = 2)$            & 0.50 & 0.80 & 0.75 \\
			ALL $(d_\theta = 4)$            & 0.53 & 0.82 & 0.84 \\
			ALL $(d_\theta = 8)$            & 0.52 & 0.85 & 0.85 \\
			\hline
		\end{tabular}
	}
\end{table}


%%%%%%%%%%%%%%%%%%%%%%%%%%%%%%%%%%%%%%%%%%%%%%%%%%%%%%%%%%%%%%%%%%%%%%%%%%%%%%%%%%%%%%%%%%%%%%%%%%%%%%
%%%%%%%%%%%%%%%%%%%%%%%%%%%%%%%%%%%%%%%%%%%%%%%%%%%%%%%%%%%%%%%%%%%%%%%%%%%%%%%%%%%%%%%%%%%%%%%%%%%%%%

\section{Related Work}\label{section:relwork}

Previous studies on temporal graph analysis are largely categorized into (a) designing algorithms for streaming graphs~\cite{lee2020temporal, eswaran2018spotlight, liben2007link, mcgregor2014graph}, (b) discovering temporal patterns in graphs~\cite{leskovec2005graphs, akoglu2008rtm, beyer2010mechanistic, akoglu2010structure, bahulkar2016analysis}, and (c) generating graphs with realistic dynamics~\cite{barabasi1999emergence, leskovec2010kronecker, akoglu2008rtm}. This work belongs to the second category.

Studies in this category have revealed (a) universal temporal patterns, such as densification~\cite{leskovec2005graphs}, shrinking diameter~\cite{leskovec2005graphs}, and power-laws between principle eigenvalues and edge counts~\cite{akoglu2008rtm}; and (b) domain-specific patterns in hyperlink networks~\cite{broder2011graph}, metabolic networks (e.g., biochemical reactions and  protein interactions)~\cite{beyer2010mechanistic}, communication networks (e.g., phone calls and
texts)~\cite{hidalgo2008dynamics,akoglu2010structure}, and friendship networks~\cite{bahulkar2016analysis}.

In particular, for the analysis of local structures, the concept of graphlets \cite{prvzulj2007biological} (i.e., the sets of isomorphic small subgraphs with a predefined number of nodes) has been extended to temporal graphs. The extensions, which are called \textit{temporal network motifs}, have multiple variants.
%Temporal network motifs are proposed in many works.
Kovanen et al.~\cite{kovanen2011temporal}  defined them as sets of temporal subgraphs with a fixed number of nodes that are (a) topologically equivalent, (b) temporally equivalent (specifically, relative orders of constituent edges are identical), (c)  consecutive (specifically, constituent edges are consecutive for every node), and (d) temporally local (specifically, arrival times of consecutive edges are close enough).
Hulovaty et al.~\cite{hulovatyy2015exploring} ignores (c); and Paranjape et al.~\cite{paranjape2017motifs} 
ignores (c) and relaxes (d) by restricting only the time difference between the first edge and the last edge.
Note that all these notions focus on temporally local subgraphs, and thus they are suitable only for analyzing short-term dynamics. 

For long-term dynamics in local structures,
David et al.~\cite{aparicio2018graphlet} proposed Graph-orbit Transition (GoT) and Orbit Temporal Agreement (OTA), which characterize the dynamic of a temporal graph by approximately counting the number of transitions between node roles.
However, due to high computational overhead, only a small fraction of snapshots can be compared for estimating the counts of transitions, and as a result, their  characterization powers are significantly weaker than our characterization method using GTGs (see Section~\ref{section:graph:transition}). Recall that our method counts ``every'' transition between graphlets, and it is still significantly faster than GOT and OTA (see Section~\ref{sec:appendix:compare_got_ota} in Appendix).

%considers induced subgraphs that the time difference between the first edge and the last edge is within a time window $\Delta$ as instances of the temporal motif. 
%However, due to the time threshold, they are suitable for analyzing the short-term changes of graphs but not for long-term changes in local structures.

For predicting the future in-degree of nodes, Yang et al.~\cite{yang2014predicting} proposed to use five features obtained from graphlets with three nodes (see Section~\ref{subsection:node_prediction} for descriptions).
As shown empirically, our proposed features tend to provide better prediction performance than these five features, and more importantly, they are complementary to each other.
Faisal and Milenkovi{\'c}~\cite{faisal2014dynamic} aimed to detect aging-related nodes, whose topological properties (e.g,. graphlet counts) change highly over time, in the gene expression process. 

On the algorithmic aspect, a great number of algorithms have been developed for the problem of counting the instances of each graphlet, which is also known as the subgraph counting problem. %, has been studied also on the algorithmic aspect.
As suggested in a survey on subgraph counting \cite{ribeiro2021survey},  subgraph-counting algorithms are largely categorized into exact counting \cite{milo2002network, schiller2015stream, ortmann2017efficient, ahmed2017graphlet} and approximated counting  \cite{wernicke2005faster, aslay2018mining}.
Those in the first category are further categorized into enumeration-based approaches \cite{milo2002network, schiller2015stream}, matrix-based approaches \cite{ortmann2017efficient}, and decomposition-based approaches \cite{ahmed2017graphlet}. 
Algorithm~\ref{alg:track_graphlet} belongs to the first subcategory, and it achieves the optimal time complexity achievable by those in this subcategory, as discussed in the beginning of Section~\ref{section:graph:time}.
It is adapted from StreaM \cite{schiller2015stream}, which maintains the counts of the instances of $4$-node undirected graphlets in a fully dynamic graph stream (i.e., a stream of edge insertions and deletions). %In this subcategory, for static graphs, algorithms that are able to enumerate the instances of larger graphlets are available \cite{ribeiro2010g}.
% In this subcategory, 
% For static graphs, fast algorithms (e.g., \cite{ribeiro2010g}) for counting the instances of graphlets of largert size


% \red{
% Ortmann et al. \cite{ortmann2017efficient} presented an algorithm for counting the instances of graphlets with 4 nodes in an undirected graph and graphlets with 3 nodes in directed graph. 
% They look into the non-induced subgraphs, build a linear equation for fast counting, and apply the clique counting algorithm.
% Ribeiro et al. \cite{ribeiro2010g} suggested the data structure for counting subgraphs instances called g-trie, which is a prefix tree for graphs that each node represents a different graph and the graph of parent node is a subgraph of its child node.}

%\blue{capture the changes of local structure of each node-pair among past snapshots, make a  predict whether this node-pair is connected or not in the next snapshot. }

%They compute the node centrality of each node in both its early and late stages, and they measure the correlation between them. 

%\red{Need to Check}  define a concept of node prominence profile based on triadic closure and preferential attachment, and they use it for predicting degree centrality of nodes at their future. They only consider the appearance of nodes at initial stage and after evolution, and they overlook their grown-up progress, which we also investigate.

%However, it is difficult to calculate various centralites whenever new edges arrive}


% \smallsection{Graphlets in Temporal Graphs:}
% Graphlets are a set of connected isomorphic subgraphs. The methods for counting the occurrence of graphlets in temporal graphs have proposed various perspectives: graphlet counting~\cite{lee2020temporal}, frequent pattern mining~\cite{raj2018mining, zhang2020seasonal}, and network motifs~\cite{paranjape2017motifs, hulovatyy2015exploring, kovanen2011temporal, lee2021thyme}.





% 다양한 종류의 graphlet in temporal graph


% \smallsection{Graph Patterns} The pattern of real-world graphs differs from the randomly generated graph. Many works suggest graph generating methods to generate random graphs containing similar graph properties of real-world graph \cite{leskovec2010kronecker, erdHos1960evolution, newman2003structure}. Analyzing graph patterns is a basic step in graph mining, such as graph attack \cite{albert2000error}, anomaly detection \cite{akoglu2015graph}, and so on.

% \smallsection{Network motif} Network motifs are a useful tool for understanding local structural patterns in graphs \cite{milo2002network, lee2020hypergraph, purohit2018temporal, dey2017motif, benson2016higher}. After the concept of the network motif is introduced in \cite{milo2002network}, network motifs are widely used in the various applications: community detection \cite{arenas2008motif}, graph embedding \cite{dareddy2019motif2vec}, and deep learning \cite{sankar2017motif}.  The concept of network motif has been extended various area: temporal graph \cite{paranjape2017motifs}, bipertite graph \cite{borgatti1997network}, and hypergraph \cite{lee2020hypergraph}. 

% \smallsection{Graph Importance} Each node and edge take roles in a graph, and various centrality measures have been proposed to measure their importance in a graph in various perspectives. PageRank \cite{page1999PageRank}, HITs \cite{kleinberg1998authoritative}, closeness, betweenness centrality is a representative measure for node, and edge-betweenness centrality is famous measure for edge. To obtain the important information in the graph, many works identify node importance in the complex system \cite{yang2016efficient, hu2015identifying}, predict node importance \cite{yang2014predicting, park2019estimating}, and measure them as the criteria of the performance.

%%%%%%%%%%%%%%%%%%%%%%%%%%%%%%%%%%%%%%%%%%%%%%%%%%%%%%%%%%%%%%%%%%%%%%%%%%%%%%%%%%%%%%%%%%%%%%%%%%%%%%
%%%%%%%%%%%%%%%%%%%%%%%%%%%%%%%%%%%%%%%%%%%%%%%%%%%%%%%%%%%%%%%%%%%%%%%%%%%%%%%%%%%%%%%%%%%%%%%%%%%%%%

%%%%%%%%%%%%%%%%%%%%%%%%%%%%%%%%%%%%%%%%%%%%%%%%%%%%%%%%%%%%%%%%%%%%%%%%%%%%%%%%%%%%%%%%%%%%%%%%%%%%%%
%%%%%%%%%%%%%%%%%%%%%%%%%%%%%%%%%%%%%%%%%%%%%%%%%%%%%%%%%%%%%%%%%%%%%%%%%%%%%%%%%%%%%%%%%%%%%%%%%%%%%%

\section{Conclusion and Future Work}\label{sec:conclusion}
In this paper, we report an approach which adopts reinforcement learning algorithms to solve the problem of robustness-guided falsification of CPS systems. We implement our approach in a prototype tool and conduct preliminary evaluations with a widely adopted CPS system. The evaluation results show that our method can reduce the number of episodes to find the falsifying input. As a future work, we plan to extend the current work to explore more reinforcement learning algorithms and evaluate our methods on more CPS benchmarks. 
\bibliographystyle{ACM-Reference-Format}
\bibliography{reference}

%\clearpage
\appendix
\renewcommand{\thesection}{\Alph{section}.\arabic{section}}
\setcounter{section}{0}
\appendix

{

}




\section{\textsc{Begin} Annotation Protocol} 
\label{sec:annotationprotocol}
Each worker was given a document, previous turn in a conversation and a generated response (either by \textsc{T5}, \textsc{GPT2}, \textsc{DoHA} or \CTRL{}).  They were asked to evaluate the response as either fully attributable, not attributable, or too generic to be informative. They also were provided with multiple examples with explanations for each category. The exact instructions were as follows:
\begin{mdframed}[leftmargin=0pt,rightmargin=0pt]
\small
Which of these best describes the highlighted utterance?
\begin{itemize}
    \item[$\circ$] {Generic: This utterance is uninformative (too bland or not specific enough to be sharing any new information) }
    \item[$\circ$] {Contains \emph{any} unsupported Information:
This utterance is sharing information that cannot be fully verified by the document.  It may include  false information, unverifiable information, and personal stories/opinions.}
    \item[$\circ$] {\emph{All} information is \emph{fully} supported by the document: This utterance contains only information that is fully supported by the document.}
\end{itemize}
\end{mdframed}



\section{Implementations}
\label{sec:hyperparam}

\paragraph{GPT2, T5} We implement these models using the TensorFlow Huggingface Transformers library \cite{wolf-etal-2020-transformers}. During training, we use the Adam optimizer \cite{DBLP:journals/corr/KingmaB14} with Dropout \cite{srivastava2014dropout} on a batch size of $32$ with a learning rate of $6.25 \times 10^{-5}$ that is linearly decayed. The maximum dialogue history length is set to $3$ utterances. The model early-stops at epoch \{6, 10, 10\} respectively for \textsc{WoW}, \textsc{CMU-DoG} and \textsc{TopicalChat}.

\paragraph{\CTRL{}} We reproduce the results from \cite{rashkin-etal-2021-increasing}, following the training details in that paper.

\paragraph{DoHA} We use the code and the pre-trained model on \textsc{CMU-DoG} that are publicly available by the authors at their Github's account \footnote{\url{https://bit.ly/3bBup2M}}. For \textsc{WoW} and \textsc{TopicalChat}, we follow closely the authors' training procedure described in \citep{prabhumoye-etal-2021-focused} and we train two models on both datasets.  


For each dataset, we save the best model based on the validation set. We use nucleus sampling with $p=0.9$. 


\section{Model-Based Metrics}
\label{app:implementation-details-metrics}




\paragraph{Semantic Similarity Models}
We use BERTScore version 0.3.11. with the
DeBERTa-xl-MNLI model \cite{he2020deberta}, which is the recommended model as of the time of investigation. For BLEURT, We use the recommended
BLEURT-20 checkpoint \cite{pu-etal-2021-learning}. For BARTScore, we use the latest publicly available checkpoint (accessed March 2022) from \url{https://github.com/neulab/BARTScore}. 






\begin{table*}[t]
	\centering
	\caption{\label{tab:node_f1}
	F1-score on the task of predicting future node importance when $d_\theta=2$.
	}
	%\scalebox{0.80}{
	\resizebox{\textwidth}{!}{
		\begin{tabular}{|c|c|c|c|c|c|c|c|c|c|}
		\hline
			\multirow{2}{*}{Centrality}  & \multirow{2}{*}{Feature} & \multicolumn{2}{c|}{Citation Networks} & \multicolumn{3}{c|}{Email/Message Networks} & \multicolumn{2}{c|}{Online Q/A Networks} & \multirow{2}{*}{\bf{Average}} \\\cline{3-9}
			&  & HepPh & Hepth & Email-EU & Email-Enron & Message-College & Mathoverflow & Askubuntu & \\ 
			\hline
			\multirow{6}{*}{Degree}
			& Local-NR     & 0.11$\pm$0.008       & 0.20$\pm$0.014      & 0.36$\pm$0.092      & 0.68$\pm$0.007      & 0.27$\pm$0.027      & 0.38$\pm$0.017      & \f{0.73}$\pm$0.002      & 0.39 \\
			& Local-NPP    & 0.12$\pm$0.013       & 0.19$\pm$0.016      & 0.40$\pm$0.102      & 0.60$\pm$0.011      & 0.27$\pm$0.040      & 0.36$\pm$0.016      & \s{0.72}$\pm$0.003      & 0.38 \\ \cline{2-10}
			& Global-NR    & \s{0.52}$\pm$0.013   & 0.56$\pm$0.013      & \s{0.51}$\pm$0.077  & \f{0.79}$\pm$0.004  & 0.37$\pm$0.025      & \f{0.52}$\pm$0.021  & 0.70$\pm$0.004          & \f{0.57} \\
			& Global-NPP   & 0.51$\pm$0.010       & \s{0.57}$\pm$0.018  & \s{0.51}$\pm$0.063  & 0.76$\pm$0.005      & \f{0.40}$\pm$0.030  & \f{0.52}$\pm$0.022  & 0.70$\pm$0.006          & \f{0.57} \\
			& Global-basic & 0.36$\pm$0.014       & 0.38$\pm$0.024      & 0.44$\pm$0.073      & \s{0.77}$\pm$0.008  & 0.34$\pm$0.050      & \s{0.51}$\pm$0.022  & 0.71$\pm$0.003          & \s{0.50} \\\cline{2-10}
			& ALL          & \f{0.53}$\pm$0.010   & \f{0.58}$\pm$0.013  & \f{0.52}$\pm$0.043  & \f{0.79}$\pm$0.008  & \s{0.38}$\pm$0.041  & \f{0.52}$\pm$0.026  & 0.70$\pm$0.005          & \f{0.57} \\ 
			\hline
			\multirow{6}{*}{Betweenness}
			& Local-NR     & 0.59$\pm$0.011       & 0.88$\pm$0.009      & 0.34$\pm$0.063      & 0.49$\pm$0.011      & 0.34$\pm$0.033      & \s{0.74}$\pm$0.011  & \s{0.73}$\pm$0.007  & 0.59 \\
			& Local-NPP    & 0.58$\pm$0.010       & 0.87$\pm$0.006      & 0.35$\pm$0.076      & 0.45$\pm$0.010      & 0.36$\pm$0.073      & \s{0.74}$\pm$0.011  & \s{0.73}$\pm$0.007  & 0.58 \\ \cline{2-10}
			& Global-NR    & \f{0.64}$\pm$0.007   & \f{0.90}$\pm$0.005  & 0.48$\pm$0.089      & \f{0.62}$\pm$0.014  & \s{0.38}$\pm$0.047  & \f{0.75}$\pm$0.008  & \f{0.74}$\pm$0.006  & \s{0.64} \\
			& Global-NPP   & \s{0.62}$\pm$0.010   & \s{0.89}$\pm$0.006  & \s{0.49}$\pm$0.037  & \s{0.58}$\pm$0.019  & \f{0.40}$\pm$0.034  & \f{0.75}$\pm$0.010  & \f{0.74}$\pm$0.006  & \s{0.64} \\
			& Global-basic & 0.01$\pm$0.003       & 0.27$\pm$0.028      & 0.40$\pm$0.079      & 0.36$\pm$0.017      & 0.25$\pm$0.038      & 0.32$\pm$0.013      & 0.10$\pm$0.013      & 0.24 \\ \cline{2-10}
			& ALL          & \f{0.64}$\pm$0.007   & \f{0.90}$\pm$0.007  & \f{0.53}$\pm$0.052  & \f{0.62}$\pm$0.016  & \s{0.38}$\pm$0.045  & \f{0.75}$\pm$0.010  & \f{0.74}$\pm$0.007  & \f{0.65}\\ 
			\hline
			\multirow{6}{*}{Closeness}
			& Local-NR     & 0.49$\pm$0.010       & 0.53$\pm$0.010      & 0.28$\pm$0.077      & 0.69$\pm$0.008      & 0.24$\pm$0.034      & 0.58$\pm$0.014      & \s{0.75}$\pm$0.005  & 0.51 \\
			& Local-NPP    & 0.37$\pm$0.015       & 0.51$\pm$0.014      & 0.31$\pm$0.067      & 0.46$\pm$0.010      & 0.25$\pm$0.038      & 0.43$\pm$0.017      & 0.66$\pm$0.006      & 0.43 \\\cline{2-10}
			& Global-NR    & \s{0.84}$\pm$0.006   & \s{0.75}$\pm$0.007  & 0.47$\pm$0.046      & \f{0.83}$\pm$0.008  & \s{0.38}$\pm$0.055  & \f{0.69}$\pm$0.024  & \f{0.81}$\pm$0.002  & \s{0.68} \\
			& Global-NPP   & 0.83$\pm$0.008       & 0.74$\pm$0.007      & \s{0.52}$\pm$0.047  & \s{0.76}$\pm$0.008  & \f{0.39}$\pm$0.022  & \s{0.64}$\pm$0.019  & 0.71$\pm$0.005      & 0.66 \\
			& Global-basic & 0.82$\pm$0.004       & 0.70$\pm$0.010      & 0.44$\pm$0.074      & 0.72$\pm$0.010      & 0.33$\pm$0.010      & 0.51$\pm$0.015      & 0.60$\pm$0.004      & 0.59 \\\cline{2-10}
			& ALL          & \f{0.85}$\pm$0.008   & \f{0.76}$\pm$0.008  & \f{0.53}$\pm$0.043  & \f{0.83}$\pm$0.007  & 0.36$\pm$0.051      & \f{0.69}$\pm$0.022  & \f{0.81}$\pm$0.003  & \f{0.69} \\ 
			\hline
			\multirow{6}{*}{PageRank}
			& Local-NR     & 0.44$\pm$0.013       & 0.15$\pm$0.018      & 0.42$\pm$0.069      & 0.64$\pm$0.008      & 0.25$\pm$0.038      & 0.46$\pm$0.016      & 0.58$\pm$0.009      & 0.42 \\
			& Local-NPP    & 0.41$\pm$0.012       & 0.18$\pm$0.017      & 0.43$\pm$0.086      & 0.39$\pm$0.009      & 0.25$\pm$0.040      & 0.41$\pm$0.013      & 0.53$\pm$0.008      & 0.37 \\\cline{2-10}
			& Global-NR    & \f{0.64}$\pm$0.014   & 0.41$\pm$0.015      & \s{0.49}$\pm$0.078  & \f{0.74}$\pm$0.006  & 0.35$\pm$0.056      & \f{0.53}$\pm$0.019  & \f{0.63}$\pm$0.009  & \s{0.54} \\
			& Global-NPP   & \f{0.64}$\pm$0.012   & \s{0.43}$\pm$0.015  & \f{0.55}$\pm$0.046  & \s{0.65}$\pm$0.011  & \f{0.38}$\pm$0.047  & \s{0.52}$\pm$0.017  & \s{0.61}$\pm$0.007  & \s{0.54} \\
			& Global-basic & \s{0.63}$\pm$0.010   & 0.31$\pm$0.023      & 0.41$\pm$0.054      & \s{0.65}$\pm$0.007  & 0.28$\pm$0.035      & 0.49$\pm$0.010      & 0.54$\pm$0.008      & 0.47 \\\cline{2-10}
			& ALL          & \f{0.64}$\pm$0.009   & \f{0.44}$\pm$0.013  & \f{0.55}$\pm$0.035  & \f{0.74}$\pm$0.006  & \s{0.37}$\pm$0.030  & \f{0.53}$\pm$0.020  & \f{0.63}$\pm$0.008  & \f{0.56} \\ 
			\hline
   
		\hline
		\end{tabular}
	}
\end{table*}

\begin{table*}[t]
	\centering
	\caption{\label{tab:node_accuracy}
	Accuracy on the task of predicting future node importance when $d_\theta=2$.
	}
	\resizebox{\textwidth}{!}{
         %\scalebox{0.8}{
		\begin{tabular}{|c|c|c|c|c|c|c|c|c|c|}
		\hline
			\multirow{2}{*}{Centrality}  & \multirow{2}{*}{Feature} & \multicolumn{2}{c|}{Citation Networks} & \multicolumn{3}{c|}{Email/Message Networks} & \multicolumn{2}{c|}{Online Q/A Networks} & \multirow{2}{*}{\bf{Average}} \\\cline{3-9}
			&  & HepPh & Hepth & Email-EU & Email-Enron & Message-College & Mathoverflow & Askubuntu & \\ 
			\hline
			\multirow{6}{*}{Degree}
			& Local-NR     & 0.71$\pm$0.008       & 0.72$\pm$0.006      & \s{0.80}$\pm$0.024  & 0.62$\pm$0.006      & \f{0.77}$\pm$0.019  & 0.66$\pm$0.008      & 0.58$\pm$0.003      & 0.69 \\
			& Local-NPP    & 0.71$\pm$0.010       & 0.72$\pm$0.007      & 0.79$\pm$0.031      & 0.55$\pm$0.006      & 0.75$\pm$0.020      & 0.65$\pm$0.016      & 0.58$\pm$0.002      & 0.68 \\ \cline{2-10}
			& Global-NR    & \f{0.76}$\pm$0.007   & \s{0.77}$\pm$0.007  & \f{0.82}$\pm$0.024  & \f{0.77}$\pm$0.004  & \s{0.76}$\pm$0.017  & \f{0.68}$\pm$0.014  & \f{0.61}$\pm$0.004  & \f{0.74} \\
			& Global-NPP   & \s{0.75}$\pm$0.003   & \s{0.77}$\pm$0.007  & \s{0.80}$\pm$0.023  & \s{0.75}$\pm$0.004  & \f{0.77}$\pm$0.019  & \f{0.68}$\pm$0.012  & \f{0.61}$\pm$0.004  & \s{0.73} \\
			& Global-basic & 0.73$\pm$0.006       & 0.74$\pm$0.008      & 0.77$\pm$0.028      & \s{0.75}$\pm$0.006  & 0.74$\pm$0.019      & \s{0.67}$\pm$0.011  & \s{0.60}$\pm$0.003  & 0.72 \\\cline{2-10}
			& ALL          & \f{0.76}$\pm$0.008   & \f{0.78}$\pm$0.006  & \f{0.82}$\pm$0.019  & \f{0.77}$\pm$0.005  & \s{0.76}$\pm$0.016  & \f{0.68}$\pm$0.014  & \f{0.61}$\pm$0.004  & \f{0.74} \\ 
			\hline
			\multirow{6}{*}{Betweenness}
			& Local-NR     & 0.79$\pm$0.005       & \s{0.93}$\pm$0.005  & 0.78$\pm$0.017      & 0.82$\pm$0.006      & \f{0.76}$\pm$0.019  & \f{0.86}$\pm$0.005  & \f{0.90}$\pm$0.003  & 0.83 \\
			& Local-NPP    & 0.79$\pm$0.004       & 0.92$\pm$0.004      & 0.75$\pm$0.028      & 0.81$\pm$0.005      & 0.65$\pm$0.032      & \f{0.86}$\pm$0.005  & \f{0.90}$\pm$0.003  & 0.81 \\ \cline{2-10}
			& Global-NR    & \f{0.81}$\pm$0.004   & \f{0.94}$\pm$0.003  & \s{0.81}$\pm$0.017  & \f{0.84}$\pm$0.007  & \s{0.75}$\pm$0.016  & \f{0.86}$\pm$0.004  & \f{0.90}$\pm$0.002  & \s{0.84} \\
			& Global-NPP   & \s{0.80}$\pm$0.005   & \f{0.94}$\pm$0.003  & 0.79$\pm$0.021      & \s{0.83}$\pm$0.007  & \f{0.76}$\pm$0.020  & \f{0.86}$\pm$0.004  & \f{0.90}$\pm$0.003  & \s{0.84} \\
			& Global-basic & 0.71$\pm$0.005       & 0.73$\pm$0.010      & 0.75$\pm$0.033      & 0.77$\pm$0.007      & 0.70$\pm$0.019      & \s{0.67}$\pm$0.008  & \s{0.77}$\pm$0.004  & 0.73 \\ \cline{2-10}
			& ALL          & \f{0.81}$\pm$0.003   & \f{0.94}$\pm$0.004  & \f{0.82}$\pm$0.018  & \f{0.84}$\pm$0.008  & \s{0.75}$\pm$0.019  & \f{0.86}$\pm$0.004  & \f{0.90}$\pm$0.003  & \f{0.85}\\ 
			\hline
			\multirow{6}{*}{Closeness}
			& Local-NR     & 0.76$\pm$0.003       & 0.76$\pm$0.005      & 0.78$\pm$0.027      & 0.75$\pm$0.005      & \f{0.76}$\pm$0.015  & 0.72$\pm$0.006      & \s{0.78}$\pm$0.003  & 0.76 \\
			& Local-NPP    & 0.73$\pm$0.003       & 0.75$\pm$0.006      & 0.75$\pm$0.032      & 0.63$\pm$0.007      & 0.74$\pm$0.016      & 0.65$\pm$0.010      & 0.64$\pm$0.004      & 0.70 \\\cline{2-10}
			& Global-NR    & \f{0.91}$\pm$0.003   & \f{0.86}$\pm$0.004  & 0.80$\pm$0.018      & \f{0.85}$\pm$0.006  & \s{0.75}$\pm$0.019  & \f{0.77}$\pm$0.011  & \f{0.82}$\pm$0.002  & \s{0.82} \\
			& Global-NPP   & \s{0.90}$\pm$0.005   & \s{0.85}$\pm$0.002  & \s{0.81}$\pm$0.013  & \s{0.80}$\pm$0.007  & \f{0.76}$\pm$0.011  & \s{0.73}$\pm$0.010  & 0.73$\pm$0.003      & 0.80 \\
			& Global-basic & \s{0.90}$\pm$0.003   & 0.82$\pm$0.004      & 0.76$\pm$0.033      & 0.77$\pm$0.008      & 0.73$\pm$0.024      & 0.66$\pm$0.009      & 0.64$\pm$0.003      & 0.75 \\\cline{2-10}
			& ALL          & \f{0.91}$\pm$0.004   & \f{0.86}$\pm$0.004  & \f{0.82}$\pm$0.020  & \f{0.85}$\pm$0.005  & 0.75$\pm$0.017      & \f{0.77}$\pm$0.011  & \f{0.82}$\pm$0.002  & \f{0.83} \\ 
			\hline
			\multirow{6}{*}{PageRank}
			& Local-NR     & \s{0.76}$\pm$0.007   & 0.72$\pm$0.009      & 0.80$\pm$0.020      & \s{0.74}$\pm$0.006  & \s{0.75}$\pm$0.017  & 0.66$\pm$0.011      & 0.65$\pm$0.006      & 0.73 \\
			& Local-NPP    & 0.74$\pm$0.007       & 0.72$\pm$0.007      & 0.77$\pm$0.023      & 0.65$\pm$0.018      & 0.73$\pm$0.023      & 0.63$\pm$0.005      & 0.62$\pm$0.006      & 0.69 \\\cline{2-10}
			& Global-NR    & \f{0.81}$\pm$0.005   & \f{0.75}$\pm$0.007  & \s{0.81}$\pm$0.022  & \f{0.80}$\pm$0.005  & \s{0.75}$\pm$0.015  & \f{0.68}$\pm$0.010  & \f{0.67}$\pm$0.006  & \f{0.75} \\
			& Global-NPP   & \f{0.81}$\pm$0.005   & \s{0.74}$\pm$0.007  & \f{0.82}$\pm$0.023  & \s{0.74}$\pm$0.011  & \f{0.76}$\pm$0.019  & \s{0.67}$\pm$0.008  & \s{0.66}$\pm$0.006  & \s{0.74} \\
			& Global-basic & \f{0.81}$\pm$0.004   & 0.73$\pm$0.008      & 0.75$\pm$0.021      & \s{0.74}$\pm$0.006  & 0.70$\pm$0.025      & 0.65$\pm$0.007      & 0.59$\pm$0.006      & 0.71 \\\cline{2-10}
			& ALL          & \f{0.81}$\pm$0.005   & \f{0.75}$\pm$0.004  & \s{0.81}$\pm$0.017  & \f{0.80}$\pm$0.006  & \s{0.75}$\pm$0.014  & \f{0.68}$\pm$0.010  & \f{0.67}$\pm$0.005  & \f{0.75} \\ 
			\hline
   
		\hline
		\end{tabular}
	}
\end{table*}

\section{Detailed Results of Future Node Importance Prediction}
\label{sec:appendix:node:details}

In Tables~\ref{tab:node_f1}-\ref{tab:node_auroc_degree},
we provide the average predictive performances and standard deviations over $10$ runs on the task of predicting future node centrality in each real-world graph in terms of several evaluation metrics. 
The detailed experimental settings can be found in Section~\ref{subsection:node_prediction} of the main paper. 

\section{Detailed Results of Future Edge Importance Prediction}
\label{sec:appendix:edge:details}
In Tables~\ref{tab:edge_f1_score}-\ref{tab:edge_auroc},
we provide the average predictive performances and standard deviations over $10$ runs on the task of predicting future node centrality in each real-world graph in terms of several evaluation metrics. 
The detailed experimental settings can be found in Section~\ref{section:edge} of the main paper. 


\begin{table*}[t]
\vspace{-2mm}
	\centering
	\caption{\label{tab:node_auroc}
	AUROC on the task of predicting future node importance when $d_\theta=2$.
	}
	\resizebox{\textwidth}{!}{
	%\scalebox{0.8}{
		\begin{tabular}{|c|c|c|c|c|c|c|c|c|c|}
		\hline
			\multirow{2}{*}{Centrality}  & \multirow{2}{*}{Feature} & \multicolumn{2}{c|}{Citation Networks} & \multicolumn{3}{c|}{Email/Message Networks} & \multicolumn{2}{c|}{Online Q/A Networks} & \multirow{2}{*}{\bf{Average}} \\\cline{3-9}
			&  & HepPh & Hepth & Email-EU & Email-Enron & Message-College & Mathoverflow & Askubuntu & \\ 
			\hline
			\multirow{6}{*}{Degree}
			& Local-NR     & 0.69$\pm$0.006       & 0.70$\pm$0.006      & 0.80$\pm$0.037      & 0.67$\pm$0.007      & 0.69$\pm$0.020      & 0.65$\pm$0.012      & 0.58$\pm$0.006      & 0.68 \\
			& Local-NPP    & 0.64$\pm$0.005       & 0.67$\pm$0.007      & 0.75$\pm$0.032      & 0.56$\pm$0.007      & 0.64$\pm$0.025      & 0.63$\pm$0.012      & 0.56$\pm$0.004      & 0.64 \\ \cline{2-10}
			& Global-NR    & \f{0.81}$\pm$0.004   & \s{0.83}$\pm$0.005  & \f{0.85}$\pm$0.037  & \f{0.86}$\pm$0.005  & \s{0.73}$\pm$0.026  & \f{0.71}$\pm$0.013  & \f{0.65}$\pm$0.006  & \f{0.78} \\
			& Global-NPP   & \s{0.80}$\pm$0.002   & \s{0.83}$\pm$0.005  & \s{0.83}$\pm$0.032  & 0.83$\pm$0.005      & \f{0.74}$\pm$0.026  & \f{0.71}$\pm$0.012  & \f{0.65}$\pm$0.005  & \s{0.77} \\
			& Global-basic & 0.74$\pm$0.004       & 0.74$\pm$0.007      & 0.82$\pm$0.035      & \s{0.84}$\pm$0.006  & 0.68$\pm$0.037      & \s{0.68}$\pm$0.009  & \s{0.62}$\pm$0.005  & 0.73 \\\cline{2-10}
			& ALL          & \f{0.81}$\pm$0.004   & \f{0.84}$\pm$0.006  & \f{0.85}$\pm$0.036  & \f{0.86}$\pm$0.005  & \s{0.73}$\pm$0.025  & \f{0.71}$\pm$0.014  & \f{0.65}$\pm$0.006  & \f{0.78} \\ 
			\hline
			\multirow{6}{*}{Betweenness}
			& Local-NR     & 0.84$\pm$0.006       & \s{0.98}$\pm$0.002  & 0.79$\pm$0.033      & 0.80$\pm$0.007      & 0.70$\pm$0.029      & \s{0.83}$\pm$0.010  & \s{0.82}$\pm$0.005  & 0.82 \\
			& Local-NPP    & 0.83$\pm$0.006       & \s{0.98}$\pm$0.003  & 0.73$\pm$0.031      & 0.68$\pm$0.008      & 0.65$\pm$0.041      & 0.82$\pm$0.010      & 0.81$\pm$0.006      & 0.79 \\ \cline{2-10}
			& Global-NR    & \f{0.87}$\pm$0.005   & \f{0.99}$\pm$0.002  & \s{0.81}$\pm$0.030  & \f{0.87}$\pm$0.007  & \s{0.71}$\pm$0.032  & \f{0.86}$\pm$0.008  & \f{0.86}$\pm$0.005  & \s{0.85} \\
			& Global-NPP   & \s{0.85}$\pm$0.007   & \s{0.98}$\pm$0.002  & \s{0.81}$\pm$0.032  & \s{0.84}$\pm$0.008  & \f{0.72}$\pm$0.033  & \f{0.86}$\pm$0.008  & \f{0.86}$\pm$0.005  & \s{0.85} \\
			& Global-basic & 0.62$\pm$0.008       & 0.74$\pm$0.013      & 0.76$\pm$0.042      & 0.74$\pm$0.005      & 0.63$\pm$0.038      & 0.63$\pm$0.014      & 0.60$\pm$0.007      & 0.67 \\ \cline{2-10}
			& ALL          & \f{0.87}$\pm$0.006   & \f{0.99}$\pm$0.001  & \f{0.83}$\pm$0.024  & \f{0.87}$\pm$0.007  & \s{0.71}$\pm$0.033  & \f{0.86}$\pm$0.009  & \f{0.86}$\pm$0.005  & \f{0.86}\\ 
			\hline
			\multirow{6}{*}{Closeness}
			& Local-NR     & 0.79$\pm$0.005       & 0.79$\pm$0.006      & 0.76$\pm$0.047      & 0.84$\pm$0.006      & 0.68$\pm$0.026      & 0.76$\pm$0.009      & 0.84$\pm$0.002      & 0.78 \\
			& Local-NPP    & 0.74$\pm$0.005       & 0.78$\pm$0.008      & 0.72$\pm$0.042      & 0.65$\pm$0.009      & 0.61$\pm$0.022      & 0.63$\pm$0.007      & 0.70$\pm$0.005      & 0.69 \\\cline{2-10}
			& Global-NR    & \f{0.97}$\pm$0.002   & \s{0.93}$\pm$0.005  & \s{0.82}$\pm$0.035  & \f{0.93}$\pm$0.003  & \s{0.73}$\pm$0.024  & \f{0.83}$\pm$0.011  & \s{0.89}$\pm$0.002  & \s{0.87} \\
			& Global-NPP   & \s{0.96}$\pm$0.003   & 0.92$\pm$0.004      & \f{0.83}$\pm$0.028  & \s{0.88}$\pm$0.004  & \s{0.73}$\pm$0.026  & \s{0.79}$\pm$0.011  & 0.81$\pm$0.003      & 0.85 \\
			& Global-basic & 0.95$\pm$0.003       & 0.89$\pm$0.005      & 0.79$\pm$0.044      & 0.85$\pm$0.007      & 0.69$\pm$0.035      & 0.68$\pm$0.012      & 0.70$\pm$0.003      & 0.79 \\\cline{2-10}
			& ALL          & \f{0.97}$\pm$0.002   & \f{0.94}$\pm$0.005  & \s{0.82}$\pm$0.027  & \f{0.93}$\pm$0.004  & \f{0.75}$\pm$0.028  & \f{0.83}$\pm$0.010  & \f{0.90}$\pm$0.001  & \f{0.88} \\ 
			\hline
			\multirow{6}{*}{PageRank}
			& Local-NR     & 0.77$\pm$0.006       & 0.65$\pm$0.010      & 0.80$\pm$0.031      & 0.81$\pm$0.004      & \s{0.67}$\pm$0.011  & 0.69$\pm$0.015      & 0.70$\pm$0.006      & 0.73 \\
			& Local-NPP    & 0.75$\pm$0.004       & 0.63$\pm$0.008      & 0.77$\pm$0.031      & 0.63$\pm$0.010      & 0.62$\pm$0.024      & 0.64$\pm$0.009      & 0.67$\pm$0.006      & 0.67 \\\cline{2-10}
			& Global-NR    & \f{0.87}$\pm$0.005   & \s{0.78}$\pm$0.006  & \s{0.84}$\pm$0.035  & \f{0.88}$\pm$0.003  & \f{0.72}$\pm$0.029  & \f{0.71}$\pm$0.012  & \f{0.73}$\pm$0.005  & \f{0.79} \\
			& Global-NPP   & \s{0.86}$\pm$0.005   & 0.77$\pm$0.007      & \f{0.85}$\pm$0.028  & \s{0.82}$\pm$0.008  & \f{0.72}$\pm$0.018  & \s{0.70}$\pm$0.011  & \s{0.71}$\pm$0.006  & \s{0.78} \\
			& Global-basic & \s{0.86}$\pm$0.005   & 0.73$\pm$0.010      & 0.81$\pm$0.033      & 0.81$\pm$0.008      & 0.66$\pm$0.026      & 0.66$\pm$0.011      & 0.62$\pm$0.004      & 0.74 \\\cline{2-10}
			& ALL          & \f{0.87}$\pm$0.005   & \f{0.79}$\pm$0.007  & \f{0.85}$\pm$0.035  & \f{0.88}$\pm$0.004  & \f{0.72}$\pm$0.031  & \f{0.71}$\pm$0.014  & \f{0.73}$\pm$0.005  & \f{0.79} \\ 
			\hline
   
		\hline
		\end{tabular}
	}
\end{table*}

\begin{table*}
    \vspace{-1mm}
	\centering
	\caption{\label{tab:node_f1_degree}
	F1-score on the task of predicting future node importance depending on $d_\theta$ (i.e., in-degree of nodes when their input features are extracted).}
	%\scalebox{0.8}{
        \resizebox{\textwidth}{!}{
		\begin{tabular}{|c|c|c|c|c|c|c|c|c|c|}
		\hline
			\multirow{2}{*}{Centrality}  & \multirow{2}{*}{Feature} & \multicolumn{2}{c|}{Citation Networks} & \multicolumn{3}{c|}{Email/Message Networks} & \multicolumn{2}{c|}{Online Q/A Networks} & \multirow{2}{*}{\bf{Average}} \\\cline{3-9}
			& & HepPh & Hepth & Email-EU & Email-Enron & Message-College & Mathoverflow & Askubuntu &  \\ 
			\hline
			\multirow{3}{*}{Degree}
			& ALL $(d_\theta=2)$ & 0.53$\pm$0.010     & 0.58$\pm$0.013       & \s{0.52}$\pm$0.043    & \f{0.19$\pm$0.005}& 0.38$\pm$0.041           & \s{0.52}$\pm$0.026      &  \f{0.70$\pm$0.005}  & 0.59 \\
			& ALL $(d_\theta=4)$ & \s{0.67}$\pm$0.007 & \s{0.78}$\pm$0.008   & \s{0.62}$\pm$0.062    & 1.00*             & \s{0.49}$\pm$0.045       & \f{0.89}$\pm$0.006      &  1.00*               & \s{0.69}\\
			& ALL $(d_\theta=8)$ & \f{0.82}$\pm$0.006 & \f{0.93}$\pm$0.006   & \f{0.74}$\pm$0.036    & 1.00*             & \f{0.71}$\pm$0.022       & 1.00*                   &  1.00*               & \f{0.80}\\
            \hline
            
            \multirow{3}{*}{Betweenness}
            & ALL $(d_\theta=2)$ & 0.64$\pm$0.007     & 0.90$\pm$0.007       & \s{0.53}$\pm$0.052    & 0.62$\pm$0.016     & 0.38$\pm$0.045       & 0.75$\pm$0.010      & 0.74$\pm$0.007          & 0.65 \\
			& ALL $(d_\theta=4)$ & \s{0.72}$\pm$0.012 & \s{0.94}$\pm$0.007   & 0.52$\pm$0.066        & \s{0.75}$\pm$0.006 & \s{0.50}$\pm$0.042   & \s{0.84}$\pm$0.009  & \s{0.84}$\pm$0.008      & \s{0.73} \\
			& ALL $(d_\theta=8)$ & \f{0.77}$\pm$0.008 & \f{0.97}$\pm$0.005   & \f{0.65}$\pm$0.045    & \f{0.86}$\pm$0.009 & \f{0.69}$\pm$0.057   & \f{0.91}$\pm$0.009  & \f{0.89}$\pm$0.007      & \f{0.82} \\
            \hline
			
            \multirow{3}{*}{Closeness}
            & ALL $(d_\theta=2)$ & 0.85$\pm$0.008       & 0.76$\pm$0.008        & 0.53$\pm$0.043        & 0.83$\pm$0.007     & 0.36$\pm$0.051      & 0.69$\pm$0.022      & 0.81$\pm$0.003      & 0.69 \\
			& ALL $(d_\theta=4)$ & \s{0.87}$\pm$0.008   & \s{0.85}$\pm$0.010    & \s{0.55}$\pm$0.072    &\s{0.91}$\pm$0.006  & \s{0.54}$\pm$0.032  & \s{0.85}$\pm$0.013  & \s{0.91}$\pm$0.004  & \s{0.78} \\
			& ALL $(d_\theta=8)$ & \f{0.88}$\pm$0.007   & \f{0.90}$\pm$0.009    & \f{0.65}$\pm$0.061    & \f{0.97}$\pm$0.004 & \f{0.74}$\pm$0.045  & \f{0.95}$\pm$0.007  & \f{0.98}$\pm$0.003  & \f{0.86} \\
            \hline
		
            \multirow{3}{*}{PageRank}
            & ALL $(d_\theta=2)$ & 0.64$\pm$0.009      & 0.44$\pm$0.013      & 0.52$\pm$0.035        & 0.74$\pm$0.006     & 0.37$\pm$0.030       & 0.53$\pm$0.020      & 0.63$\pm$0.006  & 0.55       \\
			& ALL $(d_\theta=4)$ & \s{0.74}$\pm$0.008  & \s{0.71}$\pm$0.010  & \s{0.62}$\pm$0.037    & \s{0.87}$\pm$0.006 & \s{0.48}$\pm$0.040   & \s{0.79}$\pm$0.008  & \s{0.89}$\pm$0.003      & \s{0.73}   \\
			& ALL $(d_\theta=8)$ & \f{0.83}$\pm$0.007  & \f{0.85}$\pm$0.012  & \f{0.68}$\pm$0.049    & \f{0.95}$\pm$0.003 & \f{0.72}$\pm$0.033   & \f{0.95}$\pm$0.006  & \f{0.98}$\pm$0.003  & \f{0.85}   \\
            \hline
            \multicolumn{10}{l}{* All nodes satisfying the condition on $d_\theta$ have the same class, belonging to top $20\%$ in terms of the considered centrality measure.} \\
  
		\end{tabular}
	}
\end{table*}

\begin{table*}[t]
    \vspace{-1mm}
	\centering
	\caption{\label{tab:node_accuracy_degree}
	Accuracy on the task of predicting future node importance depending on $d_\theta$.}
%	Result: Node Centrality Prediction 2. We report the change of the predictive performance among the degree of predicted node. The higher the degree of the predicted node, the higher predictive performance. \red{TODO}}
    %\scalebox{0.8}{
	\resizebox{\textwidth}{!}{
		\begin{tabular}{|c|c|c|c|c|c|c|c|c|c|}
		\hline
			\multirow{2}{*}{Centrality}  & \multirow{2}{*}{Feature} & \multicolumn{2}{c|}{Citation Networks} & \multicolumn{3}{c|}{Email/Message Networks} & \multicolumn{2}{c|}{Online Q/A Networks} & \multirow{2}{*}{\bf{Average}} \\\cline{3-9}
			& & HepPh & Hepth & Email-EU & Email-Enron & Message-College & Mathoverflow & Askubuntu &  \\ 
			\hline
			\multirow{3}{*}{Degree}
			& ALL $(d_\theta=2)$ & \s{0.76}$\pm$0.008 & 0.78$\pm$0.006       & 0.82$\pm$0.019        & \f{0.77$\pm$0.005}& \s{0.76}$\pm$0.016       & \s{0.68}$\pm$0.014  &  \f{0.61$\pm$0.004}  & 0.74 \\
			& ALL $(d_\theta=4)$ & \s{0.76}$\pm$0.006 & \s{0.80}$\pm$0.009   & \s{0.83}$\pm$0.028    & 1.00*             & 0.70$\pm$0.046           & \f{0.81}$\pm$0.009  &  1.00*               & \s{0.78}\\
			& ALL $(d_\theta=8)$ & \f{0.79}$\pm$0.006 & \f{0.88}$\pm$0.010   & \f{0.86}$\pm$0.021    & 1.00*             & \f{0.72}$\pm$0.025       & 1.00*               &  1.00*               & \f{0.81}\\
            \hline
            
            \multirow{3}{*}{Betweenness}
            & ALL $(d_\theta=2)$ & 0.81$\pm$0.003 & 0.94$\pm$0.004       & \f{0.82}$\pm$0.018       & \s{0.84}$\pm$0.008   & \f{0.75}$\pm$0.019   & \s{0.86}$\pm$0.004  & \f{0.90}$\pm$0.003  & \f{0.85} \\
			& ALL $(d_\theta=4)$ & 0.81$\pm$0.008 & \s{0.96}$\pm$0.004   & \s{0.80}$\pm$0.023       & 0.82$\pm$0.004       & \s{0.70}$\pm$0.023   & \s{0.86}$\pm$0.009  & \s{0.89}$\pm$0.006  & \s{0.83} \\
			& ALL $(d_\theta=8)$ & 0.81$\pm$0.004 & \f{0.98}$\pm$0.004   & \f{0.82}$\pm$0.019       & \f{0.85}$\pm$0.009   & \s{0.70}$\pm$0.049   & \f{0.88}$\pm$0.010  & 0.88$\pm$0.006      & \f{0.85} \\
            \hline
			
            \multirow{3}{*}{Closeness}
            & ALL $(d_\theta=2)$ & 0.91$\pm$0.004   & 0.86$\pm$0.004        & \f{0.82}$\pm$0.020        & 0.85$\pm$0.005 & \f{0.75}$\pm$0.017      & 0.77$\pm$0.011      & 0.82$\pm$0.002      & \s{0.83} \\
			& ALL $(d_\theta=4)$ & 0.91$\pm$0.005   & \s{0.88}$\pm$0.007    & \s{0.80}$\pm$0.022    & \s{0.89}$\pm$0.007 & 0.70$\pm$0.021  & \s{0.80}$\pm$0.015  & \s{0.86}$\pm$0.006  & \s{0.83} \\
			& ALL $(d_\theta=8)$ & 0.91$\pm$0.006   & \f{0.89}$\pm$0.009    & \f{0.82}$\pm$0.020    & \f{0.94}$\pm$0.006 & \s{0.73}$\pm$0.046  & \f{0.91}$\pm$0.013  & \f{0.95}$\pm$0.006  & \f{0.88} \\
            \hline
		
            \multirow{3}{*}{PageRank}
            & ALL $(d_\theta=2)$ & \s{0.81}$\pm$0.005  & \s{0.75}$\pm$0.004  & 0.81$\pm$0.017        & 0.80$\pm$0.006     & \f{0.75}$\pm$0.014 & \s{0.68}$\pm$0.010  & 0.67$\pm$0.006     & 0.75       \\
			& ALL $(d_\theta=4)$ & \s{0.81}$\pm$0.006  & \s{0.75}$\pm$0.007  & \f{0.83}$\pm$0.017    & \s{0.83}$\pm$0.007 & 0.68$\pm$0.025   & \s{0.68}$\pm$0.011  & \s{0.81}$\pm$0.003 & \s{0.77}   \\
			& ALL $(d_\theta=8)$ & \f{0.83}$\pm$0.005  & \f{0.81}$\pm$0.012  & \s{0.82}$\pm$0.023    & \f{0.92}$\pm$0.004 & \s{0.72}$\pm$0.025   & \f{0.91}$\pm$0.011  & \f{0.96}$\pm$0.003 & \f{0.85}   \\
            \hline
            \multicolumn{10}{l}{* All nodes satisfying the condition on $d_\theta$ have the same class, belonging to top $20\%$ in terms of the considered centrality measure.} \\
  
		\end{tabular}
	}
\end{table*}



\begin{table*}[t]
	\centering
	\caption{\label{tab:node_auroc_degree}
	AUROC on the task of predicting future node importance depending on $d_\theta$.}
%	Result: Node Centrality Prediction 2. We report the change of the predictive performance among the degree of predicted node. The higher the degree of the predicted node, the higher predictive performance. \red{TODO}}
	%\scalebox{0.8}{
	\resizebox{\textwidth}{!}{
		\begin{tabular}{|c|c|c|c|c|c|c|c|c|c|}
		\hline
			\multirow{2}{*}{Centrality}  & \multirow{2}{*}{Feature} & \multicolumn{2}{c|}{Citation Networks} & \multicolumn{3}{c|}{Email/Message Networks} & \multicolumn{2}{c|}{Online Q/A Networks} & \multirow{2}{*}{\bf{Average}} \\\cline{3-9}
			& & HepPh & Hepth & Email-EU & Email-Enron & Message-College & Mathoverflow & Askubuntu &  \\ 
			\hline
			\multirow{3}{*}{Degree}
			& ALL $(d_\theta=2)$ & 0.81$\pm$0.004     & 0.84$\pm$0.005       & \s{0.85}$\pm$0.035    & \f{0.86$\pm$0.005}& \s{0.73}$\pm$0.025       & \f{0.71}$\pm$0.014      &  \f{0.65$\pm$0.005}  & 0.78 \\
			& ALL $(d_\theta=4)$ & \s{0.83}$\pm$0.005 & \s{0.87}$\pm$0.006   & \s{0.85}$\pm$0.036    & 1.00*             & 0.72$\pm$0.027           & \s{0.68}$\pm$0.018      &  1.00*               & \s{0.79}\\
			& ALL $(d_\theta=8)$ & \f{0.87}$\pm$0.007 & \f{0.90}$\pm$0.013   & \f{0.88}$\pm$0.027    & 1.00*             & \f{0.78}$\pm$0.031       & 1.00*                   &  1.00*               & \f{0.86}\\
            \hline
            
            \multirow{3}{*}{Betweenness}
            & ALL $(d_\theta=2)$ & 0.87$\pm$0.005     & \s{0.99}$\pm$0.001   & \s{0.83}$\pm$0.024    & 0.87$\pm$0.007     & 0.71$\pm$0.033       & 0.86$\pm$0.009      & 0.86$\pm$0.005          & 0.86 \\
			& ALL $(d_\theta=4)$ & \s{0.89}$\pm$0.006 & \s{0.99}$\pm$0.001   & 0.81$\pm$0.040        & \s{0.89}$\pm$0.004 & \s{0.73}$\pm$0.026   & \s{0.90}$\pm$0.007  & \s{0.91}$\pm$0.004      & \s{0.87} \\
			& ALL $(d_\theta=8)$ & \f{0.90}$\pm$0.003 & \f{1.00}$\pm$0.001   & \f{0.84}$\pm$0.026    & \f{0.93}$\pm$0.006 & \f{0.77}$\pm$0.044   & \f{0.94}$\pm$0.009  & \f{0.94}$\pm$0.006      & \f{0.90} \\
            \hline
			
            \multirow{3}{*}{Closeness}
            & ALL $(d_\theta=2)$ & 0.97$\pm$0.002    & 0.94$\pm$0.005        & \s{0.84}$\pm$0.033    & 0.93$\pm$0.004     & 0.73$\pm$0.028      & 0.83$\pm$0.010      & 0.90$\pm$0.002      & 0.88 \\
			& ALL $(d_\theta=4)$ & 0.97$\pm$0.002    & \s{0.95}$\pm$0.004    & 0.82$\pm$0.027        & \s{0.95}$\pm$0.004 & \s{0.75}$\pm$0.030  & \s{0.88}$\pm$0.012  & \s{0.93}$\pm$0.004  & \s{0.89} \\
			& ALL $(d_\theta=8)$ & 0.97$\pm$0.003    & \f{0.96}$\pm$0.006    & \f{0.88}$\pm$0.024    & \f{0.98}$\pm$0.004 & \f{0.79}$\pm$0.043  & \f{0.92}$\pm$0.016  & \f{0.95}$\pm$0.013  & \f{0.92} \\
            \hline
		
            \multirow{3}{*}{PageRank}
            & ALL $(d_\theta=2)$ & 0.87$\pm$0.005      & 0.79$\pm$0.008      & \s{0.85}$\pm$0.035    & 0.88$\pm$0.004     & \s{0.72}$\pm$0.031   & \s{0.71}$\pm$0.014   & \f{0.73}$\pm$0.005  & 0.79       \\
			& ALL $(d_\theta=4)$ & \s{0.89}$\pm$0.006  & \s{0.83}$\pm$0.008  & \f{0.87}$\pm$0.018    & \s{0.90}$\pm$0.006 & 0.71$\pm$0.028       & 0.69$\pm$0.009       & 0.70$\pm$0.007      & \s{0.80}   \\
			& ALL $(d_\theta=8)$ & \f{0.91}$\pm$0.005  & \f{0.87}$\pm$0.013  & \f{0.87}$\pm$0.034    & \f{0.95}$\pm$0.009 & \f{0.79}$\pm$0.018   & \f{0.73}$\pm$0.040   & \s{0.71}$\pm$0.049  & \f{0.83}   \\
            \hline
            \multicolumn{10}{l}{* All nodes satisfying the condition on $d_\theta$ have the same class, belonging to top $20\%$ in terms of the considered centrality measure.} \\
  
		\end{tabular}
	}
\end{table*}

\begin{table*}[t]
	\centering
	\caption{\label{tab:edge_f1_score}	F1-score on the task of predicting future edge importance.
	}
%	\scalebox{0.7}{
	\resizebox{\textwidth}{!}{
		\begin{tabular}{|c|c|c|c|c|c|c|c|c|c|}
		\hline
			\multirow{2}{*}{Centrality}  & \multirow{2}{*}{Feature} & \multicolumn{2}{c|}{Citation Networks} & \multicolumn{3}{c|}{Email/Message Networks} & \multicolumn{2}{c|}{Online Q/A Networks} & \multirow{2}{*}{\bf{Average}}\\\cline{3-9}
			& & HepPh & Hepth &  Email-EU & Email-Enron & Message-College & Mathoverflow & Askubuntu & \\ 
	    	\hline
			\multirow{8}{*}{Edge Betweenness}
			& Local-ER $(d_\theta=2)$    & 0.68 $\pm$ 0.004       & 0.59 $\pm$ 0.011      & 0.14 $\pm$ 0.094      & 0.74 $\pm$ 0.013      & \f{0.41} $\pm$ 0.042  & \s{0.21} $\pm$ 0.038  & \f{0.40} $\pm$ 0.013  & 0.45 \\
			& Global-ER $(d_\theta=2)$   & \s{0.69} $\pm$ 0.015   & \s{0.63} $\pm$ 0.041  & 0.18 $\pm$ 0.122      & \s{0.79} $\pm$ 0.051  & 0.38 $\pm$ 0.060      & \f{0.23} $\pm$ 0.058  & \s{0.39} $\pm$ 0.022  & \f{0.47} \\
			& Global-Basic $(d_\theta=2)$& \s{0.69} $\pm$ 0.013   & 0.51 $\pm$ 0.168      & \s{0.22} $\pm$ 0.132  & 0.75 $\pm$ 0.072      & 0.37 $\pm$ 0.064      & 0.15 $\pm$ 0.116      & 0.26 $\pm$ 0.180  & 0.42 \\
			& ALL $(d_\theta=2)$         & \f{0.71} $\pm$ 0.005   & \f{0.68} $\pm$ 0.009  & \f{0.25} $\pm$ 0.186  & \f{0.84} $\pm$ 0.005  & \s{0.40} $\pm$ 0.062  & \f{0.23} $\pm$ 0.060  & 0.36 $\pm$ 0.018  & \s{0.50} \\
			\cline{2-10}
			& ALL $(d_\theta=2)$     & \f{0.71} $\pm$ 0.005          & 0.68 $\pm$ 0.009      & \s{0.25} $\pm$ 0.186  & \f{0.84} $\pm$ 0.005     & \s{0.40} $\pm$ 0.062      & 0.23 $\pm$ 0.060      & 0.36 $\pm$ 0.018  & 0.50 \\
			& ALL $(d_\theta=4)$     & \f{0.71} $\pm$ 0.007      & \s{0.72} $\pm$ 0.009  & \f{0.33} $\pm$ 0.104      & \s{0.77} $\pm$ 0.006         & \f{0.43} $\pm$ 0.086  & \s{0.29} $\pm$ 0.071  & \s{0.46} $\pm$ 0.014  & \f{0.53} \\
			& ALL $(d_\theta=8)$     & \s{0.69} $\pm$ 0.004      & \f{0.75} $\pm$ 0.009  & 0.17 $\pm$ 0.079  & 0.72 $\pm$ 0.011     & 0.39 $\pm$ 0.052  & \f{0.31} $\pm$ 0.055  & \f{0.53} $\pm$ 0.023  & \s{0.52} \\
            
		\hline
		\end{tabular}
	}
\end{table*}



\begin{table*}
	\centering
	\caption{\label{tab:edge_accuracy}
	Accuracy on the task of predicting future edge importance.
	}
	
    %\scalebox{0.7}{
	\resizebox{\textwidth}{!}{
		\begin{tabular}{|c|c|c|c|c|c|c|c|c|c|}
		\hline
			\multirow{2}{*}{Centrality}  & \multirow{2}{*}{Feature} & \multicolumn{2}{c|}{Citation Networks} & \multicolumn{3}{c|}{Email/Message Networks} & \multicolumn{2}{c|}{Online Q/A Networks} & \multirow{2}{*}{\bf{Average}}\\\cline{3-9}
			& & HepPh & Hepth &  Email-EU & Email-Enron & Message-College & Mathoverflow & Askubuntu & \\ 
	    	\hline
			\multirow{8}{*}{Edge Betweenness}
			& Local-ER $(d_\theta=2)$    & 0.66 $\pm$ 0.003       & 0.72 $\pm$ 0.005      & \s{0.87} $\pm$ 0.041  & 0.75 $\pm$ 0.012      & \f{0.70} $\pm$ 0.030  & 0.91 $\pm$ 0.008  & 0.91 $\pm$ 0.003  & 0.78 \\
			& Global-ER $(d_\theta=2)$   & \f{0.68} $\pm$ 0.020   & \f{0.75} $\pm$ 0.023  & \f{0.88} $\pm$ 0.041  & \f{0.81} $\pm$ 0.058  & \f{0.70} $\pm$ 0.035  & 0.91 $\pm$ 0.009  & 0.91 $\pm$ 0.003  & \f{0.81} \\
			& Global-Basic $(d_\theta=2)$& 0.66 $\pm$ 0.036       & 0.72 $\pm$ 0.046      & \s{0.87} $\pm$ 0.040  & \s{0.79} $\pm$ 0.052  & 0.67 $\pm$ 0.053      & 0.91 $\pm$ 0.009  & 0.91 $\pm$ 0.008  & 0.79 \\
			& ALL $(d_\theta=2)$         & \s{0.67} $\pm$ 0.037   & \s{0.73} $\pm$ 0.047  & \s{0.87} $\pm$ 0.042  & \f{0.81} $\pm$ 0.055  & \s{0.68} $\pm$ 0.052  & 0.91 $\pm$ 0.009  & 0.91 $\pm$ 0.007  & \s{0.80} \\
			\cline{2-10}
			& ALL $(d_\theta=2)$     & 0.67 $\pm$ 0.037          & 0.73 $\pm$ 0.005      & \s{0.87} $\pm$ 0.042  & \s{0.81} $\pm$ 0.055     & 0.68 $\pm$ 0.052      & 0.91 $\pm$ 0.009      & \f{0.91} $\pm$ 0.007      & 0.80 \\
			& ALL $(d_\theta=4)$     & \s{0.73} $\pm$ 0.006      & \s{0.79} $\pm$ 0.005  & 0.86 $\pm$ 0.050      & 0.78 $\pm$ 0.024         & \s{0.78} $\pm$ 0.024  & \s{0.93} $\pm$ 0.007  & \s{0.90} $\pm$ 0.004  & \s{0.82} \\
			& ALL $(d_\theta=8)$     & \f{0.75} $\pm$ 0.003      & \f{0.81} $\pm$ 0.005  & \f{0.88} $\pm$ 0.046  & \f{0.82} $\pm$ 0.017     & \f{0.82} $\pm$ 0.017  & \f{0.94} $\pm$ 0.005  & \f{0.91} $\pm$ 0.002  & \f{0.85} \\
            
		\hline
		\end{tabular}
	}
\end{table*}

\begin{table*}
	\centering
	\caption{\label{tab:edge_auroc}	AUROC on the task of predicting future edge importance.
	}
	\resizebox{\textwidth}{!}{
%    \scalebox{0.7}{
		\begin{tabular}{|c|c|c|c|c|c|c|c|c|c|}
		\hline
			\multirow{2}{*}{Centrality}  & \multirow{2}{*}{Feature} & \multicolumn{2}{c|}{Citation Networks} & \multicolumn{3}{c|}{Email/Message Networks} & \multicolumn{2}{c|}{Online Q/A Networks} & \multirow{2}{*}{\bf{Average}}\\\cline{3-9}
			& & HepPh & Hepth &  Email-EU & Email-Enron & Message-College & Mathoverflow & Askubuntu & \\ 
	    	\hline
			\multirow{8}{*}{Edge Betweenness}
			& Local-ER $(d_\theta=2)$    & 0.71 $\pm$ 0.003       & 0.77 $\pm$ 0.007      & \f{0.64} $\pm$ 0.080  & 0.82 $\pm$ 0.009      & \f{0.68} $\pm$ 0.035  & \f{0.85} $\pm$ 0.013  & \f{0.86} $\pm$ 0.006  & \s{0.76} \\
			& Global-ER $(d_\theta=2)$   & \f{0.74} $\pm$ 0.027   & \f{0.80} $\pm$ 0.033  & \s{0.63} $\pm$ 0.092  & \s{0.88} $\pm$ 0.058  & \f{0.68} $\pm$ 0.032  & \f{0.85} $\pm$ 0.015  & \f{0.86} $\pm$ 0.007  & \f{0.78} \\
			& Global-Basic $(d_\theta=2)$& 0.71 $\pm$ 0.053       & 0.77 $\pm$ 0.055      & \f{0.64} $\pm$ 0.093  & 0.87 $\pm$ 0.049      & 0.65 $\pm$ 0.058      & 0.73 $\pm$ 0.164      & 0.76 $\pm$ 0.150      & 0.73 \\
			& ALL $(d_\theta=2)$         & \s{0.72} $\pm$ 0.054   & \s{0.79} $\pm$ 0.057  & \f{0.64} $\pm$ 0.099  & \f{0.89} $\pm$ 0.051  & \s{0.66} $\pm$ 0.056  & \s{0.76} $\pm$ 0.150  & \s{0.78} $\pm$ 0.139  & 0.75 \\
			\cline{2-10}
			& ALL $(d_\theta=2)$     & 0.72 $\pm$ 0.054          & 0.79 $\pm$ 0.057      & 0.64 $\pm$ 0.099      & 0.89 $\pm$ 0.051  & 0.66 $\pm$ 0.056      & 0.76 $\pm$ 0.150      & 0.78 $\pm$ 0.139      & 0.75 \\
			& ALL $(d_\theta=4)$     & \s{0.80} $\pm$ 0.004      & \s{0.87} $\pm$ 0.004  & \f{0.75} $\pm$ 0.050  & \f{0.94} $\pm$ 0.002  & \s{0.74} $\pm$ 0.026  & \s{0.89} $\pm$ 0.016  & \s{0.89} $\pm$ 0.009  & \s{0.84} \\
			& ALL $(d_\theta=8)$     & \f{0.82} $\pm$ 0.005      & \f{0.89} $\pm$ 0.005  & \s{0.70} $\pm$ 0.046  & \s{0.93} $\pm$ 0.002      & \f{0.79} $\pm$ 0.021  & \f{0.90} $\pm$ 0.011  & \f{0.90} $\pm$ 0.008  & \f{0.85} \\
            
		\hline
		\end{tabular}
	}
\end{table*}

% \section*{D. Comparison the Results of Future Node and Edge Importance Prediction with Those from Random Graphs}

% \begin{figure}[ht]
%     \vspace{-1mm}
%     \includegraphics[width=0.95\columnwidth]{FIG/overall_down.pdf}
%     \caption{\label{fig:pred_rand}
%     \red{
%         AUROC score on the task of predicting future importance of nodes and edges in citation graphs and their random graphs (Shuffled and Rewired). We predict degree (DEG), betweenness (BET), closeness (CLS), PageRank (PR), and edge-betweenness (EBT) by using \textbf{Global-NR} or \textbf{Global-ER} as a set of input features. Each subfigure indicates the average score of citastion graphs and random graphs when $d_\theta = 2, 4,$ and $8$. Among figures, the score of real-world citation graphs is higher than those of random graphs.
%     }
%     }
% %    graphlet-orbit transition (GoT), and orbit temporal agreement (OTA)}
% \end{figure}

% \red{We compare the predictive performance of the local structural information from the nodes and edges in real-world graphs with following two random graphs: (1) shuffled graphs (see randomized graph in Section~2.1 of the main paper) (2) rewired graphs: the graphs consisted of randomly rewired edges (i.e. every edge $u\rightarrow v\in\SE$ is rewired to $u'\rightarrow v'$ for two nodes $u', v' \in \SV$ which are randomly selected from the last snapshot of $\SG$). 
% We measure AUROC score of the predictive performance of \textbf{Global-NR} for nodes and \textbf{Global-ER} for edges, and otherwise we generally follow the settings in Section~4.2 and Section~5.2 in the main paper. 
% \newline
% The experiment results are shown at Table~\ref{tab:pred_rand_2}, Table~\ref{tab:pred_rand_4}, and Table~\ref{tab:pred_rand_8}.
% Each table contains the experiment results for $d_\theta = 2, 4,$ and $8$, respectively, and we report the average value of the predictive performance among graphs in the same domain.
% In particular, Figure~\ref{fig:pred_rand} contains the results for the citation graphs.
% \newline
% First of all, when the timestamp of edges are shuffled, the predictive performance changes either increasing or decreasing depends on the domain of graphs. 
% For example, the predictive performance of the real-world citation graphs is higher than shuffled graphs, which indicate that the local structural features of nodes and edges are informative for their future centrality in real-world graphs rather than shuffled graphs, which is also described in Figure~\ref{fig:pred_rand}. 
% However, the predictive performance of real-world graphs is not constantly higher than shuffled graphs among all domains as seen in Email/Message networks or Online Q/A networks.
% %In short, %the insertion order of edges contains some information (differentiable with random order) and it 
% %The random order of the edge stream sometimes enhance the predictive performance, especially on the Email/Message graphs.
% %In the case of citation graphs there is a rule for edge insertion that when the new edge $u\rightarrow v$ is inserted to the citation network, the nodes $u$ should be the most recently inserted node or an unseen node since the new paper cites the existing papers. 
% %
% %However, for the other graphs, the local structural information is more informative in the .
% \newline
% Second, predictive performance of rewired graphs in Table~\ref{tab:pred_rand_2}, Table~\ref{tab:pred_rand_4}, and Table~\ref{tab:pred_rand_8}, which indicate the local structure of nodes and edges are not informative in rewired graphs.
% }

% \begin{table*}[t]
% %\begin{adjustwidth}{-2.25in}{0in}
% 	\centering
% 	\caption{\label{tab:pred_rand_2} AUROC on the task of predicting future node importance and edge importance for real-world graphs and random graphs when $d_\theta$ = 2.
% 	}
% 	\resizebox{\textwidth}{!}{
%     %\scalebox{0.95}{
% 		\begin{tabular}{|c|c|c|c|c|c|c|c|c|c|}
% 		    \hline
%             \multirow{2}{*}{} & \multicolumn{3}{c|}{Citation} & \multicolumn{3}{c|}{Email/Message} & \multicolumn{3}{c|}{Online Q/A}    \\
%             \hline
%             Centrality        & Real   & Random  & Rewire    & Real  & Random & Rewire   & Real & Random   & Rewire   \\
%             \hline
%             Degree            & 0.82   & 0.59    & 0.51     & 0.81  & 0.90   & 0.51     & 0.68 & 0.85  &  0.51 \\
%             Betweenness       & 0.93   & 0.59    & 0.50     & 0.80  & 0.90   & 0.53     & 0.86 & 0.93  &  0.50 \\
%             Closeness         & 0.95   & 0.73    & 0.51     & 0.83  & 0.90   & 0.51     & 0.86 & 0.90  &  0.50 \\
%             PageRank          & 0.83   & 0.64    & 0.50     & 0.81  & 0.89   & 0.51     & 0.72 & 0.82  &  0.51 \\
%             Edge Betweenness  & 0.77   & 0.74    & 0.63     & 0.73  & 0.68   & 0.54     & 0.86 & 0.86  &  0.67 \\
%             \hline
%             Average Score     & 0.86   & 0.66    & 0.53     & 0.80  & 0.85   & 0.52     & 0.80 & 0.87  &  0.54 \\
%             \hline
%         \end{tabular}
% 	}
% %\end{adjustwidth}
% \end{table*}

% \begin{table*}[t]
% %\begin{adjustwidth}{-2.25in}{0in}
% 	\centering
% 	\caption{\label{tab:pred_rand_4} AUROC on the task of predicting future node importance and edge importance for real-world graphs and random graphs when $d_\theta$ = 4.
% 	}
% 	\resizebox{\textwidth}{!}{
%     %\scalebox{0.95}{
% 		\begin{tabular}{|c|c|c|c|c|c|c|c|c|c|}
% 		    \hline
%             \multirow{2}{*}{} & \multicolumn{3}{c|}{Citation} & \multicolumn{3}{c|}{Email/Message} & \multicolumn{3}{c|}{Online Q/A}    \\
%             \hline
%             Centrality        & Real   & Random  & Rewire & Real  & Random & Rewire   & Real & Random   & Rewire   \\
%             \hline
%             Degree          & 0.84 & 0.55 & 0.49 & 0.78 & 0.93 & 0.54 & 0.67 & 0.89 & 0.51 \\
%             Between         & 0.94 & 0.58 & 0.50 & 0.81 & 0.91 & 0.53 & 0.92 & 0.96 & 0.51 \\
%             Closeness       & 0.95 & 0.66 & 0.49 & 0.84 & 0.92 & 0.53 & 0.91 & 0.93 & 0.50 \\
%             PageRank        & 0.85 & 0.58 & 0.50 & 0.82 & 0.91 & 0.55 & 0.70 & 0.79 & 0.50 \\
%             Edge-Between    & 0.74 & 0.70 & 0.66 & 0.78 & 0.70 & 0.55 & 0.77 & 0.77 & 0.69 \\ 
%             \hline
%             Average Score   & 0.86 & 0.61 & 0.53 & 0.81 & 0.87 & 0.54 & 0.79 & 0.87 & 0.54 \\ 
%             \hline
%         \end{tabular}
% 	}
% %\end{adjustwidth}
% \end{table*}

% \begin{table*}[t]
% %\begin{adjustwidth}{-2.25in}{0in}
% 	\centering
% 	\caption{\label{tab:pred_rand_8} AUROC on the task of predicting future node importance and edge importance for real-world graphs and random graphs when $d_\theta$ = 8.
% 	}
% 	\resizebox{\textwidth}{!}{
%     %\scalebox{0.95}{
% 		\begin{tabular}{|c|c|c|c|c|c|c|c|c|c|}
% 		    \hline
%             \multirow{2}{*}{} & \multicolumn{3}{c|}{Citation} & \multicolumn{3}{c|}{Email/Message} & \multicolumn{3}{c|}{Online Q/A}    \\
%             \hline
%             Centrality        & Real   & Random  & Rewire    & Real  & Random & Rewire   & Real & Random   & Rewire   \\
%             \hline
%             Degree          & 0.86 & 0.53 & 0.50 & 0.81 & 0.95 & 0.56 & 1.00* & 1.00* & 0.44 \\
%             Between         & 0.95 & 0.56 & 0.50 & 0.84 & 0.85 & 0.57 & 0.94 & 0.97 & 0.48 \\
%             Closeness       & 0.95 & 0.60 & 0.49 & 0.86 & 0.94 & 0.56 & 0.93 & 0.95 & 0.45 \\
%             PageRank        & 0.88 & 0.57 & 0.49 & 0.86 & 0.96 & 0.57 & 0.72 & 0.81 & 0.45 \\
%             Edge-Between    & 0.80 & 0.81 & 0.81 & 0.81 & 0.81 & 0.61 & 0.90 & 0.89 & 0.83 \\ \hline
%             Average Score   & 0.89 & 0.61 & 0.56 & 0.83 & 0.90 & 0.57 & 0.87 & 0.91 & 0.53 \\ 
%             \hline
%             \multicolumn{10}{l}{* All nodes satisfying the condition on $d_\theta$ have the same class, belonging to top $20\%$.} \\
%         \end{tabular}
% 	}
% %\end{adjustwidth}
% \end{table*}



\end{document}
\endinput
