
%%%%%%%%%%%%%%%%%%%%%%%%%%%%%%%%%%%%%%%%%%%%%%%%%%%%%%%%%%%%%%%%%%%%%%%%%%%%%%%%%%%%%%%%%%%%%%%%%%%%%%
%%%%%%%%%%%%%%%%%%%%%%%%%%%%%%%%%%%%%%%%%%%%%%%%%%%%%%%%%%%%%%%%%%%%%%%%%%%%%%%%%%%%%%%%%%%%%%%%%%%%%%

\section{Related Work}\label{section:relwork}

Previous studies on temporal graph analysis are largely categorized into (a) designing algorithms for streaming graphs~\cite{lee2020temporal, eswaran2018spotlight, liben2007link, mcgregor2014graph}, (b) discovering temporal patterns in graphs~\cite{leskovec2005graphs, akoglu2008rtm, beyer2010mechanistic, akoglu2010structure, bahulkar2016analysis}, and (c) generating graphs with realistic dynamics~\cite{barabasi1999emergence, leskovec2010kronecker, akoglu2008rtm}. This work belongs to the second category.

Studies in this category have revealed (a) universal temporal patterns, such as densification~\cite{leskovec2005graphs}, shrinking diameter~\cite{leskovec2005graphs}, and power-laws between principle eigenvalues and edge counts~\cite{akoglu2008rtm}; and (b) domain-specific patterns in hyperlink networks~\cite{broder2011graph}, metabolic networks (e.g., biochemical reactions and  protein interactions)~\cite{beyer2010mechanistic}, communication networks (e.g., phone calls and
texts)~\cite{hidalgo2008dynamics,akoglu2010structure}, and friendship networks~\cite{bahulkar2016analysis}.

In particular, for the analysis of local structures, the concept of graphlets \cite{prvzulj2007biological} (i.e., the sets of isomorphic small subgraphs with a predefined number of nodes) has been extended to temporal graphs. The extensions, which are called \textit{temporal network motifs}, have multiple variants.
%Temporal network motifs are proposed in many works.
Kovanen et al.~\cite{kovanen2011temporal}  defined them as sets of temporal subgraphs with a fixed number of nodes that are (a) topologically equivalent, (b) temporally equivalent (specifically, relative orders of constituent edges are identical), (c)  consecutive (specifically, constituent edges are consecutive for every node), and (d) temporally local (specifically, arrival times of consecutive edges are close enough).
Hulovaty et al.~\cite{hulovatyy2015exploring} ignores (c); and Paranjape et al.~\cite{paranjape2017motifs} 
ignores (c) and relaxes (d) by restricting only the time difference between the first edge and the last edge.
Note that all these notions focus on temporally local subgraphs, and thus they are suitable only for analyzing short-term dynamics. 

For long-term dynamics in local structures,
David et al.~\cite{aparicio2018graphlet} proposed Graph-orbit Transition (GoT) and Orbit Temporal Agreement (OTA), which characterize the dynamic of a temporal graph by approximately counting the number of transitions between node roles.
However, due to high computational overhead, only a small fraction of snapshots can be compared for estimating the counts of transitions, and as a result, their  characterization powers are significantly weaker than our characterization method using GTGs (see Section~\ref{section:graph:transition}). Recall that our method counts ``every'' transition between graphlets, and it is still significantly faster than GOT and OTA (see Section~\ref{sec:appendix:compare_got_ota} in Appendix).

%considers induced subgraphs that the time difference between the first edge and the last edge is within a time window $\Delta$ as instances of the temporal motif. 
%However, due to the time threshold, they are suitable for analyzing the short-term changes of graphs but not for long-term changes in local structures.

For predicting the future in-degree of nodes, Yang et al.~\cite{yang2014predicting} proposed to use five features obtained from graphlets with three nodes (see Section~\ref{subsection:node_prediction} for descriptions).
As shown empirically, our proposed features tend to provide better prediction performance than these five features, and more importantly, they are complementary to each other.
Faisal and Milenkovi{\'c}~\cite{faisal2014dynamic} aimed to detect aging-related nodes, whose topological properties (e.g,. graphlet counts) change highly over time, in the gene expression process. 

On the algorithmic aspect, a great number of algorithms have been developed for the problem of counting the instances of each graphlet, which is also known as the subgraph counting problem. %, has been studied also on the algorithmic aspect.
As suggested in a survey on subgraph counting \cite{ribeiro2021survey},  subgraph-counting algorithms are largely categorized into exact counting \cite{milo2002network, schiller2015stream, ortmann2017efficient, ahmed2017graphlet} and approximated counting  \cite{wernicke2005faster, aslay2018mining}.
Those in the first category are further categorized into enumeration-based approaches \cite{milo2002network, schiller2015stream}, matrix-based approaches \cite{ortmann2017efficient}, and decomposition-based approaches \cite{ahmed2017graphlet}. 
Algorithm~\ref{alg:track_graphlet} belongs to the first subcategory, and it achieves the optimal time complexity achievable by those in this subcategory, as discussed in the beginning of Section~\ref{section:graph:time}.
It is adapted from StreaM \cite{schiller2015stream}, which maintains the counts of the instances of $4$-node undirected graphlets in a fully dynamic graph stream (i.e., a stream of edge insertions and deletions). %In this subcategory, for static graphs, algorithms that are able to enumerate the instances of larger graphlets are available \cite{ribeiro2010g}.
% In this subcategory, 
% For static graphs, fast algorithms (e.g., \cite{ribeiro2010g}) for counting the instances of graphlets of largert size


% \red{
% Ortmann et al. \cite{ortmann2017efficient} presented an algorithm for counting the instances of graphlets with 4 nodes in an undirected graph and graphlets with 3 nodes in directed graph. 
% They look into the non-induced subgraphs, build a linear equation for fast counting, and apply the clique counting algorithm.
% Ribeiro et al. \cite{ribeiro2010g} suggested the data structure for counting subgraphs instances called g-trie, which is a prefix tree for graphs that each node represents a different graph and the graph of parent node is a subgraph of its child node.}

%\blue{capture the changes of local structure of each node-pair among past snapshots, make a  predict whether this node-pair is connected or not in the next snapshot. }

%They compute the node centrality of each node in both its early and late stages, and they measure the correlation between them. 

%\red{Need to Check}  define a concept of node prominence profile based on triadic closure and preferential attachment, and they use it for predicting degree centrality of nodes at their future. They only consider the appearance of nodes at initial stage and after evolution, and they overlook their grown-up progress, which we also investigate.

%However, it is difficult to calculate various centralites whenever new edges arrive}


% \smallsection{Graphlets in Temporal Graphs:}
% Graphlets are a set of connected isomorphic subgraphs. The methods for counting the occurrence of graphlets in temporal graphs have proposed various perspectives: graphlet counting~\cite{lee2020temporal}, frequent pattern mining~\cite{raj2018mining, zhang2020seasonal}, and network motifs~\cite{paranjape2017motifs, hulovatyy2015exploring, kovanen2011temporal, lee2021thyme}.





% 다양한 종류의 graphlet in temporal graph


% \smallsection{Graph Patterns} The pattern of real-world graphs differs from the randomly generated graph. Many works suggest graph generating methods to generate random graphs containing similar graph properties of real-world graph \cite{leskovec2010kronecker, erdHos1960evolution, newman2003structure}. Analyzing graph patterns is a basic step in graph mining, such as graph attack \cite{albert2000error}, anomaly detection \cite{akoglu2015graph}, and so on.

% \smallsection{Network motif} Network motifs are a useful tool for understanding local structural patterns in graphs \cite{milo2002network, lee2020hypergraph, purohit2018temporal, dey2017motif, benson2016higher}. After the concept of the network motif is introduced in \cite{milo2002network}, network motifs are widely used in the various applications: community detection \cite{arenas2008motif}, graph embedding \cite{dareddy2019motif2vec}, and deep learning \cite{sankar2017motif}.  The concept of network motif has been extended various area: temporal graph \cite{paranjape2017motifs}, bipertite graph \cite{borgatti1997network}, and hypergraph \cite{lee2020hypergraph}. 

% \smallsection{Graph Importance} Each node and edge take roles in a graph, and various centrality measures have been proposed to measure their importance in a graph in various perspectives. PageRank \cite{page1999PageRank}, HITs \cite{kleinberg1998authoritative}, closeness, betweenness centrality is a representative measure for node, and edge-betweenness centrality is famous measure for edge. To obtain the important information in the graph, many works identify node importance in the complex system \cite{yang2016efficient, hu2015identifying}, predict node importance \cite{yang2014predicting, park2019estimating}, and measure them as the criteria of the performance.

%%%%%%%%%%%%%%%%%%%%%%%%%%%%%%%%%%%%%%%%%%%%%%%%%%%%%%%%%%%%%%%%%%%%%%%%%%%%%%%%%%%%%%%%%%%%%%%%%%%%%%
%%%%%%%%%%%%%%%%%%%%%%%%%%%%%%%%%%%%%%%%%%%%%%%%%%%%%%%%%%%%%%%%%%%%%%%%%%%%%%%%%%%%%%%%%%%%%%%%%%%%%%

%%%%%%%%%%%%%%%%%%%%%%%%%%%%%%%%%%%%%%%%%%%%%%%%%%%%%%%%%%%%%%%%%%%%%%%%%%%%%%%%%%%%%%%%%%%%%%%%%%%%%%
%%%%%%%%%%%%%%%%%%%%%%%%%%%%%%%%%%%%%%%%%%%%%%%%%%%%%%%%%%%%%%%%%%%%%%%%%%%%%%%%%%%%%%%%%%%%%%%%%%%%%%
