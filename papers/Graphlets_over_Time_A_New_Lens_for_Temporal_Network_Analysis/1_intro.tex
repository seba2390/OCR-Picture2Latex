\begin{figure}[t]
     \captionsetup[subfigure]{justification=centering}
    \begin{subfigure}{0.23\textwidth}
         \includegraphics[width=\textwidth]{FIG/transition_graph/motif_transition_graph.pdf}
         \caption{Similarity between graphs w.r.t. \textbf{graphlet transitions} 
         \newline(classification accuracy = \textbf{97.2\%})}
     \end{subfigure}
     \begin{subfigure}{0.23\textwidth}
        \includegraphics[width=\textwidth]{FIG/transition_graph/subgraph_ratio_profile.pdf}
        \caption{Similarity between graphs w.r.t. \textbf{graphlet occurrences}
        (classification accuracy = \textbf{83.3}\%)}
     \end{subfigure}
    %  \begin{subfigure}{0.15\textwidth}
    %     \includegraphics[width=\textwidth]{FIG/transition_graph/orbit_temporal_agreement.png}
    %     \caption{Similarity between graphs w.r.t. \textbf{graphlet-orbit transitions}
    %     (classification accuracy = \textbf{85.7}\%)}
    %  \end{subfigure}
     
      \caption{\label{fig:domain_correlation} 
      Real-world temporal graphs from the same domain share similar evolution patterns captured by transitions between graphlets. The figures show the pairwise similarity between 9 graphs from 3 domains (distinguished by text colors) with respect to the transitions between graphlets (see (a)) and the occurrences of graphlets (see (b)). The domains of graphs can be classified more accurately in (a) than in (b). Specifically, with the best thresholds of similarity, the classification accuracy is $97.2\%$ in (a) and $83.3\%$ in (b). See Section~\ref{section:graph:transition} for details about the similarity measures.
      }
\end{figure}

\section{Introduction}
\label{sec:intro}
Graphs are a simple yet powerful tool, and thus they have been used for representing various types of 
interactions: email communications, online Q/As, research collaborations, to name a few.
Due to newly formed interactions, such real-world graphs are \textit{temporal}, i.e., they evolve over time with new nodes and edges.
Many studies have examined the dynamics of real-world temporal graphs and revealed interesting patterns, including densification \cite{leskovec2005graphs}, shrinking diameter \cite{leskovec2005graphs}, and temporal locality in triangle formation \cite{lee2020temporal}.


%and they are naturally extended to temporal graphs to represent their changes over time.
%They have been widely used for analyzing real-world dynamic systems such as growing social networks, email communications, and online Q/A communications.
%Many studies have revealed structural patterns of real-world temporal graphs including  and giant connected components \cite{leskovec2005graphs, faloutsos2011power, dorogovtsev2001giant}. 

Graphlets have been widely employed for analyzing local structures of graphs.
\textit{Graphlets} \cite{prvzulj2007biological} are defined as the sets of isomorphic small subgraphs with a predefined number of nodes.
Specifically, the relative counts of the instances of different graphlets effectively characterize the local structures of graphs, with successful applications in graph classification \cite{milo2002network,milo2004superfamilies}, community detection \cite{arenas2008motif,benson2016higher,tsourakakis2017scalable}, anomaly detection \cite{juszczyszyn2011motif}, and node embedding \cite{liu2021motif,lee2019graph,yu2019rum}.

%Although the sizes of each graph vary, graphs from the same domain share similar distribution of network motifs .
%Therefore, network motifs have been studied in various topics including , and so on.


As temporal graphs are pervasive, the concept of graphlets has been generalized in a number of ways for temporal graph analysis. 
\textit{Temporal network motifs} \cite{paranjape2017motifs, kovanen2011temporal} are sets of temporal subgraphs that are (a) identical not just topologically but also temporally, %\footnote{Specifically, two temporal subgraphs are isomorphic if they are topologically equivalent and the orders of their edges are identical.} 
(b) composed of a fixed number of nodes, and (c) temporally local, i.e., composed of edges whose arrival times are close enough (see Section~\ref{section:relwork} for details).
Due to the last condition, they are suitable for analyzing short-term changes of graphs but not for long-term changes in local structures, which are the focus of this paper.  
%They define the temporal motifs as the subgraphs that their temporal edges arrived within in $\delta$ time unit. 
%They 

% 문단 : Motifs in Evolving Graphs
In this paper, we examine the long-term evolution of local structures captured by graphlets, as a new lens for temporal graph analysis, in nine real-world temporal graphs from three different domains. Our analysis is at three levels: graphs, nodes, and edges. %  We report observations as three perspectives: graph, node, and edge-level.

% \begin{figure}[t]
%     \centering
% 	\subfigure[\label{fig:jewel:concise}Similarity between graphs w.r.t. \textbf{graphlet transitions} (classification accuracy = \textbf{97.2\%})]{
% 		\includegraphics[width= 0.46\linewidth]{FIG/transition_graph/motif_transition_graph.pdf}
% 	} 
% 	\subfigure[\label{fig:scalability}Similarity between graphs w.r.t. \textbf{graphlet occurrences}
%         (classification accuracy = \textbf{83.3}\%)]{
% 		\includegraphics[width= 0.46\linewidth]{FIG/transition_graph/subgraph_ratio_profile.pdf}
% 	} 
% 	\caption{\label{fig:domain_correlation} Real-world temporal graphs from the same domain share similar evolution patterns captured by transitions between graphlets. The figures show the pairwise similarity between 9 graphs from 3 domains (distinguished by text colors) with respect to the transitions between graphlets (see (a)) and the occurrences of graphlets (see (b)). The domains of graphs can be classified more accurately in (a) than in (b). Specifically, with the best thresholds of similarity, the classification accuracy is $97.2\%$ in (a) and $83.3\%$ in (b). See Section~\ref{section:graph:transition} for details about the similarity measures.}
% \end{figure}

At the graph level, we first investigate the changes in the distributions of graphlet instances over time. We find out that the evolution patterns are distinguished from those in randomized graphs that are obtained by randomly shuffling edges. Moreover, the evolution patterns in graphs from the same domain share some common characteristics.
%and they also show distinct behaviors depending on the domains of the graphs. 
In order to compare the evolution patterns in a systematic way, we introduce \textit{graphlet transition graphs}, which encode transitions between graphlets due to changes in graphs.
%where nodes indicate graphlets and edges indicate transitions between motifs due to changes in graphs. Edge weights are the number of corresponding transitions.
As shown in Figure~\ref{fig:domain_correlation}(a), graphs from the same domain share similar graphlet-transition patterns, which facilitates accurate graph classification, although the sizes of the graphs vary.

% enabling accurate classification of graph.

% \begin{figure*}[t]
%     \vspace{-2mm}
%     \centering
% 	\subfigure[$13$ graphlets]{
% 		\includegraphics[width=0.185\linewidth]{FIG/motif_and_role/motif.pdf}
% 	}
% 	\subfigure[$30$ node roles (also known as, node orbits)]{
% 		\includegraphics[width=0.37\linewidth]{FIG/motif_and_role/node_role.pdf}
% 	}
%         \subfigure[$30$ edge roles (also known as, edge orbits)]{
% 		\includegraphics[width=0.37\linewidth]{FIG/motif_and_role/edge_role.pdf}
% 	}
% 	\caption{\label{fig:graphlet_and_role} (a) The 13 graphlets \cite{prvzulj2007biological} with three nodes. (b) The 30 node roles \cite{prvzulj2007biological} within the graphlets (see the positions of black nodes). (c) The 30 edge roles within the graphlets (see the positions of edges from a red node to a blue node).
% 	}
% \end{figure*}

\begin{figure*}[t]
%    \vspace{-2mm}
    \centering
    \begin{subfigure}{0.19\textwidth}\captionsetup{justification=centering} 
        \includegraphics[height=5cm]{FIG/motif_and_role/motif.pdf}
        \caption{$13$ graphlets}
     \end{subfigure}
     \hspace{1mm} %added by KJ
     \begin{subfigure}{0.39\textwidth}
        \includegraphics[height=5cm]{FIG/motif_and_role/node_role.pdf}
        \caption{$30$ node roles (also known as, node orbits)}
     \end{subfigure}
     \begin{subfigure}{0.39\textwidth}
        \includegraphics[height=5cm]{FIG/motif_and_role/edge_role.pdf}
        \caption{$30$ edge roles (also known as, edge orbits) }
     \end{subfigure}
      \caption{\label{fig:graphlet_and_role} (a) The 13 graphlets \cite{prvzulj2007biological} with three nodes. (b) The 30 node roles \cite{prvzulj2007biological} within the graphlets (see the positions of black nodes). (c) The 30 edge roles within the graphlets (see the positions of edges from a red node to a blue node).
      }
\end{figure*}





At the node and edge levels, we investigate how local structures around each node and edge in their early stage signal their future importance.
Specifically, as local structures, we consider \textit{node roles} (formally, node automorphism orbits \cite{prvzulj2007biological}) and \textit{edge roles} \cite{hovcevar2016computation}, which are roughly sets of symmetric positions of nodes and edges within graphlets.
We also demonstrate that the counts of the roles taken by each node and edge in their early stage are more informative than previously-used features \cite{yang2014predicting}, and they are complementary to simple global features (e.g., total counts of nodes and edges) for the task of predicting future centralities (specifically, in-degree, betweenness~\cite{freeman1977set}, closeness~\cite{bavelas1950communication}, and PageRank \cite{page1999PageRank}).
%as an isomorphic position of the graphlets, and we find distributions of node roles that each node takes in provide a hint about their future centrality. 
% Then, we predict the future centrality of each node using the counts of their roles in their early stage. 
% We show that they are more informative than existing features, which consist of triad sub-structures \cite{yang2014predicting}, and they are complementary to simple global features such as the number of nodes and edges in the graphs.
% Lastly, we also study local-structural patterns of edges in the same manner.
% 문단 : Contribution


%In short, throughout this paper, we introduce a new lens about analyzing long-term changes of temporal graphs using graphlets.
We summarize our contributions as follows: 
\bit
    \item \textbf{Patterns:} We make several interesting observations about the temporal evolution of graphlets: a surprising similarity in graphs from the same domain and local-structural signals regarding the future importance of nodes and edges.
    \item \textbf{Tool:} We introduce graphlet transition graphs, which is an effective tool for measuring the similarity of local dynamics in temporal graphs of different sizes.
    \item \textbf{Prediction:} We enhance the prediction accuracy of the future importance of nodes and edges by introducing role-based local features, which are complementary to global features.
\eit
\noindent\textbf{Reproducibility:} The code and the datasets are available at 
\url{https://github.com/deukryeol-yoon/graphlets-over-time}.
%\url{https://anonymous.4open.science/r/graphlets-over-time-8489}

% Related work를 2장으로 빼면서 순서 변경
%The rest of this paper is organized as follows.
In Section~\ref{section:prelim}, we introduce basic concepts, notations, and datasets. In Section~\ref{section:graph}, we present our graph-level analysis. % of graphlet distributions over time. 
In Section~\ref{section:node} and Section~\ref{section:edge}, we present our node-level and edge-level analyses. % of the relation between local structures and future importance. 
In Section~\ref{section:relwork}, we present a brief survey of related works.
In Section~\ref{section:conclusion}, we conclude our work.