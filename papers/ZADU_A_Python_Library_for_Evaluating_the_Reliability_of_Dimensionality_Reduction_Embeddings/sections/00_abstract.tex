% Dimensionality reduction (DR) techniques are not perfect. 
Dimensionality reduction (DR) techniques inherently distort the original structure of input high-dimensional data, producing imperfect low-dimensional embeddings. Diverse distortion measures have thus been proposed to evaluate the reliability of DR embeddings.  
However, implementing and executing distortion measures in practice has so far been time-consuming and tedious. 
To address this issue, we present \library, a Python library that provides distortion measures.
\library is not only easy to install and execute but also
enables comprehensive evaluation of DR embeddings through three key features. First, the library covers a wide range of distortion measures. Second, it automatically optimizes the execution of distortion measures, substantially reducing the running time required to execute multiple measures.
Last, the library informs how individual points contribute to the overall distortions, facilitating the detailed analysis of DR embeddings. By simulating a real-world scenario of optimizing DR embeddings, we verify that our optimization scheme substantially reduces the time required to execute distortion measures. 
Finally, as an application of \library, we present another library called \vislib that allows users to easily create distortion visualizations that depict the extent to which each region of an embedding suffers from distortions.

% applications such as distortion visualizations.


% Our quantitative evaluation verifies the scalability of our 


% We build \library in three steps:
% First, we conduct an extensive search on existing distortion measures.
% Second, we design a pipeline minimizing the computation required for the execution of the measures.
% At last, we implement a pipeline as a Python library.
% Empowered by the acceleration based on CPU multithreading, our library is scalable and can be easily executed. 
% We use \library to comprehensively evaluate four variants of UMAP and analyze their characteristics. 
% We also show that \library can be leveraged with existing distortion visualizations, allowing a more detailed analysis of DR embeddings. 


