Dimensionality reduction (DR) suffers from inaccuracy.
Although DR is a useful technique for visually analyzing high-dimensional data \cite{nonato19tvcg}, distortion inevitably occurs while moving data from a broad high-dimensional space to a narrow low-dimensional space \cite{nonato19tvcg, martins14cg, jeon21tvcg, jeon22arxiv}. 
Such distortions lower the credibility of data analysis with DR embeddings. 
To avoid such risks of misinterpretation, we need to assess the reliability of the embeddings prior to their usage. 
For this purpose, various distortion measures (e.g., Trustworthiness \& Continuity \cite{lee07springer} and Steadiness \& Cohesiveness \cite{jeon21tvcg}) have been proposed \cite{nonato19tvcg}. 

However, there is a lack of an easy-to-use library that provides distortion measures, which leads to the consumption of researchers' valuable time.
 % in practice is a nontrivial problem. 
A few research works provide the source code of distortion measures \cite{jeon22vis, ingram15neurocomputing, fujiwara23pvis} (\autoref{tab:measures}). However, researchers need considerable time to install and execute such code. 
For example, they need to manually configure the environment settings and install the dependencies.
Researchers thus often implement distortion measures on their own, but the laboriousness of the task persists.

Given this background, we present \library, a unified and accessible Python library serving distortion measures. 
To save the time needed to install and execute the library, we make \library easily downloadable via the Python package index \texttt{PyPI}.
Moreover, in line with the current trend in DR research \cite{jeon22vis, fujiwara23pvis, jeon21tvcg, pedregosa11jmlr, mcinnes2020arxiv}, \library is compatible with  existing Python machine learning and visualization toolboxes (e.g., \texttt{scikit-learn} \cite{pedregosa11jmlr} and \texttt{matplotlib} \cite{hunter07cse}).


\library differentiates from previous implementations of distortion measures from three perspectives.
First, the library covers a broad range of distortion measures, with a total of 17 provided. This is over three times more than the earlier implementations with the most measures available \cite{jeon22vis}. Hence, researchers do not need to spend time searching for available codes or implementing the codes by themselves.
Second, \library automatically optimizes the execution of multiple measures, substantially reducing the amount of computation time needed. 
Last, \library supports the computation of local pointwise distortions, which illustrates the contribution of each data point to the overall distortions.
By explaining distortions in a fine-grained manner, local distortions enable a more detailed analysis of DR embeddings.

We simulate a real-world scenario of evaluating DR embeddings to assess the extent to which \library optimizes the execution of multiple measures.
The simulation verifies that our optimization substantially reduces the total running time required for executing distortion measures. 
We also demonstrate using \library to create distortion visualizations that depict how and where the embedding suffers from distortions. We have packaged our implementation of distortion visualizations as a library called \vislib, enabling users to readily create the visualizations. 



% We demonstrate a use case of \library in analyzing UMAP \cite{mcinnes2020arxiv} and its three variants: Densmap \cite{narayan21nature}, UMATO \cite{jeon22vis}, and Progressive UMAP \cite{ko20eurovis}. 
% We first investigate how the scores of distortion measures vary across the variants, revealing the tradeoff in preserving the high-dimensional structure of input data.
% Then, we conduct a detailed analysis of the cluster patterns shown by DR embeddings by visualizing the pointwise distortion produced by \library. 



% We also consider the coverage of distortion measures and scalability as important requirements of \library. 
% To maximize the coverage, we conduct a wide literature review on existing distortion measures.
% To enhance scalability, we design an execution pipeline of measures that minimize the computation and maximize parallelism. 
% During implementation, we leverage state-of-the-art data processing library (e.g., \texttt{pynndescent} \cite{mcinnes23github}) and CPU-based multithreading \cite{lam15llvm} to further accelerate \library.






% To build \library, we first conduct a wide literature review on existing distortion measures.
% We then design an execution pipeline of measures that minimizes computation. 
% While 

% However, we lack a unified and accessible toolkit to utilize distortion measures. 




% They the reliability of DR embeddings in different aspects.
% For example, local measures (e.g., Trustworthiness \& Continuity \cite{lee07springer}) evaluate how well the neighborhood structure of the original data is well preserved in the embedding. 
% Global measures (e.g., Kullback-Leibler Divergence) assess how pairwise distances are distorted.


