\section{Numerical results}\label{sec:experiments}
Table \ref{tbl:boundtable} shows the results of running \eqref{eqn:alg} on real-world instances of MAP inference to find nearby $(2,1,\psi)$-expansion stable instances. We study \emph{stereo vision} models using images from the Middlebury stereo dataset \citep{scharstein2002taxonomy} and Potts models from \citet{tappen2003comparison}. Please see Appendix \ref{sec:experiments_details} for more details about the models and experiments.

We find, surprisingly, that only relatively sparse changes are required to make the observed instances $(2,1,\psi)$-expansion stable with  $\psi = 1$.
On these instances, we evaluate the recovery guarantees by our bound from Theorem \ref{thm:curvature} and compare it to the actual value of the recovery error $||\hx - \hx^{MAP}||_1/2n$.
In all of our experiments, we choose the \emph{target} solution $x^t$ for \eqref{eqn:alg} to be equal to the MAP solution $\hat{x}^{MAP}$ of the observed instance.
Therefore, we find a $(2,1,\psi)$-expansion stable instance that has the same MAP solution as our observed instance.
The recovery error bound given by Theorem \ref{thm:curvature} is then also a bound for the recovery error between $\hat{x}$ and $\hat{x}^{MAP}$, because $\hx^{MAP} = x^t$. On these instances, the bounds from our curvature result (Theorem \ref{thm:curvature}) are reasonably close to the actual recovery value. However, this bound uses the property that $\hat{x}$ has good objective in the stable instance and so it is still a ``data-dependent'' bound in the sense that it uses an empirically observed property of the LP solution $\hat{x}$. In Appendix \ref{sec:experiments_details}, we show how to refine Corollary \ref{cor:deviation} to give non-vacuous recovery bounds that do not depend on $\hat{x}$.