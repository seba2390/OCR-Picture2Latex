\section{Expansion Stability}\label{sec:expansion-stability}
\begin{figure}[tb]
  \centering
\begin{subfigure}{.5\linewidth}
  \centering
  \scalebox{1.0}{  
  \tikzstyle{vertex}=[circle, draw=black, very thick, minimum size=5mm]
  \tikzstyle{edge} = [draw=black, line width=1]
  \tikzstyle{weight} = [font=\normalsize]
  \begin{tikzpicture}[scale=2,auto,swap]
    \foreach \pos /\name in {{(0,0)}/u,{(1,0)}/w,{(0.5,0.75)}/v}
    \node[vertex](\name) at \pos{$\name$};
    \foreach \source /\dest /\weight in {u/w/1+\epsilon}
    \path[edge] (\source) -- node[weight] {$\weight$} (\dest);
    \foreach \source /\dest /\weight/\pos in {u/v/1+\epsilon/{above left}, v/w/1+\epsilon/{above right}}
    \path[edge] (\source) -- node[weight, \pos] {$\weight$} (\dest);
  \end{tikzpicture}
  }
\end{subfigure}%
\begin{subfigure}{.5\linewidth}
  \centering
  \scalebox{1.0}{
\begin{tabular}{|l|ccc|}
\hline
\multicolumn{1}{|c|}{\textbf{Node}} & \multicolumn{3}{|c|}{\textbf{Costs}} \\
\hline
u & .5 & $\infty$        & $\infty$      \\
\hline
v & 1 & 0 & $\infty$\\
\hline
w & 1 & $\infty$ & 0\\
\hline
\end{tabular}
}
\end{subfigure}
  \caption{$(2,1)$-expansion stable instance that is not $(2,1)$-stable. In the original instance (shown left), the optimal solution labels each vertex with label 1, for an objective of $2.5$. This instance is not $(2,1)$-stable: consider the $(2,1)$-perturbation that multiples all edge weights by $1/2$. In this perturbed instance, the original solution still has objective 2.5, and the new optimal solution labels $(u,v,w) \rightarrow (1,2,3)$. This has a node cost of 0.5 and an edge cost of $(3+3\epsilon)/2$, for a total of $2+3\epsilon/2 < 2.5$. Since the original solution is not optimal in the perturbed instance, this instance is not $(2,1)$-perturbation stable. However, note that the only expansions of the original solution (which had all label 1) that have non-infinite objective are $(u,v,w) \rightarrow (1,2,1)$ and $(u,v,w) \rightarrow (1,1,3)$. These each have objective $2.5 + \epsilon$, which is strictly greater than the perturbed objective of the original solution. In fact, checking this single perturbation, known as the \emph{adversarial perturbation} is enough to verify expansion stability: this instance is $(2,1)$-expansion stable. We include the full details in Appendix \ref{sec:expansion-stability_details}.}
\label{fig:counter1}
\end{figure}


In this section, we generalize the stability result of \citet{LanSonVij18} to a much broader class of instances.
This generalization allows us to efficiently check whether a real-world instance could plausibly have the structure shown in Figure \ref{fig:main-idea} (that is, whether the instance is close to a suitably stable instance).

Consider a fixed instance $(G,c,w)$ with a unique MAP solution $\bar{x}$.
Theorem 1 of \citet{LanSonVij18} requires that for all $\theta' \in \{(c,w')\ |\ w' \in \{(2,1)\text{-perturbations of } w\}\}$, $\langle \theta', x \rangle > \langle \theta', \bx \rangle$ for \emph{all} labelings $x\ne \bx$.
That is, that result requires $\bx$ to be the unique optimal solution in any $(2,1)$-perturbation of the instance.
By contrast, our result only requires $\bx$ to have better perturbed objective than the set of \emph{expansions} of $\bx$ (c.f. Definition \ref{def:expansion}).
\begin{definition}[(2,1)-expansion stability]
\label{def:expansion-stability}
    Let $\bx$ be the unique MAP solution for $(G,c,w)$, and let $\calE_{\bx}$ be the set of expansions of $\bx$ (see Definition \ref{def:expansion}). 
    Let \[
    \Theta = \{(c,w')\ |\  w' \in \{(2,1)\text{-perturbations of } w\}\}
    \]
    be the set of all objective vectors within a $(2,1)$-perturbation of $\theta = (c,w)$. We say the instance $(G,c,w)$ is $(2,1)$-expansion stable if the following holds for all $\theta' \in \Theta$ and all $x\in \calE_{\bx}$:
    \[
        \dot{\theta'}{x} > \dot{\theta'}{\bx}.
    \]
    That is, $\bx$ is better than all of its expansions $x \ne \bx$ in every $(2,1)$-perturbation of the instance.
\end{definition}

\begin{restatable}[Local LP is tight on $(2,1)$-expansion stable instances]{theorem}{lptes}\label{thm:lptes}
Let $\bx$ and $\hx$ be the MAP and local LP solutions to a $(2,1)$-expansion stable instance $(G,c,w)$, respectively. Then $\bx = \hx$ i.e. the local LP is tight on $(G,c,w)$.
\end{restatable}
We defer the proof of this theorem to Appendix~\ref{sec:expansion-stability_details} as it is similar to the proof of Theorem 1 from \citet{LanSonVij18}. 
The $(2,1)$-expansion stability assumption is much weaker than $(2,1)$-stability because the former only compares $\bx$ to its expansions, whereas the latter compares $\bx$ to \emph{all} labelings.
While the rest of our results can also be adapted to the $(2,1)$-stability definition, this relaxed assumption gives better empirical results. Figure \ref{fig:counter1} shows an example of a $(2,1)$-expansion stable instance that is not $(2,1)$-stable. This shows that our new stability condition is less restrictive.
