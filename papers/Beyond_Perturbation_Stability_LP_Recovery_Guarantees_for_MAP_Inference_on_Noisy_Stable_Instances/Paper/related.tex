\subsection{Related work.}
\paragraph{Perturbation stability.}
Several works have given recovery guarantees for the local LP relaxation on perturbation stable instances of uniform metric labeling \citep{LanSonVij18, LanSonVij19} and for similar problems \citep{makarychev2014bilu, AngMakMak17}.
%Aside from \citet{LanSonVij19}, none of these works gives a guarantee for the local LP relaxation when the observed instance $\obsins$ does not satisfy the strict stability assumptions (that are  not satisfied in practice). \citet{LanSonVij19} give partial recovery guarantees for the local LP when parts (blocks) of the observed instance satisfy a stability-like condition (that involves the dual program on the instance), and they showed that practical instances have ``blocks'' that satisfy their condition. We take a different approach in this work: we do not make any explicit assumptions involving the linear program (or its dual) for the observed instance $\obsins$. Instead, we merely assume this instance is ``close'' to a latent stable instance $\stabins$, but we are still able to give partial recovery guarantees for the local LP. Our empirical results also indicate that real-world instances are close to stable.
Finally, \citet{makarychev2014bilu} consider a notion of weak stability for graph partitioning problems, where the optimal solution is allowed to change a bit under perturbations. However, their guarantees are for a different algorithm, and do not explain the performance of the linear programming relaxation. 

\paragraph{Easy instances corrupted with noise.}
Our random noise model is similar to several planted average-case models like stochastic block models (SBMs) considered in the context of problems like community detection, correlation clustering and partitioning \citep[see e.g.,][]{Mcsherry,Abbesurvey, GRSY15}. Instances generated from these models can also be seen as the result of random noise injected into an instance with a nice block structure that is easy to solve. 
Several works give exact recovery and approximate recovery guarantees for semidefinite programming (SDP) relaxations for such models in different parameter regimes~\cite{Abbesurvey,guedon2016community}. 
In our model however, we start with an {\em arbitrary} stable instance as opposed to an instance with a block structure (which is trivial to solve). 
Moreover, we are unaware of such analysis in the context of linear programs.
Please see Section~\ref{sec:random-model} for a more detailed comparison. To the best of our knowledge, we are the first to consider study instances generated from random perturbations to stable instances.