% \documentclass{article}
\documentclass[twoside,11pt]{article}

\usepackage{blindtext}
\usepackage{setspace}
\usepackage{enumitem}
\usepackage{quoting}
\quotingsetup{vskip=5pt,leftmargin=15pt,rightmargin=15pt}
\usepackage{tocloft}
\setlength{\cftbeforesecskip}{6pt}
\renewcommand\floatpagefraction{1.5}
\renewcommand\topfraction{1.5}
\renewcommand\bottomfraction{1.5}
\renewcommand\textfraction{0.0}
\renewcommand\dblfloatpagefraction{1.5}
\renewcommand\dbltopfraction{1.5}
\setcounter{totalnumber}{50}
\setcounter{topnumber}{50}
\setcounter{bottomnumber}{50}
\setcounter{dbltopnumber}{50}

\usepackage[preprint]{jmlr2e}
\usepackage[top=1in, bottom=1in, left=1in, right=1in]{geometry}

\usepackage[utf8]{inputenc} % allow utf-8 input
\usepackage[T1]{fontenc}    % use 8-bit T1 fonts
\usepackage{hyperref}       % hyperlinks
\usepackage{url}            % simple URL typesetting
\usepackage{booktabs}       % professional-quality tables
\usepackage{amsfonts}       % blackboard math symbols
\usepackage{nicefrac}       % compact symbols for 1/2, etc.
\usepackage{microtype}      % microtypography

% input our definitions of various math symbols, operators, and environments
\usepackage{amsfonts}
\usepackage{amssymb}

\usepackage{empheq}

\usepackage{relsize}
\usepackage{bbm}

\usepackage{etoolbox}
\usepackage{setspace}
\AtBeginEnvironment{quote}{\par\singlespacing\small}

\usepackage{graphicx}
\usepackage{outlines}

\usepackage{tikz}
\usetikzlibrary{cd}
\usepackage[outline]{contour}
\usepackage{onimage}

\usepackage{xcolor}

\newcommand{\qedsymbol}{$\blacksquare$}


% wide box for emphasized equations
\newcommand*\widefbox[1]{\fbox{\hspace{2em}#1\hspace{2em}}}

\newcommand{\no}{\noindent}
\newcommand{\pf}[1]{{\partial \over \partial #1}}
\newcommand{\pp}[2]{{\partial #1 \over \partial #2}}


\newcommand{\fndict}{\mathcal{D}}
\newcommand{\manM}{\mathcal{M}}
\newcommand{\vf}{\mathfrak{X}}
\newcommand{\tf}{\mathfrak{T}}

\DeclareMathOperator{\Lie}{Lie}
\DeclareMathOperator{\graph}{gr}
\DeclareMathOperator{\image}{im}
\DeclareMathOperator{\sect}{\Sigma}
\DeclareMathOperator{\sym}{\mathfrak{sym}}
\DeclareMathOperator{\Sym}{Sym}

\DeclareMathOperator*{\argmin}{\arg\!\min\enskip}
\DeclareMathOperator*{\argmax}{\arg\!\max\enskip}
\DeclareMathOperator*{\minimize}{\min\!imize\enskip}
\DeclareMathOperator*{\maximize}{\max\!imize\enskip}
\DeclareMathOperator{\sgn}{sgn}
\DeclareMathOperator{\qf}{qf}
\DeclareMathOperator{\Tr}{Tr}
\DeclareMathOperator{\rank}{rank}
\DeclareMathOperator{\codim}{codim}
\DeclareMathOperator{\diag}{diag}
\DeclareMathOperator{\vspan}{span}
\DeclareMathOperator{\Range}{Range}
\DeclareMathOperator{\Null}{Null}
\DeclareMathOperator{\vct}{vec}
\DeclareMathOperator{\col}{col}
\DeclareMathOperator{\row}{row}
\DeclareMathOperator{\grad}{grad}
\DeclareMathOperator{\diverg}{div}
\DeclareMathOperator{\td}{\mathrm{d}}
\DeclareMathOperator{\ddt}{\frac{\td}{\td t}}
\DeclareMathOperator{\D}{\mathrm{d}}
\DeclareMathOperator{\Id}{Id}
\DeclareMathOperator{\intmul}{\mathlarger{\lrcorner}}
\DeclareMathOperator{\vpr}{vpr} %vertical projection: Kolar, Michor, Slovak, Section 6.11
\DeclareMathOperator{\Flow}{Fl}
\DeclareMathOperator{\volform}{dV}

\DeclareMathOperator{\indic}{\mathbbm{1}}

\newcommand{\mat}[1]{{\pmb{#1}}}
\newcommand{\vect}[1]{{\boldsymbol{#1}}}

\newcommand{\mcal}[1]{{\mathcal{#1}}}
\newcommand{\mfrak}[1]{{\mathfrak{#1}}}

\newcommand{\R}{\mathbb{R}}
\newcommand{\C}{\mathbb{C}}

% separating lines in bmatrix environment
\makeatletter
\renewcommand*\env@matrix[1][*\c@MaxMatrixCols c]{%
  \hskip -\arraycolsep
  \let\@ifnextchar\new@ifnextchar
  \array{#1}}
\makeatother

\newcommand{\soremark}[1]{\textcolor{blue}{[SO: #1]}}
\newcommand{\nzremark}[1]{\textcolor{orange}{[NZ: #1]}}
\newcommand{\sbremark}[1]{\textcolor{red}{[SB: #1]}}
\newcommand{\nkremark}[1]{\textcolor{green}{[NK: #1]}}

\begin{document}

\title{\LARGE{\textbf{A Unified Framework to Enforce, Discover, and Promote Symmetry in Machine Learning}}}

\author{
    \name \hspace{-4pt}Samuel E. Otto \email ottoncdr@uw.edu \\
    \addr AI Institute in Dynamic Systems\\
    University of Washington\\
    Seattle, WA 98195-4322, USA
    \AND
    \name Nicholas Zolman \email nzolman@uw.edu \\
    \addr AI Institute in Dynamic Systems\\    
    \addr Mechanical Engineering \\
    University of Washington\\
    Seattle, WA 98195-4322, USA
    \AND
    \name J. Nathan Kutz \email kutz@uw.edu \\
    \addr AI Institute in Dynamic Systems\\
    \addr Applied Mathematics \\
    University of Washington \\
    Seattle, WA 98195-4322, USA
    \AND
    \name Steven L. Brunton \email sbrunton@uw.edu \\
    \addr AI Institute in Dynamic Systems\\
    \addr Mechanical Engineering \\
    University of Washington \\
    Seattle, WA 98195-4322, USA
}

\editor{My editor}

\maketitle

\begin{abstract}
    Symmetry is present throughout nature and continues to play an increasingly central role in physics and machine learning. 
    Fundamental symmetries, such as Poincar\'{e} invariance, allow physical laws discovered in laboratories on Earth to be extrapolated to the farthest reaches of the universe.
    Symmetry is essential to achieving this extrapolatory power in machine learning applications.
    For example, translation invariance in image classification allows models with fewer parameters, such as convolutional neural networks, to be trained on smaller data sets and achieve state-of-the-art performance.
    In this paper, we provide a unifying theoretical and methodological framework for incorporating symmetry into machine learning models in three ways:
    \textbf{1. enforcing} known symmetry when training a model; \textbf{2. discovering} unknown symmetries of a given model or data set; and \textbf{3. promoting} symmetry during training by learning a model that breaks symmetries within a user-specified group of candidates when there is sufficient evidence in the data.
    We show that these tasks can be cast within a common mathematical framework whose central object is the Lie derivative associated with fiber-linear Lie group actions on vector bundles.
    We extend and unify several existing results by showing that enforcing and discovering symmetry are linear-algebraic tasks that are dual with respect to the bilinear structure of the Lie derivative.
    We also propose a novel way to promote symmetry by introducing a class of convex regularization functions based on the Lie derivative and nuclear norm relaxation to penalize symmetry breaking during training of machine learning models. 
    We explain how these ideas can be applied to a wide range of machine learning models including basis function regression, dynamical systems discovery, multilayer perceptrons, and neural networks acting on spatial fields such as images.\\

    \noindent\textbf{Keywords:} Symmetries, machine learning, Lie groups, manifolds, invariance, equivariance, neural networks, deep learning
\end{abstract}

\newpage
\renewcommand{\baselinestretch}{.9}\normalsize
\tableofcontents
\renewcommand{\baselinestretch}{1.0}\normalsize

\newpage

\section{Introduction}

Symmetry is present throughout nature, 
and according to David \cite{Gross1996role} the discovery of fundamental symmetries has played an increasingly central role in physics since the beginning of the 20th century.
He asserts that
\vspace{-.1in}\begin{quote}
   ``Einstein’s great advance in 1905 was to put symmetry first, to regard the symmetry principle as the primary feature of nature that constrains the allowable dynamical laws.''
\end{quote}
\vspace{-.025in}
According to Einstein's special theory of relativity, physical laws including those of electromagnetism and quantum mechanics are Poincar\'{e}-invariant, meaning that after predictable transformations (actions of the Poincar\'{e} group), these laws can be applied in any non-accelerating reference frame, anywhere in the universe, at all times.
Specifically these transformations form a ten-parameter group including four translations of space-time, three rotations of space, and three shifts or ``boosts'' in velocity.
For small boosts of velocity, these transformations become the Galilean symmetries of classical mechanics.
Similarly, the theorems of Euclidean geometry are unchanged after arbitrary translations, rotations, and reflections, comprising the Euclidean group.
In fluid mechanics, the conformal (angle-preserving) symmetry of Laplace's equation is used to reduce the study of idealized flows in complicated geometries to canonical flows in simple domains.
In dynamical systems, the celebrated theorem of \citet{noether1918invariante} establishes a correspondence between symmetries and conservation laws, an idea which has become a central pillar of mechanics~\citep{Abraham2008foundations}.
These examples illustrate the diversity of symmetry groups and their physical applications.
More importantly, they illustrate how \emph{symmetric models and theories in physics automatically extrapolate in explainable ways to environments beyond the available data.}


In machine learning, models that exploit symmetry can be trained with less data and use fewer parameters compared to their asymmetric counterparts.
Early examples include augmenting data with known transformations (see \cite{shorten2019survey, van2001art}) or using convolutional neural networks (CNNs) to achieve translation invariance for image processing tasks (see \cite{fukushima1980neocognitron, lecun1989backpropagation}).
More recently, equivariant neural networks respecting Euclidean symmetries have achieved state-of-the-art performance for predicting potentials in molecular dynamics \cite{batzner2022e3atom}. 
As with physical laws, symmetries and invariances allow machine learning models to extrapolate beyond the training data, and achieve high performance with fewer modeling parameters.

However, many problems are only weakly symmetric.
Gravity, friction, and other external forces can cause some or all of the Poincar\'{e} or Galilean symmetries to be broken.
Interactions between particles can be viewed as breaking symmetries possessed by non-interacting particles.
Written characters have translation and scaling symmetry, but not rotation (cf. $6$ and $9$, d and p, N and Z) or reflection (cf. b and d, b and p).
One of the main contributions of this work is to propose a method of enforcing a new principle of parsimony in machine learning.
This principal of parsimony by maximal symmetry states that \emph{a model should break a symmetry within a set of physically reasonable transformations 
(such as Poincar\'{e}, Galilean, Euclidean, or conformal symmetry) 
only when there is sufficient evidence in the data.}

In this paper, we provide a unifying theoretical and methodological framework for incorporating symmetry into machine learning models in the following three ways:
\vspace{-.05in}
\begin{enumerate}[label=Task \arabic*., leftmargin=1.5cm,itemsep=-1.75pt]
    \item \textbf{Enforce.} Train a model with known symmetry.
    \item \textbf{Discover.} Identify the symmetries of a given model or data set.
    \item \textbf{Promote.} Train a model with as many symmetries as possible (from among candidates), breaking symmetries only when there is sufficient evidence in the data.
\end{enumerate}

While these tasks have been studied to varying extents separately,
we show how they can be situated within a common mathematical framework whose central object is
the Lie derivative associated with fiber-linear Lie group actions on vector bundles.
As a special case, the Lie derivative recovers the linear constraints derived by \citet{Finzi2021practical} for weights in equivariant multilayer perceptrons.
In full generality, \emph{we show that known symmetries can be enforced as linear constraints derived from Lie derivatives for a large class of problems} including learning vector and tensor fields on manifolds as well as learning equivariant integral operators acting on such fields.
For example the kernels of ``steerable'' CNNs developed by \cite{weiler20183d, Weiler2019general} are constructed to automatically satisfy these constraints for the groups $SO(3)$ (rotations in three dimensions) and $SE(2)$ (rotations and translations in two dimensions).
We show how analogous steerable networks for other groups, such as subgroups of $SE(n)$, can be constructed by numerically enforcing linear constraints derived from the Lie derivative on integral kernels defining each layer. 
Symmetries, conservation laws, and symplectic structure can also be enforced when learning dynamical systems via linear constraints on the vector field.
Again these constraints come from the Lie derivative and can be incorporated into neural network architectures and basis function regression models such as Sparse Identification of Nonlinear Dynamics (SINDy) \citep{Brunton2016discovering}.

\citet{Moskalev2022liegg} identifies the connected subgroup of symmetries of a trained neural network by computing the nullspace of a linear operator.
Likewise, \citet{Kaiser2018discovering,kaiser2021data} recovers the symmetries and conservation laws of a dynamical system by computing the nullspace of a different linear operator.
\emph{We observe that these operators and others whose nullspaces encode the symmetries of more general models can be derived directly from the Lie derivative in a manner dual to the construction of operators used to enforce symmetry.}
Specifically, the nullspaces of the operators we construct reveal the largest connected subgroups of symmetries for enormous classes of models.
This extends work by \cite{Gruver2022Lie} using the Lie derivative to test whether a trained neural network is equivariant with respect to a given one-parameter group, e.g., rotation of images.
Generalizing the ideas in \citet{Cahill2023Lie}, we also show that the symmetries of point clouds approximating underlying submanifolds can be recovered by computing the nullspaces of associated linear operators.
This allows for the unsupervised mining of data for hidden symmetries.

The idea of relaxed symmetry has been introduced recently by \cite{Wang2022approximately}, along with architecture-specific symmetry-promoting regularization functions involving sums or integrals over the candidate group of transformations.
The Augerino method introduced by \cite{Benton2020learning} uses regularization to promote equivariance with respect to a larger collection of candidate transformations.
Promoting physical constraints through the loss function is also a core concept of Physics-Informed Neural Networks (PINNs) introduced by \cite{Raissi2019physics}.
\emph{Our approach to the third task (promoting symmetry) is to introduce a unified and broadly applicable class of convex regularization functions based on the Lie derivative to penalize symmetry breaking during training of machine learning models.}
As we describe above, the Lie derivative yields an operator whose nullspace corresponds to the symmetries of a given model.
Hence, the lower the rank of this operator, the more symmetric the model is.
The nuclear norm has been used extensively as a convex relaxation of the rank with favorable theoretical properties for compressed sensing and low-rank matrix recovery \citep{Recht2010guaranteed, Gross2011recovering}, as well as in robust PCA \citep{Candes2011robust, Bouwmans2018applications}.
Penalizing the nuclear norm of our symmetry-encoding operator yields a convex regularization function that can be added to the loss when training machine learning models, including basis function regression and neural networks.
Likewise, we use a nuclear norm penalty to promote conservation laws and Hamiltonicity with respect to candidate symplectic structures when fitting dynamical systems to data.
This lets us train the model and enforce data-consistent symmetries simultaneously. 


\section{Related work}

\subsection{Enforcing symmetry}
\label{subsec:related_work_enforcing_symmetry}
Data-augmentation, as reviewed by \cite{shorten2019survey, van2001art}, is one of the simplest ways to incorporate known symmetry into machine learning models.
Usually this entails training a neural network architecture on training data to which known transformations have been applied.
The theoretical foundations of these methods are explored by \cite{Chen2020GroupAug}.
Data-augmentation has also been used by \cite{Benton2020learning} to construct equivariant neural networks by averaging the network's output over transformations applied to the data.

Symmetry can also be enforced directly on the machine learning architecture.
For example, Convolutional Neural Networks (CNNs), introduced by \cite{fukushima1980neocognitron} and popularized by \cite{lecun1989backpropagation}, achieve translational equivariance by employing convolutional filters with trainable kernels in each layer.
CNNs have been generalized to provide equivariance with respect to symmetry groups other than translation. 
Group-Equivariant CNNs (G-CNNs) \citep{Cohen2016GroupEqConv} provide equivariance with respect to arbitrary discrete groups generated by translations, reflections, and rotations. 
Rotational equivariance can be enforced on three-dimensional scalar, vector, or tensor fields using the 3D Steerable CNNs developed by \cite{weiler20183d}. 
Spherical CNNs \cite{cohen2018spherical, esteves2018learning} allow for rotation-equivariant maps to be learned for fields (such as projected images of 3D objects) on spheres. 
Essentially any group equivariant linear map (defining a layer of an equivariant neural network) acting fields can be described by group convolution \citep{Kondor2018generalization, Cohen2019general}, with the spaces of appropriate convolution kernel characterized by \cite{Cohen2019general}. 
\cite{Finzi2020generalizing} provides a practical way to construct convolutional layers that are equivariant with respect to arbitrary Lie groups and for general data types.
For dynamical systems, \cite{Marsden:MS,Rowley2003reduction, Abraham2008foundations} describe techniques for symmetry reduction of the original problem to a quotient space where the known symmetry group has been factored out.
Related approaches have been used by \cite{peitz2023partial, steyert2022uncovering} to approximate Koopman operators for symmetric dynamical systems (see \cite{Koopman1931Hamiltonian, Mezic2005spectral, Mauroy2020koopman, Otto2021koopman, Brunton2022siamreview}).

A general method for constructing equivariant neural networks is introduced by \cite{Finzi2021practical}, and relies on the observation that equivariance can be enforced through a set of linear constraints. 
For graph neural networks, \cite{maron2018invariant} characterizes the subspaces of linear layers satisfying permutation equivariance. 
Similarly, \cite{Ahmadi2020learning_short} shows that discrete symmetries and other types of side information can be enforced via linear or convex constraints in learning problems for dynamical systems.
Our work builds on the results of \cite{Finzi2021practical}, \cite{weiler20183d}, \cite{Cohen2019general}, and \cite{Ahmadi2020learning_short} by showing that equivariance can be enforced in a systematic and unified way via linear constraints for large classes of functions and neural networks.


\subsection{Discovering symmetry}
Early work by \cite{Rao1999learning, Miao2007learning} used nonlinear optimization to learn infinitesimal generators describing transformations between images. 
Later, it was recognized by \cite{Cahill2023Lie} that linear algebraic methods could be used to uncover the generators of continuous linear symmetries of arbitrary point clouds in Euclidean space. 
Similarly, \cite{Kaiser2018discovering} and \cite{Moskalev2022liegg} show how conserved quantities of dynamical systems and invariances of trained neural networks can be revealed by computing the nullspaces of associated linear operators.
We connect these linear-algebraic methods to the Lie derivative, and provide generalizations to nonlinear group actions on manifolds. 
The Lie derivative has been used by \cite{Gruver2022Lie} to quantify the extent to which a trained network is equivariant with respect to a given one-parameter subgroup of transformations.
Our results show how the Lie derivative can reveal the entire connected subgroup of symmetries of a trained model via symmetric eigendecomposition.

More sophisticated nonlinear optimization techniques use Generative Adversarial Networks (GANs) to learn the transformations that leave a data distribution unchanged. 
These methods include SymmetryGAN developed by \cite{Desai2022symmetry} and LieGAN developed by \cite{Yang2023generative}. 
In contrast, our methods for detecting symmetry are entirely linear-algebraic.

\cite{Liu2022hidden} discover hidden symmetries by optimizing nonlinear transformations into spaces where candidate symmetries hold. 
Similar to our approach for promoting symmetry, they use a cost function to measure whether a given symmetry holds. 
In contrast, our regularization functions enable subgroups of candidate symmetry groups to be identified.


\subsection{Promoting symmetry}

Biasing a network towards increased symmetry is discussed by \cite{Wang2022approximately}, along with architecture-specific methods, including regularization functions involving summations or integrals over the candidate group of symmetries.
While our regularization functions resemble these for discrete groups, we use a radically different regularization for continuous Lie groups.
By leveraging the Lie algebra, our regularization functions eliminate the need to numerically integrate complicated functions over the group --- a task that is already prohibitive for the $10$-dimensional non-compact group of Galilean symmetries in classical mechanics.

Automated data augmentation techniques introduced by \cite{Cubuk2019autoaugment, Hataya2020faster, Benton2020learning} are another class of methods that arguably promote symmetry.
These techniques optimize the distribution of transformations applied to augment the data during training. 
For example ``Augerino'' is an elegant method developed by \cite{Benton2020learning} which averages an arbitrary network's output over the augmentation distribution and relies on regularization to prevent the distribution of transformations from becoming concentrated near the identity. 
In essence, the regularization biases the averaged network towards increased symmetry.

In contrast, our regularization functions promote symmetry on an architectural level for the original network. 
This eliminates the need to perform averaging, which grows more costly for larger collections of symmetries. 
While a distribution over symmetries can be useful for learning interesting partial symmetries (e.g. $6$ stays $6$ for small rotations, before turning into $9$), as is done by \cite{Benton2020learning}, it is not clear how to use a continuous distribution over transformations to identify lower-dimensional subgroups which have measure zero. 
On the other hand, our linear-algebraic approach easily identifies and promotes symmetries in lower-dimensional connected subgroups.

\subsection{Additional approaches and applications}
There are several other approaches that incorporate various aspects of enforcing, discovering, and promoting symmetries. For example, \cite{baddoo2023physics} developed algorithms to enforce and promote known symmetries in dynamic mode decomposition, through manifold constrained learning and regularization, respectively.  \cite{baddoo2023physics} also showed that discovering unknown symmetries is a dual problem to enforcing symmetry.   
%
Exploiting symmetry has also been a central theme in the reduced-order modeling of fluids for decades~\citep{HLBR_turb}. As machine learning methods are becoming widely used to develop these models~\citep{Brunton2020arfm}, the themes of enforcing and discovering symmetries in machine models are increasingly relevant.  Known fluid symmetries have been enforced in SINDy for fluid systems~\citep{Loiseau2017jfm} through linear equality constraints; this approach was generalized to enforce more complex constraints~\citep{champion2020unified}. Unknown symmetries were similarly uncovered for electroconvective flows~\citep{guan2020sparse}. Symmetry breaking is also important in many turbulent flows~\citep{Callaham2022scienceadvances}. 



\section{Elementary theory of matrix Lie group actions}
\label{sec:background_on_matrix_Lie_groups}
This section provides background and notation required to understand the main results of this paper in the less abstract, but still remarkably useful setting of matrix Lie groups acting on vector spaces.
In Section~\ref{sec:fundamental_operators} we use this theory to study the symmetries of smooth functions between vector spaces.
Such functions form the basic building blocks of many machine learning models such as basis functions regression models, the layers of multilayer perceptrons, and the kernels of integral operators acting on spatial fields such as images.
We emphasize that this is not the most general setting for our results, but we provide this section and simpler versions of our main Theorems in Section~\ref{sec:fundamental_operators} in order to make the presentation more accessible.
We develop our main results in the more general and abstract setting of fiber-linear Lie group actions on sections of vector bundles in Section~\ref{sec:sections_of_vector_bundles}.

\subsection{Matrix Lie groups and subgroups}

Matrix Lie groups are ubiquitous in science and engineering.
Some familiar examples include the general linear group $GL(n)$ consisting of all real, invertible, $n\times n$ matrices; the orthogonal group
\begin{equation}
    O(n) = \left\{ Q \in \R^{n\times n} \ : \ Q^T Q = I \right\};
\end{equation}
and the special Euclidean group
\begin{equation}
    SE(n) = \left\{ \begin{bmatrix} Q & b \\ 0 & 1\end{bmatrix} \ : \ Q\in\R^{n\times n}, \ b \in \R^n, \ Q^T Q = I, \ \det(Q) = 1 \right\},
\end{equation}
which represents rotations and translations in an $n$-dimensional vector space. 
Observe that these sets are closed under matrix multiplication and inversion, making them into (non-commutative) groups.
They are also smooth manifolds, which makes them \emph{Lie groups}~\citep{MarsdenMTAA}.
While Lie groups need not consist of matrices (this general case is the setting in Section~\ref{sec:sections_of_vector_bundles}), we confine our attention for now to ``matrix Lie groups'', as they require less abstract machinery.
Following \citet{Hall2015Lie}, a matrix Lie group $G$ is defined as any closed subset of $GL(n)$ with the property that $G$ contains the identity matrix and is closed under matrix multiplication and matrix inversion.
All matrix Lie groups are smooth, embedded submanifolds of $GL(n)$ thanks to a result known as the closed subgroup theorem (Corollary~3.45, also see Theorem~20.12 in \citet{Lee2013introduction}).

The most useful and profound property of a Lie group is the fact that it is almost entirely characterized by an associated vector space called the \emph{Lie algebra}.
This allows global nonlinear questions about the group --- such as which elements leave a function unchanged --- to be answered using linear algebra.
The Lie algebra, denoted $\Lie(G)$, of a matrix Lie group $G$ is equal to the tangent space $T_I G$ to the manifold $G$ at the identity matrix $I$ (Corollary~3.46).
The identity element of a group is usually denoted $e$ for ``\textit{einselement}'', and we shall do this too.
This real vector space is closed under commutation, defined by the ``Lie bracket'':
\begin{equation}
    [X, Y] = X Y - Y X \in \Lie(G),
\end{equation}
for every $X,Y \in \Lie(G)$ (Theorem~3.20).
For example, the Lie algebra of the orthogonal group $O(n)$ consists of all skew-symmetric matrices, and is denoted
\begin{equation}
    \mathfrak{o}(n)
    = \left\{ S \in \R^{n\times n} \ : \ S + S^T = 0 \right\}.
\end{equation}

The key tool relating global properties of a matrix Lie group to its Lie algebra
is the matrix exponential, defined by the absolutely convergent series
\begin{equation}
    e^A = \sum_{k=0}^{\infty} \frac{1}{k!} A^k.
\end{equation}
Restricting the matrix exponential to the Lie algebra yields the exponential map 
$$\exp: \Lie(G) \to G,$$
which provides a diffeomorphism of an open neighborhood of the origin $0$ in $\Lie(G)$ and an open neighborhood of the identity element $e$ in $G$ (Corollary~3.44).
The connected component of $G$ containing the identity element is called the ``identity component'' of the Lie group and is denoted $G_0$.
Any element in this component can be expressed as a finite product of exponentials (Corollary~3.47), that is
\begin{equation}
    G_0 = \big\{ \exp{(X_1)} \cdots \exp{(X_N)} \ : \ X_1, \ldots, X_N \in \Lie(G), \ N = 1, 2, 3, \ldots \big\}.
\end{equation}
The identity component is a normal subgroup of $G$ (Proposition~1.10) and all of the other connected components of $G$ are diffeomorphic cosets of $G_0$ (Proposition~7.15 in \cite{Lee2013introduction}), as we illustrate in Figure~\ref{fig:Lie_group_action}.
For example, the special Euclidean group $SE(n)$ is connected, and thus equal to its identity component.
On the other hand, the orthogonal group $O(n)$ has two components consisting of orthogonal matrices $Q$ whose determinants are $1$ and $-1$.
The identity component of the orthogonal group is called the special orthogonal group and is denoted $SO(n)$.
The special orthogonal group is compact.
It is a general fact that when a Lie group is connected and compact, it is equal to the image of the exponential map without the need to consider products of exponentials, see \cite{Tao2011expsurjective} and Appendix~C.1 of \cite{Lezcano2019cheap}.

\begin{figure}[t]
    \centering
    \begin{tikzonimage}[trim=20 150 0 150, clip=true, width=0.9\textwidth]{Figures/Lie_group_action_v4.png}%[tsx/show help lines]
        %% Lie group
        \node[rotate=0] at (0.27, 0.92) {\footnotesize Lie group, $G$};
        % Identity component
        \node[rotate=0] at (0.475, 0.53) {\footnotesize $G_0$};
        \node[rotate=0, anchor=north] at (0.411, 0.71) {\footnotesize $T_e G \cong \Lie(G)$};
        \node[rotate=0, anchor=north] at (0.411, 0.81) {\footnotesize $e$};
        \draw[->] (0.411, 0.81) arc[radius=0.083, start angle=90, end angle=10];
        \node[rotate=0, anchor=west] at (0.50, 0.74) {\footnotesize $\exp(t\xi)$};
        \draw[->] (0.411, 0.81) -- (.50, 0.81);
        \node[rotate=0, anchor=west] at (0.50, 0.81) {\footnotesize $\xi$};
        % G1
        \node[rotate=0, anchor=north] at (0.312, 0.393) {\footnotesize $g_1$};
        \node[rotate=0] at (0.345, 0.155) {\footnotesize $g_1 G_0$};
        % G2
        \node[rotate=0, anchor=north] at (0.135, 0.80) {\footnotesize $g_2$};
        \node[rotate=0] at (0.20, 0.535) {\footnotesize $g_2 G_0$};
        %% Manifold
        \node[rotate=0, anchor=west] at (0.88, 0.14) {\footnotesize Manifold, $\mcal{M}$};
        \node[rotate=0] at (0.9, 0.665) {\footnotesize $T_x \mcal{M}$};
        \node[rotate=0, anchor=north] at (0.861, 0.570) {\footnotesize $x$};
        \draw[->] (0.861, 0.570) -- (0.861-0.075, 0.570+0.075);
        \node[rotate=0, anchor=east] at (0.788, 0.665) {\footnotesize $\hat{\theta}(\xi)(x)$};
        \draw[->] (0.861, 0.570) arc[radius=0.076, start angle=45, end angle=150];
        \node[rotate=0, anchor=east] at (0.742, 0.554) {\footnotesize $\theta_{\exp(t\xi)}(x)$};
        %% Action
        \draw[->] (0.51, 0.90) arc[radius=0.2, start angle=135, end angle=45];
        \node[rotate=0, anchor=south] at (0.66, 0.96) {\footnotesize Action, $\theta$};
    \end{tikzonimage}
    \vspace{-.15in}
    \caption{A Lie group $G$ and its action $\theta$ on a manifold $\mcal{M}$.
    The Lie group $G$ consists of three connected components with $G_0$ being the one that contains the identity element $e$.
    Each non-identity component of $G$ is a coset $g_i G_0$ formed by translating the identity component by an arbitrary element $g_i$ in the component.
    The Lie algebra $\Lie(G)$ is identified with the tangent space $T_e G$ and an exponential curve $\exp(t\xi)$ generated by an element $\xi \in \Lie(G)$ is shown. The infinitesimal generator $\hat{\theta}(\xi)$ is the vector field on $\mcal{M}$ whose flow corresponds with the action $\theta_{\exp(t\xi)}$ of group elements along $\exp(t\xi)$.}
    \label{fig:Lie_group_action}
\end{figure}


A subgroup $H$ of a Lie group $G$ is called a ``Lie subgroup'' when $H$ is an immersed submanifold of $G$ and the group operations are smooth when restricted to $H$.
An immersed submanifold does not necessarily inherit its topology as a subset of $G$, but rather $H$ has a topology and smooth structure such that the derivative of the inclusion $\imath_H : H \hookrightarrow G$ is injective (see \cite{Lee2013introduction}, and also Theorem~5.23).
When $G$ is a matrix Lie group in $GL(n)$, Lie subgroups $H$ cannot generally be regarded as matrix Lie groups in $GL(n)$, owing to the fact that they may not be closed in $GL(n)$ (see Section~5.9 in \cite{Hall2015Lie}).
When $H$ is a closed Lie subgroup of $G$, then $H$ is automatically a matrix Lie group in $GL(n)$.
Interestingly, it does turn out that even when $H$ is not closed in $GL(n)$, $H$ can always be embedded as a matrix Lie group in a larger $GL(n')$, $n' \geq n$ thanks to Theorem~9 in \cite{Goto1950faithful}.

The tangent space to a Lie subgroup $H \subset G$ at the identity, defined as $T_e H = \Range(\D \imath_H(I)) \subset \Lie(G)$, is closed under the Lie bracket and thus forms a ``Lie subalgebra'' of $\Lie(G)$, denoted $\Lie(H)$.
A fundamental result in Lie theory (Theorem~5.20) says that every Lie subalgebra $\mfrak{h} \subset \Lie(G)$ comes from a unique connected Lie subgroup of $G$. 
These subgroups will correspond to specific model symmetries in the context of machine learning. 

\subsection{Group representations, actions, and infinitesimal generators}

A matrix Lie group homomorphism is a smooth map $\Phi:G_1 \to G_2$ between matrix Lie groups that respects matrix multiplication, that is,
\begin{equation}
    \Phi(g_1 g_2) = \Phi(g_1) \Phi(g_2).
\end{equation}
The tangent map $\phi := \D \Phi(e): \Lie(G_1) \to \Lie(G_2)$ is a Lie algebra homomorphism (Theorem~3.28), meaning that it is a linear map respecting the Lie bracket:
\begin{equation}
    \phi\big( [\xi_1, \xi_2] \big) = \big[\phi(\xi_1), \phi(\xi_2)\big].
\end{equation}
Moreover, the Lie group homomorphism and its induced Lie algebra homomorphism are related by the exponential maps on $G_1$ and $G_2$ via the identity
\begin{equation}
    \Phi\big( \exp(\xi) \big) = \exp\big( \phi(\xi) \big).
    \label{eqn:intertwining_exp_and_rep}
\end{equation}
Another fundamental result (Theorem~5.6) is that any Lie algebra homomorphism $\Lie(G_1) \to \Lie(G_2)$ corresponds to a unique Lie group homomorphism $G_1 \to G_2$ when $G_1$ is simply connected.
When $G_2$ is the general linear group on a vector space, then the Lie group and Lie algebra homomorphisms are called Lie group and Lie algebra ``representations''.

A Lie group $G$ can act on a vector space $\mcal{V}$ via a representation $\Phi: G \to GL(\mcal{V})$ according to
\begin{equation}
    \theta: (x, g) \mapsto \Phi(g^{-1}) x,
    \label{eqn:right_action_by_representation}
\end{equation}
with $x\in\mcal{V}$ and $g \in G$.
More generally, a nonlinear right action of a Lie group $G$ on a manifold $\mcal{M}$ is any smooth map $\theta: \mcal{M} \times G \to \mcal{M}$ satisfying 
\begin{equation}
    \theta(\theta(x, g_1), g_2) = \theta(x, g_1 g_2) \qquad \mbox{and} \qquad \theta(x, e) = x
\end{equation}
for every $x\in\mcal{M}$ and $g_1, g_2\in G$.
Figure~\ref{fig:Lie_group_action} depicts the action of a Lie group on a manifold.
We make frequent use of the maps $\theta_g = \theta (\cdot, g)$, which have smooth inverses $\theta_{g^{-1}}$, and the ``orbit maps'' $\theta^{(x)} = \theta(x, \cdot)$.
For example, using a representation $\Phi : SE(3) \to GL(\R^7)$, the position $q$ and velocity $v$ of a particle in $\R^3$ 
can be rotated and translated via the action
\begin{equation*}
    \theta\left( 
    \begin{bmatrix} 
    Q & b \\ 
    0 & 1\end{bmatrix}, \ \begin{bmatrix}
        q \\
        v \\
        1
    \end{bmatrix} \right)
    =
    \Phi\left( \begin{bmatrix} 
        Q^T & -Q^T b \\ 
        0 & 1 \end{bmatrix} \right) 
    \begin{bmatrix}
        q \\
        v \\
        1
    \end{bmatrix}
    = 
    \begin{bmatrix} 
    Q^T & 0 & -Q^T b \\ 
    0 & Q^T & 0 \\ 
    0 & 0 & 1\end{bmatrix}
    \begin{bmatrix}
        q \\
        v \\
        1
    \end{bmatrix} = 
    \begin{bmatrix}
        Q^T (q - b) \\
        Q^T v \\
        1
    \end{bmatrix}.
\end{equation*}
The positions and velocities of $n$ particles arranged as a vector $(q_1, \ldots, q_n, v_1, \ldots, v_n, 1)$ can be simultaneously rotated and translated via an analogous representation $\Phi: SE(n) \to GL(\R^{6n + 1})$.
Group actions can be defined more generally, but to avoid abstraction, we temporarily confine ourselves to group actions defined by representations.
The general setting is considered later in Section~\ref{sec:sections_of_vector_bundles}.

The key fact about a group action is that it is almost completely characterized by a linear map called the \emph{infinitesimal generator}.
This map $\hat{\theta}$ assigns to each element $\xi\in\Lie(G)$ in the Lie algebra, a vector field $\hat{\theta}(\xi)$ on $\mcal{M}$ defined by
\begin{equation}
    \hat{\theta}(\xi)(x) 
    = \left.\ddt\right\vert_{t=0} \theta_{\exp(t\xi)}(x)
    = \D \theta^{(x)}(e) \xi.
\end{equation}
The infinitesimal generator and its relation to the group action are illustrated in Figure~\ref{fig:Lie_group_action}.
For the linear action in Eq.~\ref{eqn:right_action_by_representation}, the infinitesimal generator is the linear vector field $\hat{\theta}(\xi)(x) = -\phi(\xi) x$.
Crucially, the flow of the generator recovers the group action along the exponential curve $\exp(t\xi)$, i.e.,
\begin{equation}
    \Flow_{\hat{\theta}(\xi)}^t(x) 
    = \theta_{\exp(t\xi)}(x).
\end{equation}
For the linear right action in Eq.~\ref{eqn:right_action_by_representation}, this is easily verified by differentiation, applying Eq.~\ref{eqn:intertwining_exp_and_rep}, and the fact that solutions of smooth ordinary differential equations are unique.
For a nonlinear right action this follows from Lemma~20.14 and Proposition~9.13 in \cite{Lee2013introduction}.

\begin{remark}
    In contrast to a ``right'' action $\theta: \mcal{M} \times G \to \mcal{M}$, a ``left'' action $\theta: G \times \mcal{M} \to \mcal{M}$ satisfies ${\theta(g_2, \theta(g_1,x)) = \theta(g_2 g_1, x)}$.
    While our main results work for left actions too, e.g. $\theta(g,x) = \Phi(g) x$, right actions are slightly more natural because the infintesimal generator is a Lie alegbra homomorphism, i.e.,
    \begin{equation}
        \hat{\theta}([\xi, \eta]) = [\hat{\theta}(\xi), \hat{\theta}(\eta)],
    \end{equation}
    whereas this holds with a sign change for left actions.
    Every left action $\theta^L$ can be converted into an equivalent right action defined by $\theta^R(x,g) = \theta^L(g^{-1},x)$, and vice versa.
\end{remark}


\newpage
\section{Fundamental operators for studying symmetry}
\label{sec:fundamental_operators}


Here we introduce our main theoretical results for studying symmetries of machine learning models by focusing on a concrete and useful special case.
The basic building blocks of the machine learning models we consider here are smooth functions $F: \mcal{V} \to \mcal{W}$ between finite-dimensional vector spaces.
These functions could be layers of a multilayer neural network, integral kernels to be applied to spatio-temporal fields, or simply linear combinations of user-specified basis functions in a regression task as in~\cite{Brunton2016discovering}.
General versions of our results for smooth sections of vector bundles are developed later in Section~\ref{sec:sections_of_vector_bundles}.
Our main results show that two families of fundamental linear operators encode the symmetries of these functions.
The fundamental operators allow us to enforce, promote, and discover symmetry in machine learning models as we describe in Sections~\ref{sec:enforcing_symmetry},~\ref{sec:discovering_symmetry},~and~\ref{sec:promoting_symmetry}.

We consider representations $\Phi_{\mcal{V}}: G \to GL(\mcal{V})$ and $\Phi_{\mcal{W}}: G \to GL(\mcal{W})$ of a matrix Lie group $G$ along with their associated actions on the domain $\mcal{V}$ and codomain $\mcal{W}$ of $F$.
The definition of equivariance, the symmetry group of a function, and the first family of fundamental operators are introduced by the following:
\begin{definition}
    \label{def:equivariance_real_map_version}
    We say that $F$ is \textbf{equivariant} with respect to a group element $g\in G$ if
    \begin{empheq}[box=\widefbox]{equation}
        % F(\Phi_0(A) x) = \Phi_1(A) F(x)
        (\mcal{K}_g F)(x) 
        := \Phi_{\mcal{W}}(g) F(\Phi_{\mcal{V}}(g)^{-1} x) 
        = F(x)
        \label{eqn:transformation_operators_real_map}
    \end{empheq}
    for every $x \in \mcal{V}$.
    These elements form a subgroup of $G$ denoted $\Sym_G(F)$.
\end{definition}
The transformation operators $\mcal{K}_g$ are linear maps of smooth functions to smooth functions with addition and scalar multiplication defined point-wise.
These fundamental operators form a group with composition $\mcal{K}_{g} \mcal{K}_{h} = \mcal{K}_{g h}$ and inversion $\mcal{K}_g^{-1} = \mcal{K}_{g^{-1}}$.
Thus, $g \mapsto \mcal{K}_g$ is an infinite-dimensional representation of $G$.
These operators are useful for studying discrete symmetries of functions, as it is impractical to work directly with the uncountable family $\{ \mcal{K}_g \}_{g\in G}$ of a continuous group $G$.

The second family of fundamental operators are the key objects we use to study continuous symmetries of functions.
These are the Lie derivatives defined for each $\xi \in \Lie(G)$ by
\begin{empheq}[box=\widefbox]{equation}
    (\mcal{L}_{\xi} F)(x) 
    = \left.\ddt\right\vert_{t=0} \big( \mcal{K}_{\exp(t \xi)} F \big)(x) 
    = \phi_{\mcal{W}}(\xi) F(x) - \frac{\partial F(x)}{\partial x} \phi_{\mcal{V}}(\xi) x.
    \label{eqn:Lie_derivative_of_real_map}
\end{empheq}
We observe that the Lie derivative is linear with respect to both $\xi$ and $F$.
The fundamental operators and the construction of the Lie derivative are depicted in Figure~\ref{fig:baby_Lie_derivative}.
It turns out (see Proposition~\ref{prop:properties_of_Lie_derivative}) that $\xi \mapsto \mcal{L}_{\xi}$ is the Lie algebra representation corresponding to $g \mapsto \mcal{K}_g$,
meaning that
\begin{equation}
    \ddt \mcal{K}_{\exp(t\xi)} F 
    = \mcal{L}_{\xi} \mcal{K}_{\exp(t\xi)} F
    = \mcal{K}_{\exp(t\xi)} \mcal{L}_{\xi} F.
    \qquad \mbox{and} \qquad
    \mcal{L}_{[\xi,\eta]} = \mcal{L}_{\xi}\mcal{L}_{\eta} - \mcal{L}_{\eta} \mcal{L}_{\xi}.
\end{equation}
The first relation together with the closed subgroup theorem are key to proving our two main theoretical results in the setting of smooth maps between vector spaces.
The results stated below are special cases of more general results developed later in Section~\ref{sec:sections_of_vector_bundles}.

\begin{figure}
    \centering
    \vspace{-.3in}
    \begin{tikzonimage}[trim=20 150 100 150, clip=true, width=0.75\textwidth]{Figures/baby_Lie_derivative_v1.png}
        %% Subspace V
        \node[rotate=0] at (0.5, 0.3) {\footnotesize $\mcal{V}$};
        \node[rotate=0, anchor=north] at (0.275, 0.2) {\footnotesize $x$};
        \node[rotate=0, anchor=north] at (0.749, 0.165) {\footnotesize $ \Phi_{\mcal{V}}(\exp(t\xi))^{-1}x$};
        %% Subspace W (no. 1)
        \node[rotate=0, anchor=south] at (0.274, 0.83) {\footnotesize $\mcal{W}$};
        \node[rotate=0] at (0.260, 0.42) {\footnotesize $F(x)$};
        \draw[->] (0.240, 0.468) -- (0.330, 0.615);
        \node[rotate=0, anchor=west] at (0.335, 0.620) {\footnotesize $\mcal{L}_{\xi}F(x)$};
        \node[rotate=0, anchor=west] at (0.335, 0.790) {\footnotesize $\mcal{K}_{\exp(t\xi)}F(x)$};
        %% Subspace W (no. 2)
        \node[rotate=0, anchor=south] at (0.770, 0.83) {\footnotesize $\mcal{W}$};
        \node[rotate=0, anchor=west] at (0.790, 0.719) {\footnotesize $F\big( \Phi_{\mcal{V}}(\exp(t\xi))^{-1} x \big)$};
        %% between subspaces
        \draw[->] (0.750, 0.719) -- (0.345, 0.722);
        \node[rotate=0, anchor=north] at (0.600, 0.719) {\footnotesize $\Phi_{\mcal{W}}(\exp(t\xi))$};
    \end{tikzonimage}
    \vspace{-.2in}
    \caption{The fundamental operators for functions between vector spaces and linear Lie group actions defined by representations. 
    The finite transformation operators $\mcal{K}_{g}$ act on the function $F: \mcal{V} \to \mcal{W}$ by composing it with the linear transformation $\Phi_{\mcal{V}}(g)^{-1}$ and then applying $\Phi_{\mcal{W}}(g)$ to the values in $\mcal{W}$. The function is $g$-equivariant when this process does not alter the function. 
    The Lie derivative $\mcal{L}_{\xi}$ is formed by differentiating $t \mapsto \mcal{K}_{\exp(t\xi)}$ at $t=0$. 
    Geometrically, $\mcal{L}_{\xi} F(x)$ is the vector in $\mcal{W}$ tangent to the curve $t \mapsto \mcal{K}_{\exp(t\xi)} F (x)$ in $\mcal{W}$ passing through $F(x)$ at $t=0$.}
    \label{fig:baby_Lie_derivative}
\end{figure}

Our first main result provides necessary and sufficient conditions for a smooth map $F:\mcal{V} \to \mcal{W}$ to be equivariant with respect to the Lie group actions on $\mcal{V}$ and $\mcal{W}$.
This generalizes the constraints derived by \cite{Finzi2021practical} for the linear layers of equivariant multilayer perceptrons.
\begin{theorem}
    \label{thm:invariance_conditions_for_real_map}
    Let $\{ \xi_i \}_{i=1}^{\dim(G)}$ be a basis for the Lie algebra $\Lie(G)$ and let $\{ g_j \}_{j=1}^{n_G-1}$ contain one element from each non-identity component of $G$.
    Then a smooth map $F:\mcal{V} \to \mcal{W}$ is $G$-equivariant if and only if
    \begin{equation}
        \mcal{L}_{\xi_i} F = 0 \qquad \mbox{and} \qquad
        \mcal{K}_{g_j} F - F = 0
        \label{eqn:equivariance_everywhere_for_real_map}
    \end{equation}
    for every $i=1,\ldots,\dim(G)$ and every $j=1,\ldots, n_G - 1$.
    This is a special case of Theorem~\ref{thm:equivariance_conditions_for_vb_section}.
\end{theorem}
Since the fundamental operators $\mcal{L}_{\xi}$ and $\mcal{K}_g$ are linear, these yield linear constraints for a smooth map $F$ to be $G$-equivariant.

Our second main result shows that the continuous symmetries of a given smooth function $F:\mcal{V} \to \mcal{W}$ are encoded by its Lie derivatives.
\begin{theorem}
    \label{thm:symmetries_of_a_map}
    The symmetry group $\Sym_G(F)$ is a closed, embedded Lie subgroup of $G$ with Lie subalgebra
    \begin{equation}
        \sym_G(F) = \left\{ \xi \in \Lie(G) \ : \ \mcal{L}_{\xi} F = 0 \right\}.
    \end{equation}
    This is a special case of Theorem~\ref{thm:symmetries_of_sections}.
\end{theorem}
This result completely characterizes the identity component of the symmetry group $\Sym_G(F)$ because the connected Lie subgroups of $G$ are in one-to-one correspondence with Lie subalgebras of $\Lie(G)$ (see Theorem~5.20 in \cite{Hall2015Lie} or Theorem~19.26 in \cite{Lee2013introduction}).
Since the condition $\mcal{L}_{\xi} F = 0$ is linear with respect to $\xi$, the Lie subalgebra of the subgroup of symmetries of $F$ can be identified via linear algebra.
In particular it is found by computing the nullspace of the linear operator 
\begin{equation} \label{eqn:L_F_op_for_vector_space_case}
    L_F : \xi \mapsto \mcal{L}_{\xi} F.
\end{equation}
This operator is defined on the Lie algebra and takes values in the space $C^{\infty}(\mcal{V}; \mcal{W})$ of smooth functions $\mcal{V} \to \mcal{W}$.

The preceding two theorems already show the duality between enforcing and discovering continuous symmetries with respect to the Lie derivative, viewed as a bilinear form $(\xi, F) \mapsto \mcal{L}_{\xi} F$.
To discover symmetries, we seek generators $\xi \in \Lie(G)$ satisfying $\mcal{L}_{\xi} F = 0$ for a known function $F$.
On the other hand, to enforce a connected group of symmetries, we seek functions $F$ satisfying $\mcal{L}_{\xi_i} F = 0$ with known generators $\xi_1, \ldots, \xi_{\dim(G)}$ spanning $\Lie(G)$. 

\section{Enforcing symmetry with linear constraints}
\label{sec:enforcing_symmetry}
Methods to enforce symmetry in neural networks and other machine learning models have been studied extensively, as we reviewed briefly in Section~\ref{subsec:related_work_enforcing_symmetry}.
A unifying theme in these techniques has been the use of linear constraints to enforce symmetry~\citep{Finzi2021practical,Loiseau2017jfm,weiler20183d,Cohen2019general,Ahmadi2020learning_short}.
The purpose of this section is to show how several of these methods can be understood in terms of the fundamental operators and linear constraints provided by Theorem~\ref{thm:invariance_conditions_for_real_map}.

\subsection{Multilayer perceptrons}
Enforcing symmetry in multilayer percetrons was studied by \cite{Finzi2021practical}.
They provide a practical method based on enforcing linear constraints on the weights defining each layer of a neural network.
The network uses specialized nonlinearities that are automatically equivariant, meaning that the constraints need only be enforced on the linear component of each layer.
We show that the constraints derived by \cite{Finzi2021practical} are the same as those given by Theorem~\ref{thm:invariance_conditions_for_real_map}.

Specifically, each linear layer $F^{(l)}: \mcal{V}_{l-1} \to \mcal{V}_l$, for $l=1,\ldots,L$, is defined by
\begin{equation}
    F^{(l)}(x) = W^{(l)} x + b^{(l)},
\end{equation}
where $W^{(l)}$ are weight matrices and $b^{(l)}$ are bias vectors.
Defining group representations $\Phi_l: G \to GL(\mcal{V}_l)$ for each layer, yields fundamental operators given by
\begin{align}
    \mcal{K}_{g} F^{(l)}(x) - F^{(l)}(x) 
    &= \big( \Phi_{l}(g) W^{(l)} \Phi_{l-1}(g)^{-1} - W^{(l)} \big) x + \Phi_{l}(g) b^{(l)} - b^{(l)} \\
    \mcal{L}_{\xi} F^{(l)}(x) 
    &= \big(\phi_{l}(\xi) W^{(l)} - W^{(l)} \phi_{l-1}(\xi)\big) x +  \phi_{l}(\xi) b^{(l)}.
\end{align}
Let $\{ \xi_i \}_{i=1}^{\dim(G)}$ be a basis for $\Lie(G)$ and let $\{g_j\}_{j=1}^{n_G - 1}$ consist of an element from each non-identity component of $G$.
Using the fundamental operators and Theorem~\ref{thm:invariance_conditions_for_real_map}, it follows that the layer $F^{(l)}$ is $G$-equivariant if and only if the weights and biases satisfy 
\begin{equation}
    \phi_{l}(\xi_i) W^{(l)} = W^{(l)} \phi_{l-1}(\xi_i), 
    \quad \mbox{and} \quad
    \Phi_{l}(g_j) W^{(l)} = W^{(l)} \Phi_{l-1}(g_j),
\end{equation}
\begin{equation}
    \phi_{l}(\xi_i) b^{(l)} = 0,
    \quad \mbox{and} \quad
    \Phi_{l}(g_j) b^{(l)} = b^{(l)}
\end{equation}
for every $i = 1, \ldots,\dim(G)$ and $j = 1, \ldots, n_g-1$.
These are the same as the linear constraints one derives using the method by \cite{Finzi2021practical}.
The equivariant linear layers are then combined with specialized equivariant nonlinearities $\sigma^{(l)}: \mcal{V}_l \to \mcal{V}_l$
to produce an equivariant network
\begin{equation}
    F = \sigma^{(L)} \circ F^{(L)} 
    \circ \cdots \circ 
    \sigma^{(1)} \circ F^{(1)}: \mcal{V}_0 \to \mcal{V}_L.
\end{equation}
The composition of equivariant functions is equivariant, as one can easily check using Definition~\ref{def:equivariance_real_map_version}.

\subsection{Neural networks acting on fields}
\label{subsec:NNs_acting_on_fields}
Enforcing symmetry in neural networks acting on spatial fields has been studied extensively by \cite{weiler20183d, cohen2018spherical, esteves2018learning, Kondor2018generalization, Cohen2019general} among others.
Many of these techniques use integral operators to define equivariant linear layers, which are coupled with equivariant nonlinearities, such as the gated nonlinearities proposed by \cite{weiler20183d}.
The key task is to identify appropriate bases for equivariant kernels.
For certain groups, such as the Special Euclidean group $G = SE(3)$, bases can be constructed explicitly using spherical harmonics, as in \cite{weiler20183d}. 
We show that equivariance with respect to arbitrary group actions can be enforced via linear constraints on the integral kernels derived using the fundamental operators introduced in Section~\ref{sec:fundamental_operators}.
Appropriate bases of kernel functions can then be constructed numerically by computing an appropriate nullspace, as is done by \cite{Finzi2021practical} for multilayer perceptrons.

For the sake of simplicity we consider integral operators acting on vector-valued functions $F:\R^m\to \mcal{V}$, where $\mcal{V}$ is a finite-dimensional vector space.
Later on in Section~\ref{subsec:equivariant_integral_operators} we study higher-order integral operators acting on sections of vector bundles.
If $\mcal{W}$ is another finite-dimensional vector space, an integral operator acting on $F$ to produce a new function $\R^n\to \mcal{W}$ is defined by
\begin{equation}
    \mcal{T}_K F(x) = \int_{\R^m} K(x,y) F(y) \td y,
    \label{eqn:intergral_operator_on_Rn}
\end{equation}
where $K(x,y): \mcal{V} \to \mcal{W}$ are linear maps.
In other words, this kernel is a function $K : \R^n \times \R^m \to \mcal{W} \otimes \mcal{V}^*$, where $\mcal{V}^*$ denotes the dual space of $\mcal{V}$.
In machine learning applications, we seek to optimize the kernel functions $K$ defining one or more layers of a neural network.

With group actions defined by representations on $\R^m, \R^n, \mcal{V}, \mcal{W}$, functions $F:\R^m \to \mcal{V}$ transform according to 
\begin{equation}
    \mcal{K}_g^{(\R^m,\mcal{V})} F(x) = \Phi_{\mcal{V}}(g) F(\Phi_{\R^m}(g)^{-1} x)
\end{equation}
for $g \in G$.
Likewise, 
functions $\R^n \to \mcal{W}$ transform via an analogous operator $\mcal{K}_g^{(\R^n,\mcal{W})}$.
\begin{definition}
    \label{def:equivariance_of_integral_operator_linear_case}
    The integral operator $\mcal{T}_K$ in Eq.~\ref{eqn:intergral_operator_on_Rn} is \textbf{equivariant} with respect to $g \in G$ when
    \begin{equation}
         \mcal{K}_{g}^{(\R^n,\mcal{W})} \circ \mcal{T}_K \circ \mcal{K}_{g^{-1}}^{(\R^m,\mcal{V})} = \mcal{T}_K.
    \end{equation}
    The elements $g$ satisfying this equation form a subgroup of $G$ denoted $\Sym_G(\mcal{T}_K)$.
\end{definition}
By changing variables in the integral, the operator on the left is given by
\begin{equation}
    \mcal{K}_{g}^{(\R^n,\mcal{W})} \circ \mcal{T}_K \circ \mcal{K}_{g^{-1}}^{(\R^m,\mcal{V})} F(x)
    = \int_{\R^m} \mcal{K}_g K(x,y) F(y) \td y,
\end{equation}
where
\begin{empheq}[box=\widefbox]{equation}
    \mcal{K}_g K (x,y) 
    = \Phi_{\mcal{W}}(g) K\big(\Phi_{\R^n}(g)^{-1} x, \Phi_{\R^m}(g)^{-1} y\big) \Phi_{\mcal{V}}(g)^{-1}  \det\big[ \Phi_{\R^m}(g)^{-1} \big].
    \label{eqn:transformation_for_linear_integral_kernels_on_Rn}
\end{empheq}
The following result provides equivariance conditions in terms of the kernel, generalizing Lemma~1 in \cite{weiler20183d}.
\begin{proposition}
    \label{prop:symmetries_of_linear_integral_operator}
    Let $K$ be continuous and suppose that $\mcal{T}_K$ acts on a function space containing all smooth, compactly supported fields. Then 
    \begin{equation}
        \Sym_G(\mcal{T}_K) = \left\{ g \in G \ : \ \mcal{K}_g K = K \right\}.
    \end{equation}
    We give a proof in Appendix~\ref{app:proofs_of_minor_results}
\end{proposition}

The Lie derivative of the kernel is given by
\begin{empheq}[box=\widefbox]{multline}
    \mcal{L}_{\xi} K(x,y) 
    = \phi_{\mcal{W}}(\xi)K(x,y) 
    - K(x,y) \phi_{\mcal{V}}(\xi)
    - K(x,y) \Tr[\phi_{\R^m}(\xi)] \\
    - \frac{\partial K(x,y)}{\partial x} \phi_{\R^n}(\xi)x 
    - \frac{\partial K(x,y)}{\partial y} \phi_{\R^m}(\xi)y
    \label{eqn:Lie_derivative_for_linear_integral_kernels_on_Rn}
\end{empheq}
The operators $\mcal{K}_g$ and $\mcal{L}_{\xi}$ are the fundamental operators from Section~\ref{sec:fundamental_operators} because 
the transformation law for the kernel can be written as
\begin{equation}
    \mcal{K}_g K 
    = \Phi_{\mcal{W} \otimes \mcal{V}^*}(g) K \circ \Phi_{\R^m\times \R^m}(g)^{-1},
\end{equation}
where 
\begin{equation}
\begin{aligned}
    \Phi_{\R^n\times \R^m}(g) &: (x,y) \mapsto \left(\Phi_{\R^n}(g) x, \Phi_{\R^m}(g) y\right), \\
    \Phi_{\mcal{W} \otimes \mcal{V}^*}(g) &: T \mapsto \Phi_{\mcal{W}}(g) T \Phi_{\mcal{V}}(g)^{-1} \det\big[ \Phi_{\R^m}(g)^{-1} \big]
\end{aligned}
\end{equation}
are representations of $G$ in $\R^m\times \R^m$ and $\mcal{W} \otimes \mcal{V}^*$.

As an immediate consequence of Theorem~\ref{thm:invariance_conditions_for_real_map}, we have the following corollary establishing linear constraints for the kernel to produce an equivariant integral operator.
\begin{corollary}
    Let $\{ \xi_i \}_{i=1}^{\dim(G)}$ be a basis for the Lie algebra $\Lie(G)$ and let $\{ g_j \}_{j=1}^{n_G-1}$ contain one element from each non-identity component of $G$.
    Under the same hypotheses as Proposition~\ref{prop:symmetries_of_linear_integral_operator}, the integral operator $\mcal{T}_K$ in Eq.~\ref{eqn:intergral_operator_on_Rn} is $G$-equivariant in the sense of Definition~\ref{def:equivariance_of_integral_operator_linear_case} if and only if
    \begin{equation}
        \mcal{L}_{\xi_i} K = 0 \qquad \mbox{and} \qquad
        \mcal{K}_{g_j} K - K = 0
    \end{equation}
    for every $i=1,\ldots,\dim(G)$ and every $j=1,\ldots, n_G - 1$.
\end{corollary}
These linear constraint equations must be satisfied to enforced equivariance with respect to a known symmetry $G$ in the machine learning process.
By discretizing the operators $\mcal{K}_g$ and $\mcal{L}_{\xi}$, as discussed later in Section~\ref{sec:discretization}, one can solve these constraints numerically to construct a basis of kernel functions for equivariant integral operators.

As an immediate consequence of Theorem~\ref{thm:symmetries_of_a_map}, the following result shows that the Lie derivative encodes the continuous symmetries of a given integral kernel.
\begin{corollary}
    Under the same hypotheses as Proposition~\ref{prop:symmetries_of_linear_integral_operator},
    $\Sym_G(\mcal{T}_K)$ is a closed, embedded Lie subgroup of $G$ whose Lie subalgebra is
    \begin{equation}
        \sym_G(\mcal{T}_K) = \left\{ \xi\in\Lie(G) \ : \ \mcal{L}_{\xi} K = 0 \right\}.
    \end{equation}
\end{corollary}
This result will be useful for methods that promote symmetry of the integral operator, as we describe later in Section~\ref{sec:promoting_symmetry}.


\section{Discovering symmetry by computing nullspaces}
\label{sec:discovering_symmetry}
In this section we show that in a wide range of settings, the continuous symmetries of a manifold, point cloud, or map can be recovered by computing the nullspace of a linear operator.
For functions, this is already covered by Theorem~\ref{thm:symmetries_of_a_map}, which allows us to compute the connected subgroup of symmetries by identifying its Lie subalgebra
\begin{equation}
    \sym_G(F) = \Null(L_F)
\end{equation}
where $L_F: \xi \mapsto \mcal{L}_{\xi}$ is the linear operator defined by Eq.~\ref{eqn:L_F_op_for_vector_space_case}.
Hence, if a machine learning model $F$ has a symmetry group $\Sym_G(F)$, then its Lie algebra is equal to the nullspace of $L_F$.

This section explains how this is actually a special case of a more general result allowing us to reveal the symmetries of submanifolds via the nullspace of a closely related operator.
We begin with the more general case where we study the symmetries of a submanifold of Euclidean space, and we explain how to recover symmetries from point clouds approximating submanifolds.
The Lie derivative described in Section~\ref{sec:fundamental_operators} is then recovered when the submanifold is the graph of a function.
We also briefly describe how the fundamental operators from Section~\ref{sec:fundamental_operators} can be used to recover symmetries and conservation laws of dynamical systems.

\subsection{Symmetries of submanifolds}
\label{subsec:submanifolds_of_Rd}
We begin by studying the symmetries of submanifolds $\mcal{M}$ of Euclidean space $\R^d$ using an approach similar to \cite{Cahill2023Lie}.
However, we use a different operator that generalizes more naturally to nonlinear group actions on arbitrary manifolds (see Section~\ref{sec:submanifolds_and_tangency}) and recovers the Lie derivative (see Section~\ref{eqn:functions_as_submanifolds}).
With a representation $\Phi:G \to GL(\R^d)$ of a Lie group, we define invariance of a submanifold as follows:
\begin{definition}
    A submanifold $\mcal{M}\subset\R^d$ is \textbf{invariant} with respect to a group element $g\in G$ if
    \begin{equation}
        \Phi(g)z \in \mcal{M}
    \end{equation}
    for every $z\in\mcal{M}$.
    These elements form a subgroup of $G$ denoted $\Sym_G(\mcal{M})$.
\end{definition} 
The subgroup of symmetries of a submanifold is characterized by the following theorem.
\begin{theorem} \label{thm:submanifolds_of_Rd}
    Let $\mcal{M}$ be a smooth, closed, embedded submanifold of $\R^d$.
    Then $\Sym_G(\mcal{M})$ is a closed, embedded Lie subgroup of $G$ whose Lie subalgebra is 
    \begin{equation}
        \sym_G(\mcal{M}) = \{ \xi \in \Lie(G) \ : \ \phi(\xi)z \in T_z \mcal{M} \quad \forall z\in\mcal{M} \}.
    \end{equation}
    This is a special case of Theorem~\ref{thm:symmetries_of_a_submanifold}.
\end{theorem}
The meaning of this result and its practical use for detecting symmetry are illustrated in Figure~\ref{fig:manifold_tangent_symmetry}.

To reveal the connected component of $\Sym_G(\mcal{M})$, we let $P_z:\R^d \to \R^d$ 
be a family of linear projections onto $T_z \mcal{M} \subset \R^d$.
These are assumed to vary continuously with respect to $z\in\mcal{M}$.
Then under the assumptions of the above theorem, $\sym_G(\mcal{M})$ is the nullspace of the symmetric, positive-semidefinite operator $S_{\mcal{M}}:\Lie(G) \to \Lie(G)$ defined by
\begin{empheq}[box=\widefbox]{equation}
    \big\langle \eta, \ S_{\mcal{M}} \xi \big\rangle_{\Lie(G)} 
    = \int_{\mcal{M}} z^T \phi(\eta)^T (I - P_z)^T (I - P_z) \phi(\xi) z \ \td \mu(z)
    \label{eqn:symmetry_operator_for_submanifolds_of_Rd}
\end{empheq}
for every $\eta,\xi\in\Lie(G)$.
We see in Figure~\ref{fig:manifold_tangent_symmetry} that $(I - P_z) \phi(\xi) z$ measures the component of the infinitesimal generator not tangent to the submanifold at $z$.
Here, $\mu$ is any strictly positive measure on $\mcal{M}$ that makes all of these integrals finite.
The above formula is useful for computing the matrix of $S_{\mcal{M}}$ in an orthonormal basis for $\Lie(G)$.

\begin{figure}
    \centering
    \begin{tikzonimage}[trim=250 200 150 150, clip=true, width=0.45\textwidth]{Figures/manifold_tangent.png}
        %% manifold and tangent space
        \node[rotate=0] at (0.40, 0.65) {\footnotesize $\mcal{M}$};
        \node[rotate=0,anchor=west] at (0.675, 0.70) {\footnotesize $T_z \mcal{M}$};
        %% tangent vectors
        \def\zx{0.655}; \def\zy{0.593};
        \def\vx{-0.7*0.050}; \def\vy{-0.7*0.440};
        \node[rotate=0,anchor=south] at (\zx, \zy) {\footnotesize $z$};
        \draw[->] (\zx, \zy) -- (\zx+\vx, \zy+\vy);
        \node[rotate=0,anchor=west] at (\zx+\vx+0.01, \zy+\vy) {\footnotesize $\hat{\theta}(\xi)(z) = \phi(\xi)z$};
        %% group action
        \draw[->] (\zx, \zy) arc[radius=0.56, start angle=-6, end angle=-45];
        \node[rotate=0,anchor=east] at (0.5, 0.27) {\footnotesize $\theta_{\exp(t\xi)}(z)$};
    \end{tikzonimage}
    \hspace{-0.25cm}
    \begin{tikzonimage}[trim=250 200 150 150, clip=true, width=0.45\textwidth]{Figures/manifold_tangent.png}
        %% manifold and tangent space
        \node[rotate=0] at (0.30, 0.25) {\footnotesize $\mcal{M}$};
        \node[rotate=0,anchor=north] at (0.690, 0.380) {\footnotesize $T_z \mcal{M}$};
        %% tangent vectors
        \def\zx{0.655}; \def\zy{0.593};
        \def\Pvx{0.025}; \def\Pvy{0.220};
        \def\Nvx{0.15}; \def\Nvy{0.00};
        \def\vx{\Pvx+\Nvx}; \def\vy{\Pvy+\Nvy};
        \node[rotate=0,anchor=north] at (\zx, \zy) {\footnotesize $z$};
        \draw[->] (\zx, \zy) -- (\zx+\Pvx, \zy+\Pvy);
        \node[rotate=0,anchor=south] at (\zx+\Pvx, \zy+\Pvy) {\footnotesize $P_z \hat{\theta}(\xi)(z)$};
        \draw[->] (\zx, \zy) -- (\zx+\Nvx, \zy+\Nvy);
        \node[rotate=0,anchor=west] at (\zx+\Nvx, \zy+\Nvy) {\footnotesize $(I-P_z) \hat{\theta}(\xi)(z)$};
        \draw[->] (\zx, \zy) -- (\zx+\vx, \zy+\vy);
        \node[rotate=0,anchor=west] at (\zx+\vx, \zy+\vy) {\footnotesize $\hat{\theta}(\xi)(z) = \phi(\xi) z$};
        %% group action
        \draw[->] (\zx, \zy) arc[radius=1.4, start angle=-36, end angle=-19];
        \node[rotate=0,anchor=west] at (0.850, 0.99) {\footnotesize $\theta_{\exp(t\xi)}(z)$};
    \end{tikzonimage}
    \caption{Tangency of infinitesimal generators and symmetries of submanifolds. The infinitesimal generator $\hat{\theta}(\xi)$ is everywhere tangent to the submanifold $\mcal{M}$ if and only if the curves $t\mapsto\theta_{\exp(t\xi)}(z)$, with $z\in\mcal{M}$, lie in $\mcal{M}$ for all $t$. 
    The Lie algebra elements $\xi$ satisfying this tangency condition form the Lie subalgebra of symmetries of $\mcal{M}$. 
    To test for tangency of the infinitesimal generator we use a family of projections $P_z$ onto the tangent spaces $T_z \mcal{M}$ for every $z\in\mcal{M}$.
    Specifically, $(I - P_z) \hat{\theta}(\xi)$ is the component of the infinitesimal generator that does not lie tangent to $\mcal{M}$. 
    Hence, $\xi$ generates a symmetry of $\mcal{M}$ if and only if $(I - P_z) \hat{\theta}(\xi) = 0$ for all $z\in\mcal{M}$.}
    \label{fig:manifold_tangent_symmetry}
\end{figure}

Alternatively, when the dimension of $G$ is large, one can compute the nullspace using a Krylov algorithm such as the one described in \cite{Finzi2021practical}.
Such algorithms rely solely on queries of $S_{\mcal{M}}$ acting on vectors $\xi\in\Lie(G)$, which can be obtained explicitly by
\begin{equation}
    S_{\mcal{M}} \xi = \int_{\mcal{M}} \D\Phi(e)^*\left[ (I-P_z)^T (I-P_z) \phi(\xi) z z^T \right] \ \td \mu(z),
\end{equation}
where $\D\Phi(e)^*:\R^{d\times d} \to \Lie(G)$ is the adjoint of $\D\Phi(e):\Lie(G) \to \R^{d\times d}$.

In practice, one can use sample points $z_i$ on the manifold to obtain a Monte-Carlo estimate of $S_{\mcal{M}}$ with approximate projections $P_{z_i}$ computed using local principal component analysis (PCA), as described in \cite{Cahill2023Lie}.
More accurate estimates of the tangent spaces can be obtained using the methods in \cite{Berry2020spectral}.
Assuming the $P_{z_i}$ are accurate, the results of Section~\ref{sec:discretization}, below, can be used to show that the correct nullspace is obtained from finitely many sample points $z_i$ almost surely.

\subsection{Symmetries of functions as symmetries of submanifolds}
\label{eqn:functions_as_submanifolds}

The method described above for studying symmetries of submanifolds can be applied to reveal the symmetries of smooth maps between vector spaces by identifying the map $F:\mcal{V} \to \mcal{W}$ with its graph
\begin{equation}
    \graph(F) = \{ (x, F(x)) \in \mcal{V}\times\mcal{W} \ : \ x \in \mcal{V} \}.
\end{equation}
The graph is a smooth, closed, embedded submanifold of the space $\mcal{V}\times \mcal{W}$ by Proposition~5.7 in \cite{Lee2013introduction}.
We show that this approach recovers the Lie derivative and our result in Theorem~\ref{thm:symmetries_of_a_map}.
By choosing bases for the domain and codomain, it suffices to consider smooth functions $F:\R^m\to\R^n$.

Supposing that we have representations $\Phi_{\R^m}$ and $\Phi_{\R^n}$ of $G$ in the domain and codomain, we consider a combined representation
\begin{equation}
    \Phi: g \mapsto \begin{bmatrix}
        \Phi_{\R^m}(g) & 0 \\
        0 & \Phi_{\R^n}(g)
    \end{bmatrix}.
\end{equation}
Defining a smoothly-varying family of projections 
\begin{equation}
    P_{(x,F(x))} = \begin{bmatrix}
        I & 0 \\
        \D F(x) & 0
    \end{bmatrix}
    \label{eqn:graph_projection}
\end{equation}
onto $T_{(x,F(x))}\graph(F)$,
it is easy to check that
\begin{equation}
    \begin{bmatrix}
        0 \\
        \mcal{L}_\xi F (x)
    \end{bmatrix}
    = \underbrace{\left( \begin{bmatrix}
        I & 0 \\
        0 & I
    \end{bmatrix} - \begin{bmatrix}
        I & 0 \\
        \D F(x) & 0
    \end{bmatrix} \right)}_{I-P_{(x,F(x))}} 
    \underbrace{
    \begin{bmatrix}
        \phi_{\R^m}(\xi) & 0 \\
        0 & \phi_{\R^n}(\xi)
    \end{bmatrix}}_{\phi(\xi)}
    \begin{bmatrix}
        x \\
        F(x)
    \end{bmatrix}.
    \label{eqn:Lie_derivative_as_projection_for_real_maps}
\end{equation}
We note that this is a special case of Theorem~\ref{thm:characterization_of_Lie_derivative} describing the Lie derivative in terms of a projection onto the tangent space of a function's graph.
The resulting operator $S_{\graph(F)}$ defined by Eq.~\ref{eqn:symmetry_operator_for_submanifolds_of_Rd} is given by
\begin{equation}
    \left\langle \eta, S_{\graph(F)} \xi \right\rangle_{\Lie(G)} 
    = \int_{\R^m} (\mcal{L}_\eta F(x))^T \mcal{L}_\xi F(x) \ \td \mu(x),
\end{equation}
for $\eta,\xi\in\Lie(G)$ and an appropriate positive measure $\mu$ on $\R^m$ that makes the integrals finite.
Therefore, Theorem~\ref{thm:symmetries_of_a_map} is recovered from our result about symmetries of submanifolds stated in Theorem~\ref{thm:submanifolds_of_Rd}.

Related quantities have been used to study the symmetries of trained neural networks, with the $F$ being the network and its derivatives computed via back-propagation.
The quantity $\left\langle \xi, S_{\graph(F)} \xi \right\rangle_{\Lie(G)} = \Vert \mcal{L}_\xi F \Vert_{L^2(\mu)}$ was used by \citet{Gruver2022Lie} to construct the Local Equivariant Error or (LEE), measuring the extent to which a trained neural network $F$ fails to respect symmetries in the one-parameter group $\{ \exp(t\xi) \}_{t\in\R}$.
The nullspace of $\xi \mapsto \mcal{L}_{\xi} F$ in the special case where $\Phi_{\R^n}(g) = I$ acts trivially was used by \cite{Moskalev2022liegg} to identify the connected subgroup with respect to which a given network is invariant.

By viewing a function as a submanifold, we obtain a simple data-driven technique for estimating the Lie derivative and subgroup of symmetries of the function.
To approximate $\mcal{L}_\xi F$, $S_{\graph(F)}$, and $\sym_G(F)$ using input-output pairs $(x_i, y_i=F(x_i))$, one simply needs to approximate the projection in Eq.~\ref{eqn:graph_projection} using these data.
To do this, we can obtain matrices $U_i$ with $m$ columns spanning $T_{(x_i,y_i)} \graph(F)$ by applying local PCA to the data $z_i = (x_i, y_i)$, or by pruning the frames computed in \cite{Berry2020spectral}.
With $E = \begin{bmatrix}
    I_{m\times m} & 0_{m\times n}
\end{bmatrix}$ the projection in Eq.~\ref{eqn:graph_projection} is given by 
\begin{equation}
% $
    P_{z_i} = U_i (E U_i)^{-1} E
% $
\end{equation}
because any projection is uniquely determined by its range and nullspace (see Section~5.9 of \cite{Meyer2000matrix}).
This gives us a simple way to approximate $(\mcal{L}_\xi F)(z_i)$, $S_{\graph(F)}$, and $\sym_G(F)$ using the input-output pairs.
However, many such pairs are needed since the tangent space to the graph of $F$ at $x_i$ is well-approximated by local PCA only when there are at least $m$ neighboring samples sufficiently close to $x_i$.
Even more samples are needed when they are noisy.
The convergence properties of the spectral methods in \cite{Berry2020spectral} are better, but they still require enough samples to obtain accurate Monte-Carlo or quadrature-based estimates of integrals, in this case over $\R^m$.

\subsection{Symmetries and conservation laws of dynamical systems}
\label{subsec:dynamical_systems}
Here, we consider the case when $F:\R^n \to \R^n$ is a smooth function defining a dynamical system
\begin{equation}
    \ddt x(t) = F(x(t))
    \label{eqn:ds}
\end{equation}
with state variables $x(t)\in\R^n$.
The solution of this equation is described by the flow map $\Flow: (t, x(\tau)) \mapsto x(\tau + t)$, which is defined on a maximal connected open set $D$ containing $0\times\R^n$.
In many cases we write $\Flow^t(\cdot) = \Flow(t, \cdot)$.
Given a Lie group representation $\Phi: G \to GL(\R^n)$, equivariance for the dynamical system is defined as follows:
\begin{definition}
    The dynamical system in Eq.~\ref{eqn:ds} is \textbf{equivariant} with respect to a group element $g\in G$ if the flow map satisfies
    \begin{equation}
        \mcal{K}_g \Flow^t (x) := \Phi(g) \Flow^t(\Phi(g)^{-1} x) = \Flow^t(x)
    \end{equation}
    for every $(t,x) \in D$.
\end{definition}
Differentiating at $t=0$ shows that equivariance of the dynamical system implies that $F$ is equivariant in the sense of Definition~\ref{def:equivariance_real_map_version}.
The converse is also true thanks to Corollary~9.14 in \cite{Lee2013introduction}, meaning that equivariance for the dynamical system is equivalent to equivariance of $F$.
Therefore, we can study equivariance of the dynamical system in Eq.~\ref{eqn:ds} by directly applying the tools developed in Section~\ref{sec:fundamental_operators} to the function $F$.
Thanks to Theorem~\ref{thm:symmetries_of_a_map}, identifying the connected subgroup of symmetries for the dynamical system is a simple matter of computing the nullspace of the linear map $\xi \mapsto \mcal{L}_{\xi} F$, that is
\begin{equation}
    \sym_G(F) = \{ \xi \in \Lie(G) \ : \ \mcal{L}_{\xi} F = 0 \}.
\end{equation}
Here, the Lie derivative is given by 
\begin{equation}
    \mcal{L}_{\xi} F(x) 
    = \phi(\xi) F(x) - \frac{\partial F(x)}{\partial x} \phi(\xi) x
    = [\hat{\theta}(\xi), F](x),
\end{equation}
where $[\hat{\theta}(\xi), F]$ is the Lie bracket of the infinitesimal generator defined by $\hat{\theta}(\xi)(x) = -\phi(\xi) x$ and the vector field $F$.
Symmetries can also be enforced as linear constraints on $F$ described by Theorem~\ref{thm:invariance_conditions_for_real_map}.
This was done by \cite{Ahmadi2020learning_short} for polynomial dynamical systems with discrete symmetries.
Later on in Section~\ref{subsec:symmetry_of_vf} we show that analogous results apply to dynamical systems defined by vector fields on manifolds and nonlinear Lie group actions.

A conserved quantity for the system in Eq.~\ref{eqn:ds} is defined as follows:
\begin{definition}
    A scalar valued quantity $f: \R^n \to \R$ is said to be \textbf{conserved} when
    \begin{equation}
        \mcal{K}_t f(x) := f(\Flow^t(x)) = f(x) \qquad \forall (t,x) \in D.
    \end{equation}
\end{definition}
In this setting, the composition operators $\mcal{K}_t$ are often referred to as Koopman operators (see \cite{Koopman1931Hamiltonian, Mezic2005spectral, Mauroy2020koopman, Otto2021koopman, Brunton2022siamreview}).
It is easy to see that a smooth function $f$ is conserved if and only if
\begin{equation}
    \mcal{L}_F f := \left.\ddt\right\vert_{t=0} \mcal{K}_t f =  \frac{\partial f}{\partial x} F = 0.
\end{equation}
This relation is used by \cite{Kaiser2018discovering,kaiser2021data} to identify conserved quantities by computing the nullspace of $\mcal{L}_F$ restricted to finite-dimensional spaces of candidate functions.
When the flow is defined for all $t\in\R$, the operators $\mcal{K}_{t}$ and $\mcal{L}_F$ can be viewed as the fundamental operators for the nonlinear action of the flow map described later on in Section~\ref{sec:sections_of_vector_bundles}.
These are the fundamental operators from Section~\ref{sec:fundamental_operators} for representations $\Phi_{\mcal{V}}(t) = e^{-At}$ and $\Phi_{\mcal{W}}(t) = I$ of the Lie group $G=(\R,+)$ when $F(x) = A x$ is a linear dynamical system.

\begin{remark}
    For Hamiltonian dynamical systems Noether's theorem establishes a remarkable equivalence between the symmetries of the Hamiltonian and conserved quantities of the system.
    We study Hamiltonian systems later in Section~\ref{subsec:Hamiltonian_conservation_laws}.
\end{remark}

\section{Promoting symmetry with convex penalties}
\label{sec:promoting_symmetry}
In this section we show how to design custom convex regularization functions to promote symmetries within a given candidate group during training of a machine learning model.
This allows us to train a model with as many symmetries as possible from among the candidates, while breaking candidate symmetries only when the data provides sufficient evidence.
We study both discrete and continuous groups of candidate symmetries.
We quantify the extent to which symmetries within the candidate group are broken using the fundamental operators described in Section~\ref{sec:fundamental_operators}.
For discrete groups we use the transformation operators $\{\mcal{K}_g\}_{g\in G}$ and for continuous groups we use the Lie derivatives $\{\mcal{L}_{\xi}\}_{\xi\in\Lie(G)}$.
In the continuous case we penalize a convex relaxation of the codimension of the subgroup of symmetries given by a nuclear norm (Schatten $1$-norm) of the operator $\xi \mapsto \mcal{L}_{\xi} F$ defined by Eq.~\ref{eqn:L_F_op_for_vector_space_case}; minimizing this codimension via the proxy nuclear norm will promote the largest nullspace possible, and hence the largest admissible symmetry group.
Once these regularization functions are developed abstractly in Sections~\ref{subsec:discrete_symmetries}~and~\ref{subsec:continuous_symmetries}, we show how the approach can be applied to basis function regression (Section~\ref{subsec:promoting_symmetry_in_regression}) and neural networks (Section~\ref{subsec:multilayer_perceptrons}).

As in Section~\ref{sec:fundamental_operators}, the basic building blocks of the machine learning models we consider are smooth functions $F: \mcal{V} \to \mcal{W}$ between finite-dimensional vector spaces.
While we consider this restricted setting here, our results readily generalize to smooth sections of vector bundles, as we describe later in Section~\ref{sec:sections_of_vector_bundles}.
These functions could be layers of a multilayer perceptron, integral kernels to be applied to spatio-temporal fields, or simply linear combinations of user-specified basis functions in a regression task.
One does not usually optimize over all smooth functions, instead constraining $F$ to lie in a given subspace $\mcal{F} \subset C^{\infty}(\mcal{V};\mcal{W})$.
Working within a finite-dimensional subspace of smooth functions will be important when we seek to discretize the fundamental operators in Section~\ref{sec:discretization}

The candidate symmetries are described by a Lie group $G$ acting on the domain and codomain of functions $F \in \mcal{F}$ via group representations $\Phi_{\mcal{V}}:G\to GL(\mcal{V})$ and $\Phi_{\mcal{W}}: G \to GL(\mcal{W})$.
Equivariance in this setting is described by Definition~\ref{def:equivariance_real_map_version}.
When fitting the function $F$ to data, our regularization functions penalize the size of $G \setminus \Sym_G(F)$.
For reasons that will become clear, we use different penalties corresponding to different notions of ``size'' when $G$ is a discrete group versus when $G$ is continuous.
The main result describing the continuous symmetries of $F$ is Theorem~\ref{thm:symmetries_of_a_map}.

\subsection{Discrete symmetries}
\label{subsec:discrete_symmetries}
When the group $G$ has finitely many elements, one can measure the size of $G \setminus \Sym_G(F)$ simply by counting its elements:
\begin{equation}
    R_{G,0}(F) = \vert G \setminus \Sym_G(F) \vert.
\end{equation}
However, this penalty is impractical for optimization owing to its discrete values and nonconvexity.
Letting $\Vert \cdot \Vert$ be any norm on the space $\mcal{F}' = \vspan \{ \mcal{K}_g F \ : \ g\in G, \ F \in \mcal{F} \}$ yields a convex relaxation of the above penalty given by
\begin{empheq}[box=\widefbox]{equation}
    R_{G,1}(F) = \sum_{g\in G} \Vert  \mcal{K}_g F - F \Vert.
    \label{eqn:discrete_penalty_for_real_maps}
\end{empheq}
This is a convex function on $\mcal{F}$ because $\mcal{K}_g$ is a linear operator.
For example, if $\vect{c} = (c_1, \ldots, c_N)$ are the coefficients of $F$ in a basis for $\mcal{F}'$ and $\mat{K}_g$ is the matrix of $\mcal{K}_g$ in this basis, then the Euclidean norm can be used to define
\begin{equation}
    R_{G,1}(F) = \sum_{g\in G} \Vert \mat{K}_g \vect{c} - \vect{c} \Vert_2.
\end{equation}
This is directly analogous to the group sparsity penalty proposed in \cite{Yuan2006model}.

\subsection{Continuous symmetries}
\label{subsec:continuous_symmetries}
We now consider the case where $G$ is a Lie group of dimension greater than zero.
Here we use the dimension of $\Sym_G(F)$ to measure the symmetry of $F$, seeking to penalize the complementary dimension or ``codimension'', given by
\begin{equation}
    R_{G,0}(F) = \codim(\Sym_G(F)) = \dim(G) - \dim(\Sym_G(F)).
\end{equation}
We take this approach in the continuous case because it is no longer possible to simply count the number of broken symmetries.
While is is possible in principle to replace the sum in Eq.~\ref{eqn:discrete_penalty_for_real_maps} by an integral, the numerical quadrature required to approximate it becomes prohibitive for higher-dimensional candidate groups.
This is complicated even further by the fact that $\Sym_G(F)$ is a set of measure zero in $G$ whenever $\dim(\Sym_G(F)) < \dim (G)$ and our empirical observation that $\Vert F - \mcal{K}_g F \Vert$ often increases very rapidly as $g$ leaves the set $\Sym_G(F)$.

The dimension of $\Sym_G(F)$ is equal to that of its Lie algebra.
Thanks to Theorem~\ref{thm:symmetries_of_a_map}, this is the nullspace of a linear operator $L_F : \Lie(G) \to C^{\infty}(\mcal{V};\mcal{W})$ defined by
\begin{equation}
    L_F: \xi \mapsto \mcal{L}_{\xi} F,
\end{equation}
where $\mcal{L}_{\xi}$ is the Lie derivative in Eq.~\ref{eqn:Lie_derivative_of_real_map}.
By the rank and nullity theorem, the codimension of $\Sym_G(F)$ is equal to the rank of this operator:
\begin{equation}
    R_{G,0}(F) 
    = \codim(\Sym_G(F))
    = \rank(L_F).
\end{equation}
Penalizing the rank of an operator is impractical for optimization owing to its discrete values and nonconvexity.
A commonly used convex relaxation of the rank is provided by the Schatten $1$-norm, also known as the ``nuclear norm'', given by
\begin{empheq}[box=\widefbox]{equation}
    R_{G,*}(F) 
    = \Vert L_{F} \Vert_{*} 
    = \sum_{i=1}^{\dim(G)} \sigma_i(L_F).
    \label{eqn:nuclear_norm_penalty_for_real_maps}
\end{empheq}
Here $\sigma_i(L_F)$ denotes the $i$th singular value of $L_F$ with respect to inner products on $\Lie(G)$ and $\mcal{F}'' = \vspan\{ \mcal{L}_{\xi} F \ : \ \xi\in\Lie(G), \ F\in\mcal{F} \}$.
For certain rank minimization problems, penalizing the nuclear norm is guaranteed to recover the true minimum rank solution \citep{Recht2010guaranteed, Gross2011recovering}.

Eq.~\ref{eqn:nuclear_norm_penalty_for_real_maps} is a convex function on $\mcal{F}$ because $F \mapsto L_F$ is linear.
For example, if $(c_1, \ldots, c_N)$ are the coefficients of $F$ in a basis $\{ F_1, \ldots, F_N \}$ for $\mcal{F}$ and $\mat{L}_{F_i}$ are the matrices of $L_{F_i}$ in orthonormal bases for $\Lie(G)$ and $\mcal{F}''$, then
\begin{equation}
    R_{G,*}(F) = \Vert c_1 \mat{L}_{F_1} + \cdots + c_N \mat{L}_{F_N} \Vert_*.
\end{equation}
With $\{ \xi_1, \ldots, \xi_{\dim(G)} \}$ and $\{ u_{1}, \ldots, u_{N''} \}$ being the orthonormal bases for $\Lie(G)$ and $\mcal{F}''$, one can compute a store the rank-$3$ tensor $[\mat{L}_{F_i}]_{j,k} = \left\langle u_j,\ \mcal{L}_{\xi_k} F_i \right\rangle_{\mcal{F}''}$.

\subsection{Promoting symmetry in basis function regression}
\label{subsec:promoting_symmetry_in_regression}
Here we consider regression problems for maps $F:\R^m\to\R^n$ of the form $F(x) = W \fndict(x)$ where $\fndict:\R^m \to \R^N$ is a dictionary consisting of user-specified smooth functions and $W\in\R^{n\times N}$ is a weight matrix we seek to fit.  For example, the sparse identification of nonlinear dynamics (SINDy) algorithm~\citep{Brunton2016discovering} belongs to this type of learning, among other machine learning algorithms~\citep{Brunton2022book}. 
For such maps we have
\begin{align}
    (\mcal{K}_g F)(x) - F(x) &= \Phi_{\R^n}(g)^{-1} W \fndict(\Phi_{\R^m}(g)x) - W\fndict(x) \\
    (\mcal{L}_\xi F)(x) &= W\frac{\partial \fndict(x)}{\partial x}\phi_{\R^m}(\xi)x - \phi_{\R^n}(\xi) W \fndict(x),
\end{align}
which can be used in Eq.~\ref{eqn:discrete_penalty_for_real_maps} and Eq.~\ref{eqn:nuclear_norm_penalty_for_real_maps} to construct symmetry-promoting regularization functions $R_G(W)$ that are convex with respect to the weight matrix $W$.
Given a collection of data consisting of input-output pairs $\{ (x_j, y_j) \}_{j=1}^M$ we can seek a regularized least-squares fit by solving the convex optimization problem
\begin{equation}
    \minimize_{W\in\R^{n\times N}} \frac{1}{M}\sum_{j=1}^M \Vert y_j - W\fndict(x_j) \Vert^2 + \gamma R_G(W\mcal{D}).
    \label{eqn:basis_function_regression_prob}
\end{equation}
Here, $\gamma \geq 0$ is a parameter controlling the strength of the regularization that can be determined using cross-validation.

\begin{remark}
    The solutions $F = W\mcal{D}$ of Eq.~\ref{eqn:basis_function_regression_prob} do not depend on how the dictionary functions are normalized due to the fact that the function being minimized can be written entirely in terms of $F$ and the data $(x_j,y_j)$.
    This is in contrast to other types of regularized regression problems that penalize the weights $W$ directly, and therefore depend on how the functions in $\mcal{D}$ are normalized.
\end{remark}

\subsection{Promoting symmetry in neural networks}
\label{subsec:multilayer_perceptrons}
In this section we describe a convex regularizing penalty to promote $G$-equivariance in feed-forward neural networks
\begin{equation}
    F = F^{(L)} \circ \cdots \circ F^{(1)}
\end{equation}
composed of layers 
$F^{(l)}: \mcal{V}_{l-1} \to \mcal{V}_l$
with group representations 
$\Phi_l : G \to GL(\mcal{V}_l)$.
Since the composition is $g$-equivariant if every layer is $g$-equivariant, the main idea is to measure the symmetries shared by all of the layers.
Specifically, we aim to maximize the ``size'' of the subgroup
\begin{equation}
    \bigcap_{l=1}^L \Sym_G\big(F^{(l)}\big) 
    = \{ g \in G \ : \ \mcal{K}_g F^{(l)} = F^{(l)},\ l=1, \ldots, L \}
    \subset \Sym_G(F),
    \label{eqn:intersection_subgroup}
\end{equation}
where the notion of ``size'' we adopt depends on whether $G$ is discrete or continuous.
The same ideas can be applied to neural networks acting on fields with layers defined by integral operators as described in Section~\ref{subsec:NNs_acting_on_fields}.
In this case we consider symmetries shared by all of the integral kernels.

We consider the case in which the trainable layers are elements of vector spaces $\mcal{F}_l$, over which the optimization is carried out.
For example, each layer may be given by $F^{(l)} = W^{(l)}\fndict^{(l)}$ as in Section~\ref{subsec:promoting_symmetry_in_regression}, where $W^{(l)}$ is a trainable weight matrix and $\fndict^{(l)}$ is a fixed dictionary of nonlinear functions.
Alternatively, we could follow \cite{Finzi2021practical} and use trainable linear layers composed with fixed $G$-equivariant nonlinearities.
In contrast with \cite{Finzi2021practical}, we do not force the linear layers to be $G$-equivariant.
Rather, we penalize the breaking of $G$-symmetries in the linear layers as a means to regularize the neural network and to learn which subgroup of symmetries are compatible with the data and which are not.

As in Section~\ref{subsec:discrete_symmetries}, when $G$ is a discrete group with finitely many elements, a convex relaxation of the cardinality of $G \setminus \bigcap_{l=1}^L \Sym_G(F^{(l)})$ is
\begin{empheq}[box=\widefbox]{equation}
    R_{G,1}\big(F^{(1)}, \ldots, F^{(l)}\big)
    = \sum_{g\in G} \sqrt{ \sum_{l=1}^L \big\Vert \mcal{K}_g F^{(l)} - F^{(l)} \big\Vert^2 }.
\end{empheq}
Again, this is analogous to the group-LASSO penalty developed in \citet{Yuan2006model}.

When $G$ is a Lie group with nonzero dimension, we follow the approach in Section~\ref{subsec:continuous_symmetries} using the following observation:
\begin{proposition}
    \label{prop:intersection_subalgebra}
    The subgroup in Eq.~\ref{eqn:intersection_subgroup} is closed and embedded in $G$; its Lie subalgebra is
    \begin{equation}
        \bigcap_{l=1}^L \sym_G\big(F^{(l)}\big) = \left\{ \xi \in \Lie(G) \ : \ \mcal{L}_{\xi} F^{(l)} = 0, \ l=1, \ldots, L \right\}.
    \end{equation}
    We provide a proof in Appendix~\ref{app:proofs_of_minor_results}.
\end{proposition}
The Lie subalgebra in the proposition is equal to the nullspace of the linear operator $L_{F^{(1)},\ldots,F^{(L)}} : \Lie(G) \to \bigoplus_{l=1}^L C^{\infty}(\mcal{V}_{l-1}; \mcal{V}_l)$ defined by 
\begin{equation}
    L_{F^{(1)},\ldots,F^{(L)}}: \xi \mapsto
    \big( \mcal{L}_\xi F^{(1)}, \ldots, \mcal{L}_\xi F^{(L)} \big).
    \label{eqn:operator_for_multiple_real_maps}
\end{equation}
By the rank and nullity theorem, minimizing the rank of this operator is equivalent to maximizing the dimension of the subgroup of symmetries shared by all of the layers in the network.
As in Section~\ref{subsec:continuous_symmetries}, a convex relaxation of the rank is provided by the nuclear norm
\begin{empheq}[box=\widefbox]{equation}
    R_{G,*}\big( F^{(1)}, \ldots, F^{(l)} \big)
    = \big\Vert L_{F^{(1)},\ldots,F^{(L)}} \big\Vert_* 
    = \left\Vert 
    \begin{bmatrix}
        \mat{L}_{F^{(1)}} \\
        \vdots \\
        \mat{L}_{F^{(L)}}
    \end{bmatrix}
    \right\Vert_*,
\end{empheq}
where $\mat{L}_{F^{(l)}}$ are the matrices of $L_{F^{(l)}}$ in orthonormal bases for $\Lie(G)$ and the associated spaces $\mcal{F}_l''$.


\section{Discretizing the operators}
\label{sec:discretization}
This section describes how to construct matrices for the operators $\mcal{L}_\xi$ and $L_F$ for smooth functions $F$ in a user-specified finite-dimensional subspace $\mcal{F} \subset C^{\infty}(\mcal{V};\mcal{W})$.
By choosing bases for the the finite-dimensional vector spaces $\mcal{V}$ and $\mcal{W}$, it suffices without loss of generality to consider the case in which $\mcal{V} = \R^m$ and $\mcal{W} = \R^n$.
We focus on constructing the matrix of $\mcal{L}_\xi$ since the matrix of $\mcal{K}_g$ can be obtained in an analogous way with respect to appropriate bases.
We assume that $\Lie(G)$ and $\mcal{F}$ are endowed with inner products and that $\{ \xi_1, \ldots \xi_{\dim(G)} \}$ and $\{ F_1, \ldots F_{\dim(\mcal{F})} \}$ are orthonormal bases for these spaces, respectively.
The key task is to endow the subspace
\begin{equation}
    \mcal{F}' = \vspan \left\{ \mcal{L}_\xi F \ : \ \xi\in\Lie(G),\ F\in\mcal{F} \right\} \subset C^{\infty}(\R^m;\R^n)
\end{equation}
with a convenient inner product.
Once this is done, an orthonormal basis $\{ u_1, \ldots, u_N \}$ can be constructed for $\mcal{F}'$ by applying a Gram-Schmidt process to the functions $\mcal{L}_{\xi_i} F_j$.
Matrices for $\mcal{L}_\xi$ and $L_F$ are then easily obtained by computing
\begin{equation}
    \big[ \mat{\mcal{L}}_\xi \big]_{i,j} = \big\langle u_i, \ \mcal{L}_\xi F_j \big\rangle_{\mcal{F}'}, \qquad
    \big[ \mat{L}_F \big]_{i,k} = \big\langle u_i, \ \mcal{L}_{\xi_k} F \big\rangle_{\mcal{F}'}.
\end{equation}
The issue at hand is to choose the inner product on $\mcal{F}'$ in a way that makes computing these matrices easy.
A natural choice is to equip $\mcal{F}'$ with an $L^2(\R^m, \mu; \R^n)$ inner product where $\mu$ is a positive measure on $\R^m$ (such as a Guassian distribution) for which the $L^2$ norms of function in $\mcal{F}'$ are finite.
The problem is that it is usually challenging or inconvenient to compute the required inner products
\begin{equation}
    \big\langle \mcal{L}_{\xi_i} F_j,\ \mcal{L}_{\xi_k} F_l \big\rangle_{L^2(\mu)} = \int_{\R^m} \big(\mcal{L}_{\xi_i} F_j\big)(x)^T \big(\mcal{L}_{\xi_k} F_l\big)(x) \td\mu(x)
    \label{eqn:L2_mu_inner_prod_of_Lie_derivs}
\end{equation}
analytically.
In this section we discuss inner products that are easy to compute in practice.

\subsection{Numerical quadrature and Monte-Carlo}
When Eq.~\ref{eqn:L2_mu_inner_prod_of_Lie_derivs} cannot be computed analytically, 
one can resort to a numerical quadrature or Monte-Carlo approximation.
In both cases the integral is approximated by a weighted sum, yielding a semi-inner product
\begin{equation}
    \langle f, \ g \rangle_{L^2(\mu_M)} = \frac{1}{M}\sum_{i=1}^M w_i f(x_i)^T g(x_i)
    \label{eqn:empirical_L2_mu_inner_prod}
\end{equation}
that converges to $\langle f, \ g \rangle_{L^2(\mu)}$ as $M\to\infty$ for $f,g$ within certain function spaces.
The following proposition means that we do not have to pass to the limit $M\to\infty$ in order to obtain a valid inner product defined by Eq.~\ref{eqn:empirical_L2_mu_inner_prod} on $\mcal{F}'$.
\begin{proposition}
    \label{prop:Monte_Carlo_eventually_gives_an_inner_product}
    Suppose that $\mcal{F}'$ is a finite-dimensional and $\langle f, \ g \rangle_{L^2(\mu_M)} \to \langle f, \ g \rangle_{L^2(\mu)}$ as $M\to\infty$ for every $f,g\in\mcal{F}'$.
    Then there is an $M_0$ such that 
    Eq.~\ref{eqn:empirical_L2_mu_inner_prod} is an inner product on $\mcal{F}'$ for every $M \geq M_0$.
    We give a proof in Appendix~\ref{app:proofs_of_minor_results}.
\end{proposition}

For example, in Monte-Carlo approximation, the samples $x_i$ are drawn independently from a distribution $\nu$ with the assumption that both $\mu$ and $\nu$ are $\sigma$-finite and $\mu$ is absolutely continuous with respect to $\nu$.
The weights are given by the Radon-Nikodym derivative $w_i = \frac{\td \mu}{\td \nu}(x_i)$.
Then for every $f, g\in L^2(\mu)$ the approximate integral converges $\langle f, \ g \rangle_{L^2(\mu_M)} \to \langle f, \ g \rangle_{L^2(\mu)}$ as $M\to\infty$ almost surely thanks to the strong law of large numbers (see Theorem~7.7 in \cite{Koralov2012theory}).
By the proposition, there is almost surely a finite $M_0$ such that Eq.~\ref{eqn:empirical_L2_mu_inner_prod} is an inner product on $\mcal{F}'$ for every $M \geq M_0$.


\subsection{Subspaces of polynomials}
Here we consider the special case when $\mcal{F}$ is a finite-dimensional subspace consisting of polynomial functions $\R^m \to \R^n$.
Examining the expression in Eq.~\ref{eqn:Lie_derivative_of_real_map}, it is evident that $\mcal{L}_\xi F$ is also a polynomial function $\R^m \to \R^n$ with degree not greater than that of $F\in\mcal{F}$.
Thus, $\mcal{F}'$ is also a space of polynomial functions with degree not exceeding the maximum degree in $\mcal{F}$.
Since a polynomial that vanishes on an open set must be identically zero, we can take the integrals defining the inner product in Eq.~\ref{eqn:empirical_L2_mu_inner_prod} over a cube, such as $[0,1]^m\subset \R^m$.
This is convenient because polynomial integrals over the cube can be calculated analytically.

We can also use the sample-based inner product in Eq.~\ref{eqn:empirical_L2_mu_inner_prod} with randomly chosen points $x_i$ and positive weights $w_i$.
The following proposition tells us exactly how many sample points we need.
\begin{proposition}
    \label{prop:generic_polynomial_map_sampling}
    Let $\mcal{F}'$ be a space of real polynomial functions $\R^m \to \R^n$ and let $\pi_i:\R^n \to \R$ be the $i$th coordinate projection $\pi(c_1, \ldots, c_n) = c_i$.
    Let 
    \begin{equation}
        M \geq M_0 = \max_{1\leq i \leq n} \dim(\pi_i (\mcal{F}'))
    \end{equation}
    and let $w_1, \ldots, w_M > 0$ be positive weights.
    Then for almost every set of points $(x_1, \ldots, x_M) \in (\R^m)^M$ with respect to Lebesgue measure, Eq.~\ref{eqn:empirical_L2_mu_inner_prod} is an inner product on $\mcal{F}'$.  
    We give a proof in Appendix~\ref{app:generic_polynomial_map_sampling}.
\end{proposition}
This means that we can draw $M \geq M_0$ sample points independently from any absolutely continuous measure (such as a Gaussian distribution or the uniform distribution on a cube), and with probability one,  Eq.~\ref{eqn:empirical_L2_mu_inner_prod} will be an inner product on $\mcal{F}'$.
When $\mcal{F}$ consists of polynomials with degree at most $d$, then taking
\begin{equation}
    M \geq \sum_{k=0}^d \binom{k+m-1}{m-1}
\end{equation}
is sufficient.

\section{Generalization to sections of vector bundles}
\label{sec:sections_of_vector_bundles}
The machinery for promoting, discovering, and enforcing symmetry of maps $F:\mcal{V} \to \mcal{W}$ between finite-dimensional vector spaces is a special case of more general machinery for sections of vector bundles presented here.
Applications of this more general framework include studying the symmetries of vector and tensor fields on manifolds with respect to nonlinear group actions \citep{MarsdenMTAA}.
Our analysis pertains to all finite-dimensional real Lie groups, and is not confined to matrix Lie groups acting via representations as in Section~\ref{sec:promoting_symmetry}.
We rely heavily on background, definitions, and results that can be found in \cite{Lee2013introduction} and \cite{Kolar1993natural}.
All manifolds and maps are assumed to be smooth, that is infinitely continuously differentiable, $C^{\infty}$.

First, we provide some background on vector bundles that can be found in \citet[Chapter~10]{Lee2013introduction}.
A rank-$k$ vector bundle $E$ is a collection of $k$-dimensional vector spaces $E_p$, called ``fibers'', organized smoothly over a base manifold $\mcal{M}$.
This fibers are organized by the ``bundle projection'' $\pi:E \to \mcal{M}$, a surjective map whose preimages are the fibers $E_p = \pi^{-1}(p)$.
The smoothness of this organization means that $\pi$ is a smooth submersion where $E$ is a smooth manifold covered by smooth local trivializations
$$
    \psi_{\alpha}: \pi^{-1}(\mcal{U}_{\alpha})\subset E \to \mcal{U}_{\alpha} \times \R^k
$$
with $\{\mcal{U}_{\alpha}\}_{\alpha\in\mcal{A}}$ being open subsets covering $\mcal{M}$.
The transition functions between local trivializations are $\R^k$-linear, meaning that there are smooth matrix-valued functions $\mat{T}_{\alpha, \beta}: \mcal{U}_{\alpha}\cap\mcal{U}_{\beta} \to \R^{k\times k}$ satisfying
\begin{equation}
    \psi_{\alpha} \circ \psi_{\beta}^{-1}(p, \vect{v})
    = (p, \mat{T}_{\alpha, \beta}(p)\vect{v})
\end{equation}
for every $p \in \mcal{U}_{\alpha}\cap\mcal{U}_{\beta}$ and $\vect{v} \in \R^k$.
The bundle with this structure is often denoted $\pi: E \to \mcal{M}$.

A ``section'' of the rank-$k$ vector bundle $\pi:E \to \mcal{M}$ is a map $F: \mcal{M} \to E$ satisfying $\pi \circ F = \Id_{\mcal{M}}$.
The space of smooth sections, denoted $\sect(E)$, is a vector space with addition and scalar multiplication defined pointwise in each fiber $E_p$.
A vector bundle and a section are depicted in Figure~\ref{fig:Lie_derivative}, along with the fundamental operators for a group action that we introduce below.

\begin{figure}[h]
    \centering
    \begin{tikzonimage}[trim=120 80 200 170, clip=true, width=0.8\textwidth]{Figures/Lie_derivative_v1.png}
        %% Base manifold
        \node[rotate=0] at (0.6, 0.1) {\footnotesize $\mcal{M}$};
        \node[rotate=0, anchor=north] at (0.260, 0.2) {\footnotesize $p$};
        \node[rotate=0, anchor=north] at (0.675, 0.285) {\footnotesize $ q = \theta_{\exp(t\xi)}(p)$};
        %% Ep
        \node[rotate=0, anchor=south] at (0.24, 0.85) {\footnotesize $E_p$};
        \draw[->] (0.261, 0.205) -- (0.23, 0.45);
        \node[rotate=0] at (0.24, 0.49) {\footnotesize $F(p)$};
        \draw[->] (0.225, 0.530) -- (0.320, 0.660);
        \node[rotate=0, anchor=west] at (0.335, 0.675) {\footnotesize $\mcal{L}_{\xi}F(p)$};
        \node[rotate=0, anchor=east] at (0.325, 0.775) {\footnotesize $\mcal{K}_{\exp(t\xi)}F(p)$};
        %% Subspace Eq
        \node[rotate=0] at (0.640, 0.90) {\footnotesize $E_q$};
        \draw[->] (0.672, 0.285) -- (0.686, 0.73);
        \node[rotate=0] at (0.690, 0.775) {\footnotesize $F(q)$};
        %% between subspaces
        \draw[->] (0.690, 0.818) -- (0.34, 0.763);
        \node[rotate=0, anchor=south] at (0.500, 0.800) {\footnotesize $\Theta_{\exp(-t\xi)}$};
        %% Bundle
        \node[rotate=0] at (0.87, 0.6) {\footnotesize $E$};
        \draw[->] (0.87, 0.55) -- (0.87, 0.25);
        \node[rotate=0, anchor=west] at (0.87, 0.4) {\footnotesize $\pi$};
    \end{tikzonimage}
    \caption{Fundamental operators for sections of vector bundles equipped with fiber-linear Lie group actions. 
    The action $\Theta_{g}$ is linear on each fiber $E_p$ and descends under the bundle projection $\pi: E \to \mcal{M}$ to an action $\theta_g$ on $\mcal{M}$. 
    Given a section $F: \mcal{M} \to E$, the finite transformation operators $\mcal{K}_g$ produce a new section $\mcal{K}_g F$ whose value at $p$ is given by evaluating $F$ at $q=\theta_g(p)$ and pulling the value in $E_{q}$ back to $E_p$ via the linear map $\Theta_{g^{-1}}$ on $E_q$. 
    The operators $\mcal{K}_g$ are linear thanks to linearity of $\Theta_{g^{-1}}$ on every $E_q$. 
    The Lie derivative $\mcal{L}_{\xi}$ is the operator on sections formed by differentiating $t\mapsto \mcal{K}_{\exp(t\xi)}$ at $t=0$. 
    Geometrically, $\mcal{L}_{\xi} F (p)$ is the vector in $E_p$ lying tangent to the curve $t\mapsto \mcal{K}_{\exp(t\xi)} F(p)$ in $E_p$ passing through $F(p)$ at $t=0$.}
    \label{fig:Lie_derivative}
\end{figure}

We consider a ``\textbf{fiber-linear}'' right $G$-action $\Theta : E \times G \to E$, meaning that every $\Theta_g = \Theta(\cdot, g):E \to E$ is a vector bundle homomorphism.
In other words, $\Theta$ descends under the bundle projection $\pi$ to a unique right $G$-action $\theta : \mcal{M} \times G \to \mcal{M}$ so that the diagram
\begin{equation}
    \begin{tikzcd}
        E \arrow[r, "\Theta_g"] \arrow[d, "\pi"'] & E \arrow[d, "\pi"] \\ %\arrow[r, "\phi_{\xi}"] & \R \\
        \mcal{M} \arrow[r, "\theta_g"'] & \mcal{M}
    \end{tikzcd}
    \label{cd:action_by_vb_homomorphisms}
\end{equation}
commutes and the restricted maps $\Theta_g\vert_{E_p}: E_p \to E_{\theta(p,g)}$ are linear.
We define what it means for a section to be symmetric with respect to this action as follows:
\begin{definition}
    \label{def:equivariance_of_section}
    A section $F\in\Sigma(E)$ is \textbf{equivariant} with respect to a transformation $g\in G$ if
    \begin{equation}
        \mcal{K}_g F := \Theta_{g^{-1}} \circ F \circ \theta_{g} = F.
        \label{eqn:transformation_operators_vb}
    \end{equation}
    These transformations form a subgroup of $G$ denoted $\Sym_G(F)$.
\end{definition}
The operators $\mcal{K}_{g}$ are depicted in Figure~\ref{fig:Lie_derivative}.
Thanks to the vector bundle homomorphism properties of $\Theta_{g^{-1}}$, the operators $\mcal{K}_g: \sect(E) \to \sect(E)$ are well-defined and linear.
Moreover, they form a group under composition $\mcal{K}_{g_1} \mcal{K}_{g_2} = \mcal{K}_{g_1\cdot g_2}$,
with inverses given by $\mcal{K}_g^{-1} = \mcal{K}_{g^{-1}}$.

The ``infinitesimal generator'' of the group action is the linear map $\hat{\Theta}:\Lie(G) \to \vf(E)$ defined by
\begin{equation}
    \hat{\Theta}(\xi) = \left. \ddt \right\vert_{t=0} \Theta_{\exp(t\xi)}.
\end{equation}
It turns out that this vector field is $\Theta$-related to $0\times\xi\in\vf(\mcal{M}\times G)$ (see Lemma~5.13 in \cite{Kolar1993natural}, Lemma~20.14 in \cite{Lee2013introduction}),
meaning that the flow of $\hat{\Theta}(\xi)$ is given by
\begin{equation}
    \Flow_{\hat{\Theta}(\xi)}^t = \Theta_{\exp(t\xi)}.
    \label{eqn:exp_action_and_flow_on_vb}
\end{equation}
Likewise, $\theta_{\exp(t\xi)}$ is the flow of $\hat{\theta}(\xi) = \left. \ddt \right\vert_{t=0} \theta_{\exp(t\xi)} \in \vf(\mcal{M})$, which is $\pi$-related to $\hat{\Theta}(\xi)$.

Differentiating the smooth curves $t \mapsto \mcal{K}_{\exp(t\xi)} F (p)$ lying in $E_p$ for each $p\in\mcal{M}$ gives rise to the Lie derivative $\mcal{L}_{\xi}:\sect(E) \to \sect(E)$ along $\xi \in \Lie(G)$ defined by
\begin{equation}
    \boxed{
    \mcal{L}_{\xi} F 
    = \left.\ddt \right\vert_{t=0} \mcal{K}_{\exp(t\xi)} F 
    = \lim_{t\to 0} \frac{1}{t}\left( \Theta_{\exp(-t\xi)}\circ F \circ \theta_{\exp(t\xi)} - F \right),
    }
    \label{eqn:Lie_derivative}
\end{equation}
where the derivative and limit 
are understood pointwise and we identify $T_{F(p)} E_p \cong E_p$.
This construction is illustrated in Figure~\ref{fig:Lie_derivative}.
We recover Eq.~\ref{eqn:Lie_derivative_of_real_map} from Eq.~\ref{eqn:Lie_derivative} in the special case where a smooth function $F:\mcal{V} \to \mcal{W}$ is viewed as a section $x \mapsto (x,F(x))$ of the bundle $\pi:\mcal{V}\times\mcal{W} \to \mcal{V}$ and acted upon by group representations.
Critically, the Lie derivative $\mcal{L}_{\xi}$, as defined above, is a linear operator on sections of the vector bundle $E$.
This allows us to formulate convex symmetry-promoting regularization functions as in Section~\ref{sec:promoting_symmetry} using Lie derivatives in the broader setting of vector bundle sections.

\begin{remark}[Lie derivatives using flows]
    Thanks to Eq.~\ref{eqn:exp_action_and_flow_on_vb}, the Lie derivative defined in Eq.~\ref{eqn:Lie_derivative} only depends on the infinitesimal generator $\hat{\Theta}(\xi) \in \vf(E)$, and its flow for small time $t$.
    Hence, any vector field in $\vf(E)$ whose flow is fiber-linear, but not necessarily defined for all $t\in\R$, gives rise to an analogously-defined Lie derivative acting linearly on $\sect(E)$.
    These are the so-called ``linear vector fields'' described by \cite{Kolar1993natural} in Section~47.9.
    In fact, more general versions of the Lie derivative based on flows for maps between manifolds are described by \cite{Kolar1993natural} in Chapter~11.
    However, these generalizations are nonlinear operators, destroying the convex properties of the symmetry-promoting regularization functions in Section~\ref{sec:promoting_symmetry}.
\end{remark}

In addition to linearity, the key properties of the operators $\mcal{K}_g$ and $\mcal{L}_\xi$ for studying symmetries of sections are:
\begin{proposition}
    \label{prop:properties_of_Lie_derivative}
    For every $F\in\sect(E)$, $\xi, \eta \in\Lie(G)$, and $\alpha, \beta, t \in\R$, we have
    \begin{equation}
        \ddt \mcal{K}_{\exp(t\xi)}F 
        = \mcal{L}_{\xi} \mcal{K}_{\exp(t\xi)} F
        = \mcal{K}_{\exp(t\xi)} \mcal{L}_{\xi} F,
        \label{eqn:Lie_derivative_as_generator}
    \end{equation}
    \begin{equation}
        \mcal{L}_{\alpha\xi + \beta \eta} = \alpha \mcal{L}_{\xi} + \beta \mcal{L}_{\eta},
        \label{eqn:linearity_of_Lie_derivative_wrt_Lie_algebra}
    \end{equation}
    and
    \begin{equation}
        \mcal{L}_{[\xi, \eta]} 
        = \mcal{L}_{\xi}\mcal{L}_{\eta} - \mcal{L}_{\eta}\mcal{L}_{\xi}.
        \label{eqn:Lie_derivative_commutator}
    \end{equation}
    We give a proof in Appendix~\ref{app:properties_of_Lie_derivative}.
\end{proposition}
Taken together, these results mean that $\Pi:g \mapsto \mcal{K}_g$ and $\Pi_*:\xi \mapsto \mcal{L}_{\xi}$ are (infinite-dimensional) representations of $G$ and $\Lie(G)$.

The main results of this section are the following two theorems.
The first gives necessary and sufficient conditions for a section to be $G$-equivariant, while the
second completely characterizes the identity component of $\Sym_G(F)$ by correspondence with its Lie subgalgebra.
\begin{theorem}
    \label{thm:equivariance_conditions_for_vb_section}
    Let $G_0$ be the identity component of $G$. 
    A section $F \in \sect(E)$ is $G_0$-equivariant if and only if
    \begin{equation}
        \mcal{L}_{\xi} F = 0 \qquad \forall \xi \in \Lie(G).
    \end{equation}
    If, in addition, we have $\mcal{K}_{g_i} F = F$ for a single $g_i$ from each non-identity component of $G$, then $F$ is $G$-equivariant.
    We give a proof in Appendix~\ref{app:equivariance_conditions_for_vb_section}.
\end{theorem}
\begin{theorem}
    \label{thm:symmetries_of_sections}
    If $F \in \sect(E)$ is a section, then $\Sym_G(F)$ is a closed, embedded Lie subgroup of $G$ whose Lie subalgebra is
    \begin{equation}
        \sym_G(F) = \left\{ \xi \in \Lie(G) \ : \ \mcal{L}_{\xi} F = 0 \right\}.
        \label{eqn:sym_G_for_vb_section}
    \end{equation}
    We give a proof in Appendix~\ref{app:symmetries_of_sections}.
\end{theorem}
These results allow us to promote, enforce, and discover symmetries for sections of vector bundles in fundamentally the same way we did for maps between finite-dimensional vector spaces in Sections~\ref{sec:enforcing_symmetry},~\ref{sec:discovering_symmetry},~and~\ref{sec:promoting_symmetry}.
In particular, symmetry can be enforced through analogous linear constraints, discovered through nullspaces of analogous operators, and promoted through analogous convex penalties based on the nuclear norm.

\begin{remark}[Left actions]
    Theorems~\ref{thm:equivariance_conditions_for_vb_section}~and~\ref{thm:symmetries_of_sections} hold without any modification for left G-actions $\Theta^L:G\times E \to E$.
    This is because we can define a corresponding right $G$-action by $\Theta^R(p,g) = \Theta^L(g^{-1},p)$ with associated operators related by 
    \begin{equation}
        % $
        \mcal{K}^R_{g} = \mcal{K}^L_{g^{-1}}
        % $
        \qquad \mbox{and}\qquad
        % and
        % $
        \mcal{L}^R_{\xi} = - \mcal{L}^L_{\xi}.
        % $.
    \end{equation}
    The symmetry group $\Sym_G(F)$ does not depend on whether it is defined by the condition $\mcal{K}^R_{g} F = F$ or by $\mcal{K}^L_{g} F = F$.
    It is slighly less natural to work with left actions because $\Pi^L: g \mapsto \mcal{K}^L_g$ and $\Pi^L_*: \xi \mapsto \mcal{L}^L_\xi$ are Lie group and Lie algebra anti-homomorphisms, that is, 
    \begin{equation}
        \Pi^L(g_1 g_2) = \Pi^L(g_2) \Pi^L(g_1) 
        \qquad \mbox{and} \qquad
        \Pi^L_*\big([\xi,\eta]\big) 
        % = \Pi^L_*(\eta) \Pi^L_*(\xi) - \Pi^L_*(\xi) \Pi^L_*(\eta)
        = \big[\Pi^L_*(\eta), \Pi^L_*(\xi)\big].
    \end{equation}
\end{remark}

\subsection{Vector fields}
\label{subsec:symmetry_of_vf}
    Here we study the symmetries of a vector field $V\in\vf(\mcal{M})$ under a right $G$-action $\theta:\mcal{M}\times G \to\mcal{M}$.
    This allows us to extend the discussion in Section~\ref{subsec:dynamical_systems} to dynamical systems described by vector fields on smooth manifolds and acted upon nonlinearly by arbitrary Lie groups.
    The tangent map of the diffeomorhpism $\theta_g = \theta(\cdot, g):\mcal{M} \to \mcal{M}$ transforms vector fields via the pushforward map $(\theta_{g})_* : \vf(\mcal{M}) \to \vf(\mcal{M})$ defined by
    \begin{equation}
        ((\theta_{g})_* V)_{p \cdot g} = \D \theta_g(p) V_p
    \end{equation}
    for every $p\in\mcal{M}$.
    \begin{definition}
        \label{def:invariance_of_vf}
        Given $g\in G$, we say that a vector field $V\in\vf(\mcal{M})$ is $g$-\textbf{invariant} if and only if $(\theta_g)_* V = V$, that is,
        \begin{equation}
            V_{p\cdot g} = \D \theta_g(p) V_p \qquad \forall p\in\mcal{M}.
        \end{equation}
    \end{definition}
    Because $(\theta_{g^{-1}})_* (\theta_g)_* = (\theta_{g^{-1}}\circ\theta_g)_* = \Id_{\vf(\mcal{M})}$, it is clear that a vector field is $g$-invariant if and only if it is $g^{-1}$-invariant.
    
    Recall that vector fields $V\in\vf(\mcal{M})$ are smooth sections of the tangent bundle $E = T\mcal{M}$. 
    The right $G$-action $\theta$ on $\mcal{M}$ induces a right $G$-action $\Theta: T\mcal{M} \times G \to T\mcal{M}$ on the tangent bundle defined by
    \begin{equation}
        \Theta_g(v_p) = \D\theta_g(p) v_p.
    \end{equation}
    It is easy to see that each $\Theta_g$ is a vector bundle homomorphism descending to $\theta_g$ under the bundle projection $\pi$.
    Crucially, we have
    \begin{equation}
        \mcal{K}_g V 
        = \Theta_{g^{-1}} \circ V \circ \theta_g
        = (\theta_{g^{-1}})_* V,
    \end{equation}
    meaning that a vector field $V\in\vf(\mcal{M})$ is $g$-invariant if and only if it is $g$-equivariant as a section of $T\mcal{M}$ with respect to the action $\Theta$.
    Recall that (by Lemma~20.14 in \citet{Lee2013introduction}) the left-invariant vector field $\xi \in \Lie(G) \subset \vf(G)$ and its infinitesimal generator $\hat{\theta}(\xi) \in \vf(\mcal{M})$ are $\theta^{(p)}$-related, where $\theta^{(p)}: g \mapsto \theta(p,g)$ is the orbit map.
    This means that $\theta_{\exp(t\xi)}$ is the time-$t$ flow of $\hat{\theta}(\xi)$ by Proposition~9.13 in \citet{Lee2013introduction}.
    As a result, the Lie derivative in Eq.~\ref{eqn:Lie_derivative} agrees with the standard Lie derivative of $V$ along $\hat{\theta}(\xi)$ (see \citet[p.228]{Lee2013introduction}),
    that is,
    \begin{equation}
    \boxed{
        \mcal{L}_{\xi} V(p) 
        = \lim_{t\to 0} \frac{1}{t} \left[ \D\theta_{\exp(-t\xi)}(\theta_{\exp(t\xi)}(p)) V_{\theta_{\exp(t\xi)}(p)} - V_p \right]
        = [\hat{\theta}(\xi), V]_p,
        }
    \end{equation}
    where the expression on the right is the Lie bracket of $\hat{\theta}(\xi)$ and $V$.

\subsection{Tensor fields}
\label{subsec:symmetry_of_tf}
    Symmetries of a tensor field can also be revealed using our framework.
    This will be useful for our study of Hamiltonian dynamics in Section~\ref{subsec:Hamiltonian_conservation_laws} and for our study of integral operators, whose kernels can be viewed as tensor fields, in Section~\ref{subsec:equivariant_integral_operators}.
    For simplicity, we consider a rank-$k$ covariant tensor field $A \in \tf^k(\mcal{M})$, although our results extend to contravariant and mixed tensor fields with minimal modification.
    We rely on the basic definitions and machinery found in \citet[Chapter~12]{Lee2013introduction}.
    Under a right $G$-action $\theta$ on $\mcal{M}$, the tensor field transforms via the pullback map $\theta_g^*:\tf^k(\mcal{M}) \to \tf^k(\mcal{M})$ defined by
    \begin{equation}
        (\theta_g^* A)_p(v_1, \ldots, v_k) 
        = (\D\theta_g(p)^* A_{p\cdot g})(v_1, \ldots, v_k) 
        = A_{p\cdot g}(\D\theta_g(p) v_1, \ldots, \D\theta_g(p) v_k)
    \end{equation}
    for every $v_1, \ldots, v_k \in T_p\mcal{M}$.
    \begin{definition}
        \label{def:invariance_of_tf}
        Given $g\in G$, a tensor field $A\in\tf^k(\mcal{M})$ is $g$-\textbf{invariant} if and only if $\theta_g^* A = A$, that is,
        \begin{equation}
            A_{p\cdot g}(\D\theta_g(p) v_1, \ldots, \D\theta_g(p) v_k) = A_p(v_1, \ldots, v_k) \qquad \forall p\in\mcal{M}.
        \end{equation}
    \end{definition}
    
    To study the invariance of tensor fields in our framework, we recall that a tensor field is a section of the tensor bundle 
    $E = T^kT^*\mcal{M} = \coprod_{p\in\mcal{M}} (T_p^*\mcal{M})^{\otimes k}$, a vector bundle over $\mcal{M}$,
    where $T_p^*\mcal{M}$ is the dual space of $T_p\mcal{M}$.
    The right $G$-action $\theta$ on $\mcal{M}$ induces a
    right $G$-action $\Theta: T^k T\mcal{M} \times G \to T^k T\mcal{M}$ defined by
    \begin{equation}
        \Theta_g(A_p) = \D \theta_{g^{-1}}(\theta_g(p))^* A_p.
    \end{equation}
    It is clear that each $\Theta_g$ is a homomorphism of the vector bundle $T^kT^*\mcal{M}$ descending to $\theta_g$ under the bundle projection.
    Crucially, we have
    \begin{equation}
        \mcal{K}_g A 
        = \Theta_{g^{-1}}\circ A \circ \theta_g
        = \theta_g^* A,
    \end{equation}
    meaning that $A\in\tf^k(\mcal{M})$ is $g$-invariant if and only if it is $g$-equivariant as a section of $T^kT^*\mcal{M}$ with respect to the action $\Theta$.
    Since $\theta_{\exp(t\xi)}$ gives the time-$t$ flow of the vector field $\hat{\theta}(\xi)\in\vf(\mcal{M})$, the Lie derivative in Eq.~\ref{eqn:Lie_derivative} for this action agrees with the standard Lie derivative of $A\in\tf^k(\mcal{M})$ along $\hat{\theta}(\xi)$ (see \citet[p.321]{Lee2013introduction}), that is
    \begin{equation}
    \boxed{
    (\mcal{L}_{\xi} A)_p 
    = \lim_{t\to 0} \frac{1}{t}\left[ \D\theta_{\exp(t\xi)}(p)^* A_{\theta_{\exp(t\xi)}(p)} - A_p \right]
    = (\mcal{L}_{\hat{\theta}(\xi)} A)_p.
    }
    \label{eqn:Lie_derivative_for_tf}
\end{equation}
The Lie derivative for arbitrary covariant tensor fields can be computed by applying Proposition~12.32 in \cite{Lee2013introduction} and its corollaries.
More generally, thanks to 6.16-18 in \cite{Kolar1993natural}, the Lie derivative for any tensor product of sections of natural vector bundles can be computed via the formula
\begin{equation}
    \mcal{L}_{\xi}(A_1\otimes A_2) = (\mcal{L}_{\xi} A_1) \otimes A_2 + A_1 \otimes (\mcal{L}_{\xi} A_2).
\end{equation}
For example, this holds when $A_1, A_2$ are arbitrary smooth tensor fields of mixed types.
The Lie derivative of a differential form $\omega$ on $\mcal{M}$ can be computed by Cartan's magic formula
\begin{equation}
    \mcal{L}_{\xi} \omega = \hat{\theta}(\xi) \intmul (\td \omega) + \td (\hat{\theta}(\xi) \intmul \omega),
    \label{eqn:Cartan_magic_formula}
\end{equation}
where $\td$ is the exterior derivative.


\subsection{Hamiltonian dynamics}
\label{subsec:Hamiltonian_conservation_laws}

The dynamics of frictionless mechanical systems can be described by Hamiltonian vector fields on symplectic manifolds.
Roughly speaking, these encompass systems that conserve energy, such as motion of rigid bodies and particles interacting via conservative forces.
The celebrated theorem of \cite{noether1918invariante} says that conserved quantities of Hamiltonian systems correspond with symmetries of the energy function (the system's Hamiltonian).
In this section, we briefly illustrate how to enforce Hamiltonicity constraints on learned dynamical systems and how to promote, discover, and enforce conservation laws.
A thorough treatment of classical mechanics, symplectic manifolds, and Hamiltonian systems can be found in \cite{Abraham2008foundations,Marsden:MS}.
This includes methods for reduction of systems with known symmetries and conservation laws.
The following brief introduction follows Chapter~22 of \cite{Lee2013introduction}.

Hamiltonian systems are defined on symplectic manifolds.
That is, a smooth even-dimensional manifold $\mcal{S}$ together with a smooth nondegenerate differential $2$-form $\omega$, called the symplectic form.
Nondegeneracy means that the map $\hat{\omega}_p: v \mapsto \omega_p(v,\cdot)$ is a bijective linear map of $T_p \mcal{S}$ onto its dual $T_p^* \mcal{S}$ for every $p \in \mcal{S}$.
Thanks to nondegeneracy, any smooth function $H \in C^{\infty}(\mcal{S})$ gives rise to a smooth vector field
\begin{equation}
    V_H = \hat{\omega}^{-1}(\td H)
\end{equation}
called the ``Hamiltonian vector field'' of $H$.
A vector field $V \in \vf(\mcal{S})$ is said to be Hamiltonian if $V = V_H$ for some function $H$, called the Hamiltonian of $V$.
A vector field is locally Hamiltonian if it is Hamiltonian in neighborhood of each point of $\mcal{S}$.

The symplectic manifolds considered in classical mechanics usually consist of the cotangent bundle $\mcal{S} = T^* \mcal{M}$ of an $m$-dimensional manifold $\mcal{M}$ describing the ``configuration'' of the system, e.g., the positions of particles.
The cotangent bundle has a canonical symplectic form given by
\begin{equation}
    \omega = \sum_{i=1}^{m} \td x^i \wedge \td \xi_i,
    \label{eqn:canonical_symplectic_form}
\end{equation}
where $(x^i, \xi_i)$ are any choice of natural coordinates on a patch of $T^* \mcal{M}$ (see Proposition~22.11 in \cite{Lee2013introduction}).
Here, each $x^i$ is a generalized coordinate describing the configuration and $\xi_i$ is its ``conjugate'' or ``generalized'' momentum.
The Darboux theorem (Theorem~22.13 in \cite{Lee2013introduction}) says that any symplectic form on a manifold can be put into the form of Eq.~\ref{eqn:canonical_symplectic_form} by a choice of local coordinates.
In these ``Darboux'' coordinates, the dynamics of a Hamiltonian system are governed by the equations
\begin{equation}
    \ddt x^i = V_H(x^i) = \frac{\partial H}{\partial \xi_i}, 
    \qquad
    \ddt \xi_i = V_H(\xi_i) = - \frac{\partial H}{\partial x^i},
\end{equation}
which should be familiar to anyone who has studied undergraduate mechanics.

Enforcing local Hamiltonicity on a vector field $V\in\vf(\mcal{S})$ is equivalent to the linear constraint
\begin{equation}
    \mcal{L}_V \omega 
    = 0
\end{equation}
thanks to Proposition~22.17 in \cite{Lee2013introduction}. 
Here $\mcal{L}_V$ is the Lie derivative of the tensor field $\omega \in \tf^2(\mcal{S})$, i.e., Eq.~\ref{eqn:Lie_derivative_for_tf} with $\theta$ being the flow of $V$ and its generator being the identity $\hat{\theta}(V) = V$.
Note that the Lie derivative still makes sense even when the orbits $t \mapsto \theta_t(p) = \theta_{\exp(t 1)}(p)$ are only defined for small $t\in (-\varepsilon, \varepsilon)$.
In Darboux coordinates, this constraint is equivalent to the set of equations
\begin{equation}
    \frac{\partial V(x^i)}{\partial x^j} + \frac{\partial V(\xi_j)}{\partial \xi_i} = 0, \qquad
    \frac{\partial V(\xi_i)}{\partial x^j} - \frac{\partial V(\xi_j)}{\partial x^i} = 0, \qquad
    \frac{\partial V(x^i)}{\partial \xi_j} - \frac{\partial V(x^j)}{\partial \xi_i} = 0
\end{equation}
for all $1\leq i,j \leq m$.
When the first de Rham cohomology group satisfies $H^1_{\text{dR}}(\mcal{S}) = 0$, for example when $\mcal{S}$ is contractible, local Hamilonicity implies the existence of a global Hamiltonian for $V$, unique on each component of $\mcal{S}$ up to addition of a constant by \citet[Proposition~22.17]{Lee2013introduction}.

Of course our approach also makes it possible to promote Hamiltonicity with respect to candidate symplectic structures when learning a vector field $V$.
To do this, we can penalize the nuclear norm of $\mcal{L}_V$ restricted to a subspace $\tilde{\Omega}$ of candidate $2$-forms using the regularization function
\begin{equation}
    R_{\tilde{\Omega},*}(V) = \big\Vert \left. \mcal{L}_V \right\vert_{\tilde{\Omega}} \big\Vert_*.
\end{equation}
The strength of this penalty can be increased when solving a regression problem for $V$ until there is a non-degenerate $2$-form in the nullspace $\Null(\mcal{L}_V) \cap \tilde{\Omega}$.
This gives a symplectic form with respect to which $V$ is locally Hamiltonian.

Another option is to learn a (globally-defined) Hamiltonian function $H$ directly by fitting $V_H$ to data.
In this case, we can regularize the learning problem by penalizing the breaking of conservation laws.
The time-derivative of a quantity, that is, a smooth function $f\in C^{\infty}(\mcal{S})$ under the flow of $V_H$ is given by the Poisson bracket
\begin{equation}
    \{ f, H \} := \omega(V_f, V_H) = \td f(V_H) = V_H(f) = -V_f(H).
\end{equation}
Hence, $f$ is a conserved quantity if and only if $H$ is invariant under the flow of $V_f$ --- this is Noether's theorem.
It is also evident that the Poisson bracket is linear with respect to both of its arguments.
In fact, the Poisson bracket turns $C^{\infty}(\mcal{S})$ into a Lie algebra with $f \mapsto V_f$ being a Lie algebra homomorphism, i.e.,
$
    V_{\{f, g \}} = [V_f, V_g]
$.

As a result of these basic properties of the Poisson bracket, the quantities conserved by a given Hamiltonian vector field $V_H$ form a Lie subalgebra given by the nullspace of a linear operator $L_H : C^{\infty}(\mcal{S}) \mapsto C^{\infty}(\mcal{S})$ defined by
\begin{equation}
    L_H : f \mapsto \{ f, H \}.
\end{equation}
To promote conservation of quantities in a given subalgebra $\mfrak{g} \subset C^{\infty}(\mcal{S})$ when learning a Hamiltonian $H$, we can penalize the nuclear norm of $L_H$ restricted to $\mfrak{g}$, that is
\begin{equation}
    R_{\mfrak{g},*}(H) = \big\Vert \left.L_H\right\vert_{\mfrak{g}} \big\Vert_*.
\end{equation}
For example, we might expect a mechanical system to conserve angular momentum about some axes, but not others due to applied torques.
In the absence of data to the contrary, it often makes sense to assume that various linear and angular momenta are conserved.


\subsection{Equivariant integral operators}
\label{subsec:equivariant_integral_operators}
In this section we provide machinery to study the symmetries of linear and nonlinear integral operators acting on sections of vector bundles, yielding far-reaching generalizations of our results in Section~\ref{subsec:NNs_acting_on_fields}.
Such operators can form the layers of neural networks acting on various vector and tensor fields supported on manifolds.

Let $\pi_0: E_0 \to \mcal{M}_0$ and $\pi_j: E_j \to \mcal{M}_j$ be vector bundles with $\mcal{M}_j$ being $d_j$-dimensional orientable Riemannian manifolds with volume forms $\volform_j \in \Omega^{d_j}(T^*\mcal{M}_j)$, $j=1,\ldots,r$.
Note that here, $\volform_j$ does not denote the exterior derivative of a $(d_j-1)$-form.
A section $K$ of the bundle 
\begin{equation}
    E = E_0 \otimes E_1^* \otimes \cdots \otimes E_r^*
    := \coprod_{(p,q_1, \ldots, q_r)\in\mcal{M}_0\times\cdots\times\mcal{M}_r} E_{0,p} \otimes E_{1,q_1}^* \otimes \cdots \otimes E_{r,q_r}^*
\end{equation}
can be viewed as a smooth family of $r$-multilinear maps
\begin{equation}
    K(p,q_1,\ldots,q_r):\bigoplus_{j=1}^r E_{j,q_j} \to E_{0,p}.
\end{equation}
The section $K$ can serve as the kernel to define an $r$-multilinear integral operator $\mcal{T}_K : D(\mcal{T}_K) \subset \bigoplus_{j=1}^r\sect(E_j) \to \sect(E_0)$ with action on $(F_1,\ldots,F_r)\in D(\mcal{T}_K)$ given by
\begin{equation}
    \mcal{T}_K[F_1,\ldots,F_r](p) = \int_{\mcal{M}_1\times\cdots\times\mcal{M}_r} K(p,q_1,\ldots,q_r)\big[ F_1(q_1), \ldots, F_r(q_r) \big] \volform_1(q_1) \wedge \cdots \wedge \volform_r(q_r). 
    \label{eqn:general_nonlinear_integral_operator}
\end{equation}
This operator is linear when $r=1$.
When $r > 1$ and $E_1 = \cdots = E_r$, Eq.~\ref{eqn:general_nonlinear_integral_operator} can be used to define a nonlinear integral operator $\sect(E_1) \to \sect(E_0)$ with action $F \mapsto \mcal{T}_K[F,\ldots,F]$.

Given fiber-linear right $G$-actions $\Theta_j: E_j \times G \to E_j$, there is an induced fiber-linear right $G$-action $\Theta: E \times G \to E$ on $E$ defined by
\begin{equation}
    \Theta_g(K_{p,q_1,\ldots,q_r})\big[v_{1}, \ldots, v_{r}\big]
    = \Theta_{0,g} \left( K_{p,q_1,\ldots,q_r}\big[ \Theta_{1,g^{-1}}(v_{1}), \ldots, \Theta_{r,g^{-1}}(v_{r})\big]\right)
\end{equation}
for $K_{p,q_1,\ldots,q_r} \in E$ viewed as an $r$-multilinear map $E_{1,q_1}\oplus \cdots \oplus E_{r,q_r} \to E_{0,p}$ and $v_j \in E_{j,\theta_{j,g}(q_j)}$.
Sections $F_j \in \sect(E_j)$ transform according to
\begin{equation}
    \mcal{K}^{E_j}_{g} F_j = \Theta_{j,g^{-1}} \circ F_j \circ \theta_{j,g},
\end{equation}
with the section defining the integral kernel transforming according to
\begin{empheq}[box=\fbox]{multline}
    \mcal{K}^{E}_g K(p,q_1,\ldots,q_r)[v_{q_1}, \ldots, v_{q_r}] = \\ 
    \Theta_{0,g^{-1}}\Big\{
    K\big(\theta_{0,g}(p),\theta_{1,g}(q_1),\ldots,\theta_{r,g}(q_r)\big)
    \big[ \Theta_{1,g}(v_{q_1}), \ldots, \Theta_{r,g}(v_{q_r}) \big] \Big\}.
\end{empheq}
Using these transformation laws, we define equivariance for the integral operator as follows:
\begin{definition}
    The integral operator in Eq.~\ref{eqn:general_nonlinear_integral_operator} is \textbf{equivariant} with respect to $g\in G$ if
    \begin{equation}
        \mcal{K}^{E_0}_{g} \mcal{T}_K\big[ \mcal{K}^{E_1}_{g^{-1}}F_1, \ldots, \mcal{K}^{E_r}_{g^{-1}}F_r \big]
        = \mcal{T}_K\big[ F_1, \ldots, F_r \big]
    \end{equation}
    for every $(F_1,\ldots,F_r) \in D(\mcal{T}_K)$.
\end{definition}

Using the fact that the integral is invariant under pullbacks by diffeomorphisms, we can express the left-hand-side of the equivariance condition as
\begin{multline}
    \mcal{K}_{0,g} \mcal{T}_K\big[ \mcal{K}_{1,g^{-1}}F_1, \ldots, \mcal{K}_{r,g^{-1}}F_r \big](p)
    = \\
    \Theta_{0,g^{-1}}\Bigg\{
    \int_{\mcal{M}_1\times\cdots\times\mcal{M}_r} K(\theta_{0,g}(p),q_1,\ldots,q_r)
    \big[ \Theta_{1,g}\circ F_1\circ \theta_{1,g^{-1}}(q_1), \ldots, \Theta_{r,g}\circ F_r\circ \theta_{r,g^{-1}}(q_r) \big] \\
    \volform_1(q_1) \wedge \cdots \wedge \volform_r(q_r). 
    \Bigg\} = \\
    \Theta_{0,g^{-1}}\Bigg\{
    \int_{\mcal{M}_1\times\cdots\times\mcal{M}_r} K(\theta_{0,g}(p),\theta_{1,g}(q_1),\ldots,\theta_{r,g}(q_r))
    \big[ \Theta_{1,g}\circ F_1(q_1), \ldots, \Theta_{r,g}\circ F_r(q_r) \big] \\
    \theta_{1,g}^* \volform_1(q_1) \wedge \cdots \wedge \theta_{r,g}^*\volform_r(q_r). 
    \Bigg\} = \\
    \int_{\mcal{M}_1\times\cdots\times\mcal{M}_r} 
    \mcal{K}^E_{g} K (p, q_1, \ldots, q_r)\big[ F_1(q_1), \ldots, F_r(q_r) \big]
    \theta_{1,g}^* \volform_1(q_1) \wedge \cdots \wedge \theta_{r,g}^*\volform_r(q_r).
\end{multline}
Hence, the condition for equivariance of $\mcal{T}_K$ is equivalent to
\begin{empheq}[box=\fbox]{equation}
    \mcal{K}_{g}\big( K \volform_1 \wedge \cdots \wedge V_r \big)  
    := \mcal{K}^E_g(K) \ \theta_{1,g}^* \volform_1 \wedge \cdots \wedge \theta_{r,g}^*\volform_r
    = K \volform_1 \wedge \cdots \wedge \volform_r,
    \label{eqn:transformation_of_integral_kernels}
\end{empheq}
where we note that $\mcal{K}^{\Omega}_g (\volform_j) = \theta_{j,g}^* \volform_j$ is the natural transformation of the differential form $\volform_j \in \Omega^{d_j}(T^*\mcal{M}_j)$ (a covariant tensor field) described in Section~\ref{subsec:symmetry_of_tf}.
The Lie derivative of the action on volume forms is given by
\begin{equation}
    \mcal{L}^{\Omega}_{\xi} \volform 
    = \td \big( \hat{\theta}(\xi) \intmul \volform \big)
    = \diverg \hat{\theta}(\xi) \volform
\end{equation}
thanks to Cartan's magic formula and the definition of divergence (see \cite{Lee2013introduction}).
Therefore, differentiating Eq.~\ref{eqn:transformation_of_integral_kernels} along the curve $g(t) = \exp(t\xi)$ yields the Lie derivative
\begin{empheq}[box=\widefbox]{equation}
    \mcal{L}_{\xi} \big( K \volform_1 \wedge \cdots \wedge \volform_r \big)
    = \bigg[ \mcal{L}^E_{\xi}(K) + K \sum_{j=1}^r \diverg \hat{\theta}_j(\xi)\bigg] \volform_1 \wedge \cdots \wedge \volform_r.
    \label{eqn:Lie_derivative_for_integral_kernels}
\end{empheq}
For the integral operators discussed in Section~\ref{subsec:NNs_acting_on_fields}, the formulas derived here recover Eqs.~\ref{eqn:transformation_for_linear_integral_kernels_on_Rn}~and~\ref{eqn:Lie_derivative_for_linear_integral_kernels_on_Rn}.


\section{Invariant submanifolds and tangency}
\label{sec:submanifolds_and_tangency}
Studying the symmetries of maps can be cast into a more general framework in which we study the symmetries of submanifolds.
Specifically, the symmetries of a map $F: \mcal{M}_0 \to \mcal{M}_1$ between manifolds correspond to symmetries of its graph, $\graph(F)$, 
and the symmetries of a section of a vector bundle $F \in \sect(E)$ correspond to symmetries of its image, $\image(F)$ 
--- both of which are properly embedded submanifolds of $\mcal{M}_0 \times \mcal{M}_1$ and $E$, respectively.
We show that symmetries of 
a large class of submanifolds, including the above,
are revealed by checking whether the infinitesimal generators of the group action are tangent to the submanifold.
In this setting, the Lie derivative of $F\in\sect(E)$ has a geometric interpretation as a projection of the infintesimal generator onto the tangent space of the image $\image(F)$, viewed as a submanifold of the bundle.

\subsection{Symmetry of submanifolds}
\label{subsec:detecting_symmetry_of_submanifolds}
In this section we study the infinitesimal conditions for a submanifold to be invariant under the action of a Lie group.
Suppose that $\mcal{N}$ is a manifold and $\theta: \mcal{N}\times G \to \mcal{N}$ is a right action of a Lie group $G$ on $\mcal{N}$.
Sometimes we denote this action by $p\cdot g = \theta(p,g)$ when there is no ambiguity.
Though our results also hold for left actions, as we discuss later in Remark~\ref{rem:left_actions_on_manifolds}, working with right actions is standard in this context and allows us to leverage results from \citet{Lee2013introduction} more naturally in our proofs.
Fixing $p\in\mcal{N}$, the orbit map of this action is denoted $\theta^{(p)}:G\to \mcal{N}$.
Fixing $g\in G$, the map $\theta_g:\mcal{N}\to \mcal{N}$ defined by $\theta_g: p \mapsto \theta(p,g)$ is a diffeomorphism with inverse $\theta_{g^{-1}}$.
\begin{definition}
    \label{def:group_invariance_of_subset}
    A subset $\mcal{M} \subset \mcal{N}$ is \textbf{$G$-invariant} if and only if $\theta(p, g) \in \mcal{M}$ for every $g\in G$ and $p\in\mcal{M}$.
\end{definition}
\noindent
Sometimes we will denote $\mcal{M} \cdot G = \{ \theta(p, g) \ : \ p \in \mcal{M}, \ g\in G \}$, in which case $G$-invariance of $\mcal{M}$ can be stated as $\mcal{M} \cdot G \subset \mcal{M}$.

We study the group invariance of submanifolds of the following type:
\begin{definition}
    \label{def:closed_slice}
    Let $\mcal{M}$ be a weakly embedded $m$-dimensional submanifold of an $n$-dimensional manifold $\mcal{N}$. 
    We say that $\mcal{M}$ is \textbf{arcwise-closed} if any smooth curve $\gamma:[a,b] \to \mcal{N}$ satisfying $\gamma((a,b)) \subset \mcal{M}$ must also satisfy $\gamma([a,b])\subset \mcal{M}$.
\end{definition}
\noindent
Submanifolds of this type include all properly embedded submanifolds of $\mcal{N}$ because properly embedded submanifolds are closed subsets (Proposition~5.5 in \cite{Lee2013introduction}).
More interestingly, we have the following:
\begin{proposition}
    \label{prop:foliations_have_arcwise_closed_leaves}
    The leaves of any (nonsingular) foliation of $\mcal{N}$ are arcwise-closed.
    We provide a proof in Appendix~\ref{app:proofs_of_minor_results}.
\end{proposition}
This means that the kinds of submanifolds we are considering include all possible Lie subgroups (\citet[Theorem~19.25]{Lee2013introduction}) as well as their orbits under free and proper group actions (\citet[Proposition~21.7]{Lee2013introduction}).
The leaves of singular foliations associated with integrable distributions of nonconstant rank (see \citet[Sections~3.18--25]{Kolar1993natural}) can fail to be arcwise-closed.
For example, the distribution spanned by the vector field $x\frac{\partial}{\partial x}$ on $\R$ has maximal integral manifolds $(-\infty, 0)$, $\{0\}$, and $(0,\infty)$ forming a singular foliation of $\R$.
Obviously, the leaves $(-\infty,0)$ and $(0,\infty)$ are not arcwise-closed.

The following theorem provides necessary and sufficient conditions for arcwise-closed weakly-embedded submanifolds to be $G$-invariant.
\begin{theorem}
    \label{thm:symmetry_conditions_for_submanifold}
    Let $\mcal{M}$ be an arcwise-closed weakly-embedded submanifold of $\mcal{N}$ and let $\theta: \mcal{N}\times G \to \mcal{N}$ be a right action of a Lie group $G$ on $\mcal{N}$ with infinitesimal generator $\hat{\theta}:\Lie(G) \to \vf(\mcal{N})$.
    Let $G_0$ be the identity component of $G$.
    Then $\mcal{M}$ is $G_0$-invariant if and only if
    \begin{equation}
        \hat{\theta}(\xi)_p \in T_p\mcal{M} \qquad \forall p\in\mcal{M} \quad \forall \xi \in \Lie(G).
        \label{eqn:generator_manifold_tangency}
    \end{equation}
    If, in addition, we have $\mcal{M} \cdot g_i \subset \mcal{M}$ for a single $g_i$ from each non-identity component of $G$, then $\mcal{M}$ is $G$-invariant.
    A proof is provided in Appendix~\ref{app:symmetry_conditions_for_submanifold}.
\end{theorem}
Since the infinitesimal generator $\hat{\theta}$ is a linear map and $T_p\mcal{M}$ is a subspace of $T_p\mcal{N}$, the tangency condition expressed in Eq.~\ref{eqn:generator_manifold_tangency} can be viewed as a set of linear constraints satisfied by the elements of the Lie algebra.

Given a candidate Lie group $G$ and a submanifold $\mcal{M}\subset\mcal{N}$ with unknown symmetries, we can ask which elements of the Lie algebra $\Lie(G)$ satisfy the tangency conditions.
The following theorem shows that these elements form the subalgebra corresponding to the largest connected Lie subgroup of symmetries of $\mcal{M}$ in $G$.
\begin{theorem}
    \label{thm:symmetries_of_a_submanifold}
    Let $\mcal{M}$ be an immersed submanifold of $\mcal{N}$ and let $\theta: \mcal{N}\times G \to \mcal{N}$ be a right action of a Lie group $G$ on $\mcal{N}$ with infinitesimal generator $\hat{\theta}:\Lie(G) \to \vf(\mcal{N})$.
    Then
    \begin{equation}
        \sym_G(\mcal{M}) = \big\{ \xi \in \Lie(G) \ : \ \hat{\theta}(\xi)_p \in T_p\mcal{M} \quad \forall p\in\mcal{M} \big\}
    \end{equation}
    is the Lie subalgebra of a unique connected Lie subgroup $\Sym_{G}(\mcal{M})_0 \subset G$.
    If $\mcal{M}$ is weakly-embedded and arcwise-closed in $\mcal{N}$, then this subgroup has the following properties:
    \begin{enumerate}[label=(\roman*)]
        \item $\mcal{M}\cdot \Sym_G(\mcal{M})_0 \subset \mcal{M}$
        \item If $H$ is a connected Lie subgroup of $G$ such that $\mcal{M} \cdot H \subset \mcal{M}$, then $H \subset \Sym_G(\mcal{M})_0$.
    \end{enumerate}
    If $\mcal{M}$ is properly embedded in $\mcal{N}$ then $\Sym_G(\mcal{M})_0$ is the identity component of the closed, properly embedded Lie subgroup
    \begin{equation}
        \Sym_G(\mcal{M}) = \{ g\in G \ : \ \mcal{M} \cdot g \subset \mcal{M} \}.
    \end{equation}
    A proof is provided in Appendix~\ref{app:symmetries_of_a_submanifold}.
\end{theorem}

\begin{remark}[Left actions]
    \label{rem:left_actions_on_manifolds}
    When the group $G$ acts on $\mcal{N}$ from the left according to $\theta^L:G\times\mcal{N} \to \mcal{N}$, we can always construct an equivalent right-action $\theta^R:\mcal{N}\times \mcal{N} \to \mcal{N}$ by setting
    $\theta^R(p, g) = \theta^L(g^{-1}, p)$.
    The corresponding infinitesimal generators are related by
        $\hat{\theta}^R = - \hat{\theta}^L$.
    Since $\hat{\theta}^L(\xi)_p \in T_p\mcal{M}$ if and only if $\hat{\theta}^R(\xi)_p \in T_p\mcal{M}$,  Theorems~\ref{thm:symmetry_conditions_for_submanifold}~and~\ref{thm:symmetries_of_a_submanifold} hold without modification for left $G$-actions.
\end{remark}

\subsection{The Lie derivative as a projection}

We provide a geometric interpretation of the Lie derivative in Eq.~\ref{eqn:Lie_derivative} by expressing it in terms of a projection of the infinitesimal generator of the group action onto the tangent space of $\image(F)$.
This allows us to connect the Lie derivative to the tangency conditions for symmetry of submanifolds presented in Section~\ref{subsec:detecting_symmetry_of_submanifolds}.

The Lie derivative $\mcal{L}_{\xi} F(p)$ lies in $E_p$, while $T_{F(p)}\image(F)$ is a subspace of $T_{F(p)} E$.
To relate quantities in these different spaces, the following lemma introduces a lifting of each $E_p$ to a subspace of $T_{F(p)} E$.
\begin{lemma}
    \label{lem:injection_of_E_into_TE}
    For every section $F\in \sect(E)$ there is a well-defined injective vector bundle homomorphism $\imath_F : E \to T E$ that 
    is expressed in any local trivialization $\Phi : \pi^{-1}(\mcal{U}) \to \mcal{U} \times \R^k$ as
    \begin{equation}
    \begin{aligned}
        \D\Phi \circ \imath_F \circ \Phi^{-1} : \mcal{U} \times \R^k &\to T(\mcal{U} \times \R^k) \\
        (p, \vect{v}) &\mapsto (0, \vect{v})_{\Phi(F(p))}.
    \end{aligned}
    \label{eqn:injection_E_to_TE}
    \end{equation}
    We give a proof in Appendix~\ref{app:characterization_of_Lie_derivative}.
\end{lemma}
This is a special case of the ``vertical lift'' of $E$ into the vertical bundle $V E = \{ v\in TE \ : \ \D \pi v = 0 \}$ described by \cite{Kolar1993natural} in Section~6.11.
The ``vertical projection'' $\vpr_E: VE \to E$ provides a left-inverse satisfying $\vpr_E \circ \imath_F = \Id_E$.

The following result relates the Lie derivative to a projection via the vertical lifting.
\begin{theorem}
    \label{thm:characterization_of_Lie_derivative}
    Given $F\in \sect(E)$ and $p \in \mcal{M}$,
    the map $P_{F(p)}:=\D(F\circ \pi)(F(p)):T_{F(p)} E \to T_{F(p)} E$ is a linear projection onto $T_{F(p)}\image(F)$ and for every $\xi\in\Lie(G)$ we have
    \begin{equation}
        \imath_F \circ (\mcal{L}_{\xi} F) (p) = 
        -\big[ I - P_{F(p)} \big] \hat{\Theta}(\xi)_{F(p)}.
        \label{eqn:Lie_derivative_as_projection}
    \end{equation}
    We give a proof in Appendix~\ref{app:characterization_of_Lie_derivative}.
\end{theorem}
For the special case of smooth maps $F:\R^m \to \R^n$ viewed a sections $x \mapsto (x,F(x))$ of the bundle $\pi:\R^m\times\R^n \to \R^m$, this theorem reproduces Eq.~\ref{eqn:Lie_derivative_as_projection_for_real_maps}.
The following corollary provides a link between our main results for sections of vector bundles and our main results for symmetries of submanifolds.
\begin{corollary}
\label{cor:properties_of_Lie_derivative}
For every $F\in\sect(E)$, $\xi\in\Lie(G)$, and $p\in\mcal{M}$ we have
\begin{equation}
    (\mcal{L}_{\xi}F)(p) = 0 
    \qquad \Leftrightarrow \qquad 
    \hat{\Theta}(\xi)_{F(p)} \in T_{F(p)}\image(F).
\end{equation}
\end{corollary}
In particular, this means that 
Theorems~\ref{thm:equivariance_conditions_for_vb_section}~and~\ref{thm:symmetries_of_sections}
are special cases of
Theorems~\ref{thm:symmetry_conditions_for_submanifold}~and~\ref{thm:symmetries_of_a_submanifold}.

\section{Conclusion}
\label{sec:conclusion}

This paper provides a unified theoretical approach to enforce, discover, and promote symmetries in machine learning models. 
In particular, we provide theoretical foundations for Lie group symmetry in machine learning from a linear-algebraic viewpoint. 
This perspective unifies and generalizes several leading approaches in the literature, including approaches for incorporating and uncovering symmetries in neural networks and more general machine learning models. 
The central objects in this work are linear operators describing the finite and infinitesimal transformations of smooth sections of vector bundles with fiber-linear Lie group actions.
To make the paper accessible to a wide range of practitioners, Sections~\ref{sec:background_on_matrix_Lie_groups}--\ref{sec:discretization} deal with the special case where the machine learning models are built using smooth functions between vector spaces acted upon by matrix Lie group representations.
Our main results establish that the infinitesimal operators --- the Lie derivatives --- fully encode the connected subgroup of symmetries for sections of vector bundles (resp. functions between vector spaces).
In other words, the Lie derivatives encode symmetries that the machine learning models are equivariant with respect to. 
We illustrate that promoting and enforcing continuous symmetries in large classes of machine learning models are dual problems with respect to the bilinear structure of the Lie derivative.
Moreover, we describe how symmetries can be promoted as inductive biases during training of these models using convex penalties based on the fundamental operators.
Finally, we show how the linear algebraic framework for discovering continuous symmetries extends to identify continuous symmetries of arbitrary submanifolds.
When the submanifold is the image of a section of a vector bundle (resp., the graph of a function between vector spaces) the fundamental operator for the submanfold recovers the Lie derivative.
We describe rigorous data-driven methods for discretizing and approximating the fundamental operators to accomplish the tasks of enforcing, promoting, and discovering symmetry.
Importantly, these theoretical concepts, while extremely general, admit efficient computational implementations via simple linear algebra. 

The main limitations of our approach come from the need to make appropriate choices for key objects including the candidate group $G$, the space of functions $\mcal{F}$ defining the machine learning model, and appropriate inner products for discretizing the fundamental operators. 
For example, it is possible that the only $G$-symmetric functions in $\mcal{F}$ are trivial, meaning that enforcing symmetry results in learning only trivial models.
One open question is whether our framework can be used in such cases to learn relaxed symmetries, as described by \cite{Wang2022approximately}.
In other words, we may hope to find elements in $\mcal{F}$ that are nearly symmetric, and to bound the degree of asymmetry based on the quantities derived from the fundamental operators, such as their norms.
Our proposed convex penalization methods are likely affected by the choice of $G$ and the discretization of the fundamental operators.
A natural question is whether a large candidate group of symmetries can be used without biasing the model towards undesirable or nonphysical symmetries.
Additionally, the choice of inner products associated with the discretization of the fundamental operators could affect the results of nuclear norm penalization.
Our reliance on the Lie algebra to study continuous symmetries also limits the ability of our proposed methods to account for partial symmetries, such as the invariance in classifying the characters ``Z'' and ``N'' to rotations by small angles, but not large angles.

In a follow-up paper we will apply the proposed methods to a wide range of examples, explaining in detail how to implement the methods in practice.
A main goal will be to study to what extent the nuclear norm relaxation can recover underlying symmetry groups and reduce the amount of data required to train accurate machine learning models.
Additionally, we will examine how the proposed techniques perform in the presence of noisy data, with the goal of understanding the empirical effects of problem dimension, noise level, and the candidate symmetry group.
The follow-up will also explore different choices of inner products in the discretization of the fundamental operators and the effect this has on the performance of the proposed symmetry-promoting regularization methods.

Other important avenues of future work include investigating computationally efficient approaches to discretize the fundamental operators and use them to enforce, discover, and promote symmetry within our framework.
This could involve leveraging sparse structure of the discretized operators in certain bases to enable the use of efficient Krylov subspace algorithms.
It will also be useful to identify efficient optimization algorithms for training symmetry-constrained or symmetry-regularized machine learning models.
Promising candidates include projected gradient descent, proximal splitting algorithms, and the Iteratively Reweighted Least Squares (IRLS) algorithms described by \cite{Mohan2012iterative}.
Using IRLS could enable symmetry-promoting penalties to be based on non-convex Schatten $p$-norms with $0 < p < 1$, potentially improving the recovery of underlying symmetry groups compared to the nuclear norm where $p=1$.

There are also several avenues we plan to explore in future theoretical work.
These include extending the techniques presented here via jet bundle prolongation to study symmetries in machine learning for Partial Differential Equations (PDEs).
Combining analogues of our proposed methods in this setting with techniques using the weak formulation proposed by \cite{Messenger2021weak, Reinbold2020using} could provide robust ways to identify symmetric PDEs in the presence of high noise and limited training data.
We also aim to study the perturbative effects of noisy data in algorithms to discover and promote symmetry with the goal of understanding the effects of problem dimension, noise level, and number of data points on recovery of symmetry groups.
Another important direction of theoretical study will be to
build on the work of \cite{peitz2023partial, steyert2022uncovering} by studying symmetry in the setting of Koopman operators for dynamical systems.
The connection between the fundamental operators in this paper and Koopman operators may be strengthened by establishing that the fundamental operators $\{\mcal{K}_g\}_{g\in G}$ form a strongly continuous group in an $L^2$ space of sections of the appropriate vector bundle, and that the corresponding infinitesimal generators are given by closures of the Lie derivatives.
One could then show that the Koopman operators of symmetric systems commute with these generators, and use this information to constrain the spectral projections of the Koopman operators.
This might be used to discover underlying symmetries based on Koopman operators, as well as to enhance algorithms to approximate Koopman operators by enforcing and promoting symmetry.
To do this, one might follow the program set forth by \citet{colbrook2023mpedmd}, where the measure preserving property of certain dynamical systems is exploited to enhance the Extended Dynamic Mode Decomposition (EDMD) algorithm of \cite{Williams2015jnls}.


\section*{Acknowledgements}
The authors acknowledge support from the National Science Foundation AI Institute in Dynamic Systems
(grant number 2112085).  SLB acknowledges support from the Army Research Office (ARO W911NF-19-1-0045). 
The authors would also like to acknowledge valuable discussions with Tess Smidt and Matthew Colbrook.

\begin{spacing}{1}
\setlength{\bibsep}{6.pt}
\bibliography{references}
\end{spacing}

\appendix

\section{Proofs of minor results}
\label{app:proofs_of_minor_results}

\begin{proof}[Proposition~\ref{prop:symmetries_of_linear_integral_operator}]
    \label{proof:symmetries_of_linear_integral_operator}
    Obviously, if $\mcal{K}_g K = K$ then $\mcal{K}_{g}^{(\mcal{W})} \circ \mcal{T}_K \circ \mcal{K}_{g^{-1}}^{(\mcal{V})} = \mcal{T}_K$.
    On the other hand, suppose that $\mcal{K}_g K(x_0,y_0) \neq K(x_0,y_0)$ for some $(x_0,y_0) \in \R^n \times \R^m$.
    Hence, there are vectors $v\in\mcal{V}$ and $w \in \mcal{W}^*$ such that $\langle w, \ \mcal{K}_g K(x_0,y_0) v - K(x_0,y_0) v \rangle > 0$.
    This remains true for all $y$ in a neighborhood $\mcal{U}$ of $y_0$ by continuity of $K$ and $\mcal{K}_g K$.
    Letting $F(x) = v \varphi(x)$ where $\varphi$ is a smooth, nonnegative, function with $\varphi(y_0) > 0$ and support in $\mcal{U}$, we obtain
    \begin{equation}
        \big\langle w, \ \mcal{K}_{g}^{(\mcal{W})} \circ \mcal{T}_K \circ \mcal{K}_{g^{-1}}^{(\mcal{V})} F(x) - \mcal{T}_K F(x)\big\rangle
        = \int_{\R^m} \big\langle w, \ \mcal{K}_g K(x,y) v - K(x,y) v \big\rangle \varphi(y) \td y > 0,
    \end{equation}
    meaning $\mcal{K}_{g}^{(\mcal{W})} \circ \mcal{T}_K \circ \mcal{K}_{g^{-1}}^{(\mcal{V})} \neq \mcal{T}_K$.
    Therefore, $\mcal{K}_g K = K$ if and only if $\mcal{K}_{g}^{(\mcal{W})} \circ \mcal{T}_K \circ \mcal{K}_{g^{-1}}^{(\mcal{V})} = \mcal{T}_K$.
\end{proof}

\begin{proof}[Proposition~\ref{prop:intersection_subalgebra}]
    \label{proof:intersection_subalgebra}
    As an intersection of closed subgroups, $H := \bigcap_{l=1}^L \Sym_G\big(F^{(l)}\big)$ is a closed subgroup of $G$.
    By the closed subgroup theorem (see Theorem~20.12 in \citet{Lee2013introduction}), $H$ is an embedded Lie subgroup, whose Lie subalgebra we denote by $\mathfrak{h}$.
    If $\xi \in \mathfrak{h}$ then $\exp(t\xi)\in \Sym_G\big(F^{(l)}\big)$ for all $t\in\R$ and every $l=1,\ldots,L$.
    Differentiating $\mcal{K}_{\exp(t\xi)} F^{(l)} = F^{(l)}$ at $t = 0$ proves that $\mcal{L}_{\xi} F^{(l)} = 0$, i.e., $\xi\in\sym_G(F^{(l)})$ by Theorem~\ref{thm:symmetries_of_a_map}.
    Conversely, if $\mcal{L}_{\xi} F^{(l)} = 0$ for every $l=1,\ldots,L$, then by Theorem~\ref{thm:symmetries_of_a_map}, $\exp(t\xi) \in H$.
    Since $H$ is a Lie subgroup, differentiating $\exp(t\xi)$ at $t=0$ proves that $\xi \in \mathfrak{h}$.
\end{proof}

\begin{proof}[Proposition~\ref{prop:Monte_Carlo_eventually_gives_an_inner_product}]
    \label{proof:Monte_Carlo_eventually_gives_an_inner_product}
    Let $f_1, \ldots, f_N$ be a basis for $\mcal{F}'$.
    Consider the sequence of Gram matrices $\mat{G}_M$ with entries
    \begin{equation}
        [\mat{G}_M]_{i,j} = \left\langle f_i, \ f_j \right\rangle_{L^2(\mu_M)}.
    \end{equation}
    It suffices to show that $\mat{G}_M$ is positive-definite for sufficiently large $M$.
    Since the $L^2(\mu)$ inner product is positive-definite on $\mcal{F}'$, it follows that the Gram matrix $\mat{G}$ with entries
    \begin{equation}
        [\mat{G}]_{i,j} = \left\langle f_i, \ f_j \right\rangle_{L^2(\mu)}
    \end{equation}
    is positive-definite.
    Hence, its smallest eigenvalue $\lambda_{\text{min}}(\mat{G})$ is positive.
    Since the ordered eigenvalues of symmetric matrices are continuous with respect to their entries (see Corollary~4.3.15 in \cite{Horn2013matrix}) and $[\mat{G}_M]_{i,j} \to [\mat{G}]_{i,j}$ for all $1\leq i,j\leq N$ by assumption, we have $\lambda_{\text{min}}(\mat{G}_M) \to \lambda_{\text{min}}(\mat{G})$ as $M\to\infty$.
    Therefore, there is an $M_0$ so that for every $M \geq M_0$ we have $\lambda_{\text{min}}(\mat{G}_M) > 0$, i.e., $\mat{G}_M$ is positive-definite.
\end{proof}

\begin{proof}[Proposition~\ref{prop:foliations_have_arcwise_closed_leaves}]
    Consider a leaf $\mcal{M}$ of an $m$-dimensional foliation on the $n$-dimensional manifold $\mcal{N}$ and let $\gamma:[a,b] \to \mcal{N}$ be a smooth curve satisfying $\gamma((a,b)) \subset \mcal{M}$.
    First, it is clear that $\mcal{M}$ is a weakly embedded submanifold of $\mcal{N}$ since $\mcal{M}$ is an integral manifold of an involutive distribution (\citet[Proposition~19.19]{Lee2013introduction}) and the local structure theorem for integral manifolds (\citet[Proposition~19.16]{Lee2013introduction}) shows that they are weakly embedded.
    
    By continuity of $\gamma$, any neighborhood of $\gamma(b)$ in $\mcal{N}$ must have nonempty intersection with $\mcal{M}$.
    By definition of a foliation (see \cite{Lee2013introduction}), there is a coordinate chart $(\mcal{U}, \vect{x})$ for $\mcal{N}$ with $\gamma(b) \in \mcal{U}$ such that $\vect{x}(\mcal{U})$ is a coordinate-aligned cube in $\R^n$ and $\mcal{M} \cap \mcal{U}$ consists of countably many slices of the form $x^{m+1} = c^{m+1}, \ldots, x^{n} = c^n$ for constants $c^{m+1}, \ldots, c^n$.
    Since $\gamma$ is continuous, there is a $\delta > 0$ so that $\gamma((b-\delta, b]) \subset \mcal{U}$, and in particular, $\gamma((b-\delta, b)) \subset \mcal{M} \cap \mcal{U}$.
    By continuity of $\gamma$, there are constants $c^{m+1}, \ldots, c^n$ such that $x^{i}(\gamma(t)) = c^{i}$ for every $i=m+1,\ldots, n$ and $t \in (b-\delta, b)$.
    Hence, we have
    \begin{equation}
        x^i(\gamma(b)) = \lim_{t\to b} x^i(\gamma(t)) = c^i, \qquad i=m+1, \ldots, n,
    \end{equation}
    meaning that $\gamma(b) \in \mcal{M}$.
    An analogous argument shows that $\gamma(a) \in \mcal{M}$, completing the proof that $\mcal{M}$ is arcwise-closed.
\end{proof}

\section{Proof of Proposition~\ref{prop:generic_polynomial_map_sampling}}
\label{app:generic_polynomial_map_sampling}

Our proof relies on the following lemma:
\begin{lemma}
    \label{lem:generic_polynomial_sampling}
    Let $\mcal{P}$ denote a finite-dimensional vector space of polynomials $\R^m \to \R$.
    If $M \geq \dim (\mcal{P})$ then the evaluation map $T:\mcal{P} \to \R^M$ defined by
    \begin{equation}
        T_{(x_1, \ldots, x_M)}: P \mapsto (P(x_1), \ldots, P(x_M))
    \end{equation}
    is injective for almost every $(x_1, \ldots, x_M) \in (\R^m)^M$ with respect to Lebesgue measure.
\end{lemma}
\begin{proof}
    Letting $M_0 = \dim(\mcal{P})$ and
    choosing a basis $P_1, \ldots, P_{M_0}$ for $\mcal{P}$, injectivity of $T_{(x_1, \ldots, x_M)}$ is equivalent to injectivity of the $M\times M_0$ matrix 
    \begin{equation}
        \mat{T}_{(x_1, \ldots, x_M)}
        = \begin{bmatrix}
        P_1(x_1) & \cdots & P_{M_0}(x_1) \\
        \vdots & \ddots & \vdots \\
        P_1(x_M) & \cdots & P_{M_0}(x_M)
        \end{bmatrix}.
    \end{equation}
    Finally, this is equivalent to
    \begin{equation}
        \phi(x_1, \ldots, x_M) = \det\big( (\mat{T}_{(x_1, \ldots, x_M)})^T \mat{T}_{(x_1, \ldots, x_M)} \big)
    \end{equation}
    taking a nonzero value.
    We observe that $\phi$ is a polynomial on the Euclidean space $(\R^{m})^M$.
    
    Suppose there exists a set of points $(\bar{x}_1, \ldots, \bar{x}_M)\in(\R^m)^M$ such that $T_{(\bar{x}_1, \ldots, \bar{x}_M)}$ is injective.
    Then for this set $\phi(\bar{x}_1, \ldots, \bar{x}_M) \neq 0$.
    Obviously, $\phi(0, \ldots, 0) = 0$, meaning that $\phi$ cannot be constant.
    Thanks to the main result in \cite{Caron2005zero}, this means that each level set of $\phi$ has zero Lebesgue measure in $(\R^{m})^M$. 
    In particular, the level set $\phi^{-1}(0)$, consisting of those $x_1, \ldots, x_M$ for which $T_{(x_1, \ldots, x_M)}$ fails to be injective, has zero Lebesgue measure.
    Therefore, it suffices to prove that there exists $(\bar{x}_1, \ldots, \bar{x}_M)\in(\R^m)^M$ such that $T_{(\bar{x}_1, \ldots, \bar{x}_M)}$ is injective.
    We do this by induction.
    
    It is clear that there exists $\bar{x}_1$ so that the $1\times 1$ matrix
    \begin{equation}
        \mat{T}_1 = \begin{bmatrix}
            P_1(\bar{x}_1)
        \end{bmatrix}
    \end{equation}
    has full rank since $P_1$ cannot be the zero polynomial.
    Proceeding by induction, we assume that there exists $\bar{x}_1, \ldots, \bar{x}_s$ so that
    \begin{equation}
        \mat{T}_s = \begin{bmatrix}
            P_1(\bar{x}_1) & \cdots & P_{s}(\bar{x}_1) \\
            \vdots & \ddots & \vdots \\
            P_{1}(\bar{x}_s) & \cdots & P_s(\bar{x}_s)
        \end{bmatrix}
    \end{equation}
    has full rank.
    Suppose that the matrix
    \begin{equation}
        \mat{\tilde{T}}_{s+1}(x) = 
        \begin{bmatrix}
            P_1(\bar{x}_1) & \cdots & P_{s}(\bar{x}_1) & P_{s+1}(\bar{x}_1) \\
            \vdots & \ddots & \vdots & \vdots \\
            P_{1}(\bar{x}_s) & \cdots & P_s(\bar{x}_s) & P_{s+1}(\bar{x}_s) \\
            P_{1}(x) & \cdots & P_s(x) & P_{s+1}(x)
        \end{bmatrix}
    \end{equation}
    has rank $< s+1$ for every $x \in \R^m$.
    Since the upper left $s\times s$ block of $\mat{\tilde{T}}_{s+1}(x)$ is $\mat{T}_s$, we must always have $\rank(\mat{\tilde{T}}_{s+1}(x)) = s$.
    The nullspace of $\mat{\tilde{T}}_{s+1}(x)$ is contained in the nullspace of the upper $s\times(s+1)$ block of $\mat{\tilde{T}}_{s+1}(x)$.
    Since both nullspaces are one-dimensional, they are equal.
    The upper $s\times(s+1)$ block of $\mat{\tilde{T}}_{s+1}(x)$ does not depend on $x$, so there is a fixed nonzero vector $\vect{v}\in\R^{s+1}$ so that $\mat{\tilde{T}}_{s+1}(x) \vect{v} = \vect{0}$ for every $x\in\R^m$.
    The last row of this expression reads
    \begin{equation}
        v_1 P_1(x) + \cdots + v_{s+1} P_{s+1}(x) = 0 \qquad \forall x\in\R^m,
    \end{equation}
    contradicting the linear independence of $P_1, \ldots, P_{s+1}$.
    Therefore there exists $\bar{x}_{s+1}$ so that $\mat{T}_{s+1} = \mat{\tilde{T}}_{s+1}(\bar{x}_{s+1})$ has full rank.
    It follows by induction on $s$ that there exists $\bar{x}_1, \ldots \bar{x}_{M_0}\in \R^m$ so that $\mat{T}_{(\bar{x}_1, \ldots \bar{x}_{M_0})} = \mat{T}_{M_0}$ has full rank.
    Choosing any $M - M_0$ additional vectors yields an injective $\mat{T}_{(\bar{x}_1, \ldots \bar{x}_{M})}$, which completes the proof.
\end{proof}

\begin{proof}[Proposition~\ref{prop:generic_polynomial_map_sampling}]
    The sum in Eq.~\ref{eqn:empirical_L2_mu_inner_prod} clearly defines a symmetric, positive-semidefinite bilinear form on $\mcal{F}'$.
    It remains to show that this bilinear form is positive-definite.
    Suppose that there is a function $f \in \mcal{F}'$ such that $\langle f, f\rangle_{L^2(\mu_M)} = 0$.
    Thanks to Lemma~\ref{lem:generic_polynomial_sampling}, our assumption that $M \geq \dim(\pi_i(\mcal{F}'))$ means that the evaluation operator $T_{(x_1, \ldots, x_M)}$ is injective on $\pi_i(\mcal{F}')$ for almost every $(x_1, \ldots, x_M) \in (\R^m)^M$ with respect to Lebesgue measure.
    Since a countable (in this case finite) intersection of sets of measure zero has measure zero, it follows that for almost every $(x_1, \ldots, x_M) \in (\R^m)^M$ with respect to Lebesgue measure, $T_{(x_1, \ldots, x_M)}$ is injective on every $\pi_i(\mcal{F}')$, $i=1, \ldots, n$.
    Defining the positive diagonal matrix
    \begin{equation}
        \mat{D} = \frac{1}{\sqrt{N}} \begin{bmatrix}
            \sqrt{w_1} & & \\
            & \ddots & \\
            & & \sqrt{w_M}
        \end{bmatrix},
    \end{equation}
    and using Eq.~\ref{eqn:empirical_L2_mu_inner_prod} yields
    \begin{equation}
        0 = \langle f, f\rangle_{L^2(\mu_M)} = \sum_{j=1}^{n} \big(\mat{D} T_{(x_1, \ldots, x_M)} \pi_j f\big)^T \mat{D} T_{(x_1, \ldots, x_M)} \pi_j f .
    \end{equation}
    This implies that $T_{(x_1, \ldots, x_M)} \pi_j f = \vect{0}$ for $j=1, \ldots, n$.
    Since $T_{(x_1, \ldots, x_M)}$ is injective on each $\pi_j(\mcal{F}')$ it follows that each $\pi_j f = 0$, meaning that $f = 0$.
    This completes the proof.
\end{proof}

\section{Proof of Proposition~\ref{prop:properties_of_Lie_derivative}}
\label{app:properties_of_Lie_derivative}

    We begin by proving
    \begin{equation}
        \ddt \mcal{K}_{\exp(t\xi)}F 
        = \mcal{L}_{\xi} \mcal{K}_{\exp(t\xi)} F
        = \mcal{K}_{\exp(t\xi)} \mcal{L}_{\xi} F.
        \label{eqn:Lie_derivative_as_generator_proof_version}
    \end{equation}
    The first equality follows from the composition law for the operators $\mcal{K}_g$ and the definition of the Lie derivative in Eq.~\ref{eqn:Lie_derivative}, yielding
    \begin{equation}
            \left.\ddt\right\vert_{t=t_0} \mcal{K}_{\exp(t\xi)}F
            = \lim_{t\to 0} \frac{1}{t}\left[ \mcal{K}_{\exp(t\xi)}\mcal{K}_{\exp(t_0\xi)}F - \mcal{K}_{\exp(t_0\xi)}F \right]
            = \mcal{L}_{\xi} \mcal{K}_{\exp(t_0\xi)} F.
    \end{equation}
    To prove the second equality, we choose $p\in\mcal{M}$, let $p' = \theta_{\exp(t_0\xi)}(p)$, and compute
    \begin{equation}
        \begin{aligned}
            \left.\ddt\right\vert_{t=t_0} \big(\mcal{K}_{\exp(t\xi)}F\big)(p)
            &= \lim_{t\to 0} \frac{1}{t}\left[ \big(\mcal{K}_{\exp(t_0\xi)}\mcal{K}_{\exp(t\xi)}F\big)(p) - \big(\mcal{K}_{\exp(t_0\xi)}F\big)(p) \right] \\
            &= \lim_{t\to 0} \frac{1}{t} \Theta_{\exp(-t_0\xi)} \circ \left( \mcal{K}_{\exp(t\xi)} F - F \right) \circ \theta_{\exp(t_0\xi)}(p) \\
            &= \Theta_{\exp(-t_0\xi)}\left( \lim_{t\to 0} \frac{1}{t}\left[ \big(\mcal{K}_{\exp(t\xi)} F\big)(p') - F(p') \right] \right) \\
            &= \Theta_{\exp(-t_0\xi)}\left( \mcal{L}_{\xi}F (p') \right)
            = \big(\mcal{K}_{\exp(t_0\xi)} \mcal{L}_{\xi} F\big)(p).
        \end{aligned}
    \end{equation}
    In the third equality, we used the fact that $\Theta_{\exp(-t_0\xi)}$ is a continuous vector bundle homomorphism.

    Next, we prove
    \begin{equation}
        \mcal{L}_{\alpha\xi + \beta \eta} = \alpha \mcal{L}_{\xi} + \beta \mcal{L}_{\eta}.
        \label{eqn:linearity_of_Lie_derivative_wrt_Lie_algebra_proof_version}
    \end{equation}
    To do this, we choose $F\in\sect(E)$, $p\in\mcal{M}$, and define the map $h: G \to E_p$ by
    \begin{equation}
        h: g \mapsto \mcal{K}_g F(p) = \Theta\big( F(\theta(p,g)), g^{-1} \big).
        \label{eqn:pointwise_orbit_transformation}
    \end{equation}
    As a composition of smooth maps, $h$ is smooth, and its derivative at the identity is
    \begin{equation}
        \D h(e) \xi_e = \left.\ddt\right\vert_{t=0} h(\exp(t\xi)) = \mcal{L}_{\xi} F(p)
        \label{eqn:pointwise_orbit_transformation_tangent_map}
    \end{equation}
    for every $\xi_e \in T_e G  \cong \Lie(G)$.
    Since the derivative is linear, it follows that $\xi \mapsto \mcal{L}_{\xi} F(p)$ is linear.

    Finally, we prove that
    \begin{equation}
        \mcal{L}_{[\xi, \eta]} 
        = \frac{1}{2}\left.\frac{\td^2}{\td t^2}\right\vert_{t=0} \mcal{K}_{\exp(t\xi)}\mcal{K}_{\exp(t\eta)}\mcal{K}_{\exp(-t\xi)}\mcal{K}_{\exp(-t\eta)}
        = \mcal{L}_{\xi}\mcal{L}_{\eta} - \mcal{L}_{\eta}\mcal{L}_{\xi}.
        \label{eqn:Lie_derivative_commutator_and_second_derivative}
    \end{equation}
    Recall that $\Flow_{\xi}^t: g \mapsto g \cdot \exp(t\xi)$ gives the flow of the left-invariant vector field $\xi\in\Lie(G)$ (see Theorem~4.18(3) in \cite{Kolar1993natural}).
    By Theorem~3.16 in \cite{Kolar1993natural} the curve $\gamma: \R \to G$ given by
    \begin{equation}
        \gamma(t) 
        = \Flow_{-\eta}^t \circ \Flow_{-\xi}^t \circ \Flow_{\eta}^t \circ \Flow_{\xi}^t(e)
        = \exp(t\xi)\exp(t\eta)\exp(-t\xi)\exp(-t\eta).
    \end{equation}
    satisfies $\gamma(0) = e$, $\gamma'(0) = 0$, and
    \begin{equation}
        \frac{1}{2}\gamma''(0) = [\xi, \eta]_e \in T_e G
        \label{eqn:first_vanishing_derivative_recovers_Lie_bracket}
    \end{equation}
    in the sense that $\gamma''(0): f \mapsto (f \circ \gamma)''(0)$ is a derivation on $C^{\infty}(G)$.
    Composing with the map in Eq.~\ref{eqn:pointwise_orbit_transformation} yields
    \begin{equation}
        0 = \D h(e) \gamma'(0) 
        = (h\circ \gamma)'(0) 
        = \left.\ddt \right\vert_{t=0} \mcal{K}_{\gamma(t)} F(p).
        \label{eqn:vanishing_first_derivative_of_K_gamma}
    \end{equation}
    Combining Eq.~\ref{eqn:first_vanishing_derivative_recovers_Lie_bracket} and Eq.~\ref{eqn:pointwise_orbit_transformation_tangent_map} (noting the definition of the tangent map $\D h(e)$ acting on derivations, as in \cite{Kolar1993natural}, \cite{Lee2013introduction}) gives
    \begin{equation}
        \mcal{L}_{[\xi, \eta]} F(p)
        = \frac{1}{2} \D h(e) \gamma''(0)
        = \frac{1}{2} (h\circ\gamma)''(0)
        = \frac{1}{2} \left.\frac{\td^2}{\td t^2}\right\vert_{t=0} \mcal{K}_{\gamma(t)} F(p).
    \end{equation}
    This proves the first equality in Eq.~\ref{eqn:Lie_derivative_commutator_and_second_derivative} thanks to the composition law
    \begin{equation}
        \mcal{K}_{\gamma(t)} = \mcal{K}_{\exp(t\xi)}\mcal{K}_{\exp(t\eta)}\mcal{K}_{\exp(-t\xi)}\mcal{K}_{\exp(-t\eta)}.
    \end{equation}

    To differentiate the above expression, we use the following observations.
    If $F_t \in \sect(E)$ is such that $(t,p) \mapsto F_t(p)$ is smooth for $(t,p)\in\R\times\mcal{M}$, then obviously $\ddt F_t\in\sect(E)$ with the usual identification $T E_p \cong E_p$.
    Moreover, we have
    \begin{equation}
    \begin{aligned}
        \ddt \mcal{K}_g F_t(p) 
        &= \ddt \Theta_{g^{-1}}\big( F_t(\theta_g(p)) \big) \\
        &= \Theta_{g^{-1}}\Big( \ddt F_t(\theta_g(p)) \Big)
        = \mcal{K}_g \Big( \ddt F_t \Big)(p)
        \label{eqn:passing_derivative_under_Kg}
    \end{aligned}
    \end{equation}
    because $F_t(\theta_g(p)) \in E_{\theta_g(p)}$ for all $t\in\R$ and $\Theta_{g^{-1}}$ is linear on $E_{\theta_g(p)}$.
    Using this, we obtain
    \begin{equation}
    \begin{aligned}
        \ddt \mcal{L}_{\xi} F_t(p) 
        &= \ddt \left.\frac{\td}{\td \tau}\right\vert_{\tau=0} \mcal{K}_{\exp(\tau\xi)} F_t(p) \\
        &= \left.\frac{\td}{\td \tau}\right\vert_{\tau=0} \ddt \mcal{K}_{\exp(\tau\xi)} F_t(p) \\
        &= \left.\frac{\td}{\td \tau}\right\vert_{\tau=0} \mcal{K}_{\exp(\tau\xi)} \Big( \ddt F_t \Big)(p)
        = \mcal{L}_{\xi} \Big( \ddt F_t \Big)(p)
        \label{eqn:passing_derivative_under_Lxi}
    \end{aligned}
    \end{equation}
    because $(t,\tau) \mapsto \mcal{K}_{\exp(\tau\xi)} F_t(p)$ lies in the vector space $E_{p}$, allowing us to exchanged the order of differentiation.
    Since $(t_1, t_2, t_3, t_4) \mapsto \mcal{K}_{\exp(t\xi)}\mcal{K}_{\exp(t\eta)}\mcal{K}_{\exp(-t\xi)}\mcal{K}_{\exp(-t\eta)} F(p)$ lies in the vector space $E_p$ for all $(t_1, t_2, t_3, t_4)\in\R^4$, we can apply the chain rule and Eq.~\ref{eqn:passing_derivative_under_Kg} to obtain
    \begin{multline}
        \ddt \mcal{K}_{\gamma(t)}
        = \left.\frac{\partial}{\partial t_1}\right\vert_{t_1=t} \mcal{K}_{\exp(t_1\xi)}\mcal{K}_{\exp(t\eta)}\mcal{K}_{\exp(-t\xi)}\mcal{K}_{\exp(-t\eta)} \\
        + \mcal{K}_{\exp(t\xi)} \left.\frac{\partial}{\partial t_2}\right\vert_{t_2=t} \mcal{K}_{\exp(t_2\eta)}\mcal{K}_{\exp(-t\xi)}\mcal{K}_{\exp(-t\eta)} \\
        + \mcal{K}_{\exp(t\xi)}\mcal{K}_{\exp(t\eta)}\left.\frac{\partial}{\partial t_3}\right\vert_{t_3=t}\mcal{K}_{\exp(-t_3\xi)}\mcal{K}_{\exp(-t\eta)} \\
        + \mcal{K}_{\exp(t\xi)}\mcal{K}_{\exp(t\eta)}\mcal{K}_{\exp(-t\xi)}\left.\frac{\partial}{\partial t_4}\right\vert_{t_4=t}\mcal{K}_{\exp(-t_4\eta)}.
    \end{multline}
    Using Eq.~\ref{eqn:Lie_derivative_as_generator_proof_version} gives
    \begin{multline}
        \ddt \mcal{K}_{\gamma(t)}
        = \mcal{L}_{\xi} \overbrace{\mcal{K}_{\exp(t\xi)}\mcal{K}_{\exp(t\eta)}\mcal{K}_{\exp(-t\xi)}\mcal{K}_{\exp(-t\eta)}}^{\mcal{K}_{\gamma(t)}} \\
        + \mcal{K}_{\exp(t\xi)} \mcal{L}_{\eta} \mcal{K}_{\exp(t\eta)}\mcal{K}_{\exp(-t\xi)}\mcal{K}_{\exp(-t\eta)} \\
        + \mcal{K}_{\exp(t\xi)}\mcal{K}_{\exp(t\eta)}\mcal{L}_{-\xi}\mcal{K}_{\exp(-t\xi)}\mcal{K}_{\exp(-t\eta)} \\
        + \underbrace{\mcal{K}_{\exp(t\xi)}\mcal{K}_{\exp(t\eta)}\mcal{K}_{\exp(-t\xi)}\mcal{K}_{\exp(-t\eta)}}_{\mcal{K}_{\gamma(t)}}\mcal{L}_{-\eta}.
    \end{multline}
    Applying the same technique to differentiate a second time and using the linearity in Eq.~\ref{eqn:linearity_of_Lie_derivative_wrt_Lie_algebra_proof_version} to cancel terms yields
    \begin{equation}
        \left.\frac{\td^2}{\td t^2}\right\vert_{t=0} \mcal{K}_{\gamma(t)}
        = \mcal{L}_{\xi} \underbrace{\left.\ddt\right\vert_{t=0} \mcal{K}_{\gamma(t)}}_{0}
        + \underbrace{
        \mcal{L}_{\xi}\mcal{L}_{\eta} 
        + \mcal{L}_{\eta}\mcal{L}_{-\xi} 
        + \mcal{L}_{\eta}\mcal{L}_{-\xi}
        + \mcal{L}_{-\xi}\mcal{L}_{-\eta}
        }_{2\big( \mcal{L}_{\xi}\mcal{L}_{\eta} - \mcal{L}_{\eta} \mcal{L}_{\xi} \big)}
        + \underbrace{\left.\ddt\right\vert_{t=0} \mcal{K}_{\gamma(t)}}_{0} \mcal{L}_{-\eta},
    \end{equation}
    which completes the proof.
\hfill\qedsymbol

\section{Proof of Theorem~\ref{thm:equivariance_conditions_for_vb_section}}
\label{app:equivariance_conditions_for_vb_section}

    If $F\in\sect(E)$ is $G_0$-equivariant, then $\mcal{K}_{\exp(t\xi)} F = F$ for all $\xi\in\Lie(G)$ and $t\in\R$.
    Differentiating with respect to $t$ at $t=0$ gives $\mcal{L}_{\xi} F = 0$.
    Conversely, suppose that $\mcal{L}_{\xi} F = 0$ for all $\xi\in\Lie(G)$.
    Choosing $\xi\in\Lie(G)$ and $p\in\mcal{M}$, the smooth curve $\gamma:\R \to E_p$ defined by $\gamma(t) = \mcal{K}_{\exp(t\xi)} F(p)$ satisfies $\gamma(0) = F(p)$ and
    \begin{equation}
        \gamma'(t) = \ddt \mcal{K}_{\exp(t\xi)} F(p) = \mcal{K}_{\exp(t\xi)} \mcal{L}_{\xi} F(p) = 0 \in E_p
    \end{equation}
    thanks to Eq.~\ref{eqn:Lie_derivative_as_generator} of Proposition~\ref{prop:properties_of_Lie_derivative}.
    Therefore, $\gamma(t) = F(p)$ for all $t\in\R$, and in particular $\gamma(1) = F(p)$, meaning that 
    \begin{equation}
        \mcal{K}_{\exp(\xi)} F = F, \qquad \forall \xi\in\Lie(G).
    \end{equation}
    We show by induction that $\mcal{K}_{g} F = F$ for every finite product $g = h_m, \ldots, h_1$ of elements $h_i = \exp(\xi_i)$.
    Supposing this holds for any product $g$ of $m$ elements in the range of the exponential map and choosing $h_{m+1} = \exp(\xi_{m+1})$ we have
    \begin{equation}
        \mcal{K}_{h_{m+1} g} F = \mcal{K}_{h_{m+1}} \mcal{K}_{g} F = \mcal{K}_{h_{m+1}} F = F,
    \end{equation}
    proving the claim.
    The conclusion of the theorem now follows immediately from Lemma~\ref{lem:generation_by_finite_products}, below, since $G$ is the union of its connected components.
    \hfill\qedsymbol
    
    \begin{lemma}
        \label{lem:generation_by_finite_products}
        Let $G_0$ be the identity component of a Lie group $G$.
        Then every element $g\in G_0$ can be expressed as a finite product $g = h_m \cdots h_1$ of elements $h_i=\exp(\xi_i)$ for $\xi_i \in\Lie(G)$.
        Let $G_i$ be a connected component of $G$ and let $g_i \in G_i$.
        Then every element $g\in G_i$ can be expressed as $g = g_0 g_i$ for some $g_0 \in G_0$.
    \end{lemma}
    \begin{proof}
        By the inverse function theorem (more specifically by Proposition~20.8(f) in \cite{Lee2013introduction}), the range of the exponential map contains an open, connected neighborhood $\mcal{U}$ of the identity element $e \in G$.
        The inverses of the elements in $\mcal{U}$ also belong to the range of the exponential map thanks to Proposition~20.8(c) in \cite{Lee2013introduction}.
        By Proposition~7.14(b) and Proposition~7.15 in \cite{Lee2013introduction}, it follows that $\mcal{U}$ generates the identity component $G_0$ of $G$.
        That is, any element $g\in G_0$ can be written as a finite product of elements in $\mcal{U}$ and their inverses, which proves the first claim.
        
        By Proposition~7.15 in \cite{Lee2013introduction}, $G_0$ is a normal subgroup of $G$ and every connected component of $G$ is diffeomorphic to $G_0$.
        In fact in the proof of this result it is shown that every connected component of $G$ is a coset of $G_0$.
        Therefore, if $G_i$ is a non-identity connected component of $G$ and $g_i \in G_i$ then $G_i = G_0 \cdot g_i$, which proves the second claim.
    \end{proof}

\section{Proof of Theorem~\ref{thm:symmetries_of_sections}}
\label{app:symmetries_of_sections}

    We begin by showing that $\Sym_G(F)$ is a closed subgroup of $G$.
    It is obviously a subgroup, for if $g_1, g_2 \in \Sym_G(F)$ then
    \begin{equation}
        \mcal{K}_{g_1 g_2} F = \mcal{K}_{g_1} \mcal{K}_{g_2} F = \mcal{K}_{g_1} F = F,
    \end{equation}
    meaning that $g_1 g_2 \in \Sym_G(F)$.
    To show that $\Sym_G(F)$ is closed, we observe that for each $p\in\mcal{M}$, the map $h_p : G \to E$ defined by
    \begin{equation}
        h_p: g \mapsto \mcal{K}_g F(p) = \Theta\big( F(\theta(p,g)) , g^{-1} \big)
    \end{equation}
    is smooth, as it is a composition of smooth maps.
    As $F(p)$ is a single point in $E$, the preimage set $h_p^{-1}\big( \{ F(p) \} \big)$ is closed in $G$.
    Since $\Sym_G(F)$ is an intersection,
    \begin{equation}
        \Sym_G(F) = \bigcap_{p\in\mcal{M}} h_p^{-1}\big( \{ F(p) \} \big),
    \end{equation}
    of closed sets, it follows that $\Sym_G(F)$ is closed in $G$.
    By the closed subgroup theorem (Theorem~20.12 in \citet{Lee2013introduction}) it follows that $\Sym_G(F)$ is an embedded Lie subgroup of $G$.

    Let $\mfrak{h} = \Lie(\Sym_G(F))$ be the Lie algebra of $\Sym_G(F)$.
    By Theorem~\ref{thm:equivariance_conditions_for_vb_section}, (or simply by differentiating $\mcal{K}_{\exp(t\xi)} F = F$ with respect to $t$) we have $\mcal{L}_{\xi} F = 0$ for every $\xi \in \mfrak{h}$.
    This means that $\mfrak{h} \subset \sym_G(F)$, as defined by Eq.~\ref{eqn:sym_G_for_vb_section}.
    To show the reverse containment, choose $\xi \in \sym_G(F)$, and observe that Eq.~\ref{eqn:Lie_derivative_as_generator} in Proposition~\ref{prop:properties_of_Lie_derivative} implies that
    \begin{equation}
        \ddt \mcal{K}_{\exp(t\xi)} F = \mcal{K}_{\exp(t\xi)} \mcal{L}_{\xi} F = 0 \qquad \forall t\in\R.
    \end{equation}
    It follows that $\mcal{K}_{\exp(t\xi)} F = F$, that is, $\exp(t\xi)\in\Sym_G(F)$ for all $t\in\R$.
    Differentiating at $t=0$ proves that $\xi\in\mfrak{h}$.
    Therefore, $\mfrak{h} = \sym_G(F)$, completing the proof.
\hfill\qedsymbol

\section{Proof of Theorem~\ref{thm:symmetry_conditions_for_submanifold}}
\label{app:symmetry_conditions_for_submanifold}

Our proof of the theorem relies on the following technical lemma concerning the integral curves of vector fields tangent to weakly embedded, arcwise-closed submanifolds.

\begin{lemma}
    \label{lem:integral_curve_tangency}
    Let $\mcal{M}$ be an arcwise-closed weakly embedded submanifold of a manifold $\mcal{N}$. 
    Let $V\in\vf(\mcal{N})$ be a vector field tangent to $\mcal{M}$, that is
    \begin{equation}
        V_p \in T_p\mcal{M} \qquad \forall p\in\mcal{M}.
    \end{equation}
    If $\gamma:I \to \mcal{N}$ is a maximal integral curve of $V$ that intersects $\mcal{M}$, then $\gamma$ lies in $\mcal{M}$.
\end{lemma}
\begin{proof}
    By the translation lemma (Lemma~9.4 in \cite{Lee2013introduction}), we can assume without loss of generality that $0\in I$ and $p_0 = \gamma(0)\in\mcal{M}$.
    Let $\imath_{\mcal{M}}:\mcal{M} \hookrightarrow \mcal{N}$ denote the inclusion map.
    Since $\mcal{M}$ is an immersed submanifold of $\mcal{N}$ and $V$ is tangent to $\mcal{M}$, there is a unique smooth vector field $V\vert_{\mcal{M}}\in\vf(\mcal{M})$ that is $\imath_{\mcal{M}}$-related to $V$ thanks to Proposition~8.23 in \cite{Lee2013introduction}.
    Let $\tilde{\gamma}:\tilde{I} \to \mcal{M}$ be the maximal integral curve of $V\vert_{\mcal{M}}$ with $\tilde{\gamma}(0) = p_0$.
    By the naturality of integral curves (Proposition~9.6 in \cite{Lee2013introduction}) $\imath_{\mcal{M}} \circ \tilde{\gamma}$ is an integral curve of $V$ with $\imath_{\mcal{M}} \circ \tilde{\gamma}(0) = p_0$.
    Since integral curves of smooth vector fields starting at the same point are unique (Theorem~9.12, part~(a) in \cite{Lee2013introduction}) we have $\tilde{I} \subset I$ and 
    \begin{equation}
        \imath_{\mcal{M}} \circ \tilde{\gamma}(t) = \gamma(t) \qquad \forall t \in \tilde{I}.
        \label{eqn:equality_of_integral_curves}
    \end{equation}
    Therefore, it remains to show that $\tilde{I} = I$.
    
    By the local existence of integral curves (Proposition~9.2 in \cite{Lee2013introduction}), the domains $I$ and $\tilde{I}$ of the maximal integral curves $\gamma$ and $\tilde{\gamma}$ are open intervals in $\R$.
    Suppose, for the sake of producing a contradiction, that there exists $t\in I$ with $t > \tilde{I}$.
    Then it follows that the least upper bound $b = \sup \tilde{I}$ is an element of $I$.
    By Eq.~\ref{eqn:equality_of_integral_curves} and continuity of $\gamma$ we have
    \begin{equation}
        q_0 = \gamma(b) = \lim_{t\to b} \imath_{\mcal{M}} \circ \tilde{\gamma}(t).
        \label{eqn:upper_limit_of_integral_curve}
    \end{equation}
    Since $\mcal{M}$ is arcwise-closed, it follows that $q_0 \in \mcal{M}$.
    
    To complete the proof, we use the local existence of an integral curve for $V\vert_{\mcal{M}}$ starting at $q_0$ to contradict the maximality of $\tilde{\gamma}$.
    By the local existence of integral curves (Proposition~9.2 in \cite{Lee2013introduction}) and the translation lemma (Lemma~9.4 in \cite{Lee2013introduction}), there is an $\varepsilon > 0$ and an integral curve $\hat{\gamma}: (b-\varepsilon, b+\varepsilon) \to \mcal{M}$ of $V\vert_{\mcal{M}}$ such that $\hat{\gamma}(b) = q_0 = \gamma(b)$.
    Shrinking the interval, we take $0<\varepsilon < b-a$.
    Again, by nauturality and uniqueness of integral curves we must have $\imath_{\mcal{M}} \circ \hat{\gamma}(t) = \gamma(t)$ for all $t\in (b-\varepsilon, b+\varepsilon)$.
    Hence, by Eq.~\ref{eqn:equality_of_integral_curves} and injectivity of $\imath_{\mcal{M}}$ it follows that $\hat{\gamma}(t) = \tilde\gamma(t)$ for all $t\in (b-\varepsilon, b)$.
    Applying the gluing lemma (Corollary~2.8 in \cite{Lee2013introduction}) to $\tilde{\gamma}$ and $\hat{\gamma}$ yields an extension of $\tilde{\gamma}$ to the larger open interval $\tilde{I} \cup (b-\varepsilon, b+\varepsilon)$.
    Since this contradicts the maximality of $\tilde{\gamma}$, there is no $t \in I$ with $t > \tilde{I}$.
    The same argument shows that there is no $t \in I$ with $t < \tilde{I}$, and so we must have $\tilde{I} = I$.
\end{proof}

\begin{proof}[Theorem~\ref{thm:symmetry_conditions_for_submanifold}]
First, suppose that $\mcal{M}$ is $G_0$-invariant.
In particular, this means that for every $p\in\mcal{M}$ and $\xi\in\Lie(G)$, the smooth curve $\gamma_{\xi}^{(p)}:\R \to \mcal{N}$ defined by
\begin{equation}
    \gamma_{\xi}^{(p)}(t) = p \cdot \exp(t\xi)
\end{equation}
lies in $\mcal{M}$.
Since $\mcal{M}$ is weakly embedded in $\mcal{N}$, $\gamma_{\xi}^{(p)}$ is also smooth as a map into $\mcal{M}$.
Specifically, there is a smooth curve $\tilde{\gamma}_{\xi}^{(p)}:\R \to \mcal{M}$ so that $\gamma_{\xi}^{(p)} = \imath_{\mcal{M}} \circ \tilde{\gamma}_{\xi}^{(p)}$ where $\imath_{\mcal{M}} : \mcal{M} \hookrightarrow \mcal{N}$ is the inclusion map.
Differentiating at $t=0$ yields
\begin{equation}
    \hat{\theta}(\xi)_p
    = \left.\ddt \gamma_{\xi}^{(p)}(t) \right\vert_{t=0}
    = \D \imath_{\mcal{M}}(p) \left.\ddt \tilde{\gamma}_{\xi}^{(p)}(t) \right\vert_{t=0},
\end{equation}
which lies in $T_p \mcal{M} = \Range\left( \D \imath_{\mcal{M}}(p) \right)$.

Conversely, suppose that the tangency condition expressed in Eq.~\ref{eqn:generator_manifold_tangency} holds.
By Lemma~20.14 in \cite{Lee2013introduction} the vector fields $\xi \in \Lie(G)$ and $\hat{\xi} = \hat{\theta}(\xi)$ are $\theta^{(p)}$-related for every $p\in\mcal{N}$.
By the naturality of integral curves (Proposition~9.6 in \cite{Lee2013introduction}) it follows that $\gamma_{\xi}^{(p)}:\R \to \mcal{N}$ defined by
\begin{equation}
    \gamma_{\xi}^{(p)}(t) = p \cdot \exp(t\xi)
\end{equation}
is the unique maximal integral curve of $\hat{\xi}$ passing through $p$ at $t=0$.
When $p\in\mcal{M}$, this integral curve lies in $\mcal{M}$ thanks to Lemma~\ref{lem:integral_curve_tangency}.
This means that $\mcal{M}$ is invariant under the action of any group element in the range of the exponential map on $G$.

Proceeding by induction, suppose that $G$ is invariant under the action of any product of $m$ group elements in the range of the exponential map.
If $g = h_1 \cdots h_m \cdot h_{m+1}$ is a product of $m+1$ elements $h_i \in \Range(\exp)\subset G$,
then it follows from associativity and the induction hypothesis that
\begin{equation}
    \mcal{M} \cdot (h_1 \cdots h_m \cdot h_{m+1}) = (\mcal{M} \cdot h_1 \cdots h_m) \cdot h_{m+1} \subset \mcal{M} \cdot h_{m+1} \subset \mcal{M}.
\end{equation}
Therefore, $\mcal{M}$ is invariant under the action of any finite product of group elements in the range of the exponential map by induction on $m$.
Thanks to Lemma~\ref{lem:generation_by_finite_products}, it follows that $\mcal{M}$ is $G_0$-invariant, proving the first claim of the theorem.

Now suppose in addition that for each connected component $G_i$ of $G$ there is an element $g_i \in G_i$ satisfying $\mcal{M} \cdot g_i \subset \mcal{M}$.
By Lemma~\ref{lem:generation_by_finite_products}, we have $G_i = G_0 \cdot g_i$.
Therefore,
\begin{equation}
    \mcal{M} \cdot G_i 
    = \mcal{M} \cdot G_0 \cdot g_i \subset \mcal{M} \cdot g_i \subset \mcal{M},
\end{equation}
proving that $\mcal{M}$ is invariant under $G_i$.
Since $G$ is the union of its connected components it follows that $\mcal{M}$ is $G$-invariant.
\end{proof}

\section{Proof of Theorem~\ref{thm:symmetries_of_a_submanifold}}
\label{app:symmetries_of_a_submanifold}

    The set $\sym_G(\mcal{M})$ is a subspace of $\Lie(G)$, for if $\xi_1, \xi_2 \in \sym_G(\mcal{M})$ and $a_1, a_2 \in \R$ then
    \begin{equation}
        \hat{\theta}(a_1 \xi_1 + a_2 \xi_2)_p = a_1 \underbrace{\hat{\theta}(\xi_1)}_{\in T_p\mcal{M}} + a_2 \underbrace{\hat{\theta}(\xi_2)}_{\in T_p\mcal{M}} \in T_p\mcal{M}
    \end{equation}
    thanks to linearity of the infinitesimal generator $\hat{\theta}$.
    To show that $\sym_G(\mcal{M})$ is a Lie subalgrebra, we must show that it is also closed under the Lie bracket.
    Recall that $\hat{\theta}$ is a Lie algebra homomorphism (see Theorem~20.15 in \cite{Lee2013introduction}), and so $\hat{\theta}([\xi_1, \xi_2]) = [\hat{\theta}(\xi_1), \hat{\theta}(\xi_1)]$.
    Since the Lie bracket of two vector fields tangent to an immersed submanifold is also tangent to the submanifold (see Corollary~8.32 in \cite{Lee2013introduction}), it follows that $[\hat{\theta}(\xi_1), \hat{\theta}(\xi_1)]$ is tangent to $\mcal{M}$.
    Hence, $\sym_G(\mcal{M})$ is closed under the Lie bracket and is therefore a Lie subalgebra of $\Lie(G)$.
    By Theorem~19.26 in \cite{Lee2013introduction}, there is a unique connected Lie subgroup of $\Sym_G(\mcal{M})_0\subset G$ whose Lie subalgebra is $\sym_G(\mcal{M})$.

    Now suppose that $\mcal{M}$ is weakly embedded and arcwise-closed in $\mcal{N}$.
    Since $\Sym_G(\mcal{M})_0$ is connected, it is equal to its identity component.
    Applying Theorem~\ref{thm:symmetry_conditions_for_submanifold} to the induced action of $\Sym_G(\mcal{M})_0$ on $\mcal{M}$, we immediately obtain $\mcal{M} \cdot \Sym_G(\mcal{M})_0 \subset \mcal{M}$.
    Suppose that $H$ is another connected Lie subgroup of $G$ such that $\mcal{M}\cdot H \subset \mcal{M}$.
    Choosing any $p\in\mcal{M}$ and $\xi \in \Lie(H)$, we have
    \begin{equation}
        p \cdot \exp(t\xi) \in \mcal{M} \qquad \forall t\in \R.
    \end{equation}
    Since $\mcal{M}$ is weakly embedded in $\mcal{N}$, this defines a smooth curve $\gamma: \R \to \mcal{M}$ such that $\imath_{\mcal{M}} \circ \gamma(t) = p \cdot \exp(t\xi)$.
    Differentiating and using the definition of the infinitesimal generator gives
    \begin{equation}
        \hat{\theta}(\xi)_p = \left.\frac{\td}{\td t}\right\vert_{t=0} p \cdot \exp(t\xi) = \D \imath_{\mcal{M}}(p) \left.\frac{\td}{\td t}\right\vert_{t=0} \gamma(t) \in T_p \mcal{M}.
    \end{equation}
    Therefore, $\Lie(H) \subset \sym_G(\mcal{M})$ which implies that $H \subset \Sym_G(\mcal{M})_0$ by Theorem~19.26 in \cite{Lee2013introduction}.

    Now suppose that $\mcal{M}$ is properly embedded in $\mcal{N}$ and denote
    \begin{equation}
        \Sym_G(\mcal{M}) 
        = \{ g\in G \ : \mcal{M} \cdot g \subset \mcal{M} \}
        = \bigcap_{p\in\mcal{M}} \big(\theta^{(p)}\big)^{-1}(\mcal{M}).
    \end{equation}
    The equality of these expressions is a simple matter of unwinding their definitions.
    It is clear that $\Sym_G(\mcal{M})$ is a subgroup of $G$, for if $g_1, g_2 \in \Sym_G(\mcal{M})$ then the composition law for the group action gives $\mcal{M} \cdot (g_1\cdot g_2) = (\mcal{M} \cdot g_1) \cdot g_2 \subset \mcal{M} \cdot g_1 \subset \mcal{M}$.
    Since $\mcal{M}$ is properly embedded, it is closed in $\mcal{M}$ (see \citet[Proposition~5.5]{Lee2013introduction}), meaning that each preimge set $\big(\theta^{(p)}\big)^{-1}(\mcal{M})$ is closed in $G$ by continuity of $\theta^{(p)}$.
    As an intersection of closed subsets, it follows that $\Sym_G(\mcal{M})$ is closed in $G$.
    By the closed subgroup theorem (\citet[Theorem~20.12]{Lee2013introduction}), $\Sym_G(\mcal{M})$ is a properly embedded Lie subgroup of $G$.
    The same holds for the identity component $H_0$ of $\Sym_G(\mcal{M})$ since $H_0$ is closed in $\Sym_G(\mcal{M})$, which implies that $H_0$ is closed in $G$ thanks to the closure of $\Sym_G(\mcal{M})$ in $G$.
    
    Next we show that $\Sym_G(\mcal{M})_0 = H_0$ is the identity component of $\Sym_G(\mcal{M})$.
    First, we observe that $\Sym_G(\mcal{M})_0 \subset H_0$ because $\Sym_G(\mcal{M})_0$ is connected and contained in $\Sym_G(\mcal{M})$.
    The reverse containment follows from the fact that $H_0$ is a connected Lie subgroup satisfying $\mcal{M}\cdot H_0 \subset \mcal{M}$, which by our earlier result implies that $H_0 \subset \Sym_G(\mcal{M})_0$.
\hfill\qedsymbol

\section{Proof of Theorem~\ref{thm:characterization_of_Lie_derivative}}
\label{app:characterization_of_Lie_derivative}
\begin{proof}[Lemma~\ref{lem:injection_of_E_into_TE}]
    The map $\imath_F$ defined in a local trivialization $\Phi$ by Eq.~\ref{eqn:injection_E_to_TE} is injective.
    It is a vector bundle homomorphism because $\D\Phi \circ \imath_F \circ \Phi^{-1}$, $\Phi$, and $\D \Phi$ are vector bundle homomorphisms and $\Phi$ and $\D \Phi$ are invertible.
    It remains to show that the definition of $\imath_F$ does not depend on the choice of local trivialization.
    Given two local trivializations $\Phi$ and $\tilde{\Phi}$ defined on $\pi^{-1}(\mcal{U}) \subset E$ where $\mcal{U}$ is an open subset of $\mcal{M}$, it suffices to show that the following diagram commutes:
    \begin{equation}
        \begin{tikzcd}
	       {\mcal{U}\times\R^k} & {\pi^{-1}(\mcal{U})\subset E} & {\mcal{U}\times\R^k} \\
	       {T(\mcal{U}\times\R^k)} & {T(\pi^{-1}(\mcal{U}))\subset TE} & {T(\mcal{U}\times\R^k)}
	       \arrow["{\imath_F}", from=1-2, to=2-2]
	       \arrow["{\imath_{\Phi\circ F}}", from=1-1, to=2-1]
	       \arrow["{\imath_{\tilde{\Phi}\circ F}}", from=1-3, to=2-3]
	       \arrow["\Phi"', from=1-2, to=1-1]
	       \arrow["{\D\Phi}", from=2-2, to=2-1]
	       \arrow["{\tilde{\Phi}}", from=1-2, to=1-3]
	       \arrow["{\D\tilde{\Phi}}"', from=2-2, to=2-3]
        \end{tikzcd}
    \end{equation}
    Since $\tilde{\Phi} \circ \Phi^{-1}$ is a bundle homomorphism descending to the identity, it can be written as
    \begin{equation}
        \tilde{\Phi} \circ \Phi^{-1}: (p,\vect{v}) \mapsto (p, \mat{T}(p) \vect{v})
    \end{equation}
    for a matrix-valued function $\mat{T}:\mcal{U} \to \R^{k\times k}$.
    Moreover, the matrices are invertible because the local trivializations are bundle isomorphisms.
    Differentiating, we obtain
    \begin{equation}
        \D \tilde{\Phi} \circ \D \Phi^{-1} 
        : (w_p, \vect{w})_{(p, \vect{v})} \mapsto
        \big(w_p, \D \mat{T}(p) w_p + \mat{T}(p)\vect{w}\big)_{\tilde{\Phi} \circ \Phi^{-1}(p,\vect{v})},
    \end{equation}
    where $w_p \in T_p \mcal{U}$.
    Composing this with $\imath_{\Phi \circ F}:(p, \vect{v}) \mapsto (0, \vect{v})_{\Phi(F(p))}$, we obtain
    \begin{equation}
        \D \tilde{\Phi} \circ \D \Phi^{-1} \circ \imath_{\Phi \circ F}
        (p, \vect{v})
        = (0, \mat{T}(p)\vect{v})_{\tilde{\Phi}(F(p))}
        = \imath_{\tilde{\Phi}\circ F}(p, \mat{T}(p)\vect{v})
        = \imath_{\tilde{\Phi}\circ F} \circ \tilde{\Phi} \circ \Phi^{-1}(p, \vect{v}),
    \end{equation}
    proving that the diagram commutes.
\end{proof}

\begin{proof}[Theorem~\ref{thm:characterization_of_Lie_derivative}]
We observe that $F\circ\pi:E \to E$ is a smooth idempotent map whose image is $\image(F) \subset E$.
By differentiating the expression $(F\circ\pi) \circ (F \circ\pi) = F\circ \pi$ at a point $F(p) \in \image(F)$, we obtain
\begin{equation}
    \D (F\circ\pi)(F(p)) \D (F\circ\pi)(F(p)) = \D (F\circ\pi)(F(p)),
\end{equation}
meaning that $\D (F\circ\pi)(F(p)):T_{F(p)}E \to T_{F(p)}E$ is a linear projection.
Since 
\begin{equation}
    \D (F\circ\pi)(F(p)) = \D F(p) \D \pi(F(p)),
\end{equation}
we have $\Range\big(\D (F\circ\pi)(F(p))\big) \subset \Range(\D F(p)) = T_{F(p)}\image(F)$.
Differentiating $F = (F\circ\pi)\circ F$ yields 
\begin{equation}
    \D F(p) = \D (F\circ \pi)(F(p)) \D F(p),
\end{equation}
meaning that $T_{F(p)}\image(F) \subset \Range\big(\D (F\circ\pi)(F(p))\big)$.
Since $\Range\big(\D (F\circ\pi)(F(p))\big) = T_{F(p)}\image(F)$ it follows that $\D (F\circ\pi)(F(p))$ is a linear projection onto $T_{F(p)}\image(F)$. 

We observe that the generalized Lie derivative in Eq.~\ref{eqn:Lie_derivative} can be expressed as
\begin{equation}
    \begin{aligned}
        (\mcal{L}_{\xi} F)(p) &= \lim_{t\to 0} \frac{1}{t}\left[ \Theta_{\exp(-t\xi)}(F( \theta_{\exp(t\xi)}(p))) - F(p) \right] \\
        &= \lim_{t\to 0} \Theta\left(\exp(-t\xi), \ \frac{1}{t}\left[ F(\theta_{\exp(t\xi)}(p)) - \Theta_{\exp(t\xi)}(F(p)) \right] \right) \\
        &= \lim_{t\to 0} \frac{1}{t}\left[ F(\theta_{\exp(t\xi)}(p)) - \Theta_{\exp(t\xi)}(F(p)) \right].
    \end{aligned}
\end{equation}
The first equality follows because $\Theta_{g^{-1}}$ is a vector bundle homomorphism, meaning that the restricted map $\Theta_{g^{-1}}\vert_{E_{p\cdot g}}: E_{p\cdot g} \to E_p$ is linear; here $g = \exp(t\xi)$.
The second equality follows because $\Theta: E \times G \to E$ is continuous.
Note that in the first expression the limit is taken in the vector space $E_p$, whereas in the last expression the limit must be taken in $E$.

We proceed by expressing everything in a local trivialization $\Phi:\pi^{-1}(\mcal{U}) \to \mcal{U}\times \R^k$ of an open neighborhood $\mcal{U}\subset \mcal{M}$ of $p \in \mcal{M}$.
Since the maps $\Theta_g$, $\Phi$, and $\Phi^{-1}$ are vector bundle homomorphisms, there is a matrix-valued function $\mat{T}_g:\mcal{U} \to \R^{k\times k}$ such that
\begin{equation}
    \tilde{\Theta}_g = \Phi \circ \Theta_g \circ \Phi^{-1}: 
    (p,\vect{v}) \mapsto (\theta_g(p), \ \mat{T}_g(p)\vect{v}).
\end{equation}
Differentiating $\tilde{\Theta}_{\exp(t\xi)}(p,\vect{v})$ with respect to $t$ yields the generator
\begin{equation}
    \hat{\tilde{\Theta}}(\xi)_{(p,\vect{v})} 
    = \left.\ddt\right\vert_{t=0} \tilde{\Theta}_{\exp(t\xi)}(p,\vect{v})
    = \left(\hat{\theta}(\xi)_p, \ \mat{\hat{T}}(\xi)_p \vect{v}\right)_{(p,\vect{v})},
\end{equation}
where $\mat{\hat{T}}(\xi)_p = \left.\ddt\right\vert_{t=0} \mat{T}_{\exp(t\xi)}(p)$. 
We define the function $\vect{\tilde{F}}:\mcal{U} \to \R^k$ by
\begin{equation}
    (p, \vect{\tilde{F}}(p)) = \Phi \circ F(p) \qquad \forall p\in\mcal{U}.
\end{equation}

Using the above definitions, we can express the generalized Lie derivative in the local trivialization:
\begin{equation}
    \begin{aligned}
        \Phi \circ (\mcal{L}_{\xi} F) (p) 
        &= 
        \lim_{t\to 0} \left( \theta_{\exp(t\xi)}(p) ,\ 
        \frac{1}{t}\left[ \vect{\tilde{F}}(\theta_{\exp(t\xi)}(p)) - \mat{T}_{\exp(t\xi)} \vect{\tilde{F}}(p) \right] \right) \\
        &= \left( p, \ \D\vect{\tilde{F}}(p)\hat{\theta}(\xi)_p - \mat{\hat{T}}(\xi)_p\vect{\tilde{F}}(p) \right).
    \end{aligned}
    \label{eqn:Lie_derivative_in_local_trivialization}
\end{equation}
Applying Lemma~\ref{lem:injection_of_E_into_TE} allows us to express the left-hand-side of Eq.~\ref{eqn:Lie_derivative_as_projection} as
\begin{equation}
    \D\Phi \left[ \imath_F \circ (\mcal{L}_{\xi} F) (p) \right] =
    \left(0, \ \D\vect{\tilde{F}}(p)\hat{\theta}(\xi)_p - \mat{\hat{T}}(\xi)_p\vect{\tilde{F}}(p) \right)_{\Phi(F(p))}.
    \label{eqn:lifted_Lie_derivative_in_local_trivialization}
\end{equation}
We can also express the quantities on the right-hand-side of Eq.~\ref{eqn:Lie_derivative_as_projection} in the local trivialization.
To do this, we compute
\begin{equation}
    \begin{aligned}
        \D \Phi(F(p)) \D (F \circ \pi)(F(p)) \hat{{\Theta}}(\xi)_{F(p)} 
        &= \left.\ddt\right\vert_{t=0} \Phi \circ F \circ \pi \circ \Theta_{\exp(t\xi)}(F(p)) \\
        &= \left.\ddt\right\vert_{t=0} \Phi \circ F(\theta_{\exp(t\xi)}(p)) \\
        &= \left( \hat{\theta}(\xi)_p, \ \D\vect{\tilde{F}}(p)\hat{\theta}(\xi)_p \right)_{\Phi(F(p))}
    \end{aligned}
\end{equation}
and
\begin{equation}
    \begin{aligned}
        \D\Phi(F(p)) \hat{\Theta}(\xi)_{F(p)} 
        &= \left. \ddt \right\vert_{t=0} \Phi \circ \Theta_{\exp(t\xi)} \circ \Phi^{-1} \circ \Phi \circ F(p) \\
        &= \hat{\tilde{\Theta}}(\xi)_{\Phi(F(p))}
        = \left( \hat{\theta}(\xi)_p, \ \mat{\hat{T}}(\xi)_p \vect{\tilde{F}}(p) \right)_{\Phi(F(p))}.
    \end{aligned}
\end{equation}
Subtracting these yields
\begin{equation}
    \D\Phi \left[ \left( \D(F\circ\pi)(F(p)) - \Id_{T_{F(p)}E} \right) \hat{\Theta}(\xi)_{F(p)} \right]
    = \left( 0, \ \D\vect{\tilde{F}}(p)\hat{\theta}(\xi)_p - \mat{\hat{T}}(\xi)_p \vect{\tilde{F}}(p) \right)_{\Phi(F(p))},
\end{equation}
which, upon comparison with Eq.~\ref{eqn:lifted_Lie_derivative_in_local_trivialization} completes the proof.
\end{proof}
\end{document}