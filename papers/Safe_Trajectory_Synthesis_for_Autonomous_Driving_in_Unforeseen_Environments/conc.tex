\section{CONCLUSION}
\label{sec:conclusion}

% \Ram{Want to mention that we can do parameteric uncertainty \cite{mohan2016convex}, hybrid \cite{shia2014convex}, and alpha confidence \cite{holmes2016convex}
% Also we can pose the optimization problem using convex optimization based methods \cite{zhao2016control,zhao2017optimal}.}

This paper presents a method to plan safe trajectories with obstacle avoidance for autonomous vehicles.
This approach is able to guarantee safety in arbitrary environments for multiple, static obstacles.
The method begins with computing the forward reachable set (FRS) of parameterized trajectories that a vehicle can realize.
This set is computed in continuous space and time, and is robust to model uncertainty between the dynamics of the vehicle's mid- and low-level controllers.

As an example, we use a kinematic Dubin's car and dynamic unicycle model as low- and high-fidelity models.
At runtime, the FRS is intersected with obstacles to eliminate unsafe trajectories, and an optimal trajectory is chosen from the remaining, safe trajectories.
This method is proven to ensure vehicle safety for present and future static obstacles by considering the time required for path planning, the time required to stop the vehicle, and the error between the low- and high-fidelity models.
One thousand simulations with randomly-located obstacles were run to show the effectiveness of this method. 

%We are preparing a Segway as a platform to demonstrate this method on a real system.
%We will also apply this method to higher degree-of-freedom vehicle models in both simulation and experiments to further develop robustness to environmental and dynamic uncertainty.

The next step in applying this method is to expand the error function $g$ beyond modeling uncertainty.
To reduce the conservativeness of the approach, we plan to explore extension to the FRS computation that incorporate confidence level sets \cite{mohan2016convex,holmes2016convex}.
The error function $g$ can also be improved by considering time variation of trajectory parameters across planning steps by posing the FRS computation as a hybrid problem \cite{shia2014convex}.
Finally, we plan to use convex optimization to find a global solution to the nonlinear trajectory optimization problem at each planning step \cite{zhao2016control,zhao2017optimal}.