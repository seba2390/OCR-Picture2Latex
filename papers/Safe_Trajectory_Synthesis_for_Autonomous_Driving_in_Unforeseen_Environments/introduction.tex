\section{INTRODUCTION}
\label{sec:introduction}

Safe control of autonomous vehicles for use on public roads is a challenging problem that directly impacts both public safety and development costs for auto manufacturers. 
To be safe, autonomous vehicles must be able to account for errors in their sensors and dynamic models.
Moreover, since it is impractical to preplan an obstacle avoidance trajectory for every plausible environment and configuration of obstacles, controllers need to be constructed, evaluated for correctness, and selected in real-time. 
%Lastly, due to the extensive mileage and time required to validate safety \cite{kalra2016driving}, it is valuable to develop controllers with a guarantee of safety.

Autonomous vehicles typically apply a hierarchical control design structure with three levels \cite{falcone2007predictive,buehler2009darpa,gray2012predictive}. 
The highest-layer in this architecture is responsible for route planning using Dijkstra's algorithm or its variants. 
The mid-level controller generates trajectories to perform actions such as lane-keeping or obstacle avoidance. 
The low-level controller follows these generated trajectories by applying throttle, brake, and steering inputs. 
For computational efficiency, the mid-level controller often uses a lower degree of freedom model than the low-level controller. 
As a result, there is typically a non-trivial gap between the trajectories generated by the mid-level controller and what a vehicle is capable of executing.
This paper develops a real-time optimization based scheme for the mid-level controller that can provably design trajectories which can be safely followed by the low-level controller in arbitrary environments with static obstacles.

% \subsection{Related Work} \label{related_work}
Various methods exist in the literature for mid-level controller design, which rely upon either temporal or spatial discretization, or both.
Rapidly Random Exploring Trees, for example, compute safe trajectories by sampling from a vehicle's input space and numerically integrating trajectories forward using a dynamic model \cite{lavalle2001randomized}. 
These methods can handle nonlinear dynamics and non-convex constraints and can even be made asymptotically optimal \cite{karaman2011sampling}.
%, they rely on time-discretization during integration which can generate trajectories that may be impossible to perfectly execute.
Partially Observable Markov Decision Processes are also able to handle nonlinear dynamics, non-convex constraints, and bounded uncertainty, while planning trajectories in a spatially and temporally discretized space\cite{brechtel2014probabilistic}. 
Though each of these mid-level control design methods is powerful, they struggle to account for the uncertainty introduced by discretization, which can result in trajectories that are impossible to execute.
For this reason, others have focused solely on the real-time verification of the safety of a generated mid-level controller using approaches grounded in zonotopes \cite{althoff2014online}.

\begin{figure}
\includegraphics[width=1\columnwidth]{doodle_XK.pdf}
\caption{An illustration of the Forward Reachable Set in the physical world (right) and the trajectory parameter space (left). 
Each trajectory parameter on the left corresponds to a trajectory in the physical world. 
The FRS (the funnel beginning from the blue car) is intersected with obstacles (the road boundaries and the red car) at run-time to identify trajectory parameters that lead to collisions (orange and red sets in left picture). 
In this sketch, the parameters labeled 1 produce a guaranteed-safe trajectory, whereas the parameters labeled 2 result in a collision.
Optimal parameters can be selected from the safe set, $K_\text{safe}$.}
\label{fig:overallsketch}
\end{figure}

Several approaches have recently been pursued to perform control design with safety guarantees.
Robust Model Predictive Control (MPC), for example, has been demonstrated to successfully account for uncertainty in the dynamics and can be used for safe control design for nonlinear vehicle models in the presence of non-convex constraints \cite{gao2014tube,shia2014semiautonomous}.
This requires linearizing the dynamics about a pre-specified, spatially discretized trajectory to synthesize a controller that is able to safely follow this trajectory while avoiding obstacles.
Thus, this approach requires solving a scenario-specific nonlinear program, which can be challenging to solve at run-time and, as a result, has few associated safety guarantees.
To try to address the more general safe control design problem, others have employed the Hamilton-Jacobi-Bellman Equation, which is typically solved using state-space discretization, to compute the set of safe feedback controllers in the presence of bounded disturbance \cite{ding2011reachability}.
These approaches rely upon a complete characterization of obstacle locations, which can make real-time applications challenging due to the computational expense associated with state-space discretization.
Despite this limitation, extensions of this approach have been successfully employed to perform cooperative control for connected vehicles \cite{dhinakaran2017hybrid}.
Similarly, the viability kernel, which is also computed using state-space discretization, has been employed to evaluate the set of possible trajectories for autonomous vehicles on pre-defined scenarios \cite{liniger2017real}. 

To avoid this curse of dimensionality, others have employed barrier functions using Lyapunov theory to devise safety-preserving controllers for adaptive cruise control and lane-keeping \cite{nilsson2016correct,xu2016correctness,wang2017safety}.
These approaches can quickly solve a quadratic program at run-time to guarantee safety, but the barrier function, which is computed off-line, is scenario specific. 
To overcome this situation-specific limitation, an approach relying upon funnel libraries was recently proposed to perform real-time control design in the presence of uncertainty \cite{majumdar2016funnel}.
This approach relies upon pre-computing a rich-enough finite family of trajectories, which is then searched at run-time to ensure safety. 

%The obstacles are entered as constraints in discretized space or time, and recursive feasibility may not even be guaranteed. 
%, but the trajectories are planned in discrete space and time, again introducing error related to grid size . 
%Furthermore the complete reachable set is neither computed nor considered during planning.

%Overall, no approach has yet been shown to account for nonlinear dynamics and nonconvex constraints, without pre-specified environments or trajectories, and without time or state space discretization.
This paper presents an approach for mid-level control design that simultaneously accounts for nonlinear dynamics, non-convex constraints, and uncertainty without pre-specifying the scenario, pre-computing a finite family of trajectories, or relying on spatial or temporal discretization. 
The proposed method, which is depicted in Figure \ref{fig:overallsketch}, begins by computing the set of trajectories that can be reached by a high-fidelity vehicle model under a parameterized controller.
This Forward Reachable Set (FRS), which is computed off-line, is then intersected with obstacles at run-time using convex optimization to exclude controllers that could cause collisions. 
Finally, the remaining safe set of controllers is optimized over to minimize an arbitrary cost function, such as following a nominal trajectory, minimizing total acceleration, etc.


% As depicted in Figure \ref{fig:overallsketch}, the proposed method begins with the offline computation of a \emph{forward reachable set} (FRS) as a continuous space of trajectories that the high-fidelity vehicle model can follow despite tracking error.
% The FRS is then intersected with obstacles at run-time using convex optimization to exclude trajectories that would cause collisions.
% Finally, we optimize over the remaining set of safe trajectories to minimize an arbitrary cost function, such as following a nominal trajectory, minimizing total acceleration, etc.
% To ensure that at least one safe trajectory always exists, we construct minimum time and sensor horizon requirements for the FRS computation by using worst-case time horizons for braking and for planning trajectories.

% First, we are characterizing the entire set of safe trajectories before selecting an optimal one. Second, the FRS is evaluated and obstacle intersection is performed in continuous space and time. Third, bounded disturbance is incorporated to account for error in both the vehicle's sensors and dynamic model. Specifically, we account for error between low and high degree of freedom models used, respectively, in the mid and low level controllers. Lastly, we maintain the ability to avoid all future obstacles or perform an emergency stop if need be. 

%A visualization of this process is shown in Figure \ref{fig:overallsketch}. 
%Currently, there are few methods for autonomous vehicles that perform both reachability analysis and control parameter selection, and those that do are not yet suited for vehicles operating on public roads. 

% The contribution of this paper is a method that, without prior knowledge of obstacle configuration, computes the reachable set of safe trajectories and their corresponding control parameters. Uncertainty in the environment and error in dynamic models used in mid and low level controllers can be accounted for. We demonstrate a relationship between the time horizon of the reachable set, the time required to sense new obstacles, the time required to plan trajectories, and the time required to perform an emergency stop that ensures the vehicle remains safe for all currently detected and future obstacles. 

% As depicted in Figure \ref{fig:overallsketch}, the proposed approached begins by computing the Forward Reachable Set (FRS), which is an outer approximation of all trajectories that the high-fidelity vehicle model can realize and their corresponding control gains.
% This set which is computed off-line and represented as a polynomial can be intersected during run-time using convex optimization to find the set of control gains that intersect with obstacles in the environment.
% The remaining set of safe control gains, an optimal trajectory is chosen. 
% A visualization of this process is shown in Figure \ref{fig:overallsketch}. 



% \begin{figure*}
% \begin{subfigure}[t]{.22\textwidth}
% \centering
% \includegraphics[width=1\textwidth]{controlparam.pdf}
% \caption{Trajectory Parameters}
% \label{fig:gainsketch}
% \end{subfigure}%
% %\hspace*{1cm}
% \begin{subfigure}[t]{.85\textwidth}
% \centering
% \includegraphics[width=0.8\textwidth]{roadsketch.pdf}
% \caption{Trajectories}
% \label{fig:roadsketch}
% \end{subfigure}
% \caption{An illustration of the Forward Reachable Set in the physical world, Figure \ref{fig:roadsketch}, and the gain space, Figure \ref{fig:gainsketch}. ${\mathcal X}_{FRS}$ is the forward reachable set (FRS). The obstacles $X_\text{obs1}$ and $X_\text{obs2}$ are intersected with $X_{FRS}$; which identifies control gains $K_{p1}$ and $K_{p2}$ that can cause collisions, and produces a set of safe control gains, $K_\text{safe}$. Optimal control parameters can be picked picked from $K_{safe}$. In this sketch, control parameters marked by the triangle and square correspond to guaranteed safe trajectories, whereas control parameters marked by the pentagon and circle can cause a collision}
% \label{fig:overallsketch}
% \end{figure*}

% \begin{figure*}
% \begin{subfigure}[t]{.22\textwidth}
% \centering
% \includegraphics[width=1.1\textwidth]{doodle_K.pdf}
% \caption{Trajectory Parameters}
% \label{fig:gainsketch}
% \end{subfigure}%
% %\hspace*{1cm}
% \begin{subfigure}[t]{.85\textwidth}
% \centering
% \includegraphics[width=0.6\textwidth]{doodle_X.pdf}
% \caption{Trajectories}
% \label{fig:roadsketch}
% \end{subfigure}
% \caption{An illustration of the Forward Reachable Set in the physical world, Figure \ref{fig:roadsketch}, and the trajectory parameter space, Figure \ref{fig:gainsketch}. 
% ${\mathcal X}_{FRS}$ is the forward reachable set (FRS). 
% The obstacles $X_\text{obs1}$ and $X_\text{obs2}$ are intersected with $X_{FRS}$; which identifies trajectory parameters $K_{p1}$ and $K_{p2}$ that can cause collisions, and produces a set of safe trajectory parameters, $K_\text{safe}$. Optimal parameters can be picked picked from $K_{safe}$. In this sketch, the parameters labeled 1 produce a guaranteed-safe trajectory, whereas the parameters labeled 2 result in a collision}
% \label{fig:overallsketch}
% \end{figure*}



% \subsection{Outline} \label{outline}
This paper makes the following pair of contributions:
first, a convex optimization based method to synthesize trajectories which can be safely tracked by a high fidelity vehicle model;
and second, a method to select braking, sensing, and planning time horizons to guarantee that a safety preserving trajectory always exists in an environment with static obstacles.
The rest of the paper is structured as follows: 
Section \ref{sec:notation} introduces the notation and assumptions utilized throughout the paper.
Section \ref{sec:problem_formulation} outlines methods to determine the FRS in the presence of bounded uncertainty, to compute the set of safe trajectory parameters in the presence of obstacles, and to compute an optimal trajectory from this set.
Section \ref{sec:implementation} describes the implementation of the methods formulated in Section \ref{sec:problem_formulation}. 
Section \ref{sec:examples} illustrates the performance of our method on an example.

