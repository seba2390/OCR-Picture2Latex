\section{IMPLEMENTATION}
\label{sec:implementation}

This section describes the specific implementations of the FRS Computation, the set intersection, and the trajectory optimization described in Section \ref{sec:problem_formulation}.
For implementation, we require the following assumption:
\begin{assum}
The functions $f_s$ is a polynomial in $\R[t,x_s,k]$, and the error function $g$ is in $\R[t,x_s]$. In addition, the sets $K$, $X_0$, $X_s$, and $X_\text{obs}$ are semialgebraic, defined by polynomials in their respective spaces as in Equations \eqref{eq:ass_2_1} - \eqref{eq:ass_2_3} and \eqref{eq:X_obs}.
\end{assum}

\subsection{FRS Computation}
\label{subsec:frs_imp}

To solve $(D)$, we construct a sequence of convex sums-of-squares programs by relaxing the continuous function in $(D)$ to polynomial functions with degree truncated to $2l$. 
The inequality constraints in $(D)$ then transform into SOS constraints which then transforms $(D)$ into a semidefinite program\cite{parrilo2000structured}
To formulate this problem, let $h_{\tau} = t(T - t)$ and define $Q_{2l}(h_{\tau},h_{X_1},\ldots,h_{X_{n_{X_s}}},h_{K_1},\ldots,h_{K_{n_K}}) \subset \mathbb{R}_{2l}[t,x_s,k]$ to be the set of polynomials $q \in \mathbb{R}_{2l}[t,x_s,k]$ (i.e. of total degree less than $2l$) expressible as:
\begin{equation}
  q = s_0 + s_1 h_{\tau} + \sum_{i=1}^{n_{X_s}} s_{i+2} h_{X_i} + \sum_{i=1}^{n_{K}} s_{i+{n_{X_S}}+2} h_{K_i},
\end{equation}
for some polynomials $\{s_i\}_{i=0}^{n_{X_s}+n_{K}+1} \subset \R_{2k}[t,x,k]$ that are sums of squares of other polynomials, where we have dropped the dependence on $t,x,$ and $k$ in each polynomial for the sake of convenience.
Note that every such polynomial is non-negative on $[0,T] \times X_s \times K$ \cite[Theorem 2.14]{lasserre2009moments}.
Define $Q_{2l} (h_{0_1},\ldots,h_{0_{n_0}},h_{K_1},\ldots,h_{K_{n_K}}) \subset \mathbb{R}_{2l}[x_s,k]$, $Q_{2l} (h_{X_1},\ldots,h_{X_{n_{X_s}}},h_{K_1},\ldots,h_{K_{n_K}}) \subset \mathbb{R}_{2l}[x_s,k]$, $Q_{2l} (h_{\text{obs}_1},\ldots,h_{\text{obs}_{n_\text{obs}}},h_{K_1},\ldots, h_{K_{n_K}}) \subset \mathbb{R}_{2l}[x_s,k]$, and $Q_{2l}(h_{K_1},\ldots, h_{K_{n_K}}) \subset \mathbb{R}_{2l}[k]$ similarly.

Employing this notation, the $l$-th relaxed semidefinite programming representation of $(D)$, denoted $(D_l)$, is defined as:
\begin{flalign} 
% \label{eq:dual lp}
		& & \underset{v,w,q}{\text{inf}} \hspace*{0.25cm} & y_{X_s \times K}^T \textrm{vec}(w) && (D_l) \nonumber \\
		& & \text{s.t.} \hspace*{0.25cm} & -\Lf v - q \in Q_{2l}(h_{\tau},h_{X_1},\ldots,h_{X_{n_{X_s}}},h_{K_1},\ldots,h_{K_{n_K}}) && \nonumber\\
        & & & \Lg v+ q \in Q_{2l}(h_{\tau},h_{X_1},\ldots,h_{X_{n_{X_s}}},h_{K_1},\ldots,h_{K_{n_K}}) && \nonumber \\
        & & & -\Lg v + q\in Q_{2l}(h_{\tau},h_{X_1},\ldots,h_{X_{n_{X_s}}},h_{K_1},\ldots,h_{K_{n_K}}) &&  \nonumber \\
        & & & q \in Q_{2l}(h_{\tau},h_{X_1},\ldots,h_{X_{n_{X_s}}},h_{K_1},\ldots,h_{K_{n_K}}) &&  \nonumber \\
        & & & -v(0,\cdot) \in Q_{2l} (h_{0_1},\ldots,h_{0_{n_0}},h_{K_1},\ldots,h_{K_{n_K}}) &&  \nonumber \\
        & & & w \in Q_{2l} (h_{X_1},\ldots,h_{X_{n_{X_s}}},h_{K_1},\ldots,h_{K_{n_K}})&&  \nonumber \\
        & & & w + v - 1 \in Q_{2l}(h_{\tau},h_{X_1},\ldots,h_{X_{n_{X_s}}},h_{K_1},\ldots,h_{K_{n_K}}) && \nonumber 
\end{flalign}
where the infimum is taken over the vector of polynomials $(v,w,q) \in \R_l[t,x_s,k] \times \R_l[x_s,k] \times \R_l[t,x_s,k]$. 

If $w_l$ denotes the $w$-component of the solution to $(D_l)$, one can prove that $w_l$ converges from above to an indicator function on $\X\frs$ in the $L^1$ norm \cite[Theorem 6]{majumdar2014convex}.
Importantly since each such $w_l$ is feasible with respect to the constraints in $(D)$, one can apply the result of Lemma \ref{lem:feasible_w} to prove that the $1$-superlevel set of $w_l$ is an outer approximation to $\X\frs$.  
% Figure \ref{fig:FRS_slices} illustrates these generated outer approximations for the Dubin's car model for two different $g$ when $l = $ \Ram{need to fill in this with more information}.

% \begin{flalign}
% 		&\text{inf} & &  l^T \textrm{vec}(w) && (D_k) \nonumber\\
%                 &\text{s.t.} && -\Lf v - {\bf 1}^Tp \in  Q_{2k} (h_{\tau},h_{X_1},\ldots,h_{X_{n_X}}), &\nonumber\\
%                & & & p - (\Lg v)^T  \in (Q_{2k} (h_{\tau},h_{X_1},\ldots,h_{X_{n_X}}))^m,&\nonumber\\
%                & & & p + (\Lg v)^T  \in (Q_{2k} (h_{\tau},h_{X_1},\ldots,h_{X_{n_X}}))^m,&\nonumber\\
%                & & & w \in  Q_{2k} (h_{X_1},\ldots,h_{X_{n_X}}),&\nonumber\\
%                & & &  w-v(0,\cdot)- 1  \in  Q_{2k} (h_{X_1},\ldots,h_{X_{n_X}}),& \nonumber\\
% 			   & & & v(T,\cdot)  \in  Q_{2k} (h_{T_1},\ldots,h_{T_{n_T}}),&\nonumber
% \end{flalign}
% where the given data are $f,g,X,X_T$, the infimum is taken over the vector of polynomials $(v,w,p) \in \mathbb{R}_{2k}[t,x] \times \mathbb{R}_{2k}[x] \times (\mathbb{R}_{2k}[t,x])^m$, and $l$ is a vector of moments associated with the Lebesgue measure (i.e. $\int_X w\ d\lambda = l^T\textrm{vec}(w)$ for all $w \in \mathbb{R}_{2k}[x]$). For each $k \in \N$, let $d^*_k$ denote the infimum of $D_k$.

% Let this relaxation be denoted $D_l$, and $P_l$ the corresponding primal.
% Solving $D_l$  returns a polynomial $w_l^*(x_s,k)$ which is an outer-approximation of $\1\frs(x_s,k)$, where
% \begin{align}
% (x_s,k) \in \X\frs \subseteq X_s\times K \ \ \Rightarrow\ \  w_l^*(x_s,k) \geq 1
% \end{align}
% by \cite[Theorem 3.5]{majumdar2014convex}. See Figure \ref{fig:FRS_slices} for an illustration of `slices' of the FRS outer approximation.

\subsection{Set Intersection}\label{subsec:set_int_imp}

In this section, we present a semidefinite program to generate a closed inner-approximation to $K_\text{safe}$ and comment on computing $\tau_\text{plan}$.
Using the notation developed in Section \ref{subsec:frs_imp}, consider the following semidefinite formulation of Equation \eqref{eq:set_int_program}:
\begin{flalign} 
% \label{eq:dual lp}
		& & \underset{h}{\text{sup}} \hspace*{0.25cm} & y_{K}^T \textrm{vec}(h) && \label{eq:SDP_h}  \\
		& & \text{s.t.} \hspace*{0.25cm} & 1 - w - h \in Q_{2l} (h_{\text{obs}_1},\ldots,h_{\text{obs}_{n_\text{obs}}},h_{K_1},\ldots, h_{K_{n_K}}) && \nonumber\\
        & & & 1 - h \in Q_{2l}(h_{K_1},\ldots, h_{K_{n_K}}) && \nonumber 
\end{flalign}
where the given data is any feasible $w \in \R_l[x_s,k]$ to $(D_l)$ and the supremum is taken over polynomials $h \in \R_l[k]$. 
Note again that since the constraints in this optimization problem are sufficient to ensure positivity on the required sets in Equation \eqref{eq:set_int_program}, one can apply the result of Lemma \ref{lem:feasible_h} to prove that the $0$-superlevel set of $h_l$ is an inner approximation to $K_{\text{safe}}$.

% \subsubsection{Formulation as a Semi-Definite Program}

% We implement Equation \eqref{eq:set_int_inner_approx} with an SDP.
% Let an obstacle $X_\text{obs}$ be represented as a semi-algebraic set, with the function $h_\text{obs}(x_s) \in \R_{2l}[x_s]$ such that $X_\text{obs} = \{x \in X_s~|~h_\text{obs}(x_s) \geq 0\}$.
% Define the function $h_w(x_s,k) := -w_l^*(x_s,k) + 1$.
% For any point $p \in X_s$, $h_w(p,k) \geq 0$ for all $k \in K$ such that no trajectory beginning in $X_0$ passes through $p$ at any time $\tau \in \T$.

% Now, we implement Equation \eqref{eq:set_int_inner_approx} by finding $h(k)$ such that $K_{l,\text{safe}} = \{k \in K~|~h(k) \geq 0\}$ using the following program:
% \begin{flalign}\label{eq:set_int_program}
% \sup_{h}\quad&\int_K h(k)d\lambda_K \\
% \text{s.t.}\quad&h_w(x_s,k) - h(k) \geq 0\ \forall~(x_s,k)~\in~X_\text{obs}\times K\label{eq:set_int_eq_constraint}
% \end{flalign}
% Just as the FRS computation is relaxed, this program is relaxed to a finite-dimensional SDP finding $h(k) \in \R_{2l}[k]$.
% The problem as posed above finds $h(k)$ as an inner approximation of $K_\text{safe}$.

% As written, this program only finds the safe trajectories for a single obstacle.
% We expand this to multiple obstacles by adding an inequality constraint in the form of Constraint \eqref{eq:set_int_eq_constraint} for each $X_\text{obs}$.

% \subsubsection{Determining the Planning Time}
% \label{subsec:determining_planning_time}
Solving such an SDP, while possible in real time, is still computationally expensive.
Compared to the trajectory optimization step, which takes milliseconds, the set intersection step can take several tenths of a second per obstacle, and is thus the primary contributor to $\tau_\text{plan}$.
The planning time increases with the number of dimensions of $X_\text{obs}$ and with the number of obstacles.
We find $\tau_\text{plan}$ by precomputing an FRS over an arbitrary, short time horizon $T$, then empirically determining $\tau_\text{plan}$ by performing set intersection with representative obstacles and environments.

\subsection{Trajectory Optimization}\label{subsec:traj_opt_imp}
We implement trajectory optimization by directly passing $h_l$ from solving \eqref{eq:SDP_h} as a nonlinear constraint to an off-the-shelf gradient-descent solver, such as MATLAB's \texttt{fmincon}.



