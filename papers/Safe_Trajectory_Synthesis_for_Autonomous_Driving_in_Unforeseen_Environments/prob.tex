\section{PROBLEM FORMULATION}
\label{sec:problem_formulation}

This section outlines a method for trajectory design and presents a theorem relating the braking, planning, and sensing time horizons to ensure the safety of the trajectory design procedure.
%In this section, we present an outline of the method, and a theorem for safety guarantees with respect to the prior assumptions.
The approach is broken into the following steps:
\begin{enumerate}
\item Precompute a \emph{Forward Reachable Set} (FRS) that captures all possible trajectories and associated parameters, subject to error function $g$, over a time interval $[0,T]$ (Section \ref{subsec:FRS}).
\item During operation, perform \emph{set intersection} of the FRS with obstacles in the vehicle's local environment to identify trajectory parameters, $K_{\text{safe}} \subset K$, which could cause collisions using a convex optimization technique (Section \ref{subsec:set_intersection}).
\item During operation, perform \emph{trajectory optimization} over $K_{\text{safe}}$ of a user-specified cost function to drive the system towards an objective while guaranteeing safety (Section \ref{subsec:traj_opt}).
\end{enumerate}
These steps are described next.

\subsection{Forward Reachable Set}
\label{subsec:FRS}

This subsection describes our approach to compute the FRS, which contains all points that can be reached from $X_0$ under the dynamics of $f$. 
The dynamics of $f$ may be high-dimensional in general as a result, which can make computing the FRS for $f$ impractical.
To overcome this challenge, we compute the FRS of the lower-dimensional model described in Equation \eqref{eq:error_low_model}.
\begin{equation}
\begin{aligned}
\X\frs = \Big\{(x_s,k) \in X_s \times K \ | \; &  \exists~ x_0 \in X_0, \tau \in [0,T], \text{ and } \\ 
& d: [0,T] \to [-1,1] \text{ s.t. } x(\tau) = x_s, \text{ where} \\
& \dot{x}(t) = f_s(t,x(t),k) + g(t,x(t),k)\; d(t)\\
& \text{a.e. } t \in [0,T] \text{ and } x(0) = x_0 \Big\}
\end{aligned}
\end{equation}
Notice that in this definition the error function $g$ along with $d(t) \in [-1,1]$ accounts for the difference between a trajectory, parameterized by $k \in K$, of the lower dimensional model and the higher-dimensional model which follows this trajectory using a control $u_k$.
%An example of the resulting FRS, for a single trajectory parameter $k$, is shown in Figure \ref{fig:FRS_slices} for the unicycle example described in Assumption \ref{ex:model}.

% \begin{figure}
% \centering
% \includegraphics[scale=0.45,trim={1cm 8cm 1cm 8cm},clip]{Xs_FRS.pdf}
% \caption{An example outer approximation of the FRS corresponding to trajectory parameters $k_\text{speed} = 0.8 \Rightarrow v = 0.9$ [m/s], and $k_\text{steering} = \dot{\theta} = 0.8$ [rad/s] over a time horizon $T = 1$ [s].
% The vehicle, plotted as the rectangular area $X_0$, begins at $x = -0.75$ facing right and moves towards $+x$, with initial speed $v = 1$ [m/s] and initial yaw rate $\dot{\theta}$.
% The Dubins path corresponding to the trajectory parameters is shown in green.
% The purple trajectories correspond to the unicycle model tracking the Dubins path given a variety of initial conditions.}
% \label{fig:FRS_slices}
% \end{figure}

\subsubsection{Selecting $T$ to Ensure Safety}

Identifying the time horizon $T$ for the FRS is the critical first step.
If $T$ is too short, then safe behavior cannot be guaranteed; if it is too long, then computing the FRS may be impractical.
Before describing how to compute $\X\frs$, we first describe how to select the time horizon $T$ to ensure safety while planning:
\begin{thm}\label{thm:time_horizon}
Let $T_\text{sense}$, $\tau_\text{plan}$, and $\tau_\text{stop}$ be as in Assumptions \ref{ass:sense},\ref{ass:plan_and_scan}, and \ref{ass:brake}, respectively.
Suppose the vehicle plans a trajectory every $\tau_\text{plan}$ and there exists a $ T \geq 0$ such that:
\begin{align}
\tau_\text{plan} + \tau_\text{stop} &\leq T  \\
T + \tau_\text{plan} &\leq T_\text{sense},
\end{align}
and a safe trajectory over the time horizon $[0,T]$. Then, there exists a safe trajectory for all $t > T$.
% Given a trajectory over the time horizon $\T = [0,T]$, let the vehicle re-plan its trajectory every $\tau_\text{plan}$, i.e. as often as possible.
% Then, the following two conditions are sufficient to ensure that a safe trajectory will exist at any time $t \in (T, 2T]$:
% \begin{align}
% T &\geq \tau_\text{plan} + \tau_\text{stop}  \\
% T_\text{sense} &\geq T + \tau_\text{plan}
% \end{align}
% where $\tau_\text{stop}$ is from Assumption \ref{ass:brake}, $\tau_\text{plan}$ is from Assumption \ref{ass:plan_and_scan}, and $T_\text{sense}$ is defined in Assumption \ref{ass:sense}.
\end{thm}
\begin{proof}
This theorem follows directly from the assumptions, which we can use to construct the worst-case scenario.
At time $t = 0$, let the vehicle be following a trajectory that has been previously identified as safe for the time interval $[0,T]$.
Suppose trajectories are planned as described in Remark \ref{rem:plan}.
%While navigating this trajectory, the vehicle plans its next trajectory, which is applied beginning at $t = \tau_\text{plan}$ (the vehicle not travel the entirety of its first trajectory).
%The initial condition of the vehicle used for planning the upcoming trajectory is based on forward integrating $f$ over the time span $[0,\tau_\text{plan}]$.
Obstacles that could cause collisions fall into one of three cases. 
It is sufficient to show that none of these cases can cause a collision.

\noindent {\bf Case 1:} Any obstacle that can be reached within time $T$.
These obstacles will already be avoided since the trajectory starting at time $t = 0$ is safe for the entire duration $[0,T]$ by definition, even if the vehicle must come to a stop.

\noindent {\bf Case 2:} Any obstacle that can be reached at a time $t \in (T,T_\text{sense}]$.
These obstacles are sensed according to Assumption \ref{ass:sense}.
The worst-case scenario, in this instance, is an obstacle that can be reachable infinitesimally after $t = T$.
Recall that the vehicle is replanning while traveling a duration of $\tau_\text{plan}$, so the plan beginning from time $t = \tau_\text{plan}$ incorporates the positions of all sensed obstacles by adjusting their relative position using the initial condition for the upcoming trajectory.
In addition, this plan incorporates the error that accumulates over the trajectory before $\tau_\text{plan}$ because the FRS uses Assumption \ref{ass:error_func}.
Therefore, at time $t = \tau_\text{plan}$, no obstacle is reachable within a time horizon less than $T - \tau_\text{plan}$, which is greater than or equal to $\tau_\text{stop}$.
Consequently, while traversing the first time interval $[0,\tau_\text{plan}]$, the vehicle determines whether or not it must begin stopping, or if a non-braking trajectory exists, at time $t = \tau_\text{plan}$.
As a result, at $t = \tau_\text{plan}$, the obstacle falls into Case 1.

\noindent {\bf Case 3:} Any obstacle that can be reached at a time $t > T_\text{sense}$.
We assume that the vehicle is detecting obstacles continually, as per Assumption \ref{ass:sense}.
Therefore, by similar logic to Case 2, the obstacle is no closer than $T_\text{sense} - \tau_\text{plan}$ at time $t = \tau_\text{plan}$.
The obstacle thus falls into Case 2 for the trajectory starting at $t = \tau_\text{plan}$.
\end{proof}

% Note that, for a system such as a car on a highway, $\tau_\text{stop}$ can result in an unreasonably long time horizon if the stopping time is the amount of time required to come to a complete stop ($5-7$s). 
% However, the stopping time required to match the speed of surrounding traffic results in a much more reasonable time horizon of 2-3 s.
% Thus, the choice of moving local reference frames can be key to creating reasonable time horizons in this methodology.

\subsubsection{FRS Computation}
\label{subsubsec:FRS_computation}
 
To compute the FRS one can solve the following linear program over the space of functions, which is adapted from \cite[Section 3.3, Program $(D)$]{majumdar2014convex}:
\begin{flalign} 
% \label{eq:dual lp}
		& & \underset{v,w,q}{\text{inf}} \hspace*{0.25cm} & \int_{X_s \times K} w(x_s,k) ~ d\lambda_{X_s \times K} && (D) \nonumber \\
		& & \text{s.t.} \hspace*{0.25cm} & \Lf v(t,x_s,k) + q(t,x_s,k) \leq 0, && \text{on } [0,T] \times X_s \times K \nonumber\\
        & & & \Lg v(t,x_s,k) + q(t,x_s,k) \geq 0, && \text{on } [0,T] \times X_s \times K  \nonumber \\
        & & & -\Lg v(t,x_s,k) + q(t,x_s,k) \geq 0, && \text{on } [0,T] \times X_s \times K  \nonumber \\
        & & & q(t,x_s,k) \geq 0, && \text{on } [0,T] \times X_s \times K  \nonumber \\
        & & & -v(0,x_s,k) \geq 0, && \text{on } X_0 \times K  \nonumber \\
        & & & w(x_s,k) \geq 0,&& \text{on } X_s \times K \nonumber \\
        & & & w(x_s,k) + v(t,x_s,k) - 1 \geq 0, && \text{on } [0,T] \times X_s \times K \nonumber 
\end{flalign}
The given data in this problem are $f,g,X_s,X_0,K$ and $T$ and the infimum is taken over $(v,w,q) \in C^1\left([0,T] \times X_s \right) \times C(X_s) \times C([0,T] \times X_s)$. 

Notice first that the $1$-superlevel set of feasible $w$'s in $(D)$ are outer approximations of $\X\frs$:
\begin{lem}\label{lem:feasible_w}
Let $(v,w,q)$ be feasible functions to $(D)$, then $\X\frs \subset \big\{(x_s,k) \in X_s \times K \mid w(x_s,k) \geq 1 \big\}$.
\end{lem}
\begin{proof}
Suppose $(x_s,k) \in \X\frs$, then $\exists~\tau \in [0,T]$, $d:[0,T] \mapsto [-1,1]$ and $x:[0,T] \mapsto X_s$ such that $\dot{x}(t) = f_s(t,x(t),k) + g(t,x(t),k)d(t)$ for a.e. $t\in [0,T]$ with $x(\tau) = x_s$ and $x(0) \in X_0$.
Observe that:
\begin{align*}
v(\tau,x(\tau),k) &= v(0,x(0),k) + \int\limits_0^{\tau} \left(\Lf v(t,x(t),k) + \Lg v(t,x(t),k) d(t) \right) dt \\
&\geq v(0,x(0),k) + \int_0^{\tau} \left( \Lf v(t,x(t),k) + q(t,x(t),k) \right) dt \\
&\geq v(0,x(0),k),
\end{align*}
where the first equality follows from the Fundamental Theorem of Calculus, the second inequality follows from noticing that $\text{sup}_{d(t) \in [-1,1]} |\Lg v(t,x(t),k) d(t) | = |\Lg v(t,x(t),k) |$ and that $  |\Lg v(t,x(t),k) | \geq q(t,x(t),k) $ due to the constraints in $(D)$, and the final inequality follows from the constraints in $(D)$. 
Notice the desired result follows by noticing that $v(0,x(0),k) \geq 0$ for $x(0) \in X_0$ and applying the last constraint in $(D)$.
\end{proof}

The result of this lemma is that the $1$-superlevel set of any feasible $w$ contains all possible trajectories of the high fidelity vehicle model $f$ beginning from $X_0$.
In fact, one can prove that the solution to this infinite dimensional linear program allows one to compute $\X\frs$:
\begin{thm}{\cite[Theorem 3.5]{majumdar2014convex}}
There is a sequence of feasible solutions to $(D)$ whose $w$-component converges from above to an indicator function on $\X\frs$ in the $L^1$ norm and almost uniformly.
\end{thm}
In Section \ref{subsec:frs_imp} we describe how to generate feasible solutions to $(D)$ using semidefinite programming over polynomial representations of continuous functions.
Once this computation is completed once offline, the result can be used online by performing real-time set intersection as described next.

\subsection{Set Intersection}
\label{subsec:set_intersection}

Any outer-approximation to $\X\frs$ can be intersected with obstacles in the state space to return the trajectory parameters that could result in a collision. 
The complement of the set returned by this intersection are then parameters which can be safely followed by the high fidelity vehicle model.
To describe this process, consider a $w$ generated as a solution to $(D)$. 
One can construct a closed inner approximation to the set of safe gains, denoted $K_\text{safe} \subset K$ by solving the following optimization problem:
\begin{flalign} 
% \label{eq:dual lp}
		& & \underset{h}{\text{sup}} \hspace*{0.25cm} & \int_K h(k)d\lambda_K && \label{eq:set_int_program} \\
		& & \text{s.t.} \hspace*{0.25cm} & 1 - w(x_s,k) - h(k) \geq 0, && \text{on } X_\text{obs} \times K \nonumber \\
        & & & h(k) \leq 1, && \text{on } K \nonumber
\end{flalign}
where $w$ is the given data and the supremum is taken over $h \in C(K)$. 

To appreciate how his optimization problem generates an inner approximation to the set of safe gains, consider a trajectory parameterization, $k$, that generates a trajectory which intersects an obstacle as illustrated in Figure \ref{fig:set_int_ex}.
In this instance, there exists some $x_s$ in $X_\text{obs}$ such that $w(x_s,k) \geq 1$. 
In that instance the first constraint in the previous optimization problem requires that $h(k)$ be less than zero. 
This observation is formalized in the next lemma:
\begin{lem} \label{lem:feasible_h}
Let $h$ be a feasible function to Equation \eqref{eq:set_int_program}. Then $\{k \in K \mid h(k) \geq 0 \} \subset K_\text{safe}$.
\end{lem}

By exploiting the polynomial outer approximation of the $\X\frs$ which is generated in Section \ref{subsec:frs_imp} one can formulate a convex optimization method to compute a closed, inner approximation to $K_\text{safe}$.
This optimization method is described in Section \ref{subsec:set_int_imp}. 
Note that if $T$ is chosen as in Theorem \ref{thm:time_horizon}, then as a result of Assumption \ref{ass:brake}, $K_\text{safe} \neq \emptyset$.

\begin{figure}
\centering
\includegraphics[width=1\columnwidth]{set_intersection_example.pdf}
\caption{An example of set intersection, with $X_\text{obs}$ chosen as three points in $X_s$. On the right is the $(x,y)$ subspace of $X_s$ with each point obstacle shown, and the vehicle plotted in blue. On the left is the trajectory parameter space $K$, with three dashed-line contours containing an outer approximation of the trajectory parameters that would cause a collision with each point shown on the right (the colors match between the points and contours). The set intersection step returns the subset of $K$ shown by the green contour, which outer approximates all trajectory parameters that could result in collisions with any of the three points in $X_\text{obs}$. Therefore, $K_\text{safe}$ is inner-approximated.}
\label{fig:set_int_ex}
\end{figure}

\subsection{Trajectory Optimization}
\label{subsec:traj_opt}
Once $K_\text{safe}$ is determined for an obstacle, one can optimize the original dynamics directly over the set. 
Specifically, if a user specifies a continuously differentiable cost function, $J: K \to \R$ then one can optimize this cost function with a constraint that enforces that $k \in K_{\text{safe}}$.
Since each $k$ corresponds to a $u_k$ one can optimize directly over safe trajectories of the high fidelity model that travel to some waypoint, minimize total acceleration along a trajectory, match a desired system speed, etc.

This constrained nonlinear optimal control problem can be solved in a variety of different ways via collocation, solving a variational equation, sampling, or using convex relaxations \cite{zhao2016control}. 
If this optimization program is unable to conclude within $\tau_{\text{plan}}$, then one can always apply a braking maneuver which always exists as described in Section \ref{subsec:set_intersection}.


% one can solve the following optimization problem:
% \begin{align}
%     \inf_k\quad &C(k) \\
%     \text{st.}\quad &k \in K_\text{safe}\nonumber
% \label{eq:traj_opt}
% \end{align}
% The exact structure of this optimization does not matter, though it has a nonlinear constraint and may not return a globally-optimal solution.
% It can be implemented rapidly in real time either via sampling .
% As long as the constraint is not violated, any trajectory chosen is safe.

%Example cost functions are: driving towards a waypoint, minimizing total acceleration along a trajectory, or matching a desired system speed.

