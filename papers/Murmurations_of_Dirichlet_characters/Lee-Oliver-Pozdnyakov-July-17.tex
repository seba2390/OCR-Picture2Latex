\documentclass{amsart}
\usepackage{amssymb, mathtools, fullpage, color}
\usepackage[colorinlistoftodos]{todonotes}
\usepackage[colorlinks=true, pdfstartview=FitV,linkcolor=blue,citecolor=blue,urlcolor=blue]{hyperref}
%\usepackage{refcheck}

\setcounter{tocdepth}{1}

\theoremstyle{definition}
\newtheorem {theorem}{Theorem}[section]
\newtheorem {lemma}[theorem]{Lemma}
\newtheorem {corollary}[theorem]{Corollary}
\newtheorem {proposition}[theorem]{Proposition}
\newtheorem {conjecture}[theorem]{Conjecture}
\newtheorem{definition}[theorem]{Definition}
\newtheorem{remark}[theorem]{Remark}
\newtheorem{example}[theorem]{Example}

\newenvironment{red}{\relax\color{red}}{\relax}
\newenvironment{blue}{\relax\color{blue}}{\hspace*{.5ex}\relax}
\newenvironment{jaune}{\relax\color{green}}{\hspace*{.5ex}\relax}
\newcommand{\ber}{\begin{red}}
\newcommand{\er}{\end{red}}
\newcommand{\beb}{\begin{blue}}
\newcommand{\eb}{\end{blue}}
\newcommand{\prceil}{\rceil^{\mathfrak p}}
\newcommand{\seteq}{\coloneqq}

\numberwithin{equation}{section}

\begin{document}

\title[Murmurations of Dirichlet Characters]{Murmurations of Dirichlet Characters}

\date{\today}

\author[K.-H. Lee]{Kyu-Hwan Lee$^{\star}$}
\thanks{$^{\star}$This work was partially supported by a grant from the Simons Foundation (\#712100).}
\address{Department of Mathematics, University of Connecticut, Storrs, CT 06269, U.S.A.}
\email{khlee@math.uconn.edu}

\author[T. Oliver]{Thomas Oliver}
\address{Teesside University, Middlesbrough, U.K.}
\email{T.Oliver@tees.ac.uk}

\author[A. Pozdnyakov]{Alexey Pozdnyakov}
\address{Department of Mathematics, University of Connecticut, Storrs, CT 06269, U.S.A.}
\email{alexey.pozdnyakov@uconn.edu}



\begin{abstract}
Inspired by recent observations for elliptic curves, we calculate the murmuration density for Dirichlet characters, normalized by their Gauss sums, over geometric and short intervals.
\end{abstract}

\maketitle

\section{Introduction}\label{s:intro}
Murmurations of elliptic curves were discovered in \cite{HLOP}, in which a striking oscillation in the average value of $a_p(E)$ was observed.
In the original work, the average was taken over elliptic curves $E/\mathbb{Q}$ with conductor in certain intervals $I\subset\mathbb{R}$. 
Motivated by the Modularity Theorem, one might expect a similar phenomenon for Fourier coefficients $a_p(f)$ averaged over newforms $f$ with rational coefficients and level $N$ in a suitable interval $I$. 
Such expectations will be validated in \cite{HLOPS}.   



Two important ideas subsequently emerged based on contributions of J. Ellenberg, A. Sutherland, and J. Bober.
Firstly, on Ellenberg's suggestion, Sutherland pursued the idea that it is interesting to study murmurations attached not only to newforms with rational coefficients, but, moreover, Galois orbits of those with coefficients in arbitrary number fields. 
Secondly, Bober proposed a so-called ``local average'', which eliminates the need for the interval $I$.
These ideas informed a forthcoming work of N.~Zubrilina, in which a local average of $a_p(f)$ for newforms $f$ is calculated \cite{Z23}.
Zubrilina's function, and its variants for related objects, were termed ``murmuration densities'' by P.~Sarnak \cite{S23}. 

This paper is concerned with murmurations of Dirichlet characters $\chi$. More precisely, we calculate the murmuration density for Dirichlet characters $\chi$ normalized by their Gauss sums $G(\chi)$.
We note that  ${\chi}(p)/G({\chi})$ is the Fourier coefficient of $\overline{\chi}$ when expanded in terms of additive characters (see, e.g., \cite[equation~(3.12)]{IK}), and that we are taking averages over sets containing both $\chi$ and $\overline{\chi}$. Therefore, as a natural analogue of the modular form case, we plot the average value of $\chi(p)/G(\chi)$ for odd and even Dirichlet characters with conductor in a prescribed geometric interval, and find the resulting picture exhibits a {murmuration} (see Figures~\ref{fig:idft_real_dyadic_sum} and~\ref{fig:idft_dyadic_avg}). 
 
The main theorem in this paper constitutes an explanation of the murmuration behaviour for Dirichlet characters. 
Let $\mathcal{D}_{+}(N)$ (resp. $\mathcal{D}_-(N)$) denote the set of primitive even (resp. odd) Dirichlet characters mod $N$. 
For $x\in\mathbb{R}_{>0}$, denote by $\lceil x \prceil$ the smallest prime that is bigger than or equal to $x$. 
For
$c\in\mathbb{R}_{>1}$, $\delta \in (0,1)$, and $y\in\mathbb{R}_{>0}$, we define
\begin{align} 
P_\pm (y, X, c)& = \frac{\log X}{X}\sum_{\substack{N \in [X, cX] \\ N \text{ prime}}} \sum_{\chi \in \mathcal{D}_\pm(N)} \frac{\chi(\lceil yX \prceil)}{G(\chi)},  \\ 
\widetilde P_\pm (y, X, \delta)& =  \frac{\log X}{X^{\delta}}\sum_{\substack{N \in [X, X+X^\delta] \\ N \text{ prime}}} \sum_{\chi \in \mathcal{D}_\pm(N)} \frac{\chi(\lceil yX \prceil)}{G(\chi)}. \end{align}
We plot instances of the functions $P_{\pm}(y,X,c)$ and $\widetilde P_{\pm}(y,X,\delta)$ in Figure~\ref{fig:prime_lavg}.
In the case of short intervals $[X,X+X^{\delta}]$, we work conditional on the Riemann Hypothesis, which guarantees that the interval contains primes provided that $\delta>\frac12$.
Our main theorem is stated as follows:
\begin{theorem}\label{thm.main}
Fix $y\in\mathbb{R}_{>0}$.  If $c\in\mathbb{R}_{>1}$, 
then 
\begin{equation}\label{eq.Athm-1}
\lim_{X \to \infty}   P_\pm (y, X, c) = \begin{cases}
\int_1^c \cos \Big(\frac{2\pi y}{x}\Big)dx,  & \text{ if }+,\\
-i\int_1^c \sin\Big(\frac{2\pi y}{x}\Big)dx, & \text{ if }-,
\end{cases} 
\end{equation}
and, assuming the Riemann Hypothesis, if $\delta \in (\frac12,1)$, then
\begin{equation}
\label{eq.LAthm-1}
\lim_{X \to \infty} \widetilde P_\pm (y, X, \delta) = \begin{cases}
\cos(2\pi y), & \text{ if $+$},\\
-i\sin(2\pi y), & \text{ if $-$}.
\end{cases}
\end{equation}
\end{theorem}

\begin{figure}[h]
\centering
\includegraphics[width=0.7\textwidth]{idft_dyadic_prime_lavg.png}
\includegraphics[width=0.7\textwidth]{idft_short_prime_lavg.png}
\caption{\sf (Top) $P_\pm(y, 2^{10}, 2)$ for $y \in [0,10]$ with $+$ in blue and (the imaginary part of) $-$ in red. (Bottom) $\widetilde{P}_\pm(y, 2002, 0.51)$ for $y \in [0,2]$ with $+$ in blue and (the imaginary part of) $-$ in red. The discontinuity around $y=1$ will be explained in Remark~\ref{rem.discont}. }
    \label{fig:prime_lavg}
\end{figure}
The proof of Theorem~\ref{thm.main} uses the prime number theorem, and the relationship between additive and multiplicative characters (which is reviewed in Section~\ref{s:background}).
Upon closer inspection, the proof of Theorem~\ref{thm.main} establishes that equation~\eqref{eq.LAthm-1} may be reformulated to incorporate certain composite conductors.
We specify this reformulation in Section~\ref{s:generalconductors}, and furthermore establish variants of Theorem~\ref{thm.main} for arbitrary conductors (in which case we no longer need to assume the Riemann Hypothesis).

\subsection*{Acknowledgements}
The authors are grateful to Yang-Hui He and Andrew Sutherland for preliminary conversations connected to the themes of this paper, and to Peter Sarnak and Kumar Murty for their helpful comments on an early draft.


\section{Background}\label{s:background}
For $m\in\mathbb{Z}_{>0}$, a Dirichlet character mod $m$ (also known as a Dirichlet character with modulus $m$) is a completely multiplicative function $\chi:\mathbb{Z}\rightarrow\mathbb{C}$ which is periodic with period $m$ and satisfies $\chi(a)=0$ if and only if $\mathrm{gcd}(a,m)>1$. 
The Gauss sum of a Dirichlet character $\chi$ mod $m$ is defined by
\[G(\chi)=\sum_{a=1}^m\chi(a)e^{2\pi ia/m}.\]
We say that a Dirichlet character $\chi$ is even (resp. odd) if $\chi(-1)=1$ (resp. $\chi(-1)=-1$).
The conductor of a Dirichlet character $\chi$ is the minimal positive integer $N$ such that $\chi$ is a Dirichlet character mod $N$.
We say that a Dirichlet character $\chi$ is primitive if its modulus and conductor are equal. 
As in Section~\ref{s:intro}, we denote by $\mathcal{D}_{+}(N)$ (resp. $\mathcal{D}_-(N)$) the set of primitive even (resp. odd) Dirichlet characters mod $N$.
In Examples~\ref{ex.quadratic} and~\ref{ex.orbits}, we plot averages over subsets in $\mathcal{D}_{\pm}(N)$.

\begin{example}\label{ex.quadratic}
We first present the simplest analogue to the murmurations of elliptic curves over $\mathbb{Q}$ discovered in \cite{HLOP}.
A non-trivial Dirichlet character is called {\em quadratic} if its values are real.
Using quadratic reciprocity, one may relate sums of quadratic Dirichlet characters to Chebyshev's bias (cf. \cite{RS}).
We denote by $\mathcal{Q}_\pm(N)$ the subset of $\mathcal{D}_{\pm}(N)$ consisting of quadratic characters. 
Note that, for even (resp. odd) quadratic $\chi$, we have $G(\chi)= \sqrt N$ (resp. $i\sqrt N$).
In Figure~\ref{fig:idft_real_dyadic_sum}, we plot the sum of $\chi(p)/G(\chi)$ over $Q_{\pm}(X):=\bigcup_{N=X}^{2X-1} \mathcal{Q}_\pm(N)$ for $X=2^{17}$. 
\end{example}

\begin{figure}[h]
\centering
\includegraphics[width=0.7\textwidth]{idft_real_dyadic_sum.png}
\caption{\sf Plot of $\sum_{N \in [X, 2X)}\sum_{\chi \in \mathcal{Q}_\pm(N)} \chi(p)/G(\chi)$, for $X = 2^{17}$ and $2 \leq p < 4X$ with $+$ in blue and (the imaginary part of) $-$ in red.}\label{fig:idft_real_dyadic_sum}
\end{figure}

\begin{example}\label{ex.orbits}
In Figure~\ref{fig:idft_dyadic_avg}, we plot a normalized sum of $\chi(p)/G(\chi)$ over $D_{\pm}(X):=\bigcup_{N=X}^{2X-1} \mathcal{D}_\pm(N)$ for $X=2^{10}$. 
Since we have the following asymptotic relation (cf. \cite{J73}):
\begin{equation}\label{eq.chipi}
\frac{1}{\sqrt{\# D_{\pm}(X)}} \sim  \frac{\pi^2}{3\sqrt{3}}\frac{1}{X},
\end{equation}
we simply normalize by the interval length $X$ rather than by the number of characters. 
As manifested in Figure~\ref{fig:idft_real_dyadic_sum}, including non-quadratic (or non-real) characters removes the noise. 
We see a similar effect with modular forms in a forthcoming work \cite{HLOPS}.
\end{example}

\begin{figure}[h]
\includegraphics[width=0.6\textwidth]{idft_dyadic_avg.png}
\caption{\sf Plot of $\frac{1}{X}\sum_{N \in [X, 2X)}\sum_{\chi \in \mathcal{D}_\pm(N)} \chi(p)/G(\chi)$ for $X = 2^{10}$ and primes $p$ such that $2 \leq p \leq 10 X$, with $+$ in blue and (the imaginary part of) $-$ in red. }
\label{fig:idft_dyadic_avg}
\end{figure}

We denote by $\chi_0$ the principal Dirichlet character given by $\chi_0(a)=1$ for $(a,m)=1$. 
Now we will state and prove several important Lemmas which will be used throughout the paper.

\begin{lemma} \label{lem.cosi}
Let $N$ be a positive integer.
If $p$ is a prime such that $(p,N)=1$, then 
\begin{align}
\cos \left(\frac{2 \pi p}N \right) &= \frac{-1}{\phi(N)}   + \frac{1}{\phi(N)} \sum_{\substack{\chi \bmod N \\ \ \chi \neq \chi_0, \, \chi(-1) = 1}} G(\overline{\chi}) \chi(p), \label{eq.cos-1}\\
\sin\left(\frac{2 \pi p}{N} \right) &= \frac{-i}{\phi(N)} \sum_{\substack{\chi \bmod N \\ \chi(-1) = -1}} G(\overline{\chi}) \chi(p).\label{eq.sin-1}
\end{align}
\end{lemma}

\begin{proof}
This follows from \cite[(3.11)]{IK}.
\end{proof}

If $N$ is prime, then every non-trivial Dirichlet character mod $N$ is primitive  and hence
\begin{equation}\label{eq.Dpm}
\mathcal{D}_{+}(N) = \{\chi~\mathrm{mod}~N~:~\chi\neq\chi_0,\, \chi(-1) = 1 \},
\ \ 
\mathcal{D}_{-}(N) = \{\chi~\mathrm{mod}~N~:~ \chi(-1)=-1 \}, \ \ (N~\text{prime}).
\end{equation}


\begin{lemma}\label{lem.sCGcossin}
If $p$ and $N$ are two distinct primes, then
\begin{align}
\sum_{\chi \in \mathcal{D}_+(N)} \frac{\chi(p)}{G(\chi)} &= \left(\frac{N-1}{N} \right)\cos\left(\frac{2\pi p}{N}\right)+\frac{1}{N},\label{eq.mcos}\\
\sum_{\chi \in \mathcal{D}_-(N)} \frac{\chi(p)}{G(\chi)} &= -i\left(\frac{N-1}{N}\right)\sin\left(\frac{2 \pi p}{N}\right).\label{eq.msin}
\end{align}
\end{lemma}
\begin{proof}
For $\chi \in \mathcal{D}_\pm(N)$, recall 
\begin{equation}\label{eq.1/G}
\frac 1{G(\chi)} = \frac {\chi(-1)} N G(\overline{\chi}) ,
\end{equation}
(see, for example, \cite[Exercise 1.1.1]{Bump}). 
Since $\epsilon \seteq \chi(-1)$ is constant on $\chi \in \mathcal{D}_\pm(N)$, summing equation~\eqref{eq.1/G} over $\chi\in\mathcal{D}_{\pm}(N)$ yields
\begin{align}\label{eq.murm-1Gbarchi}
\sum_{\chi \in \mathcal{D}_\pm(N)} \frac{\chi(p)}{G(\chi)} = 
\frac{\epsilon}{N} \sum_{\chi \in \mathcal{D}_\pm(N)} G(\overline{\chi})\chi(p).
\end{align}
Since $N$ is prime and $(p, N)=1$, Lemma \ref{lem.cosi} implies
\begin{align}
\cos\left(\frac{2 \pi p}{N} \right) &=  \frac{-1}{N-1} + \frac{1}{N-1} \sum_{\substack{\chi \bmod N \\ \chi \neq \chi_0,  \chi(-1) = 1}} G(\overline{\chi}) \chi(p), \label{eq.cos}\\
\sin\left(\frac{2 \pi p}{N} \right) &= \frac{-i}{N-1} \sum_{\substack{\chi \bmod N \\ \chi(-1) = -1}} G(\overline{\chi}) \chi(p).\label{eq.sin}
\end{align}
The result now follows from equations~\eqref{eq.Dpm},~\eqref{eq.murm-1Gbarchi},~\eqref{eq.cos} and ~\eqref{eq.sin}.
\end{proof}

\begin{lemma} For $a \in \mathbb{R}_{>0}$ and $b \in (0,1]$, we have
\begin{equation} \label{eq.1/N}
    \lim_{X \to \infty} \frac{\log X}{X^b} \sum_{\substack{N \in [X, X+aX^b]}} \frac{1}{N} = 0.
\end{equation}
\end{lemma}
\begin{proof} Since $a,b, N$ are all positive, we have
\begin{equation*}
    \lim_{X \to \infty} \frac{\log X}{X^b} \sum_{\substack{N \in [X, X+aX^b]}} \frac{1}{N} \leq \lim_{X \to \infty} \frac{\log X}{X^b} \sum_{0 < N \leq (a+1)X} \frac{1}{N} = \lim_{X \to \infty} O\left(\frac{\log X \log ((a+1)X)}{X^b} \right) = 0.
\end{equation*}
\end{proof}

\begin{lemma}
For $y\in\mathbb{R}_{>0}$, we have
\begin{equation} \label{eq.yx} 
\lim_{X \to \infty} \frac{ \lceil yX \prceil - yX}N =0. \end{equation} 
\end{lemma}

\begin{proof}
For any $x\in\mathbb{R}_{>0}$, we have $\lceil x\prceil-x<x^\theta$ for some constant $\theta <1$ (see, e.g., \cite{BHP}). Subsequently, we deduce that
\[\lim_{X \to \infty} \frac{ \lceil yX \prceil - yX}N \le \lim_{X \to \infty} \frac{ \lceil yX \prceil - yX}X=0. \]
\end{proof}

\section{Proof of Theorem~\ref{thm.main}}\label{s:prime}


\begin{proof}[Proof of equation \eqref{eq.Athm-1}]
We will prove the case of $P_+(y,X,c)$, and simply note that the case of $P_- (y, X, c)$ is similar.
For $p\neq N$, equation~\eqref{eq.mcos} implies
\begin{equation}\label{eq.goodstart}
\lim_{X\to \infty} P_+ (y, X, c) = \lim_{X \to \infty} \frac{\log X}{X}\sum_{\substack{N \in [X, cX] \\ N \text{ prime}}} \left [ \left( \frac{N-1}{N}\right)\cos\left(\frac{2 \pi \lceil yX \prceil}{N} \right) + \frac 1 N \right ].  
\end{equation}
With $a=c-1$ and $b=1$ in \eqref{eq.1/N}, we have \begin{equation}\label{eq.1/N-1}   \lim_{X \to \infty} \frac{\log X}{X}\sum_{\substack{N \in [X, cX]}}  \frac 1 N  =0. \end{equation}
Substituting equations \eqref{eq.yx} and \eqref{eq.1/N-1} into equation~\eqref{eq.goodstart} gives:
\begin{equation}\label{eq.goodstart+}
\lim_{X\to \infty} P_+ (y, X, c) =\lim_{X \to \infty} \frac{\log X}{X}\sum_{\substack{N \in [X, cX] \\ N \text{ prime}}}\cos\left(\frac{2 \pi yX}{N} \right).
\end{equation}
We relate the sum on the right hand side of equation~\eqref{eq.goodstart+} to an integral using the following equidistribution argument.
For each $X$, consider the set $S=\{N \in [X, cX] : N \text{ prime}\}$.
If $n=\#S$, then, according to the prime number theorem, we have
\begin{equation}\label{eq.apply-pnt}
n \sim \left ( \frac {cX}{\log(cX)} - \frac X {\log X} \right )\sim \frac {(c-1) X}{\log X}.
\end{equation} 
Consider the sequence $T=(N_i/X)_{i=1}^n$ where $N_i\in S$ for $i\in\{1,\dots,n\}$.
Any subinterval $[\alpha,\beta]\subset(1,c)$ contains the following proportion of elements in $T$:
\[\frac{\pi(\beta X)-\pi(\alpha X)}{n}\sim\frac{\beta X/\log(\beta X)-\alpha X/\log(\alpha X)}{(c-1)X/\log(X)}\sim\frac{\beta-\alpha}{c-1}.\]
In other words, the sequence $T=(N_i/X)_{i=1}^n$ approaches equidistributed on $(1,c)$.
Using equations~\eqref{eq.goodstart+} and~\eqref{eq.apply-pnt}, and applying equidistribution of the sequence $T$ on $(1,c)$, we  conclude using Riemann sums that:
\begin{equation}
\lim_{X\to \infty} P_+ (y, X, c)
    = \lim_{\substack{X \to \infty \\ n \to \infty}} \frac{c-1}{n} \sum_{i=1}^n \cos\left(\frac{2 \pi y X}{N_i} \right) =\int_1^c \cos\left(\frac{2 \pi y}{x}\right)dx.
\end{equation}
\end{proof}
\begin{remark}\label{rem.discont}
The discontinuity around $y=1$ in the bottom image from Figure~\ref{fig:prime_lavg} is explained by the fact that equation~\eqref{eq.goodstart} requires $p\neq N$.
In fact, when $p=N$, the quantity $\chi(\lceil yX \prceil)/G(\chi)$ vanishes. 
This discrepancy does not affect the limit.
\end{remark}
\begin{proof}[Proof of equation \eqref{eq.LAthm-1}]
We assume the Riemann Hypothesis.
We will prove the case of $\widetilde P_+(y,X,c)$, and simply note that the case of $\widetilde P_- (y, X, \delta)$ is similar.
Equation~\eqref{eq.mcos} implies that
\begin{equation}\label{eq.thmlem}
 \widetilde P_+ (y, X, \delta)= \frac{\log X}{X^{\delta}} \sum_{\substack{N \in [X, X+X^{\delta}] \\ N \text{ prime}}} \left [ \left(\frac{N-1}{N} \right)\cos\left(\frac{2\pi \lceil y X \prceil}{N}\right) + \frac 1 N \right ].\\
\end{equation}
With $a=1$ and $b=\delta$ in \eqref{eq.1/N}, we obtain
\begin{equation} \label{eq.1/N-2}
    \lim_{X \to \infty} \frac{\log X}{X^\delta} \sum_{\substack{N \in [X, X+X^\delta]}} \frac{1}{N}  = 0.
\end{equation}
Applying equations \eqref{eq.yx} and \eqref{eq.1/N-2} to equation~\eqref{eq.thmlem}, we deduce
\begin{equation}\label{eq.important}
\lim_{X \to \infty} \widetilde P_+ (y, X, \delta) = \lim_{X \to \infty} \frac{\log X}{X^{\delta}} \sum_{\substack{N \in [X, X+X^{\delta}] \\ N \text{ prime}}}\cos\left(\frac{2\pi  y X }{N}\right).
\end{equation}
Since $N \in [X, X + X^\delta]$, we have
\begin{equation}\label{eq.squeeze}
1 = \lim_{X \to \infty} \frac{X}{X+X^\delta} \leq  \lim_{X \to \infty} \frac{X}{N} \leq \lim_{X \to \infty} \frac{X}{X} = 1.
\end{equation}
Using the prime number theorem and noting that $\delta > 0.5$, we find
\begin{equation}\label{eq.PNT}
\sum_{\substack{N \in [X, X+X^{\delta}] \\ N \text{ prime}}} 1 \sim \left (\frac{X + X^{\delta}}{\log\left(X + X^{\delta}\right)} - \frac{X}{\log X} \right ) \sim \frac{X^{\delta}}{\log X}.
\end{equation}
Applying equations \eqref{eq.squeeze} and \eqref{eq.PNT} to equation~\eqref{eq.important}, we deduce 
\begin{equation}\label{eq.almost}
\lim_{X \to \infty} \frac{\log X}{X^{\delta}} \sum_{\substack{N \in [X, X+X^{\delta}] \\ N \text{ prime}}}\cos\left(\frac{2\pi  y X }{N}\right) = \cos(2 \pi y).
\end{equation}
\end{proof}


\section{Generalisations of Theorem~\ref{thm.main}}\label{s:generalconductors}

\subsection{Preliminaries}
The main aim of this section is to extend Theorem~\ref{thm.main} to include composite conductors. 
More precisely, for $y\in\mathbb{R}_{>0}, \delta\in(0,1)$ and $c \in \mathbb{R}_{>1}$, we will analyse functions connected to the following:
\begin{align}
M_{\pm}(y,X, c)=&\frac{1}{X}\sum_{\substack{N \in[X, cX] \\ N \not\equiv 2 \bmod 4}} \sum_{\chi \in \mathcal{D}_\pm(N)} \frac{\chi(\lceil yX \prceil)}{G(\chi)},\label{eq.Mpm}\\
\widetilde{M}_{\pm}(y,X,\delta)=&\frac{1}{X^\delta}\sum_{\substack{N \in[X, X+X^\delta] \\ N \not\equiv 2 \bmod 4}} \sum_{\chi \in \mathcal{D}_\pm(N)} \frac{\chi(\lceil yX \prceil)}{G(\chi)}. 
\end{align}
Note that by \cite[equation~(3.7)]{IK}, the set $\mathcal{D}_\pm(N)$ is empty if and only if $N \equiv 2 \bmod 4$. 
For an integer $N>1$ and a prime number $p$ coprime to $N$, we may rearrange Lemma~\ref{lem.cosi} to get
\begin{align}
\sum_{\substack{\chi \bmod N \\ \ \chi \neq \chi_0, \, \chi(-1) = 1}} G(\overline{\chi}) \chi(p)&=1+\phi(N)\cos \left(\frac{2 \pi p}N \right), \label{eq.cos-2}\\
\sum_{\substack{\chi \bmod N \\ \chi(-1) = -1}} G(\overline{\chi}) \chi(p)&=i\phi(N)\sin\left(\frac{2 \pi p}{N} \right). \label{eq.sin-2}
\end{align}
We introduce the sets
\begin{equation}
\mathcal{I}_\pm(N) = \{\text{$\chi$ mod $N$ : $\chi$ imprimitive, $\chi \neq \chi_0$, $\chi(-1) = \pm 1$} \},
\end{equation}
so that equations~\eqref{eq.cos-2} and~\eqref{eq.sin-2} may be rewritten as follows:
\begin{equation}\label{eq.imprim-cos}
\sum_{\chi \in \mathcal D_+(N)} G(\overline{\chi}) \chi(p)
=1+ \phi(N)\cos \left (
\frac{2 \pi p} N \right )- \sum_{\chi \in \mathcal{I}_+(N)} G(\overline{\chi}) \chi(p),
\end{equation}
\begin{equation}\label{eq.imprim-sin}
\sum_{\chi \in \mathcal D_-(N)} G(\overline{\chi}) \chi(p)
=i\phi(N)\sin \left (
\frac{2 \pi p} N \right )- \sum_{\chi \in \mathcal{I}_-(N)} G(\overline{\chi}) \chi(p).
\end{equation}
Applying equation~\eqref{eq.1/G} to equations~\eqref{eq.imprim-cos} and~\eqref{eq.imprim-sin}, we deduce
\begin{equation}\label{eq.imprim-cos-2}
\sum_{\chi \in \mathcal D_+(N)} \frac{\chi(p)}{G(\chi)}
=\frac{1}{N}+ \frac{\phi(N)}{N}\cos \left (
\frac{2 \pi p} N \right )- \frac{1}{N}\sum_{\chi \in \mathcal{I}_+(N)} G(\overline{\chi}) \chi(p),
\end{equation}
\begin{equation}\label{eq.imprim-sin-2}
\sum_{\chi \in \mathcal D_-(N)} \frac{\chi(p)}{G(\chi)}
=\frac{-i\phi(N)}{N}\sin \left (
\frac{2 \pi p} N \right )+ \frac{1}{N}\sum_{\chi \in \mathcal{I}_-(N)} G(\overline{\chi}) \chi(p).
\end{equation}
In order to recreate the proof of Theorem~\ref{thm.main} for composite conductors, equations~\eqref{eq.imprim-cos-2} and~\eqref{eq.imprim-sin-2} suggest that we need to analyse sums of imprimitive characters. 
The following Lemma will be useful for that purpose:
\begin{lemma}
If an imprimitive character $\chi$ mod $N$ is induced by the primitive character $\chi_1$ mod $N_1$, then,  we have
\begin{equation}\label{eq.imprimGauss}
G(\overline{\chi})=\mu \left({\frac {N}{N_{1}}}\right)\overline{\chi_{1}\left({\frac {N}{N_{1}}}\right)}G\left(\overline{\chi _{1}}\right), 
\end{equation}
where $\mu(n)$ is the M\"obius function.
\end{lemma}

\begin{proof}
\cite[Lemma 3.1]{IK}.
\end{proof}
Inspired by equations~\eqref{eq.imprim-cos-2} and~\eqref{eq.imprim-sin-2}, we introduce the following functions:
\begin{equation}\label{eq.Ilavanish_dyadic}
E_{\pm}(y,X,c)=\frac{1}{X}\sum_{\substack{N \in[X, cX] \\ N \not\equiv 2 \bmod 4 }} \frac{1}{N} \sum_{\mathcal{I}_\pm(N)} G(\overline{\chi}) \chi(\lceil yX \prceil),
\end{equation}
\begin{equation}\label{eq.Ilavanish_short}
\widetilde{E}_{\pm}(y,X,\delta)=\frac{1}{X^\delta}\sum_{\substack{N \in[X, X+X^\delta] \\ N \not\equiv 2\bmod 4}} \frac{1}{N} \sum_{\mathcal{I}_\pm(N)} G(\overline{\chi}) \chi(\lceil yX \prceil).
\end{equation}
One sees that it is natural to investigate:
\begin{equation}\label{eq.Murm}
T_{\pm}(y,X,c)=M_{\pm}(y,X,c)\pm E_{\pm}(y,X,c), 
\end{equation} 
\begin{equation}\label{eq.Murm2}
\widetilde{T}_{\pm}(y,X,\delta)=\widetilde{M}_{\pm}(y,X,\delta)\pm\widetilde{E}_{\pm}(y,X,\delta).
\end{equation}
\begin{remark}
Figure \ref{fig:idft_dyadic_lavg} (resp. \ref{fig:idft_short_lavg}) suggest that $|T_\pm(y,X,c)|$ and $|M_\pm(y,X,c)|$ (resp.  $|\widetilde{T}_\pm(y,X,c)|$ and $|\widetilde{M}_\pm(y,X,\delta)|$) are significantly larger than $|E_\pm(y, X, \delta)|$ (resp. $|\widetilde{E}_\pm(y,X,\delta)|$).
Since there is a canonical bijection between Dirichlet characters mod $N$ and primitive Dirichlet characters with conductor dividing $N$, $E_\pm(y, X, \delta)$ (resp. $\widetilde{E}_\pm(y,X,\delta)$) reduces to a sum over primitive characters with conductor dividing $N$. 
This reduction introduces a M\"obius factor by \eqref{eq.imprimGauss}, so we expect that this term is smaller due to additional cancellation. 
\end{remark}
Another complexity arising from equations~\eqref{eq.imprim-cos-2} and~\eqref{eq.imprim-sin-2} is the need to understand the local average of $\phi(N)/N$. 
The following Lemma will be useful for that purpose.

\begin{lemma}\label{lem.phi_order}
If $a\in\mathbb{R}_{> 0}$ and $b \in (0,1]$, then
\begin{equation}\label{eq.5/pi2}
\lim_{X \to \infty} \frac{1}{aX^b} \sum_{\substack{N \in [X, X + aX^b] \\ N \not\equiv 2\bmod 4}} \frac{\phi(N)}{N}  
=\frac{5}{\pi^2}.
\end{equation}
\end{lemma}
\begin{proof}
It is known that
\begin{equation}\label{eq.order_phi_error}
\sum_{\substack{0<N \leq X\\N\in\mathbb{Z}}} \frac{\phi(N)}{N} = \frac{6}{\pi^2}X+O((\log X)^{2/3}(\log \log X)^{4/3}), 
\end{equation}
from which it follows that
\begin{equation}\label{eq.order_phi}
\lim_{X \to \infty} \frac{1}{X}\sum_{\substack{0<N \leq X\\N\in\mathbb{Z}}} \frac{\phi(N)}{N} = \frac{6}{\pi^2} 
\end{equation}
(cf. \cite[equation~(1.74)]{IK}).
Similarly, according to \cite{N75}, we have that
\begin{equation}
    \lim_{X \to \infty} \frac{1}{X} \sum_{\substack{0< N \leq X \\ N \in \mathbb{Z}}}  \frac{\phi(2N+1)}{2N+1} = \frac{8}{\pi^2}.\label{eq.order_phi_odd}
\end{equation}
Using equation~\eqref{eq.order_phi_odd} and the identity $\phi(4N+2) = \phi(2N+1)$, we compute:
\begin{equation}
\lim_{X \to \infty} \frac{1}{X} \sum_{\substack{0<N \leq X \\ N \equiv 2\bmod 4}}\frac{\phi(N)}{N} 
= \lim_{X \to \infty} \frac{1}{4X} \sum_{\substack{0<N \leq X\\N\in\mathbb{Z}}} \frac{\phi(4N+2)}{4N+2} = \frac{1}{8} \lim_{X \to \infty} \frac{1}{X} \sum_{\substack{0<N \leq X\\N\in\mathbb{Z}}} \frac{\phi(2N+1)}{2N+1} = \frac{1}{\pi^2}. \label{eq.order_phi_2(4)}
\end{equation}
Subtracting equation~\eqref{eq.order_phi_2(4)} from equation~\eqref{eq.order_phi}, and noting the error term in equation~\eqref{eq.order_phi_error},  we conclude: 
\begin{equation}\label{eq.N<X5pi2}
\lim_{X \to \infty} \frac{1}{X} \sum_{\substack{0<N \leq X \\ N \not\equiv 2 \bmod 4}} \frac{\phi(N)}{N} 
=\frac{5}{\pi^2}.
\end{equation}
Equation~\eqref{eq.N<X5pi2} implies that 
\begin{equation}
    \sum_{\substack{0<N \leq X \\ N \not\equiv 2 \bmod 4}} \frac{\phi(N)}{N} \sim \sum_{\substack{0<N \leq X \\ N \in \mathbb{Z}}} \frac{5}{\pi^2},
\end{equation}
from which equation~\eqref{eq.5/pi2} follows.
\end{proof}

\subsection{Geometric Intervals}
In this subsection, we will prove the following theorem, which is visualised in Figure~\ref{fig:idft_dyadic_lavg}.
\begin{theorem}\label{thm.dyadic}
If $c\in\mathbb{R}_{>1}$ and $y\in\mathbb{R}_{>0}$, then
\begin{equation}\label{eq.int}
\lim_{X \to \infty} T_{\pm}(y,X,c) =
\begin{cases}
\frac{5}{\pi^2}\int_1^c \cos\left(\frac{2\pi y}{x} \right)dx, & \text{ if }+, \\
-i\frac{5}{\pi^2}\int_1^c \sin\left(\frac{2\pi y}{x} \right)dx, & \text{ if }-.
\end{cases}
\end{equation}
\end{theorem}

\begin{figure}[h]
\centering
\includegraphics[width=0.7\textwidth]{idft_dyadic_lavg_fit.png}
\caption{\sf   Plot of $M_\pm(y, 1024, 2)$ for $0 \leq y \leq 10$ with $+$ in blue and (the imaginary part of) $-$ in red.  We also show $\frac{5}{\pi^2}\int_1^2 \cos \left(\frac{2\pi y}{x}\right)dx$ in green and $-\frac{5}{\pi^2}\int_1^2 \sin \left(\frac{2\pi y}{x}\right)dx$ in orange. }
\label{fig:idft_dyadic_lavg}
\end{figure}

\begin{proof}
We will prove the case of $T_+(y,X,c)$, and simply note that $T_-(y,X,c)$ is similar. 
Applying equations~\eqref{eq.1/N},~\eqref{eq.Mpm}, and~\eqref{eq.Ilavanish_dyadic} to equation~\eqref{eq.imprim-cos-2}, we deduce
\begin{equation}\label{eq.tobreak}
\begin{split}
\lim_{X \to \infty} M_{+}(y,X,\delta) &= \lim_{X \to \infty} \frac{1}{X} \sum_{\substack{N \in [X, cX] \\ N \not\equiv 2\bmod 4}} \frac{\phi(N)}{N}\cos\left(\frac{2\pi yX}{N}\right)-\lim_{X\rightarrow\infty}E_+(y,X,\delta)\\
&=\lim_{X \to \infty} \frac{1}{X} \sum_{i=1}^{n} \sum_{\substack{N \in I_i \\ N \not\equiv 2 \bmod 4}} \frac{\phi(N)}{N} \cos\left(\frac{2\pi y X}{N}\right)-\lim_{X\rightarrow\infty}E_+(y,X,\delta),
\end{split}
\end{equation}
where, for each $X$, we put $n=\lceil \sqrt{X} \rceil$ and, for $i\in\left\{1,\dots,n\right\}$, we write
\begin{equation}
I_i = \left[X + \frac{i-1}{n}(c-1)X, X + \frac{i}{n}(c-1)X \right).
\end{equation}
Fix $\gamma \in (0, 1]$ and, for each $X$, choose $i = \lceil \gamma n\rceil\in\{1,\dots,n\}$. 
We have
\begin{equation}\label{eq.gamma}
\lim_{X \to \infty} \frac{i-1}{n} = \lim_{X \to \infty }\frac{i}{n} = \gamma.
\end{equation}
For $N \in I_i$, equation~\eqref{eq.gamma} implies that
\begin{equation}\label{eq.another-squeeze}
\frac{1}{1+\gamma(c-1)} = \lim_{X \to \infty} \frac{X}{X+i(c-1)X/n} \leq \lim_{X \to \infty} \frac{X}{N} \leq \lim_{X \to \infty}\frac{X}{X+(i-1)(c-1)X/n} = \frac{1}{1+\gamma(c-1)}.
\end{equation}
Combining equations~\eqref{eq.Murm}, ~\eqref{eq.tobreak}, and ~\eqref{eq.another-squeeze}, we deduce:
\begin{equation}\label{eq.prelemma}
\lim_{X \to \infty} T_{+}(y,X,\delta)  =  \lim_{X \to \infty}\frac{c-1}{\sqrt{X}} \sum_{i=1}^n  \cos\left(\frac{2\pi y}{1+\gamma(c-1)}\right) \frac{1}{(c-1)\sqrt{X}} \sum_{\substack{N \in I_i \\ N \not\equiv 2 \bmod 4}} \frac{\phi(N)}{N}
\end{equation}
Using ~\eqref{eq.5/pi2}, and~\eqref{eq.prelemma} we are led to
\begin{equation}\label{eq.RSfor5/pi2}
\lim_{X \to \infty} T_{+}(y,X,\delta)  = \lim_{n \to \infty} \frac{c-1}{n} \sum_{i=1}^n   \frac{5}{\pi^2}\cos\left(\frac{2\pi y}{1+i(c-1)/n}\right).
\end{equation}
Equation~\eqref{eq.int} follows upon recognising equation~\eqref{eq.RSfor5/pi2} as a Riemann sum.
\end{proof}



\subsection{Short intervals}
\subsubsection{General conductors}
We prove the following theorem, which is visualised in Figure~\ref{fig:idft_short_lavg}.
\begin{theorem}
If $\delta\in(0,1)$ and $y\in\mathbb{R}_{>0}$, then
\begin{equation}\label{eq.moreconductors}
\lim_{X\rightarrow\infty}\widetilde{T}_{\pm}(y,X,\delta)= \begin{cases}
\frac{5}{\pi^2}\cos(2\pi y), & \text{ if $+$},\\
-i\frac{5}{\pi^2}\sin(2\pi y), & \text{ if $-$}.
\end{cases}
\end{equation}
\end{theorem}
\begin{figure}[h]
\centering
\includegraphics[width=0.7\textwidth]{idft_short_lavg_fit.png}
\caption{\sf Plot of $\widetilde{M}_\pm(y, 2002, 0.51)$ for $0 \leq y \leq 2$ with $+$ in blue and (imaginary part of) $-$ in red.  We also show $\frac{5}{\pi^2}\cos(2\pi y)$ in green and $\frac{5}{\pi^2}\sin(2\pi y)$ in orange.}
\label{fig:idft_short_lavg}
\end{figure}

\begin{proof}
We will prove the case of $\widetilde{T}_+(y,X,\delta)$ and simply note that $\widetilde{T}_-(y,X,\delta)$ is similar. Mimicking the proof of equation~\eqref{eq.LAthm-1} leads to 
\begin{equation}\label{eq.limTtild}
    \lim_{X \to \infty} \widetilde{T}_+(y,X,\delta) = \lim_{X \to \infty} \frac{\cos(2\pi y)}{X^\delta} \sum_{\substack{N \in [X, X+X^\delta] \\ N \not\equiv 2 \text{ mod } 4}} \frac{\phi(N)}{N}. 
\end{equation}
In light of equation~\eqref{eq.limTtild}, the result follows from Lemma~\ref{lem.phi_order}.
\end{proof}


\subsubsection{Special conductors}\label{s:special-short}
We obtain an extension of equation~\eqref{eq.LAthm-1} by considering a set of special conductors specified as follows.  
Let $S$ denote the set of positive integers that are not congruent to $2$ mod $4$ and are either prime or squarefull\footnote{A positive integer is {\em squarefull} if all its prime factors exponents are at least $2$.}. 
By equation~\eqref{eq.imprimGauss},
this is precisely the set $S$ of integers such that, if $N \in S$, then
\begin{equation} \label{eq.sum_imprim}
    \sum_{\mathcal{I}_\pm(N)} G(\overline{\chi}) \chi(p) = 0, \ \ (N\in S).
\end{equation}
Using equations~\eqref{eq.1/N} and~\eqref{eq.sum_imprim}, we deduce that, for $N \in S$, equation~\eqref{eq.imprim-cos-2} reduces to 
\begin{equation} \label{eq.special_cos}
    \sum_{\chi \in \mathcal D_\pm(N)} \frac{\chi(p)}{G(\chi)}  = \frac{\phi(N)}{N}\cos\left(\frac{2\pi p}{N} \right), \ \ (N\in S).
\end{equation}
Now define 
\begin{equation}\label{eq.tau}
    f(X) = \sum_{\substack{N \leq X \\ N \in S}} \frac{\phi(N)}{N},
\end{equation}
and consider
\begin{equation}
    \widetilde{M}_{\pm}^S(y, X, \delta) = \frac{1}{f(X+X^\delta) - f(X)} \sum_{\substack{N \in [X, X+X^\delta] \\ N \in S}} \sum_{\chi \in \mathcal{D}_\pm(N)} \frac{\chi(\lceil yX\rceil^p)}{G(\chi)}. 
\end{equation}
This leads to the following Corollary.
\begin{corollary}
Under the Riemann Hypothesis,
if $\delta\in(\frac12,1)$ and $y\in\mathbb{R}_{>0}$, then
\begin{equation}\label{eq.moreconductors-1}
\lim_{X\rightarrow\infty}\widetilde{M}^S_{\pm}(y,X,\delta)= \begin{cases}
\cos(2\pi y), & \text{ if $+$},\\
-i\sin(2\pi y), & \text{ if $-$}.
\end{cases}
\end{equation}
\end{corollary}
\begin{proof}
We prove the case of $\widetilde{M}_+^S(y,X,\delta)$ and simply note that $\widetilde{M}_-^S(y,X,\delta)$ is similar. Using equation~\eqref{eq.special_cos} and mimicking the proof of equation~\eqref{eq.LAthm-1} yields 
\begin{equation}\label{eq.togeneralize}
\lim_{X \to \infty} \widetilde{M}^S_{+}(y,X,\delta)= \lim_{X \to \infty} \frac{\cos(2\pi y)}{f(X+X^\delta) - f(X)}\sum_{\substack{N \in [X,X+X^\delta] \\ N \in S}} \frac{\phi(N)}{N},
\end{equation}
from which the result follows by equation~\eqref{eq.tau}. 
\end{proof}

\begin{thebibliography}{CERP}%{widest-label}
\bibitem[BHP]{BHP}
R. C. Baker, G. Harman and J. Pintz,
     {\em The difference between consecutive primes {II}},
   {Proc. London Math. Soc. (3)} {\bf 83} {(2001)} {no. 3}, {532--562}.

\bibitem[B98]{Bump}
D.~Bump, 
\newblock{\em Automorphic forms and representations}, 
\newblock Cambridge University Press (1998).

\bibitem[HLOP]{HLOP}
Y.-H.~He, K.-H.~Lee, T.~Oliver, and A. Pozdnyakov,
\newblock{\em Murmurations of elliptic curves}, 
\newblock arXiv:2204.10140.

\bibitem[HLOPS]{HLOPS}
Y.-H.~He, K.-H.~Lee, T.~Oliver, A. Pozdnyakov, and A. Sutherland,
\newblock{\em Murmurations of $L$-functions}, 
\newblock in preparation. 

\bibitem[IK21]{IK}
H.~Iwaniec and E.~Kowalski, 
\newblock{\em Analytic number theory}, 
\newblock American Mathematical Soc. (2021).


\bibitem[J73]{J73}
H. Jager,
\newblock{\em On the number of Dirichlet characters with modulus not exceeding x}, Indagationes Mathematicae (Proceedings) {\bf 76} (1973), no. 5, 452--455.

\bibitem[N75]{N75}
J.E Nymann,
\newblock{\em On the probability that k positive integers are relatively prime II},
Journal of Number Theory,
Volume 7, Issue 4,
(1975), 406--412.

\bibitem[RS94]{RS}
M.~Rubinstein and P.~Sarnak,
\newblock{\em Chebyshev's bias}, Experimental Mathematics {\bf 3} (1994), no. 3, 173-197.

\bibitem[S76]{S76}
L. Schoenfeld, \newblock{\em Sharper Bounds for the Chebyshev Functions $\theta(x)$ and $\psi(x)$. II.} Mathematics of Computation, 30(134), (1976) 337–360. 

\bibitem[S23]{S23}
P.~Sarnak, \newblock{\em Root numbers and murmurations}, ICERM Workshop, July 7, 2023.

\bibitem[Z23]{Z23} N.~Zubrilina, \newblock{\em Root number correlation bias of Fourier coefficients of modular forms}, IAS/Princeton Number Theory Seminar, April 20, 2023.

\end{thebibliography}

\end{document}
