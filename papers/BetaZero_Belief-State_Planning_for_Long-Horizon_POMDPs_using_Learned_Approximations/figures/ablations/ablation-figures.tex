\begin{figure*}[ht!]
    \centering
    \begin{minipage}{0.69\textwidth}
        \resizebox{\textwidth}{!}{
            % Recommended preamble:
% \usetikzlibrary{arrows.meta}
% \usetikzlibrary{backgrounds}
% \usepgfplotslibrary{patchplots}
% \usepgfplotslibrary{fillbetween}
% \pgfplotsset{%
%     layers/standard/.define layer set={%
%         background,axis background,axis grid,axis ticks,axis lines,axis tick labels,pre main,main,axis descriptions,axis foreground%
%     }{
%         grid style={/pgfplots/on layer=axis grid},%
%         tick style={/pgfplots/on layer=axis ticks},%
%         axis line style={/pgfplots/on layer=axis lines},%
%         label style={/pgfplots/on layer=axis descriptions},%
%         legend style={/pgfplots/on layer=axis descriptions},%
%         title style={/pgfplots/on layer=axis descriptions},%
%         colorbar style={/pgfplots/on layer=axis descriptions},%
%         ticklabel style={/pgfplots/on layer=axis tick labels},%
%         axis background@ style={/pgfplots/on layer=axis background},%
%         3d box foreground style={/pgfplots/on layer=axis foreground},%
%     },
% }

% light blue = rgb,1:red,0.3922;green,0.7098;blue,0.8039
% purple = rgb,1:red,0.5059;green,0.4471;blue,0.698

\begin{tikzpicture}[/tikz/background rectangle/.style={fill={rgb,1:red,1.0;green,1.0;blue,1.0}, fill opacity={1.0}, draw opacity={1.0}}, show background rectangle]
\begin{axis}[reverse legend, point meta max={nan}, point meta min={nan}, legend cell align={left}, legend columns={1}, title={Ablation: \(Q\)-weighted policy vector}, title style={at={{(0.5,1)}}, anchor={south}, font={{\fontsize{15 pt}{19.5 pt}\selectfont}}, color={rgb,1:red,0.0;green,0.0;blue,0.0}, draw opacity={1.0}, rotate={0.0}, align={center}}, legend style={color={rgb,1:red,0.0;green,0.0;blue,0.0}, draw opacity={1.0}, line width={1}, solid, fill={rgb,1:red,1.0;green,1.0;blue,1.0}, fill opacity={1.0}, text opacity={1.0}, font={{\fontsize{14 pt}{18.2 pt}\selectfont}}, text={rgb,1:red,0.0;green,0.0;blue,0.0}, cells={anchor={west}}, at={(0.98, 0.02)}, anchor={south east}}, axis background/.style={fill={rgb,1:red,1.0;green,1.0;blue,1.0}, opacity={1.0}}, anchor={north west}, xshift={20.0mm}, yshift={-5.0mm}, width={95.81mm}, height={75.81mm}, scaled x ticks={false}, xlabel={number of episodes trained on}, x tick style={color={rgb,1:red,0.0;green,0.0;blue,0.0}, opacity={1.0}}, x tick label style={color={rgb,1:red,0.0;green,0.0;blue,0.0}, opacity={1.0}, rotate={0}}, xlabel style={at={(ticklabel cs:0.5)}, anchor=near ticklabel, at={{(ticklabel cs:0.5)}}, anchor={near ticklabel}, font={{\fontsize{14 pt}{18.2 pt}\selectfont}}, color={rgb,1:red,0.0;green,0.0;blue,0.0}, draw opacity={1.0}, rotate={0.0}}, xmajorgrids={true}, xmin={500.0}, xmax={10000.0}, xticklabels={{2K,4K,6K,8K,10K}}, xtick={{2000,4000,6000,8000,10000}}, xtick align={inside}, xticklabel style={font={{\fontsize{14 pt}{18.2 pt}\selectfont}}, color={rgb,1:red,0.0;green,0.0;blue,0.0}, draw opacity={1.0}, rotate={0.0}}, x grid style={color={rgb,1:red,0.0;green,0.0;blue,0.0}, draw opacity={0.1}, line width={0.5}, solid}, xticklabel pos={left}, x axis line style={color={rgb,1:red,0.0;green,0.0;blue,0.0}, draw opacity={1.0}, line width={1}, solid}, scaled y ticks={false}, ylabel={mean return}, y tick style={color={rgb,1:red,0.0;green,0.0;blue,0.0}, opacity={1.0}}, y tick label style={color={rgb,1:red,0.0;green,0.0;blue,0.0}, opacity={1.0}, rotate={0}}, ylabel style={at={(ticklabel cs:0.5)}, anchor=near ticklabel, at={{(ticklabel cs:0.5)}}, anchor={near ticklabel}, font={{\fontsize{14 pt}{18.2 pt}\selectfont}}, color={rgb,1:red,0.0;green,0.0;blue,0.0}, draw opacity={1.0}, rotate={0.0}}, ymajorgrids={true}, ymin={-4.888330008394597}, ymax={14.410660984188631}, yticklabels={{$-3$,$0$,$3$,$6$,$9$,$12$}}, ytick={{-3.0,0.0,3.0,6.0,9.0,12.0}}, ytick align={inside}, yticklabel style={font={{\fontsize{14 pt}{18.2 pt}\selectfont}}, color={rgb,1:red,0.0;green,0.0;blue,0.0}, draw opacity={1.0}, rotate={0.0}}, y grid style={color={rgb,1:red,0.0;green,0.0;blue,0.0}, draw opacity={0.1}, line width={0.5}, solid}, yticklabel pos={left}, y axis line style={color={rgb,1:red,0.0;green,0.0;blue,0.0}, draw opacity={1.0}, line width={1}, solid}, colorbar={false}]
    \addplot+[line width={0}, draw opacity={0}, fill={rgb,1:red,0.5059;green,0.4471;blue,0.698}, fill opacity={0.1}, mark={none}, forget plot]
        coordinates {
            (500.0,-2.8750161650037094)
            (1000.0,-1.9514317024687209)
            (1500.0,-0.1707595787629801)
            (2000.0,2.0746458405791093)
            (2500.0,3.9991107380475643)
            (3000.0,6.178488940149618)
            (3500.0,8.579412225091334)
            (4000.0,10.130681797823957)
            (4500.0,11.10758311226294)
            (5000.0,11.75673725960729)
            (5500.0,12.345554428365457)
            (6000.0,12.285545392944314)
            (6500.0,12.684306664477461)
            (7000.0,12.731775975155488)
            (7500.0,13.179567286604001)
            (8000.0,13.256060559986636)
            (8500.0,13.228492421281157)
            (9000.0,13.618543529207098)
            (9500.0,13.688228547170578)
            (10000.0,13.350034382188108)
            (10000.0,12.854469215944205)
            (9500.0,12.965796110152525)
            (9000.0,13.042507568219701)
            (8500.0,12.701157465429311)
            (8000.0,12.752660799407703)
            (7500.0,12.504679059172794)
            (7000.0,11.949909424101172)
            (6500.0,11.686788060355035)
            (6000.0,11.286227659560513)
            (5500.0,11.252211738187931)
            (5000.0,10.32686789661991)
            (4500.0,9.522564953855362)
            (4000.0,8.021974479811172)
            (3500.0,5.7578483669704426)
            (3000.0,2.1267340526622753)
            (2500.0,-1.172253364466501)
            (2000.0,-2.1027323846479624)
            (1500.0,-3.1222471397385165)
            (1000.0,-3.502950916207894)
            (500.0,-4.0830423852533)
            (500.0,-2.8750161650037094)
        }
        ;
    \addplot+[line width={0}, draw opacity={0}, fill={rgb,1:red,0.5059;green,0.4471;blue,0.698}, fill opacity={0.1}, mark={none}, forget plot]
        coordinates {
            (500.0,-2.8750161650037094)
            (1000.0,-1.9514317024687209)
            (1500.0,-0.1707595787629801)
            (2000.0,2.0746458405791093)
            (2500.0,3.9991107380475643)
            (3000.0,6.178488940149618)
            (3500.0,8.579412225091334)
            (4000.0,10.130681797823957)
            (4500.0,11.10758311226294)
            (5000.0,11.75673725960729)
            (5500.0,12.345554428365457)
            (6000.0,12.285545392944314)
            (6500.0,12.684306664477461)
            (7000.0,12.731775975155488)
            (7500.0,13.179567286604001)
            (8000.0,13.256060559986636)
            (8500.0,13.228492421281157)
            (9000.0,13.618543529207098)
            (9500.0,13.688228547170578)
            (10000.0,13.350034382188108)
            (10000.0,13.845599548432011)
            (9500.0,14.410660984188631)
            (9000.0,14.194579490194496)
            (8500.0,13.755827377133002)
            (8000.0,13.75946032056557)
            (7500.0,13.854455514035209)
            (7000.0,13.513642526209804)
            (6500.0,13.681825268599887)
            (6000.0,13.284863126328116)
            (5500.0,13.438897118542982)
            (5000.0,13.186606622594669)
            (4500.0,12.692601270670519)
            (4000.0,12.239389115836742)
            (3500.0,11.400976083212225)
            (3000.0,10.230243827636961)
            (2500.0,9.17047484056163)
            (2000.0,6.2520240658061805)
            (1500.0,2.7807279822125563)
            (1000.0,-0.399912488729548)
            (500.0,-1.6669899447541185)
            (500.0,-2.8750161650037094)
        }
        ;
    \addplot[color={rgb,1:red,0.5059;green,0.4471;blue,0.698}, name path={c46511af-30e6-41df-972e-784fc06c583e}, legend image code/.code={{
    \draw[fill={rgb,1:red,0.5059;green,0.4471;blue,0.698}, fill opacity={0.1}] (0cm,-0.1cm) rectangle (0.6cm,0.1cm);
    }}, draw opacity={1.0}, line width={2}, solid]
        table[row sep={\\}]
        {
            \\
            500.0  -2.8750161650037094  \\
            1000.0  -1.9514317024687209  \\
            1500.0  -0.1707595787629801  \\
            2000.0  2.0746458405791093  \\
            2500.0  3.9991107380475643  \\
            3000.0  6.178488940149618  \\
            3500.0  8.579412225091334  \\
            4000.0  10.130681797823957  \\
            4500.0  11.10758311226294  \\
            5000.0  11.75673725960729  \\
            5500.0  12.345554428365457  \\
            6000.0  12.285545392944314  \\
            6500.0  12.684306664477461  \\
            7000.0  12.731775975155488  \\
            7500.0  13.179567286604001  \\
            8000.0  13.256060559986636  \\
            8500.0  13.228492421281157  \\
            9000.0  13.618543529207098  \\
            9500.0  13.688228547170578  \\
            10000.0  13.350034382188108  \\
        }
        ;
    \addlegendentry {visit counts}
    \addplot[color={rgb,1:red,0.5059;green,0.4471;blue,0.698}, name path={88312f58-e2f9-4e62-aa93-b9b5d8d943e2}, draw opacity={0.1}, line width={1}, solid, forget plot]
        table[row sep={\\}]
        {
            \\
            500.0  -1.6669899447541185  \\
            1000.0  -0.399912488729548  \\
            1500.0  2.7807279822125563  \\
            2000.0  6.2520240658061805  \\
            2500.0  9.17047484056163  \\
            3000.0  10.230243827636961  \\
            3500.0  11.400976083212225  \\
            4000.0  12.239389115836742  \\
            4500.0  12.692601270670519  \\
            5000.0  13.186606622594669  \\
            5500.0  13.438897118542982  \\
            6000.0  13.284863126328116  \\
            6500.0  13.681825268599887  \\
            7000.0  13.513642526209804  \\
            7500.0  13.854455514035209  \\
            8000.0  13.75946032056557  \\
            8500.0  13.755827377133002  \\
            9000.0  14.194579490194496  \\
            9500.0  14.410660984188631  \\
            10000.0  13.845599548432011  \\
        }
        ;
    \addplot[color={rgb,1:red,0.5059;green,0.4471;blue,0.698}, name path={b803045e-6545-4d0a-be8a-c4ab91d35439}, draw opacity={0.1}, line width={1}, solid, forget plot]
        table[row sep={\\}]
        {
            \\
            500.0  -4.0830423852533  \\
            1000.0  -3.502950916207894  \\
            1500.0  -3.1222471397385165  \\
            2000.0  -2.1027323846479624  \\
            2500.0  -1.172253364466501  \\
            3000.0  2.1267340526622753  \\
            3500.0  5.7578483669704426  \\
            4000.0  8.021974479811172  \\
            4500.0  9.522564953855362  \\
            5000.0  10.32686789661991  \\
            5500.0  11.252211738187931  \\
            6000.0  11.286227659560513  \\
            6500.0  11.686788060355035  \\
            7000.0  11.949909424101172  \\
            7500.0  12.504679059172794  \\
            8000.0  12.752660799407703  \\
            8500.0  12.701157465429311  \\
            9000.0  13.042507568219701  \\
            9500.0  12.965796110152525  \\
            10000.0  12.854469215944205  \\
        }
        ;
    \addplot+[line width={0}, draw opacity={0}, fill={rgb,1:red,0.3922;green,0.7098;blue,0.8039}, fill opacity={0.1}, mark={none}, forget plot]
        coordinates {
            (500.0,-3.595634784155036)
            (1000.0,-1.5690363628859707)
            (1500.0,1.539853038997891)
            (2000.0,5.138468425876941)
            (2500.0,7.626915313672898)
            (3000.0,9.294994513422637)
            (3500.0,10.629430615597196)
            (4000.0,11.923693728612413)
            (4500.0,12.117358273624689)
            (5000.0,12.26597181631745)
            (5500.0,12.677238619106532)
            (6000.0,12.490463461123936)
            (6500.0,12.747979205193815)
            (7000.0,12.778757232363585)
            (7500.0,13.005752557486883)
            (8000.0,13.205425785361086)
            (8500.0,13.032797883353645)
            (9000.0,13.24379954905617)
            (9500.0,13.195939416424931)
            (10000.0,13.270756529744911)
            (10000.0,12.786664419866577)
            (9500.0,12.613809700776645)
            (9000.0,12.626183179822009)
            (8500.0,12.407737931052509)
            (8000.0,12.603205571324)
            (7500.0,12.364673603139929)
            (7000.0,12.118192566369332)
            (6500.0,12.04292455769452)
            (6000.0,11.856434004416956)
            (5500.0,11.9305909062234)
            (5000.0,11.732981355753513)
            (4500.0,11.588627285188878)
            (4000.0,11.193601844038177)
            (3500.0,9.599547878128385)
            (3000.0,8.212782443197813)
            (2500.0,6.002285553609889)
            (2000.0,3.0210996425149172)
            (1500.0,-0.9729774192576883)
            (1000.0,-3.2033755917293436)
            (500.0,-4.530963131371441)
            (500.0,-3.595634784155036)
        }
        ;
    \addplot+[line width={0}, draw opacity={0}, fill={rgb,1:red,0.3922;green,0.7098;blue,0.8039}, fill opacity={0.1}, mark={none}, forget plot]
        coordinates {
            (500.0,-3.595634784155036)
            (1000.0,-1.5690363628859707)
            (1500.0,1.539853038997891)
            (2000.0,5.138468425876941)
            (2500.0,7.626915313672898)
            (3000.0,9.294994513422637)
            (3500.0,10.629430615597196)
            (4000.0,11.923693728612413)
            (4500.0,12.117358273624689)
            (5000.0,12.26597181631745)
            (5500.0,12.677238619106532)
            (6000.0,12.490463461123936)
            (6500.0,12.747979205193815)
            (7000.0,12.778757232363585)
            (7500.0,13.005752557486883)
            (8000.0,13.205425785361086)
            (8500.0,13.032797883353645)
            (9000.0,13.24379954905617)
            (9500.0,13.195939416424931)
            (10000.0,13.270756529744911)
            (10000.0,13.754848639623246)
            (9500.0,13.778069132073217)
            (9000.0,13.861415918290332)
            (8500.0,13.65785783565478)
            (8000.0,13.807645999398172)
            (7500.0,13.646831511833836)
            (7000.0,13.439321898357838)
            (6500.0,13.45303385269311)
            (6000.0,13.124492917830915)
            (5500.0,13.423886331989664)
            (5000.0,12.798962276881387)
            (4500.0,12.6460892620605)
            (4000.0,12.653785613186649)
            (3500.0,11.659313353066006)
            (3000.0,10.37720658364746)
            (2500.0,9.251545073735906)
            (2000.0,7.255837209238965)
            (1500.0,4.05268349725347)
            (1000.0,0.0653028659574022)
            (500.0,-2.6603064369386313)
            (500.0,-3.595634784155036)
        }
        ;
    \addplot[color={rgb,1:red,0.3922;green,0.7098;blue,0.8039}, name path={4e40685c-801e-46b7-aaa0-57add9ba9f5b}, legend image code/.code={{
    \draw[fill={rgb,1:red,0.3922;green,0.7098;blue,0.8039}, fill opacity={0.1}] (0cm,-0.1cm) rectangle (0.6cm,0.1cm);
    }}, draw opacity={1.0}, line width={2}, solid]
        table[row sep={\\}]
        {
            \\
            500.0  -3.595634784155036  \\
            1000.0  -1.5690363628859707  \\
            1500.0  1.539853038997891  \\
            2000.0  5.138468425876941  \\
            2500.0  7.626915313672898  \\
            3000.0  9.294994513422637  \\
            3500.0  10.629430615597196  \\
            4000.0  11.923693728612413  \\
            4500.0  12.117358273624689  \\
            5000.0  12.26597181631745  \\
            5500.0  12.677238619106532  \\
            6000.0  12.490463461123936  \\
            6500.0  12.747979205193815  \\
            7000.0  12.778757232363585  \\
            7500.0  13.005752557486883  \\
            8000.0  13.205425785361086  \\
            8500.0  13.032797883353645  \\
            9000.0  13.24379954905617  \\
            9500.0  13.195939416424931  \\
            10000.0  13.270756529744911  \\
        }
        ;
    \addlegendentry {\(Q\)-values}
    \addplot[color={rgb,1:red,0.3922;green,0.7098;blue,0.8039}, name path={f539d6cc-dcff-4323-8c7a-166aba6d9158}, draw opacity={0.1}, line width={1}, solid, forget plot]
        table[row sep={\\}]
        {
            \\
            500.0  -2.6603064369386313  \\
            1000.0  0.0653028659574022  \\
            1500.0  4.05268349725347  \\
            2000.0  7.255837209238965  \\
            2500.0  9.251545073735906  \\
            3000.0  10.37720658364746  \\
            3500.0  11.659313353066006  \\
            4000.0  12.653785613186649  \\
            4500.0  12.6460892620605  \\
            5000.0  12.798962276881387  \\
            5500.0  13.423886331989664  \\
            6000.0  13.124492917830915  \\
            6500.0  13.45303385269311  \\
            7000.0  13.439321898357838  \\
            7500.0  13.646831511833836  \\
            8000.0  13.807645999398172  \\
            8500.0  13.65785783565478  \\
            9000.0  13.861415918290332  \\
            9500.0  13.778069132073217  \\
            10000.0  13.754848639623246  \\
        }
        ;
    \addplot[color={rgb,1:red,0.3922;green,0.7098;blue,0.8039}, name path={ff1cd117-a88d-4286-9fc4-74476cf6d95c}, draw opacity={0.1}, line width={1}, solid, forget plot]
        table[row sep={\\}]
        {
            \\
            500.0  -4.530963131371441  \\
            1000.0  -3.2033755917293436  \\
            1500.0  -0.9729774192576883  \\
            2000.0  3.0210996425149172  \\
            2500.0  6.002285553609889  \\
            3000.0  8.212782443197813  \\
            3500.0  9.599547878128385  \\
            4000.0  11.193601844038177  \\
            4500.0  11.588627285188878  \\
            5000.0  11.732981355753513  \\
            5500.0  11.9305909062234  \\
            6000.0  11.856434004416956  \\
            6500.0  12.04292455769452  \\
            7000.0  12.118192566369332  \\
            7500.0  12.364673603139929  \\
            8000.0  12.603205571324  \\
            8500.0  12.407737931052509  \\
            9000.0  12.626183179822009  \\
            9500.0  12.613809700776645  \\
            10000.0  12.786664419866577  \\
        }
        ;
    \addplot+[line width={0}, draw opacity={0}, fill={rgb,1:red,0.7686;green,0.3059;blue,0.3216}, fill opacity={0.1}, mark={none}, forget plot]
        coordinates {
            (500.0,-3.5980881189297964)
            (1000.0,-0.2867294374845144)
            (1500.0,4.0832705887904535)
            (2000.0,7.410102911212197)
            (2500.0,9.029396699951246)
            (3000.0,10.42345625113408)
            (3500.0,11.625058981617599)
            (4000.0,12.268835263620472)
            (4500.0,12.47923344738277)
            (5000.0,12.805039543275493)
            (5500.0,12.820735095354177)
            (6000.0,12.794351285575349)
            (6500.0,12.957570587226547)
            (7000.0,12.813190213527836)
            (7500.0,13.165255691857741)
            (8000.0,13.292094505372539)
            (8500.0,12.986205853108508)
            (9000.0,13.549593588991645)
            (9500.0,13.517144211392985)
            (10000.0,13.62327876114251)
            (10000.0,13.04656382472852)
            (9500.0,12.8561833994349)
            (9000.0,12.959368055057919)
            (8500.0,12.193301023758542)
            (8000.0,12.734708322306043)
            (7500.0,12.545280681279596)
            (7000.0,12.175102450527469)
            (6500.0,12.353701157897433)
            (6000.0,12.181545960333956)
            (5500.0,12.312349499138001)
            (5000.0,12.21612978383136)
            (4500.0,11.749918359079977)
            (4000.0,11.480073353544572)
            (3500.0,10.738504845701035)
            (3000.0,9.496209115496134)
            (2500.0,7.790372624756252)
            (2000.0,5.720173518573083)
            (1500.0,2.670664850705334)
            (1000.0,-1.2977335748549892)
            (500.0,-4.888330008394597)
            (500.0,-3.5980881189297964)
        }
        ;
    \addplot+[line width={0}, draw opacity={0}, fill={rgb,1:red,0.7686;green,0.3059;blue,0.3216}, fill opacity={0.1}, mark={none}, forget plot]
        coordinates {
            (500.0,-3.5980881189297964)
            (1000.0,-0.2867294374845144)
            (1500.0,4.0832705887904535)
            (2000.0,7.410102911212197)
            (2500.0,9.029396699951246)
            (3000.0,10.42345625113408)
            (3500.0,11.625058981617599)
            (4000.0,12.268835263620472)
            (4500.0,12.47923344738277)
            (5000.0,12.805039543275493)
            (5500.0,12.820735095354177)
            (6000.0,12.794351285575349)
            (6500.0,12.957570587226547)
            (7000.0,12.813190213527836)
            (7500.0,13.165255691857741)
            (8000.0,13.292094505372539)
            (8500.0,12.986205853108508)
            (9000.0,13.549593588991645)
            (9500.0,13.517144211392985)
            (10000.0,13.62327876114251)
            (10000.0,14.1999936975565)
            (9500.0,14.178105023351069)
            (9000.0,14.13981912292537)
            (8500.0,13.779110682458475)
            (8000.0,13.849480688439035)
            (7500.0,13.785230702435886)
            (7000.0,13.451277976528203)
            (6500.0,13.561440016555661)
            (6000.0,13.407156610816742)
            (5500.0,13.329120691570353)
            (5000.0,13.393949302719626)
            (4500.0,13.208548535685562)
            (4000.0,13.057597173696372)
            (3500.0,12.511613117534162)
            (3000.0,11.350703386772027)
            (2500.0,10.26842077514624)
            (2000.0,9.100032303851313)
            (1500.0,5.495876326875573)
            (1000.0,0.7242746998859604)
            (500.0,-2.3078462294649955)
            (500.0,-3.5980881189297964)
        }
        ;
    \addplot[color={rgb,1:red,0.7686;green,0.3059;blue,0.3216}, name path={a3c7ad93-4cb7-4502-9229-97fe7b81d0b3}, legend image code/.code={{
    \draw[fill={rgb,1:red,0.7686;green,0.3059;blue,0.3216}, fill opacity={0.1}] (0cm,-0.1cm) rectangle (0.6cm,0.1cm);
    }}, draw opacity={1.0}, line width={2}, solid]
        table[row sep={\\}]
        {
            \\
            500.0  -3.5980881189297964  \\
            1000.0  -0.2867294374845144  \\
            1500.0  4.0832705887904535  \\
            2000.0  7.410102911212197  \\
            2500.0  9.029396699951246  \\
            3000.0  10.42345625113408  \\
            3500.0  11.625058981617599  \\
            4000.0  12.268835263620472  \\
            4500.0  12.47923344738277  \\
            5000.0  12.805039543275493  \\
            5500.0  12.820735095354177  \\
            6000.0  12.794351285575349  \\
            6500.0  12.957570587226547  \\
            7000.0  12.813190213527836  \\
            7500.0  13.165255691857741  \\
            8000.0  13.292094505372539  \\
            8500.0  12.986205853108508  \\
            9000.0  13.549593588991645  \\
            9500.0  13.517144211392985  \\
            10000.0  13.62327876114251  \\
        }
        ;
    \addlegendentry {\(Q\)-weighted counts}
    \addplot[color={rgb,1:red,0.7686;green,0.3059;blue,0.3216}, name path={baeaa9ef-94dc-4abe-8260-4ed5fe8fc066}, draw opacity={0.1}, line width={1}, solid, forget plot]
        table[row sep={\\}]
        {
            \\
            500.0  -2.3078462294649955  \\
            1000.0  0.7242746998859604  \\
            1500.0  5.495876326875573  \\
            2000.0  9.100032303851313  \\
            2500.0  10.26842077514624  \\
            3000.0  11.350703386772027  \\
            3500.0  12.511613117534162  \\
            4000.0  13.057597173696372  \\
            4500.0  13.208548535685562  \\
            5000.0  13.393949302719626  \\
            5500.0  13.329120691570353  \\
            6000.0  13.407156610816742  \\
            6500.0  13.561440016555661  \\
            7000.0  13.451277976528203  \\
            7500.0  13.785230702435886  \\
            8000.0  13.849480688439035  \\
            8500.0  13.779110682458475  \\
            9000.0  14.13981912292537  \\
            9500.0  14.178105023351069  \\
            10000.0  14.1999936975565  \\
        }
        ;
    \addplot[color={rgb,1:red,0.7686;green,0.3059;blue,0.3216}, name path={3ff42172-ba7d-4f3a-bc11-a0a997fb5286}, draw opacity={0.1}, line width={1}, solid, forget plot]
        table[row sep={\\}]
        {
            \\
            500.0  -4.888330008394597  \\
            1000.0  -1.2977335748549892  \\
            1500.0  2.670664850705334  \\
            2000.0  5.720173518573083  \\
            2500.0  7.790372624756252  \\
            3000.0  9.496209115496134  \\
            3500.0  10.738504845701035  \\
            4000.0  11.480073353544572  \\
            4500.0  11.749918359079977  \\
            5000.0  12.21612978383136  \\
            5500.0  12.312349499138001  \\
            6000.0  12.181545960333956  \\
            6500.0  12.353701157897433  \\
            7000.0  12.175102450527469  \\
            7500.0  12.545280681279596  \\
            8000.0  12.734708322306043  \\
            8500.0  12.193301023758542  \\
            9000.0  12.959368055057919  \\
            9500.0  12.8561833994349  \\
            10000.0  13.04656382472852  \\
        }
        ;
\end{axis}
\end{tikzpicture}

            % Recommended preamble:
% \usetikzlibrary{arrows.meta}
% \usetikzlibrary{backgrounds}
% \usepgfplotslibrary{patchplots}
% \usepgfplotslibrary{fillbetween}
% \pgfplotsset{%
%     layers/standard/.define layer set={%
%         background,axis background,axis grid,axis ticks,axis lines,axis tick labels,pre main,main,axis descriptions,axis foreground%
%     }{
%         grid style={/pgfplots/on layer=axis grid},%
%         tick style={/pgfplots/on layer=axis ticks},%
%         axis line style={/pgfplots/on layer=axis lines},%
%         label style={/pgfplots/on layer=axis descriptions},%
%         legend style={/pgfplots/on layer=axis descriptions},%
%         title style={/pgfplots/on layer=axis descriptions},%
%         colorbar style={/pgfplots/on layer=axis descriptions},%
%         ticklabel style={/pgfplots/on layer=axis tick labels},%
%         axis background@ style={/pgfplots/on layer=axis background},%
%         3d box foreground style={/pgfplots/on layer=axis foreground},%
%     },
% }

\begin{tikzpicture}[/tikz/background rectangle/.style={fill={rgb,1:red,1.0;green,1.0;blue,1.0}, fill opacity={1.0}, draw opacity={1.0}}, show background rectangle]
\begin{axis}[reverse legend, point meta max={nan}, point meta min={nan}, legend cell align={left}, legend columns={1}, title={Ablation: Belief representation}, title style={at={{(0.5,1)}}, anchor={south}, font={{\fontsize{15 pt}{19.5 pt}\selectfont}}, color={rgb,1:red,0.0;green,0.0;blue,0.0}, draw opacity={1.0}, rotate={0.0}, align={center}}, legend style={color={rgb,1:red,0.0;green,0.0;blue,0.0}, draw opacity={1.0}, line width={1}, solid, fill={rgb,1:red,1.0;green,1.0;blue,1.0}, fill opacity={1.0}, text opacity={1.0}, font={{\fontsize{14 pt}{18.2 pt}\selectfont}}, text={rgb,1:red,0.0;green,0.0;blue,0.0}, cells={anchor={west}}, at={(0.98, 0.48)}, anchor={south east}}, axis background/.style={fill={rgb,1:red,1.0;green,1.0;blue,1.0}, opacity={1.0}}, anchor={north west}, xshift={20.0mm}, yshift={-5.0mm}, width={95.81mm}, height={75.81mm}, scaled x ticks={false}, xlabel={number of episodes trained on}, x tick style={color={rgb,1:red,0.0;green,0.0;blue,0.0}, opacity={1.0}}, x tick label style={color={rgb,1:red,0.0;green,0.0;blue,0.0}, opacity={1.0}, rotate={0}}, xlabel style={at={(ticklabel cs:0.5)}, anchor=near ticklabel, at={{(ticklabel cs:0.5)}}, anchor={near ticklabel}, font={{\fontsize{14 pt}{18.2 pt}\selectfont}}, color={rgb,1:red,0.0;green,0.0;blue,0.0}, draw opacity={1.0}, rotate={0.0}}, xmajorgrids={true}, xmin={500.0}, xmax={15000.0}, xticklabels={{5K,10K,15K}}, xtick={{5000,10000,15000}}, xtick align={inside}, xticklabel style={font={{\fontsize{14 pt}{18.2 pt}\selectfont}}, color={rgb,1:red,0.0;green,0.0;blue,0.0}, draw opacity={1.0}, rotate={0.0}}, x grid style={color={rgb,1:red,0.0;green,0.0;blue,0.0}, draw opacity={0.1}, line width={0.5}, solid}, xticklabel pos={left}, x axis line style={color={rgb,1:red,0.0;green,0.0;blue,0.0}, draw opacity={1.0}, line width={1}, solid}, scaled y ticks={false}, y tick style={color={rgb,1:red,0.0;green,0.0;blue,0.0}, opacity={1.0}}, y tick label style={color={rgb,1:red,0.0;green,0.0;blue,0.0}, opacity={1.0}, rotate={0}}, ylabel style={at={(ticklabel cs:0.5)}, anchor=near ticklabel, at={{(ticklabel cs:0.5)}}, anchor={near ticklabel}, font={{\fontsize{14 pt}{18.2 pt}\selectfont}}, color={rgb,1:red,0.0;green,0.0;blue,0.0}, draw opacity={1.0}, rotate={0.0}}, ymajorgrids={true}, ymin={-4.888330008394597}, ymax={14.459424735433695}, yticklabels={{$-3$,$0$,$3$,$6$,$9$,$12$}}, ytick={{-3.0,0.0,3.0,6.0,9.0,12.0}}, ytick align={inside}, yticklabel style={font={{\fontsize{14 pt}{18.2 pt}\selectfont}}, color={rgb,1:red,0.0;green,0.0;blue,0.0}, draw opacity={1.0}, rotate={0.0}}, y grid style={color={rgb,1:red,0.0;green,0.0;blue,0.0}, draw opacity={0.1}, line width={0.5}, solid}, yticklabel pos={left}, y axis line style={color={rgb,1:red,0.0;green,0.0;blue,0.0}, draw opacity={1.0}, line width={1}, solid}, colorbar={false}]
    \addplot+[line width={0}, draw opacity={0}, fill={rgb,1:red,0.3333;green,0.6588;blue,0.4078}, fill opacity={0.1}, mark={none}, forget plot]
        coordinates {
            (500.0,-2.2114251961396696)
            (1000.0,-1.1076250095410896)
            (1500.0,-0.4120969429611726)
            (2000.0,0.09497872151760489)
            (2500.0,0.24549422778389016)
            (3000.0,0.1355584553423665)
            (3500.0,0.22167469067586992)
            (4000.0,0.18999581896318413)
            (4500.0,-0.0056921923827150805)
            (5000.0,-0.16617038048485203)
            (5500.0,-0.2263719812255825)
            (6000.0,-0.09764043675513495)
            (6500.0,0.026787651813827168)
            (7000.0,0.18227002978949747)
            (7500.0,0.20420215912737932)
            (8000.0,-0.037617017046228524)
            (8500.0,0.23311702689343436)
            (9000.0,0.31874254832043514)
            (9500.0,0.6595995149068118)
            (10000.0,0.47451320250974316)
            (10500.0,0.5351453183169261)
            (11000.0,0.6415337596554785)
            (11500.0,0.2723108193439695)
            (12000.0,-0.017732958689274353)
            (12500.0,-0.2763163408135647)
            (13000.0,-0.23190524448813882)
            (13500.0,-0.2669203226928833)
            (14000.0,-0.11345925761572995)
            (14500.0,-0.07011233412727796)
            (15000.0,-0.025886574774766788)
            (15000.0,-0.23472277368723954)
            (14500.0,-0.3291943595942886)
            (14000.0,-0.374347627813182)
            (13500.0,-0.5949671098200716)
            (13000.0,-0.6467994313272916)
            (12500.0,-0.7880736190800096)
            (12000.0,-0.5626697764700733)
            (11500.0,-0.5336271732707472)
            (11000.0,-0.6579072289728475)
            (10500.0,-0.7349502455954928)
            (10000.0,-0.6191729053160695)
            (9500.0,-0.8105692399245449)
            (9000.0,-0.5681363604013872)
            (8500.0,-0.44238581722026404)
            (8000.0,-0.5690847800868751)
            (7500.0,-0.6008056192422218)
            (7000.0,-0.9677439394422804)
            (6500.0,-0.6229098471021165)
            (6000.0,-0.6114283564595431)
            (5500.0,-0.8262308175341639)
            (5000.0,-0.8342918888442397)
            (4500.0,-0.8013443446086594)
            (4000.0,-0.8984744492685213)
            (3500.0,-1.1324313151736451)
            (3000.0,-1.0205919166939381)
            (2500.0,-1.3781721107510962)
            (2000.0,-1.3806106640121434)
            (1500.0,-1.3749309551172568)
            (1000.0,-2.197049579720545)
            (500.0,-3.682998980465442)
            (500.0,-2.2114251961396696)
        }
        ;
    \addplot+[line width={0}, draw opacity={0}, fill={rgb,1:red,0.3333;green,0.6588;blue,0.4078}, fill opacity={0.1}, mark={none}, forget plot]
        coordinates {
            (500.0,-2.2114251961396696)
            (1000.0,-1.1076250095410896)
            (1500.0,-0.4120969429611726)
            (2000.0,0.09497872151760489)
            (2500.0,0.24549422778389016)
            (3000.0,0.1355584553423665)
            (3500.0,0.22167469067586992)
            (4000.0,0.18999581896318413)
            (4500.0,-0.0056921923827150805)
            (5000.0,-0.16617038048485203)
            (5500.0,-0.2263719812255825)
            (6000.0,-0.09764043675513495)
            (6500.0,0.026787651813827168)
            (7000.0,0.18227002978949747)
            (7500.0,0.20420215912737932)
            (8000.0,-0.037617017046228524)
            (8500.0,0.23311702689343436)
            (9000.0,0.31874254832043514)
            (9500.0,0.6595995149068118)
            (10000.0,0.47451320250974316)
            (10500.0,0.5351453183169261)
            (11000.0,0.6415337596554785)
            (11500.0,0.2723108193439695)
            (12000.0,-0.017732958689274353)
            (12500.0,-0.2763163408135647)
            (13000.0,-0.23190524448813882)
            (13500.0,-0.2669203226928833)
            (14000.0,-0.11345925761572995)
            (14500.0,-0.07011233412727796)
            (15000.0,-0.025886574774766788)
            (15000.0,0.18294962413770594)
            (14500.0,0.18896969133973268)
            (14000.0,0.1474291125817221)
            (13500.0,0.06112646443430503)
            (13000.0,0.18298894235101393)
            (12500.0,0.2354409374528802)
            (12000.0,0.5272038590915246)
            (11500.0,1.0782488119586862)
            (11000.0,1.9409747482838045)
            (10500.0,1.805240882229345)
            (10000.0,1.5681993103355558)
            (9500.0,2.1297682697381686)
            (9000.0,1.2056214570422574)
            (8500.0,0.9086198710071328)
            (8000.0,0.49385074599441803)
            (7500.0,1.0092099374969803)
            (7000.0,1.3322839990212754)
            (6500.0,0.6764851507297708)
            (6000.0,0.4161474829492731)
            (5500.0,0.373486855082999)
            (5000.0,0.5019511278745357)
            (4500.0,0.7899599598432292)
            (4000.0,1.2784660871948894)
            (3500.0,1.575780696525385)
            (3000.0,1.2917088273786712)
            (2500.0,1.8691605663188764)
            (2000.0,1.5705681070473534)
            (1500.0,0.5507370691949117)
            (1000.0,-0.018200439361633736)
            (500.0,-0.739851411813897)
            (500.0,-2.2114251961396696)
        }
        ;
    \addplot[color={rgb,1:red,0.3333;green,0.6588;blue,0.4078}, name path={372f1e83-a0bd-41b3-afcc-dce357693c51}, legend image code/.code={{
    \draw[fill={rgb,1:red,0.3333;green,0.6588;blue,0.4078}, fill opacity={0.1}] (0cm,-0.1cm) rectangle (0.6cm,0.1cm);
    }}, draw opacity={1.0}, line width={2}, solid]
        table[row sep={\\}]
        {
            \\
            500.0  -2.2114251961396696  \\
            1000.0  -1.1076250095410896  \\
            1500.0  -0.4120969429611726  \\
            2000.0  0.09497872151760489  \\
            2500.0  0.24549422778389016  \\
            3000.0  0.1355584553423665  \\
            3500.0  0.22167469067586992  \\
            4000.0  0.18999581896318413  \\
            4500.0  -0.0056921923827150805  \\
            5000.0  -0.16617038048485203  \\
            5500.0  -0.2263719812255825  \\
            6000.0  -0.09764043675513495  \\
            6500.0  0.026787651813827168  \\
            7000.0  0.18227002978949747  \\
            7500.0  0.20420215912737932  \\
            8000.0  -0.037617017046228524  \\
            8500.0  0.23311702689343436  \\
            9000.0  0.31874254832043514  \\
            9500.0  0.6595995149068118  \\
            10000.0  0.47451320250974316  \\
            10500.0  0.5351453183169261  \\
            11000.0  0.6415337596554785  \\
            11500.0  0.2723108193439695  \\
            12000.0  -0.017732958689274353  \\
            12500.0  -0.2763163408135647  \\
            13000.0  -0.23190524448813882  \\
            13500.0  -0.2669203226928833  \\
            14000.0  -0.11345925761572995  \\
            14500.0  -0.07011233412727796  \\
            15000.0  -0.025886574774766788  \\
        }
        ;
    \addlegendentry {$\tilde{b} = \mu(b)$}
    \addplot[color={rgb,1:red,0.3333;green,0.6588;blue,0.4078}, name path={514fc5e1-7f0f-4468-a2c6-8ee0f64b8dc2}, draw opacity={0.1}, line width={1}, solid, forget plot]
        table[row sep={\\}]
        {
            \\
            500.0  -0.739851411813897  \\
            1000.0  -0.018200439361633736  \\
            1500.0  0.5507370691949117  \\
            2000.0  1.5705681070473534  \\
            2500.0  1.8691605663188764  \\
            3000.0  1.2917088273786712  \\
            3500.0  1.575780696525385  \\
            4000.0  1.2784660871948894  \\
            4500.0  0.7899599598432292  \\
            5000.0  0.5019511278745357  \\
            5500.0  0.373486855082999  \\
            6000.0  0.4161474829492731  \\
            6500.0  0.6764851507297708  \\
            7000.0  1.3322839990212754  \\
            7500.0  1.0092099374969803  \\
            8000.0  0.49385074599441803  \\
            8500.0  0.9086198710071328  \\
            9000.0  1.2056214570422574  \\
            9500.0  2.1297682697381686  \\
            10000.0  1.5681993103355558  \\
            10500.0  1.805240882229345  \\
            11000.0  1.9409747482838045  \\
            11500.0  1.0782488119586862  \\
            12000.0  0.5272038590915246  \\
            12500.0  0.2354409374528802  \\
            13000.0  0.18298894235101393  \\
            13500.0  0.06112646443430503  \\
            14000.0  0.1474291125817221  \\
            14500.0  0.18896969133973268  \\
            15000.0  0.18294962413770594  \\
        }
        ;
    \addplot[color={rgb,1:red,0.3333;green,0.6588;blue,0.4078}, name path={d492a0b1-8da1-42bd-9f88-972c3e869f54}, draw opacity={0.1}, line width={1}, solid, forget plot]
        table[row sep={\\}]
        {
            \\
            500.0  -3.682998980465442  \\
            1000.0  -2.197049579720545  \\
            1500.0  -1.3749309551172568  \\
            2000.0  -1.3806106640121434  \\
            2500.0  -1.3781721107510962  \\
            3000.0  -1.0205919166939381  \\
            3500.0  -1.1324313151736451  \\
            4000.0  -0.8984744492685213  \\
            4500.0  -0.8013443446086594  \\
            5000.0  -0.8342918888442397  \\
            5500.0  -0.8262308175341639  \\
            6000.0  -0.6114283564595431  \\
            6500.0  -0.6229098471021165  \\
            7000.0  -0.9677439394422804  \\
            7500.0  -0.6008056192422218  \\
            8000.0  -0.5690847800868751  \\
            8500.0  -0.44238581722026404  \\
            9000.0  -0.5681363604013872  \\
            9500.0  -0.8105692399245449  \\
            10000.0  -0.6191729053160695  \\
            10500.0  -0.7349502455954928  \\
            11000.0  -0.6579072289728475  \\
            11500.0  -0.5336271732707472  \\
            12000.0  -0.5626697764700733  \\
            12500.0  -0.7880736190800096  \\
            13000.0  -0.6467994313272916  \\
            13500.0  -0.5949671098200716  \\
            14000.0  -0.374347627813182  \\
            14500.0  -0.3291943595942886  \\
            15000.0  -0.23472277368723954  \\
        }
        ;
    \addplot+[line width={0}, draw opacity={0}, fill={rgb,1:red,0.7686;green,0.3059;blue,0.3216}, fill opacity={0.1}, mark={none}, forget plot]
        coordinates {
            (500.0,-3.5980881189297964)
            (1000.0,-0.2867294374845144)
            (1500.0,4.0832705887904535)
            (2000.0,7.410102911212197)
            (2500.0,9.029396699951246)
            (3000.0,10.42345625113408)
            (3500.0,11.625058981617599)
            (4000.0,12.268835263620472)
            (4500.0,12.47923344738277)
            (5000.0,12.805039543275493)
            (5500.0,12.820735095354177)
            (6000.0,12.794351285575349)
            (6500.0,12.957570587226547)
            (7000.0,12.813190213527836)
            (7500.0,13.165255691857741)
            (8000.0,13.292094505372539)
            (8500.0,12.986205853108508)
            (9000.0,13.549593588991645)
            (9500.0,13.517144211392985)
            (10000.0,13.62327876114251)
            (10500.0,13.88902067586511)
            (11000.0,13.853486696013793)
            (11500.0,14.016058273452392)
            (12000.0,13.815251084444654)
            (12500.0,13.576295201500752)
            (13000.0,13.63867123795712)
            (13500.0,13.484438586913178)
            (14000.0,13.633361629346568)
            (14500.0,13.649515566846567)
            (15000.0,13.822466213095783)
            (15000.0,13.527687511048153)
            (14500.0,13.3288059406966)
            (14000.0,13.394489526481598)
            (13500.0,13.214605522029615)
            (13000.0,13.293251735088115)
            (12500.0,13.165061773250086)
            (12000.0,13.374014151989206)
            (11500.0,13.599463960910775)
            (11000.0,13.260727575241914)
            (10500.0,13.318616616296525)
            (10000.0,13.04656382472852)
            (9500.0,12.8561833994349)
            (9000.0,12.959368055057919)
            (8500.0,12.193301023758542)
            (8000.0,12.734708322306043)
            (7500.0,12.545280681279596)
            (7000.0,12.175102450527469)
            (6500.0,12.353701157897433)
            (6000.0,12.181545960333956)
            (5500.0,12.312349499138001)
            (5000.0,12.21612978383136)
            (4500.0,11.749918359079977)
            (4000.0,11.480073353544572)
            (3500.0,10.738504845701035)
            (3000.0,9.496209115496134)
            (2500.0,7.790372624756252)
            (2000.0,5.720173518573083)
            (1500.0,2.670664850705334)
            (1000.0,-1.2977335748549892)
            (500.0,-4.888330008394597)
            (500.0,-3.5980881189297964)
        }
        ;
    \addplot+[line width={0}, draw opacity={0}, fill={rgb,1:red,0.7686;green,0.3059;blue,0.3216}, fill opacity={0.1}, mark={none}, forget plot]
        coordinates {
            (500.0,-3.5980881189297964)
            (1000.0,-0.2867294374845144)
            (1500.0,4.0832705887904535)
            (2000.0,7.410102911212197)
            (2500.0,9.029396699951246)
            (3000.0,10.42345625113408)
            (3500.0,11.625058981617599)
            (4000.0,12.268835263620472)
            (4500.0,12.47923344738277)
            (5000.0,12.805039543275493)
            (5500.0,12.820735095354177)
            (6000.0,12.794351285575349)
            (6500.0,12.957570587226547)
            (7000.0,12.813190213527836)
            (7500.0,13.165255691857741)
            (8000.0,13.292094505372539)
            (8500.0,12.986205853108508)
            (9000.0,13.549593588991645)
            (9500.0,13.517144211392985)
            (10000.0,13.62327876114251)
            (10500.0,13.88902067586511)
            (11000.0,13.853486696013793)
            (11500.0,14.016058273452392)
            (12000.0,13.815251084444654)
            (12500.0,13.576295201500752)
            (13000.0,13.63867123795712)
            (13500.0,13.484438586913178)
            (14000.0,13.633361629346568)
            (14500.0,13.649515566846567)
            (15000.0,13.822466213095783)
            (15000.0,14.117244915143413)
            (14500.0,13.970225192996534)
            (14000.0,13.872233732211539)
            (13500.0,13.754271651796742)
            (13000.0,13.984090740826124)
            (12500.0,13.987528629751418)
            (12000.0,14.256488016900102)
            (11500.0,14.43265258599401)
            (11000.0,14.446245816785671)
            (10500.0,14.459424735433695)
            (10000.0,14.1999936975565)
            (9500.0,14.178105023351069)
            (9000.0,14.13981912292537)
            (8500.0,13.779110682458475)
            (8000.0,13.849480688439035)
            (7500.0,13.785230702435886)
            (7000.0,13.451277976528203)
            (6500.0,13.561440016555661)
            (6000.0,13.407156610816742)
            (5500.0,13.329120691570353)
            (5000.0,13.393949302719626)
            (4500.0,13.208548535685562)
            (4000.0,13.057597173696372)
            (3500.0,12.511613117534162)
            (3000.0,11.350703386772027)
            (2500.0,10.26842077514624)
            (2000.0,9.100032303851313)
            (1500.0,5.495876326875573)
            (1000.0,0.7242746998859604)
            (500.0,-2.3078462294649955)
            (500.0,-3.5980881189297964)
        }
        ;
    \addplot[color={rgb,1:red,0.7686;green,0.3059;blue,0.3216}, name path={cdc32631-a3a8-4e7a-bffc-a61907be3455}, legend image code/.code={{
    \draw[fill={rgb,1:red,0.7686;green,0.3059;blue,0.3216}, fill opacity={0.1}] (0cm,-0.1cm) rectangle (0.6cm,0.1cm);
    }}, draw opacity={1.0}, line width={2}, solid]
        table[row sep={\\}]
        {
            \\
            500.0  -3.5980881189297964  \\
            1000.0  -0.2867294374845144  \\
            1500.0  4.0832705887904535  \\
            2000.0  7.410102911212197  \\
            2500.0  9.029396699951246  \\
            3000.0  10.42345625113408  \\
            3500.0  11.625058981617599  \\
            4000.0  12.268835263620472  \\
            4500.0  12.47923344738277  \\
            5000.0  12.805039543275493  \\
            5500.0  12.820735095354177  \\
            6000.0  12.794351285575349  \\
            6500.0  12.957570587226547  \\
            7000.0  12.813190213527836  \\
            7500.0  13.165255691857741  \\
            8000.0  13.292094505372539  \\
            8500.0  12.986205853108508  \\
            9000.0  13.549593588991645  \\
            9500.0  13.517144211392985  \\
            10000.0  13.62327876114251  \\
            10500.0  13.88902067586511  \\
            11000.0  13.853486696013793  \\
            11500.0  14.016058273452392  \\
            12000.0  13.815251084444654  \\
            12500.0  13.576295201500752  \\
            13000.0  13.63867123795712  \\
            13500.0  13.484438586913178  \\
            14000.0  13.633361629346568  \\
            14500.0  13.649515566846567  \\
            15000.0  13.822466213095783  \\
        }
        ;
    \addlegendentry {$\tilde{b} = [\mu(b), \sigma(b)]$}
    \addplot[color={rgb,1:red,0.7686;green,0.3059;blue,0.3216}, name path={c04990fe-e79a-4cec-a931-10e28ad7b430}, draw opacity={0.1}, line width={1}, solid, forget plot]
        table[row sep={\\}]
        {
            \\
            500.0  -2.3078462294649955  \\
            1000.0  0.7242746998859604  \\
            1500.0  5.495876326875573  \\
            2000.0  9.100032303851313  \\
            2500.0  10.26842077514624  \\
            3000.0  11.350703386772027  \\
            3500.0  12.511613117534162  \\
            4000.0  13.057597173696372  \\
            4500.0  13.208548535685562  \\
            5000.0  13.393949302719626  \\
            5500.0  13.329120691570353  \\
            6000.0  13.407156610816742  \\
            6500.0  13.561440016555661  \\
            7000.0  13.451277976528203  \\
            7500.0  13.785230702435886  \\
            8000.0  13.849480688439035  \\
            8500.0  13.779110682458475  \\
            9000.0  14.13981912292537  \\
            9500.0  14.178105023351069  \\
            10000.0  14.1999936975565  \\
            10500.0  14.459424735433695  \\
            11000.0  14.446245816785671  \\
            11500.0  14.43265258599401  \\
            12000.0  14.256488016900102  \\
            12500.0  13.987528629751418  \\
            13000.0  13.984090740826124  \\
            13500.0  13.754271651796742  \\
            14000.0  13.872233732211539  \\
            14500.0  13.970225192996534  \\
            15000.0  14.117244915143413  \\
        }
        ;
    \addplot[color={rgb,1:red,0.7686;green,0.3059;blue,0.3216}, name path={b35cc357-b2bb-4981-8107-d4e11f6d16a3}, draw opacity={0.1}, line width={1}, solid, forget plot]
        table[row sep={\\}]
        {
            \\
            500.0  -4.888330008394597  \\
            1000.0  -1.2977335748549892  \\
            1500.0  2.670664850705334  \\
            2000.0  5.720173518573083  \\
            2500.0  7.790372624756252  \\
            3000.0  9.496209115496134  \\
            3500.0  10.738504845701035  \\
            4000.0  11.480073353544572  \\
            4500.0  11.749918359079977  \\
            5000.0  12.21612978383136  \\
            5500.0  12.312349499138001  \\
            6000.0  12.181545960333956  \\
            6500.0  12.353701157897433  \\
            7000.0  12.175102450527469  \\
            7500.0  12.545280681279596  \\
            8000.0  12.734708322306043  \\
            8500.0  12.193301023758542  \\
            9000.0  12.959368055057919  \\
            9500.0  12.8561833994349  \\
            10000.0  13.04656382472852  \\
            10500.0  13.318616616296525  \\
            11000.0  13.260727575241914  \\
            11500.0  13.599463960910775  \\
            12000.0  13.374014151989206  \\
            12500.0  13.165061773250086  \\
            13000.0  13.293251735088115  \\
            13500.0  13.214605522029615  \\
            14000.0  13.394489526481598  \\
            14500.0  13.3288059406966  \\
            15000.0  13.527687511048153  \\
        }
        ;
\end{axis}
\end{tikzpicture}

            % Recommended preamble:
% \usetikzlibrary{arrows.meta}
% \usetikzlibrary{backgrounds}
% \usepgfplotslibrary{patchplots}
% \usepgfplotslibrary{fillbetween}
% \pgfplotsset{%
%     layers/standard/.define layer set={%
%         background,axis background,axis grid,axis ticks,axis lines,axis tick labels,pre main,main,axis descriptions,axis foreground%
%     }{
%         grid style={/pgfplots/on layer=axis grid},%
%         tick style={/pgfplots/on layer=axis ticks},%
%         axis line style={/pgfplots/on layer=axis lines},%
%         label style={/pgfplots/on layer=axis descriptions},%
%         legend style={/pgfplots/on layer=axis descriptions},%
%         title style={/pgfplots/on layer=axis descriptions},%
%         colorbar style={/pgfplots/on layer=axis descriptions},%
%         ticklabel style={/pgfplots/on layer=axis tick labels},%
%         axis background@ style={/pgfplots/on layer=axis background},%
%         3d box foreground style={/pgfplots/on layer=axis foreground},%
%     },
% }

\begin{tikzpicture}[/tikz/background rectangle/.style={fill={rgb,1:red,1.0;green,1.0;blue,1.0}, fill opacity={1.0}, draw opacity={1.0}}, show background rectangle]
\begin{axis}[reverse legend, point meta max={nan}, point meta min={nan}, legend cell align={left}, legend columns={1}, title={Ablation: Prioritized action widening}, title style={at={{(0.5,1)}}, anchor={south}, font={{\fontsize{15 pt}{19.5 pt}\selectfont}}, color={rgb,1:red,0.0;green,0.0;blue,0.0}, draw opacity={1.0}, rotate={0.0}, align={center}}, legend style={color={rgb,1:red,0.0;green,0.0;blue,0.0}, draw opacity={1.0}, line width={1}, solid, fill={rgb,1:red,1.0;green,1.0;blue,1.0}, fill opacity={1.0}, text opacity={1.0}, font={{\fontsize{14 pt}{18.2 pt}\selectfont}}, text={rgb,1:red,0.0;green,0.0;blue,0.0}, cells={anchor={west}}, at={(0.98, 0.02)}, anchor={south east}}, axis background/.style={fill={rgb,1:red,1.0;green,1.0;blue,1.0}, opacity={1.0}}, anchor={north west}, xshift={20.0mm}, yshift={-5.0mm}, width={95.81mm}, height={75.81mm}, scaled x ticks={false}, xlabel={number of episodes trained on}, x tick style={color={rgb,1:red,0.0;green,0.0;blue,0.0}, opacity={1.0}}, x tick label style={color={rgb,1:red,0.0;green,0.0;blue,0.0}, opacity={1.0}, rotate={0}}, xlabel style={at={(ticklabel cs:0.5)}, anchor=near ticklabel, at={{(ticklabel cs:0.5)}}, anchor={near ticklabel}, font={{\fontsize{14 pt}{18.2 pt}\selectfont}}, color={rgb,1:red,0.0;green,0.0;blue,0.0}, draw opacity={1.0}, rotate={0.0}}, xmajorgrids={true}, xmin={500.0}, xmax={25000.0}, xticklabels={{5K,10K,15K,20K,25K}}, xtick={{5000,10000,15000,20000,25000}}, xtick align={inside}, xticklabel style={font={{\fontsize{14 pt}{18.2 pt}\selectfont}}, color={rgb,1:red,0.0;green,0.0;blue,0.0}, draw opacity={1.0}, rotate={0.0}}, x grid style={color={rgb,1:red,0.0;green,0.0;blue,0.0}, draw opacity={0.1}, line width={0.5}, solid}, xticklabel pos={left}, x axis line style={color={rgb,1:red,0.0;green,0.0;blue,0.0}, draw opacity={1.0}, line width={1}, solid}, scaled y ticks={false}, y tick style={color={rgb,1:red,0.0;green,0.0;blue,0.0}, opacity={1.0}}, y tick label style={color={rgb,1:red,0.0;green,0.0;blue,0.0}, opacity={1.0}, rotate={0}}, ylabel style={at={(ticklabel cs:0.5)}, anchor=near ticklabel, at={{(ticklabel cs:0.5)}}, anchor={near ticklabel}, font={{\fontsize{14 pt}{18.2 pt}\selectfont}}, color={rgb,1:red,0.0;green,0.0;blue,0.0}, draw opacity={1.0}, rotate={0.0}}, ymajorgrids={true}, ymin={-4.888330008394597}, ymax={14.566270649951788}, yticklabels={{$-3$,$0$,$3$,$6$,$9$,$12$}}, ytick={{-3.0,0.0,3.0,6.0,9.0,12.0}}, ytick align={inside}, yticklabel style={font={{\fontsize{14 pt}{18.2 pt}\selectfont}}, color={rgb,1:red,0.0;green,0.0;blue,0.0}, draw opacity={1.0}, rotate={0.0}}, y grid style={color={rgb,1:red,0.0;green,0.0;blue,0.0}, draw opacity={0.1}, line width={0.5}, solid}, yticklabel pos={left}, y axis line style={color={rgb,1:red,0.0;green,0.0;blue,0.0}, draw opacity={1.0}, line width={1}, solid}, colorbar={false}]
    \addplot+[line width={0}, draw opacity={0}, fill={rgb,1:red,0.298;green,0.4471;blue,0.6902}, fill opacity={0.1}, mark={none}, forget plot]
        coordinates {
            (500.0,-0.9714077357399192)
            (1000.0,-0.16270912013809785)
            (1500.0,0.5482086577647214)
            (2000.0,1.0140181329989648)
            (2500.0,2.2451264342115294)
            (3000.0,2.3692150280683153)
            (3500.0,2.1964932927230816)
            (4000.0,2.434104237459663)
            (4500.0,3.1407054248797683)
            (5000.0,3.7330152663692444)
            (5500.0,4.698535353027854)
            (6000.0,4.553237938017143)
            (6500.0,4.334080924363476)
            (7000.0,4.744533227081696)
            (7500.0,5.693217002831213)
            (8000.0,5.732237631274881)
            (8500.0,5.326674979252976)
            (9000.0,5.5143669651268254)
            (9500.0,5.570847207231299)
            (10000.0,5.272514945356499)
            (10500.0,5.7599268125296375)
            (11000.0,5.375189714927283)
            (11500.0,5.519950171694841)
            (12000.0,5.3601807807711985)
            (12500.0,5.328119937105072)
            (13000.0,5.668824611135639)
            (13500.0,5.18533206034218)
            (14000.0,5.140599800458277)
            (14500.0,5.2320532797008)
            (15000.0,5.924901571024206)
            (15500.0,5.519700258618914)
            (16000.0,5.236766723206967)
            (16500.0,5.231302393917693)
            (17000.0,5.083341778329667)
            (17500.0,5.130858804995459)
            (18000.0,5.110182494317399)
            (18500.0,5.242357491007194)
            (19000.0,5.633245610893979)
            (19500.0,5.620025121763053)
            (20000.0,5.502553741326366)
            (20500.0,5.462600389641761)
            (21000.0,5.746778819498082)
            (21500.0,5.665435885413881)
            (22000.0,5.4394955013860375)
            (22500.0,4.900596844788058)
            (23000.0,4.675661000356667)
            (23500.0,5.325307679309782)
            (24000.0,5.800805932249922)
            (24500.0,5.467183162033717)
            (25000.0,5.291972645947629)
            (25000.0,4.529976667522858)
            (24500.0,4.509873580188932)
            (24000.0,4.892470027110294)
            (23500.0,4.203088127065778)
            (23000.0,3.744124489719048)
            (22500.0,4.052618988536522)
            (22000.0,4.214031324192781)
            (21500.0,4.965487742568766)
            (21000.0,4.858753659349062)
            (20500.0,4.718878741355685)
            (20000.0,4.578322890957658)
            (19500.0,4.53260653006455)
            (19000.0,4.152872940750915)
            (18500.0,4.444783527447146)
            (18000.0,4.11426555935482)
            (17500.0,4.066116695539661)
            (17000.0,4.033883949041265)
            (16500.0,4.026523545616788)
            (16000.0,3.839972639838998)
            (15500.0,4.493177277460353)
            (15000.0,4.952565332893579)
            (14500.0,4.03931251181246)
            (14000.0,3.5983686357049898)
            (13500.0,3.916289801928542)
            (13000.0,4.343219873613231)
            (12500.0,3.610318890905456)
            (12000.0,3.957333279590066)
            (11500.0,3.7562014871865728)
            (11000.0,4.238263351220496)
            (10500.0,4.4033131130299985)
            (10000.0,3.856085781023187)
            (9500.0,4.2632297830829025)
            (9000.0,4.463200692722408)
            (8500.0,4.533260249994506)
            (8000.0,4.7713275881270185)
            (7500.0,4.652932542265297)
            (7000.0,3.4200621471751598)
            (6500.0,3.705420152111818)
            (6000.0,3.873474903818127)
            (5500.0,3.7020936963972386)
            (5000.0,2.9441931380338175)
            (4500.0,2.1369140180487385)
            (4000.0,1.415559135891409)
            (3500.0,1.1367494370400006)
            (3000.0,1.089282895777092)
            (2500.0,0.993885511533267)
            (2000.0,0.30653560808755176)
            (1500.0,-0.32059386903590403)
            (1000.0,-0.6487332462481534)
            (500.0,-1.3043563489216026)
            (500.0,-0.9714077357399192)
        }
        ;
    \addplot+[line width={0}, draw opacity={0}, fill={rgb,1:red,0.298;green,0.4471;blue,0.6902}, fill opacity={0.1}, mark={none}, forget plot]
        coordinates {
            (500.0,-0.9714077357399192)
            (1000.0,-0.16270912013809785)
            (1500.0,0.5482086577647214)
            (2000.0,1.0140181329989648)
            (2500.0,2.2451264342115294)
            (3000.0,2.3692150280683153)
            (3500.0,2.1964932927230816)
            (4000.0,2.434104237459663)
            (4500.0,3.1407054248797683)
            (5000.0,3.7330152663692444)
            (5500.0,4.698535353027854)
            (6000.0,4.553237938017143)
            (6500.0,4.334080924363476)
            (7000.0,4.744533227081696)
            (7500.0,5.693217002831213)
            (8000.0,5.732237631274881)
            (8500.0,5.326674979252976)
            (9000.0,5.5143669651268254)
            (9500.0,5.570847207231299)
            (10000.0,5.272514945356499)
            (10500.0,5.7599268125296375)
            (11000.0,5.375189714927283)
            (11500.0,5.519950171694841)
            (12000.0,5.3601807807711985)
            (12500.0,5.328119937105072)
            (13000.0,5.668824611135639)
            (13500.0,5.18533206034218)
            (14000.0,5.140599800458277)
            (14500.0,5.2320532797008)
            (15000.0,5.924901571024206)
            (15500.0,5.519700258618914)
            (16000.0,5.236766723206967)
            (16500.0,5.231302393917693)
            (17000.0,5.083341778329667)
            (17500.0,5.130858804995459)
            (18000.0,5.110182494317399)
            (18500.0,5.242357491007194)
            (19000.0,5.633245610893979)
            (19500.0,5.620025121763053)
            (20000.0,5.502553741326366)
            (20500.0,5.462600389641761)
            (21000.0,5.746778819498082)
            (21500.0,5.665435885413881)
            (22000.0,5.4394955013860375)
            (22500.0,4.900596844788058)
            (23000.0,4.675661000356667)
            (23500.0,5.325307679309782)
            (24000.0,5.800805932249922)
            (24500.0,5.467183162033717)
            (25000.0,5.291972645947629)
            (25000.0,6.0539686243724)
            (24500.0,6.424492743878501)
            (24000.0,6.70914183738955)
            (23500.0,6.447527231553787)
            (23000.0,5.607197510994286)
            (22500.0,5.748574701039595)
            (22000.0,6.664959678579294)
            (21500.0,6.365384028258997)
            (21000.0,6.634803979647102)
            (20500.0,6.206322037927836)
            (20000.0,6.426784591695075)
            (19500.0,6.707443713461556)
            (19000.0,7.113618281037042)
            (18500.0,6.039931454567243)
            (18000.0,6.106099429279977)
            (17500.0,6.195600914451257)
            (17000.0,6.13279960761807)
            (16500.0,6.436081242218597)
            (16000.0,6.633560806574935)
            (15500.0,6.5462232397774756)
            (15000.0,6.8972378091548325)
            (14500.0,6.42479404758914)
            (14000.0,6.682830965211565)
            (13500.0,6.454374318755818)
            (13000.0,6.994429348658047)
            (12500.0,7.045920983304688)
            (12000.0,6.763028281952331)
            (11500.0,7.2836988562031095)
            (11000.0,6.512116078634071)
            (10500.0,7.116540512029276)
            (10000.0,6.68894410968981)
            (9500.0,6.878464631379696)
            (9000.0,6.565533237531243)
            (8500.0,6.120089708511446)
            (8000.0,6.693147674422743)
            (7500.0,6.733501463397129)
            (7000.0,6.069004306988233)
            (6500.0,4.962741696615133)
            (6000.0,5.23300097221616)
            (5500.0,5.694977009658469)
            (5000.0,4.521837394704671)
            (4500.0,4.144496831710798)
            (4000.0,3.452649339027917)
            (3500.0,3.2562371484061625)
            (3000.0,3.6491471603595387)
            (2500.0,3.4963673568897917)
            (2000.0,1.721500657910378)
            (1500.0,1.4170111845653468)
            (1000.0,0.32331500597195767)
            (500.0,-0.6384591225582359)
            (500.0,-0.9714077357399192)
        }
        ;
    \addplot[color={rgb,1:red,0.298;green,0.4471;blue,0.6902}, name path={956953be-4d43-4371-9c8e-d4d8efadafb2}, legend image code/.code={{
    \draw[fill={rgb,1:red,0.298;green,0.4471;blue,0.6902}, fill opacity={0.1}] (0cm,-0.1cm) rectangle (0.6cm,0.1cm);
    }}, draw opacity={1.0}, line width={2}, solid]
        table[row sep={\\}]
        {
            \\
            500.0  -0.9714077357399192  \\
            1000.0  -0.16270912013809785  \\
            1500.0  0.5482086577647214  \\
            2000.0  1.0140181329989648  \\
            2500.0  2.2451264342115294  \\
            3000.0  2.3692150280683153  \\
            3500.0  2.1964932927230816  \\
            4000.0  2.434104237459663  \\
            4500.0  3.1407054248797683  \\
            5000.0  3.7330152663692444  \\
            5500.0  4.698535353027854  \\
            6000.0  4.553237938017143  \\
            6500.0  4.334080924363476  \\
            7000.0  4.744533227081696  \\
            7500.0  5.693217002831213  \\
            8000.0  5.732237631274881  \\
            8500.0  5.326674979252976  \\
            9000.0  5.5143669651268254  \\
            9500.0  5.570847207231299  \\
            10000.0  5.272514945356499  \\
            10500.0  5.7599268125296375  \\
            11000.0  5.375189714927283  \\
            11500.0  5.519950171694841  \\
            12000.0  5.3601807807711985  \\
            12500.0  5.328119937105072  \\
            13000.0  5.668824611135639  \\
            13500.0  5.18533206034218  \\
            14000.0  5.140599800458277  \\
            14500.0  5.2320532797008  \\
            15000.0  5.924901571024206  \\
            15500.0  5.519700258618914  \\
            16000.0  5.236766723206967  \\
            16500.0  5.231302393917693  \\
            17000.0  5.083341778329667  \\
            17500.0  5.130858804995459  \\
            18000.0  5.110182494317399  \\
            18500.0  5.242357491007194  \\
            19000.0  5.633245610893979  \\
            19500.0  5.620025121763053  \\
            20000.0  5.502553741326366  \\
            20500.0  5.462600389641761  \\
            21000.0  5.746778819498082  \\
            21500.0  5.665435885413881  \\
            22000.0  5.4394955013860375  \\
            22500.0  4.900596844788058  \\
            23000.0  4.675661000356667  \\
            23500.0  5.325307679309782  \\
            24000.0  5.800805932249922  \\
            24500.0  5.467183162033717  \\
            25000.0  5.291972645947629  \\
        }
        ;
    \addlegendentry {sampled from action space}
    \addplot[color={rgb,1:red,0.298;green,0.4471;blue,0.6902}, name path={030bddd8-e1d8-4b57-864f-b8fb76c23383}, draw opacity={0.1}, line width={1}, solid, forget plot]
        table[row sep={\\}]
        {
            \\
            500.0  -0.6384591225582359  \\
            1000.0  0.32331500597195767  \\
            1500.0  1.4170111845653468  \\
            2000.0  1.721500657910378  \\
            2500.0  3.4963673568897917  \\
            3000.0  3.6491471603595387  \\
            3500.0  3.2562371484061625  \\
            4000.0  3.452649339027917  \\
            4500.0  4.144496831710798  \\
            5000.0  4.521837394704671  \\
            5500.0  5.694977009658469  \\
            6000.0  5.23300097221616  \\
            6500.0  4.962741696615133  \\
            7000.0  6.069004306988233  \\
            7500.0  6.733501463397129  \\
            8000.0  6.693147674422743  \\
            8500.0  6.120089708511446  \\
            9000.0  6.565533237531243  \\
            9500.0  6.878464631379696  \\
            10000.0  6.68894410968981  \\
            10500.0  7.116540512029276  \\
            11000.0  6.512116078634071  \\
            11500.0  7.2836988562031095  \\
            12000.0  6.763028281952331  \\
            12500.0  7.045920983304688  \\
            13000.0  6.994429348658047  \\
            13500.0  6.454374318755818  \\
            14000.0  6.682830965211565  \\
            14500.0  6.42479404758914  \\
            15000.0  6.8972378091548325  \\
            15500.0  6.5462232397774756  \\
            16000.0  6.633560806574935  \\
            16500.0  6.436081242218597  \\
            17000.0  6.13279960761807  \\
            17500.0  6.195600914451257  \\
            18000.0  6.106099429279977  \\
            18500.0  6.039931454567243  \\
            19000.0  7.113618281037042  \\
            19500.0  6.707443713461556  \\
            20000.0  6.426784591695075  \\
            20500.0  6.206322037927836  \\
            21000.0  6.634803979647102  \\
            21500.0  6.365384028258997  \\
            22000.0  6.664959678579294  \\
            22500.0  5.748574701039595  \\
            23000.0  5.607197510994286  \\
            23500.0  6.447527231553787  \\
            24000.0  6.70914183738955  \\
            24500.0  6.424492743878501  \\
            25000.0  6.0539686243724  \\
        }
        ;
    \addplot[color={rgb,1:red,0.298;green,0.4471;blue,0.6902}, name path={9cdf18fb-3c46-4e7c-a2dd-147496c064d4}, draw opacity={0.1}, line width={1}, solid, forget plot]
        table[row sep={\\}]
        {
            \\
            500.0  -1.3043563489216026  \\
            1000.0  -0.6487332462481534  \\
            1500.0  -0.32059386903590403  \\
            2000.0  0.30653560808755176  \\
            2500.0  0.993885511533267  \\
            3000.0  1.089282895777092  \\
            3500.0  1.1367494370400006  \\
            4000.0  1.415559135891409  \\
            4500.0  2.1369140180487385  \\
            5000.0  2.9441931380338175  \\
            5500.0  3.7020936963972386  \\
            6000.0  3.873474903818127  \\
            6500.0  3.705420152111818  \\
            7000.0  3.4200621471751598  \\
            7500.0  4.652932542265297  \\
            8000.0  4.7713275881270185  \\
            8500.0  4.533260249994506  \\
            9000.0  4.463200692722408  \\
            9500.0  4.2632297830829025  \\
            10000.0  3.856085781023187  \\
            10500.0  4.4033131130299985  \\
            11000.0  4.238263351220496  \\
            11500.0  3.7562014871865728  \\
            12000.0  3.957333279590066  \\
            12500.0  3.610318890905456  \\
            13000.0  4.343219873613231  \\
            13500.0  3.916289801928542  \\
            14000.0  3.5983686357049898  \\
            14500.0  4.03931251181246  \\
            15000.0  4.952565332893579  \\
            15500.0  4.493177277460353  \\
            16000.0  3.839972639838998  \\
            16500.0  4.026523545616788  \\
            17000.0  4.033883949041265  \\
            17500.0  4.066116695539661  \\
            18000.0  4.11426555935482  \\
            18500.0  4.444783527447146  \\
            19000.0  4.152872940750915  \\
            19500.0  4.53260653006455  \\
            20000.0  4.578322890957658  \\
            20500.0  4.718878741355685  \\
            21000.0  4.858753659349062  \\
            21500.0  4.965487742568766  \\
            22000.0  4.214031324192781  \\
            22500.0  4.052618988536522  \\
            23000.0  3.744124489719048  \\
            23500.0  4.203088127065778  \\
            24000.0  4.892470027110294  \\
            24500.0  4.509873580188932  \\
            25000.0  4.529976667522858  \\
        }
        ;
    \addplot+[line width={0}, draw opacity={0}, fill={rgb,1:red,0.7686;green,0.3059;blue,0.3216}, fill opacity={0.1}, mark={none}, forget plot]
        coordinates {
            (500.0,-3.5980881189297964)
            (1000.0,-0.2867294374845144)
            (1500.0,4.0832705887904535)
            (2000.0,7.410102911212197)
            (2500.0,9.029396699951246)
            (3000.0,10.42345625113408)
            (3500.0,11.625058981617599)
            (4000.0,12.268835263620472)
            (4500.0,12.47923344738277)
            (5000.0,12.805039543275493)
            (5500.0,12.820735095354177)
            (6000.0,12.794351285575349)
            (6500.0,12.957570587226547)
            (7000.0,12.813190213527836)
            (7500.0,13.165255691857741)
            (8000.0,13.292094505372539)
            (8500.0,12.986205853108508)
            (9000.0,13.549593588991645)
            (9500.0,13.517144211392985)
            (10000.0,13.62327876114251)
            (10500.0,13.88902067586511)
            (11000.0,13.853486696013793)
            (11500.0,14.016058273452392)
            (12000.0,13.815251084444654)
            (12500.0,13.576295201500752)
            (13000.0,13.63867123795712)
            (13500.0,13.484438586913178)
            (14000.0,13.633361629346568)
            (14500.0,13.649515566846567)
            (15000.0,13.822466213095783)
            (15500.0,13.382313812011695)
            (16000.0,13.5140154161857)
            (16500.0,13.579083950448183)
            (17000.0,13.498941531307537)
            (17500.0,13.607394815065485)
            (18000.0,13.552748483778405)
            (18500.0,13.616196638114225)
            (19000.0,13.780568072628904)
            (19500.0,13.518074675244069)
            (20000.0,13.33604031870097)
            (20500.0,13.525006940939495)
            (21000.0,13.957365016566218)
            (21500.0,13.627174204778514)
            (22000.0,13.640762977383282)
            (22500.0,13.36647540405284)
            (23000.0,13.360629294604877)
            (23500.0,13.401790171037334)
            (24000.0,13.636672003198687)
            (24500.0,13.463733550863445)
            (25000.0,13.55846124490316)
            (25000.0,13.241909732340957)
            (24500.0,13.125279877052549)
            (24000.0,13.272256732287913)
            (23500.0,13.111926719491827)
            (23000.0,12.950844067099196)
            (22500.0,12.936592189373638)
            (22000.0,13.23034260620681)
            (21500.0,13.090943387078191)
            (21000.0,13.348459383180648)
            (20500.0,12.868719669561909)
            (20000.0,12.77972645457068)
            (19500.0,13.009526961631535)
            (19000.0,13.32423630084649)
            (18500.0,13.140699657641672)
            (18000.0,12.980473430272973)
            (17500.0,12.883578399363895)
            (17000.0,12.900574755756603)
            (16500.0,13.157765079781433)
            (16000.0,13.059865348511762)
            (15500.0,13.09203834634866)
            (15000.0,13.527687511048153)
            (14500.0,13.3288059406966)
            (14000.0,13.394489526481598)
            (13500.0,13.214605522029615)
            (13000.0,13.293251735088115)
            (12500.0,13.165061773250086)
            (12000.0,13.374014151989206)
            (11500.0,13.599463960910775)
            (11000.0,13.260727575241914)
            (10500.0,13.318616616296525)
            (10000.0,13.04656382472852)
            (9500.0,12.8561833994349)
            (9000.0,12.959368055057919)
            (8500.0,12.193301023758542)
            (8000.0,12.734708322306043)
            (7500.0,12.545280681279596)
            (7000.0,12.175102450527469)
            (6500.0,12.353701157897433)
            (6000.0,12.181545960333956)
            (5500.0,12.312349499138001)
            (5000.0,12.21612978383136)
            (4500.0,11.749918359079977)
            (4000.0,11.480073353544572)
            (3500.0,10.738504845701035)
            (3000.0,9.496209115496134)
            (2500.0,7.790372624756252)
            (2000.0,5.720173518573083)
            (1500.0,2.670664850705334)
            (1000.0,-1.2977335748549892)
            (500.0,-4.888330008394597)
            (500.0,-3.5980881189297964)
        }
        ;
    \addplot+[line width={0}, draw opacity={0}, fill={rgb,1:red,0.7686;green,0.3059;blue,0.3216}, fill opacity={0.1}, mark={none}, forget plot]
        coordinates {
            (500.0,-3.5980881189297964)
            (1000.0,-0.2867294374845144)
            (1500.0,4.0832705887904535)
            (2000.0,7.410102911212197)
            (2500.0,9.029396699951246)
            (3000.0,10.42345625113408)
            (3500.0,11.625058981617599)
            (4000.0,12.268835263620472)
            (4500.0,12.47923344738277)
            (5000.0,12.805039543275493)
            (5500.0,12.820735095354177)
            (6000.0,12.794351285575349)
            (6500.0,12.957570587226547)
            (7000.0,12.813190213527836)
            (7500.0,13.165255691857741)
            (8000.0,13.292094505372539)
            (8500.0,12.986205853108508)
            (9000.0,13.549593588991645)
            (9500.0,13.517144211392985)
            (10000.0,13.62327876114251)
            (10500.0,13.88902067586511)
            (11000.0,13.853486696013793)
            (11500.0,14.016058273452392)
            (12000.0,13.815251084444654)
            (12500.0,13.576295201500752)
            (13000.0,13.63867123795712)
            (13500.0,13.484438586913178)
            (14000.0,13.633361629346568)
            (14500.0,13.649515566846567)
            (15000.0,13.822466213095783)
            (15500.0,13.382313812011695)
            (16000.0,13.5140154161857)
            (16500.0,13.579083950448183)
            (17000.0,13.498941531307537)
            (17500.0,13.607394815065485)
            (18000.0,13.552748483778405)
            (18500.0,13.616196638114225)
            (19000.0,13.780568072628904)
            (19500.0,13.518074675244069)
            (20000.0,13.33604031870097)
            (20500.0,13.525006940939495)
            (21000.0,13.957365016566218)
            (21500.0,13.627174204778514)
            (22000.0,13.640762977383282)
            (22500.0,13.36647540405284)
            (23000.0,13.360629294604877)
            (23500.0,13.401790171037334)
            (24000.0,13.636672003198687)
            (24500.0,13.463733550863445)
            (25000.0,13.55846124490316)
            (25000.0,13.875012757465363)
            (24500.0,13.802187224674341)
            (24000.0,14.00108727410946)
            (23500.0,13.691653622582841)
            (23000.0,13.770414522110558)
            (22500.0,13.796358618732043)
            (22000.0,14.051183348559755)
            (21500.0,14.163405022478837)
            (21000.0,14.566270649951788)
            (20500.0,14.18129421231708)
            (20000.0,13.89235418283126)
            (19500.0,14.026622388856604)
            (19000.0,14.236899844411319)
            (18500.0,14.091693618586778)
            (18000.0,14.125023537283838)
            (17500.0,14.331211230767074)
            (17000.0,14.097308306858471)
            (16500.0,14.000402821114934)
            (16000.0,13.968165483859638)
            (15500.0,13.67258927767473)
            (15000.0,14.117244915143413)
            (14500.0,13.970225192996534)
            (14000.0,13.872233732211539)
            (13500.0,13.754271651796742)
            (13000.0,13.984090740826124)
            (12500.0,13.987528629751418)
            (12000.0,14.256488016900102)
            (11500.0,14.43265258599401)
            (11000.0,14.446245816785671)
            (10500.0,14.459424735433695)
            (10000.0,14.1999936975565)
            (9500.0,14.178105023351069)
            (9000.0,14.13981912292537)
            (8500.0,13.779110682458475)
            (8000.0,13.849480688439035)
            (7500.0,13.785230702435886)
            (7000.0,13.451277976528203)
            (6500.0,13.561440016555661)
            (6000.0,13.407156610816742)
            (5500.0,13.329120691570353)
            (5000.0,13.393949302719626)
            (4500.0,13.208548535685562)
            (4000.0,13.057597173696372)
            (3500.0,12.511613117534162)
            (3000.0,11.350703386772027)
            (2500.0,10.26842077514624)
            (2000.0,9.100032303851313)
            (1500.0,5.495876326875573)
            (1000.0,0.7242746998859604)
            (500.0,-2.3078462294649955)
            (500.0,-3.5980881189297964)
        }
        ;
    \addplot[color={rgb,1:red,0.7686;green,0.3059;blue,0.3216}, name path={b29c4a87-3bc5-4e70-be4e-611fe84062a1}, legend image code/.code={{
    \draw[fill={rgb,1:red,0.7686;green,0.3059;blue,0.3216}, fill opacity={0.1}] (0cm,-0.1cm) rectangle (0.6cm,0.1cm);
    }}, draw opacity={1.0}, line width={2}, solid]
        table[row sep={\\}]
        {
            \\
            500.0  -3.5980881189297964  \\
            1000.0  -0.2867294374845144  \\
            1500.0  4.0832705887904535  \\
            2000.0  7.410102911212197  \\
            2500.0  9.029396699951246  \\
            3000.0  10.42345625113408  \\
            3500.0  11.625058981617599  \\
            4000.0  12.268835263620472  \\
            4500.0  12.47923344738277  \\
            5000.0  12.805039543275493  \\
            5500.0  12.820735095354177  \\
            6000.0  12.794351285575349  \\
            6500.0  12.957570587226547  \\
            7000.0  12.813190213527836  \\
            7500.0  13.165255691857741  \\
            8000.0  13.292094505372539  \\
            8500.0  12.986205853108508  \\
            9000.0  13.549593588991645  \\
            9500.0  13.517144211392985  \\
            10000.0  13.62327876114251  \\
            10500.0  13.88902067586511  \\
            11000.0  13.853486696013793  \\
            11500.0  14.016058273452392  \\
            12000.0  13.815251084444654  \\
            12500.0  13.576295201500752  \\
            13000.0  13.63867123795712  \\
            13500.0  13.484438586913178  \\
            14000.0  13.633361629346568  \\
            14500.0  13.649515566846567  \\
            15000.0  13.822466213095783  \\
            15500.0  13.382313812011695  \\
            16000.0  13.5140154161857  \\
            16500.0  13.579083950448183  \\
            17000.0  13.498941531307537  \\
            17500.0  13.607394815065485  \\
            18000.0  13.552748483778405  \\
            18500.0  13.616196638114225  \\
            19000.0  13.780568072628904  \\
            19500.0  13.518074675244069  \\
            20000.0  13.33604031870097  \\
            20500.0  13.525006940939495  \\
            21000.0  13.957365016566218  \\
            21500.0  13.627174204778514  \\
            22000.0  13.640762977383282  \\
            22500.0  13.36647540405284  \\
            23000.0  13.360629294604877  \\
            23500.0  13.401790171037334  \\
            24000.0  13.636672003198687  \\
            24500.0  13.463733550863445  \\
            25000.0  13.55846124490316  \\
        }
        ;
    \addlegendentry {sampled from policy network}
    \addplot[color={rgb,1:red,0.7686;green,0.3059;blue,0.3216}, name path={229aa9d0-fd4f-44f0-9597-c5e8289f7386}, draw opacity={0.1}, line width={1}, solid, forget plot]
        table[row sep={\\}]
        {
            \\
            500.0  -2.3078462294649955  \\
            1000.0  0.7242746998859604  \\
            1500.0  5.495876326875573  \\
            2000.0  9.100032303851313  \\
            2500.0  10.26842077514624  \\
            3000.0  11.350703386772027  \\
            3500.0  12.511613117534162  \\
            4000.0  13.057597173696372  \\
            4500.0  13.208548535685562  \\
            5000.0  13.393949302719626  \\
            5500.0  13.329120691570353  \\
            6000.0  13.407156610816742  \\
            6500.0  13.561440016555661  \\
            7000.0  13.451277976528203  \\
            7500.0  13.785230702435886  \\
            8000.0  13.849480688439035  \\
            8500.0  13.779110682458475  \\
            9000.0  14.13981912292537  \\
            9500.0  14.178105023351069  \\
            10000.0  14.1999936975565  \\
            10500.0  14.459424735433695  \\
            11000.0  14.446245816785671  \\
            11500.0  14.43265258599401  \\
            12000.0  14.256488016900102  \\
            12500.0  13.987528629751418  \\
            13000.0  13.984090740826124  \\
            13500.0  13.754271651796742  \\
            14000.0  13.872233732211539  \\
            14500.0  13.970225192996534  \\
            15000.0  14.117244915143413  \\
            15500.0  13.67258927767473  \\
            16000.0  13.968165483859638  \\
            16500.0  14.000402821114934  \\
            17000.0  14.097308306858471  \\
            17500.0  14.331211230767074  \\
            18000.0  14.125023537283838  \\
            18500.0  14.091693618586778  \\
            19000.0  14.236899844411319  \\
            19500.0  14.026622388856604  \\
            20000.0  13.89235418283126  \\
            20500.0  14.18129421231708  \\
            21000.0  14.566270649951788  \\
            21500.0  14.163405022478837  \\
            22000.0  14.051183348559755  \\
            22500.0  13.796358618732043  \\
            23000.0  13.770414522110558  \\
            23500.0  13.691653622582841  \\
            24000.0  14.00108727410946  \\
            24500.0  13.802187224674341  \\
            25000.0  13.875012757465363  \\
        }
        ;
    \addplot[color={rgb,1:red,0.7686;green,0.3059;blue,0.3216}, name path={c8399994-4f97-4b9c-89ea-6770c56eaae0}, draw opacity={0.1}, line width={1}, solid, forget plot]
        table[row sep={\\}]
        {
            \\
            500.0  -4.888330008394597  \\
            1000.0  -1.2977335748549892  \\
            1500.0  2.670664850705334  \\
            2000.0  5.720173518573083  \\
            2500.0  7.790372624756252  \\
            3000.0  9.496209115496134  \\
            3500.0  10.738504845701035  \\
            4000.0  11.480073353544572  \\
            4500.0  11.749918359079977  \\
            5000.0  12.21612978383136  \\
            5500.0  12.312349499138001  \\
            6000.0  12.181545960333956  \\
            6500.0  12.353701157897433  \\
            7000.0  12.175102450527469  \\
            7500.0  12.545280681279596  \\
            8000.0  12.734708322306043  \\
            8500.0  12.193301023758542  \\
            9000.0  12.959368055057919  \\
            9500.0  12.8561833994349  \\
            10000.0  13.04656382472852  \\
            10500.0  13.318616616296525  \\
            11000.0  13.260727575241914  \\
            11500.0  13.599463960910775  \\
            12000.0  13.374014151989206  \\
            12500.0  13.165061773250086  \\
            13000.0  13.293251735088115  \\
            13500.0  13.214605522029615  \\
            14000.0  13.394489526481598  \\
            14500.0  13.3288059406966  \\
            15000.0  13.527687511048153  \\
            15500.0  13.09203834634866  \\
            16000.0  13.059865348511762  \\
            16500.0  13.157765079781433  \\
            17000.0  12.900574755756603  \\
            17500.0  12.883578399363895  \\
            18000.0  12.980473430272973  \\
            18500.0  13.140699657641672  \\
            19000.0  13.32423630084649  \\
            19500.0  13.009526961631535  \\
            20000.0  12.77972645457068  \\
            20500.0  12.868719669561909  \\
            21000.0  13.348459383180648  \\
            21500.0  13.090943387078191  \\
            22000.0  13.23034260620681  \\
            22500.0  12.936592189373638  \\
            23000.0  12.950844067099196  \\
            23500.0  13.111926719491827  \\
            24000.0  13.272256732287913  \\
            24500.0  13.125279877052549  \\
            25000.0  13.241909732340957  \\
        }
        ;
\end{axis}
\end{tikzpicture}

    }
    \end{minipage}%%%
    \hspace*{3mm} % \hfill
    \begin{minipage}{0.3\textwidth}
        \resizebox{\textwidth}{!}{%
            \begin{tikzpicture}[]
\begin{axis}[
  height = {6cm},
  ylabel = {value influence $z_q$},
  title = {\shortstack{Influence of\\$Q$-weighed counts on return}},
  xlabel = {visit count influence $z_n$},
  width = {6cm},
  enlargelimits = false,
  axis on top,
  colormap={mycolormap}{ rgb(0cm)=(0.1,0.1,0.1) rgb(1cm)=(0.225356,0.246889,0.268731) rgb(2cm)=(0.333565,0.367922,0.391532) rgb(3cm)=(0.51593,0.541278,0.547958) rgb(4cm)=(0.753644,0.757794,0.737969) rgb(5cm)=(0.917794,0.920966,0.881936) },
  colorbar,
  xmin = 0.0,
  xmax = 1.0,
  ymin = 0.0,
  ymax = 1.0
]

\addplot[
  point meta min = -0.0,
  point meta max = 13.09,
  point meta = explicit,
  matrix plot*,
  mesh/cols = 11,
  mesh/rows = 11
] table[
  meta = data
] {figures/results/tmp_50000000000021.dat};

\end{axis}
\end{tikzpicture} % _rs15x15
        }
    \end{minipage}

    \begin{minipage}[t]{0.69\textwidth}
        \caption{
        Ablation study in \textsc{LightDark}$(10)$.
        (Left) Learning is faster when policy network is trained using $Q$-weighted visit counts.
        (Middle) Incorporating belief uncertainty is crucial for learning.
        (Right) Action widening from the policy network shows significant improvement.
        The same red curves are shown and one std is shaded from three seeds with exponential smoothing with weight $0.6$.}
        \label{fig:ablations}
    \end{minipage}%%%
    \hspace*{3mm} % \hfill
    \begin{minipage}[t]{0.3\textwidth}
        \caption{Ablation study in \textsc{RockSample}$(20,20)$.
        Combining value and count information leads to the highest return. The diagonal is identical due to the $\argmax$ of \cref{eq:policy_q_weight}.}
        \label{fig:z_sweep}
    \end{minipage}
\end{figure*}
