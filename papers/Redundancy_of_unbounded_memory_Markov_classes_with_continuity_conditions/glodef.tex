% Theorem-like environments

\newtheorem{Theorem}{Theorem}
\newtheorem{Theorem*}{Theorem}

\newtheorem{Claim}{Claim}
\newtheorem{Claim*}{Claim}
\newtheorem{Corollary}[Theorem]{Corollary}
\newtheorem{CounterExample*}{$\overline{\hbox{\bf Example}}$}
\newtheorem{Definition}{Definition}
\newtheorem{Example}{Example}
\newtheorem{Example*}{Example}
\newtheorem{Exercise}{Exercise}
\newtheorem{Intuition*}{Intuition}
\newtheorem{Joke*}{Joke}
\newtheorem{Lemma}[Theorem]{Lemma}
\newtheorem{Lemma*}{Lemma}
\newtheorem{Open problem}{Open problem}
\newtheorem{Proposition}{Proposition}
\newtheorem{Property}{Property}
\newtheorem{Question}{Question}
\newtheorem{Question*}{Question}
\newtheorem{Remark*}{Remark}
\newtheorem{Result}{Result}

% Begin / End Theorems

\def \bthm#1{\begin{#1}\upshape ~}
\def \@bthm#1[#2]{\begin{#1}[#2]\upshape ~}



\def \ethm#1{\rqed\end{#1}}
\def \ethmp#1{\end{#1}}     % No box, when ends in displayed equation

\def \bClaim     {\bthm{Claim}}
\def \eClaim     {\ethm{Claim}}
\def \eClaimp    {\ethmp{Claim}}
\def \bCorollary {\bthm{Corollary}}
\def \eCorollary {\ethm{Corollary}}
\def \eCorollaryp{\ethmp{Corollary}}
\def \bCounter   {\bthm{CounterExample*}}
\def \eCounter   {\ethm{CounterExample*}}
\def \eCounterp  {\ethmp{CounterExample}}
\def \bDefinition {\@ifnextchar[{\@bthm{Definition}}{\bthm{Definition}}}
\def \eDefinition{\ethm{Definition}}
\def \eDefinitionp{\ethmp{Definition}}
\def \bExample   {\bthm{Example}}
\def \eExample   {\ethm{Example}}
\def \eExamplep  {\ethmp{Example}}
\def \bExercise  {\bthm{Exercise}}
\def \eExercise  {\ethm{Exercise}}
\def \eExercisep {\ethmp{Exercise}}
\def \bIntuition {\bthm{Intuition}}
\def \eIntuition {\ethmp{Intuition}}
\def \bJoke      {\bthm{Joke}}
\def \eJoke      {\ethm{Joke}}
\def \bLemma     {\@ifnextchar[{\@bthm{Lemma}}{\bthm{Lemma}}}
\def \eLemma     {\ethm{Lemma}}
\def \eLemmap    {\ethmp{Lemma}}
\def \bOpen      {\bthm{Open problem}}
\def \eOpen      {\ethm{Open problem}}
\def \eOpenp     {\ethmp{Open problem}}
\def \bProperty  {\bthm{Property}}
\def \eProperty  {\ethm{Property}}
\def \ePropertyp {\ethmp{Property}}
\def \bProposition  {\bthm{Proposition}}
\def \eProposition  {\ethm{Proposition}}
\def \ePropositionp {\ethmp{Proposition}}
\def \bQuestion  {\bthm{Question}}
\def \eQuestion  {\ethm{Question}}
\def \bRemark    {\bthm{Remark}}
\def \eRemark    {\ethm{Remark}}
\def \eRemarkp    {\ethmp{Remark}}
\def \bResult    {\bthm{Result}}
\def \eResult    {\ethm{Result}}
\def \bTheorem   {\@ifnextchar[{\@bthm{Theorem}}{\bthm{Theorem}}}
\def \eTheorem   {\ethm{Theorem}}
\def \eTheoremp  {\ethmp{Theorem}}

\def \bSubexa    {\begin{subexa}}
\def \eSubexa    {\ethm{subexa}}
\def \eSubexap   {\ethmp{subexa}}

% Environments

\newenvironment{Problem}{\textbf{Problem}\\ \begin{enumerate}}{\end{enumerate}}
\newenvironment{Problems}{\textbf{Problems}\\ \begin{enumerate}}{\end{enumerate}}

%\newenvironment{Problems}{\begin{trivlist}\item[]{\textbf{Problems}}{\end{trivlist}}}

% Headers

\def \skpbld#1{\par\noindent\textbf{#1}\quad}
\def \skpblds#1{\skpbld{#1}\par\noindent}

\def \Answer   {\skpbld{Answer}}
\def \Answers  {\skpblds{Answers}}
\def \Basis    {\skpbld{Basis}}
\def \Check    {\skpbld{Check}}
\def \Intuition{\skpbld{Intuition}}
\def \Method   {\skpbld{Method}}
\def \Methods  {\skpblds{Methods}}
\def \Outline  {\skpbld{Outline}}
\def \Proof    {\skpbld{Proof}}
\def \Proofs   {\skpblds{Proofs}}
%\def \Problem  {\skpbld{Problem}}
%\def \Problems {\skpblds{Problems}}
\def \Proline  {\skpbld{Proof Outline}}
\def \Remark   {\skpbld{Remark}}
\def \Remarks  {\skpblds{Remarks}}
\def \Solution {\skpbld{Solution}}
\def \Solutions{\skpblds{Solutions}}
\def \Verify   {\skpbld{Verify}}
\def \Step     {\skpbld{Step}}

% Ignores

\newcommand{\ignore}[1]{{}}

\newcommand{\problem}[1]{#1}
%\newcommand{\problem}[1]{}

%\newcommand{\solution}[1]{\mbox{}\\ \medskip\noindent{\bf Solution\medskip}#1}
\newcommand{\solution}[1]{{\\ \bf Solution}\quad #1}
%\newcommand{\solution}[1]{#1}
%\newcommand{\solution}[1]{}

\newcommand{\source}[1]{(Taken from~\cite{#1})}
%\newcommand{\source}[1]{}

\newcommand{\modifiedfrom}[1]{(Modified from~\cite{#1})}
%\newcommand{\source}[1]{}

\newcommand{\takenfrom}[1]{(Taken from~\cite{#1})}

\newcommand{\reportedby}[1]{Reported by #1.}
%\newcommand{\reportedby}[1]{}

\newcommand{\hint}[1]{Hint: #1}
%\newcommand{\hint}[1]{}
% Formating

\newcommand{\joke}[1]{\footnote{#1}}
\newcommand{\trivia}[1]{\footnote{#1}}

% Equation formatting

\newcommand{\spreqn}[1]{{\qquad\text{#1}\qquad}}

% Handouts

\newcommand{\hohead}[3]{
\hfill\begin{minipage}{3.5cm}
\bf
#1\\ % class # (CSE20) (perhaps + term)
#2\\ % term (Fall 2K+1) or name (Alon Orlitsky)
%#3\\[2mm]
%Handout #3\\
\end{minipage}
}

\newcommand{\solhead}[3]{ % CSE20 -- Discrete Math; Fall 2K+1; 1
\begin{center}
  \Large #1\\[1ex]
  \normalsize #2\\[3ex]
  Solutions for Homework Assignment \##3\\
\end{center}
}

% Blackboard fonts

\newcommand{\EE}{\mathbb{E}}
\newcommand{\CC}{\mathbb{C}}
\newcommand{\NN}{\mathbb{N}} % \dN
\newcommand{\QQ}{\mathbb{Q}} % \dQ
\newcommand{\RR}{\mathbb{R}} % \dR
\newcommand{\ZZ}{\mathbb{Z}} % \dZ

% Numbers

\newcommand{\complex}{\CC}
\newcommand{\integers}{\ZZ}
\newcommand{\naturals}{\NN}
\newcommand{\rationals}{\QQ}
\newcommand{\reals}{\RR}
\newcommand{\realsge}{{\reals_{\ge}}}
\newcommand{\realsp}{\reals^+}
\newcommand{\RRp}{\realsp}
\newcommand{\integersp}{\ZZ_+}
\newcommand{\integerss}[1]{\ZZ_{\ge{#1}}}

% boldface

\def \ba     {{\bf a}}
\def \bx     {{\bf x}}
\def \by     {{\bf y}}

\def \bA     {{\bf A}}
\def \bB     {{\bf B}}
\def \bC     {{\bf C}}
\def \bD     {{\bf D}}
\def \bF     {{\bf F}}
\def \bG     {{\bf G}}
\def \bL     {{\bf L}}
\def \bQ     {{\bf Q}}
\def \bR     {{\bf R}}
\def \bS     {{\bf S}}
\def \bT     {{\bf T}}
\def \bX     {{\bf X}}
\def \bY     {{\bf Y}}
\def \bZ     {{\bf Z}}

% caligraphics

\def \cA     {{\cal A}}
\def \cB     {{\cal B}}
\def \cC     {{\cal C}}
\def \cD     {{\cal D}}
\def \cF     {{\cal F}}
\def \cG     {{\cal G}}
\def \cH     {{\cal H}}
\def \cI     {{\cal I}}
\def \cK     {{\cal K}}
\def \cL     {{\cal L}}
\def \cM     {{\cal M}}
\def \cN     {{\cal N}}
\def \cO     {{\cal O}}
\def \cP     {{\cal P}}
\def \cQ     {{\cal Q}}
\def \cR     {{\cal R}}
\def \cS     {{\cal S}}
\def \cT     {{\cal T}}
\def \cW     {{\cal W}}
\def \cX     {{\cal X}}
\def \cY     {{\cal Y}}
\def \cZ     {{\cal Z}}

% vectors

\def \vec#1{{\overline{#1}}}

\def \veca  {{\vec a}}
\def \vecb  {{\vec b}}
\def \vecu  {{\vec u}}
\def \vecv  {{\vec v}}
\def \vecV  {{\vec V}}
\def \vecx  {{\vec x}}
\def \vecX  {{\vec X}}
\def \vecy  {{\vec y}}
\def \vecY  {{\vec Y}}
\def \vecz  {{\vec z}}

\def \vct#1{{\overline{#1}}}

\def \vcta  {{\vct a}}
\def \vctb  {{\vct b}}
\def \vctq  {{\vct q}}
\def \vctu  {{\vct u}}
\def \vctv  {{\vct v}}
\def \vctV  {{\vct V}}
\def \vctx  {{\vct x}}
\def \vctX  {{\vct X}}
\def \vcty  {{\vct y}}
\def \vctY  {{\vct Y}}
\def \vctz  {{\vct z}}

\def \vctbeta  {{\vct\beta}}

% Random variables

\newcommand{\rnd}[1]{{\bf #1}}
\newcommand{\rndl}{{\rnd l}}
\newcommand{\rnds}{{\rnd s}}
\newcommand{\rndx}{{\rnd x}}
\newcommand{\rndy}{{\rnd y}}
\newcommand{\rndz}{{\rnd z}}

% arrow vectors

\newcommand{\rvc}{{\overrightarrow c}}

% Abbreviations

\newcommand{\eg}{\textit{e.g.,}\xspace}
\newcommand{\etc}{etc.\@\xspace}
\newcommand{\ie}{\textit{i.e.,}\xspace}  % note that overridden in spanish

% notes

%\definecolor{light}{gray}{.75}

\def \mynote#1{{}}

% marginal notes

\def \marcha{{\marginpar{CHNGD}}}
\def \marfix{{\#\marginpar{FIX \#}}}
\def \marnew{{\marginpar{NEW}}}
\def \marok{{\#\marginpar{\# OK?}}}

% qed's

\def \eqed    {\eqno{\qed}}
\def \rqed    {\hbox{}~\hfill~$\qed$}

% sequences

\def \upto  {{,}\ldots{,}}
\def \zn    {0\upto n}
\def \znmo  {0\upto n-1}
\def \znpo  {0\upto n+1}
\def \ztnmo {0\upto 2^n-1}
\def \ok    {1\upto k}
\def \on    {1\upto n}
\def \onmo  {1\upto n-1}
\def \onpo  {1\upto n+1}

% sets

\def \set#1#2{{\{{#1}\upto{#2}\}}}

\def \setpmo   {\{\pm 1\}}
\def \setmpo   {\{-1{,}1\}}
\def \setzo    {\{0{,}1\}}
\def \setzn    {\{\zn\}}
\def \setznmo  {\{\znmo\}}
\def \setztnmo {\{\ztnmo\}}
\def \setok    {\{\ok\}}
\def \seton    {\{\on\}}
\def \setonmo  {\{\onmo\}}
\def \setzon   {\setzo^n}
\def \setzos   {\setzo^*}

\def \sets#1{{\{#1\}}}
\def \Sets#1{{\left\{#1\right\}}}

\def \inseg#1{{[#1]}} % use \intsgm instead
\newcommand{\intsgm}[1]{{[#1]}}

% functions

\def \ord    {\#}

\def \suml   {\sum\limits}
\def \prodl  {\prod\limits}

% Set operations

\def \union  {\cup}
%\def \union  {\bigcup}
\def \unionl {\union\limits}
%\def \Union {\Bigcup} % want this
\def \Union  {\bigcup}
\def \Unionl {\Union\limits}

\def \inter  {\cap}
%\def \inter  {\bigcap}
\def \interl {\inter\limits}
\def \Inter {\Bigcap}
\def \Interl {\Inter\limits}

% Floors and Ceilings

\def \ceil#1{{\lceil{#1}\rceil}}
\def \Ceil#1{{\left\lceil{#1}\right\rceil}}
\def \floor#1{{\lfloor{#1}\rfloor}}
\def \Floor#1{{\left\lfloor{#1}\right\rfloor}}
\def \absvlu#1{{|#1|}}
\def \Absvlu#1{{\left|{#1}\right|}}


% Parentheses, brackets

\def \paren#1{{({#1})}}
\def \Paren#1{{\left({#1}\right)}}
\def \brack#1{{[{#1}]}}
\def \Brack#1{{\left[{#1}\right]}}

%\def \frac#1#2{{{#1}\over{#2}}}
\def \frc#1#2{{\frac{#1}{#2}}}

\def \binomial#1#2{{{#1}\choose{#2}}}

\def \gcd#1#2{{{\rm gcd}\paren{{#1},{#2}}}}
\def \Gcd#1#2{{{\rm gcd}\Paren{{#1},{#2}}}}

\def \lcm#1#2{{{\rm lcm}\paren{{#1},{#2}}}}
\def \Lcm#1#2{{{\rm lcm}\Paren{{#1},{#2}}}}

% number theory

\newcommand{\base}[2]{{[#1]_{#2}}}

% equalities

\newcommand{\ed}{\stackrel{\mathrm{def}}{=}}

\newcommand{\cnvprb}{\stackrel{\mathrm{p}}{\to}}
\newcommand{\cnvdst}{\stackrel{\mathrm{d}}{\to}}
\newcommand{\cas}{\stackrel{\mathrm{a.s.}}{\to}}

%\def \ed     {\,{\buildrel \rm def \over =}\,}
\def \gap    {\ \hbox{\raisebox{-.6ex}{$\stackrel{\textstyle>}{\sim}$}}\ }

\newcommand{\al}[1]{\stackrel{\mathit{{#1}}}{<}}
\newcommand{\ag}[1]{\stackrel{\mathit{{#1}}}{>}}
\newcommand{\ale}[1]{\stackrel{\mathit{{#1}}}{\le}}
\newcommand{\age}[1]{\stackrel{\mathit{{#1}}}{\ge}}
\newcommand{\aeq}[1]{\stackrel{\mathit{{#1}}}{=}}
\newcommand{\ato}[1]{\stackrel{\mathit{{#1}}}{\rightarrow}}

%\newcommand{\eae}{\approx} % exponentially asymptotically eq - replace by \ere
%\newcommand{\eal}{\stackrel<\approx} % replace by \erl

\newcommand{\re}{\sim}
\newcommand{\rle}{\stackrel<\sim} % roughly (asymptotically taken) <=
%\newcommand{\rl}{\stackrel<\sim}
\newcommand{\rge}{\stackrel>\sim}
%\newcommand{\rg}{\stackrel>\sim}

\newcommand{\ere}{\approx} % exponentially roughly equal
\newcommand{\erle}{\stackrel<\approx}
%\newcommand{\erl}{\stackrel<\approx}
\newcommand{\erge}{\stackrel>\approx}
%\newcommand{\erg}{\stackrel>\approx}

% values

\def \half    {{\frac12}}
\def \quarter {{\frac14}}
\def \oo#1{{\frac1{#1}}}

% notation

\def \pr     {{p}}
\def \Pr     {{\hbox{Pr}}}
\def \iff    {{\it iff }}
\def \th     {{\rm th }}

% ignore

\def\ignore#1{}

% Logic

\newcommand{\ra}{\rightarrow}
\def \contra {{\leftrightarrows}}
\newcommand{\ol}[1]{{\overline{#1}}}
%\def \ob {\overline}

\def \ve {{\lor}} % needed?
\def \eq {{\equiv}} % needed?

% spaces

\newcommand{\spcin}{\hspace{1.0in}}
\newcommand{\spchin}{\hspace{.5in}}

% Default text appears in regular print in both book and class versions. 
% There are two types of text that need to be highlighted in class:
% clson - not mentioned in book version (eg jokes)
% clsbk - regular text in book (the parts that need be said)

%For book:
%\newcommand{\clson}[1]{}
%\newcommand{\clsbk}[1]{#1}
%For class:
\newcommand{\clson}[1]{\colorbox{light}{{#1}}}
\newcommand{\clsbk}[1]{\colorbox{light}{{#1}}}

%\newcommand{\bi}{\begin{aopl}}
%\newcommand{\ei}{\end{aopl}}

\newcommand{\bi}{\begin{itemize}}
\newcommand{\ei}{\end{itemize}}
%\newcommand{\bq}{\begin{quote}}
%\newcommand{\eq}{\end{quote}}

\newenvironment{aopl}
  {\begin{list}{}{\setlength{\itemsep}{4pt plus 2pt minus 2pt}}}
  {\end{list}}

% operators

\def\orpro{\mathop{\mathchoice
   {\vee\kern-.49em\raise.7ex\hbox{$\cdot$}\kern.4em}
   {\vee\kern-.45em\raise.63ex\hbox{$\cdot$}\kern.2em}
   {\vee\kern-.4em\raise.3ex\hbox{$\cdot$}\kern.1em}
   {\vee\kern-.35em\raise2.2ex\hbox{$\cdot$}\kern.1em}}\limits}

\def\andpro{\mathop{\mathchoice
 {\wedge\kern-.46em\lower.69ex\hbox{$\cdot$}\kern.3em}
 {\wedge\kern-.46em\lower.58ex\hbox{$\cdot$}\kern.25em}
 {\wedge\kern-.38em\lower.5ex\hbox{$\cdot$}\kern.1em}
 {\wedge\kern-.3em\lower.5ex\hbox{$\cdot$}\kern.1em}}\limits}

\def\inter{\mathop{\mathchoice
   {\cap}
   {\cap}
   {\cap}
   {\cap}}}
\def\interl {\inter\limits}

\def\carpro{\mathop{\mathchoice
   {\times}
   {\times}
   {\times}
   {\times}}\limits}

\def\simge{\mathrel{%
   \rlap{\raise 0.511ex \hbox{$>$}}{\lower 0.511ex \hbox{$\sim$}}}}

\def\simle{\mathrel{
   \rlap{\raise 0.511ex \hbox{$<$}}{\lower 0.511ex \hbox{$\sim$}}}}

\newcommand{\pfx}{{\prec}}
\newcommand{\pfxeq}{{\preceq}}

\newcommand{\sfx}{{\succ}}
\newcommand{\sfxeq}{{\succeq}}

% for old documents

\newcommand{\twlrm}{\fontsize{12}{14pt}\normalfont\rmfamily}
\newcommand{\tenrm}{\fontsize{10}{12pt}\normalfont\rmfamily}

% picture macros

\newcommand{\grid}[2]{				% grid{width}{height}
  \multiput(0,0)(1,0){#1}{\line(0,1){#2}}
  \put(#1,0){\line(0,1){#2}}
  \multiput(0,0)(0,1){#2}{\line(1,0){#1}}
  \put(0,#2){\line(1,0){#1}}
}

% inductiion macros

%\newcommand{\induction}[3]{
%Inductive statement: {#1}\\
%We prove: {#2}\\
%Basis: {#3}\\
%Step: {#4}\\
%Proof of step: {#5}
%}
\def \Sets#1{{\left\{#1\right\}}}

\def \inseg#1{{[#1]}} % use \intsgm instead
\providecommand{\intsgm}[1]{{[#1]}}

% functions

\def \ord    {\#}

\def \suml   {\sum\limits}
\def \prodl  {\prod\limits}

% Set operations

\def \union  {\cup}
%\def \union  {\bigcup}
\def \unionl {\union\limits}
%\def \Union {\Bigcup} % want this
\def \Union  {\bigcup}
\def \Unionl {\Union\limits}

\def \inter  {\cap}
%\def \inter  {\bigcap}
\def \interl {\inter\limits}
\def \Inter {\Bigcap}
\def \Interl {\Inter\limits}

% Floors and Ceilings

\def \ceil#1{{\lceil{#1}\rceil}}
\def \Ceil#1{{\left\lceil{#1}\right\rceil}}
\def \floor#1{{\lfloor{#1}\rfloor}}
\def \Floor#1{{\left\lfloor{#1}\right\rfloor}}

% Parentheses, brackets

\def \paren#1{{({#1})}}
\def \Paren#1{{\left({#1}\right)}}
\def \brack#1{{[{#1}]}}
\def \Brack#1{{\left[{#1}\right]}}

%\def \frac#1#2{{{#1}\over{#2}}}
\def \frc#1#2{{\frac{#1}{#2}}}

\def \binomial#1#2{{{#1}\choose{#2}}}

\def \gcd#1#2{{{\rm gcd}\paren{{#1},{#2}}}}
\def \Gcd#1#2{{{\rm gcd}\Paren{{#1},{#2}}}}

\def \lcm#1#2{{{\rm lcm}\paren{{#1},{#2}}}}
\def \Lcm#1#2{{{\rm lcm}\Paren{{#1},{#2}}}}

% number theory


% equalities

%\def \ed     {\,{\buildrel \rm def \over =}\,}
\def \gap    {\ \hbox{\raisebox{-.6ex}{$\stackrel{\textstyle>}{\sim}$}}\ }


%\newcommand{\eae}{\approx} % exponentially asymptotically eq - replace by \ere
%\newcommand{\eal}{\stackrel<\approx} % replace by \erl

% values

\def \half    {{\frac12}}
\def \quarter {{\frac14}}
\def \oo#1{{\frac1{#1}}}

% notation

\def \pr     {{p}}
\def \Pr     {{\hbox{Pr}}}
\def \iff    {{\it iff }}
\def \th     {{\rm th }}

% ignore

\def\ignore#1{}

% Logic

\def \contra {{\leftrightarrows}}
%\def \ob {\overline}

\def \ve {{\lor}} % needed?
\def \eq {{\equiv}} % needed?

% spaces


% Default text appears in regular print in both book and class versions. 
% There are two types of text that need to be highlighted in class:
% clson - not mentioned in book version (eg jokes)
% clsbk - regular text in book (the parts that need be said)

%For book:
%\newcommand{\clson}[1]{}
%\newcommand{\clsbk}[1]{#1}
%For class:

%\newcommand{\bi}{\begin{aopl}}
%\newcommand{\ei}{\end{aopl}}

%\newcommand{\bq}{\begin{quote}}
%\newcommand{\eq}{\end{quote}}

% operators

\def\orpro{\mathop{\mathchoice
   {\vee\kern-.49em\raise.7ex\hbox{$\cdot$}\kern.4em}
   {\vee\kern-.45em\raise.63ex\hbox{$\cdot$}\kern.2em}
   {\vee\kern-.4em\raise.3ex\hbox{$\cdot$}\kern.1em}
   {\vee\kern-.35em\raise2.2ex\hbox{$\cdot$}\kern.1em}}\limits}

\def\andpro{\mathop{\mathchoice
 {\wedge\kern-.46em\lower.69ex\hbox{$\cdot$}\kern.3em}
 {\wedge\kern-.46em\lower.58ex\hbox{$\cdot$}\kern.25em}
 {\wedge\kern-.38em\lower.5ex\hbox{$\cdot$}\kern.1em}
 {\wedge\kern-.3em\lower.5ex\hbox{$\cdot$}\kern.1em}}\limits}

\def\inter{\mathop{\mathchoice
   {\cap}
   {\cap}
   {\cap}
   {\cap}}}
\def\interl {\inter\limits}

\def\carpro{\mathop{\mathchoice
   {\times}
   {\times}
   {\times}
   {\times}}\limits}

\def\simge{\mathrel{%
   \rlap{\raise 0.511ex \hbox{$>$}}{\lower 0.511ex \hbox{$\sim$}}}}

\def\simle{\mathrel{
   \rlap{\raise 0.511ex \hbox{$<$}}{\lower 0.511ex \hbox{$\sim$}}}}


% for old documents


% picture macros

% inductiion macros

%\newcommand{\induction}[3]{
%Inductive statement: {#1}\\
%We prove: {#2}\\
%Basis: {#3}\\
%Step: {#4}\\
%Proof of step: {#5}
%}
