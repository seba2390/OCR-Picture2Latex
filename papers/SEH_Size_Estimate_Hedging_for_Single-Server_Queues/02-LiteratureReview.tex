\section{Related Work} \label{LiteratureReview}

Scheduling policies and their performance evaluation in a preemptive M/G/1 queue have been a subject of interest for some time. Size-based policies are known to perform better than size-oblivious policies with respect to sojourn times. In fact, the SRPT policy is optimal in minimizing the MST \cite{schrage1968letter}. However, size-based policies have a considerable disadvantage: When the exact processing times are not known to the system before scheduling, which is often the case in practical settings, their performance may significantly degrade. Dell’Amico et al.\ \cite{dell2015psbs} study the performance of SRPT with estimated job processing times and demonstrate the consequences of job processing time underestimations under different settings. Studies in Harchol-Balter et al.\ \cite{harchol2003size} and Chang et al.\ \cite{chang2011scheduling} discuss the effect of inexact processing time information in size-based policies for web servers and MapReduce systems, respectively. Our paper assumes that the processing time is not available to the scheduler until the job is fully processed, but that processing time estimations are available. The related literature for this setting is reviewed in the following paragraph.

Lu et al.\ \cite{lu2004size} were the first to study this setting. They show that size-based policies only benefit the performance when the correlation between a job’s real and estimated processing time is high. The results in Wierman and Nuyens \cite{wierman2008scheduling}, Bender et al.\ \cite{bender2002improved}, and Becchetti et al.\ \cite{becchetti2004semi} are obtained by making assumptions that may be problematic in practice. A strict upper bound on the estimation error is assumed in \cite{wierman2008scheduling}. On the other hand, \cite{becchetti2004semi} and \cite{bender2002improved} define specific job processing time classes and schedule the jobs based on their processing time class, which can be problematic for very small or very large jobs. This setting is also known as semi-clairvoyant scheduling. In this work, we do not assume any bounds on the estimation error or assign jobs to particular processing time classes. Consistent with this body of work, we do find that SEH is not recommended for systems with large estimation error variance. However, we do find that it performs well for levels of estimation error variance that are typically found in practice.

When the job processing time distribution is available, the Gittins’ Index policy \cite{gittins1979bandit} assigns a score to each job based on the processing time it has received so far, and the scheduler chooses the job with the highest score to process at each point in time. This policy is proven to be optimal for minimizing the MST in a single-server queue when the job processing time distribution is known \cite{aalto2009gittins}. This policy is specified in the next section.



