\section{Conclusion and Future Work} \label{Conclusion}

The SRPT policy, which is optimal for scheduling in single-server systems, may have problematic performance when job processing times are estimated. This work has considered the problem of scheduling with the presence of job processing time estimates. A multiplicative error model is used to produce estimation errors proportional to the job processing times. We have introduced a novel heuristic that combines the merits of SERPT and SEPT and requires minimal calculation overhead and no information about the job processing time and estimation error distributions. We have shown that this policy is consistent with a Gittins'-like view of the problem. Our numerical results demonstrate that the SEH policy has desirable performance in minimizing both the MST and mean slowdown of the system when there is low variance in the estimation error distribution. It outperforms SERPT except in scenarios where the job processing time variance is extremely low. Examining the SEH policy under other error models as well as analytic bounds as to how far it is from optimal could be investigated in future work. It would also be useful to examine how well policies designed for worst case performance would perform with respect to the performance metrics considered in this paper. The work of Purohit et al.\ \cite{purohit2018improving} is an intriguing candidate, as it runs two policies in parallel to provide worst case performance guarantees, even when there are large estimation errors.

Not much work has been done in the area of multi-server scheduling in the presence of estimation errors. One major reason is that determining optimal policies for multi-server queues is much more challenging compared to the single-server case. Mailach and Down \cite{mailach2017scheduling} suggest that when SRPT is used in a multi-server system, the estimation error affects the system's performance to a lesser degree than in a single-server system. Grosof et al.\ \cite{grosof2018srpt} prove that multi-server SRPT is asymptotically optimal when an M/G/$k$ system is heavily loaded. Our work only evaluates the performance of SEH in a single-server framework so we leave the extension and evaluation of this policy in multi-server queues for future investigation. \\
\\ \textbf{Acknowledgment}

The authors would like to thank Ziv Scully for useful discussions on the limitations of the SEH policy.