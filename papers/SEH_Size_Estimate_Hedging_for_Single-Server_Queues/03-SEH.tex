\section{Size Estimate Hedging: A Simple Dynamic Priority Scheduling Policy} \label{SEH}


\subsection{Model} \label{GeneralModel}
Consider an M/G/1 queue where preemption is allowed and we are interested in minimizing the MST. 
We assume that a job's processing time is not known upon arrival; however, an estimated processing time is provided to the scheduler. We concentrate on a multiplicative error model where the error distribution is independent of the job processing time distribution. The estimated processing time 
$\hat S$ of a job is defined as $\hat S = SX$ where $S$ is the job processing time and $X$ is the job processing time
estimation error. We assume that the value of $\hat S$ is known upon each job's arrival and is denoted by $\hat{s}$. 
The choice of a multiplicative error model results in having an absolute error proportional to the job processing time $S$, thus avoiding situations where the estimation errors tend to be worse for small jobs than for large jobs. Furthermore, Dell'Amico et al.\ \cite{dell2015psbs} and Pastorelli et al.\ \cite{pastorelli2013hfsp} suggest that a multiplicative error model is a better reflection of reality. To define our scheduling policies, we also require the notion of a quantum of service. The job with the highest priority is processed for a quantum of service $\Delta$ until either it completes or a new job arrives. At that point, priorities are recomputed. 



\subsection{Gittins' Index Approach} \label{GittinsIndex}
The Gittins' Index Policy is an appropriate technique for determining scheduling policies when the job processing time and estimation error distributions are known. For a waiting job $i$, an index $G(a_i)$ is calculated, where $a_i$ is the elapsed processing time. At each time epoch, the Gittins' Index policy processes the job with the
highest index $G(a)$ among all of the present waiting jobs \cite{gittins1979bandit}. The Gittins' rule takes the job's elapsed processing time into account and calculates the optimal quantum of service $\Delta ^*(a)$ that it should receive. 



The associated efficiency function $J(a,\Delta ),{\rm{ }}a,{\rm{ }}\Delta  \ge 0$ of a job
with processing time $S$, elapsed processing time $a$ and quantum of service $\Delta $ is defined as




\begin{equation}\label{eq:1} J(a,\Delta ) = \frac{{P(S - a \le \Delta |S > a)}}{{E[\min \{ S - a,{\rm{
}}\Delta \} |S > a]}}.
\end{equation} 




The numerator is the probability that the job will be completed within a quantum of service $\Delta $, and the denominator is
the expected remaining processing time a job with elapsed processing time $a$ and quantum of service
$\Delta $ will require to be completed.

The server (preemptively) processes the job with the highest
index at each decision epoch. Decisions are made when  \begin {enumerate*}  \item a
new job arrives to the queue, \item the current job under processing completes, or \item
the current job receives its optimal quantum of service and
does not complete.\end {enumerate*} If there are multiple jobs that have the same highest
index and all have zero optimal quanta of service, the processor will be shared
among them as long as this situation does not change. If there is only one job
with the highest index and zero optimal quantum of service, its index should be
updated throughout its processing \cite {aalto2009gittins}. 

Although the Gittins' Index policy is optimal in terms of minimizing the mean
sojourn time in an $M/G/1$ queue \cite {aalto2009gittins}, the assumption of knowing the job size and estimation error distributions might not always be practical to make. Furthermore, forming the Gittins' Index policy's efficiency function has significant computational overhead. As a result, this policy may be a problematic choice for real environments where the scheduling speed is important. However, examining the form of optimal policies has helped us in the construction of a simple heuristic. In particular, the notion of defining a policy in terms of an index allows us to make precise our notion of combining the relative merits of SRPT and SEPT. 

\subsection{Motivation} \label{Motivation}
When a job enters the system under SERPT, there is no basis on which to assume that the estimated processing time, $\hat{s}$, is incorrect. However, when the elapsed processing time reaches $\hat{s}$, we are certain that the job processing time has been underestimated. In addition, Dell'Amico et al.\ \cite{dell2019scheduling} show that SEPT performs well when dealing with estimated processing times and in the presence of estimation errors, in particular severe underestimates. So, we would like to combine these two policies. A convenient way to do this is to introduce a Gittins'-like score function, where a higher score indicates a higher priority. We will be aggressive and use the score function for SERPT until the point that we know a job is underestimated and then freeze the score, which is similar to what SEPT's constant score function does (see \eqref{eq:11} below). In this way, instead of switching to SEPT's score function, we would like to give credit for the jobs' cumulative elapsed processing times.

The score functions for SRPT, SERPT, and
SEPT are provided in \eqref{eq:9}, \eqref{eq:10}, and \eqref{eq:11}, respectively.

\begin {equation}\label{eq:9}
G(a,s)=\frac{1}{{s - a}},
\end{equation}

\begin {equation}\label{eq:10}
% \mathop {\max }\limits_\Delta  J(a,\Delta ,\hat s)=
G(a,\hat{s})=
\begin{cases}
\frac{1}{{\hat s - a}}, &  \hat s > a ,\\
\infty,  &  \hat s \le a,
\end{cases}
\end{equation}

\begin {equation}\label{eq:11}
G(a,\hat{s})=\frac{1}{{\hat s}}.
\end{equation}

We note that \eqref{eq:9} and \eqref{eq:10} have an increasing score function, and \eqref{eq:11} always assigns a constant score for a particular job.

\subsection{The SEH Policy} \label{FormalDefiniton}

Combining the score functions for SERPT and SEPT, we now define our policy. As discussed in the previous section, we would like to transition between SERPT when we cannot determine if a job processing time is underestimated to a fixed priority like SEPT when it is determined that underestimation has occurred. One consequence of using this policy is that any underestimated small job can still receive a ``high'' score and be processed, while underestimated large jobs will have a much lower score and do not interfere, even with underestimated small jobs.  Furthermore, not needing to know the job processing time and estimation error distribution, the SEH Policy does not have much overhead. Thus, it can schedule the jobs at a speed comparable to the SEPT policy. 

We introduce the score function of our SEH policy as 




\begin {equation}\label{eq:8}
G(a,\hat{s})=
\begin{cases}
\frac{1}{{\hat s - a(1 - \frac{a}{{2\hat s}})}}, & 0 \le a < \hat s,\\
\frac{2}{{\hat s}}, & a \ge \hat s,
\end{cases}
\end{equation}

\noindent where the scheduling decisions are only made at arrivals and departures. 

With the score function in \eqref{eq:8}, a job's score will increase up to the point that
it receives processing equal to its estimated processing time and then receives a constant score of
$\frac{2}{{\hat s}}$ until it completes. The choice of $2$ was made after some experimentation, it would be worthwhile to explore the sensitivity of the performance to this choice.

\subsection{Gittins' Index vs. SEH} \label{GittinsVsSEH}
In this section, we show that the form of our policy is consistent with the Gittins' index in the setting that we only know the error estimate distribution. In particular, we have no a priori or learned knowledge of the processing time distribution. 





With our estimation model in mind, \eqref{eq:1} can be rewritten as 

\begin{equation} \label{eq:2} J(a,\Delta,\hat{s} ) = \frac{{P(\frac{{\hat s}}{X} - a \le \Delta |\frac{{\hat s}}{X} >
a)}}{{E[\min \{ \frac{{\hat s}}{X} - a,{\rm{ }}\Delta \} |\frac{{\hat s}}{X} >
a]}}.
\end{equation}
The Gittins' index $G(a,\hat{s}),{\rm{ }}a \ge 0,$ is defined by 
\begin{equation*}\label{eq:3}
G(a,\hat{s}) = \mathop {\sup
}\limits_{\Delta  \ge 0} {\rm{ }}J(a,\Delta,\hat{s} ).
\end{equation*}
The optimal quantum of service is
denoted as 
\begin{equation*}\label{eq:4}
{\Delta ^*}(a,\hat{s}) = \sup \{ \Delta  \ge 0|G(a,\hat{s}) = J(a,\Delta ,\hat{s} )\}.
\end{equation*}

Suppose that the lower and upper limits on the estimation error distribution are $l$ and $u$, respectively ($l$ may be zero and $u$ may be $\infty$). After some calculation, the Gittins' index can then be written as
\begin{equation}\label{eq:6}
G(a,\hat{s}) = 
\begin{cases}

\frac{1}{{\hat s - aE[X|X \le \frac{{\hat s}}{a}]}}, & \frac{{\hat s}}{a} < u,\\ 
\frac{1}{{\hat s - aE[X]}},& otherwise,
\end{cases}
\end{equation}
\noindent where ${\Delta ^*}  = \frac{{\hat s}}{l} - a$. For instance, the Gittins' index for a $Log - N(\mu,{\rm{
}}{\sigma ^2})$ error distribution is
\begin {equation}\label{eq:7}
G(a,\hat{s}) = \frac{1}{{\hat s - a{e^{\mu  + g(a,\hat{s})}}}}, \end{equation}

\noindent where 
\begin {equation*}\label{eq:12}
g(a,\hat{s})= \frac{{{\sigma ^2}\phi[ \frac{{\ln
(\frac{{\hat s}}{a}) - \mu  - {\sigma ^2}}}{\sigma }]}}{{{2}\phi [\frac{{\ln
(\frac{{\hat s}}{a}) - \mu }}{\sigma }]}},
\end{equation*}
\noindent and $\phi $ is the cumulative distribution function of the $Log - N(0,{\rm{
}}{\sigma ^2})$ distribution. Note that for the Log-Normal distribution as the job processing time error distribution, the second case in \eqref{eq:6} cannot happen. For the remainder of the paper, we will refer to this policy as the Gittins' Index policy. We recognize that this is a slight abuse of terminology, as we are ignoring the job processing time distribution.

Taking the score in \eqref{eq:7} into account, for any job with an estimated processing time $\hat s$, the score calculated with the Gittins' Index policy continuously increases until the job completes. Fig. \ref{fig:Gittins} shows this score for a job with an estimated processing time of 20 and an estimation error generated from a $Log - N(0,{\rm{}}{\sigma ^2})$ distribution as a function of its elapsed processing time. We observe that for larger values of elapsed processing time, the slope of the score is decreasing. Fig. \ref{fig:NP} shows the score calculated with the SEH policy for a job with an estimated processing time of 20 as a function of its elapsed processing time. The score shown in Fig. \ref{fig:Gittins} is consistent with the score function having decreasing slope at some point beyond the point at which the elapsed processing time reaches the estimated processing time, as in Fig. \ref{fig:NP}. Of course, the change in slope for SEH is more severe, but we will see in our numerical experiments that the performance of the two policies is quite close. SEH has less computational overhead and more importantly, does not require knowledge of the estimation error distribution.



\begin{figure}[t]
\centering

\subfigure[calculated with the Gittins' Index policy]{
\includegraphics[width=0.47\textwidth]{Fig1a.pdf}\label{fig:Gittins}
}
\subfigure[calculated with SEH]{
\includegraphics[width=0.47\textwidth]{Fig1b.pdf} \label{fig:NP}
}
\caption{\label{fig:Gittins_VS_SEH} Job score as a function of the elapsed processing time } 
\end{figure}


