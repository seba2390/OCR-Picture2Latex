\begin{table}[!htbp] \centering 
\scriptsize
  \caption{Variable description and data.} 
  \label{tab:var_desc} 
\begin{tabular}{@{\extracolsep{5pt}}l L{6cm} llll} 
\\[-1.8ex]\hline 
\hline \\[-1.8ex] 
Name & \multicolumn{1}{c}{Description} & \multicolumn{1}{c}{Source} & \multicolumn{1}{c}{Year} & \multicolumn{1}{c}{N} & \multicolumn{1}{c}{Type} \\
\hline \\[-1.8ex] 
Population & Municipal population over 14 years old (only people over the age of 14 were allowed to participate in an ELA). & CENSO & 2017 & 345 & Count \\
\hline
Higher education & Share of municipal population who have successfully completed a higher education degree (advanced technician, bachelor, MSc, PhD.) & CENSO & 2017 & 345 & Continuous (0-1) \\
Internet penetration rate & Constructed as the interaction of two variables from CASEN survey: (i) share of households with at least one internet-connected device, (ii) number of different uses of Internet. & CASEN & 2015 & 324 & Continuous (0-1) \\
SUBDERE groups & Socio-demographic municipal classification based on the dependence on the municipal common fund and the local population. This typology divides municipalities in 7 groups plus an exception group with the highest income municipalities. Group 1 is the most vulnerable, i.e., municipalities in this group have less population and show higher dependency on the municipal common fund. & SUBDERE & 2005 & 335 &  Categorical \\
Poverty & Income poverty rate by municipality, based on income information from CASEN survey, using a method of Small Area Estimation (SAE). & INE & 2017 & 344 & Continuous (0-1) \\
SEDI & Socio-Economic Development Index, which comprises education, income, poverty, housing and sanitation. & OCHISAP & 2013 & 324 & Continuous (0-1) \\  
Community organizations & Number of community organizations in the municipality, such as parent centers, cultural centers, sport clubs, among others. & SINIM & 2015  & 342 &  Count \\
Participation in comm. org. & Share of municipal population that declares to participate in a community organization & CASEN & 2015  & 324 & Continuous (0-1)\\
\hline
Population density & Number of people living in the municipality, per square kilometer (km$^2$)  & SINIM & 2015 & 345 & Continuous\\
Rurality & Share of municipal population living in rural areas. A rural area is defined as an agglomeration with more than 1,000 inhabitants, or between 1,001 and 2,000 inhabitants where more than 50\% of the economically active population is engaged in primary economic activities.  & CENSO & 2017 & 345 & Continuous (0-1) \\
Born in 1981 or after & Share of municipal population born in 1980 or after, which represents the youngest cohort of our study, who became adults after the return to democracy. & CENSO & 2017 & 345 &  Continuous (0-1)\\
Women & Proportion of women in the municipality & CENSO & 2017 & 345 & Continuous (0-1) \\
Single-parent family with children & Share of single-parent families, with children, in the municipality. & CENSO & 2017 &  345 & Continuous (0-1) \\
Two-parent family with children & Share of two-parent families, with children, in the municipality. & CENSO & 2017 & 345 & Continuous (0-1) \\
\hline
Party affiliation & Share of municipal population affiliated to any political party. & SERVEL & 2016 & 345 & Continuous (0-1)\\
Voter turnout & Voter turnout in 2013 presidential elections at the municipal level. & SERVEL & 2013 & 345 & Continuous (0-1)\\
Votes for standing president & Share of votes received by the winning candidate in 2013 presidential elections by municipality. & SERVEL & 2013 & 345 & Continuous (0-1)\\
Municipal officials & Share of municipal population employed by the city hall. & SINIM & 2015 & 345 & Continuous (0-1) \\
Mayor & 3 dummy variables to take into account the political party which supported the winning candidate for mayor, during the 2012  municipal elections:  the first one is equal to 1 if the party is within the government coalition, and 0 otherwise; the second one is equal to one if the party is in the opposition, and 0 otherwise; the third dummy variable is assigned to 1 when the mayor ran for office with no formal party support. & SERVEL & 2012 & 345 & Categorical\\
Incumbent mayor & Dummy variable that takes the value 1 if an incumbent mayor is reelected, and 0 otherwise. & SERVEL & 2012 & 345 & Categorical \\
Government influence & Consists of the sum of: (i) the share of votes obtained by the pro-government deputies, relative to the total votes obtained by both elected deputies; (ii) 1, if the mayor was supported by the government coalition, and 0 otherwise. & SERVEL & 2012 & 345 & Continuous (0-2) \\
\hline \\[-1.8ex] 
Evangelical Christians & Share of municipal population who declared to profess an evangelical Christian religion. & CENSO & 2012 & 341 & Continuous (0-1) \\
\hline \\[-1.8ex] 
\multicolumn{6}{p{16cm}}{Notes: (i) The administrative division of Chile consists of 346 municipalities, from which we excluded the municipality of Antártica because of its special situation. (ii) Sources: Population and housing census (CENSO); Electoral Service (SERVEL); National municipal information system (SINIM);  National office for regional development (SUBDERE); National Socio-Economic Characterization Survey (CASEN); Public Health Observatory in Chile (OCHISAP); National Institute of Statistics (INE). (iii) CASEN survey lacks representativity at municipal level.}
\end{tabular} 
\end{table} 