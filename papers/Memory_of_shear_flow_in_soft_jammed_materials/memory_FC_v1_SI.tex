\documentclass[aps,preprint,nofootinbib]{revtex4-1}
%\documentclass[a4paper,10pt]{article}
\usepackage[utf8]{inputenc}
\usepackage{graphicx}
\usepackage{color}
\usepackage{epsfig}
\usepackage{textcomp}
\usepackage{float}
\usepackage{upgreek}
\usepackage{amsmath}
\usepackage{geometry}
%\usepackage[dvipdfm,colorlinks]{hyperref}
%\hypersetup{
%    pdftitle={},
%    pdfauthor={Ki-Joo Kim},
%    pdfkeywords={pdf, latex, tex, ps2pdf, dvipdfm, pdflatex},
%    bookmarksnumbered,
%    pdfstartview={FitH},
%    urlcolor=blue,
%    citecolor=magenta,
%}
\geometry{left=0.8in,right=0.8in,top=0.8in,bottom=0.8in}
\renewcommand{\thefigure}{S\arabic{figure}}
\begin{document}
%\maketitle 
\begin{center}
%\textbf{Supplementary Information}
\textbf{Memory of shear flow in jammed suspensions (Supplementary Material)}
\\
H. A. Vinutha, Manon Marchand, Marco Caggioni, Vishwas V. Vasisht, Emanuela Del Gado, and Veronique Trappe
\date{\today}
\end{center}

%\begin{abstract}

%Here we present additional data and supplemental information regarding the analysis presented in the paper, namely: (i) Raw stress relaxation data of a model jammed suspension from experiments \& simulations. (ii) Particle displacements along different directions. (iii) Dynamical correlations for different shear rates at $\zeta = 5 \tau_o \epsilon/a^2$, different runs with the same shear rate at $\zeta = 1 \tau_o \epsilon/a^2$ and unscaled data of the correlation length.  (iv) Unscaled data of fraction of icosahedrons in steady state and the time evolution of $F_{ICO}$ for two different damping coefficients. (v) Estimation of the volume fraction of the experimental system. 
%Scaling of the correlation length as a function of the distance from yield stress. 

%\end{abstract}
%\begin{enumerate}

\section{Characteristics of stress relaxation upon flow cessation}
\begin{figure*}[h!]
%\centering
\includegraphics[scale=0.3]{sigvstime_ss_SRs_expt_T50.pdf}
\includegraphics[scale=0.3]{sigvstime_ss_SRs_expt_T20.pdf}
\includegraphics[scale=0.29]{sigr_ss_SRs_etas_unscale_expt.pdf}

\includegraphics[scale=0.3]{sigvstime_ss_SRs_N1lk_Xi10.pdf}
\includegraphics[scale=0.29]{sigvstime_ss_SRs_Xi50_N1lk.pdf}
\includegraphics[scale=0.28]{sigr_ss_SRs_xis_unscale_N1e5.pdf}
\caption{\label{stress_relax} Same data as that shown in Fig. 1(b) and (d) in the paper, here graphed without normalization. The data from experiments and simulations are shown in respectively the top and bottom row. The first two rows displays the stress relaxation data obtained for different preshear rates and different solvent viscosities, respectively damping factors. The last row displays the preshear rate dependence of the residual stress for different solvent viscosities, respectively damping factors.}
\end{figure*}


%In Fig. \ref{stress_relax}, we show the evolution of stress and the residual stress for different shear rates and solvent viscosities, upon flow cessation, obtained from both experiments and simulations. The corresponding scaled plots are shown in Fig. 1(b)(c) of the paper. 

\section{Isotropic mean squared particle displacements}

As already shown in previous work \cite{mohan} the mean squared particle displacements appears to be isotropic. As an example we show in Fig. \ref{msdall} the mean squared displacement obtained along the flow $X$, the gradient $Y$ and the vorticity $Z$ direction for $\zeta = 1 \uptau_o \epsilon/a^2$ and $\dot{\gamma} = 10^{-2} \tau_o^{-1}$ . The different data sets are essentially indistinguishable. The corresponding total mean squared displacement is shown in Fig. 2(a) in the paper. 

As denoted in our work the mean particle displacement appears to be isotropic because of the coexistence of many domains of highly correlated displacements, whose average displacement vectors point in random directions.\\

\begin{figure*}[h!]
\centering
\includegraphics[scale=0.5]{msd_ss_SR0p001.pdf}
\caption{\label{msdall} Particle mean squared displacements (MSD) along the flow $X$, the gradient $Y$ and the vorticity $Z$ direction obtained for $\zeta = 1 \uptau_o \epsilon/a^2$ and $\dot{\gamma} = 10^{-2} \tau_o^{-1}$.}
\end{figure*}
 
\section{Variations in spatial correlations of particle dynamics}
\begin{figure*}[h!]
\includegraphics[scale=0.55]{spatcorel_mobil_SR1e-03_flowc_SS_runs}
\caption{\label{corel3} Spatial correlations of particle displacement fluctuations obtained for 3 independent runs with $\zeta = 1 \tau_o \epsilon/a^2$ and $\dot{\gamma} = 10^{-2} \tau_o^{-1}$. Data obtained during flow just before flow cessation are denoted as closed symbols, data obtained upon flow cessation are denoted as open symbols.} 
\end{figure*}
 Performing different runs of a given experiment reveals that the spatial correlations of particle displacement fluctuations $C$ somewhat varies at large distances from one run to another; this is shown for the example of $\zeta = 1 \tau_o \epsilon/a^2$ and $\dot{\gamma} = 10^{-2} \tau_o^{-1}$ in Fig. S3. The agreement of data acquired just before and after flow cessation, however, remains excellent, revealing that the flow cessation characteristics depend on the exact imprint of the preceding flow state. It is also noteworthy that the initial decay used for the determination of the correlation length is nicely reproducible.\\

\section{Structural evolution during stress relaxation}
\begin{figure*}[h!]
%\centering
\includegraphics[scale=0.42]{rhoICO_SRs_xis_unscale_N1e5.pdf}
\includegraphics[scale=0.435]{voronoi_F12_localq6_SRs_Xi10n50_S1.pdf}
\caption{\label{delfico} Left panel: Shear rate dependence of the fraction of icosahedral configurations obtained at steady flow conditions. This is the same data as that shown in the main graph of Fig. 4(a), here graphed without normalization. Right panel: Temporal evolution of $F_{ICO}$ upon flow cessation obtained for  $\zeta = 1$ and $5$  $\uptau_o \epsilon/a^2$. From top to bottom: the preshear rate is $\dot{\gamma} = 10^{-4}, 10^{-3}, 10^{-2}, 10^{-1}, 10^{0}$ $\uptau_o^{-1}$. The red and blue lines denote the data obtained for respectively $\zeta = 1$ and $5$  $\uptau_o \epsilon/a^2$.  The black horizontal line marks the magnitude of $F_{ICO}$ reached at the end of the stress relaxation process for the larger preshear conditions, for which we observe $F_{ICO}$ to evolve during stress relaxation.}
\end{figure*}
In the left panel of Fig. \ref{delfico}, we show the fraction of icosahedral configurations ($F_{ICO}$) obtained at steady flow conditions as a function of shear rate instead of the viscous stress, the latter being shown in Fig. 4(a) of the paper. An increase of $F_{ICO}$ during stress relaxation is only observed when the initial value of $F_{ICO} < 14 \%$, as shown for both damping factors investigated in the right panel of Fig. \ref{delfico}. Interestingly, in all these cases $F_{ICO} \approx 14 \%$ is reached at the end of stress relaxation. However, for the condition where the initial $F_{ICO} > 14 \%$ obtained for lower preshear rates, we find that some rearrangements of the icosahedron in space persist upon flow cessation even though $F_{ICO}$ does not increase. To demonstrate this we show the movies of the Voronoi analysis of the structural evolution obtained upon flow cessation after a preshear with respectively $\dot{\gamma} = 10^{-4}$ and $10^{0}$ $\uptau_o^{-1}$ in movie 1 and 2.


\section{Estimation of the volume fraction of the experimental system}\label{secA1}
\begin{figure*}[h!]
\includegraphics[scale=0.6]{Fig1_SI_Expt.pdf}
\caption{\label{figS5} Mapping procedure used to determine the volume fraction of the dispersion of $2 w \%$ Carbopol in propylene glycol. Inset: Concentration dependence of low frequency plateau modulus of Carbopol dispersions in propylene glycol with concentrations indicated in weight percent. The vertical line denotes the experimental system used in this work. Main Figure: Volume fraction dependence of plateau modulus of emulsions (black circles) published in Ref.\cite{mason}. The plateau modulus is here normalized by the ratio of the surface tension and the radius of the emulsion droplets. The red squares denote the data obtained for the Carbopol dispersions (inset) that have been normalized so to match the emulsion data. The vertical line denotes the concentration of the experimental system used in this work.}
\end{figure*}


To estimate the volume fraction $\phi$ of our experimental system, we prepare a series of Carbopol samples with different weight concentrations. For these samples we determine  
the frequency dependence of the storage and loss modulus as a function of frequency in oscillatory strain experiments using a strain amplitude within the linear range. From this data we extract the low frequency modulus $G_p$, which we find to exhibit a concentration dependence reminiscent of that observed by Mason et al. for an emulsion system \cite{mason}.  As shown in the inset of Fig. \ref{figS5}, $G_p$ increases strongly within a narrow range of concentration, to then increase only moderately in the range of larger concentration. The corresponding data set obtained for emulsions as a function of volume fraction is shown as black circles in the main Figure, where $G_p$ is here normalized by the Laplace pressure $\mathit{\Gamma}/R$ with $\mathit{\Gamma}$ the surface tension and $R$ the radius of the emulsion droplets. We estimate the volume fraction of our system by mapping our data to those obtained by Mason et al., normalizing both the modulus and the concentration so to match the emulsion data, as shown in the main Figure of Fig. \ref{figS5}. From this mapping procedure we estimate that the volume fraction of our system with a concentration of $2 w \%$ is about $74\%$  marked by a vertical line in Fig. S5.



%\vspace{10cm}

\begin{thebibliography}{100}
\bibitem{mohan}Mohan, L., Cloitre, M., \& Bonnecaze, R. T. (2015). Build-up and two-step relaxation of internal stress in jammed suspensions. Journal of Rheology, 59(1), 63-84.
\bibitem{mason}Mason, T. G., J. Bibette, and D. A. Weitz. "Elasticity of compressed emulsions." Physical review letters 75.10 (1995): 2051.
\end{thebibliography}

\end{document}


%\begin{figure*}[h!]
%\centering
%\includegraphics[scale=0.4]{icosa_corel_shear_SRs.pdf}
%\includegraphics[scale=0.4]{icosa_corel_shear_SRs_2.pdf}
%\caption{\label{gofr} Pair correlation function of particles with icosahedron neighborhood $g_{ICO}(r)$ shown for different rates during steady state (SS) and the corresponding final jammed solids after shear cessation (SC). The curves for different rates are shifted for the clarity. $g_{ICO}(r)$ is normalized by the full pair correlation function $g(r)$ to account for the density correlations.}
%\end{figure*}


%Mohan et al. \cite{mohan2015build}, associate the second process in the stress relaxation to particles getting trapped in cages formed by their neighbors and that the rate of stress relaxation is independent of preshear rate. Whereas, in our case, we observe that the particle motion is always co-operative, unlike in caging where the particle motion is uncorrelated, and that the pre-shear rate dependence enters due to the first process which is set by the preshear rate, see Fig.5(a). 

%Further, Mohan et al. \cite{mohan2015build}, indentify the anisotropic features in the pair correlation function as the origin of the residual stress but do not provide any interpretation for the scaling of residual stress with the distance from the yield stress. Here, we understand the residual stress scaling in terms of $F_{ICO}$ which are locally stable particle packing. 


%In Fig. \ref{Icosa}(a) of the paper shows that $F_{ICO}$ for high rates during the flow cessation evolve to the value closer to the lower rates. To quantify the evolution of structural correlations, we compute the pair correlation function of only particles with icosahedron neighborhood $g_{ICO}(r)$ normalized with the full $g(r)$ to account for the underlying density correlations. Similar to $F_{ICO}$, the pair correlation functions in steady state depend on the shear rate, see Fig. \ref{gofr}. For $\dot{\gamma}=10^{-5}$, there is no change in the pair correlation function in steady state and the final jammed state. For $\dot{\gamma}=10^{-1}$, during the resolidification, the system not only increases $F_{ICO}$ but also builds their correlation. This is clearly shown by the peaks developed in the final state but were absent in the steady state, see Fig. \ref{gofr}. The data suggests that for the system to regain elasticity there is an optimal way for the distribution of local stiffer regions. 



%\section{Correlation length - Avalanches}
%\begin{figure*}[h!]
%\centering
%\includegraphics[scale=0.55]{corelleng_mobil_Xi10_FC_exstress.pdf}
%\caption{\label{scaling2} Power-law scaling of the correlation length $\xi_{FLU}$  with the distance from the yield stress $\sigma - \sigma_y$. The yield stress value $\sigma_y=2.48 \epsilon/a^3$, which is obtained from quasi-static shear simulations.}
%\end{figure*}
%The correlation length ($\xi_{FLU}$) obtained from the correlation function of displacement fluctuations $C_{FLU}$ can be related to the size of avalanches as depicted in elasto-plastic models cite[1,3,23-25]. In the elasto-plastic models, the flow in amorphous solids is due to an avalanche of plastic events and the avalanche statistics can be linked to the macroscopic rheological behaviour. 
%The idea is that in the limit $\dot{\gamma} \rightarrow 0$ or $(\sigma - \sigma_y) \rightarrow 0$, the spatial extent of the avalanches is infinite and as the shear rate increases, the size of avalanches decreases but with many avalanches occurring simultaneously. According to the elasto-plastic studies, the characteristic length scale $\xi$ associated with the avalanche dynamics obeys the relation $\xi \sim (\sigma -\sigma_y)^{-\nu}$, which can be rewritten as $\xi \sim \dot{\gamma}^{-\nu/\beta}$, where $\beta=1/n$. From our simulations, the value of exponent $\nu/\beta \approx 0.2$, see Fig. 3(b) and the value of exponent $\nu \approx 0.44$. Our estimates of the exponents are in good agreement with the exponent values reported in the previous simulations studies $\nu/\beta = 0.32 \pm 0.06, \nu = 0.48 \pm 0.05, \beta = 1.5 \pm 0.05$ cite[1-3]. From the elasto-plastic models, the exponents values are  $\nu/\beta = 0.5, \nu = 0.7, \beta = 1.38$ cite[3].

%---

%To the best of our knowledge, this is a first direct measurement of the rate dependent correlation length associated with the avalanching behaviour.
%The size of the correlated domains, set by the shear rate, determines the amount of plasticity leading to stress relaxation, where the plastic events occur at the boundaries of the correlated domains.  As a result more stress is relaxed at the higher rates compared to the lower rates and the stress relaxation time scales as $\dot{\gamma}$, as denoted by the initial scaling of the relaxation data in Fig. \ref{Fig1_expt}(b). 
%In the paper, we present the data for residual stress $\sigma_{residual}$, correlation length $\xi_{FLU}$ and $F_{ICO}$ as a function of shear rate. Here, we present the data as a function of distance from the yield stress or excess stress. The yield stress value $\sigma_y=2.48 \epsilon/a^3$ is from quasi-static shear simulations, that is, in the limit $\dot{\gamma} \rightarrow 0$. In Fig. \ref{stress_relax}(b) shows that the residual stress decays more smoothly as a function of $\dot{\gamma}$.
%In the low range of shear rates the residual stress remains on the order of the yield stress followed by a weak power-law decrease. We could also represent the data in terms of distance from the yield stress $\sigma - \sigma_y$. The closer the initial stress value is to the yield stress value, the stress relaxation is least and lead to a higher residual stress. Further it is from $\sigma_y$ lower is the residual stress. Hence, the residual stress contains the information about the distance from the yield stress.
%The data in Fig. \ref{scaling2}(a) obtained at different damping factors collapse onto a unique curve implies that the residual stresses depend only on the distance from the yield stress and consistent with Ref. \cite{mohan2015build}. 

%In Fig. \ref{scaling2}(b), we show that $\xi_{FLU}$ scales with $\sigma-\sigma_y$ with an exponent value $-0.44$. We know from the elasto-plastic studies, the characteristic length scale $\xi$ associated with the avalanche dynamics obeys the relation $\xi \sim (\sigma -\sigma_y)^{-\nu}$. From our simulations, the value of exponent $\nu \approx 0.44$ and $\nu/\beta \approx 0.2$. Our estimate of the exponents are consistent with previous simulations studies $\nu = 0.48 \pm 0.05, \nu/\beta = 0.32 \pm 0.06$ \cite{clemmer2021criticality1,clemmer2021criticality}. The predictions from the elasto-plastic models for $ \nu = 0.7, \nu/\beta = 0.5$ \cite{lin2014scaling}.

 %We show the spatial correlations in the displacement fluctuations for different shear rates and $\zeta = 5 \uptau_o \epsilon/a^2$ in the left panel of Fig. \ref{corel2}. Consistent with the data of $\zeta = 1 \uptau_o \epsilon/a^2$, we observe an excellent agreement of the data obtained before and after shear cessation. The right panel of Fig. \ref{corel2} shows the unscaled data of the correlation lengths for two different damping coefficients. The corrsponding scaled data is shown in Fig. 3(b) of the paper. 

 %which identifies two distinct processes in the stress relaxation. %The simulation data of the initial relaxation fits to a stretched exponential and the second process follows a power law or logarithmic relaxation. 
%The two-step stress relaxation reported in Ref. cite{mohan2015build} on jammed suspension of microgels show an initial rapid decay and the associated relaxation timescale scales as $\dot{\gamma}^{-0.71}$ and the long-time relaxation is exponential and its timescale is independent of preshear rate cite{mohan2015build}. Whereas in our case, the relaxation timescale associated with the initial process scales as $\dot{\gamma}^{-1}$, the difference in the exponent value reports in Ref. cite{mohan2015build} may be due to the smaller system size in their simulations. %The simulations are performed till the force balance is reached, whereas, in Ref. \cite{mohan2015build} it would take a very long time for the system to reach equilibrium and therefore the time estimate for completing the long-time relaxation is not available.

%\begin{figure*}[h!]
%\centering
%\includegraphics[scale=0.44]{spatcorel_mobil_Xi50_SRs_flowc.pdf}
%\includegraphics[scale=0.4]{corelleng_mobil_N1lk_Xi10n50.pdf}
%\caption{\label{corel2} Dynamical correlations just before (closed symbols) and after shear cessation (open symbols) obtained for $\zeta = 5 \uptau_o \epsilon/a^2$, shown for different shear rates (left panel) and unscaled correlation length data (right panel).}
%\end{figure*}