%
%%%%%%%%%%%%%%%%%%%%%%%%%%%
\section{Introduction}
\label{Sec: Introduction}
%%%%%%%%%%%%%%%%%%%%%%%%%%%
%

Fractional models construct a tractable mathematical framework to describe and predict the behavior of multi-scales multi-physics complex phenomena. Particularly, fractional differential equations, as a well-structured generalization of their integer order counterparts, provide a rigorous mathematical tool to develop models, describing anomalous behavior in complex physical systems \cite{zhang2017review,jaishankar2014,jha2003evidence, Castillo2004Plasma,jaishankar2013,naghibolhosseini2015estimation,naghibolhosseini2017fractional,meer01, meral2010fractional}, where the anomalies manifest in heavy tailed distribution of corresponding underlying stochastic processes, moreover, exhibiting sharp peaks, intermittency, and asymmetry in the underlying distributions. Significant approximations as inherent part of assumptions upon which the model is built, lack of information about true values of parameters (incomplete data), and random nature of quantities being modeled pervade uncertainty in the corresponding mathematical formulations \cite{cullen1999probabilistic,roy2011comprehensive}. In this work, we develop an uncertainty quantification (UQ) framework in the context of stochastic fractional partial differential equations (SFPDEs), in which we characterize different sources of uncertainties and further propagate the associated randomness to the system response quantity of interest (QoI).


\vspace{0.1 in}
%
%%%%%%%%%%%%%%%%%%%%%%%%%%%
\noindent\textbf{Types and Sources of Uncertainty}.
%%%%%%%%%%%%%%%%%%%%%%%%%%%
%
The model uncertainties are in general being classified as aleatory, epistemic, and mixed, according to their fundamental essence. They can also be broadly characterized as occurring in model inputs, numerical approximations, and model form. Model inputs encompass all model parameters coming from geometry, constitutive laws, and fields equation, while also pertaining surrounding interactions, such as boundary conditions and random forcing sources (noise). Numerical approximations, which are an essence of differential equations since they generally do not lend themselves to analytical solutions, introduce uncertainty by imposing different sources of discretization error, iterative convergence error, and round off error. The fractional derivatives introduce derivative orders, namely fractional indices, as new set of model parameters in addition to model coefficients. These new parameters are strongly tied to the distribution of underlying stochastic process and their statistics are estimated from experimental observations in practice, see e.g.  \cite{benson2000application,baeumer2001subordinated}. 




\vspace{0.1 in}
%
%%%%%%%%%%%%%%%%%%%%%%%%%%%
\noindent\textbf{Uncertainty Framework}.
%%%%%%%%%%%%%%%%%%%%%%%%%%%
%
Conventional approaches in parametric UQ of differential equations is based around Monte Carlo sampling (MCS) \cite{carlo1996concepts}, which performs ensemble of forward calculations to map the uncertain input space to the uncertain output space.
This method enjoys from being embarrassingly parallelizable and can be implement quite readily on high dimensional random spaces. However, the key issue is the slow rate of convergence $\sim 1/\sqrt{N}$ with $N$ number of realization, which consequently imposes exhaustively so many operations of forward solver, makes it not practical for expensive simulations. Other methods such as sequential MCS \cite{del2006sequential} and multilevel sequential MSC \cite{beskos2017multilevel} are also developed and recently used in \cite{jasra2016forward} to improve the parametric uncertainty assessment in elliptic nonlocal equations. An alternative to expensive MCS is to build surrogate models. An extensive comparison of two widely used ones, namely polynomial chaos and Gaussian process, are provided in a recent work \cite{owen2017comparison}. Polynomial chaos, in which the output of stochastic model is represented as a series expansion of input parameters, was initially applied in \cite{ghanem2003stochastic}, and later extended and used in \cite{xiu2002modeling,xiu2002wiener,knio2006uncertainty}. It is also generalized and used in constructing stochastic Galerkin methods \cite{babuska2004galerkin,babuvska2005solving,le2004uncertainty,le2004multi} for problems with higher-dimensional random spaces. Other non-sampling numerical methods, including but not limited to perturbation method \cite{schuss1980singular,babuvska2002solving,todor2005sparse,winter2002groundwater} and moment equation method \cite{liu1986probabilistic,liu1986random} are also developed, however their applications are restricted to stochastic systems with relatively low-dimensional random space. These so-called ``\textit{intrusive}" approaches typically do not treat the forward solver as a black-box, rather require some knowledge and reformulation of the governing equations and thus, may not be practical in many problems with complex codes.
%; therefore, a ``non-intrusive" approach may be more practical.

A wide range of ``\textit{non-intrusive}" techniques mostly stretch over sampling, quadrature, and regression, see \cite{owen2017comparison} and references therein. More recently, high-order probabilistic collocation methods (PCM), employing the idea of interpolation/collocation in the random spaces, are developed in \cite{xiu2005high,babuvska2007stochastic,nobile2008sparse}. These methods limit the sample points to an efficient subset of random space, while adequately sampling the necessary range. The excellence in use of PCM is twofold; it has the benefit of easily sampling at nodal points that naturally leads to independent realizations of the problem as in MCS, and the advantage of fast convergence rate. The challenging post processing of solution statistics, which can essentially be described as a high-dimensional integration problem, can also be resolved by adopting sparse grid generators, such as Smolyak algorithm \cite{nobile2008sparse,Smolyak63}. The use of sparse grids, as opposed to full tensor product construction from one-dimensional quadrature rules, will effectively reduce the number of sampling, while preserving a fast convergence rate to high level of accuracy. %Utilizing domain decomposition tools, PCM is also extended to multi-element PCM \cite{Foo08}. 






\vspace{0.1 in}
%
%%%%%%%%%%%%%%%%%%%%%%%%%%%
\noindent\textbf{Forward Solver}.
%%%%%%%%%%%%%%%%%%%%%%%%%%%
%
A core task in computational forward UQ is to form an efficient numerical method, which for each realizations of random variables can accurately solves and simulates the deterministic counterpart of stochastic model in the physical domain. Such numerical method is usually called ``\textit{forward solver}" or ``\textit{simulator}". In the case of FPDEs, the excessive cost of numerical approximations due to the inherent nonlocal nature of fractional differential equations additionally become more challenging when generally most of uncertainty propagation techniques instruct operations of forward solver many times. This requires implementation of more efficient numerical schemes, which can manage increasing computational costs while maintaining sufficiently low error level in mitigating the corresponding uncertainties. In addition to numerous finite difference methods for solving FPDEs \cite{Gorenflo2002, Sun2006,  Lin2007, wang2010direct, wang2011fast, Cao2013, zeng2015numerical, zayernouri2016_JCP_Frac_AB_AM}, recent works have elaborated efficient spectral schemes, for discretizing FPDEs in physical domain, see e.g., \cite{Rawashdeh2006, Lin2007, Khader2011, Khader2012, Li2009, Li2010, chen2014generalized, wang2015high, bhrawy2015spectral}. More recently, Zayernouri et al. \cite{Zayernouri2013, zayernouri2015tempered} developed two new spectral theories on fractional and tempered fractional Sturm-Liouville problems, and introduced explicit corresponding eigenfunctions, namely \textit{Jacobi poly-fractonomials} of first and second kind. These eignefunctions are comprised of smooth and fractional parts, where the latter can be tunned to capture singularities of true solution. They are successfully employed in constructing discrete solution/test function spaces and developing a series of high-order and efficient Petrov-Galerkin spectral methods, see \cite{lischke2017petrov, suzuki2016fractional,samiee2016,samiee2017fast,samiee2017unified,samiee2018petrov,kharazmi2017petrov,kharazmi2017sem,kharazmi2018fractional}.



\vspace{0.1 in}

The main focus of this work is to develop an operator-based computational forward UQ framework in the context of stochastic fractional partial differential equation. Assuming that the mathematical model under consideration is well-posed and accounts in principle for all features of underlying phenomena, we identify three main sources of uncertainty, \textit{i}) parametric uncertainty, including fractional indices as new set of random parameters appeared in the operator, \textit{ii}) additive noises, which incorporates all intrinsic/extrinsic unknown forcing sources as lumped random inputs, and \textit{iii}) numerical approximations. Computational challenges arise when the admissible space of random inputs is infinite-dimensional, e.g. problems subject to additive noise \cite{rizzi2012uncertainty}, and thus, the framework involves uncertainty parametrization via a finite number of random space basis. Unlike the classical approach in modeling random inputs, which considers idealized uncorrelated processes (white noises), we model the random inputs as more/fully correlated random processes (colored noises), and parametrize them via Karhunen-Lo\`{e}ve (KL) expansion by assuming finite-dimensional noise assumption. This yields the problem in finite dimensional random space. We then, propagate the parametric uncertainties into the system response by applying PCM. We obtain the corresponding deterministic FPDE for each realization of random variables, using the Smolyak sparse grid generators for low to moderately high dimensions. In order to formulate the forward solver, we follow \cite{samiee2016} and develop a high-order Petrov-Galerkin (PG) spectral method to solve for each realization of SFPDE in the physical domain, employing Jacobi poly-fractonomials in addition to Legendre polynomials as temporal and spatial basis/test functions, respectively. The smart choice of coefficients in construction of spatial basis/test functions yields symmetric properties in the resulting mass/stiffness matrix, which is then exploited to formulate an efficient fast solver. We also show that for each realization of random variables, the deterministic problem is mathematically well-posed and the proposed forward solver is stable. By adopting sufficient number of basis in the physical domain, we eliminate the epistemic uncertainty associated with numerical approximation and isolate the impact of parametric uncertainty on system response QoI.


%\vspace{0.1 in}

The organization of the paper is as follows. We recall some preliminaries on fractional calculus in section \ref{Sec: Fractional Calculus}. Then, we formulate the stochastic system in section \ref{Sec: Uncertainty framework}, and parametrize the random inputs. We also develop the stochastic sampling, namely PCM and MCS for our stochastic problem. We further construct the deterministic solver in section \ref{Sec: Forward Solver}, and provide the numerical results in section \ref{Sec: numerical results}. We end the paper with a conclusion and summary.



