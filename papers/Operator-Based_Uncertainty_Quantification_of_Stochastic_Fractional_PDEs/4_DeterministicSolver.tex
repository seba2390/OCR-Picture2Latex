%
%%%%%%%%%%%%%%%%%%%%%%%%%%%
\section{Forward Solver}
\label{Sec: Forward Solver}
%%%%%%%%%%%%%%%%%%%%%%%%%%%
%
For each realization of random variables in the employed sampling methods, the stochastic model yields a deterministic FPDE, left to be solved in the physical domain. We recall that for every $\boldsymbol{\xi}_i, \, i=1,2,\cdots$ in SFPDE \eqref{Doob_momentum-2}, the deterministic problem is recast as:
%
\begin{align}
\label{Eq: Deterministic FPDE}
%
\prescript{}{0}{\mathcal{D}}_{t}^{\alpha} u(t,\textbf{x}) 
& - \sum_{j=1}^{d} \, k_{j} \,\left[ \prescript{}{a_j}{\mathcal{D}}_{x_j}^{\beta_j}
+ \prescript{}{x_j}{\mathcal{D}}_{b_j}^{\beta_j} \right]
u(t,\textbf{x}) 
= h(t,\textbf{x}) +  f(t) ,
%
%\\
%\label{Eq: Deterministic IC}
%& u \arrowvert_{t=0} = 0 ,  
%%
%\\
%\label{Eq: Deterministic BC}
%& u \arrowvert_{x=a_j} = u \arrowvert_{x=b_j} = 0 , 
%
\end{align}
%
subject to the same initial/boundary conditions as \eqref{Eq: IC} and \eqref{Eq: BC}. In the sequel, we develop a Petrov-Galerkin spectral method to numerically solve the deterministic problem in the physical domain. We also show the wellposedness of deterministic problem in a weak sense and further investigate the stability of proposed numerical scheme.



%
%%%%%%%%%%%%%%%%%%%%%%%%%%%%%
\subsection{Mathematical Framework}
\label{Sec: Sol Test Space FPDE}
%%%%%%%%%%%%%%%%%%%%%%%%%%%%%
%
We define the useful functional spaces and their associated norms \cite{kharazmi2017petrov,Li2009}. By $H^\sigma(\mathbb{R}) = \big{\{} u(t) \vert u \in L^{2}(\mathbb{R});\, (1+\vert \omega \vert^2)^{\frac{\sigma}{2}} F(u)(\omega) \in L^{2}(\mathbb{R}) \big{\}}$, $\sigma \geq 0$, we denote the fractional Sobolev space on $\mathbb{R}$, endowed with norm $\Vert u \Vert_{H^{\sigma}_{\mathbb{R}}}=\Vert (1+\vert \omega \vert^2)^{\frac{\sigma}{2}} F(u)(\omega) \Vert_{L^{2}(\mathbb{R})}$, where $\mathcal{F}(u)$ represents the Fourier transform of $u$. Subsequently, we denote by $H^\sigma(\Lambda) =\big{\{} u\in L^{2}(\Lambda)\, \vert \,\exists \, \tilde{u} \in H^{\sigma}(\mathbb{R})\, \, s.t. \, \,\tilde{u}\vert_{\Lambda}=u \big{\}}$, $\sigma \geq 0$, the fractional Sobolev space on any finite closed interval, e.g. $\Lambda = (a,b)$, with norm $ \Vert u \Vert_{H^{\sigma}(\Lambda)}= \underset{\tilde{u}\in H^{\sigma}_{\mathbb{R}},\, \tilde{u}\vert_{\Lambda}=u }{\inf} \, \Vert \tilde{u} \Vert_{{}H^{\sigma}(\mathbb{R})}$. We define the following useful norms as:
%
\begin{align*}
%
&\Vert \cdot \Vert_{{^l}H^{\sigma}_{}(\Lambda)} = \Big(\Vert \prescript{}{a}{\mathcal{D}}_{x}^{\sigma}\, (\cdot)\Vert_{L^2(\Lambda)}^2+\Vert \cdot \Vert_{L^2(\Lambda)}^2 \Big)^{\frac{1}{2}},
\\
&\Vert \cdot \Vert_{{^r}H^{\sigma}_{}(\Lambda)} = \Big(\Vert \prescript{}{x}{\mathcal{D}}_{b}^{\sigma}\, (\cdot)\Vert_{L^2(\Lambda)}^2+\Vert \cdot \Vert_{L^2(\Lambda)}^2 \Big)^{\frac{1}{2}},
\\
&\Vert \cdot \Vert_{{^c}H^{\sigma}_{}(\Lambda)} = \Big(\Vert \prescript{}{x}{\mathcal{D}}_{b}^{\sigma}\, (\cdot)\Vert_{L^2(\Lambda)}^2+\Vert \prescript{}{a}{\mathcal{D}}_{x}^{\sigma}\, (\cdot)\Vert_{L^2(\Lambda)}^2+\Vert \cdot \Vert_{L^2(\Lambda)}^2 \Big)^{\frac{1}{2}},
%
\end{align*}
%
where the equivalence of $\Vert \cdot \Vert_{{^l}H^{\sigma}_{}(\Lambda)}$ and $\Vert \cdot \Vert_{{^r}H^{\sigma}_{}(\Lambda)}$ are shown in \cite{Li2009,ervin2007variational,Li2010}. 
%
\begin{lem}
	%
	\label{Lem: norm equivalence 1}
%	\cite{kharazmi2017FSEM}:
	Let $\sigma \geq 0$ and $\sigma \neq n-\frac{1}{2}$. Then, the norms $\Vert \cdot \Vert_{{^l}H^{\sigma}_{}(\Lambda)}$ and $\Vert \cdot \Vert_{{^r}H^{\sigma}_{}(\Lambda)}$ are equivalent to $\Vert \cdot \Vert_{{^c}H^{\sigma}_{}(\Lambda)}$.
	%
\end{lem}
%


\noindent We also define $C^{\infty}_{0}(\Lambda)$ as the space of smooth functions with compact support in $(a,b)$. We denote by $\prescript{l}{}H^{\sigma}_0(\Lambda)$, $\prescript{r}{}H^{\sigma}_0(\Lambda)$, and $\prescript{c}{}H^{\sigma}_0(\Lambda)$ as the closure of $C^{\infty}_{0}(\Lambda)$ with respect to the norms $\Vert \cdot \Vert_{{^l}H^{\sigma}_{}(\Lambda)}$, $\Vert \cdot \Vert_{{^r}H^{\sigma}_{}(\Lambda)}$, and $\Vert \cdot \Vert_{{^c}H^{\sigma}_{}(\Lambda)}$. It is shown in \cite{ervin2007variational,Li2010} that these Sobolev spaces are equal and their seminorms are also equivalent to $\vert \cdot \vert_{{}H^{\sigma}_{}(\Lambda)}^{*} = \big\vert \left( \prescript{}{a}{\mathcal{D}}_{x}^{\sigma}\, (\cdot), \prescript{}{x}{\mathcal{D}}_{b}^{\sigma}\, (\cdot) \right) \big\vert_{\Lambda}^{\frac{1}{2}}$. Therefore, we can prove that $\big{\vert}(\prescript{}{a}{\mathcal{D}}_{x}^{\sigma}\, u, \prescript{}{x}{\mathcal{D}}_{b}^{\sigma}\, v )_{\Lambda}^{} \big{\vert} \geq \beta \, \vert u \vert_{{^l}H^{\sigma}_{}(\Lambda)}\, \vert  v \vert_{{^r}H^{\sigma}_{}(\Lambda)}$ and $\big{\vert}(\prescript{}{x}{\mathcal{D}}_{b}^{\sigma}\, u, \prescript{}{a}{\mathcal{D}}_{x}^{\sigma}\, v )_{\Lambda}^{} \big{\vert} \geq \beta \, \vert u \vert_{{^r}H^{\sigma}_{}(\Lambda)}\, \vert  v \vert_{{^l}H^{\sigma}_{}(\Lambda)}$, in which $\beta$ is a positive constant. 



Moreover, by letting ${_0}C^{\infty}(I)$ and $C^{\infty}_{0}(I)$ be the space of smooth functions with compact support in $(0,T]$ and $[0,T)$, respectively, we define $\prescript{l}{}H^{s}(I)$ and $\prescript{r}{}H^{s}(I)$ as the closure of ${_0}C^{\infty}(I)$ and $C^{\infty}_{0}(I)$ with respect to the norms $\Vert \cdot \Vert_{\prescript{l}{}H^{s}(I)}$ and $\Vert \cdot \Vert_{\prescript{r}{}H^{s}(I)}$. Other equivalent useful semi-norms associated with $H^s(I)$ are also introduced in \cite{Li2009,ervin2007variational}, as $\vert \cdot \vert_{{^l}H^{s}(I)} = \Vert \prescript{}{0}{\mathcal{D}}_{t}^{s} (\cdot)\Vert_{L^2(I)}$, $ \vert \cdot \vert_{{^r}H^{s}(I)} = \Vert \prescript{}{t}{\mathcal{D}}_{T}^{s} (\cdot) \Vert_{L^2(I)}$, $\vert \cdot \vert_{{}H^{s}(I)}^* = \big\vert \left( \prescript{}{0}{\mathcal{D}}_{t}^{s}(\cdot),\prescript{}{t}{\mathcal{D}}_{T}^{s}(\cdot) \right)_{I} \big\vert^{\frac{1}{2}}$, where $\vert \cdot \vert_{{}H^{s}(I)}^* \equiv \vert \cdot \vert_{{^l}H^{s}(I)}^{\frac{1}{2}} \, \vert \cdot \vert_{{^r}H^{s}(I)}^{\frac{1}{2}}$. 



Borrowing definitions from \cite{samiee2016}, we define the following spaces, which we use later in construction of corresponding solution and test spaces of our problem. Thus, by letting $\Lambda_1 = (a_1,b_1)$, $\Lambda_j = (a_j,b_j) \times \Lambda_{j-1}$ for $j=2,\cdots,d$, we define $\mathcal{X}_1 = H^{\frac{\beta_1}{2}}_{0}(\Lambda_1)$, which is associated with the norm $ \Vert \cdot \Vert_{{^c}H^{\frac{\beta_1}{2}}_{}(\Lambda_1)}$, and accordingly, $\mathcal{X}_j, \, j=2,\cdots,d$ as
%
\begin{eqnarray}
%
\mathcal{X}_2 &=& H^{\frac{\beta_2}{2}}_0 \Big((a_2,b_2); L^2(\Lambda_1) \Big) \cap L^2((a_2,b_2); \mathcal{X}_1),
\\
&\vdots&
\nonumber
\\
\mathcal{X}_d &=& H^{\frac{\beta_d}{2}}_0 \Big((a_d,b_d); L^2(\Lambda_{d-1}) \Big) \cap L^2((a_d,b_d); \mathcal{X}_{d-1}),
%
\end{eqnarray}
%
associated with norms $\Vert \cdot \Vert_{\mathcal{X}_j} = \bigg{\{} \Vert \cdot \Vert_{H^{\frac{\beta_j}{2}}_0 \Big((a_j,b_j); L^2(\Lambda_{j-1}) \Big)}^2 + \Vert \cdot \Vert_{ L^2\Big((a_j,b_j); \mathcal{X}_{j-1}\Big)}^2 \bigg{\}}^{\frac{1}{2}}, \,\, j=2,3,\cdots,d$.

\vspace{0.1 in}
%
\begin{lem}
	%
	\label{space norm 1}
%	\cite{kharazmi2017FSEM}:
	Let $\sigma \geq 0$ and $\sigma \neq n-\frac{1}{2}$. Then, for $j=1,2,\cdots,d$ 
	%
	\begin{align*}
	%
	%	\label{norm_Xd_2}
	\Vert \cdot \Vert_{\mathcal{X}_j} \equiv \bigg{\{}  \sum_{i=1}^{j} \Big(\Vert \prescript{}{x_i}{\mathcal{D}}_{b_i}^{\beta_i/2}\, (\cdot)\Vert_{L^2(\Lambda_j)}^2+\Vert \prescript{}{a_i}{\mathcal{D}}_{x_i}^{\beta_i/2}\, (\cdot)\Vert_{L^2(\Lambda_j)}^2 \Big) + \Vert  \cdot \Vert_{L^2(\Lambda_j)}^2 \bigg{\}}^{\frac{1}{2}}.
	%
	\end{align*}
	%
\end{lem}
%




%
%%%%%%%%%%%%%%%%%%%%%%%%%%%%%
\subsection*{\textbf{Solution and Test Spaces}}
%%%%%%%%%%%%%%%%%%%%%%%%%%%%%
%
We define the ``solution space" $U$ and ``test space" $V$, respectively, as
%
\begin{align}
%
\label{Eq: solution test space}
U  = \prescript{l}{0}H^{\frac{\alpha}{2}}\Big(I; L^2(\Lambda_d) \Big) \cap L^2(I; \mathcal{X}_d),
%
\quad%
V  = \prescript{r}{0}H^{\frac{\alpha}{2}}\Big(I; L^2(\Lambda_d)\Big) \cap L^2(I; \mathcal{X}_d),
%
\end{align}
%
endowed with norms
%
\begin{align}
%
\label{Eq: solution test space norm}
\Vert u \Vert_{U} &= 
\Big{\{}\Vert u \Vert_{\prescript{l}{}H^{\frac{\alpha}{2}}(I; L^2(\Lambda_d))}^2 + \Vert u \Vert_{L^2(I; \mathcal{X}_d)}^2 \Big{\}}^{\frac{1}{2}}, 
%
\quad
%
\Vert v \Vert_{V} &= 
\Big{\{}\Vert v \Vert_{\prescript{r}{}H^{\tau}(I; L^2(\Lambda_d))}^2  + \Vert v \Vert_{ L^2(I; \mathcal{X}_d)}^2 \Big{\}}^{\frac{1}{2}},
%
\end{align}
%
where $I = [0,T]$, and 
%
\begin{align*}
%
\prescript{l}{0}H^{\frac{\alpha}{2}} \Big(I; L^2(\Lambda_d) \Big) = 
\Big{\{} u \, \big|\, \Vert u(t,\cdot) \Vert_{L^2(\Lambda_d)} \in H^{\frac{\alpha}{2}}(I), u\vert_{t=0}=u\vert_{x=a_j}=u\vert_{x=b_j}=0,\, j=1,2,\cdots,d  \Big{\}},
%
\\
\prescript{r}{0}H^{\frac{\alpha}{2}} \Big(I; L^2(\Lambda_d) \Big) = 
\Big{\{} v \,\big|\, \Vert v(t,\cdot) \Vert_{L^2(\Lambda_d)} \in H^{\frac{\alpha}{2}}(I), v\vert_{t=T}=v\vert_{x=a_j}=v\vert_{x=b_j}=0,\, j =1,2,\cdots,d  \Big{\}},
%
\end{align*}
%
equipped with norms $\Vert u \Vert_{\prescript{l}{}H^{\frac{\alpha}{2}}(I; L^2(\Lambda_d))}$ and $\Vert u \Vert_{\prescript{r}{}H^{\frac{\alpha}{2}}(I; L^2(\Lambda_d))}$, respectively. We can show that these norms take the following forms
%
\begin{align}
%
%\label{norm_222}
\label{space norm 2}
&
\Vert u \Vert_{\prescript{l}{}H^{\frac{\alpha}{2}}(I; L^2(\Lambda_d))} = \Big{\Vert} \, \Vert u(t,\cdot) \Vert_{L^2(\Lambda_d)}\, \Big{\Vert}_{{^l}H^{\frac{\alpha}{2}}(I)}
=\Big(\Vert \prescript{}{0}{\mathcal{D}}_{t}^{\frac{\alpha}{2}}\, (u)\Vert_{L^2(\Omega)}^2 + \Vert u\Vert_{L^2(\Omega)}^2 \Big)^{\frac{1}{2}},
%
\\
&\Vert u \Vert_{\prescript{r}{}H^{\frac{\alpha}{2}}(I; L^2(\Lambda_d))} = \Big{\Vert} \, \Vert u(t,\cdot) \Vert_{L^2(\Lambda_d)}\, \Big{\Vert}_{{^r}H^{\frac{\alpha}{2}}(I)}
= \Big(\Vert \prescript{}{t}{\mathcal{D}}_{T}^{\frac{\alpha}{2}}\, (u)\Vert_{L^2(\Omega)}^2+\Vert u\Vert_{L^2(\Omega)}^2\Big)^{\frac{1}{2}}.
%
\end{align}
%
Also, using Lemma \ref{space norm 1}, we can show that
%
\begin{align}
%
%\label{norm_2221}
\label{space norm 3}
\Vert u \Vert_{L^2(I; \mathcal{X}_d)}
=
\Big{\Vert} \, \Vert u(t,.) \Vert_{\mathcal{X}_d}\,\Big{\Vert}_{L^2(I)}
=
\Big{\{}  \Vert u \Vert_{L^2(\Omega)}^2 + \sum_{j=1}^{d} \big( \Vert \prescript{}{x_j}{\mathcal{D}}_{b_j}^{\frac{\beta_j}{2}}\, (u)\Vert_{L^2(\Omega)}^2 
+ \Vert \prescript{}{a_j}{\mathcal{D}}_{x_j}^{\frac{\beta_j}{2}}\, (u)\Vert_{L^2(\Omega)}^2 \big) 
\Big{\}}^{\frac{1}{2}}.
%
\end{align}
%
Therefore, \eqref{Eq: solution test space norm} can be written as
%
\begin{align}
%
\label{Eq: solution space norm}
\Vert u \Vert_{U} 
&= 
\Big{\{}  \Vert u \Vert_{L^2(\Omega)}^2 + \Vert \prescript{}{0}{\mathcal{D}}_{t}^{\frac{\alpha}{2}}\, (u)\Vert_{L^2(\Omega)}^2 
+ \sum_{j=1}^{d} \big( \Vert \prescript{}{x_j}{\mathcal{D}}_{b_j}^{\frac{\beta_j}{2}}\, (u)\Vert_{L^2(\Omega)}^2+\Vert \prescript{}{a_j}{\mathcal{D}}_{x_j}^{\frac{\beta_j}{2}}\, (u)\Vert_{L^2(\Omega)}^2 \big) \Big{\}}^{\frac{1}{2}}, 
%
\\
\label{Eq: test space norm}
%
\Vert v \Vert_{V} 
& =
\Big{\{}  \Vert v \Vert_{L^2(\Omega)}^2 + \Vert \prescript{}{t}{\mathcal{D}}_{T}^{\frac{\alpha}{2}}\, (v)\Vert_{L^2(\Omega)}^2 
+ \sum_{j=1}^{d} \big( \Vert \prescript{}{x_j}{\mathcal{D}}_{b_j}^{\frac{\beta_j}{2}}\, (v)\Vert_{L^2(\Omega)}^2
+\Vert \prescript{}{a_j}{\mathcal{D}}_{x_j}^{\frac{\beta_j}{2}}\, (v)\Vert_{L^2(\Omega)}^2 \big) \Big{\}}^{\frac{1}{2}}.
%
\end{align}
%















%
%%%%%%%%%%%%%%%%%%%%%%%%%%%%%%%%%%%%%%
%
\subsection{Weak Formulation}
\label{Sec: weak formulation}
%
%%%%%%%%%%%%%%%%%%%%%%%%%%%%%%%%%%%%%%
%
The following lemmas help us obtain the weak formulation of deterministic problem in the physical domain and construct the numerical scheme.

%\vspace{0.2 cm}
%
\begin{lem}
	\label{Lem: left frac proj}
	\cite{Li2009}: For all $\alpha \in  (0,1)$, if $u \in H^1([0,T])$ such that $u(0)=0$, and $v \in H^{\alpha/2}([0,T])$, then $( \prescript{}{0}{ \mathcal{D}}_{t}^{\,\,\alpha} u, v )_{\Omega} =  (\, \prescript{}{0}{ \mathcal{D}}_{t}^{\,\,\alpha/2} u \,,\, \prescript{}{t}{ \mathcal{D}}_{T}^{\,\,\alpha/2} v\, )_{\Omega}$, where $(\cdot , \cdot)_{\Omega}$ represents the standard inner product in $\Omega=[0,T]$. 
	%
\end{lem}
%
%\vspace{0.2 cm}
%
\begin{lem}
	\label{Lem: fractional integ-by-parts 1 and 2}
	%
	\cite{kharazmi2017petrov}: Let $1 < \beta < 2$, $a$ and $b$ be arbitrary finite or infinite real numbers. Assume $u \in H^{\beta}(a,b)$ such that $u(a)=0$, also $\prescript{}{x}{\mathcal{D}}_{b}^{\beta/2}v$ is integrable in $(a,b)$ such that $v(b) = 0$. Then, $( \prescript{}{a}{\mathcal{D}}_{x}^{\beta} u \,,\,v ) = ( \prescript{}{a}{\mathcal{D}}_{x}^{\beta/2} u \,,\,\prescript{}{x}{\mathcal{D}}_{b}^{\beta/2} v )$.
	%
\end{lem}
%
%
\begin{lem}
	%
	\label{lem_generalize}
%	\cite{kharazmi2017FSEM}:
	Let $1<\beta_j<2$ for $j=1,2,\cdots,d$, and $u,v \in  \mathcal{X}_d$. Then,  
	%
	\begin{align*}
	%
	\big(\prescript{}{a_j}{\mathcal{D}}_{x_j}^{\beta_j} u, v\big)_{\Lambda_d}=\big(\prescript{}{a_j}{\mathcal{D}}_{x_j}^{\frac{\beta_j}{2}} u, \prescript{}{x_j}{\mathcal{D}}_{b_j}^{\frac{\beta_j}{2}} v\big)_{\Lambda_d},
	\qquad
	\big(\prescript{}{x_j}{\mathcal{D}}_{b_j}^{\beta_j} u, v\big)_{\Lambda_d}=\big(\prescript{}{x_j}{\mathcal{D}}_{b_j}^{\frac{\beta_j}{2}} u, \prescript{}{a_j}{\mathcal{D}}_{x_j}^{\frac{\beta_j}{2}} v\big)_{\Lambda_d}.
	%
	\end{align*}
	%
	%Besides, if $1<2\mu_i<2$, and $u,v \in  \mathcal{X}_d$, then $\big(\prescript{}{x_i}{\mathcal{D}}_{b_i}^{2\mu_i} u, v\big)=\big(\prescript{}{x_i}{\mathcal{D}}_{b_i}^{\mu_i} u, \prescript{}{a_i}{\mathcal{D}}_{x_i}^{\mu_i} v\big),$ and
	%$\big(\prescript{}{a_i}{\mathcal{D}}_{x_i}^{2\mu_i} u, v\big)=\big(\prescript{}{a_i}{\mathcal{D}}_{x_i}^{\mu_i} u, \prescript{}{x_i}{\mathcal{D}}_{b_i}^{\mu_i} v\big).$ 
	%
\end{lem}
%



For any realization of \eqref{Doob_momentum-2}, we obtain the weak system, i.e. the variational form of the deterministic counterpart of the problem problem, subject to the given initial/boundary conditions, by multiplying the equation with proper test functions and integrate over the whole computational domain $\mathbb{D}$. Using Lemmas \ref{Lem: left frac proj}-\ref{lem_generalize}, the bilinear form can be written as
%
\begin{align}
%
%\label{Eq: general weak form_2}
\label{Eq: bilinear form}
a(u,v)
=(\prescript{}{0}{\mathcal{D}}_{t}^{\frac{\alpha}{2}}\, u, \prescript{}{t}{\mathcal{D}}_{T}^{\frac{\alpha}{2}}\, v )_{\mathbb{D}} 
%
-\sum_{j=1}^{d} 
k_{j} \Big[ ( \prescript{}{a_j}{\mathcal{D}}_{x_j}^{\frac{\beta_j}{2}}\, u,\, \prescript{}{x_j}{\mathcal{D}}_{b_j}^{\frac{\beta_j}{2}}\, v )_{\mathbb{D}}
+ ( \prescript{}{x_j}{\mathcal{D}}_{b_j}^{\frac{\beta_j}{2}}\, u , \, \prescript{}{a_j}{\mathcal{D}}_{x_j}^{\frac{\beta_j}{2}} v)_{\mathbb{D}}
\Big] ,
%
\end{align}
%
and thus, by letting $U$ and $V$ be the proper solution/test spaces, the problem reads as: find $u \in U$ such that
%
\begin{align}
%
\label{Eq: general weak form FPDE}
a(u,v) = (\text{f},v)_{\mathbb{D}}, \quad \forall v \in V ,
%
\end{align}
%
where $\text{f} =  h(t,\textbf{x}) +  f(t) $.







%
%%%%%%%%%%%%%%%%%%%%%%%%%%%%%%%%%%%%%%
%
\subsection{Petrov-Galerkin Spectral Method}
\label{Sec: Implementation}
%
%%%%%%%%%%%%%%%%%%%%%%%%%%%%%%%%%%%%%%
%
We define the following finite dimensional solution and test spaces. We employ Legendre polynomials $\phi_{m_j}(\xi), \, j=1,2,\cdots,d$, and Jacobi poly-fractonomial of first kind $\psi^{\tau}_n(\eta)$ \cite{zayernouri2015tempered,Zayernouri2013}, as the spatial and temporal bases, respectively, given in their corresponding standard domain as
%
\begin{align}
\label{Eq: Spatial Basis}
%
\phi^{}_{m_j} ( \xi )  & =  \sigma_{m_j} \big{(} P_{m_j+1} (\xi) - P_{m_j-1} (\xi)\big{)},  \quad  \xi \in [-1,1]  \qquad m_j=1,2,\cdots ,
%
\\
\label{Eq: Temporal Basis}
\psi^{\tau}_n(\eta) & = {\sigma}_{n} \prescript{(1)}{}{ \mathcal{P}}_{n}^{\,\,\tau}(\eta) = {\sigma}_{n} (1+\eta)^{\tau} P_{n-1}^{-\tau, \tau} (\eta), \quad \eta\in [-1,1]  \quad n=1,2,\cdots ,
%
\end{align}
%
in which $\sigma_{m_j} = 2 + (-1)^{m_j}$. Therefore, by performing affine mappings $\eta = 2\frac{t}{T}-1$ and $\xi = 2\frac{x-a_j}{b_j-a_j} -1$ from the computational domain to the standard domain, we construct the solution space $U_N$ as
%
\begin{align}
\label{Eq: Solution Space :PG}
U_N = 
span \, \Big\{ \,\,   
\Big( \psi^{\,\tau}_n \circ \eta \Big) ( t ) \,\,
\prod_{j=1}^{d} \Big( \phi^{}_{m_j} \circ \xi \Big)  (x_j) \,\,
: n = 1,2, \cdots, \mathcal{N}, \,\, m_j= 1,2, \cdots, \mathcal{M}_j
\,\, \Big\}.
%
\end{align}
%
We note that the choice of temporal and spatial basis functions naturally satisfy the initial and boundary conditions, respectively. The parameter $\tau$ in the temporal basis functions plays a role of fine tunning parameter, which can be chosen properly to capture the singularity of exact solution. 


Moreover, we employ Legendre polynomials $\Phi_{r_j}(\xi), \, j=1,2,\cdots,d$, and Jacobi poly-fractonomial of second kind $\Psi^{\tau}_k(\eta)$, as the spatial and temporal test functions, respectively, given in their corresponding standard domain as
%
\begin{align}
\label{Eq: Spatial Test}
%
\Phi_{r_j} ( \xi )  & =  \widetilde{\sigma}_{r_j} \big{(} P_{r_j+1}^{} (\xi) - P_{r_j-1}^{} (\xi)\big{)},  \quad  \xi \in [-1,1]  \qquad r_j =1,2,\cdots ,
%
\\
\label{Eq: Temporal Test}
\Psi^{\tau}_k(\eta) & = \widetilde{\sigma}_{k} \prescript{(2)}{}{ \mathcal{P}}_{k}^{\,\,\tau}(\eta) = \widetilde{\sigma}_{k} (1-\eta)^{\tau}\, P_{k-1}^{\tau,-\tau} (\eta), \quad \eta\in [-1,1]  \quad k=1,2,\cdots,
%
\end{align}
%
where $ \widetilde{\sigma}_{r_j} = 2\,(-1)^{r_j} + 1$. Therefore, by similar affine mapping we construct the test space $V_N$ as
%
\begin{align}
\label{Eq: Test Space: PG}
V_N = span \, \Big\{  \,\,
\Big(\Psi^{\tau}_k \circ \eta\Big)(t) \,\,
\prod_{j=1}^{d} \Big( \Phi^{}_{r_j} \circ \xi_j\Big)(x_j) \,\,
: k = 1,2, \cdots, \mathcal{N}, \,\, r_j= 1,2, \cdots, \mathcal{M}_j
\,\, \Big\}.
%
\end{align}
%
Thus, since $U_N \subset U$ and $V_N \subset  V$, the problems \eqref{Eq: general weak form FPDE} read as: find $u_N \in U_N$ such that
%
\begin{align}
\label{Eq: PG method FPDE}
%
a_h(u_N,v_N) = l(v_N), \quad \forall v_N \in V_N,
%
\end{align}
%
where $l(v_N) = (\text{f},v_N)$. The discrete bilinear form $a_h(u_N,v_N)$ can be written as 
%
\begin{align}
%
\label{Eq: discrete weak form}
a_h(u_N,v_N)
=(\prescript{}{0}{\mathcal{D}}_{t}^{\frac{\alpha}{2}}\, u_N, \prescript{}{t}{\mathcal{D}}_{T}^{\frac{\alpha}{2}}\, v_N )_{\mathbb{D}} 
%
-\sum_{j=1}^{d} 
k_{j} \Big[ ( \prescript{}{a_j}{\mathcal{D}}_{x_j}^{\frac{\beta_j}{2}}\, u_N ,\, \prescript{}{x_j}{\mathcal{D}}_{b_j}^{\frac{\beta_j}{2}}\, v_N )_{\mathbb{D}}
+ ( \prescript{}{x_j}{\mathcal{D}}_{b_j}^{\frac{\beta_j}{2}}\, u_N , \, \prescript{}{a_j}{\mathcal{D}}_{x_j}^{\frac{\beta_j}{2}} v_N)_{\mathbb{D}}
\Big].
%
\end{align}
%
We expand the approximate solution $u_N \in U_N$, satisfying the discrete bilinear form \eqref{Eq: discrete weak form}, in the following form
%
\begin{align}
\label{Eq: PG expansion}
%
u_{N}(t,\textbf{x}) = 
\sum_{n=1}^\mathcal{N}
\sum_{m_1=1}^{\mathcal{M}_1}
\cdots 
\sum_{m_d= 1}^{\mathcal{M}_d}  \,\,
\hat u_{ n,m_1,\cdots,m_d} \,\,
\Big[\psi^{\tau}_n(t)
\prod_{j=1}^{d} \phi^{}_{m_j}(x_j)
\Big] ,
%
\end{align}
%
and obtain the corresponding Lyapunov system by substituting \eqref{Eq: PG expansion} into \eqref{Eq: discrete weak form} by choosing $v_N(t,\textbf{x}) = \Psi^{\tau}_k(t) \prod_{j=1}^{d} \Phi^{}_{r_j}(x_j)$, $k = 1,2, \dots, \mathcal{N}$, $r_j= 1,2, \dots, \mathcal{M}_j$. Therefore, 
%
\begin{align}
\label{Eq: general Lyapunov}
%
\Big[
S_{T} \otimes M_1 \otimes M_2 \cdots \otimes M_d 
&+
\sum_{j=1}^{d} 
M_{T} \otimes M_1\otimes \cdots   \otimes M_{j-1} \otimes S_{j}^{{Tot}} \otimes M_{j+1}  \cdots \otimes M_d
\nonumber
\\
&+ 
\gamma \, M_{T}\otimes M_1 \otimes M_2 \cdots \otimes M_d 
\Big] \, 
\mathcal{U}= F,
%
\end{align} 
%
in which $\otimes$ represents the Kronecker product, $F$ denotes the multi-dimensional load matrix whose entries are given as
%
\begin{eqnarray}
\label{Eq: general load matrix}
%
F_{k,r_1,\cdots, r_d} = \int_{\mathbb{D}}^{} \text{f}(t,\textbf{x}) \,
\Big(
\Psi^{\,\tau}_k \circ \eta \Big)(t)
\prod_{j=1}^{d} \Big(\Phi^{}_{r_j} \circ \xi_j\Big)(x_j)\, 
%
d\mathbb{D},
%
\end{eqnarray}
%
and $\mathcal{U}$ is the matrix of unknown coefficients. The matrices $S_{T}$ and $M_{T}$ denote the temporal stiffness and mass matrices, respectively; and the matrices $S_{j}$ and $M_j$ denote the spatial stiffness and mass matrices, respectively. We obtain the entries of spatial mass matrix $M_j$ analytically and employ proper quadrature rules to accurately compute the entries of other matrices $S_{T}$, $M_{T}$ and $S_{j}$.


We note that the choices of basis/test functions, employed in developing the PG scheme leads to symmetric mass and stiffness matrices, providing useful properties to further develop a fast solver. The following Theorem \ref{Thm: fast solver} provides a unified fast solver, developed in terms of the generalized eigensolutions in order to obtain a closed-form solution to the Lyapunov system \eqref{Eq: general Lyapunov}.


%
\begin{thm}[Unified Fast FPDE Solver \cite{samiee2016}]
	\label{Thm: fast solver}
	%
	Let $\{ {\vec{e}_{}}^{\,\,\mu_j}    ,    \lambda^{}_{m_j}\,  \}_{m_j=1}^{\mathcal{M}_j}$ be the set of general eigen-solutions of the spatial stiffness matrix $S^{Tot}_j$ with respect to the mass matrix $M_{j}$. Moreover, let $\{ {\vec{e}_{n}}^{\,\,\tau}    ,    \lambda^{\tau}_{n}\,  \}_{n=1}^{\mathcal{N}}$ be the set of general eigen-solutions of the temporal mass matrix $M_{T}$ with respect to the stiffness matrix $S_{T}$. Then, the matrix of unknown coefficients $\mathcal{U}$ is explicitly obtained as
	%
	\begin{equation}
	\label{Eq: thm u expression in terms of k}
	%
	\mathcal{U} = 
	\sum_{n=1}^{\mathcal{N}}
	\,\,
	\sum_{m_1= 1}^{\mathcal{M}_1}
	\cdots 
	\sum_{m_d= 1}^{\mathcal{M}_d}
	%
	\kappa_{ n,m_1,\cdots,\,m_d  } \,
	\,\vec{e}_n^{\,\,\tau}\,
	\otimes
	\,{\vec{e}_{m_1}}^{}\,\,
	\otimes
	\cdots
	\otimes
	\,{\vec{e}_{m_d}}^{},
	%
	%
	\end{equation}
	%
	where $\kappa_{ n,m_1,\cdots,\,m_d }$ is given by 
	%
	\begin{eqnarray}
	\label{Eq: thm k fraction_1}
	%
	\kappa_{ n,m_1,\cdots,\,m_d  } =  \frac{(\,\vec{e}_n^{\,\,\tau}
		\,{\vec{e}_{m_1}}^{}
		\cdots
		\,{\vec{e}_{m_d}}^{}) F}
	{
		\Big[
		(\vec{e}_n^{\,\,\tau^T} \, S_{T} \, \vec{e}_n^{\,\,\tau}) \,
		\prod_{j=1}^{d} (\vec{e}_{m_j}^{T}  \,  M_{j} \,  {\vec{e}_{m_j}}^{}) \,
		\Big]
		%
		\Lambda_{n,m_1,\cdots,m_d}
		%
	},
	%
	\end{eqnarray}
	%
	in which the numerator represents the standard multi-dimensional inner product, and $\Lambda_{n,m_1,\cdots,m_d}$ is obtained in terms of the eigenvalues of all mass matrices as
	%
	\begin{eqnarray}
	%
	\nonumber
	%
	&\Lambda_{n,m_1,\cdots, m_d} = \Big[
	%
	(1+\gamma\,\, 
	%
	\lambda^{\tau}_n)
	%
	+
	\lambda^{\tau}_n
	%
	\sum_{j=1}^{d}
	%
	(
	%
	\lambda^{}_{m_j}
	%
	%
	)
	\Big].  &
	\end{eqnarray}
	
\end{thm}

%
%
%
%
%%%%%%%%%%%%%%%%%%%%%%%%%%%%%%%%%%%%%%
%
\subsection{Stability Analysis}
\label{Sec: Stability and Convergence of PG}
%
%%%%%%%%%%%%%%%%%%%%%%%%%%%%%%%%%%%%%%
%

We show the well-posedness of deterministic problem and prove the stability of proposed PG scheme. %, following \cite{kharazmi2017FSEM}. 
%
\begin{lem}
	%
	\label{norm_223}
	Let $\alpha \in (0,1)$, $\Omega=I \times \Lambda_d$, and $u\in \prescript{l}{0}H^{\alpha/2}(I; L^2(\Lambda_d))$. Then, 
	%
	\begin{equation*}
	%
	\big\vert \left( \prescript{}{0}{\mathcal{D}}_{t}^{\alpha/2} u, \prescript{}{t}{\mathcal{D}}_{T}^{\alpha/2} v \right)_{\Omega} \big\vert \equiv \Vert u \Vert_{\prescript{l}{}H^{\alpha/2}(I; L^2(\Lambda_d))} \, \Vert v \Vert_{\prescript{r}{}H^{\alpha/2}(I; L^2(\Lambda_d))},
	\quad 
	\forall v \in \prescript{r}{0}H^{\alpha/2}(I; L^2(\Lambda_d)).
	%
	\end{equation*}
	%
	%
\end{lem}
%
Moreover,
%
\begin{align}
%
\label{equiv_space}
\vert \big(\prescript{}{a_d}{\mathcal{D}}_{x_d}^{\beta_d/2} u, \prescript{}{x_d}{\mathcal{D}}_{b_d}^{\beta_d/2} v\big)_{\Lambda_d} \vert \equiv  \vert u \vert_{\prescript{c}{}H^{\beta_d/2} \Big((a_d,b_d); L^2(\Lambda_{d-1}) \Big)} \, \vert v \vert_{\prescript{c}{}H^{\beta_d/2} \Big((a_d,b_d); L^2(\Lambda_{d-1}) \Big)},
%
\\
\label{equiv_space2}
\vert \big(\prescript{}{x_d}{\mathcal{D}}_{b_d}^{\beta_d/2} u, \prescript{}{a_d}{\mathcal{D}}_{x_d}^{\beta_d/2} v\big)_{\Lambda_d} \vert \equiv  \vert u \vert_{\prescript{c}{}H^{\beta_d/2} \Big((a_d,b_d); L^2(\Lambda_{d-1}) \Big)} \, \vert v \vert_{\prescript{c}{}H^{\beta_d/2} \Big((a_d,b_d); L^2(\Lambda_{d-1}) \Big)}.
%
\end{align} 
%



% 
\begin{lem}[Continuity]
	%
	\label{continuity_lem}
	The bilinear form \eqref{Eq: bilinear form} is continuous, i.e.,
	%
	\begin{align}
	%
	\label{continuity_eq}
	\forall u \in U, \,\,
	%\mathcal{B}^{\tau,\nu_1,\cdots,\nu_d} (\Omega) 
	\exists \, \beta > 0,  
	\quad \text{s.t.} \quad 
	\vert a(u,v)\vert \leq 
	\beta \,\,   \Vert u \Vert_{U}   \,\,     \Vert v \Vert_{V},
	%\beta \, \Vert u \Vert_{\mathcal{B}^{\tau,\nu_1,\cdots,\nu_d}(\Omega)}\Vert v \Vert_{\mathfrak{B}^{\tau,\nu_1,\cdots,\nu_d}(\Omega)} 
	\quad \forall v \in V.
	%
	\end{align}
	%
\end{lem}
%
%
\begin{proof}
	The proof directly concludes from \eqref{equiv_space} and Lemma \ref{norm_223}.
	%With the aid of \eqref{equiv_space} and lemma \ref{norm_223}, we directly conclude \eqref{continuity_eq}.
\end{proof}	
%


\begin{thm}[Stability]
	\label{inf_sup_d_lem}
%	\cite{kharazmi2017FSEM}
	The following inf-sup condition holds for the bilinear form \eqref{Eq: bilinear form}, i.e.,
	%
	\begin{align}
	%
	\label{Eq: inf sup-time_d_well}
	%
	\underset{ u \neq 0  \in U}{\inf} \,\, \underset{ v \neq 0 \in V}{\sup}
	\frac{\vert a(u , v)\vert}{ \,\, \Vert v\Vert_{V} \,\, \Vert u \Vert_{U} } \geq \beta > 0,
	%
	%\underset{0 \neq u \in \mathcal{B}^{\tau,\nu_1,\cdots,\nu_d} (\Omega)} {\inf} \,\,\underset{0 \neq v \in\mathfrak{B}^{\tau,\nu_1,\cdots,\nu_d} (\Omega)}{\sup}
	%\frac{\vert a(u , v)\vert}{\Vert v\Vert_{\mathfrak{B}^{\tau,\nu_1,\cdots,\nu_d}(\Omega)}\Vert u\Vert_{\mathcal{B}^{\tau,\nu_1,\cdots,\nu_d}}(\Omega)} \geq \beta > 0, 
	%
	\end{align}
	%
	where $\Omega = I \times \Lambda_d$ and $\underset{u \in U}{\sup} \,\, \vert a(u , v)\vert>0$.
\end{thm}



\begin{thm}[well-posedness]
	\label{Thm: well-posedness_1D}
	For all $0<\alpha<1$, $\alpha \neq 1$, and  $1<\beta_j<2$, and $j=1,\cdots,d$, there exists a unique solution to \eqref{Eq: general weak form FPDE}, continuously dependent on  $f$, which belongs to the dual space of $U$.
\end{thm}
%
\begin{proof}
	%
	Lemmas \ref{continuity_lem} (continuity) and \ref{inf_sup_d_lem} (stability) yield the well-posedness of weak form \eqref{Eq: general weak form FPDE} in (1+d)-dimension due to the generalized Babu\v{s}ka-Lax-Milgram theorem.
	%
\end{proof}

Since the defined basis and test spaces are Hilbert spaces, and $U_N \subset U$ and $V_N \subset V$, we can prove that the developed Petrov-Gelerkin spectral method is stable and the following condition holds
%
\begin{align}
\label{Eq: inf sup-time}
%
\underset{u_N \neq 0 \in U_N}{\inf}\, \, \underset{v \neq 0 \in V_N}{\sup}
\frac{\vert a(u_N , v_N)\vert}{\Vert v_N\Vert_{V} \,\, \Vert u_N\Vert_{U}} \geq \beta > 0, 
%
\end{align}
%
with $\beta > 0$ and independent of $N$, where $\underset{u_N \in U_N}{\sup} \vert a(u_N , v_N)\vert>0$.
