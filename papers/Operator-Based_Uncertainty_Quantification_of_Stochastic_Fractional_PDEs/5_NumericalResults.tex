%\newpage


%
%%%%%%%%%%%%%%%%%%%%%%%%%%%
\section{\textbf{Numerical Results}}
\label{Sec: numerical results}
%%%%%%%%%%%%%%%%%%%%%%%%%%%
%
We investigate the performance of developed numerical methods by considering couple of numerical simulations. We compare MCS and PCM in random space discretization while using PG method in physical domain. We note that by several numerical examples, we make sure that the developed PG method is stable and accurate in solving each deterministic problem; the results are not provided here.


%
%%%%%%%%%%%%%%%%%%%%%%%%%%%
\subsection{Low-Dimensional Random Inputs}
%\label{Sec: }
%%%%%%%%%%%%%%%%%%%%%%%%%%%
%
As the first case, we consider a stochastic fractional initial value problem (IVP) with random fractional index by letting the diffusion coefficient to be zero, and also ignoring the additional random input and only taking $h(t)$ as the external forcing term. Therefore, we obtain
%
\begin{equation}
%
\label{Eq: SFIVP}
\prescript{}{0}{\mathcal{D}}_{t}^{\alpha(\xi)}  u(t;\xi) = h(t),
%
\end{equation}
%
subject to zero initial condition, where $u(t,\xi): (0,T] \times \Lambda \rightarrow \mathbb{R}$. We let $ u^{ext}(t) = \frac{\alpha}{2} \,\, t^{3+\frac{\alpha}{2}} $, $ h(t) = \prescript{}{0}{\mathcal{D}}_{t}^{\alpha(\xi)} u^{ext}(t) $ for each realization of $\alpha$. In this case, by choosing the tunning parameter $\tau$ in the temporal basis function to be $\frac{\alpha}{2}$, we can efficiently employ PG numerical scheme and also obtain the exact expectation by rendering the exact solution to be random with similar distribution as the random fractional index. Fig.\ref{Fig: MCM PCM SFIVP} shows the $L^2$-norm convergence rate of MCS and PCM in comparison of solution expectation with $\mathop{\mathbb{E}}^{ext}[u] = \mathop{\mathbb{E}}[u^{ext}]$. The results confirms converges rate of $0.5$ for MCS, while in PCM, the statistics of solution converges accurately very fast, using only few numbers of realizations. In this example, by ignoring the additional random input to the system, we take the advantage of having the exact random solution to be available. 
%
%******************************************************************************************
%
\begin{figure}[t]
	\centering
	\includegraphics[width=0.5\linewidth]{FDE-MC-PCM.pdf}
	\caption{$L^2$-norm convergence rate of MCM and PCM for stochastic fractional IVP \eqref{Eq: SFIVP}.}
	\label{Fig: MCM PCM SFIVP}
\end{figure}
%
%****************************************************************************************** 
%


As another example, we also consider \eqref{Eq: SFIVP} with additional random input, expanded by KL expansion with $M=4$, as:
%
\begin{equation}
%
\label{Eq: SFIVP-2}
\prescript{}{0}{\mathcal{D}}_{t}^{\alpha(\xi)}  u(t;\boldsymbol{\xi}) = h(t) +  \sum_{k=1}^M \, a_k \, \sin\left(\frac{2 k \pi \, t}{T}\right) \, \xi_k,
%
\end{equation}
%
with two cases $h(t) = t^2$ and $h(t) = sin(\pi t)$. Fig.\ref{Fig: MCM PCM SFIVP-2} shows the mean value and variance of solution for $10^4$ sampling of MCS compared to $625$ realizations in PCM.
%
%******************************************************************************************
%
\begin{figure}[t]
	\centering
	\begin{subfigure}{0.45\textwidth}
		\centering
		\includegraphics[width=1\linewidth]{SFODE_Noise_1.pdf}
%		\caption{}
		%\label{Fig:}
	\end{subfigure}
	%
	\begin{subfigure}{0.45\textwidth}
		\centering
		\includegraphics[width=1\linewidth]{SFODE_Noise_2.pdf}
%		\caption{}
		%\label{Fig: uext suext error FDE}
	\end{subfigure}
	\caption{Expectation of solution to \eqref{Eq: SFIVP-2} with uncertainty (standard deviation) bounds, employing MCS and PCM for (left) $h(t) = t^2$ and (right) $h(t) = sin(\pi t)$. }
	\label{Fig: MCM PCM SFIVP-2}
\end{figure}
%
%****************************************************************************************** 
%


%\newpage

Moreover, we consider (1+1)-D one-sided SFPDE given in \eqref{Doob_momentum-2}, where $d=1$ and the diffusion coefficient is $k_l$. We ignore the additional random input and consider $h(t,x)$ as the only external forcing term. Therefore, we obtain
%
\begin{align}
\label{Eq: 1+1-d SFPDE no random noise}
%
&\prescript{}{0}{\mathcal{D}}_{t}^{\alpha(\xi_1)}  u(t,x;\boldsymbol{\xi}) 
+ k_l \prescript{}{-1}{\mathcal{D}}_{x}^{\, \beta(\xi_2)} u(t,x;\boldsymbol{\xi}) 
= h(t,x),
%
\end{align}
%
subject to zero initial/boundary conditions, where $u(t,x;\boldsymbol{\xi}): (0,T] \times (-1,1) \times \Lambda \rightarrow \mathbb{R}$, and the only random variables are the fractional indices $\alpha$ and $\beta$. We let $u^{ext}(t,x) = t^{3+\tau} \, \left((1+x)^{3+\mu}-\frac{1}{2}(1+x)^{4+\mu}\right)$, and choose $\tau = \alpha/2$ and $\mu = \beta/2 $. For each realization of $\alpha$ and $\beta$, we obtain the force function $h(t,x)$ by substituting the corresponding $u^{ext}$ to \eqref{Eq: 1+1-d SFPDE no random noise}. Defining $\mathop{\mathbb{E}}^{ext}[u] = \mathop{\mathbb{E}}[u^{ext}]$, Fig.\ref{Fig: MCM PCM SDPDE no noise} shows the $L^2$-norm convergence of solution expectation as compared to the exact expectation. We observe that PCM converges accurately with only few number of realizations.
%
%******************************************************************************************
%
\begin{figure}[t]
	\centering
	\includegraphics[width=0.5\linewidth]{FPDE-MC-PCM.pdf}
	\caption{$L^2$-norm convergence rate of MCM and PCM for SFPDE \eqref{Eq: 1+1-d SFPDE no random noise}.}
	\label{Fig: MCM PCM SDPDE no noise}
\end{figure}
%
%****************************************************************************************** 
%

Considering additional random input, expanded by KL expansion with $M=4$, the problem can be recast as
%
\begin{align}
\label{1+1-d SFPDE with random noise}
%
&\prescript{}{0}{\mathcal{D}}_{t}^{\alpha(\boldsymbol{\xi})}  u(t,x;\boldsymbol{\xi}) 
+ k_l \prescript{}{-1}{\mathcal{D}}_{x}^{\, \beta(\boldsymbol{\xi})} u(t,x;\boldsymbol{\xi}) 
= h(t,x) +   \sum_{k=1}^M \, a_k \, \sin\left(\frac{2 k \pi \, t}{T}\right) \, \xi_k 
%
\end{align}
%
subject to zero initial/boundary conditions. Fig.\ref{Fig: MCM PCM SFPDE-2} shows the mean value and variance of solution for MCS and PCM at different times.

%
%******************************************************************************************
%
\begin{figure}[t]
	\centering
	\includegraphics[width=0.5\linewidth]{SFPDE_Noise_1.pdf}
	%
	\caption{Expectation of solution to \eqref{1+1-d SFPDE with random noise}, employing MCS and PCM at $t=0.125, \, 0.625, \, 1$.}
	\label{Fig: MCM PCM SFPDE-2}
\end{figure}
%
%****************************************************************************************** 
%

\begin{rem}
%	
We note that generally use of the sparse grid operators in obtaining solution statistics is more effective when dimension of the random space is higher than 6. Thus, in the numerical examples for low-dimensional random inputs, we employ the easy-to-implement tensor product nodal sets. 
%	
\end{rem}




%
%%%%%%%%%%%%%%%%%%%%%%%%%%%
\subsection{Moderate- to High-Dimensional Random Inputs}
%\label{Sec: }
%%%%%%%%%%%%%%%%%%%%%%%%%%%
%
We render the problem with higher number of terms in KL expansion of random inputs in \eqref{1+1-d SFPDE with random noise} by choosing $M=10$ and $M=20$. This yields the dimension of random space $\mathcal{N} = 12$ and $\mathcal{N} = 22$, respectively. By employing the Smolyak sparse grid generator and using the developed PCM, we obtain the solution statistics. For each case, we generate the sparse grid on two levels $w=1$ and $w=2$, i.e. $A(1,12)$, $A(2,12)$, $A(1,22)$, and $A(2,22)$, where we let the higher resolution case be a benchmark value to the solution statistics, based on which we compute and normalize the error. We observe that for both cases $\mathcal{N} = 12$ and $\mathcal{N} = 22$, the normalized error in computing the expectation and standard deviation of solution are of orders $\mathcal{O}(10^{-7})$ and $\mathcal{O}(10^{-3})$, respectively. 


