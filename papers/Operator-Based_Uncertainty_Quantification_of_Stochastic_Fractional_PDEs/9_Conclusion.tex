%\newpage

%
%%%%%%%%%%%%%%%%%%%%%%%%%%%%%%%%%%%%%%%%%
\section{Summary and Discussion}
\label{Sec: Summary and Conclusion} 
%%%%%%%%%%%%%%%%%%%%%%%%%%%%%%%%%%%%%%%%%
%
We developed a mathematical framework to numerically quantify the solution uncertainty of a stochastic FPDE, associated with the randomness of model parameters. The stochastic FPDE is reformulated by rendering the problem with random fractional indices, subject to additional random noise. We used the truncated Karhunen-Lo\'eve expansion to parametrize the additive noise. Then, by employing a non-intrusive probabilistic collocation method (PCM), we propagated the associated randomness to the system response, by using Smolyak sparse grid generator to construct the set of sample point in the random space. We also formulated a forward solver to simulate the deterministic counterpart of the stochastic problem for each realization of random variables. We showed that the deterministic problem is mathematically well-posed in a weak sense. Furthermore, by employing Jacobi poly-fractonomials and Legendre polynomials as the temporal and spatial basis/test functions, respectively, we developed a Petrove-Galerkin spectral method to solve the deterministic problem in the physical domain. We also proved that the \text{inf-sup} condition holds for the proposed numerical scheme, and thus, it is stable. By considering several numerical examples with low- to high-dimensional random spaces, we examined the performance of our stochastic discretization. We showed that in each case, PCM converges very fast to a very high level of accuracy with very few number of sampling. 

