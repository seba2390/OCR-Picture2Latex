
%
%%%%%%%%%%%%%%%%%%%%%%%%%%%
\section{Preliminaries on Fractional Calculus}
\label{Sec: Fractional Calculus}
%%%%%%%%%%%%%%%%%%%%%%%%%%%
%

Let $ \xi \in [-1,1]$. The left- and right-sided fractional derivative of order $\sigma$ are defined as (see e.g., \cite{Miller93, Podlubny99})
%
\begin{align}
\label{Eq: left RL derivative}
%
(\prescript{RL}{-1}{\mathcal{D}}_{\xi}^{\sigma}) u(\xi) = \frac{1}{\Gamma(n-\sigma)}  \frac{d^{n}}{d\xi^n} \int_{-1}^{\xi} \frac{u(s) ds}{(\xi - s)^{\sigma+1-n} },\quad \xi >-1 ,
%
\\
\label{Eq: right RL derivative}
%
(\prescript{RL}{\xi}{\mathcal{D}}_{1}^{\sigma}) u(\xi) = \frac{1}{\Gamma(n-\sigma)}  \frac{(-d)^{n}}{d\xi^n} \int_{\xi}^{1} \frac{u(s) ds}{(s - \xi)^{\sigma+1-n} },\quad \xi < 1 ,
%
\end{align}
%
respectively. 
%
An alternative approach in defining the fractional derivatives is the left- and right-sided Caputo derivatives of order $\sigma$, $n-1 < \sigma \leq n$, $n \in \mathbb{N}$, defined, as
%
\begin{align}
\label{Eq: left Caputo derivative}
%
&
(\prescript{C}{-1}{\mathcal{D}}_{\xi}^{\sigma} u) (\xi) = \frac{1}{\Gamma(n-\sigma)}  \int_{-1}^{\xi} \frac{u^{(n)}(s) ds}{(\xi - s)^{\sigma+1-n} },\quad \xi>-1,
\\
\label{Eq: right Caputo derivative}
%
&
(\prescript{C}{\xi}{\mathcal{D}}_{1}^{\sigma} u) (\xi) =  \frac{1}{\Gamma(n-\sigma)}  \int_{\xi}^{1} \frac{u^{(n)}(s) ds}{(s-\xi)^{\sigma+1-n} },\quad \xi<1.
%
\end{align}
%
By performing an affine mapping from the standard domain $[-1,1]$ to the interval $t \in [a,b]$, we obtain
%
\begin{eqnarray}
\label{Eq: RL in xL-xR}
%
\prescript{RL}{a}{\mathcal{D}}_{t}^{\sigma} u  &=&  (\frac{2}{b-a})^\sigma (\prescript{RL}{-1}{\mathcal{D}}_{\xi}^{\sigma} \, u )(\xi), 
\\ 
\label{Eq: Caputo in xL-xR}
\prescript{C}{a}{\mathcal{D}}_{t}^{\sigma} u  &=&  (\frac{2}{b-a})^\sigma (\prescript{C}{-1}{\mathcal{D}}_{\xi}^{\sigma} \, u) (\xi).
%
\end{eqnarray} 
%
Hence, we can perform the operations in the standard domain only once for any given $\sigma$ and efficiently utilize them on any arbitrary interval without resorting to repeating the calculations. Moreover, the corresponding relationship between the Riemann-Liouville and Caputo fractional derivatives in $[a,b]$ for any $\sigma \in (0,1)$ is given by 
%
\begin{equation}
\label{Eq:  Caputo vs. Riemann}
(\prescript{RL}{a}{\mathcal{D}}_{t}^{\sigma} \, u) (t)  =  \frac{ u(a)}{\Gamma(1-\sigma) (t-a)^{\sigma}}  +   (\prescript{C}{a}{\mathcal{D}}_{t}^{\sigma} \, u) (t).
\end{equation}
%











