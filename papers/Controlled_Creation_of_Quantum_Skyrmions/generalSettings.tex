\usepackage{graphicx}
\usepackage{bm}
\usepackage{dcolumn}
\usepackage{amssymb}
\usepackage{amsmath}
\usepackage{amsbsy}
\usepackage{amsfonts}
\usepackage{epsfig}
\usepackage{graphicx}
\usepackage{multirow}
\usepackage{stackrel}
\usepackage{paralist}
\usepackage{braket}
\usepackage{titlesec}
\usepackage{placeins}
\usepackage{color}
\usepackage[rgb]{xcolor}
%\usepackage[textsize=tiny,textwidth = 1.6 cm]{todonotes}
%\usepackage{dblfloatfix}    % To enable figures at the bottom of page
%\usepackage{subcaption}



\usepackage[percent]{overpic}
\usepackage{hyperref}
\hypersetup{colorlinks=true,allcolors=blue}
\usepackage{tikz}
\usetikzlibrary{intersections} 
% \usetikzlibrary{external} \tikzexternalize 
\usetikzlibrary{positioning,calc}
\usepackage[caption=false]{subfig}
%\usepackage{floatrow}
%\floatsetup[figure]{style=plain,subcapbesideposition=top}
\captionsetup[subfigure]{position=top, labelformat=brace, labelfont=bf,textfont=normalfont,singlelinecheck=off,justification=raggedright,topadjust=-1.3\baselineskip} %topadjust can be used to lower the caption, e.g., topadjust = - \baselineskip

\newcommand{\adjust}{\vspace*{0ex}\hspace*{0ex}}

\usepackage[normalem]{ulem}


\newcommand{\Tcomment}[1]{\todo[inline,color=green!40]{#1}{}}
\newcommand{\Mcomment}[1]{\todo[inline,color=blue!40]{#1}{}}

\tikzset{
    right angle quadrant/.code={
        \pgfmathsetmacro\quadranta{{1,1,-1,-1}[#1-1]}     % Arrays for selecting quadrant
        \pgfmathsetmacro\quadrantb{{1,-1,-1,1}[#1-1]}},
    right angle quadrant=1, % Make sure it is set, even if not called explicitly
    right angle length/.code={\def\rightanglelength{#1}},   % Length of symbol
    right angle length= 1 ex, % Make sure it is set...
    right angle symbol/.style n args={3}{
        insert path={
            let \p0 = ($(#1)!(#3)!(#2)$) in     % Intersection
                let \p1 =  ($(\p0)!\quadranta*\rightanglelength!(#3)$),
                \p2 = ($(\p0)!\quadranta*\rightanglelength!(#2)$) in %$(\p0)+\rightanglelength/{veclen((#2)-(\p0)))}*((#2)-(\p0))$ in % Point on perpendicular line
                let \p3 = ($(\p1)+(\p2)-(\p0)$) in  % Corner point of symbol
            (\p1) -- (\p3) -- (\p2)
        }
    }
}


\newcommand{\alert}[1]{\textcolor{red}{#1}}

\newcommand{\cancel}[1]{}%{\sout{#1}} %{}%
\newcommand{\new}[1]{{\color{blue}#1}}%{\textcolor{blue}{#1}}  % {#1}%

\newcommand{\I}{\mathrm{i}}
\DeclareMathOperator{\inv}{inv}
\DeclareMathOperator{\Tr}{Tr}
\DeclareMathOperator{\diag}{diag}
\DeclareMathOperator{\sgn}{sgn}
\DeclareMathOperator{\rect}{\sqcap}%{R}%{\sqcap}

\newcommand{\refEq}[1]{Eq.~(\ref{#1})}
\newcommand{\refFig}[1]{Fig.~\ref{#1}}
\newcommand{\refSctn}[1]{Section~\ref{#1}}
\newcommand{\refTab}[1]{Tab.~\ref{#1}}
\newcommand{\citeRef}[1]{Ref.~[\onlinecite{#1}]}
\newcommand{\refApp}[1]{Appendix~\ref{#1}}
\newcommand{\refSMOverview}{\footnote{See Supplemental Material at [URL will be inserted by publisher] section A for an overview about the intermediate statistics in one dimension.}}
% \newcommand{\refSMInterpretationWaveFunction}{\footnote{See Supplemental Material at [URL will be inserted by publisher], section B, for a physical interpretation of the anyonic wave functions in terms of anyon scattering.}}
% \newcommand{\refSMProofVanishingPhase}{\footnote{See Supplemental Material at [URL will be inserted by publisher], section C, for the proof that iteration of these conditions does not result in contradictions.}}

\newcommand{\refSMGeneral}{\cite{SupplementalMaterialAnyons}} %\footnote{See Supplemental Material at [URL will be inserted by the publisher] for details.}}
\newcommand{\refSMFusedAnyons}{\refSMGeneral}%{\footnote{See Supplemental Material at [URL will be inserted by publisher], section B, for an example for the interpretation of anyon clusters as individual anyons.}}
% \newcommand{\refSMAllowedComplexMomenta}{\footnote{See Supplemental Material at [URL will be inserted by publisher], section E, for the derivation of the allowed complex momenta in a wave function.}}
\newcommand{\refSMGeneralizedJWTrafo}{\refSMGeneral}%{\footnote{See Supplemental Material at [URL will be inserted by publisher], section D, for the generalized Jordan-Wigner transformation to convert from an anyonic second quantized algebra to a bosonic one.}}
% \newcommand{\refSMMomentumDiscretization}{\footnote{See Supplemental Material at [URL will be inserted by publisher], section G, for  the derivation of the transcendental, momentum discretizing equations for confined anyons.}}
\newcommand{\refSMRealSpaceAlgebra}{\refSMGeneral}%{\footnote{See Supplemental Material at [URL will be inserted by publisher], section C, for the real space algebra for anyonic creation and annihilation operators.}}
\definecolor{NewColor}{rgb}{0,0,1}
\definecolor{myRed}{rgb}{1,0,0}
\definecolor{myGreen}{rgb}{0.2,0.6,0.2}
\definecolor{myBlue}{rgb}{0,0,1}
\newcommand{\New}{\textcolor{NewColor}}
\graphicspath{{../Graphics/}{../Graphics/Publication/}}

\newcommand{\CO}[1]{\textcolor{red}{}}

\newcommand{\expstyle}{\scriptstyle}

\newlength{\noLengthl}
\newcommand{\nolength}[1]{\ensuremath{#1\settowidth{\noLengthl}{$#1$}\hspace*{-\noLengthl}}}
\newcommand{\intS}[2]{\ensuremath{\int_{\nolength{#1}}^{\nolength{#2}}}}
\newcommand{\underbraceS}[2]{\ensuremath{\settowidth{\noLengthl}{$#1$}\hspace*{-1\noLengthl} \underbrace{#1}_{#2} \hspace*{-1\noLengthl}}}
\newcommand{\prodS}[2]{\ensuremath{\settowidth{\noLengthl}{$#1$}\hspace*{-#2 \noLengthl} \prod_{#1} \hspace*{- #2 \noLengthl}}}
\newcommand{\sumS}[2]{\ensuremath{\settowidth{\noLengthl}{$#1$}\hspace*{-#2 \noLengthl} \sum_{#1} \hspace*{- #2 \noLengthl}}}
\renewcommand{\Im}{\mathrm{Im}}
\renewcommand{\Re}{\mathrm{Re}}

\renewcommand{\vec}[1]{{\boldsymbol{#1}}}
