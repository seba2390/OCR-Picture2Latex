\documentclass[a4paper]{jpconf}

\pdfoutput=1

\usepackage{mathrsfs}

\usepackage{graphicx}
\usepackage{amsthm}
\usepackage{amssymb}
\usepackage{bm}
\usepackage{amstext}
%\usepackage{psfrag}
\usepackage{amsmath}
\usepackage{amsfonts}
\usepackage{cite}
%\usepackage{wrapfig}
%%%%%%%%%%%%%%%%%%%%%%%%%%%%%%%%%%%%%%%%%%%%%%%%%%%%%%%%%%%%%%%%%%%%%
\newcommand{\bd}{\begin{definition}}                %inizia definizione
\newcommand{\ed}{\end{definition}}                  %fine definizione
\newcommand{\bc}{\begin{corollary}}                 %inizia corollario
\newcommand{\ec}{\end{corollary}}                   %fine corollario
\newcommand{\bl}{\begin{lemma}}                     %inizia lemma
\newcommand{\el}{\end{lemma}}                       %fine lemma
\newcommand{\bp}{\begin{proposition}}            %inizia proposizione
\newcommand{\ep}{\end{proposition}}                %fine proposizione
\newcommand{\bere}{\begin{remark}}                  %inizia osservazione
\newcommand{\ere}{\end{remark}}                     %fine oservazione

\newcommand{\bt}{\begin{theorem}}
\newcommand{\et}{\end{theorem}}

\newcommand{\be}{\begin{equation}}
\newcommand{\ee}{\end{equation}}

\newcommand{\bit}{\begin{itemize}}
\newcommand{\eit}{\end{itemize}}
\newtheorem{theorem}{Theorem}[section]
\newtheorem{corollary}[theorem]{Corollary}
\newtheorem{lemma}[theorem]{Lemma}
\newtheorem{proposition}[theorem]{Proposition}
\theoremstyle{definition}
\newtheorem{definition}[theorem]{Definition}
\theoremstyle{remark}
\newtheorem{remark}[theorem]{Remark}
\newtheorem{example}[theorem]{Example}

%%%%%%%%%%%%%%%%%%%%%%%%%%%%%%%%%%%%%%%%%%%%%%%%%%%%%%%%%%%%%%%%%%%%%%

% VEDERE I TAG
%\usepackage{color}
%\usepackage[color]{showkeys}
%\definecolor{refkey}{rgb}{0,0,1} \definecolor{labelkey}{rgb}{1,0,0}


\newcommand{\p}{\partial}
\newcommand{\eu}{{\rm e}}
\newcommand{\dd}{{\rm d}}
%\renewcommand{\thesection}{\arabic{section}}
\begin{document}

%\title{Spacetime representability through steep time functions}

\title{The representation of spacetime through steep time functions}
\author{E. Minguzzi}

\address{Dipartimento di Matematica e Informatica ``U. Dini'', Universit\`a degli Studi di Firenze,  Via
S. Marta 3,  I-50139 Firenze, Italy}

\ead{ettore.minguzzi@unifi.it}

\begin{abstract}
In a recent work I showed that  the family of smooth steep time functions can be used to recover the order, the topology and the (Lorentz-Finsler) distance of spacetime. In this work I present  the main ideas entering the proof of the (smooth) distance formula, particularly the product trick which converts metric statements into causal ones. The paper ends with a second proof of the  distance formula valid for globally hyperbolic Lorentzian spacetimes.
\end{abstract}


\section{Introduction}
In a recent work \cite{minguzzi17} I  obtained optimal conditions for the existence of steep time functions on spacetime.  The result was used to characterize the Lorentzian submanifolds of Minkowski spacetime and to prove the (smooth Lorentz-Finsler) distance formula. At the meeting  I announced the latter result while placing it into the broad context of functional representation results for topological ordered spaces \cite{nachbin65,minguzzi12d}. In this work  I shall outline the proof strategy instead, by making use of some illustrations, and by leaving the technical details to the original paper.

Unless otherwise stated, we shall work in the general context of closed cone structures and closed Lorentz-Finsler spaces. Let $M$ be a  connected, Hausdorff, second-countable $C^1$ manifold.
A {\em closed cone structure} $(M,C)$, $C\subset TM\backslash 0$, is a closed cone subbundle of the slit tangent bundle such that $C_x:=C\cap T_{x}M$ is a closed (in the topology of $T_xM\backslash 0$), sharp, convex, non-empty cone. The multivalued map $x \mapsto C_x$ turns out to be upper semi-continuous, that is as $y$ approaches $x$, using the identification of tangent spaces provided by any coordinate system, $C_y\backslash C_x\to \emptyset$, cf.\ \cite{aubin84,minguzzi17} for a more precise definition of the last limit. Other useful differentiability conditions on the multivalued map are continuity and local Lipschitzness \cite{fathi12,minguzzi17}. We stress that our cones $C_x$ are not necessarily strictly convex, a feature that will be important, as we shall see.

The closed cone structure conveys the notion of causality. The {\em causal vectors} are the elements of $C$, the {\em timelike vectors} are the elements of $\textrm{Int} C$ while those belonging to $C\backslash \textrm{Int} C$ might be called {\em lightlike vectors}. In the previous expressions $\textrm{Int}$ is the interior for the topology of $TM\backslash 0$. It can be noticed that $ (\textrm{Int} C)_x \subset \textrm{Int} C_x$ and equality holds for $C^0$ cone structures. A continuous causal curve is an absolutely continuous map $x\colon I \to M$ which has causal derivative almost everywhere. The causal relation $J$ consists of all those pairs of events $(p,q)$ such that there is a continuous causal curve connecting $p$ to $q$ or $p=q$.
{\em Closed} cone structures are particularly well behaved since for them the limit curve theorem holds true. Actually, several other non trivial results hold true for closed cone structures, such as the causal ladder of spacetimes. We shall not explore these findings in this work; the reader is referred to \cite{minguzzi17} for more details.


The length of vectors is measured by the fundamental Lorentz-Finsler function $\mathscr{F}\colon C\to [0,+\infty)$ which is positive homogeneous and concave.
 As a consequence, it satisfies the reverse triangle inequality: for every $y_1,y_2\in C$, $\mathscr{F}(y_1+y_2)\ge \mathscr{F}(y_1)+\mathscr{F}(y_2)$.
 In this work a {\em Lorentz-Finsler space} is just a pair $(M,\mathscr{F})$, otherwise called a {\em spacetime}.

 Lorentzian geometry is obtained with the choice
\begin{equation} \label{nia}
\mathscr{F}(y)=\sqrt{-g(y,y)} ,
\end{equation}
where $g$ is the Lorentzian metric. Its cones $C_x$ are said to be {\em round} since the intersection with an affine plane on $T_xM$ is an ellipsoid.

It must be mentioned that in standard Lorentz-Finsler theories the Finsler Lagrangian $\mathscr{L}=-\frac{1}{2}\mathscr{F}^2$ has Lorentzian vertical Hessian $g$ on $C$, $\mathscr{F}(\p C)=0$, and further regularity properties are demanded for $\mathscr{L}$, e.g.\ $g$ is continuous up to $\p C$. The Lorentz-Finsler spaces of this work are much more general, for instance we do not even assume $\mathscr{F}(\p C)=0$. However, we are left with the problem of introducing convenient regularity conditions on $\mathscr{F}$.

One of the key ideas of our work concerns the very definition of {\em closed Lorentz-Finsler space}. This is one instance of application of the {\em product trick}. We observe that $\mathscr{F}$ with its properties allows us to define a sharp convex non-empty cone at every tangent space of $M^\times=M\times \mathbb{R}$, which we give in two versions
\begin{align}
C^\times&=\{(y,z)\colon y\in C\cup \{0\}, \vert z\vert\le \mathscr{F}(y)\} \backslash\{0,0\} \\
C^\downarrow &=\{(y,z)\colon y\in C\cup \{0\}, z\le \mathscr{F}(y)\} \backslash\{0,0\} \label{ana}
\end{align}
The former version would seem the most natural since in Lorentzian geometry it produces a round cone, but it is actually the latter which will prove to be the most useful. Observe that $C^\downarrow$ is not strictly convex.

The main idea is that the cone structure $(M^\times,C^\downarrow)$ encodes all the information on the Lorentz-Finsler space $(M,\mathscr{F})$ so in order to get the best results for $(M,\mathscr{F})$ we have to impose those differentiability conditions on $\mathscr{F}$ which guarantee that $(M^\times,C^\downarrow)$ is a nice type of cone structure. So we impose that $(M^\times,C^\downarrow)$ is a closed cone structure, a condition which is  equivalent to  the upper semi-continuity of both $C$ and $\mathscr{F}$. By definition, with these conditions $(M,\mathscr{F})$ is a {\em closed Lorentz-Finsler space}. At this point, instead of working out new results for Lorentz-Finsler spaces we can just translate the known results for cone structures.
We shall follow this approach in the construction of steep time functions, but in the main paper one can find this idea applied in many directions, from the definition of causal geodesic to the proof of the notable singularity theorems.


A {\em time function} is a continuous function which increases over every causal curve. A {\em temporal function} is a $C^1$ function $t$ such that $\dd t(y)>0$ for every $y\in C$, hence a time function. Let $f\colon C\to [0,+\infty)$ be positive homogeneous. An $f$-steep function is a $C^1$ function $t$ such that
 \begin{equation}
\dd t(y)\ge f(y)
\end{equation}
for every $y\in C$. It is {\em strictly} $f$-steep if the inequality is strict. Since $\mathscr{F}\ge 0$, every strictly $\mathscr{F}$-steep function is temporal. Our main objective is to characterize those closed Lorentz-Finsler spaces which admit a strictly  $\mathscr{F}$-steep function. If $h$ is a Riemannian metric, with some abuse of terminology, we say that a function is $h$-steep if it is $\vert \cdot\vert_h$-steep.

The (Lorentz-Finsler) length of a continuous causal curve $x\colon [0,1]\to M$, $t \mapsto x(t)$, is \[\ell(x)=\int_0^1 \mathscr{F}(\dot x) \dd t\]
(it is independent of the parametrization).   The (Lorentz-Finsler) distance is defined by: for $(p,q) \notin J$,  $d(p,q)=0$, while for $(p,q)\in J$
\begin{equation}
d(p,q)=\textrm{sup}_x \ell(x) ,
\end{equation}
where $x$ runs over the continuous causal curves which connect $p$ to $q$.

It can be observed that if $x\colon [0,1]\to M$ is continuous causal then $x^\times\colon I \to M^\times$ given by
\[
x^\times(t)=(x(t), \int_0^t \mathscr{F}(\dot x(s)) \dd s)
\]
is continuous causal. Let us set  $x(0)=p$, $x(1)=q$, $P=(p,0)$, $Q=(q, \ell(x))$ then $Q$ is the endpoint of the continuous causal curve $x^\times$. Thus $d(p,q)$ is really an upper bound for the fiber coordinate over $(J^\downarrow)^+(P)\cap \pi^{-1}(q)$ where $J^\downarrow$ is the causal relation on  $(M,C^\downarrow)$, and $\pi$ is the projection  $M^\times\to M$. Notice that $d(p,q)$ would be the maximum if $J^\downarrow$ were a closed relation.

We write $C'>C$ if $C'$ is a $C^0$ cone structure such that $ C\subset \textrm{Int} C'$.
We say that $C$ is a {\em causal} closed cone structure if it does not admit closed continuous causal curves.
A {\em stably causal}  closed cone structure is one for which we can find $C'>C$ such that $C'$ is causal. It is by now well established that under stable causality the most useful causal relation is the Seifert relation
\[
J_S=\bigcap_{C'>C} J'
\]
since it is closed and transitive. Under stronger causality conditions, such as causal simplicity or global hyperbolicity $J$ is closed, a fact which implies that $J_S=J$. In fact, under stable causality $J_S$ is really the smallest closed and transitive relation which contains $J$. This result, conjectured by Low in 1996, and implicit in some of Seifert's works in the early seventies was proved by the author in \cite{minguzzi08b} at least for the $C^2$ Lorentzian theory. However, the proof was really  topological so we could generalize it to cone structures  \cite{minguzzi17}.
It must be said that in \cite{minguzzi17} the proof appears after the construction of the steep time functions, thus the order of presentation is reversed with respect to this work. In fact in that work we looked for the most  convenient proofs while here our goal is just that of  showing that our steep time function construction is natural and reasonable given the experience built on Lorentzian geometry.
 %given the intuition built on previous knowledge from Lorentzian geometry.

%Incidentally one might ask: why do we bother of the cone structure case? Why do we not work directly in a Lorentzian framework? The reader might be asking why do we bother of


The cone structure   $C^\downarrow$ is not round, in fact it is not even strictly convex, but it is this cone structure that will prove to be fundamental for our arguments. So it is natural to consider causality theory for non-round cone structures even if one's interest is in Lorentzian geometry.

%For this reason looking for result in Lorentzian geometry naturally leads one to

%Observe that no matter the conditions placed on spacetime, $C^\downarrow$ is not round, in fact it is not even strictly convex, when  it is this cone structure that will prove to be fundamental for our arguments.

Among the results which are known to hold in Lorentzian geometry \cite{minguzzi09c} and which survive in the cone structure case \cite{bernard16,minguzzi17} we can find the following: in a stably causal spacetime the Seifert relation can be represented with the set of smooth temporal functions, that is
\begin{equation} \label{rep}
(p,q)\in J_S\Leftrightarrow t(p)\le t(q) \ \textrm{ for every smooth temporal function } \ t.
\end{equation}

Let us return to the product trick. We know that a closed Lorentz-Finsler space is best seen as a cone structure on $M^\times$, so it is natural to ask what happens after  slightly opening the cones on $M^\times$. What is the geometrical meaning of $J^\downarrow_S$, the Seifert relation on $(M^\times, C^\downarrow)$? We have seen that $d(p,q)$ is not the maximum of the fiber coordinate on $(J^\downarrow)^+(P)\cap \pi^{-1}(q)$ just because $J^\downarrow$ is not closed. But $J^\downarrow_S$ is, and moreover it is contained in $J'{}^\downarrow$ for every opening of the cones  $C'{}^\downarrow>C^\downarrow$. Here this opening implies the enlargement of the cone, so $C'>C$, and that of the graphing function $\mathscr{F}$ thus $\mathscr{F}'>\mathscr{F}$. Actually, as a matter of notation, with the latter inequality we shall always include the validity of  the former.

It will therefore not come as a surprise that
\begin{equation} \label{vyv}
J_S^\downarrow=\{((p,r),(p',r'))\colon (p,p')\in J_S \textrm{ and } r'-r \le D(p,p')\}.
\end{equation}
where
the {\em stable distance} $D\colon M\times M\to [0,+\infty]$ is defined as follows. For $p,q\in M$
\begin{equation}
D(p,q)=\mathrm{inf}_{\mathscr{F}'>\mathscr{F}} d'(p,q),
\end{equation}
where $d'$ is the Lorentz-Finsler distance for the  Lorentz-Finsler space $(M,\mathscr{F}')$.

The closure and transitivity of $J_S^\downarrow$ are reflected in two important properties of $D$, namely upper semi-continuity and  reverse triangle inequality: for every $(p,q)\in J_S$ and $(q,r)\in J_S$
\[
D(p,r)\ge D(p,q)+D(q,r).
\]
It can be noticed that $D$ is better behaved than $d$ since the reverse triangle inequality has wider applicability. It can be shown that for globally hyperbolic cone structures $D=d$, but for less demanding causality conditions $D$, rather than $d$, should be regarded as the most convenient Lorentz-Finsler distance.
%The fact that $D=d$ in globally hyperbolic spacetimes can be understood as follows: in a globally hyperbolic spacetime $d$ is maximized and $J_S=J$ a fact which implies that $J^\times$ is given continuous

We say that a closed Lorentz-Finsler space is {\em stable } if it is stable {\em causally} and {\em metrically}, namely if it is stably causal and stably finite. By {\em stably finite} we mean that $d$ remains finite under small perturbations of the Finsler function $\mathscr{F}$, namely  there is $\mathscr{F}'>\mathscr{F}$ such that $d'$ is finite. Clearly, this condition implies that $D<+\infty$, but one of the main result of our work, a bit  lengthy to be discussed here, states that the converse is true: if $D<+\infty$ then $d$ is stably finite \cite{minguzzi17}.

Finally, we mention that globally hyperbolic spacetimes are stable no matter the choice of $\mathscr{F}$ while stably causal spacetimes are conformally stable.

\section{The distance formula}

Our goal is to prove the distance formula. Alain Connes in the early nineties proposed a strategy for the unification of all fundamental physical forces based on non-commutative geometry \cite{connes94}. A key ingredient was the {\em distance formula}, an expression for the Riemannian distance in terms of the 1-Lipschitz functions on spacetime. Unfortunately, the signature of the metric on spacetime is Lorentzian so Connes did not describe correctly the spacetime manifold: there was no causality or Lorentzian distance.

Parfionov and Zapatrin \cite{parfionov00} proposed to consider a more physical Lorentzian version and for that purpose they introduced the notion of steep time function which we already met.
% One of the main ingredients is the distance formula which allows one to recover the metric geometry of the basis by means of spectra triple, namely with
 %Less developed is the theory connected to the base geometry
%Steep time functions had first appeared in a problem
 %considered by .
 Let $d$ denote the Lorentzian distance,  and let $\mathscr{S}$ be the family of $\mathscr{F}$-steep time functions. The Lorentzian version of Connes' distance formula would be, for every $p,q\in M$
\begin{equation} \label{dap}
 d(p,q)=\mathrm{inf} \big\{[f(q)-f(p)]^+\colon \ f \in \mathscr{S}\big\}.
 %\ \textrm{is smooth and} \ \mathscr{L}\textrm{-steep} \}.
\end{equation}
where $c^+=\max\{0, c\}$.
%In the mathematical literature the expression on the right-hand side is also called the {\em intrinsic distance}.

Progress in the proof of the formula was made by Moretti \cite{moretti03} and Franco \cite{franco10} for globally hyperbolic spacetimes. However, in their versions the  functions appearing in (\ref{dap}) were not differentiable everywhere a fact which was a little annoying since in Connes' program the Dirac operator acts on them.
We proved the smooth version for stable spacetimes as part of the next more general theorem on the representability of spacetime by steep time function \cite{minguzzi17}.

\begin{theorem} \label{aas}
Let $(M,\mathscr{F})$ be a closed Lorentz-Finsler space and let $\mathscr{S}$ be the family of smooth strictly  $\mathscr{F}$-steep temporal functions. The Lorentz-Finsler space  $(M,\mathscr{F})$ is stable if and only if $\mathscr{S}$  is non-empty. In this case $\mathscr{S}$   represents
\begin{itemize}
\item[(a)] the order $J_S$, namely $(p, q)\in J_S \Leftrightarrow f(p)\le f(q), \ \forall f \in \mathscr{S}$;
\item[(b)] the manifold topology, namely for every open set $O\ni p$ we can find $f,h \in \mathscr{S}$ in such a way that
$p\in \{q\colon f(q)>0\}\cap \{q\colon h(q)<0\}\subset O$;
%In fact, we can do more, for every $\epsilon >0$ we can find $f$ and $h$ so that one has also the condition $\vert f\vert, \vert h\vert<\epsilon$ on $U$;
\item[(c)] the stable distance, in the sense that  the distance formula holds true: for every $p,q\in M$
\begin{equation}
 D(p,q)=\mathrm{inf} \big\{[f(q)-f(p)]^+\colon \ f \in \mathscr{S}\big\}.
 %\ \textrm{is smooth and} \ \mathscr{L}\textrm{-steep} \}.
\end{equation}
\end{itemize}
Moreover, {\em strictly} can be dropped.
\end{theorem}
As said above we shall concentrate on the distance formula, i.e.\ point (c). We have also a version for stably causal rather than stable spacetimes. However, in that version the steep functions have to be taken with codomain $[-\infty,+\infty]$ and steep just on the finite set.

Here the main idea is very simple and follows from Eq.\ (\ref{vyv}): the distance formula is a consequence of (\ref{rep}) applied to $(M^\times, C^\downarrow)$, thus as a first step we have to show that this product spacetime is stably causal (we shall return to it later on). At this point the level sets of temporal functions on $(M^\times, C^\downarrow)$ are really local graphs of strictly $\mathscr{F}$-steep functions on $M$. It is at this step that the shape of the cone $C^\downarrow$ becomes essential, since it is thanks to the fact that $C^\downarrow$ contains the fiber direction that the level set is a (univalued) graph, see Figure \ref{mkz}.



 \begin{figure}[ht]
\begin{center}
 \includegraphics[width=14cm]{funjoined}
\end{center}
\caption{The level set of a temporal function $\tau$ on $(M^\times, C^\times)$ and on $(M^\times, C^\downarrow)$. In the latter case, due to the shape of the cone, it is necessarily the local graph of a function $t$ which, furthermore, is strictly $\mathscr{F}$-steep: for every $y\in C$, $\dd t (y)>\mathscr{F}(y)$. The domain of $t$ might not be the whole $M$  since $t$ might blow up. This problem is resolved by constructing $\tau$ directly  through a modification of Hawking's averaging method and by using the condition $D<+\infty$.} \label{mkz}
\end{figure}


Let us see what are the implications of (\ref{rep}) on $(M^\times, C^\downarrow)$.  Let $P=(p,r)$ be a generic point, let $q\in J_S^+(p)$ and let $Q=(q,r')=(q,r+D(p,q))$. By  Eq.\ (\ref{vyv}) $(P,Q)\in J_S^\downarrow$. By (\ref{rep}) if $\tau$ is  temporal on $(M,C^\downarrow)$, $\tau(P)\le \tau(Q)$ which means that $Q$ must stay in the subgraph of the level set of $\tau$ passing through $P$. In other words $t(q)-t(p)\ge r'-r=D(p,q)$, see Fig.\ \ref{sxe}. But for every strictly $\mathscr{F}$-steep function $t\colon M\to \mathbb{R}$ we can find a temporal function $\tau(P)=t(p)-r$ which has the graph of $t$  as level set. Thus the previous argument shows that for every strictly $\mathscr{F}$-steep function $t$, $D(p,q)\le t(q)-t(p)$. We now show that the infimum of the right-hand side is really $D$, thus concluding the proof. In fact (\ref{rep}) states that the temporal functions separate points not belonging to the Seifert relation. Let $Q_\epsilon=(q,r+D(p,q)+\epsilon)$, for $\epsilon>0$, then by  Eq.\ (\ref{vyv}) $(P,Q_\epsilon)\notin J_S^\downarrow$. As a consequence, there is a temporal function $\tau$ on $(M^\times, C^\downarrow)$ which separates them in the sense that $\tau(Q_\epsilon)<\tau(P)$, which implies that the graphing function of the $\tau$-level set passing through $P$ satisfies $t(q)-t(p) \le D(p,q)+\epsilon$, see Fig.\ \ref{sxe}. Due to the arbitrariness of $\epsilon$ we get the distance formula for $(p,q)\in J_S$.

 \begin{figure}[ht]
\begin{center}
 \includegraphics[width=10cm]{boljoined}
\end{center}
\caption{Since $(P,Q_\epsilon)\notin J_S^\downarrow$, there is a temporal function $\tau\colon M^\times \to\mathbb{R}$ which separates $P$ and $Q_\epsilon$ in the sense that $\tau(Q_\epsilon)<\tau(P)$. The figure displays the level set of $\tau$ passing though $P$.} \label{sxe}
\end{figure}

Actually, the function $t$ might not be defined everywhere but it turns out that it is sufficient that there exists {\em one} (global) strict $\mathscr{F}$-steep function to prove that such functions separate points not belonging to $J_S^\downarrow$. So we are left with two problems
\begin{itemize}
\item[(i)] We have yet to prove that  $(M,C^\downarrow)$ is stably causal, a fact assumed when we made use of  property (\ref{rep}) on $(M^\times,C^\downarrow)$. This result can be proved constructing directly a time function $\tau$ on $(M,C^\downarrow)$. In fact, we can do this with a sort of Hawking's averaging method.
\item[(ii)] We have  to show that this function $\tau$, suitably smoothed, gives a temporal function on $(M,C^\downarrow)$ whose level sets intersects every $\mathbb{R}$-fiber (so that the graphing function $t$ of a level set  is globally defined).
\end{itemize}
While $(i)$ can be accomplished under stable causality of $(M,C)$, (ii) holds only if $(M,C)$ is stable. In fact, we know that the inequality
$D(p,q)\le t(p)-t(q)$ holds for every strictly $\mathscr{F}$-steep function $t$, thus the existence of just one such function implies that $D$ is finite.

Concerning step $(i)$ we recall that Hawking's averaging method \cite{hawking68,hawking73} applied to $(M^\times,C^\downarrow)$ would consist in the introduction of a positive normalized measure $\mu$ on $M^\times$, absolutely continuous with respect to the Lebesgue measure of any chart, in the introduction of a one-parameter family of cones $C^\downarrow_a{}$, $a\in [0,3]$, $C^\downarrow_a<C^\downarrow_b$, for $a<b$, $C^\downarrow_0=C^\downarrow$, and in the definition of
\[
\tau^\downarrow(P)=-\int_1^2 \mu\big(J_{C^\downarrow_a}^+(P)\big) \dd a .
\]
This construction gives a time function on $(M^\times,C^\downarrow)$ provided this spacetime is stably causal. In fact one would have to choose $C^\downarrow_3$ (stably) causal. Unfortunately, we do not know if $C^\downarrow$ is stably causal. Instead, we construct the function $\tau^\downarrow$ using a different choice for $C^\downarrow_a$ which is causal by construction, namely $C^\downarrow_a$ is built from $(C_a,\mathscr{F}_a)$ according to the analog of Eq.\ (\ref{ana}) where the one-parameter family $(C_a,\mathscr{F}_a)$ is defined for   $a\in [0,3]$, and satisfies: $C_a<C_b$ and $\mathscr{F}_a<\mathscr{F}_b$, for $a<b$, $C_0=C$ and $\mathscr{F}_0=\mathscr{F}$, $C_3$ is stably causal. In other words, the cones $C^\downarrow_a$ do not open in the direction of the $\mathbb{R}$-fiber which for all of them remains lightlike. They are causal because with this definition the projection of a continuous $C_a^\downarrow$-causal curve is a continuous $C_a$-causal curve (unless it coincides with a fiber), and so there cannot be a closed causal curve on $M^\times$ as there would be one on the base, which is impossible as $C_3$ is stably causal.

With this definition $\tau^\downarrow$ is clearly increasing over continuous $C^\downarrow$-causal curves since the  argument of the integral is for every $a$.
What is puzzling is the fact that $\tau^\downarrow$ is continuous despite the fact that the cones are not opened as in Hawking's original prescription. This point cannot be understood intuitively but it is a consequence of two facts: (a) the invariance under the fiber translations of the cone structure $C^\downarrow$, (b) the fact that $C^\downarrow$ projects to a {\em sharp} cone $C$. Notice that the latter property would not hold if $C^\downarrow$ were round (Lorentzian), since the projection would be a half-space.

 Coming to step (ii) the function $\tau$ can be smoothed thanks to a powerful result \cite{minguzzi17} which improves a previous result by Chru\'sciel, Grant and the author \cite[Th.\ 4.8]{chrusciel13}. I recall the theorem though I shall not enter into the details of its use (one can apply it directly to the function $\tau$ on $M^\times$ or to the graphing function $t$ on $M$).
\begin{theorem} \label{moz}
  Let $({ M},C)$ be a closed cone structure and
 let $\tau\colon M\to \mathbb{R}$ be a continuous function. Suppose that there is a $C^0$ proper cone structure $\hat C>C$ and continuous functions  homogeneous of degree one in the fiber $\underline F, \overline F\colon \hat C\to \mathbb{R}$ such that for every  $\hat C$-timelike  curve $x\colon [0,1]\to M$
 \begin{equation} \label{mos}
 \int_x \underline F(\dot x) \dd t\le \tau(x(1))-\tau(x(0))\le \int_x \overline F(\dot x) \dd t.
 \end{equation}
 Let $h$ be an arbitrary Riemannian metric, then for every function $\alpha\colon { M} \to (0,+\infty)$ there exists
a smooth  function $\hat{\tau}$ such that $\vert \hat\tau-\tau\vert <\alpha$ and for every $v\in C$
\begin{equation} \label{kid}
\underline F(v)- \Vert v\Vert_h \le \dd \hat \tau(v) \le \overline F(v)+ \Vert v\Vert_h .
\end{equation}
Similar versions, in which some of the functions $\underline F, \overline F$ do not exist hold true. One has just to drop the corresponding inequalities in (\ref{kid}).
\end{theorem}

More interesting, and peculiar to the proof, is how we solved the problem of constructing $\tau$ in such a way that its level sets intersect every $\mathbb{R}$-fiber. Here the idea comes from Geroch's original construction of the time function on globally hyperbolic spacetime \cite{hawking73}. Basically, we repeat the construction above  with the opposite cones, thus obtaining a time function
\[
\tau^\uparrow(P)=\int_1^2 \mu\big(J_{C^\downarrow_a}^-(P)\big) \dd a .
\]
Then we define $\tau=\log \vert{\tau^\uparrow}/{\tau^\downarrow} \vert$. If we can show that $\tau^\downarrow\to 0$ moving to the future along a fiber (the downward direction in our figures) and that $\tau^\uparrow \to 0$  moving to the past along a fiber (the upward direction in our figures) then $\tau \to +\infty$ in the former limit and $\tau \to -\infty$ in the latter, thus by continuity the level set $\tau=0$ intersects every fiber. The condition $D<+\infty$ guarantees precisely the mentioned limits. Indeed if $K\subset M$ is a compact set then $J_{C^\downarrow_a}^+(P)\cap \pi^{-1}(K)$ will be upper bounded in the extra coordinate, with bound going to $-\infty$ as the fiber coordinate of $P$ goes to $-\infty$. This fact should not totally come as a surprise since as we learned in Fig.\ \ref{sxe}, the function $D$ controls how much we can go up in the extra-coordinate with respect to the starting point.


Our discussion of points (i) and (ii) finishes our exposition of the proof that under the stable condition $D<+\infty$ there are (global) strictly $\mathscr{F}$-steep time functions, which was the missing step in our argument for proving the distance formula.

Incidentally the existence of  $\mathscr{F}$-steep functions is important in the  problem of embedding Lorentzian manifolds into Minkowski spacetime, as shown by M\"uller and S\'anchez \cite{muller11}. As a consequence we also proved a long sought for characterization of the Lorentzian submanifolds of Minkowski spacetime. According to it the Lorentzian submanifolds are precisely the stable spacetimes \cite{minguzzi17}, or more precisely

\begin{theorem}
Let $(M,g)$ be a $n+1$-dimensional Lorentzian spacetime  endowed with a $C^{k}$, $3\le k\le \infty$, metric. $(M,g)$ admits a $C^k$ isometric embedding in Minkowski spacetime $E^{N,1}$, for some $N>0$, if and only if $(M,g)$ is stable.
%A  spacetime $(M,g)$ admits an isometric embedding into  $E^{N,1}$ for some $N>0$ if and only if it is  stably causal and such that ,
\end{theorem}

As for the optimal conditions for the validity of the distance formula with $D$ replaced by $d$, i.e.\ that originally suggested by Parfionov and Zapatrin, we found that in Lorentzian manifolds endowed with $C^1$ metrics the formula holds if and only if the spacetime is causally continuous and the Lorentz-Finsler  distance is finite and continuous \cite{minguzzi17}. In particular, globally hyperbolic spacetimes are of this type and in this case the family of functions can be further restricted, so for instance, the functions can be taken Cauchy.


%Notice that thanks to the product trick the distance formula, which is a metric result, will follow from a causality result on the product spacetime.



%\newpage


\section{A second proof}

In this section we give a different proof of  the smooth distance formula for Lorentzian globally hyperbolic spacetimes endowed with $C^{2,1}$ metrics. The proof might be adapted to the Lorentz-Finsler case and to  weaker regularity assumptions, however there seems to be no point in pursuing this direction since the strategy of the previous section really allowed us to prove a stronger version  under much weaker conditions.

The approach of this section was really the first one employed by the author to prove the distance formula. Though it has the limitation that it does not give the optimal conditions for the existence of smooth steep time functions,  it  is otherwise perfectly fine if one is interested in globally hyperbolic spacetimes endowed with sufficiently regular metrics.

The proof uses Theorem \ref{moz}, a previous construction of steep functions by the author  \cite{minguzzi16a},  stability results for Cauchy temporal functions  and Lorentzian distance \cite{minguzzi17}, and a previous non-differentiable version of the distance formula proved by Franco \cite{franco10} (thus it is certainly not self contained). For shortness the final step in the proof assumes familiarity with Franco's paper and more generally with
%takes its course where Franco  left off, so in what follows the reader is assumed to be familiar with
the details of \cite{minguzzi16a,franco10}.  A {\em Cauchy} time function is a time function whose restriction to any inextendible continuous causal curve has image  $\mathbb{R}$. In the next theorems $\mathscr{F}$ is as in Eq.\ (\ref{nia}).




\begin{theorem} \label{spa}
Let $(M,g)$ be a globally hyperbolic spacetime and let $h$ be any Riemannian metric on $M$. There is  a smooth  $h$-steep Cauchy  temporal function $\tau$.
Moreover, there is  a smooth $h$-steep Cauchy  temporal function which is $\mathscr{F}$-steep.
\end{theorem}

This theorem was also obtained by Suhr in a recent work  using different methods \cite[Th.\ 2.3]{suhr15}.

\begin{proof}
 The proof of the existence of a smooth Cauchy $\mathscr{F}$-steep time function in globally hyperbolic spacetimes can be found in \cite{minguzzi16a} (for a previous proof, see \cite{muller11}). One can prove more. Since global hyperbolicity is stable \cite{minguzzi11e,fathi12} one can find a smooth Cauchy steep time function $\tau'$ for a Lorentzian Finsler function $\mathscr{F}'$ (cf.\ (\ref{nia})) with larger cones $C'>C$, such that the indicatrix $\mathscr{F}'{}^{-1}(1)$ of $\mathscr{F}'$ does not intersect that of $\mathscr{F}$. So for every $v \in C_x$,  $\dd \tau'(v) \ge   \mathscr{F}'(v)$ with the bonus that now $\mathscr{F}'$ does not vanish on lightlike vectors, i.e.\ on $\p C_x$.


%This fact brings back to the anti-Lipschitz property in a differentiable version.
Let $h$ be any auxiliary Riemannian metric. For every $x\in M$, at $T_xM$ we can shrink through homotheties the indicatrix of $\mathscr{F}'$, so redefining this function but not its cone, in such a way that its indicatrix intersected with $C_x$ is contained in $B^h(x,1)$ (the open unit ball centered at $x$ with respect to the distance induced by $h$). As a consequence, for the Cauchy $\mathscr{F}'$-steep time  function $\tau'$ we have the inclusion $\{v: \dd \tau'(v)=1\}\cap C_x \subset B^h(x,1)$, which implies $\dd \tau'(v) \ge  \mathscr{F}'(v)  \ge \Vert v \Vert_h $ for every $v\in C$. But the fact that the indicatrix of $\mathscr{F}'$ does not intersect that of $\mathscr{F}$ also implies $ \mathscr{F}'(v) \ge   \mathscr{F}(v)$ for every $v\in C$. Thus with the redefinition $\tau' \to \tau$ we get the desired result.
\end{proof}



\begin{theorem} \label{oor}
Let $h$ be an auxiliary Riemannian metric.
In a globally hyperbolic spacetime $(M,g)$ both topology and order can be recovered from the set  $\mathscr{V}$ of smooth Cauchy $h$-steep time  functions. That is:
\begin{itemize}
\item[(a)]
 $(x,y)\in J \Leftrightarrow t(x)\le t(y)$, for every $t\in \mathscr{V}$;
 \item[(b)]  for every open set $O\ni p$ we can find $f,h \in \mathscr{V}$ in such a way that
$p\in \{q\colon f(q)>0\}\cap \{q\colon h(q)<0\}\subset O$.
\end{itemize}
\end{theorem}

\begin{remark} \label{mja}
Thus topology and order can be recovered from the set of smooth  Cauchy $\mathscr{F}$-steep temporal  functions. In fact, if we take $h$ so that its balls do not intersect the indicatrices of $\mathscr{F}$ we have that every smooth Cauchy $h$-steep time  function is a smooth  Cauchy $\mathscr{F}$-steep temporal  function. Notice that $h$ can be chosen complete.
\end{remark}

\begin{proof}
Let $q\notin J^+(p)$ then the spacetime  $N=M\backslash (J^+(p)\cup J^{-}(q))$ is globally hyperbolic. Any Cauchy hypersurface $S$ for $N$ is also  a Cauchy hypersurface for $M$ with $p\in I^+(S)$ and $q\in I^{-}(S)$.
The Geroch topological splitting theorem \cite{hawking73} implies that we can find a Cauchy time function $t$ for $M$ so that $S=t^{-1}(0)$, $t(p)>1$, $t(q)<-1$ (for instance, apply the theorem to $M\backslash J^{-}(S)$ and to $M\backslash J^{+}(S)$  and reparametrize the level sets so obtained into a global Cauchy time function). Let $S_t$ be its constant slices. Inspection of the proof in \cite{minguzzi16a} shows that we can find a smooth Cauchy $\mathscr{F}$-steep temporal function  $t'$ such that  $S'_0:=t'^{-1}(0)\subset t^{-1}([-1,1])$, $t'>t$ for $t>1$, and $t'<t$ for $t<-1$.
%In particular, $S'_0$ is a Cauchy hypersurface and $t'$ is a  temporal, hence stably temporal, function. 
An improvement \cite[Th.\ 47]{minguzzi17} over the classical stability result for global hyperbolicity \cite{minguzzi11e,fathi12,samann16,minguzzi17} states that the Cauchy temporal functions are stable so
we can widen the cones while preserving global hyperbolicity and the Cauchy property of $S'_0$. Let $\mathscr{F}'$ be a Lorentzian Finsler function for a wider cone structure, $C'>C$,  chosen so that the indicatrix $\mathscr{F}'{}^{-1}(1)$ intersected with $C$ is contained in the unit ball of $h$, and hence so that any $\mathscr{F}'$-steep
time function $f$ satisfies $\dd f(v)> \Vert v \Vert_h $ for $v\in C$. Repeating the  above argument we find a smooth Cauchy $\mathscr{F}'$-steep temporal function  $t''$ such that  $S''_0\subset {t'}^{-1}([-1,1])$, $t''>t'$ for $t'>1$, and $t''<t'$ for $t'<-1$.  In particular $t''(p)>1>0>-1>t''(q)$. This result proves the representability of the order by smooth $h$-steep Cauchy time functions.

%The mentioned family of $C^1$ functions can be replaced by the analogous  smooth ones. Indeed, by \cite[Th.\ 2.6]{hirsch76} $C^\infty(M,\mathbb{R})$ is dense in
%$C^1(M,\mathbb{R})$ endowed with the Whitney strong topology
%\cite[p.\ 35]{hirsch76}, so we can approximate $t''$ with a function $\tau\in
%C^\infty(M,\mathbb{R})$, up to the first
%derivative, as accurately as we want over $M$. In particular, we can
%find $\tau$ such that $\vert \tau-t''\vert<1$ and $\dd\tau(v)> \Vert v \Vert_h $ for $v\in C$, where the former inequality implies that
%$\tau$ is Cauchy and that $\tau(p)>\tau(q)$, and the latter inequality implies that $\tau$ is
%$h$-steep.


For the topology, let $p\in O$, $O$ open, and let $t$ be a  smooth Cauchy $h$-steep time function such that $t(p)=0$, cf.\ Th.\ \ref{spa}. Let $S_0=t^{-1}(0)$ and let $Q\subset O$ be a compact neighborhood of $p$ such that $Q=J^+(Q)\cap J^{-}(S_0)\cup J^-(Q)\cap J^{+}(S_0)$, and let $\varphi$ be a positive smooth function  supported on $Q$.  Let $\tau^{-}_\varphi$ be the  volume function \cite{chrusciel13}. We know that $\tau^{-}_\varphi$ is $C^1$ with past directed timelike gradient  wherever $E^{-}(q)$ intersect the locus $\varphi>0$ and with a vanishing differential otherwise.
%Moreover, it is locally anti-Lipschitz  wherever the gradient is different from zero.
Thus $f:=t+ \tau^{-}_\varphi$ is a $C^1$   Cauchy $h$-steep time function. Notice that the locus $f=0$ coincides with $t=0$ outside $Q$, $f(p)>0$ and $f>0$ only for $t>0$ or inside $Q$. Inverting the time orientation we get a  $C^1$   Cauchy $h$-steep time function $u$ such that $u=0$ coincides with  $t=0$ outside $Q$, $u(p)<0$ and $ u<0$ only for $t<0$ or inside $Q$. Thus $p\in \{q\colon f(q)>0\}\cap \{q\colon u(q)<0\}\subset O$. The smooth version of this inclusion is due to the density of $C^\infty(M,\mathbb{R})$  in
$C^1(M,\mathbb{R})$, see \cite[Th.\ 2.6]{hirsch76}.
\end{proof}



\begin{theorem} \label{nug}
Let $(M,g)$ be a globally hyperbolic spacetime and let $\mathscr{U}$ be the family of smooth, Cauchy, $\mathscr{F}$-steep temporal functions which are $h$-steep for some complete Riemannian metric $h$ (dependent on the function). We have the identity
 %and let %$(p,q)\in J$, then
\begin{equation} \label{dsf}
d(p,q)=\inf\{[f(q)-f(p]^+\colon  f \in \mathscr{U}\}
.\end{equation}
% $(p,q)\in J$, then $[\, ]^+$ on the right-hand side can be dropped.
\end{theorem}
%\begin{remark}
%Whenever an equation such as  (\ref{dsf}) holds with $\mathscr{U}$ replaced by a set $\mathscr{V}$ we say that $\mathscr{V}$ represents or encodes the Lorentz-Finsler metric.
%From  Theorem \ref{oor} we know that  $\mathscr{U}$  encodes the order and the topology. This theorem proves that it encodes also the Lorentz-Finsler metric. Of course, any family of $C^1$ and $\mathscr{L}$-steep functions which includes  $\mathscr{U}$ has these same properties.
% %In general, the smaller the family the stronger the theorem though the members of the family must be isotone.
%\end{remark}
%The Lorentzian version of this formula, for functions which were just $C^1$ and $\mathscr{L}$-steep, was suggested by Parfionov and Zapatrin \cite{parfionov00}. They did not suggest that the formula could hold  in globally hyperbolic spacetimes, though. Moretti \cite[Theor.\ 2.2]{moretti03} proved a version for globally hyperbolic spacetimes in which the functions on the right-hand side are $\mathscr{L}$-steep almost everywhere and only inside some compact set, not being defined outside the compact set.
%
% For globally hyperbolic spacetimes  the most interesting version so far available is due to Franco \cite[Theor.\ 1]{franco10}. It holds on globally hyperbolic spacetimes and on the right-hand side one finds  globally defined continuous causal functions differentiable and steep almost everywhere.
% %Unfortunately, the differentiability properties of continuous causal functions are not straightforward, and
%
%The functions appearing on the right-hand side of (\ref{dsf})  are  $C^1$, so our version is interesting already in the Lorentzian case, as the distance formula implies that originally proposed by Parfionov and Zapatrin.


\begin{proof}
Suppose first that $(p,q)\notin J$ so that $d(p,q)=0$. We know from Remark \ref{mja} that the functions in $\mathscr{U}$ represent the order, so there is $f\in \mathscr{U}$   such that $f(p)> f(q)$, thus $d(p,q)=0=[f(q)-f(p)]^+$.

Now, suppose that  $(p,q)\in J$.
Let $f$ be smooth and $\mathscr{F}$-steep.
Given a continuous causal curve $x\colon [0,1] \to M$ connecting $p$ to $q$,
\[
f(q)-f(p)=\int_x \dot f  \dd t=\int_x \dd f(\dot x)  \dd t \ge \int_x \mathscr{F}(\dot x)  \dd t= \ell(x).
\]
Thus taking the supremum over the connecting continuous causal curves we get $d(p,q) \le f(q)-f(p)= [f(q)-f(p)]^+$, and taking the infimum over the family of functions we obtain  the inequality $\le$.

For the other inequality we have to show that for every $\epsilon>0$ we can find a smooth Cauchy  $\mathscr{F}$-steep temporal function  $f$ such that $f(q)-f(p) \le d(p,q)+\epsilon$.
We can enlarge the cones while preserving global hyperbolicity. In particular, we can find a Lorentzian $\hat{\mathscr{F}}$ globally hyperbolic, with $\hat C>C$  such that the indicatrices inside the cones do not intersect and $d(p,q) \le \hat{d}(p,q)\le d(p,q)+ \epsilon/4$. The first inequality is clear, while the second inequality follows from \cite[Th.\ 58,59,61]{minguzzi17}.
%this continuity result can be obtained from the Lorentz-Finsler version of the limit curve theorems \cite[Theor.\ 2.13]{minguzzi07c}; alternatively

%In order to get the last inequality let us introduce an auxiliary Riemannian metric $\gamma$ in a compact neighborhood $K$ of $J^+(p)\cap J^{-}(q)$, and choose $\hat{\mathscr{F}}$ so as to approximate uniformly $\mathscr{F}$ on $\hat C \cap S^{n}(\gamma)$ where  $S^{n}(\gamma)$ is the  $\gamma$-unit sphere bundle over the compact set $J^+(p)\cap J^{-}(q)$. Finally, observe that the $\gamma$-length of causal curves connecting  $p$ to $q$ is bounded \cite[Lemma 2.7]{samann16} so that for every $\hat C$-connecting curve $x$ we get a uniform approximation $\vert \hat\ell(x)-\ell(x)\vert\le\epsilon/4$ independent of $x$.

Let $K$ be a compact neighborhood of $\hat J^+(p)\cap \hat J^{-}(q)$.
Notice that apart from the definition of the cone $\hat C$, so far  $\hat{\mathscr{F}}$ is unconstrained outside $K$. Let $h'$ be a complete Riemannian metric. We choose the indicatrices $\hat{\mathscr{I}}$ of $\hat{\mathscr{F}}$ outside $K$ to be so close to the origin  that their intersection with $C$ is contained in the unit sphere bundle of $h'$, that is $\hat{\mathscr{F}}(v)>\vert v\vert_{h'}$ for every $v\in C$.

Notice that since the indicatrices $\hat{\mathscr{I}}$ and $\mathscr{I}$ do not intersect we can find a Riemannian metric $h$ so small that for every $v\in  C$, $\hat{\mathscr{F}}(v)\ge {\mathscr{F}}(v)+2 \vert v\vert_h$ and outside $K$ the unit balls of $h$ contain those of $h'$ of radius 2, that is $\vert \cdot \vert_{h'}>2\vert \cdot \vert_{h}$ outside $K$.
%Notice that we can alter $h'$ enlarging its balls in a compact neighborhood of $D'$, preserving all its previous properties, and so as obtain that everywhere the unit balls of $h$ contains

Let us apply Franco's construction to $(M, \hat{\mathscr{F}})$. Inspection of his proof shows that he constructs a continuous function $\hat f$ such that $\hat f(q)-\hat f(p) <\hat{d}(p,q)+\epsilon/4$ by means of a (locally finite) sum of functions of the form $\hat{d}_r^+(\cdot):=\hat d(r,\cdot)$, $\hat{d}_r^-(\cdot):=-\hat d(\cdot, r)$, where $r$ runs over a countable set of points $\{r_i\}$. Any point of the manifold belongs to the support of one of these functions. As a consequence, for every $(a,b)\in \hat J$, Franco's function satisfies $\hat f(b)-\hat f(a)\ge  \hat\ell(x)$ where $x$ is any $\hat C$-causal curve connecting $a$ and $b$ (just partition the causal curve so that each part belongs to the support of a function $\hat{d}_r^+$ or $\hat{d}_r^-$ and use the reverse triangle inequality for $\hat d$).

We can use Theorem \ref{moz} with $\underline F=\hat{\mathscr{F}}$,  thus we can find $f$ smooth such that $\vert  f-\hat f\vert<\epsilon/4$ and $\dd  f(v) \ge \hat{\mathscr{F}}(v)-\vert v\vert_h\ge \mathscr{F}(v)+\vert v\vert_h$ on every $C$-causal vector $v$. In particular, $f$ is
$h$-steep and
$\mathscr{F}$-steep. But we have also  $\hat{\mathscr{F}}(v)-\vert v\vert_h\ge  \vert v\vert_{h'}-\vert v\vert_h \ge \frac{1}{2} \vert v\vert_{h'}$ outside $K$, thus  $f$  is $h'/4$-steep outside $K$ and  $h$-steep inside $K$, which implies that $f$ is $h''$-steep for some complete Riemannian metric $h''$, and hence Cauchy.
Finally,
\begin{align*}
 f(q)- f(p)&\le  \hat f(q)-\hat f(p)+\vert \hat f(q)-f(q)\vert + \vert \hat f(p)-f(p)\vert \le \hat{d}(p,q)+\tfrac{3}{4}\epsilon\\
&\le  {d}(p,q)+\epsilon.
\end{align*}
\end{proof}
%Next let $g$ be a  smooth Cauchy temporal function with codomain $\mathbb{R}$ and whose absolute value is bounded by 1 over the compact set $J^+(p)\cap J^{-}(q)$, then $f=\hat f+\frac{1}{6}\epsilon g$ is  $h$-steep for a complete Riemannian metric $h$, $\mathscr{L}$-steep, Cauchy and satisfies $ f(q)-f(p)\le {d}(p,q)+\epsilon$.
%Finally, a smooth Cauchy time function $\tau$ exists, and we define $ f=\hat f+ \frac{\epsilon}{5 \Delta}\tau$ where $\Delta$ is the difference between maximum and minimum of   $\tau$ over $J^+(p)\cap J^{-}(q)$.
 %
%
%Now, the indicatrix $\mathscr{I}'$ intersects $\hat C$ on a bundle $P\subset TM$ with compact fibers, thus we can find an auxiliary Riemannian metric $h$ such that $P\subset B^h(1)$. As a consequence, for every  $\hat C$-causal curve $\ell'(x) \ge \ell^h(x)$. The inequality $f(b)-f(a)\ge \ell^h(x)$ proves that $f$ is locally $h$-anti-Lipschitz with respect to $(M,\hat C)$. By Corollary \ref{soo} there is a smooth function $\hat f$, such that $\vert \hat f-f\vert<\epsilon/3$ and $\dd \hat f(v) \ge (1-\epsilon)\vert v\vert_h$ for every $C$-causal vector $v$.


%For the other inequality, we first notice that in a globally hyperbolic Lorentz-Finsler spacetime   the Lorentzian distance is continuous and finite, and the Avez-Seifert theorem on the geodesic connectedness of globally hyperbolic spacetimes still holds true  \cite{minguzzi15}. Let $\gamma$ be a geodesic segment connecting $p$ to $q$ such that $\ell(\gamma)=d(p,q)$. Let $h$ be a complete Riemannian metric, let $K$ be a causally convex compact neighborhood of $\gamma$ and let $\mathscr{L}^s$, $\mathscr{L}^0=\mathscr{L}$, $s\in [0,1]$, be a  1-parameter family of Lorentz-Finsler Lagrangians such that (a) the timelike cones of $\mathscr{L}^{s'}$ contain the causal cone of $\mathscr{L}^{s}$, for $s<s'$; (b) $(M,\mathscr{L}^1)$ is globally hyperbolic (it exists by the stability of global hyperbolicity); (c) for every $s\ne s'$ the indicatrices do not intersect, $\mathscr{I}^s\cap \mathscr{I}^{s'}=\emptyset$; (d) for every $p\notin K$, $\mathscr{I}^s\cap C\subset B^h(1)$; (e) the dependence of the cone bundles $C^s$ on $s$ is smooth, (f) the dependence of $\mathscr{L}^s$
%
%
% let $\tau$ be a steep time function for a Lorentz-Finsler Lagrangian $\mathscr{L}'$ such that (a) the timelike cones of  $\mathscr{L}'$ contain the causal cones of $\mathscr{L}$, (b) the indicatrices do not intersect. The
%$\tau$ is
%any $\mathscr{L}'$-steep is


%
%One might ask whether global hyperbolicity is necessary in this type of distance formulas. Clearly, the formula can hold only if the Lorentz-Finsler distance is finite and it is known that any causal structure is compatible with the finiteness or even boundedness of the Lorentz-Finsler distance provided it is strongly causal \cite[Lemma 2.3]{minguzzi08e}.
%Rennie  and Whale  gave a version with no causality assumption \cite{rennie16}, however the family of functions on the right-hand side of their Lorentz distance formula includes  discontinuous functions. In order to have any chance to represent  also the topology, the representing functions must be continuous, and under this condition the causality condition in the distance formula cannot be too weak, as the next result shows
%\begin{proposition} \label{sqw}
%Suppose that a Lorentz-Finsler distance formula holds for a spacetime $(M,\mathscr{L})$, where the representing functions are continuous. Then the Lorentz-Finsler distance is finite and continuous and the spacetime is reflecting. So $(M,\mathscr{L})$ is chronological and if it is distinguishing then it is causally continuous. Finally, if it is known that the representing functions are $C^1$ and  $\mathscr{L}$-steep, then $(M,\mathscr{L})$ is causally continuous.
%\end{proposition}
%
%\begin{proof}
%Finiteness is obvious and it implies chronology. Let us come to continuity. Since the   functions $f$ are continuous then $g(p,q)=[f(q)-f(p)]^+$ is continuous on $M\times M$ and so the infimum of  a set of such functions is upper semi-continuous, which implies that $d$ is upper semi-continuous. But we know that
% the Lorentz-Finsler distance function is lower semi-continuous, hence continuous. Finally reflectivity must hold, otherwise we can find $p,q$ such that $q \in \bar{I}^+(p)$ but $p \notin \bar{I}^{-}(q)$ (or dually). Let $\gamma_n$ be causal curves starting from $p$ with endpoint $q_n \to q$. Taken $r \ll p$ so that $r \notin  \bar{I}^{-}(q)$, we have $d(r,q)=0$ but if $\sigma$ is a timelike curve connecting $r$ to $p$ we have $d(r,q_n)\ge l(\sigma)>0$, so $d$ is not continuous as can be seen taking the limit $(r, q_n)\to (r,q)$  (see also \cite[Theor.\ 4.24]{beem96}).
%
% For the last statement, the family of representing functions must be non-empty, otherwise $d=0$ for every pair of points, which is impossible. Let $f$ be an element of the family. It increases over causal curves cf.\ Prop.\ \ref{chg} so it is a time function and hence the spacetime is stably causal. By the previous case it is causally continuous.
%\end{proof}

%So the Lorentz-Finsler distance has to be particularly well behaved. Global hyperbolicity assures these properties.
%
%
%In fact, global hyperbolicity is in a sense the optimal causality condition. Any attempt at weakening the causality condition while preserving the continuity of the representing functions is doomed to failure, as  we have the following
%
%\begin{theorem}
%Suppose to have been given a general  Lorentz-Finsler distance formula which applies to spacetimes satisfying a certain causality condition $P$ stronger than non-total imprisonment.
% If the  formula makes use of a family of {\em continuous} functions then $P$ is ``global hyperbolicity''.
%\end{theorem}
%
%\begin{proof}
%Suppose that such a formula exists. Let $(M,\mathscr{L})$ be a Lorentz-Finsler spacetime which satisfies $P$, and let $(M,\mathscr{L}')$ be any other element on the same conformal class, that is the Finsler Lagrangian differ by a factor $\varphi\colon M\to (0,+infty)$. Since the causality conditions depend only on the cone structure, $(M,\mathscr{L}')$ satisfies $P$.
%
%Let us consider the Lorentz-Finsler distance $d'$ of  $(M,\mathscr{L}')$. By the formula, since the   functions $f$ are continuous then $g(p,q)=[f(q)-f(p)]^+$ is continuous on $M\times M$ and so the infimum of  a set of such functions is upper semi-continuous, which implies that $d'$ is upper semi-continuous. But we know that
% the Lorentz-Finsler distance function is lower semi-continuous, hence continuous. A result by the author \cite[Theor.\ 3.6]{minguzzi08e}, developed in the Lorentzian case, but once again with a proof that passes unaltered to the Lorentz-Finsler case, tells us that if a Lorentz-Finsler space is non-total imprisoning and has a continuous distance function for every conformal choice in his class, then it is globally hyperbolic. Thus $P$ implies global hyperbolicity.
%%We recall that a Lorentz-Finsler spacetime is globally hyperbolic if and only if  the Lorentz-Finsler distance is continuous \cite{beem96}. But if functions $f$ are continuous then $g(p,q)=[f(q)-f(p)]^+$ is continuous on $M\times M$ and so the infimum of  a set of such functions is upper semi-continuous. But the Lorentz-Finsler distance function is lower semi-continuous, hence continuous, so the spacetime is globally hyperbolic.
%\end{proof}
%
%\subsection{ A novel distance}
%
%We have seen that it is impossible to generalize the distance formula to stably causal spacetimes. In our opinion this difficulty is connected with the definition of Lorentz-Finsler distance which is adapted to the causal relation $J$, while it is known that under stable causality the natural causality relation is $K$.
%
%So we develop a different type of distance. Let us write $\mathscr{L}'>\mathscr{L}$ if $C'>C$ and $\mathscr{I}'\cap \mathscr{I} \ne \emptyset$, where $\mathscr{I}\subset C$ is the indicatrix.
%
%\begin{definition}
%Given $p,q\in M$ we define $D\colon M\times M\to [0,+\infty]$ as follows
%\begin{equation}
%D(p,q)=\mathrm{inf}_{\mathscr{L}'>\mathscr{L}} d'(p,q),
%\end{equation}
%where $d'$ is the Lorentz-Finsler distance for $(M,\mathscr{L}')$.
%\end{definition}
%Observe that for $\mathscr{L}_1>\mathscr{L}_2$ we have $d_1\ge d_2$, and the set $\{\{\mathscr{L}':\mathscr{L}'>\mathscr{L}\}$ is directed in the sense that if $\mathscr{L}_1>\mathscr{L}$ and $\mathscr{L}_2>\mathscr{L}$ there is  $\mathscr{L}_3>\mathscr{L}$ such that $\mathscr{L}_1,\mathscr{L}_2>\mathscr{L}_3$.
%
%
%\begin{proposition} \label{upq}
%The following properties hold true:
%\begin{itemize}
%\item[(a)] If $(p,q)\notin J_S$, then $D(p,q)=0$,
%\item[(b)] If $(p,q) \in \mathrm{Int} J_S$ or $q\in \mathrm{Int} J_S^+(p)$ or $p\in \mathrm{Int} J_S^{-}(q)$, then $D(p,q)>0$,
%\item[(c)]    If $(p,q)\in J_S$ and $(q,r)\in J_S$, then $(p,r)\in J_S$ and
%\begin{align*}
% D(p,q)+D(q,r)\le D(p,r),
%\end{align*}
%\item[(d)] D is upper semi-continuous,
%\item[(e)] If $D=d$ then they are continuous and the spacetime is reflecting (so causally continuous if distinguishing).
%\item[(f)] $d \le D$.
%\end{itemize}
%\end{proposition}
%%
%%\begin{proof}
%%(a). By  definition of Seifert relation, if $(p,q)\notin J_S$ then there is $\hat{\mathscr{L}}$ such that $\hat C>C$ and $(p,q) \notin \hat J$. Then for every $\mathscr{L}'$ such that $\mathscr{L}<\mathscr{L}'<\mathscr{L}$, $d'(p,q)=0$, hence the thesis.
%%
%%(b). If $(p,q) \in \mathrm{Int} J_S$ pick $p',q'$ such that $p<<p'$, $q'<<p$ then $(p',q')\in J_S$. For every $\mathscr{L}'>\mathscr{L}$, $d'(p,q)$ is larger than the sum of the $\mathscr{L}$-Lorentz-Finsler lengths of the $C$-timelike curves connecting $p$ to $p'$ and $q'$ to $q$, which are positive and independent of $\mathscr{L}'$, thus the thesis. The proofs with the assumptions  $q\in \mathrm{Int} J_S^+(p)$ or $p\in \mathrm{Int} J_S^{-}(q)$ are analogous, but there is only one timelike curve.
%%
%%(c). Let $\mathscr{L}'>\mathscr{L}$, then   $(p,q)\in J'$, $(q,r)\in J'$ and
%%\begin{align*}
%%D(p,q)+D(q,r)\le d'(p,q)+d'(q,r)\le d'(p,r),
%%\end{align*}
%%where we used the Lorentz-Finsler reverse triangle inequality.
%%Since the equation in display holds for every  $\mathscr{L}'>\mathscr{L}$, taking the infimum we obtain the desired result.
%%
%%(d). We can assume that $D(p,q)$ is finite. Suppose $D$ is not upper semi-continuous at $(p,q)$, then there is $\epsilon>0$ and a sequence $(p_n,q_n)\to (p,q)$ such that $D(p_n,q_n) \ge D(p,q)+3\epsilon$. By definition of $D(p,q)$ we can find $\mathscr{L}'>\mathscr{L}$ such that for every causal curve $\gamma$ connecting $p$ to $q$, $\ell'(\gamma)\le D(p,q)+\epsilon$.  Let $\mathscr{L}_n\to \mathscr{L}$, be a sequence such that  $\mathscr{L} <\mathscr{L}_{n+1}<\mathscr{L}_n<\mathscr{L}'$. For every $n$ we can find $\gamma_n$ $C_n$-causal curve connecting $p_n$ to $q_n$ such that $\ell_n(\gamma_n)\ge D(p_n,q_n)$. By the limit curve theorem there are two $C$-causal limit curves $\sigma^q$ ending at $q$ and $\sigma^p$ starting at $p$ (possibly inextendible on the other direction) to which a subsequence (denoted in the same way) $\gamma_n$ converges uniformly over compact subsets (recall that the limit curve is $C_n$-causal for every $n$ hence $C$-causal). Let $p'\in \sigma^p$ be chosen so small that the $\mathscr{L}'$-length of the $\sigma^p$-segment between $p'$ to $p$ is smaller than $\epsilon/3$.
%% Similarly choose $q'$ with the dual criteria. Let $p'_n\in \gamma_n$ be such that $p'_n\to p'$ and similarly for $q'_n$.  Since the limit curves are $C$-causal they are $C'$-timelike, thus, as the chronology relation is open, we can move  from $p$ to $p'_n$ follow $\gamma_n$ to $q'_n$ and then move from $q'_n$ to $q$ obtaining a $C'$-timelike curve $\eta$ whose length  for sufficiently large $n$ satisfies (by the upper semi-continuity of the length functional)
%% \[
%% \ell'(\eta)>\ell'(\gamma_n)-\epsilon\ge \ell_n(\gamma_n)-\epsilon\ge D(p_n,q_n)-\epsilon \ge  D(p,q)+2\epsilon,
%% \]
%% which gives a contradiction.
%%
%%(e). Since $D$ is upper semi-continuous and $d$ is lower semi-continuous, if they are equal they are continuous. The causal implications of the continuity of $d$ have been already discussed.
%%
%%(f). We can assume $d(p,q)>0$, the other case being trivial. Since whenever $\mathscr{L}'>\mathscr{L}$, we have $\ell'(\gamma)\ge \ell(\gamma)$ for every $C$-causal curve, the statement follows.
%%\end{proof}
%%
%%
%%\begin{theorem} \label{xzo}
%%Let $f$ be a $\mathscr{L}$-steep time function, then for every $p,q\in M$
%%\[
%% D(p,q) \le [f(q)-f(p)]^+.
%%\]
%%\end{theorem}
%%
%%\begin{proof}
%%The existence of $f$ implies that the spacetime is stably causal. Since $f$ is $\mathscr{L}$-steep it is temporal and we can find $\hat C>C$ such that $f$ is a time function for $(M,\hat C)$.
%%%and $(M,\hat C)$ is still stably causal.
%%If $(p,q)\notin J_S$, we have $D(p,q)=0$ so the inequality is satisfied. Assume $(p,q)\in J_S$, then $(p,q)\in \hat J$ and $f(q)>f(p)$. We need to prove that for every $\epsilon>0$ we can find $\mathscr{L}'$, $\mathscr{L}<\mathscr{L}'$, $C'<\hat C$, such that $d'(p,q) \le f(q)-f(p) +\epsilon$.
%%
%%
%%Let $\epsilon'>0$ be so small that $\frac{f(q)-f(p)}{1-\epsilon'}\le  f(q)-f(p) +\epsilon$.
%%As the indicatrix $\mathscr{I}'$ get closer to $\mathscr{I}$ the same occurs to the dual indicatrices, with the caveat that now $\mathscr{I}^*$ is closer to the origin than $\mathscr{I}'^*$.
%%So we can choose  $\mathscr{L}'$ so close to $\mathscr{L}$ that  $\sqrt{-2 \mathscr{H}'(-\dd f)} \ge \sqrt{-2 \mathscr{H}(-\dd f)}  -\epsilon'\ge 1-\epsilon'$.
%%
%%For
%%every $C'$-causal curve $\gamma$ connecting $p$ to $q$
%%%, denoting with $t$ the $h$-arc length
%%\begin{align*}
%%f(q)-f(p)&=\int \dd f(\dot \gamma) \dd t \ge \int \sqrt{-2 \mathscr{H}'(-\dd f)} \sqrt{-2\mathscr{L}'(\dot \gamma)} \dd t \\
%%&\ge (1-\epsilon') \int  \sqrt{-2\mathscr{L}'(\dot \gamma)} \dd t \ge (1-\epsilon') \ell'(\gamma)
%%\end{align*}
%%where we used the reverse Cauchy-Schwarz inequality for $(M,\mathscr{L}')$. By the definition of $\epsilon'$, $\ell'(\gamma)\le f(q)-f(p) +\epsilon $ over every $C'$-causal connecting curve. Taking the supremum over $\gamma$ we obtain the desired result.
%%\end{proof}
%
%\begin{corollary}
%In a globally hyperbolic spacetime $(M,\mathscr{L})$ we have $D=d$.
%\end{corollary}
%%
%%\begin{proof}
%%From Theorem \ref{nug} we know that $d=\mathrm{inf} \{[f(q)-f(p)]^+\colon f \ \textrm{is} \ \mathscr{L}-\textrm{steep}  \}$. But from Prop. \ref{upq} (f) $d \le D$ and from Theorem \ref{xzo} $D\le \mathrm{inf} \{[f(q)-f(p)]^+\colon f \ \textrm{is} \ \mathscr{L}-\textrm{steep}  \} $ from which the result follows.
%%\end{proof}
%
%
%The next result clarifies the geometrical meaning of $D$ and shows that the metrical aspects of spacetime can be reduced to causal aspects of a  cone structure on a product manifold.
%
%First we need a lemma.
%
%\begin{lemma}
%Let $V$ be a finite dimensional vector space, and let $D$ be a closed strongly  convex sharp cone with non-empty interior and with smooth boundary on $V\times \mathbb{R}\backslash 0$, such that any $\mathbb{R}$-fiber has intersection with $D$ in a segment having barycenter on $V\times\{0\}$, namely on the closed convex sharp cone  with non-empty interior  $C=D\cap [V\times\{0\}]$. Then there is a Lorentz-Finsler Lagrangian $\mathscr{L}\colon C\to \mathbb{R}$ such that
%\[
%D=\{(y,z)\colon y\in C, \ \vert z\vert  \le \sqrt{-2 \mathscr{L}(y)}  \}.
%\]
%\end{lemma}
%
%\begin{proof}
%Consider the set $\mathscr{I}\subset C$, such that $\p D\cap[V\times \{1\}]=\mathscr{I} \times \{1\}$. By the strong convexity of $D$, $\mathscr{I}$ is strongly convex and so induces a Lorentz-Finsler Lagrangian which has $\mathscr{I}$ as indicatrix \cite{laugwitz11,minguzzi14h,minguzzi15e} . By positive homogeneity, this is the Lorentz-Finsler Lagrangian we were looking for.
%\end{proof}
%%
%We recall that a {\em causally easy} spacetimes is a stably causal spacetime for which the closure of the causal relation $\bar J$ is transitive. Causally continuous spacetimes are causally easy.
%%\begin{theorem} \label{aqa}
%%Let $(M,\mathscr{L})$ be a spacetime and let $M^\times= M\times \mathbb{R}$ be endowed  with the cone structure
%%\begin{equation}
%%C^\times_{(p,r)}=\{(y,z) \colon y\in C_p, \ \vert z\vert \le \sqrt{-2 \mathscr{L}(y)} \}.
%%\end{equation}
%%Then
%%\begin{itemize}
%%\item[(a)] $(M^\times,C^\times)$ is (stably) causal if and only if $(M,C)$ is (stably) causal;
%%\item[(b)] The chronological relation for $(M^\times,C^\times)$ can be written
%%\begin{equation} \label{doa}
%%I^\times=\{((p,r),(p',r'))\colon (p,p')\in I \textrm{ and } \vert r'-r\vert < d(p,p')\} ;
%%\end{equation}
%%while its closure satisfies
%%\begin{equation} \label{dpm}
%%\overline{J^\times}=\overline{I^\times}\supset\{((p,r),(p',r'))\colon (p,p')\in \bar I \textrm{ and } \vert r'-r\vert \le d(p,p')\}
%%\end{equation}
%%\item[(c)] The Seifert relation for $(M^\times,C^\times)$ can be written
%%\begin{equation} \label{vyt}
%%J_S^\times=\{((p,r),(p',r'))\colon (p,p')\in J_S \textrm{ and } \vert r'-r\vert \le D(p,p')\},
%%\end{equation}
%%and under stable causality $D(p,p)=0$ for every $p\in M$.
%%%\item[(d)] Assume $(M,C)$ is stably causal. $(M^\times,C^\times)$  is causally continuous iff $d=D$, in which case $(M,C)$ is causally continuous;
%%%\item[(e)] If $(M,C)$ is causally easy, $d$ is continuous, and the reverse triangle inequality holds for $\overline{J}$-related events,  then $D=d$, and both $(M,C)$ and  $(M^\times,C^\times)$ are causally continuous.
%%\end{itemize}
%%\end{theorem}
%%
%%
%%\begin{proof}
%%Statement (a) is trivial, since any closed causal curve for $M$ is immersed as a closed causal curve on $M^\times$, and every closed causal curve for $M^\times$ projects into a closed causal for $M$. Similarly, if the cones can be  opened on $M$ preserving causality, then they can be  opened on $M^\times$ preserving causality, otherwise the projection of a closed casual curve would contradict stable causality of $M$. Thus the stable causality of $M$ implies that of $M^\times$. Conversely, if the cones of $M^\times$ can be  opened preserving causality then their restriction to $M\times\{0\}$ provides a distribution of cones on $M$ which is an opening of the cones of $M$ which preserves causality.
%%
%%(b). Observe that $ \textrm{Int} C^\times_{(p,r)}=\{(y,z) \colon y\in \textrm{Int} C_p, \ z < \sqrt{-2 \mathscr{L}(y)} \}$, so integration over a timelike curve on $M^\times$, easily gives (\ref{doa}). In equation (\ref{dpm}) the equality $\bar{I}=\bar{J}$ is well known. The inclusion $\supset$ can be proved as follows: let  $(p,r),(p',r')$ be such that $(p,p')\in \bar I$ and $\vert r'-r\vert \le d(p,p')$. Let $s=\mathrm{sgn}(r'-r)$. Let us consider sequences $p_i<< p$, $p_i'>>p'$, $p_i \to p$, $p_i'\to p'$. Since the chronology relation is open $(p_i,p_i')\in I$. By the lower semi-continuity of $d$ for sufficiently large $i$, $d(p_i,p_i')\ge d(p,p')-1/i$, then the sequence $((p_i,r+s/i), (p'_i, r'-s/i))$ converges to  $((p,r),(p',r'))$. The sequence belongs to $I^\times$ because \[\vert (r'-s/i)- (r+s/i) \vert=\vert r'-r\vert -\tfrac{2}{i}\le d(p,p')-\tfrac{2}{i}\le d(p_i,p_i')-\tfrac{1}{i},\] which proves (\ref{dpm}).
%%
%%(c). Let $D''>C^\times$ then  the lemma implies that we can find a narrower one $C^\times<D'<D''$ which can be written $D'=\{(y,z)\colon y\in C', \  z  \le \sqrt{-2 \mathscr{L}'(y)}$, so that $C<C'$ and $\mathscr{L}<\mathscr{L}'$. Since \cite[Lemma 3.3]{minguzzi07} $J_S=\Delta \cup \bigcap I'$ for all chronological relations relative to wider cones, we get Eq.\ (\ref{vyt}) from (b). Under stable causality, $J_S^\times$ is antisymmetric, and it cannot be $D(p,p)>0$ for some $p$, otherwise we could find constants $0<r<r'< D(p,p')$ and $ ((p,r) ,(p,r')) \in J^\times_S$, $((p,r') ,(p,r)) \in J^\times_S$, a contradiction.
%%%(d). By Prop.\ \ref{upq} (e) if $D=d$ then $(M,C)$ is causally continuous and we also obtain from the previous formulas $(J_S^\times)^\pm((p,r))=\overline{(I^\times)^{\pm}((p,r))}$ which implies that $M^\times$ is reflecting and hence causally continuous. Conversely, if it is causally continuous then $J_S^\times=\overline{I^\times}$ (it is causally easy) which implies $D=d$.
%%%
%%%(e). By (b)
%%%\[
%%%\overline{J^\times}=\overline{I^\times}=\{((p,r),(p',r'))\colon (p,p')\in \bar I \textrm{ and } \vert r'-r\vert \le d(p,p')\}
%%%\]
%%%where by assumption $\bar I=J_S$.
%%% But the relation in display is closed and transitive, thus it includes  $K^\times=J_S^\times$, hence $D=d$. The remaining statement follows from (d).
%% \end{proof}
%%
%
%
%%\begin{lemma}
%%Let $V$ be a finite dimensional vector space, let $C$ be a closed convex sharp cone on $V$, and let $D$ be a closed convex sharp cone on $V\times \mathbb{R}$, such that $C\subset \mathrm{Int}[ D\cap (V\times \{0\})]$. Then we can find a  Lorentz-Finsler Lagrangian $\mathscr{L}'\colon C'\to \mathbb{R}$, such that
%%\[
%%\{(y,z) \colon y\in C, \ \vert z\vert \le \sqrt{-2 \mathscr{L}'(y)} \} \subset D
%%\]
%%\end{lemma}
%%
%%
%%\begin{proof}
%%Let $(C^\times)'$ be a cone structure wider than $C^\times$. We can always find a narrower one, denoted in the same way, with the same property and such that then its restriction to $T[M\times\{0\}]$ gives a cone structure wider than $C$
%%\end{proof}
%
%\begin{theorem}
%Let $(M,\mathscr{L})$ be stably causal. The next  conditions are equivalent
%\begin{enumerate}
%\item $D=d$,
%\item  $d$ is continuous,
%\end{enumerate}
%and they imply the causal continuity of both $(M,C)$ and   $(M^\times,C^\times)$.
%\end{theorem}
%%
%%For the proof we need to recall \cite[Theor.\ 3.3]{minguzzi07b} the definitions of the relations $D_f=\{(p,q)\colon q\in \overline{I^+(p)}\}$, $D_p=\{(p,q)\colon p\in \overline{I^-(q)}\}$. They are reflexive and transitive (preorder), moreover $D_f$ (resp.\ $D_p$) is an order (i.e.\ antisymmetric) iff the spacetime is future (resp.\ past) distinguishing. Finally \cite[Theor.\ 3.7]{minguzzi07b}, if the spacetime is causally continuous $\overline{J}=K=J_S=D_f=D_p$.
%%%Point (e) uses the equality between the Seifert relation $J_S$ and the Sorkin and Woolgar relation $K$ on stably causal spacetimes \cite{minguzzi08b}.
%%
%%\begin{proof}
%%$1 \Rightarrow 2$. By Prop.\ \ref{upq} (e) if $D=d$ then $d$ is continuous and $(M,C)$ is causally continuous. Moreover, we have
%%\begin{equation} \label{jaw}
%%\overline{J^\times}=\overline{I^\times}=\{((p,r),(p',r'))\colon (p,p')\in \bar I \textrm{ and } \vert r'-r\vert \le d(p,p')\}
%%\end{equation}
%%indeed, $\supset$ has been proved in (\ref{dpm}) while $\subset$ is easily proved using (\ref{doa}) and the upper semi-continuity of $d$.
%% By (\ref{jaw}) $(J_S^\times)^\pm((p,r))=\overline{(I^\times)^{\pm}((p,r))}$ which implies that $(M^\times, C^\times)$ is reflecting and hence causally continuous. Indeed, past reflectivity is proved as follows
%% \begin{align*}
%% (p',r') \in \overline{(I^\times)^{+}((p,r))}&\Rightarrow (p',r') \in (J_S^\times)^+((p,r)) \Rightarrow (p,r) \in (J_S^\times)^-((p',r'))\\
%% &\Rightarrow (p,r) \in \overline{(I^\times)^{-}((p',r'))},
%%\end{align*}
%%and dually for future reflectivity.
%%$32 \Rightarrow 1$. If $d$ is continuous, by \cite[Theor.\ 4.24]{beem96} $(M,C)$ is causally continuous (see also the proof of Prop.\ \ref{sqw}).  In the previous point we have shown that (\ref{jaw}) follows from the continuity of $d$. By causal continuity $\bar I=D_f=D_p$ which is transitive. Consider three points $(p,q)\in \bar I=D_p$ and $(q,r)\in \bar{I}=D_f$, then $p\in \overline{I^-(q)}$ and $r\in \overline{I^+(q)}$, thus we can find sequences $p_i\to p$, $r_i\to r$, such that $p_i<<q<<r_i$. By the reverse triangle inequality $d(p_i,q)+d(q,r_i)\le d(p_i,r_i)$, and by the continuity of $d$, $d(p,q)+d(q,r)\le d(p,r)$, so the reverse triangle inequality holds for $\bar I$-causally related pairs. There follows that $\overline{I^\times}$, which is given by (\ref{jaw}), is really closed and transitive, thus $\overline{I^\times}=K^\times=J^\times_S$, so a comparison with (\ref{vyt}) gives $D=d$
%%
%%%$3 \Rightarrow 1$. If $(M^\times,C^\times)$ is causally continuous then $J_S^\times=\overline{I^\times}$  which implies $D=d$.
%%\end{proof}
%
%We already know that on a stably causal spacetime the Seifert relation is represented by the temporal functions. It turns out that this statement on the product manifold is precisely the distance formula.
%
%
%\begin{theorem} \label{coa}
%Let $(M,\mathscr{L})$ be a stably causal spacetime such that $D<+\infty$, then there is $\bar{\mathscr{L}}>\mathscr{L}$ such that $(M,\bar{\mathscr{L}})$ is stably causal and $\bar D<+\infty$. In particular, $d\le D\le \bar d\le \bar D<+\infty$.
%\end{theorem}
%%
%%\begin{proof}
%%%We have to prove that for every $\epsilon >0$ there is $f$ $\mathscr{L}$-steep such that $[f(q)-f(p)]^+\colon\le D(p,q)+\epsilon$. Equivalently,  for every $\epsilon >0$ there is $f$ $\mathscr{L}$-steep and $\hat{\mathscr{L}}>\mathscr{L}$ such that for every $\mathscr{L}<\mathscr{L}'<\hat{\mathscr{L}}$, $[f(q)-f(p)]^+\colon\le d'(p,q)+\epsilon$.  Equivalently,  for every $\epsilon >0$ there is $f$ $\mathscr{L}$-steep and $\hat{\mathscr{L}}>\mathscr{L}$ such that for every $\mathscr{L}<\mathscr{L}'<\hat{\mathscr{L}}$, we can find a $C'$-causal curve such that  $[f(q)-f(p)]^+\colon\le \ell'(\gamma)+\epsilon$.
%%
%%By stable causality we can find $\hat C>C$ such that  $(M,\hat C)$ is still stably causal. We need some preliminary results.
%%
%%
%%Result 1:  under the theorem assumptions, given a compact set $K$ we can find $\check{\mathscr{L}}>\mathscr{L}$, $C<\check C<\hat C$, such that $\check{d}\vert_{K\times K}<R$ for some $R(K)>0$.
%%
%%Proof of result 1. Since $D$ is upper semi-continuous  there is $R>0$ such that $D\vert_{K\times K}<R$. The compact set $K$ can be covered by  $C$-causally convex sets of the form $I^+(p_i)\cap I^-(q_i)$, $1\le i\le s<+\infty$. Since $D(p_i,q_j)<R<+\infty$, by definition of $D$ we know that there is $\mathscr{L}_{ij}>\mathscr{L}$, $C_{ij}<\hat C$, such that $d_{ij}(p_i, q_j)<R$. Let $\check{\mathscr{L}}< \mathscr{L}_{ij}$ for every $1\le i,j\le s$, then for every $p,q\in K$, either they are not $\check C$-chronologically related, so  $\check d(p,q)=0<R$ and we have finished, or they are  $\check C$-chronologically related. In the latter case
%% $p\in I^+(p_i)\subset \check I^+(p_i) $ for some $i\le s$, and $q\in I^{-}(q_j)\subset  \check I^{-}(q_j) $ for some $j\le s$, in particular $(p_i,q_j)\in \check I$, and $p,q\in  \check I^+(p_i) \cap  \check I^{-}(q_j)\subset  I^+_{ij}(p_i) \cap I^{-}_{ij}(q_j)$. But now $\check d(p,q)<R$ otherwise
%% \[
%% d_{ij}(p_i,q_j)\ge \check d(p_i,q_j) \ge \check d (p,q)\ge R,
%% \]
%%a contradiction. Result 1 is proved.
%%
%%
%%
%%Result 2: under the theorem assumptions, given a compact set $K$ we can find $\check{\mathscr{L}}>\mathscr{L}$, $C<\check C<\hat C$, such that every $\mathscr{L}'$, such that  $\mathscr{L}<\mathscr{L}'<\check{\mathscr{L}}$ on $M\backslash K$, has bounded distance $d'\vert_{K\times K}$.
%%
%%Observe that the definition of the distance $d'$ involves $C'$-curves that might escape $K$. Also observe that $\mathscr{L}'$ is not constrained inside $K$.
%%
%%Proof of result 2.
%%%In order to prove the preliminary result observe that $D$ is upper semi-continuous and so there is $R>0$ such that $D\vert_{K\times K}<R$.
%%%The intersection of the compact sets (recall that $\check d$ is lower semi-continuous)
%%%\[\{(p,q,r)\in K\times K\times \mathbb{R}:  R\le  r\le \check{d}(p,q)\}\] defined for every $\check{\mathscr{L}}$, $\mathscr{L}<\check{\mathscr{L}}$, $\check C<\hat{C}$, is empty by the definition of $D$. But  the family shares the finite intersection property, so the intersection is non-empty unless some of them is empty.
%%By Result 1 there is $\check{\mathscr{L}}>\mathscr{L}$ such that $\check{d}\vert_{K\times K}<R$. But we can enlarge the cones inside $K$, and alter $\check{\mathscr{L}}$ to $\check{\mathscr{L}}'$, as long as the new cones in $K$  are narrower than $\hat C$, preserving the boundedness of $\check{d}'$ in $K\times K$. To show this point, observe that by $\hat{C}$ stable causality, $K$ can be covered by a finite number $N$ of $\hat{C}$-causally convex neighborhoods. Every $\check{C}$-causal curve starting and ending in $K$ can escape $K$ but at most $N$ times since each time it reenters a different causally convex neighborhood. The $\check{\mathscr{L}}'$-length of $\check{C}'$-causal curves contained in $K$ is bounded by a constant $B$ (again because the curves are forced to escape the convex neighborhoods), thus such an alteration of $\check{\mathscr{L}}$ inside $K$ would nevertheless keep $\check{d}'\vert_{K\times K}$ bounded by $(B+R)N$. Result 2 is proved.
%%
%%
%%Let $h$ be an auxiliary complete Riemannian metric on $M$ and let $o\in M$. Let $K_n=\bar{B}(o,n)$ and let $\mathscr{L}_n$ be the Finsler Lagrangian with the property of the preliminary result for $K=K_n$. The sequence can be chosen so that $ \mathscr{L}_{n+1}<\mathscr{L}_n$. Let $\mathscr{L}'>\mathscr{L}$ be such that for every $n$, $\mathscr{L}'\vert_{K_n\backslash \textrm{Int}  K_{n-1}}< \mathscr{L}_k\vert_{K_n\backslash \textrm{Int} K_{n-1}}$ for every $k< n$. This is a finite number of conditions over every compact annulus, so we can find $\mathscr{L}'$ as stated. Let $m\ge 1$. For every $n>m$, on $K_n\backslash K_m$ we have $\mathscr{L}' < \mathscr{L}_m$ and since this is true for every $n$, we have $\mathscr{L}'<\mathscr{L}_m$ on $M\backslash K_m$. Let $p,q\in M$, then there is some $m$ such that $p,q\in K_m$.  By $\mathscr{L}'<\mathscr{L}_m$ on $M\backslash K_m$ and the Result 2 we have $d'(p,q)<+\infty$. By the arbitrariness of $p$ and $q$, $d'$ is finite. Taking $\bar{\mathscr{L}}$ such that $ \mathscr{L}<\bar{\mathscr{L}}< \mathscr{L}'$ gives a finite $\bar{D}$.
%%
%%%
%%%
%%%The compact set $K_1=\bar{B}(o,1)$, $K_n=\bar{B}(o,n+1)\backslash B(o,n)$ constructed with the open balls of $h$ of center $o$ are such that $M=\cup_n K_n$.
%%%
%%%
%%%
%%%Let $\epsilon >0$. Let us consider  a  $\hat C$-causal curve $\gamma$ that passes through $C_n$. It can enter and escape $K_n$ several times, but by $\hat C$-strong causality of $M$, it cannot enter twice the $\hat C$-causally convex neighborhoods that cover $K_n$. By compactness we can find a finite covering for $K_n$, thus $\gamma \cap K_n$ is made of at most $N_n$ disconnected pieces, where $N_n$ does not depend on  $\gamma$. Since $D$ is upper semi-continuous it admits a maximum $M_n$ on $K_n\times K_n$, so $d$ is also bounded by $M_n$. Thus the total $\mathscr{L}$-length of $\gamma$ on $C_n$ is bounded by a constant  $R_n\le N_n M_n$. Let $\mathscr{L}_k\to \mathscr{L}$, $ \mathscr{L}<\mathscr{L}_{k+1}<\mathscr{L}_k$, on $K_n$. Let $\gamma_k$ be a sequence of $C_k$-causal curves possibly disconnected in at most $N_n$ pieces $\gamma^i_{k}$. By the limit curve theorem each piece converges to some $C$-causal curve $\gamma^i$ which collectively give a curve $\gamma$ consisting of at most $N_n$ pieces.
%%%By the upper semi-continuity of the length functional applied to each piece, we have passing to a subsequence denoted in the same way $\limsup \ell_i(\gamma_i)\le \ell_k(\gamma)+\epsilon$. Since the right-hand side holds for every $k$, we have taking the infimum $ \limsup \ell_i \ell_i(\gamma_i)\le N_n M_n+\epsilon$. This means to we can choose $\mathscr{L}'>\mathscr{L}$ on $C_n$ so that $\ell'(\gamma') \le N_n M_n+\epsilon$ for every disconnected casual curve $\gamma'$ composed of at most $N_n$ pieces. In particular $\mathscr{L}'$ can be chosen so that this property holds for every $n$, so since for every $p,q\in M$, the pair belongs to the union $\cup^{s(p,q)}_n K_n$ for some finite $s(p,q)$, we have that the
%%%
%%%
%%%
%%%Every connected $\hat C$-causal curve on $C_n$ has a $h$-length bounded by a constant $M_n$, for otherwise by the limit curve theorem we could find an inextendible  $\hat C$-causal curve contained in $C_n$ in contradiction with the strong causality of $(M,\hat C)$. Let $S_n\subset TC_n$ denote the $h$-unit ball bundle over $C_n$. Now, we choose $\mathscr{L}'$ so close to $\mathscr{L}$ that for every $n$ we have that over $S_n\cap \hat C$, $0\le \sqrt{[-2\mathscr{L}']^+}-\sqrt{[-2\mathscr{L}]^+}\le \epsilon (M_n N_n)^{-1}$. Thus for every $C'$-causal curve
%%%
%%
%%\end{proof}
%
%%
%%
%%Now the idea is to introduce a different cone structure in $M^\times$ defined at $P=(p,r)$ by
%%\begin{equation}
%%C^\downarrow_{P}=\{(y,z) \colon y\in C_p, \ z \le \sqrt{-2 \mathscr{L}(y)} \},
%%\end{equation}
%%and to show that there is a time function on $(M^\times, C^\downarrow)$, such that its zero level set $S_0$ intersects once every $\mathbb{R}$-fiber of $M^\times$. This set $S_0$ regarded as a graph over $M$ provides a time function for $(M,\mathscr{L})$ which can be approximated by a smooth  steep time functions.
%%
%%\begin{theorem} \label{soq}
%%Let  $(M,\mathscr{L})$ be stably causal and let $\mathscr{L}'>\mathscr{L}$ be such that $(M,C')$ is stably causal, then $(M^\times, C'{}^\downarrow)$ is strongly causal.
%%\end{theorem}
%%
%%
%%\begin{proof}
%%It is sufficient to prove strong causality  at $P=(p,0)\in M^\times$, $p\in M$. Let $V$ be a  $C'$-causally convex open set which is also globally hyperbolic. One can easily construct a time function $\tau$ on $V\times \mathbb{R}$, for sufficiently small $V$, e.g.\ one whose level sets are obtained by vertically translating a local $C^\downarrow$-spacelike hypersurface passing through $P$, which intersects $V\times \mathbb{R}$ on a relatively compact set. Let $\tau$ be such that $\tau(P)=0$.
%%
%%Let $U=(V\times\mathbb{R})\cap \tau^{-1}(-\delta,\delta)$ be an open neighborhood of $P$, and let us consider a $C'{}^\downarrow$-causal curve $\Gamma$ which starts from some point of $U$. We have to show that it cannot reenter $U$ once it escapes $U$. First we show that it cannot escape $V\times\mathbb{R}$.  Let $h$ be an auxiliary Riemannian metric. The curve $\Gamma$ parametrized with respect to $h$-arc length is a absolutely continuous (AC) curve with derivative in $C'{}^\downarrow$ a.e. \cite[Theor.\ 7]{minguzzi13d}, so its projection $\gamma$ (the projection to $M$ is Lipschitz, and the composition $g\circ f$, with $f$ AC and $g$ Lipschitz, is AC) is absolutely continuous with derivative in $C'$ a.e., and so $\gamma$ is a continuous causal curves \cite[Theor.\ 7]{minguzzi13d}. As a consequence, if $\Gamma$ escapes  $V\times\mathbb{R}$, then $\gamma$ escapes $V$ and so it cannot reenter it.
%%
%%But if $\Gamma$ remains in $V\times\mathbb{R}$, $\tau$ is increasing over it, so once it escapes $U$ it cannot reenter it.
%%
%%\end{proof}
%%Thus
%%
%%
%%
%%The point of escape cannot belong t
%%
%%Suppose it is not future distinguishing, then we can find
%%$P\in M^\times$, $P=(p,0)$, a neighborhood $U\ni P$ and a sequence of $C'{}^\downarrow$-causal curves $\Gamma_i$ starting from $P$, escaping $U$ and returning to $P_i$ where $P_i\to P$. Without loss of generality the neighborhood can be chosen to be a product $U=V\times R$, where $V$ is a distinguishing neighborhood for $(M,C')$ which is also globally hyperbolic. Let $h$ be an auxiliary Riemannian metric. The curves $\Gamma_i$ parametrized with respect to $h$-arc length are absolutely continuous (AC) curves with derivative in $C'{}^\downarrow$ a.e. \cite[Theor.\ 7]{minguzzi13d}, so their projection $\gamma_i$ (the projection to $M$ is Lipschitz, and the composition $g\circ f$, with $f$ AC and $g$ Lipschitz, is AC) is absolutely continuous with derivative in $C'$ a.e., and so the projections $\gamma_i$ are continuous causal curves \cite[Theor.\ 7]{minguzzi13d}. Now, since $p_i\to p$ either there is a subsequence which escapes  $V$, which would give a contradiction with distinction of $(M,C)$, or  they are contained in $V$ and contract to $p$. However, one can easily construct a time function on $V\times \mathbb{R}$, for sufficiently small $V$, e.g.\ one whose level sets are obtained by vertically translating a local $C^\downarrow$-spacelike hypersurface passing through $P$, which shows that $\Gamma_i$ cannot exist, a contradiction.
% %there is a point $P$ such that for every ne a $C^1$  $C'{}^\downarrow$-causal curve $t \mapsto P(t)$
%%=(x(t), r(t))$, $t\in [0,1]$, where $\dot{r} \le \sqrt{-2 \mathscr{L}(\dot x)}$ and $x(0)=x(1)=p$, $r(0)=r(1)=0$.
%%Since $(M,C')$ is stably causal we can find a $C'$-time function $t'\colon M\to \mathbb{R}$, which can be lifted to $M^\times$ to a function denoted in the same way. The levels sets of the lifted $t'$ are tangent to the $C'{}^\downarrow$-cones along the fiber direction, thus $\dd t'(\dot P)\ge 0$ with equality only if $\dot P$ is vertical. But it cannot be $\dd t'(\dot P)> 0$ at any point since $P$ has to return to the starting point, so the image of $P$ is contained on a $\mathbb{R}$-fiber, which is again impossible, since the curve would not be closed as  $C'{}^\downarrow$ orients it.
%%
%%\begin{proof}
%%Suppose it is not future distinguishing, then we can find
%%$P\in M^\times$, $P=(p,0)$, a neighborhood $U\ni P$ and a sequence of $C'{}^\downarrow$-causal curves $\Gamma_i$ starting from $P$, escaping $U$ and returning to $P_i$ where $P_i\to P$. Without loss of generality the neighborhood can be chosen to be a product $U=V\times R$, where $V$ is a distinguishing neighborhood for $(M,C')$ which is also globally hyperbolic. Let $h$ be an auxiliary Riemannian metric. The curves $\Gamma_i$ parametrized with respect to $h$-arc length are absolutely continuous (AC) curves with derivative in $C'{}^\downarrow$ a.e. \cite[Theor.\ 7]{minguzzi13d}, so their projection $\gamma_i$ (the projection to $M$ is Lipschitz, and the composition $g\circ f$, with $f$ AC and $g$ Lipschitz, is AC) is absolutely continuous with derivative in $C'$ a.e., and so the projections $\gamma_i$ are continuous causal curves \cite[Theor.\ 7]{minguzzi13d}. Now, since $p_i\to p$ either there is a subsequence which escapes  $V$, which would give a contradiction with distinction of $(M,C)$, or  they are contained in $V$ and contract to $p$. However, one can easily construct a time function on $V\times \mathbb{R}$, for sufficiently small $V$, e.g.\ one whose level sets are obtained by vertically translating a local $C^\downarrow$-spacelike hypersurface passing through $P$, which shows that $\Gamma_i$ cannot exist, a contradiction. Past distinction is proved dually.
%% %there is a point $P$ such that for every ne a $C^1$  $C'{}^\downarrow$-causal curve $t \mapsto P(t)$
%%%=(x(t), r(t))$, $t\in [0,1]$, where $\dot{r} \le \sqrt{-2 \mathscr{L}(\dot x)}$ and $x(0)=x(1)=p$, $r(0)=r(1)=0$.
%%%Since $(M,C')$ is stably causal we can find a $C'$-time function $t'\colon M\to \mathbb{R}$, which can be lifted to $M^\times$ to a function denoted in the same way. The levels sets of the lifted $t'$ are tangent to the $C'{}^\downarrow$-cones along the fiber direction, thus $\dd t'(\dot P)\ge 0$ with equality only if $\dot P$ is vertical. But it cannot be $\dd t'(\dot P)> 0$ at any point since $P$ has to return to the starting point, so the image of $P$ is contained on a $\mathbb{R}$-fiber, which is again impossible, since the curve would not be closed as  $C'{}^\downarrow$ orients it.
%%\end{proof}
%%%
%%
%%\begin{proof}
%%Suppose not, then there is a $C^1$ closed $C'{}^\downarrow$-causal curve $t \mapsto P(t)$
%%%=(x(t), r(t))$, $t\in [0,1]$, where $\dot{r} \le \sqrt{-2 \mathscr{L}(\dot x)}$ and $x(0)=x(1)=p$, $r(0)=r(1)=0$.
%%Since $(M,C')$ is stably causal we can find a $C'$-time function $t'\colon M\to \mathbb{R}$, which can be lifted to $M^\times$ to a function denoted in the same way. The levels sets of the lifted $t'$ are tangent to the $C'{}^\downarrow$-cones along the fiber direction, thus $\dd t'(\dot P)\ge 0$ with equality only if $\dot P$ is vertical. But it cannot be $\dd t'(\dot P)> 0$ at any point since $P$ has to return to the starting point, so the image of $P$ is contained on a $\mathbb{R}$-fiber, which is again impossible, since the curve would not be closed as  $C'{}^\downarrow$ orients it.
%%\end{proof}
%
%
%
%Let $\mu$ be a strictly positive probability measure on $M^\times$, absolutely continuous with respect to the Lebesgue measure.  Let $(M,\mathscr{L})$ be a stably causal spacetime and let  $\bar{\mathscr{L}}>\mathscr{L}$, so that $(M,\bar{C})$ is stably causal.
%% (we do not demand $\bar{D}<+\infty$.
%%such that $D<+\infty$, and let $\bar{\mathscr{L}}$ be the Finsler Lagrangian with the properties mentioned in Theorem \ref{coa}.
%Let $\mathscr{L}_a=(1-\frac{a}{3})\mathscr{L}+\frac{a}{3} \bar{\mathscr{L}}$, $a\in[0,3]$, be a one parameter family of Finsler Lagrangians which interpolates between $\mathscr{L}$ and $\bar{\mathscr{L}}$, in such a way that $\bar{\mathscr{L}}_a<\bar{\mathscr{L}}_{a'}$ for $a<a'$. Observe that it is not true that $C^\downarrow_a<C^\downarrow_{a'}$, for $a<a'$, since both share the downward vertical vectors $(0,z)$, $z\le 0$. Still, we are going to construct a time function on $(M^\times, C^\downarrow)$ using an average procedure analogous to that introduced by Hawking \cite{hawking68,hawking73}.
%
%By distinction of $(M^\times, C^\downarrow_a)$ the function  $t^\downarrow_a(P)=-\mu(I^+_{C^\downarrow_a }(P))$, is increasing over every $C^\downarrow_a$-causal curve and $t^\downarrow_a<t^\downarrow_{a'}$ for $a<a'$. However, it is not necessarily continuous, so the idea is to take the average
%\[
%t^\downarrow(P)=\int_1^2 t^\downarrow_a(P) \dd a = -\int_1^2 \mu(I^+_{C^\downarrow_a }(P)) \dd a
%\]
%
%\begin{theorem} \label{sov}
%Function $t^\downarrow$ is continuous, hence a time function for $(M^\times, C_1^\downarrow)$. Thus $(M^\times, C_1^\downarrow)$ and hence $(M^\times, C^\downarrow)$ are stably causal. The  Seifert relation for the latter is
%\begin{equation} \label{vyv}
%J_S^\downarrow=\{((p,r),(p',r'))\colon (p,p')\in J_S \textrm{ and } r'-r \le D(p,p')\}.
%\end{equation}
%\end{theorem}
%%
%%\begin{proof}
%%It suffices to prove continuity at $P=(p,0)$, $p\in M$. Let $\epsilon >0$,
%%%let $h$ be an auxiliary Riemannian metric on $M$,
%% %let $\mathscr{L}$ be a Lorentz-Finsler Lagrangian compatible with $C^\downarrow$ and
%% let $V$ be a small $C_3$-causally convex (hence $C_a$-causally convex)  open neighborhood of $p$.
%%%$B$ be a small $h$-ball centered at $p$, which is a  neighborhood for $p$ on $(M,C)$.
%%From the proof of Theorem \ref{soq} we know that if $V$ is sufficiently small, every $C_3^\downarrow$-causal curve escaping $W:=V\times \mathbb{R}$ cannot reenter it. We take $V$ so small that $\mu (W) <\epsilon/2$. Now, observe that $Q\in (J^\downarrow_a)^+(P)$, if $Q=(q,r)$, where there is a  $C_a$-causal curve $\gamma$ contained in $V$ connecting $p$ to $q$ and $r \le \ell_a(\gamma)$. So for $P_r=(p,r)$ we also have $Q\in (J^\times_a)^+(P_r)$, in other words $(J^\downarrow_a)^+(P)=\cup_{r\le 0} (J^\times_a)^+(P_r)$.
%%
%%For $a,a'\in [0,3]$, $a<a'$, we have
%%\[
%%(J^\times_a)^+(P)\cap \p W\subset (I^\times_a)^+(P)\cap \p W.
%%\]
%%Notice that that $(J^\times_a)^+(P_r)\cap \bar W$ is really the image of a closed set of $T_{P_r} M^\times$ under an  exponential map relative to Lagrangians $\mathscr{L}_a^\times$ continuous in $a$ which are compatible with $C^\times_a$. This map results from an ODE on the tangent bundle for which the parameter $a$ is an external parameter. The theory of ODE tells us that the exponential map will change continuously with $a$. As a consequence, there is a small neighborhood $\mathscr{A}(a,a')\ni P$ such that
%%\[
%%(J^\times_a)^+(Q)\cap \p W\subset (I^\times_{a'})^+(P)\cap \p W, \quad \forall Q \in \mathscr{A}(a,a').
%%\]
%%and a small neighborhood $\mathscr{B}(a,a')\ni P$ such that
%%\[
%%(J^\times_a)^+(P)\cap \p W\subset (I^\times_{a'})^+(Q)\cap \p W, \quad \forall Q \in \mathscr{B}(a,a').
%%\]
%%By translational invariance similar inclusions holds for $P_r$ in place of $P$, where the novel sets $\mathscr{A}_r$, $\mathscr{B}_r$, are the translates of $\mathscr{A}$ and $\mathscr{B}$. Thus denoting $\mathcal{A}(a,a')=\cup_{r\le 0} \mathscr{A}_r(a,a')$ and  $\mathcal{B}(a,a')=\cup_{r\le 0} \mathscr{B}_r(a,a')$, we have
%%\begin{align*}
%%(J^\downarrow_a)^+(Q)\cap \p W &\subset (I^\downarrow_{a'})^+(P)\cap \p W, \quad \forall Q \in \mathcal{A}(a,a'), \\
%%(J^\downarrow_a)^+(P)\cap \p W&\subset (I^\downarrow_{a'})^+(Q)\cap \p W \quad \forall Q \in \mathcal{B}(a,a').
%%\end{align*}
%%Let $N$ be an integer such that $N>2/\epsilon$, and let us regard $[1,2+\frac{1}{N}]$ as the union of intervals $\mathcal{I}_k=[1+\frac{k}{2N}, 1+\frac{k+1}{2N}]$, $k=0,\cdots, 2N+1$, in such a way that inside every interval $[a,a+\frac{1}{N}]$ for $a\in [1,2]$ one can find an $\mathcal{I}_{\bar k}$ interval for some $\bar k$. Let \[A=\bigcap_{k} \mathcal{A}(1+\tfrac{k}{2N}, 1+\tfrac{k+1}{2N}), \qquad B=\bigcap_{k} \mathcal{B}(1+\tfrac{k}{2N}, 1+\tfrac{k+1}{2N}).\]
%%
%%
%%{\em Lower semi-continuity}. Let $Q\in A$ and $a\in [1,2]$; choosing $\mathcal{I}_k\subset [a,a+\frac{1}{N}]$, we have $Q\in \mathcal{A}(1+\tfrac{k}{2N},1+ \tfrac{k+1}{2N})$ and
%%\begin{align*}
%%(I^\downarrow_a)^+(Q)\cap \p W &\subset (I^\downarrow_{1+\tfrac{k+1}{2N}})^+(P)\cap \p W \subset(I^\downarrow_{a+\frac{1}{N}})^+(P)\cap \p W.
%%\end{align*}
%%Thus $(I^\downarrow_a)^+(Q) \backslash W\subset (I^\downarrow_{a+\frac{1}{N}})^+(P)\backslash W$ hence
%%\[
%%\mu((I^\downarrow_a)^+(Q)) \le \mu((I^\downarrow_{a+\frac{1}{N}})^+(P))+\mu(W)\le \mu((I^\downarrow_{a+\frac{\epsilon}{2}})^+(P))+\tfrac{\epsilon}{2}.
%%\]
%%That is, for every $Q\in A$ and $a \in [1,2]$
%%\[
%%-t^\downarrow_a(Q) \le - t^\downarrow_{a+\frac{\epsilon}{2}}(P)+\tfrac{\epsilon}{2} ,
%%\]
%% and averaging (notice that $-1\le t^\downarrow_s\downarrow\le 0$)
%%\begin{align*}
%%-t^\downarrow(Q)= \int_1^2 t^\downarrow_a(Q) \dd a\le -t^\downarrow(P)-\int_2^{2+\frac{\epsilon}{2}}  t^\downarrow_{s} (P)\dd s +\frac{\epsilon}{2}\le  -t^\downarrow(P)+\epsilon,
%%\end{align*}
%%which proves the lower semi-continuity.
%%
%%{\em Upper semi-continuity}. Let $Q\in B$ and let $a\in [1,2]$; choosing $\mathcal{I}_k\subset [a,a+\frac{1}{N}]$, we have $Q\in \mathcal{B}(1+\tfrac{k}{2N}, 1+\tfrac{k+1}{2N})$ and
%%\begin{align*}
%%(I^\downarrow_a)^+(P)\cap \p W&\subset  (I^\downarrow_{1+\tfrac{k}{2N}})^+(P)\cap \p W \subset (I^\downarrow_{1+\tfrac{k+1}{2N}})^+(Q)\cap \p W \subset (I^\downarrow_{a+\frac{1}{N}})^+(Q).
%%\end{align*}
%%Thus $(I^\downarrow_a)^+(P)\backslash W \subset (I^\downarrow_{a+\frac{1}{N}})^+(Q)\backslash W$ hence
%%\[
%%\mu((I^\downarrow_a)^+(P)) \le \mu((I^\downarrow_{a+\frac{1}{N}})^+(Q))+\mu(W)\le \mu((I^\downarrow_{a+\frac{\epsilon}{2}})^+(Q))+\tfrac{\epsilon}{2}
%%\]
%%That is, for every $Q\in B$ and $a \in [1,2]$
%%\[
%%-t^\downarrow_a(P)\le  - t^\downarrow_{a+\frac{\epsilon}{2}}(Q)+\tfrac{\epsilon}{2}
%%\]
%%and averaging (notice that $-1\le t^\downarrow_s\downarrow\le 0$)
%%\begin{align*}
%%-t^\downarrow(P)\le -t^\downarrow(Q)-\int_2^{2+\frac{\epsilon}{2}}  t^\downarrow_{s} (Q)\dd s +\frac{\epsilon}{2}\le  -t^\downarrow(Q)+\epsilon,
%%\end{align*}
%%which proves the upper semi-continuity.
%%
%%Thus $(M^\times, C^\downarrow)$ is stably causal and hence $K^\downarrow=J_S^\downarrow$. But denoting with $R$ the right-hand side of Eq.\ (\ref{vyv}) it is clear that $R\subset J_S^\downarrow$, since $R$ has been obtained with  openings of cones constrained to be tangent to the fiber direction. Moreover, $R$ is closed (by the upper semi-continuity of $D$) and transitive (by Prop.\ \ref{upq} (c) and contains $J^\downarrow$, thus $K^\downarrow\subset R$. We conclude $R=K^\downarrow=J_S^\downarrow$. (We could have proved stable causality using the equivalence between stable causality and $K$-causality, and the antisymmetry of $R$, cf.\ Theor.\ \ref{aqa} (c). We passed through $t^\downarrow$ since we need to construct a time function with specific properties).
%%\end{proof}
%%
%
%\begin{lemma} \label{idf}
%The smooth temporal functions $F$ on $(M,C^\downarrow)$ whose level sets intersect every $\mathbb{R}$-fiber exactly once and such that $F((q,r_2))-F((q,r_1))=r_1-r_2$, $\forall q\in M$, $r_1,r_2\in \mathbb{R}$, are put in one-to-one correspondence with the smooth strictly steep time functions $f$ on $(M,\mathscr{L})$, through the condition $F^{-1}(0)=\cup_q (q,f(q))$.
%\end{lemma}%
%%
%%\begin{proof}
%%If is clear that the condition $F((q,r_2))-F((q,r_1))=r_1-r_2$ determines $F$ provided  the hypersurface $F^{-1}(0)$ is given. If $F$ is temporal, $\dd F$ is positive on $C^\downarrow$, thus $F^{-1}(0)$ is transverse to every $\mathbb{R}$-fiber, and so $f$ is differentiable and smooth. Moreover, since $\dd F$ is positive on $C^\downarrow$, $0<\dd F((y,\sqrt{-2 \mathscr{L}(y)}))$ for $y\in C\backslash 0$, which reads $\dd f(y)>\sqrt{-2 \mathscr{L}(y)}$, so $f$ is strictly steep. The converse is analogous, since the steep inequality is strict, $\dd F$ is positive on $C^\downarrow$, and so it is temporal.
%%\end{proof}
%
%
%\begin{theorem} \label{sqz}
%Suppose that $(M,\mathscr{L})$ is stably causal and such that $D<+\infty$. Then it   admits a smooth strictly steep time function.
%%Moreover, $(M^\times, C^\downarrow)$ admits a smooth temporal function $G$ whose level sets intersect every $\mathbb{R}$-fiber exactly once and such that $G((q,r_2))-G((q,r_1))=r_2-r_1$.
%\end{theorem}
%%
%%\begin{proof}
%%Let $t^\downarrow$ the $C^\downarrow_1$-time function constructed in the previous theorem where $\bar{\mathscr{L}}>\mathscr{L}$ is chosen so that $\bar{D}<+\infty$. In particular this means that $d_a<\bar{D}<+\infty$, where $d_a$ is the distance function of $\mathscr{L}_a$. Let  $t^\uparrow$ be the minus time function that one would obtain taking the opposite cones  on $M^\times$, and  inverting the orientation of the fibers. Both are time function on $(M^\times, C_1^\downarrow)$, where the former uses the measure of the chronological futures to build the time function, while the latter uses the measure of the chronological pasts. The important point is that over a given fiber $(p,r)$, $r\in \mathbb{R}$, $t^\downarrow \to 0$ for $r \to -\infty$, and  $t^\uparrow \to 0$ for $r \to +\infty$. Let us prove this claim for $t^\downarrow$, the other claim being proved dually. Let $\epsilon>0$, and let $K\times [-G,G]$ be a compact set such that $\mu(M^\times\backslash K\times [-G,G])<\epsilon$. Since $\bar{D}$ is upper semi-continuous, $\bar{D}((p,0), \cdot)$ has an upper bound $R$ on $K$. Hence for every $a\in [1,2]$, $d_a((p,0), \cdot)< R$ on $K$.  As a consequence $(I^\downarrow_a)^+((0,-R-G))\cap K=\emptyset$, for every $a\in [1,2]$, which implies $\vert t ^\downarrow((0,-R-G))\vert<\epsilon$. The $C_1^\downarrow$-time  function $\tau=\log \vert t^\uparrow/t^\downarrow\vert $ is   continuous over every $\mathbb{R}$-fiber with image $(-\infty,+\infty)$ as it goes to $\pm \infty$ for $r\to \mp \infty$. At this point the proof can be concluded in two different ways depending on whether we approximate smoothly on $M^\times$ or on $M$.
%%
%%(First) In the former case we notice that $\tau$ being constructed from volume functions it is anti-Lipschitz with respect to $C_{1/2}^\downarrow$ by construction \cite{chrusciel12}, thus it can be approximated as much as desired by a smooth temporal function $F$, see Cor.\ \ref{soo}, which thus has image in $(-\infty,+\infty)$ over each fiber. Write its zero level set as a graph $\cup_r(r,f(r))$, then $f$ is a smooth strictly steep time function on $(M,C)$.
%%%The function $G$ can be constructed fixing its zero level set to $G_0$ and the using the property mentioned in the statement of the theorem.
%%
%%(Second) The zero level set $S_0=\tau^{-1}(0)$ intersects the fibers exactly once.
%%Let $S_0=\cup_p(p, t(p))$, where $t\colon M\to \mathbb{R}$ is a continuous function, then if $x\colon [0,1] \to M$, is a $C_1$-causal curve, $x(0)=p$, $x(1)=q$, we have $(J_1^\downarrow)^+((p,t(p)))\ni (q, t(p)+ \ell_1(x))$, and since $\tau$ is a $C_1^\downarrow$-time function $\tau((q, t(p)+ \ell_1(x)))\ge 0$, which implies $t(q)\ge  t(p)+ \ell_1(x)$ over every $C_1$-causal curve. Let $h$ be an auxiliary Riemannian metric such that $\sqrt{-2 \mathscr{L}_1(v)}-\sqrt{-2 \mathscr{L}(v)}> h(v)$ over every $C$-causal vector $v$. It exists because the sections of $C$ are compact. Then Theorem \ref{moz} guarantees the existence of a smooth $\mathscr{L}$-steep time function $f$ on $M$ which approximates $t$.
%%\end{proof}
%
%\begin{theorem}
%A  spacetime $(M,g)$ admits an isometric embedding into  $\mathbb{L}^N$ for some $N>0$ if and only if it is  stably causal and such that $D<+\infty$,
%\end{theorem}
%
%\begin{corollary}
%Every stably causal  spacetime $(M,g)$  for which $d$ is finite and continuous  admits an isometric embedding into  $\mathbb{L}^N$ for some $N>0$.
%\end{corollary}
%
%%
%%\begin{example}
%%It is natural to ask whether in stably causal spacetimes the finiteness of $D$ is implied by the finiteness of $d$, and so whether the finiteness of $d$ can be sufficient for the embeddability in Minkowski spacetime. The next example answers in the negative. Let $M$ be the open set of Minkowski 1+1 spacetime depicted in the figure. The metric on $M$ is conformal to Minkowski $g=\varphi(x)(-\dd t^2+\dd x^2)$, $\phi>0$. The open set $M$ is contained in the region $x>0$ from which we remove segments and half-lines. We are really removing the same vertical element, repeated and rescaled a countable number of times.  This vertical element presents two `gates' which the causal curves of $(M,g)$ cannot traverse, thus  $M$ gets separated into a sequence of causally unrelated strips. However, $g'$-causal curves for $g'>g$ can indeed traverse the gates. Let $\varphi$ be bounded on $x>\epsilon$ for every $\epsilon>0$. If $\varphi\to +\infty$ sufficiently fast for $x\to 0$, e.g.\ $\varphi=1/x^2$, then $d$ is finite but $D$ is infinite for some pairs, e.g.\ $D(p,q)=+\infty$. The reason is that by opening the cones there are curves connecting $p$ to $q$ which pass as many gates as desired, acquiring arbitrarily large Lorentzian length thanks to their vertical development near $x=0$. Since $d'(p,q)=+\infty$ for every $g'>g$, we have $D(p,q)=+\infty$.
%%\begin{figure}[ht]
%%\begin{center}
%% \includegraphics[width=5.5cm]{example}
%%\end{center}
%%\caption{An example of stably causal spacetime for which $d$ is finite but which is not isometrically embeddable in Minkowski spacetime. The example is conformal to the displayed subset of 1+1 Minkowski spacetime, where the thick black lines and the region $x\le 0$ have been removed.} \label{acausal}
%%\end{figure}
%%
%%\end{example}
%
%%\begin{lemma}
%%Let $(M,\mathscr{L})$ be a stably causal spacetime such that $D<+\infty$. Every level set of a smooth temporal function on $(M^\times, C^\downarrow)$ which intersects every $\mathbb{R}$-fiber is the graph of a smooth $\mathscr{L}$-steep time function on $(M,\mathscr{L})$.
%%\end{lemma}
%
%
%\begin{lemma} \label{dpx}
%Let $(M,\mathscr{L})$ be a stably causal spacetime. Then every $\mathscr{L}$-steep time function $f$ satisfies  $f(q)-f(p)\ge D(p,q)$ for every $(p,q)\in J_S$.
%\end{lemma}
%%
%%\begin{proof}
%%Since $f$ is steep it satisfies $f(q)-f(p)\ge d(p,q)$ for every $(p,q)\in J$. Thus $S_0=\cup_p (p,f(p))$ is an acausal hypersurface on $(M^\times, C^\downarrow)$ which intersects every $\mathbb{R}$-fiber. A $J^\downarrow$-isotone function $F$ can be easily constructed taking as level sets the translates of $S_0$. But on the stably causal spacetime $(M^\times, C^\downarrow)$  every $J^\downarrow$-isotone function is $J_S^\downarrow$-isotone by \cite[Lemma 4]{minguzzi09c}, thus $J_S^\times$-isotone, thus $S_0$ is $J_S^\times$-acausal which,  given the expression for $J_S^\times$,  reads $f(q)-f(p)\ge D(p,q)$ for every $(p,q)\in J_S$.
%%\end{proof}
%
%
%\begin{theorem}
%Suppose that $(M,\mathscr{L})$ is stably causal and such that $D<+\infty$. Let $\mathscr{S}$ the the family of smooth  steep time functions. They represent
%\begin{itemize}
%\item[(a)] the order $K=J_S$, namely $(p, q)\in J_S \Leftrightarrow f(p)\le f(q), \ \forall f \in \mathscr{S}$;
%\item[(b)] the manifold topology, namely for every open set $O\ni p$ we can find $f,h \in \mathscr{S}$ in such a way that
%$p\in \{q\colon f(q)>0\}\cap \{q\colon h(q)<0\}\subset O$.
%%In fact, we can do more, for every $\epsilon >0$ we can find $f$ and $h$ so that one has also the condition $\vert f\vert, \vert h\vert<\epsilon$ on $U$;
%\item[(c)] the distance, in the sense that  the distance formula holds true: for every $p,q\in M$
%\begin{equation}
% D(p,q)=\mathrm{inf} \big\{[f(q)-f(p)]^+\colon \ f \in \mathscr{S}\big\}.
% %\ \textrm{is smooth and} \ \mathscr{L}\textrm{-steep} \}.
%\end{equation}
%\end{itemize}
%\end{theorem}
%%
%%Actually we prove a stronger result, namely that the smooth temporal functions $F$ on $(M, C^\downarrow)$, which (i) intersect exactly  once every $\mathbb{R}$-fiber, and (ii) satisfy $F((q,r_2))-F((q,r_1))=r_1-r_2$, $\forall q\in M$, $r_1,r_2\in \mathbb{R}$, represent the order and topology of $(M,C^\downarrow)$.
%%
%%\begin{proof}
%%%If $f$ is steep then for $(p,q)\in J_S$, $f(p)\ge f(q its graph $(p,f(p))$ is a spacelike hypersurface.
%%(a). Suppose that $(p,q)\notin J_S$, and let $P=(p,0)$, $Q=(q, 0)$, so that $(P,Q)\notin J_S^\times$. Recall the construction of function $\tau=\log \vert t^\uparrow/t^\downarrow\vert $ on $M^\times$ in the proof of Theorem \ref{sqz}. We repeat it with $\bar{\mathscr{L}}>\mathscr{L}$ chosen so that $(p,q)$ are not $\bar C$-causally related. Observe that in that construction $\mu$ was arbitrary, and that in fact we could use different measures $\mu^\downarrow$ and $\mu^\uparrow$ in the construction of $t^\downarrow$ and $t^\uparrow$. We can choose $\mu^\downarrow$ so that much of its mass is in  $(I^\downarrow_2)^+(Q)$ but not on $(I^\downarrow_2)^+(P)$, in such a way that $t^\downarrow (Q)< t^\downarrow (P)\le 0$. Similarly we can choose $\mu^\uparrow$ so that $0\le t^\uparrow (Q)< t^\uparrow (P)$, ad hence $\tau(Q)<\tau(P)$. Let $\cup_r(r,t(r))$ be the graph of the $\tau$-level set passing through $Q$, then $t(p)>0=t(q)$. Proceeding as in the proof of Theorem \ref{sqz}, $t$ can be approximated as closely as desired by a smooth $\mathscr{L}$-steep time function, $f$ which therefore can be chosen so that $f(p)>f(q)$.
%%
%%(b). Let $\epsilon>0$, and $P=(p,0)$, $p\in M$. Set $O^\times:=O\times (-\epsilon,\epsilon)$.  From Theorem \ref{sqz} we know that there is a smooth temporal functions $G$ on $(M,C^\downarrow)$ whose level sets intersect every $\mathbb{R}$-fiber exactly once and such that  $G(P)=0$ and $G((q,r_2))-G((q,r_1))=r_1-r_2$, for every $q\in M$, $r_1,r_2\in \mathbb{R}$. Let $\mathscr{L}^\downarrow$ be an auxiliary Lorentz-Finsler Lagrangian for $C^\downarrow$. We can find a convex neighborhood $D\subset O^\times\cap G^{-1}((-\delta,\delta))$ where $\delta>0$. Since $G$ is temporal it is temporal and anti-Lipschitz for some $\check C>C^\downarrow$.
%% We can find a small $h$-ball $B\subset G^{-1}((0, +\infty))$ ($h$ auxiliary Riemannian metric) so small and close to $P$ that $P \in F^{-1}([0, +\infty))\cap (\check J)^{-}(\bar B) \subset D\subset O^\times$ and $B\subset (I^\downarrow)^+(P)$. Now let us consider the modification $G'=G(Q)- \mu(\check I^+(Q))$, where $\mu$ is a positive measure of total mass $<\delta$ supported in $B$. Being the sum of anti-Lipschitz functions with respect to $\check C$, see \cite{chrusciel12}, it will have the same property.  Moreover, $G'(P)<0$ and the zero set of $G$ will differ from that of $G'$ just in $D$, where the latter will move `forward'. Since $G'$ is anti-Lipschitz with respect to $\check C$, it can be replaced by a smooth $C^\downarrow$-temporal function $H$ whose zero set intersects every fiber, it is contained in $G^{-1}((-\infty,0))$ outside $O^\times$, and such that $H(P)<0$. A dual result can be obtained placing the mass $\mu$ before $P$, giving a function $F$ in place of $H$, so that $P\in H^{-1}((-\infty,0))\cap F^{-1}((0,+\infty))\subset O^\times$. Now let $h$ and $f$ be defined by $H^{-1}(0)=\cup_q(q,h(q))$, $F^{-1}(0)=\cup_q(q,f(q))$, then they are smooth strictly $\mathscr{L}$-steep temporal functions and outside $O$, $f<h$, while $h(p)<0$, $f(p)>0$ which implies $p\in h^{-1}((-\infty,0))\cap f^{-1}((0,+\infty))\subset O$, as desired.
%%
%%(c). From (a) we know that the equality holds if $(p,q)\notin J_S$. Suppose $(p,q)\in J_S$, due to Lemma \ref{dpx} we have just to prove that for every $\epsilon>0$ there is $g \in \mathscr{S}$ such that $g(q)-g(p)\le D(p,q)+\epsilon$. Let $P=(p,0)$, $Q=(q, D(p,q)+\epsilon)$ so that by Theor.\ \ref{sov} $(P,Q)\notin J^\downarrow_S$.
%%We know that on a stably causal spacetime the Seifert relation is represented by the smooth temporal functions \cite{minguzzi09c} (or Cor.\ \ref{kid}). Thus there is $H\colon M^\times \to [0,1]$ smooth temporal function on $(M^\times, C^\downarrow)$ such that $H(P)>H(Q)$. %Let $\delta=[H(P)-H(Q)]/4>0$, and $A=[H(P)+H(Q)]/2$, then $H(Q)<A-\delta<A<A+\delta<H(Q)$.
%%We know from Lemma \ref{idf} and Theorem \ref{sqz} that there is a smooth temporal function $F$ which intersects every $\mathbb{R}$-fiber and such that $F((q,r_2))-F((q,r_1))=r_1-r_2$, $\forall q\in M$, $r_1,r_2\in \mathbb{R}$. We can choose it to be positive on $P$, $F(P)>0$. Let $G=H+ kF$, $k>0$, then since $H$ is bounded, the level sets of $G$ intersect every $\mathbb{R}$-fiber. Let $0<k< \frac{1}{2} [H(P)-H(G)]/[1+\vert F(Q)\vert]$, then
%%\[
%%G(Q)\le H(Q)+ \tfrac{1}{2} [H(P)-H(G)]<H(P)<G(P).
%%\]
%%We can redefine $G$ by adding a constant to it  and so assume $G(P)=0$.  Let $G^{-1}(0)=\cup_r(r,g(r))$, then $g(p)=0$. Since $G(Q)<0$ we have $D(p,q)+\epsilon> g(q)=g(q)-g(p)$, which is what we wished to prove.
%%\end{proof}
%
%
%\begin{theorem}
%In a stably causal spacetime $(M,\mathscr{L})$
%\begin{equation}
% D(p,q)=\mathrm{inf} \{[f(q)-f(p)]^+\colon f \in \mathscr{V}  \}.
%\end{equation}
%where $\mathscr{V}$ is the set of $J$-isotone continuous function $f\colon M\to [-\infty,+\infty]$ which are $\mathscr{L}$-steep wherever they are finite. Here the convention $\pm \infty- \pm \infty=0$ applies.
%\end{theorem}

%
%\section{Appendix}
%
%For the next theorem and  corollary J.\ Grant, P.\ Chrusciel and the author should be credited, since it is really a polished and improved version of our theorem \cite[Theor.\ 4.8]{chrusciel13}. I didn't change the original wording where it wasn't unnecessary. The modified proof makes  manifest a strong feature hidden in the original proof, namely the possibility of bounding the derivative of the smoothing function.
%
%\begin{theorem} \label{moz}
%  Let $({ M},C)$ be a cone structure and
% let $\tau\colon M\to \mathbb{R}$ be continuous function. Suppose that there is $\hat C>C$ and continuous functions  homogeneous of degree one in the fiber $\underline F, \overline F\colon \hat C\to \mathbb{R}$ such that for every  $\hat C$-causal curve $x\colon [0,1]\to M$
% \begin{equation} \label{mos}
% \int_x \underline F(\dot x) \dd t\le \tau(x(1))-\tau(x(0))\le \int_x \overline F(\dot x) \dd t.
% \end{equation}
% Let $h$ be an arbitrary Riemannian metric, then for every function $\alpha\colon { M} \to (0,+\infty)$ there exists
%a smooth  function $\hat{\tau}$ such that $\vert \hat\tau-\tau\vert <\alpha$ and for every $v\in C$
%\begin{equation} \label{kid}
%\underline F(v)- \vert v\vert_h \le \dd \hat \tau(v) \le \overline F(v)+ \vert v\vert_h .
%\end{equation}
%Similar versions, in which some of the functions $\underline F, \overline F$ do not exist hold true. One has just to drop the corresponding inequalities in (\ref{kid}).
%\end{theorem}
%
%Since $h$ is arbitrary the last inequality can be made as stringent as desired, e.g.\ redefining the metric through multiplication by a small conformal factor.

%
%\begin{proof}
%%
%%Consider $p\in { M}$, let $x^\mu$ be local coordinates near $p$. Let $B_p(3\epsilon(p))$ be a coordinate ball whose closure belongs to the patch of the local coordinates. The coordinates split $TM$ over  $\overline{B_p(3\epsilon(p))}$ as $\overline{B_p(3\epsilon(p))}\times \mathbb{R}^n$, which admits a subbundle $\overline{B_p(3\epsilon(p))}\times \mathbb{S}^{n-1}$. The second projection on $\mathbb{S}^{n-1}$  provides an identification of the fiber induced by the local coordinate system.
%%Let $\hat{f}:TM\to \mathbb{R}$ be a continuous function positive homogeneous of degree one which is negative on $\textrm{Int}\hat C$ and positive on $TM\backslash \hat C$. Being continuous it is uniformly continuous on  the compact set $K=\overline{B_p(3\epsilon(p))}\times \mathbb{S}^{n-1}$, in particular, it is negative on $\{p\}\times C_p\cap\mathbb{S}^{n-1}$, and so it is negative on a projectable neighborhood of this set. In other words $\epsilon$ can be chosen so small that $\overline{B_p(3\epsilon(p))}\times C_p\subset \hat C$. That is, if $v^\mu\p_\mu$ is $C$-causal at $p$, for some components $v^\mu$, then the vector with the same components at  $q\in B_p(3\epsilon(p))$ is $\hat C$-causal.
%
%
%Consider $p\in { M}$, let $x^\mu$ be local coordinates near $p$. Let $B_p(3\epsilon(p))$ be a coordinate ball whose closure belongs to the patch of the local coordinates.
%
%The coordinates split $TM$ over  $\overline{B_p(3\epsilon(p))}$ as $\overline{B_p(3\epsilon(p))}\times \mathbb{R}^n$, which admits a subbundle $\overline{B_p(3\epsilon(p))}\times \mathbb{S}^{n-1}$. The second projection provides an identification of the fiber induced by the local coordinate system.
%
% At $p$ we can find $\check C_p$ such that $C_p<\check C_p<\hat C_p$. By continuity the constant $\epsilon$ can be chosen so small that if we define $\check C=\overline{B_p(3\epsilon(p))}\times C_p$, then we still have $C<\check C<\hat C$ over the neighborhood. Since $\check C$ is translationally invariant if $v$ is $C$-causal at $q\in B_p(3\epsilon(p))$ then  the tangent vector to the curve $q'(s)=q'+vs$, $q'\in B_p(3\epsilon(p))$  is $\check C$-causal and hence $\hat C$-causal as long as $q'(s)$ stays in the neighborhood.
%
%%Similarly, the $h$ norm is a uniformly continuous function on $K$, thus given $\kappa >0$ we can choose $\epsilon$ so small that if $v\in T_q { M}$ and $w\in T_{q'} {M}$ are two non-zero vectors on
%%$T {B_p(3\epsilon(p))}$ such that $v^\mu(q)=w^\mu(q')$, then the
% %ratio of their $h$-norms belongs to $[1/\kappa,\kappa]$.
%
%
%
%Finally, $\underline F$  is a  positive homogeneous continuous function determined by its value on the compact set $K\cap \hat C$, with $K=\overline{B_p(3\epsilon(p))}\times \mathbb{S}^{n-1}$, where it is uniformly continuous, so we can find $\epsilon$ so small that for every $(y,v) (y',v)\in \{\overline{B_p(3\epsilon(p))}\times \mathbb{R}^n\}\cap \hat C$ we have
%\[
%\vert\underline F(y',v)-\underline F(y,v)\vert < \vert v\vert_h/2,
%\]
%and similarly for $\overline F$.
%
%%By continuity, there exists $\epsilon(p)> 0$ so that
%%for all $q$, $q'$ in a relatively compact coordinate ball
%%$B_p(3\epsilon(p))$ of radius $3\epsilon(p)$ centered at $p$ and for
%%all vectors $v(q)=v^\mu(q)\partial_\mu|_q \in {C}_{q}$ the vector
%%$v(q'):=v^\mu(q)\partial_\mu|_{q'} \in T_{q'}{M}$, with
%%coordinate components $v^\mu(q')$ at $q'$ equal to its coordinate components $v^\mu(q)$ at $q$,
%%is $\hat C$-timelike at $q'$. Let $\kappa >0$. The constant $\epsilon$ can be chosen so small
%%that if $v\in T_q { M}$ and $w\in T_{q'} {M}$ are two non-zero vectors on
%%$T {B_p(3\epsilon(p))}$ such that $v^\mu(q)=w^\mu(q')$, then the
%% ratio of their $h$-norms belongs to $[1/\kappa,\kappa]$.
%
%Let $\{{\mathscr O}_i:={B_{p_i}(\epsilon_i)}\}_{i\in\mathbb{N}}$ be a locally finite
%covering of ${ M}$ by such balls.
%Let $\varphi_i$ be a partition of unity subordinate
%to the cover $\{{\mathscr O}_i\}_{i\in\mathbb{N}}$. Choose some
%$0<\eta_j<\epsilon_j$. In local coordinates on ${\mathscr O}_j$ let $\tau_j$
%be defined by convolution with an even
%non-negative function $\chi$,
%supported in the coordinate ball of radius one, with integral one:
%%
%$$
% \tau_j (x) =\left\{
%               \begin{array}{ll}
% \frac 1 {\eta_j^{n+1}} \int_{B_{p_j}(3\epsilon_j)} \chi\left(\frac{y-x}{\eta_j}\right) \tau (y)\, d^{n+1}y, & \hbox{$x\in B_{p_j}(2\epsilon_j)$;} \\
%                 0, & \hbox{otherwise.}
%               \end{array}
%             \right.
%$$
%%
%We define the smooth function
%%
%$$
% \hat \tau:= \sum_j \varphi_j \tau_j
% \;.
%$$
%The non-vanishing terms at each point are finite in number, and
%$\hat\tau$ converges pointwise to $\tau$ as we let the constants
%$\eta_j$ converge to zero. The idea is to control the constants
%$\eta_j$ to get the desired properties for $\hat\tau$.
%
%
%%
%Let $x\in {\mathscr O}_j=B_{p_j}(\epsilon_j)$, and let $v = v^\mu
%\partial_\mu$ be any $C$-causal vector at $x\in {\mathscr O}_j$, of
%$h$-length one, then the curve $x^\mu (s)= x^\mu + v^\mu s$ is
%$\hat C$-timelike
% as long as it stays within $B_{p_j}(3\epsilon_j)$. We observe that $s$ is not the $h$-arc length parametrization of the curve, however it will be sufficient to observe that $\vert \frac{\dd }{\dd s}\vert_h=\vert v\vert_h=1$ at $x$.
%
%% from our choice of $\epsilon$ we have $\inf_{\overline{{\mathscr O}_j}} \vert v^\mu\partial_\mu \vert_h\ge 1/\kappa$,
%%thus for $s_2>s_1$ it holds that $(s_2-s_1)<\kappa(t_2-t_1)$, where $t$
%%is the $h$-arc length parametrization.
%%Let $\kappa C_j$ be the $\hat{g}$-anti-Lipschitz constant over $\overline{{\mathscr O}_j}$.
%
%
%
%We write:
%%
%\begin{align*}
% \hat \tau(x(s))-\hat \tau(x)&= \underbrace{\sum_j \big(\varphi_j (x(s))-\varphi_j (x)\big)\tau_j(x(s))}_{=:I(s)} +
%  \underbrace{
%  \sum_j \varphi_j (x )\big(\tau_j(x(s))-\tau_j(x)\big)
%   }_{=:I\!I(s)}
%   \;.
%\end{align*}
%%
%We have at $x\in {\mathscr O}_j$, (here we use Eq.\ (\ref{mos}))
%%
%\begin{align*}
% %\lim_{t\to 0} \frac{I\!I (s(t)) }{t}=
% %\frac{1}{\vert v(x)\vert_h}
% \lim_{s\to 0} \frac{I\!I (s) }{s}
%     &=
%  \lim_{s\to 0} \frac 1 s
%  \sum_k \varphi_k (x )\big(\tau_k(x+ v s)-\tau_k(x))
%\\
%& =
%  \lim_{s\to 0} \frac 1 s
%  \sum_k \frac {\varphi_k (x )}{\eta_k ^{n+1}}\int_{B_{0}(\epsilon_k)} \chi\left(\frac{z}{\eta_k}\right) \big( \underbrace{\tau (x+vs +z)-\tau (x+z)}_{\ge \int_0^s \underline F(x+z+tv,v)  \dd t  }\big)
%   \, d^{n+1}z
%\\
%&\ge
%  \sum_k \frac {\varphi_k (x )}{\eta_k ^{n+1}}\int_{B_{0}(\epsilon_k)} \chi\left(\frac{z}{\eta_k}\right) \underline F(x+z,v)
%   \, d^{n+1}z.
%\end{align*}
%Thus
%\begin{align*}
% \lim_{t\to 0} \frac{I\!I (s) }{s}-\underline F(x,v)
%&\ge
%  \sum_k \frac {\varphi_k (x )}{\eta_k ^{n+1}}\int_{B_{0}(\epsilon_k)} \chi\left(\frac{z}{\eta_k}\right) [\underline F(x+z,v) - \underline F(x,v)]
%   \, d^{n+1}z\\
%   &\ge  \sum_k \frac {\varphi_k (x )}{\eta_k ^{n+1}}\int_{B_{0}(\epsilon_k)} \chi\left(\frac{z}{\eta_k}\right) (- \vert v\vert_h/2)
%   \, d^{n+1}z\ge -\vert v\vert_h/2
%\end{align*}
%So we arrive at
%\[
%\underline F(x,v)-\vert v\vert_h/2\le \lim_{s\to 0} \frac{I\!I (s) }{s} \le  \overline F(x,v)+\vert v\vert_h/2
%\]
%where the second inequality is obtained following analogous calculations and using the second inequality in (\ref{mos}).
%%The right-hand side is a smoothed approximation $\hat {\underline F}$ of $\underline F$ obtained with a usual convolution method. It is possible to choose the constants $\{\eta_j\}$ relative to the neighborhoods intersecting ${\mathscr O}_j$ so small that it is larger that $\hat {\underline F}(y,w)\ge \underline F(y,w)- \alpha(y)\vert w\vert_h$ for $(y,w)\in T{\mathscr O}_j$ , thus
%%\begin{align*}
%% \lim_{t\to 0} \frac{I\!I (s(t)) }{t} &\ge  \underline F(x,v)- \alpha(x)
%% \end{align*}
%%
%%\\
%%& \ge
%%  \sum_k  {\varphi_k (x )} C_k \ge \min_{k\,:\,{\mathscr O}_k\cap {\mathscr O}_j\ne \emptyset} C_k=:B_j >0
%%  \;,
%
%%
%
%For every $j$ let
%\[
%R_j:=\sup_{k\,:\,{\mathscr O}_k\cap {\mathscr O}_j\ne \emptyset}\sup_{x\in
%\overline{{\mathscr O}_j}} \vert \nabla^h \varphi_k(x)\vert_h\; ,
%\]
%let $N_j$ be the number of distinct sets ${\mathscr O}_k$ which have
%non-empty intersection with ${\mathscr O}_j$, and let us choose  $\eta_j$ so
%small that
%\[
%\sup_{x\in \overline{{\mathscr O}_j}} |\tau(x)-\tau_j(x)|<
%\min_{\ell:{\mathscr O}_\ell\cap {\mathscr O}_j\ne \emptyset} \{\,\frac{1}{N_\ell}
%\inf_{\overline{{\mathscr O}_\ell}}\alpha, \,\frac{1}{2N_\ell
%R_\ell}\}\; .
%\]
%%
%Let $\chi_k$ be the characteristic function of ${\mathscr O}_k$, so that
%$\varphi_k\le \chi_k$.  The sets ${\mathscr O}_j$ and
%$\overline{{\mathscr O}_j}$ intersect the same sets of the covering
%$\{{\mathscr O}_i\}$, which are $N_j$ in number, thus
%%
%\begin{align*}
% \sup_{x\in \overline{{\mathscr O}_j}}  \sum_{k:{\mathscr O}_k\cap
%{\mathscr O}_j\ne \emptyset} \!\!\![\chi_k(x) |\tau(x)-\tau_k(x)|]&\le \! \!\!
%\sum_{k:{\mathscr O}_k\cap {\mathscr O}_j\ne \emptyset}\, \sup_{x\in
%\overline{{\mathscr O}_k}}|\tau(x)-\tau_k(x)|\\
%&\le\!\!\!
%\sum_{k\,:\,{\mathscr O}_k\cap {\mathscr O}_j\ne \emptyset} \frac{1}{2R_j N_j}=
%\frac{1}{2R_j}
% \;.
%\end{align*}
%
%%for $k\,:\,{\mathscr O}_k\cap {\mathscr O}_j\ne \emptyset$, so small that $\sup_{x\in \overline{{\mathscr O}_j}} \sum_{k\,:\,{\mathscr O}_k\cap {\mathscr O}_j\ne \emptyset}|\tau(x)-\tau_k(x)|<\max(\frac{B_j}{2R_j}, \inf_{\overline{{\mathscr O}_j}}\alpha)$.
%%\[
%%\sup_{x\in \overline{{\mathscr O}_j}} |\tau(x)-\tau_j(x)|< \min_{l\,:\,{\mathscr O}_l\cap {\mathscr O}_j\ne \emptyset} \frac{B_l}{2R_l}
%%\]
%
%Then at $x\in {\mathscr O}_j$, (recall that $\vert v\vert_h=1$ at $x$)
%%
%\begin{align*}
%%\bigg | \lim_{t\to 0} \frac{I (s(t)) }{t}\bigg |&=
%\bigg |
%\lim_{s\to 0} \frac{I (s) }{s} \bigg|
%     &= \bigg |
%  \lim_{s\to 0} \sum_k \frac{\varphi_k (x(s))-\varphi_k (x)} s\, \tau_k(x(s))
% \bigg|
%\\
%& = \bigg |
%  \sum_k v\big(\varphi_k(x)\big) \tau_k (x )\bigg |
%= \bigg |\sum_k v\big(\varphi_k(x)\big) \big[\tau(x)-\big(\tau(x)
%-\tau_k (x )\big)\big]
% \bigg|
%\\
% &=  \bigg |\underbrace{v\bigg(\sum_k \varphi_k(x) \bigg)}_{=  v(1)  = 0}\tau (x )
%- \sum_k v\big(\varphi_k(x)\big) (\tau(x) -\tau_k (x ) )
% \bigg|
%\\
% & \le  \sum_k |v\big(\varphi_k(x)\big)| \,|\tau(x) -\tau_k (x )|=\sum_{k\,:\,{\mathscr O}_k\cap {\mathscr O}_j\ne \emptyset} |v\big(\varphi_k(x)\big)|\, |\tau(x) -\tau_k (x )|
%\\
% &\le  R_j \sum_{k:{\mathscr O}_k\cap {\mathscr O}_j\ne \emptyset}  \chi_k(x) |\tau(x) -\tau_k (x )|\le \frac{1}{2}=\frac{\vert v\vert_h}{2}
%  \;.
%\end{align*}
%%
%Hence, for   every $x\in M$ and every $C$-causal vector $v\in
%T_x{ M}$ of $h$-length one,
%%there exists a constant $B_j/2$ such
% we have
%%
%\begin{equation}
%\underline F(x,v)-\vert v\vert_h\le v(\hat\tau)  \le  \overline F(x,v)+\vert v\vert_h.
%\end{equation}
%By positive homogeneity we can drop the condition  $\vert v\vert_h=1$ and so this equation holds for every $C$-causal vector $v$.
%
%%In particular $\hat\tau$ is a differentiable function which is
%%strictly increasing along any $g$-causal curve. By Remark~
%%\ref{R11XII12.1}, the $g$-gradient of $\hat\tau$ is everywhere
%%$g$-timelike.
%Finally, for every $x\in {M}$, there is some $j$ such
%that $x\in {\mathscr O}_j$, hence
%%
%\begin{align*}
%\vert \tau(x)-\hat\tau(x)\vert&=\vert \sum_k \varphi_k(x)[\tau(x)-\tau_k(x)] \vert\le \sum_{k:{\mathscr O}_k\cap {\mathscr O}_j\ne \emptyset} \sup_{x\in \overline{{\mathscr O}_k}} |\tau(x)-\tau_k(x)|\\
%&\le \sum_{k:{\mathscr O}_k\cap {\mathscr O}_j\ne \emptyset} \frac{1}{N_j}
%\inf_{\overline{{\mathscr O}_j}} \alpha\le \alpha(x) \sum_{k:{\mathscr O}_k\cap
%{\mathscr O}_j\ne \emptyset} \frac{1}{N_j}= \alpha(x) \; .
%\end{align*}
%%
%
%
%
%%The desired sequence of functions $\tau_i$ converging to $\tau$ is obtained by replacing in the definition of $\hat\tau$ just given all the $\eta_j$'s by $2^{-i}\eta_j$.
%
%\end{proof}
%
%Note that the smoothness of $\hat\tau$ depends only upon the
%smoothness of ${ M}$, regardless of the smoothness of $C$.
%
%
%\begin{corollary} \label{soo}
%Let $({ M},C)$ be a stably causal cone structure and suppose that there is $\hat C>C$ and a function $\tau$ which is $\gamma$-anti-Lipschitz with respect to $\hat C$. Then for every function $\alpha,\beta \colon { M} \to (0,+\infty)$ there exists
%a smooth  function $\hat{\tau}$,  such that $\vert \tau-\hat \tau\vert<\alpha$, and  $\dd \hat{\tau}(v)\ge (1-\beta)\vert v\vert_\gamma $ for every $v\in C$.
%\end{corollary}
%
%\begin{proof}
%For every $\hat C$-causal curve $x$, $\int_x\sqrt{\gamma(v,v)}\le \tau(x(1))-\tau(x(0)) $. So let $\underline F=\sqrt{\gamma(v,v)}$ and $h=\beta^2\gamma $, then $\hat \tau$ from the theorem satisfies $(1-\beta)\sqrt{\gamma(v,v)}\le \dd \tau (v)$ for every $C$-causal vector $v$.
%\end{proof}


\section*{References}

%\bibliography{../../../bibliografie/simultaneity,../../../bibliografie/libri,../../../bibliografie/miei,../../../bibliografie/mieiPrep,../../../bibliografie/mieiProc}
%\bibliographystyle{iopart-num}
%\bibliographystyle{plain}

\providecommand{\newblock}{}
\begin{thebibliography}{10}
\expandafter\ifx\csname url\endcsname\relax
  \def\url#1{{\tt #1}}\fi
\expandafter\ifx\csname urlprefix\endcsname\relax\def\urlprefix{URL }\fi
\providecommand{\eprint}[2][]{\url{#2}}
% Bibliography created with iopart-num v2.0
% /biblio/bibtex/contrib/iopart-num

\bibitem{minguzzi17}
Minguzzi E 2017 Causality theory for closed cone structures with applications
  arXiv:1709.06494

\bibitem{nachbin65}
Nachbin L 1965 {\em Topology and order\/} (Princeton: D.\ {V}an {N}ostrand
  {C}ompany, {I}nc.)

\bibitem{minguzzi12d}
Minguzzi E 2013 {\em Topol. Appl.\/} {\bf 160} 965--978 {arXiv}:1212.3776

\bibitem{aubin84}
Aubin J~P and Cellina A 1984 {\em Differential inclusions\/} ({\em Grundlehren
  der Mathematischen Wissenschaften [Fundamental Principles of Mathematical
  Sciences]\/} vol 264) (Springer-Verlag, Berlin)

\bibitem{fathi12}
Fathi A and Siconolfi A 2012 {\em Math. {P}roc. {C}amb. {P}hil. {S}oc.\/} {\bf
  152} 303--339

\bibitem{minguzzi08b}
Minguzzi E 2009 {\em Commun. Math. Phys.\/} {\bf 290} 239--248
  {arXiv}:0809.1214

\bibitem{minguzzi09c}
Minguzzi E 2010 {\em Commun. Math. Phys.\/} {\bf 298} 855--868
  {arXiv}:0909.0890

\bibitem{bernard16}
Bernard P and Suhr S Lyapounov functions of closed cone fields: from {C}onley
  theory to time functions {arXiv}:1512.08410v2. Replaces a previous
   work by Suhr ``On the existence of steep temporal functions''.

\bibitem{connes94}
Connes A 1994 {\em Noncommutative geometry\/} (Academic Press, Inc., San Diego,
  CA)

\bibitem{parfionov00}
Parfionov G~N and Zapatrin R~R 2000 {\em J. Math. Phys.\/} {\bf 41} 7122--7128

\bibitem{moretti03}
Moretti V 2003 {\em Rev. Math. Phys.\/} {\bf 15} 1171--1217

\bibitem{franco10}
Franco N 2010 {\em SIGMA Symmetry Integrability Geom. Methods Appl.\/} {\bf 6}
  Paper 064, 11

\bibitem{hawking68}
Hawking S~W 1968 {\em Proc. {R}oy. {S}oc. {L}ondon, series {A}\/} {\bf 308}
  433--435

\bibitem{hawking73}
Hawking S~W and Ellis G~F~R 1973 {\em The Large Scale Structure of
  Space-Time\/} (Cambridge: Cambridge {U}niversity {P}ress)

\bibitem{chrusciel13}
Chru{\'s}ciel P~T, Grant J~D~E and Minguzzi E 2016 {\em Ann. Henri
  {P}oincar{\'e}\/} {\bf 17} 2801--2824 {arXiv}:1301.2909

\bibitem{muller11}
M{\"u}ller O and S{\'a}nchez M 2011 {\em Trans. Am. Math. Soc.\/} {\bf 363}
  5367--5379

\bibitem{minguzzi16a}
Minguzzi E 2016 {\em Class. Quantum Grav.\/} {\bf 33} 115001 {arXiv}:1601.05932

\bibitem{suhr15}
Suhr S On the existence of steep temporal functions {arXiv:}1512.08410

\bibitem{minguzzi11e}
{Benavides Navarro} J~J and Minguzzi E 2011 {\em J. Math. Phys.\/} {\bf 52}
  112504 {arXiv}:1108.5120

\bibitem{hirsch76}
Hirsch M~W 1976 {\em Differential topology\/} (New York: {Springer-Verlag})

\bibitem{samann16}
S{\"a}mann C 2016 {\em Ann. Henri Poincar{\'e}\/} {\bf 17} 1429--1455

\end{thebibliography}

\end{document}
