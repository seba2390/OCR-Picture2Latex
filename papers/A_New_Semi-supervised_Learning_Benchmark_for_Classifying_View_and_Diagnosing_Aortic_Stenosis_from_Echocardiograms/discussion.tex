We have developed and evaluated a semi-supervised learning pipeline that can leverage abundant unlabeled data to deliver competitive patient-level diagnostic predictions for the fully-automated preliminary assessment of aortic stenosis. These methods overcome two challenges. First, echocardiograms are challenging to label so the amount of labeled training data is limited. Second, a patient's record will contain hundreds of images, many of which are not relevant for diagnosing AS.
Here, we briefly review the limitations and advantages of our approach.

\paragraph{Limitations.} 
The most important caveat to this work is the need for further independent validation of our methodology.
For logistical reasons, all our data come from one institution.
A detailed evaluation at another institution would be needed to properly assess our proposed pipeline's utility in a prospective setting when used with different patient populations, imaging devices, and sonographers.

A critical and well-known issue with interpretation of echocardiograms is \emph{inter-rater reliability}~\citep{sacchiDopplerAssessmentAortic2018}.
In particular, the distinction between mild and moderate or moderate and severe diagnostic levels can vary across annotators.
All labels in our dataset come from less than 5 expert annotators at one institution.
Further study is needed to understand if our approach could match the consensus of a broader population of annotators.

Several other opportunities to improve our pipeline exist.
Our image processing approach prioritizes simplicity but does not take advantage of recent larger CNN architectures or region-specific attention or segmentation as in some past work on cardiac imaging~\citep{chenDeepLearningCardiac2020}. 
We could use higher-resolution images.
We could include other easily-measured covariates (besides imaging) in our diagnostic model, such as age, demographics, comorbidities, and other cardio-mechanical signals.

A final limitation is that further effort to qualitatively understand what visual signals are driving predictions is needed to build trust.
We plan to investigate saliency maps~\citep{simonyanDeepConvolutionalNetworks2014,selvarajuGradCAMVisualExplanations2020}, though we will be mindful of the limitations of these methods~\citep{adebayoSanityChecksSaliency2018}.
Qualitative insight is key, because fundamentally, MixMatch works by \emph{interpolating} images. It is surprising to us that MixMatch delivered consistently improved results (replicated across several train/test splits) in a real medical imaging scenario, because interpolated echocardiogram images have questionable meaning to human experts.
%Further investigation is need to understand if  to improve the validity of interpolation (perhaps by only interpolating between similar views or aligning anatomical features).

\paragraph{Advantages.} For potentially fatal conditions like AS, echocardiograms remain the gold standard source of information to produce a diagnosis. Our approach can already reach performance levels (90\% balanced accuracy) that might be useful in a deployed setting (naturally, these must first be reliably replicated in a prospective setting on an external cohort). 
Automatic diagnostic classification pipelines have the potential to identify individuals who would benefit from further screening who otherwise would not be discovered due to limited access to expert cardiologists.

A key aspect of our approach is demonstrating the value of \emph{semi-supervised} learning for a real medical task with class-imbalance (for our view task over 80\% of the images are ``Other'').
%Our methods were trained on labeled imaging studies from 156 patients and unlabeled studies from over 2600 patients.
Our dataset also includes truly unlabeled data from over 2400 patients, which represents a more authentic test of SSL than previous benchmark datasets.
Overall, our work motivates modern SSL as a promising cost-effective way to improve performance if unlabeled data is abundant, even for real clinical images with substantial diversity.
Especially if labeled sets are small, the gains from SSL may be even greater (see Table~\ref{tab:view classification small}). 


A final advantage of our work is the demonstration that patient-level diagnosis benefits from \emph{prioritizing relevant views}. Building on \citet{madaniDeepEchocardiographyDataefficient2018}, who showed promising SSL diagnosis given manually-curated views, our SSL methods can deliver effective diagnoses given an uncurated set of \emph{all} available images, even when most depict irrelevant views.

We hope our study marks a step toward effective early detection of aortic stenosis that can enable helpful interventions. We further hope this study and the accompanying dataset release offer a reproducible template for improving patient outcomes for other diseases where medical imaging is key and labeled data is scarce.

