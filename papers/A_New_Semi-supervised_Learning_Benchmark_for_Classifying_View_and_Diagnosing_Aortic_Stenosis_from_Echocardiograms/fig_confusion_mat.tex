\newcommand{\BW}{0.23}
\setlength{\tabcolsep}{0.1cm}
\begin{figure}
\begin{tabular}{r c c c c}
    & Split 1 & Split 2 & Split 3 & Split 4
    \\
    {\rotatebox{90}{~~~~~~Basic WRN}}
    & 
    \includegraphics[width=\BW\textwidth]{figures/fold0_multitask_FS_SimpleAverage.pdf}
    &
    \includegraphics[width=\BW\textwidth]{figures/fold1_multitask_FS_SimpleAverage.pdf}
    &
    \includegraphics[width=\BW\textwidth]{figures/fold2_multitask_FS_SimpleAverage.pdf}
    &
        \includegraphics[width=\BW\textwidth]{figures/fold3_multitask_FS_SimpleAverage.pdf}
    \\
    {\rotatebox{90}{~~pretrained MixMatch}}
    & 
    \includegraphics[width=\BW\textwidth]{figures/fold0_multitask_MTM_prioritized.pdf}
    &
    \includegraphics[width=\BW\textwidth]{figures/fold1_multitask_MTM_prioritized.pdf}
    &
    \includegraphics[width=\BW\textwidth]{figures/fold2_multitask_MTM_prioritized.pdf}
    &
    \includegraphics[width=\BW\textwidth]{figures/fold3_multitask_MTM_prioritized.pdf}
    \end{tabular}	
    \caption{Confusion matrices for the patient-level AS diagnosis classification, across all four train/test splits of the \textbf{full-size \datasetName-156-52} dataset.
    Each split's test set contains 52 patients.
    We show the labeled-set only basic WRN (top) as well as the best SSL method (bottom).
    Our method improves accuracy compared to the baseline, and most remaining mistakes confuse nearby classes (e.g. ``no AS'' vs. ``mild/moderate AS'') instead of distant classes (e.g. ``no AS'' vs. ``severe AS''). }
    \label{fig:confusion_matrix}
\end{figure}