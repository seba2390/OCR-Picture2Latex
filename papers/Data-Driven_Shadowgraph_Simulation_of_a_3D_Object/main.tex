
\documentclass{article} % For LaTeX2e
\usepackage{iclr2021_conference,times}
\usepackage[english]{babel}
% Optional math commands from https://github.com/goodfeli/dlbook_notation.
%%%%% NEW MATH DEFINITIONS %%%%%

\usepackage{amsmath,amsfonts,bm}

% Mark sections of captions for referring to divisions of figures
\newcommand{\figleft}{{\em (Left)}}
\newcommand{\figcenter}{{\em (Center)}}
\newcommand{\figright}{{\em (Right)}}
\newcommand{\figtop}{{\em (Top)}}
\newcommand{\figbottom}{{\em (Bottom)}}
\newcommand{\captiona}{{\em (a)}}
\newcommand{\captionb}{{\em (b)}}
\newcommand{\captionc}{{\em (c)}}
\newcommand{\captiond}{{\em (d)}}

% Highlight a newly defined term
\newcommand{\newterm}[1]{{\bf #1}}


% Figure reference, lower-case.
\def\figref#1{figure~\ref{#1}}
% Figure reference, capital. For start of sentence
\def\Figref#1{Figure~\ref{#1}}
\def\twofigref#1#2{figures \ref{#1} and \ref{#2}}
\def\quadfigref#1#2#3#4{figures \ref{#1}, \ref{#2}, \ref{#3} and \ref{#4}}
% Section reference, lower-case.
\def\secref#1{section~\ref{#1}}
% Section reference, capital.
\def\Secref#1{Section~\ref{#1}}
% Reference to two sections.
\def\twosecrefs#1#2{sections \ref{#1} and \ref{#2}}
% Reference to three sections.
\def\secrefs#1#2#3{sections \ref{#1}, \ref{#2} and \ref{#3}}
% Reference to an equation, lower-case.
% \def\eqref#1{equation~\ref{#1}}
 \def\eqref#1{(\ref{#1})}
% Reference to an equation, upper case
\def\Eqref#1{Equation~\ref{#1}}
% A raw reference to an equation---avoid using if possible
\def\plaineqref#1{\ref{#1}}
% Reference to a chapter, lower-case.
\def\chapref#1{chapter~\ref{#1}}
% Reference to an equation, upper case.
\def\Chapref#1{Chapter~\ref{#1}}
% Reference to a range of chapters
\def\rangechapref#1#2{chapters\ref{#1}--\ref{#2}}
% Reference to an algorithm, lower-case.
\def\algref#1{algorithm~\ref{#1}}
% Reference to an algorithm, upper case.
\def\Algref#1{Algorithm~\ref{#1}}
\def\twoalgref#1#2{algorithms \ref{#1} and \ref{#2}}
\def\Twoalgref#1#2{Algorithms \ref{#1} and \ref{#2}}
% Reference to a part, lower case
\def\partref#1{part~\ref{#1}}
% Reference to a part, upper case
\def\Partref#1{Part~\ref{#1}}
\def\twopartref#1#2{parts \ref{#1} and \ref{#2}}

\def\ceil#1{\lceil #1 \rceil}
\def\floor#1{\lfloor #1 \rfloor}
\def\1{\bm{1}}
\newcommand{\train}{\mathcal{D}}
\newcommand{\valid}{\mathcal{D_{\mathrm{valid}}}}
\newcommand{\test}{\mathcal{D_{\mathrm{test}}}}

\def\eps{{\epsilon}}


% Random variables
\def\reta{{\textnormal{$\eta$}}}
\def\ra{{\textnormal{a}}}
\def\rb{{\textnormal{b}}}
\def\rc{{\textnormal{c}}}
\def\rd{{\textnormal{d}}}
\def\re{{\textnormal{e}}}
\def\rf{{\textnormal{f}}}
\def\rg{{\textnormal{g}}}
\def\rh{{\textnormal{h}}}
\def\ri{{\textnormal{i}}}
\def\rj{{\textnormal{j}}}
\def\rk{{\textnormal{k}}}
\def\rl{{\textnormal{l}}}
% rm is already a command, just don't name any random variables m
\def\rn{{\textnormal{n}}}
\def\ro{{\textnormal{o}}}
\def\rp{{\textnormal{p}}}
\def\rq{{\textnormal{q}}}
\def\rr{{\textnormal{r}}}
\def\rs{{\textnormal{s}}}
\def\rt{{\textnormal{t}}}
\def\ru{{\textnormal{u}}}
\def\rv{{\textnormal{v}}}
\def\rw{{\textnormal{w}}}
\def\rx{{\textnormal{x}}}
\def\ry{{\textnormal{y}}}
\def\rz{{\textnormal{z}}}

% Random vectors
\def\rvepsilon{{\mathbf{\epsilon}}}
\def\rvtheta{{\mathbf{\theta}}}
\def\rva{{\mathbf{a}}}
\def\rvb{{\mathbf{b}}}
\def\rvc{{\mathbf{c}}}
\def\rvd{{\mathbf{d}}}
\def\rve{{\mathbf{e}}}
\def\rvf{{\mathbf{f}}}
\def\rvg{{\mathbf{g}}}
\def\rvh{{\mathbf{h}}}
\def\rvu{{\mathbf{i}}}
\def\rvj{{\mathbf{j}}}
\def\rvk{{\mathbf{k}}}
\def\rvl{{\mathbf{l}}}
\def\rvm{{\mathbf{m}}}
\def\rvn{{\mathbf{n}}}
\def\rvo{{\mathbf{o}}}
\def\rvp{{\mathbf{p}}}
\def\rvq{{\mathbf{q}}}
\def\rvr{{\mathbf{r}}}
\def\rvs{{\mathbf{s}}}
\def\rvt{{\mathbf{t}}}
\def\rvu{{\mathbf{u}}}
\def\rvv{{\mathbf{v}}}
\def\rvw{{\mathbf{w}}}
\def\rvx{{\mathbf{x}}}
\def\rvy{{\mathbf{y}}}
\def\rvz{{\mathbf{z}}}

% Elements of random vectors
\def\erva{{\textnormal{a}}}
\def\ervb{{\textnormal{b}}}
\def\ervc{{\textnormal{c}}}
\def\ervd{{\textnormal{d}}}
\def\erve{{\textnormal{e}}}
\def\ervf{{\textnormal{f}}}
\def\ervg{{\textnormal{g}}}
\def\ervh{{\textnormal{h}}}
\def\ervi{{\textnormal{i}}}
\def\ervj{{\textnormal{j}}}
\def\ervk{{\textnormal{k}}}
\def\ervl{{\textnormal{l}}}
\def\ervm{{\textnormal{m}}}
\def\ervn{{\textnormal{n}}}
\def\ervo{{\textnormal{o}}}
\def\ervp{{\textnormal{p}}}
\def\ervq{{\textnormal{q}}}
\def\ervr{{\textnormal{r}}}
\def\ervs{{\textnormal{s}}}
\def\ervt{{\textnormal{t}}}
\def\ervu{{\textnormal{u}}}
\def\ervv{{\textnormal{v}}}
\def\ervw{{\textnormal{w}}}
\def\ervx{{\textnormal{x}}}
\def\ervy{{\textnormal{y}}}
\def\ervz{{\textnormal{z}}}

% Random matrices
\def\rmA{{\mathbf{A}}}
\def\rmB{{\mathbf{B}}}
\def\rmC{{\mathbf{C}}}
\def\rmD{{\mathbf{D}}}
\def\rmE{{\mathbf{E}}}
\def\rmF{{\mathbf{F}}}
\def\rmG{{\mathbf{G}}}
\def\rmH{{\mathbf{H}}}
\def\rmI{{\mathbf{I}}}
\def\rmJ{{\mathbf{J}}}
\def\rmK{{\mathbf{K}}}
\def\rmL{{\mathbf{L}}}
\def\rmM{{\mathbf{M}}}
\def\rmN{{\mathbf{N}}}
\def\rmO{{\mathbf{O}}}
\def\rmP{{\mathbf{P}}}
\def\rmQ{{\mathbf{Q}}}
\def\rmR{{\mathbf{R}}}
\def\rmS{{\mathbf{S}}}
\def\rmT{{\mathbf{T}}}
\def\rmU{{\mathbf{U}}}
\def\rmV{{\mathbf{V}}}
\def\rmW{{\mathbf{W}}}
\def\rmX{{\mathbf{X}}}
\def\rmY{{\mathbf{Y}}}
\def\rmZ{{\mathbf{Z}}}

% Elements of random matrices
\def\ermA{{\textnormal{A}}}
\def\ermB{{\textnormal{B}}}
\def\ermC{{\textnormal{C}}}
\def\ermD{{\textnormal{D}}}
\def\ermE{{\textnormal{E}}}
\def\ermF{{\textnormal{F}}}
\def\ermG{{\textnormal{G}}}
\def\ermH{{\textnormal{H}}}
\def\ermI{{\textnormal{I}}}
\def\ermJ{{\textnormal{J}}}
\def\ermK{{\textnormal{K}}}
\def\ermL{{\textnormal{L}}}
\def\ermM{{\textnormal{M}}}
\def\ermN{{\textnormal{N}}}
\def\ermO{{\textnormal{O}}}
\def\ermP{{\textnormal{P}}}
\def\ermQ{{\textnormal{Q}}}
\def\ermR{{\textnormal{R}}}
\def\ermS{{\textnormal{S}}}
\def\ermT{{\textnormal{T}}}
\def\ermU{{\textnormal{U}}}
\def\ermV{{\textnormal{V}}}
\def\ermW{{\textnormal{W}}}
\def\ermX{{\textnormal{X}}}
\def\ermY{{\textnormal{Y}}}
\def\ermZ{{\textnormal{Z}}}

% Vectors
\def\vzero{{\bm{0}}}
\def\vone{{\bm{1}}}
\def\vmu{{\bm{\mu}}}
\def\vtheta{{\bm{\theta}}}
\def\va{{\bm{a}}}
\def\vb{{\bm{b}}}
\def\vc{{\bm{c}}}
\def\vd{{\bm{d}}}
\def\ve{{\bm{e}}}
\def\vf{{\bm{f}}}
\def\vg{{\bm{g}}}
\def\vh{{\bm{h}}}
\def\vi{{\bm{i}}}
\def\vj{{\bm{j}}}
\def\vk{{\bm{k}}}
\def\vl{{\bm{l}}}
\def\vm{{\bm{m}}}
\def\vn{{\bm{n}}}
\def\vo{{\bm{o}}}
\def\vp{{\bm{p}}}
\def\vq{{\bm{q}}}
\def\vr{{\bm{r}}}
\def\vs{{\bm{s}}}
\def\vt{{\bm{t}}}
\def\vu{{\bm{u}}}
\def\vv{{\bm{v}}}
\def\vw{{\bm{w}}}
\def\vx{{\bm{x}}}
\def\vy{{\bm{y}}}
\def\vz{{\bm{z}}}

% Elements of vectors
\def\evalpha{{\alpha}}
\def\evbeta{{\beta}}
\def\evepsilon{{\epsilon}}
\def\evlambda{{\lambda}}
\def\evomega{{\omega}}
\def\evmu{{\mu}}
\def\evpsi{{\psi}}
\def\evsigma{{\sigma}}
\def\evtheta{{\theta}}
\def\eva{{a}}
\def\evb{{b}}
\def\evc{{c}}
\def\evd{{d}}
\def\eve{{e}}
\def\evf{{f}}
\def\evg{{g}}
\def\evh{{h}}
\def\evi{{i}}
\def\evj{{j}}
\def\evk{{k}}
\def\evl{{l}}
\def\evm{{m}}
\def\evn{{n}}
\def\evo{{o}}
\def\evp{{p}}
\def\evq{{q}}
\def\evr{{r}}
\def\evs{{s}}
\def\evt{{t}}
\def\evu{{u}}
\def\evv{{v}}
\def\evw{{w}}
\def\evx{{x}}
\def\evy{{y}}
\def\evz{{z}}

% Matrix
\def\mA{{\bm{A}}}
\def\mB{{\bm{B}}}
\def\mC{{\bm{C}}}
\def\mD{{\bm{D}}}
\def\mE{{\bm{E}}}
\def\mF{{\bm{F}}}
\def\mG{{\bm{G}}}
\def\mH{{\bm{H}}}
\def\mI{{\bm{I}}}
\def\mJ{{\bm{J}}}
\def\mK{{\bm{K}}}
\def\mL{{\bm{L}}}
\def\mM{{\bm{M}}}
\def\mN{{\bm{N}}}
\def\mO{{\bm{O}}}
\def\mP{{\bm{P}}}
\def\mQ{{\bm{Q}}}
\def\mR{{\bm{R}}}
\def\mS{{\bm{S}}}
\def\mT{{\bm{T}}}
\def\mU{{\bm{U}}}
\def\mV{{\bm{V}}}
\def\mW{{\bm{W}}}
\def\mX{{\bm{X}}}
\def\mY{{\bm{Y}}}
\def\mZ{{\bm{Z}}}
\def\mBeta{{\bm{\beta}}}
\def\mPhi{{\bm{\Phi}}}
\def\mLambda{{\bm{\Lambda}}}
\def\mSigma{{\bm{\Sigma}}}

% Tensor
\DeclareMathAlphabet{\mathsfit}{\encodingdefault}{\sfdefault}{m}{sl}
\SetMathAlphabet{\mathsfit}{bold}{\encodingdefault}{\sfdefault}{bx}{n}
\newcommand{\tens}[1]{\bm{\mathsfit{#1}}}
\def\tA{{\tens{A}}}
\def\tB{{\tens{B}}}
\def\tC{{\tens{C}}}
\def\tD{{\tens{D}}}
\def\tE{{\tens{E}}}
\def\tF{{\tens{F}}}
\def\tG{{\tens{G}}}
\def\tH{{\tens{H}}}
\def\tI{{\tens{I}}}
\def\tJ{{\tens{J}}}
\def\tK{{\tens{K}}}
\def\tL{{\tens{L}}}
\def\tM{{\tens{M}}}
\def\tN{{\tens{N}}}
\def\tO{{\tens{O}}}
\def\tP{{\tens{P}}}
\def\tQ{{\tens{Q}}}
\def\tR{{\tens{R}}}
\def\tS{{\tens{S}}}
\def\tT{{\tens{T}}}
\def\tU{{\tens{U}}}
\def\tV{{\tens{V}}}
\def\tW{{\tens{W}}}
\def\tX{{\tens{X}}}
\def\tY{{\tens{Y}}}
\def\tZ{{\tens{Z}}}


% Graph
\def\gA{{\mathcal{A}}}
\def\gB{{\mathcal{B}}}
\def\gC{{\mathcal{C}}}
\def\gD{{\mathcal{D}}}
\def\gE{{\mathcal{E}}}
\def\gF{{\mathcal{F}}}
\def\gG{{\mathcal{G}}}
\def\gH{{\mathcal{H}}}
\def\gI{{\mathcal{I}}}
\def\gJ{{\mathcal{J}}}
\def\gK{{\mathcal{K}}}
\def\gL{{\mathcal{L}}}
\def\gM{{\mathcal{M}}}
\def\gN{{\mathcal{N}}}
\def\gO{{\mathcal{O}}}
\def\gP{{\mathcal{P}}}
\def\gQ{{\mathcal{Q}}}
\def\gR{{\mathcal{R}}}
\def\gS{{\mathcal{S}}}
\def\gT{{\mathcal{T}}}
\def\gU{{\mathcal{U}}}
\def\gV{{\mathcal{V}}}
\def\gW{{\mathcal{W}}}
\def\gX{{\mathcal{X}}}
\def\gY{{\mathcal{Y}}}
\def\gZ{{\mathcal{Z}}}

% Sets
\def\sA{{\mathbb{A}}}
\def\sB{{\mathbb{B}}}
\def\sC{{\mathbb{C}}}
\def\sD{{\mathbb{D}}}
% Don't use a set called E, because this would be the same as our symbol
% for expectation.
\def\sF{{\mathbb{F}}}
\def\sG{{\mathbb{G}}}
\def\sH{{\mathbb{H}}}
\def\sI{{\mathbb{I}}}
\def\sJ{{\mathbb{J}}}
\def\sK{{\mathbb{K}}}
\def\sL{{\mathbb{L}}}
\def\sM{{\mathbb{M}}}
\def\sN{{\mathbb{N}}}
\def\sO{{\mathbb{O}}}
\def\sP{{\mathbb{P}}}
\def\sQ{{\mathbb{Q}}}
\def\sR{{\mathbb{R}}}
\def\sS{{\mathbb{S}}}
\def\sT{{\mathbb{T}}}
\def\sU{{\mathbb{U}}}
\def\sV{{\mathbb{V}}}
\def\sW{{\mathbb{W}}}
\def\sX{{\mathbb{X}}}
\def\sY{{\mathbb{Y}}}
\def\sZ{{\mathbb{Z}}}

% Entries of a matrix
\def\emLambda{{\Lambda}}
\def\emA{{A}}
\def\emB{{B}}
\def\emC{{C}}
\def\emD{{D}}
\def\emE{{E}}
\def\emF{{F}}
\def\emG{{G}}
\def\emH{{H}}
\def\emI{{I}}
\def\emJ{{J}}
\def\emK{{K}}
\def\emL{{L}}
\def\emM{{M}}
\def\emN{{N}}
\def\emO{{O}}
\def\emP{{P}}
\def\emQ{{Q}}
\def\emR{{R}}
\def\emS{{S}}
\def\emT{{T}}
\def\emU{{U}}
\def\emV{{V}}
\def\emW{{W}}
\def\emX{{X}}
\def\emY{{Y}}
\def\emZ{{Z}}
\def\emSigma{{\Sigma}}

% entries of a tensor
% Same font as tensor, without \bm wrapper
\newcommand{\etens}[1]{\mathsfit{#1}}
\def\etLambda{{\etens{\Lambda}}}
\def\etA{{\etens{A}}}
\def\etB{{\etens{B}}}
\def\etC{{\etens{C}}}
\def\etD{{\etens{D}}}
\def\etE{{\etens{E}}}
\def\etF{{\etens{F}}}
\def\etG{{\etens{G}}}
\def\etH{{\etens{H}}}
\def\etI{{\etens{I}}}
\def\etJ{{\etens{J}}}
\def\etK{{\etens{K}}}
\def\etL{{\etens{L}}}
\def\etM{{\etens{M}}}
\def\etN{{\etens{N}}}
\def\etO{{\etens{O}}}
\def\etP{{\etens{P}}}
\def\etQ{{\etens{Q}}}
\def\etR{{\etens{R}}}
\def\etS{{\etens{S}}}
\def\etT{{\etens{T}}}
\def\etU{{\etens{U}}}
\def\etV{{\etens{V}}}
\def\etW{{\etens{W}}}
\def\etX{{\etens{X}}}
\def\etY{{\etens{Y}}}
\def\etZ{{\etens{Z}}}

% The true underlying data generating distribution
\newcommand{\pdata}{p_{\rm{data}}}
% The empirical distribution defined by the training set
\newcommand{\ptrain}{\hat{p}_{\rm{data}}}
\newcommand{\Ptrain}{\hat{P}_{\rm{data}}}
% The model distribution
\newcommand{\pmodel}{p_{\rm{model}}}
\newcommand{\Pmodel}{P_{\rm{model}}}
\newcommand{\ptildemodel}{\tilde{p}_{\rm{model}}}
% Stochastic autoencoder distributions
\newcommand{\pencode}{p_{\rm{encoder}}}
\newcommand{\pdecode}{p_{\rm{decoder}}}
\newcommand{\precons}{p_{\rm{reconstruct}}}

\newcommand{\laplace}{\mathrm{Laplace}} % Laplace distribution

\newcommand{\E}{\mathbb{E}}
\newcommand{\Ls}{\mathcal{L}}
\newcommand{\R}{\mathbb{R}}
\newcommand{\emp}{\tilde{p}}
\newcommand{\lr}{\alpha}
\newcommand{\reg}{\lambda}
\newcommand{\rect}{\mathrm{rectifier}}
\newcommand{\softmax}{\mathrm{softmax}}
\newcommand{\sigmoid}{\sigma}
\newcommand{\softplus}{\zeta}
\newcommand{\KL}{D_{\mathrm{KL}}}
\newcommand{\Var}{\mathrm{Var}}
\newcommand{\standarderror}{\mathrm{SE}}
\newcommand{\Cov}{\mathrm{Cov}}
% Wolfram Mathworld says $L^2$ is for function spaces and $\ell^2$ is for vectors
% But then they seem to use $L^2$ for vectors throughout the site, and so does
% wikipedia.
\newcommand{\normlzero}{L^0}
\newcommand{\normlone}{L^1}
\newcommand{\normltwo}{L^2}
\newcommand{\normlp}{L^p}
\newcommand{\normmax}{L^\infty}

\newcommand{\parents}{Pa} % See usage in notation.tex. Chosen to match Daphne's book.

\DeclareMathOperator*{\argmax}{arg\,max}
\DeclareMathOperator*{\argmin}{arg\,min}

\DeclareMathOperator{\sign}{sign}
\DeclareMathOperator{\Tr}{Tr}
\let\ab\allowbreak

\newcommand{\norm}[2]{\left\| #1 \right\|_{#2}}

\newcommand{\zz}[1]{\textcolor{blue}{ [{\em Zhihui:} #1]}}
\newcommand{\jz}[1]{\textcolor{red}{ [{\em JZ:} #1]}}
% \newcommand{\td}[1]{\textcolor{blue}{ [{\em TD:} #1]}}
\newcommand{\jj}[1]{\textcolor{pink}{ [{\em JJ:} #1]}}

\usepackage{graphicx}
\usepackage{wrapfig}
\usepackage{subfig}
\usepackage{float}
\usepackage{subfiles}
\iclrfinalcopy
\title{Data-Driven Shadowgraph Simulation of a 3D Object}
\author{Anna Willmann$^1$, Patrick Stiller$^1$, Alexander Debus$^1$, Arie Irman$^1$, \\ \textbf{Richard Pausch}$^1$,
\textbf{Yen-Yu Chang}$^1$, \textbf{Michael Bussmann}$^{1,2}$, \textbf{Nico Hoffmann}$^1$\\
\\
$^1$ Helmholtz-Zentrum Dresden - Rossendorf, Bautzner Landstrasse 400, 01328 Dresden, Germany\\
$^2$ CASUS - Center for Advanced Systems Understanding, Untermarkt 20, 02826 Görlitz, Germany
}

\usepackage{scalerel,stackengine}
\stackMath
\newcommand\reallywidehat[1]{%
\savestack{\tmpbox}{\stretchto{%
  \scaleto{%
    \scalerel*[\widthof{\ensuremath{#1}}]{\kern-.6pt\bigwedge\kern-.6pt}%
    {\rule[-\textheight/2]{1ex}{\textheight}}%WIDTH-LIMITED BIG WEDGE
  }{\textheight}% 
}{0.5ex}}%
\stackon[1pt]{#1}{\tmpbox}%
}

\begin{document}

\maketitle

\begin{abstract}
In this work we propose a deep neural network based surrogate model for a plasma shadowgraph - a technique for visualization of perturbations in a transparent medium.
We are substituting the numerical code by a computationally cheaper projection based surrogate model that is able to approximate the electric fields at a given time without computing all preceding electric fields as required by numerical methods. This means that the projection based surrogate model allows to recover the solution of the governing 3D partial differential equation, 3D wave equation, at any point of a given compute domain and configuration without the need to run a full simulation.
This model has shown a good quality of reconstruction in a problem of interpolation of data within a narrow range of simulation parameters and can be used for input data of large size.
\end{abstract}

\section{Introduction}
Simulations of physical processes are required in theory development and give better understanding of complex dynamics that are involved into phenomena. The more complicate system is considered the more compounded is the model of it and higher requirements to the computational power appear. For simplification of such models there are used surrogate models - models, that are created to study only certain aspects of processes that an original model represents precisely but in the same time other dynamics are excluded or have not expected behavior. Surrogate models can reduce time consumption of a research with an acceptable loss in accuracy of results.

Plasma shadowgraph is one of diagnostic techniques that provides a visualization of perturbations in a transparent medium as such phenomena are not visible by human eyes. This technique is based on refraction of probe rays when they are distributing through a medium, focused and specifically filtered in order to represent differently refracted rays by light and dark zones\citep{shadowg}.

Application of a plasma shadowgraph to some phenomena in plasma such as for example laser wakefield acceleration\citep{wakefield} can be not trivial due to intense interaction between particles inside plasma and the micron scale of fluctuations. For a correct analysis of the experimental data we need high quality simulations of these processes.

Simulation of a plasma shadowgraph consists of two steps.
At first we need to approximate the solution of Maxwell's equations and then we can calculate propagation of light in free space from Fourier optics. The model that is proposed in this paper is supposed to approximate the solution on the first step of the shadowgraph simulation, for simplification of the problem, we consider only the electric field component.

The main contribution of this work is a data-driven reduced-order model for approximation of the numerical simulation of large 3D computational domains. The neural network is approximating the solution for a simplified version of Maxwell's equations in a context of the electric field propagation problem, that can be described by the following equation:

\begin{equation*}\label{eq:ewave}
    \displaystyle \frac{\partial^2\displaystyle \tE}{\partial t^2} - \displaystyle \frac{1}{\mu\epsilon}\displaystyle\nabla^2\displaystyle \tE = 0
\end{equation*}

where $\displaystyle \tE$ is the electric field, $\displaystyle \mu$ and $\displaystyle \epsilon$ are permeability and permittivity of a medium.

The model is capable of reconstructing new simulations of the electric field for different parameters within a range limited by parameter values of existed simulations. Finally, we will be analysing the applicability of our approach for interpolation as well as extrapolation in parameter space compared to ground-truth data. 

\section{Related works}
Artificial neural network based models are widely applied in the field of radiophysics and in particular for approximation of solution for Maxwell's equations. Thus, for example physics informed neural networks, introduced in \citep{pinn} were applied in \citep{pinn_me_mmd} for approximation of solution of the frequency domain Maxwell's equation in the context of metamaterial design and for approximation of time-domain electromagnetic simulations derived by Maxwell's equations in \citep{pinn_me_tdes}. Another example of a surrogate model for Maxwell's equations is presented in the paper \citep{gm_em}, as opposed to previous two works, there authors use not only fully connected architecture but also convolutional architecture to archive better quality solution approximation of Maxwell’s equations over an arbitrary dielectric permittivity. In these works models approximate a solution directly based on the input parameters. Such method is intuitive but with increasing of outputs' size one can encounter a problem of high resource consumption. Another kind of methods, projection based models that reduce dimensionality of the original model and approximate solution in a reduced space. In \citep{pbm_me_tema} authors proposed an autoencoder based architecture for approximation of Maxwell's equations solution. A convolutional autoencoder decreases dimensionality of input data and evolution in time is recurrently computed by a modified LSTM\citep{lstm} network on the reduced space. Another work that proposes an autoencoder based solver for Maxwell's equations is \citep{ann_wp} for the problem of wave propagation. 

The model proposed in our work solves the problem of approximation using architecture of an autoencoder in order to reduce number of operations at approximation itself and in the same time replaces recurrent computations of each next time point by a direct mapping of parameters to a solution approximation in a reduced space. 

\section{Method}
The model consists of two parts, the first one, autoencoder, reduces dimensionality of input data and transforms it back to the original size and the second one, projection approximator, approximates the solution in a reduced space. The structure of the autoencoder is adopted from \citep{deepfluid}, where it is used as a part of a reduced order model for fluid dynamics simulations. The encoder consists of convolutional downsamling layers, each is preceded by a block of convolutional layers and an additive skip connection between them, the decoder is structured in the same way but layers are arranged in a reverse order. In addition, the last downsampling layer is followed by one more block of convolutional layers and afterwards there is applied a linear layer to reduce vector size to the desirable size of the latent space. In this work we define number of downsampling layers by the formula proposed in \citep{deepfluid} depending on a size of input, number of preceded convolutional layers is set to 4. Each layer in the network is followed by the activation function LeakyReLU\citep{lrelu} with leak of 0.2. 

The projection approximator can be seen as multi-layer perceptron architecture consisting of 4 fully connected layers: the input layer of size $\displaystyle k+1$, 2 hidden layers and the output layer of latent size $\displaystyle l$, each hidden layer is followed by $\displaystyle sin(x)$ activation function that captures the functional relationship of adjacent latent codes. 

\textbf{\\Compression of 3D simulation data:} let us denote the encoder by $\displaystyle R:  \displaystyle \R^{\displaystyle h\times \displaystyle w\times \displaystyle d} \rightarrow \displaystyle \R^{\displaystyle l}$, where $\displaystyle h$, $\displaystyle w$, $\displaystyle d$ - height, width and depth of an input volume $\displaystyle \tE^{\left (\displaystyle t\right )}$ at time point $\displaystyle t$, $\displaystyle l$ is a size of its latent representation. 
Then the decoder is denoted by $\displaystyle G: \displaystyle \R^{\displaystyle l} \rightarrow  \displaystyle \R^{\displaystyle h\times \displaystyle w\times \displaystyle d}$.
This autoencoder is trained by minimizing the supervised reconstruction loss $\mathcal{L}_{R,G}$ of an original volume $\displaystyle \tE^{\left (\displaystyle t\right )}$ and a reconstructed one $\reallywidehat{\displaystyle \tE}^{\left (\displaystyle t\right )} = G(R(\displaystyle \tE^{\left (\displaystyle t\right )}))$:
\begin{equation*}
    \displaystyle \mathcal{L}_{R,G}(\displaystyle \tE^{\displaystyle \left (t\right )})=\displaystyle ||\displaystyle \tE^{\left (\displaystyle t\right )}-\reallywidehat{\displaystyle \tE}{}^{\left (\displaystyle t\right )}||_1
\end{equation*}

\textbf{\\Projection approximation in Latent Space:} the projection approximator is learning the mapping \mbox{$\displaystyle F:\displaystyle \R^{\displaystyle k+1} \rightarrow \displaystyle \R^{\displaystyle l}$}, where $\displaystyle k$ is a number of simulation parameters and one additional parameter is a time point at which a solution is to be approximated. The objective function of this network is the supervised approximation error,
\begin{equation*}
    \displaystyle \mathcal{L}_F(R(\displaystyle \tE^{\left (\displaystyle t\right )}), \displaystyle \vp, \displaystyle t) = \displaystyle ||R(\displaystyle \tE^{\left (\displaystyle t\right )})-F(\displaystyle \vp,\displaystyle t)||_2
\end{equation*}

where $\displaystyle \vp$ is a vector of simulation parameters and $\displaystyle t$ some point in time. Pretrained autoencoder and projection approximator allows us to reconstruct the solution of our system for different points as of $\displaystyle \tE_{predicted}^{\left (\displaystyle t\right )}= \displaystyle G(\displaystyle F(\displaystyle \vp, \displaystyle t))$. 

\section{Results}

For validation of the proposed model there are used simulations of beam propagation. 
Each simulation is an approximation of electric field propagation through a cell with a sphere in the middle that is defined by a radius $\displaystyle r$ and a refractive index $\displaystyle n$, calculated by finite-difference time-domain method\citep{ftdt} using a library Meep\citep{meep}. An example of such simulation is shown in figure \ref{fig:cubesim}. Surfaces of a cube represent middle slices of the cell from corresponding planes. 

\begin{wrapfigure}[15]{r}{5.5cm}
    \begin{center}
    \vspace*{-7mm}
    \includegraphics[scale=0.32]{figure1.png}
    \caption{Example of simulation}
    \label{fig:cubesim}
    \end{center}
\end{wrapfigure} 
The cell has physical size of 12$\displaystyle \mu m$ in each dimension with perfectly matched layer of 2$\displaystyle \mu m$ for the absorption at the boundary, proposed in \citep{pml}. Beam propagates in direction of axes Z, time points are given in units, one unit corresponds to $104.17 \mu s$. All parameters of simulations are given in supplementary material.

For training of the autoencoder such simulations were used with the following ranges of varying parameters: radius in $\displaystyle [2.0 \mu m, 4.0\mu m]$ with a step of 0.5$\displaystyle \mu m$ and refractive index in $\displaystyle [1.1, 1.7]$ with a step of 0.1. The same data in the reduced space is used for training of the projection approximator. Permutations of these values bring to 35 simulations that were used as training data for the autoencoder, one simulation approximates the electric field at each time point in range $\displaystyle [0.0, 10.0]$ with a step of 0.03125 time units, in total each simulation consists of 321 files where each file is a 3D array of size $193\times 193\times 193$ and takes 55Mb of memory and ca. 18Gb for a one full simulation.

\subsection{Training of model}
The networks were trained sequentially since the projection approximator requires a large number of epochs for convergence which would be computationally unfeasible in an end-to-end setting. Therefore, the autoencoder was trained first allowing us to precomputed the latent codes of all volumes of our training. The precomputed latent codes are then used for further training of the projection approximator. Parameters of networks were optimized by Adam optimizer\citep{adam} with learning rate of $0.0001$ over 160 iterations (autoencoder) and $0.001$ over 18400 iterations (projection approximator). Training of autoencoder was performed on 28 GPUs NVIDIA V100 for $29\;h$ and of projection approximator on 8 GPUs for $6\;h$.

\subsection{Analysis of latent code}

The analysis of temporal evolution of certain latent codes in the reduced space can give us better understanding and interpretation of the compressed representation.

\begin{wrapfigure}[9]{r}{5.5cm}
    \begin{center}
    \vspace*{-5mm}
    \includegraphics[scale=0.25]{figure2.png}
    \caption{Evolution of components in the original and reduced spaces}
    \label{fig:latorig}
    \end{center}
\end{wrapfigure}
Figure \ref{fig:latorig} illustrates two series: one series shows the time evolution of a single voxel of our original input volume while the other series shows the time evolution of a certain latent dimension of the very same dataset.

There we see that the wave pattern with its period is preserved over encoding, then we can conclude that the field is compressed with maintaining certain physical dependencies from the original volume while parameters of simulation correspond to the range of values in latent representation.

\subsection{Generation of simulations}

The interpolation of simulations between known parameter values was performed successfully for the described model. 

Figures \ref{fig:rec}, \ref{fig:rec2} show several examples of reconstruction. 
The model has shown an ability to reconstruct propagation of the field for all considered time points in the training set maintaining the structure of the field including refraction of the field in the location of a sphere as well as a distance on which the field is propagated until a certain point in time.

\begin{figure}[h!]
  \centering
  \includegraphics[width=0.8\textwidth]{figure3.png}
  \caption{The reconstructed volumes clearly resemble the field data of the 3D wave equation.}
  \label{fig:rec}
\end{figure}

Examples from figure \ref{fig:rec} relate to time points 3.0, 6.0 and 9.0 respectively for a simulation with the same radius and refractive index. Examples from figure \ref{fig:rec2} show that the model is able to recognize different input radii of a sphere as well as different refractive indices respectively. Table \ref{tab:err_inter} provides average reconstruction errors over each simulation. 

\begin{wraptable}[9]{l}{6.5cm}
\vspace*{-4mm}
\caption{Reconstruction error for interpolation}
\label{tab:err_inter}
\begin{center}
\begin{tabular}{ll}
\multicolumn{1}{c}{\bf Parameters $\displaystyle r$, $\displaystyle n$ }  &\multicolumn{1}{c}{\bf $\displaystyle L_1$ error}
\\ \hline \\
$2.25\mu m$, $1.65$ &0.01639\\
 $3.25\mu m$, $1.25$ &0.01595\\
 $3.25\mu m$, $1.65$  &0.01813 \\
 $3.75\mu m$, $1.65$ &0.01823 \\
\end{tabular}
\end{center}
\end{wraptable}

The model can handle parameter values beyond the considered ranges (extrapolation) for training only within a small distance as new simulations contain features that did not appear in the training set. 

\begin{figure}[t]
  \centering
  \includegraphics[width=0.8\textwidth]{figure4.png}
  \caption{Examples of reconstruction}
  \label{fig:rec2}
\end{figure}

The time consumption for approximation of a single 3D volume is reduced by a factor of four compared to the computational time of conventional numerical method: the FTDT method implemented in Meep takes ca. $0.8\;s$ per time point while for the developed model it takes ca. $0.18\;s$. 
An even larger improvement in runtime can be gained as the projection approximator network not only allows us to recover certain subsequent time points but also enables us to fast-forward to any given time point of compute domain in $0.18\;s$.


\section{Conclusions}
The fast reconstruction of experimentally accessible diagnostics is a crucial task for understanding very complex systems such as Laser plasma accelerators. A general strategy for solving the involved ill-posed inverse problems requires the optimisation of numerical simulations which is computationally very demanding. We are tackling that issue by a projection-based surrogate model that successfully approximates the governing 3D field propagation with stable behavior for interpolation in parameter space. Interestingly, we found that the autoencoder part of our architecture preserves time-dependent physical properties while also encoding information about the parameters describing the system. The surrogate model promises significant acceleration compared to numerical methods by allowing direct access to the solutions of the governing equation at any point in time without the need of time-stepping schemes. Additionally, the surrogate model promises an speedup by factor of four comparing to the conventional FTDT method even for simple time-stepping. 

\newpage

%%%%%%%%%%%%%%%%%%%%%%%% referenc.tex %%%%%%%%%%%%%%%%%%%%%%%%%%%%%%
% sample references
% %
% Use this file as a template for your own input.
%
%%%%%%%%%%%%%%%%%%%%%%%% Springer-Verlag %%%%%%%%%%%%%%%%%%%%%%%%%%
%
% BibTeX users please use
% \bibliographystyle{}
% \bibliography{}
%

\biblstarthook{References may be \textit{cited} in the text either by number (preferred) or by author/year.\footnote{Make sure that all references from the list are cited in the text. Those not cited should be moved to a separate \textit{Further Reading} section or chapter.} The reference list should ideally be \textit{sorted} in alphabetical order -- even if reference numbers are used for the their citation in the text. If there are several works by the same author, the following order should be used: 
\begin{enumerate}
\item all works by the author alone, ordered chronologically by year of publication
\item all works by the author with a coauthor, ordered alphabetically by coauthor
\item all works by the author with several coauthors, ordered chronologically by year of publication.
\end{enumerate}
The \textit{styling} of references\footnote{Always use the standard abbreviation of a journal's name according to the ISSN \textit{List of Title Word Abbreviations}, see \url{http://www.issn.org/en/node/344}} depends on the subject of your book:
\begin{itemize}
\item The \textit{two} recommended styles for references in books on \textit{mathematical, physical, statistical and computer sciences} are depicted in ~\cite{science-contrib, science-online, science-mono, science-journal, science-DOI} and ~\cite{phys-online, phys-mono, phys-journal, phys-DOI, phys-contrib}.
\item Examples of the most commonly used reference style in books on \textit{Psychology, Social Sciences} are~\cite{psysoc-mono, psysoc-online,psysoc-journal, psysoc-contrib, psysoc-DOI}.
\item Examples for references in books on \textit{Humanities, Linguistics, Philosophy} are~\cite{humlinphil-journal, humlinphil-contrib, humlinphil-mono, humlinphil-online, humlinphil-DOI}.
\item Examples of the basic Springer style used in publications on a wide range of subjects such as \textit{Computer Science, Economics, Engineering, Geosciences, Life Sciences, Medicine, Biomedicine} are ~\cite{basic-contrib, basic-online, basic-journal, basic-DOI, basic-mono}. 
\end{itemize}
}

\begin{thebibliography}{99.}%
% and use \bibitem to create references.
%
% Use the following syntax and markup for your references if 
% the subject of your book is from the field 
% "Mathematics, Physics, Statistics, Computer Science"
%
% Contribution 
\bibitem{science-contrib} Broy, M.: Software engineering --- from auxiliary to key technologies. In: Broy, M., Dener, E. (eds.) Software Pioneers, pp. 10-13. Springer, Heidelberg (2002)
%
% Online Document
\bibitem{science-online} Dod, J.: Effective substances. In: The Dictionary of Substances and Their Effects. Royal Society of Chemistry (1999) Available via DIALOG. \\
\url{http://www.rsc.org/dose/title of subordinate document. Cited 15 Jan 1999}
%
% Monograph
\bibitem{science-mono} Geddes, K.O., Czapor, S.R., Labahn, G.: Algorithms for Computer Algebra. Kluwer, Boston (1992) 
%
% Journal article
\bibitem{science-journal} Hamburger, C.: Quasimonotonicity, regularity and duality for nonlinear systems of partial differential equations. Ann. Mat. Pura. Appl. \textbf{169}, 321--354 (1995)
%
% Journal article by DOI
\bibitem{science-DOI} Slifka, M.K., Whitton, J.L.: Clinical implications of dysregulated cytokine production. J. Mol. Med. (2000) doi: 10.1007/s001090000086 
%
\bigskip

% Use the following (APS) syntax and markup for your references if 
% the subject of your book is from the field 
% "Mathematics, Physics, Statistics, Computer Science"
%
% Online Document
\bibitem{phys-online} J. Dod, in \textit{The Dictionary of Substances and Their Effects}, Royal Society of Chemistry. (Available via DIALOG, 1999), 
\url{http://www.rsc.org/dose/title of subordinate document. Cited 15 Jan 1999}
%
% Monograph
\bibitem{phys-mono} H. Ibach, H. L\"uth, \textit{Solid-State Physics}, 2nd edn. (Springer, New York, 1996), pp. 45-56 
%
% Journal article
\bibitem{phys-journal} S. Preuss, A. Demchuk Jr., M. Stuke, Appl. Phys. A \textbf{61}
%
% Journal article by DOI
\bibitem{phys-DOI} M.K. Slifka, J.L. Whitton, J. Mol. Med., doi: 10.1007/s001090000086
%
% Contribution 
\bibitem{phys-contrib} S.E. Smith, in \textit{Neuromuscular Junction}, ed. by E. Zaimis. Handbook of Experimental Pharmacology, vol 42 (Springer, Heidelberg, 1976), p. 593
%
\bigskip
%
% Use the following syntax and markup for your references if 
% the subject of your book is from the field 
% "Psychology, Social Sciences"
%
%
% Monograph
\bibitem{psysoc-mono} Calfee, R.~C., \& Valencia, R.~R. (1991). \textit{APA guide to preparing manuscripts for journal publication.} Washington, DC: American Psychological Association.
%
% Online Document
\bibitem{psysoc-online} Dod, J. (1999). Effective substances. In: The dictionary of substances and their effects. Royal Society of Chemistry. Available via DIALOG. \\
\url{http://www.rsc.org/dose/Effective substances.} Cited 15 Jan 1999.
%
% Journal article
\bibitem{psysoc-journal} Harris, M., Karper, E., Stacks, G., Hoffman, D., DeNiro, R., Cruz, P., et al. (2001). Writing labs and the Hollywood connection. \textit{J Film} Writing, 44(3), 213--245.
%
% Contribution 
\bibitem{psysoc-contrib} O'Neil, J.~M., \& Egan, J. (1992). Men's and women's gender role journeys: Metaphor for healing, transition, and transformation. In B.~R. Wainrig (Ed.), \textit{Gender issues across the life cycle} (pp. 107--123). New York: Springer.
%
% Journal article by DOI
\bibitem{psysoc-DOI}Kreger, M., Brindis, C.D., Manuel, D.M., Sassoubre, L. (2007). Lessons learned in systems change initiatives: benchmarks and indicators. \textit{American Journal of Community Psychology}, doi: 10.1007/s10464-007-9108-14.
%
%
% Use the following syntax and markup for your references if 
% the subject of your book is from the field 
% "Humanities, Linguistics, Philosophy"
%
\bigskip
%
% Journal article
\bibitem{humlinphil-journal} Alber John, Daniel C. O'Connell, and Sabine Kowal. 2002. Personal perspective in TV interviews. \textit{Pragmatics} 12:257--271
%
% Contribution 
\bibitem{humlinphil-contrib} Cameron, Deborah. 1997. Theoretical debates in feminist linguistics: Questions of sex and gender. In \textit{Gender and discourse}, ed. Ruth Wodak, 99--119. London: Sage Publications.

\end{thebibliography}

\end{document}
