\documentclass[main.tex]{subfiles}
\usepackage{iclr2021_conference,times}
\usepackage{url}
\usepackage{graphicx}
\usepackage{wrapfig}
\usepackage{subfig}
\usepackage{float}
\usepackage[english]{babel}
\usepackage{amsmath,amsfonts,bm}


\begin{document}
\section*{Supplementary material}

\subsection*{Parameters of simulations used for training data}\label{sec:params}

The cell has physical size of $12 \mu m$ in each dimension and resolution of simulation is $16 \text{ pixels per } \mu m$. 
Boundary condition is a perfectly matched layer for the absorption with width of $2 \mu m$ on each side of the cell. 
The source of the current starts to distribute the field in direction of axes Z from the starting point with coordinates $(-4,0,0)$ w.r.t. the center of the cell(coordinates of center are $(0,0,0)$), wavelength of the source is $1.0 \mu m$, size of the source is $(0, 8, 8)$, i.e. a flat source in space between absorbing layers. 
Center of a sphere is located in the middle of the cell - $(0,0,0)$, with defined radius and refractive index of material. Time points are given in units, one unit corresponds to $104.17 \mu s$.

\subsection*{Additional plots to latent space analysis}

\begin{figure}[h!]
    \begin{center}
    \includegraphics[scale=0.4]{figure_supplement_1.png}
    \caption{Evolution of a single component from simulations with different parameters in the reduced space}
    \label{fig:latan}
    \end{center}
\end{figure}

\subsection*{Additional results}
\begin{figure}[H]
    \begin{center}
    \includegraphics[width=0.65\textwidth]{figure_supplement_2.png}
    \caption{Reconstruction of simulation: $ r=3.25\mu m$, $ n = 1.65$}
    \label{fig:evol}
    \end{center}
\end{figure}

\begin{figure}[H]
    \begin{center}
    \includegraphics[width=0.8\textwidth]{figure_supplement_3.png}
    \caption{Examples of reconstruction: interpolation}
    \label{fig:interp}
    \end{center}
\end{figure}
\vspace*{-0.5cm}
\begin{figure}[H]
    \begin{center}
    \includegraphics[width=0.65\textwidth]{figure_supplement_4.png}
    \caption{Examples of reconstruction: extrapolation of radius}
    \label{fig:extrp_rad}
    \end{center}
\end{figure}

\begin{figure}[H]
    \begin{center}
    \includegraphics[width=0.65\textwidth]{figure_supplement_5.png}
    \caption{Examples of reconstruction: extrapolation of refractive index}
    \label{fig:extrp_ri}
    \end{center}
\end{figure}

\begin{table}[H]
\caption{Reconstruction error for extrapolation}
\label{tab:extr}
\begin{center}
\begin{tabular}{ll}
\multicolumn{1}{c}{\bf Parameters: $ r$, $ n$ }  &\multicolumn{1}{c}{\bf ($ L_1$) error}
\\ \hline \\
$4.25\mu m$, $1.65$ &0.01639\\
 $4.50\mu m$, $1.65$ &0.01826\\
 $4.75\mu m$, $1.65$  &0.01813 \\
 & \\
 $3.25\mu m$, $1.75$ &0.01877 \\
 $3.25\mu m$, $1.90$ &0.02161\\
 $3.25\mu m$, $2.00$  &0.02421\\
\end{tabular}
\end{center}
\end{table}

\end{document}