\subsection{Pascal VOC}
We train single-task baselines with the limitations imposed by using MTL and compare those with the multitask model and reference models from the literature. Table \ref{multi-res} shows the results. Some example output images can be seen in Fig.~\ref{pascalsample}. We add color distortion for data augmentation.

\begin{table}[t]
\centering
\caption{Pascal VOC results.}
\label{multi-res}
\begin{tabular}{r|ccc|cccc}
                   & \multicolumn{3}{c}{References} \vline & \multicolumn{4}{c}{Ours} \\
                   & \cite{Laurmaa}  & \cite{SSD} & \cite{pynetbuilder} & SSD & FCN & Multi & Color \\ \hline
mIoU (\%)   &   -    &  -     &  62.5   &   -  & \textbf{56.4}  &  54.4  & 55.2 \\ 
mAP (\%)  &    74.3   & 70.4   &  - &  51.3  & - & 51.8 & \textbf{52.6}
\end{tabular}
\end{table}

\begin{figure}[t]
  \centering
  \includegraphics[width=.45\linewidth]{2008_008745.png}\hfill
  \includegraphics[width=.45\linewidth]{2011_000496.png}
  \vspace{-3mm}
  \caption{Pascal VOC detection and segmentation samples.}
  \label{pascalsample}
\end{figure}

\subsection{Aerial view}
We train single-task baselines with no limitations and compare them to our multitask model. Table \ref{aerialviewresults} shows the results in terms of accuracy and resources used when deployed on an NVIDIA GTX Titan X for the single-task models, their combination and our multitask model.  Sample output images can be seen in Fig.~\ref{aerialsample}.

\subsection{Analysis}
The compromises made due to the particularities of MTL and especially the lack of a strong data augmentation caused the final accuracy of our multitask model to lag behind that of the single-task ones trained without these although they improved in terms of speed and usage of resources, being 1.6x faster, lighter and consuming less memory than the naive solution. Compared to the single-task models trained with the same limitations, the multitask models matched or outperformed their accuracy for one of the tasks. Find a more detailed analysis in \cite{thesismiquel}.

\begin{table}[t]
\centering
\caption{Aerial view results.}
\vspace{-1mm}
\label{aerialviewresults}
\begin{tabular}{r|ccc|cc}
                   & Base & FCN & SSD & Multitask & Naive \\ \hline
mIoU (\%)   & - & 70.9    &  -        &   65.3 & \textbf{70.9}  \\ 
mAP (\%)  & -  & -   &  54.3  &  28.2 & \textbf{54.3} \\ \hline
Inf. Time (ms) & 19 & 24 & 27 & \textbf{30} & 49 \\
Memory (MB) & 1203 & 1233 & 1525 & \textbf{1552} & 2511 \\
Size (MB) & 95 & 95 & 140 & \textbf{140} & 235
\end{tabular}
\vspace{-1mm}
\end{table}

\begin{figure}[t]
  \centering
  \includegraphics[width=.45\linewidth]{deathCircle_video3_2799.png}\hfill
  \includegraphics[width=.45\linewidth]{IMG_87911.png}
  \vspace{-3mm}
  \caption{Aerial view detection and segmentation samples.}
  \label{aerialsample}
  \vspace{-1mm}
\end{figure}