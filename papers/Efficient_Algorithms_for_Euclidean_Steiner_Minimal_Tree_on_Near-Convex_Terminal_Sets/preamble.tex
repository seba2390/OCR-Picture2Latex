%generic%%%
\usepackage{amsfonts,amsthm,amsmath}
\usepackage{amstext}
\usepackage{wasysym}
\usepackage{bm}

%%%pagewise footnote numbering%%%
\makeatletter
\@addtoreset{footnote}{section}
\makeatother
%%%%%%%%%%%

\usepackage{tikz}
\usepackage{thm-restate}

\usetikzlibrary{shapes.geometric,matrix,shapes,arrows,positioning,chains}
\usepackage[normalem]{ulem}
\newcommand{\longversion}[1]{#1}
\newcommand{\shortversion}[1]{}
\newcommand{\bigoh}{{\mathcal{O}}}
\usepackage{todonotes}
\usepackage{gensymb}
\usepackage{graphicx,tikz}
\RequirePackage{fancyhdr}
\usepackage{xcolor}
\usepackage{boxedminipage}
\usepackage{float}
\usepackage[ruled,vlined,linesnumbered]{algorithm2e}
\usepackage{stmaryrd,bbm}
\usepackage{environ}

\newcommand{\myparagraph}[1]{\smallskip\noindent{\textbf{\sffamily #1} \ }}
\newcommand{\mypara}[1]{\smallskip\noindent{\textbf{\sffamily #1} \ }}

%%complexity classes%%%
\newcommand{\NP}{\textsf{NP}}
\newcommand{\NPp}{${\sf NP}$}
\newcommand{\classP}{\textsf{P}}
\newcommand{\APX}{\textsf{APX}}
\newcommand{\FPTAS}{\textsf{FPTAS}}
\newcommand{\FPT}{{\rm \textsf{FPT}}}
\newcommand{\coNP}{\text{\normalfont co-NP}}
\newcommand{\XP}{\text{\normalfont XP}}
\newcommand{\W}[1][xxxx]{\text{\normalfont W[#1]}}
\newcommand{\WOH}{\textsf{W[1]}-hard}
\newcommand{\WTH}{\textsf{W[2]}-hard}
\newcommand{\WTC}{\textsf{W[2]}-complete}
\newcommand{\NPH}{\textsf{NP}-hard}
\newcommand{\NPC}{\textsf{NP}-complete}
%%%%

%%problem definition box

\newcommand{\defparproblem}[4]{
  \vspace{1mm}
\noindent\fbox{
  \begin{minipage}{0.96\textwidth}
  \begin{tabular*}{\textwidth}{@{\extracolsep{\fill}}lr} #1  & {\bf{Parameter:}} #3 \\ \end{tabular*}
  {\bf{Input:}} #2  \\
  {\bf{Question:}} #4
  \end{minipage}
  }
  \vspace{1mm}
}

\newcommand{\defproblem}[3]{
  \vspace{1mm}
\noindent\fbox{
  \begin{minipage}{0.96\textwidth}
  \begin{tabular*}{\textwidth}{@{\extracolsep{\fill}}lr} #1 \\ \end{tabular*}
  {\bf{Input:}} #2  \\
  {\bf{Question:}} #3
  \end{minipage}
  }
  \vspace{1mm}
}

%%

%%useful math macros%
\newcommand{\C}[1]{\mathcal{#1}}
\newcommand{\SC}[1]{\mathscr{#1}}
\newcommand{\BB}[1]{\mathbb{#1}}
\newcommand{\what}{\widehat}
\newcommand{\wtilde}{\widetilde}
%%%%%%%%%%%%%%%%%%%%%

%%%generic fpt paper macros%

%\newtheorem*{theorem*}{Theorem}
%\newtheorem{theorem}{Theorem}
%\newtheorem{lemma}{Lemma}[section]
%\newtheorem{claim}{Claim}[section]
%\newtheorem{corollary}{Corollary}
%\newtheorem{definition}{Definition}[section]
%\newtheorem{observation}{Observation}[section]
%\newtheorem{proposition}{Proposition}[section]
\newcounter{reduction}
\setcounter{reduction}{0}

\newcounter{markingrule}
\setcounter{markingrule}{0}

\newtheorem{mschm}[markingrule]{Marking Scheme}
\newtheorem{redr}[reduction]{Reduction Rule}
\newtheorem{marule}[markingrule]{Marking Rule}

\newcounter{hypothesis}
\setcounter{hypothesis}{0}

\newtheorem{hypo}[hypothesis]{Conjecture}

\newcommand{\Yes}{\textsf{yes}}
\newcommand{\No}{\textsf{no}}

%%%%%%%%%%%%%%%%%%%%%%%%%%%%

% Paper specific macros%

\newcommand{\BTJ}{\textsc{Bipartite Token Jumping}}
\newcommand{\Biclique}{\textsc{Bi-clique}}
\newcommand{\MBB}{\textsc{Maximum Balanced Biclique}}
\newcommand{\optbiclq}{{\sf opt}\mbox{-}{\sf biclq}}
\newcommand{\TSAT}{\textsc{$3$-SAT}}
\newcommand{\TD}{\textsc{Skew Detection}} 
\newcommand{\BTS}{\textsc{Bipartite Skew Subgraph}}
\newcommand{\BTIS}{\textsc{Bipartite Skew Induced-Subgraph}}
\newcommand{\BTCS}{\textsc{Bipartite Skew (Induced-)Subgraph}}
\newcommand{\TJ}{\textsc{Token Jumping}} 
%%%%%%%%%%%%%%%%%%%%%%%%
