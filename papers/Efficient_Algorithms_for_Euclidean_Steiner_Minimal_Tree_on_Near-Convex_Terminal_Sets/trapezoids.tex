
\subsection{Isosceles Trapezoids and Vertical Forks} \label{trapezoids}


In this section, we discuss one particular Steiner topology when the terminal set is  formed by the four corners of a given isosceles trapezoid. However, we will limit the discussion to only the isosceles trapezoids such that the angle between the non-parallel sides is of the form $\frac{2 \pi}n$ where $n \in \mathbb{N}$, $n \ge 4$. The reason is that given $2$-CPR $n$-gons $\{A_i\}$, $\{B_i\}$, for $n \geq 4$ and for any $j \in \{1,\ldots, n-1\}$, the region $A_jA_{j + 1}B_jB_{j + 1}$ is an isosceles trapezoid such that the angle between the non-parallel sides is of the form $\frac{2 \pi}n$.

\begin{figure}[h]
\centering
\includegraphics[width=5.5cm]{trapezoid.png}
\caption{Isosceles trapezoid with $\angle AOB = \frac {2 \pi} 8$}
\label{trap_def}
\end{figure}

Let $ABQP$ be an isosceles trapezoid with $AB$, $PQ$ as the parallel sides, and $AP$, $BQ$ as the non-parallel sides. Assume without loss of generality that $AB$ is shorter in length than $PQ$. Let $\overline{AB} = 1$ and $\overline{PQ} = \lambda$, where $\lambda \geq \frac {\sqrt 3 + \tan{ \frac \pi n }} {\sqrt 3 - \tan{ \frac \pi n}} $. For brevity, we say $\lambda_v = \frac {\sqrt 3 + \tan{ \frac \pi n }} {\sqrt 3 - \tan{ \frac \pi n}}$. Let $L_{PA}$ and $L_{QB}$ be the lines containing the line segments $PA$ and $QB$ respectively. Also let $O$ be the point of intersection of $L_{PA}$ and $L_{QB}$. Further, let $M$ and $N$ be the midpoints of $AB$ and $PQ$ respectively (as in Figure \ref{trap_def}). As mentioned earlier, $\angle AOB = \frac{2 \pi}n$ where $n \in \mathbb{N}$, $n \ge 4$.



%We show that a particular Steiner topology of $\{A, B, P, Q\}$. We call it the \emph{vertical fork}.

% \vspace{0.2cm}
% \subsection{The MST (for $n = 6$)}
% \vspace{0.2cm}

% For $n = 6$, the minimum spanning tree (MST) $T_{sp} = \{PA, AB, BQ\}$ is a possible Steiner tree ($\forall$ $\lambda > 1)$.

% \begin{figure}[h!]
% \centering
% \includegraphics[width=7cm]{hexagon_trapezoid_sp.png}
% \caption{Minimum Spanning Tree $T_{sp} = \{PA, AB, BQ\}$}
% \end{figure}

% We observe that for $n > 6, \angle PAB = \angle QBA < 120 ^ \circ$. Further, for $n = 6$, we get:
% $$\color{red}|T_{sp}| = \overline{PA} + \overline{AB} + \overline{QB} = 2 \lambda - 1$$

% \vspace{0.2cm}
% \subsection{The Horizontal Fork (for $n > 6$)}
% \vspace{0.2cm}

% For $n > 6$, the minimum spanning tree (MST) no longer remains a valid Steiner tree. We explore other topology, where $P$ \& $A$ are connected to a Steiner Point $S_1$ and $B$ \& $Q$ are connected to another Steiner Point $S_2$ such that $S_1$ and $S_2$ are directly connected. We call the Steiner tree of such a topology, the \emph{Horizontal Fork}. 

% \begin{figure}[h!]
% \centering
% \includegraphics[width=7cm]{horizontal_fork.png}
% \caption{The Horizontal Fork, $T_{hf}$}
% \end{figure}

% Therefore, we have the horizontal fork as $T_{hf} = \{PS_1, AS_1, S_1S_2, S_2B, S_2Q\}$

% \subsubsection{Construction}

% \begin{itemize}
%     \item We construct equilateral triangles $ABE$ and $PQF$ towards the interior of the trapezoid $ABPQ$.
%     \item $S_1 := AE \cap PF$, $S_2 := BE \cap QF$ (These intersections occurs only if $n > 6$)
% \end{itemize}

% \subsubsection{Analysis}

% We observe that the Steiner points $S_1$, $S_2$ are obtainable only if $n > 6$. Further, for the triangles $ABE$ and $PQF$ to intersect, the following condition must hold: \\


% $ \overline{ME} + \overline{FN} \ge \overline{MN} $\\

% $ \implies \dfrac{\sqrt{3}}2 \cdot \overline {PQ} + \dfrac{\sqrt{3}}2 \cdot \overline {PQ} \ge \overline{ON} - \overline{OM}$\\

% $ \implies \dfrac{\sqrt{3}}2 \cdot (\lambda + 1) \ge \dfrac \lambda {2 \tan(\frac \pi n)} - \dfrac 1 {2 \tan(\frac \pi n)}$\\

% $ \implies \lambda \le \dfrac {1 + \sqrt 3 \tan{ \frac \pi n }} {1 - \sqrt 3 \tan{ \frac \pi n}} = \lambda_h \text{ (Say)}$\\


% Now, $\overline{AB} = \overline{AE} = \overline{BE} = 1$ and $\overline{PQ} = \overline{QF} = \overline{FP} = \lambda$. Further we observe that $ES_1S_2$ and $FS_1S_2$ are equilateral triangles. Let $\overline{ES_1} = \overline{S_1F} = \overline{FS_2} = \overline{S_2E} = \overline{S_1S_2} = \delta$. Now,\\

% $ \overline{EF} + \overline{MN} = \overline{NF} + \overline{ME}$\\

% $ \implies \sqrt 3 \cdot \delta + \dfrac{\lambda - 1}{2 \tan \frac \pi n} = \dfrac{\sqrt 3}2 (\lambda + 1)$\\

% $ \implies -3 \cdot \delta = \dfrac{\sqrt{3} (\lambda - 1)}{2 \tan \frac \pi n} - \dfrac{3}2 (\lambda + 1)$

% \noindent Again, \\


% \noindent $|T_{hf}| = \overline{PS_1} + \overline{QS_2} + \overline{S_1S_2} + \overline{S_1A} + \overline{S_2B}$\\

% $ = (\lambda - \delta) + (\lambda - \delta) + \delta + (1 - \delta) + (1 - \delta) = 2 (\lambda + 1) - 3 \cdot \delta  = 2 (\lambda + 1) + \dfrac{\sqrt{3} (\lambda - 1)}{2 \tan \frac \pi n} - \dfrac{3}2 (\lambda + 1)$\\


% \noindent Therefore we get : 
% $$\color{blue}|T_{hf}| = \dfrac{\lambda + 1}2 + \dfrac{\sqrt{3} (\lambda - 1)}{2 \tan \frac \pi n}$$

% \noindent for any $n > 6$ and $\lambda \le \lambda_h$

%\vspace{0.2cm}
%\subsection{The Vertical Fork}
%\vspace{0.2cm}

Now, we explore the following Steiner topology of the terminal set $\{A, B, P, Q\}$:
\begin{enumerate}
\item $A$ and $B$ are connected to a Steiner point $S_1$.
\item $P$ and $Q$ are connected to another Steiner point $S_2$. 
\item $S_1$ and $S_2$ are directly connected (Please see Figure \ref{T_vf}). 
\end{enumerate}
We call such a topology a \emph{vertical fork topology} and the Steiner tree realising such a topology, the \emph{vertical fork}. Note that in a vertical fork topology the only unknowns are the locations of the two Steiner points $S_1,S_2$. Therefore, we have the vertical fork topology as $T_{vf}$, with $E(T_{vf}) = \{AS_1, BS_1, S_1S_2, S_2P, S_2Q\}$.


\begin{figure}[h]
\centering
\includegraphics[width=5.5cm]{vertical_fork.png}
\caption{The Vertical Fork, $\mathcal T_{vf}$}
\label{T_vf}
\end{figure}

%\subsubsection{Construction}


%\subsubsection{Analysis}

We show the existence of a vertical fork and calculate its total length in the following lemma.

\begin{lemma} \label{lambda_v}
A vertical fork $\mathcal T_{vf}$ can be constructed for any $n \ge 4$ and for any $\lambda \ge \lambda_v$, where

$$\lambda_v = \frac {\sqrt 3 + \tan{ \frac \pi n }} {\sqrt 3 - \tan{ \frac \pi n}}$$    
such that the length of the vertical fork   
$$|\mathcal T_{vf}| = \dfrac{(\lambda - 1)}{2 \tan \frac \pi n} + \dfrac {\sqrt 3 (\lambda + 1)} {2}$$

\end{lemma}

\begin{proof}
First, we construct the Steiner points $S_1$, $S_2$ and then prove that the construction works.

In the following construction, we describe how to find the locations of $S_1$ and $S_2$ for  the vertical fork:
\begin{itemize}
    \item We construct equilateral triangles $ABE$ and $PQF$ where both points $E$ and $F$ lie outside the trapezoid $ABQP$.
    \item We construct the circumcircles $(ABE)$ and $(PQF)$ of $ABE$ and $PQF$, respectively. 
    \item Recall that $L_{MN}$ is the line segment containing $M$ and $N$. Define $S_1$ to be the point of intersection of $L_{MN}$ and the circle $(ABE)$ distinct from $E$; similarly, $S_2$ is the point of intersection of the $L_{MN}$ and $(PQF)$ distinct from $F$. Therefore, by construction, $S_2$, $M$ must lie on the same side of $N$ on $L_{MN}$, and $S_1$, $N$ must lie on the same side of $M$ on $L_{MN}$. Further, $\angle AS_1B = \angle PS_2Q = \frac{2 \pi}{3}$ by construction.
\end{itemize}

We now show that the points $S_1$ and $S_2$ indeed lie inside the line segment $MN$ and the points appear in the order: $M$, $S_1$, $S_2$, $N$. We prove the following claim to serve this purpose.

\begin{claim} \label{S1_S2_in_MN}
$\overline{S_1M} + \overline{S_2N} \le \overline{MN}$
% \begin{enumerate}
%     \item  (As $S_2$, $M$ lie on the same side of $N$ on $L_{MN}$ and $S_1$, $N$ lie on the same side of $M$ on $L_{MN}$, this implies that $S_1$, $S_2$ lie on the line segment $MN$).
% \end{enumerate}

\end{claim}
\begin{proof}
We have $\overline{MN} = \overline {ON} - \overline{OM} = \dfrac{\lambda}{2 \tan \frac{\pi}{n}} - \dfrac{1}{2 \tan \frac{\pi}{n}} = \dfrac{(\lambda - 1)}{2 \tan \frac \pi n}$\\
Again $\overline{S_1M} = \dfrac{1}{2\sqrt{3}}$ and $\overline{S_2N} = \dfrac{\lambda}{2\sqrt{3}}$.\\
Therefore we have 
\begin{align*}
   & \overline{MN} - (\overline{S_1M} + \overline{S_2M}) \\
 = & \dfrac{(\lambda - 1)}{2 \tan \frac \pi n} - \dfrac 1 {\sqrt{3}} - \dfrac \lambda {2\sqrt 3}\\
 = & \dfrac{(\lambda - 1)}{2 \tan \frac \pi n} - \dfrac {(\lambda + 1)} {2\sqrt{3}}\\
 = & \frac{\lambda(\sqrt{3} - \tan \frac{\pi}{n}) - (\sqrt{3} + \tan \frac{\pi}{n})}{2 \sqrt{3} \tan \frac \pi n}\\
 = & \frac{\sqrt{3} - \tan \frac{\pi}{n}}{2 \sqrt{3} \tan \frac \pi n} \cdot (\lambda - \lambda_v)
\end{align*}

Therefore $\lambda \ge \lambda_v$ implies $\overline{MN} \ge (\overline{S_1M} + \overline{S_2M})$. This proves~\Cref{S1_S2_in_MN}.
\end{proof}

From~\Cref{S1_S2_in_MN}, we get $\overline{S_1M}, \overline{S_2N} \le \overline{MN}$. As $S_2$, $M$ lie on the same side of $N$ on $L_{MN}$ and $S_1$, $N$ lie on the same side of $M$ on $L_{MN}$, this implies that $S_1$, $S_2$ lie on the line segment $MN$. Further, $\overline{S_1M} + \overline{S_2N} \le \overline{MN}$ implies $\overline{S_1M} \le \overline {S_2M} \le \overline{NM}$, which in turn implies that the points appear in the order: $M$, $S_1$, $S_2$. $N$.\\

Now, we calculate the total length of the vertical fork, $|\mathcal T_vf|$:
\begin{align*}
  & |\mathcal T_{vf}|\\
= & \overline{AS_1} + \overline{BS_1} + \overline{S_1S_2} + \overline{PS_2} + \overline{QS_2}\\
= & 2 \overline{PS_2} + 2 \overline{AS_1} + \overline{S_1S_2}\\
= & \dfrac{2 \lambda} {\sqrt 3} + \dfrac 2 {\sqrt 3} + \bigg( \dfrac{(\lambda - 1)}{2 \tan \frac \pi n} - \dfrac {(\lambda + 1)} {2\sqrt{3}} \bigg)\\
= & \dfrac{(\lambda - 1)}{2 \tan \frac \pi n} + \dfrac {\sqrt 3 (\lambda + 1)} {2}
\end{align*}
% We observe that, 

% \noindent $\overline{S_1S_2} = \overline{MN} - \overline{MS_1} - \overline{NS_2}$\\

% $ = \dfrac{(\lambda - 1)}{2 \tan \frac \pi n} - \dfrac 1 {\sqrt{3}} - \dfrac \lambda {\sqrt 3}$\\

% $ = \dfrac{(\lambda - 1)}{2 \tan \frac \pi n} - \dfrac {(\lambda + 1)} {\sqrt{3}}$\\

%  Again,\\

% \noindent $|T_{vf}| = \overline{AS_1} + \overline{BS_1} + \overline{S_1S_2} + \overline{PS_2} + \overline{QS_2} = 2 \overline{PS_2} + 2 \overline{AS_1} + \overline{S_1S_2}$\\

% $ = \dfrac{2 \lambda} {\sqrt 3} + \dfrac 2 {\sqrt 3} + \bigg( \dfrac{(\lambda - 1)}{2 \tan \frac \pi n} - \dfrac {(\lambda + 1)} {\sqrt{3}} \bigg)$\\

% $ = \dfrac{(\lambda - 1)}{2 \tan \frac \pi n} + \dfrac {\sqrt 3 (\lambda + 1)} {\sqrt{2}}$\\

% Also, for the construction to be possible we must have: \\

% $ \overline{S_1S_2} \ge 0 $\\

% \noindent $\implies \bigg( \dfrac{(\lambda - 1)}{2 \tan \frac \pi n} - \dfrac {(\lambda + 1)} {\sqrt{3}} \bigg) \ge 0 $\\

% \noindent $\implies \lambda \ge \dfrac {\sqrt 3 + \tan{ \frac \pi n }} {\sqrt 3 - \tan{ \frac \pi n}} = \lambda_v$\\

% Thus, we are done.

This completes the proof of~\Cref{lambda_v}.
\end{proof}




% \vspace{0.2cm}
% \subsection{Comparing various Topologies}
% \vspace{0.2cm}

% \subsubsection{The horizontal fork \textit{vs} The vertical fork ($n > 6$)} \label{HFvsVF}

% Let for some $\lambda$, $\lambda_v \le \lambda \le \lambda_h$, the \emph{vertical fork} ``is better than" (has a smaller total length than) the \emph{horizontal fork}, for some $n > 6$.\\ 

% $ \Longleftrightarrow |T_{vf}| \le |T_{hf}|$\\

% $ \Longleftrightarrow \dfrac{(\lambda - 1)}{2 \tan \frac \pi n} + \dfrac {\sqrt 3 (\lambda + 1)} {2} \le \dfrac{\lambda + 1}2 + \dfrac{\sqrt{3} (\lambda - 1)}{2 \tan \frac \pi n}$\\

% $ \Longleftrightarrow \dfrac{(\lambda - 1)}{\tan \frac \pi n} \cdot \dfrac{\sqrt 3 - 1} 2 \ge (\lambda + 1) \cdot \dfrac{\sqrt 3 - 1} 2 $\\

% $ \Longleftrightarrow (\lambda - 1) \ge (\lambda + 1) \cdot \tan \frac \pi n$\\

% $ \Longleftrightarrow \lambda \ge \dfrac{(1 + \tan \frac \pi n)} {1 - \tan \frac \pi n} = \tan( \frac \pi 4 + \frac \pi n) = \lambda_s \text{ (Say)}$


% \subsubsection{The Vertical Fork \textit{vs} The MST ($n = 6$)} \label{VFvsMST}

% Let for some $\lambda$, $\lambda_v \le \lambda \le \lambda_h$, the \emph{vertical fork} ``is better than" (has a smaller total length than) the \emph{horizontal fork}, for $n = 6$.\\ 


% $ \Longleftrightarrow |T_{vf}| \le |T_{sp}|$\\

% $ \Longleftrightarrow \dfrac{(\lambda - 1)}{2 \tan \frac \pi n} + \dfrac {\sqrt 3 (\lambda + 1)} {2} \le 2 \lambda - 1$\\

% $ \Longleftrightarrow \dfrac {\sqrt 3 (\lambda - 1)} {2} + \dfrac {\sqrt 3 (\lambda + 1)} {2} \le 2 \lambda - 1 \text{ , [} \because n = 6 \text{]} $ \\

% $ \Longleftrightarrow \lambda \ge \dfrac 1 {2 - \sqrt{3}} = 2 + \sqrt 3 = \tan( \frac \pi 4 + \frac \pi 6) = \lambda_s$\\

% \begin{remark} The MST for $n = 6$ is the degenerate limit of of the \emph{horizontal fork} for $n \rightarrow 6$.
% \end{remark}

% Combining this with the result from \ref{HFvsVF}, we obtain that for all $n \ge 6$, the \emph{Vertical Fork} is better than the other construction (\emph{Horizontal Fork} for $n > 6$ and MST for $n = 6$) for $\lambda \ge \lambda_s$.

% \subsubsection{Combining the Results}

% First of all, for \ref{HFvsVF} and \ref{VFvsMST}, to be valid, we must have $\lambda_v \le \lambda_s \le \lambda_h$ for all $n > 6$, which is true because $f(x) = \dfrac {1 + x}{1 - x}$ is an increasing function in $(0, 1)$, where we have $\lambda_v = f \Big(\dfrac 1 {\sqrt 3} \cdot \dfrac \pi n \Big)$, $\lambda_s = f \Big(1 \cdot \dfrac \pi n \Big)$ and $\lambda_h = f \Big(\sqrt 3 \cdot \dfrac \pi n \Big)$.


% \begin{figure}[h!]
% \centering
% \includegraphics[width=13cm]{Trapezoid_data.png}
% \caption{ $\lambda_v$, $\lambda_s$ and $\lambda_h$ plotted against $n$ }
% \end{figure}
