\documentclass[a4paper,UKenglish,cleveref, autoref, thm-restate]{lipics-v2021}
%This is a template for producing LIPIcs articles. 
%See lipics-v2021-authors-guidelines.pdf for further information.
%for A4 paper format use option "a4paper", for US-letter use option "letterpaper"
%for british hyphenation rules use option "UKenglish", for american hyphenation rules use option "USenglish"
%for section-numbered lemmas etc., use "numberwithinsect"
%for enabling cleveref support, use "cleveref"
%for enabling autoref support, use "autoref"
%for anonymousing the authors (e.g. for double-blind review), add "anonymous"
%for enabling thm-restate support, use "thm-restate"
%for enabling a two-column layout for the author/affilation part (only applicable for > 6 authors), use "authorcolumns"
%for producing a PDF according the PDF/A standard, add "pdfa"

%\graphicspath{{./graphics/}}%helpful if your graphic files are in another directory
\usepackage{xspace}
\bibliographystyle{plainurl}% the mandatory bibstyle

\title{Efficient Algorithms for \ESMT on Near-Convex Terminal Sets} %TODO Please add

\titlerunning{ESMT on Near-Convex Terminal Sets} %TODO optional, please use if title is longer than one line

\author{Anubhav Dhar}{Indian Institute of Technology Kharagpur, India}{anubhavdhar@kgpian.iitkgp.ac.in}{}{}%TODO mandatory, please use full name; only 1 author per \author macro; first two parameters are mandatory, other parameters can be empty. Please provide at least the name of the affiliation and the country. The full address is optional

\author{Soumita Hait}{Indian Institute of Technology Kharagpur, India}{soumitahait7321@gmail.com}{}{}

\author{Sudeshna Kolay}{Indian Institute of Technology Kharagpur, India}{skolay@cse.iitkgp.ac.in}{}{}

\authorrunning{A.~Dhar, S.~Hait, S.~Kolay} %TODO mandatory. First: Use abbreviated first/middle names. Second (only in severe cases): Use first author plus 'et al.'

\Copyright{} %TODO mandatory, please use full first names. LIPIcs license is "CC-BY";  http://creativecommons.org/licenses/by/3.0/

\ccsdesc{Theory of computation~Computational geometry}%TODO mandatory: Please choose ACM 2012 classifications from https://dl.acm.org/ccs/ccs_flat.cfm 

\keywords{Steiner minimal tree, Euclidean Geometry, Almost Convex point sets, FPTAS, strong NP-completeness}

\category{} %optional, e.g. invited paper

\relatedversion{} %optional, e.g. full version hosted on arXiv, HAL, or other respository/website
%\relatedversiondetails[linktext={opt. text shown instead of the URL}, cite=DBLP:books/mk/GrayR93]{Classification (e.g. Full Version, Extended Version, Previous Version}{URL to related version} %linktext and cite are optional

%\supplement{}%optional, e.g. related research data, source code, ... hosted on a repository like zenodo, figshare, GitHub, ...
%\supplementdetails[linktext={opt. text shown instead of the URL}, cite=DBLP:books/mk/GrayR93, subcategory={Description, Subcategory}, swhid={Software Heritage Identifier}]{General Classification (e.g. Software, Dataset, Model, ...)}{URL to related version} %linktext, cite, and subcategory are optional

%\funding{(Optional) general funding statement \dots}%optional, to capture a funding statement, which applies to all authors. Please enter author specific funding statements as fifth argument of the \author macro.

%\acknowledgements{I want to thank \dots}%optional

\nolinenumbers %uncomment to disable line numbering

\hideLIPIcs  %uncomment to remove references to LIPIcs series (logo, DOI, ...), e.g. when preparing a pre-final version to be uploaded to arXiv or another public repository

%Editor-only macros:: begin (do not touch as author)%%%%%%%%%%%%%%%%%%%%%%%%%%%%%%%%%%
\EventEditors{Petr A. Golovach and Meirav Zehavi}
\EventNoEds{2}
\EventLongTitle{16th International Symposium on Parameterized and Exact Computation (IPEC 2021)}
\EventShortTitle{IPEC 2021}
\EventAcronym{IPEC}
\EventYear{2021}
\EventDate{September 8--10, 2021}
\EventLocation{Lisbon, Portugal}
\EventLogo{}
\SeriesVolume{214}
\ArticleNo{2}
%%%%%%%%%%%%%%%%%%%%%%%%%%%%%%%%%%%%%%%%%%%%%%%%%%%%%%

\newcommand{\defparproblem}[4]{
  \vspace{1mm}
\noindent\fbox{
  \begin{minipage}{0.96\textwidth}
  \begin{tabular*}{\textwidth}{@{\extracolsep{\fill}}lr} #1  & {\bf{Parameter:}} #3 \\ \end{tabular*}
  {\bf{Input:}} #2  \\
  {\bf{Question:}} #4
  \end{minipage}
  }
  \vspace{1mm}
}


\newcommand{\defproblem}[3]{
  \vspace{1mm}
\noindent\fbox{
  \begin{minipage}{0.96\textwidth}
  \begin{tabular*}{\textwidth}{@{\extracolsep{\fill}}lr} #1 \\ \end{tabular*}
  {\bf{Input:}} #2  \\
  {\bf{Question:}} #3
  \end{minipage}
  }
  \vspace{1mm}
}

\newenvironment{problem}[1]{%
\begin{center}\fbox{\parbox{14cm}{%
    {\centering\scshape #1\par}%
    \parskip=1ex
    \everypar{\hangindent=1em}%
    \BODY
}}\end{center}}

\setlength\parindent{24pt}
\newcommand{\B}[1]{\mathbb{#1}}
\newcommand{\C}[1]{\mathcal{#1}}
\newcommand{\F}[1]{\mathfrak{#1}}
\newcommand{\SC}[1]{\mathscr{#1}}
\newcommand{\what}{\widehat}
\newcommand{\wtilde}{\widetilde}
\newcommand{\lbgt}{{\sf lbgt}}
\newcommand{\bgt}{{\sf bgt}}
\newcommand{\ubgt}{{\sf ubgt}}
\newcommand{\sumubgt}{{\sf sumub}}
\newcommand{\sumlbgt}{{\sf sumlb}}
\usepackage{todonotes}
\usepackage{algorithm}
\usepackage{algpseudocode}

\newcommand{\WOH}{\textsf{W[1]}-hard}
\newcommand{\BTJ}{\textsc{Bipartite Token Jumping}}
\newcommand{\MBB}{\textsc{Maximum Balanced Biclique}}
\newcommand{\optbiclq}{{\sf opt}\mbox{-}{\sf biclq}}
\newcommand{\BTS}{\textsc{Bipartite Half Subgraph}}
\newcommand{\BTIS}{\textsc{Bipartite Half Induced-Subgraph}}
\newcommand{\BTCS}{\textsc{Bipartite Half (Induced-)Subgraph}}
\newcommand{\TJ}{\textsc{Token Jumping}} 
\newcommand{\ESTP}{\textsc{Euclidean Steiner Tree}\xspace}
\newcommand{\ESMT}{\textsc{Euclidean Steiner Minimal Tree}\xspace}
\newcommand{\pnp}{$\mbox{P} = \mbox{NP}$\xspace}

\newcommand{\mc}{\mathcal}
\newcommand{\Oh}{\mathcal{O}}
\newcommand{\OO}{\mathcal{O}}
\newcommand{\Ohstar}{\Oh^\star}
\newcommand{\smtpoly}{SMT for $\{A_i\} \cup \{B_i\}$ \xspace}

 \definecolor{babyblue}{rgb}{0.54, 0.81, 0.94}
 \definecolor{b1}{rgb}{0.63, 0.79, 0.95}
 \definecolor{b2}{rgb}{0.74, 0.83, 0.9}
 \definecolor{b3}{rgb}{0.67, 0.9, 0.93}
 \definecolor{gentlegreen}{rgb}{0.00, 0.51, 0.00}
 

\begin{document}

\maketitle




\begin{abstract}
\label{sec:abstract}

%% 1. what is the problem 
Scientific applications that run on leadership computing facilities often face the challenge 
of being unable to fit leading science cases onto accelerator devices due to memory constraints 
(memory-bound applications).
%
% 2. what is your solution 
In this work, the authors studied one such US Department of Energy mission-critical condensed matter 
physics application, Dynamical Cluster Approximation (DCA++), and this paper discusses how device memory-bound challenges were successfully reduced  by proposing an effective 
``all-to-all'' communication method---a ring communication algorithm. 
%
This implementation takes advantage of acceleration on GPUs and remote direct memory access (RDMA) for fast data exchange between GPUs. 
%
\\Additionally, the ring algorithm was optimized with sub-ring communicators
and multi-threaded support to further reduce communication overhead and 
expose more concurrency, respectively.
%
% 3. What's the cherry-picked evaluation result you want to mention
The computation and communication were also analyzed 
by using the Autonomic Performance Environment for Exascale 
(APEX) profiling tool,  and this paper further discusses the 
performance trade-off for the ring algorithm implementation. 
%
The memory analysis on the ring algorithm shows that the allocation size for the authors' most 
memory-intensive data structure per GPU is now reduced to $1/p$ of the original size, where $p$ is the number of GPUs in the ring communicator.
%
The communication analysis suggests that 
the distributed Quantum Monte Carlo execution time grows linearly as sub-ring size increases, and the cost of messages passing through the network interface connector could be a limiting factor.


%
% \todoRed{Ronnie: Next sentence needs rewrite, too much information about Green's function that no one knows in the abstract; recommend generalizing.} \emph {However, DCA++ is currently facing memory-bound challenge as 
% a larger device array $G_t$ is limited by device memory size, where
% $G_t$ is a two-particle Green's function that allows condensed matter
% scientists to explore larger and more complex (higher fidelity)
% physics cases.}

\end{abstract}

\keywords{DCA++, Quantum Monte Carlo, GPU Remote Direct Memory Access, memory-bound issue, exascale machines}

Reinforcement learning has achieved great success in areas such as Game-playing \citep{silver2018general,vinyals2019grandmaster}, robotics \cite{kober2013reinforcement}, large language models \citep{ouyang2022training}, etc.
However, due to safety concerns or physical limitations, in some real-world reinforcement learning problems, we must consider additional constraints that may influence the optimal policy and the learning process \citep{garcia2015comprehensive}.
% For example, a robotic arm must not take actions that may cause harm to itself or the environments.
A standard framework to handle such cases is the constrained Markov Decision Process (CMDP) \citep{altman1999constrained}.
Within the CMDP framework, the agent has to maximize
the expected cumulative reward while
obeying a finite number of constraints, which are usually in the form of expected cumulative cost criteria.

However, we are sometimes concerned with the problem with a continuum of constraints.
For example,
the constraints we meet might be time-evolving or subject to uncertain parameters, which
cannot be formulated as an ordinary CMDP
(see Examples \ref{Example_Time_Evolving} and  \ref{Example_Uncertain}).
In this paper we would study a generalized CMDP  
to address the above problem.  Because the constraints are not only infinite-number but also lie
in a continuous set,
the generalization is not trivial. Fortunately, we find that we can borrow the idea behind semi-infinite programming (SIP) \citep{remez1934determination, hettich1993semi} to deal with the semi-infinite constraints.
Accordingly, we propose \emph{semi-infinitely constrained Markov decision processes} (SICMDPs)
as a novel complement to the ordinary CMDP framework.
%More specifically,  an SICMDP model %, we consider 
%contains a continuum of constraints whereas an ordinary CMDP contains a finite number of constraints. 

%This generalization is natural but not trivial. However, we can brows the idea  
%The idea is quite natural and can be backtracked
%to the practice of extending linear programming to linear semi-infinite programming (LSIP) %\cite{remez1934determination, GobernaLSIO1998}.
%In addition, 
%As a complementary approach to the ordinary CMDP framework, 
%SICMDP can be used to model these problems  which cannot be described by a finite number of constraints
%that are not covered by .
%For example,
%the restrictions we consider can be time-evolving or subject to uncertain parameters
%, thus
%cannot be described by a finite number of constraints but a continuum of constraints 
%(see Examples \ref{Example_Time_Evolving} and  \ref{Example_Uncertain}).

We also present two reinforcement learning algorithms to solve SICMDPs called SI-CRL and SI-CPO, respectively.
SI-CRL is a model-based reinforcement learning algorithm designed for tabular cases, and SI-CPO is a policy optimization algorithm for non-tabular cases.
% and analyze its performance both theoretically and empirically.
The main challenge is that we need to deal with a continuum of constraints, thus reinforcement learning algorithms for ordinary CMDPs do not work anymore.
In SI-CRL, we tackle this difficulty by first transforming the reinforcement learning problem to an equivalent LSIP problem, which can then be solved using methods in the LSIP literature like the dual exchange methods \citep{Hu1990,reemtsen1998numerical}.
In SI-CPO, we resort to the idea of cooperative stochastic approximation developed in \cite{lan2020algorithms, wei2020comirror}.
As far as we know, we are the first to introduce tools from semi-infinitely programming (SIP) into the reinforcement learning community for solving constrained reinforcement learning problems.

% To the best of our knowledge, we are the first to apply tools from semi-infinitely programming (SIP) to solve reinforcement learning problems.
Furthermore, we give theoretical analysis for both SI-CRL and SI-CPO.
We decompose the error of SI-CRL into two parts: the statistical error from approximating the true SICMDP with an offline dataset and the optimization error due to the fact that the solution of the LSIP problem obtained by the dual exchange method is inexact.
On the optimization side, we show that the iteration complexity of SI-CRL is $O\left(\left\{\mathrm{diam}(Y)L\sqrt{|\gS|^2|\gA|m}/\left[(1-\gamma)\epsilon\right]\right\}^m\right)$.
On the statistical side, we show that the sample complexity of SI-CRL is $\widetilde O\left(\frac{|S|^2|A|^2}{\epsilon^2(1-\gamma)^3}\right)$ if the offline dataset is generated by a generative model, and $\widetilde O\left(\frac{|S||A|}{\nu_{\min} \epsilon^2(1-\gamma)^3}\right)$ if the dataset is generated by a probability measure $\nu$ as considered in \cite{chen2019information}.
Here $\widetilde O$ means that all logarithm terms are discarded.
For SI-CPO, things become a little more complicated because other than the statistical error and the optimization error, we also need to consider the function approximation error, which comes from imperfect policy parametrizations.
It is shown if the function approximation error can be controlled to $O(\epsilon)$ order, the iteration complexity of SI-CPO is $\widetilde{O}\left(\frac{1}{\epsilon^2(1-\gamma)^6}\right)$ and the sample complexity of SI-CPO is $\widetilde{O}(\frac{1}{\epsilon^4(1-\gamma)^{10}})$.
Here our iteration complexity bound is equivalent to a typical $\widetilde O(1/\sqrt{T})$ global convergence rate.

We perform a set of numerical experiments to illustrate the SICMDP model and validate our proposed algorithms.
Specifically, we examine two numerical examples, namely the discharge of sewage and ship route planning.
Through the discharge of sewage example, we show the advantage of the SICMDP framework over the CMDP baseline obtained by naive discretization in modeling realistic sequential decision-making problems.
Moreover, we demonstrate the effectiveness of the SI-CRL and SI-CPO algorithms in such tabular environments. 
In the ship route planning example, we illustrate the benefits of the SICMDP framework and the ability of the SI-CPO algorithm to address complex continuous control tasks involving continuous state spaces with modern deep reinforcement learning techniques.

% In summary, our contributions are listed as follows.
% First, we present the SICMDP model, which can be viewed as a generalization of the ordinary CMDP model.
% Second, we propose an algorithm to perform reinforcement learning for SICMDPs, which is called SI-CRL, and we believe that we are the first to apply tools from SIP
% to solve reinforcement learning problems.
% Third, we give a theoretical analysis of SI-CRL and identify both its sample complexity and iteration complexity.
% In addition, we perform numerical experiments to illustrate the SICMDP model and validate the SI-CRL algorithm.
% \{This paragraph can be removed!!! \}






\section{Preliminaries}\label{sec:prelims}

\subparagraph{Notations.}
For a given positive integer $k \in \mathbb{N}$, the set of integers $\{1,2,\ldots,k\}$ is denoted for short as $[k]$. Given a graph $G$, the vertex set is denoted as $V(G)$ and the edge set as $E(G)$. Given two graphs $G_1$ and $G_2$, $G_1 \cup G_2$ denotes the graph $G$ where $V(G) = V(G_1) \cup V(G_2)$ and $E(G) = E(G_1) \cup E(G_2)$. 

In this paper, a regular $n$-gon is denoted by $A_1A_2A_3...A_n$ or $B_1B_2B_3...B_n$. For convenience, we define $A_{n + 1} := A_1$, $B_{n + 1} := B_1$, $A_0 := A_n$ and $B_0 := B_n$. We use the notation $\{A_i\}$ to denote the polygon $A_1A_2A_3 \ldots A_n$ and $\{B_i\}$ to denote the polygon $B_1B_2B_3 \ldots B_n$. For any regular polygon $A_1A_2A_3...A_n$, the circumcircle of the polygon is denoted as $(A_1A_2A_3...A_n)$. Given any $n$-vertex polygon in the Euclidean plane with vertices $\mathcal P = P_1P_2P_3\ldots P_n$, and interval in $\mathcal{K}$ is a subset of consecutive vertices $P_iP_{i+1\ldots P_j}$, $i,j\in [n]$, also denoted as $[P_i,P_j]$. Here $P_i$ is considered the starting vertex of the interval and $P_j$ the ending vertex. For any $P_k$, $i \leq k\leq j$ in the interval we will also use the notation $P_i \leq P_k \leq P_j$.

Given two points $P$, $Q$ in the Euclidean plane, we denote by ${\sf dist}(P,Q)$ the Euclidean distance between $P$ and $Q$. Given a line segment $AB$ in the Euclidean plane, $\overline{AB} = {\sf dist}(A,B)$. For two distinct points $A$ and $B$, $L_{AB}$ denotes the line containing $A$ and $B$; and $\overrightarrow{AB}$ denotes the ray originating from $A$ and containing $B$. 

When we refer to a graph $\mathcal{G}$ in the Euclidean plane then $V(\mathcal{G})$ is a set of points in the Euclidean plane, and $E(\mathcal{G})$ is a subset of the family of line segments $\{P_1P_2 | P_1,P_2 \in V(\mathcal{G})\}$. For any tree $\mathcal T$ in the Euclidean plane, we denote by the notation $|\mathcal T|$ the value of $\Sigma_{e \in E(\mathcal T)} \overline{e}$. A path in a tree $\mathcal T$ is uniquely specified by the sequence of vertices on the path; therefore, $P_1$, $P_2$, $P_3$, \ldots, $P_k$ (where $P_i \in V(\mathcal T), \forall i \in [k]$ and $P_iP_{i+1} \in E(\mathcal T), \forall i \in [k-1]$) denotes the path starting from the vertex $P_1$, going through the vertices $P_2$, $P_3$, \ldots, $P_{k-1}$ and finally ending at $P_k$. Equivalently, we can specify the same path as \emph{the path from $P_1$ to $P_k$}, since $\mathcal T$ is a tree. Consider the graph $T$ such that $V(T) = \{v_P| P \in V(\mathcal{T})\}$, $E(T) = \{v_{P_1}v_{P_2}| P_1P_2 \mbox{ is a line segment in } E(\mathcal{T})\}$. Then $T$ is said to be the topology of $\mathcal{T}$ while $\mathcal{T}$ is said to realize the topology $T$. Given two trees $\mathcal{T}_1$, $\mathcal{T}_2$ in the Euclidean plane, $\mathcal{T}' = \mathcal{T}_1\cup \mathcal{T}_2$ is the graph where $V(\mathcal{T}')= V(\mathcal{T}_1) \cup V(\mathcal{T}_2)$ and $E(\mathcal{T}')= E(\mathcal{T}_1) \cup E(\mathcal{T}_2)$. 

Given any graph $G$, a Steiner minimal tree or SMT for a terminal set $\mathcal{P} \subseteq V(G)$ is the minimum length connected subgraph $G'$ of $G$ such that $\mathcal{P} \subseteq V(G')$. The {\sc Steiner Minimal Tree} problem on graphs takes as input a set $\mathcal{P}$ of terminals and aims to find a minimum length SMT for $\mathcal{P}$. For the rest of the paper, we also refer to a Euclidean Steiner minimal tree as an SMT. Given a set of points $\mathcal{P}$ in the Euclidean plane, the convex hull of $\mathcal{P}$ is denoted as $\mathrm{CH(\mathcal P)}$.

\subparagraph{Euclidean Minimum Spanning Tree (MST).}
Given a set $\mathcal P$ of $n$ points in the Euclidean plane, let $G$ be a graph where $V(G) = \{v_P| P \in \mathcal P\}$ and $E(G) = \{v_{P_i}v_{P_j} | P_i,P_j \in \mathcal P\}$. Also, a weight function $w_{G}: E(T) \rightarrow \mathbb{R}$ is defined such that for each edge $v_{P_1}v_{P_2} \in E(T)$, $w_{G}(v_{P_1}v_{P_2}) = \overline{P_1P_2}$. The Euclidean minimum spanning tree of a set $\mathcal P$ is the minimum spanning tree of the graph $G$ with edge weights $w_G$. Note that a Steiner tree may have shorter length than a minimum spanning tree of the point set $\mathcal P$. 

In the plane, the Euclidean minimum spanning tree is a subgraph of the Delaunay triangulation. Using this fact, the Euclidean minimum spanning tree for a given set of points in the Euclidean plane can be found in $\OO(n\log n)$ time as discussed in \cite{Shamos1975ClosestpointP}. 

\subparagraph{Properties of a Euclidean Steiner minimal tree.}
A Euclidean Steiner minimal tree (SMT) has certain structural properties as given in~\cite{cockayne1967steiner}. We state them in the following Proposition.

\begin{proposition}\label{smt-prop}
Consider an SMT on $n$ terminals.
 \begin{enumerate}
   \item No two edges of the SMT intersect with each other.
 
   \item Each Steiner point has degree exactly $3$ and the incident edges meet at $120^\circ$ angles. The terminals have degree at most $3$ and the incident edges form angles that are at least $120^\circ$.
  
   \item The number of Steiner points is at most $n-2$, where $n$ is the number of terminals.

\end{enumerate}
\end{proposition}

 A full Steiner tree (FST) is a Steiner tree (need not be minimal, but may include Steiner points) having exactly $n-2$ Steiner points, where $n$ is the number of terminals. In an FST, all terminals are leaves and Steiner points are interior nodes. When the length of an FST is minimized, it is called a minimum FST.

All SMTs can be decomposed into FST components such that, in each component a terminal is always a  leaf. This decomposition is unique for a given SMT~\cite{hwang1992steiner}. A topology for an FST is called a full Steiner topology and that of a Steiner tree is called a Steiner topology.


%For a tree $\mathcal T$, we would denote the set of vertices (the terminal vertices and the Steiner points) as $V(\mathcal T)$ and the set of edges as $E(\mathcal T)$. Similarly for a topology $T$, $V(T)$ and $E(T)$ denote vertex set (the terminal vertices and the Steiner points) and the edge set respectively. \todo{\color{white}Anubhav: Added this defition}

\subparagraph{Steiner Hulls.}
A Steiner hull for a given set of points is defined to be a region which is known to contain an SMT. We get the following propositions from~\cite{hwang1992steiner}.

\begin{proposition}\label{convex-steiner}
    For a given set of terminals, every SMT is always contained inside the convex hull of those points. Thus, the convex hull is also a Steiner hull.
\end{proposition}

The next two propositions are useful in restricting the structure of SMTs and the location of Steiner points.

\begin{proposition} [The Lune property]\label{lune}
    Let $\rm UV$ be any edge of an SMT. Let $L(\rm{U},\rm{V})$ be the lune-shaped intersection of circles of radius $|\rm UV|$ centered on $\rm U$ and $\rm V$. No other vertex of the SMT can lie in $L(\rm{U},\rm{V})$, except $U$ and $V$ themselves.
\end{proposition}

\begin{proposition} [The Wedge property]\label{wedge}
    Let $W$ be any open wedge-shaped region having angle $120^\circ$ or more and containing none of the points from the input terminal set $\mathcal P$. Then $W$ contains no Steiner points from an SMT of $\mathcal P$.
\end{proposition}

\subparagraph{Approximation Algorithms.}
We define all the necessary terminology required in terms of a minimization problem, as ESMT is a minimization problem.
%\begin{definition} [Approximation Factor for a Minimization Problem]
%    Let $\mathcal{P}$ be a minimization problem. An algorithm $\mathcal{A}$ for the problem $\mathcal{P}$ is called an $\alpha$ factor approximation algorithm if, for every instance $\Pi$ of $\mathcal{P}$, we have $\rm{ALG}(\Pi) \leq \alpha \rm{OPT}(\Pi)$ where $\rm{ALG}(\Pi)$ and $\rm{OPT}(\Pi)$ are the values of the output of the algorithm and optimal solution for the instance $\Pi$ respectively. $\alpha$ can be a constant or a function of the input size $n$, and is always at least $1$.
%\end{definition}

%\begin{definition} [Polynomial Time Approximation Scheme (PTAS)]
%    An algorithm is called a polynomial time approximation scheme (PTAS) for a problem if it takes an input instance and a parameter $\epsilon > 0$, and outputs a solution with approximation factor $(1+\epsilon)$ for a minimization problem in time $\OO(n^{f(1/\epsilon)})$ where $n$ is the input size and $f(1/\epsilon)$ is any computable function.
%\end{definition}

\begin{definition} [Efficient Polynomial Time Approximation Scheme (EPTAS)]
    An algorithm is called an efficient polynomial time approximation scheme (EPTAS) for a problem if it takes an input instance and a parameter $\epsilon > 0$, and outputs a solution with approximation factor $(1+\epsilon)$ for a minimization problem in time $f(1/\epsilon)n^{\OO(1)}$ where $n$ is the input size and $f(1/\epsilon)$ is any computable function.
\end{definition}

\begin{definition} [Fully Polynomial Time Approximation Scheme (FPTAS)]
    An algorithm is called a fully polynomial time approximation scheme (FPTAS) for a problem if it takes an input instance and a parameter $\epsilon > 0$, and outputs a solution with approximation factor $(1+\epsilon)$ for a minimization problem in time $(1/\epsilon)^{\OO(1)}n^{\OO(1)}$ where $n$ is the input size.
\end{definition}

% appending preliminaries.tex --- Anubhav 


%------------- Inculding files ----------------
% \begin{abstract}
\label{sec:abstract}

%% 1. what is the problem 
Scientific applications that run on leadership computing facilities often face the challenge 
of being unable to fit leading science cases onto accelerator devices due to memory constraints 
(memory-bound applications).
%
% 2. what is your solution 
In this work, the authors studied one such US Department of Energy mission-critical condensed matter 
physics application, Dynamical Cluster Approximation (DCA++), and this paper discusses how device memory-bound challenges were successfully reduced  by proposing an effective 
``all-to-all'' communication method---a ring communication algorithm. 
%
This implementation takes advantage of acceleration on GPUs and remote direct memory access (RDMA) for fast data exchange between GPUs. 
%
\\Additionally, the ring algorithm was optimized with sub-ring communicators
and multi-threaded support to further reduce communication overhead and 
expose more concurrency, respectively.
%
% 3. What's the cherry-picked evaluation result you want to mention
The computation and communication were also analyzed 
by using the Autonomic Performance Environment for Exascale 
(APEX) profiling tool,  and this paper further discusses the 
performance trade-off for the ring algorithm implementation. 
%
The memory analysis on the ring algorithm shows that the allocation size for the authors' most 
memory-intensive data structure per GPU is now reduced to $1/p$ of the original size, where $p$ is the number of GPUs in the ring communicator.
%
The communication analysis suggests that 
the distributed Quantum Monte Carlo execution time grows linearly as sub-ring size increases, and the cost of messages passing through the network interface connector could be a limiting factor.


%
% \todoRed{Ronnie: Next sentence needs rewrite, too much information about Green's function that no one knows in the abstract; recommend generalizing.} \emph {However, DCA++ is currently facing memory-bound challenge as 
% a larger device array $G_t$ is limited by device memory size, where
% $G_t$ is a two-particle Green's function that allows condensed matter
% scientists to explore larger and more complex (higher fidelity)
% physics cases.}

\end{abstract}

\keywords{DCA++, Quantum Monte Carlo, GPU Remote Direct Memory Access, memory-bound issue, exascale machines}

% \section{Introduction}  \label{sec:introduction}

\newcommand\inexpIntro[3]{#1?(#2,#3).}
\newcommand\rinexpIntro[3]{*#1?(#2,#3).}
\newcommand\outexpIntro[3]{#1!(#2,#3).}
\newcommand\outatomIntro[3]{#1!(#2,#3)}

We propose a fully automated method for proving termination of \(\pi\)-calculus processes.
Although there have been a lot of studies on termination analysis for the \(\pi\)-calculus
and related calculi~\cite{Deng06IC,Demangeon07,SangiorgiTermination,KobayashiHybrid,Yoshida04IC,DBLP:journals/jlp/DemangeonHS10,Venet98SAS}, most of them have been rather theoretical,
and there have been surprisingly little efforts in developing  fully automated termination
verification methods and tools based on them. To our knowledge,
Kobayashi's \typical{}~\cite{TyPiCal,KobayashiHybrid} is the only exception that
can prove termination of \(\pi\)-calculus processes (extended with natural numbers)
fully automatically, but its termination analysis is quite limited (see Section~\ref{sec:relatedwork}).

Our method is based on a reduction to termination analysis for sequential programs:
we translate a \(\pi\)-calculus process \(P\) to a sequential program \(S_P\), so that
if \(S_P\) is terminating, so is \(P\). The reduction allows us to use
powerful, mature methods and tools
for termination analysis of sequential programs~\cite{heizmann2016ultimate,freqterm,DBLP:conf/lics/PodelskiR04,Kuwahara2014Termination,DBLP:journals/cacm/CookPR11}.

The idea of the translation is to convert a chain of communications on replicated input
channels to a chain of recursive function calls of the target sequential program.
Let us consider the following Fibonacci process:
\begin{align*}
    & \rinexpIntro{\fib}{n}{r}
        \ifexp{n<2}{ \soutatom{r}{1} \\ &\quad}
                   { \nuexp{s_1} \nuexp{s_2} (\outatomIntro{\fib}{n-1}{s_1} \PAR \outatomIntro{\fib}{n-2}{s_2} \PAR \sinexp{s_1}{x}\sinexp{s_2}{y}\soutatom{r}{x+y}) \\}
    & \PAR \outatomIntro{\fib}{m}{r}
\end{align*}
Here, the process
$\rinexpIntro{\fib}{n}{r} \ldots$ is a function server that computes the \(n\)-th Fibonacci number
in parallel and returns the result to \(r\),
and $\outatom{\fib}{m}{r}$ sends a request for computing the \(m\)-th Fibonacci number;
those who are not familiar with the syntax of the \(\pi\)-calculus may wish to consult
Section~\ref{sec:targetlanguage} first.
To prove that the process above is terminating for any integer \(m\),
it suffices to show that there is no infinite chain of communications on $\fib$:
\[
    \fib(m,r) \to \fib(m_1,r_1) \to \fib(m_2,r_2) \to \cdots.
\]
We convert the process above to the following program:\footnote{The actual translation
  given later is a little more complex.}
\begin{verbatim}
 let rec fib(n) = if n<2 then () else (fib(n-1) [] fib(n-2)) in
 fib(m)
\end{verbatim}
Here, \texttt{[]} represents the non-deterministic choice.
Note that, although the calculation of Fibonacci numbers is not preserved,
for each chain of communications on \texttt{fib}, there is a corresponding
sequence of recursive calls:
\[
\mathtt{fib}(m) \to \mathtt{fib}(m_1) \to \mathtt{fib}(m_2) \to \cdots.
\]
Thus, the termination of the sequential program above implies the termination of
the original process.
As shown in the example above, (i) each communication on a replicated input channel
is converted to a function call, (ii) each communication on a non-replicated input
channel is just removed (or, in the actual translation, replaced by a call of
a trivial function defined by \(f(\seq{x})=(\,)\)), and (iii) parallel composition
is replaced by a non-deterministic choice.
We formalize the translation outlined above and prove its correctness.

The basic translation sketched above sometimes loses too much information.
For example, consider the following process:
\begin{align*}
    & \rinexpIntro{\pre}{n}{r} \soutatom{r}{n-1} \\
    & \PAR \rinexpIntro{f}{n}{r} \ifexp{n<0}{ \soutatom{r}{1} }
                                       { \nuexp{s} (\outatomIntro{\pre}{n}{s} \PAR \sinexp{s}{x}\outatomIntro{f}{x}{r}) } \\
    & \PAR \outatomIntro{f}{m}{r}
\end{align*}
The translation sketched above would yield:
\begin{verbatim}
  let pred(n) = n-1 in
  let rec f(n) = if n<0 then () else (pred(n) [] f(*)) in
  f(m)
\end{verbatim}
Here, \texttt{*} represents a non-deterministic integer: since we have removed
the input $\sinatom{s}{x}$, we do not have information about the value of \( x \).
As a result, the sequential program above is non-terminating, although the original
process is terminating.
To remedy this problem, we also refine the basic translation above by using a refinement
type system for the \(\pi\)-calculus. Using the refinement type system,
we can infer that the value of \(x\) in the original process is less than \(n\),
so that we can refine the definition of \texttt{f} to:
\begin{verbatim}
 let rec f(n) = ... else (pred(n) [] let x=* in assume(x<n);f(x))
\end{verbatim}
The target program is now terminating, from which
we can deduce that the original process is also terminating.
We have implemented an automated tool based on the refined translation above.

The contributions of this paper are summarized as follows.
\begin{itemize}
\item The formalization of the basic translation from the \(\pi\)-calculus
  (extended with integers) to sequential programs, and a proof of its correctness.
\item The formalization of a refined translation based on a refinement type system.
\item An implementation of the refined translation, including automated refinement type
  inference based on CHC solving, and experiments to evaluate the effectiveness of
  our method.
\end{itemize}

The rest of this paper is structured as follows.
Section~\ref{sec:targetlanguage} introduces the source and target languages
of our translation.
Section~\ref{sec:approach} 
formalizes the basic translation, and proves its correctness.
Section~\ref{sec:refinement} refines the basic translation by using a refinement type system.
Section~\ref{sec:implementation} reports an implementation and experiments.
Section~\ref{sec:relatedwork} discusses related work,
and Section~\ref{sec:conclusion} concludes the paper.

% \section{Preliminaries}\label{chpt:preliminiaries}
In this chapter we will introduce some of the mathematical background and notation needed for this thesis. In particular, we will shortly introduce the differential geometric description of spacetime in Section \ref{sec:spacetime_geometry} and give an introduction to the notion of global hyperbolicity and its connection to Green- and normally-hyperbolic operators in Section \ref{sec:global_hyperbolicity}. In a bit more detail, we will introduce the notion of differential forms and give explicit definitions, also in terms of an index based notation, in Section \ref{sec:differential_forms}. For completeness, in Section \ref{sec:cat-theory}, we present basic definitions of category theory. The reader familiar with these topics can safely skip this chapter and refer to it when interested in the chosen conventions.
%
%
%
%
%%%%%%
%%SPACTIME GEOMETRY
%%%%%
%
%
%
\subsection{Spacetime geometry}\label{sec:spacetime_geometry}
In GR, the universe is mathematically described as a four dimensional \emph{spacetime}, consisting of a smooth, four dimensional manifold \gls{M} (assumed to be Hausdorff, connected, oriented, time-oriented and para-compact) and a Lorentzian metric $g$. We will assume the signature of the Lorentzian metric $g$ to be $(-,+,+,+)$. The Levi-Civita connection on $(\M,g)$ is as usual denoted by \gls{nabla}.
Throughout this thesis, we treat spacetime as fixed, implementing a gravitational background determined classically by Einstein's field equations. Hence, we neglect any back-reaction of the fields on the metric, both in the quantum and the classical case. In that sense, we treat the fields as \emph{test fields}.\par
For the basic mathematical theory regarding Lorentzian manifolds, we refer to the literature: An introduction to the topic with an emphasis on the physical application in GR is for example given in \cite{wald_GR} and \cite{carroll_spacetime-and-gr}.
Here, we will shortly recap the notion of a tangent space and tangent bundle and generalize to the notion of a vector bundle which we will use in the general description of normally hyperbolic operators and differential forms.
In the following, we generalize the setting to an arbitrary smooth manifold $\N$ of dimension $N$ with either Lorentzian or Riemannian metric $k$.\par
%
%
A \emph{tangent vector} $v_x$ at point $x \in \N$ is a linear map $v_x : C^\infty(\N , \IR) \to \IR$ that obeys the Leibniz rule, that is, for $f,g \in C^\infty (\N,\IR)$ it holds $v_x(fg) = f(x)v_x(g) + v_x(f)g(x)$.
We define the \emph{tangent space} \gls{TxN} of $\N$ at $x$ as the real $N$-dimensional vector space of all tangent vectors at point $x$.
The disjoint union of all tangent spaces is called the \emph{tangent bundle} \gls{TN} of $\N$ and is itself a manifold of dimension $2N$. A \emph{vector field} is a map $v: \N \to T\N$ such that $v(x) \in T_x\N$.
The respective dual spaces, that is the space of all linear functionals, the \emph{co-tangent space} and the \emph{co-tangent bundle}, are denoted by \gls{TsxN} and \gls{TsN} respectively.\par
%
For Lorentzian manifolds, we call a tangent vector $v$ at $x \in \N$ \emph{timelike} if $k_{\mu \nu} v^\mu v^\nu < 0$, \emph{spacelike} if $k_{\mu \nu} v^\mu v^\nu > 0$ and \emph{null} (or lightlike) if $k_{\mu \nu} v^\mu v^\nu = 0$. At every point $x \in \N$, we define the set of all \emph{causal}, that is, either timelike or null, tangent vectors in the tangent space at $x$. This set is called the \emph{light cone} at $x$ and it is split up into two distinct parts, one that we call the future light cone, and one that we call the past light cone at $x$. Since we assume the manifold to be time orientable, there exists a smooth vector field $t$ that is timelike at every $x \in \N$. Given this time orientation, we identify the future (past) light cone with the set of tangent vectors $v \in T_x\N$ such that $k_{\mu\nu} v^\mu t^\nu < 0$ (respectively $> 0$). Therefore, a tangent vector $v$ at $x$ is called \emph{future directed} (past directed) if it lies in the future (past) light cone at $x$.\\
Accordingly, a curve $\gamma : I \to \N$ is called timelike (spacelike, null, causal, future or past directed) if its tangent vector $\dot{\gamma}$ is timelike (spacelike, null, causal, future or past directed) at every $x \in \N$.  For every point $x \in \N$ we define the \emph{causal future/past} \gls{causalfuturepast} of $x$ as the set of all points $q \in \N$ that can be reached by a future directed causal curve originating in $x$. For any subset $S \in \N$ we define $J^\pm (S) = \bigcup_{x \in S} J^\pm(x)$ and $J(S) = J^+(S) \cup J^- (S)$. Finally, the future/past domain of dependence $\gls{futurepastdomainofdependence}$ of a set $S \subset \N$ is the set of all points $x \in \N$ such that every inextendible causal curve through $x$ intersects $S$. The \emph{domain of dependence} \gls{domainofdependence} of $S$ is the union of the future and past domain of dependence of the set $S$.
For more details on the causal structure of spacetime we refer to for example \cite[Chapter 8]{wald_GR}.\par
%
%
%
The notion of tangent bundles can be generalized to the notion of a vector bundle. Instead of ``attaching'' the vector spaces $T_x \N$ to every point $x$ of the manifold, we allow for the occurrence of arbitrary vector spaces, called the fibres of the vector bundle. A vector bundle then consists of the base manifold, in our case $\N$, the total space and a map $\pi$ from the total space to the base manifold, that can be locally trivialized. At each point of the base manifold, the pre-image of $\pi$ is the fibre of the vector bundle. To be precise we define, following \cite{rudolph_schmidt}:
\begin{definition}[Vector bundle]
	A smooth \emph{vector bundle} over $\N$ is a tuple $\gls{vectorbundle} = (E,\N, \pi)$, where $E$ is a smooth manifold and $\pi : E \to \N$ is a smooth surjective map satisfying:
	\begin{enumerate}
		\item For every $x \in \N$, $\pi^{-1}(x)$ is a vector space, called the fibre of the bundle at point $x$.
		\item There exists a finite dimensional vector space $F$, an open covering $\left\{ U_\alpha\right\}_\alpha$ of $\N$ and a family of diffeomorphisms $\chi_\alpha : \pi^{-1}(U_\alpha) \to U_\alpha \times F$ such that for all $\alpha$ it holds $\chi_\alpha \comp \text{pr}_1 =  \restr{\pi}{\pi^{-1}(U_\alpha)}$ and for every $x \in \N$ the map $\text{pr}_2 \comp \restr{\chi_\alpha}{\pi^{-1}(x)} : \pi^{-1}(x) \to F$ is linear.
	\end{enumerate}
\end{definition}
Here, the maps $\text{pr}_1$ and $\text{pr}_2$ denote the projection onto the first respectively second component of an element in $U_\alpha \times F$. The properties graphically mean that \emph{locally}, the vector bundle ``looks like" the product of the base manifold with the fibre. The tuples $(U_\alpha, \chi_\alpha)$ are called \emph{local trivializations} of the vector bundle. Like for vector spaces, we can define the sum and product of vector bundles, by using the according vector space definitions on the fibres of the bundle.\par
Let $\mathfrak{X}, \mathfrak{Y}$ be vector bundles over $\N$ with fibres $X_x$ and $Y_x$ at $x \in \N$. We denote by \gls{whitneysum} the \emph{Whitney sum} of the two vector bundles - the vector bundle over $\N$ whose fibres are given by the direct sum $X_x \oplus Y_x$. Similarly, one obtains the local trivializations of the Whitney sum from the trivializations of $\mathfrak{X}, \mathfrak{Y}$ and direct sums.\par
Accordingly, let $\mathfrak{X}, \mathfrak{Y}$ be vector bundles over $\N$ and $\widetilde{\N}$, with fibres $X_x$ and $Y_{\tilde{x}}$ at $x \in \N$, $\tilde{x} \in \widetilde{\N}$ respectively. We denote by \gls{outerproductbundle} the \emph{outer product} of the two vector bundles - the vector bundle over $\N \times \widetilde{\N}$ whose fibres are given by the tensor products $X_x \otimes Y_x$. Similarly, one obtains the local trivializations of the outer product from the trivializations of $\mathfrak{X}, \mathfrak{Y}$ and tensor products. \par
%
Finally, we generalize the notion of vector fields:
\begin{definition}[Sections of vector bundles]
Let $\mathfrak{X}=(E,\N,\pi)$ be a vector bundle with fibres $X_x=\pi^{-1}(x)$ at $x \in \N$. A \emph{smooth section} of the vector bundle is a smooth map $\gamma : \N \to E$ such that $\gamma(x) \in X_x$ for all $x \in \N$. The \emph{vector space of smooth sections} of $\mathfrak{X}$ is denoted by \gls{gammax}, the one with compactly supported sections is as usual denoted by \gls{gammaxzero}.
\end{definition}
In this language, a vector field $v$ is just a smooth section of the tangent bundle of a manifold, $v \in \Gamma(T\N)$. One may therefore identify the physical notion of fields with smooth sections of vector bundles. This point of view will be used to define the notion of differential forms in Section \ref{sec:differential_forms}.\par
In this thesis, we usually are interested in complex valued functions (or sections in general). Therefore, we view all occurring vector bundles as complex, in the sense that we take two distinct copies of the vector bundle, one representing the real, one the imaginary part of the bundle. A section of that complex vector bundle is just a pair of two sections of the real vector bundle under consideration. From now, if not specified explicitly, we will view all vector bundles, including the tangent bundle $T\N$, as complex vector bundles. Accordingly, smooth sections of those bundles will in general be complex valued.
%
%
%
%
%
%
%
%
%%%%%%%
%%PARTIAL DIFFERENTIAL OPERATORS AND GLOBAL HYPERBOLICITY
%%%%%%%
%
%
%
\subsection{Partial differential operators and global hyperbolicity}\label{sec:global_hyperbolicity}
When dealing with field theories, whether classical or quantum, one is, of course, interested in the dynamics of the fields. These are usually described by some partial differential equation, often of second order. In the following, we give a short introduction to the theory of certain partial differential operators acting on smooth sections of a vector bundle over the spacetime $(\M,g)$.\par
%
As we have seen, these smooth sections are generalizations of the notion of a field.  In the following, let $\mathfrak{X}$ denote a vector bundle over the manifold $\M$ and let $P: \Gamma(\mathfrak{X}) \to \Gamma(\mathfrak{X})$ be a partial differential operator acting on smooth sections of the bundle. As in the case of flat spacetime, we are interested in basic questions regarding the differential equation $Pf = j$, for example: Can we formulate a (globally) well posed initial value problem? Does the differential equation possess (unique) solutions? To answer these questions, we will now restrict to the case where $P$ is linear and of second order, as it is often the case in physical applications. One can show that for a certain class of such operators, namely normally hyperbolic partial differential operators of second order, we can rigorously treat these questions.\par
Choosing local coordinates $x=(x_\mu)$ on $\M$ and a local trivialization of $\mathfrak{X}$, a linear partial differential operator of second order is called \emph{normally hyperbolic} if it takes the form
\begin{align}
	P = - \sum_{\mu,\nu} g^{\mu \nu} \partial_\mu \partial_\nu + \sum_{\alpha} A_\alpha (x) \partial_\alpha + B(x) \formspace,
\end{align}
where $A_\alpha$ and $B$ are matrix-valued coefficients depending smoothly on the coordinate $x$ (see. \cite[Chapter 1.5]{baer_ginoux_pfaeffle}). One can also formulate a coordinate independent definition in terms of the principal symbol, which we will not present here (see for example \cite[Section 1.5]{baer_ginoux_pfaeffle} ). \par
%
Normally hyperbolic operators possess unique fundamental solutions (see for example the fundamental solutions to the wave operator as noted in Lemma \ref{lem:fundamental_solution_wave_operator}). These fundamental solutions fulfill certain physically important properties, such as a finite propagation speed smaller than the speed of light. Furthermore, specifying the initial data on some space-like hypersurface $X \in  \M$ specifies a unique solution on the domain of dependence $D(X)$ of $X$. Due to these properties, one often calls normally hyperbolic operators just \emph{wave operators}. But to state a \emph{globally} well posed initial value problem for a wave equation, we need to restrict the class of spacetimes $\M$ under consideration to those that possess space-like hypersurfaces $X$ whose domain of dependence is all of the spacetime, $D(X) = \M$. This leads to the notion of \emph{globally hyperbolic} spacetimes:
\begin{definition}[Global Hyperbolicity]
	A spacetime $\M$ is called \emph{globally hyperbolic} if there exists a Cauchy surface $\gls{sigma}$ in $\M$.
\end{definition}
\noindent Here, a Cauchy surface is a space-like hypersurface $\Sigma \subset \M$ such that every inextendible causal curve $\gamma$ intersects $\Sigma$ exactly once. One can show that Cauchy surfaces fulfill the desired property mentioned above, that is,  $D(\Sigma) = \M$. Furthermore, one can show that any globally hyperbolic spacetime $\M$ is foliated by a one-parameter family $\left\{ \Sigma_t \right\}_t$ of Cauchy surfaces (see for example \cite[Theorem 8.3.14]{wald_GR}). \par
In physical applications, one often finds the dynamics of a theory to be described by wave operators. Most prominently, the Klein-Gordon operator $(\square + m^2)$ acting on scalar fields, or its generalization, the wave operator acting on differential forms introduced in Section \ref{sec:differential_forms}, is normally hyperbolic. But there are also important physical field theories that are not described by wave operators, such as the Proca field treated in this thesis. It turns out that the Proca operator (see Definition \ref{def:proca_operator}) is a so called \emph{Green-hyperbolic} operator. These are again partial differential operators $P$ of second order acting on smooth sections of some vector bundle, such that $P$ (and its dual $P'$) posses fundamental solutions. Obviously, normally hyperbolic operators are Green-hyperbolic, but the opposite is not true. One can generalize some results obtained by studying normally hyperbolic operators to Green-hyperbolic operators. An introduction to this topic is given in \cite{baer_green-hyperbolic}, where it is also shown that the Proca operator is Green-hyperbolic but not normally hyperbolic.\par
For our application, the notion of Green-hyperbolicity is not of vast importance, but it is worth mentioning that there exists a more detailed mathematical background on the treatment of such operators.
A very detailed description of normally hyperbolic operators on Lorentzian manifolds, including proofs of the above statements regarding the initial value problem and the existence of fundamental solutions, is given in \cite{baer_ginoux_pfaeffle}, also with an overview of quantization. A shorter introduction to the topic is for example treated in \cite{baer-ginoux_classical-and-quantum-fields}, also with a description of quantization.
%
%
%
%
%
%
%%%
%
%
%
%%
%%%%%%%%%
%%%DIFFERENTIAL FORMS
%%%%%%%%
%
%
%
\subsection{Differential forms}\label{sec:differential_forms}
%
%
Differential forms provide an elegant, coordinate independent description of calculus on smooth manifolds. In particular, they generalize the notion of line- and volume-integrals that are known from analysis. Differential forms play a remarkable role in physics, as one can argue that they indeed describe fundamental physical entities. As an example, instead of viewing a classical force as a vector, one can think of it, more closely related to experiments, as a differential one-form that assigns a scalar to a tangent vector of a curve. This scalar is the (infinitesimal) work associated with the force along the curve. Also, differential forms allow for an elegant geometric description of field theories, for example the Maxwell and Proca field theories that we encounter in this thesis. In Maxwell's classical theory of electromagnetism, instead of viewing the electric and magnetic field (which are conceptually just forces) as the fundamental physical entities, one introduces the \emph{vector potential}, a one-form, consisting of the scalar electric potential and the vector potential associated with the magnet field. Experiments like the Aharonov-Bohm experiment allow for an interpretation of the vector potential as the fundamental physical object, rather than the associated electromagnetic field. \\
Even more fundamentally, the two main theories of physics, General Relativity and the Standard Model of particle physics, are field theories. They are deeply connected to a geometric interpretation and can be elegantly described using differential forms. \par
%
%
Despite of all this, differential forms are usually not part of the standard curriculum of physicists. We shall therefore introduce the basic aspects and definitions regarding differential forms that are used in this thesis. For a more detailed introduction we refer to the literature: For example \cite[Chapter 2 and 4]{rudolph_schmidt} or \cite[Appendix B]{wald_GR} provide introductions to the topic.\par
%
%
In the following, let $\N$ denote a smooth $N$-dimensional manifold, assumed to be Hausdorff, connected, oriented and para-compact, with either Lorentzian or Riemannian metric $k$ and Levi-Civita connection $\nabla$. For a Lorentzian manifold we use the sign convention $(-,+,\dots,+)$ of the metric $k$. The number of negative eigenvalues of $k$ is denoted by $s$, so $s=0$ for a Riemannian manifold and, in our convention, $s=1$ for a Lorentzian manifold.
Later, we will specify to a four dimensional (globally hyperbolic) spacetime consisting of a four dimensional manifold $\M$ with Lorentzian metric $g$ and Cauchy surface $\Sigma$ with induced Riemannian metric $h$.
%
We define:
\begin{definition}[Differential form]
	Let $p\in \{0,1,\dots,N\}$. A \emph{differential form} $\omega$ of degree $p$, or $p$-form for short, on the manifold $\N$ is an anti-symmetric tensor field of rank $(0,p)$. That is, at every point $x \in \N$, $\omega_x$ is an anti-symmetric multi-linear map
	\begin{align}
	\omega_x : \underbrace{T_x \N \times T_x \N \times \cdots \times T_x \N}_{p\text{-times}} \to \IR \formspace.
	\end{align}
	We denote the vector space\footnote{Naturally, addition and scalar multiplication are defined point-wise.} of $p$-forms on $\N$ by $\gls{omegap}$, the space with compactly supported ones by \gls{omegapz}.
\end{definition}
As an example, a zero-form $f \in \Omega^0(\N)$ is just a $C^\infty$-function from $\N$ to $\IR$, hence we can identify $\Omega^0(\N) = C^\infty (\N, \IR)$. A one-form $A \in \Omega^1(\N)$ is nothing more than a co-vector field and in a physical context usually denoted in local coordinates by $A_\mu$. Note, that alternatively one can directly define a $p$-form as a smooth section of the $p$-th exterior product of the co-tangent bundle and hence identify $\Omega^p(\N) = \Gamma \big( \largewedge^k T^*\N\big)$. As mentioned in Section \ref{sec:spacetime_geometry}, we view the tangent bundle as a complex bundle. Therefore, the sections of that bundle will be complex valued functionals. In that fashion, we will usually view the spaces $\Omega^p(\N)$ as complex valued differential forms.\par
%
Next we define the basic operations, besides addition and scalar multiplication, that one can perform on differential forms.
%
\begin{definition}[Exterior product]
	Let $A \in \Omega^p(\N)$ be a $p$-form and  $B\in \Omega^q(\N)$ a $q$-form on $\N$. \\
	The \emph{exterior product} $\gls{wedge}:\Omega^p(\N) \times \Omega^q(\N) \to \Omega^{p+q} (\N)$ is defined by
	\begin{align}
	(A \wedge B)_{\mu_1\dots\mu_p \nu_1\dots\nu_q} = \frac{(p+q)!}{p!q!}\, A_{[\mu_1 \dots \mu_p} B_{\nu_1\dots\nu_q]} \formspace,
	\end{align}
	where the anti-symmetrization of a tensor $T$ is given through
	\begin{align}
	T_{[\mu_1\dots\mu_p]} = \frac{1}{p!} \sum\limits_{\sigma\in S_N }\textrm{sgn}(\sigma) T_{\sigma(\mu_1)\dots\sigma(\mu_p)} \formspace.
	\end{align}
\end{definition}
Here, $S_N$ denotes the symmetric group\footnote{Usually the symmetric group is defined as the set of permutations of $\{1,2,\dots,N\}$ but we chose the index to run over $\{0,1,\dots,N-1\}$, identifying the time component with zero rather then one.} of degree $N$, consisting of permutations of the set $\{0,1,\dots,N-1\}$.
With this notion of multiplication, point-wise addition and scalar multiplication, the space $\gls{omega} \coloneqq \bigoplus_{p = 0}^\infty \Omega^p(\N) = \bigoplus_{p = 0}^N \Omega^p(\N)$ becomes an algebra, usually called the Grassmann- or \emph{exterior algebra} of differential forms on $\N$. We have used that obviously $\Omega^k(\N) =0$ for $k >N$ due to the anti-symmetrization.\par
Furthermore, we find a notion of how to \emph{pullback} differential forms on manifolds to another manifold, for example the pullback of a differential form on the spacetime $\M$ to differential forms on its Cauchy surface $\Sigma$. Given a $C^\infty$-map $\psi: \widetilde{\N} \to \N$, where $\N, \widetilde{\N}$ are manifolds, we can naturally define the pullback of a function $f \in \Omega^0(\N)$ to a function $(\psi^* f) \in \Omega^0(\widetilde{\N})$ by composing $f$ with $\psi$:
\begin{align}
\psi^* f \coloneqq f \comp \psi \formspace.
\end{align}
\newpage
With the pullback of functions defined, we can define how to \emph{push forward}, or carry along, vector fields on $\widetilde{\N}$ to vector fields on $\N$: Let $f\in \Omega^0(\N)$ and $\tilde{v} \in \Gamma(T\widetilde{\N})$ and $\tilde{x} \in \widetilde{\N}$. Then
\begin{align}
(\psi_* \tilde{v})_{\psi(\tilde{x})} (f) \coloneqq \tilde{v}_{\tilde{x}}(\psi^* f)
\end{align}
defines the vector field $(\psi_* v) \in \Gamma(T\N)$. With these basic operations at hand, we can generalize to define the pullback of differential forms:
\begin{definition}[Pullback]\label{def:pullback}
	Let $\N, \widetilde{\N}$ be manifolds of dimension $N,\widetilde{N}$ respectively, and let $\psi: \widetilde{\N} \to \N$ be a smooth map. Then, $\psi$ defines an algebra homomorphism $\psi^* : \Omega(\N) \to  \Omega(\widetilde{\N})$,
	called the \emph{pullback} of differential forms. For $\omega \in \Omega^p(\N)$, $\tilde{x} \in \widetilde{\N}$ and $\tilde{v}_i \in T_x \widetilde{\N}$, $i=1,2,\dots,p$, it is defined by
	\begin{align}
	\left( \psi^* \omega \right)_{\tilde{x}}  (\tilde{v}_1,\tilde{v}_2,\dots,\tilde{v}_p) \coloneqq \omega_{\psi(\tilde{x})} (\psi_* \tilde{v}_1, \dots , \psi_* \tilde{v}_p) \formspace.
	\end{align}
\end{definition}
%
%
%
%
On the exterior algebra we find a duality, provided by the Hodge operator:
\begin{definition}[Hodge dual]
	The hodge star operator $\gls{hodge}: \Omega^p(\N) \to \Omega^{N-p}(\N)$ is defined through
	\begin{align}
	B \wedge *A = \frac{1}{p!} B^{\mu_1\dots\mu_p}A_{\mu_1\dots\mu_p} \dvolk \formspace,
	\end{align}
	which yields the coordinate representation
	\begin{align}
	(*A)_{\mu_{p+1}\dots\mu_N} = \frac{\detk}{p!} \, \epsilon_{\mu_1\dots\mu_N} A^{\mu_1\dots\mu_p} \formspace.
	\end{align}
\end{definition}
Here, \gls{levicivita} denotes the fully antisymmetric tensor of rank $N$ (Levi-Civita symbol) satisfying $\epsilon_{12,\dots,N} =1$ and the \emph{volume element} \gls{dvolk} is defined by
\begin{align}
\left( \gls{dvolk} \right)_{\alpha_1\dots\alpha_N} = \detk \, \epsilon_{\alpha_1\dots\alpha_N} \formspace.
\end{align}
In a sense, the volume element describes how the curvature of the manifold deforms a unit volume.
The duality follows from the important property of the Hodge operator as stated in the following lemma:
\begin{lemma}
	Let $*$ denote the Hodge star operator on the exterior algebra $\Omega(\N) $. It holds that
	\begin{align}
	** = (-1)^{s+p(N-p)} \, \mathbbm{1} \formspace,
	\end{align}
	which is trivially equivalent to $*^{-1} = (-1)^{s+p(N-p)} \, *$.
\end{lemma}
\begin{proof}
	Let $A \in \Omega^p(\N)$ be a $p$-form on $\N$. Then:
	\begin{align}
	(*{*A})_{\mu_1 \dots \mu_p}
	&= \frac{\detk \, \detk}{p! \, (N-p)!} \; \epsilon_{\alpha_{p+1}\dots\alpha_N \mu_1 \dots \mu_p}\;\epsilon^{\alpha_{1}\dots\alpha_N}\;A_{\alpha_1\dots\alpha_p} \notag\\
	&= (-1)^{p(N-p)} \frac{\detk \, \detk}{p! \, (N-p)!} \; \epsilon_{\alpha_{p+1}\dots\alpha_N \mu_1 \dots \mu_p}\;\epsilon^{\alpha_{p+1}\dots\alpha_{N}\alpha_1\dots\alpha_p}\;A_{\alpha_1\dots\alpha_p}  \notag\\
	&= (-1)^{s+p(N-p)} \delta\indices{^{[\alpha_{1}}_{\mu_{1}}}\, \dots \, \delta\indices{^{\alpha_p ] }_{\mu_p}} \;A_{\alpha_1\dots\alpha_p} \notag\\
	&=  (-1)^{s+p(N-p)}\;A_{\mu_1\dots\mu_p} \formspace
	\end{align}
	We have used Lemma \ref{lem:epsilon_contraction} and, in the last step, that the anti-symmetrization is absorbed by contraction because $A$ is antisymmetric.
\end{proof}
%
%
%
%
%
Furthermore, we can equip the exterior algebra with a differentiable structure, introducing the notion of the exterior derivative.
\begin{definition}[Exterior derivative]
	The \emph{exterior derivative} $\gls{d}:\Omega^p(\N) \to \Omega^{p+1} (\N)$ is defined by the following properties:
	\begin{enumerate}
		\item $d$ is linear
		\item $d$ obeys a graded Leibniz rule: Let $A \in \Omega^p(\N)$ and  $B\in \Omega^q(\N)$, then
		\begin{align}
		d(A \wedge B) = dA \wedge B + (-1)^p \, A \wedge dB
		\end{align}
		\item $d$ is nilpotent, that is,  $d^2 = 0$.
	\end{enumerate}
	In local coordinates, this is equivalent to the representation
	\begin{align}
	(dA)_{\mu \alpha_1\dots\alpha_p} = (p+1)\, \nabla_{[\mu}A_{\alpha_1\dots\alpha_p]} \formspace.
	\end{align}
\end{definition}
An important property of the exterior derivative is that it commutes (or rather intertwines its action) with pullbacks (see \cite[Proposition 4.1.7]{rudolph_schmidt}).
A $p$-form $\omega \in \Omega^p(\N)$ is called \emph{exact} if there is a $(p-1)$-form $\alpha \in \Omega^{p-1}(\N)$ such that $\omega = d\alpha$. We call $\omega$ \emph{closed} if $d \omega =0$. Accordingly, the space of closed $p$-forms is denoted by \gls{omegapd}, the space of exact ones by \gls{domegap}. As usual, the ones with compact support are denoted by a subscript zero. Note, that every exact form is closed, using that $d$ is by definition nilpotent, but the reverse is in general not true. It does hold, however, on certain manifolds with trivial topology, such as Minkowski spacetime. This is expressed in the so called Poincar\'e-Lemma (see for example \cite[Chapter 4]{bott_tu}) based on the study of de Rham cohomology.\par
%
Moreover, $N$-forms can naturally be integrated. Using local coordinates and a partition of unity, we define the integral of $N$-forms via the well known integration on $\IR^N$:
\begin{definition}[Integration on manifolds]
	Let $\left\{U_\alpha, \psi_\alpha\right\}_\alpha$ be an atlas of the manifold $\N$ and $\left\{\chi_\alpha\right\}_\alpha$ a partition of unity subordinate to the locally finite open cover $\left\{U_\alpha\right\}_\alpha$. Let $x^\mu_{(\alpha)}$ be a coordinate basis of $\psi$ on $U_\alpha$. For any $N$-form $\omega \in \Omega^N_0(\M)$ we define the integral
	\begin{align}
	\int\limits_{\N} \omega &\coloneqq \sum_{\alpha} \int\limits_{\psi_\alpha (U_\alpha)} w(x_{(\alpha)}^0,\dots,x_{(\alpha)}^1)\; dx_{(\alpha)}^0 \cdots dx_{(\alpha)}^{N-1} \formspace,
	\end{align}
	where $w$ are the components of $\omega$ in the coordinates $x_{(\alpha)}^\mu$, that is $\omega = w dx_{(\alpha)}^0 \wedge \cdots \wedge dx_{(\alpha)}^{N-1}$.
	This definition is independent of the choice of the atlas and the partition of unity (see \cite[Proposition 3.3]{bott_tu}).
\end{definition}
With integration at our disposal, we present an important theorem regarding the integration of exact differential forms:
\begin{theorem}[Stoke's Theorem]\label{thm:stokes}
	Let $\N$ be an oriented manifold of dimension $N$ and let its boundary $\partial \N$ be endowed with the induced orientation. Let $\gls{inclusionmap} : \partial \N \hookrightarrow \N$ be the inclusion operator.
	Let $\omega \in \Omega^{N-1}_0(\N)$ be a compactly supported $(N-1)$-form on $\N$. Then it holds
	\begin{align}
	\int\limits_\N d\omega = \int\limits_{\partial \N} i^*\omega \formspace.
	\end{align}
\end{theorem}
\begin{proof}
	A proof is given in most of the introductory literature on differential geometry (see for example \cite[Chapter 17, Theorem 2.1]{lang}).
	Note that one can equivalently formulate Stoke's theorem on a \emph{compact} manifold but for {arbitrary} (that is, in general not compactly supported) $(N-1)$-forms on the manifold (see for example \cite[Theorem 4.2.14]{rudolph_schmidt}). This will be of importance in later calculations.
\end{proof}
%
Furthermore, we can define a bilinear map on $\Omega^p(\N)$ using the integration of $N$-forms:
\begin{definition}
	Let $A,B \in \Omega^p(\N)$ such that their supports have a compact intersection. Define the bilinear map $\gls{innerprod} : \Omega^p(\N) \times \Omega^p(\N) \to \IC$ by
	\begin{align}
	\langle A, B \rangle_\N \coloneqq  \int_{\N } A \wedge * B = \int_{\N } A_{\mu_1 \dots \mu_p}B^{\mu_1 \dots \mu_p}\,\dvolk \formspace.
	\end{align}
\end{definition}
Since by definition $A \wedge * B$ is a compactly supported $N$-form, this is well defined. We may sometimes refer to $\langle \cdot , \cdot \rangle_\N$ as an inner product for simplicity, even though it is not positive definite.
%
%
%
%
%
Using the exterior derivative, we define the interior or co-derivative:
\begin{definition}[Interior derivative]
	The \emph{interior derivative} $\gls{delta} : \Omega^p(\N) \to \Omega^{p-1}(\N)$ is defined by
	\begin{align}
	\delta \coloneqq (-1)^{s+1+N(p-1)}\, {*{d*}} \formspace.
	\end{align}
	From the defining properties of $d$ and $*$ it follows $\delta^2 =0$.
\end{definition}
Here, $s$ again denotes the number of negative eigenvalues of the metric $k$ of $\N$. In accordance with our nomenclature, we call a $p$-form $\omega$ co-exact if there exists a $\alpha \in \Omega^{p+1}(\N)$ such that $\omega = \delta \alpha$ and co-closed if $\delta \omega = 0$. Accordingly, the spaces of co-closed and co-exact $p$-forms are denoted by \gls{omegapdelta} and \gls{deltaomegap} respectively.\par
Using the exterior and interior derivative we define the partial differential operator:
\begin{definition}[D'Alembert Operator]
	The d'Alembert (or Laplace - de Rham) operator $\gls{dalembert}: \Omega^p(\N) \to \Omega^{p}(\N)$ is defined by
	\begin{align}
	\square \coloneqq \delta d +d \delta \formspace.
	\end{align}
\end{definition}
By definition of the exterior and interior derivative, it is easy to show that $\square$ commutes with both $d$ and $\delta$:
\begin{align}
\square d &= (\delta d + d \delta )d \notag \\
&= d \delta d \notag \\
&= d (\delta d + d \delta) \formspace,
\end{align}
and analogously for $\delta$.
The d'Alembert operator, and its generalization to $(\square + m^2)$ for some constant $m > 0$, are important examples for a normally hyperbolic differential operators (see Section \ref{sec:global_hyperbolicity}) and we may therefore sometimes just refer to them as \emph{wave operators}.\par
The sign convention in the definition of the exterior derivative is chosen such that on any Lorentzian or Riemannian manifold the interior derivative is formally adjoint to the exterior derivative, that is,  for $A \in \Omega^{p}(\N)$ and $B \in \Omega^{p+1}(\N)$ it holds that
\begin{align}
\langle dA , B \rangle_{\N} = \langle A , \delta B \rangle_\N \formspace,
\end{align}
which leads to a representation in local coordinates of the Manifold given by:
\begin{align}
(\delta A)_{\mu_2\dots\mu_p} = - \nabla^{\mu_1}A_{\mu_1\dots\mu_p} \formspace.
\end{align}
To see that this is consistent, let $A \in \Omega^{p-1}(\N)$ and $B \in \Omega^{p}(\N)$ such that their supports have compact intersection.
We obtain, using Stoke's Theorem \ref{thm:stokes}:
\begin{align}
0 &= \int \limits_{\partial \N} i^* (A \wedge *B) \notag\\
&= \int \limits_{\N} d(A \wedge *B)  \notag\\
&= \int \limits_{\N} dA \wedge *B + (-1)^{p-1} A \wedge d{*B} \notag\\
&= \int \limits_{\N} dA \wedge *B + (-1)^{p-1} A \wedge *{*^{-1}}\underbrace{d{*B}}_{\textrm{is a } (N-p+1) \textrm{ form.}} \notag\\
&= \int \limits_{\N} dA \wedge *B + (-1)^{p-1}(-1)^{s+(N-p+1)(N-N+p-1)} A \wedge *{*d{*B}} \notag\\
&= \int \limits_{\N} dA \wedge *B + (-1)^{p+(1-p)(p-1)} A \wedge *\delta B \formspace.
\end{align}
It can easily be proven by induction that $\big(p+(1-p)(p-1)\big)$ is odd for any $p \in \IN$, which yields the result
\begin{align}
\langle dA , B \rangle_{\N} = \langle A , \delta B \rangle_\N \formspace.
\end{align}
The definitions stated above thus fulfill the requirement of formal adjointness of the exterior and interior derivate on an arbitrary Lorentzian or Riemannian manifold $\N$.
In local coordinates we use a partial integration to obtain
\begin{align}
\langle dA , B \rangle_\N &= \int \limits_{\N} dA \wedge * B \notag\\
%&= \int \limits_{\N} \frac{1}{p!} (dA)^{\alpha_1\dots\alpha_p}\,B_{\alpha_1 \dots \alpha_p} \, \dvolk \notag\\
&= \int \limits_{\N}  \frac{p}{p!} \nabla^{[\alpha_1}A^{\alpha_2\dots\alpha_p]}\,B_{\alpha_1 \dots \alpha_p} \, \dvolk \notag\\
&= \int \limits_{\N}  \frac{1}{(p-1)!} \nabla^{\alpha_1}A^{\alpha_2\dots\alpha_p}\,B_{\alpha_1 \dots \alpha_p} \, \dvolk \notag\\
&= - \int \limits_{\N}  \frac{1}{(p-1)!} A^{\alpha_2\dots\alpha_p}\, \nabla^{\alpha_1}B_{\alpha_1 \dots \alpha_p} \, \dvolk \notag\\
&= \langle A, \delta B \rangle_\N \formspace,
\end{align}
which yields
\begin{align}
-\nabla^{\alpha_1}B_{\alpha_1 \dots \alpha p} = (\delta B)_{\alpha_2 \dots \alpha_p}\formspace.
\end{align}
On the four dimensional spacetime $(\M,g)$ the definitions of the Hodge star operator and the interior derivative simplify, such that
\begin{align}
*_{(\M)}*_{(\M)} &= (-1)^{p+1} \mathbbm{1} \\
\delta_{(\M)} &= *_{(\M)}{d_{(\M)}*_{(\M)}} \formspace ,
\end{align}
holds on the spacetime $(\M,g)$ and
\begin{align}
*_{(\Sigma)}*_{(\Sigma)} &= \mathbbm{1} \\
\delta_{(\Sigma)} &= (-1)^p *_{(\Sigma)}{d_{(\Sigma)}*_{(\Sigma)}}
\end{align}
holds on  $(\Sigma,h)$. In the following we will drop the subscript ${(\M)}$, since we will perform all the calculations on a four dimensional spacetime, except when explicitly noted (for example with a subscript $(\Sigma)$).
%
%
%
%
%
%
%
%
%%%%%%
%%CATEGORY THEORY
%%%%%%
\subsection{Category theory}\label{sec:cat-theory}
The description of Quantum Field Theory on Curved Spacetimes (QFTCS) in the framework of \name{Brunetti}, \name{Fredenhagen} and \name{Verch} \cite{Brunetti_Fredenhagen_Verch} is based on category theory. In this thesis, we will not go into detail on those categorical aspects, however we will need some basic definitions to formulate the theory rigorously, that is namely the notion of a category and that of covariant functors, since, in the used framework, the generally covariant QFTCS is a functor.\par
Here, we present definitions given in \cite[Appendix A.1]{baer_ginoux_pfaeffle} and refer to the appropriate literature for details. We define:
\begin{definition}[Category]
	A \emph{category} $\mathsf{Cat}$ consists of the following:
	\begin{enumerate}
		\item a class $\mathsf{Obj}_\mathsf{Cat}$ whose members are called \emph{objects},
		\item a set $\mathsf{Mor}_\mathsf{Cat}(A,B)$, for any two objects $A,B \in \mathsf{Obj}_\mathsf{Cat}$, whose elements are called \emph{morphisms},
		\item for any three objects $A,B,C \in \mathsf{Obj}_\mathsf{Cat}$ there is a map
		\begin{align}
\mathsf{Mor}_\mathsf{Cat}(B,C) \times \mathsf{Mor}_\mathsf{Cat}(A,B) &\to \mathsf{Mor}_\mathsf{Cat}(A,C) \notag\\
(\psi,\phi) &\mapsto \psi \comp \phi
		\end{align}
		called the composition of morphisms subject to the relations:\vspace{4mm}
		\begin{enumerate}[label=(\arabic*)]
			\item for non equal pairs $(A,B)$, $(A',B')$ of objects, the sets $\mathsf{Mor}_\mathsf{Cat}(A,B)$ and $\mathsf{Mor}_\mathsf{Cat}(A',B')$ are disjoint,
			\item for every object $A$ there exists a morphism $\text{id}_A \in \mathsf{Mor}_\mathsf{Cat}(A,A)$ such that it holds for all objects $B$, morphisms $\psi \in \mathsf{Mor}_\mathsf{Cat}(B,A)$ and $\phi \in \mathsf{Mor}_\mathsf{Cat}(A,B)$
			\begin{align}
				\text{id}_A \comp \psi &= \psi \quad \text{and}\\
				\phi \comp \text{id}_A &= \phi \quad,
			\end{align}
			\item the composition law is associative, that is for an objects $A,B,C,D$ and any morphisms $\psi \in \mathsf{Mor}_\mathsf{Cat}(A,B)$, $\phi \in \mathsf{Mor}_\mathsf{Cat}(B,C)$ and $\chi \in \mathsf{Mor}_\mathsf{Cat}(C,D)$ it holds
			\begin{align}
				(\chi \comp \phi) \comp \psi = \chi \comp (\phi \comp \psi) \formspace.
			\end{align}
		\end{enumerate}
	\end{enumerate}
\end{definition}
%
%
%
\begin{definition}[Functor]
	Let $\mathsf{Cat1}$ and $\mathsf{Cat2}$ be categories. A \emph{covariant functor} $\mathscr{A}: \mathsf{Cat1} \to \mathsf{Cat2}$ consists of the map $\mathscr{A} : \mathsf{Obj}_\mathsf{Cat1} \to \mathsf{Obj}_\mathsf{Cat2}$ and maps $\mathscr{A}: \mathsf{Mor}_\mathsf{Cat1}(A,B) \to \mathsf{Mor}_\mathsf{Cat2}\big(\mathscr{A}(A),\mathscr{A}(B)\big)$ for any two objects $A,B \in \mathsf{Obj}_\mathsf{Cat1}$ such that
	\begin{enumerate}
		\item {the composition is preserved, that is for all objects $A,B,C \in \mathsf{Obj}_\mathsf{Cat1}$ and for any morphisms $\psi \in \mathsf{Mor}_\mathsf{Cat1}(A,B)$ and $\phi \in \mathsf{Mor}_\mathsf{Cat1}(B,C)$ it holds
		\begin{align}
			\mathscr{A}(\phi \comp \psi) = \mathscr{A}(\phi) \comp \mathscr{A}(\psi) \formspace,
		\end{align}}
		\item{
			$\mathscr{A}$ maps identities to identities, that is for any object $A \in \mathsf{Obj}_\mathsf{Cat1}$ it holds
			\begin{align}
				\mathscr{A}(\text{id}_\mathsf{A}) = \text{id}_{\mathscr{A}(A)} \formspace.
			\end{align}
			}
	\end{enumerate}
\end{definition}
%
%
%
%
%
%
%
%
%
%
%
%
%%%%%%
%%SIGN CONVENTIONS
%%%%%%
%
%
\subsection{Sign conventions}\label{sec:sign_conventions}
At certain points throughout this chapter we have had a freedom of choice regarding the signs of some entities, in particular the sign of the signature of the Lorentzian metric $g$ and that of the interior derivative $\delta$. Though at this stage the choice can be made arbitrarily, we want to make it in a way that in the end allows us to make certain physical interpretations on some parameters. More precisely, we want to interpret the parameter $m$ of the Klein-Gordon equation\footnote{or its generalization on $p$-forms} $(\square + m^2) f = 0$ for a zero-form $f \in \Omega^0(\M)$ as a mass in the physical sense. With the chosen sign convention for $\delta$ we find, using ${\delta}f = 0$:
\begin{align}
	\square f
	&= (\delta d + d \delta) f \notag\\
	&= \delta d f \notag\\
	&= - \nabla^\mu \nabla_\mu f \formspace.
\end{align}
In the following heuristic (local) argument we see
\begin{align}
	\square + m^2
	&= -\nabla^\mu \nabla_\mu + m^2 \notag\\
	&\sim \partial_t^2 + \sum_i \partial_i^2 + m^2\notag\\
	&\sim -E^2 + \abs{\vector{p}}^2 + m^2
\end{align}
which yields the correct relativistic relation of energy, momentum and mass according to $E^2 = \abs{\vector{p}}^2 + m^2$.
A similar calculation holds for the Klein-Gordon operator generalized to act on one-forms. If we had found a ``wrong'' relation between energy, momentum and mass, we would have had to adapt the chosen signs. Usually one chooses the sign of the metric and the interior derivative such that they are in some sense mathematically convenient (although one might disagree with another one's choice). We have made the choice of the metric, such that the Cauchy surfaces become Riemannian rather that ``anti-Riemannian'' (with an all minus signature), which seems more natural to some. Also, a lot of the used references on spacetime geometry (in particular the book by \name{Wald} \cite{wald_GR}) use this sign convention, which makes the application of certain formulas easier. As mentioned, the sign of the interior derivative was chosen such that it is formally adjoint to the exterior derivative (with respect the specified inner product) on all Lorentzian and Riemannian manifolds. It seemed convenient for the actual calculations to fix the sign regardless of the signature of the metric of the underlying manifold. One could equivalently have fixed the opposite sign, yielding the two derivatives to be skew-adjoint, which is also done in the literature. However, in the end, one has one freedom left to make the energy-momentum-mass relation work: that is the sign in front of the mass in the Klein-Gordon equation and all other wave equations accordingly. Hence, one regularly also finds the Klein-Gordon equation to be defined with a flipped sign of the mass term. But for our case, we want the mass $m$ in any wave equation to appear with a positive sign.
%
%

%----------------------------------------------
\section{Polynomial cases for \ESMT}\label{polycase}
In this section, we consider the \ESMT problem for $2$-CPR $n$-gons. Throughout the section, we denote the inner $n$-gon as $\{A_i\}$ and the outer $n$-gon as $\{B_i\}$. First, we consider the configuration of an Euclidean Steiner minimal tree in a subsection of the annular area between $\{A_i\}$ and $\{B_i\}$, which will form an isosceles trapezoid. Next, we consider the simple but illustrative case of $n = 3$. Finally we prove our result for all $n$.

\subsection{Isosceles Trapezoids and Vertical Forks} \label{trapezoids}


In this section, we discuss one particular Steiner topology when the terminal set is  formed by the four corners of a given isosceles trapezoid. However, we will limit the discussion to only the isosceles trapezoids such that the angle between the non-parallel sides is of the form $\frac{2 \pi}n$ where $n \in \mathbb{N}$, $n \ge 4$. The reason is that given $2$-CPR $n$-gons $\{A_i\}$, $\{B_i\}$, for $n \geq 4$ and for any $j \in \{1,\ldots, n-1\}$, the region $A_jA_{j + 1}B_jB_{j + 1}$ is an isosceles trapezoid such that the angle between the non-parallel sides is of the form $\frac{2 \pi}n$.

\begin{figure}[h]
\centering
\includegraphics[width=5.5cm]{trapezoid.png}
\caption{Isosceles trapezoid with $\angle AOB = \frac {2 \pi} 8$}
\label{trap_def}
\end{figure}

Let $ABQP$ be an isosceles trapezoid with $AB$, $PQ$ as the parallel sides, and $AP$, $BQ$ as the non-parallel sides. Assume without loss of generality that $AB$ is shorter in length than $PQ$. Let $\overline{AB} = 1$ and $\overline{PQ} = \lambda$, where $\lambda \geq \frac {\sqrt 3 + \tan{ \frac \pi n }} {\sqrt 3 - \tan{ \frac \pi n}} $. For brevity, we say $\lambda_v = \frac {\sqrt 3 + \tan{ \frac \pi n }} {\sqrt 3 - \tan{ \frac \pi n}}$. Let $L_{PA}$ and $L_{QB}$ be the lines containing the line segments $PA$ and $QB$ respectively. Also let $O$ be the point of intersection of $L_{PA}$ and $L_{QB}$. Further, let $M$ and $N$ be the midpoints of $AB$ and $PQ$ respectively (as in Figure \ref{trap_def}). As mentioned earlier, $\angle AOB = \frac{2 \pi}n$ where $n \in \mathbb{N}$, $n \ge 4$.



%We show that a particular Steiner topology of $\{A, B, P, Q\}$. We call it the \emph{vertical fork}.

% \vspace{0.2cm}
% \subsection{The MST (for $n = 6$)}
% \vspace{0.2cm}

% For $n = 6$, the minimum spanning tree (MST) $T_{sp} = \{PA, AB, BQ\}$ is a possible Steiner tree ($\forall$ $\lambda > 1)$.

% \begin{figure}[h!]
% \centering
% \includegraphics[width=7cm]{hexagon_trapezoid_sp.png}
% \caption{Minimum Spanning Tree $T_{sp} = \{PA, AB, BQ\}$}
% \end{figure}

% We observe that for $n > 6, \angle PAB = \angle QBA < 120 ^ \circ$. Further, for $n = 6$, we get:
% $$\color{red}|T_{sp}| = \overline{PA} + \overline{AB} + \overline{QB} = 2 \lambda - 1$$

% \vspace{0.2cm}
% \subsection{The Horizontal Fork (for $n > 6$)}
% \vspace{0.2cm}

% For $n > 6$, the minimum spanning tree (MST) no longer remains a valid Steiner tree. We explore other topology, where $P$ \& $A$ are connected to a Steiner Point $S_1$ and $B$ \& $Q$ are connected to another Steiner Point $S_2$ such that $S_1$ and $S_2$ are directly connected. We call the Steiner tree of such a topology, the \emph{Horizontal Fork}. 

% \begin{figure}[h!]
% \centering
% \includegraphics[width=7cm]{horizontal_fork.png}
% \caption{The Horizontal Fork, $T_{hf}$}
% \end{figure}

% Therefore, we have the horizontal fork as $T_{hf} = \{PS_1, AS_1, S_1S_2, S_2B, S_2Q\}$

% \subsubsection{Construction}

% \begin{itemize}
%     \item We construct equilateral triangles $ABE$ and $PQF$ towards the interior of the trapezoid $ABPQ$.
%     \item $S_1 := AE \cap PF$, $S_2 := BE \cap QF$ (These intersections occurs only if $n > 6$)
% \end{itemize}

% \subsubsection{Analysis}

% We observe that the Steiner points $S_1$, $S_2$ are obtainable only if $n > 6$. Further, for the triangles $ABE$ and $PQF$ to intersect, the following condition must hold: \\


% $ \overline{ME} + \overline{FN} \ge \overline{MN} $\\

% $ \implies \dfrac{\sqrt{3}}2 \cdot \overline {PQ} + \dfrac{\sqrt{3}}2 \cdot \overline {PQ} \ge \overline{ON} - \overline{OM}$\\

% $ \implies \dfrac{\sqrt{3}}2 \cdot (\lambda + 1) \ge \dfrac \lambda {2 \tan(\frac \pi n)} - \dfrac 1 {2 \tan(\frac \pi n)}$\\

% $ \implies \lambda \le \dfrac {1 + \sqrt 3 \tan{ \frac \pi n }} {1 - \sqrt 3 \tan{ \frac \pi n}} = \lambda_h \text{ (Say)}$\\


% Now, $\overline{AB} = \overline{AE} = \overline{BE} = 1$ and $\overline{PQ} = \overline{QF} = \overline{FP} = \lambda$. Further we observe that $ES_1S_2$ and $FS_1S_2$ are equilateral triangles. Let $\overline{ES_1} = \overline{S_1F} = \overline{FS_2} = \overline{S_2E} = \overline{S_1S_2} = \delta$. Now,\\

% $ \overline{EF} + \overline{MN} = \overline{NF} + \overline{ME}$\\

% $ \implies \sqrt 3 \cdot \delta + \dfrac{\lambda - 1}{2 \tan \frac \pi n} = \dfrac{\sqrt 3}2 (\lambda + 1)$\\

% $ \implies -3 \cdot \delta = \dfrac{\sqrt{3} (\lambda - 1)}{2 \tan \frac \pi n} - \dfrac{3}2 (\lambda + 1)$

% \noindent Again, \\


% \noindent $|T_{hf}| = \overline{PS_1} + \overline{QS_2} + \overline{S_1S_2} + \overline{S_1A} + \overline{S_2B}$\\

% $ = (\lambda - \delta) + (\lambda - \delta) + \delta + (1 - \delta) + (1 - \delta) = 2 (\lambda + 1) - 3 \cdot \delta  = 2 (\lambda + 1) + \dfrac{\sqrt{3} (\lambda - 1)}{2 \tan \frac \pi n} - \dfrac{3}2 (\lambda + 1)$\\


% \noindent Therefore we get : 
% $$\color{blue}|T_{hf}| = \dfrac{\lambda + 1}2 + \dfrac{\sqrt{3} (\lambda - 1)}{2 \tan \frac \pi n}$$

% \noindent for any $n > 6$ and $\lambda \le \lambda_h$

%\vspace{0.2cm}
%\subsection{The Vertical Fork}
%\vspace{0.2cm}

Now, we explore the following Steiner topology of the terminal set $\{A, B, P, Q\}$:
\begin{enumerate}
\item $A$ and $B$ are connected to a Steiner point $S_1$.
\item $P$ and $Q$ are connected to another Steiner point $S_2$. 
\item $S_1$ and $S_2$ are directly connected (Please see Figure \ref{T_vf}). 
\end{enumerate}
We call such a topology a \emph{vertical fork topology} and the Steiner tree realising such a topology, the \emph{vertical fork}. Note that in a vertical fork topology the only unknowns are the locations of the two Steiner points $S_1,S_2$. Therefore, we have the vertical fork topology as $T_{vf}$, with $E(T_{vf}) = \{AS_1, BS_1, S_1S_2, S_2P, S_2Q\}$.


\begin{figure}[h]
\centering
\includegraphics[width=5.5cm]{vertical_fork.png}
\caption{The Vertical Fork, $\mathcal T_{vf}$}
\label{T_vf}
\end{figure}

%\subsubsection{Construction}


%\subsubsection{Analysis}

We show the existence of a vertical fork and calculate its total length in the following lemma.

\begin{lemma} \label{lambda_v}
A vertical fork $\mathcal T_{vf}$ can be constructed for any $n \ge 4$ and for any $\lambda \ge \lambda_v$, where

$$\lambda_v = \frac {\sqrt 3 + \tan{ \frac \pi n }} {\sqrt 3 - \tan{ \frac \pi n}}$$    
such that the length of the vertical fork   
$$|\mathcal T_{vf}| = \dfrac{(\lambda - 1)}{2 \tan \frac \pi n} + \dfrac {\sqrt 3 (\lambda + 1)} {2}$$

\end{lemma}

\begin{proof}
First, we construct the Steiner points $S_1$, $S_2$ and then prove that the construction works.

In the following construction, we describe how to find the locations of $S_1$ and $S_2$ for  the vertical fork:
\begin{itemize}
    \item We construct equilateral triangles $ABE$ and $PQF$ where both points $E$ and $F$ lie outside the trapezoid $ABQP$.
    \item We construct the circumcircles $(ABE)$ and $(PQF)$ of $ABE$ and $PQF$, respectively. 
    \item Recall that $L_{MN}$ is the line segment containing $M$ and $N$. Define $S_1$ to be the point of intersection of $L_{MN}$ and the circle $(ABE)$ distinct from $E$; similarly, $S_2$ is the point of intersection of the $L_{MN}$ and $(PQF)$ distinct from $F$. Therefore, by construction, $S_2$, $M$ must lie on the same side of $N$ on $L_{MN}$, and $S_1$, $N$ must lie on the same side of $M$ on $L_{MN}$. Further, $\angle AS_1B = \angle PS_2Q = \frac{2 \pi}{3}$ by construction.
\end{itemize}

We now show that the points $S_1$ and $S_2$ indeed lie inside the line segment $MN$ and the points appear in the order: $M$, $S_1$, $S_2$, $N$. We prove the following claim to serve this purpose.

\begin{claim} \label{S1_S2_in_MN}
$\overline{S_1M} + \overline{S_2N} \le \overline{MN}$
% \begin{enumerate}
%     \item  (As $S_2$, $M$ lie on the same side of $N$ on $L_{MN}$ and $S_1$, $N$ lie on the same side of $M$ on $L_{MN}$, this implies that $S_1$, $S_2$ lie on the line segment $MN$).
% \end{enumerate}

\end{claim}
\begin{proof}
We have $\overline{MN} = \overline {ON} - \overline{OM} = \dfrac{\lambda}{2 \tan \frac{\pi}{n}} - \dfrac{1}{2 \tan \frac{\pi}{n}} = \dfrac{(\lambda - 1)}{2 \tan \frac \pi n}$\\
Again $\overline{S_1M} = \dfrac{1}{2\sqrt{3}}$ and $\overline{S_2N} = \dfrac{\lambda}{2\sqrt{3}}$.\\
Therefore we have 
\begin{align*}
   & \overline{MN} - (\overline{S_1M} + \overline{S_2M}) \\
 = & \dfrac{(\lambda - 1)}{2 \tan \frac \pi n} - \dfrac 1 {\sqrt{3}} - \dfrac \lambda {2\sqrt 3}\\
 = & \dfrac{(\lambda - 1)}{2 \tan \frac \pi n} - \dfrac {(\lambda + 1)} {2\sqrt{3}}\\
 = & \frac{\lambda(\sqrt{3} - \tan \frac{\pi}{n}) - (\sqrt{3} + \tan \frac{\pi}{n})}{2 \sqrt{3} \tan \frac \pi n}\\
 = & \frac{\sqrt{3} - \tan \frac{\pi}{n}}{2 \sqrt{3} \tan \frac \pi n} \cdot (\lambda - \lambda_v)
\end{align*}

Therefore $\lambda \ge \lambda_v$ implies $\overline{MN} \ge (\overline{S_1M} + \overline{S_2M})$. This proves~\Cref{S1_S2_in_MN}.
\end{proof}

From~\Cref{S1_S2_in_MN}, we get $\overline{S_1M}, \overline{S_2N} \le \overline{MN}$. As $S_2$, $M$ lie on the same side of $N$ on $L_{MN}$ and $S_1$, $N$ lie on the same side of $M$ on $L_{MN}$, this implies that $S_1$, $S_2$ lie on the line segment $MN$. Further, $\overline{S_1M} + \overline{S_2N} \le \overline{MN}$ implies $\overline{S_1M} \le \overline {S_2M} \le \overline{NM}$, which in turn implies that the points appear in the order: $M$, $S_1$, $S_2$. $N$.\\

Now, we calculate the total length of the vertical fork, $|\mathcal T_vf|$:
\begin{align*}
  & |\mathcal T_{vf}|\\
= & \overline{AS_1} + \overline{BS_1} + \overline{S_1S_2} + \overline{PS_2} + \overline{QS_2}\\
= & 2 \overline{PS_2} + 2 \overline{AS_1} + \overline{S_1S_2}\\
= & \dfrac{2 \lambda} {\sqrt 3} + \dfrac 2 {\sqrt 3} + \bigg( \dfrac{(\lambda - 1)}{2 \tan \frac \pi n} - \dfrac {(\lambda + 1)} {2\sqrt{3}} \bigg)\\
= & \dfrac{(\lambda - 1)}{2 \tan \frac \pi n} + \dfrac {\sqrt 3 (\lambda + 1)} {2}
\end{align*}
% We observe that, 

% \noindent $\overline{S_1S_2} = \overline{MN} - \overline{MS_1} - \overline{NS_2}$\\

% $ = \dfrac{(\lambda - 1)}{2 \tan \frac \pi n} - \dfrac 1 {\sqrt{3}} - \dfrac \lambda {\sqrt 3}$\\

% $ = \dfrac{(\lambda - 1)}{2 \tan \frac \pi n} - \dfrac {(\lambda + 1)} {\sqrt{3}}$\\

%  Again,\\

% \noindent $|T_{vf}| = \overline{AS_1} + \overline{BS_1} + \overline{S_1S_2} + \overline{PS_2} + \overline{QS_2} = 2 \overline{PS_2} + 2 \overline{AS_1} + \overline{S_1S_2}$\\

% $ = \dfrac{2 \lambda} {\sqrt 3} + \dfrac 2 {\sqrt 3} + \bigg( \dfrac{(\lambda - 1)}{2 \tan \frac \pi n} - \dfrac {(\lambda + 1)} {\sqrt{3}} \bigg)$\\

% $ = \dfrac{(\lambda - 1)}{2 \tan \frac \pi n} + \dfrac {\sqrt 3 (\lambda + 1)} {\sqrt{2}}$\\

% Also, for the construction to be possible we must have: \\

% $ \overline{S_1S_2} \ge 0 $\\

% \noindent $\implies \bigg( \dfrac{(\lambda - 1)}{2 \tan \frac \pi n} - \dfrac {(\lambda + 1)} {\sqrt{3}} \bigg) \ge 0 $\\

% \noindent $\implies \lambda \ge \dfrac {\sqrt 3 + \tan{ \frac \pi n }} {\sqrt 3 - \tan{ \frac \pi n}} = \lambda_v$\\

% Thus, we are done.

This completes the proof of~\Cref{lambda_v}.
\end{proof}




% \vspace{0.2cm}
% \subsection{Comparing various Topologies}
% \vspace{0.2cm}

% \subsubsection{The horizontal fork \textit{vs} The vertical fork ($n > 6$)} \label{HFvsVF}

% Let for some $\lambda$, $\lambda_v \le \lambda \le \lambda_h$, the \emph{vertical fork} ``is better than" (has a smaller total length than) the \emph{horizontal fork}, for some $n > 6$.\\ 

% $ \Longleftrightarrow |T_{vf}| \le |T_{hf}|$\\

% $ \Longleftrightarrow \dfrac{(\lambda - 1)}{2 \tan \frac \pi n} + \dfrac {\sqrt 3 (\lambda + 1)} {2} \le \dfrac{\lambda + 1}2 + \dfrac{\sqrt{3} (\lambda - 1)}{2 \tan \frac \pi n}$\\

% $ \Longleftrightarrow \dfrac{(\lambda - 1)}{\tan \frac \pi n} \cdot \dfrac{\sqrt 3 - 1} 2 \ge (\lambda + 1) \cdot \dfrac{\sqrt 3 - 1} 2 $\\

% $ \Longleftrightarrow (\lambda - 1) \ge (\lambda + 1) \cdot \tan \frac \pi n$\\

% $ \Longleftrightarrow \lambda \ge \dfrac{(1 + \tan \frac \pi n)} {1 - \tan \frac \pi n} = \tan( \frac \pi 4 + \frac \pi n) = \lambda_s \text{ (Say)}$


% \subsubsection{The Vertical Fork \textit{vs} The MST ($n = 6$)} \label{VFvsMST}

% Let for some $\lambda$, $\lambda_v \le \lambda \le \lambda_h$, the \emph{vertical fork} ``is better than" (has a smaller total length than) the \emph{horizontal fork}, for $n = 6$.\\ 


% $ \Longleftrightarrow |T_{vf}| \le |T_{sp}|$\\

% $ \Longleftrightarrow \dfrac{(\lambda - 1)}{2 \tan \frac \pi n} + \dfrac {\sqrt 3 (\lambda + 1)} {2} \le 2 \lambda - 1$\\

% $ \Longleftrightarrow \dfrac {\sqrt 3 (\lambda - 1)} {2} + \dfrac {\sqrt 3 (\lambda + 1)} {2} \le 2 \lambda - 1 \text{ , [} \because n = 6 \text{]} $ \\

% $ \Longleftrightarrow \lambda \ge \dfrac 1 {2 - \sqrt{3}} = 2 + \sqrt 3 = \tan( \frac \pi 4 + \frac \pi 6) = \lambda_s$\\

% \begin{remark} The MST for $n = 6$ is the degenerate limit of of the \emph{horizontal fork} for $n \rightarrow 6$.
% \end{remark}

% Combining this with the result from \ref{HFvsVF}, we obtain that for all $n \ge 6$, the \emph{Vertical Fork} is better than the other construction (\emph{Horizontal Fork} for $n > 6$ and MST for $n = 6$) for $\lambda \ge \lambda_s$.

% \subsubsection{Combining the Results}

% First of all, for \ref{HFvsVF} and \ref{VFvsMST}, to be valid, we must have $\lambda_v \le \lambda_s \le \lambda_h$ for all $n > 6$, which is true because $f(x) = \dfrac {1 + x}{1 - x}$ is an increasing function in $(0, 1)$, where we have $\lambda_v = f \Big(\dfrac 1 {\sqrt 3} \cdot \dfrac \pi n \Big)$, $\lambda_s = f \Big(1 \cdot \dfrac \pi n \Big)$ and $\lambda_h = f \Big(\sqrt 3 \cdot \dfrac \pi n \Big)$.


% \begin{figure}[h!]
% \centering
% \includegraphics[width=13cm]{Trapezoid_data.png}
% \caption{ $\lambda_v$, $\lambda_s$ and $\lambda_h$ plotted against $n$ }
% \end{figure}


\subsection{\ESMT for $2$-Concentric Parallel Regular $3$-gons} \label{triangles}

Note that a regular $3$-gon is an equilateral triangle and therefore, for the rest of this section we call a regular $3$-gon as an equilateral triangle. We describe a minimal solution for \ESMT for $2$-CPR equilateral triangles.

\begin{lemma}\label{3-gon}
Consider two concentric and parallel equilateral triangles $A_1A_2A_3$ and $B_1B_2B_3$, where $B_1B_2B_3$ has side length $\lambda > 1$ and $A_1A_2A_3$ has side length $1$. An SMT of $\{A_1, A_2, A_3, B_1, B_2, B_3\}$ is an SMT of $\{B_1,B_2,B_3\}$, and has length $\sqrt 3 \cdot \lambda$. 
\end{lemma}

\begin{proof}
It is to be noted that the centre $O$ of both $A_1A_2A_3$ and $B_1B_2B_3$ is also the Torricelli point of both $A_1A_2A_3$ and $B_1B_2B_3$~\cite{du1987steiner}. On taking $O$ as the only Steiner point, the SMT for $\{B_1,B_2,B_3\}$ is $ \mathcal T_3 = \{OB_1, OB_2, OB_3\}$~\cite{du1987steiner}. However, the edges of $\mathcal T_3$ already pass through $A_1$, $A_2$ and $A_3$. Therefore, $ \mathcal T_3$ with $E(\mathcal T_3) = \{OA_1, OA_2, OA_3, A_1B_1, A_2B_2, A_3B_3\}$ is also the SMT for $\{A_1, A_2, A_3, B_1, B_2, B_3\}$ as shown in Figure \ref{conc_eq_tri}.

 From the definition of $\mathcal T_3$, we have the length of the \smtpoly, for $n = 3$ as 
$$ | \mathcal T_3 | = \sqrt 3 \cdot \lambda $$ 
\end{proof}

\begin{figure}[h]
\centering
\includegraphics[width=4cm]{3_gon_topo1.png}
\caption{SMT for $2$-CPR $3$-gons }
\label{conc_eq_tri}
\end{figure}



\subsection{\ESMT and Large Polygons with Large Aspect Ratios} \label{final_proofs}


In this section, we consider the \ESMT problem when the terminal set is formed by the vertices of $2$-CPR $n$-gons, namely $\{A_i\}$ and $\{B_i\}$. As mentioned earlier, $\{A_i\}$ is the inner polygon and $\{B_i\}$ is the outer polygon of this set of $2$-CPR $n$-gons. In particular, we consider the case when $n \geq 13$; for $n \leq 12$ these are constant sized input instances and can be solved using any brute-force technique. We also require that the aspect ratio 
$\lambda$ has a lower bound $\lambda_1$, i.e. we do not want the two polygons to have sides of very similar length. The exact value of  $\lambda_1$ will be clear during the description of the algorithm. Intuitively, when $\lambda$ is \emph{very large}, the SMT should look similar to what was derived in~\cite{weng1995steiner}. In other words, (please refer to Figure \ref{sing_con_top_fig}):
\begin{enumerate}
    \item for some $j \in [n]$, there is a  vertical fork connecting the two consecutive inner polygon points $A_j, A_{j + 1}$  with the two consecutive outer polygon points $B_j, B_{j + 1}$ - we refer to this vertical fork as the \emph{vertical gadget} for the SMT,
    \item the other points in $\{B_i\}$ are connected directly via $(n - 2)$ outer polygon edges,
    \item the other points in $\{A_i\}$ are connected via $(n - 2)$ inner polygon edges. 
\end{enumerate}

We call such a topology, a \emph{singly connected topology} as in Figure \ref{sing_con_top_fig}. 
%(generated using \href{www.geosteiner.com}{Geosteiner}). 
For the rest of this section, we consider the SMTs for a large enough aspect ratio, $\lambda$ and show that there is an SMT that must be a realisation of a \emph{singly connected topology}. We refer to an SMT for the terminal set defined by the vertices of $\{A_i\}$ and $\{B_i\}$ as the \smtpoly. 

Without loss of generality, we consider the edge length of any side $A_iA_{i+1}$ in $\{A_i\}$ to be $1$. As we defined the aspect ratio to be $\lambda$, any side $B_iB_{i+1}$ of $\{B_i\}$ must have a side length of $\lambda$. Further, we observe that for any SMT $\mathcal T$, specifying $E(\mathcal T)$ sufficiently determines the entire tree, as $V(\mathcal T) = \{P~|~\exists Q \text{ such that } PQ \in E(\mathcal T)\}$.

We start with the following formal definitions:

\begin{definition}\label{def:singly-conn}
A Steiner topology of $\{A_i\} \cup \{B_i\}$ is a \textbf{singly connected topology}, if it has the following structure:

\begin{itemize}
    \item A vertical gadget i.e.  five edges $\{A_jS_a, A_{j + 1}S_a, S_aS_b, S_bB_j, S_bB_{j + 1}\}$ for some $1 \le j \le n$, where $S_a$ and $S_b$ are newly introduced Steiner points contained in the isosceles trapezoid $\{A_j, A_{j + 1}, B_j, B_{j + 1}\}$. 
    \item All $(n - 2)$ polygon edges of $\{A_i\}$ excluding the edge $A_jA_{j + 1}$
    \item All $(n - 2)$ polygon edges of $\{B_i\}$ excluding the edge $B_jB_{j + 1}$
\end{itemize} 
\end{definition}


\begin{figure}[h]
\centering
\subfloat[\centering SMT of $2$-CPR $13$-gons]{\includegraphics[width=4cm]{13_sing.png}}
    \qquad
\subfloat[\centering SMT of $2$-CPR $20$-gons] {\includegraphics[width=4cm]{20_sing.png} }
\caption{ \emph{Singly connected topology} of $2$-CPR $n$-gons $n = 13$ and $n = 20$ }
\label{sing_con_top_fig}
\end{figure}

We define the notion of a path in an SMT for the vertices of $\{A_i\}$ and $\{B_i\}$ where the starting point is in $\{A_i\}$ and the ending point is in $\{B_i\}$.
\begin{definition}
    An \textbf{A-B path} is a path in a Steiner tree of $\{A_i\} \cup \{B_i\}$ which starts from a vertex in $\{A_i\}$ and ends at a vertex in $\{B_i\}$ with all intermediate nodes (if any) being Steiner points. 
\end{definition}

The following Definition and Figure~\ref{fig:counterpath} is useful for the design of our algorithm.
\begin{definition}\label{def:cw_ccw_path}
    A \textbf{counter-clockwise path} is a path $P_1, P_2, ... P_m$ in a Steiner tree such that for all $i \in \{2, \ldots, m - 1\}, \angle P_{i - 1}P_{i}P_{i + 1} = \frac{2 \pi}{3}$ in the counter-clockwise direction. Similarly, a \textbf{clockwise path} is a path $P_1, P_2, ... P_m$ in a Steiner tree such that for all $i \in \{2, \ldots, m - 1\}, \angle P_{i - 1}P_{i}P_{i + 1} = \frac{4 \pi}{3}$ in the counter-clockwise direction.
\end{definition}


\begin{figure}[h]
\centering
\includegraphics[width=7cm]{CW_CCW_def.png}
\caption{$\angle P_{i-1}P_iP_{i+1} = \alpha$ in the counter-clockwise direction and $\angle P_{i-1}P_iP_{i+1} = \beta$ in the clockwise direction, as in~\Cref{def:cw_ccw_path}}
\label{fig:counterpath}
\end{figure}

Now, we consider any Steiner point $S$ in any SMT. Let $P$ and $Q$ be two neighbours of $S$. We now prove that there is no point of the SMT inside the triangle $PSQ$.

\begin{observation} \label{no_pt_in_PSQ}
    Let $S$ be a Steiner point in any SMT for $\{A_i\} \cup \{B_i\}$, with neighbours $P$ and $Q$. Then, no point of the SMT lies inside the triangle $PSQ$.
\end{observation}

\begin{proof}
    By the \emph{lune property} (Proposition \ref{lune}), for any edge $P_1Q_1$ in an SMT, for the two circles centred at $P_1$ and at $Q_1$, respectively and both having a radius of $\overline{P_1Q_1}$, the intersection region does not contain any point of the SMT.

    \begin{figure}[h]
    \centering
    \includegraphics[width=10cm]{no_pt_in_PSQ.png}
    \caption{Observation \ref{no_pt_in_PSQ}: Given triangle $PSQ$, equilateral triangles $PSE$ and $QSF$ are constructed}
    \label{no_pt_in_PSQ_fig}
    \end{figure}

    Let $E$ and $F$ be points on the internal angle bisector of $\angle PSQ$, such that $\angle SPE = \angle SQF = \frac{\pi}{3}$ as shown in Figure \ref{no_pt_in_PSQ_fig}. Since $E$ and $F$ are points on the angle bisector of $\angle PSQ$, $\angle PSE = \angle QSF = \frac{\pi}{3}$. Hence, triangles $PSE$ and $QSF$ are equilateral triangles.

    Since $PS$ is an edge in the SMT, by the lune property, the intersection of the circles centred at $P$ and $S$, both with radius $\overline{PS}$ contain no point inside which is a part of the SMT. Since the lune contains the entire equilateral triangle $PSE$, no point of the SMT lies inside triangle $PSE$. Similarly, no point of the SMT lies inside the triangle $QSF$.

    Further, as $\angle PSQ = \frac{2 \pi}{3}$, $\angle SPQ + \angle SQP = \frac{\pi}{3}$. This means $\angle SPQ, \angle SQP < \frac{\pi}{3}$. Therefore, as $\angle SPE = \angle SQF = \frac{\pi}{3}$, $E$ and $F$ must lie outside the triangle $PSQ$. This implies that the triangle $PSQ$ is covered by the union of the triangles $PSE$ and $QSF$. As no point of the SMT lies in triangles $PSE$ and $QSF$, triangle $PSQ$ must contain no points of the SMT as well.
\end{proof}


Next, we show that in an \smtpoly there cannot be any Steiner point, in the interior of the polygon $\{A_i\}$, that is a direct neighbour of some point $B_k$ in the polygon $\{B_i\}$.

\begin{observation} \label{no_in_to_B}
    For any \smtpoly, there cannot exist a Steiner point $S$ lying in the interior of the polygon $\{A_i\}$ such that $SB_k$ is an edge in an SMT for some $B_k \in \{B_i\}$. 
\end{observation}

\begin{proof}
    For the sake of contradiction, we assume that for some \smtpoly there exists a Steiner point $S$ lying in the interior of the polygon $\{A_i\}$ such that $SB_k$ is an edge in the SMT for some $B_k \in \{B_i\}$. Let $A_mA_{m+1}$ be the edge such that $SB_k$ intersects $A_mA_{m+1}$. Without loss of generality, assume that $A_m$ is closer to $B_k$ than $A_{m+1}$. Therefore $\angle B_kA_mS > \angle B_kA_mA_{m+1} \ge \frac{\pi}{2} + \frac{\pi}{n} > \frac{\pi}{2}$. This means that $B_kS$ is the longest edge in the triangle $B_kSA_m$. Therefore we can remove the edge $B_kS$ from the SMT and replace it with either $B_kA_m$ or $SA_m$ to get another tree connecting the terminal set with a shorter total length than what we started with, which is a contradiction. 
\end{proof}

We further analyze SMTs for $\{A_i\}\cup \{B_i\}$.

\begin{observation} \label{subchain-smt}
    Let $\mathcal V = \{A_j, A_{j + 1}, \ldots, A_k\}$ be the interval of consecutive vertices of $\{A_i\}$ lying between $A_j$ and $A_k$ (which includes $A_{j + 1}$) such that $A_j$ is distinct from $A_{k + 1}$. Let $U$ be any point on the line segment $A_kA_{k+1}$. Then an SMT of $\mathcal V \cup \{U\}$ is $\mathcal T$, with $E(\mathcal T) = \{A_jA_{j+1}, A_{j+1}A_{j+2}, \ldots, A_{k-1}A_k\} \cup \{A_kU\}$. 
\end{observation}
\begin{proof}
    For the sake of contradiction, assume that there exists an SMT $\mathcal T'$ of $\mathcal V \cup \{U\}$ such that $|\mathcal T'| < |\mathcal T|$.  
    
    From \cite{du1987steiner}, we know that $\mathcal T_A$, with $E(\mathcal T_A) = \{A_jA_{j+1}, A_{j+1}A_{j+2}, \ldots, A_{j-2}A_{j-1}\}$ (\textit{i.e.} all edges of polygon $\{A_i\}$ except $A_{j-1}A_j$) is an SMT of $\{A_i\}$. Since $U \in A_kA_{k+1} \in E(\mathcal T_A)$, $\mathcal T_A$ must also be an SMT of $\{A_i\} \cup \{U\}$. However, $\mathcal T_A$ can be partitioned as $\mathcal T_A = \mathcal T \uplus \mathcal T_1$, where $E(\mathcal T_1) = \{UA_{k+1}\} \cup \{A_{k+1}A_{k+2}, \ldots, A_{j-2}A_{j-1}\}$.
    However, as $\mathcal T'$ is assumed to be of shorter total length than $\mathcal T$, $\mathcal T' \cup \mathcal T_1$ is a tree, containing $\{A_i\}$ as a vertex subset, which has a shorter total length than $\mathcal T_A$, contradicting the optimality of $\mathcal T_A$.
\end{proof}


We proceed by showing that in any \smtpoly there exists at least one A-B path which is also a counter-clockwise path. Symmetrically, we also show that for any \smtpoly there exists another clockwise A-B path which consists of only clockwise turns. We can intuitively see that this is true because, if all clockwise paths starting at a vertex in $\{A_i\}$ also ended in a vertex in $\{A_i\}$, there would be enough paths to form a cycle, which is not possible in a tree.

\begin{lemma}\label{left_right_turn_path}
In any \smtpoly, there exists an A-B path which is also a clockwise path and there exists an A-B path which is also a counter-clockwise path.
\end{lemma}
\begin{proof}
    For the sake of contradiction, assume that for some \smtpoly there is no A-B path which is a counter-clockwise path. We pick an arbitrary vertex $A_{i_1} \in \{A_i\}$ such that it is connected to at least one Steiner point (say $S_{i_1}$). We consider the counter-clockwise path, $\mathcal C_1$ starting from $A_{i_1}S_{i_1}$ and ending at the first terminal point in the counter-clockwise path. By assumption, there can be no vertex of $\{B_i\}$ in $\mathcal C_1$, hence the endpoint must be a vertex in $\{A_i\}$. Let $A_{i_2}$ be the other endpoint of $\mathcal C_1$. By definition, the penultimate vertex in this counter-clockwise path must be a Steiner point, we call it $S_{i_2}$. We again consider the counter-clockwise path starting from $A_{i_2}S_{i_2}$, and similarly, let $A_{i_3}$ be the first terminal that is encountered in this path. The penultimate vertex in this counter-clockwise path must be a Steiner point, we call it $S_{i_3}$. We can repeat this procedure indefinitely to obtain $A_{i_4}, A_{i_5}, A_{i_6}, \ldots $ as there are no counter-clockwise A-B paths. However, as $\{A_i\}$ has $n$ vertices, there must be a repetition of vertices among   $A_{i_1}, A_{i_2}, A_{i_3}, \ldots, A_{i_{n+1}}$, implying the existence of a cycle in the SMT, which is a contradiction.

    This symmetrically implies that there must also be a clockwise A-B path. 
\end{proof}

Our next step is to bound the number of `connections' that connect the inner polygon $\{A_i\}$ and the outer polygon $\{B_i\}$ for a large aspect ratio, $\lambda$. As $\lambda$ increases, the area of the annular region between the two polygons increases as well. Therefore, an increase in the number of connections would lead to a longer total length of the SMT considered. Consequently, we will prove that after a certain positive constant $\lambda_1$, for $\lambda > \lambda_1$ any \smtpoly will have a single `connection' between the two polygons. Moreover, \cite{weng1995steiner} gives us an evidence that as $\lambda \rightarrow \infty$, there will indeed be a single connection connecting the outer polygon and the inner polygon for $n \ge 12$. We can formalize this notion of existence of a single `connection' with the following lemma.\\

\begin{lemma} \label{mincut_1}
    For any \smtpoly with $n \ge 13$ and $\lambda > \lambda_{1}$, the number of edges needed to be removed in order to disconnect $\{A_i\}$ and $\{B_i\}$ is 1, where 
    $$
    % \lambda_{1} = \frac{2 \sin{\frac{\pi}{n}} + 1}{1 - 6 \sin \frac{\pi}{n}}
    \lambda_{1} = \frac{1}{1 - 4 \sin \frac{\pi}{n}}
    $$
\end{lemma}

\begin{proof}
        For the sake of contradiction, assume that for some \smtpoly, there are at least two distinct edges in that SMT, which are needed to be removed in order to disconnect $\{A_i\}$ and $\{B_i\}$. We start with a claim.
        
        \begin{claim} \label{more_than_lambda}
        A counter-clockwise A-B path in any SMT of $\{A_i\} \cup \{B_i\}$ must have an edge of length greater than $\lambda$.
        \end{claim}
        
\begin{proof}
        We consider a generic setting, where $\mathcal T$ is an SMT of some set of terminal points $\mathcal P$. Let $H \in V(\mathcal T)$ be vertex of $\mathcal T$. If $\mathcal C$ be a counter-clockwise path starting from $H$ such that no edge in the counter clockwise path has a length of more than $r$, for some $r \in \mathbb{R}^{+}$. Due to Lemma 2.4 (1) of \cite{weng1995steiner},  we know that $\mathcal C$ is contained entirely in the circle centred at $H$ with radius $2r$.
        
        In our case, any vertex in $\{A_i\}$ and any vertex in $\{B_i\}$ are separated by the distance of at least $\dfrac{\lambda - 1}{2 \sin{\frac{\pi}{n}}}$. Therefore, by the above fact, the maximum edge length in a counter-clockwise A-B path of any SMT of $\{A_i\} \cup \{B_i\}$ must be at least $\dfrac{\lambda - 1}{4 \sin{\frac{\pi}{n}}}$. Moreover, we have $\lambda > \lambda_1$. Therefore $\lambda > \dfrac{1}{1 - 4 \sin{\frac{\pi}{n}}} \implies \dfrac{\lambda - 1}{4 \sin{\frac{\pi}{n}}} > \lambda$. Hence, a counter-clockwise A-B path in any SMT of $\{A_i\} \cup \{B_i\}$ must have one edge greater than $\lambda$. This proves the claim.
\end{proof}
        
        Now, for any SMT of $\{A_i\} \cup \{B_i\}$, let $\mathcal C$ be a counter-clockwise A-B path (this exists due to Lemma \ref{left_right_turn_path}). From Claim \ref{more_than_lambda}, we know that there is an edge $e$ in $\mathcal C$, with a length greater than $\lambda$. On removing the edge $e$, the SMT splits into a forest of two trees. Let the trees be $\mathcal T_x$ and $\mathcal T_y$. As we assumed that there are two edges required to disconnect $\{A_i\}$ and $\{B_i\}$, there must exist an A-B path in either $\mathcal T_x$ or $\mathcal T_y$. Without loss of generality, let $\mathcal T_x$ contain an A-B path, and hence $\mathcal T_x$ contains at least one point from $\{A_i\}$ and at least one point from $\{B_i\}$. Further, $\mathcal T_y$ must contain at least one terminal point (as it must contain all terminal points in one of the sides of the removed edge $e$). If $\mathcal T_y$ contains a point from $\{A_i\}$, then the polygon $\{A_i\}$ has vertices both from $\mathcal T_x$ and $\mathcal T_y$; otherwise, if $\mathcal T_y$ contains a point from $\{B_i\}$, then the polygon $\{B_i\}$ has vertices both from $\mathcal T_x$ and $\mathcal T_y$. 
        
        This means that either the polygon $\{A_i\}$ or the polygon $\{B_i\}$ will contain at least one node from each of $\mathcal T_x$ and $\mathcal T_y$. Further, as any given vertex must be either in $\mathcal T_x$ or in $\mathcal T_y$, either $\{A_i\}$ or $\{B_i\}$ must contain two consecutive vertices $U_i$ and $U_{i + 1}$ such that one of them is in $\mathcal T_x$ and the other is in $\mathcal T_y$. We simply connect $U_i$ and $U_{i + 1}$ by the polygon edge which is of length $1$ (if $U_i, U_{i + 1} \in \{A_i\}$) or of length $\lambda$ (if $U_i, U_{i + 1} \in \{B_i\}$), giving us back a tree $\mathcal{T}'$containing all the terminals. However we discarded an edge of length greater than $\lambda$ and added back an edge of length at most $\lambda$ in this process, which means that the total length of $\mathcal{T}'$ is strictly less than the SMT we started with. This is a contradiction.
\end{proof}

% \begin{remark}
%     We can find an even tighter bound $\lambda_{0} = \dfrac{1 + (2 - \sqrt{3}) \cdot \tan{\frac{\pi}{n}}}{1 - (4 - \sqrt{3}) \cdot \tan{\frac{\pi}{n}}}$, with $\lambda_1 > \lambda_0 > \lambda_s > \lambda_v$ such that whenever $\lambda \ge \lambda_0$, the SMT follows the singly connected topology, and whenever $\lambda < \lambda_0$, we use multiple vertical gadgets instead, $\forall n \ge 12$.
% \end{remark}

We now proceed to further investigate the connectivity of $\{A_i\}$ and $\{B_i\}$. 
%We start with the portion of this connection that is closer to the polygon $\{A_i\}$.\\

\begin{lemma} \label{steiner_path_from_A_to_A}
    Consider an \smtpoly for $n \ge 13$ and  $\lambda \ge \lambda_{1}$. There must exist $j \in [n]$ and a Steiner point $S_1$, such that terminals $A_j, A_{j + 1}$ form a path $A_j$, $S_1$, $A_{j+1}$ in the SMT and each A-B path passes through $S_1$; where 
    
    $$\lambda_{1} = \frac{1}{1 - 4 \sin \frac{\pi}{n}}$$
\end{lemma}

\begin{proof}
    From Lemma \ref{left_right_turn_path}, we know that there exists one clockwise A-B path and one counterclockwise A-B path in any SMT of $\{A_i\} \cup \{B_i\}$. Let a clockwise A-B path start from $A_r$ and a counter-clockwise A-B path start from $A_l$. Further following from Lemma~\ref{mincut_1}, as there is one edge common to all A-B paths, the clockwise A-B path from $A_r$ and the counter-clockwise A-B path from $A_l$ must share a common edge $S_1S_2$. Therefore, each A-B path must pass through $S_1$ and $S_2$. Without loss of generality we assume that point $S_1$ is closer to the polygon $\{A_i\}$ than $S_2$. This means $S_1$ is either a Stiener point or a terminal vertex of $\{A_i\}$.

    \begin{claim} \label{S_1_notin_A}
        $S_1$ is not a vertex in $\{A_i\}$
    \end{claim}
\begin{proof}
    For the sake of contradiction, we assume that $S_1$
     to be a vertex in $\{A_i\}$, let $S_1 = A_k$ in some SMT $\mathcal T_0$. We disconnect the edge $S_1S_2$ from $\mathcal T_0$, which results in the formation of a forest of two trees $\mathcal T_x$ and $\mathcal T_y$ such that $S_1 = A_k \in \mathcal T_x$ and $S_2 \in \mathcal T_y$.

    $\mathcal T_x$ must contain all vertices of $\{A_i\}$ and $\mathcal T_y$ must contain all vertices of $\{B_i\}$, as there would be an A-B path in the graph otherwise (contradicting that $S_1S_2$ disconnects $\{A_i\}$ and $\{B_i\}$). We replace $\mathcal T_x$ with the SMT of $\{A_i\}$, which is also an MST (from \cite{weng1995steiner}). Since all MST's are of the same length, we choose such an MST in which $A_k$ is not a leaf node. This means $A_{k-1}A_k$ and $A_kA_{k+1}$ are edges in the chosen MST of $\{A_i\}$. We now add back the edge $A_kS_2$, resulting in a connected tree $\mathcal T_0'$ of $\{A_i\} \cup \{B_i\}$. Since we replaced the tree $\mathcal T_x$ with an SMT of $\{A_i\}$, the total length of the $\mathcal T_0'$ must not be more than the total length of the $\mathcal T_0$.

    However, we observe that $A_k$ has three neighbours in $\mathcal T_0'$, which are $A_{k+1}, A_{k-1}, S_2$. However $\angle A_{k-1}A_kA_{k+1} > \frac{2 \pi}{3}$. This means either $\angle S_2A_kA_{k+1} < \frac{2 \pi}{3}$ or $\angle A_{k-1}A_kS_2 < \frac{2 \pi}{3}$. But due to Proposition \ref{smt-prop}, this cannot form an SMT. Therefore $\mathcal T_0'$ is not optimal; and hence, $\mathcal T_0$ cannot be optimal as well. This proves  the claim.
    \end{proof}

    Therefore, $S_1$ must be a Steiner point. Let $P$ and $Q$ be the neighbours of $S_1$ other than $S_2$, such that $\angle PS_1S_2$ is a clockwise turn while $\angle QS_1S_2$ is a counter-clockwise turn. This means that the clockwise A-B path from $A_r$ passes through $P$ and the counter-clockwise A-B path from $A_l$ passes through $Q$. We prove that $P$ and $Q$ are consecutive vertices of $\{A_i\}$ in some SMT of $\{A_i\} \cup \{B_i\}$.


    \begin{claim} \label{PQ_outside_A}
         $P$ and $Q$ cannot simultaneously lie in the interior of the polygon $\{A_i\}$. 
    \end{claim}
\begin{proof}
    We assume for the sake of contradiction that both $P$ and $Q$ lie in the interior of the polygon $\{A_i\}$. On deleting the edge $S_1S_2$, the SMT of $\{A_i\} \cup \{B_i\}$ splits into two trees $\mathcal T_1$ (rooted at $S_1$) and $\mathcal T_2$ (rooted at $S_2$). Further, as all A-B paths pass through $S_1S_2$, all vertices of $\{A_i\}$ must be in $\mathcal T_1$ whereas all vertices of $\{B_i\}$ must lie in  $\mathcal T_2$. Further, $\mathcal T_1$ must be the SMT of $\{A_i\} \cup \{S_1\}$ and $\mathcal T_2$ must be the SMT of $\{B_i\} \cup \{S_2\}$.

    \begin{itemize}
    \item \textit{Case I: One point in $\{S_1, S_2\}$ lies in the interior of $\{A_i\}$ and the other point lies in the exterior of $\{A_i\}$:} This means that the edge $S_1S_2$ crosses some polygon edge of $\{A_i\}$, call it $A_mA_{m+1}$. Let $D$ be the intersection of $A_mA_{m+1}$ and $S_1S_2$. We replace $\mathcal T_1$ with an MST of $\{A_i\}$ that contains the edge $A_mA_{m+1}$ (this can never lead to increase in total tree length due to \cite{weng1995steiner}) and remove the line segment $S_1D$ from $\mathcal T_2$. This forms a tree connecting the terminal set $\{A_i\} \cup \{B_i\}$ which has a total length smaller than the SMT we started with, which is a contradiction.
    
    \item \textit{Case II: Both $S_1$ and $S_2$ lie in the interior of polygon $\{A_i\}$:} We further consider two cases for this:

    \begin{itemize}
        \item Consider that there is at least one polygon edge $A_mA_{m+1}$ of $\{A_i\}$ such that it does not intersect with $\mathcal T_2$. Then we can replace $\mathcal T_1$ by the MST of $\{A_i\}$ which does not contain the edge $A_mA_{m+1}$ without reducing the total edge. However, this will be a connecting tree of $\{A_i\} \cup \{B_i\}$ with a smaller total length than the tree we started with (as we had removed the edge $S_1S_2$ previously), which is a contradiction.
        \item Now, consider that all polygon edges of $\{A_i\}$ intersect with some edge in $\mathcal T_2$. Since $\mathcal T_2$ is rooted at $S_2$ which lies in the interior of $\{A_i\}$, there must be $n$ distinct edges crossing the polygon $\{A_i\}$. However, $\mathcal T_2$ must be the SMT of the points $\{S_2\} \cup \{B_i\}$, which means there can be at most $(n - 1)$ Steiner points other than $S_2$ (from~\Cref{smt-prop}). Therefore, one of these $n$ edges must have a point in $\{B_i\}$ as one of its endpoints, contradicting Observation \ref{no_in_to_B}. 

  
    \end{itemize}
    
    \item \textit{Case III: Both $S_1$ and $S_2$ lie in the exterior of polygon $\{A_i\}$:} This means that the edges $S_1P$ and $S_1Q$ intersect the polygon edges of $\{A_i\}$. Further, from Observation \ref{no_pt_in_PSQ}, there cannot be any terminal inside the triangle $S_1PQ$. Hence, $S_1P$ and $S_1Q$ must intersect the same polygon edge of $\{A_i\}$ (otherwise intermediate vertices from $\{A_i\}$ would lie in the triangle $S_1PQ$). Let this edge be $A_tA_{t+1}$. Let $P_1$ and $Q_1$ be the points of intersection of $A_tA_{t+1}$ with $S_1P$ and $S_1Q$ respectively.

    We now remove the line segments $S_1P_1$ and $S_1Q_1$ from $\mathcal T_1$. This results in another split into two connected trees $\mathcal T_P$ (containing $P_1$, $P$ and a subset of $\{A_i\}$) and $\mathcal T_Q$ (containing $Q_1$, $Q$ and the remaining vertices of $\{A_i\}$). 
    
    We observe that the terminals in $\mathcal T_P$ and $\mathcal T_Q$ form consecutive intervals of the edges in $\{A_i\}$. To see why, consider the opposite, \textit{i.e.} there are vertices $A_{i_1}, A_{i_2}, A_{i_3}, A_{i_4}$ appearing in that order in $\{A_i\}$ such that $A_{i_1}, A_{i_3} \in \mathcal T_P$ whereas $A_{i_2}, A_{i_4} \in \mathcal T_Q$. As $\mathcal T_P$ and $\mathcal T_Q$ lie in the interior of $\{A_i\}$, the path from $A_{i_1}$ to $A_{i_3}$ in $\mathcal T_P$ must cross the path from $A_{i_2}$ to $A_{i_4}$ in $\mathcal T_Q$. However, there cannot be crossing paths in the original SMT of $\{A_i\} \cup \{B_i\}$ (due to~\Cref{smt-prop}). 
    
    Let $\{A_{w}, A_{{w}+1}, \ldots, A_t\}$ be the terminals in $\mathcal T_P$ and the remaining terminals in $\{A_i\}$ are in $\mathcal T_Q$. Again, let $\mathcal T_P'$, $\mathcal T_Q'$ be defined as:
    $$E(\mathcal T_P') = \{A_wA_{w+1}, A_{w+1}A_{w+2}, \ldots, A_{t-1}A_{t}\} \cup \{A_tP_1\}$$
    and
    $$E(\mathcal T_Q') = \{A_{t+1}A_{t+2}, \ldots, A_{w-2}A_{w-1}\} \cup \{Q_1A_{t-1}\}$$
    From Observation \ref{subchain-smt}, we know that $\mathcal T_P'$ is the SMT of $\{A_{w}, A_{{w}+1}, \ldots, A_t\} \cup \{P_1\}$ and $\mathcal T_Q'$ is the SMT of $\{A_{t + 1}, A_{{t}+2}, \ldots, A_{w-1}\} \cup \{Q_1\}$. Therefore, $|\mathcal T_P'| \le |\mathcal T_P|$ and $|\mathcal T_Q'| \le |\mathcal T_Q|$. This means that $| \mathcal T_1 | \ge |\mathcal T_1'|$, where $\mathcal T_1' = \{S_1P_1, S_1Q_1\} \cup \mathcal T_P' \cup \mathcal T_Q'$. Further, $\mathcal T_1'$ is also a connecting tree of $\{S_1\} \cup \{A_i\}$ and as $\mathcal T_1$ is an SMT of $\{S_1\} \cup \{A_i\}$, then $\mathcal T_1'$ must also be an SMT with $|
    \mathcal T_1'| = |\mathcal T_1|$.

    However, We can remove $S_1P_1$ and $P_1A_t$ from $\mathcal T_1'$ and add $S_1A_t$ to get another connecting tree of $\{S_1\} \cup \{A_i\}$, but with shorter total length (as $\overline{S_1P_1} + \overline{P_1A_t} > \overline{S_1A_t}$ from triangle inequality). This contradicts the optimality of $\mathcal T_1'$ which was derived to be an SMT of $\{S_1\} \cup \{A_i\}$.

    \end{itemize}
This proves the claim.
\end{proof}
    


    We proceed to prove a stronger claim regarding $P$ and $Q$.
    
    \begin{claim} \label{PQ_cons}
    $P$ and $Q$ are consecutive vertices of $\{A_i\}$ in any SMT of $\{A_i\} \cup \{B_i\}$.
    
    \end{claim}
    \begin{proof}
    We first prove that $P, Q$ are vertices of $\{A_i\}$. 
    
    From Claim \ref{PQ_outside_A}, we know that at least one among $P$ and $Q$ must not be in the interior of polygon $\{A_i\}$. Without loss of generality, let it be $P$. We now show that $P$ is a vertex of $\{A_i\}$. For the sake of contradiction we assume that $P$ is not a vertex of $\{A_i\}$ \textit{i.e.} $P$ is a Steiner point. Let $\overrightarrow{PF_1}$ and $\overrightarrow{PF_2}$ be tangents from $P$ to $\{A_i\}$ where $F_1, F_2$ are the points of tangency on $\{A_i\}$. As $\overrightarrow{PF_1}$ and $\overrightarrow{PF_2}$ are tangents, $\angle F_1PF_2 < \pi$. We denote the region between the tangents $\overrightarrow{PF_1}$ and $\overrightarrow{PF_2}$ which contains the all the points in $\{A_i\}$ as $\mathcal R$.
    
    From any Steiner point $H$, which lies outside $\mathcal R$, we can choose a neighbour $H_1$ of $H$ such that $\overrightarrow{HH_1}$ is not directed towards $\mathcal R$. Further we now show that there is one neighbour $P_1$ of $P$ such that $P_1$ is not in $\mathcal R$ and $P_1 \neq S_1$. 

    \textit{Case I: $S_1$ lies in $\mathcal R$.} However there must be another neighbour $P_1$ of $P$ not in $\mathcal R$ (as $\angle F_1PF_2 < \pi$) but as $P_1$ is outside $\mathcal R$, we must have $P_1 \neq S_1$.
    
    \textit{Case II: $S_1$ does not lie in $\mathcal R$ (Figure \ref{steiner_path_from_A_to_A_fig}).} As the counter-clockwise A-B path from $A_l$ passes through $Q$, $A_l$ must be to the left of the line $L_{QS_1}$ if the line is given a orientation from $Q$ to $S_1$. This means that one of the tangents from $P$ (Without loss of generality assume it to be $\overrightarrow{PF_1}$) intersects with the line $L_{QS_1}$. Therefore, taking angles in counter-clockwise order, we have:
    
\begin{figure}[h]
\centering
\includegraphics[width=13cm]{claim_proof_diag.png}
\caption{Case II of~\Cref{PQ_cons}}
\label{steiner_path_from_A_to_A_fig}
\end{figure}

    \begin{align*}
        & \angle S_1PF_1 < \frac{\pi}{3} & [\text{as } \overrightarrow{PF_1} \text{ intersects } L_{QS_1}] \\
        \implies & \angle S_1PF_2 = \angle S_1PF_1 + \angle F_1PF_2 < \frac{\pi}{3} + \pi = \frac{4 \pi}{3}\\
        \implies & \angle F_2PS_1 = 2 \pi - \angle S_1PF_2 > 2 \pi - \frac{4 \pi}{3} = \frac{2 \pi}{3}
    \end{align*}
    
    Hence there must exist one neighbour $P_1$ of $P$ lying outside the $\mathcal R$, precisely in the region bounded by the rays $\overrightarrow{PF_2}$ and $\overrightarrow{PS_1}$ with $P_1 \ne S_1$. 
    
    Further we can choose a neighbour $P_2$ of $P_1$ such that $\overrightarrow{P_1P_2}$ is directed away from $\mathcal R$. We can continue choosing $P_2, P_3, \ldots $ such that $\overrightarrow{P_iP_{i+1}}$ is directed away from the region $\mathcal R$. Moreover, the path  $P, P_1, P_2, \ldots$ must end at some point $B_k$ as it cannot end in any vertex of $\{A_i\}$ (since all vertices of $\{A_i\}$ are in $\mathcal R$). Now, let $\mathcal C_1$ be the path from $A_r$ to $B_k$ (which passes through $P$ and $P_1$) and let $\mathcal C_2$ be the counter-clockwise A-B path from $A_l$ (passing through $Q$, $S_1$ and $S_2$). We observe that $\mathcal C_1$ and $\mathcal C_2$ are two edge disjoint A-B paths, which is a contradiction to~\Cref{mincut_1}. This proves that $P$ is indeed a vertex in $\{A_i\}$. Therefore by Claim \ref{PQ_outside_A}, $Q$ does not lie inside $\{A_i\}$ and repeating this same argument on $Q$ yields that $Q$ is also a vertex of $\{A_i\}$. 

    Now, to prove that $P$ and $Q$ are consecutive vertices of $\{A_i\}$, we use Observation \ref{no_pt_in_PSQ}. Observation \ref{no_pt_in_PSQ} implies that there must not be any other point of the SMT in the triangle $PS_1Q$. This means that $P$ and $Q$ must be consecutive vertices of $\{A_i\}$, otherwise all polygon vertices of $\{A_i\}$ occurring in between $P$ and $Q$ would be inside the triangle $PS_1Q$ (as $\angle PS_1Q = \dfrac {2 \pi}{3}$ and $n \ge 13$). This proves the claim.
    \end{proof}

    Therefore, $P$ and $Q$ are consecutive vertices $A_j, A_{j+1}$ of the polygon $\{A_i\}$, for some $j \in [n]$ such that $A_j$, $S_1$, $A_{j+1}$ is a path in the SMT, where $S_1$ is a Steiner point lying on all A-B paths.
\end{proof}

Our next step is to investigate some more structural properties of an SMT for $\{A_i\} \cup \{B_i\}$. From \cite{du1987steiner}, we may guess that there would be a lot of polygon edges of both $\{A_i\}$ and $\{B_i\}$ in an SMT. We prove the following Lemma, stating that there is an SMT of $\{A_i\} \cup \{B_i\}$ which contains $(n - 2)$ polygon edges of $\{A_i\}$.

\begin{lemma} \label{n-2_A_poly_edges}
    For an \smtpoly with aspect ratio $\lambda$, 
    $\lambda > \lambda_{1} = \frac{1}{1 - 4 \sin \frac{\pi}{n}}$, let $S_1$ be the Steiner point such that all A-B paths pass through $S_1$. Let $A_j$ and $A_{j+1}$ be vertices of $\{A_i\}$ which are connected to $S_1$. Then, there exists an SMT of $\{A_i\} \cup \{B_i\}$ having $(n - 2)$ polygon edges of $\{A_i\}$ other than $A_jA_{j+1}$.
\end{lemma}

\begin{proof}
    Let $\mathcal T_0$ be any SMT of $\{A_i\} \cup \{B_i\}$. From Lemma \ref{steiner_path_from_A_to_A}, we know that there exists $S_1$, $A_j$ and $A_{j + 1}$ such that $S_1$ is a Steiner point which is a part of all A-B paths, and $A_jS_1A_{j+1}$ is a path $\mathcal T_0$.

    From $\mathcal T_0$, we remove the edges $S_1A_j$ and $S_1A_{j + 1}$ and add the edge $A_jA_{j+1}$. This results in a forest of two disjoint trees $\mathcal T_x$ and $\mathcal T_y$. One of these trees (say $\mathcal T_x$) must contain all terminal points from $\{A_i\}$ and the other tree must contain all terminals from $\{B_i\}$, as no more A-B paths exist after we removed the edges $S_1A_j$ and $S_1A_{j+1}$. Therefore we have $|\mathcal T_0| = |\mathcal T_x| + |\mathcal T_y| - 1 + \overline{S_1A_j} + \overline{S_1A_{j+1}}$.

    We further replace $\mathcal T_x$ with a Euclidean minimum spanning tree $\mathcal T_x'$ of $\{A_i\}$ such that the edge $A_jA_{j+1}$ is present in $\mathcal T_x'$. From \cite{du1987steiner}, we know that $|\mathcal T_x'| \le |\mathcal T_x|$. We now remove the edge $A_jA_{j+1}$ and add back the edges $S_1A_j$ and $S_1A_{j+1}$ which gives a connected tree $\mathcal T_0'$ of $\{A_i\} \cup \{B_i\}$. Therefore we have:
    $$|\mathcal T_0'| = |\mathcal T_x'| + |\mathcal T_y| - 1 + \overline{S_1A_j} + \overline{S_1A_{j+1}} \le |\mathcal T_x| + |\mathcal T_y| - 1 + \overline{S_1A_j} + \overline{S_1A_{j+1}} = |\mathcal T_0|$$

    This means $\mathcal T_0'$ must be an SMT. However, all polygon edges of polygon $\{A_i\}$ appearing in $\mathcal T_x'$ also appear in $\mathcal T_0'$ as well, except $A_jA_{j+1}$. Therefore, the SMT $\mathcal T_0'$ has $(n - 2)$ polygon edges of the polygon $\{A_i\}$.
\end{proof}

With these set of results in hand, we can now show that there exists an SMT of $\{A_i\} \cup \{B_i\}$ following a \emph{singly connected topology}. To show this, we start with any \smtpoly, $\mathcal T_0$, that satisfies all the results derived so far and transform it into a Steiner tree of \emph{singly connected topology} having total length not longer than the initial Steiner tree $\mathcal T_0$.

\begin{theorem} \label{final_proof}
There exists an \smtpoly following a singly connected topology for $n \ge 13$ and $\lambda \ge \lambda_{1}$, where
$$
% \lambda_{1} = \frac{2 \sin{\frac{\pi}{n}} + 1}{1 - 6 \sin \frac{\pi}{n}}
\lambda_{1} = \frac{1}{1 - 4 \sin \frac{\pi}{n}}
$$
\end{theorem}

\begin{proof}
% We know that there is a single path between two points in $\{A_i\}$ containing a Steiner point and it is of the form $A_jS_1A_{j+1}$ (from Lemma \ref{steiner_path_from_A_to_A}). The rest of the points in $\{A_i\}$ must be connected by $(n - 2)$ polygon edges of $\{A_i\}$.

Let $\mathcal{T}_0$ be any SMT of $\{A_i\} \cup \{B_i\}$ which satisfies the properties of Lemma \ref{n-2_A_poly_edges}. Further, from Lemma \ref{steiner_path_from_A_to_A}, there is a Steiner point $S_1$ which lies on all A-B paths, and there are two consecutive vertices $A_j$, $A_{j+1}$ such that $A_j$, $S_1$, $A_{j+1}$ is a path in $\mathcal T_0$. As $\mathcal T_0$ satisfies the property of Lemma \ref{n-2_A_poly_edges}, $\mathcal T_0$ has $(n - 2)$ polygon edges of $\{A_i\}$ excluding the edge $A_jA_{j+1}$.

Let $H$ be the point in the interior of the polygon $\{A_i\}$  such that $HA_jA_{j+1}$ form an equilateral triangle. As $n > 6$, the common centre $O$ of $\{A_i\}$ and $\{B_i\}$ does not lie inside the triangle $HA_jA_{j+1}$. Now, we modify $\mathcal T_0$ as follows:

\begin{enumerate}
    \item Remove edges $A_jS_1$, $S_1A_{j+1}$ and add edge $S_1H$ to get the forest $\mathcal{T}_1$. We know from \cite{hwang1992steiner} that $S_1$, $S_2$ and $H$ are collinear and this transformation does not change the total length. Therefore $|\mathcal{T}_0| = |\mathcal{T}_1|$. Here, $|\mathcal{T}_1|$ denotes the sum of the lengths of edges present in $\mathcal{T}_1$.
    \item Add edge $HO$ and remove all polygon edges of $\{A_i\}$ to get $\mathcal T_2$. Therefore $|\mathcal T_2| = |\mathcal T_1| + \overline{HO} - (n - 2) = |\mathcal T_0| + \overline{HO} - (n - 2)$. We observe that $\mathcal T_2$ is a tree connecting  the points in $\{B_i\} \cup \{O\}$.
    \item Let $S_0$ be the Torricelli point of the triangle $OB_jB_{j+1}$. Let $\mathcal T_3$ be the Steiner tree of $\{B_i\} \cup \{O\}$ with edges $S_0O$, $S_0B_j$, $S_0B_{j+1}$ and other points in $\{B_i\}$ connected through $(n - 2)$ polygon edges of the polygon $\{B_i\}$. From \cite{weng1995steiner}, we know that $\mathcal T_3$ is the SMT of $\{B_i\} \cup \{O\}$. Therefore $|\mathcal T_3| \le |\mathcal T_2| = |\mathcal T_0| + \overline{HO} - (n - 2)$. Further we know that $H$ lies on the edge $OS_0$ (as $O$, $S_0$ and $H$ lie on the perpendicular bisector of $B_j$ and $B_{j + 1}$).
    \item Remove edge $S_0O$ and add edge $S_0H$ to get $\mathcal T_4$. As $H$ lies on the edge $OS_0$, we have $|\mathcal T_4| = |\mathcal T_3| - \overline{OH} \le |\mathcal T_0| - (n - 2)$.
    \item Let $S_3$ be the intersection of the circumcircle of triangle $A_jHA_{j + 1}$ (from Lemma \ref{lambda_v} the intersection exists as $\lambda_1 \ge \lambda_v$ for $n \ge 13$). Remove the edge $S_3H$ and add the edges $S_3A_j$ and $S_3A_{j + 1}$ to get $\mathcal T_5$. Again, from \cite{hwang1992steiner} we know that this transformation does not change the total length. Hence $|\mathcal T_5| = |\mathcal T_5| \le |\mathcal T_0| - (n - 2)$. Moreover, as $\lambda > \lambda_v$, we observe that $\{A_j, B_j, A_{j+1}, B_{j+1}, S_3, S_0\}$ form the vertices of the vertical gadget and points $O$, $H$, $S_3$, $S_0$ appear in that order on the perpendicular bisector of $B_j$ and $B_{j+1}$.
    \item Add back the $(n - 2)$ polygon edges of $\{A_i\}$ which were removed in the second step to get $\mathcal T_6$. Therefore $|\mathcal T_4| = |\mathcal T_5| + (n - 2) \le |\mathcal T_0|$. We further observe that $\mathcal T_6$ is a Steiner tree connecting the points $\{A_i\} \cup \{B_i\}$ with a singly connected topology.
\end{enumerate}

Therefore we started with an arbitrary SMT $\mathcal T_0$ and transformed it into a Steiner tree $\mathcal T_6$ with a singly connected topology (where $\{A_j, B_j, A_{j+1}, B_{j+1}, S_3, S_0\}$ form the vertices of the vertical gadget) which has a total length not worse than $\mathcal T_0$. Hence $\mathcal T_6$ must be an SMT of $\{A_i\} \cup \{B_i\}$. This proves the theorem.
\end{proof}

\begin{remark}
    \cref{final_proof} determines the exact structure of the \smtpoly. Further from~\cref{trapezoids} we determine the exact method to construct the two additional Steiner points in $\mathcal O(1)$ steps - note that this construction time is independent of the integer $n$ or the real number $\lambda$. Therefore, \smtpoly for $n \ge 13$ and $\lambda \ge \lambda_1$ is solvable in polynomial time.
\end{remark}

%This means that there is an SMT of $\{A_i\} \cup \{B_i\}$ which will have a \emph{singly connected topology} for $n \ge 13$ and $\lambda \ge \lambda_{1}$. \\

Note that the total length of any \smtpoly, when $n \ge 13$ and $\lambda \ge \lambda_{1}$, is
\begin{align*}
    & |\mathcal T_6| = |\text{vertical gadget}| + |(n - 2) \text{ edges of $\{B_i\}$}| + |(n - 2) \text{ edges of $\{A_i\}$}| \\
    \implies & |\mathcal T_6| = \bigg(\dfrac{(\lambda - 1)}{2 \tan \frac \pi n} + \dfrac {\sqrt 3 (\lambda + 1)} {2}\bigg) + (n - 2) \cdot \lambda + (n - 2)\\
    \implies & {|\mathcal T_6| = \dfrac{(\lambda - 1)}{2 \tan \frac \pi n} + \bigg(n - 2 + \frac{\sqrt{3}}{2}\bigg) (\lambda + 1)}\\
\end{align*}
\noindent Further, $\lambda_1$ converges to 1 very quickly with increasing $n$ (plotted in Figure \ref{lambda_1_plot}):

    \vspace{0.3cm}
    \begin{center}
    \begin{tabular}{| c | c | c | c | c | c |}
    \hline
    $n$ & 13 & 20 & 40 & 100 & 500 \\
    \hline
    $\lambda_1$ & 23.3987 & 2.6719 & 1.4574 & 1.1437 & 1.0258\\
    \hline
    \end{tabular}
    \end{center}
    \vspace{0.3cm}

    
\begin{figure}[h]
\centering
\includegraphics[width=15cm]{lambda_1_and_lambda_v.png}
\caption{Plot of $\lambda_1$ \& $\lambda_v$ against $n$}
\label{lambda_1_plot}
\end{figure}

    
This means for large sized $n$ and for ratios that are not too small, the SMT will follow a \textit{singly connected topology}. 

% \begin{Lemma}
%     The SMT of concentric regular polygons $\{A_i\}$ and $\{B_i\}$ with $n \ge 13$ and $\lambda \ge \lambda_{1}$, all ``connections'' between $\{A_i\}$ and $\{B_i\}$ will be full Steiner subtrees forming vertical gadgets. 
% \end{Lemma}

% \begin{proof}
%     Clearly there cannot be any Horizontal Forks appearing as Horizontal Gadgets as $\lambda \ge \lambda_{0} \ge \lambda_s \ge \lambda_v$ and we can replace the Horizontal Gadget with a vertical gadget to get an even smaller length.
    
%     For any other topology of the full Steiner subtree connecting $\{A_i\}$ and $\{B_i\}$ with only one edge needed to remove in order to disconnect $\{A_i\}$ and $\{B_i\}$ (and the rest of the tree connected via polygon edges), there must exist terminals $A_j, A_{j + 1}$ and a steiner point $S_0$ such that $A_jS_0A_{j+1}$ is a path in the SMT. Let $O$ be the common center of $\{A_i\}$ and $\{B_i\}$. We remove all the edges in the SMT of that were a edges of the inner polygon, as well as the edges $A_jS_0$ and $S_0A_{j+1}$. We then add the edge $S_0O$ Since this was the SMT for $\{A_i\} \cup \{B_i\}$, this new construction gives us an SMT for $\{B_i\} \cup \{O\}$, contradicting the main result of \cite{weng1995steiner}.

%     \begin{figure}[h]
%         \centering
%         \includegraphics[width=15cm]{vert_proof.png}
%         \caption{Contradiction of \cite{weng1995steiner}}
%         \label{fig:vert_proof}
%     \end{figure}
    
% \end{proof}

% This implies that whenever $n \ge 12$ and $\lambda > \lambda_0$, the SMT follows the Singly Connected topology. Further $\lambda_0$ quickly converges to 1:

%     \vspace{0.3cm}
%     \begin{center}
%     \begin{tabular}{| c | c | c | c | c | c |}
%     \hline
%     $n$ & 12 & 20 & 40 & 100 & 500 \\
%     \hline
%     $\lambda_0$ & 2.732 & 1.627 & 1.243 & 1.086 & 1.017\\
%     \hline
%     \end{tabular}
%     \end{center}
%     \vspace{0.3cm}
    
% This means for large sized $n$ and for ratios that are not too small, the SMT will follow the Singly connected topology. 

%Few simulated examples (with the help of \href{http://www.geosteiner.com/}{[2]}) are as follows

% \begin{figure}[h!]
% \centering
% \subfloat[\centering $\lambda = 2.73205$] {\includegraphics[width=5cm]{12_vert.png}}
%     \qquad
% \subfloat[\centering $\lambda = 2.73206$] {\includegraphics[width=5cm]{12_sing.png}}
% \caption{points for $n = 12$, $\lambda = 2.732050$}
% \end{figure}


% \begin{figure}[h!]
% \centering
% \subfloat[\centering $\lambda = 1.626797$] {\includegraphics[width=5cm]{20_vert.png}}
%     \qquad
% \subfloat[\centering $\lambda = 1.626798$] {\includegraphics[width=5cm]{20_sing.png}}
% \caption{points for $n = 20$, $\lambda = 1.6267974$}
% \end{figure}




\section{\ESMT on $f(n)$-Almost Convex Point Sets}\label{sec:exact_algo}


% \begin{figure}[h]
% \centering
% \subfloat[\centering SMT for some $\mathcal P$ with $n = 6$] {\includegraphics[width=6cm]{SMT_6.png}}
%     \qquad
% \subfloat[\centering \centering SMT for some $\mathcal P$ with $n = 10$] {\includegraphics[width=6cm]{SMT_10.png}}
% \caption{Some examples of SMT}
% \end{figure}

In this section, we design an exact algorithm for \ESMT on $f(n)$-Almost Convex Point Sets running in time $2^{\OO(f(n)\log n)}$. Note that $f(n) \leq n$ is always true. Therefore, we are given as input a set $\mathcal{P}$ of $n$ points in the Euclidean Plane such that $\mathcal P$ can be partitioned as $\mathcal P = \mathcal P_1 \uplus \mathcal P_2$, where $\mathcal P_1$ is the convex hull of $\mathcal P$ and $|\mathcal P_2| = f(n)$.

First, we look into some mathematical results and computational results to finally arrive at the algorithm for solving \ESMT on $f(n)$-Almost Convex Point Sets. 

We know that the SMT of $\mathcal P$ can be decomposed uniquely into one or more full Steiner subtrees, such that two full Steiner subtrees share at most one node~\cite{hwang1992steiner}. In the following lemma, we further characterize one full Steiner subtree.

\begin{lemma}
\label{existence_of_leaf}
Let $\mathbb{F}$ be the full Steiner decomposition of an SMT of $\mathcal{P}$.  Then there exists a full Steiner subtree $\mathcal F \in \mathbb{F}$ such that $\mathcal F$ has at most one common node with at most one other full Steiner subtree in $\mathbb{F}$.
\end{lemma}
\begin{proof}
If the SMT of $\mathcal P$ is a full Steiner tree, then the statement is trivially true.

Otherwise, we assume that the SMT of $\mathcal P$ has full Steiner subtrees, $\mathbb{F} = \{\mathcal F_1, \mathcal F_2, \ldots, \mathcal F_m\}$, $m \ge 2$. Now, for the sake of contradiction, we assume that for each full Steiner subtree $\mathcal F_j$ there are atleast two other full Steiner subtrees ${\sf Neitree}(\mathcal F_j)^1,{\sf Neitree}(\mathcal F_j)^2 \in \mathbb{F}$ and two terminals $P_1(\mathcal F_j),P_2(\mathcal F_j) \in \mathcal{P}$ such that $P_i(\mathcal F_j) \in V(\mathcal F_j) \cap V({\sf Neitree}(\mathcal F_j)^i)$, $i \in \{1,2\}$. Now, let us construct a walk $W$ in the SMT of $\mathcal{P}$. Starting from $P_1(\mathcal F_1)$ of the full Steiner subtree $\mathcal F_1 \in \mathbb{F}$, we include the path in $\mathcal F_1$ connecting to $P_2(\mathcal F_1)$. Note that $P_2(\mathcal F_1)$ is also contained in ${\sf Neitree}(\mathcal F_1)^2$. Let ${\sf Neitree}(\mathcal F_1)^2 = \mathcal F_{w_1}$ for some $w_1\in [m], w_1\neq 1$. Also let $P_2(\mathcal F_1) = P_1(\mathcal F_{w_1})$.
Then, we know that there is a $P_2(\mathcal F_{w_1})$. In $W$, we include the path in $\mathcal F_{w_1}$ connecting $P_1(\mathcal F_{w_1})$ to $P_2(\mathcal F_{w_1})$. In general, suppose the $i^{th}$ full Steiner subtree to be considered in building the walk is $\mathcal F_{w_{i-1}}$ which was reached via point $P_1(\mathcal F_{w_{i-1}})$. Then we include in $W$ the path in $\mathcal F_{w_{i-1}}$ connecting $P_1(\mathcal F_{w_{i-1}})$ and $P_2(\mathcal F_{w_{i-1}})$. Thus, we can indefinitely keep constructing the walk $W$ as for each $\mathcal F_{w_{i-1}}$ both $P_1(\mathcal F_{w_{i-1}}), P_2(\mathcal F_{w_{i-1}})$ always exist. However, since there are $m$ full Steiner subtrees this means that there is an $\mathcal F_k \in \mathbb{F}$ and two indices $i\neq j$ such that $\mathcal F_k = \mathcal F_{w_i} = \mathcal F_{w_j}$. Thus, there exists a cycle in $W$, which implies that there is a cycle in the SMT of $\mathcal{P}$ (contradiction). Therefore, there must be at least one full Steiner subtree that has at most one common terminal with at most one other full Steiner subtree.
\end{proof}
\begin{remark}
A full Steiner subtree of the SMT of $\mathcal P$ has the topology of a tree. Thus, from Lemma~\ref{existence_of_leaf}, we conclude that a full Steiner subtree, that has at most one common terminal with at most one other full Steiner subtree, has at least one leaf of the SMT.
\end{remark}


\begin{definition}
Let the full Steiner subtrees, that have at most one terminal shared with at most one other full Steiner subtree, be called \textbf{leaf full Steiner subtrees}. Let the terminal which is shared be called the \textbf{pivot} of the leaf full Steiner subtree.
\end{definition}


\begin{figure}[h]
\centering
\subfloat[\centering Full Steiner Subtrees of Figure 1(a)] {\includegraphics[width=6cm]{sub_6.png}}
    \qquad
\subfloat[\centering \centering Full Steiner Subtrees of Figure 1(b)] {\includegraphics[width=6cm]{sub_10.png}}
\caption{Leaf full Steiner subtrees enclosed in ellipses, other full Steiner subtrees enclosed in rectangles, pivots of leaf full Steiner subtrees encircled}
\end{figure}


\begin{lemma}
\label{leaf_non_leaf_structure}
Let $\mathcal F$ be a leaf full Steiner Subtree of the SMT of $\mathcal P$, with terminal points $\mathcal P_{\mathcal{\mathcal F}} \subseteq \mathcal{P}$ and having pivot $P_{\mathcal F}$. Deleting $\mathcal F\setminus \{P_{\mathcal F}\}$ from the SMT of $\mathcal P$ gives us an SMT of the terminal points $((\mathcal P - \mathcal{P}_{\mathcal F}) \cup \{P_{\mathcal F}\})$.
\end{lemma}

\begin{proof}
Firstly, we observe that deleting $\mathcal F\setminus \{P_{\mathcal F}\}$ from the SMT of $\mathcal P$ will indeed give us a tree, as $\mathcal F$ is a leaf full Steiner subtree. Let us call this tree $\mathcal Y$.

Now for the sake of contradiction, we assume that the total length of $\mathcal Y$ is strictly larger than the SMT $\mathcal F'$ of $((\mathcal P - \mathcal{P}_{\mathcal F}) \cup \{P_{\mathcal F}\})$. However, this means, the total length of $\mathcal F' \cup \mathcal F$ is strictly smaller than that of the SMT of $\mathcal P$. As $\mathcal F' \cup \mathcal F$ is also a Steiner tree of $\mathcal P$,  this contradicts the minimality of the initial SMT of $\mathcal P$.
\end{proof}

Now we are ready to describe the algorithm. Recall that $\mathcal P$ is partitioned as $\mathcal P = \mathcal{P}_1 \uplus \mathcal{P}_2$, where $\mathcal{P}_1$ is the convex hull of $\mathcal P$ and $\mathcal{P}_2$ is the set of $f(n)$ points lying in the interior of $\mathcal P_1$. For the sake of brevity of notations let $|\mathcal{P}_2| = k$.

\begin{lemma}
\label{four_power_n}
Let $\mathcal P$ be a $k$-Almost Convex Point Set. A minimum FST of a subset $\mathcal S$ of $\mathcal P$ can be found in $\mathcal O(4^{|{\mathcal S}|} \cdot |\mathcal S| ^ k)$ time.
\end{lemma}

\begin{proof}
We observe that for any $\mathcal S \subseteq \mathcal P$, $\mathcal S$ forms a convex polygon with at most $k$ points lying in the interior. For $|{\mathcal S}| \le 2$, the statement of the lemma is trivially true. Hence we assume that $|{\mathcal S}| > 2$.

From \cite{hwang1992steiner}, the number of full Steiner topologies of $\mathcal{S}$ is 

$$\frac{|\mathcal S|!}{|\mathrm{CH}(\mathcal{S})|!} \cdot \frac {\binom{2|{\mathcal S}| - 4}{|{\mathcal S}| - 2}}{|{\mathcal S}| - 1}$$

However, we know that:

$$\frac {\binom{2|{\mathcal S}| - 4}{|{\mathcal S}| - 2}}{|{\mathcal S}| - 1} < \frac{\binom{2|{\mathcal S}|}{|{\mathcal S}|}}{|{\mathcal S}|} < \frac{\sum \limits_{r = 0}^{2|{\mathcal S}|} \binom{2|{\mathcal S}|}{k}}{|{\mathcal S}|} = \frac{2 ^{2|{\mathcal S}|}}{|{\mathcal S}|} = \frac{4^{|{\mathcal S}|}}{|{\mathcal S}|}$$

And,

$$\frac{|\mathcal S|!}{|\mathrm{CH}(\mathcal{S})|!} < \frac{|\mathcal S|!}{(|\mathcal S| - k)!} < |\mathcal S|^k$$

Therefore, the number of full Steiner topologies of $\mathcal{S}$ is at most $4^{|{\mathcal S}|} |\mathcal S| ^ {k - 1}$. Each of these topologies can be enumerated and using \emph{Melzak's FST Algorithm}, we can also find the SMT realizing each such full Steiner topology in linear time, as given in \cite{hwang1992steiner}. Therefore to iterate over all topologies and find a minimum takes at most time:

$$(4^{|{\mathcal S}|} \cdot |\mathcal S| ^ {k - 1}) \cdot \mathcal O(|\mathcal S|) = \mathcal \mathcal O(4^{|{\mathcal S}|} \cdot|\mathcal S| ^ k)$$

\end{proof}

Now, we find the time required for extending the results of Lemma~\ref{four_power_n} to all subsets of $\mathcal{P}$.
\begin{lemma}
\label{five_power_n}
Let $\mathcal P$ be a $k$-Almost Convex Set. Computing a minimum FST \textbf{for all} subsets $\mathcal S \subseteq \mathcal P$ can be done in $\mathcal O(n^k \cdot 5^{n})$ time.
\end{lemma}
\begin{proof}

Using Lemma \ref{four_power_n}, we can get a minimum FST for a single subset $\mathcal S \subseteq \mathcal P$ in $\mathcal O(4^{|{\mathcal S}|} |\mathcal S| ^ k)$ time. Moreover, we know that the number of subsets of $\mathcal P$ that are of size $r$ is $\binom n r$. This means that the total time to compute a minimum FST \textbf{for all} subsets $\mathcal S \subseteq \mathcal P$, time taken is: 


$$\sum \limits_{r = 0} ^ {n} \binom{n}{r} \cdot \mathcal O(4^r \cdot r^k) = \sum \limits_{r = 0} ^ {n} \binom{n}{r} \cdot \mathcal O(n^k \cdot 4^r) = \mathcal O(n^k \cdot (1 + 4)^n) = \mathcal O(n^k \cdot 5^n)$$

\end{proof}

For each $\mathcal S \subseteq \mathcal P$, we denote by $\mathcal F_{\mathcal S} $ a minimum FST of $\mathcal S$ and by $\mathcal T_{\mathcal S} $ the SMT of $\mathcal S$.

\begin{lemma}
\label{single_subset_SMT}
The SMT of subset $\mathcal S \subseteq \mathcal P$, $\mathcal T_{\mathcal S}$, can be found in $\mathcal O(|{\mathcal S}| \cdot 2^{|{\mathcal S}|})$ time, given that we have pre-computed $\mathcal T_{\mathcal R}$ and $\mathcal F_{\mathcal R}$, $\forall \mathcal R \subseteq \mathcal S$.
\end{lemma}

\begin{proof}

If $\mathcal T_{\mathcal S}$ was a full Steiner tree then it would be $\mathcal F_{\mathcal S}$. Otherwise, $\mathcal T_{\mathcal S}$ contains multiple full Steiner subtrees.

Let $\mathcal F$ be a leaf full Steiner subtree of $\mathcal T_{\mathcal{S}}$ with pivot $P_{\mathcal F}$. Therefore from Lemma \ref{leaf_non_leaf_structure} we have $\mathcal T_{\mathcal S} = \mathcal T_{((\mathcal S - V(\mathcal F)) \cup \{P_{\mathcal F}\})} \cup \mathcal F$. Therefore we can iterate over all subsets $\mathcal R \subset \mathcal S$ and all terminals $P \in \mathcal{R}$, and take the minimum-length tree among $\mathcal T_{((\mathcal S - \mathcal R)  \cup \{P\})} \cup \mathcal F_{\mathcal R}$. Since we are iterating over all $\mathcal R \subset \mathcal S$, and all $P \in \mathcal R$, we are guaranteed to get $\mathcal R = V(\mathcal F) \cap \mathcal{S}$ and $P = P_{\mathcal F}$ on one such iteration.

Now, as there are $\mathcal O(2^{|\mathcal S|})$ possibilities of $\mathcal R \subset \mathcal S$ and $\mathcal O(|\mathcal S|)$ possibilities of $P \in \mathcal R$, we have $\mathcal O(|S| \cdot 2^{|\mathcal S|})$ possibilities of the pair $(\mathcal R, P)$. Therefore the total time required for iterating is $\mathcal O(|{\mathcal S}| \cdot 2^{|{\mathcal S}|})$.

\end{proof}

\begin{lemma}
\label{multi_subset_SMT}
SMTs for all subsets $\mathcal S \subseteq \mathcal P$, $\mathcal T_{\mathcal S}$ can be found in $\mathcal O(n \cdot 3 ^n)$ time, given that we have precomputed a minimum FST $\mathcal F_{\mathcal S}$ $\forall \mathcal S \subseteq \mathcal P$.
\end{lemma}

\begin{proof}

Using Lemma \ref{single_subset_SMT}, we can get the SMT $\mathcal T_{\mathcal S}$, for a single subset $\mathcal S \subseteq \mathcal P$ in $\mathcal O(r \cdot 2^r)$ time, where $|{\mathcal S}| = r$. Moreover, we know that the number of subsets of $\mathcal P$ that are of size $r$ is $\binom n r$. This means that the total time to compute the SMT for all subsets $\mathcal S \subseteq \mathcal P$, time taken is: 

$$\sum \limits_{k = 0} ^ {n} \binom{n}{k} \cdot \mathcal O(k \cdot 2^k) = \mathcal O(n \cdot (1 + 2)^n) = \mathcal O(n \cdot 3^n)$$

However, to apply Lemma \ref{single_subset_SMT} on some subset $\mathcal S \subseteq \mathcal P$ for computing $\mathcal T_{\mathcal S}$, we must also have $\mathcal T_{\mathcal R}$ precomputed for all $\mathcal R \subseteq \mathcal S$. This can be guaranteed by computing $\mathcal T_{\mathcal S}$ and $\mathcal{F}_{\mathcal{S}}$, for all subsets $\mathcal S \subseteq \mathcal P$ in an increasing order of $|\mathcal S|$ (or any order which guarantees that the subsets of $\mathcal S$ are processed before $\mathcal S$).

\end{proof}

Finally, we state our algorithm.

\begin{theorem}
\label{final_theorem}

An SMT $\mathcal T_{\mathcal P}$ of a $k$-Almost Convex Set $\mathcal P$ of terminals can be computed in $\mathcal O(n^k \cdot 5^n)$ time.

\end{theorem}

\begin{proof}

Consider the following algorithm: 
\begin{algorithm}[H]
\caption{Computation of $\mathcal{T_P}$ ~~~ \textbf{Input:} $\mathcal P$ }\label{alg:main algo}
\begin{algorithmic}[1]
\For{all $\mathcal S \subseteq \mathcal P$} 
\State Compute $\mathcal {F_S}$ \Comment{Using Lemma \ref{five_power_n}}
\EndFor \Comment{This takes $\mathcal O(n^k \cdot 5^n)$ time}
\For{all $\mathcal S \subseteq \mathcal P$}
\State Compute $\mathcal {T_S}$ \Comment{Using Lemma \ref{multi_subset_SMT}}
\EndFor \Comment{This takes $\mathcal O(n \cdot 3^n)$ time}
\State \Return $\mathcal{T_P}$ \Comment{Total runtime is $\mathcal O(n^k \cdot 5^n + n \cdot 3^n) = \mathcal O(n^k \cdot 5^n)$}
\end{algorithmic}
\end{algorithm}


% \begin{itemize}
%     \item Using Theorem \ref{five_power_n}, we precompute $\mathcal F_{\mathcal S}$ for all subsets $\mathcal S  \subseteq \mathcal P$. This takes $\mathcal O(5^n)$ time.
%     \item Using Theorem \ref{multi_subset_SMT} and the precomputation from the previous step, we can get $\mathcal T_{\mathcal S}$ for all subsets $\mathcal S  \subseteq \mathcal P$. This takes $\mathcal O(n \cdot 3^n)$ time.
%     \item We return $\mathcal T_{\mathcal P}$ as our final answer in time $\mathcal O(5^n + n \cdot 3^n) = \mathcal O(5^n)$.
% \end{itemize}

Hence we have an SMT of a $k$-Almost Convex Point Set $\mathcal P$ in $\mathcal O(n^k\cdot 5^n)$ time.


\end{proof}

% \vspace{0.5cm}
% \section{Extending the Algorithm}
% \vspace{0.5cm}

%     We now consider the configuration where there are a small number of points inside a convex polygon. Formally, we wish to find the SMT of $\mathcal{P} \cup \mathcal{P}_{in}$, where $\mathcal{P}_{in}$ is a set of $k$ points lying strictly inside the convex polygon $\mathcal{P}$, such that $k$ is small.\\
    
%     We can apply \textbf{Algorithm 1} (derived in \ref{final_theorem}) in this case as well. However, the analysis of run time does not remain the same. We will now analyse the runtime of the same algorithm when the input is $\mathcal{P}_1 = \mathcal{P} \cup \mathcal{P}_{in}$, to get its SMT. 
    
%     \begin{align*}
%         \text{The number of Full Steiner Topologies of } \mathcal{P}_1 = & \frac{|\mathcal{P}_1|!}{|\text{convexhull}(\mathcal{P}_1)|!} \cdot \Bigg(\frac {\binom{2|{\mathcal P_1}| - 4}{|{\mathcal P_1}| - 2}}{|{\mathcal P_1}| - 1}\Bigg)\\
%         < & \frac{(n + k)!}{|\mathcal{P}|!} \cdot \frac{4^{|\mathcal P_1|}}{|\mathcal P_1|}\\
%         = & \frac{(n + k)!}{n!} \cdot \frac{4^{(n + k)}}{n + k}\\
%     \end{align*}
    
%     Similarly for each subset $\mathcal S_1 \subset \mathcal{P}_1$ of size $m$ has number of Full Steiner topologies atmost to $\Big( \dfrac{m!}{(m - k)!} \cdot \dfrac{4^m}{m} \Big)$ as the number of points strictly inside the convex hull cannot increase. And hence running Melzak's Full Steiner Tree Algorithm for each such topology would take time of $\mathcal O \Big(\frac{m!}{(m - k)!} \cdot \frac{4^m}{m} \cdot m \Big)$ = $\mathcal O((n + k)^k \cdot 4^m)$.\\

%     Hence running \textbf{Algorithm 1} on all subsets of $\mathcal P_1$ would take time:
    
%     $$\sum \limits_{m = 0}^{n + k}{(n + k) ^ k \cdot 4^m \cdot \binom{n + k}{m}} = \mathcal O ((n + k) ^ k \cdot 5^{(n + k)})$$
    
%     Further, for the part of the algorithm performing the computation of $\mathcal T_S$, $\forall \mathcal{S} \subseteq \mathcal P_1$ (as in \ref{multi_subset_SMT}), the runtime would not exceed $\mathcal O (4 ^ {n + k})$.\\
    
%     Therefore the runtime of the entire algorithm, when ran on $\mathcal P_1$, will be $\mathcal O ((n + k) ^ k \cdot 5^{(n + k)})$. \\

The above theorem gives us several improvements in special classes of inputs, based on the number of input points lying inside the convex hull of the input set, as described in the following corollary. Let there be an $f(n)$-Almost Convex Point Set $\mathcal P$ containing $n$ points. Recall that $\mathcal{P} = \mathcal{P}_1 \uplus \mathcal{P}_2$, $\mathcal{P}_1$ containing the points on the convex hull of $\mathcal{P}$, and $|\mathcal{P}_2| = f(n)$. It is only possible that $f(n) \leq n$. 

\begin{corollary}\label{almost-better}
Let $\mathcal P$ be a $f(n)$-Almost Convex Point Set. Then, then there is an algorithm $\mathcal{A}$ for \ESMT such that, $\mathcal{A}$ runs in $2^{\OO{(n + f(n) \log n)}}$ time. In particular, 
\begin{enumerate}
    \item When $f(n) = \OO(n)$, $\mathcal A$ runs in $2^{\OO(n\log n)}$ time.
    \item When $f(n) = \Omega(\frac{n}{\log n})$ and $f(n) = o(n)$, $\mathcal{A}$ runs in $2^{o(n\log n)}$.
    \item When $f(n) = \OO(\frac{n}{\log n})$, $\mathcal A$ runs in $2^{\OO(n)}$ time.
  \end{enumerate}
\end{corollary}

Therefore, for $f(n) = o(n)$, our algorithm for \ESMT does better on $f(n)$-Almost Convex Points Sets than the current best known algorithm~\cite{hwang1986linear}.
    
%    If we consider $k$ to be a constant, the runtime would be $\mathcal O (n ^ k \cdot 5^{n})$. With a more generalized assumption of $k = o(n)$, we get the runtime to be $\mathcal O (n ^ k \cdot 5^{n}) = 2^{(\mathcal O(n) + o(n) \log_2(n))} = 2 ^ {o(n \log n)}$.\\
    
%    \begin{remark}
%        This algorithm when run on the arbitrary positioned points as input, would get number of Full Steiner Topologies to be much higher  causing the runtime to be $2 ^ {\mathcal O(n \log n)}$ instead. 
%    \end{remark} 

\section{Approximation Algorithms for \ESMT}\label{sec:apx_esmt}

The \ESMT problem is NP-hard as shown by Garey et al. in~\cite{garey1977complexity}. Garey et al. also prove that there cannot be an FPTAS (fully polynomial time approximation scheme) for this problem unless $P=NP$. At the same time, the case when all the terminals lie on the boundary of a convex region admits an FPTAS as given in~\cite{scott1988convexity}. We aim to conduct a more fine-grained analysis for the problem by considering $f(n)$-Almost Convex Point Sets of $n$ terminals and studying the existence of FPTASes for different functions $f(n)$. First, we present an FPTAS for \ESMT on $f(n)$-Almost Convex Sets of $n$ terminals, when $f(n) = \OO (\log n)$. Next, we prove that no FPTAS exists for the case when $f(n) =\Omega (n^\epsilon)$, where $\epsilon \in (0,1]$.

\subsection{FPTAS for \ESMT on Cases of Almost Convex Point Sets}\label{subsec:fptas}

We first propose an algorithm for computing the SMT of a planar graph $G$ having $N$ vertices and $n$ terminals, out of which $k$ terminals lie on the outer face of $G$ and the remaining terminals lie within the boundary. Next, following the procedure in~\cite{scott1988convexity} we get an FPTAS for \ESMT on $f(n)$-Almost Convex Sets of $n$ terminals, where $f(n) = \OO (\log n)$.

We state the following proposition from Theorem 1 in~\cite{scott1988convexity}:
\begin{proposition}\label{prop:tree_interval}
    Let $\C{P}$ be the vertices of any polygon in the plane, $\C{K}$ a subset of $\C{P}$, and $\C{T}$ a tree consisting of all the vertices of $\C{K}$ (and possibly some other vertices as well) and contained entirely inside $\C{P}$. Then on removing any edge of the tree, we get two disjoint trees $\mathcal{T}_1, \mathcal{T}_2$, such that the vertices of $\C{K}$ in each tree $\mathcal{T}_i, i \in \{1,2\}$ form an interval in $\C{K}$.
\end{proposition}

Using~\Cref{prop:tree_interval} and the Dreyfus-Wagner algorithm~\cite{dreyfus1971steiner}, we give an algorithm for obtaining the SMT of a planar graph $G$. 

Let $\C{K}$ represent the set of terminals lying on the outer face of $G$ and $\C{R}$ be the set of terminals lying inside the outer face of $G$. We have $|V(G)|=N$, $|\C{K} \cup \C{R}|=n$, and $|\C{K}|=k$. Let $C(\C{L})$ denote the SMT in $G$ for a terminal subset $\C{L} \subseteq V(G)$. Let $B(v,\C{L},[a,b))$ denote the SMT in $G$ for the terminal set $\{v\} \cup \C{L} \cup [a,b)$, where $\C{L} \subseteq \C{R}$, $[a,b)$ is the set of vertices in $\C{K}$ forming an interval from vertex $a$ to $b$ in counterclockwise direction along the outer boundary of $G$ including $a$ but excluding $b$, $v \in V(G) \setminus (\C{L} \cup [a,b))$, and the degree of $v$ is at least $2$ in $B(v,\C{L},[a,b))$. Let $A(v,\C{L},[a,b))$ denote the SMT in $G$ for the terminal set $\{v\} \cup \C{L} \cup [a,b)$, where $\C{L}$, $[a,b)$, and $v$ are as defined in the previous case, and the degree of $v$ is at most $1$ in $A(v,\C{L},[a,b))$.

Splitting the SMT at a vertex $v$ of degree at least $2$ gives rise to two smaller instances of the {\sc Steiner Minimal Tree} problem on graphs.
% \begin{equation}\label{eq:dw_B}
%     B(v,\C{L},[a,b)) = \min_{\C{L}' \subseteq \C{L}, x \in [a,b], \emptyset \subset \C{L}' \cup \{[a,x)\} \subset \C{L} \cup \{[a,b)\}} \{C(\{v\} \cup \C{L}' \cup \{[a,x)\}) + C(\{v\} \cup (\C{L}\setminus \C{L}') \cup \{[x,b)\}) \}
% \end{equation}

\begin{equation}\label{eq:dw_B}
    B(v,\C{L},[a,b)) = \min_{\Pi_1, \Pi_2, \Pi_3} \{C(\{v\} \cup \C{L}' \cup [a,x)) + C(\{v\} \cup (\C{L}\setminus \C{L}') \cup [x,b)) \}
\end{equation} where the conditions on $\C{L}'$ and $x$ are $\Pi_1: \C{L}' \subseteq \C{L}$, $\Pi_2: x \in \mathcal{K}, a< x <b$, and $\Pi_3: \emptyset \subset \C{L}' \cup [a,x) \subset \C{L} \cup [a,b)$.

The intuition is to root the tree at an internal terminal vertex and start growing the Steiner tree from there. Observe that on removing one of the internal vertices $v$ in the tree $\C{T}$, we get one, two or three disjoint subtrees. They induce a partition over the terminals. The terminals in $\C{K}$ in each of the subtrees form intervals in $\mathcal{K}$, according to~\Cref{prop:tree_interval}. Moreover, the terminals in $\C{R}$ can be partitioned in any way, not necessarily maintaining the interval structure. This is captured in the following recurrence relation:
\begin{equation}\label{eq:dw_C}
    C(\{v\} \cup \C{L} \cup [a,b)) = \min_{\Pi_1, \Pi_2, \Pi_3}\{A(v,\C{L}_1,[a,c))+A(v,\C{L}_2,[c,d))+A(v,\C{L}_3,[d,b))\}
\end{equation}

where the conditions on $\C{L}_1$, $\C{L}_2$, $\C{L}_3$, $c$, and $d$ are $\Pi_1: \C{L}_1,\C{L}_2,\C{L}_3 \subseteq \C{L}$, $\Pi_2: \C{L}_1 \cup \C{L}_2 \cup \C{L}_3 = \C{L}$, and $\Pi_3: c,d \in \mathcal{K}, a\leq c \leq d \leq b$ and we have 
\begin{equation}\label{eq:dw_A}
    A(v,\C{L}',[p,q))=\min\{\min_{u \notin \C{L}'}\{B(u,\C{L}',[p,q))+d(u,v)\},\min_{u \in \C{L}' \cup [p,q)}\{C(\C{L}' \cup [p,q))+d(u,v)\}\}
\end{equation}

Our aim is to compute $C(\{v\} \cup (\C{R}\setminus \{v\}) \cup \C{K})$, where $v \in \C{R}$. We precompute the shortest distance between all pairs of vertices. We then compute the values of $C(.)$ and $B(.)$ in increasing order of cardinality of subsets of vertices in $\C{K}$ and $\C{R}$. Let $d(u,v)$ denote the shortest path length between $u$ and $v$. The base cases are $C(\{v\} \cup \{a\}) = d(v,a)$ for all $v \in V(G)$ and $a \in \C{K} \cup \C{R}$.

\begin{algorithm}[H]
\caption{Computation of SMT of planar graph $G$ with terminal set $\C{K} \cup \C{R}$ ~~~ \textbf{Input:} $G$, $\C{K}$, $\C{R}$}\label{alg:smt_planar}
\begin{algorithmic}[1]
\State Compute the shortest distance between all pairs of vertices
\For{all $u \in V(G)$ and $a \in K \cup R$} 
\State Set $C(\{u\} \cup \{a\}) = d(u,a)$
\EndFor
\State Select a vertex $v \in \C{R}$
\For{$i = 1, \ldots, n-k-1$}
\For{each $\C{L} \subseteq \C{R}\setminus \{v\}$ of size $i$}
\For{$j = 1, \ldots, k$}
\If{j=k}
\State Compute $B(v,\C{L},\C{K})$ using~\Cref{eq:dw_B}
\State Compute $C(\{v\} \cup \C{L} \cup \C{K})$ using~\Cref{eq:dw_C}
\Else
\For{each $[a,b) \subseteq \C{K}$ of size $j$}
\State Compute $B(v,\C{L},[a,b))$ using~\Cref{eq:dw_B}
\State Compute $C(\{v\} \cup \C{L} \cup [a,b))$ using~\Cref{eq:dw_C}
\EndFor
\EndIf
\EndFor
\EndFor
\EndFor
\State \Return $C(\{v\} \cup (\C{R}\setminus \{v\}) \cup \C{K})$
\end{algorithmic}
\end{algorithm}

We analyse the correctness and running time of~\Cref{alg:smt_planar}.
    \begin{theorem} \label{thm:algo1_correct}
     Consider a planar graph $G$ on $N$ vertices and a set $\mathcal{K} \uplus \mathcal{R} \subseteq V(G)$ of $n$ terminals such that $\mathcal{K}$ is defined as the terminals lying on the outer face of $G$. Moreover, let $|\mathcal{K}| = k$. Then~\Cref{alg:smt_planar} computes the SMT for $\mathcal{K} \uplus \mathcal{R}$ in $G$ in time $\OO(N^2k^44^{n-k} + Nk^33^{n-k}+N^3)$.
    \end{theorem}
    \begin{proof}
{\bf Correctness of~\Cref{alg:smt_planar}.} 
In order to prove the correctness of~\Cref{alg:smt_planar}, we need to show that the~\Cref{eq:dw_B,eq:dw_C} are valid.

        In~\Cref{eq:dw_B}, $B(v,\C{L},[a,b))$ denotes an SMT for the terminal set $\{v\} \cup \C{L} \cup [a,b)$, conditioned on the fact that the degree of $v$ is at least $2$ in it. Let us split the SMT at vertex $v$ into two smaller subtrees. This must also split the terminals in $[a,b)$ in two intervals $[a,x)$ and $[x,b)$, respectively. Otherwise it would mean that the SMT has crossing edges, which is not possible. The vertices in $\mathcal{L}$ can be present in any of the two subtrees, hence we consider all possible partitions of $\C{L}$ into two subsets $\C{L}'$ and $\C{L} \setminus \C{L}'$. Thus for~\Cref{eq:dw_B}, $\mbox{LHS} \geq \mbox{RHS}$. On the other hand, the expression in the RHS of~\Cref{eq:dw_B} is a tree containing the vertex subset $\{v\} \cup \mathcal{L} \cup [a,b)$. Since, $B(v,\mathcal{L},[a,b))$ is an SMT for the same vertex subset, in~\Cref{eq:dw_B}  $\mbox{LHS} \geq \mbox{RHS}$. Therefore,~\Cref{eq:dw_B} is valid.

        For~\Cref{eq:dw_C}, we take $v$ as the root of the SMT $C(\{v\} \cup \C{L} \cup [a,b))$. The degree of $v$ in the SMT can be $1$, $2$, or $3$. Accordingly, on removing $v$, we will get $1$, $2$, or $3$ subtrees, with the condition that in each subtree the degree of $v$ is $1$. Again, the terminals in $[a,b)$ are divided into smaller intervals $[a,c)$, $[c,d)$ and $[d,b)$ for some $c,d \in \mathcal{K}$ satisfying $a<c<d<b$. The terminals in $\C{L}$ are divided among the subtrees in any combination. The number of such intervals and partitions is equal to the degree of $v$ in $C(\{v\} \cup \C{L} \cup [a,b))$. The term $A(v,\C{L}',[p,q))$ is obtained by minimizing across all SMTs satisfying the condition that degree of $v$ in the SMT is $1$. Thus in~\Cref{eq:dw_C}, $\mbox{LHS} \geq \mbox{RHS}$. On the other hand, the RHS of~\Cref{eq:dw_C} given a tree containing vertices $\{v\} \cup \mathcal{L} \cup [a,b)$. Since $C(\{v\} \cup \mathcal{L} \cup [a,b))$ is an SMT on the terminal set $\{v\} \cup \mathcal{L} \cup [a,b)$, in~\Cref{eq:dw_C} $\mbox{LHS} \leq \mbox{RHS}$. Thus,~\Cref{eq:dw_C} is valid.

{\bf Running time of~\Cref{alg:smt_planar}.}
All pairs shortest paths can be calculated in $\OO(N^3)$ time. The time complexity of the dynamic program has two components to it. One is due to computation of the B(.) values using~\Cref{eq:dw_B} and the other is for calculating the C(.) values using~\Cref{eq:dw_C}.
    \begin{enumerate}
        \item The number of computational steps for calculating $B(v,\mathcal{L},[a,b))$ using~\Cref{eq:dw_B} is of the order of the number of choices of $v$, $\mathcal{L}$, $\mathcal{L}'$, $[a,b)$, and $[a,x)$ such that $\mathcal{L} \subset \mathcal{R}$, $\mathcal{L}' \subseteq \mathcal{L}$, $a,x,b \in \mathcal{K}$, $a \leq x \leq b$, and $v \in V(G) \setminus (L \cup [a,b))$. Each vertex in $\mathcal{R}$ belongs to exactly one of the sets $\mathcal{L}'$, $\mathcal{L}\setminus \mathcal{L}'$, or $V(G) \setminus \mathcal{L}$. The vertices in $\mathcal{K}$ are partitioned into three intervals, $[a,x)$, $[x,b)$, and $[b,a)$. There are at most $N$ possibilities for $v$. This gives us a running time of $\OO(Nk^3 3^{n-k})$.
        \item The number of computational steps for calculating $C(\{v\} \cup \mathcal{L} \cup [a,b))$ using~\Cref{eq:dw_C} is $3N$ times the order of the number of choices of $v$, $\mathcal{L}$, $\mathcal{L}_1$, $\mathcal{L}_2$, $a$, $b$, $c$, and $d$ such that $\mathcal{L} \subset \mathcal{R}$, $\mathcal{L}_1 \subseteq \mathcal{L}$, $\mathcal{L}_2 \subseteq \mathcal{L}$, $\{a,c,d,b\} \subseteq \mathcal{K}$, $a \leq c \leq d \leq b$, and $v \in V(G) \setminus (\mathcal{L} \cup [a,b))$. Each vertex in $\mathcal{R}$ belongs to exactly one of the sets $\mathcal{L}_1$, $\mathcal{L}_2$, $\mathcal{L}_3$ or $V(G) \setminus \mathcal{L}$. The vertices in $\mathcal{K}$ are partitioned into at most four intervals, $[a,c)$, $[c,d)$, $[d,b)$ and $[b,a)$. There are at most $N$ possibilities for $v$. The $3N$ factor is because calculating each of the $C(.)$ values involves minimization over at most $3N$ terms. This gives us a running time of $\OO(N^2k^4 4^{n-k})$.
    \end{enumerate}
    Thus, the time complexity of the algorithm is $\OO(N^2k^44^{n-k}+Nk^33^{n-k}+N^3)$.
\end{proof}
We obtain the following corollary from the above theorem.

\begin{corollary}\label{graph-correct}
 Consider a planar graph $G$ on $N$ vertices and a set $\mathcal{K} \uplus \mathcal{R} \subseteq V(G)$ of $n$ terminals such that $\mathcal{K}$ is defined by the terminals lying on the outer face of $G$. Moreover, let $|\mathcal{K}| = k$ and let $|\C{R}| = n-k = \OO(\log n)$. Then~\Cref{alg:smt_planar} computes the SMT for $\mathcal{K} \uplus \mathcal{R}$ in $G$ in time $N^3k^4n^{\OO(1)}$.
\end{corollary}

Next we state the FPTAS for \ESMT on $f(n)$-Almost Convex Sets of $n$  terminals. This is achieved by converting the instance of \ESMT into an instance of {\sc Steiner Minimal Tree} on graphs. The {\sc Steiner Minimal Tree} problem shall be solved using~\Cref{alg:smt_planar}. For this, we use the Algorithm 2 in~\cite{scott1988convexity}. We restate the Algorithm 2 in~\cite{scott1988convexity} for our problem instance. We denote the set of terminals with $\C{P}$.

\begin{algorithm}[H]
\caption{Computation of $(1+\epsilon)$-approximate SMT of $\C{P}$ ~~~ \textbf{Input:} $\mathcal P, \epsilon$ }\label{alg:smt_fptas}
\begin{algorithmic}[1]
\State Compute the convex hull of the set of terminals $\C{P}$. Let the region enclosed by the convex hull $\mathrm{CH}(\C{P})$ be denoted by $\mathbb{R}_{\mathrm{CH}(\C{P})}$. Let the points in $\C{P}$ lying on $\mathrm{CH}(\C{P})$ be $\C{K}$ and $\C{R} = \C{P} \setminus \C{K}$.
\State We enclose the set of terminals $\C{P}$ with the smallest axis-parallel bounding square. Let its side length be $D$. We divide the bounding square into same sized grids of side length $\frac{D\epsilon}{8n-12}$, where $\epsilon$ is the approximation factor.
\State Let $\C{V}_0$ be the set of all lattice points introduced in the previous step, and $\C{V}_1$ be the set of all lattice points lying on the edges of $\mathrm{CH}(\C{P})$. We define the weighted graph $G_{f,\epsilon}$ to be the complete graph with vertex set $V(G_{f,\epsilon}) = \C{K} \cup \C{R} \cup (\C{V}_0 \cap \mathbb{R}_{\mathrm{CH}(\C{P})}) \cup \C{V}_1$. The edge weights are equal to the Euclidean distance between the two end points.
\State Return the SMT $\C{T}$ for the graph $G_{f,\epsilon}$ with $\C{K} \cup \C{R}$ as the terminal set using~\Cref{alg:smt_planar}. 
\end{algorithmic}
\end{algorithm}

We analyse the correctness and running time of~\Cref{alg:smt_fptas}.
    \begin{theorem} \label{thm:algo2_correct}
     Consider a set $\C{P}$ of $n$ points such that $\mathcal{K}$ is defined as the points lying on the convex hull of $\C{P}$, i.e. $\mathrm{CH}(\C{P})$, and $\C{R} = \C{P} \setminus \C{K}$. Moreover, let $|\mathcal{K}| = k$. Then~\Cref{alg:smt_fptas} computes a $(1+\epsilon)$-approximate SMT for $\mathcal{P}$ in time $\OO(\frac{n^4k^4}{\epsilon ^4}4^{n-k})$.
    \end{theorem}
    \begin{proof}
{\bf Correctness of~\Cref{alg:smt_fptas}.} 
    In order to prove the correctness of~\Cref{alg:smt_fptas}, we need to show that $\C{T}$ is a $(1+\epsilon)$-approximation of the SMT of the terminal set $\C{P}$, and $\C{T}$ is indeed the SMT for $\C{K} \cup \C{R}$ in the complete weighted graph $G_{f,\epsilon}$.

        In~\cite{scott1988convexity}, the concept of weight planar graphs is used. A graph $G$ is called weight planar if it is a non-planar graph embedded on the Euclidean plane, having non-negative edge weights, such that every pair of edges $(u,v)$ and $(u',v')$ which intersect in this embedding of $G$, satisfy the inequality: $w(u,v) + w(u',v') > d(u,u') + d(v,v')$, where $w(u,v)$ is the weight of the edge between vertices $u$ and $v$ and $d(x,y)$ is the length of the shortest path between vertices $x$ and $y$. Since the edge weights of $G_{f,\epsilon}$ are the Euclidean distances between the points, $G_{f,\epsilon}$ is a weight planar graph. 

        From Theorem 5 of \cite{scott1988convexity}, we get that the SMT of a weight planar graph does not contain any crossing edges even though the input graph is non-planar. Because the SMT does not contain any crossing edges and lies completely inside the convex hull of the terminal pointset, the terminals on the outer boundary of $G_{f,\epsilon}$, i.e. $\C{K}$, follow the interval pattern as stated in~\Cref{prop:tree_interval}. Therefore,~\Cref{alg:smt_planar} designed for planar graphs can be applied in the case of weight planar graphs as well. So, $\C{T}$ is the SMT for $\C{K} \cup \C{R}$ in the complete weighted graph $G_{f,\epsilon}$.

        Finally, from Theorem 12 in~\cite{scott1988convexity}, we get that the length of the Steiner tree obtained from~\Cref{alg:smt_fptas} is at most $(1+\epsilon)$ times the length of the SMT $\C{T}^*$ of $\C{P}$, i.e. $|\C{T}| \leq (1+\epsilon)|\C{T}^*|$. Thus, we are done.
        
{\bf Running time of~\Cref{alg:smt_fptas}.}
    Constructing the convex hull takes $\OO(n\log n)$ time. The number of lattice points contained in the bounding box is $\big(\frac{8n-12}{\epsilon}\big)^2 = \OO(\frac{n^2}{\epsilon ^2})$. Thus, the number of vertices in the resultant graph $G_{f,\epsilon}$ is $N = \OO(\frac{n^2}{\epsilon ^2}) + n = \OO(\frac{n^2}{\epsilon ^2})$. The time complexity of~\Cref{alg:smt_planar} is $\OO(N^2k^44^{n-k}+Nk^33^{n-k}+N^3) = \OO(\frac{n^4k^4}{\epsilon ^4}4^{n-k})$. This step dominates the running time resulting in the complexity of~\Cref{alg:smt_fptas} being $\OO(\frac{n^4k^4}{\epsilon ^4}4^{n-k})$.
    %If $n-k=\OO(\log n)$, the running time becomes polynomial in $n$ and $k$. $k$ can be bounded by $n$, hence the time complexity becomes polynomial in $n$.
    \end{proof}

\begin{theorem}\label{thm:esmt_fptas}
    There exists an FPTAS for \ESMT on an $f(n)$-Almost Convex Set of $n$ terminals, where $f(n) = \OO (\log n)$.
\end{theorem}

\begin{proof}
    From~\Cref{thm:algo2_correct}, we get a $(1+\epsilon)$-approximate SMT for any $(n-k)$-Almost Convex Set of $n$ terminals in time $\OO(\frac{n^4k^4}{\epsilon ^4}4^{n-k})$. For $n-k = \OO(\log n)$, we get the running time of~\Cref{alg:smt_fptas} to be $\OO(\frac{n^4k^4}{\epsilon ^4}n^{\OO(1)})$. Thus,~\Cref{alg:smt_fptas} is an FPTAS for the Euclidean Steiner Minimal Tree problem on an $\OO (\log n)$-Almost Convex Set of $n$ terminals. 
\end{proof}

\subsection{Hardness of Approximation for \ESMT on Cases of Almost Convex Sets}\label{subsec:apx_hardness}

In this section, we consider the \ESMT problem on $f(n)$-Almost Convex Sets of $n$ terminal points, where $f(n) =\Omega(n^\epsilon)$ for some $\epsilon \in (0,1]$. We show that this problem cannot have an FPTAS. The proof strategy is similar to that in~\cite{garey1977complexity}. First, we give a reduction for the problem \textsc{Exact Cover by $3$-Sets} (defined below) to our problem to show that our problem is NP-hard. Next, we consider a discrete version of our problem and reduce our problem to the discrete version. The discrete version is in NP. Therefore, this chain of reductions imply that the discrete version of our problem is Strongly NP-complete and therefore cannot have an FPTAS, following from~\cite{garey1977complexity}. Similar to the arguments in~\cite{garey1977complexity}, this also implies that our problem cannot have an FPTAS.

Before we describe our reductions, we take a look at the NP-hardness reduction of the \ESMT problem from the \textsc{Exact Cover by 3-Sets} (X3C) problem in~\cite{garey1977complexity}. In the X3C problem, we are given a universe of elements $U = \{1, 2, \ldots, 3n\}$ and a family $\mathbb{F}$ of $3$-element subsets $F_1, F_2, \ldots, F_t$ of the $3n$ elements. The objective is to decide if there exists a subcollection $\mathbb{F}' \subseteq \mathbb{F}$ such that: (i) the elements of $\mathbb{F}'$ are disjoint, and (ii) $\bigcup_{F' \in \mathbb{F}'} F' = U$. The X3C problem is NP-complete~\cite{garey1979computers}.

In~\cite{garey1977complexity}, various gadgets are constructed, i.e.~particular arrangements of a set of points. These are then arranged on the plane in a way corresponding to the given X3C instance.~\Cref{fig:redcn} shows the reduced ESMT instance obtained for $U = \{1,2,3,4,5,6\}$ and $\mathbb{F} = \{\{1, 2, 4\}, \{2, 3, 6\}, \{3, 5, 6\}\}$ (taken from~\cite{garey1977complexity}). The squares, hexagons (crossovers), shaded circles (terminators) and lines (rows) all represent specific arrangements of a subset of points. Let $X(\mathbb{F})$ denote the reduced instance. The number of points in $X(\mathbb{F})$ is bounded by a polynomial in $n$ and $t$. Let this polynomial be $\OO(t^\gamma)$, as we can assume $t \geq n$ since otherwise it trivially becomes a NO instance. Here $\gamma$ is some constant.

\begin{figure}[h]
\centering
\subfloat{\includegraphics[width=12cm]{reduction_X3C_ESMT.png}}
\caption{Reduced instance of ESMT from X3C (taken from~\cite{garey1977complexity})}
\label{fig:redcn}
\end{figure}

We restate Theorem 1 in~\cite{garey1977complexity}.
\begin{proposition}\label{thm:redcn}
    Let $\mathcal{S}^{*}$ denote an SMT of $X(\mathbb{F})$, the instance obtained by reducing the X3C instance $(n,\mathbb{F})$, and $|\mathcal{S}^{*}|$ denote its length. If $\mathbb{F}$ has an exact cover, then $|\mathcal{S}^{*}| \leq f(n,t,\hat{C})$, otherwise $|\mathcal{S}^{*}| \geq f(n,t,\hat{C}) + \frac{1}{200nt}$, where $t = |\mathbb{F}|$, $\hat{C}$ is the number of crossovers, i.e.~hexagonal gadgets, and $f$ is a positive real-valued function of $n,t,\hat{C}$.
\end{proposition}

We extend this construction to prove NP-hardness for instances of \ESMT where the terminal set $\mathcal{P}$ has $\Omega(n^\epsilon)$ points inside $\mathrm{CH}(\mathcal{P})$. Here, $\epsilon \in (0,1]$ and $n$ is the number of terminals.

Let us call the \emph{length} of a gadget to be the maximum horizontal distance between any two points in that gadget. Similarly, we define the \emph{breadth} of a gadget to be the maximum vertical distance between any two points in that gadget.

% We get the following lemma from the construction given in~\cite{garey1977complexity}.
% \begin{lemma}\label{lem:bbox}
%     The smallest rectangle bounding all the points in $X(\mathbb{F})$ has side lengths of $\OO(t)$.
% \end{lemma}

% \begin{proof}
%     From the construction of the gadgets in~\cite{garey1977complexity}, we know that all of them have length and breadth as a function of $\epsilon = \frac{1}{200nt}$, which is bounded between 0 and 1. Thus, the length and breadth of each gadget can be bounded by some constant.

%     According to the reduction, the maximum number of gadgets along any horizontal level is $\OO(t)$. Thus, the length of the smallest bounding rectangle is $c_1t = \OO(t)$. Moreover, the maximum number of gadgets along any vertical column is $\OO(n) = \OO(t)$ as $t \geq n$. Thus, the breadth of the smallest bounding rectangle is $c_2t = \OO(t)$.
% \end{proof}

\begin{figure}[h] 
\centering
\subfloat[\centering The Upward Terminator symbol]{\includegraphics[width=1.65cm]{terminator_symbol.png}}   \qquad\qquad
\subfloat[\centering The Downward Terminator symbol]{\includegraphics[width=1.5cm]{terminator_down.png}}
    \qquad\qquad
\subfloat[\centering The Terminator gadget] {\includegraphics[width=5cm]{terminator_gadget.png} }
\caption{The Terminator gadget symbol and arrangement of points}  
\label{fig:terminator}
\end{figure}

The \emph{terminator} gadget used is shown in~\Cref{fig:terminator}. The straight lines represent a row of at least $1000$ points separated at distances of $1/10$ or $1/11$. The angles between them are as shown. The upward terminator has the point $A$ above the other points in the terminator, whereas the downward terminator has the point $A$ below the other points. Firstly, we adjust the number of points in the long rows, such that the length and breadth of the terminators is same as that of the hexagonal gadgets (crossovers). We can fix this length and breadth to be some constants, such that the number of points in each gadget is also bounded by some constant. In our construction, we modify the terminators $\Omega_0$, $\Omega_1$, and $\Omega_2$ as shown in~\Cref{fig:redcn} enclosed in squares. $\Omega_1$ is the terminator corresponding to the first occurrence of the element $3n\in U$ in some set in $\mathbb{F}$ and $\Omega_2$ is the terminator corresponding to the last occurrence of $3n$ in some set in $\mathbb{F}$ (if there are more than one occurrences of $3n$). If there are no occurrences of $3n$, then it is trivially a no-instance. The modified gadgets are shown in~\Cref{fig:terminator_new}. All the other gadgets remain unaltered.

\begin{figure}[h]
\centering
\includegraphics[width=8cm]{conic_set.png}
\caption{Conic Set: $\mathrm{Cone}(T,r,n)$}
\label{fig:conic_set}
\end{figure}

We call a set of points arranged as shown in~\Cref{fig:conic_set}, as a Conic Set.
\begin{definition} \label{def:conic_set}
    A Conic Set is a set of points consisting of a point $T$, called the tip of the cone, and the remaining points denoted by $\mathcal{S}$. Let $\mathcal{C}$ be the circle with $T$ as centre and radius $r$. All the points in $\mathcal{S}$ lie on $\mathcal{C}$, such that the angle at the tip formed by the two extreme points $L,R \in \mathcal{S}$, i.e.~$\angle{LTR} = 30^\circ$ in the anticlockwise direction. So, we have $\overline{TL} = \overline{TR} = r$. The distance between any two consecutive points in $\mathcal{S}$ is the same, say $d$. Let the number of points in $\mathcal{S}$ be $n$. We denote the Conic Set as $\mathrm{Cone}(T,r,n)$ and $\mathcal{S}$ as $\mathrm{Circ}(T,r,n)$. We call $TL$ as the left slope of the Conic Set and $TR$ as the right slope of the Conic Set.
\end{definition}

\begin{figure}[h] 
\centering
\subfloat[\centering $\Omega'_0$]{\includegraphics[width=5cm]{terminator_up_new.png}}   \qquad\qquad\qquad\qquad\qquad\qquad
\subfloat[\centering $\Omega'_1$]{\includegraphics[width=5cm]{omega_1.png}}
    \qquad\qquad
\subfloat[\centering $\Omega'_2$] {\includegraphics[width=5cm]{omega_2.png} }
\caption{The modified terminator gadgets}  
\label{fig:terminator_new}
\end{figure}


We use the Conic Set in the reduction for our problem. Now, we state the reduction of an X3C instance $(n,\mathbb{F})$ to an instance $X'(\mathbb{F},\epsilon)$ of \ESMT. Later, we show that the instance will satisfy the desired properties on the number of terminals inside the convex hull of the terminal set.

\begin{description}
\item [Algorithm $\mathcal{A}$ for construction of an ESMT instance $X'(\mathbb{F},\epsilon)$ from an X3C instance $(n,\mathbb{F})$:]
\hspace{0.1cm}
\begin{itemize}
    \item Reduce the input X3C instance to the points configuration $X(\mathbb{F})$ according to the reduction given in~\cite{garey1977complexity}.
    \item Modify the terminators $\Omega_0$, $\Omega_1$, and $\Omega_2$ to as shown in~\Cref{fig:terminator_new} and call them $\Omega'_0$, $\Omega'_1$, and $\Omega'_2$. Let $DQCP$ be the smallest axis-parallel rectangle bounding $X(\mathbb{F})$ after modifying the terminators, where $D$ is the bottom leftmost point of $\Omega'_0$.
    \item Take $\alpha = \frac{1}{\epsilon}$. Define $r = ct^{\alpha} = \OO(t^{\alpha})$ and $n' = c't^{\gamma\alpha} = \OO(t^{\gamma\alpha})$, where $t=|\mathbb{F}|$ and $c$ and $c'$ are constants. Add the $\mathrm{Cone}(D,r,n')$, such that $D$ is the tip of the Conic Set, and the left slope $DE$ makes an angle of $120^\circ$ with $DP$. The right slope $DF$ also makes an angle of $120^\circ$ with $DQ$.
    % \item Add the point $E$ on $\mathcal{C}$, such that $\angle{PDE} = 120^\circ$. Add the point $F$ such that $F$ lies on $\mathcal{C}$ and $\angle{EDF} = 30^\circ$ in anticlockwise direction.
    % \item Add $c't^{\gamma\alpha} = \OO(t^{\gamma\alpha})$ points on $\mathcal{C}$, where $c'$ is a constant, such that the distance between any two consecutive points is the same. Let us call these set of points on $\mathcal{C}$ along with $E$ and $F$ as $S$.
\end{itemize}
\end{description}

\begin{figure}[h]
\centering
\includegraphics[width=10cm]{redn_instance.png}
\caption{The reduced instance $X'(\mathbb{F},\epsilon)$}
\label{fig:redn_instance}
\end{figure}

% \begin{description}
% \item [Construction of $X'(\mathbb{F},\epsilon)$:]
% \hspace{10cm}
% \begin{itemize}
%     \item Reduce the input X3C instance to the point configuration $X(\mathbb{F})$ according to the reduction given in~\cite{garey1977complexity}.
%     \item Modify the terminator $\Omega_0$ to as shown in~\Cref{fig:terminator_new} and call it $\Omega'_0$.
%     \item Add a point $e$ at a distance of $r - D$ from the point $d$ of $\Omega'_0$ in the direction of the ray $\overrightarrow{\rm od}$. Add points on the line segment $\overline{\rm de}$ such that consecutive points are at a distance of $\frac{1}{10}$ from each other.
%     \item Take $\alpha = \frac{1}{\epsilon}$. From $e$, take two points $f$ and $g$, such that $\angle{\rm feg} = 120\degree$, $\overrightarrow{\rm oe}$ bisects $\angle{\rm feg}$ and $\overline{\rm fe} = \overline{\rm ge} = c't^{1-2\alpha}$. Consider the point $C$ at a distance of $D$ from the point $d$ of $\Omega'_0$ in the direction of the ray $\overrightarrow{\rm do}$. Let $R = \overline{\rm Cf} = \OO(t + t^{1-2\alpha}) = \OO(t)$. Taking $C$ as the centre, draw a circle of radius $R$. Let the circle be $\mathcal{T}$.
%     \item Add vertices on $\mathcal{T}$ such that the distance between two consecutive points is same as $\overline{\rm fg}$.
% \end{itemize}
% \end{description}

Now we prove a few properties of the constructed instance $X'(\mathbb{F},\epsilon)$.
\begin{lemma}\label{lem:convhull_pts}
    All the points in $\mathrm{Circ}(D,r,n')$ (according to~\Cref{def:conic_set}) lie on the convex hull of the reduced ESMT instance $X'(\mathbb{F},\epsilon)$ constructed by Algorithm $\mathcal{A}$, where $\epsilon \in (0,1]$.
\end{lemma}

\begin{proof}
    By the construction of $\mathrm{Cone}(D,r,n')$ in Algorithm $\mathcal{A}$, let $\mathcal{C}$ be the circle on which all the points in $\mathrm{Circ}(D,r,n')$ lie. If we draw a tangent to $\mathcal{C}$ at any of the points in $\mathrm{Circ}(D,r,n')$, then all the remaining points in the configuration $X'(\mathbb{F},\epsilon)$ lie towards one side of the tangent. We know that if we can find a line passing through a point such that all the other points in the plane lie on one side of the line, then the point lies on the convex hull of the points in the plane. Therefore, all the points in $\mathrm{Circ}(D,r,n')$ lie on the convex hull of the reduced instance $X'(\mathbb{F},\epsilon)$.
\end{proof}

Let us denote the convex hull of $X'(\mathbb{F},\epsilon)$ by $\mathrm{CH}(X'(\mathbb{F},\epsilon))$ and that of the points lying inside or on the bounding rectangle $\mathrm{PDQC}$, i.e.~$X'(\mathbb{F},\epsilon)\setminus \mathrm{Circ}(D,r,n')$ by $\mathrm{CH}(X'(\mathbb{F},\epsilon)\setminus \mathrm{Circ}(D,r,n'))$.

\begin{lemma}\label{lem:redcn_size}
    The reduced ESMT instance $X'(\mathbb{F},\epsilon)$ constructed by Algorithm $\mathcal{A}$ has $\Omega (N^\epsilon)$ points inside the convex hull, where $\epsilon \in (0,1]$ and $N$ is the total number of terminals in $X'(\mathbb{F},\epsilon)$. 
\end{lemma}

\begin{proof}
    $\mathrm{CH}(X'(\mathbb{F},\epsilon))$ contains all the points in $\mathrm{Circ}(D,r,n')$ by~\Cref{lem:convhull_pts}. $\mathrm{Circ}(D,r,n')$ contains $n' = \OO(t^{\gamma\alpha})$ points.

    Now we need to analyze the number of points on $\mathrm{CH}(X'(\mathbb{F},\epsilon)\setminus \mathrm{Circ}(D,r,n'))$. The remaining points in $X(\mathbb{F})$, i.e.~$X(\mathbb{F})\setminus \mathrm{CH}(X'(\mathbb{F},\epsilon)\setminus \mathrm{Circ}(D,r,n'))$ must lie within the convex hull of the entire construction, i.e.~$X'(\mathbb{F},\epsilon)$. From the construction in Algorithm $\mathcal{A}$, no point on the connecting rows can be a part of $\mathrm{CH}(X'(\mathbb{F},\epsilon)\setminus \mathrm{Circ}(D,r,n'))$ as there is no line passing through it, which contains all terminals on one side of it. The same thing holds for the square and hexagonal gadgets as well, except the hexagonal gadgets corresponding to the last element of the last set in the family, i.e.~$F_t$. Thus, only the terminators and the hexagonal gadgets corresponding to the last element of $F_t$ contribute points to $\mathrm{CH}(X'(\mathbb{F},\epsilon)\setminus \mathrm{Circ}(D,r,n'))$.

    If we look at the arrangement of points in the terminators (modified as well as those left unchanged) and the hexagonal gadgets as shown in~\Cref{fig:terminator,fig:hexagon_points}, the convex hull of each of these gadgets consists of constantly many points. Therefore, the number of points each of these gadgets contribute to $\mathrm{CH}(X'(\mathbb{F},\epsilon)\setminus \mathrm{Circ}(D,r,n'))$ is bounded by some constant. The number of terminators is $6t+2$ and the number of hexagonal gadgets corresponding to the last element of $F_t$ is at most $3n$. Therefore, the number of points on $\mathrm{CH}(X'(\mathbb{F},\epsilon)\setminus \mathrm{Circ}(D,r,n'))$ is $\OO(t+n) = \OO(t)$ as $n \leq t$.
    
    The instance $X(\mathbb{F})$ obtained via reduction from X3C has $6t+2$ terminators, $t$ squares, at most $9nt$ crossovers (hexagonal gadgets), and $\OO(nt)$ connecting rows of points. The number of gadgets is $\OO(nt)$. Therefore, the total number of points in $X(\mathbb{F})$ is $\Omega(nt) = \omega(t)$. The modified terminators result in a constantly many increase in the number of points. So, we have $\gamma > 1$.

    Thus, the number of points inside the convex hull is $\Omega(t^\gamma)$ and those on the convex hull is $\OO(t^{\gamma\alpha})$. So, the total number of terminals is $N = \OO(t^{\gamma\alpha}) + \OO(t^{\gamma}) = \OO(t^{\gamma\alpha})$, and those inside the convex hull is $\Omega(t^\gamma) = \Omega(N^{1/\alpha}) = \Omega(N^\epsilon)$ as $\alpha = \frac{1}{\epsilon}$.
\end{proof}

\begin{figure}[h]
\centering
\includegraphics[width=5cm]{hexagon_points.png}
\caption{The hexagonal gadget (crossover), the convex hull of the gadget is the quadrilateral $\rm abcd$ (taken from~\cite{garey1977complexity})}
\label{fig:hexagon_points}
\end{figure}

% \begin{lemma}\label{lem:redcn_size}
%     The reduced instance $X'(\mathbb{F},\epsilon)$ has $\OO (N^\epsilon)$ points inside the convex hull, where $\epsilon \in (0,1]$ and $N$ is the total number of terminals. 
% \end{lemma}

% \begin{proof}
%     The convex hull of $X'(\mathbb{F},\epsilon)$ configuration of points is the convex polygon inscribed in the circle $\mathcal{T}$ of radius $R = \OO(t)$. Distance between consecutive points is $\OO(t^{1-2\alpha})$. So, the convex hull contains $\OO(t^{2\alpha})$ points.
    
%     The instance $X(\mathbb{F})$ obtained via reduction from X3C has $6t+2$ terminators, $t$ squares, at most $9nt$ crossovers (hexagonal gadgets), and $\OO(nt)$ connecting rows of points. Each of these gadgets contain constantly many points. Therefore, the total number of points in $X(\mathbb{F})$ is $\OO(nt) = \OO(t^2)$. The modified terminator results in a constantly many increase in the number of points. Addition of points on the line segment $\overline{\rm de}$ results in $\OO(t)$ increase in the number of points inside as length of $\overline{\rm de} = D = \OO(t)$ and every pair of consecutive points are at a distance of $\frac{1}{10}$ from each other.

%     Thus, the number of points inside the convex hull is $\OO(t^2)$ and those on the convex hull is $\OO(t^{2\alpha})$. So, the total number of terminals is $N = \OO(t^{2\alpha})$, and those inside the convex hull is $\OO(t^2) = \OO(N^{1/\alpha}) = \OO(N^\epsilon)$ as $\alpha = \frac{1}{\epsilon}$.
% \end{proof}
We further prove structural properties of SMTs of the reduced instance $X'(\mathbb{F},\epsilon)$ when considering the modified gadgets $\Omega'_0$, $\Omega'_1$, and $\Omega'_2$.
\begin{lemma}\label{lem:modified_smt}
    Consider an SMT $\mathcal{S}^*$ of the ESMT instance $X(\mathbb{F})$ obtained via reduction from the X3C instance $(n,\mathbb{F})$ as per~\cite{garey1977complexity}. Consider a tree $\mathcal{S'}^{*}$ on the terminal set of $X'(\mathbb{F},\epsilon)$ obtained from $\mathcal{S}^*$ as follows: Consider the modified terminator gadgets $\Omega'_i,~i \in \{0,1,2\}$ as in Algorithm $\mathcal{A}$. For each $i \in \{0,1,2\}$, the edge $B_iO_i$ is excluded from $\mathcal{S}^*$ and the edge $D_iO_i$ is included to form $\mathcal{S'}^{*}$. $\mathcal{S'}^{*}$ is an SMT for the terminal set of $X'(\mathbb{F},\epsilon)$.
\end{lemma}

\begin{proof}
    Consider $\mathcal{S}^*$. Due to Lemma 4 of~\cite{garey1977complexity}, Steiner points of $\mathcal{S}^*$ can only be connected to points in the triangular and square gadgets. Lemma 5 of~\cite{garey1977complexity} states that if there are two terminals $x,y \in X(\mathbb{F})$ and the distance between $x$ and $y$ does not exceed $\frac{1}{10}$, then $(x,y)$ is an edge of $\mathcal{S}^{*}$. So in $\mathcal{S}^{*}$, for each $i \in \{0,1,2\}$ all the points on $B_iO_i$ are joined together along $B_iO_i$. Lemma 5 of~\cite{garey1977complexity} also holds true on modifying the terminators to $\Omega'_i,~i \in \{0,1,2\}$. Now, we join all the points on $D_iO_i$ along $D_iO_i$. This gives us the SMT $\mathcal{S'}^*$ for the terminal set of $X'(\mathbb{F},\epsilon)$.
\end{proof}

%A similar extension works for the modification of the terminators $\Omega_1$ and $\Omega_2$. Let this new SMT be represented by ${\mathcal{S}'}^*$ and its length by $|{\mathcal{S}'}^*|$.

Now we focus on the structure of the SMT of $X'(\C{F},\epsilon)$. The SMT is basically the union of the SMT $\C{S'}^{*}$ of the points in the bounding rectangle $PDQC$ as stated in~\Cref{lem:modified_smt} and the SMT of the set of points $\mathrm{Cone}(D,r,n')$.

% \begin{lemma}\label{lem:bounding_quad}
%     The rectangle $\mathrm{pdqC}$ as shown in~\Cref{fig:bounding_quad} encloses all the points in $X'(\C{F},\epsilon)$ located inside the convex hull.
% \end{lemma}

% \begin{proof}
%     We know that $D = \max\big(\frac{2c_1t}{\sqrt{3}}, 2c_2t\big)$. So, $\overline{\rm dq} = D\cos{30^\circ} = \max\big(c_1t, \sqrt{3}c_2t\big) \geq c_1t$. Similarly, $\overline{\rm Cq} = D\sin{30^\circ} = \max\big(\frac{c_1t}{\sqrt{3}}, c_2t\big) \geq c_2t$. Thus, the smallest bounding rectangle of the points in $X(\C{F})$ lies completely inside the bounding rectangle. So, all points in the configuration located inside the convex hull are enclosed by the bounding rectangle $\mathrm{pdqC}$.
% \end{proof}

% \begin{lemma}\label{lem:steiner_point_in_circle}
%     For any two points in or on the boundary of the bounding rectangle $\mathrm{peqC}$ in~\Cref{fig:bounding_quad}, if a Steiner point is adjacent to both of them, then the Steiner point must be located inside the circle centred at $C$ and radius $r = 2D$.
% \end{lemma}

% \begin{proof}
    
% \end{proof}

% For any SMT of $X'(\C{F},\epsilon)$, we call a \emph{\textbf{connecting path}} as any path which starts from some vertex inside the bounding rectangle and ends at a vertex on the convex hull, such that all internal vertices are Steiner points. Our analysis proceeds similar to that of the pair of concentric parallel regular polygons having more than 12 sides. The following lemma is analogous to~\Cref{left_right_turn_path,mincut_1}. 

% \begin{lemma}\label{lem:singly_connected}
%     For any SMT of $X'(\C{F},\epsilon)$, there exists a counter-clockwise or clockwise connecting path having at least one edge of length at least $\frac{R-r}{2}$. Also, there exists only one edge-disjoint connecting path.
% \end{lemma}

% \begin{proof}
%     Firstly, it can be easily seen that~\Cref{left_right_turn_path} can be extended to prove the existence of at least one counter-clockwise or clockwise connecting path.
    
%     Let $H$ be arbitrary point on the Euclidean plane. If $\mathcal C$ be a counter-clockwise path starting from $H$ such that no edge in the counter clockwise path has a length of more than $\ell$, for some $\ell \in \mathbb{R}^{+}$. Then we know that $\mathcal C$ is contained entirely in the circle centred at $H$ with radius $2\ell$.

%     The distance between a point in or on the bounding rectangle and a point on the convex hull is at least $R-D = r \gg D$ because the bounding rectangle is completely contained inside the circle with centre $C$ and radius $D$. So, all counter-clockwise or clockwise connecting paths must contain an edge of length at least $\frac{R-D}{2} \gg D$. 
    
%     If there are more than one edge-disjoint connecting paths, then we can always remove the edge of length at least $\frac{R-D}{2}$ from some counter-clockwise or clockwise connecting path and add another edge to get a smaller tree.

%     Observe that in the configuration, the distance between any two points in the bounding box is at most $D$. Also, the distance between two consecutive points on the convex hull is $\OO{t^{1-2\alpha}}$. When we remove one such edge of length $\OO(t^{2\alpha})$, either all the vertices on the convex hull are not connected or all the vertices inside the hull are not connected. In both cases, we can replace it by an edge of length at most $2\pi c$ or $r = 2D = \OO(t)$, respectively. This gives us a smaller tree, thereby contradicting the minimality of a tree containing two or more edge-disjoint connecting paths.
% \end{proof}

% Here, a \emph{\textbf{singly connected}} topology stands for one in which all connecting paths have at least one edge in common, i.e. there is only one edge-disjoint connecting path. The SMT shown in~\Cref{fig:smt_new} is a singly connected topology.

$\mathrm{CH}(X'(\mathbb{F},\epsilon)\setminus \mathrm{Circ}(D,r,n'))$ is enclosed by the bounding rectangle $PDQC$ and $D$ must lie on $\mathrm{CH}(X'(\mathbb{F},\epsilon)\setminus \mathrm{Circ}(D,r,n'))$. We label the vertices of $\mathrm{CH}(X'(\mathbb{F},\epsilon)\setminus \mathrm{Circ}(D,r,n'))$ as $D,P_1,P_2,\ldots,P_k$ in the counter-clockwise order. Let $\mathrm{CH}(X'(\mathbb{F},\epsilon))$ be the convex hull of all the points. By~\Cref{lem:convhull_pts}, all the points in $\mathrm{Circ}(D,r,n')$ lie on $\mathrm{CH}(X'(\mathbb{F},\epsilon))$. Let $EP_i$ and $FP_j$ be edges in $\mathrm{CH}(X'(\mathbb{F},\epsilon))$, such that $P_i,P_j \notin \mathrm{Circ}(D,r,n')$.

The SMT of $X'(\mathbb{F},\epsilon)$ clearly lies inside its convex hull, $\mathrm{CH}(X'(\mathbb{F},\epsilon))$. We show that the Steiner hull can be further restricted to the bounding rectangle $PDQC$ and the convex polygon formed by the points in $\mathrm{Cone}(D,r,n')$. For this we use Theorem 1.5 in~\cite{hwang1992steiner}, as stated below.

\begin{proposition}~\cite{hwang1992steiner}\label{thm:steiner_hull}
    Let $H$ be a Steiner hull of $N$. By sequentially removing wedges $\rm abc$ from the remaining region, where $\rm a$, $\rm b$, $\rm c$ are terminals but $\triangle{\rm abc}$ contains no other terminal, $a$ and $c$ are on the boundary and $\angle{\rm abc} \geq 120^\circ$, a Steiner hull $H'$ invariant to the sequence of removal is obtained.
\end{proposition}

\begin{lemma}\label{thm:steiner_hull_regions}
    The region comprising of the bounding rectangle $PDQC$ according to Algorithm $\mathcal{A}$ and the convex polygon formed by the set of points $\mathrm{Cone}(D,r,n')$ is a Steiner hull of $X'(\mathbb{F},\epsilon)$.
\end{lemma}

\begin{proof}
    Firstly, let us consider the wedge $EP_{i+1}P_i$. All the points are terminals, $E$ and $P_i$ are boundary points, and $\triangle{EP_{i+1}P_i}$ contains no other terminal. Now, $\angle{EP_{i+1}P_i}$ is greater than the exterior angle of $\angle{EP_{i+1}D}$, which in turn is greater than $\angle{EDP_{i+1}}$. $\angle{EDP_{i+1}} \geq \angle{\rm EDP} = 120^\circ$, by the construction. Therefore, $\angle{EP_{i+1}P_i} \geq 120^\circ$. By applying~\Cref{thm:steiner_hull}, we can remove the wedge $EP_{i+1}P_i$ from the convex hull $\mathrm{CH}(X'(\mathbb{F},\epsilon))$ to get a smaller Steiner hull. This can be repeated for the wedges $EP_{i+2}P_{i+1}, EP_{i+3}P_{i+2}, \ldots, EDP_k$. The same argument can also be used to get rid of the wedges $FP_{j-1}P_j, FP_{j-2}P_{j-1}, \ldots, FDP_1$. So, we get the final Steiner hull $H'$ to be the union of the bounding rectangle $PDQC$ and the convex polygon formed by the points in $\mathrm{Cone}(D,r,n')$.
\end{proof}

Given the nature of the above Steiner hull, we show that we can treat $X(\mathbb{F})$ and $\mathrm{Cone}(D,r,n')$ separately. 
\begin{lemma}\label{thm:steiner_tree}
    There is an SMT of $X'(\mathbb{F},\epsilon)$ that is the union of an SMT of $X(\mathbb{F})$ and an SMT of the points in $\mathrm{Cone}(D,r,n')$, with $D$ being common to both of them.
\end{lemma}

\begin{proof}
    According to~\Cref{thm:steiner_hull_regions}, there is an SMT of $X'(\mathbb{F},\epsilon)$ that lies completely inside the the bounding quadrilateral $PDQC$ and the convex polygon formed by $\mathrm{Cone}(D,r,n')$. These two regions have $D$ as the only common point. Therefore, $D$ is an articulation point in the tree and connects these two regions. So, we have this SMT of $X'(\mathbb{F},\epsilon)$ as the union of an SMT of $X(\mathbb{F})$ and an SMT of the points in $\mathrm{Cone}(D,r,n')$.
\end{proof}

We can identify a structure for an SMT of the points in $\mathrm{Cone}(D,r,n')$ using~\cite{weng1995steiner}.
\begin{lemma}\label{thm:steiner_tree_part}
    There is an SMT of the points in $\mathrm{Cone}(D,r,n')$ that is as shown in~\Cref{fig:steiner_new}. In the SMT, $D$ is connected to the two middle points in $\mathrm{Circ}(D,r,n')$ via a Steiner point $S^t$. The other points in $\mathrm{Circ}(D,r,n')$ are connected along the circumference.
\end{lemma}

\begin{proof}
    The number of points in $\mathrm{Circ}(D,r,n')$ is $\OO(t^{\gamma\alpha})$. We can take the constant factor to be large enough so that $|\mathrm{Circ}(D,r,n')| >= 12$. If we complete the regular polygon on $\mathcal{C}$ having $\mathrm{Circ}(D,r,n')$ as a subset of its vertices, then it contains more than $12$ vertices and along with the centre $D$ has a SMT with structure given in~\cite{weng1995steiner}.

    Let the Steiner tree for $\mathrm{Cone}(D,r,n')$ as shown in~\Cref{fig:steiner_new} be denoted by $\mathcal{T}_1$. If this is not minimal, then there exists another Steiner tree $\mathcal{T}_2$ such that $|\mathcal{T}_2| < |\mathcal{T}_1|$. Then we can replace $\mathcal{T}_1$ by $\mathcal{T}_2$ in the SMT of the regular polygon and its centre to get a shorter Steiner tree. This contradicts the minimality of the structure given in~\cite{weng1995steiner}. Therefore, the SMT of $\mathrm{Cone}(D,r,n')$ follows the structure in~\Cref{fig:steiner_new}.
\end{proof}

\begin{figure}[h]
\centering
\includegraphics[width=8cm]{steiner_new.png}
\caption{SMT of $\mathrm{Cone}(D,r,n')$}
\label{fig:steiner_new}
\end{figure}
Finally, we prove the NP-hardness of \ESMT on $f(n)$-Almost Convex Sets of $n$ terminals, when $f(n) = \Omega(n^\epsilon)$ for some $\epsilon \in (0,1]$.
\begin{theorem}\label{thm:redn_final}
    Let $\C{S}^{*}_{\mathbb{F},\epsilon}$ denote an SMT of $X'(\mathbb{F},\epsilon)$ and $|\C{S}^*_{\mathbb{F},\epsilon}|$ denote its length. If $\mathbb{F}$ has an exact cover, then $|\C{S}^{*}_{\mathbb{F},\epsilon}| \leq f(n,t,\hat{C}) + |\mathcal{T}_1|$, otherwise $|\C{S}^{*}_{\mathbb{F},\epsilon}| \geq f(n,t,\hat{C}) + \frac{1}{200nt} + |\mathcal{T}_1|$, where $\hat{C}$ is the number of crossovers, i.e.~hexagonal gadgets, and $f$ is a positive real-valued function of $n,t,\hat{C}$ as stated in~\Cref{thm:redcn}.
\end{theorem}

\begin{proof}
    From~\Cref{thm:steiner_tree}, we have $|\C{S}^{*}_{\mathbb{F},\epsilon}| = |\C{S}^{*}| + |\mathcal{T}_1|$. From~\Cref{thm:steiner_tree_part}, we can compute the length of $\mathcal{T}_1$ as a function of $t$, $\alpha$, and $\gamma$. Finally, using~\Cref{thm:redcn} we get the required reduction.
\end{proof}

%Thus, the \ESMT problem on $f(n)$-Almost Convex Sets of $n$ terminals, when $f(n) = \Omega (n^\epsilon)$ and where $\epsilon \in (0,1]$, is NP-Hard.

% Strongly NP-Complete problems do not have an FPTAS.
% Therefore, we get the following result.

% \begin{theorem}\label{thm:no_fptas}
%     There does not exist any FPTAS for the ESMT problem on $f(n)$-Almost Convex Sets of $n$ terminals, where $f(n) = \Omega(n^\epsilon)$ and $\epsilon \in (0,1]$.
% \end{theorem}

Since it is not known if the ESMT problem is in NP, Garey et al.~\cite{garey1977complexity} show the NP-completeness of a related problem called the {\sc Discrete Euclidean Steiner Minimal Tree} (DESMT) problem, which is in NP. We define the DESMT problem as given in~\cite{garey1977complexity}. The DESMT problem takes as input a set $\C{X}$ of integer-coordinate points in the plane and a positive integer $L$, and asks if there exists a set $\C{Y} \supseteq \C{X}$ of integer-coordinate points such that some spanning tree $\C{T}$ for $\C{Y}$ satisfies $|\C{T}|_d \leq L$, where $|\C{T}|_d = \Sigma_{e \in E(\mathcal T)} \lceil\overline{e}\rceil$, i.e.~we round up the length of each edge to the least integer not less than it.

In order to show that DESMT is NP-hard, the same reduction as that of the ESMT problem can be used, followed by scaling and rounding the coordinates of the points. Theorem 4 of~\cite{garey1977complexity} proves that the DESMT problem is NP-Complete. Moreover, since it is Strongly NP-Complete, the DESMT problem does not admit any FPTAS. Finally in Theorem 5 of~\cite{garey1977complexity}, Garey et al. show that as a consequence, the ESMT problem does not have any FPTAS as well.

Now we show that the DESMT problem is NP-hard even on $f(n)$-Almost Convex Sets of $n$ terminals, when $f(n) = \Omega (n^\epsilon)$ and where $\epsilon \in (0,1]$.

In Section 7 of~\cite{garey1977complexity}, the reduced instance $X(\mathbb{F})$ of ESMT is converted into an instance $X_{d}(\mathbb{F})$ of DESMT. The conversion is as follows:\\
$X_{d}(\mathbb{F}) = \{(\lceil 12M\cdot 200nt\cdot x_1\rceil, \lceil 12M\cdot 200nt\cdot x_2\rceil): x=(x_1,x_2)\in X(\mathbb{F})\}$, where $M = |X(\mathbb{F})|$.

We apply a similar conversion to the reduced ESMT instance $X'(\mathbb{F},\epsilon)$, to convert it into a DESMT instance of an $\Omega(n^\epsilon)$-Almost Convex Set. The conversion goes as follows:\\
$X'_{d}(\mathbb{F},\epsilon) = \{(\lceil 12N\cdot 200nt\cdot x_1\rceil, \lceil 12N\cdot 200nt\cdot x_2\rceil): x=(x_1,x_2)\in X'(\mathbb{F},\epsilon)\}$, where $N = |X'(\mathbb{F},\epsilon)|$.

The next two lemmas establish the validity of $X'_{d}(\mathbb{F},\epsilon)$ as an instance of DESMT and the upper bounds on the size of the constructed instance. Note that the reduction from X3C followed by the conversion can be done in polynomial time.
\begin{lemma}\label{lem:valid_desmt}
    The instance $X'_{d}(\mathbb{F},\epsilon)$ constructed above is a valid DESMT instance.
\end{lemma}

\begin{proof}
    All the points in $X'_{d}(\mathbb{F},\epsilon)$ have integer coordinates according to the conversion stated above. So, it is a DESMT instance.
\end{proof}

\begin{lemma}\label{lem:numpts_same}
    The reduced DESMT instance $X'_{d}(\mathbb{F},\epsilon)$ has $N$ distinct points, where $N = |X'(\mathbb{F},\epsilon)|$.
\end{lemma}

\begin{proof}
    The minimum distance between any two points in $X'(\mathbb{F},\epsilon)$ is that between two consecutive points of $\mathrm{Circ}(D,r,n')$, which is $\OO(\frac{1}{t^\gamma})$. Recall~\Cref{lem:redcn_size} which establishes that $N = \OO(t^{\gamma\alpha})$. So, the minimum distance between any two points in $X'_{d}(\mathbb{F},\epsilon)$ is $\OO(N \cdot nt \cdot \frac{1}{t^\gamma}) = \OO(nt^{\alpha+1})$. Because of the substantial distance obtained between points after scaling, the rounding will not cause any distinct points of $X'_{d}(\mathbb{F},\epsilon)$ to coincide. Therefore, the number of points remains unchanged, i.e.~$|X'_{d}(\mathbb{F},\epsilon)| = |X'(\mathbb{F},\epsilon)| = N$.
\end{proof}

Now we present the following lemma for the constructed DESMT instance $X'_{d}(\mathbb{F},\epsilon)$ analogous to~\Cref{lem:redcn_size} for the ESMT instance $X'(\mathbb{F},\epsilon)$.

\begin{lemma}\label{lem:desmt_redn_size}
    The reduced DESMT instance $X'_{d}(\mathbb{F},\epsilon)$ constructed is an $\Omega(N^\epsilon)$-Almost Convex Set, where  $N = |X'_{d}(\mathbb{F},\epsilon)|$.
\end{lemma}

\begin{proof}
    From~\Cref{lem:redcn_size}, we know that the reduced ESMT instance $X'(\mathbb{F},\epsilon)$ has $\Omega(N^\epsilon)$ points inside its convex hull $\mathrm{CH}(X'(\mathbb{F},\epsilon))$, and $N = |X'(\mathbb{F},\epsilon)|$. The number of points after conversion remains the same by~\Cref{lem:numpts_same}. We need to show that after conversion, except for the points of $\OO(t)$ gadgets, no other points inside the convex hull $\mathrm{CH}(X'(\mathbb{F},\epsilon))$ lie on the new convex hull $\mathrm{CH}(X'_{d}(\mathbb{F},\epsilon))$. The number of points in each of the $\OO(t)$ anomalous gadgets are constant in number, and hence not too many points from the interior of $\mathrm{CH}(X'(\mathbb{F},\epsilon))$ can lie on $\mathrm{CH}(X'_{d}(\mathbb{F},\epsilon))$. 
    
    After conversion, all the points on a horizontal connecting row have the same $y$-coordinate, as they initially had the same $y$-coordinate and therefore undergo the same transformation. Thus, all the points on a horizontal connecting row still lie on a horizontal line segment in $X'_{d}(\mathbb{F},\epsilon)$. Similarly, all the points on a vertical connecting row still lie on a vertical line segment in $X'_{d}(\mathbb{F},\epsilon)$. This implies that none of the points on the connecting rows can be a part of $\mathrm{CH}(X'_{d}(\mathbb{F},\epsilon))$ as there can be no line passing through them that also contains all terminal points on one side of it. 

    The same thing holds for the square and hexagonal gadgets (crossovers) as well, except the hexagonal gadgets placed at the beginning or end of any row. This is because all the points which are a part of these square and hexagon gadgets are surrounded by connecting row points all four sides, above, below, left and right. So again, only the terminators and the hexagonal gadgets appearing at the beginning or end of any row contribute to $\mathrm{CH}(X'_{d}(\mathbb{F},\epsilon))$.

    Now, since we had adjusted the number of points in the long rows of the terminators and hexagonal gadgets such that their lengths and breadths are some constants, the number of points in each of the terminators and hexagonal gadgets can be bounded by some constant as the minimum distance between any two consecutive points on the long rows or standard rows is at least $\frac{1}{11}$. Therefore, each of these gadgets contribute some constantly many points to $\mathrm{CH}(X'_{d}(\mathbb{F},\epsilon))$.
    
    As we have seen in~\Cref{lem:redcn_size}, the number of terminators is $6t+2$ and the number of hexagonal gadgets corresponding to the beginning or end of any row is at most $6n$. Therefore, the number of points contributed by the terminators and the hexagonal gadgets placed at the beginning or the end of any row, to $\mathrm{CH}(X'_{d}(\mathbb{F},\epsilon)$ is $\OO(t+n) = \OO(t)$, as $n \leq t$. Even if all the points in $\mathrm{Circ}(D,r,n')$ lie on the new convex hull $\mathrm{CH}(X'_{d}(\mathbb{F},\epsilon)$, we have $\Omega(t^\gamma) = \Omega(N^\epsilon)$ points inside it. Thus we are done.
\end{proof}

We get the following theorem from~\Cref{lem:valid_desmt,lem:numpts_same,lem:desmt_redn_size}.

\begin{theorem}\label{thm:desmt_redn}
    The instance $X'_{d}(\mathbb{F},\epsilon)$ constructed is a valid DESMT instance on an $\Omega(N^\epsilon)$-Almost Convex Set, where  $|X'_{d}(\mathbb{F},\epsilon)| = |X'(\mathbb{F},\epsilon)| = N$.
\end{theorem}

Following Theorems 3 and 4 in~\cite{garey1977complexity}, we get that the DESMT problem is NP-Complete for $\Omega(N^\epsilon)$-Almost Convex Sets, where $N$ is the total number of terminals. Since we get the reduced instance $X'_{d}(\mathbb{F},\epsilon)$ from the X3C instance $(n,\mathbb{F})$, the DESMT problem is strongly NP-complete for $\Omega(N^\epsilon)$-Almost Convex Sets, and does not admit any FPTAS.

Using Theorem 5 of~\cite{garey1977complexity}, we get that if the ESMT problem has an FPTAS, then the X3C problem can be solved in polynomial time. The Theorem also applies for our case of $\Omega(N^\epsilon)$-Almost Convex Sets. Therefore, we get the following theorem,

\begin{theorem}\label{thm:no_fptas}
    There does not exist any FPTAS for the ESMT problem on $f(n)$-Almost Convex Sets of $n$ terminals, where $f(n) = \Omega(n^\epsilon)$ and $\epsilon \in (0,1]$, unless \pnp.
\end{theorem}

\section{Conclusion}
 In this paper, we first study ESMT on vertices of $2$-CPR $n$-gons and design a polynomial time algorithm. It remains open to design a polynomial time algorithm for ESMT on $k$-CPR $n$-gons, or show NP-hardness for the problem. Next, we study the problem on $f(n)$-Almost Convex Sets of $n$ terminals. For this NP-hard problem, we obtain an algorithm that runs in $2^{\OO(f(n)\log n)}$ time. We also design an FPTAS when $f(n) = \OO(\log n)$. On the other hand, we show that there cannot be an FPTAS if $f(n) = \Omega(n^{\epsilon})$ for any $\epsilon \in (0,1]$, unless \pnp. The question of existence of FPTASes when $f(n)$ is a polylogarithmic function remains open.
%----------------------------------------------

%\bibliographystyle{plain}
\bibliography{refs}

\end{document}
