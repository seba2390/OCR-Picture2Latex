\documentclass[a4paper,UKenglish,cleveref, autoref, thm-restate]{lipics-v2021}
%This is a template for producing LIPIcs articles. 
%See lipics-v2021-authors-guidelines.pdf for further information.
%for A4 paper format use option "a4paper", for US-letter use option "letterpaper"
%for british hyphenation rules use option "UKenglish", for american hyphenation rules use option "USenglish"
%for section-numbered lemmas etc., use "numberwithinsect"
%for enabling cleveref support, use "cleveref"
%for enabling autoref support, use "autoref"
%for anonymousing the authors (e.g. for double-blind review), add "anonymous"
%for enabling thm-restate support, use "thm-restate"
%for enabling a two-column layout for the author/affilation part (only applicable for > 6 authors), use "authorcolumns"
%for producing a PDF according the PDF/A standard, add "pdfa"

%\graphicspath{{./graphics/}}%helpful if your graphic files are in another directory
\usepackage{xspace}
\bibliographystyle{plainurl}% the mandatory bibstyle

\title{Efficient Algorithms for \ESMT on Near-Convex Terminal Sets} %TODO Please add

\titlerunning{ESMT on Near-Convex Terminal Sets} %TODO optional, please use if title is longer than one line

\author{Anubhav Dhar}{Indian Institute of Technology Kharagpur, India}{anubhavdhar@kgpian.iitkgp.ac.in}{}{}%TODO mandatory, please use full name; only 1 author per \author macro; first two parameters are mandatory, other parameters can be empty. Please provide at least the name of the affiliation and the country. The full address is optional

\author{Soumita Hait}{Indian Institute of Technology Kharagpur, India}{soumitahait7321@gmail.com}{}{}

\author{Sudeshna Kolay}{Indian Institute of Technology Kharagpur, India}{skolay@cse.iitkgp.ac.in}{}{}

\authorrunning{A.~Dhar, S.~Hait, S.~Kolay} %TODO mandatory. First: Use abbreviated first/middle names. Second (only in severe cases): Use first author plus 'et al.'

\Copyright{} %TODO mandatory, please use full first names. LIPIcs license is "CC-BY";  http://creativecommons.org/licenses/by/3.0/

\ccsdesc{Theory of computation~Computational geometry}%TODO mandatory: Please choose ACM 2012 classifications from https://dl.acm.org/ccs/ccs_flat.cfm 

\keywords{Steiner minimal tree, Euclidean Geometry, Almost Convex point sets, FPTAS, strong NP-completeness}

\category{} %optional, e.g. invited paper

\relatedversion{} %optional, e.g. full version hosted on arXiv, HAL, or other respository/website
%\relatedversiondetails[linktext={opt. text shown instead of the URL}, cite=DBLP:books/mk/GrayR93]{Classification (e.g. Full Version, Extended Version, Previous Version}{URL to related version} %linktext and cite are optional

%\supplement{}%optional, e.g. related research data, source code, ... hosted on a repository like zenodo, figshare, GitHub, ...
%\supplementdetails[linktext={opt. text shown instead of the URL}, cite=DBLP:books/mk/GrayR93, subcategory={Description, Subcategory}, swhid={Software Heritage Identifier}]{General Classification (e.g. Software, Dataset, Model, ...)}{URL to related version} %linktext, cite, and subcategory are optional

%\funding{(Optional) general funding statement \dots}%optional, to capture a funding statement, which applies to all authors. Please enter author specific funding statements as fifth argument of the \author macro.

%\acknowledgements{I want to thank \dots}%optional

\nolinenumbers %uncomment to disable line numbering

\hideLIPIcs  %uncomment to remove references to LIPIcs series (logo, DOI, ...), e.g. when preparing a pre-final version to be uploaded to arXiv or another public repository

%Editor-only macros:: begin (do not touch as author)%%%%%%%%%%%%%%%%%%%%%%%%%%%%%%%%%%
\EventEditors{Petr A. Golovach and Meirav Zehavi}
\EventNoEds{2}
\EventLongTitle{16th International Symposium on Parameterized and Exact Computation (IPEC 2021)}
\EventShortTitle{IPEC 2021}
\EventAcronym{IPEC}
\EventYear{2021}
\EventDate{September 8--10, 2021}
\EventLocation{Lisbon, Portugal}
\EventLogo{}
\SeriesVolume{214}
\ArticleNo{2}
%%%%%%%%%%%%%%%%%%%%%%%%%%%%%%%%%%%%%%%%%%%%%%%%%%%%%%

\newcommand{\defparproblem}[4]{
  \vspace{1mm}
\noindent\fbox{
  \begin{minipage}{0.96\textwidth}
  \begin{tabular*}{\textwidth}{@{\extracolsep{\fill}}lr} #1  & {\bf{Parameter:}} #3 \\ \end{tabular*}
  {\bf{Input:}} #2  \\
  {\bf{Question:}} #4
  \end{minipage}
  }
  \vspace{1mm}
}


\newcommand{\defproblem}[3]{
  \vspace{1mm}
\noindent\fbox{
  \begin{minipage}{0.96\textwidth}
  \begin{tabular*}{\textwidth}{@{\extracolsep{\fill}}lr} #1 \\ \end{tabular*}
  {\bf{Input:}} #2  \\
  {\bf{Question:}} #3
  \end{minipage}
  }
  \vspace{1mm}
}

\newenvironment{problem}[1]{%
\begin{center}\fbox{\parbox{14cm}{%
    {\centering\scshape #1\par}%
    \parskip=1ex
    \everypar{\hangindent=1em}%
    \BODY
}}\end{center}}

\setlength\parindent{24pt}
\newcommand{\B}[1]{\mathbb{#1}}
\newcommand{\C}[1]{\mathcal{#1}}
\newcommand{\F}[1]{\mathfrak{#1}}
\newcommand{\SC}[1]{\mathscr{#1}}
\newcommand{\what}{\widehat}
\newcommand{\wtilde}{\widetilde}
\newcommand{\lbgt}{{\sf lbgt}}
\newcommand{\bgt}{{\sf bgt}}
\newcommand{\ubgt}{{\sf ubgt}}
\newcommand{\sumubgt}{{\sf sumub}}
\newcommand{\sumlbgt}{{\sf sumlb}}
\usepackage{todonotes}
\usepackage{algorithm}
\usepackage{algpseudocode}

\newcommand{\WOH}{\textsf{W[1]}-hard}
\newcommand{\BTJ}{\textsc{Bipartite Token Jumping}}
\newcommand{\MBB}{\textsc{Maximum Balanced Biclique}}
\newcommand{\optbiclq}{{\sf opt}\mbox{-}{\sf biclq}}
\newcommand{\BTS}{\textsc{Bipartite Half Subgraph}}
\newcommand{\BTIS}{\textsc{Bipartite Half Induced-Subgraph}}
\newcommand{\BTCS}{\textsc{Bipartite Half (Induced-)Subgraph}}
\newcommand{\TJ}{\textsc{Token Jumping}} 
\newcommand{\ESTP}{\textsc{Euclidean Steiner Tree}\xspace}
\newcommand{\ESMT}{\textsc{Euclidean Steiner Minimal Tree}\xspace}
\newcommand{\pnp}{$\mbox{P} = \mbox{NP}$\xspace}

\newcommand{\mc}{\mathcal}
\newcommand{\Oh}{\mathcal{O}}
\newcommand{\OO}{\mathcal{O}}
\newcommand{\Ohstar}{\Oh^\star}
\newcommand{\smtpoly}{SMT for $\{A_i\} \cup \{B_i\}$ \xspace}

 \definecolor{babyblue}{rgb}{0.54, 0.81, 0.94}
 \definecolor{b1}{rgb}{0.63, 0.79, 0.95}
 \definecolor{b2}{rgb}{0.74, 0.83, 0.9}
 \definecolor{b3}{rgb}{0.67, 0.9, 0.93}
 \definecolor{gentlegreen}{rgb}{0.00, 0.51, 0.00}
 

\begin{document}

\maketitle




  In this paper, we explore the connection between secret key agreement and secure omniscience within the setting of the multiterminal source model with a wiretapper who has side information. While the secret key agreement problem considers the generation of a maximum-rate secret key through public discussion, the secure omniscience problem is concerned with communication protocols for omniscience that minimize the rate of information leakage to the wiretapper. The starting point of our work is a lower bound on the minimum leakage rate for omniscience, $\rl$, in terms of the wiretap secret key capacity, $\wskc$. Our interest is in identifying broad classes of sources for which this lower bound is met with equality, in which case we say that there is a duality between secure omniscience and secret key agreement. We show that this duality holds in the case of certain finite linear source (FLS) models, such as two-terminal FLS models and pairwise independent network models on trees with a linear wiretapper. Duality also holds for any FLS model in which $\wskc$ is achieved by a perfect linear secret key agreement scheme. We conjecture that the duality in fact holds unconditionally for any FLS model. On the negative side, we give an example of a (non-FLS) source model for which duality does not hold if we limit ourselves to communication-for-omniscience protocols with at most two (interactive) communications.  We also address the secure function computation problem and explore the connection between the minimum leakage rate for computing a function and the wiretap secret key capacity.
  
%   Finally, we demonstrate the usefulness of our lower bound on $\rl$ by using it to derive equivalent conditions for the positivity of $\wskc$ in the multiterminal model. This extends a recent result of Gohari, G\"{u}nl\"{u} and Kramer (2020) obtained for the two-user setting.
  
   
%   In this paper, we study the problem of secret key generation through an omniscience achieving communication that minimizes the 
%   leakage rate to a wiretapper who has side information in the setting of multiterminal source model.  We explore this problem by deriving a lower bound on the wiretap secret key capacity $\wskc$ in terms of the minimum leakage rate for omniscience, $\rl$. 
%   %The former quantity is defined to be the maximum secret key rate achievable, and the latter one is defined as the minimum possible leakage rate about the source through an omniscience scheme to a wiretapper. 
%   The main focus of our work is the characterization of the sources for which the lower bound holds with equality \textemdash it is referred to as a duality between secure omniscience and wiretap secret key agreement. For general source models, we show that duality need not hold if we limit to the communication protocols with at most two (interactive) communications. In the case when there is no restriction on the number of communications, whether the duality holds or not is still unknown. However, we resolve this question affirmatively for two-user finite linear sources (FLS) and pairwise independent networks (PIN) defined on trees, a subclass of FLS. Moreover, for these sources, we give a single-letter expression for $\wskc$. Furthermore, in the direction of proving the conjecture that duality holds for all FLS, we show that if $\wskc$ is achieved by a \emph{perfect} secret key agreement scheme for FLS then the duality must hold. All these results mount up the evidence in favor of the conjecture on FLS. Moreover, we demonstrate the usefulness of our lower bound on $\wskc$ in terms of $\rl$ by deriving some equivalent conditions on the positivity of secret key capacity for multiterminal source model. Our result indeed extends the work of Gohari, G\"{u}nl\"{u} and Kramer in two-user case.
% !TEX root = ../arxiv.tex

Unsupervised domain adaptation (UDA) is a variant of semi-supervised learning \cite{blum1998combining}, where the available unlabelled data comes from a different distribution than the annotated dataset \cite{Ben-DavidBCP06}.
A case in point is to exploit synthetic data, where annotation is more accessible compared to the costly labelling of real-world images \cite{RichterVRK16,RosSMVL16}.
Along with some success in addressing UDA for semantic segmentation \cite{TsaiHSS0C18,VuJBCP19,0001S20,ZouYKW18}, the developed methods are growing increasingly sophisticated and often combine style transfer networks, adversarial training or network ensembles \cite{KimB20a,LiYV19,TsaiSSC19,Yang_2020_ECCV}.
This increase in model complexity impedes reproducibility, potentially slowing further progress.

In this work, we propose a UDA framework reaching state-of-the-art segmentation accuracy (measured by the Intersection-over-Union, IoU) without incurring substantial training efforts.
Toward this goal, we adopt a simple semi-supervised approach, \emph{self-training} \cite{ChenWB11,lee2013pseudo,ZouYKW18}, used in recent works only in conjunction with adversarial training or network ensembles \cite{ChoiKK19,KimB20a,Mei_2020_ECCV,Wang_2020_ECCV,0001S20,Zheng_2020_IJCV,ZhengY20}.
By contrast, we use self-training \emph{standalone}.
Compared to previous self-training methods \cite{ChenLCCCZAS20,Li_2020_ECCV,subhani2020learning,ZouYKW18,ZouYLKW19}, our approach also sidesteps the inconvenience of multiple training rounds, as they often require expert intervention between consecutive rounds.
We train our model using co-evolving pseudo labels end-to-end without such need.

\begin{figure}[t]%
    \centering
    \def\svgwidth{\linewidth}
    \input{figures/preview/bars.pdf_tex}
    \caption{\textbf{Results preview.} Unlike much recent work that combines multiple training paradigms, such as adversarial training and style transfer, our approach retains the modest single-round training complexity of self-training, yet improves the state of the art for adapting semantic segmentation by a significant margin.}
    \label{fig:preview}
\end{figure}

Our method leverages the ubiquitous \emph{data augmentation} techniques from fully supervised learning \cite{deeplabv3plus2018,ZhaoSQWJ17}: photometric jitter, flipping and multi-scale cropping.
We enforce \emph{consistency} of the semantic maps produced by the model across these image perturbations.
The following assumption formalises the key premise:

\myparagraph{Assumption 1.}
Let $f: \mathcal{I} \rightarrow \mathcal{M}$ represent a pixelwise mapping from images $\mathcal{I}$ to semantic output $\mathcal{M}$.
Denote $\rho_{\bm{\epsilon}}: \mathcal{I} \rightarrow \mathcal{I}$ a photometric image transform and, similarly, $\tau_{\bm{\epsilon}'}: \mathcal{I} \rightarrow \mathcal{I}$ a spatial similarity transformation, where $\bm{\epsilon},\bm{\epsilon}'\sim p(\cdot)$ are control variables following some pre-defined density (\eg, $p \equiv \mathcal{N}(0, 1)$).
Then, for any image $I \in \mathcal{I}$, $f$ is \emph{invariant} under $\rho_{\bm{\epsilon}}$ and \emph{equivariant} under $\tau_{\bm{\epsilon}'}$, \ie~$f(\rho_{\bm{\epsilon}}(I)) = f(I)$ and $f(\tau_{\bm{\epsilon}'}(I)) = \tau_{\bm{\epsilon}'}(f(I))$.

\smallskip
\noindent Next, we introduce a training framework using a \emph{momentum network} -- a slowly advancing copy of the original model.
The momentum network provides stable, yet recent targets for model updates, as opposed to the fixed supervision in model distillation \cite{Chen0G18,Zheng_2020_IJCV,ZhengY20}.
We also re-visit the problem of long-tail recognition in the context of generating pseudo labels for self-supervision.
In particular, we maintain an \emph{exponentially moving class prior} used to discount the confidence thresholds for those classes with few samples and increase their relative contribution to the training loss.
Our framework is simple to train, adds moderate computational overhead compared to a fully supervised setup, yet sets a new state of the art on established benchmarks (\cf \cref{fig:preview}).

%\documentclass[main]{subfiles}

\begin{document}

\section{Preliminaries}
\label{sec:preliminaries}
%\paragraph{Notation} 
\noindent
For $n \in \N$, we denote $[n] := \{1,\ldots,n\}$ and the vector with all ones as $\1_n \in \R^n$.
%\todo{Any vector $x \in \R^n$ is a column vector and its transpose is denoted by $x^T$. The entries of $x \in \R^n$ will be $x_1, \ldots, x_n$. Inequalities like $x \geq 0$ abbreviate the statement $\forall i \in [n] \, : \, x_i \geq 0$. For $i \in [n]$, we will use $e_i \in \R^n$ to denote the $i$-th standard basis vector (with a $1$ in its $i$-th entry and $0$'s anywhere else).} 
 
%\\
%\todo{When working with sets $\{S^i\}_{i=1}^N$, we denote $\displaystyle \bigtimes_{i=1}^N S^i := S^1 \times \ldots \times S^N$.} \\


\subsection{Multiplayer Games} 
A multiplayer game $G$ specifies (a) the number of players $N \in \N, N \geq 2,$ (b) a set of pure strategies $S^i = [m_i]$ for each player~$i$ where $m_i \in \N, m_i \geq 2,$ and (c) the utility payoffs for each player~$i$ given as a function $u_i: S = S^1 \times \ldots \times S^N \longrightarrow \R$. Throughout this paper, all multiplayer games considered shall have the same number of players $N$ and the same strategy sets $S^1, \ldots, S^N$. Hence, any game $G$ will be determined by its utility functions $\{u_i\}_{i \in [N]}$. The players choose their strategies simultaneously and they cannot communicate with each other. A utility function $u_i$ can be summarized by its pure strategy outcomes for player~$i$, captured as an $N$-dimensional array $\big\{ u_i(\ks) \big\}_{\ks \in S}$.

\begin{ex}
$2$-player games are better known as bimatrix games because their $2$-dimensional payoff arrays in become matrices $A,B \in \R^{m \times n}$.
\end{ex}

As usual, we allow the players to randomize over their pure strategies. Then, player~$i$'s strategy space extends to the set of probability distributions over $S^i$. We identify this set with $\Delta(S^i) := \, \Big\{ s^i = (s_k^i)_k \, \in \R^{m_i} \, \Big| \, s_k^i \geq 0 \, \, \forall k \in [m_i] \, \, \text{and} \, \sum_{k \in [m_i]} s_k^i = 1 \Big\}$ and refer to the probability distributions as mixed strategies. A tuple $\strats = (s^1, \ldots, s^N) \in \Delta(S^1) \times \ldots \times \Delta(S^N) =: \Delta(S)$ of mixed strategies is called a strategy profile in $G$\footnote{Note that in our notation, $\Delta(S)$ is not a simplex of higher dimensions itself but only the product of $N$ simplices.}. The utility payoff of player~$i$ for the strategy profile $\strats$ is defined as the player's utility payoff in expectation 
\[ u_i(\strats) := \sum_{\ks \in S} s_{k_1}^1 \cdot \ldots \cdot s_{k_N}^N \cdot u_i(\ks) \, .\]
The goal of each player is to maximize her utility.


We will abbreviate with $S^{-i}$ the set that consists of all possible pure strategy choices $\ks_{-i} = (k_1, \ldots, k_{i-1},k_{i+1}, \ldots, k_N)$ of the opponent players (resp. $\Delta(S^{-i})$ for the set of mixed strategy choices $\strats^{-i} = (s^1, \ldots, s^{i-1},s^{i+1}, \ldots, s^N)$). We will also often use $u_i(k_i,\ks_{-i})$ instead of $u_i(\ks)$ to stress how player~$i$ can only influence her own strategy when it comes to her payoff (resp. $u_i(s^i,\strats^{-i})$ instead of $u_i(\strats)$).
\begin{defn}
The best response set of player~$i$ to the opponents' strategy choices $\strats^{-i}$ is defined as $\BR_{u_i}(\strats^{-i}) :=  \argmax_{t^i \in \Delta(S^i)} \big\{ \, u_i(t^i,\strats^{-i}) \, \big\}$. 
\end{defn} 
Best response strategies capture the idea of optimal play against the other player's strategy choices. The most popular equilibrium concept in non-cooperative games is based on best responses.
\begin{defn}
A strategy profile $\strats \in \Delta(S)$ to a game $G = \{u_i\}_{i \in [N]}$ is called a \NE{} if for all player~$i \in [N]$ we have $s^i \in \BR_{u_i}(\strats^{-i})$.
\end{defn}
\noindent
By \cite{Nash48}, any multiplayer game $G$ admits at least one \NE{}.

\subsection{Positive Affine Transformations} 

The following lemma is a well-known result for $2$-player games\footnote{ See \cite[Lemma 2.1]{heyman}, \cite[Theorem 5.35]{maschler_solan_zamir_2013}, \cite[Chapter 3]{harsanyi1988general} or \cite[Proposition 3.1]{DynGT}.}:
\begin{lemma}
\label{PAT preserves lemma}
Let $(A,B)$ be a $m \times n$ bimatrix game and take arbitrary $\alpha, \beta >0$ and $c \in \R^n, d \in \R^m$. Define $A' = \alpha A + \1_m c^T$ and $B' = \beta B + d \1_n^T$.

Then the game $(A', B')$ has the same best response sets as the game $(A,B)$. Consequently, both games have the same \NE{} set.
\end{lemma}
The intuition behind this lemma is as follows:
player~$1$ wants to maximize her utility given the strategy that player~$2$ chose. A positive rescaling of $u_1$ will change the utility payoffs but will not change the sets of best response strategies. The same holds true if we add utility payoffs to $u_1$ that are only dependent on the strategy choice $s^2$ of her opponent. In the notation of bimatrix games, this intuition yields that the transformation $A \mapsto \alpha A + \1_m c^T$ does not affect the best response sets of player~$1$. The analogous result holds for player~$2$ and the transformation $B \mapsto \beta B + d \1_n^T$.

Let us generalize PATs to multiplayer games.
\begin{defn}
\label{multiplayer PAT defn}
A positive affine transformation (PAT) specifies for each player~$i$ a scaling parameter $\alpha^i \in \R, \alpha^i >0,$ and translation constants $C^i := ( c_{\ks_{-i}})_{\ks_{-i} \in S^{-i}}$ for each choice of pure strategies from the opponents. 
The PAT $H_{\textnormal{PAT}} = \big\{ \alpha^i, C^i \big\}_{i \in [N]}$ applied to an input game $G = \{u_i\}_{i \in [N]}$ returns the transformed game $H_{\textnormal{PAT}}(G) = \{u_i'\}_{i \in [N]}$ in which (only) the utility functions changed to
\begin{align}
\label{PAT transformed utilities}
\begin{aligned}
u_i' : S &\longrightarrow \R \\
\ks &\longmapsto \alpha_i \cdot u_i(\ks) + c_{\ks_{-i}}^i \, .
\end{aligned}
\end{align}
\end{defn}
We could not find multiplayer PATs defined in the literature, so we came up with the natural generalization above. As shown in Section \ref{sec:bimatrix games}, they indeed generalize the 2-player PATs from Lemma~\ref{PAT preserves lemma} to multiplayer settings. Moreover, multiplayer PATs also preserve the best response sets and \NE{} set.
\begin{lemma}
\label{multiplayer PAT preserves}
Take a PAT $H_{\textnormal{PAT}} = \big\{ \alpha^i, C^i \big\}_{i \in [N]}$ and any game $G = \{u_i\}_{i \in [N]}$. Then, the transformed game $H_{\textnormal{PAT}}(G) = \{u_i'\}_{i \in [N]}$ has the same best response sets as the input game $G$. Consequently, $H_{\textnormal{PAT}}(G)$ also has the same \NE{} set as $G$.
\end{lemma}
\begin{proof}
See \ref{sec:helpinglemmas}.
\end{proof}
PATs have found much success as a tool for simplifying an input game precisely because of this property. We want to investigate which other game transformations also preserve the best response sets or the \NE{} set. If we found more of these transformations, we could use them to, e.g., further increase the class of efficiently solvable games.

\subsection{Game Transformations}

There are various ways in which we could define the concept of a game transformation. Section~\ref{literature review} gives an overview of some definitions from the literature that are useful for different purposes. A key component of PATs are that they operate player-wise and strategy-wise, that is, they do not change the player set nor the players' strategy sets. This allows for a direct comparison of the strategic structure between a game and its PAT-transform. We argue that this is a natural desideratum for a definition of more general game transformation.

\begin{defn}
\label{def game trafo}
A game transformation $H = \{H^i\}_{i \in [N]}$ specifies for each player~$i$ a collection of functions $H^i := \Big\{ h_{\ks}^i : \R \longrightarrow \R \Big\}_{\ks \in S}$, indexed by the different pure strategy profiles $\ks$. \\
The transformation $H$ can then be applied to any $N$-player game $G = \{u_i\}_{i \in [N]}$ to construct the transformed game $H(G) = \{H^i(u_i)\}_{i \in [N]}$ where 
\begin{align}
\label{transformed pure utilities evaluation}
    H^i(u_i) : S \to \R, \quad \ks \mapsto h_{\ks}^i \big( u_i(\ks) \big) \, .
\end{align}
\end{defn}
Observe that the utility payoff of player~$i$ in the transformed game $H(G)$ from the pure strategy outcome $\ks$ is only a function of the utility payoff from \textit{that same} player in \textit{that same} pure strategy outcome of the input game~$G$.

We extend the utility functions $H^i(u_i)$ to mixed strategy profiles $\strats \in \Delta(S)$ as usual through $H^i(u_i)(\strats) := \sum_{\ks \in S} s_{k_1}^1 \cdot \ldots \cdot s_{k_N}^N \cdot h_{\ks}^i \big( u_i(\ks) \big)$. To simplify future notation, we will often use $h_{k_i,\ks_{-i}}^i$ to refer to $h_{\ks}^i$.

\begin{rem}
A multiplayer positive affine transformation $H_{\textnormal{PAT}} = \big\{ \alpha^i, C^i \big\}_{i \in [N]}$ makes a game transformation $H = \{H^i\}_{i \in [N]}$ in the above sense by setting $h_{\ks}^i : \, \, \R \to \R$, $z \mapsto \alpha^i  \cdot z + c_{\ks_{-i}}^i$.
\end{rem}
\begin{defn}
\label{defn NE preserving}
Let $H = \{H^i\}_{i \in [N]}$ be a game transformation. Then we say that $H$ universally preserves \NE{} sets if for all input games $G = \{u_i\}_{i \in [N]}$, the transformed game $H(G) = \{H^i(u_i)\}_{i \in [N]}$ has the same \NE{} set as the input game $G$.
\end{defn}

\begin{defn}
\label{defn BR preserving}
Let map $H^i$ come from a game transformation $H$. Then we say that $H^i$ universally preserves best responses if for all utility functions $u_i : S \longrightarrow \R$ and for all opponents' strategy choices $\strats^{-i} \in \Delta(S^{-i})$:
\begin{equation*}
\BR_{H^i(u_i)}(\strats^{-i}) = \argmax_{t^i \in \Delta(S^i)} \big\{ H^i(u_i)(t^i,\strats^{-i}) \big\} = \argmax_{t^i \in \Delta(S^i)} \big\{ u_i(t^i,\strats^{-i}) \big\} = \BR_{u_i}(\strats^{-i}) \, .
\end{equation*}
\end{defn}

\begin{defn}
\label{defn opponent dependence}
Let map $H^i$ come from a game transformation $H$. Then we say that $H^i$ only depends on the strategy choice of the opponents if for all pure strategy choices $\ks_{-i} \in S^{-i}$ of the opponents, we have the map identities
    \[h_{1, \ks_{-i}}^i = \ldots = h_{m_i, \ks_{-i}}^i\,: \R \to \R \, .\]
\end{defn}

\end{document}

%------------- Inculding files ----------------
%   In this paper, we explore the connection between secret key agreement and secure omniscience within the setting of the multiterminal source model with a wiretapper who has side information. While the secret key agreement problem considers the generation of a maximum-rate secret key through public discussion, the secure omniscience problem is concerned with communication protocols for omniscience that minimize the rate of information leakage to the wiretapper. The starting point of our work is a lower bound on the minimum leakage rate for omniscience, $\rl$, in terms of the wiretap secret key capacity, $\wskc$. Our interest is in identifying broad classes of sources for which this lower bound is met with equality, in which case we say that there is a duality between secure omniscience and secret key agreement. We show that this duality holds in the case of certain finite linear source (FLS) models, such as two-terminal FLS models and pairwise independent network models on trees with a linear wiretapper. Duality also holds for any FLS model in which $\wskc$ is achieved by a perfect linear secret key agreement scheme. We conjecture that the duality in fact holds unconditionally for any FLS model. On the negative side, we give an example of a (non-FLS) source model for which duality does not hold if we limit ourselves to communication-for-omniscience protocols with at most two (interactive) communications.  We also address the secure function computation problem and explore the connection between the minimum leakage rate for computing a function and the wiretap secret key capacity.
  
%   Finally, we demonstrate the usefulness of our lower bound on $\rl$ by using it to derive equivalent conditions for the positivity of $\wskc$ in the multiterminal model. This extends a recent result of Gohari, G\"{u}nl\"{u} and Kramer (2020) obtained for the two-user setting.
  
   
%   In this paper, we study the problem of secret key generation through an omniscience achieving communication that minimizes the 
%   leakage rate to a wiretapper who has side information in the setting of multiterminal source model.  We explore this problem by deriving a lower bound on the wiretap secret key capacity $\wskc$ in terms of the minimum leakage rate for omniscience, $\rl$. 
%   %The former quantity is defined to be the maximum secret key rate achievable, and the latter one is defined as the minimum possible leakage rate about the source through an omniscience scheme to a wiretapper. 
%   The main focus of our work is the characterization of the sources for which the lower bound holds with equality \textemdash it is referred to as a duality between secure omniscience and wiretap secret key agreement. For general source models, we show that duality need not hold if we limit to the communication protocols with at most two (interactive) communications. In the case when there is no restriction on the number of communications, whether the duality holds or not is still unknown. However, we resolve this question affirmatively for two-user finite linear sources (FLS) and pairwise independent networks (PIN) defined on trees, a subclass of FLS. Moreover, for these sources, we give a single-letter expression for $\wskc$. Furthermore, in the direction of proving the conjecture that duality holds for all FLS, we show that if $\wskc$ is achieved by a \emph{perfect} secret key agreement scheme for FLS then the duality must hold. All these results mount up the evidence in favor of the conjecture on FLS. Moreover, we demonstrate the usefulness of our lower bound on $\wskc$ in terms of $\rl$ by deriving some equivalent conditions on the positivity of secret key capacity for multiterminal source model. Our result indeed extends the work of Gohari, G\"{u}nl\"{u} and Kramer in two-user case.
% % \leavevmode
% \\
% \\
% \\
% \\
% \\
\section{Introduction}
\label{introduction}

AutoML is the process by which machine learning models are built automatically for a new dataset. Given a dataset, AutoML systems perform a search over valid data transformations and learners, along with hyper-parameter optimization for each learner~\cite{VolcanoML}. Choosing the transformations and learners over which to search is our focus.
A significant number of systems mine from prior runs of pipelines over a set of datasets to choose transformers and learners that are effective with different types of datasets (e.g. \cite{NEURIPS2018_b59a51a3}, \cite{10.14778/3415478.3415542}, \cite{autosklearn}). Thus, they build a database by actually running different pipelines with a diverse set of datasets to estimate the accuracy of potential pipelines. Hence, they can be used to effectively reduce the search space. A new dataset, based on a set of features (meta-features) is then matched to this database to find the most plausible candidates for both learner selection and hyper-parameter tuning. This process of choosing starting points in the search space is called meta-learning for the cold start problem.  

Other meta-learning approaches include mining existing data science code and their associated datasets to learn from human expertise. The AL~\cite{al} system mined existing Kaggle notebooks using dynamic analysis, i.e., actually running the scripts, and showed that such a system has promise.  However, this meta-learning approach does not scale because it is onerous to execute a large number of pipeline scripts on datasets, preprocessing datasets is never trivial, and older scripts cease to run at all as software evolves. It is not surprising that AL therefore performed dynamic analysis on just nine datasets.

Our system, {\sysname}, provides a scalable meta-learning approach to leverage human expertise, using static analysis to mine pipelines from large repositories of scripts. Static analysis has the advantage of scaling to thousands or millions of scripts \cite{graph4code} easily, but lacks the performance data gathered by dynamic analysis. The {\sysname} meta-learning approach guides the learning process by a scalable dataset similarity search, based on dataset embeddings, to find the most similar datasets and the semantics of ML pipelines applied on them.  Many existing systems, such as Auto-Sklearn \cite{autosklearn} and AL \cite{al}, compute a set of meta-features for each dataset. We developed a deep neural network model to generate embeddings at the granularity of a dataset, e.g., a table or CSV file, to capture similarity at the level of an entire dataset rather than relying on a set of meta-features.
 
Because we use static analysis to capture the semantics of the meta-learning process, we have no mechanism to choose the \textbf{best} pipeline from many seen pipelines, unlike the dynamic execution case where one can rely on runtime to choose the best performing pipeline.  Observing that pipelines are basically workflow graphs, we use graph generator neural models to succinctly capture the statically-observed pipelines for a single dataset. In {\sysname}, we formulate learner selection as a graph generation problem to predict optimized pipelines based on pipelines seen in actual notebooks.

%. This formulation enables {\sysname} for effective pruning of the AutoML search space to predict optimized pipelines based on pipelines seen in actual notebooks.}
%We note that increasingly, state-of-the-art performance in AutoML systems is being generated by more complex pipelines such as Directed Acyclic Graphs (DAGs) \cite{piper} rather than the linear pipelines used in earlier systems.  
 
{\sysname} does learner and transformation selection, and hence is a component of an AutoML systems. To evaluate this component, we integrated it into two existing AutoML systems, FLAML \cite{flaml} and Auto-Sklearn \cite{autosklearn}.  
% We evaluate each system with and without {\sysname}.  
We chose FLAML because it does not yet have any meta-learning component for the cold start problem and instead allows user selection of learners and transformers. The authors of FLAML explicitly pointed to the fact that FLAML might benefit from a meta-learning component and pointed to it as a possibility for future work. For FLAML, if mining historical pipelines provides an advantage, we should improve its performance. We also picked Auto-Sklearn as it does have a learner selection component based on meta-features, as described earlier~\cite{autosklearn2}. For Auto-Sklearn, we should at least match performance if our static mining of pipelines can match their extensive database. For context, we also compared {\sysname} with the recent VolcanoML~\cite{VolcanoML}, which provides an efficient decomposition and execution strategy for the AutoML search space. In contrast, {\sysname} prunes the search space using our meta-learning model to perform hyperparameter optimization only for the most promising candidates. 

The contributions of this paper are the following:
\begin{itemize}
    \item Section ~\ref{sec:mining} defines a scalable meta-learning approach based on representation learning of mined ML pipeline semantics and datasets for over 100 datasets and ~11K Python scripts.  
    \newline
    \item Sections~\ref{sec:kgpipGen} formulates AutoML pipeline generation as a graph generation problem. {\sysname} predicts efficiently an optimized ML pipeline for an unseen dataset based on our meta-learning model.  To the best of our knowledge, {\sysname} is the first approach to formulate  AutoML pipeline generation in such a way.
    \newline
    \item Section~\ref{sec:eval} presents a comprehensive evaluation using a large collection of 121 datasets from major AutoML benchmarks and Kaggle. Our experimental results show that {\sysname} outperforms all existing AutoML systems and achieves state-of-the-art results on the majority of these datasets. {\sysname} significantly improves the performance of both FLAML and Auto-Sklearn in classification and regression tasks. We also outperformed AL in 75 out of 77 datasets and VolcanoML in 75  out of 121 datasets, including 44 datasets used only by VolcanoML~\cite{VolcanoML}.  On average, {\sysname} achieves scores that are statistically better than the means of all other systems. 
\end{itemize}


%This approach does not need to apply cleaning or transformation methods to handle different variances among datasets. Moreover, we do not need to deal with complex analysis, such as dynamic code analysis. Thus, our approach proved to be scalable, as discussed in Sections~\ref{sec:mining}.
% %!TEX root = hopfwright.tex
%

In this section we systematically recast the Hopf bifurcation problem in Fourier space. 
We introduce appropriate scalings, sequence spaces of Fourier coefficients and convenient operators on these spaces. 
To study Equation~\eqref{eq:FourierSequenceEquation} we consider Fourier sequences $ \{a_k\}$ and fix a Banach space in which these sequences reside. It is indispensable for our analysis that this space have an algebraic structure. 
The Wiener algebra of absolutely summable Fourier series is a natural candidate, which we use with minor modifications. 
In numerical applications, weighted sequence spaces with algebraic and geometric decay have been used to great effect to study periodic solutions which are $C^k$ and analytic, respectively~\cite{lessard2010recent,hungria2016rigorous}. 
Although it follows from Lemma~\ref{l:analytic} that the Fourier coefficients of any solution decay exponentially, we choose to work in a space of less regularity. 
The reason is that by working in a space with less regularity, we are better able to connect our results with the global estimates in \cite{neumaier2014global}, see Theorem~\ref{thm:UniqunessNbd2}.


%
%
%\begin{remark}
%	Although it follows from Lemma~\ref{l:analytic} that the Fourier coefficients of any solution decay exponentially, we choose to work in a space of less regularity, namely summable Fourier coefficients. This allows us to draw SOME MORE INTERESTING CONCLUSION LATER.
%	EXPLAIN WHY WE CHOOSE A NORM WITH ALMOST NO DECAY!
%	% of s Periodic solutions to Wright's equation are known to be real analytic and so their  Fourier coefficients must decay geometrically [Nussbaum].
%	% We do not use such a strong result;  any periodic solution must be continuously differentiable, by which it follows that $ \sum | c_k| < \infty$.
%\end{remark}


\begin{remark}\label{r:a0}
There is considerable redundancy in Equation~\eqref{eq:FourierSequenceEquation}. First, since we are considering real-valued solutions $y$, we assume $\c_{-k}$ is the complex conjugate of $\c_k$. This symmetry implies it suffices to consider Equation~\eqref{eq:FourierSequenceEquation} for $k \geq 0$.
Second, we may effectively ignore the zeroth Fourier coefficient of any periodic solution \cite{jones1962existence}, since it is necessarily equal to $0$. 
%In \cite{jones1962existence}, it is shown that if $y \not\equiv -1$ is a periodic solution of~\eqref{eq:Wright} with frequency $\omega$, then $ \int_0^{2\pi/\omega} y(t) dt =0$. 
		The self contained argument is as follows. 
		As mentioned in the introduction, any periodic solution to Wright's equation must satisfy $ y(t) > -1$ for all $t$. 
	By dividing Equation~\eqref{eq:Wright} by $(1+y(t))$, which never vanishes, we obtain
	\[
	\frac{d}{dt} \log (1 + y(t)) = - \alpha y(t-1).
	\]  
	Integrating over one period $L$ we derive the condition 
	$0=\int_0^L y(t) dt $.
	Hence $a_0=0$ for any periodic solution. 
	It will be shown in Theorem~\ref{thm:FourierEquivalence1} that a related argument implies that we do not need to consider Equation~\eqref{eq:FourierSequenceEquation} for $k=0$.
\end{remark}

%%%
%%%
%%%\begin{remark}\label{r:c0} 
%%%In \cite{jones1962existence}, it is shown that if $y \not\equiv -1$ is a periodic solution of~\eqref{eq:Wright} with frequency $\omega$, then $ \int_0^{2\pi/\omega} y(t) dt =0$. 
%%%PERHAPS TOO MUCH DETAIL HERE. The self contained argument is as follows.
%%%If $y \not\equiv -1$ then $y(t) \neq -1$ for all $t$, since if $y(t_0)=-1$ for some $t_0 \in \R$ then $y'(t_0)=0$ by~\eqref{eq:Wright} and in fact by differentiating~\eqref{eq:Wright} repeatedly one obtains that all derivatives of $y$ vanish at $t_0$. Hence $y \equiv -1$ by Lemma~\ref{l:analytic}, a contradiction. Now divide~\eqref{eq:Wright} by $(1+y(t))$, which never vanishes, to obtain
%%%\[
%%%  \frac{d}{dt} \log |1 + y(t)| = - \alpha y(t-1).
%%%\]  
%%%Integrating over one period we obtain $\int_0^L y(t) dt =0$.
%%%\end{remark}



%Furthermore, the condition that $y(t)$ is real forces $\c_{-k} = \overline{\c}_{k}$.  
%
We define the spaces of absolutely summable Fourier series
\begin{alignat*}{1}
	\ell^1 &:= \left\{ \{ \c_k \}_{k \geq 1} : 
    \sum_{k \geq 1} | \c_k| < \infty  \right\} , \\
	\ell^1_\bi &:= \left\{ \{ \c_k \}_{k \in \Z} : 
    \sum_{k \in \Z} | \c_k| < \infty  \right\} .
\end{alignat*} 
We identify any semi-infinite sequence $ \{ \c_k \}_{k \geq 1} \in \ell^1$ with the bi-infinite sequence $ \{ \c_k \}_{k \in \Z} \in \ell^1_\bi$ via the conventions (see Remark~\ref{r:a0})
\begin{equation}
  \c_0=0 \qquad\text{ and }\qquad \c_{-k} = \c_{k}^*. 
\end{equation}
In other word, we identify $\ell^1$ with the set
\begin{equation*}
   \ell^1_\sym := \left\{ \c \in \ell^1_\bi : 
	\c_0=0,~\c_{-k}=\c_k^* \right\} .
\end{equation*}
On $\ell^1$ we introduce the norm
\begin{equation}\label{e:lnorm}
  \| \c \| = \| \c \|_{\ell^1} := 2 \sum_{k = 1}^\infty |\c_k|.
\end{equation}
The factor $2$ in this norm is chosen to have a Banach algebra estimate.
Indeed, for $\c, \tilde{\c} \in \ell^1 \cong \ell^1_\sym$ we define
the discrete convolution 
\[
\left[ \c * \tilde{\c} \right]_k = \sum_{\substack{k_1,k_2\in\Z\\ k_1 + k_2 = k}} \c_{k_1} \tilde{\c}_{k_2} .
\]
Although $[\c*\tilde{\c}]_0$ does not necessarily vanish, we have $\{\c*\tilde{\c}\}_{k \geq 1} \in \ell^1 $ and 
\begin{equation*}
	\| \c*\tilde{\c} \| \leq \| \c \| \cdot  \| \tilde{\c} \| 
	\qquad\text{for all } \c , \tilde{\c} \in \ell^1, 
\end{equation*}
hence $\ell^1$ with norm~\eqref{e:lnorm} is a Banach algebra.

By Lemma~\ref{l:analytic} it is clear that any periodic solution of~\eqref{eq:Wright} has a well-defined Fourier series $\c \in \ell^1_\bi$. 
The next theorem shows that in order to study periodic orbits to Wright's equation we only need to study Equation~\eqref{eq:FourierSequenceEquation} 
for $k \geq 1$. For convenience we introduce the notation 
\[
G(\alpha,\omega,\c)_k=
( i \omega k + \alpha e^{ - i \omega k}) \c_k + \alpha \sum_{k_1 + k_2 = k} e^{- i \omega k_1} \c_{k_1} \c_{k_2} \qquad \text{for } k \in \N.
\]
We note that we may interpret the trivial solution $y(t)\equiv 0$ as a periodic solution of arbitrary period.
\begin{theorem}
\label{thm:FourierEquivalence1}
Let $\alpha>0$ and $\omega>0$.
If $\c \in \ell^1 \cong \ell^1_{\sym}$ solves
$G(\alpha,\omega,\c)_k =0$  for all $k \geq 1$,
then $y(t)$ given by~\eqref{eq:FourierEquation} is a periodic solution of~\eqref{eq:Wright} with period~$2\pi/\omega$.
Vice versa, if $y(t)$ is a periodic solution of~\eqref{eq:Wright} with period~$2\pi/\omega$ then its Fourier coefficients $\c \in \ell^1_\bi$ lie in $\ell^1_\sym \cong \ell^1$ and solve $G(\alpha,\omega,\c)_k =0$ for all $k \geq 1$.
\end{theorem}

\begin{proof}	
	If $y(t)$ is a periodic solution of~\eqref{eq:Wright} then it is real analytic by Lemma~\ref{l:analytic}, hence its Fourier series $\c$ is well-defined and $\c \in \ell^1_{\sym}$ by Remark~\ref{r:a0}.
	Plugging the Fourier series~\eqref{eq:FourierEquation} into~\eqref{eq:Wright} one easily derives that $\c$ solves~\eqref{eq:FourierSequenceEquation} for all $k \geq 1$.

To prove the reverse implication, assume that $\c \in \ell^1_\sym$ solves
Equation~\eqref{eq:FourierSequenceEquation} for all $k \geq 1$. Since $\c_{-k}
= \c_k^*$, Equation \eqref{eq:FourierSequenceEquation} is also satisfied for
all $k \leq -1$. It follows from the Banach algebra property and
\eqref{eq:FourierSequenceEquation} that $\{k \c_k\}_{k \in \Z} \in \ell^1_\bi$,
hence $y$, given by~\eqref{eq:FourierEquation}, is continuously differentiable.
% (and by bootstrapping one infers that $\{k^m c_k \} \in \ell^1_\bi$, 
% hence $y \in C^m$ for any $m \geq 1$).
	Since~\eqref{eq:FourierSequenceEquation} is satisfied for all $k \in \Z \setminus \{0\}$ (but not necessarily for $k=0$) one may perform the inverse Fourier transform on~\eqref{eq:FourierSequenceEquation} to conclude that
	$y$ satisfies the delay equation 
\begin{equation}\label{eq:delaywithK}
   	y'(t) = - \alpha y(t-1) [ 1 + y(t)] + C
\end{equation}
	for some constant $C \in \R$. 
   Finally, to prove that $C=0$ we argue by contradiction.
   Suppose $C \neq 0$. Then $y(t) \neq -1$ for all $t$.
   Namely, at any point where $y(t_0) =-1$ one would have $y'(t_0) = C$
   which has fixed sign,   hence it would follow that $y$ is not periodic
   ($y$ would not be able to cross $-1$ in the opposite direction, 
   preventing $y$  from being periodic).  
  We may thus divide~\eqref{eq:delaywithK} through by $1 + y(t)$ and obtain 
\begin{equation*}
	\frac{d}{dt} \log | 1 + y(t) | = - \alpha y(t-1) + \frac{C}{1+y(t)} .
\end{equation*}
	By integrating both sides of the equation over one period $L$ and by using that $\c_0=0$, we 
	obtain
	\[
	 C \int_0^L \frac{1}{1+y(t)} dt =0.
	\]
	Since the integrand is either strictly negative or strictly positive, this implies that $C=0$. Hence~\eqref{eq:delaywithK} reduces to~\eqref{eq:Wright},
	and $y$ satisfies Wright's equation. 
\end{proof}






To efficiently study Equation~\eqref{eq:FourierSequenceEquation}, we introduce the following linear operators on $ \ell^1$:
\begin{alignat*}{1}
   [K \c ]_k &:= k^{-1} \c_k  , \\ 
   [ U_\omega \c ]_k &:= e^{-i k \omega} \c_k  .
\end{alignat*}
The map $K$ is a compact operator, and it has a densely defined inverse $K^{-1}$. The domain of $K^{-1}$ is denoted by
\[
  \ell^K := \{ \c \in \ell^1 : K^{-1} \c \in \ell^1 \}.  
\]
The map $U_{\omega}$ is a unitary operator on $\ell^1$, but
it is discontinuous in $\omega$. 
With this notation, Theorem~\ref{thm:FourierEquivalence1} implies that our problem of finding a SOPS to~\eqref{eq:Wright} is equivalent to finding an $\c \in \ell^1$ such that
\begin{equation}
\label{e:defG}
  G(\alpha,\omega,\c) :=
  \left( i \omega K^{-1} + \alpha U_\omega \right) \c + \alpha \left[U_\omega \, \c \right] * \c  = 0.
\end{equation}


%In order for the solutions of Equation \ref{eq:FHat} to be isolated we need to impose a phase condition. 
%If there is a sequence $ \{ c_k \} $ which satisfies  Equation \ref{eq:FHat}, then $ y( t + \tau) = \sum_{k \in \Z} c_k e^{ i k \omega (t + \tau)}$ satisfies Wright's equation at parameter $\alpha$. 
%Fix $ \tau = - Arg[c_1] / \omega$ so that $ c_1  e^{ i \omega \tau} $ is a nonnegative real number. 
%By Proposition \ref{thm:FourierEquivalence1} it follows that $\{ c'_k \} =  \{c_k e^{ i \omega k \tau }   \}$ is a solution to Equation \ref{eq:FHat}, and furthermore that $ c'_1 = \epsilon$ for some $ \epsilon \geq 0$. 


Periodic solutions are invariant under time translation: if $y(t)$ solves Wright's equation, then so does $ y(t+\tau)$ for any $\tau \in \R$. 
We remove this degeneracy by adding a phase condition. 
Without loss of generality, if $\c \in \ell^1$ solves Equation~\eqref{e:defG}, we may assume that $\c_1 = \epsilon$ for some 
\emph{real non-negative}~$\epsilon$:
\[
  \ell^1_{\epsilon} := \{\c \in \ell^1 : \c_1 = \epsilon \} 
  \qquad \text{where } \epsilon \in \R,  \epsilon \geq 0.
\]
In the rest of our analysis, we will split elements $\c \in \ell^1$ into two parts: $\c_1$ and $\{\c_{k}\}_{k \geq 2}$.  
We define the basis elements $\e_j \in \ell^1$ for $j=1,2,\dots$ as
\[
  [\e_j]_k = \begin{cases}
  1 & \text{if } k=j, \\
  0 & \text{if } k \neq j.
  \end{cases}
\]
We note that $\| \e_j \|=2$. 
Then we can decompose
% We define
% \[
%   \tilde{\epsilon} := (\epsilon,0,0,0,\dots) \in \ell^1
% \]
% and
% For clarity when referring to sequences $\{c_{k}\}_{k \geq 2}$, we make the following definition:
% \[
% \ell^1_0  := \{ \tc \in \ell^1 : \tc_1 = 0 \}.
% \]
% With the
any $\c \in \ell^1_\epsilon$ uniquely as
\begin{equation}\label{e:aepsc}
  \c= \epsilon \e_1 + \tc \qquad \text{with}\quad 
  \tc \in \ell^1_0 := \{ \tc \in \ell^1 : \tc_1 = 0 \}.
\end{equation}
We follow the classical approach in studying Hopf bifurcations and consider 
$\c_1 = \epsilon$ to be a parameter, and then find periodic solutions with Fourier modes in $\ell^1_{\epsilon}$.
This approach rewrites the function $G: \R^2 \times \ell^K \to \ell^1$ as a function $\tilde{F}_\epsilon : \R^2 \times \ell^K_0 \to \ell^1$, where 
we denote 
\[
\ell^K_0 := \ell^1_0 \cap \ell^K.
\]
% I AM ACTUALLY NOT SURE IF YOU WANT TO DEFINE THIS WITH RANGE IN $\ell^1$
% OR WITH DOMAIN IN $\ell^1_0$ ?? IT SEEMS TO DEPEND ON WHICH GLOBAL STATEMENT YOU WANT/NEED TO MAKE!?
\begin{definition}
We define the $\epsilon$-parameterized family of  functions $\tilde{F}_\epsilon: \R^2 \times \ell^K_0  \to \ell^1$ 
by 
\begin{equation}
\label{eq:fourieroperators}
\tilde{F}_{\epsilon}(\alpha,\omega, \tc) := 
\epsilon [i \omega + \alpha e^{-i \omega}] \e_1 + 
( i \omega K^{-1} + \alpha U_{\omega}) \tc + 
\epsilon^2 \alpha e^{-i \omega}  \e_2  +
\alpha \epsilon L_\omega \tc + 
\alpha  [ U_{\omega} \tc] * \tc ,
\end{equation}
where
$L_\omega : \ell^1_0 \to \ell^1$ is given by
\[
   L_{\omega} := \sigma^+( e^{- i \omega} I + U_{\omega}) + \sigma^-(e^{i \omega} I + U_{\omega}),
\]
with $I$ the identity and  $\sigma^\pm$ the shift operators on $\ell^1$:
\begin{alignat*}{2}
\left[ \sigma^- a \right]_k &:=  a_{k+1}  , \\
\left[ \sigma^+ a \right]_k &:=  a_{k-1}  &\qquad&\text{with the convention } \c_0=0.
\end{alignat*}
The operator $ L_\omega$ is discontinuous in $\omega$ and $ \| L_\omega \| \leq 4$. 
\end{definition} 

%The maps $ \sigma^{+}$ and $ \sigma^-$ are shift up and shift down operators respectively. 
We reformulate Theorem~\ref{thm:FourierEquivalence1}  in terms of the map  $\tilde{F}$. 
We note that it follows from Lemma~\ref{l:analytic} and 
%\marginpar{Reformulate}
%one's choice of  
Equation~\eqref{eq:FourierSequenceEquation}  
%or Equation ~\eqref{eq:fourieroperators},
that the Fourier coefficients of any periodic solution of~\eqref{eq:Wright} lie in $\ell^K$.
These observations are summarized in the following theorem.
\begin{theorem}
\label{thm:FourierEquivalence2}
	Let $ \epsilon \geq 0$,  $\tc \in \ell^K_0$, $\alpha>0$ and $ \omega >0$. 
	Define $y: \R\to \R$ as 
\begin{equation}\label{e:ytc}
	y(t) = 
	\epsilon \left( e^{i \omega t }  + e^{- i \omega t }\right) 
	+  \sum_{k = 2}^\infty   \tc_k e^{i \omega k t }  + \tc_k^* e^{- i \omega k t } .
\end{equation}
%	and suppose that $ y(t) > -1$. 
	Then $y(t)$ solves~\eqref{eq:Wright} if and only if $\tilde{F}_{\epsilon}( \alpha , \omega , \tc) = 0$. 
	Furthermore, up to time translation, any periodic solution of~\eqref{eq:Wright} with period $2\pi/\omega$ is described by a Fourier series of the form~\eqref{e:ytc} with $\epsilon \geq 0$ and $\tc \in \ell^K_0$.
\end{theorem}


%We note that for $\epsilon>0$ such solutions are truly periodic, while for $\epsilon=0$ a zero of $\tilde{F}_\epsilon$ may either correspond to a periodic solution or to the trivial solution $y(t) \equiv 0$. 



% \begin{proof}
%  By Proposition \ref{thm:FourierEquivalence1}, it suffices to show that $\tilde{F}(\alpha,\omega,c) =0$ is equivalent to Equation \ref{eq:FourierSequenceEquation} being satisfied for $k \geq 1$.
%  Since Equation \ref{eq:FourierSequenceEquation} is equivalent to Equation \ref{eq:FHat}, we expand  Equation \ref{eq:FHat} by writing $ \hat{c} = \hat{\epsilon } + c$  where $ \hat{\epsilon} := (\epsilon,0,0,\dots) \in \ell^1$ as below:
%  \begin{equation}
%  0=  \left( i \omega K^{-1} + \alpha U_\omega \right) (\hat{\epsilon}+ c) + \alpha \left[U_\omega \, (\hat{\epsilon}+ c) \right] * (\hat{\epsilon}+ c) \label{eq:Intial}
%  \end{equation}
%  The RHS of Equation \ref{eq:Intial} is $ \tilde{F}(\alpha,\omega,c)$, so the theorem is proved.
% \end{proof}



Since we want to analyze a Hopf bifurcation, we will want to solve $\tilde{F}_\epsilon = 0$ for small values of~$\epsilon$. 
However, at the bifurcation point, $ D \tilde{F}_0(\pp  ,\pp , 0)$ is not invertible.
In order for our asymptotic analysis to be non-degenerate,
we work with a rescaled version of the problem. To this end, for any $\epsilon >0$, we rescale both $\tc$ and $\tilde{F}$ as follows. Let $\tc = \epsilon c$ and 
\begin{equation}\label{e:changeofvariables}
  \tilde{F}_\epsilon (\alpha,\omega,\epsilon c) = \epsilon F_\epsilon (\alpha,\omega,c).
\end{equation}
For $\epsilon>0$ the problem then reduces to finding zeros of 
\begin{equation}
\label{eq:FDefinition}
	F_\epsilon(\alpha,\omega, c) := 
	[i \omega + \alpha e^{-i \omega}] \e_1 + 
	( i \omega K^{-1} + \alpha U_{\omega}) c + 
	\epsilon \alpha e^{-i \omega} \e_2  +
	\alpha \epsilon L_\omega c + 
	\alpha \epsilon [ U_{\omega} c] * c.
\end{equation}
We denote the triple $(\alpha,\omega,c) \in \R^2 \times \ell^1_0$ by $x$.
To pinpoint the components of $x$ we use the projection operators
\[
   \pi_\alpha x = \alpha, \quad \pi_\omega x = \omega, \quad 
  \pi_c x = c \qquad\text{for any } x=(\alpha,\omega,c).
\]

After the change of variables~\eqref{e:changeofvariables} we now have an invertible Jacobian $D F_0(\pp  ,\pp , 0)$ at the bifurcation point.
On the other hand, for $\epsilon=0$ the zero finding problems for $\tilde{F}_\epsilon$ and $F_\epsilon$ are not equivalent. 
However, it follows from the following lemma that any nontrivial periodic solution having $ \epsilon=0$ must have a relatively large size when $ \alpha $ and $ \omega $ are close to the bifurcation point. 

\begin{lemma}\label{lem:Cone}
	Fix $ \epsilon \geq 0$ and $\alpha,\omega >0$. 
	Let
	\[
	b_* :=  \frac{\omega}{\alpha} - \frac{1}{2} - \epsilon  \left(\frac{2}{3}+ \frac{1}{2}\sqrt{2 + 2 |\omega-\pp| } \right).
	\]
Assume that $b_*> \sqrt{2} \epsilon$. 
Define
% \begin{equation*}%\label{e:zstar}
% 	z^{\pm}_* :=b_* \pm \sqrt{(b_*)^2- \epsilon^2 } .
% \end{equation*}
% \note[J]{Proposed change to match Lemma E.4}
\begin{equation}\label{e:zstar}
z^{\pm}_* :=b_* \pm \sqrt{(b_*)^2- 2 \epsilon^2 } .
\end{equation}
If there exists a $\tc \in \ell^1_0$ such that $\tilde{F}_\epsilon(\alpha, \omega,\tc) = 0$, then \\
\mbox{}\quad\textup{(a)} either $ \|\tc\| \leq  z_*^-$ or $ \|\tc\| \geq z_*^+  $.\\
\mbox{}\quad\textup{(b)} 
$ \| K^{-1} \tc \| \leq (2\epsilon^2+ \|\tc\|^2) / b_*$. 
\end{lemma}
\begin{proof}
	The proof follows from Lemmas~\ref{lem:gamma} and~\ref{lem:thecone} in Appendix~\ref{appendix:aprioribounds}, combined with the observation that
$\frac{\omega}{\alpha} - \gamma \geq b_*$,
% \[
%   \frac{\omega}{\alpha} - \gamma \geq b_*
%  \qquad\text{for all }
% | \alpha - \pp| \leq r_\alpha \text{ and } 
%   | \omega - \pp| \leq r_\omega.
% \]
with $\gamma$ as defined in Lemma~\ref{lem:gamma}.
\end{proof}

\begin{remark}\label{r:smalleps}
We note that for $\alpha < 2\omega$
\begin{alignat*}{1}
z^+_* &\geq   \frac{2 \omega - \alpha}{\alpha} 
- \epsilon \left(4/3+\sqrt{2 + 2 |\omega-\pp| } \, \right) + \cO(\epsilon^2)
\\[1mm]
z^-_* & \leq   \cO(\epsilon^2)
\end{alignat*}
for small $\epsilon$. 
Hence Lemma~\ref{lem:Cone} implies that for values of $(\alpha,\omega)$ near $(\pp,\pp)$ any solution has either $\|\tc\|$ of order 1 or $\|\tc\| =  \cO(\epsilon^2)$. 
The asymptotically small term bounding $z_*^-$ is explicitly calculated in Lemma~\ref{lem:ZminusBound}. 
A related consequence is that for $\epsilon=0$ there are no nontrivial solutions 
of $\tilde{F}_0(\alpha,\omega,\tc)=0$ with 
$\| \tc \| < \frac{2 \omega - \alpha}{\alpha} $. 
\end{remark}

\begin{remark}\label{r:rhobound}
In Section~\ref{s:contraction} we will work on subsets of $\ell^K_0$ of the form
\[
  \ell_\rho := \{ c \in \ell^K_0 : \|K^{-1} c\| \leq \rho \} .
\]
Part (b) of Lemma~\ref{lem:Cone} will be used in Section~\ref{s:global} to guarantee that we are not missing any solutions by considering $\ell_\rho$ (for some specific choice of $\rho$) rather than the full space $\ell^K_0$.
In particular, we infer from Remark~\ref{r:smalleps} that  small solutions (meaning roughly that $\|\tc\| \to 0$ as $\epsilon \to 0$)
satisfy $\| K^{-1} \tc \| = \cO(\epsilon^2)$.
\end{remark}

The following theorem guarantees that near the bifurcation point the problem of finding all periodic solutions is equivalent to considering the rescaled problem $F_\epsilon(\alpha,\omega,c)=0$.
\begin{theorem}
\label{thm:FourierEquivalence3}
\textup{(a)} Let $ \epsilon > 0$,  $c \in \ell^K_0$, $\alpha>0$ and $ \omega >0$. 
	Define $y: \R\to \R$ as 
\begin{equation}\label{e:yc}
	y(t) = 
	\epsilon \left( e^{i \omega t }  + e^{- i \omega t }\right) 
	+ \epsilon  \sum_{k = 2}^\infty   c_k e^{i \omega k t }  + c_k^* e^{- i \omega k t } .
\end{equation}
%	and suppose that $ y(t) > -1$. 
	Then $y(t)$ solves~\eqref{eq:Wright} if and only if $F_{\epsilon}( \alpha , \omega , c) = 0$.\\
\textup{(b)}
Let $y(t) \not\equiv 0$ be a periodic solution of~\eqref{eq:Wright} of period $2\pi/\omega$
 with Fourier coefficients $\c$.
Suppose $\alpha < 2\omega$ and $\| \c \| < \frac{2 \omega - \alpha}{\alpha} $.
Then, up to time translation, $y(t)$ is described by a Fourier series of the form~\eqref{e:yc} with $\epsilon > 0$ and $c \in \ell^K_0$.
\end{theorem}

\begin{proof}
Part (a) follows directly from Theorem~\ref{thm:FourierEquivalence2} and the  change of variables~\eqref{e:changeofvariables}.
To prove part (b) we need to exclude the possibility that there is a nontrivial solution with $\epsilon=0$. The asserted bound on the ratio of $\alpha$ and $\omega$ guarantees, by Lemma~\ref{lem:Cone} (see also Remark~\ref{r:smalleps}), that indeed $\epsilon>0$ for any nontrivial solution. 
\end{proof}

We note that in practice (see Section~\ref{s:global}) a bound on $\| \c \|$ is derived from a bound on $y$ or $y'$ using Parseval's identity.

\begin{remark}\label{r:cone}
It follows from Theorem~\ref{thm:FourierEquivalence3} and Remark~\ref{r:smalleps} that for values of $(\alpha,\omega)$ near $(\pp,\pp)$ any reasonably bounded solution satisfies $\| c\| =  O(\epsilon)$ as well as $\|K^{-1} c \| = O(\epsilon)$ asymptotically (as $\epsilon \to 0$).
These bounds will be made explicit (and non-asymptotic) for specific choices of the parameters in Section~\ref{s:global}.
\end{remark}

% We are able to rule out such large amplitude solutions using global estimates such as those in \cite{neumaier2014global}.
% Hence, near the bifurcation point, the problem of describing periodic solutions of~\eqref{eq:Wright} reduces to studying the family of zeros finding problems $F_\epsilon=0$.





%Specifically, if a solution having $ \epsilon = 0$ does in fact correspond to a nontrivial periodic solution and $\alpha  < 2\omega $, then $ \| \tilde{c} \| > 2 \omega \alpha^{-1} -1$. 
%%PERHAPS THIS NEEDS A FORMULATION AS A THEOREM AS WELL?
%%IN OTHER WORDS: ARE WE SURE WE HAVE FOUND ALL ZEROS OF $\tilde{F}_0$, I.E. ALL SOLUTIONS WITH $\epsilon=0$ NEAR THE BIFURCATION POINT? AFTER RESCALING THESE ARE INVISIBLE?
%%THERE SHOULD BE A STATEMENT ABOUT THIS SOMEWHERE! EITHER HERE OR SOME





We finish this section by defining a curve of approximate zeros $\bx_\epsilon$ of $F_\epsilon$ 
(see \cite{chow1977integral,hassard1981theory}). 
%(see \cite{chow1977integral,morris1976perturbative,hassard1981theory}). 


\begin{definition}\label{def:xepsilon}
Let
\begin{alignat*}{1}
	\balpha_\epsilon &:= \pp + \tfrac{\epsilon^2}{5} ( \tfrac{3\pi}{2} -1)  \\
	\bomega_\epsilon &:= \pp -  \tfrac{\epsilon^2}{5} \\
	\bc_\epsilon 	 &:= \left(\tfrac{2 - i}{5}\right) \epsilon \,  \e_2 \,.
\end{alignat*}
We define the approximate solution 
$ \bx_\epsilon := \left( \balpha_\epsilon , \bomega_\epsilon  , \bc_\epsilon \right)$
for all $\epsilon \geq 0$.
\end{definition}

We leave it to the reader to verify that both 
 $F_\epsilon(\pp,\pp,\bc_{\epsilon})=\cO(\epsilon^2)$ and $F_\epsilon(\bx_\epsilon)=\cO(\epsilon^2)$.
%%%	
%%%	
%%%	}{Better like this?}
%%%\annote[J]{ $F_\epsilon(\bx_0)=\cO(\epsilon^2)$ and $F_\epsilon(\bx_\epsilon)=\cO(\epsilon^2)$.}{I think we'd still need the $ \bar{c}_\epsilon$ term in $\bar{x}_0$ to be of order $ \epsilon$.}
%%%\remove[JB]{We show in Proposition A.1
%%%%\ref{prop:ApproximateSolutionWorks} 
%%% that any $ x \in \R^2 \times \ell^1_0$ which is $ \cO(\epsilon^2)$ close to $ \bar{x}_\epsilon $ will yield the estimate $F_\epsilon(x) = \cO(\epsilon^2)$.
%%%Hence choosing $\{ \pp , \pp, \bar{c}_\epsilon\}$ as our approximate solution would also have been a natural choice for performing an $\cO(\epsilon^2)$ analysis and would have simplified several of our calculations.
%%%However,} 
%%%
We choose to use the more accurate approximation 
for the $ \alpha$ and $ \omega $ components to improve our final quantitative results. 














%
% Values for $ (\alpha, \omega,c)$ which approximately solve $\tilde{F}(\alpha,\omega,c) = 0$  are computed in  \cite{chow1977integral,morris1976perturbative,hassard1981theory} and are as follows:
%  \begin{eqnarray}
%  \tilde{\alpha}( \epsilon) &:=& \pi /2 + \tfrac{\epsilon^2}{5} ( \tfrac{3\pi}{2} -1) \nonumber \\
%  \tilde{\omega}( \epsilon) &:=& \pi /2 -  \tfrac{\epsilon^2}{5} \label{eq:ScaleApprox} \\
%  \tc(\epsilon) 	  &:=& \{ \left(\tfrac{2 - i}{5}\right)  \epsilon^2 , 0,0, \dots \} \nonumber
%  \end{eqnarray}
% In Appendix \ref{sec:OperatorNorms} we illustrate an alternative method for deriving this approximation.
%
%
%
%
% We want to solve $ \tilde{F}(\alpha , \omega, \hat{c}) =0$ for small values of $ \epsilon$.
% However $ D \tilde{F}(\alpha , \omega , c)$ is not invertible at $ ( \pp , \pp , 0)$ when $ \epsilon = 0$.
% In order for our asymptotic analysis to be non-degenerate, we need to make the change of variables $ c \mapsto \epsilon c$.
% Under this change of variables, we define the function $ F$ below so that $ \tilde{F}(\alpha , \omega , \epsilon c) =\epsilon  F( \alpha , \omega , c)$.
%
%
%
% \begin{definition}
% Construct an $\epsilon$-parameterized family of densely defined functions  $F : \R^2 \oplus \ell^1 / \C \to \ell^1$ by:
% \begin{equation}
% \label{eq:FDefinition}
% 	F(\alpha,\omega, c) :=
% 	[i \omega + \alpha e^{-i \omega}]_1 +
% 	( i \omega K^{-1} + \alpha U_{\omega}) c +
% 	[\epsilon \alpha e^{-i \omega}]_2  +
% 	\alpha \epsilon L_\omega c +
% 	\alpha \epsilon [ U_{\omega} c] * c.
% \end{equation}
% \end{definition}

%%
%%
%%\begin{corollary}
%%	\label{thm:FourierEquivalence3}
%%	Fix $ \epsilon > 0$, and $ c \in \ell^1 / \C $, and $ \omega >0$. Define $y: \R\to \R$ as 
%%	\[
%%	y(t) = 
%%	\epsilon \left( e^{i \omega t }  + e^{- i \omega t }\right) 
%%	+  \epsilon  \left( \sum_{k = 2}^\infty   c_k e^{i \omega k t }  + \overline{c}_k e^{- i \omega k t } \right) 
%%	\]
%%	and suppose that $ y(t) > -1$. 
%%	Then $y(t)$ solves Wright's equation at parameter $ \alpha > 0 $ if and only if $ F( \alpha , \omega , c) = 0$ at parameter $ \epsilon$. 
%%	
%%	
%%	
%%\end{corollary}
%%
%%
%%\begin{proof}
%%	Since $ \tilde{F}(\alpha,\omega, \epsilon c) = \epsilon F( \alpha , \omega , c)$, the result follows from Theorem \ref{thm:FourierEquivalence2}.
%%\end{proof}

% If we can find $(\alpha , \omega, c)$ for which $ F( \alpha , \omega,c)=0$ at parameter $\epsilon$, then $ \tilde{F}(\alpha ,\omega, c)=0$.
% By Theorem \ref{thm:FourierEquivalence2} this amounts to finding a periodic solution to Wright's equation.
% Lastly, because we have performed the change of variables $ c \mapsto \epsilon c$, we need to  apply this change of variables to our approximate solution as well.
%
% \begin{definition}
% 	Define the approximate solution $ x( \epsilon) = \left\{ \alpha(\epsilon ) , \omega ( \epsilon ) , c(\epsilon) \right\}$ as below,  where $c(\epsilon) = \{ c_2( \epsilon) , 0 ,0 , \dots\} $.
% 	We may also write $ x_\epsilon = x(\epsilon) $.
% 	\begin{eqnarray}
% 	\alpha( \epsilon) &:=& \pi /2 + \tfrac{\epsilon^2}{5} ( \tfrac{3\pi}{2} -1) \nonumber \\
% 	\omega( \epsilon) &:=& \pi /2 -  \tfrac{\epsilon^2}{5} \label{eq:Approx} \\
% 	c_2(\epsilon) 	  &:=& \left(\tfrac{2 - i}{5}\right) \epsilon \nonumber
% 	\end{eqnarray}
%
% \end{definition}

%----------------------------------------------
\section{Polynomial cases for \ESMT}\label{polycase}
In this section, we consider the \ESMT problem for $2$-CPR $n$-gons. Throughout the section, we denote the inner $n$-gon as $\{A_i\}$ and the outer $n$-gon as $\{B_i\}$. First, we consider the configuration of an Euclidean Steiner minimal tree in a subsection of the annular area between $\{A_i\}$ and $\{B_i\}$, which will form an isosceles trapezoid. Next, we consider the simple but illustrative case of $n = 3$. Finally we prove our result for all $n$.

\subsection{Isosceles Trapezoids and Vertical Forks} \label{trapezoids}


In this section, we discuss one particular Steiner topology when the terminal set is  formed by the four corners of a given isosceles trapezoid. However, we will limit the discussion to only the isosceles trapezoids such that the angle between the non-parallel sides is of the form $\frac{2 \pi}n$ where $n \in \mathbb{N}$, $n \ge 4$. The reason is that given $2$-CPR $n$-gons $\{A_i\}$, $\{B_i\}$, for $n \geq 4$ and for any $j \in \{1,\ldots, n-1\}$, the region $A_jA_{j + 1}B_jB_{j + 1}$ is an isosceles trapezoid such that the angle between the non-parallel sides is of the form $\frac{2 \pi}n$.

\begin{figure}[h]
\centering
\includegraphics[width=5.5cm]{trapezoid.png}
\caption{Isosceles trapezoid with $\angle AOB = \frac {2 \pi} 8$}
\label{trap_def}
\end{figure}

Let $ABQP$ be an isosceles trapezoid with $AB$, $PQ$ as the parallel sides, and $AP$, $BQ$ as the non-parallel sides. Assume without loss of generality that $AB$ is shorter in length than $PQ$. Let $\overline{AB} = 1$ and $\overline{PQ} = \lambda$, where $\lambda \geq \frac {\sqrt 3 + \tan{ \frac \pi n }} {\sqrt 3 - \tan{ \frac \pi n}} $. For brevity, we say $\lambda_v = \frac {\sqrt 3 + \tan{ \frac \pi n }} {\sqrt 3 - \tan{ \frac \pi n}}$. Let $L_{PA}$ and $L_{QB}$ be the lines containing the line segments $PA$ and $QB$ respectively. Also let $O$ be the point of intersection of $L_{PA}$ and $L_{QB}$. Further, let $M$ and $N$ be the midpoints of $AB$ and $PQ$ respectively (as in Figure \ref{trap_def}). As mentioned earlier, $\angle AOB = \frac{2 \pi}n$ where $n \in \mathbb{N}$, $n \ge 4$.



%We show that a particular Steiner topology of $\{A, B, P, Q\}$. We call it the \emph{vertical fork}.

% \vspace{0.2cm}
% \subsection{The MST (for $n = 6$)}
% \vspace{0.2cm}

% For $n = 6$, the minimum spanning tree (MST) $T_{sp} = \{PA, AB, BQ\}$ is a possible Steiner tree ($\forall$ $\lambda > 1)$.

% \begin{figure}[h!]
% \centering
% \includegraphics[width=7cm]{hexagon_trapezoid_sp.png}
% \caption{Minimum Spanning Tree $T_{sp} = \{PA, AB, BQ\}$}
% \end{figure}

% We observe that for $n > 6, \angle PAB = \angle QBA < 120 ^ \circ$. Further, for $n = 6$, we get:
% $$\color{red}|T_{sp}| = \overline{PA} + \overline{AB} + \overline{QB} = 2 \lambda - 1$$

% \vspace{0.2cm}
% \subsection{The Horizontal Fork (for $n > 6$)}
% \vspace{0.2cm}

% For $n > 6$, the minimum spanning tree (MST) no longer remains a valid Steiner tree. We explore other topology, where $P$ \& $A$ are connected to a Steiner Point $S_1$ and $B$ \& $Q$ are connected to another Steiner Point $S_2$ such that $S_1$ and $S_2$ are directly connected. We call the Steiner tree of such a topology, the \emph{Horizontal Fork}. 

% \begin{figure}[h!]
% \centering
% \includegraphics[width=7cm]{horizontal_fork.png}
% \caption{The Horizontal Fork, $T_{hf}$}
% \end{figure}

% Therefore, we have the horizontal fork as $T_{hf} = \{PS_1, AS_1, S_1S_2, S_2B, S_2Q\}$

% \subsubsection{Construction}

% \begin{itemize}
%     \item We construct equilateral triangles $ABE$ and $PQF$ towards the interior of the trapezoid $ABPQ$.
%     \item $S_1 := AE \cap PF$, $S_2 := BE \cap QF$ (These intersections occurs only if $n > 6$)
% \end{itemize}

% \subsubsection{Analysis}

% We observe that the Steiner points $S_1$, $S_2$ are obtainable only if $n > 6$. Further, for the triangles $ABE$ and $PQF$ to intersect, the following condition must hold: \\


% $ \overline{ME} + \overline{FN} \ge \overline{MN} $\\

% $ \implies \dfrac{\sqrt{3}}2 \cdot \overline {PQ} + \dfrac{\sqrt{3}}2 \cdot \overline {PQ} \ge \overline{ON} - \overline{OM}$\\

% $ \implies \dfrac{\sqrt{3}}2 \cdot (\lambda + 1) \ge \dfrac \lambda {2 \tan(\frac \pi n)} - \dfrac 1 {2 \tan(\frac \pi n)}$\\

% $ \implies \lambda \le \dfrac {1 + \sqrt 3 \tan{ \frac \pi n }} {1 - \sqrt 3 \tan{ \frac \pi n}} = \lambda_h \text{ (Say)}$\\


% Now, $\overline{AB} = \overline{AE} = \overline{BE} = 1$ and $\overline{PQ} = \overline{QF} = \overline{FP} = \lambda$. Further we observe that $ES_1S_2$ and $FS_1S_2$ are equilateral triangles. Let $\overline{ES_1} = \overline{S_1F} = \overline{FS_2} = \overline{S_2E} = \overline{S_1S_2} = \delta$. Now,\\

% $ \overline{EF} + \overline{MN} = \overline{NF} + \overline{ME}$\\

% $ \implies \sqrt 3 \cdot \delta + \dfrac{\lambda - 1}{2 \tan \frac \pi n} = \dfrac{\sqrt 3}2 (\lambda + 1)$\\

% $ \implies -3 \cdot \delta = \dfrac{\sqrt{3} (\lambda - 1)}{2 \tan \frac \pi n} - \dfrac{3}2 (\lambda + 1)$

% \noindent Again, \\


% \noindent $|T_{hf}| = \overline{PS_1} + \overline{QS_2} + \overline{S_1S_2} + \overline{S_1A} + \overline{S_2B}$\\

% $ = (\lambda - \delta) + (\lambda - \delta) + \delta + (1 - \delta) + (1 - \delta) = 2 (\lambda + 1) - 3 \cdot \delta  = 2 (\lambda + 1) + \dfrac{\sqrt{3} (\lambda - 1)}{2 \tan \frac \pi n} - \dfrac{3}2 (\lambda + 1)$\\


% \noindent Therefore we get : 
% $$\color{blue}|T_{hf}| = \dfrac{\lambda + 1}2 + \dfrac{\sqrt{3} (\lambda - 1)}{2 \tan \frac \pi n}$$

% \noindent for any $n > 6$ and $\lambda \le \lambda_h$

%\vspace{0.2cm}
%\subsection{The Vertical Fork}
%\vspace{0.2cm}

Now, we explore the following Steiner topology of the terminal set $\{A, B, P, Q\}$:
\begin{enumerate}
\item $A$ and $B$ are connected to a Steiner point $S_1$.
\item $P$ and $Q$ are connected to another Steiner point $S_2$. 
\item $S_1$ and $S_2$ are directly connected (Please see Figure \ref{T_vf}). 
\end{enumerate}
We call such a topology a \emph{vertical fork topology} and the Steiner tree realising such a topology, the \emph{vertical fork}. Note that in a vertical fork topology the only unknowns are the locations of the two Steiner points $S_1,S_2$. Therefore, we have the vertical fork topology as $T_{vf}$, with $E(T_{vf}) = \{AS_1, BS_1, S_1S_2, S_2P, S_2Q\}$.


\begin{figure}[h]
\centering
\includegraphics[width=5.5cm]{vertical_fork.png}
\caption{The Vertical Fork, $\mathcal T_{vf}$}
\label{T_vf}
\end{figure}

%\subsubsection{Construction}


%\subsubsection{Analysis}

We show the existence of a vertical fork and calculate its total length in the following lemma.

\begin{lemma} \label{lambda_v}
A vertical fork $\mathcal T_{vf}$ can be constructed for any $n \ge 4$ and for any $\lambda \ge \lambda_v$, where

$$\lambda_v = \frac {\sqrt 3 + \tan{ \frac \pi n }} {\sqrt 3 - \tan{ \frac \pi n}}$$    
such that the length of the vertical fork   
$$|\mathcal T_{vf}| = \dfrac{(\lambda - 1)}{2 \tan \frac \pi n} + \dfrac {\sqrt 3 (\lambda + 1)} {2}$$

\end{lemma}

\begin{proof}
First, we construct the Steiner points $S_1$, $S_2$ and then prove that the construction works.

In the following construction, we describe how to find the locations of $S_1$ and $S_2$ for  the vertical fork:
\begin{itemize}
    \item We construct equilateral triangles $ABE$ and $PQF$ where both points $E$ and $F$ lie outside the trapezoid $ABQP$.
    \item We construct the circumcircles $(ABE)$ and $(PQF)$ of $ABE$ and $PQF$, respectively. 
    \item Recall that $L_{MN}$ is the line segment containing $M$ and $N$. Define $S_1$ to be the point of intersection of $L_{MN}$ and the circle $(ABE)$ distinct from $E$; similarly, $S_2$ is the point of intersection of the $L_{MN}$ and $(PQF)$ distinct from $F$. Therefore, by construction, $S_2$, $M$ must lie on the same side of $N$ on $L_{MN}$, and $S_1$, $N$ must lie on the same side of $M$ on $L_{MN}$. Further, $\angle AS_1B = \angle PS_2Q = \frac{2 \pi}{3}$ by construction.
\end{itemize}

We now show that the points $S_1$ and $S_2$ indeed lie inside the line segment $MN$ and the points appear in the order: $M$, $S_1$, $S_2$, $N$. We prove the following claim to serve this purpose.

\begin{claim} \label{S1_S2_in_MN}
$\overline{S_1M} + \overline{S_2N} \le \overline{MN}$
% \begin{enumerate}
%     \item  (As $S_2$, $M$ lie on the same side of $N$ on $L_{MN}$ and $S_1$, $N$ lie on the same side of $M$ on $L_{MN}$, this implies that $S_1$, $S_2$ lie on the line segment $MN$).
% \end{enumerate}

\end{claim}
\begin{proof}
We have $\overline{MN} = \overline {ON} - \overline{OM} = \dfrac{\lambda}{2 \tan \frac{\pi}{n}} - \dfrac{1}{2 \tan \frac{\pi}{n}} = \dfrac{(\lambda - 1)}{2 \tan \frac \pi n}$\\
Again $\overline{S_1M} = \dfrac{1}{2\sqrt{3}}$ and $\overline{S_2N} = \dfrac{\lambda}{2\sqrt{3}}$.\\
Therefore we have 
\begin{align*}
   & \overline{MN} - (\overline{S_1M} + \overline{S_2M}) \\
 = & \dfrac{(\lambda - 1)}{2 \tan \frac \pi n} - \dfrac 1 {\sqrt{3}} - \dfrac \lambda {2\sqrt 3}\\
 = & \dfrac{(\lambda - 1)}{2 \tan \frac \pi n} - \dfrac {(\lambda + 1)} {2\sqrt{3}}\\
 = & \frac{\lambda(\sqrt{3} - \tan \frac{\pi}{n}) - (\sqrt{3} + \tan \frac{\pi}{n})}{2 \sqrt{3} \tan \frac \pi n}\\
 = & \frac{\sqrt{3} - \tan \frac{\pi}{n}}{2 \sqrt{3} \tan \frac \pi n} \cdot (\lambda - \lambda_v)
\end{align*}

Therefore $\lambda \ge \lambda_v$ implies $\overline{MN} \ge (\overline{S_1M} + \overline{S_2M})$. This proves~\Cref{S1_S2_in_MN}.
\end{proof}

From~\Cref{S1_S2_in_MN}, we get $\overline{S_1M}, \overline{S_2N} \le \overline{MN}$. As $S_2$, $M$ lie on the same side of $N$ on $L_{MN}$ and $S_1$, $N$ lie on the same side of $M$ on $L_{MN}$, this implies that $S_1$, $S_2$ lie on the line segment $MN$. Further, $\overline{S_1M} + \overline{S_2N} \le \overline{MN}$ implies $\overline{S_1M} \le \overline {S_2M} \le \overline{NM}$, which in turn implies that the points appear in the order: $M$, $S_1$, $S_2$. $N$.\\

Now, we calculate the total length of the vertical fork, $|\mathcal T_vf|$:
\begin{align*}
  & |\mathcal T_{vf}|\\
= & \overline{AS_1} + \overline{BS_1} + \overline{S_1S_2} + \overline{PS_2} + \overline{QS_2}\\
= & 2 \overline{PS_2} + 2 \overline{AS_1} + \overline{S_1S_2}\\
= & \dfrac{2 \lambda} {\sqrt 3} + \dfrac 2 {\sqrt 3} + \bigg( \dfrac{(\lambda - 1)}{2 \tan \frac \pi n} - \dfrac {(\lambda + 1)} {2\sqrt{3}} \bigg)\\
= & \dfrac{(\lambda - 1)}{2 \tan \frac \pi n} + \dfrac {\sqrt 3 (\lambda + 1)} {2}
\end{align*}
% We observe that, 

% \noindent $\overline{S_1S_2} = \overline{MN} - \overline{MS_1} - \overline{NS_2}$\\

% $ = \dfrac{(\lambda - 1)}{2 \tan \frac \pi n} - \dfrac 1 {\sqrt{3}} - \dfrac \lambda {\sqrt 3}$\\

% $ = \dfrac{(\lambda - 1)}{2 \tan \frac \pi n} - \dfrac {(\lambda + 1)} {\sqrt{3}}$\\

%  Again,\\

% \noindent $|T_{vf}| = \overline{AS_1} + \overline{BS_1} + \overline{S_1S_2} + \overline{PS_2} + \overline{QS_2} = 2 \overline{PS_2} + 2 \overline{AS_1} + \overline{S_1S_2}$\\

% $ = \dfrac{2 \lambda} {\sqrt 3} + \dfrac 2 {\sqrt 3} + \bigg( \dfrac{(\lambda - 1)}{2 \tan \frac \pi n} - \dfrac {(\lambda + 1)} {\sqrt{3}} \bigg)$\\

% $ = \dfrac{(\lambda - 1)}{2 \tan \frac \pi n} + \dfrac {\sqrt 3 (\lambda + 1)} {\sqrt{2}}$\\

% Also, for the construction to be possible we must have: \\

% $ \overline{S_1S_2} \ge 0 $\\

% \noindent $\implies \bigg( \dfrac{(\lambda - 1)}{2 \tan \frac \pi n} - \dfrac {(\lambda + 1)} {\sqrt{3}} \bigg) \ge 0 $\\

% \noindent $\implies \lambda \ge \dfrac {\sqrt 3 + \tan{ \frac \pi n }} {\sqrt 3 - \tan{ \frac \pi n}} = \lambda_v$\\

% Thus, we are done.

This completes the proof of~\Cref{lambda_v}.
\end{proof}




% \vspace{0.2cm}
% \subsection{Comparing various Topologies}
% \vspace{0.2cm}

% \subsubsection{The horizontal fork \textit{vs} The vertical fork ($n > 6$)} \label{HFvsVF}

% Let for some $\lambda$, $\lambda_v \le \lambda \le \lambda_h$, the \emph{vertical fork} ``is better than" (has a smaller total length than) the \emph{horizontal fork}, for some $n > 6$.\\ 

% $ \Longleftrightarrow |T_{vf}| \le |T_{hf}|$\\

% $ \Longleftrightarrow \dfrac{(\lambda - 1)}{2 \tan \frac \pi n} + \dfrac {\sqrt 3 (\lambda + 1)} {2} \le \dfrac{\lambda + 1}2 + \dfrac{\sqrt{3} (\lambda - 1)}{2 \tan \frac \pi n}$\\

% $ \Longleftrightarrow \dfrac{(\lambda - 1)}{\tan \frac \pi n} \cdot \dfrac{\sqrt 3 - 1} 2 \ge (\lambda + 1) \cdot \dfrac{\sqrt 3 - 1} 2 $\\

% $ \Longleftrightarrow (\lambda - 1) \ge (\lambda + 1) \cdot \tan \frac \pi n$\\

% $ \Longleftrightarrow \lambda \ge \dfrac{(1 + \tan \frac \pi n)} {1 - \tan \frac \pi n} = \tan( \frac \pi 4 + \frac \pi n) = \lambda_s \text{ (Say)}$


% \subsubsection{The Vertical Fork \textit{vs} The MST ($n = 6$)} \label{VFvsMST}

% Let for some $\lambda$, $\lambda_v \le \lambda \le \lambda_h$, the \emph{vertical fork} ``is better than" (has a smaller total length than) the \emph{horizontal fork}, for $n = 6$.\\ 


% $ \Longleftrightarrow |T_{vf}| \le |T_{sp}|$\\

% $ \Longleftrightarrow \dfrac{(\lambda - 1)}{2 \tan \frac \pi n} + \dfrac {\sqrt 3 (\lambda + 1)} {2} \le 2 \lambda - 1$\\

% $ \Longleftrightarrow \dfrac {\sqrt 3 (\lambda - 1)} {2} + \dfrac {\sqrt 3 (\lambda + 1)} {2} \le 2 \lambda - 1 \text{ , [} \because n = 6 \text{]} $ \\

% $ \Longleftrightarrow \lambda \ge \dfrac 1 {2 - \sqrt{3}} = 2 + \sqrt 3 = \tan( \frac \pi 4 + \frac \pi 6) = \lambda_s$\\

% \begin{remark} The MST for $n = 6$ is the degenerate limit of of the \emph{horizontal fork} for $n \rightarrow 6$.
% \end{remark}

% Combining this with the result from \ref{HFvsVF}, we obtain that for all $n \ge 6$, the \emph{Vertical Fork} is better than the other construction (\emph{Horizontal Fork} for $n > 6$ and MST for $n = 6$) for $\lambda \ge \lambda_s$.

% \subsubsection{Combining the Results}

% First of all, for \ref{HFvsVF} and \ref{VFvsMST}, to be valid, we must have $\lambda_v \le \lambda_s \le \lambda_h$ for all $n > 6$, which is true because $f(x) = \dfrac {1 + x}{1 - x}$ is an increasing function in $(0, 1)$, where we have $\lambda_v = f \Big(\dfrac 1 {\sqrt 3} \cdot \dfrac \pi n \Big)$, $\lambda_s = f \Big(1 \cdot \dfrac \pi n \Big)$ and $\lambda_h = f \Big(\sqrt 3 \cdot \dfrac \pi n \Big)$.


% \begin{figure}[h!]
% \centering
% \includegraphics[width=13cm]{Trapezoid_data.png}
% \caption{ $\lambda_v$, $\lambda_s$ and $\lambda_h$ plotted against $n$ }
% \end{figure}


\subsection{\ESMT for $2$-Concentric Parallel Regular $3$-gons} \label{triangles}

Note that a regular $3$-gon is an equilateral triangle and therefore, for the rest of this section we call a regular $3$-gon as an equilateral triangle. We describe a minimal solution for \ESMT for $2$-CPR equilateral triangles.

\begin{lemma}\label{3-gon}
Consider two concentric and parallel equilateral triangles $A_1A_2A_3$ and $B_1B_2B_3$, where $B_1B_2B_3$ has side length $\lambda > 1$ and $A_1A_2A_3$ has side length $1$. An SMT of $\{A_1, A_2, A_3, B_1, B_2, B_3\}$ is an SMT of $\{B_1,B_2,B_3\}$, and has length $\sqrt 3 \cdot \lambda$. 
\end{lemma}

\begin{proof}
It is to be noted that the centre $O$ of both $A_1A_2A_3$ and $B_1B_2B_3$ is also the Torricelli point of both $A_1A_2A_3$ and $B_1B_2B_3$~\cite{du1987steiner}. On taking $O$ as the only Steiner point, the SMT for $\{B_1,B_2,B_3\}$ is $ \mathcal T_3 = \{OB_1, OB_2, OB_3\}$~\cite{du1987steiner}. However, the edges of $\mathcal T_3$ already pass through $A_1$, $A_2$ and $A_3$. Therefore, $ \mathcal T_3$ with $E(\mathcal T_3) = \{OA_1, OA_2, OA_3, A_1B_1, A_2B_2, A_3B_3\}$ is also the SMT for $\{A_1, A_2, A_3, B_1, B_2, B_3\}$ as shown in Figure \ref{conc_eq_tri}.

 From the definition of $\mathcal T_3$, we have the length of the \smtpoly, for $n = 3$ as 
$$ | \mathcal T_3 | = \sqrt 3 \cdot \lambda $$ 
\end{proof}

\begin{figure}[h]
\centering
\includegraphics[width=4cm]{3_gon_topo1.png}
\caption{SMT for $2$-CPR $3$-gons }
\label{conc_eq_tri}
\end{figure}



\subsection{\ESMT and Large Polygons with Large Aspect Ratios} \label{final_proofs}


In this section, we consider the \ESMT problem when the terminal set is formed by the vertices of $2$-CPR $n$-gons, namely $\{A_i\}$ and $\{B_i\}$. As mentioned earlier, $\{A_i\}$ is the inner polygon and $\{B_i\}$ is the outer polygon of this set of $2$-CPR $n$-gons. In particular, we consider the case when $n \geq 13$; for $n \leq 12$ these are constant sized input instances and can be solved using any brute-force technique. We also require that the aspect ratio 
$\lambda$ has a lower bound $\lambda_1$, i.e. we do not want the two polygons to have sides of very similar length. The exact value of  $\lambda_1$ will be clear during the description of the algorithm. Intuitively, when $\lambda$ is \emph{very large}, the SMT should look similar to what was derived in~\cite{weng1995steiner}. In other words, (please refer to Figure \ref{sing_con_top_fig}):
\begin{enumerate}
    \item for some $j \in [n]$, there is a  vertical fork connecting the two consecutive inner polygon points $A_j, A_{j + 1}$  with the two consecutive outer polygon points $B_j, B_{j + 1}$ - we refer to this vertical fork as the \emph{vertical gadget} for the SMT,
    \item the other points in $\{B_i\}$ are connected directly via $(n - 2)$ outer polygon edges,
    \item the other points in $\{A_i\}$ are connected via $(n - 2)$ inner polygon edges. 
\end{enumerate}

We call such a topology, a \emph{singly connected topology} as in Figure \ref{sing_con_top_fig}. 
%(generated using \href{www.geosteiner.com}{Geosteiner}). 
For the rest of this section, we consider the SMTs for a large enough aspect ratio, $\lambda$ and show that there is an SMT that must be a realisation of a \emph{singly connected topology}. We refer to an SMT for the terminal set defined by the vertices of $\{A_i\}$ and $\{B_i\}$ as the \smtpoly. 

Without loss of generality, we consider the edge length of any side $A_iA_{i+1}$ in $\{A_i\}$ to be $1$. As we defined the aspect ratio to be $\lambda$, any side $B_iB_{i+1}$ of $\{B_i\}$ must have a side length of $\lambda$. Further, we observe that for any SMT $\mathcal T$, specifying $E(\mathcal T)$ sufficiently determines the entire tree, as $V(\mathcal T) = \{P~|~\exists Q \text{ such that } PQ \in E(\mathcal T)\}$.

We start with the following formal definitions:

\begin{definition}\label{def:singly-conn}
A Steiner topology of $\{A_i\} \cup \{B_i\}$ is a \textbf{singly connected topology}, if it has the following structure:

\begin{itemize}
    \item A vertical gadget i.e.  five edges $\{A_jS_a, A_{j + 1}S_a, S_aS_b, S_bB_j, S_bB_{j + 1}\}$ for some $1 \le j \le n$, where $S_a$ and $S_b$ are newly introduced Steiner points contained in the isosceles trapezoid $\{A_j, A_{j + 1}, B_j, B_{j + 1}\}$. 
    \item All $(n - 2)$ polygon edges of $\{A_i\}$ excluding the edge $A_jA_{j + 1}$
    \item All $(n - 2)$ polygon edges of $\{B_i\}$ excluding the edge $B_jB_{j + 1}$
\end{itemize} 
\end{definition}


\begin{figure}[h]
\centering
\subfloat[\centering SMT of $2$-CPR $13$-gons]{\includegraphics[width=4cm]{13_sing.png}}
    \qquad
\subfloat[\centering SMT of $2$-CPR $20$-gons] {\includegraphics[width=4cm]{20_sing.png} }
\caption{ \emph{Singly connected topology} of $2$-CPR $n$-gons $n = 13$ and $n = 20$ }
\label{sing_con_top_fig}
\end{figure}

We define the notion of a path in an SMT for the vertices of $\{A_i\}$ and $\{B_i\}$ where the starting point is in $\{A_i\}$ and the ending point is in $\{B_i\}$.
\begin{definition}
    An \textbf{A-B path} is a path in a Steiner tree of $\{A_i\} \cup \{B_i\}$ which starts from a vertex in $\{A_i\}$ and ends at a vertex in $\{B_i\}$ with all intermediate nodes (if any) being Steiner points. 
\end{definition}

The following Definition and Figure~\ref{fig:counterpath} is useful for the design of our algorithm.
\begin{definition}\label{def:cw_ccw_path}
    A \textbf{counter-clockwise path} is a path $P_1, P_2, ... P_m$ in a Steiner tree such that for all $i \in \{2, \ldots, m - 1\}, \angle P_{i - 1}P_{i}P_{i + 1} = \frac{2 \pi}{3}$ in the counter-clockwise direction. Similarly, a \textbf{clockwise path} is a path $P_1, P_2, ... P_m$ in a Steiner tree such that for all $i \in \{2, \ldots, m - 1\}, \angle P_{i - 1}P_{i}P_{i + 1} = \frac{4 \pi}{3}$ in the counter-clockwise direction.
\end{definition}


\begin{figure}[h]
\centering
\includegraphics[width=7cm]{CW_CCW_def.png}
\caption{$\angle P_{i-1}P_iP_{i+1} = \alpha$ in the counter-clockwise direction and $\angle P_{i-1}P_iP_{i+1} = \beta$ in the clockwise direction, as in~\Cref{def:cw_ccw_path}}
\label{fig:counterpath}
\end{figure}

Now, we consider any Steiner point $S$ in any SMT. Let $P$ and $Q$ be two neighbours of $S$. We now prove that there is no point of the SMT inside the triangle $PSQ$.

\begin{observation} \label{no_pt_in_PSQ}
    Let $S$ be a Steiner point in any SMT for $\{A_i\} \cup \{B_i\}$, with neighbours $P$ and $Q$. Then, no point of the SMT lies inside the triangle $PSQ$.
\end{observation}

\begin{proof}
    By the \emph{lune property} (Proposition \ref{lune}), for any edge $P_1Q_1$ in an SMT, for the two circles centred at $P_1$ and at $Q_1$, respectively and both having a radius of $\overline{P_1Q_1}$, the intersection region does not contain any point of the SMT.

    \begin{figure}[h]
    \centering
    \includegraphics[width=10cm]{no_pt_in_PSQ.png}
    \caption{Observation \ref{no_pt_in_PSQ}: Given triangle $PSQ$, equilateral triangles $PSE$ and $QSF$ are constructed}
    \label{no_pt_in_PSQ_fig}
    \end{figure}

    Let $E$ and $F$ be points on the internal angle bisector of $\angle PSQ$, such that $\angle SPE = \angle SQF = \frac{\pi}{3}$ as shown in Figure \ref{no_pt_in_PSQ_fig}. Since $E$ and $F$ are points on the angle bisector of $\angle PSQ$, $\angle PSE = \angle QSF = \frac{\pi}{3}$. Hence, triangles $PSE$ and $QSF$ are equilateral triangles.

    Since $PS$ is an edge in the SMT, by the lune property, the intersection of the circles centred at $P$ and $S$, both with radius $\overline{PS}$ contain no point inside which is a part of the SMT. Since the lune contains the entire equilateral triangle $PSE$, no point of the SMT lies inside triangle $PSE$. Similarly, no point of the SMT lies inside the triangle $QSF$.

    Further, as $\angle PSQ = \frac{2 \pi}{3}$, $\angle SPQ + \angle SQP = \frac{\pi}{3}$. This means $\angle SPQ, \angle SQP < \frac{\pi}{3}$. Therefore, as $\angle SPE = \angle SQF = \frac{\pi}{3}$, $E$ and $F$ must lie outside the triangle $PSQ$. This implies that the triangle $PSQ$ is covered by the union of the triangles $PSE$ and $QSF$. As no point of the SMT lies in triangles $PSE$ and $QSF$, triangle $PSQ$ must contain no points of the SMT as well.
\end{proof}


Next, we show that in an \smtpoly there cannot be any Steiner point, in the interior of the polygon $\{A_i\}$, that is a direct neighbour of some point $B_k$ in the polygon $\{B_i\}$.

\begin{observation} \label{no_in_to_B}
    For any \smtpoly, there cannot exist a Steiner point $S$ lying in the interior of the polygon $\{A_i\}$ such that $SB_k$ is an edge in an SMT for some $B_k \in \{B_i\}$. 
\end{observation}

\begin{proof}
    For the sake of contradiction, we assume that for some \smtpoly there exists a Steiner point $S$ lying in the interior of the polygon $\{A_i\}$ such that $SB_k$ is an edge in the SMT for some $B_k \in \{B_i\}$. Let $A_mA_{m+1}$ be the edge such that $SB_k$ intersects $A_mA_{m+1}$. Without loss of generality, assume that $A_m$ is closer to $B_k$ than $A_{m+1}$. Therefore $\angle B_kA_mS > \angle B_kA_mA_{m+1} \ge \frac{\pi}{2} + \frac{\pi}{n} > \frac{\pi}{2}$. This means that $B_kS$ is the longest edge in the triangle $B_kSA_m$. Therefore we can remove the edge $B_kS$ from the SMT and replace it with either $B_kA_m$ or $SA_m$ to get another tree connecting the terminal set with a shorter total length than what we started with, which is a contradiction. 
\end{proof}

We further analyze SMTs for $\{A_i\}\cup \{B_i\}$.

\begin{observation} \label{subchain-smt}
    Let $\mathcal V = \{A_j, A_{j + 1}, \ldots, A_k\}$ be the interval of consecutive vertices of $\{A_i\}$ lying between $A_j$ and $A_k$ (which includes $A_{j + 1}$) such that $A_j$ is distinct from $A_{k + 1}$. Let $U$ be any point on the line segment $A_kA_{k+1}$. Then an SMT of $\mathcal V \cup \{U\}$ is $\mathcal T$, with $E(\mathcal T) = \{A_jA_{j+1}, A_{j+1}A_{j+2}, \ldots, A_{k-1}A_k\} \cup \{A_kU\}$. 
\end{observation}
\begin{proof}
    For the sake of contradiction, assume that there exists an SMT $\mathcal T'$ of $\mathcal V \cup \{U\}$ such that $|\mathcal T'| < |\mathcal T|$.  
    
    From \cite{du1987steiner}, we know that $\mathcal T_A$, with $E(\mathcal T_A) = \{A_jA_{j+1}, A_{j+1}A_{j+2}, \ldots, A_{j-2}A_{j-1}\}$ (\textit{i.e.} all edges of polygon $\{A_i\}$ except $A_{j-1}A_j$) is an SMT of $\{A_i\}$. Since $U \in A_kA_{k+1} \in E(\mathcal T_A)$, $\mathcal T_A$ must also be an SMT of $\{A_i\} \cup \{U\}$. However, $\mathcal T_A$ can be partitioned as $\mathcal T_A = \mathcal T \uplus \mathcal T_1$, where $E(\mathcal T_1) = \{UA_{k+1}\} \cup \{A_{k+1}A_{k+2}, \ldots, A_{j-2}A_{j-1}\}$.
    However, as $\mathcal T'$ is assumed to be of shorter total length than $\mathcal T$, $\mathcal T' \cup \mathcal T_1$ is a tree, containing $\{A_i\}$ as a vertex subset, which has a shorter total length than $\mathcal T_A$, contradicting the optimality of $\mathcal T_A$.
\end{proof}


We proceed by showing that in any \smtpoly there exists at least one A-B path which is also a counter-clockwise path. Symmetrically, we also show that for any \smtpoly there exists another clockwise A-B path which consists of only clockwise turns. We can intuitively see that this is true because, if all clockwise paths starting at a vertex in $\{A_i\}$ also ended in a vertex in $\{A_i\}$, there would be enough paths to form a cycle, which is not possible in a tree.

\begin{lemma}\label{left_right_turn_path}
In any \smtpoly, there exists an A-B path which is also a clockwise path and there exists an A-B path which is also a counter-clockwise path.
\end{lemma}
\begin{proof}
    For the sake of contradiction, assume that for some \smtpoly there is no A-B path which is a counter-clockwise path. We pick an arbitrary vertex $A_{i_1} \in \{A_i\}$ such that it is connected to at least one Steiner point (say $S_{i_1}$). We consider the counter-clockwise path, $\mathcal C_1$ starting from $A_{i_1}S_{i_1}$ and ending at the first terminal point in the counter-clockwise path. By assumption, there can be no vertex of $\{B_i\}$ in $\mathcal C_1$, hence the endpoint must be a vertex in $\{A_i\}$. Let $A_{i_2}$ be the other endpoint of $\mathcal C_1$. By definition, the penultimate vertex in this counter-clockwise path must be a Steiner point, we call it $S_{i_2}$. We again consider the counter-clockwise path starting from $A_{i_2}S_{i_2}$, and similarly, let $A_{i_3}$ be the first terminal that is encountered in this path. The penultimate vertex in this counter-clockwise path must be a Steiner point, we call it $S_{i_3}$. We can repeat this procedure indefinitely to obtain $A_{i_4}, A_{i_5}, A_{i_6}, \ldots $ as there are no counter-clockwise A-B paths. However, as $\{A_i\}$ has $n$ vertices, there must be a repetition of vertices among   $A_{i_1}, A_{i_2}, A_{i_3}, \ldots, A_{i_{n+1}}$, implying the existence of a cycle in the SMT, which is a contradiction.

    This symmetrically implies that there must also be a clockwise A-B path. 
\end{proof}

Our next step is to bound the number of `connections' that connect the inner polygon $\{A_i\}$ and the outer polygon $\{B_i\}$ for a large aspect ratio, $\lambda$. As $\lambda$ increases, the area of the annular region between the two polygons increases as well. Therefore, an increase in the number of connections would lead to a longer total length of the SMT considered. Consequently, we will prove that after a certain positive constant $\lambda_1$, for $\lambda > \lambda_1$ any \smtpoly will have a single `connection' between the two polygons. Moreover, \cite{weng1995steiner} gives us an evidence that as $\lambda \rightarrow \infty$, there will indeed be a single connection connecting the outer polygon and the inner polygon for $n \ge 12$. We can formalize this notion of existence of a single `connection' with the following lemma.\\

\begin{lemma} \label{mincut_1}
    For any \smtpoly with $n \ge 13$ and $\lambda > \lambda_{1}$, the number of edges needed to be removed in order to disconnect $\{A_i\}$ and $\{B_i\}$ is 1, where 
    $$
    % \lambda_{1} = \frac{2 \sin{\frac{\pi}{n}} + 1}{1 - 6 \sin \frac{\pi}{n}}
    \lambda_{1} = \frac{1}{1 - 4 \sin \frac{\pi}{n}}
    $$
\end{lemma}

\begin{proof}
        For the sake of contradiction, assume that for some \smtpoly, there are at least two distinct edges in that SMT, which are needed to be removed in order to disconnect $\{A_i\}$ and $\{B_i\}$. We start with a claim.
        
        \begin{claim} \label{more_than_lambda}
        A counter-clockwise A-B path in any SMT of $\{A_i\} \cup \{B_i\}$ must have an edge of length greater than $\lambda$.
        \end{claim}
        
\begin{proof}
        We consider a generic setting, where $\mathcal T$ is an SMT of some set of terminal points $\mathcal P$. Let $H \in V(\mathcal T)$ be vertex of $\mathcal T$. If $\mathcal C$ be a counter-clockwise path starting from $H$ such that no edge in the counter clockwise path has a length of more than $r$, for some $r \in \mathbb{R}^{+}$. Due to Lemma 2.4 (1) of \cite{weng1995steiner},  we know that $\mathcal C$ is contained entirely in the circle centred at $H$ with radius $2r$.
        
        In our case, any vertex in $\{A_i\}$ and any vertex in $\{B_i\}$ are separated by the distance of at least $\dfrac{\lambda - 1}{2 \sin{\frac{\pi}{n}}}$. Therefore, by the above fact, the maximum edge length in a counter-clockwise A-B path of any SMT of $\{A_i\} \cup \{B_i\}$ must be at least $\dfrac{\lambda - 1}{4 \sin{\frac{\pi}{n}}}$. Moreover, we have $\lambda > \lambda_1$. Therefore $\lambda > \dfrac{1}{1 - 4 \sin{\frac{\pi}{n}}} \implies \dfrac{\lambda - 1}{4 \sin{\frac{\pi}{n}}} > \lambda$. Hence, a counter-clockwise A-B path in any SMT of $\{A_i\} \cup \{B_i\}$ must have one edge greater than $\lambda$. This proves the claim.
\end{proof}
        
        Now, for any SMT of $\{A_i\} \cup \{B_i\}$, let $\mathcal C$ be a counter-clockwise A-B path (this exists due to Lemma \ref{left_right_turn_path}). From Claim \ref{more_than_lambda}, we know that there is an edge $e$ in $\mathcal C$, with a length greater than $\lambda$. On removing the edge $e$, the SMT splits into a forest of two trees. Let the trees be $\mathcal T_x$ and $\mathcal T_y$. As we assumed that there are two edges required to disconnect $\{A_i\}$ and $\{B_i\}$, there must exist an A-B path in either $\mathcal T_x$ or $\mathcal T_y$. Without loss of generality, let $\mathcal T_x$ contain an A-B path, and hence $\mathcal T_x$ contains at least one point from $\{A_i\}$ and at least one point from $\{B_i\}$. Further, $\mathcal T_y$ must contain at least one terminal point (as it must contain all terminal points in one of the sides of the removed edge $e$). If $\mathcal T_y$ contains a point from $\{A_i\}$, then the polygon $\{A_i\}$ has vertices both from $\mathcal T_x$ and $\mathcal T_y$; otherwise, if $\mathcal T_y$ contains a point from $\{B_i\}$, then the polygon $\{B_i\}$ has vertices both from $\mathcal T_x$ and $\mathcal T_y$. 
        
        This means that either the polygon $\{A_i\}$ or the polygon $\{B_i\}$ will contain at least one node from each of $\mathcal T_x$ and $\mathcal T_y$. Further, as any given vertex must be either in $\mathcal T_x$ or in $\mathcal T_y$, either $\{A_i\}$ or $\{B_i\}$ must contain two consecutive vertices $U_i$ and $U_{i + 1}$ such that one of them is in $\mathcal T_x$ and the other is in $\mathcal T_y$. We simply connect $U_i$ and $U_{i + 1}$ by the polygon edge which is of length $1$ (if $U_i, U_{i + 1} \in \{A_i\}$) or of length $\lambda$ (if $U_i, U_{i + 1} \in \{B_i\}$), giving us back a tree $\mathcal{T}'$containing all the terminals. However we discarded an edge of length greater than $\lambda$ and added back an edge of length at most $\lambda$ in this process, which means that the total length of $\mathcal{T}'$ is strictly less than the SMT we started with. This is a contradiction.
\end{proof}

% \begin{remark}
%     We can find an even tighter bound $\lambda_{0} = \dfrac{1 + (2 - \sqrt{3}) \cdot \tan{\frac{\pi}{n}}}{1 - (4 - \sqrt{3}) \cdot \tan{\frac{\pi}{n}}}$, with $\lambda_1 > \lambda_0 > \lambda_s > \lambda_v$ such that whenever $\lambda \ge \lambda_0$, the SMT follows the singly connected topology, and whenever $\lambda < \lambda_0$, we use multiple vertical gadgets instead, $\forall n \ge 12$.
% \end{remark}

We now proceed to further investigate the connectivity of $\{A_i\}$ and $\{B_i\}$. 
%We start with the portion of this connection that is closer to the polygon $\{A_i\}$.\\

\begin{lemma} \label{steiner_path_from_A_to_A}
    Consider an \smtpoly for $n \ge 13$ and  $\lambda \ge \lambda_{1}$. There must exist $j \in [n]$ and a Steiner point $S_1$, such that terminals $A_j, A_{j + 1}$ form a path $A_j$, $S_1$, $A_{j+1}$ in the SMT and each A-B path passes through $S_1$; where 
    
    $$\lambda_{1} = \frac{1}{1 - 4 \sin \frac{\pi}{n}}$$
\end{lemma}

\begin{proof}
    From Lemma \ref{left_right_turn_path}, we know that there exists one clockwise A-B path and one counterclockwise A-B path in any SMT of $\{A_i\} \cup \{B_i\}$. Let a clockwise A-B path start from $A_r$ and a counter-clockwise A-B path start from $A_l$. Further following from Lemma~\ref{mincut_1}, as there is one edge common to all A-B paths, the clockwise A-B path from $A_r$ and the counter-clockwise A-B path from $A_l$ must share a common edge $S_1S_2$. Therefore, each A-B path must pass through $S_1$ and $S_2$. Without loss of generality we assume that point $S_1$ is closer to the polygon $\{A_i\}$ than $S_2$. This means $S_1$ is either a Stiener point or a terminal vertex of $\{A_i\}$.

    \begin{claim} \label{S_1_notin_A}
        $S_1$ is not a vertex in $\{A_i\}$
    \end{claim}
\begin{proof}
    For the sake of contradiction, we assume that $S_1$
     to be a vertex in $\{A_i\}$, let $S_1 = A_k$ in some SMT $\mathcal T_0$. We disconnect the edge $S_1S_2$ from $\mathcal T_0$, which results in the formation of a forest of two trees $\mathcal T_x$ and $\mathcal T_y$ such that $S_1 = A_k \in \mathcal T_x$ and $S_2 \in \mathcal T_y$.

    $\mathcal T_x$ must contain all vertices of $\{A_i\}$ and $\mathcal T_y$ must contain all vertices of $\{B_i\}$, as there would be an A-B path in the graph otherwise (contradicting that $S_1S_2$ disconnects $\{A_i\}$ and $\{B_i\}$). We replace $\mathcal T_x$ with the SMT of $\{A_i\}$, which is also an MST (from \cite{weng1995steiner}). Since all MST's are of the same length, we choose such an MST in which $A_k$ is not a leaf node. This means $A_{k-1}A_k$ and $A_kA_{k+1}$ are edges in the chosen MST of $\{A_i\}$. We now add back the edge $A_kS_2$, resulting in a connected tree $\mathcal T_0'$ of $\{A_i\} \cup \{B_i\}$. Since we replaced the tree $\mathcal T_x$ with an SMT of $\{A_i\}$, the total length of the $\mathcal T_0'$ must not be more than the total length of the $\mathcal T_0$.

    However, we observe that $A_k$ has three neighbours in $\mathcal T_0'$, which are $A_{k+1}, A_{k-1}, S_2$. However $\angle A_{k-1}A_kA_{k+1} > \frac{2 \pi}{3}$. This means either $\angle S_2A_kA_{k+1} < \frac{2 \pi}{3}$ or $\angle A_{k-1}A_kS_2 < \frac{2 \pi}{3}$. But due to Proposition \ref{smt-prop}, this cannot form an SMT. Therefore $\mathcal T_0'$ is not optimal; and hence, $\mathcal T_0$ cannot be optimal as well. This proves  the claim.
    \end{proof}

    Therefore, $S_1$ must be a Steiner point. Let $P$ and $Q$ be the neighbours of $S_1$ other than $S_2$, such that $\angle PS_1S_2$ is a clockwise turn while $\angle QS_1S_2$ is a counter-clockwise turn. This means that the clockwise A-B path from $A_r$ passes through $P$ and the counter-clockwise A-B path from $A_l$ passes through $Q$. We prove that $P$ and $Q$ are consecutive vertices of $\{A_i\}$ in some SMT of $\{A_i\} \cup \{B_i\}$.


    \begin{claim} \label{PQ_outside_A}
         $P$ and $Q$ cannot simultaneously lie in the interior of the polygon $\{A_i\}$. 
    \end{claim}
\begin{proof}
    We assume for the sake of contradiction that both $P$ and $Q$ lie in the interior of the polygon $\{A_i\}$. On deleting the edge $S_1S_2$, the SMT of $\{A_i\} \cup \{B_i\}$ splits into two trees $\mathcal T_1$ (rooted at $S_1$) and $\mathcal T_2$ (rooted at $S_2$). Further, as all A-B paths pass through $S_1S_2$, all vertices of $\{A_i\}$ must be in $\mathcal T_1$ whereas all vertices of $\{B_i\}$ must lie in  $\mathcal T_2$. Further, $\mathcal T_1$ must be the SMT of $\{A_i\} \cup \{S_1\}$ and $\mathcal T_2$ must be the SMT of $\{B_i\} \cup \{S_2\}$.

    \begin{itemize}
    \item \textit{Case I: One point in $\{S_1, S_2\}$ lies in the interior of $\{A_i\}$ and the other point lies in the exterior of $\{A_i\}$:} This means that the edge $S_1S_2$ crosses some polygon edge of $\{A_i\}$, call it $A_mA_{m+1}$. Let $D$ be the intersection of $A_mA_{m+1}$ and $S_1S_2$. We replace $\mathcal T_1$ with an MST of $\{A_i\}$ that contains the edge $A_mA_{m+1}$ (this can never lead to increase in total tree length due to \cite{weng1995steiner}) and remove the line segment $S_1D$ from $\mathcal T_2$. This forms a tree connecting the terminal set $\{A_i\} \cup \{B_i\}$ which has a total length smaller than the SMT we started with, which is a contradiction.
    
    \item \textit{Case II: Both $S_1$ and $S_2$ lie in the interior of polygon $\{A_i\}$:} We further consider two cases for this:

    \begin{itemize}
        \item Consider that there is at least one polygon edge $A_mA_{m+1}$ of $\{A_i\}$ such that it does not intersect with $\mathcal T_2$. Then we can replace $\mathcal T_1$ by the MST of $\{A_i\}$ which does not contain the edge $A_mA_{m+1}$ without reducing the total edge. However, this will be a connecting tree of $\{A_i\} \cup \{B_i\}$ with a smaller total length than the tree we started with (as we had removed the edge $S_1S_2$ previously), which is a contradiction.
        \item Now, consider that all polygon edges of $\{A_i\}$ intersect with some edge in $\mathcal T_2$. Since $\mathcal T_2$ is rooted at $S_2$ which lies in the interior of $\{A_i\}$, there must be $n$ distinct edges crossing the polygon $\{A_i\}$. However, $\mathcal T_2$ must be the SMT of the points $\{S_2\} \cup \{B_i\}$, which means there can be at most $(n - 1)$ Steiner points other than $S_2$ (from~\Cref{smt-prop}). Therefore, one of these $n$ edges must have a point in $\{B_i\}$ as one of its endpoints, contradicting Observation \ref{no_in_to_B}. 

  
    \end{itemize}
    
    \item \textit{Case III: Both $S_1$ and $S_2$ lie in the exterior of polygon $\{A_i\}$:} This means that the edges $S_1P$ and $S_1Q$ intersect the polygon edges of $\{A_i\}$. Further, from Observation \ref{no_pt_in_PSQ}, there cannot be any terminal inside the triangle $S_1PQ$. Hence, $S_1P$ and $S_1Q$ must intersect the same polygon edge of $\{A_i\}$ (otherwise intermediate vertices from $\{A_i\}$ would lie in the triangle $S_1PQ$). Let this edge be $A_tA_{t+1}$. Let $P_1$ and $Q_1$ be the points of intersection of $A_tA_{t+1}$ with $S_1P$ and $S_1Q$ respectively.

    We now remove the line segments $S_1P_1$ and $S_1Q_1$ from $\mathcal T_1$. This results in another split into two connected trees $\mathcal T_P$ (containing $P_1$, $P$ and a subset of $\{A_i\}$) and $\mathcal T_Q$ (containing $Q_1$, $Q$ and the remaining vertices of $\{A_i\}$). 
    
    We observe that the terminals in $\mathcal T_P$ and $\mathcal T_Q$ form consecutive intervals of the edges in $\{A_i\}$. To see why, consider the opposite, \textit{i.e.} there are vertices $A_{i_1}, A_{i_2}, A_{i_3}, A_{i_4}$ appearing in that order in $\{A_i\}$ such that $A_{i_1}, A_{i_3} \in \mathcal T_P$ whereas $A_{i_2}, A_{i_4} \in \mathcal T_Q$. As $\mathcal T_P$ and $\mathcal T_Q$ lie in the interior of $\{A_i\}$, the path from $A_{i_1}$ to $A_{i_3}$ in $\mathcal T_P$ must cross the path from $A_{i_2}$ to $A_{i_4}$ in $\mathcal T_Q$. However, there cannot be crossing paths in the original SMT of $\{A_i\} \cup \{B_i\}$ (due to~\Cref{smt-prop}). 
    
    Let $\{A_{w}, A_{{w}+1}, \ldots, A_t\}$ be the terminals in $\mathcal T_P$ and the remaining terminals in $\{A_i\}$ are in $\mathcal T_Q$. Again, let $\mathcal T_P'$, $\mathcal T_Q'$ be defined as:
    $$E(\mathcal T_P') = \{A_wA_{w+1}, A_{w+1}A_{w+2}, \ldots, A_{t-1}A_{t}\} \cup \{A_tP_1\}$$
    and
    $$E(\mathcal T_Q') = \{A_{t+1}A_{t+2}, \ldots, A_{w-2}A_{w-1}\} \cup \{Q_1A_{t-1}\}$$
    From Observation \ref{subchain-smt}, we know that $\mathcal T_P'$ is the SMT of $\{A_{w}, A_{{w}+1}, \ldots, A_t\} \cup \{P_1\}$ and $\mathcal T_Q'$ is the SMT of $\{A_{t + 1}, A_{{t}+2}, \ldots, A_{w-1}\} \cup \{Q_1\}$. Therefore, $|\mathcal T_P'| \le |\mathcal T_P|$ and $|\mathcal T_Q'| \le |\mathcal T_Q|$. This means that $| \mathcal T_1 | \ge |\mathcal T_1'|$, where $\mathcal T_1' = \{S_1P_1, S_1Q_1\} \cup \mathcal T_P' \cup \mathcal T_Q'$. Further, $\mathcal T_1'$ is also a connecting tree of $\{S_1\} \cup \{A_i\}$ and as $\mathcal T_1$ is an SMT of $\{S_1\} \cup \{A_i\}$, then $\mathcal T_1'$ must also be an SMT with $|
    \mathcal T_1'| = |\mathcal T_1|$.

    However, We can remove $S_1P_1$ and $P_1A_t$ from $\mathcal T_1'$ and add $S_1A_t$ to get another connecting tree of $\{S_1\} \cup \{A_i\}$, but with shorter total length (as $\overline{S_1P_1} + \overline{P_1A_t} > \overline{S_1A_t}$ from triangle inequality). This contradicts the optimality of $\mathcal T_1'$ which was derived to be an SMT of $\{S_1\} \cup \{A_i\}$.

    \end{itemize}
This proves the claim.
\end{proof}
    


    We proceed to prove a stronger claim regarding $P$ and $Q$.
    
    \begin{claim} \label{PQ_cons}
    $P$ and $Q$ are consecutive vertices of $\{A_i\}$ in any SMT of $\{A_i\} \cup \{B_i\}$.
    
    \end{claim}
    \begin{proof}
    We first prove that $P, Q$ are vertices of $\{A_i\}$. 
    
    From Claim \ref{PQ_outside_A}, we know that at least one among $P$ and $Q$ must not be in the interior of polygon $\{A_i\}$. Without loss of generality, let it be $P$. We now show that $P$ is a vertex of $\{A_i\}$. For the sake of contradiction we assume that $P$ is not a vertex of $\{A_i\}$ \textit{i.e.} $P$ is a Steiner point. Let $\overrightarrow{PF_1}$ and $\overrightarrow{PF_2}$ be tangents from $P$ to $\{A_i\}$ where $F_1, F_2$ are the points of tangency on $\{A_i\}$. As $\overrightarrow{PF_1}$ and $\overrightarrow{PF_2}$ are tangents, $\angle F_1PF_2 < \pi$. We denote the region between the tangents $\overrightarrow{PF_1}$ and $\overrightarrow{PF_2}$ which contains the all the points in $\{A_i\}$ as $\mathcal R$.
    
    From any Steiner point $H$, which lies outside $\mathcal R$, we can choose a neighbour $H_1$ of $H$ such that $\overrightarrow{HH_1}$ is not directed towards $\mathcal R$. Further we now show that there is one neighbour $P_1$ of $P$ such that $P_1$ is not in $\mathcal R$ and $P_1 \neq S_1$. 

    \textit{Case I: $S_1$ lies in $\mathcal R$.} However there must be another neighbour $P_1$ of $P$ not in $\mathcal R$ (as $\angle F_1PF_2 < \pi$) but as $P_1$ is outside $\mathcal R$, we must have $P_1 \neq S_1$.
    
    \textit{Case II: $S_1$ does not lie in $\mathcal R$ (Figure \ref{steiner_path_from_A_to_A_fig}).} As the counter-clockwise A-B path from $A_l$ passes through $Q$, $A_l$ must be to the left of the line $L_{QS_1}$ if the line is given a orientation from $Q$ to $S_1$. This means that one of the tangents from $P$ (Without loss of generality assume it to be $\overrightarrow{PF_1}$) intersects with the line $L_{QS_1}$. Therefore, taking angles in counter-clockwise order, we have:
    
\begin{figure}[h]
\centering
\includegraphics[width=13cm]{claim_proof_diag.png}
\caption{Case II of~\Cref{PQ_cons}}
\label{steiner_path_from_A_to_A_fig}
\end{figure}

    \begin{align*}
        & \angle S_1PF_1 < \frac{\pi}{3} & [\text{as } \overrightarrow{PF_1} \text{ intersects } L_{QS_1}] \\
        \implies & \angle S_1PF_2 = \angle S_1PF_1 + \angle F_1PF_2 < \frac{\pi}{3} + \pi = \frac{4 \pi}{3}\\
        \implies & \angle F_2PS_1 = 2 \pi - \angle S_1PF_2 > 2 \pi - \frac{4 \pi}{3} = \frac{2 \pi}{3}
    \end{align*}
    
    Hence there must exist one neighbour $P_1$ of $P$ lying outside the $\mathcal R$, precisely in the region bounded by the rays $\overrightarrow{PF_2}$ and $\overrightarrow{PS_1}$ with $P_1 \ne S_1$. 
    
    Further we can choose a neighbour $P_2$ of $P_1$ such that $\overrightarrow{P_1P_2}$ is directed away from $\mathcal R$. We can continue choosing $P_2, P_3, \ldots $ such that $\overrightarrow{P_iP_{i+1}}$ is directed away from the region $\mathcal R$. Moreover, the path  $P, P_1, P_2, \ldots$ must end at some point $B_k$ as it cannot end in any vertex of $\{A_i\}$ (since all vertices of $\{A_i\}$ are in $\mathcal R$). Now, let $\mathcal C_1$ be the path from $A_r$ to $B_k$ (which passes through $P$ and $P_1$) and let $\mathcal C_2$ be the counter-clockwise A-B path from $A_l$ (passing through $Q$, $S_1$ and $S_2$). We observe that $\mathcal C_1$ and $\mathcal C_2$ are two edge disjoint A-B paths, which is a contradiction to~\Cref{mincut_1}. This proves that $P$ is indeed a vertex in $\{A_i\}$. Therefore by Claim \ref{PQ_outside_A}, $Q$ does not lie inside $\{A_i\}$ and repeating this same argument on $Q$ yields that $Q$ is also a vertex of $\{A_i\}$. 

    Now, to prove that $P$ and $Q$ are consecutive vertices of $\{A_i\}$, we use Observation \ref{no_pt_in_PSQ}. Observation \ref{no_pt_in_PSQ} implies that there must not be any other point of the SMT in the triangle $PS_1Q$. This means that $P$ and $Q$ must be consecutive vertices of $\{A_i\}$, otherwise all polygon vertices of $\{A_i\}$ occurring in between $P$ and $Q$ would be inside the triangle $PS_1Q$ (as $\angle PS_1Q = \dfrac {2 \pi}{3}$ and $n \ge 13$). This proves the claim.
    \end{proof}

    Therefore, $P$ and $Q$ are consecutive vertices $A_j, A_{j+1}$ of the polygon $\{A_i\}$, for some $j \in [n]$ such that $A_j$, $S_1$, $A_{j+1}$ is a path in the SMT, where $S_1$ is a Steiner point lying on all A-B paths.
\end{proof}

Our next step is to investigate some more structural properties of an SMT for $\{A_i\} \cup \{B_i\}$. From \cite{du1987steiner}, we may guess that there would be a lot of polygon edges of both $\{A_i\}$ and $\{B_i\}$ in an SMT. We prove the following Lemma, stating that there is an SMT of $\{A_i\} \cup \{B_i\}$ which contains $(n - 2)$ polygon edges of $\{A_i\}$.

\begin{lemma} \label{n-2_A_poly_edges}
    For an \smtpoly with aspect ratio $\lambda$, 
    $\lambda > \lambda_{1} = \frac{1}{1 - 4 \sin \frac{\pi}{n}}$, let $S_1$ be the Steiner point such that all A-B paths pass through $S_1$. Let $A_j$ and $A_{j+1}$ be vertices of $\{A_i\}$ which are connected to $S_1$. Then, there exists an SMT of $\{A_i\} \cup \{B_i\}$ having $(n - 2)$ polygon edges of $\{A_i\}$ other than $A_jA_{j+1}$.
\end{lemma}

\begin{proof}
    Let $\mathcal T_0$ be any SMT of $\{A_i\} \cup \{B_i\}$. From Lemma \ref{steiner_path_from_A_to_A}, we know that there exists $S_1$, $A_j$ and $A_{j + 1}$ such that $S_1$ is a Steiner point which is a part of all A-B paths, and $A_jS_1A_{j+1}$ is a path $\mathcal T_0$.

    From $\mathcal T_0$, we remove the edges $S_1A_j$ and $S_1A_{j + 1}$ and add the edge $A_jA_{j+1}$. This results in a forest of two disjoint trees $\mathcal T_x$ and $\mathcal T_y$. One of these trees (say $\mathcal T_x$) must contain all terminal points from $\{A_i\}$ and the other tree must contain all terminals from $\{B_i\}$, as no more A-B paths exist after we removed the edges $S_1A_j$ and $S_1A_{j+1}$. Therefore we have $|\mathcal T_0| = |\mathcal T_x| + |\mathcal T_y| - 1 + \overline{S_1A_j} + \overline{S_1A_{j+1}}$.

    We further replace $\mathcal T_x$ with a Euclidean minimum spanning tree $\mathcal T_x'$ of $\{A_i\}$ such that the edge $A_jA_{j+1}$ is present in $\mathcal T_x'$. From \cite{du1987steiner}, we know that $|\mathcal T_x'| \le |\mathcal T_x|$. We now remove the edge $A_jA_{j+1}$ and add back the edges $S_1A_j$ and $S_1A_{j+1}$ which gives a connected tree $\mathcal T_0'$ of $\{A_i\} \cup \{B_i\}$. Therefore we have:
    $$|\mathcal T_0'| = |\mathcal T_x'| + |\mathcal T_y| - 1 + \overline{S_1A_j} + \overline{S_1A_{j+1}} \le |\mathcal T_x| + |\mathcal T_y| - 1 + \overline{S_1A_j} + \overline{S_1A_{j+1}} = |\mathcal T_0|$$

    This means $\mathcal T_0'$ must be an SMT. However, all polygon edges of polygon $\{A_i\}$ appearing in $\mathcal T_x'$ also appear in $\mathcal T_0'$ as well, except $A_jA_{j+1}$. Therefore, the SMT $\mathcal T_0'$ has $(n - 2)$ polygon edges of the polygon $\{A_i\}$.
\end{proof}

With these set of results in hand, we can now show that there exists an SMT of $\{A_i\} \cup \{B_i\}$ following a \emph{singly connected topology}. To show this, we start with any \smtpoly, $\mathcal T_0$, that satisfies all the results derived so far and transform it into a Steiner tree of \emph{singly connected topology} having total length not longer than the initial Steiner tree $\mathcal T_0$.

\begin{theorem} \label{final_proof}
There exists an \smtpoly following a singly connected topology for $n \ge 13$ and $\lambda \ge \lambda_{1}$, where
$$
% \lambda_{1} = \frac{2 \sin{\frac{\pi}{n}} + 1}{1 - 6 \sin \frac{\pi}{n}}
\lambda_{1} = \frac{1}{1 - 4 \sin \frac{\pi}{n}}
$$
\end{theorem}

\begin{proof}
% We know that there is a single path between two points in $\{A_i\}$ containing a Steiner point and it is of the form $A_jS_1A_{j+1}$ (from Lemma \ref{steiner_path_from_A_to_A}). The rest of the points in $\{A_i\}$ must be connected by $(n - 2)$ polygon edges of $\{A_i\}$.

Let $\mathcal{T}_0$ be any SMT of $\{A_i\} \cup \{B_i\}$ which satisfies the properties of Lemma \ref{n-2_A_poly_edges}. Further, from Lemma \ref{steiner_path_from_A_to_A}, there is a Steiner point $S_1$ which lies on all A-B paths, and there are two consecutive vertices $A_j$, $A_{j+1}$ such that $A_j$, $S_1$, $A_{j+1}$ is a path in $\mathcal T_0$. As $\mathcal T_0$ satisfies the property of Lemma \ref{n-2_A_poly_edges}, $\mathcal T_0$ has $(n - 2)$ polygon edges of $\{A_i\}$ excluding the edge $A_jA_{j+1}$.

Let $H$ be the point in the interior of the polygon $\{A_i\}$  such that $HA_jA_{j+1}$ form an equilateral triangle. As $n > 6$, the common centre $O$ of $\{A_i\}$ and $\{B_i\}$ does not lie inside the triangle $HA_jA_{j+1}$. Now, we modify $\mathcal T_0$ as follows:

\begin{enumerate}
    \item Remove edges $A_jS_1$, $S_1A_{j+1}$ and add edge $S_1H$ to get the forest $\mathcal{T}_1$. We know from \cite{hwang1992steiner} that $S_1$, $S_2$ and $H$ are collinear and this transformation does not change the total length. Therefore $|\mathcal{T}_0| = |\mathcal{T}_1|$. Here, $|\mathcal{T}_1|$ denotes the sum of the lengths of edges present in $\mathcal{T}_1$.
    \item Add edge $HO$ and remove all polygon edges of $\{A_i\}$ to get $\mathcal T_2$. Therefore $|\mathcal T_2| = |\mathcal T_1| + \overline{HO} - (n - 2) = |\mathcal T_0| + \overline{HO} - (n - 2)$. We observe that $\mathcal T_2$ is a tree connecting  the points in $\{B_i\} \cup \{O\}$.
    \item Let $S_0$ be the Torricelli point of the triangle $OB_jB_{j+1}$. Let $\mathcal T_3$ be the Steiner tree of $\{B_i\} \cup \{O\}$ with edges $S_0O$, $S_0B_j$, $S_0B_{j+1}$ and other points in $\{B_i\}$ connected through $(n - 2)$ polygon edges of the polygon $\{B_i\}$. From \cite{weng1995steiner}, we know that $\mathcal T_3$ is the SMT of $\{B_i\} \cup \{O\}$. Therefore $|\mathcal T_3| \le |\mathcal T_2| = |\mathcal T_0| + \overline{HO} - (n - 2)$. Further we know that $H$ lies on the edge $OS_0$ (as $O$, $S_0$ and $H$ lie on the perpendicular bisector of $B_j$ and $B_{j + 1}$).
    \item Remove edge $S_0O$ and add edge $S_0H$ to get $\mathcal T_4$. As $H$ lies on the edge $OS_0$, we have $|\mathcal T_4| = |\mathcal T_3| - \overline{OH} \le |\mathcal T_0| - (n - 2)$.
    \item Let $S_3$ be the intersection of the circumcircle of triangle $A_jHA_{j + 1}$ (from Lemma \ref{lambda_v} the intersection exists as $\lambda_1 \ge \lambda_v$ for $n \ge 13$). Remove the edge $S_3H$ and add the edges $S_3A_j$ and $S_3A_{j + 1}$ to get $\mathcal T_5$. Again, from \cite{hwang1992steiner} we know that this transformation does not change the total length. Hence $|\mathcal T_5| = |\mathcal T_5| \le |\mathcal T_0| - (n - 2)$. Moreover, as $\lambda > \lambda_v$, we observe that $\{A_j, B_j, A_{j+1}, B_{j+1}, S_3, S_0\}$ form the vertices of the vertical gadget and points $O$, $H$, $S_3$, $S_0$ appear in that order on the perpendicular bisector of $B_j$ and $B_{j+1}$.
    \item Add back the $(n - 2)$ polygon edges of $\{A_i\}$ which were removed in the second step to get $\mathcal T_6$. Therefore $|\mathcal T_4| = |\mathcal T_5| + (n - 2) \le |\mathcal T_0|$. We further observe that $\mathcal T_6$ is a Steiner tree connecting the points $\{A_i\} \cup \{B_i\}$ with a singly connected topology.
\end{enumerate}

Therefore we started with an arbitrary SMT $\mathcal T_0$ and transformed it into a Steiner tree $\mathcal T_6$ with a singly connected topology (where $\{A_j, B_j, A_{j+1}, B_{j+1}, S_3, S_0\}$ form the vertices of the vertical gadget) which has a total length not worse than $\mathcal T_0$. Hence $\mathcal T_6$ must be an SMT of $\{A_i\} \cup \{B_i\}$. This proves the theorem.
\end{proof}

\begin{remark}
    \cref{final_proof} determines the exact structure of the \smtpoly. Further from~\cref{trapezoids} we determine the exact method to construct the two additional Steiner points in $\mathcal O(1)$ steps - note that this construction time is independent of the integer $n$ or the real number $\lambda$. Therefore, \smtpoly for $n \ge 13$ and $\lambda \ge \lambda_1$ is solvable in polynomial time.
\end{remark}

%This means that there is an SMT of $\{A_i\} \cup \{B_i\}$ which will have a \emph{singly connected topology} for $n \ge 13$ and $\lambda \ge \lambda_{1}$. \\

Note that the total length of any \smtpoly, when $n \ge 13$ and $\lambda \ge \lambda_{1}$, is
\begin{align*}
    & |\mathcal T_6| = |\text{vertical gadget}| + |(n - 2) \text{ edges of $\{B_i\}$}| + |(n - 2) \text{ edges of $\{A_i\}$}| \\
    \implies & |\mathcal T_6| = \bigg(\dfrac{(\lambda - 1)}{2 \tan \frac \pi n} + \dfrac {\sqrt 3 (\lambda + 1)} {2}\bigg) + (n - 2) \cdot \lambda + (n - 2)\\
    \implies & {|\mathcal T_6| = \dfrac{(\lambda - 1)}{2 \tan \frac \pi n} + \bigg(n - 2 + \frac{\sqrt{3}}{2}\bigg) (\lambda + 1)}\\
\end{align*}
\noindent Further, $\lambda_1$ converges to 1 very quickly with increasing $n$ (plotted in Figure \ref{lambda_1_plot}):

    \vspace{0.3cm}
    \begin{center}
    \begin{tabular}{| c | c | c | c | c | c |}
    \hline
    $n$ & 13 & 20 & 40 & 100 & 500 \\
    \hline
    $\lambda_1$ & 23.3987 & 2.6719 & 1.4574 & 1.1437 & 1.0258\\
    \hline
    \end{tabular}
    \end{center}
    \vspace{0.3cm}

    
\begin{figure}[h]
\centering
\includegraphics[width=15cm]{lambda_1_and_lambda_v.png}
\caption{Plot of $\lambda_1$ \& $\lambda_v$ against $n$}
\label{lambda_1_plot}
\end{figure}

    
This means for large sized $n$ and for ratios that are not too small, the SMT will follow a \textit{singly connected topology}. 

% \begin{Lemma}
%     The SMT of concentric regular polygons $\{A_i\}$ and $\{B_i\}$ with $n \ge 13$ and $\lambda \ge \lambda_{1}$, all ``connections'' between $\{A_i\}$ and $\{B_i\}$ will be full Steiner subtrees forming vertical gadgets. 
% \end{Lemma}

% \begin{proof}
%     Clearly there cannot be any Horizontal Forks appearing as Horizontal Gadgets as $\lambda \ge \lambda_{0} \ge \lambda_s \ge \lambda_v$ and we can replace the Horizontal Gadget with a vertical gadget to get an even smaller length.
    
%     For any other topology of the full Steiner subtree connecting $\{A_i\}$ and $\{B_i\}$ with only one edge needed to remove in order to disconnect $\{A_i\}$ and $\{B_i\}$ (and the rest of the tree connected via polygon edges), there must exist terminals $A_j, A_{j + 1}$ and a steiner point $S_0$ such that $A_jS_0A_{j+1}$ is a path in the SMT. Let $O$ be the common center of $\{A_i\}$ and $\{B_i\}$. We remove all the edges in the SMT of that were a edges of the inner polygon, as well as the edges $A_jS_0$ and $S_0A_{j+1}$. We then add the edge $S_0O$ Since this was the SMT for $\{A_i\} \cup \{B_i\}$, this new construction gives us an SMT for $\{B_i\} \cup \{O\}$, contradicting the main result of \cite{weng1995steiner}.

%     \begin{figure}[h]
%         \centering
%         \includegraphics[width=15cm]{vert_proof.png}
%         \caption{Contradiction of \cite{weng1995steiner}}
%         \label{fig:vert_proof}
%     \end{figure}
    
% \end{proof}

% This implies that whenever $n \ge 12$ and $\lambda > \lambda_0$, the SMT follows the Singly Connected topology. Further $\lambda_0$ quickly converges to 1:

%     \vspace{0.3cm}
%     \begin{center}
%     \begin{tabular}{| c | c | c | c | c | c |}
%     \hline
%     $n$ & 12 & 20 & 40 & 100 & 500 \\
%     \hline
%     $\lambda_0$ & 2.732 & 1.627 & 1.243 & 1.086 & 1.017\\
%     \hline
%     \end{tabular}
%     \end{center}
%     \vspace{0.3cm}
    
% This means for large sized $n$ and for ratios that are not too small, the SMT will follow the Singly connected topology. 

%Few simulated examples (with the help of \href{http://www.geosteiner.com/}{[2]}) are as follows

% \begin{figure}[h!]
% \centering
% \subfloat[\centering $\lambda = 2.73205$] {\includegraphics[width=5cm]{12_vert.png}}
%     \qquad
% \subfloat[\centering $\lambda = 2.73206$] {\includegraphics[width=5cm]{12_sing.png}}
% \caption{points for $n = 12$, $\lambda = 2.732050$}
% \end{figure}


% \begin{figure}[h!]
% \centering
% \subfloat[\centering $\lambda = 1.626797$] {\includegraphics[width=5cm]{20_vert.png}}
%     \qquad
% \subfloat[\centering $\lambda = 1.626798$] {\includegraphics[width=5cm]{20_sing.png}}
% \caption{points for $n = 20$, $\lambda = 1.6267974$}
% \end{figure}




\section{\ESMT on $f(n)$-Almost Convex Point Sets}\label{sec:exact_algo}


% \begin{figure}[h]
% \centering
% \subfloat[\centering SMT for some $\mathcal P$ with $n = 6$] {\includegraphics[width=6cm]{SMT_6.png}}
%     \qquad
% \subfloat[\centering \centering SMT for some $\mathcal P$ with $n = 10$] {\includegraphics[width=6cm]{SMT_10.png}}
% \caption{Some examples of SMT}
% \end{figure}

In this section, we design an exact algorithm for \ESMT on $f(n)$-Almost Convex Point Sets running in time $2^{\OO(f(n)\log n)}$. Note that $f(n) \leq n$ is always true. Therefore, we are given as input a set $\mathcal{P}$ of $n$ points in the Euclidean Plane such that $\mathcal P$ can be partitioned as $\mathcal P = \mathcal P_1 \uplus \mathcal P_2$, where $\mathcal P_1$ is the convex hull of $\mathcal P$ and $|\mathcal P_2| = f(n)$.

First, we look into some mathematical results and computational results to finally arrive at the algorithm for solving \ESMT on $f(n)$-Almost Convex Point Sets. 

We know that the SMT of $\mathcal P$ can be decomposed uniquely into one or more full Steiner subtrees, such that two full Steiner subtrees share at most one node~\cite{hwang1992steiner}. In the following lemma, we further characterize one full Steiner subtree.

\begin{lemma}
\label{existence_of_leaf}
Let $\mathbb{F}$ be the full Steiner decomposition of an SMT of $\mathcal{P}$.  Then there exists a full Steiner subtree $\mathcal F \in \mathbb{F}$ such that $\mathcal F$ has at most one common node with at most one other full Steiner subtree in $\mathbb{F}$.
\end{lemma}
\begin{proof}
If the SMT of $\mathcal P$ is a full Steiner tree, then the statement is trivially true.

Otherwise, we assume that the SMT of $\mathcal P$ has full Steiner subtrees, $\mathbb{F} = \{\mathcal F_1, \mathcal F_2, \ldots, \mathcal F_m\}$, $m \ge 2$. Now, for the sake of contradiction, we assume that for each full Steiner subtree $\mathcal F_j$ there are atleast two other full Steiner subtrees ${\sf Neitree}(\mathcal F_j)^1,{\sf Neitree}(\mathcal F_j)^2 \in \mathbb{F}$ and two terminals $P_1(\mathcal F_j),P_2(\mathcal F_j) \in \mathcal{P}$ such that $P_i(\mathcal F_j) \in V(\mathcal F_j) \cap V({\sf Neitree}(\mathcal F_j)^i)$, $i \in \{1,2\}$. Now, let us construct a walk $W$ in the SMT of $\mathcal{P}$. Starting from $P_1(\mathcal F_1)$ of the full Steiner subtree $\mathcal F_1 \in \mathbb{F}$, we include the path in $\mathcal F_1$ connecting to $P_2(\mathcal F_1)$. Note that $P_2(\mathcal F_1)$ is also contained in ${\sf Neitree}(\mathcal F_1)^2$. Let ${\sf Neitree}(\mathcal F_1)^2 = \mathcal F_{w_1}$ for some $w_1\in [m], w_1\neq 1$. Also let $P_2(\mathcal F_1) = P_1(\mathcal F_{w_1})$.
Then, we know that there is a $P_2(\mathcal F_{w_1})$. In $W$, we include the path in $\mathcal F_{w_1}$ connecting $P_1(\mathcal F_{w_1})$ to $P_2(\mathcal F_{w_1})$. In general, suppose the $i^{th}$ full Steiner subtree to be considered in building the walk is $\mathcal F_{w_{i-1}}$ which was reached via point $P_1(\mathcal F_{w_{i-1}})$. Then we include in $W$ the path in $\mathcal F_{w_{i-1}}$ connecting $P_1(\mathcal F_{w_{i-1}})$ and $P_2(\mathcal F_{w_{i-1}})$. Thus, we can indefinitely keep constructing the walk $W$ as for each $\mathcal F_{w_{i-1}}$ both $P_1(\mathcal F_{w_{i-1}}), P_2(\mathcal F_{w_{i-1}})$ always exist. However, since there are $m$ full Steiner subtrees this means that there is an $\mathcal F_k \in \mathbb{F}$ and two indices $i\neq j$ such that $\mathcal F_k = \mathcal F_{w_i} = \mathcal F_{w_j}$. Thus, there exists a cycle in $W$, which implies that there is a cycle in the SMT of $\mathcal{P}$ (contradiction). Therefore, there must be at least one full Steiner subtree that has at most one common terminal with at most one other full Steiner subtree.
\end{proof}
\begin{remark}
A full Steiner subtree of the SMT of $\mathcal P$ has the topology of a tree. Thus, from Lemma~\ref{existence_of_leaf}, we conclude that a full Steiner subtree, that has at most one common terminal with at most one other full Steiner subtree, has at least one leaf of the SMT.
\end{remark}


\begin{definition}
Let the full Steiner subtrees, that have at most one terminal shared with at most one other full Steiner subtree, be called \textbf{leaf full Steiner subtrees}. Let the terminal which is shared be called the \textbf{pivot} of the leaf full Steiner subtree.
\end{definition}


\begin{figure}[h]
\centering
\subfloat[\centering Full Steiner Subtrees of Figure 1(a)] {\includegraphics[width=6cm]{sub_6.png}}
    \qquad
\subfloat[\centering \centering Full Steiner Subtrees of Figure 1(b)] {\includegraphics[width=6cm]{sub_10.png}}
\caption{Leaf full Steiner subtrees enclosed in ellipses, other full Steiner subtrees enclosed in rectangles, pivots of leaf full Steiner subtrees encircled}
\end{figure}


\begin{lemma}
\label{leaf_non_leaf_structure}
Let $\mathcal F$ be a leaf full Steiner Subtree of the SMT of $\mathcal P$, with terminal points $\mathcal P_{\mathcal{\mathcal F}} \subseteq \mathcal{P}$ and having pivot $P_{\mathcal F}$. Deleting $\mathcal F\setminus \{P_{\mathcal F}\}$ from the SMT of $\mathcal P$ gives us an SMT of the terminal points $((\mathcal P - \mathcal{P}_{\mathcal F}) \cup \{P_{\mathcal F}\})$.
\end{lemma}

\begin{proof}
Firstly, we observe that deleting $\mathcal F\setminus \{P_{\mathcal F}\}$ from the SMT of $\mathcal P$ will indeed give us a tree, as $\mathcal F$ is a leaf full Steiner subtree. Let us call this tree $\mathcal Y$.

Now for the sake of contradiction, we assume that the total length of $\mathcal Y$ is strictly larger than the SMT $\mathcal F'$ of $((\mathcal P - \mathcal{P}_{\mathcal F}) \cup \{P_{\mathcal F}\})$. However, this means, the total length of $\mathcal F' \cup \mathcal F$ is strictly smaller than that of the SMT of $\mathcal P$. As $\mathcal F' \cup \mathcal F$ is also a Steiner tree of $\mathcal P$,  this contradicts the minimality of the initial SMT of $\mathcal P$.
\end{proof}

Now we are ready to describe the algorithm. Recall that $\mathcal P$ is partitioned as $\mathcal P = \mathcal{P}_1 \uplus \mathcal{P}_2$, where $\mathcal{P}_1$ is the convex hull of $\mathcal P$ and $\mathcal{P}_2$ is the set of $f(n)$ points lying in the interior of $\mathcal P_1$. For the sake of brevity of notations let $|\mathcal{P}_2| = k$.

\begin{lemma}
\label{four_power_n}
Let $\mathcal P$ be a $k$-Almost Convex Point Set. A minimum FST of a subset $\mathcal S$ of $\mathcal P$ can be found in $\mathcal O(4^{|{\mathcal S}|} \cdot |\mathcal S| ^ k)$ time.
\end{lemma}

\begin{proof}
We observe that for any $\mathcal S \subseteq \mathcal P$, $\mathcal S$ forms a convex polygon with at most $k$ points lying in the interior. For $|{\mathcal S}| \le 2$, the statement of the lemma is trivially true. Hence we assume that $|{\mathcal S}| > 2$.

From \cite{hwang1992steiner}, the number of full Steiner topologies of $\mathcal{S}$ is 

$$\frac{|\mathcal S|!}{|\mathrm{CH}(\mathcal{S})|!} \cdot \frac {\binom{2|{\mathcal S}| - 4}{|{\mathcal S}| - 2}}{|{\mathcal S}| - 1}$$

However, we know that:

$$\frac {\binom{2|{\mathcal S}| - 4}{|{\mathcal S}| - 2}}{|{\mathcal S}| - 1} < \frac{\binom{2|{\mathcal S}|}{|{\mathcal S}|}}{|{\mathcal S}|} < \frac{\sum \limits_{r = 0}^{2|{\mathcal S}|} \binom{2|{\mathcal S}|}{k}}{|{\mathcal S}|} = \frac{2 ^{2|{\mathcal S}|}}{|{\mathcal S}|} = \frac{4^{|{\mathcal S}|}}{|{\mathcal S}|}$$

And,

$$\frac{|\mathcal S|!}{|\mathrm{CH}(\mathcal{S})|!} < \frac{|\mathcal S|!}{(|\mathcal S| - k)!} < |\mathcal S|^k$$

Therefore, the number of full Steiner topologies of $\mathcal{S}$ is at most $4^{|{\mathcal S}|} |\mathcal S| ^ {k - 1}$. Each of these topologies can be enumerated and using \emph{Melzak's FST Algorithm}, we can also find the SMT realizing each such full Steiner topology in linear time, as given in \cite{hwang1992steiner}. Therefore to iterate over all topologies and find a minimum takes at most time:

$$(4^{|{\mathcal S}|} \cdot |\mathcal S| ^ {k - 1}) \cdot \mathcal O(|\mathcal S|) = \mathcal \mathcal O(4^{|{\mathcal S}|} \cdot|\mathcal S| ^ k)$$

\end{proof}

Now, we find the time required for extending the results of Lemma~\ref{four_power_n} to all subsets of $\mathcal{P}$.
\begin{lemma}
\label{five_power_n}
Let $\mathcal P$ be a $k$-Almost Convex Set. Computing a minimum FST \textbf{for all} subsets $\mathcal S \subseteq \mathcal P$ can be done in $\mathcal O(n^k \cdot 5^{n})$ time.
\end{lemma}
\begin{proof}

Using Lemma \ref{four_power_n}, we can get a minimum FST for a single subset $\mathcal S \subseteq \mathcal P$ in $\mathcal O(4^{|{\mathcal S}|} |\mathcal S| ^ k)$ time. Moreover, we know that the number of subsets of $\mathcal P$ that are of size $r$ is $\binom n r$. This means that the total time to compute a minimum FST \textbf{for all} subsets $\mathcal S \subseteq \mathcal P$, time taken is: 


$$\sum \limits_{r = 0} ^ {n} \binom{n}{r} \cdot \mathcal O(4^r \cdot r^k) = \sum \limits_{r = 0} ^ {n} \binom{n}{r} \cdot \mathcal O(n^k \cdot 4^r) = \mathcal O(n^k \cdot (1 + 4)^n) = \mathcal O(n^k \cdot 5^n)$$

\end{proof}

For each $\mathcal S \subseteq \mathcal P$, we denote by $\mathcal F_{\mathcal S} $ a minimum FST of $\mathcal S$ and by $\mathcal T_{\mathcal S} $ the SMT of $\mathcal S$.

\begin{lemma}
\label{single_subset_SMT}
The SMT of subset $\mathcal S \subseteq \mathcal P$, $\mathcal T_{\mathcal S}$, can be found in $\mathcal O(|{\mathcal S}| \cdot 2^{|{\mathcal S}|})$ time, given that we have pre-computed $\mathcal T_{\mathcal R}$ and $\mathcal F_{\mathcal R}$, $\forall \mathcal R \subseteq \mathcal S$.
\end{lemma}

\begin{proof}

If $\mathcal T_{\mathcal S}$ was a full Steiner tree then it would be $\mathcal F_{\mathcal S}$. Otherwise, $\mathcal T_{\mathcal S}$ contains multiple full Steiner subtrees.

Let $\mathcal F$ be a leaf full Steiner subtree of $\mathcal T_{\mathcal{S}}$ with pivot $P_{\mathcal F}$. Therefore from Lemma \ref{leaf_non_leaf_structure} we have $\mathcal T_{\mathcal S} = \mathcal T_{((\mathcal S - V(\mathcal F)) \cup \{P_{\mathcal F}\})} \cup \mathcal F$. Therefore we can iterate over all subsets $\mathcal R \subset \mathcal S$ and all terminals $P \in \mathcal{R}$, and take the minimum-length tree among $\mathcal T_{((\mathcal S - \mathcal R)  \cup \{P\})} \cup \mathcal F_{\mathcal R}$. Since we are iterating over all $\mathcal R \subset \mathcal S$, and all $P \in \mathcal R$, we are guaranteed to get $\mathcal R = V(\mathcal F) \cap \mathcal{S}$ and $P = P_{\mathcal F}$ on one such iteration.

Now, as there are $\mathcal O(2^{|\mathcal S|})$ possibilities of $\mathcal R \subset \mathcal S$ and $\mathcal O(|\mathcal S|)$ possibilities of $P \in \mathcal R$, we have $\mathcal O(|S| \cdot 2^{|\mathcal S|})$ possibilities of the pair $(\mathcal R, P)$. Therefore the total time required for iterating is $\mathcal O(|{\mathcal S}| \cdot 2^{|{\mathcal S}|})$.

\end{proof}

\begin{lemma}
\label{multi_subset_SMT}
SMTs for all subsets $\mathcal S \subseteq \mathcal P$, $\mathcal T_{\mathcal S}$ can be found in $\mathcal O(n \cdot 3 ^n)$ time, given that we have precomputed a minimum FST $\mathcal F_{\mathcal S}$ $\forall \mathcal S \subseteq \mathcal P$.
\end{lemma}

\begin{proof}

Using Lemma \ref{single_subset_SMT}, we can get the SMT $\mathcal T_{\mathcal S}$, for a single subset $\mathcal S \subseteq \mathcal P$ in $\mathcal O(r \cdot 2^r)$ time, where $|{\mathcal S}| = r$. Moreover, we know that the number of subsets of $\mathcal P$ that are of size $r$ is $\binom n r$. This means that the total time to compute the SMT for all subsets $\mathcal S \subseteq \mathcal P$, time taken is: 

$$\sum \limits_{k = 0} ^ {n} \binom{n}{k} \cdot \mathcal O(k \cdot 2^k) = \mathcal O(n \cdot (1 + 2)^n) = \mathcal O(n \cdot 3^n)$$

However, to apply Lemma \ref{single_subset_SMT} on some subset $\mathcal S \subseteq \mathcal P$ for computing $\mathcal T_{\mathcal S}$, we must also have $\mathcal T_{\mathcal R}$ precomputed for all $\mathcal R \subseteq \mathcal S$. This can be guaranteed by computing $\mathcal T_{\mathcal S}$ and $\mathcal{F}_{\mathcal{S}}$, for all subsets $\mathcal S \subseteq \mathcal P$ in an increasing order of $|\mathcal S|$ (or any order which guarantees that the subsets of $\mathcal S$ are processed before $\mathcal S$).

\end{proof}

Finally, we state our algorithm.

\begin{theorem}
\label{final_theorem}

An SMT $\mathcal T_{\mathcal P}$ of a $k$-Almost Convex Set $\mathcal P$ of terminals can be computed in $\mathcal O(n^k \cdot 5^n)$ time.

\end{theorem}

\begin{proof}

Consider the following algorithm: 
\begin{algorithm}[H]
\caption{Computation of $\mathcal{T_P}$ ~~~ \textbf{Input:} $\mathcal P$ }\label{alg:main algo}
\begin{algorithmic}[1]
\For{all $\mathcal S \subseteq \mathcal P$} 
\State Compute $\mathcal {F_S}$ \Comment{Using Lemma \ref{five_power_n}}
\EndFor \Comment{This takes $\mathcal O(n^k \cdot 5^n)$ time}
\For{all $\mathcal S \subseteq \mathcal P$}
\State Compute $\mathcal {T_S}$ \Comment{Using Lemma \ref{multi_subset_SMT}}
\EndFor \Comment{This takes $\mathcal O(n \cdot 3^n)$ time}
\State \Return $\mathcal{T_P}$ \Comment{Total runtime is $\mathcal O(n^k \cdot 5^n + n \cdot 3^n) = \mathcal O(n^k \cdot 5^n)$}
\end{algorithmic}
\end{algorithm}


% \begin{itemize}
%     \item Using Theorem \ref{five_power_n}, we precompute $\mathcal F_{\mathcal S}$ for all subsets $\mathcal S  \subseteq \mathcal P$. This takes $\mathcal O(5^n)$ time.
%     \item Using Theorem \ref{multi_subset_SMT} and the precomputation from the previous step, we can get $\mathcal T_{\mathcal S}$ for all subsets $\mathcal S  \subseteq \mathcal P$. This takes $\mathcal O(n \cdot 3^n)$ time.
%     \item We return $\mathcal T_{\mathcal P}$ as our final answer in time $\mathcal O(5^n + n \cdot 3^n) = \mathcal O(5^n)$.
% \end{itemize}

Hence we have an SMT of a $k$-Almost Convex Point Set $\mathcal P$ in $\mathcal O(n^k\cdot 5^n)$ time.


\end{proof}

% \vspace{0.5cm}
% \section{Extending the Algorithm}
% \vspace{0.5cm}

%     We now consider the configuration where there are a small number of points inside a convex polygon. Formally, we wish to find the SMT of $\mathcal{P} \cup \mathcal{P}_{in}$, where $\mathcal{P}_{in}$ is a set of $k$ points lying strictly inside the convex polygon $\mathcal{P}$, such that $k$ is small.\\
    
%     We can apply \textbf{Algorithm 1} (derived in \ref{final_theorem}) in this case as well. However, the analysis of run time does not remain the same. We will now analyse the runtime of the same algorithm when the input is $\mathcal{P}_1 = \mathcal{P} \cup \mathcal{P}_{in}$, to get its SMT. 
    
%     \begin{align*}
%         \text{The number of Full Steiner Topologies of } \mathcal{P}_1 = & \frac{|\mathcal{P}_1|!}{|\text{convexhull}(\mathcal{P}_1)|!} \cdot \Bigg(\frac {\binom{2|{\mathcal P_1}| - 4}{|{\mathcal P_1}| - 2}}{|{\mathcal P_1}| - 1}\Bigg)\\
%         < & \frac{(n + k)!}{|\mathcal{P}|!} \cdot \frac{4^{|\mathcal P_1|}}{|\mathcal P_1|}\\
%         = & \frac{(n + k)!}{n!} \cdot \frac{4^{(n + k)}}{n + k}\\
%     \end{align*}
    
%     Similarly for each subset $\mathcal S_1 \subset \mathcal{P}_1$ of size $m$ has number of Full Steiner topologies atmost to $\Big( \dfrac{m!}{(m - k)!} \cdot \dfrac{4^m}{m} \Big)$ as the number of points strictly inside the convex hull cannot increase. And hence running Melzak's Full Steiner Tree Algorithm for each such topology would take time of $\mathcal O \Big(\frac{m!}{(m - k)!} \cdot \frac{4^m}{m} \cdot m \Big)$ = $\mathcal O((n + k)^k \cdot 4^m)$.\\

%     Hence running \textbf{Algorithm 1} on all subsets of $\mathcal P_1$ would take time:
    
%     $$\sum \limits_{m = 0}^{n + k}{(n + k) ^ k \cdot 4^m \cdot \binom{n + k}{m}} = \mathcal O ((n + k) ^ k \cdot 5^{(n + k)})$$
    
%     Further, for the part of the algorithm performing the computation of $\mathcal T_S$, $\forall \mathcal{S} \subseteq \mathcal P_1$ (as in \ref{multi_subset_SMT}), the runtime would not exceed $\mathcal O (4 ^ {n + k})$.\\
    
%     Therefore the runtime of the entire algorithm, when ran on $\mathcal P_1$, will be $\mathcal O ((n + k) ^ k \cdot 5^{(n + k)})$. \\

The above theorem gives us several improvements in special classes of inputs, based on the number of input points lying inside the convex hull of the input set, as described in the following corollary. Let there be an $f(n)$-Almost Convex Point Set $\mathcal P$ containing $n$ points. Recall that $\mathcal{P} = \mathcal{P}_1 \uplus \mathcal{P}_2$, $\mathcal{P}_1$ containing the points on the convex hull of $\mathcal{P}$, and $|\mathcal{P}_2| = f(n)$. It is only possible that $f(n) \leq n$. 

\begin{corollary}\label{almost-better}
Let $\mathcal P$ be a $f(n)$-Almost Convex Point Set. Then, then there is an algorithm $\mathcal{A}$ for \ESMT such that, $\mathcal{A}$ runs in $2^{\OO{(n + f(n) \log n)}}$ time. In particular, 
\begin{enumerate}
    \item When $f(n) = \OO(n)$, $\mathcal A$ runs in $2^{\OO(n\log n)}$ time.
    \item When $f(n) = \Omega(\frac{n}{\log n})$ and $f(n) = o(n)$, $\mathcal{A}$ runs in $2^{o(n\log n)}$.
    \item When $f(n) = \OO(\frac{n}{\log n})$, $\mathcal A$ runs in $2^{\OO(n)}$ time.
  \end{enumerate}
\end{corollary}

Therefore, for $f(n) = o(n)$, our algorithm for \ESMT does better on $f(n)$-Almost Convex Points Sets than the current best known algorithm~\cite{hwang1986linear}.
    
%    If we consider $k$ to be a constant, the runtime would be $\mathcal O (n ^ k \cdot 5^{n})$. With a more generalized assumption of $k = o(n)$, we get the runtime to be $\mathcal O (n ^ k \cdot 5^{n}) = 2^{(\mathcal O(n) + o(n) \log_2(n))} = 2 ^ {o(n \log n)}$.\\
    
%    \begin{remark}
%        This algorithm when run on the arbitrary positioned points as input, would get number of Full Steiner Topologies to be much higher  causing the runtime to be $2 ^ {\mathcal O(n \log n)}$ instead. 
%    \end{remark} 

\section{Approximation Algorithms for \ESMT}\label{sec:apx_esmt}

The \ESMT problem is NP-hard as shown by Garey et al. in~\cite{garey1977complexity}. Garey et al. also prove that there cannot be an FPTAS (fully polynomial time approximation scheme) for this problem unless $P=NP$. At the same time, the case when all the terminals lie on the boundary of a convex region admits an FPTAS as given in~\cite{scott1988convexity}. We aim to conduct a more fine-grained analysis for the problem by considering $f(n)$-Almost Convex Point Sets of $n$ terminals and studying the existence of FPTASes for different functions $f(n)$. First, we present an FPTAS for \ESMT on $f(n)$-Almost Convex Sets of $n$ terminals, when $f(n) = \OO (\log n)$. Next, we prove that no FPTAS exists for the case when $f(n) =\Omega (n^\epsilon)$, where $\epsilon \in (0,1]$.

\subsection{FPTAS for \ESMT on Cases of Almost Convex Point Sets}\label{subsec:fptas}

We first propose an algorithm for computing the SMT of a planar graph $G$ having $N$ vertices and $n$ terminals, out of which $k$ terminals lie on the outer face of $G$ and the remaining terminals lie within the boundary. Next, following the procedure in~\cite{scott1988convexity} we get an FPTAS for \ESMT on $f(n)$-Almost Convex Sets of $n$ terminals, where $f(n) = \OO (\log n)$.

We state the following proposition from Theorem 1 in~\cite{scott1988convexity}:
\begin{proposition}\label{prop:tree_interval}
    Let $\C{P}$ be the vertices of any polygon in the plane, $\C{K}$ a subset of $\C{P}$, and $\C{T}$ a tree consisting of all the vertices of $\C{K}$ (and possibly some other vertices as well) and contained entirely inside $\C{P}$. Then on removing any edge of the tree, we get two disjoint trees $\mathcal{T}_1, \mathcal{T}_2$, such that the vertices of $\C{K}$ in each tree $\mathcal{T}_i, i \in \{1,2\}$ form an interval in $\C{K}$.
\end{proposition}

Using~\Cref{prop:tree_interval} and the Dreyfus-Wagner algorithm~\cite{dreyfus1971steiner}, we give an algorithm for obtaining the SMT of a planar graph $G$. 

Let $\C{K}$ represent the set of terminals lying on the outer face of $G$ and $\C{R}$ be the set of terminals lying inside the outer face of $G$. We have $|V(G)|=N$, $|\C{K} \cup \C{R}|=n$, and $|\C{K}|=k$. Let $C(\C{L})$ denote the SMT in $G$ for a terminal subset $\C{L} \subseteq V(G)$. Let $B(v,\C{L},[a,b))$ denote the SMT in $G$ for the terminal set $\{v\} \cup \C{L} \cup [a,b)$, where $\C{L} \subseteq \C{R}$, $[a,b)$ is the set of vertices in $\C{K}$ forming an interval from vertex $a$ to $b$ in counterclockwise direction along the outer boundary of $G$ including $a$ but excluding $b$, $v \in V(G) \setminus (\C{L} \cup [a,b))$, and the degree of $v$ is at least $2$ in $B(v,\C{L},[a,b))$. Let $A(v,\C{L},[a,b))$ denote the SMT in $G$ for the terminal set $\{v\} \cup \C{L} \cup [a,b)$, where $\C{L}$, $[a,b)$, and $v$ are as defined in the previous case, and the degree of $v$ is at most $1$ in $A(v,\C{L},[a,b))$.

Splitting the SMT at a vertex $v$ of degree at least $2$ gives rise to two smaller instances of the {\sc Steiner Minimal Tree} problem on graphs.
% \begin{equation}\label{eq:dw_B}
%     B(v,\C{L},[a,b)) = \min_{\C{L}' \subseteq \C{L}, x \in [a,b], \emptyset \subset \C{L}' \cup \{[a,x)\} \subset \C{L} \cup \{[a,b)\}} \{C(\{v\} \cup \C{L}' \cup \{[a,x)\}) + C(\{v\} \cup (\C{L}\setminus \C{L}') \cup \{[x,b)\}) \}
% \end{equation}

\begin{equation}\label{eq:dw_B}
    B(v,\C{L},[a,b)) = \min_{\Pi_1, \Pi_2, \Pi_3} \{C(\{v\} \cup \C{L}' \cup [a,x)) + C(\{v\} \cup (\C{L}\setminus \C{L}') \cup [x,b)) \}
\end{equation} where the conditions on $\C{L}'$ and $x$ are $\Pi_1: \C{L}' \subseteq \C{L}$, $\Pi_2: x \in \mathcal{K}, a< x <b$, and $\Pi_3: \emptyset \subset \C{L}' \cup [a,x) \subset \C{L} \cup [a,b)$.

The intuition is to root the tree at an internal terminal vertex and start growing the Steiner tree from there. Observe that on removing one of the internal vertices $v$ in the tree $\C{T}$, we get one, two or three disjoint subtrees. They induce a partition over the terminals. The terminals in $\C{K}$ in each of the subtrees form intervals in $\mathcal{K}$, according to~\Cref{prop:tree_interval}. Moreover, the terminals in $\C{R}$ can be partitioned in any way, not necessarily maintaining the interval structure. This is captured in the following recurrence relation:
\begin{equation}\label{eq:dw_C}
    C(\{v\} \cup \C{L} \cup [a,b)) = \min_{\Pi_1, \Pi_2, \Pi_3}\{A(v,\C{L}_1,[a,c))+A(v,\C{L}_2,[c,d))+A(v,\C{L}_3,[d,b))\}
\end{equation}

where the conditions on $\C{L}_1$, $\C{L}_2$, $\C{L}_3$, $c$, and $d$ are $\Pi_1: \C{L}_1,\C{L}_2,\C{L}_3 \subseteq \C{L}$, $\Pi_2: \C{L}_1 \cup \C{L}_2 \cup \C{L}_3 = \C{L}$, and $\Pi_3: c,d \in \mathcal{K}, a\leq c \leq d \leq b$ and we have 
\begin{equation}\label{eq:dw_A}
    A(v,\C{L}',[p,q))=\min\{\min_{u \notin \C{L}'}\{B(u,\C{L}',[p,q))+d(u,v)\},\min_{u \in \C{L}' \cup [p,q)}\{C(\C{L}' \cup [p,q))+d(u,v)\}\}
\end{equation}

Our aim is to compute $C(\{v\} \cup (\C{R}\setminus \{v\}) \cup \C{K})$, where $v \in \C{R}$. We precompute the shortest distance between all pairs of vertices. We then compute the values of $C(.)$ and $B(.)$ in increasing order of cardinality of subsets of vertices in $\C{K}$ and $\C{R}$. Let $d(u,v)$ denote the shortest path length between $u$ and $v$. The base cases are $C(\{v\} \cup \{a\}) = d(v,a)$ for all $v \in V(G)$ and $a \in \C{K} \cup \C{R}$.

\begin{algorithm}[H]
\caption{Computation of SMT of planar graph $G$ with terminal set $\C{K} \cup \C{R}$ ~~~ \textbf{Input:} $G$, $\C{K}$, $\C{R}$}\label{alg:smt_planar}
\begin{algorithmic}[1]
\State Compute the shortest distance between all pairs of vertices
\For{all $u \in V(G)$ and $a \in K \cup R$} 
\State Set $C(\{u\} \cup \{a\}) = d(u,a)$
\EndFor
\State Select a vertex $v \in \C{R}$
\For{$i = 1, \ldots, n-k-1$}
\For{each $\C{L} \subseteq \C{R}\setminus \{v\}$ of size $i$}
\For{$j = 1, \ldots, k$}
\If{j=k}
\State Compute $B(v,\C{L},\C{K})$ using~\Cref{eq:dw_B}
\State Compute $C(\{v\} \cup \C{L} \cup \C{K})$ using~\Cref{eq:dw_C}
\Else
\For{each $[a,b) \subseteq \C{K}$ of size $j$}
\State Compute $B(v,\C{L},[a,b))$ using~\Cref{eq:dw_B}
\State Compute $C(\{v\} \cup \C{L} \cup [a,b))$ using~\Cref{eq:dw_C}
\EndFor
\EndIf
\EndFor
\EndFor
\EndFor
\State \Return $C(\{v\} \cup (\C{R}\setminus \{v\}) \cup \C{K})$
\end{algorithmic}
\end{algorithm}

We analyse the correctness and running time of~\Cref{alg:smt_planar}.
    \begin{theorem} \label{thm:algo1_correct}
     Consider a planar graph $G$ on $N$ vertices and a set $\mathcal{K} \uplus \mathcal{R} \subseteq V(G)$ of $n$ terminals such that $\mathcal{K}$ is defined as the terminals lying on the outer face of $G$. Moreover, let $|\mathcal{K}| = k$. Then~\Cref{alg:smt_planar} computes the SMT for $\mathcal{K} \uplus \mathcal{R}$ in $G$ in time $\OO(N^2k^44^{n-k} + Nk^33^{n-k}+N^3)$.
    \end{theorem}
    \begin{proof}
{\bf Correctness of~\Cref{alg:smt_planar}.} 
In order to prove the correctness of~\Cref{alg:smt_planar}, we need to show that the~\Cref{eq:dw_B,eq:dw_C} are valid.

        In~\Cref{eq:dw_B}, $B(v,\C{L},[a,b))$ denotes an SMT for the terminal set $\{v\} \cup \C{L} \cup [a,b)$, conditioned on the fact that the degree of $v$ is at least $2$ in it. Let us split the SMT at vertex $v$ into two smaller subtrees. This must also split the terminals in $[a,b)$ in two intervals $[a,x)$ and $[x,b)$, respectively. Otherwise it would mean that the SMT has crossing edges, which is not possible. The vertices in $\mathcal{L}$ can be present in any of the two subtrees, hence we consider all possible partitions of $\C{L}$ into two subsets $\C{L}'$ and $\C{L} \setminus \C{L}'$. Thus for~\Cref{eq:dw_B}, $\mbox{LHS} \geq \mbox{RHS}$. On the other hand, the expression in the RHS of~\Cref{eq:dw_B} is a tree containing the vertex subset $\{v\} \cup \mathcal{L} \cup [a,b)$. Since, $B(v,\mathcal{L},[a,b))$ is an SMT for the same vertex subset, in~\Cref{eq:dw_B}  $\mbox{LHS} \geq \mbox{RHS}$. Therefore,~\Cref{eq:dw_B} is valid.

        For~\Cref{eq:dw_C}, we take $v$ as the root of the SMT $C(\{v\} \cup \C{L} \cup [a,b))$. The degree of $v$ in the SMT can be $1$, $2$, or $3$. Accordingly, on removing $v$, we will get $1$, $2$, or $3$ subtrees, with the condition that in each subtree the degree of $v$ is $1$. Again, the terminals in $[a,b)$ are divided into smaller intervals $[a,c)$, $[c,d)$ and $[d,b)$ for some $c,d \in \mathcal{K}$ satisfying $a<c<d<b$. The terminals in $\C{L}$ are divided among the subtrees in any combination. The number of such intervals and partitions is equal to the degree of $v$ in $C(\{v\} \cup \C{L} \cup [a,b))$. The term $A(v,\C{L}',[p,q))$ is obtained by minimizing across all SMTs satisfying the condition that degree of $v$ in the SMT is $1$. Thus in~\Cref{eq:dw_C}, $\mbox{LHS} \geq \mbox{RHS}$. On the other hand, the RHS of~\Cref{eq:dw_C} given a tree containing vertices $\{v\} \cup \mathcal{L} \cup [a,b)$. Since $C(\{v\} \cup \mathcal{L} \cup [a,b))$ is an SMT on the terminal set $\{v\} \cup \mathcal{L} \cup [a,b)$, in~\Cref{eq:dw_C} $\mbox{LHS} \leq \mbox{RHS}$. Thus,~\Cref{eq:dw_C} is valid.

{\bf Running time of~\Cref{alg:smt_planar}.}
All pairs shortest paths can be calculated in $\OO(N^3)$ time. The time complexity of the dynamic program has two components to it. One is due to computation of the B(.) values using~\Cref{eq:dw_B} and the other is for calculating the C(.) values using~\Cref{eq:dw_C}.
    \begin{enumerate}
        \item The number of computational steps for calculating $B(v,\mathcal{L},[a,b))$ using~\Cref{eq:dw_B} is of the order of the number of choices of $v$, $\mathcal{L}$, $\mathcal{L}'$, $[a,b)$, and $[a,x)$ such that $\mathcal{L} \subset \mathcal{R}$, $\mathcal{L}' \subseteq \mathcal{L}$, $a,x,b \in \mathcal{K}$, $a \leq x \leq b$, and $v \in V(G) \setminus (L \cup [a,b))$. Each vertex in $\mathcal{R}$ belongs to exactly one of the sets $\mathcal{L}'$, $\mathcal{L}\setminus \mathcal{L}'$, or $V(G) \setminus \mathcal{L}$. The vertices in $\mathcal{K}$ are partitioned into three intervals, $[a,x)$, $[x,b)$, and $[b,a)$. There are at most $N$ possibilities for $v$. This gives us a running time of $\OO(Nk^3 3^{n-k})$.
        \item The number of computational steps for calculating $C(\{v\} \cup \mathcal{L} \cup [a,b))$ using~\Cref{eq:dw_C} is $3N$ times the order of the number of choices of $v$, $\mathcal{L}$, $\mathcal{L}_1$, $\mathcal{L}_2$, $a$, $b$, $c$, and $d$ such that $\mathcal{L} \subset \mathcal{R}$, $\mathcal{L}_1 \subseteq \mathcal{L}$, $\mathcal{L}_2 \subseteq \mathcal{L}$, $\{a,c,d,b\} \subseteq \mathcal{K}$, $a \leq c \leq d \leq b$, and $v \in V(G) \setminus (\mathcal{L} \cup [a,b))$. Each vertex in $\mathcal{R}$ belongs to exactly one of the sets $\mathcal{L}_1$, $\mathcal{L}_2$, $\mathcal{L}_3$ or $V(G) \setminus \mathcal{L}$. The vertices in $\mathcal{K}$ are partitioned into at most four intervals, $[a,c)$, $[c,d)$, $[d,b)$ and $[b,a)$. There are at most $N$ possibilities for $v$. The $3N$ factor is because calculating each of the $C(.)$ values involves minimization over at most $3N$ terms. This gives us a running time of $\OO(N^2k^4 4^{n-k})$.
    \end{enumerate}
    Thus, the time complexity of the algorithm is $\OO(N^2k^44^{n-k}+Nk^33^{n-k}+N^3)$.
\end{proof}
We obtain the following corollary from the above theorem.

\begin{corollary}\label{graph-correct}
 Consider a planar graph $G$ on $N$ vertices and a set $\mathcal{K} \uplus \mathcal{R} \subseteq V(G)$ of $n$ terminals such that $\mathcal{K}$ is defined by the terminals lying on the outer face of $G$. Moreover, let $|\mathcal{K}| = k$ and let $|\C{R}| = n-k = \OO(\log n)$. Then~\Cref{alg:smt_planar} computes the SMT for $\mathcal{K} \uplus \mathcal{R}$ in $G$ in time $N^3k^4n^{\OO(1)}$.
\end{corollary}

Next we state the FPTAS for \ESMT on $f(n)$-Almost Convex Sets of $n$  terminals. This is achieved by converting the instance of \ESMT into an instance of {\sc Steiner Minimal Tree} on graphs. The {\sc Steiner Minimal Tree} problem shall be solved using~\Cref{alg:smt_planar}. For this, we use the Algorithm 2 in~\cite{scott1988convexity}. We restate the Algorithm 2 in~\cite{scott1988convexity} for our problem instance. We denote the set of terminals with $\C{P}$.

\begin{algorithm}[H]
\caption{Computation of $(1+\epsilon)$-approximate SMT of $\C{P}$ ~~~ \textbf{Input:} $\mathcal P, \epsilon$ }\label{alg:smt_fptas}
\begin{algorithmic}[1]
\State Compute the convex hull of the set of terminals $\C{P}$. Let the region enclosed by the convex hull $\mathrm{CH}(\C{P})$ be denoted by $\mathbb{R}_{\mathrm{CH}(\C{P})}$. Let the points in $\C{P}$ lying on $\mathrm{CH}(\C{P})$ be $\C{K}$ and $\C{R} = \C{P} \setminus \C{K}$.
\State We enclose the set of terminals $\C{P}$ with the smallest axis-parallel bounding square. Let its side length be $D$. We divide the bounding square into same sized grids of side length $\frac{D\epsilon}{8n-12}$, where $\epsilon$ is the approximation factor.
\State Let $\C{V}_0$ be the set of all lattice points introduced in the previous step, and $\C{V}_1$ be the set of all lattice points lying on the edges of $\mathrm{CH}(\C{P})$. We define the weighted graph $G_{f,\epsilon}$ to be the complete graph with vertex set $V(G_{f,\epsilon}) = \C{K} \cup \C{R} \cup (\C{V}_0 \cap \mathbb{R}_{\mathrm{CH}(\C{P})}) \cup \C{V}_1$. The edge weights are equal to the Euclidean distance between the two end points.
\State Return the SMT $\C{T}$ for the graph $G_{f,\epsilon}$ with $\C{K} \cup \C{R}$ as the terminal set using~\Cref{alg:smt_planar}. 
\end{algorithmic}
\end{algorithm}

We analyse the correctness and running time of~\Cref{alg:smt_fptas}.
    \begin{theorem} \label{thm:algo2_correct}
     Consider a set $\C{P}$ of $n$ points such that $\mathcal{K}$ is defined as the points lying on the convex hull of $\C{P}$, i.e. $\mathrm{CH}(\C{P})$, and $\C{R} = \C{P} \setminus \C{K}$. Moreover, let $|\mathcal{K}| = k$. Then~\Cref{alg:smt_fptas} computes a $(1+\epsilon)$-approximate SMT for $\mathcal{P}$ in time $\OO(\frac{n^4k^4}{\epsilon ^4}4^{n-k})$.
    \end{theorem}
    \begin{proof}
{\bf Correctness of~\Cref{alg:smt_fptas}.} 
    In order to prove the correctness of~\Cref{alg:smt_fptas}, we need to show that $\C{T}$ is a $(1+\epsilon)$-approximation of the SMT of the terminal set $\C{P}$, and $\C{T}$ is indeed the SMT for $\C{K} \cup \C{R}$ in the complete weighted graph $G_{f,\epsilon}$.

        In~\cite{scott1988convexity}, the concept of weight planar graphs is used. A graph $G$ is called weight planar if it is a non-planar graph embedded on the Euclidean plane, having non-negative edge weights, such that every pair of edges $(u,v)$ and $(u',v')$ which intersect in this embedding of $G$, satisfy the inequality: $w(u,v) + w(u',v') > d(u,u') + d(v,v')$, where $w(u,v)$ is the weight of the edge between vertices $u$ and $v$ and $d(x,y)$ is the length of the shortest path between vertices $x$ and $y$. Since the edge weights of $G_{f,\epsilon}$ are the Euclidean distances between the points, $G_{f,\epsilon}$ is a weight planar graph. 

        From Theorem 5 of \cite{scott1988convexity}, we get that the SMT of a weight planar graph does not contain any crossing edges even though the input graph is non-planar. Because the SMT does not contain any crossing edges and lies completely inside the convex hull of the terminal pointset, the terminals on the outer boundary of $G_{f,\epsilon}$, i.e. $\C{K}$, follow the interval pattern as stated in~\Cref{prop:tree_interval}. Therefore,~\Cref{alg:smt_planar} designed for planar graphs can be applied in the case of weight planar graphs as well. So, $\C{T}$ is the SMT for $\C{K} \cup \C{R}$ in the complete weighted graph $G_{f,\epsilon}$.

        Finally, from Theorem 12 in~\cite{scott1988convexity}, we get that the length of the Steiner tree obtained from~\Cref{alg:smt_fptas} is at most $(1+\epsilon)$ times the length of the SMT $\C{T}^*$ of $\C{P}$, i.e. $|\C{T}| \leq (1+\epsilon)|\C{T}^*|$. Thus, we are done.
        
{\bf Running time of~\Cref{alg:smt_fptas}.}
    Constructing the convex hull takes $\OO(n\log n)$ time. The number of lattice points contained in the bounding box is $\big(\frac{8n-12}{\epsilon}\big)^2 = \OO(\frac{n^2}{\epsilon ^2})$. Thus, the number of vertices in the resultant graph $G_{f,\epsilon}$ is $N = \OO(\frac{n^2}{\epsilon ^2}) + n = \OO(\frac{n^2}{\epsilon ^2})$. The time complexity of~\Cref{alg:smt_planar} is $\OO(N^2k^44^{n-k}+Nk^33^{n-k}+N^3) = \OO(\frac{n^4k^4}{\epsilon ^4}4^{n-k})$. This step dominates the running time resulting in the complexity of~\Cref{alg:smt_fptas} being $\OO(\frac{n^4k^4}{\epsilon ^4}4^{n-k})$.
    %If $n-k=\OO(\log n)$, the running time becomes polynomial in $n$ and $k$. $k$ can be bounded by $n$, hence the time complexity becomes polynomial in $n$.
    \end{proof}

\begin{theorem}\label{thm:esmt_fptas}
    There exists an FPTAS for \ESMT on an $f(n)$-Almost Convex Set of $n$ terminals, where $f(n) = \OO (\log n)$.
\end{theorem}

\begin{proof}
    From~\Cref{thm:algo2_correct}, we get a $(1+\epsilon)$-approximate SMT for any $(n-k)$-Almost Convex Set of $n$ terminals in time $\OO(\frac{n^4k^4}{\epsilon ^4}4^{n-k})$. For $n-k = \OO(\log n)$, we get the running time of~\Cref{alg:smt_fptas} to be $\OO(\frac{n^4k^4}{\epsilon ^4}n^{\OO(1)})$. Thus,~\Cref{alg:smt_fptas} is an FPTAS for the Euclidean Steiner Minimal Tree problem on an $\OO (\log n)$-Almost Convex Set of $n$ terminals. 
\end{proof}

\subsection{Hardness of Approximation for \ESMT on Cases of Almost Convex Sets}\label{subsec:apx_hardness}

In this section, we consider the \ESMT problem on $f(n)$-Almost Convex Sets of $n$ terminal points, where $f(n) =\Omega(n^\epsilon)$ for some $\epsilon \in (0,1]$. We show that this problem cannot have an FPTAS. The proof strategy is similar to that in~\cite{garey1977complexity}. First, we give a reduction for the problem \textsc{Exact Cover by $3$-Sets} (defined below) to our problem to show that our problem is NP-hard. Next, we consider a discrete version of our problem and reduce our problem to the discrete version. The discrete version is in NP. Therefore, this chain of reductions imply that the discrete version of our problem is Strongly NP-complete and therefore cannot have an FPTAS, following from~\cite{garey1977complexity}. Similar to the arguments in~\cite{garey1977complexity}, this also implies that our problem cannot have an FPTAS.

Before we describe our reductions, we take a look at the NP-hardness reduction of the \ESMT problem from the \textsc{Exact Cover by 3-Sets} (X3C) problem in~\cite{garey1977complexity}. In the X3C problem, we are given a universe of elements $U = \{1, 2, \ldots, 3n\}$ and a family $\mathbb{F}$ of $3$-element subsets $F_1, F_2, \ldots, F_t$ of the $3n$ elements. The objective is to decide if there exists a subcollection $\mathbb{F}' \subseteq \mathbb{F}$ such that: (i) the elements of $\mathbb{F}'$ are disjoint, and (ii) $\bigcup_{F' \in \mathbb{F}'} F' = U$. The X3C problem is NP-complete~\cite{garey1979computers}.

In~\cite{garey1977complexity}, various gadgets are constructed, i.e.~particular arrangements of a set of points. These are then arranged on the plane in a way corresponding to the given X3C instance.~\Cref{fig:redcn} shows the reduced ESMT instance obtained for $U = \{1,2,3,4,5,6\}$ and $\mathbb{F} = \{\{1, 2, 4\}, \{2, 3, 6\}, \{3, 5, 6\}\}$ (taken from~\cite{garey1977complexity}). The squares, hexagons (crossovers), shaded circles (terminators) and lines (rows) all represent specific arrangements of a subset of points. Let $X(\mathbb{F})$ denote the reduced instance. The number of points in $X(\mathbb{F})$ is bounded by a polynomial in $n$ and $t$. Let this polynomial be $\OO(t^\gamma)$, as we can assume $t \geq n$ since otherwise it trivially becomes a NO instance. Here $\gamma$ is some constant.

\begin{figure}[h]
\centering
\subfloat{\includegraphics[width=12cm]{reduction_X3C_ESMT.png}}
\caption{Reduced instance of ESMT from X3C (taken from~\cite{garey1977complexity})}
\label{fig:redcn}
\end{figure}

We restate Theorem 1 in~\cite{garey1977complexity}.
\begin{proposition}\label{thm:redcn}
    Let $\mathcal{S}^{*}$ denote an SMT of $X(\mathbb{F})$, the instance obtained by reducing the X3C instance $(n,\mathbb{F})$, and $|\mathcal{S}^{*}|$ denote its length. If $\mathbb{F}$ has an exact cover, then $|\mathcal{S}^{*}| \leq f(n,t,\hat{C})$, otherwise $|\mathcal{S}^{*}| \geq f(n,t,\hat{C}) + \frac{1}{200nt}$, where $t = |\mathbb{F}|$, $\hat{C}$ is the number of crossovers, i.e.~hexagonal gadgets, and $f$ is a positive real-valued function of $n,t,\hat{C}$.
\end{proposition}

We extend this construction to prove NP-hardness for instances of \ESMT where the terminal set $\mathcal{P}$ has $\Omega(n^\epsilon)$ points inside $\mathrm{CH}(\mathcal{P})$. Here, $\epsilon \in (0,1]$ and $n$ is the number of terminals.

Let us call the \emph{length} of a gadget to be the maximum horizontal distance between any two points in that gadget. Similarly, we define the \emph{breadth} of a gadget to be the maximum vertical distance between any two points in that gadget.

% We get the following lemma from the construction given in~\cite{garey1977complexity}.
% \begin{lemma}\label{lem:bbox}
%     The smallest rectangle bounding all the points in $X(\mathbb{F})$ has side lengths of $\OO(t)$.
% \end{lemma}

% \begin{proof}
%     From the construction of the gadgets in~\cite{garey1977complexity}, we know that all of them have length and breadth as a function of $\epsilon = \frac{1}{200nt}$, which is bounded between 0 and 1. Thus, the length and breadth of each gadget can be bounded by some constant.

%     According to the reduction, the maximum number of gadgets along any horizontal level is $\OO(t)$. Thus, the length of the smallest bounding rectangle is $c_1t = \OO(t)$. Moreover, the maximum number of gadgets along any vertical column is $\OO(n) = \OO(t)$ as $t \geq n$. Thus, the breadth of the smallest bounding rectangle is $c_2t = \OO(t)$.
% \end{proof}

\begin{figure}[h] 
\centering
\subfloat[\centering The Upward Terminator symbol]{\includegraphics[width=1.65cm]{terminator_symbol.png}}   \qquad\qquad
\subfloat[\centering The Downward Terminator symbol]{\includegraphics[width=1.5cm]{terminator_down.png}}
    \qquad\qquad
\subfloat[\centering The Terminator gadget] {\includegraphics[width=5cm]{terminator_gadget.png} }
\caption{The Terminator gadget symbol and arrangement of points}  
\label{fig:terminator}
\end{figure}

The \emph{terminator} gadget used is shown in~\Cref{fig:terminator}. The straight lines represent a row of at least $1000$ points separated at distances of $1/10$ or $1/11$. The angles between them are as shown. The upward terminator has the point $A$ above the other points in the terminator, whereas the downward terminator has the point $A$ below the other points. Firstly, we adjust the number of points in the long rows, such that the length and breadth of the terminators is same as that of the hexagonal gadgets (crossovers). We can fix this length and breadth to be some constants, such that the number of points in each gadget is also bounded by some constant. In our construction, we modify the terminators $\Omega_0$, $\Omega_1$, and $\Omega_2$ as shown in~\Cref{fig:redcn} enclosed in squares. $\Omega_1$ is the terminator corresponding to the first occurrence of the element $3n\in U$ in some set in $\mathbb{F}$ and $\Omega_2$ is the terminator corresponding to the last occurrence of $3n$ in some set in $\mathbb{F}$ (if there are more than one occurrences of $3n$). If there are no occurrences of $3n$, then it is trivially a no-instance. The modified gadgets are shown in~\Cref{fig:terminator_new}. All the other gadgets remain unaltered.

\begin{figure}[h]
\centering
\includegraphics[width=8cm]{conic_set.png}
\caption{Conic Set: $\mathrm{Cone}(T,r,n)$}
\label{fig:conic_set}
\end{figure}

We call a set of points arranged as shown in~\Cref{fig:conic_set}, as a Conic Set.
\begin{definition} \label{def:conic_set}
    A Conic Set is a set of points consisting of a point $T$, called the tip of the cone, and the remaining points denoted by $\mathcal{S}$. Let $\mathcal{C}$ be the circle with $T$ as centre and radius $r$. All the points in $\mathcal{S}$ lie on $\mathcal{C}$, such that the angle at the tip formed by the two extreme points $L,R \in \mathcal{S}$, i.e.~$\angle{LTR} = 30^\circ$ in the anticlockwise direction. So, we have $\overline{TL} = \overline{TR} = r$. The distance between any two consecutive points in $\mathcal{S}$ is the same, say $d$. Let the number of points in $\mathcal{S}$ be $n$. We denote the Conic Set as $\mathrm{Cone}(T,r,n)$ and $\mathcal{S}$ as $\mathrm{Circ}(T,r,n)$. We call $TL$ as the left slope of the Conic Set and $TR$ as the right slope of the Conic Set.
\end{definition}

\begin{figure}[h] 
\centering
\subfloat[\centering $\Omega'_0$]{\includegraphics[width=5cm]{terminator_up_new.png}}   \qquad\qquad\qquad\qquad\qquad\qquad
\subfloat[\centering $\Omega'_1$]{\includegraphics[width=5cm]{omega_1.png}}
    \qquad\qquad
\subfloat[\centering $\Omega'_2$] {\includegraphics[width=5cm]{omega_2.png} }
\caption{The modified terminator gadgets}  
\label{fig:terminator_new}
\end{figure}


We use the Conic Set in the reduction for our problem. Now, we state the reduction of an X3C instance $(n,\mathbb{F})$ to an instance $X'(\mathbb{F},\epsilon)$ of \ESMT. Later, we show that the instance will satisfy the desired properties on the number of terminals inside the convex hull of the terminal set.

\begin{description}
\item [Algorithm $\mathcal{A}$ for construction of an ESMT instance $X'(\mathbb{F},\epsilon)$ from an X3C instance $(n,\mathbb{F})$:]
\hspace{0.1cm}
\begin{itemize}
    \item Reduce the input X3C instance to the points configuration $X(\mathbb{F})$ according to the reduction given in~\cite{garey1977complexity}.
    \item Modify the terminators $\Omega_0$, $\Omega_1$, and $\Omega_2$ to as shown in~\Cref{fig:terminator_new} and call them $\Omega'_0$, $\Omega'_1$, and $\Omega'_2$. Let $DQCP$ be the smallest axis-parallel rectangle bounding $X(\mathbb{F})$ after modifying the terminators, where $D$ is the bottom leftmost point of $\Omega'_0$.
    \item Take $\alpha = \frac{1}{\epsilon}$. Define $r = ct^{\alpha} = \OO(t^{\alpha})$ and $n' = c't^{\gamma\alpha} = \OO(t^{\gamma\alpha})$, where $t=|\mathbb{F}|$ and $c$ and $c'$ are constants. Add the $\mathrm{Cone}(D,r,n')$, such that $D$ is the tip of the Conic Set, and the left slope $DE$ makes an angle of $120^\circ$ with $DP$. The right slope $DF$ also makes an angle of $120^\circ$ with $DQ$.
    % \item Add the point $E$ on $\mathcal{C}$, such that $\angle{PDE} = 120^\circ$. Add the point $F$ such that $F$ lies on $\mathcal{C}$ and $\angle{EDF} = 30^\circ$ in anticlockwise direction.
    % \item Add $c't^{\gamma\alpha} = \OO(t^{\gamma\alpha})$ points on $\mathcal{C}$, where $c'$ is a constant, such that the distance between any two consecutive points is the same. Let us call these set of points on $\mathcal{C}$ along with $E$ and $F$ as $S$.
\end{itemize}
\end{description}

\begin{figure}[h]
\centering
\includegraphics[width=10cm]{redn_instance.png}
\caption{The reduced instance $X'(\mathbb{F},\epsilon)$}
\label{fig:redn_instance}
\end{figure}

% \begin{description}
% \item [Construction of $X'(\mathbb{F},\epsilon)$:]
% \hspace{10cm}
% \begin{itemize}
%     \item Reduce the input X3C instance to the point configuration $X(\mathbb{F})$ according to the reduction given in~\cite{garey1977complexity}.
%     \item Modify the terminator $\Omega_0$ to as shown in~\Cref{fig:terminator_new} and call it $\Omega'_0$.
%     \item Add a point $e$ at a distance of $r - D$ from the point $d$ of $\Omega'_0$ in the direction of the ray $\overrightarrow{\rm od}$. Add points on the line segment $\overline{\rm de}$ such that consecutive points are at a distance of $\frac{1}{10}$ from each other.
%     \item Take $\alpha = \frac{1}{\epsilon}$. From $e$, take two points $f$ and $g$, such that $\angle{\rm feg} = 120\degree$, $\overrightarrow{\rm oe}$ bisects $\angle{\rm feg}$ and $\overline{\rm fe} = \overline{\rm ge} = c't^{1-2\alpha}$. Consider the point $C$ at a distance of $D$ from the point $d$ of $\Omega'_0$ in the direction of the ray $\overrightarrow{\rm do}$. Let $R = \overline{\rm Cf} = \OO(t + t^{1-2\alpha}) = \OO(t)$. Taking $C$ as the centre, draw a circle of radius $R$. Let the circle be $\mathcal{T}$.
%     \item Add vertices on $\mathcal{T}$ such that the distance between two consecutive points is same as $\overline{\rm fg}$.
% \end{itemize}
% \end{description}

Now we prove a few properties of the constructed instance $X'(\mathbb{F},\epsilon)$.
\begin{lemma}\label{lem:convhull_pts}
    All the points in $\mathrm{Circ}(D,r,n')$ (according to~\Cref{def:conic_set}) lie on the convex hull of the reduced ESMT instance $X'(\mathbb{F},\epsilon)$ constructed by Algorithm $\mathcal{A}$, where $\epsilon \in (0,1]$.
\end{lemma}

\begin{proof}
    By the construction of $\mathrm{Cone}(D,r,n')$ in Algorithm $\mathcal{A}$, let $\mathcal{C}$ be the circle on which all the points in $\mathrm{Circ}(D,r,n')$ lie. If we draw a tangent to $\mathcal{C}$ at any of the points in $\mathrm{Circ}(D,r,n')$, then all the remaining points in the configuration $X'(\mathbb{F},\epsilon)$ lie towards one side of the tangent. We know that if we can find a line passing through a point such that all the other points in the plane lie on one side of the line, then the point lies on the convex hull of the points in the plane. Therefore, all the points in $\mathrm{Circ}(D,r,n')$ lie on the convex hull of the reduced instance $X'(\mathbb{F},\epsilon)$.
\end{proof}

Let us denote the convex hull of $X'(\mathbb{F},\epsilon)$ by $\mathrm{CH}(X'(\mathbb{F},\epsilon))$ and that of the points lying inside or on the bounding rectangle $\mathrm{PDQC}$, i.e.~$X'(\mathbb{F},\epsilon)\setminus \mathrm{Circ}(D,r,n')$ by $\mathrm{CH}(X'(\mathbb{F},\epsilon)\setminus \mathrm{Circ}(D,r,n'))$.

\begin{lemma}\label{lem:redcn_size}
    The reduced ESMT instance $X'(\mathbb{F},\epsilon)$ constructed by Algorithm $\mathcal{A}$ has $\Omega (N^\epsilon)$ points inside the convex hull, where $\epsilon \in (0,1]$ and $N$ is the total number of terminals in $X'(\mathbb{F},\epsilon)$. 
\end{lemma}

\begin{proof}
    $\mathrm{CH}(X'(\mathbb{F},\epsilon))$ contains all the points in $\mathrm{Circ}(D,r,n')$ by~\Cref{lem:convhull_pts}. $\mathrm{Circ}(D,r,n')$ contains $n' = \OO(t^{\gamma\alpha})$ points.

    Now we need to analyze the number of points on $\mathrm{CH}(X'(\mathbb{F},\epsilon)\setminus \mathrm{Circ}(D,r,n'))$. The remaining points in $X(\mathbb{F})$, i.e.~$X(\mathbb{F})\setminus \mathrm{CH}(X'(\mathbb{F},\epsilon)\setminus \mathrm{Circ}(D,r,n'))$ must lie within the convex hull of the entire construction, i.e.~$X'(\mathbb{F},\epsilon)$. From the construction in Algorithm $\mathcal{A}$, no point on the connecting rows can be a part of $\mathrm{CH}(X'(\mathbb{F},\epsilon)\setminus \mathrm{Circ}(D,r,n'))$ as there is no line passing through it, which contains all terminals on one side of it. The same thing holds for the square and hexagonal gadgets as well, except the hexagonal gadgets corresponding to the last element of the last set in the family, i.e.~$F_t$. Thus, only the terminators and the hexagonal gadgets corresponding to the last element of $F_t$ contribute points to $\mathrm{CH}(X'(\mathbb{F},\epsilon)\setminus \mathrm{Circ}(D,r,n'))$.

    If we look at the arrangement of points in the terminators (modified as well as those left unchanged) and the hexagonal gadgets as shown in~\Cref{fig:terminator,fig:hexagon_points}, the convex hull of each of these gadgets consists of constantly many points. Therefore, the number of points each of these gadgets contribute to $\mathrm{CH}(X'(\mathbb{F},\epsilon)\setminus \mathrm{Circ}(D,r,n'))$ is bounded by some constant. The number of terminators is $6t+2$ and the number of hexagonal gadgets corresponding to the last element of $F_t$ is at most $3n$. Therefore, the number of points on $\mathrm{CH}(X'(\mathbb{F},\epsilon)\setminus \mathrm{Circ}(D,r,n'))$ is $\OO(t+n) = \OO(t)$ as $n \leq t$.
    
    The instance $X(\mathbb{F})$ obtained via reduction from X3C has $6t+2$ terminators, $t$ squares, at most $9nt$ crossovers (hexagonal gadgets), and $\OO(nt)$ connecting rows of points. The number of gadgets is $\OO(nt)$. Therefore, the total number of points in $X(\mathbb{F})$ is $\Omega(nt) = \omega(t)$. The modified terminators result in a constantly many increase in the number of points. So, we have $\gamma > 1$.

    Thus, the number of points inside the convex hull is $\Omega(t^\gamma)$ and those on the convex hull is $\OO(t^{\gamma\alpha})$. So, the total number of terminals is $N = \OO(t^{\gamma\alpha}) + \OO(t^{\gamma}) = \OO(t^{\gamma\alpha})$, and those inside the convex hull is $\Omega(t^\gamma) = \Omega(N^{1/\alpha}) = \Omega(N^\epsilon)$ as $\alpha = \frac{1}{\epsilon}$.
\end{proof}

\begin{figure}[h]
\centering
\includegraphics[width=5cm]{hexagon_points.png}
\caption{The hexagonal gadget (crossover), the convex hull of the gadget is the quadrilateral $\rm abcd$ (taken from~\cite{garey1977complexity})}
\label{fig:hexagon_points}
\end{figure}

% \begin{lemma}\label{lem:redcn_size}
%     The reduced instance $X'(\mathbb{F},\epsilon)$ has $\OO (N^\epsilon)$ points inside the convex hull, where $\epsilon \in (0,1]$ and $N$ is the total number of terminals. 
% \end{lemma}

% \begin{proof}
%     The convex hull of $X'(\mathbb{F},\epsilon)$ configuration of points is the convex polygon inscribed in the circle $\mathcal{T}$ of radius $R = \OO(t)$. Distance between consecutive points is $\OO(t^{1-2\alpha})$. So, the convex hull contains $\OO(t^{2\alpha})$ points.
    
%     The instance $X(\mathbb{F})$ obtained via reduction from X3C has $6t+2$ terminators, $t$ squares, at most $9nt$ crossovers (hexagonal gadgets), and $\OO(nt)$ connecting rows of points. Each of these gadgets contain constantly many points. Therefore, the total number of points in $X(\mathbb{F})$ is $\OO(nt) = \OO(t^2)$. The modified terminator results in a constantly many increase in the number of points. Addition of points on the line segment $\overline{\rm de}$ results in $\OO(t)$ increase in the number of points inside as length of $\overline{\rm de} = D = \OO(t)$ and every pair of consecutive points are at a distance of $\frac{1}{10}$ from each other.

%     Thus, the number of points inside the convex hull is $\OO(t^2)$ and those on the convex hull is $\OO(t^{2\alpha})$. So, the total number of terminals is $N = \OO(t^{2\alpha})$, and those inside the convex hull is $\OO(t^2) = \OO(N^{1/\alpha}) = \OO(N^\epsilon)$ as $\alpha = \frac{1}{\epsilon}$.
% \end{proof}
We further prove structural properties of SMTs of the reduced instance $X'(\mathbb{F},\epsilon)$ when considering the modified gadgets $\Omega'_0$, $\Omega'_1$, and $\Omega'_2$.
\begin{lemma}\label{lem:modified_smt}
    Consider an SMT $\mathcal{S}^*$ of the ESMT instance $X(\mathbb{F})$ obtained via reduction from the X3C instance $(n,\mathbb{F})$ as per~\cite{garey1977complexity}. Consider a tree $\mathcal{S'}^{*}$ on the terminal set of $X'(\mathbb{F},\epsilon)$ obtained from $\mathcal{S}^*$ as follows: Consider the modified terminator gadgets $\Omega'_i,~i \in \{0,1,2\}$ as in Algorithm $\mathcal{A}$. For each $i \in \{0,1,2\}$, the edge $B_iO_i$ is excluded from $\mathcal{S}^*$ and the edge $D_iO_i$ is included to form $\mathcal{S'}^{*}$. $\mathcal{S'}^{*}$ is an SMT for the terminal set of $X'(\mathbb{F},\epsilon)$.
\end{lemma}

\begin{proof}
    Consider $\mathcal{S}^*$. Due to Lemma 4 of~\cite{garey1977complexity}, Steiner points of $\mathcal{S}^*$ can only be connected to points in the triangular and square gadgets. Lemma 5 of~\cite{garey1977complexity} states that if there are two terminals $x,y \in X(\mathbb{F})$ and the distance between $x$ and $y$ does not exceed $\frac{1}{10}$, then $(x,y)$ is an edge of $\mathcal{S}^{*}$. So in $\mathcal{S}^{*}$, for each $i \in \{0,1,2\}$ all the points on $B_iO_i$ are joined together along $B_iO_i$. Lemma 5 of~\cite{garey1977complexity} also holds true on modifying the terminators to $\Omega'_i,~i \in \{0,1,2\}$. Now, we join all the points on $D_iO_i$ along $D_iO_i$. This gives us the SMT $\mathcal{S'}^*$ for the terminal set of $X'(\mathbb{F},\epsilon)$.
\end{proof}

%A similar extension works for the modification of the terminators $\Omega_1$ and $\Omega_2$. Let this new SMT be represented by ${\mathcal{S}'}^*$ and its length by $|{\mathcal{S}'}^*|$.

Now we focus on the structure of the SMT of $X'(\C{F},\epsilon)$. The SMT is basically the union of the SMT $\C{S'}^{*}$ of the points in the bounding rectangle $PDQC$ as stated in~\Cref{lem:modified_smt} and the SMT of the set of points $\mathrm{Cone}(D,r,n')$.

% \begin{lemma}\label{lem:bounding_quad}
%     The rectangle $\mathrm{pdqC}$ as shown in~\Cref{fig:bounding_quad} encloses all the points in $X'(\C{F},\epsilon)$ located inside the convex hull.
% \end{lemma}

% \begin{proof}
%     We know that $D = \max\big(\frac{2c_1t}{\sqrt{3}}, 2c_2t\big)$. So, $\overline{\rm dq} = D\cos{30^\circ} = \max\big(c_1t, \sqrt{3}c_2t\big) \geq c_1t$. Similarly, $\overline{\rm Cq} = D\sin{30^\circ} = \max\big(\frac{c_1t}{\sqrt{3}}, c_2t\big) \geq c_2t$. Thus, the smallest bounding rectangle of the points in $X(\C{F})$ lies completely inside the bounding rectangle. So, all points in the configuration located inside the convex hull are enclosed by the bounding rectangle $\mathrm{pdqC}$.
% \end{proof}

% \begin{lemma}\label{lem:steiner_point_in_circle}
%     For any two points in or on the boundary of the bounding rectangle $\mathrm{peqC}$ in~\Cref{fig:bounding_quad}, if a Steiner point is adjacent to both of them, then the Steiner point must be located inside the circle centred at $C$ and radius $r = 2D$.
% \end{lemma}

% \begin{proof}
    
% \end{proof}

% For any SMT of $X'(\C{F},\epsilon)$, we call a \emph{\textbf{connecting path}} as any path which starts from some vertex inside the bounding rectangle and ends at a vertex on the convex hull, such that all internal vertices are Steiner points. Our analysis proceeds similar to that of the pair of concentric parallel regular polygons having more than 12 sides. The following lemma is analogous to~\Cref{left_right_turn_path,mincut_1}. 

% \begin{lemma}\label{lem:singly_connected}
%     For any SMT of $X'(\C{F},\epsilon)$, there exists a counter-clockwise or clockwise connecting path having at least one edge of length at least $\frac{R-r}{2}$. Also, there exists only one edge-disjoint connecting path.
% \end{lemma}

% \begin{proof}
%     Firstly, it can be easily seen that~\Cref{left_right_turn_path} can be extended to prove the existence of at least one counter-clockwise or clockwise connecting path.
    
%     Let $H$ be arbitrary point on the Euclidean plane. If $\mathcal C$ be a counter-clockwise path starting from $H$ such that no edge in the counter clockwise path has a length of more than $\ell$, for some $\ell \in \mathbb{R}^{+}$. Then we know that $\mathcal C$ is contained entirely in the circle centred at $H$ with radius $2\ell$.

%     The distance between a point in or on the bounding rectangle and a point on the convex hull is at least $R-D = r \gg D$ because the bounding rectangle is completely contained inside the circle with centre $C$ and radius $D$. So, all counter-clockwise or clockwise connecting paths must contain an edge of length at least $\frac{R-D}{2} \gg D$. 
    
%     If there are more than one edge-disjoint connecting paths, then we can always remove the edge of length at least $\frac{R-D}{2}$ from some counter-clockwise or clockwise connecting path and add another edge to get a smaller tree.

%     Observe that in the configuration, the distance between any two points in the bounding box is at most $D$. Also, the distance between two consecutive points on the convex hull is $\OO{t^{1-2\alpha}}$. When we remove one such edge of length $\OO(t^{2\alpha})$, either all the vertices on the convex hull are not connected or all the vertices inside the hull are not connected. In both cases, we can replace it by an edge of length at most $2\pi c$ or $r = 2D = \OO(t)$, respectively. This gives us a smaller tree, thereby contradicting the minimality of a tree containing two or more edge-disjoint connecting paths.
% \end{proof}

% Here, a \emph{\textbf{singly connected}} topology stands for one in which all connecting paths have at least one edge in common, i.e. there is only one edge-disjoint connecting path. The SMT shown in~\Cref{fig:smt_new} is a singly connected topology.

$\mathrm{CH}(X'(\mathbb{F},\epsilon)\setminus \mathrm{Circ}(D,r,n'))$ is enclosed by the bounding rectangle $PDQC$ and $D$ must lie on $\mathrm{CH}(X'(\mathbb{F},\epsilon)\setminus \mathrm{Circ}(D,r,n'))$. We label the vertices of $\mathrm{CH}(X'(\mathbb{F},\epsilon)\setminus \mathrm{Circ}(D,r,n'))$ as $D,P_1,P_2,\ldots,P_k$ in the counter-clockwise order. Let $\mathrm{CH}(X'(\mathbb{F},\epsilon))$ be the convex hull of all the points. By~\Cref{lem:convhull_pts}, all the points in $\mathrm{Circ}(D,r,n')$ lie on $\mathrm{CH}(X'(\mathbb{F},\epsilon))$. Let $EP_i$ and $FP_j$ be edges in $\mathrm{CH}(X'(\mathbb{F},\epsilon))$, such that $P_i,P_j \notin \mathrm{Circ}(D,r,n')$.

The SMT of $X'(\mathbb{F},\epsilon)$ clearly lies inside its convex hull, $\mathrm{CH}(X'(\mathbb{F},\epsilon))$. We show that the Steiner hull can be further restricted to the bounding rectangle $PDQC$ and the convex polygon formed by the points in $\mathrm{Cone}(D,r,n')$. For this we use Theorem 1.5 in~\cite{hwang1992steiner}, as stated below.

\begin{proposition}~\cite{hwang1992steiner}\label{thm:steiner_hull}
    Let $H$ be a Steiner hull of $N$. By sequentially removing wedges $\rm abc$ from the remaining region, where $\rm a$, $\rm b$, $\rm c$ are terminals but $\triangle{\rm abc}$ contains no other terminal, $a$ and $c$ are on the boundary and $\angle{\rm abc} \geq 120^\circ$, a Steiner hull $H'$ invariant to the sequence of removal is obtained.
\end{proposition}

\begin{lemma}\label{thm:steiner_hull_regions}
    The region comprising of the bounding rectangle $PDQC$ according to Algorithm $\mathcal{A}$ and the convex polygon formed by the set of points $\mathrm{Cone}(D,r,n')$ is a Steiner hull of $X'(\mathbb{F},\epsilon)$.
\end{lemma}

\begin{proof}
    Firstly, let us consider the wedge $EP_{i+1}P_i$. All the points are terminals, $E$ and $P_i$ are boundary points, and $\triangle{EP_{i+1}P_i}$ contains no other terminal. Now, $\angle{EP_{i+1}P_i}$ is greater than the exterior angle of $\angle{EP_{i+1}D}$, which in turn is greater than $\angle{EDP_{i+1}}$. $\angle{EDP_{i+1}} \geq \angle{\rm EDP} = 120^\circ$, by the construction. Therefore, $\angle{EP_{i+1}P_i} \geq 120^\circ$. By applying~\Cref{thm:steiner_hull}, we can remove the wedge $EP_{i+1}P_i$ from the convex hull $\mathrm{CH}(X'(\mathbb{F},\epsilon))$ to get a smaller Steiner hull. This can be repeated for the wedges $EP_{i+2}P_{i+1}, EP_{i+3}P_{i+2}, \ldots, EDP_k$. The same argument can also be used to get rid of the wedges $FP_{j-1}P_j, FP_{j-2}P_{j-1}, \ldots, FDP_1$. So, we get the final Steiner hull $H'$ to be the union of the bounding rectangle $PDQC$ and the convex polygon formed by the points in $\mathrm{Cone}(D,r,n')$.
\end{proof}

Given the nature of the above Steiner hull, we show that we can treat $X(\mathbb{F})$ and $\mathrm{Cone}(D,r,n')$ separately. 
\begin{lemma}\label{thm:steiner_tree}
    There is an SMT of $X'(\mathbb{F},\epsilon)$ that is the union of an SMT of $X(\mathbb{F})$ and an SMT of the points in $\mathrm{Cone}(D,r,n')$, with $D$ being common to both of them.
\end{lemma}

\begin{proof}
    According to~\Cref{thm:steiner_hull_regions}, there is an SMT of $X'(\mathbb{F},\epsilon)$ that lies completely inside the the bounding quadrilateral $PDQC$ and the convex polygon formed by $\mathrm{Cone}(D,r,n')$. These two regions have $D$ as the only common point. Therefore, $D$ is an articulation point in the tree and connects these two regions. So, we have this SMT of $X'(\mathbb{F},\epsilon)$ as the union of an SMT of $X(\mathbb{F})$ and an SMT of the points in $\mathrm{Cone}(D,r,n')$.
\end{proof}

We can identify a structure for an SMT of the points in $\mathrm{Cone}(D,r,n')$ using~\cite{weng1995steiner}.
\begin{lemma}\label{thm:steiner_tree_part}
    There is an SMT of the points in $\mathrm{Cone}(D,r,n')$ that is as shown in~\Cref{fig:steiner_new}. In the SMT, $D$ is connected to the two middle points in $\mathrm{Circ}(D,r,n')$ via a Steiner point $S^t$. The other points in $\mathrm{Circ}(D,r,n')$ are connected along the circumference.
\end{lemma}

\begin{proof}
    The number of points in $\mathrm{Circ}(D,r,n')$ is $\OO(t^{\gamma\alpha})$. We can take the constant factor to be large enough so that $|\mathrm{Circ}(D,r,n')| >= 12$. If we complete the regular polygon on $\mathcal{C}$ having $\mathrm{Circ}(D,r,n')$ as a subset of its vertices, then it contains more than $12$ vertices and along with the centre $D$ has a SMT with structure given in~\cite{weng1995steiner}.

    Let the Steiner tree for $\mathrm{Cone}(D,r,n')$ as shown in~\Cref{fig:steiner_new} be denoted by $\mathcal{T}_1$. If this is not minimal, then there exists another Steiner tree $\mathcal{T}_2$ such that $|\mathcal{T}_2| < |\mathcal{T}_1|$. Then we can replace $\mathcal{T}_1$ by $\mathcal{T}_2$ in the SMT of the regular polygon and its centre to get a shorter Steiner tree. This contradicts the minimality of the structure given in~\cite{weng1995steiner}. Therefore, the SMT of $\mathrm{Cone}(D,r,n')$ follows the structure in~\Cref{fig:steiner_new}.
\end{proof}

\begin{figure}[h]
\centering
\includegraphics[width=8cm]{steiner_new.png}
\caption{SMT of $\mathrm{Cone}(D,r,n')$}
\label{fig:steiner_new}
\end{figure}
Finally, we prove the NP-hardness of \ESMT on $f(n)$-Almost Convex Sets of $n$ terminals, when $f(n) = \Omega(n^\epsilon)$ for some $\epsilon \in (0,1]$.
\begin{theorem}\label{thm:redn_final}
    Let $\C{S}^{*}_{\mathbb{F},\epsilon}$ denote an SMT of $X'(\mathbb{F},\epsilon)$ and $|\C{S}^*_{\mathbb{F},\epsilon}|$ denote its length. If $\mathbb{F}$ has an exact cover, then $|\C{S}^{*}_{\mathbb{F},\epsilon}| \leq f(n,t,\hat{C}) + |\mathcal{T}_1|$, otherwise $|\C{S}^{*}_{\mathbb{F},\epsilon}| \geq f(n,t,\hat{C}) + \frac{1}{200nt} + |\mathcal{T}_1|$, where $\hat{C}$ is the number of crossovers, i.e.~hexagonal gadgets, and $f$ is a positive real-valued function of $n,t,\hat{C}$ as stated in~\Cref{thm:redcn}.
\end{theorem}

\begin{proof}
    From~\Cref{thm:steiner_tree}, we have $|\C{S}^{*}_{\mathbb{F},\epsilon}| = |\C{S}^{*}| + |\mathcal{T}_1|$. From~\Cref{thm:steiner_tree_part}, we can compute the length of $\mathcal{T}_1$ as a function of $t$, $\alpha$, and $\gamma$. Finally, using~\Cref{thm:redcn} we get the required reduction.
\end{proof}

%Thus, the \ESMT problem on $f(n)$-Almost Convex Sets of $n$ terminals, when $f(n) = \Omega (n^\epsilon)$ and where $\epsilon \in (0,1]$, is NP-Hard.

% Strongly NP-Complete problems do not have an FPTAS.
% Therefore, we get the following result.

% \begin{theorem}\label{thm:no_fptas}
%     There does not exist any FPTAS for the ESMT problem on $f(n)$-Almost Convex Sets of $n$ terminals, where $f(n) = \Omega(n^\epsilon)$ and $\epsilon \in (0,1]$.
% \end{theorem}

Since it is not known if the ESMT problem is in NP, Garey et al.~\cite{garey1977complexity} show the NP-completeness of a related problem called the {\sc Discrete Euclidean Steiner Minimal Tree} (DESMT) problem, which is in NP. We define the DESMT problem as given in~\cite{garey1977complexity}. The DESMT problem takes as input a set $\C{X}$ of integer-coordinate points in the plane and a positive integer $L$, and asks if there exists a set $\C{Y} \supseteq \C{X}$ of integer-coordinate points such that some spanning tree $\C{T}$ for $\C{Y}$ satisfies $|\C{T}|_d \leq L$, where $|\C{T}|_d = \Sigma_{e \in E(\mathcal T)} \lceil\overline{e}\rceil$, i.e.~we round up the length of each edge to the least integer not less than it.

In order to show that DESMT is NP-hard, the same reduction as that of the ESMT problem can be used, followed by scaling and rounding the coordinates of the points. Theorem 4 of~\cite{garey1977complexity} proves that the DESMT problem is NP-Complete. Moreover, since it is Strongly NP-Complete, the DESMT problem does not admit any FPTAS. Finally in Theorem 5 of~\cite{garey1977complexity}, Garey et al. show that as a consequence, the ESMT problem does not have any FPTAS as well.

Now we show that the DESMT problem is NP-hard even on $f(n)$-Almost Convex Sets of $n$ terminals, when $f(n) = \Omega (n^\epsilon)$ and where $\epsilon \in (0,1]$.

In Section 7 of~\cite{garey1977complexity}, the reduced instance $X(\mathbb{F})$ of ESMT is converted into an instance $X_{d}(\mathbb{F})$ of DESMT. The conversion is as follows:\\
$X_{d}(\mathbb{F}) = \{(\lceil 12M\cdot 200nt\cdot x_1\rceil, \lceil 12M\cdot 200nt\cdot x_2\rceil): x=(x_1,x_2)\in X(\mathbb{F})\}$, where $M = |X(\mathbb{F})|$.

We apply a similar conversion to the reduced ESMT instance $X'(\mathbb{F},\epsilon)$, to convert it into a DESMT instance of an $\Omega(n^\epsilon)$-Almost Convex Set. The conversion goes as follows:\\
$X'_{d}(\mathbb{F},\epsilon) = \{(\lceil 12N\cdot 200nt\cdot x_1\rceil, \lceil 12N\cdot 200nt\cdot x_2\rceil): x=(x_1,x_2)\in X'(\mathbb{F},\epsilon)\}$, where $N = |X'(\mathbb{F},\epsilon)|$.

The next two lemmas establish the validity of $X'_{d}(\mathbb{F},\epsilon)$ as an instance of DESMT and the upper bounds on the size of the constructed instance. Note that the reduction from X3C followed by the conversion can be done in polynomial time.
\begin{lemma}\label{lem:valid_desmt}
    The instance $X'_{d}(\mathbb{F},\epsilon)$ constructed above is a valid DESMT instance.
\end{lemma}

\begin{proof}
    All the points in $X'_{d}(\mathbb{F},\epsilon)$ have integer coordinates according to the conversion stated above. So, it is a DESMT instance.
\end{proof}

\begin{lemma}\label{lem:numpts_same}
    The reduced DESMT instance $X'_{d}(\mathbb{F},\epsilon)$ has $N$ distinct points, where $N = |X'(\mathbb{F},\epsilon)|$.
\end{lemma}

\begin{proof}
    The minimum distance between any two points in $X'(\mathbb{F},\epsilon)$ is that between two consecutive points of $\mathrm{Circ}(D,r,n')$, which is $\OO(\frac{1}{t^\gamma})$. Recall~\Cref{lem:redcn_size} which establishes that $N = \OO(t^{\gamma\alpha})$. So, the minimum distance between any two points in $X'_{d}(\mathbb{F},\epsilon)$ is $\OO(N \cdot nt \cdot \frac{1}{t^\gamma}) = \OO(nt^{\alpha+1})$. Because of the substantial distance obtained between points after scaling, the rounding will not cause any distinct points of $X'_{d}(\mathbb{F},\epsilon)$ to coincide. Therefore, the number of points remains unchanged, i.e.~$|X'_{d}(\mathbb{F},\epsilon)| = |X'(\mathbb{F},\epsilon)| = N$.
\end{proof}

Now we present the following lemma for the constructed DESMT instance $X'_{d}(\mathbb{F},\epsilon)$ analogous to~\Cref{lem:redcn_size} for the ESMT instance $X'(\mathbb{F},\epsilon)$.

\begin{lemma}\label{lem:desmt_redn_size}
    The reduced DESMT instance $X'_{d}(\mathbb{F},\epsilon)$ constructed is an $\Omega(N^\epsilon)$-Almost Convex Set, where  $N = |X'_{d}(\mathbb{F},\epsilon)|$.
\end{lemma}

\begin{proof}
    From~\Cref{lem:redcn_size}, we know that the reduced ESMT instance $X'(\mathbb{F},\epsilon)$ has $\Omega(N^\epsilon)$ points inside its convex hull $\mathrm{CH}(X'(\mathbb{F},\epsilon))$, and $N = |X'(\mathbb{F},\epsilon)|$. The number of points after conversion remains the same by~\Cref{lem:numpts_same}. We need to show that after conversion, except for the points of $\OO(t)$ gadgets, no other points inside the convex hull $\mathrm{CH}(X'(\mathbb{F},\epsilon))$ lie on the new convex hull $\mathrm{CH}(X'_{d}(\mathbb{F},\epsilon))$. The number of points in each of the $\OO(t)$ anomalous gadgets are constant in number, and hence not too many points from the interior of $\mathrm{CH}(X'(\mathbb{F},\epsilon))$ can lie on $\mathrm{CH}(X'_{d}(\mathbb{F},\epsilon))$. 
    
    After conversion, all the points on a horizontal connecting row have the same $y$-coordinate, as they initially had the same $y$-coordinate and therefore undergo the same transformation. Thus, all the points on a horizontal connecting row still lie on a horizontal line segment in $X'_{d}(\mathbb{F},\epsilon)$. Similarly, all the points on a vertical connecting row still lie on a vertical line segment in $X'_{d}(\mathbb{F},\epsilon)$. This implies that none of the points on the connecting rows can be a part of $\mathrm{CH}(X'_{d}(\mathbb{F},\epsilon))$ as there can be no line passing through them that also contains all terminal points on one side of it. 

    The same thing holds for the square and hexagonal gadgets (crossovers) as well, except the hexagonal gadgets placed at the beginning or end of any row. This is because all the points which are a part of these square and hexagon gadgets are surrounded by connecting row points all four sides, above, below, left and right. So again, only the terminators and the hexagonal gadgets appearing at the beginning or end of any row contribute to $\mathrm{CH}(X'_{d}(\mathbb{F},\epsilon))$.

    Now, since we had adjusted the number of points in the long rows of the terminators and hexagonal gadgets such that their lengths and breadths are some constants, the number of points in each of the terminators and hexagonal gadgets can be bounded by some constant as the minimum distance between any two consecutive points on the long rows or standard rows is at least $\frac{1}{11}$. Therefore, each of these gadgets contribute some constantly many points to $\mathrm{CH}(X'_{d}(\mathbb{F},\epsilon))$.
    
    As we have seen in~\Cref{lem:redcn_size}, the number of terminators is $6t+2$ and the number of hexagonal gadgets corresponding to the beginning or end of any row is at most $6n$. Therefore, the number of points contributed by the terminators and the hexagonal gadgets placed at the beginning or the end of any row, to $\mathrm{CH}(X'_{d}(\mathbb{F},\epsilon)$ is $\OO(t+n) = \OO(t)$, as $n \leq t$. Even if all the points in $\mathrm{Circ}(D,r,n')$ lie on the new convex hull $\mathrm{CH}(X'_{d}(\mathbb{F},\epsilon)$, we have $\Omega(t^\gamma) = \Omega(N^\epsilon)$ points inside it. Thus we are done.
\end{proof}

We get the following theorem from~\Cref{lem:valid_desmt,lem:numpts_same,lem:desmt_redn_size}.

\begin{theorem}\label{thm:desmt_redn}
    The instance $X'_{d}(\mathbb{F},\epsilon)$ constructed is a valid DESMT instance on an $\Omega(N^\epsilon)$-Almost Convex Set, where  $|X'_{d}(\mathbb{F},\epsilon)| = |X'(\mathbb{F},\epsilon)| = N$.
\end{theorem}

Following Theorems 3 and 4 in~\cite{garey1977complexity}, we get that the DESMT problem is NP-Complete for $\Omega(N^\epsilon)$-Almost Convex Sets, where $N$ is the total number of terminals. Since we get the reduced instance $X'_{d}(\mathbb{F},\epsilon)$ from the X3C instance $(n,\mathbb{F})$, the DESMT problem is strongly NP-complete for $\Omega(N^\epsilon)$-Almost Convex Sets, and does not admit any FPTAS.

Using Theorem 5 of~\cite{garey1977complexity}, we get that if the ESMT problem has an FPTAS, then the X3C problem can be solved in polynomial time. The Theorem also applies for our case of $\Omega(N^\epsilon)$-Almost Convex Sets. Therefore, we get the following theorem,

\begin{theorem}\label{thm:no_fptas}
    There does not exist any FPTAS for the ESMT problem on $f(n)$-Almost Convex Sets of $n$ terminals, where $f(n) = \Omega(n^\epsilon)$ and $\epsilon \in (0,1]$, unless \pnp.
\end{theorem}

\section{Conclusion}
 In this paper, we first study ESMT on vertices of $2$-CPR $n$-gons and design a polynomial time algorithm. It remains open to design a polynomial time algorithm for ESMT on $k$-CPR $n$-gons, or show NP-hardness for the problem. Next, we study the problem on $f(n)$-Almost Convex Sets of $n$ terminals. For this NP-hard problem, we obtain an algorithm that runs in $2^{\OO(f(n)\log n)}$ time. We also design an FPTAS when $f(n) = \OO(\log n)$. On the other hand, we show that there cannot be an FPTAS if $f(n) = \Omega(n^{\epsilon})$ for any $\epsilon \in (0,1]$, unless \pnp. The question of existence of FPTASes when $f(n)$ is a polylogarithmic function remains open.
%----------------------------------------------

%\bibliographystyle{plain}
\bibliography{refs}

\end{document}
