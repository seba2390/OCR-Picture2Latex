
\section{Preliminaries}\label{sec:prelims}

\subparagraph{Notations.}
For a given positive integer $k \in \mathbb{N}$, the set of integers $\{1,2,\ldots,k\}$ is denoted for short as $[k]$. Given a graph $G$, the vertex set is denoted as $V(G)$ and the edge set as $E(G)$. Given two graphs $G_1$ and $G_2$, $G_1 \cup G_2$ denotes the graph $G$ where $V(G) = V(G_1) \cup V(G_2)$ and $E(G) = E(G_1) \cup E(G_2)$. 

In this paper, a regular $n$-gon is denoted by $A_1A_2A_3...A_n$ or $B_1B_2B_3...B_n$. For convenience, we define $A_{n + 1} := A_1$, $B_{n + 1} := B_1$, $A_0 := A_n$ and $B_0 := B_n$. We use the notation $\{A_i\}$ to denote the polygon $A_1A_2A_3 \ldots A_n$ and $\{B_i\}$ to denote the polygon $B_1B_2B_3 \ldots B_n$. For any regular polygon $A_1A_2A_3...A_n$, the circumcircle of the polygon is denoted as $(A_1A_2A_3...A_n)$. Given any $n$-vertex polygon in the Euclidean plane with vertices $\mathcal P = P_1P_2P_3\ldots P_n$, and interval in $\mathcal{K}$ is a subset of consecutive vertices $P_iP_{i+1\ldots P_j}$, $i,j\in [n]$, also denoted as $[P_i,P_j]$. Here $P_i$ is considered the starting vertex of the interval and $P_j$ the ending vertex. For any $P_k$, $i \leq k\leq j$ in the interval we will also use the notation $P_i \leq P_k \leq P_j$.

Given two points $P$, $Q$ in the Euclidean plane, we denote by ${\sf dist}(P,Q)$ the Euclidean distance between $P$ and $Q$. Given a line segment $AB$ in the Euclidean plane, $\overline{AB} = {\sf dist}(A,B)$. For two distinct points $A$ and $B$, $L_{AB}$ denotes the line containing $A$ and $B$; and $\overrightarrow{AB}$ denotes the ray originating from $A$ and containing $B$. 

When we refer to a graph $\mathcal{G}$ in the Euclidean plane then $V(\mathcal{G})$ is a set of points in the Euclidean plane, and $E(\mathcal{G})$ is a subset of the family of line segments $\{P_1P_2 | P_1,P_2 \in V(\mathcal{G})\}$. For any tree $\mathcal T$ in the Euclidean plane, we denote by the notation $|\mathcal T|$ the value of $\Sigma_{e \in E(\mathcal T)} \overline{e}$. A path in a tree $\mathcal T$ is uniquely specified by the sequence of vertices on the path; therefore, $P_1$, $P_2$, $P_3$, \ldots, $P_k$ (where $P_i \in V(\mathcal T), \forall i \in [k]$ and $P_iP_{i+1} \in E(\mathcal T), \forall i \in [k-1]$) denotes the path starting from the vertex $P_1$, going through the vertices $P_2$, $P_3$, \ldots, $P_{k-1}$ and finally ending at $P_k$. Equivalently, we can specify the same path as \emph{the path from $P_1$ to $P_k$}, since $\mathcal T$ is a tree. Consider the graph $T$ such that $V(T) = \{v_P| P \in V(\mathcal{T})\}$, $E(T) = \{v_{P_1}v_{P_2}| P_1P_2 \mbox{ is a line segment in } E(\mathcal{T})\}$. Then $T$ is said to be the topology of $\mathcal{T}$ while $\mathcal{T}$ is said to realize the topology $T$. Given two trees $\mathcal{T}_1$, $\mathcal{T}_2$ in the Euclidean plane, $\mathcal{T}' = \mathcal{T}_1\cup \mathcal{T}_2$ is the graph where $V(\mathcal{T}')= V(\mathcal{T}_1) \cup V(\mathcal{T}_2)$ and $E(\mathcal{T}')= E(\mathcal{T}_1) \cup E(\mathcal{T}_2)$. 

Given any graph $G$, a Steiner minimal tree or SMT for a terminal set $\mathcal{P} \subseteq V(G)$ is the minimum length connected subgraph $G'$ of $G$ such that $\mathcal{P} \subseteq V(G')$. The {\sc Steiner Minimal Tree} problem on graphs takes as input a set $\mathcal{P}$ of terminals and aims to find a minimum length SMT for $\mathcal{P}$. For the rest of the paper, we also refer to a Euclidean Steiner minimal tree as an SMT. Given a set of points $\mathcal{P}$ in the Euclidean plane, the convex hull of $\mathcal{P}$ is denoted as $\mathrm{CH(\mathcal P)}$.

\subparagraph{Euclidean Minimum Spanning Tree (MST).}
Given a set $\mathcal P$ of $n$ points in the Euclidean plane, let $G$ be a graph where $V(G) = \{v_P| P \in \mathcal P\}$ and $E(G) = \{v_{P_i}v_{P_j} | P_i,P_j \in \mathcal P\}$. Also, a weight function $w_{G}: E(T) \rightarrow \mathbb{R}$ is defined such that for each edge $v_{P_1}v_{P_2} \in E(T)$, $w_{G}(v_{P_1}v_{P_2}) = \overline{P_1P_2}$. The Euclidean minimum spanning tree of a set $\mathcal P$ is the minimum spanning tree of the graph $G$ with edge weights $w_G$. Note that a Steiner tree may have shorter length than a minimum spanning tree of the point set $\mathcal P$. 

In the plane, the Euclidean minimum spanning tree is a subgraph of the Delaunay triangulation. Using this fact, the Euclidean minimum spanning tree for a given set of points in the Euclidean plane can be found in $\OO(n\log n)$ time as discussed in \cite{Shamos1975ClosestpointP}. 

\subparagraph{Properties of a Euclidean Steiner minimal tree.}
A Euclidean Steiner minimal tree (SMT) has certain structural properties as given in~\cite{cockayne1967steiner}. We state them in the following Proposition.

\begin{proposition}\label{smt-prop}
Consider an SMT on $n$ terminals.
 \begin{enumerate}
   \item No two edges of the SMT intersect with each other.
 
   \item Each Steiner point has degree exactly $3$ and the incident edges meet at $120^\circ$ angles. The terminals have degree at most $3$ and the incident edges form angles that are at least $120^\circ$.
  
   \item The number of Steiner points is at most $n-2$, where $n$ is the number of terminals.

\end{enumerate}
\end{proposition}

 A full Steiner tree (FST) is a Steiner tree (need not be minimal, but may include Steiner points) having exactly $n-2$ Steiner points, where $n$ is the number of terminals. In an FST, all terminals are leaves and Steiner points are interior nodes. When the length of an FST is minimized, it is called a minimum FST.

All SMTs can be decomposed into FST components such that, in each component a terminal is always a  leaf. This decomposition is unique for a given SMT~\cite{hwang1992steiner}. A topology for an FST is called a full Steiner topology and that of a Steiner tree is called a Steiner topology.


%For a tree $\mathcal T$, we would denote the set of vertices (the terminal vertices and the Steiner points) as $V(\mathcal T)$ and the set of edges as $E(\mathcal T)$. Similarly for a topology $T$, $V(T)$ and $E(T)$ denote vertex set (the terminal vertices and the Steiner points) and the edge set respectively. \todo{\color{white}Anubhav: Added this defition}

\subparagraph{Steiner Hulls.}
A Steiner hull for a given set of points is defined to be a region which is known to contain an SMT. We get the following propositions from~\cite{hwang1992steiner}.

\begin{proposition}\label{convex-steiner}
    For a given set of terminals, every SMT is always contained inside the convex hull of those points. Thus, the convex hull is also a Steiner hull.
\end{proposition}

The next two propositions are useful in restricting the structure of SMTs and the location of Steiner points.

\begin{proposition} [The Lune property]\label{lune}
    Let $\rm UV$ be any edge of an SMT. Let $L(\rm{U},\rm{V})$ be the lune-shaped intersection of circles of radius $|\rm UV|$ centered on $\rm U$ and $\rm V$. No other vertex of the SMT can lie in $L(\rm{U},\rm{V})$, except $U$ and $V$ themselves.
\end{proposition}

\begin{proposition} [The Wedge property]\label{wedge}
    Let $W$ be any open wedge-shaped region having angle $120^\circ$ or more and containing none of the points from the input terminal set $\mathcal P$. Then $W$ contains no Steiner points from an SMT of $\mathcal P$.
\end{proposition}

\subparagraph{Approximation Algorithms.}
We define all the necessary terminology required in terms of a minimization problem, as ESMT is a minimization problem.
%\begin{definition} [Approximation Factor for a Minimization Problem]
%    Let $\mathcal{P}$ be a minimization problem. An algorithm $\mathcal{A}$ for the problem $\mathcal{P}$ is called an $\alpha$ factor approximation algorithm if, for every instance $\Pi$ of $\mathcal{P}$, we have $\rm{ALG}(\Pi) \leq \alpha \rm{OPT}(\Pi)$ where $\rm{ALG}(\Pi)$ and $\rm{OPT}(\Pi)$ are the values of the output of the algorithm and optimal solution for the instance $\Pi$ respectively. $\alpha$ can be a constant or a function of the input size $n$, and is always at least $1$.
%\end{definition}

%\begin{definition} [Polynomial Time Approximation Scheme (PTAS)]
%    An algorithm is called a polynomial time approximation scheme (PTAS) for a problem if it takes an input instance and a parameter $\epsilon > 0$, and outputs a solution with approximation factor $(1+\epsilon)$ for a minimization problem in time $\OO(n^{f(1/\epsilon)})$ where $n$ is the input size and $f(1/\epsilon)$ is any computable function.
%\end{definition}

\begin{definition} [Efficient Polynomial Time Approximation Scheme (EPTAS)]
    An algorithm is called an efficient polynomial time approximation scheme (EPTAS) for a problem if it takes an input instance and a parameter $\epsilon > 0$, and outputs a solution with approximation factor $(1+\epsilon)$ for a minimization problem in time $f(1/\epsilon)n^{\OO(1)}$ where $n$ is the input size and $f(1/\epsilon)$ is any computable function.
\end{definition}

\begin{definition} [Fully Polynomial Time Approximation Scheme (FPTAS)]
    An algorithm is called a fully polynomial time approximation scheme (FPTAS) for a problem if it takes an input instance and a parameter $\epsilon > 0$, and outputs a solution with approximation factor $(1+\epsilon)$ for a minimization problem in time $(1/\epsilon)^{\OO(1)}n^{\OO(1)}$ where $n$ is the input size.
\end{definition}

% appending preliminaries.tex --- Anubhav 
