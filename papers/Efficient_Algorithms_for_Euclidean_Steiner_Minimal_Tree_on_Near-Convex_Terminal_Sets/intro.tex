
\section{Introduction}\label{sec:intro}
% A Steiner Minimal Tree (SMT) of a set of points $\mathcal P$ on the Euclidean plane is a tree on that plane interconnecting all points in $\mathcal P$, and has the minimum length among all such interconnecting trees of $\mathcal P$. Formally, the \emph{Euclidean Steiner Tree Problem} can be stated as:\\

% \begin{problem}{Euclidean Steiner Tree Problem}
%     \textbf{Input}: A Set of points $\mathcal P = \{ A_1, A_2, \ldots, A_n\}$ on the 2D Euclidean plane.

% \textbf{Find}: The Steiner Minimum Tree (SMT) of $\mathcal P$, that is the tree of minimum total edge length interconnecting all the vertices in $\mathcal P$.
% \end{problem}

% \vspace{0.2cm}

% Due to~\cite{smt_nphard}, we know that the Euclidean Steiner Tree Problem is \texttt{NP-Complete} in the general setting. However, many were interested to explore the structure and properties of the Steiner Minimal Tree under various specialized configurations of the set of input points.\\

% For instance, in~\cite{regular_polygons}, the structures of the Steiner Minimal Trees of the set of points forming a regular $n$-gon were explored. One of the major conclusion of the paper was that for $n \ge 6$, any set of $(n - 1)$ polygon edges makes up the Minimum Spanning Tree as well as the SMT of the set of points. However, for $n \in \{3, 4, 5\}$, introduction of external Steiner points were necessary for construction of the SMT.\\

% Again, in~\cite{polygon_and_origin}, the structures of the Steiner Minimal Trees of the set of points forming a regular $n$-gon along with its centre were explored. The paper showed that the SMT of such a set of points obtained specific topologies for specific values of $n$. The paper provided individual topologies for each $n$ in $n \le 11$, but the topology for all configurations with $n \ge 12$ had a very similar structure with $(n - 2)$ polygon edges and a single Steiner Point required to connect the centre with two of the consecutive polygon vertices. \\

% In this paper, we would be exploring the structure of the Steiner Minimal Trees of the set of points which form a pair of \emph{regular}, \emph{parallel} and \emph{concentric} $n$-gons (as in Figure \ref{intro_fig}). However, for smaller values of $n$, it is easy to obtain the exact structure of an SMT using any exponential algorithm. Hence we would be focusing on the case where $n \ge 13$ and where the pair of polygons are not very `close' to each other. 

% \begin{figure}[h] 
% \centering
% \subfloat[\centering The Terminal Points]{\includegraphics[width=5cm]{dodecagons.png}}
%     \qquad
% \subfloat[\centering The required Steiner tree] {\includegraphics[width=5cm]{dodecagons_steiner.png} }
% \caption{points for $n = 12$, $\lambda = 5$}  
% \label{intro_fig}
% \end{figure}

% We would be mostly following the terminology defined in~\cite{steiner_tree_book}. We start by introducing some definitions and recalling some results from~\cite{steiner_tree_book} in Section \ref{preliminaries}. In Section \ref{trapezoids} we would discuss how isosceles trapezoids are integrated into the structure of input points as well as one of the Steiner topologies of the corners of such an isosceles trapezoid. We briefly talk about the case where $n = 3$ (concentric, parallel, equilateral triangles) in Section \ref{triangles} and move on to our main focus of $n \ge 13$ in Section \ref{final_proofs} where the pair of polygons are not very `close' to each other. We define a particular type of topology called `singly connected topology' and conclude with Theorem \ref{final_proof}, proving that there exists an SMT of the input point following the singly connected topology.

The \ESMT problem asks for a network of minimum total length interconnecting a given finite set $\mathcal P$ of $n$ points in the Euclidean plane. Formally, we define the problem as follows, taken from~\cite{brazil2014history}:

\defproblem{\ESMT}{A set $\mathcal P$ of $n$ points in the Euclidean plane}{Find a connected plane graph $\mathcal T$ such that $\mathcal P$ is a subset of the vertex set $V(\mathcal T)$, and for the edge set $E(\mathcal T)$, $\Sigma_{e \in E(\mathcal T)} \overline e$ is minimized over all connected plane graphs with $\mathcal P$ as a vertex subset.}

Note that the metric being considered is the Euclidean metric, and for any edge $e \in E(\mathcal T)$, $\overline e$ denotes the Euclidean length of the edge. Here, the input set $\C{P}$ of points is often called a set of {\it terminals}, the points in $\C{S} = V(\mathcal T) \setminus \C{P}$ are called {\it Steiner points}. A solution graph $\mathcal T$ is referred to as a {\it Euclidean Steiner minimal tree}, or simply an SMT.
%\begin{definition}[Euclidean Steiner Minimal Tree problem] \label{def:prob}
%    Find a geometric network $T = (V,E)$,~i.e. a connected graph that is embedded in the plane, such that $N \subseteq V$ and $S = V \setminus N$ is a (possibly empty) set of points known as \textit{Steiner points}, such that $\sum_{e \subseteq E}|e|$ is minimized (where $|e|$ denotes the Euclidean length of edge $e \subseteq E$). The points in the given set $N$ are called \textit{Terminal points}.
%\end{definition}

 The \ESMT problem is a classic problem in the field of Computational Geometry. The origin of the problem dates back to Fermat (1601-1665) who proposed the problem of finding a point in the plane such that the sum of its distance from three given points is minimized. This is equivalent to finding the location of the Steiner point when given three terminals as input. Torricelli proposed a geometric solution to this special case of 3 terminal points. The idea was to construct equilateral triangles outside on all three sides of the triangle formed by the terminals, and draw their circumcircles. The three circles meet at a single point, which is our required Steiner point. This point came to be known as the {\it Torricelli point}. When one of the angles in the triangle is at least $120^\circ$, the minimizing point coincides with the obtuse angle vertex of the triangle. In this case, the Torricelli point lies outside the triangle and no longer minimizes the sum of distances from the vertices. However, when vertices of polygons with more than 3 sides are considered as a set of terminals, a solution to the Fermat problem does not in general lead to a solution to the \ESMT problem. For a more detailed survey on the history of the problem, please refer to~\cite{brazil2014history,hwang1992steiner}. For convenience, we refer to the \ESMT problem as ESMT.
 
 %and the Steiner Minimal tree as the SMT for a given terminal set.


 ESMT is NP-hard. In~\cite{garey1977complexity}, Garey et al.~prove a discrete version of the problem (Discrete ESMT) to be strongly NP-complete via a reduction from the \textsc{Exact Cover by 3-Sets} (X3C) problem. Although it is not known if the ESMT problem is in NP, it is at least as hard as any NP-complete problem. So, we do not expect a polynomial time algorithm for it. A recursive method using only Euclidean constructions was given by Melzak in~\cite{melzak1961problem} for constructing all the Steiner minimal trees for any set of $n$ points in the plane by constructing full Steiner trees of subsets of the points. Full Steiner trees are interconnecting trees having the maximum number of newly introduced points (Steiner points) where all internal junctions are of degree $3$. Hwang improved the running time of Melzak’s original exponential algorithm for full Steiner tree construction to linear time in~\cite{hwang1986linear}. Using this, we can construct an Euclidean Steiner minimal tree in $2^{\OO(n\log n)}$ time for any set of $n$ points. This was the first algorithm for \ESMT. The problem is known to be NP-hard even if all the terminals lie on two parallel straight lines, or on a bent line segment where the bend has an angle of less than $120^\circ$~\cite{rubinstein1997steiner}. Since the above sets of terminals all lie on the boundary of a convex polygon (or, are in convex position), this shows that ESMT is NP-hard when restricted to a set of points that are in weakly convex position.

 Although the ESMT problem is NP-hard, there are certain arrangements of points in the plane for which the Euclidean Steiner minimal tree can be computed efficiently, say in polynomial time. One such arrangement is placing the points on the vertices of a regular polygon. This case was solved by Du et al.~\cite{du1987steiner}. Their work gives exact topologies of the Euclidean Steiner minimal trees. Weng et al.~\cite{weng1995steiner} generalized the problem by incorporating the centre point of the regular polygon as part of the terminal set, along with the vertices. This case was also found to be polynomial time solvable.

Tractability in the form of approximation algorithms for ESMT has been extensively studied. It was proved in~\cite{garey1977complexity} that a fully polynomial time approximation scheme (FPTAS) cannot exist for this problem unless \pnp. However, we do have an FPTAS when the terminals are in convex position~\cite{scott1988convexity}. Arora's celebrated polynomial time approximation scheme (PTAS) for the ESMT and other related problems is described in~\cite{arora1998polynomial}. Around the same time, Rao and Smith gave an efficient polynomial time approximation scheme (EPTAS) in~\cite{rao1998approximating}. In recent years, an EPTAS with an improved running time was designed by Kisfaludi-Bak et al.~\cite{kisfaludi2022gap}. 
 %This is the current best known running time dependence on the number of terminals and approximation factor.
\paragraph*{Our Results.}

 In this paper, we first extend the work of~\cite{du1987steiner}  and~\cite{weng1995steiner}.
We state this problem as ESMT on $k$-Concentric Parallel Regular $n$-gons.

 \begin{definition}[$k$-Concentric Parallel Regular $n$-gons] \label{def:regular conc parallel polygons toh}
     $k$-Concentric Parallel Regular $n$-gons are $k$ regular $n$-gons that are concentric and where the corresponding sides of polygons are parallel to each other. 
 \end{definition}

 Please refer to Figure~\ref{fig:examples}(a) for an example of a $2$-Concentric Parallel Regular $12$-gon. We call $k$-Concentric Parallel Regular $n$-gons as $k$-CPR $n$-gons for short.

 
 We consider terminal sets where the terminals are placed on the vertices of $2$-CPR $n$-gons. In the case of $k=2$, the $n$-gon with the smaller side length will be called the inner $n$-gon and the other $n$-gon will be called the outer $n$-gon. Also, let $a$ be the side length of the inner $n$-gon, and $b$ be the side length of the outer $n$-gon. We define $\lambda = \frac{b}{a}$ and refer to it as the \emph{aspect ratio} of the two regular polygons. In~\Cref{polycase}, we derive the exact structures of the SMTs for $2$-CPR $n$-gons when the aspect ratio $\lambda$ of the two polygons is greater than $\frac{1}{1-4\sin{(\pi/n)}}$ and $n \geq 13$.

 Next, we consider ESMT on an $f(n)$-Almost Convex Point Set.
 \begin{definition}[$f(n)$-Almost Convex Point Set] \label{def:almost_conv}
     An $f(n)$-Almost Convex Point Set $\mathcal P$ is a set of $n$ points in the Euclidean plane such that there is a partition $\mathcal P = \mathcal P_1 \uplus \mathcal P_2$ where $\mathcal P_1$ forms the convex hull of $\mathcal P$ and $|\mathcal P_2| = f(n)$.
 \end{definition}
Please refer to Figure~\ref{fig:examples}(b) for an example of a $5$-Almost Convex Set of $13$ points. In~\Cref{sec:exact_algo}, we give an exact algorithm for ESMT on $f(n)$-Almost Convex Sets of $n$ terminals. The running time of this algorithm is $2^{\OO(f(n) \log n)}$. Thus, when $f(n) = \OO(\frac{n}{\log n})$, then our algorithm runs in $2^{\OO(n)}$ time, and when $f(n) = o(n)$ then the running time is $2^{o(n\log n)}$. This is an  improvement on the best known algorithm for the general case~\cite{hwang1992steiner}.

Next, for $f(n) = \OO(\log n)$, we give an FPTAS in~\Cref{subsec:fptas}. On the other hand in~\Cref{subsec:apx_hardness} we show that, for all $\epsilon \in (0,1]$, when $f(n) \in \Omega(n^\epsilon)$, there cannot exist any FPTAS unless \pnp.


\begin{figure}[h]
\centering
\subfloat[\centering $2$-CPR $12$-gons] {\includegraphics[width=6cm]{dodecagons.png}}
    \qquad
\subfloat[\centering \centering $f(n)$-Almost Convex Point Set for $n = 13$, $f(n) = 5$] {\includegraphics[width=6cm]{5_almost_convex.png}}
\caption{Examples for~\Cref{def:regular conc parallel polygons toh} and~\Cref{def:almost_conv}}
\end{figure}\label{fig:examples}
