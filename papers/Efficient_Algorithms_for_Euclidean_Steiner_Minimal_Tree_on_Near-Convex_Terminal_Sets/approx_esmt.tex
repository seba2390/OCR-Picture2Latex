\section{Approximation Algorithms for \ESMT}\label{sec:apx_esmt}

The \ESMT problem is NP-hard as shown by Garey et al. in~\cite{garey1977complexity}. Garey et al. also prove that there cannot be an FPTAS (fully polynomial time approximation scheme) for this problem unless $P=NP$. At the same time, the case when all the terminals lie on the boundary of a convex region admits an FPTAS as given in~\cite{scott1988convexity}. We aim to conduct a more fine-grained analysis for the problem by considering $f(n)$-Almost Convex Point Sets of $n$ terminals and studying the existence of FPTASes for different functions $f(n)$. First, we present an FPTAS for \ESMT on $f(n)$-Almost Convex Sets of $n$ terminals, when $f(n) = \OO (\log n)$. Next, we prove that no FPTAS exists for the case when $f(n) =\Omega (n^\epsilon)$, where $\epsilon \in (0,1]$.

\subsection{FPTAS for \ESMT on Cases of Almost Convex Point Sets}\label{subsec:fptas}

We first propose an algorithm for computing the SMT of a planar graph $G$ having $N$ vertices and $n$ terminals, out of which $k$ terminals lie on the outer face of $G$ and the remaining terminals lie within the boundary. Next, following the procedure in~\cite{scott1988convexity} we get an FPTAS for \ESMT on $f(n)$-Almost Convex Sets of $n$ terminals, where $f(n) = \OO (\log n)$.

We state the following proposition from Theorem 1 in~\cite{scott1988convexity}:
\begin{proposition}\label{prop:tree_interval}
    Let $\C{P}$ be the vertices of any polygon in the plane, $\C{K}$ a subset of $\C{P}$, and $\C{T}$ a tree consisting of all the vertices of $\C{K}$ (and possibly some other vertices as well) and contained entirely inside $\C{P}$. Then on removing any edge of the tree, we get two disjoint trees $\mathcal{T}_1, \mathcal{T}_2$, such that the vertices of $\C{K}$ in each tree $\mathcal{T}_i, i \in \{1,2\}$ form an interval in $\C{K}$.
\end{proposition}

Using~\Cref{prop:tree_interval} and the Dreyfus-Wagner algorithm~\cite{dreyfus1971steiner}, we give an algorithm for obtaining the SMT of a planar graph $G$. 

Let $\C{K}$ represent the set of terminals lying on the outer face of $G$ and $\C{R}$ be the set of terminals lying inside the outer face of $G$. We have $|V(G)|=N$, $|\C{K} \cup \C{R}|=n$, and $|\C{K}|=k$. Let $C(\C{L})$ denote the SMT in $G$ for a terminal subset $\C{L} \subseteq V(G)$. Let $B(v,\C{L},[a,b))$ denote the SMT in $G$ for the terminal set $\{v\} \cup \C{L} \cup [a,b)$, where $\C{L} \subseteq \C{R}$, $[a,b)$ is the set of vertices in $\C{K}$ forming an interval from vertex $a$ to $b$ in counterclockwise direction along the outer boundary of $G$ including $a$ but excluding $b$, $v \in V(G) \setminus (\C{L} \cup [a,b))$, and the degree of $v$ is at least $2$ in $B(v,\C{L},[a,b))$. Let $A(v,\C{L},[a,b))$ denote the SMT in $G$ for the terminal set $\{v\} \cup \C{L} \cup [a,b)$, where $\C{L}$, $[a,b)$, and $v$ are as defined in the previous case, and the degree of $v$ is at most $1$ in $A(v,\C{L},[a,b))$.

Splitting the SMT at a vertex $v$ of degree at least $2$ gives rise to two smaller instances of the {\sc Steiner Minimal Tree} problem on graphs.
% \begin{equation}\label{eq:dw_B}
%     B(v,\C{L},[a,b)) = \min_{\C{L}' \subseteq \C{L}, x \in [a,b], \emptyset \subset \C{L}' \cup \{[a,x)\} \subset \C{L} \cup \{[a,b)\}} \{C(\{v\} \cup \C{L}' \cup \{[a,x)\}) + C(\{v\} \cup (\C{L}\setminus \C{L}') \cup \{[x,b)\}) \}
% \end{equation}

\begin{equation}\label{eq:dw_B}
    B(v,\C{L},[a,b)) = \min_{\Pi_1, \Pi_2, \Pi_3} \{C(\{v\} \cup \C{L}' \cup [a,x)) + C(\{v\} \cup (\C{L}\setminus \C{L}') \cup [x,b)) \}
\end{equation} where the conditions on $\C{L}'$ and $x$ are $\Pi_1: \C{L}' \subseteq \C{L}$, $\Pi_2: x \in \mathcal{K}, a< x <b$, and $\Pi_3: \emptyset \subset \C{L}' \cup [a,x) \subset \C{L} \cup [a,b)$.

The intuition is to root the tree at an internal terminal vertex and start growing the Steiner tree from there. Observe that on removing one of the internal vertices $v$ in the tree $\C{T}$, we get one, two or three disjoint subtrees. They induce a partition over the terminals. The terminals in $\C{K}$ in each of the subtrees form intervals in $\mathcal{K}$, according to~\Cref{prop:tree_interval}. Moreover, the terminals in $\C{R}$ can be partitioned in any way, not necessarily maintaining the interval structure. This is captured in the following recurrence relation:
\begin{equation}\label{eq:dw_C}
    C(\{v\} \cup \C{L} \cup [a,b)) = \min_{\Pi_1, \Pi_2, \Pi_3}\{A(v,\C{L}_1,[a,c))+A(v,\C{L}_2,[c,d))+A(v,\C{L}_3,[d,b))\}
\end{equation}

where the conditions on $\C{L}_1$, $\C{L}_2$, $\C{L}_3$, $c$, and $d$ are $\Pi_1: \C{L}_1,\C{L}_2,\C{L}_3 \subseteq \C{L}$, $\Pi_2: \C{L}_1 \cup \C{L}_2 \cup \C{L}_3 = \C{L}$, and $\Pi_3: c,d \in \mathcal{K}, a\leq c \leq d \leq b$ and we have 
\begin{equation}\label{eq:dw_A}
    A(v,\C{L}',[p,q))=\min\{\min_{u \notin \C{L}'}\{B(u,\C{L}',[p,q))+d(u,v)\},\min_{u \in \C{L}' \cup [p,q)}\{C(\C{L}' \cup [p,q))+d(u,v)\}\}
\end{equation}

Our aim is to compute $C(\{v\} \cup (\C{R}\setminus \{v\}) \cup \C{K})$, where $v \in \C{R}$. We precompute the shortest distance between all pairs of vertices. We then compute the values of $C(.)$ and $B(.)$ in increasing order of cardinality of subsets of vertices in $\C{K}$ and $\C{R}$. Let $d(u,v)$ denote the shortest path length between $u$ and $v$. The base cases are $C(\{v\} \cup \{a\}) = d(v,a)$ for all $v \in V(G)$ and $a \in \C{K} \cup \C{R}$.

\begin{algorithm}[H]
\caption{Computation of SMT of planar graph $G$ with terminal set $\C{K} \cup \C{R}$ ~~~ \textbf{Input:} $G$, $\C{K}$, $\C{R}$}\label{alg:smt_planar}
\begin{algorithmic}[1]
\State Compute the shortest distance between all pairs of vertices
\For{all $u \in V(G)$ and $a \in K \cup R$} 
\State Set $C(\{u\} \cup \{a\}) = d(u,a)$
\EndFor
\State Select a vertex $v \in \C{R}$
\For{$i = 1, \ldots, n-k-1$}
\For{each $\C{L} \subseteq \C{R}\setminus \{v\}$ of size $i$}
\For{$j = 1, \ldots, k$}
\If{j=k}
\State Compute $B(v,\C{L},\C{K})$ using~\Cref{eq:dw_B}
\State Compute $C(\{v\} \cup \C{L} \cup \C{K})$ using~\Cref{eq:dw_C}
\Else
\For{each $[a,b) \subseteq \C{K}$ of size $j$}
\State Compute $B(v,\C{L},[a,b))$ using~\Cref{eq:dw_B}
\State Compute $C(\{v\} \cup \C{L} \cup [a,b))$ using~\Cref{eq:dw_C}
\EndFor
\EndIf
\EndFor
\EndFor
\EndFor
\State \Return $C(\{v\} \cup (\C{R}\setminus \{v\}) \cup \C{K})$
\end{algorithmic}
\end{algorithm}

We analyse the correctness and running time of~\Cref{alg:smt_planar}.
    \begin{theorem} \label{thm:algo1_correct}
     Consider a planar graph $G$ on $N$ vertices and a set $\mathcal{K} \uplus \mathcal{R} \subseteq V(G)$ of $n$ terminals such that $\mathcal{K}$ is defined as the terminals lying on the outer face of $G$. Moreover, let $|\mathcal{K}| = k$. Then~\Cref{alg:smt_planar} computes the SMT for $\mathcal{K} \uplus \mathcal{R}$ in $G$ in time $\OO(N^2k^44^{n-k} + Nk^33^{n-k}+N^3)$.
    \end{theorem}
    \begin{proof}
{\bf Correctness of~\Cref{alg:smt_planar}.} 
In order to prove the correctness of~\Cref{alg:smt_planar}, we need to show that the~\Cref{eq:dw_B,eq:dw_C} are valid.

        In~\Cref{eq:dw_B}, $B(v,\C{L},[a,b))$ denotes an SMT for the terminal set $\{v\} \cup \C{L} \cup [a,b)$, conditioned on the fact that the degree of $v$ is at least $2$ in it. Let us split the SMT at vertex $v$ into two smaller subtrees. This must also split the terminals in $[a,b)$ in two intervals $[a,x)$ and $[x,b)$, respectively. Otherwise it would mean that the SMT has crossing edges, which is not possible. The vertices in $\mathcal{L}$ can be present in any of the two subtrees, hence we consider all possible partitions of $\C{L}$ into two subsets $\C{L}'$ and $\C{L} \setminus \C{L}'$. Thus for~\Cref{eq:dw_B}, $\mbox{LHS} \geq \mbox{RHS}$. On the other hand, the expression in the RHS of~\Cref{eq:dw_B} is a tree containing the vertex subset $\{v\} \cup \mathcal{L} \cup [a,b)$. Since, $B(v,\mathcal{L},[a,b))$ is an SMT for the same vertex subset, in~\Cref{eq:dw_B}  $\mbox{LHS} \geq \mbox{RHS}$. Therefore,~\Cref{eq:dw_B} is valid.

        For~\Cref{eq:dw_C}, we take $v$ as the root of the SMT $C(\{v\} \cup \C{L} \cup [a,b))$. The degree of $v$ in the SMT can be $1$, $2$, or $3$. Accordingly, on removing $v$, we will get $1$, $2$, or $3$ subtrees, with the condition that in each subtree the degree of $v$ is $1$. Again, the terminals in $[a,b)$ are divided into smaller intervals $[a,c)$, $[c,d)$ and $[d,b)$ for some $c,d \in \mathcal{K}$ satisfying $a<c<d<b$. The terminals in $\C{L}$ are divided among the subtrees in any combination. The number of such intervals and partitions is equal to the degree of $v$ in $C(\{v\} \cup \C{L} \cup [a,b))$. The term $A(v,\C{L}',[p,q))$ is obtained by minimizing across all SMTs satisfying the condition that degree of $v$ in the SMT is $1$. Thus in~\Cref{eq:dw_C}, $\mbox{LHS} \geq \mbox{RHS}$. On the other hand, the RHS of~\Cref{eq:dw_C} given a tree containing vertices $\{v\} \cup \mathcal{L} \cup [a,b)$. Since $C(\{v\} \cup \mathcal{L} \cup [a,b))$ is an SMT on the terminal set $\{v\} \cup \mathcal{L} \cup [a,b)$, in~\Cref{eq:dw_C} $\mbox{LHS} \leq \mbox{RHS}$. Thus,~\Cref{eq:dw_C} is valid.

{\bf Running time of~\Cref{alg:smt_planar}.}
All pairs shortest paths can be calculated in $\OO(N^3)$ time. The time complexity of the dynamic program has two components to it. One is due to computation of the B(.) values using~\Cref{eq:dw_B} and the other is for calculating the C(.) values using~\Cref{eq:dw_C}.
    \begin{enumerate}
        \item The number of computational steps for calculating $B(v,\mathcal{L},[a,b))$ using~\Cref{eq:dw_B} is of the order of the number of choices of $v$, $\mathcal{L}$, $\mathcal{L}'$, $[a,b)$, and $[a,x)$ such that $\mathcal{L} \subset \mathcal{R}$, $\mathcal{L}' \subseteq \mathcal{L}$, $a,x,b \in \mathcal{K}$, $a \leq x \leq b$, and $v \in V(G) \setminus (L \cup [a,b))$. Each vertex in $\mathcal{R}$ belongs to exactly one of the sets $\mathcal{L}'$, $\mathcal{L}\setminus \mathcal{L}'$, or $V(G) \setminus \mathcal{L}$. The vertices in $\mathcal{K}$ are partitioned into three intervals, $[a,x)$, $[x,b)$, and $[b,a)$. There are at most $N$ possibilities for $v$. This gives us a running time of $\OO(Nk^3 3^{n-k})$.
        \item The number of computational steps for calculating $C(\{v\} \cup \mathcal{L} \cup [a,b))$ using~\Cref{eq:dw_C} is $3N$ times the order of the number of choices of $v$, $\mathcal{L}$, $\mathcal{L}_1$, $\mathcal{L}_2$, $a$, $b$, $c$, and $d$ such that $\mathcal{L} \subset \mathcal{R}$, $\mathcal{L}_1 \subseteq \mathcal{L}$, $\mathcal{L}_2 \subseteq \mathcal{L}$, $\{a,c,d,b\} \subseteq \mathcal{K}$, $a \leq c \leq d \leq b$, and $v \in V(G) \setminus (\mathcal{L} \cup [a,b))$. Each vertex in $\mathcal{R}$ belongs to exactly one of the sets $\mathcal{L}_1$, $\mathcal{L}_2$, $\mathcal{L}_3$ or $V(G) \setminus \mathcal{L}$. The vertices in $\mathcal{K}$ are partitioned into at most four intervals, $[a,c)$, $[c,d)$, $[d,b)$ and $[b,a)$. There are at most $N$ possibilities for $v$. The $3N$ factor is because calculating each of the $C(.)$ values involves minimization over at most $3N$ terms. This gives us a running time of $\OO(N^2k^4 4^{n-k})$.
    \end{enumerate}
    Thus, the time complexity of the algorithm is $\OO(N^2k^44^{n-k}+Nk^33^{n-k}+N^3)$.
\end{proof}
We obtain the following corollary from the above theorem.

\begin{corollary}\label{graph-correct}
 Consider a planar graph $G$ on $N$ vertices and a set $\mathcal{K} \uplus \mathcal{R} \subseteq V(G)$ of $n$ terminals such that $\mathcal{K}$ is defined by the terminals lying on the outer face of $G$. Moreover, let $|\mathcal{K}| = k$ and let $|\C{R}| = n-k = \OO(\log n)$. Then~\Cref{alg:smt_planar} computes the SMT for $\mathcal{K} \uplus \mathcal{R}$ in $G$ in time $N^3k^4n^{\OO(1)}$.
\end{corollary}

Next we state the FPTAS for \ESMT on $f(n)$-Almost Convex Sets of $n$  terminals. This is achieved by converting the instance of \ESMT into an instance of {\sc Steiner Minimal Tree} on graphs. The {\sc Steiner Minimal Tree} problem shall be solved using~\Cref{alg:smt_planar}. For this, we use the Algorithm 2 in~\cite{scott1988convexity}. We restate the Algorithm 2 in~\cite{scott1988convexity} for our problem instance. We denote the set of terminals with $\C{P}$.

\begin{algorithm}[H]
\caption{Computation of $(1+\epsilon)$-approximate SMT of $\C{P}$ ~~~ \textbf{Input:} $\mathcal P, \epsilon$ }\label{alg:smt_fptas}
\begin{algorithmic}[1]
\State Compute the convex hull of the set of terminals $\C{P}$. Let the region enclosed by the convex hull $\mathrm{CH}(\C{P})$ be denoted by $\mathbb{R}_{\mathrm{CH}(\C{P})}$. Let the points in $\C{P}$ lying on $\mathrm{CH}(\C{P})$ be $\C{K}$ and $\C{R} = \C{P} \setminus \C{K}$.
\State We enclose the set of terminals $\C{P}$ with the smallest axis-parallel bounding square. Let its side length be $D$. We divide the bounding square into same sized grids of side length $\frac{D\epsilon}{8n-12}$, where $\epsilon$ is the approximation factor.
\State Let $\C{V}_0$ be the set of all lattice points introduced in the previous step, and $\C{V}_1$ be the set of all lattice points lying on the edges of $\mathrm{CH}(\C{P})$. We define the weighted graph $G_{f,\epsilon}$ to be the complete graph with vertex set $V(G_{f,\epsilon}) = \C{K} \cup \C{R} \cup (\C{V}_0 \cap \mathbb{R}_{\mathrm{CH}(\C{P})}) \cup \C{V}_1$. The edge weights are equal to the Euclidean distance between the two end points.
\State Return the SMT $\C{T}$ for the graph $G_{f,\epsilon}$ with $\C{K} \cup \C{R}$ as the terminal set using~\Cref{alg:smt_planar}. 
\end{algorithmic}
\end{algorithm}

We analyse the correctness and running time of~\Cref{alg:smt_fptas}.
    \begin{theorem} \label{thm:algo2_correct}
     Consider a set $\C{P}$ of $n$ points such that $\mathcal{K}$ is defined as the points lying on the convex hull of $\C{P}$, i.e. $\mathrm{CH}(\C{P})$, and $\C{R} = \C{P} \setminus \C{K}$. Moreover, let $|\mathcal{K}| = k$. Then~\Cref{alg:smt_fptas} computes a $(1+\epsilon)$-approximate SMT for $\mathcal{P}$ in time $\OO(\frac{n^4k^4}{\epsilon ^4}4^{n-k})$.
    \end{theorem}
    \begin{proof}
{\bf Correctness of~\Cref{alg:smt_fptas}.} 
    In order to prove the correctness of~\Cref{alg:smt_fptas}, we need to show that $\C{T}$ is a $(1+\epsilon)$-approximation of the SMT of the terminal set $\C{P}$, and $\C{T}$ is indeed the SMT for $\C{K} \cup \C{R}$ in the complete weighted graph $G_{f,\epsilon}$.

        In~\cite{scott1988convexity}, the concept of weight planar graphs is used. A graph $G$ is called weight planar if it is a non-planar graph embedded on the Euclidean plane, having non-negative edge weights, such that every pair of edges $(u,v)$ and $(u',v')$ which intersect in this embedding of $G$, satisfy the inequality: $w(u,v) + w(u',v') > d(u,u') + d(v,v')$, where $w(u,v)$ is the weight of the edge between vertices $u$ and $v$ and $d(x,y)$ is the length of the shortest path between vertices $x$ and $y$. Since the edge weights of $G_{f,\epsilon}$ are the Euclidean distances between the points, $G_{f,\epsilon}$ is a weight planar graph. 

        From Theorem 5 of \cite{scott1988convexity}, we get that the SMT of a weight planar graph does not contain any crossing edges even though the input graph is non-planar. Because the SMT does not contain any crossing edges and lies completely inside the convex hull of the terminal pointset, the terminals on the outer boundary of $G_{f,\epsilon}$, i.e. $\C{K}$, follow the interval pattern as stated in~\Cref{prop:tree_interval}. Therefore,~\Cref{alg:smt_planar} designed for planar graphs can be applied in the case of weight planar graphs as well. So, $\C{T}$ is the SMT for $\C{K} \cup \C{R}$ in the complete weighted graph $G_{f,\epsilon}$.

        Finally, from Theorem 12 in~\cite{scott1988convexity}, we get that the length of the Steiner tree obtained from~\Cref{alg:smt_fptas} is at most $(1+\epsilon)$ times the length of the SMT $\C{T}^*$ of $\C{P}$, i.e. $|\C{T}| \leq (1+\epsilon)|\C{T}^*|$. Thus, we are done.
        
{\bf Running time of~\Cref{alg:smt_fptas}.}
    Constructing the convex hull takes $\OO(n\log n)$ time. The number of lattice points contained in the bounding box is $\big(\frac{8n-12}{\epsilon}\big)^2 = \OO(\frac{n^2}{\epsilon ^2})$. Thus, the number of vertices in the resultant graph $G_{f,\epsilon}$ is $N = \OO(\frac{n^2}{\epsilon ^2}) + n = \OO(\frac{n^2}{\epsilon ^2})$. The time complexity of~\Cref{alg:smt_planar} is $\OO(N^2k^44^{n-k}+Nk^33^{n-k}+N^3) = \OO(\frac{n^4k^4}{\epsilon ^4}4^{n-k})$. This step dominates the running time resulting in the complexity of~\Cref{alg:smt_fptas} being $\OO(\frac{n^4k^4}{\epsilon ^4}4^{n-k})$.
    %If $n-k=\OO(\log n)$, the running time becomes polynomial in $n$ and $k$. $k$ can be bounded by $n$, hence the time complexity becomes polynomial in $n$.
    \end{proof}

\begin{theorem}\label{thm:esmt_fptas}
    There exists an FPTAS for \ESMT on an $f(n)$-Almost Convex Set of $n$ terminals, where $f(n) = \OO (\log n)$.
\end{theorem}

\begin{proof}
    From~\Cref{thm:algo2_correct}, we get a $(1+\epsilon)$-approximate SMT for any $(n-k)$-Almost Convex Set of $n$ terminals in time $\OO(\frac{n^4k^4}{\epsilon ^4}4^{n-k})$. For $n-k = \OO(\log n)$, we get the running time of~\Cref{alg:smt_fptas} to be $\OO(\frac{n^4k^4}{\epsilon ^4}n^{\OO(1)})$. Thus,~\Cref{alg:smt_fptas} is an FPTAS for the Euclidean Steiner Minimal Tree problem on an $\OO (\log n)$-Almost Convex Set of $n$ terminals. 
\end{proof}

\subsection{Hardness of Approximation for \ESMT on Cases of Almost Convex Sets}\label{subsec:apx_hardness}

In this section, we consider the \ESMT problem on $f(n)$-Almost Convex Sets of $n$ terminal points, where $f(n) =\Omega(n^\epsilon)$ for some $\epsilon \in (0,1]$. We show that this problem cannot have an FPTAS. The proof strategy is similar to that in~\cite{garey1977complexity}. First, we give a reduction for the problem \textsc{Exact Cover by $3$-Sets} (defined below) to our problem to show that our problem is NP-hard. Next, we consider a discrete version of our problem and reduce our problem to the discrete version. The discrete version is in NP. Therefore, this chain of reductions imply that the discrete version of our problem is Strongly NP-complete and therefore cannot have an FPTAS, following from~\cite{garey1977complexity}. Similar to the arguments in~\cite{garey1977complexity}, this also implies that our problem cannot have an FPTAS.

Before we describe our reductions, we take a look at the NP-hardness reduction of the \ESMT problem from the \textsc{Exact Cover by 3-Sets} (X3C) problem in~\cite{garey1977complexity}. In the X3C problem, we are given a universe of elements $U = \{1, 2, \ldots, 3n\}$ and a family $\mathbb{F}$ of $3$-element subsets $F_1, F_2, \ldots, F_t$ of the $3n$ elements. The objective is to decide if there exists a subcollection $\mathbb{F}' \subseteq \mathbb{F}$ such that: (i) the elements of $\mathbb{F}'$ are disjoint, and (ii) $\bigcup_{F' \in \mathbb{F}'} F' = U$. The X3C problem is NP-complete~\cite{garey1979computers}.

In~\cite{garey1977complexity}, various gadgets are constructed, i.e.~particular arrangements of a set of points. These are then arranged on the plane in a way corresponding to the given X3C instance.~\Cref{fig:redcn} shows the reduced ESMT instance obtained for $U = \{1,2,3,4,5,6\}$ and $\mathbb{F} = \{\{1, 2, 4\}, \{2, 3, 6\}, \{3, 5, 6\}\}$ (taken from~\cite{garey1977complexity}). The squares, hexagons (crossovers), shaded circles (terminators) and lines (rows) all represent specific arrangements of a subset of points. Let $X(\mathbb{F})$ denote the reduced instance. The number of points in $X(\mathbb{F})$ is bounded by a polynomial in $n$ and $t$. Let this polynomial be $\OO(t^\gamma)$, as we can assume $t \geq n$ since otherwise it trivially becomes a NO instance. Here $\gamma$ is some constant.

\begin{figure}[h]
\centering
\subfloat{\includegraphics[width=12cm]{reduction_X3C_ESMT.png}}
\caption{Reduced instance of ESMT from X3C (taken from~\cite{garey1977complexity})}
\label{fig:redcn}
\end{figure}

We restate Theorem 1 in~\cite{garey1977complexity}.
\begin{proposition}\label{thm:redcn}
    Let $\mathcal{S}^{*}$ denote an SMT of $X(\mathbb{F})$, the instance obtained by reducing the X3C instance $(n,\mathbb{F})$, and $|\mathcal{S}^{*}|$ denote its length. If $\mathbb{F}$ has an exact cover, then $|\mathcal{S}^{*}| \leq f(n,t,\hat{C})$, otherwise $|\mathcal{S}^{*}| \geq f(n,t,\hat{C}) + \frac{1}{200nt}$, where $t = |\mathbb{F}|$, $\hat{C}$ is the number of crossovers, i.e.~hexagonal gadgets, and $f$ is a positive real-valued function of $n,t,\hat{C}$.
\end{proposition}

We extend this construction to prove NP-hardness for instances of \ESMT where the terminal set $\mathcal{P}$ has $\Omega(n^\epsilon)$ points inside $\mathrm{CH}(\mathcal{P})$. Here, $\epsilon \in (0,1]$ and $n$ is the number of terminals.

Let us call the \emph{length} of a gadget to be the maximum horizontal distance between any two points in that gadget. Similarly, we define the \emph{breadth} of a gadget to be the maximum vertical distance between any two points in that gadget.

% We get the following lemma from the construction given in~\cite{garey1977complexity}.
% \begin{lemma}\label{lem:bbox}
%     The smallest rectangle bounding all the points in $X(\mathbb{F})$ has side lengths of $\OO(t)$.
% \end{lemma}

% \begin{proof}
%     From the construction of the gadgets in~\cite{garey1977complexity}, we know that all of them have length and breadth as a function of $\epsilon = \frac{1}{200nt}$, which is bounded between 0 and 1. Thus, the length and breadth of each gadget can be bounded by some constant.

%     According to the reduction, the maximum number of gadgets along any horizontal level is $\OO(t)$. Thus, the length of the smallest bounding rectangle is $c_1t = \OO(t)$. Moreover, the maximum number of gadgets along any vertical column is $\OO(n) = \OO(t)$ as $t \geq n$. Thus, the breadth of the smallest bounding rectangle is $c_2t = \OO(t)$.
% \end{proof}

\begin{figure}[h] 
\centering
\subfloat[\centering The Upward Terminator symbol]{\includegraphics[width=1.65cm]{terminator_symbol.png}}   \qquad\qquad
\subfloat[\centering The Downward Terminator symbol]{\includegraphics[width=1.5cm]{terminator_down.png}}
    \qquad\qquad
\subfloat[\centering The Terminator gadget] {\includegraphics[width=5cm]{terminator_gadget.png} }
\caption{The Terminator gadget symbol and arrangement of points}  
\label{fig:terminator}
\end{figure}

The \emph{terminator} gadget used is shown in~\Cref{fig:terminator}. The straight lines represent a row of at least $1000$ points separated at distances of $1/10$ or $1/11$. The angles between them are as shown. The upward terminator has the point $A$ above the other points in the terminator, whereas the downward terminator has the point $A$ below the other points. Firstly, we adjust the number of points in the long rows, such that the length and breadth of the terminators is same as that of the hexagonal gadgets (crossovers). We can fix this length and breadth to be some constants, such that the number of points in each gadget is also bounded by some constant. In our construction, we modify the terminators $\Omega_0$, $\Omega_1$, and $\Omega_2$ as shown in~\Cref{fig:redcn} enclosed in squares. $\Omega_1$ is the terminator corresponding to the first occurrence of the element $3n\in U$ in some set in $\mathbb{F}$ and $\Omega_2$ is the terminator corresponding to the last occurrence of $3n$ in some set in $\mathbb{F}$ (if there are more than one occurrences of $3n$). If there are no occurrences of $3n$, then it is trivially a no-instance. The modified gadgets are shown in~\Cref{fig:terminator_new}. All the other gadgets remain unaltered.

\begin{figure}[h]
\centering
\includegraphics[width=8cm]{conic_set.png}
\caption{Conic Set: $\mathrm{Cone}(T,r,n)$}
\label{fig:conic_set}
\end{figure}

We call a set of points arranged as shown in~\Cref{fig:conic_set}, as a Conic Set.
\begin{definition} \label{def:conic_set}
    A Conic Set is a set of points consisting of a point $T$, called the tip of the cone, and the remaining points denoted by $\mathcal{S}$. Let $\mathcal{C}$ be the circle with $T$ as centre and radius $r$. All the points in $\mathcal{S}$ lie on $\mathcal{C}$, such that the angle at the tip formed by the two extreme points $L,R \in \mathcal{S}$, i.e.~$\angle{LTR} = 30^\circ$ in the anticlockwise direction. So, we have $\overline{TL} = \overline{TR} = r$. The distance between any two consecutive points in $\mathcal{S}$ is the same, say $d$. Let the number of points in $\mathcal{S}$ be $n$. We denote the Conic Set as $\mathrm{Cone}(T,r,n)$ and $\mathcal{S}$ as $\mathrm{Circ}(T,r,n)$. We call $TL$ as the left slope of the Conic Set and $TR$ as the right slope of the Conic Set.
\end{definition}

\begin{figure}[h] 
\centering
\subfloat[\centering $\Omega'_0$]{\includegraphics[width=5cm]{terminator_up_new.png}}   \qquad\qquad\qquad\qquad\qquad\qquad
\subfloat[\centering $\Omega'_1$]{\includegraphics[width=5cm]{omega_1.png}}
    \qquad\qquad
\subfloat[\centering $\Omega'_2$] {\includegraphics[width=5cm]{omega_2.png} }
\caption{The modified terminator gadgets}  
\label{fig:terminator_new}
\end{figure}


We use the Conic Set in the reduction for our problem. Now, we state the reduction of an X3C instance $(n,\mathbb{F})$ to an instance $X'(\mathbb{F},\epsilon)$ of \ESMT. Later, we show that the instance will satisfy the desired properties on the number of terminals inside the convex hull of the terminal set.

\begin{description}
\item [Algorithm $\mathcal{A}$ for construction of an ESMT instance $X'(\mathbb{F},\epsilon)$ from an X3C instance $(n,\mathbb{F})$:]
\hspace{0.1cm}
\begin{itemize}
    \item Reduce the input X3C instance to the points configuration $X(\mathbb{F})$ according to the reduction given in~\cite{garey1977complexity}.
    \item Modify the terminators $\Omega_0$, $\Omega_1$, and $\Omega_2$ to as shown in~\Cref{fig:terminator_new} and call them $\Omega'_0$, $\Omega'_1$, and $\Omega'_2$. Let $DQCP$ be the smallest axis-parallel rectangle bounding $X(\mathbb{F})$ after modifying the terminators, where $D$ is the bottom leftmost point of $\Omega'_0$.
    \item Take $\alpha = \frac{1}{\epsilon}$. Define $r = ct^{\alpha} = \OO(t^{\alpha})$ and $n' = c't^{\gamma\alpha} = \OO(t^{\gamma\alpha})$, where $t=|\mathbb{F}|$ and $c$ and $c'$ are constants. Add the $\mathrm{Cone}(D,r,n')$, such that $D$ is the tip of the Conic Set, and the left slope $DE$ makes an angle of $120^\circ$ with $DP$. The right slope $DF$ also makes an angle of $120^\circ$ with $DQ$.
    % \item Add the point $E$ on $\mathcal{C}$, such that $\angle{PDE} = 120^\circ$. Add the point $F$ such that $F$ lies on $\mathcal{C}$ and $\angle{EDF} = 30^\circ$ in anticlockwise direction.
    % \item Add $c't^{\gamma\alpha} = \OO(t^{\gamma\alpha})$ points on $\mathcal{C}$, where $c'$ is a constant, such that the distance between any two consecutive points is the same. Let us call these set of points on $\mathcal{C}$ along with $E$ and $F$ as $S$.
\end{itemize}
\end{description}

\begin{figure}[h]
\centering
\includegraphics[width=10cm]{redn_instance.png}
\caption{The reduced instance $X'(\mathbb{F},\epsilon)$}
\label{fig:redn_instance}
\end{figure}

% \begin{description}
% \item [Construction of $X'(\mathbb{F},\epsilon)$:]
% \hspace{10cm}
% \begin{itemize}
%     \item Reduce the input X3C instance to the point configuration $X(\mathbb{F})$ according to the reduction given in~\cite{garey1977complexity}.
%     \item Modify the terminator $\Omega_0$ to as shown in~\Cref{fig:terminator_new} and call it $\Omega'_0$.
%     \item Add a point $e$ at a distance of $r - D$ from the point $d$ of $\Omega'_0$ in the direction of the ray $\overrightarrow{\rm od}$. Add points on the line segment $\overline{\rm de}$ such that consecutive points are at a distance of $\frac{1}{10}$ from each other.
%     \item Take $\alpha = \frac{1}{\epsilon}$. From $e$, take two points $f$ and $g$, such that $\angle{\rm feg} = 120\degree$, $\overrightarrow{\rm oe}$ bisects $\angle{\rm feg}$ and $\overline{\rm fe} = \overline{\rm ge} = c't^{1-2\alpha}$. Consider the point $C$ at a distance of $D$ from the point $d$ of $\Omega'_0$ in the direction of the ray $\overrightarrow{\rm do}$. Let $R = \overline{\rm Cf} = \OO(t + t^{1-2\alpha}) = \OO(t)$. Taking $C$ as the centre, draw a circle of radius $R$. Let the circle be $\mathcal{T}$.
%     \item Add vertices on $\mathcal{T}$ such that the distance between two consecutive points is same as $\overline{\rm fg}$.
% \end{itemize}
% \end{description}

Now we prove a few properties of the constructed instance $X'(\mathbb{F},\epsilon)$.
\begin{lemma}\label{lem:convhull_pts}
    All the points in $\mathrm{Circ}(D,r,n')$ (according to~\Cref{def:conic_set}) lie on the convex hull of the reduced ESMT instance $X'(\mathbb{F},\epsilon)$ constructed by Algorithm $\mathcal{A}$, where $\epsilon \in (0,1]$.
\end{lemma}

\begin{proof}
    By the construction of $\mathrm{Cone}(D,r,n')$ in Algorithm $\mathcal{A}$, let $\mathcal{C}$ be the circle on which all the points in $\mathrm{Circ}(D,r,n')$ lie. If we draw a tangent to $\mathcal{C}$ at any of the points in $\mathrm{Circ}(D,r,n')$, then all the remaining points in the configuration $X'(\mathbb{F},\epsilon)$ lie towards one side of the tangent. We know that if we can find a line passing through a point such that all the other points in the plane lie on one side of the line, then the point lies on the convex hull of the points in the plane. Therefore, all the points in $\mathrm{Circ}(D,r,n')$ lie on the convex hull of the reduced instance $X'(\mathbb{F},\epsilon)$.
\end{proof}

Let us denote the convex hull of $X'(\mathbb{F},\epsilon)$ by $\mathrm{CH}(X'(\mathbb{F},\epsilon))$ and that of the points lying inside or on the bounding rectangle $\mathrm{PDQC}$, i.e.~$X'(\mathbb{F},\epsilon)\setminus \mathrm{Circ}(D,r,n')$ by $\mathrm{CH}(X'(\mathbb{F},\epsilon)\setminus \mathrm{Circ}(D,r,n'))$.

\begin{lemma}\label{lem:redcn_size}
    The reduced ESMT instance $X'(\mathbb{F},\epsilon)$ constructed by Algorithm $\mathcal{A}$ has $\Omega (N^\epsilon)$ points inside the convex hull, where $\epsilon \in (0,1]$ and $N$ is the total number of terminals in $X'(\mathbb{F},\epsilon)$. 
\end{lemma}

\begin{proof}
    $\mathrm{CH}(X'(\mathbb{F},\epsilon))$ contains all the points in $\mathrm{Circ}(D,r,n')$ by~\Cref{lem:convhull_pts}. $\mathrm{Circ}(D,r,n')$ contains $n' = \OO(t^{\gamma\alpha})$ points.

    Now we need to analyze the number of points on $\mathrm{CH}(X'(\mathbb{F},\epsilon)\setminus \mathrm{Circ}(D,r,n'))$. The remaining points in $X(\mathbb{F})$, i.e.~$X(\mathbb{F})\setminus \mathrm{CH}(X'(\mathbb{F},\epsilon)\setminus \mathrm{Circ}(D,r,n'))$ must lie within the convex hull of the entire construction, i.e.~$X'(\mathbb{F},\epsilon)$. From the construction in Algorithm $\mathcal{A}$, no point on the connecting rows can be a part of $\mathrm{CH}(X'(\mathbb{F},\epsilon)\setminus \mathrm{Circ}(D,r,n'))$ as there is no line passing through it, which contains all terminals on one side of it. The same thing holds for the square and hexagonal gadgets as well, except the hexagonal gadgets corresponding to the last element of the last set in the family, i.e.~$F_t$. Thus, only the terminators and the hexagonal gadgets corresponding to the last element of $F_t$ contribute points to $\mathrm{CH}(X'(\mathbb{F},\epsilon)\setminus \mathrm{Circ}(D,r,n'))$.

    If we look at the arrangement of points in the terminators (modified as well as those left unchanged) and the hexagonal gadgets as shown in~\Cref{fig:terminator,fig:hexagon_points}, the convex hull of each of these gadgets consists of constantly many points. Therefore, the number of points each of these gadgets contribute to $\mathrm{CH}(X'(\mathbb{F},\epsilon)\setminus \mathrm{Circ}(D,r,n'))$ is bounded by some constant. The number of terminators is $6t+2$ and the number of hexagonal gadgets corresponding to the last element of $F_t$ is at most $3n$. Therefore, the number of points on $\mathrm{CH}(X'(\mathbb{F},\epsilon)\setminus \mathrm{Circ}(D,r,n'))$ is $\OO(t+n) = \OO(t)$ as $n \leq t$.
    
    The instance $X(\mathbb{F})$ obtained via reduction from X3C has $6t+2$ terminators, $t$ squares, at most $9nt$ crossovers (hexagonal gadgets), and $\OO(nt)$ connecting rows of points. The number of gadgets is $\OO(nt)$. Therefore, the total number of points in $X(\mathbb{F})$ is $\Omega(nt) = \omega(t)$. The modified terminators result in a constantly many increase in the number of points. So, we have $\gamma > 1$.

    Thus, the number of points inside the convex hull is $\Omega(t^\gamma)$ and those on the convex hull is $\OO(t^{\gamma\alpha})$. So, the total number of terminals is $N = \OO(t^{\gamma\alpha}) + \OO(t^{\gamma}) = \OO(t^{\gamma\alpha})$, and those inside the convex hull is $\Omega(t^\gamma) = \Omega(N^{1/\alpha}) = \Omega(N^\epsilon)$ as $\alpha = \frac{1}{\epsilon}$.
\end{proof}

\begin{figure}[h]
\centering
\includegraphics[width=5cm]{hexagon_points.png}
\caption{The hexagonal gadget (crossover), the convex hull of the gadget is the quadrilateral $\rm abcd$ (taken from~\cite{garey1977complexity})}
\label{fig:hexagon_points}
\end{figure}

% \begin{lemma}\label{lem:redcn_size}
%     The reduced instance $X'(\mathbb{F},\epsilon)$ has $\OO (N^\epsilon)$ points inside the convex hull, where $\epsilon \in (0,1]$ and $N$ is the total number of terminals. 
% \end{lemma}

% \begin{proof}
%     The convex hull of $X'(\mathbb{F},\epsilon)$ configuration of points is the convex polygon inscribed in the circle $\mathcal{T}$ of radius $R = \OO(t)$. Distance between consecutive points is $\OO(t^{1-2\alpha})$. So, the convex hull contains $\OO(t^{2\alpha})$ points.
    
%     The instance $X(\mathbb{F})$ obtained via reduction from X3C has $6t+2$ terminators, $t$ squares, at most $9nt$ crossovers (hexagonal gadgets), and $\OO(nt)$ connecting rows of points. Each of these gadgets contain constantly many points. Therefore, the total number of points in $X(\mathbb{F})$ is $\OO(nt) = \OO(t^2)$. The modified terminator results in a constantly many increase in the number of points. Addition of points on the line segment $\overline{\rm de}$ results in $\OO(t)$ increase in the number of points inside as length of $\overline{\rm de} = D = \OO(t)$ and every pair of consecutive points are at a distance of $\frac{1}{10}$ from each other.

%     Thus, the number of points inside the convex hull is $\OO(t^2)$ and those on the convex hull is $\OO(t^{2\alpha})$. So, the total number of terminals is $N = \OO(t^{2\alpha})$, and those inside the convex hull is $\OO(t^2) = \OO(N^{1/\alpha}) = \OO(N^\epsilon)$ as $\alpha = \frac{1}{\epsilon}$.
% \end{proof}
We further prove structural properties of SMTs of the reduced instance $X'(\mathbb{F},\epsilon)$ when considering the modified gadgets $\Omega'_0$, $\Omega'_1$, and $\Omega'_2$.
\begin{lemma}\label{lem:modified_smt}
    Consider an SMT $\mathcal{S}^*$ of the ESMT instance $X(\mathbb{F})$ obtained via reduction from the X3C instance $(n,\mathbb{F})$ as per~\cite{garey1977complexity}. Consider a tree $\mathcal{S'}^{*}$ on the terminal set of $X'(\mathbb{F},\epsilon)$ obtained from $\mathcal{S}^*$ as follows: Consider the modified terminator gadgets $\Omega'_i,~i \in \{0,1,2\}$ as in Algorithm $\mathcal{A}$. For each $i \in \{0,1,2\}$, the edge $B_iO_i$ is excluded from $\mathcal{S}^*$ and the edge $D_iO_i$ is included to form $\mathcal{S'}^{*}$. $\mathcal{S'}^{*}$ is an SMT for the terminal set of $X'(\mathbb{F},\epsilon)$.
\end{lemma}

\begin{proof}
    Consider $\mathcal{S}^*$. Due to Lemma 4 of~\cite{garey1977complexity}, Steiner points of $\mathcal{S}^*$ can only be connected to points in the triangular and square gadgets. Lemma 5 of~\cite{garey1977complexity} states that if there are two terminals $x,y \in X(\mathbb{F})$ and the distance between $x$ and $y$ does not exceed $\frac{1}{10}$, then $(x,y)$ is an edge of $\mathcal{S}^{*}$. So in $\mathcal{S}^{*}$, for each $i \in \{0,1,2\}$ all the points on $B_iO_i$ are joined together along $B_iO_i$. Lemma 5 of~\cite{garey1977complexity} also holds true on modifying the terminators to $\Omega'_i,~i \in \{0,1,2\}$. Now, we join all the points on $D_iO_i$ along $D_iO_i$. This gives us the SMT $\mathcal{S'}^*$ for the terminal set of $X'(\mathbb{F},\epsilon)$.
\end{proof}

%A similar extension works for the modification of the terminators $\Omega_1$ and $\Omega_2$. Let this new SMT be represented by ${\mathcal{S}'}^*$ and its length by $|{\mathcal{S}'}^*|$.

Now we focus on the structure of the SMT of $X'(\C{F},\epsilon)$. The SMT is basically the union of the SMT $\C{S'}^{*}$ of the points in the bounding rectangle $PDQC$ as stated in~\Cref{lem:modified_smt} and the SMT of the set of points $\mathrm{Cone}(D,r,n')$.

% \begin{lemma}\label{lem:bounding_quad}
%     The rectangle $\mathrm{pdqC}$ as shown in~\Cref{fig:bounding_quad} encloses all the points in $X'(\C{F},\epsilon)$ located inside the convex hull.
% \end{lemma}

% \begin{proof}
%     We know that $D = \max\big(\frac{2c_1t}{\sqrt{3}}, 2c_2t\big)$. So, $\overline{\rm dq} = D\cos{30^\circ} = \max\big(c_1t, \sqrt{3}c_2t\big) \geq c_1t$. Similarly, $\overline{\rm Cq} = D\sin{30^\circ} = \max\big(\frac{c_1t}{\sqrt{3}}, c_2t\big) \geq c_2t$. Thus, the smallest bounding rectangle of the points in $X(\C{F})$ lies completely inside the bounding rectangle. So, all points in the configuration located inside the convex hull are enclosed by the bounding rectangle $\mathrm{pdqC}$.
% \end{proof}

% \begin{lemma}\label{lem:steiner_point_in_circle}
%     For any two points in or on the boundary of the bounding rectangle $\mathrm{peqC}$ in~\Cref{fig:bounding_quad}, if a Steiner point is adjacent to both of them, then the Steiner point must be located inside the circle centred at $C$ and radius $r = 2D$.
% \end{lemma}

% \begin{proof}
    
% \end{proof}

% For any SMT of $X'(\C{F},\epsilon)$, we call a \emph{\textbf{connecting path}} as any path which starts from some vertex inside the bounding rectangle and ends at a vertex on the convex hull, such that all internal vertices are Steiner points. Our analysis proceeds similar to that of the pair of concentric parallel regular polygons having more than 12 sides. The following lemma is analogous to~\Cref{left_right_turn_path,mincut_1}. 

% \begin{lemma}\label{lem:singly_connected}
%     For any SMT of $X'(\C{F},\epsilon)$, there exists a counter-clockwise or clockwise connecting path having at least one edge of length at least $\frac{R-r}{2}$. Also, there exists only one edge-disjoint connecting path.
% \end{lemma}

% \begin{proof}
%     Firstly, it can be easily seen that~\Cref{left_right_turn_path} can be extended to prove the existence of at least one counter-clockwise or clockwise connecting path.
    
%     Let $H$ be arbitrary point on the Euclidean plane. If $\mathcal C$ be a counter-clockwise path starting from $H$ such that no edge in the counter clockwise path has a length of more than $\ell$, for some $\ell \in \mathbb{R}^{+}$. Then we know that $\mathcal C$ is contained entirely in the circle centred at $H$ with radius $2\ell$.

%     The distance between a point in or on the bounding rectangle and a point on the convex hull is at least $R-D = r \gg D$ because the bounding rectangle is completely contained inside the circle with centre $C$ and radius $D$. So, all counter-clockwise or clockwise connecting paths must contain an edge of length at least $\frac{R-D}{2} \gg D$. 
    
%     If there are more than one edge-disjoint connecting paths, then we can always remove the edge of length at least $\frac{R-D}{2}$ from some counter-clockwise or clockwise connecting path and add another edge to get a smaller tree.

%     Observe that in the configuration, the distance between any two points in the bounding box is at most $D$. Also, the distance between two consecutive points on the convex hull is $\OO{t^{1-2\alpha}}$. When we remove one such edge of length $\OO(t^{2\alpha})$, either all the vertices on the convex hull are not connected or all the vertices inside the hull are not connected. In both cases, we can replace it by an edge of length at most $2\pi c$ or $r = 2D = \OO(t)$, respectively. This gives us a smaller tree, thereby contradicting the minimality of a tree containing two or more edge-disjoint connecting paths.
% \end{proof}

% Here, a \emph{\textbf{singly connected}} topology stands for one in which all connecting paths have at least one edge in common, i.e. there is only one edge-disjoint connecting path. The SMT shown in~\Cref{fig:smt_new} is a singly connected topology.

$\mathrm{CH}(X'(\mathbb{F},\epsilon)\setminus \mathrm{Circ}(D,r,n'))$ is enclosed by the bounding rectangle $PDQC$ and $D$ must lie on $\mathrm{CH}(X'(\mathbb{F},\epsilon)\setminus \mathrm{Circ}(D,r,n'))$. We label the vertices of $\mathrm{CH}(X'(\mathbb{F},\epsilon)\setminus \mathrm{Circ}(D,r,n'))$ as $D,P_1,P_2,\ldots,P_k$ in the counter-clockwise order. Let $\mathrm{CH}(X'(\mathbb{F},\epsilon))$ be the convex hull of all the points. By~\Cref{lem:convhull_pts}, all the points in $\mathrm{Circ}(D,r,n')$ lie on $\mathrm{CH}(X'(\mathbb{F},\epsilon))$. Let $EP_i$ and $FP_j$ be edges in $\mathrm{CH}(X'(\mathbb{F},\epsilon))$, such that $P_i,P_j \notin \mathrm{Circ}(D,r,n')$.

The SMT of $X'(\mathbb{F},\epsilon)$ clearly lies inside its convex hull, $\mathrm{CH}(X'(\mathbb{F},\epsilon))$. We show that the Steiner hull can be further restricted to the bounding rectangle $PDQC$ and the convex polygon formed by the points in $\mathrm{Cone}(D,r,n')$. For this we use Theorem 1.5 in~\cite{hwang1992steiner}, as stated below.

\begin{proposition}~\cite{hwang1992steiner}\label{thm:steiner_hull}
    Let $H$ be a Steiner hull of $N$. By sequentially removing wedges $\rm abc$ from the remaining region, where $\rm a$, $\rm b$, $\rm c$ are terminals but $\triangle{\rm abc}$ contains no other terminal, $a$ and $c$ are on the boundary and $\angle{\rm abc} \geq 120^\circ$, a Steiner hull $H'$ invariant to the sequence of removal is obtained.
\end{proposition}

\begin{lemma}\label{thm:steiner_hull_regions}
    The region comprising of the bounding rectangle $PDQC$ according to Algorithm $\mathcal{A}$ and the convex polygon formed by the set of points $\mathrm{Cone}(D,r,n')$ is a Steiner hull of $X'(\mathbb{F},\epsilon)$.
\end{lemma}

\begin{proof}
    Firstly, let us consider the wedge $EP_{i+1}P_i$. All the points are terminals, $E$ and $P_i$ are boundary points, and $\triangle{EP_{i+1}P_i}$ contains no other terminal. Now, $\angle{EP_{i+1}P_i}$ is greater than the exterior angle of $\angle{EP_{i+1}D}$, which in turn is greater than $\angle{EDP_{i+1}}$. $\angle{EDP_{i+1}} \geq \angle{\rm EDP} = 120^\circ$, by the construction. Therefore, $\angle{EP_{i+1}P_i} \geq 120^\circ$. By applying~\Cref{thm:steiner_hull}, we can remove the wedge $EP_{i+1}P_i$ from the convex hull $\mathrm{CH}(X'(\mathbb{F},\epsilon))$ to get a smaller Steiner hull. This can be repeated for the wedges $EP_{i+2}P_{i+1}, EP_{i+3}P_{i+2}, \ldots, EDP_k$. The same argument can also be used to get rid of the wedges $FP_{j-1}P_j, FP_{j-2}P_{j-1}, \ldots, FDP_1$. So, we get the final Steiner hull $H'$ to be the union of the bounding rectangle $PDQC$ and the convex polygon formed by the points in $\mathrm{Cone}(D,r,n')$.
\end{proof}

Given the nature of the above Steiner hull, we show that we can treat $X(\mathbb{F})$ and $\mathrm{Cone}(D,r,n')$ separately. 
\begin{lemma}\label{thm:steiner_tree}
    There is an SMT of $X'(\mathbb{F},\epsilon)$ that is the union of an SMT of $X(\mathbb{F})$ and an SMT of the points in $\mathrm{Cone}(D,r,n')$, with $D$ being common to both of them.
\end{lemma}

\begin{proof}
    According to~\Cref{thm:steiner_hull_regions}, there is an SMT of $X'(\mathbb{F},\epsilon)$ that lies completely inside the the bounding quadrilateral $PDQC$ and the convex polygon formed by $\mathrm{Cone}(D,r,n')$. These two regions have $D$ as the only common point. Therefore, $D$ is an articulation point in the tree and connects these two regions. So, we have this SMT of $X'(\mathbb{F},\epsilon)$ as the union of an SMT of $X(\mathbb{F})$ and an SMT of the points in $\mathrm{Cone}(D,r,n')$.
\end{proof}

We can identify a structure for an SMT of the points in $\mathrm{Cone}(D,r,n')$ using~\cite{weng1995steiner}.
\begin{lemma}\label{thm:steiner_tree_part}
    There is an SMT of the points in $\mathrm{Cone}(D,r,n')$ that is as shown in~\Cref{fig:steiner_new}. In the SMT, $D$ is connected to the two middle points in $\mathrm{Circ}(D,r,n')$ via a Steiner point $S^t$. The other points in $\mathrm{Circ}(D,r,n')$ are connected along the circumference.
\end{lemma}

\begin{proof}
    The number of points in $\mathrm{Circ}(D,r,n')$ is $\OO(t^{\gamma\alpha})$. We can take the constant factor to be large enough so that $|\mathrm{Circ}(D,r,n')| >= 12$. If we complete the regular polygon on $\mathcal{C}$ having $\mathrm{Circ}(D,r,n')$ as a subset of its vertices, then it contains more than $12$ vertices and along with the centre $D$ has a SMT with structure given in~\cite{weng1995steiner}.

    Let the Steiner tree for $\mathrm{Cone}(D,r,n')$ as shown in~\Cref{fig:steiner_new} be denoted by $\mathcal{T}_1$. If this is not minimal, then there exists another Steiner tree $\mathcal{T}_2$ such that $|\mathcal{T}_2| < |\mathcal{T}_1|$. Then we can replace $\mathcal{T}_1$ by $\mathcal{T}_2$ in the SMT of the regular polygon and its centre to get a shorter Steiner tree. This contradicts the minimality of the structure given in~\cite{weng1995steiner}. Therefore, the SMT of $\mathrm{Cone}(D,r,n')$ follows the structure in~\Cref{fig:steiner_new}.
\end{proof}

\begin{figure}[h]
\centering
\includegraphics[width=8cm]{steiner_new.png}
\caption{SMT of $\mathrm{Cone}(D,r,n')$}
\label{fig:steiner_new}
\end{figure}
Finally, we prove the NP-hardness of \ESMT on $f(n)$-Almost Convex Sets of $n$ terminals, when $f(n) = \Omega(n^\epsilon)$ for some $\epsilon \in (0,1]$.
\begin{theorem}\label{thm:redn_final}
    Let $\C{S}^{*}_{\mathbb{F},\epsilon}$ denote an SMT of $X'(\mathbb{F},\epsilon)$ and $|\C{S}^*_{\mathbb{F},\epsilon}|$ denote its length. If $\mathbb{F}$ has an exact cover, then $|\C{S}^{*}_{\mathbb{F},\epsilon}| \leq f(n,t,\hat{C}) + |\mathcal{T}_1|$, otherwise $|\C{S}^{*}_{\mathbb{F},\epsilon}| \geq f(n,t,\hat{C}) + \frac{1}{200nt} + |\mathcal{T}_1|$, where $\hat{C}$ is the number of crossovers, i.e.~hexagonal gadgets, and $f$ is a positive real-valued function of $n,t,\hat{C}$ as stated in~\Cref{thm:redcn}.
\end{theorem}

\begin{proof}
    From~\Cref{thm:steiner_tree}, we have $|\C{S}^{*}_{\mathbb{F},\epsilon}| = |\C{S}^{*}| + |\mathcal{T}_1|$. From~\Cref{thm:steiner_tree_part}, we can compute the length of $\mathcal{T}_1$ as a function of $t$, $\alpha$, and $\gamma$. Finally, using~\Cref{thm:redcn} we get the required reduction.
\end{proof}

%Thus, the \ESMT problem on $f(n)$-Almost Convex Sets of $n$ terminals, when $f(n) = \Omega (n^\epsilon)$ and where $\epsilon \in (0,1]$, is NP-Hard.

% Strongly NP-Complete problems do not have an FPTAS.
% Therefore, we get the following result.

% \begin{theorem}\label{thm:no_fptas}
%     There does not exist any FPTAS for the ESMT problem on $f(n)$-Almost Convex Sets of $n$ terminals, where $f(n) = \Omega(n^\epsilon)$ and $\epsilon \in (0,1]$.
% \end{theorem}

Since it is not known if the ESMT problem is in NP, Garey et al.~\cite{garey1977complexity} show the NP-completeness of a related problem called the {\sc Discrete Euclidean Steiner Minimal Tree} (DESMT) problem, which is in NP. We define the DESMT problem as given in~\cite{garey1977complexity}. The DESMT problem takes as input a set $\C{X}$ of integer-coordinate points in the plane and a positive integer $L$, and asks if there exists a set $\C{Y} \supseteq \C{X}$ of integer-coordinate points such that some spanning tree $\C{T}$ for $\C{Y}$ satisfies $|\C{T}|_d \leq L$, where $|\C{T}|_d = \Sigma_{e \in E(\mathcal T)} \lceil\overline{e}\rceil$, i.e.~we round up the length of each edge to the least integer not less than it.

In order to show that DESMT is NP-hard, the same reduction as that of the ESMT problem can be used, followed by scaling and rounding the coordinates of the points. Theorem 4 of~\cite{garey1977complexity} proves that the DESMT problem is NP-Complete. Moreover, since it is Strongly NP-Complete, the DESMT problem does not admit any FPTAS. Finally in Theorem 5 of~\cite{garey1977complexity}, Garey et al. show that as a consequence, the ESMT problem does not have any FPTAS as well.

Now we show that the DESMT problem is NP-hard even on $f(n)$-Almost Convex Sets of $n$ terminals, when $f(n) = \Omega (n^\epsilon)$ and where $\epsilon \in (0,1]$.

In Section 7 of~\cite{garey1977complexity}, the reduced instance $X(\mathbb{F})$ of ESMT is converted into an instance $X_{d}(\mathbb{F})$ of DESMT. The conversion is as follows:\\
$X_{d}(\mathbb{F}) = \{(\lceil 12M\cdot 200nt\cdot x_1\rceil, \lceil 12M\cdot 200nt\cdot x_2\rceil): x=(x_1,x_2)\in X(\mathbb{F})\}$, where $M = |X(\mathbb{F})|$.

We apply a similar conversion to the reduced ESMT instance $X'(\mathbb{F},\epsilon)$, to convert it into a DESMT instance of an $\Omega(n^\epsilon)$-Almost Convex Set. The conversion goes as follows:\\
$X'_{d}(\mathbb{F},\epsilon) = \{(\lceil 12N\cdot 200nt\cdot x_1\rceil, \lceil 12N\cdot 200nt\cdot x_2\rceil): x=(x_1,x_2)\in X'(\mathbb{F},\epsilon)\}$, where $N = |X'(\mathbb{F},\epsilon)|$.

The next two lemmas establish the validity of $X'_{d}(\mathbb{F},\epsilon)$ as an instance of DESMT and the upper bounds on the size of the constructed instance. Note that the reduction from X3C followed by the conversion can be done in polynomial time.
\begin{lemma}\label{lem:valid_desmt}
    The instance $X'_{d}(\mathbb{F},\epsilon)$ constructed above is a valid DESMT instance.
\end{lemma}

\begin{proof}
    All the points in $X'_{d}(\mathbb{F},\epsilon)$ have integer coordinates according to the conversion stated above. So, it is a DESMT instance.
\end{proof}

\begin{lemma}\label{lem:numpts_same}
    The reduced DESMT instance $X'_{d}(\mathbb{F},\epsilon)$ has $N$ distinct points, where $N = |X'(\mathbb{F},\epsilon)|$.
\end{lemma}

\begin{proof}
    The minimum distance between any two points in $X'(\mathbb{F},\epsilon)$ is that between two consecutive points of $\mathrm{Circ}(D,r,n')$, which is $\OO(\frac{1}{t^\gamma})$. Recall~\Cref{lem:redcn_size} which establishes that $N = \OO(t^{\gamma\alpha})$. So, the minimum distance between any two points in $X'_{d}(\mathbb{F},\epsilon)$ is $\OO(N \cdot nt \cdot \frac{1}{t^\gamma}) = \OO(nt^{\alpha+1})$. Because of the substantial distance obtained between points after scaling, the rounding will not cause any distinct points of $X'_{d}(\mathbb{F},\epsilon)$ to coincide. Therefore, the number of points remains unchanged, i.e.~$|X'_{d}(\mathbb{F},\epsilon)| = |X'(\mathbb{F},\epsilon)| = N$.
\end{proof}

Now we present the following lemma for the constructed DESMT instance $X'_{d}(\mathbb{F},\epsilon)$ analogous to~\Cref{lem:redcn_size} for the ESMT instance $X'(\mathbb{F},\epsilon)$.

\begin{lemma}\label{lem:desmt_redn_size}
    The reduced DESMT instance $X'_{d}(\mathbb{F},\epsilon)$ constructed is an $\Omega(N^\epsilon)$-Almost Convex Set, where  $N = |X'_{d}(\mathbb{F},\epsilon)|$.
\end{lemma}

\begin{proof}
    From~\Cref{lem:redcn_size}, we know that the reduced ESMT instance $X'(\mathbb{F},\epsilon)$ has $\Omega(N^\epsilon)$ points inside its convex hull $\mathrm{CH}(X'(\mathbb{F},\epsilon))$, and $N = |X'(\mathbb{F},\epsilon)|$. The number of points after conversion remains the same by~\Cref{lem:numpts_same}. We need to show that after conversion, except for the points of $\OO(t)$ gadgets, no other points inside the convex hull $\mathrm{CH}(X'(\mathbb{F},\epsilon))$ lie on the new convex hull $\mathrm{CH}(X'_{d}(\mathbb{F},\epsilon))$. The number of points in each of the $\OO(t)$ anomalous gadgets are constant in number, and hence not too many points from the interior of $\mathrm{CH}(X'(\mathbb{F},\epsilon))$ can lie on $\mathrm{CH}(X'_{d}(\mathbb{F},\epsilon))$. 
    
    After conversion, all the points on a horizontal connecting row have the same $y$-coordinate, as they initially had the same $y$-coordinate and therefore undergo the same transformation. Thus, all the points on a horizontal connecting row still lie on a horizontal line segment in $X'_{d}(\mathbb{F},\epsilon)$. Similarly, all the points on a vertical connecting row still lie on a vertical line segment in $X'_{d}(\mathbb{F},\epsilon)$. This implies that none of the points on the connecting rows can be a part of $\mathrm{CH}(X'_{d}(\mathbb{F},\epsilon))$ as there can be no line passing through them that also contains all terminal points on one side of it. 

    The same thing holds for the square and hexagonal gadgets (crossovers) as well, except the hexagonal gadgets placed at the beginning or end of any row. This is because all the points which are a part of these square and hexagon gadgets are surrounded by connecting row points all four sides, above, below, left and right. So again, only the terminators and the hexagonal gadgets appearing at the beginning or end of any row contribute to $\mathrm{CH}(X'_{d}(\mathbb{F},\epsilon))$.

    Now, since we had adjusted the number of points in the long rows of the terminators and hexagonal gadgets such that their lengths and breadths are some constants, the number of points in each of the terminators and hexagonal gadgets can be bounded by some constant as the minimum distance between any two consecutive points on the long rows or standard rows is at least $\frac{1}{11}$. Therefore, each of these gadgets contribute some constantly many points to $\mathrm{CH}(X'_{d}(\mathbb{F},\epsilon))$.
    
    As we have seen in~\Cref{lem:redcn_size}, the number of terminators is $6t+2$ and the number of hexagonal gadgets corresponding to the beginning or end of any row is at most $6n$. Therefore, the number of points contributed by the terminators and the hexagonal gadgets placed at the beginning or the end of any row, to $\mathrm{CH}(X'_{d}(\mathbb{F},\epsilon)$ is $\OO(t+n) = \OO(t)$, as $n \leq t$. Even if all the points in $\mathrm{Circ}(D,r,n')$ lie on the new convex hull $\mathrm{CH}(X'_{d}(\mathbb{F},\epsilon)$, we have $\Omega(t^\gamma) = \Omega(N^\epsilon)$ points inside it. Thus we are done.
\end{proof}

We get the following theorem from~\Cref{lem:valid_desmt,lem:numpts_same,lem:desmt_redn_size}.

\begin{theorem}\label{thm:desmt_redn}
    The instance $X'_{d}(\mathbb{F},\epsilon)$ constructed is a valid DESMT instance on an $\Omega(N^\epsilon)$-Almost Convex Set, where  $|X'_{d}(\mathbb{F},\epsilon)| = |X'(\mathbb{F},\epsilon)| = N$.
\end{theorem}

Following Theorems 3 and 4 in~\cite{garey1977complexity}, we get that the DESMT problem is NP-Complete for $\Omega(N^\epsilon)$-Almost Convex Sets, where $N$ is the total number of terminals. Since we get the reduced instance $X'_{d}(\mathbb{F},\epsilon)$ from the X3C instance $(n,\mathbb{F})$, the DESMT problem is strongly NP-complete for $\Omega(N^\epsilon)$-Almost Convex Sets, and does not admit any FPTAS.

Using Theorem 5 of~\cite{garey1977complexity}, we get that if the ESMT problem has an FPTAS, then the X3C problem can be solved in polynomial time. The Theorem also applies for our case of $\Omega(N^\epsilon)$-Almost Convex Sets. Therefore, we get the following theorem,

\begin{theorem}\label{thm:no_fptas}
    There does not exist any FPTAS for the ESMT problem on $f(n)$-Almost Convex Sets of $n$ terminals, where $f(n) = \Omega(n^\epsilon)$ and $\epsilon \in (0,1]$, unless \pnp.
\end{theorem}