
\subsection{\ESMT for $2$-Concentric Parallel Regular $3$-gons} \label{triangles}

Note that a regular $3$-gon is an equilateral triangle and therefore, for the rest of this section we call a regular $3$-gon as an equilateral triangle. We describe a minimal solution for \ESMT for $2$-CPR equilateral triangles.

\begin{lemma}\label{3-gon}
Consider two concentric and parallel equilateral triangles $A_1A_2A_3$ and $B_1B_2B_3$, where $B_1B_2B_3$ has side length $\lambda > 1$ and $A_1A_2A_3$ has side length $1$. An SMT of $\{A_1, A_2, A_3, B_1, B_2, B_3\}$ is an SMT of $\{B_1,B_2,B_3\}$, and has length $\sqrt 3 \cdot \lambda$. 
\end{lemma}

\begin{proof}
It is to be noted that the centre $O$ of both $A_1A_2A_3$ and $B_1B_2B_3$ is also the Torricelli point of both $A_1A_2A_3$ and $B_1B_2B_3$~\cite{du1987steiner}. On taking $O$ as the only Steiner point, the SMT for $\{B_1,B_2,B_3\}$ is $ \mathcal T_3 = \{OB_1, OB_2, OB_3\}$~\cite{du1987steiner}. However, the edges of $\mathcal T_3$ already pass through $A_1$, $A_2$ and $A_3$. Therefore, $ \mathcal T_3$ with $E(\mathcal T_3) = \{OA_1, OA_2, OA_3, A_1B_1, A_2B_2, A_3B_3\}$ is also the SMT for $\{A_1, A_2, A_3, B_1, B_2, B_3\}$ as shown in Figure \ref{conc_eq_tri}.

 From the definition of $\mathcal T_3$, we have the length of the \smtpoly, for $n = 3$ as 
$$ | \mathcal T_3 | = \sqrt 3 \cdot \lambda $$ 
\end{proof}

\begin{figure}[h]
\centering
\includegraphics[width=4cm]{3_gon_topo1.png}
\caption{SMT for $2$-CPR $3$-gons }
\label{conc_eq_tri}
\end{figure}

