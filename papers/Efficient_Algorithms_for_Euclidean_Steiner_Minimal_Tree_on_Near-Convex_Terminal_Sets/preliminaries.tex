\section{Preliminaries} \label{preliminaries}

 Consider two regular $n$-gons with vertex sets $\{A_1,A_2,\ldots, A_n\}$ and $\{B_1,B_2,\ldots, B_n\}$, respectively. We denote $A_1A_2A_3...A_n$ and $B_1B_2B_3...B_n$. Without loss of generality, we assume that $\overline{A_iA_{i + 1}} = 1$ for $1 \le i < n$, and $\overline{B_iB_{i + 1}} = \lambda $ for $1 \le i < n$, where $\lambda > 1$. For convenience, we define $A_{n + 1} := A_1$, $B_{n + 1} := B_1$, $A_0 := A_n$ and $B_0 := B_n$. Further, we define $O$ to be the common centre of $A_1A_2A_3 \ldots A_n$ and $B_1B_2B_3 \ldots B_n$. In this paper, we would be using the notation $\{A_i\}$ to denote the polygon $A_1A_2A_3 \ldots A_n$ and $\{B_i\}$ to denote the polygon $B_1B_2B_3 \ldots B_n$. \\

Moreover, we will use the following facts from \cite{hwang1992steiner} extensively.

\begin{lemma} \label{atleast_120}
    (a) No two edges of an SMT can meet at an angle less than $\frac{2 \pi}{3}$. (b) Each Steiner point of an SMT is of degree exactly three.
\end{lemma}

From these two facts, we see that:

\begin{lemma}
    Two edges meeting at a Steiner point must make an angle of $\frac{2 \pi}{3}$.
\end{lemma}

We start by defining some preliminary notations

\begin{definition}
    In our setting, by \textbf{terminal point} or simply \textbf{terminal}, we mean any point in $\{A_i\} \cup \{B_i\}$.
\end{definition}

\begin{definition}
    For any tree $\mathcal T$ on the Euclidean Plane, we denote by the notation $|\mathcal T|$, the sum of lengths of all edges in $\mathcal T$. 
\end{definition}