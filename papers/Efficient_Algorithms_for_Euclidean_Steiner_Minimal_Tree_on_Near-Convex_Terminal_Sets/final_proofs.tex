
\subsection{\ESMT and Large Polygons with Large Aspect Ratios} \label{final_proofs}


In this section, we consider the \ESMT problem when the terminal set is formed by the vertices of $2$-CPR $n$-gons, namely $\{A_i\}$ and $\{B_i\}$. As mentioned earlier, $\{A_i\}$ is the inner polygon and $\{B_i\}$ is the outer polygon of this set of $2$-CPR $n$-gons. In particular, we consider the case when $n \geq 13$; for $n \leq 12$ these are constant sized input instances and can be solved using any brute-force technique. We also require that the aspect ratio 
$\lambda$ has a lower bound $\lambda_1$, i.e. we do not want the two polygons to have sides of very similar length. The exact value of  $\lambda_1$ will be clear during the description of the algorithm. Intuitively, when $\lambda$ is \emph{very large}, the SMT should look similar to what was derived in~\cite{weng1995steiner}. In other words, (please refer to Figure \ref{sing_con_top_fig}):
\begin{enumerate}
    \item for some $j \in [n]$, there is a  vertical fork connecting the two consecutive inner polygon points $A_j, A_{j + 1}$  with the two consecutive outer polygon points $B_j, B_{j + 1}$ - we refer to this vertical fork as the \emph{vertical gadget} for the SMT,
    \item the other points in $\{B_i\}$ are connected directly via $(n - 2)$ outer polygon edges,
    \item the other points in $\{A_i\}$ are connected via $(n - 2)$ inner polygon edges. 
\end{enumerate}

We call such a topology, a \emph{singly connected topology} as in Figure \ref{sing_con_top_fig}. 
%(generated using \href{www.geosteiner.com}{Geosteiner}). 
For the rest of this section, we consider the SMTs for a large enough aspect ratio, $\lambda$ and show that there is an SMT that must be a realisation of a \emph{singly connected topology}. We refer to an SMT for the terminal set defined by the vertices of $\{A_i\}$ and $\{B_i\}$ as the \smtpoly. 

Without loss of generality, we consider the edge length of any side $A_iA_{i+1}$ in $\{A_i\}$ to be $1$. As we defined the aspect ratio to be $\lambda$, any side $B_iB_{i+1}$ of $\{B_i\}$ must have a side length of $\lambda$. Further, we observe that for any SMT $\mathcal T$, specifying $E(\mathcal T)$ sufficiently determines the entire tree, as $V(\mathcal T) = \{P~|~\exists Q \text{ such that } PQ \in E(\mathcal T)\}$.

We start with the following formal definitions:

\begin{definition}\label{def:singly-conn}
A Steiner topology of $\{A_i\} \cup \{B_i\}$ is a \textbf{singly connected topology}, if it has the following structure:

\begin{itemize}
    \item A vertical gadget i.e.  five edges $\{A_jS_a, A_{j + 1}S_a, S_aS_b, S_bB_j, S_bB_{j + 1}\}$ for some $1 \le j \le n$, where $S_a$ and $S_b$ are newly introduced Steiner points contained in the isosceles trapezoid $\{A_j, A_{j + 1}, B_j, B_{j + 1}\}$. 
    \item All $(n - 2)$ polygon edges of $\{A_i\}$ excluding the edge $A_jA_{j + 1}$
    \item All $(n - 2)$ polygon edges of $\{B_i\}$ excluding the edge $B_jB_{j + 1}$
\end{itemize} 
\end{definition}


\begin{figure}[h]
\centering
\subfloat[\centering SMT of $2$-CPR $13$-gons]{\includegraphics[width=4cm]{13_sing.png}}
    \qquad
\subfloat[\centering SMT of $2$-CPR $20$-gons] {\includegraphics[width=4cm]{20_sing.png} }
\caption{ \emph{Singly connected topology} of $2$-CPR $n$-gons $n = 13$ and $n = 20$ }
\label{sing_con_top_fig}
\end{figure}

We define the notion of a path in an SMT for the vertices of $\{A_i\}$ and $\{B_i\}$ where the starting point is in $\{A_i\}$ and the ending point is in $\{B_i\}$.
\begin{definition}
    An \textbf{A-B path} is a path in a Steiner tree of $\{A_i\} \cup \{B_i\}$ which starts from a vertex in $\{A_i\}$ and ends at a vertex in $\{B_i\}$ with all intermediate nodes (if any) being Steiner points. 
\end{definition}

The following Definition and Figure~\ref{fig:counterpath} is useful for the design of our algorithm.
\begin{definition}\label{def:cw_ccw_path}
    A \textbf{counter-clockwise path} is a path $P_1, P_2, ... P_m$ in a Steiner tree such that for all $i \in \{2, \ldots, m - 1\}, \angle P_{i - 1}P_{i}P_{i + 1} = \frac{2 \pi}{3}$ in the counter-clockwise direction. Similarly, a \textbf{clockwise path} is a path $P_1, P_2, ... P_m$ in a Steiner tree such that for all $i \in \{2, \ldots, m - 1\}, \angle P_{i - 1}P_{i}P_{i + 1} = \frac{4 \pi}{3}$ in the counter-clockwise direction.
\end{definition}


\begin{figure}[h]
\centering
\includegraphics[width=7cm]{CW_CCW_def.png}
\caption{$\angle P_{i-1}P_iP_{i+1} = \alpha$ in the counter-clockwise direction and $\angle P_{i-1}P_iP_{i+1} = \beta$ in the clockwise direction, as in~\Cref{def:cw_ccw_path}}
\label{fig:counterpath}
\end{figure}

Now, we consider any Steiner point $S$ in any SMT. Let $P$ and $Q$ be two neighbours of $S$. We now prove that there is no point of the SMT inside the triangle $PSQ$.

\begin{observation} \label{no_pt_in_PSQ}
    Let $S$ be a Steiner point in any SMT for $\{A_i\} \cup \{B_i\}$, with neighbours $P$ and $Q$. Then, no point of the SMT lies inside the triangle $PSQ$.
\end{observation}

\begin{proof}
    By the \emph{lune property} (Proposition \ref{lune}), for any edge $P_1Q_1$ in an SMT, for the two circles centred at $P_1$ and at $Q_1$, respectively and both having a radius of $\overline{P_1Q_1}$, the intersection region does not contain any point of the SMT.

    \begin{figure}[h]
    \centering
    \includegraphics[width=10cm]{no_pt_in_PSQ.png}
    \caption{Observation \ref{no_pt_in_PSQ}: Given triangle $PSQ$, equilateral triangles $PSE$ and $QSF$ are constructed}
    \label{no_pt_in_PSQ_fig}
    \end{figure}

    Let $E$ and $F$ be points on the internal angle bisector of $\angle PSQ$, such that $\angle SPE = \angle SQF = \frac{\pi}{3}$ as shown in Figure \ref{no_pt_in_PSQ_fig}. Since $E$ and $F$ are points on the angle bisector of $\angle PSQ$, $\angle PSE = \angle QSF = \frac{\pi}{3}$. Hence, triangles $PSE$ and $QSF$ are equilateral triangles.

    Since $PS$ is an edge in the SMT, by the lune property, the intersection of the circles centred at $P$ and $S$, both with radius $\overline{PS}$ contain no point inside which is a part of the SMT. Since the lune contains the entire equilateral triangle $PSE$, no point of the SMT lies inside triangle $PSE$. Similarly, no point of the SMT lies inside the triangle $QSF$.

    Further, as $\angle PSQ = \frac{2 \pi}{3}$, $\angle SPQ + \angle SQP = \frac{\pi}{3}$. This means $\angle SPQ, \angle SQP < \frac{\pi}{3}$. Therefore, as $\angle SPE = \angle SQF = \frac{\pi}{3}$, $E$ and $F$ must lie outside the triangle $PSQ$. This implies that the triangle $PSQ$ is covered by the union of the triangles $PSE$ and $QSF$. As no point of the SMT lies in triangles $PSE$ and $QSF$, triangle $PSQ$ must contain no points of the SMT as well.
\end{proof}


Next, we show that in an \smtpoly there cannot be any Steiner point, in the interior of the polygon $\{A_i\}$, that is a direct neighbour of some point $B_k$ in the polygon $\{B_i\}$.

\begin{observation} \label{no_in_to_B}
    For any \smtpoly, there cannot exist a Steiner point $S$ lying in the interior of the polygon $\{A_i\}$ such that $SB_k$ is an edge in an SMT for some $B_k \in \{B_i\}$. 
\end{observation}

\begin{proof}
    For the sake of contradiction, we assume that for some \smtpoly there exists a Steiner point $S$ lying in the interior of the polygon $\{A_i\}$ such that $SB_k$ is an edge in the SMT for some $B_k \in \{B_i\}$. Let $A_mA_{m+1}$ be the edge such that $SB_k$ intersects $A_mA_{m+1}$. Without loss of generality, assume that $A_m$ is closer to $B_k$ than $A_{m+1}$. Therefore $\angle B_kA_mS > \angle B_kA_mA_{m+1} \ge \frac{\pi}{2} + \frac{\pi}{n} > \frac{\pi}{2}$. This means that $B_kS$ is the longest edge in the triangle $B_kSA_m$. Therefore we can remove the edge $B_kS$ from the SMT and replace it with either $B_kA_m$ or $SA_m$ to get another tree connecting the terminal set with a shorter total length than what we started with, which is a contradiction. 
\end{proof}

We further analyze SMTs for $\{A_i\}\cup \{B_i\}$.

\begin{observation} \label{subchain-smt}
    Let $\mathcal V = \{A_j, A_{j + 1}, \ldots, A_k\}$ be the interval of consecutive vertices of $\{A_i\}$ lying between $A_j$ and $A_k$ (which includes $A_{j + 1}$) such that $A_j$ is distinct from $A_{k + 1}$. Let $U$ be any point on the line segment $A_kA_{k+1}$. Then an SMT of $\mathcal V \cup \{U\}$ is $\mathcal T$, with $E(\mathcal T) = \{A_jA_{j+1}, A_{j+1}A_{j+2}, \ldots, A_{k-1}A_k\} \cup \{A_kU\}$. 
\end{observation}
\begin{proof}
    For the sake of contradiction, assume that there exists an SMT $\mathcal T'$ of $\mathcal V \cup \{U\}$ such that $|\mathcal T'| < |\mathcal T|$.  
    
    From \cite{du1987steiner}, we know that $\mathcal T_A$, with $E(\mathcal T_A) = \{A_jA_{j+1}, A_{j+1}A_{j+2}, \ldots, A_{j-2}A_{j-1}\}$ (\textit{i.e.} all edges of polygon $\{A_i\}$ except $A_{j-1}A_j$) is an SMT of $\{A_i\}$. Since $U \in A_kA_{k+1} \in E(\mathcal T_A)$, $\mathcal T_A$ must also be an SMT of $\{A_i\} \cup \{U\}$. However, $\mathcal T_A$ can be partitioned as $\mathcal T_A = \mathcal T \uplus \mathcal T_1$, where $E(\mathcal T_1) = \{UA_{k+1}\} \cup \{A_{k+1}A_{k+2}, \ldots, A_{j-2}A_{j-1}\}$.
    However, as $\mathcal T'$ is assumed to be of shorter total length than $\mathcal T$, $\mathcal T' \cup \mathcal T_1$ is a tree, containing $\{A_i\}$ as a vertex subset, which has a shorter total length than $\mathcal T_A$, contradicting the optimality of $\mathcal T_A$.
\end{proof}


We proceed by showing that in any \smtpoly there exists at least one A-B path which is also a counter-clockwise path. Symmetrically, we also show that for any \smtpoly there exists another clockwise A-B path which consists of only clockwise turns. We can intuitively see that this is true because, if all clockwise paths starting at a vertex in $\{A_i\}$ also ended in a vertex in $\{A_i\}$, there would be enough paths to form a cycle, which is not possible in a tree.

\begin{lemma}\label{left_right_turn_path}
In any \smtpoly, there exists an A-B path which is also a clockwise path and there exists an A-B path which is also a counter-clockwise path.
\end{lemma}
\begin{proof}
    For the sake of contradiction, assume that for some \smtpoly there is no A-B path which is a counter-clockwise path. We pick an arbitrary vertex $A_{i_1} \in \{A_i\}$ such that it is connected to at least one Steiner point (say $S_{i_1}$). We consider the counter-clockwise path, $\mathcal C_1$ starting from $A_{i_1}S_{i_1}$ and ending at the first terminal point in the counter-clockwise path. By assumption, there can be no vertex of $\{B_i\}$ in $\mathcal C_1$, hence the endpoint must be a vertex in $\{A_i\}$. Let $A_{i_2}$ be the other endpoint of $\mathcal C_1$. By definition, the penultimate vertex in this counter-clockwise path must be a Steiner point, we call it $S_{i_2}$. We again consider the counter-clockwise path starting from $A_{i_2}S_{i_2}$, and similarly, let $A_{i_3}$ be the first terminal that is encountered in this path. The penultimate vertex in this counter-clockwise path must be a Steiner point, we call it $S_{i_3}$. We can repeat this procedure indefinitely to obtain $A_{i_4}, A_{i_5}, A_{i_6}, \ldots $ as there are no counter-clockwise A-B paths. However, as $\{A_i\}$ has $n$ vertices, there must be a repetition of vertices among   $A_{i_1}, A_{i_2}, A_{i_3}, \ldots, A_{i_{n+1}}$, implying the existence of a cycle in the SMT, which is a contradiction.

    This symmetrically implies that there must also be a clockwise A-B path. 
\end{proof}

Our next step is to bound the number of `connections' that connect the inner polygon $\{A_i\}$ and the outer polygon $\{B_i\}$ for a large aspect ratio, $\lambda$. As $\lambda$ increases, the area of the annular region between the two polygons increases as well. Therefore, an increase in the number of connections would lead to a longer total length of the SMT considered. Consequently, we will prove that after a certain positive constant $\lambda_1$, for $\lambda > \lambda_1$ any \smtpoly will have a single `connection' between the two polygons. Moreover, \cite{weng1995steiner} gives us an evidence that as $\lambda \rightarrow \infty$, there will indeed be a single connection connecting the outer polygon and the inner polygon for $n \ge 12$. We can formalize this notion of existence of a single `connection' with the following lemma.\\

\begin{lemma} \label{mincut_1}
    For any \smtpoly with $n \ge 13$ and $\lambda > \lambda_{1}$, the number of edges needed to be removed in order to disconnect $\{A_i\}$ and $\{B_i\}$ is 1, where 
    $$
    % \lambda_{1} = \frac{2 \sin{\frac{\pi}{n}} + 1}{1 - 6 \sin \frac{\pi}{n}}
    \lambda_{1} = \frac{1}{1 - 4 \sin \frac{\pi}{n}}
    $$
\end{lemma}

\begin{proof}
        For the sake of contradiction, assume that for some \smtpoly, there are at least two distinct edges in that SMT, which are needed to be removed in order to disconnect $\{A_i\}$ and $\{B_i\}$. We start with a claim.
        
        \begin{claim} \label{more_than_lambda}
        A counter-clockwise A-B path in any SMT of $\{A_i\} \cup \{B_i\}$ must have an edge of length greater than $\lambda$.
        \end{claim}
        
\begin{proof}
        We consider a generic setting, where $\mathcal T$ is an SMT of some set of terminal points $\mathcal P$. Let $H \in V(\mathcal T)$ be vertex of $\mathcal T$. If $\mathcal C$ be a counter-clockwise path starting from $H$ such that no edge in the counter clockwise path has a length of more than $r$, for some $r \in \mathbb{R}^{+}$. Due to Lemma 2.4 (1) of \cite{weng1995steiner},  we know that $\mathcal C$ is contained entirely in the circle centred at $H$ with radius $2r$.
        
        In our case, any vertex in $\{A_i\}$ and any vertex in $\{B_i\}$ are separated by the distance of at least $\dfrac{\lambda - 1}{2 \sin{\frac{\pi}{n}}}$. Therefore, by the above fact, the maximum edge length in a counter-clockwise A-B path of any SMT of $\{A_i\} \cup \{B_i\}$ must be at least $\dfrac{\lambda - 1}{4 \sin{\frac{\pi}{n}}}$. Moreover, we have $\lambda > \lambda_1$. Therefore $\lambda > \dfrac{1}{1 - 4 \sin{\frac{\pi}{n}}} \implies \dfrac{\lambda - 1}{4 \sin{\frac{\pi}{n}}} > \lambda$. Hence, a counter-clockwise A-B path in any SMT of $\{A_i\} \cup \{B_i\}$ must have one edge greater than $\lambda$. This proves the claim.
\end{proof}
        
        Now, for any SMT of $\{A_i\} \cup \{B_i\}$, let $\mathcal C$ be a counter-clockwise A-B path (this exists due to Lemma \ref{left_right_turn_path}). From Claim \ref{more_than_lambda}, we know that there is an edge $e$ in $\mathcal C$, with a length greater than $\lambda$. On removing the edge $e$, the SMT splits into a forest of two trees. Let the trees be $\mathcal T_x$ and $\mathcal T_y$. As we assumed that there are two edges required to disconnect $\{A_i\}$ and $\{B_i\}$, there must exist an A-B path in either $\mathcal T_x$ or $\mathcal T_y$. Without loss of generality, let $\mathcal T_x$ contain an A-B path, and hence $\mathcal T_x$ contains at least one point from $\{A_i\}$ and at least one point from $\{B_i\}$. Further, $\mathcal T_y$ must contain at least one terminal point (as it must contain all terminal points in one of the sides of the removed edge $e$). If $\mathcal T_y$ contains a point from $\{A_i\}$, then the polygon $\{A_i\}$ has vertices both from $\mathcal T_x$ and $\mathcal T_y$; otherwise, if $\mathcal T_y$ contains a point from $\{B_i\}$, then the polygon $\{B_i\}$ has vertices both from $\mathcal T_x$ and $\mathcal T_y$. 
        
        This means that either the polygon $\{A_i\}$ or the polygon $\{B_i\}$ will contain at least one node from each of $\mathcal T_x$ and $\mathcal T_y$. Further, as any given vertex must be either in $\mathcal T_x$ or in $\mathcal T_y$, either $\{A_i\}$ or $\{B_i\}$ must contain two consecutive vertices $U_i$ and $U_{i + 1}$ such that one of them is in $\mathcal T_x$ and the other is in $\mathcal T_y$. We simply connect $U_i$ and $U_{i + 1}$ by the polygon edge which is of length $1$ (if $U_i, U_{i + 1} \in \{A_i\}$) or of length $\lambda$ (if $U_i, U_{i + 1} \in \{B_i\}$), giving us back a tree $\mathcal{T}'$containing all the terminals. However we discarded an edge of length greater than $\lambda$ and added back an edge of length at most $\lambda$ in this process, which means that the total length of $\mathcal{T}'$ is strictly less than the SMT we started with. This is a contradiction.
\end{proof}

% \begin{remark}
%     We can find an even tighter bound $\lambda_{0} = \dfrac{1 + (2 - \sqrt{3}) \cdot \tan{\frac{\pi}{n}}}{1 - (4 - \sqrt{3}) \cdot \tan{\frac{\pi}{n}}}$, with $\lambda_1 > \lambda_0 > \lambda_s > \lambda_v$ such that whenever $\lambda \ge \lambda_0$, the SMT follows the singly connected topology, and whenever $\lambda < \lambda_0$, we use multiple vertical gadgets instead, $\forall n \ge 12$.
% \end{remark}

We now proceed to further investigate the connectivity of $\{A_i\}$ and $\{B_i\}$. 
%We start with the portion of this connection that is closer to the polygon $\{A_i\}$.\\

\begin{lemma} \label{steiner_path_from_A_to_A}
    Consider an \smtpoly for $n \ge 13$ and  $\lambda \ge \lambda_{1}$. There must exist $j \in [n]$ and a Steiner point $S_1$, such that terminals $A_j, A_{j + 1}$ form a path $A_j$, $S_1$, $A_{j+1}$ in the SMT and each A-B path passes through $S_1$; where 
    
    $$\lambda_{1} = \frac{1}{1 - 4 \sin \frac{\pi}{n}}$$
\end{lemma}

\begin{proof}
    From Lemma \ref{left_right_turn_path}, we know that there exists one clockwise A-B path and one counterclockwise A-B path in any SMT of $\{A_i\} \cup \{B_i\}$. Let a clockwise A-B path start from $A_r$ and a counter-clockwise A-B path start from $A_l$. Further following from Lemma~\ref{mincut_1}, as there is one edge common to all A-B paths, the clockwise A-B path from $A_r$ and the counter-clockwise A-B path from $A_l$ must share a common edge $S_1S_2$. Therefore, each A-B path must pass through $S_1$ and $S_2$. Without loss of generality we assume that point $S_1$ is closer to the polygon $\{A_i\}$ than $S_2$. This means $S_1$ is either a Stiener point or a terminal vertex of $\{A_i\}$.

    \begin{claim} \label{S_1_notin_A}
        $S_1$ is not a vertex in $\{A_i\}$
    \end{claim}
\begin{proof}
    For the sake of contradiction, we assume that $S_1$
     to be a vertex in $\{A_i\}$, let $S_1 = A_k$ in some SMT $\mathcal T_0$. We disconnect the edge $S_1S_2$ from $\mathcal T_0$, which results in the formation of a forest of two trees $\mathcal T_x$ and $\mathcal T_y$ such that $S_1 = A_k \in \mathcal T_x$ and $S_2 \in \mathcal T_y$.

    $\mathcal T_x$ must contain all vertices of $\{A_i\}$ and $\mathcal T_y$ must contain all vertices of $\{B_i\}$, as there would be an A-B path in the graph otherwise (contradicting that $S_1S_2$ disconnects $\{A_i\}$ and $\{B_i\}$). We replace $\mathcal T_x$ with the SMT of $\{A_i\}$, which is also an MST (from \cite{weng1995steiner}). Since all MST's are of the same length, we choose such an MST in which $A_k$ is not a leaf node. This means $A_{k-1}A_k$ and $A_kA_{k+1}$ are edges in the chosen MST of $\{A_i\}$. We now add back the edge $A_kS_2$, resulting in a connected tree $\mathcal T_0'$ of $\{A_i\} \cup \{B_i\}$. Since we replaced the tree $\mathcal T_x$ with an SMT of $\{A_i\}$, the total length of the $\mathcal T_0'$ must not be more than the total length of the $\mathcal T_0$.

    However, we observe that $A_k$ has three neighbours in $\mathcal T_0'$, which are $A_{k+1}, A_{k-1}, S_2$. However $\angle A_{k-1}A_kA_{k+1} > \frac{2 \pi}{3}$. This means either $\angle S_2A_kA_{k+1} < \frac{2 \pi}{3}$ or $\angle A_{k-1}A_kS_2 < \frac{2 \pi}{3}$. But due to Proposition \ref{smt-prop}, this cannot form an SMT. Therefore $\mathcal T_0'$ is not optimal; and hence, $\mathcal T_0$ cannot be optimal as well. This proves  the claim.
    \end{proof}

    Therefore, $S_1$ must be a Steiner point. Let $P$ and $Q$ be the neighbours of $S_1$ other than $S_2$, such that $\angle PS_1S_2$ is a clockwise turn while $\angle QS_1S_2$ is a counter-clockwise turn. This means that the clockwise A-B path from $A_r$ passes through $P$ and the counter-clockwise A-B path from $A_l$ passes through $Q$. We prove that $P$ and $Q$ are consecutive vertices of $\{A_i\}$ in some SMT of $\{A_i\} \cup \{B_i\}$.


    \begin{claim} \label{PQ_outside_A}
         $P$ and $Q$ cannot simultaneously lie in the interior of the polygon $\{A_i\}$. 
    \end{claim}
\begin{proof}
    We assume for the sake of contradiction that both $P$ and $Q$ lie in the interior of the polygon $\{A_i\}$. On deleting the edge $S_1S_2$, the SMT of $\{A_i\} \cup \{B_i\}$ splits into two trees $\mathcal T_1$ (rooted at $S_1$) and $\mathcal T_2$ (rooted at $S_2$). Further, as all A-B paths pass through $S_1S_2$, all vertices of $\{A_i\}$ must be in $\mathcal T_1$ whereas all vertices of $\{B_i\}$ must lie in  $\mathcal T_2$. Further, $\mathcal T_1$ must be the SMT of $\{A_i\} \cup \{S_1\}$ and $\mathcal T_2$ must be the SMT of $\{B_i\} \cup \{S_2\}$.

    \begin{itemize}
    \item \textit{Case I: One point in $\{S_1, S_2\}$ lies in the interior of $\{A_i\}$ and the other point lies in the exterior of $\{A_i\}$:} This means that the edge $S_1S_2$ crosses some polygon edge of $\{A_i\}$, call it $A_mA_{m+1}$. Let $D$ be the intersection of $A_mA_{m+1}$ and $S_1S_2$. We replace $\mathcal T_1$ with an MST of $\{A_i\}$ that contains the edge $A_mA_{m+1}$ (this can never lead to increase in total tree length due to \cite{weng1995steiner}) and remove the line segment $S_1D$ from $\mathcal T_2$. This forms a tree connecting the terminal set $\{A_i\} \cup \{B_i\}$ which has a total length smaller than the SMT we started with, which is a contradiction.
    
    \item \textit{Case II: Both $S_1$ and $S_2$ lie in the interior of polygon $\{A_i\}$:} We further consider two cases for this:

    \begin{itemize}
        \item Consider that there is at least one polygon edge $A_mA_{m+1}$ of $\{A_i\}$ such that it does not intersect with $\mathcal T_2$. Then we can replace $\mathcal T_1$ by the MST of $\{A_i\}$ which does not contain the edge $A_mA_{m+1}$ without reducing the total edge. However, this will be a connecting tree of $\{A_i\} \cup \{B_i\}$ with a smaller total length than the tree we started with (as we had removed the edge $S_1S_2$ previously), which is a contradiction.
        \item Now, consider that all polygon edges of $\{A_i\}$ intersect with some edge in $\mathcal T_2$. Since $\mathcal T_2$ is rooted at $S_2$ which lies in the interior of $\{A_i\}$, there must be $n$ distinct edges crossing the polygon $\{A_i\}$. However, $\mathcal T_2$ must be the SMT of the points $\{S_2\} \cup \{B_i\}$, which means there can be at most $(n - 1)$ Steiner points other than $S_2$ (from~\Cref{smt-prop}). Therefore, one of these $n$ edges must have a point in $\{B_i\}$ as one of its endpoints, contradicting Observation \ref{no_in_to_B}. 

  
    \end{itemize}
    
    \item \textit{Case III: Both $S_1$ and $S_2$ lie in the exterior of polygon $\{A_i\}$:} This means that the edges $S_1P$ and $S_1Q$ intersect the polygon edges of $\{A_i\}$. Further, from Observation \ref{no_pt_in_PSQ}, there cannot be any terminal inside the triangle $S_1PQ$. Hence, $S_1P$ and $S_1Q$ must intersect the same polygon edge of $\{A_i\}$ (otherwise intermediate vertices from $\{A_i\}$ would lie in the triangle $S_1PQ$). Let this edge be $A_tA_{t+1}$. Let $P_1$ and $Q_1$ be the points of intersection of $A_tA_{t+1}$ with $S_1P$ and $S_1Q$ respectively.

    We now remove the line segments $S_1P_1$ and $S_1Q_1$ from $\mathcal T_1$. This results in another split into two connected trees $\mathcal T_P$ (containing $P_1$, $P$ and a subset of $\{A_i\}$) and $\mathcal T_Q$ (containing $Q_1$, $Q$ and the remaining vertices of $\{A_i\}$). 
    
    We observe that the terminals in $\mathcal T_P$ and $\mathcal T_Q$ form consecutive intervals of the edges in $\{A_i\}$. To see why, consider the opposite, \textit{i.e.} there are vertices $A_{i_1}, A_{i_2}, A_{i_3}, A_{i_4}$ appearing in that order in $\{A_i\}$ such that $A_{i_1}, A_{i_3} \in \mathcal T_P$ whereas $A_{i_2}, A_{i_4} \in \mathcal T_Q$. As $\mathcal T_P$ and $\mathcal T_Q$ lie in the interior of $\{A_i\}$, the path from $A_{i_1}$ to $A_{i_3}$ in $\mathcal T_P$ must cross the path from $A_{i_2}$ to $A_{i_4}$ in $\mathcal T_Q$. However, there cannot be crossing paths in the original SMT of $\{A_i\} \cup \{B_i\}$ (due to~\Cref{smt-prop}). 
    
    Let $\{A_{w}, A_{{w}+1}, \ldots, A_t\}$ be the terminals in $\mathcal T_P$ and the remaining terminals in $\{A_i\}$ are in $\mathcal T_Q$. Again, let $\mathcal T_P'$, $\mathcal T_Q'$ be defined as:
    $$E(\mathcal T_P') = \{A_wA_{w+1}, A_{w+1}A_{w+2}, \ldots, A_{t-1}A_{t}\} \cup \{A_tP_1\}$$
    and
    $$E(\mathcal T_Q') = \{A_{t+1}A_{t+2}, \ldots, A_{w-2}A_{w-1}\} \cup \{Q_1A_{t-1}\}$$
    From Observation \ref{subchain-smt}, we know that $\mathcal T_P'$ is the SMT of $\{A_{w}, A_{{w}+1}, \ldots, A_t\} \cup \{P_1\}$ and $\mathcal T_Q'$ is the SMT of $\{A_{t + 1}, A_{{t}+2}, \ldots, A_{w-1}\} \cup \{Q_1\}$. Therefore, $|\mathcal T_P'| \le |\mathcal T_P|$ and $|\mathcal T_Q'| \le |\mathcal T_Q|$. This means that $| \mathcal T_1 | \ge |\mathcal T_1'|$, where $\mathcal T_1' = \{S_1P_1, S_1Q_1\} \cup \mathcal T_P' \cup \mathcal T_Q'$. Further, $\mathcal T_1'$ is also a connecting tree of $\{S_1\} \cup \{A_i\}$ and as $\mathcal T_1$ is an SMT of $\{S_1\} \cup \{A_i\}$, then $\mathcal T_1'$ must also be an SMT with $|
    \mathcal T_1'| = |\mathcal T_1|$.

    However, We can remove $S_1P_1$ and $P_1A_t$ from $\mathcal T_1'$ and add $S_1A_t$ to get another connecting tree of $\{S_1\} \cup \{A_i\}$, but with shorter total length (as $\overline{S_1P_1} + \overline{P_1A_t} > \overline{S_1A_t}$ from triangle inequality). This contradicts the optimality of $\mathcal T_1'$ which was derived to be an SMT of $\{S_1\} \cup \{A_i\}$.

    \end{itemize}
This proves the claim.
\end{proof}
    


    We proceed to prove a stronger claim regarding $P$ and $Q$.
    
    \begin{claim} \label{PQ_cons}
    $P$ and $Q$ are consecutive vertices of $\{A_i\}$ in any SMT of $\{A_i\} \cup \{B_i\}$.
    
    \end{claim}
    \begin{proof}
    We first prove that $P, Q$ are vertices of $\{A_i\}$. 
    
    From Claim \ref{PQ_outside_A}, we know that at least one among $P$ and $Q$ must not be in the interior of polygon $\{A_i\}$. Without loss of generality, let it be $P$. We now show that $P$ is a vertex of $\{A_i\}$. For the sake of contradiction we assume that $P$ is not a vertex of $\{A_i\}$ \textit{i.e.} $P$ is a Steiner point. Let $\overrightarrow{PF_1}$ and $\overrightarrow{PF_2}$ be tangents from $P$ to $\{A_i\}$ where $F_1, F_2$ are the points of tangency on $\{A_i\}$. As $\overrightarrow{PF_1}$ and $\overrightarrow{PF_2}$ are tangents, $\angle F_1PF_2 < \pi$. We denote the region between the tangents $\overrightarrow{PF_1}$ and $\overrightarrow{PF_2}$ which contains the all the points in $\{A_i\}$ as $\mathcal R$.
    
    From any Steiner point $H$, which lies outside $\mathcal R$, we can choose a neighbour $H_1$ of $H$ such that $\overrightarrow{HH_1}$ is not directed towards $\mathcal R$. Further we now show that there is one neighbour $P_1$ of $P$ such that $P_1$ is not in $\mathcal R$ and $P_1 \neq S_1$. 

    \textit{Case I: $S_1$ lies in $\mathcal R$.} However there must be another neighbour $P_1$ of $P$ not in $\mathcal R$ (as $\angle F_1PF_2 < \pi$) but as $P_1$ is outside $\mathcal R$, we must have $P_1 \neq S_1$.
    
    \textit{Case II: $S_1$ does not lie in $\mathcal R$ (Figure \ref{steiner_path_from_A_to_A_fig}).} As the counter-clockwise A-B path from $A_l$ passes through $Q$, $A_l$ must be to the left of the line $L_{QS_1}$ if the line is given a orientation from $Q$ to $S_1$. This means that one of the tangents from $P$ (Without loss of generality assume it to be $\overrightarrow{PF_1}$) intersects with the line $L_{QS_1}$. Therefore, taking angles in counter-clockwise order, we have:
    
\begin{figure}[h]
\centering
\includegraphics[width=13cm]{claim_proof_diag.png}
\caption{Case II of~\Cref{PQ_cons}}
\label{steiner_path_from_A_to_A_fig}
\end{figure}

    \begin{align*}
        & \angle S_1PF_1 < \frac{\pi}{3} & [\text{as } \overrightarrow{PF_1} \text{ intersects } L_{QS_1}] \\
        \implies & \angle S_1PF_2 = \angle S_1PF_1 + \angle F_1PF_2 < \frac{\pi}{3} + \pi = \frac{4 \pi}{3}\\
        \implies & \angle F_2PS_1 = 2 \pi - \angle S_1PF_2 > 2 \pi - \frac{4 \pi}{3} = \frac{2 \pi}{3}
    \end{align*}
    
    Hence there must exist one neighbour $P_1$ of $P$ lying outside the $\mathcal R$, precisely in the region bounded by the rays $\overrightarrow{PF_2}$ and $\overrightarrow{PS_1}$ with $P_1 \ne S_1$. 
    
    Further we can choose a neighbour $P_2$ of $P_1$ such that $\overrightarrow{P_1P_2}$ is directed away from $\mathcal R$. We can continue choosing $P_2, P_3, \ldots $ such that $\overrightarrow{P_iP_{i+1}}$ is directed away from the region $\mathcal R$. Moreover, the path  $P, P_1, P_2, \ldots$ must end at some point $B_k$ as it cannot end in any vertex of $\{A_i\}$ (since all vertices of $\{A_i\}$ are in $\mathcal R$). Now, let $\mathcal C_1$ be the path from $A_r$ to $B_k$ (which passes through $P$ and $P_1$) and let $\mathcal C_2$ be the counter-clockwise A-B path from $A_l$ (passing through $Q$, $S_1$ and $S_2$). We observe that $\mathcal C_1$ and $\mathcal C_2$ are two edge disjoint A-B paths, which is a contradiction to~\Cref{mincut_1}. This proves that $P$ is indeed a vertex in $\{A_i\}$. Therefore by Claim \ref{PQ_outside_A}, $Q$ does not lie inside $\{A_i\}$ and repeating this same argument on $Q$ yields that $Q$ is also a vertex of $\{A_i\}$. 

    Now, to prove that $P$ and $Q$ are consecutive vertices of $\{A_i\}$, we use Observation \ref{no_pt_in_PSQ}. Observation \ref{no_pt_in_PSQ} implies that there must not be any other point of the SMT in the triangle $PS_1Q$. This means that $P$ and $Q$ must be consecutive vertices of $\{A_i\}$, otherwise all polygon vertices of $\{A_i\}$ occurring in between $P$ and $Q$ would be inside the triangle $PS_1Q$ (as $\angle PS_1Q = \dfrac {2 \pi}{3}$ and $n \ge 13$). This proves the claim.
    \end{proof}

    Therefore, $P$ and $Q$ are consecutive vertices $A_j, A_{j+1}$ of the polygon $\{A_i\}$, for some $j \in [n]$ such that $A_j$, $S_1$, $A_{j+1}$ is a path in the SMT, where $S_1$ is a Steiner point lying on all A-B paths.
\end{proof}

Our next step is to investigate some more structural properties of an SMT for $\{A_i\} \cup \{B_i\}$. From \cite{du1987steiner}, we may guess that there would be a lot of polygon edges of both $\{A_i\}$ and $\{B_i\}$ in an SMT. We prove the following Lemma, stating that there is an SMT of $\{A_i\} \cup \{B_i\}$ which contains $(n - 2)$ polygon edges of $\{A_i\}$.

\begin{lemma} \label{n-2_A_poly_edges}
    For an \smtpoly with aspect ratio $\lambda$, 
    $\lambda > \lambda_{1} = \frac{1}{1 - 4 \sin \frac{\pi}{n}}$, let $S_1$ be the Steiner point such that all A-B paths pass through $S_1$. Let $A_j$ and $A_{j+1}$ be vertices of $\{A_i\}$ which are connected to $S_1$. Then, there exists an SMT of $\{A_i\} \cup \{B_i\}$ having $(n - 2)$ polygon edges of $\{A_i\}$ other than $A_jA_{j+1}$.
\end{lemma}

\begin{proof}
    Let $\mathcal T_0$ be any SMT of $\{A_i\} \cup \{B_i\}$. From Lemma \ref{steiner_path_from_A_to_A}, we know that there exists $S_1$, $A_j$ and $A_{j + 1}$ such that $S_1$ is a Steiner point which is a part of all A-B paths, and $A_jS_1A_{j+1}$ is a path $\mathcal T_0$.

    From $\mathcal T_0$, we remove the edges $S_1A_j$ and $S_1A_{j + 1}$ and add the edge $A_jA_{j+1}$. This results in a forest of two disjoint trees $\mathcal T_x$ and $\mathcal T_y$. One of these trees (say $\mathcal T_x$) must contain all terminal points from $\{A_i\}$ and the other tree must contain all terminals from $\{B_i\}$, as no more A-B paths exist after we removed the edges $S_1A_j$ and $S_1A_{j+1}$. Therefore we have $|\mathcal T_0| = |\mathcal T_x| + |\mathcal T_y| - 1 + \overline{S_1A_j} + \overline{S_1A_{j+1}}$.

    We further replace $\mathcal T_x$ with a Euclidean minimum spanning tree $\mathcal T_x'$ of $\{A_i\}$ such that the edge $A_jA_{j+1}$ is present in $\mathcal T_x'$. From \cite{du1987steiner}, we know that $|\mathcal T_x'| \le |\mathcal T_x|$. We now remove the edge $A_jA_{j+1}$ and add back the edges $S_1A_j$ and $S_1A_{j+1}$ which gives a connected tree $\mathcal T_0'$ of $\{A_i\} \cup \{B_i\}$. Therefore we have:
    $$|\mathcal T_0'| = |\mathcal T_x'| + |\mathcal T_y| - 1 + \overline{S_1A_j} + \overline{S_1A_{j+1}} \le |\mathcal T_x| + |\mathcal T_y| - 1 + \overline{S_1A_j} + \overline{S_1A_{j+1}} = |\mathcal T_0|$$

    This means $\mathcal T_0'$ must be an SMT. However, all polygon edges of polygon $\{A_i\}$ appearing in $\mathcal T_x'$ also appear in $\mathcal T_0'$ as well, except $A_jA_{j+1}$. Therefore, the SMT $\mathcal T_0'$ has $(n - 2)$ polygon edges of the polygon $\{A_i\}$.
\end{proof}

With these set of results in hand, we can now show that there exists an SMT of $\{A_i\} \cup \{B_i\}$ following a \emph{singly connected topology}. To show this, we start with any \smtpoly, $\mathcal T_0$, that satisfies all the results derived so far and transform it into a Steiner tree of \emph{singly connected topology} having total length not longer than the initial Steiner tree $\mathcal T_0$.

\begin{theorem} \label{final_proof}
There exists an \smtpoly following a singly connected topology for $n \ge 13$ and $\lambda \ge \lambda_{1}$, where
$$
% \lambda_{1} = \frac{2 \sin{\frac{\pi}{n}} + 1}{1 - 6 \sin \frac{\pi}{n}}
\lambda_{1} = \frac{1}{1 - 4 \sin \frac{\pi}{n}}
$$
\end{theorem}

\begin{proof}
% We know that there is a single path between two points in $\{A_i\}$ containing a Steiner point and it is of the form $A_jS_1A_{j+1}$ (from Lemma \ref{steiner_path_from_A_to_A}). The rest of the points in $\{A_i\}$ must be connected by $(n - 2)$ polygon edges of $\{A_i\}$.

Let $\mathcal{T}_0$ be any SMT of $\{A_i\} \cup \{B_i\}$ which satisfies the properties of Lemma \ref{n-2_A_poly_edges}. Further, from Lemma \ref{steiner_path_from_A_to_A}, there is a Steiner point $S_1$ which lies on all A-B paths, and there are two consecutive vertices $A_j$, $A_{j+1}$ such that $A_j$, $S_1$, $A_{j+1}$ is a path in $\mathcal T_0$. As $\mathcal T_0$ satisfies the property of Lemma \ref{n-2_A_poly_edges}, $\mathcal T_0$ has $(n - 2)$ polygon edges of $\{A_i\}$ excluding the edge $A_jA_{j+1}$.

Let $H$ be the point in the interior of the polygon $\{A_i\}$  such that $HA_jA_{j+1}$ form an equilateral triangle. As $n > 6$, the common centre $O$ of $\{A_i\}$ and $\{B_i\}$ does not lie inside the triangle $HA_jA_{j+1}$. Now, we modify $\mathcal T_0$ as follows:

\begin{enumerate}
    \item Remove edges $A_jS_1$, $S_1A_{j+1}$ and add edge $S_1H$ to get the forest $\mathcal{T}_1$. We know from \cite{hwang1992steiner} that $S_1$, $S_2$ and $H$ are collinear and this transformation does not change the total length. Therefore $|\mathcal{T}_0| = |\mathcal{T}_1|$. Here, $|\mathcal{T}_1|$ denotes the sum of the lengths of edges present in $\mathcal{T}_1$.
    \item Add edge $HO$ and remove all polygon edges of $\{A_i\}$ to get $\mathcal T_2$. Therefore $|\mathcal T_2| = |\mathcal T_1| + \overline{HO} - (n - 2) = |\mathcal T_0| + \overline{HO} - (n - 2)$. We observe that $\mathcal T_2$ is a tree connecting  the points in $\{B_i\} \cup \{O\}$.
    \item Let $S_0$ be the Torricelli point of the triangle $OB_jB_{j+1}$. Let $\mathcal T_3$ be the Steiner tree of $\{B_i\} \cup \{O\}$ with edges $S_0O$, $S_0B_j$, $S_0B_{j+1}$ and other points in $\{B_i\}$ connected through $(n - 2)$ polygon edges of the polygon $\{B_i\}$. From \cite{weng1995steiner}, we know that $\mathcal T_3$ is the SMT of $\{B_i\} \cup \{O\}$. Therefore $|\mathcal T_3| \le |\mathcal T_2| = |\mathcal T_0| + \overline{HO} - (n - 2)$. Further we know that $H$ lies on the edge $OS_0$ (as $O$, $S_0$ and $H$ lie on the perpendicular bisector of $B_j$ and $B_{j + 1}$).
    \item Remove edge $S_0O$ and add edge $S_0H$ to get $\mathcal T_4$. As $H$ lies on the edge $OS_0$, we have $|\mathcal T_4| = |\mathcal T_3| - \overline{OH} \le |\mathcal T_0| - (n - 2)$.
    \item Let $S_3$ be the intersection of the circumcircle of triangle $A_jHA_{j + 1}$ (from Lemma \ref{lambda_v} the intersection exists as $\lambda_1 \ge \lambda_v$ for $n \ge 13$). Remove the edge $S_3H$ and add the edges $S_3A_j$ and $S_3A_{j + 1}$ to get $\mathcal T_5$. Again, from \cite{hwang1992steiner} we know that this transformation does not change the total length. Hence $|\mathcal T_5| = |\mathcal T_5| \le |\mathcal T_0| - (n - 2)$. Moreover, as $\lambda > \lambda_v$, we observe that $\{A_j, B_j, A_{j+1}, B_{j+1}, S_3, S_0\}$ form the vertices of the vertical gadget and points $O$, $H$, $S_3$, $S_0$ appear in that order on the perpendicular bisector of $B_j$ and $B_{j+1}$.
    \item Add back the $(n - 2)$ polygon edges of $\{A_i\}$ which were removed in the second step to get $\mathcal T_6$. Therefore $|\mathcal T_4| = |\mathcal T_5| + (n - 2) \le |\mathcal T_0|$. We further observe that $\mathcal T_6$ is a Steiner tree connecting the points $\{A_i\} \cup \{B_i\}$ with a singly connected topology.
\end{enumerate}

Therefore we started with an arbitrary SMT $\mathcal T_0$ and transformed it into a Steiner tree $\mathcal T_6$ with a singly connected topology (where $\{A_j, B_j, A_{j+1}, B_{j+1}, S_3, S_0\}$ form the vertices of the vertical gadget) which has a total length not worse than $\mathcal T_0$. Hence $\mathcal T_6$ must be an SMT of $\{A_i\} \cup \{B_i\}$. This proves the theorem.
\end{proof}

\begin{remark}
    \cref{final_proof} determines the exact structure of the \smtpoly. Further from~\cref{trapezoids} we determine the exact method to construct the two additional Steiner points in $\mathcal O(1)$ steps - note that this construction time is independent of the integer $n$ or the real number $\lambda$. Therefore, \smtpoly for $n \ge 13$ and $\lambda \ge \lambda_1$ is solvable in polynomial time.
\end{remark}

%This means that there is an SMT of $\{A_i\} \cup \{B_i\}$ which will have a \emph{singly connected topology} for $n \ge 13$ and $\lambda \ge \lambda_{1}$. \\

Note that the total length of any \smtpoly, when $n \ge 13$ and $\lambda \ge \lambda_{1}$, is
\begin{align*}
    & |\mathcal T_6| = |\text{vertical gadget}| + |(n - 2) \text{ edges of $\{B_i\}$}| + |(n - 2) \text{ edges of $\{A_i\}$}| \\
    \implies & |\mathcal T_6| = \bigg(\dfrac{(\lambda - 1)}{2 \tan \frac \pi n} + \dfrac {\sqrt 3 (\lambda + 1)} {2}\bigg) + (n - 2) \cdot \lambda + (n - 2)\\
    \implies & {|\mathcal T_6| = \dfrac{(\lambda - 1)}{2 \tan \frac \pi n} + \bigg(n - 2 + \frac{\sqrt{3}}{2}\bigg) (\lambda + 1)}\\
\end{align*}
\noindent Further, $\lambda_1$ converges to 1 very quickly with increasing $n$ (plotted in Figure \ref{lambda_1_plot}):

    \vspace{0.3cm}
    \begin{center}
    \begin{tabular}{| c | c | c | c | c | c |}
    \hline
    $n$ & 13 & 20 & 40 & 100 & 500 \\
    \hline
    $\lambda_1$ & 23.3987 & 2.6719 & 1.4574 & 1.1437 & 1.0258\\
    \hline
    \end{tabular}
    \end{center}
    \vspace{0.3cm}

    
\begin{figure}[h]
\centering
\includegraphics[width=15cm]{lambda_1_and_lambda_v.png}
\caption{Plot of $\lambda_1$ \& $\lambda_v$ against $n$}
\label{lambda_1_plot}
\end{figure}

    
This means for large sized $n$ and for ratios that are not too small, the SMT will follow a \textit{singly connected topology}. 

% \begin{Lemma}
%     The SMT of concentric regular polygons $\{A_i\}$ and $\{B_i\}$ with $n \ge 13$ and $\lambda \ge \lambda_{1}$, all ``connections'' between $\{A_i\}$ and $\{B_i\}$ will be full Steiner subtrees forming vertical gadgets. 
% \end{Lemma}

% \begin{proof}
%     Clearly there cannot be any Horizontal Forks appearing as Horizontal Gadgets as $\lambda \ge \lambda_{0} \ge \lambda_s \ge \lambda_v$ and we can replace the Horizontal Gadget with a vertical gadget to get an even smaller length.
    
%     For any other topology of the full Steiner subtree connecting $\{A_i\}$ and $\{B_i\}$ with only one edge needed to remove in order to disconnect $\{A_i\}$ and $\{B_i\}$ (and the rest of the tree connected via polygon edges), there must exist terminals $A_j, A_{j + 1}$ and a steiner point $S_0$ such that $A_jS_0A_{j+1}$ is a path in the SMT. Let $O$ be the common center of $\{A_i\}$ and $\{B_i\}$. We remove all the edges in the SMT of that were a edges of the inner polygon, as well as the edges $A_jS_0$ and $S_0A_{j+1}$. We then add the edge $S_0O$ Since this was the SMT for $\{A_i\} \cup \{B_i\}$, this new construction gives us an SMT for $\{B_i\} \cup \{O\}$, contradicting the main result of \cite{weng1995steiner}.

%     \begin{figure}[h]
%         \centering
%         \includegraphics[width=15cm]{vert_proof.png}
%         \caption{Contradiction of \cite{weng1995steiner}}
%         \label{fig:vert_proof}
%     \end{figure}
    
% \end{proof}

% This implies that whenever $n \ge 12$ and $\lambda > \lambda_0$, the SMT follows the Singly Connected topology. Further $\lambda_0$ quickly converges to 1:

%     \vspace{0.3cm}
%     \begin{center}
%     \begin{tabular}{| c | c | c | c | c | c |}
%     \hline
%     $n$ & 12 & 20 & 40 & 100 & 500 \\
%     \hline
%     $\lambda_0$ & 2.732 & 1.627 & 1.243 & 1.086 & 1.017\\
%     \hline
%     \end{tabular}
%     \end{center}
%     \vspace{0.3cm}
    
% This means for large sized $n$ and for ratios that are not too small, the SMT will follow the Singly connected topology. 

%Few simulated examples (with the help of \href{http://www.geosteiner.com/}{[2]}) are as follows

% \begin{figure}[h!]
% \centering
% \subfloat[\centering $\lambda = 2.73205$] {\includegraphics[width=5cm]{12_vert.png}}
%     \qquad
% \subfloat[\centering $\lambda = 2.73206$] {\includegraphics[width=5cm]{12_sing.png}}
% \caption{points for $n = 12$, $\lambda = 2.732050$}
% \end{figure}


% \begin{figure}[h!]
% \centering
% \subfloat[\centering $\lambda = 1.626797$] {\includegraphics[width=5cm]{20_vert.png}}
%     \qquad
% \subfloat[\centering $\lambda = 1.626798$] {\includegraphics[width=5cm]{20_sing.png}}
% \caption{points for $n = 20$, $\lambda = 1.6267974$}
% \end{figure}
