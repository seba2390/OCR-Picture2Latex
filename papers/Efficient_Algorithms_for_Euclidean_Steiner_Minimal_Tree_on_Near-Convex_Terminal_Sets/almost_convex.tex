


\section{\ESMT on $f(n)$-Almost Convex Point Sets}\label{sec:exact_algo}


% \begin{figure}[h]
% \centering
% \subfloat[\centering SMT for some $\mathcal P$ with $n = 6$] {\includegraphics[width=6cm]{SMT_6.png}}
%     \qquad
% \subfloat[\centering \centering SMT for some $\mathcal P$ with $n = 10$] {\includegraphics[width=6cm]{SMT_10.png}}
% \caption{Some examples of SMT}
% \end{figure}

In this section, we design an exact algorithm for \ESMT on $f(n)$-Almost Convex Point Sets running in time $2^{\OO(f(n)\log n)}$. Note that $f(n) \leq n$ is always true. Therefore, we are given as input a set $\mathcal{P}$ of $n$ points in the Euclidean Plane such that $\mathcal P$ can be partitioned as $\mathcal P = \mathcal P_1 \uplus \mathcal P_2$, where $\mathcal P_1$ is the convex hull of $\mathcal P$ and $|\mathcal P_2| = f(n)$.

First, we look into some mathematical results and computational results to finally arrive at the algorithm for solving \ESMT on $f(n)$-Almost Convex Point Sets. 

We know that the SMT of $\mathcal P$ can be decomposed uniquely into one or more full Steiner subtrees, such that two full Steiner subtrees share at most one node~\cite{hwang1992steiner}. In the following lemma, we further characterize one full Steiner subtree.

\begin{lemma}
\label{existence_of_leaf}
Let $\mathbb{F}$ be the full Steiner decomposition of an SMT of $\mathcal{P}$.  Then there exists a full Steiner subtree $\mathcal F \in \mathbb{F}$ such that $\mathcal F$ has at most one common node with at most one other full Steiner subtree in $\mathbb{F}$.
\end{lemma}
\begin{proof}
If the SMT of $\mathcal P$ is a full Steiner tree, then the statement is trivially true.

Otherwise, we assume that the SMT of $\mathcal P$ has full Steiner subtrees, $\mathbb{F} = \{\mathcal F_1, \mathcal F_2, \ldots, \mathcal F_m\}$, $m \ge 2$. Now, for the sake of contradiction, we assume that for each full Steiner subtree $\mathcal F_j$ there are atleast two other full Steiner subtrees ${\sf Neitree}(\mathcal F_j)^1,{\sf Neitree}(\mathcal F_j)^2 \in \mathbb{F}$ and two terminals $P_1(\mathcal F_j),P_2(\mathcal F_j) \in \mathcal{P}$ such that $P_i(\mathcal F_j) \in V(\mathcal F_j) \cap V({\sf Neitree}(\mathcal F_j)^i)$, $i \in \{1,2\}$. Now, let us construct a walk $W$ in the SMT of $\mathcal{P}$. Starting from $P_1(\mathcal F_1)$ of the full Steiner subtree $\mathcal F_1 \in \mathbb{F}$, we include the path in $\mathcal F_1$ connecting to $P_2(\mathcal F_1)$. Note that $P_2(\mathcal F_1)$ is also contained in ${\sf Neitree}(\mathcal F_1)^2$. Let ${\sf Neitree}(\mathcal F_1)^2 = \mathcal F_{w_1}$ for some $w_1\in [m], w_1\neq 1$. Also let $P_2(\mathcal F_1) = P_1(\mathcal F_{w_1})$.
Then, we know that there is a $P_2(\mathcal F_{w_1})$. In $W$, we include the path in $\mathcal F_{w_1}$ connecting $P_1(\mathcal F_{w_1})$ to $P_2(\mathcal F_{w_1})$. In general, suppose the $i^{th}$ full Steiner subtree to be considered in building the walk is $\mathcal F_{w_{i-1}}$ which was reached via point $P_1(\mathcal F_{w_{i-1}})$. Then we include in $W$ the path in $\mathcal F_{w_{i-1}}$ connecting $P_1(\mathcal F_{w_{i-1}})$ and $P_2(\mathcal F_{w_{i-1}})$. Thus, we can indefinitely keep constructing the walk $W$ as for each $\mathcal F_{w_{i-1}}$ both $P_1(\mathcal F_{w_{i-1}}), P_2(\mathcal F_{w_{i-1}})$ always exist. However, since there are $m$ full Steiner subtrees this means that there is an $\mathcal F_k \in \mathbb{F}$ and two indices $i\neq j$ such that $\mathcal F_k = \mathcal F_{w_i} = \mathcal F_{w_j}$. Thus, there exists a cycle in $W$, which implies that there is a cycle in the SMT of $\mathcal{P}$ (contradiction). Therefore, there must be at least one full Steiner subtree that has at most one common terminal with at most one other full Steiner subtree.
\end{proof}
\begin{remark}
A full Steiner subtree of the SMT of $\mathcal P$ has the topology of a tree. Thus, from Lemma~\ref{existence_of_leaf}, we conclude that a full Steiner subtree, that has at most one common terminal with at most one other full Steiner subtree, has at least one leaf of the SMT.
\end{remark}


\begin{definition}
Let the full Steiner subtrees, that have at most one terminal shared with at most one other full Steiner subtree, be called \textbf{leaf full Steiner subtrees}. Let the terminal which is shared be called the \textbf{pivot} of the leaf full Steiner subtree.
\end{definition}


\begin{figure}[h]
\centering
\subfloat[\centering Full Steiner Subtrees of Figure 1(a)] {\includegraphics[width=6cm]{sub_6.png}}
    \qquad
\subfloat[\centering \centering Full Steiner Subtrees of Figure 1(b)] {\includegraphics[width=6cm]{sub_10.png}}
\caption{Leaf full Steiner subtrees enclosed in ellipses, other full Steiner subtrees enclosed in rectangles, pivots of leaf full Steiner subtrees encircled}
\end{figure}


\begin{lemma}
\label{leaf_non_leaf_structure}
Let $\mathcal F$ be a leaf full Steiner Subtree of the SMT of $\mathcal P$, with terminal points $\mathcal P_{\mathcal{\mathcal F}} \subseteq \mathcal{P}$ and having pivot $P_{\mathcal F}$. Deleting $\mathcal F\setminus \{P_{\mathcal F}\}$ from the SMT of $\mathcal P$ gives us an SMT of the terminal points $((\mathcal P - \mathcal{P}_{\mathcal F}) \cup \{P_{\mathcal F}\})$.
\end{lemma}

\begin{proof}
Firstly, we observe that deleting $\mathcal F\setminus \{P_{\mathcal F}\}$ from the SMT of $\mathcal P$ will indeed give us a tree, as $\mathcal F$ is a leaf full Steiner subtree. Let us call this tree $\mathcal Y$.

Now for the sake of contradiction, we assume that the total length of $\mathcal Y$ is strictly larger than the SMT $\mathcal F'$ of $((\mathcal P - \mathcal{P}_{\mathcal F}) \cup \{P_{\mathcal F}\})$. However, this means, the total length of $\mathcal F' \cup \mathcal F$ is strictly smaller than that of the SMT of $\mathcal P$. As $\mathcal F' \cup \mathcal F$ is also a Steiner tree of $\mathcal P$,  this contradicts the minimality of the initial SMT of $\mathcal P$.
\end{proof}

Now we are ready to describe the algorithm. Recall that $\mathcal P$ is partitioned as $\mathcal P = \mathcal{P}_1 \uplus \mathcal{P}_2$, where $\mathcal{P}_1$ is the convex hull of $\mathcal P$ and $\mathcal{P}_2$ is the set of $f(n)$ points lying in the interior of $\mathcal P_1$. For the sake of brevity of notations let $|\mathcal{P}_2| = k$.

\begin{lemma}
\label{four_power_n}
Let $\mathcal P$ be a $k$-Almost Convex Point Set. A minimum FST of a subset $\mathcal S$ of $\mathcal P$ can be found in $\mathcal O(4^{|{\mathcal S}|} \cdot |\mathcal S| ^ k)$ time.
\end{lemma}

\begin{proof}
We observe that for any $\mathcal S \subseteq \mathcal P$, $\mathcal S$ forms a convex polygon with at most $k$ points lying in the interior. For $|{\mathcal S}| \le 2$, the statement of the lemma is trivially true. Hence we assume that $|{\mathcal S}| > 2$.

From \cite{hwang1992steiner}, the number of full Steiner topologies of $\mathcal{S}$ is 

$$\frac{|\mathcal S|!}{|\mathrm{CH}(\mathcal{S})|!} \cdot \frac {\binom{2|{\mathcal S}| - 4}{|{\mathcal S}| - 2}}{|{\mathcal S}| - 1}$$

However, we know that:

$$\frac {\binom{2|{\mathcal S}| - 4}{|{\mathcal S}| - 2}}{|{\mathcal S}| - 1} < \frac{\binom{2|{\mathcal S}|}{|{\mathcal S}|}}{|{\mathcal S}|} < \frac{\sum \limits_{r = 0}^{2|{\mathcal S}|} \binom{2|{\mathcal S}|}{k}}{|{\mathcal S}|} = \frac{2 ^{2|{\mathcal S}|}}{|{\mathcal S}|} = \frac{4^{|{\mathcal S}|}}{|{\mathcal S}|}$$

And,

$$\frac{|\mathcal S|!}{|\mathrm{CH}(\mathcal{S})|!} < \frac{|\mathcal S|!}{(|\mathcal S| - k)!} < |\mathcal S|^k$$

Therefore, the number of full Steiner topologies of $\mathcal{S}$ is at most $4^{|{\mathcal S}|} |\mathcal S| ^ {k - 1}$. Each of these topologies can be enumerated and using \emph{Melzak's FST Algorithm}, we can also find the SMT realizing each such full Steiner topology in linear time, as given in \cite{hwang1992steiner}. Therefore to iterate over all topologies and find a minimum takes at most time:

$$(4^{|{\mathcal S}|} \cdot |\mathcal S| ^ {k - 1}) \cdot \mathcal O(|\mathcal S|) = \mathcal \mathcal O(4^{|{\mathcal S}|} \cdot|\mathcal S| ^ k)$$

\end{proof}

Now, we find the time required for extending the results of Lemma~\ref{four_power_n} to all subsets of $\mathcal{P}$.
\begin{lemma}
\label{five_power_n}
Let $\mathcal P$ be a $k$-Almost Convex Set. Computing a minimum FST \textbf{for all} subsets $\mathcal S \subseteq \mathcal P$ can be done in $\mathcal O(n^k \cdot 5^{n})$ time.
\end{lemma}
\begin{proof}

Using Lemma \ref{four_power_n}, we can get a minimum FST for a single subset $\mathcal S \subseteq \mathcal P$ in $\mathcal O(4^{|{\mathcal S}|} |\mathcal S| ^ k)$ time. Moreover, we know that the number of subsets of $\mathcal P$ that are of size $r$ is $\binom n r$. This means that the total time to compute a minimum FST \textbf{for all} subsets $\mathcal S \subseteq \mathcal P$, time taken is: 


$$\sum \limits_{r = 0} ^ {n} \binom{n}{r} \cdot \mathcal O(4^r \cdot r^k) = \sum \limits_{r = 0} ^ {n} \binom{n}{r} \cdot \mathcal O(n^k \cdot 4^r) = \mathcal O(n^k \cdot (1 + 4)^n) = \mathcal O(n^k \cdot 5^n)$$

\end{proof}

For each $\mathcal S \subseteq \mathcal P$, we denote by $\mathcal F_{\mathcal S} $ a minimum FST of $\mathcal S$ and by $\mathcal T_{\mathcal S} $ the SMT of $\mathcal S$.

\begin{lemma}
\label{single_subset_SMT}
The SMT of subset $\mathcal S \subseteq \mathcal P$, $\mathcal T_{\mathcal S}$, can be found in $\mathcal O(|{\mathcal S}| \cdot 2^{|{\mathcal S}|})$ time, given that we have pre-computed $\mathcal T_{\mathcal R}$ and $\mathcal F_{\mathcal R}$, $\forall \mathcal R \subseteq \mathcal S$.
\end{lemma}

\begin{proof}

If $\mathcal T_{\mathcal S}$ was a full Steiner tree then it would be $\mathcal F_{\mathcal S}$. Otherwise, $\mathcal T_{\mathcal S}$ contains multiple full Steiner subtrees.

Let $\mathcal F$ be a leaf full Steiner subtree of $\mathcal T_{\mathcal{S}}$ with pivot $P_{\mathcal F}$. Therefore from Lemma \ref{leaf_non_leaf_structure} we have $\mathcal T_{\mathcal S} = \mathcal T_{((\mathcal S - V(\mathcal F)) \cup \{P_{\mathcal F}\})} \cup \mathcal F$. Therefore we can iterate over all subsets $\mathcal R \subset \mathcal S$ and all terminals $P \in \mathcal{R}$, and take the minimum-length tree among $\mathcal T_{((\mathcal S - \mathcal R)  \cup \{P\})} \cup \mathcal F_{\mathcal R}$. Since we are iterating over all $\mathcal R \subset \mathcal S$, and all $P \in \mathcal R$, we are guaranteed to get $\mathcal R = V(\mathcal F) \cap \mathcal{S}$ and $P = P_{\mathcal F}$ on one such iteration.

Now, as there are $\mathcal O(2^{|\mathcal S|})$ possibilities of $\mathcal R \subset \mathcal S$ and $\mathcal O(|\mathcal S|)$ possibilities of $P \in \mathcal R$, we have $\mathcal O(|S| \cdot 2^{|\mathcal S|})$ possibilities of the pair $(\mathcal R, P)$. Therefore the total time required for iterating is $\mathcal O(|{\mathcal S}| \cdot 2^{|{\mathcal S}|})$.

\end{proof}

\begin{lemma}
\label{multi_subset_SMT}
SMTs for all subsets $\mathcal S \subseteq \mathcal P$, $\mathcal T_{\mathcal S}$ can be found in $\mathcal O(n \cdot 3 ^n)$ time, given that we have precomputed a minimum FST $\mathcal F_{\mathcal S}$ $\forall \mathcal S \subseteq \mathcal P$.
\end{lemma}

\begin{proof}

Using Lemma \ref{single_subset_SMT}, we can get the SMT $\mathcal T_{\mathcal S}$, for a single subset $\mathcal S \subseteq \mathcal P$ in $\mathcal O(r \cdot 2^r)$ time, where $|{\mathcal S}| = r$. Moreover, we know that the number of subsets of $\mathcal P$ that are of size $r$ is $\binom n r$. This means that the total time to compute the SMT for all subsets $\mathcal S \subseteq \mathcal P$, time taken is: 

$$\sum \limits_{k = 0} ^ {n} \binom{n}{k} \cdot \mathcal O(k \cdot 2^k) = \mathcal O(n \cdot (1 + 2)^n) = \mathcal O(n \cdot 3^n)$$

However, to apply Lemma \ref{single_subset_SMT} on some subset $\mathcal S \subseteq \mathcal P$ for computing $\mathcal T_{\mathcal S}$, we must also have $\mathcal T_{\mathcal R}$ precomputed for all $\mathcal R \subseteq \mathcal S$. This can be guaranteed by computing $\mathcal T_{\mathcal S}$ and $\mathcal{F}_{\mathcal{S}}$, for all subsets $\mathcal S \subseteq \mathcal P$ in an increasing order of $|\mathcal S|$ (or any order which guarantees that the subsets of $\mathcal S$ are processed before $\mathcal S$).

\end{proof}

Finally, we state our algorithm.

\begin{theorem}
\label{final_theorem}

An SMT $\mathcal T_{\mathcal P}$ of a $k$-Almost Convex Set $\mathcal P$ of terminals can be computed in $\mathcal O(n^k \cdot 5^n)$ time.

\end{theorem}

\begin{proof}

Consider the following algorithm: 
\begin{algorithm}[H]
\caption{Computation of $\mathcal{T_P}$ ~~~ \textbf{Input:} $\mathcal P$ }\label{alg:main algo}
\begin{algorithmic}[1]
\For{all $\mathcal S \subseteq \mathcal P$} 
\State Compute $\mathcal {F_S}$ \Comment{Using Lemma \ref{five_power_n}}
\EndFor \Comment{This takes $\mathcal O(n^k \cdot 5^n)$ time}
\For{all $\mathcal S \subseteq \mathcal P$}
\State Compute $\mathcal {T_S}$ \Comment{Using Lemma \ref{multi_subset_SMT}}
\EndFor \Comment{This takes $\mathcal O(n \cdot 3^n)$ time}
\State \Return $\mathcal{T_P}$ \Comment{Total runtime is $\mathcal O(n^k \cdot 5^n + n \cdot 3^n) = \mathcal O(n^k \cdot 5^n)$}
\end{algorithmic}
\end{algorithm}


% \begin{itemize}
%     \item Using Theorem \ref{five_power_n}, we precompute $\mathcal F_{\mathcal S}$ for all subsets $\mathcal S  \subseteq \mathcal P$. This takes $\mathcal O(5^n)$ time.
%     \item Using Theorem \ref{multi_subset_SMT} and the precomputation from the previous step, we can get $\mathcal T_{\mathcal S}$ for all subsets $\mathcal S  \subseteq \mathcal P$. This takes $\mathcal O(n \cdot 3^n)$ time.
%     \item We return $\mathcal T_{\mathcal P}$ as our final answer in time $\mathcal O(5^n + n \cdot 3^n) = \mathcal O(5^n)$.
% \end{itemize}

Hence we have an SMT of a $k$-Almost Convex Point Set $\mathcal P$ in $\mathcal O(n^k\cdot 5^n)$ time.


\end{proof}

% \vspace{0.5cm}
% \section{Extending the Algorithm}
% \vspace{0.5cm}

%     We now consider the configuration where there are a small number of points inside a convex polygon. Formally, we wish to find the SMT of $\mathcal{P} \cup \mathcal{P}_{in}$, where $\mathcal{P}_{in}$ is a set of $k$ points lying strictly inside the convex polygon $\mathcal{P}$, such that $k$ is small.\\
    
%     We can apply \textbf{Algorithm 1} (derived in \ref{final_theorem}) in this case as well. However, the analysis of run time does not remain the same. We will now analyse the runtime of the same algorithm when the input is $\mathcal{P}_1 = \mathcal{P} \cup \mathcal{P}_{in}$, to get its SMT. 
    
%     \begin{align*}
%         \text{The number of Full Steiner Topologies of } \mathcal{P}_1 = & \frac{|\mathcal{P}_1|!}{|\text{convexhull}(\mathcal{P}_1)|!} \cdot \Bigg(\frac {\binom{2|{\mathcal P_1}| - 4}{|{\mathcal P_1}| - 2}}{|{\mathcal P_1}| - 1}\Bigg)\\
%         < & \frac{(n + k)!}{|\mathcal{P}|!} \cdot \frac{4^{|\mathcal P_1|}}{|\mathcal P_1|}\\
%         = & \frac{(n + k)!}{n!} \cdot \frac{4^{(n + k)}}{n + k}\\
%     \end{align*}
    
%     Similarly for each subset $\mathcal S_1 \subset \mathcal{P}_1$ of size $m$ has number of Full Steiner topologies atmost to $\Big( \dfrac{m!}{(m - k)!} \cdot \dfrac{4^m}{m} \Big)$ as the number of points strictly inside the convex hull cannot increase. And hence running Melzak's Full Steiner Tree Algorithm for each such topology would take time of $\mathcal O \Big(\frac{m!}{(m - k)!} \cdot \frac{4^m}{m} \cdot m \Big)$ = $\mathcal O((n + k)^k \cdot 4^m)$.\\

%     Hence running \textbf{Algorithm 1} on all subsets of $\mathcal P_1$ would take time:
    
%     $$\sum \limits_{m = 0}^{n + k}{(n + k) ^ k \cdot 4^m \cdot \binom{n + k}{m}} = \mathcal O ((n + k) ^ k \cdot 5^{(n + k)})$$
    
%     Further, for the part of the algorithm performing the computation of $\mathcal T_S$, $\forall \mathcal{S} \subseteq \mathcal P_1$ (as in \ref{multi_subset_SMT}), the runtime would not exceed $\mathcal O (4 ^ {n + k})$.\\
    
%     Therefore the runtime of the entire algorithm, when ran on $\mathcal P_1$, will be $\mathcal O ((n + k) ^ k \cdot 5^{(n + k)})$. \\

The above theorem gives us several improvements in special classes of inputs, based on the number of input points lying inside the convex hull of the input set, as described in the following corollary. Let there be an $f(n)$-Almost Convex Point Set $\mathcal P$ containing $n$ points. Recall that $\mathcal{P} = \mathcal{P}_1 \uplus \mathcal{P}_2$, $\mathcal{P}_1$ containing the points on the convex hull of $\mathcal{P}$, and $|\mathcal{P}_2| = f(n)$. It is only possible that $f(n) \leq n$. 

\begin{corollary}\label{almost-better}
Let $\mathcal P$ be a $f(n)$-Almost Convex Point Set. Then, then there is an algorithm $\mathcal{A}$ for \ESMT such that, $\mathcal{A}$ runs in $2^{\OO{(n + f(n) \log n)}}$ time. In particular, 
\begin{enumerate}
    \item When $f(n) = \OO(n)$, $\mathcal A$ runs in $2^{\OO(n\log n)}$ time.
    \item When $f(n) = \Omega(\frac{n}{\log n})$ and $f(n) = o(n)$, $\mathcal{A}$ runs in $2^{o(n\log n)}$.
    \item When $f(n) = \OO(\frac{n}{\log n})$, $\mathcal A$ runs in $2^{\OO(n)}$ time.
  \end{enumerate}
\end{corollary}

Therefore, for $f(n) = o(n)$, our algorithm for \ESMT does better on $f(n)$-Almost Convex Points Sets than the current best known algorithm~\cite{hwang1986linear}.
    
%    If we consider $k$ to be a constant, the runtime would be $\mathcal O (n ^ k \cdot 5^{n})$. With a more generalized assumption of $k = o(n)$, we get the runtime to be $\mathcal O (n ^ k \cdot 5^{n}) = 2^{(\mathcal O(n) + o(n) \log_2(n))} = 2 ^ {o(n \log n)}$.\\
    
%    \begin{remark}
%        This algorithm when run on the arbitrary positioned points as input, would get number of Full Steiner Topologies to be much higher  causing the runtime to be $2 ^ {\mathcal O(n \log n)}$ instead. 
%    \end{remark} 
