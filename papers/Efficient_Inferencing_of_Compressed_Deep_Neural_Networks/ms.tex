\documentclass[conference,10pt]{IEEEtran}  

%\usepackage{fullpage}
\usepackage{array}
\usepackage{amssymb}
\usepackage{amsmath}
\usepackage{graphicx}
\usepackage{xcolor}
\usepackage{subfig}
\usepackage{boxedminipage}
\usepackage{hyperref}
\usepackage[ruled,vlined]{algorithm2e}
\usepackage{algorithmic}
\usepackage{balance}
\usepackage{amsmath}
\usepackage{xy}
\xyoption{all}

\newtheorem{theorem}{\bf Theorem}[section]
\newtheorem{corollary}[theorem]{\bf Corollary}
\newtheorem{lemma}[theorem]{\bf Lemma}
\newtheorem{fact}[theorem]{\bf Fact}
\newtheorem{definition}[theorem]{\bf Definition}
\newtheorem{proposition}[theorem]{\bf Proposition}
\newtheorem{claim}[theorem]{\bf Claim}
\newtheorem{observation}[theorem]{\bf Observation}

\newcommand{\qed} {\hfill$\Box$}
\newcommand{\eat}[1] {}
\newcommand{\myproof} {\IEEEproof}
\newcommand{\calP} {{\cal P}}
\newcommand{\calA} {{\cal A}}
\newcommand{\calE} {{\cal E}}
\newcommand{\calT} {{\cal T}}
\newcommand{\calH} {{\cal H}}
\newcommand{\calC} {{\cal C}}
\newcommand{\calD} {{\cal D}}
\newcommand{\Acc} {{\rm Acc}}
\newcommand{\Inst} {{\rm Inst}}
\newcommand{\wh}[1] {\widehat{#1}}
\newcommand{\wt}[1] {\widetilde{#1}}
\newcommand{\opt} {{\tt Opt}}
\newcommand{\LCA} {{\rm LCA}}
\newcommand{\mypath} {{\rm path}}
\newcommand{\val} {{\rm val}}
\newcommand{\mypred} {{\rm pred}}
\newcommand{\mysucc} {{\rm succ}}
\newcommand{\TMIS} {{\rm Time(MIS)}}
\newcommand{\pair}[2] {\langle #1,#2 \rangle}
\newcommand{\BuildBTD} {{\tt BuildBalTD}}
\newcommand{\BuildITD} {{\tt BuildIdealTD}}
\newcommand{\ceil}[1] {\lceil #1 \rceil}
\newcommand{\floor}[1] {\lfloor #1 \rfloor}
\newcommand{\rt} {{\rm rt}}
\newcommand{\dl} {{\rm dl}}
\newcommand{\len} {{\rm len}}
\newcommand{\depth} {{\rm depth}}
\newcommand{\mymid} {{\rm mid}}
\newcommand{\calF} {{\cal F}}
\newcommand{\idx} {{\rm idx}}
%\newcommand{\color} {{\rm color}}
\newcommand{\Bmax} {B_{\max}}
\newcommand{\Bmin} {B_{\min}}
\newcommand{\hmax} {h_{\max}}
\newcommand{\hmin} {h_{\min}}
\newcommand{\pmax} {p_{\max}}
\newcommand{\pmin} {p_{\min}}
\newcommand{\IS} {{\rm IS}}

\title{Efficient Inferencing of Compressed Deep Neural Networks}
%\author{
%\IEEEauthorblockN {---}
%\IEEEauthorblockA{IBM Research, New Delhi, India.}
%}

\author{
	\IEEEauthorblockN{
		Dharma Teja Vooturi \IEEEauthorrefmark{1},
		Saurabh Goyal\IEEEauthorrefmark{2}
		Anamitra R. Choudhury\IEEEauthorrefmark{2},
		Yogish Sabharwal\IEEEauthorrefmark{2},
		Ashish Verma\IEEEauthorrefmark{2}
	}\vspace{1mm}
	\IEEEauthorblockA{
		\hspace{1mm}
		\IEEEauthorrefmark{1}
		IIIT Hyderabad, India \hspace{9mm}
		Email: dharmateja.vooturi@research.iiit.ac.in
	}
	\IEEEauthorblockA{
		\hspace{30mm}
		\IEEEauthorrefmark{2}		
		IBM Research - India\hspace{10mm}
		Email: \{sgoyal30, anamchou, ysabharwal, vashish\}@in.ibm.com
	}
}



\begin{document}

\maketitle              

\begin{abstract}
Large number of weights in deep neural networks makes the models difficult to be deployed in low memory environments such as, mobile phones, IOT edge devices as well as ``inferencing as a service" environments on cloud. Prior work has considered  reduction in the size of the models, through compression techniques like pruning, quantization, Huffman encoding etc. However, efficient inferencing using the compressed models has received little attention, specially with the Huffman encoding in place.  In this paper, we  propose efficient parallel algorithms for inferencing of single image and batches, under various memory constraints.  
Our experimental results show that our approach of using variable batch size for inferencing 
achieves 15-25\%   performance improvement  in the inference throughput for AlexNet, 
while maintaining  memory and latency constraints.


\end{abstract}

\section{Introduction}
Reinforcement learning has achieved great success in areas such as Game-playing \citep{silver2018general,vinyals2019grandmaster}, robotics \cite{kober2013reinforcement}, large language models \citep{ouyang2022training}, etc.
However, due to safety concerns or physical limitations, in some real-world reinforcement learning problems, we must consider additional constraints that may influence the optimal policy and the learning process \citep{garcia2015comprehensive}.
% For example, a robotic arm must not take actions that may cause harm to itself or the environments.
A standard framework to handle such cases is the constrained Markov Decision Process (CMDP) \citep{altman1999constrained}.
Within the CMDP framework, the agent has to maximize
the expected cumulative reward while
obeying a finite number of constraints, which are usually in the form of expected cumulative cost criteria.

However, we are sometimes concerned with the problem with a continuum of constraints.
For example,
the constraints we meet might be time-evolving or subject to uncertain parameters, which
cannot be formulated as an ordinary CMDP
(see Examples \ref{Example_Time_Evolving} and  \ref{Example_Uncertain}).
In this paper we would study a generalized CMDP  
to address the above problem.  Because the constraints are not only infinite-number but also lie
in a continuous set,
the generalization is not trivial. Fortunately, we find that we can borrow the idea behind semi-infinite programming (SIP) \citep{remez1934determination, hettich1993semi} to deal with the semi-infinite constraints.
Accordingly, we propose \emph{semi-infinitely constrained Markov decision processes} (SICMDPs)
as a novel complement to the ordinary CMDP framework.
%More specifically,  an SICMDP model %, we consider 
%contains a continuum of constraints whereas an ordinary CMDP contains a finite number of constraints. 

%This generalization is natural but not trivial. However, we can brows the idea  
%The idea is quite natural and can be backtracked
%to the practice of extending linear programming to linear semi-infinite programming (LSIP) %\cite{remez1934determination, GobernaLSIO1998}.
%In addition, 
%As a complementary approach to the ordinary CMDP framework, 
%SICMDP can be used to model these problems  which cannot be described by a finite number of constraints
%that are not covered by .
%For example,
%the restrictions we consider can be time-evolving or subject to uncertain parameters
%, thus
%cannot be described by a finite number of constraints but a continuum of constraints 
%(see Examples \ref{Example_Time_Evolving} and  \ref{Example_Uncertain}).

We also present two reinforcement learning algorithms to solve SICMDPs called SI-CRL and SI-CPO, respectively.
SI-CRL is a model-based reinforcement learning algorithm designed for tabular cases, and SI-CPO is a policy optimization algorithm for non-tabular cases.
% and analyze its performance both theoretically and empirically.
The main challenge is that we need to deal with a continuum of constraints, thus reinforcement learning algorithms for ordinary CMDPs do not work anymore.
In SI-CRL, we tackle this difficulty by first transforming the reinforcement learning problem to an equivalent LSIP problem, which can then be solved using methods in the LSIP literature like the dual exchange methods \citep{Hu1990,reemtsen1998numerical}.
In SI-CPO, we resort to the idea of cooperative stochastic approximation developed in \cite{lan2020algorithms, wei2020comirror}.
As far as we know, we are the first to introduce tools from semi-infinitely programming (SIP) into the reinforcement learning community for solving constrained reinforcement learning problems.

% To the best of our knowledge, we are the first to apply tools from semi-infinitely programming (SIP) to solve reinforcement learning problems.
Furthermore, we give theoretical analysis for both SI-CRL and SI-CPO.
We decompose the error of SI-CRL into two parts: the statistical error from approximating the true SICMDP with an offline dataset and the optimization error due to the fact that the solution of the LSIP problem obtained by the dual exchange method is inexact.
On the optimization side, we show that the iteration complexity of SI-CRL is $O\left(\left\{\mathrm{diam}(Y)L\sqrt{|\gS|^2|\gA|m}/\left[(1-\gamma)\epsilon\right]\right\}^m\right)$.
On the statistical side, we show that the sample complexity of SI-CRL is $\widetilde O\left(\frac{|S|^2|A|^2}{\epsilon^2(1-\gamma)^3}\right)$ if the offline dataset is generated by a generative model, and $\widetilde O\left(\frac{|S||A|}{\nu_{\min} \epsilon^2(1-\gamma)^3}\right)$ if the dataset is generated by a probability measure $\nu$ as considered in \cite{chen2019information}.
Here $\widetilde O$ means that all logarithm terms are discarded.
For SI-CPO, things become a little more complicated because other than the statistical error and the optimization error, we also need to consider the function approximation error, which comes from imperfect policy parametrizations.
It is shown if the function approximation error can be controlled to $O(\epsilon)$ order, the iteration complexity of SI-CPO is $\widetilde{O}\left(\frac{1}{\epsilon^2(1-\gamma)^6}\right)$ and the sample complexity of SI-CPO is $\widetilde{O}(\frac{1}{\epsilon^4(1-\gamma)^{10}})$.
Here our iteration complexity bound is equivalent to a typical $\widetilde O(1/\sqrt{T})$ global convergence rate.

We perform a set of numerical experiments to illustrate the SICMDP model and validate our proposed algorithms.
Specifically, we examine two numerical examples, namely the discharge of sewage and ship route planning.
Through the discharge of sewage example, we show the advantage of the SICMDP framework over the CMDP baseline obtained by naive discretization in modeling realistic sequential decision-making problems.
Moreover, we demonstrate the effectiveness of the SI-CRL and SI-CPO algorithms in such tabular environments. 
In the ship route planning example, we illustrate the benefits of the SICMDP framework and the ability of the SI-CPO algorithm to address complex continuous control tasks involving continuous state spaces with modern deep reinforcement learning techniques.

% In summary, our contributions are listed as follows.
% First, we present the SICMDP model, which can be viewed as a generalization of the ordinary CMDP model.
% Second, we propose an algorithm to perform reinforcement learning for SICMDPs, which is called SI-CRL, and we believe that we are the first to apply tools from SIP
% to solve reinforcement learning problems.
% Third, we give a theoretical analysis of SI-CRL and identify both its sample complexity and iteration complexity.
% In addition, we perform numerical experiments to illustrate the SICMDP model and validate the SI-CRL algorithm.
% \{This paragraph can be removed!!! \}





\section{Discussion on use cases and challenges}
\label{motivation}
Today, a large number of Artificial Intelligence (AI) applications rely on using deep learning models for various tasks, such as, image classification, speech recognition, natural language understanding, natural language generation and so on.
Due to the significant improvement in performance achieved by the deep learning models, there is a natural trend to use these models on the applications running on mobile phone and other edge devices in the context of IOT (Internet of Things). 
For example, more and more people now want to take pictures using their mobile phones and get information on the building and surroundings around them in a foreign place. Usage of voice based assistants on mobile phones and other home devices is another increasing trend. 
Applications in the area of augmented reality involves continuous image recognition with results being reported on a VR display to provide more information regarding the environment to the individual. 
For example, in security, this can be used for identity detection.
Similarly, in self-driven cars, deep learning models are used
to inference in real-time using data collected from a combination of sensing technologies including visual sensors, such as cameras, and range-to-object detecting sensors, such as lasers and radar. 
Increased instrumentation in various industries such as agriculture, manufacturing, renewable energy and retail generates lot structured and unstructured data which preferably needs to be analyzed at the edge device and so that real-time action can be taken.

For the scenarios described above, inferencing can be done either on the cloud (or server) or on the edge device itself. However, offloading  inferencing to  the cloud can be impractical in lot of situations due to  wireless  energy  overheads, turn-around latencies and data security reasons. On the other hand, given the sheer size of the deep learning models, inferencing on mobile/edge devices poses other kind of challenges on resources, such as memory, compute and energy which need to be utilized efficiently while continuing to provide high accuracy and similar latency.

%phone applications

Even when inferencing is done on the cloud, resources have to be efficiently utilized to keep the cost of inferencing minimum for the cloud vendor as the cost of inferencing is directly dependent on resource utilization. Just as an example, a vendor providing "Inferencing as a service" for image classification may want to keep hundreds of deep learning models customized for various domains and users in memory in order to provide the low response time. This calls for storing compressed models in-memory and directly inferencing using the compressed model when the requests come in. All of this has to be done without compromising on the latency and accuracy of the inferencing.

%1) inferencing on mobile phones without compromising latency and accuracy 
%2)  inferencing over a large cluster of customized models on cloud "inferencing as a service" 
%3) inferencing over edge devices in the context of IOT 
%4) inferencing on multiple models (speech, vision, ..) on driver less cars in limited memory where latency is critical

\section{Preliminaries}
\label{sec:prelims}
\section{Approach}
\begin{figure}[t]
\centering
\resizebox{0.48\textwidth}{!}{ 
  \includegraphics[width=\textwidth]{figures/workflow.PNG}
}
  \caption{Workflow of \system}
  \label{fig:workflow}
\end{figure}

Figure ~\ref{fig:workflow} shows the overall workflow of \system. The triggers for using \system are usually alert(s) from automated anomaly detection, or sometimes an SRE engineer's suspicion. There are three major steps: constructing the service  dependency graph, constructing the event causality graph,  and root cause ranking. The outputs are the root causes ranked by the likelihood. To support fast human investigation experience, we build an interactive UI as shown in  Figure~\ref{fig:UI}: the service dependency, events with causal links and additional details such as raw metrics or the developer contact (of a code deployment event) are presented to the user for next steps. As an  offline part of human investigation, we label/collect a data set, perform validation, and summarize the knowledge for further improvement on all incidents on a daily basis. %as validations and heterogeneous graph learning (HGL)~\cite{qiao2020heterogeneous} to synthesize the knowledge from existing cases in order to further improve the system.

\subsection{Constructing Service Dependency Graph}
\label{sec:appgraph}

The construction of the service dependency graph starts with the initial alerted or suspicious service(s), denoted as $I$. For example, in Figure ~\ref{fig:ex1_dep}, $I=\{\textit{Checkout}\}$. $I$ can contain multiple services based on the range of the trigger alerts or suspicions. We maintain domain service lists where domain-level alerts can be triggered because there is no clear service-level indication.

At the back end, \system maintains a global service dependency graph $G_{global}$ via distributed tracing and log analysis. The directed edge from nodes $A$ to $B$ (two services or system components) in the dependency graph indicates a service invocation or other forms of dependency. In Figure~\ref{fig:ex1_dep}, the black arrows indicate such edges. Bi-directional edges and cycles between the services can be possible and exist. In this work, the global dependency graph is updated daily.%by extracting from one day's total site traffic.

The service dependency (sub)graph $G$ is constructed using $G_{global}$ and $I$. An extended service list $L$ is first constructed by traversing each service in $I$ over $G_{global}$ for a radius range $r$. Each service $u \in L$ can be traversed by at least one service $v \in I$ within $r$ steps: $L=\{u|\exists v\in I, \ dist(u,v)\le r\ or\ dist(v,u)\le r\}$. Then, the service dependency subgraph $G$ is constructed by the nodes in $L$ and the edges between them in $G_{global}$. In our current implementation, $r$ is set to $2$, since this dependency graph may be dynamically extended in the next steps based on events' detail for longer issue chains or additional dependencies.

\subsection{Constructing Event Causality Graph}
\label{sec:causality}

In the second step, \system collects all supported events for each service in $G$ and constructs the causal links between events. 

\subsubsection{Collecting Events}

Table~\ref{tab:events} presents some example event types and detection techniques for \system's production implementation. For detection techniques, ``De Facto'' indicates that the event can be directly collected via a specific API or storage. %The detection can be done passively at the back end continuously then store anomaly events in different databases; or done actively by pulling data and run detection on the fly to save compute resources. 
The detection either runs passively in the back end to reduce delay and improve accuracy, or runs actively for only the services within the dependency graph range to save resources. %For example, low-level error signals or logs are detected actively since they are too many to scale. 

There are three major categories of events: performance metrics, status logs, and developer activities:
\begin{itemize}
    \item \emph{Performance metrics} represent an anomaly of monitored time series metrics. For example, high CPU usage indicates that the service is causing high CPU usage on a certain machine. In this category, most events are continuously and passively detected and stored. %For high CPU usage, threshold indicates the event is created when CPU usage is higher than certain predefined value. TPS spike indicates a spike in transaction per second, since TPS is a moving average value, we use some statistical model learned from historical data to detect such events.
    \item \emph{Status logs} are caused by abnormal system status, such as spike of HTTP error code metrics while accessing other services' endpoints. Different types of error metrics are important and supported in \system, including third-party APIs. For example, Bad Host indicates abnormal patterns on some machines running the service, and can be detected by a  clustering-based ML approach.%Markdown indicates that the whole service is down. 
    \item \emph{Developer activities} are the events generated when a certain activity of developers is triggered, such as code deployment and config change.
\end{itemize}

\begin{table}[t]
\centering
\caption{List of example event types used in \system}
\resizebox{0.4\textwidth}{!}{ 
\begin{tabular}{|c|c|c|}
\hline
Type                                & Event Type                  & Detection Technique  \\ \hline
\multirow{6}{*}{Performance Metrics} & High GC (Overhead)      & Rule-based        \\ \cline{2-3} 
                                    & High CPU Usage          & Rule-based        \\ \cline{2-3} 
%                                    & Out of Memory           & Rule-based        \\ \cline{2-3} 
%                                    & LB Connection Stacking  & Statistical Model \\ \cline{2-3} 
                                    & Latency Spike           & Statistical Model \\ \cline{2-3} 
                                    & TPS Spike               & Statistical Model \\ \cline{2-3} 
                                    & Database Anomaly        & ML Model          \\ \cline{2-3} 
                                    & Business Metric Anomaly & ML Model          \\ \hline
\multirow{4}{*}{Status Logs}        & WebAPI Error            & Statistical Model \\ \cline{2-3} 
                                    & Internal Error          & Statistical Model \\ \cline{2-3} 
                                    & ServiceClient Error     & Statistical Model \\ \cline{2-3} 
                                    & Bad Host                & ML Model          \\ \hline %\cline{2-3} 
%                                    & Hystrix Circuit Break   & De Facto          \\ \hline
\multirow{3}{*}{Developer Activities} & Code Deployment         & De Facto          \\ \cline{2-3} 
                                    & Configuration Change    & De Facto          \\ \cline{2-3} 
                                    & Execute URL             & De Facto          \\ \hline
\end{tabular}
}
\label{tab:events}
\end{table}

In Groot, there are more than a dozen event types such as \emph{Latency Spike} as listed in the column 2 of Table~\ref{tab:events}. 
Each event type is characterized by three aspects: $Name$ indicates the name of this event type; $Lookback Period$ %\footnote{In Figure~\ref{fig:ex2_n1}, there are two periods, 1 day indicates the look-back range if the model has already finished deployment, 4 days indicates the range if the model deployment is still ongoing(incremental deployment).} 
indicates the time range to look back (from the time when the use of \system is triggered) for collecting events of this event type;  $PropertyType$ indicates the types of the properties that an event of this event type should hold. 
$PropertType$  is characterized by a vector of pairs, each of which indicates the string type for a property's name and the primitive type for the property's value such as string, integer, and float. 
Formally, an event type is defined as a tuple: 
$ET = <Name, Lookback Period, PropertyType>$ 
where 
$PropertyType = <(string, \textit{type}_1), ..., (string, \textit{type}_{n})>$ ($n$ is the number of properties that an event of this event type holds). 
%

Each event of a certain event type $ET$ is characterized by four aspects:
$\textit{Service}$ indicates the service name that the event belongs to; $\textit{Type}$ indicates $ET$'s $\textit{Name}$;  $\textit{StartTime}$ indicates the time when the event happens; $\textit{Properties}$ indicates the properties that the event  holds.
Formally, an event is defined as a tuple: 
$e = <Service, Type, StartTime, Properties>$ 
where $Properties$ is an instantiation of $ET$'s  $PropertyType$. 


%and each event is defined as $e = \{<\textit{Property}_i, \textit{value}_i>\}$. Each event type serves as a template for the event instantiation. such as a string, an integer, a float or a set of primitive types while $\textit{value}$ is limited to primitive data types. 
%
%Each event is defined as a sequence of property-value pairs where the set size is $n$.

For example, in Figure~\ref{fig:example1}, the generated event for \emph{Latency Spike in DataCenter-A} in \emph{Service-C} would be $<``\textit{Service-C}'', ``\textit{Latency\ Spike}'', \textit{2021/08/01-12:36:04}, <(``\textit{DataCenter}'',``\textit{DC-1}''),  ...>>$. %So for each service in $G$, we detect/collect and filter the events within specified time range of the alert.

\subsubsection{Constructing Causal Link}

After collecting all events on all services in $G$, in this step, causal links between these events are constructed for RCA ranking. The causal links (red arrows) in Figure~\ref{fig:ex1_cas} are such examples. A causal link represents that the source event can possibly be caused by the target event. SRE knowledge is engineered into rules and used to create causal links between the pairs of events. %As shown in Figure~\ref{fig:example2}, there are two categories of rules: basic rules and conditional rules. 

A rule for constructing a causal link is defined as a tuple:  $Rule = <Target\mbox{-}Type,  Source\mbox{-}Events, Target\mbox{-}Events, Direction,\\ Target\mbox{-}Service,  Condition>$  ($Condition$ can be optionally specified). $Target\mbox{-}Type$ indicates the type of the rule, being either $Static$ or $Dynamic$ (explained further later). $Source\mbox{-}Events$ indicates the type of the causal link's source event ($Source\mbox{-}Events$ are listed in the names of the rules shown in Figures~\ref{fig:ex2_n1},~\ref{fig:ex2_n2} and~\ref{fig:dynamic_example}).   $Target\mbox{-}Events$ indicates the type of the causal link's target event. $Direction$ indicates the direction of the casual link between the target event and source event. $Target\mbox{-}Service$ indicates the service that the target event should belong to. Note that $Target\mbox{-}Service$ in $Static$ rules can be  $Self$, which indicates that the target event would be within the same service as the source event, or $Outgoing$/$Incoming$, which indicates that the target event would belong to the downstream/upstream services of the service that the source event belongs to in $G$.

\begin{figure}[t]
\centering
\includegraphics[width=0.56\columnwidth]{figures/example3.png}
\caption{Example of dynamic rule}
\label{fig:dynamic_example}
\end{figure}

There are two categories of special rules. The first category is \emph{dynamic} rules (i.e., rules whose $Target\mbox{-}Type$  is set to $Dynamic$) to support dynamic dependencies. Here $Target\mbox{-}Service$ does not indicate any of the three possible options listed earlier but indicates the name of the target service that \system would need to create. For example, live DB dependencies are not available due to different tech stacks and high volume. In Figure~\ref{fig:dynamic_example}, a DB issue (DB Markdown) is shown in \emph{Service-A}. Based on the listed \emph{dynamic} rule, \system creates a new ``service'' \emph{DB-1} in $G$, a new event ``Issues'' that belongs to \emph{DB-1}, and a causal link between the two events.  In practice, the SRE teams use dynamic rules to cover a lot of third-party services and database issues since the live dependencies are not easy to maintain.  %However through the internal error messages and dynamic rules, \system is still able to handle these dependencies. %we can still support external inferences. 

The second category of special rules is \emph{conditional} rules. \emph{Conditional} rules are used when some prerequisite conditions should be satisfied before a certain causal link is created. In these rules, $Condition$ is specified with a boolean predicate. As shown in Figure~\ref{fig:ex2_n2}, the SRE teams believe \emph{Latency Spike} events from different services are related only when both events happen within the same data center. Based on this observation, \system would first evaluate the predicate in $Condition$ and build only the causal link when the predicate is true. A conditional rule overwrites the basic rule on the same source-target event pair.

When constructing causal links, \system first applies the \emph{dynamic} rules so that dynamic dependencies and events are first created at once. Then for every event in the initial services (denoted as $I$), if the rule conditions are satisfied, one or many causal links are created from this event to other events from the same or upstream/downstream services. When a causal link is created, the step is repeated recursively for the target event (as a new origin) to create new causal links. After no new causal links are created, the construction of the event causality graph is finished.

% When \system constructs the causal links, \system first processes all dynamic rules as they may create new event nodes in the graph. %\system enumerates the dynamic rules on each existing event node and also on the newly added nodes (There could also be rules applicable to the newly added nodes) until no new event nodes can be created. 


%Each rule is defined as a predicate containing both events' property-value pair. If the predicate evaluates to be true between two events, then we would add the edge in the causality graph. For example, in Figure~\ref{fig:example1}, the rule used to establish the edge between \emph{GC overhead in RNO} and \emph{Latency increase in LVS, RNO, SLC} would be like this: Suppose we are now determining whether there should be a link from event $u$ to event $v$, then this rule would be $u.\text{pool} = v.\text{pool}\ and\ u.\text{type} = ``\text{High GC Overhead}"\ and\ v.\text{type} = ``\text{Latency increase}"\ and\ u.\text{center} \cap v.\text{center} \ne \emptyset$ which holds true for these two events. Each causality link is also associated with a weight which represents the likelihood of causality - we set all initial values as $1.0$. Overtime these value are updated by the statistical analysis result of the collected data set.


\subsection{Root Cause Ranking}
Finally, \system ranks and recommends the most probable root causes from the event causality graph. Similar to how search engines infer the importance of pages by page links, we customize the PageRank \cite{manning2010introduction} algorithm to calculate the root cause ranking; the customized algorithm is named as GrootRank. The input is the event causality graph from the previous step. Each edge is associated with a weighted score for weighted propagation. The default value is set as $1$, and is set lower for alerts with high false-positive rates. 

Based on the observation that dangling nodes are more likely to be the root cause, we customize the personalization vector as $P_n = f_n $ or $P_d = 1$, where $P_d$ is the personalization score for dangling nodes, and $P_n$ is for the remaining nodes; and $f_n$ is a value smaller than 1 to enhance the propagation between dangling nodes. In our work, the parameter setting is $f_n = 0.5$, $\alpha = 0.85$, $max_{iter} = 100$ (which are parameters for the PageRank algorithm). Figure \ref{fig:person} illustrates an example. The grey circles are the events collected from three services and one database. The grey arrows are the dependency links and the red ones are the causal links with the weight of $1$. Both of the PageRank and GrootRank algorithms detect $event 5$ (DB issue) as the root cause, which is expected and correct. However, the PageRank algorithm ranks $event 4$ higher than $event 3$. But $event 3$ of $\textit{Service-C}$ is more likely to be the second most possible root cause (besides $event 5$), because the scores on dangling nodes are propagated to all others equally in each iteration. We can see that $event 3$ is correctly ranked as second using the GrootRank algorithm.

The second step of GrootRank is to break the tied results from the previous step. The tied results are due to the fact that the event graph can contain multiple disconnected sub-graphs with the same shape. We design two techniques to untie the ranking: 
\begin{figure}[t]
\centering
  \includegraphics[width=0.8\columnwidth]{figures/personalvector.png}
  \caption{Example of personalization vector customization}
  \label{fig:person}
\end{figure}

\begin{figure}[t]
\centering
  \includegraphics[width=0.8\columnwidth]{figures/accessdistance.png}
  \caption{Example of using access distance to untie the ranking results}
  \label{fig:untie}
\end{figure}
\begin{enumerate}
\item For each joint event, the access distance (sum) is calculated from the initial anomaly service(s) to the service where the event belongs to. If any ``access'' is not reachable, the distance is set as $d_m+1$ where $d_m$ is the maximum possible distance. The one with shorter access distance (sum) would be ranked higher and vice versa. Figure \ref{fig:untie} presents an example, where \emph{Service-A} and \emph{Service-B} are both initial anomaly services. Since \system suspects that $event 2$ is caused by either $event 3$ or $event 1$ with the same weight. The scores of $event 3$ and $event 1$ are tied. Then, $event 3$ has a score of $1$ (i.e., $0+1$) and $event 1$ has a score of 2 (i.e., $0+2$), since it is not reachable by \emph{Service-B}). Therefore, $event 3$ is ranked first and logical. 
\item For the remaining joint results with the same access distances, \system continues to untie by using the historical root cause frequency of the event types under the same trigger conditions (e.g., checkout domain alerts). This frequency information is generated from the manually labeled dataset. A more frequently occurred root cause type is ranked higher.% than the less frequent ones.
\end{enumerate}


\subsection{Rule Customization Management}

While \system users create or update the rules,  there could be overlaps, inconsistencies, or even conflicts being introduced such as the example in Figure~\ref{fig:ex2_n2}. \system uses two graphs to manage the rule relationships and avoid conflicts for users. One graph is to represent the link rules between events in the same service (\emph{Same-Graph}) while the other is to represent links between different services (\emph{Diff-Graph}). The nodes in these two graphs are the event types defined in Section~\ref{sec:causality}. There are three statuses between each (directional) pair of event types: (1) no rule, (2) only basic rule, and (3) conditional rule (since it overwrites the basic rule). In \emph{Same-Graph}, \system does not allow self-loop as it does not build links between an event and itself.
% but it is possible that we build links between different services with the same event type.

When rule change happens, existing rules are enumerated to build edges in \emph{Same-Graph} and \emph{Diff-Graph} based on $Target\mbox{-}Events$ and $Target\mbox{-}Service$. Based on the users' operation of 
% \begin{itemize}
%     \item 
    (1) ``remove a rule'',  \system removes the corresponding edge on the graphs;
    % \item 
    (2) ``add/update a rule'',  \system checks whether there are existing edges between the given event types, and then warns the users for possible overwrites. 
    % The users can also combine the conditional rules.   % while users are adding basic rules between event types if there are existing conditional rules between them.
    If there are no conflicts, \system just adds/updates edges between the event types.
    % \item Add conditional rules. We would first alert the possible overwrite. Then if users are about to add new conditional rules on the top of existing conditional rules, we would ask the users to combine these two conditions to add a new one. We then build or change all corresponding edges to status 3.
% \end{itemize} 

After all changes, \system extracts the rules from the graphs by converting each edge to a single rule. These rules are automatically implemented, and then tested against our labeled data set. The \system users need to review the changes with validation reports before the changes go online.

% Note that currently we don't check the consistencies between dynamic rules as we cannot process the dynamic event types, but this could be solved in the future by using nodes with symbolic values to represent such event types. 
\section{Inferencing using Compressed Models}
\label{sec:inference}

In this section, we discuss the various approaches for inferencing using the compressed model,
where the compressed model is stored  in  the format as shown in the previous section.
Clearly, the trivial method of exploding the model back to the dense format  and doing the computation 
(using standard frameworks like Caffe, Tensorflow etc) is not a good choice since the entire purpose of
model compression gets defeated because  of the 
excessive memory usage. The other extreme of decoding element by element of the matrix and doing the operations on the
decoded element 
has little memory overhead, but is computationally inefficient.
This calls for the need to develop an efficient stand-alone module (independent of the
Caffe/Tensorflow framework) for inferencing using the compressed model.
The na{\"i}ve algorithm for doing the inferencing is presented in Algorithm~\ref{alg:pseudocode1}.
The idea here is to work  sequentially on the individual rows of the weight matrix (line 3).
For a particular row, the $col\_ind$ and the $val$ entries for that row are first  Huffman-decoded (line 5-6);
this is followed by converting relative column index of $col\_ind$ to absolute index (line 7) 
and creating an $abs\_val$ array which  is essentially the $val$ array with its entries replaced by the
corresponding codebook entires. 
All these steps  in fact  create the arrays in Figure~\ref{fig:repr_f1} from 
Figure~\ref{fig:repr_f5} for a particular row segment.
%Finally the dot product of $abs\_val$ and input activation is done by
%invoking MKL routine for sparse matrix-vector product. For multiple batches,
Finally we call MKL routine $mkl\_scsrmm$ for 
sparse matrix-matrix multiplication  of $abs\_val(i)$ and $a$
to compute $b[i,:]$.

%   some standard library:
%in our work, we apply Intel MKL kernels to effe
 %the 
%and then evaluating the dot product of the respective codebook entries
%corresponding to the column entries with the input activation (line 9-12).


\begin{algorithm}[t]
\caption{Na{\"i}ve algorithm for inferencing using compressed model }
\label{alg:pseudocode1}
\small
\begin{algorithmic}[1]
	\STATE Input: $row\_ptr$ array, entry $i$ of which is a tuple 
	of  starting address of row $i$ in $val$ and that in $col\_ind$.\\
	$val$ Huffman encoded cluster index bit stream. \\
	$col\_ind$ Huffman encoded rel. indexed column bit stream. \\
	$\calC$ codebook of quantized weights. \\
	$a$ input activation matrix. \\
	\STATE Output: $b$ output  activation  matrix. \\
		 
	\FOR{every entry $i$ of the $row\_ptr$ array}
		\STATE Set $val\_begin(i)$,  $val\_end(i)$, $col\_begin(i)$, $col\_end(i)$\\
		 for row $i$ as follows \\
		 \quad  \quad $\langle val\_begin(i), col\_begin(i) \rangle \leftarrow row\_ptr(i)$\\
		 \quad  \quad $\langle val\_end(i), col\_end(i) \rangle \leftarrow row\_ptr(i+1)$.
		 \STATE $dec\_val(i)$ $\leftarrow$ Huffman decoding of bit stream in $val$ between $val\_begin(i)$ and  $val\_end(i)$.
		\STATE  $dec\_col(i)$ $\leftarrow$ Huffman decoding of bit stream in \\
		$col\_ind$ between $col\_begin(i)$ and  $col\_end(i)$.
		\STATE $abs\_col(i)$ $\leftarrow$  Prefix sum of $dec\_col(i)$.
		\STATE Set $abs\_val(i)[j]$ $\leftarrow$ $\calC [dec\_val(i)[j]]$ , $\forall j$.
		\STATE $b[i, :]$ += MKL\_CSRMM($abs\_val(i)$, $a$)
%		\FOR{every entry $j$ of the $abs\_col(i)$ array}
%			\STATE col = $abs\_col(i)[j]$.
%			\STATE $b_i$+=$a_{col}$ * $\calC [dec\_val(i)[j]]$.
%		\ENDFOR	 
	\ENDFOR	
\end{algorithmic}
\end{algorithm}

The above algorithm can be parallelized by employing different threads to operate on different rows of the weight matrix.
Moreover MKL internally can use multiple threads for sparse matrix operations.
However  Algorithm~\ref{alg:pseudocode1} faces multiple drawbacks.
Firstly, the algorithm decodes an entire row of the matrix, and thus the memory requirement becomes
significant for large matrices. 
Secondly, most  algorithms for matrix multiplication work more efficiently using  blocks rather than individual elements, to achieve necessary reuse of data in local memory.
The advantage of this approach is that the small blocks can be moved into the fast local memory and their elements can then be repeatedly used.
This motivates us to employ blocking even for compressed model inferencing, which we describe next.




\subsection{Blocking of Weight Matrix}

 The
general idea of blocking is to organize the data structures in a program into  chunks called blocks. The program is
structured so that it loads a block into the L1 cache, does all the reads and writes that it needs to on that
block, then discards the block, loads in the next block, and so on. 
Similar to standard matrix multiplication, the blocking algorithm for inferencing shall work 
 by partitioning the matrices into submatrices and then exploiting
the mathematical fact that these submatrices can be manipulated just like scalars.
%However, our inferencing procedure first needs to perform Huffman decoding followed by
%absolute index computation on the submatrices before the block multiplication can be performed.
Instead of storing the  original weight matrix in row major format,
we need to ensure that any particular block of the matrix is stored in contiguous memory.  
This will make certain	 that the Huffman decoding happens on contiguous memory locations and  generates the submatrix 
corresponding to a block.


See Figure~\ref{fig:block_f1} and Figure~\ref{fig:block_f2} for illustration.
Suppose the original weight matrix  stored in dense row major format is of dimension 8x8, and we decide to work on blocks each sized 4x4. 
We first convert  this matrix to   4 x 16 format, such that each row of the new matrix stores 
elements of the corresponding block of the old matrix in contiguous locations. This new matrix 
is then stored in CSR format with relative indexing and Huffman encoding, as discussed in the 
previous section. 
\\
{\it Size of the modified model}:


It is observed that the non zeroes  in the weight matrix are uniformly distributed, thus
the size of the $val$ and $col\_ind$ vectors does not change a lot  (even with zero padding in the compressed format)  when the matrix is
stored in block contiguous fashion.
The number of rows in the modified matrix is same as the number of blocks in the original matrix, and may be larger or smaller than that
in the  original matrix depending on the block size. From experimental results, it is however observed, that
change in model size due to this difference in the size of the $row\_ptr$ is insignificant. Hence we can assume that 
storing the model in block contiguous fashion does not add to memory overhead.



\begin{figure}[!tbp]
  \centering
  \subfloat[Original Connection Matrix.]{\includegraphics[width=2in]{figures/block1.pdf}\label{fig:block_f1}}
 % \hfill
 \hspace{10mm}
  \subfloat[Modified Connection Matrix]{\includegraphics[width=2.5in]{figures/block2.pdf}\label{fig:block_f2}}
  \caption{Representation of a compressed model.}
\end{figure}



\subsection{Blocked Inferencing Procedure}

Next we present our inferencing algorithm using the blocked storage scheme. 
Our algorithm ensures that once a  row of the connection matrix (which corresponds to a block
in the original weight matrix) is decoded,
the decoded entries are used for all the computations that require them.
This is illustrated in Figure~\ref{fig:block_mult}. A row is decoded and multiplied with all possible subblocks of 
input activation matrix to generate partial results for the output activation matrix. 
The blocked inferencing algorithm is presented in Algorithm~\ref{alg:pseudocode2}.


\begin{figure}[b]
\centering
\includegraphics[width=3in]{figures/mult.pdf}
\caption{Blocked inference scheme.}
\label{fig:block_mult}
\end{figure}





\begin{algorithm}[t]
\caption{Algorithm for block inferencing}
\label{alg:pseudocode2}
\small
\begin{algorithmic}[1]
	\STATE Input: Compressed model stored in $bh$ x $bw$ block contiguous manner with \\
	$row\_ptr$ array, entry $i$ of which is a tuple $\langle x,y \rangle$ \\
	where $x$ and $y$ being respectively starting address of row $i$ in $val$ and 	that in $col\_ind$.\\
	$val$ Huffman encoded cluster index bit stream. \\
	$col\_ind$ Huffman encoded relative indexed column bit stream. \\
	$\calC$ codebook of quantized weights. \\
	$a$ input activation matrix with $a_{rows}$ rows\\
	\STATE Output: $b$ output  activation matrix\\
		 
	\FOR{every entry $i$ of the $row\_ptr$ array}
		\STATE Set $val\_begin(i)$,  $val\_end(i)$, $col\_begin(i)$, $col\_end(i)$\\
		 for row $i$ as follows \\
		 \quad  \quad $\langle val\_begin(i), col\_begin(i) \rangle \leftarrow row\_ptr(i)$\\
		 \quad  \quad $\langle val\_end(i), col\_end(i) \rangle \leftarrow row\_ptr(i+1)$.
		\STATE $dec\_val(i)$ $\leftarrow$ Huffman decoding of bit stream in $val$ between $val\_begin(i)$ and  $val\_end(i)$.
		\STATE  $dec\_col(i)$ $\leftarrow$ Huffman decoding of bit stream in \\
		$col\_ind$ between $col\_begin(i)$ and  $col\_end(i)$.
		\STATE $abs\_col(i)$ $\leftarrow$  Prefix sum of $dec\_col(i)$.
		\STATE Set $abs\_val(i)[j]$ $\leftarrow$ $\calC [dec\_val(i)[j]]$ , $\forall j$.
		\STATE Arrange $abs\_val(i)$ as $bh$ x $bw$ block. \\
		\STATE col\_id = $ ( i \% (a_{rows}/bw) ) * bw$  \\
		\STATE row\_id = $ ( i / (a_{rows}/bw) ) * bh$  \\
		\STATE b[row\_id:(row\_id+bh-1),:] += MKL\_CSRMM($abs\_val(i)$, a[col\_id:(col\_id+bw-1),:] ) \\
	\ENDFOR	
\end{algorithmic}
\end{algorithm}





\section{Experimental Results with Blocking}
\label{sec:expt1}
In this section, we present the experimental results for our block inferencing procedure. We begin by specifying the system configurations and the dataset.

\subsection{System and Dataset}

For running our experiments (also the ones in Section~\ref{sec:expt2}), we have used Intel Xeon CPU E5-2697 system. It has two NUMA nodes 
with 12 cores, each with frequency of 2.70GHZ. The system has 32KB, 256KB and 30MB of L1, L2 and L3 cache respectively.
We consider compressed models for two popular deep neural networks, AlexNet and VGG-16. 
For each of these models we consider the compressed configurations corresponding to four different pruning percentages.
The first configuration corresponds to the procedure applied in  \cite{HanMD15}.  
Table~\ref{tab:alex_pr} and Table~\ref{tab:vgg_pr}
 present the pruning percentages of all the layers in this configuration. We refer to this configuration as 
{\em conventional} in subsequent discussion.
The compressed model sizes of AlexNet and VGG-16 for this configuration are respectively 6.81 MB and 10.64 MB.
The other three configurations correspond respectively to 70\%, 80\% and 90\% pruning of {\em all} the layers of the network.  
We consider these configurations to study how our scheme performs for a wide range of sparsity spectrum of the compressed models.
8 bit (5 bit) quantization for CONV (FC) layers
and 4 bit (5 bit) relative indexing for AlexNet (VGG-16) is employed for all the configurations.

\begin{table}[!tbp]
  \centering
  \subfloat[AlexNet]{
\begin{tabular}{|c|c|}
			\hline 
			Layer & Pruning \% \\ \hline 
			conv1 & 16 \\ \hline 
			conv2 & 62 \\ \hline 
			conv3 & 65 \\ \hline 
			conv4 & 63 \\ \hline 
			conv5 & 37 \\ \hline 
			fc6 & 91  \\ \hline 
			fc7 & 91  \\ \hline 
			fc8 & 75  \\ \hline
        \end{tabular}
        \label{tab:alex_pr}
  }
  \hspace{1mm}
    \subfloat[VGG-16]{
\begin{tabular}{|c|c|m{2em}|m{2em}|}
			\hline 
			Layer & Pruning \% \\ \hline 
			conv1\_1 & 42  \\ \hline
			conv1\_2 & 78 \\ \hline
			conv2\_1 & 66 \\ \hline
			conv2\_2 & 64 \\ \hline
			conv3\_1 & 47 \\ \hline
			conv3\_2 & 76 \\ \hline
			conv3\_3 & 58 \\ \hline
			conv4\_1 & 68 \\ \hline
			conv4\_2 & 73 \\ \hline
			conv4\_3 & 66 \\ \hline
			conv5\_1 & 65 \\ \hline
			conv5\_2 & 71 \\ \hline
			conv5\_3 & 64 \\ \hline
			fc6 & 96  \\ \hline
			fc7 & 96  \\ \hline
			fc8 & 77  \\ \hline
        \end{tabular}
        \label{tab:vgg_pr}
  }
  \caption{Compressed AlexNet and VGG-16 models.}
\end{table}


%
%\begin{table}[!htb]
%    \caption{Properties of compressed AlexNet and VGGNet Models}
%    \begin{minipage}{.5\linewidth}
%      \caption{AlexNet}
%      \centering
%        \begin{tabular}{|c|c|}
%			\hline 
%			Layer & Pruning \% \\ \hline 
%			conv1 & 16 \\ \hline 
%			conv2 & 62 \\ \hline 
%			conv3 & 65 \\ \hline 
%			conv4 & 63 \\ \hline 
%			conv5 & 37 \\ \hline 
%			fc6 & 91  \\ \hline 
%			fc7 & 91  \\ \hline 
%			fc8 & 75  \\ \hline
%        \end{tabular}
%    \end{minipage}%
%    \begin{minipage}{.5\linewidth}
%      \centering
%        \caption{VGGNet}
%        \begin{tabular}{|c|c|m{2em}|m{2em}|}
%			\hline 
%			Layer & Pruning \% \\ \hline 
%			conv1\_1 & 42  \\ \hline
%			conv1\_2 & 78 \\ \hline
%			conv2\_1 & 66 \\ \hline
%			conv2\_2 & 64 \\ \hline
%			conv3\_1 & 47 \\ \hline
%			conv3\_2 & 76 \\ \hline
%			conv3\_3 & 58 \\ \hline
%			conv4\_1 & 68 \\ \hline
%			conv4\_2 & 73 \\ \hline
%			conv4\_3 & 66 \\ \hline
%			conv5\_1 & 65 \\ \hline
%			conv5\_2 & 71 \\ \hline
%			conv5\_3 & 64 \\ \hline
%			fc6 & 96  \\ \hline
%			fc7 & 96  \\ \hline
%			fc8 & 77  \\ \hline
%        \end{tabular}
%    \end{minipage} 
%    \label{tab:compression-params}
%\end{table}
%
\subsection{Blocking results}

Our first set of experiments is aimed to study the effect of variation of  block sizes on the inference time (both the decoding time and  
the computation time) for individual layers corresponding to the different configurations of the  compressed models. 
 Figure~\ref{fig:blck_f1} and ~\ref{fig:blck_f2}
show the decoding time, computation time and total time, with different block sizes for FC6 layer of AlexNet and VGGnet,
using batch size of 16. The models used for these runs  correspond to the conventional configuration. All these experiments employ MKL with 4 threads
for computation.



We observe that for very small block sizes,  the decoding and the computation time are pretty high due to overhead of the too many function calls.
For very large block sizes, the level of parallelism gets limited, leading to increase in the inference time.
Figure~\ref{fig:blck_f3} and ~\ref{fig:blck_f4} show the same charts with batch size of 256. We note that for smaller batch size, the total time is dominated by the decoding time, 
whereas the computation time takes over at larger batch sizes. However the nature of variation of  inference time with the block size is consistent across batch sizes.
We observe similar nature of plots for other configurations and batch sizes as well.



\begin{figure*}[!tbp]
  \centering
  \subfloat[AlexNet Batch size16]{\includegraphics[width=1.7in]{figures/alex16.pdf}\label{fig:blck_f1}}
  \hspace{1mm}
  \subfloat[VGG-16 Batch size 16]{\includegraphics[width=1.7in]{figures/vgg16.pdf}\label{fig:blck_f2}}
  \hspace{1mm}
  \subfloat[AlexNet Batch size 256]{\includegraphics[width=1.7in]{figures/alex256.pdf}\label{fig:blck_f3}}
  \hspace{1mm}
  \subfloat[VGG-16 Batch size 256]{\includegraphics[width=1.7in]{figures/vgg256.pdf}\label{fig:blck_f4}}
  \caption{Inference Time Variation with Block Size.}
\end{figure*}




We also note that the working memory increases with increase in block size. 
Table~\ref{tab:workmem} presents the working memory required for matrix matrix multiplication for FC6 layer of AlexNet and VGG-16.
Since there is not significant difference in the inference timings between block sizes in range 128 x 128 to 1024 x 1024, we fix 128 x 128 as our block size
for the subsequent experiments.



\begin{table}[h!]
\centering
\begin{tabular}{|c|c|c|}
           \hline
Blocksize & AlexNet  & VGG-16 \\ \hline
  16  x   16 &    1.26KB &  0.92KB \\ \hline
  32  x   32 &    4.57KB &   3.42KB \\ \hline
  64  x   64 &   17.33KB &  12.97KB \\ \hline
 128  x  128 &   67.40KB &  50.22KB \\ \hline
 256  x  256 &  265.78KB & 197.26KB \\ \hline
 512  x  512 &    1.03MB & 781.52KB \\ \hline
1024  x 1024 &    4.11MB &   2.98MB \\ \hline
2048  x 2048 &   14.76MB &  11.42MB \\ \hline
4096  x 4096 &   36.88MB &  42.38MB \\ \hline

\end{tabular}
\caption{Working Memory Requirement for FC6 layer}
\label{tab:workmem}

\end{table}




We next observe  the variation of activation memory requirement and the inference time with batch sizes. Table~\ref{tab:batch}   presents
the results for batch sizes of 16 and 256. Clearly, for a fixed batch size,  the activation memory required by the convolution layers is  more than  that of the 
fully-connected layers.  Inferencing applications on a low resource system generally come with a cap on the available memory. 
Suppose we consider a fictitious scenario where the maximum available memory is 20MB. 
From Table~\ref{tab:batch}, it makes sense to run the fully connected layers with batch size 256, since the memory required is well below the
permissible threshold, and there is significant increase in throughput if we process in batch of 256. 
For the convolution layers, however, processing in batch of 256 is not a desirable option because of the large memory overhead.
This motivates us to use different batch sizes for different layers during the inferencing. We present this in more detail in the next section.

 


\begin{table}[h!]
\centering
\begin{tabular}{|c|c|c|c|c|}
\hline
      & \multicolumn{2}{c|}{Memory (MB)} & \multicolumn{2}{c|}{Time (ms)} \\ \hline
Layer  & batch-size & batchsize & batchsize& batchsize \\
 &16 & 256 & 16 & 256 \\ \hline
conv1  &  17.72 & 283.59 &  349.93 &  5408.93 \\ \hline
norm1  &  17.72 & 283.59 &  98.87 & 1597.83 \\ \hline
pool1  &  4.27 &  68.34 & 11.68 & 176.42 \\ \hline
conv2  &  11.39 & 182.25 &  341.72 &  5745.49 \\ \hline
norm2  &  11.39 & 182.25 &  68.06 & 1081.80 \\ \hline
pool2  &  2.64 &  42.25 & 7.12 &  116.49 \\ \hline
conv3  &  3.96 &  63.38 & 153.11 &  2573.47 \\ \hline
conv4  &  3.96 &  63.38 & 204.01 &  3135.62 \\ \hline
conv5  &  2.64 &  42.25 & 135.66 &  2242.94 \\ \hline
pool5  &  0.56 &  9.00 &  1.92 &  25.72 \\ \hline
fc6  &  0.25 &  4.00 &  51.77 & 112.62 \\ \hline
fc7  &  0.25 &  4.00 &  21.06 & 46.61 \\ \hline
fc8  &  0.06 &  0.98 &  9.66 &  22.61 \\ \hline

\end{tabular}
\caption{Memory Requirement and Inference time for AlexNet individual layers}
\label{tab:batch}
\end{table}




\subsection{Inferencing with Variable Batch Size}
\label{sec:dp}
\newcommand{\OUT} {{\rm OUT}}
\newcommand{\IN} {{\rm IN}}
\newcommand{\WS} {{\rm WS}}
\newcommand{\Time} {{\rm Time}}
\newcommand{\OPT} {{\rm OPT}}
\newcommand{\TOT} {{\rm TOT}}

It is clear from the results shown in the previous section that  using a larger batch for inferencing increases the 
throughput as computing
resources are utilized more efficiently.
However, an issue with inferencing larger batches is the increase
in inferencing latency 
(due to wait time while assembling a batch, and because
larger batches take longer to process). 
Moreover the memory requirement for the input and the output
activations and buffer memory also increases for larger batch size.
Thus applications work with large batch sizes while keeping
the latency and memory utilization within certain thresholds.
The problem becomes more challenging since the 
available memory varies dynamically depending on the system load;
hence the  batch size for achieving the maximum throughput can be figured out only at the time of inferencing.
Moreover, the memory requirement and the computation time for 
inferencing varies with the layers even for a fixed batch size. 
Thus it might be advantageous to do the inferencing using different 
batch sizes for different layers.
We address this issue by proposing a dynamic programming based algorithm
for determining variable batch sizes for different layers for efficient
inferencing.
We describe our dynamic program below.

\subsection{Dynamic Programming}
Let $L_1$, $L_2$, $\cdots$, $L_f$ denote the layers of the DNN. 
For $i$ = 1,  2, $\cdots$, $f$, let \Time$(i, B)$ denote the time required to perform the inferencing computations for layer $L_i$
of the DNN using a batch size of $B$.  
Next, we let \IN$(i,B)$ and \OUT$(i,B)$ respectively denote 
the input activation and output activation memory required
to perform inferencing of layer $L_i$ with a batch size of $B$. 
Further, let \WS$(i)$ denote the size of the temporary workspace  required for layer $L_i$ computations (for instance this includes
the buffer memory required to decode blocks of the connection matrix for $L_i$). 
All the values \IN$(i,B)$,  \OUT$(i,B)$, \WS$(i)$ and \Time$(i, B)$
are obtained once for a given compressed model.
Note that the total memory required to perform inferencing computations for layer $L_i$ with a batch size of $B$ is captured by
$$ \IN(i,B) + \WS(i) +  \OUT(i,B). $$
Let {\TOT} denote the total memory available for performing the inference
computations for the entire model.

We now describe the dynamic program to determine the optimal batch size
to be used at all the individual layers in order to maximize the overall
throughput of the inferencing.
For this we define a configuration:
a configuration is a tuple $\langle i, B, A \rangle$,
where $i$ denotes the layer $L_i$, $B$ denotes a batch size and 
$A$ denotes amount of memory.
We maintain a dynamic program table {\OPT}. 
An entry \OPT$(i, B, A)$ of the dynamic program denotes the minimum time to perform the inferencing computations for layers $L_1$-$L_i$, when a batch size of 
$B$ is used for layer $L_i$, and $A$ units of memory (out of {\TOT}) are not available for performing the inferencing computations for layers $L_1$-$L_i$ (this memory is reserved for performing inferencing computations from layer $L_{i+1}$ to layer $L_f$).
Thus, we only have available ({\TOT} - $A$) units of memory
for inference computations of layers $L_1$ to $L_i$.

We say that configuration $\langle i,B,A \rangle$ is {\em feasible} if
the total memory required for performing inferencing computations at
layer $L_i$ with a batch size of $B$ is within the available memory bound,
i.e.,
$$A + \IN(i,B) + \WS(i) +  \OUT(i,B) \leq \TOT.$$

We now describe the recurrence relation for computing the 
entries of the dynamic programming table \OPT$(\cdot, \cdot, \cdot)$.
For simplicity, we assume that for every $i$, the batch size used for inferencing computations at layer $L_{i-1}$ is no more than
the batch size used for the inferencing computations at layer $L_i$.
Clearly,  \OPT$(i, B, A)$ can be finite only if  $\langle i,B,A \rangle$ is feasible.
Suppose that layer $L_i$ is computed with batch size $B$ and layer $L_{i-1}$ with a batch size $b$. 
For simplicity, we consider all $b \leq B$ such that $b$ divides $B$.
For a given $b$, 
the inferencing computations for layer $L_{i-1}$
will be performed in $ (B/b)$ phases, wherein in each phase
a batch of size $b$ will be processed up to layer $L_{i-1}$.
After the end of these phases, the $B$ output activations of 
layer $L_{i-1}$ will be fed as input activations to layer $L_i$.
Note that before the processing of the last of these phases,
\IN$(i,B-b)$ amount of output activation need to be buffered. 
Thus the total memory available for processing up to
layer $L_{i-1}$ gets reduced by \IN$(i, B-b)$ as this is required
for storing the activations before processing layer $L_i$.

We are now ready to present the recurrence relation.
For any $i > 1$, 
\begin{equation*}
\begin{split}
\OPT(i, &B, A)   =  \Time(i,B) \quad \quad  +   \\
 &\quad  \min \limits_{b \leq B} \left \{  (B/b) * \OPT(i-1,b,A+\IN(i,B-b)) \right \}\\
 &\quad \mbox{subject to} \quad \quad  \text{ $\langle i,B,A \rangle$ is feasible}
\end{split}
\end{equation*}



For the base case i.e., for $i$ = 1.
\begin{equation*}
  \OPT(1, B, A) =\begin{cases}
     \Time(1, B), & \text{if $\langle 1,B,A \rangle$ is feasible}.\\
    \infty, & \text{otherwise}.
  \end{cases}
\end{equation*} 





 
The maximum throughput for the inferencing is obtained 
by considering the configuration that yields the 
minimum inference time per input which is 
$$\min \limits_{ B} \frac{ \OPT(f, B, 0)}{B}. $$ 



The above dynamic program can be easily extended to ensure that the latency of inferencing is always less than some specified threshold. In the recurrence relation,
if \OPT$(i, B, A)$ exceeds the threshold value for some $i$, $B$, and $A$, we make   \OPT$(i, B, A)$ $\leftarrow$ $\infty$. This makes sure that our optimal solution 
never has larger latency.

\subsection{Additional Storage and Computation Overhead}
The table \OPT$(\cdot, \cdot, \cdot)$ needs to be evaluated for each entry in order to figure out the individual layer batch sizes that maximise the overall throughput.
We  now figure out the additional storage and computation that is needed for the dynamic programming space complexity for standard networks like AlexNet. 
The total number of layers in AlexNet  is 14. For requested input count of 64, we consider batch sizes 
in range 1 to 64 for the second dimension. For the case where the additional available memory is twice of the model size, the third dimension is considered from 0 to 14MB
in steps of {100KB}. Thus the total size of the table is around 500KB.
Each entry computation of the table computes the minimum over a set of possible batch sizes. Thus the computation complexity is at most $\cal B$ times the size of the table,
where $\cal B$ is the maximum number of distinct batch sizes considered.

We begin our  inferencing  with a pre-processing step, which computes the  individual layer batch sizes that maximise the overall throughput using the
above dynamic program.  The actual inferencing uses the batch sizes outputted from the dynamic program.












\section{Experimental Results with Batch Size}
\label{sec:expt2}
In this section, we validate the results of our dynamic programming algorithm on practical test cases with AlexNet model.
Suppose the user requests for inference of a set of $K$ images, and is interested to get the maximum throughput for the inference.
We consider the scenarios where the total  memory in the system (in addition to the model) is 1.5x, 2x and 2.5x  times the model size.
Our baseline is selecting a fixed batch size such that (i) running any layer of inferencing using that batch size does not violate the
memory constraints (ii) out of all possible batch sizes which satisfy	 (i),  the baseline returns the batch size with maximum throughput.
We compare this baseline from our dynamic programming output, which uses variable batch sizes for different layers.
We perform our experiments $K =  32, 64, 128$ and with all the four configurations of AlexNet model (conventional pruning and
70\%, 80\%, 90\% pruning).
Figure~\ref{fig:real_dp1} - ~\ref{fig:real_dp3} compares the results of our dynamic program algorithm with the baseline 
(fixed batch size) output for AlexNet with conventional pruning. The x-axis shows the additional memory available (w.r.t to the model size)
over the model, and the y-axis plots the total time to infer K images. Our results show that the dynamic programming approach
improves the throughput  by 15-25\% over the fixed batch size approach.

\begin{figure*}[!tbp]
  \centering
  \subfloat[K=32]{\includegraphics[height = 1.3in, width=2in]{figures/DP_real_32.pdf}\label{fig:real_dp1}}
  \hspace{3mm}
  \subfloat[K=64]{\includegraphics[height = 1.3in, width=2in]{figures/DP_real_64.pdf}\label{fig:real_dp2}}
  \hspace{3mm}
  \subfloat[K=128]{\includegraphics[height = 1.3in, width=2in]{figures/DP_real_128.pdf}\label{fig:real_dp3}}
  \caption{Fixed batch size (baseline) time vs Time outputted from Dynamic Programming for AlexNet with conventional pruning.}
\end{figure*}

 
   

\begin{table}[h!]
\centering
\begin{tabular}{|c|c|c|c|}
\hline
 Layer & 1.5x & 2x & 2.5x \\ \hline
 conv1 & 2 & 4 & 6    \\ \hline
 norm1 & 4 & 4 & 6    \\ \hline
 pool1 & 4 & 4 & 6    \\ \hline
 conv2 & 4 & 4 & 6    \\ \hline
 norm2 & 4 & 4 & 6    \\ \hline
 pool2 & 4 & 4& 6    \\ \hline
 conv3 & 4 & 4 & 6    \\ \hline
 conv4 & 4 & 4 & 6    \\ \hline
 conv5 & 4 & 4 & 6    \\ \hline
 pool5 & 4 & 4 & 32    \\ \hline
   fc6 & 64 & 64 & 60    \\ \hline
   fc7 & 64& 64 & 60    \\ \hline
   fc8 & 64& 64 & 60    \\ \hline
\end{tabular}
\caption{Variable batching for AlexNet.}
\label{tab:dp_path}
\end{table}


\begin{figure*}[!ht]
  \centering
  \subfloat[Pruning = 70\%]{\includegraphics[height = 1.3in, width=2in]{figures/DP_70_64.pdf}\label{fig:70_dp1}}
  \hspace{3mm}
  \subfloat[Pruning = 80\%]{\includegraphics[height = 1.3in, width=2in]{figures/DP_80_64.pdf}\label{fig:80_dp2}}
  \hspace{3mm}
  \subfloat[Pruning = 90\%]{\includegraphics[height = 1.3in, width=2in]{figures/DP_90_64.pdf}\label{fig:90_dp3}}
  \caption{Fixed batch size (baseline) time vs Time outputted from Dynamic Programming for AlexNet with 70\%, 80\% and 90\% pruning.}
\end{figure*}

Table~\ref{tab:dp_path} shows the dynamic programming output corresponding to the above run for K = 64.
 It is observed that the optimal inferencing scheme uses smaller batch sizes for the convolution layers (because of the larger memory overhead),
 and combines intermediate outputs to perform fully connected layer operations with larger batch sizes. This matches our intuition which motivated us to 
 develop the dynamic programming based solution. The dynamic programming solution  corresponding to column 2.5x picks 60 as the batchsize for final layers: thus for this case, we again compute the solution for requested input of 4, and report the total time for inferencing.  The baseline corresponding to these runs use fixed batch size of 3, 5 and 7 for  additional memory
 of  1.5x, 2x and 2.5x respectively.
 
 Figure~\ref{fig:70_dp1} - ~\ref{fig:90_dp3} extends our   results to the other configurations of the AlexNet model, namely, the
70\%, 80\% and 90\% pruned models. We show these results for fixed K of 64.  Our results show that the dynamic programming approach
performs well  over the fixed batch size approach for these scenarios as well.


\section{Concluding Remarks and Future work}
\label{sec:conc}

\begin{comment}
\begin{figure}
\includegraphics[width=\linewidth]{figs/beyond_tss_lesion.pdf}
\caption[]{End-to-End runtime lesion study of the entire MNIST dataset and the FMA featurized music dataset. Each of DROP's contributions provides a runtime improvement.}
\label{fig:beyond_lesion}
\end{figure}
\end{comment}



\section{Conclusion}
\label{sec:conclusion}

Advanced data analytics techniques must scale to rising data volumes. 
DR techniques offer a powerful toolkit when processing these datasets, with PCA frequently outperforming popular techniques in exchange for high computational cost. 
In response, we propose DROP, a new dimensionality reduction optimizer. 
DROP combines progressive sampling, progress estimation, and online aggregation to identify high quality low dimensional bases via PCA without processing the entire dataset by balancing the runtime of downstream tasks and achieved dimensionality. 
Thus, DROP provides a first step in bridging the gap between quality and efficiency in end-to-end DR for downstream \red{analytics}. 

%We revisit canonical operators for time series dimensionality reduction and the measurement study of~\cite{keogh-study}, and show that PCA is more effective than popular alternatives in the data mining literature often by a margin of over $2\times$ on average on gold-standard time series benchmark data sets with respect to output data dimension. More surprisingly, we empirically demonstrate that a small number of samples are sufficient to accurately characterize directions of maximum variance and obtain a high-quality low-dimensional transformation.



\bibliographystyle{plain}
\bibliography{ms}

\end{document}

