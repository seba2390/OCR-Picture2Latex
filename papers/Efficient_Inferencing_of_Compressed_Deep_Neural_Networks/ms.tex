\documentclass[conference,10pt]{IEEEtran}  

%\usepackage{fullpage}
\usepackage{array}
\usepackage{amssymb}
\usepackage{amsmath}
\usepackage{graphicx}
\usepackage{xcolor}
\usepackage{subfig}
\usepackage{boxedminipage}
\usepackage{hyperref}
\usepackage[ruled,vlined]{algorithm2e}
\usepackage{algorithmic}
\usepackage{balance}
\usepackage{amsmath}
\usepackage{xy}
\xyoption{all}

\newtheorem{theorem}{\bf Theorem}[section]
\newtheorem{corollary}[theorem]{\bf Corollary}
\newtheorem{lemma}[theorem]{\bf Lemma}
\newtheorem{fact}[theorem]{\bf Fact}
\newtheorem{definition}[theorem]{\bf Definition}
\newtheorem{proposition}[theorem]{\bf Proposition}
\newtheorem{claim}[theorem]{\bf Claim}
\newtheorem{observation}[theorem]{\bf Observation}

\newcommand{\qed} {\hfill$\Box$}
\newcommand{\eat}[1] {}
\newcommand{\myproof} {\IEEEproof}
\newcommand{\calP} {{\cal P}}
\newcommand{\calA} {{\cal A}}
\newcommand{\calE} {{\cal E}}
\newcommand{\calT} {{\cal T}}
\newcommand{\calH} {{\cal H}}
\newcommand{\calC} {{\cal C}}
\newcommand{\calD} {{\cal D}}
\newcommand{\Acc} {{\rm Acc}}
\newcommand{\Inst} {{\rm Inst}}
\newcommand{\wh}[1] {\widehat{#1}}
\newcommand{\wt}[1] {\widetilde{#1}}
\newcommand{\opt} {{\tt Opt}}
\newcommand{\LCA} {{\rm LCA}}
\newcommand{\mypath} {{\rm path}}
\newcommand{\val} {{\rm val}}
\newcommand{\mypred} {{\rm pred}}
\newcommand{\mysucc} {{\rm succ}}
\newcommand{\TMIS} {{\rm Time(MIS)}}
\newcommand{\pair}[2] {\langle #1,#2 \rangle}
\newcommand{\BuildBTD} {{\tt BuildBalTD}}
\newcommand{\BuildITD} {{\tt BuildIdealTD}}
\newcommand{\ceil}[1] {\lceil #1 \rceil}
\newcommand{\floor}[1] {\lfloor #1 \rfloor}
\newcommand{\rt} {{\rm rt}}
\newcommand{\dl} {{\rm dl}}
\newcommand{\len} {{\rm len}}
\newcommand{\depth} {{\rm depth}}
\newcommand{\mymid} {{\rm mid}}
\newcommand{\calF} {{\cal F}}
\newcommand{\idx} {{\rm idx}}
%\newcommand{\color} {{\rm color}}
\newcommand{\Bmax} {B_{\max}}
\newcommand{\Bmin} {B_{\min}}
\newcommand{\hmax} {h_{\max}}
\newcommand{\hmin} {h_{\min}}
\newcommand{\pmax} {p_{\max}}
\newcommand{\pmin} {p_{\min}}
\newcommand{\IS} {{\rm IS}}

\title{Efficient Inferencing of Compressed Deep Neural Networks}
%\author{
%\IEEEauthorblockN {---}
%\IEEEauthorblockA{IBM Research, New Delhi, India.}
%}

\author{
	\IEEEauthorblockN{
		Dharma Teja Vooturi \IEEEauthorrefmark{1},
		Saurabh Goyal\IEEEauthorrefmark{2}
		Anamitra R. Choudhury\IEEEauthorrefmark{2},
		Yogish Sabharwal\IEEEauthorrefmark{2},
		Ashish Verma\IEEEauthorrefmark{2}
	}\vspace{1mm}
	\IEEEauthorblockA{
		\hspace{1mm}
		\IEEEauthorrefmark{1}
		IIIT Hyderabad, India \hspace{9mm}
		Email: dharmateja.vooturi@research.iiit.ac.in
	}
	\IEEEauthorblockA{
		\hspace{30mm}
		\IEEEauthorrefmark{2}		
		IBM Research - India\hspace{10mm}
		Email: \{sgoyal30, anamchou, ysabharwal, vashish\}@in.ibm.com
	}
}



\begin{document}

\maketitle              

\begin{abstract}
Large number of weights in deep neural networks makes the models difficult to be deployed in low memory environments such as, mobile phones, IOT edge devices as well as ``inferencing as a service" environments on cloud. Prior work has considered  reduction in the size of the models, through compression techniques like pruning, quantization, Huffman encoding etc. However, efficient inferencing using the compressed models has received little attention, specially with the Huffman encoding in place.  In this paper, we  propose efficient parallel algorithms for inferencing of single image and batches, under various memory constraints.  
Our experimental results show that our approach of using variable batch size for inferencing 
achieves 15-25\%   performance improvement  in the inference throughput for AlexNet, 
while maintaining  memory and latency constraints.


\end{abstract}

\section{Introduction}
% !TEX root = ../arxiv.tex

Unsupervised domain adaptation (UDA) is a variant of semi-supervised learning \cite{blum1998combining}, where the available unlabelled data comes from a different distribution than the annotated dataset \cite{Ben-DavidBCP06}.
A case in point is to exploit synthetic data, where annotation is more accessible compared to the costly labelling of real-world images \cite{RichterVRK16,RosSMVL16}.
Along with some success in addressing UDA for semantic segmentation \cite{TsaiHSS0C18,VuJBCP19,0001S20,ZouYKW18}, the developed methods are growing increasingly sophisticated and often combine style transfer networks, adversarial training or network ensembles \cite{KimB20a,LiYV19,TsaiSSC19,Yang_2020_ECCV}.
This increase in model complexity impedes reproducibility, potentially slowing further progress.

In this work, we propose a UDA framework reaching state-of-the-art segmentation accuracy (measured by the Intersection-over-Union, IoU) without incurring substantial training efforts.
Toward this goal, we adopt a simple semi-supervised approach, \emph{self-training} \cite{ChenWB11,lee2013pseudo,ZouYKW18}, used in recent works only in conjunction with adversarial training or network ensembles \cite{ChoiKK19,KimB20a,Mei_2020_ECCV,Wang_2020_ECCV,0001S20,Zheng_2020_IJCV,ZhengY20}.
By contrast, we use self-training \emph{standalone}.
Compared to previous self-training methods \cite{ChenLCCCZAS20,Li_2020_ECCV,subhani2020learning,ZouYKW18,ZouYLKW19}, our approach also sidesteps the inconvenience of multiple training rounds, as they often require expert intervention between consecutive rounds.
We train our model using co-evolving pseudo labels end-to-end without such need.

\begin{figure}[t]%
    \centering
    \def\svgwidth{\linewidth}
    \input{figures/preview/bars.pdf_tex}
    \caption{\textbf{Results preview.} Unlike much recent work that combines multiple training paradigms, such as adversarial training and style transfer, our approach retains the modest single-round training complexity of self-training, yet improves the state of the art for adapting semantic segmentation by a significant margin.}
    \label{fig:preview}
\end{figure}

Our method leverages the ubiquitous \emph{data augmentation} techniques from fully supervised learning \cite{deeplabv3plus2018,ZhaoSQWJ17}: photometric jitter, flipping and multi-scale cropping.
We enforce \emph{consistency} of the semantic maps produced by the model across these image perturbations.
The following assumption formalises the key premise:

\myparagraph{Assumption 1.}
Let $f: \mathcal{I} \rightarrow \mathcal{M}$ represent a pixelwise mapping from images $\mathcal{I}$ to semantic output $\mathcal{M}$.
Denote $\rho_{\bm{\epsilon}}: \mathcal{I} \rightarrow \mathcal{I}$ a photometric image transform and, similarly, $\tau_{\bm{\epsilon}'}: \mathcal{I} \rightarrow \mathcal{I}$ a spatial similarity transformation, where $\bm{\epsilon},\bm{\epsilon}'\sim p(\cdot)$ are control variables following some pre-defined density (\eg, $p \equiv \mathcal{N}(0, 1)$).
Then, for any image $I \in \mathcal{I}$, $f$ is \emph{invariant} under $\rho_{\bm{\epsilon}}$ and \emph{equivariant} under $\tau_{\bm{\epsilon}'}$, \ie~$f(\rho_{\bm{\epsilon}}(I)) = f(I)$ and $f(\tau_{\bm{\epsilon}'}(I)) = \tau_{\bm{\epsilon}'}(f(I))$.

\smallskip
\noindent Next, we introduce a training framework using a \emph{momentum network} -- a slowly advancing copy of the original model.
The momentum network provides stable, yet recent targets for model updates, as opposed to the fixed supervision in model distillation \cite{Chen0G18,Zheng_2020_IJCV,ZhengY20}.
We also re-visit the problem of long-tail recognition in the context of generating pseudo labels for self-supervision.
In particular, we maintain an \emph{exponentially moving class prior} used to discount the confidence thresholds for those classes with few samples and increase their relative contribution to the training loss.
Our framework is simple to train, adds moderate computational overhead compared to a fully supervised setup, yet sets a new state of the art on established benchmarks (\cf \cref{fig:preview}).

\section{Discussion on use cases and challenges}
\label{motivation}
Today, a large number of Artificial Intelligence (AI) applications rely on using deep learning models for various tasks, such as, image classification, speech recognition, natural language understanding, natural language generation and so on.
Due to the significant improvement in performance achieved by the deep learning models, there is a natural trend to use these models on the applications running on mobile phone and other edge devices in the context of IOT (Internet of Things). 
For example, more and more people now want to take pictures using their mobile phones and get information on the building and surroundings around them in a foreign place. Usage of voice based assistants on mobile phones and other home devices is another increasing trend. 
Applications in the area of augmented reality involves continuous image recognition with results being reported on a VR display to provide more information regarding the environment to the individual. 
For example, in security, this can be used for identity detection.
Similarly, in self-driven cars, deep learning models are used
to inference in real-time using data collected from a combination of sensing technologies including visual sensors, such as cameras, and range-to-object detecting sensors, such as lasers and radar. 
Increased instrumentation in various industries such as agriculture, manufacturing, renewable energy and retail generates lot structured and unstructured data which preferably needs to be analyzed at the edge device and so that real-time action can be taken.

For the scenarios described above, inferencing can be done either on the cloud (or server) or on the edge device itself. However, offloading  inferencing to  the cloud can be impractical in lot of situations due to  wireless  energy  overheads, turn-around latencies and data security reasons. On the other hand, given the sheer size of the deep learning models, inferencing on mobile/edge devices poses other kind of challenges on resources, such as memory, compute and energy which need to be utilized efficiently while continuing to provide high accuracy and similar latency.

%phone applications

Even when inferencing is done on the cloud, resources have to be efficiently utilized to keep the cost of inferencing minimum for the cloud vendor as the cost of inferencing is directly dependent on resource utilization. Just as an example, a vendor providing "Inferencing as a service" for image classification may want to keep hundreds of deep learning models customized for various domains and users in memory in order to provide the low response time. This calls for storing compressed models in-memory and directly inferencing using the compressed model when the requests come in. All of this has to be done without compromising on the latency and accuracy of the inferencing.

%1) inferencing on mobile phones without compromising latency and accuracy 
%2)  inferencing over a large cluster of customized models on cloud "inferencing as a service" 
%3) inferencing over edge devices in the context of IOT 
%4) inferencing on multiple models (speech, vision, ..) on driver less cars in limited memory where latency is critical

\section{Preliminaries}
\label{sec:prelims}
\section{Our Approach}
We formulate the problem as an anisotropic diffusion process and the diffusion tensor is learned through a deep CNN directly from the given image, which guides the refinement of the output.

\begin{figure}[t]
\includegraphics[width=1.0\textwidth]{fig/CSPN_SPN2.pdf}
\caption{Comparison between the propagation process in SPN~\cite{liu2017learning} and CPSN in this work.}
\label{fig:compare}
\end{figure}

\subsection{Convolutional Spatial Propagation Network}
% demonstrate the thereom is hold when turns to be convolution.
Given a depth map $D_o \in \spa{R}^{m\times n}$ that is output from a network, and image $\ve{X} \in \spa{R}^{m\times n}$, our task is to update the depth map to a new depth map $D_n$ within $N$ iteration steps, which first reveals more details of the image, and second improves the per-pixel depth estimation results. 

\figref{fig:compare}(b) illustrates our updating operation. Formally, without loss of generality, we can embed the $D_o$ to some hidden space $\ve{H} \in \spa{R}^{m \times n \times c}$. The convolutional transformation functional with a kernel size of $k$ for each time step $t$ could be written as,
\begin{align}
    \ve{H}_{i, j, t + 1} &= \sum\nolimits_{a,b = -(k-1)/2}^{(k-1)/2} \kappa_{i,j}(a, b) \odot \ve{H}_{i-a, j-b, t} \nonumber \\
\mbox{where,~~~~}
    \kappa_{i,j}(a, b) &= \frac{\hat{\kappa}_{i,j}(a, b)}{\sum_{a,b, a, b \neq 0} |\hat{\kappa}_{i,j}(a, b)|}, \nonumber\\
    \kappa_{i,j}(0, 0) &= 1 - \sum\nolimits_{a,b, a, b \neq 0}\kappa_{i,j}(a, b)
\label{eqn:cspn}
\end{align}
where the transformation kernel $\hat{\kappa}_{i,j} \in \spa{R}^{k\times k \times c}$ is the output from an affinity network, which is spatially dependent on the input image. The kernel size $k$ is usually set as an odd number so that the computational context surrounding pixel $(i, j)$ is symmetric.
$\odot$ is element-wise product. Following~\cite{liu2017learning}, we normalize kernel weights between range of $(-1, 1)$ so that the model can be stabilized and converged by satisfying the condition $\sum_{a,b, a,b \neq 0} |\kappa_{i,j}(a, b)| \leq 1$. Finally, we perform this iteration $N$ steps to reach a stationary distribution.

% theorem, it follows diffusion with PDE 
%\addlinespace
\noindent\textbf{Correspondence to diffusion process with a partial differential equation (PDE).} \\
Similar with~\cite{liu2017learning}, here we show that our CSPN holds all the desired properties of SPN.
Formally, we can rewrite the propagation in \equref{eqn:cspn} as a process of diffusion evolution by first doing column-first vectorization of feature map $\ve{H}$ to $\ve{H}_v \in \spa{R}^{\by{mn}{c}}$.
\begin{align}
     \ve{H}_v^{t+1} = 
     \begin{bmatrix}
    1-\lambda_{0, 0}  & \kappa_{0,0}(1,0) & \cdots & 0 \\
    \kappa_{1,0}(-1,0)   & 1-\lambda_{1, 0} & \cdots & 0 \\
    \vdots & \vdots & \ddots & \vdots \\
    \vdots & \cdots & \cdots & 1-\lambda_{m,n} \\
\end{bmatrix} = \ve{G}\ve{H}_v^{t}
\label{eqn:vector}
\end{align}
where $\lambda_{i, j} = \sum_{a,b}\kappa_{i,j}(a,b)$ and $\ve{G}$ is a $\by{mn}{mn}$ transformation matrix. The diffusion process expressed with a partial differential equation (PDE) is derived as follows, 
\begin{align}
     \ve{H}_v^{t+1} &= \ve{G}\ve{H}_v^{t} = (\ve{I} - \ve{D} + \ve{A})\ve{H}_v^{t} \nonumber\\
     \ve{H}_v^{t+1} - \ve{H}_v^{t} &= - (\ve{D} - \ve{A}) \ve{H}_v^{t} \nonumber\\
     \partial_t \ve{H}_v^{t+1} &= -\ve{L}\ve{H}_v^{t}
\label{eqn:proof}
\end{align}
where $\ve{L}$ is the Laplacian matrix, $\ve{D}$ is the diagonal matrix containing all the $\lambda_{i, j}$, and $\ve{A}$ is the affinity matrix which is the off diagonal part of $\ve{G}$.

In our formulation, different from~\cite{liu2017learning} which scans the whole image in four directions~(\figref{fig:compare}(a)) sequentially, CSPN propagates a local area towards all directions at each step~(\figref{fig:compare}(b)) simultaneously, \ie with~\by{k}{k} local context, while larger context is observed when recurrent processing is performed, and the context acquiring rate is in an order of $O(kN)$.

In practical, we choose to use convolutional operation due to that it can be efficiently implemented through image vectorization, yielding real-time performance in depth refinement tasks.

Principally, CSPN could also be derived from loopy belief propagation with sum-product algorithm~\cite{kschischang2001factor}. However, since our approach adopts linear propagation, which is efficient while just a special case of pairwise potential with L2 reconstruction loss in graphical models. Therefore, to make it more accurate, we call our strategy convolutional spatial propagation in the field of diffusion process.

\begin{figure}[t]
\centering
\includegraphics[width=0.9\textwidth]{fig/hist.pdf}
\caption {(a) Histogram of RMSE with depth maps from~\cite{Ma2017SparseToDense} at given sparse depth points.  (b) Comparison of gradient error between depth maps with sparse depth replacement (blue bars) and with ours CSPN (green bars), where ours is much smaller. Check~\figref{fig:gradient} for an example. Vertical axis shows the count of pixels.}
\label{fig:hist}
\end{figure}

\subsection{Spatial Propagation with Sparse Depth Samples}
In this application, we have an additional sparse depth map $D_s$ (\figref{fig:gradient}(b)) to help estimate a depth depth map from a RGB image. Specifically, a sparse set of pixels are set with real depth values from some depth sensors, which can be used to guide our propagation process. 

Similarly, we also embed the sparse depth map $D_s = \{d_{i,j}^s\}$ to a hidden representation $\ve{H}^s$,  and we can write the updating equation of $\ve{H}$ by simply adding a replacement step after performing \equref{eqn:cspn}, 
\begin{align}
    \ve{H}_{i, j, t+1} = (1 - m_{i, j}) \ve{H}_{i, j, t+1}  +  m_{i, j} \ve{H}_{i, j}^s 
\label{eqn:cspn_sp}
\end{align}
where $m_{i, j} = \spa{I}(d_{i, j}^s > 0)$ is an indicator for the availability of sparse depth at $(i, j)$. 

In this way, we guarantee that our refined depths have the exact same value at those valid pixels in sparse depth map. Additionally, we propagate the information from those sparse depth to its surrounding pixels such that the smoothness between the sparse depths and their neighbors are maintained. 
Thirdly, thanks to the diffusion process, the final depth map is well aligned with image structures. 
This fully satisfies the desired three properties for this task which is discussed in our introduction (\ref{sec:intro}). 

% it performs a non-symmetric propagation where the information can only be diffused from the given sparse depth to others, while not the other way around.

% still follows PDE
In addition, this process is still following the diffusion process with PDE, where the transformation matrix can be built by simply replacing the rows satisfying $m_{i, j} = 1$ in $\ve{G}$ (\equref{eqn:vector}), which are corresponding to sparse depth samples, by $\ve{e}_{i + j*m}^T$. Here $\ve{e}_{i + j*m}$ is an unit vector with the value at $i + j*m$ as 1.
Therefore, the summation of each row is still $1$, and obviously the stabilization still holds in this case.

\begin{figure}[t]
\centering
\includegraphics[width=0.95\textwidth]{fig/fig2.pdf}
\caption{Comparison of depth map~\cite{Ma2017SparseToDense} with sparse depth replacement and with our CSPN \wrt smoothness of depth gradient at sparse depth points. (a) Input image. (b) Sparse depth points. (c) Depth map with sparse depth replacement. (d) Depth map with our CSPN with sparse depth points. We highlight the differences in the red box.}
\label{fig:gradient}
\end{figure}

Our strategy has several advantages over the previous state-of-the-art sparse-to-dense methods~\cite{Ma2017SparseToDense,LiaoHWKYL16}.
In \figref{fig:hist}(a), we plot a histogram of depth displacement from ground truth at given sparse depth pixels from the output of Ma \etal~\cite{Ma2017SparseToDense}. It shows the accuracy of sparse depth points cannot preserved, and some pixels could have very large displacement (0.2m), indicating that directly training a CNN for depth prediction does not preserve the value of real sparse depths provided. To acquire such property, 
one may simply replace the depths from the outputs with provided sparse depths at those pixels, however, it yields non-smooth depth gradient \wrt surrounding pixels. 
In~\figref{fig:gradient}(c), we plot such an example, at right of the figure, we compute Sobel gradient~\cite{sobel2014history} of the depth map along x direction, where we can clearly see that the gradients surrounding pixels with replaced depth values are non-smooth.
We statistically verify this in \figref{fig:hist}(b) using 500 sparse samples, the blue bars are the histogram of gradient error  at sparse pixels by comparing the gradient of the depth map with sparse depth replacement and of ground truth depth map. We can see the difference is significant, 2/3 of the sparse pixels has large gradient error.
Our method, on the other hand, as shown with the green bars in \figref{fig:hist}(b), the average gradient error is much smaller, and most pixels have zero error. In\figref{fig:gradient}(d), we show the depth gradients surrounding sparse pixels are smooth and close to ground truth, demonstrating the effectiveness of our propagation scheme. 

% Finally, in our experiments~\ref{sec:exp}, we validate the number of iterations $N$ and kernel size $k$ used for doing the CSPN.


\subsection{Complexity Analysis}
\label{subsec:time}

As formulated in~\equref{eqn:cspn}, our CSPN takes the operation of convolution, where the complexity of using CUDA with GPU for one step CSPN is $O(\log_2(k^2))$, where $k$ is the kernel size. This is because CUDA uses parallel sum reduction, which has logarithmic complexity. In addition,  convolution operation can be performed parallel for all pixels and channels, which has a constant complexity of $O(1)$. Therefore, performing $N$-step propagation, the overall complexity for CSPN is $O(\log_2(k^2)N)$, which is irrelevant to image size $(m, n)$.

SPN~\cite{liu2017learning} adopts scanning row/column-wise propagation in four directions. Using $k$-way connection and running in parallel, the complexity for one step is $O(\log_2(k))$. The propagation needs to scan full image from one side to another, thus the complexity for SPN is $O(\log_2(k)(m + n))$. Though this is already more efficient than the densely connected CRF proposed by~\cite{philipp2012dense}, whose implementation complexity with permutohedral lattice is $O(mnN)$, ours $O(\log_2(k^2)N)$ is more efficient since the number of iterations $N$ is always much smaller than the size of image $m, n$. We show in our experiments (\secref{sec:exp}), with $k=3$ and $N=12$, CSPN already outperforms SPN with a large margin (relative $30\%$), demonstrating both efficiency and effectiveness of the proposed approach.


\subsection{End-to-End Architecture}
\label{subsec:unet}
\begin{figure}[t]
\centering
\includegraphics[width=0.95\textwidth,height=0.45\textwidth]{fig/framework2.pdf}
\caption{Architecture of our networks with mirror connections for  depth estimation via transformation kernel prediction with CSPN (best view in color). Sparse depth is an optional input, which can be embedded into the CSPN to guide the depth refinement.}
\label{fig:arch}
\end{figure}

We now explain our end-to-end network architecture to predict both the transformation kernel and the depth value, which are the inputs to CSPN for depth refinement.
 As shown in \figref{fig:arch}, our network has some similarity with that from Ma \etal~\cite{Ma2017SparseToDense}, with the final CSPN layer that outputs a dense depth map.  
 
For predicting the transformation kernel $\kappa$ in \equref{eqn:cspn}, 
rather than building a new deep network for learning affinity same as Liu \etal~\cite{liu2017learning}, we branch an additional output from the given network, which shares the same feature extractor with the depth network. This helps us to save memory and time cost for joint learning of both depth estimation and transformation kernels prediction. 

Learning of affinity is dependent on fine grained spatial details of the input image. However, spatial information is weaken or lost with the down sampling operation during the forward process of the ResNet in~\cite{laina2016deeper}. Thus, we add mirror connections similar with the U-shape network~\cite{ronneberger2015u} by directed concatenating the feature from encoder to up-projection layers as illustrated by ``UpProj$\_$Cat'' layer in~\figref{fig:arch}. Notice that it is important to carefully select the end-point of mirror connections. Through experimenting three possible positions to append the connection, \ie after \textit{conv}, after \textit{bn} and after \textit{relu} as shown by the ``UpProj'' layer in~\figref{fig:arch} , we found the last position provides the best results by validating with the NYU v2 dataset (\secref{subsec:ablation}). 
In doing so, we found not only the depth output from the network is better recovered, and the results after the CSPN is additionally refined, which we will show the experiment section~(\secref{sec:exp}).
Finally we adopt the same training loss as~\cite{Ma2017SparseToDense}, yielding an end-to-end learning system.


\section{Inferencing using Compressed Models}
\label{sec:inference}

In this section, we discuss the various approaches for inferencing using the compressed model,
where the compressed model is stored  in  the format as shown in the previous section.
Clearly, the trivial method of exploding the model back to the dense format  and doing the computation 
(using standard frameworks like Caffe, Tensorflow etc) is not a good choice since the entire purpose of
model compression gets defeated because  of the 
excessive memory usage. The other extreme of decoding element by element of the matrix and doing the operations on the
decoded element 
has little memory overhead, but is computationally inefficient.
This calls for the need to develop an efficient stand-alone module (independent of the
Caffe/Tensorflow framework) for inferencing using the compressed model.
The na{\"i}ve algorithm for doing the inferencing is presented in Algorithm~\ref{alg:pseudocode1}.
The idea here is to work  sequentially on the individual rows of the weight matrix (line 3).
For a particular row, the $col\_ind$ and the $val$ entries for that row are first  Huffman-decoded (line 5-6);
this is followed by converting relative column index of $col\_ind$ to absolute index (line 7) 
and creating an $abs\_val$ array which  is essentially the $val$ array with its entries replaced by the
corresponding codebook entires. 
All these steps  in fact  create the arrays in Figure~\ref{fig:repr_f1} from 
Figure~\ref{fig:repr_f5} for a particular row segment.
%Finally the dot product of $abs\_val$ and input activation is done by
%invoking MKL routine for sparse matrix-vector product. For multiple batches,
Finally we call MKL routine $mkl\_scsrmm$ for 
sparse matrix-matrix multiplication  of $abs\_val(i)$ and $a$
to compute $b[i,:]$.

%   some standard library:
%in our work, we apply Intel MKL kernels to effe
 %the 
%and then evaluating the dot product of the respective codebook entries
%corresponding to the column entries with the input activation (line 9-12).


\begin{algorithm}[t]
\caption{Na{\"i}ve algorithm for inferencing using compressed model }
\label{alg:pseudocode1}
\small
\begin{algorithmic}[1]
	\STATE Input: $row\_ptr$ array, entry $i$ of which is a tuple 
	of  starting address of row $i$ in $val$ and that in $col\_ind$.\\
	$val$ Huffman encoded cluster index bit stream. \\
	$col\_ind$ Huffman encoded rel. indexed column bit stream. \\
	$\calC$ codebook of quantized weights. \\
	$a$ input activation matrix. \\
	\STATE Output: $b$ output  activation  matrix. \\
		 
	\FOR{every entry $i$ of the $row\_ptr$ array}
		\STATE Set $val\_begin(i)$,  $val\_end(i)$, $col\_begin(i)$, $col\_end(i)$\\
		 for row $i$ as follows \\
		 \quad  \quad $\langle val\_begin(i), col\_begin(i) \rangle \leftarrow row\_ptr(i)$\\
		 \quad  \quad $\langle val\_end(i), col\_end(i) \rangle \leftarrow row\_ptr(i+1)$.
		 \STATE $dec\_val(i)$ $\leftarrow$ Huffman decoding of bit stream in $val$ between $val\_begin(i)$ and  $val\_end(i)$.
		\STATE  $dec\_col(i)$ $\leftarrow$ Huffman decoding of bit stream in \\
		$col\_ind$ between $col\_begin(i)$ and  $col\_end(i)$.
		\STATE $abs\_col(i)$ $\leftarrow$  Prefix sum of $dec\_col(i)$.
		\STATE Set $abs\_val(i)[j]$ $\leftarrow$ $\calC [dec\_val(i)[j]]$ , $\forall j$.
		\STATE $b[i, :]$ += MKL\_CSRMM($abs\_val(i)$, $a$)
%		\FOR{every entry $j$ of the $abs\_col(i)$ array}
%			\STATE col = $abs\_col(i)[j]$.
%			\STATE $b_i$+=$a_{col}$ * $\calC [dec\_val(i)[j]]$.
%		\ENDFOR	 
	\ENDFOR	
\end{algorithmic}
\end{algorithm}

The above algorithm can be parallelized by employing different threads to operate on different rows of the weight matrix.
Moreover MKL internally can use multiple threads for sparse matrix operations.
However  Algorithm~\ref{alg:pseudocode1} faces multiple drawbacks.
Firstly, the algorithm decodes an entire row of the matrix, and thus the memory requirement becomes
significant for large matrices. 
Secondly, most  algorithms for matrix multiplication work more efficiently using  blocks rather than individual elements, to achieve necessary reuse of data in local memory.
The advantage of this approach is that the small blocks can be moved into the fast local memory and their elements can then be repeatedly used.
This motivates us to employ blocking even for compressed model inferencing, which we describe next.




\subsection{Blocking of Weight Matrix}

 The
general idea of blocking is to organize the data structures in a program into  chunks called blocks. The program is
structured so that it loads a block into the L1 cache, does all the reads and writes that it needs to on that
block, then discards the block, loads in the next block, and so on. 
Similar to standard matrix multiplication, the blocking algorithm for inferencing shall work 
 by partitioning the matrices into submatrices and then exploiting
the mathematical fact that these submatrices can be manipulated just like scalars.
%However, our inferencing procedure first needs to perform Huffman decoding followed by
%absolute index computation on the submatrices before the block multiplication can be performed.
Instead of storing the  original weight matrix in row major format,
we need to ensure that any particular block of the matrix is stored in contiguous memory.  
This will make certain	 that the Huffman decoding happens on contiguous memory locations and  generates the submatrix 
corresponding to a block.


See Figure~\ref{fig:block_f1} and Figure~\ref{fig:block_f2} for illustration.
Suppose the original weight matrix  stored in dense row major format is of dimension 8x8, and we decide to work on blocks each sized 4x4. 
We first convert  this matrix to   4 x 16 format, such that each row of the new matrix stores 
elements of the corresponding block of the old matrix in contiguous locations. This new matrix 
is then stored in CSR format with relative indexing and Huffman encoding, as discussed in the 
previous section. 
\\
{\it Size of the modified model}:


It is observed that the non zeroes  in the weight matrix are uniformly distributed, thus
the size of the $val$ and $col\_ind$ vectors does not change a lot  (even with zero padding in the compressed format)  when the matrix is
stored in block contiguous fashion.
The number of rows in the modified matrix is same as the number of blocks in the original matrix, and may be larger or smaller than that
in the  original matrix depending on the block size. From experimental results, it is however observed, that
change in model size due to this difference in the size of the $row\_ptr$ is insignificant. Hence we can assume that 
storing the model in block contiguous fashion does not add to memory overhead.



\begin{figure}[!tbp]
  \centering
  \subfloat[Original Connection Matrix.]{\includegraphics[width=2in]{figures/block1.pdf}\label{fig:block_f1}}
 % \hfill
 \hspace{10mm}
  \subfloat[Modified Connection Matrix]{\includegraphics[width=2.5in]{figures/block2.pdf}\label{fig:block_f2}}
  \caption{Representation of a compressed model.}
\end{figure}



\subsection{Blocked Inferencing Procedure}

Next we present our inferencing algorithm using the blocked storage scheme. 
Our algorithm ensures that once a  row of the connection matrix (which corresponds to a block
in the original weight matrix) is decoded,
the decoded entries are used for all the computations that require them.
This is illustrated in Figure~\ref{fig:block_mult}. A row is decoded and multiplied with all possible subblocks of 
input activation matrix to generate partial results for the output activation matrix. 
The blocked inferencing algorithm is presented in Algorithm~\ref{alg:pseudocode2}.


\begin{figure}[b]
\centering
\includegraphics[width=3in]{figures/mult.pdf}
\caption{Blocked inference scheme.}
\label{fig:block_mult}
\end{figure}





\begin{algorithm}[t]
\caption{Algorithm for block inferencing}
\label{alg:pseudocode2}
\small
\begin{algorithmic}[1]
	\STATE Input: Compressed model stored in $bh$ x $bw$ block contiguous manner with \\
	$row\_ptr$ array, entry $i$ of which is a tuple $\langle x,y \rangle$ \\
	where $x$ and $y$ being respectively starting address of row $i$ in $val$ and 	that in $col\_ind$.\\
	$val$ Huffman encoded cluster index bit stream. \\
	$col\_ind$ Huffman encoded relative indexed column bit stream. \\
	$\calC$ codebook of quantized weights. \\
	$a$ input activation matrix with $a_{rows}$ rows\\
	\STATE Output: $b$ output  activation matrix\\
		 
	\FOR{every entry $i$ of the $row\_ptr$ array}
		\STATE Set $val\_begin(i)$,  $val\_end(i)$, $col\_begin(i)$, $col\_end(i)$\\
		 for row $i$ as follows \\
		 \quad  \quad $\langle val\_begin(i), col\_begin(i) \rangle \leftarrow row\_ptr(i)$\\
		 \quad  \quad $\langle val\_end(i), col\_end(i) \rangle \leftarrow row\_ptr(i+1)$.
		\STATE $dec\_val(i)$ $\leftarrow$ Huffman decoding of bit stream in $val$ between $val\_begin(i)$ and  $val\_end(i)$.
		\STATE  $dec\_col(i)$ $\leftarrow$ Huffman decoding of bit stream in \\
		$col\_ind$ between $col\_begin(i)$ and  $col\_end(i)$.
		\STATE $abs\_col(i)$ $\leftarrow$  Prefix sum of $dec\_col(i)$.
		\STATE Set $abs\_val(i)[j]$ $\leftarrow$ $\calC [dec\_val(i)[j]]$ , $\forall j$.
		\STATE Arrange $abs\_val(i)$ as $bh$ x $bw$ block. \\
		\STATE col\_id = $ ( i \% (a_{rows}/bw) ) * bw$  \\
		\STATE row\_id = $ ( i / (a_{rows}/bw) ) * bh$  \\
		\STATE b[row\_id:(row\_id+bh-1),:] += MKL\_CSRMM($abs\_val(i)$, a[col\_id:(col\_id+bw-1),:] ) \\
	\ENDFOR	
\end{algorithmic}
\end{algorithm}





\section{Experimental Results with Blocking}
\label{sec:expt1}
In this section, we present the experimental results for our block inferencing procedure. We begin by specifying the system configurations and the dataset.

\subsection{System and Dataset}

For running our experiments (also the ones in Section~\ref{sec:expt2}), we have used Intel Xeon CPU E5-2697 system. It has two NUMA nodes 
with 12 cores, each with frequency of 2.70GHZ. The system has 32KB, 256KB and 30MB of L1, L2 and L3 cache respectively.
We consider compressed models for two popular deep neural networks, AlexNet and VGG-16. 
For each of these models we consider the compressed configurations corresponding to four different pruning percentages.
The first configuration corresponds to the procedure applied in  \cite{HanMD15}.  
Table~\ref{tab:alex_pr} and Table~\ref{tab:vgg_pr}
 present the pruning percentages of all the layers in this configuration. We refer to this configuration as 
{\em conventional} in subsequent discussion.
The compressed model sizes of AlexNet and VGG-16 for this configuration are respectively 6.81 MB and 10.64 MB.
The other three configurations correspond respectively to 70\%, 80\% and 90\% pruning of {\em all} the layers of the network.  
We consider these configurations to study how our scheme performs for a wide range of sparsity spectrum of the compressed models.
8 bit (5 bit) quantization for CONV (FC) layers
and 4 bit (5 bit) relative indexing for AlexNet (VGG-16) is employed for all the configurations.

\begin{table}[!tbp]
  \centering
  \subfloat[AlexNet]{
\begin{tabular}{|c|c|}
			\hline 
			Layer & Pruning \% \\ \hline 
			conv1 & 16 \\ \hline 
			conv2 & 62 \\ \hline 
			conv3 & 65 \\ \hline 
			conv4 & 63 \\ \hline 
			conv5 & 37 \\ \hline 
			fc6 & 91  \\ \hline 
			fc7 & 91  \\ \hline 
			fc8 & 75  \\ \hline
        \end{tabular}
        \label{tab:alex_pr}
  }
  \hspace{1mm}
    \subfloat[VGG-16]{
\begin{tabular}{|c|c|m{2em}|m{2em}|}
			\hline 
			Layer & Pruning \% \\ \hline 
			conv1\_1 & 42  \\ \hline
			conv1\_2 & 78 \\ \hline
			conv2\_1 & 66 \\ \hline
			conv2\_2 & 64 \\ \hline
			conv3\_1 & 47 \\ \hline
			conv3\_2 & 76 \\ \hline
			conv3\_3 & 58 \\ \hline
			conv4\_1 & 68 \\ \hline
			conv4\_2 & 73 \\ \hline
			conv4\_3 & 66 \\ \hline
			conv5\_1 & 65 \\ \hline
			conv5\_2 & 71 \\ \hline
			conv5\_3 & 64 \\ \hline
			fc6 & 96  \\ \hline
			fc7 & 96  \\ \hline
			fc8 & 77  \\ \hline
        \end{tabular}
        \label{tab:vgg_pr}
  }
  \caption{Compressed AlexNet and VGG-16 models.}
\end{table}


%
%\begin{table}[!htb]
%    \caption{Properties of compressed AlexNet and VGGNet Models}
%    \begin{minipage}{.5\linewidth}
%      \caption{AlexNet}
%      \centering
%        \begin{tabular}{|c|c|}
%			\hline 
%			Layer & Pruning \% \\ \hline 
%			conv1 & 16 \\ \hline 
%			conv2 & 62 \\ \hline 
%			conv3 & 65 \\ \hline 
%			conv4 & 63 \\ \hline 
%			conv5 & 37 \\ \hline 
%			fc6 & 91  \\ \hline 
%			fc7 & 91  \\ \hline 
%			fc8 & 75  \\ \hline
%        \end{tabular}
%    \end{minipage}%
%    \begin{minipage}{.5\linewidth}
%      \centering
%        \caption{VGGNet}
%        \begin{tabular}{|c|c|m{2em}|m{2em}|}
%			\hline 
%			Layer & Pruning \% \\ \hline 
%			conv1\_1 & 42  \\ \hline
%			conv1\_2 & 78 \\ \hline
%			conv2\_1 & 66 \\ \hline
%			conv2\_2 & 64 \\ \hline
%			conv3\_1 & 47 \\ \hline
%			conv3\_2 & 76 \\ \hline
%			conv3\_3 & 58 \\ \hline
%			conv4\_1 & 68 \\ \hline
%			conv4\_2 & 73 \\ \hline
%			conv4\_3 & 66 \\ \hline
%			conv5\_1 & 65 \\ \hline
%			conv5\_2 & 71 \\ \hline
%			conv5\_3 & 64 \\ \hline
%			fc6 & 96  \\ \hline
%			fc7 & 96  \\ \hline
%			fc8 & 77  \\ \hline
%        \end{tabular}
%    \end{minipage} 
%    \label{tab:compression-params}
%\end{table}
%
\subsection{Blocking results}

Our first set of experiments is aimed to study the effect of variation of  block sizes on the inference time (both the decoding time and  
the computation time) for individual layers corresponding to the different configurations of the  compressed models. 
 Figure~\ref{fig:blck_f1} and ~\ref{fig:blck_f2}
show the decoding time, computation time and total time, with different block sizes for FC6 layer of AlexNet and VGGnet,
using batch size of 16. The models used for these runs  correspond to the conventional configuration. All these experiments employ MKL with 4 threads
for computation.



We observe that for very small block sizes,  the decoding and the computation time are pretty high due to overhead of the too many function calls.
For very large block sizes, the level of parallelism gets limited, leading to increase in the inference time.
Figure~\ref{fig:blck_f3} and ~\ref{fig:blck_f4} show the same charts with batch size of 256. We note that for smaller batch size, the total time is dominated by the decoding time, 
whereas the computation time takes over at larger batch sizes. However the nature of variation of  inference time with the block size is consistent across batch sizes.
We observe similar nature of plots for other configurations and batch sizes as well.



\begin{figure*}[!tbp]
  \centering
  \subfloat[AlexNet Batch size16]{\includegraphics[width=1.7in]{figures/alex16.pdf}\label{fig:blck_f1}}
  \hspace{1mm}
  \subfloat[VGG-16 Batch size 16]{\includegraphics[width=1.7in]{figures/vgg16.pdf}\label{fig:blck_f2}}
  \hspace{1mm}
  \subfloat[AlexNet Batch size 256]{\includegraphics[width=1.7in]{figures/alex256.pdf}\label{fig:blck_f3}}
  \hspace{1mm}
  \subfloat[VGG-16 Batch size 256]{\includegraphics[width=1.7in]{figures/vgg256.pdf}\label{fig:blck_f4}}
  \caption{Inference Time Variation with Block Size.}
\end{figure*}




We also note that the working memory increases with increase in block size. 
Table~\ref{tab:workmem} presents the working memory required for matrix matrix multiplication for FC6 layer of AlexNet and VGG-16.
Since there is not significant difference in the inference timings between block sizes in range 128 x 128 to 1024 x 1024, we fix 128 x 128 as our block size
for the subsequent experiments.



\begin{table}[h!]
\centering
\begin{tabular}{|c|c|c|}
           \hline
Blocksize & AlexNet  & VGG-16 \\ \hline
  16  x   16 &    1.26KB &  0.92KB \\ \hline
  32  x   32 &    4.57KB &   3.42KB \\ \hline
  64  x   64 &   17.33KB &  12.97KB \\ \hline
 128  x  128 &   67.40KB &  50.22KB \\ \hline
 256  x  256 &  265.78KB & 197.26KB \\ \hline
 512  x  512 &    1.03MB & 781.52KB \\ \hline
1024  x 1024 &    4.11MB &   2.98MB \\ \hline
2048  x 2048 &   14.76MB &  11.42MB \\ \hline
4096  x 4096 &   36.88MB &  42.38MB \\ \hline

\end{tabular}
\caption{Working Memory Requirement for FC6 layer}
\label{tab:workmem}

\end{table}




We next observe  the variation of activation memory requirement and the inference time with batch sizes. Table~\ref{tab:batch}   presents
the results for batch sizes of 16 and 256. Clearly, for a fixed batch size,  the activation memory required by the convolution layers is  more than  that of the 
fully-connected layers.  Inferencing applications on a low resource system generally come with a cap on the available memory. 
Suppose we consider a fictitious scenario where the maximum available memory is 20MB. 
From Table~\ref{tab:batch}, it makes sense to run the fully connected layers with batch size 256, since the memory required is well below the
permissible threshold, and there is significant increase in throughput if we process in batch of 256. 
For the convolution layers, however, processing in batch of 256 is not a desirable option because of the large memory overhead.
This motivates us to use different batch sizes for different layers during the inferencing. We present this in more detail in the next section.

 


\begin{table}[h!]
\centering
\begin{tabular}{|c|c|c|c|c|}
\hline
      & \multicolumn{2}{c|}{Memory (MB)} & \multicolumn{2}{c|}{Time (ms)} \\ \hline
Layer  & batch-size & batchsize & batchsize& batchsize \\
 &16 & 256 & 16 & 256 \\ \hline
conv1  &  17.72 & 283.59 &  349.93 &  5408.93 \\ \hline
norm1  &  17.72 & 283.59 &  98.87 & 1597.83 \\ \hline
pool1  &  4.27 &  68.34 & 11.68 & 176.42 \\ \hline
conv2  &  11.39 & 182.25 &  341.72 &  5745.49 \\ \hline
norm2  &  11.39 & 182.25 &  68.06 & 1081.80 \\ \hline
pool2  &  2.64 &  42.25 & 7.12 &  116.49 \\ \hline
conv3  &  3.96 &  63.38 & 153.11 &  2573.47 \\ \hline
conv4  &  3.96 &  63.38 & 204.01 &  3135.62 \\ \hline
conv5  &  2.64 &  42.25 & 135.66 &  2242.94 \\ \hline
pool5  &  0.56 &  9.00 &  1.92 &  25.72 \\ \hline
fc6  &  0.25 &  4.00 &  51.77 & 112.62 \\ \hline
fc7  &  0.25 &  4.00 &  21.06 & 46.61 \\ \hline
fc8  &  0.06 &  0.98 &  9.66 &  22.61 \\ \hline

\end{tabular}
\caption{Memory Requirement and Inference time for AlexNet individual layers}
\label{tab:batch}
\end{table}




\subsection{Inferencing with Variable Batch Size}
\label{sec:dp}
\newcommand{\OUT} {{\rm OUT}}
\newcommand{\IN} {{\rm IN}}
\newcommand{\WS} {{\rm WS}}
\newcommand{\Time} {{\rm Time}}
\newcommand{\OPT} {{\rm OPT}}
\newcommand{\TOT} {{\rm TOT}}

It is clear from the results shown in the previous section that  using a larger batch for inferencing increases the 
throughput as computing
resources are utilized more efficiently.
However, an issue with inferencing larger batches is the increase
in inferencing latency 
(due to wait time while assembling a batch, and because
larger batches take longer to process). 
Moreover the memory requirement for the input and the output
activations and buffer memory also increases for larger batch size.
Thus applications work with large batch sizes while keeping
the latency and memory utilization within certain thresholds.
The problem becomes more challenging since the 
available memory varies dynamically depending on the system load;
hence the  batch size for achieving the maximum throughput can be figured out only at the time of inferencing.
Moreover, the memory requirement and the computation time for 
inferencing varies with the layers even for a fixed batch size. 
Thus it might be advantageous to do the inferencing using different 
batch sizes for different layers.
We address this issue by proposing a dynamic programming based algorithm
for determining variable batch sizes for different layers for efficient
inferencing.
We describe our dynamic program below.

\subsection{Dynamic Programming}
Let $L_1$, $L_2$, $\cdots$, $L_f$ denote the layers of the DNN. 
For $i$ = 1,  2, $\cdots$, $f$, let \Time$(i, B)$ denote the time required to perform the inferencing computations for layer $L_i$
of the DNN using a batch size of $B$.  
Next, we let \IN$(i,B)$ and \OUT$(i,B)$ respectively denote 
the input activation and output activation memory required
to perform inferencing of layer $L_i$ with a batch size of $B$. 
Further, let \WS$(i)$ denote the size of the temporary workspace  required for layer $L_i$ computations (for instance this includes
the buffer memory required to decode blocks of the connection matrix for $L_i$). 
All the values \IN$(i,B)$,  \OUT$(i,B)$, \WS$(i)$ and \Time$(i, B)$
are obtained once for a given compressed model.
Note that the total memory required to perform inferencing computations for layer $L_i$ with a batch size of $B$ is captured by
$$ \IN(i,B) + \WS(i) +  \OUT(i,B). $$
Let {\TOT} denote the total memory available for performing the inference
computations for the entire model.

We now describe the dynamic program to determine the optimal batch size
to be used at all the individual layers in order to maximize the overall
throughput of the inferencing.
For this we define a configuration:
a configuration is a tuple $\langle i, B, A \rangle$,
where $i$ denotes the layer $L_i$, $B$ denotes a batch size and 
$A$ denotes amount of memory.
We maintain a dynamic program table {\OPT}. 
An entry \OPT$(i, B, A)$ of the dynamic program denotes the minimum time to perform the inferencing computations for layers $L_1$-$L_i$, when a batch size of 
$B$ is used for layer $L_i$, and $A$ units of memory (out of {\TOT}) are not available for performing the inferencing computations for layers $L_1$-$L_i$ (this memory is reserved for performing inferencing computations from layer $L_{i+1}$ to layer $L_f$).
Thus, we only have available ({\TOT} - $A$) units of memory
for inference computations of layers $L_1$ to $L_i$.

We say that configuration $\langle i,B,A \rangle$ is {\em feasible} if
the total memory required for performing inferencing computations at
layer $L_i$ with a batch size of $B$ is within the available memory bound,
i.e.,
$$A + \IN(i,B) + \WS(i) +  \OUT(i,B) \leq \TOT.$$

We now describe the recurrence relation for computing the 
entries of the dynamic programming table \OPT$(\cdot, \cdot, \cdot)$.
For simplicity, we assume that for every $i$, the batch size used for inferencing computations at layer $L_{i-1}$ is no more than
the batch size used for the inferencing computations at layer $L_i$.
Clearly,  \OPT$(i, B, A)$ can be finite only if  $\langle i,B,A \rangle$ is feasible.
Suppose that layer $L_i$ is computed with batch size $B$ and layer $L_{i-1}$ with a batch size $b$. 
For simplicity, we consider all $b \leq B$ such that $b$ divides $B$.
For a given $b$, 
the inferencing computations for layer $L_{i-1}$
will be performed in $ (B/b)$ phases, wherein in each phase
a batch of size $b$ will be processed up to layer $L_{i-1}$.
After the end of these phases, the $B$ output activations of 
layer $L_{i-1}$ will be fed as input activations to layer $L_i$.
Note that before the processing of the last of these phases,
\IN$(i,B-b)$ amount of output activation need to be buffered. 
Thus the total memory available for processing up to
layer $L_{i-1}$ gets reduced by \IN$(i, B-b)$ as this is required
for storing the activations before processing layer $L_i$.

We are now ready to present the recurrence relation.
For any $i > 1$, 
\begin{equation*}
\begin{split}
\OPT(i, &B, A)   =  \Time(i,B) \quad \quad  +   \\
 &\quad  \min \limits_{b \leq B} \left \{  (B/b) * \OPT(i-1,b,A+\IN(i,B-b)) \right \}\\
 &\quad \mbox{subject to} \quad \quad  \text{ $\langle i,B,A \rangle$ is feasible}
\end{split}
\end{equation*}



For the base case i.e., for $i$ = 1.
\begin{equation*}
  \OPT(1, B, A) =\begin{cases}
     \Time(1, B), & \text{if $\langle 1,B,A \rangle$ is feasible}.\\
    \infty, & \text{otherwise}.
  \end{cases}
\end{equation*} 





 
The maximum throughput for the inferencing is obtained 
by considering the configuration that yields the 
minimum inference time per input which is 
$$\min \limits_{ B} \frac{ \OPT(f, B, 0)}{B}. $$ 



The above dynamic program can be easily extended to ensure that the latency of inferencing is always less than some specified threshold. In the recurrence relation,
if \OPT$(i, B, A)$ exceeds the threshold value for some $i$, $B$, and $A$, we make   \OPT$(i, B, A)$ $\leftarrow$ $\infty$. This makes sure that our optimal solution 
never has larger latency.

\subsection{Additional Storage and Computation Overhead}
The table \OPT$(\cdot, \cdot, \cdot)$ needs to be evaluated for each entry in order to figure out the individual layer batch sizes that maximise the overall throughput.
We  now figure out the additional storage and computation that is needed for the dynamic programming space complexity for standard networks like AlexNet. 
The total number of layers in AlexNet  is 14. For requested input count of 64, we consider batch sizes 
in range 1 to 64 for the second dimension. For the case where the additional available memory is twice of the model size, the third dimension is considered from 0 to 14MB
in steps of {100KB}. Thus the total size of the table is around 500KB.
Each entry computation of the table computes the minimum over a set of possible batch sizes. Thus the computation complexity is at most $\cal B$ times the size of the table,
where $\cal B$ is the maximum number of distinct batch sizes considered.

We begin our  inferencing  with a pre-processing step, which computes the  individual layer batch sizes that maximise the overall throughput using the
above dynamic program.  The actual inferencing uses the batch sizes outputted from the dynamic program.












\section{Experimental Results with Batch Size}
\label{sec:expt2}
In this section, we validate the results of our dynamic programming algorithm on practical test cases with AlexNet model.
Suppose the user requests for inference of a set of $K$ images, and is interested to get the maximum throughput for the inference.
We consider the scenarios where the total  memory in the system (in addition to the model) is 1.5x, 2x and 2.5x  times the model size.
Our baseline is selecting a fixed batch size such that (i) running any layer of inferencing using that batch size does not violate the
memory constraints (ii) out of all possible batch sizes which satisfy	 (i),  the baseline returns the batch size with maximum throughput.
We compare this baseline from our dynamic programming output, which uses variable batch sizes for different layers.
We perform our experiments $K =  32, 64, 128$ and with all the four configurations of AlexNet model (conventional pruning and
70\%, 80\%, 90\% pruning).
Figure~\ref{fig:real_dp1} - ~\ref{fig:real_dp3} compares the results of our dynamic program algorithm with the baseline 
(fixed batch size) output for AlexNet with conventional pruning. The x-axis shows the additional memory available (w.r.t to the model size)
over the model, and the y-axis plots the total time to infer K images. Our results show that the dynamic programming approach
improves the throughput  by 15-25\% over the fixed batch size approach.

\begin{figure*}[!tbp]
  \centering
  \subfloat[K=32]{\includegraphics[height = 1.3in, width=2in]{figures/DP_real_32.pdf}\label{fig:real_dp1}}
  \hspace{3mm}
  \subfloat[K=64]{\includegraphics[height = 1.3in, width=2in]{figures/DP_real_64.pdf}\label{fig:real_dp2}}
  \hspace{3mm}
  \subfloat[K=128]{\includegraphics[height = 1.3in, width=2in]{figures/DP_real_128.pdf}\label{fig:real_dp3}}
  \caption{Fixed batch size (baseline) time vs Time outputted from Dynamic Programming for AlexNet with conventional pruning.}
\end{figure*}

 
   

\begin{table}[h!]
\centering
\begin{tabular}{|c|c|c|c|}
\hline
 Layer & 1.5x & 2x & 2.5x \\ \hline
 conv1 & 2 & 4 & 6    \\ \hline
 norm1 & 4 & 4 & 6    \\ \hline
 pool1 & 4 & 4 & 6    \\ \hline
 conv2 & 4 & 4 & 6    \\ \hline
 norm2 & 4 & 4 & 6    \\ \hline
 pool2 & 4 & 4& 6    \\ \hline
 conv3 & 4 & 4 & 6    \\ \hline
 conv4 & 4 & 4 & 6    \\ \hline
 conv5 & 4 & 4 & 6    \\ \hline
 pool5 & 4 & 4 & 32    \\ \hline
   fc6 & 64 & 64 & 60    \\ \hline
   fc7 & 64& 64 & 60    \\ \hline
   fc8 & 64& 64 & 60    \\ \hline
\end{tabular}
\caption{Variable batching for AlexNet.}
\label{tab:dp_path}
\end{table}


\begin{figure*}[!ht]
  \centering
  \subfloat[Pruning = 70\%]{\includegraphics[height = 1.3in, width=2in]{figures/DP_70_64.pdf}\label{fig:70_dp1}}
  \hspace{3mm}
  \subfloat[Pruning = 80\%]{\includegraphics[height = 1.3in, width=2in]{figures/DP_80_64.pdf}\label{fig:80_dp2}}
  \hspace{3mm}
  \subfloat[Pruning = 90\%]{\includegraphics[height = 1.3in, width=2in]{figures/DP_90_64.pdf}\label{fig:90_dp3}}
  \caption{Fixed batch size (baseline) time vs Time outputted from Dynamic Programming for AlexNet with 70\%, 80\% and 90\% pruning.}
\end{figure*}

Table~\ref{tab:dp_path} shows the dynamic programming output corresponding to the above run for K = 64.
 It is observed that the optimal inferencing scheme uses smaller batch sizes for the convolution layers (because of the larger memory overhead),
 and combines intermediate outputs to perform fully connected layer operations with larger batch sizes. This matches our intuition which motivated us to 
 develop the dynamic programming based solution. The dynamic programming solution  corresponding to column 2.5x picks 60 as the batchsize for final layers: thus for this case, we again compute the solution for requested input of 4, and report the total time for inferencing.  The baseline corresponding to these runs use fixed batch size of 3, 5 and 7 for  additional memory
 of  1.5x, 2x and 2.5x respectively.
 
 Figure~\ref{fig:70_dp1} - ~\ref{fig:90_dp3} extends our   results to the other configurations of the AlexNet model, namely, the
70\%, 80\% and 90\% pruned models. We show these results for fixed K of 64.  Our results show that the dynamic programming approach
performs well  over the fixed batch size approach for these scenarios as well.


\section{Concluding Remarks and Future work}
\label{sec:conc}
% \vspace{-0.5em}
\section{Conclusion}
% \vspace{-0.5em}
Recent advances in multimodal single-cell technology have enabled the simultaneous profiling of the transcriptome alongside other cellular modalities, leading to an increase in the availability of multimodal single-cell data. In this paper, we present \method{}, a multimodal transformer model for single-cell surface protein abundance from gene expression measurements. We combined the data with prior biological interaction knowledge from the STRING database into a richly connected heterogeneous graph and leveraged the transformer architectures to learn an accurate mapping between gene expression and surface protein abundance. Remarkably, \method{} achieves superior and more stable performance than other baselines on both 2021 and 2022 NeurIPS single-cell datasets.

\noindent\textbf{Future Work.}
% Our work is an extension of the model we implemented in the NeurIPS 2022 competition. 
Our framework of multimodal transformers with the cross-modality heterogeneous graph goes far beyond the specific downstream task of modality prediction, and there are lots of potentials to be further explored. Our graph contains three types of nodes. While the cell embeddings are used for predictions, the remaining protein embeddings and gene embeddings may be further interpreted for other tasks. The similarities between proteins may show data-specific protein-protein relationships, while the attention matrix of the gene transformer may help to identify marker genes of each cell type. Additionally, we may achieve gene interaction prediction using the attention mechanism.
% under adequate regulations. 
% We expect \method{} to be capable of much more than just modality prediction. Note that currently, we fuse information from different transformers with message-passing GNNs. 
To extend more on transformers, a potential next step is implementing cross-attention cross-modalities. Ideally, all three types of nodes, namely genes, proteins, and cells, would be jointly modeled using a large transformer that includes specific regulations for each modality. 

% insight of protein and gene embedding (diff task)

% all in one transformer

% \noindent\textbf{Limitations and future work}
% Despite the noticeable performance improvement by utilizing transformers with the cross-modality heterogeneous graph, there are still bottlenecks in the current settings. To begin with, we noticed that the performance variations of all methods are consistently higher in the ``CITE'' dataset compared to the ``GEX2ADT'' dataset. We hypothesized that the increased variability in ``CITE'' was due to both less number of training samples (43k vs. 66k cells) and a significantly more number of testing samples used (28k vs. 1k cells). One straightforward solution to alleviate the high variation issue is to include more training samples, which is not always possible given the training data availability. Nevertheless, publicly available single-cell datasets have been accumulated over the past decades and are still being collected on an ever-increasing scale. Taking advantage of these large-scale atlases is the key to a more stable and well-performing model, as some of the intra-cell variations could be common across different datasets. For example, reference-based methods are commonly used to identify the cell identity of a single cell, or cell-type compositions of a mixture of cells. (other examples for pretrained, e.g., scbert)


%\noindent\textbf{Future work.}
% Our work is an extension of the model we implemented in the NeurIPS 2022 competition. Now our framework of multimodal transformers with the cross-modality heterogeneous graph goes far beyond the specific downstream task of modality prediction, and there are lots of potentials to be further explored. Our graph contains three types of nodes. while the cell embeddings are used for predictions, the remaining protein embeddings and gene embeddings may be further interpreted for other tasks. The similarities between proteins may show data-specific protein-protein relationships, while the attention matrix of the gene transformer may help to identify marker genes of each cell type. Additionally, we may achieve gene interaction prediction using the attention mechanism under adequate regulations. We expect \method{} to be capable of much more than just modality prediction. Note that currently, we fuse information from different transformers with message-passing GNNs. To extend more on transformers, a potential next step is implementing cross-attention cross-modalities. Ideally, all three types of nodes, namely genes, proteins, and cells, would be jointly modeled using a large transformer that includes specific regulations for each modality. The self-attention within each modality would reconstruct the prior interaction network, while the cross-attention between modalities would be supervised by the data observations. Then, The attention matrix will provide insights into all the internal interactions and cross-relationships. With the linearized transformer, this idea would be both practical and versatile.

% \begin{acks}
% This research is supported by the National Science Foundation (NSF) and Johnson \& Johnson.
% \end{acks}
\bibliographystyle{plain}
\bibliography{ms}

\end{document}

