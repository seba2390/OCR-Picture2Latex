\documentclass[a4paper, amsfonts, amssymb, amsmath, reprint, showkeys, nofootinbib, twoside]{revtex4-1}
\usepackage[english]{babel}
\usepackage[utf8]{inputenc}
\usepackage[colorinlistoftodos, color=green!40, prependcaption]{todonotes}
\usepackage{mathtools}
\usepackage{amsmath}
\usepackage{amssymb}

% OLD PREAMBLE:

% \usepackage{jsen}
% \usepackage{cite}
% \usepackage{amsmath,amssymb,amsfonts, bbm, mathtools}
% \usepackage{algorithm,algorithmic}
% \usepackage{graphicx}
% \usepackage{textcomp}
% \usepackage{wrapfig}
% \usepackage{xfrac}
% \usepackage{stackengine}
% \usepackage{subfigure}
% \def\delequal{\mathrel{\ensurestackMath{\stackon[1pt]{=}{\scriptstyle\Delta}}}}



% \usepackage{color, soul}
% \newcommand{\hlt}[1]{\hl{#1}}
% \newcommand{\red}[1]{\textcolor{red}{#1}}

% \def\BibTeX{{\rm B\kern-.05em{\sc i\kern-.025em b}\kern-.08em
%     T\kern-.1667em\lower.7ex\hbox{E}\kern-.125emX}}
% \markboth{\journalname, VOL. XX, NO. XX, XXXX 2017}
% {Author \MakeLowercase{\textit{et al.}}: Preparation of Papers for IEEE TRANSACTIONS and JOURNALS (February 2017)}
% \definecolor{abstractbg}{rgb}{0.89804,0.94510,0.83137}
% \setlength{\fboxrule}{0pt}
% \setlength{\fboxsep}{0pt}

% NEW PREAMBLE:


\usepackage{amsmath,amsfonts,amssymb,bbm, amsthm, xfrac}
\usepackage{algorithmic}
\usepackage{algorithm}
\usepackage{array, multirow}
% \usepackage[caption=false,font=normalsize,labelfont=sf,textfont=sf]{subfig}
\usepackage{caption, subcaption}
\usepackage{textcomp}
\usepackage{stfloats}
\usepackage{url}
\usepackage{verbatim}
\usepackage{graphicx}
\usepackage{cite}
\usepackage{caption}
\usepackage{subcaption}
\hyphenation{}

\theoremstyle{plain}
\newtheorem{theorem}{Theorem}

\usepackage{color, soul}
\newcommand{\hlt}[1]{\hl{#1}}
\newcommand{\red}[1]{\textcolor{red}{#1}}

\usepackage[pdftex, pdftitle={Article}, pdfauthor={Author}]{hyperref} % For hyperlinks in the PDF
%\setlength{\marginparwidth}{2.5cm}
\bibliographystyle{apsrev4-1}
\begin{document}
\title{Ancilla-Assisted Process Tomography with Bipartite Mixed Separable States}

\author{Zhuoran Bao, Daniel F. V. James}
    \email[Correspondence email address: ]{zhuoran.bao@mail.utoronto.ca\\
    dfvj@physics.utoronto.ca}% Your name
    \affiliation{Dept. of Physics, University of Toronto, Toronto, M5S 1A7, Ontario, Canada}

\date{to be submitted to PRA} 

\begin{abstract}
It has been shown that the entanglement between the system state and the ancillary state is not a strict requirement for performing ancilla-assisted process tomography(AAPT). Instead, it only requires that the system-ancilla state to be faithful, which is equivalent to the invertibility of a certain matrix representing the state. However, it is difficult to distinguish between a faithful state that brings small error amplification and one that produces larger error amplification. Restricted to two-qubit system-ancilla states, we present a theoretical analysis to connect the invertibility problem to the concept of sinisterness, which classifies the correlation of two qubits. Using sinisterness, we provide a way of constructing all two qubits faithful mixed separable states with the smallest error amplification. We show that the maximally entangled states provided the smallest error amplification, while the separable Werner states produced an uneven error amplification larger than the maximally entangled state. Nevertheless, the error amplification due to inverting the separable Werner states or isotropic states is the best any mixed separable state can do.
\end{abstract}

\keywords{ancilla-assisted process tomography, bipartite mixed-separable states, sinisterness}


\maketitle

\input{sections/section01.tex}
\input{sections/section02.tex}
\input{sections/section03.tex}
\input{sections/section04.tex}

\begin{comment}
\begin{figure}
\includegraphics[width=\linewidth]{figs/beyond_tss_lesion.pdf}
\caption[]{End-to-End runtime lesion study of the entire MNIST dataset and the FMA featurized music dataset. Each of DROP's contributions provides a runtime improvement.}
\label{fig:beyond_lesion}
\end{figure}
\end{comment}



\section{Conclusion}
\label{sec:conclusion}

Advanced data analytics techniques must scale to rising data volumes. 
DR techniques offer a powerful toolkit when processing these datasets, with PCA frequently outperforming popular techniques in exchange for high computational cost. 
In response, we propose DROP, a new dimensionality reduction optimizer. 
DROP combines progressive sampling, progress estimation, and online aggregation to identify high quality low dimensional bases via PCA without processing the entire dataset by balancing the runtime of downstream tasks and achieved dimensionality. 
Thus, DROP provides a first step in bridging the gap between quality and efficiency in end-to-end DR for downstream \red{analytics}. 

%We revisit canonical operators for time series dimensionality reduction and the measurement study of~\cite{keogh-study}, and show that PCA is more effective than popular alternatives in the data mining literature often by a margin of over $2\times$ on average on gold-standard time series benchmark data sets with respect to output data dimension. More surprisingly, we empirically demonstrate that a small number of samples are sufficient to accurately characterize directions of maximum variance and obtain a high-quality low-dimensional transformation.



%\section{Acknowledgements}

Luca Herranz-Celotti was supported by the Natural Sciences and Engineering Research Council of Canada through the Discovery Grant from professor Jean Rouat, and by CHIST-ERA IGLU. We thank Compute Canada for the clusters used to perform the experiments and NVIDIA for the donation of two GPUs. We thank Wolfgang Maass for the opportunity to visit the Institute of Theoretical Computer Science, Guillaume Bellec, Darjan Salaj and Franz Scherr, for their invaluable insights on learning with surrogate gradients, and Maryam Hosseini, Ahmad El Ferdaoussi and Guillaume Bellec for their feedback on the article.

\begin{thebibliography}{}
\bibitem{Chuang}
I. L. Chuang, M. A. Nielsen, J. Mod. Opt. 44, 2455–2467 (1997).

\bibitem{Poyatos}
J. F. Poyatos, J. I. Cirac, and P. Zoller, Phys. Rev. Lett. 78, 390 (1997).

\bibitem{Pryde}
O'Brien, J. L. and Pryde, G. J. and Gilchrist, A. and James, D. F. V. and Langford, N. K. and Ralph, T. C. and White, A. G., “Quantum Process Tomography of a Controlled-NOT Gate,” Phys. Rev. Lett. 93(10), 080502 (2004).

\bibitem{Mohseni2006}
M. Mohseni and D. A. Lidar, Phys. Rev. Lett. 97, 170501
(2006)

\bibitem{Mohseni2007}
M. Mohseni and D. A. Lidar, Phys. Rev. A 75, 062331 (2007)

\bibitem{Mohseni2008}
M. Mohseni, A. T. Rezakhani, and D. A. Lidar, Phys. Rev. A, 77, 032322 (2008).

\bibitem{Altepeter}
J. B. Altepeter, D. Branning, E. Jeffrey, T. C. Wei, P. G. Kwiat, R. T. Thew, J. L. O’Brien, M. A. Nielsen, and A. G. White, Phys. Rev. Lett. 90, 193601 (2003).

\bibitem{Ariano}
G. M. D’Ariano and P. Lo Presti, Phys. Rev. Let. 86, 4195 (2001)

\bibitem{James}
D. F. V. James, Journal of the Optical Society of America A, Vol. 39, Issue 12, pp. C86-C97 (2022).

\bibitem{DiVincenzo}
D.P. DiVincenzo, B.M. Terhal, and A.V. Thapliyal, “Optimal decomposition of barely separable states,” J. Mod. Opt. 47, 377-385 (2000).

\bibitem{Wootters}
W. K. Wootters, “Entanglement of Formation of an Arbitrary State of Two Qubits,” Phys. Rev. Lett. 80(10),
2245-2248 (1998).

\bibitem{fang}
Y. Fang, K. A. Loparo and Xiangbo Feng, "Inequalities for the trace of matrix product," in IEEE Transactions on Automatic Control, vol. 39, no. 12, pp. 2489-2490, Dec. 1994, doi: 10.1109/9.362841.

\bibitem{sedrakyan}
Sedrakyan, H., Sedrakyan, N. (2018). The HM-GM-AM-QM Inequalities. In: Algebraic Inequalities. Problem Books in Mathematics. Springer, Cham. https://doi.org/10.1007/978-3-319-77836-5\_3
\end{thebibliography}

\end{document}
