\vspace{-0.1in}
\section{Towards Implementation}
\label{sec:impl}
\vspace{-0.1in}

The GSN-IA algorithm is implemented, together with slicers and model operators, as an extension of the MMINT~\cite{DiSandro:2015} model management framework (Fig.~\ref{fig:mmint}), called MMINT-A~\cite{Fung:2018}. In order to extend MMINT-A to support annotative product line models, and subsequently the lifted change impact assessment algorithm, the following modifications are required:
(1) Model elements need to be extended with presence conditions, with $\TT$ as a default value.   This way single product models (where all elements have the default $\TT$ presence condition) are directly supported as well.
(2) Operators on models need to be modified to take presence conditions into consideration, and compute the presence conditions of their outputs. Those modifications are mostly systematic along the lines of those of \lift{\restrict} and \lift{\trace} (Listing~\ref{lst:liftedHelpers}). 
(3) Higher-level algorithms (e.g., GSN-IA) need to be modified accordingly to use the lifted versions of the operators.
(4) The user interface of MMINT-A needs to support annotating different model elements with presence conditions.
(5) Optionally, MMINT-A can check the well-formedness of presence condition annotations. For example, the presence condition of an association between two UML classes has to be subsumed by the presence conditions of its two end points.
% i.e., the association can exist only where the two end points exist.

\begin{figure}[t]
	\includegraphics[width=\textwidth]{images/mmint}	
	\vspace{-0.2in}
	\caption{Architecture of the MMINT model management framework~\cite{Fung:2018}.}
	\label{fig:mmint}	
	\vspace{-0.2in}
\end{figure}