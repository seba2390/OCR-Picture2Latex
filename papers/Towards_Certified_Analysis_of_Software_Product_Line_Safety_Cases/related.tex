\vspace{-0.1in}
\section{Related Work}
\label{sec:related}
%This section compares our approach to related work.
\vspace{-0.1in}
%\vskip 0.1in
\noindent
{\bf Model-based approaches to safety case management.}  Many methods
for modeling safety cases have been proposed, including
goal models and requirements models~\cite{Ghanavati:2011,Brunel:2012} and GSN~\cite{Kelly:2004}.  The latter is
arguably the most widely used model-based
approach to improving the structure of safety arguments.  Building on GSN,
Habli et. al.~\cite{Habli:2010} examine how model-driven development can
provide a basis for the systematic generation of functional safety
requirements and demonstrates how an automotive safety case can be developed.  Gallina~\cite{Gallina:2014}
proposes a model-driven safety certification method to derive
arguments as goal structures given in GSN from process models. The
process is illustrated by generating arguments in the context of ISO
26262. We consider this category of work complimentary to ours; we do
not focus on safety case construction but instead
assume presence of a safety case
and focus on assessing the impact of system changes.

%\vspace{0.1in} 
%\noindent
%{\bf Safety case maintenance.}  Kelly~\cite{Kelly:2001}
%presents a tool-supported process, based on GSN, that facilitates a
%systematic safety case impact assessment.  The work by Li
%et. al.~\cite{Li:2016} proposes an assessment process to
%specify typical steps in the safety case assessment. The authors
%develop a graphical safety case editor for assessing GSN-based safety
%cases and use the Evidential Reasoning (ER) algorithm to
%assess the overall confidence in a safety case.
%%by using confidence
%%assignments to each of its
%%elements.
%Jaradat and Bate~\cite{Jaradat:2016} present two techniques that
%use safety contracts to facilitate maintenance of safety
%cases. As far as we are aware, none of the approaches provide a
%structured model-based algorithm for impact assessment.
%%% furthermore, do not consider the added efficiency approaches we
%%% propose in this paper.
%In the context of safety case maintenance, Bandur and
%McDermid~\cite{Bandur:2015} present a formalization of a
%logical subset of GSN with the aim of revealing the conditions which
%must be true in order to guarantee the internal consistency of
%a safety argument.  This provides a sound basis for understanding
%logical relationships between components of a safety case and thus
%to enhance impact assessment.

\vskip 0.1in
\noindent{\bf Lifting to Software Product Lines.} Different kinds of software analyses have been re-implemented to support product lines~\cite{Thum:2014}. 
For example, the TypeChef project implements variability aware parsers and type checkers for Java and C~\cite{Kastner:2012}. The SuperC project~\cite{Gazzillo:2012} is another C language variability-aware parser. A graph transformation engine was lifted to product lines of graphs~\cite{Salay:2014}. Datalog-based analyses (e.g., pointer analysis) have been lifted by modifying the Datalog engine being used~\cite{Shahin:2019}.
SPL\textsuperscript{Lift}~\cite{Bodden:2013} lifts data flow analyses to annotative product lines. Model checkers based on Featured Transition Systems~\cite{Classen:2013} check temporal properties of transition systems where transitions can be labeled by presence conditions. 
%Both of these SPL analyses  use almost the same single-product analyses on a lifted data representation. 
Syntactic transformation techniques have been suggested for lifting abstract interpretation analyses~\cite{Midtgaard:2015} and functional analyses~\cite{Shahin:2020} to SPLs.

In this paper, our methodology tailors the lifting approach from related work to safety cases of product lines, and we demonstrate it on change impact assessment. 
%In this paper we follow the techniques used in the above related work to lift a new analysis (Change Impact Assessment) to product lines. 
We tackle a new class of product line artifacts, particularly safety cases. To the best of our knowledge, this is the first attempt to lift a safety case analysis to product lines.
 
\vskip 0.1in
\noindent{\bf Formalized Systems and Interactive Theorem Proving.} Correctness and behavioral properties of several software systems have been formalized and verified using interactive theorem provers. The CompCert compiler~\cite{Leroy:2009} is an example of a C-language compiler fully certified using the Coq theorem prover. The seL4 microkernel~\cite{Klein:2009} was verified using the Isabelle\textbackslash HOL theorem prover. Isabelle was also used to formalize the Structured Assurance Case Metamodel (SACM) notation for certified definition of assurance cases~\cite{Nemouchi:2019}. 