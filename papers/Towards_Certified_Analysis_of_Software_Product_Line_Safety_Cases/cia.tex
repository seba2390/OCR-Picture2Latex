\vspace{-0.1in}
\section{Changed Impact Assessment}
\label{sec:cia}
\vspace{-0.1in}
\newcommand{\CIA}{GSN\_IA}

In this section, we formalize the GSN-IA~\cite{Kokaly:2017} impact assessment algorithm, systematically design a lifted version of it, and prove its correctness based on the methodology in Sec.~\ref{sec:methodology}. 
%Lean has been recently used in several projects formalizing mathematics~\cite{mathlib:2020,Lewis:2019}. 

%Safety engineering is an essential part of the development process for safety-critical systems. Ensuring the correctness of analyses applied to safety artifacts is thus in many cases essential if we are to make safety-critical decisions based on the results of those analyses. Given the constructive nature of assurance cases, where evidence constructively satisfies goals, a theorem prover based on constructive logic like Lean is better suited than theorem provers based on classical logic (e.g., Isabelle/HOL~\cite{Nipkow:2002}).
\vspace{-0.1in}
\subsection{Single-Product Algorithm}
\vspace{-0.1in}

\lstinputlisting[style=lean, basicstyle=\small, firstline=8,lastline=19,caption=Type definitions of the formalized GSN\_IA algorithm.,label=lst:defs]{cia.lean}

\newcommand{\Annotation}{\code{Annotation}}
\newcommand{\Reuse}{\code{Reuse}}
\newcommand{\Recheck}{\code{Recheck}}
\newcommand{\Revise}{\code{Revise}}
\newcommand{\SysEl}{\code{SysEl}}
\newcommand{\GSNEl}{\code{GSNEl}}
\newcommand{\Sys}{\code{Sys}}
\newcommand{\GSN}{\code{GSN}}
\newcommand{\TraceRel}{\code{TraceRel}}
\newcommand{\sliceSys}{\code{sliceSys}}
\newcommand{\sliceGSNV}{\code{sliceGSN\_V}}
\newcommand{\sliceGSNR}{\code{sliceGSN\_R}}
\newcommand{\Dlta}{\code{Delta}}
\newcommand{\restrict}{\code{restrict}}
\newcommand{\trace}{\code{trace}}
\newcommand{\createAnnotation}{\code{createAnnotation}}

The data types and external dependencies of the \CIA~algorithm are defined in Listing~\ref{lst:defs}. \Annotation~is the data type of annotations assigned to GSN model elements, with the values \Reuse, \Recheck, and \Revise~(lines 1-2). \SysEl~and \GSNEl~are opaque types of system model elements and GSN model elements respectively, where a system model \Sys~and a GSN model \GSN~ are sets of each of those elements types (lines 4-6). \TraceRel~is a traceability relation between system model elements and GSN model elements, so it is a defined as a set of ordered pairs of \SysEl~and \GSNEl~(line 7). \CIA~is parameterized by three model slicers: \sliceSys~is a system model slicer, while \sliceGSNV~and \sliceGSNR~ are GSN model slicers. 
%\todo{what's the difference?} 
Each of the slicers takes a model and a set of elements used as the slicing criterion, returning a subset slice of the input model (lines 9-11). \Dlta~is composed of three sets of system elements, representing the elements added, modified and deleted (lines 12).

\lstinputlisting[style=lean, basicstyle=\small, firstline=21,lastline=43,caption=Helper functions and the formalized GSN\_IA algorithm.,label=lst:gsnia]{cia.lean}

Listing~\ref{lst:gsnia} has the definitions of the \CIA~algorithm, together with three helper functions. \restrict~ is a function taking a traceability relation ~\code{t}~and a delta ~\code{es}~as inputs, and returns a restricted subset of \code{t} only covering elements in \code{es} (lines 1-2). \trace~takes a traceability relation \code{t}~and a set of system elements~\code{es}~as inputs, and returns the set of GSN elements mapped from \code{es}~by \code{t} (lines 4-5). \createAnnotation~assigns an \Annotation~value to each element in a GSN model, given sets of elements to be rechecked and revised (lines 7-12).

%\lstinputlisting[style=lean, basicstyle=\small, firstline=48,lastline=61,caption=Formalized GSN\_IA algorithm.,label=lst:gsnia]{cia.lean}

The change impact assessment algorithm \CIA~takes two system models \name{S}~and \name{S'} and the delta \name{D}~between them. It also takes a GSN model \name{A}~and a traceability relation \name{R}~between system model elements and GSN model elements. It returns a set of ordered pairs of GSN model elements and annotations. The algorithm starts by restricting the traceability relation based on \name{D}, slices the original system model \name{S}~using the elements deleted and modified as a slicing criterion, and slices the modified system model \name{S'} using the added and modified elements as the slicing criterion (lines 16-18). Using those two slices, the corresponding GSN model elements are traced using the traceability relation (line 19). The GSN elements traced from elements deleted from the original system model are to be revised (line 20). The slice of the GSN model based on the traced elements are to be rechecked (lines 21-22), and both revise and recheck sets are used to annotate the GSN model elements (line 23). 

\vspace{-0.1in}
\subsection{Lifted Algorithm}
\label{sec:lifting}
\vspace{-0.1in}

%To construct \lift{GSN\_IA}, we start with an overview of the lifted data structures used for algorithm inputs, outputs and intermediate values, then outline \lift{GSN\_IA}, and finally give two examples of lifted helper algorithms.

%\subsection{Lifted Algorithm}
%\label{sec:lifted_algo}

%\IncMargin{1em}
%\LinesNumbered
%\begin{algorithm}[t]
%	\SetKwData{S}{S}
%	\SetKwData{Sm}{S'}
%	\SetKwData{Sys}{\hlight{\lift{Sys}}}
%	\SetKwData{A}{A}
%	\SetKwData{GSN}{\hlight{\lift{GSN}}}
%	\SetKwData{R}{R}
%	\SetKwData{TR}{\hlight{\lift{TraceRel}}}
%	\SetKwData{Rm}{R'}
%	\SetKwData{D}{D}
%	\SetKwData{K}{K}
%	\SetKwData{Caz}{C0a}
%	\SetKwData{Cdz}{C0d}
%	\SetKwData{Cmz}{C0m}
%	\SetKwData{Cdmo}{C1dm}
%	\SetKwData{Camo}{C1am}
%	\SetKwData{Ct}{C2}
%	\SetKwData{Cr}{C3}
%	\SetKwData{Ctrecheck}{C2recheck}
%	\SetKwData{Ctrevise}{C2revise}
%	\SetKwData{Crrechecko}{C3recheck1}
%	\SetKwData{Crrecheckt}{C3recheck2}
%	
%	\SetKwFunction{Restrict}{\hlight{\lift{Restrict}}}
%	\SetKwFunction{Union}{Union}
%	\SetKwFunction{Trace}{\hlight{\lift{Trace}}}
%	\SetKwFunction{Slicesys}{\hlight{\lift{Slice\_Sys}}}
%	\SetKwFunction{Slicegsnv}{\hlight{\lift{Slice\_GSN\_V}}}
%	\SetKwFunction{Slicegsnr}{\hlight{\lift{Slice\_GSN\_R}}}
%	\SetKwFunction{CreateAnnotation}{\hlight{\lift{CreateAnnotation}}}
%	
%	\SetKwInOut{Input}{Inputs}\SetKwInOut{Output}{Output}
%	
%	\SetAlgoLined
%	\Input{\typerel{\S}{\Sys}    \tcp*[f]{Initial SPL megamodel} \newline
%		   \typerel{\Sm}{\Sys}   \tcp*[f]{Modified SPL megamodel} \newline
%	   	   \typerel{\A}{\GSN}   \tcp*[f]{GSN model}                \newline
%		   \typerel{\R}{\TR}  \tcp*[f]{Traceability map}         \newline
%		   \D =$\langle \Caz, \Cdz, \Cmz \rangle$                  \tcp*[f]{Delta}}
%	\Output{\K \tcp*[f]{GSN model Annotation}}
%	\BlankLine
%	
%	\assign{\Rm}{\Restrict{\R,\D}} \;
%	\assign{\Cdmo}{\Slicesys{\S, \Union{\Cdz, \Cmz}}} \;
%	\assign{\Camo}{\Slicesys{\Sm,\Union{\Caz, \Cmz}}} \;
%	\assign{\Ctrecheck}{\Union{\Trace{\R,\Cdmo}, \Trace{\Rm, \Camo}}} \;
%	\assign{\Ctrevise}{\Trace{\R,\Cdz}} \;
%	\assign{\Crrechecko}{\Slicegsnv{\A,\Ctrevise}} \;
%	\assign{\Crrecheckt}{\Slicegsnr{\A,\Union{\Ctrecheck, \Crrechecko}}} \;
%	\assign{\K}{\CreateAnnotation{\A, \Crrecheckt, \Ctrevise}} \;
%	\Return \K;
%	\BlankLine
%	
%	\caption{\lift{GSN\_IA}}
%	\label{alg:CIA_lifted}
%\end{algorithm}
%\DecMargin{1em}
%
%\begin{algorithm}[t]
%%\vspace{-0.3in}
%	\SetKwInOut{Input}{Inputs}\SetKwInOut{Output}{Output}
%	
%	\SetKwData{R}{R}
%	\SetKwData{Rn}{R'}
%	\SetKwData{TR}{\lift{TraceRel}}
%	\SetKwData{D}{D}
%	\SetKwData{Caz}{C0a}
%	\SetKwData{Cdz}{C0d}
%	\SetKwData{Cmz}{C0m}
%	\SetKwData{E}{e}
%	\SetKwData{PC}{pc}
%	\SetKwData{PCn}{pc'}
%	\SetKwData{EPC}{(e,pce)}
%	\SetKwData{EPCn}{(e,pc')}
%	\SetKwData{XYPC}{((x,y),pc)}
%	\SetKwData{XY}{(x,y)}
%	\SetKwData{X}{x}
%	
%	\SetKwFunction{Add}{add}
%	\SetKwFunction{Remove}{remove}
%	
%	\SetKwFunction{Find}{Find}
%	\SetKwFunction{Fn}{Function}
%	\SetKwProg{Fn}{Function}{ is}{end}
%	
%	\SetAlgoLined
%	\Input{\typerel{\R}{\TR}  \tcp*[f]{Traceability map}         \newline
%		\D =$\langle \Caz, \Cdz, \Cmz \rangle$                  \tcp*[f]{Delta}}
%	\Output{\Rn \tcp*[f]{restricted traceability relation}}
%	\BlankLine
%	%
%	%\Fn{\Add(\typerel{\R}{\TR}, \EPC)}{
%	%	\assign{\tPCn}{\Find{\R,\E}} \;
%	%	\If(\tcp*[f]{was e found in R?}){(\tPCn)}{
%	%		\lIf{\tPC == \tPCn}
%	%		{\Return \R}
%	%		\lElse{\Return \R $-$ \{\EPCn\} $\union$ \{(\E, \tPC $\vee$ \tPCn)\}} 
%	%	} 
%	%	\Else{
%	%		\Return \R $\union$ \{\EPCn\} \;
%	%	}
%	%}
%	%
%	\BlankLine
%	
%	\Fn{\Remove(\typerel{\R}{\TR}, \XYPC)}{
%		\assign{\tPCn}{\Find{\R,\XY}} \;
%		\If(\tcp*[f]{was (x,y) found in R?}){(\tPCn)}{
%			\lIf{\tPC == \tPCn}
%			{\Return \R $-$ \{\XYPC\}}
%			\lElse{\Return \R $-$ \{\XYPC\} $\union$ \{(\XY, $\neg$\tPC $\wedge$ \tPCn)\}} 
%		} 
%		\Else{
%			\Return \R \;
%		}
%	}
%	
%	\BlankLine
%	
%	\assign{\Rn}{\R} \;
%	%\For{$\EPC \in \Caz$}{
%	%	\assign{\Rn}{\Add{\Rn,\EPC}}
%	%}
%	\For{$\EPC \in \Cdz$}{
%		\For{$\XYPC \in \Rn$} {
%			\If{\E == \X} {
%				\assign{\Rn}{\Remove{\Rn, \XYPC}}
%			}
%		}
%	}
%	\Return \Rn \;
%	\caption{\lift{Restrict}}
%	\label{alg:restrict_lifted}
%	%\vspace{-0.15in}
%\end{algorithm}
%\vspace{-3in}

%%%%%%%%%%%%%%%%%%%%%%%%%%%%%%%
% Lifted algorithm
%%%%%%%%%%%%%%%%%%%%%%%%%%%%%%%

\lstinputlisting[style=lean,basicstyle=\small,firstline=102,lastline=116,caption=Lifted Change Impact Assessment algorithm.,label=lst:liftedCIA]{liftedCIA.lean}

Listing~\ref{lst:liftedCIA} is the variability-aware version of the algorithm in Listing~\ref{lst:gsnia}. Both algorithms are compositions of function/operator calls, so each of those functions/operators is replaced with its lifted counterpart. We assume that lifted versions of the three slicers are provided, and that they meet the correctness criteria of Fig.~\ref{fig:correctness1}. 
%The structure of the lifted algorithm follows that of the original one.

All the set types used in GSN\_IA need to be lifted. Definitions in lines 1-4 are lifted sets of system model elements, GSN model elements, and traceability mappings. A lifted delta (line 4) is composed of three lifted sets (additions, deletions and modifications).

The proof of the correctness theorem used auxiliary correctness lemmas for each of the helper algorithms. Each of the proofs expands definitions and repeatedly applies the correctness of lifted function composition (Fig.~\ref{fig:correctness2}). 
%{\url{https://www.github.com/ramyshahin/variability/}}.
%%%%%%%%%%%%%%%%%%%%%%%%
% Lifted helpers
%%%%%%%%%%%%%%%%%%%%%%%%

\vspace{0.1in}
\noindent
{\bf Lifted Helper Algorithms.}
%\subsection{Lifted Helper Algorithms}
\label{sec:lifted_helpers}
Since the lifted CIA algorithm operates on lifted data structures, all helper algorithms need to be modified to correctly operate on lifted data structures as well. In particular, we outline lifted versions of \restrict~and \trace~(Listing~\ref{lst:liftedHelpers}).

The original implementation of \restrict~takes a traceability map and a delta as inputs, and returns the minimal subset of the traceability map that covers all the elements in the delta. We now have presence conditions associated to system model elements, assurance case elements, and also the traceability links in between. The lifted version of \restrict~(referred to as \lift{\restrict}) needs to correctly process all those presence conditions.

The lifted algorithm starts by calculating the set of relevant elements in the system model, which is the union of added, deleted and modified elements in the delta (line 2). The algorithm returns a lifted traceability mapping as a function taking \code{((s,g),pc)}, where \code{(s,g)} is a system model element-GSN model element pair, and \code{pc} is a presence condition. This function evaluates to the conjunction of applying the input traceability map \code{t} to \code{((s,g),pc)}, and applying \code{relevant} to \code{(s,g)}. Recall that variability-aware sets (as well as Lean sets) are functions mapping values of a given type to propositions.

%Removing a traceability link from $\Rn$ needs to take presence conditions into consideration. When removing $((x,y),\tPC{})$ from $\Rn$, there are three possible cases:
%(1) if $(x,y)$ exists in $\Rn$ with the same presence condition $\tPC{}$, we just remove it (line 4).
%(2) if $(x,y)$ exists in $\Rn$ with a different presence condition $\tPC{}'$, then we only need to remove $(x,y)$ for the intersection of the set of products denoted by $\tPC{}$ and $\tPC{}'$ (line 5).
%(3) if $(x,y)$ does not exist in $\Rn$ at all, we do not remove anything (line 8).

Similarly, \lift{\trace} is the lifted version of \trace. The returned lifted set is a function mapping a GSN model element \code{g}~to the set of configurations from which there exists a system model element \code{s}~in the input lifted set of system elements, where \code{(s,g)} belongs to the input traceability map.

The lifted version of \createAnnotation~(named \lift{\createAnnotation}) is of exactly the same structure as the original because it strictly uses set operations (union, set difference and image), which have been all lifted as a part of the underlying variability-aware set implementation (Listing~\ref{lst:var}).
%The algorithm starts with an empty set of GSN elements (line 1), then iterates through all the elements of the traceability relation (lines 2-7). For each traceability link $(x,y)$ with presence condition $\tPC{}$, if $x$ exists in the system model parameter $\S$ with presence condition $\tPC{}'$ that isn't \ff, and if the conjunction of $\tPC{}$ and $\tPC{}'$ is satisfiable, we add $(y, \tPC{} \land \tPC{}')$ to $A$. Here the conjunction of the two presence conditions denotes the intersection of the sets of products denoted by the traceability link presence condition, and the GSN element presence condition.

\lstinputlisting[style=lean,basicstyle=\small,firstline=46,lastline=51,caption=Lifted implementation of \code{restrict} and \code{trace}.,label=lst:liftedHelpers]{liftedCIA.lean}
%\vspace{0.3in}

%\vspace{-0.4in}
%\begin{algorithm}[t]
%	\SetKwInOut{Input}{Inputs}\SetKwInOut{Output}{Output}
%	\SetKwData{S}{S}
%	\SetKwData{Sys}{\lift{Sys}}
%	\SetKwData{R}{R}
%	\SetKwData{TR}{\lift{TraceRel}}
%	\SetKwData{A}{A}
%	\SetKwData{GSN}{\lift{GSN}}
%	\SetKwData{X}{x}
%	\SetKwData{XPC}{xpc}
%	\SetKwData{Y}{y}
%	\SetKwData{YPC}{ypc}
%	\SetKwData{XP}{(\X,\XPC)}
%	\SetKwData{YP}{(\Y,\YPC)}
%	\SetKwData{PC}{pc}
%	\SetKwData{PCn}{pc'}
%	\SetKwData{E}{e}
%	
%	\SetKwFunction{Find}{Find}
%	
%	\SetAlgoLined
%	\Input{\typerel{\R}{\TR}  \tcp*[f]{Traceability map}         \newline
%		\typerel{\S}{\Sys}   \tcp*[f]{set of model elements with presence conditions}}
%	\Output{\typerel{\A}{\GSN}   \tcp*[f]{set of GSN elements with presence conditions}}
%	
%	\assign{\A}{\{\}}
%	
%	\For{$(\X,\Y,\tPC) \in \R$}{
%		\assign{\tPCn}{\Find{\S,\X}} \;
%		\If{(\tPCn)}{
%			\lIf{$\tPC \wedge \tPCn$}
%			{\assign{\A}{\A \cup \{(\Y, \tPC \wedge \tPCn)\}}}
%		}
%	}
%	
%	\Return{\A}
%	\caption{\lift{Trace}}
%	\label{alg:trace_lifted}
%	%\vspace{-0.15in}
%\end{algorithm}

\lstinputlisting[style=lean,basicstyle=\small,firstline=147,lastline=149,caption=Correctness theorem of \lift{GSN\_IA}.,label=lst:correctness]{liftedCIA.lean}

The correctness theorem of \lift{GSN\_IA} with respect to GSN\_IA is in Listing~\ref{lst:correctness}. It is a direct instantiation of the general correctness criteria in Fig.~\ref{fig:correctness1}, applied to inputs of the GSN\_IA algorithm.

\section{Case Study and Evaluation}
\label{casestudy}
\begin{table*}[tbp]
  \centering
  \begin{tabular}{r|ccc}
    version & \texttt{join} & \texttt{vjoin} & \texttt{udot}, \texttt{sortVector}, \texttt{roundVector}\\ \hline
    $<$ 0.15     & available  & undefined & undefined\\
    $\geq$ 0.16  & deleted & available & available\\
  \end{tabular}
  \caption{Availability of functions in hmatrix before and after tha update.}
  \label{table:join}
\end{table*}
\subsection{Case Study}
We demonstrate that \mylang{} programming achieves the two benefits of programming with versions. 
The case study simulated the incompatibility of hmatrix,\footnote{\url{https://github.com/haskell-numerics/hmatrix/blob/master/packages/base/CHANGELOG}} a popular Haskell library for numeric linear algebra and matrix computations, in the VL module \mn{Matrix}.
This simulation involved updating the applications \mn{Main} depending on \mn{Matrix} to reflect incompatible changes.

Table \ref{table:join} shows the changes introduced in version 0.16 of hmatrix. Before version 0.15, hmatrix provided a \texttt{join} function for concatenating multiple vectors.
The update from version 0.15 to 0.16 replaced \texttt{join} with \texttt{vjoin}. Moreover, several new functions were introduced.
We implement two versions of \mn{Matrix} to simulate backward incompatible changes in \mylang{}.
Also, due to the absence of user-defined types in \mylang{}, we represent \texttt{Vector a} and \texttt{Matrix a} as \texttt{List Int} and \texttt{List (List Int)} respectively, using \mn{List}, a partial port of \texttt{Data.List} from the Haskell standard library.

\begin{figure}[t]

\begin{minipage}{.5\textwidth}
% \begin{lstlisting}[style=haskell]
\begin{minted}{haskell}
module Main where
import Matrix
import List
main = let
  vec = [2, 1]
  sorted = sortVector vec
  m22 = join -- [[1,2],[2,1]]
          (singleton sorted)
          (singleton vec)
  in determinant m22
-- error: version inconsistent
\end{minted}
% \end{lstlisting}
\end{minipage}
\begin{minipage}{.5\textwidth}
\begin{minted}{haskell}
module Main where
import Matrix
import List
main = let
  vec = [2, 1]
  sorted = @\vlkey{unversion}@
             (sortVector vec)
  m22 = join -- [[1,2],[2,1]]
          (singleton sorted)
          (singleton vec)
  in determinant m22 -- ->* -3
\end{minted}
\end{minipage}
\caption{Snippets of \texttt{Main} before (left) and after (right) rewriting.}
\label{fig6-5}

\end{figure}
We implement \mn{Main} working with two conflicting versions of \mn{Matrix}. The left side of Figure \ref{fig6-5} shows a snippet of \mn{Main} in the process of updating \mn{Matrix} from version 0.15.0 to 0.16.0. \fn{main} uses functions from both versions of \mn{Matrix} together: \fn{join} and \fn{sortVector} are available only in version 0.15.0 and 0.16.0 respectively, hence \mn{Main} has conflicting dependencies on both versions of \mn{Matrix}. Therefore, it will be impossible to successfully build this program in existing languages unless the developer gives up using either \fn{join} or \fn{sortVector}.

\begin{itemize}
\item \textbf{Detecting Inconsistent Version}:
\mylang{} can accept \mn{Main} in two stages. First, the compiler flags a version inconsistency error.
It is unclear which \mn{Matrix} version the \fn{main} function depends on as \fn{join} requires version 0.15.0 while \fn{sortVector} requires version 0.16.0.
The error prevents using such incompatible version combinations, which are not allowed in a single expression.

\item \textbf{Simultaneous Use of Multiple Versions}:
In this case, using \fn{join} and \fn{sortVector} simultaneously is acceptable, as their return values are vectors and matrices. Therefore, we apply \texttt{\vlkey{unversion} t} for $t$ to collaborate with other versions.
The right side of Figure \ref{fig6-5} shows a rewritten snippet of \mn{Main}, where \texttt{sortVector vec} is replaced by \texttt{\vlkey{unversion} (sortVector vec)}. Assuming we avoid using programs that depend on a specific version elsewhere in the program, we can successfully compile and execute \fn{main}.
\end{itemize}

\subsection{Scalability of Constraint Resolution}
\begin{figure}[tbp]
    \centering
    \includegraphics[height=6.5cm]{figs/ret.png}
    \caption{Constraint resolution time for the duplicated \mn{List} by \texttt{\#mod} $\times$ \texttt{\#ver}.}
    \label{fig:consres}
\end{figure}

We conducted experiments on the constraint resolution time of the \mylang{} compiler. In the experiment, we duplicated a \mylang{} module, renaming it to \texttt{\#mod} like \mn{List\_i}, and imported each module sequentially. Every module had the same number of versions, denoted as \texttt{\#ver}. Each module version was implemented identically to \mn{List}, with top-level symbols distinguished by the module name, such as \fn{concat\_List\_i}. The experiments were performed ten times on a Ryzen 9 7950X running Ubuntu 22.04, with \texttt{\#mod} and \texttt{\#ver} ranging from 1 to 5.

Figure \ref{fig:consres} shows the average constraint resolution time. 
The data suggests that the resolution time increases polynomially (at least square) for both \texttt{\#mod} and \texttt{\#ver}.
Several issues in the current implementation contribute to this inefficiency:
First, we employ sbv as a z3 interface, generating numerous redundant variables in the SMT-Lib2 script.
For instance, in a code comprising 2600 LOC (with $\texttt{\#mod} =5$ and $\texttt{\#ver} =5$), the \mylang{} compiler produces 6090 version resource variables and the sbv library creates SMT-Lib2 scripts with approximately 210,000 intermediate symbolic variables.
Second, z3 solves versions for all AST nodes, whereas the compiler's main focus should be on external variables and the subterms of \texttt{\vlkey{unversion}}.
Third, the current \mylang{} nests the constraint network, combined with $\lor$, \texttt{\#mod} times at each bundling. This approach results in an overly complex constraint network for standard programs.
Hence, to accelerate constraint solving, we can develop a more efficient constraint compiler for SMT-Lib2 scripts, implement preprocess to reduce constraints, and employ a greedy constraint resolution for each module.






% \section{Limitations of the Current \mylang{}}
% % This section discusses the limitations of the current VL language and possible solutions.

% \subsection{Lack of Support for Structural Incompatibility}
% One of the apparent problems with the current VL system is that it does not support \emph{type incompatibilities}, a key element of structural incompatibilities. We will first analyze the types of incompatibilities and then discuss ways to extend the current VL system.

% % \paragraph{Types of Incompatibilities}
% % \label{sec:typesofincompatibility}
% % Incompatibilities between old and new versions of a package caused by updates can be broadly classified into two categories \emph{structural incompatibilities} and \emph{behavioral incompatibilities}.

% % \paragraph{Structural Incompatibilities}
% % \begin{table*}[tbp]
  \centering
  \begin{tabular}{r|ccc}
    version & \texttt{join} & \texttt{vjoin} & \texttt{udot}, \texttt{sortVector}, \texttt{roundVector}\\ \hline
    $<$ 0.15     & available  & undefined & undefined\\
    $\geq$ 0.16  & deleted & available & available\\
  \end{tabular}
  \caption{Availability of functions in hmatrix before and after tha update.}
  \label{table:join}
\end{table*}
% % A structural incompatibility occurs when multiple versions of a package provide different set of definitions including function names and data structures.
% % Structural incompatibilities are caused by adding and removing definitions, internal changes to data structures, and renaming.
% % Table \ref{table6-1} shows an example of structural incompatibility in GIMP Drawing Kit (GDK).
% % GDK is a C library for creating graphical user interfaces and is used by many projects, including GNOME.

% % If the deprecated functions are not available, version 3.22 is structurally incompatible with version 3.20 because the former lacks \mscreen{} that is available in the latter.
% % GDK versions before 3.22 provide \mscreen{} that tells the number of connected physical monitors.
% % However, versions 3.22 later provide the same functionality function \mdisplay{} and deprecate \mscreen{}.
% % When we upgrade GDK to version 3.22 and build software that uses this function without modifying anything, the build system will give us an undefined reference error.
% % With a static type check, the programmer will be informed of the incompatibility problem as a compilation error.

% % \paragraph{Behavioral Incompatibilities}
% % \input{./figs/table6-2.tex}
% % A behavioral incompatibility is a situation where multiple versions of a package provide the same definition but differ in their behavior.
% % Code changes may also cause behavioral incompatibilities that include additions, removals, and changes in side effects, even if there is no change in name or type.
% % Table \ref{table6-2} shows an example of behavioral incompatibility in the Android Platform API (henceforth Android API).
% % The Android API is the standard library written mostly in Java, and its version synchronizes with Android OS.

% % Before version 19\footnote{The Android API uses \textit{levels} instead of versions as identifiers for API revisions, but we will call them versions for consistency.},
% % the Android API provided the \mset{} method in the \texttt{AlarmManager} class that schedules an alarm at a specified time.
% % However, since version 19, the Android API has changed its behavior for power management.
% % Despite having the same name and type definitions, \mset{} no longer guarantees accurate alarm delivery.
% % For developers who require accurate delivery, the method \msetExact{} is provided instead.

% % \paragraph{Extending \mylang{} to Support Structural Incompatibility}
% The current VL language system forces terms of different versions to have the same type, both on the theoretical (typing rules in \corelang{}) and implementation (bundling in \vlmini{}) aspects.
% In \corelang{}, definitions of the same type can be combined as a versioned record (even if the programmer has given them different names), while terms with different types cannot be in a versioned record. Also, the VL language system will stop compilation if it finds a definition with the same name but a different type in more than one version of the same module.

% However, more feature is needed to deal with broader incompatibilities. Raemaekers et al. conducted a comprehensive analysis of the seven-year history of library releases in Maven Central. They found that about one-third of all releases introduced at least one structural incompatibility change. The top three causes of structural incompatibilities were class, method, or field deletions, and the remaining seven were type changes.~\cite{RAEMAEKERS2017140}
% It seems an important step to extend the language system to support a wider variety of type incompatibilities and to help programmers improve dependencies.

% The current \corelang{} design is motivated by the basic design that "the type of a versioned record is similar to the type \vertype{r}{A}, a type with a resource in coeffect calculli." In the current \corelang{}, the type of versioned record $\{\overline{l=t}\,|\,l_k\}$ is $\vertype{r}{A} (r = \{\overline{l}\})$, and no difference exists between a type of versioned records and promotions of a term of type $A$. This design has the advantage that versioned records and promotions could be treated in a unified manner, making it easier to formalize dynamic and static semantics.

% One useful idea to address this problem is to decouple version inference from the type inference of coeffect calculus and implement a type system that guarantees version consistency on top of the polymorphic record calculus.~\cite{10.1145/218570.218572} The idea stems from the fact that the type $\vertype{\{l_1,l_2\}}{A}$ is structurally similar to the variant type $\langle l_1 : A,\, l_2 : A\rangle$ of $\Lambda^{\forall,\bullet}$. It is no longer required with variants that types be the same, so terms with different types can be stored as a single value, such as $\langle l_1 = true,\, l_2 : 100\rangle : \langle l_1 : Bool,\, l_2 : Int\rangle$. Although the current version inference is uniformly defined with type inference, we believe it is possible to separate its algorithm and implement it in another calculus because the type and version inference in the type system of \vlmini{} is orthogonal to each other. In the current VL system, constraints generated from type inference and constraints generated from version inference are completely independent, and all constraints passed to z3 are version constraints.



% \subsection{Inadequate Version Polymorphism}
% As we attempt to scale VL programming to a realistically sized development, incomplete version polymorphism via duplication described in section \ref{sec:adhocversionpolymorphism} becomes an obstacle. The following examples are VL programs that depend on modules \texttt{A} and \texttt{B} in Figure \ref{fig:smallexample}. Both use functions \texttt{g} and \texttt{h} provided by module \texttt{B} and the variable \texttt{a} provided by module \texttt{A}.

% \input{figs/fig6-8.tex}
% The first problem is the difference between the treatment of local and external variables. The two programs in Figure \ref{fig6-8} illustrate this problem.  The only difference between the two programs is that the program on the left is written to apply functions without local variables, whereas the program on the right binds \texttt{a} to \texttt{a'}. However, the left one succeeds, while the right fails in version inference.

% The reason for this problem is the type inference system assigns the only resource variable to the local variable \texttt{a'}. The applications \texttt{g a'} and \texttt{h a'} generate constraints that require \texttt{a'} to depend on versions 1 and 2 of module \texttt{B}, respectively, but there is no version label that satisfies both. All external variables are given unique names by duplication, but local variables are not. Therefore, the type inference results differ in the two programs in Figure \ref{fig6-8}.

% \input{figs/fig6-9.tex}
% The second problem is that there is only one version on which each version of the top-level symbol can depend. The programs in Figure \ref{fig6-9} illustrate this problem.

% The top program requires \texttt{a} of \texttt{A} versions 2.0.0 and 1.0.0 as arguments of \texttt{g} and \texttt{h}, respectively, whereas the bottom program requires \texttt{A} version 2.0.0 for both arguments. The result of type inference is that the top program has a label that satisfies this requirement, while the bottom program does not.

% The cause of this problem is that the inference system produces a variable dependency on one of the versions of the original top-level symbol. The current VL type inference creates a variable dependency on either version of the source when creating a resource variable with the same constraints as the source of the duplication. In this example, the two copies of \texttt{a}, \texttt{a\_0} (for \texttt{g (version {A=2.0.0} of a))} and \texttt{a\_1} (for \texttt{h (version {A=2.0.0} of a))}, are expected to select either version of \texttt{a}. Furthermore, the generated constraints constrain the selected version of \texttt{a}. In line 7, \texttt{g} requires \texttt{a\_0} to have a dependency on version 1.0.0 of \texttt{B}, and version {A=2.0.0} of \texttt{a\_0} requires that \texttt{a\_0} is equal to the label selected for version 2.0.0 of a, resulting in version 2.0.0 of \texttt{a}. Similarly, line 8 generates a constraint that requires that the label for version 2.0.0 of \texttt{a} must contain version 1.0.0 of \texttt{B}, so no label satisfies both simultaneously.

% It is necessary to introduce full-resource polymorphism in the core calculus instead of duplication to solve this irrational problem,.
% The idea is to store external variables and constraints that behave in a version polymorphic manner in a top-level definition environment and instantiate them with a new resource variable for each symbol occurrence. This kind of resource polymorphism is similar to that already implemented in the Gr language~\cite{Orchard:2019:Granule}. However, unlike Gr, \vlmini{} provides a type inference algorithm that collects constraints on a per-module basis, so we need the well-defined form of the principal type.
% This extension is future work.