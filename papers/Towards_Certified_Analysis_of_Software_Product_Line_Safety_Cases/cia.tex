\vspace{-0.1in}
\section{Changed Impact Assessment}
\label{sec:cia}
\vspace{-0.1in}
\newcommand{\CIA}{GSN\_IA}

In this section, we formalize the GSN-IA~\cite{Kokaly:2017} impact assessment algorithm, systematically design a lifted version of it, and prove its correctness based on the methodology in Sec.~\ref{sec:methodology}. 
%Lean has been recently used in several projects formalizing mathematics~\cite{mathlib:2020,Lewis:2019}. 

%Safety engineering is an essential part of the development process for safety-critical systems. Ensuring the correctness of analyses applied to safety artifacts is thus in many cases essential if we are to make safety-critical decisions based on the results of those analyses. Given the constructive nature of assurance cases, where evidence constructively satisfies goals, a theorem prover based on constructive logic like Lean is better suited than theorem provers based on classical logic (e.g., Isabelle/HOL~\cite{Nipkow:2002}).
\vspace{-0.1in}
\subsection{Single-Product Algorithm}
\vspace{-0.1in}

\lstinputlisting[style=lean, basicstyle=\small, firstline=8,lastline=19,caption=Type definitions of the formalized GSN\_IA algorithm.,label=lst:defs]{cia.lean}

\newcommand{\Annotation}{\code{Annotation}}
\newcommand{\Reuse}{\code{Reuse}}
\newcommand{\Recheck}{\code{Recheck}}
\newcommand{\Revise}{\code{Revise}}
\newcommand{\SysEl}{\code{SysEl}}
\newcommand{\GSNEl}{\code{GSNEl}}
\newcommand{\Sys}{\code{Sys}}
\newcommand{\GSN}{\code{GSN}}
\newcommand{\TraceRel}{\code{TraceRel}}
\newcommand{\sliceSys}{\code{sliceSys}}
\newcommand{\sliceGSNV}{\code{sliceGSN\_V}}
\newcommand{\sliceGSNR}{\code{sliceGSN\_R}}
\newcommand{\Dlta}{\code{Delta}}
\newcommand{\restrict}{\code{restrict}}
\newcommand{\trace}{\code{trace}}
\newcommand{\createAnnotation}{\code{createAnnotation}}

The data types and external dependencies of the \CIA~algorithm are defined in Listing~\ref{lst:defs}. \Annotation~is the data type of annotations assigned to GSN model elements, with the values \Reuse, \Recheck, and \Revise~(lines 1-2). \SysEl~and \GSNEl~are opaque types of system model elements and GSN model elements respectively, where a system model \Sys~and a GSN model \GSN~ are sets of each of those elements types (lines 4-6). \TraceRel~is a traceability relation between system model elements and GSN model elements, so it is a defined as a set of ordered pairs of \SysEl~and \GSNEl~(line 7). \CIA~is parameterized by three model slicers: \sliceSys~is a system model slicer, while \sliceGSNV~and \sliceGSNR~ are GSN model slicers. 
%\todo{what's the difference?} 
Each of the slicers takes a model and a set of elements used as the slicing criterion, returning a subset slice of the input model (lines 9-11). \Dlta~is composed of three sets of system elements, representing the elements added, modified and deleted (lines 12).

\lstinputlisting[style=lean, basicstyle=\small, firstline=21,lastline=43,caption=Helper functions and the formalized GSN\_IA algorithm.,label=lst:gsnia]{cia.lean}

Listing~\ref{lst:gsnia} has the definitions of the \CIA~algorithm, together with three helper functions. \restrict~ is a function taking a traceability relation ~\code{t}~and a delta ~\code{es}~as inputs, and returns a restricted subset of \code{t} only covering elements in \code{es} (lines 1-2). \trace~takes a traceability relation \code{t}~and a set of system elements~\code{es}~as inputs, and returns the set of GSN elements mapped from \code{es}~by \code{t} (lines 4-5). \createAnnotation~assigns an \Annotation~value to each element in a GSN model, given sets of elements to be rechecked and revised (lines 7-12).

%\lstinputlisting[style=lean, basicstyle=\small, firstline=48,lastline=61,caption=Formalized GSN\_IA algorithm.,label=lst:gsnia]{cia.lean}

The change impact assessment algorithm \CIA~takes two system models \name{S}~and \name{S'} and the delta \name{D}~between them. It also takes a GSN model \name{A}~and a traceability relation \name{R}~between system model elements and GSN model elements. It returns a set of ordered pairs of GSN model elements and annotations. The algorithm starts by restricting the traceability relation based on \name{D}, slices the original system model \name{S}~using the elements deleted and modified as a slicing criterion, and slices the modified system model \name{S'} using the added and modified elements as the slicing criterion (lines 16-18). Using those two slices, the corresponding GSN model elements are traced using the traceability relation (line 19). The GSN elements traced from elements deleted from the original system model are to be revised (line 20). The slice of the GSN model based on the traced elements are to be rechecked (lines 21-22), and both revise and recheck sets are used to annotate the GSN model elements (line 23). 

\vspace{-0.1in}
\subsection{Lifted Algorithm}
\label{sec:lifting}
\vspace{-0.1in}

%To construct \lift{GSN\_IA}, we start with an overview of the lifted data structures used for algorithm inputs, outputs and intermediate values, then outline \lift{GSN\_IA}, and finally give two examples of lifted helper algorithms.

%\subsection{Lifted Algorithm}
%\label{sec:lifted_algo}

%\IncMargin{1em}
%\LinesNumbered
%\begin{algorithm}[t]
%	\SetKwData{S}{S}
%	\SetKwData{Sm}{S'}
%	\SetKwData{Sys}{\hlight{\lift{Sys}}}
%	\SetKwData{A}{A}
%	\SetKwData{GSN}{\hlight{\lift{GSN}}}
%	\SetKwData{R}{R}
%	\SetKwData{TR}{\hlight{\lift{TraceRel}}}
%	\SetKwData{Rm}{R'}
%	\SetKwData{D}{D}
%	\SetKwData{K}{K}
%	\SetKwData{Caz}{C0a}
%	\SetKwData{Cdz}{C0d}
%	\SetKwData{Cmz}{C0m}
%	\SetKwData{Cdmo}{C1dm}
%	\SetKwData{Camo}{C1am}
%	\SetKwData{Ct}{C2}
%	\SetKwData{Cr}{C3}
%	\SetKwData{Ctrecheck}{C2recheck}
%	\SetKwData{Ctrevise}{C2revise}
%	\SetKwData{Crrechecko}{C3recheck1}
%	\SetKwData{Crrecheckt}{C3recheck2}
%	
%	\SetKwFunction{Restrict}{\hlight{\lift{Restrict}}}
%	\SetKwFunction{Union}{Union}
%	\SetKwFunction{Trace}{\hlight{\lift{Trace}}}
%	\SetKwFunction{Slicesys}{\hlight{\lift{Slice\_Sys}}}
%	\SetKwFunction{Slicegsnv}{\hlight{\lift{Slice\_GSN\_V}}}
%	\SetKwFunction{Slicegsnr}{\hlight{\lift{Slice\_GSN\_R}}}
%	\SetKwFunction{CreateAnnotation}{\hlight{\lift{CreateAnnotation}}}
%	
%	\SetKwInOut{Input}{Inputs}\SetKwInOut{Output}{Output}
%	
%	\SetAlgoLined
%	\Input{\typerel{\S}{\Sys}    \tcp*[f]{Initial SPL megamodel} \newline
%		   \typerel{\Sm}{\Sys}   \tcp*[f]{Modified SPL megamodel} \newline
%	   	   \typerel{\A}{\GSN}   \tcp*[f]{GSN model}                \newline
%		   \typerel{\R}{\TR}  \tcp*[f]{Traceability map}         \newline
%		   \D =$\langle \Caz, \Cdz, \Cmz \rangle$                  \tcp*[f]{Delta}}
%	\Output{\K \tcp*[f]{GSN model Annotation}}
%	\BlankLine
%	
%	\assign{\Rm}{\Restrict{\R,\D}} \;
%	\assign{\Cdmo}{\Slicesys{\S, \Union{\Cdz, \Cmz}}} \;
%	\assign{\Camo}{\Slicesys{\Sm,\Union{\Caz, \Cmz}}} \;
%	\assign{\Ctrecheck}{\Union{\Trace{\R,\Cdmo}, \Trace{\Rm, \Camo}}} \;
%	\assign{\Ctrevise}{\Trace{\R,\Cdz}} \;
%	\assign{\Crrechecko}{\Slicegsnv{\A,\Ctrevise}} \;
%	\assign{\Crrecheckt}{\Slicegsnr{\A,\Union{\Ctrecheck, \Crrechecko}}} \;
%	\assign{\K}{\CreateAnnotation{\A, \Crrecheckt, \Ctrevise}} \;
%	\Return \K;
%	\BlankLine
%	
%	\caption{\lift{GSN\_IA}}
%	\label{alg:CIA_lifted}
%\end{algorithm}
%\DecMargin{1em}
%
%\begin{algorithm}[t]
%%\vspace{-0.3in}
%	\SetKwInOut{Input}{Inputs}\SetKwInOut{Output}{Output}
%	
%	\SetKwData{R}{R}
%	\SetKwData{Rn}{R'}
%	\SetKwData{TR}{\lift{TraceRel}}
%	\SetKwData{D}{D}
%	\SetKwData{Caz}{C0a}
%	\SetKwData{Cdz}{C0d}
%	\SetKwData{Cmz}{C0m}
%	\SetKwData{E}{e}
%	\SetKwData{PC}{pc}
%	\SetKwData{PCn}{pc'}
%	\SetKwData{EPC}{(e,pce)}
%	\SetKwData{EPCn}{(e,pc')}
%	\SetKwData{XYPC}{((x,y),pc)}
%	\SetKwData{XY}{(x,y)}
%	\SetKwData{X}{x}
%	
%	\SetKwFunction{Add}{add}
%	\SetKwFunction{Remove}{remove}
%	
%	\SetKwFunction{Find}{Find}
%	\SetKwFunction{Fn}{Function}
%	\SetKwProg{Fn}{Function}{ is}{end}
%	
%	\SetAlgoLined
%	\Input{\typerel{\R}{\TR}  \tcp*[f]{Traceability map}         \newline
%		\D =$\langle \Caz, \Cdz, \Cmz \rangle$                  \tcp*[f]{Delta}}
%	\Output{\Rn \tcp*[f]{restricted traceability relation}}
%	\BlankLine
%	%
%	%\Fn{\Add(\typerel{\R}{\TR}, \EPC)}{
%	%	\assign{\tPCn}{\Find{\R,\E}} \;
%	%	\If(\tcp*[f]{was e found in R?}){(\tPCn)}{
%	%		\lIf{\tPC == \tPCn}
%	%		{\Return \R}
%	%		\lElse{\Return \R $-$ \{\EPCn\} $\union$ \{(\E, \tPC $\vee$ \tPCn)\}} 
%	%	} 
%	%	\Else{
%	%		\Return \R $\union$ \{\EPCn\} \;
%	%	}
%	%}
%	%
%	\BlankLine
%	
%	\Fn{\Remove(\typerel{\R}{\TR}, \XYPC)}{
%		\assign{\tPCn}{\Find{\R,\XY}} \;
%		\If(\tcp*[f]{was (x,y) found in R?}){(\tPCn)}{
%			\lIf{\tPC == \tPCn}
%			{\Return \R $-$ \{\XYPC\}}
%			\lElse{\Return \R $-$ \{\XYPC\} $\union$ \{(\XY, $\neg$\tPC $\wedge$ \tPCn)\}} 
%		} 
%		\Else{
%			\Return \R \;
%		}
%	}
%	
%	\BlankLine
%	
%	\assign{\Rn}{\R} \;
%	%\For{$\EPC \in \Caz$}{
%	%	\assign{\Rn}{\Add{\Rn,\EPC}}
%	%}
%	\For{$\EPC \in \Cdz$}{
%		\For{$\XYPC \in \Rn$} {
%			\If{\E == \X} {
%				\assign{\Rn}{\Remove{\Rn, \XYPC}}
%			}
%		}
%	}
%	\Return \Rn \;
%	\caption{\lift{Restrict}}
%	\label{alg:restrict_lifted}
%	%\vspace{-0.15in}
%\end{algorithm}
%\vspace{-3in}

%%%%%%%%%%%%%%%%%%%%%%%%%%%%%%%
% Lifted algorithm
%%%%%%%%%%%%%%%%%%%%%%%%%%%%%%%

\lstinputlisting[style=lean,basicstyle=\small,firstline=102,lastline=116,caption=Lifted Change Impact Assessment algorithm.,label=lst:liftedCIA]{liftedCIA.lean}

Listing~\ref{lst:liftedCIA} is the variability-aware version of the algorithm in Listing~\ref{lst:gsnia}. Both algorithms are compositions of function/operator calls, so each of those functions/operators is replaced with its lifted counterpart. We assume that lifted versions of the three slicers are provided, and that they meet the correctness criteria of Fig.~\ref{fig:correctness1}. 
%The structure of the lifted algorithm follows that of the original one.

All the set types used in GSN\_IA need to be lifted. Definitions in lines 1-4 are lifted sets of system model elements, GSN model elements, and traceability mappings. A lifted delta (line 4) is composed of three lifted sets (additions, deletions and modifications).

The proof of the correctness theorem used auxiliary correctness lemmas for each of the helper algorithms. Each of the proofs expands definitions and repeatedly applies the correctness of lifted function composition (Fig.~\ref{fig:correctness2}). 
%{\url{https://www.github.com/ramyshahin/variability/}}.
%%%%%%%%%%%%%%%%%%%%%%%%
% Lifted helpers
%%%%%%%%%%%%%%%%%%%%%%%%

\vspace{0.1in}
\noindent
{\bf Lifted Helper Algorithms.}
%\subsection{Lifted Helper Algorithms}
\label{sec:lifted_helpers}
Since the lifted CIA algorithm operates on lifted data structures, all helper algorithms need to be modified to correctly operate on lifted data structures as well. In particular, we outline lifted versions of \restrict~and \trace~(Listing~\ref{lst:liftedHelpers}).

The original implementation of \restrict~takes a traceability map and a delta as inputs, and returns the minimal subset of the traceability map that covers all the elements in the delta. We now have presence conditions associated to system model elements, assurance case elements, and also the traceability links in between. The lifted version of \restrict~(referred to as \lift{\restrict}) needs to correctly process all those presence conditions.

The lifted algorithm starts by calculating the set of relevant elements in the system model, which is the union of added, deleted and modified elements in the delta (line 2). The algorithm returns a lifted traceability mapping as a function taking \code{((s,g),pc)}, where \code{(s,g)} is a system model element-GSN model element pair, and \code{pc} is a presence condition. This function evaluates to the conjunction of applying the input traceability map \code{t} to \code{((s,g),pc)}, and applying \code{relevant} to \code{(s,g)}. Recall that variability-aware sets (as well as Lean sets) are functions mapping values of a given type to propositions.

%Removing a traceability link from $\Rn$ needs to take presence conditions into consideration. When removing $((x,y),\tPC{})$ from $\Rn$, there are three possible cases:
%(1) if $(x,y)$ exists in $\Rn$ with the same presence condition $\tPC{}$, we just remove it (line 4).
%(2) if $(x,y)$ exists in $\Rn$ with a different presence condition $\tPC{}'$, then we only need to remove $(x,y)$ for the intersection of the set of products denoted by $\tPC{}$ and $\tPC{}'$ (line 5).
%(3) if $(x,y)$ does not exist in $\Rn$ at all, we do not remove anything (line 8).

Similarly, \lift{\trace} is the lifted version of \trace. The returned lifted set is a function mapping a GSN model element \code{g}~to the set of configurations from which there exists a system model element \code{s}~in the input lifted set of system elements, where \code{(s,g)} belongs to the input traceability map.

The lifted version of \createAnnotation~(named \lift{\createAnnotation}) is of exactly the same structure as the original because it strictly uses set operations (union, set difference and image), which have been all lifted as a part of the underlying variability-aware set implementation (Listing~\ref{lst:var}).
%The algorithm starts with an empty set of GSN elements (line 1), then iterates through all the elements of the traceability relation (lines 2-7). For each traceability link $(x,y)$ with presence condition $\tPC{}$, if $x$ exists in the system model parameter $\S$ with presence condition $\tPC{}'$ that isn't \ff, and if the conjunction of $\tPC{}$ and $\tPC{}'$ is satisfiable, we add $(y, \tPC{} \land \tPC{}')$ to $A$. Here the conjunction of the two presence conditions denotes the intersection of the sets of products denoted by the traceability link presence condition, and the GSN element presence condition.

\lstinputlisting[style=lean,basicstyle=\small,firstline=46,lastline=51,caption=Lifted implementation of \code{restrict} and \code{trace}.,label=lst:liftedHelpers]{liftedCIA.lean}
%\vspace{0.3in}

%\vspace{-0.4in}
%\begin{algorithm}[t]
%	\SetKwInOut{Input}{Inputs}\SetKwInOut{Output}{Output}
%	\SetKwData{S}{S}
%	\SetKwData{Sys}{\lift{Sys}}
%	\SetKwData{R}{R}
%	\SetKwData{TR}{\lift{TraceRel}}
%	\SetKwData{A}{A}
%	\SetKwData{GSN}{\lift{GSN}}
%	\SetKwData{X}{x}
%	\SetKwData{XPC}{xpc}
%	\SetKwData{Y}{y}
%	\SetKwData{YPC}{ypc}
%	\SetKwData{XP}{(\X,\XPC)}
%	\SetKwData{YP}{(\Y,\YPC)}
%	\SetKwData{PC}{pc}
%	\SetKwData{PCn}{pc'}
%	\SetKwData{E}{e}
%	
%	\SetKwFunction{Find}{Find}
%	
%	\SetAlgoLined
%	\Input{\typerel{\R}{\TR}  \tcp*[f]{Traceability map}         \newline
%		\typerel{\S}{\Sys}   \tcp*[f]{set of model elements with presence conditions}}
%	\Output{\typerel{\A}{\GSN}   \tcp*[f]{set of GSN elements with presence conditions}}
%	
%	\assign{\A}{\{\}}
%	
%	\For{$(\X,\Y,\tPC) \in \R$}{
%		\assign{\tPCn}{\Find{\S,\X}} \;
%		\If{(\tPCn)}{
%			\lIf{$\tPC \wedge \tPCn$}
%			{\assign{\A}{\A \cup \{(\Y, \tPC \wedge \tPCn)\}}}
%		}
%	}
%	
%	\Return{\A}
%	\caption{\lift{Trace}}
%	\label{alg:trace_lifted}
%	%\vspace{-0.15in}
%\end{algorithm}

\lstinputlisting[style=lean,basicstyle=\small,firstline=147,lastline=149,caption=Correctness theorem of \lift{GSN\_IA}.,label=lst:correctness]{liftedCIA.lean}

The correctness theorem of \lift{GSN\_IA} with respect to GSN\_IA is in Listing~\ref{lst:correctness}. It is a direct instantiation of the general correctness criteria in Fig.~\ref{fig:correctness1}, applied to inputs of the GSN\_IA algorithm.

\section{Case Study: Fair Classification Using Logistic Regression} \label{casestudy}
In this section, we instantiate our approach for achieving fairness, in the context of logistic regression. We denote the trained parameters of our logistic regressor by $\theta \in \mathbb{R}^d$, and the log-likelihood of $\theta$ given training set $S$ by $ll(\theta;S)$. 

%In logistic regression, we fit the parameters $\theta\in\mathbb{R}^d$ of a model $h_{\theta}:\mathbb{R}^d\rightarrow[0,1]$, s.t. $h_{\theta}(x)=\dfrac{1}{1+e^{-\theta^{T}x}}$. Binary prediction is done using a cut-off parameter. Given a set of probabilistic assumptions, we can fit the parameters of logistic regression by maximizing the log-likelihood function of $\theta$. In what follows, we use 
%\begin{equation*}
%\Pr[y=1~|~x;\theta]=h_{\theta}(x),~ \Pr[y=0~|~x;\theta]=1-h_{\theta}(x)
%\end{equation*}

%The log-likelihood of $\theta$ in this case is
%\begin{equation*}
%ll(\theta;S)=\sum\limits_{i=1}^{n} y_{i}\log(h_{\theta}(x_i))+(1-y_{i})\log(1-h_{\theta}(x_i))
%\end{equation*}

We wish to solve the following optimization problem:


\begin{align}\label{eq:1}
\begin{split}
\underset{\theta}{\text{minimize}}
 ~~&L_{S}^{0\text{-}1}(\theta)\\
&+d_1|FPR_{A=0}(\theta;S)-FPR_{A=1}(\theta;S)|\\
&+d_2|FNR_{A=0}(\theta;S)-FNR_{A=1}(\theta;S)|
\end{split}
\end{align}
where $d_1, d_2 \geq 0$ are to be set up front, according to the desired trade-off between accuracy, FPR matching, and FNR matching. Applying our suggested relaxation ($R_{FP}$, $R_{FN}$ are to be set as either Absolute Value Difference or Squared Difference penalizers), and adding a standard $\ell_2$ regularization term, we get the following convex optimization problem:
\begin{align} \label{eq:2}
\begin{split}
\underset{\theta}{\text{minimize}}
~~&-ll(\theta;S) \\
&+c_1 R_{FP}(\theta;S) \\
&+c_2 R_{FN}(\theta;S)\\
&+q\left|\left|\theta\right|\right|_2^2 \\
\end{split}
\end{align}

For convenience, we will denote the objective in (\ref{eq:1}) by $\text{Objective}(\theta;S,d_1,d_2)$, and the objective in the proxy problem (\ref{eq:2}) by $\text{Proxy}(\theta;S,c_1,c_2,q)$. As the proxy is easy to solve using standard methods, we use it when optimizing, and then shift back to the original problem for estimating the quality of our results.

%Effectively, our (relaxed) penalizers are set under the assumption that the distance between the prediction and the true label serves as a reliable proxy for the 0-1 loss, i.e., when predictions are done with high confidence, close to 0 or 1, far from the decision boundary. The terms are minimized when the average distance is the same for both groups in $S^{pos}$ and for both groups in $S^{neg}$. 

%In order to maximize $ll(\theta;S)$, and at the same time seek fair solutions, we use the following gradient update rule:
%\begin{equation*}
%   \theta^{i+1} = \theta^{i} - \eta_t(\nabla_{\theta}(-ll)+C_1\nabla_{\theta}(R_{FPR})+C_2\nabla_{\theta}(R_{FNR})+C_3 \theta)
%\end{equation*}
%Where $\eta_t$  is the learning rate (gradient step size).