\documentclass[runningheads]{llncs}

\usepackage{amsmath}
\usepackage{oz}
\usepackage{graphicx}
\usepackage[english]{babel}
\usepackage[T1]{fontenc}
\usepackage{tikz-cd}
\usepackage{caption}
\usepackage{framed}
\usepackage{subcaption}
\captionsetup{compatibility=false}

\usepackage[utf8x]{inputenc}
\usepackage{amssymb, upgreek}

\usepackage{color}

\usepackage{rotating}

% Lean preamble
\definecolor{keywordcolor}{rgb}{0.7, 0.1, 0.1}   % red
\definecolor{commentcolor}{rgb}{0.4, 0.4, 0.4}   % grey
\definecolor{symbolcolor}{rgb}{0.0, 0.1, 0.6}    % blue
\definecolor{sortcolor}{rgb}{0.1, 0.5, 0.1}      % green
\definecolor{errorcolor}{rgb}{1, 0, 0}           % bright red
\definecolor{stringcolor}{rgb}{0.5, 0.3, 0.2}    % brown

\usepackage{listings}
\usepackage{chngcntr}
\def\lstlanguagefiles{lstlean.tex}
\lstset{
	language=lean,
	abovecaptionskip=\smallskipamount,
	belowcaptionskip=\smallskipamount,
	belowskip=-2em,
	basicstyle=\small
	}
\lstdefinestyle{lean}{
	float=t,
	%floatplacement=tp,
	%aboveskip=3pt,
	%belowskip=-3pt,
	numbers=left
}

%\captionsetup[lstlisting]{numberbychapter=false}

\newcommand{\figref}[1]{Fig.~\ref{#1}}
\newcommand{\tblref}[1]{Table~\ref{#1}}
\newcommand{\secref}[1]{Section~\ref{#1}}
\renewcommand{\eqref}[1]{Equation~(\ref{#1})}

\def\availableat{\url{url-published-on-acceptance}}

\newcommand{\todo}[1]{{\color{red} TODO: {#1}}}
\newcommand{\newstuff}[1]{{\color{red} CHECK: {#1}}}
%\newcommand{\todo}[1]{{}}

\newcommand{\ckp}[2]{$CK_{#1}P_{#2}$}
\newcommand{\cext}{\ckp{8}{16}$ext$}
\newcommand{\cfin}{$F_{CK_{X}P_{Y}}$}
\newcommand{\cray}{\ckp{8}{8}$ray$}
\newcommand{\csin}{$SK_{8}P_{8}$}
\newcommand{\casin}{$SK_{combined}$}
%\renewcommand{\cext}{$SK_{8}K_{8}P_{8}$}
\newcommand{\ckpnl}[2]{$CK_{#1}P_{#2}nl$}


\makeatletter
\newcommand{\Spvek}[2][r]{%
	\gdef\@VORNE{1}
	\left(\hskip-\arraycolsep%
	\begin{array}{#1}\vekSp@lten{#2}\end{array}%
	\hskip-\arraycolsep\right)}

\def\vekSp@lten#1{\xvekSp@lten#1;vekL@stLine;}
\def\vekL@stLine{vekL@stLine}
\def\xvekSp@lten#1;{\def\temp{#1}%
	\ifx\temp\vekL@stLine
	\else
	\ifnum\@VORNE=1\gdef\@VORNE{0}
	\else\@arraycr\fi%
	#1%
	\expandafter\xvekSp@lten
	\fi}
\makeatother
%% Software Product Line (SPL) macros
\DeclareMathOperator{\Prop}{Prop}
\DeclareMathOperator{\Conf}{Conf}

\newcommand{\pc}[1]     {pc_{#1}}
\newcommand{\SPL}       {\mathcal{L}}
\newcommand{\feat}[1]{\bm{#1}} %for features
\newcommand{\featset}{F}
\newcommand{\featmodel}{\Phi}
\newcommand{\domainmodel}{D}
\newcommand{\pcmap}{\phi}
\newcommand{\config}{\rho}
\newcommand{\PC}[1]{\texttt{#1}}
\newcommand{\indexProd}[2]{#1|_#2}
\begin{document}
\counterwithout{lstlisting}{chapter}

\title{Towards Certified Analysis of Software Product Line Safety Cases}
\author{Ramy Shahin\inst{1} \and
	Sahar Kokaly\inst{2} \and
	Marsha Chechik\inst{1}}
\institute{University of Toronto, Toronto, Canada\\
	\email{\{rshahin,chechik\}@cs.toronto.edu} \and
	General Motors, Canada\\
	\email{sahar.kokaly@gm.com}}
\maketitle

\vspace{-0.2in}
\begin{abstract}
	Safety-critical software systems are in many cases designed and implemented as families of products, usually referred to as Software Product Lines (SPLs). Products within an SPL vary from each other in terms of which features they include. Applying existing analysis techniques to SPLs and their safety cases is usually challenging because of the potentially exponential number of products with respect to the number of supported features.
	In this paper, we present a methodology and infrastructure for certified \emph{lifting} of existing single-product safety analyses to product lines. To ensure certified safety of our infrastructure, we implement it in an interactive theorem prover, including formal definitions, lemmas, correctness criteria theorems, and proofs.
	%Modifying a software system might impact its safety case in many ways.  algorithms identify how local changes to specific system elements affect safety case elements, given traceability links between the two. Those algorithms can only work on the artifacts of a single software product at a time. However, in many cases a family of software products is developed together in the form of a Software Product Line (SPL), with artifacts potentially shared between different product variants. In this paper, we present lifting of a CIA algorithm to Software Product Lines aimed at supporting the input of feature-specific modifications, and potentially returning feature-specific safety case annotations. 
	
	We apply this infrastructure to formalize and lift a Change Impact Assessment (CIA) algorithm. We present a formal definition of the lifted algorithm, outline its correctness proof (with the full machine-checked proof available online), and discuss its implementation within a model management framework. 
\keywords{Safety cases, Product lines, Lean, Certified analysis.}
\end{abstract}
\vspace{-0.2in}

Reinforcement learning has achieved great success in areas such as Game-playing \citep{silver2018general,vinyals2019grandmaster}, robotics \cite{kober2013reinforcement}, large language models \citep{ouyang2022training}, etc.
However, due to safety concerns or physical limitations, in some real-world reinforcement learning problems, we must consider additional constraints that may influence the optimal policy and the learning process \citep{garcia2015comprehensive}.
% For example, a robotic arm must not take actions that may cause harm to itself or the environments.
A standard framework to handle such cases is the constrained Markov Decision Process (CMDP) \citep{altman1999constrained}.
Within the CMDP framework, the agent has to maximize
the expected cumulative reward while
obeying a finite number of constraints, which are usually in the form of expected cumulative cost criteria.

However, we are sometimes concerned with the problem with a continuum of constraints.
For example,
the constraints we meet might be time-evolving or subject to uncertain parameters, which
cannot be formulated as an ordinary CMDP
(see Examples \ref{Example_Time_Evolving} and  \ref{Example_Uncertain}).
In this paper we would study a generalized CMDP  
to address the above problem.  Because the constraints are not only infinite-number but also lie
in a continuous set,
the generalization is not trivial. Fortunately, we find that we can borrow the idea behind semi-infinite programming (SIP) \citep{remez1934determination, hettich1993semi} to deal with the semi-infinite constraints.
Accordingly, we propose \emph{semi-infinitely constrained Markov decision processes} (SICMDPs)
as a novel complement to the ordinary CMDP framework.
%More specifically,  an SICMDP model %, we consider 
%contains a continuum of constraints whereas an ordinary CMDP contains a finite number of constraints. 

%This generalization is natural but not trivial. However, we can brows the idea  
%The idea is quite natural and can be backtracked
%to the practice of extending linear programming to linear semi-infinite programming (LSIP) %\cite{remez1934determination, GobernaLSIO1998}.
%In addition, 
%As a complementary approach to the ordinary CMDP framework, 
%SICMDP can be used to model these problems  which cannot be described by a finite number of constraints
%that are not covered by .
%For example,
%the restrictions we consider can be time-evolving or subject to uncertain parameters
%, thus
%cannot be described by a finite number of constraints but a continuum of constraints 
%(see Examples \ref{Example_Time_Evolving} and  \ref{Example_Uncertain}).

We also present two reinforcement learning algorithms to solve SICMDPs called SI-CRL and SI-CPO, respectively.
SI-CRL is a model-based reinforcement learning algorithm designed for tabular cases, and SI-CPO is a policy optimization algorithm for non-tabular cases.
% and analyze its performance both theoretically and empirically.
The main challenge is that we need to deal with a continuum of constraints, thus reinforcement learning algorithms for ordinary CMDPs do not work anymore.
In SI-CRL, we tackle this difficulty by first transforming the reinforcement learning problem to an equivalent LSIP problem, which can then be solved using methods in the LSIP literature like the dual exchange methods \citep{Hu1990,reemtsen1998numerical}.
In SI-CPO, we resort to the idea of cooperative stochastic approximation developed in \cite{lan2020algorithms, wei2020comirror}.
As far as we know, we are the first to introduce tools from semi-infinitely programming (SIP) into the reinforcement learning community for solving constrained reinforcement learning problems.

% To the best of our knowledge, we are the first to apply tools from semi-infinitely programming (SIP) to solve reinforcement learning problems.
Furthermore, we give theoretical analysis for both SI-CRL and SI-CPO.
We decompose the error of SI-CRL into two parts: the statistical error from approximating the true SICMDP with an offline dataset and the optimization error due to the fact that the solution of the LSIP problem obtained by the dual exchange method is inexact.
On the optimization side, we show that the iteration complexity of SI-CRL is $O\left(\left\{\mathrm{diam}(Y)L\sqrt{|\gS|^2|\gA|m}/\left[(1-\gamma)\epsilon\right]\right\}^m\right)$.
On the statistical side, we show that the sample complexity of SI-CRL is $\widetilde O\left(\frac{|S|^2|A|^2}{\epsilon^2(1-\gamma)^3}\right)$ if the offline dataset is generated by a generative model, and $\widetilde O\left(\frac{|S||A|}{\nu_{\min} \epsilon^2(1-\gamma)^3}\right)$ if the dataset is generated by a probability measure $\nu$ as considered in \cite{chen2019information}.
Here $\widetilde O$ means that all logarithm terms are discarded.
For SI-CPO, things become a little more complicated because other than the statistical error and the optimization error, we also need to consider the function approximation error, which comes from imperfect policy parametrizations.
It is shown if the function approximation error can be controlled to $O(\epsilon)$ order, the iteration complexity of SI-CPO is $\widetilde{O}\left(\frac{1}{\epsilon^2(1-\gamma)^6}\right)$ and the sample complexity of SI-CPO is $\widetilde{O}(\frac{1}{\epsilon^4(1-\gamma)^{10}})$.
Here our iteration complexity bound is equivalent to a typical $\widetilde O(1/\sqrt{T})$ global convergence rate.

We perform a set of numerical experiments to illustrate the SICMDP model and validate our proposed algorithms.
Specifically, we examine two numerical examples, namely the discharge of sewage and ship route planning.
Through the discharge of sewage example, we show the advantage of the SICMDP framework over the CMDP baseline obtained by naive discretization in modeling realistic sequential decision-making problems.
Moreover, we demonstrate the effectiveness of the SI-CRL and SI-CPO algorithms in such tabular environments. 
In the ship route planning example, we illustrate the benefits of the SICMDP framework and the ability of the SI-CPO algorithm to address complex continuous control tasks involving continuous state spaces with modern deep reinforcement learning techniques.

% In summary, our contributions are listed as follows.
% First, we present the SICMDP model, which can be viewed as a generalization of the ordinary CMDP model.
% Second, we propose an algorithm to perform reinforcement learning for SICMDPs, which is called SI-CRL, and we believe that we are the first to apply tools from SIP
% to solve reinforcement learning problems.
% Third, we give a theoretical analysis of SI-CRL and identify both its sample complexity and iteration complexity.
% In addition, we perform numerical experiments to illustrate the SICMDP model and validate the SI-CRL algorithm.
% \{This paragraph can be removed!!! \}






% Panoptic segmentation

% 3D segmentation

% Multi-object tracking

% Online 3D panoptic:

% PanopticFusion: (IROS 2019)
% https://arxiv.org/pdf/1903.01177.pdf
%
% - most similar to ours
% - PSPNet + M-RCNN + 2D fusion
% - volumetric mapping, 
% - greedy matching with IoU -> optimal only with 0.5 threshold
% - voxel & class weighting
% - CRF regularisation
%
% - good:
%
% - bad:
%  - CRF post-processing step
%  - greedy data-association
%    - can't be tuned for lower overlap ratios -> has to have high framerate, large changes in viewpoint could break this
%    - IoU: sensitive to 2D labels projecting over object borders (CRF and voxel weighting seem to alleviate this)

% Voxblox++: (Robotics & automation letters 2019)
% https://arxiv.org/pdf/1903.00268.pdf
% https://github.com/ethz-asl/voxblox-plusplus
%
% - M-RCNN + geometric segmentation + fusion 
% - data association of geometric segments with 3D overlap (no. points inside volume), fixed threshold for min number of points
% - instance label is assigned to a segment based on highest overlap
% - only one detected segment per reference label, as in PanopticFusion and Ours
% - TSDF Integration 
%
% good: 
% - because of geometric segmentation objects with no associated semantic class can also be segmented
% bad:
% - two different object segment types -> confusing, overly complicated ?
% - quite inaccurate (fixed below)

% Reconstructing Interactive 3D Scenes by Panoptic Mapping and CAD Model Alignments (ICRA 2021)
% https://arxiv.org/pdf/2103.16095.pdf
% https://github.com/hmz-15/Interactive-Scene-Reconstruction
%
% - based heavily on Voxblox++, much more accurate
% - Scene-graph ("contact graph") for mapping object relations
% - Search & replace voxels with CAD models, with geometrical and physical constraints
% - Object 6D pose
% - Format for robot interaction
%
% - Segmentation: bilateral fusion of geomatric and semantic segments -> reduce segmentation noise compared to Voxblox++
% - Fusion: triplet count improves consistency over Voxblox++ pairwise count strategy (take semantic label into account in addition to instance and geometry)
% - Fusion: instance labels are also combined if there is enough overlap with common geometric label for long enough time
%   - this means multiple detections can match the same reference unlike ours, voxblox++ and PanopticFusion ?
%

% Panoptic-MOPE: (ROBOTICS AND AUTOMATION LETTERS 2020)
% https://ieeexplore.ieee.org/stamp/stamp.jsp?tp=&arnumber=8977356
% https://github.com/hoangcuongbk80/Object-RPE/tree/panoptic-mope
%
% - novel RGB-D semantic segmentation model + M-RCNN
% - camera tracking based on "addaptively weighted optimization of geometric, appearance, and semantic cues"
% - surfel map: 
%   - how does it scale ? authors satate they tested on room-sized environments, but could be applied in larger scale as well ...
%     - could maybe be applied as VO in a SLAM algorithm ...
%   - demo only on a small pallet + surroundings, might not be applicable in large-scale SLAM

% US VS THEM:
%
% - based heavily on PanopticFusion, with modifications:
%   - instead of greedy data-association (which seems to be the case in others as well), we solve LAP (JPDA?)
%     - overlap threshold can be tuned, which renders the algorithm more flexible
%     - could be extended to dynamic tracking ?
%   - multiple options for association likelihood
%   - outlier rejection (either clustering or probabilistic)
%   - test different options for decreasing processing time
%   - no post-processing
%
% - model-agnostic:
%   - completely separated from segmentation
%   - does not care how point clouds are obtained -> applicable for LIDAR segmentation (e.g. EfficientLPS) as well
%
% - also agnostic to localisation method
%   - could, however, be utilised to find landmark locations / poses

% More compact version of this paragraph to introduction to save space?
%Panoptic segmentation -- proposed in \cite{panoptic_segmentation} -- aims to solve the unified task of semantic- and instance segmentation. Semantic classes are separated to \textit{stuff} -- amorphous, unquantifiable regions like sky, road or floor -- and \textit{things} -- quantifiable objects. The distinction between the two can vary depending on the application, but a semantic class can only belong to one or another. The article also proposes a unified panoptic evaluation metric, coined \textbf{Panoptic Quality} (PQ). Many 2D approaches to panoptic segmentation -- \textit{e.g.} \cite{panopticfpn,seamless,panoptic_deeplab,efficientps} -- have since been proposed. Deep neural networks for performing semantic- or instance segmentation directly on the 3D reconstruction -- \textit{e.g.} on \cite{scannet,s3dis,paris_lille_3d} -- have also been proposed, but since they require the reconstructed 3D scene, they are mostly offline approaches and therefore out of scope for this work. Some recent works also apply panoptic segmentation to point clouds -- \textit{e.g.} methods in the SemanticKITTI panoptic segmentation competition \cite{semantic_kitti} -- mostly aimed at segmenting LiDAR output. They are suitable for online processing, but similar to RGB-D images require a method for tracking object instances persistent in both time and space. In fact, our proposed method, as well as some others mentioned in this work, could use segmented LiDAR point clouds as an input similarly to RGB-D images.

PanopticFusion \cite{panopticfusion} is the first work to propose online integration of panoptic image segmentations to a 3D reconstruction. They integrate point clouds generated from segmented images to a TSDF voxel volume \cite{tsdf,voxblox} by greedily matching detected segments with the reconstruction and regulating each voxel's corresponding instance with a weighting function. Semantic labels are inferred in a bayesian manner based on confidence scores provided by the segmentation model. They also apply a Conditional Random Field (CRF) to regularise the reconstruction, improving results significantly. Voxblox++ \cite{voxblox++} -- introduced later the same year -- is a similar approach that also integrates segmented RGB-D images into a TSDF volume. It leverages geometric segmentation of depth images to improve instance segmentation accuracy. Both geometric and semantic segments are used to compute a pair-wise weight, which is used to greedily match them with segments in the reconstruction. Because of the geometric segmentation, the method allows segmentation of objects with no known semantic class in addition to objects recognised by the instance segmentation model. 

Recently, \cite{interactive_3d_scenes} built upon the idea of Voxblox++. They apply Voxblox++ for 3D instance integration, with two small but effective modifications: the pair-wise weight is replaced by a triplet weight that also takes semantic labels into account in the fusion, and -- in addition to geometric segments -- instance segments are fused if they overlap by a significant amount. The article introduces a method for searching and aligning CAD models to reconstructed objects based on geometry and semantic class, as well as geometrical and physical rules. With the CAD models, a contact graph and interactive virtual scene are reconstructed to allow a robot to simulate its interaction with the environment. SceneGraphFusion \cite{scenegraphfusion} is another approach that forms a scene graph online from a stream of RGB-D images, but unlike the above-mentioned approach, it generates the graph with a deep neural network, after which the panoptic labels for geometrically segmented portions of the 3D reconstruction are produced a side product.

Panoptic-MOPE \cite{panoptic_mope} is another recent approach, which integrates sequences of RGB-D images into a surfel reconstruction. Unlike other mentioned approaches -- which assume the camera pose either known or estimated elsewhere -- it also tracks camera movements based on geometric-, appearance- and semantic cues. The method also applies a novel RGB-D panoptic segmentation model. Although it is only tested on room-sized environments, the authors claim it could be scaled to larger environments as well.

\section{The Proposed Method}
\label{sec:Method}
% 开头阐述模型图
% 如图所示,模型的训练分为两阶段,大数据集预训练以及小数据集finetune。在预训练阶段,posterior encoder 以 linear spectrogram作为输入,输出隐变量Z,隐变量Z送入Phoneme predictor。Phoneme predictor 输出Phoneme probability,与Phoneme Look Up Table 相乘得到Phoneme embedding。该embeddding作为歌声合成输入的一部分。
% 2.1 阐述预训练框架
%     标注精良的单歌手歌声数据集例如Opencpop,往往很难有较大的规模,因为标注十分耗费人力。并且由于单歌手的音域固定,所训练出的歌声合成模型很难拥有较广的音域,从而丧失了和真人演唱相比,可能具有的音域优势。为了能够利用大的歌声合成数据集以提高歌声合成系统的音域表现,受xxx文章的启发,我们基于proposed方法的框架,采用了melody-unsupervision在大的歌声合成数据集上做预训练,但又与该文章有着不同。由于我们预训练阶段使用的歌声合成数据集没有详细的标注,仅仅只有text、phonemes,但并没有wav在时间上的详细标注,因此我们希望能够借助ASR的训练方式,通过Phoneme Predictor预测出每一个frame的音素注意力向量 。该音素注意力向量与Phonemes lookup table相乘,最终得到了frame-level的Phonemes embedding。此外,我们从wav中提取出连续的音高,并且将其量化为乐谱音高,经过embedding layer后得到frame-level的pitch embedding。同时由于Opensinger是一个多歌手的数据集,我们使用了基于ECAPA-TDNN的speaker encoder去建模不同歌手的音色信息。
% 2.1.1 Phonemes predictor
% 由于数据集没有Phonemes 的time alignments信息,因此我们采用ASR的训练方法,在pronunciation 层面上使用CTC Loss对Phonemes predictor进行训练。具体而言,Phonemes predictor包含两层FFT Blocks ,一层线性层。线性层将hidden channels 映射到Phonemes 总数对应的类别数。对于线性层的输出,我们取softmax后得到每一个phonemes的概率 p,作为注意力向量与phonemes lookup table相乘得到phonemes embedding。同时我们对p取log,与ground truth标签计算CTC Loss。值得注意的是,与xxx工作不同,由于预训练时没有duration信息,我们的phonemes predictor输出的注意力向量是frame-level的。
% 2.1.2 阐述fine-tune架构
%     finetune架构的搭建整体基于proposed 方法,在proposed 方法的基础上,提出了 d-durator 和bi-flow,提升了我们模型的性能。
% 具体来说,在finetune阶段,模型读取预训练模型的checkpoints,并且在OpenCpop上进行finetune。由于事先在大规模数据集进行了预训练,模型的可合成音域得到了提升。
% 2.2 阐述可微分的上采样模块
% 以前的歌声合成模型,大多采用简单的复制操作将phoneme-level信息转变为frame-level信息,导致模型存在韵律问题。受xxx工作启发,我们提出了可学习的时长预测器,包含一个时长预测器以及可学习的上采样层。时长预测器输出每一个phonemes占发声总时长的比例,该比例与note duration相乘,送入可微分的上采样层。该上采样层 takes a phoneme hidden sequence as input, and outputs a sequence of prior distribution at the frame level。 Compared to simply repeating each phoneme hidden sequence with the predicted duration in a hard way, the differentiable upsampling layer enables more flexible duration adjustment for each phoneme. Also, the differentiable upsampling layer makes the phoneme to frame expansion differentiable, and thus can be jointly optimized with other modules in the TTS system.
% 2.3 阐述双向flow层
% 原先的工作中,flow层在训练阶段将复杂的后验分布映射为简单的先验分布,并且在推理阶段将简单的先验分布转换为复杂的后验分布。但这里存在着training和inference时候的mismatch,也就是 train in backward direction but infer in forward direction。因此我们在训练的时候提出了一个双向的flow模型,对posterior encoder预测的后验分布以及prior encoder预测的先验分布进行转换,并计算loss。值得一提的是,我们发现将flow model两边的kl loss
The training stage of the proposed model consists of two steps: the multi-singer pre-training step and the single-singer fine-tuning step. The architecture of the proposed model is illustrated in Fig.\ref{fig: architecture}, which consists of a prior encoder, a posterior encoder, and a decoder together with a discriminator.
The proposed model is designed from our previous work \cite{zhou22f_interspeech} with the following modifications.
The posterior encoder utilizes a phoneme predictor to predict frame-level phoneme probabilities in the pre-training step. 
The prior encoder adds a speaker encoder to model the timbre variations, replaces the length regulator with a differentiable duration regulator to improve the rhythm naturalness, and upgrades the flow module to be bi-directional to improve the sound quality.

% As illustrated in Fig.\ref{fig: architecture}, the training of the proposed model consists of two stages: the pre-training stage and the fine-tuning stage. 

% In the pre-training stage, the posterior encoder takes the linear spectrogram as input and predicts the latent representation $z$. 
% The phoneme predictor estimates the frame-level phoneme probability $p$ given the latent representation $z$. 
% We multiply $p$ with the phoneme lookup table to get the phoneme embedding. The pitch is extracted from the waveform and we quantified it into note pitch. 

% we reload the checkpoint of the pre-train model and resume training in the OpenCpop\cite{wang2022opencpop} datasets. 


\subsection{The Melody-Unsupervised Multi-Singer Pre-Training Step}
Since the multi-singer training data has no phonemic timing information, in the pre-training step, this work utilizes the automatic speech recognition (ASR) training strategy to train a phoneme predictor in the posterior encoder and predict the frame-level phoneme probabilities $p$.
The probability vectors are multiplied with the phoneme look-up table to obtain the frame-level phoneme embeddings.
In addition, the continuous pitch $f_{0}$ is estimated from the audio and quantized into the note pitch.
The note pitch is passed through the embedding layer to obtain frame-level pitch embeddings.
Moreover, we apply a speaker encoder to extract frame-level speaker embeddings to model the timbre variations of different singers.
Since the pre-training step directly deals with estimated pitch values and focuses on enhancing the vocal range, the pitch predictor, the energy predictor and the duration-related modules are dropped during the pre-training step.


\subsubsection{Phoneme predictor}
We train the phoneme predictor using the connectionist temporal classification (CTC) \cite{graves2006connectionist} loss. 
It contains two layers of FFT blocks and one linear layer.
The linear layer maps the hidden channels to the number of phoneme categories. 
We obtain the probability vector $p$ after taking the softmax operation on the linear layer's output, then multiply it with the phoneme look-up table to get frame-level phoneme embeddings.
Meanwhile, we take the log function of $p$ to compute the CTC loss with the ground truth phoneme sequences. 

\subsubsection{Speaker encoder}
This work adopts one of the state-of-the-art speaker recognition models, i.e. ECAPA-TDNN \cite{desplanques2020ecapa}, as the speaker encoder. Its advanced network architecture and attentive statistics pooling layer have shown great effectiveness in both speaker recognition \cite{desplanques2020ecapa} and voice conversion \cite{guo2022improving,li2022hierarchical}. The speaker encoder is configured as the one with 512 channels in Table 1 of \cite{desplanques2020ecapa}, and it extracts 192-dimensional frame-level speaker embeddings from the audio's Mel-Spectrograms. These embeddings are given as the speaker condition in the multi-singer pre-training step. 

\subsection{The Single-Singer Fine-Tuning Step}
In the fine-tuning step, it loads the pre-trained model parameters, then utilizes the single-singer Opencpop dataset to fine-tune model parameters.
It uses the phoneme and note-pitch annotations provided by the dataset to derive the phoneme and pitch embeddings, instead of using the phoneme probability vectors and quantized f0 values in the pre-training step.
Note that the phoneme and note-pitch annotations are at the phoneme level, rather than the frame level, such that a duration regulator after the note encoder is necessary to up-sample the embedding vectors into the frame level.
As for the speaker embedding, we use the pre-trained speaker encoder to extract an averaged speaker embedding over the Opencpop dataset, then utilize it as a fixed speaker condition during the fine-tuning step.
Moreover, the energy predictor and the pitch predictor join the fine-tuning process to enhance the expressiveness and pitch accurateness of the synthesized samples, following our previous work\cite{zhou22f_interspeech}.
% The fine-tuning architecture is built based on the proposed method \cite{zhou22f_interspeech}, and on top of the proposed method, a differentiable duration predictor and bi-directional flow model are proposed to improve the performance of the synthesized singing voice.
% Specifically, in the fine-tuning phase, the model reloads the checkpoints of the pre-trained model and performs inference on OpenCpop.
The synthesizable vocal range of the model is enhanced due to the multi-singer pre-training on a large-scale dataset.

\subsection{Differentiable Duration Regulator}
Most previous SVS systems simply replicate each phoneme hidden representation with the predicted duration in a hard way, which may degrade the rhythm naturalness.
Inspired by \cite{tan2022naturalspeech}, we leverage a differentiable duration regulator, which contains a duration predictor and a differentiable up-sampling layer. 
The duration predictor outputs the ratio of each phoneme to the corresponding note duration, then the ratio is multiplied by the note duration and fed to the differentiable up-sampling layer.
The differentiable up-sampling layer leverages the predicted duration to learn a projection matrix to extend the phoneme hidden sequence from the phoneme level to the frame level.
It makes the phoneme-to-frame expansion differentiable and thus can be jointly optimized with other modules in the system.

\subsection{Bi-directional Flow}
In the previous work \cite{zhou22f_interspeech}, the flow model maps the complex posterior distribution to the simple prior distribution in the training stage while operating reversely in the inference stage.
This process suffers from the mismatch problem between the training and inference stages.
Therefore, we leverage a bi-directional flow module\cite{tan2022naturalspeech} during training, which bridges the complex posterior distribution and the simple prior distribution bi-directionally to alleviate the mismatch issue in the inference stage.
% not only maps the complex posterior distribution to the simple prior distribution during training, but also maps the simple prior distribution to the complex posterior distribution, in order to improve the quality of the synthesized singing voice.
It is worth noting that we observed that the system can easily fail to train and encounter gradient explosion when the KL losses on both sides of the flow contribute equally, so we define the reverse KL loss weight as 0.5.





\vspace{-0.1in}
\section{Changed Impact Assessment}
\label{sec:cia}
\vspace{-0.1in}
\newcommand{\CIA}{GSN\_IA}

In this section, we formalize the GSN-IA~\cite{Kokaly:2017} impact assessment algorithm, systematically design a lifted version of it, and prove its correctness based on the methodology in Sec.~\ref{sec:methodology}. 
%Lean has been recently used in several projects formalizing mathematics~\cite{mathlib:2020,Lewis:2019}. 

%Safety engineering is an essential part of the development process for safety-critical systems. Ensuring the correctness of analyses applied to safety artifacts is thus in many cases essential if we are to make safety-critical decisions based on the results of those analyses. Given the constructive nature of assurance cases, where evidence constructively satisfies goals, a theorem prover based on constructive logic like Lean is better suited than theorem provers based on classical logic (e.g., Isabelle/HOL~\cite{Nipkow:2002}).
\vspace{-0.1in}
\subsection{Single-Product Algorithm}
\vspace{-0.1in}

\lstinputlisting[style=lean, basicstyle=\small, firstline=8,lastline=19,caption=Type definitions of the formalized GSN\_IA algorithm.,label=lst:defs]{cia.lean}

\newcommand{\Annotation}{\code{Annotation}}
\newcommand{\Reuse}{\code{Reuse}}
\newcommand{\Recheck}{\code{Recheck}}
\newcommand{\Revise}{\code{Revise}}
\newcommand{\SysEl}{\code{SysEl}}
\newcommand{\GSNEl}{\code{GSNEl}}
\newcommand{\Sys}{\code{Sys}}
\newcommand{\GSN}{\code{GSN}}
\newcommand{\TraceRel}{\code{TraceRel}}
\newcommand{\sliceSys}{\code{sliceSys}}
\newcommand{\sliceGSNV}{\code{sliceGSN\_V}}
\newcommand{\sliceGSNR}{\code{sliceGSN\_R}}
\newcommand{\Dlta}{\code{Delta}}
\newcommand{\restrict}{\code{restrict}}
\newcommand{\trace}{\code{trace}}
\newcommand{\createAnnotation}{\code{createAnnotation}}

The data types and external dependencies of the \CIA~algorithm are defined in Listing~\ref{lst:defs}. \Annotation~is the data type of annotations assigned to GSN model elements, with the values \Reuse, \Recheck, and \Revise~(lines 1-2). \SysEl~and \GSNEl~are opaque types of system model elements and GSN model elements respectively, where a system model \Sys~and a GSN model \GSN~ are sets of each of those elements types (lines 4-6). \TraceRel~is a traceability relation between system model elements and GSN model elements, so it is a defined as a set of ordered pairs of \SysEl~and \GSNEl~(line 7). \CIA~is parameterized by three model slicers: \sliceSys~is a system model slicer, while \sliceGSNV~and \sliceGSNR~ are GSN model slicers. 
%\todo{what's the difference?} 
Each of the slicers takes a model and a set of elements used as the slicing criterion, returning a subset slice of the input model (lines 9-11). \Dlta~is composed of three sets of system elements, representing the elements added, modified and deleted (lines 12).

\lstinputlisting[style=lean, basicstyle=\small, firstline=21,lastline=43,caption=Helper functions and the formalized GSN\_IA algorithm.,label=lst:gsnia]{cia.lean}

Listing~\ref{lst:gsnia} has the definitions of the \CIA~algorithm, together with three helper functions. \restrict~ is a function taking a traceability relation ~\code{t}~and a delta ~\code{es}~as inputs, and returns a restricted subset of \code{t} only covering elements in \code{es} (lines 1-2). \trace~takes a traceability relation \code{t}~and a set of system elements~\code{es}~as inputs, and returns the set of GSN elements mapped from \code{es}~by \code{t} (lines 4-5). \createAnnotation~assigns an \Annotation~value to each element in a GSN model, given sets of elements to be rechecked and revised (lines 7-12).

%\lstinputlisting[style=lean, basicstyle=\small, firstline=48,lastline=61,caption=Formalized GSN\_IA algorithm.,label=lst:gsnia]{cia.lean}

The change impact assessment algorithm \CIA~takes two system models \name{S}~and \name{S'} and the delta \name{D}~between them. It also takes a GSN model \name{A}~and a traceability relation \name{R}~between system model elements and GSN model elements. It returns a set of ordered pairs of GSN model elements and annotations. The algorithm starts by restricting the traceability relation based on \name{D}, slices the original system model \name{S}~using the elements deleted and modified as a slicing criterion, and slices the modified system model \name{S'} using the added and modified elements as the slicing criterion (lines 16-18). Using those two slices, the corresponding GSN model elements are traced using the traceability relation (line 19). The GSN elements traced from elements deleted from the original system model are to be revised (line 20). The slice of the GSN model based on the traced elements are to be rechecked (lines 21-22), and both revise and recheck sets are used to annotate the GSN model elements (line 23). 

\vspace{-0.1in}
\subsection{Lifted Algorithm}
\label{sec:lifting}
\vspace{-0.1in}

%To construct \lift{GSN\_IA}, we start with an overview of the lifted data structures used for algorithm inputs, outputs and intermediate values, then outline \lift{GSN\_IA}, and finally give two examples of lifted helper algorithms.

%\subsection{Lifted Algorithm}
%\label{sec:lifted_algo}

%\IncMargin{1em}
%\LinesNumbered
%\begin{algorithm}[t]
%	\SetKwData{S}{S}
%	\SetKwData{Sm}{S'}
%	\SetKwData{Sys}{\hlight{\lift{Sys}}}
%	\SetKwData{A}{A}
%	\SetKwData{GSN}{\hlight{\lift{GSN}}}
%	\SetKwData{R}{R}
%	\SetKwData{TR}{\hlight{\lift{TraceRel}}}
%	\SetKwData{Rm}{R'}
%	\SetKwData{D}{D}
%	\SetKwData{K}{K}
%	\SetKwData{Caz}{C0a}
%	\SetKwData{Cdz}{C0d}
%	\SetKwData{Cmz}{C0m}
%	\SetKwData{Cdmo}{C1dm}
%	\SetKwData{Camo}{C1am}
%	\SetKwData{Ct}{C2}
%	\SetKwData{Cr}{C3}
%	\SetKwData{Ctrecheck}{C2recheck}
%	\SetKwData{Ctrevise}{C2revise}
%	\SetKwData{Crrechecko}{C3recheck1}
%	\SetKwData{Crrecheckt}{C3recheck2}
%	
%	\SetKwFunction{Restrict}{\hlight{\lift{Restrict}}}
%	\SetKwFunction{Union}{Union}
%	\SetKwFunction{Trace}{\hlight{\lift{Trace}}}
%	\SetKwFunction{Slicesys}{\hlight{\lift{Slice\_Sys}}}
%	\SetKwFunction{Slicegsnv}{\hlight{\lift{Slice\_GSN\_V}}}
%	\SetKwFunction{Slicegsnr}{\hlight{\lift{Slice\_GSN\_R}}}
%	\SetKwFunction{CreateAnnotation}{\hlight{\lift{CreateAnnotation}}}
%	
%	\SetKwInOut{Input}{Inputs}\SetKwInOut{Output}{Output}
%	
%	\SetAlgoLined
%	\Input{\typerel{\S}{\Sys}    \tcp*[f]{Initial SPL megamodel} \newline
%		   \typerel{\Sm}{\Sys}   \tcp*[f]{Modified SPL megamodel} \newline
%	   	   \typerel{\A}{\GSN}   \tcp*[f]{GSN model}                \newline
%		   \typerel{\R}{\TR}  \tcp*[f]{Traceability map}         \newline
%		   \D =$\langle \Caz, \Cdz, \Cmz \rangle$                  \tcp*[f]{Delta}}
%	\Output{\K \tcp*[f]{GSN model Annotation}}
%	\BlankLine
%	
%	\assign{\Rm}{\Restrict{\R,\D}} \;
%	\assign{\Cdmo}{\Slicesys{\S, \Union{\Cdz, \Cmz}}} \;
%	\assign{\Camo}{\Slicesys{\Sm,\Union{\Caz, \Cmz}}} \;
%	\assign{\Ctrecheck}{\Union{\Trace{\R,\Cdmo}, \Trace{\Rm, \Camo}}} \;
%	\assign{\Ctrevise}{\Trace{\R,\Cdz}} \;
%	\assign{\Crrechecko}{\Slicegsnv{\A,\Ctrevise}} \;
%	\assign{\Crrecheckt}{\Slicegsnr{\A,\Union{\Ctrecheck, \Crrechecko}}} \;
%	\assign{\K}{\CreateAnnotation{\A, \Crrecheckt, \Ctrevise}} \;
%	\Return \K;
%	\BlankLine
%	
%	\caption{\lift{GSN\_IA}}
%	\label{alg:CIA_lifted}
%\end{algorithm}
%\DecMargin{1em}
%
%\begin{algorithm}[t]
%%\vspace{-0.3in}
%	\SetKwInOut{Input}{Inputs}\SetKwInOut{Output}{Output}
%	
%	\SetKwData{R}{R}
%	\SetKwData{Rn}{R'}
%	\SetKwData{TR}{\lift{TraceRel}}
%	\SetKwData{D}{D}
%	\SetKwData{Caz}{C0a}
%	\SetKwData{Cdz}{C0d}
%	\SetKwData{Cmz}{C0m}
%	\SetKwData{E}{e}
%	\SetKwData{PC}{pc}
%	\SetKwData{PCn}{pc'}
%	\SetKwData{EPC}{(e,pce)}
%	\SetKwData{EPCn}{(e,pc')}
%	\SetKwData{XYPC}{((x,y),pc)}
%	\SetKwData{XY}{(x,y)}
%	\SetKwData{X}{x}
%	
%	\SetKwFunction{Add}{add}
%	\SetKwFunction{Remove}{remove}
%	
%	\SetKwFunction{Find}{Find}
%	\SetKwFunction{Fn}{Function}
%	\SetKwProg{Fn}{Function}{ is}{end}
%	
%	\SetAlgoLined
%	\Input{\typerel{\R}{\TR}  \tcp*[f]{Traceability map}         \newline
%		\D =$\langle \Caz, \Cdz, \Cmz \rangle$                  \tcp*[f]{Delta}}
%	\Output{\Rn \tcp*[f]{restricted traceability relation}}
%	\BlankLine
%	%
%	%\Fn{\Add(\typerel{\R}{\TR}, \EPC)}{
%	%	\assign{\tPCn}{\Find{\R,\E}} \;
%	%	\If(\tcp*[f]{was e found in R?}){(\tPCn)}{
%	%		\lIf{\tPC == \tPCn}
%	%		{\Return \R}
%	%		\lElse{\Return \R $-$ \{\EPCn\} $\union$ \{(\E, \tPC $\vee$ \tPCn)\}} 
%	%	} 
%	%	\Else{
%	%		\Return \R $\union$ \{\EPCn\} \;
%	%	}
%	%}
%	%
%	\BlankLine
%	
%	\Fn{\Remove(\typerel{\R}{\TR}, \XYPC)}{
%		\assign{\tPCn}{\Find{\R,\XY}} \;
%		\If(\tcp*[f]{was (x,y) found in R?}){(\tPCn)}{
%			\lIf{\tPC == \tPCn}
%			{\Return \R $-$ \{\XYPC\}}
%			\lElse{\Return \R $-$ \{\XYPC\} $\union$ \{(\XY, $\neg$\tPC $\wedge$ \tPCn)\}} 
%		} 
%		\Else{
%			\Return \R \;
%		}
%	}
%	
%	\BlankLine
%	
%	\assign{\Rn}{\R} \;
%	%\For{$\EPC \in \Caz$}{
%	%	\assign{\Rn}{\Add{\Rn,\EPC}}
%	%}
%	\For{$\EPC \in \Cdz$}{
%		\For{$\XYPC \in \Rn$} {
%			\If{\E == \X} {
%				\assign{\Rn}{\Remove{\Rn, \XYPC}}
%			}
%		}
%	}
%	\Return \Rn \;
%	\caption{\lift{Restrict}}
%	\label{alg:restrict_lifted}
%	%\vspace{-0.15in}
%\end{algorithm}
%\vspace{-3in}

%%%%%%%%%%%%%%%%%%%%%%%%%%%%%%%
% Lifted algorithm
%%%%%%%%%%%%%%%%%%%%%%%%%%%%%%%

\lstinputlisting[style=lean,basicstyle=\small,firstline=102,lastline=116,caption=Lifted Change Impact Assessment algorithm.,label=lst:liftedCIA]{liftedCIA.lean}

Listing~\ref{lst:liftedCIA} is the variability-aware version of the algorithm in Listing~\ref{lst:gsnia}. Both algorithms are compositions of function/operator calls, so each of those functions/operators is replaced with its lifted counterpart. We assume that lifted versions of the three slicers are provided, and that they meet the correctness criteria of Fig.~\ref{fig:correctness1}. 
%The structure of the lifted algorithm follows that of the original one.

All the set types used in GSN\_IA need to be lifted. Definitions in lines 1-4 are lifted sets of system model elements, GSN model elements, and traceability mappings. A lifted delta (line 4) is composed of three lifted sets (additions, deletions and modifications).

The proof of the correctness theorem used auxiliary correctness lemmas for each of the helper algorithms. Each of the proofs expands definitions and repeatedly applies the correctness of lifted function composition (Fig.~\ref{fig:correctness2}). 
%{\url{https://www.github.com/ramyshahin/variability/}}.
%%%%%%%%%%%%%%%%%%%%%%%%
% Lifted helpers
%%%%%%%%%%%%%%%%%%%%%%%%

\vspace{0.1in}
\noindent
{\bf Lifted Helper Algorithms.}
%\subsection{Lifted Helper Algorithms}
\label{sec:lifted_helpers}
Since the lifted CIA algorithm operates on lifted data structures, all helper algorithms need to be modified to correctly operate on lifted data structures as well. In particular, we outline lifted versions of \restrict~and \trace~(Listing~\ref{lst:liftedHelpers}).

The original implementation of \restrict~takes a traceability map and a delta as inputs, and returns the minimal subset of the traceability map that covers all the elements in the delta. We now have presence conditions associated to system model elements, assurance case elements, and also the traceability links in between. The lifted version of \restrict~(referred to as \lift{\restrict}) needs to correctly process all those presence conditions.

The lifted algorithm starts by calculating the set of relevant elements in the system model, which is the union of added, deleted and modified elements in the delta (line 2). The algorithm returns a lifted traceability mapping as a function taking \code{((s,g),pc)}, where \code{(s,g)} is a system model element-GSN model element pair, and \code{pc} is a presence condition. This function evaluates to the conjunction of applying the input traceability map \code{t} to \code{((s,g),pc)}, and applying \code{relevant} to \code{(s,g)}. Recall that variability-aware sets (as well as Lean sets) are functions mapping values of a given type to propositions.

%Removing a traceability link from $\Rn$ needs to take presence conditions into consideration. When removing $((x,y),\tPC{})$ from $\Rn$, there are three possible cases:
%(1) if $(x,y)$ exists in $\Rn$ with the same presence condition $\tPC{}$, we just remove it (line 4).
%(2) if $(x,y)$ exists in $\Rn$ with a different presence condition $\tPC{}'$, then we only need to remove $(x,y)$ for the intersection of the set of products denoted by $\tPC{}$ and $\tPC{}'$ (line 5).
%(3) if $(x,y)$ does not exist in $\Rn$ at all, we do not remove anything (line 8).

Similarly, \lift{\trace} is the lifted version of \trace. The returned lifted set is a function mapping a GSN model element \code{g}~to the set of configurations from which there exists a system model element \code{s}~in the input lifted set of system elements, where \code{(s,g)} belongs to the input traceability map.

The lifted version of \createAnnotation~(named \lift{\createAnnotation}) is of exactly the same structure as the original because it strictly uses set operations (union, set difference and image), which have been all lifted as a part of the underlying variability-aware set implementation (Listing~\ref{lst:var}).
%The algorithm starts with an empty set of GSN elements (line 1), then iterates through all the elements of the traceability relation (lines 2-7). For each traceability link $(x,y)$ with presence condition $\tPC{}$, if $x$ exists in the system model parameter $\S$ with presence condition $\tPC{}'$ that isn't \ff, and if the conjunction of $\tPC{}$ and $\tPC{}'$ is satisfiable, we add $(y, \tPC{} \land \tPC{}')$ to $A$. Here the conjunction of the two presence conditions denotes the intersection of the sets of products denoted by the traceability link presence condition, and the GSN element presence condition.

\lstinputlisting[style=lean,basicstyle=\small,firstline=46,lastline=51,caption=Lifted implementation of \code{restrict} and \code{trace}.,label=lst:liftedHelpers]{liftedCIA.lean}
%\vspace{0.3in}

%\vspace{-0.4in}
%\begin{algorithm}[t]
%	\SetKwInOut{Input}{Inputs}\SetKwInOut{Output}{Output}
%	\SetKwData{S}{S}
%	\SetKwData{Sys}{\lift{Sys}}
%	\SetKwData{R}{R}
%	\SetKwData{TR}{\lift{TraceRel}}
%	\SetKwData{A}{A}
%	\SetKwData{GSN}{\lift{GSN}}
%	\SetKwData{X}{x}
%	\SetKwData{XPC}{xpc}
%	\SetKwData{Y}{y}
%	\SetKwData{YPC}{ypc}
%	\SetKwData{XP}{(\X,\XPC)}
%	\SetKwData{YP}{(\Y,\YPC)}
%	\SetKwData{PC}{pc}
%	\SetKwData{PCn}{pc'}
%	\SetKwData{E}{e}
%	
%	\SetKwFunction{Find}{Find}
%	
%	\SetAlgoLined
%	\Input{\typerel{\R}{\TR}  \tcp*[f]{Traceability map}         \newline
%		\typerel{\S}{\Sys}   \tcp*[f]{set of model elements with presence conditions}}
%	\Output{\typerel{\A}{\GSN}   \tcp*[f]{set of GSN elements with presence conditions}}
%	
%	\assign{\A}{\{\}}
%	
%	\For{$(\X,\Y,\tPC) \in \R$}{
%		\assign{\tPCn}{\Find{\S,\X}} \;
%		\If{(\tPCn)}{
%			\lIf{$\tPC \wedge \tPCn$}
%			{\assign{\A}{\A \cup \{(\Y, \tPC \wedge \tPCn)\}}}
%		}
%	}
%	
%	\Return{\A}
%	\caption{\lift{Trace}}
%	\label{alg:trace_lifted}
%	%\vspace{-0.15in}
%\end{algorithm}

\lstinputlisting[style=lean,basicstyle=\small,firstline=147,lastline=149,caption=Correctness theorem of \lift{GSN\_IA}.,label=lst:correctness]{liftedCIA.lean}

The correctness theorem of \lift{GSN\_IA} with respect to GSN\_IA is in Listing~\ref{lst:correctness}. It is a direct instantiation of the general correctness criteria in Fig.~\ref{fig:correctness1}, applied to inputs of the GSN\_IA algorithm.

\section{Case Study: Fair Classification Using Logistic Regression} \label{casestudy}
In this section, we instantiate our approach for achieving fairness, in the context of logistic regression. We denote the trained parameters of our logistic regressor by $\theta \in \mathbb{R}^d$, and the log-likelihood of $\theta$ given training set $S$ by $ll(\theta;S)$. 

%In logistic regression, we fit the parameters $\theta\in\mathbb{R}^d$ of a model $h_{\theta}:\mathbb{R}^d\rightarrow[0,1]$, s.t. $h_{\theta}(x)=\dfrac{1}{1+e^{-\theta^{T}x}}$. Binary prediction is done using a cut-off parameter. Given a set of probabilistic assumptions, we can fit the parameters of logistic regression by maximizing the log-likelihood function of $\theta$. In what follows, we use 
%\begin{equation*}
%\Pr[y=1~|~x;\theta]=h_{\theta}(x),~ \Pr[y=0~|~x;\theta]=1-h_{\theta}(x)
%\end{equation*}

%The log-likelihood of $\theta$ in this case is
%\begin{equation*}
%ll(\theta;S)=\sum\limits_{i=1}^{n} y_{i}\log(h_{\theta}(x_i))+(1-y_{i})\log(1-h_{\theta}(x_i))
%\end{equation*}

We wish to solve the following optimization problem:


\begin{align}\label{eq:1}
\begin{split}
\underset{\theta}{\text{minimize}}
 ~~&L_{S}^{0\text{-}1}(\theta)\\
&+d_1|FPR_{A=0}(\theta;S)-FPR_{A=1}(\theta;S)|\\
&+d_2|FNR_{A=0}(\theta;S)-FNR_{A=1}(\theta;S)|
\end{split}
\end{align}
where $d_1, d_2 \geq 0$ are to be set up front, according to the desired trade-off between accuracy, FPR matching, and FNR matching. Applying our suggested relaxation ($R_{FP}$, $R_{FN}$ are to be set as either Absolute Value Difference or Squared Difference penalizers), and adding a standard $\ell_2$ regularization term, we get the following convex optimization problem:
\begin{align} \label{eq:2}
\begin{split}
\underset{\theta}{\text{minimize}}
~~&-ll(\theta;S) \\
&+c_1 R_{FP}(\theta;S) \\
&+c_2 R_{FN}(\theta;S)\\
&+q\left|\left|\theta\right|\right|_2^2 \\
\end{split}
\end{align}

For convenience, we will denote the objective in (\ref{eq:1}) by $\text{Objective}(\theta;S,d_1,d_2)$, and the objective in the proxy problem (\ref{eq:2}) by $\text{Proxy}(\theta;S,c_1,c_2,q)$. As the proxy is easy to solve using standard methods, we use it when optimizing, and then shift back to the original problem for estimating the quality of our results.

%Effectively, our (relaxed) penalizers are set under the assumption that the distance between the prediction and the true label serves as a reliable proxy for the 0-1 loss, i.e., when predictions are done with high confidence, close to 0 or 1, far from the decision boundary. The terms are minimized when the average distance is the same for both groups in $S^{pos}$ and for both groups in $S^{neg}$. 

%In order to maximize $ll(\theta;S)$, and at the same time seek fair solutions, we use the following gradient update rule:
%\begin{equation*}
%   \theta^{i+1} = \theta^{i} - \eta_t(\nabla_{\theta}(-ll)+C_1\nabla_{\theta}(R_{FPR})+C_2\nabla_{\theta}(R_{FNR})+C_3 \theta)
%\end{equation*}
%Where $\eta_t$  is the learning rate (gradient step size).
%==============================================================================
\section{Implementation}
\label{sec:impl}
%==============================================================================

We built our {\System} prototype for the ARMv7-M architecture.  Our
prototype provides MPU and DWT configurations as a run-time component
written in C and executed at the end of the device boot sequence.
We implemented constant island removal as a
simple intermediate representation (IR) pass in the LLVM 10.0
compiler~\cite{LLVM:CGO04}.  The constant island removal pass
simply uses the existing {\tt -mexecute-only} option in
LLVM's Clang front-end and passes it along to the link-time optimization
(LTO) code generator.  Our prototype runs the constant island removal
pass when linking the IR of the application, libraries (e.g., newlib and
compiler-rt), and MPU and DWT configurations; this
ensures that all code has no constant islands.  Our prototype adds
88~source lines of C++ code to LLVM and has 177~source
lines of C~code in the {\System} run-time.  We leave the {\System}
implementation on ARMv8-M for future work.

Different ARM microcontrollers support different numbers of MPU
regions and DWT comparators, and the maximum ranges of their
DWT comparators may vary.
Our prototype runs on an STM32F469
Discovery board which supports up to 8~MPU
regions~\cite{STM32CortexM4:Manual} and 4~DWT
comparators~\cite{STM32F469I-DISCO:Manual}.  Each DWT
comparator can only watch over a maximum address range of 32~KB
(a maximal mask value of 15), limiting
our prototype to the following two options:

\begin{inparaenum}
\item
  Use all 4~DWT comparators to support a maximum code size of
  128~KB; the application must run in unprivileged mode in order
  for the critical system registers to be write-protected.

\item
  Configure one DWT comparator to write-protect the DWT registers
  ({\tt 0xE0001000} -- {\tt 0xE0001FFF}) and another to
  write-protect the SCB ({\tt 0xE000ED00} -- {\tt 0xE000ED8F}) and
  {\tt DEMCR} ({\tt 0xE000EDFC}). This protects a maximum code size
  of 64~KB using the remaining 2~DWT comparators.
\end{inparaenum}

To accommodate a wider range of applications on our board with less
performance loss, our prototype automatically chooses one option over
the other based on the application code size.  It rejects an application
if the code size exceeds our board's 128~KB limit.

While our {\System} prototype only supports single bare-metal
embedded applications, {\System} can also support multiple applications
running on an embedded real-time operating system (RTOS) such as Amazon
FreeRTOS~\cite{FreeRTOS:Amazon}.  On embedded systems, the application
and RTOS kernel code is linked into a single shared code
segment.  {\System} can protect this code segment with little adaptation.

\textbf{Related work}:
% Object detection related datasets/algo in non-medical domain
% Locally labeled CXR dataset
A few CXR datasets have localized abnormality annotations \cite{shih2019augmenting,filice2020crowdsourcing,jaeger2014two} that are curated manually. These are high quality gold standard ground truth datasets but tend to be smaller in scale (< 30,000 images) and have a narrow coverage, with typically only 1-2 labels. In addition, since most labeling efforts only have abnormality semantics attached, no direct relationships with the affected anatomical locations are available. 

%MEHDI: repeated concepts from above. I am removing the following: 

%The lack of anatomic semantics in the annotation is a limitation for complex multi-modal clinical reasoning work, e.g., differential diagnosis, since clinicians often integrate information along anatomical lines, and for downstream report generation tasks, which often requires describing not only the abnormality but also correctly communicate the location of the abnormalities (and medical devices) to the receiving clinicians. 

Two recent CXR datasets have labels for anatomies described in the reports. In \cite{datta2020dataset}, a small manually annotated dataset (2000 reports) included 10 abnormalities that are individually associated with 29 unique spatial locations (anatomies) at the report level. Another CXR dataset has automatically extracted abnormality and anatomy labels as disconnected concepts that are only correlated at the study level from  160,000 reports using a supervised NLP algorithm \cite{bustos2020padchest}. This was trained on a smaller set of manually annotated data. Neither datasets contain localized annotations for the associated CXR images, nor any comparison relation annotations between sequential exams, both of which are available in the Chest ImaGenome dataset. In Table \ref{tab:related}, we present a comparison of our Chest ImagGenome dataset with other datasets available in the literature.

% Table -- Kashyap

% MEdical imaging datasets to go here: Discussed that we will only focus on cxr datasets that are available for this paper. 
% \caption{\color{red} Kashyap, feel free to continue with the table. We should remove the questionmarks and add a line for our dataset (since all others are not graph). For longer text, using abbreviations and explaining them in the caption often works better. If fill in the values is not possible, it is better to remove the table altogether.}


\begin{table}[t!]
\caption{Summary of existing chest X-ray datasets}
\resizebox{\textwidth}{!}{%
\begin{tabular}{@{}lllllllll@{}}
\toprule
\textbf{Dataset} & \textbf{Annotation Level} & \textbf{Annotation Method} & \textbf{Num Labels} & \textbf{Anatomy Labeled} & \textbf{Graph} & \textbf{Dataset Size} & \textbf{Temporal Labels} & \textbf{Reports} \\ \midrule
SIIM-ACR Pneumothorax Segmentation \cite{filice2020crowdsourcing} & Segmentation & Manual + augmented & 1 & No & No & 12,047 & No & No \\
RSNA Pneumonia Detection Challenge   \cite{shih2019augmenting} & Bounding Boxes & Manual & 1 & No & No & 30,000 & No & No \\
Indiana University Chest X-ray collection \cite{demner2016preparing} & Global & Automated & 10 & No & No & 3,813 & No & Yes \\
NIH CXR dataset \cite{wang2017chestx} & Global & Automated & 14 & No & No & 112,120 & No & No \\
PLCO \cite{team2000prostate} & Global & Automated & 24 & Yes & No & 236,000 & Yes & No \\
Stanford CheXpert \cite{irvin2019chexpert} & Global & Automated & 14 & No & No & 224,316 & No & No \\
MIMIC-CXR \cite{johnson2019mimic} & Global & Automated & 14 & No & No & 377,110 & No & Yes \\
Dutta \cite{datta2020dataset} & Global & Manual & 10 & Yes & Yes & 2,000 & No & Yes \\
PadChest \cite{bustos2020padchest} & Global & Manual + automated & 297 & Yes & No & 160,868 & No & Yes \\
Montgomery County Chest X-ray   \cite{jaeger2014two} & Segmentation & Manual & 1 & Yes & No & 138 & No & No \\
Shenzen Hospital Chest X-ray   \cite{jaeger2014two} & Segmentation & Manual & 1 & Yes & No & 662 & No & No \\  \hline \hline
\textbf{Chest ImaGenome} & Bounding Boxes & Automated & 131 & Yes & Yes & 242,072 & Yes & Yes \\
\bottomrule
\end{tabular}%
}
\label{tab:related}
\vspace{-0.4cm}
\end{table}
% removed (Derived from MIMIC-CXR \cite{johnson2019mimic}) % makes table really small


\begin{comment}
\begin{figure}
\includegraphics[width=\linewidth]{figs/beyond_tss_lesion.pdf}
\caption[]{End-to-End runtime lesion study of the entire MNIST dataset and the FMA featurized music dataset. Each of DROP's contributions provides a runtime improvement.}
\label{fig:beyond_lesion}
\end{figure}
\end{comment}



\section{Conclusion}
\label{sec:conclusion}

Advanced data analytics techniques must scale to rising data volumes. 
DR techniques offer a powerful toolkit when processing these datasets, with PCA frequently outperforming popular techniques in exchange for high computational cost. 
In response, we propose DROP, a new dimensionality reduction optimizer. 
DROP combines progressive sampling, progress estimation, and online aggregation to identify high quality low dimensional bases via PCA without processing the entire dataset by balancing the runtime of downstream tasks and achieved dimensionality. 
Thus, DROP provides a first step in bridging the gap between quality and efficiency in end-to-end DR for downstream \red{analytics}. 

%We revisit canonical operators for time series dimensionality reduction and the measurement study of~\cite{keogh-study}, and show that PCA is more effective than popular alternatives in the data mining literature often by a margin of over $2\times$ on average on gold-standard time series benchmark data sets with respect to output data dimension. More surprisingly, we empirically demonstrate that a small number of samples are sufficient to accurately characterize directions of maximum variance and obtain a high-quality low-dimensional transformation.




\bibliographystyle{splncs04}
\bibliography{spl,modeling,pl,safety}
\end{document}