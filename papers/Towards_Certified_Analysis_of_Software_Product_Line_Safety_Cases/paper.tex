\documentclass[runningheads]{llncs}

\usepackage{amsmath}
\usepackage{oz}
\usepackage{graphicx}
\usepackage[english]{babel}
\usepackage[T1]{fontenc}
\usepackage{tikz-cd}
\usepackage{caption}
\usepackage{framed}
\usepackage{subcaption}
\captionsetup{compatibility=false}

\usepackage[utf8x]{inputenc}
\usepackage{amssymb, upgreek}

\usepackage{color}

\usepackage{rotating}

% Lean preamble
\definecolor{keywordcolor}{rgb}{0.7, 0.1, 0.1}   % red
\definecolor{commentcolor}{rgb}{0.4, 0.4, 0.4}   % grey
\definecolor{symbolcolor}{rgb}{0.0, 0.1, 0.6}    % blue
\definecolor{sortcolor}{rgb}{0.1, 0.5, 0.1}      % green
\definecolor{errorcolor}{rgb}{1, 0, 0}           % bright red
\definecolor{stringcolor}{rgb}{0.5, 0.3, 0.2}    % brown

\usepackage{listings}
\usepackage{chngcntr}
\def\lstlanguagefiles{lstlean.tex}
\lstset{
	language=lean,
	abovecaptionskip=\smallskipamount,
	belowcaptionskip=\smallskipamount,
	belowskip=-2em,
	basicstyle=\small
	}
\lstdefinestyle{lean}{
	float=t,
	%floatplacement=tp,
	%aboveskip=3pt,
	%belowskip=-3pt,
	numbers=left
}

%\captionsetup[lstlisting]{numberbychapter=false}

%TODO
\newcommand{\todo}[1]{{\color{red}{\bf [TODO]:~{#1}}}}

%THEOREMS
\newtheorem{theorem}{Theorem}
\newtheorem{corollary}{Corollary}
\newtheorem{lemma}{Lemma}
\newtheorem{proposition}{Proposition}
\newtheorem{problem}{Problem}
\newtheorem{definition}{Definition}
\newtheorem{remark}{Remark}
\newtheorem{example}{Example}
\newtheorem{assumption}{Assumption}

%HANS' CONVENIENCES
\newcommand{\define}[1]{\textit{#1}}
\newcommand{\join}{\vee}
\newcommand{\meet}{\wedge}
\newcommand{\bigjoin}{\bigvee}
\newcommand{\bigmeet}{\bigwedge}
\newcommand{\jointimes}{\boxplus}
\newcommand{\meettimes}{\boxplus'}
\newcommand{\bigjoinplus}{\bigjoin}
\newcommand{\bigmeetplus}{\bigmeet}
\newcommand{\joinplus}{\join}
\newcommand{\meetplus}{\meet}
\newcommand{\lattice}[1]{\mathbf{#1}}
\newcommand{\semimod}{\mathcal{S}}
\newcommand{\graph}{\mathcal{G}}
\newcommand{\nodes}{\mathcal{V}}
\newcommand{\agents}{\{1,2,\dots,N\}}
\newcommand{\edges}{\mathcal{E}}
\newcommand{\neighbors}{\mathcal{N}}
\newcommand{\Weights}{\mathcal{A}}
\renewcommand{\leq}{\leqslant}
\renewcommand{\geq}{\geqslant}
\renewcommand{\preceq}{\preccurlyeq}
\renewcommand{\succeq}{\succcurlyeq}
\newcommand{\Rmax}{\mathbb{R}_{\mathrm{max}}}
\newcommand{\Rmin}{\mathbb{R}_{\mathrm{min}}}
\newcommand{\Rext}{\overline{\mathbb{R}}}
\newcommand{\R}{\mathbb{R}}
\newcommand{\N}{\mathbb{N}}
\newcommand{\A}{\mathbf{A}}
\newcommand{\B}{\mathbf{B}}
\newcommand{\x}{\mathbf{x}}
\newcommand{\e}{\mathbf{e}}
\newcommand{\X}{\mathbf{X}}
\newcommand{\W}{\mathbf{W}}
\newcommand{\weights}{\mathcal{W}}
\newcommand{\alternatives}{\mathcal{X}}
\newcommand{\xsol}{\bar{\mathbf{x}}}
\newcommand{\y}{\mathbf{y}}
\newcommand{\Y}{\mathbf{Y}}
\newcommand{\z}{\mathbf{z}}
\newcommand{\Z}{\mathbf{Z}}
\renewcommand{\a}{\mathbf{a}}
\renewcommand{\b}{\mathbf{b}}
\newcommand{\I}{\mathbf{I}}
\DeclareMathOperator{\supp}{supp}
\newcommand{\Par}[2]{\mathcal{P}_{{#1} \to {#2}}}
\newcommand{\Laplacian}{\mathcal{L}}
\newcommand{\F}{\mathcal{F}}
\newcommand{\inv}[1]{{#1}^{\sharp}}
\newcommand{\energy}{Q}
\newcommand{\err}{\mathrm{err}}
\newcommand{\argmin}{\mathrm{argmin}}
\newcommand{\argmax}{\mathrm{argmax}}
%% Software Product Line (SPL) macros
\DeclareMathOperator{\Prop}{Prop}
\DeclareMathOperator{\Conf}{Conf}

\newcommand{\pc}[1]     {pc_{#1}}
\newcommand{\SPL}       {\mathcal{L}}
\newcommand{\feat}[1]{\bm{#1}} %for features
\newcommand{\featset}{F}
\newcommand{\featmodel}{\Phi}
\newcommand{\domainmodel}{D}
\newcommand{\pcmap}{\phi}
\newcommand{\config}{\rho}
\newcommand{\PC}[1]{\texttt{#1}}
\newcommand{\indexProd}[2]{#1|_#2}
\begin{document}
\counterwithout{lstlisting}{chapter}

\title{Towards Certified Analysis of Software Product Line Safety Cases}
\author{Ramy Shahin\inst{1} \and
	Sahar Kokaly\inst{2} \and
	Marsha Chechik\inst{1}}
\institute{University of Toronto, Toronto, Canada\\
	\email{\{rshahin,chechik\}@cs.toronto.edu} \and
	General Motors, Canada\\
	\email{sahar.kokaly@gm.com}}
\maketitle

\vspace{-0.2in}
\begin{abstract}
	Safety-critical software systems are in many cases designed and implemented as families of products, usually referred to as Software Product Lines (SPLs). Products within an SPL vary from each other in terms of which features they include. Applying existing analysis techniques to SPLs and their safety cases is usually challenging because of the potentially exponential number of products with respect to the number of supported features.
	In this paper, we present a methodology and infrastructure for certified \emph{lifting} of existing single-product safety analyses to product lines. To ensure certified safety of our infrastructure, we implement it in an interactive theorem prover, including formal definitions, lemmas, correctness criteria theorems, and proofs.
	%Modifying a software system might impact its safety case in many ways.  algorithms identify how local changes to specific system elements affect safety case elements, given traceability links between the two. Those algorithms can only work on the artifacts of a single software product at a time. However, in many cases a family of software products is developed together in the form of a Software Product Line (SPL), with artifacts potentially shared between different product variants. In this paper, we present lifting of a CIA algorithm to Software Product Lines aimed at supporting the input of feature-specific modifications, and potentially returning feature-specific safety case annotations. 
	
	We apply this infrastructure to formalize and lift a Change Impact Assessment (CIA) algorithm. We present a formal definition of the lifted algorithm, outline its correctness proof (with the full machine-checked proof available online), and discuss its implementation within a model management framework. 
\keywords{Safety cases, Product lines, Lean, Certified analysis.}
\end{abstract}
\vspace{-0.2in}

% !TEX root = ../arxiv.tex

Unsupervised domain adaptation (UDA) is a variant of semi-supervised learning \cite{blum1998combining}, where the available unlabelled data comes from a different distribution than the annotated dataset \cite{Ben-DavidBCP06}.
A case in point is to exploit synthetic data, where annotation is more accessible compared to the costly labelling of real-world images \cite{RichterVRK16,RosSMVL16}.
Along with some success in addressing UDA for semantic segmentation \cite{TsaiHSS0C18,VuJBCP19,0001S20,ZouYKW18}, the developed methods are growing increasingly sophisticated and often combine style transfer networks, adversarial training or network ensembles \cite{KimB20a,LiYV19,TsaiSSC19,Yang_2020_ECCV}.
This increase in model complexity impedes reproducibility, potentially slowing further progress.

In this work, we propose a UDA framework reaching state-of-the-art segmentation accuracy (measured by the Intersection-over-Union, IoU) without incurring substantial training efforts.
Toward this goal, we adopt a simple semi-supervised approach, \emph{self-training} \cite{ChenWB11,lee2013pseudo,ZouYKW18}, used in recent works only in conjunction with adversarial training or network ensembles \cite{ChoiKK19,KimB20a,Mei_2020_ECCV,Wang_2020_ECCV,0001S20,Zheng_2020_IJCV,ZhengY20}.
By contrast, we use self-training \emph{standalone}.
Compared to previous self-training methods \cite{ChenLCCCZAS20,Li_2020_ECCV,subhani2020learning,ZouYKW18,ZouYLKW19}, our approach also sidesteps the inconvenience of multiple training rounds, as they often require expert intervention between consecutive rounds.
We train our model using co-evolving pseudo labels end-to-end without such need.

\begin{figure}[t]%
    \centering
    \def\svgwidth{\linewidth}
    \input{figures/preview/bars.pdf_tex}
    \caption{\textbf{Results preview.} Unlike much recent work that combines multiple training paradigms, such as adversarial training and style transfer, our approach retains the modest single-round training complexity of self-training, yet improves the state of the art for adapting semantic segmentation by a significant margin.}
    \label{fig:preview}
\end{figure}

Our method leverages the ubiquitous \emph{data augmentation} techniques from fully supervised learning \cite{deeplabv3plus2018,ZhaoSQWJ17}: photometric jitter, flipping and multi-scale cropping.
We enforce \emph{consistency} of the semantic maps produced by the model across these image perturbations.
The following assumption formalises the key premise:

\myparagraph{Assumption 1.}
Let $f: \mathcal{I} \rightarrow \mathcal{M}$ represent a pixelwise mapping from images $\mathcal{I}$ to semantic output $\mathcal{M}$.
Denote $\rho_{\bm{\epsilon}}: \mathcal{I} \rightarrow \mathcal{I}$ a photometric image transform and, similarly, $\tau_{\bm{\epsilon}'}: \mathcal{I} \rightarrow \mathcal{I}$ a spatial similarity transformation, where $\bm{\epsilon},\bm{\epsilon}'\sim p(\cdot)$ are control variables following some pre-defined density (\eg, $p \equiv \mathcal{N}(0, 1)$).
Then, for any image $I \in \mathcal{I}$, $f$ is \emph{invariant} under $\rho_{\bm{\epsilon}}$ and \emph{equivariant} under $\tau_{\bm{\epsilon}'}$, \ie~$f(\rho_{\bm{\epsilon}}(I)) = f(I)$ and $f(\tau_{\bm{\epsilon}'}(I)) = \tau_{\bm{\epsilon}'}(f(I))$.

\smallskip
\noindent Next, we introduce a training framework using a \emph{momentum network} -- a slowly advancing copy of the original model.
The momentum network provides stable, yet recent targets for model updates, as opposed to the fixed supervision in model distillation \cite{Chen0G18,Zheng_2020_IJCV,ZhengY20}.
We also re-visit the problem of long-tail recognition in the context of generating pseudo labels for self-supervision.
In particular, we maintain an \emph{exponentially moving class prior} used to discount the confidence thresholds for those classes with few samples and increase their relative contribution to the training loss.
Our framework is simple to train, adds moderate computational overhead compared to a fully supervised setup, yet sets a new state of the art on established benchmarks (\cf \cref{fig:preview}).

\section{Background and Motivation}

\subsection{IBM Streams}

IBM Streams is a general-purpose, distributed stream processing system. It
allows users to develop, deploy and manage long-running streaming applications
which require high-throughput and low-latency online processing.

The IBM Streams platform grew out of the research work on the Stream Processing
Core~\cite{spc-2006}.  While the platform has changed significantly since then,
that work established the general architecture that Streams still follows today:
job, resource and graph topology management in centralized services; processing
elements (PEs) which contain user code, distributed across all hosts,
communicating over typed input and output ports; brokers publish-subscribe
communication between jobs; and host controllers on each host which
launch PEs on behalf of the platform.

The modern Streams platform approaches general-purpose cluster management, as
shown in Figure~\ref{fig:streams_v4_v6}. The responsibilities of the platform
services include all job and PE life cycle management; domain name resolution
between the PEs; all metrics collection and reporting; host and resource
management; authentication and authorization; and all log collection. The
platform relies on ZooKeeper~\cite{zookeeper} for consistent, durable metadata
storage which it uses for fault tolerance.

Developers write Streams applications in SPL~\cite{spl-2017} which is a
programming language that presents streams, operators and tuples as
abstractions. Operators continuously consume and produce tuples over streams.
SPL allows programmers to write custom logic in their operators, and to invoke
operators from existing toolkits. Compiled SPL applications become archives that
contain: shared libraries for the operators; graph topology metadata which tells
both the platform and the SPL runtime how to connect those operators; and
external dependencies. At runtime, PEs contain one or more operators. Operators
inside of the same PE communicate through function calls or queues. Operators
that run in different PEs communicate over TCP connections that the PEs
establish at startup. PEs learn what operators they contain, and how to connect
to operators in other PEs, at startup from the graph topology metadata provided
by the platform.

We use ``legacy Streams'' to refer to the IBM Streams version 4 family. The
version 5 family is for Kubernetes, but is not cloud native. It uses the
lift-and-shift approach and creates a platform-within-a-platform: it deploys a
containerized version of the legacy Streams platform within Kubernetes.

\subsection{Kubernetes}

Borg~\cite{borg-2015} is a cluster management platform used internally at Google
to schedule, maintain and monitor the applications their internal infrastructure
and external applications depend on. Kubernetes~\cite{kube} is the open-source
successor to Borg that is an industry standard cloud orchestration platform.

From a user's perspective, Kubernetes abstracts running a distributed
application on a cluster of machines. Users package their applications into
containers and deploy those containers to Kubernetes, which runs those
containers in \emph{pods}. Kubernetes handles all life cycle management of pods,
including scheduling, restarting and migration in case of failures.

Internally, Kubernetes tracks all entities as \emph{objects}~\cite{kubeobjects}.
All objects have a name and a specification that describes its desired state.
Kubernetes stores objects in etcd~\cite{etcd}, making them persistent,
highly-available and reliably accessible across the cluster. Objects are exposed
to users through \emph{resources}. All resources can have
\emph{controllers}~\cite{kubecontrollers}, which react to changes in resources.
For example, when a user changes the number of replicas in a
\code{ReplicaSet}, it is the \code{ReplicaSet} controller which makes sure the
desired number of pods are running. Users can extend Kubernetes through
\emph{custom resource definitions} (CRDs)~\cite{kubecrd}. CRDs can contain
arbitrary content, and controllers for a CRD can take any kind of action.

Architecturally, a Kubernetes cluster consists of nodes. Each node runs a
\emph{kubelet} which receives pod creation requests and makes sure that the
requisite containers are running on that node. Nodes also run a
\emph{kube-proxy} which maintains the network rules for that node on behalf of
the pods. The \emph{kube-api-server} is the central point of contact: it
receives API requests, stores objects in etcd, asks the scheduler to schedule
pods, and talks to the kubelets and kube-proxies on each node. Finally,
\emph{namespaces} logically partition the cluster. Objects which should not know
about each other live in separate namespaces, which allows them to share the
same physical infrastructure without interference.

\subsection{Motivation}
\label{sec:motivation}

Systems like Kubernetes are commonly called ``container orchestration''
platforms. We find that characterization reductive to the point of being
misleading; no one would describe operating systems as ``binary executable
orchestration.'' We adopt the idea from Verma et al.~\cite{borg-2015} that
systems like Kubernetes are ``the kernel of a distributed system.'' Through CRDs
and their controllers, Kubernetes provides state-as-a-service in a distributed
system. Architectures like the one we propose are the result of taking that view 
seriously.

The Streams legacy platform has obvious parallels to the Kubernetes
architecture, and that is not a coincidence: they solve similar problems.
Both are designed to abstract running arbitrary user-code across a distributed
system.  We suspect that Streams is not unique, and that there are many
non-trivial platforms which have to provide similar levels of cluster
management.  The benefits to being cloud native and offloading the platform
to an existing cloud management system are: 
\begin{itemize}
    \item Significantly less platform code.
    \item Better scheduling and resource management, as all services on the cluster are 
        scheduled by one platform.
    \item Easier service integration.
    \item Standardized management, logging and metrics.
\end{itemize}
The rest of this paper presents the design of replacing the legacy Streams 
platform with Kubernetes itself.


\section{Evaluation}
\label{sec:eval}

This section evaluates the performance, area and power of the \ZF architecture demonstrating how it improves over the state-of-the-art  DaDianNao accelerator~\cite{DaDiannao}. 
Section~\ref{sec:eval:method} details the experimental methodology. 
Section~\ref{sec:eval:performance} evaluates the performance of \ZF. 
Sections~\ref{sec:eval:area} and \ref{sec:eval:power} evaluate the area and power of \ZF, and Section~\ref{sec:add-ineffectual} considers the removal of non-zero neurons.

%
%

%

\subsection{Methodology}
\label{sec:eval:method}


%
%
\begin{table}[t!]
\centering
\begin{tabular}{|l|l|l|}
\hline
\textbf{Network} & \pbox{5cm}{\textbf{Conv.} \\ \textbf{Layers}} & \textbf{Source} \\ \hline \hline
alex     	 & 5 & Caffe: bvlc\_reference\_caffenet \\ \hline
google 	 & 59 & Caffe: bvlc\_googlenet \\ \hline
nin 		 & 12 & Model Zoo: NIN-imagenet \\ \hline
vgg19  	 & 16 & Model Zoo: VGG 19-layer \\ \hline
cnnM  & 5 & Model Zoo: VGG\_CNN\_M\_2048 \\ \hline
cnnS  & 5 & Model Zoo: VGG\_CNN\_S \\ \hline
\end{tabular}
\caption{Networks used}
\label{table:networks}
\end{table}

The evaluation uses the set of popular~\cite{AlexNIPS2012}, and state-of-the-art convolutional neural networks~\cite{ILSVRC15}\cite{nin}\cite{vgg}\cite{vgg19} shown in Table \ref{table:networks}. 
These networks perform image classification on the ILSVRC12 dataset~\cite{ILSVRC15}, which contains $256\times256$ images across 1000 classes. 
The experiments use a randomly selected set of 1000 images, one from each class. The networks are available, pre-trained for Caffe, either as part of the distribution or at the Caffe Model Zoo~\cite{model-zoo}.

%
%
We created a cycle accurate simulator of the baseline accelerator and \ZF. 
The simulator integrates with the Caffe framework~\cite{caffe} to enable on-the-fly validation of the layer ouput neurons. 
%
%
%
%
The area and power characteristics of \ZF and \BASE are measured 
with synthesized implementations. The two designs are implemented 
in Verilog and synthesized via the Synopsis Design 
Compiler~\cite{synopsys_site} with the TSMC 65nm library. 
The NBin, NBout, and \ZF offset SRAM buffers were modeled using 
the Artisan single-ported register file memory compiler~\cite{artisan} 
using double-pumping to allow a read and write per cycle. The eDRAM 
area and energy was modeled with \textit{Destiny}~\cite{destiny}.
%



%
%
%
%
%

%
%

%
%
%
%

%
%
%
%
%
%
%
%
%
%
%
%
%

\vspace{-0.1in}
\section{Changed Impact Assessment}
\label{sec:cia}
\vspace{-0.1in}
\newcommand{\CIA}{GSN\_IA}

In this section, we formalize the GSN-IA~\cite{Kokaly:2017} impact assessment algorithm, systematically design a lifted version of it, and prove its correctness based on the methodology in Sec.~\ref{sec:methodology}. 
%Lean has been recently used in several projects formalizing mathematics~\cite{mathlib:2020,Lewis:2019}. 

%Safety engineering is an essential part of the development process for safety-critical systems. Ensuring the correctness of analyses applied to safety artifacts is thus in many cases essential if we are to make safety-critical decisions based on the results of those analyses. Given the constructive nature of assurance cases, where evidence constructively satisfies goals, a theorem prover based on constructive logic like Lean is better suited than theorem provers based on classical logic (e.g., Isabelle/HOL~\cite{Nipkow:2002}).
\vspace{-0.1in}
\subsection{Single-Product Algorithm}
\vspace{-0.1in}

\lstinputlisting[style=lean, basicstyle=\small, firstline=8,lastline=19,caption=Type definitions of the formalized GSN\_IA algorithm.,label=lst:defs]{cia.lean}

\newcommand{\Annotation}{\code{Annotation}}
\newcommand{\Reuse}{\code{Reuse}}
\newcommand{\Recheck}{\code{Recheck}}
\newcommand{\Revise}{\code{Revise}}
\newcommand{\SysEl}{\code{SysEl}}
\newcommand{\GSNEl}{\code{GSNEl}}
\newcommand{\Sys}{\code{Sys}}
\newcommand{\GSN}{\code{GSN}}
\newcommand{\TraceRel}{\code{TraceRel}}
\newcommand{\sliceSys}{\code{sliceSys}}
\newcommand{\sliceGSNV}{\code{sliceGSN\_V}}
\newcommand{\sliceGSNR}{\code{sliceGSN\_R}}
\newcommand{\Dlta}{\code{Delta}}
\newcommand{\restrict}{\code{restrict}}
\newcommand{\trace}{\code{trace}}
\newcommand{\createAnnotation}{\code{createAnnotation}}

The data types and external dependencies of the \CIA~algorithm are defined in Listing~\ref{lst:defs}. \Annotation~is the data type of annotations assigned to GSN model elements, with the values \Reuse, \Recheck, and \Revise~(lines 1-2). \SysEl~and \GSNEl~are opaque types of system model elements and GSN model elements respectively, where a system model \Sys~and a GSN model \GSN~ are sets of each of those elements types (lines 4-6). \TraceRel~is a traceability relation between system model elements and GSN model elements, so it is a defined as a set of ordered pairs of \SysEl~and \GSNEl~(line 7). \CIA~is parameterized by three model slicers: \sliceSys~is a system model slicer, while \sliceGSNV~and \sliceGSNR~ are GSN model slicers. 
%\todo{what's the difference?} 
Each of the slicers takes a model and a set of elements used as the slicing criterion, returning a subset slice of the input model (lines 9-11). \Dlta~is composed of three sets of system elements, representing the elements added, modified and deleted (lines 12).

\lstinputlisting[style=lean, basicstyle=\small, firstline=21,lastline=43,caption=Helper functions and the formalized GSN\_IA algorithm.,label=lst:gsnia]{cia.lean}

Listing~\ref{lst:gsnia} has the definitions of the \CIA~algorithm, together with three helper functions. \restrict~ is a function taking a traceability relation ~\code{t}~and a delta ~\code{es}~as inputs, and returns a restricted subset of \code{t} only covering elements in \code{es} (lines 1-2). \trace~takes a traceability relation \code{t}~and a set of system elements~\code{es}~as inputs, and returns the set of GSN elements mapped from \code{es}~by \code{t} (lines 4-5). \createAnnotation~assigns an \Annotation~value to each element in a GSN model, given sets of elements to be rechecked and revised (lines 7-12).

%\lstinputlisting[style=lean, basicstyle=\small, firstline=48,lastline=61,caption=Formalized GSN\_IA algorithm.,label=lst:gsnia]{cia.lean}

The change impact assessment algorithm \CIA~takes two system models \name{S}~and \name{S'} and the delta \name{D}~between them. It also takes a GSN model \name{A}~and a traceability relation \name{R}~between system model elements and GSN model elements. It returns a set of ordered pairs of GSN model elements and annotations. The algorithm starts by restricting the traceability relation based on \name{D}, slices the original system model \name{S}~using the elements deleted and modified as a slicing criterion, and slices the modified system model \name{S'} using the added and modified elements as the slicing criterion (lines 16-18). Using those two slices, the corresponding GSN model elements are traced using the traceability relation (line 19). The GSN elements traced from elements deleted from the original system model are to be revised (line 20). The slice of the GSN model based on the traced elements are to be rechecked (lines 21-22), and both revise and recheck sets are used to annotate the GSN model elements (line 23). 

\vspace{-0.1in}
\subsection{Lifted Algorithm}
\label{sec:lifting}
\vspace{-0.1in}

%To construct \lift{GSN\_IA}, we start with an overview of the lifted data structures used for algorithm inputs, outputs and intermediate values, then outline \lift{GSN\_IA}, and finally give two examples of lifted helper algorithms.

%\subsection{Lifted Algorithm}
%\label{sec:lifted_algo}

%\IncMargin{1em}
%\LinesNumbered
%\begin{algorithm}[t]
%	\SetKwData{S}{S}
%	\SetKwData{Sm}{S'}
%	\SetKwData{Sys}{\hlight{\lift{Sys}}}
%	\SetKwData{A}{A}
%	\SetKwData{GSN}{\hlight{\lift{GSN}}}
%	\SetKwData{R}{R}
%	\SetKwData{TR}{\hlight{\lift{TraceRel}}}
%	\SetKwData{Rm}{R'}
%	\SetKwData{D}{D}
%	\SetKwData{K}{K}
%	\SetKwData{Caz}{C0a}
%	\SetKwData{Cdz}{C0d}
%	\SetKwData{Cmz}{C0m}
%	\SetKwData{Cdmo}{C1dm}
%	\SetKwData{Camo}{C1am}
%	\SetKwData{Ct}{C2}
%	\SetKwData{Cr}{C3}
%	\SetKwData{Ctrecheck}{C2recheck}
%	\SetKwData{Ctrevise}{C2revise}
%	\SetKwData{Crrechecko}{C3recheck1}
%	\SetKwData{Crrecheckt}{C3recheck2}
%	
%	\SetKwFunction{Restrict}{\hlight{\lift{Restrict}}}
%	\SetKwFunction{Union}{Union}
%	\SetKwFunction{Trace}{\hlight{\lift{Trace}}}
%	\SetKwFunction{Slicesys}{\hlight{\lift{Slice\_Sys}}}
%	\SetKwFunction{Slicegsnv}{\hlight{\lift{Slice\_GSN\_V}}}
%	\SetKwFunction{Slicegsnr}{\hlight{\lift{Slice\_GSN\_R}}}
%	\SetKwFunction{CreateAnnotation}{\hlight{\lift{CreateAnnotation}}}
%	
%	\SetKwInOut{Input}{Inputs}\SetKwInOut{Output}{Output}
%	
%	\SetAlgoLined
%	\Input{\typerel{\S}{\Sys}    \tcp*[f]{Initial SPL megamodel} \newline
%		   \typerel{\Sm}{\Sys}   \tcp*[f]{Modified SPL megamodel} \newline
%	   	   \typerel{\A}{\GSN}   \tcp*[f]{GSN model}                \newline
%		   \typerel{\R}{\TR}  \tcp*[f]{Traceability map}         \newline
%		   \D =$\langle \Caz, \Cdz, \Cmz \rangle$                  \tcp*[f]{Delta}}
%	\Output{\K \tcp*[f]{GSN model Annotation}}
%	\BlankLine
%	
%	\assign{\Rm}{\Restrict{\R,\D}} \;
%	\assign{\Cdmo}{\Slicesys{\S, \Union{\Cdz, \Cmz}}} \;
%	\assign{\Camo}{\Slicesys{\Sm,\Union{\Caz, \Cmz}}} \;
%	\assign{\Ctrecheck}{\Union{\Trace{\R,\Cdmo}, \Trace{\Rm, \Camo}}} \;
%	\assign{\Ctrevise}{\Trace{\R,\Cdz}} \;
%	\assign{\Crrechecko}{\Slicegsnv{\A,\Ctrevise}} \;
%	\assign{\Crrecheckt}{\Slicegsnr{\A,\Union{\Ctrecheck, \Crrechecko}}} \;
%	\assign{\K}{\CreateAnnotation{\A, \Crrecheckt, \Ctrevise}} \;
%	\Return \K;
%	\BlankLine
%	
%	\caption{\lift{GSN\_IA}}
%	\label{alg:CIA_lifted}
%\end{algorithm}
%\DecMargin{1em}
%
%\begin{algorithm}[t]
%%\vspace{-0.3in}
%	\SetKwInOut{Input}{Inputs}\SetKwInOut{Output}{Output}
%	
%	\SetKwData{R}{R}
%	\SetKwData{Rn}{R'}
%	\SetKwData{TR}{\lift{TraceRel}}
%	\SetKwData{D}{D}
%	\SetKwData{Caz}{C0a}
%	\SetKwData{Cdz}{C0d}
%	\SetKwData{Cmz}{C0m}
%	\SetKwData{E}{e}
%	\SetKwData{PC}{pc}
%	\SetKwData{PCn}{pc'}
%	\SetKwData{EPC}{(e,pce)}
%	\SetKwData{EPCn}{(e,pc')}
%	\SetKwData{XYPC}{((x,y),pc)}
%	\SetKwData{XY}{(x,y)}
%	\SetKwData{X}{x}
%	
%	\SetKwFunction{Add}{add}
%	\SetKwFunction{Remove}{remove}
%	
%	\SetKwFunction{Find}{Find}
%	\SetKwFunction{Fn}{Function}
%	\SetKwProg{Fn}{Function}{ is}{end}
%	
%	\SetAlgoLined
%	\Input{\typerel{\R}{\TR}  \tcp*[f]{Traceability map}         \newline
%		\D =$\langle \Caz, \Cdz, \Cmz \rangle$                  \tcp*[f]{Delta}}
%	\Output{\Rn \tcp*[f]{restricted traceability relation}}
%	\BlankLine
%	%
%	%\Fn{\Add(\typerel{\R}{\TR}, \EPC)}{
%	%	\assign{\tPCn}{\Find{\R,\E}} \;
%	%	\If(\tcp*[f]{was e found in R?}){(\tPCn)}{
%	%		\lIf{\tPC == \tPCn}
%	%		{\Return \R}
%	%		\lElse{\Return \R $-$ \{\EPCn\} $\union$ \{(\E, \tPC $\vee$ \tPCn)\}} 
%	%	} 
%	%	\Else{
%	%		\Return \R $\union$ \{\EPCn\} \;
%	%	}
%	%}
%	%
%	\BlankLine
%	
%	\Fn{\Remove(\typerel{\R}{\TR}, \XYPC)}{
%		\assign{\tPCn}{\Find{\R,\XY}} \;
%		\If(\tcp*[f]{was (x,y) found in R?}){(\tPCn)}{
%			\lIf{\tPC == \tPCn}
%			{\Return \R $-$ \{\XYPC\}}
%			\lElse{\Return \R $-$ \{\XYPC\} $\union$ \{(\XY, $\neg$\tPC $\wedge$ \tPCn)\}} 
%		} 
%		\Else{
%			\Return \R \;
%		}
%	}
%	
%	\BlankLine
%	
%	\assign{\Rn}{\R} \;
%	%\For{$\EPC \in \Caz$}{
%	%	\assign{\Rn}{\Add{\Rn,\EPC}}
%	%}
%	\For{$\EPC \in \Cdz$}{
%		\For{$\XYPC \in \Rn$} {
%			\If{\E == \X} {
%				\assign{\Rn}{\Remove{\Rn, \XYPC}}
%			}
%		}
%	}
%	\Return \Rn \;
%	\caption{\lift{Restrict}}
%	\label{alg:restrict_lifted}
%	%\vspace{-0.15in}
%\end{algorithm}
%\vspace{-3in}

%%%%%%%%%%%%%%%%%%%%%%%%%%%%%%%
% Lifted algorithm
%%%%%%%%%%%%%%%%%%%%%%%%%%%%%%%

\lstinputlisting[style=lean,basicstyle=\small,firstline=102,lastline=116,caption=Lifted Change Impact Assessment algorithm.,label=lst:liftedCIA]{liftedCIA.lean}

Listing~\ref{lst:liftedCIA} is the variability-aware version of the algorithm in Listing~\ref{lst:gsnia}. Both algorithms are compositions of function/operator calls, so each of those functions/operators is replaced with its lifted counterpart. We assume that lifted versions of the three slicers are provided, and that they meet the correctness criteria of Fig.~\ref{fig:correctness1}. 
%The structure of the lifted algorithm follows that of the original one.

All the set types used in GSN\_IA need to be lifted. Definitions in lines 1-4 are lifted sets of system model elements, GSN model elements, and traceability mappings. A lifted delta (line 4) is composed of three lifted sets (additions, deletions and modifications).

The proof of the correctness theorem used auxiliary correctness lemmas for each of the helper algorithms. Each of the proofs expands definitions and repeatedly applies the correctness of lifted function composition (Fig.~\ref{fig:correctness2}). 
%{\url{https://www.github.com/ramyshahin/variability/}}.
%%%%%%%%%%%%%%%%%%%%%%%%
% Lifted helpers
%%%%%%%%%%%%%%%%%%%%%%%%

\vspace{0.1in}
\noindent
{\bf Lifted Helper Algorithms.}
%\subsection{Lifted Helper Algorithms}
\label{sec:lifted_helpers}
Since the lifted CIA algorithm operates on lifted data structures, all helper algorithms need to be modified to correctly operate on lifted data structures as well. In particular, we outline lifted versions of \restrict~and \trace~(Listing~\ref{lst:liftedHelpers}).

The original implementation of \restrict~takes a traceability map and a delta as inputs, and returns the minimal subset of the traceability map that covers all the elements in the delta. We now have presence conditions associated to system model elements, assurance case elements, and also the traceability links in between. The lifted version of \restrict~(referred to as \lift{\restrict}) needs to correctly process all those presence conditions.

The lifted algorithm starts by calculating the set of relevant elements in the system model, which is the union of added, deleted and modified elements in the delta (line 2). The algorithm returns a lifted traceability mapping as a function taking \code{((s,g),pc)}, where \code{(s,g)} is a system model element-GSN model element pair, and \code{pc} is a presence condition. This function evaluates to the conjunction of applying the input traceability map \code{t} to \code{((s,g),pc)}, and applying \code{relevant} to \code{(s,g)}. Recall that variability-aware sets (as well as Lean sets) are functions mapping values of a given type to propositions.

%Removing a traceability link from $\Rn$ needs to take presence conditions into consideration. When removing $((x,y),\tPC{})$ from $\Rn$, there are three possible cases:
%(1) if $(x,y)$ exists in $\Rn$ with the same presence condition $\tPC{}$, we just remove it (line 4).
%(2) if $(x,y)$ exists in $\Rn$ with a different presence condition $\tPC{}'$, then we only need to remove $(x,y)$ for the intersection of the set of products denoted by $\tPC{}$ and $\tPC{}'$ (line 5).
%(3) if $(x,y)$ does not exist in $\Rn$ at all, we do not remove anything (line 8).

Similarly, \lift{\trace} is the lifted version of \trace. The returned lifted set is a function mapping a GSN model element \code{g}~to the set of configurations from which there exists a system model element \code{s}~in the input lifted set of system elements, where \code{(s,g)} belongs to the input traceability map.

The lifted version of \createAnnotation~(named \lift{\createAnnotation}) is of exactly the same structure as the original because it strictly uses set operations (union, set difference and image), which have been all lifted as a part of the underlying variability-aware set implementation (Listing~\ref{lst:var}).
%The algorithm starts with an empty set of GSN elements (line 1), then iterates through all the elements of the traceability relation (lines 2-7). For each traceability link $(x,y)$ with presence condition $\tPC{}$, if $x$ exists in the system model parameter $\S$ with presence condition $\tPC{}'$ that isn't \ff, and if the conjunction of $\tPC{}$ and $\tPC{}'$ is satisfiable, we add $(y, \tPC{} \land \tPC{}')$ to $A$. Here the conjunction of the two presence conditions denotes the intersection of the sets of products denoted by the traceability link presence condition, and the GSN element presence condition.

\lstinputlisting[style=lean,basicstyle=\small,firstline=46,lastline=51,caption=Lifted implementation of \code{restrict} and \code{trace}.,label=lst:liftedHelpers]{liftedCIA.lean}
%\vspace{0.3in}

%\vspace{-0.4in}
%\begin{algorithm}[t]
%	\SetKwInOut{Input}{Inputs}\SetKwInOut{Output}{Output}
%	\SetKwData{S}{S}
%	\SetKwData{Sys}{\lift{Sys}}
%	\SetKwData{R}{R}
%	\SetKwData{TR}{\lift{TraceRel}}
%	\SetKwData{A}{A}
%	\SetKwData{GSN}{\lift{GSN}}
%	\SetKwData{X}{x}
%	\SetKwData{XPC}{xpc}
%	\SetKwData{Y}{y}
%	\SetKwData{YPC}{ypc}
%	\SetKwData{XP}{(\X,\XPC)}
%	\SetKwData{YP}{(\Y,\YPC)}
%	\SetKwData{PC}{pc}
%	\SetKwData{PCn}{pc'}
%	\SetKwData{E}{e}
%	
%	\SetKwFunction{Find}{Find}
%	
%	\SetAlgoLined
%	\Input{\typerel{\R}{\TR}  \tcp*[f]{Traceability map}         \newline
%		\typerel{\S}{\Sys}   \tcp*[f]{set of model elements with presence conditions}}
%	\Output{\typerel{\A}{\GSN}   \tcp*[f]{set of GSN elements with presence conditions}}
%	
%	\assign{\A}{\{\}}
%	
%	\For{$(\X,\Y,\tPC) \in \R$}{
%		\assign{\tPCn}{\Find{\S,\X}} \;
%		\If{(\tPCn)}{
%			\lIf{$\tPC \wedge \tPCn$}
%			{\assign{\A}{\A \cup \{(\Y, \tPC \wedge \tPCn)\}}}
%		}
%	}
%	
%	\Return{\A}
%	\caption{\lift{Trace}}
%	\label{alg:trace_lifted}
%	%\vspace{-0.15in}
%\end{algorithm}

\lstinputlisting[style=lean,basicstyle=\small,firstline=147,lastline=149,caption=Correctness theorem of \lift{GSN\_IA}.,label=lst:correctness]{liftedCIA.lean}

The correctness theorem of \lift{GSN\_IA} with respect to GSN\_IA is in Listing~\ref{lst:correctness}. It is a direct instantiation of the general correctness criteria in Fig.~\ref{fig:correctness1}, applied to inputs of the GSN\_IA algorithm.

\section{Case Study and Evaluation}
\label{casestudy}
\begin{table*}[tbp]
  \centering
  \begin{tabular}{r|ccc}
    version & \texttt{join} & \texttt{vjoin} & \texttt{udot}, \texttt{sortVector}, \texttt{roundVector}\\ \hline
    $<$ 0.15     & available  & undefined & undefined\\
    $\geq$ 0.16  & deleted & available & available\\
  \end{tabular}
  \caption{Availability of functions in hmatrix before and after tha update.}
  \label{table:join}
\end{table*}
\subsection{Case Study}
We demonstrate that \mylang{} programming achieves the two benefits of programming with versions. 
The case study simulated the incompatibility of hmatrix,\footnote{\url{https://github.com/haskell-numerics/hmatrix/blob/master/packages/base/CHANGELOG}} a popular Haskell library for numeric linear algebra and matrix computations, in the VL module \mn{Matrix}.
This simulation involved updating the applications \mn{Main} depending on \mn{Matrix} to reflect incompatible changes.

Table \ref{table:join} shows the changes introduced in version 0.16 of hmatrix. Before version 0.15, hmatrix provided a \texttt{join} function for concatenating multiple vectors.
The update from version 0.15 to 0.16 replaced \texttt{join} with \texttt{vjoin}. Moreover, several new functions were introduced.
We implement two versions of \mn{Matrix} to simulate backward incompatible changes in \mylang{}.
Also, due to the absence of user-defined types in \mylang{}, we represent \texttt{Vector a} and \texttt{Matrix a} as \texttt{List Int} and \texttt{List (List Int)} respectively, using \mn{List}, a partial port of \texttt{Data.List} from the Haskell standard library.

\begin{figure}[t]

\begin{minipage}{.5\textwidth}
% \begin{lstlisting}[style=haskell]
\begin{minted}{haskell}
module Main where
import Matrix
import List
main = let
  vec = [2, 1]
  sorted = sortVector vec
  m22 = join -- [[1,2],[2,1]]
          (singleton sorted)
          (singleton vec)
  in determinant m22
-- error: version inconsistent
\end{minted}
% \end{lstlisting}
\end{minipage}
\begin{minipage}{.5\textwidth}
\begin{minted}{haskell}
module Main where
import Matrix
import List
main = let
  vec = [2, 1]
  sorted = @\vlkey{unversion}@
             (sortVector vec)
  m22 = join -- [[1,2],[2,1]]
          (singleton sorted)
          (singleton vec)
  in determinant m22 -- ->* -3
\end{minted}
\end{minipage}
\caption{Snippets of \texttt{Main} before (left) and after (right) rewriting.}
\label{fig6-5}

\end{figure}
We implement \mn{Main} working with two conflicting versions of \mn{Matrix}. The left side of Figure \ref{fig6-5} shows a snippet of \mn{Main} in the process of updating \mn{Matrix} from version 0.15.0 to 0.16.0. \fn{main} uses functions from both versions of \mn{Matrix} together: \fn{join} and \fn{sortVector} are available only in version 0.15.0 and 0.16.0 respectively, hence \mn{Main} has conflicting dependencies on both versions of \mn{Matrix}. Therefore, it will be impossible to successfully build this program in existing languages unless the developer gives up using either \fn{join} or \fn{sortVector}.

\begin{itemize}
\item \textbf{Detecting Inconsistent Version}:
\mylang{} can accept \mn{Main} in two stages. First, the compiler flags a version inconsistency error.
It is unclear which \mn{Matrix} version the \fn{main} function depends on as \fn{join} requires version 0.15.0 while \fn{sortVector} requires version 0.16.0.
The error prevents using such incompatible version combinations, which are not allowed in a single expression.

\item \textbf{Simultaneous Use of Multiple Versions}:
In this case, using \fn{join} and \fn{sortVector} simultaneously is acceptable, as their return values are vectors and matrices. Therefore, we apply \texttt{\vlkey{unversion} t} for $t$ to collaborate with other versions.
The right side of Figure \ref{fig6-5} shows a rewritten snippet of \mn{Main}, where \texttt{sortVector vec} is replaced by \texttt{\vlkey{unversion} (sortVector vec)}. Assuming we avoid using programs that depend on a specific version elsewhere in the program, we can successfully compile and execute \fn{main}.
\end{itemize}

\subsection{Scalability of Constraint Resolution}
\begin{figure}[tbp]
    \centering
    \includegraphics[height=6.5cm]{figs/ret.png}
    \caption{Constraint resolution time for the duplicated \mn{List} by \texttt{\#mod} $\times$ \texttt{\#ver}.}
    \label{fig:consres}
\end{figure}

We conducted experiments on the constraint resolution time of the \mylang{} compiler. In the experiment, we duplicated a \mylang{} module, renaming it to \texttt{\#mod} like \mn{List\_i}, and imported each module sequentially. Every module had the same number of versions, denoted as \texttt{\#ver}. Each module version was implemented identically to \mn{List}, with top-level symbols distinguished by the module name, such as \fn{concat\_List\_i}. The experiments were performed ten times on a Ryzen 9 7950X running Ubuntu 22.04, with \texttt{\#mod} and \texttt{\#ver} ranging from 1 to 5.

Figure \ref{fig:consres} shows the average constraint resolution time. 
The data suggests that the resolution time increases polynomially (at least square) for both \texttt{\#mod} and \texttt{\#ver}.
Several issues in the current implementation contribute to this inefficiency:
First, we employ sbv as a z3 interface, generating numerous redundant variables in the SMT-Lib2 script.
For instance, in a code comprising 2600 LOC (with $\texttt{\#mod} =5$ and $\texttt{\#ver} =5$), the \mylang{} compiler produces 6090 version resource variables and the sbv library creates SMT-Lib2 scripts with approximately 210,000 intermediate symbolic variables.
Second, z3 solves versions for all AST nodes, whereas the compiler's main focus should be on external variables and the subterms of \texttt{\vlkey{unversion}}.
Third, the current \mylang{} nests the constraint network, combined with $\lor$, \texttt{\#mod} times at each bundling. This approach results in an overly complex constraint network for standard programs.
Hence, to accelerate constraint solving, we can develop a more efficient constraint compiler for SMT-Lib2 scripts, implement preprocess to reduce constraints, and employ a greedy constraint resolution for each module.






% \section{Limitations of the Current \mylang{}}
% % This section discusses the limitations of the current VL language and possible solutions.

% \subsection{Lack of Support for Structural Incompatibility}
% One of the apparent problems with the current VL system is that it does not support \emph{type incompatibilities}, a key element of structural incompatibilities. We will first analyze the types of incompatibilities and then discuss ways to extend the current VL system.

% % \paragraph{Types of Incompatibilities}
% % \label{sec:typesofincompatibility}
% % Incompatibilities between old and new versions of a package caused by updates can be broadly classified into two categories \emph{structural incompatibilities} and \emph{behavioral incompatibilities}.

% % \paragraph{Structural Incompatibilities}
% % \begin{table*}[tbp]
  \centering
  \begin{tabular}{r|ccc}
    version & \texttt{join} & \texttt{vjoin} & \texttt{udot}, \texttt{sortVector}, \texttt{roundVector}\\ \hline
    $<$ 0.15     & available  & undefined & undefined\\
    $\geq$ 0.16  & deleted & available & available\\
  \end{tabular}
  \caption{Availability of functions in hmatrix before and after tha update.}
  \label{table:join}
\end{table*}
% % A structural incompatibility occurs when multiple versions of a package provide different set of definitions including function names and data structures.
% % Structural incompatibilities are caused by adding and removing definitions, internal changes to data structures, and renaming.
% % Table \ref{table6-1} shows an example of structural incompatibility in GIMP Drawing Kit (GDK).
% % GDK is a C library for creating graphical user interfaces and is used by many projects, including GNOME.

% % If the deprecated functions are not available, version 3.22 is structurally incompatible with version 3.20 because the former lacks \mscreen{} that is available in the latter.
% % GDK versions before 3.22 provide \mscreen{} that tells the number of connected physical monitors.
% % However, versions 3.22 later provide the same functionality function \mdisplay{} and deprecate \mscreen{}.
% % When we upgrade GDK to version 3.22 and build software that uses this function without modifying anything, the build system will give us an undefined reference error.
% % With a static type check, the programmer will be informed of the incompatibility problem as a compilation error.

% % \paragraph{Behavioral Incompatibilities}
% % \input{./figs/table6-2.tex}
% % A behavioral incompatibility is a situation where multiple versions of a package provide the same definition but differ in their behavior.
% % Code changes may also cause behavioral incompatibilities that include additions, removals, and changes in side effects, even if there is no change in name or type.
% % Table \ref{table6-2} shows an example of behavioral incompatibility in the Android Platform API (henceforth Android API).
% % The Android API is the standard library written mostly in Java, and its version synchronizes with Android OS.

% % Before version 19\footnote{The Android API uses \textit{levels} instead of versions as identifiers for API revisions, but we will call them versions for consistency.},
% % the Android API provided the \mset{} method in the \texttt{AlarmManager} class that schedules an alarm at a specified time.
% % However, since version 19, the Android API has changed its behavior for power management.
% % Despite having the same name and type definitions, \mset{} no longer guarantees accurate alarm delivery.
% % For developers who require accurate delivery, the method \msetExact{} is provided instead.

% % \paragraph{Extending \mylang{} to Support Structural Incompatibility}
% The current VL language system forces terms of different versions to have the same type, both on the theoretical (typing rules in \corelang{}) and implementation (bundling in \vlmini{}) aspects.
% In \corelang{}, definitions of the same type can be combined as a versioned record (even if the programmer has given them different names), while terms with different types cannot be in a versioned record. Also, the VL language system will stop compilation if it finds a definition with the same name but a different type in more than one version of the same module.

% However, more feature is needed to deal with broader incompatibilities. Raemaekers et al. conducted a comprehensive analysis of the seven-year history of library releases in Maven Central. They found that about one-third of all releases introduced at least one structural incompatibility change. The top three causes of structural incompatibilities were class, method, or field deletions, and the remaining seven were type changes.~\cite{RAEMAEKERS2017140}
% It seems an important step to extend the language system to support a wider variety of type incompatibilities and to help programmers improve dependencies.

% The current \corelang{} design is motivated by the basic design that "the type of a versioned record is similar to the type \vertype{r}{A}, a type with a resource in coeffect calculli." In the current \corelang{}, the type of versioned record $\{\overline{l=t}\,|\,l_k\}$ is $\vertype{r}{A} (r = \{\overline{l}\})$, and no difference exists between a type of versioned records and promotions of a term of type $A$. This design has the advantage that versioned records and promotions could be treated in a unified manner, making it easier to formalize dynamic and static semantics.

% One useful idea to address this problem is to decouple version inference from the type inference of coeffect calculus and implement a type system that guarantees version consistency on top of the polymorphic record calculus.~\cite{10.1145/218570.218572} The idea stems from the fact that the type $\vertype{\{l_1,l_2\}}{A}$ is structurally similar to the variant type $\langle l_1 : A,\, l_2 : A\rangle$ of $\Lambda^{\forall,\bullet}$. It is no longer required with variants that types be the same, so terms with different types can be stored as a single value, such as $\langle l_1 = true,\, l_2 : 100\rangle : \langle l_1 : Bool,\, l_2 : Int\rangle$. Although the current version inference is uniformly defined with type inference, we believe it is possible to separate its algorithm and implement it in another calculus because the type and version inference in the type system of \vlmini{} is orthogonal to each other. In the current VL system, constraints generated from type inference and constraints generated from version inference are completely independent, and all constraints passed to z3 are version constraints.



% \subsection{Inadequate Version Polymorphism}
% As we attempt to scale VL programming to a realistically sized development, incomplete version polymorphism via duplication described in section \ref{sec:adhocversionpolymorphism} becomes an obstacle. The following examples are VL programs that depend on modules \texttt{A} and \texttt{B} in Figure \ref{fig:smallexample}. Both use functions \texttt{g} and \texttt{h} provided by module \texttt{B} and the variable \texttt{a} provided by module \texttt{A}.

% \input{figs/fig6-8.tex}
% The first problem is the difference between the treatment of local and external variables. The two programs in Figure \ref{fig6-8} illustrate this problem.  The only difference between the two programs is that the program on the left is written to apply functions without local variables, whereas the program on the right binds \texttt{a} to \texttt{a'}. However, the left one succeeds, while the right fails in version inference.

% The reason for this problem is the type inference system assigns the only resource variable to the local variable \texttt{a'}. The applications \texttt{g a'} and \texttt{h a'} generate constraints that require \texttt{a'} to depend on versions 1 and 2 of module \texttt{B}, respectively, but there is no version label that satisfies both. All external variables are given unique names by duplication, but local variables are not. Therefore, the type inference results differ in the two programs in Figure \ref{fig6-8}.

% \input{figs/fig6-9.tex}
% The second problem is that there is only one version on which each version of the top-level symbol can depend. The programs in Figure \ref{fig6-9} illustrate this problem.

% The top program requires \texttt{a} of \texttt{A} versions 2.0.0 and 1.0.0 as arguments of \texttt{g} and \texttt{h}, respectively, whereas the bottom program requires \texttt{A} version 2.0.0 for both arguments. The result of type inference is that the top program has a label that satisfies this requirement, while the bottom program does not.

% The cause of this problem is that the inference system produces a variable dependency on one of the versions of the original top-level symbol. The current VL type inference creates a variable dependency on either version of the source when creating a resource variable with the same constraints as the source of the duplication. In this example, the two copies of \texttt{a}, \texttt{a\_0} (for \texttt{g (version {A=2.0.0} of a))} and \texttt{a\_1} (for \texttt{h (version {A=2.0.0} of a))}, are expected to select either version of \texttt{a}. Furthermore, the generated constraints constrain the selected version of \texttt{a}. In line 7, \texttt{g} requires \texttt{a\_0} to have a dependency on version 1.0.0 of \texttt{B}, and version {A=2.0.0} of \texttt{a\_0} requires that \texttt{a\_0} is equal to the label selected for version 2.0.0 of a, resulting in version 2.0.0 of \texttt{a}. Similarly, line 8 generates a constraint that requires that the label for version 2.0.0 of \texttt{a} must contain version 1.0.0 of \texttt{B}, so no label satisfies both simultaneously.

% It is necessary to introduce full-resource polymorphism in the core calculus instead of duplication to solve this irrational problem,.
% The idea is to store external variables and constraints that behave in a version polymorphic manner in a top-level definition environment and instantiate them with a new resource variable for each symbol occurrence. This kind of resource polymorphism is similar to that already implemented in the Gr language~\cite{Orchard:2019:Granule}. However, unlike Gr, \vlmini{} provides a type inference algorithm that collects constraints on a per-module basis, so we need the well-defined form of the principal type.
% This extension is future work.
\section{Implementation}
\label{sec:impl}
%This section describes our implementation details. %for calculating aDVF.
%The core of the implementations includes a tool and a series of
%techniques to improve the tool usability and configurability.

%\subsection{ARAT: A Tool for aDVF Analysis}
To calculate the aDVF value for a data object, %to model the application resilience,
we develop a tool, named~\textit{ARAT} (standing for \textit{A}pplication-level \textit{R}esilience \textit{A}nalysis \textit{T}ool). 
Figure~\ref{fig:tool_framework} shows the tool framework.
ARAT has three components: an application trace generator, a trace analysis tool, and a deterministic fault injector.


\begin{figure*}
  \begin{center}
  \includegraphics[height=0.13\textheight,keepaspectratio]{flow_chart.pdf}
  \vspace{-5pt}
  \caption{ARAT, a tool for application-level resilience modeling based on the aDVF analysis}
  \label{fig:tool_framework}
  \end{center}
  \vspace{-15pt}
\end{figure*}

The \textbf{application trace generator} is an LLVM instrumentation pass to generate a dynamic LLVM IR trace.
LLVM IR is architecture independent, and each instruction in the IR trace corresponds to one operation.
The trace includes dynamic register values and memory addresses referenced in each operation.
The current trace generator is based on a third-party tool~\cite{ispass13:shao}, but with some extensions 
for the deterministic fault injection %intermediate state recording, 
and Phi instruction processing to identify ambiguous branches.

The \textbf{trace analysis tool} is the core of ARAT. Using an application trace as input, the tool calculates the aDVF value of a given data object.
In particular, the trace analysis tool conducts the operation-level and fault propagation analysis. %counts tier-1 and tier-2 fault masking events.
Also, for those unresolved fault propagation analyses that reach the boundary of the fault propagation analysis,   
%but cannot be determined if the fault masking exists or not,  
the trace analysis tool will output 
%the unresolved instruction information 
a set of fault injection information for the deterministic fault injection. Such information includes dynamic instruction IDs, IDs of the operands that reference the target data object, and the bit locations of the operands that have undetermined fault masking.
%(i.e., the dynamic instruction ID, the ID of the operand that references the target data object, 
%and the bit location of the operand that has undetermined fault masking) for the guided fault injection.  
After the fault injection results (i.e., the existence of algorithm-level fault masking or not) are available from the deterministic fault injector,
we re-run the trace analysis tool, and use the fault injection results to address the unresolved analyses and update the aDVF calculation. 

%In a LLVM IR-based trace, the value of the target data object can be loaded into a regist
For the fault propagation analysis, we associate data semantics (the data object name) with the data values in registers,
such that we can identify the data of the target data object in registers. %in the trace 
%and correlate the LLVM IR-based analysis  with the application-level fault masking analysis.  
%Identifying the data of the target data object in registers 
This is necessary to analyze fault propagation.
To associate data semantics with the data in registers, 
ARAT tracks the register allocation when analyzing the trace, such that we can know at any moment which registers have the data of the target data object. 

For the value shadowing analysis to determine which bits can have their bit flips masked, we ask users to provide a set of value shadowing thresholds, each of which defines
a boundary (either upper bound or lower bound) of valid data values for a data element of the target data object. 
Only those bit positions whose bit flips result in a valid data value
are determined to have the fault masking of value shadowing.
If users cannot provide such thresholds, then we will perform deterministic fault injection test for each bit of the data element of the target data object to determine the effect of bit flip on the application outcome.
%However, this method can be time-consuming because of intensive fault injection.
%We introduce a technique in Section~\ref{sec:acc_analysis} to prune fault injection tests.
To accelerate the value shadowing analysis, we further introduce
a series of optimization techniques, such as (1) using the 
deterministic fault injection results for higher-order bits to deduce
the fault injection results for lower-order bits; (2) leveraging
iterative structure of the application;
and (3) analysis parallelization. 
Those implementation details can be found in our technical report~\cite{resilience_modeling:tr}. 

The trace analysis tool is configurable and extensible. 
It gives the user flexibility to control the trace analysis. %and explore the trade-off between accuracy and analysis speed.
%Table~\ref{tab:arat-config} summarizes the major configuration parameters.
For example, the user can define a maximum fault propagation length for the fault propagation analysis; %to accelerate the analysis;
the user can also configure fault patterns for analysis. 
%and \textbf{set up a value shadowing threshold based on the application information}.
%The trace analysis tool also allows the user to configure fault patterns for analysis.
Based on the user configuration, the tool can enumerate all fault patterns during the analysis or just examine one specific pattern.
%Furthermore, the trace analysis tool supports the identification of the three operation-level fault masking (see Section~\ref{sec:statement_analysis}) 
%and the optimization technique for the fault propagation analysis (see Section~\ref{sec:fault_propagation_analysis}).
To make the trace analysis tool extensible for future improvement,
the tool also exposes APIs that allow users to hook up new
techniques to identify fault masking and optimize analysis.
%Hence, the trace analysis tool is extensible for future development.

The \textbf{deterministic fault injector} is a tool to capture the algorithm level fault masking
%and facilitate the calculation of aDVF 
for the trace analysis tool.
The input to the deterministic fault injector is a list of fault injection points
generated by the trace analysis tool for those unresolved fault masking analyses.
Each fault injection point includes a dynamic instruction ID, %in LLVM IR, 
the ID of the operand that references the target data object, and a specific bit-position of the operand for bit flipping (i.e., the fault injection).
%and the bit location of the operand that has undetermined fault masking
%Hence, the list of fault injection points works as an interface 
%between the guided fault injector and the trace analysis tool. 
The bit-positions of the operand for bit flipping are determined after the value shadowing analysis. %(see Table~\ref{tab:arat-config}).
%Given a m-bit data, if there are n bits ($ 0 \leq n \leq m$) whose bit flips can be masked, then the rest $m-n$ bits will need deterministic fault injection tests.

Similar to the application trace generation, the deterministic fault injector is also based on the LLVM instrumentation. %and dynamic LLVM IR.
We use the LLVM instrumentation to count dynamic instructions and trigger bit flips. 
After the LLVM instrumentation, the application execution will trigger bit flip when a fault injection point is encountered.
%and during the application execution, when a LLVM IR instruction specified in the list of fault injection points is to be executed,
%we trigger a bit flip. 

\begin{comment}
(\textbf{Dong: the following part in this paragraph needs to be refined.})
The bit-locations of the operand for bit flipping are determined by the value shadowing threshold. %(see Table~\ref{tab:arat-config}).
Given a data value of the operand, the product of the data value and the value shadowing threshold
defines a boundary of the valid data value.
%Given a data value in the operand, 
The bit-locations for bit flipping are those that have data values beyond 
the boundary after bit-flipping.
Those bit-locations for fault injection can cause large value deviation from the original data value, 
and we cannot determine fault masking using the statement-level and fault propagation analysis,
and have to rely on the analysis at the algorithm level.

Similar to the application trace generation, the deterministic fault injector is also based on LLVM instrumentation and dynamic LLVM IR.
We use the LLVM instrumentation to count dynamic instruction ID.
During the application execution, when a LLVM IR instruction specified in the list of fault injection points is to be executed,
we trigger a bit flip. 
\end{comment}

%\textbf{How to handle the phi instruction; how to optimize the analysis efficiency}
%\textbf{How to handle value shadowing} how to determine the threshold to determine if a giant value plus a small value is fault masking
%\textbf{A threshold for the fault propagation analysis}


\section{Related Work}\label{sec:related}
 
The authors in \cite{humphreys2007noncontact} showed that it is possible to extract the PPG signal from the video using a complementary metal-oxide semiconductor camera by illuminating a region of tissue using through external light-emitting diodes at dual-wavelength (760nm and 880nm).  Further, the authors of  \cite{verkruysse2008remote} demonstrated that the PPG signal can be estimated by just using ambient light as a source of illumination along with a simple digital camera.  Further in \cite{poh2011advancements}, the PPG waveform was estimated from the videos recorded using a low-cost webcam. The red, green, and blue channels of the images were decomposed into independent sources using independent component analysis. One of the independent sources was selected to estimate PPG and further calculate HR, and HRV. All these works showed the possibility of extracting PPG signals from the videos and proved the similarity of this signal with the one obtained using a contact device. Further, the authors in \cite{10.1109/CVPR.2013.440} showed that heart rate can be extracted from features from the head as well by capturing the subtle head movements that happen due to blood flow.

%
The authors of \cite{kumar2015distanceppg} proposed a methodology that overcomes a challenge in extracting PPG for people with darker skin tones. The challenge due to slight movement and low lighting conditions during recording a video was also addressed. They implemented the method where PPG signal is extracted from different regions of the face and signal from each region is combined using their weighted average making weights different for different people depending on their skin color. 
%

There are other attempts where authors of \cite{6523142,6909939, 7410772, 7412627} have introduced different methodologies to make algorithms for estimating pulse rate robust to illumination variation and motion of the subjects. The paper \cite{6523142} introduces a chrominance-based method to reduce the effect of motion in estimating pulse rate. The authors of \cite{6909939} used a technique in which face tracking and normalized least square adaptive filtering is used to counter the effects of variations due to illumination and subject movement. 
The paper \cite{7410772} resolves the issue of subject movement by choosing the rectangular ROI's on the face relative to the facial landmarks and facial landmarks are tracked in the video using pose-free facial landmark fitting tracker discussed in \cite{yu2016face} followed by the removal of noise due to illumination to extract noise-free PPG signal for estimating pulse rate. 

Recently, the use of machine learning in the prediction of health parameters have gained attention. The paper \cite{osman2015supervised} used a supervised learning methodology to predict the pulse rate from the videos taken from any off-the-shelf camera. Their model showed the possibility of using machine learning methods to estimate the pulse rate. However, our method outperforms their results when the root mean squared error of the predicted pulse rate is compared. The authors in \cite{hsu2017deep} proposed a deep learning methodology to predict the pulse rate from the facial videos. The researchers trained a convolutional neural network (CNN) on the images generated using Short-Time Fourier Transform (STFT) applied on the R, G, \& B channels from the facial region of interests.
The authors of \cite{osman2015supervised, hsu2017deep} only predicted pulse rate, and we extended our work in predicting variance in the pulse rate measurements as well.

All the related work discussed above utilizes filtering and digital signal processing to extract PPG signals from the video which is further used to estimate the PR and PRV.  %
The method proposed in \cite{kumar2015distanceppg} is person dependent since the weights will be different for people with different skin tone. In contrast, we propose a deep learning model to predict the PR which is independent of the person who is being trained. Thus, the model would work even if there is no prior training model built for that individual and hence, making our model robust. 

%
% \vspace{-0.5em}
\section{Conclusion}
% \vspace{-0.5em}
Recent advances in multimodal single-cell technology have enabled the simultaneous profiling of the transcriptome alongside other cellular modalities, leading to an increase in the availability of multimodal single-cell data. In this paper, we present \method{}, a multimodal transformer model for single-cell surface protein abundance from gene expression measurements. We combined the data with prior biological interaction knowledge from the STRING database into a richly connected heterogeneous graph and leveraged the transformer architectures to learn an accurate mapping between gene expression and surface protein abundance. Remarkably, \method{} achieves superior and more stable performance than other baselines on both 2021 and 2022 NeurIPS single-cell datasets.

\noindent\textbf{Future Work.}
% Our work is an extension of the model we implemented in the NeurIPS 2022 competition. 
Our framework of multimodal transformers with the cross-modality heterogeneous graph goes far beyond the specific downstream task of modality prediction, and there are lots of potentials to be further explored. Our graph contains three types of nodes. While the cell embeddings are used for predictions, the remaining protein embeddings and gene embeddings may be further interpreted for other tasks. The similarities between proteins may show data-specific protein-protein relationships, while the attention matrix of the gene transformer may help to identify marker genes of each cell type. Additionally, we may achieve gene interaction prediction using the attention mechanism.
% under adequate regulations. 
% We expect \method{} to be capable of much more than just modality prediction. Note that currently, we fuse information from different transformers with message-passing GNNs. 
To extend more on transformers, a potential next step is implementing cross-attention cross-modalities. Ideally, all three types of nodes, namely genes, proteins, and cells, would be jointly modeled using a large transformer that includes specific regulations for each modality. 

% insight of protein and gene embedding (diff task)

% all in one transformer

% \noindent\textbf{Limitations and future work}
% Despite the noticeable performance improvement by utilizing transformers with the cross-modality heterogeneous graph, there are still bottlenecks in the current settings. To begin with, we noticed that the performance variations of all methods are consistently higher in the ``CITE'' dataset compared to the ``GEX2ADT'' dataset. We hypothesized that the increased variability in ``CITE'' was due to both less number of training samples (43k vs. 66k cells) and a significantly more number of testing samples used (28k vs. 1k cells). One straightforward solution to alleviate the high variation issue is to include more training samples, which is not always possible given the training data availability. Nevertheless, publicly available single-cell datasets have been accumulated over the past decades and are still being collected on an ever-increasing scale. Taking advantage of these large-scale atlases is the key to a more stable and well-performing model, as some of the intra-cell variations could be common across different datasets. For example, reference-based methods are commonly used to identify the cell identity of a single cell, or cell-type compositions of a mixture of cells. (other examples for pretrained, e.g., scbert)


%\noindent\textbf{Future work.}
% Our work is an extension of the model we implemented in the NeurIPS 2022 competition. Now our framework of multimodal transformers with the cross-modality heterogeneous graph goes far beyond the specific downstream task of modality prediction, and there are lots of potentials to be further explored. Our graph contains three types of nodes. while the cell embeddings are used for predictions, the remaining protein embeddings and gene embeddings may be further interpreted for other tasks. The similarities between proteins may show data-specific protein-protein relationships, while the attention matrix of the gene transformer may help to identify marker genes of each cell type. Additionally, we may achieve gene interaction prediction using the attention mechanism under adequate regulations. We expect \method{} to be capable of much more than just modality prediction. Note that currently, we fuse information from different transformers with message-passing GNNs. To extend more on transformers, a potential next step is implementing cross-attention cross-modalities. Ideally, all three types of nodes, namely genes, proteins, and cells, would be jointly modeled using a large transformer that includes specific regulations for each modality. The self-attention within each modality would reconstruct the prior interaction network, while the cross-attention between modalities would be supervised by the data observations. Then, The attention matrix will provide insights into all the internal interactions and cross-relationships. With the linearized transformer, this idea would be both practical and versatile.

% \begin{acks}
% This research is supported by the National Science Foundation (NSF) and Johnson \& Johnson.
% \end{acks}

\bibliographystyle{splncs04}
\bibliography{spl,modeling,pl,safety}
\end{document}