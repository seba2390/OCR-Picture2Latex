% system model elements
\newcommand{\alarm}{Alarm}
\newcommand{\UI}{UserInterface}
\newcommand{\LDWS}{LaneDepartureWarningSystem}
\newcommand{\audio}{Audio}
\newcommand{\visual}{Visual}

\newcommand{\CtRecheck}{C2recheck}
\newcommand{\CtRevise}{C2revise}
\newcommand{\CrRechecko}{C3recheck1}
\newcommand{\CtRecheckt}{C3recheck2}

% AC elements
\newcommand{\Gfive}{G5}
\newcommand{\Snthree}{Sn3}
\newcommand{\Seight}{S8}
\newcommand{\Geighteen}{G18}
\newcommand{\Gnineteen}{G19}
\newcommand{\Gtwenty}{G20}
\newcommand{\Snfour}{Sn4}
\newcommand{\Sneleven}{Sn11}
\newcommand{\Sntwelve}{Sn12}
\newcommand{\Sneighteen}{Sn18}

% features
\newcommand{\Fldws}{LDWS}
\newcommand{\Faudio}{Audio}
\newcommand{\Fvisual}{Visual}
\newcommand{\Faudiovis}{\Fvisual~$\vee$ \Faudio}

\vspace{-0.2in}
\subsection{Examples}
\vspace{-0.15in}
%\label{sec:examples}
%\noindent
%{\bf Examples.}
%\begin{figure*}[t]
%	\centering
%	\includegraphics[width=\textwidth]{traceability}
%	\caption{Traceability from UML classes to GSN elements for a fragment of the LKA SPL.}
%	\label{fig:traceability}	
%	\vspace{-0.2in}
%\end{figure*}

\begin{figure*}[t]
\begin{framed}
\begin{subfigure}[c]{\textwidth}
	\vspace{-0.2in}
	\small
	\centering
	\[
	\begin{array}{lcl}
	A & = & \{ (\Gfive, \PC{TT}), (\Snthree, \PC{TT}), (\Seight, \PC{TT}), (\Geighteen, \PC{\Faudio}), (\Gnineteen, \PC{\Fvisual}), 
	(\Gtwenty, \PC{\TT}), \\
	  &   &   (\Snfour, \PC{TT}), (\Sneighteen, \PC{TT}), (\Sntwelve, \PC{\Faudio}), (\Sneleven, \PC{\Fvisual}) \} \\
	
	R & = & \{ (\visual, \Gnineteen, \PC{\Fvisual}), (\visual, \Gtwenty, \PC{\Fvisual}), (\audio, \Geighteen, \PC{\Faudio}), \\
	  &   &   (\audio, \Gtwenty, \PC{\Faudio}) \}
	\end{array}
	\]
	\vspace{-0.2in}
	\caption{Assurance case elements (A) and traceability links (R) used in Ex1 and Ex2.}
	\label{fig:common}
\end{subfigure}

\vfill

\begin{subfigure}[c]{\textwidth}
	\small
	\centering
	\[
	\begin{array}{lcl}
	S & = & \{ (\alarm, \PC{\TT}), (\UI, \PC{\TT)}, (\audio, \PC{\Faudio}), (\visual, \PC{\Fvisual}), \\
	  &   &     (\LDWS, \PC{\Fldws}) \} \\

	S' & = & \{ (\alarm, \PC{\TT}), (\UI, \PC{\TT)}, (\audio, \PC{\Faudio}), (\visual', \PC{\Fvisual}), \\ 
	   &   &    (\LDWS, \PC{\Fldws})\} \\

	D & = & \langle \{\}, \{\}, \{(\visual, \PC{\Fvisual)}\} \rangle
	\end{array}
	\]
	\vspace{-0.2in}
	\caption{System model (S), modified system model (S'), and delta (D) used in Ex1.}
	\label{fig:example1}
\end{subfigure}

\vfill

\begin{subfigure}[c]{\textwidth}
	\small
	\centering
	\[
	\begin{array}{lcl}
	S & = & \{ (\alarm, \PC{\TT}), (\UI, \PC{\TT)}, (\audio, \PC{\Faudio}), (\visual, \PC{\Fvisual}), \\
	  &   &    (\LDWS, \PC{\Fldws})\} \\
	
	S' & = & \{ (\alarm', \PC{\TT}), (\UI, \PC{\TT)}, (\audio, \PC{\Faudio}), (\visual, \PC{\Fvisual}), \\ 
	   &   &   (\LDWS, \PC{\Fldws})\} \\
	
	D & = & \langle \{\}, \{\}, \{(\alarm, \PC{\TT})\} \rangle
	\end{array}
	\]
	\vspace{-0.2in}
	\caption{System model (S), modified system model (S'), and delta (D) used in Ex2.} 
	\vspace{-0.1in}
	\label{fig:example2}
\end{subfigure}
\end{framed} % fbox
\vspace{-0.2in}
\caption{Inputs to the \lift{GSN-IA} algorithm used in Ex1 and Ex2.}
\label{fig:inputs}
\vspace{-0.2in}
\end{figure*}

In this section, we apply our lifted CIA algorithm to two examples of modifications to the fragment of the LMS product line presented in Sec.~\ref{sec:intro} (Fig.~\ref{fig:ex}). 
%We present two examples of modifications of the fragment of the system model and its corresponding assurance case from Sec.~\ref{sec:intro} (Fig.~\ref{fig:ex}).

\vspace{0.1in}
\noindent
{\bf Ex1:  Feature-Specific Modification.}
Suppose  that the \term{\visual} class is modified. This class is local to the \PC{\Fvisual} feature. If we only analyze the fragment in Fig.~\ref{fig:ex}, the inputs to \lift{GSN-IA} are shown in \fig{common} and \fig{example1}.

%\todo{Define backward slicer}

Tracing through the algorithm, the first step is using \lift{\restrict} to calculate \Rn = \{(\visual, \Gnineteen, \PC{\Fvisual}), (\visual, \Gtwenty, \PC{\Fvisual})\} (line 8). Because \Cza~and \Czd~are both empty, and assuming a backward slicer (returning the transitive closure of the elements that might affect the slicing criteria), \Codm~and \Coam~both become \{(\alarm, \PC{\TT}), (\visual, \PC{\Fvisual}), (\LDWS, \PC{\Fldws})\} (lines 9-10). Now tracing from \Codm~and \Coam, \CtRecheck~becomes \{(\Gnineteen, \PC{\Fvisual}), (\Gtwenty, \PC{\Fvisual})\} (line 11).
Since \Czd~is empty, \CtRevise~and \CrRechecko~ are both empty as well (lines 12-13).  Using a backward GSN slicer, 
\CtRecheckt~becomes \{(\Gnineteen, \PC{\Fvisual}), (\Gtwenty, \PC{\Fvisual}), (\Sneleven, \PC{\Fvisual}), (\Snfour, \PC{\Fvisual}), (\Sneighteen, \PC{\Fvisual})\} (line 14). The algorithm returns an empty set of GSN elements to be revised, and the set \CtRecheckt~to be rechecked. Note that \Gtwenty, \Snfour, and \Sneighteen~are all base model elements (having \PC{\TT} as a presence condition), so the algorithm output states that we need to recheck those elements only in products where the feature \PC{\Fvisual}~is present.
 
 \vspace{0.1in}
 \noindent
 {\bf Ex2:  Base System Modification.}
\label{sec:example2}
Suppose that the \term{\alarm} class is modified. This is a base system class, i.e., it is present in all products. The inputs to \lift{GSN-IA} (restricted to the fragment in Fig.~\ref{fig:ex}) are shown in \fig{common} and \fig{example2}.

Since the \alarm~class does not have any direct traceability links, $\Rn$ is empty (line 8). Using a backward slicer (like in Ex1), \Codm~and \Coam~both become \{(\alarm, \PC{\TT}), (\visual, \PC{\Fvisual}), (\audio, \PC{\Faudio}), (\LDWS, \PC{\Fldws})\} (lines 9-10). 
From \Codm~and \Coam~using the traceability links, \CtRecheck~becomes \{(\Geighteen, \PC{\Faudio}), (\Gnineteen, \PC{\Fvisual}), (\Gtwenty, \PC{\Fvisual})\} (line 11).
Again, since \Czd~is empty, \CtRevise~and \CrRechecko~ are both empty as well (lines 12-13). With a backward GSN slicer, \CtRecheckt~becomes \{(\Geighteen, \PC{\Faudio}), (\Gnineteen, \PC{\Fvisual}), (\Gtwenty, \PC{\Faudiovis}), (\Sneleven, \PC{\Fvisual}), (\Sntwelve, \PC{\Faudio}), (\Snfour, \PC{\Faudiovis}), (\Sneighteen, \PC{\Faudiovis})\} (line 14). 
The algorithm returns an empty set of GSN elements to be revised, and the set \CtRecheckt~to be rechecked. 
Note that in this example, \Gtwenty, \Snfour, and \Sneighteen~are annotated with \name{recheck} with presence condition \PC{\Faudiovis}, which means that they need to be rechecked only if either \PC{\Faudio} or \PC{\Fvisual} are present. 
