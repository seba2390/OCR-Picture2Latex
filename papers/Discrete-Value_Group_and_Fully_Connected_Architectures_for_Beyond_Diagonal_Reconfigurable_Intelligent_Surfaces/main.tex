%\documentclass[onecolumn,12pt,draftcls]{IEEEtran}
\documentclass[twocolumn,10pt]{IEEEtran}

\usepackage[T1]{fontenc}
\usepackage[latin9]{inputenc}
\usepackage{amsmath}
\usepackage{amssymb}
\usepackage{amsthm}
%\usepackage{array}
%\usepackage{mathrsfs}
%\usepackage{multirow}
%\usepackage{amsthm}
\usepackage{graphicx}
\usepackage{comment}
\usepackage{booktabs}
\usepackage[outdir=./]{epstopdf}
\usepackage[ruled,linesnumbered]{algorithm2e}
\usepackage[acronym]{glossaries}
%\usepackage{setspace}
\glsdisablehyper
\loadglsentries{glossary}
\newtheorem{proposition}{Proposition}
\usepackage{xcolor}

\let\oldnl\nl% Store \nl in \oldnl
\newcommand{\nonl}{\renewcommand{\nl}{\let\nl\oldnl}}% Remove line number for one line

\DeclareMathOperator*{\argmin}{arg\,min}

\begin{document}

%\title{Reconfigurable Intelligent Surfaces Based on Single, Group, and Fully Connected Discrete-Value Impedance Networks}
%\title{Discrete-Value Single, Group, and Fully Connected Reconfigurable Intelligent Surfaces}
%\title{Discrete-Value Group and Fully Connected Reconfigurable Intelligent Surfaces}
%\title{Beyond Diagonal Reconfigurable Intelligent Surfaces: Discrete-Value Group and Fully Connected Architectures}
\title{Discrete-Value Group and Fully Connected Architectures for Beyond Diagonal Reconfigurable Intelligent Surfaces}

\author{Matteo Nerini,~\IEEEmembership{Graduate Student Member,~IEEE},
        Shanpu Shen,~\IEEEmembership{Senior Member,~IEEE},\\
        Bruno Clerckx,~\IEEEmembership{Fellow,~IEEE}
%\author{Matteo Nerini,~\IEEEmembership{Member,~IEEE,}
%        John~Doe,~\IEEEmembership{Fellow,~OSA,}
%        and~Jane~Doe,~\IEEEmembership{Life~Fellow,~IEEE}% <-this % stops a space

\thanks{Copyright (c) 2015 IEEE.
Personal use of this material is permitted.
However, permission to use this material for any other purposes must be obtained from the IEEE by sending a request to pubs-permissions@ieee.org.
This work was supported by Hong Kong Research Grants Council through the Collaborative Research Fund under Grant C6012-20G.
\textit{(Corresponding author: Shanpu Shen.)}}% <-this % stops a space
\thanks{M. Nerini is with the Department of Electrical and Electronic Engineering, Imperial College London, London SW7 2AZ, U.K. (e-mail: m.nerini20@imperial.ac.uk).}% <-this % stops a space
\thanks{S. Shen is with the Department of Electronic and Computer Engineering, The Hong Kong University of Science and Technology, Clear Water Bay, Kowloon, Hong Kong (e-mail: sshenaa@connect.ust.hk).}
\thanks{B. Clerckx is with the Department of Electrical and Electronic Engineering, Imperial College London, London SW7 2AZ, U.K., and with Silicon Austria Labs (SAL), Graz A-8010, Austria (e-mail: b.clerckx@imperial.ac.uk).}}% <-this % stops a space
%\thanks{Manuscript received April 19, 2005; revised August 26, 2015.}}



% The paper headers
%\markboth{Journal of \LaTeX\ Class Files,~Vol.~14, No.~8, August~2015}%
%{Shell \MakeLowercase{\textit{et al.}}: Bare Demo of IEEEtran.cls for IEEE Journals}

\maketitle

\begin{abstract}
% Background
Reconfigurable intelligent surfaces (RISs) allow controlling the propagation environment in wireless networks through reconfigurable elements.
Recently, beyond diagonal RISs (BD-RISs) have been proposed as novel RIS architectures whose scattering matrix is not limited to being diagonal.
% Problem
However, BD-RISs have been studied assuming continuous-value scattering matrices, which are hard to implement in practice.
% What the paper does
In this paper, we address this problem by proposing two solutions to realize discrete-value group and fully connected RISs.
First, we propose scalar-discrete RISs, in which each entry of the RIS impedance matrix is independently discretized.
Second, we propose vector-discrete RISs, where the entries in each group of the RIS impedance matrix are jointly discretized.
In both solutions, the codebook is designed offline such as to minimize the distortion caused in the RIS impedance matrix by the discretization operation.
% Results
Numerical results show that vector-discrete RISs achieve higher performance than scalar-discrete RISs at the cost of increased optimization complexity.
Furthermore, fewer resolution bits per impedance are necessary to achieve the performance upper bound as the group size of the group connected architecture increases.
In particular, only a single resolution bit is sufficient in fully connected RISs to approximately achieve the performance upper bound.
\end{abstract}

\glsresetall

\begin{IEEEkeywords}
Beyond diagonal reconfigurable intelligent surface (BD-RIS), codebook design, discrete-value, group connected, fully connected.
\end{IEEEkeywords}

%%%%%%%%%%%%%%%%%%%%%%%%%%%%%%%%%%%%%%%%%%%%%%%%%%
\section{Introduction}

% General intro on RIS and their benefits
Reconfigurable intelligent surfaces (RISs), or intelligent reflecting surfaces, are an emerging technology that will enhance the performance of future wireless communications \cite{bas19}, \cite{wu19a}, \cite{liu21}.
This technology relies on large planar surfaces comprising multiple reflecting elements, each of them able to induce a certain amplitude and phase change to the incident electromagnetic wave.
Thus, an RIS can steer the reflected signal toward the intended direction by smartly coordinating the reflection coefficients of its elements.
RIS-aided communication systems benefit from several advantages.
RISs with passive elements are characterized by ultra-low power consumption and do not cause any active additive thermal noise or self-interference phenomena.
In addition, RIS is a low-profile and cost-effective solution, since it does not include \gls{rf} chains.
%Firstly, the received signal power is increased by means of the reconfigurable propagation environment created by the RIS.
%Secondly, less inter-user interference is experienced in multi-user systems and, consequently, a larger capacity region is achieved.
%Finally, the outage probability is decreased since RISs can better serve users in dead zones and at the cell edge.
%Beyond these conceptual benefits, RISs are a cost-effective solution, since they are composed solely of passive reflective elements, with no expensive \gls{rf} chains.

% Importance of discrete optimization
To avoid difficult optimization problems, many studies on RISs do not pose limitations on the allowed reflection coefficient values.
However, in practical implementation, they are selected from a finite number of discrete values.
Reflection coefficients tunable with finer resolution require a more complex hardware design, which can be prohibitive when the number of RIS elements is high \cite{wu21}.
%To enable the practical development of RISs, research on RISs with discrete amplitude and phase shifts is crucial.
Considering a single-user RIS-aided \gls{siso} system, the effects of discrete phase shifts have been investigated on the diversity order \cite{bad19}, \cite{xu21}, on the achievable rate \cite{zha20b}, and on the ergodic capacity \cite{li20}.
%Furthermore, the channel estimation strategies proposed for ideal RISs with continuous phase shifts are inapplicable to practical RISs with discrete phase shifts \cite{you20b}.
Furthermore, in \cite{liu20,an21,wei21,wan21},
channel estimation strategies have been proposed for RIS-aided systems.
In \cite{an22}, a codebook-based framework is studied to strike flexible trade-offs between communication performance and signaling overhead.
%In \cite{yan21}, the channel estimation problem is addressed with a focus on wideband communications, considering frequency-dependent discrete reflection coefficients.
%
% Related work
%Several methods to design RISs with discrete reflection coefficients have been recently presented with different objectives.
Discrete reflection coefficients have been optimized to enhance the performance of \gls{mimo} communications by solving rate maximization problems in single-user systems \cite{xu19}, \cite{you20}, \cite{abe20}, \cite{qi20} and sum-rate maximization problems in multi-user systems \cite{guo19}, \cite{di20}, \cite{mu20}, \cite{jun21}, \cite{zha21}, \cite{zha21b}.
%In \cite{abe20}, besides the discrete reflection coefficients assumption, a practical model capturing the phase-dependent amplitude variation in the reflection coefficients is considered.
%In \cite{zha21}, only statistical \gls{csi} is assumed to design the discrete RIS reflection coefficients; while in \cite{zha21b}, imperfect noisy \gls{csi} is assumed.
%RISs with discrete reflection coefficients have been applied also with the objective of enhancing resource utilization.
In \cite{wu19b}, \cite{fu21}, the transmit power is minimized by jointly optimizing the continuous transmit precoding and the discrete phase shifts in the RIS.
%, subject to a given set of minimum \gls{sinr} constraints at the receivers.
For RIS-aided downlink communications, works have been conducted to maximize energy efficiency \cite{hua18}, \cite{hua19}, and the spectral efficiency in the presence of statistical \gls{csi} \cite{han20}.
%In RIS-aided communications, \gls{oma} and \gls{noma} schemes have been compared in terms of minimum transmit power required with given user rates in \cite{zhe20}.
%With a focus on RIS-aided \gls{noma} systems, the minimum user rate maximization problem has been considered in \cite{yan20}, and the outage probability minimization problem has been studied in \cite{din20}.
%In \cite{abd20}, the joint usage of a decode-forward relay together with a discrete phase shift RIS has been studied to significantly improve the achievable rate.
In addition, RISs with discrete reflection coefficients have been investigated for Terahertz (THz) communications \cite{ma20a}, \cite{ma20b}, \cite{che19}, \cite{lu20}, wideband communications \cite{cai20}, \gls{swipt} \cite{wu20}, \cite{zha20}, \cite{gon21} and index modulation \cite{liu22}.
Finally, prototypes of discrete phase shift RISs have been designed in \cite{dai20}, \cite{dun20}.
%with 2-bit resolution

\begin{figure*}[t]
    \centering
    \includegraphics[width=0.9\textwidth]{figs_v7/RIS-tree-v2-discrete-v2.eps}
    \caption{RIS classification tree.}
    \label{fig:ris-tree}
\end{figure*}

% A more general RIS model has been recently proposed, but lack of discrete optimization for the general model
In the aforementioned literature \cite{bad19}-\cite{dun20}, it is always assumed that each RIS element is independently controlled by a tunable impedance connected to ground.
%\cite{yu21}.
%Recently, a more general RIS architecture has been proposed in \cite{she20}.
%More precisely, the authors have generalized the widely adopted single connected architecture by connecting all or a subset of RIS elements through a reconfigurable impedance network, resulting in the fully connected and group connected architecture, respectively.
As a result, conventional RISs are characterized by a diagonal scattering matrix, also known as phase shift matrix.
This traditional RIS architecture is denoted as single connected \cite{she20}.
Recently, beyond diagonal RISs (BD-RISs) have been proposed as a novel branch of RISs in which the scattering matrix is not limited to be diagonal \cite{she20}.
In \cite{she20}, the authors generalized the single connected architecture by connecting all or a subset of RIS elements through a reconfigurable impedance network, resulting in the fully and group-connected architecture, respectively.
In \cite{xu21b}, the concept of \gls{star-ris} has been introduced.
This RIS architecture is able to reflect and transmit the impinging signal, differently from conventional RISs working only in reflective mode.
In \cite{li22-1}, a general RIS model has been proposed to unify different modes (reflective/transmissive/hybrid) and different architectures (single/group/fully connected).
The authors also propose different inter-cell architectures, namely cell-wise group/fully connected RIS architectures, where part of/all the cells are connected to each other \cite{li22-1}.
In \cite{li22-2}, BD-RISs supporting multi-sector mode have been proposed to achieve full-space coverage.
In multi-sector BD-RISs, the antennas are divided into multiple sectors, with each sector covering a narrow region of space.
Multi-sector BD-RISs have important gains over \gls{star-ris} due to the highly directional beam of each sector.
In \cite{li22-3}, dynamically group connected BD-RIS are optimized based on a dynamic grouping strategy.
In addition, in \cite{li22}, an RIS architecture with non-diagonal phase shift matrix is proposed, able to achieve a higher rate than conventional single connected RISs.

% Our paper
%Because of the additional degrees of freedom provided by the more complex architecture, the fully connected architecture enables the best performance gain with respect to all other RIS models proposed to date \cite{she20}.
The group and fully connected RISs proposed in \cite{she20} have been studied assuming continuous-value scattering matrices, and it is not clear how to design these architectures with discrete-value scattering matrices.
%Continuous scattering matrices can be optimized with methods based on the partial derivatives of the objective function, or their approximations \cite{she20,li22-1}.
%However, the optimization of discrete scattering matrices poses two subproblems: firstly, the construction of a discrete codebook; secondly, the search of the optimal discrete values within the codebook entries.
Discrete-value single connected RISs have been typically realized by considering uniform discretization of the phase shifts in the interval $[0,2\pi)$.
This is made possible by the diagonal structure of the scattering matrix of these RISs, which is completely described by its phase shifts.
Thus, this discretization design cannot be applied to group and fully connected RISs, whose scattering matrices are not diagonal.
%In addition, in the case of group connected architectures, how to group together the RIS elements is still an open design challenge.
This gap is addressed in this paper, where we propose a design strategy for discrete-value group and fully connected RISs.
%To address this problem, we generalize the well-established approach valid for the single connected architecture to the group and fully connected ones.
%Furthermore, in the case of group connected architectures, we investigate how to group the RIS elements to maximize the performance with reduced hardware complexity.
In Fig.~\ref{fig:ris-tree}, we classify the studies on RIS to better contextualize our work.
The contributions of this paper are summarized as follows.

\begin{comment}
\begin{table}[t]
\renewcommand{\arraystretch}{0.8}
\centering
\caption{Taxonomy of studies on RIS.}
\begin{tabular}{@{}cccc@{}}
\toprule
&
Single connected RISs&
Group connected RISs&
Fully connected RISs\\
\midrule
\begin{tabular}{@{}c@{}}Continuous\\reflection\\coefficients\end{tabular}&
\begin{tabular}{@{}c@{}}Traditional literature:\\The RIS has continuous\\phase shifts in $\left[0,2\pi\right)$.\end{tabular}&
\begin{tabular}{@{}c@{}}\cite{she20}:\\The RIS has continuous\\reactance matrix entries in $\mathbb{R}$.\end{tabular}&
\begin{tabular}{@{}c@{}}\cite{she20}:\\The RIS has continuous\\reactance matrix entries in $\mathbb{R}$.\end{tabular}\\
\midrule
\begin{tabular}{@{}c@{}}Discrete\\reflection\\coefficients\end{tabular}&
\begin{tabular}{@{}c@{}}Traditional literature:\\The RIS has discrete\\phase shifts in\\$\left\{ 0,\frac{2\pi}{2^B},\ldots,\left( 2^{B} - 1 \right)\frac{2\pi}{2^B}\right\}$.\end{tabular}&
\begin{tabular}{@{}c@{}}This work:\\The RIS has discrete\\reactance matrix entries.\end{tabular}&
\begin{tabular}{@{}c@{}}This work:\\The RIS has discrete\\reactance matrix entries.\end{tabular}\\
\bottomrule
\end{tabular}
\label{tab:taxonomy}
\end{table}
\end{comment}

% 1. Proposal of a design strategy to define the quantization steps and optimize in the discrete domain. Scalar-discrete
\textit{First}, we propose a strategy to design discrete-value group and fully connected RISs that assigns a finite number of bits to each entry of the RIS impedance matrix.
The resulting RISs are denoted as scalar-discrete RISs, since they are obtained with a scalar quantization approach.
%The discretization of BD-RISs poses two problems.
Firstly, a suitable codebook is designed in the offline learning stage.
Secondly, the discrete elements of the impedance matrix are optimized based on the resulting codebook in the online deployment stage.
%We design the codebook such as to minimize the distortion caused in the impedance matrix by the discretization operation.
%The codebook design problem has never been considered before since for single connected RISs the codebook can be trivially constructed by uniformly discretizing the phase shifts.

% 2. Scaling law
\textit{Second}, we prove that our codebook always achieves a received signal power growth of $\mathcal{O}(N_I^2)$ in group connected architectures, where $N_I$ is the number of RIS elements.
%This holds even assigning one resolution bit to each impedance composing the impedance network.
Thus, the proposed discretization strategy does not degrade the growth of the received signal power as a function of $N_I$.

% 1. Proposal of a design strategy to define the quantization steps and optimize in the discrete domain. Vector-discrete
\textit{Third}, we develop a different discretization strategy for group and fully connected RISs, exploiting the block diagonal structure of their impedance matrices.
According to this strategy, we jointly discretize and optimize the entries in each block of the RIS impedance matrix through vector quantization.
RISs designed with this strategy, denoted as vector-discrete RISs, achieve higher performance than scalar-discrete RISs at the cost of increased optimization computational complexity.
Vector discretization has never been considered for conventional RISs since their impedance matrix is diagonal.
%Thus, in this case vector discretization boils down to scalar discretization.

% 3. Comparison between the single, group and fully connected in the discrete domain + Theoretical reason for better performance (4 bits necessary in SC, 1 bit in FC)
\textit{Fourth}, we assess the performance in terms of received signal power of single, group, and fully connected RISs employing different numbers of resolution bits.
%The performance of discrete-value RISs is compared with the performance of optimized continuous-value RISs.
%Two scenarios including independent and identically distributed (i.i.d.) Rayleigh fading and correlated fading channels have been considered in our comparison.
%For each RIS architecture, we analyze how many resolution bits are necessary to achieve the performance of an RIS optimized with no resolution constraints.
%Besides the simulation results, we provide theoretical reasons to justify the obtained performance.
We verify that the more connections are present in the reconfigurable impedance network, the less resolution is needed to achieve the optimal performance of continuous-value RISs.
Because of this property, in the fully connected architecture, a single resolution bit allocated to each reconfigurable impedance is sufficient to achieve the performance of ideal RISs.
%In contrast, in the single connected architecture, four resolution bits are needed to reach the upper bound obtained by the RIS with unconstrained reconfigurable impedance network.
%Furthermore, the received signal power obtained with fully connected RISs can nearly reach the proposed performance upper bound.
%Thus, once the set of possible discrete impedance values has been properly defined, a discrete optimization method can be used to achieve the same performance of an RIS with unconstrained impedances.

% Organization
\textit{Organization}: In Section~\ref{sec:system-model}, we introduce the system model and the problem formulation.
In Sections~\ref{sec:scalar-discrete} and \ref{sec:vector-discrete}, we present our novel discrete group and fully connected RIS design based on scalar and vector quantization, respectively.
In Section~\ref{sec:results}, we assess the performance of \gls{mimo} systems aided by discrete RISs in terms of received signal power.
Finally, Section~\ref{sec:conclusion} contains the concluding remarks.

% Notation
\textit{Notation}: Vectors and matrices are denoted with bold lower and bold upper letters, respectively.
Scalars are represented with letters not in bold font.
$\left|a\right|$, and $\arg\left(a\right)$ refer to the modulus and phase of a complex scalar $a$, respectively.
$\left[\mathbf{a}\right]_{i}$ and $\left\|\mathbf{a}\right\|$ refer to the $i$th element and $l_{2}$-norm of vector $\mathbf{a}$, respectively.
$\mathbf{A}^{T}$, $\mathbf{A}^{H}$, $\left[\mathbf{A}\right]_{i,j}$, and $\left\|\mathbf{A}\right\|$ refer to the transpose, conjugate transpose, $\left(i,j\right)$th element, and spectral norm of matrix $\mathbf{A}$, respectively.
$\mathbb{R}$ and $\mathbb{C}$ denote the real and complex number sets, respectively.
$j=\sqrt{-1}$ denotes the imaginary unit.
$\mathbf{0}$ and $\mathbf{I}$ denote an all-zero matrix and an identity matrix, respectively, with appropriate dimensions.
$\mathcal{CN}\left(\mathbf{0},\mathbf{I}\right)$ denotes the distribution of a circularly symmetric complex Gaussian random vector with mean vector $\mathbf{0}$ and covariance matrix $\mathbf{I}$ and $\sim$ stands for \textquotedblleft distributed as\textquotedblright.
diag$\left(a_{1},\ldots,a_{N}\right)$ refers to a diagonal matrix with diagonal elements being $a_{1},\ldots,a_{N}$.
diag$\left(\mathbf{A}_{1},\ldots,\mathbf{A}_{N}\right)$ refers to a block diagonal matrix with blocks being $\mathbf{A}_{1},\ldots,\mathbf{A}_{N}$.
ones$\left(N\right)$ refers to an $N\times N$ matrix whose elements are all ones.

%%%%%%%%%%%%%%%%%%%%%%%%%%%%%%%%%%%%%%%%%%%%%%%%%%
\section{System Model}
\label{sec:system-model}

% SU-MIMO RIS-aided system with single stream transmission
Let us consider an RIS-aided \gls{mimo} system, as represented in Fig.~\ref{fig:ris-system}.
We denote as $N_{T}$ the number of antennas at the transmitter, $N_{R}$ the number of antennas at the receiver, and $N_{I}$ the number of antennas at the RIS.
The $N_{I}$ antennas of the RIS are connected to a $N_{I}$-port reconfigurable impedance network, with scattering matrix $\boldsymbol{\Theta}\in\mathbb{C}^{N_{I}\times N_{I}}$.
To characterize the channel matrix seen by the receiver $\mathbf{H}\in\mathbb{C}^{N_{R}\times N_{T}}$ as a function of $\boldsymbol{\Theta}$, we assume all antennas perfectly matched and we assume no mutual coupling between the antennas, as widely adopted in the literature \cite{she20}\footnote{In practice, these two assumptions can be achieved by individually matching each RIS antenna to the reference impedance $Z_{0}=50\:\Omega$ and by keeping the spacing between the RIS elements larger than half-wavelength.}.
With these two assumptions, the channel matrix $\mathbf{H}$ can be written as
\begin{equation}
\mathbf{H}=\mathbf{H}_{RT}+\mathbf{H}_{RI}\boldsymbol{\Theta}\mathbf{H}_{IT},\label{eq:H}
\end{equation}
where $\mathbf{H}_{RT}\in\mathbb{C}^{N_{R}\times N_{T}}$, $\mathbf{H}_{RI}\in\mathbb{C}^{N_{R}\times N_{I}}$, and $\mathbf{H}_{IT}\in\mathbb{C}^{N_{I}\times N_{T}}$ are the channels from transmitter to receiver, RIS to receiver, and transmitter to RIS, respectively \cite{she20}.
Perfect knowledge of the channels $\mathbf{H}_{RT}$, $\mathbf{H}_{RI}$, and $\mathbf{H}_{IT}$ is assumed, that can be acquired, for instance, through the semi-passive channel estimation protocol in \cite{wu21}.

% Single-stream transmission
We consider a single-stream \gls{mimo} transmission scheme, to exploit the diversity gain offered by the multiple antennas at the transmitter and receiver, and by the reflecting elements at the RIS.
We denote the transmit signal as $\mathbf{x}=\mathbf{w}s\in\mathbb{C}^{N_{T}\times1}$, where $\mathbf{w}\in\mathbb{C}^{N_{T}\times1}$ is the precoding vector subject to the constraint $\left\|\mathbf{w}\right\|=1$, and $s\in\mathbb{C}$ is the transmit symbol with average power $P_{T}=\mathrm{E}[\left|s\right|^{2}]$.
Denoting the receive signal as $\mathbf{y}\in\mathbb{C}^{N_{R}\times1}$, we have $\mathbf{y}=\mathbf{H}\mathbf{x}+\mathbf{n}$, where $\mathbf{H}$ is the \gls{mimo} channel matrix given by \eqref{eq:H}, and $\mathbf{n}\sim\mathcal{CN}\left(\mathbf{0},\sigma_{n}^{2}\mathbf{I}\right)$ is the \gls{awgn} at the receiver with variance $\sigma_{n}^{2}$.
Thus, the signal used for detection is given by $z=\mathbf{g}\mathbf{y}\in\mathbb{C}$, where $\mathbf{g}\in\mathbb{C}^{1\times N_{R}}$ is the combining vector subject to the constraint $\left\|\mathbf{g}\right\|=1$.
Eventually, we can express the signal $z$ as
\begin{equation}
z=\mathbf{g}\mathbf{H}\mathbf{w}s+\tilde{n},\label{eq:z}
\end{equation}
where $\tilde{n}=\mathbf{g}\mathbf{n}$ is the \gls{awgn} with variance $\sigma_{n}^{2}$.

\begin{figure}[t]
    \centering
    \includegraphics[width=0.48\textwidth]{figs_v6/RIS-system-v4.eps}
    \caption{RIS-aided MIMO communication system model.}
    \label{fig:ris-system}
\end{figure}

% Meaning of \Theta
%According to network theory \cite{poz11}, denoting with $\mathbf{Z}_{I}\in\mathbb{C}^{N_{I}\times N_{I}}$ the impedance matrix of the $N_{I}$-port reconfigurable impedance network, $\boldsymbol{\Theta}$ can be expressed as
%\begin{equation}
%\boldsymbol{\Theta}=\left(\mathbf{Z}_{I}+Z_{0}\mathbf{I}\right)^{-1}\left(\mathbf{Z}_{I}-Z_{0}\mathbf{I}\right).\label{eq:T(Z)}
%\end{equation}
%We denote the impedance matrix of the $N_{I}$-port reconfigurable impedance network as $\mathbf{Z}_{I}\in\mathbb{C}^{N_{I}\times N_{I}}$.
The $N_{I}$-port reconfigurable impedance network is constructed with passive elements which can be adapted to the channel to properly reflect the incident signal.
To maximize the power reflected by the RIS, we consider the impedance matrix of the $N_{I}$-port reconfigurable impedance network $\mathbf{Z}_{I}\in\mathbb{C}^{N_{I}\times N_{I}}$ purely reactive.
Thus, we can write $\mathbf{Z}_{I}=j\mathbf{X}_{I}$, where $\mathbf{X}_{I}\in\mathbb{R}^{N_{I}\times N_{I}}$ denotes the reactance matrix of the $N_{I}$-port reconfigurable impedance network.
Hence, according to network theory \cite{poz11}, $\boldsymbol{\Theta}$ is given by
\begin{equation}
\boldsymbol{\Theta}=\left(j\mathbf{X}_{I}+Z_{0}\mathbf{I}\right)^{-1}\left(j\mathbf{X}_{I}-Z_{0}\mathbf{I}\right).\label{eq:T(X)}
\end{equation}
Furthermore, the reconfigurable impedance network is also reciprocal so that we have $\mathbf{X}_{I}=\mathbf{X}_{I}^{T}$ and $\boldsymbol{\Theta}=\boldsymbol{\Theta}^{T}$.
Depending on the topology of the reconfigurable impedance network, three different RIS architectures have been identified in \cite{she20}, which are described in the following.

\subsection{Single Connected RIS Architecture}

The single connected RIS architecture is the conventional architecture adopted in the literature \cite{bas19}, \cite{wu19a}, \cite{liu21}.
Here, each port of the reconfigurable impedance network is connected to ground with a reconfigurable impedance and is not connected to the other ports.
The reactance matrix $\mathbf{X}_{I}$ is a diagonal matrix given by $\mathbf{X}_{I}=\mathrm{diag}\left(X_{1},X_{2},\ldots,X_{N_{I}}\right)$, where $X_{n_{I}}$ is the reactance connecting the $n_{I}$th port to ground, for $n_{I}=1,\ldots,N_{I}$.
According to \eqref{eq:T(X)}, the scattering matrix $\boldsymbol{\Theta}$ is also a diagonal matrix written as
\begin{equation}
\boldsymbol{\Theta}=\mathrm{diag}\left(e^{j\theta_{1}},e^{j\theta_{2}},\ldots,e^{j\theta_{N_{I}}}\right),\label{eq:diag(T)}
\end{equation}
where $e^{j\theta_{n_{I}}}=\frac{jX_{n_{I}}-Z_{0}}{jX_{n_{I}}+Z_{0}}$ is the reflection coefficient of the reactance $X_{n_{I}}$, for $n_{I}=1,\ldots,N_{I}$.

\subsection{Fully Connected RIS Architecture}

The fully connected RIS architecture is obtained by connecting every port of the reconfigurable impedance network to all other ports.
Therefore, the reactance matrix $\mathbf{X}_{I}$ can be an arbitrary symmetric matrix.
According to \eqref{eq:T(X)}, $\boldsymbol{\Theta}$ is a complex symmetric unitary matrix
\begin{equation}
\boldsymbol{\Theta}=\boldsymbol{\Theta}^{T},\:\boldsymbol{\Theta}^{H}\boldsymbol{\Theta}=\boldsymbol{\mathrm{I}}.\label{eq:T fully}
\end{equation}

\subsection{Group Connected RIS Architecture}

The group connected RIS architecture has been proposed as a trade-off between the single connected and the fully connected to achieve a good balance between performance and complexity.
In the group connected architecture, the $N_{I}$ elements are divided into $G$ groups, each having $N_{G}=\frac{N_{I}}{G}$ elements.
Each element of the $N_{I}$-port is connected to all other elements in its group, while there is no connection inter-group. 
Thus, $\mathbf{X}_{I}$ is a block diagonal matrix given by
\begin{equation}
\mathbf{X}_{I}=\mathrm{diag}\left(\mathbf{X}_{I,1},\mathbf{X}_{I,2},\ldots,\mathbf{X}_{I,G}\right),\:\mathbf{X}_{I,g}=\mathbf{X}_{I,g}^{T},\:\forall g,\label{eq:X group}
\end{equation}
where $\mathbf{X}_{I,g}\in\mathbb{R}^{N_{G}\times N_{G}}$ is the reactance matrix of the $N_{G}$-port fully connected reconfigurable impedance network for the $g$th group.
According to \eqref{eq:T(X)}, the following constraints can be found for the scattering matrix in the group connected architecture
\begin{equation}
\boldsymbol{\Theta}=\mathrm{diag}\left(\boldsymbol{\Theta}_{1},\boldsymbol{\Theta}_{2},\ldots,\boldsymbol{\Theta}_{G}\right),\:\boldsymbol{\Theta}_{g}=\boldsymbol{\Theta}_{g}^{T},\:\boldsymbol{\Theta}_{g}^{H}\boldsymbol{\Theta}_{g}=\boldsymbol{\mathrm{I}},\:\forall g,\label{eq:T group}
\end{equation}
which show that $\boldsymbol{\Theta}$ is a block diagonal matrix with each block $\boldsymbol{\Theta}_{g}$ being a complex symmetric unitary matrix, $\forall g$.

% Our contribution for group and fully connected RISs

In the following, we propose two novel strategies to design discrete-value group and fully connected RISs.
In the first, scalar-discrete RISs, $B$ resolution bits are allocated to each reactance matrix entry $\left[\mathbf{X}_I\right]_{i,j}$.
In the second, vector-discrete RISs, $B_V$ bits are set to each reactance matrix block $\mathbf{X}_{I,g}$.

%%%%%%%%%%%%%%%%%%%%%%%%%%%%%%%%%%%%%%%%%%%%%%%%%%
\section{Scalar-Discrete Group/Fully Connected RIS}
\label{sec:scalar-discrete}

% SU-MIMO RIS-aided system received signal power maximization problem
% Introduce the problem in the continuous case, not practical
Our goal is to design the discrete-value matrix $\boldsymbol{\Theta}$ and the vectors $\mathbf{g}$ and $\mathbf{w}$ to maximize the received signal power, given by $P_{R}=P_{T}\left|\mathbf{g}\left(\mathbf{H}_{RT}+\mathbf{H}_{RI}\boldsymbol{\Theta}\mathbf{H}_{IT}\right)\mathbf{w}\right|^{2}$.
In the case of single-stream transmission, the optimal precoding and combining vectors are given by the dominant eigenmode transmission \cite{cle13}.
Thus, maximizing the received signal power is equivalent to maximize $\left\|\mathbf{H}_{RT}+\mathbf{H}_{RI}\boldsymbol{\Theta}\mathbf{H}_{IT}\right\|^{2}$.
%Let us now consider group connected architectures, including the fully connected architecture as the special case in which $G=1$ and $N_G=N_I$.
To investigate the design of discrete-value group and fully connected RISs, it is necessary to first consider the design of group and fully connected RISs with continuous-value $\boldsymbol{\Theta}$.
Then, the continuous-value optimization is used to optimize $\boldsymbol{\Theta}$ with discrete values.
%Thus, we solve the received signal power maximization problem for continuous-value $\boldsymbol{\Theta}$, which has never been solved for single-user \gls{mimo} systems aided by a group connected RIS.

\subsection{Continuous-Value Group/Fully Connected RIS}

The received signal power maximization problem for continuous-value $\boldsymbol{\Theta}$ in \gls{mimo} systems can be formulated as
\begin{align}
\underset{\mathbf{X}_{I,g}}{\mathsf{\mathrm{max}}}\;\;
& \left\|\mathbf{H}_{RT}+\mathbf{H}_{RI}\boldsymbol{\Theta}\mathbf{H}_{IT}\right\|^{2}\label{eq:P-GC-C-obj-0}\\
\mathsf{\mathrm{s.t.}}\;\;\; & \boldsymbol{\Theta}=\mathrm{diag}\left(\boldsymbol{\Theta}_{1},\ldots,\boldsymbol{\Theta}_{G}\right),\label{eq:P-GC-C-con1-0}\\
& \boldsymbol{\Theta}_{g}=\boldsymbol{\Theta}_{g}^{T},\:\boldsymbol{\Theta}_{g}^{H}\boldsymbol{\Theta}_{g}=\boldsymbol{\mathrm{I}},\:\forall g,\label{eq:P-GC-C-con2-0}
\end{align}
where the constraints derived from \eqref{eq:T group} indicate that the scattering matrix $\boldsymbol{\Theta}$ is a block diagonal matrix with each block being a complex symmetric unitary matrix.
These constraints complicate the optimization problem, which needs to be reformulated.
Thus, the relationship between $\boldsymbol{\Theta}$ and the reactance matrix $\mathbf{X}_{I}$ given by \eqref{eq:T(X)} is exploited to equivalently rewrite problem \eqref{eq:P-GC-C-obj-0}-\eqref{eq:P-GC-C-con2-0} as
\begin{align}
\underset{\mathbf{X}_{I,g}}{\mathsf{\mathrm{max}}}\;\;
& \left\|\mathbf{H}_{RT}+\mathbf{H}_{RI}\boldsymbol{\Theta}\mathbf{H}_{IT}\right\|^{2}\label{eq:P-GC-C-obj}\\
\mathsf{\mathrm{s.t.}}\;\;\; & \boldsymbol{\Theta}=\mathrm{diag}\left(\boldsymbol{\Theta}_{1},\ldots,\boldsymbol{\Theta}_{G}\right),\label{eq:P-GC-C-con1}\\
& \boldsymbol{\Theta}_{g}=\left(j\mathbf{X}_{I,g}+Z_{0}\mathbf{I}\right)^{-1}\left(j\mathbf{X}_{I,g}-Z_{0}\mathbf{I}\right),\:\forall g,\label{eq:P-GC-C-con2}\\
& \mathbf{X}_{I,g}=\mathbf{X}_{I,g}^{T},\:\forall g,\label{eq:P-GC-C-con3}
\end{align}
which can be transformed into an unconstrained problem.
More precisely, exploiting the constraints \eqref{eq:P-GC-C-con1} and \eqref{eq:P-GC-C-con2}, the objective $\left\|\mathbf{H}_{RT}+\mathbf{H}_{RI}\boldsymbol{\Theta}\mathbf{H}_{IT}\right\|^{2}$ can be expressed as a function of $\mathbf{X}_{I,g},\:\forall g$.
Since $\mathbf{X}_{I,g}$ is an arbitrary $N_{G}\times N_{G}$ real symmetric matrix, $\mathbf{X}_{I,g}$ is an unconstrained function of the $N_{G}\left(N_{G}+1\right)/2$ entries in its upper triangular part.
Thus, the obtained problem is an unconstrained optimization problem in the variables $\left[\mathbf{X}_{I,g}\right]_{i,j}$, with $i\leq j$ and $\forall g$.

% Solution
Problem \eqref{eq:P-GC-C-obj}-\eqref{eq:P-GC-C-con3} has been solved for RIS-aided \gls{siso} systems in \cite{she20} by using the quasi-Newton method to find the optimal upper triangular part of each $\mathbf{X}_{I,g}$ without any constraints.
For RIS-aided \gls{mimo} systems, we propose to solve it by alternatively optimizing $\mathbf{X}_I$ and the beamforming vectors $\mathbf{g}$ and $\mathbf{w}$, as established in the literature on single connected RISs \cite{wu21}, \cite{wu19c}, \cite{zap21}.
After $\mathbf{g}$ and $\mathbf{w}$ are initialized to feasible values, this optimization process alternates between the two following steps until convergence is reached.
With fixed $\mathbf{g}$ and $\mathbf{w}$, we update $\mathbf{X}_I$ by maximizing the objective $\left|\mathbf{g}\mathbf{H}_{RT}\mathbf{w}+\mathbf{g}\mathbf{H}_{RI}\boldsymbol{\Theta}\mathbf{H}_{IT}\mathbf{w}\right|$ with the quasi-Newton method as in \cite{she20}, where $\boldsymbol{\Theta}$ is a function of $\mathbf{X}_I$ given by \eqref{eq:T(X)}.
With fixed $\mathbf{X}_I$, and consequently fixed $\boldsymbol{\Theta}$, we update $\mathbf{g}$ and $\mathbf{w}$ as the dominant left and right singular vectors of the matrix $\mathbf{H}_{RT}+\mathbf{H}_{RI}\boldsymbol{\Theta}\mathbf{H}_{IT}$, respectively.
Note that this method is proven to converge to a stationary point of the objective.

\subsection{Problem Formulation for Discrete-Value RIS}

In \eqref{eq:P-GC-C-obj}-\eqref{eq:P-GC-C-con3}, the reactance matrix entries are allowed to assume arbitrary real values, which is hard to realize in practice.
Thus, we are interested in an RIS design strategy in which each reactance matrix entry is selected from a codebook.
We consider a discrete reactance codebook $\mathcal{C}$ symmetric around zero, written as
\begin{equation}
\mathcal{C} = \left\{\pm c_{1},\pm c_{2},\ldots,\pm c_{2^{B-1}}\right\},\label{eq:codebook}
\end{equation}
where $c_{b}>0$, for $b=1,\ldots,2^{B-1}$.
When selecting each reactance matrix entry from a codebook, the RIS optimization problem cannot be simply solved as an unconstrained problem.
Thus, instead of \eqref{eq:P-GC-C-obj}-\eqref{eq:P-GC-C-con3}, the following optimization problem can be considered to optimize $\mathbf{X}_{I}$
\begin{align}
\underset{\mathbf{X}_{I,g}}{\mathsf{\mathrm{max}}}\;\;
& \left\|\mathbf{H}_{RT}+\mathbf{H}_{RI}\boldsymbol{\Theta}\mathbf{H}_{IT}\right\|^{2}\label{eq:P-GC-D-obj1}\\
\mathsf{\mathrm{s.t.}}\;\;\;
& \boldsymbol{\Theta}=\mathrm{diag}\left(\boldsymbol{\Theta}_{1},\ldots,\boldsymbol{\Theta}_{G}\right),\label{eq:P-GC-D-con1-1}\\
& \boldsymbol{\Theta}_{g}=\left(j\mathbf{X}_{I,g}+Z_{0}\mathbf{I}\right)^{-1}\left(j\mathbf{X}_{I,g}-Z_{0}\mathbf{I}\right),\:\forall g,\label{eq:P-GC-D-con1-2}\\
& \mathbf{X}_{I,g}=\mathbf{X}_{I,g}^{T},\:\forall g,\label{eq:P-GC-D-con1-3}\\
& \left[\mathbf{X}_{I,g}\right]_{i,j}\in\mathcal{C},\:\forall g.\label{eq:P-GC-D-con1-4}
\end{align}
Since the entries of $\mathbf{X}_{I}$ are not distributed in a finite interval, to define the optimal codebook $\mathcal{C}$ is not as straightforward as in the single connected case \cite{wu21}.
Thus, we introduce a second optimization problem to design $\mathcal{C}$ as
\begin{align}
\underset{\mathcal{C}}{\mathsf{\mathrm{max}}}\;\;
& \mathrm{E}\left[\left\|\mathbf{H}_{RT}+\mathbf{H}_{RI}\boldsymbol{\Theta}\mathbf{H}_{IT}\right\|^{2}\right]\label{eq:P-GC-D-obj2}\\
\mathsf{\mathrm{s.t.}}\;\;\;
& \boldsymbol{\Theta}=\mathrm{diag}\left(\boldsymbol{\Theta}_{1},\ldots,\boldsymbol{\Theta}_{G}\right),\label{eq:P-GC-D-con2-1}\\
& \boldsymbol{\Theta}_{g}=\left(j\mathbf{X}_{I,g}+Z_{0}\mathbf{I}\right)^{-1}\left(j\mathbf{X}_{I,g}-Z_{0}\mathbf{I}\right),\:\forall g,\label{eq:P-GC-D-con2-2}\\
& \mathbf{X}_{I,g}\text{ solves \eqref{eq:P-GC-D-obj1}-\eqref{eq:P-GC-D-con1-4}},\:\forall g,\label{eq:P-GC-D-con2-3}
\end{align}
where the objective is the ergodic received signal power and the expectation operator $\mathrm{E}[\cdot]$ represents the average over the channel realizations\footnote{Note that the received signal power in \eqref{eq:P-GC-D-obj1}-\eqref{eq:P-GC-D-con2-3} can be replaced by any objective function depending on the \gls{csi} and the RIS scattering matrix.
Thus, the proposed codebook design and optimization framework is general enough to accommodate any objective function, including the metrics typically considered in multi-user systems, e.g., the sum rate.}.
Note that in \eqref{eq:P-GC-D-obj2}-\eqref{eq:P-GC-D-con2-3}, $\mathbf{X}_{I,g}$ implicitly depends on $\mathcal{C}$ because of constraint \eqref{eq:P-GC-D-con2-3}.
These two nested optimization problems form a bilevel programming problem, in which \eqref{eq:P-GC-D-obj2}-\eqref{eq:P-GC-D-con2-3} is the upper-level problem, and \eqref{eq:P-GC-D-obj1}-\eqref{eq:P-GC-D-con1-4} is the lower-level problem.
This bilevel problem is difficult to solve since in the upper-level problem we do not have an expression for the constraint \eqref{eq:P-GC-D-con2-3}, and we cannot explicitly write $\mathbf{X}_{I,g}$ as a function of $\mathcal{C}$.

% Two stages to define the codebook and optimize
For this reason, when considering the optimization of discrete group and fully connected RISs, two problems arise.
Firstly, during the offline learning stage, a suitable discrete reactance codebook has to be defined for the elements of $\mathbf{X}_{I}$.
Secondly, during the online deployment stage, the discrete values of $\mathbf{X}_{I}$ are optimized by choosing among the values of the fixed codebook.
%In Fig.~\ref{fig:ris-diagram}, we highlight the two stages in our RIS design strategy that aim to solve these two problems.
%To solve these two problems, we propose an RIS design based on two stages, namely, offline learning and online deployment.
%In the following, we describe these two stages, offline learning and online deployment.
In the following, these two stages are described.
The offline learning stage is performed differently for binary-level ($B=1$) and multi-level ($B>1$) codebooks since a lower complexity solution is available when $B=1$.

\subsection{Offline Learning for Binary-Level Codebook}
\label{sec:scalar-discrete-binary}

% binary-value codebook construction: part 1
During offline learning, we assume that the codebook design can exploit a training set of $N_0$ channel realization triplets.
Such a training set can be practically collected through a sampling campaign to be conducted offline.
During this offline sampling campaign, channel realizations are collected and stored with dedicated transceiver devices.
The $n_0$th triplet is denoted as $\mathcal{H}^{[n_0]}=(\mathbf{H}_{RT}^{[n_0]},\mathbf{H}_{RI}^{[n_0]},\mathbf{H}_{IT}^{[n_0]})$ and includes the channels from transmitter to receiver, RIS to receiver, and transmitter to RIS, respectively.
Thus, the training set is a set of triplets defined as
\begin{equation}
\mathcal{H}_0=\left\{\mathcal{H}^{[1]},\mathcal{H}^{[2]},\ldots,\mathcal{H}^{[N_0]}\right\}.
\end{equation}
When the number of resolution bits per reconfigurable reactance is $B=1$, the entries of the reactance matrix $\mathbf{X}_{I}$ can assume only the two values in $\mathcal{C}=\{-c_{1},+c_{1}\}$, where $c_{1}$ is a positive real number.
Exploiting this special structure of binary-level codebooks, we can obtain the optimal codebook value $c_{1}$ and binary-value reactance matrix $\bar{\mathbf{X}}_I$ for each channel triplet in the training set by solving the bilevel optimization problem \eqref{eq:P-GC-D-obj1}-\eqref{eq:P-GC-D-con2-3}.

% How to obtain the two sets
%To obtain the two sets $\mathcal{C}_1$ and $\bar{\mathcal{X}}_I$, we solve problem \eqref{eq:P-GC-D-obj1}-\eqref{eq:P-GC-D-con2-3} for each channel triplet in the training set as follows.
Firstly, problem \eqref{eq:P-GC-D-obj1}-\eqref{eq:P-GC-D-con1-4} can be solved given a fixed value of $c_1$ through alternating optimization.
%, also known as successive refinement, or local search \cite{you20}, \cite{abe20}, \cite{wu19b}.
With this method, each non-zero entry of the reactance matrix is optimized individually by searching among the two possible values in $\mathcal{C}$, while fixing the other $N_I\left(N_{G}+1\right)/2-1$ entries.
This procedure is repeated for all the reactance matrix entries, iterating multiple times until the convergence is reached.
In this way, we can evaluate the optimal $\mathbf{X}_{I,g}$ as function of $c_{1}$ in constraint \eqref{eq:P-GC-D-con2-3}.
Secondly, the upper-level problem \eqref{eq:P-GC-D-obj2}-\eqref{eq:P-GC-D-con2-3} is solvable with one-dimensional pattern search since its objective can be readily evaluated as a function of $c_{1}$.
We employ pattern search since it is an optimization algorithm that only requires the evaluation of the objective function in a series of points until the convergence is reached, without using derivatives \cite{aud02}.
Thus, we build the two sets
\begin{align}
\mathcal{C}_1&=\left\{c_1^{\left[1\right]},c_1^{\left[2\right]},\ldots,c_{1}^{\left[N_0\right]}\right\},\\
\bar{\mathcal{X}}_I&=\left\{\bar{\mathbf{X}}_I^{\left[1\right]},\bar{\mathbf{X}}_I^{\left[2\right]},\ldots,\bar{\mathbf{X}}_I^{\left[N_0\right]}\right\},
\end{align}
where $c_{1}^{\left[n_0\right]}$ and $\bar{\mathbf{X}}_I^{\left[n_0\right]}$ are the optimal codebook value $c_{1}$ and binary-value reactance matrix $\bar{\mathbf{X}}_I$ associated to the channel triplet $\mathcal{H}^{[n_0]}$, respectively.

% binary-value codebook construction: part 2
Given the two sets $\mathcal{C}_1$ and $\bar{\mathcal{X}}_I$, we design the optimal value for $c_1$, denoted as $c_1^\star$.
We obtain $c_{1}^\star$ as the value that minimizes the distortion caused in the reactance matrix
\begin{align}
c_1^\star=\argmin_{c_{1}}\;\;
& \frac{1}{N_0}\sum_{n_0=1}^{N_0}\left\Vert\bar{\mathbf{X}}_I^{[n_0]}-\mathbf{X}_I\right\Vert_F^2\label{eq:B1-1}\\
\mathsf{\mathrm{s.t.}}\;\;\;
& \mathbf{X}_{I}=\mathrm{diag}\left(\mathbf{X}_{I,1},\ldots,\mathbf{X}_{I,G}\right),\\
& \left[\mathbf{X}_{I,g}\right]_{i,j}\in\left\{-c_1,+c_1\right\},\:\forall g,
\end{align}
where \eqref{eq:B1-1} is a sample average computed over the channel triplets in the training set.
In \eqref{eq:B1-1}, each non-zero squared entry of the matrix $\bar{\mathbf{X}}_I^{[n_0]}-\mathbf{X}_I$ is given by $(c_1^{[n_0]}-c_1)^2$ because of the symmetry of the codebook.
Thus, problem \eqref{eq:B1-1} can be reformulated as
%\begin{equation}
%c_1^\star=\argmin_{c_{1}}\;\;\mathrm{E}\left[N_IN_G\left(c_1^{[n_0]}-c_1\right)^2\right]
%\end{equation}
\begin{equation}
c_1^\star=\argmin_{c_{1}}\;\;\frac{N_IN_G}{N_0}\sum_{n_0=1}^{N_0}\left(c_1^{[n_0]}-c_1\right)^2.\label{eq:B1-2}
\end{equation}
By setting the derivative of the objective function in \eqref{eq:B1-2} with respect to $c_1$ to zero, we obtain
\begin{equation}
c_1^\star=\frac{1}{N_0}\sum_{n_0=1}^{N_0}c_1^{[n_0]}.\label{eq:B1-3}
\end{equation}
Finally, the binary-value codebook is given by $\mathcal{C}=\{-c_1^\star,+c_1^\star\}$.

The complexity of the offline learning for binary-level codebook is driven by the complexity of solving the bilevel problem \eqref{eq:P-GC-D-obj1}-\eqref{eq:P-GC-D-con2-3} $N_0$ times to find $\mathcal{C}_1$.
Since one-dimensional pattern search requires computing twice the objective function per iteration, the complexity per iteration is given by $\mathcal{O}(2N_0N_I(N_G+1))$.

\subsection{Offline Learning for Multi-Level Codebook}
\label{sec:scalar-discrete-multi}

Pattern search can be used to solve the upper-level problem \eqref{eq:P-GC-D-obj2}-\eqref{eq:P-GC-D-con2-3} only when the optimization involves a one-dimensional search, that is only when $B=1$.
When more resolution bits are considered, pattern search becomes highly sub-optimal since it is likely to converge to a local maximum in a multi-dimensional search space.
For this reason, we rely on a different method to define multi-level codebooks.
The main idea is to learn the codebook from the distribution of the optimal continuous-value $\mathbf{X}_{I}$ which solve \eqref{eq:P-GC-C-obj}-\eqref{eq:P-GC-C-con3}.
More precisely, we find the optimal set of values $\{c_1^\star,\ldots,c_K^\star\}$, with $K=2^{B-1}$, such that the distortion on the optimal continuous-value reactance matrix is minimized.
%This can be realized with $K$-means clustering, an unsupervised machine learning technique able to group the elements of a distribution into $K$ clusters.
%More precisely, since the codebook is assumed symmetric around zero, we need to obtain the $2^{B-1}$ values of the positive half of the codebook.
%To this end, a training set is formed with the modulus of the optimal continuous $\mathbf{X}_{I}$ elements.
%Thus, $c_{1},c_{2},\ldots,c_{2^{B-1}}$ are given by the $K$ cluster centers obtained by clustering the training set, with $k=2^{B-1}$.
The process leading to the multi-level codebook $\mathcal{C}=\{\pm c_1^\star,\ldots,\pm c_{2^{B-1}}^\star\}$ is detailed in the following.

% Find optimal continuous-value reactance matrices
We firstly solve \eqref{eq:P-GC-C-obj}-\eqref{eq:P-GC-C-con3} for each training set triplet by alternatively optimizing $\mathbf{X}_I$ and the beamforming vectors $\mathbf{g}$ and $\mathbf{w}$.
As a result, we construct the set
\begin{equation}
\mathcal{X}_I=\left\{\mathbf{X}_I^{\left[1\right]},\mathbf{X}_I^{\left[2\right]},\ldots,\mathbf{X}_I^{\left[N_0\right]}\right\},
\end{equation}
where $\mathbf{X}_I^{\left[n_0\right]}$ is the optimal continuous-value reactance matrix associated to the channel triplet $\mathcal{H}^{[n_0]}$.
Given the set $\mathcal{X}_I$, the optimal values $\{c_1^\star,\ldots,c_K^\star\}$ are determined by solving
\begin{align}
\{c_1^\star,\ldots,c_K^\star\}=\argmin_{\{c_1,\ldots,c_K\}}\;\;
& \frac{1}{N_0}\sum_{n_0=1}^{N_0}\left\Vert\mathbf{X}_I^{[n_0]}-\mathbf{X}_I\right\Vert_F^2\label{eq:B2-1}\\
\mathsf{\mathrm{s.t.}}\;\;\;
& \mathbf{X}_{I}=\mathrm{diag}\left(\mathbf{X}_{I,1},\ldots,\mathbf{X}_{I,G}\right),\\
& \left[\mathbf{X}_{I,g}\right]_{i,j}\in\left\{\pm c_1,\ldots,\pm c_K\right\},\:\forall g.
\end{align}
To simplify problem \eqref{eq:B2-1}, we introduce the vectors $\mathbf{x}^{[n_0]}\in\mathbb{R}^{1\times N_IN_G}$ containing the non-zero entries of $\mathbf{X}_I^{[n_0]}$.
Then, the $N_0$ vectors $\mathbf{x}^{[n_0]}$ are concatenated in $\mathbf{x}_0\in\mathbb{R}^{1\times N_0N_IN_G}$ as $\mathbf{x}_0=[\mathbf{x}^{[1]},\ldots,\mathbf{x}^{[N_0]}]$.
In this way, problem \eqref{eq:B2-1} becomes
\begin{align}
\{c_1^\star,\ldots,c_K^\star\}=\argmin_{\{c_1,\ldots,c_K\}}\;\;
& \frac{1}{N_0}\sum_{n=1}^{N_0N_IN_G}\left\vert\left[\mathbf{x}_0\right]_n-\left[\mathbf{x}_I\right]_n\right\vert^2\label{eq:B2-2}\\
\mathsf{\mathrm{s.t.}}\;\;\;
& \left[\mathbf{x}_I\right]_n\in\left\{\pm c_1,\ldots,\pm c_K\right\},\:\forall n.
\end{align}
where the vector $\mathbf{x}_I\in\mathbb{R}^{1\times N_0N_IN_G}$ has been introduced as an auxiliary variable.
Exploiting the symmetry of the codebook $\mathcal{C}$, \eqref{eq:B2-2} can be further reformulated as
\begin{align}
\{c_1^\star,\ldots,c_K^\star\}=\argmin_{\{c_1,\ldots,c_K\}}\;\;
& \frac{1}{N_0}\sum_{n=1}^{N_0N_IN_G}\left\vert\left\vert\left[\mathbf{x}_0\right]_n\right\vert-\left[\mathbf{x}_I\right]_n\right\vert^2\label{eq:B2-3}\\
\mathsf{\mathrm{s.t.}}\;\;\;
& \left[\mathbf{x}_I\right]_n\in\left\{c_1,\ldots,c_K\right\},\:\forall n,
\end{align}
which is a $K$-means clustering problem in a one-dimensional space, with $K=2^{B-1}$ \cite{bis06}.
For each data point $x_n=\left\vert\left[\mathbf{x}_0\right]_n\right\vert$ we introduce the binary indicator $r_{nk}\in\{0,1\}$ such that $r_{nk}=1$ if $x_n$ is assigned to $c_k$, and $r_{ni}=0$ otherwise.
The resulting $K$-means clustering problem is formalized as
\begin{equation}
\{c_1^\star,\ldots,c_K^\star\}=\argmin_{\{c_1,\ldots,c_K\}}\;\;\sum_{n=1}^{N_0N_IN_G}\sum_{k=1}^{2^{B-1}}r_{nk}\left\vert x_n-c_k\right\vert^2\label{eq:B2-4}
\end{equation}
To solve \eqref{eq:B2-4}, we iteratively optimize the sets $\{r_{nk}|n=1,\ldots,N_0N_IN_G,k=1,\ldots,K\}$ and $\{c_1,\ldots,c_K\}$.
Firstly, when the values $\{c_1,\ldots,c_K\}$ are fixed, we set $\{r_{nk}\}$ by assigning the $n$th data point $x_n$ to the closest cluster, i.e.,
\begin{equation}
r_{nk}=
\begin{cases}
1 & \text{if } k=\arg \min_{j} \left\vert x_n-c_j\right\vert^2\\
0 & \text{otherwise}
\end{cases},
\end{equation}
for $n=1,\ldots,N_0N_IN_G$ and $k=1,\ldots,K$.
Secondly, when $\{r_{nk}\}$ are fixed, the cluster centers $\{c_1,\ldots,c_K\}$ are updated.
This can be done in close form by setting the derivative of the objective function in \eqref{eq:B2-4} with respect to $c_k$ to zero
\begin{equation}
c_k=\frac{\sum_n r_{nk}x_n}{\sum_n r_{nk}},
\end{equation}
for $k=1,\ldots,K$.
The two steps of assigning data points to clusters and updating the cluster
centers are repeated alternatively until there is no further change in the cluster assignment.
Finally, the multi-value codebook is given by $\mathcal{C}=\{\pm c_1^\star,\ldots,\pm c_{2^{B-1}}^\star\}$.

The complexity of the offline learning for multi-level codebook is given by $\mathcal{O}(N_0(N_I(N_G+1)/2)^2+N_0N_IN_G2^{B-1})$, where the first term is due to solving \eqref{eq:P-GC-C-obj}-\eqref{eq:P-GC-C-con3} $N_0$ times to find $\mathcal{X}_I$, and the second term is due to the $K$-means clustering algorithm used to solve \eqref{eq:B2-4}.

\begin{algorithm}[t]
\KwIn{$B$, $\mathcal{H}_0$, $\mathbf{H}_{RT}$, $\mathbf{H}_{RI}$, $\mathbf{H}_{IT}$, $N_G$}
\KwOut{$\mathcal{C}$, $\mathbf{X}_{I}$}
\nonl Offline Learning\\
\uIf{$B==1$}{
Calculate $\mathcal{C}_1$ by the bilevel problem \eqref{eq:P-GC-D-obj1}-\eqref{eq:P-GC-D-con2-3}\;
Calculate $c_1^\star$ by \eqref{eq:B1-3}\;
$\mathcal{C}\leftarrow\{-c_1^\star,+c_1^\star\}$\;
}
\Else{
Calculate $\mathcal{X}_I$ by \eqref{eq:P-GC-C-obj}-\eqref{eq:P-GC-C-con3}\;
Calculate $\{c_1^\star,\ldots,c_K^\star\}$ by iteratively solving \eqref{eq:B2-4}, with $K=2^{B-1}$\;
$\mathcal{C}\leftarrow\{\pm c_1^\star,\ldots,\pm c_{K}^\star\}$\;
}
\nonl Online Deployment\\
Initialize $\mathbf{X}_{I,g}=\mathbf{0},\:\forall g$\;
\While{no convergence of objective \eqref{eq:P-GC-D-obj1}}{
\For{$g\leftarrow 1$ \KwTo $G$}{
\For{$i\leftarrow 1$ \KwTo $N_G$}{
\For{$j\leftarrow i$ \KwTo $N_G$}{
Update $\left[\mathbf{X}_{I,g}\right]_{i,j}\in\mathcal{C}$ by searching among the $2^B$ possible values\;
$\left[\mathbf{X}_{I,g}\right]_{j,i}\leftarrow\left[\mathbf{X}_{I,g}\right]_{i,j}$
}
}
}
}
\caption{Scalar-Discrete RISs Design}
\label{alg:scalar}
\end{algorithm}

\subsection{Online Deployment}

% Online deployment

Once the codebook has been learned offline, the discrete $\mathbf{X}_{I}$ entries are optimized during the online deployment stage.
This is achieved by solving \eqref{eq:P-GC-D-obj1}-\eqref{eq:P-GC-D-con1-4} depending on the instantaneous \gls{csi} $\mathbf{H}_{RT}$, $\mathbf{H}_{RI}$, and $\mathbf{H}_{IT}$ and on the designed codebook $\mathcal{C}$.
To solve this problem, an exhaustive search could be employed since each reactance matrix entry can assume a finite number of values.
However, to reduce the search space, we employ an alternating optimization method.
In this sub-optimal method, each non-zero entry of the reactance matrix is optimized individually by searching among all possible $2^{B}$ values, while fixing the other $N_I\left(N_{G}+1\right)/2-1$ entries.
This procedure is repeated for all $N_I\left(N_{G}+1\right)/2$ reactance matrix entries, iterating multiple times until the convergence is reached.
The convergence is considered reached when the fractional increase of the objective value in a full iteration is below a certain parameter $\epsilon$.
The convergence is guaranteed by the following two facts.
First, at each iteration, the objective function $\Vert\mathbf{H}_{RT}+\mathbf{H}_{RI}\boldsymbol{\Theta}\mathbf{H}_{IT}\Vert^{2}$ is non-decreasing.
This holds since each reactance matrix entry is optimized by exhaustively searching among the possible values in the codebook $\mathcal{C}$.
Second, the objective function is bounded from above by $(\Vert\mathbf{H}_{RT}\Vert+\Vert\mathbf{H}_{RI}\Vert\Vert\mathbf{H}_{IT}\Vert)^2$.
The design of scalar-discrete RISs is summarized in Alg.~\ref{alg:scalar}, which solves the problem \eqref{eq:P-GC-D-obj1}-\eqref{eq:P-GC-D-con2-3}.

% Assumption on channel statistics
%The entire process leading to the design of scalar-discrete group connected RISs is summarized in Fig.~\ref{fig:ris-diagram}.
%The two main stages are emphasized.
%Firstly, an offline learning has the objective of building the codebook $\boldsymbol{\Psi}$.
%Secondly, during the online deployment, alternating optimization is used to find the discrete $\mathbf{X}_{I}$ which maximizes the received signal power.
Note that we assume that the statistical properties of the channels are unchanged during the offline learning and online deployment stages.
This is necessary since we approximate the expectation of the received signal power with its sample average, computed over the training set.
However, numerical simulations showed that learning the codebook based on outdated channel statistics only slightly degrades the performance.
Besides, perfect instantaneous \gls{csi} is assumed during the online deployment stage, which can be obtained with the semi-passive channel estimation strategy proposed in \cite{wu21}.
%Thus, although the offline learning stage is conducted offline, it has to be repeated when the channel statistics change.
%Since the only information needed during offline learning consists of channel samples, the training set can be obtained with no additional overhead by storing the \gls{csi} estimated at every channel realization.

\subsection{Scaling Law}

To analyze the optimality of the codebook in \eqref{eq:codebook}, we characterize the scaling law of the received signal power as a function of the number of RIS elements $N_I$, when $\mathcal{C}$ is used to discretize the reactance matrix.
Note that this analysis is necessary since the scaling law of the proposed discrete RIS architectures is not known and not straightforward given the novel constraints of BD-RISs.
To obtain fundamental insights, we consider a \gls{siso} system, i.e., $N_T=1$ and $N_R=1$.
Furthermore, the direct link $h_{RT}$ is negligible in comparison with the reflected signal power for asymptotically large $N_I$.
Considering unitary transmit power, the received signal power is given by $P_R=\left|\mathbf{h}_{RI}\boldsymbol{\Theta}\mathbf{h}_{IT}\right|^2$, where $\boldsymbol{\Theta}$ is obtained by discretizing the reactance matrix $\mathbf{X}_I$.
Such a scaling law is given in the following proposition.
\begin{proposition}
Assume i.i.d. Rayleigh fading channels, i.e., $\mathbf{h}_{RI}\sim\mathcal{CN}\left(\mathbf{0},\mathbf{I}\right)$ and $\mathbf{h}_{IT}\sim\mathcal{CN}\left(\mathbf{0},\mathbf{I}\right)$.
A group connected RIS whose reactance matrix entries are chosen from $\mathcal{C}$ can achieve an average received signal power in the order of $\mathcal{O}\left(N_I^2\right)$ as $N_I$ increases, even with one resolution bit per each reactance.
\label{pro:1}
\end{proposition}
\begin{proof}
Let us consider the worst case $B=1$, and assume that the matrix block $\mathbf{X}_{I,g}$ can only assume two values $\mathbf{X}_{I,g}=\pm c_1\text{ones}\left(N_G\right)$.
Note that proving Proposition 1 for the case $\mathcal{C}=\{\pm c_1\}$ also establishes its validity for all codebooks $\mathcal{C}=\{\pm c_{1},\pm c_{2},\ldots,\pm c_{2^{B-1}}\}$.
Through eigenmode decomposition, we write $\mathbf{X}_{I,g}=\mathbf{V}_{g}\boldsymbol{\Lambda}_{g}\mathbf{V}_{g}^T$, where $\boldsymbol{\Lambda}_{g}=\text{diag}\left(\boldsymbol{\lambda}_g\right)$ is diagonal containing the eigenvalues of $\mathbf{X}_{I,g}$ and $\mathbf{V}_{g}$ is orthonormal.
Among the eigenvalues of $\mathbf{X}_{I,g}$, only $[\boldsymbol{\lambda}_g]_1$ is non zero since $\mathbf{X}_{I,g}$ is rank one.
By applying \eqref{eq:T(X)}, the scattering matrix $\boldsymbol{\Theta}_g$ can be expressed as $\boldsymbol{\Theta}_g=\mathbf{V}_{g}\mathbf{D}_g\mathbf{V}_{g}^T$,
%\begin{align}
%\boldsymbol{\Theta}_g
%& =\left(j\mathbf{X}_{I,g}-Z_{0}\mathbf{I}\right)^{-1}\left(j\mathbf{X}_{I,g}+Z_{0}\mathbf{I}\right)\\
%& =\left(j\mathbf{V}_{g}\boldsymbol{\Lambda}_{g}\mathbf{V}_{g}^T-Z_{0}\mathbf{V}_{g}\mathbf{I}\mathbf{V}_{g}^T\right)^{-1}\left(j\mathbf{V}_{g}\boldsymbol{\Lambda}_{g}\mathbf{V}_{g}^T+Z_{0}\mathbf{V}_{g}\mathbf{I}\mathbf{V}_{g}^T\right)\\
%& =\left(\mathbf{V}_{g}\left(j\boldsymbol{\Lambda}_{g}-Z_{0}\mathbf{I}\right)\mathbf{V}_{g}^T\right)^{-1}\left(\mathbf{V}_{g}\left(j\boldsymbol{\Lambda}_{g}+Z_{0}\mathbf{I}\right)\mathbf{V}_{g}^T\right)\\
%& =\mathbf{V}_{g}\left(j\boldsymbol{\Lambda}_{g}-Z_{0}\mathbf{I}\right)^{-1}\left(j\boldsymbol{\Lambda}_{g}+Z_{0}\mathbf{I}\right)\mathbf{V}_{g}^T\\
%& =\mathbf{V}_{g}\mathbf{D}_g\mathbf{V}_{g}^T,\label{eq:T-decomposition}
%\end{align}
where $\mathbf{D}_g=\text{diag}\left(\mathbf{d}_g\right)$ is a diagonal matrix with $[\mathbf{d}_g]_{n_G}=\frac{j[\boldsymbol{\lambda}_g]_{n_{G}}-Z_{0}}{j[\boldsymbol{\lambda}_g]_{n_G}+Z_{0}}$.
According to \cite{she20}, the average received signal power for group connected RISs is written as
\begin{align}
\text{E}\left[P_{R}\right]
& =\text{E}\left[\left|\sum_{g=1}^{G}\mathbf{h}_{RI,g}\boldsymbol{\Theta}_{g}\mathbf{h}_{IT,g}\right|^2\right]\\
%& =\text{E}\left[\left|\sum_{g=1}^{G}\mathbf{h}_{RI,g}\mathbf{V}_{g}\mathbf{D}_{g}\mathbf{V}_{g}^T\mathbf{h}_{IT,g}\right|^2\right]\\
& =\text{E}\left[\left|\sum_{g=1}^{G}\sum_{n_G=1}^{N_G}\left[\bar{\mathbf{h}}_{RI,g}\right]_{n_G}\left[\mathbf{d}_g\right]_{n_G}\left[\bar{\mathbf{h}}_{IT,g}\right]_{n_G}\right|^2\right],
\end{align}
where we considered $\boldsymbol{\Theta}_g=\mathbf{V}_{g}\mathbf{D}_g\mathbf{V}_{g}^T$ and introduced  $\bar{\mathbf{h}}_{RI,g}=\mathbf{h}_{RI,g}\mathbf{V}_{g}$ and $\bar{\mathbf{h}}_{IT,g}=\mathbf{V}_{g}^T\mathbf{h}_{IT,g}$.
Recalling that $[\mathbf{d}_g]_{n_G}=-1$ if $n_G\geq 2$, we have
\begin{align}
\text{E}\left[P_{R}\right]
%& =\text{E}\left[\left|\sum_{g=1}^{G}\left(\left[\bar{\mathbf{h}}_{RI,g}\right]_{1}e^{j\left[\boldsymbol{\theta}_{g}\right]_{1}}\left[\bar{\mathbf{h}}_{IT,g}\right]_{1}+\sum_{n_G=2}^{N_G}\left[\bar{\mathbf{h}}_{RI,g}\right]_{n_G}e^{j\pi}\left[\bar{\mathbf{h}}_{IT,g}\right]_{n_G}\right)\right|^2\right]\\
%& =\text{E}\left[\left|\sum_{g=1}^{G}\left(\left[\bar{\mathbf{h}}_{RI,g}\right]_{1}e^{j\left[\boldsymbol{\theta}_{g}\right]_{1}}\left[\bar{\mathbf{h}}_{IT,g}\right]_{1}-\sum_{n_G=2}^{N_G}\left[\bar{\mathbf{h}}_{RI,g}\right]_{n_G}\left[\bar{\mathbf{h}}_{IT,g}\right]_{n_G}\right)\right|^2\right]\\
& =\text{E}\left[\left|\left(\sum_{g=1}^{G}\left[\bar{\mathbf{h}}_{RI,g}\right]_{1}\left[\mathbf{d}_g\right]_{1}\left[\bar{\mathbf{h}}_{IT,g}\right]_{1}\right)\right.\right.\\
& -\left.\left.\left(\sum_{g=1}^{G}\sum_{n_G=2}^{N_G}\left[\bar{\mathbf{h}}_{RI,g}\right]_{n_G}\left[\bar{\mathbf{h}}_{IT,g}\right]_{n_G}\right)\right|^2\right]\\
%& =\text{E}\Bigg[\left|\sum_{g=1}^{G}\left[\bar{\mathbf{h}}_{RI,g}\right]_{1}e^{j\left[\boldsymbol{\theta}_{g}\right]_{1}}\left[\bar{\mathbf{h}}_{IT,g}\right]_{1}\right|^2+\left|\sum_{g=1}^{G}\sum_{n_G=2}^{N_G}\left[\bar{\mathbf{h}}_{RI,g}\right]_{n_G}\left[\bar{\mathbf{h}}_{IT,g}\right]_{n_G}\right|^2\\
%& -\left(\sum_{g=1}^{G}\left[\bar{\mathbf{h}}_{RI,g}\right]_{1}e^{j\left[\boldsymbol{\theta}_{g}\right]_{1}}\left[\bar{\mathbf{h}}_{IT,g}\right]_{1}\right)\left(\sum_{g=1}^{G}\sum_{n_G=2}^{N_G}\left[\bar{\mathbf{h}}_{RI,g}\right]_{n_G}\left[\bar{\mathbf{h}}_{IT,g}\right]_{n_G}\right)^*\\
%& -\left(\sum_{g=1}^{G}\left[\bar{\mathbf{h}}_{RI,g}\right]_{1}e^{j\left[\boldsymbol{\theta}_{g}\right]_{1}}\left[\bar{\mathbf{h}}_{IT,g}\right]_{1}\right)^*\left(\sum_{g=1}^{G}\sum_{n_G=2}^{N_G}\left[\bar{\mathbf{h}}_{RI,g}\right]_{n_G}\left[\bar{\mathbf{h}}_{IT,g}\right]_{n_G}\right)\Bigg]\\
%& =\text{E}\left[\left|\sum_{g=1}^{G}\left[\bar{\mathbf{h}}_{RI,g}\right]_{1}\left[\mathbf{d}_g\right]_{1}\left[\bar{\mathbf{h}}_{IT,g}\right]_{1}\right|^2\right]+\text{E}\left[\left|\sum_{g=1}^{G}\sum_{n_G=2}^{N_G}\left[\bar{\mathbf{h}}_{RI,g}\right]_{n_G}\left[\bar{\mathbf{h}}_{IT,g}\right]_{n_G}\right|^2\right]\\
& >\text{E}\left[\left|\sum_{g=1}^{G}\left[\bar{\mathbf{h}}_{RI,g}\right]_1\left[\mathbf{d}_g\right]_1\left[\bar{\mathbf{h}}_{IT,g}\right]_1\right|^2\right]\\
& =\text{E}\left[\left|\sum_{g=1}^{G}\left|\left[\bar{\mathbf{h}}_{RI,g}\right]_1\right|\left|\left[\bar{\mathbf{h}}_{IT,g}\right]_1\right|e^{j\left[\text{arg}\left(\left[\mathbf{d}_g\right]_1\right)-\theta_g^\star\right]}\right|^2\right],\label{eq:last-P}
%& =\text{E}\left[\left|\sum_{g=1}^{G}\left|\left[\bar{\mathbf{h}}_{RI,g}\right]_1\right|\left|\left[\bar{\mathbf{h}}_{IT,g}\right]_1\right|e^{j\bar{\theta}_g}\right|^2\right]
\end{align}
where we introduced $\theta_g^\star=-\text{arg}\left([\bar{\mathbf{h}}_{RI,g}]_1\right)-\text{arg}\left([\bar{\mathbf{h}}_{IT,g}]_1\right)$.
We denote as $\theta_1=\arg\left(\frac{j[\boldsymbol{\lambda}_g^+]_1-Z_{0}}{j[\boldsymbol{\lambda}_g^+]_1+Z_{0}}\right)\in\left(0,\pi\right)$, where $[\boldsymbol{\lambda}_g^+]_1$ is the non zero eigenvalue of $\mathbf{X}_{I,g}^+=+c_1\text{ones}\left(N_G\right)$.
Thus, we have $-\theta_1=\arg\left(\frac{j[\boldsymbol{\lambda}_g^-]_1-Z_{0}}{j[\boldsymbol{\lambda}_g^-]_1+Z_{0}}\right)\in\left(-\pi,0\right)$, where $[\boldsymbol{\lambda}_g^-]_1=-[\boldsymbol{\lambda}_g^+]_1$ is the non zero eigenvalue of $\mathbf{X}_{I,g}^-=-c_1\text{ones}\left(N_G\right)$.
We consider a quantization approach giving $\text{arg}\left([\mathbf{d}_g]_1\right)=\theta_1$ if $\theta_g^\star\in\left[0,\pi\right)$, or $\text{arg}\left([\mathbf{d}_g]_1\right)=-\theta_1$ otherwise.
The quantization error is denoted as $\bar{\theta}_g=\text{arg}\left([\mathbf{d}_g]_1\right)-\theta_g^\star$.
Thus, \eqref{eq:last-P} can be rewritten as
\begin{multline}
\text{E}\left[P_{R}\right]
>\text{E}\Bigg[\sum_{g=1}^{G}\left|\left[\bar{\mathbf{h}}_{RI,g}\right]_1\right|^2\left|\left[\bar{\mathbf{h}}_{IT,g}\right]_1\right|^2\\
+\sum_{g\neq f}\left|\left[\bar{\mathbf{h}}_{RI,g}\right]_1\right|\left|\left[\bar{\mathbf{h}}_{IT,g}\right]_1\right|\left|\left[\bar{\mathbf{h}}_{RI,f}\right]_1\right|\left|\left[\bar{\mathbf{h}}_{IT,f}\right]_1\right|e^{j\bar{\theta}_g-j\bar{\theta}_f}\Bigg].
\end{multline}
Noting that $\left|[\bar{\mathbf{h}}_{RI,g}]_1\right|$, $\left|[\bar{\mathbf{h}}_{IT,g}]_1\right|$, and $e^{j\bar{\theta}_g}$ are independent with each other, with $\text{E}\left[\left|[\bar{\mathbf{h}}_{RI,g}]_1\right|^2\right]=1$, $\text{E}\left[\left|[\bar{\mathbf{h}}_{RI,g}]_1\right|\right]=\sqrt{\pi}/2$, and $\text{E}\left[e^{j\bar{\theta}_g}\right]=\text{E}\left[e^{-j\bar{\theta}_g}\right]=\frac{2}{\pi}\sin\left(\theta_1\right)$, we have
\begin{align}
\text{E}\left[P_{R}\right]
& >G+G\left(G-1\right)\left(\frac{\sqrt{\pi}}{2}\right)^4\left(\frac{2}{\pi}\sin\left(\theta_1\right)\right)^2\\
& >N_I^2\left(\frac{\sin\left(\theta_1\right)}{2N_G^2}\right)^2.\label{eq:O(N^2)}
\end{align}
Since $\sin\left(\theta_1\right)>0$, \eqref{eq:O(N^2)} proves that $\mathcal{C}$ can achieve an average received signal power growth of $\mathcal{O}\left(N_I^2\right)$ as $N_I$ increases.
\end{proof}

% Conclusion of the proposition
According to Proposition~\ref{pro:1}, the entries of an RIS reactance matrix can be chosen from a codebook $\mathcal{C}$ with no degradation in the received signal power growth.
This justifies the use of a symmetric codebook for our discrete-value design strategy.

%%%%%%%%%%%%%%%%%%%%%%%%%%%%%%%%%%%%%%%%%%%%%%%%%%
\section{Vector-Discrete Group/Fully Connected RIS}
\label{sec:vector-discrete}

% Introduce possibility to generalize the quantization to vector quantization
% Then, state the new optimization problem and the new complexity
%In the previous subsection, the scalar codebook $\mathcal{C}$ has been introduced to define the possible values that each element in $\mathbf{X}_{I}$ can assume.
In this section, we propose a second discretization strategy based on vector quantization.
To realize vector-discrete RISs, we assign $B_V$ resolution bits to each reactance matrix block $\mathbf{X}_{I,g}$.
Thus, we introduce the codebook $\mathcal{C}_V$ as a set of vectors
\begin{equation}
\mathcal{C}_{V} = \left\{\mathbf{c}_{1},\mathbf{c}_{2},\ldots,\mathbf{c}_{2^{B_V}}\right\}.
\end{equation}
where $\mathbf{c}_{k}\in\mathbb{R}^{\frac{N_{G}\left(N_{G}+1\right)}{2}}$, for $k=1,\ldots,2^{B_V}$.
Here, $\mathbf{c}_{k}$ is a possible value that the upper triangular part of each block $\mathbf{X}_{I,g}$ can assume.

\subsection{Offline Learning}

Similarly to the scalar discretization case, $\mathcal{C}_V$ can be obtained with $K$-means clustering through an offline learning phase, with the difference that a multi-dimensional feature space with dimension $N_{G}\left(N_{G}+1\right)/2$ is now considered, and $K=2^{B_{V}}$.
%
% Offline Learning
To construct the codebook $\mathcal{C}_V$, let us consider the set $\mathcal{X}_I$ containing the optimal reactance matrices associated with the $N_0$ training channel triplets.
To formalize the $K$-means clustering problem, we introduce the matrices $\mathbf{X}^{[n_0]}\in\mathbb{R}^{N_{G}\left(N_{G}+1\right)/2\times G}$ containing in their $g$th column the entries of the upper triangular part of $\mathbf{X}_{I,g}^{[n_0]}$, for $g=1,\ldots,G$.
Then, the $N_0$ matrices $\mathbf{X}^{[n_0]}$ are concatenated in $\mathbf{X}_0\in\mathbb{R}^{N_{G}\left(N_{G}+1\right)/2\times N_0G}$ as $\mathbf{X}_0=[\mathbf{X}^{[1]},\ldots,\mathbf{X}^{[N_0]}]$.
In this way, the resulting $K$-means clustering problem writes as
\begin{equation}
\{\mathbf{c}_1^\star,\ldots,\mathbf{c}_K^\star\}=\argmin_{\{\mathbf{c}_1,\ldots,\mathbf{c}_K\}}\;\;\sum_{n=1}^{N_0G}\sum_{k=1}^{2^{B_V}}r_{nk}\left\vert \mathbf{x}_n-\mathbf{c}_k\right\vert^2,\label{eq:BV}
\end{equation}
where the $n$th data point $\mathbf{x}_n\in\mathbb{R}^{N_{G}\left(N_{G}+1\right)/2\times 1}$ is the $n$th column of the matrix $\mathbf{X}_0$.
%To solve \eqref{eq:BV}, we iteratively optimize the sets $\{r_{nk}\}$ and $\{\mathbf{c}_1,\ldots,\mathbf{c}_K\}$, similarly as done in subsection~\ref{sec:scalar-discrete-multi}.
Remarkably, \eqref{eq:BV} is in the form of a $K$-means clustering problem.
Thus, we solve \eqref{eq:BV} by alternatively optimizing the sets $\{r_{nk}\}$ and $\{\mathbf{c}_1,\ldots,\mathbf{c}_K\}$, as described in the following.
Firstly, with fixed $\{\mathbf{c}_1,\ldots,\mathbf{c}_K\}$ are fixed, we set $\{r_{nk}\}$ by assigning the $n$th data point $\mathbf{x}_n$ to the closest cluster center $\mathbf{c}_k$, yielding
\begin{equation}
r_{nk}=
\begin{cases}
1 & \text{if } k=\arg \min_{j}\left\Vert \mathbf{x}_n-\mathbf{c}_j\right\Vert^2\\
0 & \text{otherwise}
\end{cases},
\end{equation}
for $n=1,\ldots,N_0G$ and $k=1,\ldots,K$.
Secondly, with fixed $\{r_{nk}\}$, the cluster centers $\{\mathbf{c}_1,\ldots,\mathbf{c}_K\}$ are optimally updated in closed-form as
\begin{equation}
\mathbf{c}_k=\frac{\sum_n r_{nk}\mathbf{x}_n}{\sum_n r_{nk}},
\end{equation}
for $k=1,\ldots,K$.
These two steps are alternatively repeated until convergence.
Finally, the vector-quantization codebook is given by $\mathcal{C}_V=\{\mathbf{c}_1^\star,\ldots,\mathbf{c}_{K}^\star\}$.

The complexity of the offline learning for vector-discrete group and fully connected RISs is given by $\mathcal{O}(N_0(N_I(N_G+1)/2)^2+N_0N_I(N_G+1)2^{B_V-1})$, where the first term is due to solving \eqref{eq:P-GC-C-obj}-\eqref{eq:P-GC-C-con3} $N_0$ times to find $\mathcal{X}_I$, and the second term is due to the $K$-means clustering algorithm used to solve \eqref{eq:BV}.

\begin{figure*}[t]
    \begin{centering} % all 7cm
    \includegraphics[width=0.24\textwidth]{figs_v6/fig-bit1-iid-MIMO-hRT.eps}
    \includegraphics[width=0.24\textwidth]{figs_v6/fig-bit2-iid-MIMO-hRT.eps}
    \includegraphics[width=0.24\textwidth]{figs_v6/fig-bit3-iid-MIMO-hRT.eps}
    \includegraphics[width=0.24\textwidth]{figs_v6/fig-bit4-iid-MIMO-hRT.eps}
    \par\end{centering}
    \vspace{0.1cm}
    \begin{centering}
    \includegraphics[width=0.24\textwidth]{figs_v6/fig-bit1-q-UG-MIMO-hRT.eps}
    \includegraphics[width=0.24\textwidth]{figs_v6/fig-bit2-q-UG-MIMO-hRT.eps}
    \includegraphics[width=0.24\textwidth]{figs_v6/fig-bit3-q-UG-MIMO-hRT.eps}
    \includegraphics[width=0.24\textwidth]{figs_v6/fig-bit4-q-UG-MIMO-hRT.eps}
    \par\end{centering}
    \caption{Average received signal power versus the number of RIS elements for different values of $B$.}
    \label{fig:bit}
\end{figure*}

\subsection{Online Deployment}

Once the codebook has been learned offline, the online optimization problem is formulated as
\begin{align}
\underset{\mathbf{X}_{I,g}}{\mathsf{\mathrm{max}}}\;\;
& \left\|\mathbf{H}_{RT}+\mathbf{H}_{RI}\boldsymbol{\Theta}\mathbf{H}_{IT}\right\|^{2}\label{eq:P-GC-D-obj4}\\
\mathsf{\mathrm{s.t.}}\;\;\;
& \boldsymbol{\Theta}=\mathrm{diag}\left(\boldsymbol{\Theta}_{1},\ldots,\boldsymbol{\Theta}_{G}\right),\label{eq:P-GC-D-con4-1}\\
& \boldsymbol{\Theta}_{g}=\left(j\mathbf{X}_{I,g}+Z_{0}\mathbf{I}\right)^{-1}\left(j\mathbf{X}_{I,g}-Z_{0}\mathbf{I}\right),\:\forall g,\label{eq:P-GC-D-con4-2}\\
& \mathbf{X}_{I,g}=\mathbf{X}_{I,g}^{T},\:\forall g,\label{eq:P-GC-D-con4-3}\\
& \mathbf{X}_{I,g}\in\left\{\mathbf{c}_1^\star,\ldots,\mathbf{c}_{2^{B_V}}^\star\right\},\:\forall g.\label{eq:P-GC-D-con4-4}
\end{align}
Alternating optimization is the selected strategy to solve this problem, in which the $G$ blocks $\mathbf{X}_{I,g}$ are iteratively optimized by searching among their possible $2^{B_V}$ values contained in $\mathcal{C}_V$.
The convergence of this iterative process can be proved as done for scalar-discrete RISs in Section~III~E.
The proposed design of vector-discrete RISs is summarized in Alg.~\ref{alg:vector}, where problem \eqref{eq:BV} is solved offline to design the codebook while \eqref{eq:P-GC-D-obj4}-\eqref{eq:P-GC-D-con4-4} is solved online to optimize the RIS reactance matrix.

\begin{algorithm}[t]
\KwIn{$B_V$, $\mathcal{H}_0$, $\mathbf{H}_{RT}$, $\mathbf{H}_{RI}$, $\mathbf{H}_{IT}$, $N_G$}
\KwOut{$\mathcal{C}$, $\mathbf{X}_{I}$}
\nonl Offline Learning\\
Calculate $\mathcal{X}_I$ by \eqref{eq:P-GC-C-obj}-\eqref{eq:P-GC-C-con3}\;
Calculate $\{\mathbf{c}_1^\star,\ldots,\mathbf{c}_K^\star\}$ by iteratively solving \eqref{eq:BV}, with $K=2^{B_V}$\;
$\mathcal{C}\leftarrow\{\mathbf{c}_{1}^\star,\mathbf{c}_{2}^\star,\ldots,\mathbf{c}_{K}^\star\}$\;
\nonl Online Deployment\\
Initialize $\mathbf{X}_{I,g}=\mathbf{0},\:\forall g$\;
\While{no convergence of objective \eqref{eq:P-GC-D-obj4}}{
\For{$g\leftarrow 1$ \KwTo $G$}{
Update $\mathbf{X}_{I,g}\in \mathcal{C}_V$ by searching among the $2^{B_V}$ possible values\;
}
}
\caption{Vector-Discrete RISs Design}
\label{alg:vector}
\end{algorithm}

% Drawback = complexity and benefits
The cost of vector-discrete RISs relies on their optimization complexity.
For example, if $B$ bits are allocated to each reactance element, the number of clusters needed to quantize a vector of $N_{G}\left(N_{G}+1\right)/2$ elements is $k=2^{B\frac{N_{G}\left(N_{G}+1\right)}{2}}$, growing exponentially with the square of $N_{G}$.
Consequently, the number of search times in a complete iteration of the alternating optimization algorithm becomes $G2^{B\frac{N_{G}\left(N_{G}+1\right)}{2}}$.
However, this discretization strategy based on vector quantization brings two benefits.
First, for the same number of total resolution bits, vector quantization is inherently more efficient than scalar quantization.
Second, the number of total resolution bits can be chosen with more degrees of freedom, since it is no longer limited to $B$ bits for each reactance element.

%%%%%%%%%%%%%%%%%%%%%%%%%%%%%%%%%%%%%%%%%%%%%%%%%%
\section{Performance Evaluation}
\label{sec:results}

% Received power definition
In this section, we evaluate the performance of an RIS-aided \gls{mimo} system with different discrete RIS configurations, each characterized by its number of elements $N_{I}$, number of resolution bits, and group size $N_{G}$.
The performance is measured in terms of received signal power, given by $P_{R}=P_{T}\left\|\mathbf{H}_{RT}+\mathbf{H}_{RI}\boldsymbol{\Theta}\mathbf{H}_{IT}\right\|^{2}$.

\begin{figure*}[t]
    \begin{centering} % all 7cm
    \includegraphics[width=0.24\textwidth]{figs_v6/fig-group2-iid-MIMO-hRT.eps}
    \includegraphics[width=0.24\textwidth]{figs_v6/fig-group4-iid-MIMO-hRT.eps}
    \includegraphics[width=0.24\textwidth]{figs_v6/fig-group8-iid-MIMO-hRT.eps}
    \includegraphics[width=0.24\textwidth]{figs_v6/fig-group0-iid-MIMO-hRT.eps}
    \par\end{centering}
    \vspace{0.1cm}
    \begin{centering}
    \includegraphics[width=0.24\textwidth]{figs_v6/fig-group2-q-UG-MIMO-hRT.eps}
    \includegraphics[width=0.24\textwidth]{figs_v6/fig-group4-q-UG-MIMO-hRT.eps}
    \includegraphics[width=0.24\textwidth]{figs_v6/fig-group8-q-UG-MIMO-hRT.eps}
    \includegraphics[width=0.24\textwidth]{figs_v6/fig-group0-q-CG-MIMO-hRT.eps}
    \par\end{centering}
    \caption{Average received signal power versus the number of RIS elements for different values of group size.}
    \label{fig:group-size}
\end{figure*}

\subsection{Numerical Simulation Setup}

% Scenario
Let us consider a three-dimensional coordinate system $(x,y,z)$, in which the $z$-axis represents the height above the ground in meters (m).
The transmitter, the RIS, and the receiver are \glspl{ula} composed of $N_{T}$, $N_{I}$, and $N_{R}$ antennas, respectively, with half wavelength antenna spacing\footnote{Our discrete design for BD-RIS is valid for any RIS array shape.}.
We set $N_{T}=4$ and $N_{R}=2$, while considering several values of $N_{I}$ as specified in the following.
The transmitter is located at $(5,-250,25)$ and its antennas are arranged along the $x$-axis.
The RIS is located at $(0,0,5)$ and its antennas are arranged along the $y$-axis.
Finally, the receiver is located at $(5,5,1.5)$ with random orientation.
The path loss is modeled as $L_{ij}(d_{ij})=L_{0}d_{ij}^{-\alpha_{ij}}$, where $L_{0}$ is the path loss at distance 1 m, $d_{ij}$ is the distance, and $\alpha_{ij}$ is the path loss exponent for $ij\in\{RT,RI,IT\}$.
We set $L_{0}=-30$ dB, $\alpha_{RT}=4$, $\alpha_{RI}=2.8$, $\alpha_{IT}=2$, and $P_{T}=10$ W. 

% Two small-scale fading considered
We analyze two scenarios with different small-scale fading effects: i.i.d. Rayleigh fading and correlated fading.
In the former scenario, the small-scale fading of all channels is assumed to be i.i.d. Rayleigh distributed.
Thus, we have $\mathbf{H}_{RT}\sim\mathcal{CN}\left(\boldsymbol{0},L_{RT}\mathbf{I}\right)$, $\mathbf{H}_{RI}\sim\mathcal{CN}\left(\boldsymbol{0},L_{RI}\mathbf{I}\right)$ and $\mathbf{H}_{IT}\sim\mathcal{CN}\left(\boldsymbol{0},L_{IT}\mathbf{I}\right)$.
In the latter scenario, we generate the small-scale effects with QuaDRiGa version 2.4, a \textsc{Matlab} based statistical ray-tracing channel simulator \cite{jae14}.
An urban macrocell propagation environment is simulated, with \gls{los} in the transmitter-RIS link, and with \gls{nlos} in the transmitter-receiver and RIS-receiver links.
The channel models ``3GPP\_38.901\_UMa\_LOS'' and ``3GPP\_38.901\_UMa\_NLOS'' have been used to simulate the \gls{los} and \gls{nlos} channels, respectively.
The \glspl{ula} at the transmitter, the RIS, and the receiver are modeled as ``3gpp-3d'' array and the carrier frequency is set to $f_c=2.5$ GHz.
%The performance of both uni-polarized and dual-polarized RIS is investigated.
%In the former case, the antenna elements of the RIS are all vertically polarized.
%In the latter case, the elements are $\pm45^\circ$ polarized.
%
% Normalization
%The channels generated with Quadriga can be expressed as $\widetilde{\mathbf{h}}_{RI}=\Lambda_{RI}^{-1/2}\mathbf{h}_{RI}$ and $\widetilde{\mathbf{h}}_{IT}=\Lambda_{IT}^{-1/2}\mathbf{h}_{IT}$, where the scalars $\Lambda_{RI}^{-1/2}$ and $\Lambda_{IT}^{-1/2}$ contain the path loss and the shadowing, while $\mathbf{h}_{RI}$ and $\mathbf{h}_{IT}$ are the fading effects.
%To obtain a received signal power purely dependent on the RIS configuration, the dataset has been normalized according to 
%\begin{equation}
%\mathbf{h}_{RI}\leftarrow\Lambda_{RI}^{1/2}\widetilde{\mathbf{h}}_{RI},\:\mathbf{h}_{IT}\leftarrow\Lambda_{IT}^{1/2}\widetilde{\mathbf{h}}_{IT},
%\end{equation}
%such that $\left\Vert\mathbf{h}_{RI}\right\Vert^{2}=N_{I}$ and $\left\Vert\mathbf{h}_{IT}\right\Vert^{2}=N_{I}$ for every channel realization.

% Offline training and online deployment
For both scenarios, offline learning stages have been carried out to define the optimal codebooks.
We employ a training set $\mathcal{H}_0$ composed of $N_0=100$ channel realization triplets for scalar-discrete RISs, and $N_0=500$ triplets for vector-discrete RISs.
These values of $N_0$ have been set such that in the $K$-means clustering problems involved in Alg.~\ref{alg:scalar} and Alg.~\ref{alg:vector}, the number of training samples is always higher than the number of clusters.
In the alternating optimizations, convergence is considered reached when the fractional increase of the objective value is below $\epsilon=10^{-3}$.
Finally, the average received signal power has been computed for each RIS configuration using the Monte Carlo method.

\subsection{Comparison With Single Connected RIS}

Before comparing discrete-value group and fully connected RISs with discrete-value single connected RISs, we briefly review how single connected RISs have been discretized in the literature.
In single connected RISs, $\boldsymbol{\Theta}$ is completely described by the phase shifts $\theta_{n_{I}}$.
The phase shifts have been typically discretized uniformly in $\left[0,2\pi\right)$.
By assigning $B$ resolution bits per phase shift, they are chosen in the codebook
\begin{equation}
\mathcal{C}^{\text{Single}} = \left\{ 0,\delta,\ldots,\left( 2^{B} - 1 \right) \delta \right\},
\end{equation}
where $\delta = \frac{2\pi}{2^{B}}$ is the quantization step.
Thus, the received signal power maximization problem is formulated in the discrete case as
\begin{align}
\underset{\theta_{n_{I}}}{\mathsf{\mathrm{max}}}\;\;
& \left\|\mathbf{H}_{RT}+\mathbf{H}_{RI}\boldsymbol{\Theta}\mathbf{H}_{IT}\right\|^{2}\label{eq:P-SC-D-obj}\\
\mathsf{\mathrm{s.t.}}\;\;\;
& \boldsymbol{\Theta}=\mathrm{diag}\left(e^{j\theta_{1}},\ldots,e^{j\theta_{N_{I}}}\right),\label{eq:P-SC-D-con1}\\
& \theta_{n_{I}}\in \left\{ 0,\delta,\ldots,\left( 2^{B} - 1 \right) \delta \right\},\:\forall n_{I}.\label{eq:P-SC-D-con2}
\end{align}
To solve this problem, we employ the widely adopted alternating optimization method \cite{you20}, \cite{abe20}, \cite{wu19b}.
In this method, each of the $N_I$ phase shifts is optimized individually by searching among all the possible $2^B$ values, while fixing the other $N_I-1$ phase shifts.
This procedure is repeated for all $N_I$ phase shifts, iterating multiple times until the convergence is reached, i.e., until the fractional increase of the objective value is below a certain parameter $\epsilon$.

\begin{figure*}[t]
    \centering
    \includegraphics[height=0.3\textwidth]{figs_v7/fig-ni64-iid-MIMO-hRT-v2.eps}
    \includegraphics[height=0.3\textwidth]{figs_v7/fig-ni64-q-UG-MIMO-hRT-v2.eps}
    \caption{Average received signal power versus the group size.}
    \label{fig:ni64}
\end{figure*}
\begin{figure*}[t]
    \centering
    \includegraphics[height=0.3\textwidth]{figs_v7/fig-ni64-vs-bits-iid-MIMO-hRT-v2.eps}
    \includegraphics[height=0.3\textwidth]{figs_v7/fig-ni64-vs-bits-q-UG-MIMO-hRT-v2.eps}
    \caption{Average received signal power versus the number of total resolution bits.}
    \label{fig:ni64-vs-bits}
\end{figure*}

\subsection{Scalar-Discrete Group/Fully Connected RIS}

% Single connected: discrete + continuous
% Single connected discussion
We first evaluate the received signal power in the case of scalar-discrete RISs.
In Fig.~\ref{fig:bit}, we show  the received signal power for different values of $N_{I}$, using from one to four resolution bits per reactance element.
Since this is the first work investigating the design of BD-RISs based on discrete values, the only available benchmark is given by the performance achieved by the same BD-RISs optimized with continuous values.
In each subfigure, we compare single connected RISs, group connected RIS with $N_{G}\in\{2,4,8\}$, and fully connected RISs.
The fully connected architecture achieves the highest received signal power, as in the case of continuous-value RISs \cite{she20}.
We also observe that a larger group size yields a higher received signal power.
Thus, group connected RISs always perform better than the conventional single connected RISs.
Furthermore, the discrepancy in the performance of fully and single connected is higher when fewer resolution bits are employed.
With correlated small-scale fading, the received signal power is higher than with i.i.d. Rayleigh fading channels.
This is because dominant eigenmode transmission achieves a received signal power proportional to the dominant eigenvalue of $\mathbf{H}\mathbf{H}^H$.
Thus, correlated channels, being low-rank, are beneficial compared to independent channels.

% Group 2, 4, 8, and fully connected: discrete + continuous
% Group 2, 4, 8, and fully connected discussion: fewer bits are necessary for higher group sizes
In Fig.~\ref{fig:group-size}, we show the received signal power when group connected (with group size $N_{G}\in\{2,4,8\}$) and fully connected RISs are used in the presence of i.i.d. Rayleigh fading channels and correlated channels.
The performance of scalar-discrete RISs is compared with the performance of continuous-value RISs, optimized by solving \eqref{eq:P-GC-C-obj}-\eqref{eq:P-GC-C-con3}.
We observe that the higher the group size, the fewer resolution bits are necessary to reach the performance of continuous-value RISs.
The rationale behind this behavior is given by the necessary and sufficient condition for $\boldsymbol{\Theta}$ to achieve the maximum $P_R$ with no direct link.
By extending the analysis in \cite{she20} to \gls{mimo} settings, it is possible to prove that this condition is given by
\begin{equation}
\mathbf{u}_{RI,g}=\boldsymbol{\Theta}_{g}\mathbf{u}_{IT,g},\forall g,
\label{eq:opt-cond-u}
\end{equation}
where $\mathbf{u}_{RI}=\left[\mathbf{u}_{RI,1},\mathbf{u}_{RI,2},\ldots,\mathbf{u}_{RI,G}\right]$ with $\mathbf{u}_{RI,g}\in\mathbb{C}^{N_{G}\times1}$, and $\mathbf{u}_{IT}=\left[\mathbf{u}_{IT,1},\mathbf{u}_{IT,2},\ldots,\mathbf{u}_{IT,G}\right]^{T}$ with $\mathbf{u}_{IT,g}\in\mathbb{C}^{N_{G}\times1}$, are the dominant left singular vectors of $\mathbf{H}_{RI}^{H}$ and $\mathbf{H}_{IT}$, respectively\footnote{
Condition \eqref{eq:opt-cond-u} is derived by considering $P_{R}\leq P_{T}\underset{\left\|\mathbf{x}\right\|=1}{\mathsf{\mathrm{max}}}\;\left\|\mathbf{H}_{RI}\right\|^{2}\left\|\boldsymbol{\Theta}\mathbf{H}_{IT}\mathbf{x}\right\|^{2}\leq P_{T}\underset{\left\|\mathbf{x}\right\|=1}{\mathsf{\mathrm{max}}}\;\left\|\mathbf{H}_{RI}\right\|^{2}\left\|\boldsymbol{\Theta}\mathbf{H}_{IT}\right\|^{2}\left\|\mathbf{x}\right\|^{2}$.
Note that the equality holds in the two inequalities when $\mathbf{u}_{RI}$ is equal to the dominant left singular vector of $\boldsymbol{\Theta}\mathbf{H}_{IT}$, i.e., $\boldsymbol{\Theta}\mathbf{u}_{IT}$, which is equivalent to \eqref{eq:opt-cond-u} in the case of block diagonal $\boldsymbol{\Theta}$.}.
%Here, $\mathbf{u}_{RI,g}\in\mathbb{C}^{N_{G}\times1}$ and $\mathbf{u}_{IT,g}\in\mathbb{C}^{N_{G}\times1}$ contain the $N_{G}$ elements of $\mathbf{h}_{RI}$ and $\mathbf{h}_{IT}$, respectively, which are grouped into the $g$th group.
This condition consists of an underdetermined system of $N_{I}$ equations in $N_{I}\left(N_{G}+1\right)/2$ unknowns.
Thus, a higher group size $N_{G}$ implies more degrees of freedom, and fewer resolution bits are required to satisfy it.
In particular, in the fully connected architecture, a single resolution bit is sufficient to obtain approximately the same performance as the optimized continuous-value RISs.
%
% Difference with \cite{jun21}
It is worthwhile to clarify the difference between this conclusion and the results in \cite{jun21}.
In \cite{jun21}, the authors prove that single connected RISs with discrete phase shifts can achieve an \gls{snr} in the order of $\mathcal{O}\left(N_{I}^2\right)$, regardless the number of bits $B\geq 1$ (see Proposition 1 in \cite{jun21}).
This means that a single resolution bit ($B = 1$) can achieve the same performance growth as $B=\infty$ in the limit $N_{I}\rightarrow\infty$.
However, when a practical number of RIS elements is considered, $B = 1$ cannot achieve the same performance as $B=\infty$.
Furthermore, even if $B = 1$ and $B=\infty$ achieve the same asymptotic \gls{snr} growth, $B = 1$ causes a power loss of 3.9 dB \cite{wu19b}.
Conversely, in this work, we show that $B = 1$ can achieve approximately the same performance as $B=\infty$ in fully connected architectures for any $N_{I}$, as represented in Fig.~\ref{fig:group-size}.

\begin{figure*}[t]
    \centering
    \includegraphics[height=0.3\textwidth]{figs_v7/fig-ni64-vs-complexity-iid-MIMO-hRT.eps}
    \includegraphics[height=0.3\textwidth]{figs_v7/fig-ni64-vs-complexity-q-UG-MIMO-hRT.eps}
    \caption{Average received signal power versus the computational complexity.}
    \label{fig:ni64-vs-compl}
\end{figure*}

% Plot of different group sizes with NI=64
We now compare the performance obtained with different group sizes when the number of RIS elements $N_{I}$ is fixed.
To this end, we optimize RISs with $N_{I}=64$ elements considering all the possible group sizes, spanning from the single connected architecture ($N_{G}=1$) to the fully connected ($N_{G}=64$).
In Fig.~\ref{fig:ni64}, the upper bound of the received signal power is reported together with the power maximized with different resolutions.
%, in the two considered scenarios: i.i.d. Rayleigh fading channels and correlated channels.
%For $N_{G}\geq2$,
We notice that the received signal power increases with the group size for every $B$ considered.
Furthermore, the performance of discrete-value RISs with $B=1$ approaches the performance of continuous-value RISs as the group size increases, converging to it in the fully connected architecture.
%In the single connected case, the matrix $\boldsymbol{\Theta}$ is completely described by its phase shifts $\theta_{n_{I}}\in\left[0,2\pi\right)$, for $n_{I}=1,\ldots,N_{I}$, which can be directly quantized.
%Because of this property, the curves in Fig.~\ref{fig:ni64} exhibit a different trend when $N_{G}=1$.
%In fact, despite a four-bit resolution is not sufficient to achieve the upper bound when $N_{G}=2$, it reaches the optimality when $N_{G}=1$.

% Plot of different group sizes with NI=64, vs total bits
When $B$ bits are allocated to each $\mathbf{X}_{I}$ element, the number of total bits employed scales with the number of $\mathbf{X}_{I}$ elements as $B\frac{\left(N_{G}+1\right)}{2}N_{I}$.
Fixing $N_{I}=64$, we now investigate which are the RIS configurations, described by the pairs $(N_{G},B)$, that maximize the received signal power when the number of total bits is limited.
Fig.~\ref{fig:ni64-vs-bits} shows how the single, group, and fully connected architectures with discrete reconfigurable impedances can provide the compromise between performance and hardware complexity.
Here, each point represents the received signal power achievable with a specific RIS configuration, when $B\frac{\left(N_{G}+1\right)}{2}\times 64$ total resolution bits are available.
%Here, different colors represent different resolutions $B$, while the numeric labels above the points identify the group size $N_{G}$.
Furthermore, the numeric labels above the points identify the group size $N_{G}$.
We observe that RIS configurations with small values of $B$ are selected as optimal in case of a limited number of total bits.
Therefore, RISs with larger group sizes and lower resolution are generally preferred over RISs with smaller group sizes and higher resolution.
This result confirms the effectiveness of group and fully connected RISs.
%The power is maximized by the fully connected architecture ($N_{G}=64$) with $B=1$.
%However, to realize this RIS configuration, the number of total resolution bits required is $32.5\times 64=2080$.
%Since a simpler architecture may be needed for practical reasons, 
%Here, each of the four curves represents the performance obtained with the eight group sizes (from 1 to 64), by using a specific $B$.
%Among all the possible RIS configurations, we highlight which are the optimal strategies that should be used when a certain number of total resolution bits is available.
%We observe that all the configurations lying on the 1-bit line are marked as optimal.
%In other words, to maximize the received signal power, the largest group size allowed by the number of available bits should be always used, even with the lowest resolution $B=1$.
%In the case of correlated fading, the RIS configurations designed with optimized grouping achieve a higher received signal power than their corresponding configurations using a grouping based on the channel statistics.
%Thus, we can conclude that optimized grouping is the best grouping strategy when scalar-discrete RISs are employed.
%Furthermore, among the two grouping strategies based on the channel statistics considered, uncorrelated grouping is the preferred one.

We now discuss the trade-off between computational complexity and performance.
To this end, we analyze the optimization computational complexity and the received signal power of all RIS configurations with $N_I=64$ elements, each described by the group size $N_G$ and the number of resolution bits $B$.
In Fig.~\ref{fig:ni64-vs-compl}, we report the RIS architectures achieving the most favorable trade-off between performance and computational complexity. 
Here, each point represents the received signal power achievable with a specific RIS configuration, requiring $N_{I}\frac{\left(N_{G}+1\right)}{2}2^{B}$ search times per iteration of our optimization algorithm.
Furthermore, the numeric labels above the points identify the group size $N_G$.
We observe that RIS configurations with small values of $B$ are the ones offering the best trade-off between performance and computational complexity.
This result confirms the effectiveness of group and fully connected RISs over single connected RISs also under the constraint of discrete values.

\begin{figure*}[t]
    \centering
    \includegraphics[height=0.3\textwidth]{figs_v7/fig-vec-group2-iid-MIMO-hRT-v2.eps}
    \includegraphics[height=0.3\textwidth]{figs_v7/fig-vec-group2-q-UG-MIMO-hRT-v2.eps}
    \caption{Average received signal power versus the number of RIS elements in the scalar-discrete and vector-discrete group connected architectures.}
    \label{fig:vector-discrete-ng2}
\end{figure*}
\begin{figure*}[t]
    \centering
    \includegraphics[height=0.3\textwidth]{figs_v7/fig-vec-B0-iid-MIMO-hRT-v2.eps}
    \includegraphics[height=0.3\textwidth]{figs_v7/fig-vec-B0-q-UG-MIMO-hRT-v2.eps}
    \caption{Average received signal power versus the number of RIS elements for different group sizes, with $N_{I}$ total resolution bits.}
    \label{fig:vector-discrete-B0}
\end{figure*}

\subsection{Vector-Discrete Group/Fully Connected RIS}

% Comparison with scalar-discrete
Let us analyze the performance of vector-discrete RISs, in comparison with scalar-discrete RISs.
To this end, we set the number of resolution bits assigned to each block $\mathbf{X}_{I,g}$ to $B_V=B\frac{N_{G}\left(N_{G}+1\right)}{2}$.
This is equivalent, in terms of total bits employed, to assigning $B$ bits to each reactance element, allowing a comparison between scalar-discrete and vector-discrete RISs.
Fig.~\ref{fig:vector-discrete-ng2} shows the performance achieved by vector-discrete group connected RISs with group size $N_{G}=2$, when the number of resolution bits considered is $B_V\in\{3,6,9\}$.
In terms of total resolution bits employed, $B_V\in\{3,6,9\}$ bits per impedance network block are equivalent to $B\in\{1,2,3\}$ bits per impedance element, respectively.
Thus, for the sake of comparison, we report in Fig.~\ref{fig:vector-discrete-ng2} also the performance of scalar-discrete group connected RISs with $B\in\{1,2,3\}$.
We notice that vector discretization brings a significant performance improvement over scalar discretization.
In particular, $B_V=9$ resolution bits per impedance network block are sufficient to approach the performance of continuous-value RISs.
However, $B=3$ resolution bits per impedance element are not sufficient to reach the performance of continuous-value RISs when scalar discretization is considered.
%Furthermore, we observe that correlated grouping achieves better performance than uncorrelated grouping when $B_V=3$, in both uni-polarized and dual-polarized RISs.
%This is because with correlated grouping there is a higher correlation between the optimal reactance elements within each group $\mathbf{X}_{I,g}$.
%Thus, vector-quantization can benefit from this property, achieving better performance in this case than in the uncorrelated grouping case, when a low resolution is used.
%This is different from what observed for scalar-discrete RISs, where uncorrelated grouping is always the best grouping strategy.



% Comparison with single connected
We now compare the performance of RISs with different group sizes when the same number of total bits is employed.
%To this end, vector discretization has to be considered since it offers the needed flexibility in defining the resolution.
%We fix the number of total bits used to $N_{I}$, which is the minimum possible in the case of scalar-discrete RISs.
%Indeed, $N_{I}$ total bits are required by the single connected architecture with $B=1$.
To this end, we fix the number of total bits used to $N_{I}$, which are required by the single connected architecture with $B=1$.
To use $N_{I}$ total bits, the feature space having $N_{G}\left(N_{G}+1\right)/2$ dimensions is quantized with $B_{V}=N_{G}$ resolution bits.
Fig.~\ref{fig:vector-discrete-B0} shows the received signal power achieved with different values of group size.
Here, we can notice that higher group sizes achieve better performance than lower group sizes.
%the higher the group size, the worse is the quantization resolution, since the $N_{I}$ total bits are distributed over more reactance elements.
This is due to the optimality condition \eqref{eq:opt-cond-u}, which has more degrees of freedom as the group size increases.
Thus, it can be more easily satisfied when higher group sizes are considered.
%Interestingly, correlated grouping achieves higher received signal power than uncorrelated grouping and optimized grouping.
%This is because with correlated grouping there is a higher correlation between the optimal reactance elements within each group $\mathbf{X}_{I,g}$.
%Vector-quantization can benefit from this property, achieving better performance than in the uncorrelated grouping and optimized grouping cases, when a low resolution is used.
%In conclusion, differently from what observed for scalar-discrete RISs, correlated grouping is the preferred grouping strategy in the case of vector-discrete RISs used with a low number of total resolution bits.

\begin{figure*}[t]
    \centering
    \includegraphics[height=0.3\textwidth]{figs_v6/fig-complexity-sca.eps}
    \includegraphics[height=0.3\textwidth]{figs_v6/fig-complexity-vec-group2.eps}
    \caption{Optimization computational complexity versus the number of RIS elements. The quasi-Newton method ``QN'' is compared with alternating optimization for scalar-discrete RISs (on the left) and vector-discrete RISs (on the right).}
    \label{fig:complexity}
\end{figure*}

\subsection{Optimization Computational Complexity}

To conclude our discussion, we compare continuous-value RISs with discrete-value RISs in terms of optimization computational complexity.
On the one hand, when the quasi-Newton method is used to optimize the continuous values of $\mathbf{X}_{I}$, the computational complexity of each iteration is $\mathcal{O}((N_{I}(N_{G}+1)/2)^2)$ \cite{she20}.
On the other hand, when alternating optimization is used to optimize the discrete values of $\mathbf{X}_{I}$, we distinguish two cases, namely scalar-discrete RISs and vector-discrete RISs.
In the case of scalar-discrete RISs, the total number of search times in a complete iteration of the alternating optimization algorithm is $G\frac{N_{G}\left(N_{G}+1\right)}{2}2^{B}$ in group connected architectures.
This number boils down to $N_I2^B$ in single connected and to $\frac{N_{I}\left(N_{I}+1\right)}{2}2^{B}$ in fully connected architectures.
Instead, in the case of vector-discrete RISs, the number of search times per iteration becomes $\frac{N_{I}}{N_{G}}2^{B_V}$.
The values of these computational complexities are reported in Fig.~\ref{fig:complexity}.
In the case of scalar-discrete RISs, we observe that our RIS design strategy based on discrete-value impedance networks is beneficial also in terms of optimization computational complexity.
Even with $B=4$, the complexity is far less than the complexity of the quasi-Newton method for medium-high numbers of RIS elements.

Remarkably, the complexity of our algorithm grows with $\mathcal{O}(N_I)$ in the case of scalar-discrete group connected RISs, and with $\mathcal{O}(N_I^2)$ in the case of scalar-discrete fully connected RISs.
Besides, the complexity grows with $\mathcal{O}(N_I)$ in the case of vector-discrete group connected RISs, and does not depend on $N_I$ in the case of vector-discrete fully connected RISs.
Thus, we can conclude that our algorithm is scalable and can be applied to RIS with a high number of elements $N_I$.

%%%%%%%%%%%%%%%%%%%%%%%%%%%%%%%%%%%%%%%%%%%%%%%%%%
\section{Conclusion}
\label{sec:conclusion}

% Practical RIS design
We propose a novel BD-RIS design based on discrete-value group and fully connected architectures.
%The proposed design strategy extends the previous research on single connected RIS architectures with discrete phase shifts to group and fully connected RIS architectures, recently introduced in \cite{she20}.
Our strategy is composed of two stages.
Firstly, through the offline learning stage, we build the codebook containing the possible values for the reconfigurable impedances.
Secondly, during the online deployment stage, we optimize the reconfigurable impedances with alternating optimization on the basis of the designed codebook.
Two different approaches are developed, namely scalar-discrete and vector-discrete RISs, based on scalar and vector quantization, respectively.
Vector-discrete RISs achieve higher performance than scalar-discrete RISs at the cost of increased optimization computational complexity.
Numerical results show that a single resolution bit per reconfigurable impedance is sufficient to achieve optimality in fully connected RISs.
In a practical scenario, this simplifies significantly the hardware complexity of the fully connected RIS.

% Future research directions
Two future research directions can be identified.
First, the optimization of discrete-value BD-RIS based on imperfect or statistical channel knowledge should be investigated.
Second, the proposed codebook design and optimization framework can be readily extended to multi-user systems.
However, the involved optimization problems are non-convex and hard to solve.
Their solution could be investigated in future work.

% Numerical results
%We consider a single-user RIS-aided \gls{mimo} system, and we assess the performance in terms of received signal power of the single, group, and fully connected architectures, discretized with different levels of resolution.
%Numerical results show that the performance of continuous-value RISs can be reached also allocating a few resolution bits.
%Numerical results show that a single resolution bit per reconfigurable impedance is sufficient to achieve the optimality in fully connected RISs.
%Furthermore, optimized grouping is the grouping strategy that maximizes the received signal power in the absence of resolution constraints, and in the case of scalar-discrete RISs.
%Conversely, correlated grouping is beneficial in the case of vector-discrete RISs, when they are designed with low levels of resolution.
%
% Future research
%Two future research avenues include the channel estimation and implementation circuits for group connected architectures.
%Channel estimation has been traditionally achieved for single connected RISs by exploiting the diagonal structure of the scattering matrix.
%Because of this property, the knowledge of the cascaded channel transmitter-RIS-receiver is sufficient to optimize the RIS, and the channel estimation problem can be highly simplified \cite{wu21}.
%However, efficient channel estimation techniques for group connected architectures are still unexplored.
%In the case of discrete-value group and fully connected RISs, a possible solution is to measure the received signal power for each possible configuration of the discrete scattering matrix during a beam training process.
%Assuming that the time required by the beam training is much less than the coherence time of the channel
%Thus, the selected scattering matrix is the one that provided the highest received signal power during the beam training.
%With this solution, an explicit \gls{csi} estimate is not required.
%A further research direction regards the implementation circuit design of group connected RISs.
%In this work, we assume that the RIS exhibits zero energy dissipation, as assumed in the vast majority of works on discrete phase shift single connected RISs.
%This allows us to optimize the discrete-value reactance matrix being agnostic about the RIS implementation circuit.
%However, in practical circuits, reconfigurable impedances include inevitable resistance values causing power loss, as modelled in \cite{abe20}, \cite{jun21}.
%The equivalent circuit model proposed in \cite{abe20} could be considered for designing group connected RISs based on discrete-value impedance networks in a future work.

\bibliographystyle{IEEEtran}
\bibliography{IEEEabrv,main}

\end{document}