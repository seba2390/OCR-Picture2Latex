\documentclass[preprint, superscriptaddress,amsmath, nofootinbib]{revtex4-1}
\usepackage{graphicx}
\usepackage{latexsym}
\usepackage{slashed}
\usepackage{bm}
\usepackage{color}
%\usepackage[utf8]{inputenc}
\usepackage{diagbox}
\usepackage{amsmath}
\usepackage{xcolor}
\usepackage{booktabs}
\usepackage[colorlinks,citecolor=blue,urlcolor=blue,linkcolor=blue]{hyperref}
%\usepackage[utf8]{inputenc}
\usepackage{diagbox}
\usepackage{textcomp}
% A useful Journal macro
\def\Journal#1#2#3#4{{#1} {\bf #2}, #3 (#4)}


\bibliographystyle{apsrev}

\definecolor{nicered}{rgb}{0.7,0.1,0.1}
\definecolor{nicegreen}{rgb}{0.1,0.5,0.1}
\hypersetup{colorlinks,citecolor= nicegreen,linkcolor= nicered}

\usepackage{dcolumn}% Align table columns on decimal point

% A useful Journal macro
\def\Journal#1#2#3#4{{#1} {\bf #2}, #3 (#4)}

% Some useful journal names
\def\NCA{\em Nuovo Cimento}
\def\PU{\em Phys. Usp.}
\def\PPNP{\em Prog. Part. Nucl. Phys.}
\def\Sym{\em Symmetry}
\def\RNC{\em Rivista Nuovo Cimento}
\def\NIM{\em Nucl. Instrum. Methods}
\def\NIMA{\em Nucl. Instrum. Methods A}
\def\NPB{\em Nucl. Phys. B}
\def\PLB{\em Phys. Lett.  B}
\def\PRL{\em Phys. Rev. Lett.}
\def\PRD{\em Phys. Rev. D}
\def\ZPC{\em Z. Phys. C}
\def\GaC{\em Gravitation and Cosmology}
\def\GaCS{\em Gravitation and Cosmology Suppl.}
\def\JETP{\em JETP}
\def\MNRAS{\em Mon. Not. Roy. astr. Soc.}
\def\JETPL{\em JETP Lett.}
\def\PAN{\em Phys.Atom.Nucl.}
\def\CQG{\em Class. Quantum Grav.}
\def\APJ{\em Astrophys. J.}
\def\SCI{\em Science}
\def\RAA{\em Res. Astron. Astrophys.}
\def\MPLA{\em Mod. Phys. Lett. A}
\def\IJTP{\em Int. J. Theor. Phys.}
\def\IJMPA{\em Int. J. Mod. Phys. A}
\def\IJMPD{\em Int. J. Mod. Phys.  D}
\def\NJP{\em New J. of Phys.}
\def\ARAA{\em Ann. Rev. Astron. Astrophys.}
\def\AIPCP{\em AIP Conf. Proc.}
\def\JHEP{\em JHEP}
\def\JCAP{\em JCAP}
\def\SJNP{\em Sov.J.Nucl.Phys.}
\def\SJPN{\em Sov.J.Part.Nucl.}
\def\EPHJ{\em Eur.Phys.J}
\def\EPJC{\em Eur.Phys.J. C}
\def\JPCS{\em J. Phys. Conf. Ser.}
\def\BWP{\em Bled Workshops in Physics}
\def\s{{\,\rm s}}
\def\g{{\,\rm g}}
\def\eV{\,{\rm eV}}
\def\keV{\,{\rm keV}}
\def\MeV{\,{\rm MeV}}
\def\GeV{\,{\rm GeV}}
\def\TeV{\,{\rm TeV}}
\def\sv{\left<\sigma v\right>}
\def\({\left(}
\def\){\right)}
\def\cm{{\,\rm cm}}
\def\K{{\,\rm K}}
\def\kpc{{\,\rm kpc}}
\def\beq{\begin{equation}}
\def\eeq{\end{equation}}


\begin{document}
\title{Probing Inelastic Dark Matter at the LHC, FASER and STCF}% Force line breaks with \\
%%\thanks{A footnote to the article title}%



\author{Chih-Ting Lu}
\email{ctlu@njnu.edu.cn}
\affiliation{Department of Physics and Institute of Theoretical Physics, Nanjing Normal University, Nanjing, 210023, China}
\affiliation{CAS Key Laboratory of Theoretical Physics, Institute of Theoretical Physics, Chinese Academy of Sciences, Beijing 100190, P. R. China}

\author{Jianfeng Tu}
\email{tujf@nnu.edu.cn}
\affiliation{Department of Physics and Institute of Theoretical Physics, Nanjing Normal University, Nanjing, 210023, China}

\author{Lei Wu}
\email{leiwu@njnu.edu.cn}
\affiliation{Department of Physics and Institute of Theoretical Physics, Nanjing Normal University, Nanjing, 210023, China}




\date{\today}% It is always \today, today,
             %  but any date may be explicitly specified

\begin{abstract}
In this work, we explore the potential of probing the inelastic dark matter (DM) model with an extra $U(1)_D$ gauge symmetry at the Large Hadron Collider, ForwArd Search ExpeRiment and Super Tau Charm Factory. To saturate the observed DM relic density, the mass splitting between two light dark states has to be small enough, and thus leads to some distinctive signatures at these colliders. By searching for the long-lived particle, the displaced muon-jets, the soft leptons, and the mono-photon events, we find that the inelastic DM mass in the range of 1 MeV to 210 GeV could be tested.
\end{abstract}
\pacs{Valid PACS appear here}% PACS, the Physics and Astronomy
                             % Classification Scheme.
%\keywords{Suggested keywords}%Use showkeys class option if keyword
                              %display desired
\maketitle

\tableofcontents

\section{Introduction}


{Despite the strong astrophysical and cosmological evidence to support the existence of dark matter (DM)~\cite{Trimble:1987ee,Barack:2018yly}, its nature still remains a mystery. It is widely believed that Standard Model (SM) particles may interact with DM through forces other than gravity~\cite{Pospelov:2007mp}. The Weakly Interacting Massive Particle (WIMP) is one of the most popular dark matter candidates~\cite{Feng:2010gw,Bauer:2017qwy}. However, null results of searching for WIMPs have imposed stringent constraints on its properties in the mass range of GeV to TeV~\cite{Schumann:2019eaa,Kahlhoefer:2017dnp}. For instance, the spin-independent (SI) DM-nucleon scattering cross section is limited to $6.5\times10^{-48}$ $\text{cm}^2$ at the DM mass of $30$ GeV~\cite{LZ:2022ufs,XENON:2023sxq}. 
This motivates the recent studies of the light DM models that can avoid the conventional constraints. To explore the sub-GeV DM, various new proposals and experiments have been proposed~\cite{Knapen:2017xzo,Lin:2022hnt}.
}


However, in the thermal freeze-out scenario, the sub-GeV DM models usually suffer from various astrophysical and cosmological constraints. For example, the s-wave annihilation of light DM is not favored by the cosmic microwave background (CMB), which requires the DM mass should be heavier than about $10$ GeV~\cite{Lin:2011gj}. Additionally, the DM particles with the mass less than $1$ MeV is tightly constrained by the Big Bang Nucleosynthesis (BBN)~\cite{Nollett:2014lwa}. Nevertheless, there exist some exceptions that can evade these bounds, such as the DM models with the p-wave annihilation~\cite{Matsumoto:2018acr}, the inelastic DM models~\cite{Tucker-Smith:2001myb}, the asymmetric DM models~\cite{Zurek:2013wia}, and the freeze-in mechanism DM models~\cite{Dvorkin:2020xga}. Among them, the inelastic DM models that were motivated by the explanation of the DAMA excess~\cite{Tucker-Smith:2001myb} have recently gained considerable attention~\cite{Baek:2014kna,Izaguirre:2015zva,DEramo:2016gqz,Izaguirre:2017bqb,Berlin:2018jbm,Mohlabeng:2019vrz,Tsai:2019buq,Okada:2019sbb,Ko:2019wxq,Duerr:2019dmv,Duerr:2020muu,Ema:2020fit,Kang:2021oes,Bell:2021zkr,Batell:2021ooj,Bell:2021xff,Feng:2021hyz,Guo:2021vpb,Li:2021rzt,Filimonova:2022pkj,Bertuzzo:2022ozu,Gu:2022vgb,Li:2022acp,Mongillo:2023hbs,Heeba:2023bik}. If only the lighter DM state is present in the current Universe, the up-scattering in DM-nucleon interactions becomes insensitive to direct detection\footnote{Note people can consider cosmic ray-boosted inelastic DM for the up-scattering in DM-nucleon interactions as shown in Ref.~\cite{Bell:2021xff,Feng:2021hyz}}, and the primary elastic DM-nucleon scattering occurs at the one-loop level~\cite{Izaguirre:2015zva}. Consequently, these models can evade the limits from direct detection~\cite{CarrilloGonzalez:2021lxm}. 




In this paper, we investigate the prospect of probing the inelastic DM model with an additional $U(1)_D$ gauge symmetry at colliders. The mass splitting of two dark states is induced by the interaction between the dark Higgs field and the DM sector, and the transition between two dark states is mediated by the new $U(1)_D$ gauge boson. The collider signatures of inelastic DM at accelerators strongly depend on two key parameters: the ground state DM mass ($M_{\chi_1}$) and the mass splitting between the excited and ground DM states ($\Delta_{\chi}\equiv M_{\chi_2}-M_{\chi_1}$). These two parameters are also associated with the lifetime of the excited DM state. 
For $M_{\chi_1}\lesssim 5$ GeV and $\Delta_{\chi} < 0.5 M_{\chi_1}$, the fixed target experiments~\cite{A1:2011yso,PhysRevLett.107.191804,PhysRevLett.112.221802} and low energy $e^+ e^-$ colliders such as BaBar~\cite{Filippi:2019lfq}, Belle II~\cite{Belle-II:2022yaw}, BESIII~\cite{BESIII:2022oww} and Super Tau Charm Factory (STCF)~\cite{Epifanov:2020elk} offer powerful avenues for searching for inelastic DM. On the other hand, for $M_{\chi_1}\lesssim 15$ GeV and $\Delta_{\chi}\lesssim 0.1 M_{\chi_1}$, LLP experiments like ForwArd Search ExpeRiment (FASER)~\cite{FASER:2018bac,FASER:2019aik}, MAssive Timing Hodoscope for Ultra-Stable neutral-pArticles (MATHUSLA)~\cite{MATHUSLA:2022sze}, SeaQuest~\cite{Liu:2023buo}, the COmpact Detector for EXotics at LHCb (CODEX-b)~\cite{Aielli:2022awh} and A Laboratory for Long-Lived eXotics (AL3X)~\cite{Dercks:2018wum} can explore the remaining parameter space.
In the intermediate DM mass range of $2$ GeV $\lesssim M_{\chi_1} \lesssim 200$ GeV with $\Delta_{\chi} \lesssim 0.2 M_{\chi_1}$, the Large Hadron Collider (LHC) remains the primary machine for probing inelastic DM. 

The structure of this paper is expanding as follows. In Sec.~\ref{sec:model}, we recapitulate the inelastic DM model with an $U(1)_D$ gauge symmetry. In Sec.~\ref{sec:signature}, we then study the signatures of inelastic DM at the FASER, LHC, and STCF. Finally, we summarize our findings in Sec.~\ref{sec:conclusion}. 


\section{Inelastic dark matter model} 
\label{sec:model}


{ 
In this section, we briefly review the inelastic DM models with an $U(1)_D$ gauge symmetry and focus on the fermionic DM candidates\footnote{The scalar inelastic DM models can be found in Ref.~\cite{Baek:2014kna, Izaguirre:2015zva, Okada:2019sbb, Kang:2021oes, Li:2021rzt, Bertuzzo:2022ozu}. Since we will study the on-shell $Z'$ productions and $Z'$ mainly decays to DM states, the predictions in our analysis can be applied to scalar inelastic DM models as well.}. In addition to the SM particles, a singlet complex scalar field $\Phi$ as well as a Dirac fermion field $\chi$ are involved. We assign the $U(1)_D$ charges for $\Phi$ and $\chi$ as $Q(\Phi) = +2$ and $Q(\chi) = +1$, respectively. All SM particles are neutral under the $U(1)_D$ symmetry and cannot be directly coupled to the dark sector. The relevant gauge invariant and renormalizable Lagrangian for this model can be written as 
\begin{equation}
\begin{split}
    \mathcal{L} =& \mathcal{L}_{SM}  -\frac{1}{4}X_{\mu\nu}X^{\mu\nu}-\frac{1}{2}\sin{\epsilon}X_{\mu\nu}B^{\mu\nu}+\mathcal{D}^\mu{\Phi^\dag}\mathcal{D}_\mu{\Phi} \\
    &- \mu^2_{\Phi}\Phi^\dag{\Phi}+\lambda_{\Phi}(\Phi^\dag{\Phi})^2-\lambda_{\mathcal{H}\Phi}\mathcal{H}^\dag\mathcal{H}\Phi^\dag{\Phi} \\
    &- \overline{\chi}(i\mathcal{D} \!\!\!/ - M_\chi)\chi - (\frac{\xi}{2}\Phi^{\dag}\overline{\chi^c}\chi + H.c.),
    \label{Lf}
\end{split}
\end{equation}
where $X_{\mu\nu}$ and $B_{\mu\nu}$ are field strength tensors of $U(1)_D$ and $U(1)_Y$ gauge fields, respectively. $\epsilon$ is the kinematic mixing angle between $X_{\mu\nu}$ and $B_{\mu\nu}$, $\mu_{\Phi}$ is the parameter with the same dimension as mass, and $\lambda_{\Phi}$, $\lambda_{\mathcal{H}\Phi}$ are dimensionless parameters and $\xi$ is assumed to be a positive, real and dimensionless parameter. $\mathcal{H}$ is the SM-like scalar doublet field and we expand $\mathcal{H}$ and $\Phi$ in the unitary gauge to the following form, 
\begin{align}
    {\mathcal{H}}\ = \ \frac{1}{\sqrt{2}} \begin{pmatrix}
    {0} \\
    v+h
    \end{pmatrix},\quad  \Phi=\frac{1}{\sqrt{2}}(v_X+h_X), 
\end{align}
where $v$ and $v_X$ are vacuum expectation values of $\mathcal{H}$ and $\Phi$, respectively.
The $U(1)_D$ is broken spontaneously by $\langle \Phi \rangle$ = ${v_{X}}/{\sqrt{2}}$, and electroweak symmetry is broken spontaneously as usual by $\langle \mathcal{H} \rangle$ = $(0,{v}/{\sqrt{2}})$.

We then diagonalize the $U(1)$ gauge kinetic term in Eq.~(\ref{Lf}) by redefining the gauge fields via the following transformation~\cite{Wells:2008xg,Berlin:2018jbm,Filimonova:2022pkj} :
\begin{equation}
\left(
\begin{array}{cc}
    B^{\mu}  \\
    X^{\mu}
\end{array}
\right)
=
\left(
\begin{array}{cc}
     1 & \eta \\
     0 & \eta/\epsilon
\end{array}
\right)
\left(
\begin{array}{cc}
     B^{\mu}_p  \\
     X^{\mu}_p
\end{array}
\right).
\end{equation}
The covariant derivative is given by $\mathcal{D}_{\mu} = \partial_{\mu} + i(g_{D}Q_{X}+ g_{1}\eta Q_{Y})X_{\mu}+ig_{1}Q_{Y}B_{\mu}+ig_{2}T^{3}W^{3}_{\mu}$,
where $W^{3}_{\mu}$, $B_{\mu}$, and $X_{\mu}$ correspond to the gauge potentials associated with the gauge groups $SU(2)_{L}$, $U(1)_{Y}$, and $U(1)_{X}$, respectively. The gauge couplings are denoted as $g_{2}$, $g_{1}$, and $g_{D}$. The gauge fields before mixing are represented by $B^{\mu}_p$ and $X^{\mu}_p$. Additionally, we define $\eta \equiv{} \epsilon/\sqrt{1-\epsilon^2}$, and $Q_X$ represents the $U(1)_D$ charge of either $\Phi$ or $\chi$.

After performing a GL$(2, R)$ rotation to diagonalize the kinetic terms, followed by an $\textit{O}(3)$ rotation to diagonalize the $3\times 3$ neutral gauge boson mass matrix, the mass eigenstates can be expressed through the corresponding transformation as~\cite{Filimonova:2022pkj} :
\begin{equation}
\left(
\begin{array}{cc}
    B_{p}^{\mu}  \\
    W^3  \\
    X_{p}^{\mu}
\end{array}
\right)
=
\left(
\begin{array}{ccc}
     c_W & -s_W c_X & s_W s_X \\
     s_W & c_W c_X & -c_W s_X \\
     0 & s_X & c_X
\end{array}
\right)
\left(
\begin{array}{ccc}
     A  \\
     Z  \\
     Z^\prime
\end{array}
\right),
\end{equation}
where the $s_W $ and $c_W$ are the sine and cosine values of the weinberg angle, and the new gauge mixing angle can be written as 
\begin{equation}
    \theta_X = \frac{1}{2}\arctan(\frac{-2s_W\eta}{1-s_W^2\eta^2-\Delta_Z}),
\end{equation}
where $\Delta_Z = M_X^2/M_{Z_{0}}^2$, $M^2_{X} = g^2_{D}Q^2_{X}v_{X}^2 $ and $M^2_{Z_{0}} = (g^2_{1}+g^2_{2}) v^2/4$, which $ M_X$ and $M_Z$ are the masses of two $U(1)$ gauge bosons before mixing. Finally, the photon becomes massless, and two heavier gauge boson mass eigenvalues are
\begin{equation}
    M_{Z,Z'} = \frac{M_{Z_{0}}^2}{2}[(1+s_W^2\eta^2+\Delta_Z)\pm\sqrt{(1-s_W^2\eta^2+\Delta_Z)^2+4s_W^2\eta^2}], 
    \label{U1mass}
\end{equation} 
which valid for $\Delta_Z < 1-s_W^2\eta^2$. Considering the assumption $\epsilon \ll 1$, we find $M_{Z'} \approx M_X$ from Eq.~(\ref{U1mass}) and the interactions of $Z'$ and SM fermions for the linear order approximation in $\epsilon$ can be written as  
\begin{equation}
    \mathcal{L}_{Z'\overline{f}f} = -\epsilon{e}{c_W}{Q_f}\overline{f}\gamma^{\mu}f Z'_\mu,
    \label{Lzff}
\end{equation}
where $Q_f$ is the electric charge of SM fermions. 


The Dirac fermion field can be further decomposed into two Majorana fermion fields $\chi_1$, $\chi_2$ as 
\begin{equation}
\begin{split} 
    \chi &= \frac{1}{\sqrt{2}}(\chi_2+i\chi_1), \\
    \chi_2 &= \chi^c_2,\quad \chi_1 = \chi^c_1.
\end{split}
\end{equation}
After the breaking of $U(1)_D$ gauge symmetry, the DM parts of Eq.~(\ref{Lf}) can be expanded as 
\begin{equation}
\begin{split}
    \mathcal{L}_{\chi} =& \frac{1}{2}\sum_{n=1,2}\overline{\chi_n}(i\partial \!\!\!/ -M_{\chi})\chi_n -i\frac{g_D}{2}\left(\overline{\chi_2}\chi_1 -\overline{\chi_1}\chi_2\right)X \!\!\!/  \\ 
    &-\frac{\xi}{2}(v_X+h_X)(\overline{\chi_2}\chi_2-\overline{\chi_1}\chi_1).
    \label{msat} 
\end{split}    
\end{equation}
We summarize some interesting features from the above equation. First, there is a residual $Z_2$ symmetry via Krauss-Wilczek mechanism~\cite{PhysRevLett.62.1221} which the $U(1)_D$ gauge symmetry is broken into its $Z_2$ subgroup~\cite{Baek:2014kna}. Only $\chi_{1,2}$ are $Z_2$-odd and can be DM candidate(s). Second, the mass splitting between $\chi_1$ and $\chi_2$ is triggered from the $\Phi^{\dag}\overline{\chi^c}\chi$ interaction after the symmetry breaking and can be written as 
\begin{equation}
    \Delta_\chi = 2\xi v_X.
\end{equation}
We then assign $M_{\chi_2} > M_{\chi_1}$ with the form, 
\begin{equation}
    M_{\chi_{1,2}} = M_{\chi}\mp\xi v_X.
\end{equation}
%Finally, we point out there are only off-diagonal interactions for $X$ boson and DM states, and diagonal interactions for $h_X$ and DM states in this model. 

%The $X$ boson can only couple to SM fermions through the kinetic mixing as shown in Eq.(\ref{Lf}). We define the mass eigenstate of dark gauge boson as $Z'$ and its interactions with SM fermions for the linear order approximation in $\epsilon$ can be written as 
%\begin{equation}
%    \mathcal{L}_{Z'\overline{f}f} = -\epsilon{e}{c_W}{Q_f}\overline{f}\gamma^{\mu}Z'_\mu{f},
%    \label{Lzff}
%\end{equation}
%where $c_W$ is the cosine of Weinberg angle and $Q_f$ is the electric charge of SM fermions.  
%Finally, the mass of $Z'$ can be approximated as 
%\begin{equation}
%    M_{Z'} \simeq g_X Q(\Phi) v_X.
%    \label{Mz}
%\end{equation}

Before closing this section, we have to mention that the scalar sector in this model is not the focus of this work. More details for the scalar sector in fermionic inelastic DM models and relevant search strategies can be found in Ref.~\cite{Duerr:2020muu,Kang:2021oes,Li:2021rzt}. We can properly choose model parameters to satisfy all constraints from the scalar sector part in this study. 
}

%\subsection{B. Fermion Model}
%In fermion iDM, we consider that complex scalar X is Dark Higgs as well and Dirac fermion $\chi$ is dark sector. Same with scalar model, X and $\chi$ are both under the $U(1)_X$ charges that Q(X) = +2 and Q($\chi$) = +1 for them. And Dark Higgs field is the origin of the non-zero mass dark mediator $Z'$, which possess non-zero vacuum expectation value causing the $U(1)_X$ symmetry broken into local $Z_2$ symmetry mentioned above in scalar model. In this model, SM-like Higss and SM particles also do not carry $U(1)_X$ charges so that they cannot coupling to dark sector directly.
%Then we give the gauge invariant Lagrangian in fermion iDM by
%\begin{equation}
%\begin{split}
%    \mathcal{L} &= \mathcal{L}_{SM}  -\frac{1}{4}X_{\mu\nu}X^{\mu\nu}-\frac{1}{2}\sin{\epsilon}X_{\mu\nu}B^{\mu\nu}+\mathcal{D}^\mu{X^\dag}\mathcal{D}_\mu{X} \\
%    &- \mu^2_XX^\dag{X}+\lambda_X(X^\dag{X})^2-\lambda_{\mathcal{H}X}\mathcal{H}^\dag\mathcal{H}X^\dag{X} \\
%    &- \overline{\chi}(i\mathcal{D} \!\!\!/ - M_\chi)\chi - (\frac{\xi}{2}X^{\dag}\overline{\chi^c}\chi + H.c),
%    \label{Lf}
%\end{split}
%\end{equation}
%where $\mathcal{D}_\mu$ = $\partial_\mu$+$ig_XQ_XX_\mu$ is the covariant derivative with the dark coupling constant $g_X$ and $Q_X$ represent the charges of Drak Higgs and Dark sector. $X_{\mu\nu}$ and $B_{\mu\nu}$ have been introduced in saclar model as well and $\xi$ is a postive real parameter we set. After field $X_\mu$ obtain it's non-zero VEV along with $U(1)_X$ guage symmetry broken into $Z_2$ symmetry, then Dirac fermion defined as $\chi=\frac{1}{\sqrt{2}}(\chi_2+i\chi_1)$ will decompose into two Majorana fermion $\chi_2$ and $\chi_1$, so we have 
%\begin{equation}
%\begin{split}
%    \chi_2 &= \chi^c_2,\quad \chi_1 = \chi^c_1 \\
%    \chi^c &= \frac{1}{\sqrt{2}}(\chi_2-i\chi_1).
%\end{split}
%\end{equation}

%When expanding the Lagrangian Eq.(\ref{Lf}), we wil get the mass splitting associating terms list below,
%\begin{equation}
%    \mathcal{L}_{msat}=\frac{1}{2}\sum_{n=1,2}\overline{\chi_n}(i\partial \!\!\!/ -M_{\chi_n})\chi_n -\frac{\xi}{2}(\nu_X+h_X)(\overline{\chi_2}\chi_2-\overline{\chi_1}\chi_1)
%    \label{msat}
%\end{equation}
%From above Eq.(\ref{msat}), we could get the mass of $\chi_2$ and $\chi_1$ respectively is $M_\chi + \xi\nu_X$, $M_\chi - \xi\nu_X$ therefore the mass splitting of them is clearly displying by folowing expression,
%\begin{equation}
%    \Delta_\chi = 2\xi\nu_X
%\end{equation}
%Then, from expanding term $\overline{\chi}(i\mathcal{D} \!\!\!/ - M_\chi)\chi$, we will find that the interaction term of dark mediator $Z'$ and dark sector is from here, which is 
%\begin{equation}
%    \mathcal{L}_{Z'\chi}=g_X\overline{\chi}X \!\!\!\!/ \chi
%\end{equation}
%Next, the representation of intraction between $Z'$ boson and SM fermions is same with Eq.(\ref{Lzff}) in scalar model. And the mass of $Z'$ boson is equal to Eq.(\ref{Mz}) as well. The parameter ($\lambda_H,\lambda_X,\lambda_{HX}$) also could be expressed by ($M_H,M_X,\alpha$) which is consistent with saclar model.




\section{Signatures of inelastic DM at colliders}
\label{sec:signature}

We will discuss the production of inelastic DM at the LHC and classify the signal signatures depending on the decay length of $\chi_2$. First of all, the UFO model file of the inelastic DM model is generated by \textsf{FeynRules}~\cite{Christensen:2008py} and then we apply \textsf{MadGraph5\_aMC@NLO}~\cite{Alwall:2014hca} to generate Monte Carlo events and calculate cross sections for the following signal process, 
\begin{equation}
    pp \rightarrow A'\rightarrow \chi_2\chi_1.
\end{equation}
We consider a centre-of-mass energy of $\sqrt{s}=14$ TeV and fix the following model parameters, 
\begin{equation} 
    M_{Z'} = 3 M_{\chi_1},\quad \alpha_D = \frac{g_D^2}{4\pi} = 0.1,\quad \epsilon = 0.01,
\end{equation}
but vary $M_{\chi_1}$ and $\Delta_{\chi}$ in the range below, 
\begin{equation} 
    M_{\chi_1}/\text{GeV} = \left[ 5, 100\right],\quad \Delta_{\chi}/\text{GeV} = \left[ 0.01, 0.1\right], 
\end{equation} 
with the step length $5$ GeV and $0.01$ GeV for $M_{\chi_1}$ and $\Delta_{\chi}$, respectively.

%(We may need the other plot for $(\epsilon, \beta\gamma c\tau_{\chi_2})$ for varying $M_{\chi_1}$ with the relation $\Delta_{\chi} = 0.1 M_{\chi_1}$ !) 
The time of flight for $\chi_2$ is automatically calculated in the Madgraph5@NLO. In the approximation $M_{Z'}\gg M_{\chi_2}\sim M_{\chi_1}\gg M_l$, the partial decay rate for $\chi_2\rightarrow\chi_1 l^+l^-$ can be written as~\cite{Izaguirre:2015zva} 
\begin{equation} 
    \Gamma (\chi_2\rightarrow\chi_1 l^+l^-)\simeq\frac{4\epsilon^2 \alpha_{\text{em}}\alpha_D\Delta^5_{\chi}}{15\pi M_{Z'}^4} 
\label{chi2_width}    
\end{equation}
where $l = e, \mu$ and $\alpha_{\text{em}}\simeq 1/137$ is the fine structure constant. It's clear to see that once we reduce the values of $\epsilon$, $\Delta_{\chi}$, and $M_{\chi_1}$, the lifetime of $\chi_2$ will enhance.  
In the Fig.~\ref{fig:chi2_length}, we display the relations of $M_{\chi_1}$ and $\epsilon$ to the $\chi_2$ decay length. We can find the behaviors in numerical results are consistent with the approximated formula in Eq.~(\ref{chi2_width}). 

\begin{figure}[t!]
\centering
\includegraphics[width=14cm,height=7cm]{dl_chi2.png}
\caption{The relationship of $(\Delta_{\chi}, \beta\gamma c\tau_{\chi_2})$ with varying $M_{\chi_1}$. The three dashed lines represent different special lab frame decay length numbers, $\beta\gamma c\tau_{\chi_2} = 0.1 \text{mm}$ (green), $\beta\gamma c\tau_{\chi_2} = 3 \text{m}$ (orange), and $\beta\gamma c\tau_{\chi_2} = 480 \text{m}$ (red). }
\label{fig:chi2_length}
\end{figure}

In addition, we use three dashed lines in Fig.~\ref{fig:chi2_length} to illustrate our search strategies for inelastic DM at collider experiments. Specifically, if the lab frame decay length of $\chi_2$ is as long as $\beta\gamma c\tau_{\chi_2}\simeq \mathcal{O}(500)$ m, the FASER is an ideal detector to search for inelastic DM as shown in the red dashed line in Fig.~\ref{fig:chi2_length}. Here, $\gamma$ is the Lorentz factor, $\beta$ is the velocity of $\chi_2$, and $\tau_{\chi_2} = 1/\Gamma_{\chi_2}$ is the proper decay time of $\chi_2$. 
The details for this analysis can be found in Sec.~\ref{subsec:4a}. If $\chi_2$ generates the displaced vertex at the LHC with $0.1 $ mm $ < \beta\gamma c\tau_{\chi_2}\lesssim 3$ m, the displaced muon-jet (DMJ) signature is sensitive to search for inelastic DM in this parameter space as shown in the regions below the orange dashed line in Fig.~\ref{fig:chi2_length}. We study this possibility in Sec.~\ref{subsec:4b}. Furthermore, if $\chi_2$ is prompt decay ($\beta\gamma c\tau_{\chi_2}\lesssim 0.1$ mm), the soft leptons searches at the LHC can be applied to this situation as shown in the regions below the green dashed line in Fig.~\ref{fig:chi2_length}. We recast the ATLAS analysis~\cite{ATLAS:2019lng} for soft leptons signatures in Sec.~\ref{subsec:4c}. Finally, if $\chi_2$ is the LLP with $\beta\gamma c\tau_{\chi_2} > 3$ m, the mono-jet searches can indirectly impose constraints on this model as shown in the regions above the orange dashed line in Fig.~\ref{fig:chi2_length}. However, the mono-jet constraints from the LHC are much weaker than the above ones, so we will not show the recasting for these constraints in this work. Finally, as a complementary study to cover the searches of sub-GeV inelastic DM, we utilize the mono-photon signature to search for inelastic DM at STCF. More details for this analysis will be presented in Sec.~\ref{subsec:4d}. 
%The recasting of the ATLAS mono-jet analysis~\cite{?} is shown in Sec.~?.


%{\color{blue}(OK !)}In this section, we begin by investigating the signatures of LLPs associated with inelastic DM at FASER. This analysis allows us to explore the lifetime frontier of the inelastic DM model. Subsequently, we focus on the displaced muon-jet and soft lepton signatures at the LHC, which extend the search to the energy frontier of this model. Finally, we examine the detection of inelastic DM through mono-photon searches at STCF, which provide sensitivity to lower DM mass ranges in this model. 

\subsection{LLPs at the FASER}
\label{subsec:4a}

As we know, the B-factories can explore LLPs, but the restriction of their center-of-mass energies forces the upper bound of the mass of LLPs to be less than about $10$ GeV~\cite{Acevedo:2021wiq}. On the other hand, the ATLAS/CMS detectors at the LHC are not sensitive to the new particles with mass less than $\mathcal{O}(10)$ GeV. Therefore, a new lifetime frontier detector to search for $\mathcal{O}(10)$ GeV BSM LLPs is needed. 

%\begin{figure}[t!]
%\centering
%\includegraphics[width=0.45\textwidth]{plots/dlchi2.eps}
%%%%%\includegraphics[width=0.45\textwidth]{plots/dlchi2_m.png}
%\caption{The distribution of heavier state dark matter candidate $\chi_2,\phi_2$ decay length in the laboratory system for $M_{\chi_2}$ = 11\ GeV and $M_{Z'}$ = 30\ GeV. In this plot, the histograms represent all the events we set in our simulation through MadGraph5@NLO as well as those passing the angular acceptance requirement of FASER 1 and FASER 2. }
%\label{fig:4}
%\end{figure}
In this subsection, we introduce a new detector called FASER, which has been built around the LHC to study LLPs that interact with SM particles weakly and have light masses \cite{Battaglieri:2017aum}. These particles have attracted significant attention as they could potentially explain DM and reconcile discrepancies between theoretical predictions and low-energy experiments \cite{PhysRevLett.116.042501,Feng:2017uoz,PhysRevD.73.072003,BOEHM2004219}.
Traditional detectors at the LHC primarily focus on 
heavier new particles in the central regions 
%transverse plane physics 
and lack the necessary sensitivity to detect light, weakly-coupled particles that are produced in the forward direction. Additionally, these particles, known as LLPs, can be highly boosted in the forward direction, traveling a macroscopic distance before decaying. Therefore, a detector located along the beamline axis in the forward region could enable the detection of light LLPs. The FASER experiment aims to address this by constructing a detector  which is 480 meters downstream from the ATLAS interaction point (IP). 
%{\color{red}The first results from a search for dark photons decaying to an electron-positron pair, using a dataset corresponding to an integrated luminosity of 27 $\text{fb}^{-1}$ collected at center-of-mass energy $\sqrt{s}= 13.6$ TeV in 2022 in LHC Run 3~\cite{Petersen:2023hgm}.} 
Furthermore, there is a proposal for FASER 2, which would be constructed from 2024-2026 to collect data during the HL-LHC era from 2026 to 2035 \cite{FASER:2019aik}. 
In the current investigation, we assume that LLPs produced near the IP travel along the beam axis and decay into SM particles, which can be detected by FASER. Therefore, LLPs within the acceptance angle of FASER should have high energies in the TeV range, as the decay products from LLPs would also possess energies close to the TeV scale. The full process is described as follows:
%In this section, we will introduce the new Long-Lived Particles detector built in LHC, FASER whose full name is ForwArd Search ExpeRiment, which spots on light and weak-interaction particles\cite{Battaglieri:2017aum}. And these particles have been attached high attention for they could yield dark matter with the correct relic density and resolve the deviations between theoretical predictions and low-energy experiments\cite{PhysRevLett.116.042501,Feng:2017uoz,PhysRevD.73.072003,BOEHM2004219}. 
%And the traditional detectors at LHC are mainly aimed at transverse plane physics, they don't have enough accuracy to search the light and weakly-coupled particles, which are produced in the very forward direction. Moreover, these kinds of particles called long-lived particles(LLPs) could be boosted in forward direction then they will fly a macroscopic distance before decaying. So a detector whose location is along with the collider  beamline axis and in the very forward region may help us to detect the light LLPs in the future, and the FASER experiment carries its mission coming.
%But as we all know, the current detectors at LHC experiment doesn't have enough accuracy to match these particles because of they are near the Interaction Point(IP). So, if there is inelastic scattering, the products conclude Long-Lived Particles which will fly a macroscopic distance before they decay so that cannot be detected by existing facilities near to IP\cite{Wang:2019xvx}. 
%And for example, new particles from pion decay are produced within angles of $\theta \sim \Lambda_{QCD}/E$ of the beam collision axis, in which $E$ is the energy of the particle. For $E \sim TeV$, such particles have only spread out $\sim$ 10 cm in the transverse plane even $\sim$ 480 m downstream\cite{FASER:2018bac,Feng:2017uoz}. That's all reason to push proceeding with constructing a small and inexpensive detector like FASER in a place far from IP so that may have the ability to take off the mask of proposed light and weakly interacting dark matter. 

%It has been mentioned that FASER is built to explore the LLPs, which is in plan to be constructed in the location IT12 and IT18 from 2019 to 2020 to collect data in Run 3 from 2021 to 2023, which is located at 480m downstream from ATLAS Interaction Point(IP). What's more, FASER 2 is in a proposal to be constructed from 2024-2026 in time so that could get the data from 2026 to 2035 in HL-LHC\cite{FASER:2019aik}. In the current FASER investigation, we assume these LLPs produced near IP could travel along the beam axis, then decay to SM particles which could be detected by FASER. Hence, LLPs within the accepting angle of FASER should possess high energy in Tev level as a result that the products from LLPs decay also have close to Tev energy. The full process is listed below,
\begin{equation}
    \text{pp} \rightarrow \chi_{2} + \chi_{1},\ \chi_{2} \ \text{travels} \sim480\text{m},\ \text{then} \ \chi_{2} \rightarrow \ \chi_{1} f\overline{f}.
\end{equation}
%But our concerned final state particles in this process are $\chi_1$ plus two oppositely charged leptons.
%From this process, we expect the signal to manifest as two oppositely charged tracks. 
The FASER detector is located in a region surrounded by rock, and the forward LHC infrastructure, including magnets and absorbers, helps to suppress potential background processes. Detailed simulations using FLUKA technology \cite{Ferrari:2005zk,BOHLEN2014211} have confirmed low radiation levels in LHC tunnels, with the radiative process associated with muons being the dominant background. To further mitigate backgrounds, a scintillating charged particle veto layer is employed in front of the detector \cite{Berlin:2018jbm}. Specifically, the FASER detector rejects high-energy charged particles, primarily muons, and protons to minimize additional troublesome backgrounds. With these technical measures, the background levels are considered negligible.
%From this process, we know that the signal should be two oppositely charged tracks, and also could be two photons that don't add in the above chain that emanate from a common vertex inside the detector and which have a combined momentum that points back through 10 m of concrete and 90 m of rock to the IP\cite{FASER:2018bac}. 
%There is one thing needs to be noticed that the decay products in light mass are collinear in some words with a very included angle actually after traveling a distance so that it demands FASER must contain a magnetic field to bend the opposite-charged track more precisely.

%The location of the FASER detector is surrounded by the rock, this situation and the forward LHC infrastructure consisting of magnets and absorbers could eliminate the potential background processes. Results from a detailed simulation using FLUKA technology\cite{Ferrari:2005zk,BOHLEN2014211} have confirmed low radiation levels in LHC tunnels, and the radiative process associated with muon is the dominant background. So, such backgrounds can be suppressed by using a scintillating charged particle veto layer in front of the detector\cite{Berlin:2018jbm}. Actually, FASER detector forbid charged particles with high energy, these particles primarily are muons and also conclude little protons to avoid adding more other troubling backgrounds. Having this technical support, we consider that backgrounds are down to a level that could be ignored.
To enhance the trigger efficiency at low energies, FASER requires a significant deposition of visible energy from the decay products of $\chi_2$, with ${E_{\text{vis}}} >$ 100 GeV. The specific parameters for the two-phase detectors, FASER and FASER 2, are cylindrical in shape, characterized by their length (L) and radius (R):
%In order to reduce the trigger rate at low energies, FASER requires a large visible energy deposition from LLP decay products, ${E_{(vis)}} > $100 GeV. The specific parameter of the above two-period detector FASER and FASER 2, whose shape are cylinder in length L and radius R is listed as follow, 
\begin{align}
\begin{split}
       \mathbf{FASER} &: L=1.5\text{m},\ R=0.1\text{m}, \\
       \mathbf{FASER\ 2} &: L=5\text{m},\ R=1\text{m}. 
\end{split}
\end{align}
Additionally, the integrated luminosity, $\mathcal{L}$, for FASER and FASER 2 is $150\ \text{fb}^{-1}$ and $3\ \text{ab}^{-1}$, respectively.
%and the luminosity $\mathcal{L}$ for FASER and FASR 2 are $150\ fb^{-1}$, $3\ ab^{-1}$ respectively.

\subsection{Displaced Muon-Jet at the LHC}
\label{subsec:4b} 

In this subsection, we focus on the signature of DMJ~\cite{Baumgart:2009tn,Cheung:2009su,Falkowski:2010cm,Izaguirre:2015pga,Kim:2016fdv,Dube:2017jgo,Zhang:2021orr} from inelastic DM models at the LHC. In the scenario where $M_{Z'} > M_{\chi_1} + M_{\chi_2}$, the $Z'$ can be on-shell produced in association with a QCD jet at the LHC~\cite{CMS:2023bay}. Subsequently, the $Z'$ decays into $\chi_1$ and $\chi_2$, and within the tracker system of the ATLAS and CMS detectors, the $\chi_2$ particle further decays into $\chi_1$ and two muons. However, due to the high boost of $\chi_2$, the resulting pair of muons from its decay becomes highly collimated, making it challenging to pass the muon isolation criteria. This phenomenon gives rise to a novel object known as a muon-jet. Our particular interest lies in a displaced dimuon vertex associated with a jet and missing momentum. This process is referred to as the DMJ signature, which is considered a particularly clean signal. 


The search strategy for this kind of signature was proposed by Ref.~\cite{Izaguirre:2015zva,Berlin:2018jbm}.
%In this part, we bring our sight to the displaced muon-jet signature. In the scenario that $M_Z' > M_{\chi_1}+M_{\chi_2} $, the dark gauge boson $Z'$ is produced on the shell associated with the jet, meanwhile, $Z'$ decays into $\chi_1$ and $\chi_2$, then the $\chi_2$ decay into $\chi_1$ and two muons inside the LHC detectors. In particular, our spot is on the decay $\chi_2$ to $\chi_1$, two muons, giving rise to a displaced dimuon vertex associated with a jet and missing momentum. When the heavier state dark matter $\chi_{2}$ highly boosted produced, the resulting pair of muons from the $\chi_{2}$ decay is too collimated to pass the muon isolation criteria and forms a novel object known as a muon-jet. The above process is named for the displaced muon-jet(DMJ) signature which is a particularly clean signal. Such a search has been proposed by the authors of~\cite{Izaguirre:2015zva}, and used by authors of~\cite{Berlin:2018jbm}. 
Our analysis of the DMJ signature follows the methodology outlined in the above two references. In their work, most of the relevant backgrounds were found to be relatively negligible. It is worth noting that displaced vertex tracks can also be present in QCD-initiated processes, which may involve the production and subsequent decay of LLPs such as $B$ or $K$ hadrons into $\pi$ and $\mu$. The authors of Ref.~\cite{Izaguirre:2015zva} assume that the probability of such events is small but provide an approximate upper bound on the probability of a QCD-initiated event producing a hard leading jet with transverse momentum $p^j_T > 120$ GeV and two displaced muons with transverse momenta $p^{\mu}_T > 5$ GeV, considering the small mass splitting between $\chi_2$ and $\chi_1$. Additionally, in order to ensure the displacement of the muon tracks, a minimum transverse impact parameter $d_\mu > 1$ mm is imposed. Furthermore, it is known that if the decay length of $\chi_2$ is sufficiently long to allow the production of two muons will through completely the tracking system, allowing for more precise track reconstruction. Hence, a requirement is imposed that the radial displacement ($R_{\chi_2}^{xy}$) of the $\chi_2$ decay vertex is less than $30$ cm. In summary, the selection criteria for the signal region of DMJ signature encompass the following conditions: 
%Representation in Ref.\cite{Izaguirre:2015zva} shows us many backgrounds they considered, one of that is QCD consideration, but most of the backgrounds are relatively negligible. Displaced vertex tracks could be found at the process QCD-initiated jet as well, which includes those giving rise to long-lived particles B or K hadrons which in turn decay into $\pi$ and $\mu$. The authors of~\cite{Izaguirre:2015zva} assume that this probability is too small but also determine an approximate upper bound on the probability for a QCD-initiated event to give a hard leading jet with transverse momentum $p_T > 120$ GeV. And two displaced vertex production muon's transverse momentum $p_T$ is limited to larger than 5 GeV because of the tiny mass splitting between $\chi_2$ and $\chi_1$. In addition, in order to ensure the muon tracks is displaced, we set up the transverse impact parameter $d_\mu > 1$ mm. At the same time, as we all know, if the deacy length is long enough to cause the final state production of $\chi_2$, two mouns to hit the outer layers of the tracking system, then we could get higher precision to reconstruct track successfully, therefore, we require that the radial displacement($R_{\chi_2}^{xy}$) of $\chi_2$ vertex need less than 30 cm. In conclusion, all selected DMJ signal region conditions should conclude the following items:
\begin{align}
    \begin{split}
        \mathbf{DMJ}:&\ p_T^j\ >\ 120\ \text{GeV}, \\
        &\ p_T^{\mu}\ >\ 5\ \text{GeV}, \\
        &\ d_{\mu} \ > \ 1\ \text{mm}, \\
        &\ R_{\chi_2}^{xy}\ <\ 30\ \text{cm}.
    \end{split}
\end{align}
The selection criteria outlined above have been carefully chosen in anticipation of an integrated luminosity of $\mathcal{L} = 3 \text{ ab}^{-1}$ at the High-Luminosity LHC. It is worth noting that extensive studies conducted by the authors of Ref.~\cite{Izaguirre:2015zva} have demonstrated that these criteria effectively reduce the backgrounds to a negligible level. 
%All the requirements shown above are selected based on expecting $\mathcal{L} = 3 \text{ab}^{-1}$( integrated luminosity) in High-Luminosity LHC  which carries our expectation to reduce backgrounds even leading to a background-free signal region.
\subsection{Soft Lepton Pair at the LHC}
\label{subsec:4c}
%In this section, we search for inelastic dark matter particles through soft lepton pair analyses, in which the final state particles contain two leptons that decay from $Z'$.This signature research uses the $\text{139 fb}^{-1}$ of $\sqrt{s} = \text{13}$ TeV proton-proton collision data collected by the ATLAS experiment at the Large Hadron Collider(LHC) as shown in the Ref.\cite{ATLAS:2019lng}. In their analyses, they select the events with missing transverse momentum and two same-flavor, oppositely charged, low transverse momentum leptons. Being consistent with Ref.\cite{ATLAS:2019lng}, we use the analysis file which had been uploaded to the \textsf{CheckMATE2}' Analysis Manager by them, and the \textsf{CheckMATE2}~\cite{Dercks:2016npn} is a program package which can help us determine whether the process we concerned in our inelastic DM model is excluded or not at $95\%$  Confidence Level(C.L.) by comparing to many recent experimental analyses. So adapting the ATLAS experiment analysis on soft lepton pair signature, we could seek the inelastic DM signature in the traditional detector at LHC. The simulated event samples in our research were generated with \textsf{MadGraph5\_aMC@NLO}(version 2.7.2), then after events reconstruction, every event entering all signal regions(SRs) share a common pre-selection which have been listed in the following Table~\ref{table.1}.
In this subsection, we investigate the search for inelastic DM models through the soft lepton pair(SLP) analysis. In our concerned process, the final state particles consist of two leptons are originated from $\chi_{2}$ decay via the off-shell $Z'$ boson. We utilize the data collected by the ATLAS detector, corresponding to an integrated luminosity of $\text{139 fb}^{-1}$ at $\sqrt{s} = \text{13}$ TeV, as described in Ref.~\cite{ATLAS:2019lng}, events with missing transverse momentum and two same-flavor, oppositely charged, low transverse momentum leptons are selected. To ensure consistency with the ATLAS analysis, we employ the analysis file provided by the \textsf{CheckMATE2} program package~\cite{Dercks:2016npn}. The \textsf{CheckMATE2} program allows us to determine whether the processes involved in inelastic DM models are excluded or not at a $95\%$ Confidence Level (C.L.), by comparing them with the results reported in Ref.~\cite{ATLAS:2019lng}. 
%By adapting this soft lepton pair signature analysis, we aim to identify the signature of inelastic DM in the traditional LHC detector. 
In our simulations, event samples are generated using \textsf{MadGraph5\_aMC@NLO}(version 2.7.2). We didn't put any pre-selections in parton level event generation, but the ME–PS matching was performed using the CKKW-L merging~\cite{Lonnblad:2011xx} scheme with the merging scale set to 15 GeV. To enforce an initial state radiation(ISR) topology, at least one parton in the final state was required to have a transverse momentum greater than 50 GeV. 
After the event reconstruction, all events entering the signal regions (SRs) undergo a common set of event selections, which is summarized in Table~\ref{table.1}.

\begin{table}[ht!] \begin{center}
 
 \begin{tabular}{l|p{8cm}} \toprule[1pt]
 
  Variable                                                                       & Event selection           \\ \hline
  Number of leptons                                                              & = 2 leptons   \\
  Leading lepton $p_T$ [GeV]                                              & $p^{\ell_1}_T > 5$       \\
  $\Delta R_{\ell \ell}$                                 &$\Delta R_{ee} > 0.3, \ \Delta R_{\mu \mu} > 0.05$      \\
  Lepton charge and flavor                                                       & $e^{\pm}e^{\mp}, \mu^{\pm}\mu^{\mp}$     \\
   $J/\psi$ invariant mass veto [GeV]                                                  & veto $3.0 < m_{\ell \ell} < 3.2$      \\
   Lepton invariant mass [GeV]                                       & $3 < m_{ee} < 60, \ 1 < m_{\mu\mu} < 60$    \\
  $E_T^\mathrm{miss}$ [GeV]       & $\geq 120$ \\
  $m_{\tau\tau}$ [GeV]                   & $m_{\tau\tau} < 0$ or $m_{\tau\tau} > 160$    \\
  Number of jets                  & $\geq 1$  \\
  Number of $b$-tagged jets                                                    & = 0     \\
  Leading jet $p_T$ [GeV]         &   $\geq 100$   \\
  min$(\Delta\phi(\mathrm{jets}, \mathbf{p}_T^\mathrm{miss}))$                          &$ > 0.4$     \\
  $\Delta\phi(j_{1}, \mathbf{p}_T^\mathrm{miss})$        & $\geq 2.0$   \\   
  %$Number of $b$-tagged jets                                                    & = 0     \\

  
  
  \bottomrule[1pt]
 \end{tabular}\end{center}
 \caption{The event selection requirements applied to all events entering SRs for soft lepton pair analysis.}
 \label{table.1}
\end{table}
According to Table~\ref{table.1}, the event selections for our signal events require exactly two leptons of the same flavor with opposite charges. We order the leading lepton ($\ell_1$) and subleading lepton ($\ell_2$) by the size of their transverse momentum. The $p_T^{\ell_1}$ is required to be larger than $5$ GeV, which helps to reduce backgrounds from fake/nonprompt (FNP) leptons. And the subleading lepton'$p_{T}$ ($p_T^{\ell_2}$) will have different constraints on different SRs. The lepton pair is also required to have a separation $\Delta R_{\ell \ell}$, with $\Delta R_{\mu \mu}$ larger than 0.05 for a muon pair and $\Delta R_{ee}$ larger than 0.3 for an electron pair. This requirement improves the efficiency of event reconstruction by avoiding overlapping electron showers in the electromagnetic calorimeter. The final state leptons must have opposite charge and same flavor. Furthermore, the invariant mass of the lepton pair ($m_{\ell \ell}$) should fall outside the range [3.0, 3.2] GeV, which removes contributions from the $J/\psi$ decays. The $m_{\ell \ell}$ is also required to be less than 60 GeV to reduce contributions from on-shell $Z$ boson decays. Requirements on the minimum angular separation between the lepton candidates ($\Delta R_{\ell\ell}$) and $m_{\ell\ell}$ remove events in which an energetic photon produces collinear lepton pairs. The variable $m_{\tau\tau}$ represents the invariant mass approximation of a pair of $\tau$ leptons undergoing leptonically decaying processes. It is defined as $m_{\tau \tau} = \text{sign}(m_{\tau \tau}^2)\sqrt{|m_{\tau \tau}^2|}$, which is the signed square root of $m_{\tau \tau}^2 \equiv 2p_{\ell_{1}}p_{\ell_{2}}(1+\zeta_1)(1+\zeta_2)$, where $p_{\ell_{1}},p_{\ell_{2}}$ are four-momentum of two leptons, while $\zeta_1 , \zeta_2$ are the parameters in solving $\mathbf{p}_\text{T}^\text{miss} = \zeta_1 \mathbf{p}_\text{T}^{\ell_1}+\zeta_2 \mathbf{p}_\text{T}^{\ell_2}$. In certain events, the $m_{\tau\tau}$ variable can be less than zero. This occurs when one of the lepton momenta has a smaller magnitude compared to the transverse missing energy ($E_T^\text{miss}$) and points in the hemisphere opposite to the momentum imbalance vector ($\mathbf{p}^\text{miss}_\text{T}$).  
In order to mitigate backgrounds originating from the Z boson decays into $\tau^+\tau^-$, events falling within the range of $0 < m_{\tau \tau} < 160$ GeV are excluded. This selection criterion achieves an efficiency exceeding 80 percent for the analyzed signals. 
%Besides, the invariant mass of the $\tau\tau$ pair ($m_{\tau\tau}$) is required to be either lower than $0$ GeV or larger than $160$ GeV, aiming to reduce contributions from $Z \rightarrow \tau\tau$ processes and increase the signal efficiency {\color{red}(How to increase the signal efficiency here ?)}.

Additionally, for the events in our research process, almost invisible momentum is carried by $\chi_{1}$, these requirements on $E_T^\mathrm{miss}$ suggest that the process our concerned is recoiling against additional hadronic activities, like ISR. All events are therefore required to have at least one jet with $p^j_{T}>100$ GeV, therefore, the missing transverse energy ($E_T^\mathrm{miss}$) is required to be greater than $120$ GeV in this analysis, even higher in some SRs. Additional jets in the event are also required to be separated from the $\mathbf{p}_{T}^\mathrm{miss}$ by $\text{min}(\Delta\phi(\text{jets}, \mathbf{p}_T^\mathrm{miss}))>0.4$ in order to suppress the impact of jet energy mismeasurement on $E_T^\mathrm{miss}$. 
%{\color{red}For searching involving ISR, the leading jet is required to be separated from the $\mathbf{p}_{T}^{miss}$ by at least 2.0 radians in $\phi$ to reduce the additional activities deriving from ISR and help us distinguish the specific process we interested. (I don't understand this sentence.)} {\color{red}(Why ?)} 
Our research process, focusing on the final state particles with two leptons, events with one or more $b$-tagged jets are vetoed to reduce backgrounds from SM $t\bar{t}$ production. %After applying all of the above event selection requirements, then the events need to go into the specific SRs provided by ATLAS experimental analyses.   


%From Table~\ref{table.1}, we could know that events require exactly two opposite-charge (OS) signal leptons of the same flavor. The two leptons possess higher $p_T$ considered to be leading lepton(${\ell_1}$) while the lower $p_T$ called subleading lepton($\ell_2$). Leading lepton'$p_T$ ($p^{\ell_1}_T$) is required larger than 5 GeV, which  could reduce backgrounds from fake/nonprompt(FNP) lepton. The lepton pair is requested to be separated by $\Delta R_{\ell \ell}$ that muon pair's $\Delta R_{\mu \mu}$ is larger than 0.05 and electron pair's $\Delta R_{ee}$ is larger than 0.3 which could help increasing efficiencies of events reconstruction by avoiding the overlapping electron showers in the EM Calorimeter. Obviously, the final state lepton charge and flavour must be $e$ or $\mu$. Another requirement is that $m_{\ell \ell}$ should locate outside of the range [3.0,3.2] GeV which removes the contributions from $J/\psi$ decays while the lepton invariant mass is demanded to lower than 60 GeV is aimed to reduce contributions from on shell $Z$-boson decays. In this pre-selection, we set the invariant mass of $\tau\tau$ pair($m_{\tau\tau}$) is lower than 0 GeV or larger than 160 GeV in order to reduce the process that $Z \rightarrow \tau\tau$ so that it could increase the efficiency of signal considered. Additionally, the $E_T^\mathrm{miss}$ is required to be greater than 120 GeV in pre-selection but some SRs may have higher requirements for $E_T^\mathrm{miss}$. %The jets and $\mathbf{p}_T^\mathrm{miss}$ should be separated by $min(\Delta\phi(jets, \mathbf{p}_T^\mathrm{miss})) > 0.4$ which help suppress the impact of jet energy mismeasurement on $E_T^\mathrm{miss}$. Our research  process with final state particles has two lepton demand events do not have $b$-tagged jets to reduce backgrounds from $t\hat{t}$ production. 
%Our research  process with final state particles has two lepton demand events that do not have $b$-tagged jets to reduce backgrounds from $t\hat{t}$ production. The above selections mentioned are all pre-selections for all events. Combined with the result file obtained from \textsf{CheckMATE2}, we find the major SRs for our research is High-$E_T^\mathrm{miss}$ region which requires the $E_T^\mathrm{miss}$ larger than 200 GeV. In these specific SRs, the subleading lepton $p_T$ has a limitation that selects the minimal value between 10 and $2+m_{\ell \ell}/3$(GeV). And the transverse mass of leading lepton which defined as $m_T^{\ell_1} = \sqrt{2(E_T^{\ell_1}E_T^\mathrm{miss}-\mathbf{p}_T^{\ell_1}\mathbf{p}_T^\mathrm{miss})}$($\mathbf{p}_T^\mathrm{miss}$ is related to the momentum of $\ell_1$ and $\ell_2$) is set lower than 60 GeV used in this High-$E_T^\mathrm{miss}$ region to reduce the contributions from FNP lepton. 

\subsection{Mono-photon Event at the STCF}
\label{subsec:4d}
In this subsection, we discuss the search for the light $Z'$ decay to $\chi_1\chi_2$ via the mono-photon signature at the future STCF which is an $e^+e^-$ collider project with a peak luminosity of $10^{35}$ $\text{cm}^{-2}\text{s}^{-1}$ and operating in $\sqrt{s} = 2$ to $7$ GeV~\cite{Barniakov:2019zhx,Liang:2021kgw}. The process of interest is $e^{+} e^{-} \rightarrow \gamma Z^{\prime} \rightarrow \gamma (\chi_1 \chi_2)$. However, this process is subject to both reducible and irreducible backgrounds due to the limited detection capability. The main reducible SM backgrounds include the processes $e^{+}e^{-} \rightarrow \gamma f \overline{f}$ and $e^{+}e^{-} \rightarrow \gamma \gamma (\gamma)$, where the final state particles are emitted in the solid angle region not covered by the detectors. %These backgrounds have a large contribution to the event sample. 
The process $e^{+}e^{-} \rightarrow \gamma e^{+}e^{-}$, where the final state electron and positron are collinear with the beam directions, receives a significant contribution from $t$-channel diagrams. The irreducible SM backgrounds to our process are the processes $e^{+}e^{-} \rightarrow \gamma \nu_{\ell} \overline{\nu}_{\ell}$, where $\nu_{\ell} = \nu_{e}, \nu_{\mu},\nu_{\tau}$ are SM neutrinos. 
%The process of interest for studying the dark mediator is $e^{+} e^{-} \rightarrow \gamma Z^{\prime}$ at the electron-positron colliders. There are many reducible backgrounds and irreducible backgrounds existing in our concerned process($e^{+} e^{-} \rightarrow \gamma Z^{\prime}, Z^{\prime} \rightarrow \chi_2 \chi_1 $), the decay chain of this process is an inelastic process, because of the limited detection capability of the subdetector. So it's important for us to investigate these reducible backgrounds as detail as possible to reduce them in higher possibility. The main reducible SM backgrounds conclude the processes $e^{+}e^{-} \rightarrow \gamma f \overline{f}$ and $e^{+}e^{-} \rightarrow \gamma \gamma \gamma$, whose final state particles emitting in the solid angle region that is uncovered by detectors, which have much large amount. Looking at the process $e^{+}e^{-} \rightarrow \gamma e^{+}e^{-}$, the cross-section will get more contribution from t channel diagrams while the final state electron and positron go along with the beam directions. The irreducible SM backgrounds to our process at electron-positron collider are $e^{+}e^{-} \rightarrow \gamma \nu_{\ell} \overline{\nu}_{\ell}$ processes, where $\nu_{\ell} = \nu_{e}, \nu_{\mu},\nu_{\tau}$ are SM neutrinos. 

%In our research, for the final photon, we adopt the cuts: $E_{\gamma} > 25$ MeV in the barrel ($|z_{\gamma}|  >  0.8$) or $E_{\gamma}  > 50$ MeV in the end-caps ($0.92  >  |z_{\gamma}|  >  0.86$), which are deriving from BESIII Collaboration~\cite{PhysRevD.96.112008}, and these cuts are both for reducible and irreducible SM backgrounds. The contribution from the reducible backgrounds is so large if we just apply the former cuts to the final photon. Here, the mono-photon reducible backgrounds at the electron-positron colliders could be removed by the next cut through the momentum conservation in the transverse direction and energy conservation~\cite{PhysRevD.100.115016, Liu:2019ogn}. Let us give a simple example of reducible backgrounds $e^{+}e^{-} \rightarrow \gamma e^{+}e^{-}$, from the conservation of energy in the center-of-mass frame and transverse momentum, we could get follow two relations,
In this study, we apply specific cuts on the final state photon to reduce background events. These cuts are based on the analysis from BESIII~\cite{PhysRevD.96.112008} and are used for both reducible and irreducible SM backgrounds. Specifically, we impose the conditions $E_{\gamma} > 25$ MeV in the barrel region ($|z_{\gamma}| < 0.8$) and $E_{\gamma} > 50$ MeV in the end-caps region ($0.92 > |z_{\gamma}| > 0.86$), where $E_{\gamma}$ is the photon energy and $z_{\gamma} \equiv \cos{\theta_{\gamma}}$ with $\theta_{\gamma}$ being the relative angle between the electron beam axis and the photon momentum. However, applying these cuts alone does not effectively suppress the contribution from reducible backgrounds, which remains significant. To further address this, we introduce an additional cut based on momentum conservation in the transverse direction and energy conservation~\cite{PhysRevD.100.115016, Liu:2019ogn,Zhang:2019wnz}. As an example, considering the reducible background process $e^{+}e^{-} \rightarrow \gamma e^{+}e^{-}$, we use energy conservation in the center-of-mass frame and transverse momentum conservation to obtain the following relations:
\begin{equation}
     E_{\gamma} + E_{1} + E_{2} = \sqrt{s}, 
\end{equation}       
\begin{equation}
E_{\gamma}\sin{\theta_{\gamma}} - E_{1}\sin{\theta_{1}} - E_{2}\sin{\theta_{2}}= 0.
\end{equation}
%Here, $E_{\gamma}, E_{1} $ and $ E_{2}$ are the final state photon and final state electron energies, and all the $\theta$ angles are representing their polar angle respectively. Under the condition both the $|\cos{\theta_{1,2}}|$ greater than or equal to $|\cos{\theta_{b}}|$, we get the final photon energy cut while the polar angle $|\cos{\theta_{b}}| \geq 0.95$,
Here, $E_{\gamma}$, $E_{1}$, and $E_{2}$ are energies of the final state photon, electron, and positron, respectively, and $\theta_{\gamma}$, $\theta_{1}$, and $\theta_{2}$ represent their respective polar angles. By imposing the condition that both $|\cos{\theta_{1,2}}|$ are greater than or equal to $|\cos{\theta_{b}}|$, where $|\cos{\theta_{b}}|$ is the polar angle at the boundary of the sub-detector where the final state electron and positron are emitted. We request $|\cos{\theta_{b}}| \geq 0.95$, and the final photon energy cut,
\begin{equation}
    E_{\gamma} \ > \ E_{b}(\theta_{\gamma}) \ = \ \frac{\sqrt{s}}{(1+\frac{\sin{\theta_{\gamma}}}{\sin{\theta_{b}}
    })}. 
\end{equation}
%where the energy cut $E_{b}$ is related to the polar angle $\theta_{b}$ which denotes the angle at the boundary of the sub-detector, at which the final state electron and positron are emitted. In our research for probing dark photon decay with an inelastic process via mono-photon signature, we aim at the center-of-mass energy in $\sqrt{s} = 4 $ GeV and $\sqrt{s}= 7$ GeV assuming the sensitivity of STCF on the mixing parameter $\epsilon$, in another word, whose differential luminosity is 30 $\text{ab}^{-1}$( integrated luminosity). 
The energy cut $E_{b}$ is determined by the polar angle $\theta_{b}$ and ensures that the final state photon lies outside the boundary region. In this work, we aim to probe the light $Z'$ decay to $\chi_1\chi_2$ via the mono-photon signature at $\sqrt{s} = 4$, $7$ GeV, 
%assuming the sensitivity of the STCF to the kinematic mixing parameter $\epsilon$, 
and corresponding to $\mathcal{L} = 30~\text{ab}^{-1}$.

\subsection{Numerical Results and Discussions} 
\label{sec:result}


%At the beginning of this work, we utilize FeynRules~\cite{Alloul:2013bka} in Mathematica to generate the the Universal FeynRules Output (UFO) model file for the inelastc DM model. Subsequently, we employ MadGraph@5NLO to obtain parton-level numerical data, such as cross-sections, for various signal processes of interest under different search strategies. 





%We now discuss the results for each strategy. {\color{red}Firstly, we focus on the LLPs search at the FASER. This strategy provides a sensitive region of $0.5\ \text{GeV} < M_{\chi_1} < 16\ \text{GeV}$ (for FASER2), extending to the smallest $\epsilon$ level of $10^{-3}$. (We need  careful checks for this result !)} In the FASER analyses, the visible energy from the decay products of $\chi_2$ must exceed 100 GeV, and the background-free consideration is adopted. {\color{blue} In the context of LLPs at FASER, we restrict ourselves for $M_{\chi} > 1$ GeV. For $M_{\chi_1}$ below $1$ GeV, the dominant generation mechanism for DM involves meson decays. However, we defer the investigation of this aspect to future studies. 
%We have observed that the lowest boundary value for $M_{\chi_1}$ in the sensitive region is 0.5 GeV. It's important to note that we haven't considered the case where $M_{\chi_1}$ is less than 0.5 GeV in the context of LLPs at FASER. When $M_{\chi_1}$ is less than 0.5 GeV, dark matter could be generated through meson decay processes, and we will finish this part in the follow-up work. 
%And due to the special location and shape of the FASER detector, it commands $\chi_2$ must have a proper corresponding lifetime which requires the $M_{\chi_1}$ couldn't be larger than nearly 16 GeV.
%{\color{red}This value is determined by our choice to set the minimum $M_{\chi_1}$ value at that point. In theory, we can extend the value of $M_{\chi_1}$ to a smaller range because the mass of the electron pair is just 1 MeV. (Terrible explanation in these sentences !! OMG!!)} 
%{\color{red}(There is no explanation for $M_{\chi_1} < 16\ \text{GeV}$ ??)}
%Secondly, the inelastic DM search strategy of idiographic DMJ signal exhibits sensitivity for $2$ GeV $< M_{\chi_1} < 210$ GeV, with a slight partial overlap with the sensitive interval of FASER. Notably, a resonance peak emerges around $M_{\chi_1} \approx 30$ GeV due to the mass mixing of $Z$ and $Z'$ in our inelastic DM models, where we set $M_{Z'} = 3 M_{\chi_1}$. {\color{blue}The left boundary line of this search strategy is about $2$ GeV deriving from the rest mass of the muon pair in the final state is nearing $0.2$ GeV which is satisfied with the relationship $\Delta M = 0.1 M_{\chi_1}$. The upper bound mass sensitivity of this strategy is originating from that we require a radial displacement of the $\chi_2$ decay vertex of $R_{\chi_2}<30$ cm.} 
%{\color{red}(Again, there is no explanation for $M_{\chi_1} < 210\ \text{GeV}$ ??)}%{\color{brown}Still confusing with the reason why largest bound mass is 210 GeV} 
%The third strategy is the SLP search at 13 TeV LHC, which reveals irregular exclusion regions for $19$ GeV $< M_{\chi_1} < 200$ GeV. {\color{blue}The bottom and upper bound DM mass sensitivity of this strategy is because the final state soft leptons characters, here, contain the invariant mass constraints.???} 
%{\color{red}(There is no explanation for $19$ GeV $< M_{\chi_1} < 200\ \text{GeV}$ ??)} 
%Similar to the DMJ signature, a resonance peak appears at $M_{\chi_1} = 30$ GeV. The last strategy in this work is searching for inelastic DM via mono-photon signature at the STCF. The specific boundary lines of this strategy indicate its sensitivity even when $M_{\chi_1}$ extends to sufficiently small values ($M_{\chi_1} = 10^{-3}$ GeV in our study). In this strategy, we consider two center-of-mass energies, $\sqrt{s} = 4$, $7$ GeV. The bottom sensitive boundary line in this strategy is almost parallel with the horizontal axis when we adopt the energy cut $E_b$ to reduce the reducible background. 
%The right bounds of $M_{\chi_1}$ come from the kinematic bound of $M_{Z'}$ which we only focus on the on-shell $Z'$ decays in this study. 
%We find that the 3 times values of right bound $M_{\chi_1}$ in this strategy is obey to rule below the center-of-mass energy. 
%Comparing these two scenarios, we observe that larger $\sqrt{s}$ leads to greater sensitivity along the axis of $M_{\chi_1}$. Conversely, a larger $\sqrt{s}$ covers a smaller range in the kinematic mixing parameter $\epsilon$ axis when $\epsilon$ ranges from $0.1$ to smaller values (indicating that a smaller $\sqrt{s}$ can explore smaller values of $\epsilon$). There is an overlap between the sensitive regions of FASER and STCF, as well as a slight overlap between STCF and LHC(DMJ) when the STCF is at $\sqrt{s} = 7$ GeV.

%As shown in the Fig~\ref{fig:4}, we find that all of the searching strategies in our work have ability to probe the parameter space not excluded by current experimental data in which the $\chi_1$ can be 100 percent of DM. Comparing the four different strategies, we could find they have their own advantage parameter space. It is worth noting that in our work, regardless of whether the search strategy adopted has background interference or not, the confidence level of the signal we finally obtained is 95 percent excluded or preferred. Firstly, the LLPs search at FASER give us a  $M_{\chi_1}$ region that $0.5\ \text{GeV} < M_{\chi_1} < 16\ \text{GeV}$ (for FASER2) and it can extend the smallest $\epsilon$ to $10^{-4}$ level. In the analyses at FASER, we have a command that the significant deposition of visible energy from the decay products of $\chi_2$ must larger than 100 GeV. Secondly, the idiographic signal of DMJ has a sensitivity to $M_{\chi_1}$ between 2.0 GeV and 210 GeV. There is a slight partial overlap with the sensitive interval of FASER. Obviously, at the strategy of DMJ, the $M_{\chi_1}$ can be pushed to $10^2$ level. And, by careful observation, it is easy to notice that when the $M_{\chi_1}$ is approximately 30 GeV, a prominent resonance peak will emerge at that energy because of mass mixing of $Z$ and $Z'$ in our inelastic dark matter model and we set $M_{Z'} = 3\times M_{\chi_1})$ in our analyses. Thirdly, the soft lepton pair search at 13 TeV LHC tell us that there is an irregular exclusion when $M_{\chi_1}$ between 19 GeV and 200 GeV. Same with the DMJ signal, there also have a resonance peak when the $M_{\chi_1}$ equal to 30 GeV. At last, let us focus on STCF, it's specific boundary line show us that it still have sensitivity when the $M_{\chi_1}$ extends to a sufficiently small area (in our work, the smallest $M_{\chi_1}$ is $10^{-3}$ GeV.). And in this strategy, we consider two center-of-mass energy $\sqrt{s} = 4$ GeV and $\sqrt{s} = 4$. By comparing two situation, we could easily find that the larger center-of-mass energy can have larger $M_{\chi_1}$ sensitivity in it's mass axis. On the contrary, the larger center-of-mass energy have small length in mixing parameter $\epsilon$ axis when the $\epsilon$ start from $10^{-1}$ to smaller one (mean the smaller center-of-mass energy can explore smaller $\epsilon$). There is an overlap between sensitive regions of FASER and STCF and a slight overlap between STCF and LHC(DMJ) when the center-of-mass energy of STCF is $\sqrt{s}=7$ GeV.              

%The Existing constraints (gray) and projected sensitivities (color) to models of fermionic inelastic DM in the dark matter candidate mass of $M_{\chi}$ and the gauge mixing $\epsilon$ parameter plane. The dark U(1) guage coupling $g_D$ is set as $g_D = 1.12$ , $M_{Z'}=3\times M_{\chi_1}$ and the mass-splitting $\Delta_{\chi}$ is set as $0.1\times M_{\chi_1}$. Along the black contour, the abundance of $\chi_{1}$ matches the observed dark matter energy density. The gray regions are excluded from LEP~\cite{Hook:2010tw, Curtin:2014cca} and BaBar~\cite{BaBar:2017tiz}. Here, we show the constraints from SLAC E137 and LSND beam dump, which both contain decay and scatter processes through gray regions as well. The colored contours show the projected reach of various proposed searches for displaced vertices at LHC, such as at ATLAS and CMS (orange)~\cite{PhysRevD.93.063523, Berlin:2018jbm} and FASER1 and FASER2 (dark dashed blue and dark blue)~\cite{FASER:2018bac, FASER:2019aik, Feng:2017uoz}. We also show the excluded region to soft lepton search at ATLAS by recast experimental analyses (purple)~\cite{ATLAS:2019lng}. Last, we show the sensitivity of dark photon search at STCF via monophoton by two center-of-frame energy $\sqrt{s} = 4$ GeV and $\sqrt{s} = 7$ GeV (magenta and cyan)~\cite{Epifanov:2020elk}.  }  

\begin{figure}[t!]
\centering
\includegraphics[width=0.45\textwidth]{p_theta_faser_and_dmj.pdf}
\includegraphics[width=0.45\textwidth]{p_theta_slp_and_dmj.pdf}
\caption{The $P_{\chi_2}-\theta$ distribution of $\chi_2$. The left panel is under the search strategies of FASER and LHC(DMJ) with $M_{\chi_2}$ = 5.5\ GeV and $M_{Z'} = 15$ GeV. The right panel is under the search strategies of DMJ and SLP at the LHC with $M_{\chi_2}$ = 11\ GeV and $M_{Z'} = 30$ GeV. The vertical line (dashed grey) in the left panel indicates the decay length in the lab frame for FASER 2 (480 m), which tells us there exist sensitive regions with $\chi_2$ when $\theta<10^{-2.68}$.} 
\label{fig:3}
\end{figure}

In Fig~\ref{fig:3}, we present the event distribution over the magnitude of the spatial momentum $P$ of $\chi_2$ and its angle $\theta$ along the beam line under different search strategies on various benchmark points. In the left panel, the $P_{\chi_2}-\theta$ distribution in the lower left area corresponds to the LLPs search at the FASER, while the distribution in the upper right corner represents the DMJ search at the LHC. These distributions are obtained for fixed $M_{Z'} = 15$ GeV and $\epsilon = 2 \times 10^{-3}$. The dashed grey line denotes the acceptance angle for FASER 2 ($\theta < 10^{-2.68}$), illustrating that the FASER detector has sensitivity primarily to highly boosted particles with a very small angle relative to the beam line. The reason for this is that the $\chi_2$ produced in this process has a very low transverse momentum ($P_T$), which is nearly equal to $M_{\chi_2}$, and it is emitted in the forward direction, with its trajectories collimated along the beam-line. The DMJ event distribution tells us the high off-beam-line sensitivity because there exists a high $P_T$ jet($>$120 GeV) back to $Z'$. Since we only require $P_T^{\mu} > 5 \text{GeV}$ in this analysis, it leads to the lower threshold of $P_{T}^{\chi_2}$ $\geq$ 50 GeV ($P_T^{\mu} \sim \Delta \times P_{T}^{\chi_2}$). We can clearly find that the distribution pattern in the left panel indicates a kinematic overlap between these two strategies. Next, the right panel in Fig.~\ref{fig:3} corresponds to the SLP searches at 13 TeV LHC and the DMJ searches at the LHC. These distributions are obtained for fixed $M_{Z'} = 30$ GeV and $\epsilon = 10^{-2}$. The lower left $P_{\chi_2}-\theta$ distribution represents the SLP search, while the upper right part corresponds to the DMJ search. 
%{\color{blue}In the SLP search strategy, 
 %the initial generations we gave conclude two processes($p p > \chi_2 \chi_1$ and $p p > \chi_2 \chi_1 j$), and we command that $P_T^{j} > 100 $ GeV and $P_T^{\ell_1}>5$ in the subsequent analysis. Following the above analysis, 
%we cannot find that the SLP event distribution is located between the strategies of FASER and DMJ. The DMJ event distribution in the right panel almost has the same style as the left panel but under a different parameter setting, which tells us it has the same kinematic distribution directly.} 
 We observe a larger kinematic overlap between these two strategies compared to the left panel, which is consistent with the results presented in Fig.~\ref{fig:2}.
%In the Fig~\ref{fig:2}, we give the distribution of the spatial momentum and its angle between the beam-line $\theta$ of heavier state dark matter $\chi_2$ under the different searching strategies on different benchmark points. The $P-\theta$ distribution in the lower left area of the left panel is from LLPs searching at FASER, and the same distribution in the upper right corner is from DMJ searches at LHC, which are both under the fixed $Z'$ mass $M_{Z'}$ = 15\ GeV and the gauge mixing parameter $\epsilon$ = $2 \times 10^{-3}$. The distribution situation in the left panel tell us that there is an kinematical overlap between above mentioned strategies, and the dashed grey line show us the receptive angle for FASER 2 ($\theta < 10^{-2.68}$) telling us that FASER detector only have sensitivity with the highly boosted particles with a very small angle to the beam-line. The right panel has same type with the left panel but the distribution are from the strategies SLP searches at 13 TeV LHC with ATLAS detector and DMJ searches at LHC which are both under the fixed $Z'$ mass $M_{Z'}$ = 30\ GeV and the gauge mixing parameter $\epsilon$ = 0.01. The lower left $P-\theta$ distribution is from strategy of SLP while the upper right part is from DMJ searching. We can find that these two strategies have larger kinematically overlapped regions comparing with left panel which is consistent with the results shown in Fig~\ref{fig:4}. 

%Next, in order to find out what kind of the heavier state dark matter $\chi_2$ would be captured by FASER detector. Here, in Fig \ref{fig:2}, we plot the 2D distribution over the total spatial momentum P of the $\chi_2$ and the angle $\theta$ it subtends with the beam axis which is under the condition of fixed $Z'$ mass $M_{Z'}$ = 30\ GeV. The horizontal lines indicate momenta corresponding to the decay length in the lab frame for a fixed gauge mixing parameter $\epsilon$ = $2 \times 10^{-3}$ at 480m which is the approximate distance from the location of FASER to the IP. From this plot, there is a vertical line representing the angle for FASER 2 ($\theta < 10^{-2.68}$). It tells us FASER 2 only selects the distribution of heavier state dark sector $\chi_2$ which are extremely boosted with a very small angle to the beam axis which explains that it could probe the region with large gauge mixing parameter having shown in Fig \ref{fig:2}. 

\begin{figure}[htbp]
\centering
\includegraphics[width=0.45\textwidth]{dRllv4.eps}
\includegraphics[width=0.45\textwidth]{dphi_tracksMETv3.eps}
\caption{
The left panel represents the $\Delta R$ distributions between two leptons in the final state from the $\chi_2$ decays, and the right panel shows the $\Delta \phi$ distributions between the spatial momentum of the lepton tracks and the missing energy. }
\label{fig:4}
\end{figure}

In Fig~\ref{fig:4}, we provide the kinematic distributions of $\Delta R_{\ell \ell}$ and $\Delta \phi_{\text{Tracks, MET}}$, with the fixed parameter points in their own probing regions, $M_{\chi_1} = 1$ GeV and $\epsilon = 10^{-2}$ for the FASER, $M_{\chi_1} = 10$ GeV and $\epsilon = 10^{-2}$ for the LHC (DMJ), and $M_{\chi_1} = 160$ GeV and $\epsilon = 7\times10^{-2}$ for the LHC (SLP) respectively, to illustrate their distinct kinematic properties in these three search strategies. In the left panel of Fig~\ref{fig:4}, we present the opening angle distribution of two leptons, $\Delta R_{\ell \ell}$, originating from the decay of $\chi_2$. The STCF signature is not considered in this analysis as its mass range is predominantly below the GeV scale. The DMJ search at LHC exhibits the largest collinear feature of a lepton pair compared to the other two search strategies because of the initial high $P_T$ jet in this signature. In contrast, the SLP searches at the LHC show a relatively weak collimation among the three strategies. 
%Production in the first step at FASER are $\chi_2$ and $\chi_1$, and  $\chi_2$ is LLP highly boosted and along with beam-line while the production in the first step at DMJ is on-sell $Z'$ associated with a jet and $Z'$ is highly boosted when recoiling off the high $P_T$ jet. 
%From the above kinematic analysis, we know that the energy of $\chi_2$ in the search strategy at the FASER is larger than LHC(DMJ) and LHC(SLP).} 
The right panel of Fig~\ref{fig:4} displays the relative azimuthal angle between the lepton tracks and the missing energy, $\Delta \phi_{\text{Tracks, MET}}$. We observe that the DMJ signature at the LHC exhibits a smaller $\Delta \phi_{\text{Tracks, MET}}$ compared to the signature at the FASER. This difference arises from the fact that the DMJ process involves an initial high $p_T$ jet causing a high energy $Z'$ recoiling this jet, therefore, leading the productions of $Z'$ to be relatively collinear. In contrast, the situation at the FASER results in larger relative angles between the produced $\chi_1$ and $\chi_2$. 
%Consequently, the DMJ signature shows a smaller $\phi_{\text{Tracks, MET}}$ distribution. 
For the SLP searches, which involve matching and merging effects of QCD jets, the distribution of $\phi_{\text{Tracks, MET}}$ tends to be more evenly distributed. Therefore, $\Delta R_{\ell \ell}$ and $\Delta \phi_{\text{Tracks, MET}}$ distributions provide insights into the kinematic characteristics of different search strategies. %By analyzing these distributions, we can better understand the distinct features of each strategy nd their impact on the observed kinematic variables.

%To illustrate the different properties of the strategies we adopted in this work, in Fig~\ref{fig:3}, we plot the histograms of kinematic variables $\Delta R_{\ell \ell}$ and $\Delta \phi_{\text{Tracks, MET}}$. In the left panel of Fig~\ref{fig:3}, it shows us that the relative distance $\Delta R_{\ell \ell}$ of two leptons from heavier state dark matter $\chi_2$ decay in three strategies on GeV level, we don't consider the signature at STCF because of the mass range of STCF are almost below GeV. Due to the respective feature of these three strategies, the $\Delta R_{\ell \ell}$ in LLPs searches at FASER have largest collinear characteristic while compared with the rest two strategies, and the DMJ searches at LHC have tiniest collimated feature in these strategies. The right panel display the relative phi value between lepton track and missing energy. It tell us that DMJ signature at LHC have smaller $\Delta \phi_{\text{Tracks, MET}}$ comparing with the signature at FASER because of the DMJ has an initial high $p_T$ jet, another only product in first step of this process $Z'$ is back to that jet completely but situation at FASER is that the product of first step are $chi_2$ and $chi_1$. So, the first $chi_1$ and second $chi_1$ at FASER will have larger relative angle comparing with same situation at DMJ searches at LHC leading to DMJ signature have smaller $\phi_{\text{Tracks, MET}}$. Since we need to do merging and matching in SLP searches, this signature will have a relatively average distribution of $\phi_{\text{Tracks, MET}}$. 
In our analyses, we take the benchmark parameters as~\cite{Kang:2021oes,Berlin:2018jbm,Izaguirre:2017bqb,Izaguirre:2015zva,Bertuzzo:2022ozu}: $M_{Z'} = 3 M_{\chi_{1}}$, $\alpha_{D} \equiv g_{D}^{2}/4\pi =0.1$, and $\Delta_{\chi}= 0.1 M_{\chi_{1}}$ . After applying the above event selections and search strategies, we present the final results regarding the projected sensitivity for different search strategies. These results will be displayed in the $\left(M_{\chi_1}, \epsilon\right)$ plane for $10^{-3}$ GeV $< M_{\chi_1} < 10^{3}$ GeV. This region is particularly interesting as it contains extensive areas that are still unconstrained by current experimental measurements. The strategies proposed in this study aim to cover a large portion of these unconstrained regions related to this model. 

\begin{figure}[t!]
\centering
\includegraphics[scale=0.5]{Contour_iDMv4.pdf}
\caption{The existing bounds (gray) and sensitivities (color) for the fermionic inelastic DM models in the $\left(M_{\chi_1}, \epsilon\right)$ plane with fixed $\alpha_D$=0.1, $M_{Z'}=3 M_{\chi_1}$, $\Delta_{\chi}$=$0.1 M_{\chi_1}$. The black contour represents the region where the abundance of $\chi_1$ matches the observed DM relic density~\cite{Izaguirre:2015zva,PhysRevD.96.055007,Duerr:2019dmv}. The light gray regions represent the excluded regions from LEP~\cite{Hook:2010tw, Curtin:2014cca} and BaBar~\cite{BaBar:2017tiz} as well as LSND and SLAC E137 beam-dumps~\cite{Izaguirre:2017bqb}. The colored contours indicate the projected reach of different strategies. Specifically, the orange contours correspond to the reach of searches at ATLAS and CMS~\cite{PhysRevD.93.063523, Berlin:2018jbm}, while the dark dashed blue and dark blue contours represent the reach of FASER1 and FASER2~\cite{FASER:2018bac, FASER:2019aik, Feng:2017uoz}. The purple region shows the excluded region from the soft lepton pair search at ATLAS based on recast experimental analyses~\cite{ATLAS:2019lng}. The sensitivity of the STCF search via monophoton is displayed in magenta and cyan for $\sqrt{s} = 4$, $7$ GeV~\cite{Epifanov:2020elk}. 
}
\label{fig:2}
\end{figure}

Our main findings are illustrated in Fig.~\ref{fig:2}, which includes both light gray regions and colored contour regions. The light gray regions represent the current experimental constraints 
%on the search for invisible dark photons, 
derived from experiments such as BaBar~\cite{BaBar:2017tiz}, LEP~\cite{Hook:2010tw, Curtin:2014cca}, and CMS $13$ TeV DMJ~\cite{CMS:2023bay}. Besides, the boundary lines derived from fixed target experiments E137 and LSND can be found in Ref.~\cite{Izaguirre:2017bqb}. In contrast, the colored contour regions depict the projected sensitivities obtained through the four strategies employed in our analysis\footnote{The FASER results from our analyses are slightly different from the corresponding results in Ref.~\cite{Berlin:2018jbm}. The main reasons for these differences stem from the absence of consideration for the production of DM states from meson decays, the distinction in $c_W$ resulting from the $Z'$ coupling to fermions in the SM, and the selection of different size parameters by the detectors.}. Finally, the black bold line represents the parameter space where the abundance of $\chi_1$ agrees with the measured DM relic density~\cite{Izaguirre:2015zva,PhysRevD.96.055007,Duerr:2019dmv}. As illustrated in Fig~\ref{fig:2}, our analysis demonstrates that all the search strategies employed in this work are capable of probing the parameter space that is not excluded by current constraints. 


In this study, we have chosen a fixed ratio of the $Z'$ mass to the $\chi_1$ mass, specifically $M_{Z'}/M_{\chi_1} = 3$. However, if the value of $M_{Z'}/M_{\chi_1}$ is increased, both the relic density curve and projected sensitivities line in our search strategies would shift upward compared to that in Fig.~\ref{fig:2}.  %{\color{blue}The situation that $M_{Z'}/M_{\chi_1}$ becomes particularly large will be constrained by cosmology.} 
Inversely, the relic density curve and projected sensitivities line shift downward when $M_{Z'}/M_{\chi_1}$ decreases. 
%{\color{red}(1. How about the situation for the value of $M_{Z'}/M_{\chi_1}$ decreasing ? 2. What are the impacts for our search strategies when increasing or decreasing $M_{Z'}/M_{\chi_1}$ ? Can we still use them to study these situations or we need to consider new search strategies ? If we need new search search strategies, what are they ?)} 
On the other hand, we have set the mass splitting $\Delta_{\chi}/M_{\chi_1} = 0.1$ in our analysis. If $\Delta_{\chi}/M_{\chi_1}$ is decreased, the relic density curve would shift downward, but the projected results of different strategies will shift upward to disappear sensitivity while the $\Delta_{\chi}/M_{\chi_1}$ is small enough so we need the other search strategies to further explore these parameter space. 
%like displacement experiments like MATHUSLA~\cite{MATHUSLA:2022sze}, CODEX-b~\cite{Aielli:2022awh}, and AL3X~\cite{Dercks:2018wum}. 
%In the case of the DM mass range we investigated includes GeV, so the $\Delta_{\chi}/M_{\chi_1}$ is not larger than 0.1 which is under the cosmological considerations. 
%{\color{red}(Again, 1. How about the situation for the value of $\Delta_{\chi}/M_{\chi_1}$ increasing ? 2. What are the impacts for our search strategies when increasing or decreasing $\Delta_{\chi}/M_{\chi_1}$ ? Can we still use them to study these situations or we need to consider new search strategies ? If we need new search search strategies, what are they ?)} 
It is important to note that while our work demonstrates good complementary among the four strategies in probing inelastic DM, there are still other exploration methods that can be pursued including time-delayed tracks at the LHC~\cite{Liu:2018wte, Berlin:2018jbm} and other LLPs experiments like MATHUSLA~\cite{MATHUSLA:2022sze}, CODEX-b~\cite{Aielli:2022awh}, AL3X~\cite{Dercks:2018wum}, and so on. 


\section{Conclusion} 
\label{sec:conclusion}

In this work, we study the prospects of searching for the inelastic DM at colliders. Due to the constraint of DM relic density, the mass splitting between the heavier and lighter dark states should be small to achieve the coannihialtion, which leads to some unconventional signatures in this model. For the inelastic DM mass in the range of 1 MeV to 210 GeV, we find that most of the parameter space that can provide the correct relic density could be probed by searching for the long-lived particles at the FASER, the displaced muon-jets and soft leptons at the LHC, and the mono-photon events at the STCF.

\section*{Acknowledgments} 

We thank Wei Liu, Yu Zhang and Bin Zhu for the helpful discussions. 
The work is supported by the National Natural Science Foundation of China (NNSFC) under grant No. 12147228 and Project 12047503 supported by NSFC.
\bibliography{refs}

\end{document}
