\section{Conclusions}

The general process of label-based DR evaluation relies on the assumption that the original data has good CLM, which can lead to erroneous conclusions when this assumption is violated.
We introduce two new distortion measures---\LT and \LC  (\ltc)---that use class labels for DR evaluation while eliminating the need to check the validity of the CLM assumption. Our quantitative experiments show that \ltc outperforms previous DR measures in terms of precision and sensitivity in detecting Missing and False Groups distortions. Use cases show that \ltc can be used to characterize DR techniques and their hyperparameters.

As future work, we will study new CVM to make \ltc more sensitive to the CLM distortions than using DSC or \CHb. 
Enriching the embedding with CLM distortions \cite{lespinats11cgf} could also better inform analysts about the credibility of visual patterns. Yet another direction would be to evaluate supervised DR techniques with \ltc. 
\rev{We also believe that supervised DR techniques using class labels in their optimization process could benefit from incorporating \ltc in their loss function.}
Overall, our proposal aims toward getting more trustworthy DR-based visual analysis.