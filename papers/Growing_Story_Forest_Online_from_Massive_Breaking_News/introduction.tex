%!TEX root = main.tex
\section{Introduction}
\label{sec:intro}


With information explosion in the fast-paced modern society, tremendous volumes of articles on trending and breaking news are being generated on a daily basis by various Internet media providers, e.g., Yahoo! News, CNN, Tencent News, Sina News, etc. In the meantime, it becomes increasingly difficult for normal readers to digest such a large amount of streaming news information. Search engines perform document retrieval from large corpora based on user-defined queries that specify what are interesting to the user. However, they do not provide a natural way for users to view what is going on. Furthermore, search engines return a list of ranked documents and do not provide structural summaries of trending topics or breaking news.

An emerging alternative way to visualize news corpora without pre-specified queries is to organize and present news articles through event timelines \cite{yan2011evolutionary,  wang2016socially}, event threads \cite{nallapati2004event}, event evolution graphs \cite{yang2009discovering}, or information maps \cite{shahaf2012trains,shahaf2013information,xu2013summarizing}. However, till today few existing news information organization techniques are turned into large-scale deployment due to several reasons:

First of all, despite research efforts in Topic Detection and Tracking (TDT) \cite{yang2002multi, allan1998line}, it remains challenging to extract distinguishable ``events'' at a proper granularity, as building blocks of the news graph, from today's vast amount of open-domain daily news. The news articles may cover extremely diverse topics and contain redundant information about a same conceptual event published by different sources. For example, simply connecting individual articles \cite{shahaf2012trains} or named entities \cite{faloutsos2004fast} in a graph will lead to redundant and entangled information. On the other hand, connecting co-occuring keyword sets in an information map \cite{shahaf2013information} can greatly reduce the fine details of news graphs. But even with the keyword graph, a user still needs to put additional efforts to understand the large number of articles associated with each keyword set.    

Second, many recently proposed event graphs or information maps try to link events in an arbitrary evolution graph \cite{yang2009discovering} or permitting intertwining branches in the information map \cite{shahaf2013information}. However, we would like to point out that such overly complex graph structures do not make it easy for users to quickly visualize and understand news data. In fact, unlike a novel or a complex story about a celebrity queried from a search engine, most breaking news stories follow one of a few typical developing structures. In fact, for breaking news summary that will appeal to commercial uses, simple story structures are preferred.

Most importantly, most existing event timeline or event graph generation schemes are based on \emph{offline} optimization over the entire news corpora, while for a system that visualizes breaking news, it is desirable to ``grow'' the stories in an online fashion without disrupting or restructuring the previously generated stories. On one hand, online computation can prevent repeated processing of older documents. On the other hand, an online scheme can deliver a consistent story development structure to users, so that users can quickly visualize what's new in the hot events that they are trying to follow. Furthermore, given the vast amount of collected daily news data, the entire online computation to identify new events and extend the existing story graphs will incur less delay. 


%, how to group them by the stories they are describing and capture the story development structure remains a challenging problem. Techniques and systems that are capable of clustering documents by events, extracting underlying news events structure, and ranking news stories by hotness are highly valuable.

In this paper, we present our experience of implementing \textit{StoryForest}, which is a comprehensive system to organize vast amounts of breaking news data into easily readable story trees of events in an online fashion.
We make careful design choices for each component in this large system, with the following contributions:

First, our system can accurately cluster massive amounts of long news documents into conceptually clean events through a novel two-layer document clustering procedure that leverages a wide range of feature engineering and machine learning techniques, mainly including keyword extraction, keyword community detection, a pre-trained classifier to detect whether two documents are talking about the same event, and a graph-based document clustering procedure. On a labeled news dataset, our proposed text clustering procedure significantly outperforms a number of existing text clustering schemes.

Second, our system further groups the discovered events into stories, where each story is represented by a \emph{tree} of events. A link between two events indicates the temporal migration or a causal relationship between two events. Compared with existing story generation systems such as StoryGraph \cite{yang2009discovering} and MetroMap \cite{shahaf2012trains}, we propose an online algorithm to evolve story trees incrementally based on daily news, without any churn of reforming the graph when new data arrive.
As a result, each story is presented in one of several easy-to-view structures, i.e., either a linear timeline, a flat structure, or a tree with branches, which we believe are sufficient to represent story structures of most breaking news. 
 %\textit{Third}, the final generated stories are ranked by calculated hotness, and then they can be further utilized by potential tasks such as hot news recommendation.

Finally, we evaluated the performance of our system based on 60 GB of Chinese news documents collected from all the major Internet news providers in China (including Tencent, Sina, WeChat, Sohu, etc.) in a three-month period from October 1, 2016 to December 31, 2016, covering extremely diverse topics in the open domain. 
We also conducted a detailed and extensive pilot user experience study for (long) news document clustering and news story generation to evaluate how our system as well as several baseline schemes conform to the habit of human readers.

According to the pilot user experience study, our system outperforms multiple state-of-the-art news clustering and story structure generation systems such as KeyGraph \cite{sayyadi2013graph} and StoryGraph \cite{yang2009discovering} in terms of logical validity of the generated story structures, as well as the conceptual purity of each identified event and story. Experiments show that the average time for our Java-based system to finish event clustering and story structure generation based on the daily data is less than $30$ seconds on  a MacBook Pro with a 2 GHz Intel Core i7 processor, and 8 GB memory. Therefore, our system proves to be highly efficient and practical.

%Java and running experiments on a MacBook Pro Version 2013, with a 2 GHz Intel Core i7 processor, and 8 GB 1600 MHz DDR3 memory.

It is worth mentioning that our work represents the first system that is able to efficiently process vast amounts of Chinese news data into organized story structures, although our proposed algorithms and schemes are also applicable to news data in English (and other languages) by simply replacing the word segmentation and NLP tools with the counterparts for the corresponding language.    
%\begin{itemize}
%\item We implement and present a news document processing system named \textit{Story Forest} which is capable of clustering documents by event, organizing events into stories with tree structure, online updating stories.
%\item We propose efficient algorithms for document clustering and story structure online generation.

 





%A bunch of existing research works sought to solving different sub-problems. 

%For document clustering, existing works \cite{},\cite{} mainly based on ....

%For story structure generation, ...

%However, few work has been reported that combines all different component into a unified framework. Besides, for industry, new documents are being generated all the time and online algorithms are needed for timely updating hot stories, so that users can get access to news updates quickly. However, most existing systems generates story structures by modeling it into an optimization problem, which is time consuming for real-world application. 



%The remainder of the paper is organized as follows. Sec. \ref{sec:problem} defines our problem and related notations. Sec. \ref{sec:system} presents different components of \textit{Story Forest} system in detail, describes our detailed procedure to extract events from a news corpus of large amounts of real-world news data, and our proposed story tree model for story structure generation. In Sec. \ref{sec:eval}, we describe our experimental setup and evaluate our system and proposed algorithms over a real-world large-scale dataset. Sec. \ref{sec:related} reviews the related literature. Finally, we conclude the paper in Sec. \ref{sec:conclude}.  


\begin{figure*}
\includegraphics[width=6.7in]{figure/CaseStudy}
\caption{The story tree of ``2016 U.S. presidential election.''}
\label{fig:CaseStudy}
\vspace{-2mm}
\end{figure*}

