\documentclass[sigconf]{acmart}

\usepackage{booktabs} % For formal tables


\usepackage{amssymb}
\usepackage{amsthm, amsmath}
\usepackage{algorithmic}
\usepackage[ruled]{algorithm2e}
\usepackage{color}
\usepackage{url}
\usepackage{xfrac}

\usepackage{subfigure}
\usepackage{tablefootnote}

\newcommand{\red}[1]{\textcolor{red}{#1}}
\newcommand{\blue}[1]{\textcolor{blue}{#1}}

% Copyright
%\setcopyright{none}
%\setcopyright{acmcopyright}
%\setcopyright{acmlicensed}
%\setcopyright{rightsretained}
%\setcopyright{usgov}
%\setcopyright{usgovmixed}
%\setcopyright{cagov}
%\setcopyright{cagovmixed}



%Conference
%\acmConference[WOODSTOCK'97]{ACM Woodstock conference}{July 1997}{El
%  Paso, Texas USA} 
%\acmYear{1997}
%\copyrightyear{2016}

%\acmPrice{15.00}

\begin{document}
\title{Growing Story Forest Online from Massive Breaking News}
% \titlenote{Produces the permission block, and
%   copyright information}
% \subtitle{Extended Abstract}
% \subtitlenote{The full version of the author's guide is available as
%   \texttt{acmart.pdf} document}

\copyrightyear{2017} 
\acmYear{2017} 
\setcopyright{acmcopyright}
\acmConference{CIKM'17}{}{November 6--10, 2017, Singapore.}
\acmPrice{15.00}
\acmDOI{https://doi.org/10.1145/3132847.3132852}
\acmISBN{ISBN 978-1-4503-4918-5/17/11}

\fancyhead{}
\settopmatter{printacmref=false, printfolios=false}
% \author{Bang Liu}
% %\authornote{Dr.~Trovato insisted his name be first.}
% %\orcid{1234-5678-9012}
% \affiliation{%
%   \institution{Department of Electrical and Computer Engineering}
%   \streetaddress{University of Alberta}
%   \city{Edmonton} 
%   \state{Alberta, Canada} 
%   %\postcode{43017-6221}
% }
% %\email{bang3@ualberta.ca}

% \author{Di Niu}
% %\authornote{The secretary disavows any knowledge of this author's actions.}
% \affiliation{%
%   \institution{Department of Electrical and Computer Engineering}
%   \streetaddress{University of Alberta}
%   \city{Edmonton} 
%   \state{Alberta, Canada} 
%   %\postcode{43017-6221}
% }
% %\email{}

% \author{Kunfeng Lai}
% %\authornote{This author is the one who did all the really hard work.}
% \affiliation{%
%   \institution{Mobile Internet Group}
%   \streetaddress{Tencent}
%   \city{Shenzhen} 
%   \country{China}}
% %\email{}

% \author{Linglong Kong}
% \affiliation{%
%   \institution{Department of Mathematical and Statistical Sciences}
%   \streetaddress{University of Alberta}
%   \city{Edmonton} 
%   \state{Alberta, Canada} 
%   %\postcode{43017-6221}
% }
% %\email{}

% \author{Yu Xu}
% \affiliation{%
%   \institution{Mobile Internet Group}
%   \streetaddress{Tencent}
%   \city{Shenzhen} 
%   \country{China}}
% %\email{}




% \author{Charles Palmer}
% \affiliation{%
%   \institution{Palmer Research Laboratories}
%   \streetaddress{8600 Datapoint Drive}
%   \city{San Antonio}
%   \state{Texas} 
%   \postcode{78229}}
% \email{cpalmer@prl.com}

% \author{John Smith}
% \affiliation{\institution{The Th{\o}rv{\"a}ld Group}}
% \email{jsmith@affiliation.org}

% \author{Julius P.~Kumquat}
% \affiliation{\institution{The Kumquat Consortium}}
% \email{jpkumquat@consortium.net}

% % The default list of authors is too long for headers}
% \renewcommand{\shortauthors}{B. Trovato et al.}

\author{Bang Liu$^1$, Di Niu$^1$, Kunfeng Lai$^2$, Linglong Kong$^1$, Yu Xu$^2$}
       \affiliation{$^1$University of Alberta, Edmonton, AB, Canada}
 \affiliation{$^2$Mobile Internet Group, Tencent Inc., Shenzhen, China}
       %\email{haolan@ualberta.ca}

%\email{bang3@ualberta.ca}


% \numberofauthors{5}
% \author{
% % You can go ahead and credit any number of authors here,
% % e.g. one 'row of three' or two rows (consisting of one row of three
% % and a second row of one, two or three).
% %
% % The command \alignauthor (no curly braces needed) should
% % precede each author name, affiliation/snail-mail address and
% % e-mail address. Additionally, tag each line of
% % affiliation/address with \affaddr, and tag the
% % e-mail address with \email.
% %
% % 1st. author
% %\alignauthor
% Bang Liu$^1$, Di Niu$^1$, Kunfeng Lai$^2$, Linglong Kong$^1$, Yu Xu$^$\\
%        \affaddr{$^1$University of Alberta, Edmonton, AB, Canada}\\
%  \affaddr{$^2$Mobile Internet Group, Tencent Inc., Shenzhen, China}\\
%        %\email{haolan@ualberta.ca}
% }



\begin{abstract}
We describe our experience of implementing a news content organization system at Tencent that discovers events from vast streams of breaking news and evolves news story structures in an online fashion.  
Our real-world system has distinct requirements in contrast to previous studies on topic detection and tracking (TDT) and event timeline or graph generation, in that we 1) need to accurately and quickly extract distinguishable events from massive streams of long text documents that cover diverse topics and contain highly redundant information, and 2) must develop the structures of event stories in an online manner, without repeatedly restructuring previously formed stories, in order to guarantee a consistent user viewing experience. In solving these challenges, we propose \emph{Story Forest}, a set of online schemes that automatically clusters streaming documents into events, while connecting related events in growing trees to tell evolving stories. %Our system is specifically tailored for fast processing massive amounts of trending and breaking news data at Tencent.  
We conducted extensive evaluation based on 60 GB of real-world Chinese news data, although our ideas are not language-dependent and can easily be extended to other languages, through detailed pilot user experience studies. The results demonstrate the superior capability of Story Forest to accurately identify events and organize news text into a logical structure that is appealing to human readers, compared to multiple existing algorithm frameworks.\footnote{This work was sponsored by CCF-Tencent Open Fund.}

% \footnote{This is an abstract footnote}. 
\end{abstract}

%
% The code below should be generated by the tool at
% http://dl.acm.org/ccs.cfm
% Please copy and paste the code instead of the example below. 
%
% \begin{CCSXML}
% <ccs2012>
%  <concept>
%   <concept_id>10010520.10010553.10010562</concept_id>
%   <concept_desc>Computer systems organization~Embedded systems</concept_desc>
%   <concept_significance>500</concept_significance>
%  </concept>
%  <concept>
%   <concept_id>10010520.10010575.10010755</concept_id>
%   <concept_desc>Computer systems organization~Redundancy</concept_desc>
%   <concept_significance>300</concept_significance>
%  </concept>
%  <concept>
%   <concept_id>10010520.10010553.10010554</concept_id>
%   <concept_desc>Computer systems organization~Robotics</concept_desc>
%   <concept_significance>100</concept_significance>
%  </concept>
%  <concept>
%   <concept_id>10003033.10003083.10003095</concept_id>
%   <concept_desc>Networks~Network reliability</concept_desc>
%   <concept_significance>100</concept_significance>
%  </concept>
% </ccs2012>  
% \end{CCSXML}

% \ccsdesc[500]{Computer systems organization~Embedded systems}
% \ccsdesc[300]{Computer systems organization~Redundancy}
% \ccsdesc{Computer systems organization~Robotics}
% \ccsdesc[100]{Networks~Network reliability}

% We no longer use \terms command
%\terms{Theory}

\keywords{Text Clustering; Online Story Tree; Information Retrieval}
\maketitle

\section{Introduction}  \label{sec:introduction}

\newcommand\inexpIntro[3]{#1?(#2,#3).}
\newcommand\rinexpIntro[3]{*#1?(#2,#3).}
\newcommand\outexpIntro[3]{#1!(#2,#3).}
\newcommand\outatomIntro[3]{#1!(#2,#3)}

We propose a fully automated method for proving termination of \(\pi\)-calculus processes.
Although there have been a lot of studies on termination analysis for the \(\pi\)-calculus
and related calculi~\cite{Deng06IC,Demangeon07,SangiorgiTermination,KobayashiHybrid,Yoshida04IC,DBLP:journals/jlp/DemangeonHS10,Venet98SAS}, most of them have been rather theoretical,
and there have been surprisingly little efforts in developing  fully automated termination
verification methods and tools based on them. To our knowledge,
Kobayashi's \typical{}~\cite{TyPiCal,KobayashiHybrid} is the only exception that
can prove termination of \(\pi\)-calculus processes (extended with natural numbers)
fully automatically, but its termination analysis is quite limited (see Section~\ref{sec:relatedwork}).

Our method is based on a reduction to termination analysis for sequential programs:
we translate a \(\pi\)-calculus process \(P\) to a sequential program \(S_P\), so that
if \(S_P\) is terminating, so is \(P\). The reduction allows us to use
powerful, mature methods and tools
for termination analysis of sequential programs~\cite{heizmann2016ultimate,freqterm,DBLP:conf/lics/PodelskiR04,Kuwahara2014Termination,DBLP:journals/cacm/CookPR11}.

The idea of the translation is to convert a chain of communications on replicated input
channels to a chain of recursive function calls of the target sequential program.
Let us consider the following Fibonacci process:
\begin{align*}
    & \rinexpIntro{\fib}{n}{r}
        \ifexp{n<2}{ \soutatom{r}{1} \\ &\quad}
                   { \nuexp{s_1} \nuexp{s_2} (\outatomIntro{\fib}{n-1}{s_1} \PAR \outatomIntro{\fib}{n-2}{s_2} \PAR \sinexp{s_1}{x}\sinexp{s_2}{y}\soutatom{r}{x+y}) \\}
    & \PAR \outatomIntro{\fib}{m}{r}
\end{align*}
Here, the process
$\rinexpIntro{\fib}{n}{r} \ldots$ is a function server that computes the \(n\)-th Fibonacci number
in parallel and returns the result to \(r\),
and $\outatom{\fib}{m}{r}$ sends a request for computing the \(m\)-th Fibonacci number;
those who are not familiar with the syntax of the \(\pi\)-calculus may wish to consult
Section~\ref{sec:targetlanguage} first.
To prove that the process above is terminating for any integer \(m\),
it suffices to show that there is no infinite chain of communications on $\fib$:
\[
    \fib(m,r) \to \fib(m_1,r_1) \to \fib(m_2,r_2) \to \cdots.
\]
We convert the process above to the following program:\footnote{The actual translation
  given later is a little more complex.}
\begin{verbatim}
 let rec fib(n) = if n<2 then () else (fib(n-1) [] fib(n-2)) in
 fib(m)
\end{verbatim}
Here, \texttt{[]} represents the non-deterministic choice.
Note that, although the calculation of Fibonacci numbers is not preserved,
for each chain of communications on \texttt{fib}, there is a corresponding
sequence of recursive calls:
\[
\mathtt{fib}(m) \to \mathtt{fib}(m_1) \to \mathtt{fib}(m_2) \to \cdots.
\]
Thus, the termination of the sequential program above implies the termination of
the original process.
As shown in the example above, (i) each communication on a replicated input channel
is converted to a function call, (ii) each communication on a non-replicated input
channel is just removed (or, in the actual translation, replaced by a call of
a trivial function defined by \(f(\seq{x})=(\,)\)), and (iii) parallel composition
is replaced by a non-deterministic choice.
We formalize the translation outlined above and prove its correctness.

The basic translation sketched above sometimes loses too much information.
For example, consider the following process:
\begin{align*}
    & \rinexpIntro{\pre}{n}{r} \soutatom{r}{n-1} \\
    & \PAR \rinexpIntro{f}{n}{r} \ifexp{n<0}{ \soutatom{r}{1} }
                                       { \nuexp{s} (\outatomIntro{\pre}{n}{s} \PAR \sinexp{s}{x}\outatomIntro{f}{x}{r}) } \\
    & \PAR \outatomIntro{f}{m}{r}
\end{align*}
The translation sketched above would yield:
\begin{verbatim}
  let pred(n) = n-1 in
  let rec f(n) = if n<0 then () else (pred(n) [] f(*)) in
  f(m)
\end{verbatim}
Here, \texttt{*} represents a non-deterministic integer: since we have removed
the input $\sinatom{s}{x}$, we do not have information about the value of \( x \).
As a result, the sequential program above is non-terminating, although the original
process is terminating.
To remedy this problem, we also refine the basic translation above by using a refinement
type system for the \(\pi\)-calculus. Using the refinement type system,
we can infer that the value of \(x\) in the original process is less than \(n\),
so that we can refine the definition of \texttt{f} to:
\begin{verbatim}
 let rec f(n) = ... else (pred(n) [] let x=* in assume(x<n);f(x))
\end{verbatim}
The target program is now terminating, from which
we can deduce that the original process is also terminating.
We have implemented an automated tool based on the refined translation above.

The contributions of this paper are summarized as follows.
\begin{itemize}
\item The formalization of the basic translation from the \(\pi\)-calculus
  (extended with integers) to sequential programs, and a proof of its correctness.
\item The formalization of a refined translation based on a refinement type system.
\item An implementation of the refined translation, including automated refinement type
  inference based on CHC solving, and experiments to evaluate the effectiveness of
  our method.
\end{itemize}

The rest of this paper is structured as follows.
Section~\ref{sec:targetlanguage} introduces the source and target languages
of our translation.
Section~\ref{sec:approach} 
formalizes the basic translation, and proves its correctness.
Section~\ref{sec:refinement} refines the basic translation by using a refinement type system.
Section~\ref{sec:implementation} reports an implementation and experiments.
Section~\ref{sec:relatedwork} discusses related work,
and Section~\ref{sec:conclusion} concludes the paper.

%!TEX root = main.tex
\section{Problem Definition and Notations}
\label{sec:problem}







% In this section, we will first describe key concepts and notations used in this paper, and formally define our problem. Then we will use a case study to make our idea of story tree more concrete.

% \subsection{Problem Definition and Notations}
% \label{subsec:problem-define}

We first present some definitions of key concepts in the top-down hierarchy: \textit{topic} $\rightarrow$ \textit{story} $\rightarrow$ \textit{event} to be used in this paper.

\begin{definition}
  \textit{Event}: an event $\mathcal{E}$ is a set of one or several documents that contain highly similar information.
\end{definition}

\begin{definition}
  \textit{Story}: a story $\mathcal{S}$ is a tree of events that revolve around a group of specific persons and happen at certain places during specific times. A directed edge from event $\mathcal{E}_1$ to $\mathcal{E}_2$ indicates a temporal evolution or a logical connection from $\mathcal{E}_1$ to $\mathcal{E}_2$.
\end{definition}

\begin{definition}
  \textit{Topic}: a topic consists of a set of stories that are highly correlated or similar to each other.
  \vspace{-1mm}
\end{definition}


Each topic may contain multiple story trees, and each story tree consists of multiple logically connected events.
In our work, events (instead of news documents) are the smallest atomic units. Each event is also assumed to belong to a single story and contains partial information about that story.
For instance, considering the topic \textit{American presidential election}, \textit{2016 U.S. presidential election} is a story within this topic, and  \textit{Trump and Hilary's first television debate} is an event within this story.


We now introduce some notations and describe our problem formally. Given a news document stream $D = \{ \mathcal{D}_1, \mathcal{D}_2, \ldots, \mathcal{D}_t,\ldots \}$, where $\mathcal{D}_t$ is the set of news documents collected on time period $t$, our objective is to: a) cluster all news documents $D$ into a set of events $E = \{ \mathcal{E}_1, \ldots, \mathcal{E}_{|E|} \}$, and b) connect the extracted events to form a set of stories $S = \{ \mathcal{S}_1, ..., \mathcal{S}_{|S|} \}$. Each story $\mathcal{S} = (E, L)$ contains a set of events $E$ and a set of links $L$, where $L_{i,j} := <\mathcal{E}_i, \mathcal{E}_j>$ denotes a directed link from event $\mathcal{E}_i$ to $\mathcal{E}_j$, which indicates a temporal evolution or logical connection relationship.

%We now illustrate our problem with an example. (A example Fig) Fig... shows ...
Furthermore, we require the events and story trees to be extracted in an online or incremental manner. That is, we extract events from each $\mathcal D_t$ individually when the news corpus $\mathcal D_t$ arrives in time period $t$, and \emph{merge} the discovered events into the existing story trees that were found at time $t-1$. This is a unique strength of our scheme as compared to prior work, since we do not need to repeatedly process older documents and can deliver  a set of evolving yet logically consistent story trees to users.  

% \subsection{Case Study}
% \label{subsec:case-study}

\begin{figure}
\includegraphics[width=3.4in]{figure/StoryStructures}
\caption{Different structures to characterize a story.}
\vspace{-2mm}
\label{fig:storyStructures}
\vspace{-2mm}
\end{figure}

For example, Fig.~\ref{fig:CaseStudy} illustrates the story tree of ``2016 U.S. presidential election''. The story contains $20$ nodes, where each node indicates an event in 2016 U.S. election, and each link indicates a temporal evolution or a logical connection between two events. %For example, event $19$ says America votes to elect new president, and event $20$ says Donald Trump is elected president. 
The index number on each node represents the event sequence over the timeline. There are $6$ paths within this story tree, where the path $1 \rightarrow 20$ indicates the whole presidential election process, branch $3 \rightarrow 6$ is about Hilary's health conditions, branch $7 \rightarrow 13$ talks about television debates, $14 \rightarrow 18$ depicts the investigation into Hilary's ``mail door'', etc. As we can see, by modeling the evolutionary and logical structure of a story into a story tree, users can easily grasp the logic of news stories and learn the main information quickly. 


Let us represent each story by an empty root node $s$ from which the story is originated, and denote each event by an event node $e$. The events in a story can be organized in one of the following four structures shown in Fig. \ref{fig:storyStructures}: a) a flat structure that does not include dependencies between events; b) a timeline structure that organizes events by their timestamps; c) a graph structure that checks the connection between all pairs of events and maintains a subset of most strong connections; d) a tree structure, which represents a story's evolving structure by a tree.  

Compared with a tree structure, sorting events by timestamps omits the logical connection between events, while using directed acyclic graphs to model event dependencies without considering the evolving consistency of the whole story can leads to unnecessary connections between events.
Through extensive user experience studies in Sec.~\ref{sec:eval}, we show that tree structures are the most effective way to represent breaking news stories as compared to other structures, including the more complex graph structures. 

%!TEX root = main.tex
\section{The Story Forest System}
\label{sec:system}


\begin{figure*}
\includegraphics[width=6.7in]{figure/System}
\caption{An overview of the system architecture of \textit{Story Forest}.}
\label{fig:system}
\vspace{-0mm}
\end{figure*}

%In this section, we provide an overview of the proposed \textit{Story Forest} system.
%Then we describe our detailed procedures of extracting events from a news corpus of large amounts of real-world text data in each time period, organizing related events into stories, and modeling stories' evolutionary structure by story trees.



%\subsection{System Overview}
%\label{subsec:system-overview}

An overview of our \textit{Story Forest} system is shown in Fig.~\ref{fig:system}, which mainly consists of three components: preprocessing, document clustering and story tree update, divided into 5 steps. First, the input news document stream will be processed by a variety of NLP and machine learning tools, mainly including document filtering, word segmentation and keyword extraction. Second, steps 2--3 will cluster documents into events in a novel 2-layer procedure as follows.
For news corpus $\mathcal D_t$ in each time period $t$, we form a keyword graph \cite{sayyadi2013graph} from these documents based on keyword co-occurrence, and extract topics as subgraphs from the keyword graph using community detection algorithms. The topics with few keywords will be discarded. After each topic is found, we find all the documents associated with the topic, and further cluster these documents into events through a semi-supervised document clustering procedure aided by a pre-trained document-pair relationship classifier.
Finally, in steps 4--5 we update the story trees (formed previously) by either inserting each discovered event into an existing story tree at the right place, or creating a new story tree if the event does not belong to any existing story. Note that each topic may contain multiple story trees and each story tree consists of logically connected events.
We will explain the design choices of each component in detail in the following.


\subsection{Preprocessing}
\label{subsec:preprocessing}
When a new set of news documents arrives,  we need to clean, filter documents, and extract features that will be helpful to the steps that follow. 
Our preprocessing module mainly includes the following three steps, which are critical to the overall system performance:

\textbf{Document filtering}: unimportant documents with content length smaller than a threshold (20 characters) will be discarded.

\textbf{Word segmentation}: we segment the title and body of each document using Stanford Chinese Word Segmenter \textit{Version 3.6.0} \cite{chang2008optimizing}, which has proved to yield excellent performance on Chinese word segmentation tasks. Note that for data in a different language, the corresponding word segmentation tool in that language can be used instead. 

% \textbf{Document topic classification}: we trained SVM classifiers to classify each document into one of $30$ different topic categories,  including politics, sociology, entertainment, finance, etc., based on the document's TF-IDF feature. The one with the maximum classification score will be selected.

\textbf{Keyword extraction}: extracting keywords from each document to represent the main concepts of the document is quite critical to the performance and efficiency of the entire system. We found that traditional keyword extraction approaches, such as TF-IDF based keyword extraction and TextRank \cite{mihalcea2004textrank}, are not sufficient to achieve good performance for real-world news data. For example, the TF-IDF based method measures each word's importance by frequency information; it cannot detect keywords that yet have a relatively low frequency. The TextRank algorithm utilizes the word co-occurrence information and is able to handle such cases. However, its efficiency is relatively low, with time cost increasing significantly as the document length increases.
% Alternatively, we may manually design a fine-tuned rule-based system to combine different strategies based on the observed results. However, such type of systems highly relies on the rule design, and often introduces other systematic errors.


\begin{table}
  \caption{Features for the classifier to extract keywords.}
  \label{tab:features}
  \begin{tabular}{lp{5.5cm}}
    \toprule
    Type & Features\\
    \midrule
    Word feature & Named entity or not, location name or not, contains angle brackets or not. \\
    Structural feature & TF-IDF, whether appear in title, first occurrence position in document, average occurrence position in document, distance between first and last occurrence positions, average distance between word adjacent occurrences, percentage of sentences that contains the word, TextRank score.\\
    Semantic feature & LDA\tablefootnote{We trained a $1000$-dimensional LDA model based on news data collected from January 1, 2016 to May 31, 2016 that contains $300,000+$ documents.
    % The training process costs $30$ hours.
    }\\
    \bottomrule
  \end{tabular}
  \vspace{-3mm}
\end{table}

\begin{figure}
\includegraphics[width=3.0in]{figure/KeywordClassify}
\caption{The classifier to extract keywords.}
\vspace{-1mm}
\label{fig:keywordClassify}
\vspace{-3mm}
\end{figure}

To efficiently and accurately extract keywords, we constructed a supervised learning system to classify whether a word is a keyword or not for a document.
% In particular, we trained an $1000$-dimensional LDA model based on 6 months of news documents (the dataset here is not the one we used in the evaluation in Sec.~\ref{sec:eval}).
In particular, we manually labeled the keywords of $10,000+$ documents, including $20,000+$ positive keyword samples and $350,000+$ negative samples.
Table~\ref{tab:features} lists the main features that we found critical to the binary classifier.

%As we combined different types of features, feature preprocessing must be carefully designed to improve the performance of classifiers such as Logistic Regression (LR) or Support Vector Machine (SVM).
A straightforward idea is to input the raw features listed above to a Logistic Regression (LR). However, as a linear classifier, LR relies on careful feature engineering.
To reduce the impact of human judgement in feature engineering, we combine a Gradient Boosting Decision Tree (GBDT) with the LR classifier to get the binary yes/no classification result, as shown in Fig. \ref{fig:keywordClassify}. GBDT, as a nonlinear model, can automatically discover useful cross features or feature combinations from raw features and discretize continuous features. 
The output of the GBDT will serve as the input of the LR classifier. Finally, the LR classifier will determine whether a word is a keyword or not for the document in question. We also tried SVM as the classifier in the second layer instead of LR and observed similar performance. Our final keyword extraction precision and recall rate are $0.83$ and $0.76$, while they are $0.72$ and $0.76$ respectively if we don't add the GBDT component.

\subsection{Document Clustering and Event Extraction}
\label{subsec:eventClustering}

% \begin{algorithm}
% \caption{Graph-based Document Clustering to Obtain Events}\label{alg:graph-cluster}
% \KwIn{A set of news documents $\mathcal{D} = \{ d_1, d_2, ..., d_{|\mathcal{D}|}\}$, with extracted features described in Sec. \ref{subsec:preprocessing}. }
% \KwOut{A set of events $E = \{ \mathcal{E}_1, \mathcal{E}_2, ..., \mathcal{E}_{|E|} \}$.}

% \begin{algorithmic}[1]
% 	\STATE Construct a keyword co-occurrence graph $\mathcal{G}$ of all documents' keywords $w_i$. There is an edge $e_{i,j} = <w_i, w_j>$ if the times that the keywords $w_i$ and $w_j$ co-occur exceed a certain threshold, and $\Pr\{w_j | w_i\},\ \Pr\{w_i | w_j\}$ are bigger than another threshold. 
	
% 	\STATE Split $\mathcal{G}$ into a set of small and strongly connected keyword communities $C = \{\mathcal{C}_1, \mathcal{C}_2, ..., \mathcal{C}_{|C|} \}$, based on the community detection algorithm \cite{ohsawa1998keygraph}. The algorithm keeps splitting a graph by iteratively delete edges with high betweenness centrality score, until a stop condition is satisfied. 

% 	\FOR{each keyword community $\mathcal{C}_i,\ i= 1,\ldots,|C|$}
% 		\STATE Retrieve a subset of documents $D_i$ which is highly related to this keyword community by calculating the cosine similarity between the TF-IDF vector of each document and that of the keyword community, and comparing it to a  threshold.

% 		\STATE Divide $D_i$ into clusters according to the document topics.

% 		\STATE Further split each cluster into events by comparing the titles of each pair of documents, after word segmentation and dropping stop words.
		
% 		%each pair of documents' title keywords. %Until this step, each document cluster is an event $\mathcal{E}$.
% 	\ENDFOR

% \end{algorithmic}
% \end{algorithm}

After document preprocessing, we need to extract events. Event extraction here is essentially a fine-tuned document clustering procedure to group conceptually similar documents into events. Although clustering studies are often subjective in nature, we show that our carefully designed procedure can significantly improve the accuracy of event clustering, conforming to human understanding, based on a manually labeled news dataset.
% Algorithm~\ref{alg:graph-cluster} shows the detailed steps of such document clustering.
To handle the high accuracy requirement for long news text clustering, we propose a $2$-layer clustering approach based on both keyword graphs and document graphs.

\textit{First}, we construct a large keyword co-occurrence graph \cite{sayyadi2013graph} $\mathcal{G}$. Each node in $\mathcal{G}$ is a keyword $w$ extracted by the scheme described in Sec.~\ref{subsec:preprocessing}, and each undirected edge $e_{i,j}$ indicates that $w_i$ and $w_j$ have ever co-occured in a same document. 
Edges that satisfy two conditions will be kept and other edges will be dropped: the times of co-occurrence shall be above a minimum threshold (we use $3$ in our system), and the conditional probabilities of the occurrence $\Pr\{w_j| w_i\}$ and $\Pr\{w_i | w_j\}$ also need to be bigger than a predefined threshold (we use $0.15$), where the conditional probability $\Pr\{w_j| w_i\}$ represents the probability that $w_j$ occurs in a document if the document contains word $w_i$.

\textit{Second}, we perform community detection in the constructed keyword graph. This step aims to split the whole keyword graph $\mathcal{G}$ into communities $C = \{\mathcal{C}_1, \mathcal{C}_1, ..., \mathcal{C}_{|C|}\}$, where each community $\mathcal{C}_i$ contains the keywords for a certain topic (to which multiple stories may be associated). 
The benefit of using community detection in the keyword graph is that each keyword can appear in multiple communities, which makes sense in reality. 
We also tried another method of clustering keywords by \textit{Word2Vec}.
However, the performance is worse than community detection based on co-occurrence graphs. The reason is that using word vectors tends to cluster the words with similar semantic meanings. However, unlike articles in a specialized domain, in long news documents in the open domain, it is highly possible that keywords with different semantic meanings can co-occur in the same event.

To detect keyword communities, we utilize the \emph{betweenness centrality score} \cite{sayyadi2013graph} of edges to measure the strength of each edge in the keyword graph. An edge's betweenness score is defined as the number of shortest paths between all pairs of nodes that pass through it. An edge between two communities is expected to achieve a high betweenness score. Edges with high betweenness score will be removed iteratively to extract communities. The iterative splitting process will stop until the number of nodes in each sub-graph is smaller than a predefined threshold, or until the maximum betweenness score of all edges in the sub-graph is smaller than a threshold that depends on the sub-graph's size. We refer interested readers to \cite{sayyadi2013graph} for more details about community detection.

After we obtain the keyword communities, we calculate the cosine similarity between each document and a keyword community.  The documents are represented by TF-IDF vectors. As a keyword community is a bag of words, it can also be considered as a document. We assign each document to the keyword community which gives the highest similarity and the similarity is above a predefined threshold. Up to now, we have finished document clustering in the first layer, i.e., the documents are grouped according to topics. 

\textit{Third}, we further perform the second-layer document clustering within each topic to obtain fine-grained events. We also call this process \emph{event clustering}. An event only contains documents that talk about the same semantic event. To yield fine-grained event clustering, unsupervised learning is not sufficient. 
Instead, we adopt a supervised-learning-guided clustering procedure in the second layer.


Specifically, we train an SVM classifier to determine whether a pair of documents are talking about the same event or not using a bunch of document-pair features as the input, including the cosine similarities of content TF-IDF and TF vectors, the cosine similarities of title TF-IDF and TF vectors, the similarity of the first sentences in the two documents, etc.

For each pair of documents within a same topic, we decide whether to connect them or not according to the prediction made by the document-pair relationship classifier mentioned above. Hence, the documents in each topic will form a document graph. We then apply the same community detection algorithm mention above to such document graphs. 
Note that the graph-based clustering on the second layer is highly efficient, since the number of documents contained in each topic is significantly smaller after the first-layer document clustering. 

In a nutshell, our 2-layer scheme groups documents into topics based on keyword community detection and further groups the documents within each topic into fine-grained events. For each event $\mathcal{E}$, we also record the set of keywords $\mathcal{C}_{\mathcal{E}}$ of the topic (keyword community) which it belongs to, which will be helpful in the subsequent story tree development.


%!TEX root = main.tex
\subsection{Growing Story Trees Online}
\label{sec:tree}


% According to our observations on real-world news data, a tree structure is sufficient to capture the evolving structures of most stories for breaking news and trending topics which often last for a constrained time period. Besides, as the number of event nodes increase, a graph structure will be too complex to clearly reveal the main logic flows in the stories, while a tree structure provides a clearer view of different story paths, including branches and the main thread. Furthermore, when grow trees in an online manner, it will be easy for users to identify the online change of a story, where a tree update operation is simply inserting a new event node on the right branch.


% \begin{algorithm}
% \caption{Online Story Forest Growing}\label{alg:story-structure}
% \KwIn{A stream of documents $D = \{ \mathcal{D}_1, \mathcal{D}_2, ..., \mathcal{D}_T\}$ incoming by time. Event compatibility threshold $\delta$.}
% \KwOut{A story forest $\mathcal{F} = \{\mathcal{S}_1, \mathcal{S}_2, ..., \mathcal{S}_{|\mathcal{F}|}\}$ that dynamically changes with time.}

% \begin{algorithmic}[1]
% 	\STATE Perform event clustering on the first batch of documents $\mathcal{D}_1$ to get events $E_1 = \{\mathcal{E}_1, \mathcal{E}_2, ..., \mathcal{E}_{|E_1|}\}$. 

% 	\STATE Create a story tree $\mathcal{S}_1$ from $\mathcal{E}_1$. Iteratively processing other events in $E_1$ using the same steps as described below. Then we get initial story forest $\mathcal{F}_1 = \{\mathcal{S}_1, \mathcal{S}_2, ..., \mathcal{S}_{|\mathcal{F}_1|}\}$  

% 	\WHILE{a new batch of documents $\mathcal{D}_t$ comes at time slot $t$}
% 		\STATE Perform event clustering and get a set of events $E_t = \{\mathcal{E}_1, \mathcal{E}_2, ..., \mathcal{E}_{|E_t|}\}$.

% 		\FOR{each event $\mathcal{E} \in E_t$}

% 			\STATE Match $\mathcal{E}$ with each story $\mathcal{S} \in \mathcal{F}_{t-1}$.

% 			\IF{$\mathcal{E}$ belongs to story $\mathcal{S}$}
% 				\STATE Compare $\mathcal{E}$ with each story node $\mathcal{E}_{\mathcal{S}, i} \in \mathcal{S}$ to check whether they are describing the same event. If yes, $\mathcal{E}_{\mathcal{S}, i} = merge(\mathcal{E}_{\mathcal{S}, i},\ \mathcal{E})$, and continue to process the next event. Otherwise, continue the following steps.

% 				\STATE Set $matchIdx \leftarrow -1,\ \text{linkScore}_{max} \leftarrow -1$

% 				\FOR{Each event $\mathcal{E}_{\mathcal{S}, j} \in \mathcal{S}$}

% 					\STATE Calculate $\text{linkScore}(\mathcal{E}, \mathcal{E}_{\mathcal{S}, j})$ according to Equation \ref{eqn:linkScore}.

% 					\IF{$\text{linkScore}_{max} < \text{linkScore}(\mathcal{E}, \mathcal{E}_{\mathcal{S}, j})$}

%                     	\STATE $\text{linkScore}_{max} \leftarrow \text{linkScore}(\mathcal{E}, \mathcal{E}_{\mathcal{S}, j}),\ matchIdx \leftarrow j$

% 					\ENDIF

% 				\ENDFOR

% 				 \IF{$\text{linkScore}_{max} \geq \delta$}
% 				 	\STATE Insert $\mathcal{E}$ into $\mathcal{S}$, and add link $<E_{\mathcal{S}, matchIdx},\ E>$ to $\mathcal{S}$.
% 				 \ELSE

% 				 	\STATE Insert $\mathcal{E}$ into $\mathcal{S}$, and add link $<ROOT,\ E>$ to $\mathcal{S}$.

% 				 \ENDIF
% 			\ELSE
% 				\STATE Create a new story tree $\mathcal{S}$ from $\mathcal{E}$ and add it to story forest $\mathcal{F}_t$.
% 			\ENDIF
% 		\ENDFOR
% 	\ENDWHILE

% \end{algorithmic}
% \end{algorithm}

Given the set of extracted events for a particular topic, we further organize these events into multiple stories under this topic in an online manner. Each story is represented by a \textit{Story Tree} to characterize the evolving structure of that story.
% Algorithm \ref{alg:story-structure} describes how we organize events into story trees in an online manner.
Upon the arrival of a new event and given an existing story forest, our online algorithm to grow the story forest mainly involves two steps: a) identifying the story tree to which the event belongs; b) updating the found story tree by inserting the new event at the right place. 
If this event does not belong to any existing story, we create a new story tree.


{\bf a) Identifying the related story tree.} 
Given a set of new events $E_t = \{\mathcal{E}_1, \mathcal{E}_2, ..., \mathcal{E}_{|E_t|}\}$ at time period $t$ and an existing story forest $\mathcal{F}_{t-1} = \{ \mathcal{S}_1, \mathcal{S}_2, ..., \mathcal{S}_{|\mathcal{F}_{t-1}|}\}$ that has been formed during previous $t-1$ time periods, our objective is to assign each new event $\mathcal{E} \in E_t$ to an existing story tree $\mathcal{S} \in \mathcal{F}_{t-1}$. If no story in the current story forest matches that event, a new story tree will be created and added to the story forest. %otherwise, we continue the steps after matching stories.
%also explain same event filtering.

We apply a two-step strategy to decide whether a new event $\mathcal{E}$ belongs to an existing story tree $\mathcal{S}$ formed previously.
% Specifically, we measure 1) keyword graph similarity, 2) document similarity and 3) title similarity sequentially to make a final decision about whether an event matches a certain story tree.
First, as described at the end of Sec. \ref{subsec:eventClustering}, event $\mathcal{E}$ has its own keyword set $\mathcal{C}_{\mathcal{E}}$.
Similarly, for the existing story tree $\mathcal{S}$, there is an associated keyword set $\mathcal{C}_{\mathcal{S}}$ that is a union of all the keyword sets of the events in that tree.

Then, we can calculate the compatibility between event $\mathcal{E}$ and story tree $\mathcal{S}$ as the Jaccard similarity coefficient between $\mathcal{C}_{\mathcal{S}}$ and $\mathcal{C}_{\mathcal{E}}$: 
$
  \text{compatibility}(\mathcal{C}_{\mathcal{S}}, \mathcal{C}_{\mathcal{E}}) = \frac{|\mathcal{C}_{\mathcal{S}} \cap \mathcal{C}_{\mathcal{E}}|}{|\mathcal{C}_{\mathcal{S}} \cup \mathcal{C}_{\mathcal{E}}|}.
$
If the compatibility is bigger than a threshold, we further check whether at least a document in event $\mathcal{E}$ and at least a document in story tree $\mathcal{S}$ share $n$ or more common words in their titles (with stop words removed). If yes, we assign event $\mathcal{E}$ to story tree $\mathcal{S}$. Otherwise, they are not related. In our experiments, we set $n=1$. 
% Notice that the document frequencies (DFs) of words change while news documents keep arriving. We maintain a time window of length $T_{df}$ days to update the document frequencies of words. In our case, we set $T_{df}=2$.
If the event $\mathcal{E}$ is not related to any existing story tree, a new story tree will be created.


\begin{figure}
\includegraphics[width=3.3in]{figure/NodeOperation}
\caption{Three types of operations to place a new event into its related story tree.}
\label{fig:nodeOperations}
\vspace{-2mm}
\end{figure}

{\bf b) Updating the related story tree.} After a related story tree $\mathcal{S}$ has been identified for the incoming event $\mathcal{E}$, we perform one of the 3 types of operations to place event $\mathcal{E}$ in the tree: \textit{merge}, \textit{extend} or \textit{insert}, as shown in Fig.~\ref{fig:nodeOperations}.
The \textit{merge} operation merges the new event $\mathcal{E}$ into an existing event node in the tree. The \textit{extend} operation will append event $\mathcal{E}$ as a child node to an existing event node in the tree. Finally, the \textit{insert} operation directly appends event $\mathcal{E}$ to the root node of story tree $\mathcal{S}$. Our system chooses the most appropriate operation to process the incoming event based on the following procedures.

{\bf \emph{Merge}}: we merge $\mathcal{E}$ with an existing event in the tree, if they essentially talk about the same event.
This can be achieved by checking whether the centroid documents of the two events are talking about the same thing using the document-pair relationship classifier described in Sec.~\ref{subsec:eventClustering}. The centroid document of an event is simply the concatenation of all the documents in the event.
% If yes, we merge the new incoming event with the existing event node. Otherwise, we continue the procedures below.

{\bf\emph{Extend}} \emph{and} {\bf \emph{Insert}}: if event $\mathcal{E}$ does not overlap with any existing event, we will find the parent event node in $\mathcal{S}$ to which it should be appended.
We calculate the \emph{connection strength} between the new event $\mathcal{E}$ and each existing event $\mathcal{E}_j \in \mathcal{S}$ based on three factors: 1) the time distance between $\mathcal{E}$ and $\mathcal{E}_j$, 2) the compatibility of the two events, and 3) the \emph{storyline coherence} if $\mathcal{E}$ is appended to $\mathcal{E}_j$ in the tree, i.e., 
\begin{align}
\label{eqn:linkScore} 
\begin{split}
  \text{ConnectionStrength}(\mathcal{E}_j, \mathcal{E})  :=\ \text{compatibility}(\mathcal{E}_j, \mathcal{E}) \times \\
  \text{coherence}(\mathcal{L}_{\mathcal{S}\to\mathcal{E}_j\to \mathcal{E}}) \times \text{timePenalty}(\mathcal{E}_j, \mathcal{E}).
\end{split}
\end{align}

Now we explain the three components in the above equation one by one. \emph{First}, the compatibility between two events $\mathcal{E}_i$ and $\mathcal{E}_j$ is given by
\begin{equation}
  \text{compatibility}(\mathcal{E}_i, \mathcal{E}_j) = \frac{\text{TF}(d_{c_i}) \cdot \text{TF}(d_{c_{j}})}{\|\text{TF}(d_{c_i})\| \cdot \|\text{TF}(d_{c_{j}})\|},
\end{equation}
where $d_{c_i}$ is the centroid document of event $\mathcal{E}_i$.
% Notice that here we use the term frequency (TF) vector of each document rather than TF-IDF, since this choice leads to better performance in practice.

Furthermore, the storyline of $\mathcal{E}_j$ is defined as the path in $\mathcal{S}$ starting from the root node of $\mathcal{S}$ ending at $\mathcal{E}_j$ itself, denoted by $\mathcal{L}_{\mathcal{S}\rightarrow \mathcal{E}_j}$. Similarly, the storyline of $\mathcal{E}$ appended to $\mathcal{E}_j$ is denoted by $\mathcal{L}_{\mathcal{S}\rightarrow \mathcal{E}_j\rightarrow\mathcal{E}}$.
% Previous works on online event threading \cite{wang2016socially} usually just measure the similarities between event pairs to determine whether a event belongs to an existing story line, and then attach into the line if some criteria is satisfied. However, such kind of approaches doesn't consider the \textit{coherence} \cite{xu2013summarizing} of the whole story line.
For a storyline $\mathcal{L}$ represented by a path
$\mathcal{E}^0\to \ldots \to \mathcal{E}^{|\mathcal{L}|}$, where $\mathcal{E}^0 := \mathcal S$, its \textit{coherence} \cite{xu2013summarizing} measures the theme consistency along the storyline, and is defined as
\begin{equation}
  \text{coherence}(\mathcal{L}) = \frac{1}{|\mathcal{L}|}\sum_{i=0}^{|\mathcal{L}|-1} \text{compatibility}(\mathcal{E}^i, \mathcal{E}^{i+1}),
\end{equation}

Finally, the bigger the time gap between two events, the less possible that the two events are connected. We thus calculate time penalty by
\begin{align}
  \text{timePenalty}(\mathcal{E}_j, \mathcal{E}) = \begin{cases}
  e^{\delta \cdot (t_{\mathcal{E}_j} - t_{\mathcal{E}})}
   &\ \text{if } t_{\mathcal{E}_j} - t_{\mathcal{E}} < 0\\
  0 &\ \text{otherwise}\\
  \end{cases}
\end{align}
where $t_{\mathcal{E}_j}$ and  $t_{\mathcal{E}}$ are the timestamps of event $\mathcal{E}_j$ and $\mathcal{E}$ respectively. The timestamp of an event is the minimum timestamp of all the documents in the event.

We calculate the connection strength between the new event $\mathcal{E}$ and every event node $\mathcal{E}_j \in \mathcal{S}$ using \eqref{eqn:linkScore}, and append event $\mathcal{E}$ to the existing $\mathcal{E}_j$ that leads to the maximum connection strength. 
If the maximum connection strength is lower than a threshold value, we \textit{insert} $\mathcal{E}$ into story tree $\mathcal{S}$ by directly appending it to the root node of $\mathcal{S}$. In other words, \emph{insert} is a special case of \emph{extend}.



%!TEX root = main.tex
\section{Evaluation}
\label{sec:eval}

In this section, we evaluate the performance of our unsupervised Ordered Word Mover's Distance metric and supervised Multi-scale Sentence Matching model with factorized sentences as input. We apply our algorithms to semantic textual similarity estimation tasks and sentence pair paraphrase identification tasks, based on four datasets: STSbenchmark, SICK, MSRP and MSRvid. 

\subsection{Experimental Setup}
\label{subsec:setup}


\begin{table}[tb]
  \caption{Description of evaluation datasets.}
  \label{tab:datasets}
  \begin{tabular}{lllll}
    \toprule
    Dataset & Task & Train & Dev & Test\\
    \midrule
    STSbenchmark & Similarity scoring & $5748$ & $1500$ & $1378$ \\
    SICK & Similarity scoring & $4500$ & $500$ & $4927$ \\
    MSRP & Paraphrase identification & $4076$ & - & $1725$ \\
    MSRvid & Similarity scoring & $750$ & - & $750$ \\
    \bottomrule
  \end{tabular}
  \vspace{-2mm}
\end{table}

We will start with a brief description for each dataset:
\begin{itemize}
\item \textbf{STSbenchmark}\cite{cer2017semeval}: it is a dataset for semantic textual similarity (STS) estimation. The task is to assign a similarity score to each sentence pair on a scale of 0.0 to 5.0, with 5.0 being the most similar.

\item \textbf{SICK}\cite{marelli2014sick}: it is another STS dataset from the SemEval 2014 task 1. It has the same scoring mechanism as STSbenchmark, where 0.0 represents the least amount of relatedness and 5.0 represents the most.

\item \textbf{MSRvid}: the Microsoft Research Video Description Corpus contains 1500 sentences that are concise summaries on the content of a short video. Each pair of sentences is also assigned a semantic similarity score between 0.0 and 5.0. 

\item \textbf{MSRP}\cite{quirk2004monolingual}: the Microsoft Research Paraphrase Corpus is a set of 5800 sentence pairs collected from news articles on the Internet. Each sentence pair is labeled 0 or 1, with 1 indicating that the two sentences are paraphrases of each other.
\end{itemize}

Table \ref{tab:datasets} shows a detailed breakdown of the datasets used in evaluation.
For STSbenchmark dataset we use the provided train/dev/test split.
The SICK dataset does not provide development set out of the box, so we extracted 500 instances from the training set as the development set.
For MSRP and MSRvid, since their sizes are relatively small to begin with, we did not create any development set for them.

One metric we used to evaluate the performance of our proposed models on the task of semantic textual similarity estimation is the Pearson Correlation coefficient, commonly denoted by $r$. Pearson Correlation is defined as:
\begin{equation}
\label{eq:pearson}
 r = cov(X,Y) /( \sigma_X \sigma_Y),
\end{equation}
where $cov(X,Y)$ is the co-variance between distributions X and Y, and $\sigma_X$, $\sigma_Y$ are the standard deviations of X and Y.
The Pearson Correlation coefficient can be thought as a measure of how well two distributions fit on a straight line. Its value has range [-1, 1], where a value of 1 indicates that data points from two distribution lie on the same line with a positive slope.
% Due to this unique property, we believe the Pearson Correlation coefficient is a strong indicator of the performance of our metric. 

Another metric we utilized is the Spearman's Rank Correlation coefficient. Commonly denoted by $r_s$, the Spearman's Rank Correlation coefficient shares a similar mathematical expression with the Pearson Correlation coefficient, but it is applied to ranked variables.
Formally it is defined as \cite{wiki:spearman}:
\begin{equation}
\label{eq:spearman}
 \rho = cov(rg_X, rg_Y) / (\sigma_{rg_X} \sigma_{rg_Y}),
\end{equation}
where $rg_X$, $rg_Y$ denotes the ranked variables derived from $X$ and $Y$. $cov(rg_X,rg_Y)$, $\sigma_{rg_X}$, $\sigma_{rg_Y}$ corresponds to the co-variance and standard deviations of the rank variables. The term ranked simply means that each instance in X is ranked higher or lower against every other instances in X and the same for Y. We then compare the rank values of X and Y with \ref{eq:spearman}. Like the Pearson Correlation coefficient, the Spearman's Rank Correlation coefficient has an output range of [-1, 1], and it measures the monotonic relationship between X and Y. A Spearman's Rank Correlation value of 1 implies that as X increases, Y is guaranteed to increase as well.
The Spearman's Rank Correlation is also less sensitive to noise created by outliers compared to the Pearson Correlation.

For the task of paraphrase identification, the classification accuracy of label $1$ and the F1 score are used as metrics. 

In the supervised learning portion, we conduct the experiments on the aforementioned four datasets. We use training sets to train the models, development set to tune the hyper-parameters and each test set is only used once in the final evaluation. For datasets without any development set, we will use cross-validation in the training process to prevent overfitting, that is, use $10\%$ of the training data for validation and the rest is used in training. For each model, we carry out training for 10 epochs. We then choose the model with the best validation performance to be evaluated on the test set.  


\subsection{Unsupervised Matching with OWMD}
\label{subsec:eval-owmd}

To evaluate the effectiveness of our Ordered Word Mover's Distance metric, we first take an unsupervised approach towards the similarity estimation task on the STSbenchmark, SICK and MSRvid datasets. Using the distance metrics listed in Table \ref{tab:compare-pearson} and \ref{tab:compare-spearman}, we first computed the distance between two sentences, then calculated the Pearson Correlation coefficients and the Spearman's Rank Correlation coefficients between all pair's distances and their labeled scores. We did not use the MSRP dataset since it is a binary classification problem.


In our proposed Ordered Word Mover's Distance metric, distance between two sentences is calculated using the order preserving Word Mover's Distance algorithm. For all three datasets, we performed hyper-parameter tuning using the training set and calculated the Pearson Correlation coefficients on the test and development set. We found that for the STSbenchmark dataset, setting $\lambda_1=10$, $\lambda_2=0.03$ produces the most optimal result. For the SICK dataset, a combination of $\lambda_1=3.5$, $\lambda_2=0.015$ works best. And for the MSRvid dataset, the highest Pearson Correlation is attained when $\lambda_1=0.01$, $\lambda_2=0.02$.
We maintain a max iteration of 20 since in our experiments we found that it is sufficient for the correlation result to converge.
During hyper-parameter tuning we discovered that using the Euclidean metric along with $\sigma=10$ produces better results, so all OWMD results summarized in Table \ref{tab:compare-pearson} and \ref{tab:compare-spearman} are acquired under these parameter settings. Finally, it is worth mentioning that our OWMD metric calculates the distances using factorized versions of sentences, while all other metrics use the original sentences. Sentence factorization is a necessary preprocessing step for the OWMD metric.


We compared the performance of Ordered Word Mover's Distance metric with the following methods:

\begin{itemize}
\item \textbf{Bag-of-Words (BoW)}: in the Bag-of-Words metric, distance between two sentences is computed as the cosine similarity between the word counts of the sentences.

\item \textbf{LexVec}~\cite{salle2016enhancing}: calculate the cosine similarity between the  averaged 300-dimensional LexVec word embedding of the two sentences. 

\item \textbf{GloVe}~\cite{pennington2014glove}: calculate the cosine similarity between the averaged 300-dimensional GloVe 6B word embedding of the two sentences. 

\item \textbf{Fastext}~\cite{joulin2016bag}: calculate the cosine similarity between the averaged 300-dimensional Fastext word embedding of the two sentences. 

\item \textbf{Word2vec}~\cite{mikolov2013efficient}: calculate the cosine similarity between the averaged 300-dimensional Word2vec word embedding of the two sentences.

\item \textbf{Word Mover's Distance (WMD)}~\cite{kusner2015word}: estimating the semantic distance between two sentences by WMD introduced in Sec.~\ref{sec:owmd}.
\end{itemize} 


\begin{table}[tb]
  \caption{Pearson Correlation results on different distance metrics.}
  \label{tab:compare-pearson}
  \begin{tabular}{c|cc|cc|c}
    \toprule
    \multirow{2}{*}{Algorithm} & \multicolumn{2}{c}{STSbenchmark} & \multicolumn{2}{c}{SICK} & MSRvid\\ 
     & Test & Dev & Test & Dev & Test\\
    \midrule
    BoW & $0.5705$ & $0.6561$ & $0.6114$ & $0.6087$ & $0.5044$ \\
    LexVec & $0.5759$ & $0.6852$ & $0.6948$ & $\mathbf{0.6811}$ & $0.7318$\\
    GloVe & $0.4064$ & $0.5207$ & $0.6297$ & $0.5892$  & $0.5481$ \\
    Fastext & $0.5079$ & $0.6247$ & $0.6517$ & $0.6421$  & $0.5517$  \\
    Word2vec & $0.5550$ & $0.6911$ & $\mathbf{0.7021}$ & $0.6730$  & $0.7209$  \\
    WMD & $0.4241$ & $0.5679$ & $0.5962$ & $0.5953$  & $0.3430$  \\
    OWMD & $\mathbf{0.6144}$ & $\mathbf{0.7240}$ & $0.6797$ & $0.6772$  & $\mathbf{0.7519}$  \\
    \bottomrule
  \end{tabular}
  \vspace{-4mm}
\end{table}

\begin{table}[tb]
  \caption{Spearman's Rank Correlation results on different distance metrics.}
  \label{tab:compare-spearman}
  \begin{tabular}{c|cc|cc|c}
    \toprule
    \multirow{2}{*}{Algorithm} & \multicolumn{2}{c}{STSbenchmark} & \multicolumn{2}{c}{SICK} & MSRvid\\ 
     & Test & Dev & Test & Dev & Test\\
    \midrule
    BoW & $0.5592$ & $0.6572$ & $0.5727$ & $0.5894$ & $0.5233$ \\
    LexVec & $0.5472$ & $0.7032$ & $0.5872$ & $0.5879$ & $0.7311$\\
    GloVe & $0.4268$ & $0.5862$ & $0.5505$ & $0.5490$  & $0.5828$ \\
    Fastext & $0.4874$ & $0.6424$ & $0.5739$ & $0.5941$  & $0.5634$  \\
    Word2vec & $0.5184$ & $0.7021$ & $0.6082$ & $0.6056$  & $0.7175$  \\
    WMD & $0.4270$ & $0.5781$ & $0.5488$ & $0.5612$  & $0.3699$  \\
    OWMD & $\mathbf{0.5855}$ & $\mathbf{0.7253}$ & $\mathbf{0.6133}$ & $\mathbf{0.6188}$  & $\mathbf{0.7543}$  \\
    \bottomrule
  \end{tabular}
  \vspace{-2mm}
\end{table}


Table \ref{tab:compare-pearson} and Table \ref{tab:compare-spearman} compare the performance of different metrics in terms of the Pearson Correlation coefficients and the Spearman's Rank Correlation coefficients.
We can see that the result of our OWMD metric achieves the best performance on all the datasets in terms of the Spearman's Rank Correlation coefficients.
It also produced the best Pearson Correlation results on the STSbenchmark and the MSRvid dataset, while the performance on SICK datasets are close to the best.
This can be attributed to the two characteristics of OWMD. First, the input sentence is re-organized into a predicate-argument structure using the sentence factorization tree. Therefore, corresponding semantic units in the two sentences will be aligned roughly in order. Second, our OWMD metric takes word positions into consideration and penalizes disordered matches. Therefore, it will produce less mismatches compared with the WMD metric.

% On the SICK dataset, although the result of our metric falls slightly behind Word2vec, LexVec on the test set and Word2vec on the development set, we still believe that it is a superior metric because it produced competitive results across multiple datasets. 

% Table \ref{tab:compare-spearman} presents the Spearman's Rank Correlation coefficients acquired with the same distance metrics. We can observe that our OWMD metric achieves the highest correlation scores on all three datasets. Which proves once again that OWMD is a better distance metric for the task of semantic similarity detection.

\subsection{Supervised Multi-scale Semantic Matching}
\label{subsec:eval-multilayer}

\begin{table*}[tb]
  \caption{A comparison among different supervised learning models in terms of accuracy, F1 score, Pearson's $r$ and Spearman's $\rho$ on various test sets.}
  \label{tab:sts}
  \begin{tabular}{c|cc|cc|cc|cc}
    \toprule
    \multirow{2}{*}{Model} & \multicolumn{2}{c}{MSRP} & \multicolumn{2}{c}{SICK} & \multicolumn{2}{c}{MSRvid} & \multicolumn{2}{c}{STSbenchmark}\\ 
     & Acc.(\%) & F1(\%) & $r$ & $\rho$ & $r$ & $\rho$ & $r$ & $\rho$ \\
    \midrule
    MaLSTM & $66.95$ & $73.95$ & $0.7824$ & $0.71843$ & $0.7325$ & $0.7193$ & $0.5739$ & $0.5558$\\
    Multi-scale MaLSTM & $\mathbf{74.09}$ & $\mathbf{82.18}$ & $\mathbf{0.8168}$ & $\mathbf{0.74226}$ & $\mathbf{0.8236}$ & $\mathbf{0.8188}$ & $\mathbf{0.6839}$ & $\mathbf{0.6575}$\\
    \midrule
    HCTI & $73.80$ & $80.85$ & $0.8408$ & $0.7698$ & $\mathbf{0.8848}$ & $\mathbf{0.8763}$  & $\mathbf{0.7697}$ & $\mathbf{0.7549}$ \\
    Multi-scale HCTI & $\mathbf{74.03}$ & $\mathbf{81.76}$ & $\mathbf{0.8437}$ & $\mathbf{0.7729}$ & $0.8763$ & $0.8686$  & $0.7269$ & $0.7033$  \\
    \bottomrule
  \end{tabular}
  \vspace{-2mm}
\end{table*}

The use of sentence factorization can improve both existing unsupervised metrics and existing supervised models. 
% We extend the normal Siamese model to Fig. \ref{fig:network} to take advantage of different level of information in the factorized sentence. 
To evaluate how the performance of existing Siamese neural networks can be improved by our sentence factorization technique and the multi-scale Siamese architecture, we implemented two types of Siamese sentence matching models, HCTI \cite{mueller2016siamese} and MaLSTM \cite{shao2017hcti}. HCTI is a Convolutional Neural Network (CNN) based Siamese model, which achieves the best Pearson Correlation coefficient on STSbenchmark dataset in SemEval2017 competition (compared with all the other neural network models). MaLSTM is a Siamese adaptation of the Long Short-Term Memory (LSTM) network for learning sentence similarity. As the source code of HCTI is not released in public, we implemented it according to \cite{shao2017hcti} by Keras \cite{chollet2015keras}. With the same parameter settings listed in paper \cite{shao2017hcti} and tried our best to optimize the model, we got a Pearson correlation of 0.7697 (0.7833 in paper \cite{shao2017hcti}) in STSbencmark test dataset.

We extended HCTI and MaLSTM to our proposed Siamese architecture in Fig. \ref{fig:network}, namely the Multi-scale MaLSTM and the Multi-scale HCTI. To evaluate the performance of our models, the experiment is conducted on two tasks: 1) semantic textual similarity estimation based on the STSbenchmark, MSRvid, and SICK2014 datasets; 2) paraphrase identification based on the MSRP dataset.

Table \ref{tab:sts} shows the results of HCTI, MaLSTM and our multi-scale models on different datasets. Compared with the original models, our models with multi-scale semantic units of the input sentences as network inputs significantly improved the performance on most datasets. 
Furthermore, the improvements on different tasks and datasets also proved the general applicability of our proposed architecture.

Compared with MaLSTM, our multi-scaled Siamese models with factorized sentences as input perform much better on each dataset. For MSRvid and STSbenmark dataset, both Pearson's $r$ and Spearman's $\rho$ increase about $10\%$ with Multi-scale MaLSTM. Moreover, the Multi-scale MaLSTM achieves the highest accuracy and F1 score on the MSRP dataset compared with other models listed in Table \ref{tab:sts}.

There are two reasons why our Multi-scale MaLSTM significantly outperforms MaLSTM model. First, for an input sentence pair, 
we explicitly model their semantic units with the factorization algorithm.
%we explicitly model the different scales of semantics of them with the semantic units produced by our sentence factorization algorithm. 
Second, our multi-scaled network architecture is 
specifically designed
%specially adapted to 
for multi-scaled sentences representations. Therefore, it is able to explicitly match a pair of sentences at different granularities.

We also report the results of HCTI and Multi-scale HCTI in Table \ref{tab:sts}. For the paraphrase identification task, our model shows better accuracy and F1 score on MSRP dataset. For the semantic textual similarity estimation task, the performance varies across datasets. On the SICK dataset, the performance of Multi-scale HCTI is close to HCTI with slightly better Pearson' $r$ and Spearman's $\rho$. However, the Multi-scale HCTI is not able to outperform HCTI on MSRvid and STSbenchmark. HCTI is still the best neural network model on the STSbenchmark dataset, and the MSRvid dataset is a subset of STSbenchmark.
Although HCTI has strong performance on these two datasets, it performs worse than our model on other datasets.
% Overall, the experimental results demonstrated the superior applicability and generalizability of our proposed models.
Overall, the experimental results demonstrated the general applicability of our proposed model architecture, which performs well on various semantic matching tasks.

% \begin{table}[tb]
%   \caption{Results of Accuracy and F1 score on MSRP test dataset.}
%   \label{tab:MSRP result}
%   \begin{tabular}{lllll}
%     \toprule
%     Model & Acc.(\%) & F1(\%)  \\
%     \midrule
%     MaLSTM & $66.95$ & $73.95$ \\
%     Factorized MaLSTM & $\mathbf{74.09}$ & $\mathbf{82.18}$ \\
%     HCTI & $73.80$ & $80.85$ \\
%     Factorized HCTI & $\mathbf{74.03}$ & $\mathbf{81.76}$ \\
%     \bottomrule
%   \end{tabular}
%   \vspace{0mm}
% \end{table}


% \begin{table}[tb]
%   \caption{Results of Pearson's $r$ and Spearman's $\rho$ on SICK test dataset.}
%   \label{tab:SICK result}
%   \begin{tabular}{lllll}
%     \toprule
%     Model & r & $\rho$ \\
%     \midrule
%     MaLSTM & $0.7824$ & $0.71843$ \\
%     Factorized MaLSTM & $\mathbf{0.8168}$ & $\mathbf{0.74226}$ \\
%     HCTI & $0.8408$ & $\mathbf{0.7698}$ \\
%     Factorized HCTI & $\mathbf{0.8429}$ & $0.7676$ \\
%     \bottomrule
%   \end{tabular}
%   \vspace{0mm}
% \end{table}


% \begin{table}[tb]
%   \caption{Results of Pearson's $r$ and Spearman's $\rho$ on MSRvid test dataset.}
%   \label{tab:MSRvid result}
%   \begin{tabular}{lll}
%     \toprule
%     Model & r & $\rho$  \\
%     \midrule
%     MaLSTM & $0.7325$ & $0.7193$ \\
%     Factorized MaLSTM & $\mathbf{0.8236}$ & $\mathbf{0.8188}$ \\
%     HCTI & $\mathbf{0.8848}$ & $\mathbf{0.8763}$ \\
%     Factorized HCTI & $0.8763$ & $0.8686$ \\
%     \bottomrule
%   \end{tabular}
%   \vspace{0mm}
% \end{table}



% \begin{table}[tb]
%   \caption{Results of Pearson's $r$ and Spearman's $\rho$ on STSbenchmark test dataset.}
%   \label{tab:STSbenchmark result}
%   \begin{tabular}{lllll}
%     \toprule
%     Model & r & $\rho$ \\
%     \midrule
%     MaLSTM & $0.5739$ & $0.5558$ \\
%     Factorized MaLSTM & $\mathbf{0.6839}$ & $\mathbf{0.6575}$ \\
%     HCTI & $\mathbf{0.7697}$ & $\mathbf{0.7549}$ \\
%     Factorized HCTI & $0.7269$ & $0.7033$ \\
%     \bottomrule
%   \end{tabular}
%   \vspace{0mm}
% \end{table}




\textbf{Related work}:
% Object detection related datasets/algo in non-medical domain
% Locally labeled CXR dataset
A few CXR datasets have localized abnormality annotations \cite{shih2019augmenting,filice2020crowdsourcing,jaeger2014two} that are curated manually. These are high quality gold standard ground truth datasets but tend to be smaller in scale (< 30,000 images) and have a narrow coverage, with typically only 1-2 labels. In addition, since most labeling efforts only have abnormality semantics attached, no direct relationships with the affected anatomical locations are available. 

%MEHDI: repeated concepts from above. I am removing the following: 

%The lack of anatomic semantics in the annotation is a limitation for complex multi-modal clinical reasoning work, e.g., differential diagnosis, since clinicians often integrate information along anatomical lines, and for downstream report generation tasks, which often requires describing not only the abnormality but also correctly communicate the location of the abnormalities (and medical devices) to the receiving clinicians. 

Two recent CXR datasets have labels for anatomies described in the reports. In \cite{datta2020dataset}, a small manually annotated dataset (2000 reports) included 10 abnormalities that are individually associated with 29 unique spatial locations (anatomies) at the report level. Another CXR dataset has automatically extracted abnormality and anatomy labels as disconnected concepts that are only correlated at the study level from  160,000 reports using a supervised NLP algorithm \cite{bustos2020padchest}. This was trained on a smaller set of manually annotated data. Neither datasets contain localized annotations for the associated CXR images, nor any comparison relation annotations between sequential exams, both of which are available in the Chest ImaGenome dataset. In Table \ref{tab:related}, we present a comparison of our Chest ImagGenome dataset with other datasets available in the literature.

% Table -- Kashyap

% MEdical imaging datasets to go here: Discussed that we will only focus on cxr datasets that are available for this paper. 
% \caption{\color{red} Kashyap, feel free to continue with the table. We should remove the questionmarks and add a line for our dataset (since all others are not graph). For longer text, using abbreviations and explaining them in the caption often works better. If fill in the values is not possible, it is better to remove the table altogether.}


\begin{table}[t!]
\caption{Summary of existing chest X-ray datasets}
\resizebox{\textwidth}{!}{%
\begin{tabular}{@{}lllllllll@{}}
\toprule
\textbf{Dataset} & \textbf{Annotation Level} & \textbf{Annotation Method} & \textbf{Num Labels} & \textbf{Anatomy Labeled} & \textbf{Graph} & \textbf{Dataset Size} & \textbf{Temporal Labels} & \textbf{Reports} \\ \midrule
SIIM-ACR Pneumothorax Segmentation \cite{filice2020crowdsourcing} & Segmentation & Manual + augmented & 1 & No & No & 12,047 & No & No \\
RSNA Pneumonia Detection Challenge   \cite{shih2019augmenting} & Bounding Boxes & Manual & 1 & No & No & 30,000 & No & No \\
Indiana University Chest X-ray collection \cite{demner2016preparing} & Global & Automated & 10 & No & No & 3,813 & No & Yes \\
NIH CXR dataset \cite{wang2017chestx} & Global & Automated & 14 & No & No & 112,120 & No & No \\
PLCO \cite{team2000prostate} & Global & Automated & 24 & Yes & No & 236,000 & Yes & No \\
Stanford CheXpert \cite{irvin2019chexpert} & Global & Automated & 14 & No & No & 224,316 & No & No \\
MIMIC-CXR \cite{johnson2019mimic} & Global & Automated & 14 & No & No & 377,110 & No & Yes \\
Dutta \cite{datta2020dataset} & Global & Manual & 10 & Yes & Yes & 2,000 & No & Yes \\
PadChest \cite{bustos2020padchest} & Global & Manual + automated & 297 & Yes & No & 160,868 & No & Yes \\
Montgomery County Chest X-ray   \cite{jaeger2014two} & Segmentation & Manual & 1 & Yes & No & 138 & No & No \\
Shenzen Hospital Chest X-ray   \cite{jaeger2014two} & Segmentation & Manual & 1 & Yes & No & 662 & No & No \\  \hline \hline
\textbf{Chest ImaGenome} & Bounding Boxes & Automated & 131 & Yes & Yes & 242,072 & Yes & Yes \\
\bottomrule
\end{tabular}%
}
\label{tab:related}
\vspace{-0.4cm}
\end{table}
% removed (Derived from MIMIC-CXR \cite{johnson2019mimic}) % makes table really small


\begin{comment}
\begin{figure}
\includegraphics[width=\linewidth]{figs/beyond_tss_lesion.pdf}
\caption[]{End-to-End runtime lesion study of the entire MNIST dataset and the FMA featurized music dataset. Each of DROP's contributions provides a runtime improvement.}
\label{fig:beyond_lesion}
\end{figure}
\end{comment}



\section{Conclusion}
\label{sec:conclusion}

Advanced data analytics techniques must scale to rising data volumes. 
DR techniques offer a powerful toolkit when processing these datasets, with PCA frequently outperforming popular techniques in exchange for high computational cost. 
In response, we propose DROP, a new dimensionality reduction optimizer. 
DROP combines progressive sampling, progress estimation, and online aggregation to identify high quality low dimensional bases via PCA without processing the entire dataset by balancing the runtime of downstream tasks and achieved dimensionality. 
Thus, DROP provides a first step in bridging the gap between quality and efficiency in end-to-end DR for downstream \red{analytics}. 

%We revisit canonical operators for time series dimensionality reduction and the measurement study of~\cite{keogh-study}, and show that PCA is more effective than popular alternatives in the data mining literature often by a margin of over $2\times$ on average on gold-standard time series benchmark data sets with respect to output data dimension. More surprisingly, we empirically demonstrate that a small number of samples are sufficient to accurately characterize directions of maximum variance and obtain a high-quality low-dimensional transformation.




\bibliographystyle{ACM-Reference-Format}
\bibliography{main} 

\end{document}
