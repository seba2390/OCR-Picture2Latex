%!TEX root = main.tex
\section{Related Work}
\label{sec:related}

There are mainly two research lines that are highly related to our work: Text Clustering and Story Structure Generation.
% Here we briefly introduce the most related works.



% \subsection{Text Clustering}
% \label{subsec:text-clustering}
The problem of text clustering has been well studied by researchers \cite{aggarwal2012survey,jing2005subspace,jing2010knowledge,guan2011text}. The most popular way is first extracting specific text features, such as TF-IDF, from documents, and then apply general clustering algorithms such as k-means. The selection of different feature and setting of algorithm parameters plays a key role in the final performance of clustering \cite{liu2005comparative}. There are also approaches which utilize the document keywords co-occurrence information to construct a keyword graph, and clustering documents by applying community detection techniques on the keyword graph \cite{sayyadi2013graph}. \cite{Mele2017Event} combines topic modeling, named-entity recognition, and temporal analysis to detect event clusters from news streams. \cite{Chakrabarti2010Evolutionary} proposed an evolutionary clustering framework to cluster data over time. A more comprehensive study of different text clustering algorithms can be found in \cite{aggarwal2012survey}.



% \subsection{Story Structure Generation}
% \label{subsec:story-struc}

The Topic Detection and Tracking (TDT) research spot news events and group by topics, and track previously spotted news events by attaching related new events into the same cluster \cite{allan1998line,allan2012topic,yang2009discovering,sayyadi2013graph}. However, the associations between related events are not defined or interpreted by TDT techniques. 
To help users capture the developing structure of events, different approaches have been proposed. \cite{nallapati2004event} proposed the concept of \textit{Event Threading}, and tried a series of strategies based on similarity measure to capture the dependencies among events. \cite{yang2009discovering} combines the similarity measure between events, temporal sequence and distance between events, and document distribution along the timeline to score the relationship between events, and models the event evolution structure by a directed acyclic graph (DAG). 
% \cite{mei2005discovering} discover and summarize the evolutionary patterns of themes in a text stream by first generating word clusters for each time period and then use the Kullback-Leibler divergence measure to discover coherent themes over time.

The above research works measure and model the relationship between events in a pairwise manner. However, the overall story consistency is not considered.
% \cite{wang2012generating} generates story summarization from text and image data by constructing a multi-view graph and solving a dominating set problem, but it omits the consistency of each storyline.
The \textit{Metro Map} model proposed in \cite{shahaf2013information} defines metrics such as coherence and diversity for story quality evaluation, and identifies lines of documents by solving an optimization problem to maximize the topic diversity of storylines while guarantee the coherence of each storyline. 
% \cite{xu2013summarizing} further summarize documents with key images and sentences, and then extract story lines with different definitions of coherence and diversity.
% These works consider the problem of discovering story development structure as optimizing problems with given news corpora.
However, new documents are being generated all the time, and systems that are able to catch related news and update story structures in an online manner are desired.

As studies based on unsupervised clustering techniques \cite{yan2011evolutionary} perform poorly in distinguishing storylines with overlapped events \cite{hua2016automatical},
more recent works introduce different Bayesian models to generate storyline. However, they often ignore the intrinsic structure of a story \cite{huang2013optimized} or fail to properly model the hidden relations \cite{zhou2015unsupervised}. \cite{hua2016automatical} proposes a hierarchical Bayesian model for storyline generation, and utilize twitter hashtags to ``supervise'' the generation process. However, the Gibbs sampling inference of the model is time consuming, and such twitter data is not always available for every news stories.
