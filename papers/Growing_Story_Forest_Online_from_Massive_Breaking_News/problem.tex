%!TEX root = main.tex
\section{Problem Definition and Notations}
\label{sec:problem}







% In this section, we will first describe key concepts and notations used in this paper, and formally define our problem. Then we will use a case study to make our idea of story tree more concrete.

% \subsection{Problem Definition and Notations}
% \label{subsec:problem-define}

We first present some definitions of key concepts in the top-down hierarchy: \textit{topic} $\rightarrow$ \textit{story} $\rightarrow$ \textit{event} to be used in this paper.

\begin{definition}
  \textit{Event}: an event $\mathcal{E}$ is a set of one or several documents that contain highly similar information.
\end{definition}

\begin{definition}
  \textit{Story}: a story $\mathcal{S}$ is a tree of events that revolve around a group of specific persons and happen at certain places during specific times. A directed edge from event $\mathcal{E}_1$ to $\mathcal{E}_2$ indicates a temporal evolution or a logical connection from $\mathcal{E}_1$ to $\mathcal{E}_2$.
\end{definition}

\begin{definition}
  \textit{Topic}: a topic consists of a set of stories that are highly correlated or similar to each other.
  \vspace{-1mm}
\end{definition}


Each topic may contain multiple story trees, and each story tree consists of multiple logically connected events.
In our work, events (instead of news documents) are the smallest atomic units. Each event is also assumed to belong to a single story and contains partial information about that story.
For instance, considering the topic \textit{American presidential election}, \textit{2016 U.S. presidential election} is a story within this topic, and  \textit{Trump and Hilary's first television debate} is an event within this story.


We now introduce some notations and describe our problem formally. Given a news document stream $D = \{ \mathcal{D}_1, \mathcal{D}_2, \ldots, \mathcal{D}_t,\ldots \}$, where $\mathcal{D}_t$ is the set of news documents collected on time period $t$, our objective is to: a) cluster all news documents $D$ into a set of events $E = \{ \mathcal{E}_1, \ldots, \mathcal{E}_{|E|} \}$, and b) connect the extracted events to form a set of stories $S = \{ \mathcal{S}_1, ..., \mathcal{S}_{|S|} \}$. Each story $\mathcal{S} = (E, L)$ contains a set of events $E$ and a set of links $L$, where $L_{i,j} := <\mathcal{E}_i, \mathcal{E}_j>$ denotes a directed link from event $\mathcal{E}_i$ to $\mathcal{E}_j$, which indicates a temporal evolution or logical connection relationship.

%We now illustrate our problem with an example. (A example Fig) Fig... shows ...
Furthermore, we require the events and story trees to be extracted in an online or incremental manner. That is, we extract events from each $\mathcal D_t$ individually when the news corpus $\mathcal D_t$ arrives in time period $t$, and \emph{merge} the discovered events into the existing story trees that were found at time $t-1$. This is a unique strength of our scheme as compared to prior work, since we do not need to repeatedly process older documents and can deliver  a set of evolving yet logically consistent story trees to users.  

% \subsection{Case Study}
% \label{subsec:case-study}

\begin{figure}
\includegraphics[width=3.4in]{figure/StoryStructures}
\caption{Different structures to characterize a story.}
\vspace{-2mm}
\label{fig:storyStructures}
\vspace{-2mm}
\end{figure}

For example, Fig.~\ref{fig:CaseStudy} illustrates the story tree of ``2016 U.S. presidential election''. The story contains $20$ nodes, where each node indicates an event in 2016 U.S. election, and each link indicates a temporal evolution or a logical connection between two events. %For example, event $19$ says America votes to elect new president, and event $20$ says Donald Trump is elected president. 
The index number on each node represents the event sequence over the timeline. There are $6$ paths within this story tree, where the path $1 \rightarrow 20$ indicates the whole presidential election process, branch $3 \rightarrow 6$ is about Hilary's health conditions, branch $7 \rightarrow 13$ talks about television debates, $14 \rightarrow 18$ depicts the investigation into Hilary's ``mail door'', etc. As we can see, by modeling the evolutionary and logical structure of a story into a story tree, users can easily grasp the logic of news stories and learn the main information quickly. 


Let us represent each story by an empty root node $s$ from which the story is originated, and denote each event by an event node $e$. The events in a story can be organized in one of the following four structures shown in Fig. \ref{fig:storyStructures}: a) a flat structure that does not include dependencies between events; b) a timeline structure that organizes events by their timestamps; c) a graph structure that checks the connection between all pairs of events and maintains a subset of most strong connections; d) a tree structure, which represents a story's evolving structure by a tree.  

Compared with a tree structure, sorting events by timestamps omits the logical connection between events, while using directed acyclic graphs to model event dependencies without considering the evolving consistency of the whole story can leads to unnecessary connections between events.
Through extensive user experience studies in Sec.~\ref{sec:eval}, we show that tree structures are the most effective way to represent breaking news stories as compared to other structures, including the more complex graph structures. 
