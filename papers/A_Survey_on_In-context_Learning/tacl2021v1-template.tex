% File tacl2021v1.tex
% Dec. 15, 2021

% The English content of this file was modified from various *ACL instructions
% by Lillian Lee and Kristina Toutanova
%
% LaTeXery is mostly all adapted from acl2018.sty.

\documentclass[11pt,a4paper]{article}
\usepackage{times,latexsym}
\usepackage{url}
\usepackage[T1]{fontenc}

%% Package options:
%% Short version: "hyperref" and "submission" are the defaults.
%% More verbose version:
%% Most compact command to produce a submission version with hyperref enabled
%%    \usepackage[]{tacl2021v1}
%% Most compact command to produce a "camera-ready" version
   \usepackage[acceptedWithA]{tacl2021v1}
%% Most compact command to produce a double-spaced copy-editor's version
%%    \usepackage[acceptedWithA,copyedit]{tacl2021v1}
%
%% If you need to disable hyperref in any of the above settings (see Section
%% "LaTeX files") in the TACL instructions), add ",nohyperref" in the square
%% brackets. (The comma is a delimiter in case there are multiple options specified.)

\usepackage{tacl2021v1}
% \setlength\titlebox{10cm} % <- for Option 2 below

%%%% Material in this block is specific to generating TACL instructions
\usepackage{xspace,mfirstuc,tabulary}
\usepackage{times}
\usepackage{latexsym}

% For proper rendering and hyphenation of words containing Latin characters (including in bib files)
\usepackage[T1]{fontenc}
% For Vietnamese characters
% \usepackage[T5]{fontenc}
% See https://www.latex-project.org/help/documentation/encguide.pdf for other character sets

% This assumes your files are encoded as UTF8
\usepackage[utf8]{inputenc}

% This is not strictly necessary, and may be commented out.
% However, it will improve the layout of the manuscript,
% and will typically save some space.
\usepackage{microtype}

% This is also not strictly necessary, and may be commented out.
% However, it will improve the aesthetics of text in
% the typewriter font.
\usepackage{inconsolata}

\usepackage{amsmath}
\usepackage{amsfonts}
\usepackage{amssymb}

% for professional tables
\usepackage{booktabs}
% for multiple rows
\usepackage{multirow}

% for in paragraph list
\usepackage{paralist}

% for compact itemize 
\usepackage{mdwlist}
\usepackage{graphicx}

% to use colors
\usepackage{color}

% for algorithms
\usepackage[ruled,linesnumbered]{algorithm2e}

% for defining optional space after words (but no space if it is punctuation)
\usepackage{xspace}

% ddm introduce
\usepackage{tabularx}
\usepackage{makecell}
\usepackage{amssymb}

\newcommand{\xx}{\mathbf{x}}
\newcommand{\uu}{\mathbf{u}}
\newcommand{\zz}{\mathbf{z}}
\newcommand{\yy}{\mathbf{y}}
\newcommand{\hh}{\mathbf{h}}
\newcommand{\zy}[1]{\textcolor{green}{\bf \small [#1 --zy]}}
\DeclareMathOperator*{\argmax}{arg\,max}
\DeclareMathOperator*{\argmin}{arg\,min}

\usepackage{color}
\usepackage{tikz}
\usepackage[edges]{forest}
\definecolor{hidden-draw}{RGB}{20,68,106}
\definecolor{hidden-pink}{RGB}{255,245,247}

\newcommand{\leiModify}[1]{\textcolor{orange}{#1 --Lei}}
\newcommand{\qingxiu}[1]{\textcolor{blue}{#1}}
\newcommand{\zc}[1]{\textcolor{purple}{#1 --ce}}
\newcommand{\dateOfLastUpdate}{Dec. 15, 2021}
\newcommand{\styleFileVersion}{tacl2021v1}

\newcommand{\ex}[1]{{\sf #1}}

\newif\iftaclinstructions
\taclinstructionsfalse % AUTHORS: do NOT set this to true
\iftaclinstructions
\renewcommand{\confidential}{}
\renewcommand{\anonsubtext}{(No author info supplied here, for consistency with
TACL-submission anonymization requirements)}
\newcommand{\instr}
\fi

%
\iftaclpubformat % this "if" is set by the choice of options
\newcommand{\taclpaper}{final version\xspace}
\newcommand{\taclpapers}{final versions\xspace}
\newcommand{\Taclpaper}{Final version\xspace}
\newcommand{\Taclpapers}{Final versions\xspace}
\newcommand{\TaclPapers}{Final Versions\xspace}
\else
\newcommand{\taclpaper}{submission\xspace}
\newcommand{\taclpapers}{{\taclpaper}s\xspace}
\newcommand{\Taclpaper}{Submission\xspace}
\newcommand{\Taclpapers}{{\Taclpaper}s\xspace}
\newcommand{\TaclPapers}{Submissions\xspace}
\fi

%%%% End TACL-instructions-specific macro block
%%%%

\title{A Survey on In-context Learning}

% Author information does not appear in the pdf unless the "acceptedWithA" option is given

% The author block may be formatted in one of two ways:

% Option 1. Author’s address is underneath each name, centered.
\author{
Qingxiu Dong\textsuperscript{\rm1}, 
Lei Li\textsuperscript{\rm1},
Damai Dai\textsuperscript{\rm1}, 
Ce Zheng\textsuperscript{\rm1},
Zhiyong Wu\textsuperscript{\rm2},\\
\textbf{
Baobao Chang\textsuperscript{\rm1},
Xu Sun\textsuperscript{\rm1},
Jingjing Xu\textsuperscript{\rm2}, 
Lei Li\textsuperscript{\rm3}
and Zhifang Sui\textsuperscript{\rm1}}  \\
\textsuperscript{\rm 1} MOE Key Lab of Computational Linguistics, School of Computer Science, Peking University \\
  \textsuperscript{\rm 2} Shanghai AI Lab \ \ \textsuperscript{\rm 3} University of California, Santa Barbara \\
  \texttt{ \{dqx,lilei\}@stu.pku.edu.cn,  
  wuzhiyong@pjlab.org.cn, lilei@cs.ucsb.edu }
  \\ \texttt{\{daidamai,zce1112zslx,chbb,xusun,jingjingxu,szf\}@pku.edu.cn}
}
% \author{
%   Template Author1\Thanks{The {\em actual} contributors to this instruction
%     document and corresponding template file are given in Section
%     \ref{sec:contributors}.} 
%   \\
%   Template Affiliation1/Address Line 1
%   \\
%   Template Affiliation1/Address Line 2
%   \\
%   Template Affiliation1/Address Line 2
%   \\
%   \texttt{template.email1example.com}
%   \And
%   Template Author2 
%   \\
%   Template Affiliation2/Address Line 1
%   \\
%   Template Affiliation2/Address Line 2
%   \\
%   Template Affiliation2/Address Line 2
%   \\
%   \texttt{template.email2@example.com}
% }

% % Option 2.  Author’s address is linked with superscript
% % characters to its name, author names are grouped, centered.

% \author{
%   Template Author1\Thanks{The {\em actual} contributors to this instruction
%     document and corresponding template file are given in Section
%     \ref{sec:contributors}.}$^\diamond$ 
%   \and
%   Template Author2$^\dagger$
%   \\
%   \ \\
%   $^\diamond$Template Affiliation1/Address Line 1
%   \\
%   Template Affiliation1/Address Line 2
%   \\
%   Template Affiliation1/Address Line 2
%   \\
%   \texttt{template.email1example.com}
%   \\
%   \ \\
%   \\
%   $^\dagger$Template Affiliation2/Address Line 1
%   \\
%   Template Affiliation2/Address Line 2
%   \\
%   Template Affiliation2/Address Line 2
%   \\
%   \texttt{template.email2@example.com}
% }

\date{}

\begin{document}
\maketitle

\begin{abstract}
With the increasing ability of large language models (LLMs), in-context learning (ICL) has become a new paradigm for natural language processing (NLP), where LLMs make predictions only based on contexts augmented with a few examples. It has been a new trend to explore ICL to evaluate and extrapolate the ability of LLMs.
In this paper, we aim to survey and summarize the progress and challenges of ICL. We first present a formal definition of ICL and clarify its correlation to related studies. Then, we organize and discuss advanced techniques, including training strategies, demonstration designing strategies, as well as related analysis.
Finally, we discuss the challenges of ICL and provide potential directions for further research. We hope that our work can encourage more research on uncovering how ICL works and improving ICL.
% \footnote{We sort out all the papers involved on \url{https://github.com/dqxiu/ICL_PaperList}.}

% and attempt to provide potential directions for further research
\end{abstract}

\section{Introduction}
\label{sec:intro}
% Description/Definition/Formulation
%   1. interface human understanding
%   2. No gradient update v.s. Finetuning
%   3. 从context learning
%   4. strengths, reasoning zero-shot human readable
%   5. weakness (a)unstable(order,distribution,token) (b)need large model (3)inference efficiency with long context , (4) limited training data(max\_seq\_len)

% 第一段 写一下 背景
% Prompt 那个是从 superivised -> pre-training finetuning -> prompt engineering 切入的
% ICL ability emerges from pretraining of large LM. 
% First proposed in GPT-3
With the scaling of model size and corpus size~\citep{bert,gpt2,gpt3,chowdhery2022palm}, large language models (LLMs) demonstrate an in-context learning (ICL) ability, that is, learning from a few examples in the context. 
% (in-context learning for short). 
Many studies have shown that LLMs can perform a series of complex tasks through ICL, such as solving mathematical reasoning problems~\citep{cot}. These strong abilities have been widely verified as emerging abilities for large language models~\citep{wei2022emergent}. 

%Figure~\ref{fig:icl} gives an example describing how language models make decisions under ICL. 

 %These strong abilities have been widely verified as the emerging abilities for large language models~\citep{wei2022emergent}. 
 
 %It shows that the model learns from the demonstration consisting of a few examples in the context
% For example, by feeding the model with translation text pair from the source language to the target and a sentence that needs to be translated, the model could accurately produce a translation for the given query.



% sec1. lei增加general 定义,当前主要问题和研究现状 contribution insight 
% sec2. lei补充formulation inference的各种变种的formulatioin 加table

% The development of pre-trained language models, such as BERT~\citep{bert} and GPT-2~\citep{gpt2} leads to a widely-adopted paradigm where the model is first trained on the general corpus, then finetuned on task-specific datasets for improving the in-domain performance, such as the classification accuracy.
% The idea behind this is to learn rich contextual representations of languages from free text.
% Further explorations in this direction scale up the number of causal language model parameters and tokens of training corpus, resulting in surprising few-shot learning abilities, i.e., the model can learn from the demonstration with a few examples in the context to perform a series of complex tasks, such as solving mathematical reasoning problems. 
% This ability is called in-context learning~(ICL) by \citet{gpt3} and shows great potential in a wide range of tasks, attracting great attention from the community~\citep{metaicl}.

% Informal definition, acutally, illustrate the process of ICL 
The key idea of in-context learning is to learn from analogy. Figure~\ref{fig:icl} gives an example describing how language models make decisions with ICL.  %and we illustrate the whole process in Figure~\ref{fig:icl}. 
First, ICL requires a few examples to form a demonstration context. These examples are usually written in natural language templates. 
%a few examples are first picked and represented in natural language tokens according to specific templates and order to form a demonstration context. 
%Second, the question of interest is also converted in a similar way after the demonstration text 
Then, ICL concatenates a query question and a piece of demonstration context together to form a prompt,  which is then fed into the language model for prediction.
Different from supervised learning requiring a training stage that uses backward gradients to update model parameters, ICL does not conduct parameter updates and directly performs predictions on the pretrained language models. The model is expected to learn the pattern hidden in the demonstration and accordingly make the right prediction. % The inference is performed in the form of text completion by reusing the language model head learned during the large-scale pretraining.
%If required, the predictions are transformed into an answer according to a pre-defined rule. \zy{the last sentence is kinda too detailed and unnecessary?}%, where the model performance is evaluated on. 
% We summarize the key characteristics of ICL and comparison with traditional finetuning in Figure~\ref{fig:my_label}.
% http://ai.stanford.edu/blog/in-context-learning : What is in-context learning? Informally, in-context learning describes a different paradigm of “learning” where the model is fed input normally as if it were a black box, and the input to the model describes a new task with some possible examples while the resulting output of the model reflects that new task as if the model had “learned”. While imprecise, the term is meant to capture common behavior that was noted in the GPT-3 paper by OpenAI as a phenomenon that GPT-3 displayed with surprising consistency.


% \section{Definition of In-Context Learning~(ICL)}
% Definition of ICL 


\begin{figure}[t]
    \centering
    \includegraphics[width=0.45\textwidth]{fig/icl.pdf}
    \caption{Illustration of in-context learning. ICL requires a piece of demonstration context containing a few examples written in  natural language templates. Taking the demonstration and a query as the input, large language models are responsible for making predictions.}
    \label{fig:icl}
\end{figure}

\tikzstyle{my-box}=[
    rectangle,
    draw=hidden-draw,
    rounded corners,
    text opacity=1,
    minimum height=1.5em,
    minimum width=5em,
    inner sep=2pt,
    align=center,
    fill opacity=.5,
    line width=0.8pt,
]
\tikzstyle{leaf}=[my-box, minimum height=1.5em,
    fill=hidden-pink!80, text=black, align=left,font=\normalsize,
    inner xsep=2pt,
    inner ysep=4pt,
    line width=0.8pt,
]
\begin{figure*}[t!]
    \centering
    \resizebox{\textwidth}{!}{
        \begin{forest}
            forked edges,
            for tree={
                grow=east,
                reversed=true,
                anchor=base west,
                parent anchor=east,
                child anchor=west,
                base=left,
                font=\large,
                rectangle,
                draw=hidden-draw,
                rounded corners,
                align=left,
                minimum width=4em,
                edge+={darkgray, line width=1pt},
                s sep=3pt,
                inner xsep=2pt,
                inner ysep=3pt,
                line width=0.8pt,
                ver/.style={rotate=90, child anchor=north, parent anchor=south, anchor=center},
            },
            where level=1{text width=4em,font=\normalsize,}{},
            where level=2{text width=6em,font=\normalsize,}{},
            where level=3{text width=8em,font=\normalsize,}{},
            where level=4{text width=5em,font=\normalsize,}{},
            [
                In-context Learning, ver
                [
                    Training 
                    % [
                    %     Pretrain 
                    %     [
                    %         LLMs 
                    %         [
                    %             GPT-3~\cite{gpt3}{, }GLaM~\cite{glam}{, } OPT~\cite{opt}{, } \\ AlexaTM 20B~\cite{alexatm}
                    %             , leaf, text width=33em
                    %         ]
                    %     ]
                    %     [
                    %         Pretrain Corpus
                    %         [
                    %             Pretraining Corpus Combination~cite{shin2022corpora}
                    %             , leaf, text width=25em
                    %         ]
                    %     ]
                    % ]
                    [
                        Warmup  (\S \ref{sec:warmup})
                        [
                            Supervised \\ In-context \\ Training (\S \ref{sec:s_tuning})
                            [
                                MetaICL~\cite{metaicl}{, }OPT-IML~\cite{optiml}{, }FLAN~\cite{flan}{, }\\Super-NaturalInstructions~\cite{natural}{, }Scaling Instruction~\cite{chung}{, }\\Symbol Tuning~\cite{symboltuning}
                                , leaf, text width=36em
                            ]
                        ]
                        [
                            Self-supervised \\ In-context  \\ Training  (\S \ref{sec:ss_tuning})
                            [
                                Self-supervised ICL~\cite{selfsupericl}{, }PICL~\cite{picl}
                                , leaf, text width=28em
                            ]
                        ]
                    ]
                ]
                [
                    Inference
                    [
                    Demonstration \\ Designing (\S \ref{sec:demo})
                        [
                            Organization (\S \ref{sec:organ})
                            [
                                Selecting \\ (\S \ref{sec:select})
                                [   
                                    KATE~\cite{liu2022close}{, }EPR~\cite{rubin2022learning}{, }PPL~\cite{gonen2022demystifying}{, }\\SG-ICL~\cite{kim2022self}{, }Self Adaptive~\cite{Wu2022SelfadaptiveIL}{,}MI~\cite{sorensen2022information}{, }\\Q-Learning~\cite{zhang2022active}{,} Informative Score~\cite{li2023supporting}{,}\\Topic~\cite{topic}{, }UDR~\cite{udr}
                                    , leaf, text width=36.2em
                                ]
                            ]
                            [
                                Ordering \\ (\S \ref{sec:order})
                                [
                                GlobalE\&LocalE~\cite{lu2022order}
                                , leaf, text width=15em
                                ]
                            ]
                        ]
                        [
                            Formatting (\S \ref{sec:format})
                            [
                                Instruction \\ (\S \ref{sec:formatting_instruction})
                                [
                                    Instruction Induction~\cite{induct}{, }APE~\cite{zhou2022large}{, }\\Self-Instruct~\cite{wang2022self}
                                    , leaf, text width=30em
                                ]
                            ]
                            [
                                Reasoning \\ Steps \\ (\S \ref{sec:formatting_intermediate})
                                [
                                    CoT~\cite{wang2022self}{, }Complex CoT~\cite{fu2022complexitycot}{, }\\AutoCoT~\cite{autocot}{, }Self-Ask~\cite{selfask}{, } \\
                                    MoT\citep{mot}{, } SuperICL\citep{xu2023small}\\ iCAP~\cite{wang2022iteratively}{, }Least-to-Most Prompting~\cite{least}
                                    , leaf, text width=30.5em
                                ]
                            ]
                        ]
                    ]
                    [
                        Scoring \\ Function (\S \ref{sec:scoring})
                        [
                            Channel prompt tuning~\cite{min2022noisy}{, }
                            Structrured Prompting~\cite{hao2022structured}{, }\\
                            $k$NN-Prompting~\cite{knnPrompting}
                            , leaf, text width=36em
                        ]
                    ]
                ]
            ]
        \end{forest}
    }
    \caption{Taxonomy of in-context learning. The training and the inference stage are two main stages for ICL. During the training stage, existing ICL studies mainly take a pretrained LLM as backbone, and optionally warmup the model to strengthen and generalize the ICL ability. Towards the inference stage, the demonstration designing and the scoring function selecting are crucial for the ultimate performance.  }
    \label{taxo_of_icl}
\end{figure*}
% Pros & Cons  of ICL 
% Understandable
% \leiModify{Emphasize importance, e.g., difference with finetuning}
% The characteristics of ICL have obvious advantages. 
As a new paradigm, ICL has multiple attractive advantages. %ICL opens new directions for natural language processing research.
First, since the demonstration is written in natural language, it provides an interpretable interface to communicate with LLMs~\citep{gpt3}.
This paradigm makes it much easier to incorporate human knowledge into LLMs by changing the demonstration and templates~\citep{liu2022close,lu2022order,Wu2022SelfadaptiveIL,cot}. Second, in-context learning is similar to the decision process of human beings by learning from analogy~\citep{winston1980learningByAnalogy}. %, i.e., a human makes a prediction based on a few instructions and utilizes the experience from the past to generalize to new tasks.\zy{learning by analogy (ju yi fan san)} 
Third, compared with supervised training, ICL is a training-free learning framework. %ICL is data-efficient This high data efficiency of ICL makes this paradigm more ideal than previous data-hungry tuning methods for low-resource applications and scenarios~\citep{calibrate}. 
% Computation Efficiency 
%Finally, unlike the traditional supervised learning paradigm, which requires parameter updates on the model, there is no gradient computation in ICL. 
This could not only greatly reduce the computation costs for adapting the model to new tasks, but also make language-model-as-a-service~\citep{sun2022black} possible and can be easily applied to large-scale real-world tasks. % and can be easily applied to large-scale real-world tasks
% LMasA


% Directions and limitations?
% Advantage ->  Importance  
% Previous paradigm cannot solve. 
Despite being promising, there are also interesting questions and intriguing properties that require further investigation in ICL.
While the vanilla GPT-3 model itself shows promising ICL abilities, several studies  observed that the ability could be significantly boosted via adaption during pretraining~\citep{metaicl,chen2022sensitivity}.
In addition, the performance of ICL is sensitive to specific settings, including the prompting template, the selection of in-context examples, and order of examples, and so on~\citep{calibrate}. Furthermore, while intuitively reasonable, the working mechanism of the ICL remains unclear, and few studies have provided preliminary explanations~\citep{dai2022iclft,icl_gd}.
% Why we need this in-context learning survey 

% Our paper aims to sensitize the NLP community
% towards this growing area of work

With the rapid growth of studies in ICL, our survey aims to sensitize the community toward the current progress.
Specifically, we present a detailed paper survey with a paper list that will be continuously updated, and make an in-depth discussion on related studies of ICL. We highlight the challenges and potential directions and hope our work may provide a useful roadmap for beginners interested in this area and shed light on future research.

% \leiModify{this part seems too short? maybe elaborate more on the importance of this survey paper for the rapidly developing field, and the resource we could provide for the community.}
% % Directions current 
% \begin{figure}[t]
%     \centering
%     \includegraphics[width=0.9\linewidth]{fig/icl_process.pdf}
%     \caption{Main procedures of in-context learning. Pretraining is significant for developing the ICL ability of LLMs and the optional warmup stage can further improve it. For the demonstrations, the most important procedure is demonstration designing. With a pretrained LLM and a well-designed demonstration, a proper scoring strategy finally yield the task output.}
%     \label{fig:icl_process}
% \end{figure}

\section{Overview}
% \qingxiu{Modifying...}
%As illustrated in Figure~\ref{taxo_of_icl}, 
The strong performance of ICL relies on two stages: (1) the training stage that cultivates the ICL ability of LLMs, and (2) the inference stage where LLMs predict according to task-specific demonstrations.
%In terms of the training stage, pretraining is significant for bringing up the ICL ability and the optional warmup can further improve the ICL ability.
%In terms of the training stage, the language models are pre-trained with 
In terms of the training stage, LLMs are directly trained on language modeling objectives, such as left-to-right generation. Although the models are not specifically optimized for in-context learning, they still exhibit the ICL ability.
% \leiModify{the logic here seems not very clear?} 
Existing studies on ICL basically take a well-trained LLM as the backbone, and thus this survey will not cover the details of pretraining language models. 
Towards the inference stage, as the input and output labels are all represented in interpretable natural language templates, there are multiple directions for improving ICL performance. This paper will give a detailed description and comparison, such as selecting suitable examples for demonstrations and designing specific scoring methods for different tasks.
%Given a pretrained LLM and a well-designed demonstration, various scoring methods can be adopted for particular tasks.
% As illustrated in Fig.~\ref{fig:icl_process}, the LLM and the demonstration are two crucial components of ICL.
% In terms of the LLM, the pretraining stage is significant for bringing up the ICL ability and the optional warmup stage can further improve the ICL ability.
% Towards the demonstrations, as the input and output labels are all represented in interpretable language tokens, humans could design better promptings strategies such as selecting suitable examples for demonstrations and decomposing the reasoning procedure in text for hinting the model.
% Given a pretrained LLM and a well-designed demonstration, various scoring methods can be adopted for particular tasks.
% ICL consists of four main stages: (1) LLMs pretraining that ; (2) optional model warm-up that; (3) demonstration designing that (4) scoring. The main procedures are 
% detection to identify claims that require verification; (ii) evidence retrieval to find sources
% supporting or refuting the claim; and (iii) claim
% verification to assess the veracity of the claim
% based on the retrieved evidence. 
%With the formulation, 
% We notice that there are important components in the framework. 
% First, the core of in-context learning is the model itself, whose model scale and pretraining corpus have a great influence on the final prediction performance.
% % We discuss various model training techniques for improving the ICL ability of the model proposed in \S~\ref{sec:warmup}.
% Besides, as the input and output labels are all represented in interpretable language tokens, humans could design better promptings strategies such as selecting suitable examples for demonstrations and decomposing the reasoning procedure in text for hinting the model.
% % which we provide an overview of current progress in \S\ref{sec:prompt_designing}. 

% \leiModify{How to connect to the rest parts of this paper?}

% Structure of this paper
% With the rapid development of ICL,
We organize the current progress in ICL following the taxonomy above (as shown in Figure~\ref{taxo_of_icl}). 
With a formal definition of ICL~(\S\ref{sec:formulation}), 
% by providing a comprehensive
% birds-eye view of the area and formal definition of in-context learning methods~(\S\ref{sec:formulation}).
we provide a detailed discussion of the warmup approaches~(\S\ref{sec:warmup}), the demonstration designing strategies (\S\ref{sec:prompt_tuning}), and the main scoring functions(\S\ref{sec:scoring}). \S\ref{sec:analysis} provides in-depth discussions of current explorations on unveiling the secrets behind the ICL. 
We further provide useful evaluation and resources for ICL~(\S\ref{sec:evaluation}) and introduce potential application scenarios where ICL shows its effectiveness~(\S\ref{sec:application}).
Finally, we summarize the challenges and potential directions~(\S\ref{sec:challege_future}) and hope this could pave the way for researchers in this field.
% In this paper, we present a curated survey on the related studies of in-context learning, to give a comprehensive
% birds-eye view of the area.

% Remove Structure 
% The paper is structured as follows:
% We discuss the related work in 
% \S\ref{sec:related}, by categorising the ICL methods into three types. 
% Furthermore, we introduce the prompting strategies of ICL in \S\ref{sec:approach}, as the the design of in-context demonstrations has a great impact on the model performance. Finally, we point out several challenges and future directions to facilitate the community in \S\ref{sec:impact}.


\section{Definition and Formulation}
\label{sec:formulation}
% \begin{figure}[t!]
%     \centering
%     \includegraphics[width=0.95\linewidth]{fig/icl-category.pdf}
%     \caption{Category of in-context learning prediction.}
%     \label{fig:icl_type}
% \end{figure}

% \leiModify{.}
% In this section, we first formulate the ICL framework, then we provide a brief overview of the main content of this survey.
Following the paper of GPT-3~\cite{gpt3}, we provide a definition of in-context learning:
\textsl{In-context learning is a paradigm that allows language models to learn tasks given only a few examples in the form of demonstration.
}
% Without loss of generality, we formulate the in-context learning in a classification problem setup and the adaption to other task formats discussed later.
Essentially, it estimates the likelihood of the potential answer conditioned on the demonstration by using a well-trained language model. 


Formally, given a query input text $x$ and a set of candidate answers $Y = \{y_1, \ldots, y_m\}$ (Y could be class labels or a set of free text phrases), a pretrained language model $\mathcal{M}$ takes the candidate answer with the maximum score as the prediction conditioning a demonstration set $C$. $C$ contains an optional task instruction $I$ and $k$ demonstration examples; therefore, $C = \{ I, s(x_1, y_1), \ldots, s(x_k, y_k) \}$ or $C = \{ s(x_1, y_1), \ldots, s(x_k, y_k) \}$, where $s(x_k, y_k,I)$ is an example written in natural language texts according to the task.
% therefore, $C = \{ s(x_1, y_1,I), \ldots, s(x_k, y_k,I) \}$, where $s(x_k, y_k,I)$ is an example written in natural language texts following the task instruction $I$.
% For a candidate label in a label space, the probability of a simple label candidate $y_j$ as a word prediction problem:
The likelihood of a candidate answer $y_j$ could be represented by a scoring function $f$ of the whole input sequence with the model $\mathcal{M}$:
% \leiModify{A more abstract Scoring function f, details later.}
\begin{equation}
    P( y_j \mid x) \triangleq
    f_\mathcal{M} ( y_j,  C, x)
    % \frac{\text{PPL}_\mathcal{M} ( y_j \mid C, x)} { \sum_m \text{PPL}_\mathcal{M} ( y_m \mid C, x)}.
\end{equation}
% This formulation degenerates into a word prediction problem when the label could be represented with a single word in the vocabulary of the LLM.
The final predicted label $\hat y$ is the candidate answer with the highest probability:
\begin{equation}
    \hat y = \arg\max_{y_j \in Y } P(y_j | x). 
\end{equation}
The scoring function $f$ estimates how possible the current answer is given the demonstration and the query text. For example, we could predict the class label in a binary sentiment classification by comparing the token probability of \emph{Negative} and \emph{Positive}. There are many $f$ variants for different applications, which will be elaborated in \S\ref{sec:scoring}. 

According to the definition, we can see the difference between ICL and other related concepts. (1) Prompt Learning: Prompts can be discrete templates or soft parameters that encourage the model to predict the desired output. Strictly speaking, ICL can be regarded as a subclass of prompt tuning where the demonstration is part of the prompt. %In ICL, the prompt format must be human-readable discrete texts and contain several demonstration examples. 
~\citet{liu2021pre} made a thorough survey on prompt learning. However, ICL is not included. (2) Few-shot Learning: few-shot learning is a general machine learning approach that uses parameter adaptation to learn the best model parameters for the task with a limited number of supervised examples~\cite{wang2019few}. In contrast, ICL does not require parameter updates and is directly performed on pretrained LLMs.
% given a query input text x and a candidate answer set Y = {y1, y2, . . . , ym}
% , we perform inference for an instance in interest $(x, y)$ by performing language completion with the input $x$ and demonstrated examples $c$ as the condition:
% ADD implementation variety.
% Table illustrates. 
% PPL.
% where $c$ is the context provided and is represented as a concatenated of $k$ instances $c= \{ x^c_1, y^c_1, \ldots, x^c_k, y^c_k \}$. $k$ is usually set to a relatively small number, such as $16$, and the effect of the demonstration number could be found in \S~X.
% Note that here we assume that all the input and target labels are represented by tokens in the vocabulary of the language model via a mapping function. 
% use an example to better illustrate 

% For example, we could perform binary sentiment classification for a sentence $x=$ \textit{The service of that coffee is really great.}, by comparing the conditional token probability of the token \textit{Negative} and \textit{Positive}.
% Demonstration examples with a similar form could be used as a context, e.g., \textit{The food is really bad. Negative;  The movie is crazy and I love it very much. Positive.} to assist the prediction.
% In this form, the model learns to perform the prediction based on the conditional context tokens in the prefix, without any computation-intensive parameter updates. 
% For the situation where the target answer is free text, the model is used to decode the answer directly as a text generation problem. The performance is evaluated via automatic metrics or human evaluation.
% The summarization of these two types of prediction is illustrated in Figure~\ref{fig:icl_type}.
% For tasks where the answer could not be represented by a single word in the vocabulary of the language model due to the tokenization process, an alternative is to compare the perplexity of all candidate sentences, where each sentence $S_j$ is the text concatenation of demonstration examples, input query and the candidate answer:
% Better calibration 
% \begin{align}
%     \hat y &= y_{\arg\min_{j} \text{PPL}_\mathcal{M} ( S_j)} \\ 
%     S_j &= \{ C, x, y_j \}
% \end{align}

% As recent explorations show that the vanilla probability could be sensitive to the order of demonstration examples, various calibrate methods have been introduced to stabilize the prediction.
% One representative framework is Calibrate, which eliminates the effect of biased prediction by introducing a normalization term estimated via a nonparametric method. Channel~(CITE) further shows a stronger regularization performance by computing the conditional probability of the input given the label.

% We summarize all these variants in Table~\ref{tab:icl_variants}.
% Word Prediction
% PPL prediction 
% 



% Benefits 

% on a large-scale text corpus with causal language modeling as the training objective

\section{Model Warmup}
\label{sec:warmup}

% \subsection{ICL Capability of LLMs}
Although LLMs have shown promising ICL capability, many studies also show that the ICL capability can be further improved through a continual training stage between pretraining and ICL inference, which we call model warmup for short.  %The  LLMs manifested the ICL ability. It is first shown by GPT-3~\cite{gpt3}, the 175 billion parameters autoregressive language model. Immediately following GPT-3, a growing number of papers have proposed LLMs with ICL capabilities, e.g., GLaM~\cite{glam}, OPT~\cite{opt}, and AlexaTM 20B~\cite{alexatm}.
Warmup is an optional procedure for ICL, which adjusts LLMs before ICL inference, including modifying the parameters of the LLMs or adding additional parameters. Unlike finetuning, warmup does not aim to train the LLM for specific tasks but enhances the overall ICL capability of the model. 

% \textbf{There are many training variables that affect the ICL capacity of the original LLMs.} 
% In terms of the model scale,~\citet{gpt3} find that the ICL ability  grows when the parameters in LLMs increase from 0.1 billion to 175 billion.
% In terms of the pertaining corpus, ~\citet{corpusscale} study how the source and size of the pretraining corpus influence the ICL ability of LLMs. They discover that, rather than the size of the pertaining corpus, the source and data combination is critical for the emergence of ICL abilities for LLMs.

% \subsection{Parameters Updating}
\subsection{Supervised In-context Training}
\label{sec:s_tuning}
To enhance ICL capability, researchers proposed a series of supervised in-context finetuning strategies by constructing in-context training data and multitask training.
% Although experiments validate the ICL ability of the original LLMs, the performance of ICL remains unsatisfactory and high-variance compared to finetuning, especially when the task is very XXX[cite]. 
Since the pretraining objectives are not optimized for in-context learning~\citep{selfsupericl}, \citet{metaicl} proposed a method MetaICL to eliminate the gap between pretraining and downstream ICL usage.
The pretrained LLM  is continually trained on a broad range of tasks with demonstration examples, which boosts its few-shot abilities.
To further encourage the model to learn input-label mappings from the context, \citet{symboltuning} propose symbol tuning. This approach fine-tunes language models on in-context input-label pairs, substituting natural language labels (e.g., "positive/negative sentiment") with arbitrary symbols (e.g., "foo/bar"). As a result, symbol tuning demonstrates an enhanced capacity to utilize in-context information for overriding prior semantic knowledge.
% , e.g., 
% %model ability on making predictions condition on the examples in the demonstrations. 
% MetaICL obtains performance comparable to supervised finetuning on $52$ unique datasets.

% \subsection{Instruction Tuning}
% \label{sec:instruction_tuning}
% Besides, there is a research direction focusing on supervised instruction tuning. 
Besides, recent work indicates the potential value of instructions~\cite{mishra2021cross} and there is a research direction focusing on supervised instruction tuning. Instruction tuning enhances the ICL ability of LLMs through training on task instructions. %, mainly leveraging the intuition that NLP tasks can be described via natural language instructions
% , and \citet{optiml} further include varying numbers of demonstration examples.
Tuning the 137B LaMDA-PT~\cite{lamda} on over 60 NLP datasets verbalized via natural language instruction templates, FLAN~\cite{flan} improves both the zero-shot and the few-shot ICL performance.
Compared to MetaICL, which constructs several demonstration examples for each task, instruction tuning mainly considers an explanation of the task and is more easier to scale up. \citet{chung} and \citet{natural} proposed to scale up instruction tuning with more than 1000+ task instructions. %, and ~\citet{chung} found that the majority of the instruction-tuning improvement comes from using up to 282 tasks.

\subsection{Self-supervised In-context Training}
\label{sec:ss_tuning}
Leveraging raw corpora for warmup, \citet{selfsupericl} proposed constructing self-supervised training data aligned with ICL formats in downstream tasks. They transformed raw text into input-output pairs, exploring four self-supervised objectives, including masked token prediction and classification tasks. Alternatively, PICL~\cite{picl} also utilizes raw corpora but employs a simple language modeling objective, promoting task inference and execution based on context while preserving pre-trained models' task generalization. Consequently, PICL outperforms \citet{selfsupericl}'s method in effectiveness and task generalizability.

% instructino-tuning does not have demonstration examples
% To boost the zero-shot ICL ability of LLMs, ~\citet{flan} introduce instruction tuning. Instruction tuning refers to train pretrained LLMs on a mixture of tasks described as instructions. The instruction-tuned model, FLAN, substantially manifests stronger zero-shot abilities on unseen tasks.
 
% \subsection{Additional Parameters Adding}
% \subsection{Contextual Bias Calibrating}
% \label{sec:additional}
% Considering that retraining or updating the LLMs is always costly, researchers focus on tuning additional parameters for more accurate and unbiased ICL.
% As ICL on LLMs may be biased or prompt-sensitive
% ~\cite{calibrate,chen2022sensitivity}  
% \citet{calibrate} propose contextual calibration to adjust the model predictions, which fit the calibration parameters in a linear layer so that the prediction for the input to be uniform across answers. Contextual calibration provides a simple but effective solution for mitigating the potential bias of ICL without additional training data.

% Compared to the parameters updating approaches, methods which adding additional parameters mainly focuses on minor model prediction calibrating while retraining the original model performance, especially for debiasing and sensitivity reducing.

\textbf{$\Diamond$ Takeaway}: \textit{
% Warmup is optional for ICL as many pretrained LLMs have manifested the ICL ability. 
(1) Supervised training and self-supervised training both propose to train the LLMs before ICL inference. The key idea is to bridge the gap between pretraining and downstream ICL formats by introducing objectives close to in-context learning. Compared to in-context finetuning involving demonstration, instruction finetuning without a few examples as demonstration is simpler and more popular.  % Although self-supervised training uses raw corpus, the task variety is relatively limited. 
(2) To some extent, these methods all improve the ICL capability by updating the model parameters, which implies that the ICL capability of the original LLMs has great potential for improvement. Therefore, although ICL does not strictly require model warmup, we recommend adding a warmup stage before ICL inference.
% and a reasonable warmup can also reduce the difficulty of the later prompt designing and the sensitivity of ICL inference. 
(3) 
% Whether through supervised in-context training or self-supervised in-context training, the improvement of warmup will reach a plateau when increasingly scaling up the training tasks or training corpus. This indicates that warmup only adapts the LLMs to learn from the context.
The performance advancement made by warmup encounters a plateau when increasingly scaling up the training data. This phenomenon appears both in supervised in-context training and self-supervised in-context training, indicating that LLMs only need a small amount of data to adapt to learn from the context during warmup.
% But the ICL capability can be further improved through warmup. 
% To enhance the ICL capability, 
% Training on ICL format tasks or instructions enhance the ICL capability, and adding additional parameters works for minor calibration.
%To retain the ICL performance and 
%, adding additional parameters will be a better solution.
}
% \leiModify{As a takeway, this could be more precise, straight to the point - Lei ?}


\section{Demonstration Designing}
\label{sec:prompt_tuning}
\label{sec:demo}
% \paragraph{Pre-Defined Metrics}
% \citet{chen2022sensitivity,lu2022order} define their new metrics and find them correlated to the ICL performance. 
% \citet{chen2022sensitivity} define a metric for ICL called prediction sensitivity, which measures the changes of the model output in response to small input perturbations. 
% They find there is an obvious negative correlation between prediction sensitivity and ICL accuracy, and motivated by it, propose a selective prediction method based on the prediction sensitivity. 
% \citet{lu2022order} analyze the sample order sensitivity for ICL in depth. 
% They first prove that the order sensitivity always exists for different model sizes and training samples, and a performant order is not transferable across models. 
% Then, they define the global and local entropy metrics and find a positive correlation between the entropy and the ICL performance. 
% With the entropy metrics, they can estimate a good sample ordering for ICL by choosing one that has higher local or global entropy. 


\begin{table*}[t!]
    \centering
    \setlength{\tabcolsep}{2pt}
    \small
    \resizebox{\linewidth}{!}{\begin{tabular}{@{}lcccc@{}}
    \toprule 
      \bf Category & \bf Methods &  \bf Demonstration Acquisition & \bf LLMs & \bf Main Tasks \\
    \midrule 
       \multirow{3}{*}{Demonstration Selection}  & KATE~\citep{liu2022close} & Human design & GPT-3& SST, table-to-text\\
       & SG-ICL~\cite{kim2022self} & LM generated & GPT-J & SST, NLI\\
       & EPR~\citep{rubin2022learning}  & Human design & GPT-\{J, 3\}/CodeX & Semantic parsing\\\midrule
        Demonstration Ordering & GlobalE \& LocalE~\citep{lu2022order} &  Human design & GPT-\{2, 3\} & Text classification\\ 
       \midrule%\midrule
       % Calibrate~\citep{calibrate} & $\frac{\mathcal{M} ( y_j \mid C, x)}{\mathcal{M} ( y_j \mid \text{NULL}))}$ & \\ 
       Instruction Formatting & Self Instruct~\citep{wang2022self} & LM generated & GPT-3/InstructGPT & SuperNaturalInstruction\\\midrule
       
       \multirow{3}{*}{Reasoning Steps Formatting}& CoT~\citep{cot} & Human design & GPT-3/CodeX & Reasoning tasks\\
       & AutoCoT~\citep{autocot} & LM generated & GPT-3/PaLM & Reasoning tasks\\
       &Self-Ask~\citep{selfask} & LM generated & GPT-3/InstructGPT & MultihopQA\\
       
    \bottomrule
    \end{tabular}}
    \caption{Summary of representative demonstration designing methods.}
    \label{tab:promptmethods}
\end{table*}
%In ICL, the inputs are fed to LLMs with demonstration examples to get outputs via language modeling. 
Many studies have shown that the performance of ICL strongly relies on the demonstration surface, including demonstration format, the order of demonstration examples, and so on~\citep{calibrate, lu2022order}. As demonstrations play a vital role in ICL, in this section, we survey demonstration designing strategies and classify them into two groups: demonstration organization and demonstration formatting, as shown in Table~\ref{tab:promptmethods}.
\subsection{Demonstration Organization}
\label{sec:organ}
Given a pool of training examples, demonstration organization focuses on how to select a subset of examples and the order of the selected examples. 

%Demonstrations show input-output mappings in ICL and the performance of ICL is sensitive to the combination and permutation of demonstrations. With limited demonstrations, demonstration organization strategies focus on how to \textbf{select} and \textbf{order} demonstrations.

\subsubsection{Demonstration Selection}
\label{sec:select}

% Suggestion from Xiaonna Li: 
% another taxnomy: into task-specific and instance specific selection categories.

Demonstrations selection aims to answer a fundamental question: Which examples are good examples for ICL? We classify related studies into two categories, including unsupervised methods based on pre-defined metrics and supervised methods.

\paragraph{Unsupervised Method} \citet{liu2022close} showed that selecting the closest neighbors as the in-context examples is a good solution. The distance metrics are pre-defined L2 distance or cosine-similarity distance based on sentence embeddings. %It outperformed significantly with farthest neighbors as examples. 
They proposed KATE, a $k$NN-based unsupervised retriever for selecting in-context examples. In addition to distance metrics, mutual information is also a valuable selection metric~\citep{sorensen2022information}. 
Similarly, $k$-NN cross-lingual demonstrations can be retrieved for multi-lingual ICL \citep{tanwar2023multilingual} to strengthen source-target language alignment.
%Another unsupervised example selection strategy is based on mutual information. 
%Without labeled examples and accessing to the LLMs, an example is selected when it has high mutual information. 
The advantage of mutual information is that it does not require labeled examples and specific LLMs. 
In addition, \citet{gonen2022demystifying} attempted to choose prompts with low perplexity. 
% \citet{Wu2022SelfadaptiveIL} selected the best subset permutation of $k$NN examples based on the code-length for data transmission to compress label $y$ given $x$ and $C$. This self-adaptive ranking method considered both selection and ordering. 
\citet{levy2022diverse} consider the diversity of demonstrations to improve compositional generalization. They select diverse demonstrations to cover different kinds of training demonstrations. 
%They also summarized the connection between code length and other metrics, including cross-entropy and mutual information.
Different from these studies selecting examples from human-labeled data, \citet{kim2022self} proposed to generate demonstrations from LLM itself. 

Some other methods utilized the output scores of LMs $P(y|C, x)$ as unsupervised metrics to select demonstrations.
\citet{Wu2022SelfadaptiveIL} selected the best subset permutation of $k$NN examples based on the code-length for data transmission to compress label $y$ given $x$ and $C$.
\citet{nguyen2023influence} measured the influence of a demonstration $x_i$ by calculating the difference between the average performance of the demonstration subsets $\{C|x_i\in C\}$ and $\{C|x_i\notin C\}$. Furthermore, \citet{li2023supporting} used infoscore, i.e., the average of $P(y|x_i,y_i,x) - P(y|x)$ for all $(x,y)$ pairs in a validation set with a diversity regularization.


\paragraph{Supervised Method} \citet{rubin2022learning} proposed a two-stage retrieval method to select demonstrations. For a specific input, it first built an unsupervised retriever (e.g., BM25) to recall similar examples as candidates and then built a supervised retriever EPR to select demonstrations from candidates. A scoring LM is used to evaluate the concatenation of each candidate example and the input. Candidates with high scores are labeled as positive examples, and candidates with low scores are hard negative examples. 
\citet{udr} further enhanced the EPR by adopting a unified demonstration retriever to unify the demonstration selection across different tasks.
\citet{ye2023compositional} retrieved the entire set of demonstrations instead of individual demonstrations to model inter-relationships between demonstrations. They trained a DPP retriever to align with LM output scores by contrastive learning and obtained the optimal demonstration set with maximum a posteriori at inference.

Based on prompt tuning, \citet{topic} view LLMs as topic models that can infer concepts $\theta$ from few demonstrations and generate tokens based on concept variables $\theta$. They use task-related concept tokens to represent latent concepts. Concept tokens are learned to maximize $P(y|x,\theta)$. They select demonstrations that are most likely to infer the concept variable based on $P(\theta|x,y)$. 
Besides, reinforcement learning was introduced by \citet{zhang2022active} for example selection. They formulated demonstration selection as a Markov decision process~\cite{bellman1957markovian} and selected demonstrations via Q-learning. The action is choosing an example, and the reward is defined as the accuracy of a labeled validation set. 


% \zc{}
%According to labeled candidates, EPR can be trained via in-batch contrastive learning.

% \paragraph{Reinforcement Learning Method} \citet{zhang2022active} formulated demonstration selection as a markov decision process and selected demonstrations via Q-learning. 

% Some researches (cite some works) focus on how to better organize the selected examples, including better ordering of selected examples and adding intermediate reasoning steps in demonstrations.
\subsubsection{Demonstration Ordering}
\label{sec:order}
Ordering the selected demonstration examples is also an important aspect of demonstration organization. 
\citet{lu2022order} have proven that order sensitivity is a common problem and always exists for various models. 
To handle this problem, previous studies have proposed several training-free methods to sort examples in the demonstration. \citet{liu2022close} sorted examples decently by their distances to the input, so the rightmost demonstration is the closest example.
% \citet{calibrate, lu2022order} show that the performance of ICL is sensitive to the permutation of demonstrations and the performance can vary from random guess to closely SOTA. 
%analyzed the sample order sensitivity for ICL in depth. They proved that order sensitivity is a common problem and always exists for various models. %, and a performant order is not transferable across models. 
\citet{lu2022order}  defined the global and local entropy metrics. They found a positive correlation between the entropy metric and the ICL performance. They directly used the entropy metric to select the best ordering of examples. %can estimate a good sample ordering for ICL by choosing one that has higher local or global entropy. 



% \subsubsection{Corpus- and Instance-Level Strategies}
% Overall, prompt formulation strategies can be divided into corpus-level and instance-level strategies.

\subsection{Demonstration Formatting}
\label{sec:format}
% \subsubsection{Instruction Formatting}
A common way to format demonstrations is concatenating examples $(x_1, y_1), \ldots, (x_k, y_k)$ with a template $\mathcal{T}$ directly. However, in some tasks that need complex reasoning (e.g., math word problems, commonsense reasoning), it is not easy to learn the mapping from $x_i$ to $y_i$ with only $k$ demonstrations. Although template engineering has been studied in prompting~\citep{liu2021pre},  some researchers aim to design a better format of demonstrations for ICL by describing tasks with the instruction $I$ (\S\ref{sec:formatting_instruction}) and adding intermediate reasoning steps between $x_i$ and $y_i$ (\S\ref{sec:formatting_intermediate}).


\subsubsection{Instruction Formatting}
\label{sec:formatting_instruction}
Except for the well-designed demonstration examples, good instructions which describe the task precisely are also helpful to the inference performance.
However, unlike the demonstration examples, which are common in traditional datasets, the task instructions depend heavily on human-written sentences.
\citet{induct} found that given several demonstration examples, LLMs can generate the task instruction.
According to the generation ability of LLMs, ~\citet{zhou2022large} proposed Automatic Prompt Engineer for automatic instruction generation and selection.
% To further save the human labor,
% ~\citet{} generate instruction by prompting the model to rephrase existing instruction.
To further improve the quality of the automatically generated instructions, ~\citet{wang2022self} proposed to use LLMs to bootstrap off its own generations.
Existing work has achieved good results in automatically generating instructions, which provided opportunities for future research on combining human feedback with automatic instruction generation.





\subsubsection{Reasoning Steps Formatting}
\label{sec:formatting_intermediate}
% \subsubsection{Reasoning}



\citet{cot} added intermediate reasoning steps between inputs and outputs to construct demonstrations, which are called chain-of-thoughts (CoT). With CoT, LLMs predict the reasoning steps and the final answer. CoT prompting can learn complex reasoning by decomposing input-output mappings into many intermediate steps. There are many pieces of research on CoT prompting strategies ~\citep{qiao2022reasoning} including prompt designing and process optimization. In this paper, we mainly focus on CoT designing strategies.

Similar to demonstration selection, CoT designing also considers CoT selection. Different from \citet{cot} manually writing CoTs, AutoCoT \citep{autocot} used LLMs with \textit{Let's think step by step} to generate CoTs. In addition, \citet{fu2022complexitycot} proposed a complexity-based demonstration selection method. They selected demonstrations with more reasoning steps for CoT prompting. 

As input-output mappings are decomposed into step-by-step reasoning, some researchers apply multi-stage ICL for CoT prompting and design CoT demonstrations for each step. Multi-stage ICL queries LLMs with different demonstrations in each reasoning step. Self-Ask \citep{selfask} allows LLMs to generate follow-up questions for the input and ask themselves these questions. Then the questions and intermediate answers will be added to CoTs. iCAP \citep{wang2022iteratively} proposes a context-aware prompter that can dynamically adjust contexts for each reasoning step. Least-to-Most Prompting \citep{least} is a two-stage ICL including question reduction and subquestion solution. The first stage decomposes a complex question into subquestions; in the second stage, LLMs answer subquestions sequentially, and previously answered questions and generated answers will be added into the context.

\citet{xu2023small} fine-tuned small LMs on specific task as plug-ins to generate pseudo reasoning steps. Given an input-output pair $(x_i,y_i)$, SuperICL regarded the prediction $y_i'$ and confidence $c_i$ of small LMs for the input $x_i$ as reasoning steps by concatenating $(x_i, y_i', c_i, y_i)$. 
% They further requested LMs to generate over-ride explanations if the final prediction $y$ is in-consistent with the prediction of small LMs $y'$.

% add \citep{mot} 


% \subsubsection{XXXX}
% \textit{\# Takeaway: Demonstration Designing strategies include Demonstration Organization and Demonstration Formatting. Demonstration Organization focuses on how to select and order demonstrations. Demonstration Formatting focuses on a better format of demonstrations.}
\textbf{$\Diamond$ Takeaway}: \textit{ (1) Demonstration selection strategies improve the ICL performance, but most of them are instance level. Since ICL is mainly evaluated under few-shot settings, the corpus-level selection strategy is more important yet under-explored. (2) The output score or probability distribution of LLMs plays an important role in instance selecting. (3) For $k$ demonstrations, the size of search space of permutations is $k!$. How to find the best orders efficiently or how to approximate the optimal ranking better is also a challenging question. (4) Adding chain-of-thoughts can effectively decompose complex reasoning tasks into intermediate reasoning steps. During inference, multi-stage demonstration designing strategies are applied to generate CoTs better. How to improve the CoT prompting ability of LLMs is also worth exploring (5) In addition to human-written demonstrations, the generative nature of LLMs can be utilized in demonstration designing. LLMs can generate instructions, demonstrations, probing sets, chain-of-thoughts, and so on. By using LLM-generated demonstrations, ICL can largely get rid of human efforts on writing templates. }

\section{Scoring Function}
\label{sec:scoring}
% \zy{shall we briefly discuss the pros and cons of different variants? e.g., direct limit the choices of templates, ppl is less efficient compared to direct, etc. probably add two columns in the table?}

\begin{table}[t!]
    \centering
    \resizebox{\linewidth}{!}{
    \begin{tabular}{@{}l|c|ccc@{}}
    \toprule 
      \bf Scoring Function & \bf Target & \bf Efficiency & \bf Task Coverage & \bf Stability \\
    \midrule 
       Direct  & $ \mathcal{M} ( y_j \mid C, x) $ & +++  &  + & + \\ 
       PPL &  $\text{PPL} (S_j) $&  + & +++ &  +\\ 
       % Calibrate~\citep{calibrate} & $\frac{\mathcal{M} ( y_j \mid C, x)}{\mathcal{M} ( y_j \mid \text{NULL}))}$ & \\ 
       Channel& $\mathcal{M} (x \mid C, y_j)$ & + & +  & ++ \\ 
    \bottomrule
    \end{tabular}}
    \caption{Summary of different scoring functions. }
    \label{tab:score_func}
\end{table}
% \qingxiu{citating}
The scoring function decides how we can transform the predictions of a language model into an estimation of the likelihood of a specific answer.
A direct estimation method~(Direct) adopts the conditional probability of candidate answers that can be represented by tokens in the vocabulary of language models~\citep{gpt3}. The answer with a higher probability is selected as the final answer. However, this method poses some restrictions on the template design, e.g., the answer tokens should be placed at the end of input sequences.
% \leiModify{Highlight the difference between task type and prediction type.}
Perplexity~(PPL) is another commonly-used metric, which computes the sentence perplexity of the whole input sequence $S_j = \{ C, s(x, y_j, I)\}$ consists of the tokens of demonstration examples $C$, input query $x$ and candidate label $y_j$. 
As PPL evaluates the probability of the whole sentence, it removes the limitations of token positions but requires extra computation time. 
Note that in generation tasks such as machine translation, ICL predicts the answer by decoding tokens with the highest sentence probability combined with diversity-promoting strategies such as beam search or Top-$p$ and Top-$k$~\citep{topp_sample} sampling algorithms.


Different from previous methods, which estimate the probability of the label given the input context, 
\citet{min2022noisy} proposed to utilize channel models~(Channel) to compute the conditional probability in a reversed direction, i.e., estimating the likelihood of input 
query given the label. In this way, language models are required to generate every token in the input, which could boost the performance under imbalanced training data regimes.
We summarize all three scoring functions in Table~\ref{tab:score_func}.
As ICL is sensitive to the demonstration~(see \S\ref{sec:prompt_tuning} for more details), normalizing the obtained score by subtracting a model-dependent prior with empty inputs is also effective for improving the stability and overall performance~\citep{calibrate}.

Another direction is to incorporate information beyond  the context length constrain to calibrate the score.
% demonstration examples are separately encoded with well-designed
% position embeddings, and then they are jointly attended by the test example using
% a rescaled attention mechanism.
Structured Prompting~\citep{hao2022structured} proposes to encode demonstration examples separately with special positional embeddings, which then are provided to the test examples with a rescaled attention mechanism.
$k$NN Prompting~\citep{knnPrompting} first queries LLMs with
training data for distributed representations, then predicts test instances by simply referring to nearest neighbors with closing representations with stored anchor representations.

% \textbf{$\Diamond$ Takeaway}: \textit{Selecting a proper scoring function according to the tasks in hand is important, and it has a great influence on the stability of the results.}
\textbf{$\Diamond$ Takeaway}: \textit{(1) We conclude the characteristics of three widely-used scoring functions in Table~\ref{tab:score_func}. Although directly adopting the conditional probability of candidate answers is efficient, this method still poses some restrictions on the template design. Perplexity is also a simple and widely scoring function. This method has universal applications, including both classification tasks and generation tasks. However, both methods are still sensitive to demonstration surface, while Channel is a remedy that especially works under imbalanced data regimes. (2) Existing scoring functions all compute a score straightforwardly from the conditional probability of LLMs. There is limited research on calibrating the bias or mitigating the sensitivity via scoring strategies. For instance, some studies add additional calibration parameters to adjust the model predictions~\cite{calibrate}. 
}



\section{Analysis}
\label{sec:analysis}
% \begin{table*}[t]
%     \centering
%     \begin{tabularx}{0.98\textwidth}{l l X}
%     \toprule
%     \textbf{Stage} & \textbf{Factor} & \textbf{Correlation to ICL} \\
%     \midrule
%     \multirow{6}{*}{\makecell[l]{Pretraining}} & \makecell[l]{Pretraining corpus domain \\ \citep{shin2022corpora}} & \makecell[X]{Compared to the pretraining corpus size, the corpus domain is more influential} \\
%      & \makecell[l]{Pretraining corpus combination \\ \citep{shin2022corpora}} & \makecell[X]{Combining two poor-performing pretraining corpora may introduce emergent ICL ability} \\
%      & \makecell[l]{Model scale \\ \citep{wei2022emergent,gpt3}} & \makecell[X]{Models with more pretraining steps and parameters are more likely to have emergent ability} \\
%     \midrule
%     \multirow{7}{*}{\makecell[l]{Inference}} & \makecell[l]{Label space exposure \\ \citep{min2022rethinking}} & \makecell[X]{Label space exposure has big impact on non-channel ICL models} \\
%      & \makecell[l]{Input distribution \\ \citep{min2022rethinking}} & \makecell[X]{In-distribution demonstrations substantially contribute to ICL performance} \\
%     & \makecell[l]{Format of input-label pairing \\ \citep{min2022rethinking}} & \makecell[X]{The format of input-label pairing contributes much to the correct prediction of ICL} \\
%     & \makecell[l]{Demonstration-query similarity \\ \citep{liu2022close}} & \makecell[X]{Using demonstrations that are more similar to the query will bring better performance} \\
%     \bottomrule
%     \end{tabularx}
%     \caption{Summary of influence factors that have relatively strong correlation to ICL.}
%     \label{tab:factor}
% \end{table*}

\begin{table}[t]
    \centering
    \footnotesize
    \begin{tabular}{cc}
    \toprule
    \textbf{Stage} & \textbf{Factor} \\
    \midrule
    \multirow{7}{*}{\makecell[l]{Pretraining}} & \makecell[l]{Pretraining corpus domain \\ \citep{shin2022corpora}} \\
     & \makecell[l]{Pretraining corpus combination \\ \citep{shin2022corpora}} \\
     & \makecell[l]{Number of model parameters \\ \citep{wei2022emergent,gpt3}} \\
     & \makecell[l]{Number of pretraining steps \\ \citep{wei2022emergent}} \\
    \midrule
    \multirow{15}{*}{\makecell[l]{Inference}} & \makecell[l]{Label space exposure \\ \citep{min2022rethinking}} \\
     & \makecell[l]{Demonstration input distribution \\ \citep{min2022rethinking}} \\
    & \makecell[l]{Format of input-label pairing \\ \citep{min2022rethinking,compositional_generalization}} \\
    & \makecell[l]{Demonstration input-label mapping \\ \citep{min2022rethinking,ground_truth} \\ 
    \citep{llm_icl_differently}} \\
    & \makecell[l]{Demonstration sample ordering \\ \citep{lu2022order}} \\
    & \makecell[l]{Demonstration-query similarity \\ \citep{liu2022close}} \\
    & \makecell[l]{Demonstration diversity \\ \citep{compositional_generalization}} \\
    & \makecell[l]{Demonstration complexity \\ \citep{compositional_generalization}} \\
    \bottomrule
    \end{tabular}
    \caption{Summary of factors that have a relatively strong correlation to ICL performance.}
    \label{tab:factor}
\end{table}

To understand ICL, many analytical studies attempt to investigate what factors may influence the performance and aim to figure out why ICL works. We summarize the factors that have a relatively strong correlation to ICL performance in Table~\ref{tab:factor} for easy reference.  

\subsection{What Influences ICL Performance}

% factors - pretraining
\paragraph{Pre-training Stage}
We first introduce influence factors in the LLM pretraining stage. 
\citet{shin2022corpora} investigated the influence of the pretraining corpora. 
They found that the domain source is more important than the corpus size. Putting multiple corpora together may give rise to emergent ICL ability, pretraining on corpora related to the downstream tasks does not always improve the ICL performance, and models with lower perplexity do not always perform better in the ICL scenarios.
\citet{wei2022emergent} investigated the emergent abilities of many large-scale models on multiple tasks. They suggested that a pretrained model suddenly acquires some emergent ICL abilities when it achieves a large scale of pretraining steps or model parameters. 
\citet{gpt3} also showed that the ICL ability grows as the parameters of LLMs increase from 0.1 billion to 175 billion.

% factors - inference
\paragraph{Inference Stage}
In the inference stage, the properties of the demonstration samples also influence the ICL performance. 
\citet{min2022rethinking} investigated that the influence of demonstration samples comes from four aspects: the input-label pairing format, the label space, the input distribution, and the input-label mapping. 
They prove that all of the input-label pairing formats, the exposure of label space, and the input distribution contribute substantially to the ICL performance. Counter-intuitively, the input-label mapping matters little to ICL. 
In terms of the effect of input-label mapping, \citet{ground_truth} drew an opposite conclusion that correct input-label mapping does impact the ICL performance, depending on specific experimental settings.
\citet{llm_icl_differently} further found that when a model is large enough, it will show an emergent ability to learn input-label mappings, even if the labels are flipped or semantically-unrelated. 
From the compositional generalization perspective, \citet{compositional_generalization} validated that ICL demonstrations should be diverse, simple, and similar to the test example in terms of the structure. 
\citet{lu2022order} indicated that the demonstration sample order is also an important factor.
In addition, \citet{liu2022close} found that the demonstration samples that have closer embeddings to the query samples usually bring better performance than those with farther embeddings. 



\subsection{Understanding Why ICL Works}

\paragraph{Distribution of Training Data} 
Concentrating on the pretraining data, \citet{distribution} showed that the ICL ability is driven by data distributional properties. 
They found that the ICL ability emerges when the training data have examples appearing in clusters and have enough rare classes. 
\citet{bayesian} explained ICL as implicit Bayesian inference and constructed a synthetic dataset to prove that the ICL ability emerges when the pretraining distribution follows a mixture of hidden Markov models. 

\paragraph{Learning Mechanism} 
By learning linear functions, \citet{garg2022linear} proved that Transformers could encode effective learning algorithms to learn unseen linear functions according to demonstration samples. 
They also found that the learning algorithm encoded in an ICL model can achieve a comparable error to that from a least squares estimator. 
\citet{trm_as_alg} abstracted ICL as an algorithm learning problem and showed that Transformers can implement a proper function class through implicit empirical risk minimization for the demonstrations. 
\citet{tr_and_tl} decoupled the ICL ability into task recognition ability and task learning ability, and further showed how they utilize demonstrations. 
From an information-theoretic perspective, \citet{implicit_structure_induction} showed an error bound for ICL under linguistically motivated assumptions to explain how next-token prediction can bring about the ICL ability. 
\citet{inductive_bias} found that large language models exhibit prior feature biases and showed a way to use intervention to avoid unintended features in ICL. 

Another series of work attempted to build connections between ICL and gradient descent. 
Taking linear regression as a starting point, \citet{akyurek2022algorithm} found that Transformer-based in-context learners can implement standard finetuning algorithms implicitly, and \citet{icl_gd} showed that linear attention-only Transformers with hand-constructed parameters and models learned by gradient descent are highly related. 
Based on softmax regression, \citet{icl_weight_shifting} found that self-attention-only Transformers showed similarity with models learned by gradient-descent. 
\citet{dai2022iclft} figured out a dual form between Transformer attention and gradient descent and further proposed to understand ICL as implicit finetuning. 
Further, they compared GPT-based ICL and explicit finetuning on real tasks and found that ICL indeed behaves similarly to finetuning from multiple perspectives. 

\paragraph{Functional Components} 
Focusing on specific functional modules, \citet{olsson2022induction} found that there exist some induction heads in Transformers that copy previous patterns to complete the next token. 
Further, they expanded the function of induction heads to more abstract pattern matching and completion, which may implement ICL. 
% They provided several pieces of evidence on models with different sizes, which support their viewpoint that induction heads constitute the mechanism of ICL. 
\citet{label_anchor} focused on the information flow in Transformers and found that during the ICL process, demonstration label words serves as anchors, which aggregates and distributes key information for the final prediction. 

\textbf{$\Diamond$ Takeaway}: \textit{
(1) Knowing and considering how ICL works can help us improve the ICL performance, and the factors that strongly correlate to ICL performance are listed in Table~\ref{tab:factor}.
(2) 
Although some analytical studies have taken a preliminary step to explain ICL, most of them are limited to simple tasks and small models. 
Extending analysis on extensive tasks and large models may be the next step to be considered. 
In addition, among existing work, explaining ICL with gradient descent seems to be a reasonable, general, and promising direction for future research. 
If we build clear connections between ICL and gradient-descent-based learning, we can borrow ideas from the history of traditional deep learning to improve ICL. 
}

\section{Evaluation and Resources}
\label{sec:evaluation}

\subsection{Traditional Tasks}
As a general learning paradigm, ICL can be examined on various traditional datasets and benchmarks, e.g., SuperGLUE~\cite{superglue}, SQuAD~\cite{squad}. 
Implementing ICL with 32 randomly sampled examples on SuperGLUE, ~\citet{gpt3} found that GPT-3 can achieve results comparable to state-of-the-art (SOTA) finetuning performance on COPA and ReCoRD, but still falls behind finetuning on most NLU tasks.
~\citet{hao2022structured} showed the potential of scaling up the number of demonstration examples. However, the improvement brought by scaling is very limited. At present, compared to finetuning, there still remains some room for ICL to reach on traditional NLP tasks.

%, and the number of demonstration examples is still not as unlimited as finetuning

%And XXX and XXX has shown XXXX, XXX are studying the XXX. % gpt3 paper examined on superglue and squad
\begin{table}[t!]
    \small
    \setlength{\tabcolsep}{4pt}
    \centering
    \resizebox{\linewidth}{!}{\begin{tabular}{llr}
    \toprule
     \bf Benchmark  & \bf Tasks  & \bf \#Tasks  \\
     \midrule
    \makecell[l]{BIG-Bench \\ \cite{beyond}}   & {\makecell[l]{Mixed tasks}} & 204 \\
    \makecell[l]{BBH \\ \cite{suzgun2022challenging}}   & {\makecell[l]{Unsolved problems}} & 23  \\
    \makecell[l]{PRONTOQA \\ \cite{heuristic}}  & {\makecell[l]{Question answering}} &  1 \\
    \makecell[l]{MGSM \\ \cite{shi2022language}} & {\makecell[l]{Math problems}} & 1  \\
    \makecell[l]{LLMAS \\ \cite{planbench}} & {\makecell[l]{Plan and reasoning tasks}}  & 8\\
    \makecell[l]{OPT-IML Bench \\ \cite{optiml}} & {\makecell[l]{Mixed tasks}} & 2000 \\
    \bottomrule
    \end{tabular}}
    \caption{New challenging evaluation benchmarks for ICL. For short, we use LLMAS to represent LLM Assessment Suite~\cite{planbench}.} 
    \label{tab:dataset}
\end{table}
% \begin{table*}[]
\small
    \centering
    \begin{tabular}{lccccccccc}
    \toprule
     ~ & \bf SuperGLUE & \bf BoolQ & \bf CB  & \bf COPA & \bf RTE  & \bf WiC  & \bf WSC  & \bf MultiRC & \bf  ReCoRD \\
 ~& Avg. & Acc. & Acc.,F1 & Acc. & Acc. &
 Acc.  & Acc. & Acc.,F1a & Acc.,F1\\
     \midrule
     Human Baseline & 89.8  & 89.0  & 95.8/98.9  & 100.0  & 81.8/51.9  & 91.7/91.3  & 93.6  & 80.0 & 100.0   \\
     Finetuned SOTA \\
     Vanilla GPT3  &   &  &   & \\
    \makecell[l]{Chain-of-thoughts \\ ~\cite{beyond}}   &   & &   & \\
    \makecell[l]{Least-to-most\\ ~\cite{suzgun2022challenging}}   & &  &   & \\
    \bottomrule
    \end{tabular}
    \caption{Benchmarking exsiting prompting strategies on the GLUE tasks with 8 demonstration examples on each dataset.}
    \label{tab:evaluation}
\end{table*}

% To make a fair comparison of different prompting strategies for ICL on traditional natural language understanding tasks, we implement XX prompting strategies for ICL with a same number of demonstration examples and a same backbone model. As it shown in Tab.~\ref{tab:evaluation}, XXXX.

\subsection{New Challenging Tasks}
% Except for the traditional extrinsic evaluation
%The powerful few-shot capability of ICL also gives rise to a new paradigm in datasets and evaluation.
In the era of large language models with in-context learning capabilities, researchers are more interested in evaluating the intrinsic capabilities of large language models without downstream task finetuning~\cite{foundation}.

To explore the capability limitations of LLM on various tasks, 
% more than 400 researchers from various institutions 
~\citet{beyond} proposed the BIG-Bench~\cite{beyond}, a large benchmark covering  
% 204 tasks with good diversity. 
a large range of tasks, including linguistics, chemistry, biology, social behavior, and beyond. 
The best models have already outperformed the average reported human-rater results on 65\% of the BIG-Bench tasks through ICL~\cite{suzgun2022challenging}. To further explore tasks actually unsolvable by current language models, \citet{suzgun2022challenging} proposed a more challenging ICL benchmark, BIG-Bench Hard (BBH). BBH includes 23 unsolved tasks, constructed by selecting challenging tasks where the state-of-art model performances are far below the human performances. Besides, researchers are searching for inverse scaling tasks,\footnote{\url{https://github.com/inverse-scaling/prize}} that is, tasks where model performance reduces when scaling up the model size. Such tasks also highlight potential issues with the current paradigm of ICL.
To further probe the model generalization ability, ~\citet{optiml} proposed OPT-IML Bench, consisting of 2000 NLP tasks from 8 existing benchmarks, especially benchmark for ICL on held-out categories.

Specifically, a series of studies focus on exploring the reasoning ability of ICL.~\citet{heuristic} generated an example from a synthetic world model
represented in first-order logic and parsed the ICL generations into symbolic proofs for formal analysis. They found that LLMs can make correct individual deduction steps via ICL.
~\citet{shi2022language} constructed the MGSM benchmark to evaluate the chain-of-thought reasoning abilities of LLMs in multilingual settings, finding that LLMs manifest complex reasoning across multiple languages.
To further probe more sophisticated planning and reasoning abilities of LLMs, ~\citet{planbench} provided multiple test cases for evaluating various reasoning abilities on actions and change, where existing ICL methods on LLMs show poor performance.

\subsection{Open-source Tools}
Noticing that ICL methods are often implemented differently and evaluated using different LLMs and tasks, \citet{openicl} developed OpenICL, an open-source toolkit enabling flexible and unified ICL assessment. With its adaptable architecture, OpenICL facilitates the combination of distinct components and offers state-of-the-art retrieval and inference techniques to accelerate the integration of ICL into advanced research.


%In summary, the evaluation for ICL on LLMs is manifesting the following characteristics:
 % Traditional datasets lack challenge for ICL; 2) 
% \textit{Takeaway: Intrinsic evaluation without large training set is a new tendency. The diversity and challenges are particularly valued for new benchmarks.}
% 4) Evaluation under the ICL paradigm brings new challenges, more insensitive and trustworthy evaluation paradigms is a new direction (Section~\ref{challenge: evaluation}).
\textbf{$\Diamond$ Takeaway}: \textit{(1) Due to the restrictions of ICL on the number of demonstration examples, the traditional evaluation tasks must be adapted to few-shot settings; otherwise, the traditional benchmarks cannot evaluate the ICL capability of LLMs directly.
(2) As ICL is a new paradigm that is different from traditional learning paradigms in many aspects, the evaluation of ICL presents new challenges and opportunities.
Toward the challenges, the results of existing evaluation methods are unstable, especially sensitive to the demonstration examples and the instructions. \citet{chen2022relation} observed that existing evaluations by accuracy underestimate the sensitivity towards instruction perturbation of ICL.
It is still an open question to conduct consistent ICL evaluation and OpenICL\cite{openicl} represents a valuable initial attempt to address this challenge.
% Apart from that, traditional evaluation benchmarks such as SuperGLUE~\cite{superglue} are less challenging for LLMs with billions of parameters, and even the sophisticated designed new benchmark, BigBench~\cite{beyond} can be readily surpassed.
% How to construct a challenging evaluation with long-term vitality is crucial for ICL evaluation.
Toward the opportunities for evaluation, as ICL only requires a few instances for the demonstration, it lowers the cost of evaluation data construction. 
}

%Furthermore, the test set is still supposed to faithfully reflect the world distribution. 
%Instead of the i.i.d. assumption with the training set, the test set requires new constraints.



\section{In-Context Learning Beyond Text}
\label{sec:mm_icl}
% Multimodal Applications
% Due to the great success of ICL in the area of NLP, the idea also has motivated pilot studies toward investigating the ICL ability in visual and multimodal tasks.

The tremendous success of ICL in NLP  has inspired researchers to explore its potential in different modalities, including visual, vision+language and speech tasks as well. 

\subsection{Visual In-Context Learning}
\begin{figure}
    \centering
    \includegraphics[width=0.49\textwidth]{fig/mm-case.pdf}
    \caption{Image-only and textual augmented prompting for visual in-context learning.}
    \label{fig:mm_case}
\end{figure}
\citet{bar2022visual_icl} employ an image patch infilling task in grid-like images using masked auto-encoders (MAE) to train their model. 
At the inference stage, the model generates output images consistent with provided input-output examples for a novel input image, showcasing promising ICL capabilities for unseen tasks such as image segmentation. Painter~\citep{wang2023imagesPainter} extends this approach by incorporating multiple tasks to build a generalist model, achieving competitive performance compared to task-specific models. Building upon this, SegGPT~\citep{wang2023seggpt} integrates diverse segmentation tasks into a unified framework and investigates ensemble techniques from spatial and feature perspectives to enhance the quality of prompt examples.
% \citet{wang2023prompt_diffusion} further propose to incorporate a text prompt along with the original input-output image pair to guide a generative model to comprehensively produce the desired image. The resulting Prompt Diffusion model~\citep{wang2023prompt_diffusion} is the first diffusion-based model that exhibits in-context learning ability.
\citet{wang2023prompt_diffusion} propose to utilize an extra text prompt to guide a generative model in comprehensively producing the desired image. The resulting Prompt Diffusion model is the first diffusion-based model that exhibits ICL ability. 
Figure~\ref{fig:mm_case} illustrates the key difference between the image-only and textual prompt augmented in-context learning for visual in-context learning.
% \paragraph{Visual In-Context Learning}
% \citet{bar2022visual_icl} utilize an image patch infilling task in grid-like images to train a model based on masked auto-encoders~(MAE).
% During the inference stage, given input-output image examples and a new input image, the model could automatically produce the output image that is consistent with the given examples. This paradigm demonstrates promising in-context learning on unseen tasks such as image segmentation. Painter~\citep{wang2023imagesPainter} further incorporates more tasks to construct a generalist model, achieving competitive results
% compared to task-specific models.
% Subsequently, SegGPT~\citep{wang2023seggpt} 
% unifies various segmentation tasks into a generalist framework, and explores ensemble methods from spatial 
% and feature aspects for improving the quality of prompt examples.


Similar to ICL in NLP, the effectiveness of visual in-context learning is significantly influenced by the selection of in-context demonstration images~\citep{zhang2023visual_icl_analysis,sun2023exploring_visual_icl}. 
To address this, \citet{zhang2023visual_icl_analysis} investigate two approaches: (1) an unsupervised retriever that selects nearest samples using an off-the-shelf model, and (2) a supervised method training an additional retriever model to maximize ICL performance. The retrieved samples notably enhance performance, exhibiting semantic similarity to the query and closer contextual alignment regarding viewpoint, background, and appearance.  Except for the prompt retrieval, \citet{sun2023exploring_visual_icl} further explore a prompt fusion technique for improving the results.




\subsection{Multi-Modal In-Context Learning}
In the vision-language area, \citet{tsimpoukelli2021frozen} utilize a vision encoder to represent an image as a prefix embedding sequence that is aligned with a frozen language model after training on the paired image-caption dataset.
The resulting model, Frozen, is capable of performing multi-modal few-shot learning. 
Further, \citet{alayrac2022flamingo} introduce
Flamingo, which 
combines a vision encoder with LLMs and adopts LLMs as the general interface to perform in-context learning on many multi-modal tasks.
They show that training on large-scale multi-modal web corpora with arbitrarily interleaved text and images is key to endowing them with in-context few-shot learning capabilities. 
Kosmos-1~\citep{huang2023kosmos} is another multi-modal LLMs and demonstrates promising zero-shot, few-shot,
and even multimodal chain-of-thought prompting abilities.
~\citet{hao2022language} present METALM, a general-purpose interface to models across tasks and modalities. With a semi-causal language modeling objective, METALM is pretrained and exhibits strong ICL performance across various vision-language tasks. 




It is natural to further enhance the ICL ability with instruction tuning, and the idea is also explored in the multi-modal scenarios as well.
Recent explorations first generate instruction tuning datasets transforming existing vision-language task dataset~\citep{xu2022multiinstruct,li2023otter} or with power LLMs such as GPT-4~\citep{liu2023llava,zhu2023minigpt4} , and connect LLMs with powerful vision foundational models such as BLIP-2~\citep{li2023blip2} on these multi-modal datasets~\citep{zhu2023minigpt4,dai2023instructblip}.



% \begin{table}[t!]
%     \centering
%     \begin{tabular}{l|c c}
%     \toprule
%       Modality   & Task &  Work   \\
%       \midrule
%      Vision  &  image generation & \\ 
%      Speech  & speech synthesis & \\ 
%      Vision + Language & image-to-text& \\ 
%     \bottomrule
%     \end{tabular}
%     \caption{Summarization of work investigates In-context learning beyond natural language.}
%     \label{tab:my_label}
% \end{table}

\subsection{Speech In-Context Learning}
In the speech area, ~\citet{wang2023neural} treated text-to-speech synthesis as a language modeling task. 
They use audio codec codes as an intermediate representation and propose the first TTS framework with strong in-context learning capability. 
Subsequently, VALLE-X~\citep{zhang2023valle-x} extend the idea to multi-lingual scenarios, demonstrating superior performance in zero-shot cross-lingual text-to-speech synthesis and zero-shot speech-to-speech translation tasks.

\textbf{$\Diamond$ Takeaway}: 
\textit{
(1) Recent studies have explored in-context learning beyond natural language with promising results.
Properly formatted data (e.g., interleaved image-text datasets for vision-language tasks) and architecture designs are key factors for activating the potential of in-context learning. Exploring it in a more complex structured space such as for graph data is challenging and promising~\citep{huang2023graph_icl}.
(2) Findings in textual in-context learning demonstration design and selection cannot be trivially transferred to other modalities. Domain-specific investigation is required to fully leverage the potential of in-context learning in various modalities.
% Findings in textual in-context learning demonstration design and selection cannot be trivially transferred to other modalities, and require domain-specific investigation. 
}

% 



\section{Application}
\label{sec:application}
% ICL for traditional NLP tasks (dialogue)
\label{app}
%Due to the human-friendly interface lightweight prompting advantages, ICL has been widely applied in various scenarios recently.

ICL manifests excellent performance on traditional NLP tasks and methods~\cite{kim2022self,metaicl}, such as machine translation~\cite{zhu2023multilingual,sia2023context}, information extraction~\cite{wan2023gpt,he2023icl} and text-to-SQL~\cite{pourreza2023din}.
% tasks
%And ICL manifests excellent performance in various task
%, such as sentiment classification natural language inference~\cite{kim2022self}. 
Especially, through demonstrations that explicitly guide the process of reasoning, ICL manifests remarkable effects on tasks that require complexity reasoning~\cite{cot,li2023code,teachalgo} and compositional generalization~\cite{least}. 
% methods
% ~\citet{chen2021meta} applies ICL for meta-learning. 

Moreover, ICL offers potential for popular methods such as meta-learning and instruction-tuning. \citet{chen2021meta} applied ICL to meta-learning, adapting to new tasks with frozen model parameters, thus addressing the complex nested optimization issue. \cite{ye2023context} enhanced zero-shot task generalization performance for both pretrained and instruction-finetuned models by applying in-context learning to instruction learning. 

Specifically, we explore several emerging and prevalent applications of ICL, showcasing their potential in the following paragraphs.
% ICL for model editing.


% ICL for data engineering
% \paragraph{Data Engineering} What's more, ICL has manifested the potential to be widely applied in data engineering. 
% Benefiting from the strong ICL ability,  it costs 50\% to 96\% less to use labels from GPT-3 than using labels from humans for data annotation. Combining pseudo labels from GPT-3 with human labels leads to even better performance at a small cost~\cite{want}.
% In more complex scenarios, such as knowledge graph construction, \citet{khorashadizadeh2023exploring} has demonstrated that ICL has the potential to significantly improve the state of the art of automatic construction and completion of knowledge graphs, resulting in a reduction in manual costs with minimal engineering effort. Therefore, leveraging the capabilities of ICL in various data engineering applications can yield significant benefits.
\paragraph{Data Engineering}
ICL has manifested the potential to be widely applied in data engineering. 
Benefiting from the strong ICL ability,  it costs 50\% to 96\% less to use labels from GPT-3 than using labels from humans for data annotation. Combining pseudo labels from GPT-3 with human labels leads to even better performance at a small cost~\cite{want}.
In more complex scenarios, such as knowledge graph construction, \citet{khorashadizadeh2023exploring} has demonstrated that ICL has the potential to significantly improve the state of the art of automatic construction and completion of knowledge graphs, resulting in a reduction in manual costs with minimal engineering effort. Therefore, leveraging the capabilities of ICL in various data engineering applications can yield significant benefits.
Compared to human annotation (e.g., crowd-sourcing) or noisy automatic annotation (e.g., distant supervision), ICL generates relatively high quality data at a low cost. 
% First, ICL takes only a few examples to learn the data engineering objective, which saves the cost of annotating training data. Second, the strong reasoning ability and text-generating ability of LLM show great potential to generate high-quality data.
However, how to use ICL for data annotation remains an open question. For example, ~\citet{annotation} performed a comprehensive analysis and found that generation-based methods are more cost-effective in using GPT-3 than annotating unlabeled data via ICL.

% \paragraph{Model Probing} 
% Apart from ICL, linear probing is another way of black-box tuning~\citep{sun2022black} for LLMs, which learns a linear classifier based on final representations of LLMs and is suitable for full-data settings. \citet{cho2022prompt} propose prompt-augmented linear probing, a hybrid of linear probing and ICL. They train a linear classifier based on representations enhanced with prepend additional demonstrations. The hybrid of ICL and linear probing can cover the weakness of each other and scale the ICL to the full-data setting.

% \paragraph{Retrieval-Augmented Models} 
\paragraph{Model Augmentating} The context-flexible nature of ICL demonstrates significant potential to enhance retrieval-augmented methods. By keeping the LM architecture unchanged and prepending grounding documents to the input, in-context RALM\citet{ram2023context} effectively utilizes off-the-shelf general-purpose retrievers, resulting in substantial LM gains across various model sizes and diverse corpora. Furthermore, ICL for retrieval also exhibits the potential to improve safety. In addition to efficiency and flexibility, ICL also shows potential in safety~\cite{dpicl}, \cite{meade2023using} use ICL for retrieved demonstrations to steer a model towards safer generations, reducing bias and toxicity in the model.

\paragraph{Knowledge Updating}
LLMs may contain outdated or incorrect knowledge, but ICL demonstrates the potential for effectively editing and updating this information. In an initial trial, \citet{reliable} found that GPT-3 updated its answers 85\% of the time when provided with counterfactual examples, with larger models performing better at in-context knowledge updating. However, this approach may impact other correct knowledge in LLMs.
Compared to knowledge editing for fine-tuned models~\cite{editingfact}, ICL has proven effective for lightweight model editing. \citet{reliable} explored the possibility of editing LLMs' memorized knowledge through in-context demonstrations, discovering that a larger model scale and a mix of demonstration examples improved ICL-based knowledge editing success rates. In a comprehensive study, \citet{ike} investigated ICL strategies for editing factual knowledge, finding that well-designed demonstrations enabled competitive success rates compared to gradient-based methods, with significantly fewer side effects. This underlines the potential of ICL for knowledge editing.
% \subsection{Knowledge Augmentation and Updating}
 % ICL presents new issues for enhancing and updating knowledge in LLMs.
% knowledge augmentation
% The knowledge of LLMs is entirely derived from the pretrained corpus. Therefore, LLMs may lack certain knowledge and generate hallucinations during ICL inference, especially towards long-tailed factual knowledge or commonsense knowledge rarely described in texts.
% Therefore, it is essential to augment knowledge for ICL.
% Different from traditional work, which adds knowledge adapters~\cite{kadapter} or provides structured knowledge during pretraining~\cite{zhang_2019_ernie,peters_2019_knowledge},
% retrieving correct knowledge and integrating the correct knowledge with the context in a lightweight manner is possibly promissing for ICL.
% knowledge updating or calibration
% Although LLMs can serve as knowledge bases, 
% The knowledge in LLMs could be wrong or out-of-date, and ICL can be applied to edit or update the knowledge for more factual generations.
% ~\citet{reliable} made an initial trial on in-context knowledge updating. By providing counterfactual examples in the demonstration, they found that GPT-3 updates its answers around 85\% of the time and larger models are better at in-context knowledge updating. 
% However, this approach may affect other correct knowledge in LLMs. Compared with knowledge editing for finetuned models~\cite{editingfact},  
% ICL has shown effectiveness for lightweight model editing. \citet{reliable} analyzed whether it is possible to edit or adapt memorized knowledge in LLMs to new information through in-context demonstrations. They found that a large model scale and a mixture of all types of demonstration examples strengthen the knowledge editing success rate of ICL. In a comprehensive empirical study, \citet{ike} examined ICL strategies for editing factual knowledge and found that well-designed demonstrations enabled in-context knowledge editing to achieve a competitive success rate compared to gradient-based methods, but with significantly fewer side effects. This highlights the potential of ICL for knowledge editing.
% the impact of in-context knowledge updating on paraphrased facts and irrelevant facts has not been rigorously evaluated. 
% However, ICL for knowledge editing still faces challenges such as over-editing or multihop editing, which require further exploration.



\section{Challenges and Future Directions}
\label{sec:challege_future}
% Rosy
In this section, we review some of the existing challenges and propose possible directions for future research on ICL.
\subsection{New Pretraining Strategies}
As investigated by~\citet{corpusscale}, language model objectives are not equal to   ICL abilities. Researchers have proposed to bridge the gap between pretraining objectives and ICL through intermediate tuning before inference (Section~\ref{sec:warmup}), which shows promising performance improvements. To take it further, tailored pretraining objectives and metrics for ICL have the potential to raise LLMs with superior ICl capabilities.
% \subsection{New Evaluation Paradigm}
% \label{challenge: evaluation}
% As ICL is a new paradigm varies from traditional finetuning paradigm in many aspects,
% %such as training data scale and model parameter updating, 
% the evaluation of ICL presents a new set of challenges and opportunities.

% Towards the challenges, the results of existing evaluation methods are sensitive. \citet{chen2022relation} observed that existing evaluations by accuracy underestimate the sensitivity towards instruction perturbation of ICL.
% It is still an open question to conduct consistent ICL evaluation.
% Apart from that, traditional evaluation benchmarks such as SuperGLUE~\cite{superglue} are less challenging for LLMs with billions of parameters, and even the sophisticated designed new benchmark, BigBench~\cite{beyond} can be readily surpassed.
% How to construct a challenging evaluation with long-term vitality is crucial for ICL evaluation.
% %A dangerous phenomenon for ICL evaluation is that pretraining data leakage, that is, the test data is difficult to circumvent data leakage during pretraining.
% %This pose new challenges to trustworthy evaluation on ICL.

% Towards the opportunities for evaluation, as ICL only requires a few instances for the demonstration, it lowers the cost for evaluation data construction. On the one hand, this makes it possible to evaluate on some tasks whose data is tedious to construct or needs experts to annotate. On the other hand, the test set is still supposed to faithfully reflect the world distribution. 
% Instead of the i.i.d. assumption with the training set, the test set requires new constraints.
% \leiModify{But the evaluation set is still supposed to faithfully reflect the world distribution.}


\subsection{ICL Ability Distillation}
Previous studies have shown that in-context learning for reasoning tasks emerges as the scale of computation and parameter exceed a certain threshold~\citep{wei2022emergent}. Transferring the ICL ability to smaller models could facilitate the model deployment greatly.
\citet{distill2smallmodel} showed that it is possible to distill the reasoning ability to small language models such as T5-XXL. The distillation is achieved by finetuning the small model on the chain-of-thought data~\citep{cot} generated by a large teacher model. 
%, with the question as input and the CoT process along with the answer as the target
Although promising performance is achieved, %for example, the accuracy
%of T5-XXL on GSM8K improves from 8.11\%
%to 21.99\%, 
the improvements are likely task-dependent.
Further investigation on improving the reasoning ability by learning from larger LLMs could be an interesting direction.
% \subsection{Instance-level Selection}
% % \leiModify{TODO: ce\&zhiyong}
% Preliminary exploration~\cite{Wu2022SelfadaptiveIL} has shown the enormous promise of instance-level ICL, which has the potential to bridge the performance gap between ICL and finetuning. However, the search for the best demonstration organization at the instance-level is essentially an NP-hard combinatorial optimization problem. To tackle the problem of massive search space, we need specialized yet efficient algorithms (e.g., a learned example retriever) that can quickly rule out large parts of the search space. The designing of such algorithms is difficult since they require sequential decision-making, which makes applying reinforcement learning in the design attractive.
\subsection{ICL Robustness}
Previous studies have shown that ICL performance is extremely unstable, from random guess to SOTA, and can be sensitive to many factors, including demonstration permutation, demonstration format, etc.~\citep{calibrate, lu2022order}. 
The robustness of ICL is a critical yet challenging problem. 

However, most of the existing methods fall into the dilemma of accuracy and robustness~\cite{chen2022sensitivity}, or even at the cost of sacrificing inference efficiency. To effectively improve the robustness of ICL, we need deeper analysis of the working mechanism of the ICL. We believe that the analysis of the robustness of the ICL from a more theoretical perspective rather than an empirical perspective can highlight future research on more robust ICL.

\subsection{ICL Efficiency and Scalability}
ICL necessitates prepending a significant number of demonstrations within the context. However, it presents two challenges: (1) the quantity of demonstrations is constrained by the maximum input length of LMs, which is significantly fewer compared to fine-tuning (scalability); (2) as the number of demonstrations increases, the computation cost becomes higher due to the quadratic complexity of attention mechanism (efficiency). Previous work in \S\ref{sec:demo} focused on exploring how to achieve better ICL performance using a limited number of demonstrations and proposed several demonstration designing strategies. Scaling ICL to more demonstrations and improving its efficiency remains a challenging task.


Recently, some works have been proposed to address the issues of scalability and efficiency of ICL. Efforts were made to optimize prompting strategies with structured prompting~\citep{hao2022structured}, demonstration ensembling~\citep{khalifa2023exploring}, dynamic prompting~\citep{zhou2023efficient}, and iterative forward tuning~\citep{yang2023iterative}. Additionally, \citet{li2023context} proposed EVaLM with longer context length and enhanced long-range language modeling capabilities. This model-level improvement aims to improve the scalability and efficiency of ICL. As LMs continue to scale up, exploring ways to effectively and efficiently utilize a larger number of demonstrations in ICL remains an ongoing area of research.



%To address this problem, \citet{chen2022sensitivity} defined a metric for ICL called prediction sensitivity, which measures the changes in the model output in response to small input perturbations. 
%They found an obvious negative correlation between prediction sensitivity and ICL accuracy. Motivated by this finding, they proposed a selective prediction method based on prediction sensitivity. Unlike this study, \citet{Chang2022CarefulDC} focused on selecting stable demonstrations.
%, which means the ICL performance on different orderings of these demonstrations has a low variance and a high worst-case and average performance. 
%They assigned training examples with scores about stability and proposed two scoring methods. The first scoring combines an example with random training examples and gets the average ICL performance as its score. Another method learns a linear proxy model to estimate how an example influences ICL performance.

%We hope to see more stable and robust ICL methods in the future because stable and robust ICL makes it easier to construct demonstrations without being overly concerned with influencing factors such as order. 
% \subsection{Working Mechanism}
% Understanding the working mechanism of ICL is an inevitable step to trust ICL and further improve it. 
% Although some analytical work has taken a preliminary step to explain ICL, most of them is limited to simple tasks and small models. 
% Extending analysis on extensive tasks and real large models may be the next step to be considered. 
% In addition, among existing work, understanding ICL as a process of meta-optimization seems to be a reasonable and promising direction for future research. 
% If we build clear connections between ICL and meta-optimization, we can improve ICL with a purpose by learning from the history of finetuning and optimization. 

% \subsection{ICL for Data Engineering}
% %Cost and quality have been two conflicting aspects of data engineering. 
% ICL has manifested the potential to be widely applied in data engineering. 
% Benefiting from the strong ICL ability,  it costs 50\% to 96\% less to use labels from GPT-3 than using labels from humans for data annotation. Combining pseudo labels from GPT-3 with human labels leads to even better performance at a small cost~\cite{want}.
% In more complex scenarios, such as knowledge graph construction, \citet{khorashadizadeh2023exploring} has demonstrated that ICL has the potential to significantly improve the state of the art of automatic construction and completion of knowledge graphs, resulting in a reduction in manual costs with minimal engineering effort. Therefore, leveraging the capabilities of ICL in various data engineering applications can yield significant benefits.
% Compared to human annotation (e.g., crowd-sourcing) or noisy automatic annotation (e.g., distant supervision), ICL generates relatively high quality data at a low cost. 
% First, ICL takes only a few examples to learn the data engineering objective, which saves the cost of annotating training data. Second, the strong reasoning ability and text-generating ability of LLM show great potential to generate high-quality data.

% However, how to use ICL for data annotation remains an open question. For example, ~\citet{annotation} performed a comprehensive analysis and found that generation-based methods are more cost-effective in using GPT-3 than annotating unlabeled data via ICL.
% We believe that improving ICL for data annotation is a direction with practical value, and ICL will be a new paradigm in data annotation, data augmentation, data pruning, as well as adversarial data generation.


% \section{Related Concepts}
% \label{sec:related}
% \paragraph{Few-shot Learning}

\paragraph{Fine-tuning}

\paragraph{Prompt Predicting}

\paragraph{Instruction Tuning}

\section{Conclusion}
\label{sec:conclusion}
In this paper, we survey the existing ICL literature and provide an extensive review of advanced ICL techniques, including training strategies, demonstration designing strategies, evaluation datasets and resources, as well as related analytical studies. Furthermore, we highlight critical challenges and potential directions for future research. To the best of our knowledge, this is the first survey about ICL. We hope this survey can highlight the current research status of ICL and shed light on future work on this promising paradigm. 


\bibliography{tacl2021}
\bibliographystyle{acl_natbib}

% \clearpage

% \appendix

% \section*{Appendix}
% \label{sec:appendix}
% \onecolumn


% \tableofcontents{}

% \newpage

\section*{Supplementary Material}
\addcontentsline{toc}{section}{Supplementary Material}


Throughout this discussion, 
we will make frequently use 
of the following standard results
concerning the exponential concentration 
of random variables:

\begin{lemma}[Hoeffding's inequality for independent RVs~\citep{hoeffding1994probability}] Let $Z_1, Z_2, \ldots, Z_n$ be independent bounded random variables with $Z_i \in [a,b]$ for all $i$, then 
    \begin{align*}
        \prob\left( \frac{1}{n} \sum_{i=1}^n (Z_i - \Expo{Z_i}) \ge t \right) \le \exp{\left( -\frac{2nt^2}{(b-a)^2} \right) }
    \end{align*} 
    and 
    \begin{align*}
        \prob\left( \frac{1}{n} \sum_{i=1}^n (Z_i - \Expo{Z_i}) \le -t \right) \le \exp{\left( -\frac{2nt^2}{(b-a)^2} \right) }
    \end{align*} 
    for all $t \ge 0$. 
\end{lemma}

\begin{lemma}[Hoeffding's inequality for sampling with replacement~\citep{hoeffding1994probability}] \label{lem:hoeffding_sampling} Let $\calZ = (Z_1, Z_2, \ldots, Z_N)$ be a finite population of $N$ points with $Z_i \in [a.b]$ for all $i$. Let $X_1, X_2, \ldots X_n$ be a random sample drawn without replacement from $\calZ$. Then for all $t \ge 0$, we have 
    \begin{align*}
        \prob\left( \frac{1}{n} \sum_{i=1}^n (X_i - \mu ) \ge t \right) \le \exp{\left( -\frac{2nt^2}{(b-a)^2} \right) }
    \end{align*} 
    and 
    \begin{align*}
        \prob\left( \frac{1}{n} \sum_{i=1}^n (X_i - \mu ) \le -t \right) \le \exp{\left( -\frac{2nt^2}{(b-a)^2} \right) } \,,
    \end{align*} 
    where $\mu = \frac{1}{N} \sum_{i=1}^{N} Z_i$. 
\end{lemma}

We now discuss one condition that generalizes the exponential concentration to dependent random variables.
\begin{condition}[Bounded difference inequality] \label{cond:BDC} Let $\calZ$ be some set and $\phi: \calZ^n \to \Real$. We say that $\phi$ satisfies the bounded difference assumption if 
there exists $c_1, c_2, \ldots c_n \ge 0$ s.t. for all $i$, we have 
\begin{align*}
    \sup_{Z_1,Z_2, \ldots,Z_n, Z_i^\prime \in \calZ^{n+1} } \abs{\phi (Z_1, \ldots, Z_i, \ldots, Z_n ) - \phi (Z_1, \ldots, Z_i^\prime, \ldots, Z_n ) } \le c_i \,.
\end{align*} 
\end{condition}

\begin{lemma}[McDiarmid’s inequality~\citep{mcdiarmid1989}] \label{lem:McDiarmid} Let $Z_1, Z_2, \ldots, Z_n$ be independent random variables on set $\calZ$ and $\phi : \calZ^n \to \Real$ satisfy bounded difference inequality (\codref{cond:BDC}). Then for all $t>0$, we have 
    \begin{align*}
        \prob\left( \phi(Z_1, Z_2, \ldots, Z_n) - \Expo{\phi(Z_1, Z_2, \ldots, Z_n)} \ge t \right) \le \exp{\left( -\frac{2t^2}{\sum_{i=1}^n c_i^2} \right) } 
    \end{align*} 
    and 
    \begin{align*}
        \prob\left( \phi(Z_1, Z_2, \ldots, Z_n) - \Expo{\phi(Z_1, Z_2, \ldots, Z_n)} \le -t \right) \le \exp{\left( -\frac{2t^2}{\sum_{i=1}^n c_i^2} \right) } \,.
    \end{align*} 
\end{lemma}


\section{Proofs from \secref{sec:ERM_training}}\label{app:proof_erm}

\textbf{Additional notation {} {}} Let $m_1$ be the number of mislabeled points ($\wt S_M$) and $m_2$ be the number of correctly labeled points ($\wt S_C$). Note $m_1 + m_2 = m$. 


\subsection{Proof of \thmref{thm:error_ERM}}


\begin{proof}[Proof of \lemref{lem:fit_mislabeled}] 
    The main idea of our proof is to regard 
    the clean portion of the data 
    ($S \cup \wt S_C$) as fixed.   
    Then, there exists an (unknown) classifier $f^*$ 
    that minimizes the expected risk
    calculated on the (fixed) clean data
    and (random draws of) the mislabeled data $\wt S_M$. 
    % 
    % 
    Formally, 
    \begin{align}
    f^* \defeq \argmin_{f \in \calF} \error_{\widecheck {\calD}} (f) \,, \label{eq:modified_ERM}
    \end{align}
    where $$\widecheck \calD = \frac{n}{m+n} \calS + \frac{m_2}{m+n} \wt \calS_C  + \frac{m_1}{m+n}\calDm \,.$$ 
    Note here that $\widecheck \calD$ is a combination 
    of the \emph{empirical distribution} 
    over correctly labeled data $S \cup \wt S_C$
    and the (population) distribution 
    over mislabeled data $\calDm$.
    Recall that 
    \begin{align}
    \wh f \defeq \argmin_{f \in \calF} \error_{\calS \cup \wt S} (f) \,. \label{eq:orig_ERM}
    \end{align}
    % 
    % 
    Since, $\widehat f$ minimizes 0-1 error 
    on $S \cup \wt S$, using ERM optimality on \eqref{eq:orig_ERM},  
    we have 
    \begin{align}
        \error_{\calS \cup \wt \calS}(\widehat f) \le \error_{
            \calS \cup \wt \calS}(f^*) \,.    \label{eq:step1}
    \end{align}
    Moreover, since $f^*$ is independent of $\wt S_M$, using Hoeffding's bound,
    % \footnote{For a fully rigorous argument,
    % refer to the complete proof in App.~\ref{app:proof_erm}.} 
    we have with probability at least $1-\delta$ that
    \begin{align}
      \error_{\wt \calS_M}(f^*) \le \error_{ \calDm}(f^*) +  \sqrt{\frac{\log(1/\delta)}{2 m_1}} \,. \label{eq:step2} 
    \end{align}
    %$ 
    %for some constant $c_1\le 1/2$. 
    Finally, since $f^*$ is the optimal classifier on $\widecheck \calD$, 
    we have 
    \begin{align}
        \error_{\widecheck \calD}(f^*) \le \error_{\widecheck \calD}(\widehat f) \,. \label{eq:step3}
    \end{align}
    Now to relate \eqref{eq:step1} and \eqref{eq:step3}, we multiply \eqref{eq:step2} by $\frac{m_1}{m+n}$ and add $\frac{n}{m+n} \error_{\calS} (f)  + \frac{m_2}{m+n} \error_{\wt \calS_C} (f)$ both the sides. Hence, 
    we can rewrite \eqref{eq:step2} as follows: 
    \begin{align}
        \error_{\calS \cup \wt\calS}(f^*) \le \error_{ \widecheck \calD}(f^*) +  \frac{m_1}{m+n}\sqrt{\frac{\log(1/\delta)}{2 m_1}} \,. \label{eq:step4} 
    \end{align}
    Now we combine equations \eqref{eq:step1}, \eqref{eq:step4}, and \eqref{eq:step3}, to get 
    \begin{align}
        \error_{\calS \cup \wt \calS}(\wh f) \le \error_{\widecheck \calD}(\wh f) +  \frac{m_1}{m+n}\sqrt{\frac{\log(1/\delta)}{2 m_1}} \,, 
    \end{align}
    which implies 
    \begin{align}
        \error_{ \wt \calS_M}(\wh f) \le \error_{\calDm}(\wh f) + \sqrt{\frac{\log(1/\delta)}{2 m_1}} \,. \label{eq:lemma1_final}
    \end{align}
    Since $\wt S$ is obtained by randomly labeling an unlabeled dataset, we assume $2m_1 \approx m$ \footnote{Formally, with probability at least $1-\delta$, we have  $(m - 2m_1)\le \sqrt{m\log(1/\delta)/2}$.}. Moreover, using $\error_{\calDm} = 1 - \error_{\calD}$ we obtain the desired result.   
    % Combining the above steps and using the fact 
    % that $\error_\calD = 1- \error_{\calDm} $, 
    % we obtain the desired result.
\end{proof}

\begin{proof}[Proof of \lemref{lem:mislabeled_error}]
    Recall $\error_{\wt S} (f) = \frac{m_1}{m} \error_{\wt S_M}(f) + \frac{m_2}{m} \error_{\wt S_C}(f)$. Hence, we have 
    \begin{align}
        2\error_{\wt S}(f) - \error_{\wt S_M}(f) - \error_{\wt S_C}(f) &= \left(\frac{2m_1}{m} \error_{\wt S_M}(f) - \error_{\wt S_M}(f)\right) + \left(\frac{2m_2}{m} \error_{\wt S_C}(f) - \error_{\wt S_C}(f)\right) \\ &= \left(\frac{2m_1}{m} - 1\right) \error_{\wt S_M}(f) + \left(\frac{2m_2}{m} - 1 \right)\error_{\wt S_C} (f) \,.
    \end{align} 
    Since the dataset is labeled uniformly at random, with probability at least $1-\delta$, we have  $\left(\frac{2m_1}{m} - 1\right) \le \sqrt{\frac{\log(1/\delta)}{2m}}$. Similarly, we have with probability at least $1-\delta$, $\left(\frac{2m_2}{m} - 1\right) \le \sqrt{\frac{\log(1/\delta)}{2m}}$. Using union bound, with probability at least $1-\delta$, we have
    % \begin{align}
    %     2\error_{\wt S} - \error_{\wt S_M}(f) - \error_{\wt S_C}(f) \le \sqrt{\frac{\log(2/\delta)}{2m}} \left(\error_{\wt S_M}(f) + \error_{\wt S_C}(f) \right) \le 2\sqrt{\frac{\log(2/\delta)}{2m}} \,. \label{eq:lemma2_final}
    % \end{align}
    \begin{align}
        2\error_{\wt S} - \error_{\wt S_M}(f) - \error_{\wt S_C}(f) \le \sqrt{\frac{\log(2/\delta)}{2m}} \left(\error_{\wt S_M}(f) + \error_{\wt S_C}(f) \right) \,. \label{eq:lemma2_prefinal}
    \end{align}
    With re-arranging $\error_{\wt S_M}(f) + \error_{\wt S_C}(f)$ and using the inequality $ 1- a\le \frac{1}{1+a} $, we have  
    \begin{align}
        2\error_{\wt S} - \error_{\wt S_M}(f) - \error_{\wt S_C}(f) \le 2\error_{\wt \calS} \sqrt{\frac{\log(2/\delta)}{2m}}  \,. \label{eq:lemma2_final}
    \end{align}

    % We obtain the desired result by using 
\end{proof}

\begin{proof}[Proof of \lemref{lem:clear_error}]
% Recall 0-1 error on each point  $(x,y) \in S \cup \wt S$ is given by $\I{ f(x)\ne y}$.
In the set of correctly labeled points $S \cup \wt S_C$, we have $S$ as a random subset of $S \cup \wt S_C$. Hence, using Hoeffding's inequality for sampling without replacement (\lemref{lem:hoeffding_sampling}), we have with probability at least $1-\delta$
\begin{align}
    \error_{\wt \calS_C} (\wh f)- \error_{\calS \cup \wt \calS_C}( \wh f) \le  \sqrt{\frac{\log(1/\delta)}{2m_2}} \,.
\end{align}
Re-writing $\error_{\calS \cup \wt \calS_C}( \wh f)$ as $\frac{m_2}{m_2 + n} \error_{\wt \calS_C }(\wh f) + \frac{n}{m_2 + n} \error_{\calS }(\wh f)$, we have with probability at least $1-\delta$
\begin{align}
   \left(\frac{n}{n+m_2}\right) \left(\error_{\wt \calS_C} (\wh f)- \error_{\calS}( \wh f) \right) \le  \sqrt{\frac{\log(1/\delta)}{2m_2}} \,.
\end{align}
As before, assuming $2m_2 \approx m$, we have with probability at least $1-\delta$ 
\begin{align}
    \error_{\wt \calS_C} (\wh f)- \error_{\calS}( \wh f) \le \left(1+\frac{m_2}{n}\right)  \sqrt{\frac{\log(1/\delta)}{m}} \le \left(1 + \frac{m}{2n}\right) \sqrt{\frac{\log(1/\delta)}{m}} \,. \label{eq:lemma3_final}
\end{align} 
\end{proof}

\begin{proof}[Proof of \thmref{thm:error_ERM}] 
    Having established these core intermediate results, we can now combine above three lemmas to prove the main result. 
    In particular, we bound the population error on clean data ($\error_\calD(\wh f)$) as follows:  
    \begin{enumerate}[(i)]
        \item First, use \eqref{eq:lemma1_final}, to obtain an upper bound on the population error on clean data, i.e., with probability at least $1-\delta/4$, we have
        \begin{align}
            \error_{ \calD} (\wh f) \le 1 - \error_{ \wt \calS_M}(\wh f) + \sqrt{\frac{\log(4/\delta)}{m}} \,. 
        \end{align}
        \item  Second, use \eqref{eq:lemma2_final}, to relate the error on the mislabeled fraction with error on clean portion of randomly labeled data and error on whole randomly labeled dataset, i.e., with probability at least $1-\delta/2$, we have 
        \begin{align}
            - \error_{\wt S_M}(f) \le \error_{\wt S_C}(f) - 2\error_{\wt S}  + 2\error_{\wt S} \sqrt{\frac{\log(4/\delta)}{2m}}  \,. 
        \end{align} 
        \item Finally, use \eqref{eq:lemma3_final} to relate the error on the clean portion of randomly labeled data and error on clean training data, i.e., with probability $1-\delta/4$, we have 
        \begin{align}
            \error_{\wt \calS_C} (\wh f)\le - \error_{\calS}( \wh f) + \left(1 + \frac{m}{2n} \right) \sqrt{\frac{\log(4/\delta)}{m}} \,. 
        \end{align} 
    \end{enumerate}

    Using union bound on the above three steps, we have with probability at least $1-\delta$: 
    \begin{align}
        \error_\calD (\wh f) \le \error_{\calS}(\wh f)   + 1 - 2\error_{\wt \calS}(\wh f)   + \left(\sqrt{2} \error_{\wt S} + 2 + \frac{m}{2n}\right)  \sqrt{\frac{\log(4/\delta)}{m}} \,.
    \end{align}
    % Note that $(1/\sqrt{2} + 2.5)$ is a loose constant. In experiments, we use the ratio $\frac{m}{n}$
    %  the exact error $\error_{\wt \calS}(\wh f)$ 
    % to evaluate R.H.S.    
\end{proof}

\subsection{Proof of \propref{prop:rademacher}}

\begin{proof}[Proof of \propref{prop:rademacher}]
    For a classifier $ f: \calX \to \{-1, 1\}$, we have $1 - 2\,\indict{ f(x) \ne y} = y \cdot f(x)$. Hence, by definition of $\error$, we have 
    \begin{align}
        1 -2\error_{\wt \calS}(f) = \frac{1}{m}\sum_{i=1}^m y_i \cdot f(x_i) \le \sup_{f \in \calF} \, \frac{1}{m} \sum_{i=1}^m y_i \cdot f(x_i)  \,. \label{eq:error_rademacher}
    \end{align}
    Note that for fixed inputs $(x_1, x_2, \ldots, x_m)$ in $\wt S$, $(y_1, y_2, \ldots y_m)$ are random labels. Define $\phi_1 (y_1, y_2, \ldots, y_m) \defeq \sup_{f \in \calF} \, \frac{1}{m} \sum_{i=1}^m y_i \cdot f(x_i)$. We have the following bounded difference condition on $\phi_1$. For all i, 
    \begin{align}
        \sup_{y_1, \ldots y_m, y_i^\prime \in \{-1, 1\}^{m+1} } \abs{ \phi_1 (y_1,\ldots, y_i, \ldots, y_m) - \phi_1 (y_1,\ldots, y_i^\prime, \ldots, y_m)  } \le 1/m \,. \label{cond1_rademacher}
    \end{align} 
    
    Similarly, we define $\phi_2 (x_1, x_2, \ldots, x_m) \defeq \Expt{ y_i \sim_U \{-1, 1\}  }{ \sup_{f \in \calF} \, \frac{1}{m}  \sum_{i=1}^m y_i \cdot f(x_i)}$. We have the following bounded difference condition on $\phi_2$. 
    For all i,
    \begin{align}
        \sup_{x_1, \ldots x_m, x_i^\prime \in \calX^{m+1} } \abs{ \phi_2 (x_1,\ldots, x_i, \ldots, x_m) - \phi_1 (x_1,\ldots, x_i^\prime, \ldots, x_m)  } \le 1/m \,. \label{cond2_rademacher}
    \end{align}
    Using McDiarmid’s inequality (\lemref{lem:McDiarmid}) twice 
    with Condition \eqref{cond1_rademacher} and \eqref{cond2_rademacher}, 
    with probability at least $1-\delta$, we have
    \begin{align}
        \sup_{f \in \calF} \, \frac{1}{m} \sum_{i=1}^m y_i \cdot f(x_i)  - \Expt{x,y}{\sup_{f \in \calF} \, \frac{1}{m} \sum_{i=1}^m y_i \cdot f(x_i) } \le \sqrt{\frac{2\log(2/\delta)}{m}} \,. \label{eq:final_rademacher}
    \end{align} 
    Combining \eqref{eq:error_rademacher} and \eqref{eq:final_rademacher}, we obtain the desired result. 
\end{proof}


\subsection{Proof of \thmref{thm:error_regularized_ERM}}

Proof of \thmref{thm:error_regularized_ERM} follows similar to the proof of \thmref{thm:error_ERM}. Note that the same results in \lemref{lem:fit_mislabeled}, \lemref{lem:mislabeled_error}, and \lemref{lem:clear_error} hold in the regularized ERM case. However, the arguments in the proof of \lemref{lem:fit_mislabeled} change slightly. Hence, we state the lemma for regularized ERM and prove it here for completeness. 

\begin{lemma} \label{lem:lemma1_reg}
    Assume the same setup as \thmref{thm:error_regularized_ERM}. 
    Then for any $\delta >0$, with probability at least  $1-\delta$ 
    over the random draws of mislabeled data $\wt S_M$, we have 
    \begin{align}
        \error_\calD(\widehat f)  \le 1 -\error_{\wt \calS_M}(\widehat f) + \sqrt{\frac{\log(1/\delta)}{m}}\,. 
    \end{align} 
\end{lemma}
\begin{proof}
    The main idea of the proof remains the same, i.e. regard 
    the clean portion of the data 
    ($S \cup \wt S_C$) as fixed.   
    Then, there exists a classifier $f^*$ 
    that is optimal over draws 
    of the mislabeled data $\wt S_M$. 

    
    Formally, 
    \begin{align}
    f^* \defeq \argmin_{f \in \calF} \error_{\widecheck {\calD}} (f)  + \lambda R(f) \,, \label{eq:modified_ERM_reg}
    \end{align}
    where $$\widecheck \calD = \frac{n}{m+n} \calS + \frac{m_1}{m+n} \wt \calS_C  + \frac{m_2}{m+n}\calDm \,.$$ That is, $\widecheck \calD$ a combination of 
    the \emph{empirical distribution} 
    over correctly labeled data $S \cup \wt S_C$
    % in $S\cup \wt S$ 
    and the (population) distribution 
    over mislabeled data $\calDm$.
    Recall that 
    \begin{align}
    \wh f \defeq \argmin_{f \in \calF} \error_{\calS \cup \wt S} (f) + \lambda R(f) \,. \label{eq:orig_ERM_reg}
    \end{align}
    % 
    % 
    Since, $\widehat f$ minimizes 0-1 error 
    on $S \cup \wt S$, using ERM optimality on \eqref{eq:orig_ERM},  
    we have 
    \begin{align}
        \error_{\calS \cup \wt \calS}(\widehat f) + \lambda R(\wh f) \le \error_{
            \calS \cup \wt \calS}(f^*) + \lambda R(f^*) \,.    \label{eq:step1_reg}
    \end{align}
    Moreover, since $f^*$ is independent of $\wt S_M$, using Hoeffding's bound,
    % \footnote{For a fully rigorous argument,
    % refer to the complete proof in App.~\ref{app:proof_erm}.} 
    we have with probability at least $1-\delta$ that
    \begin{align}
      \error_{\wt \calS_M}(f^*) \le \error_{ \calDm}(f^*) +  \sqrt{\frac{\log(1/\delta)}{2 m_1}} \,. \label{eq:step2_reg} 
    \end{align}
    %$ 
    %for some constant $c_1\le 1/2$. 
    Finally, since $f^*$ is the optimal classifier on $\widecheck \calD$, 
    we have 
    \begin{align}
        \error_{\widecheck \calD}(f^*) + \lambda R(f^*) \le \error_{\widecheck \calD}(\widehat f) + \lambda R(\wh f) \,. \label{eq:step3_reg}
    \end{align}
     Now to relate \eqref{eq:step1_reg} and \eqref{eq:step3_reg}, we can re-write the \eqref{eq:step2_reg} as follows: 
    \begin{align}
        \error_{\calS \cup \wt\calS}(f^*) \le \error_{ \widecheck \calD}(f^*) +  \frac{m_1}{m+n}\sqrt{\frac{\log(1/\delta)}{2 m_1}} \,. \label{eq:step4_reg} 
    \end{align}
    After adding $\lambda R(f^*)$ on both sides in \eqref{eq:step4_reg}, we combine equations \eqref{eq:step1_reg}, \eqref{eq:step4_reg}, and \eqref{eq:step3_reg}, to get 
    \begin{align}
        \error_{\calS \cup \wt \calS}(\wh f) \le \error_{\widecheck \calD}(\wh f) +  \frac{m_1}{m+n}\sqrt{\frac{\log(1/\delta)}{2 m_1}} \,, 
    \end{align}
    which implies 
    \begin{align}
        \error_{ \wt \calS_M}(\wh f) \le \error_{\calDm}(\wh f) + \sqrt{\frac{\log(1/\delta)}{2 m_1}} \,. \label{eq:lemma_reg_final}
    \end{align}
    Similar as before, since $\wt S$ is obtained by randomly labeling an unlabeled dataset, we assume 
    $2m_1 \approx m$. Moreover, using $\error_{\calDm} = 1 - \error_{\calD}$ we obtain the desired result. 
\end{proof}
% \begin{proof}[Proof of ]
    
% \end{proof}

\subsection{Proof of \thmref{thm:multiclass_ERM}}

To prove our results in the multiclass case,
we first state and prove lemmas
parallel to those
% We first state and prove lemmas 
% parallel 
% to the three lemmas 
used in the proof of balanced binary case. 
We then combine these results 
% in the three lemmas 
to obtain the result in \thmref{thm:multiclass_ERM}. 

Before stating the result, 
we define mislabeled distribution $\calDm$ for any $\calD$.
While $\calDm$ and $\calD$ share 
the same marginal distribution over inputs $\calX$,
the conditional distribution over labels $y$ 
given an input $x\sim \calD_\calX$ is changed as follows:
For any $x$, the Probability Mass Function (PMF) over $y$ is defined as:  
$p_{\calDm} (\cdot \vert x) \defeq \frac{1 - p_{\calD}(\cdot \vert x)}{k - 1}$, where $ p_{\calD}(\cdot \vert x)$ is the PMF over $y$ for the distribution $\calD$. 

\begin{lemma} \label{lem:fit_mislabeled_multi}
    Assume the same setup as \thmref{thm:multiclass_ERM}. 
    Then for any $\delta >0$, with probability at least  $1-\delta$ 
    over the random draws of mislabeled data $\wt S_M$, we have 
    \begin{align}
        \error_\calD(\widehat f)  \le (k-1)\left(1 -\error_{\wt \calS_M}(\widehat f)\right) + (k-1)\sqrt{\frac{\log(1/\delta)}{m}}\,. \label{eq:lemma1_multi}
    \end{align}   
\end{lemma} 

\begin{proof}
   
    The main idea of the proof remains the same.
    We begin by regarding the clean portion of the data 
    ($S \cup \wt S_C$) as fixed. 
    Then, there exists a classifier $f^*$ 
    that is optimal over draws 
    of the mislabeled data $\wt S_M$. 
    
    However, in the multiclass case,
    we cannot as easily relate the population error on mislabeled data 
    to the population accuracy on clean data.   
    While for binary classification, 
    % we could upper bound $\error_{\wt \calS_M}$ 
    % with $1-\error_\calD$ 
    we could lower bound the population accuracy $1-\error_\calD$
    with the empirical error on mislabeled data $\error_{\wt \calS_M}$ 
    (in the proof of \lemref{lem:fit_mislabeled}), 
    for multiclass classification, 
    error on the mislabeled data 
    and accuracy on the clean data 
    in the population 
    are not so directly related.  
    To establish \eqref{eq:lemma1_multi},
    we break the error on the 
    (unknown) mislabeled data 
    into two parts: one term corresponds 
    to predicting the true label on mislabeled data, 
    and the other corresponds to predicting 
    neither the true label 
    nor the assigned (mis-)label.  
    Finally, we relate these errors to their
    population counterparts to establish \eqref{eq:lemma1_multi}. 
    
    Formally, 
    \begin{align}
    f^* \defeq \argmin_{f \in \calF} \error_{\widecheck {\calD}} (f)  + \lambda R(f) \,, \label{eq:modified_ERM_reg2}
    \end{align}
    where $$\widecheck \calD = \frac{n}{m+n} \calS + \frac{m_1}{m+n} \wt \calS_C  + \frac{m_2}{m+n}\calDm \,.$$ 
    That is, $\widecheck \calD$ is a combination 
    of the \emph{empirical distribution} 
    over correctly labeled data $S \cup \wt S_C$
    % in $S\cup \wt S$ 
    and the (population) distribution 
    over mislabeled data $\calDm$.
    Recall that 
    \begin{align}
    \wh f \defeq \argmin_{f \in \calF} \error_{\calS \cup \wt S} (f) + \lambda R(f) \,. \label{eq:orig_ERM_reg2}
    \end{align}
    % 
    % 
    Following the exact steps from the proof of \lemref{lem:lemma1_reg}, 
    with probability at least $1-\delta$, we have  
    \begin{align}
        \error_{ \wt \calS_M}(\wh f) \le \error_{\calDm}(\wh f) + \sqrt{\frac{\log(1/\delta)}{2 m_1}} \,. \label{eq:lemma1_final_multi_prev}
    \end{align}
    Similar to before, since $\wt S$ is obtained 
    by randomly labeling an unlabeled dataset, 
    we assume 
    $\frac{k}{k-1} m_1 \approx m$. 
    
    Now we will relate $\error_{\calDm} (\wh f)$ with $\error_{\calD}(\wh f)$. 
    Let $y^T$ denote the (unknown) true label 
    for a mislabeled point $(x, y)$ 
    (i.e., label before replacing it with a mislabel). 
    \begin{align*}    
         \Expt{(x, y) \in \sim \calDm}{\indict{ \wh f(x) \ne y }}  &= \underbrace{\Expt{(x, y) \in \sim \calDm}{\indict{ \wh f(x) \ne y \land \wh f(x) \ne y^T}}}_{\RN{1}} \\ &\qquad \qquad + \underbrace{\Expt{(x, y) \in \sim \calDm}{\indict{ \wh f(x) \ne y \land \wh f(x) = y^T}}}_{\RN{2}} \,. \numberthis \label{eq:excess_term}
    \end{align*}
    Clearly, term 2 is one minus the accuracy 
    on the clean unseen data, i.e.,
    \begin{align}
        \RN{2} = 1 - \Expt{{x,y} \sim \calD}{ \indict{ \wh f(x) \ne y}} = 1- \error_{\calD}(\wh f) \,. \label{eq:term1}    
    \end{align}
    Next, we relate term 1 with the error on the unseen clean data. 
    We show that term 1 is equal to the error on the unseen clean data 
    scaled by $\frac{k-2}{k-1}$,
    where $k$ is the number of labels.
    Using the definition of mislabeled distribution $\calDm$,  
    we have 
    \begin{align}
        \RN{1} = \frac{1}{k-1} \left( \Expt{(x, y) \in \sim \calD}{ \sum_{i \in \calY \land i\ne y}  \indict{ \wh f(x) \ne i \land \wh f(x) \ne y}} \right) = \frac{k-2}{k-1} \error_{\calD}(\wh f) \,.\label{eq:term2}
    \end{align}    

    Combining the result in \eqref{eq:term1}, \eqref{eq:term2} and \eqref{eq:excess_term}, we have 
    \begin{align}
        \error_{\calDm}(\wh f) = 1- \frac{1}{k-1} \error_{\calD}(\wh f) \,.\label{eq:combine_terms}
    \end{align}
    Finally, combining the result in \eqref{eq:combine_terms} 
    with equation \eqref{eq:lemma1_final_multi_prev}, 
    we have with probability $1-\delta$, 
    \begin{align}
      \error_{\calD}(\wh f) \le  (k-1) \left( 1- \error_{ \wt \calS_M}(\wh f) \right)  + (k-1) \sqrt{\frac{k \log(1/\delta)}{ 2(k-1)m}} \,. \label{eq:lemma1_final_multi}
    \end{align}
\end{proof}

\begin{lemma} \label{lem:mislabeled_error_multi}
    Assume the same setup as \thmref{thm:multiclass_ERM}. 
    Then for any $\delta >0$, 
    with probability at least $1-\delta$ 
    over the random draws of $\wt S$, we have  
    % \begin{align}
        $$\abs{k\error_{\wt \calS}(\widehat f) - \error_{\wt \calS_C}(\widehat f) -  (k-1)\error_{\wt \calS_M}(\widehat f) } \le  2k\sqrt{\frac{\log(4/\delta)}{2m}}\,. $$ % \label{eq:lemma2}
    % \end{align}   
    %  for some constant $c_3 \le 1.0\,$.
\end{lemma} 


\begin{proof}
    Recall $\error_{\wt S} (f) = \frac{m_1}{m} \error_{\wt S_M}(f) + \frac{m_2}{m} \error_{\wt S_C}(f)$. Hence, we have 
    \begin{align*}
        k\error_{\wt S}(f) - (k-1)\error_{\wt S_M}(f) - \error_{\wt S_C}(f) &= (k-1)\left(\frac{k m_1}{(k-1) m} \error_{\wt S_M}(f) - \error_{\wt S_M}(f)\right) \\ & \qquad \qquad + \left(\frac{km_2}{m} \error_{\wt S_C}(f) - \error_{\wt S_C}(f)\right) \\ &= k \left[ \left(\frac{m_1}{m} - \frac{k-1}{k}\right) \error_{\wt S_M}(f) + \left(\frac{m_2}{m} - \frac{1}{k} \right) \error_{\wt S_C} (f) \right] \,.
    \end{align*} 
    Since the dataset is randomly labeled, 
    we have with probability at least $1-\delta$, 
    $\left(\frac{m_1}{m} - \frac{k-1}{k}\right) \le \sqrt{\frac{\log(1/\delta)}{2m}}$. 
    Similarly, we have with probability at least $1-\delta$, 
    $\left(\frac{m_2}{m} - \frac{1}{k}\right) \le \sqrt{\frac{\log(1/\delta)}{2m}}$. 
    Using union bound, we have with probability at least $1-\delta$
    % \begin{align}
    %     2\error_{\wt S} - \error_{\wt S_M}(f) - \error_{\wt S_C}(f) \le \sqrt{\frac{\log(2/\delta)}{2m}} \left(\error_{\wt S_M}(f) + \error_{\wt S_C}(f) \right) \le 2\sqrt{\frac{\log(2/\delta)}{2m}} \,. \label{eq:lemma2_final}
    % \end{align}
    \begin{align}
        k\error_{\wt S}(f) - (k-1)\error_{\wt S_M}(f) - \error_{\wt S_C}(f)  \le k \sqrt{\frac{\log(2/\delta)}{2m}} \left(\error_{\wt S_M}(f) + \error_{\wt S_C}(f) \right) \,. \label{eq:lemma2_final_multi}
    \end{align}

    % We obtain the desired result by using 
\end{proof}

\begin{lemma} \label{lem:clear_error_multi}
    Assume the same setup as \thmref{thm:multiclass_ERM}. 
    Then for any $\delta >0$, with probability at least $1-\delta$ 
    over the random draws of $\wt S_C$ and $S$, we have 
    % \begin{align}
        $$\abs{\error_{\wt \calS_C}(\widehat f) - \error_{\calS}(\widehat f) } \le 1.5 \sqrt{\frac{k\log(2/\delta)}{2m}}\,.$$ %\label{eq:lemma3}
    % \end{align}   
    % for some constant $c_2 \le 1.2\,$.
\end{lemma} 
\begin{proof}
    % Recall 0-1 error on each point  $(x,y) \in S \cup \wt S$ is given by $\I{ f(x)\ne y}$.
    In the set of correctly labeled points $S \cup \wt S_C$,
    we have $S$ as a random subset of $S \cup \wt S_C$. 
    Hence, using Hoeffding's inequality 
    for sampling without replacement 
    (\lemref{lem:hoeffding_sampling}), 
    we have with probability at least $1-\delta$
    \begin{align}
        \error_{\wt \calS_c} (\wh f)- \error_{\calS \cup \wt \calS_C}( \wh f) \le  \sqrt{\frac{\log(1/\delta)}{2m_2}} \,.
    \end{align}
    Re-writing $\error_{\calS \cup \wt \calS_C}( \wh f)$ 
    as $\frac{m_2}{m_2 + n} \error_{\wt \calS_C }(\wh f) + \frac{n}{m_2 + n} \error_{\calS }(\wh f)$, 
    we have with probability at least $1-\delta$
    \begin{align}
       \left(\frac{n}{n+m_2}\right) \left(\error_{\wt \calS_c} (\wh f)- \error_{\calS}( \wh f) \right) \le  \sqrt{\frac{\log(1/\delta)}{2m_2}} \,.
    \end{align}
    As before, assuming $km_2 \approx m$, 
    we have with probability at least $1-\delta$ 
    \begin{align}
        \error_{\wt \calS_c} (\wh f)- \error_{\calS}( \wh f) \le \left(1+\frac{m_2}{n}\right)  \sqrt{\frac{k\log(1/\delta)}{2m}} \le \left( 1 + \frac{1}{k}\right) \sqrt{\frac{k\log(1/\delta)}{2m}} \,. \label{eq:lemma3_final_multi}
    \end{align} 
\end{proof}

\begin{proof}[Proof of \thmref{thm:multiclass_ERM}] 
    Having established these core intermediate results, 
    we can now combine above three lemmas. 
    In particular, we bound the population error 
    on clean data ($\error_\calD(\wh f)$) as follows:  
    \begin{enumerate}[(i)]
        \item First, use \eqref{eq:lemma1_final_multi}, 
        to obtain an upper bound on the population error on clean data, 
        i.e., with probability at least $1-\delta/4$, we have
        \begin{align}
            \error_{ \calD} (\wh f) \le (k-1)\left(1 - \error_{ \wt \calS_M}(\wh f) \right) + (k-1) \sqrt{\frac{k\log(4/\delta)}{2(k-1)m}} \,. 
        \end{align}
        \item  Second, use \eqref{eq:lemma2_final_multi}
        to relate the error on the mislabeled fraction 
        with error on clean portion of randomly labeled data 
        and error on whole randomly labeled dataset, 
        i.e., with probability at least $1-\delta/2$, we have 
        \begin{align}
            - (k-1)\error_{\wt S_M}(f) \le \error_{\wt S_C}(f) - k\error_{\wt S}  + k\sqrt{\frac{\log(4/\delta)}{2m}}  \,. 
        \end{align} 
        \item Finally, use \eqref{eq:lemma3_final_multi} 
        to relate the error on the clean portion of randomly labeled data 
        and error on clean training data, 
        i.e., with probability $1-\delta/4$, we have 
        \begin{align}
            \error_{\wt \calS_C} (\wh f)\le - \error_{\calS}( \wh f) + \left(1 + \frac{m}{kn} \right) \sqrt{\frac{k\log(4/\delta)}{2m}} \,. 
        \end{align} 
    \end{enumerate}

    Using union bound on the above three steps, 
    we have with probability at least $1-\delta$: 
    \begin{align}
        \error_\calD (\wh f) \le \error_{\calS}(\wh f) + (k-1) - k\error_{\wt \calS}(\wh f)   + (\sqrt{k(k-1)} + k + \sqrt{k} + \frac{m}{n\sqrt{k}})  \sqrt{\frac{\log(4/\delta)}{2m}} \,.\label{eq:multiclass_ERM_final}
    \end{align}
    Simplifying the term in RHS of \eqref{eq:multiclass_ERM_final}, 
    we get the desired result. 
    % Note that since $\frac{m}{n\sqrt{k}}$ 
    % is much smaller than the sum of the other terms
    % the other terms in summation, 
    % we ignore $\frac{m}{n\sqrt{k}}$  
    % Z: ??? --- great
    % that 
    % them
    in the final bound. 
    % we ignore that in the final bound. 
    % Note that $(1/\sqrt{2} + 2.5)$ is a loose constant. In experiments, we use the ratio $\frac{m}{n}$
    %  the exact error $\error_{\wt \calS}(\wh f)$ 
    % to evaluate R.H.S.    
\end{proof}

\newpage
\section{Proofs from \secref{sec:linear_models}}\label{app:proof_gd}
We suppose that the parameters of the linear function 
are obtained via gradient descent on 
the following $L_2$ regularized problem: 
\begin{align}
    % n in denominator is avoided deliberately
    \calL_S(w; \lambda) \defeq \sum_{i=1}^n{(w^Tx_i - y_i)^2} + \lambda \norm{w}{2}^2 \,, \label{eq:l2_MSE_app}   
\end{align}
where $\lambda\ge0$ is a regularization parameter. 
We assume access to a clean dataset 
$S = \{(x_i, y_i)\}_{i=1}^n \sim \calD^n$ 
and randomly labeled dataset 
$\wt S = \{(x_i, y_i)\}_{i=n+1}^{n+m} \sim \wt \calD^m$. 
Let $\bX = [x_1, x_2, \cdots, x_{m+n}]$ 
and $\by = [y_1, y_2, \cdots, y_{m+n}]$. 
Fix a positive learning rate $\eta$ such that 
$\eta \le 1/\left(\norm{\bX^T\bX}{\text{op}} + \lambda^2\right)$ 
and an initialization $w_0 = 0$. 
% \todos{Assumption made for simplicty}. 
Consider the following gradient descent iterates 
to minimize objective \eqref{eq:l2_MSE_app} on $S \cup \wt S$:
\begin{align}
w_t = w_{t-1} - \eta \grad_w \calL_{S \cup \wt S} (w_{t-1}; \lambda) \quad \forall t=1,2,\ldots \label{eq:GD_iterates_app}
\end{align} 
Then we have $\{ w_t\}$ converge to the limiting solution 
$\wh w = \left( \bX^T\bX+\lambda \boldsymbol{I}\right)^{-1}\bX^T\by$. Define $\widehat f (x) \defeq f(x ; \wh w) $.  

% \subsection{\textcolor{red}{Errata}}

% We wish to correct the following error in the body:
% \codref{cond:error_stability} is not enough 
% to guarantee the result in \thmref{thm:linear}. 
% We now present a slightly stronger condition 
% called \emph{hypothesis stability} 
% under which we obtain a result 
% similar to \thmref{thm:linear}. 

% This error doesn't change the main arguments of the proof,
% where we show that the empirical train error 
% is less than or equal to the leave-one-out error.
% We need a stronger condition to relate leave-one-out error 
% with the population error of the original classifier. 
% Specifically, while \codref{cond:error_stability} 
% relates the average population error of leave-one-out classifiers 
% with the population error of the original classifier, 
% we need the new condition to show the concentration 
% of the empirical leave-one-out error 
% and average population error of leave-one-out classifiers. 
% main takeaway 

% Note that the new condition, 
% while being stronger than the previous one, 
% still doesn't imply generalization \citep{bousquet2002stability,elisseeff2003leave,abou2019exponential}. 
% Overall, the main results in \secref{sec:ERM_training} 
% and takeaways of the paper remain unaffected by the error.  

% We now present the new condition 
% and a corrected statement of \thmref{thm:linear}. 
% Recall, for a given training set $S \sim \calD^n $, 
% we use $S_{(i)}$ to denote the training set $S$ 
% with the $i^{\text{th}}$ point removed.

% \begin{condition}[Hypothesis Stability] 
%     \label{cond:hypothesis_stability}
%     We have $\beta$ hypothesis stability 
%     if our training algorithm $\calA$ satisfies the following: 
%     \begin{align*}
%     % ${\sum_{i=1}^n \frac{\error_{\calD}( f(\calA, S_{(i)}))}{n} - \error_\calD(f(\calA, S))} \le \beta\,$.
%     \forall i \in \{1,2,\ldots, n\}, \quad  \Expt{\calS, (x,y) \in \calD}{ \abs{\error\left( f(x) ,y  \right) - \error\left( f_{(i)}(x), y \right) }} \le \frac{\beta}{n} \,,
%     \end{align*}
%     where $f_{(i)} \defeq f(\calA, S_{(i)})$ and $ f \defeq f(\calA, S)$.
% \end{condition}

% \begin{theorem}[Correct statement of \thmref{thm:linear}] \label{thm:new_linear}
%     Assume that this gradient descent algorithm satisfies \codref{cond:hypothesis_stability}
%     with $\beta=\calO(1)$.  
%     Then for any $\delta >0$, with probability at least $1-\delta$ 
%     over the random draws of datasets $\wt S$ and $S$, we have:
%     \begin{align}
%         \error_\calD(\widehat f) \le \error_\calS(\widehat f) + 1 - 2 \error_{\wt\calS}(\widehat f) + \left(\frac{1}{\sqrt{2}} + 1.5 \right) \sqrt{\frac{\log(4/\delta)}{m}} + \sqrt{\frac{4}{\delta}\left(\frac{1}{m} +\frac{3\beta}{m+n} \right)}  \,. \label{eq:gd_error}
%     \end{align} 
%     % for some constant $c\le 3.2$.
% \end{theorem}

\subsection{Proof of \thmref{thm:linear}}
We use a standard result from linear algebra, 
namely the Shermann-Morrison formula 
\citep{sherman1950adjustment} for matrix inversion:  

\begin{lemma}[\citet{sherman1950adjustment}] \label{lem:sherman}
    Suppose $\bA \in \Real^{n \times n}$ 
    is an invertible square matrix 
    and $u,v \in \Real^n$ are column vectors. 
    Then $\bA + uv^T$ is invertible iff $1 + v^T \bA u \ne 0$ 
    and in particular
    \begin{align}
        (\bA + u v^T)^{-1} = \bA^{-1}  - \frac{\bA^{-1} uv^T \bA^{-1} }{ 1 + v^T \bA^{-1} u} \,.
    \end{align}   
\end{lemma}
\newcommand\byy[1]{\by_{\left(#1\right)}}
\newcommand\bXX[1]{\bX_{\left(#1\right)}}
\newcommand\ff[1]{\wh f_{\left(#1\right)}}

For a given training set $S \cup \wt S_C$, 
define leave-one-out error 
on mislabeled points in the training data 
as $$\error_{\text{LOO}(\wt S_M) } = \frac{\sum_{(x_i, y_i) \in \wt S_M} \error( f_{(i)}( x_i), y_i)}{ \abs{\wt S_M }} \,, $$
where $f_{(i)} \defeq f(\calA, (S \cup \wt S)_{(i)})$. 
To relate empirical leave-one-out error and population error 
with hypothesis stability condition, 
we use the following lemma:   

\begin{lemma}[\citet{bousquet2002stability}] \label{lem:stability_error}
    For the leave-one-out error, we have
    \begin{align}
        \Expo{ \left( \error_{\calDm}(\wh f) -\error_{\text{LOO}(\wt S_M) } \right)^2 } \le \frac{1}{2m_1}+  \frac{3\beta}{n + m}\,.
    \end{align}   
    % where $ f \defeq f(\calA, S \cup \wt S) $.
\end{lemma}

Proof of the above lemma is similar 
to the proof of Lemma 9 in \citet{bousquet2002stability} 
and can be found in \appref{app:proof_lem_error}. 
% 
% Before presenting the result, we introduce some notation. 
Before presenting the proof of \thmref{thm:linear}, 
we introduce some more notation. 
Let $\bX_{(i)}$ denote the matrix of covariates 
with the $i^{\text{th}}$ point removed. 
Similarly, let $\by_{(i)}$ be the array of responses 
with the $i^{\text{th}}$ point removed. 
Define the corresponding regularized GD solution 
as $\wh w_{(i)} = \left( \bXX{i}^T\bXX{i}+\lambda \boldsymbol{I}\right)^{-1}\bXX{i}^T\byy{i}$. 
Define $\ff{i}(x) \defeq f(x ; \wh w_{(i)}) $.

\begin{proof}[Proof of \thmref{thm:linear}]
    Because squared loss minimization does not imply 0-1 error minimization, 
    we cannot use arguments from \lemref{lem:fit_mislabeled}. 
    This is the main technical difficulty. 
    To compare the 0-1 error at a train point with an unseen point, 
    we use the closed-form expression for $\widehat{w}$ 
    and Shermann-Morrison formula 
    to upper bound training error 
    with leave-one-out cross validation error. 
    
    The proof is divided into three parts: 
    In part one, we show that 0-1 error 
    on mislabeled points in the training set 
    is lower than the error obtained 
    by leave-one-out error at those points. 
    In part two, we relate this leave-one-out error 
    with the population error on mislabeled distribution
    using \codref{cond:hypothesis_stability}.
    While the empirical leave-one-out error is an unbiased estimator 
    of the average population error of leave-one-out classifiers, 
    we need hypothesis stability 
    to control the variance 
    of empirical leave-one-out error. 
    Finally, in part three, we show 
    that the error on the mislabeled training points 
    can be estimated with just the randomly labeled 
    and clean training data (as in proof of \thmref{thm:error_ERM}).  

    \textbf{Part 1 {} {}} First we relate training error with leave-one-out error.        
    For any training point $(x_i, y_i)$ in $\wt S \cup S$, we have 
    \begin{align}
        \error(\wh f(x_i), y_i ) &= \indict{ y_i \cdot x_i^T \wh w < 0 } = \indict{ y_i \cdot x_i^T \left( \bX^T\bX+\lambda \boldsymbol{I}\right)^{-1}\bX^T\by < 0 } \\
        &= \indict{ y_i \cdot x_i^T \underbrace{\left( \bXX{i}^T\bXX{i} + x_i ^T x_i +\lambda \boldsymbol{I}\right)^{-1}}_{\RN{1}} (\bXX{i}^T\byy{i} + y_i \cdot x_i) < 0 } \,.
    \end{align}
    Letting $\bA = \left(\bXX{i}^T\bXX{i} +\lambda \boldsymbol{I}\right)$ 
    and using \lemref{lem:sherman} on term 1, we have 
    \begin{align}
        \error(\wh f(x_i), y_i ) &= \indict{ y_i \cdot x_i^T \left[\bA^{-1} -  \frac{\bA^{-1} x_i x_i^T \bA^{-1}}{ 1 + x_i ^T \bA^{-1} x_i } \right] (\bXX{i}^T\byy{i} + y_i \cdot x_i) < 0 } \\
        &= \indict{ y_i \cdot\left[ \frac{ x_i^T \bA^{-1} ( 1 + x_i ^T \bA^{-1} x_i ) -  x_i^T \bA^{-1} x_i x_i^T \bA^{-1}}{ 1 + x_i ^T \bA ^{-1}x_i } \right] (\bXX{i}^T\byy{i} + y_i \cdot x_i) < 0 } \\
        &= \indict{ y_i \cdot\left[ \frac{ x_i^T \bA^{-1}}{ 1 + x_i ^T \bA ^{-1}x_i } \right] (\bXX{i}^T\byy{i} + y_i \cdot x_i) < 0 } \,.
    \end{align}

    Since $1 + x_i^T \bA^{-1} x_i > 0$, we have 
    \begin{align}
        \error(\wh f(x_i), y_i ) &= \indict{ y_i \cdot x_i^T \bA^{-1} (\bXX{i}^T\byy{i} + y_i \cdot x_i) < 0 } \\
        &= \indict{ x_i^T \bA^{-1} x_i +  y_i \cdot x_i^T \bA^{-1} (\bXX{i}^T\byy{i}) < 0 } \\
        &\le \indict{ y_i \cdot x_i^T \bA^{-1} (\bXX{i}^T\byy{i}) < 0 } = \error(\ff{i}(x_i), y_i ) \,.\label{eq:LOO_error}
    \end{align}

    Using \eqref{eq:LOO_error}, we have 
    \begin{align}
        \error_{\wt \calS_M } (\wh f) \le \error_{\text{LOO} (\wt S_M)} \defeq \frac{\sum_{(x_i, y_i) \in \wt S_M} \error(\ff{i}(x_i), y_i ) }{\abs{\wt \calS_M}}\label{eq:LOO_error_final} \,.
    \end{align}
    \textbf{Part 2 {}{}} We now relate RHS in \eqref{eq:LOO_error_final} 
    with the population error on mislabeled distribution. 
    To do this, we leverage \codref{cond:hypothesis_stability} 
    and \lemref{lem:stability_error}. 
    In particular, we have 

    \begin{align}
        \Expt{\calS \cup \wt \calS_M }{ \left(\error_{\calDm}(\wh f) - \error_{\text{LOO} (\wt S_M)}\right)^2 } \le \frac{1}{2m_1} + \frac{3\beta}{m+n} \,.
    \end{align}

    Using Chebyshev's inequality, with probability at least $1-\delta$, we have 
    \begin{align}
        \error_{\text{LOO} (\wt S_M)} \le  \error_{\calDm}(\wh f)   + \sqrt{\frac{1}{\delta}\left(\frac{1}{2m_1} +\frac{3\beta}{m+n} \right)} \,. \label{eq:final_mislabeled_linear}
    \end{align}
    

    \textbf{Part 3 {}{}} Combining \eqref{eq:final_mislabeled_linear} and \eqref{eq:LOO_error_final}, we have 

    \begin{align}
        \error_{\wt \calS_M } (\wh f) \le \error_{\calDm}(\wh f)   + \sqrt{\frac{1}{\delta}\left(\frac{1}{2m_1} +\frac{3\beta}{m+n} \right)} \,. \label{eq:linear_parallel_lem1}
    \end{align}

    Compare \eqref{eq:linear_parallel_lem1} with \eqref{eq:lemma1_final} 
    in the proof of \lemref{lem:fit_mislabeled}. 
    We obtain a similar relationship 
    between $\error_{\wt \calS_M }$ and $\error_{\calDm}$ 
    but with a polynomial concentration 
    instead of exponential concentration. 
    In addition, since we just use concentration arguments 
    to relate mislabeled error to the errors
    on the clean and unlabeled portions 
    of the randomly labeled data, 
    we can directly use the results 
    in \lemref{lem:mislabeled_error} and \lemref{lem:clear_error}. 
    Therefore, combining results in \lemref{lem:mislabeled_error}, \lemref{lem:clear_error}, and \eqref{eq:linear_parallel_lem1} with union bound, 
    we have with probability at least $1-\delta$
    \begin{align}
        \error_\calD(\widehat f) \le \error_\calS(\widehat f) + 1 - 2 \error_{\wt\calS}(\widehat f) + \left(\sqrt{2}\error_{\wt\calS}(\widehat f) + 1 + \frac{m}{2n} \right) \sqrt{\frac{\log(4/\delta)}{m}} + \sqrt{\frac{4}{\delta}\left(\frac{1}{m} +\frac{3\beta}{m+n} \right)}  \,.
    \end{align}
    

       
\end{proof}

\subsection{Extension to multiclass classification} \label{app:multiclass_linear}
For multiclass problems with squared loss minimization, as standard practice, we consider one-hot encoding for the underlying label, i.e., a class label $c \in [k]$ is treated as $(0, \cdot, 0,1,0, \cdot, 0) \in \Real^k$ (with $c$-th coordinate being 1).  As before, we suppose that the parameters of the linear function 
are obtained via gradient descent on the following $L_2$ regularized problem: 
\begin{align}
    % n in denominator is avoided deliberately
    \calL_S(w; \lambda) \defeq \sum_{i=1}^n\norm{w^Tx_i - y_i}{2}^2 + \lambda \sum_{j=1}^k \norm{w_j}{2}^2 \,, \label{eq:l2_multiclass_MSE_app}   
\end{align}
where $\lambda\ge0$ is a regularization parameter. 
We assume access to a clean dataset 
$S = \{(x_i, y_i)\}_{i=1}^n \sim \calD^n$ 
and randomly labeled dataset 
$\wt S = \{(x_i, y_i)\}_{i=n+1}^{n+m} \sim \wt \calD^m$. 
Let $\bX = [x_1, x_2, \cdots, x_{m+n}]$ 
and $\by = [e_{y_1}, e_{y_2}, \cdots, e_{y_{m+n}}]$. 
Fix a positive learning rate $\eta$ such that 
$\eta \le 1/\left(\norm{\bX^T\bX}{\text{op}} + \lambda^2\right)$ 
and an initialization $w_0 = 0$. 
% \todos{Assumption made for simplicty}. 
Consider the following gradient descent iterates 
to minimize objective \eqref{eq:l2_MSE_app} on $S \cup \wt S$:
\begin{align}
{w_j}^t = {w_j}^{t-1} - \eta \grad_{w_j} \calL_{S \cup \wt S} (w^{t-1}; \lambda) \quad \forall t=1,2,\ldots \text{ and } j=1,2,\ldots,k  \,. \label{eq:GD_multi_iterates_app}
\end{align} 
Then we have $\{ {w_j}^t\}$ for all $j =1,2,\cdots, k$ converge to the limiting solution 
$\wh w_j = \left( \bX^T\bX+\lambda \boldsymbol{I}\right)^{-1}\bX^T\by_j$. Define $\widehat f (x) \defeq f(x ; \wh w) $.  

\begin{theorem}\label{thm:multi_linear}
    Assume that this gradient descent algorithm satisfies \codref{cond:hypothesis_stability}
    with $\beta=\calO(1)$.  
    Then for a multiclass classification problem wth $k$ classes, for any $\delta >0$, with probability at least $1-\delta$, we have:
    \begin{align*}
        \error_\calD(\widehat f) \le \error_\calS(\widehat f) &+ (k-1)\left(1 - \frac{k}{k-1} \error_{\wt\calS}(\widehat f) \right) \\ &+ \left(k + \sqrt{k} + \frac{m}{n\sqrt{k}} \right) \sqrt{\frac{\log(4/\delta)}{2m}} + \sqrt{k(k-1)} \sqrt{\frac{4}{\delta}\left(\frac{1}{m} +\frac{3\beta}{m+n} \right)}  \,. \numberthis \label{eq:gd_multi_error}
    \end{align*} 
    % for some constant $c\le 3.2$.
\end{theorem}
\begin{proof}
    The proof of this theorem is divided into two parts. In the first part, we relate the error on the mislabeled samples with the population error on the mislabeled data. Similar to the proof of \thmref{thm:linear}, we use Shermann-Morrison formula to upper bound training error with leave-one-out error on each $\wh w^j$. Second part of the proof follows entirely from the proof of \thmref{thm:multiclass_ERM}. In essence, the first part derives an equivalent of \eqref{eq:lemma1_final_multi_prev} for GD training with squared loss and then the second part follows from the proof  of \thmref{thm:multiclass_ERM}. 
    
    \textbf{Part-1:} Consider a training point $(x_i,y_i)$ in $\wt S \cup S $. For simplicity, we use $c_i$ to denote the class of $i$-th point and use $y_i$ as the corresponding one-hot embedding. Recall error in multiclass point is given by $\error(\wh f(x_i), y_i ) = \indict{ c_i \not \in \argmax x_i^T \wh w }$. Thus, there exists a $j \ne c_i \in [k]$, such that we have
     \begin{align}
        \error(\wh f(x_i), y_i ) &= \indict{ c_i \not \in \argmax x_i^T \wh w } = \indict{ x_i^T \wh w_{c_i} < x_i^T \wh w_{j}  } \\ &= \indict{ x_i^T \left( \bX^T\bX+\lambda \boldsymbol{I}\right)^{-1}\bX^T\by_{c_i} < x_i^T \left( \bX^T\bX+\lambda \boldsymbol{I}\right)^{-1}\bX^T\by_{j} } \\
        &= \indict{ x_i^T \underbrace{\left( \bXX{i}^T\bXX{i} + x_i ^T x_i +\lambda \boldsymbol{I}\right)^{-1}}_{\RN{1}} \left(\bXX{i}^T{\by_{c_i}}_{(i)} + x_i - \bXX{i}^T{\by_{j}}_{(i)}\right) < 0 } \,.
    \end{align}
    Letting $\bA = \left(\bXX{i}^T\bXX{i} +\lambda \boldsymbol{I}\right)$ 
    and using \lemref{lem:sherman} on term 1, we have 
    \begin{align}
        \error(\wh f(x_i), y_i ) &= \indict{ x_i^T \left[\bA^{-1} -  \frac{\bA^{-1} x_i x_i^T \bA^{-1}}{ 1 + x_i ^T \bA^{-1} x_i } \right]  \left(\bXX{i}^T{\by_{c_i}}_{(i)} + x_i - \bXX{i}^T{\by_{j}}_{(i)}\right) < 0 } \\
        &= \indict{ \left[ \frac{ x_i^T \bA^{-1} ( 1 + x_i ^T \bA^{-1} x_i ) -  x_i^T \bA^{-1} x_i x_i^T \bA^{-1}}{ 1 + x_i ^T \bA ^{-1}x_i } \right]  \left(\bXX{i}^T{\by_{c_i}}_{(i)} + x_i - \bXX{i}^T{\by_{j}}_{(i)}\right) < 0 } \\
        &= \indict{ \left[ \frac{ x_i^T \bA^{-1}}{ 1 + x_i ^T \bA ^{-1}x_i } \right]  \left(\bXX{i}^T{\by_{c_i}}_{(i)} + x_i - \bXX{i}^T{\by_{j}}_{(i)}\right) < 0} \,.
    \end{align}
    Since $1 + x_i^T \bA^{-1} x_i > 0$, we have 
    \begin{align}
        \error(\wh f(x_i), y_i ) &= \indict{ x_i^T \bA^{-1}  \left(\bXX{i}^T{\by_{c_i}}_{(i)} + x_i - \bXX{i}^T{\by_{j}}_{(i)}\right) < 0 } \\
        &= \indict{ x_i^T \bA^{-1} x_i +  x_i^T \bA^{-1}  \bXX{i}^T{\by_{c_i}}_{(i)}  - x_i^T\bA^{-1}  \bXX{i}^T{\by_{j}}_{(i)} < 0 } \\
        &\le \indict{  x_i^T \bA^{-1}  \bXX{i}^T{\by_{c_i}}_{(i)}  - x_i^T\bA^{-1}  \bXX{i}^T{\by_{j}}_{(i)} < 0  } = \error(\ff{i}(x_i), y_i ) \,.\label{eq:LOO_error_multi}
    \end{align}
    Using \eqref{eq:LOO_error_multi}, we have 
    \begin{align}
        \error_{\wt \calS_M } (\wh f) \le \error_{\text{LOO} (\wt S_M)} \defeq \frac{\sum_{(x_i, y_i) \in \wt S_M} \error(\ff{i}(x_i), y_i ) }{\abs{\wt \calS_M}}\label{eq:LOO_error_multi_final} \,.
    \end{align}
    
    We now relate RHS in \eqref{eq:LOO_error_final} 
    with the population error on mislabeled distribution. 
    Similar as before, to do this, we leverage \codref{cond:hypothesis_stability} 
    and \lemref{lem:stability_error}. Using  \eqref{eq:final_mislabeled_linear} and \eqref{eq:LOO_error_multi_final}, we have 
    \begin{align}
        \error_{\wt \calS_M } (\wh f) \le \error_{\calDm}(\wh f)   + \sqrt{\frac{1}{\delta}\left(\frac{1}{2m_1} +\frac{3\beta}{m+n} \right)} \,. \label{eq:linear_multi_parallel_lem1}
    \end{align}
    
    We have now derived a parallel to \eqref{eq:lemma1_final_multi_prev}. Using the same arguments in the proof of \lemref{lem:fit_mislabeled_multi}, we have 
    \begin{align}
      \error_{\calD}(\wh f) \le  (k-1) \left( 1- \error_{ \wt \calS_M}(\wh f) \right)  + (k-1)\sqrt{\frac{k}{\delta(k-1)}\left(\frac{1}{2m_1} +\frac{3\beta}{m+n} \right)}  \,. \label{eq:lemma1_linear_final_multi}
    \end{align}
    
    \textbf{Part-2:} We now combine the results in \lemref{lem:mislabeled_error_multi} and \lemref{lem:clear_error_multi} to obtain the final inequality in terms of quantities that can be computed from just the randomly labeled and clean data. Similar to the binary case, we obtained a polynomial concentration instead of exponential concentration. Combining \eqref{eq:lemma1_linear_final_multi} with \lemref{lem:mislabeled_error_multi} and \lemref{lem:clear_error_multi}, we have with probability at least $1-\delta$
    \begin{align*}
        \error_\calD(\widehat f) \le \error_\calS(\widehat f) &+ (k-1)\left(1 - \frac{k}{k-1} \error_{\wt\calS}(\widehat f) \right) \\ &+ \left(k + \sqrt{k} + \frac{m}{n\sqrt{k}} \right) \sqrt{\frac{\log(4/\delta)}{2m}} + \sqrt{k(k-1)} \sqrt{\frac{4}{\delta}\left(\frac{1}{m} +\frac{3\beta}{m+n} \right)}  \,. \numberthis \label{eq:gd_multi_error_proof}
    \end{align*} 
\end{proof}

\subsection{Discussion on \codref{cond:hypothesis_stability}} \label{app:discuss_cond1}
The quantity in LHS of \codref{cond:hypothesis_stability} 
measures how much the function learned by the algorithm 
(in terms of error on unseen point) will change 
when one point in the training set is removed. 
% Discussion on exponential concentration and stronger condition. 
% Notice that hypothesis stability implies error stability, i.e., \codref{cond:error_stability} \citep{bousquet2002stability}.  
% In summary, while error stability allowed us 
% to relate the average population error 
% of the leave-one-out classifiers 
% with the population error of the original classifier, 
We need hypothesis stability condition 
to control the variance of the empirical leave-one-out error to show concentration of average leave-one-error with the population error. 

Additionally, we note that while the dominating term in the RHS of \thmref{thm:linear} matches with the dominating term in ERM bound in \thmref{thm:error_ERM}, there is a polynomial concentration term 
(dependence on $1/\delta$ instead of $\log(\sqrt{1/\delta})$) 
in \thmref{thm:linear}. 
Since with hypothesis stability, 
we just bound the variance, 
the polynomial concentration is due 
to the use of Chebyshev's inequality 
instead of an exponential tail inequality
(as in \lemref{lem:fit_mislabeled}).
Recent works have highlighted that 
a slightly stronger condition than hypothesis stability 
can be used to obtain an exponential concentration 
for leave-one-out error \citep{abou2019exponential},
but we leave this for future work for now. 
% We leave 
% However, the constants 

% we also want to highlight  

\subsection{Formal statement and proof of \propref{prop:early_stop}} \label{app:formal_early_stop}

Before formally presenting the result, 
we will introduce some notation.  
By $\calL_{S}(w)$, we denote 
the objective in \eqref{eq:l2_MSE_app} with $\lambda=0$. 
Assume Singular Value Decomposition (SVD) of $\bX$
as $\sqrt{n} \bU \bS^{1/2} \bV^T$. 
Hence $\bX^T \bX = \bV \bS \bV^T$.
Consider the GD iterates defined in \eqref{eq:GD_iterates_app}. 
% 
We now derive closed form expression 
for the $t^\text{th}$ iterate of gradient descent:  
% 
\begin{align}
    w_t = w_{t-1} + \eta \cdot \bX^T (\by - \bX w_{t-1}) = (\bI - \eta \bV \bS \bV^T )w_{k-1} + \eta \bX^T \by \,.
\end{align}
Rotating by $\bV^T$, we get 
\begin{align}
    \wt w_t = (\bI - \eta\bS )\wt w_{k-1} + \eta \wt \by \label{eq:GD_recur},
\end{align}
where $\wt w_t = \bV^T w_t $ and $\wt \by = \bV^T \bX^T \by$. 
Assuming the initial point $w_0 = 0$ 
and applying the recursion in \eqref{eq:GD_recur}, we get
\begin{align}
    \wt w_t = \bS ^{-1} ( \bI - (\bI - \eta \bS)^k ) \wt \by \,, 
\end{align} 
Projecting solution back to the original space, we have 
\begin{align}
     w_t = \bV \bS ^{-1} ( \bI - (\bI - \eta \bS)^k ) \bV^T \bX^T \by \,. 
\end{align} 
% We will work with this GD solution at any iterate $t$ in the next proposition. 
Define $f_t(x) \defeq f(x;w_t)$ 
as the solution at the $t^{\text{th}}$ iterate. 
Let $\wt w_{\lambda} = \argmin_{w} \calL_\calS (w;\lambda) = (\bX^T \bX + \lambda \bI)^{-1} \bX^T \by = \bV (\bS + \lambda \bI )^{-1} \bV^T \bX^T \by $. 
% ) \,,$ for all $t=1,2,\ldots\,.$ 
and define $\wt f_\lambda(x) \defeq f(x;\wt w_\lambda)$ as the regularized solution. 
Assume $\kappa$ be the condition number 
of the population covariance matrix 
and let $s_\text{min}$ be the minimum positive 
singular value of the empirical covariance matrix. 
Our proof idea is inspired from recent work 
on relating gradient flow solution 
and regularized solution 
for regression problems \citep{ali2018continuous}. 
We will use the following lemma in the proof: 
\begin{lemma} \label{lem:ineq_soln}
    For all $x \in [0,1]$ and for all $ k \in \mathbb{N}$, 
    we have (a) $ \frac{kx}{1+kx} \le 1- (1-x)^k$ 
    and (b) $ 1- (1-x)^k \le 2 \cdot \frac{kx}{kx+1} $.
    %  where $g(c)$ is a constant dependent on $c$. For $c = 1$, $g(c) = 2.0$.   
\end{lemma}
\begin{proof}
    % [Proof of \lemref{lem:ineq_soln}]
    % Part (a) is easy. 
    Using $ (1-x)^k \le \frac{1}{1+kx}$, we have part (a). 
    For part (b), we numerically maximize 
    $\frac{ (1+kx ) (1 - (1-x)^k) }{kx}$ 
    for all $k\ge 1$ and for all $x \in [0, 1]$.  
\end{proof}

% 
% Next, 

\begin{prop}[Formal statement of \propref{prop:early_stop}] \label{prop:formal_early_stop}
Let $\lambda = \frac{1}{t\eta}$. 
For a training point $x$, we have 
\begin{align*}
    \Expt{x \sim \calS}{(f_t(x) - \wt f_\lambda(x))^2} &\le c(t,\eta) \cdot \Expt{x \sim \calS}{f_t(x)^2} \,, %\label{eq:early_stop}
\end{align*}
where $c(t, \eta) \defeq \min( 0.25, \frac{1}{s_\text{min}^2 t^2 \eta^2})$. 
Similarly for a test point, we have 
\begin{align*}
    \Expt{x \sim \calD_\calX}{(f_t(x) - \wt f_\lambda(x))^2} &\le \kappa \cdot c(t,\eta) \cdot \Expt{x \sim \calD_\calX}{f_t(x)^2} \,. %\label{eq:early_stop}
\end{align*}
\end{prop} 

\begin{proof}
    %%%%%%%%%%%%% 
    We want to analyze the expected squared difference output 
    of regularized linear regression 
    with regularization constant $\lambda = \frac{1}{\eta t}$ 
    and the gradient descent solution at the $t^\text{th}$ iterate. 
    We separately expand the algebraic expression 
    for squared difference at a training point and a test point. 
    % We start by considering the difference  
    Then the main step is to show that 
    $\left[ \bS ^{-1} ( \bI - (\bI - \eta \bS)^k )  - (\bS + \lambda \bI )^{-1}\right] \preceq c(\eta, t) \cdot \bS ^{-1} ( \bI - (\bI - \eta \bS)^k ) $.

    %%%%%%%%%%%%%
    
   \textbf{Part 1 {} {}} 
    First, we will analyze the squared difference 
    of the output at a training point 
    (for simplicity, we refer to $S \cup \wt S$ as $S$), i.e., 
    \begin{align}
        \Expt{ x \sim \calS }{\left(f_t(x) - \wt f_\lambda (x)\right)^2} &= \norm{\bX w_t - \bX \wt w_\lambda}{2}^2\\ &=   \norm{\bX \bV \bS ^{-1} ( \bI - (\bI - \eta \bS)^t ) \bV^T \bX^T \by - \bX \bV (\bS + \lambda \bI )^{-1} \bV^T \bX^T \by }{2}^2 \\
        &= \norm{\bX \bV \left(\bS ^{-1} ( \bI - (\bI - \eta \bS)^t ) - (\bS + \lambda \bI )^{-1} \right) \bV^T \bX^T \by  }{2} \\
        &=  \by^T \bV \bX \left( \underbrace{\bS ^{-1} ( \bI - (\bI - \eta \bS)^t ) - (\bS + \lambda \bI )^{-1}}_{\RN{1}} \right)^2 \bS \bV^T \bX^T \by \label{eq:train_GD_rel} \,.
        %  (\bX \bV \bS ^{-1} ( \bI - (\bI - \eta \bS)^k ) \bV^T \bX^T \by)^T \bX \bV \bS ^{-1} ( \bI - (\bI - \eta \bS)^k ) \bV^T \bX^T \by
    \end{align}
    We now separately consider term 1. 
    Substituting $\lambda = \frac{1}{t \eta}$, 
    we get
    \begin{align}
        \bS ^{-1} ( \bI - (\bI - \eta \bS)^t ) - (\bS + \lambda \bI )^{-1} &= \bS^{-1} \left( ( \bI - (\bI - \eta \bS)^t ) - (\bI + \bS^{-1} \lambda )^{-1}\right) \\
        &= \underbrace{\bS^{-1} \left( ( \bI - (\bI - \eta \bS)^t ) - (\bI + ( \bS t \eta)^{-1}  )^{-1}\right)}_{\bA} \,.
    \end{align}

    We now separately bound the diagonal entries in matrix $\bA$. 
    With $s_i$, we denote $i^{\text{th}}$ diagonal entry of $\bS$.
    Note that since $ \eta\le 1/\norm{S}{\text{op}}$, 
    for all $i$, $\eta s_i  \le 1$.  
    Consider $i^{\text{th}}$ diagonal term (which is non-zero) 
    of the diagonal matrix $\bA$, we have 
    \begin{align}
        \bA_{ii} = \frac{1}{s_i} \left(  1 - (1 - s_i \eta)^t - \frac{t \eta s_i}{1 + t \eta s_i } \right) &=  \frac{1 - (1 - s_i \eta)^t}{s_i} \left( \underbrace{ 1 - \frac{t \eta s_i}{(1 + t \eta s_i)(1 - (1 - s_i \eta)^t)}}_{\RN{2}} \right) \\ 
         &\le \frac{1}{2}\left[ \frac{1 - (1 - s_i \eta)^t}{ s_i} \right] \tag*{(Using \lemref{lem:ineq_soln} (b))} \,.
    \end{align} 
    Additionally, we can also show the following upper bound on term 2: 
    \begin{align}
         1 - \frac{t \eta s_i}{(1 + t \eta s_i)(1 - (1 - s_i \eta)^t)} &= \frac{(1 + t \eta s_i)(1 - (1 - s_i \eta)^t) - t \eta s_i }{(1 + t \eta s_i)(1 - (1 - s_i \eta)^t)} \\
         & \le  \frac{ 1 -  (1 - s_i \eta)^t - t \eta s_i (1 - s_i \eta)^t}{(1 + t \eta s_i)(1 - (1 - s_i \eta)^t)} \\
         & \le \frac{1}{t\eta s_i} \,. \tag{Using \lemref{lem:ineq_soln} (a)}
        %  &\le \frac{1}{2}\left[ \frac{1 - (1 - s_i \eta)^t}{ s_i} \right] \tag*{(Using \lemref{lem:ineq_soln})} \,.
    \end{align} 

    Combining both the upper bounds 
    on each diagonal entry $\bA_{ii}$, we have 
    \begin{align}
    \bA \preceq c_1(\eta, t) \cdot \bS^{-1} ( \bI - (\bI - \eta \bS)^t ) \,, \label{eq:upperbound_diagonal}
    \end{align}
    where $c_1(\eta, t ) = \min(0.5, \frac{1}{t s_i \eta })$. Plugging this into \eqref{eq:train_GD_rel}, we have 
    \begin{align}
        \Expt{ x \sim \calS }{\left(f_t(x) - \wt f_\lambda (x)\right)^2} &\le c(\eta, t) \cdot \by^T \bV \bX  \left( \bS^{-1} ( \bI - (\bI - \eta \bS)^t ) \right)^2 \bS \bV^T \bX^T \by \\
        &=   c(\eta, t) \cdot \by^T \bV \bX  \left( \bS^{-1} ( \bI - (\bI - \eta \bS)^t ) \right) \bS \left( \bS^{-1} ( \bI - (\bI - \eta \bS)^t ) \right) \bV^T \bX^T \by \\
        & =  c(\eta, t) \cdot \norm{\bX w_t}{2}^2 \\
        &= c(\eta, t) \cdot  \Expt{ x \sim \calS }{\left(f_t(x) \right)^2} \,,
    \end{align}
    where $c(\eta, t ) = \min(0.25, \frac{1}{t^2 s^2_i \eta^2 })$.

    \textbf{Part 2 {} {}} With $\bSigma$, 
    we denote the underlying true covariance matrix. 
    We now consider the squared difference of output at an unseen point: 
    \begin{align}
        \Expt{ x \sim \calD_{\calX} }{\left(f_t(x) - \wt f_\lambda (x)\right)^2} &= \Expt{x \sim \calD_{\calX}}{\norm{x^T w_t - x^T \wt w_\lambda}{2}} \\
        &=   \norm{x^T \bV \bS ^{-1} ( \bI - (\bI - \eta \bS)^t ) \bV^T \bX^T \by - x^T \bV (\bS + \lambda \bI )^{-1} \bV^T \bX^T \by }{2} \\
        &= \norm{x^T \bV \left(\bS ^{-1} ( \bI - (\bI - \eta \bS)^t ) - (\bS + \lambda \bI )^{-1} \right) \bV^T \bX^T \by  }{2} \\
        &= \by^T \bV \bX \left( \bS ^{-1} ( \bI - (\bI - \eta \bS)^t ) - (\bS + \lambda \bI )^{-1} \right) \bV^T \bSigma \bV \\ &\qquad \qquad \qquad \qquad \qquad \left( (\bI - (\bI - \eta \bS)^t ) - (\bS + \lambda \bI )^{-1} \right) \bV^T \bX^T \by \\
        &\le \sigma_{\text{max}} \cdot \by^T \bV \bX \left( \underbrace{\bS ^{-1} ( \bI - (\bI - \eta \bS)^t ) - (\bS + \lambda \bI )^{-1}}_{\RN{1}} \right)^2 \bV^T \bX^T \by \,, \label{eq:test_GD_rel}
        %  (\bX \bV \bS ^{-1} ( \bI - (\bI - \eta \bS)^k ) \bV^T \bX^T \by)^T \bX \bV \bS ^{-1} ( \bI - (\bI - \eta \bS)^k ) \bV^T \bX^T \by
    \end{align}
    where $\sigma_{\text{max}}$ is the maximum eigenvalue 
    of the underlying covariance matrix $\bSigma$. 
    Using the upper bound on term 1 in \eqref{eq:upperbound_diagonal}, 
    we have 
    \begin{align}
        \Expt{ x \sim \calD_{\calX} }{\left(f_t(x) - \wt f_\lambda (x)\right)^2} &\le \sigma_{\text{max}} \cdot c(\eta, t) \cdot \by^T \bV \bX  \left( \bS^{-1} ( \bI - (\bI - \eta \bS)^t ) \right)^2 \bV^T \bX^T \by \\
        &=   \kappa \cdot c(\eta, t) \cdot \sigma_{\text{min}}\cdot \norm{\bV \left( \bS^{-1} ( \bI - (\bI - \eta \bS)^t ) \right) \bV^T \bX^T \by}{2}^2 \\
        &\le \kappa \cdot c(\eta, t) \cdot \left[ \bV \left( \bS^{-1} ( \bI - (\bI - \eta \bS)^t ) \right) \bV^T \bX^T \right]^T \bSigma \\
        &\qquad \qquad \qquad \qquad \qquad \left[ \bV \left( \bS^{-1} ( \bI - (\bI - \eta \bS)^t ) \right) \bV^T \bX^T \right] \by \\
        & = \kappa \cdot c(\eta, t) \cdot \Expt{x \sim \calD_{\calX}}{\norm{x^T w_t}{2}} \,.
    \end{align}
% 
% 
    % Since $ \eta\le 1/\norm{S}{\text{op}}$, invoking \lemref{lem:ineq_soln} to upper bound term 1 with
\end{proof}

\subsection{Extension to deep learning} \label{appsubsec:ext_DL}
Under \asmpref{appsubsec:justifying_assumption1}, we present the formal result parallel to \thmref{thm:multiclass_ERM}. 
\begin{theorem} \label{thm:multiclass_ERM_algoA}
    Consider a multiclass classification problem 
    with $k$ classes. Under \asmpref{asmp:deep_models}, 
    for any $\delta >0$, with probability at least $1-\delta$,
    we have
    \vspace{-10pt}
    \begin{align*}
        \error_\calD(\widehat f)  \le \error_\calS(\widehat f) + (k-1) \left(1 - \tfrac{k}{k-1} \error_{\wt\calS}(\widehat f)\right) + c\sqrt{\frac{\log(\frac{4}{\delta})}{2m}} \,,\numberthis \label{eq:multiclass_ERM_deep}
    % \vspace{-20pt}
    \end{align*}
    for some constant $c \le ((c+1) k+\sqrt{k} + \frac{m}{n\sqrt{k}})$.
\end{theorem}

The proof follows exactly as in step (i) to (iii) in \thmref{thm:multiclass_ERM}.  

\subsection{Justifying~\asmpref{asmp:deep_models}} \label{appsubsec:justifying_assumption1}

Motivated by the analysis on linear models, we now discuss alternate (and weaker) conditions that imply \asmpref{asmp:deep_models}. 
We need hypothesis stability (\codref{cond:hypothesis_stability}) and the following assumption relating training error and leave-one-error: 

\begin{assumption} \label{asmp:loo_error}
Let $\wh f$ be a model obtained by training with algorithm $\calA$ on a mixture of clean $S$ and randomly labeled data $\wt S$. Then we assume we have 
\begin{align*}
    \error_{\wt \calS_M} (\wh f) \le  \error_{\text{LOO} (\wt S_M)} \,, 
\end{align*}
for all $(x_i, y_i) \in  \wt S_M$ where $\wh f_{(i)} \defeq f(\calA, S \cup {{}\wt S_M}_{(i)})$ and  $\error_{\text{LOO} (\wt S_M)} \defeq  \frac{\sum_{(x_i, y_i) \in \wt S_M} \error(\ff{i}(x_i), y_i ) }{\abs{\wt \calS_M}}$.  
\end{assumption}

% we assume this to extend our result (parallel to \thmref{thm:multi_linear}) for deep models. 
Intuitively, this assumption states that the error on a (mislabeled) datum $(x,y)$ included in the training set is less than the error on that datum $(x,y)$ obtained by a model trained on the training set $S - \{(x,y)\}$. We proved this for linear models trained with GD in the proof of \thmref{thm:multi_linear}. 
% 
\codref{cond:hypothesis_stability} with $\beta = \calO(1)$ and \asmpref{asmp:loo_error} together with \lemref{lem:stability_error} implies \asmpref{asmp:deep_models} with a polynomial residual term (instead of logarithmic in $1/\delta$): 
\begin{align}
     \error_{\calS_M} (\wh f) \le  \error_{\calDm}(\wh f)   + \sqrt{\frac{1}{\delta}\left(\frac{1}{m} +\frac{3\beta}{m+n} \right)} \,.
\end{align}
% Note that this  

\newpage 
\section{Additional experiments and details}\label{app:exp}
\newcommand\tab[1][1cm]{\hspace*{#1}}

\subsection{Datasets} \label{sec:app_dataset}

\textbf{Toy Dataset {} {}} Assume fixed constants $\mu$ and $\sigma$. For a given label $y$, we simulate features $x$ in our toy classification setup as follows: 
\begin{align*}
    x \defeq \texttt{concat} \left[ x_1, x_2\right] \quad \text{where} \quad  x_1 \sim  \calN( y \cdot \mu, \sigma^2 I_{d \times d}) \ \  \text{and} \ \  x_1 \sim  \calN( 0, \sigma^2 I_{d \times d}) \,.
\end{align*}  
% where $y$ is the true label and $x$ is the corresponding feature vector. 
In experiements throughout the paper, we fix dimention $d=100$, $\mu = 1.0 $, and $\sigma = \sqrt{d}$. Intuitively, $x_1$ carries the information about the underlying label and $x_2$ is additional noise independent of the underlying label. 

\textbf{CV datasets {} {}} We use MNIST~\citep{lecun1998mnist} and CIFAR10~\cite{krizhevsky2009learning}. 
% For binary tasks, 
We produce a binary variant from the multiclass classification problem by mapping classes $\{0,1,2,3,4\}$ to label $1$ and $\{ 5,6,7,8,9\}$ to label $-1$. For CIFAR dataset, we also use the standard data augementation of random crop and horizontal flip. PyTorch code is as follows: 

\texttt{(transforms.RandomCrop(32, padding=4),\\
\tab transforms.RandomHorizontalFlip())}

\textbf{NLP dataset {} {}} We use IMDb Sentiment analysis~\citep{maas2011learning} corpus.  

\subsection{Architecture Details} 

All experiments were run on NVIDIA GeForce RTX 2080 Ti GPUs. We used PyTorch~\citep{NEURIPS2019a9015} and Keras with Tensorflow~\citep{abadi2016tensorflow} backend for experiments. 
% , ELMo embeddings~\citep{Peters:2018}, and Hugging Face Transformers~\citep{wolf-etal-2020-transformers}. 

\textbf{Linear model {} {}} For the toy dataset, we simulate a linear model with scalar output and the same number of parameters as the number of dimensions.   

\textbf{Wide nets {} {}} To simulate the NTK regime, we experiment with $2-$layered wide nets. The PyTorch code for 2-layer wide MLP is as follows: 


\texttt{ nn.Sequential( \\
\tab     nn.Flatten(),\\
\tab    nn.Linear(input\_dims, 200000, bias=True),\\
\tab    nn.ReLU(),\\
\tab    nn.Linear(200000, 1, bias=True)\\
\tab     )}


We experiment both (i) with the second layer fixed at random initialization; (ii)  and updating both layers' weights.     

\textbf{Deep nets for CV tasks {} {}} We consider a 4-layered MLP. The PyTorch code for 4-layer MLP is as follows: 

\texttt{ nn.Sequential(nn.Flatten(), \\
\tab        nn.Linear(input\_dim, 5000, bias=True),\\
\tab        nn.ReLU(),\\
\tab        nn.Linear(5000, 5000, bias=True),\\
\tab        nn.ReLU(),\\
\tab        nn.Linear(5000, 5000, bias=True),\\
\tab        nn.ReLU(),\\
% \tab        nn.Linear(5000, 5000, bias=True),\\
% \tab        nn.ReLU(),\\
\tab        nn.Linear(1024, num\_label, bias=True)\\
\tab        )}

For MNIST, we use $1000$ nodes instead of $5000$ nodes in the hidden layer. 
% 
We also experiment with convolutional nets. In particular, we use ResNet18 \citep{he2016deep}. Implementation adapted from:  \url{https://github.com/kuangliu/pytorch-cifar.git}. 

\textbf{Deep nets for NLP {} {}} We use a simple LSTM model with embeddings intialized with ELMo embeddings~\citep{Peters:2018}. Code adapted from: \url{https://github.com/kamujun/elmo_experiments/blob/master/elmo_experiment/notebooks/elmo_text_classification_on_imdb.ipynb} 

We also evaluate our bounds with a BERT model. In particular, we fine-tune an off-the-shelf uncased BERT model~\citep{devlin2018bert}. Code adapted from Hugging Face Transformers~\citep{wolf-etal-2020-transformers}: \url{https://huggingface.co/transformers/v3.1.0/custom_datasets.html}. 


\subsection{Additonal experiments}

\textbf{Results with SGD on underparameterized linear models {} {}} 

\begin{figure*}[h]
    \centering 
    % \vspace{-15pt}
    % \includegraphics[width=0.9\linewidth]{example-image-a}
    \includegraphics[width=0.3\linewidth]{figures/lowdim-Gaussian-SGD.pdf}
    % \includegraphics[width=0.9\linewidth]{figures/{CIFAR10_rn=0.1_lr=0.2_wd=0.005}.png}
    \vspace{-5pt}
    \caption{ 
    % Predicted lower bound 
    % on different
    We plot the accuracy and corresponding bound 
    (RHS in \eqref{eq:erm}) at $\delta = 0.1$
    for toy binary classification task. 
    Results aggregated over $3$ seeds. 
    % i.e., $1-\error$ where $\error$ is the term in the RHS of \eqref{eq:erm}
    Accuracy vs fraction of unlabeled data (w.r.t clean data) 
    in the toy setup with a linear model trained with SGD. Results parallel to \figref{fig:error_binary}(a) with SGD.  }
    \label{fig:error_binary_linear}
    \vspace{-5pt}
\end{figure*}

\textbf{Results with wide nets on binary MNIST {} {}}

\begin{figure*}[h]
    \centering 
    % \vspace{-15pt}
    % \includegraphics[width=0.9\linewidth]{example-image-a}
    \subfigure[GD with MSE loss]{\includegraphics[width=0.3\linewidth]{figures/MNIST-GD_MSE.pdf}} \hfil
    \subfigure[SGD with CE loss]{\includegraphics[width=0.3\linewidth]{figures/MNIST-SGD_CE.pdf}}
    \subfigure[SGD with MSE loss]{\includegraphics[width=0.3\linewidth]{figures/MNIST-SGD_MSE-first-layer.pdf}}
    % \includegraphics[width=0.9\linewidth]{figures/{CIFAR10_rn=0.1_lr=0.2_wd=0.005}.png}
    \vspace{-5pt}
    \caption{ 
    % Predicted lower bound 
    % on different
    We plot the accuracy and corresponding bound 
    (RHS in \eqref{eq:erm}) at $\delta = 0.1$ 
    for binary MNIST classification. 
    Results aggregated over $3$ seeds. 
    % i.e., $1-\error$ where $\error$ is the term in the RHS of \eqref{eq:erm}
    Accuracy vs fraction of unlabeled data 
    for a 2-layer wide network on binary MNIST with both the layers training in (a,b) and only first layer training in (c). 
    Results parallel to \figref{fig:error_binary}(b) .  }
    \label{fig:error_binary_MNIST}
    \vspace{-5pt}
\end{figure*}

% \begin{figure*}[h]
%     \centering 
%     % \vspace{-15pt}
%     % \includegraphics[width=0.9\linewidth]{example-image-a}
%     \subfigure[GD with MSE loss]{\includegraphics[width=0.3\linewidth]{figures/MNIST.pdf}} \hfil
    
%     \subfigure[SGD with CE loss]{\includegraphics[width=0.3\linewidth]{figures/MNIST.pdf}}
%     % \includegraphics[width=0.9\linewidth]{figures/{CIFAR10_rn=0.1_lr=0.2_wd=0.005}.png}
%     \vspace{-5pt}
%     \caption{ 
%     % Predicted lower bound 
%     % on different
%     We plot the accuracy and corresponding bound 
%     (RHS in \eqref{eq:erm}) at $\delta = 0.1$
%     for binary MNIST classification. 
%     Results aggregated over $3$ seeds. 
%     % i.e., $1-\error$ where $\error$ is the term in the RHS of \eqref{eq:erm}
%     Accuracy vs fraction of unlabeled data 
%     for a 2-layer wide network on binary MNIST with just the first layer training. 
%     Results parallel to \figref{fig:error_binary}(b) with only the first layer training.  }
%     \label{fig:error_binary_MNIST}
%     \vspace{-5pt}
% \end{figure*}

\textbf{Results on CIFAR 10 and MNIST {} {}} 
% 
We plot epoch wise error curve for results in \tabref{table:multiclass}(\figref{fig:error_epoch_CIFAR10} and \figref{fig:error_epoch_MNIST}). We observe the same trend as in \figref{fig:error_CIFAR10}. Additionally, we plot an \emph{oracle bound} obtained by tracking the error on mislabeled data which nevertheless were predicted as true label. To obtain an exact emprical value of the oracle bound, we need underlying true labels for the randomly labeled data. 
% Note that our bound in \thmref{thm:multiclass_ERM}, lower bounds the accuracy as predicted by the oracle bound. 
While with just access to extra unlabeled data we cannot calculate oracle bound, we note that the oracle bound is very tight and never violated in practice underscoring an importamt aspect of generalization in multiclass problems. This highlight that even a stronger conjecture may hold in multiclass classification, i.e., error on mislabeled data (where nevertheless true label was predicted) lower bounds the population error on the distribution of mislabeled data and hence, the error on (a specific) mislabeled portion predicts the population accuracy on clean data. 
% 
On the other hand, the dominating term of in \thmref{thm:multiclass_ERM} is loose when compared with the oracle bound. The main reason, we believe is the pessimistic upper bound in \eqref{eq:lemma1_final_multi_prev} in the proof of \lemref{lem:fit_mislabeled_multi}. We leave an investigation on this gap for future. 
% of fit 

% However, oracle bound highlights two . One,  



\begin{figure}[h]
    \centering 
    % \vspace{-15pt}
    % \includegraphics[width=0.9\linewidth]{example-image-a}
    \subfigure[MLP]{\includegraphics[width=0.3\linewidth]{figures/CIFAR10-FNN.pdf}} \hfil
    \subfigure[ResNet]{\includegraphics[width=0.3\linewidth]{figures/CIFAR10-Resnet.pdf}}
    % \includegraphics[width=0.9\linewidth]{figures/{CIFAR10_rn=0.1_lr=0.2_wd=0.005}.png}
    % \vspace{-10pt}
    \caption{ Per epoch curves for CIFAR10 corresponding results in \tabref{table:multiclass}. As before, we just plot the dominating term in the RHS of \eqref{eq:multiclass_ERM} as predicted bound. Additionally, we also plot the predicted lower bound by the error on mislabeled data which nevertheless were predicted as true label. We refer to this as ``Oracle bound''. See text for more details. 
    % 
    % except for the stopping point. 
    % The bound predicted by RATT (RHS in \eqref{eq:multiclass_ERM}) is vacuous. 
    }\label{fig:error_epoch_CIFAR10}
    % \vspace{-15pt}
\end{figure}


\begin{figure}[h]
    \centering 
    % \vspace{-15pt}
    % \includegraphics[width=0.9\linewidth]{example-image-a}
    \subfigure[MLP]{\includegraphics[width=0.3\linewidth]{figures/MNIST-FNN.pdf}} \hfil
    \subfigure[ResNet]{\includegraphics[width=0.3\linewidth]{figures/MNIST-Resnet.pdf}}
    % \includegraphics[width=0.9\linewidth]{figures/{CIFAR10_rn=0.1_lr=0.2_wd=0.005}.png}
    % \vspace{-10pt}
    \caption{ Per epoch curves for MNIST corresponding results in \tabref{table:multiclass}. As before, we just plot the dominating term in the RHS of \eqref{eq:multiclass_ERM} as predicted bound. Additionally, we also plot the predicted lower bound by the error on mislabeled data which nevertheless were predicted as true label. We refer to this as ``Oracle bound''. See text for more details. 
    % 
    % except for the stopping point. 
    % The bound predicted by RATT (RHS in \eqref{eq:multiclass_ERM}) is vacuous. 
    }\label{fig:error_epoch_MNIST}
    % \vspace{-15pt}
\end{figure}

\textbf{Results on CIFAR 100 {} {}} 
% 
On CIFAR100, our bound in \eqref{eq:multiclass_ERM} yields vacous bounds. However, the oracle bound as explained above yields tight guarantees in the initial phase of the learning (i.e., when learning rate is less than $0.1$) (\figref{fig:error_CIFAR100}).  

\begin{figure}[h]
    \centering 
    % \vspace{-15pt}
    % \includegraphics[width=0.9\linewidth]{example-image-a}
    \includegraphics[width=0.3\linewidth]{figures/CIFAR100-Resnet.pdf}
    % \includegraphics[width=0.9\linewidth]{figures/{CIFAR10_rn=0.1_lr=0.2_wd=0.005}.png}
    % \vspace{-10pt}
    \caption{ Predicted lower bound by the error on mislabeled data which nevertheless were predicted as true label with ResNet18 on CIFAR100. We refer to this as ``Oracle bound''. See text for more details. 
    % 
    % except for the stopping point. 
    The bound predicted by RATT (RHS in \eqref{eq:multiclass_ERM}) is vacuous. 
    }\label{fig:error_CIFAR100}
    % \vspace{-15pt}
\end{figure}


% \paragraph{Experiments on CIFAR100} 


% \subsection{Model Selection using RATT}


\subsection{Hyperparameter Details}


\textbf{\figref{fig:error_CIFAR10} {} {}} We use clean training dataset of size $40,000$. We fix the amount of unlabeled data at $20\%$ of the clean size, i.e. we include additional $8,000$ points with randomly assigned labels. We use test set of $10,000$ points. For both MLP and ResNet, we use SGD with an initial learning rate of $0.1$ and momentum $0.9$. We fix the weight decay parameter at $5\times 10^{-4}$. After $100$ epochs, we decay the learning rate to $0.01$. We use SGD batch size of $100$. 

\textbf{\figref{fig:error_binary} (a) {} {}} We obtain a toy dataset according to the process described in \secref{sec:app_dataset}. We fix $d=100$ and create a dataset of $50,000$ points with balanced classes. Moreover, we sample additional covariates with the same procedure to create randomly labeled dataset. For both SGD and GD training, we use a fixed learning rate $0.1$.    

\textbf{\figref{fig:error_binary} (b) {} {}} Similar to binary CIFAR, we use clean training dataset of size $40,000$ and fix the amount of unlabeled data at $20\%$ of the clean dataset size. To train wide nets, we use a fixed learning of $0.001$ with GD and SGD. We decide the weight decay parameter and the early stopping point that maximizes our generalization bound (i.e. without peeking at unseen data ).  We use SGD batch size of $100$. 

\textbf{\figref{fig:error_binary} (c) {} {}} With IMDb dataset, we use a clean dataset of size $20,000$ and as before, fix the amount of unlabeled data at $20\%$ of the clean data. To train ELMo model, we use Adam optimizer with a fixed learning rate $0.01$ and weight decay $10^{-6}$ to minimize cross entropy loss. We train with batch size $32$ for 3 epochs. To fine-tune BERT model, we use Adam optimizer with learning rate $5\times 10^{-5}$ to minimize cross entropy loss. We train with a batch size of $16$ for 1 epoch.    

\textbf{\tabref{table:multiclass} {} {}} For multiclass datasets, we train both MLP and ResNet with the same hyperparameters as described before. We sample a clean training dataset of size $40,000$ and fix the amount of unlabeled data at $20\%$ of the clean size. We use SGD with an initial learning rate of $0.1$ and momentum $0.9$. We fix the weight decay parameter at $5\times 10^{-4}$. After $30$ epochs for ResNet and after $50$ epochs for MLP, we decay the learning rate to $0.01$.  We use SGD with batch size $100$. 
For \figref{fig:error_CIFAR100}, we use the same hyperparameters as 
CIFAR10 training, except we now decay learning rate after $100$ epochs. 


In all experiments, to identify the best possible accuracy on just the clean data, we use the exact same set of hyperparamters except the stopping point. We choose a stopping point that maximizes test performance. 

\subsection{Summary of experiments }

\begin{center}
    \begin{table}[H] 
        \centering
        \begin{tabular}{|c|c|c|c|} 
        \hline
        Classification type & Model category & Model & Dataset  \\ [0.5ex] 
        \hline
        \hline
        \multirow{10}{*}{Binary} & Low dimensional & Linear model & Toy Gaussain dataset  \\
                        \cline{2-4}
                         & Overparameterized 
                        %  & Linear model & Toy Gaussain dataset \\
                        %  \cline{3-4}
                        %  & & 2-layer wide net& Toy Gaussain dataset \\
                        %  \cline{3-4}
                         & \multirow{2}{*}{2-layer wide net} & \multirow{2}{*}{Binary MNIST} \\
                         & linear nets & &  
                         \\
                         \cline{2-4}                 
                         & \multirow{6}{*}{Deep nets} & \multirow{2}{*}{MLP} & Binary MNIST \\
                         \cline{4-4}
                         & &  & Binary CIFAR \\
                         \cline{3-4}
                         &  & \multirow{2}{*}{ResNet} & Binary MNIST \\
                         \cline{4-4}
                         & &  & Binary CIFAR \\
                         \cline{3-4}
                         &  & ELMo-LSTM model & IMDb Sentiment Analysis \\
                         \cline{3-4}
                         & & BERT pre-trained model & IMDb Sentiment Analysis \\
        \hline
        \multirow{5}{*}{Multiclass} & \multirow{5}{*}{Deep nets} & \multirow{2}{*}{MLP} & MNIST \\
                        \cline{4-4} 
                        & & & CIFAR10 \\                   
                        \cline{3-4}
                         &   & \multirow{3}{*}{ResNet} & MNIST \\
                         \cline{4-4}
                         &   & & CIFAR10 \\
                         \cline{4-4}
                         &   & & CIFAR100 \\
        \hline
        \end{tabular}
        % \caption{Summary of experiments performed} \label{table:experiments}
    \end{table}    
    % \footnotetext[6]{We use both MSE loss and cross-entropy loss.}
    % \footnotetext[6]{We try 2 variants: one with a fixed first layer and the other with both layers trainable.}
\end{center}

\newpage
\section{Proof of \lemref{lem:stability_error}} \label{app:proof_lem_error}

\begin{proof}[Proof of \lemref{lem:stability_error}]
    Recall, we have a training set $S \cup \wt S_C$. We defined leave-one-out error on mislabeled points as $$\error_{\text{LOO}(\wt S_M) } = \frac{\sum_{(x_i, y_i) \in \wt S_M} \error( f_{(i)}( x_i), y_i)}{ \abs{\wt S_M }} \,, $$
    where $f_{(i)} \defeq f(\calA, (S \cup \wt S)_{(i)})$. Define $S^\prime \defeq S \cup \wt S$. Assume $(x,y)$ and $(x^\prime,y^\prime)$ as i.i.d. samples from ${\calDm}$. 
    Using Lemma 25 in \citet{bousquet2002stability}, we have
    \begin{align*}
        \Expo{ \left( \error_{\calDm}(\wh f) -\error_{\text{LOO}(\wt S_M) } \right)^2 } \le & \Expt{ S^\prime, (x,y), (x^\prime,y^\prime) }{ \error(\wh f(x), y ) \error(\wh f(x^\prime), y^\prime )} - 2 \Expt{ S^\prime, (x,y) }{ \error(\wh f(x), y ) \error(f_{(i)}(x_i), y_i )} \\
        & + \frac{m_1-1}{m_1}\Expt{ S^\prime }{  \error(f_{(i)}(x_i), y_i )  \error(f_{(j)}(x_j), y_j )} + \frac{1}{m_1} \Expt{ S^\prime }{  \error(f_{(i)}(x_i), y_i ) } \,. \numberthis \label{eq:main_reln}
    \end{align*}
    We can rewrite the equation above as : 
    \begin{align*}
        \Expo{ \left( \error_{\calDm}(\wh f) -\error_{\text{LOO}(\wt S_M) } \right)^2 } \le &  \, \underbrace{\Expt{ S^\prime, (x,y), (x^\prime,y^\prime) }{ \error(\wh f(x), y ) \error(\wh f(x^\prime), y^\prime ) - \error(\wh f(x), y ) \error(f_{(i)}(x_i), y_i )}}_{\RN{1}} \\
        & + \underbrace{\Expt{ S^\prime }{  \error(f_{(i)}(x_i), y_i )  \error(f_{(j)}(x_j), y_j ) -  \error(\wh f(x), y ) \error(f_{(i)}(x_i), y_i )}}_{\RN{2}} \\ &+ \underbrace{\frac{1}{m_1} \Expt{ S^\prime }{  \error(f_{(i)}(x_i), y_i ) - \error(f_{(i)}(x_i), y_i )  \error(f_{(j)}(x_j), y_j ) }}_{\RN{3}} \,. \numberthis \label{eq:main_reln2}
    \end{align*}
    
    We will now bound term $\RN{3}$.  Using Cauchy-Schwarz's inequality, we have
    
    \begin{align}
        \Expt{ S^\prime }{  \error(f_{(i)}(x_i), y_i ) - \error(f_{(i)}(x_i), y_i )  \error(f_{(j)}(x_j), y_j ) }^2 &\le  \Expt{ S^\prime }{  \error(f_{(i)}(x_i), y_i ) }^2 \Expt{S^\prime}{1 -   \error(f_{(j)}(x_j), y_j ) }^2 \\
        &\le \frac{1}{4} \,.\label{eq:term1_lem12}
    \end{align}
    
    Note that since $(x_i,y_i)$, $(x_j ,y_j )$, $(x,y)$, and $(x^\prime, y^\prime)$ are all from same distribution $\calDm$, we directly incorporate the bounds on term $\RN{1}$ and $\RN{2}$ from the proof of Lemma 9 in \citet{bousquet2002stability}. Combining that with \eqref{eq:term1_lem12} and our definition of hypothesis stability in \codref{cond:hypothesis_stability}, we have the required claim. 
    
    
    % We now re-write term $\RN{1}$ as
    % \begin{align*}
    %         &\Expt{S^\prime, (x,y), (x^\prime,y^\prime) }{ \error(\wh f(x), y ) \error(\wh f(x^\prime), y^\prime ) - \error(\wh f(x), y ) \error(f_{(i)}(x_i), y_i )} \\ & \qquad = \Expt{ S^\prime, (x,y), (x^\prime,y^\prime) }{ \error(\wh f(x), y ) \error(\wh f  (x^\prime), y^\prime ) - \error(\wh f ^\prime(x), y ) \error(f_{(i)}(x^\prime), y^\prime )} \tag{Exchanging $(x_i, y_i)$ with $(x^\prime, y^\prime)$ in the second term} \\
    %         & \qquad = \Expt{ S^\prime, (x,y), (x^\prime,y^\prime) }{  \left(\error(\wh f(x), y )-  \error(f_{(i)}(x), y ) \right) \error(\wh f  (x^\prime), y^\prime )  } \\
    %         & \qquad  + \Expt{ S^\prime, (x,y), (x^\prime,y^\prime) }{  \left(\error(f_{(i)}(x), y ) -\error(\wh f ^\prime(x), y ) \right) \error(\wh f  (x^\prime), y^\prime )}  \\
    %         & \qquad +\Expt{ S^\prime, (x,y), (x^\prime,y^\prime) }{  \left( \error(\wh f  (x^\prime), y^\prime ) -  \error(f_{(i)}(x^\prime), y^\prime ) \right) \error(\wh f ^\prime(x), y ) }  \,, \numberthis \label{eq:term1_final}
    % \end{align*}
    % where $\wh f^\prime$ is the classifier obtained by training on $ S^\prime_{(i)} \cup \{ (x^\prime, y^\prime) \} $. Similarly we can re-write term $\RN{2}$ as 
    % \begin{align*}
    %     & \Expt{ S^\prime }{  \error(f_{(i)}(x_i), y_i )  \error(f_{(j)}(x_j), y_j ) -  \error(\wh f(x), y ) \error(f_{(i)}(x_i), y_i )} \\
    %     &\quad  = \Expt{ S^\prime, (x,y), (x^\prime,y^\prime)}{  \error(f^{\prime\prime}_{(i)}(x), y )  \error(f_{(j)}^{\prime}(x^\prime), y^\prime ) -  \error(\wh f(x), y ) \error(f_{(i)}(x_i), y_i )} \tag{Exchanging $(x_i, y_i)$ with $(x, y)$ and $(x_j, y_j)$ with $(x^\prime, y^\prime)$ in the first term}\\
    %     &\quad = \Expt{ S^\prime, (x,y), (x^\prime,y^\prime)}{  \error(f^{\prime\prime}_{(j)}(x), y )  \error(f_{(i)}^{\prime}(x^\prime), y^\prime ) -  \error(\wh f^\prime (x), y ) \error(f^\prime_{(j)}(x^\prime), y^\prime )} \tag{Exchanging $(x_i, y_i)$ and $(x_j, y_j)$ and then replacing $(x_j, y_j)$ with $(x^\prime, y^\prime)$ in the second term} \\
    %     & \quad = \Expt{ S^\prime, (x,y), (x^\prime,y^\prime) }{  \left( \error(f_{(i)}^{\prime}(x^\prime), y^\prime )   -  \error(\wh f^{\prime\prime}  (x^\prime), y^\prime ) \right)  \error(f^{\prime\prime}_{(j)}(x), y )   } \\
    %     & \quad  + \Expt{ S^\prime, (x,y), (x^\prime,y^\prime) }{  \left( \error(f^{\prime\prime}_{(j)}(x), y )  -\error(\wh f ^\prime(x), y ) \right) \error(\wh f^{\prime\prime}  (x^\prime), y^\prime )  }  \\
    %     & \quad+ \Expt{ S^\prime, (x,y), (x^\prime,y^\prime) }{  \left( \error(\wh f^{\prime\prime}  (x^\prime), y^\prime )  -  \error(f^\prime_{(j)}(x^\prime), y^\prime ) \right)  \error(\wh f^\prime (x), y ) }   \\
    %     & \quad = \Expt{ S^\prime, (x,y), (x^\prime,y^\prime) }{  \left( \error(f_{(i)}^{\prime}(x^\prime), y^\prime )   -  \error(\wh f (x^\prime), y^\prime ) \right)  \error(f_{(i)}(x_j), y_j )   } \\
    %     & \quad  + \Expt{ S^\prime, (x,y), (x^\prime,y^\prime) }{  \left( \error(f^{\prime\prime}_{(j)}(x), y )  -\error(\wh f (x), y ) \right) \error(\wh f^{\prime\prime}  (x_j), y_j )  }  \\
    %     & \quad+ \Expt{ S^\prime, (x,y), (x^\prime,y^\prime) }{  \left( \error(\wh f^{\prime\prime}  (x^\prime), y^\prime )  -  \error(f^\prime_{(j)}(x^\prime), y^\prime ) \right)  \error(\wh f^\prime (x^\prime), y^\prime ) }  \,, \numberthis \label{eq:term2_final}
    % \end{align*}
    % where $f^{\prime\prime}_{(j)}$ is trained on $S^\prime_{(j,i)} \cup {(x,y)}$, $f^{\prime}_{(i)}$ is trained on $S^\prime_{(j,i)} \cup {(x^\prime,y^\prime)}$, and $\wh f^{\prime\prime} $ is trained on $S^\prime_{(j)} \cup {(x,y)}$. Note in the last line we replaced $(x,y)$ by $(x_j, y_j)$ in the first term, replaced $(x^\prime,y^\prime)$ by $(x_j, y_j)$ in the second term and exchanged $(x_i,y_i)$ with $(x_j,y_j)$ and also $(x,y)$ and $(x^\prime, y^\prime)$
    
    
\end{proof}


% 
% 16th Century Version Control 
% 

% \onecolumn

% \section*{Supplementary Material}
% We will be using the following standard results
% on exponential concentration of random variables 
% all throughout the discussion:

% \begin{lemma}[Hoeffding's inequality for independent RVs~\citep{hoeffding1994probability}] Let $Z_1, Z_2, \ldots, Z_n$ be independent bounded random variables with $Z_i \in [a,b]$ for all $i$, then 
%     \begin{align*}
%         \prob\left( \frac{1}{n} \sum_{i=1}^n (Z_i - \Expo{Z_i}) \ge t \right) \le \exp{\left( -\frac{2nt^2}{(b-a)^2} \right) }
%     \end{align*} 
%     and 
%     \begin{align*}
%         \prob\left( \frac{1}{n} \sum_{i=1}^n (Z_i - \Expo{Z_i}) \le -t \right) \le \exp{\left( -\frac{2nt^2}{(b-a)^2} \right) }
%     \end{align*} 
%     for all $t \ge 0$. 
% \end{lemma}

% \begin{lemma}[Hoeffding's inequality for sampling with replacement~\citep{hoeffding1994probability}] \label{lem:hoeffding_sampling} Let $\calZ = (Z_1, Z_2, \ldots, Z_N)$ be a finite population of $N$ points with $Z_i \in [a.b]$ for all $i$. Let $X_1, X_2, \ldots X_n$ be a random sample drawn without replacement from $\calZ$. Then for all $t \ge 0$, we have 
%     \begin{align*}
%         \prob\left( \frac{1}{n} \sum_{i=1}^n (X_i - \mu ) \ge t \right) \le \exp{\left( -\frac{2nt^2}{(b-a)^2} \right) }
%     \end{align*} 
%     and 
%     \begin{align*}
%         \prob\left( \frac{1}{n} \sum_{i=1}^n (X_i - \mu ) \le -t \right) \le \exp{\left( -\frac{2nt^2}{(b-a)^2} \right) } \,,
%     \end{align*} 
%     where $\mu = \frac{1}{N} \sum_{i=1}^{N} Z_i$. 
% \end{lemma}

% We now discuss one condition that generalizes the exponential concentration to dependent random variables.
% \begin{condition}[Bounded difference inequality] \label{cond:BDC} Let $\calZ$ be some set and $\phi: \calZ^n \to \Real$. We say that $\phi$ satisfies the bounded difference assumption if 
% there exists $c_1, c_2, \ldots c_n \ge 0$ s.t. for all $i$, we have 
% \begin{align*}
%     \sup_{Z_1,Z_2, \ldots,Z_n, Z_i^\prime in \calZ^{n+1} } \abs{\phi (Z_1, \ldots, Z_i, \ldots, Z_n ) - \phi (Z_1, \ldots, Z_i^\prime, \ldots, Z_n ) } \le c_i \,.
% \end{align*} 
% \end{condition}

% \begin{lemma}[McDiarmid’s inequality~\citep{mcdiarmid1989}] \label{lem:McDiarmid} Let $Z_1, Z_2, \ldots, Z_n$ be independent random variables on set $\calZ$ and $\phi : \calZ^n \to \Real$ satisfy bounded difference assumption (\codref{cond:BDC}). Then for all $t>0$, we have 
%     \begin{align*}
%         \prob\left( \phi(Z_1, Z_2, \ldots, Z_n) - \Expo{\phi(Z_1, Z_2, \ldots, Z_n)} \ge t \right) \le \exp{\left( -\frac{2t^2}{\sum_{i=1}^n c_i^2} \right) } 
%     \end{align*} 
%     and 
%     \begin{align*}
%         \prob\left( \phi(Z_1, Z_2, \ldots, Z_n) - \Expo{\phi(Z_1, Z_2, \ldots, Z_n)} \le -t \right) \le \exp{\left( -\frac{2t^2}{\sum_{i=1}^n c_i^2} \right) } \,
%     \end{align*} 
% \end{lemma}


% \section{Proofs from \secref{sec:ERM_training}}\label{app:proof_erm}

% \textbf{Additional notation {} {}} Let $m_1$ be the number of mislabeled points ($\wt S_M$) and $m_2$ be the number of correctly labeled points ($\wt S_C$). Note $m_1 + m_2 = m$. 


% \subsection{Proof of \thmref{thm:error_ERM}}


% \begin{proof}[Proof of \lemref{lem:fit_mislabeled}] 
%     The main idea of our proof is to regard 
%     the clean portion of the data 
%     ($S \cup \wt S_C$) as fixed.   
%     Then, there exists a classifier $f^*$ 
%     that is optimal over draws 
%     of the mislabeled data $\wt S_M$. 
% % 
%     % 
%     Formally, 
%     \begin{align}
%     f^* \defeq \argmin_{f \in \calF} \error_{\widecheck {\calD}} (f) \,, \label{eq:modified_ERM}
%     \end{align}
%     where $$\widecheck \calD = \frac{n}{m+n} \calS + \frac{m_1}{m+n} \wt \calS_C  + \frac{m_2}{m+n}\calDm \,.$$ That is, $\widecheck \calD$ a combination of 
%     the \emph{empirical distribution} 
%     over correctly labeled data $S \cup \wt S_C$
%     % in $S\cup \wt S$ 
%     and the (population) distribution 
%     over mislabeled data $\calDm$.
%     Recall that 
%     \begin{align}
%     \wh f \defeq \argmin_{f \in \calF} \error_{\calS \cup \wt S} (f) \,. \label{eq:orig_ERM}
%     \end{align}
%     % 
%     % 
%     Since, $\widehat f$ minimizes 0-1 error 
%     on $S \cup \wt S$, using ERM optimality on \eqref{eq:orig_ERM},  
%     we have 
%     \begin{align}
%         \error_{\calS \cup \wt \calS}(\widehat f) \le \error_{
%             \calS \cup \wt \calS}(f^*) \,.    \label{eq:step1}
%     \end{align}
%     Moreover, since $f^*$ is independent of $\wt S_M$, using Hoeffding's bound,
%     % \footnote{For a fully rigorous argument,
%     % refer to the complete proof in App.~\ref{app:proof_erm}.} 
%     we have with probability at least $1-\delta$ that
%     \begin{align}
%       \error_{\wt \calS_M}(f^*) \le \error_{ \calDm}(f^*) +  \sqrt{\frac{\log(1/\delta)}{2 m_1}} \,. \label{eq:step2} 
%     \end{align}
%     %$ 
%     %for some constant $c_1\le 1/2$. 
%     Finally, since $f^*$ is the optimal classifier on $\widecheck \calD$, 
%     we have 
%     \begin{align}
%         \error_{\widecheck \calD}(f^*) \le \error_{\widecheck \calD}(\widehat f) \label{eq:step3}
%     \end{align}
%      Now to relate \eqref{eq:step1} and \eqref{eq:step3}, we can re-write the \eqref{eq:step2} as follows: 
%     \begin{align}
%         \error_{\calS \cup \wt\calS}(f^*) \le \error_{ \widecheck \calD}(f^*) +  \frac{m_1}{m+n}\sqrt{\frac{\log(1/\delta)}{2 m_1}} \,. \label{eq:step4} 
%     \end{align}
%     Now we combine equations \eqref{eq:step1}, \eqref{eq:step4}, and \eqref{eq:step3}, to get 
%     \begin{align}
%         \error_{\calS \cup \wt \calS}(\wh f) \le \error_{\widecheck \calD}(\wh f) +  \frac{m_1}{m+n}\sqrt{\frac{\log(1/\delta)}{2 m_1}} \,, 
%     \end{align}
%     which implies 
%     \begin{align}
%         \error_{ \wt \calS_M}(\wh f) \le \error_{\calDm}(\wh f) + \sqrt{\frac{\log(1/\delta)}{2 m_1}} \,. \label{eq:lemma1_final}
%     \end{align}
%     Since $\wt S$ is obtained by randomly labeling an unlabeled dataset, we assume $2m_1 \approx m$ \footnote{Formally, with probability at least $1-\delta$, we have  $(m - 2m_1)\le \sqrt{m\log(1/\delta)/2}$ }. Moreover, using $\error_{\calDm} = 1 - \error_{\calD}$ we obtain the desired result.   
%     % Combining the above steps and using the fact 
%     % that $\error_\calD = 1- \error_{\calDm} $, 
%     % we obtain the desired result.
% \end{proof}

% \begin{proof}[Proof of \lemref{lem:mislabeled_error}]
%     Recall $\error_{\wt S} (f) = \frac{m_1}{m} \error_{\wt S_M}(f) + \frac{m_2}{m} \error_{\wt S_C}(f)$. Hence, we have 
%     \begin{align}
%         2\error_{\wt S}(f) - \error_{\wt S_M}(f) - \error_{\wt S_C}(f) &= \left(\frac{2m_1}{m} \error_{\wt S_M}(f) - \error_{\wt S_M}(f)\right) + \left(\frac{2m_2}{m} \error_{\wt S_C}(f) - \error_{\wt S_C}(f)\right) \\ &= \left(\frac{2m_1}{m} - 1\right) \error_{\wt S_M}(f) + \left(\frac{2m_2}{m} - 1 \right)\error_{\wt S_C} (f) \,.
%     \end{align} 
%     Since the dataset is randomly labeled, with probability at least $1-\delta$, we have  $\left(\frac{2m_1}{m} - 1\right) \le \sqrt{\frac{\log(1/\delta)}{2m}}$. Similarly, we have with probability at least $1-\delta$, $\left(\frac{2m_2}{m} - 1\right) \le \sqrt{\frac{\log(1/\delta)}{2m}}$. Using union bound, we have with probability at least $1-\delta$
%     % \begin{align}
%     %     2\error_{\wt S} - \error_{\wt S_M}(f) - \error_{\wt S_C}(f) \le \sqrt{\frac{\log(2/\delta)}{2m}} \left(\error_{\wt S_M}(f) + \error_{\wt S_C}(f) \right) \le 2\sqrt{\frac{\log(2/\delta)}{2m}} \,. \label{eq:lemma2_final}
%     % \end{align}
%     \begin{align}
%         2\error_{\wt S} - \error_{\wt S_M}(f) - \error_{\wt S_C}(f) \le \sqrt{\frac{\log(2/\delta)}{2m}} \left(\error_{\wt S_M}(f) + \error_{\wt S_C}(f) \right) \,. \label{eq:lemma2_prefinal}
%     \end{align}
%     With re-arranging $\error_{\wt S_M}(f) + \error_{\wt S_C}(f)$ and using the inequality $ 1- a\le \frac{1}{1+a} $, we have  
%     \begin{align}
%         2\error_{\wt S} - \error_{\wt S_M}(f) - \error_{\wt S_C}(f) \le 2\error_{\wt \calS} \sqrt{\frac{\log(2/\delta)}{2m}}  \,. \label{eq:lemma2_final}
%     \end{align}

%     % We obtain the desired result by using 
% \end{proof}

% \begin{proof}[Proof of \lemref{lem:clear_error}]
% % Recall 0-1 error on each point  $(x,y) \in S \cup \wt S$ is given by $\I{ f(x)\ne y}$.
% In the set of correctly labeled points $S \cup \wt S_C$, we have $S$ as a random subset of $S \cup \wt S_C$. Hence, using Hoeffding's inequality for sampling without replacement (\lemref{lem:hoeffding_sampling}), we have with probability at least $1-\delta$
% \begin{align}
%     \error_{\wt \calS_c} (\wh f)- \error_{\calS \cup \wt \calS_C}( \wh f) \le  \sqrt{\frac{\log(1/\delta)}{2m_2}} \,.
% \end{align}
% Re-writing $\error_{\calS \cup \wt \calS_C}( \wh f)$ as $\frac{m_2}{m_2 + n} \error_{\wt \calS_C }(\wh f) + \frac{n}{m_2 + n} \error_{\calS }(\wh f)$, we have with probability at least $1-\delta$
% \begin{align}
%   \left(\frac{n}{n+m_2}\right) \left(\error_{\wt \calS_c} (\wh f)- \error_{\calS}( \wh f) \right) \le  \sqrt{\frac{\log(1/\delta)}{2m_2}} \,.
% \end{align}
% As before, assuming $2m_2 \approx m$, we have with probability at least $1-\delta$ 
% \begin{align}
%     \error_{\wt \calS_c} (\wh f)- \error_{\calS}( \wh f) \le \left(1+\frac{m_2}{n}\right)  \sqrt{\frac{\log(1/\delta)}{m}} \le 1.5 \sqrt{\frac{\log(1/\delta)}{m}} \,. \label{eq:lemma3_final}
% \end{align} 
% \end{proof}

% \begin{proof}[Proof of \thmref{thm:error_ERM}] 
%     Having established these core intermediate results, we can now combine above three lemmas to prove the main result. 
%     In particular, we bound the population error on clean data ($\error_\calD(\wh f)$) as follows:  
%     \begin{enumerate}[(i)]
%         \item First, use \eqref{eq:lemma1_final}, to obtain an upper bound on the population error on clean data, i.e., with probability at least $1-\delta/4$, we have
%         \begin{align}
%             \error_{ \calD} (\wh f) \le 1 - \error_{ \wt \calS_M}(\wh f) + \sqrt{\frac{\log(4/\delta)}{m}} \,. 
%         \end{align}
%         \item  Second, use \eqref{eq:lemma2_final}, to relate the error on the mislabeled fraction with error on clean portion of randomly labeled data and error on whole randomly labeled dataset, i.e., with probability at least $1-\delta/2$, we have 
%         \begin{align}
%             - \error_{\wt S_M}(f) \le \error_{\wt S_C}(f) - 2\error_{\wt S}  + \sqrt{\frac{\log(4/\delta)}{2m}}  \,. 
%         \end{align} 
%         \item Finally, use \eqref{eq:lemma3_final} to relate the error on the clean portion of randomly labeled data and error on clean training data, i.e., with probability $1-\delta/4$, we have 
%         \begin{align}
%             \error_{\wt \calS_C} (\wh f)\le - \error_{\calS}( \wh f) + \left(1 + \frac{m}{2n} \right) \sqrt{\frac{\log(4/\delta)}{m}} \,. 
%         \end{align} 
%     \end{enumerate}

%     Using union bound on the above three steps, we have with probability at least $1-\delta$: 
%     \begin{align}
%         \error_\calD (\wh f) \le \error_{\calS}(\wh f)   + 1 - 2\error_{\wt \calS}(\wh f)   + (1/\sqrt{2} + 2.5)  \sqrt{\frac{\log(4/\delta)}{m}} \,.
%     \end{align}
%     Note that $(1/\sqrt{2} + 2.5)$ is a loose constant. In experiments, we use the ratio $\frac{m}{n}$
%     %  the exact error $\error_{\wt \calS}(\wh f)$ 
%     to evaluate R.H.S.    
% \end{proof}

% \subsection{Proof of \propref{prop:rademacher}}

% \begin{proof}[Proof of \propref{prop:rademacher}]
%     For a classifier $ f: \calX \to \{-1, 1\}$, we have $1 - 2\,\indict{ f(x) \ne y} = y \cdot f(x)$. Hence, by definition of $\error$, we have 
%     \begin{align}
%         1 -2\error_{\wt \calS}(f) = \frac{1}{m}\sum_{i=1}^m y_i \cdot f(x_i) \le \sup_{f \in \calF} \, \frac{1}{m} \sum_{i=1}^m y_i \cdot f(x_i)  \,. \label{eq:error_rademacher}
%     \end{align}
%     Note that for fixed inputs $(x_1, x_2, \ldots, x_m)$ in $\wt S$, $(y_1, y_2, \ldots y_m)$ are random labels. Define $\phi_1 (y_1, y_2, \ldots, y_m) \defeq \sup_{f \in \calF} \, \frac{1}{m} \sum_{i=1}^m y_i \cdot f(x_i)$. We have the following bounded difference condition on $\phi_1$. For all i, 
%     \begin{align}
%         \sup_{y_1, \ldots y_m, y_i^\prime \in \{-1, 1\}^{m+1} } \abs{ \phi_1 (y_1,\ldots, y_i, \ldots, y_m) - \phi_1 (y_1,\ldots, y_i^\prime, \ldots, y_m)  } \le 1/m \,. \label{cond1_rademacher}
%     \end{align} 
    
%     Similarly define $\phi_2 (x_1, x_2, \ldots, x_m) \defeq \Expt{ y_i \sim_U \{-1, 1\}  }{ \sup_{f \in \calF} \, \frac{1}{m}  \sum_{i=1}^m y_i \cdot f(x_i)}$. We have the following bounded difference condition on $\phi_2$. For all i,
%     \begin{align}
%         \sup_{x_1, \ldots x_m, x_i^\prime \in \calX^{m+1} } \abs{ \phi_2 (x_1,\ldots, x_i, \ldots, x_m) - \phi_1 (x_1,\ldots, x_i^\prime, \ldots, x_m)  } \le 1/m \,. \label{cond2_rademacher}
%     \end{align}
%     Using McDiarmid’s inequality (\lemref{lem:McDiarmid}) twice with Condition \eqref{cond1_rademacher} and \eqref{cond2_rademacher}, with probability at least $1-\delta$, we have
%     \begin{align}
%         \sup_{f \in \calF} \, \frac{1}{m} \sum_{i=1}^m y_i \cdot f(x_i)  - \Expt{x,y}{\sup_{f \in \calF} \, \frac{1}{m} \sum_{i=1}^m y_i \cdot f(x_i) } \le \sqrt{\frac{2\log(2/\delta)}{m}} \label{eq:final_rademacher}
%     \end{align} 
%     Combining \eqref{eq:error_rademacher} and \eqref{eq:final_rademacher}, we obtain the desired result. 
% \end{proof}


% \subsection{Proof of \thmref{thm:error_regularized_ERM}}

% Proof of \thmref{thm:error_regularized_ERM} follows similar to the proof of \thmref{thm:error_ERM}. Note that the same results in \lemref{lem:fit_mislabeled}, \lemref{lem:mislabeled_error}, and \lemref{lem:clear_error} hold in the regularized ERM case. However, the arguments in the proof of \lemref{lem:fit_mislabeled} changes slightly. Hence, we state and prove a lemma parallel to \lemref{lem:fit_mislabeled} for completeness. 

% \begin{lemma} \label{lem:lemma1_reg}
%     Assume the same setup as \thmref{thm:error_regularized_ERM}. 
%     Then for any $\delta >0$, with probability at least  $1-\delta$ 
%     over the random draws of mislabeled data $\wt S_M$, we have 
%     \begin{align}
%         \error_\calD(\widehat f)  \le 1 -\error_{\wt \calS_M}(\widehat f) + \sqrt{\frac{\log(1/\delta)}{m}}\,. 
%     \end{align} 
% \end{lemma}
% \begin{proof}
%     The main idea of the proof remains the same, i.e. regard 
%     the clean portion of the data 
%     ($S \cup \wt S_C$) as fixed.   
%     Then, there exists a classifier $f^*$ 
%     that is optimal over draws 
%     of the mislabeled data $\wt S_M$. 

    
%     Formally, 
%     \begin{align}
%     f^* \defeq \argmin_{f \in \calF} \error_{\widecheck {\calD}} (f)  + \lambda R(f) \,, \label{eq:modified_ERM_reg}
%     \end{align}
%     where $$\widecheck \calD = \frac{n}{m+n} \calS + \frac{m_1}{m+n} \wt \calS_C  + \frac{m_2}{m+n}\calDm \,.$$ That is, $\widecheck \calD$ a combination of 
%     the \emph{empirical distribution} 
%     over correctly labeled data $S \cup \wt S_C$
%     % in $S\cup \wt S$ 
%     and the (population) distribution 
%     over mislabeled data $\calDm$.
%     Recall that 
%     \begin{align}
%     \wh f \defeq \argmin_{f \in \calF} \error_{\calS \cup \wt S} (f) + \lambda R(f) \,. \label{eq:orig_ERM_reg}
%     \end{align}
%     % 
%     % 
%     Since, $\widehat f$ minimizes 0-1 error 
%     on $S \cup \wt S$, using ERM optimality on \eqref{eq:orig_ERM},  
%     we have 
%     \begin{align}
%         \error_{\calS \cup \wt \calS}(\widehat f) + \lambda R(\wh f) \le \error_{
%             \calS \cup \wt \calS}(f^*) + \lambda R(f^*) \,.    \label{eq:step1_reg}
%     \end{align}
%     Moreover, since $f^*$ is independent of $\wt S_M$, using Hoeffding's bound,
%     % \footnote{For a fully rigorous argument,
%     % refer to the complete proof in App.~\ref{app:proof_erm}.} 
%     we have with probability at least $1-\delta$ that
%     \begin{align}
%       \error_{\wt \calS_M}(f^*) \le \error_{ \calDm}(f^*) +  \sqrt{\frac{\log(1/\delta)}{2 m_1}} \,. \label{eq:step2_reg} 
%     \end{align}
%     %$ 
%     %for some constant $c_1\le 1/2$. 
%     Finally, since $f^*$ is the optimal classifier on $\widecheck \calD$, 
%     we have 
%     \begin{align}
%         \error_{\widecheck \calD}(f^*) + \lambda R(f^*) \le \error_{\widecheck \calD}(\widehat f) + \lambda R(\wh f) \label{eq:step3_reg}
%     \end{align}
%      Now to relate \eqref{eq:step1_reg} and \eqref{eq:step3_reg}, we can re-write the \eqref{eq:step2_reg} as follows: 
%     \begin{align}
%         \error_{\calS \cup \wt\calS}(f^*) \le \error_{ \widecheck \calD}(f^*) +  \frac{m_1}{m+n}\sqrt{\frac{\log(1/\delta)}{2 m_1}} \,. \label{eq:step4_reg} 
%     \end{align}
%     After adding $\lambda R(f^*)$ on both sides in \eqref{eq:step4_reg}, we combine equations \eqref{eq:step1_reg}, \eqref{eq:step4_reg}, and \eqref{eq:step3_reg}, to get 
%     \begin{align}
%         \error_{\calS \cup \wt \calS}(\wh f) \le \error_{\widecheck \calD}(\wh f) +  \frac{m_1}{m+n}\sqrt{\frac{\log(1/\delta)}{2 m_1}} \,, 
%     \end{align}
%     which implies 
%     \begin{align}
%         \error_{ \wt \calS_M}(\wh f) \le \error_{\calDm}(\wh f) + \sqrt{\frac{\log(1/\delta)}{2 m_1}} \,. \label{eq:lemma_reg_final}
%     \end{align}
%     Similar as before, since $\wt S$ is obtained by randomly labeling an unlabeled dataset, we assume 
%     $2m_1 \approx m$. Moreover, using $\error_{\calDm} = 1 - \error_{\calD}$ we obtain the desired result. 
% \end{proof}
% % \begin{proof}[Proof of ]
    
% % \end{proof}

% \subsection{Proof of \thmref{thm:multiclass_ERM}}

% We first state and prove lemmas parallel to three lemmas used in the proof of balanced binary case. Then we combine the results in the three lemmas to obtain the result in \thmref{thm:multiclass_ERM}. 

% Before stating the result, we define mislabeled distribution $\calDm$ for any $\calD$. While $\calDm$ and $\calD$ share 
% the same marginal distribution over $\calX$, 
% the distribution over labels $y$ 
% given an input $x\sim \calD_\calX$ is changed.
% In particular, for any $x$, the pdf over $y$ is changed to:  
% $p_{\calDm} (\cdot \vert x) \defeq \frac{1 - p_{\calD}(\cdot \vert x)}{k - 1}$.

% \begin{lemma} \label{lem:fit_mislabeled_multi}
%     Assume the same setup as \thmref{thm:multiclass_ERM}. 
%     Then for any $\delta >0$, with probability at least  $1-\delta$ 
%     over the random draws of mislabeled data $\wt S_M$, we have 
%     \begin{align}
%         \error_\calD(\widehat f)  \le (k-1)\left(1 -\error_{\wt \calS_M}(\widehat f)\right) + (k-1)\sqrt{\frac{\log(1/\delta)}{m}}\,. \label{eq:lemma1_multi}
%     \end{align}   
% \end{lemma} 

% \begin{proof}
%     The main idea of the proof remains the same, i.e. regard 
%     the clean portion of the data 
%     ($S \cup \wt S_C$) as fixed. 
%     Then, there exists a classifier $f^*$ 
%     that is optimal over draws 
%     of the mislabeled data $\wt S_M$. 
    
%     However, we need to be careful while relating population error on mislabeled data with population accuracy on clean data.   
%     While for binary classification,  we could upper bound $\error_{\wt \calS_M}$ 
%     with $1-\error_\calD$  (in the proof of \lemref{lem:fit_mislabeled}), 
%     for multiclass classification, 
%     error on the mislabeled data 
%     and accuracy on the clean data 
%     in the population 
%     are not so directly related.  
%     To establish \eqref{eq:lemma1_multi},
%     we break the error on the 
%     (unknown) mislabeled data 
%     into two parts: one term corresponds 
%     to predicting the true label on mislabeled data, 
%     and the other corresponds to predicting 
%     neither the true label 
%     nor the assigned (mis-)label.  
%     Finally, we relate these errors to their
%     population counterparts to establish \eqref{eq:lemma1_multi}. 
    
%     Formally, 
%     \begin{align}
%     f^* \defeq \argmin_{f \in \calF} \error_{\widecheck {\calD}} (f)  + \lambda R(f) \,, \label{eq:modified_ERM_reg2}
%     \end{align}
%     where $$\widecheck \calD = \frac{n}{m+n} \calS + \frac{m_1}{m+n} \wt \calS_C  + \frac{m_2}{m+n}\calDm \,.$$ That is, $\widecheck \calD$ a combination of 
%     the \emph{empirical distribution} 
%     over correctly labeled data $S \cup \wt S_C$
%     % in $S\cup \wt S$ 
%     and the (population) distribution 
%     over mislabeled data $\calDm$.
%     Recall that 
%     \begin{align}
%     \wh f \defeq \argmin_{f \in \calF} \error_{\calS \cup \wt S} (f) + \lambda R(f) \,. \label{eq:orig_ERM_reg2}
%     \end{align}
%     % 
%     % 
%     Following the exact steps from the proof of \lemref{lem:lemma1_reg}, with probability at least $1-\delta$, we have  
%     \begin{align}
%         \error_{ \wt \calS_M}(\wh f) \le \error_{\calDm}(\wh f) + \sqrt{\frac{\log(1/\delta)}{2 m_1}} \,. \label{eq:lemma1_final_multi_prev}
%     \end{align}
%     Similar to before, since $\wt S$ is obtained by randomly labeling an unlabeled dataset, we assume 
%     $\frac{k}{k-1} m_1 \approx m$. 
    
%     Now we will relate $\error_\calDm (\wh f)$ with $\error_{\calD}(\wh f)$. Let $y^T$ denote the (unknown) true label for a mislabeled point $(x, y)$ (i.e., label before replacing it with a mislabel). 
%     \begin{align}    
%          \Expt{(x, y) \in \sim \calDm}{\indict{ \wh f(x) \ne y }}  &= \underbrace{\Expt{(x, y) \in \sim \calDm}{\indict{ \wh f(x) \ne y \land \wh f(x) \ne y^T}}}_{\RN{1}} + \underbrace{\Expt{(x, y) \in \sim \calDm}{\indict{ \wh f(x) \ne y \land \wh f(x) = y^T}}}_{\RN{2}} \,. \label{eq:excess_term}
%     \end{align}
%     Clearly, term 2 is one minus the accuracy on the clean unseen data, i.e. 
%     \begin{align}
%         \RN{2} = 1 - \Expt{{x,y} \sim \calD}{ \indict{ \wh f(x) \ne y}} = 1- \error_{\calD}(\wh f) \,. \label{eq:term1}    
%     \end{align}
%     Next, we  relate term 1 with the error on the unseen clean data. We show that term 1 is equal to the error on the unseen clean data scaled by $\frac{k-2}{k-1}$ where $k$ is the number of labels. Using the definition of mislabeled distribution $\calDm$,  we have 
%     \begin{align}
%         \RN{1} = \frac{1}{k-1} \left( \Expt{(x, y) \in \sim \calD}{ \sum_{i \in \calY \land i\ne y}  \indict{ \wh f(x) \ne i \land \wh f(x) \ne y}} \right) = \frac{k-2}{k-1} \error_{\calD}(\wh f) \,.\label{eq:term2}
%     \end{align}    

%     Combining the result in \eqref{eq:term1}, \eqref{eq:term2} and \eqref{eq:excess_term}, we have 
%     \begin{align}
%         \error_{\calDm}(\wh f) = 1- \frac{1}{k-1} \error_{\calD}(\wh f) \,.\label{eq:combine_terms}
%     \end{align}
%     Finally, combining the result in \eqref{eq:combine_terms} with equation \eqref{eq:lemma1_final_multi_prev}, we have with probability $1-\delta$, 
%     \begin{align}
%       \error_{\calD}(\wh f) \le  (k-1) \left( 1- \error_{ \wt \calS_M}(\wh f) \right)  + (k-1) \sqrt{\frac{k \log(1/\delta)}{ 2(k-1)m}} \,. \label{eq:lemma1_final_multi}
%     \end{align}
% \end{proof}

% \begin{lemma} \label{lem:mislabeled_error_multi}
%     Assume the same setup as \thmref{thm:multiclass_ERM}.  Then for any $\delta >0$, with probability at least $1-\delta$ over the random draws of $\wt S$, we have  
%     % \begin{align}
%         $$\abs{k\error_{\wt \calS}(\widehat f) - \error_{\wt \calS_C}(\widehat f) -  (k-1)\error_{\wt \calS_M}(\widehat f) } \le  2k\sqrt{\frac{\log(4/\delta)}{2m}}\,. $$ % \label{eq:lemma2}
%     % \end{align}   
%     %  for some constant $c_3 \le 1.0\,$.
% \end{lemma} 


% \begin{proof}
%     Recall $\error_{\wt S} (f) = \frac{m_1}{m} \error_{\wt S_M}(f) + \frac{m_2}{m} \error_{\wt S_C}(f)$. Hence, we have 
%     \begin{align}
%         k\error_{\wt S}(f) - (k-1)\error_{\wt S_M}(f) - \error_{\wt S_C}(f) &= (k-1)\left(\frac{k m_1}{(k-1) m} \error_{\wt S_M}(f) - \error_{\wt S_M}(f)\right) + \left(\frac{km_2}{m} \error_{\wt S_C}(f) - \error_{\wt S_C}(f)\right) \\ &= k \left[ \left(\frac{m_1}{m} - \frac{k-1}{k}\right) \error_{\wt S_M}(f) + \left(\frac{m_2}{m} - \frac{1}{k} \right) \error_{\wt S_C} (f) \right] \,.
%     \end{align} 
%     Since the dataset is randomly labeled, we have with probability at least $1-\delta$, $\left(\frac{m_1}{m} - \frac{k-1}{k}\right) \le \sqrt{\frac{\log(1/\delta)}{2m}}$. Similarly, we have with probability at least $1-\delta$, $\left(\frac{m_2}{m} - \frac{1}{k}\right) \le \sqrt{\frac{\log(1/\delta)}{2m}}$. Using union bound, we have with probability at least $1-\delta$
%     % \begin{align}
%     %     2\error_{\wt S} - \error_{\wt S_M}(f) - \error_{\wt S_C}(f) \le \sqrt{\frac{\log(2/\delta)}{2m}} \left(\error_{\wt S_M}(f) + \error_{\wt S_C}(f) \right) \le 2\sqrt{\frac{\log(2/\delta)}{2m}} \,. \label{eq:lemma2_final}
%     % \end{align}
%     \begin{align}
%         k\error_{\wt S}(f) - (k-1)\error_{\wt S_M}(f) - \error_{\wt S_C}(f)  \le k \sqrt{\frac{\log(2/\delta)}{2m}} \left(\error_{\wt S_M}(f) + \error_{\wt S_C}(f) \right) \,. \label{eq:lemma2_final_multi}
%     \end{align}

%     % We obtain the desired result by using 
% \end{proof}

% \begin{lemma} \label{lem:clear_error_multi}
%     Assume the same setup as \thmref{thm:multiclass_ERM}. 
%     Then for any $\delta >0$, with probability at least $1-\delta$ 
%     over the random draws of $\wt S_C$ and $S$, we have 
%     % \begin{align}
%         $$\abs{\error_{\wt \calS_C}(\widehat f) - \error_{\calS}(\widehat f) } \le 1.5 \sqrt{\frac{k\log(2/\delta)}{2m}}\,.$$ %\label{eq:lemma3}
%     % \end{align}   
%     % for some constant $c_2 \le 1.2\,$.
% \end{lemma} 
% \begin{proof}
%     % Recall 0-1 error on each point  $(x,y) \in S \cup \wt S$ is given by $\I{ f(x)\ne y}$.
%     In the set of correctly labeled points $S \cup \wt S_C$, we have $S$ as a random subset of $S \cup \wt S_C$. Hence, using Hoeffding's inequality for sampling without replacement (\lemref{lem:hoeffding_sampling}), we have with probability at least $1-\delta$
%     \begin{align}
%         \error_{\wt \calS_c} (\wh f)- \error_{\calS \cup \wt \calS_C}( \wh f) \le  \sqrt{\frac{\log(1/\delta)}{2m_2}} \,.
%     \end{align}
%     Re-writing $\error_{\calS \cup \wt \calS_C}( \wh f)$ as $\frac{m_2}{m_2 + n} \error_{\wt \calS_C }(\wh f) + \frac{n}{m_2 + n} \error_{\calS }(\wh f)$, we have with probability at least $1-\delta$
%     \begin{align}
%       \left(\frac{n}{n+m_2}\right) \left(\error_{\wt \calS_c} (\wh f)- \error_{\calS}( \wh f) \right) \le  \sqrt{\frac{\log(1/\delta)}{2m_2}} \,.
%     \end{align}
%     As before, assuming $km_2 \approx m$, we have with probability at least $1-\delta$ 
%     \begin{align}
%         \error_{\wt \calS_c} (\wh f)- \error_{\calS}( \wh f) \le \left(1+\frac{m_2}{n}\right)  \sqrt{\frac{k\log(1/\delta)}{2m}} \le \left( 1 + \frac{1}{k}\right) \sqrt{\frac{k\log(1/\delta)}{2m}} \,. \label{eq:lemma3_final_multi}
%     \end{align} 
% \end{proof}

% \begin{proof}[Proof of \thmref{thm:multiclass_ERM}] 
%     Having established these core intermediate results, we can now combine above three lemmas. 
%     In particular, we bound the population error on clean data ($\error_\calD(\wh f)$) as follows:  
%     \begin{enumerate}[(i)]
%         \item First, use \eqref{eq:lemma1_final_multi}, to obtain an upper bound on the population error on clean data, i.e., with probability at least $1-\delta/4$, we have
%         \begin{align}
%             \error_{ \calD} (\wh f) \le (k-1)\left(1 - \error_{ \wt \calS_M}(\wh f) \right) + (k-1) \sqrt{\frac{k\log(4/\delta)}{2(k-1)m}} \,. 
%         \end{align}
%         \item  Second, use \eqref{eq:lemma2_final_multi}, to relate the error on the mislabeled fraction with error on clean portion of randomly labeled data and error on whole randomly labeled dataset, i.e., with probability at least $1-\delta/2$, we have 
%         \begin{align}
%             - (k-1)\error_{\wt S_M}(f) \le \error_{\wt S_C}(f) - k\error_{\wt S}  + k\sqrt{\frac{\log(4/\delta)}{2m}}  \,. 
%         \end{align} 
%         \item Finally, use \eqref{eq:lemma3_final_multi} to relate the error on the clean portion of randomly labeled data and error on clean training data, i.e., with probability $1-\delta/4$, we have 
%         \begin{align}
%             \error_{\wt \calS_C} (\wh f)\le - \error_{\calS}( \wh f) + \left(1 + \frac{m}{kn} \right) \sqrt{\frac{k\log(4/\delta)}{2m}} \,. 
%         \end{align} 
%     \end{enumerate}

%     Using union bound on the above three steps, we have with probability at least $1-\delta$: 
%     \begin{align}
%         \error_\calD (\wh f) \le \error_{\calS}(\wh f) + (k-1) - k\error_{\wt \calS}(\wh f)   + (\sqrt{k(k-1)} + k + \sqrt{k} + \frac{m}{n\sqrt{k}})  \sqrt{\frac{\log(4/\delta)}{2m}} \,.
%     \end{align}
%     % Note that $\frac{m}{n\sqrt{k}}$ is much smaller than the other terms in addition. Hence, we ignore this in the final bound. 
%     % Note that $(1/\sqrt{2} + 2.5)$ is a loose constant. In experiments, we use the ratio $\frac{m}{n}$
%     %  the exact error $\error_{\wt \calS}(\wh f)$ 
%     % to evaluate R.H.S.    
% \end{proof}

% \newpage
% \section{Proofs from \secref{sec:linear_models}}\label{app:proof_gd}

% We suppose that the parameters of the linear function 
% are obtained via gradient descent on 
% the following $L_2$ regularized problem: 
% \begin{align}
%     % n in denominator is avoided deliberately
%     \calL_S(w; \lambda) \defeq \sum_{i=1}^n{(w^Tx_i - y_i)^2} + \lambda \norm{w}{2}^2 \,, \label{eq:l2_MSE_app}   
% \end{align}
% where $\lambda\ge0$ is a regularization parameter. 
% We assume access to a clean dataset 
% $S = \{(x_i, y_i)\}_{i=1}^n \sim \calD^n$ 
% and randomly labeled dataset 
% $\wt S = \{(x_i, y_i)\}_{i=n+1}^{n+m} \sim \wt \calD^m$. 
% Let $\bX = [x_1, x_2, \cdots, x_{m+n}]$ 
% and $\by = [y_1, y_2, \cdots, y_{m+n}]$. 
% Fix a positive learning rate $\eta$ such that 
% $\eta \le 1/\left(\norm{\bX^T\bX}{\text{op}} + \lambda^2\right)$ 
% and an initialization $w_0 = 0$. 
% % \todos{Assumption made for simplicty}. 
% Consider the following gradient descent iterates 
% to minimize objective \eqref{eq:l2_MSE_app} on $S \cup \wt S$:
% \begin{align}
% w_t = w_{t-1} - \eta \grad_w \calL_{S \cup \wt S} (w_{t-1}; \lambda) \quad \forall t=1,2,\ldots \label{eq:GD_iterates_app}
% \end{align} 
% Then we have $\{ w_t\}$ converge to the limiting solution 
% $\wh w = \left( \bX^T\bX+\lambda \boldsymbol{I}\right)^{-1}\bX^T\by$. Define $\widehat f (x) \defeq f(x ; \wh w) $.  

% \subsection{\textcolor{red}{Errata}}

% We wish to correct the following error in the body: \codref{cond:error_stability} is not enough to guarantee the result in \thmref{thm:linear}. We now present a slightly stronger condition called \emph{hypothesis stability} under which we obtain a result similar to \thmref{thm:linear}. 

% This error doesn't change the main arguments of the proof where we show that the empirical train error is less than or equal to the leave-one-out error. We need a stronger condition to relate leave-one-out error with the population error of the original classifier. Specifically, while \codref{cond:error_stability} relates the average population error of leave-one-out classifiers with the population error of the original classifier, we need the new condition to show the concentration of the empirical leave-one-out error and  average population error of leave-one-out classifiers. 
% % main takeaway 

% Note that the new condition, while being stronger than the previous one, still doesn't imply generalization~\cite{bousquet2002stability,elisseeff2003leave,abou2019exponential}. Overall, the main results in \secref{sec:ERM_training} and takeaways of the paper remain unaffected by the error.  

% We now present the new condition and a corrected statement of \thmref{thm:linear}. Recall, for a given training set $S \sim \calD^n $, 
% we use $S_{(i)}$ to denote the training set $S$ 
% with the $i^{\text{th}}$ point removed.

% \begin{condition}[Hypothesis Stability] 
%     \label{cond:hypothesis_stability}
%     We have $\beta$ hypothesis stability 
%     if our training algorithm $\calA$ satisfies the following: 
%     \begin{align*}
%     % ${\sum_{i=1}^n \frac{\error_{\calD}( f(\calA, S_{(i)}))}{n} - \error_\calD(f(\calA, S))} \le \beta\,$.
%     \forall i \in \{1,2,\ldots, n\}, \quad  \Expt{\calS, (x,y) \in \calD}{ \abs{\error\left( f(x) ,y  \right) - \error\left( f_{(i)}(x), y \right) }} \le \frac{\beta}{n} \,,
%     \end{align*}
%     where $f_{(i)} \defeq f(\calA, S_{(i)})$ and $ f \defeq f(\calA, S)$.
% \end{condition}

% \begin{theorem}[Correct statement of \thmref{thm:linear}] \label{thm:new_linear}
%     Assume that this gradient descent algorithm satisfies \codref{cond:hypothesis_stability}
%     with $\beta=\calO(1)$.  
%     Then for any $\delta >0$, with probability at least $1-\delta$ 
%     over the random draws of datasets $\wt S$ and $S$, we have:
%     \begin{align}
%         \error_\calD(\widehat f) \le \error_\calS(\widehat f) + 1 - 2 \error_{\wt\calS}(\widehat f) + \left(\frac{1}{\sqrt{2}} + 1.5 \right) \sqrt{\frac{\log(4/\delta)}{m}} + \sqrt{\frac{4}{\delta}\left(\frac{1}{m} +\frac{3\beta}{m+n} \right)}  \,. \label{eq:gd_error}
%     \end{align} 
%     % for some constant $c\le 3.2$.
% \end{theorem}

% \subsection{Proof of \thmref{thm:new_linear}}
% We use a standard result from linear algebra, namely Shermann-Morrison formula~\citep{sherman1950adjustment} for matrix inversion:  

% \begin{lemma}[\citet{sherman1950adjustment}] \label{lem:sherman}
%     Suppose $\bA \in \Real^{n \times n}$ is an invertible square matrix and $u,v \in \Real^n$ are column vectors. Then $\bA + uv^T$ is invertible iff $1 + v^T \bA u \ne 0$ and in particular
%     \begin{align}
%         (\bA + u v^T)^{-1} = \bA^{-1}  - \frac{\bA^{-1} uv^T \bA^{-1} }{ 1 + v^T \bA^{-1} u} \,.
%     \end{align}   
% \end{lemma}
% \newcommand\byy[1]{\by_{\left(#1\right)}}
% \newcommand\bXX[1]{\bX_{\left(#1\right)}}
% \newcommand\ff[1]{\wh f_{\left(#1\right)}}

% For a given training set $S \cup \wt S_C$, define leave-one-out error on mislabeled points in the training data as $$\error_{\text{LOO}(\wt S_M) } = \frac{\sum_{(x_i, y_i) \in \wt S_M} \error( f_{(i)}( x_i), y_i)}{ \abs{\wt S_M }} \,, $$
% where $f_{(i)} \defeq f(\calA, (S \cup \wt S)_{(i)})$. To relate empirical leave-one-out error and population error with hypothesis stability condition, we use the following lemma:   

% \begin{lemma}[\citet{bousquet2002stability}] \label{lem:stability_error}
%     For the leave-one-out error, we have
%     \begin{align}
%         \Expo{ \left( \error_{\calDm}(\wh f) -\error_{\text{LOO}(\wt S_M) } \right)^2 } \le \frac{1}{2m_1}+  \frac{3\beta}{n + m}\,.
%     \end{align}   
%     % where $ f \defeq f(\calA, S \cup \wt S) $.
% \end{lemma}

% Proof of the above lemma is similar to the proof of  Lemma 9 in \citet{bousquet2002stability} and can be found in \appref{app:proof_lem_error}. 
% % 
% % Before presenting the result, we introduce some notation. 
% Before presenting the proof of \thmref{thm:new_linear}, we introduce some more notation. Let $\bX_{(i)}$ denote the matrix of covariates with $i^{\text{th}}$ point removed. Similarly let $\by_{(i)}$ be the array of responses with $i^{\text{th}}$ point removed. Define the corresponding regularized GD solution as $\wh w_{(i)} = \left( \bXX{i}^T\bXX{i}+\lambda \boldsymbol{I}\right)^{-1}\bXX{i}^T\byy{i}$. Define $\ff{i}(x) \defeq f(x ; \wh w_{(i)}) $.

% \begin{proof}[Proof of \thmref{thm:new_linear}]
%     Because squared loss minimization does not imply 0-1 error minimization, we cannot use arguments from \lemref{lem:fit_mislabeled}. This is the main technical difficulty. To compare the 0-1 error at a train point with an unseen point, 
%     we use the closed-form expression for $\widehat{w}$ and Shermann-Morrison formula to upper bound training error with leave-one-out cross validation error. 
    
%     The proof is divided into three parts: In part one, we show that 0-1 error on mislabeled points in the training set is lower than the error obtained by leave-one-out error at those points. In part two, we relate this leave-one-out error with the population error on mislabeled distribution using \codref{cond:hypothesis_stability}. While the empirical leave-one-out error is unbiased estimator of the average population error of leave-one-out classifiers, we need hypothesis stability to control the variance of empirical leave-one-out error. Finally in part three, we show that the error on the mislabeled training points can be estimated with just the randomly labeled and  clean training data (as in proof of \thmref{thm:error_ERM}).  

%     \textbf{Part 1 {} {}} First we relate training error with leave-one-out error.        
%     For any 
%     training point $(x_i, y_i)$ in $\wt S \cup S$, we have 
%     \begin{align}
%         \error(\wh f(x_i), y_i ) &= \indict{ y_i \cdot x_i^T \wh w < 0 } = \indict{ y_i \cdot x_i^T \left( \bX^T\bX+\lambda \boldsymbol{I}\right)^{-1}\bX^T\by < 0 } \\
%         &= \indict{ y_i \cdot x_i^T \underbrace{\left( \bXX{i}^T\bXX{i} + x_i ^T x_i +\lambda \boldsymbol{I}\right)^{-1}}_{\RN{1}} (\bXX{i}^T\byy{i} + y \cdot x_i) < 0 }
%     \end{align}
%     Letting $\bA = \left(\bXX{i}^T\bXX{i} +\lambda \boldsymbol{I}\right)$ and using \lemref{lem:sherman} on term 1, we have 
%     \begin{align}
%         \error(\wh f(x_i), y_i ) &= \indict{ y_i \cdot x_i^T \left[\bA^{-1} -  \frac{\bA^{-1} x_i x_i^T \bA^{-1}}{ 1 + x_i ^T \bA^{-1} x_i } \right] (\bXX{i}^T\byy{i} + y \cdot x_i) < 0 } \\
%         &= \indict{ y_i \cdot\left[ \frac{ x_i^T \bA^{-1} ( 1 + x_i ^T \bA^{-1} x_i ) -  x_i^T \bA^{-1} x_i x_i^T \bA^{-1}}{ 1 + x_i ^T \bA ^{-1}x_i } \right] (\bXX{i}^T\byy{i} + y \cdot x_i) < 0 } \\
%         &= \indict{ y_i \cdot\left[ \frac{ x_i^T \bA^{-1}}{ 1 + x_i ^T \bA ^{-1}x_i } \right] (\bXX{i}^T\byy{i} + y \cdot x_i) < 0 } \,.
%     \end{align}

%     Since $1 + x_i^T \bA^{-1} x_i > 0$, we have 
%     \begin{align}
%         \error(\wh f(x_i), y_i ) &= \indict{ y_i \cdot x_i^T \bA^{-1} (\bXX{i}^T\byy{i} + y \cdot x_i) < 0 } \\
%         &= \indict{ x_i^T \bA^{-1} x_i +  y_i \cdot x_i^T \bA^{-1} (\bXX{i}^T\byy{i}) < 0 } \\
%         &\le \indict{ y_i \cdot x_i^T \bA^{-1} (\bXX{i}^T\byy{i}) < 0 } = \error(\ff{i}(x_i), y_i ) \,.\label{eq:LOO_error}
%     \end{align}

%     Using \eqref{eq:LOO_error}, we have 
%     \begin{align}
%         \error_{\wt \calS_M } (\wh f) \le \error_{\text{LOO} (S_M)} \defeq \frac{\sum_{(x_i, y_i) \in \wt S_M} \error(\ff{i}(x_i), y_i ) }{\abs{\wt \calS_M}}\label{eq:LOO_error_final}
%     \end{align}
%     \textbf{Part 2 {}{}} We now relate RHS in \eqref{eq:LOO_error_final} with the population error on mislabeled distribution. To do this, we leverage \codref{cond:hypothesis_stability} and \lemref{lem:stability_error}. In particular, we have 

%     \begin{align}
%         \Expt{\calS \cup \wt \calS_M }{ \left(\error_{\calDm}(\wh f) - \error_{\text{LOO} (S_M)}\right)^2 } \le \frac{1}{2m_1} + \frac{3\beta}{m+n} \,.
%     \end{align}

%     Using Chebyshev's inequality, with probability at least $1-\delta$, we have 
%     \begin{align}
%         \error_{\text{LOO} (S_M)} \le  \error_{\calDm}(\wh f)   + \sqrt{\frac{1}{\delta}\left(\frac{1}{2m_1} +\frac{3\beta}{m+n} \right)} \,. \label{eq:final_mislabeled_linear}
%     \end{align}
    

%     \textbf{Part 3 {}{}} Combining \eqref{eq:final_mislabeled_linear} and \eqref{eq:LOO_error_final}, we have 

%     \begin{align}
%         \error_{\wt \calS_M } (\wh f) \le \error_{\calDm}(\wh f)   + \sqrt{\frac{1}{\delta}\left(\frac{1}{2m_1} +\frac{3\beta}{m+n} \right)} \,. \label{eq:linear_parallel_lem1}
%     \end{align}

%     Compare \eqref{eq:linear_parallel_lem1}, with \eqref{eq:lemma1_final} in the proof of \lemref{lem:fit_mislabeled}. We obtain a similar relationship between $\error_{\wt \calS_M }$ and $\error_{\calDm}$ but with a polynomial concentration instead of exponential concentration. 
%     In addition, since we just use concentration arguments to relate mislabeled error with the error on clean portion and unlabeled portion, we can directly use the results in \lemref{lem:mislabeled_error} and \lemref{lem:clear_error}. Therefore, combining results in \lemref{lem:mislabeled_error}, \lemref{lem:clear_error}, and \eqref{eq:linear_parallel_lem1} with union bound, we have with probability at least $1-\delta$

%     \begin{align}
%         \error_\calD(\widehat f) \le \error_\calS(\widehat f) + 1 - 2 \error_{\wt\calS}(\widehat f) + \left(\frac{1}{\sqrt{2}} + 1.5 \right) \sqrt{\frac{\log(4/\delta)}{m}} + \sqrt{\frac{4}{\delta}\left(\frac{1}{m} +\frac{3\beta}{m+n} \right)}  \,.
%     \end{align}
    

       
% \end{proof}

% \subsection{Discussion on \codref{cond:hypothesis_stability}}

% The quantity in LHS of \codref{cond:hypothesis_stability} measures how much the function learned by the algorithm (in terms of error on unseen point) will change when one point in the training set is removed. 
% % Discussion on exponential concentration and stronger condition. 
% Notice that hypothesis stability implies error stability, i.e., \codref{cond:error_stability} ~\cite{bousquet2002stability}.  In summary, while error stability allowed us to relate the average population error of the leave-one-out classifiers with the population error of the original classifier, we need hypothesis stability condition to control the variance of the empirical leave-one-out error. 

% Additionally, we note that while the dominating term in the RHS of \thmref{thm:new_linear} matches with the dominating term in ERM bound in \thmref{thm:error_ERM}, there is a polynomial concentration term (dependence on $1/\delta$ instead of $\log(\sqrt{1/\delta})$) in  \thmref{thm:new_linear}. 
% Since with hypothesis stability, we just bound the variance,  the polynomial concentration is due to the use of Chebyshev's inequality instead of an exponential tail inequality (as in \lemref{lem:fit_mislabeled}).
% Recent works have highlighted that slightly stronger condition than hypothesis stability can be used to obtained an exponential concentration for leave-one-out error~\citep{abou2019exponential}, but we leave this for future work for now. 
% % We leave 
% % However, the constants 

% % we also want to highlight  

% \subsection{Formal statement and proof of  of \propref{prop:early_stop}}

% Before formally presenting the result, we will introduce some notation.  By $\calL_{S}(w)$, we denote 
% the objective in \eqref{eq:l2_MSE_app} with $\lambda=0$. 
% Assume Singular Value Decomposition (SVD) of $\bX$  as $\sqrt{n} \bU \bS^{1/2} \bV^T$. Hence $\bX^T \bX = \bV \bS \bV^T$.
% Consider the GD iterates defined in \eqref{eq:GD_iterates_app}. 
% % 
% We now derive closed form expression for the $t^\text{th}$ iterate of gradient descent:  
% % 
% \begin{align}
%     w_t = w_{t-1} + \eta \cdot \bX^T (\by - \bX w_{t-1}) = (\bI - \eta \bV \bS \bV^T )w_{k-1} + \eta \bX^T \by \,.
% \end{align}
% Rotating by $\bV^T$, we get 
% \begin{align}
%     \wt w_t = (\bI - \eta\bS )\wt w_{k-1} + \eta \wt \by \,, \label{eq:GD_recur}
% \end{align}
% where $\wt w_t = \bV^T w_t $ and $\wt \by = \bV^T \bX^T \by$. Assuming the initial point $w_0 = 0$ and applying the recursion in \eqref{eq:GD_recur}, we get
% \begin{align}
%     \wt w_t = \bS ^{-1} ( \bI - (\bI - \eta \bS)^k ) \wt \by \,, 
% \end{align} 
% Projecting solution back to the original space, we have 
% \begin{align}
%      w_t = \bV \bS ^{-1} ( \bI - (\bI - \eta \bS)^k ) \bV^T \bX^T \by \,, 
% \end{align} 
% % We will work with this GD solution at any iterate $t$ in the next proposition. 
% Define $f_t(x) \defeq f(x;w_t)$ as the solution at the $t^{\text{th}}$ iterate. 
% Let $\wt w_{\lambda} = \argmin_{w} \calL_\calS (w;\lambda) = (\bX^T \bX + \lambda \bI)^{-1} \bX^T \by = \bV (\bS + \lambda \bI )^{-1} \bV^T \bX^T \by $. 
% % ) \,,$ for all $t=1,2,\ldots\,.$ 
% and define $\wt f_\lambda(x) \defeq f(x;\wt w_\lambda)$ as the regularized solution. 
% Assume $\kappa$ be the condition number of the population covariance matrix 
% and 
% let $s_\text{min}$ be the minimum positive singular value of the empirical covariance matrix. Our proof idea is inspired from recent work on relating gradient flow solution and regularized solution for regression problems \citep{ali2018continuous}. We will use the following lemma in the proof: 
% \begin{lemma} \label{lem:ineq_soln}
%     For all $x \in [0,1]$ and for all $ k \in \mathbb{N}$, we have (a) $ \frac{kx}{1+kx} \le 1- (1-x)^k$ and (b) $ 1- (1-x)^k \le 2 \cdot \frac{kx}{kx+1} $.
%     %  where $g(c)$ is a constant dependent on $c$. For $c = 1$, $g(c) = 2.0$.   
% \end{lemma}
% \begin{proof}
%     % [Proof of \lemref{lem:ineq_soln}]
%     % Part (a) is easy. 
%     Using $ (1-x)^k \le \frac{1}{1+kx}$, we have part (a). For part (b), we numerically maximize $\frac{ (1+kx ) (1 - (1-x)^k) }{kx}$ for all $k\ge 1$ and for all $x \in [0, 1]$.  
% \end{proof}

% % 
% % Next, 

% \begin{prop}[Formal statement of \propref{prop:early_stop}] \label{prop:formal_early_stop}
% Let $\lambda = \frac{1}{t\eta}$. For a training point $x$, we have 
% \begin{align*}
%     \Expt{x \sim \calS}{(f_t(x) - \wt f_\lambda(x))^2} &\le c(t,\eta) \cdot \Expt{x \sim \calS}{f_t(x)^2} \,, %\label{eq:early_stop}
% \end{align*}
% where $c(t, \eta) \defeq \min( 0.25, \frac{1}{s_\text{min}^2 t^2 \eta^2})$. Similarly for a test point, we have 
% \begin{align*}
%     \Expt{x \sim \calD_\calX}{(f_t(x) - \wt f_\lambda(x))^2} &\le \kappa \cdot c(t,\eta) \cdot \Expt{x \sim \calD_\calX}{f_t(x)^2} \,. %\label{eq:early_stop}
% \end{align*}
% \end{prop} 

% \begin{proof}
%     %%%%%%%%%%%%% 
%     We want to analyze the expected squared difference output of regularized linear regression with regularization constant $\lambda = \frac{1}{\eta t}$ and gradient descent solution at $t^\text{th}$ iterate. We separately expand the algebraic expression for squared difference at a training point and a test point. 
%     % We start by considering the difference  
%     Then the main step is to show that  $\left[ \bS ^{-1} ( \bI - (\bI - \eta \bS)^k )  - (\bS + \lambda \bI )^{-1}\right] \preceq c(\eta, t) \cdot \bS ^{-1} ( \bI - (\bI - \eta \bS)^k ) $.

%     %%%%%%%%%%%%%
    
%   \textbf{Part 1 {} {}} 
%     First, we will analyze the squared difference of output at a training point (for simplicity, we refer to $S \cup \wt S$ as $S$), i.e. 
%     \begin{align}
%         \Expt{ x \sim \calS }{\left(f_t(x) - \wt f_\lambda (x)\right)^2} &= \norm{\bX w_t - \bX \wt w_\lambda}{2}^2 =   \norm{\bX \bV \bS ^{-1} ( \bI - (\bI - \eta \bS)^t ) \bV^T \bX^T \by - \bX \bV (\bS + \lambda \bI )^{-1} \bV^T \bX^T \by }{2}^2 \\
%         &= \norm{\bX \bV \left(\bS ^{-1} ( \bI - (\bI - \eta \bS)^t ) - (\bS + \lambda \bI )^{-1} \right) \bV^T \bX^T \by  }{2} \\
%         &=  \by^T \bV \bX \left( \underbrace{\bS ^{-1} ( \bI - (\bI - \eta \bS)^t ) - (\bS + \lambda \bI )^{-1}}_{\RN{1}} \right)^2 \bS \bV^T \bX^T \by \label{eq:train_GD_rel}
%         %  (\bX \bV \bS ^{-1} ( \bI - (\bI - \eta \bS)^k ) \bV^T \bX^T \by)^T \bX \bV \bS ^{-1} ( \bI - (\bI - \eta \bS)^k ) \bV^T \bX^T \by
%     \end{align}
%     We now separately consider term 1. Substituting $\lambda = \frac{1}{t \eta}$, we get
%     \begin{align}
%         \bS ^{-1} ( \bI - (\bI - \eta \bS)^t ) - (\bS + \lambda \bI )^{-1} &= \bS^{-1} \left( ( \bI - (\bI - \eta \bS)^t ) - (\bI + \bS^{-1} \lambda )^{-1}\right) \\
%         &= \underbrace{\bS^{-1} \left( ( \bI - (\bI - \eta \bS)^t ) - (\bI + ( \bS t \eta)^{-1}  )^{-1}\right)}_{\bA}
%     \end{align}

%     We now separately bound the diagonal entries in matrix $\bA$. 
%     With $s_i$, we denote $i^{\text{th}}$ diagonal entry of $\bS$. Note that since $ \eta\le 1/\norm{S}{\text{op}}$, for all $i$, $\eta s_i  \le 1$.  Consider $i^{\text{th}}$ diagonal term (which is non-zero) of the diagonal matrix $\bA$, we have 
%     \begin{align}
%         \bA_{ii} = \frac{1}{s_i} \left(  1 - (1 - s_i \eta)^t - \frac{t \eta s_i}{1 + t \eta s_i } \right) &=  \frac{1 - (1 - s_i \eta)^t}{s_i} \left( \underbrace{ 1 - \frac{t \eta s_i}{(1 + t \eta s_i)(1 - (1 - s_i \eta)^t)}}_{\RN{2}} \right) \\ 
%          &\le \frac{1}{2}\left[ \frac{1 - (1 - s_i \eta)^t}{ s_i} \right] \tag*{(Using \lemref{lem:ineq_soln} (b))} \,.
%     \end{align} 
%     Additionally, we can also show the following upper bound on term 2: 
%     \begin{align}
%          1 - \frac{t \eta s_i}{(1 + t \eta s_i)(1 - (1 - s_i \eta)^t)} &= \frac{(1 + t \eta s_i)(1 - (1 - s_i \eta)^t) - t \eta s_i }{(1 + t \eta s_i)(1 - (1 - s_i \eta)^t)} \\
%          & \le  \frac{ 1 -  (1 - s_i \eta)^t - t \eta s_i (1 - s_i \eta)^t}{(1 + t \eta s_i)(1 - (1 - s_i \eta)^t)} \\
%          & \le \frac{1}{t\eta s_i} \,. \tag{Using \lemref{lem:ineq_soln} (a)}
%         %  &\le \frac{1}{2}\left[ \frac{1 - (1 - s_i \eta)^t}{ s_i} \right] \tag*{(Using \lemref{lem:ineq_soln})} \,.
%     \end{align} 

%     Combining both the upper bounds on each diagonal entry $\bA_{ii}$, we have 
%     \begin{align}
%     \bA \preceq c_1(\eta, t) \cdot \bS^{-1} ( \bI - (\bI - \eta \bS)^t ) \,, \label{eq:upperbound_diagonal}
%     \end{align}
%     where $c_1(\eta, t ) = \min(0.5, \frac{1}{t s_i \eta })$. Plugging this into \eqref{eq:train_GD_rel}, we have 
%     \begin{align}
%         \Expt{ x \sim \calS }{\left(f_t(x) - \wt f_\lambda (x)\right)^2} &\le c(\eta, t) \cdot \by^T \bV \bX  \left( \bS^{-1} ( \bI - (\bI - \eta \bS)^t ) \right)^2 \bS \bV^T \bX^T \by \\
%         &=   c(\eta, t) \cdot \by^T \bV \bX  \left( \bS^{-1} ( \bI - (\bI - \eta \bS)^t ) \right) \bS \left( \bS^{-1} ( \bI - (\bI - \eta \bS)^t ) \right) \bV^T \bX^T \by \\
%         & =  c(\eta, t) \cdot \norm{\bX w_t}{2}^2 \\
%         &= c(\eta, t) \cdot  \Expt{ x \sim \calS }{\left(f_t(x) \right)^2} \,,
%     \end{align}
%     where $c(\eta, t ) = \min(0.25, \frac{1}{t^2 s^2_i \eta^2 })$.

%     \textbf{Part 2 {} {}} With $\bSigma$, we denote the underlying true covariance matrix. We now consider the squared difference of output at an unseen point: 
%     \begin{align}
%         \Expt{ x \sim \calD_{\calX} }{\left(f_t(x) - \wt f_\lambda (x)\right)^2} &= \Expt{x \sim \calD_{\calX}}{\norm{x^T w_t - x^T \wt w_\lambda}{2}} \\
%         &=   \norm{x^T \bV \bS ^{-1} ( \bI - (\bI - \eta \bS)^t ) \bV^T \bX^T \by - x^T \bV (\bS + \lambda \bI )^{-1} \bV^T \bX^T \by }{2} \\
%         &= \norm{x^T \bV \left(\bS ^{-1} ( \bI - (\bI - \eta \bS)^t ) - (\bS + \lambda \bI )^{-1} \right) \bV^T \bX^T \by  }{2} \\
%         &= \by^T \bV \bX \left( \bS ^{-1} ( \bI - (\bI - \eta \bS)^t ) - (\bS + \lambda \bI )^{-1} \right) \bV^T \bSigma \bV \\ &\qquad \qquad \qquad \qquad \qquad \left( (\bI - (\bI - \eta \bS)^t ) - (\bS + \lambda \bI )^{-1} \right) \bV^T \bX^T \by \\
%         &\le \sigma_{\text{max}} \cdot \by^T \bV \bX \left( \underbrace{\bS ^{-1} ( \bI - (\bI - \eta \bS)^t ) - (\bS + \lambda \bI )^{-1}}_{\RN{1}} \right)^2 \bV^T \bX^T \by \,, \label{eq:test_GD_rel}
%         %  (\bX \bV \bS ^{-1} ( \bI - (\bI - \eta \bS)^k ) \bV^T \bX^T \by)^T \bX \bV \bS ^{-1} ( \bI - (\bI - \eta \bS)^k ) \bV^T \bX^T \by
%     \end{align}
%     where $\sigma_{\text{max}}$ is the maximum eigenvalue of the underlying covariance matrix $\bSigma$. Using the upper bound on term 1 in \eqref{eq:upperbound_diagonal}, we have 
%     \begin{align}
%         \Expt{ x \sim \calD_{\calX} }{\left(f_t(x) - \wt f_\lambda (x)\right)^2} &\le \sigma_{\text{max}} \cdot c(\eta, t) \cdot \by^T \bV \bX  \left( \bS^{-1} ( \bI - (\bI - \eta \bS)^t ) \right)^2 \bV^T \bX^T \by \\
%         &=   \kappa \cdot c(\eta, t) \cdot \sigma_{\text{min}}\cdot \norm{\bV \left( \bS^{-1} ( \bI - (\bI - \eta \bS)^t ) \right) \bV^T \bX^T \by}{2}^2 \\
%         &\le \kappa \cdot c(\eta, t) \cdot \left[ \bV \left( \bS^{-1} ( \bI - (\bI - \eta \bS)^t ) \right) \bV^T \bX^T \right]^T \bSigma \\
%         &\qquad \qquad \qquad \qquad \qquad \left[ \bV \left( \bS^{-1} ( \bI - (\bI - \eta \bS)^t ) \right) \bV^T \bX^T \right] \by \\
%         & = \kappa \cdot c(\eta, t) \cdot \Expt{x \sim \calD_{\calX}}{\norm{x^T w_t}{2}} \,.
%     \end{align}
% % 
% % 
%     % Since $ \eta\le 1/\norm{S}{\text{op}}$, invoking \lemref{lem:ineq_soln} to upper bound term 1 with
% \end{proof}


% \newpage
% \section{Additional experiments and details}\label{app:exp}
% \newcommand\tab[1][1cm]{\hspace*{#1}}

% \subsection{Datasets} \label{sec:app_dataset}

% \textbf{Toy Dataset {} {}} Assume fixed constants $\mu$ and $\sigma$. For a given label $y$, we simulate features $x$ in our toy classification setup as follows: 
% \begin{align*}
%     x \defeq \texttt{concat} \left[ x_1, x_2\right] \quad \text{where} \quad  x_1 \sim  \calN( y \cdot \mu, \sigma^2 I_{d \times d}) \ \  \text{and} \ \  x_1 \sim  \calN( 0, \sigma^2 I_{d \times d}) \,.
% \end{align*}  
% % where $y$ is the true label and $x$ is the corresponding feature vector. 
% In experiements throughout the paper, we fix dimention $d=100$, $\mu = 1.0 $, and $\sigma = \sqrt{d}$. Intuitively, $x_1$ carries the information about the underlying label and $x_2$ is additional noise independent of the underlying label. 

% \textbf{CV datasets {} {}} We use MNIST~\citep{lecun1998mnist} and CIFAR10~\cite{krizhevsky2009learning}. 
% % For binary tasks, 
% We produce a binary variant from the multiclass classification problem by mapping classes $\{0,1,2,3,4\}$ to label $1$ and $\{ 5,6,7,8,9\}$ to label $-1$. For CIFAR dataset, we also use the standard data augementation of random crop and horizontal flip. PyTorch code is as follows: 

% \texttt{(transforms.RandomCrop(32, padding=4),\\
% \tab transforms.RandomHorizontalFlip())}

% \textbf{NLP dataset {} {}} We use IMDb Sentiment analysis~\citep{maas2011learning} corpus.  

% \subsection{Architecture Details} 

% All experiments were run on NVIDIA GeForce RTX 2080 Ti GPUs. We used PyTorch~\citep{NEURIPS2019a9015} and Keras with Tensorflow~\citep{abadi2016tensorflow} backend for experiments. 
% % , ELMo embeddings~\citep{Peters:2018}, and Hugging Face Transformers~\citep{wolf-etal-2020-transformers}. 

% \textbf{Linear model {} {}} For the toy dataset, we simulate a linear model with scalar output and the same number of parameters as the number of dimensions.   

% \textbf{Wide nets {} {}} To simulate the NTK regime, we experiment with $2-$layered wide nets. The PyTorch code for 2-layer wide MLP is as follows: 


% \texttt{ nn.Sequential( \\
% \tab     nn.Flatten(),\\
% \tab    nn.Linear(input\_dims, 200000, bias=True),\\
% \tab    nn.ReLU(),\\
% \tab    nn.Linear(200000, 1, bias=True)\\
% \tab     )}


% We experiment both (i) with the first layer fixed at random initialization; (ii)  and updating both layers' weights.     

% \textbf{Deep nets for CV tasks {} {}} We consider a 4-layered MLP. The PyTorch code for 4-layer MLP is as follows: 

% \texttt{ nn.Sequential(nn.Flatten(), \\
% \tab        nn.Linear(input\_dim, 5000, bias=True),\\
% \tab        nn.ReLU(),\\
% \tab        nn.Linear(5000, 5000, bias=True),\\
% \tab        nn.ReLU(),\\
% \tab        nn.Linear(5000, 5000, bias=True),\\
% \tab        nn.ReLU(),\\
% % \tab        nn.Linear(5000, 5000, bias=True),\\
% % \tab        nn.ReLU(),\\
% \tab        nn.Linear(1024, num\_label, bias=True)\\
% \tab        )}

% For MNIST, we use $1000$ nodes instead of $5000$ nodes in the hidden layer. 
% % 
% We also experiment with convolutional nets. In particular, we use ResNet18 \citep{he2016deep}. Implementation adapted from:  \url{https://github.com/kuangliu/pytorch-cifar.git}. 

% \textbf{Deep nets for NLP {} {}} We use a simple LSTM model with embeddings intialized with ELMo embeddings~\citep{Peters:2018}. Code adapted from: \url{https://github.com/kamujun/elmo_experiments/blob/master/elmo_experiment/notebooks/elmo_text_classification_on_imdb.ipynb} 

% We also evaluate our bounds with a BERT model. In particular, we fine-tune an off-the-shelf uncased BERT model~\citep{devlin2018bert}. Code adapted from Hugging Face Transformers~\citep{wolf-etal-2020-transformers}: \url{https://huggingface.co/transformers/v3.1.0/custom_datasets.html}. 


% \subsection{Additonal experiments}

% 1. SGD with linear models on cross entropy and MSE loss. 

% 2. CE loss and SGD. GD with MSE loss 

% 3. Binary MNIST with MLP. multiclass MNIST  

% \textbf{Results on CIFAR 10 {} {}} 
% % 
% We plot epoch wise error curve for results in \tabref{table:multiclass}. We observe the same trend as in \figref{fig:error_CIFAR10}. Additionally, we plot an \emph{oracle bound} obtained by tracking the error on mislabeled data which nevertheless were predicted as true label. To obtain an exact emprical value of the oracle bound, we need underlying true labels for the randomly labeled data. 
% % Note that our bound in \thmref{thm:multiclass_ERM}, lower bounds the accuracy as predicted by the oracle bound. 
% While with just access to extra unlabeled data we cannot calculate oracle bound, we note that the oracle bound is very tight and never violated in practice underscoring an importamt aspect of generalization in multiclass problems. This highlight that even a stronger conjecture may hold in multiclass classification, i.e., error on mislabeled data (where nevertheless true label was predicted) lower bounds the population error on the distribution of mislabeled data and hence, the error on (a specific) mislabeled portion predicts the population accuracy on clean data. 
% % 
% On the other hand, the dominating term of in \thmref{thm:multiclass_ERM} is loose when compared with the oracle bound. The main reason, we believe is the pessimistic upper bound in \eqref{eq:lemma1_final_multi_prev} in the proof of \lemref{lem:fit_mislabeled_multi}. We leave an investigation on this gap for future. 
% % of fit 

% % However, oracle bound highlights two . One,  



% \begin{figure}[h]
%     \centering 
%     % \vspace{-15pt}
%     % \includegraphics[width=0.9\linewidth]{example-image-a}
%     \includegraphics[width=0.4\linewidth]{figures/CIFAR10-FNN.pdf} \hfil
%     \includegraphics[width=0.4\linewidth]{figures/CIFAR10-Resnet.pdf}
%     % \includegraphics[width=0.9\linewidth]{figures/{CIFAR10_rn=0.1_lr=0.2_wd=0.005}.png}
%     % \vspace{-10pt}
%     \caption{ Per epoch curves for CIFAR10 corresponding results in \tabref{table:multiclass}. As before, we just plot the dominating term in the RHS of \eqref{eq:multiclass_ERM} as predicted bound. Additionally, we also plot the predicted lower bound by the error on mislabeled data which nevertheless were predicted as true label. We refer to this as ``Oracle bound''. See text for more details. 
%     % 
%     % except for the stopping point. 
%     % The bound predicted by RATT (RHS in \eqref{eq:multiclass_ERM}) is vacuous. 
%     }\label{fig:error_epoch_CIFAR10}
%     % \vspace{-15pt}
% \end{figure}


% \textbf{Results on CIFAR 100 {} {}} 
% % 
% On CIFAR100, our bound in \eqref{eq:multiclass_ERM} yields vacous bounds. However, the oracle bound as explained above yields tight guarantees in the initial phase of the learning (i.e., when learning rate is less than $0.1$). 

% \begin{figure}[h]
%     \centering 
%     % \vspace{-15pt}
%     % \includegraphics[width=0.9\linewidth]{example-image-a}
%     \includegraphics[width=0.4\linewidth]{figures/CIFAR100-Resnet.pdf}
%     % \includegraphics[width=0.9\linewidth]{figures/{CIFAR10_rn=0.1_lr=0.2_wd=0.005}.png}
%     % \vspace{-10pt}
%     \caption{ Predicted lower bound by the error on mislabeled data which nevertheless were predicted as true label with ResNet18 on CIFAR100. We refer to this as ``Oracle bound''. See text for more details. 
%     % 
%     % except for the stopping point. 
%     The bound predicted by RATT (RHS in \eqref{eq:multiclass_ERM}) is vacuous. 
%     }\label{fig:error_CIFAR100}
%     % \vspace{-15pt}
% \end{figure}


% % \paragraph{Experiments on CIFAR100} 



% \subsection{Hyperparameter Details}


% \textbf{\figref{fig:error_CIFAR10} {} {}} We use clean training dataset of size $40,000$. We fix the amount of unlabeled data at $20\%$ of the clean size, i.e. we include additional $8,000$ points with randomly assigned labels. We use test set of $10,000$ points. For both MLP and ResNet, we use SGD with an initial learning rate of $0.1$ and momentum $0.9$. We fix the weight decay parameter at $5\times 10^{-4}$. After $100$ epochs, we decay the learning rate to $0.01$. We use SGD batch size of $100$. 

% \textbf{\figref{fig:error_binary} (a) {} {}} We obtain a toy dataset according to the process described in \secref{sec:app_dataset}. We fix $d=100$ and create a dataset of $50,000$ points with balanced classes. Moreover, we sample additional covariates with the same procedure to create randomly labeled dataset. For both SGD and GD training, we use a fixed learning rate $0.1$.    

% \textbf{\figref{fig:error_binary} (b) {} {}} Similar to binary CIFAR, we use clean training dataset of size $40,000$ and fix the amount of unlabeled data at $20\%$ of the clean dataset size. To train wide nets, we use a fixed learning of $0.001$ with GD and SGD. We decide the weight decay parameter and the early stopping point that maximizes our generalization bound (i.e. without peeking at unseen data ).  We use SGD batch size of $100$. 

% \textbf{\figref{fig:error_binary} (c) {} {}} With IMDb dataset, we use a clean dataset of size $20,000$ and as before, fix the amount of unlabeled data at $20\%$ of the clean data. To train ELMo model, we use Adam optimizer with a fixed learning rate $0.01$ and weight decay $10^{-6}$ to minimize cross entropy loss. We train with batch size $32$ for 3 epochs. To fine-tune BERT model, we use Adam optimizer with learning rate $5\times 10^{-5}$ to minimize cross entropy loss. We train with a batch size of $16$ for 1 epoch.    

% \textbf{\tabref{table:multiclass} {} {}} For multiclass datasets, we train both MLP and ResNet with the same hyperparameters as described before. We sample a clean training dataset of size $40,000$ and fix the amount of unlabeled data at $20\%$ of the clean size. We use SGD with an initial learning rate of $0.1$ and momentum $0.9$. We fix the weight decay parameter at $5\times 10^{-4}$. After $30$ epochs for ResNet and after $50$ epochs for MLP, we decay the learning rate to $0.01$.  We use SGD with batch size $100$. 
% For \figref{fig:error_CIFAR100}, we use the same hyperparameters as 
% CIFAR10 training, except we now decay learning rate after $100$ epochs. 


% In all experiments, to identify the best possible accuracy on just the clean data, we use the exact same set of hyperparamters except the stopping point. We choose a stopping point that maximizes test performance. 

% \subsection{Summary of experiments }

% \begin{center}
%     \begin{table}[H] 
%         \centering
%         \begin{tabular}{|c|c|c|c|} 
%         \hline
%         Classification type & Model category & Model & Dataset  \\ [0.5ex] 
%         \hline
%         \hline
%         \multirow{9}{*}{Binary} & Low dimensional & Linear model & Toy Gaussain dataset  \\
%                         \cline{2-4}
%                          & \multirow{1}{*}{Overparameterized linear nets} 
%                         %  & Linear model & Toy Gaussain dataset \\
%                         %  \cline{3-4}
%                         %  & & 2-layer wide net& Toy Gaussain dataset \\
%                         %  \cline{3-4}
%                          & 2-layer wide net & Binary MNIST \\
%                          \cline{2-4}                 
%                          & \multirow{6}{*}{Deep nets} & \multirow{2}{*}{MLP} & Binary MNIST \\
%                          \cline{4-4}
%                          & &  & Binary CIFAR \\
%                          \cline{3-4}
%                          &  & \multirow{2}{*}{ResNet} & Binary MNIST \\
%                          \cline{4-4}
%                          & &  & Binary CIFAR \\
%                          \cline{3-4}
%                          &  & ELMo-LSTM model & IMDb Sentiment Analysis \\
%                          \cline{3-4}
%                          & & BERT pre-trained model & IMDb Sentiment Analysis \\
%         \hline
%         \multirow{5}{*}{Multiclass} & \multirow{5}{*}{Deep nets} & \multirow{2}{*}{MLP} & MNIST \\
%                         \cline{4-4} 
%                         & & & CIFAR10 \\                   
%                         \cline{3-4}
%                          &   & \multirow{3}{*}{ResNet} & MNIST \\
%                          \cline{4-4}
%                          &   & & CIFAR10 \\
%                          \cline{4-4}
%                          &   & & CIFAR100 \\
%         \hline
%         \end{tabular}
%         % \caption{Summary of experiments performed} \label{table:experiments}
%     \end{table}    
%     % \footnotetext[6]{We use both MSE loss and cross-entropy loss.}
%     % \footnotetext[6]{We try 2 variants: one with a fixed first layer and the other with both layers trainable.}
% \end{center}

% \newpage
% \section{Proof of \lemref{lem:stability_error}} \label{app:proof_lem_error}

% \begin{proof}[Proof of \lemref{lem:stability_error}]
%     Recall, we have a training set $S \cup \wt S_C$. We defined leave-one-out error on mislabeled points as $$\error_{\text{LOO}(\wt S_M) } = \frac{\sum_{(x_i, y_i) \in \wt S_M} \error( f_{(i)}( x_i), y_i)}{ \abs{\wt S_M }} \,, $$
%     where $f_{(i)} \defeq f(\calA, (S \cup \wt S)_{(i)})$. Define $S^\prime \defeq S \cup \wt S$. Assume $(x,y)$ and $(x^\prime,y^\prime)$ as i.i.d. samples from ${\calDm}$. 
%     Using Lemma 25 in \citet{bousquet2002stability}, we have
%     \begin{align*}
%         \Expo{ \left( \error_{\calDm}(\wh f) -\error_{\text{LOO}(\wt S_M) } \right)^2 } \le & \Expt{ S^\prime, (x,y), (x^\prime,y^\prime) }{ \error(\wh f(x), y ) \error(\wh f(x^\prime), y^\prime )} - 2 \Expt{ S^\prime, (x,y) }{ \error(\wh f(x), y ) \error(f_{(i)}(x_i), y_i )} \\
%         & + \frac{m_1-1}{m_1}\Expt{ S^\prime }{  \error(f_{(i)}(x_i), y_i )  \error(f_{(j)}(x_j), y_j )} + \frac{1}{m_1} \Expt{ S^\prime }{  \error(f_{(i)}(x_i), y_i ) } \,. \numberthis \label{eq:main_reln}
%     \end{align*}
%     We can rewrite the equation above as : 
%     \begin{align*}
%         \Expo{ \left( \error_{\calDm}(\wh f) -\error_{\text{LOO}(\wt S_M) } \right)^2 } \le &  \, \underbrace{\Expt{ S^\prime, (x,y), (x^\prime,y^\prime) }{ \error(\wh f(x), y ) \error(\wh f(x^\prime), y^\prime ) - \error(\wh f(x), y ) \error(f_{(i)}(x_i), y_i )}}_{\RN{1}} \\
%         & + \underbrace{\Expt{ S^\prime }{  \error(f_{(i)}(x_i), y_i )  \error(f_{(j)}(x_j), y_j ) -  \error(\wh f(x), y ) \error(f_{(i)}(x_i), y_i )}}_{\RN{2}} \\ &+ \underbrace{\frac{1}{m_1} \Expt{ S^\prime }{  \error(f_{(i)}(x_i), y_i ) - \error(f_{(i)}(x_i), y_i )  \error(f_{(j)}(x_j), y_j ) }}_{\RN{3}} \,. \numberthis \label{eq:main_reln2}
%     \end{align*}
    
%     We will now bound term $\RN{3}$.  Using Schwarz's inequality, we have
    
%     \begin{align}
%         \Expt{ S^\prime }{  \error(f_{(i)}(x_i), y_i ) - \error(f_{(i)}(x_i), y_i )  \error(f_{(j)}(x_j), y_j ) }^2 &\le  \Expt{ S^\prime }{  \error(f_{(i)}(x_i), y_i ) }^2 \Expt{S^\prime}{1 -   \error(f_{(j)}(x_j), y_j ) }^2 \\
%         &\le \frac{1}{4} \label{eq:term1_lem12}
%     \end{align}
    
%     Note that since $(x_i,y_i)$, $(x_j ,y_j )$, $(x,y)$, and $(x^\prime, y^\prime)$ are all from same distribution $\calDm$, we directly incorporate the bounds on term $\RN{1}$ and $\RN{2}$ from proof of Lemma 9 in \citet{bousquet2002stability}. Combining that with \eqref{eq:term1_lem12} and our definition of hypothesis stability in \codref{cond:hypothesis_stability}, we have the required claim. 
    
    
%     % We now re-write term $\RN{1}$ as
%     % \begin{align*}
%     %         &\Expt{S^\prime, (x,y), (x^\prime,y^\prime) }{ \error(\wh f(x), y ) \error(\wh f(x^\prime), y^\prime ) - \error(\wh f(x), y ) \error(f_{(i)}(x_i), y_i )} \\ & \qquad = \Expt{ S^\prime, (x,y), (x^\prime,y^\prime) }{ \error(\wh f(x), y ) \error(\wh f  (x^\prime), y^\prime ) - \error(\wh f ^\prime(x), y ) \error(f_{(i)}(x^\prime), y^\prime )} \tag{Exchanging $(x_i, y_i)$ with $(x^\prime, y^\prime)$ in the second term} \\
%     %         & \qquad = \Expt{ S^\prime, (x,y), (x^\prime,y^\prime) }{  \left(\error(\wh f(x), y )-  \error(f_{(i)}(x), y ) \right) \error(\wh f  (x^\prime), y^\prime )  } \\
%     %         & \qquad  + \Expt{ S^\prime, (x,y), (x^\prime,y^\prime) }{  \left(\error(f_{(i)}(x), y ) -\error(\wh f ^\prime(x), y ) \right) \error(\wh f  (x^\prime), y^\prime )}  \\
%     %         & \qquad +\Expt{ S^\prime, (x,y), (x^\prime,y^\prime) }{  \left( \error(\wh f  (x^\prime), y^\prime ) -  \error(f_{(i)}(x^\prime), y^\prime ) \right) \error(\wh f ^\prime(x), y ) }  \,, \numberthis \label{eq:term1_final}
%     % \end{align*}
%     % where $\wh f^\prime$ is the classifier obtained by training on $ S^\prime_{(i)} \cup \{ (x^\prime, y^\prime) \} $. Similarly we can re-write term $\RN{2}$ as 
%     % \begin{align*}
%     %     & \Expt{ S^\prime }{  \error(f_{(i)}(x_i), y_i )  \error(f_{(j)}(x_j), y_j ) -  \error(\wh f(x), y ) \error(f_{(i)}(x_i), y_i )} \\
%     %     &\quad  = \Expt{ S^\prime, (x,y), (x^\prime,y^\prime)}{  \error(f^{\prime\prime}_{(i)}(x), y )  \error(f_{(j)}^{\prime}(x^\prime), y^\prime ) -  \error(\wh f(x), y ) \error(f_{(i)}(x_i), y_i )} \tag{Exchanging $(x_i, y_i)$ with $(x, y)$ and $(x_j, y_j)$ with $(x^\prime, y^\prime)$ in the first term}\\
%     %     &\quad = \Expt{ S^\prime, (x,y), (x^\prime,y^\prime)}{  \error(f^{\prime\prime}_{(j)}(x), y )  \error(f_{(i)}^{\prime}(x^\prime), y^\prime ) -  \error(\wh f^\prime (x), y ) \error(f^\prime_{(j)}(x^\prime), y^\prime )} \tag{Exchanging $(x_i, y_i)$ and $(x_j, y_j)$ and then replacing $(x_j, y_j)$ with $(x^\prime, y^\prime)$ in the second term} \\
%     %     & \quad = \Expt{ S^\prime, (x,y), (x^\prime,y^\prime) }{  \left( \error(f_{(i)}^{\prime}(x^\prime), y^\prime )   -  \error(\wh f^{\prime\prime}  (x^\prime), y^\prime ) \right)  \error(f^{\prime\prime}_{(j)}(x), y )   } \\
%     %     & \quad  + \Expt{ S^\prime, (x,y), (x^\prime,y^\prime) }{  \left( \error(f^{\prime\prime}_{(j)}(x), y )  -\error(\wh f ^\prime(x), y ) \right) \error(\wh f^{\prime\prime}  (x^\prime), y^\prime )  }  \\
%     %     & \quad+ \Expt{ S^\prime, (x,y), (x^\prime,y^\prime) }{  \left( \error(\wh f^{\prime\prime}  (x^\prime), y^\prime )  -  \error(f^\prime_{(j)}(x^\prime), y^\prime ) \right)  \error(\wh f^\prime (x), y ) }   \\
%     %     & \quad = \Expt{ S^\prime, (x,y), (x^\prime,y^\prime) }{  \left( \error(f_{(i)}^{\prime}(x^\prime), y^\prime )   -  \error(\wh f (x^\prime), y^\prime ) \right)  \error(f_{(i)}(x_j), y_j )   } \\
%     %     & \quad  + \Expt{ S^\prime, (x,y), (x^\prime,y^\prime) }{  \left( \error(f^{\prime\prime}_{(j)}(x), y )  -\error(\wh f (x), y ) \right) \error(\wh f^{\prime\prime}  (x_j), y_j )  }  \\
%     %     & \quad+ \Expt{ S^\prime, (x,y), (x^\prime,y^\prime) }{  \left( \error(\wh f^{\prime\prime}  (x^\prime), y^\prime )  -  \error(f^\prime_{(j)}(x^\prime), y^\prime ) \right)  \error(\wh f^\prime (x^\prime), y^\prime ) }  \,, \numberthis \label{eq:term2_final}
%     % \end{align*}
%     % where $f^{\prime\prime}_{(j)}$ is trained on $S^\prime_{(j,i)} \cup {(x,y)}$, $f^{\prime}_{(i)}$ is trained on $S^\prime_{(j,i)} \cup {(x^\prime,y^\prime)}$, and $\wh f^{\prime\prime} $ is trained on $S^\prime_{(j)} \cup {(x,y)}$. Note in the last line we replaced $(x,y)$ by $(x_j, y_j)$ in the first term, replaced $(x^\prime,y^\prime)$ by $(x_j, y_j)$ in the second term and exchanged $(x_i,y_i)$ with $(x_j,y_j)$ and also $(x,y)$ and $(x^\prime, y^\prime)$
    
    
% \end{proof}


\end{document}


