% File tacl2021v1.tex
% Dec. 15, 2021

% The English content of this file was modified from various *ACL instructions
% by Lillian Lee and Kristina Toutanova
%
% LaTeXery is mostly all adapted from acl2018.sty.

\documentclass[11pt,a4paper]{article}
\usepackage{times,latexsym}
\usepackage{url}
\usepackage[T1]{fontenc}

%% Package options:
%% Short version: "hyperref" and "submission" are the defaults.
%% More verbose version:
%% Most compact command to produce a submission version with hyperref enabled
%%    \usepackage[]{tacl2021v1}
%% Most compact command to produce a "camera-ready" version
   \usepackage[acceptedWithA]{tacl2021v1}
%% Most compact command to produce a double-spaced copy-editor's version
%%    \usepackage[acceptedWithA,copyedit]{tacl2021v1}
%
%% If you need to disable hyperref in any of the above settings (see Section
%% "LaTeX files") in the TACL instructions), add ",nohyperref" in the square
%% brackets. (The comma is a delimiter in case there are multiple options specified.)

\usepackage{tacl2021v1}
% \setlength\titlebox{10cm} % <- for Option 2 below

%%%% Material in this block is specific to generating TACL instructions
\usepackage{xspace,mfirstuc,tabulary}
\usepackage{times}
\usepackage{latexsym}

% For proper rendering and hyphenation of words containing Latin characters (including in bib files)
\usepackage[T1]{fontenc}
% For Vietnamese characters
% \usepackage[T5]{fontenc}
% See https://www.latex-project.org/help/documentation/encguide.pdf for other character sets

% This assumes your files are encoded as UTF8
\usepackage[utf8]{inputenc}

% This is not strictly necessary, and may be commented out.
% However, it will improve the layout of the manuscript,
% and will typically save some space.
\usepackage{microtype}

% This is also not strictly necessary, and may be commented out.
% However, it will improve the aesthetics of text in
% the typewriter font.
\usepackage{inconsolata}

\usepackage{amsmath}
\usepackage{amsfonts}
\usepackage{amssymb}

% for professional tables
\usepackage{booktabs}
% for multiple rows
\usepackage{multirow}

% for in paragraph list
\usepackage{paralist}

% for compact itemize 
\usepackage{mdwlist}
\usepackage{graphicx}

% to use colors
\usepackage{color}

% for algorithms
\usepackage[ruled,linesnumbered]{algorithm2e}

% for defining optional space after words (but no space if it is punctuation)
\usepackage{xspace}

% ddm introduce
\usepackage{tabularx}
\usepackage{makecell}
\usepackage{amssymb}

\newcommand{\xx}{\mathbf{x}}
\newcommand{\uu}{\mathbf{u}}
\newcommand{\zz}{\mathbf{z}}
\newcommand{\yy}{\mathbf{y}}
\newcommand{\hh}{\mathbf{h}}
\newcommand{\zy}[1]{\textcolor{green}{\bf \small [#1 --zy]}}
\DeclareMathOperator*{\argmax}{arg\,max}
\DeclareMathOperator*{\argmin}{arg\,min}

\usepackage{color}
\usepackage{tikz}
\usepackage[edges]{forest}
\definecolor{hidden-draw}{RGB}{20,68,106}
\definecolor{hidden-pink}{RGB}{255,245,247}

\newcommand{\leiModify}[1]{\textcolor{orange}{#1 --Lei}}
\newcommand{\qingxiu}[1]{\textcolor{blue}{#1}}
\newcommand{\zc}[1]{\textcolor{purple}{#1 --ce}}
\newcommand{\dateOfLastUpdate}{Dec. 15, 2021}
\newcommand{\styleFileVersion}{tacl2021v1}

\newcommand{\ex}[1]{{\sf #1}}

\newif\iftaclinstructions
\taclinstructionsfalse % AUTHORS: do NOT set this to true
\iftaclinstructions
\renewcommand{\confidential}{}
\renewcommand{\anonsubtext}{(No author info supplied here, for consistency with
TACL-submission anonymization requirements)}
\newcommand{\instr}
\fi

%
\iftaclpubformat % this "if" is set by the choice of options
\newcommand{\taclpaper}{final version\xspace}
\newcommand{\taclpapers}{final versions\xspace}
\newcommand{\Taclpaper}{Final version\xspace}
\newcommand{\Taclpapers}{Final versions\xspace}
\newcommand{\TaclPapers}{Final Versions\xspace}
\else
\newcommand{\taclpaper}{submission\xspace}
\newcommand{\taclpapers}{{\taclpaper}s\xspace}
\newcommand{\Taclpaper}{Submission\xspace}
\newcommand{\Taclpapers}{{\Taclpaper}s\xspace}
\newcommand{\TaclPapers}{Submissions\xspace}
\fi

%%%% End TACL-instructions-specific macro block
%%%%

\title{A Survey on In-context Learning}

% Author information does not appear in the pdf unless the "acceptedWithA" option is given

% The author block may be formatted in one of two ways:

% Option 1. Author’s address is underneath each name, centered.
\author{
Qingxiu Dong\textsuperscript{\rm1}, 
Lei Li\textsuperscript{\rm1},
Damai Dai\textsuperscript{\rm1}, 
Ce Zheng\textsuperscript{\rm1},
Zhiyong Wu\textsuperscript{\rm2},\\
\textbf{
Baobao Chang\textsuperscript{\rm1},
Xu Sun\textsuperscript{\rm1},
Jingjing Xu\textsuperscript{\rm2}, 
Lei Li\textsuperscript{\rm3}
and Zhifang Sui\textsuperscript{\rm1}}  \\
\textsuperscript{\rm 1} MOE Key Lab of Computational Linguistics, School of Computer Science, Peking University \\
  \textsuperscript{\rm 2} Shanghai AI Lab \ \ \textsuperscript{\rm 3} University of California, Santa Barbara \\
  \texttt{ \{dqx,lilei\}@stu.pku.edu.cn,  
  wuzhiyong@pjlab.org.cn, lilei@cs.ucsb.edu }
  \\ \texttt{\{daidamai,zce1112zslx,chbb,xusun,jingjingxu,szf\}@pku.edu.cn}
}
% \author{
%   Template Author1\Thanks{The {\em actual} contributors to this instruction
%     document and corresponding template file are given in Section
%     \ref{sec:contributors}.} 
%   \\
%   Template Affiliation1/Address Line 1
%   \\
%   Template Affiliation1/Address Line 2
%   \\
%   Template Affiliation1/Address Line 2
%   \\
%   \texttt{template.email1example.com}
%   \And
%   Template Author2 
%   \\
%   Template Affiliation2/Address Line 1
%   \\
%   Template Affiliation2/Address Line 2
%   \\
%   Template Affiliation2/Address Line 2
%   \\
%   \texttt{template.email2@example.com}
% }

% % Option 2.  Author’s address is linked with superscript
% % characters to its name, author names are grouped, centered.

% \author{
%   Template Author1\Thanks{The {\em actual} contributors to this instruction
%     document and corresponding template file are given in Section
%     \ref{sec:contributors}.}$^\diamond$ 
%   \and
%   Template Author2$^\dagger$
%   \\
%   \ \\
%   $^\diamond$Template Affiliation1/Address Line 1
%   \\
%   Template Affiliation1/Address Line 2
%   \\
%   Template Affiliation1/Address Line 2
%   \\
%   \texttt{template.email1example.com}
%   \\
%   \ \\
%   \\
%   $^\dagger$Template Affiliation2/Address Line 1
%   \\
%   Template Affiliation2/Address Line 2
%   \\
%   Template Affiliation2/Address Line 2
%   \\
%   \texttt{template.email2@example.com}
% }

\date{}

\begin{document}
\maketitle

\begin{abstract}


Transformers are expensive to train due to the quadratic time and space complexity in the self-attention mechanism. On the other hand, although kernel machines suffer from the same computation bottleneck in pairwise dot products, several approximation schemes have been successfully incorporated to considerably reduce their computational cost without sacrificing too much accuracy. In this work, we leverage the computation methods for kernel machines to alleviate the high computational cost and introduce Skyformer, which replaces the softmax structure with a Gaussian kernel to stabilize the model training and adapts the \nystrom method to a  non-positive semidefinite matrix to accelerate the computation. We further conduct theoretical analysis by showing that the matrix approximation error of our proposed method is small in the spectral norm. Experiments on Long Range Arena benchmark show that the proposed method is sufficient in getting comparable or even better performance than the full self-attention while requiring fewer computation resources. 


\end{abstract}

\section{Introduction}
\label{sec:intro}
% Description/Definition/Formulation
%   1. interface human understanding
%   2. No gradient update v.s. Finetuning
%   3. 从context learning
%   4. strengths, reasoning zero-shot human readable
%   5. weakness (a)unstable(order,distribution,token) (b)need large model (3)inference efficiency with long context , (4) limited training data(max\_seq\_len)

% 第一段 写一下 背景
% Prompt 那个是从 superivised -> pre-training finetuning -> prompt engineering 切入的
% ICL ability emerges from pretraining of large LM. 
% First proposed in GPT-3
With the scaling of model size and corpus size~\citep{bert,gpt2,gpt3,chowdhery2022palm}, large language models (LLMs) demonstrate an in-context learning (ICL) ability, that is, learning from a few examples in the context. 
% (in-context learning for short). 
Many studies have shown that LLMs can perform a series of complex tasks through ICL, such as solving mathematical reasoning problems~\citep{cot}. These strong abilities have been widely verified as emerging abilities for large language models~\citep{wei2022emergent}. 

%Figure~\ref{fig:icl} gives an example describing how language models make decisions under ICL. 

 %These strong abilities have been widely verified as the emerging abilities for large language models~\citep{wei2022emergent}. 
 
 %It shows that the model learns from the demonstration consisting of a few examples in the context
% For example, by feeding the model with translation text pair from the source language to the target and a sentence that needs to be translated, the model could accurately produce a translation for the given query.



% sec1. lei增加general 定义,当前主要问题和研究现状 contribution insight 
% sec2. lei补充formulation inference的各种变种的formulatioin 加table

% The development of pre-trained language models, such as BERT~\citep{bert} and GPT-2~\citep{gpt2} leads to a widely-adopted paradigm where the model is first trained on the general corpus, then finetuned on task-specific datasets for improving the in-domain performance, such as the classification accuracy.
% The idea behind this is to learn rich contextual representations of languages from free text.
% Further explorations in this direction scale up the number of causal language model parameters and tokens of training corpus, resulting in surprising few-shot learning abilities, i.e., the model can learn from the demonstration with a few examples in the context to perform a series of complex tasks, such as solving mathematical reasoning problems. 
% This ability is called in-context learning~(ICL) by \citet{gpt3} and shows great potential in a wide range of tasks, attracting great attention from the community~\citep{metaicl}.

% Informal definition, acutally, illustrate the process of ICL 
The key idea of in-context learning is to learn from analogy. Figure~\ref{fig:icl} gives an example describing how language models make decisions with ICL.  %and we illustrate the whole process in Figure~\ref{fig:icl}. 
First, ICL requires a few examples to form a demonstration context. These examples are usually written in natural language templates. 
%a few examples are first picked and represented in natural language tokens according to specific templates and order to form a demonstration context. 
%Second, the question of interest is also converted in a similar way after the demonstration text 
Then, ICL concatenates a query question and a piece of demonstration context together to form a prompt,  which is then fed into the language model for prediction.
Different from supervised learning requiring a training stage that uses backward gradients to update model parameters, ICL does not conduct parameter updates and directly performs predictions on the pretrained language models. The model is expected to learn the pattern hidden in the demonstration and accordingly make the right prediction. % The inference is performed in the form of text completion by reusing the language model head learned during the large-scale pretraining.
%If required, the predictions are transformed into an answer according to a pre-defined rule. \zy{the last sentence is kinda too detailed and unnecessary?}%, where the model performance is evaluated on. 
% We summarize the key characteristics of ICL and comparison with traditional finetuning in Figure~\ref{fig:my_label}.
% http://ai.stanford.edu/blog/in-context-learning : What is in-context learning? Informally, in-context learning describes a different paradigm of “learning” where the model is fed input normally as if it were a black box, and the input to the model describes a new task with some possible examples while the resulting output of the model reflects that new task as if the model had “learned”. While imprecise, the term is meant to capture common behavior that was noted in the GPT-3 paper by OpenAI as a phenomenon that GPT-3 displayed with surprising consistency.


% \section{Definition of In-Context Learning~(ICL)}
% Definition of ICL 


\begin{figure}[t]
    \centering
    \includegraphics[width=0.45\textwidth]{fig/icl.pdf}
    \caption{Illustration of in-context learning. ICL requires a piece of demonstration context containing a few examples written in  natural language templates. Taking the demonstration and a query as the input, large language models are responsible for making predictions.}
    \label{fig:icl}
\end{figure}

\section{Results and discussion}
\label{sec:taxonomy}


In this section, we present the result of our research. As stated in Section \ref{sec:reviewProtocol}, the selected VPs have been labeled according to the taxonomy of terms summarized in Tables \ref{table:instructionalDesignTable} and \ref{table:technicalDesignTable}. The definition of the identified categories (which differs to a large extent from the one presented in \cite{lee2020effective}) is introduced in the following subsections, where we also discuss the survey results relative to each group. We first introduce the instructional design elements, which are connected to the technical elements necessary to realize them; afterwards, we discuss the design choices related to the technical and technological components of the simulations. Finally, we discuss the experimental evidences related to the effectiveness of the identified design elements.



% Approaching health care provider-patient communication training through computer-based learning methods has been an active research area during the last ten years. As a result, the number of works in this area has grown significantly \cite{lee2020effective,peddle2016virtual,richardson2019virtualreview}. 

% As stated in Section \ref{sec:reviewProtocol}, during the process of reading articles and collecting data, we identified several recurring characteristics and design elements that we subsequently organized in a taxonomy of terms (summarized in Tables \ref{table:instructionalDesignTable} and \ref{table:technicalDesignTable}). The definition of the identified categories (which differs to a large extent from the one presented in \cite{lee2020effective}) is introduced in the following subsections, where we also discuss the survey results relative to each group. We first introduce the instructional design elements, which are connected to the technical elements necessary to realize them, and then we discuss the design choices related to the technical and technological component of the simulations. Finally, we discuss the experimental evidences related to the effectiveness of the identified design elements. 




% %COMPREHENSIVE TABLE WITH EVERYTHING

\begin{landscape}
%\begin{longtable}{ | p{1cm} | *{15}{l} |}
%\begin{tabularx}{\linewidth} {>{\raggedleft\arraybackslash}X >{\raggedleft\arraybackslash}X >{\raggedleft\arraybackslash}X >{\raggedleft\arraybackslash}X >{\raggedleft\arraybackslash}X >{\raggedleft\arraybackslash}X >{\raggedleft\arraybackslash}X >{\raggedleft\arraybackslash}X >{\raggedleft\arraybackslash}X >{\raggedleft\arraybackslash}X}
\tiny
{\rowcolors{3}{mywhite}{mygray}
%\begin{tabularx}{\linewidth} {X | X | X | X | X | X | X | X | X | X}
\begin{tabularx}{\linewidth}
{|>{\hsize=.5\hsize\linewidth=\hsize}X |
>{\hsize=.75\hsize\linewidth=\hsize}X |
>{\hsize=.75\hsize\linewidth=\hsize}X |
>{\hsize=1.5\hsize\linewidth=\hsize}X |
>{\hsize=1\hsize\linewidth=\hsize}X |
>{\hsize=1.5\hsize\linewidth=\hsize}X |
>{\hsize=.75\hsize\linewidth=\hsize}X |
>{\hsize=.5\hsize\linewidth=\hsize}X |
>{\hsize=.75\hsize\linewidth=\hsize}X |
>{\hsize=2\hsize\linewidth=\hsize}X |}


%\textbf{Study} & \textbf{Geographical Location and Level} & \textbf{Identification} & 
%\textbf{Multiplier Result} \\ \hline \hline
%\endfirsthead
\rowcolor{lightgray}
\textbf{Article}  & \multicolumn{4}{|l|}{\textbf{Instructional Design}} & \multicolumn{5}{|l|}{\textbf{Technical Design}} \\
\rowcolor{lightgray}
& Category & Navigation & Feedback & Gamification & Hardware & Presentation & Input Interface & Distribution & Other Tech. Features \\
%& \multicolumn{1}{|l|}{Category} & \multicolumn{1}{|l|}{Navigation} & \multicolumn{1}{|l|}{Feedback} & \multicolumn{1}{|l|}{Gamification} & \multicolumn{1}{|l|}{Hardware} & \multicolumn{1}{|l|}{Presentation} & \multicolumn{1}{|l|}{Input} & \multicolumn{1}{|l|}{Distribution} & \multicolumn{1}{|l|}{Other Tech. Features}  \\ 
%\hline 
\specialrule{.1em}{.05em}{.05em} 
\endhead

\cite{adefila2020students} & Narrative, PS & \textbf{\emph{Unclear}} &	Timed events and health status &	Tamagotchi style serious game & Any device with a web browser	& Graphic (Image) & Typed &	Web-Based & /\\ 

\cite{albright2018using} & Narrative &	Closed-Option &
	Trust meter and Virtual Instructor discussing the user’s choices & / &	Any device with a web browser & Graphic (Image) & Typed	& Web-Based & /\\ 

\cite{banszki2018clinical} + \cite{quail2016student} & Narrative	& Open-Option, Non-Verbal &	/ &	/ &	PC, 64” monitor, Microphone & Graphic (3D) & Voice-Controlled & Standalone &	Human-Controlled,Gesture and Facial Expression Output \\ 

\cite{dupuy2019virtual} & Narrative &	Closed-Option, Non-Verbal &	Errors, score, time at the end. Possibility to replay certain parts. &	Score, Time	& PC, Vertical 40” screen, Microphone, Camera & Graphic (3D) & Voice-Controlled & Standalone &	Facial Expression Detection through face tracking\\ 

\cite{foster2016using} & Narrative & Open-Option & Empathy feedback available at the end of the simulation (human-generated) & / &	Any device with a web browser & Text-Based, Graphic (Video) & Typed & Web-Based & Human-Controlled Empathy Feedback\\ 

\cite{guetterman2019medical} + \cite{kron2017using} & Narrative &	Closed-Option, Non-Verbal &	Feedback at the end giving evidence on each choice of words and a score &	Score &	PC, Microsoft Kinect, Microphone & Graphic (3D) & Voice-Controlled &	Standalone &	Facial Expression and Body Posture Detection and output, Recorded Voiceover, Multiple VHs\\ 

\cite{hirumi2016advancingPart2} + \cite{hirumi2016advancing} + \cite{kleinsmith2015understanding} & Narrative, PS &	Closed OR Open-Option & Available topics, discoveries made	& NO &	Any device with a web browser	& Graphic (3D) & Typed & Web-Based & /\\ 

\cite{jacklin2019virtual} + \cite{jacklin2018improving} & Narrative & Closed-Option &	Unspecified Feedback at the end of the simulation &	/ &	Any device with a web browser & Graphic (3D) & Typed &	Web-Based & Body Posture Output, Recorded Voiceover	\\ 

\cite{jeuring2015communicate} & Narrative & Closed-Option & Scores, recap of choices made & Envisioned as a Serious Game, Scores for each learning goal	& \textbf{\emph{Unclear}} & Graphic (3D) & Typed & \textbf{\emph{Unclear}} & Scenario Editor\\ 

\cite{maicher2017developing} & Narrative, PS &	Open-Option, Non-Verbal & /	& / & PC, Microsoft Kinect, Multi-Array Microphone & Graphic (3D) & Typed + Voice-Controlled & Web-Based + Standalone & Motion-Captured Animations, Movement and Posture Detection and Output\\

\cite{marei2018use} & Narrative	& Closed-Option &	/ &	/ &	PC & Graphic (Image) &	Typed & \textbf{\emph{Unclear}} & /\\ 

\cite{ochs2019training} & Narrative	& Hybrid & / & / & PC, HMD (Oculus Rift), CAVE, High-End Microphone & Graphic (3D, IVR) & Voice-Controlled &  Standalone & Same VP deployed in desktop, VR and Cave versions. Speech recognition is Human-Controlled, Non-Verbal output (facial expression and body posture). Text-To-Speech, Lip Synch, Virtual Playback\\

\cite{o2019suicide} & Narrative, PS &	Closed-Option &	Immediate and after-action guidance, via Virtual Instructor & / & Any device with a web browser & Graphic (Video) & 	Typed & Web-Based & /\\ 

\cite{peddle2019exploring} + \cite{peddle2019development} & Narrative, PS & Closed-Option	& Possibility to replay certain parts & / & Any device with a web browser & Graphic (Video) & Typed & Web-Based & /\\ 

\cite{richardson2019virtual} & Narrative, PS &	Closed-Option &	Personalized feedback at the end to enhance counselling ability	& / &	Any device with a web browser & Graphic (3D) &	Typed &	Web-Based & /\\

\cite{sapkaroski2018implementation} & Narrative, PS & Closed-Option & / & /	& PC, HMD (Various), Leap Motion, Oculus Touch & Graphic (3D, IVR) & Typed + Voice-Controlled + NUI & Standalone & Remote progress tracking for educators\\ 

\cite{schoenthaler2017simulated} & Narrative & Closed-Option & Virtual Instructor feedback during (discussing each choice) and after the simulation, Trust Meter & Scores for each learning goal after the simulation & \textbf{\emph{Unclear}} &  Graphic (3D) & Typed & 	\textbf{\emph{Unclear}} &	Can play as both provider and patient, Non-Verbal Output\\ 

\cite{szilas2019virtual} & Narrative &	Closed-Option &	/ &	Envisioned as a serious game &	PC	& Graphic (3D) &	Typed & Standalone & /\\ 

\cite{washburn2020virtual} & Narrative, PS & Open-Option & / &	/ &	PC or Laptop, Large Screen & Graphic (3D) & Hybrid & Standalone & Human-Transcribed Voice Controls\\ 

\cite{zielke2016beyond} + \cite{zielke2016using} & Narrative, PS &	Closed-Option &	Points and badges assigned during and after the interview & Envisioned as a Serious Game, Scores, Badges	& Any device with a web browser & Graphic (3D) & Typed & 	Web-Based & Multiple VHs, Non-Verbal Output\\ 

\cite{zlotos2016scenario} & Narrative, PS & Closed-Option	& All possible outcomes are played after the simulation is over & / & Any device with a web browser & Graphic (3D) & Typed  & Web-Based & Motion-Captured Animations, Recorded Voiceover\\ 

\hline
%\end{longtable}
\end{tabularx}
}

\end{landscape}
\normalsize
% %TABLE WITH ONLY INSTRUCTIONAL DESIGN

%\begin{longtable}{ | p{1cm} | *{15}{l} |}
%\begin{tabularx}{\linewidth} {>{\raggedleft\arraybackslash}X >{\raggedleft\arraybackslash}X >{\raggedleft\arraybackslash}X >{\raggedleft\arraybackslash}X >{\raggedleft\arraybackslash}X >{\raggedleft\arraybackslash}X >{\raggedleft\arraybackslash}X >{\raggedleft\arraybackslash}X >{\raggedleft\arraybackslash}X >{\raggedleft\arraybackslash}X}

\scriptsize 
{\rowcolors{3}{mywhite}{mygray}
%\begin{tabularx}{\linewidth} {X | X | X | X | X | X | X | X | X | X}
\begin{tabularx}{\linewidth}
{|>{\hsize=.5\hsize\linewidth=\hsize}X |
>{\hsize=.75\hsize\linewidth=\hsize}X |
>{\hsize=.75\hsize\linewidth=\hsize}X |
>{\hsize=1.5\hsize\linewidth=\hsize}X |
>{\hsize=1.5\hsize\linewidth=\hsize}X |}


%\textbf{Study} & \textbf{Geographical Location and Level} & \textbf{Identification} & 
%\textbf{Multiplier Result} \\ \hline \hline
%\endfirsthead
\rowcolor{lightgray}
\textbf{Article}  & \multicolumn{4}{|l|}{\textbf{Instructional Design}}\\
\rowcolor{lightgray}
& Category & Navigation & Feedback & Gamification\\
%& \multicolumn{1}{|l|}{Category} & \multicolumn{1}{|l|}{Navigation} & \multicolumn{1}{|l|}{Feedback} & \multicolumn{1}{|l|}{Gamification} & \multicolumn{1}{|l|}{Hardware} & \multicolumn{1}{|l|}{Presentation} & \multicolumn{1}{|l|}{Input} & \multicolumn{1}{|l|}{Distribution} & \multicolumn{1}{|l|}{Other Tech. Features}  \\ 
%\hline 
\specialrule{.1em}{.05em}{.05em} 
\endhead

\cite{adefila2020students} & Narrative, PS & \textbf{\emph{Unclear}} &	Timed events and health status &	Tamagotchi style serious game\\ 

\cite{albright2018using} & Narrative &	Closed-Option &
	Trust meter and Virtual Instructor discussing the user’s choices & /\\ 

\cite{banszki2018clinical} + \cite{quail2016student} & Narrative & Open-Option, Non-Verbal & / & /\\ 

\cite{dupuy2019virtual} & Narrative &	Closed-Option, Non-Verbal &	Errors, score, time at the end. Possibility to replay certain parts. &	Score, Time\\ 

\cite{foster2016using} & Narrative & Open-Option & Empathy feedback available at the end of the simulation (human-generated) & /\\ 

\cite{guetterman2019medical} + \cite{kron2017using} & Narrative &	Closed-Option, Non-Verbal &	Feedback at the end giving evidence on each choice of words and a score &	Score\\ 

\cite{hirumi2016advancingPart2} + \cite{hirumi2016advancing} + \cite{kleinsmith2015understanding} & Narrative, PS &	Closed OR Open-Option & Available topics, discoveries made	& NO\\ 

\cite{jacklin2019virtual} + \cite{jacklin2018improving} & Narrative & Closed-Option &	Unspecified Feedback at the end of the simulation &	/\\ 

\cite{jeuring2015communicate} & Narrative & Closed-Option & Scores, recap of choices made & Envisioned as a Serious Game, Scores for each learning goal\\ 

\cite{maicher2017developing} & Narrative, PS &	Open-Option, Non-Verbal & /	& /\\

\cite{marei2018use} & Narrative	& Closed-Option &	/ &	/\\ 

\cite{ochs2019training} & Narrative	& Hybrid & / & /\\

\cite{o2019suicide} & Narrative, PS &	Closed-Option &	Immediate and after-action guidance, via Virtual Instructor & /\\ 

\cite{peddle2019exploring} + \cite{peddle2019development} & Narrative, PS & Closed-Option	& Possibility to replay certain parts & /\\ 

\cite{richardson2019virtual} & Narrative, PS &	Closed-Option &	Personalized feedback at the end to enhance counselling ability	& /\\

\cite{sapkaroski2018implementation} & Narrative, PS & Closed-Option & / & /\\ 

\cite{schoenthaler2017simulated} & Narrative & Closed-Option & Virtual Instructor feedback during (discussing each choice) and after the simulation, Trust Meter & Scores for each learning goal after the simulation\\ 

\cite{szilas2019virtual} & Narrative &	Closed-Option &	/ &	Envisioned as a serious game\\ 

\cite{washburn2020virtual} & Narrative, PS & Open-Option & / &	/\\ 

\cite{zielke2016beyond} + \cite{zielke2016using} & Narrative, PS &	Closed-Option &	Points and badges assigned during and after the interview & Envisioned as a Serious Game, Scores, Badges\\ 

\cite{zlotos2016scenario} & Narrative, PS & Closed-Option	& All possible outcomes are played after the simulation is over & /\\ 

\hline
%\end{longtable}
\end{tabularx}
}

\normalsize


\subsection{Instructional design}
\label{sec:instructionalDesign}

This category encompasses various instructional design aspects implemented in the VP scenario, such as how the VP delivers (and facilitates) learning activities and if (and how) it provides scaffolded support to improve learner's performance.


%Instructional Table: NEW VERSION
%ditched tabularx, use only normal tabular

\begin{table} [t]
\scriptsize{
\begin{center}
    \caption{Synopsis of the reviewed VPs for each instructional design category}
    \label{table:instructionalDesignTable}
    \begin{tabular}{| p{1.8cm} | p{2cm} | p{8cm} |}
    \hline
        \rowcolor{mygray}
        \multicolumn{3}{|c|}{\textbf{Instructional design}}\\
    \hline
        \rowcolor{lightgray}
        \textbf{Category}  & \textbf{Subcategory} & \textbf{Virtual Patients}\\
    \hline
    %STRUCTURE
         \multirow{2}{*}{Structure} & \emph{Narrative} & HOLLIE \cite{adefila2020students}, AtRiskInPrimaryCare \cite{albright2018using}, Dupuy \cite{dupuy2019virtual}, CynthiaYoungVP \cite{foster2016using}, Jacklin \cite{jacklin2019virtual,jacklin2018improving}, MPathic-VR \cite{guetterman2019medical,kron2017using}, Communicate! \cite{jeuring2015communicate}, Marei \cite{marei2018use},  Ochs \cite{ochs2019training}, Szilas \cite{szilas2019virtual} \\
    \cline{2-3}
        & \emph{Narrative + Problem solving} & Banszki \cite{banszki2018clinical,quail2016student}, NERVE
        \cite{hirumi2016advancingPart2,hirumi2016advancing,kleinsmith2015understanding},  Maicher \cite{maicher2017developing}, Suicide Prevention \cite{o2019suicide}, VSPR \cite{peddle2019exploring,peddle2019development}, Richardson \cite{richardson2019virtual}, CESTOLVRClinic \cite{sapkaroski2018implementation}, Schoenthaler \cite{schoenthaler2017simulated},   Washburn \cite{washburn2020virtual},  UTTimePortal \cite{zielke2016beyond,zielke2016using}, Zlotos \cite{zlotos2016scenario}\\
    \hline
    %UNFOLDING
         \multirow{3}{*}{Unfolding} & \emph{Closed-option} & HOLLIE \cite{adefila2020students}, AtRiskInPrimaryCare \cite{albright2018using}, Dupuy \cite{dupuy2019virtual}, MPathic-VR \cite{guetterman2019medical,kron2017using}, NERVE
        \cite{hirumi2016advancingPart2,hirumi2016advancing,kleinsmith2015understanding}, Jacklin \cite{jacklin2019virtual,jacklin2018improving}, Communicate! \cite{jeuring2015communicate}, Marei \cite{marei2018use}, Suicide Prevention \cite{o2019suicide}, VSPR \cite{peddle2019exploring,peddle2019development}, Richardson \cite{richardson2019virtual}, CESTOLVRClinic \cite{sapkaroski2018implementation}, Schoenthaler \cite{schoenthaler2017simulated}, Szilas \cite{szilas2019virtual}, UTTimePortal \cite{zielke2016beyond,zielke2016using}, Zlotos \cite{zlotos2016scenario}, \\
    \cline{2-3}
        & \emph{Open-option} & Banszki \cite{banszki2018clinical,quail2016student}, CynthiaYoungVP\cite{foster2016using}, NERVE
        \cite{hirumi2016advancingPart2,hirumi2016advancing,kleinsmith2015understanding}, Maicher \cite{maicher2017developing}, Washburn \cite{washburn2020virtual} \\
    \cline{2-3}
        & \emph{Hybrid} & Ochs \cite{ochs2019training}  \\
    \hline
    %FEEDBACK
        \multirow{8}{*}{Feedback} & \emph{Replay feature} & Dupuy \cite{dupuy2019virtual}, Communicate! \cite{jeuring2015communicate}, Ochs \cite{ochs2019training}, VSPR \cite{peddle2019exploring,peddle2019development}, Zlotos \cite{zlotos2016scenario}\\
    \cline{2-3}
        & \emph{Virtual instructor} & At-Risk in Primary Care \cite{albright2018using}, Suicide Prevention \cite{o2019suicide}, Schoenthaler \cite{schoenthaler2017simulated}\\
    \cline{2-3}
        & \emph{Multiple session structure} & MPathic-VR \cite{guetterman2019medical,kron2017using}  \\
    \cline{2-3}
        & \emph{Quantitative emotional feedback} & At-Risk in Primary Care \cite{albright2018using}, Schoenthaler \cite{schoenthaler2017simulated}\\
    \cline{2-3}
        & \emph{Qualitative personalized post-feedback} & Jacklin \cite{jacklin2019virtual,jacklin2018improving}, Richardson \cite{richardson2019virtual}\\
    \cline{2-3}
        & \emph{Empathy feedback} & CynthiaYoungVP \cite{foster2016using}\\
    \cline{2-3}
        & \emph{Clinical discoveries available} & NERVE
        \cite{hirumi2016advancingPart2,hirumi2016advancing,kleinsmith2015understanding}\\
    \cline{2-3}
        & \emph{Game elements} & Dupuy \cite{dupuy2019virtual}, MPathic-VR \cite{guetterman2019medical,kron2017using}, Communicate! \cite{jeuring2015communicate}, Schoenthaler \cite{schoenthaler2017simulated}, UT-Time Portal \cite{zielke2016beyond,zielke2016using}\\
    \hline
        \multirow{3}{*}{Gamification} & \emph{Scoring system} & Dupuy \cite{dupuy2019virtual}, MPathic-VR \cite{guetterman2019medical,kron2017using}, Communicate! \cite{jeuring2015communicate}, Schoenthaler \cite{schoenthaler2017simulated}, UT-Time Portal \cite{zielke2016beyond,zielke2016using}\\
    \cline{2-3}
        & \emph{Badge system} & UTTimePortal \cite{zielke2016beyond,zielke2016using}  \\
    \cline{2-3}
        & \emph{Countdown timed events} & HOLLIE \cite{adefila2020students}\\
    \hline
     \end{tabular}
\end{center}
}
\end{table}
\normalsize

%Technical Table: NEW VERSION
%ditched tabularx, use only normal tabular

\begin{table} [t]
\scriptsize{
\begin{center}
    \caption{Synopsis of the reviewed VPs for each technical design category}
    \label{table:technicalDesignTable}
    \begin{tabular}{| p{1.8cm} | p{2cm} | p{8cm} |}
    \hline
        \rowcolor{mygray}
        \multicolumn{3}{|c|}{\textbf{Technical Design}}\\
    \hline
        \rowcolor{lightgray}
        \textbf{Category}  & \textbf{Subcategory} & \textbf{Virtual Patients}\\
    \hline
    %PRESENTATION FORMAT
        \multirow{5}{1.8cm}{Presentation format} & \emph{Image} & HOLLIE \cite{adefila2020students},  Marei \cite{marei2018use}\\
    \cline{2-3}
        & \emph{Video} & CynthiaYoungVP \cite{foster2016using}, Suicide Prevention \cite{o2019suicide}, VSPR \cite{peddle2019exploring,peddle2019development}\\
    \cline{2-3}
        & \emph{Desktop VR} & AtRiskInPrimaryCare \cite{albright2018using}, MPathic-VR \cite{guetterman2019medical,kron2017using}, NERVE
        \cite{hirumi2016advancingPart2,hirumi2016advancing,kleinsmith2015understanding}, Jacklin \cite{jacklin2019virtual,jacklin2018improving}, 
        Communicate! \cite{jeuring2015communicate}, Richardson \cite{richardson2019virtual}, Schoenthaler \cite{schoenthaler2017simulated}, Szilas \cite{szilas2019virtual}, UTTimePortal \cite{zielke2016beyond,zielke2016using}, Zlotos \cite{zlotos2016scenario}\\
    \cline{2-3}
        & \emph{Large volume display} & Dupuy \cite{dupuy2019virtual}, Banszki \cite{banszki2018clinical,quail2016student}, Maicher \cite{maicher2017developing}, Washburn \cite{washburn2020virtual}\\
    \cline{2-3}
        & \emph{Immersive VR} & Ochs \cite{ochs2019training}, CESTOLVRClinic \cite{sapkaroski2018implementation}\\
    \hline
    %INPUT INTERFACE
        \multirow{4}{1.8cm}{Input interface} & \emph{Typed} & HOLLIE \cite{adefila2020students}, AtRiskInPrimaryCare \cite{albright2018using}, CynthiaYoungVP \cite{foster2016using}, NERVE
        \cite{hirumi2016advancingPart2,hirumi2016advancing,kleinsmith2015understanding}, Jacklin \cite{jacklin2019virtual,jacklin2018improving}, Communicate! \cite{jeuring2015communicate}, Maicher \cite{maicher2017developing}, Marei \cite{marei2018use}, Suicide Prevention \cite{o2019suicide}, VSPR \cite{peddle2019exploring,peddle2019development}, Richardson \cite{richardson2019virtual}, CESTOLVRClinic \cite{sapkaroski2018implementation}, Schoenthaler \cite{schoenthaler2017simulated}, Szilas \cite{szilas2019virtual}, UTTimePortal \cite{zielke2016beyond,zielke2016using}, Zlotos \cite{zlotos2016scenario}\\
    \cline{2-3}
        & \emph{Voice-controlled} & Banszki \cite{banszki2018clinical,quail2016student}, Dupuy \cite{dupuy2019virtual}, MPathic-VR \cite{guetterman2019medical,kron2017using}, Maicher \cite{maicher2017developing}, Ochs \cite{ochs2019training}, CESTOLVRClinic \cite{sapkaroski2018implementation}\\
    \cline{2-3}
        %& \emph{NUI} & CESTOLVRClinic \cite{sapkaroski2018implementation}\\
    %\cline{2-3}
        & \emph{Hybrid} & Washburn \cite{washburn2020virtual} \\
    \cline{2-3}
        & \emph{Non-verbal} & Banszki \cite{banszki2018clinical,quail2016student}, Dupuy \cite{dupuy2019virtual}, MPathic-VR \cite{guetterman2019medical,kron2017using}, Maicher \cite{maicher2017developing}, CESTOLVRClinic \cite{sapkaroski2018implementation}\\
    \hline
    % DISTRIBUTION
        \multirow{3}{*}{Distribution} & \emph{Standalone} & Banszki \cite{banszki2018clinical,quail2016student}, Dupuy \cite{dupuy2019virtual}, MPathic-VR \cite{guetterman2019medical,kron2017using}, Maicher \cite{maicher2017developing}, Marei \cite{marei2018use}, Ochs \cite{ochs2019training}, CESTOLVRClinic \cite{sapkaroski2018implementation}, Szilas \cite{szilas2019virtual}, Washburn \cite{washburn2020virtual}\\
    \cline{2-3}
        & \emph{Web-based} & HOLLIE \cite{adefila2020students}, AtRiskInPrimaryCare \cite{albright2018using}, CynthiaYoungVP \cite{foster2016using}, NERVE
        \cite{hirumi2016advancingPart2,hirumi2016advancing,kleinsmith2015understanding}, Jacklin \cite{jacklin2019virtual,jacklin2018improving}, Maicher \cite{maicher2017developing}, Suicide Prevention \cite{o2019suicide}, VSPR \cite{peddle2019exploring,peddle2019development}, Richardson \cite{richardson2019virtual}, UTTimePortal \cite{zielke2016beyond,zielke2016using}, Zlotos \cite{zlotos2016scenario}\\
    \cline{2-3}
         & \emph{Undisclosed} &  Communicate! \cite{jeuring2015communicate}, Schoenthaler \cite{schoenthaler2017simulated} \\
    \hline
     \end{tabular}
\end{center}
}
\end{table}
\normalsize

\textbf{Structure}. The \emph{structure} defines the hierarchical organization and presentation of VP-related information within the simulation.
According to \cite{bearman2001random}, two non-mutually exclusive approaches (i.e., \emph{narrative} and \emph{problem solving}) can be defined. 
The narrative VPs are characterized by a coherent storyline, with a focus on cause-effect decisions that have a direct impact on the evolution of the simulation. These VPs present the patient as more than a mere collection of data and statistics, and devote a certain degree of attention to interpersonal and communication aspects of the provider-patient interaction.  On the contrary, the \emph{problem solving} VPs are mainly used to support inquiry-based learning scenarios such as teaching clinical reasoning, differential diagnosis, and history-taking skills. These contexts do not usually concern  portraying authentic communicative acts, since they mainly involve making questions and observations. 

Scholars and researchers recognize the power of \emph{narrative} design in the creation of meaningful learning experiences \cite{bearman2001random,marei2018use}.
Narrative-based simulations that reflect the consequences of the choices and the actions made by the learner can lead to the development of more effective VPs.  In particular, for VPs used to teach communication skills, experimental  evidence supports the value of \emph{narrative} design \cite{bearman2001random}. Thus, it is not surprising that all the VPs presented in the selected works are based on this approach. Nevertheless, it is interesting to note that 10 out the \totalVPs VPs analyzed integrate the \emph{narrative} design with a \emph{problem solving} component. 
This component aims to teach particular skills like history-taking (Cynthia Young VP \cite{foster2016using}, NERVE \cite{hirumi2016advancingPart2,hirumi2016advancing,kleinsmith2015understanding}, Maicher \cite{maicher2017developing}), clinical reasoning (VSPR \cite{peddle2019exploring,peddle2019development}, Richardson \cite{richardson2019virtual},  Washburn \cite{washburn2020virtual}, UT-Time Portal \cite{zielke2016beyond,zielke2016using}, Zlotos \cite{zlotos2016scenario}),  physical examinations (HOLLIE \cite{adefila2020students}, NERVE \cite{hirumi2016advancingPart2,hirumi2016advancing,kleinsmith2015understanding},  CESTOL VR Clinic \cite{sapkaroski2018implementation}), compilation and consultation of electronic medical records (HOLLIE \cite{adefila2020students}, Maicher \cite{maicher2017developing}), and medication administration (HOLLIE \cite{adefila2020students}). 





\textbf{Unfolding}. Given the prominence of narrative design in the development of VPs for communication skill training, another relevant design element is defining how the narrative may unfold, and how the simulation can evolve between different states. A preliminary subdivision can be made among \textit{linear} and \textit{non-linear} narratives. In the former design, VPs have a linear path to follow and the decisions, questions and options possibly presented to the learner do not influence the simulation outcome. It is clear that this design severely limits learning effectiveness, and none of the works included in this survey implemented it.

On the contrary, the \emph{non-linear} navigational structure of VPs offers learners a greater flexibility, and an higher degrees of interactivity and control. In this case, two further choices are possible. In the \textit{closed-option} design, the simulation advances to the next state by selecting one of the possible alternatives or explicit paths offered to learners. Simulation states are organized in a hierarchical structure (similar to that of the \quotes{choose your own adventure} books), which stresses the cause-effect relation of the user's choices. 
The \textit{open-option} design (sometimes referred in the literature as \quotes{free-text} \cite{jacklin2019virtual,janda2004simulation,mccoy2016evaluating}  or \quotes{open-chat} \cite{hirumi2016advancing}) can be used to develop  free-form simulations where states are organized in a partially or fully interconnected structure, and users are free to interact with the VP as they wish, thus emulating the flow of a real conversation. As we will discuss  more in detail in Section \ref{sec:technicalDesign}, learners can formulate questions and statements by either typing or having their speech transcribed into written words using speech-to-text software. Then, the application parses the text and elaborates a proper response. The VP state progression can be influenced also by non-verbal cues such as gestures, body posture, expressions and sight.


%In regards to unfolding, several works report that many users feel 'restricted' by the closed-option interface \cite{dupuy2019virtual,peddle2019development,jacklin2019virtual,hirumi2016advancing}, preferring an open-option structure or the possibility to chose between the two variants, like in NERVE \cite{hirumi2016advancingPart2,hirumi2016advancing,kleinsmith2015understanding}.


The \textit{closed-option} design characterizes most of the analyzed VPs (15), with only four works  based on an \textit{open-option} design; as for the remaining, one VP implemented both options (NERVE \cite{hirumi2016advancingPart2,hirumi2016advancing,kleinsmith2015understanding}), whereas the other one can be considered an hybrid between the two designs (Ochs \cite{ochs2019training}). One explanation for this result is the lower complexity of the \textit{closed-option} implementation, although some Authors \cite{carnell2015adapting,jacklin2019virtual} argued that a such an approach may be more suitable for novices who, for example, may still be inexperienced about the procedures to follow in a patient encounter. However, other works reported that many students feel restricted by the \textit{closed-option} interface \cite{dupuy2019virtual,hirumi2016advancing,jacklin2019virtual,peddle2019development}, preferring either an \textit{open-option} structure or the possibility to chose between the two variants. It is worth noting that implementing both options allows the use of the same VP in different educational settings. %\edo{precisazione}For instance, in NERVE \cite{hirumi2016advancingPart2,hirumi2016advancing,kleinsmith2015understanding} the two systems coexist, to leverage the less stress-inducing \textit{closed-option} variant in the learning process, and the less restrictive \textit{open-option} setting for rehearsal sessions. 
%Ho cambiato leggermente questo paragrafo perché sembrava che in NERVE ci fosse una netta distinzione tra sessioni di evaluation e learning, in realtà questa potenziale divisione è solo una proposta/considerazione degli autori
For instance, in NERVE \cite{hirumi2016advancingPart2,hirumi2016advancing,kleinsmith2015understanding}, the less stress-inducing \textit{closed-option} variant is used in the learning sessions, while rehearsal sessions leverage the less restrictive \textit{open-option} setting. 
% This concept ties to another common feedback given by users: the idea that the same simulations should have diversified levels of difficulty \cite{dupuy2019virtual}, \cite{peddle2019development}, \cite{peddle2019exploring} to appropriately challenge learners of different skill levels.
% \edo{end of addon about effectiveness}
Another interesting approach is the hybrid model implemented in Ochs \cite{ochs2019training}, where the user can freely interact through voice with the VP. Then, a human facilitator selects, from a set of possible closed-options, the utterance that semantically resembles the original phrase the most, prompting the appropriate response from the patient. The advantage of this approach is that it ease the development burden of what appears to learners as an \textit{open-option} VP, with the clear disadvantage of preventing its use as a self-learning and self-evaluation tool.



\textbf{Feedback}. With the term \emph{feedback} we refer to any form of instructional scaffolding enclosed in the simulation itself (i.e., we exclude any feedback external to the simulation, such as post-simulation debrief and reflection sessions with mentors and peers).
%(i.e., we exclude any out of simulation feedback such as post-simulation debrief and reflection sessions with mentors and peers).
Feedback can be given in many different forms, from explicit messages to discoveries made, questions answered, and visual representations of the current VP state. 

While researchers recognize the relevance of immediate and after-action feedback as an essential feature in communication-based VPs (\cite{adefila2020students,jacklin2018improving,marei2018use,peddle2019exploring,quail2016student}), 
%\cite{kleinsmith2015understanding,fiorella2012applying,mckimm2006abc,albright2018using,maicher2017developing,mcgaghie2011does,barry2005features,ericsson1993role,huwendiek2009design,botezatu2010virtual}), 
it is surprising that six of the VPs surveyed (Banszki \cite{banszki2018clinical,quail2016student}, Maicher \cite{maicher2017developing}, Marei \cite{marei2018use}, Szilas \cite{szilas2019virtual}, Washburn \cite{washburn2020virtual}, CESTOLVRClinic \cite{sapkaroski2018implementation}) do not embed any type of built-in feedback. 
The remaining works take different approaches. Four VPs (Dupuy \cite{dupuy2019virtual}, Communicate! \cite{jeuring2015communicate}, VSPR \cite{peddle2019exploring,peddle2019development}, Zlotos \cite{zlotos2016scenario}) offer the possibility, after the simulation is completed, to \textit{replay} some of its parts and analyze the outcome of different choices. 
Three simulations (At-Risk in Primary Care \cite{albright2018using}, Suicide Prevention \cite{o2019suicide},  Schoenthaler \cite{schoenthaler2017simulated}) include a \emph{virtual instructor}, i.e., a virtual tutor that gives advice or feedback based on the user's choices.
MPathic-VR \cite{guetterman2019medical,kron2017using} employs a \emph{multiple session structure}, where the first run acts as a learning phase, concluded by an automated and complete feedback provided by the system, whereas the second run (set in the same scenario) serves as an evaluation phase. This type of structure appears to be highly appreciated by students since they can immediately put into practice what they learned during the first phase, taking into account the feedback received.
Two VPs (At-Risk in Primary Care \cite{albright2018using}, Schoenthaler \cite{schoenthaler2017simulated}) feature \emph{quantitative emotional feedback} in the form of an on-screen trust meter that indicates users how effective their communication choices were at building a relation with the patient. Two VPs (Jacklin \cite{jacklin2019virtual,jacklin2018improving}, Richardson \cite{richardson2019virtual}) offer learners a \emph{personalized qualitative feedback} at the end of the simulation. CynthiaYoungVP \cite{foster2016using} uses an hybrid approach between automated and human feedback. At the end of the simulation, the students can access a web page containing \emph{empathy feedback} and scores for each response given, where scores are manually assigned by human experts. 
The approach adopted in NERVE \cite{hirumi2016advancing,hirumi2016advancingPart2,kleinsmith2015understanding} is to inform learners about the number of \emph{clinical discoveries} and empathic responses available, thus providing inexperienced users with useful guidelines on how to proceed with the conversation. Although it has not been  implemented yet, the work in  \cite{hirumi2016advancingPart2} put forth the proposition of providing \quotes{cumulative feedback} on how users are developing their skills across multiple patient scenarios. Authors suggest that this feature could both encourage repeated use of the system and act as a motivator for performance improvement. 
Finally, there is a number of VPs (HOLLIE \cite{adefila2020students}, Dupuy \cite{dupuy2019virtual}, Communicate! \cite{jeuring2015communicate}, MPathic-VR \cite{guetterman2019medical,kron2017using}, Schoenthaler \cite{schoenthaler2017simulated}, UT-Time Portal \cite{zielke2016beyond,zielke2016using}) that leverage \emph{game elements} as feedback. Since the introduction of game elements is a relevant design feature, it is discussed in detail in the  following subsection. 


%GAMIFICATION 
\textbf{Gamification}. The idea of introducing game mechanics in any learning experience is to make them more enjoyable and engaging.  Researchers and practitioners recognize that game mechanics contribute to making the learning experience more effective, fostering self-improvement and healthy competition between peers \cite{benedict2013promotion,festinger1954theory}. 
The mechanics used the most in \quotes{gamified} experiences are \emph{scores}, \emph{badges} and \emph{leaderboards}. 
Scores are a quantitative and immediate form of feedback that acts as an extrinsic motivator to foster users to improve their performance. \emph{Scoring systems} can be found in Dupuy \cite{dupuy2019virtual}, Communicate! \cite{jeuring2015communicate}, MPathic-VR \cite{guetterman2019medical,kron2017using}, Schoenthaler \cite{schoenthaler2017simulated} and UT-Time Portal \cite{zielke2016beyond,zielke2016using}. In particular, Communicate! \cite{jeuring2015communicate} and Schoenthaler \cite{schoenthaler2017simulated} provide separate scores for each learning goal (e.g., empathy control, language clarity, and pick up of patient's concerns). Such a feature can help learners to tell the areas in which they are already proficient, distinguishing them from those needing improvement.

Badges are visual representations used in games to prove that the player has reached an intermediate goal on his/her road to mastery. Their purpose is twofold: they are a form of gratification to the learners, and they allow trainees to share achievements with peers and educators. Thus, they also represent an extrinsic motivator for improvement.
%Thus, they represent both an intrinsic and extrinsic motivator for improvement.
In our survey,  the only simulation we found that implements a  \emph{badge system} is UT-Time Portal \cite{zielke2016beyond,zielke2016using}. VSPR \cite{peddle2019exploring,peddle2019development} features a system of certificates issued to users at the end of each learning module which shall be regarded as an \quotes{intrinsic-only} motivator, since there is no overarching structure that enables users to see each others' achievements.
%A cosa servono i rankings, perché nessuno li implementa?

Finally, leaderboards (or rankings) are a primarily extrinsic motivator that leverage competition with peers when they compare their performance to that of others (it should be noted that, for highly competitive individuals, the act of \quotes{climbing the leaderboard} can also be seen as a relevant intrinsic motivator independent of the context).
%Finally, leaderboards (or rankings) are a relevant intrinsic motivator for learners that leverage competition with peers when they compare their performance to that of others.
Surprisingly, despite their demonstrated benefits for learning, in our survey, we found no example of public ranking and leaderboards.

A final note is for HOLLIE \cite{adefila2020students}, which implements the very peculiar idea of a Tamagotchi-style VP the learners have to care for (adequately, at regular intervals and in real-time) over two weeks. Here, the leading game mechanics (constant care over a long period) reproduces quite accurately the daily tasks of a nurse leveraging the innate sense of responsibility in the players. 


%%TABLE WITH TECHNICAL DESGIN ONLY
%\begin{longtable}{ | p{1cm} | *{15}{l} |}
%\begin{tabularx}{\linewidth} {>{\raggedleft\arraybackslash}X >{\raggedleft\arraybackslash}X >{\raggedleft\arraybackslash}X >{\raggedleft\arraybackslash}X >{\raggedleft\arraybackslash}X >{\raggedleft\arraybackslash}X >{\raggedleft\arraybackslash}X >{\raggedleft\arraybackslash}X >{\raggedleft\arraybackslash}X >{\raggedleft\arraybackslash}X}
{\rowcolors{3}{mywhite}{mygray}
%\begin{tabularx}{\linewidth} {X | X | X | X | X | X | X | X | X | X}
\begin{tabularx}{\linewidth}
\footnotesize
{|>{\hsize=0.5\hsize\linewidth=\hsize}X |
>{\hsize=1.5\hsize\linewidth=\hsize}X |
>{\hsize=.75\hsize\linewidth=\hsize}X |
>{\hsize=.5\hsize\linewidth=\hsize}X |
>{\hsize=.75\hsize\linewidth=\hsize}X |
>{\hsize=2\hsize\linewidth=\hsize}X |}


%\textbf{Study} & \textbf{Geographical Location and Level} & \textbf{Identification} & 
%\textbf{Multiplier Result} \\ \hline \hline
%\endfirsthead
\rowcolor{lightgray}
\textbf{Article} & \multicolumn{5}{|l|}{\textbf{Technical Design}} \\
\rowcolor{lightgray}
& Hardware & Presentation & Input Interface & Distribution & Other Tech. Features \\
%& \multicolumn{1}{|l|}{Category} & \multicolumn{1}{|l|}{Navigation} & \multicolumn{1}{|l|}{Feedback} & \multicolumn{1}{|l|}{Gamification} & \multicolumn{1}{|l|}{Hardware} & \multicolumn{1}{|l|}{Presentation} & \multicolumn{1}{|l|}{Input} & \multicolumn{1}{|l|}{Distribution} & \multicolumn{1}{|l|}{Other Tech. Features}  \\ 
%\hline 
\specialrule{.1em}{.05em}{.05em} 
\endhead

\cite{adefila2020students} & Any device with a web browser	& Graphic (Image) & Typed &	Web-Based & /\\ 

\cite{albright2018using} & Any device with a web browser & Graphic (Image) & Typed	& Web-Based & /\\ 

\cite{banszki2018clinical} + \cite{quail2016student} & PC, 64” monitor, Microphone & Graphic (3D) & Voice-Controlled & Standalone &	Human-Controlled,Gesture and Facial Expression Output \\  

\cite{dupuy2019virtual} & PC, Vertical 40” screen, Microphone, Camera & Graphic (3D) & Voice-Controlled & Standalone &	Facial Expression Detection through face tracking\\ 

\cite{foster2016using} & Any device with a web browser & Text-Based, Graphic (Video) & Typed & Web-Based & Human-Controlled Empathy Feedback\\ 

\cite{guetterman2019medical} + \cite{kron2017using} & PC, Microsoft Kinect, Microphone & Graphic (3D) & Voice-Controlled &	Standalone &	Facial Expression and Body Posture Detection and output, Recorded Voiceover, Multiple VHs\\ 

\cite{hirumi2016advancingPart2} + \cite{hirumi2016advancing} + \cite{kleinsmith2015understanding} & Any device with a web browser	& Graphic (3D) & Typed & Web-Based & /\\ 

\cite{jacklin2019virtual} + \cite{jacklin2018improving} & Any device with a web browser & Graphic (3D) & Typed &	Web-Based & Body Posture Output, Recorded Voiceover	\\ 

\cite{jeuring2015communicate} & \textbf{\emph{Unclear}} & Graphic (3D) & Typed & \textbf{\emph{Unclear}} & Scenario Editor\\ 

\cite{maicher2017developing} & PC, Microsoft Kinect, Multi-Array Microphone & Graphic (3D) & Typed + Voice-Controlled & Web-Based + Standalone & Motion-Captured Animations, Movement Posture Detection and Output\\

\cite{marei2018use} & PC & Graphic (Image) &	Typed & \textbf{\emph{Unclear}} & /\\ 

\cite{ochs2019training} & PC, HMD (Oculus Rift), CAVE, High-End Microphone & Graphic (3D, IVR) & Voice-Controlled &  Standalone & Same VP deployed in desktop, VR and Cave versions. Speech recognition is Human-Controlled, Non-Verbal output (facial expression and body posture). Text-To-Speech, Lip Synch, Virtual Playback\\

\cite{o2019suicide} & Any device with a web browser & Graphic (Video) & 	Typed & Web-Based & /\\ 

\cite{peddle2019exploring} + \cite{peddle2019development} & Any device with a web browser & Graphic (Video) & Typed & Web-Based & /\\ 

\cite{richardson2019virtual} & Any device with a web browser & Graphic (3D) &	Typed &	Web-Based & /\\

\cite{sapkaroski2018implementation} & PC, HMD (Various), Leap Motion, Oculus Touch & Graphic (3D, IVR) & Typed + Voice-Controlled + NUI & Standalone & Remote progress tracking for educators\\ 

\cite{schoenthaler2017simulated} & \textbf{\emph{Unclear}} &  Graphic (3D) & Typed & 	\textbf{\emph{Unclear}} &	Can play as both provider and patient, Non-Verbal Output\\ 

\cite{szilas2019virtual} & PC	& Graphic (3D) &	Typed & Standalone & /\\ 

\cite{washburn2020virtual} & PC or Laptop, Large Screen & Graphic (3D) & Hybrid & Standalone & Human-Transcribed Voice Controls\\ 

\cite{zielke2016beyond} + \cite{zielke2016using} & Any device with a web browser & Graphic (3D) & Typed & 	Web-Based & Multiple VHs, Non-Verbal Output\\ 

\cite{zlotos2016scenario} & Any device with a web browser & Graphic (3D) & Typed  & Web-Based & Motion-Captured Animations, Recorded Voiceover\\ 

\hline
%\end{longtable}
\end{tabularx}
}



\subsection{Technical features}
\label{sec:technicalDesign}

This category explores, from a technical perspective, the different solutions that can support (and implement) the choices made in the instructional design, i.e., which are the technical features that enable the accomplishment of the envisioned learning activities. These features include the physical devices required to guarantee the exchange of information between the learner and the system, and the possible communication infrastructure needed to run the simulation.

%%Technical Table: NEW VERSION
%ditched tabularx, use only normal tabular

\begin{table} [t]
\scriptsize{
\begin{center}
    \caption{Synopsis of the reviewed VPs for each technical design category}
    \label{table:technicalDesignTable}
    \begin{tabular}{| p{1.8cm} | p{2cm} | p{8cm} |}
    \hline
        \rowcolor{mygray}
        \multicolumn{3}{|c|}{\textbf{Technical Design}}\\
    \hline
        \rowcolor{lightgray}
        \textbf{Category}  & \textbf{Subcategory} & \textbf{Virtual Patients}\\
    \hline
    %PRESENTATION FORMAT
        \multirow{5}{1.8cm}{Presentation format} & \emph{Image} & HOLLIE \cite{adefila2020students},  Marei \cite{marei2018use}\\
    \cline{2-3}
        & \emph{Video} & CynthiaYoungVP \cite{foster2016using}, Suicide Prevention \cite{o2019suicide}, VSPR \cite{peddle2019exploring,peddle2019development}\\
    \cline{2-3}
        & \emph{Desktop VR} & AtRiskInPrimaryCare \cite{albright2018using}, MPathic-VR \cite{guetterman2019medical,kron2017using}, NERVE
        \cite{hirumi2016advancingPart2,hirumi2016advancing,kleinsmith2015understanding}, Jacklin \cite{jacklin2019virtual,jacklin2018improving}, 
        Communicate! \cite{jeuring2015communicate}, Richardson \cite{richardson2019virtual}, Schoenthaler \cite{schoenthaler2017simulated}, Szilas \cite{szilas2019virtual}, UTTimePortal \cite{zielke2016beyond,zielke2016using}, Zlotos \cite{zlotos2016scenario}\\
    \cline{2-3}
        & \emph{Large volume display} & Dupuy \cite{dupuy2019virtual}, Banszki \cite{banszki2018clinical,quail2016student}, Maicher \cite{maicher2017developing}, Washburn \cite{washburn2020virtual}\\
    \cline{2-3}
        & \emph{Immersive VR} & Ochs \cite{ochs2019training}, CESTOLVRClinic \cite{sapkaroski2018implementation}\\
    \hline
    %INPUT INTERFACE
        \multirow{4}{1.8cm}{Input interface} & \emph{Typed} & HOLLIE \cite{adefila2020students}, AtRiskInPrimaryCare \cite{albright2018using}, CynthiaYoungVP \cite{foster2016using}, NERVE
        \cite{hirumi2016advancingPart2,hirumi2016advancing,kleinsmith2015understanding}, Jacklin \cite{jacklin2019virtual,jacklin2018improving}, Communicate! \cite{jeuring2015communicate}, Maicher \cite{maicher2017developing}, Marei \cite{marei2018use}, Suicide Prevention \cite{o2019suicide}, VSPR \cite{peddle2019exploring,peddle2019development}, Richardson \cite{richardson2019virtual}, CESTOLVRClinic \cite{sapkaroski2018implementation}, Schoenthaler \cite{schoenthaler2017simulated}, Szilas \cite{szilas2019virtual}, UTTimePortal \cite{zielke2016beyond,zielke2016using}, Zlotos \cite{zlotos2016scenario}\\
    \cline{2-3}
        & \emph{Voice-controlled} & Banszki \cite{banszki2018clinical,quail2016student}, Dupuy \cite{dupuy2019virtual}, MPathic-VR \cite{guetterman2019medical,kron2017using}, Maicher \cite{maicher2017developing}, Ochs \cite{ochs2019training}, CESTOLVRClinic \cite{sapkaroski2018implementation}\\
    \cline{2-3}
        %& \emph{NUI} & CESTOLVRClinic \cite{sapkaroski2018implementation}\\
    %\cline{2-3}
        & \emph{Hybrid} & Washburn \cite{washburn2020virtual} \\
    \cline{2-3}
        & \emph{Non-verbal} & Banszki \cite{banszki2018clinical,quail2016student}, Dupuy \cite{dupuy2019virtual}, MPathic-VR \cite{guetterman2019medical,kron2017using}, Maicher \cite{maicher2017developing}, CESTOLVRClinic \cite{sapkaroski2018implementation}\\
    \hline
    % DISTRIBUTION
        \multirow{3}{*}{Distribution} & \emph{Standalone} & Banszki \cite{banszki2018clinical,quail2016student}, Dupuy \cite{dupuy2019virtual}, MPathic-VR \cite{guetterman2019medical,kron2017using}, Maicher \cite{maicher2017developing}, Marei \cite{marei2018use}, Ochs \cite{ochs2019training}, CESTOLVRClinic \cite{sapkaroski2018implementation}, Szilas \cite{szilas2019virtual}, Washburn \cite{washburn2020virtual}\\
    \cline{2-3}
        & \emph{Web-based} & HOLLIE \cite{adefila2020students}, AtRiskInPrimaryCare \cite{albright2018using}, CynthiaYoungVP \cite{foster2016using}, NERVE
        \cite{hirumi2016advancingPart2,hirumi2016advancing,kleinsmith2015understanding}, Jacklin \cite{jacklin2019virtual,jacklin2018improving}, Maicher \cite{maicher2017developing}, Suicide Prevention \cite{o2019suicide}, VSPR \cite{peddle2019exploring,peddle2019development}, Richardson \cite{richardson2019virtual}, UTTimePortal \cite{zielke2016beyond,zielke2016using}, Zlotos \cite{zlotos2016scenario}\\
    \cline{2-3}
         & \emph{Undisclosed} &  Communicate! \cite{jeuring2015communicate}, Schoenthaler \cite{schoenthaler2017simulated} \\
    \hline
     \end{tabular}
\end{center}
}
\end{table}
\normalsize


\textbf{Presentation format}. The surveyed works provide learners with different types of outputs aimed to deliver VP information to the learner and presenting the VP itself. A first rough subdivision is between \textit{text-based} and \textit{graphic} representations. In screen-based text simulators, the VP is presented mainly in the form of a collection of text and structured data, with the possible inclusion of images portraying a static patient or exam results. However, the lack of a graphic component capable of displaying a patient that can express emotions as the simulation unfolds (and, consequently, change posture and facial expressions according to its current  state) is one of the main limitations of these approaches. Therefore, researchers started extending text-based simulations into learning activities with a relevant graphic component. 

All the VPs surveyed in this work fall in the \emph{graphic} category, which can be further classified in \emph{image}, \emph{video} and \emph{3D}. VPs in the \emph{image} subclass are presented through a series of static images (either photographs or drawings), such as HOLLIE \cite{adefila2020students} and Marei \cite{marei2018use}.  Some VPs present their case using \emph{video}, either in the form of live footage (Suicide Prevention \cite{o2019suicide}, VSPR \cite{peddle2019exploring,peddle2019development}) or as a computer-generated offline video (Cynthia Young VP \cite{foster2016using}). However, a clear limitation of this approach is its lack of flexibility, since the actor video cannot be re-purposed to portray a different clinical case. 

The majority of surveyed simulations fall in the \emph{3D} subclass and present the patient and the environment as 3D models rendered in real-time.  Their main advantage is that tweaking and expanding a simulation using 3D characters can be done in a much more modular fashion than with \emph{image} and \emph{video}-based VPs. Another advantage is that the sense of immersion and presence are greater than those that can be delivered by \emph{image} and \emph{video}-based VPs. 

Most of the 3D approaches rely on standard \emph{desktop VR} (DVR) settings (10, namely AtRiskInPrimaryCare \cite{albright2018using}, MPathic-VR \cite{guetterman2019medical,kron2017using}, NERVE \cite{hirumi2016advancingPart2,hirumi2016advancing,kleinsmith2015understanding}, Jacklin \cite{jacklin2019virtual,jacklin2018improving}, Communicate! \cite{jeuring2015communicate}, Richardson \cite{richardson2019virtual}, Schoenthaler \cite{schoenthaler2017simulated}, Szilas \cite{szilas2019virtual}, UTTimePortal \cite{zielke2016beyond,zielke2016using}, Zlotos \cite{zlotos2016scenario}). However, since trying to maximize the feeling of immersion and presence is extremely relevant for engaging learners and helping them achieve the expected learning outcomes, some works integrate (partially or fully) immersive technologies. Four of them (Dupuy \cite{dupuy2019virtual}, Banszki \cite{banszki2018clinical,quail2016student}, Maicher \cite{maicher2017developing}, Washburn \cite{washburn2020virtual}) take advantage of \emph{large volume displays}  to portray a life-sized and more natural interaction with the patient, and CESTOL VR Clinic \cite{sapkaroski2018implementation} uses an HMD for the same purpose. In Ochs \cite{ochs2019training}, three different setups (DVR, immersive VR with an HMD, and immersive VR in a CAVE) are compared to analyze their effect on the sense of presence. The outcome of this experiment demonstrates that immersive environments improve the sense of presence and perception of the VP, with the CAVE scoring slightly better than the HMD.
It should be noted, however, that while \emph{immersive VR} (IVR) offers a higher degree of immersion and presence over DVR, there are still accessibility issues that limit its use, in particular when the VP is intended for self-learning and self-training.


\textbf{Input Interface}. This category describes the input methods through which the user influences the unfolding of the VP simulation.
In the case of \emph{typed} interfaces, user's textual intents are entered by typing on a keyboard or selecting an item in a predefined list of choices.
\emph{Voice-controlled} simulations use natural language, which is then parsed into text through a speech-to-text module, usually offered by external Natural Language Processing (NLP) APIs \cite{foster2016using,maicher2017developing}. 
Finally, the integration within the simulation of Natural User Interfaces (NUI) allows to influence the VP state evolution through additional \textit{non-verbal} input cues such as eye contact, distance, facial expression, gestures and body posture, which can be captured with cameras and other hardware. % to provide additional input cues for the simulation.


Among the analyzed VPs, 14 feature a \emph{typed}-only input, only five are \emph{voice-controlled} (Banszki \cite{banszki2018clinical,quail2016student}, Dupuy \cite{dupuy2019virtual}, MPathic-VR \cite{guetterman2019medical,kron2017using}, Ochs \cite{ochs2019training}, CESTOL VR Clinic \cite{sapkaroski2018implementation}), Maicher \cite{maicher2017developing} has both options, and Washburn \cite{washburn2020virtual} can be considered as a hybrid solution since a human facilitator transcribes the spoken commands through a \emph{typed} interface. 
One of the reasons behind the limited use of voice controls is the fear, expressed by some Authors \cite{maicher2017developing,ochs2019training}, that  NLP systems may be technically hard to implement and prone to wrong transcriptions, which may lead to misunderstandings or unrecognized utterances, break the sense of immersion and cause frustration in the user \cite{bloodworth2010initial}.
This is why some Authors (e.g., Banszki \cite{banszki2018clinical,quail2016student}, Ochs \cite{ochs2019training}, and Washburn \cite{washburn2020virtual}) decided to have a human facilitator taking over the function of the NLP module. 
 Moreover, a VP featuring only voice controls cannot be used by learners with speech impairments \cite{maicher2017developing}. However, it should be stressed that, nowadays, speech-to-text APIs have become widely available, and their quality keeps improving; thus, problems related to imprecise transcriptions should be less and less daunting in the coming years. As for the impaired people, a smart solution to achieve maximum flexibility and accessibility could be to let the users  choose between  \emph{typed} and \emph{voice-controlled} interfaces freely. It should also be noted that IVR environments favour the use of \emph{voice-controlled} interfaces over alternative solutions such as virtual keyboards or situation-specific control boards  %, which requires for interaction the use of specific controllers, gestures or gaze/head tracking 
 \cite{sapkaroski2018implementation}, which are likely to break the sense of immersion and presence and are often cumbersome to use.

Among the analyzed VPs, five of them support also \emph{non-verbal} input, by either leveraging NUI-based approaches (e.g., using RGBD sensors, like in MPathic-VR \cite{guetterman2019medical,kron2017using} and Maicher \cite{maicher2017developing}, or standard RGB cameras, like in Dupuy \cite{dupuy2019virtual}) or having a human controller that observes the user interacting with the VP and updates the VP's response accordingly in terms of gestures and facial expressions (like in Banszki \cite{banszki2018clinical,quail2016student}). 
However, apart from Banszki \cite{banszki2018clinical,quail2016student}, it appears that this information is largely underutilized to influence the VP's behavior. In Dupuy \cite{dupuy2019virtual}, the users' facial expressions are detected to merely assess their emotional state at the end of the simulation. In Maicher \cite{maicher2017developing}, the tracked user position is simply used to adjust the agent's gaze, and there is no specific mention of the way the simulation exploits gestures. Finally, in  MPathic-VR \cite{guetterman2019medical,kron2017using}, instead of continuously capturing \emph{non-verbal} communication, learners are forced to assume specific expressions and poses when prompted by the system. In summary, the above discussion highlights that sounder ways of using \emph{non-verbal} inputs are sorely needed in this particular research field.


\textbf{Distribution}. One relevant technological parameter of the VP simulation is the way the application is distributed (and how learners can access it). In principle, there are two main options. The first option is to deploy the VP as a \emph{web-based} application that can be accessed over the Internet. Such a simulation often runs inside a web browser (which makes it device-independent), and generally requires a low amount of computational resources. This flexibility can also foster self-learning (since simulations can be accessed at places and times convenient for the learner) and helps reduce costs (since learning can be carried out online). However, since \emph{web-based} applications are required to be portable on many devices (including mobile ones), they generally sacrifice technical characteristics and computational complexity in favour of accessibility. On the other hand, \emph{standalone} applications are deployed locally on a computer or workstation. These simulations can implement more advanced and complex features since they can leverage the full computational power of a dedicated machine, and integrate external devices or sensors (such as high-quality cameras and microphones).

In our survey, we found a total of ten \emph{web-based} and eight \emph{standalone} applications; in two cases, this information was undisclosed in the paper, whereas in one case (Maicher \cite{maicher2017developing}), the VP was deployed in both variants. This latter work is interesting since it shows how a \emph{standalone} version can trade off some of the flexibility of the web one with a broad array of  features (such as voice control and gesture/posture detection). 
The Authors observed that students were significantly more engaged with the \emph{standalone} VP, whereas in the web version they had to focus on typing and reading, which make them be less prone to notice the subtle non-verbal cues manifested by the patient.

% \edo{addon about effectiveness}
% This doesn't mean that a VP simulation must sacrifice accessibility over technical capabilities at all costs, in fact \cite{richardson2019virtual} mentions user feedback asking for a simulation that is available on multiple platforms (phones, tablets, desktop). So, while a VP that incorporates many interesting technical features may be preferable for maximizing user engagement, it being deployed also in a more accessible but feature-light is a good idea to maximize its potential use cases.
% \edo{end of addon about effectiveness}

Nonetheless, it should be stressed that technology is advancing rapidly, and personal devices come equipped with ever better microphones, cameras and computational power, which can reduce the technological gap between (desktop-only) \emph{standalone} applications and \emph{web-based} ones. Further discussions on this topic are included in Section \ref{sec:openResearch}.  
    

    
\subsection{Effectiveness of design elements}
\label{sec:effectiveness}
The general effectiveness of VPs on developing communication skills has been discussed by several Authors \cite{lee2020effective,peddle2016virtual,richardson2019virtualreview}. A common complaint in VP-related literature is the lack of a standardized terminology that, coupled with a considerable heterogeneity in study design, makes the retrieval and evaluation of relevant works a troublesome task. Despite this situation, both \cite{lee2020effective} and \cite{peddle2016virtual} concluded that, when appropriately contextualized in a well thought out educational context, VPs are indeed useful for developing, practising and building confidence about communication and other skills like, e.g., decision making and teamwork.  

Based on these findings, one possible question arising from our review is if the surveyed papers provide pieces of evidence about the effects on learning outcomes and efficacy of the simulation of the different instructional design elements and the technical features available. 
Unfortunately, the answer is negative. In most of the analyzed works, the Authors reported only users' feedback or comments about a particular element/feature, and a direct comparison between different design choices is missing. The only notable exceptions are three. The first one is represented by \cite{ochs2019training}, in which  different presentation formats were assessed, showing that immersive VR technologies yield superior results when compared to non-immersive ones. The second one concerns the distribution method  \cite{maicher2017developing}. The Authors found that a standalone application can provide a considerably higher level of engagement than its web-based counterpart thanks to the possibility to leverage advanced technical features (voice-controlled input and large volume displays) to increase immersion and focus on the task at hand. The third one compared closed and open-option unfolding designs, highlighting the advantages and disadvantages of each variant \cite{hirumi2016advancing}.
%reporting learners' preferences for the latter.  %Non proprio, hirumi non si sbilancia sull'una o sull'altra, semplicemente evidenzia i vantaggi e gli svantaggi dell'una o dell'altra

% In conclusion, the current literature lacks a thorough evaluation of the effectiveness of alternative designs, and further work has to be done to develop a better understanding of instructional elements and technical features that VP simulations can offer in order to achieve the desired learning outcomes. 
% \andrea{If we have something interesting to add (something like "In our opinion, a relevant contribution to this issue would be ...") we can do it here or in the open research questions}




% Pros & Cons  of ICL 
% Understandable
% \leiModify{Emphasize importance, e.g., difference with finetuning}
% The characteristics of ICL have obvious advantages. 
As a new paradigm, ICL has multiple attractive advantages. %ICL opens new directions for natural language processing research.
First, since the demonstration is written in natural language, it provides an interpretable interface to communicate with LLMs~\citep{gpt3}.
This paradigm makes it much easier to incorporate human knowledge into LLMs by changing the demonstration and templates~\citep{liu2022close,lu2022order,Wu2022SelfadaptiveIL,cot}. Second, in-context learning is similar to the decision process of human beings by learning from analogy~\citep{winston1980learningByAnalogy}. %, i.e., a human makes a prediction based on a few instructions and utilizes the experience from the past to generalize to new tasks.\zy{learning by analogy (ju yi fan san)} 
Third, compared with supervised training, ICL is a training-free learning framework. %ICL is data-efficient This high data efficiency of ICL makes this paradigm more ideal than previous data-hungry tuning methods for low-resource applications and scenarios~\citep{calibrate}. 
% Computation Efficiency 
%Finally, unlike the traditional supervised learning paradigm, which requires parameter updates on the model, there is no gradient computation in ICL. 
This could not only greatly reduce the computation costs for adapting the model to new tasks, but also make language-model-as-a-service~\citep{sun2022black} possible and can be easily applied to large-scale real-world tasks. % and can be easily applied to large-scale real-world tasks
% LMasA


% Directions and limitations?
% Advantage ->  Importance  
% Previous paradigm cannot solve. 
Despite being promising, there are also interesting questions and intriguing properties that require further investigation in ICL.
While the vanilla GPT-3 model itself shows promising ICL abilities, several studies  observed that the ability could be significantly boosted via adaption during pretraining~\citep{metaicl,chen2022sensitivity}.
In addition, the performance of ICL is sensitive to specific settings, including the prompting template, the selection of in-context examples, and order of examples, and so on~\citep{calibrate}. Furthermore, while intuitively reasonable, the working mechanism of the ICL remains unclear, and few studies have provided preliminary explanations~\citep{dai2022iclft,icl_gd}.
% Why we need this in-context learning survey 

% Our paper aims to sensitize the NLP community
% towards this growing area of work

With the rapid growth of studies in ICL, our survey aims to sensitize the community toward the current progress.
Specifically, we present a detailed paper survey with a paper list that will be continuously updated, and make an in-depth discussion on related studies of ICL. We highlight the challenges and potential directions and hope our work may provide a useful roadmap for beginners interested in this area and shed light on future research.

% \leiModify{this part seems too short? maybe elaborate more on the importance of this survey paper for the rapidly developing field, and the resource we could provide for the community.}
% % Directions current 
% \begin{figure}[t]
%     \centering
%     \includegraphics[width=0.9\linewidth]{fig/icl_process.pdf}
%     \caption{Main procedures of in-context learning. Pretraining is significant for developing the ICL ability of LLMs and the optional warmup stage can further improve it. For the demonstrations, the most important procedure is demonstration designing. With a pretrained LLM and a well-designed demonstration, a proper scoring strategy finally yield the task output.}
%     \label{fig:icl_process}
% \end{figure}

\section{Overview}
% \qingxiu{Modifying...}
%As illustrated in Figure~\ref{taxo_of_icl}, 
The strong performance of ICL relies on two stages: (1) the training stage that cultivates the ICL ability of LLMs, and (2) the inference stage where LLMs predict according to task-specific demonstrations.
%In terms of the training stage, pretraining is significant for bringing up the ICL ability and the optional warmup can further improve the ICL ability.
%In terms of the training stage, the language models are pre-trained with 
In terms of the training stage, LLMs are directly trained on language modeling objectives, such as left-to-right generation. Although the models are not specifically optimized for in-context learning, they still exhibit the ICL ability.
% \leiModify{the logic here seems not very clear?} 
Existing studies on ICL basically take a well-trained LLM as the backbone, and thus this survey will not cover the details of pretraining language models. 
Towards the inference stage, as the input and output labels are all represented in interpretable natural language templates, there are multiple directions for improving ICL performance. This paper will give a detailed description and comparison, such as selecting suitable examples for demonstrations and designing specific scoring methods for different tasks.
%Given a pretrained LLM and a well-designed demonstration, various scoring methods can be adopted for particular tasks.
% As illustrated in Fig.~\ref{fig:icl_process}, the LLM and the demonstration are two crucial components of ICL.
% In terms of the LLM, the pretraining stage is significant for bringing up the ICL ability and the optional warmup stage can further improve the ICL ability.
% Towards the demonstrations, as the input and output labels are all represented in interpretable language tokens, humans could design better promptings strategies such as selecting suitable examples for demonstrations and decomposing the reasoning procedure in text for hinting the model.
% Given a pretrained LLM and a well-designed demonstration, various scoring methods can be adopted for particular tasks.
% ICL consists of four main stages: (1) LLMs pretraining that ; (2) optional model warm-up that; (3) demonstration designing that (4) scoring. The main procedures are 
% detection to identify claims that require verification; (ii) evidence retrieval to find sources
% supporting or refuting the claim; and (iii) claim
% verification to assess the veracity of the claim
% based on the retrieved evidence. 
%With the formulation, 
% We notice that there are important components in the framework. 
% First, the core of in-context learning is the model itself, whose model scale and pretraining corpus have a great influence on the final prediction performance.
% % We discuss various model training techniques for improving the ICL ability of the model proposed in \S~\ref{sec:warmup}.
% Besides, as the input and output labels are all represented in interpretable language tokens, humans could design better promptings strategies such as selecting suitable examples for demonstrations and decomposing the reasoning procedure in text for hinting the model.
% % which we provide an overview of current progress in \S\ref{sec:prompt_designing}. 

% \leiModify{How to connect to the rest parts of this paper?}

% Structure of this paper
% With the rapid development of ICL,
We organize the current progress in ICL following the taxonomy above (as shown in Figure~\ref{taxo_of_icl}). 
With a formal definition of ICL~(\S\ref{sec:formulation}), 
% by providing a comprehensive
% birds-eye view of the area and formal definition of in-context learning methods~(\S\ref{sec:formulation}).
we provide a detailed discussion of the warmup approaches~(\S\ref{sec:warmup}), the demonstration designing strategies (\S\ref{sec:prompt_tuning}), and the main scoring functions(\S\ref{sec:scoring}). \S\ref{sec:analysis} provides in-depth discussions of current explorations on unveiling the secrets behind the ICL. 
We further provide useful evaluation and resources for ICL~(\S\ref{sec:evaluation}) and introduce potential application scenarios where ICL shows its effectiveness~(\S\ref{sec:application}).
Finally, we summarize the challenges and potential directions~(\S\ref{sec:challege_future}) and hope this could pave the way for researchers in this field.
% In this paper, we present a curated survey on the related studies of in-context learning, to give a comprehensive
% birds-eye view of the area.

% Remove Structure 
% The paper is structured as follows:
% We discuss the related work in 
% \S\ref{sec:related}, by categorising the ICL methods into three types. 
% Furthermore, we introduce the prompting strategies of ICL in \S\ref{sec:approach}, as the the design of in-context demonstrations has a great impact on the model performance. Finally, we point out several challenges and future directions to facilitate the community in \S\ref{sec:impact}.


\section{Definition and Formulation}
\label{sec:formulation}
% \begin{figure}[t!]
%     \centering
%     \includegraphics[width=0.95\linewidth]{fig/icl-category.pdf}
%     \caption{Category of in-context learning prediction.}
%     \label{fig:icl_type}
% \end{figure}

% \leiModify{.}
% In this section, we first formulate the ICL framework, then we provide a brief overview of the main content of this survey.
Following the paper of GPT-3~\cite{gpt3}, we provide a definition of in-context learning:
\textsl{In-context learning is a paradigm that allows language models to learn tasks given only a few examples in the form of demonstration.
}
% Without loss of generality, we formulate the in-context learning in a classification problem setup and the adaption to other task formats discussed later.
Essentially, it estimates the likelihood of the potential answer conditioned on the demonstration by using a well-trained language model. 


Formally, given a query input text $x$ and a set of candidate answers $Y = \{y_1, \ldots, y_m\}$ (Y could be class labels or a set of free text phrases), a pretrained language model $\mathcal{M}$ takes the candidate answer with the maximum score as the prediction conditioning a demonstration set $C$. $C$ contains an optional task instruction $I$ and $k$ demonstration examples; therefore, $C = \{ I, s(x_1, y_1), \ldots, s(x_k, y_k) \}$ or $C = \{ s(x_1, y_1), \ldots, s(x_k, y_k) \}$, where $s(x_k, y_k,I)$ is an example written in natural language texts according to the task.
% therefore, $C = \{ s(x_1, y_1,I), \ldots, s(x_k, y_k,I) \}$, where $s(x_k, y_k,I)$ is an example written in natural language texts following the task instruction $I$.
% For a candidate label in a label space, the probability of a simple label candidate $y_j$ as a word prediction problem:
The likelihood of a candidate answer $y_j$ could be represented by a scoring function $f$ of the whole input sequence with the model $\mathcal{M}$:
% \leiModify{A more abstract Scoring function f, details later.}
\begin{equation}
    P( y_j \mid x) \triangleq
    f_\mathcal{M} ( y_j,  C, x)
    % \frac{\text{PPL}_\mathcal{M} ( y_j \mid C, x)} { \sum_m \text{PPL}_\mathcal{M} ( y_m \mid C, x)}.
\end{equation}
% This formulation degenerates into a word prediction problem when the label could be represented with a single word in the vocabulary of the LLM.
The final predicted label $\hat y$ is the candidate answer with the highest probability:
\begin{equation}
    \hat y = \arg\max_{y_j \in Y } P(y_j | x). 
\end{equation}
The scoring function $f$ estimates how possible the current answer is given the demonstration and the query text. For example, we could predict the class label in a binary sentiment classification by comparing the token probability of \emph{Negative} and \emph{Positive}. There are many $f$ variants for different applications, which will be elaborated in \S\ref{sec:scoring}. 

According to the definition, we can see the difference between ICL and other related concepts. (1) Prompt Learning: Prompts can be discrete templates or soft parameters that encourage the model to predict the desired output. Strictly speaking, ICL can be regarded as a subclass of prompt tuning where the demonstration is part of the prompt. %In ICL, the prompt format must be human-readable discrete texts and contain several demonstration examples. 
~\citet{liu2021pre} made a thorough survey on prompt learning. However, ICL is not included. (2) Few-shot Learning: few-shot learning is a general machine learning approach that uses parameter adaptation to learn the best model parameters for the task with a limited number of supervised examples~\cite{wang2019few}. In contrast, ICL does not require parameter updates and is directly performed on pretrained LLMs.
% given a query input text x and a candidate answer set Y = {y1, y2, . . . , ym}
% , we perform inference for an instance in interest $(x, y)$ by performing language completion with the input $x$ and demonstrated examples $c$ as the condition:
% ADD implementation variety.
% Table illustrates. 
% PPL.
% where $c$ is the context provided and is represented as a concatenated of $k$ instances $c= \{ x^c_1, y^c_1, \ldots, x^c_k, y^c_k \}$. $k$ is usually set to a relatively small number, such as $16$, and the effect of the demonstration number could be found in \S~X.
% Note that here we assume that all the input and target labels are represented by tokens in the vocabulary of the language model via a mapping function. 
% use an example to better illustrate 

% For example, we could perform binary sentiment classification for a sentence $x=$ \textit{The service of that coffee is really great.}, by comparing the conditional token probability of the token \textit{Negative} and \textit{Positive}.
% Demonstration examples with a similar form could be used as a context, e.g., \textit{The food is really bad. Negative;  The movie is crazy and I love it very much. Positive.} to assist the prediction.
% In this form, the model learns to perform the prediction based on the conditional context tokens in the prefix, without any computation-intensive parameter updates. 
% For the situation where the target answer is free text, the model is used to decode the answer directly as a text generation problem. The performance is evaluated via automatic metrics or human evaluation.
% The summarization of these two types of prediction is illustrated in Figure~\ref{fig:icl_type}.
% For tasks where the answer could not be represented by a single word in the vocabulary of the language model due to the tokenization process, an alternative is to compare the perplexity of all candidate sentences, where each sentence $S_j$ is the text concatenation of demonstration examples, input query and the candidate answer:
% Better calibration 
% \begin{align}
%     \hat y &= y_{\arg\min_{j} \text{PPL}_\mathcal{M} ( S_j)} \\ 
%     S_j &= \{ C, x, y_j \}
% \end{align}

% As recent explorations show that the vanilla probability could be sensitive to the order of demonstration examples, various calibrate methods have been introduced to stabilize the prediction.
% One representative framework is Calibrate, which eliminates the effect of biased prediction by introducing a normalization term estimated via a nonparametric method. Channel~(CITE) further shows a stronger regularization performance by computing the conditional probability of the input given the label.

% We summarize all these variants in Table~\ref{tab:icl_variants}.
% Word Prediction
% PPL prediction 
% 



% Benefits 

% on a large-scale text corpus with causal language modeling as the training objective

\section{Model Warmup}
\label{sec:warmup}

% \subsection{ICL Capability of LLMs}
Although LLMs have shown promising ICL capability, many studies also show that the ICL capability can be further improved through a continual training stage between pretraining and ICL inference, which we call model warmup for short.  %The  LLMs manifested the ICL ability. It is first shown by GPT-3~\cite{gpt3}, the 175 billion parameters autoregressive language model. Immediately following GPT-3, a growing number of papers have proposed LLMs with ICL capabilities, e.g., GLaM~\cite{glam}, OPT~\cite{opt}, and AlexaTM 20B~\cite{alexatm}.
Warmup is an optional procedure for ICL, which adjusts LLMs before ICL inference, including modifying the parameters of the LLMs or adding additional parameters. Unlike finetuning, warmup does not aim to train the LLM for specific tasks but enhances the overall ICL capability of the model. 

% \textbf{There are many training variables that affect the ICL capacity of the original LLMs.} 
% In terms of the model scale,~\citet{gpt3} find that the ICL ability  grows when the parameters in LLMs increase from 0.1 billion to 175 billion.
% In terms of the pertaining corpus, ~\citet{corpusscale} study how the source and size of the pretraining corpus influence the ICL ability of LLMs. They discover that, rather than the size of the pertaining corpus, the source and data combination is critical for the emergence of ICL abilities for LLMs.

% \subsection{Parameters Updating}
\subsection{Supervised In-context Training}
\label{sec:s_tuning}
To enhance ICL capability, researchers proposed a series of supervised in-context finetuning strategies by constructing in-context training data and multitask training.
% Although experiments validate the ICL ability of the original LLMs, the performance of ICL remains unsatisfactory and high-variance compared to finetuning, especially when the task is very XXX[cite]. 
Since the pretraining objectives are not optimized for in-context learning~\citep{selfsupericl}, \citet{metaicl} proposed a method MetaICL to eliminate the gap between pretraining and downstream ICL usage.
The pretrained LLM  is continually trained on a broad range of tasks with demonstration examples, which boosts its few-shot abilities.
To further encourage the model to learn input-label mappings from the context, \citet{symboltuning} propose symbol tuning. This approach fine-tunes language models on in-context input-label pairs, substituting natural language labels (e.g., "positive/negative sentiment") with arbitrary symbols (e.g., "foo/bar"). As a result, symbol tuning demonstrates an enhanced capacity to utilize in-context information for overriding prior semantic knowledge.
% , e.g., 
% %model ability on making predictions condition on the examples in the demonstrations. 
% MetaICL obtains performance comparable to supervised finetuning on $52$ unique datasets.

% \subsection{Instruction Tuning}
% \label{sec:instruction_tuning}
% Besides, there is a research direction focusing on supervised instruction tuning. 
Besides, recent work indicates the potential value of instructions~\cite{mishra2021cross} and there is a research direction focusing on supervised instruction tuning. Instruction tuning enhances the ICL ability of LLMs through training on task instructions. %, mainly leveraging the intuition that NLP tasks can be described via natural language instructions
% , and \citet{optiml} further include varying numbers of demonstration examples.
Tuning the 137B LaMDA-PT~\cite{lamda} on over 60 NLP datasets verbalized via natural language instruction templates, FLAN~\cite{flan} improves both the zero-shot and the few-shot ICL performance.
Compared to MetaICL, which constructs several demonstration examples for each task, instruction tuning mainly considers an explanation of the task and is more easier to scale up. \citet{chung} and \citet{natural} proposed to scale up instruction tuning with more than 1000+ task instructions. %, and ~\citet{chung} found that the majority of the instruction-tuning improvement comes from using up to 282 tasks.

\subsection{Self-supervised In-context Training}
\label{sec:ss_tuning}
Leveraging raw corpora for warmup, \citet{selfsupericl} proposed constructing self-supervised training data aligned with ICL formats in downstream tasks. They transformed raw text into input-output pairs, exploring four self-supervised objectives, including masked token prediction and classification tasks. Alternatively, PICL~\cite{picl} also utilizes raw corpora but employs a simple language modeling objective, promoting task inference and execution based on context while preserving pre-trained models' task generalization. Consequently, PICL outperforms \citet{selfsupericl}'s method in effectiveness and task generalizability.

% instructino-tuning does not have demonstration examples
% To boost the zero-shot ICL ability of LLMs, ~\citet{flan} introduce instruction tuning. Instruction tuning refers to train pretrained LLMs on a mixture of tasks described as instructions. The instruction-tuned model, FLAN, substantially manifests stronger zero-shot abilities on unseen tasks.
 
% \subsection{Additional Parameters Adding}
% \subsection{Contextual Bias Calibrating}
% \label{sec:additional}
% Considering that retraining or updating the LLMs is always costly, researchers focus on tuning additional parameters for more accurate and unbiased ICL.
% As ICL on LLMs may be biased or prompt-sensitive
% ~\cite{calibrate,chen2022sensitivity}  
% \citet{calibrate} propose contextual calibration to adjust the model predictions, which fit the calibration parameters in a linear layer so that the prediction for the input to be uniform across answers. Contextual calibration provides a simple but effective solution for mitigating the potential bias of ICL without additional training data.

% Compared to the parameters updating approaches, methods which adding additional parameters mainly focuses on minor model prediction calibrating while retraining the original model performance, especially for debiasing and sensitivity reducing.

\textbf{$\Diamond$ Takeaway}: \textit{
% Warmup is optional for ICL as many pretrained LLMs have manifested the ICL ability. 
(1) Supervised training and self-supervised training both propose to train the LLMs before ICL inference. The key idea is to bridge the gap between pretraining and downstream ICL formats by introducing objectives close to in-context learning. Compared to in-context finetuning involving demonstration, instruction finetuning without a few examples as demonstration is simpler and more popular.  % Although self-supervised training uses raw corpus, the task variety is relatively limited. 
(2) To some extent, these methods all improve the ICL capability by updating the model parameters, which implies that the ICL capability of the original LLMs has great potential for improvement. Therefore, although ICL does not strictly require model warmup, we recommend adding a warmup stage before ICL inference.
% and a reasonable warmup can also reduce the difficulty of the later prompt designing and the sensitivity of ICL inference. 
(3) 
% Whether through supervised in-context training or self-supervised in-context training, the improvement of warmup will reach a plateau when increasingly scaling up the training tasks or training corpus. This indicates that warmup only adapts the LLMs to learn from the context.
The performance advancement made by warmup encounters a plateau when increasingly scaling up the training data. This phenomenon appears both in supervised in-context training and self-supervised in-context training, indicating that LLMs only need a small amount of data to adapt to learn from the context during warmup.
% But the ICL capability can be further improved through warmup. 
% To enhance the ICL capability, 
% Training on ICL format tasks or instructions enhance the ICL capability, and adding additional parameters works for minor calibration.
%To retain the ICL performance and 
%, adding additional parameters will be a better solution.
}
% \leiModify{As a takeway, this could be more precise, straight to the point - Lei ?}


\section{Demonstration Designing}
\label{sec:prompt_tuning}
\label{sec:demo}
% \paragraph{Pre-Defined Metrics}
% \citet{chen2022sensitivity,lu2022order} define their new metrics and find them correlated to the ICL performance. 
% \citet{chen2022sensitivity} define a metric for ICL called prediction sensitivity, which measures the changes of the model output in response to small input perturbations. 
% They find there is an obvious negative correlation between prediction sensitivity and ICL accuracy, and motivated by it, propose a selective prediction method based on the prediction sensitivity. 
% \citet{lu2022order} analyze the sample order sensitivity for ICL in depth. 
% They first prove that the order sensitivity always exists for different model sizes and training samples, and a performant order is not transferable across models. 
% Then, they define the global and local entropy metrics and find a positive correlation between the entropy and the ICL performance. 
% With the entropy metrics, they can estimate a good sample ordering for ICL by choosing one that has higher local or global entropy. 


\begin{table*}[t!]
    \centering
    \setlength{\tabcolsep}{2pt}
    \small
    \resizebox{\linewidth}{!}{\begin{tabular}{@{}lcccc@{}}
    \toprule 
      \bf Category & \bf Methods &  \bf Demonstration Acquisition & \bf LLMs & \bf Main Tasks \\
    \midrule 
       \multirow{3}{*}{Demonstration Selection}  & KATE~\citep{liu2022close} & Human design & GPT-3& SST, table-to-text\\
       & SG-ICL~\cite{kim2022self} & LM generated & GPT-J & SST, NLI\\
       & EPR~\citep{rubin2022learning}  & Human design & GPT-\{J, 3\}/CodeX & Semantic parsing\\\midrule
        Demonstration Ordering & GlobalE \& LocalE~\citep{lu2022order} &  Human design & GPT-\{2, 3\} & Text classification\\ 
       \midrule%\midrule
       % Calibrate~\citep{calibrate} & $\frac{\mathcal{M} ( y_j \mid C, x)}{\mathcal{M} ( y_j \mid \text{NULL}))}$ & \\ 
       Instruction Formatting & Self Instruct~\citep{wang2022self} & LM generated & GPT-3/InstructGPT & SuperNaturalInstruction\\\midrule
       
       \multirow{3}{*}{Reasoning Steps Formatting}& CoT~\citep{cot} & Human design & GPT-3/CodeX & Reasoning tasks\\
       & AutoCoT~\citep{autocot} & LM generated & GPT-3/PaLM & Reasoning tasks\\
       &Self-Ask~\citep{selfask} & LM generated & GPT-3/InstructGPT & MultihopQA\\
       
    \bottomrule
    \end{tabular}}
    \caption{Summary of representative demonstration designing methods.}
    \label{tab:promptmethods}
\end{table*}
%In ICL, the inputs are fed to LLMs with demonstration examples to get outputs via language modeling. 
Many studies have shown that the performance of ICL strongly relies on the demonstration surface, including demonstration format, the order of demonstration examples, and so on~\citep{calibrate, lu2022order}. As demonstrations play a vital role in ICL, in this section, we survey demonstration designing strategies and classify them into two groups: demonstration organization and demonstration formatting, as shown in Table~\ref{tab:promptmethods}.
\subsection{Demonstration Organization}
\label{sec:organ}
Given a pool of training examples, demonstration organization focuses on how to select a subset of examples and the order of the selected examples. 

%Demonstrations show input-output mappings in ICL and the performance of ICL is sensitive to the combination and permutation of demonstrations. With limited demonstrations, demonstration organization strategies focus on how to \textbf{select} and \textbf{order} demonstrations.

\subsubsection{Demonstration Selection}
\label{sec:select}

% Suggestion from Xiaonna Li: 
% another taxnomy: into task-specific and instance specific selection categories.

Demonstrations selection aims to answer a fundamental question: Which examples are good examples for ICL? We classify related studies into two categories, including unsupervised methods based on pre-defined metrics and supervised methods.

\paragraph{Unsupervised Method} \citet{liu2022close} showed that selecting the closest neighbors as the in-context examples is a good solution. The distance metrics are pre-defined L2 distance or cosine-similarity distance based on sentence embeddings. %It outperformed significantly with farthest neighbors as examples. 
They proposed KATE, a $k$NN-based unsupervised retriever for selecting in-context examples. In addition to distance metrics, mutual information is also a valuable selection metric~\citep{sorensen2022information}. 
Similarly, $k$-NN cross-lingual demonstrations can be retrieved for multi-lingual ICL \citep{tanwar2023multilingual} to strengthen source-target language alignment.
%Another unsupervised example selection strategy is based on mutual information. 
%Without labeled examples and accessing to the LLMs, an example is selected when it has high mutual information. 
The advantage of mutual information is that it does not require labeled examples and specific LLMs. 
In addition, \citet{gonen2022demystifying} attempted to choose prompts with low perplexity. 
% \citet{Wu2022SelfadaptiveIL} selected the best subset permutation of $k$NN examples based on the code-length for data transmission to compress label $y$ given $x$ and $C$. This self-adaptive ranking method considered both selection and ordering. 
\citet{levy2022diverse} consider the diversity of demonstrations to improve compositional generalization. They select diverse demonstrations to cover different kinds of training demonstrations. 
%They also summarized the connection between code length and other metrics, including cross-entropy and mutual information.
Different from these studies selecting examples from human-labeled data, \citet{kim2022self} proposed to generate demonstrations from LLM itself. 

Some other methods utilized the output scores of LMs $P(y|C, x)$ as unsupervised metrics to select demonstrations.
\citet{Wu2022SelfadaptiveIL} selected the best subset permutation of $k$NN examples based on the code-length for data transmission to compress label $y$ given $x$ and $C$.
\citet{nguyen2023influence} measured the influence of a demonstration $x_i$ by calculating the difference between the average performance of the demonstration subsets $\{C|x_i\in C\}$ and $\{C|x_i\notin C\}$. Furthermore, \citet{li2023supporting} used infoscore, i.e., the average of $P(y|x_i,y_i,x) - P(y|x)$ for all $(x,y)$ pairs in a validation set with a diversity regularization.


\paragraph{Supervised Method} \citet{rubin2022learning} proposed a two-stage retrieval method to select demonstrations. For a specific input, it first built an unsupervised retriever (e.g., BM25) to recall similar examples as candidates and then built a supervised retriever EPR to select demonstrations from candidates. A scoring LM is used to evaluate the concatenation of each candidate example and the input. Candidates with high scores are labeled as positive examples, and candidates with low scores are hard negative examples. 
\citet{udr} further enhanced the EPR by adopting a unified demonstration retriever to unify the demonstration selection across different tasks.
\citet{ye2023compositional} retrieved the entire set of demonstrations instead of individual demonstrations to model inter-relationships between demonstrations. They trained a DPP retriever to align with LM output scores by contrastive learning and obtained the optimal demonstration set with maximum a posteriori at inference.

Based on prompt tuning, \citet{topic} view LLMs as topic models that can infer concepts $\theta$ from few demonstrations and generate tokens based on concept variables $\theta$. They use task-related concept tokens to represent latent concepts. Concept tokens are learned to maximize $P(y|x,\theta)$. They select demonstrations that are most likely to infer the concept variable based on $P(\theta|x,y)$. 
Besides, reinforcement learning was introduced by \citet{zhang2022active} for example selection. They formulated demonstration selection as a Markov decision process~\cite{bellman1957markovian} and selected demonstrations via Q-learning. The action is choosing an example, and the reward is defined as the accuracy of a labeled validation set. 


% \zc{}
%According to labeled candidates, EPR can be trained via in-batch contrastive learning.

% \paragraph{Reinforcement Learning Method} \citet{zhang2022active} formulated demonstration selection as a markov decision process and selected demonstrations via Q-learning. 

% Some researches (cite some works) focus on how to better organize the selected examples, including better ordering of selected examples and adding intermediate reasoning steps in demonstrations.
\subsubsection{Demonstration Ordering}
\label{sec:order}
Ordering the selected demonstration examples is also an important aspect of demonstration organization. 
\citet{lu2022order} have proven that order sensitivity is a common problem and always exists for various models. 
To handle this problem, previous studies have proposed several training-free methods to sort examples in the demonstration. \citet{liu2022close} sorted examples decently by their distances to the input, so the rightmost demonstration is the closest example.
% \citet{calibrate, lu2022order} show that the performance of ICL is sensitive to the permutation of demonstrations and the performance can vary from random guess to closely SOTA. 
%analyzed the sample order sensitivity for ICL in depth. They proved that order sensitivity is a common problem and always exists for various models. %, and a performant order is not transferable across models. 
\citet{lu2022order}  defined the global and local entropy metrics. They found a positive correlation between the entropy metric and the ICL performance. They directly used the entropy metric to select the best ordering of examples. %can estimate a good sample ordering for ICL by choosing one that has higher local or global entropy. 



% \subsubsection{Corpus- and Instance-Level Strategies}
% Overall, prompt formulation strategies can be divided into corpus-level and instance-level strategies.

\subsection{Demonstration Formatting}
\label{sec:format}
% \subsubsection{Instruction Formatting}
A common way to format demonstrations is concatenating examples $(x_1, y_1), \ldots, (x_k, y_k)$ with a template $\mathcal{T}$ directly. However, in some tasks that need complex reasoning (e.g., math word problems, commonsense reasoning), it is not easy to learn the mapping from $x_i$ to $y_i$ with only $k$ demonstrations. Although template engineering has been studied in prompting~\citep{liu2021pre},  some researchers aim to design a better format of demonstrations for ICL by describing tasks with the instruction $I$ (\S\ref{sec:formatting_instruction}) and adding intermediate reasoning steps between $x_i$ and $y_i$ (\S\ref{sec:formatting_intermediate}).


\subsubsection{Instruction Formatting}
\label{sec:formatting_instruction}
Except for the well-designed demonstration examples, good instructions which describe the task precisely are also helpful to the inference performance.
However, unlike the demonstration examples, which are common in traditional datasets, the task instructions depend heavily on human-written sentences.
\citet{induct} found that given several demonstration examples, LLMs can generate the task instruction.
According to the generation ability of LLMs, ~\citet{zhou2022large} proposed Automatic Prompt Engineer for automatic instruction generation and selection.
% To further save the human labor,
% ~\citet{} generate instruction by prompting the model to rephrase existing instruction.
To further improve the quality of the automatically generated instructions, ~\citet{wang2022self} proposed to use LLMs to bootstrap off its own generations.
Existing work has achieved good results in automatically generating instructions, which provided opportunities for future research on combining human feedback with automatic instruction generation.





\subsubsection{Reasoning Steps Formatting}
\label{sec:formatting_intermediate}
% \subsubsection{Reasoning}



\citet{cot} added intermediate reasoning steps between inputs and outputs to construct demonstrations, which are called chain-of-thoughts (CoT). With CoT, LLMs predict the reasoning steps and the final answer. CoT prompting can learn complex reasoning by decomposing input-output mappings into many intermediate steps. There are many pieces of research on CoT prompting strategies ~\citep{qiao2022reasoning} including prompt designing and process optimization. In this paper, we mainly focus on CoT designing strategies.

Similar to demonstration selection, CoT designing also considers CoT selection. Different from \citet{cot} manually writing CoTs, AutoCoT \citep{autocot} used LLMs with \textit{Let's think step by step} to generate CoTs. In addition, \citet{fu2022complexitycot} proposed a complexity-based demonstration selection method. They selected demonstrations with more reasoning steps for CoT prompting. 

As input-output mappings are decomposed into step-by-step reasoning, some researchers apply multi-stage ICL for CoT prompting and design CoT demonstrations for each step. Multi-stage ICL queries LLMs with different demonstrations in each reasoning step. Self-Ask \citep{selfask} allows LLMs to generate follow-up questions for the input and ask themselves these questions. Then the questions and intermediate answers will be added to CoTs. iCAP \citep{wang2022iteratively} proposes a context-aware prompter that can dynamically adjust contexts for each reasoning step. Least-to-Most Prompting \citep{least} is a two-stage ICL including question reduction and subquestion solution. The first stage decomposes a complex question into subquestions; in the second stage, LLMs answer subquestions sequentially, and previously answered questions and generated answers will be added into the context.

\citet{xu2023small} fine-tuned small LMs on specific task as plug-ins to generate pseudo reasoning steps. Given an input-output pair $(x_i,y_i)$, SuperICL regarded the prediction $y_i'$ and confidence $c_i$ of small LMs for the input $x_i$ as reasoning steps by concatenating $(x_i, y_i', c_i, y_i)$. 
% They further requested LMs to generate over-ride explanations if the final prediction $y$ is in-consistent with the prediction of small LMs $y'$.

% add \citep{mot} 


% \subsubsection{XXXX}
% \textit{\# Takeaway: Demonstration Designing strategies include Demonstration Organization and Demonstration Formatting. Demonstration Organization focuses on how to select and order demonstrations. Demonstration Formatting focuses on a better format of demonstrations.}
\textbf{$\Diamond$ Takeaway}: \textit{ (1) Demonstration selection strategies improve the ICL performance, but most of them are instance level. Since ICL is mainly evaluated under few-shot settings, the corpus-level selection strategy is more important yet under-explored. (2) The output score or probability distribution of LLMs plays an important role in instance selecting. (3) For $k$ demonstrations, the size of search space of permutations is $k!$. How to find the best orders efficiently or how to approximate the optimal ranking better is also a challenging question. (4) Adding chain-of-thoughts can effectively decompose complex reasoning tasks into intermediate reasoning steps. During inference, multi-stage demonstration designing strategies are applied to generate CoTs better. How to improve the CoT prompting ability of LLMs is also worth exploring (5) In addition to human-written demonstrations, the generative nature of LLMs can be utilized in demonstration designing. LLMs can generate instructions, demonstrations, probing sets, chain-of-thoughts, and so on. By using LLM-generated demonstrations, ICL can largely get rid of human efforts on writing templates. }

\section{Scoring Function}
\label{sec:scoring}
% \zy{shall we briefly discuss the pros and cons of different variants? e.g., direct limit the choices of templates, ppl is less efficient compared to direct, etc. probably add two columns in the table?}

\begin{table}[t!]
    \centering
    \resizebox{\linewidth}{!}{
    \begin{tabular}{@{}l|c|ccc@{}}
    \toprule 
      \bf Scoring Function & \bf Target & \bf Efficiency & \bf Task Coverage & \bf Stability \\
    \midrule 
       Direct  & $ \mathcal{M} ( y_j \mid C, x) $ & +++  &  + & + \\ 
       PPL &  $\text{PPL} (S_j) $&  + & +++ &  +\\ 
       % Calibrate~\citep{calibrate} & $\frac{\mathcal{M} ( y_j \mid C, x)}{\mathcal{M} ( y_j \mid \text{NULL}))}$ & \\ 
       Channel& $\mathcal{M} (x \mid C, y_j)$ & + & +  & ++ \\ 
    \bottomrule
    \end{tabular}}
    \caption{Summary of different scoring functions. }
    \label{tab:score_func}
\end{table}
% \qingxiu{citating}
The scoring function decides how we can transform the predictions of a language model into an estimation of the likelihood of a specific answer.
A direct estimation method~(Direct) adopts the conditional probability of candidate answers that can be represented by tokens in the vocabulary of language models~\citep{gpt3}. The answer with a higher probability is selected as the final answer. However, this method poses some restrictions on the template design, e.g., the answer tokens should be placed at the end of input sequences.
% \leiModify{Highlight the difference between task type and prediction type.}
Perplexity~(PPL) is another commonly-used metric, which computes the sentence perplexity of the whole input sequence $S_j = \{ C, s(x, y_j, I)\}$ consists of the tokens of demonstration examples $C$, input query $x$ and candidate label $y_j$. 
As PPL evaluates the probability of the whole sentence, it removes the limitations of token positions but requires extra computation time. 
Note that in generation tasks such as machine translation, ICL predicts the answer by decoding tokens with the highest sentence probability combined with diversity-promoting strategies such as beam search or Top-$p$ and Top-$k$~\citep{topp_sample} sampling algorithms.


Different from previous methods, which estimate the probability of the label given the input context, 
\citet{min2022noisy} proposed to utilize channel models~(Channel) to compute the conditional probability in a reversed direction, i.e., estimating the likelihood of input 
query given the label. In this way, language models are required to generate every token in the input, which could boost the performance under imbalanced training data regimes.
We summarize all three scoring functions in Table~\ref{tab:score_func}.
As ICL is sensitive to the demonstration~(see \S\ref{sec:prompt_tuning} for more details), normalizing the obtained score by subtracting a model-dependent prior with empty inputs is also effective for improving the stability and overall performance~\citep{calibrate}.

Another direction is to incorporate information beyond  the context length constrain to calibrate the score.
% demonstration examples are separately encoded with well-designed
% position embeddings, and then they are jointly attended by the test example using
% a rescaled attention mechanism.
Structured Prompting~\citep{hao2022structured} proposes to encode demonstration examples separately with special positional embeddings, which then are provided to the test examples with a rescaled attention mechanism.
$k$NN Prompting~\citep{knnPrompting} first queries LLMs with
training data for distributed representations, then predicts test instances by simply referring to nearest neighbors with closing representations with stored anchor representations.

% \textbf{$\Diamond$ Takeaway}: \textit{Selecting a proper scoring function according to the tasks in hand is important, and it has a great influence on the stability of the results.}
\textbf{$\Diamond$ Takeaway}: \textit{(1) We conclude the characteristics of three widely-used scoring functions in Table~\ref{tab:score_func}. Although directly adopting the conditional probability of candidate answers is efficient, this method still poses some restrictions on the template design. Perplexity is also a simple and widely scoring function. This method has universal applications, including both classification tasks and generation tasks. However, both methods are still sensitive to demonstration surface, while Channel is a remedy that especially works under imbalanced data regimes. (2) Existing scoring functions all compute a score straightforwardly from the conditional probability of LLMs. There is limited research on calibrating the bias or mitigating the sensitivity via scoring strategies. For instance, some studies add additional calibration parameters to adjust the model predictions~\cite{calibrate}. 
}



\section{Analysis}
\label{sec:analysis}
% \begin{table*}[t]
%     \centering
%     \begin{tabularx}{0.98\textwidth}{l l X}
%     \toprule
%     \textbf{Stage} & \textbf{Factor} & \textbf{Correlation to ICL} \\
%     \midrule
%     \multirow{6}{*}{\makecell[l]{Pretraining}} & \makecell[l]{Pretraining corpus domain \\ \citep{shin2022corpora}} & \makecell[X]{Compared to the pretraining corpus size, the corpus domain is more influential} \\
%      & \makecell[l]{Pretraining corpus combination \\ \citep{shin2022corpora}} & \makecell[X]{Combining two poor-performing pretraining corpora may introduce emergent ICL ability} \\
%      & \makecell[l]{Model scale \\ \citep{wei2022emergent,gpt3}} & \makecell[X]{Models with more pretraining steps and parameters are more likely to have emergent ability} \\
%     \midrule
%     \multirow{7}{*}{\makecell[l]{Inference}} & \makecell[l]{Label space exposure \\ \citep{min2022rethinking}} & \makecell[X]{Label space exposure has big impact on non-channel ICL models} \\
%      & \makecell[l]{Input distribution \\ \citep{min2022rethinking}} & \makecell[X]{In-distribution demonstrations substantially contribute to ICL performance} \\
%     & \makecell[l]{Format of input-label pairing \\ \citep{min2022rethinking}} & \makecell[X]{The format of input-label pairing contributes much to the correct prediction of ICL} \\
%     & \makecell[l]{Demonstration-query similarity \\ \citep{liu2022close}} & \makecell[X]{Using demonstrations that are more similar to the query will bring better performance} \\
%     \bottomrule
%     \end{tabularx}
%     \caption{Summary of influence factors that have relatively strong correlation to ICL.}
%     \label{tab:factor}
% \end{table*}

\begin{table}[t]
    \centering
    \footnotesize
    \begin{tabular}{cc}
    \toprule
    \textbf{Stage} & \textbf{Factor} \\
    \midrule
    \multirow{7}{*}{\makecell[l]{Pretraining}} & \makecell[l]{Pretraining corpus domain \\ \citep{shin2022corpora}} \\
     & \makecell[l]{Pretraining corpus combination \\ \citep{shin2022corpora}} \\
     & \makecell[l]{Number of model parameters \\ \citep{wei2022emergent,gpt3}} \\
     & \makecell[l]{Number of pretraining steps \\ \citep{wei2022emergent}} \\
    \midrule
    \multirow{15}{*}{\makecell[l]{Inference}} & \makecell[l]{Label space exposure \\ \citep{min2022rethinking}} \\
     & \makecell[l]{Demonstration input distribution \\ \citep{min2022rethinking}} \\
    & \makecell[l]{Format of input-label pairing \\ \citep{min2022rethinking,compositional_generalization}} \\
    & \makecell[l]{Demonstration input-label mapping \\ \citep{min2022rethinking,ground_truth} \\ 
    \citep{llm_icl_differently}} \\
    & \makecell[l]{Demonstration sample ordering \\ \citep{lu2022order}} \\
    & \makecell[l]{Demonstration-query similarity \\ \citep{liu2022close}} \\
    & \makecell[l]{Demonstration diversity \\ \citep{compositional_generalization}} \\
    & \makecell[l]{Demonstration complexity \\ \citep{compositional_generalization}} \\
    \bottomrule
    \end{tabular}
    \caption{Summary of factors that have a relatively strong correlation to ICL performance.}
    \label{tab:factor}
\end{table}

To understand ICL, many analytical studies attempt to investigate what factors may influence the performance and aim to figure out why ICL works. We summarize the factors that have a relatively strong correlation to ICL performance in Table~\ref{tab:factor} for easy reference.  

\subsection{What Influences ICL Performance}

% factors - pretraining
\paragraph{Pre-training Stage}
We first introduce influence factors in the LLM pretraining stage. 
\citet{shin2022corpora} investigated the influence of the pretraining corpora. 
They found that the domain source is more important than the corpus size. Putting multiple corpora together may give rise to emergent ICL ability, pretraining on corpora related to the downstream tasks does not always improve the ICL performance, and models with lower perplexity do not always perform better in the ICL scenarios.
\citet{wei2022emergent} investigated the emergent abilities of many large-scale models on multiple tasks. They suggested that a pretrained model suddenly acquires some emergent ICL abilities when it achieves a large scale of pretraining steps or model parameters. 
\citet{gpt3} also showed that the ICL ability grows as the parameters of LLMs increase from 0.1 billion to 175 billion.

% factors - inference
\paragraph{Inference Stage}
In the inference stage, the properties of the demonstration samples also influence the ICL performance. 
\citet{min2022rethinking} investigated that the influence of demonstration samples comes from four aspects: the input-label pairing format, the label space, the input distribution, and the input-label mapping. 
They prove that all of the input-label pairing formats, the exposure of label space, and the input distribution contribute substantially to the ICL performance. Counter-intuitively, the input-label mapping matters little to ICL. 
In terms of the effect of input-label mapping, \citet{ground_truth} drew an opposite conclusion that correct input-label mapping does impact the ICL performance, depending on specific experimental settings.
\citet{llm_icl_differently} further found that when a model is large enough, it will show an emergent ability to learn input-label mappings, even if the labels are flipped or semantically-unrelated. 
From the compositional generalization perspective, \citet{compositional_generalization} validated that ICL demonstrations should be diverse, simple, and similar to the test example in terms of the structure. 
\citet{lu2022order} indicated that the demonstration sample order is also an important factor.
In addition, \citet{liu2022close} found that the demonstration samples that have closer embeddings to the query samples usually bring better performance than those with farther embeddings. 



\subsection{Understanding Why ICL Works}

\paragraph{Distribution of Training Data} 
Concentrating on the pretraining data, \citet{distribution} showed that the ICL ability is driven by data distributional properties. 
They found that the ICL ability emerges when the training data have examples appearing in clusters and have enough rare classes. 
\citet{bayesian} explained ICL as implicit Bayesian inference and constructed a synthetic dataset to prove that the ICL ability emerges when the pretraining distribution follows a mixture of hidden Markov models. 

\paragraph{Learning Mechanism} 
By learning linear functions, \citet{garg2022linear} proved that Transformers could encode effective learning algorithms to learn unseen linear functions according to demonstration samples. 
They also found that the learning algorithm encoded in an ICL model can achieve a comparable error to that from a least squares estimator. 
\citet{trm_as_alg} abstracted ICL as an algorithm learning problem and showed that Transformers can implement a proper function class through implicit empirical risk minimization for the demonstrations. 
\citet{tr_and_tl} decoupled the ICL ability into task recognition ability and task learning ability, and further showed how they utilize demonstrations. 
From an information-theoretic perspective, \citet{implicit_structure_induction} showed an error bound for ICL under linguistically motivated assumptions to explain how next-token prediction can bring about the ICL ability. 
\citet{inductive_bias} found that large language models exhibit prior feature biases and showed a way to use intervention to avoid unintended features in ICL. 

Another series of work attempted to build connections between ICL and gradient descent. 
Taking linear regression as a starting point, \citet{akyurek2022algorithm} found that Transformer-based in-context learners can implement standard finetuning algorithms implicitly, and \citet{icl_gd} showed that linear attention-only Transformers with hand-constructed parameters and models learned by gradient descent are highly related. 
Based on softmax regression, \citet{icl_weight_shifting} found that self-attention-only Transformers showed similarity with models learned by gradient-descent. 
\citet{dai2022iclft} figured out a dual form between Transformer attention and gradient descent and further proposed to understand ICL as implicit finetuning. 
Further, they compared GPT-based ICL and explicit finetuning on real tasks and found that ICL indeed behaves similarly to finetuning from multiple perspectives. 

\paragraph{Functional Components} 
Focusing on specific functional modules, \citet{olsson2022induction} found that there exist some induction heads in Transformers that copy previous patterns to complete the next token. 
Further, they expanded the function of induction heads to more abstract pattern matching and completion, which may implement ICL. 
% They provided several pieces of evidence on models with different sizes, which support their viewpoint that induction heads constitute the mechanism of ICL. 
\citet{label_anchor} focused on the information flow in Transformers and found that during the ICL process, demonstration label words serves as anchors, which aggregates and distributes key information for the final prediction. 

\textbf{$\Diamond$ Takeaway}: \textit{
(1) Knowing and considering how ICL works can help us improve the ICL performance, and the factors that strongly correlate to ICL performance are listed in Table~\ref{tab:factor}.
(2) 
Although some analytical studies have taken a preliminary step to explain ICL, most of them are limited to simple tasks and small models. 
Extending analysis on extensive tasks and large models may be the next step to be considered. 
In addition, among existing work, explaining ICL with gradient descent seems to be a reasonable, general, and promising direction for future research. 
If we build clear connections between ICL and gradient-descent-based learning, we can borrow ideas from the history of traditional deep learning to improve ICL. 
}

\section{Evaluation and Resources}
\label{sec:evaluation}

\subsection{Traditional Tasks}
As a general learning paradigm, ICL can be examined on various traditional datasets and benchmarks, e.g., SuperGLUE~\cite{superglue}, SQuAD~\cite{squad}. 
Implementing ICL with 32 randomly sampled examples on SuperGLUE, ~\citet{gpt3} found that GPT-3 can achieve results comparable to state-of-the-art (SOTA) finetuning performance on COPA and ReCoRD, but still falls behind finetuning on most NLU tasks.
~\citet{hao2022structured} showed the potential of scaling up the number of demonstration examples. However, the improvement brought by scaling is very limited. At present, compared to finetuning, there still remains some room for ICL to reach on traditional NLP tasks.

%, and the number of demonstration examples is still not as unlimited as finetuning

%And XXX and XXX has shown XXXX, XXX are studying the XXX. % gpt3 paper examined on superglue and squad
\begin{table}[t!]
    \small
    \setlength{\tabcolsep}{4pt}
    \centering
    \resizebox{\linewidth}{!}{\begin{tabular}{llr}
    \toprule
     \bf Benchmark  & \bf Tasks  & \bf \#Tasks  \\
     \midrule
    \makecell[l]{BIG-Bench \\ \cite{beyond}}   & {\makecell[l]{Mixed tasks}} & 204 \\
    \makecell[l]{BBH \\ \cite{suzgun2022challenging}}   & {\makecell[l]{Unsolved problems}} & 23  \\
    \makecell[l]{PRONTOQA \\ \cite{heuristic}}  & {\makecell[l]{Question answering}} &  1 \\
    \makecell[l]{MGSM \\ \cite{shi2022language}} & {\makecell[l]{Math problems}} & 1  \\
    \makecell[l]{LLMAS \\ \cite{planbench}} & {\makecell[l]{Plan and reasoning tasks}}  & 8\\
    \makecell[l]{OPT-IML Bench \\ \cite{optiml}} & {\makecell[l]{Mixed tasks}} & 2000 \\
    \bottomrule
    \end{tabular}}
    \caption{New challenging evaluation benchmarks for ICL. For short, we use LLMAS to represent LLM Assessment Suite~\cite{planbench}.} 
    \label{tab:dataset}
\end{table}
% \begin{table*}[]
\small
    \centering
    \begin{tabular}{lccccccccc}
    \toprule
     ~ & \bf SuperGLUE & \bf BoolQ & \bf CB  & \bf COPA & \bf RTE  & \bf WiC  & \bf WSC  & \bf MultiRC & \bf  ReCoRD \\
 ~& Avg. & Acc. & Acc.,F1 & Acc. & Acc. &
 Acc.  & Acc. & Acc.,F1a & Acc.,F1\\
     \midrule
     Human Baseline & 89.8  & 89.0  & 95.8/98.9  & 100.0  & 81.8/51.9  & 91.7/91.3  & 93.6  & 80.0 & 100.0   \\
     Finetuned SOTA \\
     Vanilla GPT3  &   &  &   & \\
    \makecell[l]{Chain-of-thoughts \\ ~\cite{beyond}}   &   & &   & \\
    \makecell[l]{Least-to-most\\ ~\cite{suzgun2022challenging}}   & &  &   & \\
    \bottomrule
    \end{tabular}
    \caption{Benchmarking exsiting prompting strategies on the GLUE tasks with 8 demonstration examples on each dataset.}
    \label{tab:evaluation}
\end{table*}

% To make a fair comparison of different prompting strategies for ICL on traditional natural language understanding tasks, we implement XX prompting strategies for ICL with a same number of demonstration examples and a same backbone model. As it shown in Tab.~\ref{tab:evaluation}, XXXX.

\subsection{New Challenging Tasks}
% Except for the traditional extrinsic evaluation
%The powerful few-shot capability of ICL also gives rise to a new paradigm in datasets and evaluation.
In the era of large language models with in-context learning capabilities, researchers are more interested in evaluating the intrinsic capabilities of large language models without downstream task finetuning~\cite{foundation}.

To explore the capability limitations of LLM on various tasks, 
% more than 400 researchers from various institutions 
~\citet{beyond} proposed the BIG-Bench~\cite{beyond}, a large benchmark covering  
% 204 tasks with good diversity. 
a large range of tasks, including linguistics, chemistry, biology, social behavior, and beyond. 
The best models have already outperformed the average reported human-rater results on 65\% of the BIG-Bench tasks through ICL~\cite{suzgun2022challenging}. To further explore tasks actually unsolvable by current language models, \citet{suzgun2022challenging} proposed a more challenging ICL benchmark, BIG-Bench Hard (BBH). BBH includes 23 unsolved tasks, constructed by selecting challenging tasks where the state-of-art model performances are far below the human performances. Besides, researchers are searching for inverse scaling tasks,\footnote{\url{https://github.com/inverse-scaling/prize}} that is, tasks where model performance reduces when scaling up the model size. Such tasks also highlight potential issues with the current paradigm of ICL.
To further probe the model generalization ability, ~\citet{optiml} proposed OPT-IML Bench, consisting of 2000 NLP tasks from 8 existing benchmarks, especially benchmark for ICL on held-out categories.

Specifically, a series of studies focus on exploring the reasoning ability of ICL.~\citet{heuristic} generated an example from a synthetic world model
represented in first-order logic and parsed the ICL generations into symbolic proofs for formal analysis. They found that LLMs can make correct individual deduction steps via ICL.
~\citet{shi2022language} constructed the MGSM benchmark to evaluate the chain-of-thought reasoning abilities of LLMs in multilingual settings, finding that LLMs manifest complex reasoning across multiple languages.
To further probe more sophisticated planning and reasoning abilities of LLMs, ~\citet{planbench} provided multiple test cases for evaluating various reasoning abilities on actions and change, where existing ICL methods on LLMs show poor performance.

\subsection{Open-source Tools}
Noticing that ICL methods are often implemented differently and evaluated using different LLMs and tasks, \citet{openicl} developed OpenICL, an open-source toolkit enabling flexible and unified ICL assessment. With its adaptable architecture, OpenICL facilitates the combination of distinct components and offers state-of-the-art retrieval and inference techniques to accelerate the integration of ICL into advanced research.


%In summary, the evaluation for ICL on LLMs is manifesting the following characteristics:
 % Traditional datasets lack challenge for ICL; 2) 
% \textit{Takeaway: Intrinsic evaluation without large training set is a new tendency. The diversity and challenges are particularly valued for new benchmarks.}
% 4) Evaluation under the ICL paradigm brings new challenges, more insensitive and trustworthy evaluation paradigms is a new direction (Section~\ref{challenge: evaluation}).
\textbf{$\Diamond$ Takeaway}: \textit{(1) Due to the restrictions of ICL on the number of demonstration examples, the traditional evaluation tasks must be adapted to few-shot settings; otherwise, the traditional benchmarks cannot evaluate the ICL capability of LLMs directly.
(2) As ICL is a new paradigm that is different from traditional learning paradigms in many aspects, the evaluation of ICL presents new challenges and opportunities.
Toward the challenges, the results of existing evaluation methods are unstable, especially sensitive to the demonstration examples and the instructions. \citet{chen2022relation} observed that existing evaluations by accuracy underestimate the sensitivity towards instruction perturbation of ICL.
It is still an open question to conduct consistent ICL evaluation and OpenICL\cite{openicl} represents a valuable initial attempt to address this challenge.
% Apart from that, traditional evaluation benchmarks such as SuperGLUE~\cite{superglue} are less challenging for LLMs with billions of parameters, and even the sophisticated designed new benchmark, BigBench~\cite{beyond} can be readily surpassed.
% How to construct a challenging evaluation with long-term vitality is crucial for ICL evaluation.
Toward the opportunities for evaluation, as ICL only requires a few instances for the demonstration, it lowers the cost of evaluation data construction. 
}

%Furthermore, the test set is still supposed to faithfully reflect the world distribution. 
%Instead of the i.i.d. assumption with the training set, the test set requires new constraints.



\section{In-Context Learning Beyond Text}
\label{sec:mm_icl}
% Multimodal Applications
% Due to the great success of ICL in the area of NLP, the idea also has motivated pilot studies toward investigating the ICL ability in visual and multimodal tasks.

The tremendous success of ICL in NLP  has inspired researchers to explore its potential in different modalities, including visual, vision+language and speech tasks as well. 

\subsection{Visual In-Context Learning}
\begin{figure}
    \centering
    \includegraphics[width=0.49\textwidth]{fig/mm-case.pdf}
    \caption{Image-only and textual augmented prompting for visual in-context learning.}
    \label{fig:mm_case}
\end{figure}
\citet{bar2022visual_icl} employ an image patch infilling task in grid-like images using masked auto-encoders (MAE) to train their model. 
At the inference stage, the model generates output images consistent with provided input-output examples for a novel input image, showcasing promising ICL capabilities for unseen tasks such as image segmentation. Painter~\citep{wang2023imagesPainter} extends this approach by incorporating multiple tasks to build a generalist model, achieving competitive performance compared to task-specific models. Building upon this, SegGPT~\citep{wang2023seggpt} integrates diverse segmentation tasks into a unified framework and investigates ensemble techniques from spatial and feature perspectives to enhance the quality of prompt examples.
% \citet{wang2023prompt_diffusion} further propose to incorporate a text prompt along with the original input-output image pair to guide a generative model to comprehensively produce the desired image. The resulting Prompt Diffusion model~\citep{wang2023prompt_diffusion} is the first diffusion-based model that exhibits in-context learning ability.
\citet{wang2023prompt_diffusion} propose to utilize an extra text prompt to guide a generative model in comprehensively producing the desired image. The resulting Prompt Diffusion model is the first diffusion-based model that exhibits ICL ability. 
Figure~\ref{fig:mm_case} illustrates the key difference between the image-only and textual prompt augmented in-context learning for visual in-context learning.
% \paragraph{Visual In-Context Learning}
% \citet{bar2022visual_icl} utilize an image patch infilling task in grid-like images to train a model based on masked auto-encoders~(MAE).
% During the inference stage, given input-output image examples and a new input image, the model could automatically produce the output image that is consistent with the given examples. This paradigm demonstrates promising in-context learning on unseen tasks such as image segmentation. Painter~\citep{wang2023imagesPainter} further incorporates more tasks to construct a generalist model, achieving competitive results
% compared to task-specific models.
% Subsequently, SegGPT~\citep{wang2023seggpt} 
% unifies various segmentation tasks into a generalist framework, and explores ensemble methods from spatial 
% and feature aspects for improving the quality of prompt examples.


Similar to ICL in NLP, the effectiveness of visual in-context learning is significantly influenced by the selection of in-context demonstration images~\citep{zhang2023visual_icl_analysis,sun2023exploring_visual_icl}. 
To address this, \citet{zhang2023visual_icl_analysis} investigate two approaches: (1) an unsupervised retriever that selects nearest samples using an off-the-shelf model, and (2) a supervised method training an additional retriever model to maximize ICL performance. The retrieved samples notably enhance performance, exhibiting semantic similarity to the query and closer contextual alignment regarding viewpoint, background, and appearance.  Except for the prompt retrieval, \citet{sun2023exploring_visual_icl} further explore a prompt fusion technique for improving the results.




\subsection{Multi-Modal In-Context Learning}
In the vision-language area, \citet{tsimpoukelli2021frozen} utilize a vision encoder to represent an image as a prefix embedding sequence that is aligned with a frozen language model after training on the paired image-caption dataset.
The resulting model, Frozen, is capable of performing multi-modal few-shot learning. 
Further, \citet{alayrac2022flamingo} introduce
Flamingo, which 
combines a vision encoder with LLMs and adopts LLMs as the general interface to perform in-context learning on many multi-modal tasks.
They show that training on large-scale multi-modal web corpora with arbitrarily interleaved text and images is key to endowing them with in-context few-shot learning capabilities. 
Kosmos-1~\citep{huang2023kosmos} is another multi-modal LLMs and demonstrates promising zero-shot, few-shot,
and even multimodal chain-of-thought prompting abilities.
~\citet{hao2022language} present METALM, a general-purpose interface to models across tasks and modalities. With a semi-causal language modeling objective, METALM is pretrained and exhibits strong ICL performance across various vision-language tasks. 




It is natural to further enhance the ICL ability with instruction tuning, and the idea is also explored in the multi-modal scenarios as well.
Recent explorations first generate instruction tuning datasets transforming existing vision-language task dataset~\citep{xu2022multiinstruct,li2023otter} or with power LLMs such as GPT-4~\citep{liu2023llava,zhu2023minigpt4} , and connect LLMs with powerful vision foundational models such as BLIP-2~\citep{li2023blip2} on these multi-modal datasets~\citep{zhu2023minigpt4,dai2023instructblip}.



% \begin{table}[t!]
%     \centering
%     \begin{tabular}{l|c c}
%     \toprule
%       Modality   & Task &  Work   \\
%       \midrule
%      Vision  &  image generation & \\ 
%      Speech  & speech synthesis & \\ 
%      Vision + Language & image-to-text& \\ 
%     \bottomrule
%     \end{tabular}
%     \caption{Summarization of work investigates In-context learning beyond natural language.}
%     \label{tab:my_label}
% \end{table}

\subsection{Speech In-Context Learning}
In the speech area, ~\citet{wang2023neural} treated text-to-speech synthesis as a language modeling task. 
They use audio codec codes as an intermediate representation and propose the first TTS framework with strong in-context learning capability. 
Subsequently, VALLE-X~\citep{zhang2023valle-x} extend the idea to multi-lingual scenarios, demonstrating superior performance in zero-shot cross-lingual text-to-speech synthesis and zero-shot speech-to-speech translation tasks.

\textbf{$\Diamond$ Takeaway}: 
\textit{
(1) Recent studies have explored in-context learning beyond natural language with promising results.
Properly formatted data (e.g., interleaved image-text datasets for vision-language tasks) and architecture designs are key factors for activating the potential of in-context learning. Exploring it in a more complex structured space such as for graph data is challenging and promising~\citep{huang2023graph_icl}.
(2) Findings in textual in-context learning demonstration design and selection cannot be trivially transferred to other modalities. Domain-specific investigation is required to fully leverage the potential of in-context learning in various modalities.
% Findings in textual in-context learning demonstration design and selection cannot be trivially transferred to other modalities, and require domain-specific investigation. 
}

% 



\section{Application}
\label{sec:application}
% ICL for traditional NLP tasks (dialogue)
\label{app}
%Due to the human-friendly interface lightweight prompting advantages, ICL has been widely applied in various scenarios recently.

ICL manifests excellent performance on traditional NLP tasks and methods~\cite{kim2022self,metaicl}, such as machine translation~\cite{zhu2023multilingual,sia2023context}, information extraction~\cite{wan2023gpt,he2023icl} and text-to-SQL~\cite{pourreza2023din}.
% tasks
%And ICL manifests excellent performance in various task
%, such as sentiment classification natural language inference~\cite{kim2022self}. 
Especially, through demonstrations that explicitly guide the process of reasoning, ICL manifests remarkable effects on tasks that require complexity reasoning~\cite{cot,li2023code,teachalgo} and compositional generalization~\cite{least}. 
% methods
% ~\citet{chen2021meta} applies ICL for meta-learning. 

Moreover, ICL offers potential for popular methods such as meta-learning and instruction-tuning. \citet{chen2021meta} applied ICL to meta-learning, adapting to new tasks with frozen model parameters, thus addressing the complex nested optimization issue. \cite{ye2023context} enhanced zero-shot task generalization performance for both pretrained and instruction-finetuned models by applying in-context learning to instruction learning. 

Specifically, we explore several emerging and prevalent applications of ICL, showcasing their potential in the following paragraphs.
% ICL for model editing.


% ICL for data engineering
% \paragraph{Data Engineering} What's more, ICL has manifested the potential to be widely applied in data engineering. 
% Benefiting from the strong ICL ability,  it costs 50\% to 96\% less to use labels from GPT-3 than using labels from humans for data annotation. Combining pseudo labels from GPT-3 with human labels leads to even better performance at a small cost~\cite{want}.
% In more complex scenarios, such as knowledge graph construction, \citet{khorashadizadeh2023exploring} has demonstrated that ICL has the potential to significantly improve the state of the art of automatic construction and completion of knowledge graphs, resulting in a reduction in manual costs with minimal engineering effort. Therefore, leveraging the capabilities of ICL in various data engineering applications can yield significant benefits.
\paragraph{Data Engineering}
ICL has manifested the potential to be widely applied in data engineering. 
Benefiting from the strong ICL ability,  it costs 50\% to 96\% less to use labels from GPT-3 than using labels from humans for data annotation. Combining pseudo labels from GPT-3 with human labels leads to even better performance at a small cost~\cite{want}.
In more complex scenarios, such as knowledge graph construction, \citet{khorashadizadeh2023exploring} has demonstrated that ICL has the potential to significantly improve the state of the art of automatic construction and completion of knowledge graphs, resulting in a reduction in manual costs with minimal engineering effort. Therefore, leveraging the capabilities of ICL in various data engineering applications can yield significant benefits.
Compared to human annotation (e.g., crowd-sourcing) or noisy automatic annotation (e.g., distant supervision), ICL generates relatively high quality data at a low cost. 
% First, ICL takes only a few examples to learn the data engineering objective, which saves the cost of annotating training data. Second, the strong reasoning ability and text-generating ability of LLM show great potential to generate high-quality data.
However, how to use ICL for data annotation remains an open question. For example, ~\citet{annotation} performed a comprehensive analysis and found that generation-based methods are more cost-effective in using GPT-3 than annotating unlabeled data via ICL.

% \paragraph{Model Probing} 
% Apart from ICL, linear probing is another way of black-box tuning~\citep{sun2022black} for LLMs, which learns a linear classifier based on final representations of LLMs and is suitable for full-data settings. \citet{cho2022prompt} propose prompt-augmented linear probing, a hybrid of linear probing and ICL. They train a linear classifier based on representations enhanced with prepend additional demonstrations. The hybrid of ICL and linear probing can cover the weakness of each other and scale the ICL to the full-data setting.

% \paragraph{Retrieval-Augmented Models} 
\paragraph{Model Augmentating} The context-flexible nature of ICL demonstrates significant potential to enhance retrieval-augmented methods. By keeping the LM architecture unchanged and prepending grounding documents to the input, in-context RALM\citet{ram2023context} effectively utilizes off-the-shelf general-purpose retrievers, resulting in substantial LM gains across various model sizes and diverse corpora. Furthermore, ICL for retrieval also exhibits the potential to improve safety. In addition to efficiency and flexibility, ICL also shows potential in safety~\cite{dpicl}, \cite{meade2023using} use ICL for retrieved demonstrations to steer a model towards safer generations, reducing bias and toxicity in the model.

\paragraph{Knowledge Updating}
LLMs may contain outdated or incorrect knowledge, but ICL demonstrates the potential for effectively editing and updating this information. In an initial trial, \citet{reliable} found that GPT-3 updated its answers 85\% of the time when provided with counterfactual examples, with larger models performing better at in-context knowledge updating. However, this approach may impact other correct knowledge in LLMs.
Compared to knowledge editing for fine-tuned models~\cite{editingfact}, ICL has proven effective for lightweight model editing. \citet{reliable} explored the possibility of editing LLMs' memorized knowledge through in-context demonstrations, discovering that a larger model scale and a mix of demonstration examples improved ICL-based knowledge editing success rates. In a comprehensive study, \citet{ike} investigated ICL strategies for editing factual knowledge, finding that well-designed demonstrations enabled competitive success rates compared to gradient-based methods, with significantly fewer side effects. This underlines the potential of ICL for knowledge editing.
% \subsection{Knowledge Augmentation and Updating}
 % ICL presents new issues for enhancing and updating knowledge in LLMs.
% knowledge augmentation
% The knowledge of LLMs is entirely derived from the pretrained corpus. Therefore, LLMs may lack certain knowledge and generate hallucinations during ICL inference, especially towards long-tailed factual knowledge or commonsense knowledge rarely described in texts.
% Therefore, it is essential to augment knowledge for ICL.
% Different from traditional work, which adds knowledge adapters~\cite{kadapter} or provides structured knowledge during pretraining~\cite{zhang_2019_ernie,peters_2019_knowledge},
% retrieving correct knowledge and integrating the correct knowledge with the context in a lightweight manner is possibly promissing for ICL.
% knowledge updating or calibration
% Although LLMs can serve as knowledge bases, 
% The knowledge in LLMs could be wrong or out-of-date, and ICL can be applied to edit or update the knowledge for more factual generations.
% ~\citet{reliable} made an initial trial on in-context knowledge updating. By providing counterfactual examples in the demonstration, they found that GPT-3 updates its answers around 85\% of the time and larger models are better at in-context knowledge updating. 
% However, this approach may affect other correct knowledge in LLMs. Compared with knowledge editing for finetuned models~\cite{editingfact},  
% ICL has shown effectiveness for lightweight model editing. \citet{reliable} analyzed whether it is possible to edit or adapt memorized knowledge in LLMs to new information through in-context demonstrations. They found that a large model scale and a mixture of all types of demonstration examples strengthen the knowledge editing success rate of ICL. In a comprehensive empirical study, \citet{ike} examined ICL strategies for editing factual knowledge and found that well-designed demonstrations enabled in-context knowledge editing to achieve a competitive success rate compared to gradient-based methods, but with significantly fewer side effects. This highlights the potential of ICL for knowledge editing.
% the impact of in-context knowledge updating on paraphrased facts and irrelevant facts has not been rigorously evaluated. 
% However, ICL for knowledge editing still faces challenges such as over-editing or multihop editing, which require further exploration.



\section{Challenges and Future Directions}
\label{sec:challege_future}
% Rosy
In this section, we review some of the existing challenges and propose possible directions for future research on ICL.
\subsection{New Pretraining Strategies}
As investigated by~\citet{corpusscale}, language model objectives are not equal to   ICL abilities. Researchers have proposed to bridge the gap between pretraining objectives and ICL through intermediate tuning before inference (Section~\ref{sec:warmup}), which shows promising performance improvements. To take it further, tailored pretraining objectives and metrics for ICL have the potential to raise LLMs with superior ICl capabilities.
% \subsection{New Evaluation Paradigm}
% \label{challenge: evaluation}
% As ICL is a new paradigm varies from traditional finetuning paradigm in many aspects,
% %such as training data scale and model parameter updating, 
% the evaluation of ICL presents a new set of challenges and opportunities.

% Towards the challenges, the results of existing evaluation methods are sensitive. \citet{chen2022relation} observed that existing evaluations by accuracy underestimate the sensitivity towards instruction perturbation of ICL.
% It is still an open question to conduct consistent ICL evaluation.
% Apart from that, traditional evaluation benchmarks such as SuperGLUE~\cite{superglue} are less challenging for LLMs with billions of parameters, and even the sophisticated designed new benchmark, BigBench~\cite{beyond} can be readily surpassed.
% How to construct a challenging evaluation with long-term vitality is crucial for ICL evaluation.
% %A dangerous phenomenon for ICL evaluation is that pretraining data leakage, that is, the test data is difficult to circumvent data leakage during pretraining.
% %This pose new challenges to trustworthy evaluation on ICL.

% Towards the opportunities for evaluation, as ICL only requires a few instances for the demonstration, it lowers the cost for evaluation data construction. On the one hand, this makes it possible to evaluate on some tasks whose data is tedious to construct or needs experts to annotate. On the other hand, the test set is still supposed to faithfully reflect the world distribution. 
% Instead of the i.i.d. assumption with the training set, the test set requires new constraints.
% \leiModify{But the evaluation set is still supposed to faithfully reflect the world distribution.}


\subsection{ICL Ability Distillation}
Previous studies have shown that in-context learning for reasoning tasks emerges as the scale of computation and parameter exceed a certain threshold~\citep{wei2022emergent}. Transferring the ICL ability to smaller models could facilitate the model deployment greatly.
\citet{distill2smallmodel} showed that it is possible to distill the reasoning ability to small language models such as T5-XXL. The distillation is achieved by finetuning the small model on the chain-of-thought data~\citep{cot} generated by a large teacher model. 
%, with the question as input and the CoT process along with the answer as the target
Although promising performance is achieved, %for example, the accuracy
%of T5-XXL on GSM8K improves from 8.11\%
%to 21.99\%, 
the improvements are likely task-dependent.
Further investigation on improving the reasoning ability by learning from larger LLMs could be an interesting direction.
% \subsection{Instance-level Selection}
% % \leiModify{TODO: ce\&zhiyong}
% Preliminary exploration~\cite{Wu2022SelfadaptiveIL} has shown the enormous promise of instance-level ICL, which has the potential to bridge the performance gap between ICL and finetuning. However, the search for the best demonstration organization at the instance-level is essentially an NP-hard combinatorial optimization problem. To tackle the problem of massive search space, we need specialized yet efficient algorithms (e.g., a learned example retriever) that can quickly rule out large parts of the search space. The designing of such algorithms is difficult since they require sequential decision-making, which makes applying reinforcement learning in the design attractive.
\subsection{ICL Robustness}
Previous studies have shown that ICL performance is extremely unstable, from random guess to SOTA, and can be sensitive to many factors, including demonstration permutation, demonstration format, etc.~\citep{calibrate, lu2022order}. 
The robustness of ICL is a critical yet challenging problem. 

However, most of the existing methods fall into the dilemma of accuracy and robustness~\cite{chen2022sensitivity}, or even at the cost of sacrificing inference efficiency. To effectively improve the robustness of ICL, we need deeper analysis of the working mechanism of the ICL. We believe that the analysis of the robustness of the ICL from a more theoretical perspective rather than an empirical perspective can highlight future research on more robust ICL.

\subsection{ICL Efficiency and Scalability}
ICL necessitates prepending a significant number of demonstrations within the context. However, it presents two challenges: (1) the quantity of demonstrations is constrained by the maximum input length of LMs, which is significantly fewer compared to fine-tuning (scalability); (2) as the number of demonstrations increases, the computation cost becomes higher due to the quadratic complexity of attention mechanism (efficiency). Previous work in \S\ref{sec:demo} focused on exploring how to achieve better ICL performance using a limited number of demonstrations and proposed several demonstration designing strategies. Scaling ICL to more demonstrations and improving its efficiency remains a challenging task.


Recently, some works have been proposed to address the issues of scalability and efficiency of ICL. Efforts were made to optimize prompting strategies with structured prompting~\citep{hao2022structured}, demonstration ensembling~\citep{khalifa2023exploring}, dynamic prompting~\citep{zhou2023efficient}, and iterative forward tuning~\citep{yang2023iterative}. Additionally, \citet{li2023context} proposed EVaLM with longer context length and enhanced long-range language modeling capabilities. This model-level improvement aims to improve the scalability and efficiency of ICL. As LMs continue to scale up, exploring ways to effectively and efficiently utilize a larger number of demonstrations in ICL remains an ongoing area of research.



%To address this problem, \citet{chen2022sensitivity} defined a metric for ICL called prediction sensitivity, which measures the changes in the model output in response to small input perturbations. 
%They found an obvious negative correlation between prediction sensitivity and ICL accuracy. Motivated by this finding, they proposed a selective prediction method based on prediction sensitivity. Unlike this study, \citet{Chang2022CarefulDC} focused on selecting stable demonstrations.
%, which means the ICL performance on different orderings of these demonstrations has a low variance and a high worst-case and average performance. 
%They assigned training examples with scores about stability and proposed two scoring methods. The first scoring combines an example with random training examples and gets the average ICL performance as its score. Another method learns a linear proxy model to estimate how an example influences ICL performance.

%We hope to see more stable and robust ICL methods in the future because stable and robust ICL makes it easier to construct demonstrations without being overly concerned with influencing factors such as order. 
% \subsection{Working Mechanism}
% Understanding the working mechanism of ICL is an inevitable step to trust ICL and further improve it. 
% Although some analytical work has taken a preliminary step to explain ICL, most of them is limited to simple tasks and small models. 
% Extending analysis on extensive tasks and real large models may be the next step to be considered. 
% In addition, among existing work, understanding ICL as a process of meta-optimization seems to be a reasonable and promising direction for future research. 
% If we build clear connections between ICL and meta-optimization, we can improve ICL with a purpose by learning from the history of finetuning and optimization. 

% \subsection{ICL for Data Engineering}
% %Cost and quality have been two conflicting aspects of data engineering. 
% ICL has manifested the potential to be widely applied in data engineering. 
% Benefiting from the strong ICL ability,  it costs 50\% to 96\% less to use labels from GPT-3 than using labels from humans for data annotation. Combining pseudo labels from GPT-3 with human labels leads to even better performance at a small cost~\cite{want}.
% In more complex scenarios, such as knowledge graph construction, \citet{khorashadizadeh2023exploring} has demonstrated that ICL has the potential to significantly improve the state of the art of automatic construction and completion of knowledge graphs, resulting in a reduction in manual costs with minimal engineering effort. Therefore, leveraging the capabilities of ICL in various data engineering applications can yield significant benefits.
% Compared to human annotation (e.g., crowd-sourcing) or noisy automatic annotation (e.g., distant supervision), ICL generates relatively high quality data at a low cost. 
% First, ICL takes only a few examples to learn the data engineering objective, which saves the cost of annotating training data. Second, the strong reasoning ability and text-generating ability of LLM show great potential to generate high-quality data.

% However, how to use ICL for data annotation remains an open question. For example, ~\citet{annotation} performed a comprehensive analysis and found that generation-based methods are more cost-effective in using GPT-3 than annotating unlabeled data via ICL.
% We believe that improving ICL for data annotation is a direction with practical value, and ICL will be a new paradigm in data annotation, data augmentation, data pruning, as well as adversarial data generation.


% \section{Related Concepts}
% \label{sec:related}
% \paragraph{Few-shot Learning}

\paragraph{Fine-tuning}

\paragraph{Prompt Predicting}

\paragraph{Instruction Tuning}

\section{Conclusion}
\label{sec:conclusion}
In this paper, we survey the existing ICL literature and provide an extensive review of advanced ICL techniques, including training strategies, demonstration designing strategies, evaluation datasets and resources, as well as related analytical studies. Furthermore, we highlight critical challenges and potential directions for future research. To the best of our knowledge, this is the first survey about ICL. We hope this survey can highlight the current research status of ICL and shed light on future work on this promising paradigm. 


\bibliography{tacl2021}
\bibliographystyle{acl_natbib}

% \clearpage

% \appendix

% \section*{Appendix}
% \label{sec:appendix}
% \chapter{Supplementary Material}
\label{appendix}

In this appendix, we present supplementary material for the techniques and
experiments presented in the main text.

\section{Baseline Results and Analysis for Informed Sampler}
\label{appendix:chap3}

Here, we give an in-depth
performance analysis of the various samplers and the effect of their
hyperparameters. We choose hyperparameters with the lowest PSRF value
after $10k$ iterations, for each sampler individually. If the
differences between PSRF are not significantly different among
multiple values, we choose the one that has the highest acceptance
rate.

\subsection{Experiment: Estimating Camera Extrinsics}
\label{appendix:chap3:room}

\subsubsection{Parameter Selection}
\paragraph{Metropolis Hastings (\MH)}

Figure~\ref{fig:exp1_MH} shows the median acceptance rates and PSRF
values corresponding to various proposal standard deviations of plain
\MH~sampling. Mixing gets better and the acceptance rate gets worse as
the standard deviation increases. The value $0.3$ is selected standard
deviation for this sampler.

\paragraph{Metropolis Hastings Within Gibbs (\MHWG)}

As mentioned in Section~\ref{sec:room}, the \MHWG~sampler with one-dimensional
updates did not converge for any value of proposal standard deviation.
This problem has high correlation of the camera parameters and is of
multi-modal nature, which this sampler has problems with.

\paragraph{Parallel Tempering (\PT)}

For \PT~sampling, we took the best performing \MH~sampler and used
different temperature chains to improve the mixing of the
sampler. Figure~\ref{fig:exp1_PT} shows the results corresponding to
different combination of temperature levels. The sampler with
temperature levels of $[1,3,27]$ performed best in terms of both
mixing and acceptance rate.

\paragraph{Effect of Mixture Coefficient in Informed Sampling (\MIXLMH)}

Figure~\ref{fig:exp1_alpha} shows the effect of mixture
coefficient ($\alpha$) on the informed sampling
\MIXLMH. Since there is no significant different in PSRF values for
$0 \le \alpha \le 0.7$, we chose $0.7$ due to its high acceptance
rate.


% \end{multicols}

\begin{figure}[h]
\centering
  \subfigure[MH]{%
    \includegraphics[width=.48\textwidth]{figures/supplementary/camPose_MH.pdf} \label{fig:exp1_MH}
  }
  \subfigure[PT]{%
    \includegraphics[width=.48\textwidth]{figures/supplementary/camPose_PT.pdf} \label{fig:exp1_PT}
  }
\\
  \subfigure[INF-MH]{%
    \includegraphics[width=.48\textwidth]{figures/supplementary/camPose_alpha.pdf} \label{fig:exp1_alpha}
  }
  \mycaption{Results of the `Estimating Camera Extrinsics' experiment}{PRSFs and Acceptance rates corresponding to (a) various standard deviations of \MH, (b) various temperature level combinations of \PT sampling and (c) various mixture coefficients of \MIXLMH sampling.}
\end{figure}



\begin{figure}[!t]
\centering
  \subfigure[\MH]{%
    \includegraphics[width=.48\textwidth]{figures/supplementary/occlusionExp_MH.pdf} \label{fig:exp2_MH}
  }
  \subfigure[\BMHWG]{%
    \includegraphics[width=.48\textwidth]{figures/supplementary/occlusionExp_BMHWG.pdf} \label{fig:exp2_BMHWG}
  }
\\
  \subfigure[\MHWG]{%
    \includegraphics[width=.48\textwidth]{figures/supplementary/occlusionExp_MHWG.pdf} \label{fig:exp2_MHWG}
  }
  \subfigure[\PT]{%
    \includegraphics[width=.48\textwidth]{figures/supplementary/occlusionExp_PT.pdf} \label{fig:exp2_PT}
  }
\\
  \subfigure[\INFBMHWG]{%
    \includegraphics[width=.5\textwidth]{figures/supplementary/occlusionExp_alpha.pdf} \label{fig:exp2_alpha}
  }
  \mycaption{Results of the `Occluding Tiles' experiment}{PRSF and
    Acceptance rates corresponding to various standard deviations of
    (a) \MH, (b) \BMHWG, (c) \MHWG, (d) various temperature level
    combinations of \PT~sampling and; (e) various mixture coefficients
    of our informed \INFBMHWG sampling.}
\end{figure}

%\onecolumn\newpage\twocolumn
\subsection{Experiment: Occluding Tiles}
\label{appendix:chap3:tiles}

\subsubsection{Parameter Selection}

\paragraph{Metropolis Hastings (\MH)}

Figure~\ref{fig:exp2_MH} shows the results of
\MH~sampling. Results show the poor convergence for all proposal
standard deviations and rapid decrease of AR with increasing standard
deviation. This is due to the high-dimensional nature of
the problem. We selected a standard deviation of $1.1$.

\paragraph{Blocked Metropolis Hastings Within Gibbs (\BMHWG)}

The results of \BMHWG are shown in Figure~\ref{fig:exp2_BMHWG}. In
this sampler we update only one block of tile variables (of dimension
four) in each sampling step. Results show much better performance
compared to plain \MH. The optimal proposal standard deviation for
this sampler is $0.7$.

\paragraph{Metropolis Hastings Within Gibbs (\MHWG)}

Figure~\ref{fig:exp2_MHWG} shows the result of \MHWG sampling. This
sampler is better than \BMHWG and converges much more quickly. Here
a standard deviation of $0.9$ is found to be best.

\paragraph{Parallel Tempering (\PT)}

Figure~\ref{fig:exp2_PT} shows the results of \PT sampling with various
temperature combinations. Results show no improvement in AR from plain
\MH sampling and again $[1,3,27]$ temperature levels are found to be optimal.

\paragraph{Effect of Mixture Coefficient in Informed Sampling (\INFBMHWG)}

Figure~\ref{fig:exp2_alpha} shows the effect of mixture
coefficient ($\alpha$) on the blocked informed sampling
\INFBMHWG. Since there is no significant different in PSRF values for
$0 \le \alpha \le 0.8$, we chose $0.8$ due to its high acceptance
rate.



\subsection{Experiment: Estimating Body Shape}
\label{appendix:chap3:body}

\subsubsection{Parameter Selection}
\paragraph{Metropolis Hastings (\MH)}

Figure~\ref{fig:exp3_MH} shows the result of \MH~sampling with various
proposal standard deviations. The value of $0.1$ is found to be
best.

\paragraph{Metropolis Hastings Within Gibbs (\MHWG)}

For \MHWG sampling we select $0.3$ proposal standard
deviation. Results are shown in Fig.~\ref{fig:exp3_MHWG}.


\paragraph{Parallel Tempering (\PT)}

As before, results in Fig.~\ref{fig:exp3_PT}, the temperature levels
were selected to be $[1,3,27]$ due its slightly higher AR.

\paragraph{Effect of Mixture Coefficient in Informed Sampling (\MIXLMH)}

Figure~\ref{fig:exp3_alpha} shows the effect of $\alpha$ on PSRF and
AR. Since there is no significant differences in PSRF values for $0 \le
\alpha \le 0.8$, we choose $0.8$.


\begin{figure}[t]
\centering
  \subfigure[\MH]{%
    \includegraphics[width=.48\textwidth]{figures/supplementary/bodyShape_MH.pdf} \label{fig:exp3_MH}
  }
  \subfigure[\MHWG]{%
    \includegraphics[width=.48\textwidth]{figures/supplementary/bodyShape_MHWG.pdf} \label{fig:exp3_MHWG}
  }
\\
  \subfigure[\PT]{%
    \includegraphics[width=.48\textwidth]{figures/supplementary/bodyShape_PT.pdf} \label{fig:exp3_PT}
  }
  \subfigure[\MIXLMH]{%
    \includegraphics[width=.48\textwidth]{figures/supplementary/bodyShape_alpha.pdf} \label{fig:exp3_alpha}
  }
\\
  \mycaption{Results of the `Body Shape Estimation' experiment}{PRSFs and
    Acceptance rates corresponding to various standard deviations of
    (a) \MH, (b) \MHWG; (c) various temperature level combinations
    of \PT sampling and; (d) various mixture coefficients of the
    informed \MIXLMH sampling.}
\end{figure}


\subsection{Results Overview}
Figure~\ref{fig:exp_summary} shows the summary results of the all the three
experimental studies related to informed sampler.
\begin{figure*}[h!]
\centering
  \subfigure[Results for: Estimating Camera Extrinsics]{%
    \includegraphics[width=0.9\textwidth]{figures/supplementary/camPose_ALL.pdf} \label{fig:exp1_all}
  }
  \subfigure[Results for: Occluding Tiles]{%
    \includegraphics[width=0.9\textwidth]{figures/supplementary/occlusionExp_ALL.pdf} \label{fig:exp2_all}
  }
  \subfigure[Results for: Estimating Body Shape]{%
    \includegraphics[width=0.9\textwidth]{figures/supplementary/bodyShape_ALL.pdf} \label{fig:exp3_all}
  }
  \label{fig:exp_summary}
  \mycaption{Summary of the statistics for the three experiments}{Shown are
    for several baseline methods and the informed samplers the
    acceptance rates (left), PSRFs (middle), and RMSE values
    (right). All results are median results over multiple test
    examples.}
\end{figure*}

\subsection{Additional Qualitative Results}

\subsubsection{Occluding Tiles}
In Figure~\ref{fig:exp2_visual_more} more qualitative results of the
occluding tiles experiment are shown. The informed sampling approach
(\INFBMHWG) is better than the best baseline (\MHWG). This still is a
very challenging problem since the parameters for occluded tiles are
flat over a large region. Some of the posterior variance of the
occluded tiles is already captured by the informed sampler.

\begin{figure*}[h!]
\begin{center}
\centerline{\includegraphics[width=0.95\textwidth]{figures/supplementary/occlusionExp_Visual.pdf}}
\mycaption{Additional qualitative results of the occluding tiles experiment}
  {From left to right: (a)
  Given image, (b) Ground truth tiles, (c) OpenCV heuristic and most probable estimates
  from 5000 samples obtained by (d) MHWG sampler (best baseline) and
  (e) our INF-BMHWG sampler. (f) Posterior expectation of the tiles
  boundaries obtained by INF-BMHWG sampling (First 2000 samples are
  discarded as burn-in).}
\label{fig:exp2_visual_more}
\end{center}
\end{figure*}

\subsubsection{Body Shape}
Figure~\ref{fig:exp3_bodyMeshes} shows some more results of 3D mesh
reconstruction using posterior samples obtained by our informed
sampling \MIXLMH.

\begin{figure*}[t]
\begin{center}
\centerline{\includegraphics[width=0.75\textwidth]{figures/supplementary/bodyMeshResults.pdf}}
\mycaption{Qualitative results for the body shape experiment}
  {Shown is the 3D mesh reconstruction results with first 1000 samples obtained
  using the \MIXLMH informed sampling method. (blue indicates small
  values and red indicates high values)}
\label{fig:exp3_bodyMeshes}
\end{center}
\end{figure*}

\clearpage



\section{Additional Results on the Face Problem with CMP}

Figure~\ref{fig:shading-qualitative-multiple-subjects-supp} shows inference results for reflectance maps, normal maps and lights for randomly chosen test images, and Fig.~\ref{fig:shading-qualitative-same-subject-supp} shows reflectance estimation results on multiple images of the same subject produced under different illumination conditions. CMP is able to produce estimates that are closer to the groundtruth across different subjects and illumination conditions.

\begin{figure*}[h]
  \begin{center}
  \centerline{\includegraphics[width=1.0\columnwidth]{figures/face_cmp_visual_results_supp.pdf}}
  \vspace{-1.2cm}
  \end{center}
	\mycaption{A visual comparison of inference results}{(a)~Observed images. (b)~Inferred reflectance maps. \textit{GT} is the photometric stereo groundtruth, \textit{BU} is the Biswas \etal (2009) reflectance estimate and \textit{Forest} is the consensus prediction. (c)~The variance of the inferred reflectance estimate produced by \MTD (normalized across rows).(d)~Visualization of inferred light directions. (e)~Inferred normal maps.}
	\label{fig:shading-qualitative-multiple-subjects-supp}
\end{figure*}


\begin{figure*}[h]
	\centering
	\setlength\fboxsep{0.2mm}
	\setlength\fboxrule{0pt}
	\begin{tikzpicture}

		\matrix at (0, 0) [matrix of nodes, nodes={anchor=east}, column sep=-0.05cm, row sep=-0.2cm]
		{
			\fbox{\includegraphics[width=1cm]{figures/sample_3_4_X.png}} &
			\fbox{\includegraphics[width=1cm]{figures/sample_3_4_GT.png}} &
			\fbox{\includegraphics[width=1cm]{figures/sample_3_4_BISWAS.png}}  &
			\fbox{\includegraphics[width=1cm]{figures/sample_3_4_VMP.png}}  &
			\fbox{\includegraphics[width=1cm]{figures/sample_3_4_FOREST.png}}  &
			\fbox{\includegraphics[width=1cm]{figures/sample_3_4_CMP.png}}  &
			\fbox{\includegraphics[width=1cm]{figures/sample_3_4_CMPVAR.png}}
			 \\

			\fbox{\includegraphics[width=1cm]{figures/sample_3_5_X.png}} &
			\fbox{\includegraphics[width=1cm]{figures/sample_3_5_GT.png}} &
			\fbox{\includegraphics[width=1cm]{figures/sample_3_5_BISWAS.png}}  &
			\fbox{\includegraphics[width=1cm]{figures/sample_3_5_VMP.png}}  &
			\fbox{\includegraphics[width=1cm]{figures/sample_3_5_FOREST.png}}  &
			\fbox{\includegraphics[width=1cm]{figures/sample_3_5_CMP.png}}  &
			\fbox{\includegraphics[width=1cm]{figures/sample_3_5_CMPVAR.png}}
			 \\

			\fbox{\includegraphics[width=1cm]{figures/sample_3_6_X.png}} &
			\fbox{\includegraphics[width=1cm]{figures/sample_3_6_GT.png}} &
			\fbox{\includegraphics[width=1cm]{figures/sample_3_6_BISWAS.png}}  &
			\fbox{\includegraphics[width=1cm]{figures/sample_3_6_VMP.png}}  &
			\fbox{\includegraphics[width=1cm]{figures/sample_3_6_FOREST.png}}  &
			\fbox{\includegraphics[width=1cm]{figures/sample_3_6_CMP.png}}  &
			\fbox{\includegraphics[width=1cm]{figures/sample_3_6_CMPVAR.png}}
			 \\
	     };

       \node at (-3.85, -2.0) {\small Observed};
       \node at (-2.55, -2.0) {\small `GT'};
       \node at (-1.27, -2.0) {\small BU};
       \node at (0.0, -2.0) {\small MP};
       \node at (1.27, -2.0) {\small Forest};
       \node at (2.55, -2.0) {\small \textbf{CMP}};
       \node at (3.85, -2.0) {\small Variance};

	\end{tikzpicture}
	\mycaption{Robustness to varying illumination}{Reflectance estimation on a subject images with varying illumination. Left to right: observed image, photometric stereo estimate (GT)
  which is used as a proxy for groundtruth, bottom-up estimate of \cite{Biswas2009}, VMP result, consensus forest estimate, CMP mean, and CMP variance.}
	\label{fig:shading-qualitative-same-subject-supp}
\end{figure*}

\clearpage

\section{Additional Material for Learning Sparse High Dimensional Filters}
\label{sec:appendix-bnn}

This part of supplementary material contains a more detailed overview of the permutohedral
lattice convolution in Section~\ref{sec:permconv}, more experiments in
Section~\ref{sec:addexps} and additional results with protocols for
the experiments presented in Chapter~\ref{chap:bnn} in Section~\ref{sec:addresults}.

\vspace{-0.2cm}
\subsection{General Permutohedral Convolutions}
\label{sec:permconv}

A core technical contribution of this work is the generalization of the Gaussian permutohedral lattice
convolution proposed in~\cite{adams2010fast} to the full non-separable case with the
ability to perform back-propagation. Although, conceptually, there are minor
differences between Gaussian and general parameterized filters, there are non-trivial practical
differences in terms of the algorithmic implementation. The Gauss filters belong to
the separable class and can thus be decomposed into multiple
sequential one dimensional convolutions. We are interested in the general filter
convolutions, which can not be decomposed. Thus, performing a general permutohedral
convolution at a lattice point requires the computation of the inner product with the
neighboring elements in all the directions in the high-dimensional space.

Here, we give more details of the implementation differences of separable
and non-separable filters. In the following, we will explain the scalar case first.
Recall, that the forward pass of general permutohedral convolution
involves 3 steps: \textit{splatting}, \textit{convolving} and \textit{slicing}.
We follow the same splatting and slicing strategies as in~\cite{adams2010fast}
since these operations do not depend on the filter kernel. The main difference
between our work and the existing implementation of~\cite{adams2010fast} is
the way that the convolution operation is executed. This proceeds by constructing
a \emph{blur neighbor} matrix $K$ that stores for every lattice point all
values of the lattice neighbors that are needed to compute the filter output.

\begin{figure}[t!]
  \centering
    \includegraphics[width=0.6\columnwidth]{figures/supplementary/lattice_construction}
  \mycaption{Illustration of 1D permutohedral lattice construction}
  {A $4\times 4$ $(x,y)$ grid lattice is projected onto the plane defined by the normal
  vector $(1,1)^{\top}$. This grid has $s+1=4$ and $d=2$ $(s+1)^{d}=4^2=16$ elements.
  In the projection, all points of the same color are projected onto the same points in the plane.
  The number of elements of the projected lattice is $t=(s+1)^d-s^d=4^2-3^2=7$, that is
  the $(4\times 4)$ grid minus the size of lattice that is $1$ smaller at each size, in this
  case a $(3\times 3)$ lattice (the upper right $(3\times 3)$ elements).
  }
\label{fig:latticeconstruction}
\end{figure}

The blur neighbor matrix is constructed by traversing through all the populated
lattice points and their neighboring elements.
% For efficiency, we do this matrix construction recursively with shared computations
% since $n^{th}$ neighbourhood elements are $1^{st}$ neighborhood elements of $n-1^{th}$ neighbourhood elements. \pg{do not understand}
This is done recursively to share computations. For any lattice point, the neighbors that are
$n$ hops away are the direct neighbors of the points that are $n-1$ hops away.
The size of a $d$ dimensional spatial filter with width $s+1$ is $(s+1)^{d}$ (\eg, a
$3\times 3$ filter, $s=2$ in $d=2$ has $3^2=9$ elements) and this size grows
exponentially in the number of dimensions $d$. The permutohedral lattice is constructed by
projecting a regular grid onto the plane spanned by the $d$ dimensional normal vector ${(1,\ldots,1)}^{\top}$. See
Fig.~\ref{fig:latticeconstruction} for an illustration of the 1D lattice construction.
Many corners of a grid filter are projected onto the same point, in total $t = {(s+1)}^{d} -
s^{d}$ elements remain in the permutohedral filter with $s$ neighborhood in $d-1$ dimensions.
If the lattice has $m$ populated elements, the
matrix $K$ has size $t\times m$. Note that, since the input signal is typically
sparse, only a few lattice corners are being populated in the \textit{slicing} step.
We use a hash-table to keep track of these points and traverse only through
the populated lattice points for this neighborhood matrix construction.

Once the blur neighbor matrix $K$ is constructed, we can perform the convolution
by the matrix vector multiplication
\begin{equation}
\ell' = BK,
\label{eq:conv}
\end{equation}
where $B$ is the $1 \times t$ filter kernel (whose values we will learn) and $\ell'\in\mathbb{R}^{1\times m}$
is the result of the filtering at the $m$ lattice points. In practice, we found that the
matrix $K$ is sometimes too large to fit into GPU memory and we divided the matrix $K$
into smaller pieces to compute Eq.~\ref{eq:conv} sequentially.

In the general multi-dimensional case, the signal $\ell$ is of $c$ dimensions. Then
the kernel $B$ is of size $c \times t$ and $K$ stores the $c$ dimensional vectors
accordingly. When the input and output points are different, we slice only the
input points and splat only at the output points.


\subsection{Additional Experiments}
\label{sec:addexps}
In this section, we discuss more use-cases for the learned bilateral filters, one
use-case of BNNs and two single filter applications for image and 3D mesh denoising.

\subsubsection{Recognition of subsampled MNIST}\label{sec:app_mnist}

One of the strengths of the proposed filter convolution is that it does not
require the input to lie on a regular grid. The only requirement is to define a distance
between features of the input signal.
We highlight this feature with the following experiment using the
classical MNIST ten class classification problem~\cite{lecun1998mnist}. We sample a
sparse set of $N$ points $(x,y)\in [0,1]\times [0,1]$
uniformly at random in the input image, use their interpolated values
as signal and the \emph{continuous} $(x,y)$ positions as features. This mimics
sub-sampling of a high-dimensional signal. To compare against a spatial convolution,
we interpolate the sparse set of values at the grid positions.

We take a reference implementation of LeNet~\cite{lecun1998gradient} that
is part of the Caffe project~\cite{jia2014caffe} and compare it
against the same architecture but replacing the first convolutional
layer with a bilateral convolution layer (BCL). The filter size
and numbers are adjusted to get a comparable number of parameters
($5\times 5$ for LeNet, $2$-neighborhood for BCL).

The results are shown in Table~\ref{tab:all-results}. We see that training
on the original MNIST data (column Original, LeNet vs. BNN) leads to a slight
decrease in performance of the BNN (99.03\%) compared to LeNet
(99.19\%). The BNN can be trained and evaluated on sparse
signals, and we resample the image as described above for $N=$ 100\%, 60\% and
20\% of the total number of pixels. The methods are also evaluated
on test images that are subsampled in the same way. Note that we can
train and test with different subsampling rates. We introduce an additional
bilinear interpolation layer for the LeNet architecture to train on the same
data. In essence, both models perform a spatial interpolation and thus we
expect them to yield a similar classification accuracy. Once the data is of
higher dimensions, the permutohedral convolution will be faster due to hashing
the sparse input points, as well as less memory demanding in comparison to
naive application of a spatial convolution with interpolated values.

\begin{table}[t]
  \begin{center}
    \footnotesize
    \centering
    \begin{tabular}[t]{lllll}
      \toprule
              &     & \multicolumn{3}{c}{Test Subsampling} \\
       Method  & Original & 100\% & 60\% & 20\%\\
      \midrule
       LeNet &  \textbf{0.9919} & 0.9660 & 0.9348 & \textbf{0.6434} \\
       BNN &  0.9903 & \textbf{0.9844} & \textbf{0.9534} & 0.5767 \\
      \hline
       LeNet 100\% & 0.9856 & 0.9809 & 0.9678 & \textbf{0.7386} \\
       BNN 100\% & \textbf{0.9900} & \textbf{0.9863} & \textbf{0.9699} & 0.6910 \\
      \hline
       LeNet 60\% & 0.9848 & 0.9821 & 0.9740 & 0.8151 \\
       BNN 60\% & \textbf{0.9885} & \textbf{0.9864} & \textbf{0.9771} & \textbf{0.8214}\\
      \hline
       LeNet 20\% & \textbf{0.9763} & \textbf{0.9754} & 0.9695 & 0.8928 \\
       BNN 20\% & 0.9728 & 0.9735 & \textbf{0.9701} & \textbf{0.9042}\\
      \bottomrule
    \end{tabular}
  \end{center}
\vspace{-.2cm}
\caption{Classification accuracy on MNIST. We compare the
    LeNet~\cite{lecun1998gradient} implementation that is part of
    Caffe~\cite{jia2014caffe} to the network with the first layer
    replaced by a bilateral convolution layer (BCL). Both are trained
    on the original image resolution (first two rows). Three more BNN
    and CNN models are trained with randomly subsampled images (100\%,
    60\% and 20\% of the pixels). An additional bilinear interpolation
    layer samples the input signal on a spatial grid for the CNN model.
  }
  \label{tab:all-results}
\vspace{-.5cm}
\end{table}

\subsubsection{Image Denoising}

The main application that inspired the development of the bilateral
filtering operation is image denoising~\cite{aurich1995non}, there
using a single Gaussian kernel. Our development allows to learn this
kernel function from data and we explore how to improve using a \emph{single}
but more general bilateral filter.

We use the Berkeley segmentation dataset
(BSDS500)~\cite{arbelaezi2011bsds500} as a test bed. The color
images in the dataset are converted to gray-scale,
and corrupted with Gaussian noise with a standard deviation of
$\frac {25} {255}$.

We compare the performance of four different filter models on a
denoising task.
The first baseline model (`Spatial' in Table \ref{tab:denoising}, $25$
weights) uses a single spatial filter with a kernel size of
$5$ and predicts the scalar gray-scale value at the center pixel. The next model
(`Gauss Bilateral') applies a bilateral \emph{Gaussian}
filter to the noisy input, using position and intensity features $\f=(x,y,v)^\top$.
The third setup (`Learned Bilateral', $65$ weights)
takes a Gauss kernel as initialization and
fits all filter weights on the train set to minimize the
mean squared error with respect to the clean images.
We run a combination
of spatial and permutohedral convolutions on spatial and bilateral
features (`Spatial + Bilateral (Learned)') to check for a complementary
performance of the two convolutions.

\label{sec:exp:denoising}
\begin{table}[!h]
\begin{center}
  \footnotesize
  \begin{tabular}[t]{lr}
    \toprule
    Method & PSNR \\
    \midrule
    Noisy Input & $20.17$ \\
    Spatial & $26.27$ \\
    Gauss Bilateral & $26.51$ \\
    Learned Bilateral & $26.58$ \\
    Spatial + Bilateral (Learned) & \textbf{$26.65$} \\
    \bottomrule
  \end{tabular}
\end{center}
\vspace{-0.5em}
\caption{PSNR results of a denoising task using the BSDS500
  dataset~\cite{arbelaezi2011bsds500}}
\vspace{-0.5em}
\label{tab:denoising}
\end{table}
\vspace{-0.2em}

The PSNR scores evaluated on full images of the test set are
shown in Table \ref{tab:denoising}. We find that an untrained bilateral
filter already performs better than a trained spatial convolution
($26.27$ to $26.51$). A learned convolution further improve the
performance slightly. We chose this simple one-kernel setup to
validate an advantage of the generalized bilateral filter. A competitive
denoising system would employ RGB color information and also
needs to be properly adjusted in network size. Multi-layer perceptrons
have obtained state-of-the-art denoising results~\cite{burger12cvpr}
and the permutohedral lattice layer can readily be used in such an
architecture, which is intended future work.

\subsection{Additional results}
\label{sec:addresults}

This section contains more qualitative results for the experiments presented in Chapter~\ref{chap:bnn}.

\begin{figure*}[th!]
  \centering
    \includegraphics[width=\columnwidth,trim={5cm 2.5cm 5cm 4.5cm},clip]{figures/supplementary/lattice_viz.pdf}
    \vspace{-0.7cm}
  \mycaption{Visualization of the Permutohedral Lattice}
  {Sample lattice visualizations for different feature spaces. All pixels falling in the same simplex cell are shown with
  the same color. $(x,y)$ features correspond to image pixel positions, and $(r,g,b) \in [0,255]$ correspond
  to the red, green and blue color values.}
\label{fig:latticeviz}
\end{figure*}

\subsubsection{Lattice Visualization}

Figure~\ref{fig:latticeviz} shows sample lattice visualizations for different feature spaces.

\newcolumntype{L}[1]{>{\raggedright\let\newline\\\arraybackslash\hspace{0pt}}b{#1}}
\newcolumntype{C}[1]{>{\centering\let\newline\\\arraybackslash\hspace{0pt}}b{#1}}
\newcolumntype{R}[1]{>{\raggedleft\let\newline\\\arraybackslash\hspace{0pt}}b{#1}}

\subsubsection{Color Upsampling}\label{sec:color_upsampling}
\label{sec:col_upsample_extra}

Some images of the upsampling for the Pascal
VOC12 dataset are shown in Fig.~\ref{fig:Colour_upsample_visuals}. It is
especially the low level image details that are better preserved with
a learned bilateral filter compared to the Gaussian case.

\begin{figure*}[t!]
  \centering
    \subfigure{%
   \raisebox{2.0em}{
    \includegraphics[width=.06\columnwidth]{figures/supplementary/2007_004969.jpg}
   }
  }
  \subfigure{%
    \includegraphics[width=.17\columnwidth]{figures/supplementary/2007_004969_gray.pdf}
  }
  \subfigure{%
    \includegraphics[width=.17\columnwidth]{figures/supplementary/2007_004969_gt.pdf}
  }
  \subfigure{%
    \includegraphics[width=.17\columnwidth]{figures/supplementary/2007_004969_bicubic.pdf}
  }
  \subfigure{%
    \includegraphics[width=.17\columnwidth]{figures/supplementary/2007_004969_gauss.pdf}
  }
  \subfigure{%
    \includegraphics[width=.17\columnwidth]{figures/supplementary/2007_004969_learnt.pdf}
  }\\
    \subfigure{%
   \raisebox{2.0em}{
    \includegraphics[width=.06\columnwidth]{figures/supplementary/2007_003106.jpg}
   }
  }
  \subfigure{%
    \includegraphics[width=.17\columnwidth]{figures/supplementary/2007_003106_gray.pdf}
  }
  \subfigure{%
    \includegraphics[width=.17\columnwidth]{figures/supplementary/2007_003106_gt.pdf}
  }
  \subfigure{%
    \includegraphics[width=.17\columnwidth]{figures/supplementary/2007_003106_bicubic.pdf}
  }
  \subfigure{%
    \includegraphics[width=.17\columnwidth]{figures/supplementary/2007_003106_gauss.pdf}
  }
  \subfigure{%
    \includegraphics[width=.17\columnwidth]{figures/supplementary/2007_003106_learnt.pdf}
  }\\
  \setcounter{subfigure}{0}
  \small{
  \subfigure[Inp.]{%
  \raisebox{2.0em}{
    \includegraphics[width=.06\columnwidth]{figures/supplementary/2007_006837.jpg}
   }
  }
  \subfigure[Guidance]{%
    \includegraphics[width=.17\columnwidth]{figures/supplementary/2007_006837_gray.pdf}
  }
   \subfigure[GT]{%
    \includegraphics[width=.17\columnwidth]{figures/supplementary/2007_006837_gt.pdf}
  }
  \subfigure[Bicubic]{%
    \includegraphics[width=.17\columnwidth]{figures/supplementary/2007_006837_bicubic.pdf}
  }
  \subfigure[Gauss-BF]{%
    \includegraphics[width=.17\columnwidth]{figures/supplementary/2007_006837_gauss.pdf}
  }
  \subfigure[Learned-BF]{%
    \includegraphics[width=.17\columnwidth]{figures/supplementary/2007_006837_learnt.pdf}
  }
  }
  \vspace{-0.5cm}
  \mycaption{Color Upsampling}{Color $8\times$ upsampling results
  using different methods, from left to right, (a)~Low-resolution input color image (Inp.),
  (b)~Gray scale guidance image, (c)~Ground-truth color image; Upsampled color images with
  (d)~Bicubic interpolation, (e) Gauss bilateral upsampling and, (f)~Learned bilateral
  updampgling (best viewed on screen).}

\label{fig:Colour_upsample_visuals}
\end{figure*}

\subsubsection{Depth Upsampling}
\label{sec:depth_upsample_extra}

Figure~\ref{fig:depth_upsample_visuals} presents some more qualitative results comparing bicubic interpolation, Gauss
bilateral and learned bilateral upsampling on NYU depth dataset image~\cite{silberman2012indoor}.

\subsubsection{Character Recognition}\label{sec:app_character}

 Figure~\ref{fig:nnrecognition} shows the schematic of different layers
 of the network architecture for LeNet-7~\cite{lecun1998mnist}
 and DeepCNet(5, 50)~\cite{ciresan2012multi,graham2014spatially}. For the BNN variants, the first layer filters are replaced
 with learned bilateral filters and are learned end-to-end.

\subsubsection{Semantic Segmentation}\label{sec:app_semantic_segmentation}
\label{sec:semantic_bnn_extra}

Some more visual results for semantic segmentation are shown in Figure~\ref{fig:semantic_visuals}.
These include the underlying DeepLab CNN\cite{chen2014semantic} result (DeepLab),
the 2 step mean-field result with Gaussian edge potentials (+2stepMF-GaussCRF)
and also corresponding results with learned edge potentials (+2stepMF-LearnedCRF).
In general, we observe that mean-field in learned CRF leads to slightly dilated
classification regions in comparison to using Gaussian CRF thereby filling-in the
false negative pixels and also correcting some mis-classified regions.

\begin{figure*}[t!]
  \centering
    \subfigure{%
   \raisebox{2.0em}{
    \includegraphics[width=.06\columnwidth]{figures/supplementary/2bicubic}
   }
  }
  \subfigure{%
    \includegraphics[width=.17\columnwidth]{figures/supplementary/2given_image}
  }
  \subfigure{%
    \includegraphics[width=.17\columnwidth]{figures/supplementary/2ground_truth}
  }
  \subfigure{%
    \includegraphics[width=.17\columnwidth]{figures/supplementary/2bicubic}
  }
  \subfigure{%
    \includegraphics[width=.17\columnwidth]{figures/supplementary/2gauss}
  }
  \subfigure{%
    \includegraphics[width=.17\columnwidth]{figures/supplementary/2learnt}
  }\\
    \subfigure{%
   \raisebox{2.0em}{
    \includegraphics[width=.06\columnwidth]{figures/supplementary/32bicubic}
   }
  }
  \subfigure{%
    \includegraphics[width=.17\columnwidth]{figures/supplementary/32given_image}
  }
  \subfigure{%
    \includegraphics[width=.17\columnwidth]{figures/supplementary/32ground_truth}
  }
  \subfigure{%
    \includegraphics[width=.17\columnwidth]{figures/supplementary/32bicubic}
  }
  \subfigure{%
    \includegraphics[width=.17\columnwidth]{figures/supplementary/32gauss}
  }
  \subfigure{%
    \includegraphics[width=.17\columnwidth]{figures/supplementary/32learnt}
  }\\
  \setcounter{subfigure}{0}
  \small{
  \subfigure[Inp.]{%
  \raisebox{2.0em}{
    \includegraphics[width=.06\columnwidth]{figures/supplementary/41bicubic}
   }
  }
  \subfigure[Guidance]{%
    \includegraphics[width=.17\columnwidth]{figures/supplementary/41given_image}
  }
   \subfigure[GT]{%
    \includegraphics[width=.17\columnwidth]{figures/supplementary/41ground_truth}
  }
  \subfigure[Bicubic]{%
    \includegraphics[width=.17\columnwidth]{figures/supplementary/41bicubic}
  }
  \subfigure[Gauss-BF]{%
    \includegraphics[width=.17\columnwidth]{figures/supplementary/41gauss}
  }
  \subfigure[Learned-BF]{%
    \includegraphics[width=.17\columnwidth]{figures/supplementary/41learnt}
  }
  }
  \mycaption{Depth Upsampling}{Depth $8\times$ upsampling results
  using different upsampling strategies, from left to right,
  (a)~Low-resolution input depth image (Inp.),
  (b)~High-resolution guidance image, (c)~Ground-truth depth; Upsampled depth images with
  (d)~Bicubic interpolation, (e) Gauss bilateral upsampling and, (f)~Learned bilateral
  updampgling (best viewed on screen).}

\label{fig:depth_upsample_visuals}
\end{figure*}

\subsubsection{Material Segmentation}\label{sec:app_material_segmentation}
\label{sec:material_bnn_extra}

In Fig.~\ref{fig:material_visuals-app2}, we present visual results comparing 2 step
mean-field inference with Gaussian and learned pairwise CRF potentials. In
general, we observe that the pixels belonging to dominant classes in the
training data are being more accurately classified with learned CRF. This leads to
a significant improvements in overall pixel accuracy. This also results
in a slight decrease of the accuracy from less frequent class pixels thereby
slightly reducing the average class accuracy with learning. We attribute this
to the type of annotation that is available for this dataset, which is not
for the entire image but for some segments in the image. We have very few
images of the infrequent classes to combat this behaviour during training.

\subsubsection{Experiment Protocols}
\label{sec:protocols}

Table~\ref{tbl:parameters} shows experiment protocols of different experiments.

 \begin{figure*}[t!]
  \centering
  \subfigure[LeNet-7]{
    \includegraphics[width=0.7\columnwidth]{figures/supplementary/lenet_cnn_network}
    }\\
    \subfigure[DeepCNet]{
    \includegraphics[width=\columnwidth]{figures/supplementary/deepcnet_cnn_network}
    }
  \mycaption{CNNs for Character Recognition}
  {Schematic of (top) LeNet-7~\cite{lecun1998mnist} and (bottom) DeepCNet(5,50)~\cite{ciresan2012multi,graham2014spatially} architectures used in Assamese
  character recognition experiments.}
\label{fig:nnrecognition}
\end{figure*}

\definecolor{voc_1}{RGB}{0, 0, 0}
\definecolor{voc_2}{RGB}{128, 0, 0}
\definecolor{voc_3}{RGB}{0, 128, 0}
\definecolor{voc_4}{RGB}{128, 128, 0}
\definecolor{voc_5}{RGB}{0, 0, 128}
\definecolor{voc_6}{RGB}{128, 0, 128}
\definecolor{voc_7}{RGB}{0, 128, 128}
\definecolor{voc_8}{RGB}{128, 128, 128}
\definecolor{voc_9}{RGB}{64, 0, 0}
\definecolor{voc_10}{RGB}{192, 0, 0}
\definecolor{voc_11}{RGB}{64, 128, 0}
\definecolor{voc_12}{RGB}{192, 128, 0}
\definecolor{voc_13}{RGB}{64, 0, 128}
\definecolor{voc_14}{RGB}{192, 0, 128}
\definecolor{voc_15}{RGB}{64, 128, 128}
\definecolor{voc_16}{RGB}{192, 128, 128}
\definecolor{voc_17}{RGB}{0, 64, 0}
\definecolor{voc_18}{RGB}{128, 64, 0}
\definecolor{voc_19}{RGB}{0, 192, 0}
\definecolor{voc_20}{RGB}{128, 192, 0}
\definecolor{voc_21}{RGB}{0, 64, 128}
\definecolor{voc_22}{RGB}{128, 64, 128}

\begin{figure*}[t]
  \centering
  \small{
  \fcolorbox{white}{voc_1}{\rule{0pt}{6pt}\rule{6pt}{0pt}} Background~~
  \fcolorbox{white}{voc_2}{\rule{0pt}{6pt}\rule{6pt}{0pt}} Aeroplane~~
  \fcolorbox{white}{voc_3}{\rule{0pt}{6pt}\rule{6pt}{0pt}} Bicycle~~
  \fcolorbox{white}{voc_4}{\rule{0pt}{6pt}\rule{6pt}{0pt}} Bird~~
  \fcolorbox{white}{voc_5}{\rule{0pt}{6pt}\rule{6pt}{0pt}} Boat~~
  \fcolorbox{white}{voc_6}{\rule{0pt}{6pt}\rule{6pt}{0pt}} Bottle~~
  \fcolorbox{white}{voc_7}{\rule{0pt}{6pt}\rule{6pt}{0pt}} Bus~~
  \fcolorbox{white}{voc_8}{\rule{0pt}{6pt}\rule{6pt}{0pt}} Car~~ \\
  \fcolorbox{white}{voc_9}{\rule{0pt}{6pt}\rule{6pt}{0pt}} Cat~~
  \fcolorbox{white}{voc_10}{\rule{0pt}{6pt}\rule{6pt}{0pt}} Chair~~
  \fcolorbox{white}{voc_11}{\rule{0pt}{6pt}\rule{6pt}{0pt}} Cow~~
  \fcolorbox{white}{voc_12}{\rule{0pt}{6pt}\rule{6pt}{0pt}} Dining Table~~
  \fcolorbox{white}{voc_13}{\rule{0pt}{6pt}\rule{6pt}{0pt}} Dog~~
  \fcolorbox{white}{voc_14}{\rule{0pt}{6pt}\rule{6pt}{0pt}} Horse~~
  \fcolorbox{white}{voc_15}{\rule{0pt}{6pt}\rule{6pt}{0pt}} Motorbike~~
  \fcolorbox{white}{voc_16}{\rule{0pt}{6pt}\rule{6pt}{0pt}} Person~~ \\
  \fcolorbox{white}{voc_17}{\rule{0pt}{6pt}\rule{6pt}{0pt}} Potted Plant~~
  \fcolorbox{white}{voc_18}{\rule{0pt}{6pt}\rule{6pt}{0pt}} Sheep~~
  \fcolorbox{white}{voc_19}{\rule{0pt}{6pt}\rule{6pt}{0pt}} Sofa~~
  \fcolorbox{white}{voc_20}{\rule{0pt}{6pt}\rule{6pt}{0pt}} Train~~
  \fcolorbox{white}{voc_21}{\rule{0pt}{6pt}\rule{6pt}{0pt}} TV monitor~~ \\
  }
  \subfigure{%
    \includegraphics[width=.18\columnwidth]{figures/supplementary/2007_001423_given.jpg}
  }
  \subfigure{%
    \includegraphics[width=.18\columnwidth]{figures/supplementary/2007_001423_gt.png}
  }
  \subfigure{%
    \includegraphics[width=.18\columnwidth]{figures/supplementary/2007_001423_cnn.png}
  }
  \subfigure{%
    \includegraphics[width=.18\columnwidth]{figures/supplementary/2007_001423_gauss.png}
  }
  \subfigure{%
    \includegraphics[width=.18\columnwidth]{figures/supplementary/2007_001423_learnt.png}
  }\\
  \subfigure{%
    \includegraphics[width=.18\columnwidth]{figures/supplementary/2007_001430_given.jpg}
  }
  \subfigure{%
    \includegraphics[width=.18\columnwidth]{figures/supplementary/2007_001430_gt.png}
  }
  \subfigure{%
    \includegraphics[width=.18\columnwidth]{figures/supplementary/2007_001430_cnn.png}
  }
  \subfigure{%
    \includegraphics[width=.18\columnwidth]{figures/supplementary/2007_001430_gauss.png}
  }
  \subfigure{%
    \includegraphics[width=.18\columnwidth]{figures/supplementary/2007_001430_learnt.png}
  }\\
    \subfigure{%
    \includegraphics[width=.18\columnwidth]{figures/supplementary/2007_007996_given.jpg}
  }
  \subfigure{%
    \includegraphics[width=.18\columnwidth]{figures/supplementary/2007_007996_gt.png}
  }
  \subfigure{%
    \includegraphics[width=.18\columnwidth]{figures/supplementary/2007_007996_cnn.png}
  }
  \subfigure{%
    \includegraphics[width=.18\columnwidth]{figures/supplementary/2007_007996_gauss.png}
  }
  \subfigure{%
    \includegraphics[width=.18\columnwidth]{figures/supplementary/2007_007996_learnt.png}
  }\\
   \subfigure{%
    \includegraphics[width=.18\columnwidth]{figures/supplementary/2010_002682_given.jpg}
  }
  \subfigure{%
    \includegraphics[width=.18\columnwidth]{figures/supplementary/2010_002682_gt.png}
  }
  \subfigure{%
    \includegraphics[width=.18\columnwidth]{figures/supplementary/2010_002682_cnn.png}
  }
  \subfigure{%
    \includegraphics[width=.18\columnwidth]{figures/supplementary/2010_002682_gauss.png}
  }
  \subfigure{%
    \includegraphics[width=.18\columnwidth]{figures/supplementary/2010_002682_learnt.png}
  }\\
     \subfigure{%
    \includegraphics[width=.18\columnwidth]{figures/supplementary/2010_004789_given.jpg}
  }
  \subfigure{%
    \includegraphics[width=.18\columnwidth]{figures/supplementary/2010_004789_gt.png}
  }
  \subfigure{%
    \includegraphics[width=.18\columnwidth]{figures/supplementary/2010_004789_cnn.png}
  }
  \subfigure{%
    \includegraphics[width=.18\columnwidth]{figures/supplementary/2010_004789_gauss.png}
  }
  \subfigure{%
    \includegraphics[width=.18\columnwidth]{figures/supplementary/2010_004789_learnt.png}
  }\\
       \subfigure{%
    \includegraphics[width=.18\columnwidth]{figures/supplementary/2007_001311_given.jpg}
  }
  \subfigure{%
    \includegraphics[width=.18\columnwidth]{figures/supplementary/2007_001311_gt.png}
  }
  \subfigure{%
    \includegraphics[width=.18\columnwidth]{figures/supplementary/2007_001311_cnn.png}
  }
  \subfigure{%
    \includegraphics[width=.18\columnwidth]{figures/supplementary/2007_001311_gauss.png}
  }
  \subfigure{%
    \includegraphics[width=.18\columnwidth]{figures/supplementary/2007_001311_learnt.png}
  }\\
  \setcounter{subfigure}{0}
  \subfigure[Input]{%
    \includegraphics[width=.18\columnwidth]{figures/supplementary/2010_003531_given.jpg}
  }
  \subfigure[Ground Truth]{%
    \includegraphics[width=.18\columnwidth]{figures/supplementary/2010_003531_gt.png}
  }
  \subfigure[DeepLab]{%
    \includegraphics[width=.18\columnwidth]{figures/supplementary/2010_003531_cnn.png}
  }
  \subfigure[+GaussCRF]{%
    \includegraphics[width=.18\columnwidth]{figures/supplementary/2010_003531_gauss.png}
  }
  \subfigure[+LearnedCRF]{%
    \includegraphics[width=.18\columnwidth]{figures/supplementary/2010_003531_learnt.png}
  }
  \vspace{-0.3cm}
  \mycaption{Semantic Segmentation}{Example results of semantic segmentation.
  (c)~depicts the unary results before application of MF, (d)~after two steps of MF with Gaussian edge CRF potentials, (e)~after
  two steps of MF with learned edge CRF potentials.}
    \label{fig:semantic_visuals}
\end{figure*}


\definecolor{minc_1}{HTML}{771111}
\definecolor{minc_2}{HTML}{CAC690}
\definecolor{minc_3}{HTML}{EEEEEE}
\definecolor{minc_4}{HTML}{7C8FA6}
\definecolor{minc_5}{HTML}{597D31}
\definecolor{minc_6}{HTML}{104410}
\definecolor{minc_7}{HTML}{BB819C}
\definecolor{minc_8}{HTML}{D0CE48}
\definecolor{minc_9}{HTML}{622745}
\definecolor{minc_10}{HTML}{666666}
\definecolor{minc_11}{HTML}{D54A31}
\definecolor{minc_12}{HTML}{101044}
\definecolor{minc_13}{HTML}{444126}
\definecolor{minc_14}{HTML}{75D646}
\definecolor{minc_15}{HTML}{DD4348}
\definecolor{minc_16}{HTML}{5C8577}
\definecolor{minc_17}{HTML}{C78472}
\definecolor{minc_18}{HTML}{75D6D0}
\definecolor{minc_19}{HTML}{5B4586}
\definecolor{minc_20}{HTML}{C04393}
\definecolor{minc_21}{HTML}{D69948}
\definecolor{minc_22}{HTML}{7370D8}
\definecolor{minc_23}{HTML}{7A3622}
\definecolor{minc_24}{HTML}{000000}

\begin{figure*}[t]
  \centering
  \small{
  \fcolorbox{white}{minc_1}{\rule{0pt}{6pt}\rule{6pt}{0pt}} Brick~~
  \fcolorbox{white}{minc_2}{\rule{0pt}{6pt}\rule{6pt}{0pt}} Carpet~~
  \fcolorbox{white}{minc_3}{\rule{0pt}{6pt}\rule{6pt}{0pt}} Ceramic~~
  \fcolorbox{white}{minc_4}{\rule{0pt}{6pt}\rule{6pt}{0pt}} Fabric~~
  \fcolorbox{white}{minc_5}{\rule{0pt}{6pt}\rule{6pt}{0pt}} Foliage~~
  \fcolorbox{white}{minc_6}{\rule{0pt}{6pt}\rule{6pt}{0pt}} Food~~
  \fcolorbox{white}{minc_7}{\rule{0pt}{6pt}\rule{6pt}{0pt}} Glass~~
  \fcolorbox{white}{minc_8}{\rule{0pt}{6pt}\rule{6pt}{0pt}} Hair~~ \\
  \fcolorbox{white}{minc_9}{\rule{0pt}{6pt}\rule{6pt}{0pt}} Leather~~
  \fcolorbox{white}{minc_10}{\rule{0pt}{6pt}\rule{6pt}{0pt}} Metal~~
  \fcolorbox{white}{minc_11}{\rule{0pt}{6pt}\rule{6pt}{0pt}} Mirror~~
  \fcolorbox{white}{minc_12}{\rule{0pt}{6pt}\rule{6pt}{0pt}} Other~~
  \fcolorbox{white}{minc_13}{\rule{0pt}{6pt}\rule{6pt}{0pt}} Painted~~
  \fcolorbox{white}{minc_14}{\rule{0pt}{6pt}\rule{6pt}{0pt}} Paper~~
  \fcolorbox{white}{minc_15}{\rule{0pt}{6pt}\rule{6pt}{0pt}} Plastic~~\\
  \fcolorbox{white}{minc_16}{\rule{0pt}{6pt}\rule{6pt}{0pt}} Polished Stone~~
  \fcolorbox{white}{minc_17}{\rule{0pt}{6pt}\rule{6pt}{0pt}} Skin~~
  \fcolorbox{white}{minc_18}{\rule{0pt}{6pt}\rule{6pt}{0pt}} Sky~~
  \fcolorbox{white}{minc_19}{\rule{0pt}{6pt}\rule{6pt}{0pt}} Stone~~
  \fcolorbox{white}{minc_20}{\rule{0pt}{6pt}\rule{6pt}{0pt}} Tile~~
  \fcolorbox{white}{minc_21}{\rule{0pt}{6pt}\rule{6pt}{0pt}} Wallpaper~~
  \fcolorbox{white}{minc_22}{\rule{0pt}{6pt}\rule{6pt}{0pt}} Water~~
  \fcolorbox{white}{minc_23}{\rule{0pt}{6pt}\rule{6pt}{0pt}} Wood~~ \\
  }
  \subfigure{%
    \includegraphics[width=.18\columnwidth]{figures/supplementary/000010868_given.jpg}
  }
  \subfigure{%
    \includegraphics[width=.18\columnwidth]{figures/supplementary/000010868_gt.png}
  }
  \subfigure{%
    \includegraphics[width=.18\columnwidth]{figures/supplementary/000010868_cnn.png}
  }
  \subfigure{%
    \includegraphics[width=.18\columnwidth]{figures/supplementary/000010868_gauss.png}
  }
  \subfigure{%
    \includegraphics[width=.18\columnwidth]{figures/supplementary/000010868_learnt.png}
  }\\[-2ex]
  \subfigure{%
    \includegraphics[width=.18\columnwidth]{figures/supplementary/000006011_given.jpg}
  }
  \subfigure{%
    \includegraphics[width=.18\columnwidth]{figures/supplementary/000006011_gt.png}
  }
  \subfigure{%
    \includegraphics[width=.18\columnwidth]{figures/supplementary/000006011_cnn.png}
  }
  \subfigure{%
    \includegraphics[width=.18\columnwidth]{figures/supplementary/000006011_gauss.png}
  }
  \subfigure{%
    \includegraphics[width=.18\columnwidth]{figures/supplementary/000006011_learnt.png}
  }\\[-2ex]
    \subfigure{%
    \includegraphics[width=.18\columnwidth]{figures/supplementary/000008553_given.jpg}
  }
  \subfigure{%
    \includegraphics[width=.18\columnwidth]{figures/supplementary/000008553_gt.png}
  }
  \subfigure{%
    \includegraphics[width=.18\columnwidth]{figures/supplementary/000008553_cnn.png}
  }
  \subfigure{%
    \includegraphics[width=.18\columnwidth]{figures/supplementary/000008553_gauss.png}
  }
  \subfigure{%
    \includegraphics[width=.18\columnwidth]{figures/supplementary/000008553_learnt.png}
  }\\[-2ex]
   \subfigure{%
    \includegraphics[width=.18\columnwidth]{figures/supplementary/000009188_given.jpg}
  }
  \subfigure{%
    \includegraphics[width=.18\columnwidth]{figures/supplementary/000009188_gt.png}
  }
  \subfigure{%
    \includegraphics[width=.18\columnwidth]{figures/supplementary/000009188_cnn.png}
  }
  \subfigure{%
    \includegraphics[width=.18\columnwidth]{figures/supplementary/000009188_gauss.png}
  }
  \subfigure{%
    \includegraphics[width=.18\columnwidth]{figures/supplementary/000009188_learnt.png}
  }\\[-2ex]
  \setcounter{subfigure}{0}
  \subfigure[Input]{%
    \includegraphics[width=.18\columnwidth]{figures/supplementary/000023570_given.jpg}
  }
  \subfigure[Ground Truth]{%
    \includegraphics[width=.18\columnwidth]{figures/supplementary/000023570_gt.png}
  }
  \subfigure[DeepLab]{%
    \includegraphics[width=.18\columnwidth]{figures/supplementary/000023570_cnn.png}
  }
  \subfigure[+GaussCRF]{%
    \includegraphics[width=.18\columnwidth]{figures/supplementary/000023570_gauss.png}
  }
  \subfigure[+LearnedCRF]{%
    \includegraphics[width=.18\columnwidth]{figures/supplementary/000023570_learnt.png}
  }
  \mycaption{Material Segmentation}{Example results of material segmentation.
  (c)~depicts the unary results before application of MF, (d)~after two steps of MF with Gaussian edge CRF potentials, (e)~after two steps of MF with learned edge CRF potentials.}
    \label{fig:material_visuals-app2}
\end{figure*}


\begin{table*}[h]
\tiny
  \centering
    \begin{tabular}{L{2.3cm} L{2.25cm} C{1.5cm} C{0.7cm} C{0.6cm} C{0.7cm} C{0.7cm} C{0.7cm} C{1.6cm} C{0.6cm} C{0.6cm} C{0.6cm}}
      \toprule
& & & & & \multicolumn{3}{c}{\textbf{Data Statistics}} & \multicolumn{4}{c}{\textbf{Training Protocol}} \\

\textbf{Experiment} & \textbf{Feature Types} & \textbf{Feature Scales} & \textbf{Filter Size} & \textbf{Filter Nbr.} & \textbf{Train}  & \textbf{Val.} & \textbf{Test} & \textbf{Loss Type} & \textbf{LR} & \textbf{Batch} & \textbf{Epochs} \\
      \midrule
      \multicolumn{2}{c}{\textbf{Single Bilateral Filter Applications}} & & & & & & & & & \\
      \textbf{2$\times$ Color Upsampling} & Position$_{1}$, Intensity (3D) & 0.13, 0.17 & 65 & 2 & 10581 & 1449 & 1456 & MSE & 1e-06 & 200 & 94.5\\
      \textbf{4$\times$ Color Upsampling} & Position$_{1}$, Intensity (3D) & 0.06, 0.17 & 65 & 2 & 10581 & 1449 & 1456 & MSE & 1e-06 & 200 & 94.5\\
      \textbf{8$\times$ Color Upsampling} & Position$_{1}$, Intensity (3D) & 0.03, 0.17 & 65 & 2 & 10581 & 1449 & 1456 & MSE & 1e-06 & 200 & 94.5\\
      \textbf{16$\times$ Color Upsampling} & Position$_{1}$, Intensity (3D) & 0.02, 0.17 & 65 & 2 & 10581 & 1449 & 1456 & MSE & 1e-06 & 200 & 94.5\\
      \textbf{Depth Upsampling} & Position$_{1}$, Color (5D) & 0.05, 0.02 & 665 & 2 & 795 & 100 & 654 & MSE & 1e-07 & 50 & 251.6\\
      \textbf{Mesh Denoising} & Isomap (4D) & 46.00 & 63 & 2 & 1000 & 200 & 500 & MSE & 100 & 10 & 100.0 \\
      \midrule
      \multicolumn{2}{c}{\textbf{DenseCRF Applications}} & & & & & & & & &\\
      \multicolumn{2}{l}{\textbf{Semantic Segmentation}} & & & & & & & & &\\
      \textbf{- 1step MF} & Position$_{1}$, Color (5D); Position$_{1}$ (2D) & 0.01, 0.34; 0.34  & 665; 19  & 2; 2 & 10581 & 1449 & 1456 & Logistic & 0.1 & 5 & 1.4 \\
      \textbf{- 2step MF} & Position$_{1}$, Color (5D); Position$_{1}$ (2D) & 0.01, 0.34; 0.34 & 665; 19 & 2; 2 & 10581 & 1449 & 1456 & Logistic & 0.1 & 5 & 1.4 \\
      \textbf{- \textit{loose} 2step MF} & Position$_{1}$, Color (5D); Position$_{1}$ (2D) & 0.01, 0.34; 0.34 & 665; 19 & 2; 2 &10581 & 1449 & 1456 & Logistic & 0.1 & 5 & +1.9  \\ \\
      \multicolumn{2}{l}{\textbf{Material Segmentation}} & & & & & & & & &\\
      \textbf{- 1step MF} & Position$_{2}$, Lab-Color (5D) & 5.00, 0.05, 0.30  & 665 & 2 & 928 & 150 & 1798 & Weighted Logistic & 1e-04 & 24 & 2.6 \\
      \textbf{- 2step MF} & Position$_{2}$, Lab-Color (5D) & 5.00, 0.05, 0.30 & 665 & 2 & 928 & 150 & 1798 & Weighted Logistic & 1e-04 & 12 & +0.7 \\
      \textbf{- \textit{loose} 2step MF} & Position$_{2}$, Lab-Color (5D) & 5.00, 0.05, 0.30 & 665 & 2 & 928 & 150 & 1798 & Weighted Logistic & 1e-04 & 12 & +0.2\\
      \midrule
      \multicolumn{2}{c}{\textbf{Neural Network Applications}} & & & & & & & & &\\
      \textbf{Tiles: CNN-9$\times$9} & - & - & 81 & 4 & 10000 & 1000 & 1000 & Logistic & 0.01 & 100 & 500.0 \\
      \textbf{Tiles: CNN-13$\times$13} & - & - & 169 & 6 & 10000 & 1000 & 1000 & Logistic & 0.01 & 100 & 500.0 \\
      \textbf{Tiles: CNN-17$\times$17} & - & - & 289 & 8 & 10000 & 1000 & 1000 & Logistic & 0.01 & 100 & 500.0 \\
      \textbf{Tiles: CNN-21$\times$21} & - & - & 441 & 10 & 10000 & 1000 & 1000 & Logistic & 0.01 & 100 & 500.0 \\
      \textbf{Tiles: BNN} & Position$_{1}$, Color (5D) & 0.05, 0.04 & 63 & 1 & 10000 & 1000 & 1000 & Logistic & 0.01 & 100 & 30.0 \\
      \textbf{LeNet} & - & - & 25 & 2 & 5490 & 1098 & 1647 & Logistic & 0.1 & 100 & 182.2 \\
      \textbf{Crop-LeNet} & - & - & 25 & 2 & 5490 & 1098 & 1647 & Logistic & 0.1 & 100 & 182.2 \\
      \textbf{BNN-LeNet} & Position$_{2}$ (2D) & 20.00 & 7 & 1 & 5490 & 1098 & 1647 & Logistic & 0.1 & 100 & 182.2 \\
      \textbf{DeepCNet} & - & - & 9 & 1 & 5490 & 1098 & 1647 & Logistic & 0.1 & 100 & 182.2 \\
      \textbf{Crop-DeepCNet} & - & - & 9 & 1 & 5490 & 1098 & 1647 & Logistic & 0.1 & 100 & 182.2 \\
      \textbf{BNN-DeepCNet} & Position$_{2}$ (2D) & 40.00  & 7 & 1 & 5490 & 1098 & 1647 & Logistic & 0.1 & 100 & 182.2 \\
      \bottomrule
      \\
    \end{tabular}
    \mycaption{Experiment Protocols} {Experiment protocols for the different experiments presented in this work. \textbf{Feature Types}:
    Feature spaces used for the bilateral convolutions. Position$_1$ corresponds to un-normalized pixel positions whereas Position$_2$ corresponds
    to pixel positions normalized to $[0,1]$ with respect to the given image. \textbf{Feature Scales}: Cross-validated scales for the features used.
     \textbf{Filter Size}: Number of elements in the filter that is being learned. \textbf{Filter Nbr.}: Half-width of the filter. \textbf{Train},
     \textbf{Val.} and \textbf{Test} corresponds to the number of train, validation and test images used in the experiment. \textbf{Loss Type}: Type
     of loss used for back-propagation. ``MSE'' corresponds to Euclidean mean squared error loss and ``Logistic'' corresponds to multinomial logistic
     loss. ``Weighted Logistic'' is the class-weighted multinomial logistic loss. We weighted the loss with inverse class probability for material
     segmentation task due to the small availability of training data with class imbalance. \textbf{LR}: Fixed learning rate used in stochastic gradient
     descent. \textbf{Batch}: Number of images used in one parameter update step. \textbf{Epochs}: Number of training epochs. In all the experiments,
     we used fixed momentum of 0.9 and weight decay of 0.0005 for stochastic gradient descent. ```Color Upsampling'' experiments in this Table corresponds
     to those performed on Pascal VOC12 dataset images. For all experiments using Pascal VOC12 images, we use extended
     training segmentation dataset available from~\cite{hariharan2011moredata}, and used standard validation and test splits
     from the main dataset~\cite{voc2012segmentation}.}
  \label{tbl:parameters}
\end{table*}

\clearpage

\section{Parameters and Additional Results for Video Propagation Networks}

In this Section, we present experiment protocols and additional qualitative results for experiments
on video object segmentation, semantic video segmentation and video color
propagation. Table~\ref{tbl:parameters_supp} shows the feature scales and other parameters used in different experiments.
Figures~\ref{fig:video_seg_pos_supp} show some qualitative results on video object segmentation
with some failure cases in Fig.~\ref{fig:video_seg_neg_supp}.
Figure~\ref{fig:semantic_visuals_supp} shows some qualitative results on semantic video segmentation and
Fig.~\ref{fig:color_visuals_supp} shows results on video color propagation.

\newcolumntype{L}[1]{>{\raggedright\let\newline\\\arraybackslash\hspace{0pt}}b{#1}}
\newcolumntype{C}[1]{>{\centering\let\newline\\\arraybackslash\hspace{0pt}}b{#1}}
\newcolumntype{R}[1]{>{\raggedleft\let\newline\\\arraybackslash\hspace{0pt}}b{#1}}

\begin{table*}[h]
\tiny
  \centering
    \begin{tabular}{L{3.0cm} L{2.4cm} L{2.8cm} L{2.8cm} C{0.5cm} C{1.0cm} L{1.2cm}}
      \toprule
\textbf{Experiment} & \textbf{Feature Type} & \textbf{Feature Scale-1, $\Lambda_a$} & \textbf{Feature Scale-2, $\Lambda_b$} & \textbf{$\alpha$} & \textbf{Input Frames} & \textbf{Loss Type} \\
      \midrule
      \textbf{Video Object Segmentation} & ($x,y,Y,Cb,Cr,t$) & (0.02,0.02,0.07,0.4,0.4,0.01) & (0.03,0.03,0.09,0.5,0.5,0.2) & 0.5 & 9 & Logistic\\
      \midrule
      \textbf{Semantic Video Segmentation} & & & & & \\
      \textbf{with CNN1~\cite{yu2015multi}-NoFlow} & ($x,y,R,G,B,t$) & (0.08,0.08,0.2,0.2,0.2,0.04) & (0.11,0.11,0.2,0.2,0.2,0.04) & 0.5 & 3 & Logistic \\
      \textbf{with CNN1~\cite{yu2015multi}-Flow} & ($x+u_x,y+u_y,R,G,B,t$) & (0.11,0.11,0.14,0.14,0.14,0.03) & (0.08,0.08,0.12,0.12,0.12,0.01) & 0.65 & 3 & Logistic\\
      \textbf{with CNN2~\cite{richter2016playing}-Flow} & ($x+u_x,y+u_y,R,G,B,t$) & (0.08,0.08,0.2,0.2,0.2,0.04) & (0.09,0.09,0.25,0.25,0.25,0.03) & 0.5 & 4 & Logistic\\
      \midrule
      \textbf{Video Color Propagation} & ($x,y,I,t$)  & (0.04,0.04,0.2,0.04) & No second kernel & 1 & 4 & MSE\\
      \bottomrule
      \\
    \end{tabular}
    \mycaption{Experiment Protocols} {Experiment protocols for the different experiments presented in this work. \textbf{Feature Types}:
    Feature spaces used for the bilateral convolutions, with position ($x,y$) and color
    ($R,G,B$ or $Y,Cb,Cr$) features $\in [0,255]$. $u_x$, $u_y$ denotes optical flow with respect
    to the present frame and $I$ denotes grayscale intensity.
    \textbf{Feature Scales ($\Lambda_a, \Lambda_b$)}: Cross-validated scales for the features used.
    \textbf{$\alpha$}: Exponential time decay for the input frames.
    \textbf{Input Frames}: Number of input frames for VPN.
    \textbf{Loss Type}: Type
     of loss used for back-propagation. ``MSE'' corresponds to Euclidean mean squared error loss and ``Logistic'' corresponds to multinomial logistic loss.}
  \label{tbl:parameters_supp}
\end{table*}

% \begin{figure}[th!]
% \begin{center}
%   \centerline{\includegraphics[width=\textwidth]{figures/video_seg_visuals_supp_small.pdf}}
%     \mycaption{Video Object Segmentation}
%     {Shown are the different frames in example videos with the corresponding
%     ground truth (GT) masks, predictions from BVS~\cite{marki2016bilateral},
%     OFL~\cite{tsaivideo}, VPN (VPN-Stage2) and VPN-DLab (VPN-DeepLab) models.}
%     \label{fig:video_seg_small_supp}
% \end{center}
% \vspace{-1.0cm}
% \end{figure}

\begin{figure}[th!]
\begin{center}
  \centerline{\includegraphics[width=0.7\textwidth]{figures/video_seg_visuals_supp_positive.pdf}}
    \mycaption{Video Object Segmentation}
    {Shown are the different frames in example videos with the corresponding
    ground truth (GT) masks, predictions from BVS~\cite{marki2016bilateral},
    OFL~\cite{tsaivideo}, VPN (VPN-Stage2) and VPN-DLab (VPN-DeepLab) models.}
    \label{fig:video_seg_pos_supp}
\end{center}
\vspace{-1.0cm}
\end{figure}

\begin{figure}[th!]
\begin{center}
  \centerline{\includegraphics[width=0.7\textwidth]{figures/video_seg_visuals_supp_negative.pdf}}
    \mycaption{Failure Cases for Video Object Segmentation}
    {Shown are the different frames in example videos with the corresponding
    ground truth (GT) masks, predictions from BVS~\cite{marki2016bilateral},
    OFL~\cite{tsaivideo}, VPN (VPN-Stage2) and VPN-DLab (VPN-DeepLab) models.}
    \label{fig:video_seg_neg_supp}
\end{center}
\vspace{-1.0cm}
\end{figure}

\begin{figure}[th!]
\begin{center}
  \centerline{\includegraphics[width=0.9\textwidth]{figures/supp_semantic_visual.pdf}}
    \mycaption{Semantic Video Segmentation}
    {Input video frames and the corresponding ground truth (GT)
    segmentation together with the predictions of CNN~\cite{yu2015multi} and with
    VPN-Flow.}
    \label{fig:semantic_visuals_supp}
\end{center}
\vspace{-0.7cm}
\end{figure}

\begin{figure}[th!]
\begin{center}
  \centerline{\includegraphics[width=\textwidth]{figures/colorization_visuals_supp.pdf}}
  \mycaption{Video Color Propagation}
  {Input grayscale video frames and corresponding ground-truth (GT) color images
  together with color predictions of Levin et al.~\cite{levin2004colorization} and VPN-Stage1 models.}
  \label{fig:color_visuals_supp}
\end{center}
\vspace{-0.7cm}
\end{figure}

\clearpage

\section{Additional Material for Bilateral Inception Networks}
\label{sec:binception-app}

In this section of the Appendix, we first discuss the use of approximate bilateral
filtering in BI modules (Sec.~\ref{sec:lattice}).
Later, we present some qualitative results using different models for the approach presented in
Chapter~\ref{chap:binception} (Sec.~\ref{sec:qualitative-app}).

\subsection{Approximate Bilateral Filtering}
\label{sec:lattice}

The bilateral inception module presented in Chapter~\ref{chap:binception} computes a matrix-vector
product between a Gaussian filter $K$ and a vector of activations $\bz_c$.
Bilateral filtering is an important operation and many algorithmic techniques have been
proposed to speed-up this operation~\cite{paris2006fast,adams2010fast,gastal2011domain}.
In the main paper we opted to implement what can be considered the
brute-force variant of explicitly constructing $K$ and then using BLAS to compute the
matrix-vector product. This resulted in a few millisecond operation.
The explicit way to compute is possible due to the
reduction to super-pixels, e.g., it would not work for DenseCRF variants
that operate on the full image resolution.

Here, we present experiments where we use the fast approximate bilateral filtering
algorithm of~\cite{adams2010fast}, which is also used in Chapter~\ref{chap:bnn}
for learning sparse high dimensional filters. This
choice allows for larger dimensions of matrix-vector multiplication. The reason for choosing
the explicit multiplication in Chapter~\ref{chap:binception} was that it was computationally faster.
For the small sizes of the involved matrices and vectors, the explicit computation is sufficient and we had no
GPU implementation of an approximate technique that matched this runtime. Also it
is conceptually easier and the gradient to the feature transformations ($\Lambda \mathbf{f}$) is
obtained using standard matrix calculus.

\subsubsection{Experiments}

We modified the existing segmentation architectures analogous to those in Chapter~\ref{chap:binception}.
The main difference is that, here, the inception modules use the lattice
approximation~\cite{adams2010fast} to compute the bilateral filtering.
Using the lattice approximation did not allow us to back-propagate through feature transformations ($\Lambda$)
and thus we used hand-specified feature scales as will be explained later.
Specifically, we take CNN architectures from the works
of~\cite{chen2014semantic,zheng2015conditional,bell2015minc} and insert the BI modules between
the spatial FC layers.
We use superpixels from~\cite{DollarICCV13edges}
for all the experiments with the lattice approximation. Experiments are
performed using Caffe neural network framework~\cite{jia2014caffe}.

\begin{table}
  \small
  \centering
  \begin{tabular}{p{5.5cm}>{\raggedright\arraybackslash}p{1.4cm}>{\centering\arraybackslash}p{2.2cm}}
    \toprule
		\textbf{Model} & \emph{IoU} & \emph{Runtime}(ms) \\
    \midrule

    %%%%%%%%%%%% Scores computed by us)%%%%%%%%%%%%
		\deeplablargefov & 68.9 & 145ms\\
    \midrule
    \bi{7}{2}-\bi{8}{10}& \textbf{73.8} & +600 \\
    \midrule
    \deeplablargefovcrf~\cite{chen2014semantic} & 72.7 & +830\\
    \deeplabmsclargefovcrf~\cite{chen2014semantic} & \textbf{73.6} & +880\\
    DeepLab-EdgeNet~\cite{chen2015semantic} & 71.7 & +30\\
    DeepLab-EdgeNet-CRF~\cite{chen2015semantic} & \textbf{73.6} & +860\\
  \bottomrule \\
  \end{tabular}
  \mycaption{Semantic Segmentation using the DeepLab model}
  {IoU scores on the Pascal VOC12 segmentation test dataset
  with different models and our modified inception model.
  Also shown are the corresponding runtimes in milliseconds. Runtimes
  also include superpixel computations (300 ms with Dollar superpixels~\cite{DollarICCV13edges})}
  \label{tab:largefovresults}
\end{table}

\paragraph{Semantic Segmentation}
The experiments in this section use the Pascal VOC12 segmentation dataset~\cite{voc2012segmentation} with 21 object classes and the images have a maximum resolution of 0.25 megapixels.
For all experiments on VOC12, we train using the extended training set of
10581 images collected by~\cite{hariharan2011moredata}.
We modified the \deeplab~network architecture of~\cite{chen2014semantic} and
the CRFasRNN architecture from~\cite{zheng2015conditional} which uses a CNN with
deconvolution layers followed by DenseCRF trained end-to-end.

\paragraph{DeepLab Model}\label{sec:deeplabmodel}
We experimented with the \bi{7}{2}-\bi{8}{10} inception model.
Results using the~\deeplab~model are summarized in Tab.~\ref{tab:largefovresults}.
Although we get similar improvements with inception modules as with the
explicit kernel computation, using lattice approximation is slower.

\begin{table}
  \small
  \centering
  \begin{tabular}{p{6.4cm}>{\raggedright\arraybackslash}p{1.8cm}>{\raggedright\arraybackslash}p{1.8cm}}
    \toprule
    \textbf{Model} & \emph{IoU (Val)} & \emph{IoU (Test)}\\
    \midrule
    %%%%%%%%%%%% Scores computed by us)%%%%%%%%%%%%
    CNN &  67.5 & - \\
    \deconv (CNN+Deconvolutions) & 69.8 & 72.0 \\
    \midrule
    \bi{3}{6}-\bi{4}{6}-\bi{7}{2}-\bi{8}{6}& 71.9 & - \\
    \bi{3}{6}-\bi{4}{6}-\bi{7}{2}-\bi{8}{6}-\gi{6}& 73.6 &  \href{http://host.robots.ox.ac.uk:8080/anonymous/VOTV5E.html}{\textbf{75.2}}\\
    \midrule
    \deconvcrf (CRF-RNN)~\cite{zheng2015conditional} & 73.0 & 74.7\\
    Context-CRF-RNN~\cite{yu2015multi} & ~~ - ~ & \textbf{75.3} \\
    \bottomrule \\
  \end{tabular}
  \mycaption{Semantic Segmentation using the CRFasRNN model}{IoU score corresponding to different models
  on Pascal VOC12 reduced validation / test segmentation dataset. The reduced validation set consists of 346 images
  as used in~\cite{zheng2015conditional} where we adapted the model from.}
  \label{tab:deconvresults-app}
\end{table}

\paragraph{CRFasRNN Model}\label{sec:deepinception}
We add BI modules after score-pool3, score-pool4, \fc{7} and \fc{8} $1\times1$ convolution layers
resulting in the \bi{3}{6}-\bi{4}{6}-\bi{7}{2}-\bi{8}{6}
model and also experimented with another variant where $BI_8$ is followed by another inception
module, G$(6)$, with 6 Gaussian kernels.
Note that here also we discarded both deconvolution and DenseCRF parts of the original model~\cite{zheng2015conditional}
and inserted the BI modules in the base CNN and found similar improvements compared to the inception modules with explicit
kernel computaion. See Tab.~\ref{tab:deconvresults-app} for results on the CRFasRNN model.

\paragraph{Material Segmentation}
Table~\ref{tab:mincresults-app} shows the results on the MINC dataset~\cite{bell2015minc}
obtained by modifying the AlexNet architecture with our inception modules. We observe
similar improvements as with explicit kernel construction.
For this model, we do not provide any learned setup due to very limited segment training
data. The weights to combine outputs in the bilateral inception layer are
found by validation on the validation set.

\begin{table}[t]
  \small
  \centering
  \begin{tabular}{p{3.5cm}>{\centering\arraybackslash}p{4.0cm}}
    \toprule
    \textbf{Model} & Class / Total accuracy\\
    \midrule

    %%%%%%%%%%%% Scores computed by us)%%%%%%%%%%%%
    AlexNet CNN & 55.3 / 58.9 \\
    \midrule
    \bi{7}{2}-\bi{8}{6}& 68.5 / 71.8 \\
    \bi{7}{2}-\bi{8}{6}-G$(6)$& 67.6 / 73.1 \\
    \midrule
    AlexNet-CRF & 65.5 / 71.0 \\
    \bottomrule \\
  \end{tabular}
  \mycaption{Material Segmentation using AlexNet}{Pixel accuracy of different models on
  the MINC material segmentation test dataset~\cite{bell2015minc}.}
  \label{tab:mincresults-app}
\end{table}

\paragraph{Scales of Bilateral Inception Modules}
\label{sec:scales}

Unlike the explicit kernel technique presented in the main text (Chapter~\ref{chap:binception}),
we didn't back-propagate through feature transformation ($\Lambda$)
using the approximate bilateral filter technique.
So, the feature scales are hand-specified and validated, which are as follows.
The optimal scale values for the \bi{7}{2}-\bi{8}{2} model are found by validation for the best performance which are
$\sigma_{xy}$ = (0.1, 0.1) for the spatial (XY) kernel and $\sigma_{rgbxy}$ = (0.1, 0.1, 0.1, 0.01, 0.01) for color and position (RGBXY)  kernel.
Next, as more kernels are added to \bi{8}{2}, we set scales to be $\alpha$*($\sigma_{xy}$, $\sigma_{rgbxy}$).
The value of $\alpha$ is chosen as  1, 0.5, 0.1, 0.05, 0.1, at uniform interval, for the \bi{8}{10} bilateral inception module.


\subsection{Qualitative Results}
\label{sec:qualitative-app}

In this section, we present more qualitative results obtained using the BI module with explicit
kernel computation technique presented in Chapter~\ref{chap:binception}. Results on the Pascal VOC12
dataset~\cite{voc2012segmentation} using the DeepLab-LargeFOV model are shown in Fig.~\ref{fig:semantic_visuals-app},
followed by the results on MINC dataset~\cite{bell2015minc}
in Fig.~\ref{fig:material_visuals-app} and on
Cityscapes dataset~\cite{Cordts2015Cvprw} in Fig.~\ref{fig:street_visuals-app}.


\definecolor{voc_1}{RGB}{0, 0, 0}
\definecolor{voc_2}{RGB}{128, 0, 0}
\definecolor{voc_3}{RGB}{0, 128, 0}
\definecolor{voc_4}{RGB}{128, 128, 0}
\definecolor{voc_5}{RGB}{0, 0, 128}
\definecolor{voc_6}{RGB}{128, 0, 128}
\definecolor{voc_7}{RGB}{0, 128, 128}
\definecolor{voc_8}{RGB}{128, 128, 128}
\definecolor{voc_9}{RGB}{64, 0, 0}
\definecolor{voc_10}{RGB}{192, 0, 0}
\definecolor{voc_11}{RGB}{64, 128, 0}
\definecolor{voc_12}{RGB}{192, 128, 0}
\definecolor{voc_13}{RGB}{64, 0, 128}
\definecolor{voc_14}{RGB}{192, 0, 128}
\definecolor{voc_15}{RGB}{64, 128, 128}
\definecolor{voc_16}{RGB}{192, 128, 128}
\definecolor{voc_17}{RGB}{0, 64, 0}
\definecolor{voc_18}{RGB}{128, 64, 0}
\definecolor{voc_19}{RGB}{0, 192, 0}
\definecolor{voc_20}{RGB}{128, 192, 0}
\definecolor{voc_21}{RGB}{0, 64, 128}
\definecolor{voc_22}{RGB}{128, 64, 128}

\begin{figure*}[!ht]
  \small
  \centering
  \fcolorbox{white}{voc_1}{\rule{0pt}{4pt}\rule{4pt}{0pt}} Background~~
  \fcolorbox{white}{voc_2}{\rule{0pt}{4pt}\rule{4pt}{0pt}} Aeroplane~~
  \fcolorbox{white}{voc_3}{\rule{0pt}{4pt}\rule{4pt}{0pt}} Bicycle~~
  \fcolorbox{white}{voc_4}{\rule{0pt}{4pt}\rule{4pt}{0pt}} Bird~~
  \fcolorbox{white}{voc_5}{\rule{0pt}{4pt}\rule{4pt}{0pt}} Boat~~
  \fcolorbox{white}{voc_6}{\rule{0pt}{4pt}\rule{4pt}{0pt}} Bottle~~
  \fcolorbox{white}{voc_7}{\rule{0pt}{4pt}\rule{4pt}{0pt}} Bus~~
  \fcolorbox{white}{voc_8}{\rule{0pt}{4pt}\rule{4pt}{0pt}} Car~~\\
  \fcolorbox{white}{voc_9}{\rule{0pt}{4pt}\rule{4pt}{0pt}} Cat~~
  \fcolorbox{white}{voc_10}{\rule{0pt}{4pt}\rule{4pt}{0pt}} Chair~~
  \fcolorbox{white}{voc_11}{\rule{0pt}{4pt}\rule{4pt}{0pt}} Cow~~
  \fcolorbox{white}{voc_12}{\rule{0pt}{4pt}\rule{4pt}{0pt}} Dining Table~~
  \fcolorbox{white}{voc_13}{\rule{0pt}{4pt}\rule{4pt}{0pt}} Dog~~
  \fcolorbox{white}{voc_14}{\rule{0pt}{4pt}\rule{4pt}{0pt}} Horse~~
  \fcolorbox{white}{voc_15}{\rule{0pt}{4pt}\rule{4pt}{0pt}} Motorbike~~
  \fcolorbox{white}{voc_16}{\rule{0pt}{4pt}\rule{4pt}{0pt}} Person~~\\
  \fcolorbox{white}{voc_17}{\rule{0pt}{4pt}\rule{4pt}{0pt}} Potted Plant~~
  \fcolorbox{white}{voc_18}{\rule{0pt}{4pt}\rule{4pt}{0pt}} Sheep~~
  \fcolorbox{white}{voc_19}{\rule{0pt}{4pt}\rule{4pt}{0pt}} Sofa~~
  \fcolorbox{white}{voc_20}{\rule{0pt}{4pt}\rule{4pt}{0pt}} Train~~
  \fcolorbox{white}{voc_21}{\rule{0pt}{4pt}\rule{4pt}{0pt}} TV monitor~~\\


  \subfigure{%
    \includegraphics[width=.15\columnwidth]{figures/supplementary/2008_001308_given.png}
  }
  \subfigure{%
    \includegraphics[width=.15\columnwidth]{figures/supplementary/2008_001308_sp.png}
  }
  \subfigure{%
    \includegraphics[width=.15\columnwidth]{figures/supplementary/2008_001308_gt.png}
  }
  \subfigure{%
    \includegraphics[width=.15\columnwidth]{figures/supplementary/2008_001308_cnn.png}
  }
  \subfigure{%
    \includegraphics[width=.15\columnwidth]{figures/supplementary/2008_001308_crf.png}
  }
  \subfigure{%
    \includegraphics[width=.15\columnwidth]{figures/supplementary/2008_001308_ours.png}
  }\\[-2ex]


  \subfigure{%
    \includegraphics[width=.15\columnwidth]{figures/supplementary/2008_001821_given.png}
  }
  \subfigure{%
    \includegraphics[width=.15\columnwidth]{figures/supplementary/2008_001821_sp.png}
  }
  \subfigure{%
    \includegraphics[width=.15\columnwidth]{figures/supplementary/2008_001821_gt.png}
  }
  \subfigure{%
    \includegraphics[width=.15\columnwidth]{figures/supplementary/2008_001821_cnn.png}
  }
  \subfigure{%
    \includegraphics[width=.15\columnwidth]{figures/supplementary/2008_001821_crf.png}
  }
  \subfigure{%
    \includegraphics[width=.15\columnwidth]{figures/supplementary/2008_001821_ours.png}
  }\\[-2ex]



  \subfigure{%
    \includegraphics[width=.15\columnwidth]{figures/supplementary/2008_004612_given.png}
  }
  \subfigure{%
    \includegraphics[width=.15\columnwidth]{figures/supplementary/2008_004612_sp.png}
  }
  \subfigure{%
    \includegraphics[width=.15\columnwidth]{figures/supplementary/2008_004612_gt.png}
  }
  \subfigure{%
    \includegraphics[width=.15\columnwidth]{figures/supplementary/2008_004612_cnn.png}
  }
  \subfigure{%
    \includegraphics[width=.15\columnwidth]{figures/supplementary/2008_004612_crf.png}
  }
  \subfigure{%
    \includegraphics[width=.15\columnwidth]{figures/supplementary/2008_004612_ours.png}
  }\\[-2ex]


  \subfigure{%
    \includegraphics[width=.15\columnwidth]{figures/supplementary/2009_001008_given.png}
  }
  \subfigure{%
    \includegraphics[width=.15\columnwidth]{figures/supplementary/2009_001008_sp.png}
  }
  \subfigure{%
    \includegraphics[width=.15\columnwidth]{figures/supplementary/2009_001008_gt.png}
  }
  \subfigure{%
    \includegraphics[width=.15\columnwidth]{figures/supplementary/2009_001008_cnn.png}
  }
  \subfigure{%
    \includegraphics[width=.15\columnwidth]{figures/supplementary/2009_001008_crf.png}
  }
  \subfigure{%
    \includegraphics[width=.15\columnwidth]{figures/supplementary/2009_001008_ours.png}
  }\\[-2ex]




  \subfigure{%
    \includegraphics[width=.15\columnwidth]{figures/supplementary/2009_004497_given.png}
  }
  \subfigure{%
    \includegraphics[width=.15\columnwidth]{figures/supplementary/2009_004497_sp.png}
  }
  \subfigure{%
    \includegraphics[width=.15\columnwidth]{figures/supplementary/2009_004497_gt.png}
  }
  \subfigure{%
    \includegraphics[width=.15\columnwidth]{figures/supplementary/2009_004497_cnn.png}
  }
  \subfigure{%
    \includegraphics[width=.15\columnwidth]{figures/supplementary/2009_004497_crf.png}
  }
  \subfigure{%
    \includegraphics[width=.15\columnwidth]{figures/supplementary/2009_004497_ours.png}
  }\\[-2ex]



  \setcounter{subfigure}{0}
  \subfigure[\scriptsize Input]{%
    \includegraphics[width=.15\columnwidth]{figures/supplementary/2010_001327_given.png}
  }
  \subfigure[\scriptsize Superpixels]{%
    \includegraphics[width=.15\columnwidth]{figures/supplementary/2010_001327_sp.png}
  }
  \subfigure[\scriptsize GT]{%
    \includegraphics[width=.15\columnwidth]{figures/supplementary/2010_001327_gt.png}
  }
  \subfigure[\scriptsize Deeplab]{%
    \includegraphics[width=.15\columnwidth]{figures/supplementary/2010_001327_cnn.png}
  }
  \subfigure[\scriptsize +DenseCRF]{%
    \includegraphics[width=.15\columnwidth]{figures/supplementary/2010_001327_crf.png}
  }
  \subfigure[\scriptsize Using BI]{%
    \includegraphics[width=.15\columnwidth]{figures/supplementary/2010_001327_ours.png}
  }
  \mycaption{Semantic Segmentation}{Example results of semantic segmentation
  on the Pascal VOC12 dataset.
  (d)~depicts the DeepLab CNN result, (e)~CNN + 10 steps of mean-field inference,
  (f~result obtained with bilateral inception (BI) modules (\bi{6}{2}+\bi{7}{6}) between \fc~layers.}
  \label{fig:semantic_visuals-app}
\end{figure*}


\definecolor{minc_1}{HTML}{771111}
\definecolor{minc_2}{HTML}{CAC690}
\definecolor{minc_3}{HTML}{EEEEEE}
\definecolor{minc_4}{HTML}{7C8FA6}
\definecolor{minc_5}{HTML}{597D31}
\definecolor{minc_6}{HTML}{104410}
\definecolor{minc_7}{HTML}{BB819C}
\definecolor{minc_8}{HTML}{D0CE48}
\definecolor{minc_9}{HTML}{622745}
\definecolor{minc_10}{HTML}{666666}
\definecolor{minc_11}{HTML}{D54A31}
\definecolor{minc_12}{HTML}{101044}
\definecolor{minc_13}{HTML}{444126}
\definecolor{minc_14}{HTML}{75D646}
\definecolor{minc_15}{HTML}{DD4348}
\definecolor{minc_16}{HTML}{5C8577}
\definecolor{minc_17}{HTML}{C78472}
\definecolor{minc_18}{HTML}{75D6D0}
\definecolor{minc_19}{HTML}{5B4586}
\definecolor{minc_20}{HTML}{C04393}
\definecolor{minc_21}{HTML}{D69948}
\definecolor{minc_22}{HTML}{7370D8}
\definecolor{minc_23}{HTML}{7A3622}
\definecolor{minc_24}{HTML}{000000}

\begin{figure*}[!ht]
  \small % scriptsize
  \centering
  \fcolorbox{white}{minc_1}{\rule{0pt}{4pt}\rule{4pt}{0pt}} Brick~~
  \fcolorbox{white}{minc_2}{\rule{0pt}{4pt}\rule{4pt}{0pt}} Carpet~~
  \fcolorbox{white}{minc_3}{\rule{0pt}{4pt}\rule{4pt}{0pt}} Ceramic~~
  \fcolorbox{white}{minc_4}{\rule{0pt}{4pt}\rule{4pt}{0pt}} Fabric~~
  \fcolorbox{white}{minc_5}{\rule{0pt}{4pt}\rule{4pt}{0pt}} Foliage~~
  \fcolorbox{white}{minc_6}{\rule{0pt}{4pt}\rule{4pt}{0pt}} Food~~
  \fcolorbox{white}{minc_7}{\rule{0pt}{4pt}\rule{4pt}{0pt}} Glass~~
  \fcolorbox{white}{minc_8}{\rule{0pt}{4pt}\rule{4pt}{0pt}} Hair~~\\
  \fcolorbox{white}{minc_9}{\rule{0pt}{4pt}\rule{4pt}{0pt}} Leather~~
  \fcolorbox{white}{minc_10}{\rule{0pt}{4pt}\rule{4pt}{0pt}} Metal~~
  \fcolorbox{white}{minc_11}{\rule{0pt}{4pt}\rule{4pt}{0pt}} Mirror~~
  \fcolorbox{white}{minc_12}{\rule{0pt}{4pt}\rule{4pt}{0pt}} Other~~
  \fcolorbox{white}{minc_13}{\rule{0pt}{4pt}\rule{4pt}{0pt}} Painted~~
  \fcolorbox{white}{minc_14}{\rule{0pt}{4pt}\rule{4pt}{0pt}} Paper~~
  \fcolorbox{white}{minc_15}{\rule{0pt}{4pt}\rule{4pt}{0pt}} Plastic~~\\
  \fcolorbox{white}{minc_16}{\rule{0pt}{4pt}\rule{4pt}{0pt}} Polished Stone~~
  \fcolorbox{white}{minc_17}{\rule{0pt}{4pt}\rule{4pt}{0pt}} Skin~~
  \fcolorbox{white}{minc_18}{\rule{0pt}{4pt}\rule{4pt}{0pt}} Sky~~
  \fcolorbox{white}{minc_19}{\rule{0pt}{4pt}\rule{4pt}{0pt}} Stone~~
  \fcolorbox{white}{minc_20}{\rule{0pt}{4pt}\rule{4pt}{0pt}} Tile~~
  \fcolorbox{white}{minc_21}{\rule{0pt}{4pt}\rule{4pt}{0pt}} Wallpaper~~
  \fcolorbox{white}{minc_22}{\rule{0pt}{4pt}\rule{4pt}{0pt}} Water~~
  \fcolorbox{white}{minc_23}{\rule{0pt}{4pt}\rule{4pt}{0pt}} Wood~~\\
  \subfigure{%
    \includegraphics[width=.15\columnwidth]{figures/supplementary/000008468_given.png}
  }
  \subfigure{%
    \includegraphics[width=.15\columnwidth]{figures/supplementary/000008468_sp.png}
  }
  \subfigure{%
    \includegraphics[width=.15\columnwidth]{figures/supplementary/000008468_gt.png}
  }
  \subfigure{%
    \includegraphics[width=.15\columnwidth]{figures/supplementary/000008468_cnn.png}
  }
  \subfigure{%
    \includegraphics[width=.15\columnwidth]{figures/supplementary/000008468_crf.png}
  }
  \subfigure{%
    \includegraphics[width=.15\columnwidth]{figures/supplementary/000008468_ours.png}
  }\\[-2ex]

  \subfigure{%
    \includegraphics[width=.15\columnwidth]{figures/supplementary/000009053_given.png}
  }
  \subfigure{%
    \includegraphics[width=.15\columnwidth]{figures/supplementary/000009053_sp.png}
  }
  \subfigure{%
    \includegraphics[width=.15\columnwidth]{figures/supplementary/000009053_gt.png}
  }
  \subfigure{%
    \includegraphics[width=.15\columnwidth]{figures/supplementary/000009053_cnn.png}
  }
  \subfigure{%
    \includegraphics[width=.15\columnwidth]{figures/supplementary/000009053_crf.png}
  }
  \subfigure{%
    \includegraphics[width=.15\columnwidth]{figures/supplementary/000009053_ours.png}
  }\\[-2ex]




  \subfigure{%
    \includegraphics[width=.15\columnwidth]{figures/supplementary/000014977_given.png}
  }
  \subfigure{%
    \includegraphics[width=.15\columnwidth]{figures/supplementary/000014977_sp.png}
  }
  \subfigure{%
    \includegraphics[width=.15\columnwidth]{figures/supplementary/000014977_gt.png}
  }
  \subfigure{%
    \includegraphics[width=.15\columnwidth]{figures/supplementary/000014977_cnn.png}
  }
  \subfigure{%
    \includegraphics[width=.15\columnwidth]{figures/supplementary/000014977_crf.png}
  }
  \subfigure{%
    \includegraphics[width=.15\columnwidth]{figures/supplementary/000014977_ours.png}
  }\\[-2ex]


  \subfigure{%
    \includegraphics[width=.15\columnwidth]{figures/supplementary/000022922_given.png}
  }
  \subfigure{%
    \includegraphics[width=.15\columnwidth]{figures/supplementary/000022922_sp.png}
  }
  \subfigure{%
    \includegraphics[width=.15\columnwidth]{figures/supplementary/000022922_gt.png}
  }
  \subfigure{%
    \includegraphics[width=.15\columnwidth]{figures/supplementary/000022922_cnn.png}
  }
  \subfigure{%
    \includegraphics[width=.15\columnwidth]{figures/supplementary/000022922_crf.png}
  }
  \subfigure{%
    \includegraphics[width=.15\columnwidth]{figures/supplementary/000022922_ours.png}
  }\\[-2ex]


  \subfigure{%
    \includegraphics[width=.15\columnwidth]{figures/supplementary/000025711_given.png}
  }
  \subfigure{%
    \includegraphics[width=.15\columnwidth]{figures/supplementary/000025711_sp.png}
  }
  \subfigure{%
    \includegraphics[width=.15\columnwidth]{figures/supplementary/000025711_gt.png}
  }
  \subfigure{%
    \includegraphics[width=.15\columnwidth]{figures/supplementary/000025711_cnn.png}
  }
  \subfigure{%
    \includegraphics[width=.15\columnwidth]{figures/supplementary/000025711_crf.png}
  }
  \subfigure{%
    \includegraphics[width=.15\columnwidth]{figures/supplementary/000025711_ours.png}
  }\\[-2ex]


  \subfigure{%
    \includegraphics[width=.15\columnwidth]{figures/supplementary/000034473_given.png}
  }
  \subfigure{%
    \includegraphics[width=.15\columnwidth]{figures/supplementary/000034473_sp.png}
  }
  \subfigure{%
    \includegraphics[width=.15\columnwidth]{figures/supplementary/000034473_gt.png}
  }
  \subfigure{%
    \includegraphics[width=.15\columnwidth]{figures/supplementary/000034473_cnn.png}
  }
  \subfigure{%
    \includegraphics[width=.15\columnwidth]{figures/supplementary/000034473_crf.png}
  }
  \subfigure{%
    \includegraphics[width=.15\columnwidth]{figures/supplementary/000034473_ours.png}
  }\\[-2ex]


  \subfigure{%
    \includegraphics[width=.15\columnwidth]{figures/supplementary/000035463_given.png}
  }
  \subfigure{%
    \includegraphics[width=.15\columnwidth]{figures/supplementary/000035463_sp.png}
  }
  \subfigure{%
    \includegraphics[width=.15\columnwidth]{figures/supplementary/000035463_gt.png}
  }
  \subfigure{%
    \includegraphics[width=.15\columnwidth]{figures/supplementary/000035463_cnn.png}
  }
  \subfigure{%
    \includegraphics[width=.15\columnwidth]{figures/supplementary/000035463_crf.png}
  }
  \subfigure{%
    \includegraphics[width=.15\columnwidth]{figures/supplementary/000035463_ours.png}
  }\\[-2ex]


  \setcounter{subfigure}{0}
  \subfigure[\scriptsize Input]{%
    \includegraphics[width=.15\columnwidth]{figures/supplementary/000035993_given.png}
  }
  \subfigure[\scriptsize Superpixels]{%
    \includegraphics[width=.15\columnwidth]{figures/supplementary/000035993_sp.png}
  }
  \subfigure[\scriptsize GT]{%
    \includegraphics[width=.15\columnwidth]{figures/supplementary/000035993_gt.png}
  }
  \subfigure[\scriptsize AlexNet]{%
    \includegraphics[width=.15\columnwidth]{figures/supplementary/000035993_cnn.png}
  }
  \subfigure[\scriptsize +DenseCRF]{%
    \includegraphics[width=.15\columnwidth]{figures/supplementary/000035993_crf.png}
  }
  \subfigure[\scriptsize Using BI]{%
    \includegraphics[width=.15\columnwidth]{figures/supplementary/000035993_ours.png}
  }
  \mycaption{Material Segmentation}{Example results of material segmentation.
  (d)~depicts the AlexNet CNN result, (e)~CNN + 10 steps of mean-field inference,
  (f)~result obtained with bilateral inception (BI) modules (\bi{7}{2}+\bi{8}{6}) between
  \fc~layers.}
\label{fig:material_visuals-app}
\end{figure*}


\definecolor{city_1}{RGB}{128, 64, 128}
\definecolor{city_2}{RGB}{244, 35, 232}
\definecolor{city_3}{RGB}{70, 70, 70}
\definecolor{city_4}{RGB}{102, 102, 156}
\definecolor{city_5}{RGB}{190, 153, 153}
\definecolor{city_6}{RGB}{153, 153, 153}
\definecolor{city_7}{RGB}{250, 170, 30}
\definecolor{city_8}{RGB}{220, 220, 0}
\definecolor{city_9}{RGB}{107, 142, 35}
\definecolor{city_10}{RGB}{152, 251, 152}
\definecolor{city_11}{RGB}{70, 130, 180}
\definecolor{city_12}{RGB}{220, 20, 60}
\definecolor{city_13}{RGB}{255, 0, 0}
\definecolor{city_14}{RGB}{0, 0, 142}
\definecolor{city_15}{RGB}{0, 0, 70}
\definecolor{city_16}{RGB}{0, 60, 100}
\definecolor{city_17}{RGB}{0, 80, 100}
\definecolor{city_18}{RGB}{0, 0, 230}
\definecolor{city_19}{RGB}{119, 11, 32}
\begin{figure*}[!ht]
  \small % scriptsize
  \centering


  \subfigure{%
    \includegraphics[width=.18\columnwidth]{figures/supplementary/frankfurt00000_016005_given.png}
  }
  \subfigure{%
    \includegraphics[width=.18\columnwidth]{figures/supplementary/frankfurt00000_016005_sp.png}
  }
  \subfigure{%
    \includegraphics[width=.18\columnwidth]{figures/supplementary/frankfurt00000_016005_gt.png}
  }
  \subfigure{%
    \includegraphics[width=.18\columnwidth]{figures/supplementary/frankfurt00000_016005_cnn.png}
  }
  \subfigure{%
    \includegraphics[width=.18\columnwidth]{figures/supplementary/frankfurt00000_016005_ours.png}
  }\\[-2ex]

  \subfigure{%
    \includegraphics[width=.18\columnwidth]{figures/supplementary/frankfurt00000_004617_given.png}
  }
  \subfigure{%
    \includegraphics[width=.18\columnwidth]{figures/supplementary/frankfurt00000_004617_sp.png}
  }
  \subfigure{%
    \includegraphics[width=.18\columnwidth]{figures/supplementary/frankfurt00000_004617_gt.png}
  }
  \subfigure{%
    \includegraphics[width=.18\columnwidth]{figures/supplementary/frankfurt00000_004617_cnn.png}
  }
  \subfigure{%
    \includegraphics[width=.18\columnwidth]{figures/supplementary/frankfurt00000_004617_ours.png}
  }\\[-2ex]

  \subfigure{%
    \includegraphics[width=.18\columnwidth]{figures/supplementary/frankfurt00000_020880_given.png}
  }
  \subfigure{%
    \includegraphics[width=.18\columnwidth]{figures/supplementary/frankfurt00000_020880_sp.png}
  }
  \subfigure{%
    \includegraphics[width=.18\columnwidth]{figures/supplementary/frankfurt00000_020880_gt.png}
  }
  \subfigure{%
    \includegraphics[width=.18\columnwidth]{figures/supplementary/frankfurt00000_020880_cnn.png}
  }
  \subfigure{%
    \includegraphics[width=.18\columnwidth]{figures/supplementary/frankfurt00000_020880_ours.png}
  }\\[-2ex]



  \subfigure{%
    \includegraphics[width=.18\columnwidth]{figures/supplementary/frankfurt00001_007285_given.png}
  }
  \subfigure{%
    \includegraphics[width=.18\columnwidth]{figures/supplementary/frankfurt00001_007285_sp.png}
  }
  \subfigure{%
    \includegraphics[width=.18\columnwidth]{figures/supplementary/frankfurt00001_007285_gt.png}
  }
  \subfigure{%
    \includegraphics[width=.18\columnwidth]{figures/supplementary/frankfurt00001_007285_cnn.png}
  }
  \subfigure{%
    \includegraphics[width=.18\columnwidth]{figures/supplementary/frankfurt00001_007285_ours.png}
  }\\[-2ex]


  \subfigure{%
    \includegraphics[width=.18\columnwidth]{figures/supplementary/frankfurt00001_059789_given.png}
  }
  \subfigure{%
    \includegraphics[width=.18\columnwidth]{figures/supplementary/frankfurt00001_059789_sp.png}
  }
  \subfigure{%
    \includegraphics[width=.18\columnwidth]{figures/supplementary/frankfurt00001_059789_gt.png}
  }
  \subfigure{%
    \includegraphics[width=.18\columnwidth]{figures/supplementary/frankfurt00001_059789_cnn.png}
  }
  \subfigure{%
    \includegraphics[width=.18\columnwidth]{figures/supplementary/frankfurt00001_059789_ours.png}
  }\\[-2ex]


  \subfigure{%
    \includegraphics[width=.18\columnwidth]{figures/supplementary/frankfurt00001_068208_given.png}
  }
  \subfigure{%
    \includegraphics[width=.18\columnwidth]{figures/supplementary/frankfurt00001_068208_sp.png}
  }
  \subfigure{%
    \includegraphics[width=.18\columnwidth]{figures/supplementary/frankfurt00001_068208_gt.png}
  }
  \subfigure{%
    \includegraphics[width=.18\columnwidth]{figures/supplementary/frankfurt00001_068208_cnn.png}
  }
  \subfigure{%
    \includegraphics[width=.18\columnwidth]{figures/supplementary/frankfurt00001_068208_ours.png}
  }\\[-2ex]

  \subfigure{%
    \includegraphics[width=.18\columnwidth]{figures/supplementary/frankfurt00001_082466_given.png}
  }
  \subfigure{%
    \includegraphics[width=.18\columnwidth]{figures/supplementary/frankfurt00001_082466_sp.png}
  }
  \subfigure{%
    \includegraphics[width=.18\columnwidth]{figures/supplementary/frankfurt00001_082466_gt.png}
  }
  \subfigure{%
    \includegraphics[width=.18\columnwidth]{figures/supplementary/frankfurt00001_082466_cnn.png}
  }
  \subfigure{%
    \includegraphics[width=.18\columnwidth]{figures/supplementary/frankfurt00001_082466_ours.png}
  }\\[-2ex]

  \subfigure{%
    \includegraphics[width=.18\columnwidth]{figures/supplementary/lindau00033_000019_given.png}
  }
  \subfigure{%
    \includegraphics[width=.18\columnwidth]{figures/supplementary/lindau00033_000019_sp.png}
  }
  \subfigure{%
    \includegraphics[width=.18\columnwidth]{figures/supplementary/lindau00033_000019_gt.png}
  }
  \subfigure{%
    \includegraphics[width=.18\columnwidth]{figures/supplementary/lindau00033_000019_cnn.png}
  }
  \subfigure{%
    \includegraphics[width=.18\columnwidth]{figures/supplementary/lindau00033_000019_ours.png}
  }\\[-2ex]

  \subfigure{%
    \includegraphics[width=.18\columnwidth]{figures/supplementary/lindau00052_000019_given.png}
  }
  \subfigure{%
    \includegraphics[width=.18\columnwidth]{figures/supplementary/lindau00052_000019_sp.png}
  }
  \subfigure{%
    \includegraphics[width=.18\columnwidth]{figures/supplementary/lindau00052_000019_gt.png}
  }
  \subfigure{%
    \includegraphics[width=.18\columnwidth]{figures/supplementary/lindau00052_000019_cnn.png}
  }
  \subfigure{%
    \includegraphics[width=.18\columnwidth]{figures/supplementary/lindau00052_000019_ours.png}
  }\\[-2ex]




  \subfigure{%
    \includegraphics[width=.18\columnwidth]{figures/supplementary/lindau00027_000019_given.png}
  }
  \subfigure{%
    \includegraphics[width=.18\columnwidth]{figures/supplementary/lindau00027_000019_sp.png}
  }
  \subfigure{%
    \includegraphics[width=.18\columnwidth]{figures/supplementary/lindau00027_000019_gt.png}
  }
  \subfigure{%
    \includegraphics[width=.18\columnwidth]{figures/supplementary/lindau00027_000019_cnn.png}
  }
  \subfigure{%
    \includegraphics[width=.18\columnwidth]{figures/supplementary/lindau00027_000019_ours.png}
  }\\[-2ex]



  \setcounter{subfigure}{0}
  \subfigure[\scriptsize Input]{%
    \includegraphics[width=.18\columnwidth]{figures/supplementary/lindau00029_000019_given.png}
  }
  \subfigure[\scriptsize Superpixels]{%
    \includegraphics[width=.18\columnwidth]{figures/supplementary/lindau00029_000019_sp.png}
  }
  \subfigure[\scriptsize GT]{%
    \includegraphics[width=.18\columnwidth]{figures/supplementary/lindau00029_000019_gt.png}
  }
  \subfigure[\scriptsize Deeplab]{%
    \includegraphics[width=.18\columnwidth]{figures/supplementary/lindau00029_000019_cnn.png}
  }
  \subfigure[\scriptsize Using BI]{%
    \includegraphics[width=.18\columnwidth]{figures/supplementary/lindau00029_000019_ours.png}
  }%\\[-2ex]

  \mycaption{Street Scene Segmentation}{Example results of street scene segmentation.
  (d)~depicts the DeepLab results, (e)~result obtained by adding bilateral inception (BI) modules (\bi{6}{2}+\bi{7}{6}) between \fc~layers.}
\label{fig:street_visuals-app}
\end{figure*}



\end{document}


