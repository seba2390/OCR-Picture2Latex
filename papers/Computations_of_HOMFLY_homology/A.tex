\newgeometry{top=2.5cm, bottom=2.5cm, left=4cm, right=4cm} 

\appendix
\section{Results}\label{sec:results}

The following results are computed by the program \cite{kr-calc}, with the braid representatives of the \textit{KnotInfo Database} \cite{knotinfo} as the input data. Beware that, compared with the Rolfsen knot table, there are ambiguities with respect to mirrors.

The tables here are read off as follows. The horizontal (resp.\ vertical) axis corresponds to the degree $j$ (resp.\ $k$). Each monomial $q^i$ written at the position $(j,k)$ indicates a generator of the homology (as a vector space) with triple grading $(i,j,k)$. Again, the Poincar\'e polynomial $\mathcal{P}(K)$ of the homology $\Hbar(K)$ is given by
\[
    \mathcal{P}(K)(q, a, t) = \sum_{i,j,k} q^ia^jt^{(k-j)/2}\dim \Hbar^{i,\,j,\,k}(K).
\]
For example, the Poincar\'e polynomial of $3_1$ is $q^{-2}a^2+q^2a^2t^2+a^4t^3$.
We note that we are omitting empty rows and columns to save spaces.

\vspace{3em}

\begin{multicols}{2}
    \setlength\parindent{0pt}
    \begin{minipage}{\linewidth}
$\bullet\ $ $3_{1}$ \vspace{0.5em} \\
\begin{tabular}{l|ll}
$k \setminus j$ & $2$ & $4$ \\
\hline
$2$ & $q^{-2}$ &  \\
$-2$ & $q^{2}$ & $1$ \\
\end{tabular}
\vspace{2em}
\end{minipage}
%
\begin{minipage}{\linewidth}
$\bullet\ $ $4_{1}$ \vspace{0.5em} \\
\begin{tabular}{l|lll}
$k \setminus j$ & $-2$ & $0$ & $2$ \\
\hline
$2$ & $1$ & $q^{-2}$ &  \\
$0$ &  & $1$ &  \\
$-2$ &  & $q^{2}$ & $1$ \\
\end{tabular}
\vspace{2em}
\end{minipage}
%
\begin{minipage}{\linewidth}
$\bullet\ $ $5_{1}$ \vspace{0.5em} \\
\begin{tabular}{l|ll}
$k \setminus j$ & $4$ & $6$ \\
\hline
$4$ & $q^{-4}$ &  \\
$0$ & $1$ & $q^{-2}$ \\
$-4$ & $q^{4}$ & $q^{2}$ \\
\end{tabular}
\vspace{2em}
\end{minipage}
%
\begin{minipage}{\linewidth}
$\bullet\ $ $5_{2}$ \vspace{0.5em} \\
\begin{tabular}{l|lll}
$k \setminus j$ & $2$ & $4$ & $6$ \\
\hline
$2$ & $q^{-2}$ &  &  \\
$0$ & $1$ & $q^{-2}$ &  \\
$-2$ & $q^{2}$ & $1$ &  \\
$-4$ &  & $q^{2}$ & $1$ \\
\end{tabular}
\vspace{2em}
\end{minipage}
%
\begin{minipage}{\linewidth}
$\bullet\ $ $6_{1}$ \vspace{0.5em} \\
\begin{tabular}{l|llll}
$k \setminus j$ & $-2$ & $0$ & $2$ & $4$ \\
\hline
$2$ & $1$ & $q^{-2}$ &  &  \\
$0$ &  & $2$ & $q^{-2}$ &  \\
$-2$ &  & $q^{2}$ & $1$ &  \\
$-4$ &  &  & $q^{2}$ & $1$ \\
\end{tabular}
\vspace{2em}
\end{minipage}
%
\begin{minipage}{\linewidth}
$\bullet\ $ $6_{2}$ \vspace{0.5em} \\
\begin{tabular}{l|lll}
$k \setminus j$ & $0$ & $2$ & $4$ \\
\hline
$4$ & $q^{-2}$ & $q^{-4}$ &  \\
$2$ &  & $q^{-2}$ &  \\
$0$ & $q^{2}$ & $2$ & $q^{-2}$ \\
$-2$ &  & $q^{2}$ & $1$ \\
$-4$ &  & $q^{4}$ & $q^{2}$ \\
\end{tabular}
\vspace{2em}
\end{minipage}
%
\begin{minipage}{\linewidth}
$\bullet\ $ $6_{3}$ \vspace{0.5em} \\
\begin{tabular}{l|lll}
$k \setminus j$ & $-2$ & $0$ & $2$ \\
\hline
$4$ & $q^{-2}$ & $q^{-4}$ &  \\
$2$ & $1$ & $q^{-2}$ &  \\
$0$ & $q^{2}$ & $3$ & $q^{-2}$ \\
$-2$ &  & $q^{2}$ & $1$ \\
$-4$ &  & $q^{4}$ & $q^{2}$ \\
\end{tabular}
\vspace{2em}
\end{minipage}
%
\begin{minipage}{\linewidth}
$\bullet\ $ $7_{1}$ \vspace{0.5em} \\
\begin{tabular}{l|ll}
$k \setminus j$ & $6$ & $8$ \\
\hline
$6$ & $q^{-6}$ &  \\
$2$ & $q^{-2}$ & $q^{-4}$ \\
$-2$ & $q^{2}$ & $1$ \\
$-6$ & $q^{6}$ & $q^{4}$ \\
\end{tabular}
\vspace{2em}
\end{minipage}
%
\begin{minipage}{\linewidth}
$\bullet\ $ $7_{2}$ \vspace{0.5em} \\
\begin{tabular}{l|llll}
$k \setminus j$ & $2$ & $4$ & $6$ & $8$ \\
\hline
$2$ & $q^{-2}$ &  &  &  \\
$0$ & $1$ & $q^{-2}$ &  &  \\
$-2$ & $q^{2}$ & $2$ & $q^{-2}$ &  \\
$-4$ &  & $q^{2}$ & $1$ &  \\
$-6$ &  &  & $q^{2}$ & $1$ \\
\end{tabular}
\vspace{2em}
\end{minipage}
%
\begin{minipage}{\linewidth}
$\bullet\ $ $7_{3}$ \vspace{0.5em} \\
\begin{tabular}{l|lll}
$k \setminus j$ & $4$ & $6$ & $8$ \\
\hline
$4$ & $q^{-4}$ &  &  \\
$2$ & $q^{-2}$ & $q^{-4}$ &  \\
$0$ & $1$ & $q^{-2}$ &  \\
$-2$ & $q^{2}$ & $2$ & $q^{-2}$ \\
$-4$ & $q^{4}$ & $q^{2}$ &  \\
$-6$ &  & $q^{4}$ & $q^{2}$ \\
\end{tabular}
\vspace{2em}
\end{minipage}
%
\begin{minipage}{\linewidth}
$\bullet\ $ $7_{4}$ \vspace{0.5em} \\
\begin{tabular}{l|llll}
$k \setminus j$ & $2$ & $4$ & $6$ & $8$ \\
\hline
$2$ & $q^{-2}$ &  &  &  \\
$0$ & $2$ & $2q^{-2}$ &  &  \\
$-2$ & $q^{2}$ & $2$ & $q^{-2}$ &  \\
$-4$ &  & $2q^{2}$ & $2$ &  \\
$-6$ &  &  & $q^{2}$ & $1$ \\
\end{tabular}
\vspace{2em}
\end{minipage}
%
\begin{minipage}{\linewidth}
$\bullet\ $ $7_{5}$ \vspace{0.5em} \\
\begin{tabular}{l|lll}
$k \setminus j$ & $4$ & $6$ & $8$ \\
\hline
$4$ & $q^{-4}$ &  &  \\
$2$ & $q^{-2}$ & $q^{-4}$ &  \\
$0$ & $2$ & $2q^{-2}$ &  \\
$-2$ & $q^{2}$ & $2$ & $q^{-2}$ \\
$-4$ & $q^{4}$ & $2q^{2}$ & $1$ \\
$-6$ &  & $q^{4}$ & $q^{2}$ \\
\end{tabular}
\vspace{2em}
\end{minipage}
%
\begin{minipage}{\linewidth}
$\bullet\ $ $7_{6}$ \vspace{0.5em} \\
\begin{tabular}{l|llll}
$k \setminus j$ & $0$ & $2$ & $4$ & $6$ \\
\hline
$4$ & $q^{-2}$ & $q^{-4}$ &  &  \\
$2$ & $1$ & $2q^{-2}$ &  &  \\
$0$ & $q^{2}$ & $3$ & $2q^{-2}$ &  \\
$-2$ &  & $2q^{2}$ & $2$ &  \\
$-4$ &  & $q^{4}$ & $2q^{2}$ & $1$ \\
\end{tabular}
\vspace{2em}
\end{minipage}
%
\begin{minipage}{\linewidth}
$\bullet\ $ $7_{7}$ \vspace{0.5em} \\
\begin{tabular}{l|llll}
$k \setminus j$ & $-4$ & $-2$ & $0$ & $2$ \\
\hline
$4$ & $1$ & $2q^{-2}$ & $q^{-4}$ &  \\
$2$ &  & $2$ & $2q^{-2}$ &  \\
$0$ &  & $2q^{2}$ & $4$ & $q^{-2}$ \\
$-2$ &  &  & $2q^{2}$ & $2$ \\
$-4$ &  &  & $q^{4}$ & $q^{2}$ \\
\end{tabular}
\vspace{2em}
\end{minipage}
%
\begin{minipage}{\linewidth}
$\bullet\ $ $8_{1}$ \vspace{0.5em} \\
\begin{tabular}{l|lllll}
$k \setminus j$ & $-2$ & $0$ & $2$ & $4$ & $6$ \\
\hline
$2$ & $1$ & $q^{-2}$ &  &  &  \\
$0$ &  & $2$ & $q^{-2}$ &  &  \\
$-2$ &  & $q^{2}$ & $2$ & $q^{-2}$ &  \\
$-4$ &  &  & $q^{2}$ & $1$ &  \\
$-6$ &  &  &  & $q^{2}$ & $1$ \\
\end{tabular}
\vspace{2em}
\end{minipage}
%
\begin{minipage}{\linewidth}
$\bullet\ $ $8_{2}$ \vspace{0.5em} \\
\begin{tabular}{l|lll}
$k \setminus j$ & $2$ & $4$ & $6$ \\
\hline
$6$ & $q^{-4}$ & $q^{-6}$ &  \\
$4$ &  & $q^{-4}$ &  \\
$2$ & $1$ & $2q^{-2}$ & $q^{-4}$ \\
$0$ &  & $1$ & $q^{-2}$ \\
$-2$ & $q^{4}$ & $2q^{2}$ & $1$ \\
$-4$ &  & $q^{4}$ & $q^{2}$ \\
$-6$ &  & $q^{6}$ & $q^{4}$ \\
\end{tabular}
\vspace{2em}
\end{minipage}
%
\begin{minipage}{\linewidth}
$\bullet\ $ $8_{3}$ \vspace{0.5em} \\
\begin{tabular}{l|lllll}
$k \setminus j$ & $-4$ & $-2$ & $0$ & $2$ & $4$ \\
\hline
$4$ & $1$ & $q^{-2}$ &  &  &  \\
$2$ &  & $2$ & $2q^{-2}$ &  &  \\
$0$ &  & $q^{2}$ & $3$ & $q^{-2}$ &  \\
$-2$ &  &  & $2q^{2}$ & $2$ &  \\
$-4$ &  &  &  & $q^{2}$ & $1$ \\
\end{tabular}
\vspace{2em}
\end{minipage}
%
\begin{minipage}{\linewidth}
$\bullet\ $ $8_{4}$ \vspace{0.5em} \\
\begin{tabular}{l|llll}
$k \setminus j$ & $-4$ & $-2$ & $0$ & $2$ \\
\hline
$4$ & $q^{-2}$ & $q^{-4}$ &  &  \\
$2$ & $1$ & $2q^{-2}$ & $q^{-4}$ &  \\
$0$ & $q^{2}$ & $2$ & $q^{-2}$ &  \\
$-2$ &  & $2q^{2}$ & $2$ & $q^{-2}$ \\
$-4$ &  & $q^{4}$ & $q^{2}$ &  \\
$-6$ &  &  & $q^{4}$ & $q^{2}$ \\
\end{tabular}
\vspace{2em}
\end{minipage}
%
\begin{minipage}{\linewidth}
$\bullet\ $ $8_{5}$ \vspace{0.5em} \\
\begin{tabular}{l|lll}
$k \setminus j$ & $2$ & $4$ & $6$ \\
\hline
$6$ & $q^{-4}$ & $q^{-6}$ &  \\
$4$ &  & $q^{-4}$ &  \\
$2$ & $2$ & $3q^{-2}$ & $q^{-4}$ \\
$0$ &  & $1$ & $q^{-2}$ \\
$-2$ & $q^{4}$ & $3q^{2}$ & $2$ \\
$-4$ &  & $q^{4}$ & $q^{2}$ \\
$-6$ &  & $q^{6}$ & $q^{4}$ \\
\end{tabular}
\vspace{2em}
\end{minipage}
%
\begin{minipage}{\linewidth}
$\bullet\ $ $8_{6}$ \vspace{0.5em} \\
\begin{tabular}{l|llll}
$k \setminus j$ & $0$ & $2$ & $4$ & $6$ \\
\hline
$4$ & $q^{-2}$ & $q^{-4}$ &  &  \\
$2$ &  & $2q^{-2}$ & $q^{-4}$ &  \\
$0$ & $q^{2}$ & $3$ & $2q^{-2}$ &  \\
$-2$ &  & $2q^{2}$ & $3$ & $q^{-2}$ \\
$-4$ &  & $q^{4}$ & $2q^{2}$ & $1$ \\
$-6$ &  &  & $q^{4}$ & $q^{2}$ \\
\end{tabular}
\vspace{2em}
\end{minipage}
%
\begin{minipage}{\linewidth}
$\bullet\ $ $8_{7}$ \vspace{0.5em} \\
\begin{tabular}{l|lll}
$k \setminus j$ & $-4$ & $-2$ & $0$ \\
\hline
$6$ & $q^{-4}$ & $q^{-6}$ &  \\
$4$ & $q^{-2}$ & $q^{-4}$ &  \\
$2$ & $2$ & $3q^{-2}$ & $q^{-4}$ \\
$0$ & $q^{2}$ & $2$ & $q^{-2}$ \\
$-2$ & $q^{4}$ & $3q^{2}$ & $1$ \\
$-4$ &  & $q^{4}$ & $q^{2}$ \\
$-6$ &  & $q^{6}$ & $q^{4}$ \\
\end{tabular}
\vspace{2em}
\end{minipage}
%
\begin{minipage}{\linewidth}
$\bullet\ $ $8_{8}$ \vspace{0.5em} \\
\begin{tabular}{l|llll}
$k \setminus j$ & $-4$ & $-2$ & $0$ & $2$ \\
\hline
$6$ & $q^{-2}$ & $q^{-4}$ &  &  \\
$4$ & $1$ & $2q^{-2}$ & $q^{-4}$ &  \\
$2$ & $q^{2}$ & $3$ & $2q^{-2}$ &  \\
$0$ &  & $2q^{2}$ & $4$ & $q^{-2}$ \\
$-2$ &  & $q^{4}$ & $2q^{2}$ & $1$ \\
$-4$ &  &  & $q^{4}$ & $q^{2}$ \\
\end{tabular}
\vspace{2em}
\end{minipage}
%
\begin{minipage}{\linewidth}
$\bullet\ $ $8_{9}$ \vspace{0.5em} \\
\begin{tabular}{l|lll}
$k \setminus j$ & $-2$ & $0$ & $2$ \\
\hline
$6$ & $q^{-4}$ & $q^{-6}$ &  \\
$4$ & $q^{-2}$ & $q^{-4}$ &  \\
$2$ & $2$ & $3q^{-2}$ & $q^{-4}$ \\
$0$ & $q^{2}$ & $3$ & $q^{-2}$ \\
$-2$ & $q^{4}$ & $3q^{2}$ & $2$ \\
$-4$ &  & $q^{4}$ & $q^{2}$ \\
$-6$ &  & $q^{6}$ & $q^{4}$ \\
\end{tabular}
\vspace{2em}
\end{minipage}
%
\begin{minipage}{\linewidth}
$\bullet\ $ $8_{10}$ \vspace{0.5em} \\
\begin{tabular}{l|lll}
$k \setminus j$ & $-4$ & $-2$ & $0$ \\
\hline
$6$ & $q^{-4}$ & $q^{-6}$ &  \\
$4$ & $q^{-2}$ & $q^{-4}$ &  \\
$2$ & $3$ & $4q^{-2}$ & $q^{-4}$ \\
$0$ & $q^{2}$ & $2$ & $q^{-2}$ \\
$-2$ & $q^{4}$ & $4q^{2}$ & $2$ \\
$-4$ &  & $q^{4}$ & $q^{2}$ \\
$-6$ &  & $q^{6}$ & $q^{4}$ \\
\end{tabular}
\vspace{2em}
\end{minipage}
%
\begin{minipage}{\linewidth}
$\bullet\ $ $8_{11}$ \vspace{0.5em} \\
\begin{tabular}{l|llll}
$k \setminus j$ & $0$ & $2$ & $4$ & $6$ \\
\hline
$4$ & $q^{-2}$ & $q^{-4}$ &  &  \\
$2$ & $1$ & $3q^{-2}$ & $q^{-4}$ &  \\
$0$ & $q^{2}$ & $3$ & $2q^{-2}$ &  \\
$-2$ &  & $3q^{2}$ & $4$ & $q^{-2}$ \\
$-4$ &  & $q^{4}$ & $2q^{2}$ & $1$ \\
$-6$ &  &  & $q^{4}$ & $q^{2}$ \\
\end{tabular}
\vspace{2em}
\end{minipage}
%
\begin{minipage}{\linewidth}
$\bullet\ $ $8_{12}$ \vspace{0.5em} \\
\begin{tabular}{l|lllll}
$k \setminus j$ & $-4$ & $-2$ & $0$ & $2$ & $4$ \\
\hline
$4$ & $1$ & $2q^{-2}$ & $q^{-4}$ &  &  \\
$2$ &  & $3$ & $3q^{-2}$ &  &  \\
$0$ &  & $2q^{2}$ & $5$ & $2q^{-2}$ &  \\
$-2$ &  &  & $3q^{2}$ & $3$ &  \\
$-4$ &  &  & $q^{4}$ & $2q^{2}$ & $1$ \\
\end{tabular}
\vspace{2em}
\end{minipage}
%
\begin{minipage}{\linewidth}
$\bullet\ $ $8_{13}$ \vspace{0.5em} \\
\begin{tabular}{l|llll}
$k \setminus j$ & $-4$ & $-2$ & $0$ & $2$ \\
\hline
$6$ & $q^{-2}$ & $q^{-4}$ &  &  \\
$4$ & $1$ & $2q^{-2}$ & $q^{-4}$ &  \\
$2$ & $q^{2}$ & $4$ & $3q^{-2}$ &  \\
$0$ &  & $2q^{2}$ & $4$ & $q^{-2}$ \\
$-2$ &  & $q^{4}$ & $3q^{2}$ & $2$ \\
$-4$ &  &  & $q^{4}$ & $q^{2}$ \\
\end{tabular}
\vspace{2em}
\end{minipage}
%
\begin{minipage}{\linewidth}
$\bullet\ $ $8_{14}$ \vspace{0.5em} \\
\begin{tabular}{l|llll}
$k \setminus j$ & $0$ & $2$ & $4$ & $6$ \\
\hline
$4$ & $q^{-2}$ & $q^{-4}$ &  &  \\
$2$ & $1$ & $3q^{-2}$ & $q^{-4}$ &  \\
$0$ & $q^{2}$ & $4$ & $3q^{-2}$ &  \\
$-2$ &  & $3q^{2}$ & $4$ & $q^{-2}$ \\
$-4$ &  & $q^{4}$ & $3q^{2}$ & $2$ \\
$-6$ &  &  & $q^{4}$ & $q^{2}$ \\
\end{tabular}
\vspace{2em}
\end{minipage}
%
\begin{minipage}{\linewidth}
$\bullet\ $ $8_{15}$ \vspace{0.5em} \\
\begin{tabular}{l|llll}
$k \setminus j$ & $4$ & $6$ & $8$ & $10$ \\
\hline
$4$ & $q^{-4}$ &  &  &  \\
$2$ & $2q^{-2}$ & $2q^{-4}$ &  &  \\
$0$ & $3$ & $3q^{-2}$ &  &  \\
$-2$ & $2q^{2}$ & $5$ & $3q^{-2}$ &  \\
$-4$ & $q^{4}$ & $3q^{2}$ & $2$ &  \\
$-6$ &  & $2q^{4}$ & $3q^{2}$ & $1$ \\
\end{tabular}
\vspace{2em}
\end{minipage}
%
\begin{minipage}{\linewidth}
$\bullet\ $ $8_{16}$ \vspace{0.5em} \\
\begin{tabular}{l|lll}
$k \setminus j$ & $-4$ & $-2$ & $0$ \\
\hline
$6$ & $q^{-4}$ & $q^{-6}$ &  \\
$4$ & $2q^{-2}$ & $2q^{-4}$ &  \\
$2$ & $3$ & $4q^{-2}$ & $q^{-4}$ \\
$0$ & $2q^{2}$ & $4$ & $2q^{-2}$ \\
$-2$ & $q^{4}$ & $4q^{2}$ & $2$ \\
$-4$ &  & $2q^{4}$ & $2q^{2}$ \\
$-6$ &  & $q^{6}$ & $q^{4}$ \\
\end{tabular}
\vspace{2em}
\end{minipage}
%
\begin{minipage}{\linewidth}
$\bullet\ $ $8_{17}$ \vspace{0.5em} \\
\begin{tabular}{l|lll}
$k \setminus j$ & $-2$ & $0$ & $2$ \\
\hline
$6$ & $q^{-4}$ & $q^{-6}$ &  \\
$4$ & $2q^{-2}$ & $2q^{-4}$ &  \\
$2$ & $3$ & $4q^{-2}$ & $q^{-4}$ \\
$0$ & $2q^{2}$ & $5$ & $2q^{-2}$ \\
$-2$ & $q^{4}$ & $4q^{2}$ & $3$ \\
$-4$ &  & $2q^{4}$ & $2q^{2}$ \\
$-6$ &  & $q^{6}$ & $q^{4}$ \\
\end{tabular}
\vspace{2em}
\end{minipage}
%
\begin{minipage}{\linewidth}
$\bullet\ $ $8_{18}$ \vspace{0.5em} \\
\begin{tabular}{l|lll}
$k \setminus j$ & $-2$ & $0$ & $2$ \\
\hline
$6$ & $q^{-4}$ & $q^{-6}$ &  \\
$4$ & $3q^{-2}$ & $3q^{-4}$ &  \\
$2$ & $3$ & $4q^{-2}$ & $q^{-4}$ \\
$0$ & $3q^{2}$ & $7$ & $3q^{-2}$ \\
$-2$ & $q^{4}$ & $4q^{2}$ & $3$ \\
$-4$ &  & $3q^{4}$ & $3q^{2}$ \\
$-6$ &  & $q^{6}$ & $q^{4}$ \\
\end{tabular}
\vspace{2em}
\end{minipage}
%
\begin{minipage}{\linewidth}
$\bullet\ $ $8_{19}$ \vspace{0.5em} \\
\begin{tabular}{l|lll}
$k \setminus j$ & $6$ & $8$ & $10$ \\
\hline
$6$ & $q^{-6}$ &  &  \\
$2$ & $q^{-2}$ & $q^{-4}$ &  \\
$-2$ & $1$ + $q^{2}$ & $q^{-2}$ + $1$ &  \\
$-6$ & $q^{6}$ & $q^{2}$ + $q^{4}$ & $1$ \\
\end{tabular}
\vspace{2em}
\end{minipage}
%
\begin{minipage}{\linewidth}
$\bullet\ $ $8_{20}$ \vspace{0.5em} \\
\begin{tabular}{l|lll}
$k \setminus j$ & $-4$ & $-2$ & $0$ \\
\hline
$6$ & $q^{-2}$ & $q^{-4}$ &  \\
$2$ & $q^{2}$ & $2$ & $q^{-2}$ \\
$0$ &  &  & $1$ \\
$-2$ &  & $q^{4}$ & $q^{2}$ \\
\end{tabular}
\vspace{2em}
\end{minipage}
%
\begin{minipage}{\linewidth}
$\bullet\ $ $8_{21}$ \vspace{0.5em} \\
\begin{tabular}{l|lll}
$k \setminus j$ & $2$ & $4$ & $6$ \\
\hline
$2$ & $2q^{-2}$ & $q^{-4}$ &  \\
$0$ & $1$ & $q^{-2}$ &  \\
$-2$ & $2q^{2}$ & $3$ & $q^{-2}$ \\
$-4$ &  & $q^{2}$ & $1$ \\
$-6$ &  & $q^{4}$ & $q^{2}$ \\
\end{tabular}
\vspace{2em}
\end{minipage}
%
    \begin{minipage}{\linewidth}
$\bullet\ $ $9_{1}$ \vspace{0.5em} \\
\begin{tabular}{l|ll}
$k \setminus j$ & $8$ & $10$ \\
\hline
$8$ & $q^{-8}$ &  \\
$4$ & $q^{-4}$ & $q^{-6}$ \\
$0$ & $1$ & $q^{-2}$ \\
$-4$ & $q^{4}$ & $q^{2}$ \\
$-8$ & $q^{8}$ & $q^{6}$ \\
\end{tabular}
\vspace{2em}
\end{minipage}
%
\begin{minipage}{\linewidth}
$\bullet\ $ $9_{2}$ \vspace{0.5em} \\
\begin{tabular}{l|lllll}
$k \setminus j$ & $2$ & $4$ & $6$ & $8$ & $10$ \\
\hline
$2$ & $q^{-2}$ &  &  &  &  \\
$0$ & $1$ & $q^{-2}$ &  &  &  \\
$-2$ & $q^{2}$ & $2$ & $q^{-2}$ &  &  \\
$-4$ &  & $q^{2}$ & $2$ & $q^{-2}$ &  \\
$-6$ &  &  & $q^{2}$ & $1$ &  \\
$-8$ &  &  &  & $q^{2}$ & $1$ \\
\end{tabular}
\vspace{2em}
\end{minipage}
%
\begin{minipage}{\linewidth}
$\bullet\ $ $9_{3}$ \vspace{0.5em} \\
\begin{tabular}{l|lll}
$k \setminus j$ & $6$ & $8$ & $10$ \\
\hline
$6$ & $q^{-6}$ &  &  \\
$4$ & $q^{-4}$ & $q^{-6}$ &  \\
$2$ & $q^{-2}$ & $q^{-4}$ &  \\
$0$ & $1$ & $2q^{-2}$ & $q^{-4}$ \\
$-2$ & $q^{2}$ & $1$ &  \\
$-4$ & $q^{4}$ & $2q^{2}$ & $1$ \\
$-6$ & $q^{6}$ & $q^{4}$ &  \\
$-8$ &  & $q^{6}$ & $q^{4}$ \\
\end{tabular}
\vspace{2em}
\end{minipage}
%
\begin{minipage}{\linewidth}
$\bullet\ $ $9_{4}$ \vspace{0.5em} \\
\begin{tabular}{l|llll}
$k \setminus j$ & $4$ & $6$ & $8$ & $10$ \\
\hline
$4$ & $q^{-4}$ &  &  &  \\
$2$ & $q^{-2}$ & $q^{-4}$ &  &  \\
$0$ & $1$ & $2q^{-2}$ & $q^{-4}$ &  \\
$-2$ & $q^{2}$ & $2$ & $q^{-2}$ &  \\
$-4$ & $q^{4}$ & $2q^{2}$ & $2$ & $q^{-2}$ \\
$-6$ &  & $q^{4}$ & $q^{2}$ &  \\
$-8$ &  &  & $q^{4}$ & $q^{2}$ \\
\end{tabular}
\vspace{2em}
\end{minipage}
%
\begin{minipage}{\linewidth}
$\bullet\ $ $9_{5}$ \vspace{0.5em} \\
\begin{tabular}{l|lllll}
$k \setminus j$ & $2$ & $4$ & $6$ & $8$ & $10$ \\
\hline
$2$ & $q^{-2}$ &  &  &  &  \\
$0$ & $2$ & $2q^{-2}$ &  &  &  \\
$-2$ & $q^{2}$ & $3$ & $2q^{-2}$ &  &  \\
$-4$ &  & $2q^{2}$ & $3$ & $q^{-2}$ &  \\
$-6$ &  &  & $2q^{2}$ & $2$ &  \\
$-8$ &  &  &  & $q^{2}$ & $1$ \\
\end{tabular}
\vspace{2em}
\end{minipage}
%
\begin{minipage}{\linewidth}
$\bullet\ $ $9_{6}$ \vspace{0.5em} \\
\begin{tabular}{l|lll}
$k \setminus j$ & $6$ & $8$ & $10$ \\
\hline
$6$ & $q^{-6}$ &  &  \\
$4$ & $q^{-4}$ & $q^{-6}$ &  \\
$2$ & $2q^{-2}$ & $2q^{-4}$ &  \\
$0$ & $1$ & $2q^{-2}$ & $q^{-4}$ \\
$-2$ & $2q^{2}$ & $3$ & $q^{-2}$ \\
$-4$ & $q^{4}$ & $2q^{2}$ & $1$ \\
$-6$ & $q^{6}$ & $2q^{4}$ & $q^{2}$ \\
$-8$ &  & $q^{6}$ & $q^{4}$ \\
\end{tabular}
\vspace{2em}
\end{minipage}
%
\begin{minipage}{\linewidth}
$\bullet\ $ $9_{7}$ \vspace{0.5em} \\
\begin{tabular}{l|llll}
$k \setminus j$ & $4$ & $6$ & $8$ & $10$ \\
\hline
$4$ & $q^{-4}$ &  &  &  \\
$2$ & $q^{-2}$ & $q^{-4}$ &  &  \\
$0$ & $2$ & $3q^{-2}$ & $q^{-4}$ &  \\
$-2$ & $q^{2}$ & $3$ & $2q^{-2}$ &  \\
$-4$ & $q^{4}$ & $3q^{2}$ & $3$ & $q^{-2}$ \\
$-6$ &  & $q^{4}$ & $2q^{2}$ & $1$ \\
$-8$ &  &  & $q^{4}$ & $q^{2}$ \\
\end{tabular}
\vspace{2em}
\end{minipage}
%
\begin{minipage}{\linewidth}
$\bullet\ $ $9_{8}$ \vspace{0.5em} \\
\begin{tabular}{l|lllll}
$k \setminus j$ & $-2$ & $0$ & $2$ & $4$ & $6$ \\
\hline
$6$ & $q^{-2}$ & $q^{-4}$ &  &  &  \\
$4$ & $1$ & $2q^{-2}$ & $q^{-4}$ &  &  \\
$2$ & $q^{2}$ & $3$ & $3q^{-2}$ &  &  \\
$0$ &  & $2q^{2}$ & $4$ & $2q^{-2}$ &  \\
$-2$ &  & $q^{4}$ & $3q^{2}$ & $2$ &  \\
$-4$ &  &  & $q^{4}$ & $2q^{2}$ & $1$ \\
\end{tabular}
\vspace{2em}
\end{minipage}
%
\begin{minipage}{\linewidth}
$\bullet\ $ $9_{9}$ \vspace{0.5em} \\
\begin{tabular}{l|lll}
$k \setminus j$ & $6$ & $8$ & $10$ \\
\hline
$6$ & $q^{-6}$ &  &  \\
$4$ & $q^{-4}$ & $q^{-6}$ &  \\
$2$ & $2q^{-2}$ & $2q^{-4}$ &  \\
$0$ & $2$ & $3q^{-2}$ & $q^{-4}$ \\
$-2$ & $2q^{2}$ & $3$ & $q^{-2}$ \\
$-4$ & $q^{4}$ & $3q^{2}$ & $2$ \\
$-6$ & $q^{6}$ & $2q^{4}$ & $q^{2}$ \\
$-8$ &  & $q^{6}$ & $q^{4}$ \\
\end{tabular}
\vspace{2em}
\end{minipage}
%
\begin{minipage}{\linewidth}
$\bullet\ $ $9_{10}$ \vspace{0.5em} \\
\begin{tabular}{l|llll}
$k \setminus j$ & $4$ & $6$ & $8$ & $10$ \\
\hline
$4$ & $q^{-4}$ &  &  &  \\
$2$ & $2q^{-2}$ & $2q^{-4}$ &  &  \\
$0$ & $2$ & $3q^{-2}$ & $q^{-4}$ &  \\
$-2$ & $2q^{2}$ & $4$ & $2q^{-2}$ &  \\
$-4$ & $q^{4}$ & $3q^{2}$ & $3$ & $q^{-2}$ \\
$-6$ &  & $2q^{4}$ & $2q^{2}$ &  \\
$-8$ &  &  & $q^{4}$ & $q^{2}$ \\
\end{tabular}
\vspace{2em}
\end{minipage}
%
\begin{minipage}{\linewidth}
$\bullet\ $ $9_{11}$ \vspace{0.5em} \\
\begin{tabular}{l|llll}
$k \setminus j$ & $-8$ & $-6$ & $-4$ & $-2$ \\
\hline
$6$ & $q^{-2}$ & $2q^{-4}$ & $q^{-6}$ &  \\
$4$ &  & $2q^{-2}$ & $2q^{-4}$ &  \\
$2$ & $q^{2}$ & $3$ & $3q^{-2}$ & $q^{-4}$ \\
$0$ &  & $2q^{2}$ & $3$ & $q^{-2}$ \\
$-2$ &  & $2q^{4}$ & $3q^{2}$ & $1$ \\
$-4$ &  &  & $2q^{4}$ & $q^{2}$ \\
$-6$ &  &  & $q^{6}$ & $q^{4}$ \\
\end{tabular}
\vspace{2em}
\end{minipage}
%
\begin{minipage}{\linewidth}
$\bullet\ $ $9_{12}$ \vspace{0.5em} \\
\begin{tabular}{l|lllll}
$k \setminus j$ & $0$ & $2$ & $4$ & $6$ & $8$ \\
\hline
$4$ & $q^{-2}$ & $q^{-4}$ &  &  &  \\
$2$ & $1$ & $3q^{-2}$ & $q^{-4}$ &  &  \\
$0$ & $q^{2}$ & $4$ & $3q^{-2}$ &  &  \\
$-2$ &  & $3q^{2}$ & $5$ & $2q^{-2}$ &  \\
$-4$ &  & $q^{4}$ & $3q^{2}$ & $2$ &  \\
$-6$ &  &  & $q^{4}$ & $2q^{2}$ & $1$ \\
\end{tabular}
\vspace{2em}
\end{minipage}
%
\begin{minipage}{\linewidth}
$\bullet\ $ $9_{13}$ \vspace{0.5em} \\
\begin{tabular}{l|llll}
$k \setminus j$ & $4$ & $6$ & $8$ & $10$ \\
\hline
$4$ & $q^{-4}$ &  &  &  \\
$2$ & $2q^{-2}$ & $2q^{-4}$ &  &  \\
$0$ & $2$ & $3q^{-2}$ & $q^{-4}$ &  \\
$-2$ & $2q^{2}$ & $5$ & $3q^{-2}$ &  \\
$-4$ & $q^{4}$ & $3q^{2}$ & $3$ & $q^{-2}$ \\
$-6$ &  & $2q^{4}$ & $3q^{2}$ & $1$ \\
$-8$ &  &  & $q^{4}$ & $q^{2}$ \\
\end{tabular}
\vspace{2em}
\end{minipage}
%
\begin{minipage}{\linewidth}
$\bullet\ $ $9_{14}$ \vspace{0.5em} \\
\begin{tabular}{l|lllll}
$k \setminus j$ & $-6$ & $-4$ & $-2$ & $0$ & $2$ \\
\hline
$6$ & $1$ & $2q^{-2}$ & $q^{-4}$ &  &  \\
$4$ &  & $2$ & $3q^{-2}$ & $q^{-4}$ &  \\
$2$ &  & $2q^{2}$ & $5$ & $3q^{-2}$ &  \\
$0$ &  &  & $3q^{2}$ & $5$ & $q^{-2}$ \\
$-2$ &  &  & $q^{4}$ & $3q^{2}$ & $2$ \\
$-4$ &  &  &  & $q^{4}$ & $q^{2}$ \\
\end{tabular}
\vspace{2em}
\end{minipage}
%
\begin{minipage}{\linewidth}
$\bullet\ $ $9_{15}$ \vspace{0.5em} \\
\begin{tabular}{l|lllll}
$k \setminus j$ & $-8$ & $-6$ & $-4$ & $-2$ & $0$ \\
\hline
$6$ & $1$ & $2q^{-2}$ & $q^{-4}$ &  &  \\
$4$ &  & $3$ & $4q^{-2}$ & $q^{-4}$ &  \\
$2$ &  & $2q^{2}$ & $5$ & $3q^{-2}$ &  \\
$0$ &  &  & $4q^{2}$ & $5$ & $q^{-2}$ \\
$-2$ &  &  & $q^{4}$ & $3q^{2}$ & $1$ \\
$-4$ &  &  &  & $q^{4}$ & $q^{2}$ \\
\end{tabular}
\vspace{2em}
\end{minipage}
%
\begin{minipage}{\linewidth}
$\bullet\ $ $9_{16}$ \vspace{0.5em} \\
\begin{tabular}{l|lll}
$k \setminus j$ & $6$ & $8$ & $10$ \\
\hline
$6$ & $q^{-6}$ &  &  \\
$4$ & $q^{-4}$ & $q^{-6}$ &  \\
$2$ & $3q^{-2}$ & $3q^{-4}$ &  \\
$0$ & $2$ & $3q^{-2}$ & $q^{-4}$ \\
$-2$ & $3q^{2}$ & $5$ & $2q^{-2}$ \\
$-4$ & $q^{4}$ & $3q^{2}$ & $2$ \\
$-6$ & $q^{6}$ & $3q^{4}$ & $2q^{2}$ \\
$-8$ &  & $q^{6}$ & $q^{4}$ \\
\end{tabular}
\vspace{2em}
\end{minipage}
%
\begin{minipage}{\linewidth}
$\bullet\ $ $9_{17}$ \vspace{0.5em} \\
\begin{tabular}{l|llll}
$k \setminus j$ & $-2$ & $0$ & $2$ & $4$ \\
\hline
$6$ & $q^{-2}$ & $2q^{-4}$ & $q^{-6}$ &  \\
$4$ &  & $2q^{-2}$ & $2q^{-4}$ &  \\
$2$ & $q^{2}$ & $3$ & $4q^{-2}$ & $q^{-4}$ \\
$0$ &  & $2q^{2}$ & $4$ & $2q^{-2}$ \\
$-2$ &  & $2q^{4}$ & $4q^{2}$ & $2$ \\
$-4$ &  &  & $2q^{4}$ & $2q^{2}$ \\
$-6$ &  &  & $q^{6}$ & $q^{4}$ \\
\end{tabular}
\vspace{2em}
\end{minipage}
%
\begin{minipage}{\linewidth}
$\bullet\ $ $9_{18}$ \vspace{0.5em} \\
\begin{tabular}{l|llll}
$k \setminus j$ & $4$ & $6$ & $8$ & $10$ \\
\hline
$4$ & $q^{-4}$ &  &  &  \\
$2$ & $2q^{-2}$ & $2q^{-4}$ &  &  \\
$0$ & $3$ & $4q^{-2}$ & $q^{-4}$ &  \\
$-2$ & $2q^{2}$ & $5$ & $3q^{-2}$ &  \\
$-4$ & $q^{4}$ & $4q^{2}$ & $4$ & $q^{-2}$ \\
$-6$ &  & $2q^{4}$ & $3q^{2}$ & $1$ \\
$-8$ &  &  & $q^{4}$ & $q^{2}$ \\
\end{tabular}
\vspace{2em}
\end{minipage}
%
\begin{minipage}{\linewidth}
$\bullet\ $ $9_{19}$ \vspace{0.5em} \\
\begin{tabular}{l|lllll}
$k \setminus j$ & $-4$ & $-2$ & $0$ & $2$ & $4$ \\
\hline
$4$ & $1$ & $2q^{-2}$ & $q^{-4}$ &  &  \\
$2$ &  & $3$ & $4q^{-2}$ & $q^{-4}$ &  \\
$0$ &  & $2q^{2}$ & $6$ & $3q^{-2}$ &  \\
$-2$ &  &  & $4q^{2}$ & $5$ & $q^{-2}$ \\
$-4$ &  &  & $q^{4}$ & $3q^{2}$ & $2$ \\
$-6$ &  &  &  & $q^{4}$ & $q^{2}$ \\
\end{tabular}
\vspace{2em}
\end{minipage}
%
\begin{minipage}{\linewidth}
$\bullet\ $ $9_{20}$ \vspace{0.5em} \\
\begin{tabular}{l|llll}
$k \setminus j$ & $2$ & $4$ & $6$ & $8$ \\
\hline
$6$ & $q^{-4}$ & $q^{-6}$ &  &  \\
$4$ & $q^{-2}$ & $2q^{-4}$ &  &  \\
$2$ & $2$ & $4q^{-2}$ & $2q^{-4}$ &  \\
$0$ & $q^{2}$ & $4$ & $3q^{-2}$ &  \\
$-2$ & $q^{4}$ & $4q^{2}$ & $4$ & $q^{-2}$ \\
$-4$ &  & $2q^{4}$ & $3q^{2}$ & $1$ \\
$-6$ &  & $q^{6}$ & $2q^{4}$ & $q^{2}$ \\
\end{tabular}
\vspace{2em}
\end{minipage}
%
\begin{minipage}{\linewidth}
$\bullet\ $ $9_{21}$ \vspace{0.5em} \\
\begin{tabular}{l|lllll}
$k \setminus j$ & $-8$ & $-6$ & $-4$ & $-2$ & $0$ \\
\hline
$6$ & $1$ & $2q^{-2}$ & $q^{-4}$ &  &  \\
$4$ &  & $3$ & $4q^{-2}$ & $q^{-4}$ &  \\
$2$ &  & $2q^{2}$ & $6$ & $4q^{-2}$ &  \\
$0$ &  &  & $4q^{2}$ & $5$ & $q^{-2}$ \\
$-2$ &  &  & $q^{4}$ & $4q^{2}$ & $2$ \\
$-4$ &  &  &  & $q^{4}$ & $q^{2}$ \\
\end{tabular}
\vspace{2em}
\end{minipage}
%
\begin{minipage}{\linewidth}
$\bullet\ $ $9_{22}$ \vspace{0.5em} \\
\begin{tabular}{l|llll}
$k \setminus j$ & $-2$ & $0$ & $2$ & $4$ \\
\hline
$6$ & $q^{-2}$ & $2q^{-4}$ & $q^{-6}$ &  \\
$4$ &  & $2q^{-2}$ & $2q^{-4}$ &  \\
$2$ & $q^{2}$ & $4$ & $5q^{-2}$ & $q^{-4}$ \\
$0$ &  & $2q^{2}$ & $4$ & $2q^{-2}$ \\
$-2$ &  & $2q^{4}$ & $5q^{2}$ & $3$ \\
$-4$ &  &  & $2q^{4}$ & $2q^{2}$ \\
$-6$ &  &  & $q^{6}$ & $q^{4}$ \\
\end{tabular}
\vspace{2em}
\end{minipage}
%
\begin{minipage}{\linewidth}
$\bullet\ $ $9_{23}$ \vspace{0.5em} \\
\begin{tabular}{l|llll}
$k \setminus j$ & $4$ & $6$ & $8$ & $10$ \\
\hline
$4$ & $q^{-4}$ &  &  &  \\
$2$ & $2q^{-2}$ & $2q^{-4}$ &  &  \\
$0$ & $3$ & $4q^{-2}$ & $q^{-4}$ &  \\
$-2$ & $2q^{2}$ & $6$ & $4q^{-2}$ &  \\
$-4$ & $q^{4}$ & $4q^{2}$ & $4$ & $q^{-2}$ \\
$-6$ &  & $2q^{4}$ & $4q^{2}$ & $2$ \\
$-8$ &  &  & $q^{4}$ & $q^{2}$ \\
\end{tabular}
\vspace{2em}
\end{minipage}
%
\begin{minipage}{\linewidth}
$\bullet\ $ $9_{24}$ \vspace{0.5em} \\
\begin{tabular}{l|llll}
$k \setminus j$ & $-2$ & $0$ & $2$ & $4$ \\
\hline
$6$ & $q^{-4}$ & $q^{-6}$ &  &  \\
$4$ & $2q^{-2}$ & $2q^{-4}$ &  &  \\
$2$ & $3$ & $5q^{-2}$ & $2q^{-4}$ &  \\
$0$ & $2q^{2}$ & $5$ & $2q^{-2}$ &  \\
$-2$ & $q^{4}$ & $5q^{2}$ & $5$ & $q^{-2}$ \\
$-4$ &  & $2q^{4}$ & $2q^{2}$ &  \\
$-6$ &  & $q^{6}$ & $2q^{4}$ & $q^{2}$ \\
\end{tabular}
\vspace{2em}
\end{minipage}
%
\begin{minipage}{\linewidth}
$\bullet\ $ $9_{25}$ \vspace{0.5em} \\
\begin{tabular}{l|lllll}
$k \setminus j$ & $0$ & $2$ & $4$ & $6$ & $8$ \\
\hline
$4$ & $q^{-2}$ & $q^{-4}$ &  &  &  \\
$2$ & $1$ & $4q^{-2}$ & $2q^{-4}$ &  &  \\
$0$ & $q^{2}$ & $5$ & $4q^{-2}$ &  &  \\
$-2$ &  & $4q^{2}$ & $7$ & $3q^{-2}$ &  \\
$-4$ &  & $q^{4}$ & $4q^{2}$ & $3$ &  \\
$-6$ &  &  & $2q^{4}$ & $3q^{2}$ & $1$ \\
\end{tabular}
\vspace{2em}
\end{minipage}
%
\begin{minipage}{\linewidth}
$\bullet\ $ $9_{26}$ \vspace{0.5em} \\
\begin{tabular}{l|llll}
$k \setminus j$ & $-6$ & $-4$ & $-2$ & $0$ \\
\hline
$6$ & $q^{-2}$ & $2q^{-4}$ & $q^{-6}$ &  \\
$4$ & $1$ & $3q^{-2}$ & $2q^{-4}$ &  \\
$2$ & $q^{2}$ & $5$ & $5q^{-2}$ & $q^{-4}$ \\
$0$ &  & $3q^{2}$ & $5$ & $2q^{-2}$ \\
$-2$ &  & $2q^{4}$ & $5q^{2}$ & $2$ \\
$-4$ &  &  & $2q^{4}$ & $2q^{2}$ \\
$-6$ &  &  & $q^{6}$ & $q^{4}$ \\
\end{tabular}
\vspace{2em}
\end{minipage}
%
\begin{minipage}{\linewidth}
$\bullet\ $ $9_{27}$ \vspace{0.5em} \\
\begin{tabular}{l|llll}
$k \setminus j$ & $-2$ & $0$ & $2$ & $4$ \\
\hline
$6$ & $q^{-4}$ & $q^{-6}$ &  &  \\
$4$ & $2q^{-2}$ & $2q^{-4}$ &  &  \\
$2$ & $3$ & $5q^{-2}$ & $2q^{-4}$ &  \\
$0$ & $2q^{2}$ & $6$ & $3q^{-2}$ &  \\
$-2$ & $q^{4}$ & $5q^{2}$ & $5$ & $q^{-2}$ \\
$-4$ &  & $2q^{4}$ & $3q^{2}$ & $1$ \\
$-6$ &  & $q^{6}$ & $2q^{4}$ & $q^{2}$ \\
\end{tabular}
\vspace{2em}
\end{minipage}
%
\begin{minipage}{\linewidth}
$\bullet\ $ $9_{28}$ \vspace{0.5em} \\
\begin{tabular}{l|llll}
$k \setminus j$ & $0$ & $2$ & $4$ & $6$ \\
\hline
$6$ & $q^{-4}$ & $q^{-6}$ &  &  \\
$4$ & $2q^{-2}$ & $2q^{-4}$ &  &  \\
$2$ & $3$ & $6q^{-2}$ & $2q^{-4}$ &  \\
$0$ & $2q^{2}$ & $5$ & $3q^{-2}$ &  \\
$-2$ & $q^{4}$ & $6q^{2}$ & $6$ & $q^{-2}$ \\
$-4$ &  & $2q^{4}$ & $3q^{2}$ & $1$ \\
$-6$ &  & $q^{6}$ & $2q^{4}$ & $q^{2}$ \\
\end{tabular}
\vspace{2em}
\end{minipage}
%
\begin{minipage}{\linewidth}
$\bullet\ $ $9_{29}$ \vspace{0.5em} \\
\begin{tabular}{l|llll}
$k \setminus j$ & $-4$ & $-2$ & $0$ & $2$ \\
\hline
$6$ & $q^{-4}$ & $q^{-6}$ &  &  \\
$4$ & $2q^{-2}$ & $2q^{-4}$ &  &  \\
$2$ & $4$ & $6q^{-2}$ & $2q^{-4}$ &  \\
$0$ & $2q^{2}$ & $5$ & $3q^{-2}$ &  \\
$-2$ & $q^{4}$ & $6q^{2}$ & $5$ & $q^{-2}$ \\
$-4$ &  & $2q^{4}$ & $3q^{2}$ & $1$ \\
$-6$ &  & $q^{6}$ & $2q^{4}$ & $q^{2}$ \\
\end{tabular}
\vspace{2em}
\end{minipage}
%
\begin{minipage}{\linewidth}
$\bullet\ $ $9_{30}$ \vspace{0.5em} \\
\begin{tabular}{l|llll}
$k \setminus j$ & $-4$ & $-2$ & $0$ & $2$ \\
\hline
$6$ & $q^{-2}$ & $2q^{-4}$ & $q^{-6}$ &  \\
$4$ & $1$ & $3q^{-2}$ & $2q^{-4}$ &  \\
$2$ & $q^{2}$ & $6$ & $6q^{-2}$ & $q^{-4}$ \\
$0$ &  & $3q^{2}$ & $6$ & $2q^{-2}$ \\
$-2$ &  & $2q^{4}$ & $6q^{2}$ & $4$ \\
$-4$ &  &  & $2q^{4}$ & $2q^{2}$ \\
$-6$ &  &  & $q^{6}$ & $q^{4}$ \\
\end{tabular}
\vspace{2em}
\end{minipage}
%
\begin{minipage}{\linewidth}
$\bullet\ $ $9_{31}$ \vspace{0.5em} \\
\begin{tabular}{l|llll}
$k \setminus j$ & $0$ & $2$ & $4$ & $6$ \\
\hline
$6$ & $q^{-4}$ & $q^{-6}$ &  &  \\
$4$ & $2q^{-2}$ & $2q^{-4}$ &  &  \\
$2$ & $3$ & $6q^{-2}$ & $2q^{-4}$ &  \\
$0$ & $2q^{2}$ & $6$ & $4q^{-2}$ &  \\
$-2$ & $q^{4}$ & $6q^{2}$ & $6$ & $q^{-2}$ \\
$-4$ &  & $2q^{4}$ & $4q^{2}$ & $2$ \\
$-6$ &  & $q^{6}$ & $2q^{4}$ & $q^{2}$ \\
\end{tabular}
\vspace{2em}
\end{minipage}
%
\begin{minipage}{\linewidth}
$\bullet\ $ $9_{32}$ \vspace{0.5em} \\
\begin{tabular}{l|llll}
$k \setminus j$ & $-6$ & $-4$ & $-2$ & $0$ \\
\hline
$6$ & $q^{-2}$ & $2q^{-4}$ & $q^{-6}$ &  \\
$4$ & $1$ & $4q^{-2}$ & $3q^{-4}$ &  \\
$2$ & $q^{2}$ & $6$ & $6q^{-2}$ & $q^{-4}$ \\
$0$ &  & $4q^{2}$ & $7$ & $3q^{-2}$ \\
$-2$ &  & $2q^{4}$ & $6q^{2}$ & $3$ \\
$-4$ &  &  & $3q^{4}$ & $3q^{2}$ \\
$-6$ &  &  & $q^{6}$ & $q^{4}$ \\
\end{tabular}
\vspace{2em}
\end{minipage}
%
\begin{minipage}{\linewidth}
$\bullet\ $ $9_{33}$ \vspace{0.5em} \\
\begin{tabular}{l|llll}
$k \setminus j$ & $-4$ & $-2$ & $0$ & $2$ \\
\hline
$6$ & $q^{-2}$ & $2q^{-4}$ & $q^{-6}$ &  \\
$4$ & $1$ & $4q^{-2}$ & $3q^{-4}$ &  \\
$2$ & $q^{2}$ & $6$ & $6q^{-2}$ & $q^{-4}$ \\
$0$ &  & $4q^{2}$ & $8$ & $3q^{-2}$ \\
$-2$ &  & $2q^{4}$ & $6q^{2}$ & $4$ \\
$-4$ &  &  & $3q^{4}$ & $3q^{2}$ \\
$-6$ &  &  & $q^{6}$ & $q^{4}$ \\
\end{tabular}
\vspace{2em}
\end{minipage}
%
\begin{minipage}{\linewidth}
$\bullet\ $ $9_{34}$ \vspace{0.5em} \\
\begin{tabular}{l|llll}
$k \setminus j$ & $-4$ & $-2$ & $0$ & $2$ \\
\hline
$6$ & $q^{-2}$ & $2q^{-4}$ & $q^{-6}$ &  \\
$4$ & $2$ & $5q^{-2}$ & $3q^{-4}$ &  \\
$2$ & $q^{2}$ & $7$ & $7q^{-2}$ & $q^{-4}$ \\
$0$ &  & $5q^{2}$ & $9$ & $3q^{-2}$ \\
$-2$ &  & $2q^{4}$ & $7q^{2}$ & $5$ \\
$-4$ &  &  & $3q^{4}$ & $3q^{2}$ \\
$-6$ &  &  & $q^{6}$ & $q^{4}$ \\
\end{tabular}
\vspace{2em}
\end{minipage}
%
\begin{minipage}{\linewidth}
$\bullet\ $ $9_{35}$ \vspace{0.5em} \\
\begin{tabular}{l|lllll}
$k \setminus j$ & $2$ & $4$ & $6$ & $8$ & $10$ \\
\hline
$2$ & $q^{-2}$ &  &  &  &  \\
$0$ & $2$ & $2q^{-2}$ &  &  &  \\
$-2$ & $q^{2}$ & $4$ & $3q^{-2}$ &  &  \\
$-4$ &  & $2q^{2}$ & $3$ & $q^{-2}$ &  \\
$-6$ &  &  & $3q^{2}$ & $3$ &  \\
$-8$ &  &  &  & $q^{2}$ & $1$ \\
\end{tabular}
\vspace{2em}
\end{minipage}
%
\begin{minipage}{\linewidth}
$\bullet\ $ $9_{36}$ \vspace{0.5em} \\
\begin{tabular}{l|llll}
$k \setminus j$ & $-8$ & $-6$ & $-4$ & $-2$ \\
\hline
$6$ & $q^{-2}$ & $2q^{-4}$ & $q^{-6}$ &  \\
$4$ &  & $2q^{-2}$ & $2q^{-4}$ &  \\
$2$ & $q^{2}$ & $4$ & $4q^{-2}$ & $q^{-4}$ \\
$0$ &  & $2q^{2}$ & $3$ & $q^{-2}$ \\
$-2$ &  & $2q^{4}$ & $4q^{2}$ & $2$ \\
$-4$ &  &  & $2q^{4}$ & $q^{2}$ \\
$-6$ &  &  & $q^{6}$ & $q^{4}$ \\
\end{tabular}
\vspace{2em}
\end{minipage}
%
\begin{minipage}{\linewidth}
$\bullet\ $ $9_{37}$ \vspace{0.5em} \\
\begin{tabular}{l|lllll}
$k \setminus j$ & $-4$ & $-2$ & $0$ & $2$ & $4$ \\
\hline
$4$ & $1$ & $2q^{-2}$ & $q^{-4}$ &  &  \\
$2$ &  & $4$ & $5q^{-2}$ & $q^{-4}$ &  \\
$0$ &  & $2q^{2}$ & $6$ & $3q^{-2}$ &  \\
$-2$ &  &  & $5q^{2}$ & $6$ & $q^{-2}$ \\
$-4$ &  &  & $q^{4}$ & $3q^{2}$ & $2$ \\
$-6$ &  &  &  & $q^{4}$ & $q^{2}$ \\
\end{tabular}
\vspace{2em}
\end{minipage}
%
\begin{minipage}{\linewidth}
$\bullet\ $ $9_{38}$ \vspace{0.5em} \\
\begin{tabular}{l|llll}
$k \setminus j$ & $4$ & $6$ & $8$ & $10$ \\
\hline
$4$ & $q^{-4}$ &  &  &  \\
$2$ & $3q^{-2}$ & $3q^{-4}$ &  &  \\
$0$ & $4$ & $5q^{-2}$ & $q^{-4}$ &  \\
$-2$ & $3q^{2}$ & $8$ & $5q^{-2}$ &  \\
$-4$ & $q^{4}$ & $5q^{2}$ & $5$ & $q^{-2}$ \\
$-6$ &  & $3q^{4}$ & $5q^{2}$ & $2$ \\
$-8$ &  &  & $q^{4}$ & $q^{2}$ \\
\end{tabular}
\vspace{2em}
\end{minipage}
%
\begin{minipage}{\linewidth}
$\bullet\ $ $9_{39}$ \vspace{0.5em} \\
\begin{tabular}{l|lllll}
$k \setminus j$ & $-8$ & $-6$ & $-4$ & $-2$ & $0$ \\
\hline
$6$ & $1$ & $3q^{-2}$ & $2q^{-4}$ &  &  \\
$4$ &  & $4$ & $5q^{-2}$ & $q^{-4}$ &  \\
$2$ &  & $3q^{2}$ & $8$ & $5q^{-2}$ &  \\
$0$ &  &  & $5q^{2}$ & $6$ & $q^{-2}$ \\
$-2$ &  &  & $2q^{4}$ & $5q^{2}$ & $2$ \\
$-4$ &  &  &  & $q^{4}$ & $q^{2}$ \\
\end{tabular}
\vspace{2em}
\end{minipage}
%
\begin{minipage}{\linewidth}
$\bullet\ $ $9_{40}$ \vspace{0.5em} \\
\begin{tabular}{l|llll}
$k \setminus j$ & $0$ & $2$ & $4$ & $6$ \\
\hline
$6$ & $q^{-4}$ & $q^{-6}$ &  &  \\
$4$ & $4q^{-2}$ & $4q^{-4}$ &  &  \\
$2$ & $4$ & $7q^{-2}$ & $2q^{-4}$ &  \\
$0$ & $4q^{2}$ & $10$ & $6q^{-2}$ &  \\
$-2$ & $q^{4}$ & $7q^{2}$ & $7$ & $q^{-2}$ \\
$-4$ &  & $4q^{4}$ & $6q^{2}$ & $2$ \\
$-6$ &  & $q^{6}$ & $2q^{4}$ & $q^{2}$ \\
\end{tabular}
\vspace{2em}
\end{minipage}
%
\begin{minipage}{\linewidth}
$\bullet\ $ $9_{41}$ \vspace{0.5em} \\
\begin{tabular}{l|lllll}
$k \setminus j$ & $-6$ & $-4$ & $-2$ & $0$ & $2$ \\
\hline
$6$ & $1$ & $3q^{-2}$ & $2q^{-4}$ &  &  \\
$4$ &  & $3$ & $4q^{-2}$ & $q^{-4}$ &  \\
$2$ &  & $3q^{2}$ & $7$ & $4q^{-2}$ &  \\
$0$ &  &  & $4q^{2}$ & $6$ & $q^{-2}$ \\
$-2$ &  &  & $2q^{4}$ & $4q^{2}$ & $2$ \\
$-4$ &  &  &  & $q^{4}$ & $q^{2}$ \\
\end{tabular}
\vspace{2em}
\end{minipage}
%
\begin{minipage}{\linewidth}
$\bullet\ $ $9_{42}$ \vspace{0.5em} \\
\begin{tabular}{l|lll}
$k \setminus j$ & $-2$ & $0$ & $2$ \\
\hline
$2$ & $q^{-2}$ & $q^{-4}$ &  \\
$0$ &  & $1$ &  \\
$-2$ & $q^{2}$ & $2$ & $q^{-2}$ \\
$-6$ &  & $q^{4}$ & $q^{2}$ \\
\end{tabular}
\vspace{2em}
\end{minipage}
%
\begin{minipage}{\linewidth}
$\bullet\ $ $9_{43}$ \vspace{0.5em} \\
\begin{tabular}{l|llll}
$k \setminus j$ & $2$ & $4$ & $6$ & $8$ \\
\hline
$6$ & $q^{-4}$ & $q^{-6}$ &  &  \\
$4$ &  & $q^{-4}$ &  &  \\
$2$ & $1$ & $2q^{-2}$ & $q^{-4}$ &  \\
$0$ &  & $1$ & $q^{-2}$ &  \\
$-2$ & $q^{4}$ & $1$ + $2q^{2}$ & $q^{-2}$ + $1$ &  \\
$-4$ &  & $q^{4}$ & $q^{2}$ &  \\
$-6$ &  & $q^{6}$ & $q^{2}$ + $q^{4}$ & $1$ \\
\end{tabular}
\vspace{2em}
\end{minipage}
%
\begin{minipage}{\linewidth}
$\bullet\ $ $9_{44}$ \vspace{0.5em} \\
\begin{tabular}{l|llll}
$k \setminus j$ & $-2$ & $0$ & $2$ & $4$ \\
\hline
$2$ & $1$ & $2q^{-2}$ & $q^{-4}$ &  \\
$0$ &  & $2$ & $q^{-2}$ &  \\
$-2$ &  & $2q^{2}$ & $3$ & $q^{-2}$ \\
$-4$ &  &  & $q^{2}$ & $1$ \\
$-6$ &  &  & $q^{4}$ & $q^{2}$ \\
\end{tabular}
\vspace{2em}
\end{minipage}
%
\begin{minipage}{\linewidth}
$\bullet\ $ $9_{45}$ \vspace{0.5em} \\
\begin{tabular}{l|llll}
$k \setminus j$ & $-8$ & $-6$ & $-4$ & $-2$ \\
\hline
$6$ & $1$ & $2q^{-2}$ & $q^{-4}$ &  \\
$4$ &  & $2$ & $2q^{-2}$ &  \\
$2$ &  & $2q^{2}$ & $4$ & $2q^{-2}$ \\
$0$ &  &  & $2q^{2}$ & $2$ \\
$-2$ &  &  & $q^{4}$ & $2q^{2}$ \\
\end{tabular}
\vspace{2em}
\end{minipage}
%
\begin{minipage}{\linewidth}
$\bullet\ $ $9_{46}$ \vspace{0.5em} \\
\begin{tabular}{l|llll}
$k \setminus j$ & $-6$ & $-4$ & $-2$ & $0$ \\
\hline
$6$ & $1$ & $q^{-2}$ &  &  \\
$4$ &  & $1$ & $q^{-2}$ &  \\
$2$ &  & $q^{2}$ & $1$ &  \\
$0$ &  &  & $q^{2}$ & $2$ \\
\end{tabular}
\vspace{2em}
\end{minipage}
%
\begin{minipage}{\linewidth}
$\bullet\ $ $9_{47}$ \vspace{0.5em} \\
\begin{tabular}{l|llll}
$k \setminus j$ & $0$ & $2$ & $4$ & $6$ \\
\hline
$6$ & $q^{-4}$ & $q^{-6}$ &  &  \\
$4$ & $2q^{-2}$ & $2q^{-4}$ &  &  \\
$2$ & $1$ & $3q^{-2}$ & $q^{-4}$ &  \\
$0$ & $2q^{2}$ & $4$ & $2q^{-2}$ &  \\
$-2$ & $q^{4}$ & $1$ + $3q^{2}$ & $q^{-2}$ + $2$ &  \\
$-4$ &  & $2q^{4}$ & $2q^{2}$ &  \\
$-6$ &  & $q^{6}$ & $q^{2}$ + $q^{4}$ & $1$ \\
\end{tabular}
\vspace{2em}
\end{minipage}
%
\begin{minipage}{\linewidth}
$\bullet\ $ $9_{48}$ \vspace{0.5em} \\
\begin{tabular}{l|llll}
$k \setminus j$ & $-6$ & $-4$ & $-2$ & $0$ \\
\hline
$4$ & $2$ & $3q^{-2}$ & $q^{-4}$ &  \\
$2$ &  & $3$ & $3q^{-2}$ &  \\
$0$ &  & $3q^{2}$ & $4$ & $q^{-2}$ \\
$-2$ &  &  & $3q^{2}$ & $2$ \\
$-4$ &  &  & $q^{4}$ & $q^{2}$ \\
\end{tabular}
\vspace{2em}
\end{minipage}
%
\begin{minipage}{\linewidth}
$\bullet\ $ $9_{49}$ \vspace{0.5em} \\
\begin{tabular}{l|lll}
$k \setminus j$ & $4$ & $6$ & $8$ \\
\hline
$4$ & $q^{-4}$ &  &  \\
$2$ & $2q^{-2}$ & $2q^{-4}$ &  \\
$0$ & $2$ & $2q^{-2}$ &  \\
$-2$ & $2q^{2}$ & $4$ & $2q^{-2}$ \\
$-4$ & $q^{4}$ & $2q^{2}$ & $1$ \\
$-6$ &  & $2q^{4}$ & $2q^{2}$ \\
\end{tabular}
\vspace{2em}
\end{minipage}
%
    
    \newpage
    \small
    \setlength{\tabcolsep}{4pt}
    \begin{minipage}{\linewidth}
$\bullet\ $ $10_{1}$ \vspace{0.5em} \\
\begin{tabular}{l|llllll}
$k \setminus j$ & $-2$ & $0$ & $2$ & $4$ & $6$ & $8$ \\
\hline
$2$ & $1$ & $q^{-2}$ &  &  &  &  \\
$0$ &  & $2$ & $q^{-2}$ &  &  &  \\
$-2$ &  & $q^{2}$ & $2$ & $q^{-2}$ &  &  \\
$-4$ &  &  & $q^{2}$ & $2$ & $q^{-2}$ &  \\
$-6$ &  &  &  & $q^{2}$ & $1$ &  \\
$-8$ &  &  &  &  & $q^{2}$ & $1$ \\
\end{tabular}
\vspace{2em}
\end{minipage}
%
\begin{minipage}{\linewidth}
$\bullet\ $ $10_{2}$ \vspace{0.5em} \\
\begin{tabular}{l|lll}
$k \setminus j$ & $4$ & $6$ & $8$ \\
\hline
$8$ & $q^{-6}$ & $q^{-8}$ &  \\
$6$ &  & $q^{-6}$ &  \\
$4$ & $q^{-2}$ & $2q^{-4}$ & $q^{-6}$ \\
$2$ &  & $q^{-2}$ & $q^{-4}$ \\
$0$ & $q^{2}$ & $2$ & $q^{-2}$ \\
$-2$ &  & $q^{2}$ & $1$ \\
$-4$ & $q^{6}$ & $2q^{4}$ & $q^{2}$ \\
$-6$ &  & $q^{6}$ & $q^{4}$ \\
$-8$ &  & $q^{8}$ & $q^{6}$ \\
\end{tabular}
\vspace{2em}
\end{minipage}
%
\begin{minipage}{\linewidth}
$\bullet\ $ $10_{3}$ \vspace{0.5em} \\
\begin{tabular}{l|llllll}
$k \setminus j$ & $-4$ & $-2$ & $0$ & $2$ & $4$ & $6$ \\
\hline
$4$ & $1$ & $q^{-2}$ &  &  &  &  \\
$2$ &  & $2$ & $2q^{-2}$ &  &  &  \\
$0$ &  & $q^{2}$ & $4$ & $2q^{-2}$ &  &  \\
$-2$ &  &  & $2q^{2}$ & $3$ & $q^{-2}$ &  \\
$-4$ &  &  &  & $2q^{2}$ & $2$ &  \\
$-6$ &  &  &  &  & $q^{2}$ & $1$ \\
\end{tabular}
\vspace{2em}
\end{minipage}
%
\begin{minipage}{\linewidth}
$\bullet\ $ $10_{4}$ \vspace{0.5em} \\
\begin{tabular}{l|lllll}
$k \setminus j$ & $-4$ & $-2$ & $0$ & $2$ & $4$ \\
\hline
$4$ & $q^{-2}$ & $q^{-4}$ &  &  &  \\
$2$ & $1$ & $2q^{-2}$ & $q^{-4}$ &  &  \\
$0$ & $q^{2}$ & $2$ & $2q^{-2}$ & $q^{-4}$ &  \\
$-2$ &  & $2q^{2}$ & $2$ & $q^{-2}$ &  \\
$-4$ &  & $q^{4}$ & $2q^{2}$ & $2$ & $q^{-2}$ \\
$-6$ &  &  & $q^{4}$ & $q^{2}$ &  \\
$-8$ &  &  &  & $q^{4}$ & $q^{2}$ \\
\end{tabular}
\vspace{2em}
\end{minipage}
%
\begin{minipage}{\linewidth}
$\bullet\ $ $10_{5}$ \vspace{0.5em} \\
\begin{tabular}{l|lll}
$k \setminus j$ & $-6$ & $-4$ & $-2$ \\
\hline
$8$ & $q^{-6}$ & $q^{-8}$ &  \\
$6$ & $q^{-4}$ & $q^{-6}$ &  \\
$4$ & $2q^{-2}$ & $3q^{-4}$ & $q^{-6}$ \\
$2$ & $1$ & $2q^{-2}$ & $q^{-4}$ \\
$0$ & $2q^{2}$ & $3$ & $q^{-2}$ \\
$-2$ & $q^{4}$ & $2q^{2}$ & $1$ \\
$-4$ & $q^{6}$ & $3q^{4}$ & $q^{2}$ \\
$-6$ &  & $q^{6}$ & $q^{4}$ \\
$-8$ &  & $q^{8}$ & $q^{6}$ \\
\end{tabular}
\vspace{2em}
\end{minipage}
%
\begin{minipage}{\linewidth}
$\bullet\ $ $10_{6}$ \vspace{0.5em} \\
\begin{tabular}{l|llll}
$k \setminus j$ & $2$ & $4$ & $6$ & $8$ \\
\hline
$6$ & $q^{-4}$ & $q^{-6}$ &  &  \\
$4$ &  & $2q^{-4}$ & $q^{-6}$ &  \\
$2$ & $1$ & $3q^{-2}$ & $2q^{-4}$ &  \\
$0$ &  & $2$ & $3q^{-2}$ & $q^{-4}$ \\
$-2$ & $q^{4}$ & $3q^{2}$ & $3$ & $q^{-2}$ \\
$-4$ &  & $2q^{4}$ & $3q^{2}$ & $1$ \\
$-6$ &  & $q^{6}$ & $2q^{4}$ & $q^{2}$ \\
$-8$ &  &  & $q^{6}$ & $q^{4}$ \\
\end{tabular}
\vspace{2em}
\end{minipage}
%
\begin{minipage}{\linewidth}
$\bullet\ $ $10_{7}$ \vspace{0.5em} \\
\begin{tabular}{l|lllll}
$k \setminus j$ & $0$ & $2$ & $4$ & $6$ & $8$ \\
\hline
$4$ & $q^{-2}$ & $q^{-4}$ &  &  &  \\
$2$ & $1$ & $3q^{-2}$ & $q^{-4}$ &  &  \\
$0$ & $q^{2}$ & $4$ & $4q^{-2}$ & $q^{-4}$ &  \\
$-2$ &  & $3q^{2}$ & $5$ & $2q^{-2}$ &  \\
$-4$ &  & $q^{4}$ & $4q^{2}$ & $4$ & $q^{-2}$ \\
$-6$ &  &  & $q^{4}$ & $2q^{2}$ & $1$ \\
$-8$ &  &  &  & $q^{4}$ & $q^{2}$ \\
\end{tabular}
\vspace{2em}
\end{minipage}
%
\begin{minipage}{\linewidth}
$\bullet\ $ $10_{8}$ \vspace{0.5em} \\
\begin{tabular}{l|llll}
$k \setminus j$ & $0$ & $2$ & $4$ & $6$ \\
\hline
$8$ & $q^{-4}$ & $q^{-6}$ &  &  \\
$6$ &  & $q^{-4}$ & $q^{-6}$ &  \\
$4$ & $1$ & $2q^{-2}$ & $2q^{-4}$ &  \\
$2$ &  & $1$ & $2q^{-2}$ & $q^{-4}$ \\
$0$ & $q^{4}$ & $2q^{2}$ & $2$ & $q^{-2}$ \\
$-2$ &  & $q^{4}$ & $2q^{2}$ & $1$ \\
$-4$ &  & $q^{6}$ & $2q^{4}$ & $q^{2}$ \\
$-6$ &  &  & $q^{6}$ & $q^{4}$ \\
\end{tabular}
\vspace{2em}
\end{minipage}
%
\begin{minipage}{\linewidth}
$\bullet\ $ $10_{9}$ \vspace{0.5em} \\
\begin{tabular}{l|lll}
$k \setminus j$ & $0$ & $2$ & $4$ \\
\hline
$8$ & $q^{-6}$ & $q^{-8}$ &  \\
$6$ & $q^{-4}$ & $q^{-6}$ &  \\
$4$ & $2q^{-2}$ & $3q^{-4}$ & $q^{-6}$ \\
$2$ & $1$ & $3q^{-2}$ & $q^{-4}$ \\
$0$ & $2q^{2}$ & $4$ & $2q^{-2}$ \\
$-2$ & $q^{4}$ & $3q^{2}$ & $2$ \\
$-4$ & $q^{6}$ & $3q^{4}$ & $2q^{2}$ \\
$-6$ &  & $q^{6}$ & $q^{4}$ \\
$-8$ &  & $q^{8}$ & $q^{6}$ \\
\end{tabular}
\vspace{2em}
\end{minipage}
%
\begin{minipage}{\linewidth}
$\bullet\ $ $10_{10}$ \vspace{0.5em} \\
\begin{tabular}{l|lllll}
$k \setminus j$ & $-6$ & $-4$ & $-2$ & $0$ & $2$ \\
\hline
$8$ & $q^{-2}$ & $q^{-4}$ &  &  &  \\
$6$ & $1$ & $2q^{-2}$ & $q^{-4}$ &  &  \\
$4$ & $q^{2}$ & $4$ & $4q^{-2}$ & $q^{-4}$ &  \\
$2$ &  & $2q^{2}$ & $5$ & $3q^{-2}$ &  \\
$0$ &  & $q^{4}$ & $4q^{2}$ & $5$ & $q^{-2}$ \\
$-2$ &  &  & $q^{4}$ & $3q^{2}$ & $2$ \\
$-4$ &  &  &  & $q^{4}$ & $q^{2}$ \\
\end{tabular}
\vspace{2em}
\end{minipage}
%
\begin{minipage}{\linewidth}
$\bullet\ $ $10_{11}$ \vspace{0.5em} \\
\begin{tabular}{l|lllll}
$k \setminus j$ & $-2$ & $0$ & $2$ & $4$ & $6$ \\
\hline
$6$ & $q^{-2}$ & $q^{-4}$ &  &  &  \\
$4$ &  & $2q^{-2}$ & $2q^{-4}$ &  &  \\
$2$ & $q^{2}$ & $3$ & $4q^{-2}$ & $q^{-4}$ &  \\
$0$ &  & $2q^{2}$ & $5$ & $3q^{-2}$ &  \\
$-2$ &  & $q^{4}$ & $4q^{2}$ & $4$ & $q^{-2}$ \\
$-4$ &  &  & $2q^{4}$ & $3q^{2}$ & $1$ \\
$-6$ &  &  &  & $q^{4}$ & $q^{2}$ \\
\end{tabular}
\vspace{2em}
\end{minipage}
%
\begin{minipage}{\linewidth}
$\bullet\ $ $10_{12}$ \vspace{0.5em} \\
\begin{tabular}{l|llll}
$k \setminus j$ & $-6$ & $-4$ & $-2$ & $0$ \\
\hline
$8$ & $q^{-4}$ & $q^{-6}$ &  &  \\
$6$ & $q^{-2}$ & $2q^{-4}$ & $q^{-6}$ &  \\
$4$ & $2$ & $4q^{-2}$ & $2q^{-4}$ &  \\
$2$ & $q^{2}$ & $4$ & $4q^{-2}$ & $q^{-4}$ \\
$0$ & $q^{4}$ & $4q^{2}$ & $4$ & $q^{-2}$ \\
$-2$ &  & $2q^{4}$ & $4q^{2}$ & $1$ \\
$-4$ &  & $q^{6}$ & $2q^{4}$ & $q^{2}$ \\
$-6$ &  &  & $q^{6}$ & $q^{4}$ \\
\end{tabular}
\vspace{2em}
\end{minipage}
%
\begin{minipage}{\linewidth}
$\bullet\ $ $10_{13}$ \vspace{0.5em} \\
\begin{tabular}{l|llllll}
$k \setminus j$ & $-4$ & $-2$ & $0$ & $2$ & $4$ & $6$ \\
\hline
$4$ & $1$ & $2q^{-2}$ & $q^{-4}$ &  &  &  \\
$2$ &  & $4$ & $5q^{-2}$ & $q^{-4}$ &  &  \\
$0$ &  & $2q^{2}$ & $7$ & $4q^{-2}$ &  &  \\
$-2$ &  &  & $5q^{2}$ & $7$ & $2q^{-2}$ &  \\
$-4$ &  &  & $q^{4}$ & $4q^{2}$ & $3$ &  \\
$-6$ &  &  &  & $q^{4}$ & $2q^{2}$ & $1$ \\
\end{tabular}
\vspace{2em}
\end{minipage}
%
\begin{minipage}{\linewidth}
$\bullet\ $ $10_{14}$ \vspace{0.5em} \\
\begin{tabular}{l|llll}
$k \setminus j$ & $2$ & $4$ & $6$ & $8$ \\
\hline
$6$ & $q^{-4}$ & $q^{-6}$ &  &  \\
$4$ & $q^{-2}$ & $3q^{-4}$ & $q^{-6}$ &  \\
$2$ & $1$ & $4q^{-2}$ & $3q^{-4}$ &  \\
$0$ & $q^{2}$ & $5$ & $5q^{-2}$ & $q^{-4}$ \\
$-2$ & $q^{4}$ & $4q^{2}$ & $5$ & $2q^{-2}$ \\
$-4$ &  & $3q^{4}$ & $5q^{2}$ & $2$ \\
$-6$ &  & $q^{6}$ & $3q^{4}$ & $2q^{2}$ \\
$-8$ &  &  & $q^{6}$ & $q^{4}$ \\
\end{tabular}
\vspace{2em}
\end{minipage}
%
\begin{minipage}{\linewidth}
$\bullet\ $ $10_{15}$ \vspace{0.5em} \\
\begin{tabular}{l|llll}
$k \setminus j$ & $-4$ & $-2$ & $0$ & $2$ \\
\hline
$6$ & $q^{-4}$ & $q^{-6}$ &  &  \\
$4$ & $q^{-2}$ & $2q^{-4}$ & $q^{-6}$ &  \\
$2$ & $2$ & $4q^{-2}$ & $2q^{-4}$ &  \\
$0$ & $q^{2}$ & $3$ & $3q^{-2}$ & $q^{-4}$ \\
$-2$ & $q^{4}$ & $4q^{2}$ & $3$ & $q^{-2}$ \\
$-4$ &  & $2q^{4}$ & $3q^{2}$ & $1$ \\
$-6$ &  & $q^{6}$ & $2q^{4}$ & $q^{2}$ \\
$-8$ &  &  & $q^{6}$ & $q^{4}$ \\
\end{tabular}
\vspace{2em}
\end{minipage}
%
\begin{minipage}{\linewidth}
$\bullet\ $ $10_{16}$ \vspace{0.5em} \\
\begin{tabular}{l|lllll}
$k \setminus j$ & $-2$ & $0$ & $2$ & $4$ & $6$ \\
\hline
$6$ & $q^{-2}$ & $q^{-4}$ &  &  &  \\
$4$ & $1$ & $3q^{-2}$ & $2q^{-4}$ &  &  \\
$2$ & $q^{2}$ & $3$ & $4q^{-2}$ & $q^{-4}$ &  \\
$0$ &  & $3q^{2}$ & $6$ & $3q^{-2}$ &  \\
$-2$ &  & $q^{4}$ & $4q^{2}$ & $4$ & $q^{-2}$ \\
$-4$ &  &  & $2q^{4}$ & $3q^{2}$ & $1$ \\
$-6$ &  &  &  & $q^{4}$ & $q^{2}$ \\
\end{tabular}
\vspace{2em}
\end{minipage}
%
\begin{minipage}{\linewidth}
$\bullet\ $ $10_{17}$ \vspace{0.5em} \\
\begin{tabular}{l|lll}
$k \setminus j$ & $-2$ & $0$ & $2$ \\
\hline
$8$ & $q^{-6}$ & $q^{-8}$ &  \\
$6$ & $q^{-4}$ & $q^{-6}$ &  \\
$4$ & $2q^{-2}$ & $3q^{-4}$ & $q^{-6}$ \\
$2$ & $2$ & $3q^{-2}$ & $q^{-4}$ \\
$0$ & $2q^{2}$ & $5$ & $2q^{-2}$ \\
$-2$ & $q^{4}$ & $3q^{2}$ & $2$ \\
$-4$ & $q^{6}$ & $3q^{4}$ & $2q^{2}$ \\
$-6$ &  & $q^{6}$ & $q^{4}$ \\
$-8$ &  & $q^{8}$ & $q^{6}$ \\
\end{tabular}
\vspace{2em}
\end{minipage}
%
\begin{minipage}{\linewidth}
$\bullet\ $ $10_{18}$ \vspace{0.5em} \\
\begin{tabular}{l|lllll}
$k \setminus j$ & $-2$ & $0$ & $2$ & $4$ & $6$ \\
\hline
$6$ & $q^{-2}$ & $q^{-4}$ &  &  &  \\
$4$ & $1$ & $3q^{-2}$ & $2q^{-4}$ &  &  \\
$2$ & $q^{2}$ & $4$ & $5q^{-2}$ & $q^{-4}$ &  \\
$0$ &  & $3q^{2}$ & $7$ & $4q^{-2}$ &  \\
$-2$ &  & $q^{4}$ & $5q^{2}$ & $5$ & $q^{-2}$ \\
$-4$ &  &  & $2q^{4}$ & $4q^{2}$ & $2$ \\
$-6$ &  &  &  & $q^{4}$ & $q^{2}$ \\
\end{tabular}
\vspace{2em}
\end{minipage}
%
\begin{minipage}{\linewidth}
$\bullet\ $ $10_{19}$ \vspace{0.5em} \\
\begin{tabular}{l|llll}
$k \setminus j$ & $-2$ & $0$ & $2$ & $4$ \\
\hline
$8$ & $q^{-4}$ & $q^{-6}$ &  &  \\
$6$ & $q^{-2}$ & $2q^{-4}$ & $q^{-6}$ &  \\
$4$ & $1$ & $4q^{-2}$ & $3q^{-4}$ &  \\
$2$ & $q^{2}$ & $3$ & $4q^{-2}$ & $q^{-4}$ \\
$0$ & $q^{4}$ & $4q^{2}$ & $5$ & $2q^{-2}$ \\
$-2$ &  & $2q^{4}$ & $4q^{2}$ & $2$ \\
$-4$ &  & $q^{6}$ & $3q^{4}$ & $2q^{2}$ \\
$-6$ &  &  & $q^{6}$ & $q^{4}$ \\
\end{tabular}
\vspace{2em}
\end{minipage}
%
\begin{minipage}{\linewidth}
$\bullet\ $ $10_{20}$ \vspace{0.5em} \\
\begin{tabular}{l|lllll}
$k \setminus j$ & $0$ & $2$ & $4$ & $6$ & $8$ \\
\hline
$4$ & $q^{-2}$ & $q^{-4}$ &  &  &  \\
$2$ &  & $2q^{-2}$ & $q^{-4}$ &  &  \\
$0$ & $q^{2}$ & $3$ & $3q^{-2}$ & $q^{-4}$ &  \\
$-2$ &  & $2q^{2}$ & $4$ & $2q^{-2}$ &  \\
$-4$ &  & $q^{4}$ & $3q^{2}$ & $3$ & $q^{-2}$ \\
$-6$ &  &  & $q^{4}$ & $2q^{2}$ & $1$ \\
$-8$ &  &  &  & $q^{4}$ & $q^{2}$ \\
\end{tabular}
\vspace{2em}
\end{minipage}
%
\begin{minipage}{\linewidth}
$\bullet\ $ $10_{21}$ \vspace{0.5em} \\
\begin{tabular}{l|llll}
$k \setminus j$ & $2$ & $4$ & $6$ & $8$ \\
\hline
$6$ & $q^{-4}$ & $q^{-6}$ &  &  \\
$4$ & $q^{-2}$ & $3q^{-4}$ & $q^{-6}$ &  \\
$2$ & $1$ & $3q^{-2}$ & $2q^{-4}$ &  \\
$0$ & $q^{2}$ & $4$ & $4q^{-2}$ & $q^{-4}$ \\
$-2$ & $q^{4}$ & $3q^{2}$ & $3$ & $q^{-2}$ \\
$-4$ &  & $3q^{4}$ & $4q^{2}$ & $1$ \\
$-6$ &  & $q^{6}$ & $2q^{4}$ & $q^{2}$ \\
$-8$ &  &  & $q^{6}$ & $q^{4}$ \\
\end{tabular}
\vspace{2em}
\end{minipage}
%
\begin{minipage}{\linewidth}
$\bullet\ $ $10_{22}$ \vspace{0.5em} \\
\begin{tabular}{l|llll}
$k \setminus j$ & $-2$ & $0$ & $2$ & $4$ \\
\hline
$6$ & $q^{-4}$ & $q^{-6}$ &  &  \\
$4$ & $q^{-2}$ & $2q^{-4}$ & $q^{-6}$ &  \\
$2$ & $2$ & $4q^{-2}$ & $2q^{-4}$ &  \\
$0$ & $q^{2}$ & $5$ & $4q^{-2}$ & $q^{-4}$ \\
$-2$ & $q^{4}$ & $4q^{2}$ & $4$ & $q^{-2}$ \\
$-4$ &  & $2q^{4}$ & $4q^{2}$ & $2$ \\
$-6$ &  & $q^{6}$ & $2q^{4}$ & $q^{2}$ \\
$-8$ &  &  & $q^{6}$ & $q^{4}$ \\
\end{tabular}
\vspace{2em}
\end{minipage}
%
\begin{minipage}{\linewidth}
$\bullet\ $ $10_{23}$ \vspace{0.5em} \\
\begin{tabular}{l|llll}
$k \setminus j$ & $-6$ & $-4$ & $-2$ & $0$ \\
\hline
$8$ & $q^{-4}$ & $q^{-6}$ &  &  \\
$6$ & $q^{-2}$ & $2q^{-4}$ & $q^{-6}$ &  \\
$4$ & $2$ & $5q^{-2}$ & $3q^{-4}$ &  \\
$2$ & $q^{2}$ & $5$ & $5q^{-2}$ & $q^{-4}$ \\
$0$ & $q^{4}$ & $5q^{2}$ & $6$ & $2q^{-2}$ \\
$-2$ &  & $2q^{4}$ & $5q^{2}$ & $2$ \\
$-4$ &  & $q^{6}$ & $3q^{4}$ & $2q^{2}$ \\
$-6$ &  &  & $q^{6}$ & $q^{4}$ \\
\end{tabular}
\vspace{2em}
\end{minipage}
%
\begin{minipage}{\linewidth}
$\bullet\ $ $10_{24}$ \vspace{0.5em} \\
\begin{tabular}{l|lllll}
$k \setminus j$ & $0$ & $2$ & $4$ & $6$ & $8$ \\
\hline
$4$ & $q^{-2}$ & $q^{-4}$ &  &  &  \\
$2$ & $1$ & $4q^{-2}$ & $2q^{-4}$ &  &  \\
$0$ & $q^{2}$ & $5$ & $5q^{-2}$ & $q^{-4}$ &  \\
$-2$ &  & $4q^{2}$ & $7$ & $3q^{-2}$ &  \\
$-4$ &  & $q^{4}$ & $5q^{2}$ & $5$ & $q^{-2}$ \\
$-6$ &  &  & $2q^{4}$ & $3q^{2}$ & $1$ \\
$-8$ &  &  &  & $q^{4}$ & $q^{2}$ \\
\end{tabular}
\vspace{2em}
\end{minipage}
%
\begin{minipage}{\linewidth}
$\bullet\ $ $10_{25}$ \vspace{0.5em} \\
\begin{tabular}{l|llll}
$k \setminus j$ & $2$ & $4$ & $6$ & $8$ \\
\hline
$6$ & $q^{-4}$ & $q^{-6}$ &  &  \\
$4$ & $q^{-2}$ & $3q^{-4}$ & $q^{-6}$ &  \\
$2$ & $2$ & $5q^{-2}$ & $3q^{-4}$ &  \\
$0$ & $q^{2}$ & $6$ & $6q^{-2}$ & $q^{-4}$ \\
$-2$ & $q^{4}$ & $5q^{2}$ & $6$ & $2q^{-2}$ \\
$-4$ &  & $3q^{4}$ & $6q^{2}$ & $3$ \\
$-6$ &  & $q^{6}$ & $3q^{4}$ & $2q^{2}$ \\
$-8$ &  &  & $q^{6}$ & $q^{4}$ \\
\end{tabular}
\vspace{2em}
\end{minipage}
%
\begin{minipage}{\linewidth}
$\bullet\ $ $10_{26}$ \vspace{0.5em} \\
\begin{tabular}{l|llll}
$k \setminus j$ & $-2$ & $0$ & $2$ & $4$ \\
\hline
$6$ & $q^{-4}$ & $q^{-6}$ &  &  \\
$4$ & $2q^{-2}$ & $3q^{-4}$ & $q^{-6}$ &  \\
$2$ & $3$ & $5q^{-2}$ & $2q^{-4}$ &  \\
$0$ & $2q^{2}$ & $7$ & $5q^{-2}$ & $q^{-4}$ \\
$-2$ & $q^{4}$ & $5q^{2}$ & $5$ & $q^{-2}$ \\
$-4$ &  & $3q^{4}$ & $5q^{2}$ & $2$ \\
$-6$ &  & $q^{6}$ & $2q^{4}$ & $q^{2}$ \\
$-8$ &  &  & $q^{6}$ & $q^{4}$ \\
\end{tabular}
\vspace{2em}
\end{minipage}
%
\begin{minipage}{\linewidth}
$\bullet\ $ $10_{27}$ \vspace{0.5em} \\
\begin{tabular}{l|llll}
$k \setminus j$ & $-6$ & $-4$ & $-2$ & $0$ \\
\hline
$8$ & $q^{-4}$ & $q^{-6}$ &  &  \\
$6$ & $2q^{-2}$ & $3q^{-4}$ & $q^{-6}$ &  \\
$4$ & $3$ & $6q^{-2}$ & $3q^{-4}$ &  \\
$2$ & $2q^{2}$ & $7$ & $6q^{-2}$ & $q^{-4}$ \\
$0$ & $q^{4}$ & $6q^{2}$ & $7$ & $2q^{-2}$ \\
$-2$ &  & $3q^{4}$ & $6q^{2}$ & $2$ \\
$-4$ &  & $q^{6}$ & $3q^{4}$ & $2q^{2}$ \\
$-6$ &  &  & $q^{6}$ & $q^{4}$ \\
\end{tabular}
\vspace{2em}
\end{minipage}
%
\begin{minipage}{\linewidth}
$\bullet\ $ $10_{28}$ \vspace{0.5em} \\
\begin{tabular}{l|lllll}
$k \setminus j$ & $-6$ & $-4$ & $-2$ & $0$ & $2$ \\
\hline
$8$ & $q^{-2}$ & $q^{-4}$ &  &  &  \\
$6$ & $1$ & $3q^{-2}$ & $2q^{-4}$ &  &  \\
$4$ & $q^{2}$ & $4$ & $4q^{-2}$ & $q^{-4}$ &  \\
$2$ &  & $3q^{2}$ & $7$ & $4q^{-2}$ &  \\
$0$ &  & $q^{4}$ & $4q^{2}$ & $5$ & $q^{-2}$ \\
$-2$ &  &  & $2q^{4}$ & $4q^{2}$ & $2$ \\
$-4$ &  &  &  & $q^{4}$ & $q^{2}$ \\
\end{tabular}
\vspace{2em}
\end{minipage}
%
\begin{minipage}{\linewidth}
$\bullet\ $ $10_{29}$ \vspace{0.5em} \\
\begin{tabular}{l|lllll}
$k \setminus j$ & $-2$ & $0$ & $2$ & $4$ & $6$ \\
\hline
$6$ & $q^{-2}$ & $2q^{-4}$ & $q^{-6}$ &  &  \\
$4$ &  & $3q^{-2}$ & $3q^{-4}$ &  &  \\
$2$ & $q^{2}$ & $4$ & $6q^{-2}$ & $2q^{-4}$ &  \\
$0$ &  & $3q^{2}$ & $7$ & $4q^{-2}$ &  \\
$-2$ &  & $2q^{4}$ & $6q^{2}$ & $5$ & $q^{-2}$ \\
$-4$ &  &  & $3q^{4}$ & $4q^{2}$ & $1$ \\
$-6$ &  &  & $q^{6}$ & $2q^{4}$ & $q^{2}$ \\
\end{tabular}
\vspace{2em}
\end{minipage}
%
\begin{minipage}{\linewidth}
$\bullet\ $ $10_{30}$ \vspace{0.5em} \\
\begin{tabular}{l|lllll}
$k \setminus j$ & $0$ & $2$ & $4$ & $6$ & $8$ \\
\hline
$4$ & $q^{-2}$ & $q^{-4}$ &  &  &  \\
$2$ & $2$ & $5q^{-2}$ & $2q^{-4}$ &  &  \\
$0$ & $q^{2}$ & $6$ & $6q^{-2}$ & $q^{-4}$ &  \\
$-2$ &  & $5q^{2}$ & $9$ & $4q^{-2}$ &  \\
$-4$ &  & $q^{4}$ & $6q^{2}$ & $6$ & $q^{-2}$ \\
$-6$ &  &  & $2q^{4}$ & $4q^{2}$ & $2$ \\
$-8$ &  &  &  & $q^{4}$ & $q^{2}$ \\
\end{tabular}
\vspace{2em}
\end{minipage}
%
\begin{minipage}{\linewidth}
$\bullet\ $ $10_{31}$ \vspace{0.5em} \\
\begin{tabular}{l|lllll}
$k \setminus j$ & $-4$ & $-2$ & $0$ & $2$ & $4$ \\
\hline
$6$ & $q^{-2}$ & $q^{-4}$ &  &  &  \\
$4$ & $1$ & $3q^{-2}$ & $2q^{-4}$ &  &  \\
$2$ & $q^{2}$ & $5$ & $5q^{-2}$ & $q^{-4}$ &  \\
$0$ &  & $3q^{2}$ & $8$ & $4q^{-2}$ &  \\
$-2$ &  & $q^{4}$ & $5q^{2}$ & $5$ & $q^{-2}$ \\
$-4$ &  &  & $2q^{4}$ & $4q^{2}$ & $2$ \\
$-6$ &  &  &  & $q^{4}$ & $q^{2}$ \\
\end{tabular}
\vspace{2em}
\end{minipage}
%
\begin{minipage}{\linewidth}
$\bullet\ $ $10_{32}$ \vspace{0.5em} \\
\begin{tabular}{l|llll}
$k \setminus j$ & $-2$ & $0$ & $2$ & $4$ \\
\hline
$6$ & $q^{-4}$ & $q^{-6}$ &  &  \\
$4$ & $2q^{-2}$ & $3q^{-4}$ & $q^{-6}$ &  \\
$2$ & $3$ & $6q^{-2}$ & $3q^{-4}$ &  \\
$0$ & $2q^{2}$ & $7$ & $5q^{-2}$ & $q^{-4}$ \\
$-2$ & $q^{4}$ & $6q^{2}$ & $7$ & $2q^{-2}$ \\
$-4$ &  & $3q^{4}$ & $5q^{2}$ & $2$ \\
$-6$ &  & $q^{6}$ & $3q^{4}$ & $2q^{2}$ \\
$-8$ &  &  & $q^{6}$ & $q^{4}$ \\
\end{tabular}
\vspace{2em}
\end{minipage}
%
\begin{minipage}{\linewidth}
$\bullet\ $ $10_{33}$ \vspace{0.5em} \\
\begin{tabular}{l|lllll}
$k \setminus j$ & $-4$ & $-2$ & $0$ & $2$ & $4$ \\
\hline
$6$ & $q^{-2}$ & $q^{-4}$ &  &  &  \\
$4$ & $2$ & $4q^{-2}$ & $2q^{-4}$ &  &  \\
$2$ & $q^{2}$ & $6$ & $6q^{-2}$ & $q^{-4}$ &  \\
$0$ &  & $4q^{2}$ & $9$ & $4q^{-2}$ &  \\
$-2$ &  & $q^{4}$ & $6q^{2}$ & $6$ & $q^{-2}$ \\
$-4$ &  &  & $2q^{4}$ & $4q^{2}$ & $2$ \\
$-6$ &  &  &  & $q^{4}$ & $q^{2}$ \\
\end{tabular}
\vspace{2em}
\end{minipage}
%
\begin{minipage}{\linewidth}
$\bullet\ $ $10_{34}$ \vspace{0.5em} \\
\begin{tabular}{l|lllll}
$k \setminus j$ & $-6$ & $-4$ & $-2$ & $0$ & $2$ \\
\hline
$8$ & $q^{-2}$ & $q^{-4}$ &  &  &  \\
$6$ & $1$ & $2q^{-2}$ & $q^{-4}$ &  &  \\
$4$ & $q^{2}$ & $3$ & $3q^{-2}$ & $q^{-4}$ &  \\
$2$ &  & $2q^{2}$ & $4$ & $2q^{-2}$ &  \\
$0$ &  & $q^{4}$ & $3q^{2}$ & $4$ & $q^{-2}$ \\
$-2$ &  &  & $q^{4}$ & $2q^{2}$ & $1$ \\
$-4$ &  &  &  & $q^{4}$ & $q^{2}$ \\
\end{tabular}
\vspace{2em}
\end{minipage}
%
\begin{minipage}{\linewidth}
$\bullet\ $ $10_{35}$ \vspace{0.5em} \\
\begin{tabular}{l|llllll}
$k \setminus j$ & $-6$ & $-4$ & $-2$ & $0$ & $2$ & $4$ \\
\hline
$6$ & $1$ & $2q^{-2}$ & $q^{-4}$ &  &  &  \\
$4$ &  & $3$ & $4q^{-2}$ & $q^{-4}$ &  &  \\
$2$ &  & $2q^{2}$ & $6$ & $4q^{-2}$ &  &  \\
$0$ &  &  & $4q^{2}$ & $7$ & $2q^{-2}$ &  \\
$-2$ &  &  & $q^{4}$ & $4q^{2}$ & $3$ &  \\
$-4$ &  &  &  & $q^{4}$ & $2q^{2}$ & $1$ \\
\end{tabular}
\vspace{2em}
\end{minipage}
%
\begin{minipage}{\linewidth}
$\bullet\ $ $10_{36}$ \vspace{0.5em} \\
\begin{tabular}{l|lllll}
$k \setminus j$ & $0$ & $2$ & $4$ & $6$ & $8$ \\
\hline
$4$ & $q^{-2}$ & $q^{-4}$ &  &  &  \\
$2$ & $1$ & $3q^{-2}$ & $q^{-4}$ &  &  \\
$0$ & $q^{2}$ & $5$ & $5q^{-2}$ & $q^{-4}$ &  \\
$-2$ &  & $3q^{2}$ & $6$ & $3q^{-2}$ &  \\
$-4$ &  & $q^{4}$ & $5q^{2}$ & $5$ & $q^{-2}$ \\
$-6$ &  &  & $q^{4}$ & $3q^{2}$ & $2$ \\
$-8$ &  &  &  & $q^{4}$ & $q^{2}$ \\
\end{tabular}
\vspace{2em}
\end{minipage}
%
\begin{minipage}{\linewidth}
$\bullet\ $ $10_{37}$ \vspace{0.5em} \\
\begin{tabular}{l|lllll}
$k \setminus j$ & $-4$ & $-2$ & $0$ & $2$ & $4$ \\
\hline
$6$ & $q^{-2}$ & $q^{-4}$ &  &  &  \\
$4$ & $1$ & $3q^{-2}$ & $2q^{-4}$ &  &  \\
$2$ & $q^{2}$ & $5$ & $5q^{-2}$ & $q^{-4}$ &  \\
$0$ &  & $3q^{2}$ & $7$ & $3q^{-2}$ &  \\
$-2$ &  & $q^{4}$ & $5q^{2}$ & $5$ & $q^{-2}$ \\
$-4$ &  &  & $2q^{4}$ & $3q^{2}$ & $1$ \\
$-6$ &  &  &  & $q^{4}$ & $q^{2}$ \\
\end{tabular}
\vspace{2em}
\end{minipage}
%
\begin{minipage}{\linewidth}
$\bullet\ $ $10_{38}$ \vspace{0.5em} \\
\begin{tabular}{l|lllll}
$k \setminus j$ & $0$ & $2$ & $4$ & $6$ & $8$ \\
\hline
$4$ & $q^{-2}$ & $q^{-4}$ &  &  &  \\
$2$ & $1$ & $4q^{-2}$ & $2q^{-4}$ &  &  \\
$0$ & $q^{2}$ & $5$ & $5q^{-2}$ & $q^{-4}$ &  \\
$-2$ &  & $4q^{2}$ & $8$ & $4q^{-2}$ &  \\
$-4$ &  & $q^{4}$ & $5q^{2}$ & $5$ & $q^{-2}$ \\
$-6$ &  &  & $2q^{4}$ & $4q^{2}$ & $2$ \\
$-8$ &  &  &  & $q^{4}$ & $q^{2}$ \\
\end{tabular}
\vspace{2em}
\end{minipage}
%
\begin{minipage}{\linewidth}
$\bullet\ $ $10_{39}$ \vspace{0.5em} \\
\begin{tabular}{l|llll}
$k \setminus j$ & $2$ & $4$ & $6$ & $8$ \\
\hline
$6$ & $q^{-4}$ & $q^{-6}$ &  &  \\
$4$ & $q^{-2}$ & $3q^{-4}$ & $q^{-6}$ &  \\
$2$ & $2$ & $5q^{-2}$ & $3q^{-4}$ &  \\
$0$ & $q^{2}$ & $5$ & $5q^{-2}$ & $q^{-4}$ \\
$-2$ & $q^{4}$ & $5q^{2}$ & $6$ & $2q^{-2}$ \\
$-4$ &  & $3q^{4}$ & $5q^{2}$ & $2$ \\
$-6$ &  & $q^{6}$ & $3q^{4}$ & $2q^{2}$ \\
$-8$ &  &  & $q^{6}$ & $q^{4}$ \\
\end{tabular}
\vspace{2em}
\end{minipage}
%
\begin{minipage}{\linewidth}
$\bullet\ $ $10_{40}$ \vspace{0.5em} \\
\begin{tabular}{l|llll}
$k \setminus j$ & $-6$ & $-4$ & $-2$ & $0$ \\
\hline
$8$ & $q^{-4}$ & $q^{-6}$ &  &  \\
$6$ & $2q^{-2}$ & $3q^{-4}$ & $q^{-6}$ &  \\
$4$ & $3$ & $6q^{-2}$ & $3q^{-4}$ &  \\
$2$ & $2q^{2}$ & $8$ & $7q^{-2}$ & $q^{-4}$ \\
$0$ & $q^{4}$ & $6q^{2}$ & $7$ & $2q^{-2}$ \\
$-2$ &  & $3q^{4}$ & $7q^{2}$ & $3$ \\
$-4$ &  & $q^{6}$ & $3q^{4}$ & $2q^{2}$ \\
$-6$ &  &  & $q^{6}$ & $q^{4}$ \\
\end{tabular}
\vspace{2em}
\end{minipage}
%
\begin{minipage}{\linewidth}
$\bullet\ $ $10_{41}$ \vspace{0.5em} \\
\begin{tabular}{l|lllll}
$k \setminus j$ & $-2$ & $0$ & $2$ & $4$ & $6$ \\
\hline
$6$ & $q^{-2}$ & $2q^{-4}$ & $q^{-6}$ &  &  \\
$4$ & $1$ & $4q^{-2}$ & $3q^{-4}$ &  &  \\
$2$ & $q^{2}$ & $5$ & $7q^{-2}$ & $2q^{-4}$ &  \\
$0$ &  & $4q^{2}$ & $8$ & $4q^{-2}$ &  \\
$-2$ &  & $2q^{4}$ & $7q^{2}$ & $6$ & $q^{-2}$ \\
$-4$ &  &  & $3q^{4}$ & $4q^{2}$ & $1$ \\
$-6$ &  &  & $q^{6}$ & $2q^{4}$ & $q^{2}$ \\
\end{tabular}
\vspace{2em}
\end{minipage}
%
\begin{minipage}{\linewidth}
$\bullet\ $ $10_{42}$ \vspace{0.5em} \\
\begin{tabular}{l|lllll}
$k \setminus j$ & $-4$ & $-2$ & $0$ & $2$ & $4$ \\
\hline
$6$ & $q^{-2}$ & $2q^{-4}$ & $q^{-6}$ &  &  \\
$4$ & $1$ & $4q^{-2}$ & $3q^{-4}$ &  &  \\
$2$ & $q^{2}$ & $7$ & $8q^{-2}$ & $2q^{-4}$ &  \\
$0$ &  & $4q^{2}$ & $10$ & $5q^{-2}$ &  \\
$-2$ &  & $2q^{4}$ & $8q^{2}$ & $7$ & $q^{-2}$ \\
$-4$ &  &  & $3q^{4}$ & $5q^{2}$ & $2$ \\
$-6$ &  &  & $q^{6}$ & $2q^{4}$ & $q^{2}$ \\
\end{tabular}
\vspace{2em}
\end{minipage}
%
\begin{minipage}{\linewidth}
$\bullet\ $ $10_{43}$ \vspace{0.5em} \\
\begin{tabular}{l|lllll}
$k \setminus j$ & $-4$ & $-2$ & $0$ & $2$ & $4$ \\
\hline
$6$ & $q^{-2}$ & $2q^{-4}$ & $q^{-6}$ &  &  \\
$4$ & $1$ & $4q^{-2}$ & $3q^{-4}$ &  &  \\
$2$ & $q^{2}$ & $6$ & $7q^{-2}$ & $2q^{-4}$ &  \\
$0$ &  & $4q^{2}$ & $9$ & $4q^{-2}$ &  \\
$-2$ &  & $2q^{4}$ & $7q^{2}$ & $6$ & $q^{-2}$ \\
$-4$ &  &  & $3q^{4}$ & $4q^{2}$ & $1$ \\
$-6$ &  &  & $q^{6}$ & $2q^{4}$ & $q^{2}$ \\
\end{tabular}
\vspace{2em}
\end{minipage}
%
\begin{minipage}{\linewidth}
$\bullet\ $ $10_{44}$ \vspace{0.5em} \\
\begin{tabular}{l|lllll}
$k \setminus j$ & $-2$ & $0$ & $2$ & $4$ & $6$ \\
\hline
$6$ & $q^{-2}$ & $2q^{-4}$ & $q^{-6}$ &  &  \\
$4$ & $1$ & $4q^{-2}$ & $3q^{-4}$ &  &  \\
$2$ & $q^{2}$ & $6$ & $8q^{-2}$ & $2q^{-4}$ &  \\
$0$ &  & $4q^{2}$ & $9$ & $5q^{-2}$ &  \\
$-2$ &  & $2q^{4}$ & $8q^{2}$ & $7$ & $q^{-2}$ \\
$-4$ &  &  & $3q^{4}$ & $5q^{2}$ & $2$ \\
$-6$ &  &  & $q^{6}$ & $2q^{4}$ & $q^{2}$ \\
\end{tabular}
\vspace{2em}
\end{minipage}
%
\begin{minipage}{\linewidth}
$\bullet\ $ $10_{45}$ \vspace{0.5em} \\
\begin{tabular}{l|lllll}
$k \setminus j$ & $-4$ & $-2$ & $0$ & $2$ & $4$ \\
\hline
$6$ & $q^{-2}$ & $2q^{-4}$ & $q^{-6}$ &  &  \\
$4$ & $2$ & $5q^{-2}$ & $3q^{-4}$ &  &  \\
$2$ & $q^{2}$ & $8$ & $9q^{-2}$ & $2q^{-4}$ &  \\
$0$ &  & $5q^{2}$ & $11$ & $5q^{-2}$ &  \\
$-2$ &  & $2q^{4}$ & $9q^{2}$ & $8$ & $q^{-2}$ \\
$-4$ &  &  & $3q^{4}$ & $5q^{2}$ & $2$ \\
$-6$ &  &  & $q^{6}$ & $2q^{4}$ & $q^{2}$ \\
\end{tabular}
\vspace{2em}
\end{minipage}
%
\begin{minipage}{\linewidth}
$\bullet\ $ $10_{46}$ \vspace{0.5em} \\
\begin{tabular}{l|lll}
$k \setminus j$ & $4$ & $6$ & $8$ \\
\hline
$8$ & $q^{-6}$ & $q^{-8}$ &  \\
$6$ &  & $q^{-6}$ &  \\
$4$ & $2q^{-2}$ & $3q^{-4}$ & $q^{-6}$ \\
$2$ &  & $q^{-2}$ & $q^{-4}$ \\
$0$ & $2q^{2}$ & $4$ & $2q^{-2}$ \\
$-2$ &  & $q^{2}$ & $1$ \\
$-4$ & $q^{6}$ & $3q^{4}$ & $2q^{2}$ \\
$-6$ &  & $q^{6}$ & $q^{4}$ \\
$-8$ &  & $q^{8}$ & $q^{6}$ \\
\end{tabular}
\vspace{2em}
\end{minipage}
%
\begin{minipage}{\linewidth}
$\bullet\ $ $10_{47}$ \vspace{0.5em} \\
\begin{tabular}{l|lll}
$k \setminus j$ & $-6$ & $-4$ & $-2$ \\
\hline
$8$ & $q^{-6}$ & $q^{-8}$ &  \\
$6$ & $q^{-4}$ & $q^{-6}$ &  \\
$4$ & $3q^{-2}$ & $4q^{-4}$ & $q^{-6}$ \\
$2$ & $1$ & $2q^{-2}$ & $q^{-4}$ \\
$0$ & $3q^{2}$ & $5$ & $2q^{-2}$ \\
$-2$ & $q^{4}$ & $2q^{2}$ & $1$ \\
$-4$ & $q^{6}$ & $4q^{4}$ & $2q^{2}$ \\
$-6$ &  & $q^{6}$ & $q^{4}$ \\
$-8$ &  & $q^{8}$ & $q^{6}$ \\
\end{tabular}
\vspace{2em}
\end{minipage}
%
\begin{minipage}{\linewidth}
$\bullet\ $ $10_{48}$ \vspace{0.5em} \\
\begin{tabular}{l|lll}
$k \setminus j$ & $-2$ & $0$ & $2$ \\
\hline
$8$ & $q^{-6}$ & $q^{-8}$ &  \\
$6$ & $q^{-4}$ & $q^{-6}$ &  \\
$4$ & $3q^{-2}$ & $4q^{-4}$ & $q^{-6}$ \\
$2$ & $2$ & $3q^{-2}$ & $q^{-4}$ \\
$0$ & $3q^{2}$ & $7$ & $3q^{-2}$ \\
$-2$ & $q^{4}$ & $3q^{2}$ & $2$ \\
$-4$ & $q^{6}$ & $4q^{4}$ & $3q^{2}$ \\
$-6$ &  & $q^{6}$ & $q^{4}$ \\
$-8$ &  & $q^{8}$ & $q^{6}$ \\
\end{tabular}
\vspace{2em}
\end{minipage}
%
\begin{minipage}{\linewidth}
$\bullet\ $ $10_{49}$ \vspace{0.5em} \\
\begin{tabular}{l|llll}
$k \setminus j$ & $6$ & $8$ & $10$ & $12$ \\
\hline
$6$ & $q^{-6}$ &  &  &  \\
$4$ & $2q^{-4}$ & $2q^{-6}$ &  &  \\
$2$ & $3q^{-2}$ & $3q^{-4}$ &  &  \\
$0$ & $3$ & $6q^{-2}$ & $3q^{-4}$ &  \\
$-2$ & $3q^{2}$ & $5$ & $2q^{-2}$ &  \\
$-4$ & $2q^{4}$ & $6q^{2}$ & $5$ & $q^{-2}$ \\
$-6$ & $q^{6}$ & $3q^{4}$ & $2q^{2}$ &  \\
$-8$ &  & $2q^{6}$ & $3q^{4}$ & $q^{2}$ \\
\end{tabular}
\vspace{2em}
\end{minipage}
%
\begin{minipage}{\linewidth}
$\bullet\ $ $10_{50}$ \vspace{0.5em} \\
\begin{tabular}{l|llll}
$k \setminus j$ & $2$ & $4$ & $6$ & $8$ \\
\hline
$6$ & $q^{-4}$ & $q^{-6}$ &  &  \\
$4$ & $q^{-2}$ & $3q^{-4}$ & $q^{-6}$ &  \\
$2$ & $2$ & $4q^{-2}$ & $2q^{-4}$ &  \\
$0$ & $q^{2}$ & $5$ & $5q^{-2}$ & $q^{-4}$ \\
$-2$ & $q^{4}$ & $4q^{2}$ & $4$ & $q^{-2}$ \\
$-4$ &  & $3q^{4}$ & $5q^{2}$ & $2$ \\
$-6$ &  & $q^{6}$ & $2q^{4}$ & $q^{2}$ \\
$-8$ &  &  & $q^{6}$ & $q^{4}$ \\
\end{tabular}
\vspace{2em}
\end{minipage}
%
\begin{minipage}{\linewidth}
$\bullet\ $ $10_{51}$ \vspace{0.5em} \\
\begin{tabular}{l|llll}
$k \setminus j$ & $-6$ & $-4$ & $-2$ & $0$ \\
\hline
$8$ & $q^{-4}$ & $q^{-6}$ &  &  \\
$6$ & $q^{-2}$ & $2q^{-4}$ & $q^{-6}$ &  \\
$4$ & $3$ & $6q^{-2}$ & $3q^{-4}$ &  \\
$2$ & $q^{2}$ & $6$ & $6q^{-2}$ & $q^{-4}$ \\
$0$ & $q^{4}$ & $6q^{2}$ & $7$ & $2q^{-2}$ \\
$-2$ &  & $2q^{4}$ & $6q^{2}$ & $3$ \\
$-4$ &  & $q^{6}$ & $3q^{4}$ & $2q^{2}$ \\
$-6$ &  &  & $q^{6}$ & $q^{4}$ \\
\end{tabular}
\vspace{2em}
\end{minipage}
%
\begin{minipage}{\linewidth}
$\bullet\ $ $10_{52}$ \vspace{0.5em} \\
\begin{tabular}{l|llll}
$k \setminus j$ & $-2$ & $0$ & $2$ & $4$ \\
\hline
$8$ & $q^{-4}$ & $q^{-6}$ &  &  \\
$6$ & $q^{-2}$ & $2q^{-4}$ & $q^{-6}$ &  \\
$4$ & $2$ & $5q^{-2}$ & $3q^{-4}$ &  \\
$2$ & $q^{2}$ & $4$ & $5q^{-2}$ & $q^{-4}$ \\
$0$ & $q^{4}$ & $5q^{2}$ & $6$ & $2q^{-2}$ \\
$-2$ &  & $2q^{4}$ & $5q^{2}$ & $3$ \\
$-4$ &  & $q^{6}$ & $3q^{4}$ & $2q^{2}$ \\
$-6$ &  &  & $q^{6}$ & $q^{4}$ \\
\end{tabular}
\vspace{2em}
\end{minipage}
%
\begin{minipage}{\linewidth}
$\bullet\ $ $10_{53}$ \vspace{0.5em} \\
\begin{tabular}{l|lllll}
$k \setminus j$ & $4$ & $6$ & $8$ & $10$ & $12$ \\
\hline
$4$ & $q^{-4}$ &  &  &  &  \\
$2$ & $3q^{-2}$ & $3q^{-4}$ &  &  &  \\
$0$ & $4$ & $6q^{-2}$ & $2q^{-4}$ &  &  \\
$-2$ & $3q^{2}$ & $9$ & $6q^{-2}$ &  &  \\
$-4$ & $q^{4}$ & $6q^{2}$ & $8$ & $3q^{-2}$ &  \\
$-6$ &  & $3q^{4}$ & $6q^{2}$ & $3$ &  \\
$-8$ &  &  & $2q^{4}$ & $3q^{2}$ & $1$ \\
\end{tabular}
\vspace{2em}
\end{minipage}
%
\begin{minipage}{\linewidth}
$\bullet\ $ $10_{54}$ \vspace{0.5em} \\
\begin{tabular}{l|llll}
$k \setminus j$ & $-4$ & $-2$ & $0$ & $2$ \\
\hline
$6$ & $q^{-4}$ & $q^{-6}$ &  &  \\
$4$ & $q^{-2}$ & $2q^{-4}$ & $q^{-6}$ &  \\
$2$ & $2$ & $4q^{-2}$ & $2q^{-4}$ &  \\
$0$ & $q^{2}$ & $4$ & $4q^{-2}$ & $q^{-4}$ \\
$-2$ & $q^{4}$ & $4q^{2}$ & $3$ & $q^{-2}$ \\
$-4$ &  & $2q^{4}$ & $4q^{2}$ & $2$ \\
$-6$ &  & $q^{6}$ & $2q^{4}$ & $q^{2}$ \\
$-8$ &  &  & $q^{6}$ & $q^{4}$ \\
\end{tabular}
\vspace{2em}
\end{minipage}
%
\begin{minipage}{\linewidth}
$\bullet\ $ $10_{55}$ \vspace{0.5em} \\
\begin{tabular}{l|lllll}
$k \setminus j$ & $4$ & $6$ & $8$ & $10$ & $12$ \\
\hline
$4$ & $q^{-4}$ &  &  &  &  \\
$2$ & $2q^{-2}$ & $2q^{-4}$ &  &  &  \\
$0$ & $3$ & $5q^{-2}$ & $2q^{-4}$ &  &  \\
$-2$ & $2q^{2}$ & $7$ & $5q^{-2}$ &  &  \\
$-4$ & $q^{4}$ & $5q^{2}$ & $7$ & $3q^{-2}$ &  \\
$-6$ &  & $2q^{4}$ & $5q^{2}$ & $3$ &  \\
$-8$ &  &  & $2q^{4}$ & $3q^{2}$ & $1$ \\
\end{tabular}
\vspace{2em}
\end{minipage}
%
\begin{minipage}{\linewidth}
$\bullet\ $ $10_{56}$ \vspace{0.5em} \\
\begin{tabular}{l|llll}
$k \setminus j$ & $2$ & $4$ & $6$ & $8$ \\
\hline
$6$ & $q^{-4}$ & $q^{-6}$ &  &  \\
$4$ & $q^{-2}$ & $3q^{-4}$ & $q^{-6}$ &  \\
$2$ & $2$ & $5q^{-2}$ & $3q^{-4}$ &  \\
$0$ & $q^{2}$ & $6$ & $6q^{-2}$ & $q^{-4}$ \\
$-2$ & $q^{4}$ & $5q^{2}$ & $6$ & $2q^{-2}$ \\
$-4$ &  & $3q^{4}$ & $6q^{2}$ & $3$ \\
$-6$ &  & $q^{6}$ & $3q^{4}$ & $2q^{2}$ \\
$-8$ &  &  & $q^{6}$ & $q^{4}$ \\
\end{tabular}
\vspace{2em}
\end{minipage}
%
\begin{minipage}{\linewidth}
$\bullet\ $ $10_{57}$ \vspace{0.5em} \\
\begin{tabular}{l|llll}
$k \setminus j$ & $-6$ & $-4$ & $-2$ & $0$ \\
\hline
$8$ & $q^{-4}$ & $q^{-6}$ &  &  \\
$6$ & $2q^{-2}$ & $3q^{-4}$ & $q^{-6}$ &  \\
$4$ & $4$ & $7q^{-2}$ & $3q^{-4}$ &  \\
$2$ & $2q^{2}$ & $8$ & $7q^{-2}$ & $q^{-4}$ \\
$0$ & $q^{4}$ & $7q^{2}$ & $8$ & $2q^{-2}$ \\
$-2$ &  & $3q^{4}$ & $7q^{2}$ & $3$ \\
$-4$ &  & $q^{6}$ & $3q^{4}$ & $2q^{2}$ \\
$-6$ &  &  & $q^{6}$ & $q^{4}$ \\
\end{tabular}
\vspace{2em}
\end{minipage}
%
\begin{minipage}{\linewidth}
$\bullet\ $ $10_{58}$ \vspace{0.5em} \\
\begin{tabular}{l|llllll}
$k \setminus j$ & $-4$ & $-2$ & $0$ & $2$ & $4$ & $6$ \\
\hline
$4$ & $1$ & $2q^{-2}$ & $q^{-4}$ &  &  &  \\
$2$ &  & $4$ & $6q^{-2}$ & $2q^{-4}$ &  &  \\
$0$ &  & $2q^{2}$ & $8$ & $5q^{-2}$ &  &  \\
$-2$ &  &  & $6q^{2}$ & $9$ & $3q^{-2}$ &  \\
$-4$ &  &  & $q^{4}$ & $5q^{2}$ & $4$ &  \\
$-6$ &  &  &  & $2q^{4}$ & $3q^{2}$ & $1$ \\
\end{tabular}
\vspace{2em}
\end{minipage}
%
\begin{minipage}{\linewidth}
$\bullet\ $ $10_{59}$ \vspace{0.5em} \\
\begin{tabular}{l|lllll}
$k \setminus j$ & $-2$ & $0$ & $2$ & $4$ & $6$ \\
\hline
$6$ & $q^{-2}$ & $2q^{-4}$ & $q^{-6}$ &  &  \\
$4$ & $1$ & $4q^{-2}$ & $3q^{-4}$ &  &  \\
$2$ & $q^{2}$ & $6$ & $8q^{-2}$ & $2q^{-4}$ &  \\
$0$ &  & $4q^{2}$ & $8$ & $4q^{-2}$ &  \\
$-2$ &  & $2q^{4}$ & $8q^{2}$ & $7$ & $q^{-2}$ \\
$-4$ &  &  & $3q^{4}$ & $4q^{2}$ & $1$ \\
$-6$ &  &  & $q^{6}$ & $2q^{4}$ & $q^{2}$ \\
\end{tabular}
\vspace{2em}
\end{minipage}
%
\begin{minipage}{\linewidth}
$\bullet\ $ $10_{60}$ \vspace{0.5em} \\
\begin{tabular}{l|lllll}
$k \setminus j$ & $-6$ & $-4$ & $-2$ & $0$ & $2$ \\
\hline
$6$ & $1$ & $3q^{-2}$ & $3q^{-4}$ & $q^{-6}$ &  \\
$4$ &  & $3$ & $6q^{-2}$ & $3q^{-4}$ &  \\
$2$ &  & $3q^{2}$ & $10$ & $8q^{-2}$ & $q^{-4}$ \\
$0$ &  &  & $6q^{2}$ & $10$ & $3q^{-2}$ \\
$-2$ &  &  & $3q^{4}$ & $8q^{2}$ & $5$ \\
$-4$ &  &  &  & $3q^{4}$ & $3q^{2}$ \\
$-6$ &  &  &  & $q^{6}$ & $q^{4}$ \\
\end{tabular}
\vspace{2em}
\end{minipage}
%
\begin{minipage}{\linewidth}
$\bullet\ $ $10_{61}$ \vspace{0.5em} \\
\begin{tabular}{l|llll}
$k \setminus j$ & $0$ & $2$ & $4$ & $6$ \\
\hline
$8$ & $q^{-4}$ & $q^{-6}$ &  &  \\
$6$ &  & $q^{-4}$ & $q^{-6}$ &  \\
$4$ & $2$ & $3q^{-2}$ & $2q^{-4}$ &  \\
$2$ &  & $1$ & $2q^{-2}$ & $q^{-4}$ \\
$0$ & $q^{4}$ & $3q^{2}$ & $3$ & $q^{-2}$ \\
$-2$ &  & $q^{4}$ & $2q^{2}$ & $1$ \\
$-4$ &  & $q^{6}$ & $2q^{4}$ & $q^{2}$ \\
$-6$ &  &  & $q^{6}$ & $q^{4}$ \\
\end{tabular}
\vspace{2em}
\end{minipage}
%
\begin{minipage}{\linewidth}
$\bullet\ $ $10_{62}$ \vspace{0.5em} \\
\begin{tabular}{l|lll}
$k \setminus j$ & $-6$ & $-4$ & $-2$ \\
\hline
$8$ & $q^{-6}$ & $q^{-8}$ &  \\
$6$ & $q^{-4}$ & $q^{-6}$ &  \\
$4$ & $3q^{-2}$ & $4q^{-4}$ & $q^{-6}$ \\
$2$ & $2$ & $3q^{-2}$ & $q^{-4}$ \\
$0$ & $3q^{2}$ & $5$ & $2q^{-2}$ \\
$-2$ & $q^{4}$ & $3q^{2}$ & $2$ \\
$-4$ & $q^{6}$ & $4q^{4}$ & $2q^{2}$ \\
$-6$ &  & $q^{6}$ & $q^{4}$ \\
$-8$ &  & $q^{8}$ & $q^{6}$ \\
\end{tabular}
\vspace{2em}
\end{minipage}
%
\begin{minipage}{\linewidth}
$\bullet\ $ $10_{63}$ \vspace{0.5em} \\
\begin{tabular}{l|lllll}
$k \setminus j$ & $4$ & $6$ & $8$ & $10$ & $12$ \\
\hline
$4$ & $q^{-4}$ &  &  &  &  \\
$2$ & $2q^{-2}$ & $2q^{-4}$ &  &  &  \\
$0$ & $3$ & $5q^{-2}$ & $2q^{-4}$ &  &  \\
$-2$ & $2q^{2}$ & $6$ & $4q^{-2}$ &  &  \\
$-4$ & $q^{4}$ & $5q^{2}$ & $7$ & $3q^{-2}$ &  \\
$-6$ &  & $2q^{4}$ & $4q^{2}$ & $2$ &  \\
$-8$ &  &  & $2q^{4}$ & $3q^{2}$ & $1$ \\
\end{tabular}
\vspace{2em}
\end{minipage}
%
\begin{minipage}{\linewidth}
$\bullet\ $ $10_{64}$ \vspace{0.5em} \\
\begin{tabular}{l|lll}
$k \setminus j$ & $0$ & $2$ & $4$ \\
\hline
$8$ & $q^{-6}$ & $q^{-8}$ &  \\
$6$ & $q^{-4}$ & $q^{-6}$ &  \\
$4$ & $3q^{-2}$ & $4q^{-4}$ & $q^{-6}$ \\
$2$ & $2$ & $4q^{-2}$ & $q^{-4}$ \\
$0$ & $3q^{2}$ & $6$ & $3q^{-2}$ \\
$-2$ & $q^{4}$ & $4q^{2}$ & $3$ \\
$-4$ & $q^{6}$ & $4q^{4}$ & $3q^{2}$ \\
$-6$ &  & $q^{6}$ & $q^{4}$ \\
$-8$ &  & $q^{8}$ & $q^{6}$ \\
\end{tabular}
\vspace{2em}
\end{minipage}
%
\begin{minipage}{\linewidth}
$\bullet\ $ $10_{65}$ \vspace{0.5em} \\
\begin{tabular}{l|llll}
$k \setminus j$ & $-6$ & $-4$ & $-2$ & $0$ \\
\hline
$8$ & $q^{-4}$ & $q^{-6}$ &  &  \\
$6$ & $q^{-2}$ & $2q^{-4}$ & $q^{-6}$ &  \\
$4$ & $3$ & $6q^{-2}$ & $3q^{-4}$ &  \\
$2$ & $q^{2}$ & $5$ & $5q^{-2}$ & $q^{-4}$ \\
$0$ & $q^{4}$ & $6q^{2}$ & $7$ & $2q^{-2}$ \\
$-2$ &  & $2q^{4}$ & $5q^{2}$ & $2$ \\
$-4$ &  & $q^{6}$ & $3q^{4}$ & $2q^{2}$ \\
$-6$ &  &  & $q^{6}$ & $q^{4}$ \\
\end{tabular}
\vspace{2em}
\end{minipage}
%
\begin{minipage}{\linewidth}
$\bullet\ $ $10_{66}$ \vspace{0.5em} \\
\begin{tabular}{l|llll}
$k \setminus j$ & $6$ & $8$ & $10$ & $12$ \\
\hline
$6$ & $q^{-6}$ &  &  &  \\
$4$ & $2q^{-4}$ & $2q^{-6}$ &  &  \\
$2$ & $4q^{-2}$ & $4q^{-4}$ &  &  \\
$0$ & $4$ & $7q^{-2}$ & $3q^{-4}$ &  \\
$-2$ & $4q^{2}$ & $8$ & $4q^{-2}$ &  \\
$-4$ & $2q^{4}$ & $7q^{2}$ & $6$ & $q^{-2}$ \\
$-6$ & $q^{6}$ & $4q^{4}$ & $4q^{2}$ & $1$ \\
$-8$ &  & $2q^{6}$ & $3q^{4}$ & $q^{2}$ \\
\end{tabular}
\vspace{2em}
\end{minipage}
%
\begin{minipage}{\linewidth}
$\bullet\ $ $10_{67}$ \vspace{0.5em} \\
\begin{tabular}{l|lllll}
$k \setminus j$ & $0$ & $2$ & $4$ & $6$ & $8$ \\
\hline
$4$ & $q^{-2}$ & $q^{-4}$ &  &  &  \\
$2$ & $1$ & $4q^{-2}$ & $2q^{-4}$ &  &  \\
$0$ & $q^{2}$ & $6$ & $6q^{-2}$ & $q^{-4}$ &  \\
$-2$ &  & $4q^{2}$ & $8$ & $4q^{-2}$ &  \\
$-4$ &  & $q^{4}$ & $6q^{2}$ & $6$ & $q^{-2}$ \\
$-6$ &  &  & $2q^{4}$ & $4q^{2}$ & $2$ \\
$-8$ &  &  &  & $q^{4}$ & $q^{2}$ \\
\end{tabular}
\vspace{2em}
\end{minipage}
%
\begin{minipage}{\linewidth}
$\bullet\ $ $10_{68}$ \vspace{0.5em} \\
\begin{tabular}{l|lllll}
$k \setminus j$ & $-6$ & $-4$ & $-2$ & $0$ & $2$ \\
\hline
$8$ & $q^{-2}$ & $q^{-4}$ &  &  &  \\
$6$ & $1$ & $3q^{-2}$ & $2q^{-4}$ &  &  \\
$4$ & $q^{2}$ & $5$ & $5q^{-2}$ & $q^{-4}$ &  \\
$2$ &  & $3q^{2}$ & $7$ & $4q^{-2}$ &  \\
$0$ &  & $q^{4}$ & $5q^{2}$ & $6$ & $q^{-2}$ \\
$-2$ &  &  & $2q^{4}$ & $4q^{2}$ & $2$ \\
$-4$ &  &  &  & $q^{4}$ & $q^{2}$ \\
\end{tabular}
\vspace{2em}
\end{minipage}
%
\begin{minipage}{\linewidth}
$\bullet\ $ $10_{69}$ \vspace{0.5em} \\
\begin{tabular}{l|lllll}
$k \setminus j$ & $-8$ & $-6$ & $-4$ & $-2$ & $0$ \\
\hline
$6$ & $1$ & $3q^{-2}$ & $3q^{-4}$ & $q^{-6}$ &  \\
$4$ &  & $4$ & $7q^{-2}$ & $3q^{-4}$ &  \\
$2$ &  & $3q^{2}$ & $10$ & $8q^{-2}$ & $q^{-4}$ \\
$0$ &  &  & $7q^{2}$ & $10$ & $3q^{-2}$ \\
$-2$ &  &  & $3q^{4}$ & $8q^{2}$ & $4$ \\
$-4$ &  &  &  & $3q^{4}$ & $3q^{2}$ \\
$-6$ &  &  &  & $q^{6}$ & $q^{4}$ \\
\end{tabular}
\vspace{2em}
\end{minipage}
%
\begin{minipage}{\linewidth}
$\bullet\ $ $10_{70}$ \vspace{0.5em} \\
\begin{tabular}{l|lllll}
$k \setminus j$ & $-6$ & $-4$ & $-2$ & $0$ & $2$ \\
\hline
$6$ & $q^{-2}$ & $2q^{-4}$ & $q^{-6}$ &  &  \\
$4$ & $1$ & $4q^{-2}$ & $3q^{-4}$ &  &  \\
$2$ & $q^{2}$ & $6$ & $7q^{-2}$ & $2q^{-4}$ &  \\
$0$ &  & $4q^{2}$ & $7$ & $3q^{-2}$ &  \\
$-2$ &  & $2q^{4}$ & $7q^{2}$ & $5$ & $q^{-2}$ \\
$-4$ &  &  & $3q^{4}$ & $3q^{2}$ &  \\
$-6$ &  &  & $q^{6}$ & $2q^{4}$ & $q^{2}$ \\
\end{tabular}
\vspace{2em}
\end{minipage}
%
\begin{minipage}{\linewidth}
$\bullet\ $ $10_{71}$ \vspace{0.5em} \\
\begin{tabular}{l|lllll}
$k \setminus j$ & $-4$ & $-2$ & $0$ & $2$ & $4$ \\
\hline
$6$ & $q^{-2}$ & $2q^{-4}$ & $q^{-6}$ &  &  \\
$4$ & $1$ & $4q^{-2}$ & $3q^{-4}$ &  &  \\
$2$ & $q^{2}$ & $7$ & $8q^{-2}$ & $2q^{-4}$ &  \\
$0$ &  & $4q^{2}$ & $9$ & $4q^{-2}$ &  \\
$-2$ &  & $2q^{4}$ & $8q^{2}$ & $7$ & $q^{-2}$ \\
$-4$ &  &  & $3q^{4}$ & $4q^{2}$ & $1$ \\
$-6$ &  &  & $q^{6}$ & $2q^{4}$ & $q^{2}$ \\
\end{tabular}
\vspace{2em}
\end{minipage}
%
\begin{minipage}{\linewidth}
$\bullet\ $ $10_{72}$ \vspace{0.5em} \\
\begin{tabular}{l|llll}
$k \setminus j$ & $2$ & $4$ & $6$ & $8$ \\
\hline
$6$ & $q^{-4}$ & $q^{-6}$ &  &  \\
$4$ & $q^{-2}$ & $3q^{-4}$ & $q^{-6}$ &  \\
$2$ & $2$ & $6q^{-2}$ & $4q^{-4}$ &  \\
$0$ & $q^{2}$ & $6$ & $6q^{-2}$ & $q^{-4}$ \\
$-2$ & $q^{4}$ & $6q^{2}$ & $8$ & $3q^{-2}$ \\
$-4$ &  & $3q^{4}$ & $6q^{2}$ & $3$ \\
$-6$ &  & $q^{6}$ & $4q^{4}$ & $3q^{2}$ \\
$-8$ &  &  & $q^{6}$ & $q^{4}$ \\
\end{tabular}
\vspace{2em}
\end{minipage}
%
\begin{minipage}{\linewidth}
$\bullet\ $ $10_{73}$ \vspace{0.5em} \\
\begin{tabular}{l|lllll}
$k \setminus j$ & $-8$ & $-6$ & $-4$ & $-2$ & $0$ \\
\hline
$6$ & $1$ & $3q^{-2}$ & $3q^{-4}$ & $q^{-6}$ &  \\
$4$ &  & $3$ & $6q^{-2}$ & $3q^{-4}$ &  \\
$2$ &  & $3q^{2}$ & $10$ & $8q^{-2}$ & $q^{-4}$ \\
$0$ &  &  & $6q^{2}$ & $9$ & $3q^{-2}$ \\
$-2$ &  &  & $3q^{4}$ & $8q^{2}$ & $4$ \\
$-4$ &  &  &  & $3q^{4}$ & $3q^{2}$ \\
$-6$ &  &  &  & $q^{6}$ & $q^{4}$ \\
\end{tabular}
\vspace{2em}
\end{minipage}
%
\begin{minipage}{\linewidth}
$\bullet\ $ $10_{74}$ \vspace{0.5em} \\
\begin{tabular}{l|lllll}
$k \setminus j$ & $0$ & $2$ & $4$ & $6$ & $8$ \\
\hline
$4$ & $q^{-2}$ & $q^{-4}$ &  &  &  \\
$2$ & $2$ & $5q^{-2}$ & $2q^{-4}$ &  &  \\
$0$ & $q^{2}$ & $6$ & $6q^{-2}$ & $q^{-4}$ &  \\
$-2$ &  & $5q^{2}$ & $8$ & $3q^{-2}$ &  \\
$-4$ &  & $q^{4}$ & $6q^{2}$ & $6$ & $q^{-2}$ \\
$-6$ &  &  & $2q^{4}$ & $3q^{2}$ & $1$ \\
$-8$ &  &  &  & $q^{4}$ & $q^{2}$ \\
\end{tabular}
\vspace{2em}
\end{minipage}
%
\begin{minipage}{\linewidth}
$\bullet\ $ $10_{75}$ \vspace{0.5em} \\
\begin{tabular}{l|lllll}
$k \setminus j$ & $-6$ & $-4$ & $-2$ & $0$ & $2$ \\
\hline
$6$ & $1$ & $3q^{-2}$ & $3q^{-4}$ & $q^{-6}$ &  \\
$4$ &  & $3$ & $6q^{-2}$ & $3q^{-4}$ &  \\
$2$ &  & $3q^{2}$ & $9$ & $7q^{-2}$ & $q^{-4}$ \\
$0$ &  &  & $6q^{2}$ & $10$ & $3q^{-2}$ \\
$-2$ &  &  & $3q^{4}$ & $7q^{2}$ & $4$ \\
$-4$ &  &  &  & $3q^{4}$ & $3q^{2}$ \\
$-6$ &  &  &  & $q^{6}$ & $q^{4}$ \\
\end{tabular}
\vspace{2em}
\end{minipage}
%
\begin{minipage}{\linewidth}
$\bullet\ $ $10_{76}$ \vspace{0.5em} \\
\begin{tabular}{l|llll}
$k \setminus j$ & $2$ & $4$ & $6$ & $8$ \\
\hline
$6$ & $q^{-4}$ & $q^{-6}$ &  &  \\
$4$ &  & $2q^{-4}$ & $q^{-6}$ &  \\
$2$ & $2$ & $5q^{-2}$ & $3q^{-4}$ &  \\
$0$ &  & $4$ & $5q^{-2}$ & $q^{-4}$ \\
$-2$ & $q^{4}$ & $5q^{2}$ & $6$ & $2q^{-2}$ \\
$-4$ &  & $2q^{4}$ & $5q^{2}$ & $3$ \\
$-6$ &  & $q^{6}$ & $3q^{4}$ & $2q^{2}$ \\
$-8$ &  &  & $q^{6}$ & $q^{4}$ \\
\end{tabular}
\vspace{2em}
\end{minipage}
%
\begin{minipage}{\linewidth}
$\bullet\ $ $10_{77}$ \vspace{0.5em} \\
\begin{tabular}{l|llll}
$k \setminus j$ & $-6$ & $-4$ & $-2$ & $0$ \\
\hline
$8$ & $q^{-4}$ & $q^{-6}$ &  &  \\
$6$ & $2q^{-2}$ & $3q^{-4}$ & $q^{-6}$ &  \\
$4$ & $3$ & $5q^{-2}$ & $2q^{-4}$ &  \\
$2$ & $2q^{2}$ & $7$ & $6q^{-2}$ & $q^{-4}$ \\
$0$ & $q^{4}$ & $5q^{2}$ & $5$ & $q^{-2}$ \\
$-2$ &  & $3q^{4}$ & $6q^{2}$ & $2$ \\
$-4$ &  & $q^{6}$ & $2q^{4}$ & $q^{2}$ \\
$-6$ &  &  & $q^{6}$ & $q^{4}$ \\
\end{tabular}
\vspace{2em}
\end{minipage}
%
\begin{minipage}{\linewidth}
$\bullet\ $ $10_{78}$ \vspace{0.5em} \\
\begin{tabular}{l|lllll}
$k \setminus j$ & $2$ & $4$ & $6$ & $8$ & $10$ \\
\hline
$6$ & $q^{-4}$ & $q^{-6}$ &  &  &  \\
$4$ & $2q^{-2}$ & $3q^{-4}$ &  &  &  \\
$2$ & $3$ & $6q^{-2}$ & $3q^{-4}$ &  &  \\
$0$ & $2q^{2}$ & $7$ & $5q^{-2}$ &  &  \\
$-2$ & $q^{4}$ & $6q^{2}$ & $8$ & $3q^{-2}$ &  \\
$-4$ &  & $3q^{4}$ & $5q^{2}$ & $2$ &  \\
$-6$ &  & $q^{6}$ & $3q^{4}$ & $3q^{2}$ & $1$ \\
\end{tabular}
\vspace{2em}
\end{minipage}
%
\begin{minipage}{\linewidth}
$\bullet\ $ $10_{79}$ \vspace{0.5em} \\
\begin{tabular}{l|lll}
$k \setminus j$ & $-2$ & $0$ & $2$ \\
\hline
$8$ & $q^{-6}$ & $q^{-8}$ &  \\
$6$ & $q^{-4}$ & $q^{-6}$ &  \\
$4$ & $4q^{-2}$ & $5q^{-4}$ & $q^{-6}$ \\
$2$ & $3$ & $4q^{-2}$ & $q^{-4}$ \\
$0$ & $4q^{2}$ & $9$ & $4q^{-2}$ \\
$-2$ & $q^{4}$ & $4q^{2}$ & $3$ \\
$-4$ & $q^{6}$ & $5q^{4}$ & $4q^{2}$ \\
$-6$ &  & $q^{6}$ & $q^{4}$ \\
$-8$ &  & $q^{8}$ & $q^{6}$ \\
\end{tabular}
\vspace{2em}
\end{minipage}
%
\begin{minipage}{\linewidth}
$\bullet\ $ $10_{80}$ \vspace{0.5em} \\
\begin{tabular}{l|llll}
$k \setminus j$ & $6$ & $8$ & $10$ & $12$ \\
\hline
$6$ & $q^{-6}$ &  &  &  \\
$4$ & $2q^{-4}$ & $2q^{-6}$ &  &  \\
$2$ & $4q^{-2}$ & $4q^{-4}$ &  &  \\
$0$ & $4$ & $7q^{-2}$ & $3q^{-4}$ &  \\
$-2$ & $4q^{2}$ & $7$ & $3q^{-2}$ &  \\
$-4$ & $2q^{4}$ & $7q^{2}$ & $6$ & $q^{-2}$ \\
$-6$ & $q^{6}$ & $4q^{4}$ & $3q^{2}$ &  \\
$-8$ &  & $2q^{6}$ & $3q^{4}$ & $q^{2}$ \\
\end{tabular}
\vspace{2em}
\end{minipage}
%
\begin{minipage}{\linewidth}
$\bullet\ $ $10_{81}$ \vspace{0.5em} \\
\begin{tabular}{l|lllll}
$k \setminus j$ & $-4$ & $-2$ & $0$ & $2$ & $4$ \\
\hline
$6$ & $q^{-2}$ & $2q^{-4}$ & $q^{-6}$ &  &  \\
$4$ & $1$ & $5q^{-2}$ & $4q^{-4}$ &  &  \\
$2$ & $q^{2}$ & $7$ & $8q^{-2}$ & $2q^{-4}$ &  \\
$0$ &  & $5q^{2}$ & $11$ & $5q^{-2}$ &  \\
$-2$ &  & $2q^{4}$ & $8q^{2}$ & $7$ & $q^{-2}$ \\
$-4$ &  &  & $4q^{4}$ & $5q^{2}$ & $1$ \\
$-6$ &  &  & $q^{6}$ & $2q^{4}$ & $q^{2}$ \\
\end{tabular}
\vspace{2em}
\end{minipage}
%
\begin{minipage}{\linewidth}
$\bullet\ $ $10_{82}$ \vspace{0.5em} \\
\begin{tabular}{l|lll}
$k \setminus j$ & $0$ & $2$ & $4$ \\
\hline
$8$ & $q^{-6}$ & $q^{-8}$ &  \\
$6$ & $2q^{-4}$ & $2q^{-6}$ &  \\
$4$ & $3q^{-2}$ & $4q^{-4}$ & $q^{-6}$ \\
$2$ & $3$ & $6q^{-2}$ & $2q^{-4}$ \\
$0$ & $3q^{2}$ & $6$ & $3q^{-2}$ \\
$-2$ & $2q^{4}$ & $6q^{2}$ & $4$ \\
$-4$ & $q^{6}$ & $4q^{4}$ & $3q^{2}$ \\
$-6$ &  & $2q^{6}$ & $2q^{4}$ \\
$-8$ &  & $q^{8}$ & $q^{6}$ \\
\end{tabular}
\vspace{2em}
\end{minipage}
%
\begin{minipage}{\linewidth}
$\bullet\ $ $10_{83}$ \vspace{0.5em} \\
\begin{tabular}{l|llll}
$k \setminus j$ & $-6$ & $-4$ & $-2$ & $0$ \\
\hline
$8$ & $q^{-4}$ & $q^{-6}$ &  &  \\
$6$ & $2q^{-2}$ & $3q^{-4}$ & $q^{-6}$ &  \\
$4$ & $3$ & $7q^{-2}$ & $4q^{-4}$ &  \\
$2$ & $2q^{2}$ & $8$ & $7q^{-2}$ & $q^{-4}$ \\
$0$ & $q^{4}$ & $7q^{2}$ & $9$ & $3q^{-2}$ \\
$-2$ &  & $3q^{4}$ & $7q^{2}$ & $3$ \\
$-4$ &  & $q^{6}$ & $4q^{4}$ & $3q^{2}$ \\
$-6$ &  &  & $q^{6}$ & $q^{4}$ \\
\end{tabular}
\vspace{2em}
\end{minipage}
%
\begin{minipage}{\linewidth}
$\bullet\ $ $10_{84}$ \vspace{0.5em} \\
\begin{tabular}{l|llll}
$k \setminus j$ & $0$ & $2$ & $4$ & $6$ \\
\hline
$6$ & $q^{-4}$ & $q^{-6}$ &  &  \\
$4$ & $2q^{-2}$ & $3q^{-4}$ & $q^{-6}$ &  \\
$2$ & $3$ & $8q^{-2}$ & $4q^{-4}$ &  \\
$0$ & $2q^{2}$ & $8$ & $7q^{-2}$ & $q^{-4}$ \\
$-2$ & $q^{4}$ & $8q^{2}$ & $10$ & $3q^{-2}$ \\
$-4$ &  & $3q^{4}$ & $7q^{2}$ & $4$ \\
$-6$ &  & $q^{6}$ & $4q^{4}$ & $3q^{2}$ \\
$-8$ &  &  & $q^{6}$ & $q^{4}$ \\
\end{tabular}
\vspace{2em}
\end{minipage}
%
\begin{minipage}{\linewidth}
$\bullet\ $ $10_{85}$ \vspace{0.5em} \\
\begin{tabular}{l|lll}
$k \setminus j$ & $-6$ & $-4$ & $-2$ \\
\hline
$8$ & $q^{-6}$ & $q^{-8}$ &  \\
$6$ & $2q^{-4}$ & $2q^{-6}$ &  \\
$4$ & $3q^{-2}$ & $4q^{-4}$ & $q^{-6}$ \\
$2$ & $3$ & $5q^{-2}$ & $2q^{-4}$ \\
$0$ & $3q^{2}$ & $5$ & $2q^{-2}$ \\
$-2$ & $2q^{4}$ & $5q^{2}$ & $3$ \\
$-4$ & $q^{6}$ & $4q^{4}$ & $2q^{2}$ \\
$-6$ &  & $2q^{6}$ & $2q^{4}$ \\
$-8$ &  & $q^{8}$ & $q^{6}$ \\
\end{tabular}
\vspace{2em}
\end{minipage}
%
\begin{minipage}{\linewidth}
$\bullet\ $ $10_{86}$ \vspace{0.5em} \\
\begin{tabular}{l|llll}
$k \setminus j$ & $-2$ & $0$ & $2$ & $4$ \\
\hline
$6$ & $q^{-4}$ & $q^{-6}$ &  &  \\
$4$ & $3q^{-2}$ & $4q^{-4}$ & $q^{-6}$ &  \\
$2$ & $4$ & $7q^{-2}$ & $3q^{-4}$ &  \\
$0$ & $3q^{2}$ & $10$ & $7q^{-2}$ & $q^{-4}$ \\
$-2$ & $q^{4}$ & $7q^{2}$ & $8$ & $2q^{-2}$ \\
$-4$ &  & $4q^{4}$ & $7q^{2}$ & $3$ \\
$-6$ &  & $q^{6}$ & $3q^{4}$ & $2q^{2}$ \\
$-8$ &  &  & $q^{6}$ & $q^{4}$ \\
\end{tabular}
\vspace{2em}
\end{minipage}
%
\begin{minipage}{\linewidth}
$\bullet\ $ $10_{87}$ \vspace{0.5em} \\
\begin{tabular}{l|llll}
$k \setminus j$ & $-2$ & $0$ & $2$ & $4$ \\
\hline
$6$ & $q^{-4}$ & $q^{-6}$ &  &  \\
$4$ & $2q^{-2}$ & $3q^{-4}$ & $q^{-6}$ &  \\
$2$ & $3$ & $7q^{-2}$ & $4q^{-4}$ &  \\
$0$ & $2q^{2}$ & $8$ & $6q^{-2}$ & $q^{-4}$ \\
$-2$ & $q^{4}$ & $7q^{2}$ & $9$ & $3q^{-2}$ \\
$-4$ &  & $3q^{4}$ & $6q^{2}$ & $3$ \\
$-6$ &  & $q^{6}$ & $4q^{4}$ & $3q^{2}$ \\
$-8$ &  &  & $q^{6}$ & $q^{4}$ \\
\end{tabular}
\vspace{2em}
\end{minipage}
%
\begin{minipage}{\linewidth}
$\bullet\ $ $10_{88}$ \vspace{0.5em} \\
\begin{tabular}{l|lllll}
$k \setminus j$ & $-4$ & $-2$ & $0$ & $2$ & $4$ \\
\hline
$6$ & $q^{-2}$ & $2q^{-4}$ & $q^{-6}$ &  &  \\
$4$ & $2$ & $6q^{-2}$ & $4q^{-4}$ &  &  \\
$2$ & $q^{2}$ & $9$ & $10q^{-2}$ & $2q^{-4}$ &  \\
$0$ &  & $6q^{2}$ & $13$ & $6q^{-2}$ &  \\
$-2$ &  & $2q^{4}$ & $10q^{2}$ & $9$ & $q^{-2}$ \\
$-4$ &  &  & $4q^{4}$ & $6q^{2}$ & $2$ \\
$-6$ &  &  & $q^{6}$ & $2q^{4}$ & $q^{2}$ \\
\end{tabular}
\vspace{2em}
\end{minipage}
%
\begin{minipage}{\linewidth}
$\bullet\ $ $10_{89}$ \vspace{0.5em} \\
\begin{tabular}{l|lllll}
$k \setminus j$ & $-8$ & $-6$ & $-4$ & $-2$ & $0$ \\
\hline
$6$ & $1$ & $3q^{-2}$ & $3q^{-4}$ & $q^{-6}$ &  \\
$4$ &  & $4$ & $8q^{-2}$ & $4q^{-4}$ &  \\
$2$ &  & $3q^{2}$ & $11$ & $9q^{-2}$ & $q^{-4}$ \\
$0$ &  &  & $8q^{2}$ & $12$ & $4q^{-2}$ \\
$-2$ &  &  & $3q^{4}$ & $9q^{2}$ & $5$ \\
$-4$ &  &  &  & $4q^{4}$ & $4q^{2}$ \\
$-6$ &  &  &  & $q^{6}$ & $q^{4}$ \\
\end{tabular}
\vspace{2em}
\end{minipage}
%
\begin{minipage}{\linewidth}
$\bullet\ $ $10_{90}$ \vspace{0.5em} \\
\begin{tabular}{l|llll}
$k \setminus j$ & $-2$ & $0$ & $2$ & $4$ \\
\hline
$6$ & $q^{-4}$ & $q^{-6}$ &  &  \\
$4$ & $2q^{-2}$ & $3q^{-4}$ & $q^{-6}$ &  \\
$2$ & $4$ & $7q^{-2}$ & $3q^{-4}$ &  \\
$0$ & $2q^{2}$ & $8$ & $6q^{-2}$ & $q^{-4}$ \\
$-2$ & $q^{4}$ & $7q^{2}$ & $8$ & $2q^{-2}$ \\
$-4$ &  & $3q^{4}$ & $6q^{2}$ & $3$ \\
$-6$ &  & $q^{6}$ & $3q^{4}$ & $2q^{2}$ \\
$-8$ &  &  & $q^{6}$ & $q^{4}$ \\
\end{tabular}
\vspace{2em}
\end{minipage}
%
\begin{minipage}{\linewidth}
$\bullet\ $ $10_{91}$ \vspace{0.5em} \\
\begin{tabular}{l|lll}
$k \setminus j$ & $-2$ & $0$ & $2$ \\
\hline
$8$ & $q^{-6}$ & $q^{-8}$ &  \\
$6$ & $2q^{-4}$ & $2q^{-6}$ &  \\
$4$ & $4q^{-2}$ & $5q^{-4}$ & $q^{-6}$ \\
$2$ & $4$ & $6q^{-2}$ & $2q^{-4}$ \\
$0$ & $4q^{2}$ & $9$ & $4q^{-2}$ \\
$-2$ & $2q^{4}$ & $6q^{2}$ & $4$ \\
$-4$ & $q^{6}$ & $5q^{4}$ & $4q^{2}$ \\
$-6$ &  & $2q^{6}$ & $2q^{4}$ \\
$-8$ &  & $q^{8}$ & $q^{6}$ \\
\end{tabular}
\vspace{2em}
\end{minipage}
%
\begin{minipage}{\linewidth}
$\bullet\ $ $10_{92}$ \vspace{0.5em} \\
\begin{tabular}{l|llll}
$k \setminus j$ & $2$ & $4$ & $6$ & $8$ \\
\hline
$6$ & $q^{-4}$ & $q^{-6}$ &  &  \\
$4$ & $2q^{-2}$ & $4q^{-4}$ & $q^{-6}$ &  \\
$2$ & $3$ & $7q^{-2}$ & $4q^{-4}$ &  \\
$0$ & $2q^{2}$ & $9$ & $8q^{-2}$ & $q^{-4}$ \\
$-2$ & $q^{4}$ & $7q^{2}$ & $9$ & $3q^{-2}$ \\
$-4$ &  & $4q^{4}$ & $8q^{2}$ & $4$ \\
$-6$ &  & $q^{6}$ & $4q^{4}$ & $3q^{2}$ \\
$-8$ &  &  & $q^{6}$ & $q^{4}$ \\
\end{tabular}
\vspace{2em}
\end{minipage}
%
\begin{minipage}{\linewidth}
$\bullet\ $ $10_{93}$ \vspace{0.5em} \\
\begin{tabular}{l|llll}
$k \setminus j$ & $-4$ & $-2$ & $0$ & $2$ \\
\hline
$6$ & $q^{-4}$ & $q^{-6}$ &  &  \\
$4$ & $2q^{-2}$ & $3q^{-4}$ & $q^{-6}$ &  \\
$2$ & $3$ & $6q^{-2}$ & $3q^{-4}$ &  \\
$0$ & $2q^{2}$ & $6$ & $5q^{-2}$ & $q^{-4}$ \\
$-2$ & $q^{4}$ & $6q^{2}$ & $6$ & $2q^{-2}$ \\
$-4$ &  & $3q^{4}$ & $5q^{2}$ & $2$ \\
$-6$ &  & $q^{6}$ & $3q^{4}$ & $2q^{2}$ \\
$-8$ &  &  & $q^{6}$ & $q^{4}$ \\
\end{tabular}
\vspace{2em}
\end{minipage}
%
\begin{minipage}{\linewidth}
$\bullet\ $ $10_{94}$ \vspace{0.5em} \\
\begin{tabular}{l|lll}
$k \setminus j$ & $0$ & $2$ & $4$ \\
\hline
$8$ & $q^{-6}$ & $q^{-8}$ &  \\
$6$ & $2q^{-4}$ & $2q^{-6}$ &  \\
$4$ & $4q^{-2}$ & $5q^{-4}$ & $q^{-6}$ \\
$2$ & $3$ & $6q^{-2}$ & $2q^{-4}$ \\
$0$ & $4q^{2}$ & $8$ & $4q^{-2}$ \\
$-2$ & $2q^{4}$ & $6q^{2}$ & $4$ \\
$-4$ & $q^{6}$ & $5q^{4}$ & $4q^{2}$ \\
$-6$ &  & $2q^{6}$ & $2q^{4}$ \\
$-8$ &  & $q^{8}$ & $q^{6}$ \\
\end{tabular}
\vspace{2em}
\end{minipage}
%
\begin{minipage}{\linewidth}
$\bullet\ $ $10_{95}$ \vspace{0.5em} \\
\begin{tabular}{l|llll}
$k \setminus j$ & $-6$ & $-4$ & $-2$ & $0$ \\
\hline
$8$ & $q^{-4}$ & $q^{-6}$ &  &  \\
$6$ & $2q^{-2}$ & $3q^{-4}$ & $q^{-6}$ &  \\
$4$ & $4$ & $8q^{-2}$ & $4q^{-4}$ &  \\
$2$ & $2q^{2}$ & $9$ & $8q^{-2}$ & $q^{-4}$ \\
$0$ & $q^{4}$ & $8q^{2}$ & $10$ & $3q^{-2}$ \\
$-2$ &  & $3q^{4}$ & $8q^{2}$ & $4$ \\
$-4$ &  & $q^{6}$ & $4q^{4}$ & $3q^{2}$ \\
$-6$ &  &  & $q^{6}$ & $q^{4}$ \\
\end{tabular}
\vspace{2em}
\end{minipage}
%
\begin{minipage}{\linewidth}
$\bullet\ $ $10_{96}$ \vspace{0.5em} \\
\begin{tabular}{l|lllll}
$k \setminus j$ & $-6$ & $-4$ & $-2$ & $0$ & $2$ \\
\hline
$6$ & $1$ & $3q^{-2}$ & $3q^{-4}$ & $q^{-6}$ &  \\
$4$ &  & $4$ & $7q^{-2}$ & $3q^{-4}$ &  \\
$2$ &  & $3q^{2}$ & $11$ & $9q^{-2}$ & $q^{-4}$ \\
$0$ &  &  & $7q^{2}$ & $11$ & $3q^{-2}$ \\
$-2$ &  &  & $3q^{4}$ & $9q^{2}$ & $6$ \\
$-4$ &  &  &  & $3q^{4}$ & $3q^{2}$ \\
$-6$ &  &  &  & $q^{6}$ & $q^{4}$ \\
\end{tabular}
\vspace{2em}
\end{minipage}
%
\begin{minipage}{\linewidth}
$\bullet\ $ $10_{97}$ \vspace{0.5em} \\
\begin{tabular}{l|lllll}
$k \setminus j$ & $0$ & $2$ & $4$ & $6$ & $8$ \\
\hline
$4$ & $q^{-2}$ & $q^{-4}$ &  &  &  \\
$2$ & $2$ & $6q^{-2}$ & $3q^{-4}$ &  &  \\
$0$ & $q^{2}$ & $8$ & $8q^{-2}$ & $q^{-4}$ &  \\
$-2$ &  & $6q^{2}$ & $12$ & $6q^{-2}$ &  \\
$-4$ &  & $q^{4}$ & $8q^{2}$ & $8$ & $q^{-2}$ \\
$-6$ &  &  & $3q^{4}$ & $6q^{2}$ & $3$ \\
$-8$ &  &  &  & $q^{4}$ & $q^{2}$ \\
\end{tabular}
\vspace{2em}
\end{minipage}
%
\begin{minipage}{\linewidth}
$\bullet\ $ $10_{98}$ \vspace{0.5em} \\
\begin{tabular}{l|llll}
$k \setminus j$ & $2$ & $4$ & $6$ & $8$ \\
\hline
$6$ & $q^{-4}$ & $q^{-6}$ &  &  \\
$4$ & $2q^{-2}$ & $4q^{-4}$ & $q^{-6}$ &  \\
$2$ & $3$ & $6q^{-2}$ & $3q^{-4}$ &  \\
$0$ & $2q^{2}$ & $9$ & $8q^{-2}$ & $q^{-4}$ \\
$-2$ & $q^{4}$ & $6q^{2}$ & $7$ & $2q^{-2}$ \\
$-4$ &  & $4q^{4}$ & $8q^{2}$ & $4$ \\
$-6$ &  & $q^{6}$ & $3q^{4}$ & $2q^{2}$ \\
$-8$ &  &  & $q^{6}$ & $q^{4}$ \\
\end{tabular}
\vspace{2em}
\end{minipage}
%
\begin{minipage}{\linewidth}
$\bullet\ $ $10_{99}$ \vspace{0.5em} \\
\begin{tabular}{l|lll}
$k \setminus j$ & $-2$ & $0$ & $2$ \\
\hline
$8$ & $q^{-6}$ & $q^{-8}$ &  \\
$6$ & $2q^{-4}$ & $2q^{-6}$ &  \\
$4$ & $5q^{-2}$ & $6q^{-4}$ & $q^{-6}$ \\
$2$ & $4$ & $6q^{-2}$ & $2q^{-4}$ \\
$0$ & $5q^{2}$ & $11$ & $5q^{-2}$ \\
$-2$ & $2q^{4}$ & $6q^{2}$ & $4$ \\
$-4$ & $q^{6}$ & $6q^{4}$ & $5q^{2}$ \\
$-6$ &  & $2q^{6}$ & $2q^{4}$ \\
$-8$ &  & $q^{8}$ & $q^{6}$ \\
\end{tabular}
\vspace{2em}
\end{minipage}
%
\begin{minipage}{\linewidth}
$\bullet\ $ $10_{100}$ \vspace{0.5em} \\
\begin{tabular}{l|lll}
$k \setminus j$ & $-6$ & $-4$ & $-2$ \\
\hline
$8$ & $q^{-6}$ & $q^{-8}$ &  \\
$6$ & $2q^{-4}$ & $2q^{-6}$ &  \\
$4$ & $4q^{-2}$ & $5q^{-4}$ & $q^{-6}$ \\
$2$ & $3$ & $5q^{-2}$ & $2q^{-4}$ \\
$0$ & $4q^{2}$ & $7$ & $3q^{-2}$ \\
$-2$ & $2q^{4}$ & $5q^{2}$ & $3$ \\
$-4$ & $q^{6}$ & $5q^{4}$ & $3q^{2}$ \\
$-6$ &  & $2q^{6}$ & $2q^{4}$ \\
$-8$ &  & $q^{8}$ & $q^{6}$ \\
\end{tabular}
\vspace{2em}
\end{minipage}
%
\begin{minipage}{\linewidth}
$\bullet\ $ $10_{101}$ \vspace{0.5em} \\
\begin{tabular}{l|lllll}
$k \setminus j$ & $4$ & $6$ & $8$ & $10$ & $12$ \\
\hline
$4$ & $q^{-4}$ &  &  &  &  \\
$2$ & $3q^{-2}$ & $3q^{-4}$ &  &  &  \\
$0$ & $4$ & $7q^{-2}$ & $3q^{-4}$ &  &  \\
$-2$ & $3q^{2}$ & $10$ & $7q^{-2}$ &  &  \\
$-4$ & $q^{4}$ & $7q^{2}$ & $10$ & $4q^{-2}$ &  \\
$-6$ &  & $3q^{4}$ & $7q^{2}$ & $4$ &  \\
$-8$ &  &  & $3q^{4}$ & $4q^{2}$ & $1$ \\
\end{tabular}
\vspace{2em}
\end{minipage}
%
\begin{minipage}{\linewidth}
$\bullet\ $ $10_{102}$ \vspace{0.5em} \\
\begin{tabular}{l|llll}
$k \setminus j$ & $-2$ & $0$ & $2$ & $4$ \\
\hline
$6$ & $q^{-4}$ & $q^{-6}$ &  &  \\
$4$ & $2q^{-2}$ & $3q^{-4}$ & $q^{-6}$ &  \\
$2$ & $3$ & $6q^{-2}$ & $3q^{-4}$ &  \\
$0$ & $2q^{2}$ & $8$ & $6q^{-2}$ & $q^{-4}$ \\
$-2$ & $q^{4}$ & $6q^{2}$ & $7$ & $2q^{-2}$ \\
$-4$ &  & $3q^{4}$ & $6q^{2}$ & $3$ \\
$-6$ &  & $q^{6}$ & $3q^{4}$ & $2q^{2}$ \\
$-8$ &  &  & $q^{6}$ & $q^{4}$ \\
\end{tabular}
\vspace{2em}
\end{minipage}
%
\begin{minipage}{\linewidth}
$\bullet\ $ $10_{103}$ \vspace{0.5em} \\
\begin{tabular}{l|llll}
$k \setminus j$ & $-6$ & $-4$ & $-2$ & $0$ \\
\hline
$8$ & $q^{-4}$ & $q^{-6}$ &  &  \\
$6$ & $2q^{-2}$ & $3q^{-4}$ & $q^{-6}$ &  \\
$4$ & $3$ & $6q^{-2}$ & $3q^{-4}$ &  \\
$2$ & $2q^{2}$ & $8$ & $7q^{-2}$ & $q^{-4}$ \\
$0$ & $q^{4}$ & $6q^{2}$ & $7$ & $2q^{-2}$ \\
$-2$ &  & $3q^{4}$ & $7q^{2}$ & $3$ \\
$-4$ &  & $q^{6}$ & $3q^{4}$ & $2q^{2}$ \\
$-6$ &  &  & $q^{6}$ & $q^{4}$ \\
\end{tabular}
\vspace{2em}
\end{minipage}
%
\begin{minipage}{\linewidth}
$\bullet\ $ $10_{104}$ \vspace{0.5em} \\
\begin{tabular}{l|lll}
$k \setminus j$ & $-2$ & $0$ & $2$ \\
\hline
$8$ & $q^{-6}$ & $q^{-8}$ &  \\
$6$ & $2q^{-4}$ & $2q^{-6}$ &  \\
$4$ & $4q^{-2}$ & $5q^{-4}$ & $q^{-6}$ \\
$2$ & $5$ & $7q^{-2}$ & $2q^{-4}$ \\
$0$ & $4q^{2}$ & $9$ & $4q^{-2}$ \\
$-2$ & $2q^{4}$ & $7q^{2}$ & $5$ \\
$-4$ & $q^{6}$ & $5q^{4}$ & $4q^{2}$ \\
$-6$ &  & $2q^{6}$ & $2q^{4}$ \\
$-8$ &  & $q^{8}$ & $q^{6}$ \\
\end{tabular}
\vspace{2em}
\end{minipage}
%
\begin{minipage}{\linewidth}
$\bullet\ $ $10_{105}$ \vspace{0.5em} \\
\begin{tabular}{l|lllll}
$k \setminus j$ & $-2$ & $0$ & $2$ & $4$ & $6$ \\
\hline
$6$ & $q^{-2}$ & $2q^{-4}$ & $q^{-6}$ &  &  \\
$4$ & $1$ & $5q^{-2}$ & $4q^{-4}$ &  &  \\
$2$ & $q^{2}$ & $7$ & $9q^{-2}$ & $2q^{-4}$ &  \\
$0$ &  & $5q^{2}$ & $11$ & $6q^{-2}$ &  \\
$-2$ &  & $2q^{4}$ & $9q^{2}$ & $8$ & $q^{-2}$ \\
$-4$ &  &  & $4q^{4}$ & $6q^{2}$ & $2$ \\
$-6$ &  &  & $q^{6}$ & $2q^{4}$ & $q^{2}$ \\
\end{tabular}
\vspace{2em}
\end{minipage}
%
\begin{minipage}{\linewidth}
$\bullet\ $ $10_{106}$ \vspace{0.5em} \\
\begin{tabular}{l|lll}
$k \setminus j$ & $0$ & $2$ & $4$ \\
\hline
$8$ & $q^{-6}$ & $q^{-8}$ &  \\
$6$ & $2q^{-4}$ & $2q^{-6}$ &  \\
$4$ & $4q^{-2}$ & $5q^{-4}$ & $q^{-6}$ \\
$2$ & $4$ & $7q^{-2}$ & $2q^{-4}$ \\
$0$ & $4q^{2}$ & $8$ & $4q^{-2}$ \\
$-2$ & $2q^{4}$ & $7q^{2}$ & $5$ \\
$-4$ & $q^{6}$ & $5q^{4}$ & $4q^{2}$ \\
$-6$ &  & $2q^{6}$ & $2q^{4}$ \\
$-8$ &  & $q^{8}$ & $q^{6}$ \\
\end{tabular}
\vspace{2em}
\end{minipage}
%
\begin{minipage}{\linewidth}
$\bullet\ $ $10_{107}$ \vspace{0.5em} \\
\begin{tabular}{l|lllll}
$k \setminus j$ & $-4$ & $-2$ & $0$ & $2$ & $4$ \\
\hline
$6$ & $q^{-2}$ & $2q^{-4}$ & $q^{-6}$ &  &  \\
$4$ & $1$ & $5q^{-2}$ & $4q^{-4}$ &  &  \\
$2$ & $q^{2}$ & $8$ & $9q^{-2}$ & $2q^{-4}$ &  \\
$0$ &  & $5q^{2}$ & $12$ & $6q^{-2}$ &  \\
$-2$ &  & $2q^{4}$ & $9q^{2}$ & $8$ & $q^{-2}$ \\
$-4$ &  &  & $4q^{4}$ & $6q^{2}$ & $2$ \\
$-6$ &  &  & $q^{6}$ & $2q^{4}$ & $q^{2}$ \\
\end{tabular}
\vspace{2em}
\end{minipage}
%
\begin{minipage}{\linewidth}
$\bullet\ $ $10_{108}$ \vspace{0.5em} \\
\begin{tabular}{l|llll}
$k \setminus j$ & $-2$ & $0$ & $2$ & $4$ \\
\hline
$8$ & $q^{-4}$ & $q^{-6}$ &  &  \\
$6$ & $2q^{-2}$ & $3q^{-4}$ & $q^{-6}$ &  \\
$4$ & $2$ & $5q^{-2}$ & $3q^{-4}$ &  \\
$2$ & $2q^{2}$ & $5$ & $5q^{-2}$ & $q^{-4}$ \\
$0$ & $q^{4}$ & $5q^{2}$ & $6$ & $2q^{-2}$ \\
$-2$ &  & $3q^{4}$ & $5q^{2}$ & $2$ \\
$-4$ &  & $q^{6}$ & $3q^{4}$ & $2q^{2}$ \\
$-6$ &  &  & $q^{6}$ & $q^{4}$ \\
\end{tabular}
\vspace{2em}
\end{minipage}
%
\begin{minipage}{\linewidth}
$\bullet\ $ $10_{109}$ \vspace{0.5em} \\
\begin{tabular}{l|lll}
$k \setminus j$ & $-2$ & $0$ & $2$ \\
\hline
$8$ & $q^{-6}$ & $q^{-8}$ &  \\
$6$ & $2q^{-4}$ & $2q^{-6}$ &  \\
$4$ & $5q^{-2}$ & $6q^{-4}$ & $q^{-6}$ \\
$2$ & $5$ & $7q^{-2}$ & $2q^{-4}$ \\
$0$ & $5q^{2}$ & $11$ & $5q^{-2}$ \\
$-2$ & $2q^{4}$ & $7q^{2}$ & $5$ \\
$-4$ & $q^{6}$ & $6q^{4}$ & $5q^{2}$ \\
$-6$ &  & $2q^{6}$ & $2q^{4}$ \\
$-8$ &  & $q^{8}$ & $q^{6}$ \\
\end{tabular}
\vspace{2em}
\end{minipage}
%
\begin{minipage}{\linewidth}
$\bullet\ $ $10_{110}$ \vspace{0.5em} \\
\begin{tabular}{l|lllll}
$k \setminus j$ & $-2$ & $0$ & $2$ & $4$ & $6$ \\
\hline
$6$ & $q^{-2}$ & $2q^{-4}$ & $q^{-6}$ &  &  \\
$4$ & $1$ & $5q^{-2}$ & $4q^{-4}$ &  &  \\
$2$ & $q^{2}$ & $6$ & $8q^{-2}$ & $2q^{-4}$ &  \\
$0$ &  & $5q^{2}$ & $10$ & $5q^{-2}$ &  \\
$-2$ &  & $2q^{4}$ & $8q^{2}$ & $7$ & $q^{-2}$ \\
$-4$ &  &  & $4q^{4}$ & $5q^{2}$ & $1$ \\
$-6$ &  &  & $q^{6}$ & $2q^{4}$ & $q^{2}$ \\
\end{tabular}
\vspace{2em}
\end{minipage}
%
\begin{minipage}{\linewidth}
$\bullet\ $ $10_{111}$ \vspace{0.5em} \\
\begin{tabular}{l|llll}
$k \setminus j$ & $2$ & $4$ & $6$ & $8$ \\
\hline
$6$ & $q^{-4}$ & $q^{-6}$ &  &  \\
$4$ & $2q^{-2}$ & $4q^{-4}$ & $q^{-6}$ &  \\
$2$ & $3$ & $6q^{-2}$ & $3q^{-4}$ &  \\
$0$ & $2q^{2}$ & $8$ & $7q^{-2}$ & $q^{-4}$ \\
$-2$ & $q^{4}$ & $6q^{2}$ & $7$ & $2q^{-2}$ \\
$-4$ &  & $4q^{4}$ & $7q^{2}$ & $3$ \\
$-6$ &  & $q^{6}$ & $3q^{4}$ & $2q^{2}$ \\
$-8$ &  &  & $q^{6}$ & $q^{4}$ \\
\end{tabular}
\vspace{2em}
\end{minipage}
%
\begin{minipage}{\linewidth}
$\bullet\ $ $10_{112}$ \vspace{0.5em} \\
\begin{tabular}{l|lll}
$k \setminus j$ & $-4$ & $-2$ & $0$ \\
\hline
$8$ & $q^{-6}$ & $q^{-8}$ &  \\
$6$ & $3q^{-4}$ & $3q^{-6}$ &  \\
$4$ & $4q^{-2}$ & $5q^{-4}$ & $q^{-6}$ \\
$2$ & $6$ & $9q^{-2}$ & $3q^{-4}$ \\
$0$ & $4q^{2}$ & $8$ & $4q^{-2}$ \\
$-2$ & $3q^{4}$ & $9q^{2}$ & $5$ \\
$-4$ & $q^{6}$ & $5q^{4}$ & $4q^{2}$ \\
$-6$ &  & $3q^{6}$ & $3q^{4}$ \\
$-8$ &  & $q^{8}$ & $q^{6}$ \\
\end{tabular}
\vspace{2em}
\end{minipage}
%
\begin{minipage}{\linewidth}
$\bullet\ $ $10_{113}$ \vspace{0.5em} \\
\begin{tabular}{l|llll}
$k \setminus j$ & $0$ & $2$ & $4$ & $6$ \\
\hline
$6$ & $q^{-4}$ & $q^{-6}$ &  &  \\
$4$ & $3q^{-2}$ & $4q^{-4}$ & $q^{-6}$ &  \\
$2$ & $4$ & $10q^{-2}$ & $5q^{-4}$ &  \\
$0$ & $3q^{2}$ & $11$ & $9q^{-2}$ & $q^{-4}$ \\
$-2$ & $q^{4}$ & $10q^{2}$ & $13$ & $4q^{-2}$ \\
$-4$ &  & $4q^{4}$ & $9q^{2}$ & $5$ \\
$-6$ &  & $q^{6}$ & $5q^{4}$ & $4q^{2}$ \\
$-8$ &  &  & $q^{6}$ & $q^{4}$ \\
\end{tabular}
\vspace{2em}
\end{minipage}
%
\begin{minipage}{\linewidth}
$\bullet\ $ $10_{114}$ \vspace{0.5em} \\
\begin{tabular}{l|llll}
$k \setminus j$ & $-4$ & $-2$ & $0$ & $2$ \\
\hline
$8$ & $q^{-4}$ & $q^{-6}$ &  &  \\
$6$ & $3q^{-2}$ & $4q^{-4}$ & $q^{-6}$ &  \\
$4$ & $3$ & $7q^{-2}$ & $4q^{-4}$ &  \\
$2$ & $3q^{2}$ & $10$ & $8q^{-2}$ & $q^{-4}$ \\
$0$ & $q^{4}$ & $7q^{2}$ & $10$ & $3q^{-2}$ \\
$-2$ &  & $4q^{4}$ & $8q^{2}$ & $4$ \\
$-4$ &  & $q^{6}$ & $4q^{4}$ & $3q^{2}$ \\
$-6$ &  &  & $q^{6}$ & $q^{4}$ \\
\end{tabular}
\vspace{2em}
\end{minipage}
%
\begin{minipage}{\linewidth}
$\bullet\ $ $10_{115}$ \vspace{0.5em} \\
\begin{tabular}{l|lllll}
$k \setminus j$ & $-4$ & $-2$ & $0$ & $2$ & $4$ \\
\hline
$6$ & $q^{-2}$ & $2q^{-4}$ & $q^{-6}$ &  &  \\
$4$ & $2$ & $7q^{-2}$ & $5q^{-4}$ &  &  \\
$2$ & $q^{2}$ & $9$ & $10q^{-2}$ & $2q^{-4}$ &  \\
$0$ &  & $7q^{2}$ & $15$ & $7q^{-2}$ &  \\
$-2$ &  & $2q^{4}$ & $10q^{2}$ & $9$ & $q^{-2}$ \\
$-4$ &  &  & $5q^{4}$ & $7q^{2}$ & $2$ \\
$-6$ &  &  & $q^{6}$ & $2q^{4}$ & $q^{2}$ \\
\end{tabular}
\vspace{2em}
\end{minipage}
%
\begin{minipage}{\linewidth}
$\bullet\ $ $10_{116}$ \vspace{0.5em} \\
\begin{tabular}{l|lll}
$k \setminus j$ & $-4$ & $-2$ & $0$ \\
\hline
$8$ & $q^{-6}$ & $q^{-8}$ &  \\
$6$ & $3q^{-4}$ & $3q^{-6}$ &  \\
$4$ & $5q^{-2}$ & $6q^{-4}$ & $q^{-6}$ \\
$2$ & $6$ & $9q^{-2}$ & $3q^{-4}$ \\
$0$ & $5q^{2}$ & $10$ & $5q^{-2}$ \\
$-2$ & $3q^{4}$ & $9q^{2}$ & $5$ \\
$-4$ & $q^{6}$ & $6q^{4}$ & $5q^{2}$ \\
$-6$ &  & $3q^{6}$ & $3q^{4}$ \\
$-8$ &  & $q^{8}$ & $q^{6}$ \\
\end{tabular}
\vspace{2em}
\end{minipage}
%
\begin{minipage}{\linewidth}
$\bullet\ $ $10_{117}$ \vspace{0.5em} \\
\begin{tabular}{l|llll}
$k \setminus j$ & $-6$ & $-4$ & $-2$ & $0$ \\
\hline
$8$ & $q^{-4}$ & $q^{-6}$ &  &  \\
$6$ & $3q^{-2}$ & $4q^{-4}$ & $q^{-6}$ &  \\
$4$ & $5$ & $9q^{-2}$ & $4q^{-4}$ &  \\
$2$ & $3q^{2}$ & $11$ & $9q^{-2}$ & $q^{-4}$ \\
$0$ & $q^{4}$ & $9q^{2}$ & $11$ & $3q^{-2}$ \\
$-2$ &  & $4q^{4}$ & $9q^{2}$ & $4$ \\
$-4$ &  & $q^{6}$ & $4q^{4}$ & $3q^{2}$ \\
$-6$ &  &  & $q^{6}$ & $q^{4}$ \\
\end{tabular}
\vspace{2em}
\end{minipage}
%
\begin{minipage}{\linewidth}
$\bullet\ $ $10_{118}$ \vspace{0.5em} \\
\begin{tabular}{l|lll}
$k \setminus j$ & $-2$ & $0$ & $2$ \\
\hline
$8$ & $q^{-6}$ & $q^{-8}$ &  \\
$6$ & $3q^{-4}$ & $3q^{-6}$ &  \\
$4$ & $5q^{-2}$ & $6q^{-4}$ & $q^{-6}$ \\
$2$ & $6$ & $9q^{-2}$ & $3q^{-4}$ \\
$0$ & $5q^{2}$ & $11$ & $5q^{-2}$ \\
$-2$ & $3q^{4}$ & $9q^{2}$ & $6$ \\
$-4$ & $q^{6}$ & $6q^{4}$ & $5q^{2}$ \\
$-6$ &  & $3q^{6}$ & $3q^{4}$ \\
$-8$ &  & $q^{8}$ & $q^{6}$ \\
\end{tabular}
\vspace{2em}
\end{minipage}
%
\begin{minipage}{\linewidth}
$\bullet\ $ $10_{119}$ \vspace{0.5em} \\
\begin{tabular}{l|llll}
$k \setminus j$ & $-4$ & $-2$ & $0$ & $2$ \\
\hline
$8$ & $q^{-4}$ & $q^{-6}$ &  &  \\
$6$ & $3q^{-2}$ & $4q^{-4}$ & $q^{-6}$ &  \\
$4$ & $4$ & $8q^{-2}$ & $4q^{-4}$ &  \\
$2$ & $3q^{2}$ & $11$ & $9q^{-2}$ & $q^{-4}$ \\
$0$ & $q^{4}$ & $8q^{2}$ & $11$ & $3q^{-2}$ \\
$-2$ &  & $4q^{4}$ & $9q^{2}$ & $5$ \\
$-4$ &  & $q^{6}$ & $4q^{4}$ & $3q^{2}$ \\
$-6$ &  &  & $q^{6}$ & $q^{4}$ \\
\end{tabular}
\vspace{2em}
\end{minipage}
%
\begin{minipage}{\linewidth}
$\bullet\ $ $10_{120}$ \vspace{0.5em} \\
\begin{tabular}{l|lllll}
$k \setminus j$ & $4$ & $6$ & $8$ & $10$ & $12$ \\
\hline
$4$ & $q^{-4}$ &  &  &  &  \\
$2$ & $4q^{-2}$ & $4q^{-4}$ &  &  &  \\
$0$ & $6$ & $9q^{-2}$ & $3q^{-4}$ &  &  \\
$-2$ & $4q^{2}$ & $13$ & $9q^{-2}$ &  &  \\
$-4$ & $q^{4}$ & $9q^{2}$ & $12$ & $4q^{-2}$ &  \\
$-6$ &  & $4q^{4}$ & $9q^{2}$ & $5$ &  \\
$-8$ &  &  & $3q^{4}$ & $4q^{2}$ & $1$ \\
\end{tabular}
\vspace{2em}
\end{minipage}
%
\begin{minipage}{\linewidth}
$\bullet\ $ $10_{121}$ \vspace{0.5em} \\
\begin{tabular}{l|llll}
$k \setminus j$ & $0$ & $2$ & $4$ & $6$ \\
\hline
$6$ & $q^{-4}$ & $q^{-6}$ &  &  \\
$4$ & $4q^{-2}$ & $5q^{-4}$ & $q^{-6}$ &  \\
$2$ & $5$ & $10q^{-2}$ & $4q^{-4}$ &  \\
$0$ & $4q^{2}$ & $13$ & $10q^{-2}$ & $q^{-4}$ \\
$-2$ & $q^{4}$ & $10q^{2}$ & $12$ & $3q^{-2}$ \\
$-4$ &  & $5q^{4}$ & $10q^{2}$ & $5$ \\
$-6$ &  & $q^{6}$ & $4q^{4}$ & $3q^{2}$ \\
$-8$ &  &  & $q^{6}$ & $q^{4}$ \\
\end{tabular}
\vspace{2em}
\end{minipage}
%
\begin{minipage}{\linewidth}
$\bullet\ $ $10_{122}$ \vspace{0.5em} \\
\begin{tabular}{l|llll}
$k \setminus j$ & $-4$ & $-2$ & $0$ & $2$ \\
\hline
$8$ & $q^{-4}$ & $q^{-6}$ &  &  \\
$6$ & $4q^{-2}$ & $5q^{-4}$ & $q^{-6}$ &  \\
$4$ & $4$ & $8q^{-2}$ & $4q^{-4}$ &  \\
$2$ & $4q^{2}$ & $12$ & $9q^{-2}$ & $q^{-4}$ \\
$0$ & $q^{4}$ & $8q^{2}$ & $11$ & $3q^{-2}$ \\
$-2$ &  & $5q^{4}$ & $9q^{2}$ & $4$ \\
$-4$ &  & $q^{6}$ & $4q^{4}$ & $3q^{2}$ \\
$-6$ &  &  & $q^{6}$ & $q^{4}$ \\
\end{tabular}
\vspace{2em}
\end{minipage}
%
\begin{minipage}{\linewidth}
$\bullet\ $ $10_{123}$ \vspace{0.5em} \\
\begin{tabular}{l|lll}
$k \setminus j$ & $-2$ & $0$ & $2$ \\
\hline
$8$ & $q^{-6}$ & $q^{-8}$ &  \\
$6$ & $4q^{-4}$ & $4q^{-6}$ &  \\
$4$ & $6q^{-2}$ & $7q^{-4}$ & $q^{-6}$ \\
$2$ & $8$ & $12q^{-2}$ & $4q^{-4}$ \\
$0$ & $6q^{2}$ & $13$ & $6q^{-2}$ \\
$-2$ & $4q^{4}$ & $12q^{2}$ & $8$ \\
$-4$ & $q^{6}$ & $7q^{4}$ & $6q^{2}$ \\
$-6$ &  & $4q^{6}$ & $4q^{4}$ \\
$-8$ &  & $q^{8}$ & $q^{6}$ \\
\end{tabular}
\vspace{2em}
\end{minipage}
%
\begin{minipage}{\linewidth}
$\bullet\ $ $10_{124}$ \vspace{0.5em} \\
\begin{tabular}{l|lll}
$k \setminus j$ & $8$ & $10$ & $12$ \\
\hline
$8$ & $q^{-8}$ &  &  \\
$4$ & $q^{-4}$ & $q^{-6}$ &  \\
$0$ & $q^{-2}$ + $1$ & $q^{-4}$ + $q^{-2}$ &  \\
$-4$ & $q^{2}$ + $q^{4}$ & $2$ + $q^{2}$ & $q^{-2}$ \\
$-8$ & $q^{8}$ & $q^{4}$ + $q^{6}$ & $q^{2}$ \\
\end{tabular}
\vspace{2em}
\end{minipage}
%
\begin{minipage}{\linewidth}
$\bullet\ $ $10_{125}$ \vspace{0.5em} \\
\begin{tabular}{l|lll}
$k \setminus j$ & $-2$ & $0$ & $2$ \\
\hline
$4$ & $q^{-4}$ & $q^{-6}$ &  \\
$0$ & $1$ & $2q^{-2}$ + $1$ & $q^{-4}$ \\
$-4$ & $q^{4}$ & $2q^{2}$ & $1$ \\
$-8$ &  & $q^{6}$ & $q^{4}$ \\
\end{tabular}
\vspace{2em}
\end{minipage}
%
\begin{minipage}{\linewidth}
$\bullet\ $ $10_{126}$ \vspace{0.5em} \\
\begin{tabular}{l|lll}
$k \setminus j$ & $-6$ & $-4$ & $-2$ \\
\hline
$8$ & $q^{-4}$ & $q^{-6}$ &  \\
$4$ & $2$ & $3q^{-2}$ & $q^{-4}$ \\
$2$ &  & $1$ & $q^{-2}$ \\
$0$ & $q^{4}$ & $3q^{2}$ & $2$ \\
$-2$ &  &  & $q^{2}$ \\
$-4$ &  & $q^{6}$ & $q^{4}$ \\
\end{tabular}
\vspace{2em}
\end{minipage}
%
\begin{minipage}{\linewidth}
$\bullet\ $ $10_{127}$ \vspace{0.5em} \\
\begin{tabular}{l|lll}
$k \setminus j$ & $4$ & $6$ & $8$ \\
\hline
$4$ & $2q^{-4}$ & $q^{-6}$ &  \\
$2$ & $q^{-2}$ & $q^{-4}$ &  \\
$0$ & $3$ & $4q^{-2}$ & $q^{-4}$ \\
$-2$ & $q^{2}$ & $2$ & $q^{-2}$ \\
$-4$ & $2q^{4}$ & $4q^{2}$ & $2$ \\
$-6$ &  & $q^{4}$ & $q^{2}$ \\
$-8$ &  & $q^{6}$ & $q^{4}$ \\
\end{tabular}
\vspace{2em}
\end{minipage}
%
\begin{minipage}{\linewidth}
$\bullet\ $ $10_{128}$ \vspace{0.5em} \\
\begin{tabular}{l|llll}
$k \setminus j$ & $6$ & $8$ & $10$ & $12$ \\
\hline
$6$ & $q^{-6}$ &  &  &  \\
$4$ & $q^{-4}$ & $q^{-6}$ &  &  \\
$2$ & $q^{-2}$ & $q^{-4}$ &  &  \\
$0$ & $1$ & $2q^{-2}$ & $q^{-4}$ &  \\
$-2$ & $1$ + $q^{2}$ & $q^{-2}$ + $1$ &  &  \\
$-4$ & $q^{4}$ & $1$ + $2q^{2}$ & $q^{-2}$ + $1$ &  \\
$-6$ & $q^{6}$ & $q^{2}$ + $q^{4}$ & $1$ &  \\
$-8$ &  & $q^{6}$ & $q^{2}$ + $q^{4}$ & $1$ \\
\end{tabular}
\vspace{2em}
\end{minipage}
%
\begin{minipage}{\linewidth}
$\bullet\ $ $10_{129}$ \vspace{0.5em} \\
\begin{tabular}{l|llll}
$k \setminus j$ & $-2$ & $0$ & $2$ & $4$ \\
\hline
$4$ & $q^{-2}$ & $q^{-4}$ &  &  \\
$2$ & $1$ & $2q^{-2}$ & $q^{-4}$ &  \\
$0$ & $q^{2}$ & $4$ & $2q^{-2}$ &  \\
$-2$ &  & $2q^{2}$ & $3$ & $q^{-2}$ \\
$-4$ &  & $q^{4}$ & $2q^{2}$ & $1$ \\
$-6$ &  &  & $q^{4}$ & $q^{2}$ \\
\end{tabular}
\vspace{2em}
\end{minipage}
%
\begin{minipage}{\linewidth}
$\bullet\ $ $10_{130}$ \vspace{0.5em} \\
\begin{tabular}{l|llll}
$k \setminus j$ & $-6$ & $-4$ & $-2$ & $0$ \\
\hline
$8$ & $q^{-2}$ & $q^{-4}$ &  &  \\
$6$ &  & $q^{-2}$ & $q^{-4}$ &  \\
$4$ & $q^{2}$ & $2$ & $q^{-2}$ &  \\
$2$ &  & $q^{2}$ & $2$ & $q^{-2}$ \\
$0$ &  & $q^{4}$ & $q^{2}$ & $1$ \\
$-2$ &  &  & $q^{4}$ & $q^{2}$ \\
\end{tabular}
\vspace{2em}
\end{minipage}
%
\begin{minipage}{\linewidth}
$\bullet\ $ $10_{131}$ \vspace{0.5em} \\
\begin{tabular}{l|llll}
$k \setminus j$ & $2$ & $4$ & $6$ & $8$ \\
\hline
$2$ & $2q^{-2}$ & $q^{-4}$ &  &  \\
$0$ & $2$ & $3q^{-2}$ & $q^{-4}$ &  \\
$-2$ & $2q^{2}$ & $4$ & $2q^{-2}$ &  \\
$-4$ &  & $3q^{2}$ & $4$ & $q^{-2}$ \\
$-6$ &  & $q^{4}$ & $2q^{2}$ & $1$ \\
$-8$ &  &  & $q^{4}$ & $q^{2}$ \\
\end{tabular}
\vspace{2em}
\end{minipage}
%
\begin{minipage}{\linewidth}
$\bullet\ $ $10_{132}$ \vspace{0.5em} \\
\begin{tabular}{l|lll}
$k \setminus j$ & $-6$ & $-4$ & $-2$ \\
\hline
$8$ & $q^{-2}$ & $q^{-4}$ &  \\
$4$ & $q^{2}$ & $2$ & $q^{-2}$ \\
$2$ &  & $1$ & $q^{-2}$ \\
$0$ &  & $q^{4}$ & $q^{2}$ \\
$-2$ &  &  & $q^{2}$ \\
\end{tabular}
\vspace{2em}
\end{minipage}
%
\begin{minipage}{\linewidth}
$\bullet\ $ $10_{133}$ \vspace{0.5em} \\
\begin{tabular}{l|llll}
$k \setminus j$ & $2$ & $4$ & $6$ & $8$ \\
\hline
$2$ & $q^{-2}$ &  &  &  \\
$0$ & $1$ & $2q^{-2}$ & $q^{-4}$ &  \\
$-2$ & $q^{2}$ & $2$ & $q^{-2}$ &  \\
$-4$ &  & $2q^{2}$ & $3$ & $q^{-2}$ \\
$-6$ &  &  & $q^{2}$ & $1$ \\
$-8$ &  &  & $q^{4}$ & $q^{2}$ \\
\end{tabular}
\vspace{2em}
\end{minipage}
%
\begin{minipage}{\linewidth}
$\bullet\ $ $10_{134}$ \vspace{0.5em} \\
\begin{tabular}{l|llll}
$k \setminus j$ & $6$ & $8$ & $10$ & $12$ \\
\hline
$6$ & $q^{-6}$ &  &  &  \\
$4$ & $q^{-4}$ & $q^{-6}$ &  &  \\
$2$ & $2q^{-2}$ & $2q^{-4}$ &  &  \\
$0$ & $1$ & $2q^{-2}$ & $q^{-4}$ &  \\
$-2$ & $2q^{2}$ & $3$ & $q^{-2}$ &  \\
$-4$ & $q^{4}$ & $1$ + $2q^{2}$ & $q^{-2}$ + $1$ &  \\
$-6$ & $q^{6}$ & $2q^{4}$ & $q^{2}$ &  \\
$-8$ &  & $q^{6}$ & $q^{2}$ + $q^{4}$ & $1$ \\
\end{tabular}
\vspace{2em}
\end{minipage}
%
\begin{minipage}{\linewidth}
$\bullet\ $ $10_{135}$ \vspace{0.5em} \\
\begin{tabular}{l|llll}
$k \setminus j$ & $-2$ & $0$ & $2$ & $4$ \\
\hline
$4$ & $2q^{-2}$ & $2q^{-4}$ &  &  \\
$2$ & $2$ & $3q^{-2}$ & $q^{-4}$ &  \\
$0$ & $2q^{2}$ & $6$ & $3q^{-2}$ &  \\
$-2$ &  & $3q^{2}$ & $4$ & $q^{-2}$ \\
$-4$ &  & $2q^{4}$ & $3q^{2}$ & $1$ \\
$-6$ &  &  & $q^{4}$ & $q^{2}$ \\
\end{tabular}
\vspace{2em}
\end{minipage}
%
\begin{minipage}{\linewidth}
$\bullet\ $ $10_{136}$ \vspace{0.5em} \\
\begin{tabular}{l|llll}
$k \setminus j$ & $-4$ & $-2$ & $0$ & $2$ \\
\hline
$2$ & $1$ & $2q^{-2}$ & $q^{-4}$ &  \\
$0$ &  & $1$ & $q^{-2}$ + $1$ &  \\
$-2$ &  & $2q^{2}$ & $3$ & $q^{-2}$ \\
$-4$ &  &  & $q^{2}$ & $1$ \\
$-6$ &  &  & $q^{4}$ & $q^{2}$ \\
\end{tabular}
\vspace{2em}
\end{minipage}
%
\begin{minipage}{\linewidth}
$\bullet\ $ $10_{137}$ \vspace{0.5em} \\
\begin{tabular}{l|lllll}
$k \setminus j$ & $-2$ & $0$ & $2$ & $4$ & $6$ \\
\hline
$2$ & $1$ & $2q^{-2}$ & $q^{-4}$ &  &  \\
$0$ &  & $3$ & $2q^{-2}$ &  &  \\
$-2$ &  & $2q^{2}$ & $4$ & $2q^{-2}$ &  \\
$-4$ &  &  & $2q^{2}$ & $2$ &  \\
$-6$ &  &  & $q^{4}$ & $2q^{2}$ & $1$ \\
\end{tabular}
\vspace{2em}
\end{minipage}
%
\begin{minipage}{\linewidth}
$\bullet\ $ $10_{138}$ \vspace{0.5em} \\
\begin{tabular}{l|lllll}
$k \setminus j$ & $-2$ & $0$ & $2$ & $4$ & $6$ \\
\hline
$6$ & $q^{-2}$ & $2q^{-4}$ & $q^{-6}$ &  &  \\
$4$ &  & $2q^{-2}$ & $2q^{-4}$ &  &  \\
$2$ & $q^{2}$ & $3$ & $4q^{-2}$ & $q^{-4}$ &  \\
$0$ &  & $2q^{2}$ & $4$ & $2q^{-2}$ &  \\
$-2$ &  & $2q^{4}$ & $1$ + $4q^{2}$ & $q^{-2}$ + $2$ &  \\
$-4$ &  &  & $2q^{4}$ & $2q^{2}$ &  \\
$-6$ &  &  & $q^{6}$ & $q^{2}$ + $q^{4}$ & $1$ \\
\end{tabular}
\vspace{2em}
\end{minipage}
%
\begin{minipage}{\linewidth}
$\bullet\ $ $10_{139}$ \vspace{0.5em} \\
\begin{tabular}{l|lll}
$k \setminus j$ & $8$ & $10$ & $12$ \\
\hline
$8$ & $q^{-8}$ &  &  \\
$4$ & $q^{-4}$ & $q^{-6}$ &  \\
$0$ & $q^{-2}$ + $1$ & $q^{-4}$ + $q^{-2}$ &  \\
$-2$ & $1$ & $q^{-2}$ &  \\
$-4$ & $q^{2}$ + $q^{4}$ & $2$ + $q^{2}$ & $q^{-2}$ \\
$-6$ &  & $q^{2}$ & $1$ \\
$-8$ & $q^{8}$ & $q^{4}$ + $q^{6}$ & $q^{2}$ \\
\end{tabular}
\vspace{2em}
\end{minipage}
%
\begin{minipage}{\linewidth}
$\bullet\ $ $10_{140}$ \vspace{0.5em} \\
\begin{tabular}{l|llll}
$k \setminus j$ & $0$ & $2$ & $4$ & $6$ \\
\hline
$0$ & $1$ & $q^{-2}$ & $q^{-4}$ &  \\
$-4$ &  & $q^{2}$ & $2$ & $q^{-2}$ \\
$-8$ &  &  & $q^{4}$ & $q^{2}$ \\
\end{tabular}
\vspace{2em}
\end{minipage}
%
\begin{minipage}{\linewidth}
$\bullet\ $ $10_{141}$ \vspace{0.5em} \\
\begin{tabular}{l|lll}
$k \setminus j$ & $0$ & $2$ & $4$ \\
\hline
$4$ & $q^{-4}$ & $q^{-6}$ &  \\
$2$ & $q^{-2}$ & $q^{-4}$ &  \\
$0$ & $2$ & $2q^{-2}$ & $q^{-4}$ \\
$-2$ & $q^{2}$ & $2$ & $q^{-2}$ \\
$-4$ & $q^{4}$ & $2q^{2}$ & $1$ \\
$-6$ &  & $q^{4}$ & $q^{2}$ \\
$-8$ &  & $q^{6}$ & $q^{4}$ \\
\end{tabular}
\vspace{2em}
\end{minipage}
%
\begin{minipage}{\linewidth}
$\bullet\ $ $10_{142}$ \vspace{0.5em} \\
\begin{tabular}{l|llll}
$k \setminus j$ & $6$ & $8$ & $10$ & $12$ \\
\hline
$6$ & $q^{-6}$ &  &  &  \\
$4$ & $q^{-4}$ & $q^{-6}$ &  &  \\
$2$ & $q^{-2}$ & $q^{-4}$ &  &  \\
$0$ & $1$ & $2q^{-2}$ & $q^{-4}$ &  \\
$-2$ & $q^{2}$ & $1$ &  &  \\
$-4$ & $q^{4}$ & $1$ + $2q^{2}$ & $q^{-2}$ + $1$ &  \\
$-6$ & $q^{6}$ & $q^{4}$ &  &  \\
$-8$ &  & $q^{6}$ & $q^{2}$ + $q^{4}$ & $1$ \\
\end{tabular}
\vspace{2em}
\end{minipage}
%
\begin{minipage}{\linewidth}
$\bullet\ $ $10_{143}$ \vspace{0.5em} \\
\begin{tabular}{l|lll}
$k \setminus j$ & $-6$ & $-4$ & $-2$ \\
\hline
$8$ & $q^{-4}$ & $q^{-6}$ &  \\
$6$ & $q^{-2}$ & $q^{-4}$ &  \\
$4$ & $2$ & $3q^{-2}$ & $q^{-4}$ \\
$2$ & $q^{2}$ & $3$ & $2q^{-2}$ \\
$0$ & $q^{4}$ & $3q^{2}$ & $2$ \\
$-2$ &  & $q^{4}$ & $2q^{2}$ \\
$-4$ &  & $q^{6}$ & $q^{4}$ \\
\end{tabular}
\vspace{2em}
\end{minipage}
%
\begin{minipage}{\linewidth}
$\bullet\ $ $10_{144}$ \vspace{0.5em} \\
\begin{tabular}{l|llll}
$k \setminus j$ & $0$ & $2$ & $4$ & $6$ \\
\hline
$4$ & $2q^{-2}$ & $2q^{-4}$ &  &  \\
$2$ & $1$ & $3q^{-2}$ & $q^{-4}$ &  \\
$0$ & $2q^{2}$ & $6$ & $4q^{-2}$ &  \\
$-2$ &  & $3q^{2}$ & $4$ & $q^{-2}$ \\
$-4$ &  & $2q^{4}$ & $4q^{2}$ & $2$ \\
$-6$ &  &  & $q^{4}$ & $q^{2}$ \\
\end{tabular}
\vspace{2em}
\end{minipage}
%
\begin{minipage}{\linewidth}
$\bullet\ $ $10_{145}$ \vspace{0.5em} \\
\begin{tabular}{l|llll}
$k \setminus j$ & $-10$ & $-8$ & $-6$ & $-4$ \\
\hline
$8$ & $1$ & $q^{-2}$ &  &  \\
$6$ &  & $1$ & $q^{-2}$ &  \\
$4$ &  & $q^{2}$ & $q^{-2}$ + $1$ & $q^{-4}$ \\
$2$ &  &  & $q^{2}$ & $1$ \\
$0$ &  &  & $q^{2}$ & $1$ \\
$-4$ &  &  &  & $q^{4}$ \\
\end{tabular}
\vspace{2em}
\end{minipage}
%
\begin{minipage}{\linewidth}
$\bullet\ $ $10_{146}$ \vspace{0.5em} \\
\begin{tabular}{l|llll}
$k \setminus j$ & $-4$ & $-2$ & $0$ & $2$ \\
\hline
$6$ & $q^{-2}$ & $q^{-4}$ &  &  \\
$4$ & $2$ & $3q^{-2}$ & $q^{-4}$ &  \\
$2$ & $q^{2}$ & $4$ & $3q^{-2}$ &  \\
$0$ &  & $3q^{2}$ & $5$ & $q^{-2}$ \\
$-2$ &  & $q^{4}$ & $3q^{2}$ & $2$ \\
$-4$ &  &  & $q^{4}$ & $q^{2}$ \\
\end{tabular}
\vspace{2em}
\end{minipage}
%
\begin{minipage}{\linewidth}
$\bullet\ $ $10_{147}$ \vspace{0.5em} \\
\begin{tabular}{l|llll}
$k \setminus j$ & $-2$ & $0$ & $2$ & $4$ \\
\hline
$6$ & $q^{-2}$ & $q^{-4}$ &  &  \\
$4$ & $1$ & $2q^{-2}$ & $q^{-4}$ &  \\
$2$ & $q^{2}$ & $3$ & $3q^{-2}$ &  \\
$0$ &  & $2q^{2}$ & $3$ & $q^{-2}$ \\
$-2$ &  & $q^{4}$ & $3q^{2}$ & $2$ \\
$-4$ &  &  & $q^{4}$ & $q^{2}$ \\
\end{tabular}
\vspace{2em}
\end{minipage}
%
\begin{minipage}{\linewidth}
$\bullet\ $ $10_{148}$ \vspace{0.5em} \\
\begin{tabular}{l|lll}
$k \setminus j$ & $-6$ & $-4$ & $-2$ \\
\hline
$8$ & $q^{-4}$ & $q^{-6}$ &  \\
$6$ & $q^{-2}$ & $q^{-4}$ &  \\
$4$ & $3$ & $4q^{-2}$ & $q^{-4}$ \\
$2$ & $q^{2}$ & $3$ & $2q^{-2}$ \\
$0$ & $q^{4}$ & $4q^{2}$ & $3$ \\
$-2$ &  & $q^{4}$ & $2q^{2}$ \\
$-4$ &  & $q^{6}$ & $q^{4}$ \\
\end{tabular}
\vspace{2em}
\end{minipage}
%
\begin{minipage}{\linewidth}
$\bullet\ $ $10_{149}$ \vspace{0.5em} \\
\begin{tabular}{l|lll}
$k \setminus j$ & $4$ & $6$ & $8$ \\
\hline
$4$ & $2q^{-4}$ & $q^{-6}$ &  \\
$2$ & $2q^{-2}$ & $2q^{-4}$ &  \\
$0$ & $4$ & $5q^{-2}$ & $q^{-4}$ \\
$-2$ & $2q^{2}$ & $4$ & $2q^{-2}$ \\
$-4$ & $2q^{4}$ & $5q^{2}$ & $3$ \\
$-6$ &  & $2q^{4}$ & $2q^{2}$ \\
$-8$ &  & $q^{6}$ & $q^{4}$ \\
\end{tabular}
\vspace{2em}
\end{minipage}
%
\begin{minipage}{\linewidth}
$\bullet\ $ $10_{150}$ \vspace{0.5em} \\
\begin{tabular}{l|lll}
$k \setminus j$ & $2$ & $4$ & $6$ \\
\hline
$6$ & $q^{-4}$ & $q^{-6}$ &  \\
$4$ & $q^{-2}$ & $2q^{-4}$ &  \\
$2$ & $2$ & $3q^{-2}$ & $q^{-4}$ \\
$0$ & $q^{2}$ & $3$ & $2q^{-2}$ \\
$-2$ & $q^{4}$ & $3q^{2}$ & $2$ \\
$-4$ &  & $2q^{4}$ & $2q^{2}$ \\
$-6$ &  & $q^{6}$ & $q^{4}$ \\
\end{tabular}
\vspace{2em}
\end{minipage}
%
\begin{minipage}{\linewidth}
$\bullet\ $ $10_{151}$ \vspace{0.5em} \\
\begin{tabular}{l|llll}
$k \setminus j$ & $-6$ & $-4$ & $-2$ & $0$ \\
\hline
$6$ &  & $q^{-4}$ & $q^{-6}$ &  \\
$4$ & $1$ & $3q^{-2}$ & $2q^{-4}$ &  \\
$2$ &  & $4$ & $5q^{-2}$ & $q^{-4}$ \\
$0$ &  & $3q^{2}$ & $5$ & $2q^{-2}$ \\
$-2$ &  & $q^{4}$ & $5q^{2}$ & $3$ \\
$-4$ &  &  & $2q^{4}$ & $2q^{2}$ \\
$-6$ &  &  & $q^{6}$ & $q^{4}$ \\
\end{tabular}
\vspace{2em}
\end{minipage}
%
\begin{minipage}{\linewidth}
$\bullet\ $ $10_{152}$ \vspace{0.5em} \\
\begin{tabular}{l|lll}
$k \setminus j$ & $8$ & $10$ & $12$ \\
\hline
$8$ & $q^{-8}$ &  &  \\
$4$ & $q^{-4}$ & $q^{-6}$ &  \\
$0$ & $2q^{-2}$ + $1$ & $2q^{-4}$ + $q^{-2}$ &  \\
$-2$ & $1$ & $q^{-2}$ &  \\
$-4$ & $2q^{2}$ + $q^{4}$ & $4$ + $q^{2}$ & $2q^{-2}$ \\
$-6$ &  & $q^{2}$ & $1$ \\
$-8$ & $q^{8}$ & $2q^{4}$ + $q^{6}$ & $2q^{2}$ \\
\end{tabular}
\vspace{2em}
\end{minipage}
%
\begin{minipage}{\linewidth}
$\bullet\ $ $10_{153}$ \vspace{0.5em} \\
\begin{tabular}{l|llll}
$k \setminus j$ & $-4$ & $-2$ & $0$ & $2$ \\
\hline
$6$ & $q^{-2}$ & $q^{-4}$ &  &  \\
$4$ & $1$ & $q^{-4}$ + $q^{-2}$ & $q^{-6}$ &  \\
$2$ & $q^{2}$ & $2$ & $q^{-2}$ &  \\
$0$ &  & $1$ + $q^{2}$ & $2q^{-2}$ + $2$ & $q^{-4}$ \\
$-2$ &  & $q^{4}$ & $q^{2}$ &  \\
$-4$ &  & $q^{4}$ & $2q^{2}$ & $1$ \\
$-8$ &  &  & $q^{6}$ & $q^{4}$ \\
\end{tabular}
\vspace{2em}
\end{minipage}
%
\begin{minipage}{\linewidth}
$\bullet\ $ $10_{154}$ \vspace{0.5em} \\
\begin{tabular}{l|llll}
$k \setminus j$ & $6$ & $8$ & $10$ & $12$ \\
\hline
$6$ & $q^{-6}$ &  &  &  \\
$2$ & $q^{-2}$ & $q^{-4}$ &  &  \\
$0$ & $q^{-2}$ & $q^{-4}$ &  &  \\
$-2$ & $2$ + $q^{2}$ & $2q^{-2}$ + $1$ &  &  \\
$-4$ & $q^{2}$ & $3$ & $2q^{-2}$ &  \\
$-6$ & $q^{6}$ & $2q^{2}$ + $q^{4}$ & $2$ &  \\
$-8$ &  & $q^{4}$ & $2q^{2}$ & $1$ \\
\end{tabular}
\vspace{2em}
\end{minipage}
%
\begin{minipage}{\linewidth}
$\bullet\ $ $10_{155}$ \vspace{0.5em} \\
\begin{tabular}{l|lll}
$k \setminus j$ & $-4$ & $-2$ & $0$ \\
\hline
$8$ & $q^{-4}$ & $q^{-6}$ &  \\
$6$ & $q^{-2}$ & $q^{-4}$ &  \\
$4$ & $2$ & $3q^{-2}$ & $q^{-4}$ \\
$2$ & $q^{2}$ & $2$ & $q^{-2}$ \\
$0$ & $q^{4}$ & $3q^{2}$ & $3$ \\
$-2$ &  & $q^{4}$ & $q^{2}$ \\
$-4$ &  & $q^{6}$ & $q^{4}$ \\
\end{tabular}
\vspace{2em}
\end{minipage}
%
\begin{minipage}{\linewidth}
$\bullet\ $ $10_{156}$ \vspace{0.5em} \\
\begin{tabular}{l|lll}
$k \setminus j$ & $-4$ & $-2$ & $0$ \\
\hline
$6$ & $q^{-4}$ & $q^{-6}$ &  \\
$4$ & $2q^{-2}$ & $2q^{-4}$ &  \\
$2$ & $3$ & $4q^{-2}$ & $q^{-4}$ \\
$0$ & $2q^{2}$ & $4$ & $2q^{-2}$ \\
$-2$ & $q^{4}$ & $4q^{2}$ & $2$ \\
$-4$ &  & $2q^{4}$ & $2q^{2}$ \\
$-6$ &  & $q^{6}$ & $q^{4}$ \\
\end{tabular}
\vspace{2em}
\end{minipage}
%
\begin{minipage}{\linewidth}
$\bullet\ $ $10_{157}$ \vspace{0.5em} \\
\begin{tabular}{l|lll}
$k \setminus j$ & $-8$ & $-6$ & $-4$ \\
\hline
$8$ & $q^{-4}$ & $q^{-6}$ &  \\
$6$ & $3q^{-2}$ & $3q^{-4}$ &  \\
$4$ & $3$ & $5q^{-2}$ & $2q^{-4}$ \\
$2$ & $3q^{2}$ & $6$ & $3q^{-2}$ \\
$0$ & $q^{4}$ & $5q^{2}$ & $4$ \\
$-2$ &  & $3q^{4}$ & $3q^{2}$ \\
$-4$ &  & $q^{6}$ & $2q^{4}$ \\
\end{tabular}
\vspace{2em}
\end{minipage}
%
\begin{minipage}{\linewidth}
$\bullet\ $ $10_{158}$ \vspace{0.5em} \\
\begin{tabular}{l|llll}
$k \setminus j$ & $-2$ & $0$ & $2$ & $4$ \\
\hline
$6$ & $q^{-4}$ & $q^{-6}$ &  &  \\
$4$ & $2q^{-2}$ & $2q^{-4}$ &  &  \\
$2$ & $4$ & $5q^{-2}$ & $q^{-4}$ &  \\
$0$ & $2q^{2}$ & $6$ & $3q^{-2}$ &  \\
$-2$ & $q^{4}$ & $5q^{2}$ & $4$ &  \\
$-4$ &  & $2q^{4}$ & $3q^{2}$ & $1$ \\
$-6$ &  & $q^{6}$ & $q^{4}$ &  \\
\end{tabular}
\vspace{2em}
\end{minipage}
%
\begin{minipage}{\linewidth}
$\bullet\ $ $10_{159}$ \vspace{0.5em} \\
\begin{tabular}{l|lll}
$k \setminus j$ & $2$ & $4$ & $6$ \\
\hline
$4$ & $q^{-4}$ & $q^{-6}$ &  \\
$2$ & $3q^{-2}$ & $2q^{-4}$ &  \\
$0$ & $3$ & $4q^{-2}$ & $q^{-4}$ \\
$-2$ & $3q^{2}$ & $5$ & $2q^{-2}$ \\
$-4$ & $q^{4}$ & $4q^{2}$ & $3$ \\
$-6$ &  & $2q^{4}$ & $2q^{2}$ \\
$-8$ &  & $q^{6}$ & $q^{4}$ \\
\end{tabular}
\vspace{2em}
\end{minipage}
%
\begin{minipage}{\linewidth}
$\bullet\ $ $10_{160}$ \vspace{0.5em} \\
\begin{tabular}{l|llll}
$k \setminus j$ & $2$ & $4$ & $6$ & $8$ \\
\hline
$6$ & $q^{-4}$ & $q^{-6}$ &  &  \\
$4$ & $q^{-2}$ & $2q^{-4}$ &  &  \\
$2$ & $1$ & $2q^{-2}$ & $q^{-4}$ &  \\
$0$ & $q^{2}$ & $3$ & $2q^{-2}$ &  \\
$-2$ & $q^{4}$ & $1$ + $2q^{2}$ & $q^{-2}$ + $1$ &  \\
$-4$ &  & $2q^{4}$ & $2q^{2}$ &  \\
$-6$ &  & $q^{6}$ & $q^{2}$ + $q^{4}$ & $1$ \\
\end{tabular}
\vspace{2em}
\end{minipage}
%
\begin{minipage}{\linewidth}
$\bullet\ $ $10_{161}$ \vspace{0.5em} \\
\begin{tabular}{l|lll}
$k \setminus j$ & $6$ & $8$ & $10$ \\
\hline
$6$ & $q^{-6}$ &  &  \\
$2$ & $q^{-2}$ & $q^{-4}$ &  \\
$0$ & $q^{-2}$ & $q^{-4}$ &  \\
$-2$ & $1$ + $q^{2}$ & $q^{-2}$ + $1$ &  \\
$-4$ & $q^{2}$ & $2$ & $q^{-2}$ \\
$-6$ & $q^{6}$ & $q^{2}$ + $q^{4}$ & $1$ \\
$-8$ &  & $q^{4}$ & $q^{2}$ \\
\end{tabular}
\vspace{2em}
\end{minipage}
%
\begin{minipage}{\linewidth}
$\bullet\ $ $10_{162}$ \vspace{0.5em} \\
\begin{tabular}{l|llll}
$k \setminus j$ & $-6$ & $-4$ & $-2$ & $0$ \\
\hline
$6$ & $q^{-2}$ & $q^{-4}$ &  &  \\
$4$ & $1$ & $3q^{-2}$ & $2q^{-4}$ &  \\
$2$ & $q^{2}$ & $4$ & $3q^{-2}$ &  \\
$0$ &  & $3q^{2}$ & $5$ & $2q^{-2}$ \\
$-2$ &  & $q^{4}$ & $3q^{2}$ & $1$ \\
$-4$ &  &  & $2q^{4}$ & $2q^{2}$ \\
\end{tabular}
\vspace{2em}
\end{minipage}
%
\begin{minipage}{\linewidth}
$\bullet\ $ $10_{163}$ \vspace{0.5em} \\
\begin{tabular}{l|llll}
$k \setminus j$ & $0$ & $2$ & $4$ & $6$ \\
\hline
$6$ & $q^{-4}$ & $q^{-6}$ &  &  \\
$4$ & $3q^{-2}$ & $3q^{-4}$ &  &  \\
$2$ & $3$ & $5q^{-2}$ & $q^{-4}$ &  \\
$0$ & $3q^{2}$ & $7$ & $4q^{-2}$ &  \\
$-2$ & $q^{4}$ & $5q^{2}$ & $4$ &  \\
$-4$ &  & $3q^{4}$ & $4q^{2}$ & $1$ \\
$-6$ &  & $q^{6}$ & $q^{4}$ &  \\
\end{tabular}
\vspace{2em}
\end{minipage}
%
\begin{minipage}{\linewidth}
$\bullet\ $ $10_{164}$ \vspace{0.5em} \\
\begin{tabular}{l|llll}
$k \setminus j$ & $-4$ & $-2$ & $0$ & $2$ \\
\hline
$6$ & $q^{-2}$ & $q^{-4}$ &  &  \\
$4$ & $2$ & $4q^{-2}$ & $2q^{-4}$ &  \\
$2$ & $q^{2}$ & $5$ & $4q^{-2}$ &  \\
$0$ &  & $4q^{2}$ & $7$ & $2q^{-2}$ \\
$-2$ &  & $q^{4}$ & $4q^{2}$ & $3$ \\
$-4$ &  &  & $2q^{4}$ & $2q^{2}$ \\
\end{tabular}
\vspace{2em}
\end{minipage}
%
\begin{minipage}{\linewidth}
$\bullet\ $ $10_{165}$ \vspace{0.5em} \\
\begin{tabular}{l|llll}
$k \setminus j$ & $-8$ & $-6$ & $-4$ & $-2$ \\
\hline
$8$ & $q^{-2}$ & $q^{-4}$ &  &  \\
$6$ & $2$ & $3q^{-2}$ & $q^{-4}$ &  \\
$4$ & $q^{2}$ & $5$ & $4q^{-2}$ &  \\
$2$ &  & $3q^{2}$ & $5$ & $2q^{-2}$ \\
$0$ &  & $q^{4}$ & $4q^{2}$ & $3$ \\
$-2$ &  &  & $q^{4}$ & $2q^{2}$ \\
\end{tabular}
\vspace{2em}
\end{minipage}
%
\end{multicols}

\newgeometry{top=2.5cm, bottom=2.5cm, left=3.5cm, right=3.5cm} 

\begin{multicols}{2}
    \setlength\parindent{0pt}
    \small
    \setlength{\tabcolsep}{4pt}
    \begin{minipage}{\linewidth}
$\bullet\ $ $11a_{8}$ \vspace{0.5em} \\
\begin{tabular}{l|lllll}
$k \setminus j$ & $-6$ & $-4$ & $-2$ & $0$ & $2$ \\
\hline
$8$ & $q^{-2}$ & $2q^{-4}$ & $q^{-6}$ &  &  \\
$6$ & $1$ & $4q^{-2}$ & $4q^{-4}$ & $q^{-6}$ &  \\
$4$ & $q^{2}$ & $7$ & $10q^{-2}$ & $4q^{-4}$ &  \\
$2$ &  & $4q^{2}$ & $12$ & $9q^{-2}$ & $q^{-4}$ \\
$0$ &  & $2q^{4}$ & $10q^{2}$ & $12$ & $3q^{-2}$ \\
$-2$ &  &  & $4q^{4}$ & $9q^{2}$ & $5$ \\
$-4$ &  &  & $q^{6}$ & $4q^{4}$ & $3q^{2}$ \\
$-6$ &  &  &  & $q^{6}$ & $q^{4}$ \\
\end{tabular}
\vspace{2em}
\end{minipage}
%
\begin{minipage}{\linewidth}
$\bullet\ $ $11a_{9}$ \vspace{0.5em} \\
\begin{tabular}{l|llll}
$k \setminus j$ & $0$ & $2$ & $4$ & $6$ \\
\hline
$8$ & $q^{-4}$ & $2q^{-6}$ & $q^{-8}$ &  \\
$6$ &  & $2q^{-4}$ & $2q^{-6}$ &  \\
$4$ & $1$ & $4q^{-2}$ & $5q^{-4}$ & $q^{-6}$ \\
$2$ &  & $2$ & $4q^{-2}$ & $2q^{-4}$ \\
$0$ & $q^{4}$ & $4q^{2}$ & $6$ & $3q^{-2}$ \\
$-2$ &  & $2q^{4}$ & $4q^{2}$ & $2$ \\
$-4$ &  & $2q^{6}$ & $5q^{4}$ & $3q^{2}$ \\
$-6$ &  &  & $2q^{6}$ & $2q^{4}$ \\
$-8$ &  &  & $q^{8}$ & $q^{6}$ \\
\end{tabular}
\vspace{2em}
\end{minipage}
%
\begin{minipage}{\linewidth}
$\bullet\ $ $11a_{11}$ \vspace{0.5em} \\
\begin{tabular}{l|lllll}
$k \setminus j$ & $-4$ & $-2$ & $0$ & $2$ & $4$ \\
\hline
$6$ & $q^{-2}$ & $2q^{-4}$ & $q^{-6}$ &  &  \\
$4$ & $1$ & $4q^{-2}$ & $4q^{-4}$ & $q^{-6}$ &  \\
$2$ & $q^{2}$ & $7$ & $10q^{-2}$ & $4q^{-4}$ &  \\
$0$ &  & $4q^{2}$ & $12$ & $8q^{-2}$ & $q^{-4}$ \\
$-2$ &  & $2q^{4}$ & $10q^{2}$ & $11$ & $3q^{-2}$ \\
$-4$ &  &  & $4q^{4}$ & $8q^{2}$ & $4$ \\
$-6$ &  &  & $q^{6}$ & $4q^{4}$ & $3q^{2}$ \\
$-8$ &  &  &  & $q^{6}$ & $q^{4}$ \\
\end{tabular}
\vspace{2em}
\end{minipage}
%
\begin{minipage}{\linewidth}
$\bullet\ $ $11a_{12}$ \vspace{0.5em} \\
\begin{tabular}{l|lllll}
$k \setminus j$ & $-2$ & $0$ & $2$ & $4$ & $6$ \\
\hline
$6$ & $q^{-2}$ & $2q^{-4}$ & $q^{-6}$ &  &  \\
$4$ &  & $3q^{-2}$ & $4q^{-4}$ & $q^{-6}$ &  \\
$2$ & $q^{2}$ & $5$ & $9q^{-2}$ & $4q^{-4}$ &  \\
$0$ &  & $3q^{2}$ & $10$ & $8q^{-2}$ & $q^{-4}$ \\
$-2$ &  & $2q^{4}$ & $9q^{2}$ & $10$ & $3q^{-2}$ \\
$-4$ &  &  & $4q^{4}$ & $8q^{2}$ & $4$ \\
$-6$ &  &  & $q^{6}$ & $4q^{4}$ & $3q^{2}$ \\
$-8$ &  &  &  & $q^{6}$ & $q^{4}$ \\
\end{tabular}
\vspace{2em}
\end{minipage}
%
\begin{minipage}{\linewidth}
$\bullet\ $ $11a_{13}$ \vspace{0.5em} \\
\begin{tabular}{l|llllll}
$k \setminus j$ & $-4$ & $-2$ & $0$ & $2$ & $4$ & $6$ \\
\hline
$4$ & $1$ & $2q^{-2}$ & $q^{-4}$ &  &  &  \\
$2$ &  & $3$ & $4q^{-2}$ & $q^{-4}$ &  &  \\
$0$ &  & $2q^{2}$ & $7$ & $5q^{-2}$ & $q^{-4}$ &  \\
$-2$ &  &  & $4q^{2}$ & $7$ & $3q^{-2}$ &  \\
$-4$ &  &  & $q^{4}$ & $5q^{2}$ & $5$ & $q^{-2}$ \\
$-6$ &  &  &  & $q^{4}$ & $3q^{2}$ & $2$ \\
$-8$ &  &  &  &  & $q^{4}$ & $q^{2}$ \\
\end{tabular}
\vspace{2em}
\end{minipage}
%
\begin{minipage}{\linewidth}
$\bullet\ $ $11a_{14}$ \vspace{0.5em} \\
\begin{tabular}{l|llll}
$k \setminus j$ & $-2$ & $0$ & $2$ & $4$ \\
\hline
$8$ & $q^{-6}$ & $q^{-8}$ &  &  \\
$6$ & $2q^{-4}$ & $2q^{-6}$ &  &  \\
$4$ & $6q^{-2}$ & $8q^{-4}$ & $2q^{-6}$ &  \\
$2$ & $6$ & $10q^{-2}$ & $4q^{-4}$ &  \\
$0$ & $6q^{2}$ & $16$ & $10q^{-2}$ & $q^{-4}$ \\
$-2$ & $2q^{4}$ & $10q^{2}$ & $10$ & $2q^{-2}$ \\
$-4$ & $q^{6}$ & $8q^{4}$ & $10q^{2}$ & $3$ \\
$-6$ &  & $2q^{6}$ & $4q^{4}$ & $2q^{2}$ \\
$-8$ &  & $q^{8}$ & $2q^{6}$ & $q^{4}$ \\
\end{tabular}
\vspace{2em}
\end{minipage}
%
\begin{minipage}{\linewidth}
$\bullet\ $ $11a_{15}$ \vspace{0.5em} \\
\begin{tabular}{l|llll}
$k \setminus j$ & $-4$ & $-2$ & $0$ & $2$ \\
\hline
$8$ & $q^{-6}$ & $q^{-8}$ &  &  \\
$6$ & $2q^{-4}$ & $2q^{-6}$ &  &  \\
$4$ & $5q^{-2}$ & $7q^{-4}$ & $2q^{-6}$ &  \\
$2$ & $5$ & $8q^{-2}$ & $3q^{-4}$ &  \\
$0$ & $5q^{2}$ & $12$ & $8q^{-2}$ & $q^{-4}$ \\
$-2$ & $2q^{4}$ & $8q^{2}$ & $6$ & $q^{-2}$ \\
$-4$ & $q^{6}$ & $7q^{4}$ & $8q^{2}$ & $2$ \\
$-6$ &  & $2q^{6}$ & $3q^{4}$ & $q^{2}$ \\
$-8$ &  & $q^{8}$ & $2q^{6}$ & $q^{4}$ \\
\end{tabular}
\vspace{2em}
\end{minipage}
%
\begin{minipage}{\linewidth}
$\bullet\ $ $11a_{16}$ \vspace{0.5em} \\
\begin{tabular}{l|lllll}
$k \setminus j$ & $-4$ & $-2$ & $0$ & $2$ & $4$ \\
\hline
$8$ & $q^{-4}$ & $q^{-6}$ &  &  &  \\
$6$ & $2q^{-2}$ & $3q^{-4}$ & $q^{-6}$ &  &  \\
$4$ & $4$ & $8q^{-2}$ & $4q^{-4}$ &  &  \\
$2$ & $2q^{2}$ & $9$ & $9q^{-2}$ & $2q^{-4}$ &  \\
$0$ & $q^{4}$ & $8q^{2}$ & $12$ & $4q^{-2}$ &  \\
$-2$ &  & $3q^{4}$ & $9q^{2}$ & $7$ & $q^{-2}$ \\
$-4$ &  & $q^{6}$ & $4q^{4}$ & $4q^{2}$ & $1$ \\
$-6$ &  &  & $q^{6}$ & $2q^{4}$ & $q^{2}$ \\
\end{tabular}
\vspace{2em}
\end{minipage}
%
\begin{minipage}{\linewidth}
$\bullet\ $ $11a_{17}$ \vspace{0.5em} \\
\begin{tabular}{l|llllll}
$k \setminus j$ & $-2$ & $0$ & $2$ & $4$ & $6$ & $8$ \\
\hline
$6$ & $q^{-2}$ & $2q^{-4}$ & $q^{-6}$ &  &  &  \\
$4$ & $1$ & $6q^{-2}$ & $5q^{-4}$ &  &  &  \\
$2$ & $q^{2}$ & $8$ & $11q^{-2}$ & $3q^{-4}$ &  &  \\
$0$ &  & $6q^{2}$ & $15$ & $9q^{-2}$ &  &  \\
$-2$ &  & $2q^{4}$ & $11q^{2}$ & $12$ & $3q^{-2}$ &  \\
$-4$ &  &  & $5q^{4}$ & $9q^{2}$ & $4$ &  \\
$-6$ &  &  & $q^{6}$ & $3q^{4}$ & $3q^{2}$ & $1$ \\
\end{tabular}
\vspace{2em}
\end{minipage}
%
\begin{minipage}{\linewidth}
$\bullet\ $ $11a_{18}$ \vspace{0.5em} \\
\begin{tabular}{l|lllll}
$k \setminus j$ & $0$ & $2$ & $4$ & $6$ & $8$ \\
\hline
$6$ & $q^{-4}$ & $q^{-6}$ &  &  &  \\
$4$ & $2q^{-2}$ & $4q^{-4}$ & $2q^{-6}$ &  &  \\
$2$ & $3$ & $9q^{-2}$ & $5q^{-4}$ &  &  \\
$0$ & $2q^{2}$ & $11$ & $12q^{-2}$ & $3q^{-4}$ &  \\
$-2$ & $q^{4}$ & $9q^{2}$ & $13$ & $5q^{-2}$ &  \\
$-4$ &  & $4q^{4}$ & $12q^{2}$ & $9$ & $q^{-2}$ \\
$-6$ &  & $q^{6}$ & $5q^{4}$ & $5q^{2}$ & $1$ \\
$-8$ &  &  & $2q^{6}$ & $3q^{4}$ & $q^{2}$ \\
\end{tabular}
\vspace{2em}
\end{minipage}
%
\begin{minipage}{\linewidth}
$\bullet\ $ $11a_{19}$ \vspace{0.5em} \\
\begin{tabular}{l|llll}
$k \setminus j$ & $0$ & $2$ & $4$ & $6$ \\
\hline
$8$ & $q^{-6}$ & $q^{-8}$ &  &  \\
$6$ & $3q^{-4}$ & $3q^{-6}$ &  &  \\
$4$ & $7q^{-2}$ & $9q^{-4}$ & $2q^{-6}$ &  \\
$2$ & $7$ & $13q^{-2}$ & $5q^{-4}$ &  \\
$0$ & $7q^{2}$ & $17$ & $11q^{-2}$ & $q^{-4}$ \\
$-2$ & $3q^{4}$ & $13q^{2}$ & $12$ & $2q^{-2}$ \\
$-4$ & $q^{6}$ & $9q^{4}$ & $11q^{2}$ & $3$ \\
$-6$ &  & $3q^{6}$ & $5q^{4}$ & $2q^{2}$ \\
$-8$ &  & $q^{8}$ & $2q^{6}$ & $q^{4}$ \\
\end{tabular}
\vspace{2em}
\end{minipage}
%
\begin{minipage}{\linewidth}
$\bullet\ $ $11a_{20}$ \vspace{0.5em} \\
\begin{tabular}{l|lllll}
$k \setminus j$ & $2$ & $4$ & $6$ & $8$ & $10$ \\
\hline
$6$ & $q^{-4}$ & $q^{-6}$ &  &  &  \\
$4$ & $q^{-2}$ & $4q^{-4}$ & $2q^{-6}$ &  &  \\
$2$ & $2$ & $7q^{-2}$ & $5q^{-4}$ &  &  \\
$0$ & $q^{2}$ & $9$ & $11q^{-2}$ & $3q^{-4}$ &  \\
$-2$ & $q^{4}$ & $7q^{2}$ & $11$ & $5q^{-2}$ &  \\
$-4$ &  & $4q^{4}$ & $11q^{2}$ & $8$ & $q^{-2}$ \\
$-6$ &  & $q^{6}$ & $5q^{4}$ & $5q^{2}$ & $1$ \\
$-8$ &  &  & $2q^{6}$ & $3q^{4}$ & $q^{2}$ \\
\end{tabular}
\vspace{2em}
\end{minipage}
%
\begin{minipage}{\linewidth}
$\bullet\ $ $11a_{21}$ \vspace{0.5em} \\
\begin{tabular}{l|llllll}
$k \setminus j$ & $0$ & $2$ & $4$ & $6$ & $8$ & $10$ \\
\hline
$4$ & $q^{-2}$ & $q^{-4}$ &  &  &  &  \\
$2$ & $1$ & $4q^{-2}$ & $2q^{-4}$ &  &  &  \\
$0$ & $q^{2}$ & $5$ & $6q^{-2}$ & $2q^{-4}$ &  &  \\
$-2$ &  & $4q^{2}$ & $9$ & $5q^{-2}$ &  &  \\
$-4$ &  & $q^{4}$ & $6q^{2}$ & $8$ & $3q^{-2}$ &  \\
$-6$ &  &  & $2q^{4}$ & $5q^{2}$ & $3$ &  \\
$-8$ &  &  &  & $2q^{4}$ & $3q^{2}$ & $1$ \\
\end{tabular}
\vspace{2em}
\end{minipage}
%
\begin{minipage}{\linewidth}
$\bullet\ $ $11a_{22}$ \vspace{0.5em} \\
\begin{tabular}{l|llll}
$k \setminus j$ & $2$ & $4$ & $6$ & $8$ \\
\hline
$8$ & $q^{-6}$ & $q^{-8}$ &  &  \\
$6$ & $2q^{-4}$ & $2q^{-6}$ &  &  \\
$4$ & $4q^{-2}$ & $7q^{-4}$ & $2q^{-6}$ &  \\
$2$ & $4$ & $7q^{-2}$ & $3q^{-4}$ &  \\
$0$ & $4q^{2}$ & $11$ & $8q^{-2}$ & $q^{-4}$ \\
$-2$ & $2q^{4}$ & $7q^{2}$ & $6$ & $q^{-2}$ \\
$-4$ & $q^{6}$ & $7q^{4}$ & $8q^{2}$ & $2$ \\
$-6$ &  & $2q^{6}$ & $3q^{4}$ & $q^{2}$ \\
$-8$ &  & $q^{8}$ & $2q^{6}$ & $q^{4}$ \\
\end{tabular}
\vspace{2em}
\end{minipage}
%
\begin{minipage}{\linewidth}
$\bullet\ $ $11a_{23}$ \vspace{0.5em} \\
\begin{tabular}{l|lllll}
$k \setminus j$ & $0$ & $2$ & $4$ & $6$ & $8$ \\
\hline
$6$ & $q^{-4}$ & $q^{-6}$ &  &  &  \\
$4$ & $2q^{-2}$ & $3q^{-4}$ & $q^{-6}$ &  &  \\
$2$ & $3$ & $8q^{-2}$ & $4q^{-4}$ &  &  \\
$0$ & $2q^{2}$ & $9$ & $9q^{-2}$ & $2q^{-4}$ &  \\
$-2$ & $q^{4}$ & $8q^{2}$ & $11$ & $4q^{-2}$ &  \\
$-4$ &  & $3q^{4}$ & $9q^{2}$ & $7$ & $q^{-2}$ \\
$-6$ &  & $q^{6}$ & $4q^{4}$ & $4q^{2}$ & $1$ \\
$-8$ &  &  & $q^{6}$ & $2q^{4}$ & $q^{2}$ \\
\end{tabular}
\vspace{2em}
\end{minipage}
%
\begin{minipage}{\linewidth}
$\bullet\ $ $11a_{24}$ \vspace{0.5em} \\
\begin{tabular}{l|llll}
$k \setminus j$ & $-2$ & $0$ & $2$ & $4$ \\
\hline
$8$ & $q^{-6}$ & $q^{-8}$ &  &  \\
$6$ & $3q^{-4}$ & $3q^{-6}$ &  &  \\
$4$ & $7q^{-2}$ & $9q^{-4}$ & $2q^{-6}$ &  \\
$2$ & $8$ & $13q^{-2}$ & $5q^{-4}$ &  \\
$0$ & $7q^{2}$ & $18$ & $11q^{-2}$ & $q^{-4}$ \\
$-2$ & $3q^{4}$ & $13q^{2}$ & $12$ & $2q^{-2}$ \\
$-4$ & $q^{6}$ & $9q^{4}$ & $11q^{2}$ & $3$ \\
$-6$ &  & $3q^{6}$ & $5q^{4}$ & $2q^{2}$ \\
$-8$ &  & $q^{8}$ & $2q^{6}$ & $q^{4}$ \\
\end{tabular}
\vspace{2em}
\end{minipage}
%
\begin{minipage}{\linewidth}
$\bullet\ $ $11a_{25}$ \vspace{0.5em} \\
\begin{tabular}{l|llll}
$k \setminus j$ & $0$ & $2$ & $4$ & $6$ \\
\hline
$8$ & $q^{-6}$ & $q^{-8}$ &  &  \\
$6$ & $3q^{-4}$ & $3q^{-6}$ &  &  \\
$4$ & $7q^{-2}$ & $9q^{-4}$ & $2q^{-6}$ &  \\
$2$ & $7$ & $13q^{-2}$ & $5q^{-4}$ &  \\
$0$ & $7q^{2}$ & $17$ & $11q^{-2}$ & $q^{-4}$ \\
$-2$ & $3q^{4}$ & $13q^{2}$ & $12$ & $2q^{-2}$ \\
$-4$ & $q^{6}$ & $9q^{4}$ & $11q^{2}$ & $3$ \\
$-6$ &  & $3q^{6}$ & $5q^{4}$ & $2q^{2}$ \\
$-8$ &  & $q^{8}$ & $2q^{6}$ & $q^{4}$ \\
\end{tabular}
\vspace{2em}
\end{minipage}
%
\begin{minipage}{\linewidth}
$\bullet\ $ $11a_{26}$ \vspace{0.5em} \\
\begin{tabular}{l|llll}
$k \setminus j$ & $-2$ & $0$ & $2$ & $4$ \\
\hline
$8$ & $q^{-6}$ & $q^{-8}$ &  &  \\
$6$ & $3q^{-4}$ & $3q^{-6}$ &  &  \\
$4$ & $7q^{-2}$ & $9q^{-4}$ & $2q^{-6}$ &  \\
$2$ & $8$ & $13q^{-2}$ & $5q^{-4}$ &  \\
$0$ & $7q^{2}$ & $18$ & $11q^{-2}$ & $q^{-4}$ \\
$-2$ & $3q^{4}$ & $13q^{2}$ & $12$ & $2q^{-2}$ \\
$-4$ & $q^{6}$ & $9q^{4}$ & $11q^{2}$ & $3$ \\
$-6$ &  & $3q^{6}$ & $5q^{4}$ & $2q^{2}$ \\
$-8$ &  & $q^{8}$ & $2q^{6}$ & $q^{4}$ \\
\end{tabular}
\vspace{2em}
\end{minipage}
%
\begin{minipage}{\linewidth}
$\bullet\ $ $11a_{27}$ \vspace{0.5em} \\
\begin{tabular}{l|lllll}
$k \setminus j$ & $-8$ & $-6$ & $-4$ & $-2$ & $0$ \\
\hline
$8$ & $q^{-2}$ & $2q^{-4}$ & $q^{-6}$ &  &  \\
$6$ & $2$ & $7q^{-2}$ & $6q^{-4}$ & $q^{-6}$ &  \\
$4$ & $q^{2}$ & $9$ & $12q^{-2}$ & $4q^{-4}$ &  \\
$2$ &  & $7q^{2}$ & $17$ & $11q^{-2}$ & $q^{-4}$ \\
$0$ &  & $2q^{4}$ & $12q^{2}$ & $13$ & $3q^{-2}$ \\
$-2$ &  &  & $6q^{4}$ & $11q^{2}$ & $4$ \\
$-4$ &  &  & $q^{6}$ & $4q^{4}$ & $3q^{2}$ \\
$-6$ &  &  &  & $q^{6}$ & $q^{4}$ \\
\end{tabular}
\vspace{2em}
\end{minipage}
%
\begin{minipage}{\linewidth}
$\bullet\ $ $11a_{28}$ \vspace{0.5em} \\
\begin{tabular}{l|llll}
$k \setminus j$ & $-4$ & $-2$ & $0$ & $2$ \\
\hline
$8$ & $q^{-4}$ & $2q^{-6}$ & $q^{-8}$ &  \\
$6$ & $q^{-2}$ & $4q^{-4}$ & $3q^{-6}$ &  \\
$4$ & $2$ & $8q^{-2}$ & $7q^{-4}$ & $q^{-6}$ \\
$2$ & $q^{2}$ & $8$ & $10q^{-2}$ & $3q^{-4}$ \\
$0$ & $q^{4}$ & $8q^{2}$ & $13$ & $5q^{-2}$ \\
$-2$ &  & $4q^{4}$ & $10q^{2}$ & $6$ \\
$-4$ &  & $2q^{6}$ & $7q^{4}$ & $5q^{2}$ \\
$-6$ &  &  & $3q^{6}$ & $3q^{4}$ \\
$-8$ &  &  & $q^{8}$ & $q^{6}$ \\
\end{tabular}
\vspace{2em}
\end{minipage}
%
\begin{minipage}{\linewidth}
$\bullet\ $ $11a_{31}$ \vspace{0.5em} \\
\begin{tabular}{l|lllll}
$k \setminus j$ & $2$ & $4$ & $6$ & $8$ & $10$ \\
\hline
$6$ & $q^{-4}$ & $q^{-6}$ &  &  &  \\
$4$ & $2q^{-2}$ & $5q^{-4}$ & $2q^{-6}$ &  &  \\
$2$ & $3$ & $8q^{-2}$ & $5q^{-4}$ &  &  \\
$0$ & $2q^{2}$ & $11$ & $12q^{-2}$ & $3q^{-4}$ &  \\
$-2$ & $q^{4}$ & $8q^{2}$ & $12$ & $5q^{-2}$ &  \\
$-4$ &  & $5q^{4}$ & $12q^{2}$ & $8$ & $q^{-2}$ \\
$-6$ &  & $q^{6}$ & $5q^{4}$ & $5q^{2}$ & $1$ \\
$-8$ &  &  & $2q^{6}$ & $3q^{4}$ & $q^{2}$ \\
\end{tabular}
\vspace{2em}
\end{minipage}
%
\begin{minipage}{\linewidth}
$\bullet\ $ $11a_{32}$ \vspace{0.5em} \\
\begin{tabular}{l|lllll}
$k \setminus j$ & $0$ & $2$ & $4$ & $6$ & $8$ \\
\hline
$6$ & $q^{-4}$ & $q^{-6}$ &  &  &  \\
$4$ & $3q^{-2}$ & $5q^{-4}$ & $2q^{-6}$ &  &  \\
$2$ & $4$ & $10q^{-2}$ & $5q^{-4}$ &  &  \\
$0$ & $3q^{2}$ & $13$ & $13q^{-2}$ & $3q^{-4}$ &  \\
$-2$ & $q^{4}$ & $10q^{2}$ & $14$ & $5q^{-2}$ &  \\
$-4$ &  & $5q^{4}$ & $13q^{2}$ & $9$ & $q^{-2}$ \\
$-6$ &  & $q^{6}$ & $5q^{4}$ & $5q^{2}$ & $1$ \\
$-8$ &  &  & $2q^{6}$ & $3q^{4}$ & $q^{2}$ \\
\end{tabular}
\vspace{2em}
\end{minipage}
%
\begin{minipage}{\linewidth}
$\bullet\ $ $11a_{33}$ \vspace{0.5em} \\
\begin{tabular}{l|llll}
$k \setminus j$ & $-4$ & $-2$ & $0$ & $2$ \\
\hline
$8$ & $q^{-6}$ & $q^{-8}$ &  &  \\
$6$ & $2q^{-4}$ & $2q^{-6}$ &  &  \\
$4$ & $4q^{-2}$ & $6q^{-4}$ & $2q^{-6}$ &  \\
$2$ & $4$ & $7q^{-2}$ & $3q^{-4}$ &  \\
$0$ & $4q^{2}$ & $10$ & $7q^{-2}$ & $q^{-4}$ \\
$-2$ & $2q^{4}$ & $7q^{2}$ & $5$ & $q^{-2}$ \\
$-4$ & $q^{6}$ & $6q^{4}$ & $7q^{2}$ & $2$ \\
$-6$ &  & $2q^{6}$ & $3q^{4}$ & $q^{2}$ \\
$-8$ &  & $q^{8}$ & $2q^{6}$ & $q^{4}$ \\
\end{tabular}
\vspace{2em}
\end{minipage}
%
\begin{minipage}{\linewidth}
$\bullet\ $ $11a_{34}$ \vspace{0.5em} \\
\begin{tabular}{l|llll}
$k \setminus j$ & $0$ & $2$ & $4$ & $6$ \\
\hline
$8$ & $q^{-6}$ & $q^{-8}$ &  &  \\
$6$ & $2q^{-4}$ & $2q^{-6}$ &  &  \\
$4$ & $5q^{-2}$ & $7q^{-4}$ & $2q^{-6}$ &  \\
$2$ & $4$ & $9q^{-2}$ & $4q^{-4}$ &  \\
$0$ & $5q^{2}$ & $13$ & $9q^{-2}$ & $q^{-4}$ \\
$-2$ & $2q^{4}$ & $9q^{2}$ & $9$ & $2q^{-2}$ \\
$-4$ & $q^{6}$ & $7q^{4}$ & $9q^{2}$ & $3$ \\
$-6$ &  & $2q^{6}$ & $4q^{4}$ & $2q^{2}$ \\
$-8$ &  & $q^{8}$ & $2q^{6}$ & $q^{4}$ \\
\end{tabular}
\vspace{2em}
\end{minipage}
%
\begin{minipage}{\linewidth}
$\bullet\ $ $11a_{35}$ \vspace{0.5em} \\
\begin{tabular}{l|llll}
$k \setminus j$ & $-2$ & $0$ & $2$ & $4$ \\
\hline
$8$ & $q^{-6}$ & $q^{-8}$ &  &  \\
$6$ & $2q^{-4}$ & $2q^{-6}$ &  &  \\
$4$ & $5q^{-2}$ & $7q^{-4}$ & $2q^{-6}$ &  \\
$2$ & $5$ & $9q^{-2}$ & $4q^{-4}$ &  \\
$0$ & $5q^{2}$ & $14$ & $9q^{-2}$ & $q^{-4}$ \\
$-2$ & $2q^{4}$ & $9q^{2}$ & $9$ & $2q^{-2}$ \\
$-4$ & $q^{6}$ & $7q^{4}$ & $9q^{2}$ & $3$ \\
$-6$ &  & $2q^{6}$ & $4q^{4}$ & $2q^{2}$ \\
$-8$ &  & $q^{8}$ & $2q^{6}$ & $q^{4}$ \\
\end{tabular}
\vspace{2em}
\end{minipage}
%
\begin{minipage}{\linewidth}
$\bullet\ $ $11a_{37}$ \vspace{0.5em} \\
\begin{tabular}{l|lllll}
$k \setminus j$ & $-2$ & $0$ & $2$ & $4$ & $6$ \\
\hline
$6$ & $q^{-4}$ & $q^{-6}$ &  &  &  \\
$4$ & $2q^{-2}$ & $3q^{-4}$ & $q^{-6}$ &  &  \\
$2$ & $4$ & $7q^{-2}$ & $3q^{-4}$ &  &  \\
$0$ & $2q^{2}$ & $9$ & $8q^{-2}$ & $2q^{-4}$ &  \\
$-2$ & $q^{4}$ & $7q^{2}$ & $9$ & $3q^{-2}$ &  \\
$-4$ &  & $3q^{4}$ & $8q^{2}$ & $6$ & $q^{-2}$ \\
$-6$ &  & $q^{6}$ & $3q^{4}$ & $3q^{2}$ & $1$ \\
$-8$ &  &  & $q^{6}$ & $2q^{4}$ & $q^{2}$ \\
\end{tabular}
\vspace{2em}
\end{minipage}
%
\begin{minipage}{\linewidth}
$\bullet\ $ $11a_{39}$ \vspace{0.5em} \\
\begin{tabular}{l|lllll}
$k \setminus j$ & $-2$ & $0$ & $2$ & $4$ & $6$ \\
\hline
$6$ & $q^{-4}$ & $q^{-6}$ &  &  &  \\
$4$ & $2q^{-2}$ & $4q^{-4}$ & $2q^{-6}$ &  &  \\
$2$ & $3$ & $7q^{-2}$ & $4q^{-4}$ &  &  \\
$0$ & $2q^{2}$ & $9$ & $9q^{-2}$ & $3q^{-4}$ &  \\
$-2$ & $q^{4}$ & $7q^{2}$ & $9$ & $3q^{-2}$ &  \\
$-4$ &  & $4q^{4}$ & $9q^{2}$ & $6$ & $q^{-2}$ \\
$-6$ &  & $q^{6}$ & $4q^{4}$ & $3q^{2}$ &  \\
$-8$ &  &  & $2q^{6}$ & $3q^{4}$ & $q^{2}$ \\
\end{tabular}
\vspace{2em}
\end{minipage}
%
\begin{minipage}{\linewidth}
$\bullet\ $ $11a_{40}$ \vspace{0.5em} \\
\begin{tabular}{l|llll}
$k \setminus j$ & $2$ & $4$ & $6$ & $8$ \\
\hline
$8$ & $q^{-6}$ & $q^{-8}$ &  &  \\
$6$ & $2q^{-4}$ & $2q^{-6}$ &  &  \\
$4$ & $3q^{-2}$ & $6q^{-4}$ & $2q^{-6}$ &  \\
$2$ & $3$ & $6q^{-2}$ & $3q^{-4}$ &  \\
$0$ & $3q^{2}$ & $9$ & $7q^{-2}$ & $q^{-4}$ \\
$-2$ & $2q^{4}$ & $6q^{2}$ & $5$ & $q^{-2}$ \\
$-4$ & $q^{6}$ & $6q^{4}$ & $7q^{2}$ & $2$ \\
$-6$ &  & $2q^{6}$ & $3q^{4}$ & $q^{2}$ \\
$-8$ &  & $q^{8}$ & $2q^{6}$ & $q^{4}$ \\
\end{tabular}
\vspace{2em}
\end{minipage}
%
\begin{minipage}{\linewidth}
$\bullet\ $ $11a_{41}$ \vspace{0.5em} \\
\begin{tabular}{l|lllll}
$k \setminus j$ & $0$ & $2$ & $4$ & $6$ & $8$ \\
\hline
$6$ & $q^{-4}$ & $q^{-6}$ &  &  &  \\
$4$ & $3q^{-2}$ & $4q^{-4}$ & $q^{-6}$ &  &  \\
$2$ & $4$ & $9q^{-2}$ & $4q^{-4}$ &  &  \\
$0$ & $3q^{2}$ & $11$ & $10q^{-2}$ & $2q^{-4}$ &  \\
$-2$ & $q^{4}$ & $9q^{2}$ & $12$ & $4q^{-2}$ &  \\
$-4$ &  & $4q^{4}$ & $10q^{2}$ & $7$ & $q^{-2}$ \\
$-6$ &  & $q^{6}$ & $4q^{4}$ & $4q^{2}$ & $1$ \\
$-8$ &  &  & $q^{6}$ & $2q^{4}$ & $q^{2}$ \\
\end{tabular}
\vspace{2em}
\end{minipage}
%
\begin{minipage}{\linewidth}
$\bullet\ $ $11a_{42}$ \vspace{0.5em} \\
\begin{tabular}{l|llllll}
$k \setminus j$ & $-2$ & $0$ & $2$ & $4$ & $6$ & $8$ \\
\hline
$6$ & $q^{-2}$ & $2q^{-4}$ & $q^{-6}$ &  &  &  \\
$4$ & $1$ & $5q^{-2}$ & $4q^{-4}$ &  &  &  \\
$2$ & $q^{2}$ & $7$ & $10q^{-2}$ & $3q^{-4}$ &  &  \\
$0$ &  & $5q^{2}$ & $12$ & $7q^{-2}$ &  &  \\
$-2$ &  & $2q^{4}$ & $10q^{2}$ & $11$ & $3q^{-2}$ &  \\
$-4$ &  &  & $4q^{4}$ & $7q^{2}$ & $3$ &  \\
$-6$ &  &  & $q^{6}$ & $3q^{4}$ & $3q^{2}$ & $1$ \\
\end{tabular}
\vspace{2em}
\end{minipage}
%
\begin{minipage}{\linewidth}
$\bullet\ $ $11a_{44}$ \vspace{0.5em} \\
\begin{tabular}{l|llll}
$k \setminus j$ & $-2$ & $0$ & $2$ & $4$ \\
\hline
$8$ & $q^{-6}$ & $q^{-8}$ &  &  \\
$6$ & $2q^{-4}$ & $2q^{-6}$ &  &  \\
$4$ & $6q^{-2}$ & $8q^{-4}$ & $2q^{-6}$ &  \\
$2$ & $5$ & $8q^{-2}$ & $3q^{-4}$ &  \\
$0$ & $6q^{2}$ & $15$ & $9q^{-2}$ & $q^{-4}$ \\
$-2$ & $2q^{4}$ & $8q^{2}$ & $7$ & $q^{-2}$ \\
$-4$ & $q^{6}$ & $8q^{4}$ & $9q^{2}$ & $2$ \\
$-6$ &  & $2q^{6}$ & $3q^{4}$ & $q^{2}$ \\
$-8$ &  & $q^{8}$ & $2q^{6}$ & $q^{4}$ \\
\end{tabular}
\vspace{2em}
\end{minipage}
%
\begin{minipage}{\linewidth}
$\bullet\ $ $11a_{45}$ \vspace{0.5em} \\
\begin{tabular}{l|llllll}
$k \setminus j$ & $-2$ & $0$ & $2$ & $4$ & $6$ & $8$ \\
\hline
$6$ & $q^{-2}$ & $q^{-4}$ &  &  &  &  \\
$4$ & $1$ & $4q^{-2}$ & $3q^{-4}$ &  &  &  \\
$2$ & $q^{2}$ & $5$ & $7q^{-2}$ & $2q^{-4}$ &  &  \\
$0$ &  & $4q^{2}$ & $11$ & $7q^{-2}$ &  &  \\
$-2$ &  & $q^{4}$ & $7q^{2}$ & $9$ & $3q^{-2}$ &  \\
$-4$ &  &  & $3q^{4}$ & $7q^{2}$ & $4$ &  \\
$-6$ &  &  &  & $2q^{4}$ & $3q^{2}$ & $1$ \\
\end{tabular}
\vspace{2em}
\end{minipage}
%
\begin{minipage}{\linewidth}
$\bullet\ $ $11a_{46}$ \vspace{0.5em} \\
\begin{tabular}{l|lllll}
$k \setminus j$ & $-2$ & $0$ & $2$ & $4$ & $6$ \\
\hline
$8$ & $q^{-4}$ & $q^{-6}$ &  &  &  \\
$6$ & $2q^{-2}$ & $3q^{-4}$ & $q^{-6}$ &  &  \\
$4$ & $3$ & $6q^{-2}$ & $3q^{-4}$ &  &  \\
$2$ & $2q^{2}$ & $7$ & $8q^{-2}$ & $2q^{-4}$ &  \\
$0$ & $q^{4}$ & $6q^{2}$ & $8$ & $3q^{-2}$ &  \\
$-2$ &  & $3q^{4}$ & $8q^{2}$ & $6$ & $q^{-2}$ \\
$-4$ &  & $q^{6}$ & $3q^{4}$ & $3q^{2}$ & $1$ \\
$-6$ &  &  & $q^{6}$ & $2q^{4}$ & $q^{2}$ \\
\end{tabular}
\vspace{2em}
\end{minipage}
%
\begin{minipage}{\linewidth}
$\bullet\ $ $11a_{47}$ \vspace{0.5em} \\
\begin{tabular}{l|llll}
$k \setminus j$ & $-2$ & $0$ & $2$ & $4$ \\
\hline
$8$ & $q^{-6}$ & $q^{-8}$ &  &  \\
$6$ & $2q^{-4}$ & $2q^{-6}$ &  &  \\
$4$ & $6q^{-2}$ & $8q^{-4}$ & $2q^{-6}$ &  \\
$2$ & $5$ & $8q^{-2}$ & $3q^{-4}$ &  \\
$0$ & $6q^{2}$ & $15$ & $9q^{-2}$ & $q^{-4}$ \\
$-2$ & $2q^{4}$ & $8q^{2}$ & $7$ & $q^{-2}$ \\
$-4$ & $q^{6}$ & $8q^{4}$ & $9q^{2}$ & $2$ \\
$-6$ &  & $2q^{6}$ & $3q^{4}$ & $q^{2}$ \\
$-8$ &  & $q^{8}$ & $2q^{6}$ & $q^{4}$ \\
\end{tabular}
\vspace{2em}
\end{minipage}
%
\begin{minipage}{\linewidth}
$\bullet\ $ $11a_{48}$ \vspace{0.5em} \\
\begin{tabular}{l|lllll}
$k \setminus j$ & $-10$ & $-8$ & $-6$ & $-4$ & $-2$ \\
\hline
$8$ & $q^{-2}$ & $2q^{-4}$ & $q^{-6}$ &  &  \\
$6$ & $1$ & $5q^{-2}$ & $5q^{-4}$ & $q^{-6}$ &  \\
$4$ & $q^{2}$ & $7$ & $10q^{-2}$ & $4q^{-4}$ &  \\
$2$ &  & $5q^{2}$ & $12$ & $8q^{-2}$ & $q^{-4}$ \\
$0$ &  & $2q^{4}$ & $10q^{2}$ & $10$ & $2q^{-2}$ \\
$-2$ &  &  & $5q^{4}$ & $8q^{2}$ & $3$ \\
$-4$ &  &  & $q^{6}$ & $4q^{4}$ & $2q^{2}$ \\
$-6$ &  &  &  & $q^{6}$ & $q^{4}$ \\
\end{tabular}
\vspace{2em}
\end{minipage}
%
\begin{minipage}{\linewidth}
$\bullet\ $ $11a_{49}$ \vspace{0.5em} \\
\begin{tabular}{l|lllll}
$k \setminus j$ & $2$ & $4$ & $6$ & $8$ & $10$ \\
\hline
$6$ & $q^{-4}$ & $q^{-6}$ &  &  &  \\
$4$ & $q^{-2}$ & $4q^{-4}$ & $2q^{-6}$ &  &  \\
$2$ & $2$ & $7q^{-2}$ & $5q^{-4}$ &  &  \\
$0$ & $q^{2}$ & $8$ & $10q^{-2}$ & $3q^{-4}$ &  \\
$-2$ & $q^{4}$ & $7q^{2}$ & $10$ & $4q^{-2}$ &  \\
$-4$ &  & $4q^{4}$ & $10q^{2}$ & $7$ & $q^{-2}$ \\
$-6$ &  & $q^{6}$ & $5q^{4}$ & $4q^{2}$ &  \\
$-8$ &  &  & $2q^{6}$ & $3q^{4}$ & $q^{2}$ \\
\end{tabular}
\vspace{2em}
\end{minipage}
%
\begin{minipage}{\linewidth}
$\bullet\ $ $11a_{50}$ \vspace{0.5em} \\
\begin{tabular}{l|llllll}
$k \setminus j$ & $0$ & $2$ & $4$ & $6$ & $8$ & $10$ \\
\hline
$4$ & $q^{-2}$ & $q^{-4}$ &  &  &  &  \\
$2$ & $1$ & $4q^{-2}$ & $2q^{-4}$ &  &  &  \\
$0$ & $q^{2}$ & $6$ & $7q^{-2}$ & $2q^{-4}$ &  &  \\
$-2$ &  & $4q^{2}$ & $10$ & $6q^{-2}$ &  &  \\
$-4$ &  & $q^{4}$ & $7q^{2}$ & $9$ & $3q^{-2}$ &  \\
$-6$ &  &  & $2q^{4}$ & $6q^{2}$ & $4$ &  \\
$-8$ &  &  &  & $2q^{4}$ & $3q^{2}$ & $1$ \\
\end{tabular}
\vspace{2em}
\end{minipage}
%
\begin{minipage}{\linewidth}
$\bullet\ $ $11a_{51}$ \vspace{0.5em} \\
\begin{tabular}{l|llllll}
$k \setminus j$ & $-8$ & $-6$ & $-4$ & $-2$ & $0$ & $2$ \\
\hline
$6$ & $1$ & $3q^{-2}$ & $3q^{-4}$ & $q^{-6}$ &  &  \\
$4$ &  & $4$ & $8q^{-2}$ & $4q^{-4}$ &  &  \\
$2$ &  & $3q^{2}$ & $12$ & $11q^{-2}$ & $2q^{-4}$ &  \\
$0$ &  &  & $8q^{2}$ & $13$ & $5q^{-2}$ &  \\
$-2$ &  &  & $3q^{4}$ & $11q^{2}$ & $8$ & $q^{-2}$ \\
$-4$ &  &  &  & $4q^{4}$ & $5q^{2}$ & $1$ \\
$-6$ &  &  &  & $q^{6}$ & $2q^{4}$ & $q^{2}$ \\
\end{tabular}
\vspace{2em}
\end{minipage}
%
\begin{minipage}{\linewidth}
$\bullet\ $ $11a_{53}$ \vspace{0.5em} \\
\begin{tabular}{l|llll}
$k \setminus j$ & $-8$ & $-6$ & $-4$ & $-2$ \\
\hline
$8$ & $q^{-4}$ & $2q^{-6}$ & $q^{-8}$ &  \\
$6$ & $q^{-2}$ & $4q^{-4}$ & $3q^{-6}$ &  \\
$4$ & $1$ & $6q^{-2}$ & $6q^{-4}$ & $q^{-6}$ \\
$2$ & $q^{2}$ & $6$ & $8q^{-2}$ & $3q^{-4}$ \\
$0$ & $q^{4}$ & $6q^{2}$ & $8$ & $3q^{-2}$ \\
$-2$ &  & $4q^{4}$ & $8q^{2}$ & $4$ \\
$-4$ &  & $2q^{6}$ & $6q^{4}$ & $3q^{2}$ \\
$-6$ &  &  & $3q^{6}$ & $3q^{4}$ \\
$-8$ &  &  & $q^{8}$ & $q^{6}$ \\
\end{tabular}
\vspace{2em}
\end{minipage}
%
\begin{minipage}{\linewidth}
$\bullet\ $ $11a_{54}$ \vspace{0.5em} \\
\begin{tabular}{l|lllll}
$k \setminus j$ & $-8$ & $-6$ & $-4$ & $-2$ & $0$ \\
\hline
$8$ & $q^{-2}$ & $2q^{-4}$ & $q^{-6}$ &  &  \\
$6$ & $1$ & $5q^{-2}$ & $5q^{-4}$ & $q^{-6}$ &  \\
$4$ & $q^{2}$ & $8$ & $12q^{-2}$ & $5q^{-4}$ &  \\
$2$ &  & $5q^{2}$ & $15$ & $11q^{-2}$ & $q^{-4}$ \\
$0$ &  & $2q^{4}$ & $12q^{2}$ & $14$ & $4q^{-2}$ \\
$-2$ &  &  & $5q^{4}$ & $11q^{2}$ & $5$ \\
$-4$ &  &  & $q^{6}$ & $5q^{4}$ & $4q^{2}$ \\
$-6$ &  &  &  & $q^{6}$ & $q^{4}$ \\
\end{tabular}
\vspace{2em}
\end{minipage}
%
\begin{minipage}{\linewidth}
$\bullet\ $ $11a_{55}$ \vspace{0.5em} \\
\begin{tabular}{l|llll}
$k \setminus j$ & $-4$ & $-2$ & $0$ & $2$ \\
\hline
$8$ & $q^{-6}$ & $q^{-8}$ &  &  \\
$6$ & $2q^{-4}$ & $2q^{-6}$ &  &  \\
$4$ & $3q^{-2}$ & $5q^{-4}$ & $2q^{-6}$ &  \\
$2$ & $3$ & $5q^{-2}$ & $2q^{-4}$ &  \\
$0$ & $3q^{2}$ & $7$ & $5q^{-2}$ & $q^{-4}$ \\
$-2$ & $2q^{4}$ & $5q^{2}$ & $2$ &  \\
$-4$ & $q^{6}$ & $5q^{4}$ & $5q^{2}$ & $1$ \\
$-6$ &  & $2q^{6}$ & $2q^{4}$ &  \\
$-8$ &  & $q^{8}$ & $2q^{6}$ & $q^{4}$ \\
\end{tabular}
\vspace{2em}
\end{minipage}
%
\begin{minipage}{\linewidth}
$\bullet\ $ $11a_{57}$ \vspace{0.5em} \\
\begin{tabular}{l|llll}
$k \setminus j$ & $-4$ & $-2$ & $0$ & $2$ \\
\hline
$8$ & $q^{-6}$ & $q^{-8}$ &  &  \\
$6$ & $2q^{-4}$ & $2q^{-6}$ &  &  \\
$4$ & $5q^{-2}$ & $7q^{-4}$ & $2q^{-6}$ &  \\
$2$ & $5$ & $7q^{-2}$ & $2q^{-4}$ &  \\
$0$ & $5q^{2}$ & $12$ & $8q^{-2}$ & $q^{-4}$ \\
$-2$ & $2q^{4}$ & $7q^{2}$ & $4$ &  \\
$-4$ & $q^{6}$ & $7q^{4}$ & $8q^{2}$ & $2$ \\
$-6$ &  & $2q^{6}$ & $2q^{4}$ &  \\
$-8$ &  & $q^{8}$ & $2q^{6}$ & $q^{4}$ \\
\end{tabular}
\vspace{2em}
\end{minipage}
%
\begin{minipage}{\linewidth}
$\bullet\ $ $11a_{58}$ \vspace{0.5em} \\
\begin{tabular}{l|lllll}
$k \setminus j$ & $-4$ & $-2$ & $0$ & $2$ & $4$ \\
\hline
$8$ & $q^{-4}$ & $q^{-6}$ &  &  &  \\
$6$ & $2q^{-2}$ & $3q^{-4}$ & $q^{-6}$ &  &  \\
$4$ & $3$ & $6q^{-2}$ & $3q^{-4}$ &  &  \\
$2$ & $2q^{2}$ & $7$ & $7q^{-2}$ & $2q^{-4}$ &  \\
$0$ & $q^{4}$ & $6q^{2}$ & $8$ & $2q^{-2}$ &  \\
$-2$ &  & $3q^{4}$ & $7q^{2}$ & $5$ & $q^{-2}$ \\
$-4$ &  & $q^{6}$ & $3q^{4}$ & $2q^{2}$ &  \\
$-6$ &  &  & $q^{6}$ & $2q^{4}$ & $q^{2}$ \\
\end{tabular}
\vspace{2em}
\end{minipage}
%
\begin{minipage}{\linewidth}
$\bullet\ $ $11a_{59}$ \vspace{0.5em} \\
\begin{tabular}{l|llllll}
$k \setminus j$ & $-4$ & $-2$ & $0$ & $2$ & $4$ & $6$ \\
\hline
$8$ & $q^{-2}$ & $q^{-4}$ &  &  &  &  \\
$6$ & $1$ & $2q^{-2}$ & $q^{-4}$ &  &  &  \\
$4$ & $q^{2}$ & $3$ & $3q^{-2}$ & $q^{-4}$ &  &  \\
$2$ &  & $2q^{2}$ & $4$ & $3q^{-2}$ &  &  \\
$0$ &  & $q^{4}$ & $3q^{2}$ & $4$ & $2q^{-2}$ &  \\
$-2$ &  &  & $q^{4}$ & $3q^{2}$ & $2$ &  \\
$-4$ &  &  &  & $q^{4}$ & $2q^{2}$ & $1$ \\
\end{tabular}
\vspace{2em}
\end{minipage}
%
\begin{minipage}{\linewidth}
$\bullet\ $ $11a_{60}$ \vspace{0.5em} \\
\begin{tabular}{l|lllll}
$k \setminus j$ & $2$ & $4$ & $6$ & $8$ & $10$ \\
\hline
$6$ & $q^{-4}$ & $q^{-6}$ &  &  &  \\
$4$ & $q^{-2}$ & $4q^{-4}$ & $2q^{-6}$ &  &  \\
$2$ & $1$ & $5q^{-2}$ & $4q^{-4}$ &  &  \\
$0$ & $q^{2}$ & $6$ & $8q^{-2}$ & $3q^{-4}$ &  \\
$-2$ & $q^{4}$ & $5q^{2}$ & $7$ & $3q^{-2}$ &  \\
$-4$ &  & $4q^{4}$ & $8q^{2}$ & $5$ & $q^{-2}$ \\
$-6$ &  & $q^{6}$ & $4q^{4}$ & $3q^{2}$ &  \\
$-8$ &  &  & $2q^{6}$ & $3q^{4}$ & $q^{2}$ \\
\end{tabular}
\vspace{2em}
\end{minipage}
%
\begin{minipage}{\linewidth}
$\bullet\ $ $11a_{61}$ \vspace{0.5em} \\
\begin{tabular}{l|llllll}
$k \setminus j$ & $0$ & $2$ & $4$ & $6$ & $8$ & $10$ \\
\hline
$4$ & $q^{-2}$ & $q^{-4}$ &  &  &  &  \\
$2$ & $2$ & $6q^{-2}$ & $3q^{-4}$ &  &  &  \\
$0$ & $q^{2}$ & $8$ & $9q^{-2}$ & $2q^{-4}$ &  &  \\
$-2$ &  & $6q^{2}$ & $13$ & $7q^{-2}$ &  &  \\
$-4$ &  & $q^{4}$ & $9q^{2}$ & $11$ & $3q^{-2}$ &  \\
$-6$ &  &  & $3q^{4}$ & $7q^{2}$ & $4$ &  \\
$-8$ &  &  &  & $2q^{4}$ & $3q^{2}$ & $1$ \\
\end{tabular}
\vspace{2em}
\end{minipage}
%
\begin{minipage}{\linewidth}
$\bullet\ $ $11a_{62}$ \vspace{0.5em} \\
\begin{tabular}{l|llll}
$k \setminus j$ & $-10$ & $-8$ & $-6$ & $-4$ \\
\hline
$8$ & $q^{-4}$ & $2q^{-6}$ & $q^{-8}$ &  \\
$6$ &  & $2q^{-4}$ & $2q^{-6}$ &  \\
$4$ & $1$ & $4q^{-2}$ & $4q^{-4}$ & $q^{-6}$ \\
$2$ &  & $2$ & $3q^{-2}$ & $q^{-4}$ \\
$0$ & $q^{4}$ & $4q^{2}$ & $5$ & $2q^{-2}$ \\
$-2$ &  & $2q^{4}$ & $3q^{2}$ & $1$ \\
$-4$ &  & $2q^{6}$ & $4q^{4}$ & $2q^{2}$ \\
$-6$ &  &  & $2q^{6}$ & $q^{4}$ \\
$-8$ &  &  & $q^{8}$ & $q^{6}$ \\
\end{tabular}
\vspace{2em}
\end{minipage}
%
\begin{minipage}{\linewidth}
$\bullet\ $ $11a_{63}$ \vspace{0.5em} \\
\begin{tabular}{l|lllll}
$k \setminus j$ & $-10$ & $-8$ & $-6$ & $-4$ & $-2$ \\
\hline
$8$ & $q^{-2}$ & $2q^{-4}$ & $q^{-6}$ &  &  \\
$6$ &  & $3q^{-2}$ & $4q^{-4}$ & $q^{-6}$ &  \\
$4$ & $q^{2}$ & $5$ & $8q^{-2}$ & $4q^{-4}$ &  \\
$2$ &  & $3q^{2}$ & $9$ & $7q^{-2}$ & $q^{-4}$ \\
$0$ &  & $2q^{4}$ & $8q^{2}$ & $8$ & $2q^{-2}$ \\
$-2$ &  &  & $4q^{4}$ & $7q^{2}$ & $3$ \\
$-4$ &  &  & $q^{6}$ & $4q^{4}$ & $2q^{2}$ \\
$-6$ &  &  &  & $q^{6}$ & $q^{4}$ \\
\end{tabular}
\vspace{2em}
\end{minipage}
%
\begin{minipage}{\linewidth}
$\bullet\ $ $11a_{64}$ \vspace{0.5em} \\
\begin{tabular}{l|lllll}
$k \setminus j$ & $-10$ & $-8$ & $-6$ & $-4$ & $-2$ \\
\hline
$8$ & $q^{-2}$ & $2q^{-4}$ & $q^{-6}$ &  &  \\
$6$ & $1$ & $5q^{-2}$ & $5q^{-4}$ & $q^{-6}$ &  \\
$4$ & $q^{2}$ & $6$ & $8q^{-2}$ & $3q^{-4}$ &  \\
$2$ &  & $5q^{2}$ & $11$ & $7q^{-2}$ & $q^{-4}$ \\
$0$ &  & $2q^{4}$ & $8q^{2}$ & $7$ & $q^{-2}$ \\
$-2$ &  &  & $5q^{4}$ & $7q^{2}$ & $2$ \\
$-4$ &  &  & $q^{6}$ & $3q^{4}$ & $q^{2}$ \\
$-6$ &  &  &  & $q^{6}$ & $q^{4}$ \\
\end{tabular}
\vspace{2em}
\end{minipage}
%
\begin{minipage}{\linewidth}
$\bullet\ $ $11a_{65}$ \vspace{0.5em} \\
\begin{tabular}{l|llllll}
$k \setminus j$ & $-10$ & $-8$ & $-6$ & $-4$ & $-2$ & $0$ \\
\hline
$8$ & $1$ & $2q^{-2}$ & $q^{-4}$ &  &  &  \\
$6$ &  & $3$ & $4q^{-2}$ & $q^{-4}$ &  &  \\
$4$ &  & $2q^{2}$ & $6$ & $5q^{-2}$ & $q^{-4}$ &  \\
$2$ &  &  & $4q^{2}$ & $7$ & $3q^{-2}$ &  \\
$0$ &  &  & $q^{4}$ & $5q^{2}$ & $5$ & $q^{-2}$ \\
$-2$ &  &  &  & $q^{4}$ & $3q^{2}$ & $1$ \\
$-4$ &  &  &  &  & $q^{4}$ & $q^{2}$ \\
\end{tabular}
\vspace{2em}
\end{minipage}
%
\begin{minipage}{\linewidth}
$\bullet\ $ $11a_{66}$ \vspace{0.5em} \\
\begin{tabular}{l|llll}
$k \setminus j$ & $-6$ & $-4$ & $-2$ & $0$ \\
\hline
$8$ & $q^{-4}$ & $2q^{-6}$ & $q^{-8}$ &  \\
$6$ & $q^{-2}$ & $4q^{-4}$ & $3q^{-6}$ &  \\
$4$ & $2$ & $8q^{-2}$ & $7q^{-4}$ & $q^{-6}$ \\
$2$ & $q^{2}$ & $8$ & $10q^{-2}$ & $3q^{-4}$ \\
$0$ & $q^{4}$ & $8q^{2}$ & $12$ & $5q^{-2}$ \\
$-2$ &  & $4q^{4}$ & $10q^{2}$ & $5$ \\
$-4$ &  & $2q^{6}$ & $7q^{4}$ & $5q^{2}$ \\
$-6$ &  &  & $3q^{6}$ & $3q^{4}$ \\
$-8$ &  &  & $q^{8}$ & $q^{6}$ \\
\end{tabular}
\vspace{2em}
\end{minipage}
%
\begin{minipage}{\linewidth}
$\bullet\ $ $11a_{67}$ \vspace{0.5em} \\
\begin{tabular}{l|lllll}
$k \setminus j$ & $-6$ & $-4$ & $-2$ & $0$ & $2$ \\
\hline
$8$ & $q^{-2}$ & $2q^{-4}$ & $q^{-6}$ &  &  \\
$6$ & $1$ & $5q^{-2}$ & $5q^{-4}$ & $q^{-6}$ &  \\
$4$ & $q^{2}$ & $7$ & $10q^{-2}$ & $4q^{-4}$ &  \\
$2$ &  & $5q^{2}$ & $14$ & $10q^{-2}$ & $q^{-4}$ \\
$0$ &  & $2q^{4}$ & $10q^{2}$ & $12$ & $3q^{-2}$ \\
$-2$ &  &  & $5q^{4}$ & $10q^{2}$ & $5$ \\
$-4$ &  &  & $q^{6}$ & $4q^{4}$ & $3q^{2}$ \\
$-6$ &  &  &  & $q^{6}$ & $q^{4}$ \\
\end{tabular}
\vspace{2em}
\end{minipage}
%
\begin{minipage}{\linewidth}
$\bullet\ $ $11a_{68}$ \vspace{0.5em} \\
\begin{tabular}{l|llll}
$k \setminus j$ & $-2$ & $0$ & $2$ & $4$ \\
\hline
$8$ & $q^{-4}$ & $2q^{-6}$ & $q^{-8}$ &  \\
$6$ & $q^{-2}$ & $4q^{-4}$ & $3q^{-6}$ &  \\
$4$ & $1$ & $6q^{-2}$ & $6q^{-4}$ & $q^{-6}$ \\
$2$ & $q^{2}$ & $6$ & $9q^{-2}$ & $3q^{-4}$ \\
$0$ & $q^{4}$ & $6q^{2}$ & $9$ & $4q^{-2}$ \\
$-2$ &  & $4q^{4}$ & $9q^{2}$ & $5$ \\
$-4$ &  & $2q^{6}$ & $6q^{4}$ & $4q^{2}$ \\
$-6$ &  &  & $3q^{6}$ & $3q^{4}$ \\
$-8$ &  &  & $q^{8}$ & $q^{6}$ \\
\end{tabular}
\vspace{2em}
\end{minipage}
%
\begin{minipage}{\linewidth}
$\bullet\ $ $11a_{71}$ \vspace{0.5em} \\
\begin{tabular}{l|llll}
$k \setminus j$ & $0$ & $2$ & $4$ & $6$ \\
\hline
$8$ & $q^{-6}$ & $q^{-8}$ &  &  \\
$6$ & $3q^{-4}$ & $3q^{-6}$ &  &  \\
$4$ & $6q^{-2}$ & $8q^{-4}$ & $2q^{-6}$ &  \\
$2$ & $7$ & $14q^{-2}$ & $6q^{-4}$ &  \\
$0$ & $6q^{2}$ & $16$ & $11q^{-2}$ & $q^{-4}$ \\
$-2$ & $3q^{4}$ & $14q^{2}$ & $14$ & $3q^{-2}$ \\
$-4$ & $q^{6}$ & $8q^{4}$ & $11q^{2}$ & $4$ \\
$-6$ &  & $3q^{6}$ & $6q^{4}$ & $3q^{2}$ \\
$-8$ &  & $q^{8}$ & $2q^{6}$ & $q^{4}$ \\
\end{tabular}
\vspace{2em}
\end{minipage}
%
\begin{minipage}{\linewidth}
$\bullet\ $ $11a_{72}$ \vspace{0.5em} \\
\begin{tabular}{l|llll}
$k \setminus j$ & $-2$ & $0$ & $2$ & $4$ \\
\hline
$8$ & $q^{-6}$ & $q^{-8}$ &  &  \\
$6$ & $3q^{-4}$ & $3q^{-6}$ &  &  \\
$4$ & $6q^{-2}$ & $8q^{-4}$ & $2q^{-6}$ &  \\
$2$ & $7$ & $13q^{-2}$ & $6q^{-4}$ &  \\
$0$ & $6q^{2}$ & $16$ & $10q^{-2}$ & $q^{-4}$ \\
$-2$ & $3q^{4}$ & $13q^{2}$ & $13$ & $3q^{-2}$ \\
$-4$ & $q^{6}$ & $8q^{4}$ & $10q^{2}$ & $3$ \\
$-6$ &  & $3q^{6}$ & $6q^{4}$ & $3q^{2}$ \\
$-8$ &  & $q^{8}$ & $2q^{6}$ & $q^{4}$ \\
\end{tabular}
\vspace{2em}
\end{minipage}
%
\begin{minipage}{\linewidth}
$\bullet\ $ $11a_{73}$ \vspace{0.5em} \\
\begin{tabular}{l|llll}
$k \setminus j$ & $-4$ & $-2$ & $0$ & $2$ \\
\hline
$8$ & $q^{-4}$ & $2q^{-6}$ & $q^{-8}$ &  \\
$6$ & $3q^{-2}$ & $7q^{-4}$ & $4q^{-6}$ &  \\
$4$ & $3$ & $11q^{-2}$ & $9q^{-4}$ & $q^{-6}$ \\
$2$ & $3q^{2}$ & $15$ & $16q^{-2}$ & $4q^{-4}$ \\
$0$ & $q^{4}$ & $11q^{2}$ & $18$ & $7q^{-2}$ \\
$-2$ &  & $7q^{4}$ & $16q^{2}$ & $9$ \\
$-4$ &  & $2q^{6}$ & $9q^{4}$ & $7q^{2}$ \\
$-6$ &  &  & $4q^{6}$ & $4q^{4}$ \\
$-8$ &  &  & $q^{8}$ & $q^{6}$ \\
\end{tabular}
\vspace{2em}
\end{minipage}
%
\begin{minipage}{\linewidth}
$\bullet\ $ $11a_{74}$ \vspace{0.5em} \\
\begin{tabular}{l|llll}
$k \setminus j$ & $0$ & $2$ & $4$ & $6$ \\
\hline
$8$ & $q^{-4}$ & $2q^{-6}$ & $q^{-8}$ &  \\
$6$ &  & $2q^{-4}$ & $2q^{-6}$ &  \\
$4$ & $2$ & $5q^{-2}$ & $5q^{-4}$ & $q^{-6}$ \\
$2$ &  & $3$ & $5q^{-2}$ & $2q^{-4}$ \\
$0$ & $q^{4}$ & $5q^{2}$ & $7$ & $3q^{-2}$ \\
$-2$ &  & $2q^{4}$ & $5q^{2}$ & $3$ \\
$-4$ &  & $2q^{6}$ & $5q^{4}$ & $3q^{2}$ \\
$-6$ &  &  & $2q^{6}$ & $2q^{4}$ \\
$-8$ &  &  & $q^{8}$ & $q^{6}$ \\
\end{tabular}
\vspace{2em}
\end{minipage}
%
\begin{minipage}{\linewidth}
$\bullet\ $ $11a_{75}$ \vspace{0.5em} \\
\begin{tabular}{l|lllll}
$k \setminus j$ & $-2$ & $0$ & $2$ & $4$ & $6$ \\
\hline
$6$ & $q^{-2}$ & $2q^{-4}$ & $q^{-6}$ &  &  \\
$4$ &  & $3q^{-2}$ & $4q^{-4}$ & $q^{-6}$ &  \\
$2$ & $q^{2}$ & $4$ & $7q^{-2}$ & $3q^{-4}$ &  \\
$0$ &  & $3q^{2}$ & $8$ & $6q^{-2}$ & $q^{-4}$ \\
$-2$ &  & $2q^{4}$ & $7q^{2}$ & $7$ & $2q^{-2}$ \\
$-4$ &  &  & $4q^{4}$ & $6q^{2}$ & $2$ \\
$-6$ &  &  & $q^{6}$ & $3q^{4}$ & $2q^{2}$ \\
$-8$ &  &  &  & $q^{6}$ & $q^{4}$ \\
\end{tabular}
\vspace{2em}
\end{minipage}
%
\begin{minipage}{\linewidth}
$\bullet\ $ $11a_{76}$ \vspace{0.5em} \\
\begin{tabular}{l|llll}
$k \setminus j$ & $-4$ & $-2$ & $0$ & $2$ \\
\hline
$8$ & $q^{-4}$ & $2q^{-6}$ & $q^{-8}$ &  \\
$6$ & $2q^{-2}$ & $5q^{-4}$ & $3q^{-6}$ &  \\
$4$ & $3$ & $10q^{-2}$ & $8q^{-4}$ & $q^{-6}$ \\
$2$ & $2q^{2}$ & $11$ & $12q^{-2}$ & $3q^{-4}$ \\
$0$ & $q^{4}$ & $10q^{2}$ & $16$ & $6q^{-2}$ \\
$-2$ &  & $5q^{4}$ & $12q^{2}$ & $7$ \\
$-4$ &  & $2q^{6}$ & $8q^{4}$ & $6q^{2}$ \\
$-6$ &  &  & $3q^{6}$ & $3q^{4}$ \\
$-8$ &  &  & $q^{8}$ & $q^{6}$ \\
\end{tabular}
\vspace{2em}
\end{minipage}
%
\begin{minipage}{\linewidth}
$\bullet\ $ $11a_{77}$ \vspace{0.5em} \\
\begin{tabular}{l|lllll}
$k \setminus j$ & $-8$ & $-6$ & $-4$ & $-2$ & $0$ \\
\hline
$8$ & $q^{-2}$ & $2q^{-4}$ & $q^{-6}$ &  &  \\
$6$ & $2$ & $6q^{-2}$ & $5q^{-4}$ & $q^{-6}$ &  \\
$4$ & $q^{2}$ & $8$ & $11q^{-2}$ & $4q^{-4}$ &  \\
$2$ &  & $6q^{2}$ & $15$ & $10q^{-2}$ & $q^{-4}$ \\
$0$ &  & $2q^{4}$ & $11q^{2}$ & $12$ & $3q^{-2}$ \\
$-2$ &  &  & $5q^{4}$ & $10q^{2}$ & $4$ \\
$-4$ &  &  & $q^{6}$ & $4q^{4}$ & $3q^{2}$ \\
$-6$ &  &  &  & $q^{6}$ & $q^{4}$ \\
\end{tabular}
\vspace{2em}
\end{minipage}
%
\begin{minipage}{\linewidth}
$\bullet\ $ $11a_{79}$ \vspace{0.5em} \\
\begin{tabular}{l|llll}
$k \setminus j$ & $-6$ & $-4$ & $-2$ & $0$ \\
\hline
$8$ & $q^{-4}$ & $2q^{-6}$ & $q^{-8}$ &  \\
$6$ & $2q^{-2}$ & $5q^{-4}$ & $3q^{-6}$ &  \\
$4$ & $3$ & $10q^{-2}$ & $8q^{-4}$ & $q^{-6}$ \\
$2$ & $2q^{2}$ & $11$ & $12q^{-2}$ & $3q^{-4}$ \\
$0$ & $q^{4}$ & $10q^{2}$ & $15$ & $6q^{-2}$ \\
$-2$ &  & $5q^{4}$ & $12q^{2}$ & $6$ \\
$-4$ &  & $2q^{6}$ & $8q^{4}$ & $6q^{2}$ \\
$-6$ &  &  & $3q^{6}$ & $3q^{4}$ \\
$-8$ &  &  & $q^{8}$ & $q^{6}$ \\
\end{tabular}
\vspace{2em}
\end{minipage}
%
\begin{minipage}{\linewidth}
$\bullet\ $ $11a_{80}$ \vspace{0.5em} \\
\begin{tabular}{l|llll}
$k \setminus j$ & $-4$ & $-2$ & $0$ & $2$ \\
\hline
$8$ & $q^{-4}$ & $2q^{-6}$ & $q^{-8}$ &  \\
$6$ & $2q^{-2}$ & $5q^{-4}$ & $3q^{-6}$ &  \\
$4$ & $3$ & $9q^{-2}$ & $7q^{-4}$ & $q^{-6}$ \\
$2$ & $2q^{2}$ & $11$ & $12q^{-2}$ & $3q^{-4}$ \\
$0$ & $q^{4}$ & $9q^{2}$ & $14$ & $5q^{-2}$ \\
$-2$ &  & $5q^{4}$ & $12q^{2}$ & $7$ \\
$-4$ &  & $2q^{6}$ & $7q^{4}$ & $5q^{2}$ \\
$-6$ &  &  & $3q^{6}$ & $3q^{4}$ \\
$-8$ &  &  & $q^{8}$ & $q^{6}$ \\
\end{tabular}
\vspace{2em}
\end{minipage}
%
\begin{minipage}{\linewidth}
$\bullet\ $ $11a_{81}$ \vspace{0.5em} \\
\begin{tabular}{l|llll}
$k \setminus j$ & $-2$ & $0$ & $2$ & $4$ \\
\hline
$8$ & $q^{-4}$ & $2q^{-6}$ & $q^{-8}$ &  \\
$6$ & $2q^{-2}$ & $5q^{-4}$ & $3q^{-6}$ &  \\
$4$ & $2$ & $8q^{-2}$ & $7q^{-4}$ & $q^{-6}$ \\
$2$ & $2q^{2}$ & $9$ & $11q^{-2}$ & $3q^{-4}$ \\
$0$ & $q^{4}$ & $8q^{2}$ & $12$ & $5q^{-2}$ \\
$-2$ &  & $5q^{4}$ & $11q^{2}$ & $6$ \\
$-4$ &  & $2q^{6}$ & $7q^{4}$ & $5q^{2}$ \\
$-6$ &  &  & $3q^{6}$ & $3q^{4}$ \\
$-8$ &  &  & $q^{8}$ & $q^{6}$ \\
\end{tabular}
\vspace{2em}
\end{minipage}
%
\begin{minipage}{\linewidth}
$\bullet\ $ $11a_{82}$ \vspace{0.5em} \\
\begin{tabular}{l|llll}
$k \setminus j$ & $-4$ & $-2$ & $0$ & $2$ \\
\hline
$8$ & $q^{-6}$ & $q^{-8}$ &  &  \\
$6$ & $2q^{-4}$ & $2q^{-6}$ &  &  \\
$4$ & $4q^{-2}$ & $6q^{-4}$ & $2q^{-6}$ &  \\
$2$ & $4$ & $7q^{-2}$ & $3q^{-4}$ &  \\
$0$ & $4q^{2}$ & $10$ & $7q^{-2}$ & $q^{-4}$ \\
$-2$ & $2q^{4}$ & $7q^{2}$ & $5$ & $q^{-2}$ \\
$-4$ & $q^{6}$ & $6q^{4}$ & $7q^{2}$ & $2$ \\
$-6$ &  & $2q^{6}$ & $3q^{4}$ & $q^{2}$ \\
$-8$ &  & $q^{8}$ & $2q^{6}$ & $q^{4}$ \\
\end{tabular}
\vspace{2em}
\end{minipage}
%
\begin{minipage}{\linewidth}
$\bullet\ $ $11a_{83}$ \vspace{0.5em} \\
\begin{tabular}{l|llll}
$k \setminus j$ & $2$ & $4$ & $6$ & $8$ \\
\hline
$8$ & $q^{-6}$ & $q^{-8}$ &  &  \\
$6$ & $2q^{-4}$ & $2q^{-6}$ &  &  \\
$4$ & $4q^{-2}$ & $7q^{-4}$ & $2q^{-6}$ &  \\
$2$ & $4$ & $8q^{-2}$ & $4q^{-4}$ &  \\
$0$ & $4q^{2}$ & $12$ & $9q^{-2}$ & $q^{-4}$ \\
$-2$ & $2q^{4}$ & $8q^{2}$ & $8$ & $2q^{-2}$ \\
$-4$ & $q^{6}$ & $7q^{4}$ & $9q^{2}$ & $3$ \\
$-6$ &  & $2q^{6}$ & $4q^{4}$ & $2q^{2}$ \\
$-8$ &  & $q^{8}$ & $2q^{6}$ & $q^{4}$ \\
\end{tabular}
\vspace{2em}
\end{minipage}
%
\begin{minipage}{\linewidth}
$\bullet\ $ $11a_{84}$ \vspace{0.5em} \\
\begin{tabular}{l|lllll}
$k \setminus j$ & $-4$ & $-2$ & $0$ & $2$ & $4$ \\
\hline
$8$ & $q^{-4}$ & $q^{-6}$ &  &  &  \\
$6$ & $2q^{-2}$ & $3q^{-4}$ & $q^{-6}$ &  &  \\
$4$ & $3$ & $7q^{-2}$ & $4q^{-4}$ &  &  \\
$2$ & $2q^{2}$ & $9$ & $9q^{-2}$ & $2q^{-4}$ &  \\
$0$ & $q^{4}$ & $7q^{2}$ & $11$ & $4q^{-2}$ &  \\
$-2$ &  & $3q^{4}$ & $9q^{2}$ & $7$ & $q^{-2}$ \\
$-4$ &  & $q^{6}$ & $4q^{4}$ & $4q^{2}$ & $1$ \\
$-6$ &  &  & $q^{6}$ & $2q^{4}$ & $q^{2}$ \\
\end{tabular}
\vspace{2em}
\end{minipage}
%
\begin{minipage}{\linewidth}
$\bullet\ $ $11a_{85}$ \vspace{0.5em} \\
\begin{tabular}{l|lllll}
$k \setminus j$ & $0$ & $2$ & $4$ & $6$ & $8$ \\
\hline
$6$ & $q^{-4}$ & $q^{-6}$ &  &  &  \\
$4$ & $2q^{-2}$ & $3q^{-4}$ & $q^{-6}$ &  &  \\
$2$ & $3$ & $8q^{-2}$ & $4q^{-4}$ &  &  \\
$0$ & $2q^{2}$ & $9$ & $9q^{-2}$ & $2q^{-4}$ &  \\
$-2$ & $q^{4}$ & $8q^{2}$ & $12$ & $5q^{-2}$ &  \\
$-4$ &  & $3q^{4}$ & $9q^{2}$ & $7$ & $q^{-2}$ \\
$-6$ &  & $q^{6}$ & $4q^{4}$ & $5q^{2}$ & $2$ \\
$-8$ &  &  & $q^{6}$ & $2q^{4}$ & $q^{2}$ \\
\end{tabular}
\vspace{2em}
\end{minipage}
%
\begin{minipage}{\linewidth}
$\bullet\ $ $11a_{86}$ \vspace{0.5em} \\
\begin{tabular}{l|llll}
$k \setminus j$ & $-2$ & $0$ & $2$ & $4$ \\
\hline
$8$ & $q^{-4}$ & $2q^{-6}$ & $q^{-8}$ &  \\
$6$ & $q^{-2}$ & $3q^{-4}$ & $2q^{-6}$ &  \\
$4$ & $1$ & $6q^{-2}$ & $6q^{-4}$ & $q^{-6}$ \\
$2$ & $q^{2}$ & $5$ & $7q^{-2}$ & $2q^{-4}$ \\
$0$ & $q^{4}$ & $6q^{2}$ & $9$ & $4q^{-2}$ \\
$-2$ &  & $3q^{4}$ & $7q^{2}$ & $4$ \\
$-4$ &  & $2q^{6}$ & $6q^{4}$ & $4q^{2}$ \\
$-6$ &  &  & $2q^{6}$ & $2q^{4}$ \\
$-8$ &  &  & $q^{8}$ & $q^{6}$ \\
\end{tabular}
\vspace{2em}
\end{minipage}
%
\begin{minipage}{\linewidth}
$\bullet\ $ $11a_{88}$ \vspace{0.5em} \\
\begin{tabular}{l|llll}
$k \setminus j$ & $-4$ & $-2$ & $0$ & $2$ \\
\hline
$8$ & $q^{-4}$ & $2q^{-6}$ & $q^{-8}$ &  \\
$6$ & $q^{-2}$ & $3q^{-4}$ & $2q^{-6}$ &  \\
$4$ & $2$ & $7q^{-2}$ & $6q^{-4}$ & $q^{-6}$ \\
$2$ & $q^{2}$ & $7$ & $8q^{-2}$ & $2q^{-4}$ \\
$0$ & $q^{4}$ & $7q^{2}$ & $11$ & $4q^{-2}$ \\
$-2$ &  & $3q^{4}$ & $8q^{2}$ & $5$ \\
$-4$ &  & $2q^{6}$ & $6q^{4}$ & $4q^{2}$ \\
$-6$ &  &  & $2q^{6}$ & $2q^{4}$ \\
$-8$ &  &  & $q^{8}$ & $q^{6}$ \\
\end{tabular}
\vspace{2em}
\end{minipage}
%
\begin{minipage}{\linewidth}
$\bullet\ $ $11a_{89}$ \vspace{0.5em} \\
\begin{tabular}{l|lllll}
$k \setminus j$ & $-2$ & $0$ & $2$ & $4$ & $6$ \\
\hline
$6$ & $q^{-2}$ & $2q^{-4}$ & $q^{-6}$ &  &  \\
$4$ & $1$ & $5q^{-2}$ & $5q^{-4}$ & $q^{-6}$ &  \\
$2$ & $q^{2}$ & $6$ & $10q^{-2}$ & $4q^{-4}$ &  \\
$0$ &  & $5q^{2}$ & $13$ & $9q^{-2}$ & $q^{-4}$ \\
$-2$ &  & $2q^{4}$ & $10q^{2}$ & $11$ & $3q^{-2}$ \\
$-4$ &  &  & $5q^{4}$ & $9q^{2}$ & $4$ \\
$-6$ &  &  & $q^{6}$ & $4q^{4}$ & $3q^{2}$ \\
$-8$ &  &  &  & $q^{6}$ & $q^{4}$ \\
\end{tabular}
\vspace{2em}
\end{minipage}
%
\begin{minipage}{\linewidth}
$\bullet\ $ $11a_{90}$ \vspace{0.5em} \\
\begin{tabular}{l|lllll}
$k \setminus j$ & $-6$ & $-4$ & $-2$ & $0$ & $2$ \\
\hline
$6$ & $q^{-2}$ & $2q^{-4}$ & $q^{-6}$ &  &  \\
$4$ & $1$ & $4q^{-2}$ & $4q^{-4}$ & $q^{-6}$ &  \\
$2$ & $q^{2}$ & $5$ & $7q^{-2}$ & $3q^{-4}$ &  \\
$0$ &  & $4q^{2}$ & $9$ & $6q^{-2}$ & $q^{-4}$ \\
$-2$ &  & $2q^{4}$ & $7q^{2}$ & $6$ & $2q^{-2}$ \\
$-4$ &  &  & $4q^{4}$ & $6q^{2}$ & $2$ \\
$-6$ &  &  & $q^{6}$ & $3q^{4}$ & $2q^{2}$ \\
$-8$ &  &  &  & $q^{6}$ & $q^{4}$ \\
\end{tabular}
\vspace{2em}
\end{minipage}
%
\begin{minipage}{\linewidth}
$\bullet\ $ $11a_{91}$ \vspace{0.5em} \\
\begin{tabular}{l|lllll}
$k \setminus j$ & $-4$ & $-2$ & $0$ & $2$ & $4$ \\
\hline
$6$ & $q^{-2}$ & $2q^{-4}$ & $q^{-6}$ &  &  \\
$4$ & $2$ & $6q^{-2}$ & $5q^{-4}$ & $q^{-6}$ &  \\
$2$ & $q^{2}$ & $8$ & $11q^{-2}$ & $4q^{-4}$ &  \\
$0$ &  & $6q^{2}$ & $15$ & $9q^{-2}$ & $q^{-4}$ \\
$-2$ &  & $2q^{4}$ & $11q^{2}$ & $12$ & $3q^{-2}$ \\
$-4$ &  &  & $5q^{4}$ & $9q^{2}$ & $4$ \\
$-6$ &  &  & $q^{6}$ & $4q^{4}$ & $3q^{2}$ \\
$-8$ &  &  &  & $q^{6}$ & $q^{4}$ \\
\end{tabular}
\vspace{2em}
\end{minipage}
%
\begin{minipage}{\linewidth}
$\bullet\ $ $11a_{92}$ \vspace{0.5em} \\
\begin{tabular}{l|llll}
$k \setminus j$ & $-6$ & $-4$ & $-2$ & $0$ \\
\hline
$8$ & $q^{-4}$ & $2q^{-6}$ & $q^{-8}$ &  \\
$6$ & $q^{-2}$ & $3q^{-4}$ & $2q^{-6}$ &  \\
$4$ & $3$ & $8q^{-2}$ & $6q^{-4}$ & $q^{-6}$ \\
$2$ & $q^{2}$ & $7$ & $8q^{-2}$ & $2q^{-4}$ \\
$0$ & $q^{4}$ & $8q^{2}$ & $11$ & $4q^{-2}$ \\
$-2$ &  & $3q^{4}$ & $8q^{2}$ & $4$ \\
$-4$ &  & $2q^{6}$ & $6q^{4}$ & $4q^{2}$ \\
$-6$ &  &  & $2q^{6}$ & $2q^{4}$ \\
$-8$ &  &  & $q^{8}$ & $q^{6}$ \\
\end{tabular}
\vspace{2em}
\end{minipage}
%
\begin{minipage}{\linewidth}
$\bullet\ $ $11a_{93}$ \vspace{0.5em} \\
\begin{tabular}{l|lllll}
$k \setminus j$ & $-6$ & $-4$ & $-2$ & $0$ & $2$ \\
\hline
$8$ & $q^{-2}$ & $2q^{-4}$ & $q^{-6}$ &  &  \\
$6$ & $1$ & $4q^{-2}$ & $4q^{-4}$ & $q^{-6}$ &  \\
$4$ & $q^{2}$ & $5$ & $7q^{-2}$ & $3q^{-4}$ &  \\
$2$ &  & $4q^{2}$ & $10$ & $7q^{-2}$ & $q^{-4}$ \\
$0$ &  & $2q^{4}$ & $7q^{2}$ & $8$ & $2q^{-2}$ \\
$-2$ &  &  & $4q^{4}$ & $7q^{2}$ & $3$ \\
$-4$ &  &  & $q^{6}$ & $3q^{4}$ & $2q^{2}$ \\
$-6$ &  &  &  & $q^{6}$ & $q^{4}$ \\
\end{tabular}
\vspace{2em}
\end{minipage}
%
\begin{minipage}{\linewidth}
$\bullet\ $ $11a_{94}$ \vspace{0.5em} \\
\begin{tabular}{l|llll}
$k \setminus j$ & $6$ & $8$ & $10$ & $12$ \\
\hline
$6$ & $q^{-6}$ &  &  &  \\
$4$ & $2q^{-4}$ & $2q^{-6}$ &  &  \\
$2$ & $4q^{-2}$ & $5q^{-4}$ & $q^{-6}$ &  \\
$0$ & $4$ & $9q^{-2}$ & $5q^{-4}$ &  \\
$-2$ & $4q^{2}$ & $10$ & $7q^{-2}$ & $q^{-4}$ \\
$-4$ & $2q^{4}$ & $9q^{2}$ & $10$ & $3q^{-2}$ \\
$-6$ & $q^{6}$ & $5q^{4}$ & $7q^{2}$ & $3$ \\
$-8$ &  & $2q^{6}$ & $5q^{4}$ & $3q^{2}$ \\
$-10$ &  &  & $q^{6}$ & $q^{4}$ \\
\end{tabular}
\vspace{2em}
\end{minipage}
%
\begin{minipage}{\linewidth}
$\bullet\ $ $11a_{96}$ \vspace{0.5em} \\
\begin{tabular}{l|llllll}
$k \setminus j$ & $-6$ & $-4$ & $-2$ & $0$ & $2$ & $4$ \\
\hline
$6$ & $1$ & $3q^{-2}$ & $3q^{-4}$ & $q^{-6}$ &  &  \\
$4$ &  & $4$ & $8q^{-2}$ & $4q^{-4}$ &  &  \\
$2$ &  & $3q^{2}$ & $12$ & $11q^{-2}$ & $2q^{-4}$ &  \\
$0$ &  &  & $8q^{2}$ & $15$ & $6q^{-2}$ &  \\
$-2$ &  &  & $3q^{4}$ & $11q^{2}$ & $9$ & $q^{-2}$ \\
$-4$ &  &  &  & $4q^{4}$ & $6q^{2}$ & $2$ \\
$-6$ &  &  &  & $q^{6}$ & $2q^{4}$ & $q^{2}$ \\
\end{tabular}
\vspace{2em}
\end{minipage}
%
\begin{minipage}{\linewidth}
$\bullet\ $ $11a_{97}$ \vspace{0.5em} \\
\begin{tabular}{l|lllll}
$k \setminus j$ & $-4$ & $-2$ & $0$ & $2$ & $4$ \\
\hline
$8$ & $q^{-2}$ & $2q^{-4}$ & $q^{-6}$ &  &  \\
$6$ &  & $2q^{-2}$ & $3q^{-4}$ & $q^{-6}$ &  \\
$4$ & $q^{2}$ & $4$ & $6q^{-2}$ & $3q^{-4}$ &  \\
$2$ &  & $2q^{2}$ & $5$ & $5q^{-2}$ & $q^{-4}$ \\
$0$ &  & $2q^{4}$ & $6q^{2}$ & $6$ & $2q^{-2}$ \\
$-2$ &  &  & $3q^{4}$ & $5q^{2}$ & $2$ \\
$-4$ &  &  & $q^{6}$ & $3q^{4}$ & $2q^{2}$ \\
$-6$ &  &  &  & $q^{6}$ & $q^{4}$ \\
\end{tabular}
\vspace{2em}
\end{minipage}
%
\begin{minipage}{\linewidth}
$\bullet\ $ $11a_{98}$ \vspace{0.5em} \\
\begin{tabular}{l|llllll}
$k \setminus j$ & $-6$ & $-4$ & $-2$ & $0$ & $2$ & $4$ \\
\hline
$6$ & $1$ & $2q^{-2}$ & $q^{-4}$ &  &  &  \\
$4$ &  & $3$ & $5q^{-2}$ & $2q^{-4}$ &  &  \\
$2$ &  & $2q^{2}$ & $8$ & $7q^{-2}$ & $q^{-4}$ &  \\
$0$ &  &  & $5q^{2}$ & $10$ & $4q^{-2}$ &  \\
$-2$ &  &  & $q^{4}$ & $7q^{2}$ & $7$ & $q^{-2}$ \\
$-4$ &  &  &  & $2q^{4}$ & $4q^{2}$ & $2$ \\
$-6$ &  &  &  &  & $q^{4}$ & $q^{2}$ \\
\end{tabular}
\vspace{2em}
\end{minipage}
%
\begin{minipage}{\linewidth}
$\bullet\ $ $11a_{99}$ \vspace{0.5em} \\
\begin{tabular}{l|llll}
$k \setminus j$ & $-2$ & $0$ & $2$ & $4$ \\
\hline
$8$ & $q^{-4}$ & $2q^{-6}$ & $q^{-8}$ &  \\
$6$ & $2q^{-2}$ & $5q^{-4}$ & $3q^{-6}$ &  \\
$4$ & $2$ & $9q^{-2}$ & $8q^{-4}$ & $q^{-6}$ \\
$2$ & $2q^{2}$ & $9$ & $11q^{-2}$ & $3q^{-4}$ \\
$0$ & $q^{4}$ & $9q^{2}$ & $14$ & $6q^{-2}$ \\
$-2$ &  & $5q^{4}$ & $11q^{2}$ & $6$ \\
$-4$ &  & $2q^{6}$ & $8q^{4}$ & $6q^{2}$ \\
$-6$ &  &  & $3q^{6}$ & $3q^{4}$ \\
$-8$ &  &  & $q^{8}$ & $q^{6}$ \\
\end{tabular}
\vspace{2em}
\end{minipage}
%
\begin{minipage}{\linewidth}
$\bullet\ $ $11a_{100}$ \vspace{0.5em} \\
\begin{tabular}{l|lllll}
$k \setminus j$ & $2$ & $4$ & $6$ & $8$ & $10$ \\
\hline
$6$ & $q^{-4}$ & $q^{-6}$ &  &  &  \\
$4$ & $2q^{-2}$ & $5q^{-4}$ & $2q^{-6}$ &  &  \\
$2$ & $3$ & $9q^{-2}$ & $6q^{-4}$ &  &  \\
$0$ & $2q^{2}$ & $12$ & $13q^{-2}$ & $3q^{-4}$ &  \\
$-2$ & $q^{4}$ & $9q^{2}$ & $15$ & $7q^{-2}$ &  \\
$-4$ &  & $5q^{4}$ & $13q^{2}$ & $9$ & $q^{-2}$ \\
$-6$ &  & $q^{6}$ & $6q^{4}$ & $7q^{2}$ & $2$ \\
$-8$ &  &  & $2q^{6}$ & $3q^{4}$ & $q^{2}$ \\
\end{tabular}
\vspace{2em}
\end{minipage}
%
\begin{minipage}{\linewidth}
$\bullet\ $ $11a_{105}$ \vspace{0.5em} \\
\begin{tabular}{l|lllll}
$k \setminus j$ & $2$ & $4$ & $6$ & $8$ & $10$ \\
\hline
$6$ & $q^{-4}$ & $q^{-6}$ &  &  &  \\
$4$ & $q^{-2}$ & $4q^{-4}$ & $2q^{-6}$ &  &  \\
$2$ & $2$ & $7q^{-2}$ & $5q^{-4}$ &  &  \\
$0$ & $q^{2}$ & $8$ & $10q^{-2}$ & $3q^{-4}$ &  \\
$-2$ & $q^{4}$ & $7q^{2}$ & $11$ & $5q^{-2}$ &  \\
$-4$ &  & $4q^{4}$ & $10q^{2}$ & $7$ & $q^{-2}$ \\
$-6$ &  & $q^{6}$ & $5q^{4}$ & $5q^{2}$ & $1$ \\
$-8$ &  &  & $2q^{6}$ & $3q^{4}$ & $q^{2}$ \\
\end{tabular}
\vspace{2em}
\end{minipage}
%
\begin{minipage}{\linewidth}
$\bullet\ $ $11a_{106}$ \vspace{0.5em} \\
\begin{tabular}{l|llll}
$k \setminus j$ & $-2$ & $0$ & $2$ & $4$ \\
\hline
$8$ & $q^{-6}$ & $q^{-8}$ &  &  \\
$6$ & $2q^{-4}$ & $2q^{-6}$ &  &  \\
$4$ & $4q^{-2}$ & $6q^{-4}$ & $2q^{-6}$ &  \\
$2$ & $4$ & $7q^{-2}$ & $3q^{-4}$ &  \\
$0$ & $4q^{2}$ & $10$ & $6q^{-2}$ & $q^{-4}$ \\
$-2$ & $2q^{4}$ & $7q^{2}$ & $6$ & $q^{-2}$ \\
$-4$ & $q^{6}$ & $6q^{4}$ & $6q^{2}$ & $1$ \\
$-6$ &  & $2q^{6}$ & $3q^{4}$ & $q^{2}$ \\
$-8$ &  & $q^{8}$ & $2q^{6}$ & $q^{4}$ \\
\end{tabular}
\vspace{2em}
\end{minipage}
%
\begin{minipage}{\linewidth}
$\bullet\ $ $11a_{108}$ \vspace{0.5em} \\
\begin{tabular}{l|llll}
$k \setminus j$ & $-4$ & $-2$ & $0$ & $2$ \\
\hline
$8$ & $q^{-6}$ & $q^{-8}$ &  &  \\
$6$ & $2q^{-4}$ & $2q^{-6}$ &  &  \\
$4$ & $4q^{-2}$ & $6q^{-4}$ & $2q^{-6}$ &  \\
$2$ & $5$ & $8q^{-2}$ & $3q^{-4}$ &  \\
$0$ & $4q^{2}$ & $10$ & $7q^{-2}$ & $q^{-4}$ \\
$-2$ & $2q^{4}$ & $8q^{2}$ & $6$ & $q^{-2}$ \\
$-4$ & $q^{6}$ & $6q^{4}$ & $7q^{2}$ & $2$ \\
$-6$ &  & $2q^{6}$ & $3q^{4}$ & $q^{2}$ \\
$-8$ &  & $q^{8}$ & $2q^{6}$ & $q^{4}$ \\
\end{tabular}
\vspace{2em}
\end{minipage}
%
\begin{minipage}{\linewidth}
$\bullet\ $ $11a_{109}$ \vspace{0.5em} \\
\begin{tabular}{l|llll}
$k \setminus j$ & $-2$ & $0$ & $2$ & $4$ \\
\hline
$8$ & $q^{-6}$ & $q^{-8}$ &  &  \\
$6$ & $2q^{-4}$ & $2q^{-6}$ &  &  \\
$4$ & $5q^{-2}$ & $7q^{-4}$ & $2q^{-6}$ &  \\
$2$ & $5$ & $9q^{-2}$ & $4q^{-4}$ &  \\
$0$ & $5q^{2}$ & $13$ & $8q^{-2}$ & $q^{-4}$ \\
$-2$ & $2q^{4}$ & $9q^{2}$ & $9$ & $2q^{-2}$ \\
$-4$ & $q^{6}$ & $7q^{4}$ & $8q^{2}$ & $2$ \\
$-6$ &  & $2q^{6}$ & $4q^{4}$ & $2q^{2}$ \\
$-8$ &  & $q^{8}$ & $2q^{6}$ & $q^{4}$ \\
\end{tabular}
\vspace{2em}
\end{minipage}
%
\begin{minipage}{\linewidth}
$\bullet\ $ $11a_{110}$ \vspace{0.5em} \\
\begin{tabular}{l|lllll}
$k \setminus j$ & $-4$ & $-2$ & $0$ & $2$ & $4$ \\
\hline
$8$ & $q^{-4}$ & $q^{-6}$ &  &  &  \\
$6$ & $2q^{-2}$ & $3q^{-4}$ & $q^{-6}$ &  &  \\
$4$ & $3$ & $7q^{-2}$ & $4q^{-4}$ &  &  \\
$2$ & $2q^{2}$ & $8$ & $8q^{-2}$ & $2q^{-4}$ &  \\
$0$ & $q^{4}$ & $7q^{2}$ & $11$ & $4q^{-2}$ &  \\
$-2$ &  & $3q^{4}$ & $8q^{2}$ & $6$ & $q^{-2}$ \\
$-4$ &  & $q^{6}$ & $4q^{4}$ & $4q^{2}$ & $1$ \\
$-6$ &  &  & $q^{6}$ & $2q^{4}$ & $q^{2}$ \\
\end{tabular}
\vspace{2em}
\end{minipage}
%
\begin{minipage}{\linewidth}
$\bullet\ $ $11a_{111}$ \vspace{0.5em} \\
\begin{tabular}{l|lllll}
$k \setminus j$ & $-2$ & $0$ & $2$ & $4$ & $6$ \\
\hline
$8$ & $q^{-4}$ & $q^{-6}$ &  &  &  \\
$6$ & $2q^{-2}$ & $3q^{-4}$ & $q^{-6}$ &  &  \\
$4$ & $3$ & $7q^{-2}$ & $4q^{-4}$ &  &  \\
$2$ & $2q^{2}$ & $8$ & $9q^{-2}$ & $2q^{-4}$ &  \\
$0$ & $q^{4}$ & $7q^{2}$ & $11$ & $5q^{-2}$ &  \\
$-2$ &  & $3q^{4}$ & $9q^{2}$ & $7$ & $q^{-2}$ \\
$-4$ &  & $q^{6}$ & $4q^{4}$ & $5q^{2}$ & $2$ \\
$-6$ &  &  & $q^{6}$ & $2q^{4}$ & $q^{2}$ \\
\end{tabular}
\vspace{2em}
\end{minipage}
%
\begin{minipage}{\linewidth}
$\bullet\ $ $11a_{112}$ \vspace{0.5em} \\
\begin{tabular}{l|llll}
$k \setminus j$ & $-4$ & $-2$ & $0$ & $2$ \\
\hline
$8$ & $q^{-4}$ & $2q^{-6}$ & $q^{-8}$ &  \\
$6$ & $q^{-2}$ & $4q^{-4}$ & $3q^{-6}$ &  \\
$4$ & $2$ & $8q^{-2}$ & $7q^{-4}$ & $q^{-6}$ \\
$2$ & $q^{2}$ & $9$ & $11q^{-2}$ & $3q^{-4}$ \\
$0$ & $q^{4}$ & $8q^{2}$ & $13$ & $5q^{-2}$ \\
$-2$ &  & $4q^{4}$ & $11q^{2}$ & $7$ \\
$-4$ &  & $2q^{6}$ & $7q^{4}$ & $5q^{2}$ \\
$-6$ &  &  & $3q^{6}$ & $3q^{4}$ \\
$-8$ &  &  & $q^{8}$ & $q^{6}$ \\
\end{tabular}
\vspace{2em}
\end{minipage}
%
\begin{minipage}{\linewidth}
$\bullet\ $ $11a_{113}$ \vspace{0.5em} \\
\begin{tabular}{l|llll}
$k \setminus j$ & $-8$ & $-6$ & $-4$ & $-2$ \\
\hline
$8$ & $q^{-4}$ & $2q^{-6}$ & $q^{-8}$ &  \\
$6$ & $q^{-2}$ & $4q^{-4}$ & $3q^{-6}$ &  \\
$4$ & $1$ & $7q^{-2}$ & $7q^{-4}$ & $q^{-6}$ \\
$2$ & $q^{2}$ & $7$ & $9q^{-2}$ & $3q^{-4}$ \\
$0$ & $q^{4}$ & $7q^{2}$ & $10$ & $4q^{-2}$ \\
$-2$ &  & $4q^{4}$ & $9q^{2}$ & $5$ \\
$-4$ &  & $2q^{6}$ & $7q^{4}$ & $4q^{2}$ \\
$-6$ &  &  & $3q^{6}$ & $3q^{4}$ \\
$-8$ &  &  & $q^{8}$ & $q^{6}$ \\
\end{tabular}
\vspace{2em}
\end{minipage}
%
\begin{minipage}{\linewidth}
$\bullet\ $ $11a_{117}$ \vspace{0.5em} \\
\begin{tabular}{l|lllll}
$k \setminus j$ & $2$ & $4$ & $6$ & $8$ & $10$ \\
\hline
$6$ & $q^{-4}$ & $q^{-6}$ &  &  &  \\
$4$ & $2q^{-2}$ & $4q^{-4}$ & $q^{-6}$ &  &  \\
$2$ & $3$ & $8q^{-2}$ & $5q^{-4}$ &  &  \\
$0$ & $2q^{2}$ & $10$ & $10q^{-2}$ & $2q^{-4}$ &  \\
$-2$ & $q^{4}$ & $8q^{2}$ & $13$ & $6q^{-2}$ &  \\
$-4$ &  & $4q^{4}$ & $10q^{2}$ & $7$ & $q^{-2}$ \\
$-6$ &  & $q^{6}$ & $5q^{4}$ & $6q^{2}$ & $2$ \\
$-8$ &  &  & $q^{6}$ & $2q^{4}$ & $q^{2}$ \\
\end{tabular}
\vspace{2em}
\end{minipage}
%
\begin{minipage}{\linewidth}
$\bullet\ $ $11a_{118}$ \vspace{0.5em} \\
\begin{tabular}{l|lllll}
$k \setminus j$ & $-2$ & $0$ & $2$ & $4$ & $6$ \\
\hline
$6$ & $q^{-2}$ & $2q^{-4}$ & $q^{-6}$ &  &  \\
$4$ &  & $3q^{-2}$ & $4q^{-4}$ & $q^{-6}$ &  \\
$2$ & $q^{2}$ & $4$ & $7q^{-2}$ & $3q^{-4}$ &  \\
$0$ &  & $3q^{2}$ & $9$ & $7q^{-2}$ & $q^{-4}$ \\
$-2$ &  & $2q^{4}$ & $7q^{2}$ & $7$ & $2q^{-2}$ \\
$-4$ &  &  & $4q^{4}$ & $7q^{2}$ & $3$ \\
$-6$ &  &  & $q^{6}$ & $3q^{4}$ & $2q^{2}$ \\
$-8$ &  &  &  & $q^{6}$ & $q^{4}$ \\
\end{tabular}
\vspace{2em}
\end{minipage}
%
\begin{minipage}{\linewidth}
$\bullet\ $ $11a_{119}$ \vspace{0.5em} \\
\begin{tabular}{l|llllll}
$k \setminus j$ & $-4$ & $-2$ & $0$ & $2$ & $4$ & $6$ \\
\hline
$4$ & $1$ & $2q^{-2}$ & $q^{-4}$ &  &  &  \\
$2$ &  & $4$ & $6q^{-2}$ & $2q^{-4}$ &  &  \\
$0$ &  & $2q^{2}$ & $8$ & $6q^{-2}$ & $q^{-4}$ &  \\
$-2$ &  &  & $6q^{2}$ & $10$ & $4q^{-2}$ &  \\
$-4$ &  &  & $q^{4}$ & $6q^{2}$ & $6$ & $q^{-2}$ \\
$-6$ &  &  &  & $2q^{4}$ & $4q^{2}$ & $2$ \\
$-8$ &  &  &  &  & $q^{4}$ & $q^{2}$ \\
\end{tabular}
\vspace{2em}
\end{minipage}
%
\begin{minipage}{\linewidth}
$\bullet\ $ $11a_{120}$ \vspace{0.5em} \\
\begin{tabular}{l|lllll}
$k \setminus j$ & $-10$ & $-8$ & $-6$ & $-4$ & $-2$ \\
\hline
$8$ & $q^{-2}$ & $2q^{-4}$ & $q^{-6}$ &  &  \\
$6$ & $1$ & $5q^{-2}$ & $5q^{-4}$ & $q^{-6}$ &  \\
$4$ & $q^{2}$ & $6$ & $9q^{-2}$ & $4q^{-4}$ &  \\
$2$ &  & $5q^{2}$ & $12$ & $8q^{-2}$ & $q^{-4}$ \\
$0$ &  & $2q^{4}$ & $9q^{2}$ & $9$ & $2q^{-2}$ \\
$-2$ &  &  & $5q^{4}$ & $8q^{2}$ & $3$ \\
$-4$ &  &  & $q^{6}$ & $4q^{4}$ & $2q^{2}$ \\
$-6$ &  &  &  & $q^{6}$ & $q^{4}$ \\
\end{tabular}
\vspace{2em}
\end{minipage}
%
\begin{minipage}{\linewidth}
$\bullet\ $ $11a_{121}$ \vspace{0.5em} \\
\begin{tabular}{l|llllll}
$k \setminus j$ & $-8$ & $-6$ & $-4$ & $-2$ & $0$ & $2$ \\
\hline
$6$ & $1$ & $3q^{-2}$ & $3q^{-4}$ & $q^{-6}$ &  &  \\
$4$ &  & $4$ & $8q^{-2}$ & $4q^{-4}$ &  &  \\
$2$ &  & $3q^{2}$ & $12$ & $11q^{-2}$ & $2q^{-4}$ &  \\
$0$ &  &  & $8q^{2}$ & $14$ & $6q^{-2}$ &  \\
$-2$ &  &  & $3q^{4}$ & $11q^{2}$ & $8$ & $q^{-2}$ \\
$-4$ &  &  &  & $4q^{4}$ & $6q^{2}$ & $2$ \\
$-6$ &  &  &  & $q^{6}$ & $2q^{4}$ & $q^{2}$ \\
\end{tabular}
\vspace{2em}
\end{minipage}
%
\begin{minipage}{\linewidth}
$\bullet\ $ $11a_{122}$ \vspace{0.5em} \\
\begin{tabular}{l|lllll}
$k \setminus j$ & $-8$ & $-6$ & $-4$ & $-2$ & $0$ \\
\hline
$8$ & $q^{-2}$ & $2q^{-4}$ & $q^{-6}$ &  &  \\
$6$ & $1$ & $5q^{-2}$ & $5q^{-4}$ & $q^{-6}$ &  \\
$4$ & $q^{2}$ & $8$ & $11q^{-2}$ & $4q^{-4}$ &  \\
$2$ &  & $5q^{2}$ & $14$ & $10q^{-2}$ & $q^{-4}$ \\
$0$ &  & $2q^{4}$ & $11q^{2}$ & $12$ & $3q^{-2}$ \\
$-2$ &  &  & $5q^{4}$ & $10q^{2}$ & $4$ \\
$-4$ &  &  & $q^{6}$ & $4q^{4}$ & $3q^{2}$ \\
$-6$ &  &  &  & $q^{6}$ & $q^{4}$ \\
\end{tabular}
\vspace{2em}
\end{minipage}
%
\begin{minipage}{\linewidth}
$\bullet\ $ $11a_{124}$ \vspace{0.5em} \\
\begin{tabular}{l|llll}
$k \setminus j$ & $6$ & $8$ & $10$ & $12$ \\
\hline
$6$ & $q^{-6}$ &  &  &  \\
$4$ & $3q^{-4}$ & $3q^{-6}$ &  &  \\
$2$ & $6q^{-2}$ & $7q^{-4}$ & $q^{-6}$ &  \\
$0$ & $7$ & $14q^{-2}$ & $7q^{-4}$ &  \\
$-2$ & $6q^{2}$ & $15$ & $10q^{-2}$ & $q^{-4}$ \\
$-4$ & $3q^{4}$ & $14q^{2}$ & $15$ & $4q^{-2}$ \\
$-6$ & $q^{6}$ & $7q^{4}$ & $10q^{2}$ & $4$ \\
$-8$ &  & $3q^{6}$ & $7q^{4}$ & $4q^{2}$ \\
$-10$ &  &  & $q^{6}$ & $q^{4}$ \\
\end{tabular}
\vspace{2em}
\end{minipage}
%
\begin{minipage}{\linewidth}
$\bullet\ $ $11a_{125}$ \vspace{0.5em} \\
\begin{tabular}{l|llll}
$k \setminus j$ & $0$ & $2$ & $4$ & $6$ \\
\hline
$8$ & $q^{-6}$ & $q^{-8}$ &  &  \\
$6$ & $3q^{-4}$ & $3q^{-6}$ &  &  \\
$4$ & $7q^{-2}$ & $9q^{-4}$ & $2q^{-6}$ &  \\
$2$ & $8$ & $15q^{-2}$ & $6q^{-4}$ &  \\
$0$ & $7q^{2}$ & $19$ & $13q^{-2}$ & $q^{-4}$ \\
$-2$ & $3q^{4}$ & $15q^{2}$ & $15$ & $3q^{-2}$ \\
$-4$ & $q^{6}$ & $9q^{4}$ & $13q^{2}$ & $5$ \\
$-6$ &  & $3q^{6}$ & $6q^{4}$ & $3q^{2}$ \\
$-8$ &  & $q^{8}$ & $2q^{6}$ & $q^{4}$ \\
\end{tabular}
\vspace{2em}
\end{minipage}
%
\begin{minipage}{\linewidth}
$\bullet\ $ $11a_{126}$ \vspace{0.5em} \\
\begin{tabular}{l|llll}
$k \setminus j$ & $-2$ & $0$ & $2$ & $4$ \\
\hline
$8$ & $q^{-6}$ & $q^{-8}$ &  &  \\
$6$ & $2q^{-4}$ & $2q^{-6}$ &  &  \\
$4$ & $6q^{-2}$ & $8q^{-4}$ & $2q^{-6}$ &  \\
$2$ & $6$ & $11q^{-2}$ & $5q^{-4}$ &  \\
$0$ & $6q^{2}$ & $17$ & $11q^{-2}$ & $q^{-4}$ \\
$-2$ & $2q^{4}$ & $11q^{2}$ & $12$ & $3q^{-2}$ \\
$-4$ & $q^{6}$ & $8q^{4}$ & $11q^{2}$ & $4$ \\
$-6$ &  & $2q^{6}$ & $5q^{4}$ & $3q^{2}$ \\
$-8$ &  & $q^{8}$ & $2q^{6}$ & $q^{4}$ \\
\end{tabular}
\vspace{2em}
\end{minipage}
%
\begin{minipage}{\linewidth}
$\bullet\ $ $11a_{127}$ \vspace{0.5em} \\
\begin{tabular}{l|llll}
$k \setminus j$ & $2$ & $4$ & $6$ & $8$ \\
\hline
$8$ & $q^{-6}$ & $q^{-8}$ &  &  \\
$6$ & $3q^{-4}$ & $3q^{-6}$ &  &  \\
$4$ & $5q^{-2}$ & $8q^{-4}$ & $2q^{-6}$ &  \\
$2$ & $6$ & $11q^{-2}$ & $5q^{-4}$ &  \\
$0$ & $5q^{2}$ & $14$ & $10q^{-2}$ & $q^{-4}$ \\
$-2$ & $3q^{4}$ & $11q^{2}$ & $10$ & $2q^{-2}$ \\
$-4$ & $q^{6}$ & $8q^{4}$ & $10q^{2}$ & $3$ \\
$-6$ &  & $3q^{6}$ & $5q^{4}$ & $2q^{2}$ \\
$-8$ &  & $q^{8}$ & $2q^{6}$ & $q^{4}$ \\
\end{tabular}
\vspace{2em}
\end{minipage}
%
\begin{minipage}{\linewidth}
$\bullet\ $ $11a_{128}$ \vspace{0.5em} \\
\begin{tabular}{l|llllll}
$k \setminus j$ & $-6$ & $-4$ & $-2$ & $0$ & $2$ & $4$ \\
\hline
$6$ & $1$ & $3q^{-2}$ & $3q^{-4}$ & $q^{-6}$ &  &  \\
$4$ &  & $5$ & $9q^{-2}$ & $4q^{-4}$ &  &  \\
$2$ &  & $3q^{2}$ & $13$ & $12q^{-2}$ & $2q^{-4}$ &  \\
$0$ &  &  & $9q^{2}$ & $16$ & $6q^{-2}$ &  \\
$-2$ &  &  & $3q^{4}$ & $12q^{2}$ & $10$ & $q^{-2}$ \\
$-4$ &  &  &  & $4q^{4}$ & $6q^{2}$ & $2$ \\
$-6$ &  &  &  & $q^{6}$ & $2q^{4}$ & $q^{2}$ \\
\end{tabular}
\vspace{2em}
\end{minipage}
%
\begin{minipage}{\linewidth}
$\bullet\ $ $11a_{129}$ \vspace{0.5em} \\
\begin{tabular}{l|llll}
$k \setminus j$ & $-8$ & $-6$ & $-4$ & $-2$ \\
\hline
$8$ & $q^{-4}$ & $2q^{-6}$ & $q^{-8}$ &  \\
$6$ & $q^{-2}$ & $4q^{-4}$ & $3q^{-6}$ &  \\
$4$ & $2$ & $8q^{-2}$ & $7q^{-4}$ & $q^{-6}$ \\
$2$ & $q^{2}$ & $7$ & $9q^{-2}$ & $3q^{-4}$ \\
$0$ & $q^{4}$ & $8q^{2}$ & $11$ & $4q^{-2}$ \\
$-2$ &  & $4q^{4}$ & $9q^{2}$ & $5$ \\
$-4$ &  & $2q^{6}$ & $7q^{4}$ & $4q^{2}$ \\
$-6$ &  &  & $3q^{6}$ & $3q^{4}$ \\
$-8$ &  &  & $q^{8}$ & $q^{6}$ \\
\end{tabular}
\vspace{2em}
\end{minipage}
%
\begin{minipage}{\linewidth}
$\bullet\ $ $11a_{130}$ \vspace{0.5em} \\
\begin{tabular}{l|lllll}
$k \setminus j$ & $-8$ & $-6$ & $-4$ & $-2$ & $0$ \\
\hline
$8$ & $q^{-2}$ & $2q^{-4}$ & $q^{-6}$ &  &  \\
$6$ & $1$ & $5q^{-2}$ & $5q^{-4}$ & $q^{-6}$ &  \\
$4$ & $q^{2}$ & $7$ & $10q^{-2}$ & $4q^{-4}$ &  \\
$2$ &  & $5q^{2}$ & $14$ & $10q^{-2}$ & $q^{-4}$ \\
$0$ &  & $2q^{4}$ & $10q^{2}$ & $11$ & $3q^{-2}$ \\
$-2$ &  &  & $5q^{4}$ & $10q^{2}$ & $4$ \\
$-4$ &  &  & $q^{6}$ & $4q^{4}$ & $3q^{2}$ \\
$-6$ &  &  &  & $q^{6}$ & $q^{4}$ \\
\end{tabular}
\vspace{2em}
\end{minipage}
%
\begin{minipage}{\linewidth}
$\bullet\ $ $11a_{131}$ \vspace{0.5em} \\
\begin{tabular}{l|llll}
$k \setminus j$ & $-6$ & $-4$ & $-2$ & $0$ \\
\hline
$8$ & $q^{-4}$ & $2q^{-6}$ & $q^{-8}$ &  \\
$6$ & $q^{-2}$ & $4q^{-4}$ & $3q^{-6}$ &  \\
$4$ & $2$ & $9q^{-2}$ & $8q^{-4}$ & $q^{-6}$ \\
$2$ & $q^{2}$ & $9$ & $11q^{-2}$ & $3q^{-4}$ \\
$0$ & $q^{4}$ & $9q^{2}$ & $14$ & $6q^{-2}$ \\
$-2$ &  & $4q^{4}$ & $11q^{2}$ & $6$ \\
$-4$ &  & $2q^{6}$ & $8q^{4}$ & $6q^{2}$ \\
$-6$ &  &  & $3q^{6}$ & $3q^{4}$ \\
$-8$ &  &  & $q^{8}$ & $q^{6}$ \\
\end{tabular}
\vspace{2em}
\end{minipage}
%
\begin{minipage}{\linewidth}
$\bullet\ $ $11a_{133}$ \vspace{0.5em} \\
\begin{tabular}{l|llllll}
$k \setminus j$ & $0$ & $2$ & $4$ & $6$ & $8$ & $10$ \\
\hline
$4$ & $q^{-2}$ & $q^{-4}$ &  &  &  &  \\
$2$ & $1$ & $4q^{-2}$ & $2q^{-4}$ &  &  &  \\
$0$ & $q^{2}$ & $6$ & $7q^{-2}$ & $2q^{-4}$ &  &  \\
$-2$ &  & $4q^{2}$ & $9$ & $5q^{-2}$ &  &  \\
$-4$ &  & $q^{4}$ & $7q^{2}$ & $9$ & $3q^{-2}$ &  \\
$-6$ &  &  & $2q^{4}$ & $5q^{2}$ & $3$ &  \\
$-8$ &  &  &  & $2q^{4}$ & $3q^{2}$ & $1$ \\
\end{tabular}
\vspace{2em}
\end{minipage}
%
\begin{minipage}{\linewidth}
$\bullet\ $ $11a_{134}$ \vspace{0.5em} \\
\begin{tabular}{l|lllll}
$k \setminus j$ & $0$ & $2$ & $4$ & $6$ & $8$ \\
\hline
$6$ & $q^{-4}$ & $q^{-6}$ &  &  &  \\
$4$ & $2q^{-2}$ & $4q^{-4}$ & $2q^{-6}$ &  &  \\
$2$ & $3$ & $9q^{-2}$ & $5q^{-4}$ &  &  \\
$0$ & $2q^{2}$ & $10$ & $11q^{-2}$ & $3q^{-4}$ &  \\
$-2$ & $q^{4}$ & $9q^{2}$ & $13$ & $5q^{-2}$ &  \\
$-4$ &  & $4q^{4}$ & $11q^{2}$ & $8$ & $q^{-2}$ \\
$-6$ &  & $q^{6}$ & $5q^{4}$ & $5q^{2}$ & $1$ \\
$-8$ &  &  & $2q^{6}$ & $3q^{4}$ & $q^{2}$ \\
\end{tabular}
\vspace{2em}
\end{minipage}
%
\begin{minipage}{\linewidth}
$\bullet\ $ $11a_{135}$ \vspace{0.5em} \\
\begin{tabular}{l|lllll}
$k \setminus j$ & $-4$ & $-2$ & $0$ & $2$ & $4$ \\
\hline
$6$ & $q^{-2}$ & $2q^{-4}$ & $q^{-6}$ &  &  \\
$4$ & $2$ & $6q^{-2}$ & $5q^{-4}$ & $q^{-6}$ &  \\
$2$ & $q^{2}$ & $10$ & $14q^{-2}$ & $5q^{-4}$ &  \\
$0$ &  & $6q^{2}$ & $17$ & $11q^{-2}$ & $q^{-4}$ \\
$-2$ &  & $2q^{4}$ & $14q^{2}$ & $16$ & $4q^{-2}$ \\
$-4$ &  &  & $5q^{4}$ & $11q^{2}$ & $6$ \\
$-6$ &  &  & $q^{6}$ & $5q^{4}$ & $4q^{2}$ \\
$-8$ &  &  &  & $q^{6}$ & $q^{4}$ \\
\end{tabular}
\vspace{2em}
\end{minipage}
%
\begin{minipage}{\linewidth}
$\bullet\ $ $11a_{136}$ \vspace{0.5em} \\
\begin{tabular}{l|lllll}
$k \setminus j$ & $-8$ & $-6$ & $-4$ & $-2$ & $0$ \\
\hline
$8$ & $q^{-2}$ & $2q^{-4}$ & $q^{-6}$ &  &  \\
$6$ & $2$ & $7q^{-2}$ & $6q^{-4}$ & $q^{-6}$ &  \\
$4$ & $q^{2}$ & $10$ & $14q^{-2}$ & $5q^{-4}$ &  \\
$2$ &  & $7q^{2}$ & $19$ & $13q^{-2}$ & $q^{-4}$ \\
$0$ &  & $2q^{4}$ & $14q^{2}$ & $16$ & $4q^{-2}$ \\
$-2$ &  &  & $6q^{4}$ & $13q^{2}$ & $6$ \\
$-4$ &  &  & $q^{6}$ & $5q^{4}$ & $4q^{2}$ \\
$-6$ &  &  &  & $q^{6}$ & $q^{4}$ \\
\end{tabular}
\vspace{2em}
\end{minipage}
%
\begin{minipage}{\linewidth}
$\bullet\ $ $11a_{139}$ \vspace{0.5em} \\
\begin{tabular}{l|llll}
$k \setminus j$ & $0$ & $2$ & $4$ & $6$ \\
\hline
$8$ & $q^{-6}$ & $q^{-8}$ &  &  \\
$6$ & $2q^{-4}$ & $2q^{-6}$ &  &  \\
$4$ & $4q^{-2}$ & $6q^{-4}$ & $2q^{-6}$ &  \\
$2$ & $4$ & $8q^{-2}$ & $3q^{-4}$ &  \\
$0$ & $4q^{2}$ & $10$ & $7q^{-2}$ & $q^{-4}$ \\
$-2$ & $2q^{4}$ & $8q^{2}$ & $7$ & $q^{-2}$ \\
$-4$ & $q^{6}$ & $6q^{4}$ & $7q^{2}$ & $2$ \\
$-6$ &  & $2q^{6}$ & $3q^{4}$ & $q^{2}$ \\
$-8$ &  & $q^{8}$ & $2q^{6}$ & $q^{4}$ \\
\end{tabular}
\vspace{2em}
\end{minipage}
%
\begin{minipage}{\linewidth}
$\bullet\ $ $11a_{140}$ \vspace{0.5em} \\
\begin{tabular}{l|lllll}
$k \setminus j$ & $0$ & $2$ & $4$ & $6$ & $8$ \\
\hline
$8$ & $q^{-4}$ & $q^{-6}$ &  &  &  \\
$6$ & $q^{-2}$ & $2q^{-4}$ & $q^{-6}$ &  &  \\
$4$ & $2$ & $4q^{-2}$ & $3q^{-4}$ &  &  \\
$2$ & $q^{2}$ & $4$ & $5q^{-2}$ & $2q^{-4}$ &  \\
$0$ & $q^{4}$ & $4q^{2}$ & $6$ & $3q^{-2}$ &  \\
$-2$ &  & $2q^{4}$ & $5q^{2}$ & $4$ & $q^{-2}$ \\
$-4$ &  & $q^{6}$ & $3q^{4}$ & $3q^{2}$ & $1$ \\
$-6$ &  &  & $q^{6}$ & $2q^{4}$ & $q^{2}$ \\
\end{tabular}
\vspace{2em}
\end{minipage}
%
\begin{minipage}{\linewidth}
$\bullet\ $ $11a_{141}$ \vspace{0.5em} \\
\begin{tabular}{l|lllll}
$k \setminus j$ & $-6$ & $-4$ & $-2$ & $0$ & $2$ \\
\hline
$6$ & $q^{-2}$ & $2q^{-4}$ & $q^{-6}$ &  &  \\
$4$ & $1$ & $4q^{-2}$ & $4q^{-4}$ & $q^{-6}$ &  \\
$2$ & $q^{2}$ & $6$ & $9q^{-2}$ & $4q^{-4}$ &  \\
$0$ &  & $4q^{2}$ & $10$ & $7q^{-2}$ & $q^{-4}$ \\
$-2$ &  & $2q^{4}$ & $9q^{2}$ & $9$ & $3q^{-2}$ \\
$-4$ &  &  & $4q^{4}$ & $7q^{2}$ & $3$ \\
$-6$ &  &  & $q^{6}$ & $4q^{4}$ & $3q^{2}$ \\
$-8$ &  &  &  & $q^{6}$ & $q^{4}$ \\
\end{tabular}
\vspace{2em}
\end{minipage}
%
\begin{minipage}{\linewidth}
$\bullet\ $ $11a_{142}$ \vspace{0.5em} \\
\begin{tabular}{l|llll}
$k \setminus j$ & $-10$ & $-8$ & $-6$ & $-4$ \\
\hline
$8$ & $q^{-4}$ & $2q^{-6}$ & $q^{-8}$ &  \\
$6$ &  & $2q^{-4}$ & $2q^{-6}$ &  \\
$4$ & $1$ & $4q^{-2}$ & $4q^{-4}$ & $q^{-6}$ \\
$2$ &  & $3$ & $4q^{-2}$ & $q^{-4}$ \\
$0$ & $q^{4}$ & $4q^{2}$ & $5$ & $2q^{-2}$ \\
$-2$ &  & $2q^{4}$ & $4q^{2}$ & $2$ \\
$-4$ &  & $2q^{6}$ & $4q^{4}$ & $2q^{2}$ \\
$-6$ &  &  & $2q^{6}$ & $q^{4}$ \\
$-8$ &  &  & $q^{8}$ & $q^{6}$ \\
\end{tabular}
\vspace{2em}
\end{minipage}
%
\begin{minipage}{\linewidth}
$\bullet\ $ $11a_{143}$ \vspace{0.5em} \\
\begin{tabular}{l|lllll}
$k \setminus j$ & $-10$ & $-8$ & $-6$ & $-4$ & $-2$ \\
\hline
$8$ & $q^{-2}$ & $2q^{-4}$ & $q^{-6}$ &  &  \\
$6$ &  & $3q^{-2}$ & $4q^{-4}$ & $q^{-6}$ &  \\
$4$ & $q^{2}$ & $5$ & $8q^{-2}$ & $4q^{-4}$ &  \\
$2$ &  & $3q^{2}$ & $8$ & $6q^{-2}$ & $q^{-4}$ \\
$0$ &  & $2q^{4}$ & $8q^{2}$ & $8$ & $2q^{-2}$ \\
$-2$ &  &  & $4q^{4}$ & $6q^{2}$ & $2$ \\
$-4$ &  &  & $q^{6}$ & $4q^{4}$ & $2q^{2}$ \\
$-6$ &  &  &  & $q^{6}$ & $q^{4}$ \\
\end{tabular}
\vspace{2em}
\end{minipage}
%
\begin{minipage}{\linewidth}
$\bullet\ $ $11a_{144}$ \vspace{0.5em} \\
\begin{tabular}{l|lllll}
$k \setminus j$ & $-10$ & $-8$ & $-6$ & $-4$ & $-2$ \\
\hline
$8$ & $q^{-2}$ & $2q^{-4}$ & $q^{-6}$ &  &  \\
$6$ &  & $3q^{-2}$ & $4q^{-4}$ & $q^{-6}$ &  \\
$4$ & $q^{2}$ & $4$ & $6q^{-2}$ & $3q^{-4}$ &  \\
$2$ &  & $3q^{2}$ & $7$ & $5q^{-2}$ & $q^{-4}$ \\
$0$ &  & $2q^{4}$ & $6q^{2}$ & $5$ & $q^{-2}$ \\
$-2$ &  &  & $4q^{4}$ & $5q^{2}$ & $1$ \\
$-4$ &  &  & $q^{6}$ & $3q^{4}$ & $q^{2}$ \\
$-6$ &  &  &  & $q^{6}$ & $q^{4}$ \\
\end{tabular}
\vspace{2em}
\end{minipage}
%
\begin{minipage}{\linewidth}
$\bullet\ $ $11a_{145}$ \vspace{0.5em} \\
\begin{tabular}{l|llllll}
$k \setminus j$ & $-10$ & $-8$ & $-6$ & $-4$ & $-2$ & $0$ \\
\hline
$8$ & $1$ & $2q^{-2}$ & $q^{-4}$ &  &  &  \\
$6$ &  & $4$ & $6q^{-2}$ & $2q^{-4}$ &  &  \\
$4$ &  & $2q^{2}$ & $8$ & $7q^{-2}$ & $q^{-4}$ &  \\
$2$ &  &  & $6q^{2}$ & $11$ & $5q^{-2}$ &  \\
$0$ &  &  & $q^{4}$ & $7q^{2}$ & $7$ & $q^{-2}$ \\
$-2$ &  &  &  & $2q^{4}$ & $5q^{2}$ & $2$ \\
$-4$ &  &  &  &  & $q^{4}$ & $q^{2}$ \\
\end{tabular}
\vspace{2em}
\end{minipage}
%
\begin{minipage}{\linewidth}
$\bullet\ $ $11a_{146}$ \vspace{0.5em} \\
\begin{tabular}{l|llll}
$k \setminus j$ & $-6$ & $-4$ & $-2$ & $0$ \\
\hline
$8$ & $q^{-4}$ & $2q^{-6}$ & $q^{-8}$ &  \\
$6$ & $q^{-2}$ & $4q^{-4}$ & $3q^{-6}$ &  \\
$4$ & $2$ & $8q^{-2}$ & $7q^{-4}$ & $q^{-6}$ \\
$2$ & $q^{2}$ & $9$ & $11q^{-2}$ & $3q^{-4}$ \\
$0$ & $q^{4}$ & $8q^{2}$ & $12$ & $5q^{-2}$ \\
$-2$ &  & $4q^{4}$ & $11q^{2}$ & $6$ \\
$-4$ &  & $2q^{6}$ & $7q^{4}$ & $5q^{2}$ \\
$-6$ &  &  & $3q^{6}$ & $3q^{4}$ \\
$-8$ &  &  & $q^{8}$ & $q^{6}$ \\
\end{tabular}
\vspace{2em}
\end{minipage}
%
\begin{minipage}{\linewidth}
$\bullet\ $ $11a_{147}$ \vspace{0.5em} \\
\begin{tabular}{l|llll}
$k \setminus j$ & $0$ & $2$ & $4$ & $6$ \\
\hline
$8$ & $q^{-6}$ & $q^{-8}$ &  &  \\
$6$ & $3q^{-4}$ & $3q^{-6}$ &  &  \\
$4$ & $6q^{-2}$ & $8q^{-4}$ & $2q^{-6}$ &  \\
$2$ & $7$ & $13q^{-2}$ & $5q^{-4}$ &  \\
$0$ & $6q^{2}$ & $16$ & $11q^{-2}$ & $q^{-4}$ \\
$-2$ & $3q^{4}$ & $13q^{2}$ & $12$ & $2q^{-2}$ \\
$-4$ & $q^{6}$ & $8q^{4}$ & $11q^{2}$ & $4$ \\
$-6$ &  & $3q^{6}$ & $5q^{4}$ & $2q^{2}$ \\
$-8$ &  & $q^{8}$ & $2q^{6}$ & $q^{4}$ \\
\end{tabular}
\vspace{2em}
\end{minipage}
%
\begin{minipage}{\linewidth}
$\bullet\ $ $11a_{149}$ \vspace{0.5em} \\
\begin{tabular}{l|lllll}
$k \setminus j$ & $-2$ & $0$ & $2$ & $4$ & $6$ \\
\hline
$6$ & $q^{-2}$ & $2q^{-4}$ & $q^{-6}$ &  &  \\
$4$ & $1$ & $5q^{-2}$ & $5q^{-4}$ & $q^{-6}$ &  \\
$2$ & $q^{2}$ & $7$ & $11q^{-2}$ & $4q^{-4}$ &  \\
$0$ &  & $5q^{2}$ & $14$ & $10q^{-2}$ & $q^{-4}$ \\
$-2$ &  & $2q^{4}$ & $11q^{2}$ & $12$ & $3q^{-2}$ \\
$-4$ &  &  & $5q^{4}$ & $10q^{2}$ & $5$ \\
$-6$ &  &  & $q^{6}$ & $4q^{4}$ & $3q^{2}$ \\
$-8$ &  &  &  & $q^{6}$ & $q^{4}$ \\
\end{tabular}
\vspace{2em}
\end{minipage}
%
\begin{minipage}{\linewidth}
$\bullet\ $ $11a_{150}$ \vspace{0.5em} \\
\begin{tabular}{l|lllll}
$k \setminus j$ & $-10$ & $-8$ & $-6$ & $-4$ & $-2$ \\
\hline
$8$ & $q^{-2}$ & $2q^{-4}$ & $q^{-6}$ &  &  \\
$6$ & $1$ & $5q^{-2}$ & $5q^{-4}$ & $q^{-6}$ &  \\
$4$ & $q^{2}$ & $7$ & $11q^{-2}$ & $5q^{-4}$ &  \\
$2$ &  & $5q^{2}$ & $13$ & $9q^{-2}$ & $q^{-4}$ \\
$0$ &  & $2q^{4}$ & $11q^{2}$ & $12$ & $3q^{-2}$ \\
$-2$ &  &  & $5q^{4}$ & $9q^{2}$ & $4$ \\
$-4$ &  &  & $q^{6}$ & $5q^{4}$ & $3q^{2}$ \\
$-6$ &  &  &  & $q^{6}$ & $q^{4}$ \\
\end{tabular}
\vspace{2em}
\end{minipage}
%
\begin{minipage}{\linewidth}
$\bullet\ $ $11a_{151}$ \vspace{0.5em} \\
\begin{tabular}{l|llll}
$k \setminus j$ & $-6$ & $-4$ & $-2$ & $0$ \\
\hline
$8$ & $q^{-4}$ & $2q^{-6}$ & $q^{-8}$ &  \\
$6$ & $q^{-2}$ & $4q^{-4}$ & $3q^{-6}$ &  \\
$4$ & $3$ & $9q^{-2}$ & $7q^{-4}$ & $q^{-6}$ \\
$2$ & $q^{2}$ & $9$ & $11q^{-2}$ & $3q^{-4}$ \\
$0$ & $q^{4}$ & $9q^{2}$ & $13$ & $5q^{-2}$ \\
$-2$ &  & $4q^{4}$ & $11q^{2}$ & $6$ \\
$-4$ &  & $2q^{6}$ & $7q^{4}$ & $5q^{2}$ \\
$-6$ &  &  & $3q^{6}$ & $3q^{4}$ \\
$-8$ &  &  & $q^{8}$ & $q^{6}$ \\
\end{tabular}
\vspace{2em}
\end{minipage}
%
\begin{minipage}{\linewidth}
$\bullet\ $ $11a_{152}$ \vspace{0.5em} \\
\begin{tabular}{l|lllll}
$k \setminus j$ & $-6$ & $-4$ & $-2$ & $0$ & $2$ \\
\hline
$8$ & $q^{-2}$ & $2q^{-4}$ & $q^{-6}$ &  &  \\
$6$ & $1$ & $5q^{-2}$ & $5q^{-4}$ & $q^{-6}$ &  \\
$4$ & $q^{2}$ & $6$ & $9q^{-2}$ & $4q^{-4}$ &  \\
$2$ &  & $5q^{2}$ & $13$ & $9q^{-2}$ & $q^{-4}$ \\
$0$ &  & $2q^{4}$ & $9q^{2}$ & $11$ & $3q^{-2}$ \\
$-2$ &  &  & $5q^{4}$ & $9q^{2}$ & $4$ \\
$-4$ &  &  & $q^{6}$ & $4q^{4}$ & $3q^{2}$ \\
$-6$ &  &  &  & $q^{6}$ & $q^{4}$ \\
\end{tabular}
\vspace{2em}
\end{minipage}
%
\begin{minipage}{\linewidth}
$\bullet\ $ $11a_{153}$ \vspace{0.5em} \\
\begin{tabular}{l|lllll}
$k \setminus j$ & $-2$ & $0$ & $2$ & $4$ & $6$ \\
\hline
$6$ & $q^{-4}$ & $q^{-6}$ &  &  &  \\
$4$ & $2q^{-2}$ & $3q^{-4}$ & $q^{-6}$ &  &  \\
$2$ & $3$ & $7q^{-2}$ & $4q^{-4}$ &  &  \\
$0$ & $2q^{2}$ & $8$ & $7q^{-2}$ & $2q^{-4}$ &  \\
$-2$ & $q^{4}$ & $7q^{2}$ & $9$ & $3q^{-2}$ &  \\
$-4$ &  & $3q^{4}$ & $7q^{2}$ & $5$ & $q^{-2}$ \\
$-6$ &  & $q^{6}$ & $4q^{4}$ & $3q^{2}$ &  \\
$-8$ &  &  & $q^{6}$ & $2q^{4}$ & $q^{2}$ \\
\end{tabular}
\vspace{2em}
\end{minipage}
%
\begin{minipage}{\linewidth}
$\bullet\ $ $11a_{154}$ \vspace{0.5em} \\
\begin{tabular}{l|llllll}
$k \setminus j$ & $-2$ & $0$ & $2$ & $4$ & $6$ & $8$ \\
\hline
$6$ & $q^{-2}$ & $q^{-4}$ &  &  &  &  \\
$4$ & $1$ & $3q^{-2}$ & $2q^{-4}$ &  &  &  \\
$2$ & $q^{2}$ & $5$ & $6q^{-2}$ & $q^{-4}$ &  &  \\
$0$ &  & $3q^{2}$ & $8$ & $5q^{-2}$ &  &  \\
$-2$ &  & $q^{4}$ & $6q^{2}$ & $7$ & $2q^{-2}$ &  \\
$-4$ &  &  & $2q^{4}$ & $5q^{2}$ & $3$ &  \\
$-6$ &  &  &  & $q^{4}$ & $2q^{2}$ & $1$ \\
\end{tabular}
\vspace{2em}
\end{minipage}
%
\begin{minipage}{\linewidth}
$\bullet\ $ $11a_{156}$ \vspace{0.5em} \\
\begin{tabular}{l|llll}
$k \setminus j$ & $-6$ & $-4$ & $-2$ & $0$ \\
\hline
$8$ & $q^{-4}$ & $2q^{-6}$ & $q^{-8}$ &  \\
$6$ &  & $2q^{-4}$ & $2q^{-6}$ &  \\
$4$ & $2$ & $7q^{-2}$ & $6q^{-4}$ & $q^{-6}$ \\
$2$ &  & $5$ & $7q^{-2}$ & $2q^{-4}$ \\
$0$ & $q^{4}$ & $7q^{2}$ & $10$ & $4q^{-2}$ \\
$-2$ &  & $2q^{4}$ & $7q^{2}$ & $4$ \\
$-4$ &  & $2q^{6}$ & $6q^{4}$ & $4q^{2}$ \\
$-6$ &  &  & $2q^{6}$ & $2q^{4}$ \\
$-8$ &  &  & $q^{8}$ & $q^{6}$ \\
\end{tabular}
\vspace{2em}
\end{minipage}
%
\begin{minipage}{\linewidth}
$\bullet\ $ $11a_{157}$ \vspace{0.5em} \\
\begin{tabular}{l|llll}
$k \setminus j$ & $-6$ & $-4$ & $-2$ & $0$ \\
\hline
$8$ & $q^{-4}$ & $2q^{-6}$ & $q^{-8}$ &  \\
$6$ & $q^{-2}$ & $4q^{-4}$ & $3q^{-6}$ &  \\
$4$ & $3$ & $10q^{-2}$ & $8q^{-4}$ & $q^{-6}$ \\
$2$ & $q^{2}$ & $9$ & $11q^{-2}$ & $3q^{-4}$ \\
$0$ & $q^{4}$ & $10q^{2}$ & $15$ & $6q^{-2}$ \\
$-2$ &  & $4q^{4}$ & $11q^{2}$ & $6$ \\
$-4$ &  & $2q^{6}$ & $8q^{4}$ & $6q^{2}$ \\
$-6$ &  &  & $3q^{6}$ & $3q^{4}$ \\
$-8$ &  &  & $q^{8}$ & $q^{6}$ \\
\end{tabular}
\vspace{2em}
\end{minipage}
%
\begin{minipage}{\linewidth}
$\bullet\ $ $11a_{158}$ \vspace{0.5em} \\
\begin{tabular}{l|llll}
$k \setminus j$ & $-4$ & $-2$ & $0$ & $2$ \\
\hline
$8$ & $q^{-6}$ & $q^{-8}$ &  &  \\
$6$ & $2q^{-4}$ & $2q^{-6}$ &  &  \\
$4$ & $5q^{-2}$ & $7q^{-4}$ & $2q^{-6}$ &  \\
$2$ & $5$ & $9q^{-2}$ & $4q^{-4}$ &  \\
$0$ & $5q^{2}$ & $13$ & $9q^{-2}$ & $q^{-4}$ \\
$-2$ & $2q^{4}$ & $9q^{2}$ & $8$ & $2q^{-2}$ \\
$-4$ & $q^{6}$ & $7q^{4}$ & $9q^{2}$ & $3$ \\
$-6$ &  & $2q^{6}$ & $4q^{4}$ & $2q^{2}$ \\
$-8$ &  & $q^{8}$ & $2q^{6}$ & $q^{4}$ \\
\end{tabular}
\vspace{2em}
\end{minipage}
%
\begin{minipage}{\linewidth}
$\bullet\ $ $11a_{159}$ \vspace{0.5em} \\
\begin{tabular}{l|llllll}
$k \setminus j$ & $-2$ & $0$ & $2$ & $4$ & $6$ & $8$ \\
\hline
$6$ & $q^{-2}$ & $2q^{-4}$ & $q^{-6}$ &  &  &  \\
$4$ & $1$ & $5q^{-2}$ & $4q^{-4}$ &  &  &  \\
$2$ & $q^{2}$ & $7$ & $10q^{-2}$ & $3q^{-4}$ &  &  \\
$0$ &  & $5q^{2}$ & $13$ & $8q^{-2}$ &  &  \\
$-2$ &  & $2q^{4}$ & $10q^{2}$ & $11$ & $3q^{-2}$ &  \\
$-4$ &  &  & $4q^{4}$ & $8q^{2}$ & $4$ &  \\
$-6$ &  &  & $q^{6}$ & $3q^{4}$ & $3q^{2}$ & $1$ \\
\end{tabular}
\vspace{2em}
\end{minipage}
%
\begin{minipage}{\linewidth}
$\bullet\ $ $11a_{160}$ \vspace{0.5em} \\
\begin{tabular}{l|llll}
$k \setminus j$ & $-2$ & $0$ & $2$ & $4$ \\
\hline
$8$ & $q^{-6}$ & $q^{-8}$ &  &  \\
$6$ & $3q^{-4}$ & $3q^{-6}$ &  &  \\
$4$ & $6q^{-2}$ & $8q^{-4}$ & $2q^{-6}$ &  \\
$2$ & $7$ & $12q^{-2}$ & $5q^{-4}$ &  \\
$0$ & $6q^{2}$ & $16$ & $10q^{-2}$ & $q^{-4}$ \\
$-2$ & $3q^{4}$ & $12q^{2}$ & $11$ & $2q^{-2}$ \\
$-4$ & $q^{6}$ & $8q^{4}$ & $10q^{2}$ & $3$ \\
$-6$ &  & $3q^{6}$ & $5q^{4}$ & $2q^{2}$ \\
$-8$ &  & $q^{8}$ & $2q^{6}$ & $q^{4}$ \\
\end{tabular}
\vspace{2em}
\end{minipage}
%
\begin{minipage}{\linewidth}
$\bullet\ $ $11a_{161}$ \vspace{0.5em} \\
\begin{tabular}{l|lllll}
$k \setminus j$ & $-8$ & $-6$ & $-4$ & $-2$ & $0$ \\
\hline
$6$ & $q^{-2}$ & $2q^{-4}$ & $q^{-6}$ &  &  \\
$4$ &  & $2q^{-2}$ & $3q^{-4}$ & $q^{-6}$ &  \\
$2$ & $q^{2}$ & $3$ & $4q^{-2}$ & $2q^{-4}$ &  \\
$0$ &  & $2q^{2}$ & $5$ & $4q^{-2}$ & $q^{-4}$ \\
$-2$ &  & $2q^{4}$ & $4q^{2}$ & $3$ & $q^{-2}$ \\
$-4$ &  &  & $3q^{4}$ & $4q^{2}$ & $2$ \\
$-6$ &  &  & $q^{6}$ & $2q^{4}$ & $q^{2}$ \\
$-8$ &  &  &  & $q^{6}$ & $q^{4}$ \\
\end{tabular}
\vspace{2em}
\end{minipage}
%
\begin{minipage}{\linewidth}
$\bullet\ $ $11a_{162}$ \vspace{0.5em} \\
\begin{tabular}{l|llll}
$k \setminus j$ & $-6$ & $-4$ & $-2$ & $0$ \\
\hline
$8$ & $q^{-4}$ & $2q^{-6}$ & $q^{-8}$ &  \\
$6$ & $2q^{-2}$ & $6q^{-4}$ & $4q^{-6}$ &  \\
$4$ & $3$ & $11q^{-2}$ & $9q^{-4}$ & $q^{-6}$ \\
$2$ & $2q^{2}$ & $13$ & $15q^{-2}$ & $4q^{-4}$ \\
$0$ & $q^{4}$ & $11q^{2}$ & $17$ & $7q^{-2}$ \\
$-2$ &  & $6q^{4}$ & $15q^{2}$ & $8$ \\
$-4$ &  & $2q^{6}$ & $9q^{4}$ & $7q^{2}$ \\
$-6$ &  &  & $4q^{6}$ & $4q^{4}$ \\
$-8$ &  &  & $q^{8}$ & $q^{6}$ \\
\end{tabular}
\vspace{2em}
\end{minipage}
%
\begin{minipage}{\linewidth}
$\bullet\ $ $11a_{163}$ \vspace{0.5em} \\
\begin{tabular}{l|llll}
$k \setminus j$ & $-2$ & $0$ & $2$ & $4$ \\
\hline
$8$ & $q^{-4}$ & $2q^{-6}$ & $q^{-8}$ &  \\
$6$ & $q^{-2}$ & $4q^{-4}$ & $3q^{-6}$ &  \\
$4$ & $2$ & $8q^{-2}$ & $7q^{-4}$ & $q^{-6}$ \\
$2$ & $q^{2}$ & $7$ & $10q^{-2}$ & $3q^{-4}$ \\
$0$ & $q^{4}$ & $8q^{2}$ & $12$ & $5q^{-2}$ \\
$-2$ &  & $4q^{4}$ & $10q^{2}$ & $6$ \\
$-4$ &  & $2q^{6}$ & $7q^{4}$ & $5q^{2}$ \\
$-6$ &  &  & $3q^{6}$ & $3q^{4}$ \\
$-8$ &  &  & $q^{8}$ & $q^{6}$ \\
\end{tabular}
\vspace{2em}
\end{minipage}
%
\begin{minipage}{\linewidth}
$\bullet\ $ $11a_{164}$ \vspace{0.5em} \\
\begin{tabular}{l|llll}
$k \setminus j$ & $-4$ & $-2$ & $0$ & $2$ \\
\hline
$8$ & $q^{-4}$ & $2q^{-6}$ & $q^{-8}$ &  \\
$6$ & $2q^{-2}$ & $6q^{-4}$ & $4q^{-6}$ &  \\
$4$ & $3$ & $11q^{-2}$ & $9q^{-4}$ & $q^{-6}$ \\
$2$ & $2q^{2}$ & $13$ & $15q^{-2}$ & $4q^{-4}$ \\
$0$ & $q^{4}$ & $11q^{2}$ & $18$ & $7q^{-2}$ \\
$-2$ &  & $6q^{4}$ & $15q^{2}$ & $9$ \\
$-4$ &  & $2q^{6}$ & $9q^{4}$ & $7q^{2}$ \\
$-6$ &  &  & $4q^{6}$ & $4q^{4}$ \\
$-8$ &  &  & $q^{8}$ & $q^{6}$ \\
\end{tabular}
\vspace{2em}
\end{minipage}
%
\begin{minipage}{\linewidth}
$\bullet\ $ $11a_{165}$ \vspace{0.5em} \\
\begin{tabular}{l|lllll}
$k \setminus j$ & $-2$ & $0$ & $2$ & $4$ & $6$ \\
\hline
$6$ & $q^{-4}$ & $q^{-6}$ &  &  &  \\
$4$ & $2q^{-2}$ & $3q^{-4}$ & $q^{-6}$ &  &  \\
$2$ & $3$ & $6q^{-2}$ & $3q^{-4}$ &  &  \\
$0$ & $2q^{2}$ & $8$ & $7q^{-2}$ & $2q^{-4}$ &  \\
$-2$ & $q^{4}$ & $6q^{2}$ & $7$ & $2q^{-2}$ &  \\
$-4$ &  & $3q^{4}$ & $7q^{2}$ & $5$ & $q^{-2}$ \\
$-6$ &  & $q^{6}$ & $3q^{4}$ & $2q^{2}$ &  \\
$-8$ &  &  & $q^{6}$ & $2q^{4}$ & $q^{2}$ \\
\end{tabular}
\vspace{2em}
\end{minipage}
%
\begin{minipage}{\linewidth}
$\bullet\ $ $11a_{166}$ \vspace{0.5em} \\
\begin{tabular}{l|llllll}
$k \setminus j$ & $-2$ & $0$ & $2$ & $4$ & $6$ & $8$ \\
\hline
$6$ & $q^{-2}$ & $q^{-4}$ &  &  &  &  \\
$4$ & $1$ & $3q^{-2}$ & $2q^{-4}$ &  &  &  \\
$2$ & $q^{2}$ & $4$ & $5q^{-2}$ & $q^{-4}$ &  &  \\
$0$ &  & $3q^{2}$ & $7$ & $4q^{-2}$ &  &  \\
$-2$ &  & $q^{4}$ & $5q^{2}$ & $6$ & $2q^{-2}$ &  \\
$-4$ &  &  & $2q^{4}$ & $4q^{2}$ & $2$ &  \\
$-6$ &  &  &  & $q^{4}$ & $2q^{2}$ & $1$ \\
\end{tabular}
\vspace{2em}
\end{minipage}
%
\begin{minipage}{\linewidth}
$\bullet\ $ $11a_{167}$ \vspace{0.5em} \\
\begin{tabular}{l|lllll}
$k \setminus j$ & $-4$ & $-2$ & $0$ & $2$ & $4$ \\
\hline
$6$ & $q^{-2}$ & $2q^{-4}$ & $q^{-6}$ &  &  \\
$4$ & $1$ & $4q^{-2}$ & $4q^{-4}$ & $q^{-6}$ &  \\
$2$ & $q^{2}$ & $7$ & $10q^{-2}$ & $4q^{-4}$ &  \\
$0$ &  & $4q^{2}$ & $12$ & $8q^{-2}$ & $q^{-4}$ \\
$-2$ &  & $2q^{4}$ & $10q^{2}$ & $11$ & $3q^{-2}$ \\
$-4$ &  &  & $4q^{4}$ & $8q^{2}$ & $4$ \\
$-6$ &  &  & $q^{6}$ & $4q^{4}$ & $3q^{2}$ \\
$-8$ &  &  &  & $q^{6}$ & $q^{4}$ \\
\end{tabular}
\vspace{2em}
\end{minipage}
%
\begin{minipage}{\linewidth}
$\bullet\ $ $11a_{169}$ \vspace{0.5em} \\
\begin{tabular}{l|lllll}
$k \setminus j$ & $-6$ & $-4$ & $-2$ & $0$ & $2$ \\
\hline
$8$ & $q^{-2}$ & $2q^{-4}$ & $q^{-6}$ &  &  \\
$6$ & $1$ & $5q^{-2}$ & $5q^{-4}$ & $q^{-6}$ &  \\
$4$ & $q^{2}$ & $7$ & $10q^{-2}$ & $4q^{-4}$ &  \\
$2$ &  & $5q^{2}$ & $13$ & $9q^{-2}$ & $q^{-4}$ \\
$0$ &  & $2q^{4}$ & $10q^{2}$ & $12$ & $3q^{-2}$ \\
$-2$ &  &  & $5q^{4}$ & $9q^{2}$ & $4$ \\
$-4$ &  &  & $q^{6}$ & $4q^{4}$ & $3q^{2}$ \\
$-6$ &  &  &  & $q^{6}$ & $q^{4}$ \\
\end{tabular}
\vspace{2em}
\end{minipage}
%
\begin{minipage}{\linewidth}
$\bullet\ $ $11a_{170}$ \vspace{0.5em} \\
\begin{tabular}{l|llll}
$k \setminus j$ & $-2$ & $0$ & $2$ & $4$ \\
\hline
$8$ & $q^{-6}$ & $q^{-8}$ &  &  \\
$6$ & $3q^{-4}$ & $3q^{-6}$ &  &  \\
$4$ & $7q^{-2}$ & $9q^{-4}$ & $2q^{-6}$ &  \\
$2$ & $9$ & $16q^{-2}$ & $7q^{-4}$ &  \\
$0$ & $7q^{2}$ & $20$ & $13q^{-2}$ & $q^{-4}$ \\
$-2$ & $3q^{4}$ & $16q^{2}$ & $17$ & $4q^{-2}$ \\
$-4$ & $q^{6}$ & $9q^{4}$ & $13q^{2}$ & $5$ \\
$-6$ &  & $3q^{6}$ & $7q^{4}$ & $4q^{2}$ \\
$-8$ &  & $q^{8}$ & $2q^{6}$ & $q^{4}$ \\
\end{tabular}
\vspace{2em}
\end{minipage}
%
\begin{minipage}{\linewidth}
$\bullet\ $ $11a_{171}$ \vspace{0.5em} \\
\begin{tabular}{l|llll}
$k \setminus j$ & $0$ & $2$ & $4$ & $6$ \\
\hline
$8$ & $q^{-6}$ & $q^{-8}$ &  &  \\
$6$ & $4q^{-4}$ & $4q^{-6}$ &  &  \\
$4$ & $7q^{-2}$ & $9q^{-4}$ & $2q^{-6}$ &  \\
$2$ & $9$ & $17q^{-2}$ & $7q^{-4}$ &  \\
$0$ & $7q^{2}$ & $18$ & $12q^{-2}$ & $q^{-4}$ \\
$-2$ & $4q^{4}$ & $17q^{2}$ & $16$ & $3q^{-2}$ \\
$-4$ & $q^{6}$ & $9q^{4}$ & $12q^{2}$ & $4$ \\
$-6$ &  & $4q^{6}$ & $7q^{4}$ & $3q^{2}$ \\
$-8$ &  & $q^{8}$ & $2q^{6}$ & $q^{4}$ \\
\end{tabular}
\vspace{2em}
\end{minipage}
%
\begin{minipage}{\linewidth}
$\bullet\ $ $11a_{174}$ \vspace{0.5em} \\
\begin{tabular}{l|llll}
$k \setminus j$ & $-4$ & $-2$ & $0$ & $2$ \\
\hline
$8$ & $q^{-6}$ & $q^{-8}$ &  &  \\
$6$ & $2q^{-4}$ & $2q^{-6}$ &  &  \\
$4$ & $3q^{-2}$ & $5q^{-4}$ & $2q^{-6}$ &  \\
$2$ & $3$ & $6q^{-2}$ & $3q^{-4}$ &  \\
$0$ & $3q^{2}$ & $7$ & $5q^{-2}$ & $q^{-4}$ \\
$-2$ & $2q^{4}$ & $6q^{2}$ & $4$ & $q^{-2}$ \\
$-4$ & $q^{6}$ & $5q^{4}$ & $5q^{2}$ & $1$ \\
$-6$ &  & $2q^{6}$ & $3q^{4}$ & $q^{2}$ \\
$-8$ &  & $q^{8}$ & $2q^{6}$ & $q^{4}$ \\
\end{tabular}
\vspace{2em}
\end{minipage}
%
\begin{minipage}{\linewidth}
$\bullet\ $ $11a_{175}$ \vspace{0.5em} \\
\begin{tabular}{l|llll}
$k \setminus j$ & $-2$ & $0$ & $2$ & $4$ \\
\hline
$8$ & $q^{-6}$ & $q^{-8}$ &  &  \\
$6$ & $2q^{-4}$ & $2q^{-6}$ &  &  \\
$4$ & $4q^{-2}$ & $6q^{-4}$ & $2q^{-6}$ &  \\
$2$ & $4$ & $8q^{-2}$ & $4q^{-4}$ &  \\
$0$ & $4q^{2}$ & $11$ & $7q^{-2}$ & $q^{-4}$ \\
$-2$ & $2q^{4}$ & $8q^{2}$ & $8$ & $2q^{-2}$ \\
$-4$ & $q^{6}$ & $6q^{4}$ & $7q^{2}$ & $2$ \\
$-6$ &  & $2q^{6}$ & $4q^{4}$ & $2q^{2}$ \\
$-8$ &  & $q^{8}$ & $2q^{6}$ & $q^{4}$ \\
\end{tabular}
\vspace{2em}
\end{minipage}
%
\begin{minipage}{\linewidth}
$\bullet\ $ $11a_{176}$ \vspace{0.5em} \\
\begin{tabular}{l|llll}
$k \setminus j$ & $0$ & $2$ & $4$ & $6$ \\
\hline
$8$ & $q^{-6}$ & $q^{-8}$ &  &  \\
$6$ & $2q^{-4}$ & $2q^{-6}$ &  &  \\
$4$ & $4q^{-2}$ & $6q^{-4}$ & $2q^{-6}$ &  \\
$2$ & $4$ & $9q^{-2}$ & $4q^{-4}$ &  \\
$0$ & $4q^{2}$ & $11$ & $8q^{-2}$ & $q^{-4}$ \\
$-2$ & $2q^{4}$ & $9q^{2}$ & $9$ & $2q^{-2}$ \\
$-4$ & $q^{6}$ & $6q^{4}$ & $8q^{2}$ & $3$ \\
$-6$ &  & $2q^{6}$ & $4q^{4}$ & $2q^{2}$ \\
$-8$ &  & $q^{8}$ & $2q^{6}$ & $q^{4}$ \\
\end{tabular}
\vspace{2em}
\end{minipage}
%
\begin{minipage}{\linewidth}
$\bullet\ $ $11a_{177}$ \vspace{0.5em} \\
\begin{tabular}{l|llll}
$k \setminus j$ & $2$ & $4$ & $6$ & $8$ \\
\hline
$8$ & $q^{-6}$ & $q^{-8}$ &  &  \\
$6$ & $2q^{-4}$ & $2q^{-6}$ &  &  \\
$4$ & $3q^{-2}$ & $6q^{-4}$ & $2q^{-6}$ &  \\
$2$ & $3$ & $7q^{-2}$ & $4q^{-4}$ &  \\
$0$ & $3q^{2}$ & $9$ & $7q^{-2}$ & $q^{-4}$ \\
$-2$ & $2q^{4}$ & $7q^{2}$ & $7$ & $2q^{-2}$ \\
$-4$ & $q^{6}$ & $6q^{4}$ & $7q^{2}$ & $2$ \\
$-6$ &  & $2q^{6}$ & $4q^{4}$ & $2q^{2}$ \\
$-8$ &  & $q^{8}$ & $2q^{6}$ & $q^{4}$ \\
\end{tabular}
\vspace{2em}
\end{minipage}
%
\begin{minipage}{\linewidth}
$\bullet\ $ $11a_{178}$ \vspace{0.5em} \\
\begin{tabular}{l|lllll}
$k \setminus j$ & $0$ & $2$ & $4$ & $6$ & $8$ \\
\hline
$6$ & $q^{-4}$ & $q^{-6}$ &  &  &  \\
$4$ & $3q^{-2}$ & $4q^{-4}$ & $q^{-6}$ &  &  \\
$2$ & $4$ & $9q^{-2}$ & $4q^{-4}$ &  &  \\
$0$ & $3q^{2}$ & $12$ & $11q^{-2}$ & $2q^{-4}$ &  \\
$-2$ & $q^{4}$ & $9q^{2}$ & $13$ & $5q^{-2}$ &  \\
$-4$ &  & $4q^{4}$ & $11q^{2}$ & $8$ & $q^{-2}$ \\
$-6$ &  & $q^{6}$ & $4q^{4}$ & $5q^{2}$ & $2$ \\
$-8$ &  &  & $q^{6}$ & $2q^{4}$ & $q^{2}$ \\
\end{tabular}
\vspace{2em}
\end{minipage}
%
\begin{minipage}{\linewidth}
$\bullet\ $ $11a_{179}$ \vspace{0.5em} \\
\begin{tabular}{l|llll}
$k \setminus j$ & $0$ & $2$ & $4$ & $6$ \\
\hline
$8$ & $q^{-4}$ & $2q^{-6}$ & $q^{-8}$ &  \\
$6$ &  & $2q^{-4}$ & $2q^{-6}$ &  \\
$4$ & $1$ & $3q^{-2}$ & $4q^{-4}$ & $q^{-6}$ \\
$2$ &  & $2$ & $4q^{-2}$ & $2q^{-4}$ \\
$0$ & $q^{4}$ & $3q^{2}$ & $4$ & $2q^{-2}$ \\
$-2$ &  & $2q^{4}$ & $4q^{2}$ & $2$ \\
$-4$ &  & $2q^{6}$ & $4q^{4}$ & $2q^{2}$ \\
$-6$ &  &  & $2q^{6}$ & $2q^{4}$ \\
$-8$ &  &  & $q^{8}$ & $q^{6}$ \\
\end{tabular}
\vspace{2em}
\end{minipage}
%
\begin{minipage}{\linewidth}
$\bullet\ $ $11a_{180}$ \vspace{0.5em} \\
\begin{tabular}{l|llll}
$k \setminus j$ & $-4$ & $-2$ & $0$ & $2$ \\
\hline
$8$ & $q^{-4}$ & $2q^{-6}$ & $q^{-8}$ &  \\
$6$ & $q^{-2}$ & $3q^{-4}$ & $2q^{-6}$ &  \\
$4$ & $2$ & $6q^{-2}$ & $5q^{-4}$ & $q^{-6}$ \\
$2$ & $q^{2}$ & $6$ & $7q^{-2}$ & $2q^{-4}$ \\
$0$ & $q^{4}$ & $6q^{2}$ & $9$ & $3q^{-2}$ \\
$-2$ &  & $3q^{4}$ & $7q^{2}$ & $4$ \\
$-4$ &  & $2q^{6}$ & $5q^{4}$ & $3q^{2}$ \\
$-6$ &  &  & $2q^{6}$ & $2q^{4}$ \\
$-8$ &  &  & $q^{8}$ & $q^{6}$ \\
\end{tabular}
\vspace{2em}
\end{minipage}
%
\begin{minipage}{\linewidth}
$\bullet\ $ $11a_{182}$ \vspace{0.5em} \\
\begin{tabular}{l|llll}
$k \setminus j$ & $-8$ & $-6$ & $-4$ & $-2$ \\
\hline
$8$ & $q^{-4}$ & $2q^{-6}$ & $q^{-8}$ &  \\
$6$ & $q^{-2}$ & $3q^{-4}$ & $2q^{-6}$ &  \\
$4$ & $1$ & $5q^{-2}$ & $5q^{-4}$ & $q^{-6}$ \\
$2$ & $q^{2}$ & $4$ & $5q^{-2}$ & $2q^{-4}$ \\
$0$ & $q^{4}$ & $5q^{2}$ & $6$ & $2q^{-2}$ \\
$-2$ &  & $3q^{4}$ & $5q^{2}$ & $2$ \\
$-4$ &  & $2q^{6}$ & $5q^{4}$ & $2q^{2}$ \\
$-6$ &  &  & $2q^{6}$ & $2q^{4}$ \\
$-8$ &  &  & $q^{8}$ & $q^{6}$ \\
\end{tabular}
\vspace{2em}
\end{minipage}
%
\begin{minipage}{\linewidth}
$\bullet\ $ $11a_{183}$ \vspace{0.5em} \\
\begin{tabular}{l|lllll}
$k \setminus j$ & $-8$ & $-6$ & $-4$ & $-2$ & $0$ \\
\hline
$8$ & $q^{-2}$ & $2q^{-4}$ & $q^{-6}$ &  &  \\
$6$ & $1$ & $4q^{-2}$ & $4q^{-4}$ & $q^{-6}$ &  \\
$4$ & $q^{2}$ & $7$ & $10q^{-2}$ & $4q^{-4}$ &  \\
$2$ &  & $4q^{2}$ & $12$ & $9q^{-2}$ & $q^{-4}$ \\
$0$ &  & $2q^{4}$ & $10q^{2}$ & $11$ & $3q^{-2}$ \\
$-2$ &  &  & $4q^{4}$ & $9q^{2}$ & $4$ \\
$-4$ &  &  & $q^{6}$ & $4q^{4}$ & $3q^{2}$ \\
$-6$ &  &  &  & $q^{6}$ & $q^{4}$ \\
\end{tabular}
\vspace{2em}
\end{minipage}
%
\begin{minipage}{\linewidth}
$\bullet\ $ $11a_{184}$ \vspace{0.5em} \\
\begin{tabular}{l|llll}
$k \setminus j$ & $-6$ & $-4$ & $-2$ & $0$ \\
\hline
$8$ & $q^{-4}$ & $2q^{-6}$ & $q^{-8}$ &  \\
$6$ & $q^{-2}$ & $3q^{-4}$ & $2q^{-6}$ &  \\
$4$ & $2$ & $6q^{-2}$ & $5q^{-4}$ & $q^{-6}$ \\
$2$ & $q^{2}$ & $6$ & $7q^{-2}$ & $2q^{-4}$ \\
$0$ & $q^{4}$ & $6q^{2}$ & $8$ & $3q^{-2}$ \\
$-2$ &  & $3q^{4}$ & $7q^{2}$ & $3$ \\
$-4$ &  & $2q^{6}$ & $5q^{4}$ & $3q^{2}$ \\
$-6$ &  &  & $2q^{6}$ & $2q^{4}$ \\
$-8$ &  &  & $q^{8}$ & $q^{6}$ \\
\end{tabular}
\vspace{2em}
\end{minipage}
%
\begin{minipage}{\linewidth}
$\bullet\ $ $11a_{185}$ \vspace{0.5em} \\
\begin{tabular}{l|lllll}
$k \setminus j$ & $-6$ & $-4$ & $-2$ & $0$ & $2$ \\
\hline
$8$ & $q^{-2}$ & $2q^{-4}$ & $q^{-6}$ &  &  \\
$6$ & $1$ & $4q^{-2}$ & $4q^{-4}$ & $q^{-6}$ &  \\
$4$ & $q^{2}$ & $6$ & $9q^{-2}$ & $4q^{-4}$ &  \\
$2$ &  & $4q^{2}$ & $11$ & $8q^{-2}$ & $q^{-4}$ \\
$0$ &  & $2q^{4}$ & $9q^{2}$ & $11$ & $3q^{-2}$ \\
$-2$ &  &  & $4q^{4}$ & $8q^{2}$ & $4$ \\
$-4$ &  &  & $q^{6}$ & $4q^{4}$ & $3q^{2}$ \\
$-6$ &  &  &  & $q^{6}$ & $q^{4}$ \\
\end{tabular}
\vspace{2em}
\end{minipage}
%
\begin{minipage}{\linewidth}
$\bullet\ $ $11a_{186}$ \vspace{0.5em} \\
\begin{tabular}{l|llll}
$k \setminus j$ & $6$ & $8$ & $10$ & $12$ \\
\hline
$6$ & $q^{-6}$ &  &  &  \\
$4$ & $2q^{-4}$ & $2q^{-6}$ &  &  \\
$2$ & $4q^{-2}$ & $5q^{-4}$ & $q^{-6}$ &  \\
$0$ & $4$ & $8q^{-2}$ & $4q^{-4}$ &  \\
$-2$ & $4q^{2}$ & $9$ & $6q^{-2}$ & $q^{-4}$ \\
$-4$ & $2q^{4}$ & $8q^{2}$ & $8$ & $2q^{-2}$ \\
$-6$ & $q^{6}$ & $5q^{4}$ & $6q^{2}$ & $2$ \\
$-8$ &  & $2q^{6}$ & $4q^{4}$ & $2q^{2}$ \\
$-10$ &  &  & $q^{6}$ & $q^{4}$ \\
\end{tabular}
\vspace{2em}
\end{minipage}
%
\begin{minipage}{\linewidth}
$\bullet\ $ $11a_{188}$ \vspace{0.5em} \\
\begin{tabular}{l|lllll}
$k \setminus j$ & $-4$ & $-2$ & $0$ & $2$ & $4$ \\
\hline
$8$ & $q^{-2}$ & $2q^{-4}$ & $q^{-6}$ &  &  \\
$6$ &  & $2q^{-2}$ & $3q^{-4}$ & $q^{-6}$ &  \\
$4$ & $q^{2}$ & $3$ & $5q^{-2}$ & $3q^{-4}$ &  \\
$2$ &  & $2q^{2}$ & $5$ & $5q^{-2}$ & $q^{-4}$ \\
$0$ &  & $2q^{4}$ & $5q^{2}$ & $5$ & $2q^{-2}$ \\
$-2$ &  &  & $3q^{4}$ & $5q^{2}$ & $2$ \\
$-4$ &  &  & $q^{6}$ & $3q^{4}$ & $2q^{2}$ \\
$-6$ &  &  &  & $q^{6}$ & $q^{4}$ \\
\end{tabular}
\vspace{2em}
\end{minipage}
%
\begin{minipage}{\linewidth}
$\bullet\ $ $11a_{189}$ \vspace{0.5em} \\
\begin{tabular}{l|llll}
$k \setminus j$ & $-4$ & $-2$ & $0$ & $2$ \\
\hline
$8$ & $q^{-4}$ & $2q^{-6}$ & $q^{-8}$ &  \\
$6$ & $2q^{-2}$ & $5q^{-4}$ & $3q^{-6}$ &  \\
$4$ & $3$ & $10q^{-2}$ & $8q^{-4}$ & $q^{-6}$ \\
$2$ & $2q^{2}$ & $12$ & $13q^{-2}$ & $3q^{-4}$ \\
$0$ & $q^{4}$ & $10q^{2}$ & $16$ & $6q^{-2}$ \\
$-2$ &  & $5q^{4}$ & $13q^{2}$ & $8$ \\
$-4$ &  & $2q^{6}$ & $8q^{4}$ & $6q^{2}$ \\
$-6$ &  &  & $3q^{6}$ & $3q^{4}$ \\
$-8$ &  &  & $q^{8}$ & $q^{6}$ \\
\end{tabular}
\vspace{2em}
\end{minipage}
%
\begin{minipage}{\linewidth}
$\bullet\ $ $11a_{190}$ \vspace{0.5em} \\
\begin{tabular}{l|lllll}
$k \setminus j$ & $-2$ & $0$ & $2$ & $4$ & $6$ \\
\hline
$6$ & $q^{-4}$ & $q^{-6}$ &  &  &  \\
$4$ & $2q^{-2}$ & $3q^{-4}$ & $q^{-6}$ &  &  \\
$2$ & $3$ & $6q^{-2}$ & $3q^{-4}$ &  &  \\
$0$ & $2q^{2}$ & $8$ & $7q^{-2}$ & $2q^{-4}$ &  \\
$-2$ & $q^{4}$ & $6q^{2}$ & $8$ & $3q^{-2}$ &  \\
$-4$ &  & $3q^{4}$ & $7q^{2}$ & $5$ & $q^{-2}$ \\
$-6$ &  & $q^{6}$ & $3q^{4}$ & $3q^{2}$ & $1$ \\
$-8$ &  &  & $q^{6}$ & $2q^{4}$ & $q^{2}$ \\
\end{tabular}
\vspace{2em}
\end{minipage}
%
\begin{minipage}{\linewidth}
$\bullet\ $ $11a_{191}$ \vspace{0.5em} \\
\begin{tabular}{l|llll}
$k \setminus j$ & $6$ & $8$ & $10$ & $12$ \\
\hline
$6$ & $q^{-6}$ &  &  &  \\
$4$ & $2q^{-4}$ & $2q^{-6}$ &  &  \\
$2$ & $3q^{-2}$ & $4q^{-4}$ & $q^{-6}$ &  \\
$0$ & $3$ & $7q^{-2}$ & $4q^{-4}$ &  \\
$-2$ & $3q^{2}$ & $7$ & $5q^{-2}$ & $q^{-4}$ \\
$-4$ & $2q^{4}$ & $7q^{2}$ & $7$ & $2q^{-2}$ \\
$-6$ & $q^{6}$ & $4q^{4}$ & $5q^{2}$ & $2$ \\
$-8$ &  & $2q^{6}$ & $4q^{4}$ & $2q^{2}$ \\
$-10$ &  &  & $q^{6}$ & $q^{4}$ \\
\end{tabular}
\vspace{2em}
\end{minipage}
%
\begin{minipage}{\linewidth}
$\bullet\ $ $11a_{193}$ \vspace{0.5em} \\
\begin{tabular}{l|lllll}
$k \setminus j$ & $-8$ & $-6$ & $-4$ & $-2$ & $0$ \\
\hline
$8$ & $q^{-2}$ & $2q^{-4}$ & $q^{-6}$ &  &  \\
$6$ & $1$ & $4q^{-2}$ & $4q^{-4}$ & $q^{-6}$ &  \\
$4$ & $q^{2}$ & $6$ & $8q^{-2}$ & $3q^{-4}$ &  \\
$2$ &  & $4q^{2}$ & $10$ & $7q^{-2}$ & $q^{-4}$ \\
$0$ &  & $2q^{4}$ & $8q^{2}$ & $8$ & $2q^{-2}$ \\
$-2$ &  &  & $4q^{4}$ & $7q^{2}$ & $2$ \\
$-4$ &  &  & $q^{6}$ & $3q^{4}$ & $2q^{2}$ \\
$-6$ &  &  &  & $q^{6}$ & $q^{4}$ \\
\end{tabular}
\vspace{2em}
\end{minipage}
%
\begin{minipage}{\linewidth}
$\bullet\ $ $11a_{194}$ \vspace{0.5em} \\
\begin{tabular}{l|llll}
$k \setminus j$ & $-8$ & $-6$ & $-4$ & $-2$ \\
\hline
$8$ & $q^{-4}$ & $2q^{-6}$ & $q^{-8}$ &  \\
$6$ & $q^{-2}$ & $3q^{-4}$ & $2q^{-6}$ &  \\
$4$ & $2$ & $7q^{-2}$ & $6q^{-4}$ & $q^{-6}$ \\
$2$ & $q^{2}$ & $6$ & $7q^{-2}$ & $2q^{-4}$ \\
$0$ & $q^{4}$ & $7q^{2}$ & $9$ & $3q^{-2}$ \\
$-2$ &  & $3q^{4}$ & $7q^{2}$ & $4$ \\
$-4$ &  & $2q^{6}$ & $6q^{4}$ & $3q^{2}$ \\
$-6$ &  &  & $2q^{6}$ & $2q^{4}$ \\
$-8$ &  &  & $q^{8}$ & $q^{6}$ \\
\end{tabular}
\vspace{2em}
\end{minipage}
%
\begin{minipage}{\linewidth}
$\bullet\ $ $11a_{195}$ \vspace{0.5em} \\
\begin{tabular}{l|llllll}
$k \setminus j$ & $-8$ & $-6$ & $-4$ & $-2$ & $0$ & $2$ \\
\hline
$8$ & $1$ & $2q^{-2}$ & $q^{-4}$ &  &  &  \\
$6$ &  & $2$ & $3q^{-2}$ & $q^{-4}$ &  &  \\
$4$ &  & $2q^{2}$ & $5$ & $4q^{-2}$ & $q^{-4}$ &  \\
$2$ &  &  & $3q^{2}$ & $6$ & $3q^{-2}$ &  \\
$0$ &  &  & $q^{4}$ & $4q^{2}$ & $5$ & $q^{-2}$ \\
$-2$ &  &  &  & $q^{4}$ & $3q^{2}$ & $2$ \\
$-4$ &  &  &  &  & $q^{4}$ & $q^{2}$ \\
\end{tabular}
\vspace{2em}
\end{minipage}
%
\begin{minipage}{\linewidth}
$\bullet\ $ $11a_{196}$ \vspace{0.5em} \\
\begin{tabular}{l|llll}
$k \setminus j$ & $-6$ & $-4$ & $-2$ & $0$ \\
\hline
$8$ & $q^{-4}$ & $2q^{-6}$ & $q^{-8}$ &  \\
$6$ & $2q^{-2}$ & $5q^{-4}$ & $3q^{-6}$ &  \\
$4$ & $3$ & $10q^{-2}$ & $8q^{-4}$ & $q^{-6}$ \\
$2$ & $2q^{2}$ & $12$ & $13q^{-2}$ & $3q^{-4}$ \\
$0$ & $q^{4}$ & $10q^{2}$ & $15$ & $6q^{-2}$ \\
$-2$ &  & $5q^{4}$ & $13q^{2}$ & $7$ \\
$-4$ &  & $2q^{6}$ & $8q^{4}$ & $6q^{2}$ \\
$-6$ &  &  & $3q^{6}$ & $3q^{4}$ \\
$-8$ &  &  & $q^{8}$ & $q^{6}$ \\
\end{tabular}
\vspace{2em}
\end{minipage}
%
\begin{minipage}{\linewidth}
$\bullet\ $ $11a_{203}$ \vspace{0.5em} \\
\begin{tabular}{l|llll}
$k \setminus j$ & $4$ & $6$ & $8$ & $10$ \\
\hline
$8$ & $q^{-6}$ & $q^{-8}$ &  &  \\
$6$ & $q^{-4}$ & $2q^{-6}$ &  &  \\
$4$ & $2q^{-2}$ & $4q^{-4}$ & $2q^{-6}$ &  \\
$2$ & $1$ & $4q^{-2}$ & $3q^{-4}$ &  \\
$0$ & $2q^{2}$ & $5$ & $4q^{-2}$ & $q^{-4}$ \\
$-2$ & $q^{4}$ & $4q^{2}$ & $4$ & $q^{-2}$ \\
$-4$ & $q^{6}$ & $4q^{4}$ & $4q^{2}$ & $1$ \\
$-6$ &  & $2q^{6}$ & $3q^{4}$ & $q^{2}$ \\
$-8$ &  & $q^{8}$ & $2q^{6}$ & $q^{4}$ \\
\end{tabular}
\vspace{2em}
\end{minipage}
%
\begin{minipage}{\linewidth}
$\bullet\ $ $11a_{204}$ \vspace{0.5em} \\
\begin{tabular}{l|lllll}
$k \setminus j$ & $2$ & $4$ & $6$ & $8$ & $10$ \\
\hline
$6$ & $q^{-4}$ & $q^{-6}$ &  &  &  \\
$4$ & $2q^{-2}$ & $4q^{-4}$ & $q^{-6}$ &  &  \\
$2$ & $3$ & $7q^{-2}$ & $4q^{-4}$ &  &  \\
$0$ & $2q^{2}$ & $9$ & $9q^{-2}$ & $2q^{-4}$ &  \\
$-2$ & $q^{4}$ & $7q^{2}$ & $10$ & $4q^{-2}$ &  \\
$-4$ &  & $4q^{4}$ & $9q^{2}$ & $6$ & $q^{-2}$ \\
$-6$ &  & $q^{6}$ & $4q^{4}$ & $4q^{2}$ & $1$ \\
$-8$ &  &  & $q^{6}$ & $2q^{4}$ & $q^{2}$ \\
\end{tabular}
\vspace{2em}
\end{minipage}
%
\begin{minipage}{\linewidth}
$\bullet\ $ $11a_{205}$ \vspace{0.5em} \\
\begin{tabular}{l|lllll}
$k \setminus j$ & $-6$ & $-4$ & $-2$ & $0$ & $2$ \\
\hline
$6$ & $q^{-2}$ & $2q^{-4}$ & $q^{-6}$ &  &  \\
$4$ & $1$ & $4q^{-2}$ & $4q^{-4}$ & $q^{-6}$ &  \\
$2$ & $q^{2}$ & $6$ & $8q^{-2}$ & $3q^{-4}$ &  \\
$0$ &  & $4q^{2}$ & $9$ & $6q^{-2}$ & $q^{-4}$ \\
$-2$ &  & $2q^{4}$ & $8q^{2}$ & $7$ & $2q^{-2}$ \\
$-4$ &  &  & $4q^{4}$ & $6q^{2}$ & $2$ \\
$-6$ &  &  & $q^{6}$ & $3q^{4}$ & $2q^{2}$ \\
$-8$ &  &  &  & $q^{6}$ & $q^{4}$ \\
\end{tabular}
\vspace{2em}
\end{minipage}
%
\begin{minipage}{\linewidth}
$\bullet\ $ $11a_{206}$ \vspace{0.5em} \\
\begin{tabular}{l|llll}
$k \setminus j$ & $-10$ & $-8$ & $-6$ & $-4$ \\
\hline
$8$ & $q^{-4}$ & $2q^{-6}$ & $q^{-8}$ &  \\
$6$ &  & $2q^{-4}$ & $2q^{-6}$ &  \\
$4$ & $1$ & $3q^{-2}$ & $3q^{-4}$ & $q^{-6}$ \\
$2$ &  & $2$ & $3q^{-2}$ & $q^{-4}$ \\
$0$ & $q^{4}$ & $3q^{2}$ & $3$ & $q^{-2}$ \\
$-2$ &  & $2q^{4}$ & $3q^{2}$ & $1$ \\
$-4$ &  & $2q^{6}$ & $3q^{4}$ & $q^{2}$ \\
$-6$ &  &  & $2q^{6}$ & $q^{4}$ \\
$-8$ &  &  & $q^{8}$ & $q^{6}$ \\
\end{tabular}
\vspace{2em}
\end{minipage}
%
\begin{minipage}{\linewidth}
$\bullet\ $ $11a_{207}$ \vspace{0.5em} \\
\begin{tabular}{l|lllll}
$k \setminus j$ & $-10$ & $-8$ & $-6$ & $-4$ & $-2$ \\
\hline
$8$ & $q^{-2}$ & $2q^{-4}$ & $q^{-6}$ &  &  \\
$6$ &  & $3q^{-2}$ & $4q^{-4}$ & $q^{-6}$ &  \\
$4$ & $q^{2}$ & $4$ & $7q^{-2}$ & $4q^{-4}$ &  \\
$2$ &  & $3q^{2}$ & $8$ & $6q^{-2}$ & $q^{-4}$ \\
$0$ &  & $2q^{4}$ & $7q^{2}$ & $7$ & $2q^{-2}$ \\
$-2$ &  &  & $4q^{4}$ & $6q^{2}$ & $2$ \\
$-4$ &  &  & $q^{6}$ & $4q^{4}$ & $2q^{2}$ \\
$-6$ &  &  &  & $q^{6}$ & $q^{4}$ \\
\end{tabular}
\vspace{2em}
\end{minipage}
%
\begin{minipage}{\linewidth}
$\bullet\ $ $11a_{208}$ \vspace{0.5em} \\
\begin{tabular}{l|lllll}
$k \setminus j$ & $-10$ & $-8$ & $-6$ & $-4$ & $-2$ \\
\hline
$8$ & $q^{-2}$ & $2q^{-4}$ & $q^{-6}$ &  &  \\
$6$ & $1$ & $5q^{-2}$ & $5q^{-4}$ & $q^{-6}$ &  \\
$4$ & $q^{2}$ & $6$ & $9q^{-2}$ & $4q^{-4}$ &  \\
$2$ &  & $5q^{2}$ & $11$ & $7q^{-2}$ & $q^{-4}$ \\
$0$ &  & $2q^{4}$ & $9q^{2}$ & $9$ & $2q^{-2}$ \\
$-2$ &  &  & $5q^{4}$ & $7q^{2}$ & $2$ \\
$-4$ &  &  & $q^{6}$ & $4q^{4}$ & $2q^{2}$ \\
$-6$ &  &  &  & $q^{6}$ & $q^{4}$ \\
\end{tabular}
\vspace{2em}
\end{minipage}
%
\begin{minipage}{\linewidth}
$\bullet\ $ $11a_{210}$ \vspace{0.5em} \\
\begin{tabular}{l|llllll}
$k \setminus j$ & $-6$ & $-4$ & $-2$ & $0$ & $2$ & $4$ \\
\hline
$6$ & $1$ & $2q^{-2}$ & $q^{-4}$ &  &  &  \\
$4$ &  & $3$ & $5q^{-2}$ & $2q^{-4}$ &  &  \\
$2$ &  & $2q^{2}$ & $7$ & $6q^{-2}$ & $q^{-4}$ &  \\
$0$ &  &  & $5q^{2}$ & $10$ & $4q^{-2}$ &  \\
$-2$ &  &  & $q^{4}$ & $6q^{2}$ & $6$ & $q^{-2}$ \\
$-4$ &  &  &  & $2q^{4}$ & $4q^{2}$ & $2$ \\
$-6$ &  &  &  &  & $q^{4}$ & $q^{2}$ \\
\end{tabular}
\vspace{2em}
\end{minipage}
%
\begin{minipage}{\linewidth}
$\bullet\ $ $11a_{211}$ \vspace{0.5em} \\
\begin{tabular}{l|llllll}
$k \setminus j$ & $-10$ & $-8$ & $-6$ & $-4$ & $-2$ & $0$ \\
\hline
$8$ & $1$ & $2q^{-2}$ & $q^{-4}$ &  &  &  \\
$6$ &  & $3$ & $4q^{-2}$ & $q^{-4}$ &  &  \\
$4$ &  & $2q^{2}$ & $7$ & $6q^{-2}$ & $q^{-4}$ &  \\
$2$ &  &  & $4q^{2}$ & $8$ & $4q^{-2}$ &  \\
$0$ &  &  & $q^{4}$ & $6q^{2}$ & $6$ & $q^{-2}$ \\
$-2$ &  &  &  & $q^{4}$ & $4q^{2}$ & $2$ \\
$-4$ &  &  &  &  & $q^{4}$ & $q^{2}$ \\
\end{tabular}
\vspace{2em}
\end{minipage}
%
\begin{minipage}{\linewidth}
$\bullet\ $ $11a_{215}$ \vspace{0.5em} \\
\begin{tabular}{l|llll}
$k \setminus j$ & $2$ & $4$ & $6$ & $8$ \\
\hline
$8$ & $q^{-6}$ & $q^{-8}$ &  &  \\
$6$ & $3q^{-4}$ & $3q^{-6}$ &  &  \\
$4$ & $5q^{-2}$ & $8q^{-4}$ & $2q^{-6}$ &  \\
$2$ & $6$ & $11q^{-2}$ & $5q^{-4}$ &  \\
$0$ & $5q^{2}$ & $13$ & $9q^{-2}$ & $q^{-4}$ \\
$-2$ & $3q^{4}$ & $11q^{2}$ & $10$ & $2q^{-2}$ \\
$-4$ & $q^{6}$ & $8q^{4}$ & $9q^{2}$ & $2$ \\
$-6$ &  & $3q^{6}$ & $5q^{4}$ & $2q^{2}$ \\
$-8$ &  & $q^{8}$ & $2q^{6}$ & $q^{4}$ \\
\end{tabular}
\vspace{2em}
\end{minipage}
%
\begin{minipage}{\linewidth}
$\bullet\ $ $11a_{216}$ \vspace{0.5em} \\
\begin{tabular}{l|llll}
$k \setminus j$ & $0$ & $2$ & $4$ & $6$ \\
\hline
$8$ & $q^{-6}$ & $q^{-8}$ &  &  \\
$6$ & $3q^{-4}$ & $3q^{-6}$ &  &  \\
$4$ & $6q^{-2}$ & $8q^{-4}$ & $2q^{-6}$ &  \\
$2$ & $7$ & $13q^{-2}$ & $5q^{-4}$ &  \\
$0$ & $6q^{2}$ & $15$ & $10q^{-2}$ & $q^{-4}$ \\
$-2$ & $3q^{4}$ & $13q^{2}$ & $12$ & $2q^{-2}$ \\
$-4$ & $q^{6}$ & $8q^{4}$ & $10q^{2}$ & $3$ \\
$-6$ &  & $3q^{6}$ & $5q^{4}$ & $2q^{2}$ \\
$-8$ &  & $q^{8}$ & $2q^{6}$ & $q^{4}$ \\
\end{tabular}
\vspace{2em}
\end{minipage}
%
\begin{minipage}{\linewidth}
$\bullet\ $ $11a_{217}$ \vspace{0.5em} \\
\begin{tabular}{l|llll}
$k \setminus j$ & $-6$ & $-4$ & $-2$ & $0$ \\
\hline
$8$ & $q^{-4}$ & $2q^{-6}$ & $q^{-8}$ &  \\
$6$ & $q^{-2}$ & $4q^{-4}$ & $3q^{-6}$ &  \\
$4$ & $3$ & $10q^{-2}$ & $8q^{-4}$ & $q^{-6}$ \\
$2$ & $q^{2}$ & $10$ & $12q^{-2}$ & $3q^{-4}$ \\
$0$ & $q^{4}$ & $10q^{2}$ & $15$ & $6q^{-2}$ \\
$-2$ &  & $4q^{4}$ & $12q^{2}$ & $7$ \\
$-4$ &  & $2q^{6}$ & $8q^{4}$ & $6q^{2}$ \\
$-6$ &  &  & $3q^{6}$ & $3q^{4}$ \\
$-8$ &  &  & $q^{8}$ & $q^{6}$ \\
\end{tabular}
\vspace{2em}
\end{minipage}
%
\begin{minipage}{\linewidth}
$\bullet\ $ $11a_{218}$ \vspace{0.5em} \\
\begin{tabular}{l|llllll}
$k \setminus j$ & $-8$ & $-6$ & $-4$ & $-2$ & $0$ & $2$ \\
\hline
$6$ & $1$ & $3q^{-2}$ & $3q^{-4}$ & $q^{-6}$ &  &  \\
$4$ &  & $4$ & $9q^{-2}$ & $5q^{-4}$ &  &  \\
$2$ &  & $3q^{2}$ & $13$ & $12q^{-2}$ & $2q^{-4}$ &  \\
$0$ &  &  & $9q^{2}$ & $16$ & $7q^{-2}$ &  \\
$-2$ &  &  & $3q^{4}$ & $12q^{2}$ & $9$ & $q^{-2}$ \\
$-4$ &  &  &  & $5q^{4}$ & $7q^{2}$ & $2$ \\
$-6$ &  &  &  & $q^{6}$ & $2q^{4}$ & $q^{2}$ \\
\end{tabular}
\vspace{2em}
\end{minipage}
%
\begin{minipage}{\linewidth}
$\bullet\ $ $11a_{220}$ \vspace{0.5em} \\
\begin{tabular}{l|lllll}
$k \setminus j$ & $2$ & $4$ & $6$ & $8$ & $10$ \\
\hline
$6$ & $q^{-4}$ & $q^{-6}$ &  &  &  \\
$4$ & $q^{-2}$ & $3q^{-4}$ & $q^{-6}$ &  &  \\
$2$ & $2$ & $6q^{-2}$ & $4q^{-4}$ &  &  \\
$0$ & $q^{2}$ & $6$ & $7q^{-2}$ & $2q^{-4}$ &  \\
$-2$ & $q^{4}$ & $6q^{2}$ & $9$ & $4q^{-2}$ &  \\
$-4$ &  & $3q^{4}$ & $7q^{2}$ & $5$ & $q^{-2}$ \\
$-6$ &  & $q^{6}$ & $4q^{4}$ & $4q^{2}$ & $1$ \\
$-8$ &  &  & $q^{6}$ & $2q^{4}$ & $q^{2}$ \\
\end{tabular}
\vspace{2em}
\end{minipage}
%
\begin{minipage}{\linewidth}
$\bullet\ $ $11a_{221}$ \vspace{0.5em} \\
\begin{tabular}{l|llll}
$k \setminus j$ & $-6$ & $-4$ & $-2$ & $0$ \\
\hline
$8$ & $q^{-4}$ & $2q^{-6}$ & $q^{-8}$ &  \\
$6$ &  & $2q^{-4}$ & $2q^{-6}$ &  \\
$4$ & $2$ & $6q^{-2}$ & $5q^{-4}$ & $q^{-6}$ \\
$2$ &  & $4$ & $6q^{-2}$ & $2q^{-4}$ \\
$0$ & $q^{4}$ & $6q^{2}$ & $8$ & $3q^{-2}$ \\
$-2$ &  & $2q^{4}$ & $6q^{2}$ & $3$ \\
$-4$ &  & $2q^{6}$ & $5q^{4}$ & $3q^{2}$ \\
$-6$ &  &  & $2q^{6}$ & $2q^{4}$ \\
$-8$ &  &  & $q^{8}$ & $q^{6}$ \\
\end{tabular}
\vspace{2em}
\end{minipage}
%
\begin{minipage}{\linewidth}
$\bullet\ $ $11a_{222}$ \vspace{0.5em} \\
\begin{tabular}{l|lllll}
$k \setminus j$ & $-6$ & $-4$ & $-2$ & $0$ & $2$ \\
\hline
$8$ & $q^{-2}$ & $2q^{-4}$ & $q^{-6}$ &  &  \\
$6$ &  & $3q^{-2}$ & $4q^{-4}$ & $q^{-6}$ &  \\
$4$ & $q^{2}$ & $5$ & $8q^{-2}$ & $4q^{-4}$ &  \\
$2$ &  & $3q^{2}$ & $10$ & $8q^{-2}$ & $q^{-4}$ \\
$0$ &  & $2q^{4}$ & $8q^{2}$ & $10$ & $3q^{-2}$ \\
$-2$ &  &  & $4q^{4}$ & $8q^{2}$ & $4$ \\
$-4$ &  &  & $q^{6}$ & $4q^{4}$ & $3q^{2}$ \\
$-6$ &  &  &  & $q^{6}$ & $q^{4}$ \\
\end{tabular}
\vspace{2em}
\end{minipage}
%
\begin{minipage}{\linewidth}
$\bullet\ $ $11a_{223}$ \vspace{0.5em} \\
\begin{tabular}{l|llll}
$k \setminus j$ & $4$ & $6$ & $8$ & $10$ \\
\hline
$8$ & $q^{-6}$ & $q^{-8}$ &  &  \\
$6$ & $q^{-4}$ & $2q^{-6}$ &  &  \\
$4$ & $3q^{-2}$ & $5q^{-4}$ & $2q^{-6}$ &  \\
$2$ & $2$ & $5q^{-2}$ & $3q^{-4}$ &  \\
$0$ & $3q^{2}$ & $7$ & $5q^{-2}$ & $q^{-4}$ \\
$-2$ & $q^{4}$ & $5q^{2}$ & $5$ & $q^{-2}$ \\
$-4$ & $q^{6}$ & $5q^{4}$ & $5q^{2}$ & $1$ \\
$-6$ &  & $2q^{6}$ & $3q^{4}$ & $q^{2}$ \\
$-8$ &  & $q^{8}$ & $2q^{6}$ & $q^{4}$ \\
\end{tabular}
\vspace{2em}
\end{minipage}
%
\begin{minipage}{\linewidth}
$\bullet\ $ $11a_{224}$ \vspace{0.5em} \\
\begin{tabular}{l|lllll}
$k \setminus j$ & $2$ & $4$ & $6$ & $8$ & $10$ \\
\hline
$6$ & $q^{-4}$ & $q^{-6}$ &  &  &  \\
$4$ & $q^{-2}$ & $3q^{-4}$ & $q^{-6}$ &  &  \\
$2$ & $2$ & $6q^{-2}$ & $4q^{-4}$ &  &  \\
$0$ & $q^{2}$ & $7$ & $8q^{-2}$ & $2q^{-4}$ &  \\
$-2$ & $q^{4}$ & $6q^{2}$ & $9$ & $4q^{-2}$ &  \\
$-4$ &  & $3q^{4}$ & $8q^{2}$ & $6$ & $q^{-2}$ \\
$-6$ &  & $q^{6}$ & $4q^{4}$ & $4q^{2}$ & $1$ \\
$-8$ &  &  & $q^{6}$ & $2q^{4}$ & $q^{2}$ \\
\end{tabular}
\vspace{2em}
\end{minipage}
%
\begin{minipage}{\linewidth}
$\bullet\ $ $11a_{225}$ \vspace{0.5em} \\
\begin{tabular}{l|lllll}
$k \setminus j$ & $-8$ & $-6$ & $-4$ & $-2$ & $0$ \\
\hline
$6$ & $q^{-2}$ & $2q^{-4}$ & $q^{-6}$ &  &  \\
$4$ &  & $2q^{-2}$ & $3q^{-4}$ & $q^{-6}$ &  \\
$2$ & $q^{2}$ & $3$ & $4q^{-2}$ & $2q^{-4}$ &  \\
$0$ &  & $2q^{2}$ & $4$ & $3q^{-2}$ & $q^{-4}$ \\
$-2$ &  & $2q^{4}$ & $4q^{2}$ & $3$ & $q^{-2}$ \\
$-4$ &  &  & $3q^{4}$ & $3q^{2}$ & $1$ \\
$-6$ &  &  & $q^{6}$ & $2q^{4}$ & $q^{2}$ \\
$-8$ &  &  &  & $q^{6}$ & $q^{4}$ \\
\end{tabular}
\vspace{2em}
\end{minipage}
%
\begin{minipage}{\linewidth}
$\bullet\ $ $11a_{226}$ \vspace{0.5em} \\
\begin{tabular}{l|llllll}
$k \setminus j$ & $-8$ & $-6$ & $-4$ & $-2$ & $0$ & $2$ \\
\hline
$6$ & $1$ & $2q^{-2}$ & $q^{-4}$ &  &  &  \\
$4$ &  & $3$ & $5q^{-2}$ & $2q^{-4}$ &  &  \\
$2$ &  & $2q^{2}$ & $7$ & $6q^{-2}$ & $q^{-4}$ &  \\
$0$ &  &  & $5q^{2}$ & $9$ & $4q^{-2}$ &  \\
$-2$ &  &  & $q^{4}$ & $6q^{2}$ & $5$ & $q^{-2}$ \\
$-4$ &  &  &  & $2q^{4}$ & $4q^{2}$ & $2$ \\
$-6$ &  &  &  &  & $q^{4}$ & $q^{2}$ \\
\end{tabular}
\vspace{2em}
\end{minipage}
%
\begin{minipage}{\linewidth}
$\bullet\ $ $11a_{227}$ \vspace{0.5em} \\
\begin{tabular}{l|llll}
$k \setminus j$ & $6$ & $8$ & $10$ & $12$ \\
\hline
$6$ & $q^{-6}$ &  &  &  \\
$4$ & $3q^{-4}$ & $3q^{-6}$ &  &  \\
$2$ & $6q^{-2}$ & $7q^{-4}$ & $q^{-6}$ &  \\
$0$ & $7$ & $13q^{-2}$ & $6q^{-4}$ &  \\
$-2$ & $6q^{2}$ & $14$ & $9q^{-2}$ & $q^{-4}$ \\
$-4$ & $3q^{4}$ & $13q^{2}$ & $13$ & $3q^{-2}$ \\
$-6$ & $q^{6}$ & $7q^{4}$ & $9q^{2}$ & $3$ \\
$-8$ &  & $3q^{6}$ & $6q^{4}$ & $3q^{2}$ \\
$-10$ &  &  & $q^{6}$ & $q^{4}$ \\
\end{tabular}
\vspace{2em}
\end{minipage}
%
\begin{minipage}{\linewidth}
$\bullet\ $ $11a_{228}$ \vspace{0.5em} \\
\begin{tabular}{l|llllll}
$k \setminus j$ & $-6$ & $-4$ & $-2$ & $0$ & $2$ & $4$ \\
\hline
$6$ & $1$ & $3q^{-2}$ & $3q^{-4}$ & $q^{-6}$ &  &  \\
$4$ &  & $4$ & $9q^{-2}$ & $5q^{-4}$ &  &  \\
$2$ &  & $3q^{2}$ & $13$ & $12q^{-2}$ & $2q^{-4}$ &  \\
$0$ &  &  & $9q^{2}$ & $17$ & $7q^{-2}$ &  \\
$-2$ &  &  & $3q^{4}$ & $12q^{2}$ & $10$ & $q^{-2}$ \\
$-4$ &  &  &  & $5q^{4}$ & $7q^{2}$ & $2$ \\
$-6$ &  &  &  & $q^{6}$ & $2q^{4}$ & $q^{2}$ \\
\end{tabular}
\vspace{2em}
\end{minipage}
%
\begin{minipage}{\linewidth}
$\bullet\ $ $11a_{229}$ \vspace{0.5em} \\
\begin{tabular}{l|llllll}
$k \setminus j$ & $0$ & $2$ & $4$ & $6$ & $8$ & $10$ \\
\hline
$4$ & $q^{-2}$ & $q^{-4}$ &  &  &  &  \\
$2$ & $1$ & $4q^{-2}$ & $2q^{-4}$ &  &  &  \\
$0$ & $q^{2}$ & $6$ & $6q^{-2}$ & $q^{-4}$ &  &  \\
$-2$ &  & $4q^{2}$ & $9$ & $5q^{-2}$ &  &  \\
$-4$ &  & $q^{4}$ & $6q^{2}$ & $7$ & $2q^{-2}$ &  \\
$-6$ &  &  & $2q^{4}$ & $5q^{2}$ & $3$ &  \\
$-8$ &  &  &  & $q^{4}$ & $2q^{2}$ & $1$ \\
\end{tabular}
\vspace{2em}
\end{minipage}
%
\begin{minipage}{\linewidth}
$\bullet\ $ $11a_{230}$ \vspace{0.5em} \\
\begin{tabular}{l|llllll}
$k \setminus j$ & $0$ & $2$ & $4$ & $6$ & $8$ & $10$ \\
\hline
$4$ & $q^{-2}$ & $q^{-4}$ &  &  &  &  \\
$2$ & $1$ & $3q^{-2}$ & $q^{-4}$ &  &  &  \\
$0$ & $q^{2}$ & $4$ & $4q^{-2}$ & $q^{-4}$ &  &  \\
$-2$ &  & $3q^{2}$ & $6$ & $3q^{-2}$ &  &  \\
$-4$ &  & $q^{4}$ & $4q^{2}$ & $5$ & $2q^{-2}$ &  \\
$-6$ &  &  & $q^{4}$ & $3q^{2}$ & $2$ &  \\
$-8$ &  &  &  & $q^{4}$ & $2q^{2}$ & $1$ \\
\end{tabular}
\vspace{2em}
\end{minipage}
%
\begin{minipage}{\linewidth}
$\bullet\ $ $11a_{231}$ \vspace{0.5em} \\
\begin{tabular}{l|llll}
$k \setminus j$ & $-4$ & $-2$ & $0$ & $2$ \\
\hline
$8$ & $q^{-6}$ & $q^{-8}$ &  &  \\
$6$ & $2q^{-4}$ & $2q^{-6}$ &  &  \\
$4$ & $5q^{-2}$ & $7q^{-4}$ & $2q^{-6}$ &  \\
$2$ & $5$ & $7q^{-2}$ & $2q^{-4}$ &  \\
$0$ & $5q^{2}$ & $12$ & $8q^{-2}$ & $q^{-4}$ \\
$-2$ & $2q^{4}$ & $7q^{2}$ & $4$ &  \\
$-4$ & $q^{6}$ & $7q^{4}$ & $8q^{2}$ & $2$ \\
$-6$ &  & $2q^{6}$ & $2q^{4}$ &  \\
$-8$ &  & $q^{8}$ & $2q^{6}$ & $q^{4}$ \\
\end{tabular}
\vspace{2em}
\end{minipage}
%
\begin{minipage}{\linewidth}
$\bullet\ $ $11a_{232}$ \vspace{0.5em} \\
\begin{tabular}{l|llll}
$k \setminus j$ & $-4$ & $-2$ & $0$ & $2$ \\
\hline
$8$ & $q^{-6}$ & $q^{-8}$ &  &  \\
$6$ & $2q^{-4}$ & $2q^{-6}$ &  &  \\
$4$ & $5q^{-2}$ & $7q^{-4}$ & $2q^{-6}$ &  \\
$2$ & $6$ & $10q^{-2}$ & $4q^{-4}$ &  \\
$0$ & $5q^{2}$ & $13$ & $9q^{-2}$ & $q^{-4}$ \\
$-2$ & $2q^{4}$ & $10q^{2}$ & $9$ & $2q^{-2}$ \\
$-4$ & $q^{6}$ & $7q^{4}$ & $9q^{2}$ & $3$ \\
$-6$ &  & $2q^{6}$ & $4q^{4}$ & $2q^{2}$ \\
$-8$ &  & $q^{8}$ & $2q^{6}$ & $q^{4}$ \\
\end{tabular}
\vspace{2em}
\end{minipage}
%
\begin{minipage}{\linewidth}
$\bullet\ $ $11a_{233}$ \vspace{0.5em} \\
\begin{tabular}{l|llll}
$k \setminus j$ & $-2$ & $0$ & $2$ & $4$ \\
\hline
$8$ & $q^{-6}$ & $q^{-8}$ &  &  \\
$6$ & $3q^{-4}$ & $3q^{-6}$ &  &  \\
$4$ & $7q^{-2}$ & $9q^{-4}$ & $2q^{-6}$ &  \\
$2$ & $9$ & $15q^{-2}$ & $6q^{-4}$ &  \\
$0$ & $7q^{2}$ & $19$ & $12q^{-2}$ & $q^{-4}$ \\
$-2$ & $3q^{4}$ & $15q^{2}$ & $15$ & $3q^{-2}$ \\
$-4$ & $q^{6}$ & $9q^{4}$ & $12q^{2}$ & $4$ \\
$-6$ &  & $3q^{6}$ & $6q^{4}$ & $3q^{2}$ \\
$-8$ &  & $q^{8}$ & $2q^{6}$ & $q^{4}$ \\
\end{tabular}
\vspace{2em}
\end{minipage}
%
\begin{minipage}{\linewidth}
$\bullet\ $ $11a_{234}$ \vspace{0.5em} \\
\begin{tabular}{l|lll}
$k \setminus j$ & $8$ & $10$ & $12$ \\
\hline
$8$ & $q^{-8}$ &  &  \\
$6$ & $q^{-6}$ & $q^{-8}$ &  \\
$4$ & $2q^{-4}$ & $2q^{-6}$ &  \\
$2$ & $q^{-2}$ & $2q^{-4}$ & $q^{-6}$ \\
$0$ & $2$ & $3q^{-2}$ & $q^{-4}$ \\
$-2$ & $q^{2}$ & $2$ & $q^{-2}$ \\
$-4$ & $2q^{4}$ & $3q^{2}$ & $1$ \\
$-6$ & $q^{6}$ & $2q^{4}$ & $q^{2}$ \\
$-8$ & $q^{8}$ & $2q^{6}$ & $q^{4}$ \\
$-10$ &  & $q^{8}$ & $q^{6}$ \\
\end{tabular}
\vspace{2em}
\end{minipage}
%
\begin{minipage}{\linewidth}
$\bullet\ $ $11a_{235}$ \vspace{0.5em} \\
\begin{tabular}{l|llll}
$k \setminus j$ & $6$ & $8$ & $10$ & $12$ \\
\hline
$6$ & $q^{-6}$ &  &  &  \\
$4$ & $2q^{-4}$ & $2q^{-6}$ &  &  \\
$2$ & $3q^{-2}$ & $4q^{-4}$ & $q^{-6}$ &  \\
$0$ & $3$ & $6q^{-2}$ & $3q^{-4}$ &  \\
$-2$ & $3q^{2}$ & $6$ & $4q^{-2}$ & $q^{-4}$ \\
$-4$ & $2q^{4}$ & $6q^{2}$ & $5$ & $q^{-2}$ \\
$-6$ & $q^{6}$ & $4q^{4}$ & $4q^{2}$ & $1$ \\
$-8$ &  & $2q^{6}$ & $3q^{4}$ & $q^{2}$ \\
$-10$ &  &  & $q^{6}$ & $q^{4}$ \\
\end{tabular}
\vspace{2em}
\end{minipage}
%
\begin{minipage}{\linewidth}
$\bullet\ $ $11a_{236}$ \vspace{0.5em} \\
\begin{tabular}{l|llll}
$k \setminus j$ & $6$ & $8$ & $10$ & $12$ \\
\hline
$6$ & $q^{-6}$ &  &  &  \\
$4$ & $2q^{-4}$ & $2q^{-6}$ &  &  \\
$2$ & $4q^{-2}$ & $5q^{-4}$ & $q^{-6}$ &  \\
$0$ & $4$ & $8q^{-2}$ & $4q^{-4}$ &  \\
$-2$ & $4q^{2}$ & $10$ & $7q^{-2}$ & $q^{-4}$ \\
$-4$ & $2q^{4}$ & $8q^{2}$ & $8$ & $2q^{-2}$ \\
$-6$ & $q^{6}$ & $5q^{4}$ & $7q^{2}$ & $3$ \\
$-8$ &  & $2q^{6}$ & $4q^{4}$ & $2q^{2}$ \\
$-10$ &  &  & $q^{6}$ & $q^{4}$ \\
\end{tabular}
\vspace{2em}
\end{minipage}
%
\begin{minipage}{\linewidth}
$\bullet\ $ $11a_{239}$ \vspace{0.5em} \\
\begin{tabular}{l|llll}
$k \setminus j$ & $0$ & $2$ & $4$ & $6$ \\
\hline
$8$ & $q^{-6}$ & $q^{-8}$ &  &  \\
$6$ & $4q^{-4}$ & $4q^{-6}$ &  &  \\
$4$ & $8q^{-2}$ & $10q^{-4}$ & $2q^{-6}$ &  \\
$2$ & $10$ & $18q^{-2}$ & $7q^{-4}$ &  \\
$0$ & $8q^{2}$ & $20$ & $13q^{-2}$ & $q^{-4}$ \\
$-2$ & $4q^{4}$ & $18q^{2}$ & $17$ & $3q^{-2}$ \\
$-4$ & $q^{6}$ & $10q^{4}$ & $13q^{2}$ & $4$ \\
$-6$ &  & $4q^{6}$ & $7q^{4}$ & $3q^{2}$ \\
$-8$ &  & $q^{8}$ & $2q^{6}$ & $q^{4}$ \\
\end{tabular}
\vspace{2em}
\end{minipage}
%
\begin{minipage}{\linewidth}
$\bullet\ $ $11a_{240}$ \vspace{0.5em} \\
\begin{tabular}{l|lll}
$k \setminus j$ & $8$ & $10$ & $12$ \\
\hline
$8$ & $q^{-8}$ &  &  \\
$6$ & $q^{-6}$ & $q^{-8}$ &  \\
$4$ & $3q^{-4}$ & $3q^{-6}$ &  \\
$2$ & $2q^{-2}$ & $3q^{-4}$ & $q^{-6}$ \\
$0$ & $4$ & $6q^{-2}$ & $2q^{-4}$ \\
$-2$ & $2q^{2}$ & $4$ & $2q^{-2}$ \\
$-4$ & $3q^{4}$ & $6q^{2}$ & $3$ \\
$-6$ & $q^{6}$ & $3q^{4}$ & $2q^{2}$ \\
$-8$ & $q^{8}$ & $3q^{6}$ & $2q^{4}$ \\
$-10$ &  & $q^{8}$ & $q^{6}$ \\
\end{tabular}
\vspace{2em}
\end{minipage}
%
\begin{minipage}{\linewidth}
$\bullet\ $ $11a_{241}$ \vspace{0.5em} \\
\begin{tabular}{l|llll}
$k \setminus j$ & $6$ & $8$ & $10$ & $12$ \\
\hline
$6$ & $q^{-6}$ &  &  &  \\
$4$ & $2q^{-4}$ & $2q^{-6}$ &  &  \\
$2$ & $4q^{-2}$ & $5q^{-4}$ & $q^{-6}$ &  \\
$0$ & $4$ & $8q^{-2}$ & $4q^{-4}$ &  \\
$-2$ & $4q^{2}$ & $9$ & $6q^{-2}$ & $q^{-4}$ \\
$-4$ & $2q^{4}$ & $8q^{2}$ & $8$ & $2q^{-2}$ \\
$-6$ & $q^{6}$ & $5q^{4}$ & $6q^{2}$ & $2$ \\
$-8$ &  & $2q^{6}$ & $4q^{4}$ & $2q^{2}$ \\
$-10$ &  &  & $q^{6}$ & $q^{4}$ \\
\end{tabular}
\vspace{2em}
\end{minipage}
%
\begin{minipage}{\linewidth}
$\bullet\ $ $11a_{242}$ \vspace{0.5em} \\
\begin{tabular}{l|llll}
$k \setminus j$ & $6$ & $8$ & $10$ & $12$ \\
\hline
$6$ & $q^{-6}$ &  &  &  \\
$4$ & $q^{-4}$ & $q^{-6}$ &  &  \\
$2$ & $2q^{-2}$ & $3q^{-4}$ & $q^{-6}$ &  \\
$0$ & $1$ & $3q^{-2}$ & $2q^{-4}$ &  \\
$-2$ & $2q^{2}$ & $4$ & $3q^{-2}$ & $q^{-4}$ \\
$-4$ & $q^{4}$ & $3q^{2}$ & $3$ & $q^{-2}$ \\
$-6$ & $q^{6}$ & $3q^{4}$ & $3q^{2}$ & $1$ \\
$-8$ &  & $q^{6}$ & $2q^{4}$ & $q^{2}$ \\
$-10$ &  &  & $q^{6}$ & $q^{4}$ \\
\end{tabular}
\vspace{2em}
\end{minipage}
%
\begin{minipage}{\linewidth}
$\bullet\ $ $11a_{244}$ \vspace{0.5em} \\
\begin{tabular}{l|llll}
$k \setminus j$ & $6$ & $8$ & $10$ & $12$ \\
\hline
$6$ & $q^{-6}$ &  &  &  \\
$4$ & $3q^{-4}$ & $3q^{-6}$ &  &  \\
$2$ & $6q^{-2}$ & $7q^{-4}$ & $q^{-6}$ &  \\
$0$ & $7$ & $13q^{-2}$ & $6q^{-4}$ &  \\
$-2$ & $6q^{2}$ & $15$ & $10q^{-2}$ & $q^{-4}$ \\
$-4$ & $3q^{4}$ & $13q^{2}$ & $13$ & $3q^{-2}$ \\
$-6$ & $q^{6}$ & $7q^{4}$ & $10q^{2}$ & $4$ \\
$-8$ &  & $3q^{6}$ & $6q^{4}$ & $3q^{2}$ \\
$-10$ &  &  & $q^{6}$ & $q^{4}$ \\
\end{tabular}
\vspace{2em}
\end{minipage}
%
\begin{minipage}{\linewidth}
$\bullet\ $ $11a_{245}$ \vspace{0.5em} \\
\begin{tabular}{l|llll}
$k \setminus j$ & $6$ & $8$ & $10$ & $12$ \\
\hline
$6$ & $q^{-6}$ &  &  &  \\
$4$ & $q^{-4}$ & $q^{-6}$ &  &  \\
$2$ & $3q^{-2}$ & $4q^{-4}$ & $q^{-6}$ &  \\
$0$ & $2$ & $5q^{-2}$ & $3q^{-4}$ &  \\
$-2$ & $3q^{2}$ & $8$ & $6q^{-2}$ & $q^{-4}$ \\
$-4$ & $q^{4}$ & $5q^{2}$ & $6$ & $2q^{-2}$ \\
$-6$ & $q^{6}$ & $4q^{4}$ & $6q^{2}$ & $3$ \\
$-8$ &  & $q^{6}$ & $3q^{4}$ & $2q^{2}$ \\
$-10$ &  &  & $q^{6}$ & $q^{4}$ \\
\end{tabular}
\vspace{2em}
\end{minipage}
%
\begin{minipage}{\linewidth}
$\bullet\ $ $11a_{248}$ \vspace{0.5em} \\
\begin{tabular}{l|llll}
$k \setminus j$ & $-6$ & $-4$ & $-2$ & $0$ \\
\hline
$8$ & $q^{-4}$ & $2q^{-6}$ & $q^{-8}$ &  \\
$6$ & $3q^{-2}$ & $6q^{-4}$ & $3q^{-6}$ &  \\
$4$ & $4$ & $11q^{-2}$ & $8q^{-4}$ & $q^{-6}$ \\
$2$ & $3q^{2}$ & $14$ & $14q^{-2}$ & $3q^{-4}$ \\
$0$ & $q^{4}$ & $11q^{2}$ & $16$ & $6q^{-2}$ \\
$-2$ &  & $6q^{4}$ & $14q^{2}$ & $7$ \\
$-4$ &  & $2q^{6}$ & $8q^{4}$ & $6q^{2}$ \\
$-6$ &  &  & $3q^{6}$ & $3q^{4}$ \\
$-8$ &  &  & $q^{8}$ & $q^{6}$ \\
\end{tabular}
\vspace{2em}
\end{minipage}
%
\begin{minipage}{\linewidth}
$\bullet\ $ $11a_{249}$ \vspace{0.5em} \\
\begin{tabular}{l|lllll}
$k \setminus j$ & $-6$ & $-4$ & $-2$ & $0$ & $2$ \\
\hline
$8$ & $q^{-2}$ & $2q^{-4}$ & $q^{-6}$ &  &  \\
$6$ & $2$ & $5q^{-2}$ & $4q^{-4}$ & $q^{-6}$ &  \\
$4$ & $q^{2}$ & $7$ & $10q^{-2}$ & $4q^{-4}$ &  \\
$2$ &  & $5q^{2}$ & $12$ & $8q^{-2}$ & $q^{-4}$ \\
$0$ &  & $2q^{4}$ & $10q^{2}$ & $12$ & $3q^{-2}$ \\
$-2$ &  &  & $4q^{4}$ & $8q^{2}$ & $4$ \\
$-4$ &  &  & $q^{6}$ & $4q^{4}$ & $3q^{2}$ \\
$-6$ &  &  &  & $q^{6}$ & $q^{4}$ \\
\end{tabular}
\vspace{2em}
\end{minipage}
%
\begin{minipage}{\linewidth}
$\bullet\ $ $11a_{250}$ \vspace{0.5em} \\
\begin{tabular}{l|llll}
$k \setminus j$ & $0$ & $2$ & $4$ & $6$ \\
\hline
$8$ & $q^{-4}$ & $2q^{-6}$ & $q^{-8}$ &  \\
$6$ &  & $2q^{-4}$ & $2q^{-6}$ &  \\
$4$ & $2$ & $6q^{-2}$ & $6q^{-4}$ & $q^{-6}$ \\
$2$ &  & $4$ & $6q^{-2}$ & $2q^{-4}$ \\
$0$ & $q^{4}$ & $6q^{2}$ & $9$ & $4q^{-2}$ \\
$-2$ &  & $2q^{4}$ & $6q^{2}$ & $4$ \\
$-4$ &  & $2q^{6}$ & $6q^{4}$ & $4q^{2}$ \\
$-6$ &  &  & $2q^{6}$ & $2q^{4}$ \\
$-8$ &  &  & $q^{8}$ & $q^{6}$ \\
\end{tabular}
\vspace{2em}
\end{minipage}
%
\begin{minipage}{\linewidth}
$\bullet\ $ $11a_{251}$ \vspace{0.5em} \\
\begin{tabular}{l|llll}
$k \setminus j$ & $-4$ & $-2$ & $0$ & $2$ \\
\hline
$8$ & $q^{-4}$ & $2q^{-6}$ & $q^{-8}$ &  \\
$6$ & $q^{-2}$ & $4q^{-4}$ & $3q^{-6}$ &  \\
$4$ & $2$ & $9q^{-2}$ & $8q^{-4}$ & $q^{-6}$ \\
$2$ & $q^{2}$ & $9$ & $11q^{-2}$ & $3q^{-4}$ \\
$0$ & $q^{4}$ & $9q^{2}$ & $15$ & $6q^{-2}$ \\
$-2$ &  & $4q^{4}$ & $11q^{2}$ & $7$ \\
$-4$ &  & $2q^{6}$ & $8q^{4}$ & $6q^{2}$ \\
$-6$ &  &  & $3q^{6}$ & $3q^{4}$ \\
$-8$ &  &  & $q^{8}$ & $q^{6}$ \\
\end{tabular}
\vspace{2em}
\end{minipage}
%
\begin{minipage}{\linewidth}
$\bullet\ $ $11a_{252}$ \vspace{0.5em} \\
\begin{tabular}{l|llll}
$k \setminus j$ & $-2$ & $0$ & $2$ & $4$ \\
\hline
$8$ & $q^{-4}$ & $2q^{-6}$ & $q^{-8}$ &  \\
$6$ & $q^{-2}$ & $4q^{-4}$ & $3q^{-6}$ &  \\
$4$ & $2$ & $9q^{-2}$ & $8q^{-4}$ & $q^{-6}$ \\
$2$ & $q^{2}$ & $8$ & $11q^{-2}$ & $3q^{-4}$ \\
$0$ & $q^{4}$ & $9q^{2}$ & $14$ & $6q^{-2}$ \\
$-2$ &  & $4q^{4}$ & $11q^{2}$ & $7$ \\
$-4$ &  & $2q^{6}$ & $8q^{4}$ & $6q^{2}$ \\
$-6$ &  &  & $3q^{6}$ & $3q^{4}$ \\
$-8$ &  &  & $q^{8}$ & $q^{6}$ \\
\end{tabular}
\vspace{2em}
\end{minipage}
%
\begin{minipage}{\linewidth}
$\bullet\ $ $11a_{253}$ \vspace{0.5em} \\
\begin{tabular}{l|llll}
$k \setminus j$ & $-4$ & $-2$ & $0$ & $2$ \\
\hline
$8$ & $q^{-4}$ & $2q^{-6}$ & $q^{-8}$ &  \\
$6$ & $q^{-2}$ & $4q^{-4}$ & $3q^{-6}$ &  \\
$4$ & $2$ & $9q^{-2}$ & $8q^{-4}$ & $q^{-6}$ \\
$2$ & $q^{2}$ & $9$ & $11q^{-2}$ & $3q^{-4}$ \\
$0$ & $q^{4}$ & $9q^{2}$ & $15$ & $6q^{-2}$ \\
$-2$ &  & $4q^{4}$ & $11q^{2}$ & $7$ \\
$-4$ &  & $2q^{6}$ & $8q^{4}$ & $6q^{2}$ \\
$-6$ &  &  & $3q^{6}$ & $3q^{4}$ \\
$-8$ &  &  & $q^{8}$ & $q^{6}$ \\
\end{tabular}
\vspace{2em}
\end{minipage}
%
\begin{minipage}{\linewidth}
$\bullet\ $ $11a_{254}$ \vspace{0.5em} \\
\begin{tabular}{l|llll}
$k \setminus j$ & $-2$ & $0$ & $2$ & $4$ \\
\hline
$8$ & $q^{-4}$ & $2q^{-6}$ & $q^{-8}$ &  \\
$6$ & $q^{-2}$ & $4q^{-4}$ & $3q^{-6}$ &  \\
$4$ & $2$ & $9q^{-2}$ & $8q^{-4}$ & $q^{-6}$ \\
$2$ & $q^{2}$ & $8$ & $11q^{-2}$ & $3q^{-4}$ \\
$0$ & $q^{4}$ & $9q^{2}$ & $14$ & $6q^{-2}$ \\
$-2$ &  & $4q^{4}$ & $11q^{2}$ & $7$ \\
$-4$ &  & $2q^{6}$ & $8q^{4}$ & $6q^{2}$ \\
$-6$ &  &  & $3q^{6}$ & $3q^{4}$ \\
$-8$ &  &  & $q^{8}$ & $q^{6}$ \\
\end{tabular}
\vspace{2em}
\end{minipage}
%
\begin{minipage}{\linewidth}
$\bullet\ $ $11a_{255}$ \vspace{0.5em} \\
\begin{tabular}{l|llll}
$k \setminus j$ & $-6$ & $-4$ & $-2$ & $0$ \\
\hline
$8$ & $q^{-4}$ & $2q^{-6}$ & $q^{-8}$ &  \\
$6$ & $2q^{-2}$ & $5q^{-4}$ & $3q^{-6}$ &  \\
$4$ & $3$ & $10q^{-2}$ & $8q^{-4}$ & $q^{-6}$ \\
$2$ & $2q^{2}$ & $11$ & $12q^{-2}$ & $3q^{-4}$ \\
$0$ & $q^{4}$ & $10q^{2}$ & $15$ & $6q^{-2}$ \\
$-2$ &  & $5q^{4}$ & $12q^{2}$ & $6$ \\
$-4$ &  & $2q^{6}$ & $8q^{4}$ & $6q^{2}$ \\
$-6$ &  &  & $3q^{6}$ & $3q^{4}$ \\
$-8$ &  &  & $q^{8}$ & $q^{6}$ \\
\end{tabular}
\vspace{2em}
\end{minipage}
%
\begin{minipage}{\linewidth}
$\bullet\ $ $11a_{256}$ \vspace{0.5em} \\
\begin{tabular}{l|lllll}
$k \setminus j$ & $-6$ & $-4$ & $-2$ & $0$ & $2$ \\
\hline
$8$ & $q^{-2}$ & $2q^{-4}$ & $q^{-6}$ &  &  \\
$6$ & $2$ & $6q^{-2}$ & $5q^{-4}$ & $q^{-6}$ &  \\
$4$ & $q^{2}$ & $8$ & $11q^{-2}$ & $4q^{-4}$ &  \\
$2$ &  & $6q^{2}$ & $15$ & $10q^{-2}$ & $q^{-4}$ \\
$0$ &  & $2q^{4}$ & $11q^{2}$ & $13$ & $3q^{-2}$ \\
$-2$ &  &  & $5q^{4}$ & $10q^{2}$ & $5$ \\
$-4$ &  &  & $q^{6}$ & $4q^{4}$ & $3q^{2}$ \\
$-6$ &  &  &  & $q^{6}$ & $q^{4}$ \\
\end{tabular}
\vspace{2em}
\end{minipage}
%
\begin{minipage}{\linewidth}
$\bullet\ $ $11a_{257}$ \vspace{0.5em} \\
\begin{tabular}{l|llll}
$k \setminus j$ & $-4$ & $-2$ & $0$ & $2$ \\
\hline
$8$ & $q^{-4}$ & $2q^{-6}$ & $q^{-8}$ &  \\
$6$ & $q^{-2}$ & $3q^{-4}$ & $2q^{-6}$ &  \\
$4$ & $2$ & $7q^{-2}$ & $6q^{-4}$ & $q^{-6}$ \\
$2$ & $q^{2}$ & $6$ & $7q^{-2}$ & $2q^{-4}$ \\
$0$ & $q^{4}$ & $7q^{2}$ & $11$ & $4q^{-2}$ \\
$-2$ &  & $3q^{4}$ & $7q^{2}$ & $4$ \\
$-4$ &  & $2q^{6}$ & $6q^{4}$ & $4q^{2}$ \\
$-6$ &  &  & $2q^{6}$ & $2q^{4}$ \\
$-8$ &  &  & $q^{8}$ & $q^{6}$ \\
\end{tabular}
\vspace{2em}
\end{minipage}
%
\begin{minipage}{\linewidth}
$\bullet\ $ $11a_{258}$ \vspace{0.5em} \\
\begin{tabular}{l|lllll}
$k \setminus j$ & $-4$ & $-2$ & $0$ & $2$ & $4$ \\
\hline
$8$ & $q^{-2}$ & $2q^{-4}$ & $q^{-6}$ &  &  \\
$6$ &  & $2q^{-2}$ & $3q^{-4}$ & $q^{-6}$ &  \\
$4$ & $q^{2}$ & $4$ & $6q^{-2}$ & $3q^{-4}$ &  \\
$2$ &  & $2q^{2}$ & $6$ & $6q^{-2}$ & $q^{-4}$ \\
$0$ &  & $2q^{4}$ & $6q^{2}$ & $6$ & $2q^{-2}$ \\
$-2$ &  &  & $3q^{4}$ & $6q^{2}$ & $3$ \\
$-4$ &  &  & $q^{6}$ & $3q^{4}$ & $2q^{2}$ \\
$-6$ &  &  &  & $q^{6}$ & $q^{4}$ \\
\end{tabular}
\vspace{2em}
\end{minipage}
%
\begin{minipage}{\linewidth}
$\bullet\ $ $11a_{259}$ \vspace{0.5em} \\
\begin{tabular}{l|llll}
$k \setminus j$ & $4$ & $6$ & $8$ & $10$ \\
\hline
$8$ & $q^{-6}$ & $q^{-8}$ &  &  \\
$6$ & $q^{-4}$ & $2q^{-6}$ &  &  \\
$4$ & $3q^{-2}$ & $5q^{-4}$ & $2q^{-6}$ &  \\
$2$ & $2$ & $5q^{-2}$ & $3q^{-4}$ &  \\
$0$ & $3q^{2}$ & $8$ & $6q^{-2}$ & $q^{-4}$ \\
$-2$ & $q^{4}$ & $5q^{2}$ & $5$ & $q^{-2}$ \\
$-4$ & $q^{6}$ & $5q^{4}$ & $6q^{2}$ & $2$ \\
$-6$ &  & $2q^{6}$ & $3q^{4}$ & $q^{2}$ \\
$-8$ &  & $q^{8}$ & $2q^{6}$ & $q^{4}$ \\
\end{tabular}
\vspace{2em}
\end{minipage}
%
\begin{minipage}{\linewidth}
$\bullet\ $ $11a_{260}$ \vspace{0.5em} \\
\begin{tabular}{l|lllll}
$k \setminus j$ & $2$ & $4$ & $6$ & $8$ & $10$ \\
\hline
$6$ & $q^{-4}$ & $q^{-6}$ &  &  &  \\
$4$ & $q^{-2}$ & $3q^{-4}$ & $q^{-6}$ &  &  \\
$2$ & $2$ & $5q^{-2}$ & $3q^{-4}$ &  &  \\
$0$ & $q^{2}$ & $5$ & $6q^{-2}$ & $2q^{-4}$ &  \\
$-2$ & $q^{4}$ & $5q^{2}$ & $6$ & $2q^{-2}$ &  \\
$-4$ &  & $3q^{4}$ & $6q^{2}$ & $4$ & $q^{-2}$ \\
$-6$ &  & $q^{6}$ & $3q^{4}$ & $2q^{2}$ &  \\
$-8$ &  &  & $q^{6}$ & $2q^{4}$ & $q^{2}$ \\
\end{tabular}
\vspace{2em}
\end{minipage}
%
\begin{minipage}{\linewidth}
$\bullet\ $ $11a_{261}$ \vspace{0.5em} \\
\begin{tabular}{l|llll}
$k \setminus j$ & $-8$ & $-6$ & $-4$ & $-2$ \\
\hline
$8$ & $q^{-4}$ & $2q^{-6}$ & $q^{-8}$ &  \\
$6$ & $2q^{-2}$ & $5q^{-4}$ & $3q^{-6}$ &  \\
$4$ & $3$ & $9q^{-2}$ & $7q^{-4}$ & $q^{-6}$ \\
$2$ & $2q^{2}$ & $10$ & $11q^{-2}$ & $3q^{-4}$ \\
$0$ & $q^{4}$ & $9q^{2}$ & $12$ & $4q^{-2}$ \\
$-2$ &  & $5q^{4}$ & $11q^{2}$ & $6$ \\
$-4$ &  & $2q^{6}$ & $7q^{4}$ & $4q^{2}$ \\
$-6$ &  &  & $3q^{6}$ & $3q^{4}$ \\
$-8$ &  &  & $q^{8}$ & $q^{6}$ \\
\end{tabular}
\vspace{2em}
\end{minipage}
%
\begin{minipage}{\linewidth}
$\bullet\ $ $11a_{262}$ \vspace{0.5em} \\
\begin{tabular}{l|lllll}
$k \setminus j$ & $-8$ & $-6$ & $-4$ & $-2$ & $0$ \\
\hline
$8$ & $q^{-2}$ & $2q^{-4}$ & $q^{-6}$ &  &  \\
$6$ & $1$ & $4q^{-2}$ & $4q^{-4}$ & $q^{-6}$ &  \\
$4$ & $q^{2}$ & $6$ & $9q^{-2}$ & $4q^{-4}$ &  \\
$2$ &  & $4q^{2}$ & $11$ & $8q^{-2}$ & $q^{-4}$ \\
$0$ &  & $2q^{4}$ & $9q^{2}$ & $10$ & $3q^{-2}$ \\
$-2$ &  &  & $4q^{4}$ & $8q^{2}$ & $3$ \\
$-4$ &  &  & $q^{6}$ & $4q^{4}$ & $3q^{2}$ \\
$-6$ &  &  &  & $q^{6}$ & $q^{4}$ \\
\end{tabular}
\vspace{2em}
\end{minipage}
%
\begin{minipage}{\linewidth}
$\bullet\ $ $11a_{263}$ \vspace{0.5em} \\
\begin{tabular}{l|llll}
$k \setminus j$ & $8$ & $10$ & $12$ & $14$ \\
\hline
$8$ & $q^{-8}$ &  &  &  \\
$6$ & $q^{-6}$ & $q^{-8}$ &  &  \\
$4$ & $4q^{-4}$ & $4q^{-6}$ &  &  \\
$2$ & $3q^{-2}$ & $4q^{-4}$ & $q^{-6}$ &  \\
$0$ & $6$ & $9q^{-2}$ & $3q^{-4}$ &  \\
$-2$ & $3q^{2}$ & $6$ & $3q^{-2}$ &  \\
$-4$ & $4q^{4}$ & $1$ + $9q^{2}$ & $q^{-2}$ + $5$ &  \\
$-6$ & $q^{6}$ & $4q^{4}$ & $3q^{2}$ &  \\
$-8$ & $q^{8}$ & $4q^{6}$ & $q^{2}$ + $3q^{4}$ & $1$ \\
$-10$ &  & $q^{8}$ & $q^{6}$ &  \\
\end{tabular}
\vspace{2em}
\end{minipage}
%
\begin{minipage}{\linewidth}
$\bullet\ $ $11a_{264}$ \vspace{0.5em} \\
\begin{tabular}{l|llll}
$k \setminus j$ & $0$ & $2$ & $4$ & $6$ \\
\hline
$8$ & $q^{-6}$ & $q^{-8}$ &  &  \\
$6$ & $3q^{-4}$ & $3q^{-6}$ &  &  \\
$4$ & $5q^{-2}$ & $7q^{-4}$ & $2q^{-6}$ &  \\
$2$ & $6$ & $12q^{-2}$ & $5q^{-4}$ &  \\
$0$ & $5q^{2}$ & $13$ & $9q^{-2}$ & $q^{-4}$ \\
$-2$ & $3q^{4}$ & $12q^{2}$ & $11$ & $2q^{-2}$ \\
$-4$ & $q^{6}$ & $7q^{4}$ & $9q^{2}$ & $3$ \\
$-6$ &  & $3q^{6}$ & $5q^{4}$ & $2q^{2}$ \\
$-8$ &  & $q^{8}$ & $2q^{6}$ & $q^{4}$ \\
\end{tabular}
\vspace{2em}
\end{minipage}
%
\begin{minipage}{\linewidth}
$\bullet\ $ $11a_{265}$ \vspace{0.5em} \\
\begin{tabular}{l|lllll}
$k \setminus j$ & $-2$ & $0$ & $2$ & $4$ & $6$ \\
\hline
$6$ & $q^{-4}$ & $q^{-6}$ &  &  &  \\
$4$ & $3q^{-2}$ & $4q^{-4}$ & $q^{-6}$ &  &  \\
$2$ & $4$ & $8q^{-2}$ & $4q^{-4}$ &  &  \\
$0$ & $3q^{2}$ & $11$ & $9q^{-2}$ & $2q^{-4}$ &  \\
$-2$ & $q^{4}$ & $8q^{2}$ & $11$ & $4q^{-2}$ &  \\
$-4$ &  & $4q^{4}$ & $9q^{2}$ & $6$ & $q^{-2}$ \\
$-6$ &  & $q^{6}$ & $4q^{4}$ & $4q^{2}$ & $1$ \\
$-8$ &  &  & $q^{6}$ & $2q^{4}$ & $q^{2}$ \\
\end{tabular}
\vspace{2em}
\end{minipage}
%
\begin{minipage}{\linewidth}
$\bullet\ $ $11a_{266}$ \vspace{0.5em} \\
\begin{tabular}{l|llll}
$k \setminus j$ & $-4$ & $-2$ & $0$ & $2$ \\
\hline
$8$ & $q^{-4}$ & $2q^{-6}$ & $q^{-8}$ &  \\
$6$ & $4q^{-2}$ & $8q^{-4}$ & $4q^{-6}$ &  \\
$4$ & $5$ & $14q^{-2}$ & $10q^{-4}$ & $q^{-6}$ \\
$2$ & $4q^{2}$ & $19$ & $19q^{-2}$ & $4q^{-4}$ \\
$0$ & $q^{4}$ & $14q^{2}$ & $22$ & $8q^{-2}$ \\
$-2$ &  & $8q^{4}$ & $19q^{2}$ & $11$ \\
$-4$ &  & $2q^{6}$ & $10q^{4}$ & $8q^{2}$ \\
$-6$ &  &  & $4q^{6}$ & $4q^{4}$ \\
$-8$ &  &  & $q^{8}$ & $q^{6}$ \\
\end{tabular}
\vspace{2em}
\end{minipage}
%
\begin{minipage}{\linewidth}
$\bullet\ $ $11a_{267}$ \vspace{0.5em} \\
\begin{tabular}{l|llll}
$k \setminus j$ & $-6$ & $-4$ & $-2$ & $0$ \\
\hline
$8$ & $q^{-4}$ & $2q^{-6}$ & $q^{-8}$ &  \\
$6$ & $3q^{-2}$ & $7q^{-4}$ & $4q^{-6}$ &  \\
$4$ & $4$ & $13q^{-2}$ & $10q^{-4}$ & $q^{-6}$ \\
$2$ & $3q^{2}$ & $16$ & $17q^{-2}$ & $4q^{-4}$ \\
$0$ & $q^{4}$ & $13q^{2}$ & $20$ & $8q^{-2}$ \\
$-2$ &  & $7q^{4}$ & $17q^{2}$ & $9$ \\
$-4$ &  & $2q^{6}$ & $10q^{4}$ & $8q^{2}$ \\
$-6$ &  &  & $4q^{6}$ & $4q^{4}$ \\
$-8$ &  &  & $q^{8}$ & $q^{6}$ \\
\end{tabular}
\vspace{2em}
\end{minipage}
%
\begin{minipage}{\linewidth}
$\bullet\ $ $11a_{268}$ \vspace{0.5em} \\
\begin{tabular}{l|llll}
$k \setminus j$ & $-2$ & $0$ & $2$ & $4$ \\
\hline
$8$ & $q^{-4}$ & $2q^{-6}$ & $q^{-8}$ &  \\
$6$ & $2q^{-2}$ & $5q^{-4}$ & $3q^{-6}$ &  \\
$4$ & $2$ & $9q^{-2}$ & $8q^{-4}$ & $q^{-6}$ \\
$2$ & $2q^{2}$ & $10$ & $12q^{-2}$ & $3q^{-4}$ \\
$0$ & $q^{4}$ & $9q^{2}$ & $14$ & $6q^{-2}$ \\
$-2$ &  & $5q^{4}$ & $12q^{2}$ & $7$ \\
$-4$ &  & $2q^{6}$ & $8q^{4}$ & $6q^{2}$ \\
$-6$ &  &  & $3q^{6}$ & $3q^{4}$ \\
$-8$ &  &  & $q^{8}$ & $q^{6}$ \\
\end{tabular}
\vspace{2em}
\end{minipage}
%
\begin{minipage}{\linewidth}
$\bullet\ $ $11a_{269}$ \vspace{0.5em} \\
\begin{tabular}{l|llll}
$k \setminus j$ & $-2$ & $0$ & $2$ & $4$ \\
\hline
$8$ & $q^{-4}$ & $2q^{-6}$ & $q^{-8}$ &  \\
$6$ & $3q^{-2}$ & $6q^{-4}$ & $3q^{-6}$ &  \\
$4$ & $3$ & $10q^{-2}$ & $8q^{-4}$ & $q^{-6}$ \\
$2$ & $3q^{2}$ & $12$ & $13q^{-2}$ & $3q^{-4}$ \\
$0$ & $q^{4}$ & $10q^{2}$ & $15$ & $6q^{-2}$ \\
$-2$ &  & $6q^{4}$ & $13q^{2}$ & $7$ \\
$-4$ &  & $2q^{6}$ & $8q^{4}$ & $6q^{2}$ \\
$-6$ &  &  & $3q^{6}$ & $3q^{4}$ \\
$-8$ &  &  & $q^{8}$ & $q^{6}$ \\
\end{tabular}
\vspace{2em}
\end{minipage}
%
\begin{minipage}{\linewidth}
$\bullet\ $ $11a_{274}$ \vspace{0.5em} \\
\begin{tabular}{l|llll}
$k \setminus j$ & $-4$ & $-2$ & $0$ & $2$ \\
\hline
$8$ & $q^{-4}$ & $2q^{-6}$ & $q^{-8}$ &  \\
$6$ & $2q^{-2}$ & $5q^{-4}$ & $3q^{-6}$ &  \\
$4$ & $4$ & $12q^{-2}$ & $9q^{-4}$ & $q^{-6}$ \\
$2$ & $2q^{2}$ & $13$ & $14q^{-2}$ & $3q^{-4}$ \\
$0$ & $q^{4}$ & $12q^{2}$ & $19$ & $7q^{-2}$ \\
$-2$ &  & $5q^{4}$ & $14q^{2}$ & $9$ \\
$-4$ &  & $2q^{6}$ & $9q^{4}$ & $7q^{2}$ \\
$-6$ &  &  & $3q^{6}$ & $3q^{4}$ \\
$-8$ &  &  & $q^{8}$ & $q^{6}$ \\
\end{tabular}
\vspace{2em}
\end{minipage}
%
\begin{minipage}{\linewidth}
$\bullet\ $ $11a_{277}$ \vspace{0.5em} \\
\begin{tabular}{l|llll}
$k \setminus j$ & $0$ & $2$ & $4$ & $6$ \\
\hline
$8$ & $q^{-6}$ & $q^{-8}$ &  &  \\
$6$ & $3q^{-4}$ & $3q^{-6}$ &  &  \\
$4$ & $6q^{-2}$ & $8q^{-4}$ & $2q^{-6}$ &  \\
$2$ & $5$ & $11q^{-2}$ & $5q^{-4}$ &  \\
$0$ & $6q^{2}$ & $14$ & $9q^{-2}$ & $q^{-4}$ \\
$-2$ & $3q^{4}$ & $11q^{2}$ & $10$ & $2q^{-2}$ \\
$-4$ & $q^{6}$ & $8q^{4}$ & $9q^{2}$ & $2$ \\
$-6$ &  & $3q^{6}$ & $5q^{4}$ & $2q^{2}$ \\
$-8$ &  & $q^{8}$ & $2q^{6}$ & $q^{4}$ \\
\end{tabular}
\vspace{2em}
\end{minipage}
%
\begin{minipage}{\linewidth}
$\bullet\ $ $11a_{278}$ \vspace{0.5em} \\
\begin{tabular}{l|lllll}
$k \setminus j$ & $-2$ & $0$ & $2$ & $4$ & $6$ \\
\hline
$6$ & $q^{-4}$ & $q^{-6}$ &  &  &  \\
$4$ & $3q^{-2}$ & $4q^{-4}$ & $q^{-6}$ &  &  \\
$2$ & $6$ & $11q^{-2}$ & $5q^{-4}$ &  &  \\
$0$ & $3q^{2}$ & $14$ & $12q^{-2}$ & $2q^{-4}$ &  \\
$-2$ & $q^{4}$ & $11q^{2}$ & $16$ & $6q^{-2}$ &  \\
$-4$ &  & $4q^{4}$ & $12q^{2}$ & $9$ & $q^{-2}$ \\
$-6$ &  & $q^{6}$ & $5q^{4}$ & $6q^{2}$ & $2$ \\
$-8$ &  &  & $q^{6}$ & $2q^{4}$ & $q^{2}$ \\
\end{tabular}
\vspace{2em}
\end{minipage}
%
\begin{minipage}{\linewidth}
$\bullet\ $ $11a_{281}$ \vspace{0.5em} \\
\begin{tabular}{l|llll}
$k \setminus j$ & $-2$ & $0$ & $2$ & $4$ \\
\hline
$8$ & $q^{-4}$ & $2q^{-6}$ & $q^{-8}$ &  \\
$6$ & $2q^{-2}$ & $5q^{-4}$ & $3q^{-6}$ &  \\
$4$ & $3$ & $11q^{-2}$ & $9q^{-4}$ & $q^{-6}$ \\
$2$ & $2q^{2}$ & $11$ & $13q^{-2}$ & $3q^{-4}$ \\
$0$ & $q^{4}$ & $11q^{2}$ & $17$ & $7q^{-2}$ \\
$-2$ &  & $5q^{4}$ & $13q^{2}$ & $8$ \\
$-4$ &  & $2q^{6}$ & $9q^{4}$ & $7q^{2}$ \\
$-6$ &  &  & $3q^{6}$ & $3q^{4}$ \\
$-8$ &  &  & $q^{8}$ & $q^{6}$ \\
\end{tabular}
\vspace{2em}
\end{minipage}
%
\begin{minipage}{\linewidth}
$\bullet\ $ $11a_{282}$ \vspace{0.5em} \\
\begin{tabular}{l|llll}
$k \setminus j$ & $-4$ & $-2$ & $0$ & $2$ \\
\hline
$8$ & $q^{-6}$ & $q^{-8}$ &  &  \\
$6$ & $3q^{-4}$ & $3q^{-6}$ &  &  \\
$4$ & $5q^{-2}$ & $7q^{-4}$ & $2q^{-6}$ &  \\
$2$ & $6$ & $11q^{-2}$ & $5q^{-4}$ &  \\
$0$ & $5q^{2}$ & $12$ & $8q^{-2}$ & $q^{-4}$ \\
$-2$ & $3q^{4}$ & $11q^{2}$ & $9$ & $2q^{-2}$ \\
$-4$ & $q^{6}$ & $7q^{4}$ & $8q^{2}$ & $2$ \\
$-6$ &  & $3q^{6}$ & $5q^{4}$ & $2q^{2}$ \\
$-8$ &  & $q^{8}$ & $2q^{6}$ & $q^{4}$ \\
\end{tabular}
\vspace{2em}
\end{minipage}
%
\begin{minipage}{\linewidth}
$\bullet\ $ $11a_{284}$ \vspace{0.5em} \\
\begin{tabular}{l|llll}
$k \setminus j$ & $0$ & $2$ & $4$ & $6$ \\
\hline
$8$ & $q^{-6}$ & $q^{-8}$ &  &  \\
$6$ & $4q^{-4}$ & $4q^{-6}$ &  &  \\
$4$ & $8q^{-2}$ & $10q^{-4}$ & $2q^{-6}$ &  \\
$2$ & $9$ & $16q^{-2}$ & $6q^{-4}$ &  \\
$0$ & $8q^{2}$ & $19$ & $12q^{-2}$ & $q^{-4}$ \\
$-2$ & $4q^{4}$ & $16q^{2}$ & $14$ & $2q^{-2}$ \\
$-4$ & $q^{6}$ & $10q^{4}$ & $12q^{2}$ & $3$ \\
$-6$ &  & $4q^{6}$ & $6q^{4}$ & $2q^{2}$ \\
$-8$ &  & $q^{8}$ & $2q^{6}$ & $q^{4}$ \\
\end{tabular}
\vspace{2em}
\end{minipage}
%
\begin{minipage}{\linewidth}
$\bullet\ $ $11a_{286}$ \vspace{0.5em} \\
\begin{tabular}{l|llll}
$k \setminus j$ & $-2$ & $0$ & $2$ & $4$ \\
\hline
$8$ & $q^{-4}$ & $2q^{-6}$ & $q^{-8}$ &  \\
$6$ & $2q^{-2}$ & $5q^{-4}$ & $3q^{-6}$ &  \\
$4$ & $3$ & $10q^{-2}$ & $8q^{-4}$ & $q^{-6}$ \\
$2$ & $2q^{2}$ & $11$ & $13q^{-2}$ & $3q^{-4}$ \\
$0$ & $q^{4}$ & $10q^{2}$ & $15$ & $6q^{-2}$ \\
$-2$ &  & $5q^{4}$ & $13q^{2}$ & $8$ \\
$-4$ &  & $2q^{6}$ & $8q^{4}$ & $6q^{2}$ \\
$-6$ &  &  & $3q^{6}$ & $3q^{4}$ \\
$-8$ &  &  & $q^{8}$ & $q^{6}$ \\
\end{tabular}
\vspace{2em}
\end{minipage}
%
\begin{minipage}{\linewidth}
$\bullet\ $ $11a_{287}$ \vspace{0.5em} \\
\begin{tabular}{l|llll}
$k \setminus j$ & $-4$ & $-2$ & $0$ & $2$ \\
\hline
$8$ & $q^{-4}$ & $2q^{-6}$ & $q^{-8}$ &  \\
$6$ & $2q^{-2}$ & $6q^{-4}$ & $4q^{-6}$ &  \\
$4$ & $3$ & $12q^{-2}$ & $10q^{-4}$ & $q^{-6}$ \\
$2$ & $2q^{2}$ & $14$ & $16q^{-2}$ & $4q^{-4}$ \\
$0$ & $q^{4}$ & $12q^{2}$ & $20$ & $8q^{-2}$ \\
$-2$ &  & $6q^{4}$ & $16q^{2}$ & $10$ \\
$-4$ &  & $2q^{6}$ & $10q^{4}$ & $8q^{2}$ \\
$-6$ &  &  & $4q^{6}$ & $4q^{4}$ \\
$-8$ &  &  & $q^{8}$ & $q^{6}$ \\
\end{tabular}
\vspace{2em}
\end{minipage}
%
\begin{minipage}{\linewidth}
$\bullet\ $ $11a_{288}$ \vspace{0.5em} \\
\begin{tabular}{l|llll}
$k \setminus j$ & $-2$ & $0$ & $2$ & $4$ \\
\hline
$8$ & $q^{-6}$ & $q^{-8}$ &  &  \\
$6$ & $4q^{-4}$ & $4q^{-6}$ &  &  \\
$4$ & $9q^{-2}$ & $11q^{-4}$ & $2q^{-6}$ &  \\
$2$ & $11$ & $18q^{-2}$ & $7q^{-4}$ &  \\
$0$ & $9q^{2}$ & $23$ & $14q^{-2}$ & $q^{-4}$ \\
$-2$ & $4q^{4}$ & $18q^{2}$ & $17$ & $3q^{-2}$ \\
$-4$ & $q^{6}$ & $11q^{4}$ & $14q^{2}$ & $4$ \\
$-6$ &  & $4q^{6}$ & $7q^{4}$ & $3q^{2}$ \\
$-8$ &  & $q^{8}$ & $2q^{6}$ & $q^{4}$ \\
\end{tabular}
\vspace{2em}
\end{minipage}
%
\begin{minipage}{\linewidth}
$\bullet\ $ $11a_{289}$ \vspace{0.5em} \\
\begin{tabular}{l|llll}
$k \setminus j$ & $-2$ & $0$ & $2$ & $4$ \\
\hline
$8$ & $q^{-6}$ & $q^{-8}$ &  &  \\
$6$ & $3q^{-4}$ & $3q^{-6}$ &  &  \\
$4$ & $6q^{-2}$ & $8q^{-4}$ & $2q^{-6}$ &  \\
$2$ & $7$ & $12q^{-2}$ & $5q^{-4}$ &  \\
$0$ & $6q^{2}$ & $16$ & $10q^{-2}$ & $q^{-4}$ \\
$-2$ & $3q^{4}$ & $12q^{2}$ & $11$ & $2q^{-2}$ \\
$-4$ & $q^{6}$ & $8q^{4}$ & $10q^{2}$ & $3$ \\
$-6$ &  & $3q^{6}$ & $5q^{4}$ & $2q^{2}$ \\
$-8$ &  & $q^{8}$ & $2q^{6}$ & $q^{4}$ \\
\end{tabular}
\vspace{2em}
\end{minipage}
%
\begin{minipage}{\linewidth}
$\bullet\ $ $11a_{291}$ \vspace{0.5em} \\
\begin{tabular}{l|llll}
$k \setminus j$ & $6$ & $8$ & $10$ & $12$ \\
\hline
$6$ & $q^{-6}$ &  &  &  \\
$4$ & $3q^{-4}$ & $3q^{-6}$ &  &  \\
$2$ & $4q^{-2}$ & $5q^{-4}$ & $q^{-6}$ &  \\
$0$ & $4$ & $9q^{-2}$ & $5q^{-4}$ &  \\
$-2$ & $4q^{2}$ & $8$ & $5q^{-2}$ & $q^{-4}$ \\
$-4$ & $3q^{4}$ & $9q^{2}$ & $8$ & $2q^{-2}$ \\
$-6$ & $q^{6}$ & $5q^{4}$ & $5q^{2}$ & $1$ \\
$-8$ &  & $3q^{6}$ & $5q^{4}$ & $2q^{2}$ \\
$-10$ &  &  & $q^{6}$ & $q^{4}$ \\
\end{tabular}
\vspace{2em}
\end{minipage}
%
\begin{minipage}{\linewidth}
$\bullet\ $ $11a_{293}$ \vspace{0.5em} \\
\begin{tabular}{l|llll}
$k \setminus j$ & $-6$ & $-4$ & $-2$ & $0$ \\
\hline
$8$ & $q^{-6}$ & $q^{-8}$ &  &  \\
$6$ & $2q^{-4}$ & $2q^{-6}$ &  &  \\
$4$ & $4q^{-2}$ & $6q^{-4}$ & $2q^{-6}$ &  \\
$2$ & $2$ & $5q^{-2}$ & $3q^{-4}$ &  \\
$0$ & $4q^{2}$ & $8$ & $5q^{-2}$ & $q^{-4}$ \\
$-2$ & $2q^{4}$ & $5q^{2}$ & $4$ & $q^{-2}$ \\
$-4$ & $q^{6}$ & $6q^{4}$ & $5q^{2}$ & $1$ \\
$-6$ &  & $2q^{6}$ & $3q^{4}$ & $q^{2}$ \\
$-8$ &  & $q^{8}$ & $2q^{6}$ & $q^{4}$ \\
\end{tabular}
\vspace{2em}
\end{minipage}
%
\begin{minipage}{\linewidth}
$\bullet\ $ $11a_{298}$ \vspace{0.5em} \\
\begin{tabular}{l|llll}
$k \setminus j$ & $6$ & $8$ & $10$ & $12$ \\
\hline
$6$ & $q^{-6}$ &  &  &  \\
$4$ & $3q^{-4}$ & $3q^{-6}$ &  &  \\
$2$ & $5q^{-2}$ & $6q^{-4}$ & $q^{-6}$ &  \\
$0$ & $6$ & $12q^{-2}$ & $6q^{-4}$ &  \\
$-2$ & $5q^{2}$ & $12$ & $8q^{-2}$ & $q^{-4}$ \\
$-4$ & $3q^{4}$ & $12q^{2}$ & $12$ & $3q^{-2}$ \\
$-6$ & $q^{6}$ & $6q^{4}$ & $8q^{2}$ & $3$ \\
$-8$ &  & $3q^{6}$ & $6q^{4}$ & $3q^{2}$ \\
$-10$ &  &  & $q^{6}$ & $q^{4}$ \\
\end{tabular}
\vspace{2em}
\end{minipage}
%
\begin{minipage}{\linewidth}
$\bullet\ $ $11a_{300}$ \vspace{0.5em} \\
\begin{tabular}{l|llll}
$k \setminus j$ & $-4$ & $-2$ & $0$ & $2$ \\
\hline
$8$ & $q^{-4}$ & $2q^{-6}$ & $q^{-8}$ &  \\
$6$ & $2q^{-2}$ & $5q^{-4}$ & $3q^{-6}$ &  \\
$4$ & $4$ & $11q^{-2}$ & $8q^{-4}$ & $q^{-6}$ \\
$2$ & $2q^{2}$ & $12$ & $13q^{-2}$ & $3q^{-4}$ \\
$0$ & $q^{4}$ & $11q^{2}$ & $17$ & $6q^{-2}$ \\
$-2$ &  & $5q^{4}$ & $13q^{2}$ & $8$ \\
$-4$ &  & $2q^{6}$ & $8q^{4}$ & $6q^{2}$ \\
$-6$ &  &  & $3q^{6}$ & $3q^{4}$ \\
$-8$ &  &  & $q^{8}$ & $q^{6}$ \\
\end{tabular}
\vspace{2em}
\end{minipage}
%
\begin{minipage}{\linewidth}
$\bullet\ $ $11a_{301}$ \vspace{0.5em} \\
\begin{tabular}{l|llll}
$k \setminus j$ & $-6$ & $-4$ & $-2$ & $0$ \\
\hline
$8$ & $q^{-4}$ & $2q^{-6}$ & $q^{-8}$ &  \\
$6$ & $3q^{-2}$ & $7q^{-4}$ & $4q^{-6}$ &  \\
$4$ & $5$ & $14q^{-2}$ & $10q^{-4}$ & $q^{-6}$ \\
$2$ & $3q^{2}$ & $17$ & $18q^{-2}$ & $4q^{-4}$ \\
$0$ & $q^{4}$ & $14q^{2}$ & $21$ & $8q^{-2}$ \\
$-2$ &  & $7q^{4}$ & $18q^{2}$ & $10$ \\
$-4$ &  & $2q^{6}$ & $10q^{4}$ & $8q^{2}$ \\
$-6$ &  &  & $4q^{6}$ & $4q^{4}$ \\
$-8$ &  &  & $q^{8}$ & $q^{6}$ \\
\end{tabular}
\vspace{2em}
\end{minipage}
%
\begin{minipage}{\linewidth}
$\bullet\ $ $11a_{302}$ \vspace{0.5em} \\
\begin{tabular}{l|llll}
$k \setminus j$ & $-8$ & $-6$ & $-4$ & $-2$ \\
\hline
$8$ & $q^{-4}$ & $2q^{-6}$ & $q^{-8}$ &  \\
$6$ & $2q^{-2}$ & $6q^{-4}$ & $4q^{-6}$ &  \\
$4$ & $3$ & $11q^{-2}$ & $9q^{-4}$ & $q^{-6}$ \\
$2$ & $2q^{2}$ & $12$ & $14q^{-2}$ & $4q^{-4}$ \\
$0$ & $q^{4}$ & $11q^{2}$ & $16$ & $6q^{-2}$ \\
$-2$ &  & $6q^{4}$ & $14q^{2}$ & $8$ \\
$-4$ &  & $2q^{6}$ & $9q^{4}$ & $6q^{2}$ \\
$-6$ &  &  & $4q^{6}$ & $4q^{4}$ \\
$-8$ &  &  & $q^{8}$ & $q^{6}$ \\
\end{tabular}
\vspace{2em}
\end{minipage}
%
\begin{minipage}{\linewidth}
$\bullet\ $ $11a_{305}$ \vspace{0.5em} \\
\begin{tabular}{l|llll}
$k \setminus j$ & $-4$ & $-2$ & $0$ & $2$ \\
\hline
$8$ & $q^{-6}$ & $q^{-8}$ &  &  \\
$6$ & $3q^{-4}$ & $3q^{-6}$ &  &  \\
$4$ & $5q^{-2}$ & $7q^{-4}$ & $2q^{-6}$ &  \\
$2$ & $7$ & $12q^{-2}$ & $5q^{-4}$ &  \\
$0$ & $5q^{2}$ & $13$ & $9q^{-2}$ & $q^{-4}$ \\
$-2$ & $3q^{4}$ & $12q^{2}$ & $10$ & $2q^{-2}$ \\
$-4$ & $q^{6}$ & $7q^{4}$ & $9q^{2}$ & $3$ \\
$-6$ &  & $3q^{6}$ & $5q^{4}$ & $2q^{2}$ \\
$-8$ &  & $q^{8}$ & $2q^{6}$ & $q^{4}$ \\
\end{tabular}
\vspace{2em}
\end{minipage}
%
\begin{minipage}{\linewidth}
$\bullet\ $ $11a_{306}$ \vspace{0.5em} \\
\begin{tabular}{l|llll}
$k \setminus j$ & $-8$ & $-6$ & $-4$ & $-2$ \\
\hline
$8$ & $q^{-4}$ & $2q^{-6}$ & $q^{-8}$ &  \\
$6$ & $2q^{-2}$ & $4q^{-4}$ & $2q^{-6}$ &  \\
$4$ & $3$ & $8q^{-2}$ & $6q^{-4}$ & $q^{-6}$ \\
$2$ & $2q^{2}$ & $8$ & $8q^{-2}$ & $2q^{-4}$ \\
$0$ & $q^{4}$ & $8q^{2}$ & $10$ & $3q^{-2}$ \\
$-2$ &  & $4q^{4}$ & $8q^{2}$ & $4$ \\
$-4$ &  & $2q^{6}$ & $6q^{4}$ & $3q^{2}$ \\
$-6$ &  &  & $2q^{6}$ & $2q^{4}$ \\
$-8$ &  &  & $q^{8}$ & $q^{6}$ \\
\end{tabular}
\vspace{2em}
\end{minipage}
%
\begin{minipage}{\linewidth}
$\bullet\ $ $11a_{307}$ \vspace{0.5em} \\
\begin{tabular}{l|lllll}
$k \setminus j$ & $-8$ & $-6$ & $-4$ & $-2$ & $0$ \\
\hline
$8$ & $q^{-2}$ & $2q^{-4}$ & $q^{-6}$ &  &  \\
$6$ & $1$ & $3q^{-2}$ & $3q^{-4}$ & $q^{-6}$ &  \\
$4$ & $q^{2}$ & $5$ & $7q^{-2}$ & $3q^{-4}$ &  \\
$2$ &  & $3q^{2}$ & $8$ & $6q^{-2}$ & $q^{-4}$ \\
$0$ &  & $2q^{4}$ & $7q^{2}$ & $7$ & $2q^{-2}$ \\
$-2$ &  &  & $3q^{4}$ & $6q^{2}$ & $2$ \\
$-4$ &  &  & $q^{6}$ & $3q^{4}$ & $2q^{2}$ \\
$-6$ &  &  &  & $q^{6}$ & $q^{4}$ \\
\end{tabular}
\vspace{2em}
\end{minipage}
%
\begin{minipage}{\linewidth}
$\bullet\ $ $11a_{308}$ \vspace{0.5em} \\
\begin{tabular}{l|llll}
$k \setminus j$ & $4$ & $6$ & $8$ & $10$ \\
\hline
$8$ & $q^{-6}$ & $q^{-8}$ &  &  \\
$6$ & $q^{-4}$ & $2q^{-6}$ &  &  \\
$4$ & $2q^{-2}$ & $4q^{-4}$ & $2q^{-6}$ &  \\
$2$ & $2$ & $5q^{-2}$ & $3q^{-4}$ &  \\
$0$ & $2q^{2}$ & $6$ & $5q^{-2}$ & $q^{-4}$ \\
$-2$ & $q^{4}$ & $5q^{2}$ & $5$ & $q^{-2}$ \\
$-4$ & $q^{6}$ & $4q^{4}$ & $5q^{2}$ & $2$ \\
$-6$ &  & $2q^{6}$ & $3q^{4}$ & $q^{2}$ \\
$-8$ &  & $q^{8}$ & $2q^{6}$ & $q^{4}$ \\
\end{tabular}
\vspace{2em}
\end{minipage}
%
\begin{minipage}{\linewidth}
$\bullet\ $ $11a_{309}$ \vspace{0.5em} \\
\begin{tabular}{l|lllll}
$k \setminus j$ & $2$ & $4$ & $6$ & $8$ & $10$ \\
\hline
$6$ & $q^{-4}$ & $q^{-6}$ &  &  &  \\
$4$ & $2q^{-2}$ & $4q^{-4}$ & $q^{-6}$ &  &  \\
$2$ & $2$ & $6q^{-2}$ & $4q^{-4}$ &  &  \\
$0$ & $2q^{2}$ & $8$ & $8q^{-2}$ & $2q^{-4}$ &  \\
$-2$ & $q^{4}$ & $6q^{2}$ & $9$ & $4q^{-2}$ &  \\
$-4$ &  & $4q^{4}$ & $8q^{2}$ & $5$ & $q^{-2}$ \\
$-6$ &  & $q^{6}$ & $4q^{4}$ & $4q^{2}$ & $1$ \\
$-8$ &  &  & $q^{6}$ & $2q^{4}$ & $q^{2}$ \\
\end{tabular}
\vspace{2em}
\end{minipage}
%
\begin{minipage}{\linewidth}
$\bullet\ $ $11a_{310}$ \vspace{0.5em} \\
\begin{tabular}{l|lllll}
$k \setminus j$ & $2$ & $4$ & $6$ & $8$ & $10$ \\
\hline
$6$ & $q^{-4}$ & $q^{-6}$ &  &  &  \\
$4$ & $q^{-2}$ & $3q^{-4}$ & $q^{-6}$ &  &  \\
$2$ & $1$ & $4q^{-2}$ & $3q^{-4}$ &  &  \\
$0$ & $q^{2}$ & $4$ & $5q^{-2}$ & $2q^{-4}$ &  \\
$-2$ & $q^{4}$ & $4q^{2}$ & $5$ & $2q^{-2}$ &  \\
$-4$ &  & $3q^{4}$ & $5q^{2}$ & $3$ & $q^{-2}$ \\
$-6$ &  & $q^{6}$ & $3q^{4}$ & $2q^{2}$ &  \\
$-8$ &  &  & $q^{6}$ & $2q^{4}$ & $q^{2}$ \\
\end{tabular}
\vspace{2em}
\end{minipage}
%
\begin{minipage}{\linewidth}
$\bullet\ $ $11a_{311}$ \vspace{0.5em} \\
\begin{tabular}{l|llllll}
$k \setminus j$ & $0$ & $2$ & $4$ & $6$ & $8$ & $10$ \\
\hline
$4$ & $q^{-2}$ & $q^{-4}$ &  &  &  &  \\
$2$ & $2$ & $5q^{-2}$ & $2q^{-4}$ &  &  &  \\
$0$ & $q^{2}$ & $7$ & $7q^{-2}$ & $q^{-4}$ &  &  \\
$-2$ &  & $5q^{2}$ & $10$ & $5q^{-2}$ &  &  \\
$-4$ &  & $q^{4}$ & $7q^{2}$ & $8$ & $2q^{-2}$ &  \\
$-6$ &  &  & $2q^{4}$ & $5q^{2}$ & $3$ &  \\
$-8$ &  &  &  & $q^{4}$ & $2q^{2}$ & $1$ \\
\end{tabular}
\vspace{2em}
\end{minipage}
%
\begin{minipage}{\linewidth}
$\bullet\ $ $11a_{314}$ \vspace{0.5em} \\
\begin{tabular}{l|llll}
$k \setminus j$ & $0$ & $2$ & $4$ & $6$ \\
\hline
$8$ & $q^{-6}$ & $q^{-8}$ &  &  \\
$6$ & $4q^{-4}$ & $4q^{-6}$ &  &  \\
$4$ & $8q^{-2}$ & $10q^{-4}$ & $2q^{-6}$ &  \\
$2$ & $8$ & $15q^{-2}$ & $6q^{-4}$ &  \\
$0$ & $8q^{2}$ & $18$ & $11q^{-2}$ & $q^{-4}$ \\
$-2$ & $4q^{4}$ & $15q^{2}$ & $13$ & $2q^{-2}$ \\
$-4$ & $q^{6}$ & $10q^{4}$ & $11q^{2}$ & $2$ \\
$-6$ &  & $4q^{6}$ & $6q^{4}$ & $2q^{2}$ \\
$-8$ &  & $q^{8}$ & $2q^{6}$ & $q^{4}$ \\
\end{tabular}
\vspace{2em}
\end{minipage}
%
\begin{minipage}{\linewidth}
$\bullet\ $ $11a_{315}$ \vspace{0.5em} \\
\begin{tabular}{l|llll}
$k \setminus j$ & $-2$ & $0$ & $2$ & $4$ \\
\hline
$8$ & $q^{-6}$ & $q^{-8}$ &  &  \\
$6$ & $3q^{-4}$ & $3q^{-6}$ &  &  \\
$4$ & $7q^{-2}$ & $9q^{-4}$ & $2q^{-6}$ &  \\
$2$ & $8$ & $13q^{-2}$ & $5q^{-4}$ &  \\
$0$ & $7q^{2}$ & $18$ & $11q^{-2}$ & $q^{-4}$ \\
$-2$ & $3q^{4}$ & $13q^{2}$ & $12$ & $2q^{-2}$ \\
$-4$ & $q^{6}$ & $9q^{4}$ & $11q^{2}$ & $3$ \\
$-6$ &  & $3q^{6}$ & $5q^{4}$ & $2q^{2}$ \\
$-8$ &  & $q^{8}$ & $2q^{6}$ & $q^{4}$ \\
\end{tabular}
\vspace{2em}
\end{minipage}
%
\begin{minipage}{\linewidth}
$\bullet\ $ $11a_{316}$ \vspace{0.5em} \\
\begin{tabular}{l|llll}
$k \setminus j$ & $-2$ & $0$ & $2$ & $4$ \\
\hline
$8$ & $q^{-6}$ & $q^{-8}$ &  &  \\
$6$ & $2q^{-4}$ & $2q^{-6}$ &  &  \\
$4$ & $5q^{-2}$ & $7q^{-4}$ & $2q^{-6}$ &  \\
$2$ & $5$ & $9q^{-2}$ & $4q^{-4}$ &  \\
$0$ & $5q^{2}$ & $14$ & $9q^{-2}$ & $q^{-4}$ \\
$-2$ & $2q^{4}$ & $9q^{2}$ & $9$ & $2q^{-2}$ \\
$-4$ & $q^{6}$ & $7q^{4}$ & $9q^{2}$ & $3$ \\
$-6$ &  & $2q^{6}$ & $4q^{4}$ & $2q^{2}$ \\
$-8$ &  & $q^{8}$ & $2q^{6}$ & $q^{4}$ \\
\end{tabular}
\vspace{2em}
\end{minipage}
%
\begin{minipage}{\linewidth}
$\bullet\ $ $11a_{318}$ \vspace{0.5em} \\
\begin{tabular}{l|llll}
$k \setminus j$ & $6$ & $8$ & $10$ & $12$ \\
\hline
$6$ & $q^{-6}$ &  &  &  \\
$4$ & $3q^{-4}$ & $3q^{-6}$ &  &  \\
$2$ & $6q^{-2}$ & $7q^{-4}$ & $q^{-6}$ &  \\
$0$ & $7$ & $12q^{-2}$ & $5q^{-4}$ &  \\
$-2$ & $6q^{2}$ & $14$ & $9q^{-2}$ & $q^{-4}$ \\
$-4$ & $3q^{4}$ & $12q^{2}$ & $11$ & $2q^{-2}$ \\
$-6$ & $q^{6}$ & $7q^{4}$ & $9q^{2}$ & $3$ \\
$-8$ &  & $3q^{6}$ & $5q^{4}$ & $2q^{2}$ \\
$-10$ &  &  & $q^{6}$ & $q^{4}$ \\
\end{tabular}
\vspace{2em}
\end{minipage}
%
\begin{minipage}{\linewidth}
$\bullet\ $ $11a_{319}$ \vspace{0.5em} \\
\begin{tabular}{l|llll}
$k \setminus j$ & $6$ & $8$ & $10$ & $12$ \\
\hline
$6$ & $q^{-6}$ &  &  &  \\
$4$ & $3q^{-4}$ & $3q^{-6}$ &  &  \\
$2$ & $5q^{-2}$ & $6q^{-4}$ & $q^{-6}$ &  \\
$0$ & $6$ & $11q^{-2}$ & $5q^{-4}$ &  \\
$-2$ & $5q^{2}$ & $11$ & $7q^{-2}$ & $q^{-4}$ \\
$-4$ & $3q^{4}$ & $11q^{2}$ & $10$ & $2q^{-2}$ \\
$-6$ & $q^{6}$ & $6q^{4}$ & $7q^{2}$ & $2$ \\
$-8$ &  & $3q^{6}$ & $5q^{4}$ & $2q^{2}$ \\
$-10$ &  &  & $q^{6}$ & $q^{4}$ \\
\end{tabular}
\vspace{2em}
\end{minipage}
%
\begin{minipage}{\linewidth}
$\bullet\ $ $11a_{323}$ \vspace{0.5em} \\
\begin{tabular}{l|lllll}
$k \setminus j$ & $-4$ & $-2$ & $0$ & $2$ & $4$ \\
\hline
$6$ & $q^{-4}$ & $q^{-6}$ &  &  &  \\
$4$ & $2q^{-2}$ & $3q^{-4}$ & $q^{-6}$ &  &  \\
$2$ & $3$ & $6q^{-2}$ & $3q^{-4}$ &  &  \\
$0$ & $2q^{2}$ & $7$ & $7q^{-2}$ & $2q^{-4}$ &  \\
$-2$ & $q^{4}$ & $6q^{2}$ & $7$ & $3q^{-2}$ &  \\
$-4$ &  & $3q^{4}$ & $7q^{2}$ & $5$ & $q^{-2}$ \\
$-6$ &  & $q^{6}$ & $3q^{4}$ & $3q^{2}$ & $1$ \\
$-8$ &  &  & $q^{6}$ & $2q^{4}$ & $q^{2}$ \\
\end{tabular}
\vspace{2em}
\end{minipage}
%
\begin{minipage}{\linewidth}
$\bullet\ $ $11a_{326}$ \vspace{0.5em} \\
\begin{tabular}{l|llll}
$k \setminus j$ & $-2$ & $0$ & $2$ & $4$ \\
\hline
$8$ & $q^{-6}$ & $q^{-8}$ &  &  \\
$6$ & $3q^{-4}$ & $3q^{-6}$ &  &  \\
$4$ & $7q^{-2}$ & $9q^{-4}$ & $2q^{-6}$ &  \\
$2$ & $8$ & $14q^{-2}$ & $6q^{-4}$ &  \\
$0$ & $7q^{2}$ & $19$ & $12q^{-2}$ & $q^{-4}$ \\
$-2$ & $3q^{4}$ & $14q^{2}$ & $14$ & $3q^{-2}$ \\
$-4$ & $q^{6}$ & $9q^{4}$ & $12q^{2}$ & $4$ \\
$-6$ &  & $3q^{6}$ & $6q^{4}$ & $3q^{2}$ \\
$-8$ &  & $q^{8}$ & $2q^{6}$ & $q^{4}$ \\
\end{tabular}
\vspace{2em}
\end{minipage}
%
\begin{minipage}{\linewidth}
$\bullet\ $ $11a_{330}$ \vspace{0.5em} \\
\begin{tabular}{l|llll}
$k \setminus j$ & $-6$ & $-4$ & $-2$ & $0$ \\
\hline
$8$ & $q^{-6}$ & $q^{-8}$ &  &  \\
$6$ & $2q^{-4}$ & $2q^{-6}$ &  &  \\
$4$ & $4q^{-2}$ & $6q^{-4}$ & $2q^{-6}$ &  \\
$2$ & $3$ & $6q^{-2}$ & $3q^{-4}$ &  \\
$0$ & $4q^{2}$ & $9$ & $6q^{-2}$ & $q^{-4}$ \\
$-2$ & $2q^{4}$ & $6q^{2}$ & $5$ & $q^{-2}$ \\
$-4$ & $q^{6}$ & $6q^{4}$ & $6q^{2}$ & $2$ \\
$-6$ &  & $2q^{6}$ & $3q^{4}$ & $q^{2}$ \\
$-8$ &  & $q^{8}$ & $2q^{6}$ & $q^{4}$ \\
\end{tabular}
\vspace{2em}
\end{minipage}
%
\begin{minipage}{\linewidth}
$\bullet\ $ $11a_{332}$ \vspace{0.5em} \\
\begin{tabular}{l|llll}
$k \setminus j$ & $-2$ & $0$ & $2$ & $4$ \\
\hline
$8$ & $q^{-6}$ & $q^{-8}$ &  &  \\
$6$ & $4q^{-4}$ & $4q^{-6}$ &  &  \\
$4$ & $9q^{-2}$ & $11q^{-4}$ & $2q^{-6}$ &  \\
$2$ & $10$ & $16q^{-2}$ & $6q^{-4}$ &  \\
$0$ & $9q^{2}$ & $22$ & $13q^{-2}$ & $q^{-4}$ \\
$-2$ & $4q^{4}$ & $16q^{2}$ & $14$ & $2q^{-2}$ \\
$-4$ & $q^{6}$ & $11q^{4}$ & $13q^{2}$ & $3$ \\
$-6$ &  & $4q^{6}$ & $6q^{4}$ & $2q^{2}$ \\
$-8$ &  & $q^{8}$ & $2q^{6}$ & $q^{4}$ \\
\end{tabular}
\vspace{2em}
\end{minipage}
%
\begin{minipage}{\linewidth}
$\bullet\ $ $11a_{333}$ \vspace{0.5em} \\
\begin{tabular}{l|llllll}
$k \setminus j$ & $-8$ & $-6$ & $-4$ & $-2$ & $0$ & $2$ \\
\hline
$8$ & $1$ & $2q^{-2}$ & $q^{-4}$ &  &  &  \\
$6$ &  & $2$ & $4q^{-2}$ & $2q^{-4}$ &  &  \\
$4$ &  & $2q^{2}$ & $6$ & $5q^{-2}$ & $q^{-4}$ &  \\
$2$ &  &  & $4q^{2}$ & $8$ & $4q^{-2}$ &  \\
$0$ &  &  & $q^{4}$ & $5q^{2}$ & $6$ & $q^{-2}$ \\
$-2$ &  &  &  & $2q^{4}$ & $4q^{2}$ & $2$ \\
$-4$ &  &  &  &  & $q^{4}$ & $q^{2}$ \\
\end{tabular}
\vspace{2em}
\end{minipage}
%
\begin{minipage}{\linewidth}
$\bullet\ $ $11a_{334}$ \vspace{0.5em} \\
\begin{tabular}{l|lll}
$k \setminus j$ & $8$ & $10$ & $12$ \\
\hline
$8$ & $q^{-8}$ &  &  \\
$6$ & $q^{-6}$ & $q^{-8}$ &  \\
$4$ & $2q^{-4}$ & $2q^{-6}$ &  \\
$2$ & $2q^{-2}$ & $3q^{-4}$ & $q^{-6}$ \\
$0$ & $3$ & $4q^{-2}$ & $q^{-4}$ \\
$-2$ & $2q^{2}$ & $4$ & $2q^{-2}$ \\
$-4$ & $2q^{4}$ & $4q^{2}$ & $2$ \\
$-6$ & $q^{6}$ & $3q^{4}$ & $2q^{2}$ \\
$-8$ & $q^{8}$ & $2q^{6}$ & $q^{4}$ \\
$-10$ &  & $q^{8}$ & $q^{6}$ \\
\end{tabular}
\vspace{2em}
\end{minipage}
%
\begin{minipage}{\linewidth}
$\bullet\ $ $11a_{335}$ \vspace{0.5em} \\
\begin{tabular}{l|llll}
$k \setminus j$ & $6$ & $8$ & $10$ & $12$ \\
\hline
$6$ & $q^{-6}$ &  &  &  \\
$4$ & $2q^{-4}$ & $2q^{-6}$ &  &  \\
$2$ & $3q^{-2}$ & $4q^{-4}$ & $q^{-6}$ &  \\
$0$ & $3$ & $6q^{-2}$ & $3q^{-4}$ &  \\
$-2$ & $3q^{2}$ & $7$ & $5q^{-2}$ & $q^{-4}$ \\
$-4$ & $2q^{4}$ & $6q^{2}$ & $5$ & $q^{-2}$ \\
$-6$ & $q^{6}$ & $4q^{4}$ & $5q^{2}$ & $2$ \\
$-8$ &  & $2q^{6}$ & $3q^{4}$ & $q^{2}$ \\
$-10$ &  &  & $q^{6}$ & $q^{4}$ \\
\end{tabular}
\vspace{2em}
\end{minipage}
%
\begin{minipage}{\linewidth}
$\bullet\ $ $11a_{336}$ \vspace{0.5em} \\
\begin{tabular}{l|llll}
$k \setminus j$ & $6$ & $8$ & $10$ & $12$ \\
\hline
$6$ & $q^{-6}$ &  &  &  \\
$4$ & $2q^{-4}$ & $2q^{-6}$ &  &  \\
$2$ & $2q^{-2}$ & $3q^{-4}$ & $q^{-6}$ &  \\
$0$ & $2$ & $5q^{-2}$ & $3q^{-4}$ &  \\
$-2$ & $2q^{2}$ & $4$ & $3q^{-2}$ & $q^{-4}$ \\
$-4$ & $2q^{4}$ & $5q^{2}$ & $4$ & $q^{-2}$ \\
$-6$ & $q^{6}$ & $3q^{4}$ & $3q^{2}$ & $1$ \\
$-8$ &  & $2q^{6}$ & $3q^{4}$ & $q^{2}$ \\
$-10$ &  &  & $q^{6}$ & $q^{4}$ \\
\end{tabular}
\vspace{2em}
\end{minipage}
%
\begin{minipage}{\linewidth}
$\bullet\ $ $11a_{338}$ \vspace{0.5em} \\
\begin{tabular}{l|lll}
$k \setminus j$ & $8$ & $10$ & $12$ \\
\hline
$8$ & $q^{-8}$ &  &  \\
$6$ & $q^{-6}$ & $q^{-8}$ &  \\
$4$ & $3q^{-4}$ & $3q^{-6}$ &  \\
$2$ & $3q^{-2}$ & $4q^{-4}$ & $q^{-6}$ \\
$0$ & $4$ & $6q^{-2}$ & $2q^{-4}$ \\
$-2$ & $3q^{2}$ & $6$ & $3q^{-2}$ \\
$-4$ & $3q^{4}$ & $6q^{2}$ & $3$ \\
$-6$ & $q^{6}$ & $4q^{4}$ & $3q^{2}$ \\
$-8$ & $q^{8}$ & $3q^{6}$ & $2q^{4}$ \\
$-10$ &  & $q^{8}$ & $q^{6}$ \\
\end{tabular}
\vspace{2em}
\end{minipage}
%
\begin{minipage}{\linewidth}
$\bullet\ $ $11a_{339}$ \vspace{0.5em} \\
\begin{tabular}{l|llll}
$k \setminus j$ & $6$ & $8$ & $10$ & $12$ \\
\hline
$6$ & $q^{-6}$ &  &  &  \\
$4$ & $q^{-4}$ & $q^{-6}$ &  &  \\
$2$ & $2q^{-2}$ & $3q^{-4}$ & $q^{-6}$ &  \\
$0$ & $2$ & $4q^{-2}$ & $2q^{-4}$ &  \\
$-2$ & $2q^{2}$ & $5$ & $4q^{-2}$ & $q^{-4}$ \\
$-4$ & $q^{4}$ & $4q^{2}$ & $4$ & $q^{-2}$ \\
$-6$ & $q^{6}$ & $3q^{4}$ & $4q^{2}$ & $2$ \\
$-8$ &  & $q^{6}$ & $2q^{4}$ & $q^{2}$ \\
$-10$ &  &  & $q^{6}$ & $q^{4}$ \\
\end{tabular}
\vspace{2em}
\end{minipage}
%
\begin{minipage}{\linewidth}
$\bullet\ $ $11a_{340}$ \vspace{0.5em} \\
\begin{tabular}{l|llll}
$k \setminus j$ & $6$ & $8$ & $10$ & $12$ \\
\hline
$6$ & $q^{-6}$ &  &  &  \\
$4$ & $2q^{-4}$ & $2q^{-6}$ &  &  \\
$2$ & $3q^{-2}$ & $4q^{-4}$ & $q^{-6}$ &  \\
$0$ & $4$ & $8q^{-2}$ & $4q^{-4}$ &  \\
$-2$ & $3q^{2}$ & $7$ & $5q^{-2}$ & $q^{-4}$ \\
$-4$ & $2q^{4}$ & $8q^{2}$ & $8$ & $2q^{-2}$ \\
$-6$ & $q^{6}$ & $4q^{4}$ & $5q^{2}$ & $2$ \\
$-8$ &  & $2q^{6}$ & $4q^{4}$ & $2q^{2}$ \\
$-10$ &  &  & $q^{6}$ & $q^{4}$ \\
\end{tabular}
\vspace{2em}
\end{minipage}
%
\begin{minipage}{\linewidth}
$\bullet\ $ $11a_{346}$ \vspace{0.5em} \\
\begin{tabular}{l|llll}
$k \setminus j$ & $0$ & $2$ & $4$ & $6$ \\
\hline
$8$ & $q^{-4}$ & $2q^{-6}$ & $q^{-8}$ &  \\
$6$ & $q^{-2}$ & $3q^{-4}$ & $2q^{-6}$ &  \\
$4$ & $2$ & $6q^{-2}$ & $6q^{-4}$ & $q^{-6}$ \\
$2$ & $q^{2}$ & $6$ & $7q^{-2}$ & $2q^{-4}$ \\
$0$ & $q^{4}$ & $6q^{2}$ & $9$ & $4q^{-2}$ \\
$-2$ &  & $3q^{4}$ & $7q^{2}$ & $4$ \\
$-4$ &  & $2q^{6}$ & $6q^{4}$ & $4q^{2}$ \\
$-6$ &  &  & $2q^{6}$ & $2q^{4}$ \\
$-8$ &  &  & $q^{8}$ & $q^{6}$ \\
\end{tabular}
\vspace{2em}
\end{minipage}
%
\begin{minipage}{\linewidth}
$\bullet\ $ $11a_{348}$ \vspace{0.5em} \\
\begin{tabular}{l|llll}
$k \setminus j$ & $2$ & $4$ & $6$ & $8$ \\
\hline
$8$ & $q^{-6}$ & $q^{-8}$ &  &  \\
$6$ & $4q^{-4}$ & $4q^{-6}$ &  &  \\
$4$ & $5q^{-2}$ & $8q^{-4}$ & $2q^{-6}$ &  \\
$2$ & $7$ & $13q^{-2}$ & $6q^{-4}$ &  \\
$0$ & $5q^{2}$ & $13$ & $9q^{-2}$ & $q^{-4}$ \\
$-2$ & $4q^{4}$ & $13q^{2}$ & $11$ & $2q^{-2}$ \\
$-4$ & $q^{6}$ & $8q^{4}$ & $9q^{2}$ & $2$ \\
$-6$ &  & $4q^{6}$ & $6q^{4}$ & $2q^{2}$ \\
$-8$ &  & $q^{8}$ & $2q^{6}$ & $q^{4}$ \\
\end{tabular}
\vspace{2em}
\end{minipage}
%
\begin{minipage}{\linewidth}
$\bullet\ $ $11a_{349}$ \vspace{0.5em} \\
\begin{tabular}{l|lllll}
$k \setminus j$ & $0$ & $2$ & $4$ & $6$ & $8$ \\
\hline
$6$ & $q^{-4}$ & $q^{-6}$ &  &  &  \\
$4$ & $4q^{-2}$ & $5q^{-4}$ & $q^{-6}$ &  &  \\
$2$ & $5$ & $12q^{-2}$ & $6q^{-4}$ &  &  \\
$0$ & $4q^{2}$ & $15$ & $13q^{-2}$ & $2q^{-4}$ &  \\
$-2$ & $q^{4}$ & $12q^{2}$ & $18$ & $7q^{-2}$ &  \\
$-4$ &  & $5q^{4}$ & $13q^{2}$ & $9$ & $q^{-2}$ \\
$-6$ &  & $q^{6}$ & $6q^{4}$ & $7q^{2}$ & $2$ \\
$-8$ &  &  & $q^{6}$ & $2q^{4}$ & $q^{2}$ \\
\end{tabular}
\vspace{2em}
\end{minipage}
%
\begin{minipage}{\linewidth}
$\bullet\ $ $11a_{350}$ \vspace{0.5em} \\
\begin{tabular}{l|llll}
$k \setminus j$ & $-2$ & $0$ & $2$ & $4$ \\
\hline
$8$ & $q^{-6}$ & $q^{-8}$ &  &  \\
$6$ & $4q^{-4}$ & $4q^{-6}$ &  &  \\
$4$ & $7q^{-2}$ & $9q^{-4}$ & $2q^{-6}$ &  \\
$2$ & $10$ & $17q^{-2}$ & $7q^{-4}$ &  \\
$0$ & $7q^{2}$ & $19$ & $12q^{-2}$ & $q^{-4}$ \\
$-2$ & $4q^{4}$ & $17q^{2}$ & $16$ & $3q^{-2}$ \\
$-4$ & $q^{6}$ & $9q^{4}$ & $12q^{2}$ & $4$ \\
$-6$ &  & $4q^{6}$ & $7q^{4}$ & $3q^{2}$ \\
$-8$ &  & $q^{8}$ & $2q^{6}$ & $q^{4}$ \\
\end{tabular}
\vspace{2em}
\end{minipage}
%
\begin{minipage}{\linewidth}
$\bullet\ $ $11a_{351}$ \vspace{0.5em} \\
\begin{tabular}{l|llll}
$k \setminus j$ & $-2$ & $0$ & $2$ & $4$ \\
\hline
$8$ & $q^{-6}$ & $q^{-8}$ &  &  \\
$6$ & $4q^{-4}$ & $4q^{-6}$ &  &  \\
$4$ & $7q^{-2}$ & $9q^{-4}$ & $2q^{-6}$ &  \\
$2$ & $9$ & $15q^{-2}$ & $6q^{-4}$ &  \\
$0$ & $7q^{2}$ & $17$ & $10q^{-2}$ & $q^{-4}$ \\
$-2$ & $4q^{4}$ & $15q^{2}$ & $13$ & $2q^{-2}$ \\
$-4$ & $q^{6}$ & $9q^{4}$ & $10q^{2}$ & $2$ \\
$-6$ &  & $4q^{6}$ & $6q^{4}$ & $2q^{2}$ \\
$-8$ &  & $q^{8}$ & $2q^{6}$ & $q^{4}$ \\
\end{tabular}
\vspace{2em}
\end{minipage}
%
\begin{minipage}{\linewidth}
$\bullet\ $ $11a_{353}$ \vspace{0.5em} \\
\begin{tabular}{l|llll}
$k \setminus j$ & $6$ & $8$ & $10$ & $12$ \\
\hline
$6$ & $q^{-6}$ &  &  &  \\
$4$ & $3q^{-4}$ & $3q^{-6}$ &  &  \\
$2$ & $5q^{-2}$ & $6q^{-4}$ & $q^{-6}$ &  \\
$0$ & $6$ & $11q^{-2}$ & $5q^{-4}$ &  \\
$-2$ & $5q^{2}$ & $12$ & $8q^{-2}$ & $q^{-4}$ \\
$-4$ & $3q^{4}$ & $11q^{2}$ & $10$ & $2q^{-2}$ \\
$-6$ & $q^{6}$ & $6q^{4}$ & $8q^{2}$ & $3$ \\
$-8$ &  & $3q^{6}$ & $5q^{4}$ & $2q^{2}$ \\
$-10$ &  &  & $q^{6}$ & $q^{4}$ \\
\end{tabular}
\vspace{2em}
\end{minipage}
%
\begin{minipage}{\linewidth}
$\bullet\ $ $11a_{355}$ \vspace{0.5em} \\
\begin{tabular}{l|lll}
$k \setminus j$ & $8$ & $10$ & $12$ \\
\hline
$8$ & $q^{-8}$ &  &  \\
$6$ & $q^{-6}$ & $q^{-8}$ &  \\
$4$ & $2q^{-4}$ & $2q^{-6}$ &  \\
$2$ & $2q^{-2}$ & $3q^{-4}$ & $q^{-6}$ \\
$0$ & $2$ & $3q^{-2}$ & $q^{-4}$ \\
$-2$ & $2q^{2}$ & $4$ & $2q^{-2}$ \\
$-4$ & $2q^{4}$ & $3q^{2}$ & $1$ \\
$-6$ & $q^{6}$ & $3q^{4}$ & $2q^{2}$ \\
$-8$ & $q^{8}$ & $2q^{6}$ & $q^{4}$ \\
$-10$ &  & $q^{8}$ & $q^{6}$ \\
\end{tabular}
\vspace{2em}
\end{minipage}
%
\begin{minipage}{\linewidth}
$\bullet\ $ $11a_{356}$ \vspace{0.5em} \\
\begin{tabular}{l|llll}
$k \setminus j$ & $6$ & $8$ & $10$ & $12$ \\
\hline
$6$ & $q^{-6}$ &  &  &  \\
$4$ & $2q^{-4}$ & $2q^{-6}$ &  &  \\
$2$ & $3q^{-2}$ & $4q^{-4}$ & $q^{-6}$ &  \\
$0$ & $4$ & $7q^{-2}$ & $3q^{-4}$ &  \\
$-2$ & $3q^{2}$ & $7$ & $5q^{-2}$ & $q^{-4}$ \\
$-4$ & $2q^{4}$ & $7q^{2}$ & $6$ & $q^{-2}$ \\
$-6$ & $q^{6}$ & $4q^{4}$ & $5q^{2}$ & $2$ \\
$-8$ &  & $2q^{6}$ & $3q^{4}$ & $q^{2}$ \\
$-10$ &  &  & $q^{6}$ & $q^{4}$ \\
\end{tabular}
\vspace{2em}
\end{minipage}
%
\begin{minipage}{\linewidth}
$\bullet\ $ $11a_{357}$ \vspace{0.5em} \\
\begin{tabular}{l|llll}
$k \setminus j$ & $6$ & $8$ & $10$ & $12$ \\
\hline
$6$ & $q^{-6}$ &  &  &  \\
$4$ & $2q^{-4}$ & $2q^{-6}$ &  &  \\
$2$ & $3q^{-2}$ & $4q^{-4}$ & $q^{-6}$ &  \\
$0$ & $4$ & $8q^{-2}$ & $4q^{-4}$ &  \\
$-2$ & $3q^{2}$ & $8$ & $6q^{-2}$ & $q^{-4}$ \\
$-4$ & $2q^{4}$ & $8q^{2}$ & $8$ & $2q^{-2}$ \\
$-6$ & $q^{6}$ & $4q^{4}$ & $6q^{2}$ & $3$ \\
$-8$ &  & $2q^{6}$ & $4q^{4}$ & $2q^{2}$ \\
$-10$ &  &  & $q^{6}$ & $q^{4}$ \\
\end{tabular}
\vspace{2em}
\end{minipage}
%
\begin{minipage}{\linewidth}
$\bullet\ $ $11a_{358}$ \vspace{0.5em} \\
\begin{tabular}{l|llll}
$k \setminus j$ & $6$ & $8$ & $10$ & $12$ \\
\hline
$6$ & $q^{-6}$ &  &  &  \\
$4$ & $q^{-4}$ & $q^{-6}$ &  &  \\
$2$ & $q^{-2}$ & $2q^{-4}$ & $q^{-6}$ &  \\
$0$ & $1$ & $2q^{-2}$ & $q^{-4}$ &  \\
$-2$ & $q^{2}$ & $2$ & $2q^{-2}$ & $q^{-4}$ \\
$-4$ & $q^{4}$ & $2q^{2}$ & $1$ &  \\
$-6$ & $q^{6}$ & $2q^{4}$ & $2q^{2}$ & $1$ \\
$-8$ &  & $q^{6}$ & $q^{4}$ &  \\
$-10$ &  &  & $q^{6}$ & $q^{4}$ \\
\end{tabular}
\vspace{2em}
\end{minipage}
%
\begin{minipage}{\linewidth}
$\bullet\ $ $11a_{364}$ \vspace{0.5em} \\
\begin{tabular}{l|lll}
$k \setminus j$ & $8$ & $10$ & $12$ \\
\hline
$8$ & $q^{-8}$ &  &  \\
$6$ & $q^{-6}$ & $q^{-8}$ &  \\
$4$ & $q^{-4}$ & $q^{-6}$ &  \\
$2$ & $q^{-2}$ & $2q^{-4}$ & $q^{-6}$ \\
$0$ & $1$ & $q^{-2}$ &  \\
$-2$ & $q^{2}$ & $2$ & $q^{-2}$ \\
$-4$ & $q^{4}$ & $q^{2}$ &  \\
$-6$ & $q^{6}$ & $2q^{4}$ & $q^{2}$ \\
$-8$ & $q^{8}$ & $q^{6}$ &  \\
$-10$ &  & $q^{8}$ & $q^{6}$ \\
\end{tabular}
\vspace{2em}
\end{minipage}
%
\begin{minipage}{\linewidth}
$\bullet\ $ $11a_{365}$ \vspace{0.5em} \\
\begin{tabular}{l|llll}
$k \setminus j$ & $6$ & $8$ & $10$ & $12$ \\
\hline
$6$ & $q^{-6}$ &  &  &  \\
$4$ & $2q^{-4}$ & $2q^{-6}$ &  &  \\
$2$ & $2q^{-2}$ & $3q^{-4}$ & $q^{-6}$ &  \\
$0$ & $2$ & $4q^{-2}$ & $2q^{-4}$ &  \\
$-2$ & $2q^{2}$ & $4$ & $3q^{-2}$ & $q^{-4}$ \\
$-4$ & $2q^{4}$ & $4q^{2}$ & $2$ &  \\
$-6$ & $q^{6}$ & $3q^{4}$ & $3q^{2}$ & $1$ \\
$-8$ &  & $2q^{6}$ & $2q^{4}$ &  \\
$-10$ &  &  & $q^{6}$ & $q^{4}$ \\
\end{tabular}
\vspace{2em}
\end{minipage}
%
\begin{minipage}{\linewidth}
$\bullet\ $ $11a_{367}$ \vspace{0.5em} \\
\begin{tabular}{l|ll}
$k \setminus j$ & $10$ & $12$ \\
\hline
$10$ & $q^{-10}$ &  \\
$6$ & $q^{-6}$ & $q^{-8}$ \\
$2$ & $q^{-2}$ & $q^{-4}$ \\
$-2$ & $q^{2}$ & $1$ \\
$-6$ & $q^{6}$ & $q^{4}$ \\
$-10$ & $q^{10}$ & $q^{8}$ \\
\end{tabular}
\vspace{2em}
\end{minipage}
%
\begin{minipage}{\linewidth}
$\bullet\ $ $11n_{1}$ \vspace{0.5em} \\
\begin{tabular}{l|lllll}
$k \setminus j$ & $-10$ & $-8$ & $-6$ & $-4$ & $-2$ \\
\hline
$8$ & $1$ & $2q^{-2}$ & $q^{-4}$ &  &  \\
$6$ &  & $2$ & $2q^{-2}$ &  &  \\
$4$ &  & $2q^{2}$ & $4$ & $2q^{-2}$ &  \\
$2$ &  &  & $2q^{2}$ & $3$ & $q^{-2}$ \\
$0$ &  &  & $q^{4}$ & $2q^{2}$ & $1$ \\
$-2$ &  &  &  &  & $q^{2}$ \\
\end{tabular}
\vspace{2em}
\end{minipage}
%
\begin{minipage}{\linewidth}
$\bullet\ $ $11n_{2}$ \vspace{0.5em} \\
\begin{tabular}{l|lllll}
$k \setminus j$ & $2$ & $4$ & $6$ & $8$ & $10$ \\
\hline
$6$ & $q^{-4}$ & $q^{-6}$ &  &  &  \\
$4$ & $q^{-2}$ & $3q^{-4}$ & $q^{-6}$ &  &  \\
$2$ & $2$ & $5q^{-2}$ & $3q^{-4}$ &  &  \\
$0$ & $q^{2}$ & $5$ & $5q^{-2}$ & $q^{-4}$ &  \\
$-2$ & $q^{4}$ & $5q^{2}$ & $6$ & $2q^{-2}$ &  \\
$-4$ &  & $3q^{4}$ & $1$ + $5q^{2}$ & $q^{-2}$ + $2$ &  \\
$-6$ &  & $q^{6}$ & $3q^{4}$ & $2q^{2}$ &  \\
$-8$ &  &  & $q^{6}$ & $q^{2}$ + $q^{4}$ & $1$ \\
\end{tabular}
\vspace{2em}
\end{minipage}
%
\begin{minipage}{\linewidth}
$\bullet\ $ $11n_{3}$ \vspace{0.5em} \\
\begin{tabular}{l|lllll}
$k \setminus j$ & $-6$ & $-4$ & $-2$ & $0$ & $2$ \\
\hline
$4$ & $1$ & $3q^{-2}$ & $2q^{-4}$ &  &  \\
$2$ &  & $3$ & $4q^{-2}$ & $q^{-4}$ &  \\
$0$ &  & $3q^{2}$ & $6$ & $3q^{-2}$ &  \\
$-2$ &  &  & $4q^{2}$ & $4$ & $q^{-2}$ \\
$-4$ &  &  & $2q^{4}$ & $3q^{2}$ & $1$ \\
$-6$ &  &  &  & $q^{4}$ & $q^{2}$ \\
\end{tabular}
\vspace{2em}
\end{minipage}
%
\begin{minipage}{\linewidth}
$\bullet\ $ $11n_{4}$ \vspace{0.5em} \\
\begin{tabular}{l|llll}
$k \setminus j$ & $-2$ & $0$ & $2$ & $4$ \\
\hline
$4$ & $q^{-2}$ & $2q^{-4}$ & $q^{-6}$ &  \\
$2$ & $1$ & $3q^{-2}$ & $2q^{-4}$ &  \\
$0$ & $q^{2}$ & $6$ & $5q^{-2}$ & $q^{-4}$ \\
$-2$ &  & $3q^{2}$ & $5$ & $2q^{-2}$ \\
$-4$ &  & $2q^{4}$ & $5q^{2}$ & $3$ \\
$-6$ &  &  & $2q^{4}$ & $2q^{2}$ \\
$-8$ &  &  & $q^{6}$ & $q^{4}$ \\
\end{tabular}
\vspace{2em}
\end{minipage}
%
\begin{minipage}{\linewidth}
$\bullet\ $ $11n_{5}$ \vspace{0.5em} \\
\begin{tabular}{l|llll}
$k \setminus j$ & $0$ & $2$ & $4$ & $6$ \\
\hline
$4$ & $2q^{-2}$ & $3q^{-4}$ & $q^{-6}$ &  \\
$2$ & $1$ & $5q^{-2}$ & $3q^{-4}$ &  \\
$0$ & $2q^{2}$ & $8$ & $7q^{-2}$ & $q^{-4}$ \\
$-2$ &  & $5q^{2}$ & $8$ & $3q^{-2}$ \\
$-4$ &  & $3q^{4}$ & $7q^{2}$ & $4$ \\
$-6$ &  &  & $3q^{4}$ & $3q^{2}$ \\
$-8$ &  &  & $q^{6}$ & $q^{4}$ \\
\end{tabular}
\vspace{2em}
\end{minipage}
%
\begin{minipage}{\linewidth}
$\bullet\ $ $11n_{6}$ \vspace{0.5em} \\
\begin{tabular}{l|lllll}
$k \setminus j$ & $-6$ & $-4$ & $-2$ & $0$ & $2$ \\
\hline
$8$ & $q^{-2}$ & $2q^{-4}$ & $q^{-6}$ &  &  \\
$6$ &  & $q^{-2}$ & $q^{-4}$ &  &  \\
$4$ & $q^{2}$ & $3$ & $3q^{-2}$ & $q^{-4}$ &  \\
$2$ &  & $q^{2}$ & $q^{-2}$ + $2$ & $q^{-4}$ + $q^{-2}$ &  \\
$0$ &  & $2q^{4}$ & $1$ + $3q^{2}$ & $q^{-2}$ + $2$ &  \\
$-2$ &  &  & $q^{2}$ + $q^{4}$ & $2$ + $q^{2}$ & $q^{-2}$ \\
$-4$ &  &  & $q^{6}$ & $q^{2}$ + $q^{4}$ & $1$ \\
$-6$ &  &  &  & $q^{4}$ & $q^{2}$ \\
\end{tabular}
\vspace{2em}
\end{minipage}
%
\begin{minipage}{\linewidth}
$\bullet\ $ $11n_{7}$ \vspace{0.5em} \\
\begin{tabular}{l|lllll}
$k \setminus j$ & $-2$ & $0$ & $2$ & $4$ & $6$ \\
\hline
$6$ & $q^{-2}$ & $2q^{-4}$ & $q^{-6}$ &  &  \\
$4$ & $1$ & $4q^{-2}$ & $3q^{-4}$ &  &  \\
$2$ & $q^{2}$ & $6$ & $7q^{-2}$ & $q^{-4}$ &  \\
$0$ &  & $4q^{2}$ & $8$ & $4q^{-2}$ &  \\
$-2$ &  & $2q^{4}$ & $7q^{2}$ & $5$ &  \\
$-4$ &  &  & $3q^{4}$ & $4q^{2}$ & $1$ \\
$-6$ &  &  & $q^{6}$ & $q^{4}$ &  \\
\end{tabular}
\vspace{2em}
\end{minipage}
%
\begin{minipage}{\linewidth}
$\bullet\ $ $11n_{8}$ \vspace{0.5em} \\
\begin{tabular}{l|llll}
$k \setminus j$ & $-8$ & $-6$ & $-4$ & $-2$ \\
\hline
$6$ & $q^{-2}$ & $2q^{-4}$ & $q^{-6}$ &  \\
$4$ & $1$ & $4q^{-2}$ & $3q^{-4}$ &  \\
$2$ & $q^{2}$ & $5$ & $5q^{-2}$ & $q^{-4}$ \\
$0$ &  & $4q^{2}$ & $6$ & $2q^{-2}$ \\
$-2$ &  & $2q^{4}$ & $5q^{2}$ & $3$ \\
$-4$ &  &  & $3q^{4}$ & $2q^{2}$ \\
$-6$ &  &  & $q^{6}$ & $q^{4}$ \\
\end{tabular}
\vspace{2em}
\end{minipage}
%
\begin{minipage}{\linewidth}
$\bullet\ $ $11n_{9}$ \vspace{0.5em} \\
\begin{tabular}{l|llll}
$k \setminus j$ & $4$ & $6$ & $8$ & $10$ \\
\hline
$8$ & $q^{-6}$ & $q^{-8}$ &  &  \\
$6$ &  & $q^{-6}$ &  &  \\
$4$ & $q^{-2}$ & $2q^{-4}$ & $q^{-6}$ &  \\
$2$ &  & $q^{-2}$ & $q^{-4}$ &  \\
$0$ & $1$ + $q^{2}$ & $3q^{-2}$ + $2$ & $2q^{-4}$ + $q^{-2}$ &  \\
$-2$ &  & $2$ + $q^{2}$ & $2q^{-2}$ + $1$ &  \\
$-4$ & $q^{6}$ & $3q^{2}$ + $2q^{4}$ & $5$ + $q^{2}$ & $2q^{-2}$ \\
$-6$ &  & $q^{6}$ & $2q^{2}$ + $q^{4}$ & $2$ \\
$-8$ &  & $q^{8}$ & $2q^{4}$ + $q^{6}$ & $2q^{2}$ \\
\end{tabular}
\vspace{2em}
\end{minipage}
%
\begin{minipage}{\linewidth}
$\bullet\ $ $11n_{10}$ \vspace{0.5em} \\
\begin{tabular}{l|llll}
$k \setminus j$ & $-10$ & $-8$ & $-6$ & $-4$ \\
\hline
$8$ & $q^{-2}$ & $2q^{-4}$ & $q^{-6}$ &  \\
$6$ & $1$ & $4q^{-2}$ & $3q^{-4}$ &  \\
$4$ & $q^{2}$ & $6$ & $7q^{-2}$ & $2q^{-4}$ \\
$2$ &  & $4q^{2}$ & $7$ & $3q^{-2}$ \\
$0$ &  & $2q^{4}$ & $7q^{2}$ & $5$ \\
$-2$ &  &  & $3q^{4}$ & $3q^{2}$ \\
$-4$ &  &  & $q^{6}$ & $2q^{4}$ \\
\end{tabular}
\vspace{2em}
\end{minipage}
%
\begin{minipage}{\linewidth}
$\bullet\ $ $11n_{11}$ \vspace{0.5em} \\
\begin{tabular}{l|llll}
$k \setminus j$ & $0$ & $2$ & $4$ & $6$ \\
\hline
$4$ & $q^{-2}$ & $2q^{-4}$ & $q^{-6}$ &  \\
$2$ & $1$ & $4q^{-2}$ & $2q^{-4}$ &  \\
$0$ & $q^{2}$ & $6$ & $6q^{-2}$ & $q^{-4}$ \\
$-2$ &  & $4q^{2}$ & $6$ & $2q^{-2}$ \\
$-4$ &  & $2q^{4}$ & $6q^{2}$ & $4$ \\
$-6$ &  &  & $2q^{4}$ & $2q^{2}$ \\
$-8$ &  &  & $q^{6}$ & $q^{4}$ \\
\end{tabular}
\vspace{2em}
\end{minipage}
%
\begin{minipage}{\linewidth}
$\bullet\ $ $11n_{12}$ \vspace{0.5em} \\
\begin{tabular}{l|llll}
$k \setminus j$ & $0$ & $2$ & $4$ & $6$ \\
\hline
$2$ &  & $q^{-2}$ &  &  \\
$0$ & $1$ & $2q^{-2}$ & $q^{-4}$ &  \\
$-2$ &  & $1$ + $q^{2}$ & $q^{-2}$ + $1$ &  \\
$-4$ &  & $2q^{2}$ & $3$ & $q^{-2}$ \\
$-6$ &  &  & $q^{2}$ & $1$ \\
$-8$ &  &  & $q^{4}$ & $q^{2}$ \\
\end{tabular}
\vspace{2em}
\end{minipage}
%
\begin{minipage}{\linewidth}
$\bullet\ $ $11n_{13}$ \vspace{0.5em} \\
\begin{tabular}{l|llll}
$k \setminus j$ & $4$ & $6$ & $8$ & $10$ \\
\hline
$8$ & $q^{-6}$ & $q^{-8}$ &  &  \\
$6$ &  & $q^{-6}$ &  &  \\
$4$ & $q^{-2}$ & $2q^{-4}$ & $q^{-6}$ &  \\
$2$ &  & $q^{-2}$ & $q^{-4}$ &  \\
$0$ & $q^{2}$ & $q^{-2}$ + $2$ & $q^{-4}$ + $q^{-2}$ &  \\
$-2$ &  & $q^{2}$ & $1$ &  \\
$-4$ & $q^{6}$ & $q^{2}$ + $2q^{4}$ & $2$ + $q^{2}$ & $q^{-2}$ \\
$-6$ &  & $q^{6}$ & $q^{4}$ &  \\
$-8$ &  & $q^{8}$ & $q^{4}$ + $q^{6}$ & $q^{2}$ \\
\end{tabular}
\vspace{2em}
\end{minipage}
%
\begin{minipage}{\linewidth}
$\bullet\ $ $11n_{14}$ \vspace{0.5em} \\
\begin{tabular}{l|llll}
$k \setminus j$ & $-10$ & $-8$ & $-6$ & $-4$ \\
\hline
$8$ & $q^{-2}$ & $2q^{-4}$ & $q^{-6}$ &  \\
$6$ &  & $2q^{-2}$ & $2q^{-4}$ &  \\
$4$ & $q^{2}$ & $4$ & $5q^{-2}$ & $2q^{-4}$ \\
$2$ &  & $2q^{2}$ & $4$ & $2q^{-2}$ \\
$0$ &  & $2q^{4}$ & $5q^{2}$ & $3$ \\
$-2$ &  &  & $2q^{4}$ & $2q^{2}$ \\
$-4$ &  &  & $q^{6}$ & $2q^{4}$ \\
\end{tabular}
\vspace{2em}
\end{minipage}
%
\begin{minipage}{\linewidth}
$\bullet\ $ $11n_{15}$ \vspace{0.5em} \\
\begin{tabular}{l|llll}
$k \setminus j$ & $0$ & $2$ & $4$ & $6$ \\
\hline
$4$ & $q^{-2}$ & $2q^{-4}$ & $q^{-6}$ &  \\
$2$ &  & $2q^{-2}$ & $q^{-4}$ &  \\
$0$ & $q^{2}$ & $4$ & $4q^{-2}$ & $q^{-4}$ \\
$-2$ &  & $2q^{2}$ & $3$ & $q^{-2}$ \\
$-4$ &  & $2q^{4}$ & $4q^{2}$ & $2$ \\
$-6$ &  &  & $q^{4}$ & $q^{2}$ \\
$-8$ &  &  & $q^{6}$ & $q^{4}$ \\
\end{tabular}
\vspace{2em}
\end{minipage}
%
\begin{minipage}{\linewidth}
$\bullet\ $ $11n_{16}$ \vspace{0.5em} \\
\begin{tabular}{l|lllll}
$k \setminus j$ & $2$ & $4$ & $6$ & $8$ & $10$ \\
\hline
$6$ & $q^{-4}$ & $q^{-6}$ &  &  &  \\
$4$ & $q^{-2}$ & $3q^{-4}$ & $q^{-6}$ &  &  \\
$2$ & $1$ & $3q^{-2}$ & $2q^{-4}$ &  &  \\
$0$ & $q^{2}$ & $4$ & $4q^{-2}$ & $q^{-4}$ &  \\
$-2$ & $q^{4}$ & $1$ + $3q^{2}$ & $q^{-2}$ + $3$ & $q^{-2}$ &  \\
$-4$ &  & $3q^{4}$ & $1$ + $4q^{2}$ & $q^{-2}$ + $1$ &  \\
$-6$ &  & $q^{6}$ & $q^{2}$ + $2q^{4}$ & $1$ + $q^{2}$ &  \\
$-8$ &  &  & $q^{6}$ & $q^{2}$ + $q^{4}$ & $1$ \\
\end{tabular}
\vspace{2em}
\end{minipage}
%
\begin{minipage}{\linewidth}
$\bullet\ $ $11n_{17}$ \vspace{0.5em} \\
\begin{tabular}{l|lllll}
$k \setminus j$ & $-10$ & $-8$ & $-6$ & $-4$ & $-2$ \\
\hline
$8$ & $1$ & $2q^{-2}$ & $q^{-4}$ &  &  \\
$6$ &  & $3$ & $4q^{-2}$ & $q^{-4}$ &  \\
$4$ &  & $2q^{2}$ & $6$ & $4q^{-2}$ &  \\
$2$ &  &  & $4q^{2}$ & $6$ & $2q^{-2}$ \\
$0$ &  &  & $q^{4}$ & $4q^{2}$ & $3$ \\
$-2$ &  &  &  & $q^{4}$ & $2q^{2}$ \\
\end{tabular}
\vspace{2em}
\end{minipage}
%
\begin{minipage}{\linewidth}
$\bullet\ $ $11n_{18}$ \vspace{0.5em} \\
\begin{tabular}{l|lllll}
$k \setminus j$ & $-2$ & $0$ & $2$ & $4$ & $6$ \\
\hline
$2$ & $1$ & $2q^{-2}$ & $q^{-4}$ &  &  \\
$0$ &  & $3$ & $3q^{-2}$ & $q^{-4}$ &  \\
$-2$ &  & $2q^{2}$ & $4$ & $2q^{-2}$ &  \\
$-4$ &  &  & $3q^{2}$ & $4$ & $q^{-2}$ \\
$-6$ &  &  & $q^{4}$ & $2q^{2}$ & $1$ \\
$-8$ &  &  &  & $q^{4}$ & $q^{2}$ \\
\end{tabular}
\vspace{2em}
\end{minipage}
%
\begin{minipage}{\linewidth}
$\bullet\ $ $11n_{19}$ \vspace{0.5em} \\
\begin{tabular}{l|llll}
$k \setminus j$ & $-6$ & $-4$ & $-2$ & $0$ \\
\hline
$4$ & $1$ & $q^{-4}$ + $q^{-2}$ & $q^{-6}$ &  \\
$2$ &  & $1$ & $q^{-2}$ &  \\
$0$ &  & $1$ + $q^{2}$ & $2q^{-2}$ + $1$ & $q^{-4}$ \\
$-2$ &  &  & $q^{2}$ &  \\
$-4$ &  & $q^{4}$ & $2q^{2}$ & $1$ \\
$-8$ &  &  & $q^{6}$ & $q^{4}$ \\
\end{tabular}
\vspace{2em}
\end{minipage}
%
\begin{minipage}{\linewidth}
$\bullet\ $ $11n_{20}$ \vspace{0.5em} \\
\begin{tabular}{l|llll}
$k \setminus j$ & $-4$ & $-2$ & $0$ & $2$ \\
\hline
$4$ & $q^{-2}$ & $q^{-4}$ &  &  \\
$2$ & $1$ & $2q^{-2}$ & $q^{-4}$ &  \\
$0$ & $q^{2}$ & $3$ & $2q^{-2}$ + $1$ &  \\
$-2$ &  & $2q^{2}$ & $3$ & $q^{-2}$ \\
$-4$ &  & $q^{4}$ & $2q^{2}$ & $1$ \\
$-6$ &  &  & $q^{4}$ & $q^{2}$ \\
\end{tabular}
\vspace{2em}
\end{minipage}
%
\begin{minipage}{\linewidth}
$\bullet\ $ $11n_{21}$ \vspace{0.5em} \\
\begin{tabular}{l|llll}
$k \setminus j$ & $-2$ & $0$ & $2$ & $4$ \\
\hline
$4$ & $q^{-2}$ & $2q^{-4}$ & $q^{-6}$ &  \\
$2$ & $1$ & $3q^{-2}$ & $2q^{-4}$ &  \\
$0$ & $q^{2}$ & $6$ & $5q^{-2}$ & $q^{-4}$ \\
$-2$ &  & $3q^{2}$ & $5$ & $2q^{-2}$ \\
$-4$ &  & $2q^{4}$ & $5q^{2}$ & $3$ \\
$-6$ &  &  & $2q^{4}$ & $2q^{2}$ \\
$-8$ &  &  & $q^{6}$ & $q^{4}$ \\
\end{tabular}
\vspace{2em}
\end{minipage}
%
\begin{minipage}{\linewidth}
$\bullet\ $ $11n_{22}$ \vspace{0.5em} \\
\begin{tabular}{l|llll}
$k \setminus j$ & $2$ & $4$ & $6$ & $8$ \\
\hline
$4$ & $q^{-4}$ & $q^{-6}$ &  &  \\
$2$ & $3q^{-2}$ & $2q^{-4}$ &  &  \\
$0$ & $4$ & $6q^{-2}$ & $2q^{-4}$ &  \\
$-2$ & $3q^{2}$ & $6$ & $3q^{-2}$ &  \\
$-4$ & $q^{4}$ & $6q^{2}$ & $6$ & $q^{-2}$ \\
$-6$ &  & $2q^{4}$ & $3q^{2}$ & $1$ \\
$-8$ &  & $q^{6}$ & $2q^{4}$ & $q^{2}$ \\
\end{tabular}
\vspace{2em}
\end{minipage}
%
\begin{minipage}{\linewidth}
$\bullet\ $ $11n_{23}$ \vspace{0.5em} \\
\begin{tabular}{l|llll}
$k \setminus j$ & $2$ & $4$ & $6$ & $8$ \\
\hline
$8$ & $q^{-6}$ & $q^{-8}$ &  &  \\
$6$ & $q^{-4}$ & $q^{-6}$ &  &  \\
$4$ & $2q^{-2}$ & $4q^{-4}$ & $q^{-6}$ &  \\
$2$ & $1$ & $2q^{-2}$ & $q^{-4}$ &  \\
$0$ & $2q^{2}$ & $q^{-2}$ + $5$ & $q^{-4}$ + $3q^{-2}$ &  \\
$-2$ & $q^{4}$ & $1$ + $2q^{2}$ & $q^{-2}$ + $1$ &  \\
$-4$ & $q^{6}$ & $q^{2}$ + $4q^{4}$ & $2$ + $3q^{2}$ & $q^{-2}$ \\
$-6$ &  & $q^{6}$ & $q^{2}$ + $q^{4}$ & $1$ \\
$-8$ &  & $q^{8}$ & $q^{4}$ + $q^{6}$ & $q^{2}$ \\
\end{tabular}
\vspace{2em}
\end{minipage}
%
\begin{minipage}{\linewidth}
$\bullet\ $ $11n_{24}$ \vspace{0.5em} \\
\begin{tabular}{l|lll}
$k \setminus j$ & $-2$ & $0$ & $2$ \\
\hline
$8$ & $q^{-4}$ & $q^{-6}$ &  \\
$6$ & $q^{-2}$ & $q^{-4}$ &  \\
$4$ & $2$ & $3q^{-2}$ & $q^{-4}$ \\
$2$ & $q^{2}$ & $2$ & $q^{-2}$ \\
$0$ & $q^{4}$ & $1$ + $3q^{2}$ & $2$ \\
$-2$ &  & $q^{4}$ & $q^{2}$ \\
$-4$ &  & $q^{6}$ & $q^{4}$ \\
\end{tabular}
\vspace{2em}
\end{minipage}
%
\begin{minipage}{\linewidth}
$\bullet\ $ $11n_{25}$ \vspace{0.5em} \\
\begin{tabular}{l|llll}
$k \setminus j$ & $0$ & $2$ & $4$ & $6$ \\
\hline
$4$ & $q^{-2}$ & $2q^{-4}$ & $q^{-6}$ &  \\
$2$ &  & $3q^{-2}$ & $2q^{-4}$ &  \\
$0$ & $q^{2}$ & $5$ & $5q^{-2}$ & $q^{-4}$ \\
$-2$ &  & $3q^{2}$ & $5$ & $2q^{-2}$ \\
$-4$ &  & $2q^{4}$ & $5q^{2}$ & $3$ \\
$-6$ &  &  & $2q^{4}$ & $2q^{2}$ \\
$-8$ &  &  & $q^{6}$ & $q^{4}$ \\
\end{tabular}
\vspace{2em}
\end{minipage}
%
\begin{minipage}{\linewidth}
$\bullet\ $ $11n_{26}$ \vspace{0.5em} \\
\begin{tabular}{l|llll}
$k \setminus j$ & $0$ & $2$ & $4$ & $6$ \\
\hline
$4$ & $q^{-4}$ & $q^{-6}$ &  &  \\
$2$ & $2q^{-2}$ & $2q^{-4}$ &  &  \\
$0$ & $3$ & $4q^{-2}$ & $2q^{-4}$ &  \\
$-2$ & $2q^{2}$ & $4$ & $2q^{-2}$ &  \\
$-4$ & $q^{4}$ & $4q^{2}$ & $4$ & $q^{-2}$ \\
$-6$ &  & $2q^{4}$ & $2q^{2}$ &  \\
$-8$ &  & $q^{6}$ & $2q^{4}$ & $q^{2}$ \\
\end{tabular}
\vspace{2em}
\end{minipage}
%
\begin{minipage}{\linewidth}
$\bullet\ $ $11n_{27}$ \vspace{0.5em} \\
\begin{tabular}{l|llll}
$k \setminus j$ & $4$ & $6$ & $8$ & $10$ \\
\hline
$8$ & $q^{-6}$ & $q^{-8}$ &  &  \\
$6$ &  & $q^{-6}$ &  &  \\
$4$ & $2q^{-2}$ & $3q^{-4}$ & $q^{-6}$ &  \\
$2$ &  & $q^{-2}$ & $q^{-4}$ &  \\
$0$ & $2q^{2}$ & $q^{-2}$ + $4$ & $q^{-4}$ + $2q^{-2}$ &  \\
$-2$ &  & $1$ + $q^{2}$ & $q^{-2}$ + $1$ &  \\
$-4$ & $q^{6}$ & $q^{2}$ + $3q^{4}$ & $2$ + $2q^{2}$ & $q^{-2}$ \\
$-6$ &  & $q^{6}$ & $q^{2}$ + $q^{4}$ & $1$ \\
$-8$ &  & $q^{8}$ & $q^{4}$ + $q^{6}$ & $q^{2}$ \\
\end{tabular}
\vspace{2em}
\end{minipage}
%
\begin{minipage}{\linewidth}
$\bullet\ $ $11n_{28}$ \vspace{0.5em} \\
\begin{tabular}{l|lllll}
$k \setminus j$ & $-2$ & $0$ & $2$ & $4$ & $6$ \\
\hline
$2$ & $1$ & $q^{-2}$ &  &  &  \\
$0$ &  & $2$ & $2q^{-2}$ & $q^{-4}$ &  \\
$-2$ &  & $q^{2}$ & $2$ & $q^{-2}$ &  \\
$-4$ &  &  & $2q^{2}$ & $3$ & $q^{-2}$ \\
$-6$ &  &  &  & $q^{2}$ & $1$ \\
$-8$ &  &  &  & $q^{4}$ & $q^{2}$ \\
\end{tabular}
\vspace{2em}
\end{minipage}
%
\begin{minipage}{\linewidth}
$\bullet\ $ $11n_{29}$ \vspace{0.5em} \\
\begin{tabular}{l|lllll}
$k \setminus j$ & $0$ & $2$ & $4$ & $6$ & $8$ \\
\hline
$4$ & $2q^{-2}$ & $2q^{-4}$ &  &  &  \\
$2$ & $2$ & $4q^{-2}$ & $q^{-4}$ &  &  \\
$0$ & $2q^{2}$ & $7$ & $5q^{-2}$ &  &  \\
$-2$ &  & $4q^{2}$ & $6$ & $2q^{-2}$ &  \\
$-4$ &  & $2q^{4}$ & $5q^{2}$ & $3$ &  \\
$-6$ &  &  & $q^{4}$ & $2q^{2}$ & $1$ \\
\end{tabular}
\vspace{2em}
\end{minipage}
%
\begin{minipage}{\linewidth}
$\bullet\ $ $11n_{30}$ \vspace{0.5em} \\
\begin{tabular}{l|lllll}
$k \setminus j$ & $2$ & $4$ & $6$ & $8$ & $10$ \\
\hline
$6$ & $q^{-4}$ & $q^{-6}$ &  &  &  \\
$4$ &  & $2q^{-4}$ & $q^{-6}$ &  &  \\
$2$ & $1$ & $3q^{-2}$ & $2q^{-4}$ &  &  \\
$0$ &  & $2$ & $3q^{-2}$ & $q^{-4}$ &  \\
$-2$ & $q^{4}$ & $3q^{2}$ & $3$ & $q^{-2}$ &  \\
$-4$ &  & $2q^{4}$ & $1$ + $3q^{2}$ & $q^{-2}$ + $1$ &  \\
$-6$ &  & $q^{6}$ & $2q^{4}$ & $q^{2}$ &  \\
$-8$ &  &  & $q^{6}$ & $q^{2}$ + $q^{4}$ & $1$ \\
\end{tabular}
\vspace{2em}
\end{minipage}
%
\begin{minipage}{\linewidth}
$\bullet\ $ $11n_{31}$ \vspace{0.5em} \\
\begin{tabular}{l|lllll}
$k \setminus j$ & $2$ & $4$ & $6$ & $8$ & $10$ \\
\hline
$6$ & $q^{-4}$ & $q^{-6}$ &  &  &  \\
$4$ &  & $q^{-4}$ &  &  &  \\
$2$ & $1$ & $2q^{-2}$ & $q^{-4}$ &  &  \\
$0$ &  & $q^{-2}$ + $1$ & $q^{-4}$ + $q^{-2}$ &  &  \\
$-2$ & $q^{4}$ & $2$ + $2q^{2}$ & $2q^{-2}$ + $1$ &  &  \\
$-4$ &  & $q^{2}$ + $q^{4}$ & $3$ + $q^{2}$ & $2q^{-2}$ &  \\
$-6$ &  & $q^{6}$ & $2q^{2}$ + $q^{4}$ & $2$ &  \\
$-8$ &  &  & $q^{4}$ & $2q^{2}$ & $1$ \\
\end{tabular}
\vspace{2em}
\end{minipage}
%
\begin{minipage}{\linewidth}
$\bullet\ $ $11n_{32}$ \vspace{0.5em} \\
\begin{tabular}{l|lllll}
$k \setminus j$ & $-4$ & $-2$ & $0$ & $2$ & $4$ \\
\hline
$6$ &  & $q^{-4}$ & $q^{-6}$ &  &  \\
$4$ & $1$ & $4q^{-2}$ & $3q^{-4}$ &  &  \\
$2$ &  & $5$ & $7q^{-2}$ & $2q^{-4}$ &  \\
$0$ &  & $4q^{2}$ & $9$ & $4q^{-2}$ &  \\
$-2$ &  & $q^{4}$ & $7q^{2}$ & $7$ & $q^{-2}$ \\
$-4$ &  &  & $3q^{4}$ & $4q^{2}$ & $1$ \\
$-6$ &  &  & $q^{6}$ & $2q^{4}$ & $q^{2}$ \\
\end{tabular}
\vspace{2em}
\end{minipage}
%
\begin{minipage}{\linewidth}
$\bullet\ $ $11n_{33}$ \vspace{0.5em} \\
\begin{tabular}{l|llll}
$k \setminus j$ & $-2$ & $0$ & $2$ & $4$ \\
\hline
$6$ & $q^{-2}$ & $2q^{-4}$ & $q^{-6}$ &  \\
$4$ &  & $3q^{-2}$ & $3q^{-4}$ &  \\
$2$ & $q^{2}$ & $4$ & $5q^{-2}$ & $q^{-4}$ \\
$0$ &  & $3q^{2}$ & $6$ & $3q^{-2}$ \\
$-2$ &  & $2q^{4}$ & $5q^{2}$ & $3$ \\
$-4$ &  &  & $3q^{4}$ & $3q^{2}$ \\
$-6$ &  &  & $q^{6}$ & $q^{4}$ \\
\end{tabular}
\vspace{2em}
\end{minipage}
%
\begin{minipage}{\linewidth}
$\bullet\ $ $11n_{34}$ \vspace{0.5em} \\
\begin{tabular}{l|llll}
$k \setminus j$ & $-4$ & $-2$ & $0$ & $2$ \\
\hline
$8$ & $q^{-4}$ & $q^{-6}$ &  &  \\
$6$ & $q^{-2}$ & $q^{-4}$ &  &  \\
$4$ & $2$ & $q^{-4}$ + $3q^{-2}$ & $q^{-6}$ + $q^{-4}$ &  \\
$2$ & $q^{2}$ & $q^{-2}$ + $2$ & $q^{-4}$ + $q^{-2}$ &  \\
$0$ & $q^{4}$ & $2$ + $3q^{2}$ & $3q^{-2}$ + $3$ & $q^{-4}$ \\
$-2$ &  & $q^{2}$ + $q^{4}$ & $2$ + $q^{2}$ & $q^{-2}$ \\
$-4$ &  & $q^{4}$ + $q^{6}$ & $3q^{2}$ + $q^{4}$ & $2$ \\
$-6$ &  &  & $q^{4}$ & $q^{2}$ \\
$-8$ &  &  & $q^{6}$ & $q^{4}$ \\
\end{tabular}
\vspace{2em}
\end{minipage}
%
\begin{minipage}{\linewidth}
$\bullet\ $ $11n_{35}$ \vspace{0.5em} \\
\begin{tabular}{l|llll}
$k \setminus j$ & $4$ & $6$ & $8$ & $10$ \\
\hline
$4$ & $3q^{-4}$ & $2q^{-6}$ &  &  \\
$2$ & $4q^{-2}$ & $4q^{-4}$ &  &  \\
$0$ & $7$ & $10q^{-2}$ & $3q^{-4}$ &  \\
$-2$ & $4q^{2}$ & $9$ & $5q^{-2}$ &  \\
$-4$ & $3q^{4}$ & $10q^{2}$ & $8$ & $q^{-2}$ \\
$-6$ &  & $4q^{4}$ & $5q^{2}$ & $1$ \\
$-8$ &  & $2q^{6}$ & $3q^{4}$ & $q^{2}$ \\
\end{tabular}
\vspace{2em}
\end{minipage}
%
\begin{minipage}{\linewidth}
$\bullet\ $ $11n_{36}$ \vspace{0.5em} \\
\begin{tabular}{l|llll}
$k \setminus j$ & $0$ & $2$ & $4$ & $6$ \\
\hline
$8$ & $q^{-6}$ & $q^{-8}$ &  &  \\
$6$ & $2q^{-4}$ & $2q^{-6}$ &  &  \\
$4$ & $4q^{-2}$ & $5q^{-4}$ & $q^{-6}$ &  \\
$2$ & $4$ & $7q^{-2}$ & $2q^{-4}$ &  \\
$0$ & $4q^{2}$ & $q^{-2}$ + $8$ & $q^{-4}$ + $4q^{-2}$ &  \\
$-2$ & $2q^{4}$ & $7q^{2}$ & $5$ &  \\
$-4$ & $q^{6}$ & $q^{2}$ + $5q^{4}$ & $2$ + $4q^{2}$ & $q^{-2}$ \\
$-6$ &  & $2q^{6}$ & $2q^{4}$ &  \\
$-8$ &  & $q^{8}$ & $q^{4}$ + $q^{6}$ & $q^{2}$ \\
\end{tabular}
\vspace{2em}
\end{minipage}
%
\begin{minipage}{\linewidth}
$\bullet\ $ $11n_{37}$ \vspace{0.5em} \\
\begin{tabular}{l|lll}
$k \setminus j$ & $-4$ & $-2$ & $0$ \\
\hline
$8$ & $q^{-4}$ & $q^{-6}$ &  \\
$6$ & $q^{-2}$ & $q^{-4}$ &  \\
$4$ & $2$ & $3q^{-2}$ & $q^{-4}$ \\
$2$ & $q^{2}$ & $2$ & $q^{-2}$ \\
$0$ & $q^{4}$ & $3q^{2}$ & $3$ \\
$-2$ &  & $q^{4}$ & $q^{2}$ \\
$-4$ &  & $q^{6}$ & $q^{4}$ \\
\end{tabular}
\vspace{2em}
\end{minipage}
%
\begin{minipage}{\linewidth}
$\bullet\ $ $11n_{38}$ \vspace{0.5em} \\
\begin{tabular}{l|llll}
$k \setminus j$ & $-4$ & $-2$ & $0$ & $2$ \\
\hline
$8$ & $q^{-2}$ & $q^{-4}$ &  &  \\
$4$ & $q^{2}$ & $2$ & $q^{-2}$ &  \\
$2$ &  & $1$ & $q^{-2}$ &  \\
$0$ &  & $q^{4}$ & $1$ + $q^{2}$ &  \\
$-2$ &  &  & $q^{2}$ & $1$ \\
\end{tabular}
\vspace{2em}
\end{minipage}
%
\begin{minipage}{\linewidth}
$\bullet\ $ $11n_{39}$ \vspace{0.5em} \\
\begin{tabular}{l|llll}
$k \setminus j$ & $-2$ & $0$ & $2$ & $4$ \\
\hline
$8$ & $q^{-4}$ & $q^{-6}$ &  &  \\
$4$ & $2$ & $q^{-4}$ + $3q^{-2}$ & $q^{-6}$ + $q^{-4}$ &  \\
$2$ & $1$ & $3q^{-2}$ & $2q^{-4}$ &  \\
$0$ & $q^{4}$ & $4$ + $3q^{2}$ & $4q^{-2}$ + $2$ & $q^{-4}$ \\
$-2$ &  & $3q^{2}$ & $5$ & $2q^{-2}$ \\
$-4$ &  & $q^{4}$ + $q^{6}$ & $4q^{2}$ + $q^{4}$ & $3$ \\
$-6$ &  &  & $2q^{4}$ & $2q^{2}$ \\
$-8$ &  &  & $q^{6}$ & $q^{4}$ \\
\end{tabular}
\vspace{2em}
\end{minipage}
%
\begin{minipage}{\linewidth}
$\bullet\ $ $11n_{40}$ \vspace{0.5em} \\
\begin{tabular}{l|llll}
$k \setminus j$ & $2$ & $4$ & $6$ & $8$ \\
\hline
$4$ & $2q^{-4}$ & $2q^{-6}$ &  &  \\
$2$ & $4q^{-2}$ & $3q^{-4}$ &  &  \\
$0$ & $6$ & $9q^{-2}$ & $3q^{-4}$ &  \\
$-2$ & $4q^{2}$ & $8$ & $4q^{-2}$ &  \\
$-4$ & $2q^{4}$ & $9q^{2}$ & $8$ & $q^{-2}$ \\
$-6$ &  & $3q^{4}$ & $4q^{2}$ & $1$ \\
$-8$ &  & $2q^{6}$ & $3q^{4}$ & $q^{2}$ \\
\end{tabular}
\vspace{2em}
\end{minipage}
%
\begin{minipage}{\linewidth}
$\bullet\ $ $11n_{41}$ \vspace{0.5em} \\
\begin{tabular}{l|llll}
$k \setminus j$ & $2$ & $4$ & $6$ & $8$ \\
\hline
$8$ & $q^{-6}$ & $q^{-8}$ &  &  \\
$6$ & $2q^{-4}$ & $2q^{-6}$ &  &  \\
$4$ & $3q^{-2}$ & $5q^{-4}$ & $q^{-6}$ &  \\
$2$ & $3$ & $5q^{-2}$ & $2q^{-4}$ &  \\
$0$ & $3q^{2}$ & $q^{-2}$ + $7$ & $q^{-4}$ + $4q^{-2}$ &  \\
$-2$ & $2q^{4}$ & $1$ + $5q^{2}$ & $q^{-2}$ + $3$ &  \\
$-4$ & $q^{6}$ & $q^{2}$ + $5q^{4}$ & $2$ + $4q^{2}$ & $q^{-2}$ \\
$-6$ &  & $2q^{6}$ & $q^{2}$ + $2q^{4}$ & $1$ \\
$-8$ &  & $q^{8}$ & $q^{4}$ + $q^{6}$ & $q^{2}$ \\
\end{tabular}
\vspace{2em}
\end{minipage}
%
\begin{minipage}{\linewidth}
$\bullet\ $ $11n_{42}$ \vspace{0.5em} \\
\begin{tabular}{l|llll}
$k \setminus j$ & $-4$ & $-2$ & $0$ & $2$ \\
\hline
$8$ & $q^{-4}$ & $q^{-6}$ &  &  \\
$6$ & $q^{-2}$ & $q^{-4}$ &  &  \\
$4$ & $2$ & $q^{-4}$ + $3q^{-2}$ & $q^{-6}$ + $q^{-4}$ &  \\
$2$ & $q^{2}$ & $q^{-2}$ + $2$ & $q^{-4}$ + $q^{-2}$ &  \\
$0$ & $q^{4}$ & $2$ + $3q^{2}$ & $3q^{-2}$ + $3$ & $q^{-4}$ \\
$-2$ &  & $q^{2}$ + $q^{4}$ & $2$ + $q^{2}$ & $q^{-2}$ \\
$-4$ &  & $q^{4}$ + $q^{6}$ & $3q^{2}$ + $q^{4}$ & $2$ \\
$-6$ &  &  & $q^{4}$ & $q^{2}$ \\
$-8$ &  &  & $q^{6}$ & $q^{4}$ \\
\end{tabular}
\vspace{2em}
\end{minipage}
%
\begin{minipage}{\linewidth}
$\bullet\ $ $11n_{43}$ \vspace{0.5em} \\
\begin{tabular}{l|llll}
$k \setminus j$ & $4$ & $6$ & $8$ & $10$ \\
\hline
$4$ & $3q^{-4}$ & $2q^{-6}$ &  &  \\
$2$ & $4q^{-2}$ & $4q^{-4}$ &  &  \\
$0$ & $7$ & $10q^{-2}$ & $3q^{-4}$ &  \\
$-2$ & $4q^{2}$ & $9$ & $5q^{-2}$ &  \\
$-4$ & $3q^{4}$ & $10q^{2}$ & $8$ & $q^{-2}$ \\
$-6$ &  & $4q^{4}$ & $5q^{2}$ & $1$ \\
$-8$ &  & $2q^{6}$ & $3q^{4}$ & $q^{2}$ \\
\end{tabular}
\vspace{2em}
\end{minipage}
%
\begin{minipage}{\linewidth}
$\bullet\ $ $11n_{44}$ \vspace{0.5em} \\
\begin{tabular}{l|llll}
$k \setminus j$ & $0$ & $2$ & $4$ & $6$ \\
\hline
$8$ & $q^{-6}$ & $q^{-8}$ &  &  \\
$6$ & $2q^{-4}$ & $2q^{-6}$ &  &  \\
$4$ & $4q^{-2}$ & $5q^{-4}$ & $q^{-6}$ &  \\
$2$ & $4$ & $7q^{-2}$ & $2q^{-4}$ &  \\
$0$ & $4q^{2}$ & $q^{-2}$ + $8$ & $q^{-4}$ + $4q^{-2}$ &  \\
$-2$ & $2q^{4}$ & $7q^{2}$ & $5$ &  \\
$-4$ & $q^{6}$ & $q^{2}$ + $5q^{4}$ & $2$ + $4q^{2}$ & $q^{-2}$ \\
$-6$ &  & $2q^{6}$ & $2q^{4}$ &  \\
$-8$ &  & $q^{8}$ & $q^{4}$ + $q^{6}$ & $q^{2}$ \\
\end{tabular}
\vspace{2em}
\end{minipage}
%
\begin{minipage}{\linewidth}
$\bullet\ $ $11n_{45}$ \vspace{0.5em} \\
\begin{tabular}{l|llll}
$k \setminus j$ & $-2$ & $0$ & $2$ & $4$ \\
\hline
$8$ & $q^{-4}$ & $q^{-6}$ &  &  \\
$4$ & $2$ & $q^{-4}$ + $3q^{-2}$ & $q^{-6}$ + $q^{-4}$ &  \\
$2$ & $1$ & $3q^{-2}$ & $2q^{-4}$ &  \\
$0$ & $q^{4}$ & $4$ + $3q^{2}$ & $4q^{-2}$ + $2$ & $q^{-4}$ \\
$-2$ &  & $3q^{2}$ & $5$ & $2q^{-2}$ \\
$-4$ &  & $q^{4}$ + $q^{6}$ & $4q^{2}$ + $q^{4}$ & $3$ \\
$-6$ &  &  & $2q^{4}$ & $2q^{2}$ \\
$-8$ &  &  & $q^{6}$ & $q^{4}$ \\
\end{tabular}
\vspace{2em}
\end{minipage}
%
\begin{minipage}{\linewidth}
$\bullet\ $ $11n_{46}$ \vspace{0.5em} \\
\begin{tabular}{l|llll}
$k \setminus j$ & $2$ & $4$ & $6$ & $8$ \\
\hline
$4$ & $2q^{-4}$ & $2q^{-6}$ &  &  \\
$2$ & $4q^{-2}$ & $3q^{-4}$ &  &  \\
$0$ & $6$ & $9q^{-2}$ & $3q^{-4}$ &  \\
$-2$ & $4q^{2}$ & $8$ & $4q^{-2}$ &  \\
$-4$ & $2q^{4}$ & $9q^{2}$ & $8$ & $q^{-2}$ \\
$-6$ &  & $3q^{4}$ & $4q^{2}$ & $1$ \\
$-8$ &  & $2q^{6}$ & $3q^{4}$ & $q^{2}$ \\
\end{tabular}
\vspace{2em}
\end{minipage}
%
\begin{minipage}{\linewidth}
$\bullet\ $ $11n_{47}$ \vspace{0.5em} \\
\begin{tabular}{l|llll}
$k \setminus j$ & $2$ & $4$ & $6$ & $8$ \\
\hline
$8$ & $q^{-6}$ & $q^{-8}$ &  &  \\
$6$ & $2q^{-4}$ & $2q^{-6}$ &  &  \\
$4$ & $3q^{-2}$ & $5q^{-4}$ & $q^{-6}$ &  \\
$2$ & $3$ & $5q^{-2}$ & $2q^{-4}$ &  \\
$0$ & $3q^{2}$ & $q^{-2}$ + $7$ & $q^{-4}$ + $4q^{-2}$ &  \\
$-2$ & $2q^{4}$ & $1$ + $5q^{2}$ & $q^{-2}$ + $3$ &  \\
$-4$ & $q^{6}$ & $q^{2}$ + $5q^{4}$ & $2$ + $4q^{2}$ & $q^{-2}$ \\
$-6$ &  & $2q^{6}$ & $q^{2}$ + $2q^{4}$ & $1$ \\
$-8$ &  & $q^{8}$ & $q^{4}$ + $q^{6}$ & $q^{2}$ \\
\end{tabular}
\vspace{2em}
\end{minipage}
%
\begin{minipage}{\linewidth}
$\bullet\ $ $11n_{48}$ \vspace{0.5em} \\
\begin{tabular}{l|llll}
$k \setminus j$ & $-4$ & $-2$ & $0$ & $2$ \\
\hline
$8$ & $q^{-4}$ & $q^{-6}$ &  &  \\
$6$ & $q^{-2}$ & $q^{-4}$ &  &  \\
$4$ & $2$ & $3q^{-2}$ & $q^{-4}$ &  \\
$2$ & $q^{2}$ & $3$ & $2q^{-2}$ &  \\
$0$ & $q^{4}$ & $3q^{2}$ & $3$ &  \\
$-2$ &  & $q^{4}$ & $2q^{2}$ & $1$ \\
$-4$ &  & $q^{6}$ & $q^{4}$ &  \\
\end{tabular}
\vspace{2em}
\end{minipage}
%
\begin{minipage}{\linewidth}
$\bullet\ $ $11n_{49}$ \vspace{0.5em} \\
\begin{tabular}{l|lllll}
$k \setminus j$ & $-4$ & $-2$ & $0$ & $2$ & $4$ \\
\hline
$8$ & $q^{-2}$ & $q^{-4}$ &  &  &  \\
$4$ & $q^{2}$ & $2$ & $q^{-2}$ &  &  \\
$2$ &  & $1$ & $q^{-2}$ &  &  \\
$0$ &  & $q^{4}$ & $2$ + $q^{2}$ & $q^{-2}$ &  \\
$-2$ &  &  & $q^{2}$ & $1$ &  \\
$-4$ &  &  &  & $q^{2}$ & $1$ \\
\end{tabular}
\vspace{2em}
\end{minipage}
%
\begin{minipage}{\linewidth}
$\bullet\ $ $11n_{50}$ \vspace{0.5em} \\
\begin{tabular}{l|llll}
$k \setminus j$ & $-6$ & $-4$ & $-2$ & $0$ \\
\hline
$8$ & $q^{-2}$ & $q^{-4}$ &  &  \\
$6$ & $1$ & $2q^{-2}$ & $q^{-4}$ &  \\
$4$ & $q^{2}$ & $3$ & $2q^{-2}$ &  \\
$2$ &  & $2q^{2}$ & $3$ & $q^{-2}$ \\
$0$ &  & $q^{4}$ & $2q^{2}$ & $2$ \\
$-2$ &  &  & $q^{4}$ & $q^{2}$ \\
\end{tabular}
\vspace{2em}
\end{minipage}
%
\begin{minipage}{\linewidth}
$\bullet\ $ $11n_{51}$ \vspace{0.5em} \\
\begin{tabular}{l|llll}
$k \setminus j$ & $0$ & $2$ & $4$ & $6$ \\
\hline
$4$ & $q^{-4}$ & $q^{-6}$ &  &  \\
$2$ & $q^{-2}$ & $q^{-4}$ &  &  \\
$0$ & $2$ & $3q^{-2}$ & $2q^{-4}$ &  \\
$-2$ & $q^{2}$ & $2$ & $q^{-2}$ &  \\
$-4$ & $q^{4}$ & $3q^{2}$ & $3$ & $q^{-2}$ \\
$-6$ &  & $q^{4}$ & $q^{2}$ &  \\
$-8$ &  & $q^{6}$ & $2q^{4}$ & $q^{2}$ \\
\end{tabular}
\vspace{2em}
\end{minipage}
%
\begin{minipage}{\linewidth}
$\bullet\ $ $11n_{52}$ \vspace{0.5em} \\
\begin{tabular}{l|llll}
$k \setminus j$ & $0$ & $2$ & $4$ & $6$ \\
\hline
$4$ & $2q^{-2}$ & $3q^{-4}$ & $q^{-6}$ &  \\
$2$ & $1$ & $4q^{-2}$ & $2q^{-4}$ &  \\
$0$ & $2q^{2}$ & $7$ & $6q^{-2}$ & $q^{-4}$ \\
$-2$ &  & $4q^{2}$ & $6$ & $2q^{-2}$ \\
$-4$ &  & $3q^{4}$ & $6q^{2}$ & $3$ \\
$-6$ &  &  & $2q^{4}$ & $2q^{2}$ \\
$-8$ &  &  & $q^{6}$ & $q^{4}$ \\
\end{tabular}
\vspace{2em}
\end{minipage}
%
\begin{minipage}{\linewidth}
$\bullet\ $ $11n_{53}$ \vspace{0.5em} \\
\begin{tabular}{l|llll}
$k \setminus j$ & $-4$ & $-2$ & $0$ & $2$ \\
\hline
$8$ & $q^{-4}$ & $q^{-6}$ &  &  \\
$6$ & $q^{-2}$ & $q^{-4}$ &  &  \\
$4$ & $2$ & $4q^{-2}$ & $2q^{-4}$ &  \\
$2$ & $q^{2}$ & $3$ & $2q^{-2}$ &  \\
$0$ & $q^{4}$ & $4q^{2}$ & $5$ & $q^{-2}$ \\
$-2$ &  & $q^{4}$ & $2q^{2}$ & $1$ \\
$-4$ &  & $q^{6}$ & $2q^{4}$ & $q^{2}$ \\
\end{tabular}
\vspace{2em}
\end{minipage}
%
\begin{minipage}{\linewidth}
$\bullet\ $ $11n_{54}$ \vspace{0.5em} \\
\begin{tabular}{l|llll}
$k \setminus j$ & $2$ & $4$ & $6$ & $8$ \\
\hline
$4$ & $q^{-4}$ & $q^{-6}$ &  &  \\
$2$ & $2q^{-2}$ & $q^{-4}$ &  &  \\
$0$ & $3$ & $5q^{-2}$ & $2q^{-4}$ &  \\
$-2$ & $2q^{2}$ & $4$ & $2q^{-2}$ &  \\
$-4$ & $q^{4}$ & $5q^{2}$ & $5$ & $q^{-2}$ \\
$-6$ &  & $q^{4}$ & $2q^{2}$ & $1$ \\
$-8$ &  & $q^{6}$ & $2q^{4}$ & $q^{2}$ \\
\end{tabular}
\vspace{2em}
\end{minipage}
%
\begin{minipage}{\linewidth}
$\bullet\ $ $11n_{55}$ \vspace{0.5em} \\
\begin{tabular}{l|llll}
$k \setminus j$ & $-2$ & $0$ & $2$ & $4$ \\
\hline
$4$ & $2q^{-2}$ & $3q^{-4}$ & $q^{-6}$ &  \\
$2$ & $2$ & $4q^{-2}$ & $2q^{-4}$ &  \\
$0$ & $2q^{2}$ & $8$ & $6q^{-2}$ & $q^{-4}$ \\
$-2$ &  & $4q^{2}$ & $6$ & $2q^{-2}$ \\
$-4$ &  & $3q^{4}$ & $6q^{2}$ & $3$ \\
$-6$ &  &  & $2q^{4}$ & $2q^{2}$ \\
$-8$ &  &  & $q^{6}$ & $q^{4}$ \\
\end{tabular}
\vspace{2em}
\end{minipage}
%
\begin{minipage}{\linewidth}
$\bullet\ $ $11n_{56}$ \vspace{0.5em} \\
\begin{tabular}{l|llll}
$k \setminus j$ & $-2$ & $0$ & $2$ & $4$ \\
\hline
$8$ & $q^{-4}$ & $q^{-6}$ &  &  \\
$6$ & $q^{-2}$ & $q^{-4}$ &  &  \\
$4$ & $2$ & $4q^{-2}$ & $2q^{-4}$ &  \\
$2$ & $q^{2}$ & $2$ & $2q^{-2}$ &  \\
$0$ & $q^{4}$ & $4q^{2}$ & $4$ & $q^{-2}$ \\
$-2$ &  & $q^{4}$ & $2q^{2}$ & $1$ \\
$-4$ &  & $q^{6}$ & $2q^{4}$ & $q^{2}$ \\
\end{tabular}
\vspace{2em}
\end{minipage}
%
\begin{minipage}{\linewidth}
$\bullet\ $ $11n_{57}$ \vspace{0.5em} \\
\begin{tabular}{l|llll}
$k \setminus j$ & $4$ & $6$ & $8$ & $10$ \\
\hline
$8$ & $q^{-6}$ & $q^{-8}$ &  &  \\
$6$ &  & $q^{-6}$ &  &  \\
$4$ & $q^{-2}$ & $2q^{-4}$ & $q^{-6}$ &  \\
$2$ &  & $q^{-2}$ & $q^{-4}$ &  \\
$0$ & $1$ + $q^{2}$ & $2q^{-2}$ + $2$ & $q^{-4}$ + $q^{-2}$ &  \\
$-2$ &  & $1$ + $q^{2}$ & $q^{-2}$ + $1$ &  \\
$-4$ & $q^{6}$ & $2q^{2}$ + $2q^{4}$ & $3$ + $q^{2}$ & $q^{-2}$ \\
$-6$ &  & $q^{6}$ & $q^{2}$ + $q^{4}$ & $1$ \\
$-8$ &  & $q^{8}$ & $q^{4}$ + $q^{6}$ & $q^{2}$ \\
\end{tabular}
\vspace{2em}
\end{minipage}
%
\begin{minipage}{\linewidth}
$\bullet\ $ $11n_{58}$ \vspace{0.5em} \\
\begin{tabular}{l|llll}
$k \setminus j$ & $-4$ & $-2$ & $0$ & $2$ \\
\hline
$4$ & $q^{-2}$ & $2q^{-4}$ & $q^{-6}$ &  \\
$2$ & $1$ & $2q^{-2}$ & $q^{-4}$ &  \\
$0$ & $q^{2}$ & $4$ & $4q^{-2}$ & $q^{-4}$ \\
$-2$ &  & $2q^{2}$ & $2$ & $q^{-2}$ \\
$-4$ &  & $2q^{4}$ & $4q^{2}$ & $2$ \\
$-6$ &  &  & $q^{4}$ & $q^{2}$ \\
$-8$ &  &  & $q^{6}$ & $q^{4}$ \\
\end{tabular}
\vspace{2em}
\end{minipage}
%
\begin{minipage}{\linewidth}
$\bullet\ $ $11n_{59}$ \vspace{0.5em} \\
\begin{tabular}{l|llll}
$k \setminus j$ & $4$ & $6$ & $8$ & $10$ \\
\hline
$4$ & $2q^{-4}$ & $q^{-6}$ &  &  \\
$2$ & $2q^{-2}$ & $2q^{-4}$ &  &  \\
$0$ & $4$ & $6q^{-2}$ & $2q^{-4}$ &  \\
$-2$ & $2q^{2}$ & $5$ & $3q^{-2}$ &  \\
$-4$ & $2q^{4}$ & $6q^{2}$ & $5$ & $q^{-2}$ \\
$-6$ &  & $2q^{4}$ & $3q^{2}$ & $1$ \\
$-8$ &  & $q^{6}$ & $2q^{4}$ & $q^{2}$ \\
\end{tabular}
\vspace{2em}
\end{minipage}
%
\begin{minipage}{\linewidth}
$\bullet\ $ $11n_{60}$ \vspace{0.5em} \\
\begin{tabular}{l|llll}
$k \setminus j$ & $0$ & $2$ & $4$ & $6$ \\
\hline
$8$ & $q^{-6}$ & $q^{-8}$ &  &  \\
$6$ & $q^{-4}$ & $q^{-6}$ &  &  \\
$4$ & $2q^{-2}$ & $3q^{-4}$ & $q^{-6}$ &  \\
$2$ & $1$ & $3q^{-2}$ & $q^{-4}$ &  \\
$0$ & $2q^{2}$ & $q^{-2}$ + $4$ & $q^{-4}$ + $2q^{-2}$ &  \\
$-2$ & $q^{4}$ & $3q^{2}$ & $2$ &  \\
$-4$ & $q^{6}$ & $q^{2}$ + $3q^{4}$ & $2$ + $2q^{2}$ & $q^{-2}$ \\
$-6$ &  & $q^{6}$ & $q^{4}$ &  \\
$-8$ &  & $q^{8}$ & $q^{4}$ + $q^{6}$ & $q^{2}$ \\
\end{tabular}
\vspace{2em}
\end{minipage}
%
\begin{minipage}{\linewidth}
$\bullet\ $ $11n_{61}$ \vspace{0.5em} \\
\begin{tabular}{l|llll}
$k \setminus j$ & $2$ & $4$ & $6$ & $8$ \\
\hline
$8$ & $q^{-6}$ & $q^{-8}$ &  &  \\
$6$ & $q^{-4}$ & $q^{-6}$ &  &  \\
$4$ & $q^{-2}$ & $3q^{-4}$ & $q^{-6}$ &  \\
$2$ & $1$ & $2q^{-2}$ & $q^{-4}$ &  \\
$0$ & $1$ + $q^{2}$ & $2q^{-2}$ + $3$ & $q^{-4}$ + $2q^{-2}$ &  \\
$-2$ & $q^{4}$ & $1$ + $2q^{2}$ & $q^{-2}$ + $1$ &  \\
$-4$ & $q^{6}$ & $2q^{2}$ + $3q^{4}$ & $3$ + $2q^{2}$ & $q^{-2}$ \\
$-6$ &  & $q^{6}$ & $q^{2}$ + $q^{4}$ & $1$ \\
$-8$ &  & $q^{8}$ & $q^{4}$ + $q^{6}$ & $q^{2}$ \\
\end{tabular}
\vspace{2em}
\end{minipage}
%
\begin{minipage}{\linewidth}
$\bullet\ $ $11n_{62}$ \vspace{0.5em} \\
\begin{tabular}{l|lllll}
$k \setminus j$ & $-4$ & $-2$ & $0$ & $2$ & $4$ \\
\hline
$4$ & $1$ & $2q^{-2}$ & $q^{-4}$ &  &  \\
$2$ &  & $2$ & $3q^{-2}$ & $q^{-4}$ &  \\
$0$ &  & $2q^{2}$ & $5$ & $2q^{-2}$ &  \\
$-2$ &  &  & $3q^{2}$ & $4$ & $q^{-2}$ \\
$-4$ &  &  & $q^{4}$ & $2q^{2}$ & $1$ \\
$-6$ &  &  &  & $q^{4}$ & $q^{2}$ \\
\end{tabular}
\vspace{2em}
\end{minipage}
%
\begin{minipage}{\linewidth}
$\bullet\ $ $11n_{63}$ \vspace{0.5em} \\
\begin{tabular}{l|lllll}
$k \setminus j$ & $2$ & $4$ & $6$ & $8$ & $10$ \\
\hline
$2$ & $2q^{-2}$ & $q^{-4}$ &  &  &  \\
$0$ & $2$ & $3q^{-2}$ & $q^{-4}$ &  &  \\
$-2$ & $2q^{2}$ & $5$ & $3q^{-2}$ &  &  \\
$-4$ &  & $3q^{2}$ & $5$ & $2q^{-2}$ &  \\
$-6$ &  & $q^{4}$ & $3q^{2}$ & $2$ &  \\
$-8$ &  &  & $q^{4}$ & $2q^{2}$ & $1$ \\
\end{tabular}
\vspace{2em}
\end{minipage}
%
\begin{minipage}{\linewidth}
$\bullet\ $ $11n_{64}$ \vspace{0.5em} \\
\begin{tabular}{l|lllll}
$k \setminus j$ & $0$ & $2$ & $4$ & $6$ & $8$ \\
\hline
$8$ & $q^{-4}$ & $q^{-6}$ &  &  &  \\
$6$ &  & $q^{-4}$ & $q^{-6}$ &  &  \\
$4$ & $1$ & $2q^{-2}$ & $2q^{-4}$ &  &  \\
$2$ &  & $1$ & $2q^{-2}$ & $q^{-4}$ &  \\
$0$ & $q^{4}$ & $1$ + $2q^{2}$ & $q^{-2}$ + $2$ & $q^{-2}$ &  \\
$-2$ &  & $q^{4}$ & $1$ + $2q^{2}$ & $q^{-2}$ + $1$ &  \\
$-4$ &  & $q^{6}$ & $q^{2}$ + $2q^{4}$ & $1$ + $q^{2}$ &  \\
$-6$ &  &  & $q^{6}$ & $q^{2}$ + $q^{4}$ & $1$ \\
\end{tabular}
\vspace{2em}
\end{minipage}
%
\begin{minipage}{\linewidth}
$\bullet\ $ $11n_{65}$ \vspace{0.5em} \\
\begin{tabular}{l|llll}
$k \setminus j$ & $-6$ & $-4$ & $-2$ & $0$ \\
\hline
$8$ & $q^{-2}$ & $q^{-4}$ &  &  \\
$6$ & $1$ & $3q^{-2}$ & $2q^{-4}$ &  \\
$4$ & $q^{2}$ & $3$ & $2q^{-2}$ &  \\
$2$ &  & $3q^{2}$ & $5$ & $2q^{-2}$ \\
$0$ &  & $q^{4}$ & $2q^{2}$ & $2$ \\
$-2$ &  &  & $2q^{4}$ & $2q^{2}$ \\
\end{tabular}
\vspace{2em}
\end{minipage}
%
\begin{minipage}{\linewidth}
$\bullet\ $ $11n_{66}$ \vspace{0.5em} \\
\begin{tabular}{l|llll}
$k \setminus j$ & $0$ & $2$ & $4$ & $6$ \\
\hline
$6$ & $q^{-4}$ & $q^{-6}$ &  &  \\
$4$ & $3q^{-2}$ & $3q^{-4}$ &  &  \\
$2$ & $4$ & $8q^{-2}$ & $3q^{-4}$ &  \\
$0$ & $3q^{2}$ & $8$ & $5q^{-2}$ &  \\
$-2$ & $q^{4}$ & $8q^{2}$ & $9$ & $2q^{-2}$ \\
$-4$ &  & $3q^{4}$ & $5q^{2}$ & $2$ \\
$-6$ &  & $q^{6}$ & $3q^{4}$ & $2q^{2}$ \\
\end{tabular}
\vspace{2em}
\end{minipage}
%
\begin{minipage}{\linewidth}
$\bullet\ $ $11n_{68}$ \vspace{0.5em} \\
\begin{tabular}{l|lllll}
$k \setminus j$ & $0$ & $2$ & $4$ & $6$ & $8$ \\
\hline
$4$ & $2q^{-2}$ & $2q^{-4}$ &  &  &  \\
$2$ & $2$ & $5q^{-2}$ & $2q^{-4}$ &  &  \\
$0$ & $2q^{2}$ & $8$ & $6q^{-2}$ &  &  \\
$-2$ &  & $5q^{2}$ & $8$ & $3q^{-2}$ &  \\
$-4$ &  & $2q^{4}$ & $6q^{2}$ & $4$ &  \\
$-6$ &  &  & $2q^{4}$ & $3q^{2}$ & $1$ \\
\end{tabular}
\vspace{2em}
\end{minipage}
%
\begin{minipage}{\linewidth}
$\bullet\ $ $11n_{69}$ \vspace{0.5em} \\
\begin{tabular}{l|lllll}
$k \setminus j$ & $2$ & $4$ & $6$ & $8$ & $10$ \\
\hline
$6$ & $q^{-4}$ & $q^{-6}$ &  &  &  \\
$4$ & $q^{-2}$ & $3q^{-4}$ & $q^{-6}$ &  &  \\
$2$ & $2$ & $4q^{-2}$ & $2q^{-4}$ &  &  \\
$0$ & $q^{2}$ & $4$ & $4q^{-2}$ & $q^{-4}$ &  \\
$-2$ & $q^{4}$ & $4q^{2}$ & $4$ & $q^{-2}$ &  \\
$-4$ &  & $3q^{4}$ & $1$ + $4q^{2}$ & $q^{-2}$ + $1$ &  \\
$-6$ &  & $q^{6}$ & $2q^{4}$ & $q^{2}$ &  \\
$-8$ &  &  & $q^{6}$ & $q^{2}$ + $q^{4}$ & $1$ \\
\end{tabular}
\vspace{2em}
\end{minipage}
%
\begin{minipage}{\linewidth}
$\bullet\ $ $11n_{70}$ \vspace{0.5em} \\
\begin{tabular}{l|lll}
$k \setminus j$ & $0$ & $2$ & $4$ \\
\hline
$8$ & $q^{-4}$ & $q^{-6}$ &  \\
$4$ & $2$ & $3q^{-2}$ & $q^{-4}$ \\
$2$ &  & $q^{-2}$ &  \\
$0$ & $q^{4}$ & $3q^{2}$ & $2$ \\
$-2$ &  & $q^{2}$ & $1$ \\
$-4$ &  & $q^{6}$ & $q^{4}$ \\
\end{tabular}
\vspace{2em}
\end{minipage}
%
\begin{minipage}{\linewidth}
$\bullet\ $ $11n_{71}$ \vspace{0.5em} \\
\begin{tabular}{l|llll}
$k \setminus j$ & $2$ & $4$ & $6$ & $8$ \\
\hline
$4$ & $2q^{-4}$ & $2q^{-6}$ &  &  \\
$2$ & $3q^{-2}$ & $2q^{-4}$ &  &  \\
$0$ & $5$ & $8q^{-2}$ & $3q^{-4}$ &  \\
$-2$ & $3q^{2}$ & $5$ & $2q^{-2}$ &  \\
$-4$ & $2q^{4}$ & $8q^{2}$ & $7$ & $q^{-2}$ \\
$-6$ &  & $2q^{4}$ & $2q^{2}$ &  \\
$-8$ &  & $2q^{6}$ & $3q^{4}$ & $q^{2}$ \\
\end{tabular}
\vspace{2em}
\end{minipage}
%
\begin{minipage}{\linewidth}
$\bullet\ $ $11n_{72}$ \vspace{0.5em} \\
\begin{tabular}{l|llll}
$k \setminus j$ & $4$ & $6$ & $8$ & $10$ \\
\hline
$4$ & $3q^{-4}$ & $2q^{-6}$ &  &  \\
$2$ & $3q^{-2}$ & $3q^{-4}$ &  &  \\
$0$ & $7$ & $10q^{-2}$ & $3q^{-4}$ &  \\
$-2$ & $3q^{2}$ & $7$ & $4q^{-2}$ &  \\
$-4$ & $3q^{4}$ & $10q^{2}$ & $8$ & $q^{-2}$ \\
$-6$ &  & $3q^{4}$ & $4q^{2}$ & $1$ \\
$-8$ &  & $2q^{6}$ & $3q^{4}$ & $q^{2}$ \\
\end{tabular}
\vspace{2em}
\end{minipage}
%
\begin{minipage}{\linewidth}
$\bullet\ $ $11n_{73}$ \vspace{0.5em} \\
\begin{tabular}{l|llll}
$k \setminus j$ & $-2$ & $0$ & $2$ & $4$ \\
\hline
$8$ & $q^{-4}$ & $q^{-6}$ &  &  \\
$4$ & $2$ & $q^{-4}$ + $3q^{-2}$ & $q^{-6}$ + $q^{-4}$ &  \\
$2$ &  & $q^{-2}$ & $q^{-4}$ &  \\
$0$ & $q^{4}$ & $3$ + $3q^{2}$ & $3q^{-2}$ + $2$ & $q^{-4}$ \\
$-2$ &  & $q^{2}$ & $2$ & $q^{-2}$ \\
$-4$ &  & $q^{4}$ + $q^{6}$ & $3q^{2}$ + $q^{4}$ & $2$ \\
$-6$ &  &  & $q^{4}$ & $q^{2}$ \\
$-8$ &  &  & $q^{6}$ & $q^{4}$ \\
\end{tabular}
\vspace{2em}
\end{minipage}
%
\begin{minipage}{\linewidth}
$\bullet\ $ $11n_{74}$ \vspace{0.5em} \\
\begin{tabular}{l|llll}
$k \setminus j$ & $-2$ & $0$ & $2$ & $4$ \\
\hline
$8$ & $q^{-4}$ & $q^{-6}$ &  &  \\
$4$ & $2$ & $q^{-4}$ + $3q^{-2}$ & $q^{-6}$ + $q^{-4}$ &  \\
$2$ &  & $q^{-2}$ & $q^{-4}$ &  \\
$0$ & $q^{4}$ & $3$ + $3q^{2}$ & $3q^{-2}$ + $2$ & $q^{-4}$ \\
$-2$ &  & $q^{2}$ & $2$ & $q^{-2}$ \\
$-4$ &  & $q^{4}$ + $q^{6}$ & $3q^{2}$ + $q^{4}$ & $2$ \\
$-6$ &  &  & $q^{4}$ & $q^{2}$ \\
$-8$ &  &  & $q^{6}$ & $q^{4}$ \\
\end{tabular}
\vspace{2em}
\end{minipage}
%
\begin{minipage}{\linewidth}
$\bullet\ $ $11n_{75}$ \vspace{0.5em} \\
\begin{tabular}{l|llll}
$k \setminus j$ & $-8$ & $-6$ & $-4$ & $-2$ \\
\hline
$8$ & $q^{-2}$ & $3q^{-4}$ & $2q^{-6}$ &  \\
$6$ &  & $2q^{-2}$ & $2q^{-4}$ &  \\
$4$ & $q^{2}$ & $7$ & $8q^{-2}$ & $2q^{-4}$ \\
$2$ &  & $2q^{2}$ & $5$ & $3q^{-2}$ \\
$0$ &  & $3q^{4}$ & $8q^{2}$ & $5$ \\
$-2$ &  &  & $2q^{4}$ & $3q^{2}$ \\
$-4$ &  &  & $2q^{6}$ & $2q^{4}$ \\
\end{tabular}
\vspace{2em}
\end{minipage}
%
\begin{minipage}{\linewidth}
$\bullet\ $ $11n_{76}$ \vspace{0.5em} \\
\begin{tabular}{l|llll}
$k \setminus j$ & $-8$ & $-6$ & $-4$ & $-2$ \\
\hline
$8$ & $q^{-2}$ & $q^{-6}$ + $q^{-4}$ & $q^{-8}$ &  \\
$6$ &  & $q^{-4}$ & $q^{-6}$ &  \\
$4$ & $q^{2}$ & $4q^{-2}$ + $2$ & $5q^{-4}$ + $q^{-2}$ & $q^{-6}$ \\
$2$ &  & $2$ & $3q^{-2}$ & $q^{-4}$ \\
$0$ &  & $4q^{2}$ + $q^{4}$ & $7$ + $q^{2}$ & $3q^{-2}$ \\
$-2$ &  & $q^{4}$ & $3q^{2}$ & $2$ \\
$-4$ &  & $q^{6}$ & $5q^{4}$ & $3q^{2}$ \\
$-6$ &  &  & $q^{6}$ & $q^{4}$ \\
$-8$ &  &  & $q^{8}$ & $q^{6}$ \\
\end{tabular}
\vspace{2em}
\end{minipage}
%
\begin{minipage}{\linewidth}
$\bullet\ $ $11n_{77}$ \vspace{0.5em} \\
\begin{tabular}{l|llll}
$k \setminus j$ & $8$ & $10$ & $12$ & $14$ \\
\hline
$8$ & $q^{-8}$ &  &  &  \\
$4$ & $q^{-4}$ & $q^{-6}$ &  &  \\
$0$ & $3q^{-2}$ + $1$ & $3q^{-4}$ + $q^{-2}$ &  &  \\
$-2$ & $2$ & $2q^{-2}$ &  &  \\
$-4$ & $3q^{2}$ + $q^{4}$ & $7$ + $q^{2}$ & $4q^{-2}$ &  \\
$-6$ &  & $2q^{2}$ & $2$ &  \\
$-8$ & $q^{8}$ & $3q^{4}$ + $q^{6}$ & $4q^{2}$ & $1$ \\
\end{tabular}
\vspace{2em}
\end{minipage}
%
\begin{minipage}{\linewidth}
$\bullet\ $ $11n_{78}$ \vspace{0.5em} \\
\begin{tabular}{l|llll}
$k \setminus j$ & $2$ & $4$ & $6$ & $8$ \\
\hline
$8$ & $q^{-6}$ & $q^{-8}$ &  &  \\
$6$ & $q^{-4}$ & $q^{-6}$ &  &  \\
$4$ & $3q^{-2}$ & $5q^{-4}$ & $q^{-6}$ &  \\
$2$ & $2$ & $3q^{-2}$ & $q^{-4}$ &  \\
$0$ & $3q^{2}$ & $q^{-2}$ + $7$ & $q^{-4}$ + $4q^{-2}$ &  \\
$-2$ & $q^{4}$ & $3q^{2}$ & $2$ &  \\
$-4$ & $q^{6}$ & $q^{2}$ + $5q^{4}$ & $2$ + $4q^{2}$ & $q^{-2}$ \\
$-6$ &  & $q^{6}$ & $q^{4}$ &  \\
$-8$ &  & $q^{8}$ & $q^{4}$ + $q^{6}$ & $q^{2}$ \\
\end{tabular}
\vspace{2em}
\end{minipage}
%
\begin{minipage}{\linewidth}
$\bullet\ $ $11n_{79}$ \vspace{0.5em} \\
\begin{tabular}{l|llll}
$k \setminus j$ & $-2$ & $0$ & $2$ & $4$ \\
\hline
$6$ & $q^{-2}$ & $q^{-4}$ &  &  \\
$4$ &  & $q^{-2}$ & $q^{-4}$ &  \\
$2$ & $q^{2}$ & $2$ & $q^{-2}$ &  \\
$0$ &  & $1$ + $q^{2}$ & $2$ & $q^{-2}$ \\
$-2$ &  & $q^{4}$ & $q^{2}$ &  \\
$-4$ &  &  & $q^{4}$ & $q^{2}$ \\
\end{tabular}
\vspace{2em}
\end{minipage}
%
\begin{minipage}{\linewidth}
$\bullet\ $ $11n_{80}$ \vspace{0.5em} \\
\begin{tabular}{l|lllll}
$k \setminus j$ & $-8$ & $-6$ & $-4$ & $-2$ & $0$ \\
\hline
$6$ & $1$ & $2q^{-2}$ & $q^{-4}$ &  &  \\
$4$ &  & $3$ & $q^{-4}$ + $3q^{-2}$ & $q^{-6}$ &  \\
$2$ &  & $2q^{2}$ & $4$ & $2q^{-2}$ &  \\
$0$ &  &  & $1$ + $3q^{2}$ & $2q^{-2}$ + $3$ & $q^{-4}$ \\
$-2$ &  &  & $q^{4}$ & $2q^{2}$ &  \\
$-4$ &  &  & $q^{4}$ & $2q^{2}$ & $1$ \\
$-8$ &  &  &  & $q^{6}$ & $q^{4}$ \\
\end{tabular}
\vspace{2em}
\end{minipage}
%
\begin{minipage}{\linewidth}
$\bullet\ $ $11n_{81}$ \vspace{0.5em} \\
\begin{tabular}{l|llll}
$k \setminus j$ & $4$ & $6$ & $8$ & $10$ \\
\hline
$8$ & $q^{-6}$ & $q^{-8}$ &  &  \\
$6$ &  & $q^{-6}$ &  &  \\
$4$ & $3q^{-2}$ & $4q^{-4}$ & $q^{-6}$ &  \\
$2$ &  & $q^{-2}$ & $q^{-4}$ &  \\
$0$ & $3q^{2}$ & $q^{-2}$ + $6$ & $q^{-4}$ + $3q^{-2}$ &  \\
$-2$ &  & $1$ + $q^{2}$ & $q^{-2}$ + $1$ &  \\
$-4$ & $q^{6}$ & $q^{2}$ + $4q^{4}$ & $2$ + $3q^{2}$ & $q^{-2}$ \\
$-6$ &  & $q^{6}$ & $q^{2}$ + $q^{4}$ & $1$ \\
$-8$ &  & $q^{8}$ & $q^{4}$ + $q^{6}$ & $q^{2}$ \\
\end{tabular}
\vspace{2em}
\end{minipage}
%
\begin{minipage}{\linewidth}
$\bullet\ $ $11n_{82}$ \vspace{0.5em} \\
\begin{tabular}{l|lll}
$k \setminus j$ & $-2$ & $0$ & $2$ \\
\hline
$8$ & $q^{-4}$ & $q^{-6}$ &  \\
$6$ & $q^{-2}$ & $q^{-4}$ &  \\
$4$ & $1$ & $2q^{-2}$ & $q^{-4}$ \\
$2$ & $q^{2}$ & $2$ & $q^{-2}$ \\
$0$ & $q^{4}$ & $1$ + $2q^{2}$ & $1$ \\
$-2$ &  & $q^{4}$ & $q^{2}$ \\
$-4$ &  & $q^{6}$ & $q^{4}$ \\
\end{tabular}
\vspace{2em}
\end{minipage}
%
\begin{minipage}{\linewidth}
$\bullet\ $ $11n_{83}$ \vspace{0.5em} \\
\begin{tabular}{l|lllll}
$k \setminus j$ & $-4$ & $-2$ & $0$ & $2$ & $4$ \\
\hline
$6$ & $q^{-2}$ & $q^{-4}$ &  &  &  \\
$4$ & $2$ & $4q^{-2}$ & $2q^{-4}$ &  &  \\
$2$ & $q^{2}$ & $5$ & $4q^{-2}$ &  &  \\
$0$ &  & $4q^{2}$ & $8$ & $3q^{-2}$ &  \\
$-2$ &  & $q^{4}$ & $4q^{2}$ & $3$ &  \\
$-4$ &  &  & $2q^{4}$ & $3q^{2}$ & $1$ \\
\end{tabular}
\vspace{2em}
\end{minipage}
%
\begin{minipage}{\linewidth}
$\bullet\ $ $11n_{84}$ \vspace{0.5em} \\
\begin{tabular}{l|llll}
$k \setminus j$ & $-8$ & $-6$ & $-4$ & $-2$ \\
\hline
$8$ & $q^{-2}$ & $q^{-4}$ &  &  \\
$6$ & $2$ & $3q^{-2}$ & $q^{-4}$ &  \\
$4$ & $q^{2}$ & $4$ & $3q^{-2}$ &  \\
$2$ &  & $3q^{2}$ & $5$ & $2q^{-2}$ \\
$0$ &  & $q^{4}$ & $3q^{2}$ & $2$ \\
$-2$ &  &  & $q^{4}$ & $2q^{2}$ \\
\end{tabular}
\vspace{2em}
\end{minipage}
%
\begin{minipage}{\linewidth}
$\bullet\ $ $11n_{85}$ \vspace{0.5em} \\
\begin{tabular}{l|llll}
$k \setminus j$ & $-2$ & $0$ & $2$ & $4$ \\
\hline
$4$ & $q^{-2}$ & $2q^{-4}$ & $q^{-6}$ &  \\
$2$ & $1$ & $3q^{-2}$ & $2q^{-4}$ &  \\
$0$ & $q^{2}$ & $5$ & $4q^{-2}$ & $q^{-4}$ \\
$-2$ &  & $3q^{2}$ & $5$ & $2q^{-2}$ \\
$-4$ &  & $2q^{4}$ & $4q^{2}$ & $2$ \\
$-6$ &  &  & $2q^{4}$ & $2q^{2}$ \\
$-8$ &  &  & $q^{6}$ & $q^{4}$ \\
\end{tabular}
\vspace{2em}
\end{minipage}
%
\begin{minipage}{\linewidth}
$\bullet\ $ $11n_{86}$ \vspace{0.5em} \\
\begin{tabular}{l|lll}
$k \setminus j$ & $0$ & $2$ & $4$ \\
\hline
$4$ & $q^{-4}$ & $q^{-6}$ &  \\
$2$ & $2q^{-2}$ & $2q^{-4}$ &  \\
$0$ & $3$ & $3q^{-2}$ & $q^{-4}$ \\
$-2$ & $2q^{2}$ & $4$ & $2q^{-2}$ \\
$-4$ & $q^{4}$ & $3q^{2}$ & $2$ \\
$-6$ &  & $2q^{4}$ & $2q^{2}$ \\
$-8$ &  & $q^{6}$ & $q^{4}$ \\
\end{tabular}
\vspace{2em}
\end{minipage}
%
\begin{minipage}{\linewidth}
$\bullet\ $ $11n_{87}$ \vspace{0.5em} \\
\begin{tabular}{l|llll}
$k \setminus j$ & $-8$ & $-6$ & $-4$ & $-2$ \\
\hline
$8$ & $q^{-2}$ & $2q^{-4}$ & $q^{-6}$ &  \\
$6$ & $1$ & $3q^{-2}$ & $2q^{-4}$ &  \\
$4$ & $q^{2}$ & $5$ & $5q^{-2}$ & $q^{-4}$ \\
$2$ &  & $3q^{2}$ & $6$ & $3q^{-2}$ \\
$0$ &  & $2q^{4}$ & $5q^{2}$ & $3$ \\
$-2$ &  &  & $2q^{4}$ & $3q^{2}$ \\
$-4$ &  &  & $q^{6}$ & $q^{4}$ \\
\end{tabular}
\vspace{2em}
\end{minipage}
%
\begin{minipage}{\linewidth}
$\bullet\ $ $11n_{88}$ \vspace{0.5em} \\
\begin{tabular}{l|llll}
$k \setminus j$ & $4$ & $6$ & $8$ & $10$ \\
\hline
$8$ & $q^{-6}$ & $q^{-8}$ &  &  \\
$6$ &  & $q^{-6}$ &  &  \\
$4$ & $q^{-2}$ & $2q^{-4}$ & $q^{-6}$ &  \\
$2$ &  & $q^{-2}$ & $q^{-4}$ &  \\
$0$ & $q^{2}$ & $q^{-2}$ + $2$ & $q^{-4}$ + $q^{-2}$ &  \\
$-2$ &  & $1$ + $q^{2}$ & $q^{-2}$ + $1$ &  \\
$-4$ & $q^{6}$ & $q^{2}$ + $2q^{4}$ & $2$ + $q^{2}$ & $q^{-2}$ \\
$-6$ &  & $q^{6}$ & $q^{2}$ + $q^{4}$ & $1$ \\
$-8$ &  & $q^{8}$ & $q^{4}$ + $q^{6}$ & $q^{2}$ \\
\end{tabular}
\vspace{2em}
\end{minipage}
%
\begin{minipage}{\linewidth}
$\bullet\ $ $11n_{89}$ \vspace{0.5em} \\
\begin{tabular}{l|llll}
$k \setminus j$ & $-10$ & $-8$ & $-6$ & $-4$ \\
\hline
$8$ & $q^{-2}$ & $2q^{-4}$ & $q^{-6}$ &  \\
$6$ & $1$ & $4q^{-2}$ & $3q^{-4}$ &  \\
$4$ & $q^{2}$ & $5$ & $6q^{-2}$ & $2q^{-4}$ \\
$2$ &  & $4q^{2}$ & $7$ & $3q^{-2}$ \\
$0$ &  & $2q^{4}$ & $6q^{2}$ & $4$ \\
$-2$ &  &  & $3q^{4}$ & $3q^{2}$ \\
$-4$ &  &  & $q^{6}$ & $2q^{4}$ \\
\end{tabular}
\vspace{2em}
\end{minipage}
%
\begin{minipage}{\linewidth}
$\bullet\ $ $11n_{90}$ \vspace{0.5em} \\
\begin{tabular}{l|lllll}
$k \setminus j$ & $2$ & $4$ & $6$ & $8$ & $10$ \\
\hline
$6$ & $q^{-4}$ & $q^{-6}$ &  &  &  \\
$4$ & $q^{-2}$ & $3q^{-4}$ & $q^{-6}$ &  &  \\
$2$ & $1$ & $3q^{-2}$ & $2q^{-4}$ &  &  \\
$0$ & $q^{2}$ & $4$ & $4q^{-2}$ & $q^{-4}$ &  \\
$-2$ & $q^{4}$ & $3q^{2}$ & $3$ & $q^{-2}$ &  \\
$-4$ &  & $3q^{4}$ & $1$ + $4q^{2}$ & $q^{-2}$ + $1$ &  \\
$-6$ &  & $q^{6}$ & $2q^{4}$ & $q^{2}$ &  \\
$-8$ &  &  & $q^{6}$ & $q^{2}$ + $q^{4}$ & $1$ \\
\end{tabular}
\vspace{2em}
\end{minipage}
%
\begin{minipage}{\linewidth}
$\bullet\ $ $11n_{91}$ \vspace{0.5em} \\
\begin{tabular}{l|lllll}
$k \setminus j$ & $-10$ & $-8$ & $-6$ & $-4$ & $-2$ \\
\hline
$8$ & $1$ & $2q^{-2}$ & $q^{-4}$ &  &  \\
$6$ &  & $2$ & $2q^{-2}$ &  &  \\
$4$ &  & $2q^{2}$ & $5$ & $3q^{-2}$ &  \\
$2$ &  &  & $2q^{2}$ & $3$ & $q^{-2}$ \\
$0$ &  &  & $q^{4}$ & $3q^{2}$ & $2$ \\
$-2$ &  &  &  &  & $q^{2}$ \\
\end{tabular}
\vspace{2em}
\end{minipage}
%
\begin{minipage}{\linewidth}
$\bullet\ $ $11n_{92}$ \vspace{0.5em} \\
\begin{tabular}{l|llll}
$k \setminus j$ & $-4$ & $-2$ & $0$ & $2$ \\
\hline
$4$ & $1$ & $q^{-4}$ + $q^{-2}$ & $q^{-6}$ &  \\
$2$ &  & $q^{-2}$ & $q^{-4}$ &  \\
$0$ &  & $1$ + $q^{2}$ & $2q^{-2}$ + $2$ & $q^{-4}$ \\
$-2$ &  & $q^{2}$ & $2$ & $q^{-2}$ \\
$-4$ &  & $q^{4}$ & $2q^{2}$ & $1$ \\
$-6$ &  &  & $q^{4}$ & $q^{2}$ \\
$-8$ &  &  & $q^{6}$ & $q^{4}$ \\
\end{tabular}
\vspace{2em}
\end{minipage}
%
\begin{minipage}{\linewidth}
$\bullet\ $ $11n_{93}$ \vspace{0.5em} \\
\begin{tabular}{l|llll}
$k \setminus j$ & $6$ & $8$ & $10$ & $12$ \\
\hline
$6$ & $q^{-6}$ &  &  &  \\
$4$ & $2q^{-4}$ & $2q^{-6}$ &  &  \\
$2$ & $3q^{-2}$ & $3q^{-4}$ &  &  \\
$0$ & $3$ & $5q^{-2}$ & $2q^{-4}$ &  \\
$-2$ & $3q^{2}$ & $5$ & $2q^{-2}$ &  \\
$-4$ & $2q^{4}$ & $1$ + $5q^{2}$ & $q^{-2}$ + $3$ &  \\
$-6$ & $q^{6}$ & $3q^{4}$ & $2q^{2}$ &  \\
$-8$ &  & $2q^{6}$ & $q^{2}$ + $2q^{4}$ & $1$ \\
\end{tabular}
\vspace{2em}
\end{minipage}
%
\begin{minipage}{\linewidth}
$\bullet\ $ $11n_{94}$ \vspace{0.5em} \\
\begin{tabular}{l|llll}
$k \setminus j$ & $-2$ & $0$ & $2$ & $4$ \\
\hline
$4$ & $q^{-2}$ & $2q^{-4}$ & $q^{-6}$ &  \\
$2$ & $1$ & $4q^{-2}$ & $3q^{-4}$ &  \\
$0$ & $q^{2}$ & $6$ & $5q^{-2}$ & $q^{-4}$ \\
$-2$ &  & $4q^{2}$ & $7$ & $3q^{-2}$ \\
$-4$ &  & $2q^{4}$ & $5q^{2}$ & $3$ \\
$-6$ &  &  & $3q^{4}$ & $3q^{2}$ \\
$-8$ &  &  & $q^{6}$ & $q^{4}$ \\
\end{tabular}
\vspace{2em}
\end{minipage}
%
\begin{minipage}{\linewidth}
$\bullet\ $ $11n_{95}$ \vspace{0.5em} \\
\begin{tabular}{l|llll}
$k \setminus j$ & $4$ & $6$ & $8$ & $10$ \\
\hline
$4$ & $2q^{-4}$ & $q^{-6}$ &  &  \\
$2$ & $2q^{-2}$ & $2q^{-4}$ &  &  \\
$0$ & $3$ & $4q^{-2}$ & $q^{-4}$ &  \\
$-2$ & $2q^{2}$ & $4$ & $2q^{-2}$ &  \\
$-4$ & $2q^{4}$ & $1$ + $4q^{2}$ & $q^{-2}$ + $2$ &  \\
$-6$ &  & $2q^{4}$ & $2q^{2}$ &  \\
$-8$ &  & $q^{6}$ & $q^{2}$ + $q^{4}$ & $1$ \\
\end{tabular}
\vspace{2em}
\end{minipage}
%
\begin{minipage}{\linewidth}
$\bullet\ $ $11n_{96}$ \vspace{0.5em} \\
\begin{tabular}{l|llll}
$k \setminus j$ & $-2$ & $0$ & $2$ & $4$ \\
\hline
$8$ & $q^{-4}$ & $q^{-6}$ &  &  \\
$6$ & $q^{-2}$ & $q^{-4}$ &  &  \\
$4$ & $1$ & $2q^{-2}$ & $q^{-4}$ &  \\
$2$ & $q^{2}$ & $q^{-2}$ + $2$ & $q^{-4}$ + $q^{-2}$ &  \\
$0$ & $q^{4}$ & $2$ + $2q^{2}$ & $q^{-2}$ + $1$ &  \\
$-2$ &  & $q^{2}$ + $q^{4}$ & $2$ + $q^{2}$ & $q^{-2}$ \\
$-4$ &  & $q^{6}$ & $q^{2}$ + $q^{4}$ & $1$ \\
$-6$ &  &  & $q^{4}$ & $q^{2}$ \\
\end{tabular}
\vspace{2em}
\end{minipage}
%
\begin{minipage}{\linewidth}
$\bullet\ $ $11n_{98}$ \vspace{0.5em} \\
\begin{tabular}{l|llll}
$k \setminus j$ & $-4$ & $-2$ & $0$ & $2$ \\
\hline
$6$ & $2q^{-2}$ & $3q^{-4}$ & $q^{-6}$ &  \\
$4$ & $1$ & $4q^{-2}$ & $3q^{-4}$ &  \\
$2$ & $2q^{2}$ & $8$ & $7q^{-2}$ & $q^{-4}$ \\
$0$ &  & $4q^{2}$ & $8$ & $3q^{-2}$ \\
$-2$ &  & $3q^{4}$ & $7q^{2}$ & $4$ \\
$-4$ &  &  & $3q^{4}$ & $3q^{2}$ \\
$-6$ &  &  & $q^{6}$ & $q^{4}$ \\
\end{tabular}
\vspace{2em}
\end{minipage}
%
\begin{minipage}{\linewidth}
$\bullet\ $ $11n_{99}$ \vspace{0.5em} \\
\begin{tabular}{l|llll}
$k \setminus j$ & $-8$ & $-6$ & $-4$ & $-2$ \\
\hline
$8$ & $q^{-2}$ & $q^{-4}$ &  &  \\
$6$ & $1$ & $3q^{-2}$ & $2q^{-4}$ &  \\
$4$ & $q^{2}$ & $4$ & $3q^{-2}$ &  \\
$2$ &  & $3q^{2}$ & $6$ & $3q^{-2}$ \\
$0$ &  & $q^{4}$ & $3q^{2}$ & $2$ \\
$-2$ &  &  & $2q^{4}$ & $3q^{2}$ \\
\end{tabular}
\vspace{2em}
\end{minipage}
%
\begin{minipage}{\linewidth}
$\bullet\ $ $11n_{100}$ \vspace{0.5em} \\
\begin{tabular}{l|lllll}
$k \setminus j$ & $-2$ & $0$ & $2$ & $4$ & $6$ \\
\hline
$4$ & $q^{-2}$ & $q^{-4}$ &  &  &  \\
$2$ & $3$ & $4q^{-2}$ & $q^{-4}$ &  &  \\
$0$ & $q^{2}$ & $6$ & $4q^{-2}$ &  &  \\
$-2$ &  & $4q^{2}$ & $6$ & $2q^{-2}$ &  \\
$-4$ &  & $q^{4}$ & $4q^{2}$ & $3$ &  \\
$-6$ &  &  & $q^{4}$ & $2q^{2}$ & $1$ \\
\end{tabular}
\vspace{2em}
\end{minipage}
%
\begin{minipage}{\linewidth}
$\bullet\ $ $11n_{101}$ \vspace{0.5em} \\
\begin{tabular}{l|lllll}
$k \setminus j$ & $-6$ & $-4$ & $-2$ & $0$ & $2$ \\
\hline
$4$ & $1$ & $2q^{-2}$ & $q^{-4}$ &  &  \\
$2$ &  & $3$ & $4q^{-2}$ & $q^{-4}$ &  \\
$0$ &  & $2q^{2}$ & $5$ & $3q^{-2}$ &  \\
$-2$ &  &  & $4q^{2}$ & $4$ & $q^{-2}$ \\
$-4$ &  &  & $q^{4}$ & $3q^{2}$ & $2$ \\
$-6$ &  &  &  & $q^{4}$ & $q^{2}$ \\
\end{tabular}
\vspace{2em}
\end{minipage}
%
\begin{minipage}{\linewidth}
$\bullet\ $ $11n_{102}$ \vspace{0.5em} \\
\begin{tabular}{l|lllll}
$k \setminus j$ & $-8$ & $-6$ & $-4$ & $-2$ & $0$ \\
\hline
$8$ & $1$ & $q^{-2}$ &  &  &  \\
$6$ &  & $1$ & $q^{-2}$ &  &  \\
$4$ &  & $q^{2}$ & $q^{-2}$ + $1$ & $q^{-4}$ &  \\
$2$ &  &  & $1$ + $q^{2}$ & $q^{-2}$ + $1$ &  \\
$0$ &  &  & $q^{2}$ & $2$ & $q^{-2}$ \\
$-2$ &  &  &  & $q^{2}$ &  \\
$-4$ &  &  &  & $q^{4}$ & $q^{2}$ \\
\end{tabular}
\vspace{2em}
\end{minipage}
%
\begin{minipage}{\linewidth}
$\bullet\ $ $11n_{103}$ \vspace{0.5em} \\
\begin{tabular}{l|llll}
$k \setminus j$ & $-8$ & $-6$ & $-4$ & $-2$ \\
\hline
$6$ & $2q^{-2}$ & $3q^{-4}$ & $q^{-6}$ &  \\
$4$ & $2$ & $5q^{-2}$ & $3q^{-4}$ &  \\
$2$ & $2q^{2}$ & $7$ & $6q^{-2}$ & $q^{-4}$ \\
$0$ &  & $5q^{2}$ & $7$ & $2q^{-2}$ \\
$-2$ &  & $3q^{4}$ & $6q^{2}$ & $3$ \\
$-4$ &  &  & $3q^{4}$ & $2q^{2}$ \\
$-6$ &  &  & $q^{6}$ & $q^{4}$ \\
\end{tabular}
\vspace{2em}
\end{minipage}
%
\begin{minipage}{\linewidth}
$\bullet\ $ $11n_{104}$ \vspace{0.5em} \\
\begin{tabular}{l|llll}
$k \setminus j$ & $4$ & $6$ & $8$ & $10$ \\
\hline
$8$ & $q^{-6}$ & $q^{-8}$ &  &  \\
$6$ &  & $q^{-6}$ &  &  \\
$4$ & $q^{-2}$ & $2q^{-4}$ & $q^{-6}$ &  \\
$2$ &  & $q^{-2}$ & $q^{-4}$ &  \\
$0$ & $1$ + $q^{2}$ & $2q^{-2}$ + $2$ & $q^{-4}$ + $q^{-2}$ &  \\
$-2$ &  & $2$ + $q^{2}$ & $2q^{-2}$ + $1$ &  \\
$-4$ & $q^{6}$ & $2q^{2}$ + $2q^{4}$ & $3$ + $q^{2}$ & $q^{-2}$ \\
$-6$ &  & $q^{6}$ & $2q^{2}$ + $q^{4}$ & $2$ \\
$-8$ &  & $q^{8}$ & $q^{4}$ + $q^{6}$ & $q^{2}$ \\
\end{tabular}
\vspace{2em}
\end{minipage}
%
\begin{minipage}{\linewidth}
$\bullet\ $ $11n_{105}$ \vspace{0.5em} \\
\begin{tabular}{l|llll}
$k \setminus j$ & $4$ & $6$ & $8$ & $10$ \\
\hline
$4$ & $2q^{-4}$ & $q^{-6}$ &  &  \\
$2$ & $3q^{-2}$ & $3q^{-4}$ &  &  \\
$0$ & $5$ & $7q^{-2}$ & $2q^{-4}$ &  \\
$-2$ & $3q^{2}$ & $8$ & $5q^{-2}$ &  \\
$-4$ & $2q^{4}$ & $7q^{2}$ & $6$ & $q^{-2}$ \\
$-6$ &  & $3q^{4}$ & $5q^{2}$ & $2$ \\
$-8$ &  & $q^{6}$ & $2q^{4}$ & $q^{2}$ \\
\end{tabular}
\vspace{2em}
\end{minipage}
%
\begin{minipage}{\linewidth}
$\bullet\ $ $11n_{106}$ \vspace{0.5em} \\
\begin{tabular}{l|llll}
$k \setminus j$ & $-4$ & $-2$ & $0$ & $2$ \\
\hline
$4$ &  & $q^{-4}$ & $q^{-6}$ &  \\
$2$ & $1$ & $2q^{-2}$ & $q^{-4}$ &  \\
$0$ &  & $2$ & $3q^{-2}$ & $q^{-4}$ \\
$-2$ &  & $2q^{2}$ & $2$ & $q^{-2}$ \\
$-4$ &  & $q^{4}$ & $3q^{2}$ & $2$ \\
$-6$ &  &  & $q^{4}$ & $q^{2}$ \\
$-8$ &  &  & $q^{6}$ & $q^{4}$ \\
\end{tabular}
\vspace{2em}
\end{minipage}
%
\begin{minipage}{\linewidth}
$\bullet\ $ $11n_{107}$ \vspace{0.5em} \\
\begin{tabular}{l|llll}
$k \setminus j$ & $2$ & $4$ & $6$ & $8$ \\
\hline
$8$ & $q^{-6}$ & $q^{-8}$ &  &  \\
$6$ & $q^{-4}$ & $q^{-6}$ &  &  \\
$4$ & $q^{-2}$ & $3q^{-4}$ & $q^{-6}$ &  \\
$2$ & $1$ & $2q^{-2}$ & $q^{-4}$ &  \\
$0$ & $q^{2}$ & $q^{-2}$ + $3$ & $q^{-4}$ + $2q^{-2}$ &  \\
$-2$ & $q^{4}$ & $1$ + $2q^{2}$ & $q^{-2}$ + $1$ &  \\
$-4$ & $q^{6}$ & $q^{2}$ + $3q^{4}$ & $2$ + $2q^{2}$ & $q^{-2}$ \\
$-6$ &  & $q^{6}$ & $q^{2}$ + $q^{4}$ & $1$ \\
$-8$ &  & $q^{8}$ & $q^{4}$ + $q^{6}$ & $q^{2}$ \\
\end{tabular}
\vspace{2em}
\end{minipage}
%
\begin{minipage}{\linewidth}
$\bullet\ $ $11n_{108}$ \vspace{0.5em} \\
\begin{tabular}{l|llll}
$k \setminus j$ & $-10$ & $-8$ & $-6$ & $-4$ \\
\hline
$8$ & $q^{-2}$ & $2q^{-4}$ & $q^{-6}$ &  \\
$6$ & $1$ & $5q^{-2}$ & $4q^{-4}$ &  \\
$4$ & $q^{2}$ & $6$ & $7q^{-2}$ & $2q^{-4}$ \\
$2$ &  & $5q^{2}$ & $9$ & $4q^{-2}$ \\
$0$ &  & $2q^{4}$ & $7q^{2}$ & $5$ \\
$-2$ &  &  & $4q^{4}$ & $4q^{2}$ \\
$-4$ &  &  & $q^{6}$ & $2q^{4}$ \\
\end{tabular}
\vspace{2em}
\end{minipage}
%
\begin{minipage}{\linewidth}
$\bullet\ $ $11n_{109}$ \vspace{0.5em} \\
\begin{tabular}{l|llll}
$k \setminus j$ & $-10$ & $-8$ & $-6$ & $-4$ \\
\hline
$8$ & $q^{-2}$ & $2q^{-4}$ & $q^{-6}$ &  \\
$6$ &  & $3q^{-2}$ & $3q^{-4}$ &  \\
$4$ & $q^{2}$ & $5$ & $6q^{-2}$ & $2q^{-4}$ \\
$2$ &  & $3q^{2}$ & $6$ & $3q^{-2}$ \\
$0$ &  & $2q^{4}$ & $6q^{2}$ & $4$ \\
$-2$ &  &  & $3q^{4}$ & $3q^{2}$ \\
$-4$ &  &  & $q^{6}$ & $2q^{4}$ \\
\end{tabular}
\vspace{2em}
\end{minipage}
%
\begin{minipage}{\linewidth}
$\bullet\ $ $11n_{110}$ \vspace{0.5em} \\
\begin{tabular}{l|llll}
$k \setminus j$ & $-2$ & $0$ & $2$ & $4$ \\
\hline
$4$ &  & $q^{-4}$ & $q^{-6}$ &  \\
$2$ & $1$ & $3q^{-2}$ & $2q^{-4}$ &  \\
$0$ &  & $4$ & $4q^{-2}$ & $q^{-4}$ \\
$-2$ &  & $3q^{2}$ & $5$ & $2q^{-2}$ \\
$-4$ &  & $q^{4}$ & $4q^{2}$ & $3$ \\
$-6$ &  &  & $2q^{4}$ & $2q^{2}$ \\
$-8$ &  &  & $q^{6}$ & $q^{4}$ \\
\end{tabular}
\vspace{2em}
\end{minipage}
%
\begin{minipage}{\linewidth}
$\bullet\ $ $11n_{111}$ \vspace{0.5em} \\
\begin{tabular}{l|llll}
$k \setminus j$ & $0$ & $2$ & $4$ & $6$ \\
\hline
$8$ & $q^{-4}$ & $q^{-6}$ &  &  \\
$4$ & $1$ & $2q^{-2}$ & $q^{-4}$ &  \\
$2$ &  & $2q^{-2}$ & $q^{-4}$ &  \\
$0$ & $q^{4}$ & $2$ + $2q^{2}$ & $2q^{-2}$ + $1$ &  \\
$-2$ &  & $2q^{2}$ & $3$ & $q^{-2}$ \\
$-4$ &  & $q^{6}$ & $2q^{2}$ + $q^{4}$ & $2$ \\
$-6$ &  &  & $q^{4}$ & $q^{2}$ \\
\end{tabular}
\vspace{2em}
\end{minipage}
%
\begin{minipage}{\linewidth}
$\bullet\ $ $11n_{112}$ \vspace{0.5em} \\
\begin{tabular}{l|llll}
$k \setminus j$ & $0$ & $2$ & $4$ & $6$ \\
\hline
$4$ & $q^{-2}$ & $2q^{-4}$ & $q^{-6}$ &  \\
$2$ & $1$ & $4q^{-2}$ & $2q^{-4}$ &  \\
$0$ & $q^{2}$ & $6$ & $6q^{-2}$ & $q^{-4}$ \\
$-2$ &  & $4q^{2}$ & $6$ & $2q^{-2}$ \\
$-4$ &  & $2q^{4}$ & $6q^{2}$ & $4$ \\
$-6$ &  &  & $2q^{4}$ & $2q^{2}$ \\
$-8$ &  &  & $q^{6}$ & $q^{4}$ \\
\end{tabular}
\vspace{2em}
\end{minipage}
%
\begin{minipage}{\linewidth}
$\bullet\ $ $11n_{113}$ \vspace{0.5em} \\
\begin{tabular}{l|lllll}
$k \setminus j$ & $-10$ & $-8$ & $-6$ & $-4$ & $-2$ \\
\hline
$8$ & $1$ & $2q^{-2}$ & $q^{-4}$ &  &  \\
$6$ &  & $3$ & $3q^{-2}$ &  &  \\
$4$ &  & $2q^{2}$ & $5$ & $3q^{-2}$ &  \\
$2$ &  &  & $3q^{2}$ & $4$ & $q^{-2}$ \\
$0$ &  &  & $q^{4}$ & $3q^{2}$ & $2$ \\
$-2$ &  &  &  &  & $q^{2}$ \\
\end{tabular}
\vspace{2em}
\end{minipage}
%
\begin{minipage}{\linewidth}
$\bullet\ $ $11n_{114}$ \vspace{0.5em} \\
\begin{tabular}{l|lllll}
$k \setminus j$ & $-4$ & $-2$ & $0$ & $2$ & $4$ \\
\hline
$4$ & $1$ & $3q^{-2}$ & $2q^{-4}$ &  &  \\
$2$ &  & $4$ & $5q^{-2}$ & $q^{-4}$ &  \\
$0$ &  & $3q^{2}$ & $8$ & $4q^{-2}$ &  \\
$-2$ &  &  & $5q^{2}$ & $6$ & $q^{-2}$ \\
$-4$ &  &  & $2q^{4}$ & $4q^{2}$ & $2$ \\
$-6$ &  &  &  & $q^{4}$ & $q^{2}$ \\
\end{tabular}
\vspace{2em}
\end{minipage}
%
\begin{minipage}{\linewidth}
$\bullet\ $ $11n_{115}$ \vspace{0.5em} \\
\begin{tabular}{l|lllll}
$k \setminus j$ & $-4$ & $-2$ & $0$ & $2$ & $4$ \\
\hline
$6$ & $q^{-2}$ & $2q^{-4}$ & $q^{-6}$ &  &  \\
$4$ & $2$ & $5q^{-2}$ & $3q^{-4}$ &  &  \\
$2$ & $q^{2}$ & $8$ & $8q^{-2}$ & $q^{-4}$ &  \\
$0$ &  & $5q^{2}$ & $10$ & $4q^{-2}$ &  \\
$-2$ &  & $2q^{4}$ & $8q^{2}$ & $6$ &  \\
$-4$ &  &  & $3q^{4}$ & $4q^{2}$ & $1$ \\
$-6$ &  &  & $q^{6}$ & $q^{4}$ &  \\
\end{tabular}
\vspace{2em}
\end{minipage}
%
\begin{minipage}{\linewidth}
$\bullet\ $ $11n_{116}$ \vspace{0.5em} \\
\begin{tabular}{l|lllll}
$k \setminus j$ & $-6$ & $-4$ & $-2$ & $0$ & $2$ \\
\hline
$6$ & $1$ & $q^{-2}$ &  &  &  \\
$4$ &  & $1$ & $q^{-2}$ &  &  \\
$2$ &  & $q^{2}$ & $q^{-2}$ + $1$ & $q^{-4}$ &  \\
$0$ &  &  & $q^{2}$ & $2$ &  \\
$-2$ &  &  & $q^{2}$ & $2$ & $q^{-2}$ \\
$-6$ &  &  &  & $q^{4}$ & $q^{2}$ \\
\end{tabular}
\vspace{2em}
\end{minipage}
%
\begin{minipage}{\linewidth}
$\bullet\ $ $11n_{117}$ \vspace{0.5em} \\
\begin{tabular}{l|llll}
$k \setminus j$ & $-2$ & $0$ & $2$ & $4$ \\
\hline
$6$ & $q^{-2}$ & $q^{-4}$ &  &  \\
$4$ & $1$ & $3q^{-2}$ & $2q^{-4}$ &  \\
$2$ & $q^{2}$ & $3$ & $3q^{-2}$ &  \\
$0$ &  & $3q^{2}$ & $5$ & $2q^{-2}$ \\
$-2$ &  & $q^{4}$ & $3q^{2}$ & $2$ \\
$-4$ &  &  & $2q^{4}$ & $2q^{2}$ \\
\end{tabular}
\vspace{2em}
\end{minipage}
%
\begin{minipage}{\linewidth}
$\bullet\ $ $11n_{118}$ \vspace{0.5em} \\
\begin{tabular}{l|llll}
$k \setminus j$ & $4$ & $6$ & $8$ & $10$ \\
\hline
$4$ & $2q^{-4}$ & $q^{-6}$ &  &  \\
$2$ & $q^{-2}$ & $q^{-4}$ &  &  \\
$0$ & $2$ & $3q^{-2}$ & $q^{-4}$ &  \\
$-2$ & $q^{2}$ & $2$ & $q^{-2}$ &  \\
$-4$ & $2q^{4}$ & $1$ + $3q^{2}$ & $q^{-2}$ + $1$ &  \\
$-6$ &  & $q^{4}$ & $q^{2}$ &  \\
$-8$ &  & $q^{6}$ & $q^{2}$ + $q^{4}$ & $1$ \\
\end{tabular}
\vspace{2em}
\end{minipage}
%
\begin{minipage}{\linewidth}
$\bullet\ $ $11n_{119}$ \vspace{0.5em} \\
\begin{tabular}{l|llll}
$k \setminus j$ & $-4$ & $-2$ & $0$ & $2$ \\
\hline
$8$ & $q^{-4}$ & $q^{-6}$ &  &  \\
$6$ & $2q^{-2}$ & $2q^{-4}$ &  &  \\
$4$ & $4$ & $7q^{-2}$ & $3q^{-4}$ &  \\
$2$ & $2q^{2}$ & $7$ & $5q^{-2}$ &  \\
$0$ & $q^{4}$ & $7q^{2}$ & $9$ & $2q^{-2}$ \\
$-2$ &  & $2q^{4}$ & $5q^{2}$ & $3$ \\
$-4$ &  & $q^{6}$ & $3q^{4}$ & $2q^{2}$ \\
\end{tabular}
\vspace{2em}
\end{minipage}
%
\begin{minipage}{\linewidth}
$\bullet\ $ $11n_{120}$ \vspace{0.5em} \\
\begin{tabular}{l|llll}
$k \setminus j$ & $0$ & $2$ & $4$ & $6$ \\
\hline
$8$ & $q^{-6}$ & $q^{-8}$ &  &  \\
$6$ & $2q^{-4}$ & $2q^{-6}$ &  &  \\
$4$ & $3q^{-2}$ & $4q^{-4}$ & $q^{-6}$ &  \\
$2$ & $2$ & $5q^{-2}$ & $2q^{-4}$ &  \\
$0$ & $3q^{2}$ & $q^{-2}$ + $6$ & $q^{-4}$ + $3q^{-2}$ &  \\
$-2$ & $2q^{4}$ & $1$ + $5q^{2}$ & $q^{-2}$ + $3$ &  \\
$-4$ & $q^{6}$ & $q^{2}$ + $4q^{4}$ & $2$ + $3q^{2}$ & $q^{-2}$ \\
$-6$ &  & $2q^{6}$ & $q^{2}$ + $2q^{4}$ & $1$ \\
$-8$ &  & $q^{8}$ & $q^{4}$ + $q^{6}$ & $q^{2}$ \\
\end{tabular}
\vspace{2em}
\end{minipage}
%
\begin{minipage}{\linewidth}
$\bullet\ $ $11n_{121}$ \vspace{0.5em} \\
\begin{tabular}{l|llll}
$k \setminus j$ & $-8$ & $-6$ & $-4$ & $-2$ \\
\hline
$6$ & $q^{-2}$ & $2q^{-4}$ & $q^{-6}$ &  \\
$4$ &  & $3q^{-2}$ & $3q^{-4}$ &  \\
$2$ & $q^{2}$ & $4$ & $4q^{-2}$ & $q^{-4}$ \\
$0$ &  & $3q^{2}$ & $5$ & $2q^{-2}$ \\
$-2$ &  & $2q^{4}$ & $4q^{2}$ & $2$ \\
$-4$ &  &  & $3q^{4}$ & $2q^{2}$ \\
$-6$ &  &  & $q^{6}$ & $q^{4}$ \\
\end{tabular}
\vspace{2em}
\end{minipage}
%
\begin{minipage}{\linewidth}
$\bullet\ $ $11n_{122}$ \vspace{0.5em} \\
\begin{tabular}{l|llll}
$k \setminus j$ & $-8$ & $-6$ & $-4$ & $-2$ \\
\hline
$8$ & $q^{-2}$ & $q^{-4}$ &  &  \\
$6$ & $1$ & $2q^{-2}$ & $q^{-4}$ &  \\
$4$ & $q^{2}$ & $3$ & $2q^{-2}$ &  \\
$2$ &  & $2q^{2}$ & $4$ & $2q^{-2}$ \\
$0$ &  & $q^{4}$ & $2q^{2}$ & $1$ \\
$-2$ &  &  & $q^{4}$ & $2q^{2}$ \\
\end{tabular}
\vspace{2em}
\end{minipage}
%
\begin{minipage}{\linewidth}
$\bullet\ $ $11n_{123}$ \vspace{0.5em} \\
\begin{tabular}{l|lllll}
$k \setminus j$ & $-6$ & $-4$ & $-2$ & $0$ & $2$ \\
\hline
$6$ & $1$ & $3q^{-2}$ & $2q^{-4}$ &  &  \\
$4$ &  & $4$ & $5q^{-2}$ & $q^{-4}$ &  \\
$2$ &  & $3q^{2}$ & $8$ & $5q^{-2}$ &  \\
$0$ &  &  & $5q^{2}$ & $7$ & $q^{-2}$ \\
$-2$ &  &  & $2q^{4}$ & $5q^{2}$ & $3$ \\
$-4$ &  &  &  & $q^{4}$ & $q^{2}$ \\
\end{tabular}
\vspace{2em}
\end{minipage}
%
\begin{minipage}{\linewidth}
$\bullet\ $ $11n_{124}$ \vspace{0.5em} \\
\begin{tabular}{l|llll}
$k \setminus j$ & $-2$ & $0$ & $2$ & $4$ \\
\hline
$6$ & $q^{-2}$ & $2q^{-4}$ & $q^{-6}$ &  \\
$4$ & $1$ & $4q^{-2}$ & $3q^{-4}$ &  \\
$2$ & $q^{2}$ & $5$ & $6q^{-2}$ & $q^{-4}$ \\
$0$ &  & $4q^{2}$ & $7$ & $3q^{-2}$ \\
$-2$ &  & $2q^{4}$ & $6q^{2}$ & $4$ \\
$-4$ &  &  & $3q^{4}$ & $3q^{2}$ \\
$-6$ &  &  & $q^{6}$ & $q^{4}$ \\
\end{tabular}
\vspace{2em}
\end{minipage}
%
\begin{minipage}{\linewidth}
$\bullet\ $ $11n_{125}$ \vspace{0.5em} \\
\begin{tabular}{l|llll}
$k \setminus j$ & $0$ & $2$ & $4$ & $6$ \\
\hline
$4$ & $q^{-2}$ & $2q^{-4}$ & $q^{-6}$ &  \\
$2$ & $1$ & $5q^{-2}$ & $3q^{-4}$ &  \\
$0$ & $q^{2}$ & $6$ & $6q^{-2}$ & $q^{-4}$ \\
$-2$ &  & $5q^{2}$ & $8$ & $3q^{-2}$ \\
$-4$ &  & $2q^{4}$ & $6q^{2}$ & $4$ \\
$-6$ &  &  & $3q^{4}$ & $3q^{2}$ \\
$-8$ &  &  & $q^{6}$ & $q^{4}$ \\
\end{tabular}
\vspace{2em}
\end{minipage}
%
\begin{minipage}{\linewidth}
$\bullet\ $ $11n_{126}$ \vspace{0.5em} \\
\begin{tabular}{l|llll}
$k \setminus j$ & $6$ & $8$ & $10$ & $12$ \\
\hline
$6$ & $q^{-6}$ &  &  &  \\
$4$ & $2q^{-4}$ & $2q^{-6}$ &  &  \\
$2$ & $2q^{-2}$ & $2q^{-4}$ &  &  \\
$0$ & $2$ & $4q^{-2}$ & $2q^{-4}$ &  \\
$-2$ & $1$ + $2q^{2}$ & $q^{-2}$ + $3$ & $q^{-2}$ &  \\
$-4$ & $2q^{4}$ & $2$ + $4q^{2}$ & $2q^{-2}$ + $2$ &  \\
$-6$ & $q^{6}$ & $q^{2}$ + $2q^{4}$ & $1$ + $q^{2}$ &  \\
$-8$ &  & $2q^{6}$ & $2q^{2}$ + $2q^{4}$ & $2$ \\
\end{tabular}
\vspace{2em}
\end{minipage}
%
\begin{minipage}{\linewidth}
$\bullet\ $ $11n_{127}$ \vspace{0.5em} \\
\begin{tabular}{l|llll}
$k \setminus j$ & $-8$ & $-6$ & $-4$ & $-2$ \\
\hline
$8$ & $q^{-2}$ & $2q^{-4}$ & $q^{-6}$ &  \\
$6$ & $1$ & $3q^{-2}$ & $2q^{-4}$ &  \\
$4$ & $q^{2}$ & $6$ & $6q^{-2}$ & $q^{-4}$ \\
$2$ &  & $3q^{2}$ & $6$ & $3q^{-2}$ \\
$0$ &  & $2q^{4}$ & $6q^{2}$ & $4$ \\
$-2$ &  &  & $2q^{4}$ & $3q^{2}$ \\
$-4$ &  &  & $q^{6}$ & $q^{4}$ \\
\end{tabular}
\vspace{2em}
\end{minipage}
%
\begin{minipage}{\linewidth}
$\bullet\ $ $11n_{128}$ \vspace{0.5em} \\
\begin{tabular}{l|llll}
$k \setminus j$ & $-2$ & $0$ & $2$ & $4$ \\
\hline
$8$ & $q^{-4}$ & $q^{-6}$ &  &  \\
$6$ & $2q^{-2}$ & $2q^{-4}$ &  &  \\
$4$ & $2$ & $4q^{-2}$ & $2q^{-4}$ &  \\
$2$ & $2q^{2}$ & $4$ & $3q^{-2}$ &  \\
$0$ & $q^{4}$ & $4q^{2}$ & $4$ & $q^{-2}$ \\
$-2$ &  & $2q^{4}$ & $3q^{2}$ & $1$ \\
$-4$ &  & $q^{6}$ & $2q^{4}$ & $q^{2}$ \\
\end{tabular}
\vspace{2em}
\end{minipage}
%
\begin{minipage}{\linewidth}
$\bullet\ $ $11n_{129}$ \vspace{0.5em} \\
\begin{tabular}{l|llll}
$k \setminus j$ & $-6$ & $-4$ & $-2$ & $0$ \\
\hline
$8$ & $q^{-4}$ & $q^{-6}$ &  &  \\
$6$ & $q^{-2}$ & $q^{-4}$ &  &  \\
$4$ & $3$ & $5q^{-2}$ & $2q^{-4}$ &  \\
$2$ & $q^{2}$ & $4$ & $3q^{-2}$ &  \\
$0$ & $q^{4}$ & $5q^{2}$ & $5$ & $q^{-2}$ \\
$-2$ &  & $q^{4}$ & $3q^{2}$ & $1$ \\
$-4$ &  & $q^{6}$ & $2q^{4}$ & $q^{2}$ \\
\end{tabular}
\vspace{2em}
\end{minipage}
%
\begin{minipage}{\linewidth}
$\bullet\ $ $11n_{130}$ \vspace{0.5em} \\
\begin{tabular}{l|llll}
$k \setminus j$ & $-4$ & $-2$ & $0$ & $2$ \\
\hline
$8$ & $q^{-4}$ & $q^{-6}$ &  &  \\
$6$ & $2q^{-2}$ & $2q^{-4}$ &  &  \\
$4$ & $3$ & $5q^{-2}$ & $2q^{-4}$ &  \\
$2$ & $2q^{2}$ & $6$ & $4q^{-2}$ &  \\
$0$ & $q^{4}$ & $5q^{2}$ & $6$ & $q^{-2}$ \\
$-2$ &  & $2q^{4}$ & $4q^{2}$ & $2$ \\
$-4$ &  & $q^{6}$ & $2q^{4}$ & $q^{2}$ \\
\end{tabular}
\vspace{2em}
\end{minipage}
%
\begin{minipage}{\linewidth}
$\bullet\ $ $11n_{131}$ \vspace{0.5em} \\
\begin{tabular}{l|llll}
$k \setminus j$ & $-6$ & $-4$ & $-2$ & $0$ \\
\hline
$6$ & $q^{-2}$ & $2q^{-4}$ & $q^{-6}$ &  \\
$4$ & $2$ & $5q^{-2}$ & $3q^{-4}$ &  \\
$2$ & $q^{2}$ & $7$ & $7q^{-2}$ & $q^{-4}$ \\
$0$ &  & $5q^{2}$ & $8$ & $3q^{-2}$ \\
$-2$ &  & $2q^{4}$ & $7q^{2}$ & $4$ \\
$-4$ &  &  & $3q^{4}$ & $3q^{2}$ \\
$-6$ &  &  & $q^{6}$ & $q^{4}$ \\
\end{tabular}
\vspace{2em}
\end{minipage}
%
\begin{minipage}{\linewidth}
$\bullet\ $ $11n_{132}$ \vspace{0.5em} \\
\begin{tabular}{l|llll}
$k \setminus j$ & $-6$ & $-4$ & $-2$ & $0$ \\
\hline
$8$ & $q^{-2}$ & $q^{-4}$ &  &  \\
$6$ & $1$ & $2q^{-2}$ & $q^{-4}$ &  \\
$4$ & $q^{2}$ & $3$ & $2q^{-2}$ &  \\
$2$ &  & $2q^{2}$ & $3$ & $q^{-2}$ \\
$0$ &  & $q^{4}$ & $2q^{2}$ & $2$ \\
$-2$ &  &  & $q^{4}$ & $q^{2}$ \\
\end{tabular}
\vspace{2em}
\end{minipage}
%
\begin{minipage}{\linewidth}
$\bullet\ $ $11n_{133}$ \vspace{0.5em} \\
\begin{tabular}{l|llll}
$k \setminus j$ & $2$ & $4$ & $6$ & $8$ \\
\hline
$8$ & $q^{-6}$ & $q^{-8}$ &  &  \\
$6$ & $2q^{-4}$ & $2q^{-6}$ &  &  \\
$4$ & $q^{-2}$ & $3q^{-4}$ & $q^{-6}$ &  \\
$2$ & $2$ & $4q^{-2}$ & $2q^{-4}$ &  \\
$0$ & $1$ + $q^{2}$ & $2q^{-2}$ + $3$ & $q^{-4}$ + $2q^{-2}$ &  \\
$-2$ & $2q^{4}$ & $2$ + $4q^{2}$ & $2q^{-2}$ + $2$ &  \\
$-4$ & $q^{6}$ & $2q^{2}$ + $3q^{4}$ & $3$ + $2q^{2}$ & $q^{-2}$ \\
$-6$ &  & $2q^{6}$ & $2q^{2}$ + $2q^{4}$ & $2$ \\
$-8$ &  & $q^{8}$ & $q^{4}$ + $q^{6}$ & $q^{2}$ \\
\end{tabular}
\vspace{2em}
\end{minipage}
%
\begin{minipage}{\linewidth}
$\bullet\ $ $11n_{134}$ \vspace{0.5em} \\
\begin{tabular}{l|llll}
$k \setminus j$ & $-8$ & $-6$ & $-4$ & $-2$ \\
\hline
$8$ & $q^{-2}$ & $q^{-4}$ &  &  \\
$6$ & $2$ & $4q^{-2}$ & $2q^{-4}$ &  \\
$4$ & $q^{2}$ & $5$ & $4q^{-2}$ &  \\
$2$ &  & $4q^{2}$ & $7$ & $3q^{-2}$ \\
$0$ &  & $q^{4}$ & $4q^{2}$ & $3$ \\
$-2$ &  &  & $2q^{4}$ & $3q^{2}$ \\
\end{tabular}
\vspace{2em}
\end{minipage}
%
\begin{minipage}{\linewidth}
$\bullet\ $ $11n_{135}$ \vspace{0.5em} \\
\begin{tabular}{l|llll}
$k \setminus j$ & $2$ & $4$ & $6$ & $8$ \\
\hline
$6$ & $q^{-4}$ & $q^{-6}$ &  &  \\
$4$ &  & $q^{-4}$ &  &  \\
$2$ & $1$ & $2q^{-2}$ & $q^{-4}$ &  \\
$0$ &  & $q^{-2}$ + $1$ & $q^{-4}$ + $q^{-2}$ &  \\
$-2$ & $q^{4}$ & $1$ + $2q^{2}$ & $q^{-2}$ + $1$ &  \\
$-4$ &  & $q^{2}$ + $q^{4}$ & $2$ + $q^{2}$ & $q^{-2}$ \\
$-6$ &  & $q^{6}$ & $q^{2}$ + $q^{4}$ & $1$ \\
$-8$ &  &  & $q^{4}$ & $q^{2}$ \\
\end{tabular}
\vspace{2em}
\end{minipage}
%
\begin{minipage}{\linewidth}
$\bullet\ $ $11n_{136}$ \vspace{0.5em} \\
\begin{tabular}{l|lll}
$k \setminus j$ & $6$ & $8$ & $10$ \\
\hline
$6$ & $q^{-6}$ &  &  \\
$4$ & $2q^{-4}$ & $2q^{-6}$ &  \\
$2$ & $4q^{-2}$ & $4q^{-4}$ &  \\
$0$ & $4$ & $6q^{-2}$ & $2q^{-4}$ \\
$-2$ & $4q^{2}$ & $7$ & $3q^{-2}$ \\
$-4$ & $2q^{4}$ & $6q^{2}$ & $4$ \\
$-6$ & $q^{6}$ & $4q^{4}$ & $3q^{2}$ \\
$-8$ &  & $2q^{6}$ & $2q^{4}$ \\
\end{tabular}
\vspace{2em}
\end{minipage}
%
\begin{minipage}{\linewidth}
$\bullet\ $ $11n_{137}$ \vspace{0.5em} \\
\begin{tabular}{l|llll}
$k \setminus j$ & $-8$ & $-6$ & $-4$ & $-2$ \\
\hline
$6$ & $2q^{-2}$ & $3q^{-4}$ & $q^{-6}$ &  \\
$4$ & $1$ & $4q^{-2}$ & $3q^{-4}$ &  \\
$2$ & $2q^{2}$ & $6$ & $5q^{-2}$ & $q^{-4}$ \\
$0$ &  & $4q^{2}$ & $6$ & $2q^{-2}$ \\
$-2$ &  & $3q^{4}$ & $5q^{2}$ & $2$ \\
$-4$ &  &  & $3q^{4}$ & $2q^{2}$ \\
$-6$ &  &  & $q^{6}$ & $q^{4}$ \\
\end{tabular}
\vspace{2em}
\end{minipage}
%
\begin{minipage}{\linewidth}
$\bullet\ $ $11n_{138}$ \vspace{0.5em} \\
\begin{tabular}{l|llll}
$k \setminus j$ & $-4$ & $-2$ & $0$ & $2$ \\
\hline
$8$ & $q^{-2}$ & $q^{-4}$ &  &  \\
$6$ &  & $q^{-2}$ & $q^{-4}$ &  \\
$4$ & $q^{2}$ & $2$ & $q^{-2}$ &  \\
$2$ &  & $q^{2}$ & $2$ & $q^{-2}$ \\
$0$ &  & $q^{4}$ & $1$ + $q^{2}$ &  \\
$-2$ &  &  & $q^{4}$ & $q^{2}$ \\
\end{tabular}
\vspace{2em}
\end{minipage}
%
\begin{minipage}{\linewidth}
$\bullet\ $ $11n_{139}$ \vspace{0.5em} \\
\begin{tabular}{l|lllll}
$k \setminus j$ & $0$ & $2$ & $4$ & $6$ & $8$ \\
\hline
$0$ & $1$ &  &  &  &  \\
$-2$ &  & $1$ & $q^{-2}$ &  &  \\
$-4$ &  &  & $1$ & $q^{-2}$ &  \\
$-6$ &  &  & $q^{2}$ & $1$ &  \\
$-8$ &  &  &  & $q^{2}$ & $1$ \\
\end{tabular}
\vspace{2em}
\end{minipage}
%
\begin{minipage}{\linewidth}
$\bullet\ $ $11n_{142}$ \vspace{0.5em} \\
\begin{tabular}{l|lllll}
$k \setminus j$ & $-6$ & $-4$ & $-2$ & $0$ & $2$ \\
\hline
$6$ & $1$ & $2q^{-2}$ & $q^{-4}$ &  &  \\
$4$ &  & $3$ & $3q^{-2}$ &  &  \\
$2$ &  & $2q^{2}$ & $5$ & $3q^{-2}$ &  \\
$0$ &  &  & $3q^{2}$ & $4$ &  \\
$-2$ &  &  & $q^{4}$ & $3q^{2}$ & $2$ \\
\end{tabular}
\vspace{2em}
\end{minipage}
%
\begin{minipage}{\linewidth}
$\bullet\ $ $11n_{143}$ \vspace{0.5em} \\
\begin{tabular}{l|llll}
$k \setminus j$ & $-2$ & $0$ & $2$ & $4$ \\
\hline
$6$ & $q^{-2}$ & $q^{-4}$ &  &  \\
$4$ & $1$ & $q^{-4}$ + $q^{-2}$ & $q^{-6}$ &  \\
$2$ & $q^{2}$ & $q^{-2}$ + $2$ & $q^{-4}$ + $q^{-2}$ &  \\
$0$ &  & $2$ + $q^{2}$ & $2q^{-2}$ + $1$ & $q^{-4}$ \\
$-2$ &  & $q^{2}$ + $q^{4}$ & $2$ + $q^{2}$ & $q^{-2}$ \\
$-4$ &  & $q^{4}$ & $2q^{2}$ & $1$ \\
$-6$ &  &  & $q^{4}$ & $q^{2}$ \\
$-8$ &  &  & $q^{6}$ & $q^{4}$ \\
\end{tabular}
\vspace{2em}
\end{minipage}
%
\begin{minipage}{\linewidth}
$\bullet\ $ $11n_{144}$ \vspace{0.5em} \\
\begin{tabular}{l|llll}
$k \setminus j$ & $4$ & $6$ & $8$ & $10$ \\
\hline
$4$ & $2q^{-4}$ & $q^{-6}$ &  &  \\
$2$ & $3q^{-2}$ & $3q^{-4}$ &  &  \\
$0$ & $5$ & $7q^{-2}$ & $2q^{-4}$ &  \\
$-2$ & $3q^{2}$ & $7$ & $4q^{-2}$ &  \\
$-4$ & $2q^{4}$ & $7q^{2}$ & $6$ & $q^{-2}$ \\
$-6$ &  & $3q^{4}$ & $4q^{2}$ & $1$ \\
$-8$ &  & $q^{6}$ & $2q^{4}$ & $q^{2}$ \\
\end{tabular}
\vspace{2em}
\end{minipage}
%
\begin{minipage}{\linewidth}
$\bullet\ $ $11n_{145}$ \vspace{0.5em} \\
\begin{tabular}{l|llll}
$k \setminus j$ & $-2$ & $0$ & $2$ & $4$ \\
\hline
$8$ & $q^{-4}$ & $q^{-6}$ &  &  \\
$4$ & $1$ & $2q^{-2}$ & $q^{-4}$ &  \\
$2$ & $1$ & $2q^{-2}$ & $q^{-4}$ &  \\
$0$ & $q^{4}$ & $3$ + $2q^{2}$ & $2q^{-2}$ + $1$ &  \\
$-2$ &  & $2q^{2}$ & $3$ & $q^{-2}$ \\
$-4$ &  & $q^{6}$ & $2q^{2}$ + $q^{4}$ & $2$ \\
$-6$ &  &  & $q^{4}$ & $q^{2}$ \\
\end{tabular}
\vspace{2em}
\end{minipage}
%
\begin{minipage}{\linewidth}
$\bullet\ $ $11n_{146}$ \vspace{0.5em} \\
\begin{tabular}{l|llll}
$k \setminus j$ & $2$ & $4$ & $6$ & $8$ \\
\hline
$4$ & $q^{-4}$ & $q^{-6}$ &  &  \\
$2$ & $3q^{-2}$ & $2q^{-4}$ &  &  \\
$0$ & $5$ & $7q^{-2}$ & $2q^{-4}$ &  \\
$-2$ & $3q^{2}$ & $7$ & $4q^{-2}$ &  \\
$-4$ & $q^{4}$ & $7q^{2}$ & $7$ & $q^{-2}$ \\
$-6$ &  & $2q^{4}$ & $4q^{2}$ & $2$ \\
$-8$ &  & $q^{6}$ & $2q^{4}$ & $q^{2}$ \\
\end{tabular}
\vspace{2em}
\end{minipage}
%
\begin{minipage}{\linewidth}
$\bullet\ $ $11n_{147}$ \vspace{0.5em} \\
\begin{tabular}{l|llll}
$k \setminus j$ & $2$ & $4$ & $6$ & $8$ \\
\hline
$8$ & $q^{-6}$ & $q^{-8}$ &  &  \\
$6$ & $2q^{-4}$ & $2q^{-6}$ &  &  \\
$4$ & $2q^{-2}$ & $4q^{-4}$ & $q^{-6}$ &  \\
$2$ & $2$ & $4q^{-2}$ & $2q^{-4}$ &  \\
$0$ & $2q^{2}$ & $q^{-2}$ + $5$ & $q^{-4}$ + $3q^{-2}$ &  \\
$-2$ & $2q^{4}$ & $2$ + $4q^{2}$ & $2q^{-2}$ + $2$ &  \\
$-4$ & $q^{6}$ & $q^{2}$ + $4q^{4}$ & $2$ + $3q^{2}$ & $q^{-2}$ \\
$-6$ &  & $2q^{6}$ & $2q^{2}$ + $2q^{4}$ & $2$ \\
$-8$ &  & $q^{8}$ & $q^{4}$ + $q^{6}$ & $q^{2}$ \\
\end{tabular}
\vspace{2em}
\end{minipage}
%
\begin{minipage}{\linewidth}
$\bullet\ $ $11n_{148}$ \vspace{0.5em} \\
\begin{tabular}{l|llll}
$k \setminus j$ & $0$ & $2$ & $4$ & $6$ \\
\hline
$8$ & $q^{-6}$ & $q^{-8}$ &  &  \\
$6$ & $3q^{-4}$ & $3q^{-6}$ &  &  \\
$4$ & $4q^{-2}$ & $5q^{-4}$ & $q^{-6}$ &  \\
$2$ & $5$ & $9q^{-2}$ & $3q^{-4}$ &  \\
$0$ & $4q^{2}$ & $q^{-2}$ + $8$ & $q^{-4}$ + $4q^{-2}$ &  \\
$-2$ & $3q^{4}$ & $1$ + $9q^{2}$ & $q^{-2}$ + $6$ &  \\
$-4$ & $q^{6}$ & $q^{2}$ + $5q^{4}$ & $2$ + $4q^{2}$ & $q^{-2}$ \\
$-6$ &  & $3q^{6}$ & $q^{2}$ + $3q^{4}$ & $1$ \\
$-8$ &  & $q^{8}$ & $q^{4}$ + $q^{6}$ & $q^{2}$ \\
\end{tabular}
\vspace{2em}
\end{minipage}
%
\begin{minipage}{\linewidth}
$\bullet\ $ $11n_{149}$ \vspace{0.5em} \\
\begin{tabular}{l|llll}
$k \setminus j$ & $2$ & $4$ & $6$ & $8$ \\
\hline
$8$ & $q^{-6}$ & $q^{-8}$ &  &  \\
$6$ & $2q^{-4}$ & $2q^{-6}$ &  &  \\
$4$ & $q^{-2}$ & $3q^{-4}$ & $q^{-6}$ &  \\
$2$ & $2$ & $4q^{-2}$ & $2q^{-4}$ &  \\
$0$ & $q^{2}$ & $q^{-2}$ + $3$ & $q^{-4}$ + $2q^{-2}$ &  \\
$-2$ & $2q^{4}$ & $1$ + $4q^{2}$ & $q^{-2}$ + $2$ &  \\
$-4$ & $q^{6}$ & $q^{2}$ + $3q^{4}$ & $2$ + $2q^{2}$ & $q^{-2}$ \\
$-6$ &  & $2q^{6}$ & $q^{2}$ + $2q^{4}$ & $1$ \\
$-8$ &  & $q^{8}$ & $q^{4}$ + $q^{6}$ & $q^{2}$ \\
\end{tabular}
\vspace{2em}
\end{minipage}
%
\begin{minipage}{\linewidth}
$\bullet\ $ $11n_{150}$ \vspace{0.5em} \\
\begin{tabular}{l|lllll}
$k \setminus j$ & $0$ & $2$ & $4$ & $6$ & $8$ \\
\hline
$6$ & $q^{-4}$ & $q^{-6}$ &  &  &  \\
$4$ & $3q^{-2}$ & $4q^{-4}$ & $q^{-6}$ &  &  \\
$2$ & $3$ & $7q^{-2}$ & $3q^{-4}$ &  &  \\
$0$ & $3q^{2}$ & $8$ & $6q^{-2}$ & $q^{-4}$ &  \\
$-2$ & $q^{4}$ & $7q^{2}$ & $8$ & $2q^{-2}$ &  \\
$-4$ &  & $4q^{4}$ & $1$ + $6q^{2}$ & $q^{-2}$ + $2$ &  \\
$-6$ &  & $q^{6}$ & $3q^{4}$ & $2q^{2}$ &  \\
$-8$ &  &  & $q^{6}$ & $q^{2}$ + $q^{4}$ & $1$ \\
\end{tabular}
\vspace{2em}
\end{minipage}
%
\begin{minipage}{\linewidth}
$\bullet\ $ $11n_{151}$ \vspace{0.5em} \\
\begin{tabular}{l|llll}
$k \setminus j$ & $0$ & $2$ & $4$ & $6$ \\
\hline
$8$ & $q^{-4}$ & $q^{-6}$ &  &  \\
$4$ & $2$ & $q^{-4}$ + $3q^{-2}$ & $q^{-6}$ + $q^{-4}$ &  \\
$2$ &  & $3q^{-2}$ & $2q^{-4}$ &  \\
$0$ & $q^{4}$ & $3$ + $3q^{2}$ & $4q^{-2}$ + $2$ & $q^{-4}$ \\
$-2$ &  & $3q^{2}$ & $5$ & $2q^{-2}$ \\
$-4$ &  & $q^{4}$ + $q^{6}$ & $4q^{2}$ + $q^{4}$ & $3$ \\
$-6$ &  &  & $2q^{4}$ & $2q^{2}$ \\
$-8$ &  &  & $q^{6}$ & $q^{4}$ \\
\end{tabular}
\vspace{2em}
\end{minipage}
%
\begin{minipage}{\linewidth}
$\bullet\ $ $11n_{152}$ \vspace{0.5em} \\
\begin{tabular}{l|llll}
$k \setminus j$ & $0$ & $2$ & $4$ & $6$ \\
\hline
$8$ & $q^{-4}$ & $q^{-6}$ &  &  \\
$4$ & $2$ & $q^{-4}$ + $3q^{-2}$ & $q^{-6}$ + $q^{-4}$ &  \\
$2$ &  & $3q^{-2}$ & $2q^{-4}$ &  \\
$0$ & $q^{4}$ & $3$ + $3q^{2}$ & $4q^{-2}$ + $2$ & $q^{-4}$ \\
$-2$ &  & $3q^{2}$ & $5$ & $2q^{-2}$ \\
$-4$ &  & $q^{4}$ + $q^{6}$ & $4q^{2}$ + $q^{4}$ & $3$ \\
$-6$ &  &  & $2q^{4}$ & $2q^{2}$ \\
$-8$ &  &  & $q^{6}$ & $q^{4}$ \\
\end{tabular}
\vspace{2em}
\end{minipage}
%
\begin{minipage}{\linewidth}
$\bullet\ $ $11n_{153}$ \vspace{0.5em} \\
\begin{tabular}{l|llll}
$k \setminus j$ & $-2$ & $0$ & $2$ & $4$ \\
\hline
$8$ & $q^{-6}$ & $q^{-8}$ &  &  \\
$6$ & $2q^{-4}$ & $2q^{-6}$ &  &  \\
$4$ & $3q^{-2}$ & $4q^{-4}$ & $q^{-6}$ &  \\
$2$ & $4$ & $6q^{-2}$ & $2q^{-4}$ &  \\
$0$ & $3q^{2}$ & $q^{-2}$ + $7$ & $q^{-4}$ + $3q^{-2}$ &  \\
$-2$ & $2q^{4}$ & $6q^{2}$ & $4$ &  \\
$-4$ & $q^{6}$ & $q^{2}$ + $4q^{4}$ & $2$ + $3q^{2}$ & $q^{-2}$ \\
$-6$ &  & $2q^{6}$ & $2q^{4}$ &  \\
$-8$ &  & $q^{8}$ & $q^{4}$ + $q^{6}$ & $q^{2}$ \\
\end{tabular}
\vspace{2em}
\end{minipage}
%
\begin{minipage}{\linewidth}
$\bullet\ $ $11n_{154}$ \vspace{0.5em} \\
\begin{tabular}{l|llll}
$k \setminus j$ & $0$ & $2$ & $4$ & $6$ \\
\hline
$4$ & $2q^{-2}$ & $3q^{-4}$ & $q^{-6}$ &  \\
$2$ & $2$ & $6q^{-2}$ & $3q^{-4}$ &  \\
$0$ & $2q^{2}$ & $9$ & $8q^{-2}$ & $q^{-4}$ \\
$-2$ &  & $6q^{2}$ & $9$ & $3q^{-2}$ \\
$-4$ &  & $3q^{4}$ & $8q^{2}$ & $5$ \\
$-6$ &  &  & $3q^{4}$ & $3q^{2}$ \\
$-8$ &  &  & $q^{6}$ & $q^{4}$ \\
\end{tabular}
\vspace{2em}
\end{minipage}
%
\begin{minipage}{\linewidth}
$\bullet\ $ $11n_{155}$ \vspace{0.5em} \\
\begin{tabular}{l|lllll}
$k \setminus j$ & $-2$ & $0$ & $2$ & $4$ & $6$ \\
\hline
$8$ & $q^{-4}$ & $q^{-6}$ &  &  &  \\
$6$ & $2q^{-2}$ & $3q^{-4}$ & $q^{-6}$ &  &  \\
$4$ & $1$ & $4q^{-2}$ & $3q^{-4}$ &  &  \\
$2$ & $2q^{2}$ & $5$ & $5q^{-2}$ & $q^{-4}$ &  \\
$0$ & $q^{4}$ & $1$ + $4q^{2}$ & $q^{-2}$ + $5$ & $2q^{-2}$ &  \\
$-2$ &  & $3q^{4}$ & $1$ + $5q^{2}$ & $q^{-2}$ + $2$ &  \\
$-4$ &  & $q^{6}$ & $q^{2}$ + $3q^{4}$ & $1$ + $2q^{2}$ &  \\
$-6$ &  &  & $q^{6}$ & $q^{2}$ + $q^{4}$ & $1$ \\
\end{tabular}
\vspace{2em}
\end{minipage}
%
\begin{minipage}{\linewidth}
$\bullet\ $ $11n_{156}$ \vspace{0.5em} \\
\begin{tabular}{l|llll}
$k \setminus j$ & $-2$ & $0$ & $2$ & $4$ \\
\hline
$4$ & $2q^{-2}$ & $3q^{-4}$ & $q^{-6}$ &  \\
$2$ & $3$ & $6q^{-2}$ & $3q^{-4}$ &  \\
$0$ & $2q^{2}$ & $9$ & $7q^{-2}$ & $q^{-4}$ \\
$-2$ &  & $6q^{2}$ & $9$ & $3q^{-2}$ \\
$-4$ &  & $3q^{4}$ & $7q^{2}$ & $4$ \\
$-6$ &  &  & $3q^{4}$ & $3q^{2}$ \\
$-8$ &  &  & $q^{6}$ & $q^{4}$ \\
\end{tabular}
\vspace{2em}
\end{minipage}
%
\begin{minipage}{\linewidth}
$\bullet\ $ $11n_{157}$ \vspace{0.5em} \\
\begin{tabular}{l|llll}
$k \setminus j$ & $-4$ & $-2$ & $0$ & $2$ \\
\hline
$8$ & $q^{-4}$ & $q^{-6}$ &  &  \\
$6$ & $3q^{-2}$ & $3q^{-4}$ &  &  \\
$4$ & $4$ & $6q^{-2}$ & $2q^{-4}$ &  \\
$2$ & $3q^{2}$ & $8$ & $5q^{-2}$ &  \\
$0$ & $q^{4}$ & $6q^{2}$ & $7$ & $q^{-2}$ \\
$-2$ &  & $3q^{4}$ & $5q^{2}$ & $2$ \\
$-4$ &  & $q^{6}$ & $2q^{4}$ & $q^{2}$ \\
\end{tabular}
\vspace{2em}
\end{minipage}
%
\begin{minipage}{\linewidth}
$\bullet\ $ $11n_{158}$ \vspace{0.5em} \\
\begin{tabular}{l|llll}
$k \setminus j$ & $2$ & $4$ & $6$ & $8$ \\
\hline
$8$ & $q^{-6}$ & $q^{-8}$ &  &  \\
$6$ & $2q^{-4}$ & $2q^{-6}$ &  &  \\
$4$ & $2q^{-2}$ & $4q^{-4}$ & $q^{-6}$ &  \\
$2$ & $3$ & $5q^{-2}$ & $2q^{-4}$ &  \\
$0$ & $2q^{2}$ & $q^{-2}$ + $5$ & $q^{-4}$ + $3q^{-2}$ &  \\
$-2$ & $2q^{4}$ & $1$ + $5q^{2}$ & $q^{-2}$ + $3$ &  \\
$-4$ & $q^{6}$ & $q^{2}$ + $4q^{4}$ & $2$ + $3q^{2}$ & $q^{-2}$ \\
$-6$ &  & $2q^{6}$ & $q^{2}$ + $2q^{4}$ & $1$ \\
$-8$ &  & $q^{8}$ & $q^{4}$ + $q^{6}$ & $q^{2}$ \\
\end{tabular}
\vspace{2em}
\end{minipage}
%
\begin{minipage}{\linewidth}
$\bullet\ $ $11n_{159}$ \vspace{0.5em} \\
\begin{tabular}{l|llll}
$k \setminus j$ & $-8$ & $-6$ & $-4$ & $-2$ \\
\hline
$8$ & $q^{-2}$ & $2q^{-4}$ & $q^{-6}$ &  \\
$6$ & $2$ & $5q^{-2}$ & $3q^{-4}$ &  \\
$4$ & $q^{2}$ & $7$ & $7q^{-2}$ & $q^{-4}$ \\
$2$ &  & $5q^{2}$ & $9$ & $4q^{-2}$ \\
$0$ &  & $2q^{4}$ & $7q^{2}$ & $5$ \\
$-2$ &  &  & $3q^{4}$ & $4q^{2}$ \\
$-4$ &  &  & $q^{6}$ & $q^{4}$ \\
\end{tabular}
\vspace{2em}
\end{minipage}
%
\begin{minipage}{\linewidth}
$\bullet\ $ $11n_{160}$ \vspace{0.5em} \\
\begin{tabular}{l|llll}
$k \setminus j$ & $0$ & $2$ & $4$ & $6$ \\
\hline
$4$ & $2q^{-2}$ & $3q^{-4}$ & $q^{-6}$ &  \\
$2$ & $2$ & $5q^{-2}$ & $2q^{-4}$ &  \\
$0$ & $2q^{2}$ & $8$ & $7q^{-2}$ & $q^{-4}$ \\
$-2$ &  & $5q^{2}$ & $7$ & $2q^{-2}$ \\
$-4$ &  & $3q^{4}$ & $7q^{2}$ & $4$ \\
$-6$ &  &  & $2q^{4}$ & $2q^{2}$ \\
$-8$ &  &  & $q^{6}$ & $q^{4}$ \\
\end{tabular}
\vspace{2em}
\end{minipage}
%
\begin{minipage}{\linewidth}
$\bullet\ $ $11n_{163}$ \vspace{0.5em} \\
\begin{tabular}{l|llll}
$k \setminus j$ & $0$ & $2$ & $4$ & $6$ \\
\hline
$6$ & $q^{-4}$ & $q^{-6}$ &  &  \\
$4$ & $4q^{-2}$ & $4q^{-4}$ &  &  \\
$2$ & $5$ & $9q^{-2}$ & $3q^{-4}$ &  \\
$0$ & $4q^{2}$ & $11$ & $7q^{-2}$ &  \\
$-2$ & $q^{4}$ & $9q^{2}$ & $10$ & $2q^{-2}$ \\
$-4$ &  & $4q^{4}$ & $7q^{2}$ & $3$ \\
$-6$ &  & $q^{6}$ & $3q^{4}$ & $2q^{2}$ \\
\end{tabular}
\vspace{2em}
\end{minipage}
%
\begin{minipage}{\linewidth}
$\bullet\ $ $11n_{164}$ \vspace{0.5em} \\
\begin{tabular}{l|llll}
$k \setminus j$ & $2$ & $4$ & $6$ & $8$ \\
\hline
$6$ & $q^{-4}$ & $q^{-6}$ &  &  \\
$4$ & $2q^{-2}$ & $3q^{-4}$ &  &  \\
$2$ & $3$ & $4q^{-2}$ & $q^{-4}$ &  \\
$0$ & $2q^{2}$ & $6$ & $4q^{-2}$ &  \\
$-2$ & $q^{4}$ & $4q^{2}$ & $3$ &  \\
$-4$ &  & $3q^{4}$ & $4q^{2}$ & $1$ \\
$-6$ &  & $q^{6}$ & $q^{4}$ &  \\
\end{tabular}
\vspace{2em}
\end{minipage}
%
\begin{minipage}{\linewidth}
$\bullet\ $ $11n_{165}$ \vspace{0.5em} \\
\begin{tabular}{l|lllll}
$k \setminus j$ & $-4$ & $-2$ & $0$ & $2$ & $4$ \\
\hline
$6$ &  & $q^{-4}$ & $q^{-6}$ &  &  \\
$4$ & $1$ & $5q^{-2}$ & $4q^{-4}$ &  &  \\
$2$ &  & $6$ & $8q^{-2}$ & $2q^{-4}$ &  \\
$0$ &  & $5q^{2}$ & $12$ & $6q^{-2}$ &  \\
$-2$ &  & $q^{4}$ & $8q^{2}$ & $8$ & $q^{-2}$ \\
$-4$ &  &  & $4q^{4}$ & $6q^{2}$ & $2$ \\
$-6$ &  &  & $q^{6}$ & $2q^{4}$ & $q^{2}$ \\
\end{tabular}
\vspace{2em}
\end{minipage}
%
\begin{minipage}{\linewidth}
$\bullet\ $ $11n_{166}$ \vspace{0.5em} \\
\begin{tabular}{l|llll}
$k \setminus j$ & $0$ & $2$ & $4$ & $6$ \\
\hline
$8$ & $q^{-6}$ & $q^{-8}$ &  &  \\
$6$ & $2q^{-4}$ & $2q^{-6}$ &  &  \\
$4$ & $4q^{-2}$ & $5q^{-4}$ & $q^{-6}$ &  \\
$2$ & $3$ & $6q^{-2}$ & $2q^{-4}$ &  \\
$0$ & $4q^{2}$ & $q^{-2}$ + $8$ & $q^{-4}$ + $4q^{-2}$ &  \\
$-2$ & $2q^{4}$ & $1$ + $6q^{2}$ & $q^{-2}$ + $4$ &  \\
$-4$ & $q^{6}$ & $q^{2}$ + $5q^{4}$ & $2$ + $4q^{2}$ & $q^{-2}$ \\
$-6$ &  & $2q^{6}$ & $q^{2}$ + $2q^{4}$ & $1$ \\
$-8$ &  & $q^{8}$ & $q^{4}$ + $q^{6}$ & $q^{2}$ \\
\end{tabular}
\vspace{2em}
\end{minipage}
%
\begin{minipage}{\linewidth}
$\bullet\ $ $11n_{167}$ \vspace{0.5em} \\
\begin{tabular}{l|llll}
$k \setminus j$ & $0$ & $2$ & $4$ & $6$ \\
\hline
$4$ & $q^{-2}$ & $2q^{-4}$ & $q^{-6}$ &  \\
$2$ & $2$ & $5q^{-2}$ & $2q^{-4}$ &  \\
$0$ & $q^{2}$ & $7$ & $7q^{-2}$ & $q^{-4}$ \\
$-2$ &  & $5q^{2}$ & $7$ & $2q^{-2}$ \\
$-4$ &  & $2q^{4}$ & $7q^{2}$ & $5$ \\
$-6$ &  &  & $2q^{4}$ & $2q^{2}$ \\
$-8$ &  &  & $q^{6}$ & $q^{4}$ \\
\end{tabular}
\vspace{2em}
\end{minipage}
%
\begin{minipage}{\linewidth}
$\bullet\ $ $11n_{168}$ \vspace{0.5em} \\
\begin{tabular}{l|lllll}
$k \setminus j$ & $-2$ & $0$ & $2$ & $4$ & $6$ \\
\hline
$6$ & $q^{-2}$ & $2q^{-4}$ & $q^{-6}$ &  &  \\
$4$ & $2$ & $5q^{-2}$ & $3q^{-4}$ &  &  \\
$2$ & $q^{2}$ & $7$ & $8q^{-2}$ & $q^{-4}$ &  \\
$0$ &  & $5q^{2}$ & $9$ & $4q^{-2}$ &  \\
$-2$ &  & $2q^{4}$ & $8q^{2}$ & $6$ &  \\
$-4$ &  &  & $3q^{4}$ & $4q^{2}$ & $1$ \\
$-6$ &  &  & $q^{6}$ & $q^{4}$ &  \\
\end{tabular}
\vspace{2em}
\end{minipage}
%
\begin{minipage}{\linewidth}
$\bullet\ $ $11n_{169}$ \vspace{0.5em} \\
\begin{tabular}{l|llll}
$k \setminus j$ & $6$ & $8$ & $10$ & $12$ \\
\hline
$6$ & $q^{-6}$ &  &  &  \\
$4$ & $2q^{-4}$ & $2q^{-6}$ &  &  \\
$2$ & $2q^{-2}$ & $2q^{-4}$ &  &  \\
$0$ & $2$ & $4q^{-2}$ & $2q^{-4}$ &  \\
$-2$ & $2q^{2}$ & $3$ & $q^{-2}$ &  \\
$-4$ & $2q^{4}$ & $1$ + $4q^{2}$ & $q^{-2}$ + $2$ &  \\
$-6$ & $q^{6}$ & $2q^{4}$ & $q^{2}$ &  \\
$-8$ &  & $2q^{6}$ & $q^{2}$ + $2q^{4}$ & $1$ \\
\end{tabular}
\vspace{2em}
\end{minipage}
%
\begin{minipage}{\linewidth}
$\bullet\ $ $11n_{172}$ \vspace{0.5em} \\
\begin{tabular}{l|llll}
$k \setminus j$ & $-6$ & $-4$ & $-2$ & $0$ \\
\hline
$8$ & $q^{-2}$ & $2q^{-4}$ & $q^{-6}$ &  \\
$6$ & $1$ & $3q^{-2}$ & $2q^{-4}$ &  \\
$4$ & $q^{2}$ & $5$ & $5q^{-2}$ & $q^{-4}$ \\
$2$ &  & $3q^{2}$ & $5$ & $2q^{-2}$ \\
$0$ &  & $2q^{4}$ & $5q^{2}$ & $4$ \\
$-2$ &  &  & $2q^{4}$ & $2q^{2}$ \\
$-4$ &  &  & $q^{6}$ & $q^{4}$ \\
\end{tabular}
\vspace{2em}
\end{minipage}
%
\begin{minipage}{\linewidth}
$\bullet\ $ $11n_{173}$ \vspace{0.5em} \\
\begin{tabular}{l|llll}
$k \setminus j$ & $2$ & $4$ & $6$ & $8$ \\
\hline
$8$ & $q^{-6}$ & $q^{-8}$ &  &  \\
$6$ & $2q^{-4}$ & $2q^{-6}$ &  &  \\
$4$ & $3q^{-2}$ & $5q^{-4}$ & $q^{-6}$ &  \\
$2$ & $2$ & $4q^{-2}$ & $2q^{-4}$ &  \\
$0$ & $3q^{2}$ & $q^{-2}$ + $7$ & $q^{-4}$ + $4q^{-2}$ &  \\
$-2$ & $2q^{4}$ & $2$ + $4q^{2}$ & $2q^{-2}$ + $2$ &  \\
$-4$ & $q^{6}$ & $q^{2}$ + $5q^{4}$ & $2$ + $4q^{2}$ & $q^{-2}$ \\
$-6$ &  & $2q^{6}$ & $2q^{2}$ + $2q^{4}$ & $2$ \\
$-8$ &  & $q^{8}$ & $q^{4}$ + $q^{6}$ & $q^{2}$ \\
\end{tabular}
\vspace{2em}
\end{minipage}
%
\begin{minipage}{\linewidth}
$\bullet\ $ $11n_{174}$ \vspace{0.5em} \\
\begin{tabular}{l|llll}
$k \setminus j$ & $4$ & $6$ & $8$ & $10$ \\
\hline
$4$ & $3q^{-4}$ & $2q^{-6}$ &  &  \\
$2$ & $5q^{-2}$ & $5q^{-4}$ &  &  \\
$0$ & $7$ & $10q^{-2}$ & $3q^{-4}$ &  \\
$-2$ & $5q^{2}$ & $11$ & $6q^{-2}$ &  \\
$-4$ & $3q^{4}$ & $10q^{2}$ & $8$ & $q^{-2}$ \\
$-6$ &  & $5q^{4}$ & $6q^{2}$ & $1$ \\
$-8$ &  & $2q^{6}$ & $3q^{4}$ & $q^{2}$ \\
\end{tabular}
\vspace{2em}
\end{minipage}
%
\begin{minipage}{\linewidth}
$\bullet\ $ $11n_{176}$ \vspace{0.5em} \\
\begin{tabular}{l|llll}
$k \setminus j$ & $-8$ & $-6$ & $-4$ & $-2$ \\
\hline
$8$ & $q^{-2}$ & $2q^{-4}$ & $q^{-6}$ &  \\
$6$ & $1$ & $4q^{-2}$ & $3q^{-4}$ &  \\
$4$ & $q^{2}$ & $6$ & $6q^{-2}$ & $q^{-4}$ \\
$2$ &  & $4q^{2}$ & $8$ & $4q^{-2}$ \\
$0$ &  & $2q^{4}$ & $6q^{2}$ & $4$ \\
$-2$ &  &  & $3q^{4}$ & $4q^{2}$ \\
$-4$ &  &  & $q^{6}$ & $q^{4}$ \\
\end{tabular}
\vspace{2em}
\end{minipage}
%
\begin{minipage}{\linewidth}
$\bullet\ $ $11n_{177}$ \vspace{0.5em} \\
\begin{tabular}{l|llll}
$k \setminus j$ & $0$ & $2$ & $4$ & $6$ \\
\hline
$8$ & $q^{-6}$ & $q^{-8}$ &  &  \\
$6$ & $3q^{-4}$ & $3q^{-6}$ &  &  \\
$4$ & $5q^{-2}$ & $6q^{-4}$ & $q^{-6}$ &  \\
$2$ & $5$ & $9q^{-2}$ & $3q^{-4}$ &  \\
$0$ & $5q^{2}$ & $q^{-2}$ + $10$ & $q^{-4}$ + $5q^{-2}$ &  \\
$-2$ & $3q^{4}$ & $1$ + $9q^{2}$ & $q^{-2}$ + $6$ &  \\
$-4$ & $q^{6}$ & $q^{2}$ + $6q^{4}$ & $2$ + $5q^{2}$ & $q^{-2}$ \\
$-6$ &  & $3q^{6}$ & $q^{2}$ + $3q^{4}$ & $1$ \\
$-8$ &  & $q^{8}$ & $q^{4}$ + $q^{6}$ & $q^{2}$ \\
\end{tabular}
\vspace{2em}
\end{minipage}
%
\begin{minipage}{\linewidth}
$\bullet\ $ $11n_{178}$ \vspace{0.5em} \\
\begin{tabular}{l|llll}
$k \setminus j$ & $2$ & $4$ & $6$ & $8$ \\
\hline
$4$ & $2q^{-4}$ & $2q^{-6}$ &  &  \\
$2$ & $5q^{-2}$ & $4q^{-4}$ &  &  \\
$0$ & $7$ & $10q^{-2}$ & $3q^{-4}$ &  \\
$-2$ & $5q^{2}$ & $11$ & $6q^{-2}$ &  \\
$-4$ & $2q^{4}$ & $10q^{2}$ & $9$ & $q^{-2}$ \\
$-6$ &  & $4q^{4}$ & $6q^{2}$ & $2$ \\
$-8$ &  & $2q^{6}$ & $3q^{4}$ & $q^{2}$ \\
\end{tabular}
\vspace{2em}
\end{minipage}
%
\begin{minipage}{\linewidth}
$\bullet\ $ $11n_{179}$ \vspace{0.5em} \\
\begin{tabular}{l|llll}
$k \setminus j$ & $-2$ & $0$ & $2$ & $4$ \\
\hline
$6$ & $q^{-4}$ & $q^{-6}$ &  &  \\
$4$ & $3q^{-2}$ & $3q^{-4}$ &  &  \\
$2$ & $5$ & $8q^{-2}$ & $3q^{-4}$ &  \\
$0$ & $3q^{2}$ & $9$ & $5q^{-2}$ &  \\
$-2$ & $q^{4}$ & $8q^{2}$ & $9$ & $2q^{-2}$ \\
$-4$ &  & $3q^{4}$ & $5q^{2}$ & $2$ \\
$-6$ &  & $q^{6}$ & $3q^{4}$ & $2q^{2}$ \\
\end{tabular}
\vspace{2em}
\end{minipage}
%
\begin{minipage}{\linewidth}
$\bullet\ $ $11n_{180}$ \vspace{0.5em} \\
\begin{tabular}{l|lll}
$k \setminus j$ & $6$ & $8$ & $10$ \\
\hline
$6$ & $q^{-6}$ &  &  \\
$4$ & $2q^{-4}$ & $2q^{-6}$ &  \\
$2$ & $3q^{-2}$ & $3q^{-4}$ &  \\
$0$ & $4$ & $6q^{-2}$ & $2q^{-4}$ \\
$-2$ & $3q^{2}$ & $5$ & $2q^{-2}$ \\
$-4$ & $2q^{4}$ & $6q^{2}$ & $4$ \\
$-6$ & $q^{6}$ & $3q^{4}$ & $2q^{2}$ \\
$-8$ &  & $2q^{6}$ & $2q^{4}$ \\
\end{tabular}
\vspace{2em}
\end{minipage}
%
\begin{minipage}{\linewidth}
$\bullet\ $ $11n_{182}$ \vspace{0.5em} \\
\begin{tabular}{l|llll}
$k \setminus j$ & $-4$ & $-2$ & $0$ & $2$ \\
\hline
$8$ & $q^{-2}$ & $q^{-6}$ + $q^{-4}$ & $q^{-8}$ &  \\
$6$ &  & $3q^{-4}$ & $3q^{-6}$ &  \\
$4$ & $q^{2}$ & $5q^{-2}$ + $2$ & $6q^{-4}$ + $q^{-2}$ & $q^{-6}$ \\
$2$ &  & $7$ & $10q^{-2}$ & $3q^{-4}$ \\
$0$ &  & $5q^{2}$ + $q^{4}$ & $11$ + $q^{2}$ & $5q^{-2}$ \\
$-2$ &  & $3q^{4}$ & $10q^{2}$ & $7$ \\
$-4$ &  & $q^{6}$ & $6q^{4}$ & $5q^{2}$ \\
$-6$ &  &  & $3q^{6}$ & $3q^{4}$ \\
$-8$ &  &  & $q^{8}$ & $q^{6}$ \\
\end{tabular}
\vspace{2em}
\end{minipage}
%
\begin{minipage}{\linewidth}
$\bullet\ $ $11n_{183}$ \vspace{0.5em} \\
\begin{tabular}{l|llll}
$k \setminus j$ & $6$ & $8$ & $10$ & $12$ \\
\hline
$6$ & $q^{-6}$ &  &  &  \\
$2$ & $q^{-2}$ & $q^{-4}$ &  &  \\
$0$ & $2q^{-2}$ & $2q^{-4}$ &  &  \\
$-2$ & $2$ + $q^{2}$ & $2q^{-2}$ + $1$ &  &  \\
$-4$ & $2q^{2}$ & $5$ & $3q^{-2}$ &  \\
$-6$ & $q^{6}$ & $2q^{2}$ + $q^{4}$ & $2$ &  \\
$-8$ &  & $2q^{4}$ & $3q^{2}$ & $1$ \\
\end{tabular}
\vspace{2em}
\end{minipage}
%
\begin{minipage}{\linewidth}
$\bullet\ $ $11n_{184}$ \vspace{0.5em} \\
\begin{tabular}{l|llll}
$k \setminus j$ & $2$ & $4$ & $6$ & $8$ \\
\hline
$4$ & $2q^{-4}$ & $2q^{-6}$ &  &  \\
$2$ & $5q^{-2}$ & $4q^{-4}$ &  &  \\
$0$ & $6$ & $9q^{-2}$ & $3q^{-4}$ &  \\
$-2$ & $5q^{2}$ & $10$ & $5q^{-2}$ &  \\
$-4$ & $2q^{4}$ & $9q^{2}$ & $8$ & $q^{-2}$ \\
$-6$ &  & $4q^{4}$ & $5q^{2}$ & $1$ \\
$-8$ &  & $2q^{6}$ & $3q^{4}$ & $q^{2}$ \\
\end{tabular}
\vspace{2em}
\end{minipage}
%
\begin{minipage}{\linewidth}
$\bullet\ $ $11n_{185}$ \vspace{0.5em} \\
\begin{tabular}{l|llll}
$k \setminus j$ & $4$ & $6$ & $8$ & $10$ \\
\hline
$4$ & $3q^{-4}$ & $2q^{-6}$ &  &  \\
$2$ & $5q^{-2}$ & $5q^{-4}$ &  &  \\
$0$ & $8$ & $11q^{-2}$ & $3q^{-4}$ &  \\
$-2$ & $5q^{2}$ & $12$ & $7q^{-2}$ &  \\
$-4$ & $3q^{4}$ & $11q^{2}$ & $9$ & $q^{-2}$ \\
$-6$ &  & $5q^{4}$ & $7q^{2}$ & $2$ \\
$-8$ &  & $2q^{6}$ & $3q^{4}$ & $q^{2}$ \\
\end{tabular}
\vspace{2em}
\end{minipage}
\end{multicols}