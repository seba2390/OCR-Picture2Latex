\section{Future Prospects}\label{sec:future}

\subsection{On spectral sequences}
The celebrated conjecture given by Dunfield, Gukov, and Rasmussen \cite{DGR} claims that the HOMFLY homology unifies categorifications of the skein link polynomials. 
More precisely, they conjectured that for each knot $K$, there are differentials $\{d_N\}_{N \in \mathbb{Z}}$ on the HOMFLY homology $\Hbar(K)$ and an involution $\Phi$ on $\Hbar(K)$ such that 
\begin{itemize}
    \item the homology of $\Hbar(K)$ with respect to the differential $d_N$ gives the (reduced) $sl(N)$ homology $\Hbar_N(K)$ for $N > 0$,
    \item the homology of $\Hbar(K)$ with respect to $d_0$ gives the knot Floer homology of $K$,
    \item the involution $\Phi$ gives isomorphisms $\Phi \colon \Hbar^{i,\,j,\,k}(L) \cong \Hbar^{-i,\,j,\,k+2i}(L)$, 
    \item $d_Nd_M = -d_Md_N$ and $\Phi d_N = d_{-N}\Phi$ for all $N,M \in \ZZ$.
\end{itemize}
Rasmussen \cite{Ras15} found a spectral sequence from the HOMFLY homology to the $sl(N)$ homology for each $N > 0$ which partially supports the first statement. 
The $m$-th differential $d_m(N)$ in the spectral sequence has triple degree $(2mN,-2m,2-2m)$. If the sequence converges after the first differential $d_1(N)$, then the first statement is verified. 

For the knots we have computed, for $N > 2$ the differentials $d_m(N)$
are necessarily trivial by degree reasons. In other words, the HOMFLY homology $\Hbar(K)$ directly gives the $sl(N)$ homology $\Hbar_N(K)$ for $N > 2$.

For $N = 2$, the first differential $d_1(2)$ is non-trivial for many KR-thick knots listed in \Cref{prop:KR-thick-10,prop:KR-thick-11}.
For some of them, we have verified by hand that $d_m(2) = 0$ for $m > 1$ by observing their reduced Khovanov homology and using the anti-commutativity of $d_1(1)$ and $d_1(2)$.

A future objective for the authors is to refine the algorithm so that we can compute Rasmussen's spectral sequences and check the Dunfield--Gukov--Rasmussen conjecture for more knots. It will also allow us to compute the integer valued invariant $S(K)$ \cite{DGR}, which comes from the spectral sequence for $N = 1$ in a similar way with Rasmussen's $s$-invariant \cite{Rasmussen:2010}.

\subsection{On homologies for general knot diagrams}\label{subsec:generalsym}

As noted in \Cref{sec:prelim}, $\Hbar(D)$ does not generally give the reduced HOMFLY homology unless $D$ is a braid closure diagram. Still our program can partially\footnote{Since the symmetry of the homology can not be assumed, it is possible that there are more non-trivial summands in level $l > -n$.} compute $\Hbar(D)$ for an arbitrary knot diagram $D$. 
%
For example, consider the following diagram of $8_{15}$ represented by the planar diagram code \cite{knotinfo,knotatlas}
\[
    D = X_{1726}X_{3,15,4,14}X_{5968}X_{7382}X_{9,13,10,12}X_{11,1,12,16}X_{13,5,14,4}X_{15,11,16,10}.
    % [[1,7,2,6],[3,15,4,14],[5,9,6,8],[7,3,8,2],[9,13,10,12],[11,1,12,16],[13,5,14,4],[15,11,16,10]]
\]
\Cref{table:815} shows the homologies $\Hbar(8_{15})$ and $\Hbar(D)$ (possibly partial). It is obviously not isomorphic to $\Hbar(8_{15})$, nor is it symmetric as in \Cref{prop:q-sym}.

\begin{table}[h]
\begin{minipage}[t]{.49\linewidth}
\small
\centering
\setlength{\tabcolsep}{2pt}
\begin{tabular}[t]{l|llll}
$k \setminus j$ & $4$ & $6$ & $8$ & $10$ \\
\hline
$4$ & $q^{-4}$ &  &  &  \\
$2$ & $2q^{-2}$ & $2q^{-4}$ &  &  \\
$0$ & $3$ & $3q^{-2}$ &  &  \\
$-2$ & $2q^{2}$ & $5$ & $3q^{-2}$ &  \\
$-4$ & $q^{4}$ & $3q^{2}$ & $2$ &  \\
$-6$ &  & $2q^{4}$ & $3q^{2}$ & $1$ \\
\end{tabular}
\end{minipage}
\begin{minipage}[t]{.49\linewidth}
\small
\centering
\setlength{\tabcolsep}{2pt}
\begin{tabular}[t]{l|lllll}
$k \setminus j$ & $4$ & $6$ & $8$ & $10$ & $12$ \\
\hline
$4$ & $q^{-4}$ & & & & \\
$2$ & $2q^{-2}$ & $2q^{-4}$ & & & \\
$0$ & $3$ & $6q^{-2}$ & $3q^{-4}$ & & \\
$-2$ & $2q^2$ & $8$ & $7q^{-2}$ & $q^{-4}$ & \\
$-4$ & $q^4$ & $6q^2$ & $10$ & $5q^{-2}$ & \\
$-6$ & & $2q^4$ & $6q^2$ & $5$ & $q^{-2}$ \\
$-8$ & & & $3q^4$ & $5q^2$ & $2$ \\
\end{tabular}
\end{minipage}
\caption{$\Hbar(8_{15})$ and $\Hbar(D)$ for a diagram $D$ of $8_{15}$}
\label{table:815}
\end{table}

At the time of writing, we do not know how $\Hbar(D)$ varies under the RIIb moves  (\Cref{question:refine-H}). Nevertheless, we can possibly extend the usage of $\Hbar(D)$. For $\Hbar(D)$ to give the correct HOMFLY homology of the corresponding knot $K$, it is \textit{not} necessary that $D$ be a closed braid diagram of $K$.
Indeed, if a spectral sequence from $\Hbar(D)$ to $\Hbar_N(D)$ can be constructed for each $N > 0$ in Rasmussen's manner \cite{Ras15}, then for sufficiently large $N > 0$ we have $\Hbar(D) \cong \Hbar_N(D) \cong \Hbar_N(K) \cong \Hbar(K)$. This holds when a variant of the MOY calculus used in \cite{Ras15, Wu08} runs until the end for $D$. We say $D$ is \textit{good} if $\Hbar(D) \cong \Hbar(K)$.

Finding a good diagram for a knot $K$ can be beneficial for homology computation. 
If there is a good diagram of a knot $K$ which has less crossings than its braid length, then the computational cost for $\Hbar(K)$ can be reduced. For instance, the following diagram of $11n_{39}$
\[
D =
\begin{array}{l}
    X_{4251}X_{8493}X_{11,17,12,16}X_{12,5,13,6}X_{6,13,7,14}X_{17,22,18,1}\\
    X_{9,18,10,19}X_{21,10,22,11}X_{15,21,16,20}X_{19,15,20,14}X_{2837}
\end{array}
\]
is good and has $11$ crossings, while the braid length of $11n_{39}$ is $13$.

\begin{question}
    Can we give geometrical or combinatorial characterization of good diagrams? Is there a systematic (or algorithmic) way to produce good diagrams for any knot $K$?
\end{question}

\begin{question}
    Are the spectral sequences for good diagrams also knot invariants?
\end{question}