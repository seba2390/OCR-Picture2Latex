%abstract rewriting to get better reviewers.  signal for do we want numerical linear algebra stuff at neurips.  classical empirical compression people 
%change the title to automated model preserving neural network compression  %emphasise the model correlation idea, frame the goals we want model preservation structure preservation and not a lot of fine tuning.  preserving per example decisions.  
%in the appendix talk a little bit more about the problem with pruning and how we demonstrate the 
%take the intro that is there and re-write it.  
% use some sort of "chemical notation" 
%add the two references that people noted.  fix our own references that were missing years.  
%re-do some of the plots, don't make it look like the background.  re-do the visuals to look like that.  make id dashed, and make the composed methods more obvious.  hit hte reader over the head with structure preserving. 
%add the checklist back in.  



\documentclass{article}
% if you need to pass options to natbib, use, e.g.:
\PassOptionsToPackage{numbers, compress}{natbib}
% before loading neurips_2022

% ready for submission
% \usepackage{neurips_2022}

% to compile a preprint version, e.g., for submission to arXiv, add add the
% [preprint] option:
\usepackage[preprint]{neurips_2022}

% to compile a camera-ready version, add the [final] option, e.g.:
%\usepackage[final]{neurips_2022}

% to avoid loading the natbib package, add option nonatbib:
%    \usepackage[nonatbib]{neurips_2022}


\usepackage[utf8]{inputenc} % allow utf-8 input
\usepackage[T1]{fontenc}    % use 8-bit T1 fonts
\usepackage{hyperref}       % hyperlinks
\usepackage{url}            % simple URL typesetting
\usepackage{booktabs}       % professional-quality tables
\usepackage{amsfonts}       % blackboard math symbols
\usepackage{nicefrac}       % compact symbols for 1/2, etc.
\usepackage{microtype}      % microtypography
\usepackage{xcolor}         % colors

%%%% User defined macros %%%%%
% Get better typography
% \usepackage[protrusion=true,expansion=true]{microtype}		

% Small margins
%\usepackage[top=2cm, bottom=2cm, left = 1cm, right = 1cm,columnsep=20pt]{geometry}


% For basic math, align, fonts, etc.
\usepackage{amsmath}
\usepackage{amsthm}
\usepackage{amssymb}
\usepackage{bbm}
\usepackage{mathtools}
\usepackage{mathrsfs}
\usepackage{dsfont}
\mathtoolsset{showonlyrefs}

\newtheorem{defn}{Definition}[]
\newtheorem{thm}{Theorem}[]
\newtheorem{ass}{Assumption}[]
\newtheorem{cor}{Corollary}[]

% For creating sub-groups of Assumptions/Theorems etc.
\newtheorem{assumpA}{Assumption}
\renewcommand\theassumpA{A\arabic{assumpA}}

\usepackage{courier} % For \texttt{foo} to put foo in Courier (for code / variables)

\usepackage{lipsum} % For dummy text

% For images
% \usepackage{graphicx}
% \usepackage{subcaption}
\usepackage[space]{grffile} % For spaces in image names

% For bibliography
\usepackage[round]{natbib}

% For links (e.g., clicking a reference takes you to the phy)
\usepackage{hyperref}

% For color
\usepackage{graphicx}
\usepackage{wrapfig}
\usepackage{xcolor}
\definecolor{dark-red}{rgb}{0.4,0.15,0.15}
\definecolor{dark-blue}{rgb}{0,0,0.7}
\hypersetup{
    colorlinks, linkcolor={dark-blue},
    citecolor={dark-blue}, urlcolor={dark-blue}
}


% Macros



\usepackage{algpseudocode}
% \usepackage[noend,linesnumbered,ruled]{algorithm2e}
\usepackage[vlined,linesnumbered,ruled,algo2e]{algorithm2e}
% \usepackage[vlined,ruled,algo2e]{algorithm2e}

% \usepackage[algo2e]{algorithm2e} 
\SetKwProg{Fn}{Function}{}{}
\let\oldnl\nl% Store \nl in \oldnl
\newcommand{\nonl}{\renewcommand{\nl}{\let\nl\oldnl}}% Remove line number for one line

% Miscelaneous headers
\usepackage{multicol}
\usepackage{enumitem}
\usepackage[utf8]{inputenc} % allow utf-8 input
\usepackage[T1]{fontenc}    % use 8-bit T1 fonts
\usepackage{hyperref}       % hyperlinks
\usepackage{url}            % simple URL typesetting
\usepackage{booktabs}       % professional-quality tables
\usepackage{nicefrac}       % compact symbols for 1/2, etc.


\usepackage{theoremref}
%%%%%%%%%%%%%%%%%%%%%%%%%%%%%%


\title{
Model Preserving Compression for Neural Networks
%Compressing Neural Networks while Preserving the Original Model
}


% The \author macro works with any number of authors. There are two commands
% used to separate the names and addresses of multiple authors: \And and \AND.
%
% Using \And between authors leaves it to LaTeX to determine where to break the
% lines. Using \AND forces a line break at that point. So, if LaTeX puts 3 of 4
% authors names on the first line, and the last on the second line, try using
% \AND instead of \And before the third author name.


\author{%
  %David S.~Hippocampus\thanks{Use footnote for providing further information
  %  about author (webpage, alternative address)---\emph{not} for acknowledging
  %  funding agencies.} \\
  %Department of Computer Science\\
  %Cranberry-Lemon University\\
  %Pittsburgh, PA 15213 \\
  %\texttt{hippo@cs.cranberry-lemon.edu} \\
  % examples of more authors
  % \And
  % Coauthor \\
  % Affiliation \\
  % Address \\
  % \texttt{email} \\
  % \AND
  % Coauthor \\
  % Affiliation \\
  % Address \\
  % \texttt{email} \\
  % \And
  % Coauthor \\
  % Affiliation \\
  % Address \\
  % \texttt{email} \\
  % \And
  % Coauthor \\
  % Affiliation \\
  % Address \\
  % \texttt{email} \\
}




\author{
  Jerry Chee\thanks{equal contribution} \\
  Department of Computer Science\\
  Cornell University\\
  \texttt{jerrychee@cs.cornell.edu}\\
  %Pittsburgh, PA 15213 \\
  %\texttt{hippo@cs.cranberry-lemon.edu} \\
  % examples of more authors
   \And
   Megan Renz\footnotemark[1]\\
   Department of Physics\\
   Cornell University \\
   \texttt{mr2268@cornell.edu}\\
  % Address \\
  % \texttt{email} \\
   \AND
   Anil Damle \\
   Department of Computer Science\\
   Cornell University \\
   \texttt{damle@cornell.edu}\\
  % Address \\
  % \texttt{email} \\
   \And
   Christopher De Sa \\
   Department of Computer Science\\
   Cornell University \\
   \texttt{cdesa@cs.cornell.edu}
  % Address \\
  % \texttt{email} \\
  % \And
  % Coauthor \\
  % Affiliation \\
  % Address \\
  % \texttt{email} \\
}



\begin{document}


\maketitle


\begin{abstract}
After training complex deep learning models, a common task is to compress the model to reduce compute and storage demands. When compressing, it is desirable to preserve the original model's per-example decisions (e.g., to go beyond top-1 accuracy or preserve robustness), maintain the network's structure, automatically determine per-layer compression levels, and eliminate the need for fine tuning. No existing compression methods simultaneously satisfy these criteria---we introduce a principled approach that does by leveraging interpolative decompositions. Our approach simultaneously selects and eliminates channels 
%a structured low-rank matrix approximation known as the interpolative decomposition.
%By explicitly building an approximation to the activation output of each layer, we simultaneously select and eliminate channels
(analogously, neurons), then constructs an interpolation matrix that propagates a correction into the next layer, preserving the network's structure. 
Consequently, our method achieves good performance even without fine tuning and admits theoretical analysis.
Our theoretical generalization bound for a one layer network lends itself naturally to a heuristic that allows our method to automatically choose per-layer sizes for deep networks.
% Since our method simply makes networks narrower, it can easily be combined with other matrix decomposition techniques.
We demonstrate the efficacy of our approach with strong empirical performance on a variety of tasks, models, and datasets---from simple one-hidden-layer networks to deep networks on ImageNet.

% Preserving the original model's per-example decisions beyond top-1 accuracy and enabling effortless plug-and-play within machine learning pipelines by preserving network structure and minimizing fine tuning are both desirable properties for network compression methods. 
%by preserving the structure of the network while minimizing 
%Practical neural network compression methods should have the following qualities:  First to preserve the model's per-example decisions to maintain properties beyond top-1 accuracy (like
%sub-class accuracy
%fairness criteria
%or adversarial robustness), and second to enable
%effortless plug-and-play within a machine learning pipeline 
%by preserving the structure of the network while minimizing fine tuning and hyperparameter search.
%to ensure trivial plug-and-play within a machine learning pipeline.  
% A neural network compression method that is effective for practitioners must preserve the model's decisions as well as be practically usable.
% Preserving the per-example decisions ensures that properties beyond top-1 accuracy, such as sub-class accuracy or adversarial robustness, are retained.
% To ensure ease of use, the structure of the network must be preserved to enable trivial plug-and-play with the rest of the machine learning pipeline, and hyper-parameter tuning must be kept to a realistic minimum.
%Our goal is to compress deep networks into narrower but identically structured models that closely mirror the per-example decisions of the original model, 
%with minimal hyper-parameter tuning.
%To satisfy these criteria 
%Our method (uniquely?) combines the advantages of two types of well known compression methods: matrix approximation preserves the model's decisions, and structured pruning preserves the network's structure.
%Matrix approximation methods typically only satisfy the first criteria, while structured pruning methods typically only satisfy the second.
% Matrix approximation methods preserve the output, while structured pruning methods preserve the computational structure. 
% Our method combines advantages of both these types of methods.
%Matrix approximations typically preserve the model well, but change the network structure by adding additional layers.
%Structured pruning retains the network's structure, but does not typically preserve the model's decisions.

% We introduce a principled approach that leverages the interpolative decomposition to build a structured low-rank approximation of the activation output of each layer. 
% By doing so, we simultaneously select and eliminate channels 
% %a structured low-rank matrix approximation known as the interpolative decomposition.
% %By explicitly building an approximation to the activation output of each layer, we simultaneously select and eliminate channels
% (analogously, neurons) and construct an interpolation matrix that propagates a correction into the next layer, preserving the network's structure.
% Consequently, our method achieves good performance even without fine tuning and admits theoretical analysis.
% Our theoretical generalization bound for a one layer network lends itself naturally to a heuristic that allows our method to automatically choose per-layer sizes for deep networks.
% % Since our method simply makes networks narrower, it can easily be combined with other matrix decomposition techniques.
% We demonstrate the efficacy of our approach with strong empirical performance on a variety of tasks, models, and datasets---from simple one hidden layer networks to deep networks on ImageNet.
\end{abstract}


% \listoftodos

\section{Introduction}
\label{sec:intro}
\section{Introduciton}

Today's WiFi networks use advanced authentication and encryption mechanisms (such as WPA3) to protect our privacy and security by stopping unauthorized devices from accessing our devices and data. Despite all these mechanisms, WiFi networks remain vulnerable to attacks mainly due to their physical layer behaviors and requirements defined by WiFi standards. In this paper, we find two loopholes in the IEEE 802.11 standard for the first time and show how they can put our privacy and security at risk. 

\textbf{a) WiFi radios respond when they should not.}  In a WiFi network, when a device sends a packet to another device, the receiving device sends an acknowledgment back to the transmitter. In particular, upon receiving a frame, the receiver calculates the cyclic redundancy check (CRC) of the packet in the physical layer to detect possible errors. If it passes CRC, then the receiver sends an Acknowledgment (ACK) to the transmitter to notify the correct reception of the frame. Surprisingly, we have found that all existing WiFi devices send back ACKs to even fake packets received from unauthorized WiFi devices outside of their network. Why should a WiFi device respond to a fake packet from an unauthorized device?! 

\textbf{b) WiFi radios stay awake when they should not.}
WiFi chipsets are mostly in sleep mode to save power. However, to make sure that they do not miss their incoming packets, they notify their WiFi access point before entering sleep mode so that the access point buffers any incoming packets for them. Then, WiFi devices wake up periodically to receive beacon frames sent by the associated access point. In regular operation, only the access point sends beacon frames to notify the devices that have buffered packets. When a device is notified, it stays awake to receive them. However, these beacon frames are not encrypted. Hence, we find that an unauthorized user can forge those beacon frames to keep a specific device awake for receiving the (non-existent) buffered frames. %has packets waiting for it. 
%This keeps the WiFi radio awake and prevents it from going to sleep mode to save power.

We examine these behaviors and loopholes in detail over different WiFi chipsets from different vendors. Our examination of over 5,000 WiFi devices from 186 vendors shows that these are widespread issues. We then study the root cause of these issues and show that, unfortunately, they cannot be fixed by a simple solution such as updating WiFi chipsets firmware.  Finally, we implement and demonstrate two attacks based on these loopholes. In the first attack, we show that by forcing WiFi devices to stay awake and continuously transmit, an adversary can continuously analyze the signal and extract personal information such as the breathing rate of the WiFi users. In the second attack, we show that by forcing WiFi devices to stay awake and continuously transmit, the adversary can quickly drain the battery, and hence disable WiFi devices such as home and office security sensors. These attacks can be performed from outside buildings despite the WiFi network and devices being completely secured. All the attacker needs is a \$10 microcontroller with integrated WiFi (such as ESP32) and a battery bank. The attacker device can easily be carried in a pocket or hidden somewhere near the target building. 

The main contributions of this work are:
%\footnote{We discussed our project and experiments with our institution’s IRB office and they determined that no IRB review nor IRB approval is required.}:
\begin{itemize}
    \item We find that WiFi devices respond to fake 802.11 frames with ACK, even when they are from unauthorized devices. We also find that WiFi radios can be kept awake by sending them fake beacon frames indicating they have packets waiting for them. 
    \item We study these loopholes and their root causes in detail, and have tested more than 5,000 WiFi access points and client devices from more than 186 vendors.  
    
    \item We implement two attacks based on these loopholes using just a 10-dollar off-the-shelf WiFi module and validate them in real-world settings.
    



    
\end{itemize}






%\section{Context and Related Work}
\section{Related Work}
\label{sec:related}
There are a number of different design choices to be made in the compression and pruning process.  Classical pruning involves eliminating either channels (analogously neurons) or individual weights, sometimes in a structured way.  
Magnitude pruning on both weights and neurons is still considered an effective approach~\cite{blalock2020state,frankle2018lottery,frankle2020pruning,gale2019state,liu2019rethink,li2017l1}.  When pruning, the method may incorporate a correction to future layers, though it often does not ~\cite{he2019fpgm,luo2017thinet}. Some methods that correct the network chose to do local fine tuning ~\cite{luo2017thinet,he2017feat,zhuang2018dcp,peng2019ccp,liu2017netslim}, whereas others do not ~\cite{liebenwein2020provable,he2018amc}.  

Of particular note, matrix approximation methods
~\cite{denten2014svd,idel2020lrank,liebenwein2021alds,peng2018group,lebedev2015cpdecomp}
%~\cite{denten2014svd,idel2020lrank,jarderberg2014lowrank,liebenwein2021alds,peng2018group,lebedev2015cpdecomp,zhang20153dfilter}
often satisfy the first criteria we desire for compression methods, but typically not the second as they add additional layers. These methods sometimes incorporate local fine tuning after compression ~\cite{idel2020lrank,jaderberg2014speeding,liebenwein2021alds,peng2018group,zhang20153dfilter} and sometimes do not ~\cite{denten2014svd,lebedev2015cpdecomp}.
In contrast, structured pruning methods
%~\cite{he2019fpgm,he2018amc,liebenwein2020provable,liu2019rethink,luo2017thinet}
~\cite{he2019fpgm,he2018amc,liebenwein2020provable,liu2017netslim,liu2019rethink,luo2017thinet}
can satisfy the second criteria we outlined, but typically not the first as they do a poor job of preserving the model's decisions and often require excessive amounts of fine tuning.


Pruning with coresets~\cite{mussay2020coreset} is the closest in spirit to our own work and provides a way to select a subset of neurons in the current layer that can approximate those in the next layer as well as new weight connections. Of note, \citet{mussay2020coreset} provide a sample complexity result,
and demonstrate their method on fully connected (but not convolution) layers.
The HRank method~\cite{lin2020hrank} is also close in spirit to our own, and works by selectively pruning channels that produce low-rank feature maps.   However, the method does not propagate updates into the next layer and instead relies on excessive amounts of fine tuning (30 epochs for each layer pruned) to fix the network's accuracy.  


Recently the literature has started to consider criteria beyond topline accuracy metrics, and \citet{liebenwein2021lost} use measures of functional approximation to conclude that pruned networks well approximate the original models. 
\citet{marx2020multiplicity} characterize when linear models can achieve similar accuracy but with competing predictions.

% It is interesting to explore how their approximation measures differ from our per-label correlation.


\section{Interpolative decompositions}
\label{sec:ID}
%\jerry{maybe emphasize that we're not a low-rank pruning method, even though we're using a low-rank matrix approxiation method}

Our pruning strategy relies on a structured low-rank approximation known as an interpolative decomposition (ID). 
% jerry 1.25.22: we talk more about preserving structure in intro now, don't think we need it here
%\jerrytwo{Note that a major benefit of the ID is that it preserves the structure of the network.
%Unlike other low-rank approximation methods, we do not need to store additional matrix factorizations, and we do not need to pay the additional reconstruction cost during inference.
%}
Classically, the Singular Value Decomposition (SVD) (see, e.g.,~\cite{GVL}) provides an optimal low-rank approximation. However, because we consider matrices that include the non-linear activation function the SVD cannot be directly used to either subselect neurons or generate new ones (since it is unclear how to propagate singular vectors ``backwards'' through the non-linearity). In contrast, an ID constructs a structured low-rank approximation of a matrix $A$ where the basis used for the approximation is constrained to be a subset of the columns of $A$. For a matrix $A\in\R^{n\times m}$ we let $A_{\mathcal{J},\I}$ denote a sub-selection of the matrix $A$ using sets $\mathcal{J}\subset [n]$ to denote the selected rows and $\I\subset [m]$ to denote the selected columns; $:$ denotes a selection of all rows or columns.
% The optimal basis is the leading left singular vectors of $A$, but generically these will not be columns of $A$.
%Shortly, we will see how the rank-revealing QR factorization is used to compute the interpolative decomposition.
%\begin{definition}[Interpolative decomposition]
%Given a matrix $A \in \R^{n \times m}$ and some $k \leq \ell$ (as before, $\ell = \min(n,m)$) a \emph{rank-k interpolative decomposition} is an index subset $\I \subset [m]$ with $|\I| = k$ and an interpolation matrix $T \in \R^{k \times m}$ with $T_{:,\I} = I_k$ such that
%\begin{equation}
%\label{eq:ID}
 %   A \approx A_{:,\I} T.
%\end{equation}
%\end{definition}

\begin{definition}[Interpolative Decomposition]
\label{def:ID}
Let $A \in \R^{n \times m} $and $\epsilon \geq 0$. 
An $\epsilon$-accurate \emph{interpolative decomposition} 
$A \approx A_{:,\I} T$
is a subset of columns of A, denoted with the index subset $\I \subset [m],$ and an associated interpolation matrix $T$ such that $\|A - A_{:,\I} T \|_2 \leq \epsilon\|A\|_2.$
\end{definition}






\begin{remarks}
\hspace{-0.8em}
When computing an ID, we would like to find the smallest possible $k\equiv\lvert\I\rvert$ such that the accuracy requirement is satisfied. Moreover, we would like $T$ to have entries of reasonable magnitude and approximation error not much larger than the best possible for a given $k$ (i.e., $\|A - A_{:,\I}T\|_2 = t \sigma_{k+1}(A)$ for some small $t \geq 1$). While necessarily sub-optimal, the advantage is that we explicitly use a subset of the columns of $A$ to build the approximation.
\end{remarks}

IDs are well studied~\cite{cheng2005compression,martinsson2011randomized}, widely used in the domain of rank-structured matrices~\cite{ho2012fast,ho2015hierarchical,ho2016hierarchical,martinsson2005fast,martinsson2019fast,minden2017recursive}, and are closely related to CUR decompositions~\cite{mahoney2009cur,voronin2017efficient} and subset selection problems~\cite{boutsidis2009improved,civril2009selecting,tropp2009column}. While these decompositions always exist, finding them optimally is a difficult task and in this work we appeal to what are known as (strong) rank-revealing QR factorizations~\cite{businger1965linear,chan1992some,chandrasekaran1994rank,gu1996efficient,hong1992rank}.


%\subsection{Rank-Revealing QR factorization}
% \paragraph{Rank-revealing QR factorizations}
\begin{definition}[Rank-revealing QR factorization]
Let $A \in \R^{n \times m}$, $\ell = \min(n,m)$, and take any $k \leq \ell$.
A \emph{rank-revealing QR factorization} of $A$ computes a permutation matrix $\Pi \in \R^{m \times m}$, an upper-trapezoidal matrix $R \in \R^{\ell \times m}$ (i.e. $R_{i,j}=0 \text{ if } i > j$), and a matrix $Q \in \R^{n \times \ell}$ with orthonormal columns (i.e., $Q^\top Q = I$) such that $A \Pi = Q R$ and $Q$ and $R$ satisfy certain properties.
Splitting $\Pi, Q$, and $R$ into $\Pi_1 \in \R^{m \times k}$, $\Pi_2 \in \R^{m \times (m-k)}$, $Q_1 \in \R^{n \times k}$, $Q_2 \in \R^{n \times (\ell-k)}$, $R_{11} \in \R^{k \times k}$, 
$R_{12} \in \R^{k \times (m-k)}$, and $R_{22} \in \R^{(\ell-k)\times(m-k)}$ we can write
%$R_{12} \in \R^{k \times (\ell-m)}$, and $R_{22} \in \R^{(\ell-k)\times(\ell-n)}$ we can write
\begin{equation}
\label{eq:rankQR}
    A
    \begin{bmatrix}
    \Pi_1 & \Pi_2
    \end{bmatrix}
    =
    \begin{bmatrix}
    Q_1 & Q_2
    \end{bmatrix}
    \begin{bmatrix}
    R_{11} & R_{12} \\
    & R_{22}
    \end{bmatrix}.
\end{equation}
\end{definition}

\begin{remarks}
What makes~\eqref{eq:rankQR} a rank-revealing QR factorization is that the permutation $\Pi$ is computed to ensure that $R_{11}$ is as well-conditioned as possible and $R_{22}$ is as small as possible. While more formal statements of these conditions exist, we omit them here as they do not factor into our work.
\end{remarks}

Critically, any rank-revealing QR factorization yields a natural rank-$k$ approximation of $A$ with error
\begin{equation*}
    \| A - Q_1 
    \begin{bmatrix}
    R_{11} & R_{22}
    \end{bmatrix}
    \Pi^\top \|_2
    =
    \| R_{22} \|_2.
\end{equation*}
While finding the optimal rank-revealing QR factorization (\emph{i.e.,} minimizing the error for a given $k$) is closely related to a provably hard problem~\cite{civril2009selecting}, we find the original algorithm of Businger and Golub~\cite{businger1965linear} works well in practice. This routine is available in LAPACK~\cite{lapack,blas3QRCP}, can be easily incorporated into existing code, and has computational complexity $\mathcal{O}(nmk)$ when run for $k$ steps.  
%While optimally finding a rank-revealing QR is closely related to the provably hard problem of finding maximum volume subsets~\cite{civril2009selecting}, many good algorithms exist~\cite{businger1965linear,blas3QRCP,chandrasekaran1994rank,gu1996efficient,golub1976rank} that are effective in practice.
%More specifically, these algorithms seek to ensure that (where $\sigma$ is used to denote the appropriate singular values)
%\begin{equation*}
 %   \sigma_{\min}(R_{11})
%    \geq 
%    \frac{\sigma_k(A)}{f_1(n,k)}
%    \quad\text{and}\quad
%    \sigma_{\max}(R_{22}) 
%    \leq
%    f_2(n,k) \sigma_{k+1}(A),
%\end{equation*}
%for functions $f_1$ and $f_2$ that grow mildly in $n$ and $k$.
%(For many of the specific algorithms referenced explicit expressions of $f_1$ and $f_2$ are known.)
%For our purposes, the original algorithm of Businger and Golub~\cite{businger1965linear} suffices and has runtime $\mathcal{O}(nm\ell)$; readily available implementations exist in LAPACK~\cite{lapack,blas3QRCP}, distributed memory implementations are available (e.g., in ScaLAPAK~\cite{Scalapack}), and randomization can be used to provide efficiency gains in practice~~\cite{liberty2007randomized,duersch2017randomized,martinsson2011randomized,martinsson2017householder,randomreview}.



\paragraph{Computing interpolative decompositions}
%\subsection{Computing interpolative decompositions}
Given a rank-revealing QR factorization, we can immediately construct an ID (a formal algorithmic statement is given in the appendix).
Let $\I\subset [m]$ be such that $A_{:,\I} = A \Pi_1$ and define the interpolation matrix
\begin{equation*}
    T = 
    \begin{bmatrix}
    I_k & R_{11}^{-1} R_{12}
    \end{bmatrix}
    \Pi^\top.
\end{equation*}
With the choice $A_{:,\I} = Q_1 R_{11}$ it follows that the error of the ID as defined by $\I$ and $T$ is 
$%$\begin{align*}
    \|A - A_{:,\I}T\|_2
    %&=
    %\|Q\begin{bmatrix}
    %0 & R_{22}
    %\end{bmatrix}
    %\Pi^\top \|_2 \\
    = \|R_{22}\|_2 
$. %$\end{align*}  
Picking $k$ such that $\|R_{22}\|_2\leq \epsilon \|A\|_2$ yields the desired relative error. Notably, since $\kappa(A_{:,\I}) = \kappa(R_{11})$ and $T = 
\begin{bmatrix}
I_k & R_{11}^{-1} R_{12}
\end{bmatrix}
\Pi^\top$ 
the desired criteria for an ID map back to those of a rank-revealing QR factorization---if $R_{11}$ is well conditioned then the  basis we use for approximation is as well and entries of $T$ are not too large.
If $\sigma_{\max}(R_{22})$ is not much larger than $\sigma_{k+1}(A)$ we get near optimal approximation accuracy.







\paragraph{Accuracy of the matrix approximation}
%\subsection{Accuracy of the matrix approximation}
%\label{sec:accuracyID}
A key feature of using a column-pivoted QR factorization to compute an ID is that it allows us to dynamically determine the approximation rank $k$ as a function of $\epsilon$. 
This can be accomplished by monitoring $\|R_{22}\|_2$ at each step of the column-pivoted QR algorithm. 
However, repeatedly computing $\|R_{22}\|_2$ is expensive and often unnecessary in practice. 
When using the algorithm by Businger and Golub~\cite{businger1965linear} the magnitude of the diagonal entries of $R$ are non-increasing and it is common to use $\lvert r_{k+1,k+1} / r_{1,1}\rvert$ as a proxy for 
%$\|R_{22}\|_2/\|R\|_2$.
$\|R_{22}\|_2/\|A\|_2$.
While formal bounds indicate the approximation may be loose in the worst case, it is effective in practice (see appendix Figure~\ref{fig:chosingk}) and once a candidate $k$ has been identified $\|R_{22}\|_2$ can be computed if desired to certify the result---if the accuracy is unacceptable $k$ can be increased until it is. 
In some settings it may be preferable to fix $k$ and simply accept whatever accuracy is achieved.


% The matrix R gives us useful information about the potential accuracy vs. complexity of the decomposition. 
% In the section above, we showed that the error from our decomposition is $||R_{22}||$. 
% However, computing the value of $||R_{22}||$ for each k can be expensive. 
% In this situation, we can use properties of the column-pivoted QR factorization to speed our algorithm. 
% The largest magnitude entry in $R_{22}$ is  in the upper left-hand corner of the matrix.  
% While this does not give us a useful hard bound on $||R_{22}||/||R||$, in practice we can use this as a proxy for the relative size of $||R_{22}||/||R||$.  
% \megan{cite places this is done in practice/ well established?}.  
% Let $r_{11} = R_{1,1}$ denote the matrix $R$ indexed at $(0,0)$, the top left, and $r_{kk} = R_{k,k}$ denote the matrix $R$ indexed at the $k$-th diagonal.
% In figure {\ref{fig:chosingk}}, we see that the value of $r_{kk}/r_{11}$ tracks well with the optimal bound as determined by the singular value decay, as well as with the actual error $||R_{22}||$. 
% This allows us to control the increase in error that we create for the network locally.  

%In figure (\megan{number}), we see that the degradation in accuracy matches what we would expect given the ratio $r_{kk}/r_{00}$.  




%\jerry{Maybe some simple simulations comparing RRQR to SVD?
%Similar to Fig2 in grant.}


%Given a matrix $A$ an interpolative decomposition may be computed using any rank-revealing QR factorization, though most commonly the column-pivoted QR factorization of Golub and Businger is used \todo{cite}.
%The procedure is given in Algorithm~\ref{alg:genericID}, which we elaborate on in further detail.
%For simplicity let $l = \min(m,n)$ and assume $k < l$.
%
%Concretely, we first compute the column-pivoted QR factorization
%\begin{equation*}
%    A \Pi = Q R
%\end{equation*}
%where $\Pi \in \R^{n \times n}$ is a permutation, $Q \in \R^{m \times l}$ has orthonormal columns and $R \in \R^{l \times n}$ is upper triangular.
%We now partition $\Pi$ and $R$ as 
%\begin{equation*}
%    \Pi = 
%    \begin{bmatrix}
%    \Pi_1 & \Pi_2
%    \end{bmatrix}
%    \quad\text{and}\quad
%    R =
%    \begin{bmatrix}
%    R_{11} & R_{12} \\
%    0    & R_{22}
%    \end{bmatrix},
%\end{equation*}
%where $\Pi_1 \in \R^{n \times k}, R_{11} \in \R^{k \times k}, R_{12} \in \R^{k \times n - k}$, and the remaining dimensions are as required.
%Let $\I$ denote the columns selected by the subselection matrix $\Pi$ such that
%\begin{equation*}
%    A_{:,\I} = A \Pi_1.
%\end{equation*}
%Letting the interpolation matrix
%\begin{equation*}
%    T = 
%    \begin{bmatrix}
%    I_k & R_{11}^{-1} R_{12}
%    \end{bmatrix}
%    \Pi^\top
%\end{equation*}
%we have the interpolative decomposition.
%Note that the error is 
%\begin{align*}
%    \|A - A_{:,\I}T\|_2
%    &=
%    \|Q\begin{bmatrix}
%    0 & R_{22}
%    \end{bmatrix}
%    \Pi^\top \|_2 \\
%    &= \|R_{22}\|_2.
%\end{align*}



\section{Pruning with interpolative decompositions}
\label{sec:pruneID}
The core of our approach is a novel use of IDs to 
prune neural networks. Here we illustrate the scheme for a single fully connected layer and we extend the scheme to more complex layers (e.g., convolution layers) and deeper networks in Section~\ref{sec:extendid}.
Consider a simple two layer (one hidden layer) fully connected neural network $h_{FC} : \R^d \to \R^c$ of width $m$ defined as 
\begin{equation*}
\label{eq:1hiddenfc}
    h_{FC}(x; W, U)
    =
    U^\top g(W^\top x)
\end{equation*}
with hidden layer $W \in \R^{d \times m}$, output layer $U \in \R^{m \times c}$, and activation function $g$. 
We omit bias terms: they may be readily incorporated by adding a row to $W$ and suitably augmenting the data.

To prune the model we will use an \emph{unlabeled} pruning data set  $\{x_i\}^{n}_{i=1}$ with $x_i \in \R^d$.
Let $X \in \R^{d \times n}$ be the matrix such that $X_{:,i} = x_i$.
Preserving the action of the two layer network to accuracy $\epsilon > 0$ on the data with fewer neurons is synonymous with finding an $\epsilon$ accurate approximation
$h_{FC}(x;W,U) \approx h_{FC}(x; \widehat{W},\widehat{U})$ 
where $\widehat{W}$ has fewer columns than $W$.
We can do this by computing an ID of the activation output of the first layer.

Concretely, let $Z \in \R^{m \times n}$ be the first-layer output, i.e., $Z = g(W^\top X),$ and let $Z^\top \approx (Z^\top)_{:,\I}  T $
be a rank-$k$ ID of $Z^\top$
with $|\I| = k$ and interpolation matrix $T \in \R^{k \times m}$ that achieves accuracy $\epsilon$ as in Definition~\ref{def:ID} (note that this means $k$ is a function of $\epsilon$).
%Importantly, 
Because the activation function $g$ commutes with the sub-selection operator, if $Z \approx T^\top Z_{\I,:}$ then
\begin{align*}
    g(W^\top X) 
    \approx T^\top \left( g(W^\top X) \right)_{\I,:}
    % \approx
    % T^\top g\left(\left( W^\top X \right)_{\I,:} \right)
    =
    T^\top g\left( W_{:,\I}^\top X \right).
\end{align*}
Multiplying both sides by $U^\top$ now gives an approximation of the original network by a pruned one,
\begin{equation}
    h_{FC}\left(x; W, U \right) 
    =
    U^\top g(W^\top X) \approx \\
    h_{FC}\left(x; W_{:,\I}, T U \right)
    =
    U^\top T^\top g\left( W_{:,\I}^\top X \right).
    \label{eqn:PruneApprox}
\end{equation}
% The activation output $Z$ can then be approximated
% \begin{equation}
%     Z \approx 
%     T^\top g(W_{1(:,\I)}^\top X)
% \end{equation}
% by setting $\widehat{W}_1 = W_{1(:,\I)}$ to select neurons in the current layer.
% In order to encode the interpolation matrix $T$ into the neural network, set $\widehat{W}_2 = T W_2$ 
% %to propagate the interpolation matrix into 
% in the next layer.
That is, the ID has pruned the network of width $m$ into a dense sub-network of width $k$ with $\widehat W \equiv W_{:,\I} \in \R^{d \times k}$ and $\widehat U \equiv T U \in \R^{k \times c}$.
% \begin{equation}
% \label{eq:fcID}
%     h_{FC}(x; W_1, W_2)
%     \approx
%     h_{FC}(x; W_{1(:,\I)} , T W_2) 
% \end{equation}
%\jerry{not sure if we want to say this:}
%Furthermore, if the interpolative decomposition achieves $\epsilon$ accuracy, then the neural network achieves \megan{at least}$\epsilon \|W_2\|$ accuracy.
Importantly, 
%An important point is that 
the SVD of $Z^\top$ cannot be used for this task since
it is not clear how to map the dominant left singular vectors back through the activation function $g$ to either a subset of the existing neurons or a small set of new neurons.
%In contrast the sub-selection operator of the interpolative decomposition commutes with the activation function.
This makes use of the ID essential and we provide additional intuition for this scheme in the appendix, specifically in Figure ~\ref{fig:patches}.  




%\begin{algorithm}[t]
%\SetAlgoLined
%\DontPrintSemicolon
%\KwIn{
%FC layer $f_{FC}(x;W_1)$, 
%next layer $f_{2}(x;W_2)$,
%%intermediate layer $g(x; w_3)$ (batch norm, pooling),
%pruning data $X_{p} \in \R^{d \times n_{p}}$,
%rank-$k(\epsilon)$
%}
%\KwOut{
%pruned layers
%$\widehat{f}_{FC}(x;\widehat{W}_1)$, 
%$\widehat{f}_{2}(x;\widehat{W}_2)$,
%%$\widehat{g}(x;\widehat{w}_3)$
%}
%
%%\tcc{apply batch norm or pool layer if it follows $f_{FC1}$}
%$Z \gets \gamma(X_{p}^\top W_1)$\;
%$Z_{:,\I}, T \gets \ID(Z, k(\epsilon))$\;
%$\widehat{W}_1 \gets (W_1)_{(:,\I)}$
%\tcp{select neurons via ID}
%$\widehat{W}_2 \gets \operatorname{matmul}(T, W_2)$
%\tcp{propagate interpolation matrix to next layer}
%%\tcp{depends on if next layer is FC or Conv}
%\caption{ID Pruning a FC Layer\jerry{might get rid of}}
%\label{alg:fcID}
%\end{algorithm}

\subsection{A generalization bound for the pruned network}
% \paragraph{A generalization bound for the pruned network}
A key feature of our ID based pruning method is that it can be used to dynamically select the width of the pruned network to maintain a desired accuracy when compared with the full model. This allows us to provide generalization guarantees for the compressed network in terms of generalization of the full model and the accuracy of the ID. We state the results for a single hidden fully connected layer with scalar output (i.e., $c=1$) and squared loss. They can be extended to more complex networks (at the expense of more complicated dependence on the accuracy each layer is pruned to), more general Lipschitz continuous loss functions, and vector valued output. We defer all proofs to the supplementary material.

Assume $(x,y)\sim \cD$ where $x\in\R^d,$ $y\in\R,$ and the distribution $\cD$ is supported on a compact domain $\Omega_x\times\Omega_y$. We let $\mathcal{R}_0 = \mathbb{E}_{(x,y) \sim \cD}(\|({u}^\top g({W}^\top x)- y)\|^2)$ denote the true risk of the trained full model and $\mathcal{R}_p = \mathbb{E}_{(x,y) \sim \cD}(\|({\widehat{u}}^\top g({\widehat{W}}^\top x)- y)\|^2)$ be the risk of the pruned model. (Since $c=1$ we let $u\in\R^m$ denote the last layer.)
We also define the empirical risk $\widehat{\cR}_{ID}$ of approximating the full model with our pruned model as
\begin{equation*}
\label{eq:pruneRisk}
    \widehat{\cR}_{ID}=\frac{1}{n}\sum_{i=1}^n \left\lvert{u}^\top g({W}^\top x_i) - {\widehat{u}}^\top g({\widehat{W}}^\top x_i) \right\rvert^2,
\end{equation*}
where $\{x_i\}_{i=1}^n$ are $n$ i.i.d.\ samples from $\cD$ (note that we do not need labels for these samples). Using this notation, Theorem~\ref{thm:generalization} controls the generalization error of the pruned model.


\begin{theorem}[Single hidden layer FC]
\label{thm:generalization}
Consider a model $h_{FC}=u^\top g(W^\top x)$ with m hidden neurons and a pruned model $\widehat{h}_{FC}=\widehat{u}^\top g(\widehat{W}^\top x)$ constructed using an $\epsilon$ accurate ID with $n$ data points drawn i.i.d\ from $\cD.$ The risk of the pruned model $\mathcal{R}_p$ on a data set $(x,y) \sim D$ assuming $\cD$ is compactly supported on $\Omega_x\times\Omega$ is bounded by  
\begin{equation}
\label{eq:RiskDcomp}
    \mathcal{R}_p \leq \mathcal{R}_{ID} + \mathcal{R}_0+ 2  \sqrt{ \mathcal{R}_{ID}  \mathcal{R}_0},
\end{equation}
where $\mathcal{R}_{ID}$ is the risk associated with approximating the full model by a pruned one and with probability $1-\delta$ satisfies
\begin{align*}
    {\mathcal{R}}_{ID} 
    &\leq 
    \epsilon^2M+M(1+\|T\|_2)^2n^{-\frac{1}{2}} 
    &\left( \sqrt{2\zeta dm \log (dm)\log\frac{en}{\zeta dm \log (dm)}}+ \sqrt{\frac{\log (1/\delta)}{2}}\right).
\end{align*} 
Here, $M = \sup_{x\in\Omega_x} \|u\|_2^2 \| g(W^T x)\|_2^2$ and $\zeta$ is a universal constant that depends on $g$. %the activation function.  


\end{theorem}
Theorem~\ref{thm:generalization} is developed by considering the ID as a learning algorithm applied to the output of the full model using unlabeled pruning data. This allows us to control the risk of the pruned model in terms of the risk of the original model and the additional risk introduced by the ID. Importantly, here we can control the additional risk in terms of the empirical risk of the ID and an additive term that decays as additional pruning data is used. Lemma~\ref{lem:prunedRisk} codifies this decomposition.
% \begin{remarks} 
% The statement $
%     \mathcal{R} \leq \mathcal{R}_p + \mathcal{R}_0+ 2 
%     \sqrt{ \mathcal{R}_p  \mathcal{R}_0}
% $ in Theorem~\ref{thm:generalization} follows naturally from the squared loss and the Cauchy-Schwartz inequality and holds for all pruning schemes.
% \end{remarks}


\begin{lemma}
\label{lem:prunedRisk}
Under the assumptions of Theorem~\ref{thm:generalization}, for any $\delta\in(0,1)$, $\cR_{ID}$ satisfies  
\begin{equation*}
    \mathcal{R}_{ID} 
    \leq \widehat{\cR}_{ID} + M(1+\|T\|_2)^2n^{-\frac{1}{2}} 
     \left( \sqrt{2p\log(en/p)}+ 2^{-\frac{1}{2}}\sqrt{\log (1/\delta)}\right)
\end{equation*}
with probability $1-\delta,$ where $M = \sup_{x\in\Omega_x} \|u\| ^2 \| g(W^T x)\|^2$ and $p=\zeta dm \log (dm)$ for some universal constant $\zeta$ that depends only on the activation function.
\end{lemma}
%\begin{lemma}
%The p-dimension of the network is 
%\begin{equation}
%    p \leq \zeta dm \log (dm)
%\end{equation}
%\end{lemma}
\begin{remarks}
We believe that the second part of the bound in Lemma~\ref{lem:prunedRisk} is likely loose since it relies on a pseudo-dimension bound for fully connected neural networks. However, when pruning with an ID we only consider subsets of existent neurons and it is plausible that in this setting the upper bound for the pseudo-dimension could be improved.
\end{remarks}

Crucially, an immediate consequence of using an ID for pruning is that we can explicitly control $\cR_{ID}$ in terms of the accuracy parameter. This relation between the ID accuracy and empirical risk is given in Lemma~\ref{lem:IDEmperical} and is what allows us to express the risk of the pruned network in Theorem~\ref{thm:generalization}. 

% Since the interpolative decomposition preserves the action of the network on the pruning data, (Lemma \ref{lem:IDrisk}), the empirical risk on the pruning data is bounded. 
%We can bound each term, beginning with $\hat{R}_s$.  
\begin{lemma}
\label{lem:IDEmperical} Following the notation of Theorem~\ref{thm:generalization}, an ID pruning to accuracy $\epsilon$ yields a compressed network that satisfies
$\widehat{\cR}_{ID} \leq  \epsilon^2 \|u\|_2^2 \| g(W^T X)\|_2^2 / n,$
where $X\in\R^{d\times n}$ is a matrix whose columns are the pruning data.
\end{lemma}


%; given a fixed $k$ this could be computed to assess the quality of an upper bound. 

%Finally, we bound p.  Since the network we are using has a subset of the hypothesis class of the full network, its p-dimension must be the same or smaller.  From [\megan{cite }], we have that:

%for some constant C.  

%Putting this all together, we get:  

%\begin{equation}
%    {\mathcal{R}}_p \leq \epsilon^2 \|u\| ^2 \| g(W^T x)\|^2 +(\|u^\top g(W^\top x) \| (1+2m))^2 ( \sqrt{\frac{2Cdm \log (dm)\log\frac{eN}{Cdm \log (dm)}}{N}}+ \sqrt{\frac{\log \frac{1}{\delta}}{2N}})
%\end{equation} 






% OLD
%To demonstrate why we expect that this method will be effective, we created a simple synthetic data set for which we know the form of a relatively minimal representation.  More details about the data set and experiment can be found in the supplement. We see that the interpolative decomposition keeps a set of neurons which represents the underlying function slightly better than the original model, ignoring duplicates but taking into account differences in the bias.  Magnitude pruning keeps duplicate neurons and fails to find important information with the same number of neurons.    
%We can train an overparameterized single hidden layer network to perform well on this task, and given a good initialization scale of the parameters, the neurons do not need to move very far[cite].  
%A pruning method based on magnitude pruning will not necessarily recognize the pairs of neurons which can perform well on this task when we prune to very small network sizes, since these neurons may not have a large magnitude in the larger model.    
%
%In practice, we see that ID is able to select neurons which resemble a close to minimal representative network.  


\section{Convolutional and deep networks}
\label{sec:extendid}
\subsection{Convolution layers}
% \paragraph{Convolution layers}
%\label{sec:convid}


To prune convolution layers with the ID at the channel level we reshape the %order-4 
output tensor into a matrix where each column represents a single output channel.
After this transformation, the key idea is the same as in Section~\ref{sec:pruneID}.
%The weights of convolution layers are an order-4 tensor ($\operatorname{input channels} \times \operatorname{output channels} \times \operatorname{filter width} \times \operatorname{filter height}$).
%and their action on the activations is via a convolution.
%We first reshape the activation tensor matrix of shape $( \operatorname{filter width} \cdot \operatorname{filter height} \cdot \operatorname{input channels} ) \times \operatorname{output channels}$, and then use an interpolative decomposition to prune the output channels.
Consider a simple two layer convolution neural network defined as 
$%\begin{equation*}
    h_{\Conv}(x; W, U)
    =
    \Conv(U, g( \Conv(W, x))))
$, %\end{equation*}
%$h_{Conv}: \R^{q \times r \times s} \to \R^{q' \times r' \times s'}$
with the convolution-layer operator $\Conv$, weight tensors $W$ and $U$,
%such that the output channels of $W$ and the input channels of $U$ match,
and activation function $g$.
%Again we omit bias terms, though they may be readily incorporated.
%We omit specifying the kernel dimension, size, stride, padding, and dilation because they have no effect on the interpolative decomposition at the output channel level.
The kernel dimension, size, stride, padding, and dilation do not change the form of the ID at the output channel level.
Let $Z = g(\Conv(W, X))$ be the activation output of the first layer with unlabeled pruning data $X$, and define $\Reshape$ as the operator which reshapes a tensor into a matrix with the output channels as columns, i.e., $\Reshape(Z)\in\R^{n_i\times m_c}$ where $m_c$ is the number of channels and $n_i$ is the product of all other dimensions (e.g. in the case of a 2d convolution, $n_i$ would be $\text{width} \times \text{height} \times \text{number of examples}$). 

We now compute\footnote{When $n_i$ is large we can appeal to randomized ID algorithms~\cite{martinsson2011randomized}, or the TSQR~\cite{ballard2014tsqr}.
%if needed.
} a rank-$k$ ID
$\Reshape(Z) \approx \Reshape(Z)_{:,\I} T.$
%the reshaped activation output.
%Because the activation function $g$ and reshaping operator $\Reshape$ commute with the sub-selection operator
%\begin{equation*}
%    Z 
%    \approx
%\end{equation*}
%Define $\Reshape(Z)$ as the operator to reshape the order-4 tensor into an order-2 tensor with columns $\operatorname{output channels}$.
%The steps to prune this two layer convolution network are similar to the fully connected case.
%Compute a rank-$k(\epsilon)$ interpolative decomposition on the reshaped activation output
%$
%    \Reshape(Z) 
%    \approx
%    \Reshape(Z)_{:,\I} T
%$.
The activation function $g$ and reshape operator both commute with the sub-selection operator, so
\begin{align*}
    \Reshape(g(\Conv(W, X)))
    &\approx
    \Reshape(g(\Conv(W, X)))_{:,\I} T \\
    &=
    \Reshape(g(\Conv(W_{\I,\ldots}, X))) T,
\end{align*}
where $W_{\I,\ldots}$ denotes an indexing sub-selection of $W$ along its output-channel dimension. 
%$\I$.
Next, we need to propagate this $T$ into the next layer, which we can do with a ``matrix multiply'' by the next-layer's weights along its input channel dimension: we call this operation $\Matmul$.\footnote{If $T \in \R^{m \times n}$ and $U$ is a weight-tensor with $n$ input channels, then to compute $\Matmul(T,U)$ we: (1) reshape $U$ to be an $n \times p$ matrix for some $p$, (2) multiply the reshaped matrix by $T$, producing a $m \times p$ matrix, and (3) reshape the result back to a tensor with $m$ input channels and all other dimensions the same as $U$.}
With this, a little algebraic manipulation of our approximate equality above gives
\begin{equation}
    \Conv(U, g( \Conv(W, X)))) \\
    \approx 
    \Conv(\Matmul(T,U), g( \Conv(W_{\I,\ldots}, X)))),
\end{equation}
and so if we set $\widehat U = \Matmul(T,U)$ and $\widehat W = W_{\I,\ldots}$,
% Thus by selecting channels $\widehat{W} = W_{\I},$ where $W_{\I}$ is the reshaping of $\Reshape(Z)_{:,\I}$ back into an order 4 tensor, and propagating $T$ into the next layer as $\widehat{U} = \Matmul(T,U)$ 
we can preserve the action of the two layer network with fewer channels as
$%\begin{equation*}
%\label{eq:convID}
    h_{\Conv}(x; W, U)
    \approx
    h_{\Conv}(x; \widehat{W}, \widehat{U}).
$ %\end{equation*}
This gives us a recipe for pruning convolution layers analogous to (\ref{eqn:PruneApprox}).
This recipe can be directly applied to a composition of a convolution layer followed by a pooling layer~\cite{goodfellow2016deep} (or any other layer that acts independently by channel) by treating the conv layer/ pooling layer pair as a single convolution layer with a ``fancy'' activation function $g$: we just run the ID on the output post-pooling, and use that to sub-select the convolution layer's weights.
Flatten layers, for connecting to FC layers, are equally straightforward.

%It is straightforward to extend this to arbitrary compositions of linear and convolutional layers: for more details, see Algorithm~\ref{alg:deepID}.

% Here, $\Matmul(\ )$ applies a batch matrix multiplication between $T$ and $\Reshape(U)$ followed by reshaping to the original order-4 tensor dimensions.\footnote{
% %So far we have discussed how to prune two fully connected layers (FC-FC), and two convolution layers (Conv-Conv).
% For a convolution layer followed by a fully connected layer, % (Conv-FC),
% the details of the ID and channel selection remain the same.
% To propagate $T$, 
% %define the propagation function
% let $\Matmul(T, W_2) = (T \otimes I_a) W_2$ with $a = \text{input channels} * \text{filter width} * \text{filter height}$ and $\otimes$ the Kronecker product.
% }
%\begin{equation}
%    Z 
%    \approx
%    T^\top g(\Conv(W_{1(:,\I,:,:)}, X))
%\end{equation}

%\jerry{How much detail for Conv algebra: full details a bit messy.}


%\begin{algorithm}[t]
%\SetAlgoLined
%\DontPrintSemicolon
%\KwIn{
%Conv layer $f_{Conv}(x;W_1)$,
%next layer $f_{2}(x;W_2)$,
%%intermediate layer $g(x; w_3)$ (batch norm, pooling),
%pruning data $X_{p} \in \R^{d \times n_{p}}$,
%rank-$k(\epsilon)$
%}
%\KwOut{
%pruned layers
%$\widehat{f}_{Conv}(x;\widehat{W}_1)$,
%$\widehat{f}_{2}(x;\widehat{W}_2)$,
%%$\widehat{g}(x;\widehat{w}_3)$
%}
%
%%\tcc{apply batch norm or pool layer if it follows $f_{Conv1}$}
%$Z \gets f_{Conv1}(X_{p},W_1)$\;
%$Z \gets Z.\operatorname{reshape}(\operatorname{output channels}, \operatorname{filter width} \cdot \operatorname{filter height} \cdot \operatorname{input channels})$\;
%$Z_{:,\I}, T \gets \ID(Z, k(\epsilon))$\;
%$\widehat{W}_1 \gets (W_1)_{(:,\I,:,:)}$
%\tcp{select output channels via ID}
%$\widehat{W}_2 \gets \operatorname{matmul}(T, W_2)$
%\tcp{propagate interpolation matrix to next layer}
%%\tcp{depends on if next layer is FC or Conv}
%%\todo{presentation} \;
%%$\widehat{W}_2 \gets W_2.\operatorname{reshape}(\operatorname{output channels}, \operatorname{filter width}, \operatorname{filter height},  \operatorname{input channels})$\;
%%$\widehat{W}_2 \gets \operatorname{batch broadcast matmul}(T, \widehat{W}_2)$\;
%%$\widehat{W}_2 \gets W_2.\operatorname{reshape}(\operatorname{input channels}, \operatorname{output channels}, \operatorname{filter width}, \operatorname{filter height})$\;
%\caption{ID Pruning a Conv Layer\jerry{might get rid of}}
%\label{alg:convID}
%\end{algorithm}




\subsection{Deep networks}
% \paragraph{Deep networks}
\label{sec:deepid}

\begin{algorithm}[t]{\small
%\SetAlgoLined
%\DontPrintSemicolon
\caption{Pruning a multilayer network with interpolative decompositions}
\begin{algorithmic}[1]
\label{alg:deepID}
%\INPUT
\REQUIRE
Neural net $h(x; W^{(1)},\ldots,W^{(L)})$,
layers to not prune $S \subset [L]$,
pruning set $X$,
%prune mode $\in \{\operatorname{acc},\operatorname{frac}\}$,
pruning fraction $\alpha$
%Ordered list 
%index list $L$ 
%of layers to be pruned $\{(l_{i}, l_{i+1})\}_{i \in L}$ and activation outputs $\{Z_i\}_{i \in L}$, \\
%prune mode $\in \{\operatorname{frac}, \operatorname{acc}\}$, accuracy $\epsilon$
%\OUTPUT
\ENSURE
Pruned network $h(x; \widehat{W}^{(1)},\ldots,\widehat{W}^{(L)})$
%Pruned layers $\{(\hat{l}_{i}, \hat{l}_{i+1})\}_{i \in L}$
\vspace{0.5em}
%Absorb batch norm layers into their preceding layer\;
%$\operatorname{acc}$: set $k$ to achieve desired accuracy $\epsilon$ or $\operatorname{frac}$: as $\epsilon$ fraction of neurons \;
\STATE $T^{(0)} \gets I$ \;
\FOR{$l \in \{1 \dots L\}$}
\STATE $Z \gets h_{1:l}(X; W^{(1)}, \dots, W^{(l)})$
\COMMENT{layer l activations}
%\COMMENT{compute activations of layer l}
\IF{layer $l$ is a FC layer}
\STATE $(\I, T^{(l)}) \gets \operatorname{ID}(Z^T; \alpha) \textbf{ if } l \notin S \textbf{ else } (:, I)$ \;
%prune $\alpha\%$ of neurons with ID of $Z^\top$: $\I, T$\;
%compute rank-$k$ ID of $Z^\top$: $\I, T$\;
\STATE $\widehat{W}^{(l)} \gets T^{(l-1)} W^{(l)}_{:,\I}$
\COMMENT{sub-select neurons, multiply T of prev layer's ID}
%\tcp{select neurons in current layer}
%$\widehat{W}^{(l+1)} \gets T \widehat{W}^{(l+1)}$
%\;
%\tcp{propagate T to next layer}
\ELSIF{layer l is a Conv layer (or Conv+Pool)}
\STATE $(\I, T^{(l)}) \gets \operatorname{ID}(\Reshape(Z); \alpha) \textbf{ if } l \notin S \textbf{ else } (:, I)$ \;
% prune $\alpha\%$ of channels with ID of $\Reshape(Z)$: $\I, T$\;
%compute rank-$k$ ID of $\Reshape(Z)$: $\I, T$\;
\STATE $\widehat{W}^{(l)} \gets \Matmul(T^{(l-1)}, W^{(l)}_{\I,\ldots})$
%\;
\COMMENT{select channels; multiply T} %in current layer
% $\widehat{W}^{(l+1)} \gets \Matmul(T, \widehat{W}^{(l+1)})$
%\tcp{propagate T to next layer}
%\tcp{depends if next layer is FC or Conv}
\ELSIF{layer l is a Flatten layer}
\STATE $T^{(l)} \gets T^{(l-1)} \otimes I \,\,$ 
\COMMENT{expand T to have the expected size}
\ENDIF
\ENDFOR
\end{algorithmic}
%%Specify direction\;
%%How to compute Z\;
%\caption{ID pruning a multi-layer neural network}
}\end{algorithm}


%% OLD Version
%%\begin{algorithm}[t]
%%\caption{ID pruning a multi-layer neural network}
%%%\caption{Pruning a multilayer network with interpolative decompositions}
%%\label{alg:deepID_acc}
%%\begin{algorithmic}
%%\INPUT
%%Neural network $h(x; W^{(1)},W^{(2)},\dots,W^{(L)})$,
%%layers to not prune $S \subset [L]$,\\
%%%layers to skip during pruning $S \subset [L]$,\\
%%pruning set $X$,
%%%prune mode $\in \{\operatorname{acc},\operatorname{frac}\}$,
%%pruning proportion $\alpha$
%%%Ordered list 
%%%index list $L$ 
%%%of layers to be pruned $\{(l_{i}, l_{i+1})\}_{i \in L}$ and activation outputs $\{Z_i\}_{i \in L}$, \\
%%%prune mode $\in \{\operatorname{frac}, \operatorname{acc}\}$, accuracy $\epsilon$
%%\OUTPUT
%%Pruned network $h(x; \widehat{W}^{(1)},\widehat{W}^{(2)},\dots,\widehat{W}^{(L)})$
%%%Pruned layers $\{(\hat{l}_{i}, \hat{l}_{i+1})\}_{i \in L}$
%%\vspace{0.5em}
%%\hrule
%%\vspace{0.5em}
%%%Absorb batch norm layers into their preceding layer\;
%%%$\operatorname{acc}$: set $k$ to achieve desired accuracy $\epsilon$ or $\operatorname{frac}$: as $\epsilon$ fraction of neurons \;
%%\FOR{$l \in \{1,\dots L\}$}
%%\STATE $\widehat{W}^{(l)} \gets W^{(l)}$
%%\FOR{$l \in \{1 \dots L\} \setminus S$}
%%\IF{layer l is a FC layer}
%%\STATE $Z \gets h_{1:l}(X; W^{(1)}, \dots, W^{(l)})$
%%\COMMENT{output of layer l}
%%\STATE prune $\alpha\%$ of neurons with ID of $Z^\top$: $\I, T$\;
%%%compute rank-$k$ ID of $Z^\top$: $\I, T$\;
%%\STATE $\widehat{W}^{(l)} \gets \widehat{W}^{(l)}_{:,\I}$
%%\;
%%%\tcp{select neurons in current layer}
%%\STATE $\widehat{W}^{(l+1)} \gets T \widehat{W}^{(l+1)}$
%%%\tcp{propagate T to next layer}
%%\ELSIF{layer l is a Conv layer}
%%\STATE $Z \gets h_{1:l}(X; W^{(1)}, \dots, W^{(l)})$\;
%%\STATE prune $\alpha\%$ of channels with ID of $\Reshape(Z)$: $\I, T$\;
%%%compute rank-$k$ ID of $\Reshape(Z)$: $\I, T$\;
%%$\widehat{W}^{(l)} \gets \widehat{W}^{(l)}_{(:,\I,:,:)}$
%%%\;
%%\COMMENT{select channels} %in current layer
%%\STATE $\widehat{W}^{(l+1)} \gets \Matmul(T, \widehat{W}^{(l+1)})$
%%\;
%%\ENDIF
%%%\tcp{propagate T to next layer}
%%%\tcp{depends if next layer is FC or Conv}
%%\ENDFOR
%%\ENDFOR
%%\end{algorithmic}
%%%\SetAlgoLined
%%%\DontPrintSemicolon
%%%%Specify direction\;
%%%%How to compute Z\;
%%\end{algorithm}



The ID pruning primitives for fully connected and convolution layers can now be composed together to prune deep networks.
Algorithm~\ref{alg:deepID} specifies how we chain together the fully connected and convolution primitives to prune feedforward networks, for simplicity we assume for the moment we know the desired layer sizes.
%of arbitrary composition.
%In contrast most pruning methods act on a single layer at a time, and
%only modify the weights in the current layer, and 
%do not make any corrective changes to the next layers.
A multi-layer neural network is pruned from the beginning to the end, where the ID is used to approximate the outputs of the original network.
The ID pruning primitives sub-select neurons (or channels) in the current layer and propagate the corrective interpolation matrix to the next layer.
There are many ways one could prune a multi-layer network with these ID pruning primitives;
we selected the approach in Algorithm~\ref{alg:deepID} through empirical observations (though we do not assert that it is optimal).
% Discussion on the experiments which guided our design choices can be found in the supplement.
%The activation outputs are computed using a held out pruning set with either the original network, or the pruned network.
%For deep networks we found that it was crucial to use the activation outputs from the original network to mitigate harmful cascading approximation errors across the layers.
Note that as a pre-processing step before running Algorithm~\ref{alg:deepID}, batch normalization layers~\cite{batchnorm} should be absorbed into their preceding fully-connected or convolution layers, and dropout~\cite{srivastava2014dropout} layers should be removed.
% Pooling layers~\cite{goodfellow2016deep} operate at the neuron or channel level, and can be incorporated into computing $Z$ without affecting the interpolative decomposition.
% We also consider residual networks~\cite{resnet} which consist of blocks with two or more convolution layers and a skip connection: for such residual blocks we prune all but the last layer in the block.


\paragraph{Iterative Pruning}
While Algorithm~\ref{alg:deepID} is illustrative, in practice we would often like to be able to either specify a desired accuracy or choose layer sizes optimally for a desired compression ratio. Our approach allows us to accomplish this by iteratively selecting layers to compress. We introduce a score function for layers that is the ratio of the estimated relative error $\lvert r_{k+1,k+1} / r_{1,1}\rvert$ introduced by compressing a layer to the number of flops $f_l$ that would be cut if we pruned a layer $l$ to size $k$. We call this score $s_l(k)= \lvert r_{k+1,k+1} / r_{1,1}\rvert /f_l$ and it is heuristic for the compressability of each layer---lower scores imply a layer is easier to compress. However, different layers of the network are connected, and compressing a layer early in the network can effect how well later layers can be compressed.  
Therefore, we prune the network iteratively, measuring the score for each layer at a given pruning percentage (or step size), choosing the layer with the lowest score, pruning it, and then re-calculating the scores for the later layers. 
We repeat the process until the network reaches a desired compression, or until the network performance degrades unacceptably. 
% This method is particularly useful for sequential networks where the layer sizes were not chosen efficiently.
We refer to this method as Iterative ID, and refer to cutting a constant fraction of all neurons in each layer as Constant Fraction ID.  
For full details see Appendix~\ref{app:sec:iterativeID} and Algorithm~\ref{alg:deepIDIter}.
%\todo[inline]{Put IDIter algorithm psuedo code somewhere}

%\begin{algorithm}[t]{\small
%%\SetAlgoLined
%%\DontPrintSemicolon
%\KwIn{
%Neural net $h(x; W^{(1)},\ldots,W^{(L)})$,
%layers to not prune $S \subset [L]$,
%pruning set $X$,
%%prune mode $\in \{\operatorname{acc},\operatorname{frac}\}$,
%pruning fraction $\alpha$
%%Ordered list 
%%index list $L$ 
%%of layers to be pruned $\{(l_{i}, l_{i+1})\}_{i \in L}$ and activation outputs $\{Z_i\}_{i \in L}$, \\
%%prune mode $\in \{\operatorname{frac}, \operatorname{acc}\}$, accuracy $\epsilon$
%}
%\KwOut{
%Pruned network $h(x; \widehat{W}^{(1)},\ldots,\widehat{W}^{(L)})$
%%Pruned layers $\{(\hat{l}_{i}, \hat{l}_{i+1})\}_{i \in L}$
%}
%\vspace{0.5em}
%%Absorb batch norm layers into their preceding layer\;
%%$\operatorname{acc}$: set $k$ to achieve desired accuracy $\epsilon$ or $\operatorname{frac}$: as $\epsilon$ fraction of neurons \;
%\textbf{set} $T^{(0)} \gets I$ \;
%\For{$l \in \{1 \dots L\}$}{
%$Z \gets h_{1:l}(X; W^{(1)}, \dots, W^{(l)})$
%\hfill\tcp{compute activations of layer l}
%\uIf{layer $l$ is a FC layer}{
%$(\I, T^{(l)}) \gets \operatorname{InterpolativeDecomposition}(Z^T; \alpha) \textbf{ if } l \notin S \textbf{ else } (:, I)$ \;
%%prune $\alpha\%$ of neurons with ID of $Z^\top$: $\I, T$\;
%%compute rank-$k$ ID of $Z^\top$: $\I, T$\;
%$\widehat{W}^{(l)} \gets T^{(l-1)} W^{(l)}_{:,\I}$
%\hfill\tcp{sub-select neurons, multiply T of prev layer's ID}
%%\tcp{select neurons in current layer}
%%$\widehat{W}^{(l+1)} \gets T \widehat{W}^{(l+1)}$
%%\;
%%\tcp{propagate T to next layer}
%}
%\uElseIf{layer l is a Conv layer (or Conv+Pool)}{
%$(\I, T^{(l)}) \gets \operatorname{InterpolativeDecomposition}(\Reshape(Z); \alpha) \textbf{ if } l \notin S \textbf{ else } (:, I)$ \;
%% prune $\alpha\%$ of channels with ID of $\Reshape(Z)$: $\I, T$\;
%%compute rank-$k$ ID of $\Reshape(Z)$: $\I, T$\;
%$\widehat{W}^{(l)} \gets \Matmul(T^{(l-1)}, W^{(l)}_{\I,\ldots})$
%%\;
%\hfill\tcp{select channels; multiply T} %in current layer
%% $\widehat{W}^{(l+1)} \gets \Matmul(T, \widehat{W}^{(l+1)})$
%%\tcp{propagate T to next layer}
%%\tcp{depends if next layer is FC or Conv}
%}
%\uElseIf{layer l is a Flatten layer}{
%    $T^{(l)} \gets T^{(l-1)} \otimes I \,\,$ \hfill\tcp{expand T to have the expected size}
%}
%}
%%Specify direction\;
%%How to compute Z\;
%\caption{ID pruning a multi-layer neural network}
%%\caption{Pruning a multilayer network with interpolative decompositions}
%\label{alg:deepID}
%}\end{algorithm}
%
%
%\begin{algorithm}[t]
%\SetAlgoLined
%\DontPrintSemicolon
%\KwIn{
%Neural network $h(x; W^{(1)},W^{(2)},\dots,W^{(L)})$,
%layers to skip during pruning $S \subset [L]$,\\
%pruning set $X$,
%%prune mode $\in \{\operatorname{acc},\operatorname{frac}\}$,
%pruning proportion $\alpha$
%%Ordered list 
%%index list $L$ 
%%of layers to be pruned $\{(l_{i}, l_{i+1})\}_{i \in L}$ and activation outputs $\{Z_i\}_{i \in L}$, \\
%%prune mode $\in \{\operatorname{frac}, \operatorname{acc}\}$, accuracy $\epsilon$
%}
%\KwOut{
%Pruned network $h(x; \widehat{W}^{(1)},\widehat{W}^{(2)},\dots,\widehat{W}^{(L)})$
%%Pruned layers $\{(\hat{l}_{i}, \hat{l}_{i+1})\}_{i \in L}$
%}
%\vspace{0.5em}
%\hrule
%\vspace{0.5em}
%%Absorb batch norm layers into their preceding layer\;
%%$\operatorname{acc}$: set $k$ to achieve desired accuracy $\epsilon$ or $\operatorname{frac}$: as $\epsilon$ fraction of neurons \;
%\lFor{$l \in \{1,\dots L\}$}{
%$\widehat{W}^{(l)} \gets W^{(l)}$
%}
%\For{$l \in \{1 \dots L\} \setminus S$}{
%\uIf{layer l is a FC layer}{
%$Z \gets h_{1:l}(X; W^{(1)}, \dots, W^{(l)})$
%\tcp{output of layer l}
%prune $\alpha\%$ of neurons with ID of $Z^\top$: $\I, T$\;
%%compute rank-$k$ ID of $Z^\top$: $\I, T$\;
%$\widehat{W}^{(l)} \gets \widehat{W}^{(l)}_{:,\I}$
%\;
%%\tcp{select neurons in current layer}
%$\widehat{W}^{(l+1)} \gets T \widehat{W}^{(l+1)}$
%\;
%%\tcp{propagate T to next layer}
%}
%\uElseIf{layer l is a Conv layer}{
%$Z \gets h_{1:l}(X; W^{(1)}, \dots, W^{(l)})$\;
%prune $\alpha\%$ of channels with ID of $\Reshape(Z)$: $\I, T$\;
%%compute rank-$k$ ID of $\Reshape(Z)$: $\I, T$\;
%$\widehat{W}^{(l)} \gets \widehat{W}^{(l)}_{(:,\I,:,:)}$
%%\;
%\tcp{select channels} %in current layer
%$\widehat{W}^{(l+1)} \gets \Matmul(T, \widehat{W}^{(l+1)})$
%\;
%%\tcp{propagate T to next layer}
%%\tcp{depends if next layer is FC or Conv}
%}
%}
%%Specify direction\;
%%How to compute Z\;
%\caption{ID pruning a multi-layer neural network}
%%\caption{Pruning a multilayer network with interpolative decompositions}
%\label{alg:deepID}
%\end{algorithm}

%\section{Contextualizing our work}
%\label{sec:context}
%% \begin{table}[h!]
% \centering
% \begin{tabular}{|c c c c| c c c|} 
%  \hline
%  & & & & \multicolumn{3}{c|}{Compression type} \\ [0.5ex]
%  & & & & \multirow{3}{*}{Sparse} & \multicolumn{2}{|c|}{Dense} \\
%  & & & & & \multicolumn{2}{|c|}{Preserves net structure?} \\ 
%  & & & & & \multicolumn{1}{|c}{Yes} & \multicolumn{1}{|c|}{No} \\ [0.5ex] %extra space to next row
%  \hline%\hline
%  \multirow{3}{*}{\rotatebox[origin=c]{90}{Correction}} & & & 
%  None & Cat. A & Cat. B & Cat. C \\ [1ex]
%  \cline{2-4}
%  & \multicolumn{2}{c}{Local}
%  %& \multirow{2}{*}{\rotatebox[origin=c]{90}{Local}} 
%  %& \multirow{2}{*}{\rotatebox[origin=c]{90}{FT?}}
%  & No & Cat. D & Cat. E & Cat. F \\ [1ex]
%  \cline{4-4}
%  & \multicolumn{2}{c}{FT?}
%  & Yes & Cat. G & Cat. H & Cat. I\\ [1ex]
%  \hline
% % \multirow{6}{*}{\rotatebox[origin=c]{90}{Correction}} & 
% % \multirow{2}{*}{None} & weight & channel & low- \\
% % & & prune & prune & rank \\ [1ex]
% % \cline{2-5}
% % & Yes, w/out &\multirow{2}{*}{N/A} & \multirow{2}{*}{ID} & \multirow{2}{*}{N/A}\\
% % & locFT &  & & \\ [1ex]
% % \cline{2-5}
% % & Yes, w/ & \multirow{2}{*}{WP+} & \multirow{2}{*}{CP+} & \multirow{2}{*}{LR+}\\
% % & locFT &  & & \\ [1ex]
% % \hline
% \end{tabular}
% \caption{Taxonomy of parameter compressing methods via 2 axes: the type of compression (how parameters are removed), and what type of correction is taken to improve the network after removing parameters.
% Often, local fine tuning is interleaved into compression methods.
% }
% \label{tab:taxonomy}
% \end{table}


%Relative to existing techniques, our use of the interpolative decomposition provides several key advantages. 
To facilitate a careful discussion of how our method fits within the current literature, Table~\ref{tab:taxonomy} provides a taxonomy of parameter space compression methods.
%Our formulation of compression as preserving the per-example labels necessitates us to focus on the paradigm of compressing a pre-trained model.
We focus on preserving the per-example labels of a pre-trained model. 
Thus we ignore methods which do not take a pre-trained model as input.
The ID~(Cat. E) combines benefits of both low-rank and channel pruning methods, while also incorporating a parameter correction done without additional local fine tuning.
Unlike low-rank methods~(Cat. F), the ID preserves the computational structure of the network.
This allows us to trivially compose the ID with other compression methods, achieving model recovery comparable to that of global fine tuning.
And unlike channel pruning methods~(Cat. B), the ID can fully recover the original model at minimal FLOPs reduction, without any fine tuning. 
%Low-rank methods modify the network structure by decomposing the weight matrices, and do not directly incorporate a corrective step.
%Channel pruning methods retain the network structure, but often poorly preserve model performance without  fine tuning (whether local or global).
%A key feature of our method is that it retains the network structure---just reducing the width---and, therefore, is able to leverage the computational benefits of specific network architectures.




%We can also prune to a fixed compression level and accept the resulting degradation in accuracy.  
%While this process requires additional data, it can be unlabeled data as our process does not require ``ground truth'' labels; in fact, it effectively uses pseudo-labels generated by the trained model. 
% Jerry removed 1/25/22
%Another key feature of our method is that it retains the network structure---just reducing the width---and, therefore, is able to leverage the computational benefits of specific network architectures.

In our taxonomy, we choose to separate any fine tuning (i.e., optimization) steps from the parameter reduction step.
What we call ``fine tuning'' can take two primary forms.
%: either locally interleaved into the parameter reduction, or performed end-to-end globally on the reduced model.
We define ``global'' fine tuning as optimizing the end-to-end loss.
``Local'' fine tuning is where per-layer structures are optimized.
It is valuable to compare compression methods that use similar a type and amount of fine tuning, as well as to compare before and after its application.
%This allows us to better evaluate whether accuracies are achieved due to a fine-tuning technique or due to the compression method itself.
%We do not interleave local fine tuning with the ID in order to better understand the contribution of our compression method.
%However, we recognize that local fine tuning is a powerful method. 
%Because the interpolative decomposition preserves the model structure, we can compose other methods that use local fine tuning on top of it.  
%We demonstrate the efficacy of this approach in section \ref{sec:experiments}.  


\section{Evaluating compression beyond accuracy}
\label{sec:correlation}
An important benefit of our approach is that we are actually able to preserve the original model's predictions better than other methods.
Traditional pruning methods typically do a poor job at preserving the original predictions, due to their heavy reliance on fine tuning that effectively retrains the model.
Here we explain why we might care about preserving per-example predictions beyond top-line accuracy.
We argue that in many situations compression methods must well-approximate the original model, and that accuracy is a poor metric for this use case.

% What does it mean to ``compress'' a model?
% There are two criteria. 
% First, the compressed model must be smaller than the original, e.g. in memory of in inference runtime.
% %We have no qualm with how the reduction in size is measured.
% Second, the compressed model must well-approximate the original.
% We argue that accuracy is a poor metric for measuring if one model well-approximates another.

Consider a pretrained model $M$, a resulting ``compressed'' model $M_C$, and an evaluation set $(X,Y)$.
Accuracy measures the similarity between the true labels $Y$ and the predicted labels from the compressed model $M_C(X)$.
This metric does not directly compare the original model $M$ and compressed model $M_C$.
As we have seen in Section~\ref{sec:intro}, a compressed model can recover the accuracy of the original model, but still differ widely on the predictions.
Instead our model correlation measures the similarity between the original model's predictions $M(X)$ and the compressed model's predictions $M_C(X)$.
One can think of this metric as an ``accuracy to the original predictions'', instead of an accuracy to the true labels.

We propose model correlation as the percent of test example predictions two models agree on---details of this metric are discussed in Appendix~\ref{app:sec:correlation}.
Model correlation is a general metric to measure the similarity between the learned functions 
%decision boundaries
of two models.
% However we propose model correlation as a generic metric to measure preservation between a pre-trained and compressed model.
Our claim is that by better preserving a model's per-example decisions, we can better preserve special properties of the model.
In the following section we provide experimental evidence for this claim.
Models are now often trained to have properties that go beyond test accuracy---for example robustness to adversarial attacks, sub-class classification accuracy, fairness, etc., and this measure of model correlation is likely to correlate with many of these criteria.
% Fairness and robustness properties are two such specific examples.
% Better preserving specialize fairness or adversarial robustness properties is a corollary of better preserving the original model's learned function.
Note that we do not believe preserving per-example decisions ``boosts'' any of these properties, we are simply preserving properties of the baseline model.







% We propose a set of evaluation metrics for pruning methods that goes beyond simple post-fine tuning test accuracy. The first metric we define is correlation between two models, defined as the percent of test example predictions the two models agree on---details of this metric are discussed in Appendix~\ref{app:sec:correlation}.  We begin with a pretrained model, $M$, and compress it to create a model $M_c$.  We define model correlation as an aggregate measure of how well $M_c$ preserves the per-example decisions of $M$ on the test set.  This captures information about both examples which $M$  correctly and incorrectly classified.  Models are now often trained to have properties that go beyond test accuracy---for example robustness to adversarial attacks, sub-class classification accuracy, fairness, etc. On a practical level, no single number could perfectly capture the preservation on every possible metric. Our claim is that by better preserving per-example decisions, we may better preserve other special properties of the model. In order to demonstrate this, we show how our method can be used to prune a network while maintaining sub-class accuracy when neither the pruning nor fine-tuning methods have access to data from one of the classes in Appendix \ref{sec:sensitivity}.




\section{Experiments}
\label{sec:experiments}

\section{Experiments}\label{sec:experiments}
We validate our approach using multiple datasets containing real-life data from the fields of criminal risk assessment, credit, lending, and college admissions. In each of the datasets we select a binary feature and treat it as the protected attribute (e.g., race or gender), which is the feature we require our trained classifier to behave fairly upon. Our proposed method performs well on all of these datasets, succeeding in removing unfairness almost entirely, at a very modest price in terms of accuracy.


\begin{table*}[h]
\centering
\resizebox{\textwidth}{!}{
\def\arraystretch{1.2}

\begin{tabular}{c c c | c | c | c || c | c | c || c | c | c |}

\cline{4-12}
&&&
\multicolumn{9}{ c| }{\textbf{COMPAS Dataset}}
\\ \cline{4-12}
&&&
\multicolumn{3}{ c|| }{\textbf{FPR Considerations}}&
\multicolumn{3}{ c|| }{\textbf{FNR Considerations}}&
\multicolumn{3}{ c| }{\textbf{Both Considerations}}
\\ \cline{4-12}
&&&
 $\mathbf{Acc.}$ &  $\mathbf{D_{FPR}}$ &  $\mathbf{D_{FNR}}$ &  $\mathbf{Acc.}$ &  $\mathbf{D_{FPR}}$ &  $\mathbf{D_{FNR}}$ &  $\mathbf{Acc.}$ &  $\mathbf{D_{FPR}}$ &  $\mathbf{D_{FNR}}$
\\  \cline{4-12}
\vspace*{-0.5ex}
\\ \cline{1-2} \cline{4-12}
\multicolumn{1}{ |c  }{} &
\multicolumn{1}{ c|  }{  \textbf{Our Method (AVD Penalizers)}}  &&
$\mathbf{0.660}$    &  $\mathbf{0.01}$  &  $0.04$ &
$\mathbf{0.653}$    &  $0.02$   &  $\mathbf{0.04}$ &
$\mathbf{0.654}$    &  $\mathbf{0.02}$  &  $\mathbf{0.04}$
\\ \cline{1-2} \cline{4-12}
\multicolumn{1}{ |c  }{} &
\multicolumn{1}{ c|  }{  \textbf{Our Method (SD Penalizers)}}  &&
$\mathbf{0.664}$    &  $\mathbf{0.02}$  &  $0.09$ &
$\mathbf{0.661}$    &  $0.05$   &  $\mathbf{0.03}$ &
$\mathbf{0.661}$    &  $\mathbf{0.02}$  &  $\mathbf{0.03}$
\\ \cline{1-2} \cline{4-12}
\multicolumn{1}{ |c  }{} &
\multicolumn{1}{ c|  }{  Zafar et al.~(\citeyear{disparatemistreatment})}  &&
$0.660$    &   $0.06$    &   $0.14$  &
$0.662$    &   $0.03$    &   $0.10$  &
$0.661$    &   $0.03$    &   $0.11$
\\ \cline{1-2} \cline{4-12}
\multicolumn{1}{ |c  }{} &
\multicolumn{1}{ c|  }{  Zafar et al. Baseline~(\citeyear{disparatemistreatment})}  &&
$0.643$    &   $0.03$    &   $0.11$  &
$0.660$    &   $0.00$    &   $0.07$  &
$0.660$    &   $0.01$    &   $0.09$
\\ \cline{1-2} \cline{4-12}
\multicolumn{1}{ |c  }{} &
\multicolumn{1}{ c|  }{  Hardt et al.~(\citeyear{hardt})}  &&
$0.659$    &  $0.02$    &   $0.08$  &
$0.653$    &  $0.06$   &    $0.01$  &
$0.645$    &  $0.01$   &    $0.01$
\\ \cline{1-2} \cline{4-12}
\multicolumn{1}{ |c  }{} &
\multicolumn{1}{ c|  }{  \textbf{Vanilla Regularized Logistic Regression}}  &&
$\mathbf{0.672}$    &   $\mathbf{0.20}$    &   $\mathbf{0.30}$  &
$\mathbf{0.672}$    &   $\mathbf{0.20}$    &   $\mathbf{0.30}$  &
$\mathbf{0.672}$    &   $\mathbf{0.20}$    &   $\mathbf{0.30}$
\\ \cline{1-2} \cline{4-12}
\end{tabular}
}
\vspace{3mm}
\caption{Performance comparison on the COMPAS dataset. For the approaches in bold -- Accuracy, FPR difference and FNR difference are evaluated on the test set, averaging over five runs and using a 70-30 training/test split. The performance of the remaining three approaches is stated as reported in Zafar et al.~(\citeyear{disparatemistreatment}).} \label{table:comparison_results}
\end{table*}



\begin{figure*}[b]
  \includegraphics[scale=0.6]{compas0-400.png}
  \caption{COMPAS Dataset. Accuracy, FPR difference ($\mathbf{D_{FPR}}$), and FNR difference ($\mathbf{D_{FNR}}$) (all evaluated on the test set) of the learned classifier, as a function of the weight $c=c_1 = c_2 \geq 0$ placed on the fairness penalizer terms. On the left we use the Absolute Value Difference (AVD) penalizer, and the Squared Difference (SD) penalizer on the right, both as presented in Section~\ref{regularization}. ``Relaxed FPR/FNR Diff.'' plots the value of the relevant penalization term.} %In this particular run, parameters chosen for the absolute value relaxation were: $c=80, q_c=60$, and for the squared relaxation: $c=220, q_c=30$.}
  \label{fig:compas}
\end{figure*}


\subsection{Implementation}
\textbf{Our method} 
%We instantiate our method in the following way: Given dataset $Q$, we split it randomly into a training set $S$ (which we will use for learning) and a test set $T$ (which we will only use for reporting performance). 
For the purpose of comparison with  Zafar et al.~(\citeyear{disparatemistreatment}) and Hardt et al.~\cite{hardt} on the COMPAS data, we use a parameter $c$ to induce three possible combinations of weights on the FPR and FNR penalization terms: $c = c_1$ and $c_2 = 0$; $c_1 = 0$ and $c = c_2$; and $c = c_1 = c_2$. For the other three datasets, we consider only $c = c_1 = c_2$.\footnote{The reason for varying the values of $c$ in the training phase is since we shifted to a proxy problem, in which we rely on the distance from the decision boundary rather the actual classifications. 
%Our hope is that there is no need for a worst-case cross validation between all of the combinations of $c_1, c_2, c_3$, and that the training scheme we propose is sufficient. 
It is possible, of course, that even better results are attainable using our scheme with other combinations of $c_1, c_2$, and $q$.} To explore the accuracy/fairness trade-off curve for the relaxed optimization problem~(\ref{eq:2}), we train for different values of $c$, starting at $c=0$ (which is just standard logistic regression), and growing gradually.



Given a dataset $Q$ and fixing a $d_1, d_2 \in \{0, 1\}$ of interest, we use the following training scheme:
\begin{enumerate}
\item Split $Q$ at random into training set $S$ and test set $T$.
\item For each $c$, perform cross-validation on $S$ to select the corresponding best value $q_c$ for the regularization parameter.
\item For each $(c,q_c)$, let $\theta_c = \argmin\limits_{\theta} \text{Proxy}(\theta;S,c,c,q_c)$.
\item Select $\theta^* \in \argmin\limits_{\theta_c} \text{Objective}(\theta_c;S,d_1,d_2)$.
\item Evaluate performance using $\theta^*$ on test set $T$.
\end{enumerate}
We report the average of five such runs, each with a fresh training-test split.




%We instantiate our method by solving the relaxed optimization problem~(\ref{eq:2}), in place of the original, non-convex problem~(\ref{eq:1}).  
%We test our approach with three different combinations of weights on the penalization terms:
%\katrina{What are the $d$, and how are they related to the $c$s?}
%\begin{enumerate}
%\item FPR considerations only: $d_1 = 1, d_2 = 0$.
%\item FNR considerations only: $d_1 = 0, d_2 = 1$.
%\item Both FPR, FNR considerations, assigned similar significance: $d_1 = 1, d_2 = 1$.
%\end{enumerate}
%One could, of course, pick any other combination of the FPR and FNR penalty weights.

%\katrina{I don't understand how the below is distinct from the list above}
%Learning is done by training the parameters of a logistic regressor to solve~\ref{eq:2}, while picking the value of $c_1, %c_2$ as the following:
%\begin{enumerate}
%\item FPR considerations only: $c_1 = c \geq 0$, $c_2 = 0$.
%\item FNR considerations only: $c_1 = 0$, $c_2 = c \geq 0$.
%\item Both FPR, FNR considerations, assigned similar significance: $c_1 = c_2 = c \geq 0$
%\end{enumerate}



% We then cross-validate to pick the best $c_3$ (the weight on the standard $\ell_2$-regularization term) given $c$.\footnote{The reason for varying the values of $c$ in the training phase is since we shifted to a proxy problem, in which we rely on the distance from the decision boundary rather the actual classifications. 
%Our hope is that there is no need for a worst-case cross validation between all of the combinations of $c_1, c_2, c_3$, and that the training scheme we propose is sufficient. 
%It is possible, of course, that even better results are attainable using our scheme with other combinations of $c_1, c_2, c_3$.} For each such combination, we report results as the averages of multiple \katrina{how many?} different runs, each time splitting data randomly into training and test sets.
%\yahav{We need to shorten this description.}

We solve the relaxed convex optimization problem using the CVXPY solver. Due to stability issues with large training sets, we use a train/test split of 30-70 on the larger datasets, rather than 70-30 as on the COMPAS dataset\footnote{The code implementing our method can be found at https://github.com/jjgold012/lab-project-fairness}.

%
%
%We then report the results (as evaluated on the test set) attained by a regressor $\theta \in \mathbb{R}^d$ that minimizes (on the training set $S$) a weighted combination of the $0$-$1$ loss and the differences in FPR and FNR across populations:
%\begin{equation*}
%\begin{aligned}
%&\underset{\theta}{\text{argmin}}
%& & L_{S}^{0\text{-}1}(\theta) \\
%&&& + d_1|FPR_{A=0}(\theta;S)-FPR_{A=1}(\theta;S)| \\
%&&& + d_2|FNR_{A=0}(\theta;S)-FNR_{A=1}(\theta;S)|
%\end{aligned}
%\end{equation*}
%
%\katrina{What is $d_1$ vs. $c_1$ etc.?}



%For classification, we decided use a standard cut-off threshold of $c=0.5$. There are of course, further possible interactions between the FPR, FNR considerations, and picking a certain cut-off level. These are not straightforward, since  these interactions are data-specific. 



%allows for flexibility in picking the values of $c_1, c_2$, which reflect the significance we wish to place on the objectives of achieving accuracy, equal FPR, and equal FNR. As for $c_3$, we will want to find the value of it that achieves the best results, for any combined objective of accuracy and fairness defined by a specific selection of $c_1,c_2$. Therefore, given a specific selection of $c_1, c_2$, we apply cross-validation to select the value of $c_3$. 




We briefly describe the other algorithmic approaches to which we compare:\\
\textbf{Zafar et al.}~(\citeyear{disparatemistreatment}) performs optimization by considering a proxy for the bias: the covariance between the samples' sensitive attributes and the signed distance between the feature vectors of misclassified users and the classifier decision boundary.\\
\textbf{Zafar et al. Baseline}~(\citeyear{disparatemistreatment}) tries to enforce equal FP/FN rates on the different groups by introducing different penalties for misclassified data points with different sensitive attribute values during the training phase.\\
\textbf{Hardt et al.}~(\citeyear{hardt}) performs post-processing on a standard trained (unfair) logistic regressor, picking different decision thresholds for different groups, and possibly adding randomization.


\subsection{Experimental Results}

In what follows, we use the following notation, given a trained classifier $\hat{Y}$:
\begin{align*}
\mathbf{D_{FPR}}&=\left|FPR_{A=0}(\hat{Y})-FPR_{A=1}(\hat{Y})\right| \\ 
\mathbf{D_{FNR}}&=\left|FNR_{A=0}(\hat{Y})-FNR_{A=1}(\hat{Y})\right|
\end{align*}
The values $FPR_{A=0}(\hat{Y})$, $FPR_{A=1}(\hat{Y})$, $FNR_{A=0}(\hat{Y})$, $FNR_{A=1}(\hat{Y})$ are reported as evaluated on the test set.

\paragraph{The COMPAS Dataset\footnote{https://github.com/propublica/compas-analysis}} The Correctional Offender Management Profiling for Alternative Sanctions (COMPAS) records from Broward County, Florida 2013-2014, made available online by ProPublica, are perhaps the best-studied data in the context of fairness.  The goal in this scenario is to successfully predict recidivism within two years, based on features such as age, gender, race, number of prior offenses, and charge degree. The dataset contains 5,278 samples. The protected attribute in this scenario is race, where $A$ indicates black or white. We filtered the dataset using the same features as Zafar et al.~(\citeyear{disparatemistreatment}), to allow for comparison.

%\begin{table}[h]
%\centering
%\begin{tabularx}{\columnwidth}{c|c|c|c}
%\hline
%  &  Recid. ($y = 1$)        & No Recid.  ($y = 0$)       & Total \\ \hline
%Black &  $ 1661   $ & $ 1514 $ &  $ 3175 $ \\ \hline
%White &  $ 822   $  & $1281  $ &  $ 2103 $ \\ \hline
%Total &  $ 2483  $  & $2795 $ &  $ 5278 $ \\\hline
%\end{tabularx}
%\caption{Statistics of the ProPublica COMPAS data.} \label{table:compas-stats}
%\label{tab:stats}
%\end{table}
%\vspace{-1em}

%\begin{table}[h]
%\centering
%\begin{tabularx}{\columnwidth}{c|c}
%\hline
%Feature  &  Description \\ \hline
%Age Category &  $<25$, between $25$ and $45$, $>45$ \\
%Gender &  Male or Female \\
%Race &  White or Black \\
%Priors Count &  0--37 \\
%Charge Degree &  Misconduct or Felony \\
%\hline
%2-year-recid. & Whether or not the  \\
%(target feature)  & defendant recidivated within two years
%\end{tabularx}
%\caption{Description of features used from ProPublica COMPAS data.} \label{table:compas-features}
%\label{tab:features}
%\end{table}




\begin{table*}[t]
\centering
\caption{A description of the datasets used, along with parameters of the training procedure used for each.}
\label{table:datasets_description}
\begin{adjustbox}{max width=\textwidth}
\begin{tabular}{|l|l|l|l|l|l|l|l|}
\hline
\textbf{Dataset} & \textbf{No. Samples} & \textbf{No. Features} & \textbf{Train/Test Split} & \textbf{No. Repetitions} & \textbf{No. Folds in CV} & \textbf{Protected Feature} & \textbf{Target Variable} \\ \hline
COMPAS           & 5,278                     & 5                          & 70-30                     & 5                        & 5                                 & Race                       & 2-Year-Recidivism        \\ \hline
Adult            & 30,162                    & 10                         & 30-70                     & 5                        & 5                                 & Gender                     & Income Over/Under 50K    \\ \hline
Default          & 30,000                    & 23                         & 30-70                     & 5                        & 3                                 & Gender                     & Defaulting On Payments   \\ \hline
Admissions       & 20,839                    & 17                         & 30-70                     & 5                        & 3                                 & Race                       & Passing Bar Exam         \\ \hline
\end{tabular}
\end{adjustbox}
\end{table*}


\begin{table*}[t]
\centering
\resizebox{\textwidth}{!}{
\def\arraystretch{1.2}

\begin{tabular}{c c c | c | c | c || c | c | c || c | c | c |}

\cline{4-12}
&&&
\multicolumn{3}{ c|| }{\textbf{Adult Dataset}}&
\multicolumn{3}{ c|| }{\textbf{Default Dataset}}&
\multicolumn{3}{ c| }{\textbf{Admissions Dataset}}
\\ \cline{4-12}
%&&&
%\multicolumn{3}{ c|| }{\textbf{Both Considerations}}&
%\multicolumn{3}{ c|| }{\textbf{Both Considerations}}&
%\multicolumn{3}{ c| }{\textbf{Both Considerations}}
%\\ \cline{4-12}
&&&
 $\mathbf{Acc.}$ &  $\mathbf{D_{FPR}}$ &  $\mathbf{D_{FNR}}$ &  $\mathbf{Acc.}$ &  $\mathbf{D_{FPR}}$ &  $\mathbf{D_{FNR}}$ &  $\mathbf{Acc.}$ &  $\mathbf{D_{FPR}}$ &  $\mathbf{D_{FNR}}$
\\  \cline{4-12}
\vspace*{-0.5ex}
\\ \cline{1-2} \cline{4-12}
\multicolumn{1}{ |c  }{} &
\multicolumn{1}{ c|  }{  \textbf{Our Method (AVD Penalizers)}}  &&
$\mathbf{0.776}$    &  $\mathbf{0.00}$  &  $\mathbf{0.04}$ &
$\mathbf{0.807}$    &  $\mathbf{0.00}$   &  $\mathbf{0.01}$ &
$\mathbf{0.950}$    &  $\mathbf{0.01}$  &  $\mathbf{0.00}$
\\ \cline{1-2} \cline{4-12}
\multicolumn{1}{ |c  }{} &
\multicolumn{1}{ c|  }{  \textbf{Our Method (SD Penalizers)}}  &&
$\mathbf{0.783}$    &  $\mathbf{0.00}$  &  $\mathbf{0.09}$ &
$\mathbf{0.806}$    &  $\mathbf{0.01}$   &  $\mathbf{0.02}$ &
$\mathbf{0.950}$    &  $\mathbf{0.00}$  &  $\mathbf{0.00}$
\\ \cline{1-2} \cline{4-12}
\multicolumn{1}{ |c  }{} &
\multicolumn{1}{ c|  }{  \textbf{Vanilla Regularized Logistic Regression}}  &&
$\mathbf{0.800}$    &   $\mathbf{0.08}$    &   $\mathbf{0.39}$  &
$\mathbf{0.807}$    &   $\mathbf{0.01}$    &   $\mathbf{0.05}$  &
$\mathbf{0.951}$    &   $\mathbf{0.16}$    &   $\mathbf{0.02}$
\\ \cline{1-2} \cline{4-12}
\end{tabular}
}
\vspace{3mm}
\caption{Performance on the Adult, Loan Default, and Admissions datasets, penalizing for both FPR and FNR difference. Accuracy, FPR difference and FNR difference are evaluated on the test set, averaging over five runs and using a 30-70 training/test split.} \label{table:comparison_results_rest}
\end{table*}


In Table~\ref{table:comparison_results}, we compare the performance of our approach with that of three other techniques from the literature. Each method was trained based on logistic regression.  As a basis for comparison, we also present the performance of vanilla logistic regression, absent fairness considerations, with the regularization parameter selected via cross-validation.\footnote{Zafar et al.~(\citeyear{disparatemistreatment}) do not incorporate regularization in any of the approaches they report.}
%Results are reported as the averages of 5 different runs \katrina{Is that still correct?}, each time splitting data evenly and randomly into training and test sets. 
Results for Zafar et al., Zafar et al. baseline, and Hardt et al. appear here as reported in Zafar et al.~(\citeyear{disparatemistreatment}).\footnote{Our method selects the classifier based on the training set only and reports its performance over the test set. Results for the three other approaches, reported by Zafar et al.~(\citeyear{disparatemistreatment}), are based on tuning parameters after seeing the trade-off curve over the test set, and reporting according to the best selection of these parameters.}
%\katrina{Perhaps here is the right place for a footnote about the discrepancy with the Zafar baseline}

We find that the vanilla logistic regressor (absent fairness considerations) results in significant unfairness, as $\mathbf{D_{FPR}}=0.20$, and $\mathbf{D_{FNR}}=0.30$. The overall accuracy of this classifier measured on the test set was $0.672$.\footnote{Zafar et al.~(\citeyear{disparatemistreatment}) report a slightly different baseline of: Accuracy = 0.668, $\mathbf{D_{FPR}}=0.18$, $\mathbf{D_{FNR}}=0.30$.} Our SD penalization approach empirically achieves approximately the same accuracy as the Zafar et al.~(\citeyear{disparatemistreatment}) approach, with significantly better fairness. It is difficult to compare fairness-accuracy tradeoffs with the Hardt et al.~(\citeyear{hardt}) approach, since their accuracy is significantly lower than ours. A more direct comparison is possible by noting that our learned classifier can be post-processed to improve its fairness at a direct cost to accuracy. Hence, we can achieve accuracy of $0.659$ with $\mathbf{D_{FPR}} = \mathbf{D_{FNR}} = 0.01$, which compares very favorably with the Hardt et al. accuracy rate of 0.645 given the same FPR and FNR rates.\footnote{For completeness, we note that using a 50-50 training-test split (again not using the test set for parameter selection), our method (SD, both considerations) produces a classifier that provides: Accuracy = 0.659, $\mathbf{D_{FPR}} = 0.01, \mathbf{D_{FNR}} = 0.05$. This classifier can be post-processed to achieve rates of: Accuracy = 0.655, $\mathbf{D_{FPR}} = \mathbf{D_{FNR}} = 0.01$.}

Figure \ref{fig:compas} illustrates the accuracy/fairness trade-offs achievable using our scheme. Increasing the weight $c$ on the proxy fairness penalizers results in reducing their magnitude. The figure also illustrates how our relaxed penalizers succeed in tracking the real FPR and FNR differences. 
%
%
%\katrina{Must rewrite the following paragraph}
%We observe that our method succeeds in eliminating unfairness almost completely on the COMPAS dataset, while retaining most of the accuracy, when compared to the vanilla logistic regression. We achieve very low difference rates when penalizing for achieving each of the FPR and FNR criteria individually, and also for both. We achieve preferable results comparing to Zafar et al. and Zafar et al. baseline in all 3 scenarios, and also comparing to Hardt et al. in the settings of false positive/false negative considerations only. In the setting of both considerations - The Hardt et al. method removes a larger portion of the unfairness, however it results in major accuracy loss as it achieves accuracy rate of 0.645 in comparison to our method which results in accuracy of 0.665, retaining most of the original accuracy rate while removing most of the unfairness.




%The Hardt et al.~\cite{hardt} approach as reported removes a smaller portion of the bias in the different scenarios, however for FP/FN constraints alone, it provides higher accuracy rates. The Zafar et al.~(\citeyear{disparatemistreatment}) approach as reported retains significant bias (in most cases), but in some cases  achieves slightly superior accuracy rates to the methods above. 

%These performance comparisons are incomplete in the sense that each of the compared techniques has the potential to trade off between accuracy and fairness, using some degree of parameter tuning; what we report here is only one point on the achievable trade-off frontier for each algorithm. The ``correct'' trade-off, and, in particular, the best manner in which to weigh unfairness in the FPR against unfairness in the FNR, are matters of opinion. We have chosen to report our method's performance under parameters designed to very aggressively mitigate unfairness, at some cost to the accuracy.

%It would certainly be desirable to evaluate these and other approaches to fair learning on other datasets and on different tasks, particularly on larger datasets, which might afford both greater accuracy and better bias-reduction. The present empirical evaluations, however, suggest that our regularization-based approach provides a new tool worthy of consideration---we succeed in almost entirely eliminating bias on the hold-out set, at a modest price in terms of accuracy.

%Due to the fact that our true objective includes the original non-convex penalization terms, our approach does not carry any formal guarantees. However, the ease of implementation, generality, and empirical results are encouraging. Figure~\ref{fig:test1} illustrates the rate of convergence to a fair, accurate classifier on this dataset.
%In terms of computation costs, given that at each iteration we must calculate the gradient according to the FPR and FNR regularizers, we are required to predict the labels for the entire training set at each step. 
%However, this does not pose a computational burden, as it is already required by the (classic) gradient descent algorithm in our logistic regressor fitting scheme. Furthermore, when given a sufficiently large dataset (one or two orders of magnitude larger than the one currently available for the COMPAS scores data), this could be relaxed to sampling only a mini-batch of samples from the training data set at each iteration (much as is done in stochastic gradient descent).






\subsection{Additional Datasets}


Table~\ref{table:datasets_description} provides summary statistics on each of the datasets on which we tested our approach. We also briefly describe the datasets below. 


{\bf The Adult Dataset}\footnote{http://archive.ics.uci.edu/ml/datasets/Adult} is based on 1994 US Census data. The task we consider is to predict whether the income of each individual is over or under 50K dollars per year, based on features such as occupation, marital status, and education. The protected attribute selected in this task is gender. 

{\bf The Loan Default Dataset}\footnote{{\scriptsize https://archive.ics.uci.edu/ml/datasets/default+of+credit+card+clients}}
contains data regrading Taiwanese credit card users. The task we consider is to predict whether an individual will default on payments, based on features such as history of past payments, age, and the amount of given credit. The protected attribute is gender.

{\bf The Admissions Dataset}\footnote{http://www2.law.ucla.edu/sander/Systemic/Data.htm}
contains records of law school students who went on to take the bar exam. The task we consider is to predict whether a student will pass the exam based on features such as LSAT score, undergraduate GPA, and family income. The protected attribute is set to race.

Table~\ref{table:comparison_results_rest} describes the performance of our approach on these datasets, and Figures~\ref{fig:adult},~\ref{fig:default}, and~\ref{fig:lawschool} illustrate the fairness-accuracy trade-offs we achieve in each context. Overall, we see that unfairness is nearly eliminated while accuracy remains quite high. The dataset on which accuracy suffers most under our approach is the Adult dataset, which is also the dataset on which the vanilla regression is the most unfair.


\begin{figure*}[]
  \includegraphics[scale=0.6]{adult0-800.png}
  \caption{Adult Dataset. Fairness-Accuracy tradeoffs, as in Figure~\ref{fig:compas}.}
  \label{fig:adult}  
\end{figure*}



\begin{figure*}[]
  \includegraphics[scale=0.6]{default0-50.png}
  \caption{Loan Default Dataset. Fairness-Accuracy tradeoffs, as in Figure~\ref{fig:compas}.}
  \label{fig:default}
\end{figure*}



\begin{figure*}[]
  \includegraphics[scale=0.6]{admissions0-400.png}
  \caption{Admissions Dataset. Fairness-Accuracy tradeoffs, as in Figure~\ref{fig:compas}.}
  \label{fig:lawschool}
\end{figure*}




\vspace{-2em}
\section{Limitations and future work}
\label{sec:dis}
%% theory does not specifically encode for fine grain properties like fairness or robustness
Our method comes with several limitations and possible extensions for future work.  
While preserving model correlation suggests that we are likely to preserve sub-class loss, our theory does not currently extend to that regime and requires that the unlabeled pruning set be from the same distribution as the test set. More broadly, we do not yet fully understand how trainable the models produced by ID are in different conditions and we cannot make claims about how compressable a given model will be. In future work we will explore modifying training process to improve prunability, which is a common approach~\cite{huang2018sss,yang2020hoyer,liu2017netslim,zhuang2020polar}.  
We will also explore ways to refine our iterative pruning approach to work with
%residual layers and 
more complicated architectures. Of particular note, our method typically exposes a sizable ``free FLOPs" regime and we explore how this can be leveraged more broadly.


\vspace{-1em}
\begin{ack}
\vspace{-1em}
This work was partially funded by the National Science Foundation under awards
DMS-1830274, DGE-1922551, and NSF CAREER Award 2046760.
\end{ack}


{
\small
\bibliographystyle{plainnat}
\bibliography{id_for_nn.bib}
}


%%%%%%%%%%%%%%%%%%%%%%%%%%%%%%%%%%%%%%%%%%%%%%%%%%%%%%%%%%%%



%%% BEGIN INSTRUCTIONS %%%
%The checklist follows the references.  Please
%read the checklist guidelines carefully for information on how to answer these
%questions.  For each question, change the default \answerTODO{} to \answerYes{},
%\answerNo{}, or \answerNA{}.  You are strongly encouraged to include a {\bf
%justification to your answer}, either by referencing the appropriate section of
%your paper or providing a brief inline description.  For example:
%\begin{itemize}
%  \item Did you include the license to the code and datasets? \answerTODO{}
%  \item Did you include the license to the code and datasets? \answerTODO{}
%  \item Did you include the license to the code and datasets? \answerNA{}
%\end{itemize}
%Please do not modify the questions and only use the provided macros for your
%answers.  Note that the Checklist section does not count towards the page
%limit.  In your paper, please delete this instructions block and only keep the
%Checklist section heading above along with the questions/answers below.
%%% END INSTRUCTIONS %%%

% \section*{Checklist}
% \begin{enumerate}


% \item For all authors...
% \begin{enumerate}
%   \item Do the main claims made in the abstract and introduction accurately reflect the paper's contributions and scope?
%     \answerYes{}
%   \item Did you describe the limitations of your work?
%     \answerYes{} Section 7
%   \item Did you discuss any potential negative societal impacts of your work?
%     \answerNA{} 
%   \item Have you read the ethics review guidelines and ensured that your paper conforms to them?
%     \answerYes{}
% \end{enumerate}


% \item If you are including theoretical results...
% \begin{enumerate}
%   \item Did you state the full set of assumptions of all theoretical results?
%     \answerYes{}
%         \item Did you include complete proofs of all theoretical results?
%     \answerYes{} - In the Appendix.  
% \end{enumerate}


% \item If you ran experiments...
% \begin{enumerate}
%   \item Did you include the code, data, and instructions needed to reproduce the main experimental results (either in the supplemental material or as a URL)?
%     \answerYes{} % - Supplemental material
%     % \href{https://github.com/jerry-chee/ModelPreserveCompressionNN}{https://github.com/jerry-chee/ModelPreserveCompressionNN}
% %   \item Did you specify all the training details (e.g., data splits, hyperparameters, how they were chosen)?
%     \answerYes{} Appendix
%         \item Did you report error bars (e.g., with respect to the random seed after running experiments multiple times)?
%     \answerYes{} Where including error bars was computationally feasible we did so.  This was not computationally feasible on ImageNet.  
%         \item Did you include the total amount of compute and the type of resources used (e.g., type of GPUs, internal cluster, or cloud provider)? 
%     \answerYes{} Appendix
% \end{enumerate}


% \item If you are using existing assets (e.g., code, data, models) or curating/releasing new assets...
% \begin{enumerate}
%   \item If your work uses existing assets, did you cite the creators?
%     \answerYes{}
%   \item Did you mention the license of the assets?
%     \answerYes{}{}
%   \item Did you include any new assets either in the supplemental material or as a URL?
%     \answerNo{}
%   \item Did you discuss whether and how consent was obtained from people whose data you're using/curating?
%     \answerNA{}
%   \item Did you discuss whether the data you are using/curating contains personally identifiable information or offensive content?
%     \answerNA{}
% \end{enumerate}


% \item If you used crowdsourcing or conducted research with human subjects...
% \begin{enumerate}
%   \item Did you include the full text of instructions given to participants and screenshots, if applicable?
%     \answerNA{}
%   \item Did you describe any potential participant risks, with links to Institutional Review Board (IRB) approvals, if applicable?
%     \answerNA{}
%   \item Did you include the estimated hourly wage paid to participants and the total amount spent on participant compensation?
%     \answerNA{}
% \end{enumerate}


% \end{enumerate}


%%%%%%%%%%%%%%%%%%%%%%%%%%%%%%%%%%%%%%%%%%%%%%%%%%%%%%%%%%%%

\newpage
\appendix
\section*{Appendix}
\documentclass[journal]{IEEEtran}

% correct bad hyphenation here
\hyphenation{op-tical net-works semi-conduc-tor}

% packages used in document
\usepackage{amsfonts}
\usepackage{microtype}
\usepackage{graphicx}
\usepackage{subfigure}
\usepackage{booktabs}
\usepackage{multirow}
\usepackage{mathtools,cases}
\usepackage{xcolor}
\usepackage{hyperref}

\begin{document}

% set url style
\urlstyle{tt}

\title{Leveraging power grid topology in machine learning assisted optimal power flow}
%
%
% author names and IEEE memberships
% note positions of commas and nonbreaking spaces $ ~ $ LaTeX will not break
% a structure at a ~ so this keeps an author's name from being broken across
% two lines.
% use \thanks{} to gain access to the first footnote area
% a separate \thanks must be used for each paragraph as LaTeX2e's \thanks
% was not built to handle multiple paragraphs
%

\author{Thomas~Falconer
        and~Letif~Mones% <-this % stops a space
\thanks{All authors are with Invenia Labs, 95 Regent Street, Cambridge, CB2 1AW, United Kingdom (e-mails: \{firstname.lastname\}@invenialabs.co.uk).}% <-this % stops a space
}

\maketitle

\IEEEpeerreviewmaketitle

\section{Supplementary Material}

\IEEEPARstart{T}{o} complete the set of our empirical results, we provide additional information regarding: (1) the training time statistics for local and extended global regression GNN models (Table~\ref{tab:train_times_reg_local}); (2) the number of parameters for global classification models with fixed topology (Table~\ref{tab:params_clf_global}); (3) the training time statistics for global classification models (Table~\ref{tab:train_times_clf_global}); (4) the MSE statistics of the test sets for global, local and extended global GNN regression models using the admittance matrix (Tables~\ref{tab:error_reg_global_admittance}, \ref{tab:error_reg_local_admittance}); and lastly (5) the number of parameters for global classification models with varying topology (Table~\ref{tab:params_clf_global_contingency}).

\begin{table*}[!ht]
\small
\caption{Training time statistics (mean and two-sided 95\% confidence intervals) for local and extended global regression GNN models}
\label{tab:train_times_reg_local}
\def\na{---}
\centering
    %\resizebox{\textwidth}{!}{
    \begin{tabular}{lrrr|rrr}
    \toprule
    \multirow{2}{*}{Case} & \multicolumn{6}{c}{Training time ($\times 10^{2}$ s)} \\
    \cmidrule(r){2-7}
    & 
    $\mathcal{M}^{\textrm{GCN}}_{\textrm{local-3}}$ & $\mathcal{M}^{\textrm{CHC}}_{\textrm{local-3}}$ & $\mathcal{M}^{\textrm{GAT}}_{\textrm{local-3}}$ & $\mathcal{M}^{\textrm{GCN}}_{\textrm{global-4}}$ & $\mathcal{M}^{\textrm{CHC}}_{\textrm{global-4}}$ & $\mathcal{M}^{\textrm{GAT}}_{\textrm{global-4}}$ \\
    \midrule
    $\textrm{24-ieee-rts}$ & ${2.98} \pm {0.49}$ & $12.95 \pm 1.76$ & $6.30 \pm 1.38$ & ${8.25} \pm {1.55}$ & $14.92 \pm 2.25$ & $10.03 \pm 1.65$ \\
    $\textrm{30-ieee}$ & ${2.57} \pm {0.41}$ & $10.22 \pm 1.17$ & $6.54 \pm 1.31$ & ${9.57} \pm {2.36}$ & $15.86 \pm 1.72$ & $9.79 \pm 1.63$ \\
    $\textrm{39-epri}$ & ${2.60} \pm {0.54}$ & $11.03 \pm 0.91$ & $4.21 \pm 0.69$ & $13.18 \pm 2.14$ & $15.60 \pm 2.24$ & ${12.92} \pm {1.91}$ \\
    $\textrm{57-ieee}$ & ${2.47} \pm {0.77}$ & $9.47 \pm 0.71$ & $5.24 \pm 1.07$ & ${9.74} \pm {1.31}$ & $10.30 \pm 2.60$ & $10.87 \pm 1.92$ \\
    $\textrm{73-ieee-rts}$ & ${6.44} \pm {0.88}$ & $17.77 \pm 3.08$ & $12.12 \pm 3.67$ & ${11.58} \pm {1.69}$ & $23.40 \pm 3.17$ & $19.56 \pm 5.16$ \\
    $\textrm{118-ieee}$ & ${4.72} \pm {0.72}$ & $12.75 \pm 1.60$ & $9.44 \pm 1.65$ & $9.85 \pm 1.07$ & ${6.79} \pm {0.77}$ & $17.46 \pm 5.86$ \\
    $\textrm{162-ieee-dtc}$ & ${4.91} \pm {0.70}$ & $16.81 \pm 2.60$ & $10.05 \pm 1.52$ & $7.49 \pm 0.88$ & ${4.58} \pm {0.29}$ & $11.64 \pm 2.33$ \\
    $\textrm{300-ieee}$ & ${11.60} \pm {2.52}$ & $15.42 \pm 1.18$ & $20.00 \pm 4.91$ & $16.92 \pm 1.68$ & ${8.40} \pm {0.43}$ & $19.86 \pm 6.88$ \\
    $\textrm{588-sdet}$ & $43.33 \pm 9.31$ & ${27.33} \pm {2.18}$ & $75.15 \pm 9.95$ & $39.04 \pm 4.37$ & ${16.67} \pm {0.83}$ & $34.38 \pm 8.83$ \\
    \bottomrule
    \end{tabular}
    %}
\end{table*}

\begin{table*}[!ht]
\small
\caption{Number of parameters for global classification models (fixed topology)}
\label{tab:params_clf_global}
\def\na{---}
\centering
    %\resizebox{\columnwidth}{!}{
    \begin{tabular}{lr|rrrrrr}
    \toprule
    \multirow{2}{*}{Case} & \multicolumn{7}{c}{\# of parameters} \\ 
    \cmidrule(r){2-8}
    & $\mathcal{M}^{\textrm{FCNN}}_{\textrm{global-3}}$ & $\mathcal{M}^{\textrm{FCNN}}_{\textrm{global-1}}$ & $\mathcal{M}^{\textrm{CNN}}_{\textrm{global-4}}$ & $\mathcal{M}^{\textrm{GCN}}_{\textrm{global-3}}$ & $\mathcal{M}^{\textrm{CHC}}_{\textrm{global-3}}$ & $\mathcal{M}^{\textrm{SC}}_{\textrm{global-3}}$ & $\mathcal{M}^{\textrm{GC}}_{\textrm{global-3}}$ \\
    \midrule
    $\textrm{24-ieee-rts}$ & $3443$ & $833$ & $877$ & $1034$ & $1514$ & $1674$ & $1194$ \\
    $\textrm{30-ieee}$ & $3643$ & $244$ & $720$ & $359$ & $839$ & $999$ & $519$ \\
    $\textrm{39-epri}$ & $8120$ & $1659$ & $1617$ & $1075$ & $1555$ & $1715$ & $1235$ \\
    $\textrm{57-ieee}$ & $13301$ & $1150$ & $1398$ & $815$ & $1295$ & $1455$ & $975$ \\
    $\textrm{73-ieee-rts}$ & $30251$ & $7203$ & $6125$ & $7095$ & $7575$ & $7735$ & $7255$ \\
    $\textrm{118-ieee}$ & $63807$ & $9954$ & $9366$ & $10712$ & $12536$ & $13144$ & $11320$ \\
    $\textrm{162-ieee-dtc}$ & $128323$ & $24050$ & $21974$ & $24808$ & $26632$ & $27240$ & $25416$ \\
    $\textrm{300-ieee}$ & $494219$ & $114791$ & $107739$ & $115549$ & $117373$ & $117981$ & $116157$ \\
    $\textrm{588-sdet}$ & $1460783$ & $171842$ & $166590$ & $176688$ & $178512$ & $179120$ & $177296$ \\
    $\textrm{1354-pegase}$ & $7486643$ & $734139$ & $724971$ & $734897$ & $736721$ & $737329$ & $735505$ \\
    $\textrm{2853-sdet}$ & $43369442$ & $9690486$ & $9619758$ & $9759164$ & $9760988$ & $9761596$ & $9759772$ \\
    \bottomrule
    \end{tabular}
    %}
\end{table*}

\begin{table*}[!ht]
\small
\caption{Training time statistics (mean and two-sided 95\% confidence intervals) for global classification models}
\label{tab:train_times_clf_global}
\def\na{---}
\centering
    %\resizebox{\textwidth}{!}{
    \begin{tabular}{lr|rrrrrr}
    \toprule
    \multirow{2}{*}{Case} & \multicolumn{7}{c}{Training time ($\times 10^{2}$ s)} \\
    \cmidrule(r){2-8}
    & $\mathcal{M}^{\textrm{FCNN}}_{\textrm{global-3}}$ & $\mathcal{M}^{\textrm{FCNN}}_{\textrm{global-1}}$ & $\mathcal{M}^{\textrm{CNN}}_{\textrm{global-4}}$ & $\mathcal{M}^{\textrm{GCN}}_{\textrm{global-3}}$ & $\mathcal{M}^{\textrm{CHC}}_{\textrm{global-3}}$ & $\mathcal{M}^{\textrm{SC}}_{\textrm{global-3}}$ & $\mathcal{M}^{\textrm{GC}}_{\textrm{global-3}}$ \\
    \midrule
    $\textrm{24-ieee-rts}$ & $0.55 \pm 0.08$ & $4.21 \pm 0.02$ & ${0.79} \pm {0.05}$ & $9.14 \pm 2.29$ & $10.36 \pm 1.46$ & $7.87 \pm 0.62$ & $10.50 \pm 1.62$ \\
    $\textrm{30-ieee}$ & $0.36 \pm 0.04$ & $3.87 \pm 0.44$ & ${0.96} \pm {0.14}$ & $5.23 \pm 1.22$ & $11.35 \pm 1.49$ & $7.64 \pm 1.71$ & $8.32 \pm 1.35$ \\
    $\textrm{39-epri}$ & $0.51 \pm 0.06$ & $3.32 \pm 0.48$ & ${0.70} \pm {0.14}$ & $14.50 \pm 2.71$ & $11.81 \pm 1.37$ & $9.36 \pm 1.44$ & $10.94 \pm 1.68$ \\
    $\textrm{57-ieee}$ & $0.20 \pm 0.02$ & $1.98 \pm 0.73$ & ${0.49} \pm {0.04}$ & $9.29 \pm 1.79$ & $4.96 \pm 0.45$ & $6.37 \pm 1.10$ & $5.61 \pm 0.54$ \\
    $\textrm{73-ieee-rts}$ & $0.27 \pm 0.03$ & $4.44 \pm 0.16$ & ${1.06} \pm {0.30}$ & $12.92 \pm 2.53$ & $10.71 \pm 1.42$ & $9.45 \pm 1.52$ & $8.99 \pm 1.22$ \\
    $\textrm{118-ieee}$ & $0.20 \pm 0.01$ & $1.85 \pm 0.40$ & ${0.59} \pm {0.10}$ & $15.21 \pm 4.01$ & $4.62 \pm 0.28$ & $5.05 \pm 0.25$ & $4.36 \pm 0.39$ \\
    $\textrm{162-ieee-dtc}$ & $0.18 \pm 0.01$ & $0.81 \pm 0.09$ & ${0.52} \pm {0.05}$ & $7.86 \pm 0.87$ & $3.37 \pm 0.18$ & $4.37 \pm 0.41$ & $3.44 \pm 0.19$ \\
    $\textrm{300-ieee}$ & $0.15 \pm 0.00$ & $0.58 \pm 0.02$ & ${0.34} \pm {0.03}$ & $5.29 \pm 0.53$ & $3.44 \pm 0.11$ & $4.53 \pm 0.23$ & $2.81 \pm 0.15$ \\
    $\textrm{588-sdet}$ & $0.14 \pm 0.00$ & $0.50 \pm 0.02$ & ${0.23} \pm {0.01}$ & $4.00 \pm 0.41$ & $3.77 \pm 0.07$ & $5.05 \pm 0.19$ & $2.94 \pm 0.10$ \\
    $\textrm{1354-pegase}$ & $0.27 \pm 0.00$ & $0.50 \pm 0.02$ & ${0.20} \pm {0.00}$ & $2.61 \pm 0.13$ & $4.52 \pm 0.09$ & $7.07 \pm 0.17$ & $2.82 \pm 0.08$ \\
    $\textrm{2853-sdet}$ & $1.32 \pm 0.01$ & $0.65 \pm 0.02$ & ${0.40} \pm {0.01}$ & $8.53 \pm 0.19$ & $12.61 \pm 0.11$ & $17.26 \pm 0.44$ & $9.28 \pm 0.12$ \\
    \bottomrule
    \end{tabular}
    %}
\end{table*}

\begin{table*}[!ht]
\small
\caption{MSE statistics (mean and two-sided 95\% confidence intervals) of the test sets for global GNN regression models using admittance matrix (fixed topology)}
\label{tab:error_reg_global_admittance}
\def\na{---}
\centering
    %\resizebox{\textwidth}{!}{
    \begin{tabular}{lr|rrrrr}
    \toprule
    \multirow{2}{*}{Case} & \multicolumn{6}{c}{MSE ($\times 10^{-3}$)} \\
    \cmidrule(r){2-7}
    & 
    $\mathcal{M}^{\textrm{FCNN}}_{\textrm{global-3}}$ &  $\mathcal{M}^{\textrm{GCN}}_{\textrm{global-3}}$ & $\mathcal{M}^{\textrm{CHC}}_{\textrm{global-3}}$ & $\mathcal{M}^{\textrm{SC}}_{\textrm{global-3}}$ & $\mathcal{M}^{\textrm{GC}}_{\textrm{global-3}}$ & $\mathcal{M}^{\textrm{GAT}}_{\textrm{global-3}}$ \\
    \midrule
    $\textrm{24-ieee-rts}$ & $0.18 \pm 0.02$ & $2.48 \pm 0.12$ & $0.69 \pm 0.05$ & $0.89 \pm 0.07$ & $1.71 \pm 0.34$ & $2.78 \pm 0.18$ \\
    $\textrm{30-ieee}$ & $0.05 \pm 0.01$ & $1.41 \pm 0.28$ & $0.08 \pm 0.01$ & $0.11 \pm 0.03$ & $0.51 \pm 0.12$ & $2.95 \pm 0.22$ \\
    $\textrm{39-epri}$ & $0.89 \pm 0.10$ & $4.44 \pm 0.12$ & $2.42 \pm 0.05$ & $2.41 \pm 0.14$ & $4.86 \pm 0.95$ & $4.75 \pm 0.29$ \\
    $\textrm{57-ieee}$ & $0.52 \pm 0.11$ & $1.91 \pm 0.15$ & $1.25 \pm 0.13$ & $1.31 \pm 0.13$ & $1.67 \pm 0.15$ & $2.29 \pm 0.16$ \\
    $\textrm{73-ieee-rts}$ & $0.21 \pm 0.07$ & $1.37 \pm 0.26$ & $0.61 \pm 0.02$ & $0.67 \pm 0.02$ & $1.15 \pm 0.13$ & $1.83 \pm 0.11$ \\
    $\textrm{118-ieee}$ & $0.39 \pm 0.03$ & $2.12 \pm 0.07$ & $1.26 \pm 0.06$ & $1.27 \pm 0.07$ & $1.47 \pm 0.12$ & $2.43 \pm 0.12$ \\
    $\textrm{162-ieee-dtc}$ & $2.61 \pm 0.10$ & $3.75 \pm 0.11$ & $3.07 \pm 0.09$ & $2.87 \pm 0.13$ & $3.55 \pm 0.22$ & $4.84 \pm 0.22$ \\
    $\textrm{300-ieee}$ & $2.06 \pm 0.06$ & $3.96 \pm 0.16$ & $2.42 \pm 0.05$ & $2.35 \pm 0.07$ & $3.19 \pm 0.23$ & $3.59 \pm 0.24$ \\
    $\textrm{588-sdet}$ & $2.56 \pm 0.06$ & $3.66 \pm 0.07$ & $3.18 \pm 0.05$ & $3.21 \pm 0.08$ & $10.21 \pm 2.51$ & $5.01 \pm 0.24$ \\
    $\textrm{1354-pegase}$ & $0.83 \pm 0.12$ & $2.09 \pm 0.11$ & $1.43 \pm 0.09$ & $1.36 \pm 0.09$ & $2.64 \pm 0.11$ & $2.51 \pm 0.14$ \\
    $\textrm{2853-sdet}$ & $5.99 \pm 0.16$ & $11.54 \pm 0.48$ & $9.03 \pm 0.29$ & $8.35 \pm 0.13$ & $13.98 \pm 0.44$ & $11.15 \pm 0.59$ \\
    \bottomrule
    \end{tabular}
   %}
\end{table*}

\begin{table*}[!ht]
\small
\caption{MSE statistics (mean and two-sided 95\% confidence intervals) of the test sets for local and extended global regression GNN models using admittance matrix (fixed topology)}
\label{tab:error_reg_local_admittance}
\def\na{---}
\centering
    %\resizebox{\textwidth}{!}{
    \begin{tabular}{lrrr|rrr}
    \toprule
    \multirow{2}{*}{Case} & \multicolumn{6}{c}{MSE ($\times 10^{-3}$)} \\
    \cmidrule(r){2-7}
    & $\mathcal{M}^{\textrm{GCN}}_{\textrm{local-3}}$ & $\mathcal{M}^{\textrm{CHC}}_{\textrm{local-3}}$ & $\mathcal{M}^{\textrm{GAT}}_{\textrm{local-3}}$ & $\mathcal{M}^{\textrm{GCN}}_{\textrm{global-4}}$ & $\mathcal{M}^{\textrm{CHC}}_{\textrm{global-4}}$ & $\mathcal{M}^{\textrm{GAT}}_{\textrm{global-4}}$ \\
    \midrule
    $\textrm{24-ieee-rts}$ & $60.26 \pm 0.53$ & $25.89 \pm 0.14$ & $67.83 \pm 9.26$ & $2.16 \pm 0.08$ & $0.46 \pm 0.04$ & $2.56 \pm 0.15$ \\
    $\textrm{30-ieee}$ & $8.67 \pm 1.05$ & $0.52 \pm 0.14$ & $28.47 \pm 9.82$ & $1.17 \pm 0.38$ & $0.11 \pm 0.02$ & $2.85 \pm 0.22$ \\
    $\textrm{39-epri}$ &$16.12 \pm 6.49$ & $3.86 \pm 0.16$ & $13.94 \pm 2.79$ & $3.14 \pm 0.19$ & $2.21 \pm 0.08$ & $3.08 \pm 0.22$ \\
    $\textrm{57-ieee}$ & $4.85 \pm 0.21$ & $2.25 \pm 0.37$ & $8.87 \pm 3.25$ & $1.75 \pm 0.14$ & $1.26 \pm 0.21$ & $2.41 \pm 0.22$ \\
    $\textrm{73-ieee-rts}$ & $43.09 \pm 6.61$ & $31.58 \pm 0.04$ & $57.05 \pm 3.62$ & $0.88 \pm 0.03$ & $0.33 \pm 0.03$ & $2.32 \pm 1.33$ \\
    $\textrm{118-ieee}$ & $25.37 \pm 4.74$ & $7.75 \pm 0.19$ & $35.44 \pm 1.51$ & $2.65 \pm 0.11$ & $1.45 \pm 0.08$ & $4.66 \pm 0.26$\\
    $\textrm{162-ieee-dtc}$ & $11.19 \pm 0.18$ & $8.29 \pm 0.16$ & $11.51 \pm 0.13$ & $4.12 \pm 0.16$ & $3.33 \pm 0.11$ & $5.95 \pm 1.12$ \\
    $\textrm{300-ieee}$ & $11.56 \pm 0.46$ & $9.51 \pm 0.07$ & $78.56 \pm 9.67$ & $4.35 \pm 0.17$ & $2.71 \pm 0.06$ & $5.29 \pm 1.63$ \\
    $\textrm{588-sdet}$ & $21.11 \pm 1.15$ & $16.28 \pm 0.19$ & $28.56 \pm 8.91$ & $3.87 \pm 0.06$ & $3.83 \pm 0.12$ & $71.49 \pm 9.76$ \\
    \bottomrule
    \end{tabular}
    %}
\end{table*}

\begin{table*}[!ht]
\small
\caption{Number of parameters for global classification models (varying topology)}
\label{tab:params_clf_global_contingency}
\def\na{---}
\centering
    %\resizebox{\columnwidth}{!}{
    \begin{tabular}{lr|rrrrrr}
    \toprule
    \multirow{2}{*}{Case} & \multicolumn{7}{c}{\# of parameters} \\ 
    \cmidrule(r){2-8}
    & $\mathcal{M}^{\textrm{FCNN}}_{\textrm{global-3}}$ & $\mathcal{M}^{\textrm{FCNN}}_{\textrm{global-1}}$ & $\mathcal{M}^{\textrm{CNN}}_{\textrm{global-4}}$ & $\mathcal{M}^{\textrm{GCN}}_{\textrm{global-3}}$ & $\mathcal{M}^{\textrm{CHC}}_{\textrm{global-3}}$ & $\mathcal{M}^{\textrm{SC}}_{\textrm{global-3}}$ & $\mathcal{M}^{\textrm{GC}}_{\textrm{global-3}}$ \\
    \midrule
    $\textrm{24-ieee-rts}$ & $14372$ & $4263$ & $2067$ & $4324$ & $4804$ & $4964$ & $4484$ \\
    $\textrm{30-ieee}$ & $4740$ & $915$ & $1083$ & $700$ & $1180$ & $1340$ & $860$ \\
    $\textrm{39-epri}$ & $13661$ & $4345$ & $3283$ & $2435$ & $2915$ & $3075$ & $2595$ \\
    $\textrm{57-ieee}$ & $17233$ & $3680$ & $3180$ & $2091$ & $2571$ & $2731$ & $2251$ \\
    $\textrm{73-ieee-rts}$ & $77552$ & $25431$ & $20137$ & $24455$ & $24935$ & $25095$ & $24615$ \\
    $\textrm{118-ieee}$ & $112037$ & $34839$ & $31311$ & $35597$ & $37421$ & $38029$ & $36205$ \\
    $\textrm{162-ieee-dtc}$ & $171139$ & $47450$ & $42782$ & $48208$ & $50032$ & $50640$ & $48816$ \\
    $\textrm{300-ieee}$ & $600416$ & $169482$ & $158790$ & $170240$ & $172064$ & $172672$ & $170848$ \\
    $\textrm{588-sdet}$ & $1866920$ & $423720$ & $409908$ & $434558$ & $436382$ & $436990$ & $435166$ \\
    \bottomrule
    \end{tabular}
    %}
\end{table*}

\ifCLASSOPTIONcaptionsoff
  \newpage
\fi

\end{document}






\end{document}