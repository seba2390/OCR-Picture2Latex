%abstract rewriting to get better reviewers.  signal for do we want numerical linear algebra stuff at neurips.  classical empirical compression people 
%change the title to automated model preserving neural network compression  %emphasise the model correlation idea, frame the goals we want model preservation structure preservation and not a lot of fine tuning.  preserving per example decisions.  
%in the appendix talk a little bit more about the problem with pruning and how we demonstrate the 
%take the intro that is there and re-write it.  
% use some sort of "chemical notation" 
%add the two references that people noted.  fix our own references that were missing years.  
%re-do some of the plots, don't make it look like the background.  re-do the visuals to look like that.  make id dashed, and make the composed methods more obvious.  hit hte reader over the head with structure preserving. 
%add the checklist back in.  



\documentclass{article}
% if you need to pass options to natbib, use, e.g.:
\PassOptionsToPackage{numbers, compress}{natbib}
% before loading neurips_2022

% ready for submission
% \usepackage{neurips_2022}

% to compile a preprint version, e.g., for submission to arXiv, add add the
% [preprint] option:
\usepackage[preprint]{neurips_2022}

% to compile a camera-ready version, add the [final] option, e.g.:
%\usepackage[final]{neurips_2022}

% to avoid loading the natbib package, add option nonatbib:
%    \usepackage[nonatbib]{neurips_2022}


\usepackage[utf8]{inputenc} % allow utf-8 input
\usepackage[T1]{fontenc}    % use 8-bit T1 fonts
\usepackage{hyperref}       % hyperlinks
\usepackage{url}            % simple URL typesetting
\usepackage{booktabs}       % professional-quality tables
\usepackage{amsfonts}       % blackboard math symbols
\usepackage{nicefrac}       % compact symbols for 1/2, etc.
\usepackage{microtype}      % microtypography
\usepackage{xcolor}         % colors

%%%% User defined macros %%%%%
% Get better typography
% \usepackage[protrusion=true,expansion=true]{microtype}		

% Small margins
%\usepackage[top=2cm, bottom=2cm, left = 1cm, right = 1cm,columnsep=20pt]{geometry}


% For basic math, align, fonts, etc.
\usepackage{amsmath}
\usepackage{amsthm}
\usepackage{amssymb}
\usepackage{bbm}
\usepackage{mathtools}
\usepackage{mathrsfs}
\usepackage{dsfont}
\mathtoolsset{showonlyrefs}

\newtheorem{defn}{Definition}[]
\newtheorem{thm}{Theorem}[]
\newtheorem{ass}{Assumption}[]
\newtheorem{cor}{Corollary}[]

% For creating sub-groups of Assumptions/Theorems etc.
\newtheorem{assumpA}{Assumption}
\renewcommand\theassumpA{A\arabic{assumpA}}

\usepackage{courier} % For \texttt{foo} to put foo in Courier (for code / variables)

\usepackage{lipsum} % For dummy text

% For images
% \usepackage{graphicx}
% \usepackage{subcaption}
\usepackage[space]{grffile} % For spaces in image names

% For bibliography
%\usepackage[round]{natbib}

% For links (e.g., clicking a reference takes you to the phy)
\usepackage{hyperref}

% For color
\usepackage{graphicx}
\usepackage{wrapfig}
\usepackage{xcolor}
\definecolor{dark-red}{rgb}{0.4,0.15,0.15}
\definecolor{dark-blue}{rgb}{0,0,0.7}
\hypersetup{
    colorlinks, linkcolor={dark-blue},
    citecolor={dark-blue}, urlcolor={dark-blue}
}


% Macros



\usepackage{algpseudocode}
% \usepackage[noend,linesnumbered,ruled]{algorithm2e}
\usepackage[vlined,linesnumbered,ruled,algo2e]{algorithm2e}
% \usepackage[vlined,ruled,algo2e]{algorithm2e}

% \usepackage[algo2e]{algorithm2e} 
\SetKwProg{Fn}{Function}{}{}
\let\oldnl\nl% Store \nl in \oldnl
\newcommand{\nonl}{\renewcommand{\nl}{\let\nl\oldnl}}% Remove line number for one line

% Miscelaneous headers
\usepackage{multicol}
\usepackage{enumitem}
\usepackage[utf8]{inputenc} % allow utf-8 input
\usepackage[T1]{fontenc}    % use 8-bit T1 fonts
\usepackage{hyperref}       % hyperlinks
\usepackage{url}            % simple URL typesetting
\usepackage{booktabs}       % professional-quality tables
\usepackage{nicefrac}       % compact symbols for 1/2, etc.


\usepackage{theoremref}

%%%%%%%%%%%%%%%%%%%%%%%%%%%%%%


\title{
Model Preserving Compression for Neural Networks
%Compressing Neural Networks while Preserving the Original Model
}


% The \author macro works with any number of authors. There are two commands
% used to separate the names and addresses of multiple authors: \And and \AND.
%
% Using \And between authors leaves it to LaTeX to determine where to break the
% lines. Using \AND forces a line break at that point. So, if LaTeX puts 3 of 4
% authors names on the first line, and the last on the second line, try using
% \AND instead of \And before the third author name.


\author{%
  %David S.~Hippocampus\thanks{Use footnote for providing further information
  %  about author (webpage, alternative address)---\emph{not} for acknowledging
  %  funding agencies.} \\
  %Department of Computer Science\\
  %Cranberry-Lemon University\\
  %Pittsburgh, PA 15213 \\
  %\texttt{hippo@cs.cranberry-lemon.edu} \\
  % examples of more authors
  % \And
  % Coauthor \\
  % Affiliation \\
  % Address \\
  % \texttt{email} \\
  % \AND
  % Coauthor \\
  % Affiliation \\
  % Address \\
  % \texttt{email} \\
  % \And
  % Coauthor \\
  % Affiliation \\
  % Address \\
  % \texttt{email} \\
  % \And
  % Coauthor \\
  % Affiliation \\
  % Address \\
  % \texttt{email} \\
}




\author{
  Jerry Chee\thanks{equal contribution} \\
  Department of Computer Science\\
  Cornell University\\
  \texttt{jerrychee@cs.cornell.edu}\\
  %Pittsburgh, PA 15213 \\
  %\texttt{hippo@cs.cranberry-lemon.edu} \\
  % examples of more authors
   \And
   Megan Renz\footnotemark[1]\\
   Department of Physics\\
   Cornell University \\
   \texttt{mr2268@cornell.edu}\\
  % Address \\
  % \texttt{email} \\
   \AND
   Anil Damle \\
   Department of Computer Science\\
   Cornell University \\
   \texttt{damle@cornell.edu}\\
  % Address \\
  % \texttt{email} \\
   \And
   Christopher De Sa \\
   Department of Computer Science\\
   Cornell University \\
   \texttt{cdesa@cs.cornell.edu}
  % Address \\
  % \texttt{email} \\
  % \And
  % Coauthor \\
  % Affiliation \\
  % Address \\
  % \texttt{email} \\
}



\begin{document}


\maketitle


\begin{abstract}
After training complex deep learning models, a common task is to compress the model to reduce compute and storage demands. When compressing, it is desirable to preserve the original model's per-example decisions (e.g., to go beyond top-1 accuracy or preserve robustness), maintain the network's structure, automatically determine per-layer compression levels, and eliminate the need for fine tuning. No existing compression methods simultaneously satisfy these criteria---we introduce a principled approach that does by leveraging interpolative decompositions. Our approach simultaneously selects and eliminates channels 
%a structured low-rank matrix approximation known as the interpolative decomposition.
%By explicitly building an approximation to the activation output of each layer, we simultaneously select and eliminate channels
(analogously, neurons), then constructs an interpolation matrix that propagates a correction into the next layer, preserving the network's structure. 
Consequently, our method achieves good performance even without fine tuning and admits theoretical analysis.
Our theoretical generalization bound for a one layer network lends itself naturally to a heuristic that allows our method to automatically choose per-layer sizes for deep networks.
% Since our method simply makes networks narrower, it can easily be combined with other matrix decomposition techniques.
We demonstrate the efficacy of our approach with strong empirical performance on a variety of tasks, models, and datasets---from simple one-hidden-layer networks to deep networks on ImageNet.

% Preserving the original model's per-example decisions beyond top-1 accuracy and enabling effortless plug-and-play within machine learning pipelines by preserving network structure and minimizing fine tuning are both desirable properties for network compression methods. 
%by preserving the structure of the network while minimizing 
%Practical neural network compression methods should have the following qualities:  First to preserve the model's per-example decisions to maintain properties beyond top-1 accuracy (like
%sub-class accuracy
%fairness criteria
%or adversarial robustness), and second to enable
%effortless plug-and-play within a machine learning pipeline 
%by preserving the structure of the network while minimizing fine tuning and hyperparameter search.
%to ensure trivial plug-and-play within a machine learning pipeline.  
% A neural network compression method that is effective for practitioners must preserve the model's decisions as well as be practically usable.
% Preserving the per-example decisions ensures that properties beyond top-1 accuracy, such as sub-class accuracy or adversarial robustness, are retained.
% To ensure ease of use, the structure of the network must be preserved to enable trivial plug-and-play with the rest of the machine learning pipeline, and hyper-parameter tuning must be kept to a realistic minimum.
%Our goal is to compress deep networks into narrower but identically structured models that closely mirror the per-example decisions of the original model, 
%with minimal hyper-parameter tuning.
%To satisfy these criteria 
%Our method (uniquely?) combines the advantages of two types of well known compression methods: matrix approximation preserves the model's decisions, and structured pruning preserves the network's structure.
%Matrix approximation methods typically only satisfy the first criteria, while structured pruning methods typically only satisfy the second.
% Matrix approximation methods preserve the output, while structured pruning methods preserve the computational structure. 
% Our method combines advantages of both these types of methods.
%Matrix approximations typically preserve the model well, but change the network structure by adding additional layers.
%Structured pruning retains the network's structure, but does not typically preserve the model's decisions.

% We introduce a principled approach that leverages the interpolative decomposition to build a structured low-rank approximation of the activation output of each layer. 
% By doing so, we simultaneously select and eliminate channels 
% %a structured low-rank matrix approximation known as the interpolative decomposition.
% %By explicitly building an approximation to the activation output of each layer, we simultaneously select and eliminate channels
% (analogously, neurons) and construct an interpolation matrix that propagates a correction into the next layer, preserving the network's structure.
% Consequently, our method achieves good performance even without fine tuning and admits theoretical analysis.
% Our theoretical generalization bound for a one layer network lends itself naturally to a heuristic that allows our method to automatically choose per-layer sizes for deep networks.
% % Since our method simply makes networks narrower, it can easily be combined with other matrix decomposition techniques.
% We demonstrate the efficacy of our approach with strong empirical performance on a variety of tasks, models, and datasets---from simple one hidden layer networks to deep networks on ImageNet.
\end{abstract}


% \listoftodos

\section{Introduction}
\label{sec:intro}
\section{Introduciton}

Today's WiFi networks use advanced authentication and encryption mechanisms (such as WPA3) to protect our privacy and security by stopping unauthorized devices from accessing our devices and data. Despite all these mechanisms, WiFi networks remain vulnerable to attacks mainly due to their physical layer behaviors and requirements defined by WiFi standards. In this paper, we find two loopholes in the IEEE 802.11 standard for the first time and show how they can put our privacy and security at risk. 

\textbf{a) WiFi radios respond when they should not.}  In a WiFi network, when a device sends a packet to another device, the receiving device sends an acknowledgment back to the transmitter. In particular, upon receiving a frame, the receiver calculates the cyclic redundancy check (CRC) of the packet in the physical layer to detect possible errors. If it passes CRC, then the receiver sends an Acknowledgment (ACK) to the transmitter to notify the correct reception of the frame. Surprisingly, we have found that all existing WiFi devices send back ACKs to even fake packets received from unauthorized WiFi devices outside of their network. Why should a WiFi device respond to a fake packet from an unauthorized device?! 

\textbf{b) WiFi radios stay awake when they should not.}
WiFi chipsets are mostly in sleep mode to save power. However, to make sure that they do not miss their incoming packets, they notify their WiFi access point before entering sleep mode so that the access point buffers any incoming packets for them. Then, WiFi devices wake up periodically to receive beacon frames sent by the associated access point. In regular operation, only the access point sends beacon frames to notify the devices that have buffered packets. When a device is notified, it stays awake to receive them. However, these beacon frames are not encrypted. Hence, we find that an unauthorized user can forge those beacon frames to keep a specific device awake for receiving the (non-existent) buffered frames. %has packets waiting for it. 
%This keeps the WiFi radio awake and prevents it from going to sleep mode to save power.

We examine these behaviors and loopholes in detail over different WiFi chipsets from different vendors. Our examination of over 5,000 WiFi devices from 186 vendors shows that these are widespread issues. We then study the root cause of these issues and show that, unfortunately, they cannot be fixed by a simple solution such as updating WiFi chipsets firmware.  Finally, we implement and demonstrate two attacks based on these loopholes. In the first attack, we show that by forcing WiFi devices to stay awake and continuously transmit, an adversary can continuously analyze the signal and extract personal information such as the breathing rate of the WiFi users. In the second attack, we show that by forcing WiFi devices to stay awake and continuously transmit, the adversary can quickly drain the battery, and hence disable WiFi devices such as home and office security sensors. These attacks can be performed from outside buildings despite the WiFi network and devices being completely secured. All the attacker needs is a \$10 microcontroller with integrated WiFi (such as ESP32) and a battery bank. The attacker device can easily be carried in a pocket or hidden somewhere near the target building. 

The main contributions of this work are:
%\footnote{We discussed our project and experiments with our institution’s IRB office and they determined that no IRB review nor IRB approval is required.}:
\begin{itemize}
    \item We find that WiFi devices respond to fake 802.11 frames with ACK, even when they are from unauthorized devices. We also find that WiFi radios can be kept awake by sending them fake beacon frames indicating they have packets waiting for them. 
    \item We study these loopholes and their root causes in detail, and have tested more than 5,000 WiFi access points and client devices from more than 186 vendors.  
    
    \item We implement two attacks based on these loopholes using just a 10-dollar off-the-shelf WiFi module and validate them in real-world settings.
    



    
\end{itemize}






%\section{Context and Related Work}
\section{Related Work}
\label{sec:related}
There are a number of different design choices to be made in the compression and pruning process.  Classical pruning involves eliminating either channels (analogously neurons) or individual weights, sometimes in a structured way.  
Magnitude pruning on both weights and neurons is still considered an effective approach~\cite{blalock2020state,frankle2018lottery,frankle2020pruning,gale2019state,liu2019rethink,li2017l1}.  When pruning, the method may incorporate a correction to future layers, though it often does not ~\cite{he2019fpgm,luo2017thinet}. Some methods that correct the network chose to do local fine tuning ~\cite{luo2017thinet,he2017feat,zhuang2018dcp,peng2019ccp,liu2017netslim}, whereas others do not ~\cite{liebenwein2020provable,he2018amc}.  

Of particular note, matrix approximation methods
~\cite{denten2014svd,idel2020lrank,liebenwein2021alds,peng2018group,lebedev2015cpdecomp}
%~\cite{denten2014svd,idel2020lrank,jarderberg2014lowrank,liebenwein2021alds,peng2018group,lebedev2015cpdecomp,zhang20153dfilter}
often satisfy the first criteria we desire for compression methods, but typically not the second as they add additional layers. These methods sometimes incorporate local fine tuning after compression ~\cite{idel2020lrank,jaderberg2014speeding,liebenwein2021alds,peng2018group,zhang20153dfilter} and sometimes do not ~\cite{denten2014svd,lebedev2015cpdecomp}.
In contrast, structured pruning methods
%~\cite{he2019fpgm,he2018amc,liebenwein2020provable,liu2019rethink,luo2017thinet}
~\cite{he2019fpgm,he2018amc,liebenwein2020provable,liu2017netslim,liu2019rethink,luo2017thinet}
can satisfy the second criteria we outlined, but typically not the first as they do a poor job of preserving the model's decisions and often require excessive amounts of fine tuning.


Pruning with coresets~\cite{mussay2020coreset} is the closest in spirit to our own work and provides a way to select a subset of neurons in the current layer that can approximate those in the next layer as well as new weight connections. Of note, \citet{mussay2020coreset} provide a sample complexity result,
and demonstrate their method on fully connected (but not convolution) layers.
The HRank method~\cite{lin2020hrank} is also close in spirit to our own, and works by selectively pruning channels that produce low-rank feature maps.   However, the method does not propagate updates into the next layer and instead relies on excessive amounts of fine tuning (30 epochs for each layer pruned) to fix the network's accuracy.  


Recently the literature has started to consider criteria beyond topline accuracy metrics, and \citet{liebenwein2021lost} use measures of functional approximation to conclude that pruned networks well approximate the original models. 
\citet{marx2020multiplicity} characterize when linear models can achieve similar accuracy but with competing predictions.

% It is interesting to explore how their approximation measures differ from our per-label correlation.


\section{Interpolative decompositions}
\label{sec:ID}
%\jerry{maybe emphasize that we're not a low-rank pruning method, even though we're using a low-rank matrix approxiation method}

Our pruning strategy relies on a structured low-rank approximation known as an interpolative decomposition (ID). 
% jerry 1.25.22: we talk more about preserving structure in intro now, don't think we need it here
%\jerrytwo{Note that a major benefit of the ID is that it preserves the structure of the network.
%Unlike other low-rank approximation methods, we do not need to store additional matrix factorizations, and we do not need to pay the additional reconstruction cost during inference.
%}
Classically, the Singular Value Decomposition (SVD) (see, e.g.,~\cite{GVL}) provides an optimal low-rank approximation. However, because we consider matrices that include the non-linear activation function the SVD cannot be directly used to either subselect neurons or generate new ones (since it is unclear how to propagate singular vectors ``backwards'' through the non-linearity). In contrast, an ID constructs a structured low-rank approximation of a matrix $A$ where the basis used for the approximation is constrained to be a subset of the columns of $A$. For a matrix $A\in\R^{n\times m}$ we let $A_{\mathcal{J},\I}$ denote a sub-selection of the matrix $A$ using sets $\mathcal{J}\subset [n]$ to denote the selected rows and $\I\subset [m]$ to denote the selected columns; $:$ denotes a selection of all rows or columns.
% The optimal basis is the leading left singular vectors of $A$, but generically these will not be columns of $A$.
%Shortly, we will see how the rank-revealing QR factorization is used to compute the interpolative decomposition.
%\begin{definition}[Interpolative decomposition]
%Given a matrix $A \in \R^{n \times m}$ and some $k \leq \ell$ (as before, $\ell = \min(n,m)$) a \emph{rank-k interpolative decomposition} is an index subset $\I \subset [m]$ with $|\I| = k$ and an interpolation matrix $T \in \R^{k \times m}$ with $T_{:,\I} = I_k$ such that
%\begin{equation}
%\label{eq:ID}
 %   A \approx A_{:,\I} T.
%\end{equation}
%\end{definition}

\begin{definition}[Interpolative Decomposition]
\label{def:ID}
Let $A \in \R^{n \times m} $and $\epsilon \geq 0$. 
An $\epsilon$-accurate \emph{interpolative decomposition} 
$A \approx A_{:,\I} T$
is a subset of columns of A, denoted with the index subset $\I \subset [m],$ and an associated interpolation matrix $T$ such that $\|A - A_{:,\I} T \|_2 \leq \epsilon\|A\|_2.$
\end{definition}






\begin{remarks}
\hspace{-0.8em}
When computing an ID, we would like to find the smallest possible $k\equiv\lvert\I\rvert$ such that the accuracy requirement is satisfied. Moreover, we would like $T$ to have entries of reasonable magnitude and approximation error not much larger than the best possible for a given $k$ (i.e., $\|A - A_{:,\I}T\|_2 = t \sigma_{k+1}(A)$ for some small $t \geq 1$). While necessarily sub-optimal, the advantage is that we explicitly use a subset of the columns of $A$ to build the approximation.
\end{remarks}

IDs are well studied~\cite{cheng2005compression,martinsson2011randomized}, widely used in the domain of rank-structured matrices~\cite{ho2012fast,ho2015hierarchical,ho2016hierarchical,martinsson2005fast,martinsson2019fast,minden2017recursive}, and are closely related to CUR decompositions~\cite{mahoney2009cur,voronin2017efficient} and subset selection problems~\cite{boutsidis2009improved,civril2009selecting,tropp2009column}. While these decompositions always exist, finding them optimally is a difficult task and in this work we appeal to what are known as (strong) rank-revealing QR factorizations~\cite{businger1965linear,chan1992some,chandrasekaran1994rank,gu1996efficient,hong1992rank}.


%\subsection{Rank-Revealing QR factorization}
% \paragraph{Rank-revealing QR factorizations}
\begin{definition}[Rank-revealing QR factorization]
Let $A \in \R^{n \times m}$, $\ell = \min(n,m)$, and take any $k \leq \ell$.
A \emph{rank-revealing QR factorization} of $A$ computes a permutation matrix $\Pi \in \R^{m \times m}$, an upper-trapezoidal matrix $R \in \R^{\ell \times m}$ (i.e. $R_{i,j}=0 \text{ if } i > j$), and a matrix $Q \in \R^{n \times \ell}$ with orthonormal columns (i.e., $Q^\top Q = I$) such that $A \Pi = Q R$ and $Q$ and $R$ satisfy certain properties.
Splitting $\Pi, Q$, and $R$ into $\Pi_1 \in \R^{m \times k}$, $\Pi_2 \in \R^{m \times (m-k)}$, $Q_1 \in \R^{n \times k}$, $Q_2 \in \R^{n \times (\ell-k)}$, $R_{11} \in \R^{k \times k}$, 
$R_{12} \in \R^{k \times (m-k)}$, and $R_{22} \in \R^{(\ell-k)\times(m-k)}$ we can write
%$R_{12} \in \R^{k \times (\ell-m)}$, and $R_{22} \in \R^{(\ell-k)\times(\ell-n)}$ we can write
\begin{equation}
\label{eq:rankQR}
    A
    \begin{bmatrix}
    \Pi_1 & \Pi_2
    \end{bmatrix}
    =
    \begin{bmatrix}
    Q_1 & Q_2
    \end{bmatrix}
    \begin{bmatrix}
    R_{11} & R_{12} \\
    & R_{22}
    \end{bmatrix}.
\end{equation}
\end{definition}

\begin{remarks}
What makes~\eqref{eq:rankQR} a rank-revealing QR factorization is that the permutation $\Pi$ is computed to ensure that $R_{11}$ is as well-conditioned as possible and $R_{22}$ is as small as possible. While more formal statements of these conditions exist, we omit them here as they do not factor into our work.
\end{remarks}

Critically, any rank-revealing QR factorization yields a natural rank-$k$ approximation of $A$ with error
\begin{equation*}
    \| A - Q_1 
    \begin{bmatrix}
    R_{11} & R_{22}
    \end{bmatrix}
    \Pi^\top \|_2
    =
    \| R_{22} \|_2.
\end{equation*}
While finding the optimal rank-revealing QR factorization (\emph{i.e.,} minimizing the error for a given $k$) is closely related to a provably hard problem~\cite{civril2009selecting}, we find the original algorithm of Businger and Golub~\cite{businger1965linear} works well in practice. This routine is available in LAPACK~\cite{lapack,blas3QRCP}, can be easily incorporated into existing code, and has computational complexity $\mathcal{O}(nmk)$ when run for $k$ steps.  
%While optimally finding a rank-revealing QR is closely related to the provably hard problem of finding maximum volume subsets~\cite{civril2009selecting}, many good algorithms exist~\cite{businger1965linear,blas3QRCP,chandrasekaran1994rank,gu1996efficient,golub1976rank} that are effective in practice.
%More specifically, these algorithms seek to ensure that (where $\sigma$ is used to denote the appropriate singular values)
%\begin{equation*}
 %   \sigma_{\min}(R_{11})
%    \geq 
%    \frac{\sigma_k(A)}{f_1(n,k)}
%    \quad\text{and}\quad
%    \sigma_{\max}(R_{22}) 
%    \leq
%    f_2(n,k) \sigma_{k+1}(A),
%\end{equation*}
%for functions $f_1$ and $f_2$ that grow mildly in $n$ and $k$.
%(For many of the specific algorithms referenced explicit expressions of $f_1$ and $f_2$ are known.)
%For our purposes, the original algorithm of Businger and Golub~\cite{businger1965linear} suffices and has runtime $\mathcal{O}(nm\ell)$; readily available implementations exist in LAPACK~\cite{lapack,blas3QRCP}, distributed memory implementations are available (e.g., in ScaLAPAK~\cite{Scalapack}), and randomization can be used to provide efficiency gains in practice~~\cite{liberty2007randomized,duersch2017randomized,martinsson2011randomized,martinsson2017householder,randomreview}.



\paragraph{Computing interpolative decompositions}
%\subsection{Computing interpolative decompositions}
Given a rank-revealing QR factorization, we can immediately construct an ID (a formal algorithmic statement is given in the appendix).
Let $\I\subset [m]$ be such that $A_{:,\I} = A \Pi_1$ and define the interpolation matrix
\begin{equation*}
    T = 
    \begin{bmatrix}
    I_k & R_{11}^{-1} R_{12}
    \end{bmatrix}
    \Pi^\top.
\end{equation*}
With the choice $A_{:,\I} = Q_1 R_{11}$ it follows that the error of the ID as defined by $\I$ and $T$ is 
$%$\begin{align*}
    \|A - A_{:,\I}T\|_2
    %&=
    %\|Q\begin{bmatrix}
    %0 & R_{22}
    %\end{bmatrix}
    %\Pi^\top \|_2 \\
    = \|R_{22}\|_2 
$. %$\end{align*}  
Picking $k$ such that $\|R_{22}\|_2\leq \epsilon \|A\|_2$ yields the desired relative error. Notably, since $\kappa(A_{:,\I}) = \kappa(R_{11})$ and $T = 
\begin{bmatrix}
I_k & R_{11}^{-1} R_{12}
\end{bmatrix}
\Pi^\top$ 
the desired criteria for an ID map back to those of a rank-revealing QR factorization---if $R_{11}$ is well conditioned then the  basis we use for approximation is as well and entries of $T$ are not too large.
If $\sigma_{\max}(R_{22})$ is not much larger than $\sigma_{k+1}(A)$ we get near optimal approximation accuracy.







\paragraph{Accuracy of the matrix approximation}
%\subsection{Accuracy of the matrix approximation}
%\label{sec:accuracyID}
A key feature of using a column-pivoted QR factorization to compute an ID is that it allows us to dynamically determine the approximation rank $k$ as a function of $\epsilon$. 
This can be accomplished by monitoring $\|R_{22}\|_2$ at each step of the column-pivoted QR algorithm. 
However, repeatedly computing $\|R_{22}\|_2$ is expensive and often unnecessary in practice. 
When using the algorithm by Businger and Golub~\cite{businger1965linear} the magnitude of the diagonal entries of $R$ are non-increasing and it is common to use $\lvert r_{k+1,k+1} / r_{1,1}\rvert$ as a proxy for 
%$\|R_{22}\|_2/\|R\|_2$.
$\|R_{22}\|_2/\|A\|_2$.
While formal bounds indicate the approximation may be loose in the worst case, it is effective in practice (see appendix Figure~\ref{fig:chosingk}) and once a candidate $k$ has been identified $\|R_{22}\|_2$ can be computed if desired to certify the result---if the accuracy is unacceptable $k$ can be increased until it is. 
In some settings it may be preferable to fix $k$ and simply accept whatever accuracy is achieved.


% The matrix R gives us useful information about the potential accuracy vs. complexity of the decomposition. 
% In the section above, we showed that the error from our decomposition is $||R_{22}||$. 
% However, computing the value of $||R_{22}||$ for each k can be expensive. 
% In this situation, we can use properties of the column-pivoted QR factorization to speed our algorithm. 
% The largest magnitude entry in $R_{22}$ is  in the upper left-hand corner of the matrix.  
% While this does not give us a useful hard bound on $||R_{22}||/||R||$, in practice we can use this as a proxy for the relative size of $||R_{22}||/||R||$.  
% \megan{cite places this is done in practice/ well established?}.  
% Let $r_{11} = R_{1,1}$ denote the matrix $R$ indexed at $(0,0)$, the top left, and $r_{kk} = R_{k,k}$ denote the matrix $R$ indexed at the $k$-th diagonal.
% In figure {\ref{fig:chosingk}}, we see that the value of $r_{kk}/r_{11}$ tracks well with the optimal bound as determined by the singular value decay, as well as with the actual error $||R_{22}||$. 
% This allows us to control the increase in error that we create for the network locally.  

%In figure (\megan{number}), we see that the degradation in accuracy matches what we would expect given the ratio $r_{kk}/r_{00}$.  




%\jerry{Maybe some simple simulations comparing RRQR to SVD?
%Similar to Fig2 in grant.}


%Given a matrix $A$ an interpolative decomposition may be computed using any rank-revealing QR factorization, though most commonly the column-pivoted QR factorization of Golub and Businger is used \todo{cite}.
%The procedure is given in Algorithm~\ref{alg:genericID}, which we elaborate on in further detail.
%For simplicity let $l = \min(m,n)$ and assume $k < l$.
%
%Concretely, we first compute the column-pivoted QR factorization
%\begin{equation*}
%    A \Pi = Q R
%\end{equation*}
%where $\Pi \in \R^{n \times n}$ is a permutation, $Q \in \R^{m \times l}$ has orthonormal columns and $R \in \R^{l \times n}$ is upper triangular.
%We now partition $\Pi$ and $R$ as 
%\begin{equation*}
%    \Pi = 
%    \begin{bmatrix}
%    \Pi_1 & \Pi_2
%    \end{bmatrix}
%    \quad\text{and}\quad
%    R =
%    \begin{bmatrix}
%    R_{11} & R_{12} \\
%    0    & R_{22}
%    \end{bmatrix},
%\end{equation*}
%where $\Pi_1 \in \R^{n \times k}, R_{11} \in \R^{k \times k}, R_{12} \in \R^{k \times n - k}$, and the remaining dimensions are as required.
%Let $\I$ denote the columns selected by the subselection matrix $\Pi$ such that
%\begin{equation*}
%    A_{:,\I} = A \Pi_1.
%\end{equation*}
%Letting the interpolation matrix
%\begin{equation*}
%    T = 
%    \begin{bmatrix}
%    I_k & R_{11}^{-1} R_{12}
%    \end{bmatrix}
%    \Pi^\top
%\end{equation*}
%we have the interpolative decomposition.
%Note that the error is 
%\begin{align*}
%    \|A - A_{:,\I}T\|_2
%    &=
%    \|Q\begin{bmatrix}
%    0 & R_{22}
%    \end{bmatrix}
%    \Pi^\top \|_2 \\
%    &= \|R_{22}\|_2.
%\end{align*}



\section{Pruning with interpolative decompositions}
\label{sec:pruneID}
The core of our approach is a novel use of IDs to 
prune neural networks. Here we illustrate the scheme for a single fully connected layer and we extend the scheme to more complex layers (e.g., convolution layers) and deeper networks in Section~\ref{sec:extendid}.
Consider a simple two layer (one hidden layer) fully connected neural network $h_{FC} : \R^d \to \R^c$ of width $m$ defined as 
\begin{equation*}
\label{eq:1hiddenfc}
    h_{FC}(x; W, U)
    =
    U^\top g(W^\top x)
\end{equation*}
with hidden layer $W \in \R^{d \times m}$, output layer $U \in \R^{m \times c}$, and activation function $g$. 
We omit bias terms: they may be readily incorporated by adding a row to $W$ and suitably augmenting the data.

To prune the model we will use an \emph{unlabeled} pruning data set  $\{x_i\}^{n}_{i=1}$ with $x_i \in \R^d$.
Let $X \in \R^{d \times n}$ be the matrix such that $X_{:,i} = x_i$.
Preserving the action of the two layer network to accuracy $\epsilon > 0$ on the data with fewer neurons is synonymous with finding an $\epsilon$ accurate approximation
$h_{FC}(x;W,U) \approx h_{FC}(x; \widehat{W},\widehat{U})$ 
where $\widehat{W}$ has fewer columns than $W$.
We can do this by computing an ID of the activation output of the first layer.

Concretely, let $Z \in \R^{m \times n}$ be the first-layer output, i.e., $Z = g(W^\top X),$ and let $Z^\top \approx (Z^\top)_{:,\I}  T $
be a rank-$k$ ID of $Z^\top$
with $|\I| = k$ and interpolation matrix $T \in \R^{k \times m}$ that achieves accuracy $\epsilon$ as in Definition~\ref{def:ID} (note that this means $k$ is a function of $\epsilon$).
%Importantly, 
Because the activation function $g$ commutes with the sub-selection operator, if $Z \approx T^\top Z_{\I,:}$ then
\begin{align*}
    g(W^\top X) 
    \approx T^\top \left( g(W^\top X) \right)_{\I,:}
    % \approx
    % T^\top g\left(\left( W^\top X \right)_{\I,:} \right)
    =
    T^\top g\left( W_{:,\I}^\top X \right).
\end{align*}
Multiplying both sides by $U^\top$ now gives an approximation of the original network by a pruned one,
\begin{equation}
    h_{FC}\left(x; W, U \right) 
    =
    U^\top g(W^\top X) \approx \\
    h_{FC}\left(x; W_{:,\I}, T U \right)
    =
    U^\top T^\top g\left( W_{:,\I}^\top X \right).
    \label{eqn:PruneApprox}
\end{equation}
% The activation output $Z$ can then be approximated
% \begin{equation}
%     Z \approx 
%     T^\top g(W_{1(:,\I)}^\top X)
% \end{equation}
% by setting $\widehat{W}_1 = W_{1(:,\I)}$ to select neurons in the current layer.
% In order to encode the interpolation matrix $T$ into the neural network, set $\widehat{W}_2 = T W_2$ 
% %to propagate the interpolation matrix into 
% in the next layer.
That is, the ID has pruned the network of width $m$ into a dense sub-network of width $k$ with $\widehat W \equiv W_{:,\I} \in \R^{d \times k}$ and $\widehat U \equiv T U \in \R^{k \times c}$.
% \begin{equation}
% \label{eq:fcID}
%     h_{FC}(x; W_1, W_2)
%     \approx
%     h_{FC}(x; W_{1(:,\I)} , T W_2) 
% \end{equation}
%\jerry{not sure if we want to say this:}
%Furthermore, if the interpolative decomposition achieves $\epsilon$ accuracy, then the neural network achieves \megan{at least}$\epsilon \|W_2\|$ accuracy.
Importantly, 
%An important point is that 
the SVD of $Z^\top$ cannot be used for this task since
it is not clear how to map the dominant left singular vectors back through the activation function $g$ to either a subset of the existing neurons or a small set of new neurons.
%In contrast the sub-selection operator of the interpolative decomposition commutes with the activation function.
This makes use of the ID essential and we provide additional intuition for this scheme in the appendix, specifically in Figure ~\ref{fig:patches}.  




%\begin{algorithm}[t]
%\SetAlgoLined
%\DontPrintSemicolon
%\KwIn{
%FC layer $f_{FC}(x;W_1)$, 
%next layer $f_{2}(x;W_2)$,
%%intermediate layer $g(x; w_3)$ (batch norm, pooling),
%pruning data $X_{p} \in \R^{d \times n_{p}}$,
%rank-$k(\epsilon)$
%}
%\KwOut{
%pruned layers
%$\widehat{f}_{FC}(x;\widehat{W}_1)$, 
%$\widehat{f}_{2}(x;\widehat{W}_2)$,
%%$\widehat{g}(x;\widehat{w}_3)$
%}
%
%%\tcc{apply batch norm or pool layer if it follows $f_{FC1}$}
%$Z \gets \gamma(X_{p}^\top W_1)$\;
%$Z_{:,\I}, T \gets \ID(Z, k(\epsilon))$\;
%$\widehat{W}_1 \gets (W_1)_{(:,\I)}$
%\tcp{select neurons via ID}
%$\widehat{W}_2 \gets \operatorname{matmul}(T, W_2)$
%\tcp{propagate interpolation matrix to next layer}
%%\tcp{depends on if next layer is FC or Conv}
%\caption{ID Pruning a FC Layer\jerry{might get rid of}}
%\label{alg:fcID}
%\end{algorithm}

\subsection{A generalization bound for the pruned network}
% \paragraph{A generalization bound for the pruned network}
A key feature of our ID based pruning method is that it can be used to dynamically select the width of the pruned network to maintain a desired accuracy when compared with the full model. This allows us to provide generalization guarantees for the compressed network in terms of generalization of the full model and the accuracy of the ID. We state the results for a single hidden fully connected layer with scalar output (i.e., $c=1$) and squared loss. They can be extended to more complex networks (at the expense of more complicated dependence on the accuracy each layer is pruned to), more general Lipschitz continuous loss functions, and vector valued output. We defer all proofs to the supplementary material.

Assume $(x,y)\sim \cD$ where $x\in\R^d,$ $y\in\R,$ and the distribution $\cD$ is supported on a compact domain $\Omega_x\times\Omega_y$. We let $\mathcal{R}_0 = \mathbb{E}_{(x,y) \sim \cD}(\|({u}^\top g({W}^\top x)- y)\|^2)$ denote the true risk of the trained full model and $\mathcal{R}_p = \mathbb{E}_{(x,y) \sim \cD}(\|({\widehat{u}}^\top g({\widehat{W}}^\top x)- y)\|^2)$ be the risk of the pruned model. (Since $c=1$ we let $u\in\R^m$ denote the last layer.)
We also define the empirical risk $\widehat{\cR}_{ID}$ of approximating the full model with our pruned model as
\begin{equation*}
\label{eq:pruneRisk}
    \widehat{\cR}_{ID}=\frac{1}{n}\sum_{i=1}^n \left\lvert{u}^\top g({W}^\top x_i) - {\widehat{u}}^\top g({\widehat{W}}^\top x_i) \right\rvert^2,
\end{equation*}
where $\{x_i\}_{i=1}^n$ are $n$ i.i.d.\ samples from $\cD$ (note that we do not need labels for these samples). Using this notation, Theorem~\ref{thm:generalization} controls the generalization error of the pruned model.


\begin{theorem}[Single hidden layer FC]
\label{thm:generalization}
Consider a model $h_{FC}=u^\top g(W^\top x)$ with m hidden neurons and a pruned model $\widehat{h}_{FC}=\widehat{u}^\top g(\widehat{W}^\top x)$ constructed using an $\epsilon$ accurate ID with $n$ data points drawn i.i.d\ from $\cD.$ The risk of the pruned model $\mathcal{R}_p$ on a data set $(x,y) \sim D$ assuming $\cD$ is compactly supported on $\Omega_x\times\Omega$ is bounded by  
\begin{equation}
\label{eq:RiskDcomp}
    \mathcal{R}_p \leq \mathcal{R}_{ID} + \mathcal{R}_0+ 2  \sqrt{ \mathcal{R}_{ID}  \mathcal{R}_0},
\end{equation}
where $\mathcal{R}_{ID}$ is the risk associated with approximating the full model by a pruned one and with probability $1-\delta$ satisfies
\begin{align*}
    {\mathcal{R}}_{ID} 
    &\leq 
    \epsilon^2M+M(1+\|T\|_2)^2n^{-\frac{1}{2}} 
    &\left( \sqrt{2\zeta dm \log (dm)\log\frac{en}{\zeta dm \log (dm)}}+ \sqrt{\frac{\log (1/\delta)}{2}}\right).
\end{align*} 
Here, $M = \sup_{x\in\Omega_x} \|u\|_2^2 \| g(W^T x)\|_2^2$ and $\zeta$ is a universal constant that depends on $g$. %the activation function.  


\end{theorem}
Theorem~\ref{thm:generalization} is developed by considering the ID as a learning algorithm applied to the output of the full model using unlabeled pruning data. This allows us to control the risk of the pruned model in terms of the risk of the original model and the additional risk introduced by the ID. Importantly, here we can control the additional risk in terms of the empirical risk of the ID and an additive term that decays as additional pruning data is used. Lemma~\ref{lem:prunedRisk} codifies this decomposition.
% \begin{remarks} 
% The statement $
%     \mathcal{R} \leq \mathcal{R}_p + \mathcal{R}_0+ 2 
%     \sqrt{ \mathcal{R}_p  \mathcal{R}_0}
% $ in Theorem~\ref{thm:generalization} follows naturally from the squared loss and the Cauchy-Schwartz inequality and holds for all pruning schemes.
% \end{remarks}


\begin{lemma}
\label{lem:prunedRisk}
Under the assumptions of Theorem~\ref{thm:generalization}, for any $\delta\in(0,1)$, $\cR_{ID}$ satisfies  
\begin{equation*}
    \mathcal{R}_{ID} 
    \leq \widehat{\cR}_{ID} + M(1+\|T\|_2)^2n^{-\frac{1}{2}} 
     \left( \sqrt{2p\log(en/p)}+ 2^{-\frac{1}{2}}\sqrt{\log (1/\delta)}\right)
\end{equation*}
with probability $1-\delta,$ where $M = \sup_{x\in\Omega_x} \|u\| ^2 \| g(W^T x)\|^2$ and $p=\zeta dm \log (dm)$ for some universal constant $\zeta$ that depends only on the activation function.
\end{lemma}
%\begin{lemma}
%The p-dimension of the network is 
%\begin{equation}
%    p \leq \zeta dm \log (dm)
%\end{equation}
%\end{lemma}
\begin{remarks}
We believe that the second part of the bound in Lemma~\ref{lem:prunedRisk} is likely loose since it relies on a pseudo-dimension bound for fully connected neural networks. However, when pruning with an ID we only consider subsets of existent neurons and it is plausible that in this setting the upper bound for the pseudo-dimension could be improved.
\end{remarks}

Crucially, an immediate consequence of using an ID for pruning is that we can explicitly control $\cR_{ID}$ in terms of the accuracy parameter. This relation between the ID accuracy and empirical risk is given in Lemma~\ref{lem:IDEmperical} and is what allows us to express the risk of the pruned network in Theorem~\ref{thm:generalization}. 

% Since the interpolative decomposition preserves the action of the network on the pruning data, (Lemma \ref{lem:IDrisk}), the empirical risk on the pruning data is bounded. 
%We can bound each term, beginning with $\hat{R}_s$.  
\begin{lemma}
\label{lem:IDEmperical} Following the notation of Theorem~\ref{thm:generalization}, an ID pruning to accuracy $\epsilon$ yields a compressed network that satisfies
$\widehat{\cR}_{ID} \leq  \epsilon^2 \|u\|_2^2 \| g(W^T X)\|_2^2 / n,$
where $X\in\R^{d\times n}$ is a matrix whose columns are the pruning data.
\end{lemma}


%; given a fixed $k$ this could be computed to assess the quality of an upper bound. 

%Finally, we bound p.  Since the network we are using has a subset of the hypothesis class of the full network, its p-dimension must be the same or smaller.  From [\megan{cite }], we have that:

%for some constant C.  

%Putting this all together, we get:  

%\begin{equation}
%    {\mathcal{R}}_p \leq \epsilon^2 \|u\| ^2 \| g(W^T x)\|^2 +(\|u^\top g(W^\top x) \| (1+2m))^2 ( \sqrt{\frac{2Cdm \log (dm)\log\frac{eN}{Cdm \log (dm)}}{N}}+ \sqrt{\frac{\log \frac{1}{\delta}}{2N}})
%\end{equation} 






% OLD
%To demonstrate why we expect that this method will be effective, we created a simple synthetic data set for which we know the form of a relatively minimal representation.  More details about the data set and experiment can be found in the supplement. We see that the interpolative decomposition keeps a set of neurons which represents the underlying function slightly better than the original model, ignoring duplicates but taking into account differences in the bias.  Magnitude pruning keeps duplicate neurons and fails to find important information with the same number of neurons.    
%We can train an overparameterized single hidden layer network to perform well on this task, and given a good initialization scale of the parameters, the neurons do not need to move very far[cite].  
%A pruning method based on magnitude pruning will not necessarily recognize the pairs of neurons which can perform well on this task when we prune to very small network sizes, since these neurons may not have a large magnitude in the larger model.    
%
%In practice, we see that ID is able to select neurons which resemble a close to minimal representative network.  


\section{Convolutional and deep networks}
\label{sec:extendid}
\subsection{Convolution layers}
% \paragraph{Convolution layers}
%\label{sec:convid}


To prune convolution layers with the ID at the channel level we reshape the %order-4 
output tensor into a matrix where each column represents a single output channel.
After this transformation, the key idea is the same as in Section~\ref{sec:pruneID}.
%The weights of convolution layers are an order-4 tensor ($\operatorname{input channels} \times \operatorname{output channels} \times \operatorname{filter width} \times \operatorname{filter height}$).
%and their action on the activations is via a convolution.
%We first reshape the activation tensor matrix of shape $( \operatorname{filter width} \cdot \operatorname{filter height} \cdot \operatorname{input channels} ) \times \operatorname{output channels}$, and then use an interpolative decomposition to prune the output channels.
Consider a simple two layer convolution neural network defined as 
$%\begin{equation*}
    h_{\Conv}(x; W, U)
    =
    \Conv(U, g( \Conv(W, x))))
$, %\end{equation*}
%$h_{Conv}: \R^{q \times r \times s} \to \R^{q' \times r' \times s'}$
with the convolution-layer operator $\Conv$, weight tensors $W$ and $U$,
%such that the output channels of $W$ and the input channels of $U$ match,
and activation function $g$.
%Again we omit bias terms, though they may be readily incorporated.
%We omit specifying the kernel dimension, size, stride, padding, and dilation because they have no effect on the interpolative decomposition at the output channel level.
The kernel dimension, size, stride, padding, and dilation do not change the form of the ID at the output channel level.
Let $Z = g(\Conv(W, X))$ be the activation output of the first layer with unlabeled pruning data $X$, and define $\Reshape$ as the operator which reshapes a tensor into a matrix with the output channels as columns, i.e., $\Reshape(Z)\in\R^{n_i\times m_c}$ where $m_c$ is the number of channels and $n_i$ is the product of all other dimensions (e.g. in the case of a 2d convolution, $n_i$ would be $\text{width} \times \text{height} \times \text{number of examples}$). 

We now compute\footnote{When $n_i$ is large we can appeal to randomized ID algorithms~\cite{martinsson2011randomized}, or the TSQR~\cite{ballard2014tsqr}.
%if needed.
} a rank-$k$ ID
$\Reshape(Z) \approx \Reshape(Z)_{:,\I} T.$
%the reshaped activation output.
%Because the activation function $g$ and reshaping operator $\Reshape$ commute with the sub-selection operator
%\begin{equation*}
%    Z 
%    \approx
%\end{equation*}
%Define $\Reshape(Z)$ as the operator to reshape the order-4 tensor into an order-2 tensor with columns $\operatorname{output channels}$.
%The steps to prune this two layer convolution network are similar to the fully connected case.
%Compute a rank-$k(\epsilon)$ interpolative decomposition on the reshaped activation output
%$
%    \Reshape(Z) 
%    \approx
%    \Reshape(Z)_{:,\I} T
%$.
The activation function $g$ and reshape operator both commute with the sub-selection operator, so
\begin{align*}
    \Reshape(g(\Conv(W, X)))
    &\approx
    \Reshape(g(\Conv(W, X)))_{:,\I} T \\
    &=
    \Reshape(g(\Conv(W_{\I,\ldots}, X))) T,
\end{align*}
where $W_{\I,\ldots}$ denotes an indexing sub-selection of $W$ along its output-channel dimension. 
%$\I$.
Next, we need to propagate this $T$ into the next layer, which we can do with a ``matrix multiply'' by the next-layer's weights along its input channel dimension: we call this operation $\Matmul$.\footnote{If $T \in \R^{m \times n}$ and $U$ is a weight-tensor with $n$ input channels, then to compute $\Matmul(T,U)$ we: (1) reshape $U$ to be an $n \times p$ matrix for some $p$, (2) multiply the reshaped matrix by $T$, producing a $m \times p$ matrix, and (3) reshape the result back to a tensor with $m$ input channels and all other dimensions the same as $U$.}
With this, a little algebraic manipulation of our approximate equality above gives
\begin{equation}
    \Conv(U, g( \Conv(W, X)))) \\
    \approx 
    \Conv(\Matmul(T,U), g( \Conv(W_{\I,\ldots}, X)))),
\end{equation}
and so if we set $\widehat U = \Matmul(T,U)$ and $\widehat W = W_{\I,\ldots}$,
% Thus by selecting channels $\widehat{W} = W_{\I},$ where $W_{\I}$ is the reshaping of $\Reshape(Z)_{:,\I}$ back into an order 4 tensor, and propagating $T$ into the next layer as $\widehat{U} = \Matmul(T,U)$ 
we can preserve the action of the two layer network with fewer channels as
$%\begin{equation*}
%\label{eq:convID}
    h_{\Conv}(x; W, U)
    \approx
    h_{\Conv}(x; \widehat{W}, \widehat{U}).
$ %\end{equation*}
This gives us a recipe for pruning convolution layers analogous to (\ref{eqn:PruneApprox}).
This recipe can be directly applied to a composition of a convolution layer followed by a pooling layer~\cite{goodfellow2016deep} (or any other layer that acts independently by channel) by treating the conv layer/ pooling layer pair as a single convolution layer with a ``fancy'' activation function $g$: we just run the ID on the output post-pooling, and use that to sub-select the convolution layer's weights.
Flatten layers, for connecting to FC layers, are equally straightforward.

%It is straightforward to extend this to arbitrary compositions of linear and convolutional layers: for more details, see Algorithm~\ref{alg:deepID}.

% Here, $\Matmul(\ )$ applies a batch matrix multiplication between $T$ and $\Reshape(U)$ followed by reshaping to the original order-4 tensor dimensions.\footnote{
% %So far we have discussed how to prune two fully connected layers (FC-FC), and two convolution layers (Conv-Conv).
% For a convolution layer followed by a fully connected layer, % (Conv-FC),
% the details of the ID and channel selection remain the same.
% To propagate $T$, 
% %define the propagation function
% let $\Matmul(T, W_2) = (T \otimes I_a) W_2$ with $a = \text{input channels} * \text{filter width} * \text{filter height}$ and $\otimes$ the Kronecker product.
% }
%\begin{equation}
%    Z 
%    \approx
%    T^\top g(\Conv(W_{1(:,\I,:,:)}, X))
%\end{equation}

%\jerry{How much detail for Conv algebra: full details a bit messy.}


%\begin{algorithm}[t]
%\SetAlgoLined
%\DontPrintSemicolon
%\KwIn{
%Conv layer $f_{Conv}(x;W_1)$,
%next layer $f_{2}(x;W_2)$,
%%intermediate layer $g(x; w_3)$ (batch norm, pooling),
%pruning data $X_{p} \in \R^{d \times n_{p}}$,
%rank-$k(\epsilon)$
%}
%\KwOut{
%pruned layers
%$\widehat{f}_{Conv}(x;\widehat{W}_1)$,
%$\widehat{f}_{2}(x;\widehat{W}_2)$,
%%$\widehat{g}(x;\widehat{w}_3)$
%}
%
%%\tcc{apply batch norm or pool layer if it follows $f_{Conv1}$}
%$Z \gets f_{Conv1}(X_{p},W_1)$\;
%$Z \gets Z.\operatorname{reshape}(\operatorname{output channels}, \operatorname{filter width} \cdot \operatorname{filter height} \cdot \operatorname{input channels})$\;
%$Z_{:,\I}, T \gets \ID(Z, k(\epsilon))$\;
%$\widehat{W}_1 \gets (W_1)_{(:,\I,:,:)}$
%\tcp{select output channels via ID}
%$\widehat{W}_2 \gets \operatorname{matmul}(T, W_2)$
%\tcp{propagate interpolation matrix to next layer}
%%\tcp{depends on if next layer is FC or Conv}
%%\todo{presentation} \;
%%$\widehat{W}_2 \gets W_2.\operatorname{reshape}(\operatorname{output channels}, \operatorname{filter width}, \operatorname{filter height},  \operatorname{input channels})$\;
%%$\widehat{W}_2 \gets \operatorname{batch broadcast matmul}(T, \widehat{W}_2)$\;
%%$\widehat{W}_2 \gets W_2.\operatorname{reshape}(\operatorname{input channels}, \operatorname{output channels}, \operatorname{filter width}, \operatorname{filter height})$\;
%\caption{ID Pruning a Conv Layer\jerry{might get rid of}}
%\label{alg:convID}
%\end{algorithm}




\subsection{Deep networks}
% \paragraph{Deep networks}
\label{sec:deepid}

\begin{algorithm}[t]{\small
%\SetAlgoLined
%\DontPrintSemicolon
\caption{Pruning a multilayer network with interpolative decompositions}
\begin{algorithmic}[1]
\label{alg:deepID}
%\INPUT
\REQUIRE
Neural net $h(x; W^{(1)},\ldots,W^{(L)})$,
layers to not prune $S \subset [L]$,
pruning set $X$,
%prune mode $\in \{\operatorname{acc},\operatorname{frac}\}$,
pruning fraction $\alpha$
%Ordered list 
%index list $L$ 
%of layers to be pruned $\{(l_{i}, l_{i+1})\}_{i \in L}$ and activation outputs $\{Z_i\}_{i \in L}$, \\
%prune mode $\in \{\operatorname{frac}, \operatorname{acc}\}$, accuracy $\epsilon$
%\OUTPUT
\ENSURE
Pruned network $h(x; \widehat{W}^{(1)},\ldots,\widehat{W}^{(L)})$
%Pruned layers $\{(\hat{l}_{i}, \hat{l}_{i+1})\}_{i \in L}$
\vspace{0.5em}
%Absorb batch norm layers into their preceding layer\;
%$\operatorname{acc}$: set $k$ to achieve desired accuracy $\epsilon$ or $\operatorname{frac}$: as $\epsilon$ fraction of neurons \;
\STATE $T^{(0)} \gets I$ \;
\FOR{$l \in \{1 \dots L\}$}
\STATE $Z \gets h_{1:l}(X; W^{(1)}, \dots, W^{(l)})$
\COMMENT{layer l activations}
%\COMMENT{compute activations of layer l}
\IF{layer $l$ is a FC layer}
\STATE $(\I, T^{(l)}) \gets \operatorname{ID}(Z^T; \alpha) \textbf{ if } l \notin S \textbf{ else } (:, I)$ \;
%prune $\alpha\%$ of neurons with ID of $Z^\top$: $\I, T$\;
%compute rank-$k$ ID of $Z^\top$: $\I, T$\;
\STATE $\widehat{W}^{(l)} \gets T^{(l-1)} W^{(l)}_{:,\I}$
\COMMENT{sub-select neurons, multiply T of prev layer's ID}
%\tcp{select neurons in current layer}
%$\widehat{W}^{(l+1)} \gets T \widehat{W}^{(l+1)}$
%\;
%\tcp{propagate T to next layer}
\ELSIF{layer l is a Conv layer (or Conv+Pool)}
\STATE $(\I, T^{(l)}) \gets \operatorname{ID}(\Reshape(Z); \alpha) \textbf{ if } l \notin S \textbf{ else } (:, I)$ \;
% prune $\alpha\%$ of channels with ID of $\Reshape(Z)$: $\I, T$\;
%compute rank-$k$ ID of $\Reshape(Z)$: $\I, T$\;
\STATE $\widehat{W}^{(l)} \gets \Matmul(T^{(l-1)}, W^{(l)}_{\I,\ldots})$
%\;
\COMMENT{select channels; multiply T} %in current layer
% $\widehat{W}^{(l+1)} \gets \Matmul(T, \widehat{W}^{(l+1)})$
%\tcp{propagate T to next layer}
%\tcp{depends if next layer is FC or Conv}
\ELSIF{layer l is a Flatten layer}
\STATE $T^{(l)} \gets T^{(l-1)} \otimes I \,\,$ 
\COMMENT{expand T to have the expected size}
\ENDIF
\ENDFOR
\end{algorithmic}
%%Specify direction\;
%%How to compute Z\;
%\caption{ID pruning a multi-layer neural network}
}\end{algorithm}


%% OLD Version
%%\begin{algorithm}[t]
%%\caption{ID pruning a multi-layer neural network}
%%%\caption{Pruning a multilayer network with interpolative decompositions}
%%\label{alg:deepID_acc}
%%\begin{algorithmic}
%%\INPUT
%%Neural network $h(x; W^{(1)},W^{(2)},\dots,W^{(L)})$,
%%layers to not prune $S \subset [L]$,\\
%%%layers to skip during pruning $S \subset [L]$,\\
%%pruning set $X$,
%%%prune mode $\in \{\operatorname{acc},\operatorname{frac}\}$,
%%pruning proportion $\alpha$
%%%Ordered list 
%%%index list $L$ 
%%%of layers to be pruned $\{(l_{i}, l_{i+1})\}_{i \in L}$ and activation outputs $\{Z_i\}_{i \in L}$, \\
%%%prune mode $\in \{\operatorname{frac}, \operatorname{acc}\}$, accuracy $\epsilon$
%%\OUTPUT
%%Pruned network $h(x; \widehat{W}^{(1)},\widehat{W}^{(2)},\dots,\widehat{W}^{(L)})$
%%%Pruned layers $\{(\hat{l}_{i}, \hat{l}_{i+1})\}_{i \in L}$
%%\vspace{0.5em}
%%\hrule
%%\vspace{0.5em}
%%%Absorb batch norm layers into their preceding layer\;
%%%$\operatorname{acc}$: set $k$ to achieve desired accuracy $\epsilon$ or $\operatorname{frac}$: as $\epsilon$ fraction of neurons \;
%%\FOR{$l \in \{1,\dots L\}$}
%%\STATE $\widehat{W}^{(l)} \gets W^{(l)}$
%%\FOR{$l \in \{1 \dots L\} \setminus S$}
%%\IF{layer l is a FC layer}
%%\STATE $Z \gets h_{1:l}(X; W^{(1)}, \dots, W^{(l)})$
%%\COMMENT{output of layer l}
%%\STATE prune $\alpha\%$ of neurons with ID of $Z^\top$: $\I, T$\;
%%%compute rank-$k$ ID of $Z^\top$: $\I, T$\;
%%\STATE $\widehat{W}^{(l)} \gets \widehat{W}^{(l)}_{:,\I}$
%%\;
%%%\tcp{select neurons in current layer}
%%\STATE $\widehat{W}^{(l+1)} \gets T \widehat{W}^{(l+1)}$
%%%\tcp{propagate T to next layer}
%%\ELSIF{layer l is a Conv layer}
%%\STATE $Z \gets h_{1:l}(X; W^{(1)}, \dots, W^{(l)})$\;
%%\STATE prune $\alpha\%$ of channels with ID of $\Reshape(Z)$: $\I, T$\;
%%%compute rank-$k$ ID of $\Reshape(Z)$: $\I, T$\;
%%$\widehat{W}^{(l)} \gets \widehat{W}^{(l)}_{(:,\I,:,:)}$
%%%\;
%%\COMMENT{select channels} %in current layer
%%\STATE $\widehat{W}^{(l+1)} \gets \Matmul(T, \widehat{W}^{(l+1)})$
%%\;
%%\ENDIF
%%%\tcp{propagate T to next layer}
%%%\tcp{depends if next layer is FC or Conv}
%%\ENDFOR
%%\ENDFOR
%%\end{algorithmic}
%%%\SetAlgoLined
%%%\DontPrintSemicolon
%%%%Specify direction\;
%%%%How to compute Z\;
%%\end{algorithm}



The ID pruning primitives for fully connected and convolution layers can now be composed together to prune deep networks.
Algorithm~\ref{alg:deepID} specifies how we chain together the fully connected and convolution primitives to prune feedforward networks, for simplicity we assume for the moment we know the desired layer sizes.
%of arbitrary composition.
%In contrast most pruning methods act on a single layer at a time, and
%only modify the weights in the current layer, and 
%do not make any corrective changes to the next layers.
A multi-layer neural network is pruned from the beginning to the end, where the ID is used to approximate the outputs of the original network.
The ID pruning primitives sub-select neurons (or channels) in the current layer and propagate the corrective interpolation matrix to the next layer.
There are many ways one could prune a multi-layer network with these ID pruning primitives;
we selected the approach in Algorithm~\ref{alg:deepID} through empirical observations (though we do not assert that it is optimal).
% Discussion on the experiments which guided our design choices can be found in the supplement.
%The activation outputs are computed using a held out pruning set with either the original network, or the pruned network.
%For deep networks we found that it was crucial to use the activation outputs from the original network to mitigate harmful cascading approximation errors across the layers.
Note that as a pre-processing step before running Algorithm~\ref{alg:deepID}, batch normalization layers~\cite{batchnorm} should be absorbed into their preceding fully-connected or convolution layers, and dropout~\cite{srivastava2014dropout} layers should be removed.
% Pooling layers~\cite{goodfellow2016deep} operate at the neuron or channel level, and can be incorporated into computing $Z$ without affecting the interpolative decomposition.
% We also consider residual networks~\cite{resnet} which consist of blocks with two or more convolution layers and a skip connection: for such residual blocks we prune all but the last layer in the block.


\paragraph{Iterative Pruning}
While Algorithm~\ref{alg:deepID} is illustrative, in practice we would often like to be able to either specify a desired accuracy or choose layer sizes optimally for a desired compression ratio. Our approach allows us to accomplish this by iteratively selecting layers to compress. We introduce a score function for layers that is the ratio of the estimated relative error $\lvert r_{k+1,k+1} / r_{1,1}\rvert$ introduced by compressing a layer to the number of flops $f_l$ that would be cut if we pruned a layer $l$ to size $k$. We call this score $s_l(k)= \lvert r_{k+1,k+1} / r_{1,1}\rvert /f_l$ and it is heuristic for the compressability of each layer---lower scores imply a layer is easier to compress. However, different layers of the network are connected, and compressing a layer early in the network can effect how well later layers can be compressed.  
Therefore, we prune the network iteratively, measuring the score for each layer at a given pruning percentage (or step size), choosing the layer with the lowest score, pruning it, and then re-calculating the scores for the later layers. 
We repeat the process until the network reaches a desired compression, or until the network performance degrades unacceptably. 
% This method is particularly useful for sequential networks where the layer sizes were not chosen efficiently.
We refer to this method as Iterative ID, and refer to cutting a constant fraction of all neurons in each layer as Constant Fraction ID.  
For full details see Appendix~\ref{app:sec:iterativeID} and Algorithm~\ref{alg:deepIDIter}.
%\todo[inline]{Put IDIter algorithm psuedo code somewhere}

%\begin{algorithm}[t]{\small
%%\SetAlgoLined
%%\DontPrintSemicolon
%\KwIn{
%Neural net $h(x; W^{(1)},\ldots,W^{(L)})$,
%layers to not prune $S \subset [L]$,
%pruning set $X$,
%%prune mode $\in \{\operatorname{acc},\operatorname{frac}\}$,
%pruning fraction $\alpha$
%%Ordered list 
%%index list $L$ 
%%of layers to be pruned $\{(l_{i}, l_{i+1})\}_{i \in L}$ and activation outputs $\{Z_i\}_{i \in L}$, \\
%%prune mode $\in \{\operatorname{frac}, \operatorname{acc}\}$, accuracy $\epsilon$
%}
%\KwOut{
%Pruned network $h(x; \widehat{W}^{(1)},\ldots,\widehat{W}^{(L)})$
%%Pruned layers $\{(\hat{l}_{i}, \hat{l}_{i+1})\}_{i \in L}$
%}
%\vspace{0.5em}
%%Absorb batch norm layers into their preceding layer\;
%%$\operatorname{acc}$: set $k$ to achieve desired accuracy $\epsilon$ or $\operatorname{frac}$: as $\epsilon$ fraction of neurons \;
%\textbf{set} $T^{(0)} \gets I$ \;
%\For{$l \in \{1 \dots L\}$}{
%$Z \gets h_{1:l}(X; W^{(1)}, \dots, W^{(l)})$
%\hfill\tcp{compute activations of layer l}
%\uIf{layer $l$ is a FC layer}{
%$(\I, T^{(l)}) \gets \operatorname{InterpolativeDecomposition}(Z^T; \alpha) \textbf{ if } l \notin S \textbf{ else } (:, I)$ \;
%%prune $\alpha\%$ of neurons with ID of $Z^\top$: $\I, T$\;
%%compute rank-$k$ ID of $Z^\top$: $\I, T$\;
%$\widehat{W}^{(l)} \gets T^{(l-1)} W^{(l)}_{:,\I}$
%\hfill\tcp{sub-select neurons, multiply T of prev layer's ID}
%%\tcp{select neurons in current layer}
%%$\widehat{W}^{(l+1)} \gets T \widehat{W}^{(l+1)}$
%%\;
%%\tcp{propagate T to next layer}
%}
%\uElseIf{layer l is a Conv layer (or Conv+Pool)}{
%$(\I, T^{(l)}) \gets \operatorname{InterpolativeDecomposition}(\Reshape(Z); \alpha) \textbf{ if } l \notin S \textbf{ else } (:, I)$ \;
%% prune $\alpha\%$ of channels with ID of $\Reshape(Z)$: $\I, T$\;
%%compute rank-$k$ ID of $\Reshape(Z)$: $\I, T$\;
%$\widehat{W}^{(l)} \gets \Matmul(T^{(l-1)}, W^{(l)}_{\I,\ldots})$
%%\;
%\hfill\tcp{select channels; multiply T} %in current layer
%% $\widehat{W}^{(l+1)} \gets \Matmul(T, \widehat{W}^{(l+1)})$
%%\tcp{propagate T to next layer}
%%\tcp{depends if next layer is FC or Conv}
%}
%\uElseIf{layer l is a Flatten layer}{
%    $T^{(l)} \gets T^{(l-1)} \otimes I \,\,$ \hfill\tcp{expand T to have the expected size}
%}
%}
%%Specify direction\;
%%How to compute Z\;
%\caption{ID pruning a multi-layer neural network}
%%\caption{Pruning a multilayer network with interpolative decompositions}
%\label{alg:deepID}
%}\end{algorithm}
%
%
%\begin{algorithm}[t]
%\SetAlgoLined
%\DontPrintSemicolon
%\KwIn{
%Neural network $h(x; W^{(1)},W^{(2)},\dots,W^{(L)})$,
%layers to skip during pruning $S \subset [L]$,\\
%pruning set $X$,
%%prune mode $\in \{\operatorname{acc},\operatorname{frac}\}$,
%pruning proportion $\alpha$
%%Ordered list 
%%index list $L$ 
%%of layers to be pruned $\{(l_{i}, l_{i+1})\}_{i \in L}$ and activation outputs $\{Z_i\}_{i \in L}$, \\
%%prune mode $\in \{\operatorname{frac}, \operatorname{acc}\}$, accuracy $\epsilon$
%}
%\KwOut{
%Pruned network $h(x; \widehat{W}^{(1)},\widehat{W}^{(2)},\dots,\widehat{W}^{(L)})$
%%Pruned layers $\{(\hat{l}_{i}, \hat{l}_{i+1})\}_{i \in L}$
%}
%\vspace{0.5em}
%\hrule
%\vspace{0.5em}
%%Absorb batch norm layers into their preceding layer\;
%%$\operatorname{acc}$: set $k$ to achieve desired accuracy $\epsilon$ or $\operatorname{frac}$: as $\epsilon$ fraction of neurons \;
%\lFor{$l \in \{1,\dots L\}$}{
%$\widehat{W}^{(l)} \gets W^{(l)}$
%}
%\For{$l \in \{1 \dots L\} \setminus S$}{
%\uIf{layer l is a FC layer}{
%$Z \gets h_{1:l}(X; W^{(1)}, \dots, W^{(l)})$
%\tcp{output of layer l}
%prune $\alpha\%$ of neurons with ID of $Z^\top$: $\I, T$\;
%%compute rank-$k$ ID of $Z^\top$: $\I, T$\;
%$\widehat{W}^{(l)} \gets \widehat{W}^{(l)}_{:,\I}$
%\;
%%\tcp{select neurons in current layer}
%$\widehat{W}^{(l+1)} \gets T \widehat{W}^{(l+1)}$
%\;
%%\tcp{propagate T to next layer}
%}
%\uElseIf{layer l is a Conv layer}{
%$Z \gets h_{1:l}(X; W^{(1)}, \dots, W^{(l)})$\;
%prune $\alpha\%$ of channels with ID of $\Reshape(Z)$: $\I, T$\;
%%compute rank-$k$ ID of $\Reshape(Z)$: $\I, T$\;
%$\widehat{W}^{(l)} \gets \widehat{W}^{(l)}_{(:,\I,:,:)}$
%%\;
%\tcp{select channels} %in current layer
%$\widehat{W}^{(l+1)} \gets \Matmul(T, \widehat{W}^{(l+1)})$
%\;
%%\tcp{propagate T to next layer}
%%\tcp{depends if next layer is FC or Conv}
%}
%}
%%Specify direction\;
%%How to compute Z\;
%\caption{ID pruning a multi-layer neural network}
%%\caption{Pruning a multilayer network with interpolative decompositions}
%\label{alg:deepID}
%\end{algorithm}

%\section{Contextualizing our work}
%\label{sec:context}
%% \begin{table}[h!]
% \centering
% \begin{tabular}{|c c c c| c c c|} 
%  \hline
%  & & & & \multicolumn{3}{c|}{Compression type} \\ [0.5ex]
%  & & & & \multirow{3}{*}{Sparse} & \multicolumn{2}{|c|}{Dense} \\
%  & & & & & \multicolumn{2}{|c|}{Preserves net structure?} \\ 
%  & & & & & \multicolumn{1}{|c}{Yes} & \multicolumn{1}{|c|}{No} \\ [0.5ex] %extra space to next row
%  \hline%\hline
%  \multirow{3}{*}{\rotatebox[origin=c]{90}{Correction}} & & & 
%  None & Cat. A & Cat. B & Cat. C \\ [1ex]
%  \cline{2-4}
%  & \multicolumn{2}{c}{Local}
%  %& \multirow{2}{*}{\rotatebox[origin=c]{90}{Local}} 
%  %& \multirow{2}{*}{\rotatebox[origin=c]{90}{FT?}}
%  & No & Cat. D & Cat. E & Cat. F \\ [1ex]
%  \cline{4-4}
%  & \multicolumn{2}{c}{FT?}
%  & Yes & Cat. G & Cat. H & Cat. I\\ [1ex]
%  \hline
% % \multirow{6}{*}{\rotatebox[origin=c]{90}{Correction}} & 
% % \multirow{2}{*}{None} & weight & channel & low- \\
% % & & prune & prune & rank \\ [1ex]
% % \cline{2-5}
% % & Yes, w/out &\multirow{2}{*}{N/A} & \multirow{2}{*}{ID} & \multirow{2}{*}{N/A}\\
% % & locFT &  & & \\ [1ex]
% % \cline{2-5}
% % & Yes, w/ & \multirow{2}{*}{WP+} & \multirow{2}{*}{CP+} & \multirow{2}{*}{LR+}\\
% % & locFT &  & & \\ [1ex]
% % \hline
% \end{tabular}
% \caption{Taxonomy of parameter compressing methods via 2 axes: the type of compression (how parameters are removed), and what type of correction is taken to improve the network after removing parameters.
% Often, local fine tuning is interleaved into compression methods.
% }
% \label{tab:taxonomy}
% \end{table}


%Relative to existing techniques, our use of the interpolative decomposition provides several key advantages. 
To facilitate a careful discussion of how our method fits within the current literature, Table~\ref{tab:taxonomy} provides a taxonomy of parameter space compression methods.
%Our formulation of compression as preserving the per-example labels necessitates us to focus on the paradigm of compressing a pre-trained model.
We focus on preserving the per-example labels of a pre-trained model. 
Thus we ignore methods which do not take a pre-trained model as input.
The ID~(Cat. E) combines benefits of both low-rank and channel pruning methods, while also incorporating a parameter correction done without additional local fine tuning.
Unlike low-rank methods~(Cat. F), the ID preserves the computational structure of the network.
This allows us to trivially compose the ID with other compression methods, achieving model recovery comparable to that of global fine tuning.
And unlike channel pruning methods~(Cat. B), the ID can fully recover the original model at minimal FLOPs reduction, without any fine tuning. 
%Low-rank methods modify the network structure by decomposing the weight matrices, and do not directly incorporate a corrective step.
%Channel pruning methods retain the network structure, but often poorly preserve model performance without  fine tuning (whether local or global).
%A key feature of our method is that it retains the network structure---just reducing the width---and, therefore, is able to leverage the computational benefits of specific network architectures.




%We can also prune to a fixed compression level and accept the resulting degradation in accuracy.  
%While this process requires additional data, it can be unlabeled data as our process does not require ``ground truth'' labels; in fact, it effectively uses pseudo-labels generated by the trained model. 
% Jerry removed 1/25/22
%Another key feature of our method is that it retains the network structure---just reducing the width---and, therefore, is able to leverage the computational benefits of specific network architectures.

In our taxonomy, we choose to separate any fine tuning (i.e., optimization) steps from the parameter reduction step.
What we call ``fine tuning'' can take two primary forms.
%: either locally interleaved into the parameter reduction, or performed end-to-end globally on the reduced model.
We define ``global'' fine tuning as optimizing the end-to-end loss.
``Local'' fine tuning is where per-layer structures are optimized.
It is valuable to compare compression methods that use similar a type and amount of fine tuning, as well as to compare before and after its application.
%This allows us to better evaluate whether accuracies are achieved due to a fine-tuning technique or due to the compression method itself.
%We do not interleave local fine tuning with the ID in order to better understand the contribution of our compression method.
%However, we recognize that local fine tuning is a powerful method. 
%Because the interpolative decomposition preserves the model structure, we can compose other methods that use local fine tuning on top of it.  
%We demonstrate the efficacy of this approach in section \ref{sec:experiments}.  


\section{Evaluating compression beyond accuracy}
\label{sec:correlation}
An important benefit of our approach is that we are actually able to preserve the original model's predictions better than other methods.
Traditional pruning methods typically do a poor job at preserving the original predictions, due to their heavy reliance on fine tuning that effectively retrains the model.
Here we explain why we might care about preserving per-example predictions beyond top-line accuracy.
We argue that in many situations compression methods must well-approximate the original model, and that accuracy is a poor metric for this use case.

% What does it mean to ``compress'' a model?
% There are two criteria. 
% First, the compressed model must be smaller than the original, e.g. in memory of in inference runtime.
% %We have no qualm with how the reduction in size is measured.
% Second, the compressed model must well-approximate the original.
% We argue that accuracy is a poor metric for measuring if one model well-approximates another.

Consider a pretrained model $M$, a resulting ``compressed'' model $M_C$, and an evaluation set $(X,Y)$.
Accuracy measures the similarity between the true labels $Y$ and the predicted labels from the compressed model $M_C(X)$.
This metric does not directly compare the original model $M$ and compressed model $M_C$.
As we have seen in Section~\ref{sec:intro}, a compressed model can recover the accuracy of the original model, but still differ widely on the predictions.
Instead our model correlation measures the similarity between the original model's predictions $M(X)$ and the compressed model's predictions $M_C(X)$.
One can think of this metric as an ``accuracy to the original predictions'', instead of an accuracy to the true labels.

We propose model correlation as the percent of test example predictions two models agree on---details of this metric are discussed in Appendix~\ref{app:sec:correlation}.
Model correlation is a general metric to measure the similarity between the learned functions 
%decision boundaries
of two models.
% However we propose model correlation as a generic metric to measure preservation between a pre-trained and compressed model.
Our claim is that by better preserving a model's per-example decisions, we can better preserve special properties of the model.
In the following section we provide experimental evidence for this claim.
Models are now often trained to have properties that go beyond test accuracy---for example robustness to adversarial attacks, sub-class classification accuracy, fairness, etc., and this measure of model correlation is likely to correlate with many of these criteria.
% Fairness and robustness properties are two such specific examples.
% Better preserving specialize fairness or adversarial robustness properties is a corollary of better preserving the original model's learned function.
Note that we do not believe preserving per-example decisions ``boosts'' any of these properties, we are simply preserving properties of the baseline model.







% We propose a set of evaluation metrics for pruning methods that goes beyond simple post-fine tuning test accuracy. The first metric we define is correlation between two models, defined as the percent of test example predictions the two models agree on---details of this metric are discussed in Appendix~\ref{app:sec:correlation}.  We begin with a pretrained model, $M$, and compress it to create a model $M_c$.  We define model correlation as an aggregate measure of how well $M_c$ preserves the per-example decisions of $M$ on the test set.  This captures information about both examples which $M$  correctly and incorrectly classified.  Models are now often trained to have properties that go beyond test accuracy---for example robustness to adversarial attacks, sub-class classification accuracy, fairness, etc. On a practical level, no single number could perfectly capture the preservation on every possible metric. Our claim is that by better preserving per-example decisions, we may better preserve other special properties of the model. In order to demonstrate this, we show how our method can be used to prune a network while maintaining sub-class accuracy when neither the pruning nor fine-tuning methods have access to data from one of the classes in Appendix \ref{sec:sensitivity}.




\section{Experiments}
\label{sec:experiments}
In this section we conduct comprehensive experiments to emphasise the effectiveness of DIAL, including evaluations under white-box and black-box settings, robustness to unforeseen adversaries, robustness to unforeseen corruptions, transfer learning, and ablation studies. Finally, we present a new measurement to test the balance between robustness and natural accuracy, which we named $F_1$-robust score. 


\subsection{A case study on SVHN and CIFAR-100}
In the first part of our analysis, we conduct a case study experiment on two benchmark datasets: SVHN \citep{netzer2011reading} and CIFAR-100 \cite{krizhevsky2009learning}. We follow common experiment settings as in \cite{rice2020overfitting, wu2020adversarial}. We used the PreAct ResNet-18 \citep{he2016identity} architecture on which we integrate a domain classification layer. The adversarial training is done using 10-step PGD adversary with perturbation size of 0.031 and a step size of 0.003 for SVHN and 0.007 for CIFAR-100. The batch size is 128, weight decay is $7e^{-4}$ and the model is trained for 100 epochs. For SVHN, the initial learinnig rate is set to 0.01 and decays by a factor of 10 after 55, 75 and 90 iteration. For CIFAR-100, the initial learning rate is set to 0.1 and decays by a factor of 10 after 75 and 90 iterations. 
%We compared DIAL to \cite{madry2017towards} and TRADES \citep{zhang2019theoretically}. 
%The evaluation is done using Auto-Attack~\citep{croce2020reliable}, which is an ensemble of three white-box and one black-box parameter-free attacks, and various $\ell_{\infty}$ adversaries: PGD$^{20}$, PGD$^{100}$, PGD$^{1000}$ and CW$_{\infty}$ with step size of 0.003. 
Results are averaged over 3 restarts while omitting one standard deviation (which is smaller than 0.2\% in all experiments). As can be seen by the results in Tables~\ref{black-and_white-svhn} and \ref{black-and_white-cifar100}, DIAL presents consistent improvement in robustness (e.g., 5.75\% improved robustness on SVHN against AA) compared to the standard AT 
%under variety of attacks 
while also improving the natural accuracy. More results are presented in Appendix \ref{cifar100-svhn-appendix}.


\begin{table}[!ht]
  \caption{Robustness against white-box, black-box attacks and Auto-Attack (AA) on SVHN. Black-box attacks are generated using naturally trained surrogate model. Natural represents the naturally trained (non-adversarial) model.
  %and applied to the best performing robust models.
  }
  \vskip 0.1in
  \label{black-and_white-svhn}
  \centering
  \small
  \begin{tabular}{l@{\hspace{1\tabcolsep}}c@{\hspace{1\tabcolsep}}c@{\hspace{1\tabcolsep}}c@{\hspace{1\tabcolsep}}c@{\hspace{1\tabcolsep}}c@{\hspace{1\tabcolsep}}c@{\hspace{1\tabcolsep}}c@{\hspace{1\tabcolsep}}c@{\hspace{1\tabcolsep}}c@{\hspace{1\tabcolsep}}c}
    \toprule
    & & \multicolumn{4}{c}{White-box} & \multicolumn{4}{c}{Black-Box}  \\
    \cmidrule(r){3-6} 
    \cmidrule(r){7-10}
    Defense Model & Natural & PGD$^{20}$ & PGD$^{100}$  & PGD$^{1000}$  & CW$^{\infty}$ & PGD$^{20}$ & PGD$^{100}$ & PGD$^{1000}$  & CW$^{\infty}$ & AA \\
    \midrule
    NATURAL & 96.85 & 0 & 0 & 0 & 0 & 0 & 0 & 0 & 0 & 0 \\
    \midrule
    AT & 89.90 & 53.23 & 49.45 & 49.23 & 48.25 & 86.44 & 86.28 & 86.18 & 86.42 & 45.25 \\
    % TRADES & 90.35 & 57.10 & 54.13 & 54.08 & 52.19 & 86.89 & 86.73 & 86.57 & 86.70 &  49.50 \\
    $\DIAL_{\kl}$ (Ours) & 90.66 & \textbf{58.91} & \textbf{55.30} & \textbf{55.11} & \textbf{53.67} & 87.62 & 87.52 & 87.41 & 87.63 & \textbf{51.00} \\
    $\DIAL_{\ce}$ (Ours) & \textbf{92.88} & 55.26  & 50.82 & 50.54 & 49.66 & \textbf{89.12} & \textbf{89.01} & \textbf{88.74} & \textbf{89.10} &  46.52  \\
    \bottomrule
  \end{tabular}
\end{table}


\begin{table}[!ht]
  \caption{Robustness against white-box, black-box attacks and Auto-Attack (AA) on CIFAR100. Black-box attacks are generated using naturally trained surrogate model. Natural represents the naturally trained (non-adversarial) model.
  %and applied to the best performing robust models.
  }
  \vskip 0.1in
  \label{black-and_white-cifar100}
  \centering
  \small
  \begin{tabular}{l@{\hspace{1\tabcolsep}}c@{\hspace{1\tabcolsep}}c@{\hspace{1\tabcolsep}}c@{\hspace{1\tabcolsep}}c@{\hspace{1\tabcolsep}}c@{\hspace{1\tabcolsep}}c@{\hspace{1\tabcolsep}}c@{\hspace{1\tabcolsep}}c@{\hspace{1\tabcolsep}}c@{\hspace{1\tabcolsep}}c}
    \toprule
    & & \multicolumn{4}{c}{White-box} & \multicolumn{4}{c}{Black-Box}  \\
    \cmidrule(r){3-6} 
    \cmidrule(r){7-10}
    Defense Model & Natural & PGD$^{20}$ & PGD$^{100}$  & PGD$^{1000}$  & CW$^{\infty}$ & PGD$^{20}$ & PGD$^{100}$ & PGD$^{1000}$  & CW$^{\infty}$ & AA \\
    \midrule
    NATURAL & 79.30 & 0 & 0 & 0 & 0 & 0 & 0 & 0 & 0 & 0 \\
    \midrule
    AT & 56.73 & 29.57 & 28.45 & 28.39 & 26.6 & 55.52 & 55.29 & 55.26 & 55.40 & 24.12 \\
    % TRADES & 58.24 & 30.10 & 29.66 & 29.64 & 25.97 & 57.05 & 56.71 & 56.67 & 56.77 & 24.92 \\
    $\DIAL_{\kl}$ (Ours) & 58.47 & \textbf{31.19} & \textbf{30.50} & \textbf{30.42} & \textbf{26.91} & 57.16 & 56.81 & 56.80 & 57.00 & \textbf{25.87} \\
    $\DIAL_{\ce}$ (Ours) & \textbf{60.77} & 27.87 & 26.66 & 26.61 & 25.98 & \textbf{59.48} & \textbf{59.06} & \textbf{58.96} & \textbf{59.20} & 23.51  \\
    \bottomrule
  \end{tabular}
\end{table}


% \begin{table}[!ht]
%   \caption{Robustness comparison of DIAL to Madry et al. and TRADES defense models on the SVHN dataset under different PGD white-box attacks and the ensemble Auto-Attack (AA).}
%   \label{svhn}
%   \centering
%   \begin{tabular}{llllll|l}
%     \toprule
%     \cmidrule(r){1-5}
%     Defense Model & Natural & PGD$^{20}$ & PGD$^{100}$ & PGD$^{1000}$ & CW$_{\infty}$ & AA\\
%     \midrule
%     $\DIAL_{\kl}$ (Ours) & $\mathbf{90.66}$ & $\mathbf{58.91}$ & $\mathbf{55.30}$ & $\mathbf{55.12}$ & $\mathbf{53.67}$  & $\mathbf{51.00}$  \\
%     Madry et al. & 89.90 & 53.23 & 49.45 & 49.23 & 48.25 & 45.25  \\
%     TRADES & 90.35 & 57.10 & 54.13 & 54.08 & 52.19 & 49.50 \\
%     \bottomrule
%   \end{tabular}
% \end{table}


\subsection{Performance comparison on CIFAR-10} \label{defence-settings}
In this part, we evaluate the performance of DIAL compared to other well-known methods on CIFAR-10. 
%To be consistent with other methods, 
We follow the same experiment setups as in~\cite{madry2017towards, wang2019improving, zhang2019theoretically}. When experiment settings are not identical between tested methods, we choose the most commonly used settings, and apply it to all experiments. This way, we keep the comparison as fair as possible and avoid reporting changes in results which are caused by inconsistent experiment settings \citep{pang2020bag}. To show that our results are not caused because of what is referred to as \textit{obfuscated gradients}~\citep{athalye2018obfuscated}, we evaluate our method with same setup as in our defense model, under strong attacks (e.g., PGD$^{1000}$) in both white-box, black-box settings, Auto-Attack ~\citep{croce2020reliable}, unforeseen "natural" corruptions~\citep{hendrycks2018benchmarking}, and unforeseen adversaries. To make sure that the reported improvements are not caused by \textit{adversarial overfitting}~\citep{rice2020overfitting}, we report best robust results for each method on average of 3 restarts, while omitting one standard deviation (which is smaller than 0.2\% in all experiments). Additional results for CIFAR-10 as well as comprehensive evaluation on MNIST can be found in Appendix \ref{mnist-results} and \ref{additional_res}.
%To further keep the comparison consistent, we followed the same attack settings for all methods.


\begin{table}[ht]
  \caption{Robustness against white-box, black-box attacks and Auto-Attack (AA) on CIFAR-10. Black-box attacks are generated using naturally trained surrogate model. Natural represents the naturally trained (non-adversarial) model.
  %and applied to the best performing robust models.
  }
  \vskip 0.1in
  \label{black-and_white-cifar}
  \centering
  \small
  \begin{tabular}{cccccccc@{\hspace{1\tabcolsep}}c}
    \toprule
    & & \multicolumn{3}{c}{White-box} & \multicolumn{3}{c}{Black-Box} \\
    \cmidrule(r){3-5} 
    \cmidrule(r){6-8}
    Defense Model & Natural & PGD$^{20}$ & PGD$^{100}$ & CW$^{\infty}$ & PGD$^{20}$ & PGD$^{100}$ & CW$^{\infty}$ & AA \\
    \midrule
    NATURAL & 95.43 & 0 & 0 & 0 & 0 & 0 & 0 &  0 \\
    \midrule
    TRADES & 84.92 & 56.60 & 55.56 & 54.20 & 84.08 & 83.89 & 83.91 &  53.08 \\
    MART & 83.62 & 58.12 & 56.48 & 53.09 & 82.82 & 82.52 & 82.80 & 51.10 \\
    AT & 85.10 & 56.28 & 54.46 & 53.99 & 84.22 & 84.14 & 83.92 & 51.52 \\
    ATDA & 76.91 & 43.27 & 41.13 & 41.01 & 75.59 & 75.37 & 75.35 & 40.08\\
    $\DIAL_{\kl}$ (Ours) & 85.25 & $\mathbf{58.43}$ & $\mathbf{56.80}$ & $\mathbf{55.00}$ & 84.30 & 84.18 & 84.05 & \textbf{53.75} \\
    $\DIAL_{\ce}$ (Ours)  & $\mathbf{89.59}$ & 54.31 & 51.67 & 52.04 &$ \mathbf{88.60}$ & $\mathbf{88.39}$ & $\mathbf{88.44}$ & 49.85 \\
    \midrule
    $\DIAL_{\awp}$ (Ours) & $\mathbf{85.91}$ & $\mathbf{61.10}$ & $\mathbf{59.86}$ & $\mathbf{57.67}$ & $\mathbf{85.13}$ & $\mathbf{84.93}$ & $\mathbf{85.03}$  & \textbf{56.78} \\
    $\TRADES_{\awp}$ & 85.36 & 59.27 & 59.12 & 57.07 & 84.58 & 84.58 & 84.59 & 56.17 \\
    \bottomrule
  \end{tabular}
\end{table}



\paragraph{CIFAR-10 setup.} We use the wide residual network (WRN-34-10)~\citep{zagoruyko2016wide} architecture. %used in the experiments of~\cite{madry2017towards, wang2019improving, zhang2019theoretically}. 
Sidelong this architecture, we integrate a domain classification layer. To generate the adversarial domain dataset, we use a perturbation size of $\epsilon=0.031$. We apply 10 of inner maximization iterations with perturbation step size of 0.007. Batch size is set to 128, weight decay is set to $7e^{-4}$, and the model is trained for 100 epochs. Similar to the other methods, the initial learning rate was set to 0.1, and decays by a factor of 10 at iterations 75 and 90. 
%For being consistent with other methods, the natural images are padded with 4-pixel padding with 32-random crop and random horizontal flip. Furthermore, all methods are trained using SGD with momentum 0.9. For $\DIAL_{\kl}$, we balance the robust loss with $\lambda=6$ and the domains loss with $r=4$. For $\DIAL_{\ce}$, we balance the robust loss with $\lambda=1$ and the domains loss with $r=2$. 
%We also introduce a version of our method that incorporates the AWP double-perturbation mechanism, named DIAL-AWP.
%which is trained using the same learning rate schedule used in ~\cite{wu2020adversarial}, where the initial 0.1 learning rate decays by a factor of 10 after 100 and 150 iterations. 
See Appendix \ref{cifar10-additional-setup} for additional details.

\begin{table}[ht]
  \caption{Black-box attack using the adversarially trained surrogate models on CIFAR-10.}
  \vskip 0.1in
  \label{black-box-cifar-adv}
  \centering
  \small
  \begin{tabular}{ll|c}
    \toprule
    \cmidrule(r){1-2}
    Surrogate (source) model & Target model & robustness \% \\
    % \midrule
    \midrule
    TRADES & $\DIAL_{\ce}$ & $\mathbf{67.77}$ \\
    $\DIAL_{\ce}$ & TRADES & 65.75 \\
    \midrule
    MART & $\DIAL_{\ce}$ & $\mathbf{70.30}$ \\
    $\DIAL_{\ce}$ & MART & 64.91 \\
    \midrule
    AT & $\DIAL_{\ce}$ & $\mathbf{65.32}$ \\
    $\DIAL_{\ce}$ & AT  & 63.54 \\
    \midrule
    ATDA & $\DIAL_{\ce}$ & $\mathbf{66.77}$ \\
    $\DIAL_{\ce}$ & ATDA & 52.56 \\
    \bottomrule
  \end{tabular}
\end{table}

\paragraph{White-box/Black-box robustness.} 
%We evaluate all defense models using Auto-Attack, PGD$^{20}$, PGD$^{100}$, PGD$^{1000}$ and CW$_{\infty}$ with step size 0.003. We constrain all attacks by the same perturbation $\epsilon=0.031$. 
As reported in Table~\ref{black-and_white-cifar} and Appendix~\ref{additional_res}, our method achieves better robustness compared to the other methods. Specifically, in the white-box settings, our method improves robustness over~\citet{madry2017towards} and TRADES by 2\% 
%using the common PGD$^{20}$ attack 
while keeping higher natural accuracy. We also observe better natural accuracy of 1.65\% over MART while also achieving better robustness over all attacks. Moreover, our method presents significant improvement of up to 15\% compared to the the domain invariant method suggested by~\citet{song2018improving} (ATDA).
%in both natural and robust accuracy. 
When incorporating 
%the double-perturbation mechanism of 
AWP, our method improves the results of $\TRADES_{\awp}$ by almost 2\%.
%and reaches state-of-the-art results for robust models with no additional data. 
% Additional results are available in Appendix~\ref{additional_res}.
When tested on black-box settings, $\DIAL_{\ce}$ presents a significant improvement of more than 4.4\% over the second-best performing method, and up to 13\%. In Table~\ref{black-box-cifar-adv}, we also present the black-box results when the source model is taken from one of the adversarially trained models. %Then, we compare our model to each one of them both as the source model and target model. 
In addition to the improvement in black-box robustness, $\DIAL_{\ce}$ also manages to achieve better clean accuracy of more than 4.5\% over the second-best performing method.
% Moreover, based on the auto-attack leader-board \footnote{\url{https://github.com/fra31/auto-attack}}, our method achieves the 1st place among models without additional data using the WRN-34-10 architecture.

% \begin{table}
%   \caption{White-box robustness on CIFAR-10 using WRN-34-10}
%   \label{white-box-cifar-10}
%   \centering
%   \begin{tabular}{lllll}
%     \toprule
%     \cmidrule(r){1-2}
%     Defense Model & Natural & PGD$^{20}$ & PGD$^{100}$ & PGD$^{1000}$ \\
%     \midrule
%     TRADES ~\cite{zhang2019theoretically} & 84.92  & 56.6 & 55.56 & 56.43  \\
%     MART ~\cite{wang2019improving} & 83.62  & 58.12 & 56.48 & 56.55  \\
%     Madry et al. ~\cite{madry2017towards} & 85.1  & 56.28 & 54.46 & 54.4  \\
%     Song et al. ~\cite{song2018improving} & 76.91 & 43.27 & 41.13 & 41.02  \\
%     $\DIAL_{\ce}$ (Ours) & $ \mathbf{90}$  & 52.12 & 48.88 & 48.78  \\
%     $\DIAL_{\kl}$ (Ours) & 85.25 & $\mathbf{58.43}$ & $\mathbf{56.8}$ & $\mathbf{56.73}$ \\
%     \midrule
%     $\DIAL_{\kl}$+AWP (Ours) & $\mathbf{85.91}$ & $\mathbf{61.1}$ & - & -  \\
%     TRADES+AWP \cite{wu2020adversarial} & 85.36 & 59.27 & 59.12 & -  \\
%     % MART + AWP & 84.43 & 60.68 & 59.32 & -  \\
%     \bottomrule
%   \end{tabular}
% \end{table}


% \begin{table}
%   \caption{White-box robustness on MNIST}
%   \label{white-box-mnist}
%   \centering
%   \begin{tabular}{llllll}
%     \toprule
%     \cmidrule(r){1-2}
%     Defense Model & Natural & PGD$^{40}$ & PGD$^{100}$ & PGD$^{1000}$ \\
%     \midrule
%     TRADES ~\cite{zhang2019theoretically} & 99.48 & 96.07 & 95.52 & 95.22 \\
%     MART ~\cite{wang2019improving} & 99.38  & 96.99 & 96.11 & 95.74  \\
%     Madry et al. ~\cite{madry2017towards} & 99.41  & 96.01 & 95.49 & 95.36 \\
%     Song et al. ~\cite{song2018improving}  & 98.72 & 96.82 & 96.26 & 96.2  \\
%     $\DIAL_{\kl}$ (Ours) & 99.46 & 97.05 & 96.06 & 95.99  \\
%     $\DIAL_{\ce}$ (Ours) & $\mathbf{99.49}$  & $\mathbf{97.38}$ & $\mathbf{96.45}$ & $\mathbf{96.33}$ \\
%     \bottomrule
%   \end{tabular}
% \end{table}


% \paragraph{Attacking MNIST.} For consistency, we use the same perturbation and step sizes. For MNIST, we use $\epsilon=0.3$ and step size of $0.01$. The natural accuracy of our surrogate (source) model is 99.51\%. Attacks results are reported in Table~\ref{black-and_white-mnist}. It is worth noting that the improvement margin is not conclusive on MNIST as it is on CIFAR-10, which is a more complex task.

% \begin{table}
%   \caption{Black-box robustness on MNIST and CIFAR-10 using naturally trained surrogate model and best performing robust models}
%   \label{black-box-mnist-cifar}
%   \centering
%   \begin{tabular}{lllllll}
%     \toprule
%     & \multicolumn{3}{c}{MNIST} & \multicolumn{3}{c}{CIFAR-10} \\
%     \cmidrule(r){2-4} 
%     \cmidrule(r){5-7}  
%     Defense Model & PGD$^{40}$ & PGD$^{100}$ & PGD$^{1000}$ & PGD$^{20}$ & PGD$^{100}$ & PGD$^{1000}$ \\
%     \midrule
%     TRADES ~\cite{zhang2019theoretically} & 98.12 & 97.86 & 97.81 & 84.08 & 83.89 & 83.8 \\
%     MART ~\cite{wang2019improving} & 98.16 & 97.96 & 97.89  & 82.82 & 82.52 & 82.47 \\
%     Madry et al. ~\cite{madry2017towards}  & 98.05 & 97.73 & 97.78 & 84.22 & 84.14 & 83.96 \\
%     Song et al. ~\cite{song2018improving} & 97.74 & 97.28 & 97.34 & 75.59 & 75.37 & 75.11 \\
%     $\DIAL_{\kl}$ (Ours) & 98.14 & 97.83 & 97.87  & 84.3 & 84.18 & 84.0 \\
%     $\DIAL_{\ce}$ (Ours)  & $\mathbf{98.37}$ & $\mathbf{98.12}$ & $\mathbf{98.05}$  & $\mathbf{89.13}$ & $\mathbf{88.89}$ & $\mathbf{88.78}$ \\
%     \bottomrule
%   \end{tabular}
% \end{table}



% \subsubsection{Ensemble attack} In addition to the white-box and black-box settings, we evaluate our method on the Auto-Attack ~\citep{croce2020reliable} using $\ell_{\infty}$ threat model with perturbation $\epsilon=0.031$. Auto-Attack is an ensemble of parameter-free attacks. It consists of three white-box attacks: APGD-CE which is a step size-free version of PGD on the cross-entropy ~\citep{croce2020reliable}. APGD-DLR which is a step size-free version of PGD on the DLR loss ~\citep{croce2020reliable} and FAB which  minimizes the norm of the adversarial perturbations, and one black-box attack: square attack which is a query-efficient black-box attack ~\citep{andriushchenko2020square}. Results are presented in Table~\ref{auto-attack}. Based on the auto-attack leader-board \footnote{\url{https://github.com/fra31/auto-attack}}, our method achieves the 1st place among models without additional data using the WRN-34-10 architecture.

%Additional results can be found in Appendix ~\ref{additional_res}.

% \begin{table}
%   \caption{Auto-Attack (AA) on CIFAR-10 with perturbation size $\epsilon=0.031$ with $\ell_{\infty}$ threat model}
%   \label{auto-attack}
%   \centering
%   \begin{tabular}{lll}
%     \toprule
%     \cmidrule(r){1-2}
%     Defense Model & AA \\
%     \midrule
%     TRADES ~\cite{zhang2019theoretically} & 53.08  \\
%     MART ~\cite{wang2019improving} & 51.1  \\
%     Madry et al. ~\cite{madry2017towards} & 51.52    \\
%     Song et al.   ~\cite{song2018improving} & 40.18 \\
%     $\DIAL_{\ce}$ (Ours) & 47.33  \\
%     $\DIAL_{\kl}$ (Ours) & $\mathbf{53.75}$ \\
%     \midrule
%     DIAL-AWP (Ours) & $\mathbf{56.78}$ \\
%     TRADES-AWP \cite{wu2020adversarial} & 56.17 \\
%     \bottomrule
%   \end{tabular}
% \end{table}


% \begin{table}[!ht]
%   \caption{Auto-Attack (AA) Robustness (\%) on CIFAR-10 with $\epsilon=0.031$ using an $\ell_{\infty}$ threat model}
%   \label{auto-attack}
%   \centering
%   \begin{tabular}{cccccc|cc}
%     \toprule
%     % \multicolumn{8}{c}{Defence Model}  \\
%     % \cmidrule(r){1-8} 
%     TRADES & MART & Madry & Song & $\DIAL_{\ce}$ & $\DIAL_{\kl}$ & DIAL-AWP  & TRADES-AWP\\
%     \midrule
%     53.08 & 51.10 & 51.52 &  40.08 & 47.33  & $\mathbf{53.75}$ & $\mathbf{56.78}$ & 56.17 \\

%     \bottomrule
%   \end{tabular}
% \end{table}

% \begin{table}[!ht]
% \caption{$F_1$-robust measurement using PGD$^{20}$ attack in white-box and black-box settings on CIFAR-10}
%   \label{f1-robust}
%   \centering
%   \begin{tabular}{ccccccc|cc}
%     \toprule
%     % \multicolumn{8}{c}{Defence Model}  \\
%     % \cmidrule(r){1-8} 
%     Defense Model & TRADES & MART & Madry & Song & $\DIAL_{\kl}$ & $\DIAL_{\ce}$ & DIAL-AWP  & TRADES-AWP\\
%     \midrule
%     White-box & 0.659 & 0.666 & 0.657 & 0.518 & $\mathbf{0.675}$  & 0.643 & $\mathbf{0.698}$ & 0.682 \\
%     Black-box & 0.844 & 0.831 & 0.846 & 0.761 & 0.847 & $\mathbf{0.895}$ & $\mathbf{0.854}$ &  0.849 \\
%     \bottomrule
%   \end{tabular}
% \end{table}

\subsubsection{Robustness to Unforeseen Attacks and Corruptions}
\paragraph{Unforeseen Adversaries.} To further demonstrate the effectiveness of our approach, we test our method against various adversaries that were not used during the training process. We attack the model under the white-box settings with $\ell_{2}$-PGD, $\ell_{1}$-PGD, $\ell_{\infty}$-DeepFool and $\ell_{2}$-DeepFool \citep{moosavi2016deepfool} adversaries using Foolbox \citep{rauber2017foolbox}. We applied commonly used attack budget 
%(perturbation for PGD adversaries and overshot for DeepFool adversaries) 
with 20 and 50 iterations for PGD and DeepFool, respectively.
Results are presented in Table \ref{unseen-attacks}. As can be seen, our approach  gains an improvement of up to 4.73\% over the second best method under the various attack types and an average improvement of 3.7\% over all threat models.


\begin{table}[ht]
  \caption{Robustness on CIFAR-10 against unseen adversaries under white-box settings.}
  \vskip 0.1in
  \label{unseen-attacks}
  \centering
%   \small
  \begin{tabular}{c@{\hspace{1.5\tabcolsep}}c@{\hspace{1.5\tabcolsep}}c@{\hspace{1.5\tabcolsep}}c@{\hspace{1.5\tabcolsep}}c@{\hspace{1.5\tabcolsep}}c@{\hspace{1.5\tabcolsep}}c@{\hspace{1.5\tabcolsep}}c}
    \toprule
    Threat Model & Attack Constraints & $\DIAL_{\kl}$ & $\DIAL_{\ce}$ & AT & TRADES & MART & ATDA \\
    \midrule
    \multirow{2}{*}{$\ell_{2}$-PGD} & $\epsilon=0.5$ & 76.05 & \textbf{80.51} & 76.82 & 76.57 & 75.07 & 66.25 \\
    & $\epsilon=0.25$ & 80.98 & \textbf{85.38} & 81.41 & 81.10 & 80.04 & 71.87 \\\midrule
    \multirow{2}{*}{$\ell_{1}$-PGD} & $\epsilon=12$ & 74.84 & \textbf{80.00} & 76.17 & 75.52 & 75.95 & 65.76 \\
    & $\epsilon=7.84$ & 78.69 & \textbf{83.62} & 79.86 & 79.16 & 78.55 & 69.97 \\
    \midrule
    $\ell_{2}$-DeepFool & overshoot=0.02 & 84.53 & \textbf{88.88} & 84.15 & 84.23 & 82.96 & 76.08 \\\midrule
    $\ell_{\infty}$-DeepFool & overshoot=0.02 & 68.43 & \textbf{69.50} & 67.29 & 67.60 & 66.40 & 57.35 \\
    \bottomrule
  \end{tabular}
\end{table}


%%%%%%%%%%%%%%%%%%%%%%%%% conference version %%%%%%%%%%%%%%%%%%%%%%%%%%%%%%%%%%%%%
\paragraph{Unforeseen Corruptions.}
We further demonstrate that our method consistently holds against unforeseen ``natural'' corruptions, consists of 18 unforeseen diverse corruption types proposed by \citet{hendrycks2018benchmarking} on CIFAR-10, which we refer to as CIFAR10-C. The CIFAR10-C benchmark covers noise, blur, weather, and digital categories. As can be shown in Figure \ref{corruption}, our method gains a significant and consistent improvement over all the other methods. Our method leads to an average improvement of 4.7\% with minimum improvement of 3.5\% and maximum improvement of 5.9\% compared to the second best method over all unforeseen attacks. See Appendix \ref{corruptions-apendix} for the full experiment results.


\begin{figure}[h]
 \centering
  \includegraphics[width=0.4\textwidth]{figures/spider_full.png}
%   \caption{Summary of accuracy over all unforeseen corruptions compared to the second and third best performing methods.}
  \caption{Accuracy comparison over all unforeseen corruptions.}
  \label{corruption}
\end{figure}


%%%%%%%%%%%%%%%%%%%%%%%%% conference version %%%%%%%%%%%%%%%%%%%%%%%%%%%%%%%%%%%%%

%%%%%%%%%%%%%%%%%%%%%%%%% Arxiv version %%%%%%%%%%%%%%%%%%%%%%%%%%%%%%%%%%%%%
% \newpage
% \paragraph{Unforeseen Corruptions.}
% We further demonstrate that our method consistently holds against unforeseen "natural" corruptions, consists of 18 unforeseen diverse corruption types proposed by \cite{hendrycks2018benchmarking} on CIFAR-10, which we refer to as CIFAR10-C. The CIFAR10-C benchmark covers noise, blur, weather, and digital categories. As can be shown in Figure  \ref{spider-full-graph}, our method gains a significant and consistent improvement over all the other methods. Our approach leads to an average improvement of 4.7\% with minimum improvement of 3.5\% and maximum improvement of 5.9\% compared to the second best method over all unforeseen attacks. Full accuracy results against unforeseen corruptions are presented in Tables \ref{corruption-table1} and \ref{corruption-table2}. 

% \begin{table}[!ht]
%   \caption{Accuracy (\%) against unforeseen corruptions.}
%   \label{corruption-table1}
%   \centering
%   \tiny
%   \begin{tabular}{lcccccccccccccccccc}
%     \toprule
%     Defense Model & brightness & defocus blur & fog & glass blur & jpeg compression & motion blur & saturate & snow & speckle noise  \\
%     \midrule
%     TRADES & 82.63 & 80.04 & 60.19 & 78.00 & 82.81 & 76.49 & 81.53 & 80.68 & 80.14 \\
%     MART & 80.76 & 78.62 & 56.78 & 76.60 & 81.26 & 74.58 & 80.74 & 78.22 & 79.42 \\
%     AT &  83.30 & 80.42 & 60.22 & 77.90 & 82.73 & 76.64 & 82.31 & 80.37 & 80.74 \\
%     ATDA & 72.67 & 69.36 & 45.52 & 64.88 & 73.22 & 63.47 & 72.07 & 68.76 & 72.27 \\
%     DIAL (Ours)  & \textbf{87.14} & \textbf{84.84} & \textbf{66.08} & \textbf{81.82} & \textbf{87.07} & \textbf{81.20} & \textbf{86.45} & \textbf{84.18} & \textbf{84.94} \\
%     \bottomrule
%   \end{tabular}
% \end{table}


% \begin{table}[!ht]
%   \caption{Accuracy (\%) against unforeseen corruptions.}
%   \label{corruption-table2}
%   \centering
%   \tiny
%   \begin{tabular}{lcccccccccccccccccc}
%     \toprule
%     Defense Model & contrast & elastic transform & frost & gaussian noise & impulse noise & pixelate & shot noise & spatter & zoom blur \\
%     \midrule
%     TRADES & 43.11 & 79.11 & 76.45 & 79.21 & 73.72 & 82.73 & 80.42 & 80.72 & 78.97 \\
%     MART & 41.22 & 77.77 & 73.07 & 78.30 & 74.97 & 81.31 & 79.53 & 79.28 & 77.8 \\
%     AT & 43.30 & 79.58 & 77.53 & 79.47 & 73.76 & 82.78 & 80.86 & 80.49 & 79.58 \\
%     ATDA & 36.06 & 67.06 & 62.56 & 70.33 & 64.63 & 73.46 & 72.28 & 70.50 & 67.31 \\
%     DIAL (Ours) & \textbf{48.84} & \textbf{84.13} & \textbf{81.76} & \textbf{83.76} & \textbf{78.26} & \textbf{87.24} & \textbf{85.13} & \textbf{84.84} & \textbf{83.93}  \\
%     \bottomrule
%   \end{tabular}
% \end{table}


% \begin{figure}[!ht]
%   \centering
%   \includegraphics[width=9cm]{figures/spider_full.png}
%   \caption{Accuracy comparison with all tested methods over unforeseen corruptions.}
%   \label{spider-full-graph}
% \end{figure}
% %%%%%%%%%%%%%%%%%%%%%%%%% Arxiv version %%%%%%%%%%%%%%%%%%%%%%%%%%%%%%%%%%%%%
%%%%%%%%%%%%%%%%%%%%%%%%% Arxiv version %%%%%%%%%%%%%%%%%%%%%%%%%%%%%%%%%%%%%

\subsubsection{Transfer Learning}
Recent works \citep{salman2020adversarially,utrera2020adversarially} suggested that robust models transfer better on standard downstream classification tasks. In Table \ref{transfer-res} we demonstrate the advantage of our method when applied for transfer learning across CIFAR10 and CIFAR100 using the common linear evaluation protocol. see Appendix \ref{transfer-learning-settings} for detailed settings.

\begin{table}[H]
  \caption{Transfer learning results comparison.}
  \vskip 0.1in
  \label{transfer-res}
  \centering
  \small
\begin{tabular}{c|c|c|c}
\toprule

\multicolumn{2}{l}{} & \multicolumn{2}{c}{Target} \\
\cmidrule(r){3-4}
Source & Defence Model & CIFAR10 & CIFAR100 \\
\midrule
\multirow{3}{*}{CIFAR10} & DIAL & \multirow{3}{*}{-} & \textbf{28.57} \\
 & AT &  & 26.95  \\
 & TRADES &  & 25.40  \\
 \midrule
\multirow{3}{*}{CIFAR100} & DIAL & \textbf{73.68} & \multirow{3}{*}{-} \\
 & AT & 71.41 & \\
 & TRADES & 71.42 &  \\
%  \midrule
% \multirow{3}{}{SVHN} & DIAL &  &  & \multirow{3}{}{-} \\
%  & Madry et al. &  &  &  \\
%  & TRADES &  &  &  \\ 
\bottomrule
\end{tabular}
\end{table}


\subsubsection{Modularity and Ablation Studies}

We note that the domain classifier is a modular component that can be integrated into existing models for further improvements. Removing the domain head and related loss components from the different DIAL formulations results in some common adversarial training techniques. For $\DIAL_{\kl}$, removing the domain and related loss components results in the formulation of TRADES. For $\DIAL_{\ce}$, removing the domain and related loss components results in the original formulation of the standard adversarial training, and for $\DIAL_{\awp}$ the removal results in $\TRADES_{\awp}$. Therefore, the ablation studies will demonstrate the effectiveness of combining DIAL on top of different adversarial training methods. 

We investigate the contribution of the additional domain head component introduced in our method. Experiment configuration are as in \ref{defence-settings}, and robust accuracy is based on white-box PGD$^{20}$ on CIFAR-10 dataset. We remove the domain head from both $\DIAL_{\kl}$, $\DIAL_{\awp}$, and $\DIAL_{\ce}$ (equivalent to $r=0$) and report the natural and robust accuracy. We perform 3 random restarts and omit one standard deviation from the results. Results are presented in Figure \ref{ablation}. All DIAL variants exhibits stable improvements on both natural accuracy and robust accuracy. $\DIAL_{\ce}$, $\DIAL_{\kl}$, and $\DIAL_{\awp}$ present an improvement of 1.82\%, 0.33\%, and 0.55\% on natural accuracy and an improvement of 2.5\%, 1.87\%, and 0.83\% on robust accuracy, respectively. This evaluation empirically demonstrates the benefits of incorporating DIAL on top of different adversarial training techniques.
% \paragraph{semi-supervised extensions.} Since the domain classifier does not require the class labels, we argue that additional unlabeled data can be leveraged in future work.
%for improved results. 

\begin{figure}[ht]
  \centering
  \includegraphics[width=0.35\textwidth]{figures/ablation_graphs3.png}
  \caption{Ablation studies for $\DIAL_{\kl}$, $\DIAL_{\ce}$, and $\DIAL_{\awp}$ on CIFAR-10. Circle represent the robust-natural accuracy without using DIAL, and square represent the robust-natural accuracy when incorporating DIAL.
  %to further investigate the impact of the domain head and loss on natural and robust accuracy.
  }
  \label{ablation}
\end{figure}

\subsubsection{Visualizing DIAL}
To further illustrate the superiority of our method, we visualize the model outputs from the different methods on both natural and adversarial test data.
% adversarial test data generated using PGD$^{20}$ white-box attack with step size 0.003 and $\epsilon=0.031$ on CIFAR-10. 
Figure~\ref{tsne1} shows the embedding received after applying t-SNE ~\citep{van2008visualizing} with two components on the model output for our method and for TRADES. DIAL seems to preserve strong separation between classes on both natural test data and adversarial test data. Additional illustrations for the other methods are attached in Appendix~\ref{additional_viz}. 

\begin{figure}[h]
\centering
  \subfigure[\textbf{DIAL} on natural logits]{\includegraphics[width=0.21\textwidth]{figures/domain_ce_test.png}}
  \hspace{0.03\textwidth}
  \subfigure[\textbf{DIAL} on adversarial logits]{\includegraphics[width=0.21\textwidth]{figures/domain_ce_adversarial.png}}
  \hspace{0.03\textwidth}
    \subfigure[\textbf{TRADES} on natural logits]{\includegraphics[width=0.21\textwidth]{figures/trades_test.png}}
    \hspace{0.03\textwidth}
    \subfigure[\textbf{TRADES} on adversarial logits]{\includegraphics[width=0.21\textwidth]{figures/trades_adversarial.png}}
  \caption{t-SNE embedding of model output (logits) into two-dimensional space for DIAL and TRADES using the CIFAR-10 natural test data and the corresponding PGD$^{20}$ generated adversarial examples.}
  \label{tsne1}
\end{figure}


% \begin{figure}[ht]
% \centering
%   \begin{subfigure}{4cm}
%     \centering\includegraphics[width=3.3cm]{figures/domain_ce_test.png}
%     \caption{\textbf{DIAL} on nat. examples}
%   \end{subfigure}
%   \begin{subfigure}{4cm}
%     \centering\includegraphics[width=3.3cm]{figures/domain_ce_adversarial.png}
%     \caption{\textbf{DIAL} on adv. examples}
%   \end{subfigure}
  
%   \begin{subfigure}{4cm}
%     \centering\includegraphics[width=3.3cm]{figures/trades_test.png}
%     \caption{\textbf{TRADES} on nat. examples}
%   \end{subfigure}
%   \begin{subfigure}{4cm}
%     \centering\includegraphics[width=3.3cm]{figures/trades_adversarial.png}
%     \caption{\textbf{TRADES} on adv. examples}
%   \end{subfigure}
%   \caption{t-SNE embedding of model output (logits) into two-dimensional space for DIAL and TRADES using the CIFAR-10 natural test data and the corresponding adversarial examples.}
%   \label{tsne1}
% \end{figure}



% \begin{figure}[ht]
% \centering
%   \begin{subfigure}{6cm}
%     \centering\includegraphics[width=5cm]{figures/domain_ce_test.png}
%     \caption{\textbf{DIAL} on nat. examples}
%   \end{subfigure}
%   \begin{subfigure}{6cm}
%     \centering\includegraphics[width=5cm]{figures/domain_ce_adversarial.png}
%     \caption{\textbf{DIAL} on adv. examples}
%   \end{subfigure}
  
%   \begin{subfigure}{6cm}
%     \centering\includegraphics[width=5cm]{figures/trades_test.png}
%     \caption{\textbf{TRADES} on nat. examples}
%   \end{subfigure}
%   \begin{subfigure}{6cm}
%     \centering\includegraphics[width=5cm]{figures/trades_adversarial.png}
%     \caption{\textbf{TRADES} on adv. examples}
%   \end{subfigure}
%   \caption{t-SNE embedding of model output (logits) into two-dimensional space for DIAL and TRADES using the CIFAR-10 natural test data and the corresponding adversarial examples.}
%   \label{tsne1}
% \end{figure}



\subsection{Balanced measurement for robust-natural accuracy}
One of the goals of our method is to better balance between robust and natural accuracy under a given model. For a balanced metric, we adopt the idea of $F_1$-score, which is the harmonic mean between the precision and recall. However, rather than using precision and recall, we measure the $F_1$-score between robustness and natural accuracy,
using a measure we call
%We named it
the
\textbf{$\mathbf{F_1}$-robust} score.
\begin{equation}
F_1\text{-robust} = \dfrac{\text{true\_robust}}
{\text{true\_robust}+\frac{1}{2}
%\cdot
(\text{false\_{robust}}+
\text{false\_natural})},
\end{equation}
where $\text{true\_robust}$ are the adversarial examples that were correctly classified, $\text{false\_{robust}}$ are the adversarial examples that were miss-classified, and $\text{false\_natural}$ are the natural examples that were miss-classified.
%We tested the proposed $F_1$-robust score using PGD$^{20}$ on CIFAR-10 dataset in white-box and black-box settings. 
Results are presented in Table~\ref{f1-robust} and demonstrate that our method achieves the best $F_1$-robust score in both settings, which supports our findings from previous sections.

% \begin{table}[!ht]
%   \caption{$F_1$-robust measurement using PGD$^{20}$ attack in white and black box settings on CIFAR-10}
%   \label{f1-robust}
%   \centering
%   \begin{tabular}{lll}
%     \toprule
%     \cmidrule(r){1-2}
%     Defense Model & White-box & Black-box \\
%     \midrule
%     TRADES & 0.65937  & 0.84435 \\
%     MART & 0.66613  & 0.83153  \\
%     Madry et al. & 0.65755 & 0.84574   \\
%     Song et al. & 0.51823 & 0.76092  \\
%     $\DIAL_{\ce}$ (Ours) & 0.65318   & $\mathbf{0.88806}$  \\
%     $\DIAL_{\kl}$ (Ours) & $\mathbf{0.67479}$ & 0.84702 \\
%     \midrule
%     \midrule
%     DIAL-AWP (Ours) & $\mathbf{0.69753}$  & $\mathbf{0.85406}$  \\
%     TRADES-AWP & 0.68162 & 0.84917 \\
%     \bottomrule
%   \end{tabular}
% \end{table}

\begin{table}[ht]
\small
  \caption{$F_1$-robust measurement using PGD$^{20}$ attack in white and black box settings on CIFAR-10.}
  \vskip 0.1in
  \label{f1-robust}
  \centering
%   \small
  \begin{tabular}{c
  @{\hspace{1.5\tabcolsep}}c @{\hspace{1.5\tabcolsep}}c @{\hspace{1.5\tabcolsep}}c @{\hspace{1.5\tabcolsep}}c
  @{\hspace{1.5\tabcolsep}}c @{\hspace{1.5\tabcolsep}}c @{\hspace{1.5\tabcolsep}}|
  @{\hspace{1.5\tabcolsep}}c
  @{\hspace{1.5\tabcolsep}}c}
    \toprule
    % \cmidrule(r){8-9}
     & TRADES & MART & AT & ATDA & $\DIAL_{\ce}$ & $\DIAL_{\kl}$ & $\DIAL_{\awp}$ & $\TRADES_{\awp}$ \\
    \midrule
    White-box & 0.659 & 0.666 & 0.657 & 0.518 & 0.660 & \textbf{0.675} & \textbf{0.698} & 0.682 \\
    Black-box & 0.844 & 0.831 & 0.845 & 0.761 & \textbf{0.890} & 0.847 & \textbf{0.854} & 0.849 \\ 
    \bottomrule
  \end{tabular}
\end{table}


\vspace{-2em}
\section{Limitations and future work}
\label{sec:dis}
%% theory does not specifically encode for fine grain properties like fairness or robustness
Our method comes with several limitations and possible extensions for future work.  
While preserving model correlation suggests that we are likely to preserve sub-class loss, our theory does not currently extend to that regime and requires that the unlabeled pruning set be from the same distribution as the test set. More broadly, we do not yet fully understand how trainable the models produced by ID are in different conditions and we cannot make claims about how compressable a given model will be. In future work we will explore modifying training process to improve prunability, which is a common approach~\cite{huang2018sss,yang2020hoyer,liu2017netslim,zhuang2020polar}.  
We will also explore ways to refine our iterative pruning approach to work with
%residual layers and 
more complicated architectures. Of particular note, our method typically exposes a sizable ``free FLOPs" regime and we explore how this can be leveraged more broadly.


\vspace{-1em}
\begin{ack}
\vspace{-1em}
This work was partially funded by the National Science Foundation under awards
DMS-1830274, DGE-1922551, and NSF CAREER Award 2046760.
\end{ack}


{
\small
\bibliographystyle{plainnat}
\bibliography{id_for_nn.bib}
}


%%%%%%%%%%%%%%%%%%%%%%%%%%%%%%%%%%%%%%%%%%%%%%%%%%%%%%%%%%%%



%%% BEGIN INSTRUCTIONS %%%
%The checklist follows the references.  Please
%read the checklist guidelines carefully for information on how to answer these
%questions.  For each question, change the default \answerTODO{} to \answerYes{},
%\answerNo{}, or \answerNA{}.  You are strongly encouraged to include a {\bf
%justification to your answer}, either by referencing the appropriate section of
%your paper or providing a brief inline description.  For example:
%\begin{itemize}
%  \item Did you include the license to the code and datasets? \answerTODO{}
%  \item Did you include the license to the code and datasets? \answerTODO{}
%  \item Did you include the license to the code and datasets? \answerNA{}
%\end{itemize}
%Please do not modify the questions and only use the provided macros for your
%answers.  Note that the Checklist section does not count towards the page
%limit.  In your paper, please delete this instructions block and only keep the
%Checklist section heading above along with the questions/answers below.
%%% END INSTRUCTIONS %%%

% \section*{Checklist}
% \begin{enumerate}


% \item For all authors...
% \begin{enumerate}
%   \item Do the main claims made in the abstract and introduction accurately reflect the paper's contributions and scope?
%     \answerYes{}
%   \item Did you describe the limitations of your work?
%     \answerYes{} Section 7
%   \item Did you discuss any potential negative societal impacts of your work?
%     \answerNA{} 
%   \item Have you read the ethics review guidelines and ensured that your paper conforms to them?
%     \answerYes{}
% \end{enumerate}


% \item If you are including theoretical results...
% \begin{enumerate}
%   \item Did you state the full set of assumptions of all theoretical results?
%     \answerYes{}
%         \item Did you include complete proofs of all theoretical results?
%     \answerYes{} - In the Appendix.  
% \end{enumerate}


% \item If you ran experiments...
% \begin{enumerate}
%   \item Did you include the code, data, and instructions needed to reproduce the main experimental results (either in the supplemental material or as a URL)?
%     \answerYes{} % - Supplemental material
%     % \href{https://github.com/jerry-chee/ModelPreserveCompressionNN}{https://github.com/jerry-chee/ModelPreserveCompressionNN}
% %   \item Did you specify all the training details (e.g., data splits, hyperparameters, how they were chosen)?
%     \answerYes{} Appendix
%         \item Did you report error bars (e.g., with respect to the random seed after running experiments multiple times)?
%     \answerYes{} Where including error bars was computationally feasible we did so.  This was not computationally feasible on ImageNet.  
%         \item Did you include the total amount of compute and the type of resources used (e.g., type of GPUs, internal cluster, or cloud provider)? 
%     \answerYes{} Appendix
% \end{enumerate}


% \item If you are using existing assets (e.g., code, data, models) or curating/releasing new assets...
% \begin{enumerate}
%   \item If your work uses existing assets, did you cite the creators?
%     \answerYes{}
%   \item Did you mention the license of the assets?
%     \answerYes{}{}
%   \item Did you include any new assets either in the supplemental material or as a URL?
%     \answerNo{}
%   \item Did you discuss whether and how consent was obtained from people whose data you're using/curating?
%     \answerNA{}
%   \item Did you discuss whether the data you are using/curating contains personally identifiable information or offensive content?
%     \answerNA{}
% \end{enumerate}


% \item If you used crowdsourcing or conducted research with human subjects...
% \begin{enumerate}
%   \item Did you include the full text of instructions given to participants and screenshots, if applicable?
%     \answerNA{}
%   \item Did you describe any potential participant risks, with links to Institutional Review Board (IRB) approvals, if applicable?
%     \answerNA{}
%   \item Did you include the estimated hourly wage paid to participants and the total amount spent on participant compensation?
%     \answerNA{}
% \end{enumerate}


% \end{enumerate}


%%%%%%%%%%%%%%%%%%%%%%%%%%%%%%%%%%%%%%%%%%%%%%%%%%%%%%%%%%%%

\newpage
\appendix
\section*{Appendix}
\appendix


\section{Notation}

Matrices are denoted with capital letters such as $A$, and vectors with lower case $a$.  In situations where we partition a matrix into pieces, the partitions will be referred to as $A_{ij}$.  Individual entries in a matrix will be referred to as lower case letters with two subscripts, $a_{ij}$.  
$\sigma_k(A)$ denotes the $k$-th leading singular value of $A$, and $\kappa(A)$ the condition number.
For a matrix $A\in\R^{n\times m}$ we let $A_{\mathcal{J},\I}$ denote a sub-selection of the matrix $A$ using sets $\mathcal{J}\subset [n]$ to denote the selected rows and $\I\subset [m]$ to denote the selected columns; $:$ denotes a selection of all rows or columns.

\section{Fixed-rank interpolative decompositions}
%\paragraph{Further applications: choosing important sub-samples}
%Aside from pruning, interpolative decompositions can be used to select an important sub-sample of the %training
%data which can act as a surrogate for the whole set.
%Following our notation in Section~\ref{sec:pruneID} for a one hidden layer network, let $Z \in \R^{m \times n}$ be the first-layer output, i.e. $Z = g(W^\top X)$.
%Now we wish to preserve the network outputs with fewer data points by computing a rank-$k$ interpolative decomposition $Z^\top \approx (X^\top)_{:,\I} H$ where $H$ is an interpolation matrix.
%In particular, we compute a rank-$k$ interpolative decomposition of 
%%$\gamma(W^\top X)$ (i.e., 
%$Z^\top$ denoted $\gamma(W^\top X) \approx \gamma(W^\top X_{:,\mathcal{J}})H$, where $|\mathcal{J}|=k$ is the subset of the data and $H$ the associated interpolation matrix. 

As stated in Section~\ref{sec:ID}, a formal algorithmic statement is given for computing fixed-rank interpolative decompositions.
    
\begin{algorithm}[ht]%t
\begin{algorithmic}
\label{alg:genericID}
%\INPUT
\REQUIRE
%\KwIn{
matrix $A \in \R^{n \times m}$, rank-$k$
%}
%\OUTPUT
\ENSURE
%\KwOut{
interpolative decomposition $A_{:,\I} T$
%}
\STATE Compute column-pivoted QR factorization 
\[
    A
    \begin{bmatrix}
    \Pi_1 & \Pi_2
    \end{bmatrix}
    =
    \begin{bmatrix}
    Q_1 & Q_2
    \end{bmatrix}
    \begin{bmatrix}
    R_{11} & R_{12} \\
    & R_{22}
    \end{bmatrix}.
\]
where
$\Pi_1 \in \R^{m \times k}$, 
%$\Pi_2 \in \R^{m \times (m-k)}$, 
%$Q_1 \in \R^{n \times k}$, 
%$Q_2 \in \R^{n \times (\ell-k)}$, 
$R_{11} \in \R^{k \times k}$, 
$R_{12} \in \R^{k \times (m-k)}$, 
%and $R_{22} \in \R^{(\ell-k)\times(\ell-n)}$
and remaining dimensions as required.
\;
\STATE $A_{:,\I} \gets A \Pi_1$ \;
\STATE $T \gets 
\begin{bmatrix}
I_k & R_{11}^{-1} R_{12}
\end{bmatrix}
\Pi^\top$\;
\end{algorithmic}
%\SetAlgoLined
%\DontPrintSemicolon
\caption{Interpolative Decomposition}
\end{algorithm}

\section{Proofs}

\begin{customthm}{\ref{thm:generalization}}
%\label{thm:generalization}
Consider a model $h_{FC}=u^\top g(W^\top x)$ with m hidden neurons and a pruned model $\widehat{h}_{FC}=\widehat{u}^\top g(\widehat{W}^\top x)$ constructed using an $\epsilon$ accurate ID with $n$ data points drawn i.i.d\ from $\cD.$ The risk of the pruned model $\mathcal{R}_p$ on a data set $(x,y) \sim D$ assuming $\cD$ is compactly supported on $\Omega_x\times\Omega$ is bounded by  
\begin{equation*}
    \mathcal{R}_p \leq \mathcal{R}_{ID} + \mathcal{R}_0+ 2  \sqrt{ \mathcal{R}_{ID}  \mathcal{R}_0},
\end{equation*}
where $\mathcal{R}_{ID}$ is the risk associated with approximating the full model by a pruned one and with probability $1-\delta$ satisfies
\begin{equation*}
    {\mathcal{R}}_{ID} \leq \epsilon^2M+M(1+\|T\|_2)^2n^{-\frac{1}{2}} \left( \sqrt{2\zeta dm \log (dm)\log\frac{en}{\zeta dm \log (dm)}}+ \sqrt{\frac{\log (1/\delta)}{2}}\right).
\end{equation*} 
Here, $M = \sup_{x\in\Omega_x} \|u\|_2^2 \| g(W^T x)\|_2^2$ and $\zeta$ is a universal constant that depends on $g$. %the activation function.  


\end{customthm}


\begin{proof}
We can write the risk for this network as
\begin{equation*}
    \mathcal{R}_p=\mathbb{E}(\|\widehat{u}^\top g(\widehat{W}^\top x)- y\|^2),
\end{equation*} and adding and subtract the original network yields

\begin{equation*}
\begin{aligned}
    \mathbb{E}(\|\widehat{u}^\top g(\widehat{W}^\top x)- y\|^2) &= \mathbb{E}(\|(\widehat{u}^\top g(\widehat{W}^\top x)-u^\top g (w^\top x))+(u^\top g (w^\top x)- y)\|^2)\\
    &\leq \mathbb{E}((\|(\widehat{u}^\top g(\widehat{W}^\top x)-u^\top g (w^\top x))\|+\|(u^\top g (w^\top x)- y)\|)^2)\\
   &\leq \mathbb{E}(\|(\widehat{u}^\top g(\widehat{W}^\top x)-u^\top g (w^\top x))\|^2)\\
   &\phantom{\leq}+ \mathbb{E}(2\|(\widehat{u}^\top g(\widehat{W}^\top x)-u^\top g (w^\top x))\|\|(u^\top g (w^\top x)- y)\|)\\
   &\phantom{\leq}+\mathbb{E}(\|(u^\top g (w^\top x)- y)\|^2)\\
   &\leq  \mathcal{R}_{ID} + 2  \sqrt{ \mathcal{R}_{ID}  \mathcal{R}_0}+ \mathcal{R}_0.
\end{aligned}
\end{equation*}

Now, we bound $\mathcal{R}_{ID}$ by 
considering the interpolative decomposition to be a learning algorithm learning the function $u^\top g(W^\top X)$.
Specifically, we use Lemma \ref{lem:prunedRisk} to bound $\mathcal{R}_{ID}$ as  
\begin{equation*}
    \mathcal{R}_{ID} \leq \widehat{\cR}_{ID} + M(1+\|T\|_2)^2n^{-\frac{1}{2}}\left( \sqrt{2p\log(en/p)}+ 2^{-\frac{1}{2}}\sqrt{\log (1/\delta)}\right).
\end{equation*}
where p is the pseudo-dimension. We can then use Lemma \ref{lem:IDEmperical} to bound the empirical risk of the interpolative decomposition as  
\begin{equation*}
    \widehat{\cR}_{ID} \leq  \epsilon^2 \|u\|_2^2 \| g(W^T X)\|_2^2 / n,
\end{equation*}
and it follows that
\begin{equation*}
    \widehat{\cR}_{ID} \leq \epsilon^2 \sup_{x\in\Omega_x} \|u\|_2^2 \| g(W^T x)\|_2^2.
\end{equation*}

\end{proof}


\begin{customlemma}{\ref{lem:prunedRisk}}
Under the assumptions of Theorem~\ref{thm:generalization}, for any $\delta\in(0,1)$, $\cR_{ID}$ satisfies  
\begin{equation*}
    \mathcal{R}_{ID} \leq \widehat{\cR}_{ID} + M(1+\|T\|_2)^2n^{-\frac{1}{2}}\left( \sqrt{2p\log(en/p)}+ 2^{-\frac{1}{2}}\sqrt{\log (1/\delta)}\right)
\end{equation*}
with probability $1-\delta,$ where $M = \sup_{x\in\Omega_x} \|u\| ^2 \| g(W^T x)\|^2$ and $p=\zeta dm \log (dm)$ for some universal constant $\zeta$ that depends only on the activation function.
\end{customlemma}
\begin{proof} 
Considering the interpolative decomposition as a learning algorithm to learn $u^\top g(W^\top X)$, we can use Theorem 11.8 in~\cite{foundationsML} to bound the risk on the data distribution. Given a maximum on the loss function $\eta$, and the ReLU activation function, 


\begin{equation*}
    \mathcal{R}_{ID} \leq \widehat{\mathcal{R}}_{ID} + \frac{\eta}{n^{1/2}}( \sqrt{2p\log en/p}+ 2^{-1/2} \sqrt{\log (1/\delta)})
\end{equation*}
with probability $(1-\delta)$.\footnote{e is the base of the natural log.} Here, the constant $\eta$ is bounded by Lemma~\ref{lem:eta}. Bartlett et al.~\cite{pmlr-v65-harvey17a}  show that the p-dimension for a ReLU network is $O(\mathcal{W}Llog(\mathcal{W}))$ where $\mathcal{W}$ is the number of weights and L is the number of layers.   Here, that translates to $p=\zeta dm \log(dm)$ for some constant $\zeta$ that depends only on the choice of activation function.  
\end{proof}




\begin{customlemma}{\ref{lem:IDEmperical}}

Following the notation of Theorem~\ref{thm:generalization}, an ID pruning to accuracy $\epsilon$ yields a compressed network that satisfies
\begin{equation*}
    \widehat{\cR}_{ID} \leq  \epsilon^2 \|u\|_2^2 \| g(W^T X)\|_2^2 / n,
\end{equation*}
where $X\in\R^{d\times n}$ is a matrix whose columns are the pruning data.
\end{customlemma}

\begin{proof}
\begin{equation}
    \hat{\mathcal{R}}_{ID}=\frac{1}{n}\sum_{i=1}^n |u^\top g(W^\top x_i) - \widehat{u}^\top g(\widehat{W}^\top x_i) |^2
\end{equation}
Here, we can appeal to our definition of the ID to bound each term in the sum.    

\begin{equation}
    |u^\top g(W^\top x_i) - \widehat{u}^\top g(\widehat{W}^\top x_i) | =|u^\top g(W^\top x_i) - {u}^\top  T^\top g(P^\top {W}^\top x_i) |
\end{equation}
By our definition of an $\epsilon$-accurate interpolative decomposition, 

\begin{equation}
    |u^\top g(W^\top x_i) - \widehat{u}^\top g(\widehat{W}^\top x_i) | \leq  \epsilon \|u\| \| g(W^T x_i)\|
\end{equation}

Therefore, 

\begin{equation}
    \hat{\mathcal{R}}_{ID} \leq \frac{1}{n} \epsilon^2 \|u\| ^2 \| g(W^T X)\|^2
\end{equation}
\end{proof}

\begin{lemma}
\label{lem:eta}
The maximum $\eta$ of the loss function associated with approximating the full network with the pruned on is bounded as   
\[
    \eta \leq \sup_{x\in\Omega_x} \|u\|_2^2 \| g(W^T x)\|_2^2 \|(1+ \|T\|))^2
\]

\end{lemma}
\begin{proof} 
\begin{equation*}
    \eta=\max_{x, W, u} \| u^\top g(W^\top x ) - \widehat{ u}^\top g (\widehat{W}^\top x) \|^2
\end{equation*}
For any $x\in\Omega_x$ we have the bound 
\begin{equation*}
\begin{aligned}
    \| u^\top g(W^\top x ) - \widehat{ u}^\top g (\widehat{W}^\top x) \|^2 &\leq (\|u^\top g(W^\top x) \| + \|\widehat{ u}^\top g (\widehat{W}^\top x) \|)^2\\
     &\leq (\|u^\top g(W^\top x) \| + \|{u}^\top  T^\top g(P^\top {W}^\top x)\|)^2\\
     &\leq (\|u^\top g(W^\top x)\|(1+ \|T\|))^2.\\
\end{aligned}
\end{equation*}
Therefore, 
\begin{equation*}
    \eta \leq \sup_{x\in\Omega_x} \|u\|_2^2 \| g(W^T x)\|_2^2 \|(1+ \|T\|))^2
\end{equation*}

%Here, we appeal to the existence of a matrix T which interpolates $g(W^\top x)$ such that each $t_{ij} \leq 2$ \megan{cite}.  This bounds $\|T\| $ \megan{check what the tightest bound on this is is it 2m?  }
\end{proof}

\begin{remarks}
We can explicitly measure the norm of the interpolation matrix $T$ that appears in the upper bound of Lemma~\ref{lem:prunedRisk}.  Moreover, we expect this to be small because there exists an interpolation matrix such that  $t_{ij} \leq 2 \forall \{i,j\}$~\cite{liberty2007randomized}. The better the interpolation matrix, the better the bound.    
\end{remarks}
%\subsection{A note about sparse regression}
%\jerry{Keep?}
%Instead of a rank-revealing QR factorization, one could also use a sparse regression approach.
%Ultimately we did not proceed with this approach because it empirically did not perform as well.





\section{Correlation between random trials}
\label{app:sec:correlation}
We introduce a metric that we call "model correlation" in order to evaluate different pruning methods.  We define the correlation between two models on a data set as the percent of labels that the two models agree on, irrespective of if those labels agree with the ground truth.  There are situations in which the user of a network may care about more than just the simple accuracy --- it may matter which items a network is most likely to get wrong, and how.  It is also possible for two networks to have the same accuracy but perform very differently on subsets of the dataset.  We include this metric after fine tuning to demonstrate the efficacy of our compression method relative to methods that necessitate extensive fine tuning and, therefore, may not correlate well with the original model.  Here we provide some baseline measurements of what affects model correlation, in order to better understand this metric.  

Our baseline measurement of model correlation is the model correlation between two same-sized but randomly initialized networks trained using the same hyper parameters but with different (random) data orders.  We first test the effects for a fully connected one hidden layer network on FashionMNIST. 

\begin{table}[h!]
\begin{center}
 \begin{tabular}{||c c c c c||} 
 \hline
Size & Same Initialization & Same Data Order & Vary Both & Accuracy\\ [0.5ex] 
 \hline\hline
 500 & 94.6 & 94.0 & 93.3 & 89.21 \\ 
 \hline
 2000 & 96.7 & 94.7 & 94.5& 89.63\\
 \hline
 4000 & 97.3 & 95.4 & 95.1 &89.74\\
 \hline
\end{tabular}
\caption{Correlation data for a single hidden layer fully connected network on the FashionMNIST data set. We keep the same initialization but vary the data order (Same Initialization), use the same data order but vary the initialization (Same Data Order) or vary both the data order and initialization (Vary both). This data is averaged over 9 trials.  We see that starting at the same initialization increases the correlation at the end of training, and, interestingly, that using the same data order during training can increase the correlation as well. }
\end{center}
\end{table}
We continue our experiments on the CIFAR10 dataset using the VGG-16 architecture.  We find that two differently randomly initialized models trained using different data orders to  the same state-of-the-art accuracy (93.6\%) agree on classifications 93.0\% of the time.  The effect of data order is also seen on CIFAR10 VGG-16 networks.  When we prune to 50\% FLOPS and then re-train a VGG-16 network using magnitude pruning, if the data order is the same as the original network, then the correlation is 94.01\%.  However, using a different data order the correlation is 93.15\%. The model correlation breaks down quickly when we use a large learning rate (.1 for 200 epochs).  

\section{Comparison methods}
\label{app:sec:comparison_methods}
Here we provide a key for the various methods we compare to in the main text, along with their classifications within our taxonomy. %(Table~\ref{tab:taxonomy}).  
We give both the paper citation and implementation citation.

\begin{table}[h!]
\label{tab:citations}
%\small
\begin{center}
 %\begin{tabular}{||c c c c c||} 
 \begin{tabular}{c c c c c} 
 %\hline 
 \toprule
 \textbf{Dense} & \textbf{Structure Preserving} & \textbf{Corrects Next Layer} & \textbf{No Local FT} & \\ 
 %\textbf{Dense}&\textbf{Structure Preserving}& \textbf{Corrects Next Layer}&\textbf{No Local FT}&\\ 
 \midrule
 %\textbf{Name} & \textbf{Citation} &\textbf{Implementation} & \textbf{Table} & \textbf{Figures}\\ 
 Name & Citation & Implementation & Table & Figures \\ 
 \midrule
 ID & (Ours) &(Ours) & All & All\\
 PFP & \citet{liebenwein2020provable} &\citet{liebenwein2020provable}& - & \ref{fig:vgg16preft}\\ 
 AMC & \citet{he2018amc} & \citet{he2018amc} & \ref{tab:vggImgNet} & -\\
 \midrule
 \\\\
 \toprule
 \textbf{Dense}&\textbf{Structure Preserving}& \textbf{Corrects Next Layer}&\textbf{Local FT}&\\ 
 \midrule
 Name & Citation & Implementation & Table & Figures \\ 
 \midrule
 Thi & \citet{luo2017thinet} & \citet{liebenwein2020provable} & \ref{tab:vggImgNet} & \ref{fig:vgg16preft}\\ 
 CP & \citet{he2017feat} & \citet{he2017feat} &  \ref{tab:vggImgNet} & -\\ 
 NS & \citet{liu2017netslim} &\citet{zhuang2020polar} & \ref{tab:cifarVgg} & -\\ 
 \midrule
 \\\\
 \toprule
 \textbf{Dense}&\textbf{Structure Preserving}&\textbf{No Correction}&&\\
 \midrule
 Name & Citation & Implementation & Table & Figures \\ 
 \midrule
 Uni & Uniform Random Filter Pruning &\citet{liebenwein2020provable}&  - & \ref{fig:vgg16preft}\\ 
 Soft & \citet{softnetHe} &\citet{liebenwein2020provable}&  - & \ref{fig:vgg16preft}\\ 
 StructMag &\citet{li2017l1} & \citet{liebenwein2020provable} & \ref{tab:cifarVgg} & \ref{fig:vgg16preft}\\ 
 FPGM & \citet{he2019fpgm} &(Ours)&  \ref{tab:cifarVgg} & -\\
 {HRank} & ~\citet{lin2020hrank}  &~\citet{lin2020hrank} &  \ref{tab:cifarVgg} & -\\ 
 \midrule
 \\\\
 \toprule
 \textbf{Dense}&\textbf{Extra Layers}&&&\\
 \midrule
 Name & Citation & Implementation & Table & Figures \\ 
 \midrule
 ALDS & \citet{liebenwein2021alds} &\citet{liebenwein2021alds}&  - & \ref{fig:mobilenet_atom3d},\ref{fig:combiningID} \\ 
 Messi & \citet{Maalouf2021DeepLM} &\citet{liebenwein2021alds}&  - & \ref{fig:mobilenet_atom3d},\ref{fig:vgg16preft}\\ 
 PCA & \citet{zhang20153dfilter} &\citet{liebenwein2021alds}&  - &\ref{fig:mobilenet_atom3d}, \ref{fig:combiningID}\\ 
 SVD & \citet{denten2014svd} &\citet{liebenwein2021alds}&  - & \ref{fig:mobilenet_atom3d}, \ref{fig:vgg16preft}\\ 
 LRank & \citet{idel2020lrank} &\citet{liebenwein2021alds} &  \ref{tab:cifarVgg},\ref{tab:vggImgNet}& \ref{fig:combiningID}\\ 
 Polar & \citet{zhuang2020polar}&\citet{zhuang2020polar}&  \ref{tab:cifarVgg} & -\\
 \midrule
 \\\\
 \toprule
 \textbf{Sparse}&&&&\\
 \midrule
 Name & Citation & Implementation & Table & Figures \\ 
 \midrule
 SiPP & \citet{sippBayal}&\citet{sippBayal} &  - & \ref{fig:vgg16preft}\\ 
 Snip & \citet{lee2019snip} &\citet{sippBayal}& - & \ref{fig:vgg16preft}\\
 Thres & \citet{li2017l1} &\citet{liebenwein2020provable}&  - & \ref{fig:vgg16preft}\\ 

\hline
 \end{tabular}

 \end{center}
       \caption{
    Taxonomy of pruning and methods and look-up table for references.  As you go further down the list, methods tend to become more different from our own.  
    }
 \end{table}
 
% \begin{table}[h!]
%%\small
%\begin{center}
% \begin{tabular}{||c c c c c c c||} \hline
% Name & Full Name & Citation &Impl. & Cat.& Tab. & Figs.\\ \hline
% ID & Interpolative Decomposition & (Ours) &(Ours)& E & All & All\\ 
% ALDS & Automatic Layer-wise Decomposition Selector & \citet{liebenwein2021alds} &\citet{liebenwein2021alds}& I& - & \ref{fig:mobilenetCifar},\ref{fig:CombineCifar}, \ref{fig:compose_vggImgNet} \\ 
% Messi & Multiple Estimated SVDs for Smaller Intralayers & \citet{Maalouf2021DeepLM} &\cite{liebenwein2021alds}& I& - & \ref{fig:mobilenetCifar},\ref{fig:vggCifar}\\ 
% PCA & Accel Very Deep ConvNets for Class. \& Detect. & \citet{zhang20153dfilter} &\citet{liebenwein2021alds}& I& - & \ref{fig:mobilenetCifar},\ref{fig:CombineCifar}, \ref{fig:compose_vggImgNet}\\ 
% SVD & Exploiting Linear Struct. w/in ConvNets & \citet{denten2014svd} &\citet{liebenwein2021alds}& F& - & \ref{fig:mobilenetCifar}, \ref{fig:mobilenetImgNet}, \ref{fig:compressonly_vggImgNet}\\ 
% Uni & Full Name & Citation &\citet{liebenwein2020provable}& Cat.& - & \ref{fig:vggCifar}\\ 
% PFP & Provable Filter Pruning & \citet{liebenwein2020provable} &\citet{liebenwein2020provable}& Cat& - & \ref{fig:vggCifar}\\ 
% SiPP & Sensitivity-informed Provable Pruning & \citet{sippBayal}&\cite{sippBayal} & A& - & \ref{fig:vggCifar}\\ 
% Snip & Single-shot Network Pruning & \citet{} &\citet{sippBayal}&G& - & \ref{fig:vggCifar}\\ 
% Thi & ThiNet: A Filter Level Pruning Method & \citet{luo2017thinet} & \citet{liebenwein2020provable} & Cat& \ref{tab:vggImgNet} & \ref{fig:vggCifar}, \ref{fig:compressonly_vggImgNet}\\ 
% Soft & Soft filter pruning & \citet{softnetHe} &\citet{liebenwein2020provable}& Cat& - & \ref{fig:vggCifar}\\ 
% Thres &  Pruning Filters for Efficient ConvNets &\citet{li2017l1} &\citet{liebenwein2020provable}& A& - & \ref{fig:vggCifar}\\ 
% StructMag & Pruning Filters for Efficient ConvNets & \citet{li2017l1} & Impl. & B& \ref{tab:cifarVgg} & \ref{fig:compressonly_vggImgNet}\\ 
%  LRank & Low-Rank Compressino of NNs & \citet{idel2020lrank} &\citet{liebenwein2021alds} & I& \ref{tab:cifarVgg},\ref{tab:vggImgNet}& \ref{fig:CombineCifar}, \ref{fig:compose_vggImgNet}\\ 
%   CP & Channel Pruning for Accel Very Deep NN & \citet{he2017feat} & \citet{he2017feat} & cat& \ref{tab:vggImgNet} & -\\ 
%   AMC & AUtoML for Model Compression & \citet{he2018amc} & \citet{he2018amc} & cat& \ref{tab:vggImgNet} & -\\ 
%   Polar &Structured Pruning using Polarization Regularizer & \citet{zhuang2020polar}&\cite{zhuang2020polar}& cat& \ref{tab:cifarVgg} & -\\ 
%  NS &Network Slimming & \citet{liu2017netslim} &\citet{zhuang2020polar}& cat& \ref{tab:cifarVgg} & -\\ 
%  FPGM &Filter Pruning via Geometric Median & \citet{he2019fpgm} &(Ours)& cat& \ref{tab:cifarVgg} & -\\ 
%
%\hline
%
% \end{tabular}
% \end{center}
%% \label{tab:citations}
% \end{table}
 
\section{Setting $k(\epsilon)$ per layer in deep networks with iterative pruning}
\label{app:sec:iterativeID}
By Definition~\ref{def:ID}, an $\epsilon$-accurate interpolative decomposition is associated with a number $k$ of selected columns.
We observe that for deep networks it is not straightforward to apply a single accuracy $\epsilon$ to the entire network.
Figure \ref{fig:vggMet} illustrates the representative variety in the layer-wise singular value decay for a trained VGG-16 model.
For our method a sharper singular value decay indicates greater prunability. 
However, not all layers contribute equally to the number of FLOPS and the number of parameters.  Typically convolutional layers contribute disproportionately to the number of FLOPs compared to the number of parameters they contain.
When we prune neurons or channels in a layer, that has an effect on the number of FLOPS performed by that layer, and also in the next layer.  Therefore, we iteratively prune the network by finding the layer that will allow us to prune the most FLOPs compared to the error that we expect from pruning that layer, and pruning that layer.  This is given in Algorithm \ref{alg:deepIDIter}.  

\begin{algorithm}[t]{\small
\caption{Pruning a multilayer network with iterative interpolative decompositions}
\begin{algorithmic}
\label{alg:deepIDIter}
%\INPUT
\REQUIRE
Neural net $h(x; W^{(1)},\ldots,W^{(L)})$,
pruning set $X$,
step size $\lambda$,
FLOPs ratio $\rho$
%\OUTPUT
\ENSURE
Pruned network $h(x; \widehat{W}^{(1)},\ldots,\widehat{W}^{(L)})$
\vspace{0.5em}
\FOR{$l \in \{1, \dots, L\}$}
\STATE $\widehat{W}^{(l)} \gets W^{(l)}$
\ENDFOR
\STATE $F \gets \operatorname{Compute\_FLOPs}(h(x; \widehat{W}^{(1)},\ldots,\widehat{W}^{(L)})$
\STATE $F_\rho \gets F * \rho$
\WHILE{$F > F_\rho$}
\STATE $S, K \gets \{\}$
\FOR{$l \in \{1, \dots, L\}$}
\STATE $Z \gets h_{1:l}(X; W^{(1)}, \dots, W^{(l)})$
\COMMENT{layer l activations}
\STATE $R \gets \operatorname{Pivot\_QR}(\operatorname{Reshape}(Z))$
\COMMENT{reshape if Conv layer}
\STATE $k \gets \operatorname{num\_channel}(\widehat{W}^{(l)} \times \lambda$
\COMMENT{calculates proportion of channels or neurons to potentially remove}
\STATE $Err \gets |R[k+1,k+1]/R[1,1]|$
\COMMENT{error from ID approximation}
\STATE $F_l \gets \operatorname{Compute\_FLOPs}(\widehat{W}^{(l)}, \widehat{W}^{(l+1)})$
\COMMENT{compute FLOPs of current and next layer}
\STATE $S.\operatorname{append}(Err / F_l)$
\COMMENT{weighted layer prunability score}
\STATE $K.\operatorname{append}(k)$
\ENDFOR
\STATE $l \gets \operatorname{argmin}(S)$
\STATE $\widehat{W}^{(l)} \gets \operatorname{ID}$ prune layer $l$ to $K[l]$ neurons or channels
\STATE $F \gets \operatorname{Compute\_FLOPs}(h(x; \widehat{W}^{(1)},\ldots,\widehat{W}^{(L)})$
\ENDWHILE
%\STATE $T^{(0)} \gets I$ \;
%\FOR{$l \in \{1 \dots L\}$}
%\STATE $Z \gets h_{1:l}(X; W^{(1)}, \dots, W^{(l)})$
%\COMMENT{layer l activations}
%%\COMMENT{compute activations of layer l}
%\IF{layer $l$ is a FC layer}
%\STATE $(\I, T^{(l)}) \gets \operatorname{ID}(Z^T; \alpha) \textbf{ if } l \notin S \textbf{ else } (:, I)$ \;
%%prune $\alpha\%$ of neurons with ID of $Z^\top$: $\I, T$\;
%%compute rank-$k$ ID of $Z^\top$: $\I, T$\;
%\STATE $\widehat{W}^{(l)} \gets T^{(l-1)} W^{(l)}_{:,\I}$
%\COMMENT{sub-select neurons, multiply T of prev layer's ID}
%%\tcp{select neurons in current layer}
%%$\widehat{W}^{(l+1)} \gets T \widehat{W}^{(l+1)}$
%%\;
%%\tcp{propagate T to next layer}
%\ELSIF{layer l is a Conv layer (or Conv+Pool)}
%\STATE $(\I, T^{(l)}) \gets \operatorname{ID}(\Reshape(Z); \alpha) \textbf{ if } l \notin S \textbf{ else } (:, I)$ \;
%% prune $\alpha\%$ of channels with ID of $\Reshape(Z)$: $\I, T$\;
%%compute rank-$k$ ID of $\Reshape(Z)$: $\I, T$\;
%\STATE $\widehat{W}^{(l)} \gets \Matmul(T^{(l-1)}, W^{(l)}_{\I,\ldots})$
%%\;
%\COMMENT{select channels; multiply T} %in current layer
%% $\widehat{W}^{(l+1)} \gets \Matmul(T, \widehat{W}^{(l+1)})$
%%\tcp{propagate T to next layer}
%%\tcp{depends if next layer is FC or Conv}
%\ELSIF{layer l is a Flatten layer}
%\STATE $T^{(l)} \gets T^{(l-1)} \otimes I \,\,$ 
%\COMMENT{expand T to have the expected size}
%\ENDIF
%\ENDFOR
\end{algorithmic}
%%Specify direction\;
%%How to compute Z\;
%\caption{ID pruning a multi-layer neural network}
}\end{algorithm}






\begin{figure}[!tbp]
\centering

  \includegraphics[width=.25\linewidth]{figures/layer_1.pdf}
  \includegraphics[width=.25\linewidth]{figures/layer_8.pdf}
  \includegraphics[width=.25\linewidth]{figures/layer_13.pdf}

    \caption{Metrics for different layers in VGG-16 for Cifar-10.  The leftmost layer is the first convolutional layer.  The center figure is one of the middle convolutional layers, and the rightmost figure is the last convolutional layer.  As we can see, the singular value decay varies throughout the network. }
\label{fig:vggMet}
\end{figure}
\section{Sensitivity of parameters}
\label{sec:sens}
When we prune using Iterative ID, we have a choice of hyperparameter in how what percent of channels to cut per iteration (pruning fraction $\alpha$ in Algorithm~\ref{alg:deepIDIter}). In Figure~\ref{fig:sensitivity}, we demonstrate that the choice of pruning fraction does not greatly affect the accuracy of the network after iterative pruning.  However, using a smaller $\alpha$ results in the network taking significantly more time to prune.  

\begin{figure}[h!]
    \centering
    \includegraphics[width=\linewidth]{figures/sensitivity.pdf}
    \caption{Iterative pruning performed at different amounts of channels/neurons cut per iteration in the chosen layer.  We see that the choice of number of channels per iteration does not have a large impact on the outcome, though smaller percents take a longer time.}
    \label{fig:sensitivity}
\end{figure}

%One benefit of maintaining properties of the original model may include maintaining the fairness of an already fair model.  In order to accomplish this, we want our method to be robust to the contents of the pruning set, including when classes are underrepresented.  While we do not recommend 


\section{Additional experimental details and results}
%FLOPs reduction = 1 - pruned FLOPs / orinal FLOPs.

\subsection{Illustrative example}


We created a simple synthetic data set for which we know the form of a relatively minimal model representation.

Draw $n$ points iid. from the unit circle in 2 dimensions.
Next select 2 (normalized) random vectors $v_1$ and $v_2$.
All labels are initialized to zero, and add or subtract 1 to the label for each time it produces a positive inner product with one of the random vectors.
With the ReLU activation function it is possible to correctly label all points with a one hidden layer fully connected network of width 4.
Construct pairs of neurons, with each pair aligned with the center of one of the random vectors.  We can think of the pair as creating a flat function by using one neuron to "cut the top" off of the round part of the function created by the other.

We parameterize this with an angle $\phi$ away from the pair. If each neuron in the pair has the same weight magnitude $w$, coefficient $\pm u$ and a bias $b \pm \delta/2$, then given a particular $w$, $b$ and $\delta$, we can write:

\begin{equation*}
       u ({\ReLU( w \cos \phi -b +\delta/2)- \ReLU(w \cos \phi -(b-\delta/2))})
\end{equation*}

As long as $w>b$, this looks like a step function, where the sides get steeper as $w$ approaches infinity.  This allows us to perfectly label all of the points given twice the number of neurons as we have random vectors $v_1$ and $v_2$, which is much smaller than the number of data points $n$.

We train an over-parameterized single hidden layer network to perform fairly well on this task, using an initialization scale that is common in some machine learning platforms such as TensorFlow~\cite{tensorflow2015-whitepaper}. This is shown in figure \ref{fig:patchesBonus}.  
However, magnitude pruning will not necessarily recognize the structure of the minimum representative network because both of the neurons in the pair construction may not have a large magnitude.
On this example we see that the interpolative decomposition is able to select neurons which resemble a close to minimal representative network.  

\begin{figure}[!tbp]
\centering
  \includegraphics[width=.4\linewidth]{figures/patchesAcc.pdf}
  \includegraphics[width=.4\linewidth]{figures/patchesFuncEval.pdf}
\caption{A plot of loss v.s. number of neurons kept (left) and a plot of the function evaluation for the full network, ID pruning, and magnitude pruned model (right).  The data is drawn from the unit circle, and we parameterize X in terms of an angle $\theta$. To see more detail in the magnitude pruned function, we draw it with a scale of 10x applied uniformly.  We see that it takes very few (12) neurons to completely represent the function approximated by the network. In fact, the ID pruned function is visually indistinguishable from the full model in this case.  }
\label{fig:patchesBonus}
\end{figure}


In Figure~\ref{fig:patches} we see that the ID well represents the original network by ignoring duplicate neurons and taking into account differences in the bias.
The ID even achieves slightly better test loss than the original model.
Magnitude pruning does not approximate the original model well; it keeps duplicate neurons and fails to find important information with the same number of neurons.

\begin{figure}[!tbp]
\centering
  \includegraphics[width=.3\linewidth]{figures/fullPatch.pdf}
  \includegraphics[width=.3\linewidth]{figures/IDpatch.pdf}
  \includegraphics[width=.3\linewidth]{figures/magPatch.pdf}
    \caption{Neuron visualization for a simple 2-d regression task. The axes are the weights of the neurons, and the color is the bias.  We have the weights trained for a full model(left) with 5000 hidden nodes.  The 12 neurons kept by the ID (center) represent the function well. These are re-scaled by the magnitude of their coefficients in the second layer to display the effect of the ID.  Magnitude pruning (right) keeps duplicate neurons that do not represent the function.  The test loss is 0.037786 for the full model, 0.037781 for the model pruned with ID, and 0.33 magnitude-pruned model. }
\label{fig:patches}
\end{figure}
  
%We created a synthetic dataset, with points drawn iid. from the unit sphere in 3 dimensions.  We select 2 random vectors and normalize.  All of the labels are initialized to zero, and we add 1 to the labels of each of the points with a positive inner product with each of the vectors.  If we use the ReLU activation function, it is possible to correctly label all of the points using a one hidden layer neural network with 2n neurons in its hidden layer.   One arrangement that accomplishes this is by pairing the the neurons, with each pair aligned with each of the centers of the patches.  

%\subsection{Data set and code licenses}
%Fashion MNIST~\cite{datafashionmnist} is licensed under the MIT License.
%We are unaware of a license for the CIFAR-10 data set~\cite{datacifar10}.
%We adapted code and hyper-parameters from \textcite{liu2019rethink} (MIT License), \textcite{li2017l1} (license unaware), and \textcite{he2019fpgm} (license unaware).

%\subsection{Estimating compute time}
%The experiments on the illustrative example and Fashion MNIST were performed on a 2019 iMac running an Intel I9. The illustrative example computes in minutes. We trained 15 different model configurations, 8 one hidden layer, and 7 two hidden layer network sizes.  For 5 random seeds, we used $5*(50+10+10)=350$ epochs per model size, and trained 15 different model configurations.   We used 10,000 images for the pruning set, however, the runtime for computing the ID is cheap compared to training, using the scipy QR decomposition function which calls a LAPACK subroutine~\cite{lapack}.

%The experiments on CIFAR-10 were performed on a NVIDIA GeForce GTX 1080Ti. 
%For 5 random seeds and two models a total of $5*2*160=1600$ epochs was used for training the %ResNet-56 and
%VGG-16 models.
%The Table~\ref{tab:cifar10mag} our results required $5*3*40=1200$ epochs of fine-tuning for 5 random seeds and 2 pruning configurations on ResNet-56, 1 configuration on VGG-16.
%Computing the interpolative decomposition were comparatively cheap, only requiring 1000 data points. 
%The Table~\ref{tab:cifar10modern} and~\ref{tab:cifar10beforeFT} ID results required $5 * 2 * 200 = 2000$ total epochs of fine-tuning for 5 random seeds and 2 different models.






%\paragraph{Fashion MNIST}
\subsection{Fashion MNIST}
\label{sec:fashionmnist}

For one and two hidden layer networks on Fashion MNIST~\cite{datafashionmnist} we compare the performance of ID and magnitude pruning to networks of the same size trained from scratch.
Each pruning method is used to prune to half the number of neurons of the original model, and is then fine-tuned for 10 epochs.
%For our first experiment, we train simple fully connected one and two hidden layer networks on the Fashion MNIST~\cite{datafashionmnist} data set and prune each network to one half of the neurons in each layer. 
%This consists of 60000 grey-scale training images and 10000 test images with 10 classes.  
We use a stochastic gradient descent optimizer with a learning rate of 0.3 which decays by a factor of 0.9 for each epoch to train the initial networks.  Each was trained for 50 epochs.  Our pruning set was 10000 images, which did not need to be held out from training for the simple data set. The fine tuning ran for 10 epochs with an initial learning rate of 0.1 with a decay rate of 0.6 for two layer networks, and 0.2 with a learning rate decay of 0.7 for one layer.  Larger ID-pruned models (greater than 512 neurons) can be fine-tuned with a much smaller learning rate of 0.002. These learning rates and number of epochs may not be optimal but were determined through brief empirical tests. We used 5 random seeds for each size of model.  The error bars are reported as the uncertainty in the mean, defined in terms of the standard deviation $\sigma$, and number of independent trials $N$,  $\sigma_{mean}=\sigma/\sqrt{N}$.
%We compare the performance of the networks trained from scratch, and pruned using magnitude pruning and ID pruning.  
Results are shown in Figure~\ref{fig:fmnist}; before fine-tuning the ID achieves a significantly higher test accuracy than magnitude pruning.
ID pruning with fine-tuning outperforms training from scratch. %, and all methods begin to converge at large network sizes (except just magnitude pruning).
%We see that the pruning and then fine tuning technique out performs training from scratch for moderate network sizes, and that all of the methods except magnitude pruning without fine tuning begin to converge at large network sizes,
At sufficiently large network sizes, ID pruning alone performs similarly to training a network from scratch. 
When we start to see diminishing returns from adding more neurons, the performance of the various methods begin to converge.  
%In addition by about $2^9$ hidden neurons we've surpassed the minimal network size and see diminishing returns from more neurons.  


\begin{figure}
\centering
  \centering
  \includegraphics[width=.47\linewidth]{figures/OHL.pdf}
  \hspace{2mm}
  \includegraphics[width=.47\linewidth]{figures/THL.pdf}
  
    \caption{Accuracy for a one hidden layer (left) and two hidden layer (right) fully connected neural networks of varying size on Fashion-MNIST for ID and magnitude pruning, as well as training directly.
    These curves are averaged over 5 trials; error bars report uncertainty in the mean. 
    }
\label{fig:fmnist}
\end{figure}

\begin{figure}[ht]
\centering
  \includegraphics[width=.45\linewidth]{figures/metrics.pdf}
  %\caption{Normalized singular value decay, $||R_{22}||/||R||$, and the magnitude of$r_{kk}/r_{00}$ for a matrix Z from a one hidden layer neural network trained on Fashion MNIST.  We see that the three metrics generally correlate well in a practical setting.}
  \includegraphics[width=.45\linewidth]{figures/accuracyvsrkk.pdf}
  %\caption{Accuracy of a single hidden layer fully connected neural network pruned from 4096 to k neurons.  The horizontal lines are the accuracy of the full sized network  }

    \caption{Left: Normalized singular value decay, $\|R_{22}\|_2/\|R\|_2$, and $\lvert r_{k+1,k+1}/r_{1,1}\rvert$ for a matrix from~\eqref{eq:1hiddenfc}) from a one hidden layer network (i.e., $g(W^\top X)$) trained on Fashion MNIST. We see that the metrics generally correlate well in this setting. Right: Accuracy of a single hidden layer network pruned from width 4096 to $k.$  The horizontal lines are the accuracy of the full sized network. The difference  between pruning and test accuracy is due to a slight class imbalance in the canonical test set.}%Note that the slight difference between the pruning and test set exists for the full network and is due to a slight imbalance in classes for the canonical testing set.}
\label{fig:chosingk}
\end{figure}


%\subsection{Ablation studies for design of Algorithm~\ref{alg:deepID}}
%We found that using a held-out pruning set was important to prevent the interpolative decomposition from over-fitting to the training data, and improve generalization performance to the test set.
%Table~\ref{tab:prunesetAblation} reports our results.
%
%\begin{table}[]
%    \centering
%    \caption{Ablation study on the effect of using a held-out (or not) pruning set for the interpolative decomposition.
%    A ResNet-56 model on CIFAR-10 was pruned to 51\% FLOPs reduction with a pruning set of 1000.
%    The pruning set was either held-out from the test set, or randomly sampled from the training set.
%    Accuracies are reported as mean and standard deviation over 5 independent trials.
%    \\}
%    \label{tab:prunesetAblation}
%    \begin{tabular}{ccc}
%    \toprule
%    Interpolative Decomposition &
%    Baseline  &
%    Pruning \\
%    Pruning Set & 
%    Acc. (\%) &
%    Acc. (\%)\\
%    \midrule
%    Held Out &
%    \multirow{2}{*}{93.04 ($\pm$ 0.25)} & 
%    69.64 ($\pm$ 2.14) \\
%    Not Held Out & &
%    66.34 ($\pm$ 1.30)  \\
%    \bottomrule
%    \end{tabular}
%\end{table}
%
%
%In Algorithm~\ref{alg:deepID} we apply the ID from the beginning to the end of the multi-layer network.
%Doing so one could use the interpolative decomposition to approximate the activation outputs of either the original model or the model.
%Table~\ref{tab:Zablation} gives an ablation study which compares the pruning accuracy for both such cases.
%We observe that using the ID to approximate the original model achieves higher pruning accuracy (before fine-tuning).
%We believe that in deep networks there is greater concern for the ID to propagate errors forward through the matrix.
%Thus by approximating the original model's activation outputs, we can mitigate some of this error propagation.
%
%
%\begin{table}[]
%    \centering
%    \caption{Ablation study on the effect of approximating the activation outputs of the original or pruned model using the interpolative decomposition.
%    A ResNet-56 model on CIFAR-10 was pruned to 51\% FLOPs reduction with a held-out pruning set of size 1000.
%    Accuracies are reported as mean and standard deviation over 5 independent trials.
%    \\}
%    \label{tab:Zablation}
%    \begin{tabular}{ccc}
%    \toprule
%    Interpolative Decomposition &
%    Baseline  &
%    Pruning \\
%    Approximation Target & 
%    Acc. (\%) &
%    Acc. (\%)\\
%    \midrule
%    Original Model &
%    \multirow{2}{*}{93.04 ($\pm$ 0.25)} & 
%    69.64 ($\pm$ 2.14) \\
%    Pruned Model & &
%    63.19 ($\pm$ 4.66)  \\
%    \bottomrule
%    \end{tabular}
%\end{table}


%\subsection{Learning rate and prunability}
%\outline{LR and SVD}

\subsection{Additional ImageNet experiments on MobileNet V1}
\label{app:sec:imgnet_extra}
Here we give additional pruning experiments on Mobilenet V1 for ImageNet.
We see that the ID is competitive against compression methods which do not do any local fine tuning (Figure~\ref{fig:mobilenetImgNet_nolocFT}.
Figure~\ref{fig:mobilenetImgNet_yeslocFT} includes methods which do use local fine tuning; ID performs better than two out of the three.
Interestingly, we see that although the PCA and LRank methods performed well on an overparameterized network such as VGG-16, they do not work well on a much more efficient network.

\begin{figure}
\centering
\begin{subfigure}{.45\textwidth}
\centering
\includegraphics[width=.95\linewidth]{figures/mob_nolocFT_ImgNet.pdf}
\caption{Compare with methods that do not use any fine tuning.
% We see that the ID is superior in this setting.
}
\label{fig:mobilenetImgNet_nolocFT}
\end{subfigure}
\begin{subfigure}{0.45\textwidth}
\centering
\includegraphics[width=.95\linewidth]{figures/mob_yeslocFT_ImgNet.pdf}
\caption{Comparing with methods that use a local fine tuning correction.  
%We see that the ID is superior in this setting.
}
\label{fig:mobilenetImgNet_yeslocFT}
\end{subfigure}
\centering
\caption{MobileNet V1 compression results on ImageNet. Note that the matrix methods often work by changing the depth of convolution in the network, and given that MobileNet uses depth-wise separable convolutions, it is not surprising that matrix methods would perform poorly.  }
\end{figure}


\subsection{Imagenet pre fine tuning correlation}
\label{app:sec:imgnet_preft_corr}

\begin{figure}
    \centering
    \includegraphics[width=0.47\linewidth]{figures/mobCorr_ImgNet.pdf}
    \caption{MobileNet correlation results compared against various compression methods. Note that several other compression methods work by doing decompositions on the weight matrix and changing the number of groups in the convolution.  Due to the MobileNet architecture, this may result in a very low accuracy for methods that work well on other architectures.}
    \label{fig:movCorr_ImgNet}
\end{figure}

\begin{figure}
\centering
\begin{subfigure}{.47\textwidth}
\centering
\includegraphics[width=.95\linewidth]{figures/vggCorr_nolocFT_ImgNet.pdf}
\caption{
Comparing with methods that do not use a local fine tuning correction.
}
\label{fig:vggCorrImgNet_nolocFT}
\end{subfigure}
\begin{subfigure}{0.47\textwidth}
\centering
\includegraphics[width=.95\linewidth]{figures/vggCorr_yeslocFT_ImgNet.pdf}
\caption{
Comparing with methods that use a fine tuning correction.
}
\label{fig:vggCorrImgNet_yeslocFT}
\end{subfigure}
\centering
\caption{VGG-16 correlation results on ImageNet}
\end{figure}

Here we report correlation results om ImageNet with the MobileNet V1 and VGG-16 models.
For methods which do not incorporate a fine tuning correction (Figure~\ref{fig:vggCorrImgNet_nolocFT}), we see that the ID proves as good as any other method we compare to.
Figure~\ref{fig:vggCorrImgNet_yeslocFT} compares to methods which do incorporate a local fine tuning correction. 
Similar to accuracy, we see that composing ID with another compression method improves upon either method.





\subsection{Hyper-parameter details}
\paragraph{CIFAR-10}
The VGG-16 models are trained with 5 random seeds and with hyper-parameter specifications and code provided by \citet{liu2019rethink}.
The test set is randomly partitioned into a prune set and new test set.
The Iterative ID used a held out set of size 1000.
%The held-out pruning set is randomly sampled from the test set.
%This shrinks the test set slightly, from 10000 to 9000.
%For magnitude pruning we use the hyper-parameters specified by \textcite{liu2019rethink} to fine-tune for 40 epochs and learning rate 0.001.
%The interpolative decomposition uses a held out pruning set of 1000 data points and the same fine-tuning hyper-parameters, except the ResNet-56 is retrained with initial learning rate 0.01 and decreased to 0.001 after 10 epochs.
%The VGG-16 uses the same hyper-parameters as~\cite{liu2019rethink}.
%For Table~\ref{tab:cifarVgg}, we implement the method by \textcite{he2019fpgm} with their provided code and hyper-parameter settings on our trained models to fine-tune for 200 epochs.
%The interpolative decomposition uses a held out pruning set of 1000 data points with a hyper-parameter configuration from~\textcite{he2019fpgm}: use a starting learning rate 0.1 decaying to 0.02, 0.004, 0.0008 at epochs 60, 120, 160 to train for 200 epochs total with batch size 128 and weight decay 5e-4.
For the iterative ID on VGG-16, we prune 10\% of a layer per iteration.  %This results in pruning the 
For VGG-16 fine tuning we use SGD with initial learning rate of 5e-3, reduced to 2.5e-3 after 20 epochs and again reduced to 1e-3 after a another 20 epochs have passed.
We use a batch size of 128, momentum of 0.9, and weight decay of 5e-4.

%\paragraph{Mobilenet V1}
The full-size Mobilenet V1 network for Cifar-10 was trained using the ADAM optimizer for 120 epochs, using the default parameters of lr=.001, betas= (.9, .999).  The test set is randomly partitioned into a prune set and new test set.  

\paragraph{ImageNet}
For VGG-16 the iterative ID algorithm uses a randomly held out set of 5000 images with a stepsize parameter of 5\%.
For fine-tuning, we use SGD with a learning rate of 1e-7, batch size of 256, momentum of 0.9, and weight decay of 1e-4.
For ID+PCA, we switched to learning rate 1e-8 at 75 epochs.

For Mobilenet V1 we prune to a constant fraction using the ID with a randomly sampled held out set of 1000 images.


\subsection{Estimating compute}
The experiments on the illustrative example and Fashion MNIST were performed on a 2019 iMac running an Intel I9. The illustrative example computes in minutes. We trained 15 different model configurations, 8 one hidden layer, and 7 two hidden layer network sizes.  For 5 random seeds, we used $5*(50+10+10)=350$ epochs per model size, and trained 15 different model configurations.   We used 10,000 images for the pruning set, however, the runtime for computing the ID is cheap compared to training, using the scipy QR decomposition function which calls a LAPACK subroutine~\cite{lapack}.

The experiments on CIFAR-10 were performed on a NVIDIA GeForce GTX 1080Ti on a university compute cluster. 
Epochs of fine tuning are specified in the tables of the main paper.
Computing the interpolative decomposition were comparatively cheap, only requiring 1000 data points. 
Computing costs for any compression method were negligible compared to fine tuning.

ImageNet experiments were conducted on a university compute cluster.
For fine tuning NVIDIA GeForce RTX 3090, RTX A6000, or TITAN RTX were used.
The compression portion was conducted on a NIVIDIA GeForce GTX 1080Ti or FTX 2080Ti.
Epochs of fine tuning are specified in the tables of the main paper.
The compute cost of the compression methods before fine tuning (including interpolative decomposition) were trivial compared to any amount of global fine tuning.



\section{Sensitivity to pruning set}
\label{sec:sensitivity}
\subsection{Pruning set size}
We compare the pre-fine-tuning accuracy as a function of the pruning set size. 
Here we show that the accuracy is not particularly sensitive to the pruning set size.
Table~\ref{tab:id_sensitivity_imagenet} shows on ImageNet that our ID-based pruning method is robust to the prune set size, and in fact quite efficient.
Figure~\ref{fig:id_sensitivity_cifar10} shows on CIFAR-10 that this trend persists across a wide range of compression levels.
Note that the number of pruning examples must be at least the number of neurons or channels that we prune to for each layer.
%Pruning using the interpolative decomposition only takes a modes pruning set size.  At a minimum, it requires at least as many pruning examples as the number of neurons or channels that we wish to prune to. Here we show that the outcome of pruning is not particularly sensitive to the pruning set size. 

\begin{table}[]
    \centering
    \begin{tabular}{cc}
        Prune set size & Pre-fine-tuning accuracy \\
        \hline
        5k & 68.03 \\
        10k & 68.06 
    \end{tabular}
    \caption{Pre-fine-tuning accuracy compared to prune set size for VGG16 model pruned to 25\% FLOPs reduction on ImageNet.}
    \label{tab:id_sensitivity_imagenet}
\end{table}

\begin{figure}
\centering
%\begin{subfigure}{.87\textwidth}
\centering
\includegraphics[width=.5\linewidth]{figures/setsize.pdf}
\caption{
Comparing the pre-fine-tuning accuracy and FLOP reduction for different pruning set sizes on VGG-16 Cifar-10 model.  We see that above the minimum threshold, pruning set size has a minimial impact on the accuracy of the pruned model. 
}
\label{fig:id_sensitivity_cifar10}
%\end{subfigure}
\end{figure}

\subsection{Pruning set contents}

We want our method to be robust to the data selection method used to generate our pruning set. 
We test our method to this specific sensitivity by removing an entire class from the pruning set.
%This may include situations where an entire class is missing from the pruning set. 

We begin with a full-sized VGG-16 CIFAR-10 model that was trained
%to perform reasonably well
on all 10 classes.  Then we draw a pruning set from only 9 of the classes, completely leaving out images from one of the classes. We prune to 50\% of the original FLOPs, using Iterative ID, and compare the per-class test accuracies for each of the 10 classes. Figure~\ref{fig:id_sensitivity_class} shows that the ID maintains good accuracy even on images from the class that was excluded from the pruning set.  

We expect that other methods which preserve the model's decision boundaries (and therefore correlation) will likely show similar results.  However, methods which require extensive fine tuning and effectively re-train the network will likely not recover accuracy on the missing class.  To demonstrate this, we prune the same full-sized network using magnitude pruning using the same pruning set to 50\% FLOPs reduction, and then fine tune with only 9 classes in the fine tuning set.  As shown in Figure~\ref{fig:mag_sensitivity_class}, the accuracy for the other 9 classes recovers, however, the accuracy for the class that was removed does not.  

This experiment
%is a contrived example, but it 
suggests that pruning methods which maintain properties of the original model may potentially be able to maintain fairness (i.e. per-class accuracy) even when some classes of data are under-represented.  
Our compression method was able to reasonably preserve per-class accuracies (our measure of fairness), even without access to data from one of those classes.

%We began with a ``fair model'' which performed well on all 10 classes of CIFAR-10, and prune it to 50\% of the original FLOPs while maintaining accuracy on all classes, even without access to data from one of those classes.  

\begin{figure}
\centering
\begin{subfigure}{.47\textwidth}
\includegraphics[width=.95\linewidth]{figures/subclass.pdf}
\caption{Per class accuracies while pruning a VGG-16 model using only data from 9 classes.}
\label{fig:id_sensitivity_class}
\end{subfigure}
\begin{subfigure}{.47\textwidth}
\includegraphics[width=.95\linewidth]{figures/IDcorr.pdf}
\caption{Model Correlation when ID pruning using data from only 9 classes}
\label{fig:id_sensitivity_corr}
\end{subfigure}
\caption{
Per-class accuracies as we prune a VGG-16 model on CIFAR-10 with only access to data from 9/10 classes.  No fine tuning was done.  We see that ID maintains reasonable accuracy on all (including the unrepresented) classes, and stays correlated with the original model.  
}

\end{figure}

\begin{figure}
\centering
\begin{subfigure}{.47\textwidth}
\includegraphics[width=.95\linewidth]{figures/magFT.pdf}
\caption{Per class accuracies while fine tuning a magnitude-pruned model with only 9 classes}
\label{fig:mag_sensitivity_class}
\end{subfigure}
\begin{subfigure}{.47\textwidth}
\includegraphics[width=.95\linewidth]{figures/magFTcorr.pdf}
\caption{Model Correlation while fine-tuning a magnitude pruned model using data from only 9 classes.}
\label{fig:mag_sensitivity_corr}
\end{subfigure}
\caption{
Comparing class accuracies for a 50\% FLOPS VGG-16 model pruned using magnitude pruning, and then fine-tuned using only 9 out of 10 classes of CIFAR-10.  We see that the model recovers on the represented classes, but not on the unrepresented class. 
}

\end{figure}




\end{document}