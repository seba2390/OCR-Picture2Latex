%\mbcomment{Add motivation - examples (I have copied some from the other paper). Also add related work.}

%In this paper we study price competition between sellers of perfect complement goods.

In this paper we study a model of a pricing game between two firms
that sell goods that are perfect complements to each other. These goods are only demanded in bundles, at equal quantities, and there is no demand for each good by itself.
%(each buyer demands a pair of items, one from each seller, and each item has zero value by itself).
%In this paper we study of model of a pricing game between two firms
%that sell complementary goods (each buyer demands a pair of items, one from each seller, and each item has zero value by itself).
The two sellers simultaneously choose prices $p_1, p_2$
and the demand at these prices is given by $\mathcal{D}(p_1+p_2)$ where
$\mathcal{D}$ is the demand for the bundle of these two complementary
goods. %  (we assume that each item by itself has zero value to the buyers).
The revenue of seller $i$ is thus $p_i \cdot \mathcal{D}(p_1+p_2)$, and as
we assume zero production costs, this is taken as his utility.

%This model was first studied in Cournot's famous work \cite{Cournot1838}.  In addition to the Cournot Oligopoly model about sellers who compete through quantities, Cournot also considered this model of a duopoly selling perfect complements, zinc and copper.


This model was first studied in Cournot's famous work \cite{Cournot1838}.
In \cite{Cournot1838}, Cournot studied two seminal oligopoly models.
The first, and the more famous, model is the well known Cournot oligopoly model about
sellers who compete through quantities. We study a second model that was proposed by Cournot in the same work, regarding price competition between sellers of perfect complements.\footnote{
\cite{Son68} showed that these two different models by Cournot actually share the same formal structure.
}
In Cournot's example, a manufacturer of zinc may observe that some of her major customers produce brass (made of zinc and copper);
Therefore, zinc manufacturers
%compete not only with other zinc sellers, but they also
indirectly compete with manufacturers of copper, as both target the money of brass producers.
Another classic example of a duopoly selling perfect complements is by \cite{Ell39}, who studied how owners of two consecutive segments of a canal determine the tolls for shippers; Clearly, every shipper must purchase a permit from both owners for being granted the right to cross the canal.
Another, more contemporary, example for perfect complements might be
% \mbedit{two registered patents that both must be licensed by any high-tech or pharmaceutical firm that wants to manufacture a product.}
% a
high-tech or pharmaceutical firms that must buy the rights to use two registered patents to manufacture its product;
The owners of the two patents quote prices for the usage rights, and these patents can be viewed as perfect complements. % for the firm.
%As a final example, consider an international trader who wishes to export goods from country X to country Y, and needs to pay license fees to both countries.

Cournot, in his 1838 book, proved a counterintuitive result saying that
competition among multiple sellers of complement goods lead to a \emph{worse} social outcome than the result reached by a monopoly that controls the two sellers. Moreover, both the profits of the firms and the consumer surplus increase in the monopoly outcome.
In the legal literature, this phenomenon was termed ``the tragedy of the anticommons" (see, \cite{BY00,Hel98,PSD05}). In our work, we will quantify the severity of this phenomenon.
%showed that equilibria of this game never give better
%social welfare than had the two firms been controlled by a single
%monopolist (or behaving as a cartel, maintaining the monopolist price for the bundle).

%
%As mentioned above, this
%model was already studied by Cournot in his classical paper
%\cite{} who showed that equilibria of this game never give better
%social welfare than had the two firms been controlled by a single
%monopolist (or behaving as a cartel, maintaining the monopolist price for the bundle).

Clearly, if the demand at a sufficiently high price is zero,
then there are {\em trivial} equilibria in which both sellers
price prohibitively high, and nothing is sold.
This raises the following question: {\em Do non-trivial equilibria, in which some pairs of items are sold, always exist? }
We study this question as well as some natural follow-ups:
{\em What are the revenue and welfare properties of such equilibria?
What are the properties of equilibria that might arise as a result of best-response dynamics?}

%\ignore{ MB: STILL EDITING
%A sufficient condition for existence of non-trivial equilibria is existence of equilibria in which every agent with zero revenue prices at zero, such equilibria are called ''non-malicious''.
% We show that if the demand has

%\mbcomment{I have added the below to explain why we need descrete demand.}
%It is easy to see that if the demand function is completely arbitrary,  then a seller might not have a best response to any positive price of the other seller:  assume the demand function is  $\mathcal{D}(p)=1/p$ for $p>0$, then for any positive price of one seller, the other seller's revenue if strictly increasing in his price.

% \mbcomment{Should we try and give a more concrete justification? }

For the sake of quantification, we study a {\em discretized version} of this game in which the demand changes only finitely many times.
The number of discrete steps in the demand function, also viewed as the number of possible types of buyers,
is denoted by $n$ and is called the number of {\em demand levels}.

Our first result proves the existence of non-trivial pure Nash equilibria.
% by showing a slightly stronger result - we show that there always exists an equilibrium in which a seller with zero revenue prices at zero, we call such equilibria ''non-malicious''.  Since both sellers pricing at zero is never an equilibrium, existence of non-malicious equilibria implies existence of non-trivial equilibria.

\begin{theorem}
	For any demand function with $n$ demand levels there exists at least one non-trivial pure Nash equilibrium.
\end{theorem}
We prove the theorem using an artificial dynamics which starts from zero prices and continues in steps.
In each one of these steps one seller best responds to the other seller's price, % (breaking ties in favor of zero price),
and after each seller best responds, the total price of both is symmetrized: both prices are replaced by their average.
We show that %in this dynamics
	the total price is monotonically non-decreasing, and thus it
terminates after at most $n$ steps in the non-trivial equilibrium of highest revenue and welfare.
% We actually prove a slightly stronger claim, showing that there always exists an equilibrium in which a seller with zero revenue prices at $0$, we call such equilibria ''non-malicious''.
% % A sufficient condition for existence of non-trivial equilibria is existence of equilibria in which every agent with zero revenue prices at zero, such equilibria are called ''non-malicious''.
% }

In our model, it is easy to observe that there are multiple equilibria for some demand functions. How different can the welfare and revenue of these equilibria be?
A useful parameter for bounding the difference, as well as bounding the inefficiency of equilibria, turns out to be $D$,
the ratio between the demand at price $0$ and the demand at the highest price $v_{max}$
%$v_{max}$
for which there is non-zero demand. % \mbedit{We refer to $D$ as the {\em total demand}.}

Consider the following example with two ($n=2$) types of buyers:
a single buyer that is willing to pay ``a lot'', $2$,
for the bundle of the two goods, and many,
$D-1>>2$, buyers that are willing to pay ``a little'', $1$, each, for the bundle.
A monopolist (that controls both sellers) would have sold
the bundle at the low price $1$. At this price, all the $D$ buyers decide to buy, leading to revenue $D$ and optimal social welfare of $D+1$.
Equilibria here belong to two types:
%There are two equilibria here:
the ``bad'' equilibria\footnote{It turns out that in our model there is no conflict between welfare and revenue in equilibria - the lower the total price, the higher the welfare and the revenue in equilibria (see Proposition \ref{obs:best-NE-well-defined}).}
have high prices, $p_1+p_2=2$, (which certainly is
an equilibrium when, say, $p_1 = p_2 = 1$) and achieve low revenue and
low social welfare of $2$.
The ``good'' equilibria have low prices, $p_1+p_2=1$ (which is an equilibrium as long
as $p_1, p_2 \ge 1/D$), and achieve optimal social welfare as well as the monopolist revenue, both values are at least $D$.
Thus, we see that the ratio of welfare (and revenue) between the ``good'' and ``bad'' equilibria can be very high, as high as $\Omega(D)$.
This can be viewed as a negative ``Price of Anarchy''  result.  % How about the quality of the best equilibria? }
% \mbedit{We note that this is essentially the worst ratio as the welfare ratio between any two non-trivial outcomes is at most $O(D)$.}\lbcomment{maybe "we show/prove" instead of "we note"? Also, do we actually prove it anywhere in the paper?}

We next focus on the best equilibria and  present bounds on the ``Price of Stability" of this game;
	We show that the ratio between the optimal social welfare and the best equilibrium revenue\footnote{Note that this also shows  the same bounds on the ratio between the optimal welfare and the welfare in the best equilibrium, as well as the ratio between the monopolist revenue and the revenue in the best equilibrium.}
	is bounded by $O(\sqrt{D})$, and that this is tight when $D=n$.
	When $n$ is very small, the ratio can only grow as $2^n$ and not more. In particular, for constant $n$ the ratio is a constant, in contrast to the lower bound of $\Omega(D)$ on ``Price of Anarchy'' for $n=2$, presented above.
	
%    We show that the ratio of the best equilibrium welfare and the optimal welfare can grow exponentially in $n$, as $2^n$,	but the proof of this lower bound uses a huge $D$, that is exponential in $n$. 	For $D$ that is not so large, the ratio can be upper bounded by $O(\sqrt{D})$:
%	%We show that there always exist an equilibrium with revenue of at least $\Omega(\frac{1}{ n+\sqrt{D} })$ fraction of the monopolist revenue (the same bound also holds for social welfare).


% OLD: Our next result gives a tight bound for the price of stability in this game. We show that there always exist an equilibrium with revenue of at least $\Omega(\frac{1}{ n+\sqrt{D} })$ fraction of the monopolist revenue (the same bound also holds for social welfare). \lbedit{I removed the paragraph from here, including the discussion on POA}

%
%We study %continue by studying
%the statics of the the game in general:
%how bad can the best and worst non-trivial equilibria of this game be
%compared to each other, as well as compared to the welfare optimal and
%to the monopolist's outcome,
%both for social
%welfare and revenue.
%The example above already shows a gap as large as possible
%(proportional to the market size, $D$) between the worst and best
%equilibria even for very simple markets ($n=2$), both for revenue and for social welfare.
%This also settles the ``price of anarchy'' question as it matches also the trivial
%upper bound of $D$.  The ``price of stability'' question turns out to be more
%delicate, and the best equilibrium is competitive for simple markets, and not completely
%bad, in general:

\begin{theorem}	
% 	There are instances with $n$ demand levels for which the best equilibrium has welfare and revenue that are only $O(2^{-n})$	fraction of the monopolist's revenue and of the optimal welfare, respectively. This result it tight.
	For any instance, the optimal welfare and the monopolist revenue are at most
	$O(\min \{2^n,\sqrt{D}\})$ times the revenue of the best equilibrium. These bounds are tight.
	

	% For any instance, the optimal welfare as well as the monopolist revenue are bounded by $O(n+\sqrt{D})$ times the revenue of the best equilibrium (and thus also its welfare).
	
	%Moreover, there are instances for which the monopolist revenue is $\Omega(2^n)$ larger than the welfare and revenue of any equilibrium, and this result is tight.
	%% in which\footnote{We believe that this technical condition is not necessary but was not able to verify that this is indeed the case.} demands are integer multiples of $1/D$,
	%the best equilibrium has welfare and revenue that is at least  $\Omega(1/(n+\sqrt{D}))$ fraction of the monopolist's revenue and of the optimal welfare.
% 	There are instances with total demand $D$ for which best equilibrium has welfare and revenue that is only $O(1/\sqrt{D})$	fraction of the monopolist's revenue and of the optimal welfare. For any instance with demands that are integers,	
% 	There always exists an equilibrium whose revenue and welfare are at least $\Omega(2^{-n})$ 	fraction of the monopolist's revenue and welfare. Both bounds are tight.		
% 	There always exists an equilibrium whose revenue and welfare are at least $\Omega(1/\sqrt{D})$	fraction of the monopolist's revenue and welfare.
\end{theorem}

\ignore{ % Noam did not like this
	
In the above example we saw that the ratio between the welfare of the best and {\em worst} non-trivial equilibria can be as large as $\Omega(D)$ even with only two demand levels.
\mbedit{In contrast, the theorem shows that %while the gap between the welfare of the best and worst non-trivial equilibria can be as large as $\Omega(D)$ even with only two demand levels,
the ratio between  the {\em optimal welfare} and the best equilibrium welfare is only a constant when the number of demand levels is a constant, and that it cannot grow faster than $O(\sqrt{D})$ when $n\leq \sqrt{D}$.}
%If the number of demand levels is not very large, the ratio can only grow as $n+\sqrt{D}$.
% re are many demand levels it can only be as large as $2^n$
% %(under a mild technical condition).

} % Comment

%\vspace{2mm}

We now turn to discuss how such markets converge to equilibria, and in case of multiple equilibria, which of them will be reached?
%In the case that there are multiple equilibria, which of them will be reached?
% A natural process
We consider best response dynamics in which
players start with some initial prices %$p_1^0, p_2^0$
and repeatedly best-reply to each other.
 We study the quality of equilibria reached by the dynamics, compared to the best equilibria.

Clearly, if the dynamics happen to start at an equilibrium, best replying will leave the prices there,
whether the equilibrium is good or bad.
But what happens in general:
which equilibrium will they ``converge'' to when starting from ``natural" starting points, if any,
and how long can that take?
Zero  prices (or other, very low prices) are probably the most natural starting point.
However, as can be seen by the example above, starting from zero prices may result in the worst equilibrium.\footnote{In this example, the best response to price of $0$ is price of $1$. Next, the first seller will move from price of $0$ to price of $1$ as well, resulting in the worst equilibrium.}
Another natural starting point is a situation where the two sellers form a cartel and decide to post prices that sum to the monopoly price.
Indeed, in our example above, if the two sellers equally split the monopoly price, this will be the best equilibrium.
However, we know that cartel solutions are typically unstable, and the participants will have incentives to deviate to other prices and thus start a price updating process. We prove a negative result in this context, showing that starting from any split of the monopoly price might result in bad equilibria.
We also check what would be the result of dynamics that start at random prices. Again, we prove a negative result showing situations where dynamics starting from random prices almost surely converge to bad equilibria.
Finally, we show that convergence might take a long time, even with only two demand levels. Following is a more formal description of these results about the best-response dynamics:
%
%, but equal split of the monopolist price is the best equilibrium.
%Will starting from a split of a price of a monopolist ensures convergence to a good equilibrium in general?
%If not, how about starting from random prices?
%Our results are negative, showing that starting from any split of the monopolist price might result in bad equilibria,
%and starting from random price might almost surely converge to bad equilibria.
%Moreover, convergence might take long time, even with only two demand levels.


\begin{theorem}
	The following statements hold:
	\begin{itemize}
		\item There are instances with $3$ demand levels for which a best-response dynamics starting from any split of a monopoly price reaches the worst equilibrium that
		is factor $\Omega(\sqrt{D})$ worse than the best equilibrium in terms of both revenue and welfare.
		
		\item For any $\epsilon>0$ and $D>2/\epsilon$  there are instances with $2$ demand levels for which a best-response dynamics starting from uniform random prices in $[0,v_{max}]^2$ reaches the worst equilibrium with probability $1-\epsilon$, while the best equilibrium has welfare and revenue that is factor $\epsilon\cdot D$ larger.
		
		\item For any $n\geq 2$ and $\epsilon>0$ there are instances with $n$ demand levels for which a best-response dynamics starting from uniform random prices in $[0,v_{max}]^2$ almost surely (with probability 1) reaches the worst equilibrium, while the best equilibrium has welfare and revenue that is factor $\Omega(2^n)$ larger.
			
		\item (Slow convergence.) For any $K>0$ there is an instance with only $2$ demand levels ($n=2$) and $D<2$ for which a best-response dynamics continues for at least $K$ steps before converging to an equilibrium.
		% \item Best reply dynamics may take $\Omega(D)$ time to converge to an equilibrium even for $n=2$.  {\bf MOSHE: this seems wrong, it should be a function of W}
		% \item For $n=2$, it always converges in $O(D)$ steps.  {\bf MOSHE: this seems wrong, it should be a function of W}
	\end{itemize}
\end{theorem}


Thus, best-reply dynamics may take a very long time to converge, and then typically end up at a very bad equilibrium.
% At least for simple markets there is a non-zero  probability of reaching a reasonable equilibrium.
While for very simple ($n=2$) markets we know that convergence will always occur, we do not know whether convergence
is assured for every market.

\vspace{0.1in}
\noindent
{\bf Open Problem:} {\em Do best reply dynamics always converge to an equilibrium or may
they loop infinitely? }  We do not know the answer even for $n=3$.
% \mbedit{If it does converge, how long might it take?}
\vspace{0.1in}



\noindent \textbf{More related work.}
While this paper studies price competition between sellers of perfect complements, the classic Bertrand competition \cite{Bert83} studied a similar situation between sellers of perfect substitutes. Bertrand competition leads to an efficient outcome with zero profits for the sellers.
\cite{BLN13} studied Bertrand-like competition over a network of sellers.
In another paper \cite{BBN16}, we studied a network of sellers
of perfect complements, where we showed how equilibrium properties depend on the graph structure, and we proved price-of-stability results for lines, cycles, trees etc.
%\cite{CR08} and \cite{CN09} studied the price of anarchy in a pricing game  between sellers that face combinatorial bidders.
%liad added July 2016
Chawla and Roughgarden \cite{CR08} studied the price of anarchy in two-sided markets with consumers interested in buying flows in a graph from multiple sellers, each selling limited bandwidth on a single edge. Their model is fundamentally different than ours (e.g., they consider combinatorial demand by buyers, and sellers with limited capacities) and their PoA results are with respect to unrestricted Nash Equilibrium, while we focus on non-trivial ones (in our model the analysis of PoA is straightforward for unrestricted NE). A similar model was also studied in \cite{CN09}.

\cite{ES92} extended the complements model of Cournot to accommodate multiple brands of compatible goods. \cite{EK06} studied pricing strategies for complementary software products. The paper by \cite{FK01} directly studied the Cournot/Ellet model, but when buyers approach the sellers (or the tollbooths on the canal) sequentially.

\cite{FKLMO13} discussed best-response dynamics in a Cournot Oligopoly model with linear demand functions, and proved that they converge to equilibria.
%liad Feb17 added the following reference
Another recent paper \cite{NP10} studied how no-regret strategies converge to Nash equilibria in Cournot and Bertrand oligopoly settings;
The main results in \cite{NP10} are positive, showing how such strategies lead to a positive-payoff outcomes in Bertrand competition, but they do not consider such a model with complement items.

Best-response dynamics is a natural description of how decentralized markets converge to equilibria, see, e.g., \cite{FPT04,NSVZ11}, or to approximate equilibria, e.g., \cite{AAEMS08,SV08}.
The inefficiency of equilibria in various settings has been extensively studied, see, \cite{KP99, RT02, Rou15, ADKTWR08,HHT14}.

\vspace{2mm}

We continue as follows: Section \ref{sec:model} defines our model and some basic equilibrium properties. In Section \ref{sec:existence} we prove the existence of non trivial equilibria. In Section \ref{sec:best-response} we study the results of best-response dynamics. Finally, Section \ref{sec:pos} compares the quality of the best equilibria to the optimal outcomes.


%\noindent{\textbf{More related research:}}
%In his celebrated work, Cournot presented two competition models. The first model is known as Cournot Oligopoly, and describes firms that compete via output levels of an homogenous good. Cournot also presented a model of firms selling perfect complements. \cite{Son68} showed that these two different models actually share the same formal structure.
%Cournot's described a surprising , saying that multiple sellers for complement goods lead to a worse result than the outcome of a monopoly.
%In the legal literature, this phenomenon was termed ``the tragedy of the anticommons" (\cite{BY00, Hel98, PSD05}).


% An important parameter in such dynamics is the starting prices. A natural starting prices is for both sellers to price at zero, but that might result in a very ''bad'' equilibrium. Indeed, in the example about, such dynamics will move from both sellers pricing at zero, to one pricing at $1$ and then to both pricing at $1$, and this would be an equilibrium with low revenue and welfare of $2$ (compared to $\Omega(d)$ in the best equilibria). But maybe other starting prices ensure convergence to beter equilibria? We prove the following negative results:


\ignore{
\mbcomment{OLD:}

%In this paper we study both the dynamics of this game and the revenue and welfare properties of its equilibria.
For the sake of quantification, we study a
{\em discretized version}\footnote{In the appendix we
go back to the continuous model and discuss how are discretized results
are translated to it.} of this game and derive our results in
terms of two
basic parameters of the demand function:  The first
parameter is the number of discrete steps in the
demand function, also viewed as the number of possible types of
buyers, which we denote by $n$ and called the number of demand levels.
The second  parameter is the ratio $D$ between
the total demand at price $0$ and
the demand at the highest price for which there is non-zero demand.
% We normalize the demand at the highest price to be $1$ and denote the total demand at price $0$ by $D$.
One may view $D$ as
the {\em size} of the market and view $n$ as the {\em complexity} of the market.

Let us start with an example: consider the  case where
there are two ($n=2$) types of buyers: a single buyer that is willing to pay ``a lot'',
$V>>1$, for the bundle of the two items, and many,
$D-1>>V>>1$, buyers that are willing to pay ``a little'', $1$, each, for the bundle.
A monopolist (that controls both sellers) would have sold
the bundle at the low price $1$, leading to revenue $D$ and optimal
social welfare of $D-1+V$.
There are two equilibria here: the ``bad'' one has high prices, $p_1+p_2=V$, (which certainly is
an equilibrium when, say, $p_1, p_2 \ge 1$) and achieves low revenue and
low social welfare compared to the monopolist.
The ``good'' one has low prices, $p_1+p_2=1$ (which is an equilibrium as long
as $p_1, p_2 \ge (V-1)/D$), and achieves
optimal social welfare as well as the monopolist revenue.

We start by studying the statics of this game: how bad can the best and worst
equilibria of this game be
compared to to each other, as well as compared to the optimal and
to the monopolist's outcome,
both for revenue and for social
welfare.  The example above already shows a gap as large as possible
(proportional to the market size, $D$) between the worst and best
equilibria
even for very simple markets ($n=2$), both for revenue and for social welfare.
This also settles the ``price of anarchy'' question as it matches also the trivial
upper bound of $D$.  The ``price of stability'' question turns out to be more
delicate, and the best equilibrium is competitive for simple markets, and not completely
bad, in general:

\begin{theorem}
*** FIX TO BE EXACTLY CORRECT ***

There always exists an equilibrium whose revenue and welfare are at least $1/\sqrt{D}$
fraction of the monopolist's revenue and welfare.  There always exists an equilibrium whose revenue and welfare are at least $2^{-n}$
fraction of the monopolist's revenue and welfare. Both bounds are tight.
\end{theorem}

We then continue with the focus of our paper of how to ensure that we reach a ``good''
equilibrium rather than a ``bad'' one.  We start be studying the best-reply dynamics: players
start with some initial prices $p_1^0, p_2^0$ and repeatedly best-reply to each other.
Clearly if they happen to start at an equilibrium, best replying will leave them there,
whether the equilibrium is good or bad.  But what happens in general: which equilibrium
will they ``converge'' to when starting from a ``natural starting point'', if any,
and how long can that take?  Our results are mostly negative here, with just a bit of
positive results:

\begin{theorem}
\begin{itemize}
	\item For any $K>0$ there is instance with only $2$ demand levels ($n=2$) and $D<2$ for which best response dynamics continues for at least $K$ steps before converging to an equilibrium.
% \item Best reply dynamics may take $\Omega(D)$ time to converge to an equilibrium even for $n=2$.  {\bf MOSHE: this seems wrong, it should be a function of W}
% \item For $n=2$, it always converges in $O(D)$ steps.  {\bf MOSHE: this seems wrong, it should be a function of W}
\item
For any $n \ge 2$, dynamics starting at either zero prices or at a $1-\epsilon$ measure
of starting prices (for any fixed $\epsilon>0$) may reach the worst equilibrium that
is $\Omega(D)$ worse than the best one in terms of revenue and welfare.
\item
For every
$n$, there is non-zero measure of starting points that reaches an equilibrium whose
revenue and welfare is at least $(2+\delta)^{-n}$ fraction of those of the best one
(for any fixed $\delta>0$).
\item
It is possible to have only a zero-measure of starting prices
that reaches an equilibrium with at least $(2-\delta)^{-n}$ fraction of the best
revenue or welfare.
\end{itemize}
\end{theorem}

Thus best reply dynamics may take a very long time to converge, and then may typically
end up at a very bad equilibrium.  At least for simple markets there is a non-zero
probability of reaching a reasonable equilibrium.  While for very simple ($n=2$) markets
we know that convergence will always occur, we do not know whether convergence
is assured for every market.

\vspace{0.1in}
\noindent
{\bf Open Problem:} Do best reply dynamics always converge to an equilibrium or may
they loop infinitely?  We do not know the answer even for $n=3$.
\vspace{0.1in}

As opposed to simple best-reply dynamics, we introduce a symmetrized version of
best-reply dynamics that is guaranteed to converge both quickly and to the best equilibrium.
In this version the players start with some prices, and at each stage one player best-replies
to the other one, but then we split the joint revenue equally between the two palyers.
I.e. if at a certain stage we have prices $p_1^i,p_2^i$, then player one updates his price
to $p'_1 = BR(p_2^i)$ and then we split the joint
revenue between them: $p_1^{i+1}=p_2^{i+1}=(p'_1+p_2^i)/2$.

\begin{theorem}
\begin{itemize}
\item
Symmetric best reply dynamics always reach an equilibrium after at most $O(n)$ steps
(and this bound is tight).
\item
Symmetric best-reply dynamics starting with zero prices, or any sufficiently low prices,
always end up at the best equilibrum.
\end{itemize}
\end{theorem}

We thus see that intervening in the best-reply dynamics in a way that every step
equalizes the revenue share between the two bidders improves efficiency, welfare, and
convergence time.


} % ignore
