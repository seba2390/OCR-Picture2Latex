
% We prove the following results:
Up to this point we considered the quality of equilibria reached by best response dynamics. In this section, we will show that not only that best response dynamics reach equilibria of poor quality, it may also take them arbitrary long time to converge.
%We next consider the time to convergence.
Moreover, the long convergence time is possible even with only $2$ demand levels and total demand that is close to $1$.

Specifically, we will show that as the difference between the demand of adjacent values becomes smaller, the convergence time can increase.
More formally, we let
$W=\frac{d_n}{min_{i=2}^n \{d_i-d_{i-1}\}}$ be the ratio between the maximal demand and the minimal change in demand.
Note that if $d_1=1$ and every $d_i$ is an integer, then 
$d_1=min_{i=2}^n \{d_i-d_{i-1}\}$ and thus
$W=D$; %liad feb17 elaborated a bit
if demands are not restricted to be integers, $W$ might be much larger than $D$ even in the case that $d_1=1$,
for example if $d_1=1$ and $d_2=1+\epsilon=D$
then $W=1/\epsilon$ is large while $D=1+\epsilon\approx 1$.
We show a simple setting with only two demand levels and with $D$ close to $1$ in which the dynamics takes time linear in $W$.
% , and in particular the time cannot be bounded as a function of $n$ and $D$.
%Of course, we can choose parameters where $W$ is arbitrarily high so the convergence time is essentially unbounded.


%%Specifically, we show that even for only two demand levels, and even for $D$ close to $1$, the dynamics can take time that is linear in $W=\frac{d_n}{min_{i=1}^n d_i-d_{i-1}}$, the ratio between the maximal demand and the minimal change in demand.
%Note that $W\geq D$ and additionally, if $d_1=1$ and every $d_i$ is an integer, then $W=D$, \textbf{Liad: do we need this sentence?} while if demands are not restricted to be integers, $W$ might be much larger than $D$, for example if $d_1=1$ and $d_2=1+\epsilon=D$  then $W=1/\epsilon$ while $D=1+\epsilon\approx 1$.
% {\bf MOSHE: I still need to merge the two paragraphs above. }

\begin{theorem}[Slow convergence]
	% For any instance with $n=2$ demand levels, best response dynamics starting from any prices will stop in an equilibrium after $O(d)$ steps, and
	For any $W$, %for some instances with $2$ demand levels ($n=2$),
	best response dynamics starting from zero prices may require each seller to update his price $W-1$ times
	to converge to an equilibrium.
	% {\bf seems like we prove W-1, why use $\Omega()$? }
	Moreover, this holds even with 2 demand levels ($n=2$) and with $D= \frac{W}{W-1}$ which is close to $1$ when $W$ is large.
	% {\bf Moshe: I have just realized the number of steps can be inversely proportional to the normalized difference between demands $(d_2-d_1)/d_1$, not the ratio of demands $D= d_2/d_1$. So it seems like there is a third parameter that is important!  }
\end{theorem}
\begin{proof}
We consider the following setting given some $\epsilon>0$ that is small enough:
$v_1=1$ and $d_1=1$, $v_2=1-\epsilon$ and $d_2=\frac{1}{1-2\epsilon}$.
In this case, $W=d_2/(d_2-d_1)=\frac{1}{2\epsilon}$. We will show that for this instance best response dynamics starting at $(0,0)$
 takes at least $W-1=\frac{1}{2\epsilon}-1$ steps to converge to an equilibrium. % {\bf Moshe: I think we were only be able to prove $W/2$. In any case this might not be an integer and we might need to round down - I took out 2 of W - I think we need to take out 2, need to make sure this is explained in the proof. note that for epsilon=1/3 we stop after one step, while W=3.}

Let $p_m,q_m$ denote the price offered by the two sellers after $m$ best-response steps for each seller ($p_m$ is the offer of the seller who plays first).
We will prove by induction that $p_m=1-m\epsilon$ and $q_m=m\epsilon$ whenever $m+1 < \frac{1}{2\epsilon}$.


We first handle the base case. With zero prices, the first seller can price at $v_1=1$ and get profit $1$, or price at $v_2= 1-\epsilon$ and get profit
$(1-\epsilon)\cdot \frac{1}{1-2\epsilon} >1$. Thus, $p_1=1-\epsilon$.
Now, the best response of the other seller is clearly $q_1=\epsilon$ as
pricing at total price of $1-\epsilon$ gains her $0$ profit.

We next move to the induction step. Assume that the claim is true for some $m$, i.e., $(p_m,q_m)=(1-m\epsilon, m\epsilon)$, and we prove it for $m+1$
(as long as $m+1<\frac{1}{2\epsilon}$).
If the second seller prices at $m\epsilon$, the first seller will maximize profit by pricing either at $1-(m+1)\epsilon$  or at  $1-m\epsilon$ (recall that
by Observation \ref{obs:BR-price-is-value} after a seller is best responding, the price will be equal to either $v_1$ or $v_2$).

The gain from the first price is $(1-(m+1)\epsilon)\cdot \frac{1}{1-2\epsilon}$ and the gain from the latter price is $1-m\epsilon$.
Simple algebra shows that $(1-(m+1)\epsilon)\cdot \frac{1}{1-2\epsilon} > 1-m\epsilon$ iff $m<\frac{1}{2\epsilon}$.

	Now, assume that the first seller prices at  $1-(m+1)\epsilon$,
the second seller maximizes profit by pricing either at $(m+1)\epsilon$  or at $1-\epsilon-(1-(m+1)\epsilon)=m\epsilon$.
The second seller chooses a price of $(m+1)\epsilon$ if
$(m+1)\epsilon > \frac{1}{1-2\epsilon}m\epsilon$.
Simple algebra shows that this holds iff $m+1<\frac{1}{2\epsilon}$.
This concludes the induction step and completes the proof.
%	We consider demand with two values, $1$ and $v=1-\epsilon<1$ (and $v>0$). The demand at value of $1$ is normalized to $1$, and the demand at $v$ is $d>1$.
%	
%	Fix any $\epsilon<\frac{1}{2k+1}$ and set $v=1-\epsilon$. Pick any $d$ such that $\frac{k+1}{k}>d>\frac{1-k\epsilon}{1-(k+1)\epsilon}$.
%	Observe that for $\epsilon<\frac{1}{2k+1}$ such a $d$ exists since this condition on $\epsilon$ is a condition that is equivalent to  $\frac{k+1}{k}>\frac{1-k\epsilon}{1-(k+1)\epsilon}$.
%	
%	To prove the claim we show by induction that starting with a price of $0$ for the second seller, for any $m\leq k$,
%	the first seller will price at price $1-m\epsilon$ at the $m$-$th$ update, while the second seller will price at $m\epsilon$ at his $m$-$th$ update.
%	
%	Indeed, for the base case observe that the best response of the first seller to the price of $0$ is $v=1-\epsilon$ since $v\cdot d>1$ as $d> \frac{1-k\epsilon}{1-(k+1)\epsilon}= 1+ \frac{\epsilon}{1-(k+1)\epsilon}> 1+ \frac{\epsilon}{1-\epsilon}= \frac{1}{1-\epsilon} = \frac{1}{v} $. Now the second seller's best response to $v$ is clearly $\epsilon=1-v$ as it increases his utility from $0$ to $\epsilon$.
%	
%	We next move to the induction step. Assume that the claim is true for any $m<k$, we prove it for $m+1$.
%	Assuming that the second seller prices at $m\epsilon$ we would like to show that the first seller prefers $1-(m+1)\epsilon$  over $1-m\epsilon$. Indeed, his revenue from price $1-m\epsilon$ is $1-m\epsilon$  while his revenue from $1-(m+1)\epsilon$ is
%	$d(1-(m+1)\epsilon)$  and he prefers $1-(m+1)\epsilon$ since $1-m\epsilon< d(1-(m+1)\epsilon)$ holds as
%	$d> \frac{1-k\epsilon}{1-(k+1)\epsilon}= 1+ \frac{\epsilon}{1-(k+1)\epsilon}>  1+ \frac{\epsilon}{1-(m+1)\epsilon} = \frac{1-m\epsilon}{1-(m+1)\epsilon}$.
%	
%	Now, assume that the first seller prices at  $1-(m+1)\epsilon$, we show that the second seller prefers $(m+1)\epsilon$ over $m\epsilon$.
%	This holds if $(m+1)\epsilon > d m\epsilon$ which holds as $d<\frac{k+1}{k}\leq \frac{m+1}{m}$ since $k>m$.
%	This concludes the induction step and completes the proof.
\end{proof}


\ignore{OLD:
The following theorem shows that except of a measure zero (finite) set of starting prices for the best response dynamics, every dynamics will end up in a very low welfare equilibrium, although equilibrium with high welfare exists. The welfare gap between the good and bad equilibria increases exponentially in the number of demand levels $n$.

\begin{theorem}[Almost sure convergence to bad equilibria, large $n$]
	For any % total demand $D>1$,
	demand levels $n\geq 2$  and $\epsilon>0$ that is small enough,
	there exists instance % with total demand $D$
	that has the following properties:
	\begin{itemize}
		\item it has equilibria with welfare at least $\frac{(2-\epsilon)^{n}-1}{1-\epsilon}$ and monopolist revenue of at least $(2-\epsilon)^{n-1}$.
		\item if the best response of the first seller in the dynamics does not result in an equilibrium, then
		the best response dynamics  will end up in an equilibrium with welfare and revenue of $1$.
		\item the set of prices $\{(p,q) | (p,q)\in NE \ \&\ SW(p,q)\neq 1\}$ is finite.
		\item the set of prices $\{(p,q) | (p,q)\in NE \ \&\ SW(p,q)= 1\}$ is infinite and uncountable.
	\end{itemize}
\end{theorem}
\begin{proof}
% Fix $\alpha<2$ that is close to $2$.
Let $\alpha=2-\epsilon$.
For $i\in [n]$ let $v_i=\epsilon^{i-1} $ and $d_i=\alpha^{i-1}/v_i$.
% Let $v_1 = \frac{\alpha}{D^{1/n}}$. For $i\in [n]$ let $v_i=(v_1)^i$ and let $d_i = \frac{\alpha^n}{2v_1^i-v_1^n}$.
We argue that for any $i\in [n]$ prices $(v_i/2, v_i/2)$ form an equilibrium, and that best response dynamics starting by best responding to any price $p\notin \{v_i/2 \ for \ i\in [n]\}$, ends in an equilibrium with $p+q=v_1$, and thus the set of starting prices for the dynamics that result in equilibrium welfare higher than $1$ is finite.
% Finally we observe that

FINISH THE PROOF
\end{proof}

We next show that for any total demand $D$, even with only $2$ demand levels, the gap in welfare between the best and worst equilibria can be as large as $\sqrt{D}$ and moreover, with non-malicious prices being in $[0,1]$, if the starting price for the dynamics is sampled uniformly from these prices, the dynamics will converge to the bad equilibria with probability at least $1-1/\sqrt{D}$.

\begin{theorem}[High probability of convergence to bad equilibria, $n=2$]
	\label{thm:conv-bad-NE-n is 2}
	For any total demand $D>1$ that is large enough, there exists instance with $n=2$ demand levels
	that has the following properties:
	\begin{itemize}
		\item Non-malicious prices are in $[0,1]$.
		\item it has (good) equilibria with welfare and monopolist revenue of at least $\sqrt{D}$.
		\item it has (bad) equilibria with welfare and monopolist revenue of $1$.
		\item if the price from which the best response dynamics starts (the first best response is to that price) is sampled uniformly in $[0,1]$, then  the best response dynamics ends up in an equilibrium with welfare and revenue of $1$ with probability at least $1-1/\sqrt{D}$.
	\end{itemize}
\end{theorem}
\begin{proof}
	 % Let $\epsilon = 1/\sqrt{D}$ and
	 Consider the input with $n=2$ demand levels satisfying $v_1=1> v_2=1/\sqrt{D}$ and $d_1=1<d_2=D$. % Note that $D=1/\epsilon^2$.
	 A pair of prices $(p,q)$ with $p+q=v_2$ results with welfare and monopolist revenue of $\sqrt{D}$, and for large enough $D$, the pair $(v_2/2, v_2/2) $ is indeed an equilibrium. On the other hand, $(1/2,1/2)$ is also an equilibrium, and its welfare and revenue are only $1$. Finally, observe that unless the price that the dynamics starts with is at most $v_2=1/\sqrt{D}$, the first best response result in an equilibrium with total price of $1$, and welfare of $1$, immediately after the first best response.
\end{proof}
}

We observe that with two demand levels, convergence to equilibrium is guaranteed, and the above linear bound is actually tight.
Proof appears in Appendix \ref{app:BR-stops-at-time-W}.
\begin{proposition}
	\label{obs:bsd-2-stops}
	For any instance with $2$ demand levels ($n=2$), best response dynamics starting from any price profile will stop in an
	equilibrium after each seller updates his price at most $W$ times.
\end{proposition}
%\begin{proof}
%	We prove the result for every market with two demand levels.
%	Let $v_1=1$ and $d_1=1$ (we normalize the two values to 1 w.l.o.g.), and let $v_2=1-\epsilon$ for $\epsilon>0$ and $d_2=D$.
%	For these parameters, $W=\frac{D}{D-1}$.
%	
%	We will first show that as long as the best response process proceeds,
%	there is an increase of exactly $\epsilon$ between any two consecutive prices one seller sets, and a decrease of exactly $\epsilon$ between prices set by the other seller.
%	
%	Recall that due to Observation \ref{obs:BR-price-is-value}, after a seller is best responding, the price will be equal to either $1$ or $1-\epsilon$. % When a seller best responds,
%
%	
%	Consider first a set of prices $(p,\mathbf{q})$ where $q+p=1$ (the price that is marked in bold indicates the price of the player whose turn is to best respond, in this case, the second seller).
%	The seller sets his price to a price in $BR(p)$,
%	and for the dynamics to continue it must hold that the total price is now equal to $v_2$:
%	$BR(p)$ is unique and not equal to $q$ and it satisfies $BR(p)+p=1-\epsilon$. Since $p+q=1$, we get that
%	$BR(p)=q-\epsilon$ and the new pair of offers is $(\mathbf{p},q-\epsilon)$.
%	
%	Consider now some set of prices $(\mathbf{p'},q')$ where $q'+p'=1-\epsilon$.
%	The seller sets his price to a price in $BR(q')$,
%	and for the dynamics to continue it must hold that $BR(q')$ is unique and not equal to $p'$ and it satisfies
%	$BR(q')+q'=1$. Since $q'+p'=1-\epsilon$, we get that the new pair of prices is $(p'+\epsilon,\mathbf{q'})$.
%	
%	We conclude that every best response dynamics have the following form.
%	After the first step, the sum of prices will either $v_1$ or $v_2$.
%	As long as the process proceeds, we will have the following sequence of prices when the initial total price is $v_1$ (otherwise, consider the sequence starting from the second price vector):
%	$(p_0,\pmb{q_0)}$,
%	$(\pmb{p_0},q_0-\epsilon)$,
%	$(p_0+\epsilon,\pmb{q_0-\epsilon})$,
%	$(\pmb{p_0+\epsilon},q_0-2\epsilon)$,$(p_0+2\epsilon,\pmb{q_0-2\epsilon})$,
%	$...$,$(p_0+m\epsilon,\pmb{q_0-m\epsilon})$,
%	$(\pmb{p_0+m\epsilon},q_0-(m+1)\epsilon)$, and so on.
%	
%	As prices are bounded in $[0,1]$, the number of updates by one seller clearly cannot be more than $1/\epsilon$.
%	
%	We are left to bound the number of iterations as a function of $D$.
%	Indeed, consider the price vector $(\pmb{p_0+m\epsilon},q_0-(m+1)\epsilon)$.
%	For the dynamics to continue, the currently responding player must prefer
%	increasing his price and selling to the lower demand at price $v_1$:
%	\begin{align}
%	1\cdot(p_0+(m+1)\epsilon) > D\cdot(p_0+m\epsilon)
%	\end{align}
%	It follows that $m<\frac{1}{D-1}-\frac{p_0}{\epsilon}<\frac{D}{D-1}=W$. Therefore, in every best response dynamics each player will change its price at most $W$ times.
%	%We claim that for one seller there is an increase of exactly $\epsilon$ between any two consecutive prices he sets, and for the other, there is a decrease of exactly $\epsilon$ between prices.
%	%$As prices are bounded in $[0,1]$, the number of updates by one seller cannot be more than $1/\epsilon$ as claimed.
%	%
%	%Due to Observation \ref{obs:BR-price-is-value}, after a seller is best responding, the price will be equal to either $1$ or $1-\epsilon$.
%	%First consider a seller that by best responding induces a total price of $1$; That is, the seller sets a price of $y$ such that $y=BR(1-y)$ where $1-y$ was the price set by the other seller.
%	%	For the dynamics to continue, %such that the seller will get to best responses again,
%	%it must hold that $BR(y) = v_2-y$ and $BR(v_2-y)=1-(v_2-y) = y+\epsilon$, thus the price increased by exactly $\epsilon$ as claimed.
%	%	
%	%	Now consider the other seller. For the dynamics to continue one more step, it must hold that $BR(y+\epsilon) = v_2-(y+\epsilon) = v_2-y-\epsilon$, so his price deceases by $\epsilon$ as claimed.
%\end{proof}
%
