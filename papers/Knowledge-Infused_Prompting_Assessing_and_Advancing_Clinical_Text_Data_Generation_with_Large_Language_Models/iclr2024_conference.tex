
\documentclass{article} % For LaTeX2e
\usepackage{pifont}

\usepackage{iclr2024_conference,times}

% Optional math commands from https://github.com/goodfeli/dlbook_notation.
%%%%% NEW MATH DEFINITIONS %%%%%

\usepackage{amsmath,amsfonts,bm}
\usepackage{xifthen}

% Highlight a newly defined term
\newcommand{\newterm}[1]{{\bf #1}}

\def\eps{{\epsilon}}


% Utility for ticks 
\newcommand{\cmark}{\ding{51}}%
\newcommand{\xmark}{\ding{55}}%

% Theorem styles 
\theoremstyle{definition}
\newtheorem{theorem}{Theorem}[section]
\newtheorem{definition}{Definition}[section]
% \newtheorem{remark}{Remark}[theorem] %numbered remark
\newtheorem*{remark}{Remark} %unnumbered remark
\newtheorem{lemma}{Lemma}[section]
\newtheorem{prop}{Proposition}[section]
\newtheorem{corollary}{Corollary}[theorem]
\newtheorem{conjecture}{Conjecture}[section]
\newtheorem{assumption}{Assumption}[section]

\newtheorem{manualtheoreminner}{Theorem}
\newenvironment{manualtheorem}[1]{%
  \renewcommand\themanualtheoreminner{#1}%
  \manualtheoreminner
}{\endmanualtheoreminner}


% Math helper - standard function
\DeclareMathOperator*{\argmax}{arg\,max}
\DeclareMathOperator*{\argmin}{arg\,min}
\DeclareMathOperator{\support}{support}
\DeclareMathOperator{\MAX}{MAX}
\DeclareMathOperator{\term}{\texttt{term}}
\DeclareMathOperator*{\logsumexp}{log-sum-exp}
\DeclareMathOperator*{\TV}{TV}
\newcommand{\norm}[1]{\left\lVert#1\right\rVert}
\DeclarePairedDelimiter\set\{\}
\DeclarePairedDelimiter\abs{\lvert}{\rvert}%
\newcommand*{\mytop}{\mathrel{\scalebox{0.5}{$\top$}}}
\newcommand*{\mybot}{\mathrel{\scalebox{0.5}{$\bot$}}}
\newcommand*{\mydiese}{\mathrel{\scalebox{0.5}{$\#$}}}
\newcommand*{\myplus}{\mathrel{\scalebox{0.5}{$+$}}}
\newcommand*{\myminus}{\mathrel{\scalebox{0.5}{$-$}}}
\newcommand*{\bmg}{\bm{\gamma}}
\newcommand*{\bml}{\bm{\lambda}}

% MDP notation
\renewcommand{\S}{\mathcal{S}}
\newcommand{\X}{\mathcal{X}}
\newcommand{\A}{\mathcal{A}}
\newcommand{\T}{\mathcal{T}}
\newcommand{\M}{\mathcal{M}}
\newcommand{\B}{\mathcal{B}}
\newcommand{\Bset}{\mathfrak{B}}
\newcommand{\Dist}{\mathscr{P}}
\newcommand{\D}{\mathcal{D}}
\newcommand{\Real}{\mathbb{R}}
\renewcommand{\P}{\mathcal{P}}
\newcommand{\E}{\mathop{\mathbb{E}}}
\renewcommand{\H}{\mathcal{H}}
% \newcommand{\R}{\mathcal{R}}
% \newcommand{\C}{\mathcal{C}}

% Extended MDP notation
\newcommand{\Pstar}{p^{\star}}
\newcommand{\Rstar}{\bm{r}^{\star}}
\newcommand{\Cstar}{C^{\star}}
% \newcommand{\rmax}{\textsc{Rmax}}
\newcommand{\rmax}{r_{\mytop}}
\newcommand{\cmax}{\textsc{Cmax}}

\newcommand{\mstar}{m^{\star}}
\newcommand{\mhat}{\hat{m}}
\newcommand{\mopt}{m^{\star}}

\newcommand{\Phat}{\hat{p}}
\newcommand{\Rhat}{\hat{\bm{r}}}
\newcommand{\Chat}{\hat{C}}

% Math helper - custom function
\newcommand{\expwrtpi}[1]{\E_{\pi} [\sum_{t=0}^{\infty} \gamma^t #1(s_t, a_t)]}
\newcommand{\expangle}[1]{\langle #1  \rangle}

% helper function for return and constraints

% for value function, takes arguments:
% #1: policy 
% #2: the function of interest, R or C_i
% #3 (optional): the MDP for which this is estimated
\newcommand{\V}[3]{ %
    \ifthenelse{\isempty{#3}}%
    {V^{#1}(#2)}% #3 is empty 
    {V^{#1}_{#3}(#2)}%
}

\newcommand{\Q}[3]{
    \ifthenelse{\isempty{#3}}
    {Q^{#1}(#2)}% #3 is empty 
    {Q^{#1}_{#3}(#2)}%
}


\newcommand{\Adv}[3]{
    \ifthenelse{\isempty{#3}}
    {A^{#1}(#2)}% #3 is empty 
    {A^{#1}_{#3}(#2)}%
}

% careful diff notation
% 1: pi
% 2: R/C
% 3: M
\newcommand{\J}[3]{
    \ifthenelse{\isempty{#3}}
    {\mathcal{J}^{#1}_{#2}}% #3 is empty -> eg V^{\pi}(x ; R)
    {\mathcal{J}^{#1}_{#3,#2}}% -? eg V^{\pi}_{M}(x ; C)
    % {J_{#2}(#1)}% #3 is empty 
    % {J_{#2}(#1, #3)} %
}



\newcommand{\MRkern}{%
  \mkern-6.5mu
  \mathchoice{}{}{\mkern0.2mu}{\mkern0.5mu}%
}

% for value function, takes arguments:
% #1: policy 
% #2: the function of interest, R or C_i
% #3 (optional): the MDP for which this is estimated
% #4: variables to be given input (x) or (x,a)
\newcommand{\val}[4]{ %
    \ifthenelse{\isempty{#3}}%
    {v^{#1}_{#2}(#4)}% #3 is empty -> eg V^{\pi}(x ; R)
    {v^{#1}_{#3,#2}(#4)}% -? eg V^{\pi}_{M}(x ; C)
    % {V^{#1}_{#3}(#4 ;#2)}% -? eg V^{\pi}_{M}(x ; C)
    % {V_{#2}(#4 ; #1)}% #3 is empty -> eg V_R(x ; \pi)
    % {V_{#2}(#4 ;#1, #3)}% -? eg V_C(x ; \pi, M)
    % {#2 \MRkern V^{#1}_{#3}(#4)}% -? eg V^{\pi}_{M}(x ; C) # combines the letter V and R together
}

\newcommand{\qval}[4]{
    \ifthenelse{\isempty{#3}}
    {q^{#1}_{#2}(#4)}% #3 is empty -> eg V^{\pi}(x ; R)
    {q^{#1}_{#3,#2}(#4)}% -? eg V^{\pi}_{M}(x ; C)
    % {Q^{#1}(#4 ; #2)}% #3 is empty -> eg Q^{\pi}(x,a ; R)
    % {Q^{#1}_{#3}(#4 ;#2)}% -? eg Q^{\pi}_{M}(x,a ; C)
    % {Q_{#2}(#4 ; #1)}% #3 is empty -> eg Q_R(x,a ; \pi)
    % {Q_{#2}(#4 ;#1, #3)}% -? eg Q_C(x,a ; \pi, M)
}
\DeclareMathOperator*{\advantage}{Adv}

\newcommand{\adv}[4]{
    \ifthenelse{\isempty{#3}}
    {\advantage^{#1}_{#2}(#4)}% #3 is empty -> eg V^{\pi}(x ; R)
    {\advantage^{#1}_{#3,#2}(#4)}% -? eg V^{\pi}_{M}(x ; C)
    % {A^{#1}(#4 ; #2)}% #3 is empty -> eg Q^{\pi}(x,a ; R)
    % {A^{#1}_{#3}(#4 ;#2)}% -? eg Q^{\pi}_{M}(x,a ; C)
    % {A_{#2}(#4 ; #1)}% #3 is empty -> eg A_R(x,a ; \pi)
    % {A_{#2}(#4 ;#1, #3)}% -? eg A_C(x,a ; \pi, M)
}




\newcommand{\ci}{C}

\newcommand{\pib}{\pi_{b}}
\newcommand{\piopt}{\pi^{*}}
\newcommand{\pie}{\pi_{t}}

\newcommand{\lR}{\lambda_{R}}
\newcommand{\lC}{\lambda_{C}}
\newcommand{\ephi}{e_{\phi}}

\newcommand{\pr}{\text{Pr}}
\newcommand{\IS}{\text{IS}}
\newcommand{\CI}{\text{CI}}


% SPIBB symbols 
\newcommand{\EpsPib}{(\pi_b, e, \epsilon)}
\iclrfinalcopy
\usepackage[colorlinks,citecolor=teal]{hyperref}
\usepackage{url}
\usepackage{multirow}
\usepackage{booktabs}
\usepackage{floatrow}
\usepackage{graphicx} % For adding images
\usepackage{smile}
\usepackage{enumitem}
\usepackage{colortbl}

% \title{Clinically-informed Prompting Makes Large Language Models Better Few-shot Clinical Text Generators}
% \title{Assessing and Advancing LLM on Clinical Text Generation via Knowledge-infused Prompting}
\title{Knowledge-Infused Prompting: Assessing and Advancing Clinical Text Data Generation with Large Language Models}

% Authors must not appear in the submitted version. They should be hidden
% as long as the \iclrfinalcopy macro remains commented out below.
% Non-anonymous submissions will be rejected without review.

\author{Ran Xu$^{\heartsuit}$, Hejie Cui$^{\heartsuit}$, Yue Yu{$^\spadesuit$}, Xuan Kan$^{\heartsuit}$, Wenqi Shi{$^\spadesuit$}, Yuchen Zhuang{$^\spadesuit$}, \\ \bf Wei Jin$^{\heartsuit}$, Joyce C. Ho$^{\heartsuit}$, Carl Yang$^{\heartsuit}$ \\
${\heartsuit}$ Emory University \quad {$^\spadesuit$} Georgia Institute of Technology \\
Atlanta, GA, USA \\
\texttt{\{ran.xu,hejie.cui,xuan.kan,wei.jin,joyce.c.ho,j.carlyang\}@emory.edu} \\
\texttt{\{yueyu,wshi83,yczhuang\}@gatech.edu}
}

% The \author macro works with any number of authors. There are two commands
% used to separate the names and addresses of multiple authors: \And and \AND.
%
% Using \And between authors leaves it to \LaTeX{} to determine where to break
% the lines. Using \AND forces a linebreak at that point. So, if \LaTeX{}
% puts 3 of 4 authors names on the first line, and the last on the second
% line, try using \AND instead of \And before the third author name.

\newcommand{\fix}{\marginpar{FIX}}
\newcommand{\new}{\marginpar{NEW}}
\usepackage{xspace}
\usepackage{xcolor,colortbl}
\usepackage{subfigure}
\usepackage{caption}
\usepackage[normalem]{ulem} % Load the ulem package
\usepackage{listings}


\newcommand{\wei}[1]{\textcolor{magenta}{@Wei:~#1@}}
\newcommand{\joyce}[1]{\textcolor{blue}{@J:~#1@}}
\newcommand{\ours}{\textsc{ClinGen}\xspace}
\newcommand{\ran}[1]{{\color{cyan}[Ran: #1]}}

\lstdefinestyle{mystyle}{
  basicstyle=\ttfamily,
  frame=single,
  breaklines=true,
  breakindent=0pt,
  backgroundcolor=\color{gray!10}, % Change the background color
}


%\iclrfinalcopy % Uncomment for camera-ready version, but NOT for submission.
\begin{document}


\maketitle

\begin{abstract}
% The abstract paragraph should be indented 1/2~inch (3~picas) on both left and
% right-hand margins. Use 10~point type, with a vertical spacing of 11~points.
% The word \textsc{Abstract} must be centered, in small caps, and in point size 12. Two
% line spaces precede the abstract. The abstract must be limited to one
% paragraph.
Clinical natural language processing requires methods that can address domain-specific challenges, such as complex medical terminology and clinical contexts. 
Recently, large language models (LLMs) have shown promise in this domain. Yet, their direct deployment can lead to privacy issues and are constrained by resources. 
To address this challenge, we delve into synthetic clinical text generation using LLMs for clinical NLP tasks. We propose an innovative, resource-efficient approach, {\ours}, which infuses knowledge into the process. Our model involves clinical knowledge extraction and context-informed LLM prompting. Both clinical topics and writing styles are drawn from external domain-specific knowledge graphs and LLMs to guide data generation. 
Our extensive empirical study across 7 clinical NLP tasks and 16 datasets reveals that {\ours} consistently enhances performance across various tasks, effectively aligning the distribution of real datasets and significantly enriching the diversity of generated training instances. We will publish our code and all the generated data in \url{https://github.com/ritaranx/ClinGen}.
\end{abstract}





\section{Introduction}
\label{sec1:introduction}
\vspace{-5pt}

Language Models (LMs) have opened up a new era in Natural Language Processing (NLP) by leveraging extensive datasets and billions of parameters\,\citep{llm_survey, gpt4_report, scaling_law}. These LMs excel at In-Context Learning (ICL), generating responses based on a few demonstrations without needing further parameter adjustments\,\citep{emergent_abilities_llms, gpt3, icl_survey}. The rise of instruction-tuning has further enhanced this capability, optimizing LMs to align their outputs closely with human-specified instructions\,\citep{flan, t0, gpt3, lms_are_unsupervised_multitask_learners}. This approach has demonstrated a significant improvement in zero-shot scenarios, underscoring its importance for tackling diverse tasks.

However, instruction-tuned models often struggle with unfamiliar tasks due to limitations in their training datasets, whether the datasets are human-annotated\,\citep{ni_dataset, sni_dataset} or model-generated\,\citep{self_instruct, unnatural_ni_dataset}. Refining these datasets is essential but requires substantial effort and computational resources, highlighting the need for more efficient approaches\,\citep{flan_t5, lima}. Moreover, the depth of a model's understanding of and how they respond to instructions remains an area of active research. While recent studies have provided some insights\,\citep{do_really_follows_instructions, did_you_read_instructions}, many questions remain unanswered. Techniques such as prompt-engineering\,\citep{prompt_analysis} and utilizing diversified outputs\,\citep{self_consistency} aim to increase the quality of outputs. However, the effectiveness of these techniques often depends on the fortuitous alignment of prompts or initial conditions, making them labor-intensive since the tuning process must be tailored.

In pursuit of refining the behavior of LMs, some researchers have begun to explore the \textit{anchoring effect}\,\citep{kahneman1982judgment}—a well-known cognitive bias where initial information exerts disproportionate influence on subsequent judgments. Intriguingly, this cognitive principle has been demonstrated to extend to LMs. For example, through effective prompting, the outputs generated by LMs can be steered towards a specific intent\,\citep{jones2022capturing}. Similarly, emphasizing the first few sentences of a long context enhances the model's overall comprehension of the content\,\citep{coherence_boosting}. Given these observations on LMs—parallels that mirror human tendencies—and the influential role of initial prompts, we hypothesize that the strategic application of the anchoring effect could substantially improve LMs' fidelity to instructions.

In this work, we propose \textit{Instructive Decoding}\,(ID)\,(\autoref{fig:main}), a novel method that enhances the attention of instruction-tuned LMs towards provided instructions during the generation phase without any parameter updates. The core idea of ID is deploying \textit{noisy} variants of instructions, crafted to induce a clear \textit{anchoring effect} within the LMs, to adjust the output anchored by the original instruction. More precisely, this effect aims to steer the models toward particular, potentially sub-optimal predictions. Our range of variants spans from simple strategies such as instruction truncation and more aggressive alterations, the most extreme of which is the \textit{opposite} instruction. By intentionally introducing such deviations, ID capitalizes on the resulting disparities. Within a contrastive framework, next-token prediction logits that are influenced by the noisy instructions are systematically compared to those derived from the original instruction. This process refines the model's responses to align more closely with the intended instruction.

\begin{figure}[t!]
\centering
\vspace{-10pt}
\includegraphics[width=\textwidth]{materials/figures/fig1_main_fix.pdf}
\vspace{-10pt}
\caption{Overview of Instructive Decoding\,(ID). The example in this figure is from \texttt{task442\_com\_qa\_paraphrase\_question\_generation} in \textsc{SupNatInst}\,\citep{sni_dataset}. The original response not only fails to meet the task requirements (Question Rewriting) but also contains incorrect information\protect\footnotemark. In contrast, ID generates a more relevant response by refining its next-token predictions based on the noisy instruction (here, opposite prompting is used for ID).}
\vspace{-12pt}
\label{fig:main}
\end{figure}
%
\footnotetext{According to the \href{https://population.un.org/wpp/}{2022 U.N. Revision}, the population of USA is approximately 338.3 million as of 2022.}

\begin{wrapfigure}{r}{0.54\textwidth}
\vspace{-15pt}
    \includegraphics[width=1.0\linewidth]{materials/figures/main_result_fig.pdf}
    \vspace{-15pt}
    \caption{Zero-shot Rouge-L comparison on the \textsc{SupNatInst} heldout dataset\,\citep{sni_dataset}. Models not instruction-tuned on \textsc{SupNatInst} are in \textcolor{blue}{blue} dotted boxes, while those instruction-tuned are in \textcolor{green!50!black}{green}.} 
    \label{fig:result_overall}
\vspace{-10pt}
\end{wrapfigure}

Experiments on unseen task generalization with \textsc{SupNatInst}\,\citep{sni_dataset} and \textsc{UnNatInst}\,\citep{unnatural_ni_dataset} held-out datasets show that instruction-tuned models enhanced by ID consistently outperform baseline models across various setups. Intriguingly, T\textit{k}-XL combined with our method outperforms its larger version, T\textit{k}-XXL, with standard inference\,(\autoref{fig:result_overall}). Models not previously trained on the \textsc{SupNatInst} dataset, including Alpaca (7B) and T0 (3B), also show marked enhancements in performance. Additionally, the overall Rouge-L score of the GPT3 (175B) is strikingly competitive, closely mirroring the performance of OpenSNI (7B) when augmented with our method. 
We further observe that ID's generation exhibits increased both adherence to the instruction and an improvement in semantic quality. To provide a comprehensive understanding, we investigate the anchoring effect of noisy instructions. Our findings suggest that as the model's comprehension of the noisy instruction intensifies, the anchoring effect becomes more potent, making ID more effective. Our main contributions are as follows:
%
\begin{itemize}
    \item We introduce \textit{Instructive Decoding}\,(ID), a novel method to enhance the instruction following capabilities in instruction-tuned LMs. By using distorted versions of the original instruction, ID directs the model to bring its attention to the instruction during generation \textbf{(\Autoref{sec2:method})}.
    %
    \item We show that steering the noisy instruction towards more degrading predictions leads to improved decoding performance. Remarkably, the \textit{opposite} variant, which is designed for the most significant deviation from the original instruction yet plausible, consistently shows notable performance gains across various models and tasks (\textbf{\Autoref{sec3:experiment}}).
    %
    \item We provide a comprehensive analysis of the behavior of ID, demonstrating its efficacy from various perspectives. The generated responses via ID also improve in terms of label adherence and coherence, and contribute to mitigate the typical imbalances observed in the standard decoding process. (\textbf{\Autoref{sec4:discuss}})
\end{itemize}

\section{Related Work}
Generating additional training data enables a more precise analysis of medical text, and has gained more attention in the past years. 
Earlier research has employed data augmentation techniques to generate similar samples to existing instances with word substitution~\citep{kang2021umls,checklist}, back translation~\citep{uda}, interpolation between samples~\citep{chen2020mixtext,kan2023r}, pretrained transformers~\citep{kumar2020data,zhou2021flipda,melm} for enhancing model generalization. But they often yield rigid transformations and the quality of the augmented text cannot be always guaranteed. 
Another line of work focuses on leveraging external knowledge to create weak labels~\citep{ratner2017snorkel,fries2017swellshark,wang2019clinical,yu-etal-2021-fine,dunnmon2020cross,gao2022classifying,wander}\footnote{Some works also name it as `distant supervision'~\citep{mintz2009distant,min2013distant,liang2020bond}.}. 
These methods typically require domain expertise and additional task-specific corpora, which can be resource-intensive to obtain for low-resource clinical tasks~\citep{zhang2021wrench}. 
% .

The emergence of LLMs has presented new possibilities, and some studies attempt to use LLM to generate training data~\citep{meng2022generating,meng2023tuning,ye2022zerogen,yu2023large,chung2023increasing}, often with few demonstrations~\citep{gpt3mix}. However, these methods often use generic and simple prompts that may not fully capture domain-specific knowledge, thus potentially limiting the quality of the generated data. 
\citet{liu2022wanli,chung2023increasing} employ interactive learning to generate additional instances to refine the existing dataset, at the cost of additional human efforts.
One recent study \citep{tang2023does} explores synthetic data generation for clinical NLP. Nevertheless, their proposed approach relies on a \emph{much larger training set} to generate candidate entities, which {disregards the practical low-resource setting}~\citep{perez2021true}. Furthermore, their study is limited to {a narrow range of tasks (2 tasks and 4 datasets only)}, lacking breadth in terms of exploring a diverse set of clinical NLP applications.

On the other hand, several works aimed at optimizing prompts using LLMs~\citep{mishra-etal-2022-reframing,zhou2023large,yang2023large} or knowledge graphs~\citep{chen2022knowprompt,hu-etal-2022-knowledgeable,liu-etal-2022-generated,kan2021zero}, yet they mainly focus on refining prompts to obtain the answer for the given input, and the prompt template often remains unchanged. 
Instead, we focus on the different task of generating training instances. By composing different topics and styles together, we are able to generate diverse templates for prompting LLMs to improve the quality of the synthetic data.
% \vspace{-3ex}
\section{Preliminary Study}
\vspace{-1ex}
\label{sec:preliminary}
This section first presents the foundational setup of synthetic data generation. 
Then, we provide an in-depth investigation into the pitfalls of existing synthetic data generation methods. 


\vspace{-1ex}
\subsection{Problem Setup}
In this paper, we study the synthetic data generation problem in the few-shot setting.
The input consists of a training set $\cD_{train}=\{(x_i,y_i)\}_{i=1}^K$, where $(x_i, y_i)$ represents the text and its label for the $i$-th example. $K$ denotes the total number of training samples, which is intentionally kept at a very small value (5-shot per label). The primary objective is to harness the capabilities of an LLM $\cM$ to generate a synthetic dataset, denoted as $\cD_{\text{syn}}=\{(\tilde{x_i},\tilde{y_i})\}_{i=1}^N$, where $N$ is the number of generated samples ($N \gg K$). 
% To use  for 
For each downstream task, we fine-tune an additional pre-trained classifier $\cC_{\theta}$ parameterized by $\theta$ on the synthetic dataset $\cD_{\text{syn}}$ for evaluating on the target task\footnote{While In-context Learning~\citep{brown2020language} can also be utilized, it is often hard to fit all generated instances into the context window, especially for datasets with high cardinality.}. 
% This deliberate design is aimed at leveraging $\cM$ to create a substantially augmented dataset for downstream tasks.





% The training set Dtrain = {(x, y)i}
% consists of K training samples per label where x =
% [x1, x2, . . . , xn] is a text sequence with n tokens.
\vspace{-1ex}
\subsection{Limitations of Existing Synthetic Data Generation Methods}
\label{sec:limitations}
Here, we take a closer look at the synthetic text data generated by two representative approaches: ZeroGen~\citep{ye2022zerogen}, which directly instructs LLMs for data generation, and DemoGen~\citep{gpt3mix,meng2023tuning}, which augments the prompt with few-shot demonstrations. 
We observe that these methods often introduce distribution shifts and exhibit limited diversity, which can be suboptimal for improving downstream performance. The illustration is as follows, and we include additional figures in Appendix~\ref{sec:add_prelim}.



\begin{figure}
	\centering
	\vspace{-3ex}
	\subfigure[t-SNE plot]{
		\includegraphics[width=0.28\linewidth]{figures/bc5cdr_disease_sentencebert_bsl.pdf}
		\label{fig:bc5cdr_disease_sentencebert_bsl}
	} %\hfill
         \hspace{-2.3ex}
	% \subfigure[MEDIQA-RQE]{
	% 	\includegraphics[width=0.25\linewidth]{figures/mediqa_rqe_sentencebert_bsl.pdf}
	% 	\label{fig:mediqa_rqe_sentencebert_bsl}
	% }\hspace{-1.5ex}
     \subfigure[Case study of generated examples]{
		\includegraphics[width=0.69\linewidth]{figures/case_study_prelim.pdf}
		\label{fig:case_study_prelim}
	}
	\caption{Preliminary Studies. (a) is from BC5CDR-Disease and (b) is from MEDIQA-RQE.\vspace{-1ex}}
	\vspace{-1ex}
\label{fig:prelim1}
% \vspace{-3ex}
\end{figure}



\textbf{Distribution Shift.} An inherent challenge when adapting LLMs to specific domains for text generation is the issue of \emph{distribution shift}, given that LLMs are primarily trained on vast amounts of web text in general domains. In Figure \ref{fig:bc5cdr_disease_sentencebert_bsl}, we visualize the embeddings\footnote{We employ SentenceBERT~\citep{reimers2019sentence} as the text encoder.} of both the ground truth training data and synthetic datasets generated via two representative methods. Overall, these methods use generic prompts (see Appendix~\ref{sec:prompt_format_bsl} for details) with minimal domain-specific constraints.
This limitation remains evident even when incorporating few-shot demonstrations into the process, with a notable disparity between the embeddings of the ground truth data and synthetic data.
% as there exists a large discrepancy between the embeddings of the ground truth data and the synthetic data. 
% \ran{sentencebert, more in appendix} 

To quantify the data distribution shift, we employ Central Moment Discrepancy (CMD)~\citep{zellinger2017central} to measure the gap between synthetic and real data across six clinical NLP datasets. Particularly, a high CMD value indicates a large gap between the two given distributions. Figure \ref{fig:cmd-baseline} illustrates that both ZeroGen and DemoGen exhibit elevated CMD scores, indicating substantial dissimilarity between the synthetic data and those of the real dataset.

% TSNE-embedding
% Case Study

 \begin{figure}
	\centering
	\vspace{-4.5ex}
	\subfigure[CMD]{
		\includegraphics[width=0.345\linewidth]{figures/cmd-baseline.pdf}
		\label{fig:cmd-baseline}
	} %\hfill
         \hspace{-1.5ex}
	% \subfigure[MEDIQA-RQE]{
	% 	\includegraphics[width=0.25\linewidth]{figures/mediqa_rqe_sentencebert_bsl.pdf}
	% 	\label{fig:mediqa_rqe_sentencebert_bsl}
	% }\hspace{-1.5ex}
     \subfigure[Entity Coverage]{
		\includegraphics[width=0.345\linewidth]{figures/avg-entity-baseline.pdf}
		\label{fig:avg-entity-baseline}
	}
 \hspace{-1.5ex}
      \subfigure[Entity Frequency]{
		\includegraphics[width=0.28\linewidth]{figures/bc5cdr_disease_freq_bsl.pdf}
		\label{fig:bc5cdr_disease_freq_bsl}
	}
 \vspace{-2ex}
	\caption{Preliminary Studies. (c) is from BC5CDR-Disease and is in log scale. \vspace{-3ex}}

\label{fig:prelim2}
\end{figure}

\textbf{Limited Diversity.}
Clinical datasets in real-world scenarios harbor a wealth of valuable knowledge that can be challenging to replicate within synthetically generated data by AI models. We evaluate synthetic dataset diversity by using both entity quantity and their normalized frequencies. The results are illustrated in Figures~\ref{fig:avg-entity-baseline} and \ref{fig:bc5cdr_disease_freq_bsl}. Our analysis reveals that datasets generated by ZeroGen and DemoGen exhibit a limited number of clinical entities, having a substantial discrepancy with the ground truth. 
Furthermore, it is highlighted that only a minority of potential entities and relations are frequently referenced across instances, while the majority are generated infrequently.

% Furthermore, these synthetic entities exhibit a \textit{long-tailed distribution} of normalized frequencies, highlighting that only a minority are frequently referenced across instances, while the majority of potential entities and relations are generated infrequently.

To explicitly illustrate the aforementioned limitations of synthetic datasets created using existing methods, we present a case study in Figure~\ref{fig:case_study_prelim}. In this case study, we randomly select one sample from each class within the training set generated by ZeroGen and DemoGen. These selected samples are compared with the ground truth data from the MEDIQA-RQE dataset, which aims to predict whether a consumer health query can entail an existing Frequently Asked Question (FAQ). The comparison reveals that the samples generated by ZeroGen and DemoGen tend to be more straightforward, lacking the \textit{sufficient details} and \textit{real-case nuances} present in the ground truth data. 
Furthermore, the generated samples adhere to a more uniform style and structure, while the ground truth encompasses various situations and writing styles, including urgent and informal inquiries.
% Furthermore, the generated samples lack the \textit{situation diversity} and \textit{style variability} present in the ground truth. The ground truth encompasses various situations and writing styles, including urgent and informal inquiries, while the generated samples adhere to a more uniform style and structure.


\section{Clinical Knowledge Infused Data Generation}
The revealed insights from the preliminary studies assert the necessity of domain-tailored knowledge for clinical synthetic data generation. In pursuit of efficient, effective, and scalable data generation for clinical domains, we introduce our novel framework, {\ours}, a prior knowledge-informed clinical data generation. The overview of {\ours} is shown in Figure~\ref{fig:overall}. This innovative two-step methodology harnesses the emergent capabilities of LLMs and external knowledge from KGs to facilitate the synthesis of clinical data, even when only presented with few-shot examples. 



\subsection{Clinical knowledge extraction}
Contrary to previous studies~\citep{ye2022zerogen,ye2022progen,meng2022generating,meng2023tuning} which employ generic queries to prompt LLMs for text generation, {\ours} emphasizes refining clinically informed prompts. This approach aims to extract rich clinically relevant knowledge from parametric (\eg LLMs) or nonparametric sources (\eg knowledge graphs) and tailor it to clinical NLP tasks.
% Different from prior research~\citep{ye2022zerogen,ye2022progen,meng2022generating,meng2023tuning} that use the simple task-specific queries to prompt Language Model (LLM) for data generation,
% {\ours}'s primary focus is to optimize clinically informed prompts to better harvest the clinical-relevant knowledge from LLMs and adapt them to clinical NLP tasks.
% emphasize on the initial ground of contextual domain knowledge.
To realize this objective, our modeling contains two dimensions including \emph{clinical topics} and \emph{writing styles}, which are integrated into the original prompts to infuse domain-specific knowledge. 
By dynamically composing different topics and writing styles together, {\ours} can provide a diverse suite of prompts, resulting in a wider spectrum of text produced from LLM.


% harness the synergistic potential of Knowledge Graphs (KGs) and Large Language Models (LLMs) in constructing a candidate set enriched with prior knowledge, specifically (1) the sampling of pertinent entities or relations from external knowledge graphs (KGs), and (2) the extraction of relevant hints through queries to Language Models (LLMs).\ran{style}

\subsubsection{Clinical Topics Generation}
We provide two choices to generate clinical topics -- one is to sample related entities or relations from external KG, and the other is to query relevant knowledge from LLM.

\textbf{Topics sampled from Non-Parametric KGs.} 
Healthcare KGs offer 
% comprehensive view of medical concepts and their relationships.
a rich collection of medical concepts and their complex relationships, and have emerged as a promising tool for organizing medical knowledge in a structured way~\citep{li2022graph,cui2023survey}. 
In our methodology, we employ the iBKH KG~\citep{su2023biomedical} due to its broad coverage over clinical entities. 
To illustrate, for the Disease Recognition task (NCBI)~\citep{ncbi-disease}, we extract all medication nodes from the iBKH to bolster the pharmaceutical information. 
As another example, we retrieve links between drug and disease nodes for the chemical and disease relation extraction (CDR) task~\citep{cdr_dataset}. 
By integrating the information from the clinical KG into our data generation process, we guarantee that our generated samples exhibit a high degree of contextual accuracy, diversity, and semantic richness.


\begin{figure}[t]
    \centering
    \vspace{-4ex}
    \includegraphics[width=0.96\linewidth]{figures/clingen-framework.pdf}
    \caption{The overview of \ours. \vspace{-2.5ex}
    % The left orange panel illustrates the knowledge extraction part. The middle purple panel shows the synthetic data generation module. The right green one is the fine-tuning step.}
    }
    % \vspace{-ex}
    \label{fig:overall}
\end{figure}

\textbf{Topics queried from Parametric Model (LLMs).} 
LLMs provide an alternative method for acquiring domain knowledge, as they are pre-trained on extensive text corpora, including medical literature.
% Large Language Models (LLMs) offer another viable avenue for acquiring foundational domain knowledge, owing to their extensive training on diverse text corpora, including the vast expanse of medical literature. 
Specifically, we aim to harness the rich clinical domain knowledge encoded in ChatGPT (\texttt{gpt-3.5-turbo-0301}) to augment the prompt. 
The incorporated prior knowledge from LLMs is focused on entity categories that hold significant relevance within clinical text datasets, including diseases, drugs, symptoms, and side effects.
For each of these pivotal entity types, we prompt the LLMs by formulating inquiries, \eg, ``\texttt{Suppose you are a clinician and want to collect a set of <Entity Type>. Could you list 100 entities about <Entity Type>?}''. These crafted conversational cues serve as effective prompts, aiding in the retrieval of clinically significant entities from the extensive domain knowledge within LLMs. For each entity type, we generate 300 entities which will be used for synthetic data generation.

\vspace{-0.5ex}
\subsubsection{Writing Styles Suggestion}
\vspace{-0.5ex}
\textbf{Styles suggested by LLMs.} To address the limitations mentioned in Sec~\ref{sec:limitations} and introduce a diverse range of writing styles into the generated samples, we leverage the powerful LLM again by suggesting candidate writing styles for each task. Specifically, we incorporate task names into our prompts (e.g., disease entity recognition, recognizing text entailment, etc.) and integrate few-shot demonstrations. We then engage ChatGPT in suggesting several potential sources, speakers, or authors of the sentences. See Appendix~\ref{sec: style_prompt} for detailed prompt. Responses such as ``\texttt{medical literature}" or ``\texttt{patient-doctor dialogues}" are augmented into the prompts to imitate the writing styles found in real datasets. 
% \joyce{I'm not quite sure what this quite looks like. Is there an example in the appendix/supplemental part? Seems you're downplaying it here otherwise.}
% explicitly query 

\vspace{-0.5ex}
\subsection{Knowledge-infused Synthetic Data Generation}
\vspace{-0.5ex}
% After building the candidate set with one of the previous approaches, we randomly sample one keyword each time and augment it into the prompt. 
% % An example of the prompt on xxx dataset is shown in Figure xxx. 
% For example, a keyword for the NCBI dataset could be ``\texttt{stroke}", then we enrich the prompt for querying ChatGPT by adding an additional sentence ``\texttt{generate a sentence about stroke}" (see Figure/Appendix for full prompt). 
With the generated entities as well as styles, the key challenge becomes how to leverage them to extract rich clinical information from the LLM for improving synthetic data quality.
% hat, if used wisely, can be potentially leveraged for generalizing over limited context.
% if they are guided effectively. 
Directly putting all the elements to enrich the prompt is often infeasible due to the massive size of entities.
% and over-complicated 
To balance informativeness as well as diversity, we propose a  knowledge-infused strategy, where the collected clinical topics and writing styles serve as the base unit. 
% for leading LLMs towards clinical domains.  
% To elucidate, each extracted hinting keyword from the candidate set takes a role in shaping a clinically informed prompt structure. 
In each step, we randomly sample a topic and a writing style from the candidate set to augment the prompts.
For instance, for the Disease Recognition (NCBI) task, consider a clinical entity like ``\texttt{stroke}" . We enrich the prompt query for LLM by appending ``\texttt{generate a sentence about stroke}" as a generation guidance. For a comprehensive view of the prompt formats across various tasks, please refer to Appendix~\ref{sec:prompt_format}. 
Despite its simplicity, this knowledge-infused strategy ensures that the clinical context is incorporated into the prompts while encouraging prompt diversity (via composing different entities and writing styles), 
thereby enhancing the quality and clinical relevance of the generated synthetic data.


\subsection{Language model fine-tuning}
After generating synthetic data $\cD_{\text{syn}}$ through LLMs, we fine-tune a pre-trained classifier $\cC_{\theta}$ to each downstream task. Following \citep{meng2023tuning}, we first fine-tune $\cC_{\theta}$ on $\cD_{\text{train}}$ with standard supervised training objectives (denoted as $\ell(\cdot)$), then on the synthetic data $\cD_{\text{syn}}$ as  
\begin{align}
    \label{eq:stage1}
    \mathrm{Stage~I}: \theta^{(1)} = \min_{\theta}~\mathbb{E}_{(x, y) \sim \cD_{\text{train}}} \ell\left( f(x; \theta), y \right), \\
    \mathrm{Stage~II}: \theta^{(2)} =  \min_{\theta}~\mathbb{E}_{(x, y) \sim \cD_{\text{syn}}} \ell\left( f(x; \theta), y \right),  \theta_{\text{init}} = \theta^{(1)}.
\end{align} 
% \end{equation}
It's important to highlight that we strictly follow a standard fine-tuning process and avoid using any extra techniques: (1) for standard classification tasks, $\ell(\cdot)$ is the cross-entropy loss; (2) for multi-label classification tasks, $\ell(\cdot)$ is the binary cross-entropy loss; 
(3) for token-level classification tasks, we stack an additional linear layer as the classification head and  $\ell(\cdot)$ is the token-level cross-entropy loss. 
% This is to ensure methodological consistency and transparency throughout the evaluation across different methods and tasks. 
The design of \emph{advanced learning objectives} as well as \emph{data mixing strategies}, while important, are orthogonal to the scope of this paper. 






% It is important to underscore that, in pursuit of a rigorous quality evaluation of the synthetic data, we rigorously adhere to a conventional fine-tuning pipeline, without any auxiliary techniques. Our chosen learning objective is the minimization of the cross-entropy loss against the task-specific target, ensuring methodological consistency and transparency throughout the evaluation across different methods and tasks.

% \subsubsection{Style}

\section{Empirical Evaluation}
Given our focus on synthetic text generation, our primary interest lies in faithfully evaluating different synthetic text generation approaches under few-shot scenarios, rather than engaging in a ``state-of-the-art" race with general few-shot learning approaches (\ie we never claim that we achieve \emph{state-of-the-art} performance on these tasks). In this context, the following questions particularly intrigue us:
\textbf{RQ1}: How does {\ours} perform when compared with baselines on different downstream tasks?  
\textbf{RQ2}: How do different factors such as LLM generators and the size of synthetic data affect the performance of {\ours}? 
\textbf{RQ3}: How is the quality of the synthetic datasets generated by {\ours} and baselines?
These questions are addressed in Sec~\ref{sec:model_perf}, Sec~\ref{sec:ablation} and Sec~\ref{sec:quality_analysis}, respectively.

% GPT annotation~\citep{wang2021want,ding-etal-2023-gpt,gu2023distilling}

% In pursuing few-shot synthetic data generation, we aim to delve into practical applications in clinical scenarios. We committed to conducting a thorough evaluation that 
\subsection{Experiment Setup}
We conduct experiments in the few-shot settings with 5 examples available for each class. We employ ChatGPT~\citep{chatgpt} (\texttt{gpt-3.5-turbo-0301}) as the generator for both {\ours} and baselines for a fair comparison. The pre-trained PubMedBERT~\citep{gu2021domain} is then applied to fine-tune on the generated synthetic data for evaluation, where we consider both the \texttt{Base} and \texttt{Large} variants. See Appendix~\ref{sec:implementation_details} for implementation details.

\textbf{Datasets and Tasks.}
In our exploration of few-shot synthetic data generation, we undertake a comprehensive evaluation of \textbf{16 datasets} across a diverse array of tasks typically encountered in clinical NLP benchmarks~\citep{blue,fries2022bigbio}. Specifically, we consider 2 text classification, 3 relation extraction (RE), 3 natural language inference (NLI), 2 fact verification, 1 sentence similarity (STS), 4 NER, and 1 attribute extraction tasks. Please see Appendix~\ref{sec:dataset_description} for detailed dataset descriptions and the statistics of each dataset.

% The evaluation tasks and datasets are summarized in Table \ref{tab:datastats}. Note that the number of training samples indicates the size of the \textit{original} training set. Please see Appendix~\ref{sec:dataset_description} for detailed dataset descriptions.

\textbf{Baselines.}
We compare {\ours} with \textbf{9 baselines} in total, including
6 data augmentation methods and 3 LLM-based data generation techniques. The data augmentation models include Word Substitution~\citep{checklist}, Back Translation~\citep{uda}, Mixup~\citep{chen2020mixtext,seqmix}, Transformer~\citep{kumar2020data,melm}, LightNER~\citep{lightner}, and KGPC~\citep{chen2023few}.
% Note that LightNER and KGPC are designed specifically for NER tasks. Back Translation cannot be applied to NER and RE tasks as it's non-trivial to locate the related entities in the generated sentences. 
For LLM-based generation models, we consider ZeroGen~\citep{ye2022zerogen}, DemoGen~\citep{meng2023tuning,gpt3mix} and ProGen~\citep{ye2022progen} as representative methods. See Appendix~\ref{sec:baseline_details} for details.


\subsection{Model Performance with the Synthetic Data}
\label{sec:model_perf}
\begin{table}[t]
% \floatconts
  \caption{Experimental results aggregated by tasks. \textbf{Bold} and \underline{underline} indicate the best and second best results for each dataset, respectively. $\dagger$: The models can only be applied to NER tasks, and the number is reported from the original paper. $*$: Since the two $\dagger$ models only report results on two NER datasets, we report an average performance on those two datasets for a fair comparison.}
  \resizebox{\linewidth}{!}{
  \begin{tabular}{lcc|cccc|ccccc}
  \toprule
  \multirow{3.5}{*}{\bf Task}  & \multicolumn{2}{c|}{\textit {Single-Sentence Tasks}} &  \multicolumn{4}{c|}{\textit{Sentence-Pair Tasks}} & \multicolumn{5}{c}{\textit{Token Classification Tasks}} \\
  \cmidrule(lr){2-3} \cmidrule(lr){4-7} \cmidrule(lr){8-12} 
  & \bfseries Text Class & \bfseries RE & \bfseries NLI & \multicolumn{2}{c}{\textbf{Fact Verification}} & \bfseries STS & \multicolumn{2}{c}{\textbf{NER}} & \multicolumn{3}{c}{\textbf{MedAttr}} \\
  % \midrule
  \cmidrule(lr){2-2} \cmidrule(lr){3-3} \cmidrule(lr){4-4} \cmidrule(lr){5-6} \cmidrule(lr){7-7} \cmidrule(lr){8-9} \cmidrule(lr){10-12}
  & F1 & F1 & Acc & Acc & F1 & Acc & F1 & F1-subset$^*$ & P & R & F1\\
  \midrule
  \multicolumn{12}{l}{\textbf{PubMedBERT$_{\texttt{Base}}$}} \\
  \midrule
  Supervised-Full & 77.01 & 77.34 & 79.20 & 67.58 & 65.49 & 75.70 & 89.67 & 87.27 & --- & --- & ---\\
  Supervised-Few & 18.61 & 43.89 & 44.64 & 29.43 & 27.10 & 55.70 & 39.41 & 34.12 & 38.11 & 43.82 & 40.77 \\
  \midrule
  DA-Word Sub & 40.74 & 38.14 & 55.08 & 28.86 & 25.83 & 54.40 & 44.30 & 40.41 & 40.25 & 47.65 & 43.64\\
  DA-Back Trans & 47.24 & --- & 54.30 & 32.15 & 28.04 & 55.80 & --- & --- & --- & --- & ---\\
  DA-Mixup & 45.09 & 43.37 & 53.52 & 32.78 & 29.12 & 58.20 & 42.20 & 37.65 & 42.37 & 48.96 & 45.43\\
  DA-Transformer & 41.02 & 47.56 & 55.71 & 35.32 & 31.77 & 58.80 & 44.75 & 39.66 & 37.82 & 44.28 & 40.80\\
  LightNER$^\dagger$ & --- & --- & --- & --- & --- & --- & --- & 39.49 & --- & --- & ---\\
  % DA-MELM$^\dagger$& --- & --- & --- & --- & --- & --- & 44.75 & 39.66 & 37.82 & 44.28 & 40.80\\
  KGPC$^\dagger$ & --- & --- & --- & --- & --- & --- & --- & 51.60 & --- & --- & ---\\
  \midrule
  ZeroGen & 59.02 & 63.84 & 55.96 & 35.30 & 32.50 & 68.35 & 56.97 & 48.26 & 52.80 & 49.53 & 51.11\\
  DemoGen & 64.09 & 67.46 & 59.80 & 40.30 & 35.95 & 70.85 & 60.16 & 53.91 & 58.15 & 56.84 & 57.49\\
  ProGen & 65.16 & 67.23 & 59.57 & 37.71 & 34.54 & 69.30 & 60.49 & 55.11 & 57.76 & 58.57 & 58.16\\  
  \midrule
  % \hline
  \rowcolor{teal!10} {\ours} w/ KG & \underline{67.15} & \underline{69.01} & \underline{64.89} & \underline{43.83} & \underline{39.43} & \underline{72.17} & \textbf{64.26} & \textbf{60.11} & \textbf{71.75} & \underline{65.20} & \textbf{68.32}\\
  \rowcolor{teal!10} {\ours} w/ LLM & \textbf{67.82} & \textbf{70.06} & \textbf{67.24} & \textbf{46.50} & \textbf{41.46} & \textbf{73.31} & \underline{63.17} & \underline{58.49} & \underline{68.19} & \textbf{66.79} & \underline{67.48}\\
  % \midrule
  \rowcolor{gray!15} Performance Gain & 4.08\% & 3.85\% & 12.44\% & 15.38\% & 15.33\% & 3.47\% & 6.23\% & --- & --- & --- & 17.47\% \\
  \midrule
  \multicolumn{12}{l}{\textbf{PubMedBERT$_{\texttt{Large}}$}} \\
  \midrule
  Supervised-Full & 80.06 & 79.64 & 82.65 & 72.97 & 69.23 & 78.80 & 90.15 & 87.68 & --- & --- & ---\\
  Supervised-Few & 17.86 & 52.68 & 50.00& 40.90 & 30.50 & 59.73 & 42.84 & 37.57 & 41.30 & 45.02 & 43.08 \\
  \midrule
  DA-Word Sub & 43.99 & 44.35 & 57.66 & 35.51 & 31.95 & 55.30 & 46.67 & 43.70 & 46.77 & 43.52 & 45.09\\
  DA-Back Trans & 50.98 & --- & 58.39 & 34.12 & 31.36 & 56.40 & --- & --- & --- & --- & ---\\
  DA-Mixup & 46.74 & 50.97 & 57.35 & 34.01 & 31.10 & 58.50 & 46.69 & 43.01 & 41.25 & 52.09 & 46.04\\
  DA-Transformer & 44.41 & 46.12 & 58.94 & 35.09 & 30.95 & 58.10 & 46.94 & 43.50 & 43.36 & 45.78 & 44.54\\
  LightNER$^\dagger$ & --- & --- & --- & --- & --- & --- & --- & --- & --- & --- & ---\\
  % DA-MELM$^\dagger$ & --- & --- & --- & --- & --- & --- & 46.94 & 43.50 & 43.36 & 45.78 & 44.54\\
  KGPC$^\dagger$ & --- & --- & --- & --- & --- & --- & --- & --- & --- & --- & ---\\
  \midrule
  ZeroGen & 61.51 & 65.18 & 63.47 & 41.12 & 36.10 & 72.69 & 57.79 & 49.10 & 54.04 & 51.40 & 52.69\\
  DemoGen & 64.97 & 68.65 & 64.58 & 42.61 & 38.69 & 74.37 & 61.43 & 55.61 & 62.67 & 61.02 & 61.83\\
  ProGen & 65.01 & 69.23 & 63.32 & 42.79 & 38.63 & 74.89 & 62.47 & 57.31 & 57.21 & 63.70 & 60.28\\
  \midrule
  % \hline
  \rowcolor{teal!10} {\ours} w/ KG & \underline{66.76} & \underline{71.47} & \textbf{70.90} & \underline{48.62} & \underline{42.45} & \underline{75.82} & \textbf{65.48} & \textbf{62.23} & \underline{70.96} & \textbf{69.66} & \textbf{70.30}\\
  \rowcolor{teal!10} {\ours} w/ LLM & \textbf{67.61} & \textbf{72.81} & \underline{70.50} & \textbf{49.51} & \textbf{43.72} & \textbf{76.21} & \underline{65.36} & \underline{61.89} & \textbf{71.61} & \underline{66.86} & \underline{69.15}\\
  \rowcolor{gray!15} {Performance Gain} & 4.00\% & 5.17\% & 9.79\% & 15.70\% & 13.00\% & 3.47\% & 1.76\% & --- & --- & --- & 13.70\% \\
  \bottomrule
  \end{tabular}
  }
  \vspace{-1ex}
  \label{tab:main-table}
\end{table}
Table~\ref{tab:main-table} summarizes the experimental results on different datasets. We also conduct supervised learning on the original training data and the extracted few-shot examples, denoted as ``Supervised-Full" and ``Supervised-Few", respectively.
Due to space limits, we report the average performance over all datasets for each task, but provide the detailed results for each dataset in Tables~\ref{tab:single-sent}, \ref{tab:sent-pair}, \ref{tab:token-class} in Appendix~\ref{sec:more_experimental_results}. 
Based on the experimental results, we have the following findings:

\noindent $\diamond$ Our proposed approach, {\ours}, consistently outperforms the baselines across all tasks. The average performance gain over all \textit{main} metrics is 8.98\% at \texttt{Base} scale and 7.27\% at \texttt{Large} scale. In addition, methods utilizing LLMs have better performance than traditional data augmentation techniques, illustrating the capacity of LLMs to extract valuable information from limited examples. The performance gain of DemoGen and ProGen over ZeroGen further demonstrates the positive influence of few-shot examples on overall comprehension.

\noindent $\diamond$ In \textit{token classification tasks}, {\ours} performs better with KG compared to LLM. This improvement stems from the strong alignment between the task's target and the generated domain knowledge, where the extracted topics serve as direct labels for these datasets. The \textit{single-sentence} and \textit{sentence-pair tasks}, on the other hand, display an advantage for the LLM-based knowledge extraction. This can be attributed to two potential reasons: first, these tasks prioritize understanding entire sentences over specific terminologies, and some specialized terms might even impede LLM comprehension. Second, KGs may not always contain the required information. For example, in a RE task involving chemicals and proteins, some types of the relations are absent from the KG, thus the performance gain is rather limited.
% as this information is absent from the KG, we have to extract drug-gene relations as a substitute.

\noindent $\diamond$ Some data augmentation methods are task-specific, limiting their generalizability. For example, LightNER and KGPC are designed specifically for NER tasks. It is also non-trivial to apply Back Translation to NER or RE tasks, as it requires locating related entities in the generated sentence accurately.
In contrast, {\ours} is flexible and can be effectively applied to various tasks.


% \begin{itemize}[leftmargin=0.5cm]

% \item \textbf{In Terms of Model Types}\\
% $\diamond$ Methods utilizing LLMs consistently outperform traditional data augmentation techniques, underscoring the capacity of LLMs to extract valuable information from limited examples while preserving diversity in terms of content, structures, and lexicons. This observation is further supported by the findings presented in Section~\ref{sec:diversity_measures}.

% $\diamond$ The consistent superiority of DemoGen and ProGen over ZeroGen underscores the positive influence of few-shot examples on overall comprehension.

% $\diamond$ Our proposed approach, {\ours}, consistently outperforms the alternatives across all 16 datasets, showcasing relative improvements of up to 14.03\%.

% \item \textbf{In Terms of Clinical Task Types}\\
% $\diamond$ The performance gain in \textit{token classification tasks} is particularly significant, which predominantly involves the identification of disease or drug entities. The strong alignment between this task target and our generated prior knowledge guarantees the exceptional quality of the synthetic dataset created by {\ours}.

% % $\diamond$ \textit{single-sentence tasks}

% $\diamond$ The proposed model, {\ours}, is flexible and can be effortlessly and effectively applied to various tasks.

% \item \textbf{The Way of Clinical Knowledge Extraction}\\
% $\diamond$ {\ours} with KG or LLM demonstrates comparable performance on \textit{in-depth token classification tasks}, with a slight advantage for the KG-based model. This can be attributed to the enhanced diversity and comprehensiveness of entities derived from the knowledge graph.

% $\diamond$ When combined with LLMs, the proposed approach {\ours} demonstrates superior performance in \textit{sentence-pair tasks} compared to its performance with KG. The reason for this difference can be attributed to the inherent ability of LLM  to generalize from few-shot examples and to capture nuanced semantic relationships, which is beneficial in sentence-pair tasks that require understanding contextual information and complex inference.

% \end{itemize}

% LLM better than DA
% DemoGen and ProGen better than ZeroGen
% Ours the best (prior knowledge)
% KG and LLM
% NER knowledge related tasks -> more gain

% \paragraph{Single Sentence}

% \paragraph{Sentence Pair}

% \paragraph{Token Classification}
\subsection{Ablation and Parameter Studies}
\label{sec:ablation}
\textbf{Effect of Different LLM Generators.}
To investigate the impact of various LLMs on {\ours}, we leverage other models in the GPT-family as the text generator. Specifically, we utilize InstructGPT (\texttt{text-curie-001})~\citep{ouyang2022training} and GPT-4~\citep{gpt4}. Note that we only generate 500 samples in the GPT-4 setting due to budget constraints, but we provide the results of GPT-3.5 with same amount of synthetic samples for a fair comparison. 
From Figure~\ref{fig:generator} we observe that {\ours} generally outperforms the best baseline in all settings.
% except for the NCBI-Disease dataset. 
Additionally, we observe generally improved performance with larger models, as they often have better capabilities to follow our designed instructions for the given prompts. See Appendix~\ref{sec:add_ablation_para} for more figures.

\begin{figure}[t!]
    \centering
    \begin{minipage}{0.48\textwidth}
        \centering
        \subfigure[HOC]{
            \includegraphics[width=0.48\textwidth]{figures/generator-HOC.pdf}
            \label{fig:generator-HOC}
        } \hspace{-3mm}
        \subfigure[MEDIQA-RQE]{
            \includegraphics[width=0.48\textwidth]{figures/generator-MEDIQA-RQE.pdf}
            \label{fig:generator-MEDIQA-RQE}
        }
        \vspace{-2ex}
        \RawCaption{\caption{Different generators at \texttt{Base}.}\label{fig:generator}}
    \end{minipage}%
    % \begin{minipage}{0.33\textwidth}
    %     \centering
    %     \vspace{3.5mm}
    %     \includegraphics[width=0.96\textwidth]{figures/size-HOC.pdf}
    %     \caption{\textit{Accuracy vs. Confidence score.}}
    %     \label{fig:confscore}
    % \end{minipage}
    \begin{minipage}{0.48\textwidth}
        \centering
        \subfigure[HOC]{
            \includegraphics[width=0.5\textwidth]{figures/size-HOC.pdf}
            \label{fig:size-HOC}
        } \hspace{-6mm}
        \subfigure[MEDIQA-RQE]{
            \includegraphics[width=0.5\textwidth]{figures/size-Mediqa-rqe.pdf}
            \label{fig:size-Mediqa-rqe}
        }
        \vspace{-2ex}
        \RawCaption{\caption{Different proportion of data at \texttt{Base}.}\label{fig:size-synthetic}}
    \end{minipage}%
    \vspace{-0.5ex}
\end{figure}



\textbf{Effect of Size of Synthetic Data.}
In Figure~\ref{fig:size-synthetic} (and more in Appendix~\ref{sec:add_ablation_para}), we study the effect of the size of synthetic data. The result shows that {\ours} consistently outperforms the best baseline, using only around 10\% of the synthetic examples. This illustrates that incorporating domain knowledge and increasing the diversity of the prompts could be an effective way to improve the sample efficiency, and narrow the gap between the performance of synthetic and ground-truth dataset.
% \paragraph{How Many Clean Data Points is Synthetic Data Worth?}

\textbf{Comparison with few-shot inference via prompting ChatGPT.}
% How Does the Value of Synthetic Data Compare to Clean Data Points?
\begin{table}[t]
% \floatconts
% \vspace{-1ex}
  \caption{Comparison between prompting ChatGPT for inference and {\ours} at \texttt{Large} scale.\vspace{-2ex}}
  \resizebox{\linewidth}{!}{
  \begin{tabular}{lcccccccccccccc}
  \toprule
  & \bfseries HOC & \multicolumn{3}{c}{\textbf{GAD}} & \bfseries ChemProt & \bfseries MEDIQA-RQE & \multicolumn{2}{c}{\textbf{PUBHEALTH}} & \multicolumn{3}{c}{\textbf{NCBI-Disease}} & \multicolumn{3}{c}{\textbf{CASI}}\\
  % \midrule
  \cmidrule(lr){2-2} \cmidrule(lr){3-5} \cmidrule(lr){6-6} \cmidrule(lr){7-7} \cmidrule(lr){8-9} \cmidrule(lr){10-12} \cmidrule(lr){13-15}
  & F1 & P & R & F1 & F1 & ACC & ACC & F1 & P & R & F1 & P & R & F1\\
  \midrule
ChatGPT Inference & 68.76 & 84.21 & \textbf{97.46} & 90.35 & 49.42 & 74.31 & \textbf{69.50} & \textbf{52.47} & 46.62 & 52.31 & 49.30 & 48.82 & \textbf{74.75} & 59.07\\ 
  \midrule
  % \hline
  \rowcolor{teal!10} {\ours} w/ KG & 77.71 & 94.30 & 89.09 & \textbf{91.62} & 60.12 & \textbf{79.92} & 50.20 & 41.26 & \textbf{62.46} & \textbf{64.08} & \textbf{63.26} & 70.96 & 69.66 & \textbf{70.30} \\
  \rowcolor{teal!10} {\ours} w/ LLM & \textbf{78.14} & \textbf{95.08} & 86.14 & 90.39 & \textbf{63.05} & 77.36 & 52.96 & 43.31 & 61.12 & 60.16 & 60.64 & \textbf{71.61} & 66.86 & 69.15 \\
  \bottomrule
  \end{tabular}
  }
  \label{tab:gpt_inference}
  \vspace{-1ex}
\end{table}
As we employ ChatGPT as our backbone model for generating synthetic data, we also evaluate its ability for direct inference from the original training sets using few-shot in-context learning. Due to budget limits, we only run experiments on datasets with few testing samples for each task.  
As presented in Table~\ref{tab:gpt_inference}, {\ours} at PubMedBERT$_{\texttt{Large}}$ scale achieves better results on 5 out of 6 datasets than ChatGPT few-shot learning, which uses $\sim 530 \times$ more parameters. 
One exception is for PUBHEALTH, as it requires complex reasoning abilities that PubMedBERT$_{\texttt{Large}}$ may not fully possess. 
% Overall, the performance gain verifies that the synthetic dataset generated by our model effectively captures task-relevant information, enabling PubMedBERT$_{\texttt{Large}}$ to generalize well across these tasks.  
Overall, {\ours} offers cost-effective and time-efficient advantages. 
While it entails a one-time investment in both money and time for synthetic training data generation, subsequent prediction relying on a moderate-sized pretrained language model is much more efficient. 
Besides, the continued use of ChatGPT for inference on new testing data 
incurs ongoing time and financial costs, while our model requires zero additional costs for querying APIs. The price information is exhibited in Appendix \ref{sec:apd_cost}.

\textbf{Effect of Topic Extraction and Style Suggestion.}
We inspect different components of {\ours} in Table~\ref{tab:ablation}. It is observed that both Topics Extraction and Style Suggestion contribute to model performance as they enhance the relevance of generated samples to domain knowledge and introduce greater structural diversity. Different from other datasets, MEDIQA-RQE shows more performance gain incorporating writing style than topics. It is because NLI tasks focus on capturing the relationships between two sentences while incorporating additional knowledge entities does not directly help the model improve the reasoning ability.

% \joyce{Style seemingly plays a key role in MEDIQA-RQE. Thoughts as to why?}

% \begin{table}[hbtp]
% % \floatconts
%   \caption{Ablation studies on topic extraction and style suggestion at \texttt{Base} scale.}
%   \resizebox{0.5\linewidth}{!}{
%   \begin{tabular}{lcc|cc|cc|cc}
%   \toprule
%   & \multicolumn{2}{c}{\textbf{HOC}} & \multicolumn{2}{c}{\textbf{CDR}} & \multicolumn{2}{c}{\textbf{MEDIQA-RQE}} & \multicolumn{2}{c}{\textbf{NCBI-Disease}}\\
%   % \midrule
%   \cmidrule(lr){2-3} \cmidrule(lr){4-5} \cmidrule(lr){6-7} \cmidrule(lr){8-9}
%   & w/ KG & w/ LLM & w/ KG & w/ LLM & w/ KG & w/ LLM & w/ KG & w/ LLM \\
%   \midrule
%   {\ours} & \textbf{76.28} & \textbf{76.42} & \textbf{61.74} & \textbf{63.34} & \textbf{74.85} & \textbf{72.40} & \textbf{59.46} & \textbf{55.95} \\
%   w/o Styles & 73.25 & 74.40 & 59.10 & 60.15 & 67.21 & 66.50 & 57.97 & 54.70 \\
%   % w/o Topics & 70.86 & 70.86 & 58.51 & 58.51 & 64.87 & 64.87 & 57.30 & 57.30\\
%   w/o Topics & \multicolumn{2}{c|}{70.86} & \multicolumn{2}{c|}{58.51} & \multicolumn{2}{c|}{64.87} & \multicolumn{2}{c}{55.09} \\
%   \bottomrule
%   \end{tabular}
%   }
%   \label{tab:ablation}
% \end{table}

\begin{table}[H]
\vspace{-1ex}
    \begin{minipage}{0.54\linewidth}
        \centering
        \hspace{-4mm}
        \resizebox{\linewidth}{!}{
  \begin{tabular}{lcc|cc|cc|cc}
  \toprule
  & \multicolumn{2}{c}{\textbf{HOC}} & \multicolumn{2}{c}{\textbf{CDR}} & \multicolumn{2}{c}{\textbf{MEDIQA-RQE}} & \multicolumn{2}{c}{\textbf{NCBI-Disease}}\\
  % \midrule
  \cmidrule(lr){2-3} \cmidrule(lr){4-5} \cmidrule(lr){6-7} \cmidrule(lr){8-9}
  & w/ KG & w/ LLM & w/ KG & w/ LLM & w/ KG & w/ LLM & w/ KG & w/ LLM \\
  \midrule
  \rowcolor{teal!10} {\ours} & \textbf{76.28} & \textbf{76.42} & \textbf{61.74} & \textbf{63.34} & \textbf{74.85} & \textbf{72.40} & \textbf{59.46} & \textbf{55.95} \\
  w/o Styles & 73.25 & 74.40 & 59.10 & 60.15 & 67.21 & 66.50 & 57.97 & 54.70 \\
  % w/o Topics & 70.86 & 70.86 & 58.51 & 58.51 & 64.87 & 64.87 & 57.30 & 57.30\\
  w/o Topics & \multicolumn{2}{c|}{70.86} & \multicolumn{2}{c|}{58.51} & \multicolumn{2}{c|}{69.86} & \multicolumn{2}{c}{55.09} \\
  \bottomrule
  \end{tabular}
  }
        \RawCaption{\caption{Ablation studies on topic extraction and style suggestion at \texttt{Base} scale.}\label{tab:ablation}}
    \end{minipage}%
    \hspace{3mm}
    \begin{minipage}{0.43\linewidth}
        \centering
        \resizebox{\linewidth}{!}{
  \begin{tabular}{lcccc}
  \toprule
  % & \multicolumn{2}{c}{\textbf{HOC}} & \multicolumn{2}{c}{\textbf{CDR}} & \multicolumn{2}{c}{\textbf{MEDIQA-RQE}} & \multicolumn{2}{c}{\textbf{NCBI-Disease}}\\
  & \bfseries HOC & \bfseries CDR & \bfseries MEDIQA-RQE & \bfseries NCBI-Disease \\
  % \midrule
  % \cmidrule(lr){2-3} \cmidrule(lr){4-5} \cmidrule(lr){6-7} \cmidrule(lr){8-9}
  % & Avg.Sim. & Avg.Sim. & Avg.Sim. & Avg.Sim. \\
  \midrule
  ZeroGen & 0.512 & 0.469  &   0.277 &  0.528  \\
  DemoGen & 0.463 &    0.377   & 0.289  &  0.281  \\
  ProGen  & 0.481 &   0.321   & 0.290  &  0.357  \\
  \rowcolor{teal!10} {\ours} w/ KG  & 0.440 & \textbf{0.291}   & \textbf{0.243} &  0.180    \\
  \rowcolor{teal!10} {\ours} w/ LLM & \textbf{0.432}  & 0.338  &  0.255 &  \textbf{0.155} \\
  % \midrule
  Ground truth   & 0.265  & 0.268  &  0.164 &  0.262  \\
  \bottomrule
  \end{tabular}
  }
        \RawCaption{\caption{Average Pairwise Similarity.} \label{table:aps}}
    \end{minipage} 
    \vspace{-1ex}
\end{table}



% \vspace{-1ex}
\section{Quality Analysis of the Synthetic Data}
% \vspace{-1ex}
\label{sec:quality_analysis}
\textbf{Data Distribution Measures.}
In this section, we present the data distribution and diversity measurement of the synthetic dataset generated by {\ours}.
Figure~\ref{fig:bc5cdr_disease_sentencebert_ours} shows the t-SNE plot of data generated by {\ours} and baselines compared with the ground truth. This visualization clearly demonstrates that {\ours} exhibits a greater overlap with the ground truth, indicating a similar distribution as the original dataset.
In addition, as depicted in Figure \ref{fig:cmd-all}, the embedding of \textit{\ours} aligns more closely with the ground truth distribution than other baselines across all six datasets, further justifying the efficacy of {\ours} for mitigating the distribution shift issue.
% Notably, \textit{\ours} w/ LLM tends to replicate the ground truth more precisely than \textit{\ours} w/ KG. 


\textbf{Diversity Measures.}
Table~\ref{table:aps} calculates the average cosine similarity for sample pairs using SentenceBERT embeddings.
Compared to baselines, the dataset generated with {\ours} exhibits lower cosine similarity and the average similarity is close to that of the ground truth training data, which shows {\ours} could render more diverse data.  
% Moreover, while {\ours} w/ KG envelops more entities, {\ours} w/ LLM delivers a data distribution that is most congruent with the ground truth, further verifying the observations from Figure \ref{fig:cmd-all} and \ref{fig:avg-entity-all}.
Moreover, Figure \ref{fig:avg-entity-all} highlights the ability of \textit{\ours} to cover a broader range of entities in comparison to the baselines. We find that {\ours} w/ KG captures a larger variety of entities than {\ours} w/ LLM, because KG tends to cover more extensive knowledge, including relatively uncommon information that may not be present in LLMs.
% attributed to the rich information, especially rarer knowledge, contained in the knowledge graph.
Figure \ref{fig:bc5cdr_disease_freq} reflects that the entity frequency distribution of {\ours} is more in line with the ground truth, having a relatively balanced distribution among all entities. This ensures that {\ours} generates synthetic data with a wide range of diverse topics.


\label{sec:diversity_measures}

\begin{figure}
	\centering
	\vspace{-2ex}
	\subfigure[t-SNE plot]{
		\includegraphics[width=0.26\linewidth]{figures/bc5cdr_disease_sentencebert_ours.pdf}
		\label{fig:bc5cdr_disease_sentencebert_ours}
	} %\hfill
         \hspace{-2ex}
	% \subfigure[MEDIQA-RQE]{
	% 	\includegraphics[width=0.25\linewidth]{figures/mediqa_rqe_sentencebert_bsl.pdf}
	% 	\label{fig:mediqa_rqe_sentencebert_bsl}
	% }\hspace{-1.5ex}
     \subfigure[Case study of generated examples]{
		\includegraphics[width=0.68\linewidth]{figures/case_study_llm.pdf}
		\label{fig:case_study_llm}
	}
	\caption{Data distribution and diversity measures on {\ours}. (a) is from BC5CDR-Disease and (b) is from MEDIQA-RQE using {\ours} with LLM. \vspace{-1.5ex}}
	\vspace{-1ex}
\label{fig:quality_ana1}
\end{figure}

 \begin{figure}
	\centering
	\vspace{-2ex}
	\subfigure[CMD]{
		\includegraphics[width=0.362\linewidth]{figures/cmd-all.pdf}
		\label{fig:cmd-all}
	} %\hfill
         \hspace{-1.5ex}
	% \subfigure[MEDIQA-RQE]{
	% 	\includegraphics[width=0.25\linewidth]{figures/mediqa_rqe_sentencebert_bsl.pdf}
	% 	\label{fig:mediqa_rqe_sentencebert_bsl}
	% }\hspace{-1.5ex}
     \subfigure[Entity Coverage]{
		\includegraphics[width=0.362\linewidth]{figures/avg-entity-all.pdf}
		\label{fig:avg-entity-all}
	}
 \hspace{-1.5ex}
      \subfigure[Entity Frequency]{
		\includegraphics[width=0.248\linewidth]{figures/bc5cdr_disease_freq.pdf}
		\label{fig:bc5cdr_disease_freq}
	}
	\caption{Data distribution and diversity measures on {\ours}. (c) is from BC5CDR-Disease.\vspace{-2ex}}
	\vspace{-1ex}
\label{fig:quality_ana2}
\end{figure}


% rare diseases

\textbf{Case Study.}
In Figure~\ref{fig:case_study_llm}, we present a case study of examples generated by {\ours} with LLM on MEDIQA-RQE dataset, which consists of consumer health queries. The showcased examples reveal that the sentences generated by {\ours} include more extensive contextual information compared with the baseline as shown in Figure~\ref{fig:case_study_prelim}. These sentences closely resemble the queries people might pose in real-life scenarios.
\section{Conclusion}
% Large language models (LLMs) inherently capture significant clinical knowledge. 
% In this work, we focus on an effective and generic approach for clinical text data generation with LLMs. 
% We thoroughly examine existing synthetic data generation methods for clinical tasks and identify significant issues, such as large distribution shifts and limited diversity in the generated data. 
% To address these challenges, we proposed a novel clinical knowledge-infuse framework, {\ours}, where clinical knowledge from both non-parametric KGs and parametric LLMs are leveraged to contextualize the clinical data generation. Specifically, clinical topic knowledge and real-world writing styles are elicited to serve as the foundation to create domain-specific prompts. 
% Through extensive empirical evaluations across 7 clinical NLP tasks spanning 16 datasets and comparing against 9 baseline methods from various categories, our results consistently demonstrate that {\ours}-generated data improves task performance, closely aligns with real data distributions, and significantly enhances data diversity compared to existing approaches.
% We anticipate this knowledge-infused clinical data generation paradigm be readily adapted for any future clinical text tasks, serving as a versatile approach for clinical NLP studies.  

In this work, we propose a versatile approach to clinical text data generation using LLMs. We thoroughly assess existing methods for clinical data generation and identify issues including distribution shifts and limited diversity. To tackle these challenges, we introduce {\ours}, a new framework that leverages clinical knowledge from non-parametric KGs and parametric LLMs. This knowledge empowers data generation by utilizing clinical topic knowledge and real-world writing styles in domain-specific prompts. Our extensive empirical evaluations across 7 clinical NLP tasks and 16 datasets, comparing to 9 baseline methods, consistently show that {\ours} improves task performance, aligns closely with real data, and enhances data diversity. We expect this approach can be seamlessly incorporated into a broad suite of clinical text tasks to advance clinical NLP research.


\bibliography{iclr2024_conference,dataset}
\bibliographystyle{iclr2024_conference}

\clearpage
\appendix
\section{Limitation, Future Works and Ethics Issues}
In this work, we propose {\ours} to better harness the LLM for synthetic text data generation. 
Despite the strong performance of {\ours} on 16 clinical NLP tasks, we mainly verify their efficacy from their empirical performance, sample diversity, and distribution gaps. 
However, there still exist gaps between the performance of the model $\cC_\theta$ fine-tuned using our generated synthetic data and ground-truth data. 
To further improve {\ours}, there are several avenues of future works:

\textbf{Using Clinical LLMs as Data Generator}:
Our method {\ours} relies on an LLM with instruction following ability. We mainly evaluate {\ours} using GPT-family models as the LLM.
Recently, there are many LLMs that have been fine-tuned on additional clinical contexts as well as instructions (e.g. Med-PALM\footnote{\url{https://sites.research.google/med-palm/}}), and achieved superior performance on challenging clinical NLP benchmarks.
However, they are not open-sourced, thus we cannot run them in our experiments. An interesting future work could be how to leverage these Clinical LLMs as Data Generator  to further boost the performance.

\textbf{Data Evaluation}:
In this work, we consider the distribution gap and sample diversity as our optimization objective. However, there might be many other aspects for synthetic quality estimation \citep{pmlr-v162-alaa22a}. 
We need more tools to capture, analyze, and improve this new aspect of data-centric AI.

% Additional Tasks and Methods

\textbf{Factuality}: One issue with LLM-based synthetic data generation is the phenomenon of \emph{hallucination}, wherein the model generates information that does not ground in reality~\citep{zhang2023siren}. This can lead to the propagation of misinformation, which may have negative impacts on the clinical domain.  
It is crucial to cross-verify the generated text with a reliable knowledge base or dataset. 
% Furthermore, incorporating an additional layer of human review can also help in mitigating hallucinations and ensuring the faithfulness of LLM-generated synthetic outputs~\citep{zhou2023context}. 


\section{Additional Preliminary Studies}
\label{sec:add_prelim}
We present additional preliminary studies of the t-SNE plots in Figure~\ref{fig:add_prelim_tsne} and the regularized entity frequencies in Figure~\ref{fig:add_prelim_freq}.
These results further justify the distribution shift issue mentioned in section \ref{sec:limitations}, demonstrating that the limited diversity as well as the distribution shift issue generally exists for a broad range of clinical NLP tasks.
\clearpage
 \begin{figure}[H]
	\centering
	% \vspace{-2ex}
	\subfigure[LitCovid]{
		\includegraphics[width=0.31\linewidth]{figures/litcovid_sentencebert_bsl.pdf}
	} %\hfill
         % \hspace{-1.5ex}
     \subfigure[GAD]{
		\includegraphics[width=0.31\linewidth]{figures/gad_sentencebert_bsl.pdf}
	}
        % \hspace{-1.5ex}
      \subfigure[CDR]{
		\includegraphics[width=0.31\linewidth]{figures/cdr_sentencebert_bsl.pdf}
	}
 	\subfigure[MEDIQA-RQE]{
		\includegraphics[width=0.31\linewidth]{figures/mediqa_rqe_sentencebert_bsl.pdf}
	} %\hfill
         % \hspace{-1.5ex}
     \subfigure[MQP]{
		\includegraphics[width=0.31\linewidth]{figures/mqp_sentencebert_bsl.pdf}
	}
        % \hspace{-1.5ex}
      \subfigure[CHEMDNER]{
		\includegraphics[width=0.31\linewidth]{figures/chemdner_sentencebert_bsl.pdf}
	}
	\caption{The t-SNE plots of datasets generated by ZeroGen and DemoGen compared with the ground truth.}
\label{fig:add_prelim_tsne}
\end{figure}

 \begin{figure}[H]
	\centering
	\vspace{-2ex}
	\subfigure[LitCovid]{
		\includegraphics[width=0.31\linewidth]{figures/litcovid_freq_bsl.pdf}
	} %\hfill
         % \hspace{-1.5ex}
     \subfigure[GAD]{
		\includegraphics[width=0.31\linewidth]{figures/gad_freq_bsl.pdf}
	}
        % \hspace{-1.5ex}
      \subfigure[CDR]{
		\includegraphics[width=0.31\linewidth]{figures/cdr_freq_bsl.pdf}
	}
 	\subfigure[MEDIQA-RQE]{
		\includegraphics[width=0.31\linewidth]{figures/mediqa_rqe_freq_bsl.pdf}
	} %\hfill
         % \hspace{-1.5ex}
     \subfigure[MQP]{
		\includegraphics[width=0.31\linewidth]{figures/mqp_freq_bsl.pdf}
	}
        % \hspace{-1.5ex}
      \subfigure[CHEMDNER]{
		\includegraphics[width=0.31\linewidth]{figures/chemdner_freq_bsl.pdf}
	}
	\caption{The regularized entity frequencies of datasets generated by ZeroGen and DemoGen compared with the ground truth in log scale.}
\label{fig:add_prelim_freq}
\end{figure}


\section{Implementation Details}
\label{sec:implementation_details}
For implementation, we use PyTorch~\citep{paszke2019pytorch} and HuggingFace~\citep{wolf2019huggingface}. For each dataset, we randomly sample 5 examples from each class to provide few-shot demonstrations and keep a validation set of the same size. In the experiments, We generate 5000 synthetic training data for both {\ours} and the baselines and report the average performance over 3 random seeds for all the results.

During the data generation process when we call the ChatGPT APIs~\citep{chatgpt}, we set the parameter $\operatorname{top\_p}=1.0$ and temperature $t=1.0$ to balance between the quality of the generated text as well as diversity~\citep{chung2023increasing,yu2023large}\footnote{We do not further increase $t$, as previous analysis  \citep{chung2023increasing,yu2023large} has shown that increasing $t$ to larger value does not help with additional performance gain.}. 
With the generated synthetic dataset, we follow the common few-shot learning setting~\citep{perez2021true} to train all the models for 6 epochs and use the model with the best performance on the validation set for evaluation.

During the PubMedBERT fine-tuning, we adopt AdamW~\citep{loshchilov2017decoupled} for optimization with a linear warmup of the first 5\% steps and linear learning rate decay. The learning rate is set to 2e-5 for \texttt{Base} and 4e-5 for \texttt{Large}, and the maximum number of tokens per sequence is 256. 

\section{Dataset Description}
\label{sec:dataset_description}

% \begin{table}[hbtp]
% \floatconts
%     {tab:datasets}
%     {\caption{Dataset statistics. For \# of hyperedges in MIMIC-III, the first number indicates the hyperedges without labels, while the second one indicates ones with labels.}}
%     {
%     \resizebox{0.8\linewidth}{!}{
%     \begin{tabular}{lcc}
%     \toprule
%     \bfseries Stats & \bfseries MIMIC-III    & \bfseries \dataset          \\ 
%     \midrule 
%     \# of diagnosis & 846 & 7915\\
%     \# of medication & 4525 & 489\\
%     \# of procedure & 2032 & 4321\\
%     \# of service & 20 & ---\\
%     \# of hyperedges & 36875/12353 & 36611 \\
%     \bottomrule
%     \end{tabular}
%     }
%     }
% \end{table}

\begin{table}[h]
% \floatconts
  \caption{Dataset statistics. We do not count the non-entity/non-relation class for relation extraction and token classification tasks to align with existing works. P and R stand for Precision and Recall. Metrics in \textbf{bold} are considered as the main metrics. $*$ is not allowed to put into GPT and $\dagger$ does not provide training data, so we sample few-shot examples from the SciTail~\citep{khot2018scitail} instead.}
  \label{tab:datastats}
  
  {
  \resizebox{0.95\linewidth}{!}{
  \begin{tabular}{lcccc}
  \toprule
  \bfseries Corpus & \bfseries Tasks & \bfseries \#Class & \bfseries \#Train/\#Test & \bfseries Metrics \\
  \midrule
  \multicolumn{5}{l}{\textbf{Single-Sentence Tasks}}\\
  \midrule
  LitCovid~\citep{litcovid} & Text Classification & 7 & 24960/6238 & \textbf{F1}\\
  HOC~\citep{hoc} & Text Classification & 10 & 3091/898 &\textbf{F1}\\
  GAD~\citep{gad} & Relation Extraction (RE) & 1 & 4750/350 & P, R, \textbf{F1}\\
  CDR~\citep{cdr_dataset} & Relation Extraction (RE) & 1 & 8431/2522 &  P, R, \textbf{F1}\\
  ChemProt~\citep{chemprot} & Relation Extraction (RE) & 5 & 8793/1087 & \textbf{F1}\\
  \midrule
  \multicolumn{5}{l}{\textbf{Sentence-Pair Tasks}}\\
  \midrule
  MedNLI$^*$~\citep{mednli} & Natural Language Inference (NLI) & 3 & 11232/1422 & \textbf{Acc}\\
  MEDIQA-NLI$^\dagger$~\citep{mediqa-nli} & Natural Language Inference (NLI) & 3 & -/405 & \textbf{Acc}\\
  MEDIQA-RQE~\citep{abacha2016recognizing} & Natural Language Inference (NLI) & 2 & 8588/302 & \textbf{Acc}\\
  PUBHEALTH~\citep{PUBHEALTH} & Fact Verification & 4 & 9804/1231 & Acc, \textbf{F1}\\
  HealthVer~\citep{healthver} & Fact Verification & 3 & 10591/1824 & Acc, \textbf{F1}\\
  MQP~\citep{mqp} & Sentences Similarity (STS) & 2 & 10/3033 & \textbf{Acc}\\
  \midrule
  \multicolumn{5}{l}{\textbf{Token Classification Tasks}}\\
  \midrule
  BC5CDR-Disease~\citep{bc5cdr} & Named Entity Recognition (NER) & 1 & 4882/5085 & P, R, \textbf{F1}\\
  BC5CDR-Chemical~\citep{bc5cdr} & Named Entity Recognition (NER) & 1 & 4882/5085 & P, R, \textbf{F1}\\
  NCBI-Disease~\citep{ncbi-disease} & Named Entity Recognition (NER) & 1 & 5336/921 &  P, R, \textbf{F1}\\
  CHEMDNER~\citep{chemdner} & Named Entity Recognition (NER) & 1 & 14522/12430 &  P, R, \textbf{F1}\\
  CASI~\citep{agrawal2022large,claim} & Attribute Extraction & 6 & 5/100 & \textbf{F1}\\
  \bottomrule
  \end{tabular}
     }
  }

\end{table}

% \begin{table}[hbtp]
% % \floatconts
%   {\caption{Dataset statistics. We do not count the non-entity/non-relation class for relation extraction and token classification tasks to align with existing works. P and R stand for Precision and Recall. Metrics in \textbf{bold} are considered as the main metrics.}\label{tab:datastats}}
  
%   {
%   \resizebox{0.8\linewidth}{!}{
%   \begin{tabular}{lcccc}
%   \toprule
%   \bfseries Corpus & \bfseries Tasks & \bfseries \#Class & \bfseries \#Train/\#Test & \bfseries Metrics \\
%   \midrule
%   \multicolumn{5}{c}{\textbf{Single-Sentence Tasks}}\\
%   \midrule
%   LitCovid & Text Classification & 7 & 24960/6238 & \textbf{F1}\\
%   HOC & Text Classification & 10 & 3091/898 &\textbf{F1}\\
%   GAD & Relation Extraction & 1 & 8431/2522 & P, R, \textbf{F1}\\
%   CDR & Relation Extraction & 1 & 8431/2522 &  P, R, \textbf{F1}\\
%   ChemProt & Relation Extraction & 5 & 8793/1087 & \textbf{F1}\\
%   \midrule
%   \multicolumn{5}{c}{\textbf{Sentence-Pair Tasks}}\\
%   \midrule
%   MedNLI* & NLI & 3 & 11232/1422 & \textbf{Acc}\\
%   MEDIQA-NLI & NLI & 3 & -/405 & \textbf{Acc}\\
%   MEDIQA-RQE & NLI & 2 & 8588/302 & \textbf{Acc}\\
%   PUBHEALTH & Fact Verification & 4 & 9804/1231 & Acc, \textbf{F1}\\
%   HealthVer & Fact Verification & 3 & 10591/1824 & Acc, \textbf{F1}\\
%   MQP & Sentences Similarity & 2 & 10/3033 & \textbf{Acc}\\
%   \midrule
%   \multicolumn{5}{c}{\textbf{Token Classification Tasks}}\\
%   \midrule
%   BC5CDR-Disease & NER & 1 & 4882/5085 & P, R, \textbf{F1}\\
%   BC5CDR-Chemical & NER & 1 & 4882/5085 & P, R, \textbf{F1}\\
%   NCBI-Disease & NER & 1 & 5336/921 &  P, R, \textbf{F1}\\
%   CHEMDNER & NER & 1 & 14522/12430 &  P, R, \textbf{F1}\\
%   CASI & Attribute Extraction & 6 & 5/100 & \textbf{F1}\\
%   \bottomrule
%   \end{tabular}
%      }
%   }

% \end{table}
The evaluation tasks and datasets are summarized in Table \ref{tab:datastats}. Note that the number of training samples indicates the size of the \textit{original} training set. Specifically, we consider the following datasets:

\begin{itemize}[leftmargin=0.3cm]

\item \textbf{Single-Sentence Tasks}
\begin{itemize}[label=$\circ$]
% \noindent$\bullet$ \textbf{Text Classification}: 
\item \uline{Text Classification}: 

\begin{itemize}
    \item The \textit{LitCovid} dataset~\citep{litcovid} consists of COVID-19-related publications from PubMed. The task is to predict the topics of the sentences, including ``Epidemic Forecasting", ``Treatment", ``Prevention", ``Mechanism", ``Case Report", ``Transmission", and ``Diagnosis".
    \item The \textit{HOC} dataset~\citep{hoc} also extracts sentences from PubMed articles, each annotated at the sentence level. The task is to predict the topics of the sentences, including ``evading growth suppressors", ``tumor promoting inflammation", ``enabling replicative immortality", ``cellular energetics", ``resisting cell death", ``activating invasion and metastasis", genomic instability and mutation", ``inducing angiogenesis", ``sustaining proliferative signaling", and ``avoiding immune destruction".
\end{itemize}

% Our evaluation aligns with prior research~\citep{blue}, focusing on the document-level F1-score for the multi-label text classification task.

% \noindent$\bullet$ \textbf{Relation Extraction}: 
\item \uline{Relation Extraction}:
\begin{itemize}
    \item The \textit{GAD}~\citep{gad} dataset is to predict whether there is a relation between the given disease and gene in the sentences. Note that the original annotation for this dataset is Noisy. To remedy this issue, we \emph{relabel} 350 examples from the original test set to form a clean subset for faithful evaluation.
    \item The \textit{CDR}~\citep{cdr_dataset} dataset is to predict whether the provided chemical can induce the disease in the sentences.
    \item The \textit{ChemProt}~\citep{chemprot} dataset focuses on the chemical-protein relations, and the labels include ``Upregulator", ``Downregulator", ``Agonist", ``Antagonist", ``Product\_of" and ``No relation".
\end{itemize}
% The relation extraction tasks involve identifying relationships and their associated types among entities within sentences. These tasks encompass the detection of gene-disease relation in the \textit{GAD}~\citep{gad} dataset, chemical-disease relation in the \textit{CDR}~\citep{cdr_dataset} dataset, and chemical-protein interactions in the \textit{ChemProt}~\citep{chemprot} dataset. 
% Performance evaluation hinges on a comparison of the predicted relation types against the annotated data, utilizing metrics precision, recall, and F1-score.
\end{itemize}

\item \textbf{Sentence-Pair Tasks}
\begin{itemize}[label=$\circ$]
% \noindent$\bullet$ \textbf{Natural Language Inference (NLI)}: 
\item \uline{Natural Language Inference (NLI)}: 

\begin{itemize}
    \item The \textit{MedNLI}~\citep{mednli} dateset consists of sentences pairs derived from MIMIC-III, where we predict the relations between the sentences. The labels include ``entailment", ``neutral" and ``contradiction".
    \item The \textit{MEDIQA-NLI}~\citep{mediqa-nli} dataset  comprises text-hypothesis pairs. Their relations include ``entailment", ``neutral" and ``contradiction".
    \item The \textit{MEDIQA-RQE}~\citep{abacha2016recognizing} dataset contains NIH consumer health question pairs, and the task is to recognize if the first question can entail the second one. 
\end{itemize}
% To identify inference relations between sentences, we leveraged three clinical NLI datasets: \textit{MedNLI*}~\citep{mednli} with 14,049 pairs derived from MIMIC-III, \textit{MEDIQA-NLI}~\citep{mediqa-nli} comprising 405 text-hypothesis pairs, and \textit{MEDIQA-RQE}~\citep{abacha2016recognizing} with 230 NIH consumer health question pairs for recognizing question entailment (RQE). 
% We evaluate NLI and RQE tasks using accuracy as our primary evaluation metric.

% \noindent$\bullet$ \textbf{Fact Verification}: 
\item \uline{Fact Verification}:
\begin{itemize}
    \item The \textit{PUBHEALTH}~\citep{PUBHEALTH} encompasses claims paired with journalist-crafted explanations. The task is to predict the relations between the claim and evidence, including ``Refute", ``Unproven", ``Support", and ``Mixture".
    \item The \textit{HealthVer}~\citep{healthver} contains evidence-claim pairs from search engine snippets regarding COVID-19 questions. The relations between claims and evidences are chosen from ``Refute", ``Unproven", and ``Support".
\end{itemize}
% In verifying claims against credible evidence, our investigation delves into two real-world evidence-based fact-checking datasets: \textit{PUBHEALTH}~\citep{PUBHEALTH}
% encompassing 11.8K claims paired with journalist-crafted explanations, and \textit{HealthVer}~\citep{healthver} containing 14,330 evidence-claim pairs from search engine snippets regarding COVID-19 questions.
% Our evaluation of the verification prediction task is based on accuracy and F1-score as the key performance metrics.

% \noindent$\bullet$ \textbf{Sentence Similarity (STS)}: 
\item \uline{Sentence Similarity (STS)}:
\begin{itemize}
    \item the \textit{MQP}~\citep{mqp} dataset comprises a collection of medical question pairs designed for identifying semantically similar questions. The task is to predict whether the two questions are equivalent or not.
\end{itemize}
% \textit{MQP}~\citep{mqp} dataset comprises a collection of 4,567 unique medical question pairs designed for identifying semantically similar questions. Our evaluation of model performance involves comparing the predicted similarity scores with the ground truth labels, and we report accuracy as the comparison metric.
\end{itemize}

\item \textbf{Token Classification Tasks}
\begin{itemize}[label=$\circ$]
% \noindent$\bullet$ \textbf{Named Entity Recognition (NER)}: 
\item \uline{Named Entity Recognition (NER)}:
\begin{itemize}
    \item The \textit{BC5CDR-Disease}~\citep{li2016biocreative} is to recognize diseases in the sentences.
    \item The \textit{BC5CDR-Chemical}~\citep{li2016biocreative} is to recognize chemicals in the sentences.
    \item The \textit{NCBI-Disease}~\citep{ncbi-disease} is to recognize diseases in the sentences.
    \item The \textit{CHEMDNER}~\citep{chemdner} is to recognize chemicals in the sentences.
\end{itemize}
% For recognizing and predicting entities (e.g., diseases, chemicals) from text, we examine three datasets commonly used for biomedical NER: \textit{BC5CDR}~\citep{li2016biocreative}, \textit{NCBI-Disease}~\citep{ncbi-disease}, \textit{CHEMDNER}~\citep{chemdner}. Our evaluation compares annotated mention spans in the documents to model predictions, utilizing precision, recall, and F1-score.

% \noindent$\bullet$ \textbf{Attribute Extraction (MedAttr)}:
\item \uline{Attribute Extraction (MedAttr)}:
\begin{itemize}
    \item The \textit{CASI} dataset~\citep{agrawal2022large,claim} aims to identify interventions including medication, dosage, route, freq, reason, duration
\end{itemize}
% For biomedical evidence extraction, we concentrate on identifying interventions from the manually re-annotated \textit{CASI} dataset~\citep{agrawal2022large,claim}, using the F1-score.
\end{itemize}
\end{itemize}

\section{Baseline Details}
\label{sec:baseline_details}
\textbf{Data Augmentation Methods:}
\begin{itemize}[leftmargin=0.5cm]
    \item \textbf{DA-Word Sub}: Word substitution, following Checklist~\citep{checklist}, maintains a word list to generate new examples by filling in a template. 
    \item \textbf{DA-Back Translation}: Following UDA~\citep{uda}, we employ back translation to augment the training data, including translating text from the target language to the source language and then back to the target language. 
    \item \textbf{DA-Mixup}~\citep{chen2020mixtext,seqmix}: We use the TMix version of MixText for data interpolation on the few-shot labeled dataset. For token-level classification tasks, we employ SeqMix~\citep{seqmix} that augments the queried samples by generating additional labeled sequences iteratively.
    \item \textbf{DA-Transformer (MELM)}~\citep{kumar2020data,melm}: It introduces a conditional data augmentation technique that prepends class labels to text sequences for pre-trained transformer-based models. For token-level classification tasks, it incorporates token labels into the sentence context and predicts masked entity tokens by explicitly conditioning on their labels.
    \item  \textbf{LightNER}~\citep{lightner}: It adopts a seq2seq framework, generating the entity span sequence and entity categories under the guidance of a self-attention-based prompting module. It is designed specifically for NER tasks.
    \item  \textbf{KGPC}~\citep{chen2023few}: Leveraging the semantic relations of the knowledge graph, it performs knowledge-guided instance generation for few-shot biomedical NER. It also only applies to NER tasks.
\end{itemize}


\textbf{LLM-based Generation Methods.} 
% In this study, we consider both zero-shot and few-shot learning data generation methods as baselines. 

% In zero-shot learning,
\begin{itemize}[leftmargin=0.5cm]
\item \textbf{ZeroGen}~\citep{ye2022zerogen}: It generates a dataset using a carefully designed instruction and then trains a tiny task-specific model for zero-shot inference. 
We follow the prompting method mentioned in their original paper as implementation, which \emph{does not consider} any style information as well as domain knowledge.
\item \textbf{DemoGen} ~\citep{meng2023tuning,gpt3mix}: It leverages LLMs to synthesize novel training data by  feeding few-shot samples as demonstrations. Note that we focus on using the black-box LLM as the generator, thus we do not tune the LLM as \citep{meng2023tuning}.
\item \textbf{ProGen}~\citep{ye2022progen}: It divides the entire dataset generation process into multiple phases. In each phase, feedback from the previously generated dataset guides the generation towards higher quality. It is also in the few-shot setting.
\end{itemize}

We do not compare with \cite{tang2023does} in the main experiments as it leverages entities extracted from the entire training set and violates the true few-shot learning setting.
% The comparison is listed in Appendix~\ref{}.



\section{Prompt Format}
\label{sec:prompt_format}
\subsection{The prompts for Writing Styles Suggestion with {\ours}}
\label{sec: style_prompt}

\begin{lstlisting}[style=mystyle, caption={Prompt Format for writing styles suggestion with {\ours}.}, label=lst:prompt, escapeinside={<@}{@>}]
Suppose you need to generate a synthetic clinical text dataset on <@\textcolor{blue}{[task]}@> tasks. Here are a few examples from the original training set:
<@\textcolor{blue}{[demonstrations]}@>
Please write three potential sources, speakers or authors of the sentences.

\end{lstlisting}

\texttt{[task]}: The task names for each specific task.
\texttt{[demonstrations]}: The few-shot demonstrations from the original training set.


\subsection{The prompts for Data Generation with {\ours}}
\label{sec:generation_prompt}
In the following prompt format, \texttt{[topic]} and \texttt{[style]} are randomly sampled from the topics candidate set and styles candidate set we formulate in the knowledge extraction step, respectively.

\textbf{Named entity recognition tasks:}

\begin{lstlisting}[style=mystyle, caption={Prompt Format for NER tasks with {\ours}.}, label=lst:prompt, escapeinside={<@}{@>}]
Suppose you need to create a dataset for <@\textcolor{blue}{[domain]}@> recognition. Your task is to:
1. generate a sentence about <@\textcolor{blue}{[domain]}@>,
2. output a list of named entity about <@\textcolor{blue}{[domain]}@> only,
3. the sentence should mimic the style of <@\textcolor{blue}{[style]}@>,
4. the sentence should mention the <@\textcolor{blue}{[domain]}@> named <@\textcolor{blue}{[topic]}@>.
\end{lstlisting}

\texttt{[domain]}: ``disease" for BC5CDR-Disease and NCBI-Disease; ``chemical" for BC5CDR-Chemical and CHEMDNER.

\textbf{Medication attributes tasks:}

\begin{lstlisting}[style=mystyle, caption={Prompt Format for medication attributes tasks with {\ours}.}, label=lst:prompt, escapeinside={<@}{@>}]
Suppose you need to create a dataset for clinical attributes recognition. Your task is to:
1. generate a sentence about clinical attributes, The Clinical Attributes you need to extract include "Medication", "Dosage", "Route", "Frequency", "Reason", "Duration". For each attribute class, please return a list of attributes within the class that occurs in the Sentence.
2. the sentence should mimic the style of <@\textcolor{blue}{[style]}@>,
3. the sentence should be relevant to <@\textcolor{blue}{[topic]}@>.
\end{lstlisting}


\textbf{Text classification tasks:}

\begin{lstlisting}[style=mystyle, caption={Prompt Format for text classification tasks with {\ours}.}, label=lst:prompt, escapeinside={<@}{@>}]
 Suppose you are a writer for <@\textcolor{blue}{[domain]}@>. Your task is to:
 1. give a synthetic <@\textcolor{blue}{[domain]}@> about <@\textcolor{blue}{[class\_name]}@>. 
 2. discuss about the subtopic of <@\textcolor{blue}{[topic]}@> for <@\textcolor{blue}{[class\_name]}@> in the <@\textcolor{blue}{[domain]}@>.
 3. the sentence should mimic the style of <@\textcolor{blue}{[style]}@>.
\end{lstlisting}
\texttt{[domain]}: ``COVID-19 Literature" for LitCovid and ``Cancer Document" for HOC.

\texttt{[class\_name]}: the label name for this generated sample.

\textbf{Relation extraction tasks:}

\begin{lstlisting}[style=mystyle, caption={Prompt Format for relation extraction tasks with {\ours}.}, label=lst:prompt, escapeinside={<@}{@>}]
Suppose you need to generate synthetic data for the biomedical <@\textcolor{blue}{[domain]}@> task. Your task is to:
1. give a sentence about <@\textcolor{blue}{[class\_name]}@> relation between <@\textcolor{blue}{[entity0]}@> and <@\textcolor{blue}{[entity1]}@>
2. the sentence should discuss the <@\textcolor{blue}{[entity0]}@>: <@\textcolor{blue}{[topic0]}@> and <@\textcolor{blue}{[entity1]}@>: <@\textcolor{blue}{[topic1]}@> with the relation <@\textcolor{blue}{[label\_desc]}@>.
3. the sentence should mimic the style of <@\textcolor{blue}{[style]}@>.
\end{lstlisting}
\texttt{[domain]}: ``Disease Gene Relation" for GAD, ``Chemical Disease Relation" for CDR, and ``Chemical Protein Relation" for ChemProt.

\texttt{[entity0]} and \texttt{[entity1]}: ``disease" and ``gene" for GAD, ``chemical" and ``disease: for CDR, and ``chemical" and ``protein" for ChemProt.

\texttt{[class\_name]}: the label name for this generated sample.

\texttt{[label\_desc]}: the description of the selected label. For example, the label ``upregulator" in ChemProt has a description of ``the chemical activates expression of the protein."

\textbf{Natural language inference tasks:}
\begin{lstlisting}[style=mystyle, caption={Prompt Format for generating the first sentence in NLI tasks with {\ours}.}, label=lst:prompt, escapeinside={<@}{@>}]
Suppose you need to create a set of <@\textcolor{blue}{[content]}@>. Your task is to:
1. generate one sentence for a <@\textcolor{blue}{[content]}@>.
2. the <@\textcolor{blue}{[content]}@> should be relevant to <@\textcolor{blue}{[topic]}@>,
3. The <@\textcolor{blue}{[content]}@> should mimic the style of <@\textcolor{blue}{[style]}@>.
\end{lstlisting}
\texttt{[content]}: ``health question" for MEDIQA-RQE, ``claim" for MEDIQA-NLI, MedNLI and MQP, and ``health news" for PUBHEALTH and HealthVer.


% \vspace{8ex}
\begin{lstlisting}[style=mystyle, caption={Prompt Format for generating the second sentence in NLI tasks with {\ours}.}, label=lst:prompt, escapeinside={<@}{@>}]
Suppose you need to create a pair of sentences for the <@\textcolor{blue}{[domain]}@> task with the label '<@\textcolor{blue}{[class\_name]}@>'. Given the <@\textcolor{blue}{[content]}@>: '<@\textcolor{blue}{[first\_sentence]}@>', Your task is to:
1. generate one short <@\textcolor{blue}{[content]}@> about <@\textcolor{blue}{[topic]}@> so that <@\textcolor{blue}{[label\_desc]}@>.
2. The <@\textcolor{blue}{[content]}@> should mimic the style of the first sentence.
\end{lstlisting}
\texttt{[domain]}: ``Question Entailment" for MEDIQA-RQE, ``Natural Language Entailment" for MEDIQA-NLI and MedNLI, ``Fact Verification" for PUBHEALTH and HealthVer, and ``Sentence Similarity Calculation" for MQP.

\texttt{[content]}: ``health question" for MEDIQA-RQE, ``hypothesis" for MEDIQA-NLI, MedNLI, ``evidence" for PUBHEALTH and HealthVer, and ``sentence" for MQP.

\texttt{[class\_name]}: the label name for this generated sample.

\texttt{[label\_desc]}: the description of the selected label.

\texttt{[first\_sentence]}: the first sentence we generate

\subsection{Prompts for ZeroGen, DemoGen, ProGen}
\label{sec:prompt_format_bsl}
We use the same set of prompts for ZeroGen, DemoGen and ProGen, while DemoGen and ProGen have additional demonstrations augmented to the prompts. DemoGen uses the few-shot examples in the training set as demonstrations, and ProGen leverages feedbacks from previous rounds to iteratively guide the generation.

\textbf{Named entity recognition tasks:}

\begin{lstlisting}[style=mystyle, caption={Prompt Format for NER tasks with baselines.}, label=lst:prompt, escapeinside={<@}{@>}]
Suppose you need to create a dataset for <@\textcolor{blue}{[domain]}@> recognition. Your task is to generate a sentence about <@\textcolor{blue}{[domain]}@> and output a list of named entity about <@\textcolor{blue}{[domain]}@> only.
\end{lstlisting}

\texttt{[domain]}: ``disease" for BC5CDR-Disease and NCBI-Disease; ``chemical" for BC5CDR-Chemical and CHEMDNER.

\textbf{Medication attributes tasks:}

\begin{lstlisting}[style=mystyle, caption={Prompt Format for medication attributes tasks with baselines.}, label=lst:prompt, escapeinside={<@}{@>}]
Suppose you need to create a dataset for clinical attributes recognition. Your task is to generate a sentence about clinical attributes, The Clinical Attributes you need to extract include "Medication", "Dosage", "Route", "Frequency", "Reason", "Duration". For each attribute class, please return a list of attributes within the class that occurs in the Sentence.
\end{lstlisting}


\textbf{Text classification tasks:}

\begin{lstlisting}[style=mystyle, caption={Prompt Format for text classification tasks with baselines.}, label=lst:prompt, escapeinside={<@}{@>}]
 Suppose you are a writer for <@\textcolor{blue}{[domain]}@>. Your task is to give a synthetic <@\textcolor{blue}{[domain]}@> about <@\textcolor{blue}{[class\_name]}@>. 
\end{lstlisting}
\texttt{[domain]}: ``COVID-19 Literature" for LitCovid and ``Cancer Document" for HOC.

\texttt{[class\_name]}: the label name for this generated sample.

\textbf{Relation extraction tasks:}

\begin{lstlisting}[style=mystyle, caption={Prompt Format for relation extraction tasks with baselines.}, label=lst:prompt, escapeinside={<@}{@>}]
Suppose you need to generate synthetic data for the biomedical <@\textcolor{blue}{[domain]}@> task. Your task is to give a sentence about <@\textcolor{blue}{[class\_name]}@> relation between <@\textcolor{blue}{[entity0]}@> and <@\textcolor{blue}{[entity1]}@> so that <@\textcolor{blue}{[label\_desc]}@>.
\end{lstlisting}
\texttt{[domain]}: ``Disease Gene Relation" for GAD, ``Chemical Disease Relation" for CDR, and ``Chemical Protein Relation" for ChemProt.

\texttt{[entity0]} and \texttt{[entity1]}: ``disease" and ``gene" for GAD, ``chemical" and ``disease: for CDR, and ``chemical" and ``protein" for ChemProt.

\texttt{[class\_name]}: the label name for this generated sample.

\texttt{[label\_desc]}: the description of the selected label. For example, the label ``upregulator" in ChemProt has a description of ``the chemical activates expression of the protein."

\textbf{Natural language inference tasks:}
\begin{lstlisting}[style=mystyle, caption={Prompt Format for generating the first sentence in NLI tasks with baselines.}, label=lst:prompt, escapeinside={<@}{@>}]
Suppose you need to create a set of <@\textcolor{blue}{[content]}@>. Your task is to generate one sentence for a <@\textcolor{blue}{[content]}@>.
\end{lstlisting}
\texttt{[content]}: ``health question" for MEDIQA-RQE, ``claim" for MEDIQA-NLI, MedNLI and MQP, and ``health news" for PUBHEALTH and HealthVer.


% \vspace{8ex}
\begin{lstlisting}[style=mystyle, caption={Prompt Format for generating the second sentence in NLI tasks with baselines.}, label=lst:prompt, escapeinside={<@}{@>}]
Suppose you need to create a pair of sentences for the <@\textcolor{blue}{[domain]}@> task with the label '<@\textcolor{blue}{[class\_name]}@>'. Given the <@\textcolor{blue}{[content]}@>: '<@\textcolor{blue}{[first\_sentence]}@>', Your task is to generate one short <@\textcolor{blue}{[content]}@> so that <@\textcolor{blue}{[label\_desc]}@>.
\end{lstlisting}
\texttt{[domain]}: ``Question Entailment" for MEDIQA-RQE, ``Natural Language Entailment" for MEDIQA-NLI and MedNLI, ``Fact Verification" for PUBHEALTH and HealthVer, and ``Sentence Similarity Calculation" for MQP.

\texttt{[content]}: ``health question" for MEDIQA-RQE, ``hypothesis" for MEDIQA-NLI, MedNLI, ``evidence" for PUBHEALTH and HealthVer, and ``sentence" for MQP.

\texttt{[class\_name]}: the label name for this generated sample.

\texttt{[label\_desc]}: the description of the selected label.

\texttt{[first\_sentence]}: the first sentence we generate

\section{Additional Experimental Results}
In this section, we present additional experimental results on every dataset in Tables~\ref{tab:single-sent}, ~\ref{tab:sent-pair}, ~\ref{tab:token-class}.
\label{sec:more_experimental_results}
\begin{table}[h]
% \floatconts
  \caption{Performance on single-sentence tasks evaluated by PubMedBERT$_{\texttt{Base}}$ and PubMedBERT$_{\texttt{Large}}$. \textbf{Bold} and \underline{underline} indicate the best and second best results for each dataset, respectively (Same as below).}
  \resizebox{\linewidth}{!}{
  \begin{tabular}{lccccccccc}
  \toprule
  & \bfseries LitCovid & \bfseries HOC & \multicolumn{3}{c}{\textbf{CDR}} & \multicolumn{3}{c}{\textbf{GAD}} & \bfseries ChemProt \\
  % \midrule
  \cmidrule(lr){2-2} \cmidrule(lr){3-3} \cmidrule(lr){4-6} \cmidrule(lr){7-9} \cmidrule(lr){10-10}
  & F1 & F1 & P & R & F1 & P & R & F1 & F1\\
  \midrule
  \multicolumn{10}{l}{\textbf{PubMedBERT$_{\texttt{Base}}$}} \\
  \midrule
  Supervised-Full & 71.70 & 82.32 & 67.81 & 76.60 & 71.96 & 82.55 & 85.10 & 83.81 & 76.24 \\
  Supervised-Few & 24.08 & 13.13 & 41.62 & 52.96 & 46.61 & 57.71 & 46.54 & 51.53 & 33.54 \\
  \midrule
  DA-Word Sub & 36.49 & 44.98 & 40.50 & 46.20 & 43.16 & 51.15 & 32.10 & 39.45 & 31.82 \\
  DA-Back Trans & 39.7 & 54.78 & --- & --- & --- & --- & --- & --- & --- \\
  DA-Mixup & 40.82 & 49.35 & 41.4 & 44.8 & 43.03 & 55.44 & 48.30 & 51.62 & 35.45 \\
  DA-Transformer & 39.86 & 42.18 & 44.6 & 61.7 & 51.77 & 59.4 & 46.5 & 52.16 & 38.73 \\ \midrule
  ZeroGen & 50.50 & 67.90 & 38.82 & \textbf{91.82} & 54.57 & 84.38 & 80.68 & 82.49 & 54.46 \\
  DemoGen & 57.65 & 70.52 & 46.9 & \underline{83.3} & 60.01 & 93.14 & 80.19 & 86.18 & 56.18 \\
  ProGen & \underline{58.06} & 72.25 & 51.35 & {71.58} & 59.80 & 90.52 & \textbf{85.14} & \underline{87.75} & 54.15 \\  
  \midrule
  % \hline
  \rowcolor{teal!10} {\ours} w/ KG & 58.01 & \underline{76.28} & \underline{56.98} & {67.38} & \underline{61.75} & \underline{93.33} & \underline{83.68} & \textbf{88.24} & \underline{57.04} \\
  \rowcolor{teal!10} {\ours} w/ LLM & \textbf{59.22} & \textbf{76.42} & \textbf{60.6} & 66.35 & \textbf{63.34} & \textbf{94.61} & 78.17 & 85.61 & \textbf{61.22} \\
  \midrule
  \multicolumn{10}{l}{\textbf{PubMedBERT$_{\texttt{Large}}$}} \\
  \midrule
  Supervised-Full & 74.59 & 85.53 & 72.31 & 74.88 & 73.57 & 84.95 & 88.75 & 86.81 & 78.55 \\
  Supervised-Few & 22.59 & 13.13 & 42.27 & 67.51 & 51.99 & 57.58 & 90.07 & 70.25 & 35.80 \\
  \midrule
  DA-Word Sub & 37.20 & 50.78 & 47.70 & 43.50 & 45.50 & 63.40 & 42.00 & 50.53 & 37.01 \\
  DA-Back Trans & 40.50 & 61.46 & --- & --- & --- & --- & --- & --- & --- \\
  DA-Mixup & 40.03 & 53.45 & 43.34 & 73.50 & 54.53 & 62.20 & 59.93 & 60.52 & 37.87 \\
  DA-Transformer & 38.95 & 49.86 & 50.70 & 31.60 & 38.93 & 59.80 & 57.76 & 58.76 & 40.66 \\
  \midrule
  ZeroGen & 52.86 & 70.16 & 42.95 & \textbf{80.67} & 56.06 & 92.26 & 76.73 & 83.78 & 55.71 \\
  DemoGen & \underline{56.29} & 73.65 & 50.86 & 74.30 & 60.39 & \textbf{96.85} & 76.83 & 85.69 & 59.88 \\
  ProGen & 54.71 & 75.31 & 50.36 & \underline{76.08} & 60.60 & 91.11 & 85.63 & 88.29 & 58.79 \\
  \midrule
  \rowcolor{teal!10} {\ours} w/ KG & 55.81 & \underline{77.71} & \underline{60.45} & 65.04 & \underline{62.66} & 94.30 & \textbf{89.08} & \textbf{91.62} & \underline{60.12} \\
  \rowcolor{teal!10} {\ours} w/ LLM & \textbf{57.07} & \textbf{78.14} & \textbf{67.13} & 62.98 & \textbf{64.99} & \underline{95.08} & \underline{86.14} & \underline{90.39} & \textbf{63.05} \\
  \bottomrule
  \end{tabular}
  }
  \label{tab:single-sent}
\end{table}
\begin{table}[h]
% \floatconts
  \caption{Performance on sentence-pair tasks evaluated by PubMedBERT$_{\texttt{Base}}$ and PubMedBERT$_{\texttt{Large}}$.}
  \resizebox{\linewidth}{!}{
  \begin{tabular}{lcccccccc}
  \toprule
  & \bfseries MEDIQA-RQE & \bfseries MEDIQA-NLI & \bfseries MedNLI & \multicolumn{2}{c}{\textbf{PUBHEALTH}} & \multicolumn{2}{c}{\textbf{HealthVer}} & \bfseries MQP \\
  % \midrule
  \cmidrule(lr){2-2} \cmidrule(lr){3-3} \cmidrule(lr){4-4} \cmidrule(lr){5-6} \cmidrule(lr){7-8} \cmidrule(lr){9-9}
  & ACC & ACC & ACC & ACC & F1 & ACC & F1 & ACC\\
  \midrule
  \multicolumn{9}{l}{\textbf{PubMedBERT$_{\texttt{Base}}$}} \\
  \midrule
  Supervised-Full & 77.15 & 79.01 & 81.43 & 65.16 & 62.96 & 70.00 & 68.02 & 75.70 \\
  Supervised-Few & 57.51 & 40 & 36.40 & 28.30 & 23.70 & 30.55 & 30.49 & 55.70 \\
  \midrule
  DA-Word Sub & 58.60 & 50.24 & 56.4 & 23.67 & 17.64 & 34.05 & 34.02 & 54.40 \\
  DA-Back Trans & 59.16 & 49.92 & 53.82 & 30.70 & 23.32 & 33.60 & 32.76 & 55.80 \\
  DA-Mixup & 57.71 & 49.38 & 53.47 & 31.45 & 24.45 & 34.11 & 33.78 & 58.20 \\
  \midrule
  ZeroGen & 63.28 & 52.89 & 57.71 & 35.80 & 31.50 & 34.80 & 33.50 & 68.35 \\
  DemoGen & 66.56 & 56.29 & 58.56 & 42.60 & 35.40 & 38.00 & 36.50 & 70.85 \\
  ProGen & 65.94 & 57.28 & 59.49 & 38.70 & 33.10 & 36.72 & 35.97 & 69.30 \\
  \midrule
  \rowcolor{teal!10} {\ours} w/ KG & \textbf{74.85} & \underline{58.03} & \underline{61.80} & \underline{44.60} & \underline{36.80} & \underline{43.05} & \underline{42.06} & \underline{72.17} \\
  \rowcolor{teal!10} {\ours} w/ LLM & \underline{72.40} & \textbf{64.44} & \textbf{64.89} & \textbf{48.50} & \textbf{40.60} & \textbf{44.50} & \textbf{42.32} & \textbf{73.31} \\
  \midrule
  \multicolumn{9}{l}{\textbf{PubMedBERT$_{\texttt{Large}}$}} \\
  \midrule
  Supervised-Full & 81.10 & 82.89 & 83.96 & 70.21 & 63.45 & 75.72 & 75.01 & 78.80 \\
  Supervised-Few & 63.79 & 47.40 & 38.80 & 46.20 & 27.20 & 35.60 & 33.80 & 59.73 \\
  \midrule
  DA-Word Sub & 64.26 & 51.20 & 57.53 & 35.60 & 31.60 & 35.41 & 32.29 & 55.30 \\
  DA-Back Trans & 65.52 & 51.43 & 58.21 & 34.45 & 30.50 & 33.78 & 32.21 & 56.40\\
  DA-Mixup & 64.10 & 50.91 & 57.03 & 34.23 & 30.78 & 33.79 & 31.42 & 58.50 \\
  \midrule
  ZeroGen & 67.26 & 60.74 & 62.42 & 42.50 & 33.30 & 39.74 & 38.90 & 72.69 \\
  DemoGen & 69.22 & 62.97 & 64.55 & 44.50 & 36.80 & 40.72 & 40.57 & 74.37 \\
  ProGen & 67.82 & 60.98 & 63.15 & 44.15 & 36.37 & 41.42 & 40.89 & 74.89 \\
  \midrule
  \rowcolor{teal!10} {\ours} w/ KG & \textbf{79.92} & \underline{63.59} & \underline{69.19} & \underline{50.20} & \underline{41.26} & \textbf{47.03} & \underline{43.64} & \underline{75.82} \\
  \rowcolor{teal!10} {\ours} w/ LLM & \underline{77.36} & \textbf{64.69} & \textbf{69.46} & \textbf{52.96} & \textbf{43.31} & \underline{46.05} & \textbf{44.12} & \textbf{76.21} \\
  \bottomrule
  \end{tabular}
  }
  \label{tab:sent-pair}
\end{table}
% \begin{table}[hbtp]
% \floatconts
  \caption{Performance on sentence-pair tasks evaluated by PubMedBERT$_{\texttt{Base}}$.}
  \resizebox{0.9\linewidth}{!}{
  \begin{tabular}{lcccccccc}
  \toprule
  & \bfseries MEDIQA-RQE & \bfseries MEDIQA-NLI & \bfseries MedNLI & \multicolumn{2}{c}{\textbf{PUBHEALTH}} & \multicolumn{2}{c}{\textbf{HealthVer}} & \bfseries MQP \\
  % \midrule
  \cmidrule(lr){2-2} \cmidrule(lr){3-3} \cmidrule(lr){4-4} \cmidrule(lr){5-6} \cmidrule(lr){7-8} \cmidrule(lr){9-9}
  & ACC & ACC & ACC & ACC & F1 & ACC & F1 & ACC\\
  \midrule
  Supervised-Full & 77.15 & 79.01 & 81.43 & 65.16 & 62.96 & 70.00 & 68.02 & 75.70 \\
  Supervised-Few & 57.51 & 40 & 36.40 & 28.30 & 23.70 & 30.55 & 30.49 & 55.70 \\
  \midrule
  DA-Word Sub & 58.60 & 50.24 & 56.4 & 23.67 & 17.64 & 34.05 & 34.02 & 54.40 \\
  DA-Back Trans & 59.16 & 49.92 & 53.82 & 30.70 & 23.32 & 33.60 & 32.76 & 55.80 \\
  DA-Mixup & 57.71 & 49.38 & 53.47 & 31.45 & 24.45 & 34.11 & 33.78 & 58.20 \\
  \midrule
  ZeroGen & 63.28 & 52.89 & 57.71 & 35.80 & 31.50 & 34.80 & 33.50 & 68.35 \\
  DemoGen & 66.56 & 56.29 & 58.56 & 42.60 & 35.40 & 38.00 & 36.50 & 70.85 \\
  ProGen & 65.94 & 57.28 & 59.49 & 38.70 & 33.10 & 36.72 & 35.97 & 69.30 \\
  \midrule
  \rowcolor{teal!10} {\ours} w/ KG & \textbf{74.85} & \underline{58.03} & \underline{61.80} & \underline{44.60} & \underline{36.80} & \underline{43.05} & \underline{42.06} & \underline{72.17} \\
  \rowcolor{teal!10} {\ours} w/ LLM & \underline{72.40} & \textbf{64.44} & \textbf{64.89} & \textbf{48.50} & \textbf{40.60} & \textbf{44.50} & \textbf{42.32} & \textbf{73.31} \\
  \bottomrule
  \end{tabular}
  }
  \label{tab:pair-base}
\end{table}
% \begin{table}[hbtp]
% \floatconts
  \caption{Performance on sentence-pair tasks evaluated by PubMedBERT$_{\texttt{Large}}$.}
  \resizebox{0.9\linewidth}{!}{
  \begin{tabular}{lcccccccc}
  \toprule
  & \bfseries MEDIQA-RQE & \bfseries MEDIQA-NLI & \bfseries MedNLI & \multicolumn{2}{c}{\textbf{PUBHEALTH}} & \multicolumn{2}{c}{\textbf{HealthVer}} & \bfseries MQP \\
  % \midrule
  \cmidrule(lr){2-2} \cmidrule(lr){3-3} \cmidrule(lr){4-4} \cmidrule(lr){5-6} \cmidrule(lr){7-8} \cmidrule(lr){9-9}
  & ACC & ACC & ACC & ACC & F1 & ACC & F1 & ACC\\
  \midrule
  Supervised-Full & 81.10 & 82.89 & 83.96 & 70.21 & 63.45 & 75.72 & 75.01 & 78.80 \\
  Supervised-Few & 63.79 & 47.40 & 38.80 & 46.20 & 27.20 & 35.60 & 33.80 & 59.73 \\
  \midrule
  DA-Word Sub & 64.26 & 51.20 & 57.53 & 35.60 & 31.60 & 35.41 & 32.29 & 55.30 \\
  DA-Back Trans & 65.52 & 51.43 & 58.21 & 34.45 & 30.50 & 33.78 & 32.21 & 56.40\\
  DA-Mixup & 64.10 & 50.91 & 57.03 & 34.23 & 30.78 & 33.79 & 31.42 & 58.50 \\
  \midrule
  ZeroGen & 67.26 & 60.74 & 62.42 & 42.50 & 33.30 & 39.74 & 38.90 & 72.69 \\
  DemoGen & 69.22 & 62.97 & 64.55 & 44.50 & 36.80 & 40.72 & 40.57 & 74.37 \\
  ProGen & 67.82 & 60.98 & 63.15 & 44.15 & 36.37 & 41.42 & 40.89 & 74.89 \\
  \midrule
  \rowcolor{teal!10} {\ours} w/ KG & 79.92 & 63.59 & 69.19 & 50.20 & 41.26 & 47.03 & 43.64 & 75.82 \\
  \rowcolor{teal!10} {\ours} w/ LLM & 77.36 & 64.69 & 69.46 & 52.96 & 43.31 & 46.05 & 44.12 & 76.21 \\
  \bottomrule
  \end{tabular}
  }
  \label{tab:pair-large}
\end{table}
\begin{table}[h]
% \floatconts
  \caption{Performance on token-classification tasks evaluated by PubMedBERT$_{\texttt{Base}}$ and PubMedBERT$_{\texttt{Large}}$.}
  \resizebox{\linewidth}{!}{
  \begin{tabular}{lccccccccccccccc}
  \toprule
  & \multicolumn{3}{c}{\textbf{BC5CDR-Disease}} & \multicolumn{3}{c}{\textbf{BC5CDR-Chemical}} & \multicolumn{3}{c}{\textbf{NCBI-Disease}} & \multicolumn{3}{c}{\textbf{CHEMDNER}} & \multicolumn{3}{c}{\textbf{CASI}} \\
  % \midrule
  \cmidrule(lr){2-4} \cmidrule(lr){5-7} \cmidrule(lr){8-10} \cmidrule(lr){11-13} \cmidrule(lr){14-16}
  & P & R & F1 & P & R & F1 & P & R & F1 & P & R & F1 & P & R & F1\\
  \midrule
  \multicolumn{16}{l}{\textbf{PubMedBERT$_{\texttt{Base}}$}} \\
  \midrule
  Full & 83.84 & 87.92 & 85.83 & 92.22 & 91.74 & 91.98 & 87.54 & 89.92 & 88.71 & 91.84 & 92.45 & 92.14 & --- & --- & --- \\
  Few & 24.86 & 39.47 & 30.51 & 63.73 & 46.07 & 53.48 & 36.16 & 39.47 & 37.74 & 48.00 & 28.70 & 35.92 & 38.11 & 43.82 & 40.77 \\
  \midrule
  DA-Word Sub & 35.34 & 39.54 & 37.32 & 63.13 & 52.52 & 57.34 & 53.40 & 36.70 & 43.50 & 47.45 & 33.15 & 39.03 & 40.25 & 47.65 & 43.64 \\
  DA-Mixup & 36.13 & 42.90 & 39.23 & 66.43 & 50.54 & 57.41 & 56.57 & 26.48 & 36.07 & 52.40 & 27.53 & 36.10 & 42.37 & 48.96 & 45.43 \\
  LightNER & 39.80 & 33.20 & 36.20 & --- & ---  & --- & 43.70 & 41.90 & 42.78 & --- & --- & --- & --- & --- & --- \\						
  DA-MELM & 34.20 & 41.30 & 37.42 & 47.23 & 72.81 & 57.29 & 36.90 & 48.50 & 41.91 & 39.33 & 45.95 & 42.38 & 37.82 & 44.28 & 40.80 \\
  KGPC & 50.80 & 51.30 & 51.05 & --- & --- & --- & 52.20 & 52.10 & 52.15 & --- & --- & --- & --- & --- & --- \\
  \midrule
  ZeroGen & 55.60 & 39.10 & 45.91 & 73.20 & 82.85 & 77.73 & 56.25 & 45.98 & 50.60 & \textbf{54.34} & 52.93 & 53.63 & 52.80 & 49.53 & 51.11 \\
  DemoGen & \underline{63.10} & 48.44 & 54.81 & 76.40 & 81.65 & 78.94 & 57.65 & 49.08 & 53.02 & \underline{54.00} & 53.77 & 53.88 & 58.15 & 56.84 & 57.49 \\
  ProGen & 61.60 & 50.5 & 55.50 & \underline{77.10} & 82.02 & 79.48 & 56.01 & \underline{53.50} & 54.73 & 51.55 & 53.00 & 52.26 & 57.76 & 58.57 & 58.16 \\
  \midrule
  \rowcolor{teal!10} {\ours} w/ KG & 58.64 & \textbf{63.02} & \underline{60.75} & 74.96 & \textbf{85.45} & \underline{79.86} & \textbf{62.62} & \textbf{56.62} & \textbf{59.47} & 48.33 & \textbf{69.28} & \textbf{56.94} & \textbf{71.75} & \underline{65.20} & \textbf{68.32} \\
  \rowcolor{teal!10} {\ours} w/ LLM & \textbf{63.41} & \underline{58.83} & \textbf{61.03} & \textbf{77.68} & \underline{84.33} & \textbf{80.87} & \underline{62.58} & 50.59 & \underline{55.95} & 51.40 & \underline{58.77} & \underline{54.84} & \underline{68.19} & \textbf{66.79} & \underline{67.48} \\
  \midrule
  \multicolumn{16}{l}{\textbf{PubMedBERT$_{\texttt{Large}}$}} \\
  \midrule
  Supervised-Full & 86.77 & 85.92 & 86.34 & 92.80 & 92.94 & 92.87 & 87.97 & 90.09 & 89.02 & 92.23 & 92.48 & 92.35 & --- & --- & --- \\
  Supervised-Few & 25.52 & 45.85 & 32.79 & 61.40 & 54.41 & 57.69 & 44.86 & 40.12 & 42.35 & 43.40 & 34.60 & 38.50 & 41.30 & 45.02 & 43.08 \\
  \midrule
  DA-Word Sub & 38.54 & 38.85 & 38.69 & 64.85 & 53.96 & 58.91 & 52.59 & 45.35 & 48.70 & 44.85 & 36.69 & 40.36 & 46.77 & 43.52 & 45.09 \\
  DA-Mixup  & 36.27 & 46.67 & 40.82 & 67.63 & 54.15 & 60.14 & 55.64 & 38.06 & 45.20 & 45.51 & 36.66 & 40.61 & 41.25 & 52.09 & 46.04 \\
  LightNER & --- & --- & --- & --- & --- & --- & --- & --- & --- & --- & --- & --- & --- & --- & --- \\
  DA-MELM  & 33.40 & 41.61 & 37.06 & 53.80 & 66.71 & 59.56 & 44.20 & 57.40 & 49.94 & 36.40 & 47.41 & 41.18 & 43.36 & 45.78 & 44.54 \\
  KGPC & --- & --- & --- & --- & --- & --- & --- & --- & --- & --- & --- & --- & --- & --- & --- \\
  \midrule
  ZeroGen & 57.40 & 39.21 & 46.59 & 78.08 & 80.97 & 79.49 & 54.52 & 49.00 & 51.61 & 48.56 & 59.44 & 53.45 & 54.04 & 51.40 & 52.69 \\
  DemoGen & 57.34 & 49.48 & 53.12 & \underline{78.27} & 83.90 & 80.99 & 59.43 & 56.83 & 58.10 & 48.03 & 60.39 & 53.51 & 62.67 & 61.02 & 61.83 \\
  ProGen & \underline{60.34} & 54.13 & 57.07 & \textbf{78.42} & 82.94 & 80.62 & 60.02 & 55.28 & 57.55 & \underline{50.40} & 59.64 & 54.63 & 57.21 & 63.70 & 60.28 \\
  \midrule
  \rowcolor{teal!10} {\ours} w/ KG & 54.28 & \textbf{70.14} & \underline{61.21} & 77.88 & \underline{86.32} & \underline{81.88} & \textbf{62.46} & \textbf{64.08} & \textbf{63.26} & 47.03 & \textbf{67.86} & \textbf{55.56} & \underline{70.96} & \textbf{69.66} & \textbf{70.30} \\
  \rowcolor{teal!10} {\ours} w/ LLM & \textbf{61.05} & \underline{65.40} & \textbf{63.15} & 78.08 & \textbf{86.98} & \textbf{82.29} & \underline{61.12} & \underline{60.16} & \underline{60.64} & \textbf{50.92} & \underline{60.67} & \underline{55.37} & \textbf{71.61} & \underline{66.86} & \underline{69.15} \\
  \bottomrule
  \end{tabular}
  }
  \label{tab:token-class}
\end{table}

\clearpage
\section{Additional Ablation and Parameter Studies}
\label{sec:add_ablation_para}

Figure~\ref{fig:generator-add} and \ref{fig:size-synthetic-add} show the effect of different generators and the effect of the proportion of data on two additional datasets, respectively. Overall, our method generally outperform the best baseline. One interesting finding for the NCBI-Disease dataset is that {\ours} performs worse than the best on one variant. We hypothesize that it is because this task involves more complex input and output, potentially posing a challenge for moderate-size LLMs to follow the instructions. 


Besides, as few-shot sample selection is important for the final performance, we show the performance of different 3 random seeds (with different seed examples/training process), and observe that our method {\ours} generally outperforms the baselines with non-negligible margins, which indicates the robustness of {\ours} as it does not rely on a specific subset of few-shot training examples to perform well. 

\begin{figure}[t!]
    \centering
    \begin{minipage}{0.48\textwidth}
        \centering
        \subfigure[CDR]{
            \includegraphics[width=0.48\textwidth]{figures/generator-CDR.pdf}
            \label{fig:generator-CDR}
        } \hspace{-3mm}
        \subfigure[NCBI-Disease]{
            \includegraphics[width=0.48\textwidth]{figures/generator-NCBI.pdf}
            \label{fig:generator-NCBI}
        }
        \vspace{-2ex}
        \RawCaption{\caption{Different generators at \texttt{Base}.}\label{fig:generator-add}}
    \end{minipage}%
    \begin{minipage}{0.48\textwidth}
        \centering
        \subfigure[CDR]{
            \includegraphics[width=0.5\textwidth]{figures/size-CDR.pdf}
            \label{fig:size-CDR}
        } \hspace{-6mm}
        \subfigure[NCBI-Disease]{
            \includegraphics[width=0.5\textwidth]{figures/size-NCBI.pdf}
            \label{fig:size-NCBI}
        }
        \vspace{-2ex}
        \RawCaption{\caption{Different proportion of data at \texttt{Base}.}\label{fig:size-synthetic-add}}
    \end{minipage}%
    \vspace{-0.5ex}
\end{figure}

\begin{table}[t]
% \floatconts
% \vspace{-1ex}
  \caption{Performance with Different Random Seeds using PubMedBERT$_{\texttt{Base}}$.}
  \resizebox{\linewidth}{!}{
  \begin{tabular}{l ccc | ccc | ccc| ccc}
  \toprule
  & \multicolumn{3}{c}{\textbf{HOC}} & \multicolumn{3}{c}{\textbf{CDR}} & \multicolumn{3}{c}{\textbf{MEDIQA-RQE}} & \multicolumn{3}{c}{\textbf{NCBI-Disease}}\\
  % \midrule
  \cmidrule(lr){2-4} \cmidrule(lr){5-7} \cmidrule(lr){8-10} \cmidrule(lr){11-13}
  & Best Baseline & {\ours}-KG& {\ours}-LLM & Best Baseline & {\ours}-KG& {\ours}-LLM& Best Baseline & {\ours}-KG& {\ours}-LLM & Best Baseline & {\ours}-KG& {\ours}-LLM\\
  \midrule
  1 &70.04 &	74.30 &	77.30	 &61.52 &	61.66 &	63.34	 & 68.30	 &76.85 &	74.50 &	56.12	 &60.22	 &54.51 \\
2 & 75.30	& 79.73& 	73.63	& 60.69& 	63.77	& 64.66	& 64.20	& 71.80	& 71.19	& 54.19& 	60.64& 	57.81 \\
3 & 71.41& 	74.81	& 78.33& 	57.82& 	59.79	& 62.02& 	67.18	& 75.90	& 71.51	& 53.85& 	57.52& 	55.50\\
  \bottomrule
  \end{tabular}
  }
  \label{tab:random_seed}
  \vspace{-1ex}
\end{table}


\section{Additional Quality Analysis}
We present additional quality analysis of the synthetic dataset with t-SNE plots in Figure~\ref{fig:add_quality_tsne} and the regularized entity frequencies in Figure~\ref{fig:add_quality_freq}.

 \begin{figure}[h]
	\centering
	\vspace{-2ex}
	\subfigure[LitCovid]{
		\includegraphics[width=0.31\linewidth]{figures/litcovid_sentencebert_ours.pdf}
	} %\hfill
         % \hspace{-1.5ex}
     \subfigure[GAD]{
		\includegraphics[width=0.31\linewidth]{figures/gad_sentencebert_ours.pdf}
	}
        % \hspace{-1.5ex}
      \subfigure[CDR]{
		\includegraphics[width=0.31\linewidth]{figures/cdr_sentencebert_ours.pdf}
	}
 	\subfigure[MEDIQA-RQE]{
		\includegraphics[width=0.31\linewidth]{figures/mediqa_rqe_sentencebert_ours.pdf}
	} %\hfill
         % \hspace{-1.5ex}
     \subfigure[MQP]{
		\includegraphics[width=0.31\linewidth]{figures/mqp_sentencebert_ours.pdf}
	}
        % \hspace{-1.5ex}
      \subfigure[CHEMDNER]{
		\includegraphics[width=0.31\linewidth]{figures/chemdner_sentencebert_ours.pdf}
	}
	\caption{The t-SNE plots of datasets generated by {\ours}, ZeroGen and DemoGen compared with the ground truth.}
\label{fig:add_quality_tsne}
\end{figure}

 \begin{figure}[h]
	\centering
	\vspace{-2ex}
	\subfigure[LitCovid]{
		\includegraphics[width=0.31\linewidth]{figures/litcovid_freq.pdf}
	} %\hfill
         % \hspace{-1.5ex}
     \subfigure[GAD]{
		\includegraphics[width=0.31\linewidth]{figures/gad_freq.pdf}
	}
        % \hspace{-1.5ex}
      \subfigure[CDR]{
		\includegraphics[width=0.31\linewidth]{figures/cdr_freq.pdf}
	}
 	\subfigure[MEDIQA-RQE]{
		\includegraphics[width=0.31\linewidth]{figures/mediqa_rqe_freq.pdf}
	} %\hfill
         % \hspace{-1.5ex}
     \subfigure[MQP]{
		\includegraphics[width=0.31\linewidth]{figures/mqp_freq.pdf}
	}
        % \hspace{-1.5ex}
      \subfigure[CHEMDNER]{
		\includegraphics[width=0.31\linewidth]{figures/chemdner_freq.pdf}
	}
	\caption{The regularized entity frequencies of datasets generated by {\ours}, ZeroGen and DemoGen compared with the ground truth in log scale.}
\label{fig:add_quality_freq}
\end{figure}


\section{Monetary Cost}
\label{sec:apd_cost}
We display the monetary cost of {\ours} for calling the OpenAI APIs, with a comparison with prompting GPT-3.5 for direct inference and DemoGen. From the values shown in Figure~\ref{tab:money_cost}, we observe that inference via GPT-3.5 generally has a higher cost, as it needs to input all the testing samples for prompting. In contrast, DemoGen has a relatively lower cost, because it does not include the topics and writing styles to the prompts as {\ours} does.

\begin{table}[h]
% \floatconts
  \caption{The average cost (in US dollars) of running {\ours} on various datasets per 1000 samples, compared with prompting GPT-3.5 for inference and DemoGen.}
  \resizebox{\linewidth}{!}{
  \begin{tabular}{lccccccc}
  \toprule
  & \bfseries HOC & \bfseries GAD & \bfseries ChemProt & \bfseries MEDIQA-RQE & \bfseries PUBHEALTH & \bfseries NCBI-Disease & \bfseries CASI\\
  % \midrule
  \midrule
  GPT-3.5 Inference & 1.09 & 1.05 & 5.75 & 2.15 & 2.80 & 0.90 & 1.30 \\ 
  % \hline
  DemoGen & 0.59 & 0.66 & 1.35 & 0.81 & 0.92 & 1.12 & 1.28 \\
  \rowcolor{teal!10} {\ours} w/ KG & 0.65 & 0.73 & 1.47 & 0.86 & 1.01 & 1.41 & 1.55 \\
  \rowcolor{teal!10} {\ours} w/ LLM & 0.72 & 0.84 & 1.51 & 0.90 & 1.34 & 1.49 & 1.62 \\
  \bottomrule
  \end{tabular}
  }
  \label{tab:money_cost}
  \vspace{-1ex}
\end{table}

\end{document}
