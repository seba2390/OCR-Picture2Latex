% !TEX root = DistrSeqCtrlAttacks_TASE19.tex

\section{Introduction}
\label{sec:intro}


Advanced  capabilities of smart Internet of Things (IoT) devices %affect their widespread adoption in numerous application fields.
have lead to their  widespread adoption in industrial automation system,
%Cyber-Physical Systems (CPS) and Internet of Things (IoT) technologies are
rapidly advancing reconfigurable manufacturing~\cite{jakovljevic_icamm17}; the rise of the fourth industrial revolution, known as Industry 4.0~\cite{I4.0}, introduces the new era of highly-customized (rather than highly-serialized) manufacturing~\cite{Hoda}. In this vision, manufacturing resources are highly modularized, providing the necessary flexibility to adapt to dynamical market demands. Efficient structural and functional changes are supported by Reconfigurable Manufacturing Systems (RMS) that can be configured \emph{ad-hoc} with little or zero downtime~\cite{Koren2018121}.

The foundation of RMS are modules controlled by smart Industrial IoT (IIoT)-enabled controllers. IIoT endpoints (sometimes referred to as \emph{industrial assets}) are heterogeneous by definition---they represent multi-vendor components whose deployment environment dynamically changes depending on the process needs and current configuration of RMS. Furthermore, a plethora of different communication technologies (wired and wireless) and protocols are employed~\cite{Boyes20181}. Seamless reconfiguration, integration and reliable functioning of RMS requires that components are highly autonomous. Specifically, they must be capable of seamlessly communicating with each other using compatible protocols (integrability), exchanging both low-level control-related and high-level process-bound information (interoperability) and to interact with each other in different ways to enable operation in a plethora of configurations (composability)~\cite{IIRA}.

Reconfigurability is naturally supported by distributed control architectures; conventionally centralized controllers are responsible of all aspects of control---from low-level event signaling, to high-level coordination. Their complexity hinders reconfigurability both from the hardware perspective (i.e., requiring component re-wiring), and the software aspect (i.e., having to ensure the control software is aware of and functions correctly under the new hardware configuration). Thus, the new generation of smart manufacturing resources must exploit not only functionally-required components (such as sensors and actuators) but also intrinsic computation and communication capabilities of IIoT-enabled controllers in order to enable a higher level of automation and autonomy. Control distribution enables decoupling of fine-grained details about \emph{how} control over the specific physical resource is performed, from the resources coordination problem which only needs to worry about \emph{what} the manufacturing resources are capable~of~performing.

%\PA{here, we are missing a short part why distributed systems are needed for RMS -- otherwise, why were we talking about RMS?}
%To address the diversity of technologies used in industrial networking, Industrial Internet Reference Architecture \cite{IIRA} proposes two-layer structuring of connectivity. First abstract layer refers to the Communication Transport Layer and provides technical interoperability between endpoints. The second layer is Connectivity Framework Layer; it provides syntactic interoperability and it is agnostic to the technologies implemented in the first layer.

However, the networked nature of the new generation of distributed automation systems makes them susceptible to network-based attacks~\cite{wang_arxiv19}, similar to security vulnerabilities reported in other cyber-physical systems domains (e.g.,~\cite{pajic_csm17}). For example, an adversary may inject false events~\cite{802.15.4auth}, delay or deny network access to legitimate controllers~\cite{wirelessattack}, or manipulate control commands~\cite{ackattack2} sent over unsecure communication channels. On the one hand, providing security guarantees is critical in distributed sequential control systems where progress is directly impacted by communication unavailability. Yet, despite devastating effects such attacks could have on operation of distributed industrial automation systems, existing approaches to securing such systems are somewhat ad-hoc; commonly, the benefits of included security mechanisms for control performance (i.e., Quality-of-Control---QoC) are unclear and hard to evaluate if no formal system analysis can be performed.
%
Consequently, to enable building of secure and correct-by-design RMS, in this work we introduce efficient techniques for systematic security analysis of distributed control applications deployed on IIoT-enabled local controllers (LCs). We also show how results of the security analysis can be used to improve automation performance and safety guarantees in the presence of attacks, by adding suitable security mechanisms that address the detected vulnerabilities. This results in the overall framework, shown  in Fig.~\ref{fig:methodology}, for formal safety analysis and patching of distributed sequential automation systems under adversarial influences.
%\paragraph{Related Work}
%Security analysis techniques for other IoT domains have been recently proposed.
%In~\cite{soteria}, smart home IoT applications are formally surveyed for anomalous behaviors. However, formal adversarial models and implications of security vulnerabilities on  system operation, such as QoC and safety in the presence of attacks, are not considered. Similarly, \cite{celikiotguard} introduces a dynamic policy-based enforcement system for securing against unauthorized and unwanted control scenarios, but focuses only on architectures and platforms for consumer IoT applications mostly in smart home automation. In~\cite{iotsat}, \todo{which conf/journal}  an SMT-based framework for IoT security analysis is presented; yet, only abstract threat models are used, and the software architecture of IoT nodes is masked by behavioral modeling.
%On the other hand, in the field of IIoT~\cite{jakovljevic_tcst19} presents formal techniques for distribution of existing centralized sequential automation designs towards deployment of legacy control applications over IIoT-enabled smart controllers, but without reliability or security considerations.

%
\begin{figure}[t]
	\centering
	\includegraphics[width=0.5\textwidth]{methodology.pdf}
	\caption{Our methodology for resilient IIoT-based distributed automation---in Phase 1, the composition of existing distributed control models, which are used to generate executable code for IIoT controllers, with channel and plant models is used to formally verify properties of interest in Phase 2. Finally, in Phase 3, the results of the security analysis are used to enhance system resiliency, by adding suitable  security mechanisms during code generation.}
	\label{fig:methodology}
\end{figure}


Coordination between components in a large fraction of IoT systems is based on discrete events. While a plethora of formal modeling frameworks is employed under the umbrella of IoT (e.g.,~\cite{iotsat,iotz3,iotautomata}), industrial automation systems are commonly based on GRAFCET (IEC~60848)/SFC (IEC~61131-3) control designs, and consequently on the underlying formal semantics of Control Interpreted Petri Nets (CIPN). Therefore, we focus on formal security analysis of IIoT-enabled controllers that are described using CIPNs; such controllers may be  developed directly, or automatically derived using methods for distribution of existing centralized sequential automation designs (e.g.,~\cite{jakovljevic_tcst19}), which allow for deployment of legacy control applications over IIoT-enabled smart controllers.


While inherent determinism of CIPNs is not a limitation when the formalism is used to specify controllers' behaviors, it prevents the use of CIPNs to model malicious actions~\cite{wang_arxiv19}. On the other hand, the sister formalism of Time Petri Nets (TPN) supports nondeterminism, which makes it a great candidate for security-aware modeling.
%Consequently, in this paper we consider
Thus, for the first phase of our security analysis, we introduce methods for
automatic transformation of domain-specific CIPN-based controller specifications (i.e., designs) into TPN-compliant representations. These TPN models enable closed-loop system modeling and analysis, by composing them with corresponding non-deterministic plant and security-aware communication channel models; we show how such security-aware models can be developed with the desired level of abstraction that allows us to capture impacts of attacks on automation performance. While our framework generally supports any communication channel implementation, we focus on the IEEE~802.15.4-based implementation featured in our evaluation setup.

In Phase 2, we employ open verification tools (e.g.,~\cite{romeo}) to perform system-wide verification of safety and QoC-relevant properties in the presence of attacks, based on the aforementioned security-aware closed-loop system model; it is important to highlight that we make no assumptions about the attacker's choice among all possible malicious actions nor the times when they (i.e., attack actions) may occur.
%
By enabling security analysis within the same family of formalisms (i.e., using a formalism that is closely related to the formalism used to design controllers),   we provide convenient domain-specific interpretation of analysis results. This allows us to exploit verification results in Phase 3 to orchestrate security patches in code generation which is performed based on original CIPN-based models.
%
Finally, we show the applicability of our methodology on a real-world industrial case study---a security analysis of an IIoT-enabled manipulator system. %Our framework for formal functional safety analysis of distributed sequential automation systems under adversarial influences is summarized in Fig~\ref{fig:methodology}.

{
Specifically, the contributions of this work are as follows:
\begin{itemize}
  \item Security-aware framework for verification of system-level properties for distributed discrete-event controllers (based on CIPNs) in the presence of network-based attacks;
  \item TPN-based non-deterministic modeling of network-based attacks on distributed controller communication, with emphasis on capturing impacts on automation performance;
  \item Extension of the control software development cycle from security-aware analysis to firmware patching, in order to ensure correct operation in the presence of attacks;
  \item Full-stack proof-of-concept case study based on industry-grade components demonstrating applicability of the developed secure automation framework.
\end{itemize}
}

%\LE{Loop to security patching missing}
%\PA{We then show how to introduce a security-aware communication channel model along with suitable plant models, and base them on Time Petri Nets (TPN); we transform the CIPN-based controller representation into a TPN-compliant model to obtain a system-level security-aware model of the cyber-physical system. We use the composition of the controller, plant, and channel models to show violation of relevant safety properties in the presence of attacks, and demonstrate how inclusion of security services affects modeling and verification.}
%\PA{On the other hand, standard formalism employed in control system design and control firmware generation based on CIPNs does not support aforementioned semantics necessary for design of resilient distributed automation.}
%

This paper is organized as follows. Sec.~\ref{sec:relatedWork} gives an overview of relevant related work, while Sec.~\ref{sec:motivation} provides the problem definition with emphasis on distributed IIoT-based automation. Sec.~\ref{sec:modeling} introduces TPN-based security-aware modelings and Sec.~\ref{subsec:channelAndCtrlChannelInteraction} derives the security-aware communication model. Specification and verification of relevant formal properties is presented in Sec.~\ref{sec:verification}, as well as the loop closure from verification to code generation to include security patches and improve system resiliency. Industrial case studies are discussed in Sec.~\ref{sec:evaluation}, before concluding remarks (Sec.~\ref{sec:conclusion}).

%In the rest of the section, we present automatic generation of TPN-based controller models, which can be used to reason about system resiliency, from existing CIPN-based models, which are in turn used to generate control~code.\todo{mention in the intro}


\section{Related Work}
\label{sec:relatedWork}
%\vspace{-2pt}
In~\cite{faruque_cpsattacks}, a model-based approach for simulating attacks on CPS is presented, but no formal verification is supported and experimental results are obtained based on specific attack implementations. In~\cite{faruque_survivability}, additional formal security assessment of industrial CPS controllers is performed, but analysis remains constrained to high-level vulnerabilities at the level of functional models. On the other hand, a comprehensive formal security analysis of wireless IoT communications under a specific attack model is presented in~\cite{wirelessIOTsecurity}, but no relations to implications for Quality-of-Control of the underlying physical process are considered.

Security analysis techniques for other IoT domains have also been recently proposed. In~\cite{soteria}, smart home IoT applications are formally surveyed for anomalous behaviors. However, formal adversarial models and implications of security vulnerabilities on  system operation, such as QoC and safety in the presence of attacks, are not considered. Similarly, \cite{celikiotguard} introduces a dynamic policy-based enforcement system for securing against unauthorized and unwanted control scenarios, but focuses only on architectures and platforms for consumer IoT applications mostly in smart home automation. In~\cite{iotsat}, an SMT-based framework for IoT security analysis is presented; yet, only abstract threat models are used, and the software architecture of IoT nodes is masked by behavioral modeling.

Note that Petri Nets (PNs) have been used for security-aware modeling and analysis.
%In the domain of security,
For example, penetration analysis using attack trees was formalized through %Petri nets (PN)
PNs (e.g.,~\cite{PNattackNets1,PNattackNets2}).
Coordinated cyber-physical attack modeling for smart grids was done in~\cite{PNgridAttacks}, but high-level attack scenarios were modeled, and the structure of the system components was coarsely abstracted. Modeling of risks and vulnerabilities towards avoidance and discovery for Unix-like software was performed (e.g.,~\cite{PNunixAttacks}) but without specifics of the underlying software architecture. \cite{SPNattacksCPS,SPNattacksGRID}~adopt stochastic PN-based attack models for CPS threats, while in~\cite{lesi_iotdi19}, authors formulate a framework for formal reliability analysis of networked IIoT sequential control applications based on CIPNs. On the other hand, \cite{petrinets_faults}~deals with fault detection in systems modeled by PNs. However, fault/failure models are limited to stochastic behaviors that cannot accurately capture adversarial actions (as described in~\cite{wang_arxiv19}). While cooperation and communication protocols were modeled with PNs (e.g.,~\cite{PNprotocols1,PNprotocols2})
, to the best of our knowledge, nondeterminism in Petri nets was not exploited for adversarial modeling.

