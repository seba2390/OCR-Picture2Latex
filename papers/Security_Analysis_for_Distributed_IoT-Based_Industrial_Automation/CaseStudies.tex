% !TEX root = DistrSeqCtrlAttacks_TASE19.tex


\section{Case Studies: Industrial Manipulators}
\label{sec:evaluation}
We consider a full physical implementation of a reconfigurable industrial pneumatic manipulator with a variable number of modules/degrees~of~freedom (DOF) controlled in a distributed fashion; i.e., one local controller per module/DOF. We demonstrate effectiveness of our framework on multiple module configurations (i.e., 2-DOF, 3-DOF).

%
\begin{figure}[t]
	\centering
	\includegraphics[width=0.49\textwidth]{caseStudies}
	\caption{Pneumatic manipulator in multiple configurations: (a) 2-DOF~pick \&~place configuration; (b) 3-DOF pick-immerse-shake-return configuration; (c,1) upper portion of the physical setup of the configuration (b) shows cylinders; (c,2) low-cost ARM Cortex-M3-based networked controller; each physical component (cylinders and the gripper) are equipped with one LC.}%; (d) parallel loading/unloading configuration.}%---parts are loaded by cylinder A, and unloaded by cylinder B simultaneously, while the rotary cylinder C `replaces' the loading/unloading positions by means of rotating the loading base.}
	\label{fig:caseStudies}
\end{figure}
%

\subsection{2-DOF Industrial Pneumatic Manipulator}
The pneumatic industrial manipulator in the 2-DOF configuration is depicted in Fig.~\ref{fig:caseStudies}(a); two double-acting cylinders (denoted $A$ and $B$) provide translational degrees of freedom, while the pneumatic gripper (denoted $C$) provides means of handling the workpiece. All actuation commands are issued by updating electrical signals \verb!xp!, $\verb!x!\in\{\verb!a!,\verb!b!,\verb!c!\}$ which activate monostable dual control pneumatic valves.\footnote{ A control valve is the interface between the controller and the pneumatic cylinder; it converts the actuation signal from the controller into mechanical movement that controls flow of pressured air towards pneumatic cylinders.} Notice that signals are denoted with \verb!x! while cylinders are denoted with $X$. Cylinders $A$ an $B$ are equipped with two proximity switches which allow position (i.e., fully retracted, fully extended) sensing. Signals corresponding to fully retracted (home) position are denoted \verb!x0!, while fully extended (end) position signals are denoted \verb!x1!. Additionally, the system contains a start switch whose corresponding signal is denoted by \verb!st!.


%
\begin{figure}[t]
	\centering
	\includegraphics[width=0.498\textwidth]{CIPNdistrExample.pdf}
	\caption{CIPN-based {distributed} controller of a 2-DOF pneumatic manipulator.}
	\label{fig:CIPNdistrCtrl}
\end{figure}
%



CIPN-based models of three LCs are shown in Fig.~\ref{fig:CIPNdistrCtrl}. Initially, cylinders $A$ and $B$ are fully retracted, and gripper $C$ released---in this state the manipulator is ready to begin its work cycle. The initial work cycle of the manipulator is started by pressing the start switch (\verb!st==1!), after which operation is fully automated. First, cylinder $B$ extends towards the workpiece picking position (due to actuation command \verb!bp=1!). Once cylinder $B$ reaches its end position (\verb!b1==1!), gripper $C$ is commanded gripping (\verb!cp=1!). Controller $B$ waits for $500~ms$ for the part to be gripped.\footnote{The gripper $C$ does not have end position sensing due to size constraints; thus a timed delay is used to permit secure gripping/releasing of the workpiece.}
Then, cylinder $B$ retracts (due to command \verb!bp=0!), and once it reaches home position (\verb!b0==1!), cylinder $A$ extends (due to command \verb!ap=1!). After reaching its end position (\verb!a1==1!), cylinder $B$ extends towards the placing position (due to command \verb!bp=1!). Once it reaches its end position (\verb!b1==1!), gripper $C$ is commanded release of the workpiece (command \verb!cp=0!). $500~ms$ later, cylinder B retracts (\verb!bp=0! followed by \verb!b0==1!), after which cylinder $A$ retracts (\verb!ap=0! followed by \verb!a0==1!). The manipulator returnees into its initial state, after which the next cycle is automatically executed. Signals (i.e., sensors outputs and actuator inputs) are allocated to LCs according to their physical proximity: $\{\verb!ap!,\verb!a0!,\verb!a1!\}$ are mapped to controller A (i.e., $LC_A$), $\{\verb!bp!,\verb!b0!,\verb!b1!,\verb!st!\}$ to controller B ($LC_B$), and $\{\verb!cp!\}$ to controller C ($LC_C$).

TPN models are obtained from these specifications as described in Section~\ref{sec:modeling}, but are omitted here due to their size. On the other hand, pneumatic cylinders are modeled as two-state plants with bounded, non-deterministic extending/retracting times obtained from experimental measurements. We extract timing parameters (i.e., bounds on time-to-transmit, time-to-acknowledge, and back-off timing) from experimental measurements---histograms for $10,000$ messages are shown in Fig.~\ref{fig:histograms}, for the employed low-cost ARM Cortex-M3-based controllers equipped with an IEEE~802.15.4-compliant transceiver. On the other hand, we obtain transceiver-related timings (e.g., back-off time during clear channel assessment) from the radio specifications~\cite{mrf}. While we verified a large number of safety and liveness properties for this setup, we~illustrate~verification and security patching on a more complex 3-DOF setup.

\begin{figure}[t]
	\includegraphics[width=0.49\textwidth,left]{nomPlot}\vspace{5pt}
	\includegraphics[width=0.49\textwidth,left]{injPlot}\vspace{5pt}
	\includegraphics[width=0.49\textwidth,left]{intPlot}
	\caption{Sensing/actuation signal timings for a nominal pick~\&~place run (a), a run where a signal injection is performed resulting in a dropped workpiece (b), and a run where progress is inhibited due to a DoS attack (c). Messages exchanged by LCs are marked with blue arrows. $X$ axis is unlabeled as the speed of the workcycle can be controlled by regulating air pressure in the system and is thus not crucial.}
	\label{fig:plots}
\end{figure}


\subsection{3-DOF Industrial Pneumatic Manipulator}
A 3-DOF configuration of the described manipulator is shown in Fig.~\ref{fig:caseStudies}(b). The additional rotational DOF, provided by cylinder C, introduces an additional LC and increases the complexity of the LC coordination. This configuration may be used to prepare workpieces for painting by immersing them into a pool with cleaning solution, and returning them to the pick-up position for further processing by another machine.

Fig.~\ref{fig:caseStudies}(c,1) shows the physical setup for  this configuration; the upper portion of the manipulator is shown such that cylinders are visible. Fig.~\ref{fig:caseStudies}(c,2) shows the low-cost ARM Cortex-M3-based LC with the corresponding IEEE~802.15.4 transceiver. While the models are more complex than in the 2-DOF case, they are semantically similar and thus omitted. Fig.~\ref{fig:plots}(a) shows event timing---i.e., states of all sensing and actuation signals, for one sample pick-immerse-shake-return run; messages exchanged by LCs are denoted with blue arrows originating at the source event and terminating at the triggered event. Among the many safety liveness and QoC properties, we illustrate verification on the following examples.
%
\begin{property}\label{prop:caseStudyP1}
Gripper D is always gripped before cylinder B picks the workpiece; formally captured as, \verb!AG(M(PdGRIP_Gripped)==1 and!\\\verb!M(PbCTRL_bCYL_Retract1)==1)!.
\end{property}
%
\begin{property}\label{prop:caseStudyP2}
A workpiece is eventually processed, once the work cycle is started. Formally, \verb!M(PbCTRL_bCyl_Extend1)==1-->!\\\verb!(M(PcCTRL_cGRIP_Release)==1!.
%Gripper D is never released while cylinder A is moving. Formally,\\
%\verb!AG(not(M(PdGRIP_Gripped)==0 and M(PaCTRL_aCYL_Extend)==1) or!\\\verb!(M(PdGRIP_Gripped)==0 and M(PaCTRL_aCYL_Retract)==1))!.
\end{property}
%

When no security mechanisms are employed, we verified violation of these properties.  Property~\ref{prop:caseStudyP1} is violated due to a possible impersonation attack at the gripper controller; an attacker may send the command to release the workpiece before it was returned to the return position. Fig.~\ref{fig:plots}(b) shows signal timings acquired on a sample cycle run in which the workpiece is dropped due to a maliciously injected command to release the gripper (potentially causing mechanical damage to the workpiece and/or the manipulator).

However, if transmissions are authenticated, and the model adjusted correspondingly as described in Section~\ref{subsubsec:authentication}, this vulnerability is alleviated. We applied a software security patch by including the \emph{mbed TLS} (Secure Sockets Layer) library that our IIoT controllers are fully compatible with. Signing a $128~bit$ message authentication code over one transmitted signal incurs computational overhead of $\sim100~\mu s$ on the employed low-cost ARM Cortex-M3-based LCs; this practically negligibly slows down manipulator's work cycle, while providing security guarantees. Hence, Property~\ref{prop:caseStudyP1} is satisfied following this security patch.

Property~\ref{prop:caseStudyP2} is violated due to the possibility of a DoS attack that infinitely delays progress. From the model's perspective, this attack does not cause a deadlock---while the \emph{physical} process is stalled, the \emph{cyber} process is in fact livelocked reattempting to access the channel (i.e., same places are revisited and same transitions fire infinitely often). Fig.~\ref{fig:plots}(c) shows signal timings acquired on a sample run where a DoS attack is launched by enabling carrier transmission on the attacker's transceiver, in order to jam messages after the workpiece was picked up from the immersion pool. As described in Section~\ref{subsubsec:DoSdetection}, wireless control nodes can keep track of unsuccessful medium access attempts, and promptly halt operation when a DoS attack is detected.
%
%As noted in Sec.~\ref{subsubsec:DoSdetection},
In such cases, distributing the information about DoS detection requires a secure channel, which we do not consider in this work.

%Note that while a jamming attack on a wireless medium is hard to mitigate without additional channel support, certain protocols designed for wired media support access control based on design-time reservations (e.g., Flexray), increasing the bar for the attacker's actions required to mount a successful DoS.



%\paragraph{Industrial Pneumatic Manipulator With Parallel Processes}
%Fig.~\ref{fig:caseStudies}(d) shows a configuration with parallel processes; i.e., workpieces are simultaneously loaded/unloaded by cylinders $A$ and $B$ while the two workpieces are ``replaced" by a rotating base attached to rotary cylinder $C$






