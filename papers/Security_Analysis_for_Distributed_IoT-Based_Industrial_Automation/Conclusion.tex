% !TEX root = DistrSeqCtrlAttacks_TASE19.tex

%\section{Related Work}
%\label{sec:relWork}


\section{Conclusion}
\label{sec:conclusion}
In this paper, we have developed a framework for security analysis of distributed sequential control systems captured as widely-adopted CIPN-based models. As CIPNs do not support verification of formal safety properties in the presence of attacks, we transform controller models into TPNs that inherently enable this verification by supporting non-deterministic timed transition as well as non-deterministic choice among transitions; this is crucial as it imposes minimal assumptions on adversarial actions. We have shown how a model of a network-based attacker can be integrated into the non-deterministic communication channel model, and have verified violation of safety properties in presence of attacks. 

Additionally, we have shown how results of verification can be used to pinpoint vulnerabilities in the control software implementation, and suggest security patches to alleviate impact of these vulnerabilities on control performance; we have also provided the loop back to the modeling stage enabling verification of the same safety properties that are now satisfied due to the use of appropriate security mechanisms. Finally, we have evaluated our framework on an industrial case study of a realistic scale and complexity. 