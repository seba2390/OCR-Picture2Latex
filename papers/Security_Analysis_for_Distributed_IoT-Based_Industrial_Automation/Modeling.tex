% !TEX root = DistrSeqCtrlAttacks_TASE19.tex


%\section{Security-Aware Modeling}
%\label{sec:modeling}
\section{TPN-Based  Automation Modeling} %Distributed
\label{sec:modeling}

TPNs extend PNs by introducing timed transitions; in a TPN, every transition is characterised by an interval $(\underline{t}_f,\overline{t}_f),[\underline{t}_f,\overline{t}_f),(\underline{t}_f,\overline{t}_f],$ or $[\underline{t}_f,\overline{t}_f]$, where $\underline{t}_f$ and $\overline{t}_f$ are the lower/upper bound on the transition firing times, %respectively,
which may be \emph{zero} or \emph{infinity}---time interval next to \emph{immediate} transitions (i.e., with zero firing time) is not specified.
Also, %$\underline{t}_f$ and $\overline{t}_f$
the firing times are defined relative to the moment of transition enabling, without any assumptions %are imposed
on their distribution. % of firing times.
This enables modeling of timed properties of real-time control software~\cite{TPNforRT1,TPNforRT2,TPNforRT3}.
In addition, by supporting non-determinism, TPNs facilitate attack modeling %of complex adversarial impact
that cannot be accurately done with deterministic or stochastic models. %, as we will show later in this section.

%{The state of a TPN can be represented as a pair $S=(M,I)$, where $M$ is the marking, and $I$ is the vector of time intervals corresponding to each of the enabled transitions, forming the \emph{firing domain}.}

%On the other hand, standard formalism employed in control system design and control firmware generation based on CIPNs does not support aforementioned semantics necessary for design of resilient distributed automation.
Therefore, we transform formal distributed control specifications expressed by $\mathbf{CIPN}_i$, $i=1,...,N$, into TPN-compatible models $\mathbf{TPN}_i^{ctrl}$, $i=1,...,N$. We then compose these models with plant models $\mathbf{TPN}_i^{plant}$, $i=1,...,N$, and security-aware communication channel models $\mathbf{TPN}_j^{channel}$, $j=1,...,M$,\footnote{This captures the general case where time or frequency multiplexing may be used to provide multiple communication channels over the same medium.} which enables us to reason about system-level safety and security properties under the modeled adversarial influences. Since both CIPNs and TPNs originate from PNs, the translation from $\mathbf{CIPN}_i$, $i=1,...,N$ controller models to $\mathbf{TPN}_i^{ctrl}$, $i=1,...,N$ is direct for all places and transitions except where platform-implemented API is called,  i.e., %constructs requiring special translation semantics are:
\begin{enumerate}
  \item Places handling actuation, and transitions handling sensing by issuing I/O API calls %(respectively
  (\verb!actuator=value! and \verb!sensor==value!, respectively),
  \item Places handling transmissions over a shared channel, and transitions handling receiving of communication signals %communicated
  %over the shared channel %by issuing communication
  using API calls (\verb!Send(destination,signal)! and \verb!signal==value!, respectively),
  \item Places calling other platform-dependent API, such as request for execution delays.
\end{enumerate}
%
These CIPN constructs, which directly rely on the underlying platform used to implement the controller , % runtime environment
must be explicitly modeled as nets that capture: 1)~interaction between LCs $\mathbf{TPN}_i^{ctrl}$ and the plant $\mathbf{TPN}_i^{plant}$, 2)~interaction between LCs $\mathbf{TPN}_i^{ctrl}$ and communication channel(s) $\mathbf{TPN}_j^{channel}$, and 3)~runtime environment %state
changes based on issued commands (e.g., variable updates, execution delays).

Thus, we introduce methods for automatic extraction of TPN-based controller models from existing CIPN-based models that may be used to generate control code. We capture interaction between the automation system and plant, as well as the platforms' runtime environment in Section~\ref{subsec:plantAndControllerInteraction}, %\todo{where is 2nd part done?}%and~\ref{sec:runtimemodel}),
and security-aware modeling of communication channels and their interfaces with controllers in Section~\ref{subsec:channelAndCtrlChannelInteraction}. These methods enable developing a complete closed-loop system model that can then be used to reason about system resiliency to~attacks.

\subsection{Modeling Plants and Controller-Plant Interaction}
\label{subsec:plantAndControllerInteraction}
Nominal behavior of the controlled physical system is typically known at control design time. Thus, since the formalism of CIPNs is universally adopted for automation design, we assume that a PN-based (i.e., $\mathbf{CIPN}$ or $\mathbf{TPN}$) plant model is available.\footnote{On the other hand, if an automata or other discrete-event system representation of the plant is available, existing tools and methodologies can be used to translate such models into a PN-based representation (e.g.,~\cite{PNtransformation1,PNtransformation2}).}
%We begin by describing development of a TPN-based plant model for
On the running example, we describe development of such TPN-based plant model, along with a TPN-compliant controller-plant interface implemented through marking-dependent \emph{guard functions}. %\todo{what is in figure 3.(c); there is nothing in caption; also, where do you use it in the text?}

%The rules for the

%\vspace{40pt}

%
\begin{figure}[t]
%\vspace{-14pt}
	\centering
    \includegraphics[width=0.46\textwidth]{ctrlPlantInteraction.pdf}
	\caption{Plant and plant-controller interaction modeling: (a) $\mathbf{TPN}$ model of the pick~\&~place station; % ($[\underline{t}_{p\&p}^{proc},\overline{t}_{p\&p}^{proc}]$/$[\underline{t}_{p\&p}^{ret},\overline{t}_{p\&p}^{ret}]$ denote the range of pick~\&~place/return times);
	(b) extended model of $LC_2$ from~Fig.~\ref{fig:exampleCIPN}(c) with TPN-compatible sensing/actuation -- the model is not a $\mathbf{TPN}$ as it still relies on the communication API (in {\color{red}red}) for interaction with other LCs; (c) model of incoming workpieces with a lower bound on the workpiece interarrival.}
	\label{fig:ctrlPlantInteraction}
\end{figure}

%\subsubsection{Plant Modeling}
Fig.~\ref{fig:ctrlPlantInteraction}(a) shows a $\mathbf{TPN}$ model of the pick~\&~place~station from Fig.~\ref{fig:exampleCIPN}; Fig.~\ref{fig:ctrlPlantInteraction}(c) models  the incoming workpieces arrivals, with a lower bound on the interarrival times.
 Place \verb!Pp&p_Init! represents the station's initial state. Token flow from this place is conditioned by the corresponding commands of $LC_2$ \verb!PP_Act==1! and \verb!PP_Act==2! from the controller model in Fig.~\ref{fig:exampleCIPN}(c). In the formalism of TPN, \emph{marking-dependent guard functions} can be used to restrict state changes (i.e., token flow) in the plant model, i.e., a marking-dependent function (denoted with \verb!M(!$\cdot$\verb!)!) assesses the state of the controller model (i.e., token distribution) and returns the current number of tokens inside the argument place. Hence, guard function \verb!G:M(Pctrl_P&P1)==1! (or \verb!G:M(Pctrl_P&P2)==1!  for the other conveyer belt) is associated with transition \verb!Tp&p_wfPick1! (or \verb!Tp&p_wfPick2!).
%
%
Once the pick~\&~place process is triggered (i.e., $LC_2$ initiates it from a specific conveyor belt) by the $LC_2$'s model advancing its token to \verb!Pctrl_P&P1! (or \verb!Pctrl_P&P2!), the station's token transitions to place \verb!Pp&p_P&P1! (or \verb!Pp&p_P&P2!) if conveyor belt 1 (or 2) has a workpiece waiting to be processed.
Note that in order to capture realistic executions, the actual times to complete the pick~\&~place and return processes are not deterministic; this is natively supported by TPN's timed transitions; transitions \verb!Tp&p_P&P1! / \verb!Tp&p_P&P2! have firing times from the interval $[\underline{t}_{p\&p}^{proc},\overline{t}_{p\&p}^{proc}]$, as shown in Fig.~\ref{fig:ctrlPlantInteraction}(a).\footnote{These intervals may be obtained experimentally under nominal conditions at runtime, or based on design constraints %(e.g., from first principles)
at design-time. Also, different durations of the picking process from different conveyors are supported but set equal in the above model/figure to simplify our presentation.} %Additionally, after the station completes the process, it signals this event to $LC_2$; this is formally modeled via a transition \emph{update function}, denoted as \verb!U:P&P_Complete=1! next to the transition.


Now, the station dwells in place \verb!Pp&p_wfRet! waiting for the signal from $LC_2$ to return to the home position, i.e., guard function \verb!G:M(Pctrl_Ret)==1! conditions transition \verb!Tp&p_wfRet! on the corresponding controller's state (i.e., where the return action is issued). Once return is commanded, the pick~\&~place station takes non-deterministic time from %the interval
$[\underline{t}_{p\&p}^{ret},\overline{t}_{p\&p}^{ret}]$ to return (i.e., transition over \verb!Tp&p_RetInit!). After this transition, % (i.e., returning to home position),
the station model is back in the initial state, waiting to process the next workpiece. %Finally, note that we do not denote a time interval next to \emph{immediate} transitions (i.e., transitions with zero firing time).


The actuation part of the plant-controller interface is managed by guard functions assessing the controller's marking; therefore, explicit actuation input updates (e.g., \verb!PP_Act=1!) in the CIPN places are omitted in the transformation to the TPN model as TPN places do not feature any attributes. This interface can also be achieved alternatively, by utilizing \emph{update functions} which are triggered on the firing of controller's transitions and can update markings or variables. The choice of the transformation semantics from CIPN to TPN can therefore be adjusted to the specific platform implementation.

%\subsubsection{Modeling Controller-Plant Interaction}
%\label{subsubsec:controllerPlantInteraction}

%The plant side of the controller-plant interaction is enforced through transitions in the plant model that are conditioned by guard functions as previously described. For instance, the transition \verb!Tp&p_wfRet! in the plant model is guarded by the Boolean function \verb!G:M(Pctrl_Ret)==1! which evaluates to \emph{true} if the controller's model has a token in place \verb!Pctrl_Ret! (i.e., station return to home is commanded).


%the TPN-based controller model must contain places whose number of tokens are assessed by the guard functions in the plant model. Therefore, places handling actuation through commands \verb!actuator=value! within the CIPN-based controller model are transformed into aforementioned places in the TPN-based controller model. In essence, \textbf{activation} of guard functions dependent on the controller's state that enables transitions in the plant model corresponds to the controller \textbf{sending} actuation commands to locally connected actuators.

Dually to the actuation part of the interaction, % of controller-plant interaction,
sensing is modeled by introducing plant-marking-dependent guard functions on controller's transitions. Specifically, transitions conditioned by sensor values in the form \verb!sensor==value! in the CIPN controller model are replaced with immediate transitions guarded by a Boolean function evaluating to \emph{true} if the plant model marking corresponds to the plant state where \verb!sensor==value! is satisfied, and to \emph{false} otherwise.

For instance, once $LC_2$ commands return of the pick~\&~place station, (i.e., controller model from Fig.~\ref{fig:exampleCIPN}(c) has the token in \verb!Pctrl_Ret!), it is blocked on the transition \verb!Tctrl_wfRet! which is guarded by the condition \verb!Ret_Complete==1!. This transition in the CIPN model is transformed into a transition in the TPN model in Fig.~\ref{fig:ctrlPlantInteraction}(b) that is guarded by a function dependent on the marking of the pick~\&~place station model from Fig.~\ref{fig:ctrlPlantInteraction}(a)~(i.e., controller waits for the station to reach home position). Guard function \verb!G:M(Pp&p_Init)==1! returns \emph{true} when the token in the plant model transitions from \verb!Pp&p_Ret! to \verb!Pp&p_Init! over the timed transition \verb!Tp&p_RetInit! (here, \verb!M(!$\cdot$\verb!)! denotes a function that returns the current number of tokens inside the argument place) -- hence, controller $LC_2$ can transition over \verb!Tctrl_wfRet!. Therefore, this guard function is used for the transition \verb!Tctrl_wfRet! in the TPN model of $LC_2$. This models the controller side of the controller-plant interaction, i.e., sensor sampling. More complex conditions based on multiple sensors are implemented by forming an arbitrary plant marking-dependent Boolean guard function.



Fig.~\ref{fig:ctrlPlantInteraction}(b) shows a controller model of %that implements
the described controller-plant interface. %  which is TPN-compliant.
The model, obtained from the CIPN-based model in~Fig.~\ref{fig:exampleCIPN}(c), is intermediary, and not fully TPN-compliant; the CIPN-based communication semantics (i.e., signal transmissions via \verb!Send(destination,signal)!, and receptions through \verb!signal==value! denoted in red in Fig.~\ref{fig:ctrlPlantInteraction}(b)) is still present in the model. {However, to allow for verification of properties for systems where networking is not a concern, this communication semantics can be easily adapted to TPNs by applying the same \emph{guard/update} functions as described; this results in a model architecture from Fig.~\ref{fig:modeling}(a).

Additionally, note that the conveyor monitor model from Fig.~\ref{fig:exampleCIPN}(b) can be directly transformed into a TPN, with guards \verb!Pres1==1! and \verb!Pres2==1! conditioning progress based on external inputs (i.e., the process) triggered by the process model shown in Fig.~\ref{fig:ctrlPlantInteraction}(c). This process model abstracts away the nature of the process of incoming workpieces on the conveyors with a minimum inter-arrival time (i.e., time in the interval $[t_{min}^{proc},\infty]$). %, without imposing any assumptions on the process iteself.

%\subsubsection{Controller Runtime Environment Modeling}
\subsubsection{Controller Runtime Environment Modeling}
\label{sec:runtimemodel}

Another challenge for the automatic mapping of CIPN-based control models into TPN-compliant models is mapping of places issuing system calls from the runtime environment (e.g., execution delays, requests for timer interrupts, setting counter events) or updating local controller state (e.g., manipulating global variables). Requested execution delays can easily be modeled as timed transitions with the exact firing times (i.e., where the lower and upper firing time bound are the same); in general, however, event timings with different semantics are available depending on the control implementation---i.e., GRAFCET or SFC. In~\cite{GRAFCETtoTPN,SFCtoTPN}, authors provide detailed translational semantics between CIPNs and TPNs in these cases by introducing \emph{event sequencers} as certain conditions exist where transitions can be taken while time to some events generated internally in places has still not elapsed.%\QEDE
%\end{remark}
\begin{remark}[Modeling more Complex Execution Environments]
While we consider single-threaded automation examples (as most sequential control implementations are), existing techniques for modeling parallel systems can be applied given the expressiveness of TPNs. For example, for multithreaded applications where task preemption is allowed, the operating system scheduler can be modeled as a separate component, even in case of multi-processor platforms~\cite{TPNscheduling}. %or the upper bound on the firing time of a specific transition adjusted to include worst-case blocking time incurred from other lower-priority workload~\cite{}.
%
%On the other hand, definition of global variables and functions is supported by state-of-the-art TPN tools (e.g.,~\cite{romeo}). Update functions that are invoked on firing of the transition to which the function is assigned and guard functions are generally not only marking- but also global variable-dependent, which allows for rich modeling of complex internal LC state updates, similarly as used to implement the controller-plant interface in Sec.~\ref{subsec:plantAndControllerInteraction}.
\QEDE
\end{remark}

\subsection{CIPN and TPN Controller Equivalence}
  An execution path in CIPNs can be defined as a sequence of markings, where a change in the marking occurs due to firing of a transition. Recall that places are associated with actions; hence, each marking is associated with a set of actions, while transitions are associated with guards---firing of each transition is thus conditioned by a set of conditions.\footnote{We employ the standard assumption that all inputs are re-evaluated after firing of every transition (e.g., as done in~\cite{SFCtoTPN}).} Therefore, an execution path is a sequence $M_0,\mathcal{T}_1,M_1,\mathcal{T}_2,...$, where $\mathcal{T}_{i}$ is the transition taking the net from marking $M_i$ to $M_{i+1}$. In the TPN model, a path is characterized by a similar sequence with the addition of transition timing.\footnote{Strictly speaking, two types of time intervals characterize each transition: static intervals (i.e., design-time bounds) when they may fire, and dynamic (i.e., runtime) intervals when they can fire at any given instant, conditioned by all other enabled transitions. However, for purposes of showing marking-based equivalence with CIPNs, time can be abstracted away.} In our case, it is sufficient to maintain the CIPN \emph{controller} execution paths in the TPN model, as our objective is operational equivalence of the source (centralized), and the target (distributed) control models.

{
CIPN controller specifications are fully deterministic by design, and have only \emph{immediate} transitions.\footnote{Immediate transitions become fireable immediately after enabling, i.e., without a time delay.}
Thus, the corresponding target TPN-based controller models obtained by directly constructing the same model in the TPN formalism without additional constructs (i.e., as previously described in Sec.~\ref{subsec:plantAndControllerInteraction}), %when composed
do not introduce behavior which is not covered by the source CIPN-based models. Consequently, execution paths of the composition of the TPN models match with that of the CIPN, from the input-output (i.e., sensing-actuation) perspective. In other words, no execution path is added when we transform the CIPN-based controller into the TPN-based representation~\cite{jakovljevic_tcst19}. Intuitively, the TPN models obtained by direct mapping from CIPN (i.e., place-by-place, and transition-by-transition), are still \emph{fully deterministic} (isolated from the intrinsically non-deterministic plant and channel); i.e., their behavior is identical to their CIPN counterparts, and same behavioral assumptions (e.g., 1-boundedness) hold~\cite{David20101}.
}

%However, given that the following are satisfied:
%\begin{itemize}
%  \item Controller TPN is considered in isolation from the plant (which is usually non-deterministic),
%  \item Controller TPN is obtained from a CIPN specification that is fully deterministic by design (and has only immediate transitions),
%  \item Labels can be associated with TPN places (like outputs with CIPN places) and transitions conditioned by the same inputs,
%\end{itemize}






%
\begin{figure}[t]
	\centering
    \includegraphics[width=0.45\textwidth]{modeling.pdf}
	\caption{Model architecture: (a) Model captures local controllers $LC_i$, plants $PLANT_i$ and their interactions, (b) Model also captures the employed communication transceivers $XCVR_i$, and the underlying communication channel $CH_i$. Note that local controller models  $LC_i$ in schemes (a) and (b) are not the same; i.e., in (b), controller places/transitions invoking communication APIs are made compatible with the transceiver model.}
	\label{fig:modeling}
\end{figure}

