
%% bare_jrnl.tex
%% V1.3
%% 2007/01/11
%% by Michael Shell
%% see http://www.michaelshell.org/
%% for current contact information.
%%
%% This is a skeleton file demonstrating the use of IEEEtran.cls
%% (requires IEEEtran.cls version 1.7 or later) with an IEEE journal paper.
%%
%% Support sites:
%% http://www.michaelshell.org/tex/ieeetran/
%% http://www.ctan.org/tex-archive/macros/latex/contrib/IEEEtran/
%% and
%% http://www.ieee.org/



% *** Authors should verify (and, if needed, correct) their LaTeX system  ***
% *** with the testflow diagnostic prior to trusting their LaTeX platform ***
% *** with production work. IEEE's font choices can trigger bugs that do  ***
% *** not appear when using other class files.                            ***
% The testflow support page is at:
% http://www.michaelshell.org/tex/testflow/


%%*************************************************************************
%% Legal Notice:
%% This code is offered as-is without any warranty either expressed or
%% implied; without even the implied warranty of MERCHANTABILITY or
%% FITNESS FOR A PARTICULAR PURPOSE!
%% User assumes all risk.
%% In no event shall IEEE or any contributor to this code be liable for
%% any damages or losses, including, but not limited to, incidental,
%% consequential, or any other damages, resulting from the use or misuse
%% of any information contained here.
%%
%% All comments are the opinions of their respective authors and are not
%% necessarily endorsed by the IEEE.
%%
%% This work is distributed under the LaTeX Project Public License (LPPL)
%% ( http://www.latex-project.org/ ) version 1.3, and may be freely used,
%% distributed and modified. A copy of the LPPL, version 1.3, is included
%% in the base LaTeX documentation of all distributions of LaTeX released
%% 2003/12/01 or later.
%% Retain all contribution notices and credits.
%% ** Modified files should be clearly indicated as such, including  **
%% ** renaming them and changing author support contact information. **
%%
%% File list of work: IEEEtran.cls, IEEEtran_HOWTO.pdf, bare_adv.tex,
%%                    bare_conf.tex, bare_jrnl.tex, bare_jrnl_compsoc.tex
%%*************************************************************************

% Note that the a4paper option is mainly intended so that authors in
% countries using A4 can easily print to A4 and see how their papers will
% look in print - the typesetting of the document will not typically be
% affected with changes in paper size (but the bottom and side margins will).
% Use the testflow package mentioned above to verify correct handling of
% both paper sizes by the user's LaTeX system.
%
% Also note that the "draftcls" or "draftclsnofoot", not "draft", option
% should be used if it is desired that the figures are to be displayed in
% draft mode.
%
\documentclass[journal]{IEEEtran}
%
% If IEEEtran.cls has not been installed into the LaTeX system files,
% manually specify the path to it like:
% \documentclass[journal]{../sty/IEEEtran}


% Vuk: Our lines here
\usepackage{graphicx}
\graphicspath{{Figures/}}

%\usepackage{listings}

\usepackage{fancyvrb} % For verbatim markup in footnotes
%\lstset{
%    breaklines=true,
%    breakatwhitespace=true,
%    fancyvrb=true,
%    morefvcmdparams=\PY 2,
%    basicstyle=\ttfamily,
%    postbreak=\mbox{\textcolor{red}{$\hookrightarrow$}\space},
%}

\usepackage{amssymb}
\newcommand*{\QEDF}{\hfill\ensuremath{\blacksquare}}%
\newcommand*{\QEDE}{\hfill\ensuremath{\square}}%


\newtheorem{problem}{Problem}
\newtheorem{remark}{Remark}
\newtheorem{example}{Example}
%\newtheorem*{example*}{Example}
\newtheorem{algorithm}{Algorithm}
\newtheorem{property}{Property}
\newtheorem{scenario}{Attack Scenario}

\newcommand{\LE}[1]{$\clubsuit$\footnote{VL: #1}}
\newcommand{\PA}[1]{$\spadesuit$\footnote{MP: #1}}
\newcommand{\ZJ}[1]{$\clubsuit$\footnote{ZJ: #1}}

\usepackage{tabularx}

% multi-row cells in tables
\usepackage{multirow}
%\multirow{number rows}{width}{text}
% multi-column cells in tables
%\multicolumn{number cols}{align}{text} % align: l,c,r

\usepackage[export]{adjustbox}

\usepackage{todonotes}


%\renewcommand{\labelenumi}{\alph{enumi})} % For letter-based enumeration
% Vuk: End our lines here


% Some very useful LaTeX packages include:
% (uncomment the ones you want to load)


% *** MISC UTILITY PACKAGES ***
%
%\usepackage{ifpdf}
% Heiko Oberdiek's ifpdf.sty is very useful if you need conditional
% compilation based on whether the output is pdf or dvi.
% usage:
% \ifpdf
%   % pdf code
% \else
%   % dvi code
% \fi
% The latest version of ifpdf.sty can be obtained from:
% http://www.ctan.org/tex-archive/macros/latex/contrib/oberdiek/
% Also, note that IEEEtran.cls V1.7 and later provides a builtin
% \ifCLASSINFOpdf conditional that works the same way.
% When switching from latex to pdflatex and vice-versa, the compiler may
% have to be run twice to clear warning/error messages.






% *** CITATION PACKAGES ***
%
%\usepackage{cite}
% cite.sty was written by Donald Arseneau
% V1.6 and later of IEEEtran pre-defines the format of the cite.sty package
% \cite{} output to follow that of IEEE. Loading the cite package will
% result in citation numbers being automatically sorted and properly
% "compressed/ranged". e.g., [1], [9], [2], [7], [5], [6] without using
% cite.sty will become [1], [2], [5]--[7], [9] using cite.sty. cite.sty's
% \cite will automatically add leading space, if needed. Use cite.sty's
% noadjust option (cite.sty V3.8 and later) if you want to turn this off.
% cite.sty is already installed on most LaTeX systems. Be sure and use
% version 4.0 (2003-05-27) and later if using hyperref.sty. cite.sty does
% not currently provide for hyperlinked citations.
% The latest version can be obtained at:
% http://www.ctan.org/tex-archive/macros/latex/contrib/cite/
% The documentation is contained in the cite.sty file itself.






% *** GRAPHICS RELATED PACKAGES ***
%
\ifCLASSINFOpdf
  % \usepackage[pdftex]{graphicx}
  % declare the path(s) where your graphic files are
  % \graphicspath{{../pdf/}{../jpeg/}}
  % and their extensions so you won't have to specify these with
  % every instance of \includegraphics
  % \DeclareGraphicsExtensions{.pdf,.jpeg,.png}
\else
  % or other class option (dvipsone, dvipdf, if not using dvips). graphicx
  % will default to the driver specified in the system graphics.cfg if no
  % driver is specified.
  % \usepackage[dvips]{graphicx}
  % declare the path(s) where your graphic files are
  % \graphicspath{{../eps/}}
  % and their extensions so you won't have to specify these with
  % every instance of \includegraphics
  % \DeclareGraphicsExtensions{.eps}
\fi
% graphicx was written by David Carlisle and Sebastian Rahtz. It is
% required if you want graphics, photos, etc. graphicx.sty is already
% installed on most LaTeX systems. The latest version and documentation can
% be obtained at:
% http://www.ctan.org/tex-archive/macros/latex/required/graphics/
% Another good source of documentation is "Using Imported Graphics in
% LaTeX2e" by Keith Reckdahl which can be found as epslatex.ps or
% epslatex.pdf at: http://www.ctan.org/tex-archive/info/
%
% latex, and pdflatex in dvi mode, support graphics in encapsulated
% postscript (.eps) format. pdflatex in pdf mode supports graphics
% in .pdf, .jpeg, .png and .mps (metapost) formats. Users should ensure
% that all non-photo figures use a vector format (.eps, .pdf, .mps) and
% not a bitmapped formats (.jpeg, .png). IEEE frowns on bitmapped formats
% which can result in "jaggedy"/blurry rendering of lines and letters as
% well as large increases in file sizes.
%
% You can find documentation about the pdfTeX application at:
% http://www.tug.org/applications/pdftex





% *** MATH PACKAGES ***
%
%\usepackage[cmex10]{amsmath}
% A popular package from the American Mathematical Society that provides
% many useful and powerful commands for dealing with mathematics. If using
% it, be sure to load this package with the cmex10 option to ensure that
% only type 1 fonts will utilized at all point sizes. Without this option,
% it is possible that some math symbols, particularly those within
% footnotes, will be rendered in bitmap form which will result in a
% document that can not be IEEE Xplore compliant!
%
% Also, note that the amsmath package sets \interdisplaylinepenalty to 10000
% thus preventing page breaks from occurring within multiline equations. Use:
%\interdisplaylinepenalty=2500
% after loading amsmath to restore such page breaks as IEEEtran.cls normally
% does. amsmath.sty is already installed on most LaTeX systems. The latest
% version and documentation can be obtained at:
% http://www.ctan.org/tex-archive/macros/latex/required/amslatex/math/





% *** SPECIALIZED LIST PACKAGES ***
%
%\usepackage{algorithmic}
% algorithmic.sty was written by Peter Williams and Rogerio Brito.
% This package provides an algorithmic environment fo describing algorithms.
% You can use the algorithmic environment in-text or within a figure
% environment to provide for a floating algorithm. Do NOT use the algorithm
% floating environment provided by algorithm.sty (by the same authors) or
% algorithm2e.sty (by Christophe Fiorio) as IEEE does not use dedicated
% algorithm float types and packages that provide these will not provide
% correct IEEE style captions. The latest version and documentation of
% algorithmic.sty can be obtained at:
% http://www.ctan.org/tex-archive/macros/latex/contrib/algorithms/
% There is also a support site at:
% http://algorithms.berlios.de/index.html
% Also of interest may be the (relatively newer and more customizable)
% algorithmicx.sty package by Szasz Janos:
% http://www.ctan.org/tex-archive/macros/latex/contrib/algorithmicx/




% *** ALIGNMENT PACKAGES ***
%
%\usepackage{array}
% Frank Mittelbach's and David Carlisle's array.sty patches and improves
% the standard LaTeX2e array and tabular environments to provide better
% appearance and additional user controls. As the default LaTeX2e table
% generation code is lacking to the point of almost being broken with
% respect to the quality of the end results, all users are strongly
% advised to use an enhanced (at the very least that provided by array.sty)
% set of table tools. array.sty is already installed on most systems. The
% latest version and documentation can be obtained at:
% http://www.ctan.org/tex-archive/macros/latex/required/tools/


%\usepackage{mdwmath}
%\usepackage{mdwtab}
% Also highly recommended is Mark Wooding's extremely powerful MDW tools,
% especially mdwmath.sty and mdwtab.sty which are used to format equations
% and tables, respectively. The MDWtools set is already installed on most
% LaTeX systems. The lastest version and documentation is available at:
% http://www.ctan.org/tex-archive/macros/latex/contrib/mdwtools/


% IEEEtran contains the IEEEeqnarray family of commands that can be used to
% generate multiline equations as well as matrices, tables, etc., of high
% quality.


%\usepackage{eqparbox}
% Also of notable interest is Scott Pakin's eqparbox package for creating
% (automatically sized) equal width boxes - aka "natural width parboxes".
% Available at:
% http://www.ctan.org/tex-archive/macros/latex/contrib/eqparbox/





% *** SUBFIGURE PACKAGES ***
%\usepackage[tight,footnotesize]{subfigure}
% subfigure.sty was written by Steven Douglas Cochran. This package makes it
% easy to put subfigures in your figures. e.g., "Figure 1a and 1b". For IEEE
% work, it is a good idea to load it with the tight package option to reduce
% the amount of white space around the subfigures. subfigure.sty is already
% installed on most LaTeX systems. The latest version and documentation can
% be obtained at:
% http://www.ctan.org/tex-archive/obsolete/macros/latex/contrib/subfigure/
% subfigure.sty has been superceeded by subfig.sty.



%\usepackage[caption=false]{caption}
%\usepackage[font=footnotesize]{subfig}
% subfig.sty, also written by Steven Douglas Cochran, is the modern
% replacement for subfigure.sty. However, subfig.sty requires and
% automatically loads Axel Sommerfeldt's caption.sty which will override
% IEEEtran.cls handling of captions and this will result in nonIEEE style
% figure/table captions. To prevent this problem, be sure and preload
% caption.sty with its "caption=false" package option. This is will preserve
% IEEEtran.cls handing of captions. Version 1.3 (2005/06/28) and later
% (recommended due to many improvements over 1.2) of subfig.sty supports
% the caption=false option directly:
%\usepackage[caption=false,font=footnotesize]{subfig}
%
% The latest version and documentation can be obtained at:
% http://www.ctan.org/tex-archive/macros/latex/contrib/subfig/
% The latest version and documentation of caption.sty can be obtained at:
% http://www.ctan.org/tex-archive/macros/latex/contrib/caption/




% *** FLOAT PACKAGES ***
%
%\usepackage{fixltx2e}
% fixltx2e, the successor to the earlier fix2col.sty, was written by
% Frank Mittelbach and David Carlisle. This package corrects a few problems
% in the LaTeX2e kernel, the most notable of which is that in current
% LaTeX2e releases, the ordering of single and double column floats is not
% guaranteed to be preserved. Thus, an unpatched LaTeX2e can allow a
% single column figure to be placed prior to an earlier double column
% figure. The latest version and documentation can be found at:
% http://www.ctan.org/tex-archive/macros/latex/base/



%\usepackage{stfloats}
% stfloats.sty was written by Sigitas Tolusis. This package gives LaTeX2e
% the ability to do double column floats at the bottom of the page as well
% as the top. (e.g., "\begin{figure*}[!b]" is not normally possible in
% LaTeX2e). It also provides a command:
%\fnbelowfloat
% to enable the placement of footnotes below bottom floats (the standard
% LaTeX2e kernel puts them above bottom floats). This is an invasive package
% which rewrites many portions of the LaTeX2e float routines. It may not work
% with other packages that modify the LaTeX2e float routines. The latest
% version and documentation can be obtained at:
% http://www.ctan.org/tex-archive/macros/latex/contrib/sttools/
% Documentation is contained in the stfloats.sty comments as well as in the
% presfull.pdf file. Do not use the stfloats baselinefloat ability as IEEE
% does not allow \baselineskip to stretch. Authors submitting work to the
% IEEE should note that IEEE rarely uses double column equations and
% that authors should try to avoid such use. Do not be tempted to use the
% cuted.sty or midfloat.sty packages (also by Sigitas Tolusis) as IEEE does
% not format its papers in such ways.


%\ifCLASSOPTIONcaptionsoff
%  \usepackage[nomarkers]{endfloat}
% \let\MYoriglatexcaption\caption
% \renewcommand{\caption}[2][\relax]{\MYoriglatexcaption[#2]{#2}}
%\fi
% endfloat.sty was written by James Darrell McCauley and Jeff Goldberg.
% This package may be useful when used in conjunction with IEEEtran.cls'
% captionsoff option. Some IEEE journals/societies require that submissions
% have lists of figures/tables at the end of the paper and that
% figures/tables without any captions are placed on a page by themselves at
% the end of the document. If needed, the draftcls IEEEtran class option or
% \CLASSINPUTbaselinestretch interface can be used to increase the line
% spacing as well. Be sure and use the nomarkers option of endfloat to
% prevent endfloat from "marking" where the figures would have been placed
% in the text. The two hack lines of code above are a slight modification of
% that suggested by in the endfloat docs (section 8.3.1) to ensure that
% the full captions always appear in the list of figures/tables - even if
% the user used the short optional argument of \caption[]{}.
% IEEE papers do not typically make use of \caption[]'s optional argument,
% so this should not be an issue. A similar trick can be used to disable
% captions of packages such as subfig.sty that lack options to turn off
% the subcaptions:
% For subfig.sty:
% \let\MYorigsubfloat\subfloat
% \renewcommand{\subfloat}[2][\relax]{\MYorigsubfloat[]{#2}}
% For subfigure.sty:
% \let\MYorigsubfigure\subfigure
% \renewcommand{\subfigure}[2][\relax]{\MYorigsubfigure[]{#2}}
% However, the above trick will not work if both optional arguments of
% the \subfloat/subfig command are used. Furthermore, there needs to be a
% description of each subfigure *somewhere* and endfloat does not add
% subfigure captions to its list of figures. Thus, the best approach is to
% avoid the use of subfigure captions (many IEEE journals avoid them anyway)
% and instead reference/explain all the subfigures within the main caption.
% The latest version of endfloat.sty and its documentation can obtained at:
% http://www.ctan.org/tex-archive/macros/latex/contrib/endfloat/
%
% The IEEEtran \ifCLASSOPTIONcaptionsoff conditional can also be used
% later in the document, say, to conditionally put the References on a
% page by themselves.





% *** PDF, URL AND HYPERLINK PACKAGES ***
%
%\usepackage{url}
% url.sty was written by Donald Arseneau. It provides better support for
% handling and breaking URLs. url.sty is already installed on most LaTeX
% systems. The latest version can be obtained at:
% http://www.ctan.org/tex-archive/macros/latex/contrib/misc/
% Read the url.sty source comments for usage information. Basically,
% \url{my_url_here}.





% *** Do not adjust lengths that control margins, column widths, etc. ***
% *** Do not use packages that alter fonts (such as pslatex).         ***
% There should be no need to do such things with IEEEtran.cls V1.6 and later.
% (Unless specifically asked to do so by the journal or conference you plan
% to submit to, of course. )


% correct bad hyphenation here
\hyphenation{op-tical net-works semi-conduc-tor}

%\linespread{.977}

\begin{document}
\VerbatimFootnotes
% paper title
% can use linebreaks \\ within to get better formatting as desired
\title{Security Analysis for Distributed IoT-Based Industrial Automation}%
%
% author names and IEEE memberships
% note positions of commas and nonbreaking spaces ( ~ ) LaTeX will not break
% a structure at a ~ so this keeps an author's name from being broken across
% two lines.
% use \thanks{} to gain access to the first footnote area
% a separate \thanks must be used for each paragraph as LaTeX2e's \thanks
% was not built to handle multiple paragraphs
%

\author{Vuk~Lesi,~\IEEEmembership{Student Member,~IEEE,}
        Zivana~Jakovljevic,~\IEEEmembership{Member,~IEEE,}
        and~Miroslav~Pajic,~\IEEEmembership{Senior Member,~IEEE}% <-this % stops a space
%\thanks{This work is based on research sponsored in part by the ONR under agreements N00014-17-1-2012 and N00014-17-1-2504, as well as the NSF CNS-1652544 grant. It was also partially supported by Serbian Ministry of Education, Science and Technological Development, research grants TR35004 and TR35020.}%
\thanks{V. Lesi and M. Pajic are with the Department
of Electrical and Computer Engineering, Duke University, Durham, NC, 27708 USA (email:
vuk.lesi@duke.edu; miroslav.pajic@duke.edu.}% <-this % stops a space
\thanks{Z. Jakovljevic is with University of Belgrade, Faculty of Mechanical
Engineering, Department for Production Engineering, 11000 Belgrade, Serbia (e-mail: zjakovljevic@mas.bg.ac.rs).}%
%\thanks{Manuscript received April 19, 2005; revised January 11, 2007.}
\thanks{Manuscript composed December 15, 2019.}
}

% note the % following the last \IEEEmembership and also \thanks -
% these prevent an unwanted space from occurring between the last author name
% and the end of the author line. i.e., if you had this:
%
% \author{....lastname \thanks{...} \thanks{...} }
%                     ^------------^------------^----Do not want these spaces!
%
% a space would be appended to the last name and could cause every name on that
% line to be shifted left slightly. This is one of those "LaTeX things". For
% instance, "\textbf{A} \textbf{B}" will typeset as "A B" not "AB". To get
% "AB" then you have to do: "\textbf{A}\textbf{B}"
% \thanks is no different in this regard, so shield the last } of each \thanks
% that ends a line with a % and do not let a space in before the next \thanks.
% Spaces after \IEEEmembership other than the last one are OK (and needed) as
% you are supposed to have spaces between the names. For what it is worth,
% this is a minor point as most people would not even notice if the said evil
% space somehow managed to creep in.



% The paper headers
%\markboth{Journal of \LaTeX\ Class Files,~Vol.~6, No.~1, January~2007}%
%{Shell \MakeLowercase{\textit{et al.}}: Bare Demo of IEEEtran.cls for Journals}



%\markboth{IEEE Transactions on Automation Science and Engineering}%
%{Shell \MakeLowercase{\textit{et al.}}: Bare Demo of IEEEtran.cls for Journals}


% The only time the second header will appear is for the odd numbered pages
% after the title page when using the twoside option.
%
% *** Note that you probably will NOT want to include the author's ***
% *** name in the headers of peer review papers.                   ***
% You can use \ifCLASSOPTIONpeerreview for conditional compilation here if
% you desire.




% If you want to put a publisher's ID mark on the page you can do it like
% this:
%\IEEEpubid{0000--0000/00\$00.00~\copyright~2007 IEEE}
% Remember, if you use this you must call \IEEEpubidadjcol in the second
% column for its text to clear the IEEEpubid mark.



% use for special paper notices
%\IEEEspecialpapernotice{(Invited Paper)}




% make the title area
\maketitle


\begin{abstract}
%\boldmath
With ever-expanding computation and communication capabilities of modern embedded platforms, Internet of Things (IoT) technologies enable development of Reconfigurable Manufacturing Systems---a new generation of highly modularized industrial equipment suitable for highly-customized manufacturing. Sequential control in these systems is largely based on discrete events, while their formal execution semantics is specified as Control Interpreted Petri Nets (CIPN). Despite industry-wide use of programming languages based on the CIPN formalism, formal verification of such control applications in the presence of adversarial activity is not supported.
%
Consequently, in this paper we focus on security-aware modeling and verification challenges for CIPN-based sequential control applications. Specifically, we show how CIPN models of networked industrial IoT controllers can be transformed into Time Petri Net (TPN)-based models, and composed with plant and security-aware channel models in order to enable system-level verification of safety properties in the presence of network-based attacks. Additionally, we introduce realistic channel-specific attack models that capture adversarial behavior using nondeterminism. Moreover, we show how verification results can be utilized to introduce security patches and motivate design of attack detectors that improve overall system resiliency, and allow satisfaction of critical safety properties. Finally, we evaluate our framework on an industrial case study.\end{abstract}
% IEEEtran.cls defaults to using nonbold math in the Abstract.
% This preserves the distinction between vectors and scalars. However,
% if the journal you are submitting to favors bold math in the abstract,
% then you can use LaTeX's standard command \boldmath at the very start
% of the abstract to achieve this. Many IEEE journals frown on math
% in the abstract anyway.

\renewcommand\abstractname{Note to Practitioners}
\begin{abstract}
Our main goal is to provide formal security guarantees for distributed sequential controllers. Specifically, we target smart automation controllers geared towards Industrial IoT applications, that are typically programmed in C/C++, and are running applications originally designed in e.g., GRAFCET (\mbox{IEC 60848})/SFC (\mbox{IEC 61131-3}) automation programming languages. Since existing tools for design of distributed automation do not support system-level verification of relevant safety properties, %in this work,
we show how security-aware transceiver and communication models can be developed and composed with distributed controller models. Then, we show how existing tools for verification of Time Petri Nets can be used to verify relevant properties including safety and liveness of the distributed automation system in the presence of network-based attacks. To provide an end-to-end analysis as well as security patching, results of our analysis can be used to deploy suitable firmware updates during the stage when executable code for target %embedded
controllers (e.g., in C/C++) is generated based on GRAFCET/SFC control models. We also show that security guarantees can be improved as the relevant safety/liveness properties can be verified after corresponding security patches are deployed. %While
Finally, we~show applicability of our methodology on a realistic distributed pneumatic manipulator. % in an academic setup, in the future we would strive to evaluate scalability of our security analysis framework on significantly larger setups.

%"Note to Practitioners" goes here.  For format and style, please see:
%http://www.ieee-ras.org/tase/ntp
\end{abstract}

%Primary and Secondary Keywords
%\begin{IEEEkeywords}
%Primary Topic Number 1, Primary Topic Number 2, ...,
%Secondary Topic Keywords: ...
%See list
%\end{IEEEkeywords}


\renewcommand\abstractname{Primary and Secondary Keywords}
\begin{abstract}
Primary Topics: Sequential control systems, Secure distributed automation, Industrial Internet of Things;
Secondary Topics: Petri nets, Non-deterministic analysis
\end{abstract}




% For peer review papers, you can put extra information on the cover
% page as needed:
% \ifCLASSOPTIONpeerreview
% \begin{center} \bfseries EDICS Category: 3-BBND \end{center}
% \fi
%
% For peerreview papers, this IEEEtran command inserts a page break and
% creates the second title. It will be ignored for other modes.
\IEEEpeerreviewmaketitle



%\section{Introduction}
%% The very first letter is a 2 line initial drop letter followed
%% by the rest of the first word in caps.
%%
%% form to use if the first word consists of a single letter:
%% \IEEEPARstart{A}{demo} file is ....
%%
%% form to use if you need the single drop letter followed by
%% normal text (unknown if ever used by IEEE):
%% \IEEEPARstart{A}{}demo file is ....
%%
%% Some journals put the first two words in caps:
%% \IEEEPARstart{T}{his demo} file is ....
%%
%% Here we have the typical use of a "T" for an initial drop letter
%% and "HIS" in caps to complete the first word.
%\IEEEPARstart{T}{his} demo file is intended to serve as a ``starter file''
%for IEEE journal papers produced under \LaTeX\ using
%IEEEtran.cls version 1.7 and later.
%% You must have at least 2 lines in the paragraph with the drop letter
%% (should never be an issue)
%I wish you the best of success.
%
%\hfill mds
%
%\hfill January 11, 2007
%
%\subsection{Subsection Heading Here}
%Subsection text here.
%
%% needed in second column of first page if using \IEEEpubid
%%\IEEEpubidadjcol
%
%\subsubsection{Subsubsection Heading Here}
%Subsubsection text here.
%
%
%% An example of a floating figure using the graphicx package.
%% Note that \label must occur AFTER (or within) \caption.
%% For figures, \caption should occur after the \includegraphics.
%% Note that IEEEtran v1.7 and later has special internal code that
%% is designed to preserve the operation of \label within \caption
%% even when the captionsoff option is in effect. However, because
%% of issues like this, it may be the safest practice to put all your
%% \label just after \caption rather than within \caption{}.
%%
%% Reminder: the "draftcls" or "draftclsnofoot", not "draft", class
%% option should be used if it is desired that the figures are to be
%% displayed while in draft mode.
%%
%%\begin{figure}[!t]
%%\centering
%%\includegraphics[width=2.5in]{myfigure}
%% where an .eps filename suffix will be assumed under latex,
%% and a .pdf suffix will be assumed for pdflatex; or what has been declared
%% via \DeclareGraphicsExtensions.
%%\caption{Simulation Results}
%%\label{fig_sim}
%%\end{figure}
%
%% Note that IEEE typically puts floats only at the top, even when this
%% results in a large percentage of a column being occupied by floats.
%
%
%% An example of a double column floating figure using two subfigures.
%% (The subfig.sty package must be loaded for this to work.)
%% The subfigure \label commands are set within each subfloat command, the
%% \label for the overall figure must come after \caption.
%% \hfil must be used as a separator to get equal spacing.
%% The subfigure.sty package works much the same way, except \subfigure is
%% used instead of \subfloat.
%%
%%\begin{figure*}[!t]
%%\centerline{\subfloat[Case I]\includegraphics[width=2.5in]{subfigcase1}%
%%\label{fig_first_case}}
%%\hfil
%%\subfloat[Case II]{\includegraphics[width=2.5in]{subfigcase2}%
%%\label{fig_second_case}}}
%%\caption{Simulation results}
%%\label{fig_sim}
%%\end{figure*}
%%
%% Note that often IEEE papers with subfigures do not employ subfigure
%% captions (using the optional argument to \subfloat), but instead will
%% reference/describe all of them (a), (b), etc., within the main caption.
%
%
%% An example of a floating table. Note that, for IEEE style tables, the
%% \caption command should come BEFORE the table. Table text will default to
%% \footnotesize as IEEE normally uses this smaller font for tables.
%% The \label must come after \caption as always.
%%
%%\begin{table}[!t]
%%% increase table row spacing, adjust to taste
%%\renewcommand{\arraystretch}{1.3}
%% if using array.sty, it might be a good idea to tweak the value of
%% \extrarowheight as needed to properly center the text within the cells
%%\caption{An Example of a Table}
%%\label{table_example}
%%\centering
%%% Some packages, such as MDW tools, offer better commands for making tables
%%% than the plain LaTeX2e tabular which is used here.
%%\begin{tabular}{|c||c|}
%%\hline
%%One & Two\\
%%\hline
%%Three & Four\\
%%\hline
%%\end{tabular}
%%\end{table}
%
%
%% Note that IEEE does not put floats in the very first column - or typically
%% anywhere on the first page for that matter. Also, in-text middle ("here")
%% positioning is not used. Most IEEE journals use top floats exclusively.
%% Note that, LaTeX2e, unlike IEEE journals, places footnotes above bottom
%% floats. This can be corrected via the \fnbelowfloat command of the
%% stfloats package.
%
%
%
%\section{Conclusion}
%The conclusion goes here.
%

\VerbatimFootnotes

\section{Introduction}
\label{sec: intro}

% Operating safely in dynamic environments is crucial for autonomous robots in real-world scenarios. Existing control barrier functions for obstacle avoidance often assume point or circular robots, limiting their applicability to robots with more complex geometries. In this paper, we address this limitation by presenting an analytic approach to compute the distance between a polygonal robot and moving elliptical obstacles in a 2D environment. This distance computation is utilized in constructing a control barrier function for safe control synthesis, enabling the operation of a robot with a more intricate shape. Our proposed approach offers real-time tight elliptical obstacle avoidance for polygon-shaped robots. 

Obstacle avoidance in static and dynamic environments is a central challenge for safe mobile robot autonomy. 

At the planning level, several motion planning algorithms have been developed to provide a feasible path that ensures obstacle avoidance, including prominent approaches like A$^*$~\cite{A_star_planning}, RRT$^*$~\cite{RRT_star}, and their variants~\cite{informed_rrt_star, neural_rrt_star}. These algorithms typically assume that a low-level tracking controller can execute the planned path. However, in dynamic environments where obstacles and conditions change rapidly, reliance on such a controller can be limiting. A significant contribution to the field was made by Khatib \cite{potential-field}, who introduced artificial potential fields to enable collision avoidance during not only the motion planning stage but also the real-time control of a mobile robot. Later, Rimon and Koditschek \cite{navigation-function} developed navigation functions, a particular form of artificial potential functions that guarantees simultaneous collision avoidance and stabilization to a goal configuration.
%. These functions strive to ensure collision avoidance and stabilization towards a goal configuration simultaneously. 
% Meanwhile, Fox \cite{Fox1997TheDW} introduced the dynamics window concept, an influential approach to obstacle avoidance that proactively filters out unsafe control actions. 
In recent years, research has delved into the domain of trajectory generation and optimization, with innovative algorithms proposed for quadrotor safe navigation \cite{mellinger_snap_2011, zhou2019robust, tordesillas2019faster}. In parallel, the rise of learning-based approaches \cite{michels2005high, pfeiffer2018reinforced, loquercio2021learning} has added a new direction to the field, utilizing machine learning to facilitate both planning and real-time obstacle avoidance. Despite their promise, these methods often face challenges in dynamic environments and in providing safety guarantees.


In the field of safe control synthesis, integrating control Lyapunov functions (CLFs) and control barrier functions (CBFs) into a quadratic program (QP) has proven to be a reliable and efficient strategy for formulating safe stabilizing controls across a wide array of robotic tasks \cite{glotfelter2017nonsmooth, grandia_2021_legged, wang2017_aerial}. While CBF-based methodologies have been deployed for obstacle avoidance \cite{srinivasan2020synthesis, Long_learningcbf_ral21, almubarak2022safety, dawson2022learning, abdi2023safe}, such strategies typically simplify the robot as a point or circle and assume static environments when constructing CBFs for control synthesis. Some recent advances have also explored the use of time-varying CBFs to facilitate safe control in dynamic environments \cite{he2021rule, molnar2022safety, hamdipoor2023safe}. However, this concept has yet to be thoroughly investigated in the context of obstacle avoidance for rigid-body robots. For the safe autonomy of robot arms, Koptev \textit{et al}. \cite{Koptev2023_neural_joint_control} introduced a neural network approach to approximate the signed distance function of a robot arm and use it for safe reactive control in dynamic environments. In \cite{Hamatani2020arm}, a CBF construction formula is proposed for a robot arm with a static and circular obstacle. A configuration-aware control approach for the robot arm was proposed in \cite{ding2022configurationaware} by integrating geometric restrictions with CBFs. Thirugnanam \textit{et al}. \cite{discrete_polytope_cbf} introduced a discrete CBF constraint between polytopes and further incorporated the constraint in a model predictive control to enable safe navigation. The authors also extended the formulation for continuous-time systems in \cite{polytopic_cbf} but the CBF computation between polytopes is numerical, requiring a duality-based formulation with non-smooth CBFs. 

\subsubsection*{Notations}

The sets of non-negative real and natural numbers are denoted $\bbR_{\geq 0}$ and $\bbN$. For $N \in \bbN$, $[N] := \{1,2, \dots N\}$. The orientation of a 2D body is denoted by $0 \leq \theta < 2\pi$ for counter-clockwise rotation. We denote the corresponding rotation matrix as 
% \begin{equation}
% \label{eq: rotation}
    $\bfR(\theta) = \begin{bmatrix} \cos \theta & -\sin \theta \\ \sin \theta & \cos \theta \end{bmatrix}.$
% \end{equation}
The configuration of a 2D rigid-body is described by position and orientation, and the space of the positions and orientations in 2D is called the special Euclidean group, denoted as $SE(2)$. Also, we use $\|\bfx\|$ to denote the $L_2$ norm for a vector $\bfx$ and $\otimes$ to denote the Kronecker product. The gradient of a differentiable function $V$ is denoted by $\nabla V$, and its Lie derivative along a vector field $f$ by $\calL_f V  = \nabla V \cdot f$. A continuous function $\alpha: [0,a)\rightarrow [0,\infty )$ is of class $\calK$ if it is strictly increasing and $\alpha(0) = 0$. A continuous function $\alpha:\mathbb{R} \rightarrow \mathbb{R}$ is of extended class $\calK_{\infty}$ if it is of class $\calK$ and $\lim_{r \rightarrow \infty} \alpha(r) = \infty$. Lastly, consider the body-fixed frame of the ellipse $\calE'$. The signed distance function (SDF) of the ellipse $\psi_\calE: \mathbb{R}^2 \to \mathbb{R}$ is defined as 
\begin{equation}
%\label{eq: SDF}
    \psi_\calE(\bfp') 
    = \left\{
    \begin{array}{ll}
        d(\calE',\bfp'), & \text{if } \bfp' \in \calE^c,  \\
        -d(\calE',\bfp'), & \text{if } \bfp' \in \calE,
    \end{array} 
    \right. \notag
\end{equation}
where $d$ is the Euclidean distance. In addition, $\|\nabla \psi_\calE (\bfp')\| = 1$ for all $\bfp'$ except on the boundary of the ellipse and its center of mass, the origin.


\textbf{Contributions}: (i) We present an analytic distance formula in $SE(2)$ for elliptical and polygonal objects, enabling closed-form calculations for distance and its gradient. (ii) We introduce a novel time-varying control barrier function, specifically for rigid-body robots described by one or multiple $SE(2)$ configurations. Its efficacy of ensuring safe autonomy is demonstrated in ground robot navigation and multi-link robot arm problems. 


% !TEX root = DistrSeqCtrlAttacks_TASE19.tex

%\PA{@Vuk: in \figref{exampleCIPN}, put the current (c) part on top and (a) and (b) below + adapt the references in the text accordingly}

\section{Motivating Example and Problem Description}
\label{sec:motivation}
%Building secure and correct-by-design RMS necessitates efficient techniques for systematic security analysis of distributed control applications deployed on IIoT-enabled local controllers (LCs). While a plethora of formal modeling frameworks is employed under the umbrella of IoT (e.g.,~\cite{iotsat,iotz3,iotautomata}), industrial automation is commonly based on GRAFCET (IEC~60848)/SFC (IEC~61131-3) control design, and consequently on the underlying formal semantics of Control Interpreted Petri Nets (CIPN). Therefore, in this paper we focus on security analysis of IIoT-enabled controllers formally described using CIPNs.

We first introduce CIPNs and the mother formalism of PNs, before presenting an illustrative distributed control application, used as a running example in this work to highlight security vulnerabilities caused by automation (i.e.,~control)~distribution. %We then show how to introduce a security-aware communication channel model along with suitable plant models, and base them on Time Petri Nets (TPN); we transform the CIPN-based controller representation into a TPN-compliant model to obtain a system-level security-aware model of the cyber-physical system. We use the composition of the controller, plant, and channel models to show violation of relevant safety properties in the presence of attacks, and demonstrate how inclusion of security services affects modeling and verification.

%
\begin{figure}[!t]
%\vspace{-16pt}
	\centering
	\includegraphics[width=0.462\textwidth]{exampleCIPN.pdf}
	\caption{Distributed Automation Example: Simple CIPN-based distributed control model of (b) conveyor monitor~and (c)~pick~\&~place controller; physical setup is illustrated in~(a).}
	\label{fig:exampleCIPN}
\end{figure}


\subsubsection*{Petri Nets (PNs)}
A Petri net is a 5-tuple $\mathbf{PN}=(P,T,F,W,M_0)$, where $P=\{P_1,...,P_m\}$ is a set of places (represented by circles), $T=\{T_1,...,T_n\}$ is a set of transitions (represented by bars) such that $P\cup T \neq \emptyset$ and $P\cap T = \emptyset$, while $F\subseteq\{P \times T\}\cup \{T\times P\}$ is the set of arcs between places and transitions (no arc connects two places or two transitions). $W$ is the vector of arc cardinalities which determines how many tokens are removed/deposited over specific arcs upon firing of corresponding transitions. The $\mathbf{PN}$'s state  is defined by its \emph{marking}, i.e., distribution of tokens (captured by dots inside places); $M_0$ is the initial marking (i.e., the initial token distribution). Functionally, current PN marking determines the system's state, while transition firing (i.e., token flow) represents a state change. A formal description of the PN semantics can be found~in~\cite{Murata1989541}.

\subsubsection*{Distributed Automation with CIPNs}
CIPNs are a version of PNs where arc cardinality is fixed to $1$ and the initial marking $M_0$ may initialize only one place with a token. In CIPNs, transitions' firing can be conditioned by system inputs (i.e., sensors) in the form \verb!sensor==value!, while actuation commands can be associated with places in the form \verb!actuator=command!. For distributed automation, functionality of each local controller (LC) is captured by the corresponding CIPN.
For (event) information exchange between LCs, places of a CIPN may invoke the communication API exposed by the LC runtime environment; % to transmit event information,
this is denoted as \verb!Send(signal,value)! for broadcast or \verb!Send({dest1,dest2,...},signal,value)! for uni/multicast transmissions. Dually, the receiving LC can condition its transitions with statements similar to conditioning on locally connected sensors (i.e., as \verb!signal==value!)~\cite{jakovljevic_tcst19}.

Formally, a CIPN is a 6-tuple $\mathbf{CIPN}=(P,T,F,C, A, M_0)$ where $P$, $T$, $F$, and $M_0$ are defined as for PNs; $C=\{C_1,...,C_n\}$ is a set of logical conditions %(i.e., Boolean functions)
enabling synchronization of the controller with sensors by guarding corresponding transitions in the $\mathbf{CIPN}$ model; %within the controller model;
$A=\{A_1,...,A_m\}$ is a set of actions %(i.e., Boolean or other functions)
on actuator outputs that are allocated to places; formal CIPN semantics is available in~\cite{David20101}. By its definition, CIPN semantics is deterministic (does~not support nondeterminism due to CIPNs use to only model controllers), which needs to be ensured during model~design~\cite{david1994petri}. % -- this has not been a limitation, .\todo{Z/V - check}

Distributed CIPN-based controller models are obtained directly by design, or by distributing existing  (i.e., centralized) controllers % distribution techniques
(e.g.,~as done in~\cite{jakovljevic_tcst19}). %starting from a conventional (i.e., centralized)~design.
%
Fig.~\ref{fig:exampleCIPN} shows a simple control application, built with two wireless nodes, that we will use as a %{illustrative}
running example in this work. %We begin by describing the system, whose physical configuration is shown in Fig.~\ref{fig:exampleCIPN}(a)), while referencing the corresponding control model, shown in Fig.~\ref{fig:exampleCIPN}(b-c).



\begin{example}
\label{ex:motivation}
%As a motivating example,
Consider a simple application from~Fig.~\ref{fig:exampleCIPN}.
Control over the physical system (Fig.~\ref{fig:exampleCIPN}(a)) is performed by the conveyor monitor ($LC_1$) and the pick~\&~place station controller ($LC_2$). Two sensors (e.g.,~proximity, retro-reflective) locally connected to $LC_1$ overlook two parallel incoming conveyor belts; they sense if a workpiece is ready to be picked from either of the conveyors % (by the pick~\&~place station),
and placed on the third, outgoing conveyor.
The CIPN-based controllers %(i.e., their CIPN-based model/specification)
for $LC_1$ and $LC_2$ are shown in Fig.~\ref{fig:exampleCIPN}(b-c).
Initially, $LC_1$ is in state \verb!Pcm_Init! where it is waiting for either of %the locally connected
its sensors to indicate workpiece presence (i.e.,~transition \verb!Tcm_Pres1! / \verb!Tcm_Pres2! is conditioned by the sensing event \verb!Pres1==1! / \verb!Pres2==1!).\footnote{For model readability, we employ descriptive notation for~places, transitions, conditions, and actions; e.g., transition \verb!Tctrl_wfRet! in controller $LC_2$ waits for the return cycle to finish, while place \verb+Pcm_TxCtrl_Pick1+ on the conveyor monitor sends signal \verb+Pick==1+ to the pick~\&~place controller.} Upon detection of a workpiece (i.e.,~when one of %the conditions
\verb!Pres1==1! / \verb!Pres2==1! is satisfied, and thus transition \verb!Tcm_Pres1! / \verb!Tcm_Pres2! is enabled), $LC_1$ sends a message to $LC_2$ %(through communication
(via API call \verb!Send(Pick,1)! / \verb!Send(Pick,2)! in place \verb!Pcm_TxCtrl_Pick1! / \verb!..._Pick2!); % \verb!Pcm_TxCtrl_Pick2!);
the message %containing the corresponding signal indicating
indicates which conveyor has a workpiece ready to be picked. $LC_1$ then waits for completion of the pick~\&~place operation.

Concurrently, $LC_2$'s initial state is \verb!Pctrl_Init! where the pick~\&~place station is commanded to halt (by \verb!PP_Act=0!). Once the signal \verb!Pick! is received from $LC_1$, based on its value the token in $LC_2$ model transitions over \verb!Tctrl_wfPick1! / \verb!Tctrl_wfPick2! into place \verb!Pctrl_P&P1! / \verb!Pctrl_P&P2! where the corresponding actuation command is given to the pick~\&~place station (i.e.,~\verb!PP_Act=1! / \verb!PP_Act=2!). After completion of the pick~\&~place operation, condition \verb!P&P_Complete==1! is satisfied, allowing the $LC_2$ to transition over \verb!Tctrl_P&Pcomplete! / \verb!Tctrl_P&Pcomplete! to \verb!Pctrl_Ret! where it commands the pick~\&~place station to return to home position (by \verb!PP_Act=-1!). $LC_2$ waits for completion of the pick~\&~place station return stroke (when %condition
\verb!Ret_Complete==1! evaluates to true); after it transitions back into \verb!Pctrl_SendCMfin!, where it signals %conveyor monitor
$LC_1$ that the workcycle is complete. $LC_2$ transitions over \verb!Tctrl_RetInit! back into the initial place \verb!Pctrl_Init!.
%
Finally, conveyor monitor $LC_1$ can also return to its initial state (formally, the token is deposited back into \verb!Pbm_Init! over \verb!Tbm_RetInit1! / \verb!Tbm_RetInit2!), as the condition \verb!Complete==1! is satisfied.\QEDE
\end{example}

%On the other hand, this model assumes ideal communication packet delivery rates, and absence of unpredictable communication channel behaviors. For instance, consider that an adversary with network access mounts an \emph{impersonation} attack~\footnote{This type of attack is also known as \emph{masquerade} attack.} at the beginning of the workcycle when $LC_A$ is waiting in place \verb!PaCTRL_wfRxV_bHome1! to receive a message from $LC_B$ signalling that the workpiece is picked up and cylinder $B$ is retracted. If the adversary impersonates $LC_B$ and sends the corresponding message to $LC_A$ before the workpiece is gripped and cylinder $B$ retracted, the manipulator and currently handled workpieces may incur significant physical damage due to early extending of cylinder $A$. Another example would be an adversary \emph{delaying} or \emph{blocking} some transmissions or acknowledgements between controllers.\footnote{This type of attack is also known as Denial-of-Service (DoS) attack.} For instance, $LC_B$ informs $LC_C$ when cylinder $B$ reaches its end position, so that gripper $C$ can perform gripping/releasing of the workpiece. Since gripper $C$ does not feature end position sensing, and $LC_B$ deterministically waits for $500~ms$ for the gripping/releasing operation to be completed before resuming execution, an adversary delaying or blocking messages from $LC_B$ could cause $LC_C$ to not perform as expected, compromising system safety. These examples indicate that the system can be easily compromised in the case when distribution of control tasks is performed without implementation of suitable security measures.

The CIPN-based control models from Example~\ref{ex:motivation} assume ideal communication (i.e., packet delivery), without unpredictable %communication
channel behaviors. For instance, consider an adversary with network access that mounts an \emph{impersonation} (i.e., spoofing or masquerade~\cite{wirelessattack}) attack % ~\footnote{This type of attack is also known as \emph{masquerade} attack.}
when $LC_2$ is waiting for a message from $LC_1$ %signaling
that a workpiece should be picked up. By sending % If the adversary sends
the corresponding message (e.g., by signaling \verb!Send(Pick,1)!), the attack will result in the pick~\&~place station  being commanded pickup by $LC_2$; hence, it may collide with upcoming workpieces, potentially incurring mechanical damage, or just waste a workcycle. Similar %consequences
 holds for \emph{message modification}~\cite{wirelessattack} (i.e., signal replacement~\cite{DESsupervisoryControl}) attacks,
 when the right conveyor belt contains a workpiece ready to be picked up (i.e., \verb!Pres_R==1!), %but $LC_1$ maliciously signals \verb!Send(Pick,2).!
 but the attacker intercepts the corresponding message and maliciously signals \verb!Send(Pick,2)!. %\todo{but $LC_1$ maliciously signals !Send(Pick,2).! -- where is the attacker?} %\PA{actually this section misses attacker model}
Also, if an adversary \emph{delays} or \emph{blocks} some transmissions or acknowledgements (ACKs)~between LCs (i.e., launching a Denial-of-Service (DoS) attack~\cite{wirelessattack}) %. For instance, $LC_2$ informs $LC_1$ when the pick~\&~place cycle is completed; delaying this signal may result in
the system may experience excessive downtime.

These examples illustrate that distributing control and automation functionalities may introduce security vulnerabilities as the system operation can be easily affected by an attacker with network~access.
Hence, in this work, we focus on security aspects of % the discrete-event distributed industrial IoT systems
IIoT-enabled distributed automation systems; our goal is to provide techniques to model and analyze system behaviors in the presence of  \emph{network-based attacks}, while enabling the use of analysis results %of the provided analysis
to modify (i.e., update)) the system to achieve attack-resilient operation.

\subsection{Overview of our Approach}
We start from a functional description %(i.e., specification) for each
of $N$ LCs expressed by $\mathbf{CIPN}_i$, $i=1,...,N$. We consider an attacker with full access to the network with {$M$ communication channels}. %We also assume that
The attacker is not able to compromise LCs, but has full knowledge of the state of each~LC. %\todo{check}
%
We introduce a design-time methodology illustrated in Fig.~\ref{fig:methodology} that starts with automatic transformation of CIPN-based control models to Timed Petri Net~(TPN)-based models; such models allow for explicit capturing of (i)~the communication semantics, (ii)~platform-based effects using timed transitions to model real non-zero execution and message propagation times, and most importantly (iii)~non-deterministic behaviors necessary to model adversarial actions. %While inherent determinism of CIPNs does not present a limitation when the formalism is used to specify controllers' behaviors, it prevents the use of CIPNs to model malicious actions~\cite{wang_arxiv19}. On the other hand, TPN's support for nondeterminism, as well as availability of open verification tools (e.g.,~\cite{romeo}), makes them a great candidate for security-aware modeling.

%\todo{give an outline here}
We show how the remaining closed-loop system components (i.e., the plant and communication channel in the presence of attacks) can be modeled within the TPN formalism. Furthermore, we  demonstrate how composition of these models enables system-wide analysis of control performance in the presence of attacks. Finally, we show how design-time formal verification results can be used during code generation for smart IIoT-based controllers, which facilitates adaptation  of LCs' firmware in order to address exposed security~concerns. %Our framework is summarized in~Fig.~\ref{fig:methodology}.

\begin{remark}[Petri nets vs. Automata/Finite-State Machines]
%In this work, we
We employ Petri Net-based modeling formalisms since CIPNs are the main formalism used to capture existing (including distributed) automation  systems. % (including distributed systems).
For example, GRAFCET~(IEC~60848) and SFC (IEC 61131-3) languages for programming
industrial control systems originate from Petri Nets, with their behavioral equivalence discussed in~\cite{David20101,GRAFCETtoTPN,SFCtoTPN}.
However, %it is important to highlight that
the proposed methodology, including network and attack modeling, can be directly extended to other discrete-event IoT systems whose behavior can be captured with automata/finite-state machines due to the fact that formal mappings between semantics of Petri nets and automata have been defined (e.g.,~as in~\cite{PNtransformation1,PNtransformation2}).\QEDE
%  Intuitively, Petri nets present means for automatic state spaces/underlying state machine generation. Infinitely large state spaces can be compactly represented with Petri nets~\cite{bobbioPNs}, and translation semantics between Petri nets and automata have been defined (e.g.,~\cite{PNtransformation1,PNtransformation2}). On the other hand, GRAFCET/SFC languages for programming industrial control gear originate from Petri nets, with their behavioral equivalence discussed in~\cite{David20101,CIPNtoTPN,Wightkin2011455}.\PA{needs to be fixed}
\end{remark}
%\PA{add a remark on relationship between PN and FSM/Automata; EMSOFT crowd is more comfortable with automata} 
% !TEX root = DistrSeqCtrlAttacks_TASE19.tex


%\section{Security-Aware Modeling}
%\label{sec:modeling}
\section{TPN-Based  Automation Modeling} %Distributed
\label{sec:modeling}

TPNs extend PNs by introducing timed transitions; in a TPN, every transition is characterised by an interval $(\underline{t}_f,\overline{t}_f),[\underline{t}_f,\overline{t}_f),(\underline{t}_f,\overline{t}_f],$ or $[\underline{t}_f,\overline{t}_f]$, where $\underline{t}_f$ and $\overline{t}_f$ are the lower/upper bound on the transition firing times, %respectively,
which may be \emph{zero} or \emph{infinity}---time interval next to \emph{immediate} transitions (i.e., with zero firing time) is not specified.
Also, %$\underline{t}_f$ and $\overline{t}_f$
the firing times are defined relative to the moment of transition enabling, without any assumptions %are imposed
on their distribution. % of firing times.
This enables modeling of timed properties of real-time control software~\cite{TPNforRT1,TPNforRT2,TPNforRT3}.
In addition, by supporting non-determinism, TPNs facilitate attack modeling %of complex adversarial impact
that cannot be accurately done with deterministic or stochastic models. %, as we will show later in this section.

%{The state of a TPN can be represented as a pair $S=(M,I)$, where $M$ is the marking, and $I$ is the vector of time intervals corresponding to each of the enabled transitions, forming the \emph{firing domain}.}

%On the other hand, standard formalism employed in control system design and control firmware generation based on CIPNs does not support aforementioned semantics necessary for design of resilient distributed automation.
Therefore, we transform formal distributed control specifications expressed by $\mathbf{CIPN}_i$, $i=1,...,N$, into TPN-compatible models $\mathbf{TPN}_i^{ctrl}$, $i=1,...,N$. We then compose these models with plant models $\mathbf{TPN}_i^{plant}$, $i=1,...,N$, and security-aware communication channel models $\mathbf{TPN}_j^{channel}$, $j=1,...,M$,\footnote{This captures the general case where time or frequency multiplexing may be used to provide multiple communication channels over the same medium.} which enables us to reason about system-level safety and security properties under the modeled adversarial influences. Since both CIPNs and TPNs originate from PNs, the translation from $\mathbf{CIPN}_i$, $i=1,...,N$ controller models to $\mathbf{TPN}_i^{ctrl}$, $i=1,...,N$ is direct for all places and transitions except where platform-implemented API is called,  i.e., %constructs requiring special translation semantics are:
\begin{enumerate}
  \item Places handling actuation, and transitions handling sensing by issuing I/O API calls %(respectively
  (\verb!actuator=value! and \verb!sensor==value!, respectively),
  \item Places handling transmissions over a shared channel, and transitions handling receiving of communication signals %communicated
  %over the shared channel %by issuing communication
  using API calls (\verb!Send(destination,signal)! and \verb!signal==value!, respectively),
  \item Places calling other platform-dependent API, such as request for execution delays.
\end{enumerate}
%
These CIPN constructs, which directly rely on the underlying platform used to implement the controller , % runtime environment
must be explicitly modeled as nets that capture: 1)~interaction between LCs $\mathbf{TPN}_i^{ctrl}$ and the plant $\mathbf{TPN}_i^{plant}$, 2)~interaction between LCs $\mathbf{TPN}_i^{ctrl}$ and communication channel(s) $\mathbf{TPN}_j^{channel}$, and 3)~runtime environment %state
changes based on issued commands (e.g., variable updates, execution delays).

Thus, we introduce methods for automatic extraction of TPN-based controller models from existing CIPN-based models that may be used to generate control code. We capture interaction between the automation system and plant, as well as the platforms' runtime environment in Section~\ref{subsec:plantAndControllerInteraction}, %\todo{where is 2nd part done?}%and~\ref{sec:runtimemodel}),
and security-aware modeling of communication channels and their interfaces with controllers in Section~\ref{subsec:channelAndCtrlChannelInteraction}. These methods enable developing a complete closed-loop system model that can then be used to reason about system resiliency to~attacks.

\subsection{Modeling Plants and Controller-Plant Interaction}
\label{subsec:plantAndControllerInteraction}
Nominal behavior of the controlled physical system is typically known at control design time. Thus, since the formalism of CIPNs is universally adopted for automation design, we assume that a PN-based (i.e., $\mathbf{CIPN}$ or $\mathbf{TPN}$) plant model is available.\footnote{On the other hand, if an automata or other discrete-event system representation of the plant is available, existing tools and methodologies can be used to translate such models into a PN-based representation (e.g.,~\cite{PNtransformation1,PNtransformation2}).}
%We begin by describing development of a TPN-based plant model for
On the running example, we describe development of such TPN-based plant model, along with a TPN-compliant controller-plant interface implemented through marking-dependent \emph{guard functions}. %\todo{what is in figure 3.(c); there is nothing in caption; also, where do you use it in the text?}

%The rules for the

%\vspace{40pt}

%
\begin{figure}[t]
%\vspace{-14pt}
	\centering
    \includegraphics[width=0.46\textwidth]{ctrlPlantInteraction.pdf}
	\caption{Plant and plant-controller interaction modeling: (a) $\mathbf{TPN}$ model of the pick~\&~place station; % ($[\underline{t}_{p\&p}^{proc},\overline{t}_{p\&p}^{proc}]$/$[\underline{t}_{p\&p}^{ret},\overline{t}_{p\&p}^{ret}]$ denote the range of pick~\&~place/return times);
	(b) extended model of $LC_2$ from~Fig.~\ref{fig:exampleCIPN}(c) with TPN-compatible sensing/actuation -- the model is not a $\mathbf{TPN}$ as it still relies on the communication API (in {\color{red}red}) for interaction with other LCs; (c) model of incoming workpieces with a lower bound on the workpiece interarrival.}
	\label{fig:ctrlPlantInteraction}
\end{figure}

%\subsubsection{Plant Modeling}
Fig.~\ref{fig:ctrlPlantInteraction}(a) shows a $\mathbf{TPN}$ model of the pick~\&~place~station from Fig.~\ref{fig:exampleCIPN}; Fig.~\ref{fig:ctrlPlantInteraction}(c) models  the incoming workpieces arrivals, with a lower bound on the interarrival times.
 Place \verb!Pp&p_Init! represents the station's initial state. Token flow from this place is conditioned by the corresponding commands of $LC_2$ \verb!PP_Act==1! and \verb!PP_Act==2! from the controller model in Fig.~\ref{fig:exampleCIPN}(c). In the formalism of TPN, \emph{marking-dependent guard functions} can be used to restrict state changes (i.e., token flow) in the plant model, i.e., a marking-dependent function (denoted with \verb!M(!$\cdot$\verb!)!) assesses the state of the controller model (i.e., token distribution) and returns the current number of tokens inside the argument place. Hence, guard function \verb!G:M(Pctrl_P&P1)==1! (or \verb!G:M(Pctrl_P&P2)==1!  for the other conveyer belt) is associated with transition \verb!Tp&p_wfPick1! (or \verb!Tp&p_wfPick2!).
%
%
Once the pick~\&~place process is triggered (i.e., $LC_2$ initiates it from a specific conveyor belt) by the $LC_2$'s model advancing its token to \verb!Pctrl_P&P1! (or \verb!Pctrl_P&P2!), the station's token transitions to place \verb!Pp&p_P&P1! (or \verb!Pp&p_P&P2!) if conveyor belt 1 (or 2) has a workpiece waiting to be processed.
Note that in order to capture realistic executions, the actual times to complete the pick~\&~place and return processes are not deterministic; this is natively supported by TPN's timed transitions; transitions \verb!Tp&p_P&P1! / \verb!Tp&p_P&P2! have firing times from the interval $[\underline{t}_{p\&p}^{proc},\overline{t}_{p\&p}^{proc}]$, as shown in Fig.~\ref{fig:ctrlPlantInteraction}(a).\footnote{These intervals may be obtained experimentally under nominal conditions at runtime, or based on design constraints %(e.g., from first principles)
at design-time. Also, different durations of the picking process from different conveyors are supported but set equal in the above model/figure to simplify our presentation.} %Additionally, after the station completes the process, it signals this event to $LC_2$; this is formally modeled via a transition \emph{update function}, denoted as \verb!U:P&P_Complete=1! next to the transition.


Now, the station dwells in place \verb!Pp&p_wfRet! waiting for the signal from $LC_2$ to return to the home position, i.e., guard function \verb!G:M(Pctrl_Ret)==1! conditions transition \verb!Tp&p_wfRet! on the corresponding controller's state (i.e., where the return action is issued). Once return is commanded, the pick~\&~place station takes non-deterministic time from %the interval
$[\underline{t}_{p\&p}^{ret},\overline{t}_{p\&p}^{ret}]$ to return (i.e., transition over \verb!Tp&p_RetInit!). After this transition, % (i.e., returning to home position),
the station model is back in the initial state, waiting to process the next workpiece. %Finally, note that we do not denote a time interval next to \emph{immediate} transitions (i.e., transitions with zero firing time).


The actuation part of the plant-controller interface is managed by guard functions assessing the controller's marking; therefore, explicit actuation input updates (e.g., \verb!PP_Act=1!) in the CIPN places are omitted in the transformation to the TPN model as TPN places do not feature any attributes. This interface can also be achieved alternatively, by utilizing \emph{update functions} which are triggered on the firing of controller's transitions and can update markings or variables. The choice of the transformation semantics from CIPN to TPN can therefore be adjusted to the specific platform implementation.

%\subsubsection{Modeling Controller-Plant Interaction}
%\label{subsubsec:controllerPlantInteraction}

%The plant side of the controller-plant interaction is enforced through transitions in the plant model that are conditioned by guard functions as previously described. For instance, the transition \verb!Tp&p_wfRet! in the plant model is guarded by the Boolean function \verb!G:M(Pctrl_Ret)==1! which evaluates to \emph{true} if the controller's model has a token in place \verb!Pctrl_Ret! (i.e., station return to home is commanded).


%the TPN-based controller model must contain places whose number of tokens are assessed by the guard functions in the plant model. Therefore, places handling actuation through commands \verb!actuator=value! within the CIPN-based controller model are transformed into aforementioned places in the TPN-based controller model. In essence, \textbf{activation} of guard functions dependent on the controller's state that enables transitions in the plant model corresponds to the controller \textbf{sending} actuation commands to locally connected actuators.

Dually to the actuation part of the interaction, % of controller-plant interaction,
sensing is modeled by introducing plant-marking-dependent guard functions on controller's transitions. Specifically, transitions conditioned by sensor values in the form \verb!sensor==value! in the CIPN controller model are replaced with immediate transitions guarded by a Boolean function evaluating to \emph{true} if the plant model marking corresponds to the plant state where \verb!sensor==value! is satisfied, and to \emph{false} otherwise.

For instance, once $LC_2$ commands return of the pick~\&~place station, (i.e., controller model from Fig.~\ref{fig:exampleCIPN}(c) has the token in \verb!Pctrl_Ret!), it is blocked on the transition \verb!Tctrl_wfRet! which is guarded by the condition \verb!Ret_Complete==1!. This transition in the CIPN model is transformed into a transition in the TPN model in Fig.~\ref{fig:ctrlPlantInteraction}(b) that is guarded by a function dependent on the marking of the pick~\&~place station model from Fig.~\ref{fig:ctrlPlantInteraction}(a)~(i.e., controller waits for the station to reach home position). Guard function \verb!G:M(Pp&p_Init)==1! returns \emph{true} when the token in the plant model transitions from \verb!Pp&p_Ret! to \verb!Pp&p_Init! over the timed transition \verb!Tp&p_RetInit! (here, \verb!M(!$\cdot$\verb!)! denotes a function that returns the current number of tokens inside the argument place) -- hence, controller $LC_2$ can transition over \verb!Tctrl_wfRet!. Therefore, this guard function is used for the transition \verb!Tctrl_wfRet! in the TPN model of $LC_2$. This models the controller side of the controller-plant interaction, i.e., sensor sampling. More complex conditions based on multiple sensors are implemented by forming an arbitrary plant marking-dependent Boolean guard function.



Fig.~\ref{fig:ctrlPlantInteraction}(b) shows a controller model of %that implements
the described controller-plant interface. %  which is TPN-compliant.
The model, obtained from the CIPN-based model in~Fig.~\ref{fig:exampleCIPN}(c), is intermediary, and not fully TPN-compliant; the CIPN-based communication semantics (i.e., signal transmissions via \verb!Send(destination,signal)!, and receptions through \verb!signal==value! denoted in red in Fig.~\ref{fig:ctrlPlantInteraction}(b)) is still present in the model. {However, to allow for verification of properties for systems where networking is not a concern, this communication semantics can be easily adapted to TPNs by applying the same \emph{guard/update} functions as described; this results in a model architecture from Fig.~\ref{fig:modeling}(a).

Additionally, note that the conveyor monitor model from Fig.~\ref{fig:exampleCIPN}(b) can be directly transformed into a TPN, with guards \verb!Pres1==1! and \verb!Pres2==1! conditioning progress based on external inputs (i.e., the process) triggered by the process model shown in Fig.~\ref{fig:ctrlPlantInteraction}(c). This process model abstracts away the nature of the process of incoming workpieces on the conveyors with a minimum inter-arrival time (i.e., time in the interval $[t_{min}^{proc},\infty]$). %, without imposing any assumptions on the process iteself.

%\subsubsection{Controller Runtime Environment Modeling}
\subsubsection{Controller Runtime Environment Modeling}
\label{sec:runtimemodel}

Another challenge for the automatic mapping of CIPN-based control models into TPN-compliant models is mapping of places issuing system calls from the runtime environment (e.g., execution delays, requests for timer interrupts, setting counter events) or updating local controller state (e.g., manipulating global variables). Requested execution delays can easily be modeled as timed transitions with the exact firing times (i.e., where the lower and upper firing time bound are the same); in general, however, event timings with different semantics are available depending on the control implementation---i.e., GRAFCET or SFC. In~\cite{GRAFCETtoTPN,SFCtoTPN}, authors provide detailed translational semantics between CIPNs and TPNs in these cases by introducing \emph{event sequencers} as certain conditions exist where transitions can be taken while time to some events generated internally in places has still not elapsed.%\QEDE
%\end{remark}
\begin{remark}[Modeling more Complex Execution Environments]
While we consider single-threaded automation examples (as most sequential control implementations are), existing techniques for modeling parallel systems can be applied given the expressiveness of TPNs. For example, for multithreaded applications where task preemption is allowed, the operating system scheduler can be modeled as a separate component, even in case of multi-processor platforms~\cite{TPNscheduling}. %or the upper bound on the firing time of a specific transition adjusted to include worst-case blocking time incurred from other lower-priority workload~\cite{}.
%
%On the other hand, definition of global variables and functions is supported by state-of-the-art TPN tools (e.g.,~\cite{romeo}). Update functions that are invoked on firing of the transition to which the function is assigned and guard functions are generally not only marking- but also global variable-dependent, which allows for rich modeling of complex internal LC state updates, similarly as used to implement the controller-plant interface in Sec.~\ref{subsec:plantAndControllerInteraction}.
\QEDE
\end{remark}

\subsection{CIPN and TPN Controller Equivalence}
  An execution path in CIPNs can be defined as a sequence of markings, where a change in the marking occurs due to firing of a transition. Recall that places are associated with actions; hence, each marking is associated with a set of actions, while transitions are associated with guards---firing of each transition is thus conditioned by a set of conditions.\footnote{We employ the standard assumption that all inputs are re-evaluated after firing of every transition (e.g., as done in~\cite{SFCtoTPN}).} Therefore, an execution path is a sequence $M_0,\mathcal{T}_1,M_1,\mathcal{T}_2,...$, where $\mathcal{T}_{i}$ is the transition taking the net from marking $M_i$ to $M_{i+1}$. In the TPN model, a path is characterized by a similar sequence with the addition of transition timing.\footnote{Strictly speaking, two types of time intervals characterize each transition: static intervals (i.e., design-time bounds) when they may fire, and dynamic (i.e., runtime) intervals when they can fire at any given instant, conditioned by all other enabled transitions. However, for purposes of showing marking-based equivalence with CIPNs, time can be abstracted away.} In our case, it is sufficient to maintain the CIPN \emph{controller} execution paths in the TPN model, as our objective is operational equivalence of the source (centralized), and the target (distributed) control models.

{
CIPN controller specifications are fully deterministic by design, and have only \emph{immediate} transitions.\footnote{Immediate transitions become fireable immediately after enabling, i.e., without a time delay.}
Thus, the corresponding target TPN-based controller models obtained by directly constructing the same model in the TPN formalism without additional constructs (i.e., as previously described in Sec.~\ref{subsec:plantAndControllerInteraction}), %when composed
do not introduce behavior which is not covered by the source CIPN-based models. Consequently, execution paths of the composition of the TPN models match with that of the CIPN, from the input-output (i.e., sensing-actuation) perspective. In other words, no execution path is added when we transform the CIPN-based controller into the TPN-based representation~\cite{jakovljevic_tcst19}. Intuitively, the TPN models obtained by direct mapping from CIPN (i.e., place-by-place, and transition-by-transition), are still \emph{fully deterministic} (isolated from the intrinsically non-deterministic plant and channel); i.e., their behavior is identical to their CIPN counterparts, and same behavioral assumptions (e.g., 1-boundedness) hold~\cite{David20101}.
}

%However, given that the following are satisfied:
%\begin{itemize}
%  \item Controller TPN is considered in isolation from the plant (which is usually non-deterministic),
%  \item Controller TPN is obtained from a CIPN specification that is fully deterministic by design (and has only immediate transitions),
%  \item Labels can be associated with TPN places (like outputs with CIPN places) and transitions conditioned by the same inputs,
%\end{itemize}






%
\begin{figure}[t]
	\centering
    \includegraphics[width=0.45\textwidth]{modeling.pdf}
	\caption{Model architecture: (a) Model captures local controllers $LC_i$, plants $PLANT_i$ and their interactions, (b) Model also captures the employed communication transceivers $XCVR_i$, and the underlying communication channel $CH_i$. Note that local controller models  $LC_i$ in schemes (a) and (b) are not the same; i.e., in (b), controller places/transitions invoking communication APIs are made compatible with the transceiver model.}
	\label{fig:modeling}
\end{figure}


% !TEX root = DistrSeqCtrlAttacks_TASE19.tex

%
\begin{figure*}[t]
	\centering
	\includegraphics[width=0.98\textwidth]{CIPNtoTPN.pdf}
	\caption{Transformation between CIPN-based and TPN-compatible communication models; (a) a Tx/Rx place/transition pair in the CIPN formalism; (b) the same Tx/Rx place/transition pair modeled with as a TPN adjusted to the half-duplex, acknowledge-required unicast CSMA-CA-based channel, whose model is shown in (c); (d) model of the employed radio transceiver (i.e., the governing RF state machine TPN model). Note that each Tx/Rx net pair in (a) (from the model in Fig.~\ref{fig:modeling}(a)) is extended into a corresponding pair in (b), while only a single model from (c) and (d) are added to obtain the model from Fig.~\ref{fig:modeling}(b).}
	\label{fig:CIPNtoTPN}
\end{figure*}
%


\section{Security-Aware Modeling of the Channel  and Controller-Channel Interaction}
\label{subsec:channelAndCtrlChannelInteraction}

%We  now discuss how to complete the transformation of the controller model into a model from Fig.~\ref{fig:modeling}(b), by also capturing a suitable controller-channel interface (extracted from the transceiver specifications) that is compatible with a security-aware channel model. Specifically,
We now introduce a security-aware channel model, including a TPN-compliant controller-channel interface that enables model composition. Hence, we address modeling challenges to enable the transition from the security-agnostic model structure from Fig.~\ref{fig:modeling}(a), to the security-aware model composition shown in Fig.~\ref{fig:modeling}(b). We start by defining the attack model.

\subsection{Attack Model}
\label{subsec:attackModel}
%In this work, we
We assume %presence of
a powerful network-based adversary that:
\begin{enumerate}
  \item Has the full knowledge of the distributed system, %i.e., has access to the same CIPN controller models that are used to generate control code,
  including the CIPN models, generated code and analysis framework, as well as  the current state of all LCs (and their transceivers).
  \item Has network access and full communication protocol compliance, i.e., is able to transmit unsigned messages as any of the LCs, or intercept messages or %acknowledgement (ACK) packets
ACKs exchanged by~LCs,
  \item Is able to precisely time actions and align transmissions with legitimate network traffic, e.g., to interfere with legitimate messages by transmitting the carrier signal or a protocol-compliant message.
%  \item Has full access to the current state of the system, i.e., has knowledge of the current state of all LCs (and their transceivers).
\end{enumerate}

Thus, the adversary may mount the following attacks:
\begin{enumerate}
  \item \emph{Interception or delaying of legitimate packets (DoS):} With these attacks, adversarial transmissions occupy the channel (a)~blocking transmissions %initiated by
  from legitimate LCs to prevent or delay their access to the network, % access of the targeted LCs,
  or (b)~blocking ACKs on legitimate LC's transmissions to cause unnecessary retransmissions and slow down progression of the targeted transmitter~\cite{wirelessattack}.

  \item \emph{ACK spoofing:} The attacker may impersonate an ACK expected by a legitimate transmitter; e.g., following by interception of the transmission, the attacker may spoof the ACK misleading the legitimate transmitter into believing that the transmitted signal was received by the intended receiver~\cite{802.15.4auth}, both for regular and `heartbeat'/sync messages~\cite{ackattack1}.%\PA{@Vuk: check, new wording instead of old " Alternatively, the attacker may mislead a transmitter that the intended receiver of a packet is alive by spoofing the ACK~\cite{ackattack1}."}

  \item \emph{Impersonation/Masquerade:} %This type of attack pertains to an
  The adversary may transmit false event signals on behalf of a legitimate LC (i.e., impersonating another controller), with the goal to inject false commands~\cite{802.15.4auth} or sensor measurements; such attacks could e.g., allow the targeted receiver to resume execution while it is blocked waiting for %reception of
  an event signal, before the event %is physically attainable (i.e., before
   it is sent by the legitimate LC.

  \item \emph{Signal replacing/Message modification:} The adversary may modify content of a legitimate message %with the intention
  to deliver false event information. While logically the same, the attack procedure differs from intercepting a legitimate transmission followed~by a masquerading attack~\cite{ackattack2}, and thus is modeled~differently.\footnote{This type of attack is technically more challenging to perform compared to other attacks, especially over a wireless medium.}

  \item \emph{Replay attack:} The attack characterizes an adversary that records events signaled by the LCs and replays the sequence of events on behalf of one or more LCs; thus, maliciously emulating activity of LCs whose operation (s)he is interfering with~\cite{ackattack2}.% \todo{@Vuk: a ref in indusrial setup would be good}
\end{enumerate}

The above attack set, considered in this work, covers all reported attacks that, from the standpoint of low-level signaling of events, could have direct impact on Quality-of-Control (QoC) of the underlying physical process~\cite{wang_arxiv19,wirelessControl}. %\todo{@Vuk - fix}
%this attack models is adequate.~\cite{wirelessControl} \todo{@Vuk - fix}
Other attacks, such as attacks on {network routing policies}, are focused on higher-level information flows and are thus harder to directly relate to automation QoC~\cite{ackattack2}. Consequently, our goal is to model attacks by capturing their influence~on~the sequential control system and the resulting QoC, rather than the employed attack vector for any specific attacks; i.e., the attack model should be agnostic to the actual attack implementation.

\subsection{TPN-Based Modeling of Attack Impact}

Recall that CIPN models rely on platform-provided communication APIs for passing events between LCs; e.g., as in Fig.~\ref{fig:CIPNtoTPN}(a),
\verb!Send(destination,signal=value)! command within a place sends the updated \verb!value! of \verb!signal! to the \verb!destination! LC, while condition \verb!signal==value! on a transition within the model blocks execution until the signal corresponding to the desired value is received over the network. % (as shown in Fig.~\ref{fig:CIPNtoTPN}(a)).
To %solve the main challenge towards security-aware analysis, i.e.,
enable formal analysis of the attack impact on QoC of distributed automation, it is necessary to develop a TPN-compliant model of the interface (i.e., transceiver) between the controller and security-aware channel model; such model can be then composed with the TPN-based models described in Section~\ref{subsec:plantAndControllerInteraction}, % and \ref{sec:runtimemodel},
resulting in Fig.~\ref{fig:modeling}(b) architecture.

{
%There exist three main challenges to security-aware distributed automation modeling; i.e., a
Such security-aware formal model has to capture:
(1)~application-level (i.e., controller side) communication stack behavior, directly affected by
(2)~the channel-side (i.e., communication medium) attack model, and
(3)~the controller-channel interface.
%\end{enumerate}
Specifically, application-level (i.e., control-related) communication stack behavior, such as delays or blocking on communication peripheral resources, is of interest for security analysis, as this presents the main reflection of the communication-level attacks onto the control functionality. Therefore, when translating the CIPN communication model from Fig.~\ref{fig:CIPNtoTPN}(a) into a TPN-compliant model, it is necessary to capture application software states that directly affect progress of the control functionality, conditioned by data dependencies resolved via communication. Such models can be obtained from the actual application firmware running on the embedded LCs (i.e., source code). For example, when IEEE~\mbox{802.15.4} protocol is used, as in the case study presented in Sec~\ref{sec:evaluation}, the state-machine/TPN representation can be directly extracted from the radio driver (as done in Fig.~\ref{fig:CIPNtoTPN}(b)).
On the other hand, if more complex communication stack is considered (i.e., implementing higher communication layers), exiting state-machine extraction techniques (e.g.,~\cite{soteria}) can be employed.

Second, the channel model has to explicitly capture the channel states essential for supporting the attack models presented in Section~\ref{subsec:attackModel}---channel states that are not observable (or alterable) need not be modeled (e.g., bit-level signaling, or carrier-level modulation). Finally, a TPN-compliant interface between the controller and security-aware channel models is needed to allow for their formal composition, %of the controller and channel models
enabling system-level analysis of adversarial influence on the entire system. Therein, specific data link layer (OSI model layer 2) features are crucial for understanding retransmissions and ACK mechanics which, as we will show, affects design of attack detectors. Therefore, while controller models should capture application-level communication semantics, it is also necessary to include protocol-level details within the transceiver (XCVR) models, which act as the interface between the controllers and the medium (as shown in Fig.~\ref{fig:modeling}(b)). XCVR specifics are commonly available for the specific employed radio communication chip as RF circuitry control is usually state-machine based (e.g.,~referred to as the \emph{internal RF state machine}~\cite{mrf} in the case of radios used in our implementation).



On the other hand, explicit security-aware channel modeling is \emph{medium-, protocol-, and attack-dependent}. Fig.~\ref{fig:CIPNtoTPN}(c) and Fig.~\ref{fig:attackModels} show a security-aware model of a \emph{half-duplex, acknowledge-required unicast CSMA-CA-based communication channel} with respect to the previously defined attack model. While other medium/protocol variants can be easily modeled due to the expressiveness of TPNs, we consider this specific model as it applies to our physical setup described later in Section~\ref{sec:evaluation}. In the remaining of this section, we describe the transformation from the CIPN-based LC communication model to a TPN-compliant model assuming the aforementioned channel, while aiming to balance between the model expressiveness and capturing security-aware behavior required for analysis of QoC under attack.
}

%\todo{why is this here? you tell one story, high level list of things that need to be done, and all of a sudden there is a model of something that you didn't talk much about + what is `security-aware' here?}
%\todo{In general, I have troubles following this section, and how it relates to Fig. 5a, and 4a; After next subsection it goes better; what is also somewhat unclear here is what are the main challenges}




\begin{table}[t]
  \centering
    \begin{tabular}{p{1.6cm}|p{5.4cm}|p{0.4cm}}
      % after \\: \hline or \cline{col1-col2} \cline{col3-col4} ...
      Symbol & Description & SW acc.\\ \hline\hline
      \verb!ChBusy! & Indicator whether the channel is currently busy with a packet or ACK & YES \\ \hline
      \verb!N_RxBuf! & Local Rx buffer & YES \\ \hline
      \verb!N_RxAck! & Local flag indicating successful transmission, i.e., ACK reception & YES \\ \hline
      \verb!NXCVR_PTx! & Transceiver Tx payload buffer & YES \\ \hline
      \verb!NXCVR_PRx! & Transceiver Rx payload buffer & YES \\ \hline
      \verb!NXCVR_Tx! & Signal to XCVR initiating transmission & YES \\ \hline
      \verb!NXCVR_Txd! & Signal to XCVR indicating transmission & NO \\ \hline
      \verb!NXCVR_Rx! & Signal from XCVR indicating reception & YES \\ \hline
      \verb!NXCVR_TxAck! & XCVR signal initiating ACK transmission & NO \\ \hline
      \verb!NXCVR_RxAck! & XCVR signal indicating ACK reception & NO \\ \hline
      \verb!NXCVR_TxCnt! & XCVR retry counter & NO \\ \hline
      $\underline{t}_{Tx}^{Msg},\overline{t}_{Tx}^{Msg}$ & Message transmission time (bounds) & --- \\ \hline
      $\underline{t}_{Tx}^{Ack},\overline{t}_{Tx}^{Ack}$ & ACK transmission time (bounds) & --- \\ \hline
      $\underline{t}_{Tx}^{Boff},\overline{t}_{Tx}^{Boff}$ & Back-off time (bounds) & --- \\ \hline
      $t_{Tx}^{AckTO}$ & Data link layer ACK timeout & --- \\ \hline
      $t_{wfAck}$ & Application-level ACK timeout & --- \\ \hline
      $\underline{t}_{Ch}^{DoS},\overline{t}_{Ch}^{DoS}$ & Contention time due to DoS (bounds) & --- \\ \hline\hline
    \end{tabular}
  \caption{Symbols used in Fig.~\ref{fig:CIPNtoTPN} and Fig.~\ref{fig:attackModels}; third column (where applicable) indicates accessibility to application software (or only to the transceiver's internal RF state machine).}\label{tab:CIPNtoTPNsymbols}
\end{table}
%

\subsection{Security-Aware Modeling of the Channel and Controller-Channel Interaction}

Fig.~\ref{fig:CIPNtoTPN}(b) shows the TPN transmitter/receiver models that replace the platform-independent CIPN transmitter/receiver model in Fig.~\ref{fig:CIPNtoTPN}(a). Fig.~\ref{fig:CIPNtoTPN}(c) shows the nominal channel model (i.e., without adversarial influences), while Fig.~\ref{fig:CIPNtoTPN}(d) shows the transceiver (XCVR) model. Notice that both $LC_A$ and $LC_B$ have identical transceivers; thus, $N\in\{A,B\}$ in place/transition names. Table~\ref{tab:CIPNtoTPNsymbols} enumerates symbols (local flags, variables, and transition timing parameters) used in the models in Fig.~\ref{fig:CIPNtoTPN},~\ref{fig:attackModels}. The internal RF state machine can be in the \emph{listening}, \emph{transmitting a packet}, \emph{waiting for acknowledgement}, or \emph{transmitting an acknowledgement} states. The transceiver employed in our case study (in Sec.~\ref{sec:evaluation}), performs up to three retransmissions before signaling a transmission failure to the application. On the application level, an unbounded number of retransmissions are performed in case the transceiver returns failure. The TPN model in Fig.~\ref{fig:CIPNtoTPN}(b-d) models this interaction.


%%%% SOME DETAILED EXPLANATIONS GOING THROUGH THE NETS ARE HERE---KEEP HERE
%Initially, when the receiver's (i.e., $LC_B$'s) token is deposited into place \verb!Pb_wfRx!, the receiver is blocked on transition \verb!Tb_wfRx! waiting to receive signal \verb!s! with value \verb!1! (i.e., transition is guarded by receive buffer condition \verb!G:B_RxBuf==1!). On the other hand, the clear channel assessment (CCA) is performed by the transmitter ($LC_A$), when its token is deposited into the place \verb!Pa_wfTx!. If the channel is not busy (i.e., guard \verb!G:ChBusy==0! on transition \verb!Ta_wfTx! is true), the transmitter's token transitions to place \verb!Pa_wfAck!, where a transmission of signal \verb!s! with value \verb!1! is initiated (i.e., XCVR transmit payload buffer is initialized (\verb!U:AXCVR_PTx=1!) and transmission is initiated (\verb!U:AXCVR_Tx=1!)). As a result of the initiated transmission, after a random back-off time in the range $[\underline{t}_{Tx}^{B-off},\overline{t}_{Tx}^{B-off}]$ elapses, $LC_A$'s XCVR transitions from \emph{listen} to \emph{transmit} mode (i.e., \verb!PaXCVR_Tx!). Consequently, the transition \verb!Tch_wfTx! in the channel model is enabled and fires; message transmission takes non-deterministic time in the range $[\underline{t}_{Tx}^{Msg},\overline{t}_{Tx}^{Msg}]$, after which the channel can return to the idle place (i.e., transition over \verb!Tch_TxMsgFin! back to \verb!Pch_Idle!). This enables $LC_A$'s XCVR to switch to waiting for the ACK, while internally increasing the number of retransmissions (through update function \verb!U:AXCVR_TxCnt++!). An ACK is expected within the timeout $[t_{Tx}^{AckTO},t_{Tx}^{AckTO}]$.
%
%On the other hand, $LC_B$'s XCVR
%
%
% as the channel is declared busy and $LC_A$ has released the packet to the radio (i.e., \verb!G:M(Pa_wfAck)==1! is satisfied). $LC_A$ dwells in \verb!Pa_wfAck! until either the acknowledgement packet is successfully received on the transition \verb!Ta_ackSucc! (i.e., guard \verb!G:A_RxAck==1! becomes satisfied), or a timeout transition \verb!Ta_ackFail! fires (i.e., $t_{wfAck}$ elapses) after which the transmission is retried. The channel model now dwells in \verb!Pch_TxMsg!.
%
%The message transmission takes non-deterministic time in the range $[\underline{t}_{Tx}^{Msg},\overline{t}_{Tx}^{Msg}]$, after which the channel can return to the idle place (i.e., transition over \verb!Tch_TxMsgFin! back to \verb!Pch_Idle!), triggering update functions that mark $LC_A$'s transmission as finished, update channel state, and allow $LC_B$ to use the transmitted signal (i.e., \verb!G:A_sTxd=1!, \verb!ChBusy=0!, and \verb!B_sRx=A_sTx!, respectively). Consequently, $LC_B$ is no longer blocked on \verb!Tb_wfRx! as \verb!G:B_sRx==1! is satisfied (i.e., signal \verb!s==1! is received), and can acknowledge reception and continue execution. When $LC_B$ transitions over \verb!Tb_wfRx!, the channel is immediately set to busy and the ACK transmitted (via update functions \verb!U:ChBusy=1! and \verb!U:B_TxAck=1!), as CCA is not performed for ACK packets. As a result, \verb!Tch_wfAck! is enabled and the channel becomes busy transmitting the ACK by transitioning to place \verb!Pch_TxAck!. The ACK transmission takes non-deterministic time in the range $[\underline{t}_{Tx}^{Ack},\overline{t}_{Tx}^{Ack}]$, after which the channel returns to place \verb!Pch_Idle!, when update functions are invoked updating channel state, and variable \verb!A_RxAck! allowing the transmitter to resume execution.
%
%Nominally, the ACK process takes less than $t_{wfAck}$, and thus $LC_A$ successfully receives the ACK; it consequently transitions over \verb!Ta_ackSucc! as guard \verb!G:A_RxAck==1! is satisfied, which concludes the transmission process. Notice that $LC_A$'s local variable \verb!A_sTxd! (which is initially \verb!0!, is set in the channel model when $LC_A$'s message is transmitted, and reset by $LC_A$ when an ACK is not received) is not needed but will be used in Sec.~\ref{subsubsec:attackModeling} when attacks are introduced. Additionally, initial values of all model variables are \verb!0!. In the following, we show how this communication model can be extended to include adversarial influences defined in Sec.~\ref{subsec:channelAndCtrlChannelInteraction}.
%%%% SOME DETAILED EXPLANATIONS GOING THROUGH THE NETS ARE HERE---KEEP HERE

%
\begin{figure*}[t]
\vspace{-10pt}
	\centering
	\includegraphics[width=0.98\textwidth]{attackModels.pdf}
	\caption{Additional places, transitions, and arcs required to obtain a security-aware channel model for different attack types: (a) DoS, (b) ACK intercept/spoof, (c) message modification, and (d) masquerade. All attack-related components of the model are depicted in red color, while nominal components are shown partially for completeness in black (where relevant).}
	\label{fig:attackModels}
\end{figure*}
%
%\subsubsection{Attack Modelling}
%\label{subsubsec:attackModeling}
%
In the remaining of this section, we show how  the attacks described in Section~\ref{subsec:channelAndCtrlChannelInteraction} can be modeled as TPNs. Specifically, we describe additional places, transition, and arcs to be added to the nominal channel model shown in Fig.~\ref{fig:CIPNtoTPN}(c) to capture the attacks. To enhance model readability, Fig.~\ref{fig:attackModels} depicts only additional places and transitions in red color required to model a specific attack, while the nominal places and transitions are depicted in black (all parts of the nominal model not relevant for the specific attack are omitted therein).

%%%
%%%
%%%
\paragraph{DoS attack submodel} Fig.~\ref{fig:attackModels}(a) shows the DoS attack submodel. When the channel is idle, the attacker may decide to occupy the channel to prevent legitimate transmissions. He/she may do so at any time (non-deterministic choice) when the channel is not busy, and keep the channel busy arbitrarily long. In the model, the channel is kept busy for some non-deterministic time in the range $[\underline{t}_{Ch}^{DoS},\overline{t}_{Ch}^{DoS}]$, after which it is released by the attacker.% Packet and ACK interceptions are included in the following submodels.

%%%
%%%
%%%
\paragraph{ACK interception/spoofing submodel} Fig.~\ref{fig:attackModels}(b) shows the ACK intercept/spoof submodel. To model ACK interception, an additional transition is needed allowing the channel to return to idle state following ACK transmission, without the transmitter ($LC_A$) receiving the ACK sent by the receiver ($LC_B$); i.e., transition \verb!Tch_TxAckInt! is added as shown in Fig.~\ref{fig:attackModels}(b), and is \emph{not} associated with the update function \verb!U:BXCVR_RxAck=1!.
However, when ACK spoofing is considered, the attacker may transmit an ACK when the targeted receiver \emph{is not} in the process of acknowledging, while the targeted transmitter \emph{is} in the process of waiting an ACK.

Additionally, malicious ACK spoofing may be performed when the signal to which the ACK is intended to correspond is transmitted already, but the ACK has not yet been received by the sender (e.g., due to an intercepted ACK from the legitimate receiver). This is enabled with the additional net branch in Fig.~\ref{fig:attackModels}(b) starting with transition \verb!Tch_wfAckImp!. As a result of firing of this transition, the channel is declared busy and the spoofed ACK is assumed to take the same time as transmitting legitimate ACKs; thus,  the transition \verb!Tch_TxAckImp! has the same attributes as \verb!Tch_TxAck!, with the exception of the signal to the targeted transmitter signalling ACK transmission is done (i.e., update \verb!U:AXCVR_RxAck=1! is omitted).

%%%
%%%
%%%
\paragraph{Message intercept/modify submodel} Fig.~\ref{fig:attackModels}(c) shows the message intercept/modify submodel, where an additional transition \verb!Tch_TxMsgMod! (\verb!Tch_TxMsgInt!), {represented as one transition for conciseness}, is added. In the case of the modification attack, this transition in the model allows the attacker to deliver a signal different form the one originally transmitted (i.e., \verb!U:BXCVR_PRx=Pmod! where \verb!Pmod! is the payload modified by the attacker). In the case of packet interception, no update to the receiver's XCVR receiver buffer is made, and consequently the XCVR is not notified of a received packet (i.e., update functions are omitted and denoted as \verb!(.)! in Fig.~\ref{fig:attackModels}(c)).


%%%
%%%
%%%
\paragraph{Message impersonation submodel} Fig.~\ref{fig:attackModels}(d) presents the masquerade submodel. The additional transitions and places allow the attacker to make a non-deterministic choice to impersonate transmission of the expected transmitter whenever the channel is not busy, the targeted receiver is waiting for the corresponding signal, and the original transmitter is not in the process of sending this signal. Then, similarly to the nominal (legitimate) transmission model (shown in Fig.~\ref{fig:CIPNtoTPN}(c)), the transmission takes a non-deterministic time in the same range as legitimate transmissions. Note that the received payload on $LC_B$ is in this case the value \verb!Pinj! crafted by the attacker, rather than \verb!AXCVR_PTx!, normally transmitted by $LC_A$ in the adversary-free case.


%%%
%%%
\begin{remark}[Replay attacks]
Due to the introduced non-determinism, any specific sequence of attack actions are contained within the presented model (as long as the individual actions correspond to the attack model from Section~\ref{subsec:channelAndCtrlChannelInteraction}). Thus, replay attacks are covered by the presented model as they are only specific executions of the presented security-aware channel model. On the other hand, using a similar approach, finite memory replay attacks can be captured by a model that restricts inserted attack signals only to the previously transmitted messages, as done in~\cite{wang_arxiv19}.\QEDE
\end{remark}
%%%
%%%
\begin{remark}[Controller-plant VS. controller-channel interface modeling fidelity]
Control interface to the channel is modeled in far more detail than the interface to the plant, by abstracting away locally-connected actuator drives, relays, analog amplifiers, etc. The reason is that, in this work, we do not consider physical plant-level attacks. Hence, modeling the controller-plant interaction at a lower level of abstraction would unnecessarily increase model complexity. However, the presented techniques can be easily extended and the framework fully adapted to also cover physical attacks on the plant.\QEDE
\end{remark}
%%%
%%%

















%\subsection{Controller Runtime State Modeling}
%Finally, the last remaining construct in the CIPN-based controller model to be made TPN-compliant are places issuing system calls provided by the runtime environment (e.g., execution delay) or updating local controller state (e.g., manipulating global variables). Execution delays requested from the runtime environment can easily be modeled as timed transition with exact firing time (i.e., where the lower and upper firing time bound are the same, as in the example for ACK timeout). While in this work we consider single-threaded sequential automation examples, existing techniques for modeling parallel systems can be applied given the rich expressiveness of TPNs~\cite{CIPNtoTPN}. For example, in the case of multithreaded applications where task preemption is allowed, the operating system scheduler can be modeled as a separate component, even in the case of multi-processor platforms~\cite{TPNscheduling}. %or the upper bound on the firing time of a specific transition adjusted to include worst-case blocking time incurred from other lower-priority workload~\cite{}.
%%
%On the other hand, definition of global variables and functions is supported by state-of-the-art TPN tools (e.g.,~\cite{romeo}). Update functions that are invoked on firing of the transition to which the function is assigned and guard functions are generally not only marking- but also global variable-dependent, which allows for rich modeling of complex internal LC state updates, similarly as used to implement the controller-plant and controller-channel interface in Sec.~\ref{subsubsec:controllerPlantInteraction}~and~\ref{subsec:channelAndCtrlChannelInteraction}. 
% !TEX root = DistrSeqCtrlAttacks_TASE19.tex

%


%
\section{Resiliency Analysis and Security Patching}
\label{sec:verification}
%
A security-aware closed-loop system model obtained by composing the developed security-aware TPN models can be used to verify system-level safety and QoC properties in the presence of attacks. TPN analysis tools (e.g.,~\cite{romeo,tina}) allow for verification of formal properties specified as Linear Temporal Logic (LTL), Computational Tree Logic (CTL), or Timed CTL (TCTL) formulas~\cite{modelChecking}. In this work, we employ the tool Romeo~\cite{romeo} that enables verification of TCTL-based formal queries, such as  traditional safety (e.g., 1-boundedness~\cite{David20101}) and liveness properties (e.g., absence of deadlock).
In addition, as plant models are included, we can specify relevant domain-related plant-state-bound properties that are crucial for functional safety and QoC assessment. For our running example, the considered properties include: %as~follows.


\begin{property}\label{prop:p1}
  A workpiece on conveyor~1~never triggers a pick-up from conveyor~2; this can be formally captured as:
 \verb!AG(not(M(Pp&p_P&P2)==1 and! \verb! activeConveyor==1))!.\footnote{Variable \verb!activeConveyor! is set when a workpiece presence is detected (i.e., on transitions \verb!Tcm_Pres1! or \verb!Tcm_Pres2! of the conveyor monitor, shown as CIPN in Fig.~\ref{fig:exampleCIPN}) and reset when conveyor monitor returns to initial state (i.e., over \verb!Tcm_RetInit!)}, where \verb!A! and \verb!G! are quantifiers signifying formula satisfaction along \emph{all} paths and \emph{always} (i.e., along all subsequent paths), respectively.
\end{property}

\begin{property}\label{prop:p2}
  A workpiece detected on any of the conveyors is eventually picked-up: \verb!(M(Pcm_TxCtrl_Pick1)==1 or! \verb!M(Pcm_TxCtrl_Pick2)==1)-->(M(Pp&p_wfRet)==1)!, where \verb!-->! denotes the {"leads to"} property; i.e., \verb!p-->q! means that for all executions, continuous satisfaction of property \verb!p! implies always eventual satisfaction of property \verb!q!, or formally \verb!AG(p => AF(q))!.%\LE{showcase for DoS}
\end{property}

\begin{property}\label{prop:p3}
  The pick~\&~place station does not commence cycle (i.e., it is neither in \verb!Pp&p_P&P1! nor \verb!Pp&p_P&P2!), while the conveyor monitor is waiting for incoming workpieces (i.e., in the place \verb!Pcm_Init!). Formally, \verb!AG(M(Pcm_Init)+M(Pp&p_P&P1)+M(Pp&p_P&P2)<=1)!.%\LE{showcase for injection - MAC solves this}
\end{property}

Using the Romeo tool, we verified that these as well as other QoC- and safety-critical properties are \emph{not} satisfied in the presence of attacks, as the attacker is capable of significantly altering the intended interaction between LCs, at arbitrary moments in time. For example, Property~\ref{prop:p1} is violated under \emph{message modification} attacks, Property~\ref{prop:p2} under possible infinite \emph{DoS}, while Property~\ref{prop:p3} fails under \emph{spoofing}.
The aforementioned properties are violated regardless of the values of timing parameters used in the model. Note that bounds on time-to-transmit and time-to-acknowledge can be obtained from experimental measurements, or directly from network specifications. Also, transceiver-related timings (e.g., back-off time during clear channel assessment) can be obtained from the employed transceivers' specifications. % Notice that aforementioned properties are violated regardless of these timing parameters.}




Regarding verification scalability---in the system model, one nominal pick~\&~place cycle,~with all attacks disabled, contains around $35$ transitions, which is on the order of the number of states in the model. Model complexity increases with the addition of non-deterministic attack choices, besides the time-induced non-determinism (in the plant model).\footnote{Romeo does not output statistics of the state space underlying the model.} Yet, in all cases, the tool takes less than $1~s$ to find an execution path violating the properties, on a workstation with an Intel~i7-8086K CPU ($4~GHz$ clock) and $64~GB$~memory.

\subsection{Addressing the Discovered Vulnerabilities}
As our previous analysis have shown, attack actions may significantly affect performance of distributed IoT-based industrial automation systems; to address them, it is necessary to add certain security mechanisms. In this section, we discuss how such security mechanisms affect  system models and verifiability of the relevant~properties.


\begin{figure}[!t]
	\centering
	%\vspace{-4pt}
	\includegraphics[width=0.48\textwidth]{MACandDOSmodels.pdf}
	\caption{Model adaptation to addition of security services; (a) for DoS detection, and (b) against spoofing. Additional places and transitions are shown in blue color.}
	\label{fig:MACandDOSmodels}
\end{figure}
%
\begin{figure}[!t]
	\centering
	%\vspace{-8pt}
	\includegraphics[width=0.44\textwidth]{histograms}
	\caption{Tx-to-Rx and Rx-to-Ack times measured on our IEEE~802.15.4-enabled LC platform described in Section~\ref{sec:verification}.}
	\label{fig:histograms}
\end{figure}
%


\subsubsection{Detecting Denial-of-Service Attacks}
\label{subsubsec:DoSdetection}
Packet and acknowledgement (ACK) dropouts are common in wireless communication, and hence ACK and retransmission mechanisms are commonly used in such setups. For instance, in our experimental setup described in Sectio~\ref{sec:evaluation}, ACK request can be disabled in transceiver settings, in which case no retransmissions are attempted on the data link layer. For two isolated transceivers, this amounts to the one-way packet success rate of approximately $99~\%$ (see histograms in~Fig.~\ref{fig:histograms} that exclude unsuccessful transmissions). Thus, when ACK requests are enabled, up to three data-link layer retransmissions are performed,\footnote{The XCVR model in Fig.~\ref{fig:CIPNtoTPN}(d) is compatible with these specifications.} and we experimentally observed that no application-level retries are required beyond the three low-level protocol-provided retransmissions, in the case when a single industrial machine operates in isolation.

On the other hand, to increase network utilization, we emulated a number of additional machines communicating over the same wireless channel in physical vicinity (described in more detail in Section~\ref{sec:evaluation}); we experimentally observed the one-way packet success rate of approximately $98~\%$. Thus, two application-level retries were sufficient to enable reliable exchange of events, ensuring correct operation. Intuitively, protocol-provided retries are issued in short bursts while application-level retransmissions incur significant delay; the channel is more likely to be continuously busy for a short period of time (e.g., occupied by other legitimate transmissions). Yet, an adversary may repeatedly deny network access to legitimate controllers preventing the system from progressing. {Consequently, the modeled system does not satisfy Property~\ref{prop:p2}, despite application-level retransmissions, unless DoS attacks can be detected and system halted (or other precautionary actions taken), using e.g., a separate secure channel.}

From the operational perspective, every LC may implement a limited number of successive application-level retransmissions before declaring that it is under attack. For instance, if in our setup from Section~\ref{sec:evaluation}, we limited the number of retransmissions to five, amounting to a theoretical one-way packet success rate of \emph{eight nines} (if application-level retransmissions are assumed to be independent). To address this from the modeling perspective, we add an additional place where the transmitter's model transitions to, when application-level retries are exhausted (see Fig.~\ref{fig:MACandDOSmodels}(a)). Hence, we can verify that if infinite blocking of medium access is allowed, LCs may end up in the place \verb!Pa_DoSdetect!. Conversely, if DoS attacks are limited to four consecutive channel access denials, Property~\ref{prop:p2} is satisfied. Note that immediate emergency halt of the machinery may not be possible if a secure communication channel is not available or the DoS attacks cannot be isolated from the network (e.g., using bus guardians).


\subsubsection{Authenticating network flows}
\label{subsubsec:authentication}
Traditional cryptographic techniques for ensuring integrity of network flows rely on signing packets with Message Authentication Codes (MAC)~\cite{802.15.4auth}, and can be used to defend against spoofing attacks. In this setting, every transmission between LCs is signed by the transmitter using a secret key, and the signature is verified by the receiver; therefore, the attacker cannot tamper with the message payload, or else he/she will be detected.

From the modeling perspective, introducing authentication can be modeled as an additional condition on the receiving transitions (in the controller models) where the received payload is compared to desired values; i.e., the MAC portion of the payload is compared to a secret value that cannot be altered (in the case of modification) or generated (in the case of spoofing attacks) by the attacker. Specifically, the transition \verb!Tb_wfRx! in the $LC_B$ model in Fig.~\ref{fig:CIPNtoTPN}(b) would feature an additional guard function on the \verb!B_RxMAC! variable. Optionally, if the signature verification fails, a transition to a place modeling intrusion detection reaction can be added as shown in Fig.~\ref{fig:MACandDOSmodels}(b); this is left to the application designer as reacting to detected intrusions is highly application-specific.

Using the developed framework, we verified that if non-authenticated transmissions are not allowed (i.e., authentication implemented), Properties~\ref{prop:p1}~and~\ref{prop:p3} can be verified over our running example, under the condition that infinite denial of network access to LCs is not allowed, as previously discussed.


\subsubsection{Acknowledgement Spoofing}
Authenticating transmission does not affect ACKs as the data-link layer is responsible for ACK packets while MACs are added to the packet payload. In addition, non-encrypted sequence numbers, which are part of the packet frame, can be overheard by the attacker. Thus, valid ACKs can be generated on behalf of inactive (failed) LCs. Also, undelivered (i.e., intercepted) transmissions can be falsely acknowledged, even when authentication is used. This is a well-known shortcoming of data link layer ACKs~\cite{ackattack2,802.15.4sec}, and could be alleviated by application-level ACKs. Enforcing consensus over event-propagation in discrete event systems spans beyond the scope of this paper; yet, the presented modeling techniques can be utilized to model additional implemented protocols.

While this section introduced the general security-aware modeling aspects, with occasional focus on specific medium access techniques to avoid overly general discussions, in the following section we demonstrate the use of the presented framework on real-world industrial case studies.

% !TEX root = DistrSeqCtrlAttacks_TASE19.tex


\section{Case Studies: Industrial Manipulators}
\label{sec:evaluation}
We consider a full physical implementation of a reconfigurable industrial pneumatic manipulator with a variable number of modules/degrees~of~freedom (DOF) controlled in a distributed fashion; i.e., one local controller per module/DOF. We demonstrate effectiveness of our framework on multiple module configurations (i.e., 2-DOF, 3-DOF).

%
\begin{figure}[t]
	\centering
	\includegraphics[width=0.49\textwidth]{caseStudies}
	\caption{Pneumatic manipulator in multiple configurations: (a) 2-DOF~pick \&~place configuration; (b) 3-DOF pick-immerse-shake-return configuration; (c,1) upper portion of the physical setup of the configuration (b) shows cylinders; (c,2) low-cost ARM Cortex-M3-based networked controller; each physical component (cylinders and the gripper) are equipped with one LC.}%; (d) parallel loading/unloading configuration.}%---parts are loaded by cylinder A, and unloaded by cylinder B simultaneously, while the rotary cylinder C `replaces' the loading/unloading positions by means of rotating the loading base.}
	\label{fig:caseStudies}
\end{figure}
%

\subsection{2-DOF Industrial Pneumatic Manipulator}
The pneumatic industrial manipulator in the 2-DOF configuration is depicted in Fig.~\ref{fig:caseStudies}(a); two double-acting cylinders (denoted $A$ and $B$) provide translational degrees of freedom, while the pneumatic gripper (denoted $C$) provides means of handling the workpiece. All actuation commands are issued by updating electrical signals \verb!xp!, $\verb!x!\in\{\verb!a!,\verb!b!,\verb!c!\}$ which activate monostable dual control pneumatic valves.\footnote{ A control valve is the interface between the controller and the pneumatic cylinder; it converts the actuation signal from the controller into mechanical movement that controls flow of pressured air towards pneumatic cylinders.} Notice that signals are denoted with \verb!x! while cylinders are denoted with $X$. Cylinders $A$ an $B$ are equipped with two proximity switches which allow position (i.e., fully retracted, fully extended) sensing. Signals corresponding to fully retracted (home) position are denoted \verb!x0!, while fully extended (end) position signals are denoted \verb!x1!. Additionally, the system contains a start switch whose corresponding signal is denoted by \verb!st!.


%
\begin{figure}[t]
	\centering
	\includegraphics[width=0.498\textwidth]{CIPNdistrExample.pdf}
	\caption{CIPN-based {distributed} controller of a 2-DOF pneumatic manipulator.}
	\label{fig:CIPNdistrCtrl}
\end{figure}
%



CIPN-based models of three LCs are shown in Fig.~\ref{fig:CIPNdistrCtrl}. Initially, cylinders $A$ and $B$ are fully retracted, and gripper $C$ released---in this state the manipulator is ready to begin its work cycle. The initial work cycle of the manipulator is started by pressing the start switch (\verb!st==1!), after which operation is fully automated. First, cylinder $B$ extends towards the workpiece picking position (due to actuation command \verb!bp=1!). Once cylinder $B$ reaches its end position (\verb!b1==1!), gripper $C$ is commanded gripping (\verb!cp=1!). Controller $B$ waits for $500~ms$ for the part to be gripped.\footnote{The gripper $C$ does not have end position sensing due to size constraints; thus a timed delay is used to permit secure gripping/releasing of the workpiece.}
Then, cylinder $B$ retracts (due to command \verb!bp=0!), and once it reaches home position (\verb!b0==1!), cylinder $A$ extends (due to command \verb!ap=1!). After reaching its end position (\verb!a1==1!), cylinder $B$ extends towards the placing position (due to command \verb!bp=1!). Once it reaches its end position (\verb!b1==1!), gripper $C$ is commanded release of the workpiece (command \verb!cp=0!). $500~ms$ later, cylinder B retracts (\verb!bp=0! followed by \verb!b0==1!), after which cylinder $A$ retracts (\verb!ap=0! followed by \verb!a0==1!). The manipulator returnees into its initial state, after which the next cycle is automatically executed. Signals (i.e., sensors outputs and actuator inputs) are allocated to LCs according to their physical proximity: $\{\verb!ap!,\verb!a0!,\verb!a1!\}$ are mapped to controller A (i.e., $LC_A$), $\{\verb!bp!,\verb!b0!,\verb!b1!,\verb!st!\}$ to controller B ($LC_B$), and $\{\verb!cp!\}$ to controller C ($LC_C$).

TPN models are obtained from these specifications as described in Section~\ref{sec:modeling}, but are omitted here due to their size. On the other hand, pneumatic cylinders are modeled as two-state plants with bounded, non-deterministic extending/retracting times obtained from experimental measurements. We extract timing parameters (i.e., bounds on time-to-transmit, time-to-acknowledge, and back-off timing) from experimental measurements---histograms for $10,000$ messages are shown in Fig.~\ref{fig:histograms}, for the employed low-cost ARM Cortex-M3-based controllers equipped with an IEEE~802.15.4-compliant transceiver. On the other hand, we obtain transceiver-related timings (e.g., back-off time during clear channel assessment) from the radio specifications~\cite{mrf}. While we verified a large number of safety and liveness properties for this setup, we~illustrate~verification and security patching on a more complex 3-DOF setup.

\begin{figure}[t]
	\includegraphics[width=0.49\textwidth,left]{nomPlot}\vspace{5pt}
	\includegraphics[width=0.49\textwidth,left]{injPlot}\vspace{5pt}
	\includegraphics[width=0.49\textwidth,left]{intPlot}
	\caption{Sensing/actuation signal timings for a nominal pick~\&~place run (a), a run where a signal injection is performed resulting in a dropped workpiece (b), and a run where progress is inhibited due to a DoS attack (c). Messages exchanged by LCs are marked with blue arrows. $X$ axis is unlabeled as the speed of the workcycle can be controlled by regulating air pressure in the system and is thus not crucial.}
	\label{fig:plots}
\end{figure}


\subsection{3-DOF Industrial Pneumatic Manipulator}
A 3-DOF configuration of the described manipulator is shown in Fig.~\ref{fig:caseStudies}(b). The additional rotational DOF, provided by cylinder C, introduces an additional LC and increases the complexity of the LC coordination. This configuration may be used to prepare workpieces for painting by immersing them into a pool with cleaning solution, and returning them to the pick-up position for further processing by another machine.

Fig.~\ref{fig:caseStudies}(c,1) shows the physical setup for  this configuration; the upper portion of the manipulator is shown such that cylinders are visible. Fig.~\ref{fig:caseStudies}(c,2) shows the low-cost ARM Cortex-M3-based LC with the corresponding IEEE~802.15.4 transceiver. While the models are more complex than in the 2-DOF case, they are semantically similar and thus omitted. Fig.~\ref{fig:plots}(a) shows event timing---i.e., states of all sensing and actuation signals, for one sample pick-immerse-shake-return run; messages exchanged by LCs are denoted with blue arrows originating at the source event and terminating at the triggered event. Among the many safety liveness and QoC properties, we illustrate verification on the following examples.
%
\begin{property}\label{prop:caseStudyP1}
Gripper D is always gripped before cylinder B picks the workpiece; formally captured as, \verb!AG(M(PdGRIP_Gripped)==1 and!\\\verb!M(PbCTRL_bCYL_Retract1)==1)!.
\end{property}
%
\begin{property}\label{prop:caseStudyP2}
A workpiece is eventually processed, once the work cycle is started. Formally, \verb!M(PbCTRL_bCyl_Extend1)==1-->!\\\verb!(M(PcCTRL_cGRIP_Release)==1!.
%Gripper D is never released while cylinder A is moving. Formally,\\
%\verb!AG(not(M(PdGRIP_Gripped)==0 and M(PaCTRL_aCYL_Extend)==1) or!\\\verb!(M(PdGRIP_Gripped)==0 and M(PaCTRL_aCYL_Retract)==1))!.
\end{property}
%

When no security mechanisms are employed, we verified violation of these properties.  Property~\ref{prop:caseStudyP1} is violated due to a possible impersonation attack at the gripper controller; an attacker may send the command to release the workpiece before it was returned to the return position. Fig.~\ref{fig:plots}(b) shows signal timings acquired on a sample cycle run in which the workpiece is dropped due to a maliciously injected command to release the gripper (potentially causing mechanical damage to the workpiece and/or the manipulator).

However, if transmissions are authenticated, and the model adjusted correspondingly as described in Section~\ref{subsubsec:authentication}, this vulnerability is alleviated. We applied a software security patch by including the \emph{mbed TLS} (Secure Sockets Layer) library that our IIoT controllers are fully compatible with. Signing a $128~bit$ message authentication code over one transmitted signal incurs computational overhead of $\sim100~\mu s$ on the employed low-cost ARM Cortex-M3-based LCs; this practically negligibly slows down manipulator's work cycle, while providing security guarantees. Hence, Property~\ref{prop:caseStudyP1} is satisfied following this security patch.

Property~\ref{prop:caseStudyP2} is violated due to the possibility of a DoS attack that infinitely delays progress. From the model's perspective, this attack does not cause a deadlock---while the \emph{physical} process is stalled, the \emph{cyber} process is in fact livelocked reattempting to access the channel (i.e., same places are revisited and same transitions fire infinitely often). Fig.~\ref{fig:plots}(c) shows signal timings acquired on a sample run where a DoS attack is launched by enabling carrier transmission on the attacker's transceiver, in order to jam messages after the workpiece was picked up from the immersion pool. As described in Section~\ref{subsubsec:DoSdetection}, wireless control nodes can keep track of unsuccessful medium access attempts, and promptly halt operation when a DoS attack is detected.
%
%As noted in Sec.~\ref{subsubsec:DoSdetection},
In such cases, distributing the information about DoS detection requires a secure channel, which we do not consider in this work.

%Note that while a jamming attack on a wireless medium is hard to mitigate without additional channel support, certain protocols designed for wired media support access control based on design-time reservations (e.g., Flexray), increasing the bar for the attacker's actions required to mount a successful DoS.



%\paragraph{Industrial Pneumatic Manipulator With Parallel Processes}
%Fig.~\ref{fig:caseStudies}(d) shows a configuration with parallel processes; i.e., workpieces are simultaneously loaded/unloaded by cylinders $A$ and $B$ while the two workpieces are ``replaced" by a rotating base attached to rotary cylinder $C$







\section{Conclusion}
We have presented a neural performance rendering system to generate high-quality geometry and photo-realistic textures of human-object interaction activities in novel views using sparse RGB cameras only. 
%
Our layer-wise scene decoupling strategy enables explicit disentanglement of human and object for robust reconstruction and photo-realistic rendering under challenging occlusion caused by interactions. 
%
Specifically, the proposed implicit human-object capture scheme with occlusion-aware human implicit regression and human-aware object tracking enables consistent 4D human-object dynamic geometry reconstruction.
%
Additionally, our layer-wise human-object rendering scheme encodes the occlusion information and human motion priors to provide high-resolution and photo-realistic texture results of interaction activities in the novel views.
%
Extensive experimental results demonstrate the effectiveness of our approach for compelling performance capture and rendering in various challenging scenarios with human-object interactions under the sparse setting.
%
We believe that it is a critical step for dynamic reconstruction under human-object interactions and neural human performance analysis, with many potential applications in VR/AR, entertainment,  human behavior analysis and immersive telepresence.








%-- \PA{@Vuk: related work needs to be added, especially on the use of PNs for network/security modeling}


% if have a single appendix:
%\appendix[Proof of the Zonklar Equations]
% or
%\appendix  % for no appendix heading
% do not use \section anymore after \appendix, only \section*
% is possibly needed

% use appendices with more than one appendix
% then use \section to start each appendix
% you must declare a \section before using any
% \subsection or using \label (\appendices by itself
% starts a section numbered zero.)
%


%\appendices
%\section{Proof of the First Zonklar Equation}
%Appendix one text goes here.

% you can choose not to have a title for an appendix
% if you want by leaving the argument blank
%\section{}
%Appendix two text goes here.


% use section* for acknowledgement
\section*{Acknowledgment}
This work is sponsored in part by the ONR under agreements N00014-17-1-2012 and N00014-17-1-2504, as well as the NSF CNS-1652544 grant. It was also partially supported by Serbian Ministry of Education, Science and Technological Development, research grants TR35004 and TR35020.

% Can use something like this to put references on a page
% by themselves when using endfloat and the captionsoff option.
\ifCLASSOPTIONcaptionsoff
  \newpage
\fi



% trigger a \newpage just before the given reference
% number - used to balance the columns on the last page
% adjust value as needed - may need to be readjusted if
% the document is modified later
%\IEEEtriggeratref{8}
% The "triggered" command can be changed if desired:
%\IEEEtriggercmd{\enlargethispage{-5in}}

% references section

% can use a bibliography generated by BibTeX as a .bbl file
% BibTeX documentation can be easily obtained at:
% http://www.ctan.org/tex-archive/biblio/bibtex/contrib/doc/
% The IEEEtran BibTeX style support page is at:
% http://www.michaelshell.org/tex/ieeetran/bibtex/
%\bibliographystyle{IEEEtran}
% argument is your BibTeX string definitions and bibliography database(s)
%\bibliography{IEEEabrv,../bib/paper}
%
% <OR> manually copy in the resultant .bbl file
% set second argument of \begin to the number of references
% (used to reserve space for the reference number labels box)
%\begin{thebibliography}{1}
%
%\bibitem{IEEEhowto:kopka}
%H.~Kopka and P.~W. Daly, \emph{A Guide to \LaTeX}, 3rd~ed.\hskip 1em plus
%  0.5em minus 0.4em\relax Harlow, England: Addison-Wesley, 1999.
%
%\end{thebibliography}


\bibliographystyle{IEEEtran}
{
    %\small
    \bibliography{DistrSeqCtrlAttacks_TASE19}
}

% biography section
%
% If you have an EPS/PDF photo (graphicx package needed) extra braces are
% needed around the contents of the optional argument to biography to prevent
% the LaTeX parser from getting confused when it sees the complicated
% \includegraphics command within an optional argument. (You could create
% your own custom macro containing the \includegraphics command to make things
% simpler here.)
%\begin{biography}[{\includegraphics[width=1in,height=1.25in,clip,keepaspectratio]{mshell}}]{Michael Shell}
% or if you just want to reserve a space for a photo:

%\begin{IEEEbiography}{Michael Shell}
%Biography text here.
%\end{IEEEbiography}

% if you will not have a photo at all:
%\begin{IEEEbiographynophoto}{John Doe}
%Biography text here.
%\end{IEEEbiographynophoto}

% insert where needed to balance the two columns on the last page with
% biographies
%\newpage

%\begin{IEEEbiographynophoto}{Jane Doe}
%Biography text here.
%\end{IEEEbiographynophoto}

% You can push biographies down or up by placing
% a \vfill before or after them. The appropriate
% use of \vfill depends on what kind of text is
% on the last page and whether or not the columns
% are being equalized.

%\vfill

% Can be used to pull up biographies so that the bottom of the last one
% is flush with the other column.
%\enlargethispage{-5in}



% that's all folks
\end{document}


