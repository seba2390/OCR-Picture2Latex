% !TEX root = DistrSeqCtrlAttacks_TASE19.tex

%


%
\section{Resiliency Analysis and Security Patching}
\label{sec:verification}
%
A security-aware closed-loop system model obtained by composing the developed security-aware TPN models can be used to verify system-level safety and QoC properties in the presence of attacks. TPN analysis tools (e.g.,~\cite{romeo,tina}) allow for verification of formal properties specified as Linear Temporal Logic (LTL), Computational Tree Logic (CTL), or Timed CTL (TCTL) formulas~\cite{modelChecking}. In this work, we employ the tool Romeo~\cite{romeo} that enables verification of TCTL-based formal queries, such as  traditional safety (e.g., 1-boundedness~\cite{David20101}) and liveness properties (e.g., absence of deadlock).
In addition, as plant models are included, we can specify relevant domain-related plant-state-bound properties that are crucial for functional safety and QoC assessment. For our running example, the considered properties include: %as~follows.


\begin{property}\label{prop:p1}
  A workpiece on conveyor~1~never triggers a pick-up from conveyor~2; this can be formally captured as:
 \verb!AG(not(M(Pp&p_P&P2)==1 and! \verb! activeConveyor==1))!.\footnote{Variable \verb!activeConveyor! is set when a workpiece presence is detected (i.e., on transitions \verb!Tcm_Pres1! or \verb!Tcm_Pres2! of the conveyor monitor, shown as CIPN in Fig.~\ref{fig:exampleCIPN}) and reset when conveyor monitor returns to initial state (i.e., over \verb!Tcm_RetInit!)}, where \verb!A! and \verb!G! are quantifiers signifying formula satisfaction along \emph{all} paths and \emph{always} (i.e., along all subsequent paths), respectively.
\end{property}

\begin{property}\label{prop:p2}
  A workpiece detected on any of the conveyors is eventually picked-up: \verb!(M(Pcm_TxCtrl_Pick1)==1 or! \verb!M(Pcm_TxCtrl_Pick2)==1)-->(M(Pp&p_wfRet)==1)!, where \verb!-->! denotes the {"leads to"} property; i.e., \verb!p-->q! means that for all executions, continuous satisfaction of property \verb!p! implies always eventual satisfaction of property \verb!q!, or formally \verb!AG(p => AF(q))!.%\LE{showcase for DoS}
\end{property}

\begin{property}\label{prop:p3}
  The pick~\&~place station does not commence cycle (i.e., it is neither in \verb!Pp&p_P&P1! nor \verb!Pp&p_P&P2!), while the conveyor monitor is waiting for incoming workpieces (i.e., in the place \verb!Pcm_Init!). Formally, \verb!AG(M(Pcm_Init)+M(Pp&p_P&P1)+M(Pp&p_P&P2)<=1)!.%\LE{showcase for injection - MAC solves this}
\end{property}

Using the Romeo tool, we verified that these as well as other QoC- and safety-critical properties are \emph{not} satisfied in the presence of attacks, as the attacker is capable of significantly altering the intended interaction between LCs, at arbitrary moments in time. For example, Property~\ref{prop:p1} is violated under \emph{message modification} attacks, Property~\ref{prop:p2} under possible infinite \emph{DoS}, while Property~\ref{prop:p3} fails under \emph{spoofing}.
The aforementioned properties are violated regardless of the values of timing parameters used in the model. Note that bounds on time-to-transmit and time-to-acknowledge can be obtained from experimental measurements, or directly from network specifications. Also, transceiver-related timings (e.g., back-off time during clear channel assessment) can be obtained from the employed transceivers' specifications. % Notice that aforementioned properties are violated regardless of these timing parameters.}




Regarding verification scalability---in the system model, one nominal pick~\&~place cycle,~with all attacks disabled, contains around $35$ transitions, which is on the order of the number of states in the model. Model complexity increases with the addition of non-deterministic attack choices, besides the time-induced non-determinism (in the plant model).\footnote{Romeo does not output statistics of the state space underlying the model.} Yet, in all cases, the tool takes less than $1~s$ to find an execution path violating the properties, on a workstation with an Intel~i7-8086K CPU ($4~GHz$ clock) and $64~GB$~memory.

\subsection{Addressing the Discovered Vulnerabilities}
As our previous analysis have shown, attack actions may significantly affect performance of distributed IoT-based industrial automation systems; to address them, it is necessary to add certain security mechanisms. In this section, we discuss how such security mechanisms affect  system models and verifiability of the relevant~properties.


\begin{figure}[!t]
	\centering
	%\vspace{-4pt}
	\includegraphics[width=0.48\textwidth]{MACandDOSmodels.pdf}
	\caption{Model adaptation to addition of security services; (a) for DoS detection, and (b) against spoofing. Additional places and transitions are shown in blue color.}
	\label{fig:MACandDOSmodels}
\end{figure}
%
\begin{figure}[!t]
	\centering
	%\vspace{-8pt}
	\includegraphics[width=0.44\textwidth]{histograms}
	\caption{Tx-to-Rx and Rx-to-Ack times measured on our IEEE~802.15.4-enabled LC platform described in Section~\ref{sec:verification}.}
	\label{fig:histograms}
\end{figure}
%


\subsubsection{Detecting Denial-of-Service Attacks}
\label{subsubsec:DoSdetection}
Packet and acknowledgement (ACK) dropouts are common in wireless communication, and hence ACK and retransmission mechanisms are commonly used in such setups. For instance, in our experimental setup described in Sectio~\ref{sec:evaluation}, ACK request can be disabled in transceiver settings, in which case no retransmissions are attempted on the data link layer. For two isolated transceivers, this amounts to the one-way packet success rate of approximately $99~\%$ (see histograms in~Fig.~\ref{fig:histograms} that exclude unsuccessful transmissions). Thus, when ACK requests are enabled, up to three data-link layer retransmissions are performed,\footnote{The XCVR model in Fig.~\ref{fig:CIPNtoTPN}(d) is compatible with these specifications.} and we experimentally observed that no application-level retries are required beyond the three low-level protocol-provided retransmissions, in the case when a single industrial machine operates in isolation.

On the other hand, to increase network utilization, we emulated a number of additional machines communicating over the same wireless channel in physical vicinity (described in more detail in Section~\ref{sec:evaluation}); we experimentally observed the one-way packet success rate of approximately $98~\%$. Thus, two application-level retries were sufficient to enable reliable exchange of events, ensuring correct operation. Intuitively, protocol-provided retries are issued in short bursts while application-level retransmissions incur significant delay; the channel is more likely to be continuously busy for a short period of time (e.g., occupied by other legitimate transmissions). Yet, an adversary may repeatedly deny network access to legitimate controllers preventing the system from progressing. {Consequently, the modeled system does not satisfy Property~\ref{prop:p2}, despite application-level retransmissions, unless DoS attacks can be detected and system halted (or other precautionary actions taken), using e.g., a separate secure channel.}

From the operational perspective, every LC may implement a limited number of successive application-level retransmissions before declaring that it is under attack. For instance, if in our setup from Section~\ref{sec:evaluation}, we limited the number of retransmissions to five, amounting to a theoretical one-way packet success rate of \emph{eight nines} (if application-level retransmissions are assumed to be independent). To address this from the modeling perspective, we add an additional place where the transmitter's model transitions to, when application-level retries are exhausted (see Fig.~\ref{fig:MACandDOSmodels}(a)). Hence, we can verify that if infinite blocking of medium access is allowed, LCs may end up in the place \verb!Pa_DoSdetect!. Conversely, if DoS attacks are limited to four consecutive channel access denials, Property~\ref{prop:p2} is satisfied. Note that immediate emergency halt of the machinery may not be possible if a secure communication channel is not available or the DoS attacks cannot be isolated from the network (e.g., using bus guardians).


\subsubsection{Authenticating network flows}
\label{subsubsec:authentication}
Traditional cryptographic techniques for ensuring integrity of network flows rely on signing packets with Message Authentication Codes (MAC)~\cite{802.15.4auth}, and can be used to defend against spoofing attacks. In this setting, every transmission between LCs is signed by the transmitter using a secret key, and the signature is verified by the receiver; therefore, the attacker cannot tamper with the message payload, or else he/she will be detected.

From the modeling perspective, introducing authentication can be modeled as an additional condition on the receiving transitions (in the controller models) where the received payload is compared to desired values; i.e., the MAC portion of the payload is compared to a secret value that cannot be altered (in the case of modification) or generated (in the case of spoofing attacks) by the attacker. Specifically, the transition \verb!Tb_wfRx! in the $LC_B$ model in Fig.~\ref{fig:CIPNtoTPN}(b) would feature an additional guard function on the \verb!B_RxMAC! variable. Optionally, if the signature verification fails, a transition to a place modeling intrusion detection reaction can be added as shown in Fig.~\ref{fig:MACandDOSmodels}(b); this is left to the application designer as reacting to detected intrusions is highly application-specific.

Using the developed framework, we verified that if non-authenticated transmissions are not allowed (i.e., authentication implemented), Properties~\ref{prop:p1}~and~\ref{prop:p3} can be verified over our running example, under the condition that infinite denial of network access to LCs is not allowed, as previously discussed.


\subsubsection{Acknowledgement Spoofing}
Authenticating transmission does not affect ACKs as the data-link layer is responsible for ACK packets while MACs are added to the packet payload. In addition, non-encrypted sequence numbers, which are part of the packet frame, can be overheard by the attacker. Thus, valid ACKs can be generated on behalf of inactive (failed) LCs. Also, undelivered (i.e., intercepted) transmissions can be falsely acknowledged, even when authentication is used. This is a well-known shortcoming of data link layer ACKs~\cite{ackattack2,802.15.4sec}, and could be alleviated by application-level ACKs. Enforcing consensus over event-propagation in discrete event systems spans beyond the scope of this paper; yet, the presented modeling techniques can be utilized to model additional implemented protocols.

While this section introduced the general security-aware modeling aspects, with occasional focus on specific medium access techniques to avoid overly general discussions, in the following section we demonstrate the use of the presented framework on real-world industrial case studies.
