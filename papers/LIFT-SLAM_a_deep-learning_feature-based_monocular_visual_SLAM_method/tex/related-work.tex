\section{Related Work}

\textbf{Feature-based approaches.} The first monocular feature-based VSLAM system proposed was MonoSLAM \cite{monoslam}. In this method, camera motion and 3D structure of an unknown environment are simultaneously estimated using an extended Kalman filter (EKF). There is no loop closure detection in this method, and it performs map initialization using a known object. This method's main problem is the computational cost, as it increases in proportion to the size of an environment. The Parallel Tracking and Mapping (PTAM) \cite{ptam} algorithm was proposed to solve the problems of MonoSLAM. To reduce computational costs, the authors propose to split tracking and mapping into two separate tasks, processed in parallel threads. That way, the tracking estimates camera motion in real-time, and the mapping estimates accurate 3D positions of feature points with a computational cost \cite{survey-visual-slam}. It is the first real-time method that was able to incorporate bundle adjustment (BA). They have also created an automatic initialization with a 5-point algorithm. The main ideas of PTAM were used in ORB-SLAM. 

Mur-Artal et al. proposed ORB-SLAM \cite{orb-slam}, a feature-based monocular VSLAM system with three threads: Tracking, Local Mapping, and Loop Closing. It relies on Oriented FAST and rotated BRIEF (ORB) features and uses a place recognition system based on Bag-of-Words (BoW). The mapping step adopts graph representations, which allow the system to perform local and global pose-graph optimization. Later, the authors of ORB-SLAM proposed an extension of ORB-SLAM applied to stereo and RGB-D cameras \cite{orb-slam2}. This is currently one of the state-of-the-art feature-based monocular VSLAM algorithms. However, because it is based on traditional features, ORB-SLAM can still fail in some situations, as shown in \cite{gcnv2}, requiring parameter tuning per dataset, as we will demonstrate later.

%The system adopts two graph representations: a covisibility graph and an essential graph. %In the covisibility graph, each node is a keyframe, and an edge between two keyframes exists if they share observations of the same map points weighted by the number of map points in common. The essential graph is a lighter version of the covisibility graph. It contains the same nodes (keyframes) but fewer edges, which allows faster pose graph optimization. 

\textbf{End-to-end deep learning-based approaches.} One of the most notable end-to-end approaches is called DeepVO \cite{deep-vo}. In DeepVO, a Recurrent Neural Network (RNN) estimates the camera pose from features learned by a Convolutional Neural Network (CNN). The CNN architecture proposed is based on an architecture used to compute optical flow from a sequence of images called Flownet \cite{flownet}. Then two stacked Long-Short Term Memory (LSTM) layers are applied to estimate temporal changes from the features predicted by a CNN. Another end-to-end approach, based on unsupervised learning called UnDeepVO, is presented in \cite{undeep-vo}. The network relies on stereo image pairs to recover the scale during training while using consecutive monocular images for testing. Moreover, the loss function defined for training the networks uses spatial and temporal dense information. The system successfully estimates the pose of a monocular camera and the depth of its view. 

In \cite{attention-based}, an end-to-end system that uses a similar architecture to DeepVO is proposed. However, instead of employing LSTMs, they include an attention phase, which is called Neural Graph Optimization. It considers that poses that are temporally adjacent should have similar outputs and should be visually identical. Still, temporally different poses should also have related outputs, enabling a loop closure-like correction of drift. Although these approaches presented promising results, they are still not accurate enough to overcome the results of traditional methods.

\textbf{Hybrid approaches.} Hybrid approaches replace some modules of the traditional VSLAM pipeline. In \cite{pose-graph-optimization}, Li et. al. proposed a monocular system called Neural Bundler. It is an unsupervised DNN that estimates motion. Then, it constructs a conventional pose graph, enabling an efficient loop closing procedure based on the pose graph's optimization. A recent hybrid approach called SuperGlue \cite{superglue} proposed a graph neural network with an attention mechanism to perform the matching between two sets of local features. They use the DNN between feature extraction and pose estimation, which they call a learnable "middle-end," as it lies between the front-end and back-end of a traditional VSLAM system. 

Recently, some papers proposed using locally learned features to replace the traditional local features such as ORB and Scale-invariant feature transform (SIFT) of VSLAM systems. In DF-SLAM \cite{df-slam}, the TFeat network \cite{tfeat} is used to create descriptors for features extracted from stereo images with the FAST corner detector. The feature descriptors are then used in a traditional VSLAM pipeline, based on ORB-SLAM2 \cite{orb-slam2}. A self-supervised approach called SuperPointVO is proposed in \cite{self-improving-vo}, where they combine a DNN based in SuperPoint \cite{superpoint} feature extractor as a VSLAM front-end with a traditional back-end, using the stability of keypoints in the images to aid in learning. These approaches can leverage deep learning-based methods robustly and still be as accurate as a traditional feature-based approach. However, none of these papers evaluate their methods' robustness in challenging scenarios or in multiple datasets with different characteristics.