Bayesian optimization is a sample efficient method for the optimization of black box functions. The optimization occurs by constructing a Gaussian Process (GP) model of the function. GP is used as a prior over the space of smooth functions and can be fully described by a mean and covariance function \cite{Rasmussen:2005:GPM:1162254}. Assuming a mean of zero, the GP is then defined by its covariance function $f \sim GP(0,k(\textbf{x},\textbf{x'}))$. Given some observations $D = \{\textbf{x}_i,y_i\}_{i=1}^{t}$, the posterior is also a GP for which the predictive mean and covariance can be computed via,

\begin{equation} \label{eq:mu}
\mu_{t}(\textbf{x}) = \textbf{k}_{t}(\textbf{x})^T(K_{t} + \sigma_{n}^2\mathbf{I})^{-1}\textbf{y}_{t}
\end{equation}
\begin{equation} \label{eq:sigma}
k_{t}(\textbf{x},\textbf{x}') = k(\textbf{x},\textbf{x}') - \textbf{k}_t(\textbf{x})^T(K_t + \sigma_{n}^2\mathbf{I})^{-1}\textbf{k}_t(\textbf{x}')
\end{equation}
with variance $\sigma_{t}^2(\textbf{x}) = k_{t}(\textbf{x},\textbf{x})$, $\textbf{k}_t(\textbf{x}) = [k(\textbf{x}_i,\textbf{x})]_{i=1}^t$ and, $K_t = [k(\textbf{x}_t,\textbf{x}_{t^{'}})]_{t,t^{'}}$ is the kernel Gram matrix.

The squared exponential kernel of the form shown in (\ref{eq:sekernel}) is a popular choice of kernel and will be used in this paper.

\begin{equation} \label{eq:sekernel}
k(\textbf{x}_i,\textbf{x}_j) = \sigma_{f}^2 \text{exp}(-\frac{1}{2} \sum_{d=1}^{D} \frac{(x_{d,i} - x_{d,j})^2}{l_{d}^2})
\end{equation}
where $\sigma_{f}^2$ is the signal variance, and $l_{d}$ is the length scale for the $d$-th dimension.

The mean and variance from the GP is used to construct an acquisition function $a(\boldsymbol{x})$ which is then optimized to select the next best sample point. 

\begin{equation}\label{eq:BO_acquisition}
\textbf{x}_{t} = \operatorname*{argmax}_{x\in \chi}  a_{t}(\textbf{x})
\end{equation}

There are multiple acquisition functions used in Bayesian optimization problems including probability of improvement, GP-UCB, Expected Improvement (EI) and Thompson Sampling \cite{DBLP:journals/corr/abs-1012-2599,Basu2017}.