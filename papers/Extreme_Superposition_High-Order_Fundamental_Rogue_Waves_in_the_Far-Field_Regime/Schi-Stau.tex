\label{sec:Schi-Stau}
In this section, we prove Theorem~\ref{thm:exterior}.  Since that result is specialized to the case of fundamental rogue waves of order $k\in\mathbb{Z}_{>0}$ for which $\mathbf{G}=\mathbf{Q}^{-s}$ with $s=(-1)^k$ and $M=\tfrac{1}{2}k+\tfrac{1}{4}$ (assumptions that are essential to the proof), in this section we will write $\mathbf{S}^{(k)}(\lambda;\chi,\tau)=\mathbf{S}(\lambda;\chi,\tau,\mathbf{Q}^{-s},M)$.
\subsection{Deformation to a dumbbell-shaped contour}
\label{sec:dumbbell}
When $(\chi,\tau)\in \exterior$, we will find it useful to begin by replacing the Jordan jump contour $\Sigma_\circ$ for $\mathbf{S}^{(k)}(\lambda;\chi,\tau)$ with a dumbbell-shaped contour consisting of a closed loop $\Gamma^+$ in the upper half-plane surrounding the point $\lambda=\ii$ in the clockwise sense, its reflection $\Gamma^-$ in the real axis (also oriented in the clockwise sense), and a ``neck'' $N$ consisting of an upward-oriented arc against the left side of the branch cut $\Sigma_\mathrm{c}$ for $\vartheta(\lambda;\chi,\tau)$ and a downward-oriented arc against the right side of the same cut.  Combining these two jump conditions with the jump discontinuity of the function $\vartheta(\lambda;\chi,\tau)$ across the central arc $\Sigma_\mathrm{c}$ of the neck, we can write a single jump condition for $\mathbf{S}^{(k)}(\lambda;\chi,\tau)$ across $N$, which we take to be oriented in the upward direction.  For this calculation, we assume that initially the Jordan curve $\Sigma_\circ$ contains $\Gamma^+\cup N\cup \Gamma^-$ in its interior and we introduce a substitution by setting
%\begin{equation}
%\tilde{\mathbf{S}}^{(k)}(\lambda;\chi,\tau)\defeq
%\mathbf{S}^{(k)}(\lambda;\chi,\tau)\ee^{-\ii n\vartheta(\lambda;\chi,\tau)\sigma_3}\omega(\lambda)^{s\sigma_3}\mathbf{Q}^{-s}\omega(\lambda)^{-s\sigma_3}\ee^{\ii n\vartheta(\lambda;\chi,\tau)\sigma_3},
%\label{eq:S-Stilde}
%\end{equation}
%\textcolor{red}{The alternate version of this reads:
\begin{equation}
\tilde{\mathbf{S}}^{(k)}(\lambda;\chi,\tau)\defeq
\mathbf{S}^{(k)}(\lambda;\chi,\tau)\ee^{-\ii M\vartheta(\lambda;\chi,\tau)\sigma_3}\mathbf{Q}^{-s}\ee^{\ii M\vartheta(\lambda;\chi,\tau)\sigma_3},
\label{eq:S-Stilde-ALT}
\end{equation}
%}
for $\lambda$ between $\Sigma_\circ$ and $\Gamma^+\cup N\cup \Gamma^-$, and we set $\tilde{\mathbf{S}}^{(k)}(\lambda;\chi,\tau)\defeq\mathbf{S}^{(k)}(\lambda;\chi,\tau)$ elsewhere, i.e., in the exterior of $\Sigma_\circ$ and in the interior of $\Gamma^+$ and of $\Gamma^-$.  
Dropping the tilde, the jump contour for $\mathbf{S}^{(k)}(\lambda;\chi,\tau)$ becomes $\Gamma^+\cup N\cup \Gamma^-$.  The jump condition for $\mathbf{S}^{(k)}(\lambda;\chi,\tau)$ across $\Gamma^+$ and $\Gamma^-$ reads exactly the same as the original jump condition \eqref{eq:S-jump} across $\Sigma_\circ$.  
%To compute the jump condition across $N$, we recall that $\ee^{\ii n\vartheta(\lambda;\chi,\tau)}$ is single-valued and analytic for $\lambda\in N$, and then obtain from \eqref{eq:S-Stilde} by distinguishing the boundary values of $\omega(\lambda)$ and using the fact that the original version of $\mathbf{S}^{(k)}(\lambda;\chi,\tau)$ is analytic on $N$ that
%\begin{multline}
%\mathbf{S}_+^{(k)}(\lambda;\chi,\tau)=\\
%\mathbf{S}_-^{(k)}(\lambda;\chi,\tau)\ee^{-\ii n\vartheta(\lambda;\chi,\tau)\sigma_3}
%\omega_-(\lambda)^{s\sigma_3}\mathbf{Q}^s\omega_-(\lambda)^{-s\sigma_3}\omega_+(\lambda)^{s\sigma_3}\mathbf{Q}^{-s}
%\omega_+(\lambda)^{-s\sigma_3}\ee^{\ii n\vartheta(\lambda;\chi,\tau)\sigma_3},\\
%\lambda\in N.
%\end{multline}
%We will now simplify this jump across $N$, showing that the jump matrix is off-diagonal.  Indeed, we first observe that regardless of the value of the parity index $s=\pm 1$, 
%\begin{equation}
%\mathbf{Q}^s\omega_-(\lambda)^{-s\sigma_3}\omega_+(\lambda)^{s\sigma_3}\mathbf{Q}^{-s}=\frac{1}{2}
%\begin{bmatrix}\displaystyle \frac{\omega_+(\lambda)}{\omega_-(\lambda)}+\frac{\omega_-(\lambda)}{\omega_+(\lambda)} & 
%\displaystyle \frac{\omega_+(\lambda)}{\omega_-(\lambda)}-\frac{\omega_-(\lambda)}{\omega_+(\lambda)} \\\\
%\displaystyle \frac{\omega_+(\lambda)}{\omega_-(\lambda)}-\frac{\omega_-(\lambda)}{\omega_+(\lambda)} &
%\displaystyle \frac{\omega_+(\lambda)}{\omega_-(\lambda)}+\frac{\omega_-(\lambda)}{\omega_+(\lambda)}
%\end{bmatrix}.
%\end{equation}
%Then, using \eqref{eq:omega-jump} we find that
%%
%%Now, by the definition \eqref{eq:omega-def} of $\omega(\lambda)$, 
%%\begin{equation}
%%\begin{split}
%%\frac{\omega_+(\lambda)}{\omega_-(\lambda)}\pm\frac{\omega_-(\lambda)}{\omega_+(\lambda)}&=\frac{\omega_+(\lambda)^2\pm\omega_-(\lambda)^2}{\omega_+(\lambda)\omega_-(\lambda)}\\
%%&=\frac{f_+(\lambda)^2(1+\ii(\lambda-\rho_+(\lambda)))^2\pm f_-(\lambda)^2(1+\ii(\lambda-\rho_-(\lambda)))^2}{f_+(\lambda)f_-(\lambda)(1+\ii(\lambda-\rho_+(\lambda)))(1+\ii(\lambda-\rho_-(\lambda)))}.
%%\end{split}
%%\end{equation}
%%Then, using \eqref{eq:f-squared} we have
%%\begin{equation}
%%\begin{split}
%%[f_+(\lambda)f_-(\lambda)]^2&=\frac{(\lambda+\rho_+(\lambda))(\lambda+\rho_-(\lambda))}{4\rho_+(\lambda)\rho_-(\lambda)}\\
%%&=\frac{(\lambda+\rho_+(\lambda))(\lambda-\rho_+(\lambda))}{-4\rho_+(\lambda)^2}\\
%%&=\frac{\rho_+(\lambda)^2-\lambda^2}{4\rho_+(\lambda)^2}\\
%%&=\frac{1}{4(\lambda^2+1)}.
%%\end{split}
%%\end{equation}
%%Using the fact that $f(\lambda)\to 1$ as $\lambda\to\infty$ and that $f(\lambda)$ is analytic for $\lambda\in\mathbb{C}\setminus\Sigma_\mathrm{c}$, we take the appropriate square root and obtain
%%\begin{equation}
%%f_+(\lambda)f_-(\lambda)=\frac{1}{2\rho_-(\lambda)}=-\frac{1}{2\rho_+(\lambda)}.
%%\end{equation}
%%Using this along with $\rho_-(\lambda)=-\rho_+(\lambda)$ and $\rho_\pm(\lambda)^2=\lambda^2+1$ then shows that 
%%\begin{equation}
%%\frac{\omega_+(\lambda)}{\omega_-(\lambda)}\pm\frac{\omega_-(\lambda)}{\omega_+(\lambda)}=\ii(1\mp 1).
%%\end{equation}
%%It follows that regardless of the parity index $s=\pm 1$, 
%\begin{equation}
%\mathbf{Q}^s\omega_-(\lambda)^{-s\sigma_3}\omega_+(\lambda)^{s\sigma_3}\mathbf{Q}^{-s}=\ii\sigma_1,\quad\lambda\in N.
%\end{equation}
%Moreover, it then follows easily from \eqref{eq:omega-jump} again that
%\begin{equation}
%\begin{split}
%\mathbf{S}^{(k)}_+(\lambda;\chi,\tau)&=\mathbf{S}^{(k)}_-(\lambda;\chi,\tau)
%\begin{bmatrix}
%0 & s\omega_+(\lambda)^{2s}\ee^{-2\ii n\vartheta(\lambda;\chi,\tau)}\\
%-s\omega_+(\lambda)^{-2s}\ee^{2\ii n\vartheta(\lambda;\chi,\tau)} & 0\end{bmatrix}\\
%&=
%\mathbf{S}^{(k)}_-(\lambda;\chi,\tau)
%\begin{bmatrix}
%0 & -s\omega_-(\lambda)^{2s}\ee^{-2\ii n\vartheta(\lambda;\chi,\tau)}\\
%s\omega_-(\lambda)^{-2s}\ee^{2\ii n\vartheta(\lambda;\chi,\tau)} & 0\end{bmatrix},\quad\lambda\in N.
%\end{split}
%\label{eq:S-N-jump}
%\end{equation}
%\textcolor{red}{In the alternate approach, the way to compute the jump across $N$ is easier.  
To compute the jump of the redefined $\mathbf{S}^{(k)}(\lambda;\chi,\tau)$ across $N$,
we start from its definition and using the fact that $\vartheta(\lambda;\chi,\tau)$ takes distinct boundary values on $N$ from either side we get
\begin{multline}
\mathbf{S}^{(k)}_+(\lambda;\chi,\tau)=\mathbf{S}_-^{(k)}(\lambda;\chi,\tau)\ee^{-\ii M\vartheta_-(\lambda;\chi,\tau)\sigma_3}\mathbf{Q}^s\ee^{\ii M\vartheta_-(\lambda;\chi,\tau)\sigma_3}
\ee^{-\ii M\vartheta_+(\lambda;\chi,\tau)\sigma_3}\mathbf{Q}^{-s}\ee^{\ii M\vartheta_+(\lambda;\chi,\tau)\sigma_3}\\
{}=\mathbf{S}_-^{(k)}(\lambda;\chi,\tau)\frac{1}{2}
\begin{bmatrix} 1+\ee^{2\ii M(\vartheta_+(\lambda;\chi,\tau)-\vartheta_-(\lambda;\chi,\tau))} & 
s\left(\ee^{-2\ii M\vartheta_+(\lambda;\chi,\tau)}-\ee^{-2\ii M\vartheta_-(\lambda;\chi,\tau)}\right)\\
-s\left(\ee^{2\ii M\vartheta_+(\lambda;\chi,\tau)}-\ee^{2\ii M\vartheta_-(\lambda;\chi,\tau)}\right) & 
1+\ee^{-2\ii M(\vartheta_+(\lambda;\chi,\tau)-\vartheta_-(\lambda;\chi,\tau))}\end{bmatrix},\\
\quad\quad\quad\quad\quad\quad\lambda\in N.
\end{multline}
But by \eqref{eq:vartheta} we have $\vartheta_+(\lambda;\chi,\tau)-\vartheta_-(\lambda;\chi,\tau)=-2\pi$.  Since $M=\tfrac{1}{2}k+\tfrac{1}{4}$ for $k\in\mathbb{Z}_{>0}$, this easily reduces to
\begin{equation}
\begin{split}
\mathbf{S}^{(k)}_+(\lambda;\chi,\tau)&=\mathbf{S}_-^{(k)}(\lambda;\chi,\tau)\begin{bmatrix}0 & s\ee^{-2\ii M\vartheta_+(\lambda;\chi,\tau)}\\-s\ee^{2\ii M\vartheta_+(\lambda;\chi,\tau)} & 0\end{bmatrix}\\
&=\mathbf{S}_-^{(k)}(\lambda;\chi,\tau)\begin{bmatrix}0 & -s\ee^{-2\ii M\vartheta_-(\lambda;\chi,\tau)}\\
s\ee^{2\ii M\vartheta_-(\lambda;\chi,\tau)} & 0\end{bmatrix},\quad\lambda\in N.
\end{split}
\label{eq:S-N-jump-ALT}
\end{equation}
%}

\begin{remark}
The fact that the jump matrix on $N$ is off-diagonal is a consequence of the quantization of $M>0$ via $M=\tfrac{1}{2}k+\tfrac{1}{4}$, $k\in\mathbb{Z}_{>0}$, and the choice of ``core'' matrix $\mathbf{G}=\mathbf{Q}^{-s}$ for $s=(-1)^k$.  
More generally, if we express $M\ge 0$ in the modular form $M=\tfrac{1}{2}k+r$ with $k\in\mathbb{Z}_{\ge 0}$ and $0\le r<\tfrac{1}{2}$, then for $\mathbf{G}=\mathbf{Q}^{-s}$ with $s=\pm 1$ arbitrary we obtain
\begin{equation}
\mathbf{S}_+(\lambda;\chi,\tau,\mathbf{Q}^{-s},M)=
\mathbf{S}_-(\lambda;\chi,\tau,\mathbf{Q}^{-s},M)
\ee^{-\ii M\vartheta_-(\lambda;\chi,\tau)\sigma_3}\mathbf{Z}
\ee^{\ii M\vartheta_+(\lambda;\chi,\tau)\sigma_3},\quad\lambda\in N
\end{equation}
in place of \eqref{eq:S-N-jump-ALT}, where $\mathbf{Z}$ is the constant matrix
\begin{equation}
\mathbf{Z}\defeq\begin{bmatrix}
(-1)^k\cos(2\pi r) & s (-1)^k\ii\sin(2\pi r)\\ s(-1)^k\ii\sin(2\pi r) & (-1)^k\cos(2\pi r)\end{bmatrix}.
\end{equation}
It is then clear that the \emph{only} values of $M\ge 0$ for which $\mathbf{Z}$ is off-diagonal are those corresponding to rogue waves.  
This is the reason why fundamental rogue waves behave differently for $(\chi,\tau)\in \exterior$ than other solutions obtained from Riemann-Hilbert Problem~\ref{rhp:rogue-wave-reformulation} for different parameters as described in Section~\ref{sec:M-arbitrary}, such as the high-order multiple-pole solitons for which $M\in\tfrac{1}{2}\mathbb{Z}_{\ge 0}$.  The latter solutions are special once again, in that they are precisely the solutions for which $\mathbf{Z}$ is diagonal (in fact $\mathbf{Z}=(-1)^k\mathbb{I}$).
In the general case, all four entries of $\mathbf{Z}$ are nonzero and hence available for use as pivots in matrix factorizations, and this distinguishes the asymptotic behavior on $\exterior$ from both special cases as described in Section~\ref{sec:general}.
\label{rem:M-quantum}
\end{remark}

Next, we explain how the contours $\Gamma^+$ and $N$ should be chosen (recall that $\Gamma^-$ is the Schwarz reflection of $\Gamma^+$ with clockwise orientation).  Recall from Section~\ref{sec:zero-level-curve} that as $(\chi,\tau)$ ranges over $\exterior$, there exists a simple closed curve surrounding the point $\lambda=\ii$ and passing through the point $\lambda=A+\ii B$ such that all roots of $2\tau\lambda^2+u\lambda+v=0$ are in the exterior of this curve, and importantly, such that $h'(\lambda;\chi,\tau)\,\dd\lambda$ is purely real along the curve.  In other words, the circle domain for the rational quadratic differential $h'(\lambda;\chi,\tau)^2\,\dd\lambda^2$ containing the pole $\lambda=\ii$ (reality of the residue due to the condition \eqref{eq:hprime-residues} guarantees that this point is indeed contained in a circle domain) has only the critical point $\lambda=A+\ii B$ on its boundary.  We take the boundary curve, which is a critical trajectory for $h'(\lambda;\chi,\tau)^2\,\dd\lambda^2$, to be the loop $\Gamma^+$.  Then we choose $N$ to be any Schwarz-symmetric arc from $\lambda=A-\ii B$ to $\lambda=A+\ii B$ that lies in the exterior of both loops $\Gamma^+\cup \Gamma^-$.  Later we will fix its direction near the endpoints of $N$.  See the left-hand panels of Figures~\ref{fig:Schi1}--\ref{fig:Stau1}.

\begin{figure}[h]
\begin{center}
\includegraphics{Schi1.pdf}
\end{center}
\caption{Left:  for $(\chi,\tau)=(2.5,0.7)\in \exterior_\chi$, the regions in the $\lambda$-plane where $\mathrm{Re}(\ii h(\lambda;\chi,\tau))<0$ (shaded) and $\mathrm{Re}(\ii h(\lambda;\chi,\tau))>0$ (unshaded), and the modified jump contour $\Gamma^+\cup N\cup\Gamma^-$.  The jump contour $\Sigma_\mathrm{c}$ for $\vartheta(\lambda;\chi,\tau)$ consists of the union of $N$ and the dashed red arcs terminating at $\lambda=\pm\ii$ (red dots).  Critical points of $h(\lambda;\chi,\tau)$ are shown with black dots.  Also shown are the ``lens'' regions $L^\pm$ and $R^\pm$ lying to the left and right respectively of $\Gamma^\pm$.  Right:  the jump contour for $\mathbf{W}^{(k)}(\lambda;\chi,\tau)$.  Note that for $(\chi,\tau)\in \exterior_\chi$ we may choose the branch cut $N=\Sigma_g$ (highlighted in orange) to coincide with a level curve of $\mathrm{Re}(\ii h(\lambda;\chi,\tau))$ and with this choice $\mathrm{Re}(\ii h(\lambda;\chi,\tau))$ is a continuous function with the exception of the points $\lambda=\pm \ii$.}
\label{fig:Schi1}
\end{figure}

\begin{figure}[h]
\begin{center}
\includegraphics{Stau1.pdf}
\end{center}
\caption{As in Figure~\ref{fig:Schi1} but now for $(\chi,\tau)=(2.0,1.2)\in \exterior_\tau$.  In this case a pair of real critical points of $h(\lambda;\chi,\tau)$ on $\exterior_\chi$ have merged and split into a conjugate pair that is necessarily on a nonzero level of $\mathrm{Re}(\ii h(\lambda;\chi,\tau))$.  The consequence is that it is no longer possible on $\exterior_\tau$ to choose the branch cut $N=\Sigma_g$ to be a level curve of $\mathrm{Re}(\ii h(\lambda;\chi,\tau))$ which therefore experiences a jump discontinuity across the cut.  We illustrate this fact in this figure by taking $N$ as a somewhat arbitrary union of straight line segments instead of any natural trajectory of $h'(\lambda;\chi,\tau)^2\,\dd\lambda^2$.  Crucially, this issue plays no role in the subsequent analysis, because it is never necessary to factor the jump matrix carried by $N$.}
\label{fig:Stau1}
\end{figure}

%\textcolor{red}{Prove this claim\dots}
\subsection{Introduction of $g$ and steepest descent deformation of the Riemann-Hilbert problem}
Now with the contours $\Gamma^\pm$ and $N$ set up in this way, we introduce the $g$-function via the transformation \eqref{eq:T-to-S} 
%\textcolor{red}{(or its alternate version)} 
taking $\mathbf{S}^{(k)}(\lambda;\chi,\tau)$ to $\mathbf{T}^{(k)}(\lambda;\chi,\tau)$.  We assume that the Schwarz-symmetric arc $\Sigma_g$ where $g(\lambda;\chi,\tau)$ fails to be analytic coincides with $N$, which in turn is a sub-arc of $\Sigma_\mathrm{c}$.  Therefore, we need the version of the construction of $g(\lambda;\chi,\tau)$ described in Section~\ref{sec:g-function-dumbbell}.  Recalling from \eqref{eq:hpm-gamma} that $h_+(\lambda;\chi,\tau)+h_-(\lambda;\chi,\tau)=2\gamma(\chi,\tau)$ for $\lambda\in\Sigma_g=N$, where 
%$\kappa(\chi,\tau)$ 
$\gamma(\chi,\tau)$
is a real quantity 
given by \eqref{eq:gamma-formula}
%determined mod $2\pi$ by \eqref{eq:kappa-formula} 
%\textcolor{red}{(we need the form that is the average over two paths from $-\ii$ to $\ii$ on either side of $\Sigma_g$, or just the sum of an integral from $-\ii$ to $\lambda_0(\chi,\tau)^*$ and an integral from $\lambda_0(\chi,\tau)$ to $\ii$)}, 
we obtain from \eqref{eq:S-N-jump-ALT} the jump condition for $\mathbf{T}^{(k)}(\lambda;\chi,\tau)$ along $N$ in the form
%\begin{equation}
%\begin{split}
%\mathbf{T}^{(k)}_+(\lambda;\chi,\tau)&=\mathbf{T}^{(k)}_-(\lambda;\chi,\tau)\begin{bmatrix}
%0 & s\omega_+(\lambda)^{2s}\ee^{-2\ii n\kappa(\chi,\tau)}\\-s\omega_+(\lambda)^{-2s}\ee^{2\ii n\kappa(\chi,\tau)} & 0\end{bmatrix}\\
%&=\mathbf{T}^{(k)}_-(\lambda;\chi,\tau)\begin{bmatrix}
%0 & -s\omega_-(\lambda)^{2s}\ee^{-2\ii n\kappa(\chi,\tau)}\\s\omega_-(\lambda)^{-2s}\ee^{2\ii n\kappa(\chi,\tau)} & 0\end{bmatrix},\quad \lambda\in N.
%\end{split}
%\label{eq:T-jump-N-Schi-Stau}
%\end{equation}  
%\textcolor{red}{The alternate version of this formula reads
\begin{equation}
\mathbf{T}^{(k)}_+(\lambda;\chi,\tau)=\mathbf{T}^{(k)}_-(\lambda;\chi,\tau)\begin{bmatrix}
0 & \ii\ee^{-2\ii M\gamma(\chi,\tau)}\\\ii\ee^{2\ii M\gamma(\chi,\tau)} & 0\end{bmatrix},\quad\lambda\in N.
\label{eq:T-jump-N-Schi-Stau-ALT}
\end{equation}
%}
We next take advantage of the fact that $\mathrm{Re}(\ii h(\lambda;\chi,\tau))=0$ on $\Gamma^\pm$ to transform $\mathbf{T}^{(k)}(\lambda;\chi,\tau)$ explicitly into $\mathbf{W}^{(k)}(\lambda;\chi,\tau)$ by a substitution based on the same elementary factorization \eqref{eq:Q-factorizations} used in Section~\ref{sec:channels}.  Let $\Omega^\pm$ denote small disks centered at $\lambda=\pm\ii$ and enclosed by $\Gamma^\pm$ respectively, let $R^\pm$ denote the interior of $\Gamma^\pm$ with the closure of $\Omega^\pm$ excluded, and let $L^\pm$ denote lens-shaped regions on the exterior of $\Gamma^\pm$ as shown in the left-hand panels of Figures~\ref{fig:Schi1}--\ref{fig:Stau1}.  Then, we make the following definition (compare with \eqref{eq:T-S-L-plus-ALT}--\eqref{eq:T-S-L-minus-ALT}):
%\begin{equation}
%\mathbf{W}^{(k)}(\lambda;\chi,\tau)\defeq\mathbf{T}^{(k)}(\lambda;\chi,\tau)\begin{bmatrix}
%1 & 0\\ s\omega(\lambda)^{-2s}\ee^{2nh(\lambda;\chi,\tau)} & 1\end{bmatrix},\quad\lambda\in L^+,
%\end{equation}
%\begin{equation}
%\mathbf{W}^{(k)}(\lambda;\chi,\tau)\defeq\mathbf{T}^{(k)}(\lambda;\chi,\tau)2^{\frac{1}{2}\sigma_3}\begin{bmatrix}1 & \tfrac{1}{2}s\omega(\lambda)^{2s}\ee^{-2 n h(\lambda;\chi,\tau)}\\ 0 & 1\end{bmatrix},\quad\lambda\in R^+,
%\label{eq:W-def-Schi-Stau-Rplus}
%\end{equation}
%\textcolor{red}{with alternate versions:
\begin{equation}
\mathbf{W}^{(k)}(\lambda;\chi,\tau)\defeq\mathbf{T}^{(k)}(\lambda;\chi,\tau)\begin{bmatrix}
1 & 0\\ s\ee^{2\ii Mh(\lambda;\chi,\tau)} & 1\end{bmatrix},\quad\lambda\in L^+,
\label{eq:W-def-Schi-Stau-Lplus-ALT}
\end{equation}
\begin{equation}
\mathbf{W}^{(k)}(\lambda;\chi,\tau)\defeq\mathbf{T}^{(k)}(\lambda;\chi,\tau)2^{\frac{1}{2}\sigma_3}\begin{bmatrix}1 & \tfrac{1}{2}s\ee^{-2\ii M h(\lambda;\chi,\tau)}\\ 0 & 1\end{bmatrix},\quad\lambda\in R^+,
\label{eq:W-def-Schi-Stau-Rplus-ALT}
\end{equation}
%}
\begin{equation}
\mathbf{W}^{(k)}(\lambda;\chi,\tau)\defeq\mathbf{T}^{(k)}(\lambda;\chi,\tau)2^{\frac{1}{2}\sigma_3},\quad
\lambda\in\Omega^+,
\end{equation}
\begin{equation}
\mathbf{W}^{(k)}(\lambda;\chi,\tau)\defeq\mathbf{T}^{(k)}(\lambda;\chi,\tau)2^{-\frac{1}{2}\sigma_3},\quad
\lambda\in\Omega^-,
\end{equation}
%\begin{equation}
%\mathbf{W}^{(k)}(\lambda;\chi,\tau)\defeq\mathbf{T}^{(k)}(\lambda;\chi,\tau)2^{-\frac{1}{2}\sigma_3}\begin{bmatrix} 1 & 0\\-\tfrac{1}{2}s\omega(\lambda)^{-2s}\ee^{2 nh(\lambda;\chi,\tau)} & 1\end{bmatrix},\quad\lambda\in R^-,
%\label{eq:W-def-Schi-Stau-Rminus}
%\end{equation}
%\begin{equation}
%\mathbf{W}^{(k)}(\lambda;\chi,\tau)\defeq\mathbf{T}^{(k)}(\lambda;\chi,\tau)\begin{bmatrix}
%1 & -s\omega(\lambda)^{2s}\ee^{-2 nh(\lambda;\chi,\tau)}\\ 0 & 1\end{bmatrix},\quad
%\lambda\in L^-,
%\end{equation}
%\textcolor{red}{with alternate versions:
\begin{equation}
\mathbf{W}^{(k)}(\lambda;\chi,\tau)\defeq\mathbf{T}^{(k)}(\lambda;\chi,\tau)2^{-\frac{1}{2}\sigma_3}\begin{bmatrix} 1 & 0\\-\tfrac{1}{2}s\ee^{2\ii Mh(\lambda;\chi,\tau)} & 1\end{bmatrix},\quad\lambda\in R^-,\quad\text{and}
\label{eq:W-def-Schi-Stau-Rminus-ALT}
\end{equation}
\begin{equation}
\mathbf{W}^{(k)}(\lambda;\chi,\tau)\defeq\mathbf{T}^{(k)}(\lambda;\chi,\tau)\begin{bmatrix}
1 & -s\ee^{-2\ii Mh(\lambda;\chi,\tau)}\\ 0 & 1\end{bmatrix},\quad
\lambda\in L^-,
 \label{eq:W-def-Schi-Stau-Lminus-ALT}
\end{equation}
%}
and elsewhere that $\mathbf{T}^{(k)}(\lambda;\chi,\tau)$ is defined we set $\mathbf{W}^{(k)}(\lambda;\chi,\tau)\defeq\mathbf{T}^{(k)}(\lambda;\chi,\tau)$.  One can check that $\mathbf{W}^{(k)}(\lambda;\chi,\tau)$ can be defined on $\Gamma^\pm$ to be analytic there.  Taking into account that 
%$\omega(\lambda)$ \textcolor{red}{(or 
$\ee^{\pm 2\ii Mh(\lambda;\chi,\tau)}$
%)} 
has jump discontinuities across arcs $N^\pm$ within the annular domains $R^\pm$, 
%(although $\ee^{\pm 2nh(\lambda;\chi,\tau)}$ are both single-valued \textcolor{red}{[omit parenthetical remark for alternate approach]}), 
the jump contour for $\mathbf{W}^{(k)}(\lambda;\chi,\tau)$ is as shown in the right-hand panels of Figures~\ref{fig:Schi1}--\ref{fig:Stau1}.
The jump conditions satisfied by $\mathbf{W}^{(k)}(\lambda;\chi,\tau)$ are then the following.  Firstly, since $\mathbf{W}^{(k)}(\lambda;\chi,\tau)=\mathbf{T}^{(k)}(\lambda;\chi,\tau)$ holds for both boundary values taken along $N$, the same jump condition \eqref{eq:T-jump-N-Schi-Stau-ALT} holds for $\mathbf{W}^{(k)}(\lambda;\chi,\tau)$ also.  Next, comparing with \eqref{eq:Tjump-channels-CLplus-ALT}--\eqref{eq:Tjump-channels-CRplus-ALT} and \eqref{eq:Tjump-channels-CRminus-ALT}--\eqref{eq:Tjump-channels-CLminus-ALT}, we have
%\begin{equation}
%\mathbf{W}^{(k)}_+(\lambda;\chi,\tau)=\mathbf{W}^{(k)}_-(\lambda;\chi,\tau)\begin{bmatrix}1 & 0\\
%-s\omega(\lambda)^{-2s}\ee^{2 nh(\lambda;\chi,\tau)} & 1\end{bmatrix},\quad\lambda\in C_L^+,
%\end{equation}
%\begin{equation}
%\mathbf{W}^{(k)}_+(\lambda;\chi,\tau)=\mathbf{W}^{(k)}_-(\lambda;\chi,\tau)\begin{bmatrix}1 & \tfrac{1}{2}s\omega(\lambda)^{2s}\ee^{-2 nh(\lambda;\chi,\tau)} \\ 0 & 1\end{bmatrix},\quad\lambda\in C_R^+,
%\end{equation}
%\begin{equation}
%\mathbf{W}^{(k)}_+(\lambda;\chi,\tau)=\mathbf{W}^{(k)}_-(\lambda;\chi,\tau)\begin{bmatrix}1 & 0\\
%-\tfrac{1}{2}s\omega(\lambda)^{-2s}\ee^{2 nh(\lambda;\chi,\tau)} & 1\end{bmatrix},\quad
%\lambda\in C_R^-,\quad\text{and}
%\end{equation}
%\begin{equation}
%\mathbf{W}^{(k)}_+(\lambda;\chi,\tau)=\mathbf{W}^{(k)}_-(\lambda;\chi,\tau)\begin{bmatrix} 1 & s\omega(\lambda)^{2s}\ee^{-2 nh(\lambda;\chi,\tau)}\\ 0 & 1\end{bmatrix},\quad\lambda\in C_L^-.
%\end{equation}
%\textcolor{red}{The alternate versions of these read:
\begin{equation}
\mathbf{W}^{(k)}_+(\lambda;\chi,\tau)=\mathbf{W}^{(k)}_-(\lambda;\chi,\tau)\begin{bmatrix}1 & 0\\
-s\ee^{2\ii Mh(\lambda;\chi,\tau)} & 1\end{bmatrix},\quad\lambda\in C_L^+,
\label{eq:Wjump-exterior-CLplus}
\end{equation}
\begin{equation}
\mathbf{W}^{(k)}_+(\lambda;\chi,\tau)=\mathbf{W}^{(k)}_-(\lambda;\chi,\tau)\begin{bmatrix}1 & \tfrac{1}{2}s\ee^{-2\ii Mh(\lambda;\chi,\tau)} \\ 0 & 1\end{bmatrix},\quad\lambda\in C_R^+,
\end{equation}
\begin{equation}
\mathbf{W}^{(k)}_+(\lambda;\chi,\tau)=\mathbf{W}^{(k)}_-(\lambda;\chi,\tau)\begin{bmatrix}1 & 0\\
-\tfrac{1}{2}s\ee^{2\ii Mh(\lambda;\chi,\tau)} & 1\end{bmatrix},\quad
\lambda\in C_R^-,\quad\text{and}
\end{equation}
\begin{equation}
\mathbf{W}^{(k)}_+(\lambda;\chi,\tau)=\mathbf{W}^{(k)}_-(\lambda;\chi,\tau)\begin{bmatrix} 1 & s\ee^{-2\ii Mh(\lambda;\chi,\tau)}\\ 0 & 1\end{bmatrix},\quad\lambda\in C_L^-.
\label{eq:Wjump-exterior-CLminus}
\end{equation}
%}
Finally, for $\lambda\in N^\pm$ we compute 
%the jump condition from \eqref{eq:W-def-Schi-Stau-Rplus} and 
%\eqref{eq:W-def-Schi-Stau-Rminus}:
%\begin{equation}
%\mathbf{W}_+^{(k)}(\lambda;\chi,\tau)=\mathbf{W}_-^{(k)}(\lambda;\chi,\tau)\begin{bmatrix}1 & 
%\tfrac{1}{2}s(\omega_+(\lambda)^{2s}-\omega_-(\lambda)^{2s})\ee^{-2nh(\lambda;\chi,\tau)}\\0 & 1\end{bmatrix},\quad\lambda\in N^+,\quad\text{and}
%\end{equation}
%\begin{equation}
%\mathbf{W}^{(k)}_+(\lambda;\chi,\tau)=\mathbf{W}^{(k)}_-(\lambda;\chi,\tau)\begin{bmatrix}
%1 & 0\\-\tfrac{1}{2}s(\omega_+(\lambda)^{-2s}-\omega_-(\lambda)^{-2s})\ee^{2nh(\lambda;\chi,\tau)} & 1
%\end{bmatrix},\quad\lambda\in N^-.
%\end{equation}
%Note that according to \eqref{eq:omega-jump}, $\omega_+(\lambda)^{\pm 2s}-\omega_-(\lambda)^{\pm 2s}=2\omega_+(\lambda)^{\pm 2s}=-2\omega_-(\lambda)^{\pm 2s}$.  
%\textcolor{red}{The alternate version of this calculation reads:
\begin{equation}
\mathbf{W}^{(k)}_+(\lambda;\chi,\tau)=\mathbf{W}_-^{(k)}(\lambda;\chi,\tau)\begin{bmatrix}1 & \tfrac{1}{2}s\left(\ee^{-2\ii Mh_+(\lambda;\chi,\tau)}-\ee^{-2\ii Mh_-(\lambda;\chi,\tau)}\right)\\0 & 1\end{bmatrix},\quad
\lambda\in N^+,\quad\text{and}
\end{equation}
\begin{equation}
\mathbf{W}^{(k)}_+(\lambda;\chi,\tau)=\mathbf{W}^{(k)}_-(\lambda;\chi,\tau)\begin{bmatrix}1 & 0\\
-\tfrac{1}{2}s\left(\ee^{2\ii Mh_+(\lambda;\chi,\tau)}-\ee^{2\ii Mh_-(\lambda;\chi,\tau)}\right) & 1\end{bmatrix},\quad\lambda\in N^-.
\end{equation}
Then, since for $\lambda\in N^\pm$, $g(\lambda;\chi,\tau)$ has no jump discontinuity and $\vartheta_+(\lambda;\chi,\tau)-\vartheta_-(\lambda;\chi,\tau)=-2\pi$, and since $M=\tfrac{1}{2}k+\tfrac{1}{4}$ for $k\in\mathbb{Z}_{>0}$, these simplify to
\begin{equation}
\mathbf{W}^{(k)}_+(\lambda;\chi,\tau)=\mathbf{W}^{(k)}_-(\lambda;\chi,\tau)\begin{bmatrix}1 & \ii\ee^{-\ii M(h_+(\lambda;\chi,\tau)+h_-(\lambda;\chi,\tau))}\\0 & 1\end{bmatrix},\quad\lambda\in N^+,\quad\text{and}
\end{equation}
\begin{equation}
\mathbf{W}^{(k)}_+(\lambda;\chi,\tau)=\mathbf{W}^{(k)}_-(\lambda;\chi,\tau)\begin{bmatrix}1 & 0\\
\ii\ee^{\ii M(h_+(\lambda;\chi,\tau)+h_-(\lambda;\chi,\tau))} & 1\end{bmatrix},\quad\lambda\in N^-.
\end{equation}
%}

%As a final step to prepare for the construction of parametrices, we introduce a \emph{Szeg\H{o} function} $S(\lambda;\chi,\tau)$ whose purpose is to remove the non-constant factors involving boundary values of $\omega(\lambda)^{\pm 2}$ from the jump condition along $N$ (see \eqref{eq:T-jump-N-Schi-Stau}).  
%Recalling that $N$ is chosen to coincide with $\Sigma_g$, $S:\mathbb{C}\setminus\Sigma_g\to\mathbb{C}$ is required to be analytic and bounded at the endpoints $\lambda=A(\chi,\tau)\pm \ii B(\chi,\tau)$ of $\Sigma_g$, to satisfy the normalization condition $S(\lambda;\chi,\tau)\to 0$ as $\lambda\to\infty$, and to admit continuous boundary values $S_\pm(\lambda;\chi,\tau)$ on $\Sigma_g$ related by 
%\begin{equation}
%S_+(\lambda;\chi,\tau) + S_-(\lambda;\chi,\tau) - 2s \log(\omega_+(\lambda)) = 2 \ii {s}\gamma,\quad \lambda\in \Sigma_g, 
%\label{eq:Szego-jump-Schi-Stau}
%\end{equation}
%for some constant $\gamma=\gamma(\chi,\tau)$ which may depend on the parameters $\chi$ and $\tau$. 
%Unlike in the construction of $g(\lambda;\chi,\tau)$ described in Section~\ref{sec:g-function} where the endpoints $A(\chi,\tau)\pm\ii B(\chi,\tau)$ of $\Sigma_g$ were used as free parameters chosen to guarantee existence, these have now been determined and are no longer available; here the constant $\gamma$ will play a similar role of a parameter to be chosen so that $S$ exists.  In the jump condition \eqref{eq:Szego-jump-Schi-Stau}, $\log(\cdot)$ is taken to be the principal branch, so that $\log(\omega(\lambda))$ is an analytic function on the domain $\mathbb{C}\setminus\overline{N\cup N^+\cup N^-}$ that vanishes as $\lambda\to\infty$.  The boundary values $\log(\omega_\pm(\lambda))$ are analytic for $\lambda\in\Sigma_g$ (in fact, on a neighborhood of this arc) but unequal, and we have $\omega_+(\lambda)^{2s} = \ee^{2s \log(\omega_+(\lambda))}$.
%% with the right-hand side being well-defined for all $\lambda\in\Sigma_g$ including the end points since $\omega(\lambda)$ is nonzero, continuous and bounded (in fact, analytic) in a suitably small\footnote{Suitably small in the sense that the neighborhood leaves $\Sigma_\omega$ in its exterior.} neighborhood containing $\Sigma_g$. 
%It is easy to verify that
%\begin{equation}
%S(\lambda;\chi,\tau)\defeq \frac{R(\lambda;\chi,\tau)}{2\pi \ii} \int_{\Sigma_g} \frac{2s \log(\omega_+(\eta))+  2\ii{s} \gamma}{R_+(\eta; \chi,\tau)(\eta-\lambda)}\,\dd \eta
%\label{eq:Szego-def-Schi-Stau}
%\end{equation}
%is analytic in $\mathbb{C}\setminus\Sigma_g$, satisfies the jump condition given in \eqref{eq:Szego-jump-Schi-Stau}, and it is bounded as $\lambda \to A(\chi,\tau) \pm \ii B(\chi,\tau)$. Since $R(\lambda;\chi,\tau)=\lambda+O(1)$ as $\lambda\to\infty$, enforcing the normalization condition $S(\lambda;\chi,\tau)\to 0$ as $\lambda\to\infty$ requires 
%\begin{equation}
%\int_{\Sigma_g} \frac{\log(\omega_+(\eta))+ \ii \gamma(\chi,\tau)}{R_+(\eta;\chi,\tau)}\, \dd \eta = 0.
%\label{eq:gamma-condition-Schi-Stau}
%\end{equation}
%This condition determines the constant $\gamma=\gamma(\chi,\tau)$; indeed,
%recalling \eqref{eq:integral-R-plus} 
%we obtain from \eqref{eq:gamma-condition-Schi-Stau} that
%\begin{equation}
%\gamma(\chi,\tau)  = -\frac{1}{\pi} \displaystyle \int_{\Sigma_g} \frac{ \log(\omega_+(\eta)) }{R_+(\eta;\chi,\tau)}\,\dd \eta.
%\label{eq:gamma-def-Schi-Stau}
%\end{equation}
%We can simplify this formula as follows.  Firstly, let $\Sigma_{\tilde{\omega}}$ be an arc with the same endpoints ($\lambda=\pm\ii$) as but lying to the right of $\overline{N\cup N^+\cup N^-}$, and let  $\log(\tilde{\omega}(\lambda))$ denote the analytic continuation to $\mathbb{C}\setminus\Sigma_{\tilde{\omega}}$ from $\Sigma_g$ of $\log(\omega_+(\lambda))$.  Then, let $C$ be a clockwise-oriented loop surrounding the branch cut $\Sigma_g$ of $R(\lambda;\chi,\tau)$ excluding the branch cut $\Sigma_{\tilde{\omega}}$ of $\tilde{\omega}(\lambda)$. Taking $C'$ to be a counter-clockwise-oriented contour that encircles $\Sigma_{\tilde{\omega}}$ but excludes $\Sigma_g$ and using the fact that the integrand in \eqref{eq:gamma-def-Schi-Stau} is integrable at $\eta=\infty$, we obtain:
%\begin{equation}
%\gamma(\chi,\tau) = -\frac{1}{2\pi} \oint_C \frac{\log(\tilde{\omega}(\eta))}{R(\eta;\chi,\tau)}\,\dd \eta
%= -\frac{1}{2\pi} \oint_{C'} \frac{\log(\tilde{\omega}(\eta))}{R(\eta;\chi,\tau)}\,\dd \eta
%= -\frac{1}{8\pi} \oint_{C'} \log\left(\frac{\eta-\ii}{\eta+\ii}\right) \frac{\dd\eta}{R(\eta;\chi,\tau)},
%\end{equation}
%where we used the identity \eqref{eq:omega-fourth-power} and where the logarithm has $\Sigma_{\tilde{\omega}}$ as its branch cut. We may now collapse $C'$ to both sides of $\Sigma_{\tilde{\omega}}$, where $R(\lambda;\chi,\tau)$ is analytic but the boundary values of the logarithm differ by $2\pi \ii$, and see that
%\begin{equation}
%\gamma(\chi,\tau) = -\frac{1}{4\ii} \int_{\Sigma_{\tilde{\omega}}} \frac{\dd\eta}{R(\eta;\chi,\tau)},
%\label{eq:gamma-explicit-Schi-Stau}
%\end{equation}
%and since $\Sigma_{\tilde{\omega}}$ is a Schwarz-symmetric contour, it follows that $\gamma(\chi,\tau)\in\mathbb{R}$.  This completes the construction of the Szeg\H{o} function $S(\lambda;\chi,\tau)$.
%
%We now make a global substitution by setting
%\begin{equation}
%\mathbf{X}^{(k)}(\lambda;\chi,\tau) \defeq \mathbf{W}^{(k)}(\lambda;\chi,\tau)\ee^{S(\lambda;\chi,\tau)\sigma_3}
%\end{equation}
%in the whole domain of analyticity of $\mathbf{W}^{(k)}(\lambda;\chi,\tau)$.  From the jump condition \eqref{eq:Szego-jump-Schi-Stau}, the jump matrix on $\Sigma_g=N$ for $\mathbf{X}^{(k)}(\lambda;\chi,\tau)$
%becomes constant as desired:
%\begin{equation}
%\mathbf{X}_+^{(k)}(\lambda;\chi,\tau)=\mathbf{X}_-^{(k)}(\lambda;\chi,\tau)\begin{bmatrix}
%0 & s\ee^{-2\ii (n\kappa(\chi,\tau)+s\gamma(\chi,\tau))}\\
%-s\ee^{2\ii (n\kappa(\chi,\tau)+s\gamma(\chi,\tau))} & 0\end{bmatrix},\quad \lambda\in \Sigma_g=N.
%\label{eq:X-jump-N-Schi-Stau}
%\end{equation}
%The remaining jump conditions for $\mathbf{W}^{(k)}(\lambda;\chi,\tau)$ are merely modified by conjugation of the jump matrix, since $S_+(\lambda;\chi,\tau)=S_-(\lambda;\chi,\tau)$ on all other contour arcs:
%\begin{equation}
%\mathbf{X}^{(k)}_+(\lambda;\chi,\tau)=\mathbf{X}^{(k)}_-(\lambda;\chi,\tau)\begin{bmatrix}1 & 0\\
%-s\omega(\lambda)^{-2s}\ee^{2S(\lambda;\chi,\tau)}\ee^{2 nh(\lambda;\chi,\tau)} & 1\end{bmatrix},\quad\lambda\in C_L^+,
%\end{equation}
%\begin{equation}
%\mathbf{X}^{(k)}_+(\lambda;\chi,\tau)=\mathbf{X}^{(k)}_-(\lambda;\chi,\tau)\begin{bmatrix}1 & \tfrac{1}{2}s\omega(\lambda)^{2s}\ee^{-2S(\lambda;\chi,\tau)}\ee^{-2 nh(\lambda;\chi,\tau)} \\ 0 & 1\end{bmatrix},\quad\lambda\in C_R^+,
%\end{equation}
%\begin{equation}
%\mathbf{X}^{(k)}_+(\lambda;\chi,\tau)=\mathbf{X}^{(k)}_-(\lambda;\chi,\tau)\begin{bmatrix}1 & 0\\
%-\tfrac{1}{2}s\omega(\lambda)^{-2s}\ee^{2S(\lambda;\chi,\tau)}\ee^{2 nh(\lambda;\chi,\tau)} & 1\end{bmatrix},\quad
%\lambda\in C_R^-,\quad\text{and}
%\end{equation}
%\begin{equation}
%\mathbf{X}^{(k)}_+(\lambda;\chi,\tau)=\mathbf{X}^{(k)}_-(\lambda;\chi,\tau)\begin{bmatrix} 1 & s\omega(\lambda)^{2s}\ee^{-2S(\lambda;\chi,\tau)}\ee^{-2 nh(\lambda;\chi,\tau)}\\ 0 & 1\end{bmatrix},\quad\lambda\in C_L^-,
%\end{equation}
%\begin{equation}
%\mathbf{X}_+^{(k)}(\lambda;\chi,\tau)=\mathbf{X}_-^{(k)}(\lambda;\chi,\tau)\begin{bmatrix}1 & 
%\tfrac{1}{2}s(\omega_+(\lambda)^{2s}-\omega_-(\lambda)^{2s})\ee^{-2S(\lambda;\chi,\tau)}\ee^{-2nh(\lambda;\chi,\tau)}\\0 & 1\end{bmatrix},\quad\lambda\in N^+,\quad\text{and}
%\end{equation}
%\begin{equation}
%\mathbf{X}^{(k)}_+(\lambda;\chi,\tau)=\mathbf{X}^{(k)}_-(\lambda;\chi,\tau)\begin{bmatrix}
%1 & 0\\-\tfrac{1}{2}s(\omega_+(\lambda)^{-2s}-\omega_-(\lambda)^{-2s})\ee^{2S(\lambda;\chi,\tau)}\ee^{2nh(\lambda;\chi,\tau)} & 1
%\end{bmatrix},\quad\lambda\in N^-.
%\end{equation}
%\textcolor{red}{In the alternative approach, the entire discussion about the Szeg\H{o} function and the transformation from $\mathbf{W}^{(k)}(\lambda;\chi,\tau)$ to $\mathbf{X}^{(k)}(\lambda;\chi,\tau)$ is to be removed, and we just stick with $\mathbf{W}^{(k)}(\lambda;\chi,\tau)$ going forward (no need for $\mathbf{X}^{(k)}(\lambda;\chi,\tau)$ at all).}

\subsection{Parametrix construction}
\label{sec:Airy-parametrix}
From the sign structure of $\mathrm{Re}(\ii h(\lambda;\chi,\tau))$ as indicated with shading in Figures~\ref{fig:Schi1}--\ref{fig:Stau1}, it is then clear that the jump matrices are exponentially small perturbations of the identity matrix except when $\lambda\in N=\Sigma_g$ and in small neighborhoods of the branch points $\lambda=A\pm \ii B$.  To deal with these, we first construct an \emph{outer parametrix} denoted 
%$\dot{\mathbf{X}}^{(k),\mathrm{out}}(\lambda;\chi,\tau)$ \textcolor{red}{or, 
$\dot{\mathbf{W}}^{(k),\mathrm{out}}(\lambda;\chi,\tau)$
%)} 
designed to solve the jump condition %\eqref{eq:X-jump-N-Schi-Stau} 
%\textcolor{red}{(equation for jump of $\mathbf{W}^{(k)}$ across $N$)} 
\eqref{eq:T-jump-N-Schi-Stau-ALT} for $\lambda\in\Sigma_g$ exactly, to be analytic for $\lambda\in\mathbb{C}\setminus\Sigma_g$, and to tend to the identity as $\lambda\to\infty$.  This is easily accomplished simply by diagonalization of the constant jump matrix, the eigenvalues of which are $\pm \ii$.  All solutions of the jump condition 
%\eqref{eq:X-jump-N-Schi-Stau} 
%\textcolor{red}{(equation for jump of $\mathbf{W}^{(k)}$ across $N$)} 
\eqref{eq:T-jump-N-Schi-Stau-ALT}
have singularities at the endpoints of $\Sigma_g$, and we select the unique solution with the mildest rate of growth at these two points:
%\begin{multline}
%\dot{\mathbf{X}}^{(k),\mathrm{out}}(\lambda;\chi,\tau)\defeq\\
%\ee^{-\ii(n\kappa(\chi,\tau)+s\gamma(\chi,\tau))\sigma_3}\ee^{\frac{1}{4}(1-s)\ii\pi\sigma_3}\mathbf{O}\left(\frac{\lambda-\lambda_0(\chi,\tau)}{\lambda-\lambda_0(\chi,\tau)^*}\right)^{\frac{1}{4}\sigma_3}\mathbf{O}^{-1}\ee^{-\frac{1}{4}(1-s)\ii\pi\sigma_3}\ee^{\ii(n\kappa(\chi,\tau)+s\gamma(\chi,\tau))\sigma_3},
%\label{eq:outer-parametrix-Schi-Stau}
%\end{multline}
%where 
%\begin{equation}
%\mathbf{O}\defeq\frac{1}{\sqrt{2}}\begin{bmatrix} 1 & \ii\\\ii & 1\end{bmatrix},\quad\det(\mathbf{O})=1,
%\label{eq:O-def-Schi-Stau}
%\end{equation}
%and where the central factor is uniquely determined to be analytic for $\lambda\in\mathbb{C}\setminus\Sigma_g$ and to tend to the identity matrix as $\lambda\to\infty$.  
%\textcolor{red}{The alternative version of this formula reads:
\begin{equation}
\dot{\mathbf{W}}^{(k),\mathrm{out}}(\lambda;\chi,\tau)\defeq\\
\ee^{-\ii M\gamma(\chi,\tau)\sigma_3}\mathbf{Q}y(\lambda;\chi,\tau)^{\sigma_3}\mathbf{Q}^{-1}\ee^{\ii M\gamma(\chi,\tau)\sigma_3},
\label{eq:outer-parametrix-Schi-Stau-ALT}
\end{equation}
where $\mathbf{Q}$ is the matrix defined in \eqref{eq:Q-def}, and where $y(\lambda;\chi,\tau)$ is the function analytic for $\lambda\in\mathbb{C}\setminus\Sigma_g$ determined by the conditions
\begin{equation}
y(\lambda;\chi,\tau)^4=\frac{\lambda-\lambda_0(\chi,\tau)}{\lambda-\lambda_0(\chi,\tau)^*},\quad\text{and $y(\lambda;\chi,\tau)\to 1$ as $\lambda\to\infty$}.
\label{eq:y-def}
\end{equation}
%}
Note that the only dependence on 
%$n$ \textcolor{red}{(or, 
$M$
%)} 
enters via the oscillatory factors 
%$\ee^{\pm \ii n\kappa(\chi,\tau)\sigma_3}$ 
%\textcolor{red}{(or, $\ee^{\pm\ii M\kappa(\chi,\tau)\sigma_3}$)}, 
$\ee^{\pm\ii M\gamma(\chi,\tau)\sigma_3}$,
so the outer parametrix
%$\dot{\mathbf{X}}^{(k),\mathrm{out}}(\lambda;\chi,\tau)$ 
%\textcolor{red}{(or, $\dot{\mathbf{W}}^{(k),\mathrm{out}}(\lambda;\chi,\tau)$)} 
$\dot{\mathbf{W}}^{(k),\mathrm{out}}(\lambda;\chi,\tau)$
is bounded as 
%$n\to\infty$ \textcolor{red}{(or, as $M\to\infty$)}, 
$M\to\infty$,
provided that $\lambda$ is bounded away from $\lambda_0(\chi,\tau)$ and $\lambda_0(\chi,\tau)^*$.

Next, we let $D_{\lambda_0}(\delta)$ and $D_{\lambda_0^*}(\delta)=D_{\lambda_0}(\delta)^*$ be disks of small radius $\delta$ independent of 
%$n$ 
$M$
centered at $\lambda=\lambda_0(\chi,\tau)=A(\chi,\tau)+\ii B(\chi,\tau)$ and $\lambda=\lambda_0(\chi,\tau)^*$ respectively.  
%Since $h'(\lambda;\chi,\tau)$ vanishes like a square root as $\lambda\to \lambda_0(\chi,\tau)$ and $h(\lambda_0(\chi,\tau);\chi,\tau)=\ii\kappa(\chi,\tau)$, there is a univalent function $f^+(\lambda;\chi,\tau)$ defined on $D^+(\delta)$ with $f^+(\lambda_0(\chi,\tau);\chi,\tau)=0$ such that 
%\begin{equation}
%f^+(\lambda;\chi,\tau)^3=(2h(\lambda;\chi,\tau)-2\ii\kappa(\chi,\tau))^2,\quad \lambda\in D^+(\delta).
%\label{eq:Airy-map-Schi-Stau}
%\end{equation}
%Moreover, the univalent solution of \eqref{eq:Airy-map-Schi-Stau} and the jump contours $N\cap D^+(\delta)$, $N^+\cap D^+(\delta)$, and $C_L^+\cap D^+(\delta)$ can be chosen so that $\lambda\in N\cap D^+(\delta)$ implies $f^+(\lambda;\chi,\tau)<0$, $\lambda\in N^+\cap D^+(\delta)$ implies $f^+(\lambda;\chi,\tau)>0$, and $\lambda\in C_L^+\cap D^+(\delta)$ implies that either $\arg(f^+(\lambda;\chi,\tau))=\tfrac{2}{3}\pi$ or $\arg(f^+(\lambda;\chi,\tau))=-\tfrac{2}{3}\pi$.  Let $D^+_+(\delta)$ (resp., $D^+_-(\delta)$) denote the part of $D^+(\delta)$ to the left (resp., right) of $N\cup N^+$.  Define a matrix $\mathbf{Y}^{(k)}(\lambda;\chi,\tau)$ within $D^+(\delta)$ by
%\begin{equation}
%\mathbf{Y}^{(k)}(\lambda;\chi,\tau)\defeq\mathbf{X}^{(k)}(\lambda;\chi,\tau)\omega(\lambda)^{s\sigma_3}\ee^{-S(\lambda;\chi,\tau)\sigma_3}\ee^{-\ii n\kappa(\chi,\tau)\sigma_3}\begin{cases}
%\ee^{\frac{1}{4}(1-s)\ii\pi\sigma_3},&\lambda\in D^+_+(\delta)\\
%\ee^{\frac{1}{4}(1+s)\ii\pi\sigma_3},&\lambda\in D^+_-(\delta).
%\end{cases}
%\label{eq:Y-from-X-Schi-Stau}
%\end{equation}
%\textcolor{red}{The alternate version of this construction is as follows:
Since $h'(\lambda;\chi,\tau)$ vanishes like a square root as $\lambda\to \lambda_0(\chi,\tau)$ and $h_+(\lambda_0(\chi,\tau);\chi,\tau)+h_-(\lambda_0(\chi,\tau);\chi,\tau)=2\gamma(\chi,\tau)$, there is a univalent function $f_{\lambda_0}(\lambda;\chi,\tau)$ defined on $D_{\lambda_0}(\delta)$ with $f_{\lambda_0}(\lambda_0(\chi,\tau);\chi,\tau)=0$ such that 
\begin{equation}
f_{\lambda_0}(\lambda;\chi,\tau)^3=-(h_+(\lambda;\chi,\tau)+h_-(\lambda;\chi,\tau)-2\gamma(\chi,\tau))^2,\quad \lambda\in D_{\lambda_0}(\delta),
\label{eq:Airy-map-Schi-Stau-ALT}
\end{equation}
in which the sum of boundary values of $h$ is analytically continued from $N^+$ to $D_{\lambda_0}(\delta)\setminus N$ by means of the identity $h_+(\lambda;\chi,\tau)-h_-(\lambda;\chi,\tau)=-2\pi$ for $\lambda\in N^+$.
Moreover, the univalent solution of \eqref{eq:Airy-map-Schi-Stau-ALT} and the jump contours $N\cap D_{\lambda_0}(\delta)$, $N^+\cap D_{\lambda_0}(\delta)$, and $C_L^+\cap D_{\lambda_0}(\delta)$ can be chosen so that $\lambda\in N\cap D_{\lambda_0}(\delta)$ implies $f_{\lambda_0}(\lambda;\chi,\tau)<0$, $\lambda\in N^+\cap D_{\lambda_0}(\delta)$ implies $f_{\lambda_0}(\lambda;\chi,\tau)>0$, and $\lambda\in C_L^+\cap D_{\lambda_0}(\delta)$ implies that either $\arg(f_{\lambda_0}(\lambda;\chi,\tau))=\tfrac{2}{3}\pi$ or $\arg(f_{\lambda_0}(\lambda;\chi,\tau))=-\tfrac{2}{3}\pi$.  
Define a matrix $\mathbf{X}^{(k)}(\lambda;\chi,\tau)$ within $D_{\lambda_0}(\delta)$ by
\begin{equation}
\mathbf{X}^{(k)}(\lambda;\chi,\tau)\defeq\mathbf{W}^{(k)}(\lambda;\chi,\tau)\ee^{-\ii M\gamma(\chi,\tau)\sigma_3}\ee^{\frac{1}{4}\ii\pi\sigma_3},\quad\lambda\in D_{\lambda_0}(\delta).
\label{eq:Y-from-X-Schi-Stau-ALT}
\end{equation}
%}
Then, using again $M=\tfrac{1}{2}k+\tfrac{1}{4}$ for $k\in\mathbb{Z}_{>0}$, the jump conditions satisfied by 
%$\mathbf{Y}^{(k)}(\lambda;\chi,\tau)$ 
%\textcolor{red}{(or, $\mathbf{X}^{(k)}(\lambda;\chi,\tau)$)} 
$\mathbf{X}^{(k)}(\lambda;\chi,\tau)$
can be written in a simple form, in terms of the variable (rescaled conformal coordinate on $D_{\lambda_0}(\delta)$) 
%$\zeta\defeq n^{\frac{2}{3}}f^+(\lambda;\chi,\tau)$ 
%\textcolor{red}{(or, $\zeta\defeq M^\frac{2}{3}f^+(\lambda;\chi,\tau)$)}:
$\zeta\defeq M^\frac{2}{3}f_{\lambda_0}(\lambda;\chi,\tau)$:
\begin{equation}
\mathbf{X}^{(k)}_+(\lambda;\chi,\tau)=\mathbf{X}^{(k)}_-(\lambda;\chi,\tau)\begin{bmatrix}1 &\ee^{-\zeta^\frac{3}{2}} \\ 0 & 1\end{bmatrix},\quad \arg(\zeta)=0,
\label{eq:Airy-jump-first}
\end{equation}
\begin{equation}
\mathbf{X}^{(k)}_+(\lambda;\chi,\tau)=\mathbf{X}^{(k)}_-(\lambda;\chi,\tau)\begin{bmatrix}1 & 0\\-\ee^{\zeta^\frac{3}{2}} & 1\end{bmatrix},\quad\arg(\zeta)=\pm\tfrac{2}{3}\pi,\quad\text{and}
\end{equation}
\begin{equation}
\mathbf{X}^{(k)}_+(\lambda;\chi,\tau)=\mathbf{X}^{(k)}_-(\lambda;\chi,\tau)\begin{bmatrix}0 & -1\\
1 & 0\end{bmatrix},\quad\arg(-\zeta)=0,
\label{eq:Airy-jump-last}
\end{equation}
where for uniformity all four rays are taken to be oriented away from the origin in the $\zeta$-plane.
%\textcolor{red}{(Alternately, with $\mathbf{Y}^{(k)}$ replaced by $\mathbf{X}^{(k)}$ in \eqref{eq:Airy-jump-first}--\eqref{eq:Airy-jump-last}.)}
There exists a unique matrix function $\mathbf{A}(\zeta)$ with the following properties:
\begin{itemize}
\item
$\mathbf{A}(\zeta)$ is analytic for $0<|\arg(\zeta)|<\tfrac{2}{3}\pi$ and $\tfrac{2}{3}\pi<|\arg(\zeta)|<\pi$ (four sectors);
\item $\mathbf{A}(\zeta)$ takes continuous boundary values from each sector satisfying the same jump conditions written in \eqref{eq:Airy-jump-first}--\eqref{eq:Airy-jump-last};
\item $\mathbf{A}(\zeta)$ has uniform asymptotics in all directions of the complex plane given by 
%\begin{equation}
%\mathbf{A}(\zeta)\mathbf{O}\zeta^{-\frac{1}{4}\sigma_3}=\begin{bmatrix}1+O(\zeta^{-3}) & O(\zeta^{-1})\\O(\zeta^{-2}) & 1+O(\zeta^{-3})\end{bmatrix},\quad\zeta\to\infty,
%\end{equation}
%\textcolor{red}{Alternately (equivalantly):
\begin{equation}
\mathbf{A}(\zeta)\ee^{-\frac{1}{4}\ii\pi\sigma_3}\mathbf{Q}\ee^{\frac{1}{4}\ii\pi\sigma_3}\zeta^{-\frac{1}{4}\sigma_3}=\begin{bmatrix}1+O(\zeta^{-3}) & O(\zeta^{-1})\\O(\zeta^{-2}) & 1+O(\zeta^{-3})\end{bmatrix},\quad\zeta\to\infty,
\end{equation}
%}
%where $\mathbf{O}$ is the matrix defined in \eqref{eq:O-def-Schi-Stau}.
where $\mathbf{Q}$ is the matrix defined in \eqref{eq:Q-def}.
\end{itemize}
It is well-known that the unique solution of these Riemann-Hilbert conditions can be written explicitly in terms of Airy functions, and the reader can find a complete development of the solution in \cite[Appendix B]{BothnerM20}.
%We make a similar transformation of the outer parametrix as in \eqref{eq:Y-from-X-Schi-Stau}:
%\begin{equation}
%\dot{\mathbf{Y}}^{(k),\mathrm{out}}(\lambda;\chi,\tau)\defeq\dot{\mathbf{X}}^{(k),\mathrm{out}}(\lambda;\chi,\tau)\omega(\lambda)^{s\sigma_3}\ee^{-S(\lambda;\chi,\tau)\sigma_3}\ee^{-\ii n\kappa(\chi,\tau)\sigma_3}\begin{cases}\ee^{\frac{1}{4}(1-s)\ii\pi\sigma_3},& \lambda\in D^+_+(\delta)\\
%\ee^{\frac{1}{4}(1+s)\ii\pi\sigma_3},&\lambda\in D^+_-(\delta).
%\end{cases}
%\end{equation}
%It is straightforward to check that $\dot{\mathbf{Y}}^{(k),\mathrm{out}}(\lambda;\chi,\tau)$ satisfies the jump condition \eqref{eq:Airy-jump-last} but otherwise is analytic within $D^+$; it then follows that the matrix 
%\begin{equation}
%\mathbf{H}(\lambda;\chi,\tau)\defeq\ee^{\ii n\kappa(\chi,\tau)\sigma_3}\dot{\mathbf{Y}}^{(k),\mathrm{out}}(\lambda;\chi,\tau)\mathbf{O}f^+(\lambda;\chi,\tau)^{-\frac{1}{4}\sigma_3}
%\end{equation}
%is analytic for $\lambda\in D^+$ and is independent of $n$ (although it depends on the parity index $s$).
%\textcolor{red}{Alternatively, we simply define the matrix
Next, we define the matrix function
\begin{equation}
\mathbf{H}(\lambda;\chi,\tau)\defeq\ee^{\ii M\gamma(\chi,\tau)\sigma_3}\dot{\mathbf{W}}^{(k),\mathrm{out}}(\lambda;\chi,\tau)\ee^{-\ii M\gamma(\chi,\tau)\sigma_3}\mathbf{Q}f_{\lambda_0}(\lambda;\chi,\tau)^{-\frac{1}{4}\sigma_3}\ee^{\frac{1}{4}\ii\pi\sigma_3}
\end{equation}
and note that it follows from the definition of the conformal map $\lambda\mapsto f_{\lambda_0}(\lambda;\chi,\tau)$ and the definition \eqref{eq:outer-parametrix-Schi-Stau-ALT} of $\dot{\mathbf{W}}^{(k),\mathrm{out}}(\lambda;\chi,\tau)$ that $\mathbf{H}(\lambda;\chi,\tau)$ is analytic for $\lambda\in D_{\lambda_0}(\delta)$ and is independent of $M$.
%}
We use $\mathbf{H}(\lambda;\chi,\tau)$ and $\mathbf{A}(\zeta)$ to define an \emph{inner parametrix} on $D_{\lambda_0}(\delta)$ as follows:
%\begin{multline}
%\dot{\mathbf{X}}^{(k),\mathrm{in}}(\lambda;\chi,\tau)\defeq\\
%\ee^{-\ii n\kappa(\chi,\tau)\sigma_3}\mathbf{H}(\lambda;\chi,\tau)n^{-\frac{1}{6}\sigma_3}\mathbf{A}(n^\frac{2}{3}f^+(\lambda;\chi,\tau))\ee^{\ii n\kappa(\chi,\tau)\sigma_3}\ee^{S(\lambda;\chi,\tau)\sigma_3}\omega(\lambda)^{-s\sigma_3}\begin{cases}
%\ee^{-\frac{1}{4}(1-s)\ii\pi\sigma_3},&\lambda\in D_+^+(\delta)\\
%\ee^{-\frac{1}{4}(1+s)\ii\pi\sigma_3},&\lambda\in D_-^+(\delta).
%\end{cases}
%\end{multline}
%\textcolor{red}{The alternate version reads:
\begin{equation}
\dot{\mathbf{W}}^{(k),\lambda_0}(\lambda;\chi,\tau):=\ee^{-\ii M\gamma(\chi,\tau)\sigma_3}\mathbf{H}(\lambda;\chi,\tau)M^{-\frac{1}{6}\sigma_3}\mathbf{A}(M^\frac{2}{3}f_{\lambda_0}(\lambda;\chi,\tau))\ee^{-\frac{1}{4}\ii\pi\sigma_3}\ee^{\ii M\gamma(\chi,\tau)\sigma_3},\quad\lambda\in D_{\lambda_0}(\delta).
\end{equation}
%}
It is easy to check that 
%$\dot{\mathbf{X}}^{(k),\mathrm{in}}(\lambda;\chi,\tau)$ 
%\textcolor{red}{(or, $\dot{\mathbf{W}}^{(k),\mathrm{in}}(\lambda;\chi,\tau)$)} 
$\dot{\mathbf{W}}^{(k),\lambda_0}(\lambda;\chi,\tau)$
takes continuous boundary values that satisfy exactly the same jump conditions within $D_{\lambda_0}(\delta)$ as do those of 
%$\mathbf{X}^{(k)}(\lambda;\chi,\tau)$ \textcolor{red}{(or, $\mathbf{W}^{(k)}(\lambda;\chi,\tau)$)} 
$\mathbf{W}^{(k)}(\lambda;\chi,\tau)$
itself. Also, since $\zeta$ is large of size 
%$n^\frac{2}{3}$ \textcolor{red}{(or, 
$M^\frac{2}{3}$
%)} 
when $\lambda\in \partial D_{\lambda_0}(\delta)$, 
%\begin{multline}
%\dot{\mathbf{X}}^{(k),\mathrm{in}}(\lambda;\chi,\tau)\dot{\mathbf{X}}^{(k),\mathrm{out}}(\lambda;\chi,\tau)^{-1} \\
%\begin{aligned}
%&= \ee^{-\ii n\kappa(\chi,\tau)\sigma_3}\mathbf{H}(\lambda;\chi,\tau)n^{-\frac{1}{6}\sigma_3}\mathbf{A}(\zeta)\mathbf{O}\zeta^{-\frac{1}{4}\sigma_3}n^{\frac{1}{6}\sigma_3}\mathbf{H}(\lambda;\chi,\tau)^{-1}\ee^{\ii n\kappa(\chi,\tau)\sigma_3}\\
%&=\ee^{-\ii n\kappa(\chi,\tau)\sigma_3}\mathbf{H}(\lambda;\chi,\tau)\begin{bmatrix}1+O(\zeta^{-3}) & O(n^{-\frac{1}{3}}\zeta^{-1})\\O(n^\frac{1}{3}\zeta^{-2}) & 1+O(\zeta^{-3})\end{bmatrix}
%\mathbf{H}(\lambda;\chi,\tau)^{-1}\ee^{\ii n\kappa(\chi,\tau)\sigma_3}\\
%&=\mathbb{I}+O(n^{-1}),\quad\lambda\in\partial D^+(\delta).
%\end{aligned}
%\label{eq:Dplus-mismatch-Schi-Stau}
%\end{multline}
%\textcolor{red}{The alternative version reads:
\begin{multline}
\dot{\mathbf{W}}^{(k),\lambda_0}(\lambda;\chi,\tau)\dot{\mathbf{W}}^{(k),\mathrm{out}}(\lambda;\chi,\tau)^{-1} \\
\begin{aligned}
&= \ee^{-\ii M\gamma(\chi,\tau)\sigma_3}\mathbf{H}(\lambda;\chi,\tau)M^{-\frac{1}{6}\sigma_3}\mathbf{A}(\zeta)\ee^{-\frac{1}{4}\ii\pi\sigma_3}\mathbf{Q}\ee^{\frac{1}{4}\ii\pi\sigma_3}\zeta^{-\frac{1}{4}\sigma_3}M^{\frac{1}{6}\sigma_3}\mathbf{H}(\lambda;\chi,\tau)^{-1}\ee^{\ii M\gamma(\chi,\tau)\sigma_3}\\
&=\ee^{-\ii M\gamma(\chi,\tau)\sigma_3}\mathbf{H}(\lambda;\chi,\tau)\begin{bmatrix}1+O(\zeta^{-3}) & O(M^{-\frac{1}{3}}\zeta^{-1})\\O(M^\frac{1}{3}\zeta^{-2}) & 1+O(\zeta^{-3})\end{bmatrix}
\mathbf{H}(\lambda;\chi,\tau)^{-1}\ee^{\ii M\gamma(\chi,\tau)\sigma_3}\\
&=\mathbb{I}+O(M^{-1}),\quad\lambda\in\partial D_{\lambda_0}(\delta).
\end{aligned}
\label{eq:Dplus-mismatch-Schi-Stau-ALT}
\end{multline}
%}
Since the matrix $\mathbf{W}^{(k)}(\lambda;\chi,\tau)$ satisfies $\mathbf{W}^{(k)}(\lambda^*;\chi,\tau)=\sigma_2\mathbf{W}^{(k)}(\lambda;\chi,\tau)^*\sigma_2$, 
%\textcolor{red}{(or, replacing $\mathbf{X}\mapsto\mathbf{W}$)}, 
we may define a second inner parametrix for $\lambda\in D_{\lambda_0^*}(\delta)$ to respect this symmetry.

We combine the inner and outer parametrices into a \emph{global parametrix} by setting
%\begin{equation}
%\dot{\mathbf{X}}^{(k)}(\lambda;\chi,\tau)\defeq\begin{cases}
%\dot{\mathbf{X}}^{(k),\mathrm{in}}(\lambda;\chi,\tau),&\quad\lambda\in D^+(\delta)\\
%\sigma_2\dot{\mathbf{X}}^{(k),\mathrm{in}}(\lambda^*;\chi,\tau)^*\sigma_2,&\quad\lambda\in D^-(\delta)\\
%\dot{\mathbf{X}}^{(k),\mathrm{out}}(\lambda;\chi,\tau),&\quad\lambda\in\mathbb{C}\setminus(D^+(\delta)\cup D^-(\delta)\cup\Sigma_g).
%\end{cases}
%\label{eq:global-parametrix-Schi-Stau}
%\end{equation}
%\textcolor{red}{Alternatively, this reads:
\begin{equation}
\dot{\mathbf{W}}^{(k)}(\lambda;\chi,\tau)\defeq\begin{cases}
\dot{\mathbf{W}}^{(k),\lambda_0}(\lambda;\chi,\tau),&\quad\lambda\in D_{\lambda_0}(\delta),\\
\sigma_2\dot{\mathbf{W}}^{(k),\lambda_0}(\lambda^*;\chi,\tau)^*\sigma_2,&\quad\lambda\in D_{\lambda_0^*}(\delta),\\
\dot{\mathbf{W}}^{(k),\mathrm{out}}(\lambda;\chi,\tau),&\quad\lambda\in\mathbb{C}\setminus(\overline{D_{\lambda_0}(\delta)\cup D_{\lambda_0^*}(\delta)}\cup\Sigma_g).
\end{cases}
\label{eq:global-parametrix-Schi-Stau-ALT}
\end{equation}
%}

\subsection{Error analysis and asymptotic formula for $\psi_k(M\chi,M\tau)$ for $(\chi,\tau)\in \exterior$.}
As in Section~\ref{sec:small-norm-channels}, we define an error matrix to compare $\mathbf{W}^{(k)}(\lambda;\chi,\tau)$ with its global parametrix defined in \eqref{eq:global-parametrix-Schi-Stau-ALT}:
\begin{equation}
\mathbf{F}^{(k)}(\lambda;\chi,\tau):=\mathbf{W}^{(k)}(\lambda;\chi,\tau)\dot{\mathbf{W}}^{(k)}(\lambda;\chi,\tau)^{-1}.
\end{equation}
This matrix can be considered to be analytic in $\lambda$ except on a contour $\Sigma_\mathbf{F}$ consisting of the union of (i) those arcs of the jump contour for $\mathbf{W}^{(k)}(\lambda;\chi,\tau)$ other than $\Sigma_g$ outside the disks $D_{\lambda_0}(\delta)$ and $D_{\lambda_0^*}(\delta)$ and (ii) the disk boundaries $\partial D_{\lambda_0}(\delta)$ and $\partial D_{\lambda_0^*}(\delta)$, which we take to have clockwise orientation.  Also, $\mathbf{F}^{(k)}(\lambda;\chi,\tau)$ takes continuous boundary values on $\Sigma_\mathbf{F}$ from each connected component of $\mathbb{C}\setminus\Sigma_\mathbf{F}$, and $\mathbf{F}^{(k)}(\lambda;\chi,\tau)\to\mathbb{I}$ as $\lambda\to\infty$. Because $\delta>0$ is held fixed as $M\to\infty$ and because the outer parametrix is uniformly bounded on arcs of type (i), there is a constant $\nu>0$ such that on those arcs we have the uniform estimate $\mathbf{F}^{(k)}_+(\lambda;\chi,\tau)=\mathbf{F}^{(k)}_-(\lambda;\chi,\tau)(\mathbb{I}+O(\ee^{-\nu M}))$.  On the circular arcs of type (ii), the estimate \eqref{eq:Dplus-mismatch-Schi-Stau-ALT} and its Schwarz reflection guarantee that on those arcs we have the uniform estimate $\mathbf{F}^{(k)}_+(\lambda;\chi,\tau)=\mathbf{F}^{(k)}_-(\lambda;\chi,\tau)(\mathbb{I}+O(M^{-1}))$. By small-norm theory it then follows that $\mathbf{F}^{(k)}_-(\lambda;\chi,\tau)=\mathbb{I}+O(M^{-1})$ holds in the $L^2$ sense on the union of arcs of types (i) and (ii) as $M\to+\infty$.  Using the Cauchy integral representation \eqref{eq:F-Cauchy-channels} then shows that $\mathbf{F}^{(k)}(\lambda;\chi,\tau)=\mathbb{I}+\lambda^{-1}\mathbf{F}_1^{(k)}(\chi,\tau) + O(\lambda^{-2})$ as $\lambda\to\infty$ where $\mathbf{F}_1^{(k)}(\chi,\tau)=O(M^{-1})$ holds uniformly for $(\chi,\tau)$ in compact subsets of $\exterior$.

Using the fact that for $|\lambda|$ sufficiently large, 
%$\mathbf{S}^{(k)}(\lambda;\chi,\tau)=\mathbf{X}^{(k)}(\lambda;\chi,\tau)\ee^{-ng(\lambda;\chi,\tau)\sigma_3}\ee^{-S(\lambda;\chi,\tau)\sigma_3}$ while $\dot{\mathbf{X}}^{(k)}(\lambda;\chi,\tau)=\dot{\mathbf{X}}^{(k),\mathrm{out}}(\lambda;\chi,\tau)$ 
%\textcolor{red}{(or, 
$\mathbf{S}^{(k)}(\lambda;\chi,\tau)=\mathbf{W}^{(k)}(\lambda;\chi,\tau)\ee^{-\ii Mg(\lambda;\chi,\tau)\sigma_3}$ while $\dot{\mathbf{W}}^{(k)}(\lambda;\chi,\tau)=\dot{\mathbf{W}}^{(k),\mathrm{out}}(\lambda;\chi,\tau)$,
%)}, 
%the analogue of \eqref{eq:psi-k-exact-channels} is 
from \eqref{eq:psi-k-S} we have the following exact formula
%\begin{multline}
%\psi_k(n\chi,n\tau)=2\ii\ee^{-\ii n\tau}\lim_{\lambda\to\infty}\lambda X^{(k)}_{12}(\lambda;\chi,\tau)\ee^{ng(\lambda;\chi,\tau)}\ee^{S(\lambda;\chi,\tau)}\\
%{}=2\ii\ee^{-\ii n\tau}\lim_{\lambda\to\infty}\lambda\left[F^{(k)}_{11}(\lambda;\chi,\tau)\dot{X}^{(k),\mathrm{out}}_{12}(\lambda;\chi,\tau)+F^{(k)}_{12}(\lambda;\chi,\tau)\dot{X}^{(k),\mathrm{out}}_{22}(\lambda;\chi,\tau)\right]\ee^{ng(\lambda;\chi,\tau)}\ee^{S(\lambda;\chi,\tau)}.
%\end{multline}
%\textcolor{red}{Alternatively, this reads:
\begin{equation}
\begin{split}
\psi_k(M\chi,M\tau)&=2\ii\ee^{-\ii M\tau}\lim_{\lambda\to\infty}\lambda W^{(k)}_{12}(\lambda;\chi,\tau)\ee^{\ii Mg(\lambda;\chi,\tau)}\\
&=2\ii\ee^{-\ii M\tau}\lim_{\lambda\to\infty}\lambda\left[F^{(k)}_{11}(\lambda;\chi,\tau)\dot{W}^{(k),\mathrm{out}}_{12}(\lambda;\chi,\tau)\right.\\
&\qquad\qquad\qquad\qquad\qquad\qquad{}\left.+F^{(k)}_{12}(\lambda;\chi,\tau)\dot{W}^{(k),\mathrm{out}}_{22}(\lambda;\chi,\tau)\right]\ee^{\ii Mg(\lambda;\chi,\tau)}.
\end{split}
\end{equation}
%}
%Since $\mathbf{F}^{(k)}(\lambda;\chi,\tau)\to\mathbb{I}$, $\dot{\mathbf{X}}^{(k),\mathrm{out}}(\lambda;\chi,\tau)\to\mathbb{I}$, $g(\lambda;\chi,\tau)\to 0$, and $S(\lambda;\chi,\tau)\to 0$ as $\lambda\to\infty$, this simplifies to
%\begin{equation}
%\psi_k(n\chi,n\tau)=2\ii\ee^{-\ii n\tau}\lim_{\lambda\to\infty}\lambda\left[\dot{X}^{(k),\mathrm{out}}_{12}(\lambda;\chi,\tau) +F^{(k)}_{12}(\lambda;\chi,\tau)\right].
%\end{equation}
%\textcolor{red}{Alternately, since 
Since $\mathbf{F}^{(k)}(\lambda;\chi,\tau)\to\mathbb{I}$, $\dot{\mathbf{W}}^{(k),\mathrm{out}}(\lambda;\chi,\tau)\to\mathbb{I}$, and $g(\lambda;\chi,\tau)\to 0$ as $\lambda\to\infty$, this simplifies to
\begin{equation}
\psi_k(M\chi,M\tau)=2\ii\ee^{-\ii M\tau}\lim_{\lambda\to\infty}\lambda\left[\dot{W}^{(k),\mathrm{out}}_{12}(\lambda;\chi,\tau) +F^{(k)}_{12}(\lambda;\chi,\tau)\right].
\end{equation}
%}
%Using \eqref{eq:outer-parametrix-Schi-Stau} and that $\mathbf{F}_1^{(k)}(\chi,\tau)=O(n^{-1})$,
%recalling $B(\chi,\tau)=\mathrm{Im}(\lambda_0(\chi,\tau))>0$ we obtain
%\begin{equation}
%\psi_k(n\chi,n\tau)=-\ii s\ee^{-\ii n\tau}\ee^{-2\ii (n\kappa(\chi,\tau)+s\gamma(\chi,\tau))}B(\chi,\tau) + O(n^{-1}).
%\label{eq:psi-k-shelves-chi-tau}
%\end{equation}
%\textcolor{red}{Alternately, using 
Using \eqref{eq:outer-parametrix-Schi-Stau-ALT} and that $\mathbf{F}_1^{(k)}(\chi,\tau)=O(M^{-1})$, recalling $B(\chi,\tau)=\mathrm{Im}(\lambda_0(\chi,\tau))>0$ we obtain 
\begin{equation}
\psi_k(M\chi,M\tau)=B(\chi,\tau)\ee^{-\ii M\tau}\ee^{-2\ii M\gamma(\chi,\tau)} + O(M^{-1}),
\label{eq:psi-k-shelves-chi-tau-ALT}
\end{equation}
which completes the proof of Theorem~\ref{thm:exterior}.
%}