This paper is a continuation of a study, begun in \cite{BilmanLM20}, of high-order rogue-wave solutions of the focusing nonlinear Schr\"odinger equation.  As in \cite{BilmanLM20}, the starting point is a Riemann-Hilbert problem characterization of the fundamental rogue wave of order $k$, $k\geq 0$, which was originally obtained in \cite{BilmanM19} and which we now describe. Let $\Sigma_\mathrm{c}$ denote a Schwarz-symmetric simple arc connecting endpoints $\lambda=\pm\ii$ with upward orientation, let $\rho:\mathbb{C}\setminus\Sigma_\mathrm{c}\to\mathbb{C}$ be the analytic function satisfying $\rho(\lambda)^2=\lambda^2+1$ and $\rho(\lambda)=\lambda+O(\lambda^{-1})$ as $\lambda\to\infty$, and let $\Sigma_\circ$ denote a Schwarz-symmetric Jordan curve with $\Sigma_\mathrm{c}$ in its interior and let $\Sigma_\circ$ have clockwise orientation.  
Also, let
\begin{equation}
\mathbf{Q}:=\frac{1}{\sqrt{2}}\begin{bmatrix} 1 & -1 \\ 1 & 1\end{bmatrix},
\label{eq:Q-def}
\end{equation}
and let $\mathbf{E}(\lambda)$ denote the matrix function defined for $\lambda\in\mathbb{C}\setminus\Sigma_\mathrm{c}$ by
\begin{equation}
\mathbf{E}(\lambda):= f(\lambda)\begin{bmatrix}1 & \ii (\lambda-\rho(\lambda))\\\ii(\lambda-\rho(\lambda)) & 1\end{bmatrix},\quad\lambda\in\mathbb{C}\setminus\Sigma_\mathrm{c},
\label{eq:E-def}
\end{equation}
where $f(\lambda)$ is the function analytic for $\lambda\in\mathbb{C}\setminus\Sigma_\mathrm{c}$ that satisfies 
\begin{equation}
f(\lambda)^2=\frac{\lambda+\rho(\lambda)}{2\rho(\lambda)}\quad\text{and $f(\lambda)\to 1$ as $\lambda\to\infty$.}
\label{eq:f-squared}
\end{equation}
This matrix $\mathbf{E}(\lambda)$ is analytic in its domain of definition and has unit determinant. 
It is convenient to introduce a sign $s=(-1)^k$ and express the order $k\in\mathbb{Z}_{\ge 0}$ in terms of another integer $n\in\mathbb{Z}_{\ge 0}$ and $s$ by
\begin{equation}
k=2n+\frac{1}{2}(s-1)\quad\Longleftrightarrow\quad n=\frac{1}{4}(2k+1-s).
\label{eq:k-vs-n}
\end{equation}
%Thus $s=1$ (resp., $s=-1$) means $k=2n$ is even (resp., $k=2n-1$ is odd), i.e., $s=(-1)^k$.  
Each value of $n\in\mathbb{Z}_{>0}$ corresponds to two consecutive values of $k$, one of each parity; however $n=0$ corresponds to $k=0$ only.  Finally, let $B(\lambda)$ denote the elementary Blaschke factor
\begin{equation}
B(\lambda):=\frac{\lambda-\ii}{\lambda+\ii}.
\label{eq:Blaschke}
\end{equation}
In the following problem as in the rest of the paper, boundary values taken from the left/right are denoted with a subscript $+$/$-$, and $\sigma_3$ denotes one of the Pauli matrices:
\begin{equation}
\sigma_1:=\begin{bmatrix}0&1\\1 & 0\end{bmatrix},\quad
\sigma_2:=\begin{bmatrix}0 & -\ii\\\ii & 0\end{bmatrix},\quad
\sigma_3:=\begin{bmatrix}1 & 0\\0 & -1\end{bmatrix}.
\end{equation} 
\begin{rhp}[Rogue wave of order $k$]
Let $(x,t)\in\mathbb{R}^2$ be arbitrary parameters, and let $k\in\mathbb{Z}_{\ge 0}$.  Find a $2\times 2$ matrix $\mathbf{M}^{(k)}(\lambda;x,t)$ with the following properties:
\begin{itemize}
\item[]\textbf{Analyticity:}  $\mathbf{M}^{(k)}(\lambda;x,t)$ is analytic in $\lambda$ for $\lambda\in\mathbb{C}\setminus(\Sigma_\circ\cup\Sigma_\mathrm{c})$, and it takes continuous boundary values on $\Sigma_\circ\cup\Sigma_\mathrm{c}$.
\item[]\textbf{Jump conditions:}  The boundary values on the jump contour $\Sigma_\circ\cup\Sigma_\mathrm{c}$ are related as follows:
\begin{equation}
\mathbf{M}_+^{(k)}(\lambda;x,t)=\mathbf{M}_-^{(k)}(\lambda;x,t)\ee^{2\ii\rho_+(\lambda)(x+\lambda t)\sigma_3},\quad \lambda\in\Sigma_\mathrm{c},
\label{eq:jump-cut}
\end{equation}
and
\begin{equation}
\mathbf{M}_+^{(k)}(\lambda;x,t)=\mathbf{M}_-^{(k)}(\lambda;x,t)\ee^{-\ii\rho(\lambda)(x+\lambda t)\sigma_3}\mathbf{Q}
%\left(\frac{\lambda-\ii}{\lambda+\ii}\right)^{sn\sigma_3}
B(\lambda)^{sn\sigma_3}
\mathbf{Q}^{-1}\mathbf{E}(\lambda)\ee^{\ii\rho(\lambda)(x+\lambda t)\sigma_3},\quad
\lambda\in\Sigma_\circ,
\end{equation}
where $s=\pm 1$ is the parity index of $k$, and $n$ is given by \eqref{eq:k-vs-n}.
%and if $k=2n$, $n\in\mathbb{Z}_{\ge 0}$,
%\begin{equation}
%\mathbf{M}_+^{(k)}(\lambda;x,t)=\mathbf{M}_-^{(k)}(\lambda;x,t)\ee^{-\ii\rho(\lambda)(x+\lambda t)\sigma_3}\mathbf{Q}\left(\frac{\lambda-\ii}{\lambda+\ii}\right)^{n\sigma_3}\mathbf{Q}^{-1}\mathbf{E}(\lambda)\ee^{\ii\rho(\lambda)(x+\lambda t)\sigma_3},\quad
%\lambda\in\Sigma_\circ
%\end{equation}
%while if instead $k=2n-1$, $n\in\mathbb{Z}_{>0}$,
%\begin{equation}
%\mathbf{M}_+^{(k)}(\lambda;x,t)=\mathbf{M}_-^{(k)}(\lambda;x,t)\ee^{-\ii\rho(\lambda)(x+\lambda t)\sigma_3}\mathbf{Q}
%\left(\frac{\lambda+\ii}{\lambda-\ii}\right)^{n\sigma_3}\mathbf{Q}^{-1}\mathbf{E}(\lambda)\ee^{\ii\rho(\lambda)(x+\lambda t)\sigma_3},\quad\lambda\in\Sigma_\circ.
%\end{equation}
\item[]\textbf{Normalization:}  $\mathbf{M}^{(k)}(\lambda;x,t)\to\mathbb{I}$ as $\lambda\to\infty$. 
\end{itemize}
\label{rhp:rogue-wave}
\end{rhp}
The \emph{fundamental rogue wave of order $k$} is then defined by the limit 
\begin{equation}
\psi(x,t)=\psi_k(x,t)\defeq 2\ii\lim_{\lambda\to\infty}\lambda M_{12}^{(k)}(\lambda;x,t), \quad k\in\mathbb{Z}_{\ge 0},
\label{eq:psi-from-M}
\end{equation}
and it is a rational solution of the focusing nonlinear Schr\"odinger equation in the form 
\begin{equation}
\ii\psi_t +\tfrac{1}{2}\psi_{xx}+(|\psi|^2-1)\psi=0,
\label{eq:NLS}
\end{equation}
that tends to the background solution $\psi=\psi_0(x,t)\equiv 1$ as $(x,t)\to\infty$ in $\mathbb{R}^2$.  It is this feature of simultaneous spatio-temporal localization that explains the terminology of \emph{rogue waves} for such solutions.  

As the parameter $k$ increases, the fundamental rogue wave has increasing amplitude (see \cite[Proposition 2]{BilmanLM20} and also \cite{AAS09, WangYWH17}).  This large maximum amplitude is achieved exactly at the origin $(x,t)=(0,0)$, and the aim of the previous paper \cite{BilmanLM20} was to study the fundamental rogue wave of order $k$ in a small neighborhood of this amplitude peak.  It was discovered in \cite{BilmanLM20} that for fixed $s=(-1)^k$, $sn^{-1}\psi_k(n^{-1}X,n^{-2}T)$ converges as $n\to+\infty$ to a limiting function $\Psi(X,T)$, the \emph{rogue wave of infinite order}, that solves the focusing nonlinear Schr\"odinger equation in the form $\ii\Psi_T+\tfrac{1}{2}\Psi_{XX}+|\Psi|^2\Psi=0$.  This limiting function is a highly-transcendental solution having a number of remarkable properties described in \cite{BilmanLM20}, for instance:  (i) it satisfies also ordinary differential equations of Painlev\'e type in the two independent variables, (ii) it has its own Riemann-Hilbert representation, and (iii) $\Psi(X,T)\to 0$ for large $X$ and $T$ (even though $\psi_k\to 1$ for large $x$ and $t$).  The decay for large $X$ is sufficient for the function $\Psi(\cdot,T)$ to lie in $L^2(\mathbb{R})$ for every $T\in\mathbb{R}$, but $\Psi(\cdot,T)\not\in L^1(\mathbb{R})$, and the decay in $T$ is even slower.    The function $\Psi(X,T)$ has recently also been shown to be important in several other problems; for the same equation it describes also high-order multiple-pole soliton solutions \cite{BilmanB19} and self-similar focusing in the setting of weak dispersion \cite{Suleimanov17,BuckinghamJM21}, and for the sharp-line Maxwell-Bloch system in characteristic coordinates it models initial/boundary layers \cite{LiM21}.

\subsection{Reformulated characterization of fundamental rogue waves}
The purpose of this paper is to describe the fundamental rogue-wave solution of high order $k$ in a different regime for the independent variables on which both $x$ and $t$ are instead proportional to $k$.  To this end, in
place of $\mathbf{M}^{(k)}(\lambda;x,t)$, consider the matrix $\mathbf{P}^{(k)}(\lambda;x,t)$ defined by
\begin{multline}
\mathbf{P}^{(k)}(\lambda;x,t)\defeq
\ee^{\frac{1}{2}\ii t\sigma_3}\mathbf{M}^{(k)}(\lambda;x,t)\\{}\cdot
\begin{cases}
\ee^{-\ii\rho(\lambda)(x+\lambda t)\sigma_3}\mathbf{Q}^{s}\ee^{\ii(\lambda x+\lambda^2 t)\sigma_3},&\quad
\text{$\lambda$ inside $\Sigma_\circ$},\\ \displaystyle 
\ee^{\ii[\lambda x+\lambda^2t-\rho(\lambda)(x+\lambda t)]\sigma_3}
%\left(\frac{\lambda-\ii}{\lambda+\ii}\right)^{-n\sigma_3}
B(\lambda)^{-n\sigma_3}
\omega(\lambda)^{-s\sigma_3},&\quad 
\text{$\lambda$ exterior to $\Sigma_\circ$},
\end{cases}
\label{eq:M-P-bulk}
\end{multline}
where we recall that $s=\pm 1$ is the parity index of $k$, where $n$ is defined by \eqref{eq:k-vs-n}, 
and where
\begin{equation}
\omega(\lambda)\defeq f(\lambda)(1+\ii (\lambda-\rho(\lambda))).
\label{eq:omega-def}
\end{equation}
%\begin{remark}
An alternate formula for $\omega(\lambda)$ %defined in \eqref{eq:omega-def} 
can be found as follows.  First we observe that $\omega(\lambda)$ is analytic for $\lambda\in\mathbb{C}\setminus\Sigma_\mathrm{c}$ and satisfies $\omega(\lambda)\to 1$ as $\lambda\to\infty$.  Using \eqref{eq:f-squared} and $\rho(\lambda)^2=\lambda^2+1$, we easily calculate that
\begin{equation}
\omega(\lambda)^4 = B(\lambda).
\label{eq:omega-fourth-power}
\end{equation}
In particular, it follows from this that, recalling the upward orientation of $\Sigma_\mathrm{c}$,
\begin{equation}
\omega_+(\lambda)=\ii\omega_-(\lambda),\quad\lambda\in\Sigma_\mathrm{c}.
\label{eq:omega-jump}
\end{equation}
%Also, it will be convenient to allow the branch cut $\Sigma_\mathrm{c}$ to be taken as a more general Schwarz-symmetric simple arc joining the endpoints $\lambda=\pm\ii$.
%Also, $B_k(\lambda)$ in \eqref{eq:P-bulk-jump-rewrite} can be written as $\omega(\lambda)^{2k+1}$.
%\label{rem:omega}
%\end{remark}

It is easy to check that $\mathbf{P}^{(k)}(\lambda;x,t)$ is an analytic function of $\lambda$ for $\lambda\in\mathbb{C}\setminus\Sigma_\circ$, i.e., the jump of $\mathbf{M}^{(k)}(\lambda;x,t)$ across the cut $\Sigma_\mathrm{c}$ between $\pm\ii$ is removed by the substitution, and no additional singularities are introduced.  Since $\lambda x+\lambda^2t-\rho(\lambda)(x+\lambda t) = -\tfrac{1}{2}t+O(\lambda^{-1})$ as $\lambda\to\infty$, it is easy to check that $\mathbf{P}^{(k)}(\lambda;x,t)\to\mathbb{I}$ in the same limit.  One directly calculates that the jump condition satisfied by $\mathbf{P}^{(k)}(\lambda;x,t)$ across the closed curve $\Sigma_\circ$ with clockwise orientation is then
\begin{multline}
\mathbf{P}^{(k)}_+(\lambda;x,t)=\\
\mathbf{P}^{(k)}_-(\lambda;x,t)\ee^{-\ii(\lambda x+\lambda^2t)\sigma_3}\mathbf{Q}^{-s}\mathbf{Q}
%\left(\frac{\lambda-\ii}{\lambda+\ii}\right)^{s n\sigma_3}
B(\lambda)^{sn\sigma_3}
\mathbf{Q}^{-1}\mathbf{E}(\lambda)\omega(\lambda)^{-s\sigma_3}
%\left(\frac{\lambda-\ii}{\lambda+\ii}\right)^{-n\sigma_3}
B(\lambda)^{-n\sigma_3}
\ee^{\ii(\lambda x+\lambda^2t)\sigma_3},\\
\lambda\in\Sigma_\circ.
\end{multline}
But the eigenvalues of $\mathbf{E}(\lambda)$ are precisely $\omega(\lambda)^{\pm 1}$ and $\mathbf{E}(\lambda)$ is diagonalized by the constant orthogonal eigenvector matrix $\mathbf{Q}$, so $\mathbf{Q}^{-1}\mathbf{E}(\lambda)=\omega(\lambda)^{\sigma_3}\mathbf{Q}^{-1}$.  Using this identity as well as $\mathbf{Q}^2=-\ii\sigma_2$ along with \eqref{eq:k-vs-n} and \eqref{eq:omega-fourth-power}, we see that $\mathbf{P}(\lambda;x,t,\mathbf{G},M)=\mathbf{P}^{(k)}(\lambda;x,t)$ solves the following Riemann-Hilbert problem with matrix $\mathbf{G}$ and positive parameter $M$ determined from $k\in\mathbb{Z}_{\ge 0}$ by
\begin{equation}
\mathbf{G}\defeq \mathbf{Q}^{-s}\quad\text{and}\quad  M\defeq n+\tfrac{1}{4}s=\tfrac{1}{2}k+\tfrac{1}{4}.
\label{eq:G-and-M-RogueWaves}
\end{equation}
\begin{rhp}[Reformulated problem for rogue waves]
Let $(x,t)\in\mathbb{R}^2$ and $M\in\mathbb{R}$ be arbitrary parameters, and let $\mathbf{G}$ be a $2\times 2$ matrix satisfying $\det(\mathbf{G})=1$ and $\mathbf{G}^*=\sigma_2\mathbf{G}\sigma_2$.  Find a $2\times 2$ matrix $\mathbf{P}(\lambda)=\mathbf{P}(\lambda;x,t,\mathbf{G},M)$ with the following properties:
\begin{itemize}
\item[]\textbf{Analyticity:}  $\mathbf{P}(\lambda)$ is analytic in $\lambda$ for $\lambda\in\mathbb{C}\setminus\Sigma_\circ$, and it takes continuous boundary values on $\Sigma_\circ$.
\item[]\textbf{Jump conditions:}  The boundary values on the jump contour $\Sigma_\circ$ are related as follows:
%\begin{multline}
\begin{equation}
\mathbf{P}_+(\lambda)=%\\
\mathbf{P}_-(\lambda)
\ee^{-\ii (\lambda x+\lambda^2t)\sigma_3}
%\left(\frac{\lambda-\ii}{\lambda+\ii}\right)^{n\sigma_3}\omega(\lambda)^{s\sigma_3}
B(\lambda)^{M\sigma_3}
%\omega(\lambda)^{N\sigma_3}
%\mathbf{Q}^{-s}
\mathbf{G}
%\omega(\lambda)^{-N\sigma_3}
B(\lambda)^{-M\sigma_3}
%\omega(\lambda)^{-s\sigma_3}\left(\frac{\lambda-\ii}{\lambda+\ii}\right)^{-n\sigma_3}
\ee^{\ii(\lambda x+\lambda^2t)\sigma_3},\quad %\\
\lambda\in\Sigma_\circ,
\label{eq:P-bulk-jump}
\end{equation}
%\end{multline}
where scalar powers of the Blaschke factor $B(\lambda)$ are analytic for $\lambda\in\mathbb{C}\setminus\Sigma_\mathrm{c}$ and tend to $1$ as $\lambda\to\infty$.
%where $s=\pm 1$ is the parity index of $k$ and where $N\defeq 4n+s=2k+1$ (see \eqref{eq:k-vs-n}).
\item[]\textbf{Normalization:}  $\mathbf{P}(\lambda)\to\mathbb{I}$ as $\lambda\to\infty$. 
\end{itemize}
\label{rhp:rogue-wave-reformulation}
\end{rhp}
It follows from \eqref{eq:psi-from-M} and the substitution \eqref{eq:M-P-bulk} that $\psi_k(x,t)$ can be recovered from $\mathbf{P}^{(k)}(\lambda;x,t)$ by the formula
\begin{equation}
\psi_k(x,t)=2\ii\ee^{-\ii t}\lim_{\lambda\to\infty} \lambda P^{(k)}_{12}(\lambda;x,t).
\end{equation}
Note that to prove Theorem~\ref{thm:exterior} below, it will be useful to work with a limiting case for the Jordan curve $\Sigma_\circ$ in which it is squeezed into a dumbbell shape; on the ``neck'' of the dumbbell there is then a different form of the jump condition.  See Section~\ref{sec:dumbbell}.

\subsection{Continuous interpolation between rogue waves and multiple-pole solitons of arbitrary orders}
\label{sec:M-arbitrary}
Even though it is only related to fundamental rogue waves when the parameters $\mathbf{G}$ and $M$ are related to the order $k\in\mathbb{Z}_{\ge 0}$ by \eqref{eq:G-and-M-RogueWaves}, more generally it follows from the vanishing lemma \cite{Zhou89} that Riemann-Hilbert Problem~\ref{rhp:rogue-wave-reformulation} is uniquely solvable globally in $(x,t)\in\mathbb{R}^2$ for any $M\in\mathbb{R}$ and matrix $\mathbf{G}$ with $\det(\mathbf{G})=1$ and $\mathbf{G}=\sigma_2\mathbf{G}^*\sigma_2$.  From the dressing method it then follows that the function 
\begin{equation}
q=q(x,t;\mathbf{G},M)\defeq 2\ii\lim_{\lambda\to\infty}\lambda P_{12}(\lambda;x,t,\mathbf{G},M)
\label{eq:q-define}
\end{equation}
is a well-defined solution of the focusing nonlinear Schr\"odinger equation in the form
\begin{equation}
\ii q_t+\tfrac{1}{2}q_{xx}+|q|^2q=0.
\label{eq:NLS-ZBC}
\end{equation}
This implies, in particular, that $q(x,t;\mathbf{G},M)$ provides a continuous interpolation via solutions of \eqref{eq:NLS-ZBC} of fundamental rogue waves of different (integral) orders.  The intermediate interpolating solutions can be of independent interest.
For instance, noting that a general matrix $\mathbf{G}$ satisfying $\det(\mathbf{G})=1$ and $\mathbf{G}=\sigma_2\mathbf{G}^*\sigma_2$ can be written in the form
\begin{equation}
\mathbf{G}=\frac{1}{\sqrt{|a|^2+|b|^2}}\begin{bmatrix}a & b^*\\-b & a^*\end{bmatrix},\quad a,b\in\mathbb{C},
\label{eq:G-form}
\end{equation}
comparing with \cite{BilmanBW19} one sees that if $M\in\mathbb{Z}_{>0}$, then $q(x,t;\mathbf{G},M)$ is a multiple-pole soliton solution of \eqref{eq:NLS-ZBC} of order $2M$, which satisfies quite different boundary conditions than do rogue waves.  In fact, it is easy to see directly that the jump matrix in Riemann-Hilbert Problem~\ref{rhp:rogue-wave-reformulation} is single-valued meromorphic if only $M\in\tfrac{1}{2}\mathbb{Z}$, with poles of order $2|M|$ at $\lambda=\pm\ii$.  This immediately allows the problem to be reduced to the solution of a finite-dimensional linear system for all such $M$, and hence $q(x,t;\mathbf{G},\tfrac{1}{2}k)$ is a $k^\mathrm{th}$ order pole soliton solution for $k\in\mathbb{Z}_{\ge 0}$.  
In this way, we see that as $M>0$ continuously increases, $q(x,t;\mathbf{G},M)$ remains a solution of the same equation \eqref{eq:NLS-ZBC} that satisfies zero boundary conditions for $M\in\tfrac{1}{2}\mathbb{Z}_{\ge 0}$ and satisfies constant-amplitude nonzero boundary conditions for $M\in\tfrac{1}{2}\mathbb{Z}_{\ge 0}+\tfrac{1}{4}$.  This proves the following.
\begin{theorem}
Let $\mathbf{G}$ be a $2\times 2$ constant matrix with $\det(\mathbf{G})=1$ and $\mathbf{G}=\sigma_2\mathbf{G}^*\sigma_2$, and let $M>0$ be arbitrary.  Then the function $q(x,t;\mathbf{G},M)$ given in terms of the well-defined solution of Riemann-Hilbert Problem~\ref{rhp:rogue-wave-reformulation} by \eqref{eq:q-define} is a global solution of the focusing nonlinear Schr\"odinger equation in the form \eqref{eq:NLS-ZBC} that is a rogue wave of order $k\in\mathbb{Z}_{\ge 0}$ whenever $M=\tfrac{1}{2}k+\tfrac{1}{4}$ and that is a multiple-pole soliton solution of order $k$ for $k\in\mathbb{Z}_{\ge 0}$ whenever $M=\tfrac{1}{2}k$.
\label{thm:solution-family}
\end{theorem}

This strikes us as a remarkable result.  For instance, it asserts that in a precise sense the famous Peregrine solution $\psi_1(x,t)$ can be regarded as a soliton of order $\tfrac{3}{2}$, because $M=\tfrac{3}{4}$ (Peregrine) lies halfway between $M=\tfrac{1}{2}$ (stationary simple-pole soliton for zero boundary conditions) and $M=1$ (stationary double-pole soliton for zero boundary conditions).  
For values of $M\ge 0$ corresponding to neither solitons ($M\in\tfrac{1}{2}\mathbb{Z}_{\ge 0}$) nor rogue waves ($M\in\tfrac{1}{2}\mathbb{Z}_{\ge 0}+\tfrac{1}{4}$), $q(x,t;\mathbf{G},M)$ satisfies the same nonzero boundary conditions as $|x|\to\infty$ as in the rogue-wave case, \emph{except} that the decay to the background is so slow that the difference is not even in $L^2(\mathbb{R})$; by contrast it is well-known that for rogue waves the difference is in $L^1(\mathbb{R})$.  We will give the proof of this slow decay in a subsequent paper devoted to the study of the solutions for general $M\ge 0$.

Despite the fact that the boundary conditions are quite different, because the solitons and rogue waves have now been placed within the same family of solutions, they have certain properties in common.  From \cite{BilmanB19,BilmanLM20} it is known that both types of solutions exhibit the same asymptotic behavior in the large-$M$ limit near the peak amplitude point.  Choosing $\Sigma_\circ$ in Riemann-Hilbert Problem~\ref{rhp:rogue-wave-reformulation} to be a circle of radius $M$ and scaling $(x,t)$ by $x=M^{-1}X$ and $t=M^{-2}T$ produces a limiting jump condition in the $\Lambda=M^{-1}\lambda$ plane that shows immediately that the same limiting behavior near the peak in terms of the rogue wave of infinite order is also valid in the limit $M\to\infty$ along any sequence, so the ``near field'' behavior is universal with respect to $M$.  We will show in this paper that this common asymptotic behavior for the whole solution family extends to a large region of the $(x,t)$-plane, expanding in size as $M\to+\infty$ at a rate proportional to $M$.  Within this region,
the large-$M$ asymptotic behavior of $q(x,t;\mathbf{G},M)$ is rather insensitive to any particular choice of specific unbounded and increasing sequence $\{M_k\}_{k=1}^\infty$.  On the other hand, in the complementary region one sees qualitatively different asymptotic behavior along different sequences.  See Figure~\ref{fig:roguewaves-and-solitons}.

\begin{figure}[h]
\begin{center}
\phantom{!}\hfill\includegraphics[width=0.4\linewidth]{rogue-wave-k8.pdf}\hfill%
\includegraphics[width=0.4\linewidth]{multi-pole-soliton-k8.pdf}\hfill\phantom{!}\\
\phantom{!}\hfill\includegraphics[width=0.4\linewidth]{rogue-wave-k8-closeup-overlay.pdf}\hfill%
\includegraphics[width=0.4\linewidth]{multi-pole-soliton-k8-closeup-overlay.pdf}\hfill\phantom{!}
\end{center}
\caption{Top row:  amplitude density plots of the fundamental rogue wave of order $k=8$ (left) and a multiple-pole soliton of order $k=8$ (right).  Bottom row:  as in the top row, but closeup plots showing the region (bounded by yellow curves with red vertices) on which we prove common asymptotic behavior as $k\to+\infty$ for both types of solutions (see Theorem~\ref{thm:channels} and Theorem~\ref{thm:shelves} below).}
\label{fig:roguewaves-and-solitons}
\end{figure}

\subsection{Symmetry assumptions}
The function $q(x,t;\mathbf{G},M)$ is obviously unaffected by any transformation of $\mathbf{P}(\lambda;x,t,\mathbf{G},M)$ within the interior of $\Sigma_\circ$; furthermore, it is easy to see that the form of the jump condition and the symmetry property $\mathbf{G}^*=\sigma_2\mathbf{G}\sigma_2$ are both preserved if the latter transformation is taken to be right-multiplication by $w^{\sigma_3}$ where $w$ is any constant with $|w|=1$.  Thus one sees easily that there is no loss of generality in assuming $a>0$ in the form \eqref{eq:G-form}.  Under this assumption, there are only two matrices $\mathbf{G}$ that build in additional useful symmetries, namely $\mathbf{G}=\mathbf{Q}$ and $\mathbf{G}=\mathbf{Q}^{-1}$.
\begin{proposition}
For all $M>0$ and arbitrary sign $s=\pm 1$,
\begin{equation}
q(-x,t;\mathbf{Q}^{-s},M)=q(x,t;\mathbf{Q}^{-s},M)\quad\text{and}\quad
q(x,-t;\mathbf{Q}^{-s},M)=q(x,t;\mathbf{Q}^{-s},M)^*.
\label{eq:q-symmetries}  
\end{equation}
\label{prop:symmetry}
\end{proposition}
The proof is an elementary application of the representation of $q(x,t;\mathbf{G},M)$ via Riemann-Hilbert Problem~\ref{rhp:rogue-wave-reformulation} and can be found in Appendix~\ref{A:Proofs}.
The specific choice of $\mathbf{G}=\mathbf{Q}^{-s}$ with $s=\pm 1$ in Riemann-Hilbert Problem~\ref{rhp:rogue-wave-reformulation} makes the rogue wave (for $M=\tfrac{1}{2}k+\tfrac{1}{4}$ and $k\in\mathbb{Z}_{\ge 0}$ with $s=(-1)^k$) or soliton (for $M=\tfrac{1}{2}k$ with $k\in\mathbb{Z}_{\ge 0}$ and $s=\pm 1$ arbitrary) ``fundamental''.  For rogue waves the correlation of the sign $s$ with the order $k$ is important\footnote{The alternation of sign in the exponent of $\mathbf{Q}^{-s}$ is necessary to achieve the correct boundary condition $\psi_k(x,t)\to 1$ as $(x,t)\to\infty$.  Using $\mathbf{Q}^{-1}=\ii^{\sigma_3}\mathbf{Q}\ii^{-\sigma_3}$ it is easy to see that exchanging $\mathbf{Q}$ for $\mathbf{Q}^{-1}$ at fixed $M=\tfrac{1}{2}k+\tfrac{1}{4}$ corresponds to the transformation $\mathbf{P}\mapsto \ii^{\sigma_3}\mathbf{P}\ii^{-\sigma_3}$ which implies via \eqref{eq:q-define} that $q\mapsto -q$ and hence yields a rogue wave solution satisfying $\psi_k(x,t)\to -1$ as $(x,t)\to\infty$.} to fix the boundary conditions. 

This result allows us to assume, as we do for the rest of this paper, that $x\ge 0$ and $t\ge 0$.  
%Although we will formulate some of our results for $M>0$ tending to $\infty$ in an arbitrary manner, for simplicity we will restrict our attention to ``core'' jump matrices $\mathbf{G}=\mathbf{Q}^{-s}$, where $s=\pm 1$ in general while for rogue waves of order $k$ we also insist that $s=(-1)^k$.  

\subsection{The far-field regime}
A more important reason for characterizing rogue waves and solitons via Riemann-Hilbert Problem~\ref{rhp:rogue-wave-reformulation} is that its jump condition is well-suited for steepest-descent asymptotic analysis in the large $M>0$ regime where $x$ and $t$ are proportional to $M$.  Indeed, introducing rescaled variables by setting
\begin{equation}
\chi\defeq \frac{x}{M}\quad \text{and} \quad \tau\defeq \frac{t}{M},
\end{equation}
and then defining 
\begin{equation}
\vartheta(\lambda;\chi,\tau)\defeq\chi\lambda + \tau\lambda^2 +\ii\log\left(B(\lambda)\right),
\label{eq:vartheta}
\end{equation}
in which the logarithm is taken to be the principal branch (i.e., $\log(B(\lambda))$ is analytic for $\lambda\in\mathbb{C}\setminus\Sigma_\mathrm{c}$ and $\log(B(\lambda))\to 0$ as $\lambda\to\infty$), we set
\begin{equation}
\mathbf{S}(\lambda;\chi,\tau,\mathbf{G},M)\defeq\mathbf{P}(\lambda;M\chi,M\tau,\mathbf{G},M).
\label{eq:S-from-P}
\end{equation}
Then the jump condition for $\mathbf{S}(\lambda;\chi,\tau,\mathbf{G},M)$ on the jump contour $\Sigma_\circ$ reads
\begin{equation}
\mathbf{S}_+(\lambda;\chi,\tau,\mathbf{G},M)=\mathbf{S}_-(\lambda;\chi,\tau,\mathbf{G},M)\ee^{-\ii M\vartheta(\lambda;\chi,\tau)\sigma_3}\mathbf{G}\ee^{\ii M\vartheta(\lambda;\chi,\tau)\sigma_3},\quad\lambda\in\Sigma_\circ.
\label{eq:S-jump}
\end{equation}
Thus the large parameter $M\gg 1$ enters only via an exponential conjugation.  In general, a solution $q(x,t;\mathbf{G},M)$ of \eqref{eq:NLS-ZBC} is obtained from $\mathbf{S}(\lambda;\chi,\tau,\mathbf{G},M)$ via
\begin{equation}
q(M\chi,M\tau;\mathbf{G},M)=2\ii\lim_{\lambda\to\infty} \lambda S_{12}(\lambda;\chi,\tau,\mathbf{G},M).
\label{eq:q-S}
\end{equation}
To obtain the fundamental rogue wave of order $k$ we tie $\mathbf{G}$ and $M$ to $k$ via \eqref{eq:G-and-M-RogueWaves} and include an additional exponential factor:
\begin{equation}
\psi_k(M\chi,M\tau)=2\ii\ee^{-\ii M\tau}\lim_{\lambda\to\infty}\lambda S_{12}(\lambda;\chi,\tau,\mathbf{Q}^{-s},M),\quad s=(-1)^k,\quad M=\tfrac{1}{2}k+\tfrac{1}{4}.
\label{eq:psi-k-S}
\end{equation}
The regime in which the independent variables $(x,t)$ are proportional to the order $k$ (or more generally, to the parameter $M$) when the latter is large is called the \emph{far-field regime}.  The near-field regime where $x$ and $t$ are small when $k$ or $M$ is large was studied for high-order multiple-pole solitons in \cite{BilmanB19} and for fundamental rogue waves in \cite{BilmanLM20}.  It is important to observe that the near-field and far-field regimes \emph{do not} actually overlap.  There is, however, no expectation of any new phenomena occurring in the intermediate region; the near-field and far-field asymptotic formul\ae\ extend consistently to an expected overlap domain, but the conclusion of common validity over such a domain does not follow from the proofs we will give below.

\subsection{The basic exponent function $\vartheta(\lambda;\chi,\tau)$ and the domain $\channels$}
\label{sec:basic-exponent}
The exponent function $\vartheta(\lambda;\chi,\tau)$ has been studied before in the context of high-order multiple-pole soliton solutions of the focusing nonlinear Schr\"odinger equation \cite{BilmanBW19}; in the notation of that reference, we have $\varphi(\lambda;\chi,\tau,\ii)=\ii\vartheta(\lambda;\chi,\tau)$.  In particular, it is known that $\vartheta(\lambda;\chi,\tau)$ has simple critical points except when $(\chi,\tau)\in\mathbb{R}_{\ge 0}\times\mathbb{R}_{\ge 0}$ are related by the equation
\begin{equation}
16\tau^4 + (8\chi^2-72\chi+108)\tau^2 +\chi^4-2\chi^3=0.
\label{eq:boundary-curve}
\end{equation}
Clearly we can only have $\tau=0$ for $\chi\ge 0$ if $\chi=0$ or $\chi=2$.  
Solving for $\tau^2$ gives
\begin{equation}
\tau^2 
%= \frac{1}{32}\left[-8\chi^2+72\chi-108\pm 8(9-4\chi)^\frac{3}{2}\right].
=\tfrac{1}{8}\left[-2\chi^2+18\chi-27\pm 2(9-4\chi)^\frac{3}{2}\right].
\label{eq:tau-squared}
\end{equation}
Reality of $\tau^2$ for $\chi\ge 0$ requires $0\le\chi\le\tfrac{9}{4}$.  If $2\le\chi\le\tfrac{9}{4}$, then both solutions for $\tau^2$ are non-negative.  The values of $\tau^2$ coincide only at the upper endpoint $\chi=\tfrac{9}{4}$ with common value $\tau^2=\tfrac{27}{64}$, and at the lower endpoint $\chi=2$ the smaller value of $\tau^2$ changes sign.  On the interval $0\le\chi< 2$, only the branch of $\tau^2$ with the ``$+$'' sign in \eqref{eq:tau-squared} is nonnegative (and strictly positive except at the lower endpoint $\chi=0$).  Counting with multiplicity, $\vartheta(\lambda;\chi,\tau)$ has three critical points for $\tau\neq 0$, two critical points for $\tau=0$ and $\chi>0$, and no critical points for $\tau=\chi=0$.  The critical points $\lambda$ satisfy the cubic equation
\begin{equation}
2\tau\lambda^3+\chi\lambda^2 + 2\tau\lambda +\chi-2=0,
\end{equation}
and having real coefficients the roots are in general either all real or form a conjugate pair and an isolated real root.  However, in the special case that $\tau=0$ and $0\le\chi\le 2$, there are only two roots, and the critical points are exactly the opposite real numbers
\begin{equation}
\lambda = \pm\sqrt{\frac{2}{\chi}-1},\quad 0\le\chi\le 2,\quad \tau=0.
\label{eq:tau-zero-critical-points}
\end{equation}
It follows that the graphs of the positive square roots of the positive branches of \eqref{eq:tau-squared} border a bounded and relatively open subset $\channels$ of the quadrant $(\chi,\tau)\in\mathbb{R}_{\ge 0}\times\mathbb{R}_{\ge 0}$ such that $(\chi,\tau)\in \channels$ implies that all critical points of $\vartheta(\lambda;\chi,\tau)$ are real and distinct.  In \cite{BilmanBW19} $\channels$ is called the ``algebraic-decay region''.  The same graphs border on the exterior an unbounded and relatively open subset of $\mathbb{R}_{\ge 0}\times\mathbb{R}_{\ge 0}$ on which $\vartheta(\lambda;\chi,\tau)$ has a conjugate pair of critical points with nonzero imaginary part.  The boundary of $\channels$ (shown with a red curve in Figure~\ref{fig:RegionsPlot} below) defined by the relation \eqref{eq:boundary-curve} or \eqref{eq:tau-squared} is smooth except for one point $(\chi^\sharp,\tau^\sharp)$ with coordinates
\begin{equation}
(\chi^\sharp,\tau^\sharp)\defeq\left(\tfrac{9}{4},\tfrac{3\sqrt{3}}{8}\right).
\label{eq:corner-point}
\end{equation}

Although it plays no role in the analysis of high-order fundamental rogue waves, on the exterior of $\channels$ there is a distinguished curve emanating from $(\chi^\sharp,\tau^\sharp)$ that we denote by $\ell_\mathrm{sol}$ along which the level set $\mathrm{Re}(\ii\vartheta(\lambda;\chi,\tau))=0$ is connected. This curve is determined by the condition 
\begin{equation}
\ell_\mathrm{sol}:  \mathrm{Re}\left(\int_\Gamma\ii\vartheta'(\lambda;\chi,\tau)\,\dd\lambda\right)=0,
\label{eq:DegenerateBoutroux}
\end{equation}
where $\Gamma$ is any Schwarz-symmetric contour avoiding $\lambda=\pm\ii$ and having endpoints equal to the complex-conjugate critical points of $\vartheta(\lambda;\chi,\tau)$.  The curve $\ell_\mathrm{sol}$ is shown with a black dotted line in Figure~\ref{fig:RegionsPlot} below; it is important in the asymptotic description of $q(x,t;\mathbf{G},M)$ for large $M\not\in \tfrac{1}{2}\mathbb{Z}_{\ge 0}+\tfrac{1}{4}$.

\subsection{Genus-zero modification of $\vartheta(\lambda;\chi,\tau)$ and the regions $\shelves$ and $\exterior$}
\label{sec:h-intro}
When $(\chi,\tau)\in (\mathbb{R}_{\ge 0}\times\mathbb{R}_{\ge 0})\setminus\overline{\channels}$ it will be necessary to modify the phase $\vartheta(\lambda;\chi,\tau)$ with a genus-zero $g$-function.  Let $\Sigma_g$ be a Schwarz-symmetric sub-arc of 
the jump contour for $\mathbf{S}(\lambda;\chi,\tau,\mathbf{G},M)$ with complex-conjugate endpoints $\lambda_0(\chi,\tau)=A(\chi,\tau)+\ii B(\chi,\tau)$ and $\lambda_0(\chi,\tau)^*=A(\chi,\tau)-\ii B(\chi,\tau)$, and let $g(\lambda;\chi,\tau)$ be bounded and analytic for $\lambda\in\mathbb{C}\setminus\Sigma_g$ with $g(\lambda;\chi,\tau)\to 0$ as $\lambda\to\infty$.  
Consider the matrix $\mathbf{T}(\lambda;\chi,\tau,\mathbf{G},M)$ defined in terms of $g$ and $\mathbf{S}(\lambda;\chi,\tau,\mathbf{G},M)$
by the formula
%\begin{equation}
%\mathbf{T}^{(k)}(\lambda;\chi,\tau)\defeq\mathbf{S}^{(k)}(\lambda;\chi,\tau)\ee^{ng(\lambda;\chi,\tau)\sigma_3}.
%\label{eq:T-to-S}
%\end{equation}
%Then using \eqref{eq:S-jump}, we see that
%\begin{equation}
%\mathbf{T}^{(k)}_+(\lambda;\chi,\tau)=\mathbf{T}^{(k)}_-(\lambda;\chi,\tau)\ee^{-n(g_-(\lambda;\chi,\tau)+\ii\vartheta(\lambda;\chi,\tau))\sigma_3}\omega(\lambda)^{s\sigma_3}\mathbf{Q}^{-s}\omega(\lambda)^{-s\sigma_3}
%\ee^{n(g_+(\lambda;\chi,\tau)+\ii\vartheta(\lambda;\chi,\tau))\sigma_3},\;\;\lambda\in\Sigma_\circ.
%\label{eq:T-jump}
%\end{equation}
%\textcolor{red}{For the alternate scaling of $x$ and $t$, for $M=N/4$ we would set
\begin{equation}
\mathbf{T}(\lambda;\chi,\tau,\mathbf{G},M)\defeq\mathbf{S}(\lambda;\chi,\tau,\mathbf{G},M)\ee^{\ii Mg(\lambda;\chi,\tau)\sigma_3}.
\label{eq:T-to-S}
\end{equation}
Then using \eqref{eq:S-jump}, we see that on the jump contour we have
\begin{equation}
\mathbf{T}_+(\lambda;\chi,\tau,\mathbf{G},M)=\mathbf{T}_-(\lambda;\chi,\tau,\mathbf{G},M)\ee^{-\ii Mh_-(\lambda;\chi,\tau)\sigma_3}%\mathbf{Q}^{-s}
\mathbf{G}
\ee^{\ii Mh_+(\lambda;\chi,\tau)\sigma_3},
\label{eq:T-jump}
\end{equation}
where
\begin{equation}
h(\lambda;\chi,\tau)\defeq \vartheta(\lambda;\chi,\tau)+g(\lambda;\chi,\tau),\quad\lambda\in\mathbb{C}\setminus(\Sigma_g\cup\Sigma_\mathrm{c})
\label{eq:h-define}
\end{equation}
%}
is the modification of $\vartheta(\lambda;\chi,\tau)$ referred to in the section title.  
%Of course $g_+=g_-=g$ when $\lambda\in\Sigma_\circ\setminus\Sigma_g$.  Setting $h(\lambda;\chi,\tau)\defeq\ii\vartheta(\lambda;\chi,\tau)+g(\lambda;\chi,\tau)$ yields  
We impose the additional condition that the sum of boundary values $h_+(\lambda;\chi,\tau)+h_-(\lambda;\chi,\tau)$ is independent of $\lambda\in\Sigma_g$, which simplifies the jump condition \eqref{eq:T-jump} for $\lambda\in\Sigma_g\subset\Sigma_\circ$.  Thus, $h'(\lambda;\chi,\tau)$ is analytic for $\lambda\in\mathbb{C}\setminus(\Sigma_g\cup\{\ii,-\ii\})$, satisfies $h_+'(\lambda;\chi,\tau)+h_-'(\lambda;\chi,\tau)=0$ on $\Sigma_g$, has simple poles inherited from $\vartheta'(\lambda;\chi,\tau)$ at $\lambda=\pm\ii$ with
\begin{equation}
\mathop{\mathrm{Res}}_{\lambda=\pm \ii}h'(\lambda;\chi,\tau)=\pm \ii,
\label{eq:hprime-residues}
\end{equation}
and has the large-$\lambda$ expansion
\begin{equation}
h'(\lambda;\chi,\tau)=2\tau\lambda+\chi + O(\lambda^{-2}),\quad\lambda\to\infty.
\label{eq:hprime-expansion}
\end{equation}
Letting $R(\lambda;\chi,\tau)$ be the analytic function for $\lambda\in\mathbb{C}\setminus\Sigma_g$ satisfying 
\begin{equation}
\begin{split}
R(\lambda;\chi,\tau)^2&=(\lambda-\lambda_0(\chi,\tau))(\lambda-\lambda_0(\chi,\tau)^*)\\
&=(\lambda-A(\chi,\tau))^2+B(\chi,\tau)^2,\quad\text{and $R(\lambda;\chi,\tau)=\lambda + O(1)$ as $\lambda\to\infty$}, 
\end{split}
\label{eq:R-define}
\end{equation}
it follows that $h'(\lambda;\chi,\tau)$ necessarily has the form
\begin{equation}
h'(\lambda;\chi,\tau)=\frac{2\tau\lambda^2+u(\chi,\tau)\lambda+v(\chi,\tau)}{\lambda^2+1}R(\lambda;\chi,\tau),
\label{eq:hprime-formula}
\end{equation}
where $A(\chi,\tau)\in\mathbb{R}$, $B(\chi,\tau)^2>0$, $u(\chi,\tau)\in\mathbb{R}$, and $v(\chi,\tau)\in\mathbb{R}$ are to be determined (uniquely, see Section~\ref{sec:g-function}) so that $h'(\lambda;\chi,\tau)$ has the desired residues \eqref{eq:hprime-residues} and large-$\lambda$ expansion \eqref{eq:hprime-expansion}.  This determination also places conditions on the location of the branch cut $\Sigma_g$ relative to the points $\lambda=\pm\ii$; see Remark~\ref{rem:Sigma_g}.

It turns out that the boundary curve \eqref{eq:boundary-curve} reappears in the analysis of the modified phase function $h(\lambda;\chi,\tau)$ as the condition that $B(\chi,\tau)^2=0$.  
In other words,
the roots of $R(\lambda;\chi,\tau)^2$ form a well-defined conjugate pair for all $(\chi,\tau)$ in the part of the first quadrant complementary to the domain $\channels$ on which the unmodified phase $\vartheta(\lambda;\chi,\tau)$ has three real critical points, and both $A(\chi,\tau)$ and $B(\chi,\tau)>0$ are real analytic functions of $(\chi,\tau)\in (\mathbb{R}_{\ge 0}\times\mathbb{R}_{\ge 0})\setminus\overline{\channels}$.  It is easy to show from the construction of $h'(\lambda;\chi,\tau)$ in Section~\ref{sec:g-function} that if one introduces polar coordinates via $\chi=r\cos(\theta)$ and $\tau=r\sin(\theta)$, then $A(\chi,\tau)\pm\ii B(\chi,\tau)\to\pm\ii$ as $r\to\infty$ uniformly with respect to $\theta$.  Also, $A(\chi,\tau)\pm\ii B(\chi,\tau)\to\infty$ as $(\chi,\tau)\to 0$ from the exterior of $\channels$.

In the study of high-order multiple-pole soliton solutions of the focusing nonlinear Schr\"odinger equation carried out in \cite{BilmanBW19}, the exterior of $\channels$ is further divided into three sub-regions, two unbounded and one bounded, on each of which a different modified phase function is needed (trivial modification, genus zero as described above, and genus one).  The rogue wave problem is simpler in that the single genus-zero phase function $h(\lambda;\chi,\tau)$ suffices to control the large-$k$ asymptotics throughout the exterior of $\channels$; however on the bounded component of the exterior identified in \cite{BilmanBW19} (where it is called the ``non-oscillatory region'' and which we denote by $\shelves$) (i) an additional $O(1)$ contribution to the phase appears in the leading term and (ii) the higher-order correction takes a different form than on the remaining unbounded component of the exterior, which we denote by $\exterior$.  The latter effect is observable in plots for finite order $k$.  The domain $\shelves$ abuts the domain $\channels$ along
the curve given by \eqref{eq:tau-squared} taken with the ``$+$'' sign, and the other part of its boundary in the first quadrant consists of a curve connecting the point $(\chi^\sharp,\tau^\sharp)$ defined in \eqref{eq:corner-point} with $(0,1)$.  While on the interior of $\shelves$ the roots of the quadratic factor in the numerator of \eqref{eq:hprime-formula} are real and distinct, denoted by $a(\chi,\tau)<b(\chi,\tau)$, the quadratic discriminant vanishes on this second boundary curve, which is shown with a solid blue line in Figure~\ref{fig:RegionsPlot}.


\begin{figure}[h]
\begin{center}
\includegraphics[width=0.5\linewidth]{RegionsPlot.pdf}
\end{center}
\caption{The first quadrant in the $(\chi,\tau)$-plane and the regions $\channels$, $\shelves$, and $\exterior$ (which is further divided into $\exterior_\tau$ and $\exterior_\chi$).  The red curve is given by \eqref{eq:boundary-curve} or \eqref{eq:tau-squared}.  The distinguished point $(\chi^\sharp,\tau^\sharp)$ defined by \eqref{eq:corner-point} is indicated with a red dot.  Along the solid and dotted blue curves, $h'(\lambda;\chi,\tau)$ given by \eqref{eq:hprime-formula} has a (real) double root.  Important phase transitions for high-order fundamental rogue waves occur along the red curve and solid blue curve; the unbounded dotted blue curve in $\exterior$ separating $\exterior_\tau$ from $\exterior_\chi$ is of mere technical significance in our analysis.  The dotted gray line $\chi=\sqrt{8}\tau$ is an asymptote for large $\chi$ to the unbounded branch (see Section~\ref{sec:critical-points}).  The curve $\ell_\mathrm{sol}$ emanating from $(\chi^\sharp,\tau^\sharp)$ into $\exterior_\tau$ and shown with a black dotted line is of no importance at all for high-order fundamental rogue waves but it is crucial in the study of high-order multiple-pole solitons and of secondary importance in the study of $q(x,t;\mathbf{G},M)$ for other solutions characterized by Riemann-Hilbert Problem~\ref{rhp:rogue-wave-reformulation}.}
\label{fig:RegionsPlot}
\end{figure}

Exiting $\shelves$ through that curve, the roots of the quadratic factor become a complex-conjugate pair.  There is one additional unbounded curve emanating from $(\chi^\sharp,\tau^\sharp)$ into the exterior (denoted $\exterior$) of $\overline{\channels\cup\shelves}$ along which the discriminant vanishes again.  Crossing this curve (shown as a dotted blue curve in Figure~\ref{fig:RegionsPlot}), the roots of the quadratic factor become real once again.  We refer to the two unbounded components of $\exterior$ separated by this curve as $\exterior_\chi$ (the unbounded component abutting the positive $\chi$-axis for $\chi>2$) and $\exterior_\tau$ (the unbounded component abutting the positive $\tau$-axis for $\tau>1$).
The roots of the quadratic factor $2\tau\lambda^2+u(\chi,\tau)\lambda+v(\chi,\tau)$ appearing in \eqref{eq:hprime-formula} are real when $(\chi,\tau)\in \exterior_\chi\cup \shelves$ and form a complex-conjugate pair when $(\chi,\tau)\in \exterior_\tau$.

\begin{remark}
Although the bounded domain $\shelves$ coincides exactly with the ``non-oscillatory region'' identified in \cite{BilmanBW19}, the curve separating $\exterior_\tau$ from $\exterior_\chi$ is not the same as the curve $\ell_\mathrm{sol}$ separating the two unbounded components of $\exterior$ (the ``oscillatory region'' and the ``exponential-decay region'') identified in \cite{BilmanBW19} and relevant for the study of $q(x,t;\mathbf{G},M)$ for large $M\in\tfrac{1}{2}\mathbb{Z}_{\ge 0}$.  The latter curve is shown with a black dotted line in Figure~\ref{fig:RegionsPlot}.
\end{remark}

A discussion of the qualitative features of high-order fundamental rogue waves can be found in \cite[Section 1.1]{BilmanLM20}.  Near the origin in the $(x,t)$-plane one observes a narrow wedge-shaped region centered on each half of the $x$-axis containing small-amplitude oscillations and larger complementary regions centered on each half of the $t$-axis containing waves of higher amplitude.  In \cite[Section 1.1]{BilmanLM20} these types of regions near the origin were called ``channels'' and ``shelves'' respectively.  The channels and shelves were proven in \cite{BilmanLM20} to have significance for the asymptotic behavior of the rogue wave of infinite order.  In this paper we show that the channels and shelves extend also to the macroscopic regime of bounded $(\chi,\tau)$ as $\channels$ and $\shelves$ respectively. As the paper \cite{BilmanLM20} was concerned with fundamental rogue waves in a neighborhood of the origin only, the identification of the exterior domain $\exterior$ is new in this work.  Note that by definition $\channels$, $\shelves$, and $\exterior$ are all relatively open pairwise disjoint subsets of the closed first quadrant, whose union excludes only the boundary curves shown with solid lines in Figure~\ref{fig:RegionsPlot}.  Likewise $\exterior_\chi$ and $\exterior_\tau$ are relatively open disjoint subsets of $\exterior$, whose union excludes only the dotted blue curve shown in Figure~\ref{fig:RegionsPlot}.

The significance of these regions for high-order fundamental rogue waves can be seen in Figure~\ref{fig:2D-Plots}.  
\begin{figure}[h]
\begin{center}
\phantom{!}\hfill%
\includegraphics[width=0.33\linewidth]{AltScaling-amplitude-regions-plot-orderk-8.pdf}%
\hfill%
\includegraphics[width=0.33\linewidth]{AltScaling-amplitude-regions-plot-orderk-16.pdf}%
\hfill%
\includegraphics[width=0.33\linewidth]{AltScaling-amplitude-regions-plot-orderk-32.pdf}%
\hfill\phantom{!}%
\end{center}
\caption{Density plots of $|\psi_k(M\chi,M\tau)|$ with the region boundaries superimposed for $k=8$ and $M=4.25$ (left), for $k=16$ and $M=8.25$ (center), and for $k=32$ and $M=16.25$ (right).}
\label{fig:2D-Plots}
\end{figure}

\subsection{Results}
\subsubsection{Asymptotic behavior of $q(x,t;\mathbf{Q}^{-s},M)$ and fundamental rogue waves for $(\chi,\tau)\in\channels$}
\label{sec:results-channels}
Recall from Section~\ref{sec:basic-exponent} that when $(\chi,\tau)\in \channels$, the phase $\vartheta(\lambda;\chi,\tau)$ defined in \eqref{eq:vartheta} has only real and simple critical points.  When also $\tau=0$, there are precisely two of them, given by \eqref{eq:tau-zero-critical-points}.  We denote the unique continuation of these critical points to the domain $\channels$ by $a=a(\chi,\tau)$ and $b=b(\chi,\tau)$ with $a<b$.  For $\tau\neq 0$ there is a third critical point born from $\lambda=\infty$, with leading asymptotic $\lambda=-\chi/(2\tau)+ O(1)$ as $\tau\to 0$.  Hence this critical point lies to the left of $\lambda=a$ for $\tau>0$ and to the right of $\lambda=b$ for $\tau<0$, since $\chi>0$ holds throughout $\channels$.  
For $M>0$, define real phases $\Theta_a^{[\channels]}(\chi,\tau;M)$ and $\Theta_b^{[\channels]}(\chi,\tau;M)$ by 
\begin{equation}
\begin{split}
\Theta_a^{[\channels]}(\chi,\tau;M)&\defeq M\Phi_a^{[\channels]}(\chi,\tau)-\ln(M)\frac{\ln(2)}{2\pi}+\eta_a^{[\channels]}(\chi,\tau),\\
\Theta_b^{[\channels]}(\chi,\tau;M)&\defeq M\Phi_b^{[\channels]}(\chi,\tau)+\ln(M)\frac{\ln(2)}{2\pi} + \eta_b^{[\channels]}(\chi,\tau),
\end{split}
\label{eq:channels-phases}
\end{equation}
in which, noting that $\vartheta(a(\chi,\tau);\chi,\tau)$ and $\vartheta(b(\chi,\tau);\chi,\tau)$ are both real,
\begin{equation}
\begin{split}
\Phi_a^{[\channels]}(\chi,\tau)&\defeq -2\vartheta(a(\chi,\tau);\chi,\tau),\\
\Phi_b^{[\channels]}(\chi,\tau)&\defeq -2\vartheta(b(\chi,\tau);\chi,\tau)
\end{split}
\label{eq:channels-principal-phases}
\end{equation}
and, noting that $\vartheta''(a(\chi,\tau);\chi,\tau)<0$ and $\vartheta''(b(\chi,\tau);\chi,\tau)>0$ (derivatives with respect to $\lambda$),
\begin{equation}
\begin{split}
\eta_a^{[\channels]}(\chi,\tau)&\defeq -\frac{\ln(2)}{2\pi}\ln\left(-(b(\chi,\tau)-a(\chi,\tau))^2\vartheta''(a(\chi,\tau);\chi,\tau)\right)\\&\qquad\qquad\qquad\qquad{}-\frac{\ln(2)^2}{2\pi}-\frac{1}{4}\pi+\arg\left(\Gamma\left(\frac{\ii\ln(2)}{2\pi}\right)\right),\\
\eta_b^{[\channels]}(\chi,\tau)&\defeq \frac{\ln(2)}{2\pi}\ln\left((b(\chi,\tau)-a(\chi,\tau))^2\vartheta''(b(\chi,\tau);\chi,\tau)\right)\\
&\qquad\qquad\qquad\qquad{}+\frac{\ln(2)^2}{2\pi}+\frac{1}{4}\pi-\arg\left(\Gamma\left(\frac{\ii\ln(2)}{2\pi}\right)\right).
\end{split}
\label{eq:channels-lower-order-phases}
\end{equation}
Also, define two positive amplitudes by
\begin{equation}
\begin{split}
F_a^{[\channels]}(\chi,\tau)&\defeq \sqrt{-\frac{\ln(2)}{\pi\vartheta''(a(\chi,\tau);\chi,\tau)}},\\
F_b^{[\channels]}(\chi,\tau)&\defeq \sqrt{\frac{\ln(2)}{\pi\vartheta''(b(\chi,\tau);\chi,\tau)}}.
\end{split}
\label{eq:channels-amplitudes}
\end{equation}
%For $M>0$, define a phase $\phi\in\mathbb{R}$ by
%\begin{equation}
%\phi\defeq -\frac{\ln(2)}{2\pi}\ln(M)-\frac{\ln(2)}{\pi}\ln(b(\chi,\tau)-a(\chi,\tau))-\frac{\ln(2)^2}{2\pi}-\frac{1}{4}\pi+\arg\left(\Gamma\left(\frac{\ii\ln(2)}{2\pi}\right)\right).
%\label{eq:intro-phi-def}
%\end{equation}
%Also, noting that $\vartheta(a(\chi,\tau);\chi,\tau)$ and $\vartheta(b(\chi,\tau);\chi,\tau)$ are both real while $\vartheta''(a(\chi,\tau);\chi,\tau)<0$ and $\vartheta''(b(\chi,\tau);\chi,\tau)>0$ (derivatives with respect to $\lambda$), we define two real phases by
%\begin{equation}
%\begin{split}
%\Phi_a&\defeq -2M\vartheta(a(\chi,\tau);\chi,\tau)-\frac{\ln(2)}{2\pi}\ln(-\vartheta''(a(\chi,\tau);\chi,\tau))\\
%\Phi_b&\defeq -2M\vartheta(b(\chi,\tau);\chi,\tau)+\frac{\ln(2)}{2\pi}\ln(\vartheta''(b(\chi,\tau);\chi,\tau)).
%\end{split}
%\label{eq:intro-Phia-Phib}
%\end{equation}
Our first result is then the following:  
\begin{theorem}[Far-field asymptotics of $q(x,t;\mathbf{Q}^{-s},M)$ for $(\chi,\tau)\in\channels$]
Let $s=\pm 1$ be arbitrary.  Then, as $M\to+\infty$ through an arbitrary sequence of values,
$q(M\chi,M\tau;\mathbf{Q}^{-s},M)=\mathfrak{L}_s^{[\channels]}(\chi,\tau;M)+O(M^{-\frac{3}{2}})$, where
%\begin{equation}
%\ell_s^{[\channels]}(\chi,\tau;M)\defeq s\sqrt{\frac{\ln(2)}{\pi M}}\left[\frac{\ee^{\ii(\Phi_a+\phi)}}{(-\vartheta''(a(\chi,\tau);\chi,\tau))^{\frac{1}{2}}}+\frac{\ee^{\ii(\Phi_b-\phi)}}{\vartheta''(b(\chi,\tau);\chi,\tau)^\frac{1}{2}}\right],
%\label{eq:leading-term-channels-q}
%\end{equation}
\begin{equation}
\mathfrak{L}_s^{[\channels]}(\chi,\tau;M)\defeq sM^{-\frac{1}{2}}\left[F_a^{[\channels]}(\chi,\tau)\ee^{\ii\Theta_a^{[\channels]}(\chi,\tau;M)} + F_b^{[\channels]}(\chi,\tau)\ee^{\ii\Theta_b^{[\channels]}(\chi,\tau;M)}\right],
\label{eq:leading-term-channels-q}
\end{equation}
and where the error term is uniform for $(\chi,\tau)$ in any compact subset of $\channels$.
\label{thm:channels}
\end{theorem}
We present the proof in Section~\ref{sec:channels}.  Note that as this result allows for $M$ to take any positive values tending to $+\infty$, it describes both high-order multiple-pole soliton solutions and fundamental rogue waves over the same domain $\channels$ (as well as many other families of solutions interpolating between the two types); hence $\channels$ with its reflections in the coordinate axes forms a component of the region bounded by the yellow curves in Figure~\ref{fig:roguewaves-and-solitons}.   For the high-order multiple-pole soliton case corresponding to large $M\in\tfrac{1}{2}\mathbb{Z}_{\ge 0}$, it implies one of the results in \cite{BilmanBW19}, although we sharpen the error estimate from $O(M^{-1})$ to $O(M^{-\frac{3}{2}})$.  For the rogue wave case of most interest to us here, we need to correlate the values of $M$ and the index $s$ to the order $k$ and include an additional exponential factor:
\begin{corollary}
The fundamental rogue wave of order $k\in\mathbb{Z}_{>0}$ satisfies $\psi_k(M\chi,M\tau)=L_k^{[\channels]}(\chi,\tau)+O(k^{-\frac{3}{2}})$, where
\begin{equation}
L_k^{[\channels]}(\chi,\tau)\defeq 
\ee^{-\ii M\tau}\mathfrak{L}_s^{[\channels]}(\chi,\tau;M),
%=(-1)^k\ee^{-\ii M\tau}\sqrt{\frac{\ln(2)}{\pi M}}\left[\frac{\ee^{\ii(\Phi_a+\phi)}}{(-\vartheta''(a(\chi,\tau);\chi,\tau))^{\frac{1}{2}}}+\frac{\ee^{\ii(\Phi_b-\phi)}}{\vartheta''(b(\chi,\tau);\chi,\tau)^\frac{1}{2}}\right],
\quad s=(-1)^k,\quad M=\tfrac{1}{2}k+\tfrac{1}{4},
\label{eq:leading-term-channels}
\end{equation}
in which $\mathfrak{L}_s^{[\channels]}(\chi,\tau;M)$ is given by \eqref{eq:leading-term-channels-q} 
and where the error term is uniform for $(\chi,\tau)$ in any compact subset of $\channels$.
\label{cor:rogue-wave-channels}
\end{corollary}
When $\tau=0$, the two oscillations in the leading term have a common amplitude, and the formula simplifies further.  Indeed,
\begin{equation}
L_k^{[\channels]}(\chi,0)=\sqrt{\frac{\ln(2)}{\pi M}}\frac{2}{\chi^\frac{3}{4}(2-\chi)^\frac{1}{4}}\cos\left(2MF(\chi)-\frac{\ln(2)}{2\pi}\ln(M)-\Omega(\chi)+\Phi_0\right),\quad 0<\chi<2,
\label{eq:leading-term-channels-tau-zero}
\end{equation}
where
\begin{equation}
F(\chi)\defeq\chi\sqrt{\frac{2}{\chi}-1}+\pi-2\tan^{-1}\left(\sqrt{\frac{2}{\chi}-1}\right),\quad
\Omega(\chi)\defeq\frac{\ln(2)}{\pi}\ln(\chi)+\frac{3\ln(2)}{2\pi}\ln\left(\sqrt{\frac{2}{\chi}-1}\right),
\end{equation}
and
\begin{equation}
\Phi_0\defeq\left(k-\frac{1}{4}\right)\pi-\frac{3\ln(2)^2}{2\pi}+\arg\left(\Gamma\left(\frac{\ii\ln(2)}{2\pi}\right)\right).
\label{eq:leading-term-channels-tau-zero-last}
\end{equation}

\subsubsection{Asymptotic behavior of fundamental rogue waves for $(\chi,\tau)\in\exterior$}
\label{sec:Results-Exterior}
Unlike the analysis for $(\chi,\tau)\in\channels$, our next result pertains to fundamental rogue waves only, i.e., we cannot allow $M$ to tend to $\infty$ in an arbitrary fashion without making substantial modifications that are beyond the scope of this work.  Let $\gamma(\chi,\tau)$ be defined on $\exterior$ by
\begin{equation}
\gamma(\chi,\tau)\defeq \chi A(\chi,\tau)+\tau(A(\chi,\tau)^2-\tfrac{1}{2}B(\chi,\tau)^2)+\ii\int_{-\ii}^{\lambda_0(\chi,\tau)^*}\frac{\dd\lambda}{R(\lambda;\chi,\tau)}+\ii\int_{\lambda_0(\chi,\tau)}^{\ii}
\frac{\dd\lambda}{R(\lambda;\chi,\tau)},
\label{eq:gamma-formula-intro}
\end{equation}
in which the path of integration in each integral is arbitrary in the part of the upper/lower half-plane complementary to $\Sigma_g$.  (In practice, to compute $\gamma(\chi,\tau)$ for $\chi>0$ in $\exterior$ it suffices to let $R(\lambda;\chi,\tau)$ have a vertical branch cut connecting $\lambda_0(\chi,\tau)$ and $\lambda_0(\chi,\tau)^*$; recall that $A(\chi,\tau)=\mathrm{Re}(\lambda_0(\chi,\tau))$ and $B(\chi,\tau)=\mathrm{Im}(\lambda_0(\chi,\tau))$.)  
\begin{theorem}[Far-field asymptotics of $\psi_k(x,t)$ for $(\chi,\tau)\in\exterior$]
The fundamental rogue wave $\psi_k(x,t)$ of order $k\in\mathbb{Z}_{>0}$ satisfies $\psi_k(M\chi,M\tau)=L_k^{[\exterior]}(\chi,\tau)+O(k^{-1})$, where
\begin{equation}
L_k^{[\exterior]}(\chi,\tau)\defeq B(\chi,\tau)\ee^{-\ii M\tau}\ee^{-2\ii M\gamma(\chi,\tau)},\quad M=\tfrac{1}{2}k+\tfrac{1}{4},
\label{eq:leading-term-exterior}
\end{equation}
and where the error term is uniform for $(\chi,\tau)$ in compact subsets of $\exterior$.
\label{thm:exterior}
\end{theorem}
This result shows a marked difference between high-order fundamental rogue waves and high-order soliton solutions of the focusing nonlinear Schr\"odinger equation.  Indeed, for $(\chi,\tau)\in\exterior$, fundamental rogue waves behave like a slowly-modulated plane-wave solution of the same equation.  By contrast, high-order multiple-pole solitons behave like a slowly-modulated elliptic function solution or decay exponentially to zero on complementary subregions of $\exterior$ \cite{BilmanBW19}. It is not difficult to show that $\gamma(\chi,\tau)+\tfrac{1}{2}\tau\to 0$ as $(\chi,\tau)\to\infty$ in $\exterior$.  In conjunction with the fact that $A(\chi,\tau)\pm\ii B(\chi,\tau)\to \pm\ii$ as $(\chi,\tau)\to\infty$ in $\exterior$, this shows that the leading term $L_k^{[\exterior]}(\chi,\tau)$ tends to the background solution $\psi_0\equiv 1$ as $(\chi,\tau)\to\infty$ in $\exterior$, a result that is consistent with the known asymptotic $\psi_k(x,t)\to 1$ as $(x,t)\to\infty$ in $\mathbb{R}^2$, although our proof of Theorem~\ref{thm:exterior} as given in Section~\ref{sec:Schi-Stau} does not allow $(\chi,\tau)$ to become unbounded.  

It is also worth noting that the leading term $L_k^{[\exterior]}(\chi,\tau)$ becomes explicit if $\tau=0$.  Indeed, using \eqref{eq:AB-tau-small} below, one sees that $A(\chi,0)=0$ and $B(\chi,\tau)^2 = 1-4/\chi^2$ for $\chi>2$, and hence also from \eqref{eq:gamma-formula-intro} one obtains $\gamma(\chi,0)=0$.  Therefore,
\begin{equation}
L_k^{[\exterior]}(\chi,0)=\sqrt{1-\frac{4}{\chi^2}},\quad \chi>2,
\end{equation}
a formula that, in light of Theorem~\ref{thm:exterior}, describes precisely how $\psi_k(M\chi,0)$ rises from being small of size $k^{-\frac{1}{2}}$ for $0<\chi<2$ (as given in Corollary~\ref{cor:rogue-wave-channels} and \eqref{eq:leading-term-channels-tau-zero}--\eqref{eq:leading-term-channels-tau-zero-last}) to ultimately approach the unit background value for large $x$.

\subsubsection{Asymptotic behavior of $q(x,t;\mathbf{Q}^{-s},M)$ and fundamental rogue waves for $(\chi,\tau)\in \shelves$}
\label{sec:Results-Shelves}
The next results again allow $M$ to become large in an arbitrary fashion, and they concern the asymptotic description of $q(x,t;\mathbf{Q}^{-s},M)$ for rescaled coordinates $(\chi,\tau)\in \shelves$.  An obvious feature of the plots shown in Figure~\ref{fig:2D-Plots} as well as similar plots of high-order multiple-pole solitons \cite{BilmanBW19} is that in the domain $\shelves$ there are evidently amplitude oscillations of small (on the scale of $\chi$ and $\tau$) wavelength and period.  To capture these oscillations it is necessary to include both a leading term and a first error term in an asymptotic formula for $q(x,t;\mathbf{Q}^{-s},M)$.  To formulate our result, we first define some quantities.  Recall that for $(\chi,\tau)\in \shelves$, the function $h'(\lambda;\chi,\tau)$ has two real simple zeros $a(\chi,\tau)<b(\chi,\tau)$, as well as a conjugate pair $A(\chi,\tau)\pm\ii B(\chi,\tau)$ of branch points.  For such $(\chi,\tau)$ we assume that the Schwarz-symmetric logarithmic branch cut $\Sigma_\mathrm{c}$ connecting $\pm\ii$ with upward orientation crosses the real axis at a unique point between $a(\chi,\tau)$ and $b(\chi,\tau)$.  First, set
\begin{equation}
\kappa(\chi,\tau)\defeq \chi A(\chi,\tau)+\tau(A(\chi,\tau)^2-\tfrac{1}{2}B(\chi,\tau)^2)+\ii\int_{\Sigma_\mathrm{c}}\frac{\dd\lambda}{R(\lambda;\chi,\tau)},\quad (\chi,\tau)\in \shelves,
\label{eq:kappa-formula}
\end{equation}
which is well defined under the assumption that the path of integration $\Sigma_\mathrm{c}$ lies to the right of the Schwarz-symmetric branch cut $\Sigma_g$ of $R$, which we assume crosses the real axis only at $\lambda=a(\chi,\tau)$.  Then define
\begin{equation}
\mu(\chi,\tau)\defeq\frac{\ln(2)}{2\pi}\int_{a(\chi,\tau)}^{b(\chi,\tau)}\frac{\dd\lambda}{R(\lambda;\chi,\tau)}>0,
\label{eq:mu-formula-intro}
\end{equation}
where the integration is on the real line where the integrand is strictly positive.  Next, we set
\begin{equation}
K_a(\chi,\tau)\defeq \frac{\ln(2)}{\pi}\frac{|a(\chi,\tau)-\lambda_0(\chi,\tau)|}{2\pi\ii}
\int_{C}\log\left(\frac{\lambda-a(\chi,\tau)}{\lambda-b(\chi,\tau)}\right)\frac{\dd\lambda}{R(\lambda;\chi,\tau)(\lambda-a(\chi,\tau))}
\label{eq:intro-Ka}
\end{equation}
and
\begin{equation}
K_b(\chi,\tau)\defeq \frac{\ln(2)}{\pi}\frac{|b(\chi,\tau)-\lambda_0(\chi,\tau)|}{2\pi\ii}
\int_{C}\log\left(\frac{\lambda-a(\chi,\tau)}{\lambda-b(\chi,\tau)}\right)\frac{\dd\lambda}{R(\lambda;\chi,\tau)(\lambda-b(\chi,\tau))},
\label{eq:intro-Kb}
\end{equation}
where the contour $C$ lies to the left of $\Sigma_g$ with the same endpoints and orientation, and where the logarithm is cut on the real line in $[a(\chi,\tau),b(\chi,\tau)]$ and tends to zero as $\lambda\to\infty$.  Now we define real phases $\Theta_a^{[\shelves]}(\chi,\tau;M)$ and $\Theta_b^{[\shelves]}(\chi,\tau;M)$ by (compare with \eqref{eq:channels-phases})
\begin{equation}
\begin{split}
\Theta_a^{[\shelves]}(\chi,\tau;M)&\defeq M\Phi_a^{[\shelves]}(\chi,\tau)-\ln(M)\frac{\ln(2)}{2\pi}+\eta_a^{[\shelves]}(\chi,\tau)
%+\delta_a(\chi,\tau)
\\
\Theta_b^{[\shelves]}(\chi,\tau;M)&\defeq M\Phi_b^{[\shelves]}(\chi,\tau)+\ln(M)\frac{\ln(2)}{2\pi} +
\eta_b^{[\shelves]}(\chi,\tau) %+\delta_b(\chi,\tau)
\end{split}
\label{eq:Thetas-shelves}
\end{equation}
in which, noting that $h_-(a(\chi,\tau);\chi,\tau)$ and $h(b(\chi,\tau);\chi,\tau)$ are both real and comparing with \eqref{eq:channels-principal-phases},
\begin{equation}
\begin{split}
\Phi_a^{[\shelves]}(\chi,\tau)&\defeq -2h_-(a(\chi,\tau);\chi,\tau)\\
\Phi_b^{[\shelves]}(\chi,\tau)&\defeq -2h(b(\chi,\tau);\chi,\tau)
\end{split}
\label{eq:Phis-shelves}
\end{equation}
and, noting that $h''_-(a(\chi,\tau);\chi,\tau)<0$ and $h''(b(\chi,\tau);\chi,\tau)>0$ and comparing with \eqref{eq:channels-lower-order-phases},
\begin{equation}
\begin{split}
\eta_a^{[\shelves]}(\chi,\tau)&\defeq -\frac{\ln(2)}{2\pi}\ln\left(-(b(\chi,\tau)-a(\chi,\tau))^2h''_-(a(\chi,\tau);\chi,\tau)\right)\\
&\qquad\qquad\qquad\qquad{}-\frac{\ln(2)^2}{2\pi}-\frac{1}{4}\pi+\arg\left(\Gamma\left(\frac{\ii\ln(2)}{2\pi}\right)\right),\\
\eta_b^{[\shelves]}(\chi,\tau)&\defeq \frac{\ln(2)}{2\pi}\ln\left((b(\chi,\tau)-a(\chi,\tau))^2h''(b(\chi,\tau);\chi,\tau)\right)\\
&\qquad\qquad\qquad\qquad{}+\frac{\ln(2)^2}{2\pi}+\frac{1}{4}\pi-\arg\left(\Gamma\left(\frac{\ii\ln(2)}{2\pi}\right)\right).
\end{split}
\end{equation}
We also define additional real phases by
\begin{equation}
\begin{split}
\delta_a(\chi,\tau)&\defeq \pi-2(K_a(\chi,\tau)+\mu(\chi,\tau)),\\
\delta_b(\chi,\tau)&\defeq -2(K_b(\chi,\tau)+\mu(\chi,\tau)).
\end{split}
\end{equation}
By analogy with \eqref{eq:channels-amplitudes} define positive amplitudes by
\begin{equation}
\begin{split}
F_a^{[\shelves]}(\chi,\tau)&\defeq \sqrt{-\frac{\ln(2)}{\pi h''_-(a(\chi,\tau);\chi,\tau)}},\\
F_b^{[\shelves]}(\chi,\tau)&\defeq \sqrt{\frac{\ln(2)}{\pi h''(b(\chi,\tau);\chi,\tau)}}.
\end{split}
\end{equation}
Finally, define four positive modulation factors with range $[0,1]$ by
\begin{equation}
\begin{split}
m_a^\pm(\chi,\tau)&\defeq\tfrac{1}{2}\left(1\pm\cos\left(\arg\left(a(\chi,\tau)-\lambda_0(\chi,\tau)\right)\right)\right),\\
m_b^\pm(\chi,\tau)&\defeq\tfrac{1}{2}\left(1\pm\cos\left(\arg\left(b(\chi,\tau)-\lambda_0(\chi,\tau)\right)\right)\right).
\end{split}
\label{eq:m-a-b-shelves}
\end{equation}
Our main result for the region $\shelves$ is then the following.
\begin{theorem}[Far-field asymptotics of $q(x,t;\mathbf{Q}^{-s},M)$ for $(\chi,\tau)\in\shelves$]
Let $s=\pm 1$ be arbitrary.  Then, as $M\to+\infty$ through an arbitrary sequence of values,
$q(M\chi,M\tau;\mathbf{Q}^{-s};M)=\mathfrak{L}_s^{[\shelves]}(\chi,\tau;M)+\mathfrak{S}_s^{[\shelves]}(\chi,\tau;M) + O(M^{-1})$, where
\begin{equation}
\mathfrak{L}_s^{[\shelves]}(\chi,\tau;M)\defeq B(\chi,\tau)\ee^{-2\ii (M\kappa(\chi,\tau)+\mu(\chi,\tau)+\frac{1}{4}s\pi)},
\label{eq:leading-term-shelves-q}
\end{equation}
and 
\begin{multline}
\mathfrak{S}_s^{[\shelves]}(\chi,\tau;M)\defeq sM^{-\frac{1}{2}}\Big[m_a^+(\chi,\tau)\ee^{\ii\delta_a(\chi,\tau)}F_a^{[\shelves]}(\chi,\tau)\ee^{\ii\Theta_a^{[\shelves]}(\chi,\tau;M)} \\
{}+ m_b^+(\chi,\tau)\ee^{\ii\delta_b(\chi,\tau)}F_b^{[\shelves]}(\chi,\tau)\ee^{\ii\Theta_b^{[\shelves]}(\chi,\tau;M)} \\
{}- m_a^-(\chi,\tau)\ee^{-\ii\delta_a(\chi,\tau)}F_a^{[\shelves]}(\chi,\tau)\ee^{-\ii[\Theta_a^{[\shelves]}(\chi,\tau;M)+4M\kappa(\chi,\tau)+4\mu(\chi,\tau)]} \\
{}- m_b^-(\chi,\tau)\ee^{-\ii\delta_b(\chi,\tau)}F_b^{[\shelves]}(\chi,\tau)\ee^{-\ii[\Theta_b^{[\shelves]}(\chi,\tau;M)+4M\kappa(\chi,\tau)+4\mu(\chi,\tau)]}\Big],
\label{eq:subleading-term-shelves-q}
\end{multline}
and where the error term is uniform for $(\chi,\tau)$ in any compact subset of $\shelves$.
\label{thm:shelves}
\end{theorem}
This result therefore provides both a leading term $\mathfrak{L}_s^{[\shelves]}(\chi,\tau;M)$ (which in the case of high-order multiple-pole solitons with $M\in\mathbb{Z}_{>0}$ was obtained in \cite{BilmanBW19}) and a sub-leading term $\mathfrak{S}_s^{[\shelves]}(\chi,\tau;M)$.  As with Theorem~\ref{thm:channels}, this result applies to the full family of solutions including both solitons and rogue waves, and hence $\shelves$ with its reflections in the coordinate axes forms the remaining components of the region bounded by the yellow curves in Figure~\ref{fig:roguewaves-and-solitons}.  To write the formula in the rogue wave case requires just cosmetic modification; the analogue of Corollary~\ref{cor:rogue-wave-channels} when $(\chi,\tau)\in\shelves$ is the following.
\begin{corollary}
The fundamental rogue wave of order $k\in\mathbb{Z}_{>0}$ satisfies $\psi_k(M\chi,M\tau)=L_k^{[\shelves]}(\chi,\tau)+S_k^{[\shelves]}(\chi,\tau)+O(k^{-1})$, where
\begin{equation}
L_k^{[\shelves]}(\chi,\tau)\defeq\ee^{-\ii M\tau}\mathfrak{L}_s^{[\shelves]}(\chi,\tau;M),\quad
S_k^{[\shelves]}(\chi,\tau)\defeq\ee^{-\ii M\tau}\mathfrak{S}_s^{[\shelves]}(\chi,\tau;M),\quad
s=(-1)^k,\quad M=\tfrac{1}{2}k+\tfrac{1}{4},
\label{eq:rw-terms-shelves}
\end{equation}
in which $\mathfrak{L}_s^{[\shelves]}(\chi,\tau;M)$ and $\mathfrak{S}_s^{[\shelves]}(\chi,\tau;M)$ are given by \eqref{eq:leading-term-shelves-q} and \eqref{eq:subleading-term-shelves-q} respectively, and
where the error term is uniform for $(\chi,\tau)$ in any compact subset of $\shelves$.
\label{cor:rogue-wave-shelves}
\end{corollary}

%\textcolor{red}{This result generalizes \cite[Theorem 3]{BilmanBW19} in two important ways:  (i) it is valid in the limit $M\to+\infty$ along any sequence and (ii) it includes the first correction to the leading term.}  

The first correction on the domain $\shelves$ resolves the obvious oscillations visible in plots of high-order multiple-pole soliton solutions \cite{BilmanBW19} and in plots of high-order fundamental rogue waves such as those shown in Figure~\ref{fig:2D-Plots}.  On two-dimensional plots such as these, one observes that these fluctuations form a highly-regular interference pattern.  To see how Theorem~\ref{thm:shelves} yields such a pattern, we can rewrite the combination $\mathfrak{L}_s^{[\shelves]}(\chi,\tau;M)+\mathfrak{S}_s^{[\shelves]}(\chi,\tau;M)$ in a different form by factoring out a phase factor, which has the effect of producing some symmetry in the four phases present in \eqref{eq:subleading-term-shelves-q}.  Therefore, using $s=\pm 1$, we write:
\begin{multline}
\mathfrak{L}_s^{[\shelves]}(\chi,\tau;M)+\mathfrak{S}_s^{[\shelves]}(\chi,\tau;M) = s\ee^{-2\ii\phi(\chi,\tau;M)}\Big[-\ii B(\chi,\tau)\\
{}+M^{-\frac{1}{2}}\Big(m_a^+(\chi,\tau)F_a^{[\shelves]}(\chi,\tau)\ee^{\ii\phi_a(\chi,\tau;M)} -
m_a^-(\chi,\tau)F_a^{[\shelves]}(\chi,\tau)\ee^{-\ii\phi_a(\chi,\tau;M)} \\
{}+
m_b^+(\chi,\tau)F_b^{[\shelves]}(\chi,\tau)\ee^{\ii\phi_b(\chi,\tau;M)} -
m_b^-(\chi,\tau)F_b^{[\shelves]}(\chi,\tau)\ee^{-\ii\phi_b(\chi,\tau;M)}\Big)
\Big],
\label{eq:leading-and-subleading-shelves-rewritten}
\end{multline}
in which
\begin{equation}
\begin{split}
\phi(\chi,\tau;M)&\defeq M\kappa(\chi,\tau)+\mu(\chi,\tau),\\
\phi_a(\chi,\tau;M)&\defeq \Theta^{[\shelves]}_a(\chi,\tau;M)+2M\kappa(\chi,\tau) + \delta_a(\chi,\tau) + 2\mu(\chi,\tau),\\
\phi_b(\chi,\tau;M)&\defeq \Theta^{[\shelves]}_b(\chi,\tau;M)+2M\kappa(\chi,\tau) + \delta_b(\chi,\tau) + 2\mu(\chi,\tau).
\end{split}
\label{eq:symmetrical-phases}
\end{equation}
Using the fact that $m_a^+(\chi,\tau)+m_a^-(\chi,\tau)=m_b^+(\chi,\tau)+m_b^-(\chi,\tau)=1$ to expand the square modulus of the right-hand side of \eqref{eq:leading-and-subleading-shelves-rewritten} through terms proportional to $M^{-\frac{1}{2}}$, and combining with Theorem~\ref{thm:shelves} then gives the following.
\begin{corollary}
Let $s=\pm 1$ be arbitrary.  Then as $M\to+\infty$ through an arbitrary sequence of values, 
\begin{multline}
|q(M\chi,M\tau;\mathbf{Q}^{-s},M)|^2 = 
B(\chi,\tau)^2\\
{}-2M^{-\frac{1}{2}}B(\chi,\tau)\left[F_a^{[\shelves]}(\chi,\tau)\sin(\phi_a(\chi,\tau;M))+F_b^{[\shelves]}(\chi,\tau)\sin(\phi_b(\chi,\tau;M))\right]+O(M^{-1}),
\label{eq:interference}
\end{multline}
where the error is uniform for $(\chi,\tau)$ in compact subsets of $\shelves$.
\label{cor:pattern}
\end{corollary}
Since $|q|^2=|\psi_k|^2$ when $s=(-1)^k$ and $M=\tfrac{1}{2}k+\tfrac{1}{4}$, this result explains the interference pattern seen in amplitude plots of high-order fundamental rogue waves such as in \cite[Figure 2]{BilmanLM20} and in Figure~\ref{fig:2D-Plots} of this paper.  However, as it is valid for arbitrary $M\to+\infty$, the same formula also explains the similar patterns observed in plots of $k^\mathrm{th}$-order pole solitons for $M=\tfrac{1}{2}k$ large such as can be found in \cite{BilmanB19,BilmanBW19}.  It is equally valid for all other increasing sequences of $M$-values that do not correspond to either type of solution.
Corollary~\ref{cor:pattern} shows that for $(\chi,\tau)\in\shelves$, the squared modulus of $q(M\chi, M \tau; \mathbf{Q}^{-s}, M)$ consists of a slowly-varying ``shelf'' of size $O(1)$ and a rapidly-varying perturbation proportional to $M^{-\frac{1}{2}}$. To leading order, this perturbation is a superposition of two sine functions with different phases $\phi_a(\chi,\tau;M)$ and $\phi_b(\chi,\tau;M)$ whose derivatives are large for $M\gg 1$ due to the presence of the terms $2M[\kappa(\chi,\tau)-h_-(a(\chi,\tau);\chi,\tau)]$ and $2M[\kappa(\chi,\tau)-h(b(\chi,\tau);\chi,\tau)]$ respectively, see \eqref{eq:Thetas-shelves}, \eqref{eq:Phis-shelves}, and \eqref{eq:symmetrical-phases}.  
Since $F_a^{[\shelves]}(\chi,\tau)$ and $F_b^{[\shelves]}(\chi,\tau)$ are both positive, the two terms proportional to $M^{-\frac{1}{2}}$ in \eqref{eq:interference} are individually maximized when $\phi_a(\chi,\tau;M)\in (-\tfrac{1}{2}+2\mathbb{Z})\pi$ and where $\phi_b(\chi,\tau;M)\in (-\tfrac{1}{2}+2\mathbb{Z})\pi$, each condition of which produces a ($M$-dependent) system of curves that can be plotted over the region $\shelves$ in the $(\chi,\tau)$-plane.  Provided that $\nabla\phi_a(\chi,\tau;M)$ and $\nabla\phi_b(\chi,\tau;M)$ are linearly-independent vectors near a given point $(\chi,\tau)\in\shelves$, the two systems of maximizing curves will intersect each other transversely and there will be isolated local maxima of $|q|^2$ that form a locally-regular parallelogram lattice of increasing density as $M\to+\infty$.  
When $M>0$ is large, the gradient vectors of the phases $\phi_a(\chi,\tau;M)$ and $\phi_b(\chi,\tau;M)$ are dominated by the terms proportional to $M$.  Then, since in the limit that $(\chi,\tau)$ approaches the common boundary of $\shelves$ and $\exterior$ the real critical points $a(\chi,\tau)<b(\chi,\tau)$ coalesce, one can see that these leading terms coincide at the boundary curve, implying that the systems of maximizing curves nearly coincide at this boundary of $\shelves$.  Therefore, in this limit, the lattice of local maxima degenerates into a pattern of stripes instead, such as can be seen along the blue curves in Figure~\ref{fig:2D-Plots}.  For high-order multiple-pole solitons, the stripes in the square modulus $|q|^2$ grow as $(\chi,\tau)$ exits $\shelves$ and form a stripe pattern of $O(1)$ size that are modeled by an elliptic function in the ``oscillatory region'' that is a proper subset of $\exterior$ abutting $\shelves$ \cite{BilmanBW19}; for high-order fundamental rogue waves the stripes instead decay away as $(\chi,\tau)$ exits $\shelves$, leaving only the slowly-varying background amplitude $B(\chi,\tau)>0$ as described on the whole of $\exterior$ by Theorem~\ref{thm:exterior}. See also Figure~\ref{fig:roguewaves-and-solitons}.

As a final corollary of Theorem~\ref{thm:shelves}, we present a space-time localized asymptotic formula for $q(M\chi,M\tau;\mathbf{Q}^{-s},M)$.
\begin{corollary}
Let $s=\pm 1$ be arbitrary, and fix $(\chi_0,\tau_0)\in\shelves$.  Then, as $M\to+\infty$ through an arbitrary sequence of values,
\begin{equation}
q(M\chi_0+\Delta x,M\tau_0+\Delta t;\mathbf{Q}^{-s},M)=Q(\Delta x,\Delta t)\left(1+M^{-\frac{1}{2}}\left(p_a(\Delta x,\Delta t)+p_b(\Delta x,\Delta t)\right)\right)+O(M^{-1})
\label{eq:Q-perturbation-shelves}
\end{equation}
holds uniformly for bounded $(\Delta x,\Delta t)$, where 
\begin{equation}
Q(\Delta x,\Delta t)\defeq \mathcal{A}\ee^{\ii(\xi_0\Delta x-\Omega_0\Delta t)},\quad \mathcal{A}\defeq -\ii s\ee^{-2\ii\phi(\chi_0,\tau_0;M)}B(\chi_0,\tau_0),
\label{eq:leading-plane-wave-intro}
\end{equation}
and
\begin{equation}
\begin{split}
p_a(\Delta x,\Delta t)&\defeq \ii\frac{F_a^{[\shelves]}(\chi_0,\tau_0)}{B(\chi_0,\tau_0)}\Big[
m_a^+(\chi_0,\tau_0)\ee^{\ii\phi_a(\chi_0,\tau_0;M)}\ee^{\ii(\xi_a\Delta x-\Omega_a\Delta t)} \\
&\qquad\qquad\qquad {}- 
m_a^-(\chi_0,\tau_0)\ee^{-\ii\phi_a(\chi_0,\tau_0;M)}\ee^{-\ii(\xi_a\Delta x-\Omega_a\Delta t)}\Big],\\
p_b(\Delta x,\Delta t)&\defeq \ii\frac{F_b^{[\shelves]}(\chi_0,\tau_0)}{B(\chi_0,\tau_0)}\Big[
m_b^+(\chi_0,\tau_0)\ee^{\ii\phi_b(\chi_0,\tau_0;M)}\ee^{\ii(\xi_b\Delta x-\Omega_b\Delta t)} \\
&\qquad\qquad\qquad {}- 
m_b^-(\chi_0,\tau_0)\ee^{-\ii\phi_b(\chi_0,\tau_0;M)}\ee^{-\ii(\xi_b\Delta x-\Omega_b\Delta t)}\Big],
\end{split}
\label{eq:p-a-b-shelves}
\end{equation}
in which real local wavenumbers are defined by
\begin{equation}
\begin{split}
\xi_0&\defeq -2\kappa_\chi(\chi_0,\tau_0),\\
\xi_a&\defeq 2(\kappa_\chi(\chi_0,\tau_0)-h_{\chi-}(a(\chi_0,\tau_0);\chi_0,\tau_0)),\\
\xi_b&\defeq 2(\kappa_\chi(\chi_0,\tau_0)-h_{\chi}(b(\chi_0,\tau_0);\chi_0,\tau_0)),
\end{split}
\label{eq:wavenumbers-intro}
\end{equation}
and real local frequencies are defined by
\begin{equation}
\begin{split}
\Omega_0&\defeq 2\kappa_\tau(\chi_0,\tau_0),\\
\Omega_a&\defeq -2(\kappa_\tau(\chi_0,\tau_0)-h_{\tau-}(a(\chi_0,\tau_0);\chi_0,\tau_0)),\\
\Omega_b&\defeq -2(\kappa_\tau(\chi_0,\tau_0)-h_{\tau}(b(\chi_0,\tau_0);\chi_0,\tau_0)).
\end{split}
\label{eq:frequencies-intro}
\end{equation}
Moreover, $Q(\Delta x,\Delta t)$ is a plane-wave solution of the focusing nonlinear Schr\"odinger equation in the form
\begin{equation}
\ii Q_{\Delta t} + \tfrac{1}{2} Q_{\Delta x\Delta x} + |Q|^2Q=0,
\label{eq:NLS-Deltas}
\end{equation}
and both $p_a(\Delta x,\Delta t)$ and $p_b(\Delta x,\Delta t)$ are particular plane-wave solutions of the formal linearization of \eqref{eq:NLS-Deltas} about $Q(\Delta x,\Delta t)$ written in the frame rotating with the phase of that solution:
\begin{equation}
\ii p_{\Delta t} + \ii\xi_0p_{\Delta x} + \tfrac{1}{2}p_{\Delta x\Delta x} +|\mathcal{A}|^2(p+p^*)=0.
\label{eq:linearization-intro}
\end{equation}
The relative wavenumbers $\xi_a$ and $\xi_b$ also satisfy the inequalities
\begin{equation}
\xi_a^2>4|\mathcal{A}|^2\quad\text{and}\quad \xi_b^2>4|\mathcal{A}|^2.
\label{eq:relative-wavenumber-inequalities}
\end{equation}
\label{cor:shelves-local}
\end{corollary}
Note that in defining the local wavenumbers and frequencies, it makes no difference whether one first evaluates $h(\lambda;\chi,\tau)$ at $\lambda=b(\chi,\tau)$ or $h_-(\lambda;\chi,\tau)$ at $\lambda=a(\chi,\tau)\in\Sigma_g$ and then differentiates with respect to $\chi$ or $\tau$, or the other way around.  This is because $\lambda=a(\chi,\tau)$ and $\lambda=b(\chi,\tau)$ are the roots of the quadratic factor in the  numerator of \eqref{eq:hprime-formula}.

The well-known theory of plane-wave solutions of the focusing nonlinear Schr\"odinger equation 
of arbitrary amplitude $|\mathcal{A}|$ and the formal linearized theory of their perturbations is briefly summarized in Appendix~\ref{A:perturbations}.  A key result of that theory is the existence of an unstable band of relative wavenumbers $\xi$ given by the inequality $\xi^2\le 4|\mathcal{A}|^2$.  It follows from \eqref{eq:relative-wavenumber-inequalities} that the solutions $p_a(\Delta x,\Delta t)$ and $p_b(\Delta x,\Delta t)$ are linearly stable perturbations of the underlying plane wave $Q(\Delta x,\Delta t)$.

The proof of Corollary~\ref{cor:shelves-local} is given in Section~\ref{sec:wave-theoretic-interpretation} below.

\subsubsection{Relations between asymptotic formul\ae\ for $q(M\chi,M\tau;\mathbf{Q}^{-s},M)$ on $\channels$ and $\shelves$}
The asymptotic description of $q(M\chi,M\tau;\mathbf{Q}^{-s},M)$ when $(\chi,\tau)\in\shelves$ given in Theorem~\ref{thm:shelves} is substantially more complicated than for $(\chi,\tau)\in\channels$ (cf., Theorem~\ref{thm:channels}). 
However, comparing \eqref{eq:leading-term-channels-q} and \eqref{eq:subleading-term-shelves-q}, one notices that the part of $\mathfrak{S}_s^{[\shelves]}(\chi,\tau;M)$ written on the first two lines of \eqref{eq:subleading-term-shelves-q} bears a striking resemblance to the leading term $\mathfrak{L}_s^{[\channels]}(\chi,\tau;M)$ valid on the other side of the $\shelves$--$\channels$ boundary curve.  Indeed, $h(\lambda;\chi,\tau)$ degenerates at this curve into the unmodified phase $\vartheta(\lambda;\chi,\tau)$, making the indicated terms match except for the slowly-varying complex factors $m_{a,b}^+(\chi,\tau)\ee^{\ii\delta_{a,b}(\chi,\tau)}$ present within $\shelves$.  Approaching this same curve from $\shelves$, $B(\chi,\tau)\to 0$, so it is also true that the leading term $\mathfrak{L}_s^{[\shelves]}(\chi,\tau;M)$ vanishes in the limit.  However, it is difficult to compare the two asymptotic formul\ae\ quantitatively near the $\shelves$--$\channels$ boundary because $\vartheta''(a(\chi,\tau);\chi,\tau)$ and $h''(a(\chi,\tau);\chi,\tau)$ both vanish as the boundary curve is approached from $\channels$ and from $\shelves$, respectively (we also note that $(\chi,\tau)\mapsto a(\chi,\tau)$ denotes two different real-analytic functions on $\channels$ and $\shelves$ that happen to agree along the common boundary curve).  This makes one of the terms in $\mathfrak{L}^{[\channels]}_s(\chi,\tau;M)$ and two of the terms in $\mathfrak{S}^{[\shelves]}_s(\chi,\tau;M)$ blow up at the boundary curve.  Of course, neither Theorem~\ref{thm:channels} nor Theorem~\ref{thm:shelves} accurately describes $q(M\chi,M\tau;\mathbf{Q}^{-s},M)$ near this curve, so this blow up merely signals the need for further double-scaling asymptotic analysis to resolve the wave field in its vicinity.

\subsubsection{Relations between the asymptotic formula for $\psi_k(M\chi,M\tau)$ on $\exterior$ with those valid on $\channels$ and $\shelves$}
To discuss the region $\exterior$ in light of Theorem~\ref{thm:exterior}, we need to restrict attention to the fundamental rogue-wave solutions $\psi_k(M\chi,M\tau)$ where $M=\tfrac{1}{2}k+\tfrac{1}{4}$.  As the region $\exterior$ abuts both $\channels$ and $\shelves$, it is interesting and useful to compare asymptotic formul\ae\ for $\psi_k(M\chi,M\tau)$ valid on all three regions.  

The simplest observation is that since $B(\chi,\tau)\downarrow 0$ as $(\chi,\tau)$ approaches $\channels$ from anywhere in the exterior,  in particular from $\exterior$, $L^{[\exterior]}_k(\chi,\tau)\to 0$ also in this limit.  This fact is consistent with the fact that $L^{[\channels]}_k(\chi,\tau)$ is small of order $k^{-\frac{1}{2}}$.  However, we note that neither Corollary~\ref{cor:rogue-wave-channels} nor Theorem~\ref{thm:exterior} is valid on a neighborhood of any common boundary point of $\channels$ and $\exterior$.  Like the problem of studying $q(M\chi,M\tau;\mathbf{Q}^{-s},M)$ near the common boundary of $\channels$ and $\shelves$, some new phenomena may be uncovered by a suitable double-scaling analysis to zoom in on points on the curve separating $\exterior$ from $\channels$.  

We can give a more quantitative comparison between the asymptotic formul\ae\ for $\psi_k(M\chi,M\tau)$ on the domains $\exterior$ and $\shelves$.  First, note that the integral in \eqref{eq:gamma-formula-intro} originally defined for $(\chi,\tau)\in\exterior$ admits continuation to $(\mathbb{R}_{\ge 0}\times\mathbb{R}_{\ge 0})\setminus\overline{\channels}$ as a real analytic function, and the latter domain contains also $\shelves$.  Thus for $(\chi,\tau)\in\shelves$, $\gamma(\chi,\tau)$ and $\kappa(\chi,\tau)$ given by \eqref{eq:kappa-formula} can be compared.  Indeed, deforming the integration path $\Sigma_\mathrm{c}$ in \eqref{eq:kappa-formula} leftward to lie partly along the right edge of the branch cut $\Sigma_g$ by its upward orientation, one can replace the resulting integral along $\Sigma_g$ of $1/R_-(\lambda;\chi,\tau)$ by half of the integral of $1/R(\lambda;\chi,\tau)$ over a positively oriented loop enclosing $\Sigma_g$.  Evaluating the latter integral by residues using $R(\lambda;\chi,\tau)=\lambda + O(1)$ as $\lambda\to\infty$ and comparing with \eqref{eq:gamma-formula-intro} one obtains the following identity:
%can be viewed as an interpretation of that in \eqref{eq:kappa-formula} in which the integral over $\Sigma_\mathrm{c}$ is replaced with an average of two integrals over paths from $-\ii$ to $\ii$ passing on opposite sides of $\Sigma_g$.  This interpretation allows us to relate $\kappa(\chi,\tau)$ with $\gamma(\chi,\tau)$ in situations when to compute $\kappa(\chi,\tau)$ it is assumed that $\Sigma_g$ lies to the left of $\Sigma_\mathrm{c}$:
\begin{equation}
\kappa(\chi,\tau)=\gamma(\chi,\tau)-\pi,
\quad (\chi,\tau)\in\shelves.
%\quad\text{$\Sigma_g$ to the left of $\Sigma_\mathrm{c}$ for $\kappa(\chi,\tau)$}.
\label{eq:kappa-gamma}
\end{equation}
We then have the following, which uses the fact that $M=\tfrac{1}{2}k+\tfrac{1}{4}$ and $s=(-1)^k$ for the fundamental rogue wave of order $k$.
\begin{corollary}
The phase and amplitude of the leading term $L_k^{[\exterior]}(\chi,\tau)$ admit real analytic continuation from $\exterior$ into $\shelves$, in which the following identity holds:
\begin{equation}
L_k^{[\shelves]}(\chi,\tau)=\ee^{-2\ii\mu(\chi,\tau)}L_k^{[\exterior]}(\chi,\tau),\quad (\chi,\tau)\in\shelves.
\end{equation}
Therefore, for fundamental rogue waves of high order $k$, the leading terms agree for $(\chi,\tau)\in\exterior$ and for $(\chi,\tau)\in\shelves$, up to a phase $-2\mu(\chi,\tau)$ that vanishes as the common boundary is approached from $\shelves$.
\label{cor:leading-term-match-shelves-exterior}
\end{corollary}
The amplitude $|\psi_k(M\chi,M\tau)|$ is compared with that of the common leading term, namely $B(\chi,\tau)$, on the exterior of $\channels$ in Figure~\ref{fig:AbsLeadingTermShelves}.  
\begin{figure}[h]
\begin{center}
\phantom{!}\hfill\includegraphics[width=0.4\linewidth]{AltScaling-amplitude-regions-plot-orderk-32.pdf}\hfill%
\includegraphics[width=0.4\linewidth]{AbsLeadingTerm.pdf}\hfill\phantom{!}%
\end{center}
\caption{Left:  same as the right-hand panel of Figure~\ref{fig:2D-Plots}.  Right:  amplitude $B(\chi,\tau)>0$ of the leading term as a function of $(\chi,\tau)\in\shelves\cup\exterior$.  Both plots employ the same colormap.}
\label{fig:AbsLeadingTermShelves}
\end{figure}

In terms of derivatives with respect to $(\chi,\tau)$ rather than $(x,t)$, the focusing nonlinear Schr\"odinger equation \eqref{eq:NLS-ZBC} satisfied by $q$ takes the rescaled ``semiclassical'' form:
\begin{equation}
\ii\epsilon q_\tau +\tfrac{1}{2}\epsilon^2q_{\chi\chi} + |q|^2q=0,\quad\epsilon=\frac{1}{M}\ll 1.
\label{eq:NLS-semiclassical}
\end{equation}
To study this equation for small $\epsilon$ it is convenient to introduce in place of $q$ Madelung's real variables $\rho$ and $U$ given by
\begin{equation}
\rho(\chi,\tau)\defeq |q|^2\quad\text{and}\quad U(\chi,\tau)\defeq \epsilon\mathrm{Im}\left(\frac{q_\chi}{q}\right).
\label{eq:Madelung-transform}
\end{equation}
Then, without approximation \eqref{eq:NLS-semiclassical} can be written in the form
\begin{equation}
\rho_\tau+(\rho U)_\chi=0\quad\text{and}\quad U_\tau + \left(\tfrac{1}{2}U^2-\rho\right)_\chi = \tfrac{1}{2}\epsilon^2 F[\rho]_{\chi\chi},\quad\text{where}\quad
F[\rho]\defeq\frac{\rho_{\chi\chi}}{2\rho}-\left(\frac{\rho_\chi}{2\rho}\right)^2.
\label{eq:Madelung-exact}
\end{equation}
Assuming that $\rho\neq 0$, it may appear reasonable to neglect the formally small term in \eqref{eq:Madelung-exact} and hence obtain the approximating system
\begin{equation}
\rho_\tau+(\rho U)_\chi=0\quad\text{and}\quad U_\tau + \left(\tfrac{1}{2}U^2-\rho\right)_\chi = 0.
\label{eq:dispersionless-NLS}
\end{equation}
This is an elliptic quasilinear system on $\rho$ and $U$ known as the (focusing) \emph{dispersionless nonlinear Schr\"odinger system}.  Now, observing that $\rho$ and $U$ defined by \eqref{eq:Madelung-transform} are invariant under $q\mapsto \psi\defeq \ee^{-\ii\tau/\epsilon}q$, we may apply these definitions to the leading terms $L_k^{[\exterior]}(\chi,\tau)$ and $L_k^{[\shelves]}(\chi,\tau)$ of $\psi_k$ on $\exterior$ and $\shelves$, respectively.  Up to a correction term in $U$ proportional to  $\epsilon$ that is only present when $(\chi,\tau)\in\shelves$ (originating from the phase correction $-2\mu(\chi,\tau)$), the formul\ae\ in both regions read
\begin{equation}
\rho(\chi,\tau)=B(\chi,\tau)^2\quad\text{and}\quad U(\chi,\tau)=-2\gamma_\chi(\chi,\tau).
\label{eq:Madelung-transform-on-leading-term}
\end{equation}
\begin{corollary}
The expressions \eqref{eq:Madelung-transform-on-leading-term} satisfy the dispersionless nonlinear Schr\"odinger system \eqref{eq:dispersionless-NLS} for $(\chi,\tau)\in(\mathbb{R}_{\ge 0}\times\mathbb{R}_{\ge 0})\setminus\overline{\channels}$, i.e., for $(\chi,\tau)\in\exterior$ or $(\chi,\tau)\in\shelves$, or on the common boundary curve.
\label{cor:Whitham}
\end{corollary}
Note that the elliptic nature of \eqref{eq:dispersionless-NLS} is consistent with the real analyticity of $A(\chi,\tau)$ and $B(\chi,\tau)$ (and, via \eqref{eq:gamma-formula-intro}, $\gamma(\chi,\tau)$).  The dispersionless nonlinear Schr\"odinger system \eqref{eq:dispersionless-NLS} is also sometimes called the genus-zero Whitham modulation system.  In the Whitham modulation theory it arises from an ansatz of a solution of \eqref{eq:NLS-semiclassical} in the form of a modulated plane wave.  The proof of Corollary~\ref{cor:Whitham} relies on the identity $\gamma_\chi(\chi,\tau)=\kappa_\chi(\chi,\tau)=A(\chi,\tau)$ that is established in Lemma~\ref{lemma:g-derivatives}, and is given in Remark~\ref{rem:Whitham} below.

\begin{remark}
A version of Corollary~\ref{cor:Whitham} holds more generally for $\rho$ and $U$ extracted from the leading term $\mathfrak{L}_s^{[\shelves]}(\chi,\tau;M)$ valid as an approximation for $q(M\chi,M\tau;\mathbf{Q}^{-s},M)$ for general large $M$ when $(\chi,\tau)\in\shelves$.  One need only replace $\gamma_\chi(\chi,\tau)$ with $\kappa_\chi(\chi,\tau)$ in the definition \eqref{eq:Madelung-transform-on-leading-term}.  As such it also holds for the high-order multiple-pole soliton solutions studied in \cite{BilmanBW19}.
\end{remark}

\subsection{The behavior of $q(M\chi,M\tau;\mathbf{Q}^{-s},M)$ for $(\chi,\tau)\in\exterior$ for general large $M$}
\label{sec:general}
It is shown in \cite{BilmanBW19} that in the unbounded domain $\exterior$, high-order multiple-pole solitons behave quite differently from high-order fundamental rogue waves as reported in Theorem~\ref{thm:exterior}.  For the soliton solutions, the domain $\exterior$ is divided into two components by the curve $\ell_\mathrm{sol}$ described by \eqref{eq:DegenerateBoutroux} and shown with a dotted black line in Figure~\ref{fig:RegionsPlot}.  On the component adjacent to the positive $\chi$-axis, the solution is exponentially small when $M=\tfrac{1}{2}k\to+\infty$.  This is consistent with the exponential decay of $q(x,t;\mathbf{Q}^{-s},\tfrac{1}{2}k)$ as $x\to \pm\infty$ for fixed $t$, although for technical reasons the proof given in \cite{BilmanBW19} does not allow $(\chi,\tau)$ to become unbounded.  On the complementary component adjacent to the positive $\tau$-axis, the solution behaves completely differently.  Here $q(\tfrac{1}{2}k\chi,\tfrac{1}{2}k\tau;\mathbf{Q}^{-s},\tfrac{1}{2}k)$ is approximated by a modulated elliptic function of amplitude asymptotically independent of $M=\tfrac{1}{2}k$.  The elliptic modulus $m=m(\chi,\tau)$ approaches $m=0$ as $(\chi,\tau)$ approaches the common boundary with $\shelves$, while it approaches $m=1$ instead as $(\chi,\tau)$ approaches the curve $\ell_\mathrm{sol}$.  In the former limit the elliptic wave degenerates onto the trigonometric plane-wave leading term given in all cases of $M\to\infty$ in Theorem~\ref{thm:shelves}, and in the latter limit the elliptic wave degenerates into a train of isolated solitons (which explains our notation $\ell_\mathrm{sol}$).   

When we consider solutions $q(x,t;\mathbf{Q}^{-s},M)$ that do not fit into either family, we see both common features and substantial differences comparing with the special cases of solitons and rogue waves.  The way to take the limit $M\to\infty$ in this situation is to represent $M$ in modular form as $M=\tfrac{1}{2}k+r$ with quotient $k\in\mathbb{Z}_{\ge 0}$ and remainder $0\le r<\tfrac{1}{2}$.  Then we fix the remainder and let $k\to+\infty$.  Of course the soliton case is $r=0$, and the rogue wave case is $r=\tfrac{1}{4}$.  When $r\neq 0$ and $r\neq\tfrac{1}{4}$, the large-$M$ asymptotic behavior of $q(M\chi,M\tau;\mathbf{Q}^{-s},M)$ for bounded $(\chi,\tau)$ in $\exterior$ depends on whether $(\chi,\tau)\in\exterior_\chi$ or $(\chi,\tau)\in\exterior_\tau$ (see Figure~\ref{fig:RegionsPlot}).  Because we think it will be interesting for the reader, in the following paragraphs we describe what we have learned about these solutions; however full details and proofs will be given in a subsequent paper devoted to the case of general $M\ge 0$.

If $(\chi,\tau)\in\exterior_\chi$, then a version of Theorem~\ref{thm:exterior} applies to $\ee^{-\ii M\tau}q(M\chi,M\tau;\mathbf{Q}^{-s},M)$ for $r\neq 0$ and $r\neq\tfrac{1}{4}$, in which the leading term is multiplied by an $M$-independent but $(\chi,\tau)$-dependent phase factor, and in which the error term is larger, of size $O(M^{-\frac{1}{2}})$.  The sub-leading term proportional to $M^{-\frac{1}{2}}$ is simpler than for $(\chi,\tau)\in\shelves$, consisting of only one of the two waves present for instance in \eqref{eq:p-a-b-shelves}; this means that the amplitude fluctuations will form a stripe pattern rather than an interference pattern such as occurs in $\shelves$.  The approximation of $\ee^{-\ii M\tau}q(M\chi,M\tau;\mathbf{Q}^{-s},M)$ tends to the background plane wave $\psi=\pm 1$ as $(\chi,\tau)\to\infty$ in $\exterior_\chi$, which is consistent with the exact boundary conditions satisfied by the fundamental rogue-wave solutions occurring for $r=\tfrac{1}{4}$; however the proof we have in mind of this result is not valid for technical reasons when $(\chi,\tau)$ become unbounded.  Nonetheless, it follows from a different proof that, like the rogue-wave solutions, all solutions $q(x,t;\mathbf{Q}^{-s},M)$ for remainder $r\neq 0$ satisfy nonzero boundary conditions with unit limiting amplitude as $x\to\pm\infty$; however for $r\neq\tfrac{1}{4}$ the decay is so slow that the difference between $q$ and the background does not even lie in $L^2(\mathbb{R})$.

On the other hand, if $(\chi,\tau)\in\exterior_\tau$, then as $M\to\infty$ with $r\neq 0$ and $r\neq\tfrac{1}{4}$ fixed, $q(M\chi,M\tau;\mathbf{Q}^{-s},M)$ is approximated by a modulated elliptic function of amplitude neither small nor large.  In the part of $\exterior_\tau$ above (i.e., for larger $\tau$) the curve $\ell_\mathrm{sol}$, the leading term of the approximation differs from that valid in the same region for the multiple-pole soliton case of $r=0$ only in phase modifications that are independent of $M\gg 1$.  However the error term is of order $O(M^{-\frac{1}{2}})$ rather than $O(M^{-1})$.  In the part of $\exterior_\tau$ lying below the curve $\ell_\mathrm{sol}$, the solution $q(M\chi,M\tau;\mathbf{Q}^{-s},M)$ evidently behaves neither like the rogue-wave solutions for $r=\tfrac{1}{4}$ (approximated by modulated plane waves) nor like the multiple-pole soliton solutions for $r=0$ (exponentially small).  The elliptic modulus varies with $(\chi,\tau)$ from $m=1$ on the curve $\ell_\mathrm{sol}$ to $m=0$ on the curve $\partial\exterior_\chi\cap\partial\exterior_\tau$ (the blue dotted curve in Figure~\ref{fig:RegionsPlot}).  

The asymptotic description of the solution $q(M\chi,M\tau;\mathbf{Q}^{-s},M)$ for remainder $r\neq 0$ and $r\neq\tfrac{1}{4}$ and $(\chi,\tau)\in\exterior$ is consistent with the universal long-time asymptotics for solutions of the focusing nonlinear Schr\"odinger equation with nonzero boundary conditions at $x=\pm\infty$ established by Biondini and Mantzavinos \cite{BiondiniM17}.  These authors showed that for a wide variety of initial conditions, the solution depends asymptotically only on the ratio $\xi\defeq x/t=\chi/\tau$, and as a function of $\xi$ is approximated for $|\xi|<\sqrt{8}$ (translating to our scaling of the equation from theirs) by a modulated elliptic function solution with elliptic modulus $m(\xi)$ varying between $m(0)=1$ and $m(\sqrt{8})=0$, and approximated for $|\xi|>\sqrt{8}$ by a plane-wave solution of constant amplitude equal to that specified by the large-$x$ boundary conditions.  This is consistent with our description of $q(M\chi,M\tau;\mathbf{Q}^{-s},M)$ for remainder $r\neq 0$ and $r\neq\tfrac{1}{4}$ because 
\begin{itemize}
\item
the condition $\chi/\tau=\sqrt{8}$ is precisely the linear asymptote valid for large $\tau$ (dotted gray line in Figure~\ref{fig:RegionsPlot}) for the curve $\partial\exterior_\chi\cap\partial\exterior_\tau$ (dotted blue curve in Figure~\ref{fig:RegionsPlot}), and 
\item
the condition $m(0)=1$ is consistent with $m(\chi,\tau)\to 1$ as $(\chi,\tau)\to\ell_\mathrm{sol}$ because the latter curve, while not asymptotic to any line for large $\tau$, satisfies 
$\chi=\ln(\tau)+O(1)$ as $\tau\to +\infty$.  Hence the whole region above the curve $\ell_\mathrm{sol}$ in Figure~\ref{fig:RegionsPlot} can be found to the left of $\xi=\xi_0$ for any $\xi_0>0$, asymptotically in the large-$\tau$ limit.
\end{itemize}
On the other hand, the class of solutions considered in \cite{BiondiniM17} does not contain $q(x,t;\mathbf{Q}^{-s},M)$ for remainder $r\neq 0$ and large $M$ because increasing $M$ by half-integer increments amounts to iteration of a Darboux transformation \cite{BilmanM19} that injects solitons/rogue waves into the solution at the distinguished value of the spectral parameter corresponding to the nonzero background solution (here, $\lambda=\pm\ii$).  The slow decay to the background as $x\to\pm\infty$ for  $r\neq 0,\tfrac{1}{4}$ also obstructs analysis by inverse-scattering methods.  There are some extensions of the results of \cite{BiondiniM17} that allow for finitely many solitons with generic spectral parameters but no results for the case that the injected solitons are at the distinguished value.  It is also true that, as has been mentioned several times already, it is not possible to directly compare large-$(x,t)$ asymptotics with large-$M$ and bounded $(\chi,\tau)$ asymptotics without additional arguments that are not part of our proofs.

The reason why the solution $q(M\chi,M\tau;\mathbf{Q}^{-s},M)$ is so sensitive to the value of the remainder $r$ when $(\chi,\tau)\in\exterior$ is that in this domain we need to use the limiting form of the jump contour $\Sigma_\circ$ in which it is deformed into a dumbbell shape consisting of two loops connected by a ``neck'' that we denote by $N$ in Section~\ref{sec:dumbbell}.  When $\lambda\in N$, the algebraic form of the jump condition for this deformed problem depends explicitly on $r$; see Remark~\ref{rem:M-quantum} below.  In particular, for the cases $r=0$ and $r=\tfrac{1}{4}$ the jump matrix has two elements that vanish exactly, which prohibits the use of two of the four canonical factorizations of unit-determinant $2\times 2$ matrices:
\begin{equation}
\begin{alignedat}{3}
\begin{bmatrix} a&b\\c&d\end{bmatrix}&=\begin{bmatrix}1 & 0\\ca^{-1} & 1\end{bmatrix}a^{\sigma_3}\begin{bmatrix}1 & ba^{-1}\\0 & 1\end{bmatrix},&&\quad a\neq 0, &&\quad\text{(``LDU'')},\\
\begin{bmatrix} a&b\\c&d\end{bmatrix}&=\begin{bmatrix}1 & 0\\db^{-1}&1\end{bmatrix}\begin{bmatrix}0&b\\-b^{-1} & 0\end{bmatrix}\begin{bmatrix}1&0\\ab^{-1}&1\end{bmatrix},&&\quad b\neq 0,&&\quad\text{(``LTL'')},\\
\begin{bmatrix} a&b\\c&d\end{bmatrix}&=\begin{bmatrix}1 & ac^{-1}\\0 & 1\end{bmatrix}
\begin{bmatrix}0 & -c^{-1}\\c & 0\end{bmatrix}\begin{bmatrix}1 & dc^{-1}\\0 & 1\end{bmatrix},&&\quad c\neq 0,&&\quad\text{(``UTU'')},\\
\begin{bmatrix} a&b\\c&d\end{bmatrix}&=\begin{bmatrix}1 & bd^{-1}\\0 & 1\end{bmatrix}d^{-\sigma_3}
\begin{bmatrix}1 & 0\\cd^{-1}&1\end{bmatrix},&&\quad d\neq 0,&&\quad\text{(``UDL'')}.
\end{alignedat}
\end{equation}
It turns out that for $r=0$ (multiple-pole soliton case) only the LDU and UDL factorizations are possible and they are both trivial as the jump matrix on $N$ is already diagonal.  Similarly for $r=\tfrac{1}{4}$ (rogue wave case) only the LTL and UTU factorizations are possible and they are both trivial as the jump matrix on $N$ is already off-diagonal.  On the other hand, for $r\neq 0$ and $r\neq \tfrac{1}{4}$ there are four nonzero pivots and hence all four factorizations are admissible; moreover all four are essential to the steepest-descent arguments behind the proofs.

As they do not concern rogue waves and require substantially different proofs, all of the results reported in Section~\ref{sec:general} describing $q(M\chi,M\tau;\mathbf{Q}^{-s},M)$ for large $M$ with remainder $r\neq 0$ and $r\neq \tfrac{1}{4}$ and $(\chi,\tau)\in\exterior$ will be given in more detail and fully proven in a forthcoming paper.


\subsection{Numerical illustration of the results}
%\input{NumericalVerification}

We now illustrate the accuracy of the asymptotic formul\ae{} obtained for the fundamental rogue waves $\psi_k(M \chi, M \tau)$, $M=\tfrac{1}{2}k+\tfrac{1}{4}$, for $(\chi,\tau)$ in the regions $\channels$, $\exterior$, and $\shelves$. %We present two numerical verifications.
%To verify the the accuracy of the asymptotic formul\ae{} \eqref{eq:psi-k-channels} for $(\chi,\tau)\in \channels$, \eqref{eq:psi-k-shelves-chi-tau} for $(\chi,\tau)\in \exterior$, and \eqref{eq:psi-k-exterior} for $(\chi,\tau) \in \shelves$ obtained for $\psi_k(n \chi, n \tau)$ in the preceding sections, we present two numerical comparisons. 
In each subsubsection that follows, we first plot the exact solution $\psi_k$ against the approximation provided by the asymptotic formul\ae{} for certain values of $k$. Second, we study the trends in the relevant error sizes by comparing the approximations with the family of exact solutions $\psi_k$ as the value of $k$ increases over a set of positive integers $\mathcal{K}$.
The solutions $\psi_{k}$ are computed by numerically solving linear systems system obtained from their representations given by \rhref{rhp:rogue-wave}. We refer the reader to \cite[Section 3.5]{BilmanM19} for the derivation of the linear system used in this work. 
%Finally we recall \eqref{eq:k-vs-n} for the relationship between $k$ and $n$.


\subsubsection{Numerical illustration of the asymptotic formula for $\psi_k(M\chi,M\tau)$ in $\channels$} 
Here we give numerical evidence confirming Corollary~\ref{cor:rogue-wave-channels}.  Recall the leading term approximation of $\psi_k(M\chi,M\tau)$ denoted $L_k^{[\channels]}(\chi,\tau)$ and defined in \eqref{eq:leading-term-channels}.
%To begin, we recall the leading term $L_k^{[\channels]}(\chi,\tau)$ given in \eqref{eq:leading-term-channels} and recall that $\psi_k(M\chi,M\tau) = L_k^{[\channels]}(\chi,\tau) + O(k^{-\frac{3}{2}})$ as $k\to +\infty$.
%To begin, we denote by $L_n^{[\channels]}(\chi,\tau)$ the leading term given in \eqref{eq:psi-k-channels}, namely,
%\begin{multline}
%L_n^{[\channels]}(\chi,\tau) :=
%\ee^{-\ii n\tau}\frac{(-1)^k}{n^\frac{1}{2}}\sqrt{\frac{\ln(2)}{\pi}}
%\left[
%\ee^{\ii\phi}
%\frac{\ee^{-2\ii n\vartheta_a(\chi,\tau)}\omega(a(\chi,\tau))^{2s}(-\vartheta''_a(\chi,\tau))^{-\ii p}}{(-\vartheta''_a(\chi,\tau))^\frac{1}{2}}\right.\\
%\left.{}+\ee^{-\ii\phi}
%\frac{\ee^{-2\ii n\vartheta_b(\chi,\tau)}\omega(b(\chi,\tau))^{2s}\vartheta''_b(\chi,\tau)^{\ii p}}{\vartheta''_b(\chi,\tau)^\frac{1}{2}}\right],
%\end{multline}
%so that the formula \eqref{eq:psi-k-channels} takes the form
%\begin{equation}
%\psi_{k}(n \chi, n\tau) = L_n^{[\channels]}(\chi,\tau) + O(n^{-1}),\quad n\to +\infty.
%\end{equation}
\paragraph{\textit{Comparison plots}} We first consider $\tau = 0$, in which case $\channels$ comprises the open interval $0<\chi<2$, and we fix the proper subset $[0.25,1.75]$ of values for $\chi$. We plot $\psi_{k}(M \chi, M\tau)$ versus $L_k^{[\channels]}(\chi,\tau)$ for $k=15$ and for $k=32$ in Figure~\ref{f:comparison-channels-tau-0-k-15-32}. Note that the solution is real-valued for $\tau=0$ hence no plots for the imaginary parts are given.
%fix the range $\chi\in[0.25,1.75]$ which is a proper subset of the maximal range $(0,2)$, and plot $\psi_{k}(n \chi, n\tau)$ versus\@ $L_n^{[\channels]}(\chi,\tau)$ for the two cases $k=15$ (odd, $s=-1$ and $n=8$) and $k=32$ (even, $s=1$ and $n=16$)
%in Figure~\ref{f:comparison-channels-k-15-32}. Note that the solution is real-valued for $\tau=0$ hence no plots for the imaginary parts are given.
\begin{figure}[h]
\includegraphics[width=0.45\textwidth]{AltScaling-Ch-Compare-tau0-orderk-15}\qquad
\includegraphics[width=0.45\textwidth]{AltScaling-Ch-Compare-tau0-orderk-32}
\caption{Comparison of the exact solution $\psi_k(M \chi, M\tau)$ (dots) with the approximation $L_k^{[\channels]}(\chi,\tau)$ (solid curve) at $\tau=0$ for $0.25\leq \chi \leq 1.75$. Left panel: $k=15$, right panel: $k=32$. The solution is real-valued for $\tau=0$.}
\label{f:comparison-channels-tau-0-k-15-32}
\end{figure}
Next, we set $\tau=0.125$ and consider the range $1 \leq \chi \leq 1.75$ which lies inside the region $\channels$. Figure~\ref{f:comparison-channels-tau-0p125-k-31} presents plots of the real and imaginary parts of $\psi_{k}(M \chi, M\tau)$ with those of $L_k^{[\channels]}(\chi,\tau)$ for $k=31$ and Figure~\ref{f:comparison-channels-tau-0p125-k-32} presents the same comparisons for $k=32$.
 \begin{figure}[h]
\includegraphics[width=0.45\textwidth]{AltScaling-Ch-Compare-tau0p125-Re-orderk-31}\qquad
\includegraphics[width=0.45\textwidth]{AltScaling-Ch-Compare-tau0p125-Im-orderk-31}
\caption{Comparison of the exact solution $\psi_k(M \chi, M\tau)$ (dots) with the approximation $L_k^{[\channels]}(\chi,\tau)$ (solid curve) for $k=31$, at $\tau=0.125$ for $1\leq \chi \leq 1.75$. Left panel: real parts, right panel: imaginary parts.}
\label{f:comparison-channels-tau-0p125-k-31}
\end{figure}
 \begin{figure}[h]
\includegraphics[width=0.45\textwidth]{AltScaling-Ch-Compare-tau0p125-Re-orderk-32}\qquad
\includegraphics[width=0.45\textwidth]{AltScaling-Ch-Compare-tau0p125-Im-orderk-32}
\caption{Comparison of the exact solution $\psi_k(M \chi, M\tau)$ (dots) with the approximation $L_k^{[\channels]}(\chi,\tau)$ (solid curve) for $k=32$, at $\tau=0.125$ for $1 \leq \chi \leq 1.75$. Left panel: real parts, right panel: imaginary parts.}
\label{f:comparison-channels-tau-0p125-k-32}
\end{figure}

\paragraph{\textit{Error plots}} To validate the size of the error term $O(k^{-\frac{3}{2}})$ predicted in Corollary~\ref{cor:rogue-wave-channels}, we fix $\tau=0.125$ and the range $1\leq \chi \leq 1.75$ for the values of $\chi$. We construct a grid $\mathcal{G}_k$ on this interval with step size
%\footnote{This choice results in a fixed grid-spacing size on the $x$-axis.}
$\delta\chi = (4M)^{-1}$ starting at the left endpoint $\chi=1$. 
%\lceil 4M(\chi_\mathrm{max} - \chi_\mathrm{min}) \rceil$ equally spaced points.
%We fix the proper subset defined by the inequality $\chi_\mathrm{min}\defeq 0.25 \leq \chi \leq 0.75 \eqdef \chi_\mathrm{max}$ and construct the grid $\mathcal{G}_k$ on this interval with $\lceil 4M(\chi_\mathrm{max} - \chi_\mathrm{min}) \rceil$ equally spaced points.
We then compute the absolute errors made in approximating the fundamental rogue waves $\psi_k(M\chi,M\tau)$ with the leading terms $L_k^{[\channels]}(\chi,\tau)$ measured in the sup-norm over the grid $\mathcal{G}_k$, for $k$ ranging over the set $\mathcal{K} := \{16, 32, 48, 64, 80, 96 \}$:
\begin{equation}
E_k^{[\channels]}\defeq \sup_{\chi \in \mathcal{G}_k} \left| \psi_{k}(M\chi,M\tau) - L_k^{[\channels]}(\chi,\tau) \right|, \quad k\in\mathcal{K},\quad \tau=0.125.
\end{equation}
We plot $\ln(E_k^{[\channels]})$ versus $\ln(k)$ in Figure~\ref{f:channels-errors} and perform linear regression, which yields the best-fit line $\ln(E_k^{[\channels]}) =0.518758 - 1.48135 \ln(k)$ with $R$-squared value of $0.998502$. The slope of this line recovers approximately the exponent $-\tfrac{3}{2}$ in the error term predicted in Corollary~\ref{cor:rogue-wave-channels}.
\begin{figure}[h]
\includegraphics[width=0.45\textwidth]{AltScaling-Ch-Error}
\caption{Plot of $\ln(E_k^{[\channels]})$ versus $\ln(k)$ (diamond markers) for $k\in\mathcal{K}$. The dashed line is the best-fit line for the plotted data set.}
\label{f:channels-errors}
\end{figure}



\subsubsection{Numerical illustration of the asymptotic formula for $\psi_k(M\chi,M\tau)$ in $\exterior$} 
Next, we give numerical evidence confirming Theorem~\ref{thm:exterior}.  Recall the leading term approximation of $\psi_k(M\chi,M\tau)$ denoted $L_k^{[\exterior]}(\chi,\tau)$ and defined in \eqref{eq:leading-term-exterior}.
%
%In analogy with the previous subsection, we recall the leading term $L_k^{\exterior}(\chi,\tau)$ given in \eqref{eq:leading-term-exterior} and recall that $\psi_k(M\chi,M\tau) = L_k^{[\exterior]}(\chi,\tau) + O(k^{-1})$ as $k\to +\infty$.
\paragraph{\textit{Comparison plots}} We fix $\tau = 1.25>1$ and consider the range $0\leq \chi \leq 4$, which contains points $(\chi,\tau)$ from $\exterior_\chi$ and from $\exterior_\tau$, and plot the real and imaginary parts of $\psi_k(M\chi,M \tau)$ and its approximation $L^{[\exterior]}_k(\chi,\tau)$ for $k=16$ in Figure~\ref{f:comparison-exterior-k-16} and for $k=31$ in Figure~\ref{f:comparison-exterior-k-31}.
\begin{figure}[h]
\includegraphics[width=0.45\textwidth]{AltScaling-Ext-Compare-tau1p25-Re-orderk-16}\qquad
\includegraphics[width=0.45\textwidth]{AltScaling-Ext-Compare-tau1p25-Im-orderk-16}
\caption{Comparison of the exact solution $\psi_k(M \chi, M\tau)$ (dots) with the approximation $L_k^{[\exterior]}(\chi,\tau)$ (solid curve) for $k=16$, at $\tau=1.25$ for $0\leq \chi \leq 4$. Left panel: real parts, right panel: imaginary parts.}
\label{f:comparison-exterior-k-16}
\end{figure}
\begin{figure}[h]
\includegraphics[width=0.45\textwidth]{AltScaling-Ext-Compare-tau1p25-Re-orderk-31}\qquad
\includegraphics[width=0.45\textwidth]{AltScaling-Ext-Compare-tau1p25-Im-orderk-31}
\caption{Comparison of the exact solution $\psi_k(M \chi, M\tau)$ (dots) with the approximation $L_k^{[\exterior]}(\chi,\tau)$ (solid curve) for $k=31$, at $\tau=1.25$ for $0\leq \chi \leq 4$. Left panel: real parts, right panel: imaginary parts.}
\label{f:comparison-exterior-k-31}
\end{figure}
%
\paragraph{\textit{Error plots}} We now validate the size of the error term
$O(k^{-1})$ predicted in Theorem~\ref{thm:exterior}. We fix $\tau=1.25$ and the range $3\leq \chi \leq 4$ for the values of $\chi$, and again construct a grid $\mathcal{G}_k$ on this interval with step size
%\footnote{This choice results in a fixed grid-spacing size on the $x$-axis.}
$\delta\chi = (4M)^{-1}$ starting at the left endpoint $\chi=3$. 
%\lceil 4M(\chi_\mathrm{max} - \chi_\mathrm{min}) \rceil$ equally spaced points.
%We fix the proper subset defined by the inequality $\chi_\mathrm{min}\defeq 0.25 \leq \chi \leq 0.75 \eqdef \chi_\mathrm{max}$ and construct the grid $\mathcal{G}_k$ on this interval with $\lceil 4M(\chi_\mathrm{max} - \chi_\mathrm{min}) \rceil$ equally spaced points.
We then compute the absolute errors made in approximating the fundamental rogue waves $\psi_k(M\chi,M\tau)$ with the leading terms $L_k^{[\exterior]}(\chi,\tau)$ measured in the sup-norm over the grid $\mathcal{G}_k$, for $k$ ranging over the set $\mathcal{K} := \{16, 32, 48, 64, 80, 96 \}$:
\begin{equation}
E_k^{[\exterior]}:= \sup_{\chi \in \mathcal{G}_k} \left| \psi_{k}(M\chi,M\tau) - L_k^{[\exterior]}(\chi,\tau) \right|, \quad k\in\mathcal{K},\quad \tau=1.25.
\end{equation}
We plot $\ln(E_k^{[\exterior]})$ versus $\ln(k)$ in Figure~\ref{f:exterior-errors} and perform linear regression, which yields the best-fit line $\ln(E_k^{[\exterior]}) =-1.98314 - 0.927044 \ln(k)$ with $R$-squared value of $0.999274$. The slope of this line recovers approximately the exponent $-1$ in the error predicted in Theorem~\ref{thm:exterior}.
\begin{figure}[h]
\includegraphics[width=0.45\textwidth]{AltScaling-Ext-Error}
\caption{Plot of $\ln(E_k^{[\exterior]})$ versus $\ln(k)$ (diamond markers) for $k\in\mathcal{K}$. The dashed line is the best-fit line for the plotted data set.}
\label{f:exterior-errors}
\end{figure}
\subsubsection{Numerical illustration of the asymptotic formula for $\psi_k(M\chi,M\tau)$ in $\shelves$} 
Finally, we turn to the illustration of Corollary~\ref{cor:rogue-wave-shelves}, recalling the approximate formula $L_k^{[\shelves]}(\chi,\tau)+S_k^{[\shelves]}(\chi,\tau)$ defined in \eqref{eq:rw-terms-shelves}.

\paragraph{\textit{Comparison plots}} We fix $\tau = 0.5$ for which $\shelves$ consists of the interval $(0,2)$ for the values of $\chi$. We consider the range $0.25\leq\chi\leq1.75$ which is a proper subset of $(0,2)$ and plot the real and imaginary parts of $\psi_k(M\chi, M\tau)$ and of
$ L_k^{[\shelves]}(\chi,\tau) + S_k^{[\shelves]}(\chi,\tau)$ for $k=23$ in Figure~\ref{f:comparison-bun-k-23} and for $k=32$ in Figure~\ref{f:comparison-bun-k-32}.
\begin{figure}[h]
\includegraphics[width=0.45\textwidth]{AltScaling-Bun-Compare-tau0p5-Re-orderk-23}\qquad
\includegraphics[width=0.45\textwidth]{AltScaling-Bun-Compare-tau0p5-Im-orderk-23}
\caption{Comparison of the exact solution $\psi_k(M \chi, M\tau)$ (dots) with the approximation $L_k^{[\shelves]}(\chi,\tau)+S_k^{[\shelves]}(\chi,\tau)$ (solid curve) for $k=23$, at $\tau=0.5$ for $0.25< \chi < 1.75$. Left panel: real parts, right panel: imaginary parts.}
\label{f:comparison-bun-k-23}
\end{figure}
\begin{figure}[h]
\includegraphics[width=0.45\textwidth]{AltScaling-Bun-Compare-tau0p5-Re-orderk-32}\qquad
\includegraphics[width=0.45\textwidth]{AltScaling-Bun-Compare-tau0p5-Im-orderk-32}
\caption{Comparison of the exact solution $\psi_k(M \chi, M\tau)$ (dots) with the approximation $L_k^{[\shelves]}(\chi,\tau)+S_k^{[\shelves]}(\chi,\tau)$ (solid curve) for $k=32$, at $\tau=0.5$ for $0.25< \chi < 1.75$. Left panel: real parts, right panel: imaginary parts.}
\label{f:comparison-bun-k-32}
\end{figure}

\paragraph{\textit{Error plots}} To validate the size of the error term predicted in Corollary~\ref{cor:rogue-wave-shelves}, we fix $\tau=0.5$, consider the range $0.5 \leq \chi \leq 1.5$ corresponding to a line segment contained in $\shelves$, and construct a grid $\mathcal{G}_k$ on this interval with step size $\delta \chi = (4M)^{-1}$ starting at the left endpoint $\chi=0.5$. 
We then again compute the absolute errors made in approximating the fundamental rogue waves  $\psi_k(M\chi,M\tau)$ with $L_k^{[\shelves]}(\chi,\tau) + S_k^{[\shelves]}(\chi,\tau)$ measured in the sup-norm over the grid $\mathcal{G}_k$, for $k$ ranging over the set $\mathcal{K} := \{16, 32, 48, 64, 80, 96 \}$:
\begin{equation}
E_k^{[\shelves]}:= \sup_{\chi \in \mathcal{G}_k} \left| \psi_{k}(M\chi,M\tau) - \left( L_k^{[\exterior]}(\chi,\tau) + S_k^{[\shelves]}(\chi,\tau) \right)\right|, \quad k\in\mathcal{K},\quad \tau=0.5.
\end{equation}
We plot $\ln(E_k^{[\shelves]})$ versus $\ln(k)$ in Figure~\ref{f:shelves-errors} and perform linear regression, which yields the best-fit line $\ln(E_k^{[\shelves]}) =-0.366203 - 1.01873 \ln(k)$ with $R$-squared value of $0.995771$. The slope of this line recovers approximately the exponent $-1$ in the error predicted in Corollary~\ref{cor:rogue-wave-shelves}.
\begin{figure}[h]
\includegraphics[width=0.45\textwidth]{AltScaling-Bun-Error}
\caption{Plot of $\ln(E_k^{[\shelves]})$ versus $\ln(k)$ (diamond markers) for $k\in\mathcal{K}$. The dashed line is the best-fit line for the plotted data set.}
\label{f:shelves-errors}
\end{figure}

\paragraph{\textit{Interference pattern}} We now illustrate how Corollary~\ref{cor:pattern} accurately predicts the complicated wave pattern seen in the plots in Figure~\ref{fig:2D-Plots} for $(\chi,\tau)\in\shelves$.  
Although Corollary~\ref{cor:pattern} applies to more general solutions, in keeping with the setting we restrict attention to special case of the fundamental rogue-wave solutions.
%the calculation \eqref{eq:interference} in Section~\ref{sec:interference} explaining the mechanism that produces the interference pattern observed in Figure 2 in \cite{BilmanLM20} away from the peak of the wave field which occurs at $(x,t)=(0,0)$. The region in which this field emerges coincides with $\shelves$ in this work. Recalling that $f_a'(a(\chi,\tau))<0$ and $f_b'(b(\chi,\tau))>0$, and taking into account the overall minus sign in the term proportional to $M^{-1}$ in the expansion \eqref{eq:interference} for $|\psi_k(M \chi, M\tau)|^2$, we see that the amplitude of $\sin(\Theta_a(\chi,\tau;M))$ is positive and the amplitude of $\sin(\Theta_b(\chi,\tau;M))$ is negative. 
Thus, we compute the following $M$-dependent unions of level curves inside $\shelves$ in the $(\chi,\tau)$-plane:
\begin{align}
\mathcal{M}_a &=\bigcup_{j\in\mathbb{Z}} \{ (\chi,\tau)\in \shelves \colon \phi_a(\chi,\tau;M) = - \frac{\pi}{2} + 2\pi j \},\\
\mathcal{M}_b &=\bigcup_{j\in\mathbb{Z}} \{ (\chi,\tau)\in \shelves \colon \phi_b(\chi,\tau;M) = - \frac{\pi}{2} + 2\pi j \},
\end{align}
so that $\sin(\phi_a(\chi,\tau;M)) =  -1$ for $(\chi,\tau)\in \mathcal{M}_a$ and  $\sin(\phi_b(\chi,\tau;M)) =  -1$ for $(\chi,\tau)\in \mathcal{M}_b$. 
Accordingly, the claim is that the intersection points of $\mathcal{M}_a$ and $\mathcal{M}_b$ locate the amplitude peaks formed by $|\psi_k(M \chi, M\tau)|^2$. To verify that this is the case, we fix the box $[0.2, 1.2]\times[0.2,0.8] \subset \shelves$ and plot $|\psi_k(M \chi, M\tau)|^2$ and the set of points $\mathcal{M}_a \cup \mathcal{M}_b$. Figure~\ref{f:interference-overlays} illustrates this formation as it is described.

\begin{figure}[h]
\includegraphics[width=0.32\textwidth]{AltScaling-bun-overlay-orderk-4}\quad \includegraphics[width=0.32\textwidth]{AltScaling-bun-overlay-orderk-8}
\includegraphics[width=0.32\textwidth]{AltScaling-bun-overlay-orderk-16}
\caption{Monochrome density plot of $|\psi_k(M \chi, M\tau)|^2$ and plots of the unions of level curves $\mathcal{M}_a$ (red contours) and $\mathcal{M}_b$ (cyan contours) for values of $k=4,8,16$. The brighter colors correspond to larger amplitude, and hence the white spots are where the peaks are formed. Left panel: $k=4$, center panel: $k=8$, right panel: $k=16$.}
\label{f:interference-overlays}
\end{figure}

