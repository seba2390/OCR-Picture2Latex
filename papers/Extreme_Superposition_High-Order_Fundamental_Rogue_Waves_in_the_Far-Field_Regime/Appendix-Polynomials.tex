We remind the reader that the discriminant $\Delta_f$ of a polynomial $f(z)= a_n z^n + a_{n-1} z^{n-1} + \cdots + a z+ a_0$, $n\geq 1$, $a_n\neq 0$, with roots (counted with multiplicity) $\xi_1, \xi_2,\dots,\xi_n\in\mathbb{C}$ can be expressed as
\begin{equation}
\Delta_f = a_n^{2n-2}\prod_{1 \leq j < k \leq n}\left(\xi_{j}-\xi_{k}\right)^{2}.
\label{eq:discriminant-roots}
\end{equation}
We assume that the coefficients $a_k$, $k=1,\dots,n$ of the polynomial $f$ are real in the rest of this appendix. In this case, the representation \eqref{eq:discriminant-roots} provides information about the number of non-real roots of $f$. Since the non-real roots of $f$ come in complex conjugate pairs, it is seen from \eqref{eq:discriminant-roots} that $\Delta_f>0$ if and only if $f$ has all distinct real roots or the number of non-real roots are a multiple of $4$. On the other hand, in case $n\geq 2$, $\Delta_f<0$ if and only if the number of non-real roots of $f$ is $2~\mathrm{mod}(4)$. 

The following method is useful for obtaining information about the real roots of a univariate polynomial $f(z)$. We first give a definition (\cite{Sturm1829}, see also \cite[Section 1.3]{Sturmfels02}).
\begin{definition}[Sturm sequence]
Given a polynomial $f(z)$ of degree $n$, define polynomials $f_k(z)$, $k=0,1,2,\dots$ by
\begin{equation}
\begin{split}
f_0(z) &:= f(z),\\
f_1(z) &:= f'(z),\\
f_k(z) &:= -\rem(f_{k-2}(z), f_{k-1}(z)),\quad \text{for $k\geq 2$},
\end{split}
\end{equation}
where $\rem(f_{k-2}(z), f_{k-1}(z))$ denotes the remainder arising in the division of $f_{k-2}(z)$ by $f_{k-1}(z)$. For sufficiently large $k$ we have $f_k(z)\equiv 0$, so let $m$ be the index of the last non-trivial polynomial $f_m(z)$. The \emph{Sturm sequence} of $f(z)$ is the finite sequence of polynomials $(f_0(z), f_1(z),\ldots, f_m(z))$, where necessarily $m\leq n = \deg(f)$.
\label{def:Sturm-sequence}
\end{definition}
We denote by $\Xi[f](a)$ the sequence of \emph{signs} of the Sturm sequence of $f(z)$ evaluated at a point $a\in\mathbb{R}$:
\begin{equation}
\Xi[f](a) := (\sign(f_0(a)),\sign(f_1(a)),\sign(f_2(a)),\ldots, \sign(f_m(a)) ),
\end{equation}
and we let $\#(\Xi[f](a))$ denote the number of sign variations in $\Xi[f](a)$, i.e., the number of sign changes ignoring any zeros when counting. For instance, for $f(z)=4z^3 + z^2 -2$, we have the Sturm sequence
\begin{align}
f_0(z)=4z^3 + z^2 -2,\quad
f_1(z):=12 z^2 +2z,\quad
f_2(z):= \frac{1}{18}z +2,\quad
f_3(z):=-15480,\quad
f_4(z):=0,
\end{align}
and hence at $z=4$, for example, we have
\begin{equation}
 \Xi[f](4) = (\sign(-30), \sign(44), \sign(17/9), \sign(-15480))=(-,+,+,-),
\end{equation}
which gives $\#(\Xi[f](-2))=2$. As a more complicated example, we obtain $\#(\Xi[f](a))= 3$ if $\Xi[f](a)=(+,+,0,+,-,0,+,+,0,-)$. The following theorem (\cite{Sturm1829}, see also \cite[Theorem 1.4]{Sturmfels02}) gives an \emph{exact} count of real zeros of $f(z)$ weighted by multiplicity in an interval using $\#(\Xi[f](\cdot))$.
\begin{theorem}[Sturm's Theorem] Suppose that $a<b$ and neither $a$ nor $b$ is a zero of $f(z)$. Then $\#(\Xi[f](a))\geq \#(\Xi[f](b))$, and the number of real zeros, weighted by multiplicity, of the polynomial $f(z)$ in the interval $[a,b]$ is equal to $\#(\Xi[f](a)) - \#(\Xi[f](b))$.
\label{t:Sturm}
\end{theorem}
The theorem also applies to the case where $a=-\infty$ or $b=+\infty$ by considering the asymptotic behavior of the polynomials in the Sturm sequence, which amounts to looking at the signs of the leading coefficients of the polynomials $f_k(z)$ in the Sturm sequence of $f$.

Another result on the real roots of a polynomial is the following:

\begin{theorem}[D\'escartes' Rule of Signs] Let $f(z)=a_n z^n + a_{n-1} z^{n-1} + \cdots + a_1 z + a_0$, $a_n\neq 0$. The number of positive real roots of $f$ is at most the number of sign variations in its coefficient sequence $(a_n,a_{n-1},\ldots, a_1,a_0)$. Moreover, the number of positive real roots of $f$ differs from the the number of sign variations of the coefficients sequence by an even (nonnegative) integer.
\label{t:Descartes}
\end{theorem}
