In this section, we prove Theorem~\ref{thm:shelves} and its corollaries. 
%As indicated in the statement of theorem, 
Our analysis is valid for all $M\in \mathbb{Z}_{>0}$, with $\mathbf{G}=\mathbf{Q}^{-s}$, $s=\pm 1$ in contrast to that in the preceding section. The analysis will be guided by the sign chart of $\Re(\ii h(\lambda;\chi,\tau))$, $h(\lambda;\chi,\tau) = g(\lambda;\chi,\tau) +  \vartheta(\lambda;\chi,\tau)$.
Recall from the discussion in Section~\ref{sec:GenusZeroModification} that for $(\chi,\tau)\in\shelves$, $h'(\lambda;\chi,\tau)^2$ has 2 real double roots denoted by $a(\chi,\tau)<b(\chi,\tau)$, and two simple roots $A(\chi,\tau)\pm \ii B(\chi,\tau)$ for which we write $\lambda_0(\chi,\tau)\defeq  A(\chi,\tau) + \ii B(\chi,\tau)$, where $B(\chi,\tau)>0$, and we also have $A(\chi,\tau)<0$ because the endpoints $A(\chi,\tau) \pm \ii B(\chi,\tau)$ of $\Sigma_g$, along which $g(\lambda;\chi,\tau)$ has a jump discontinuity, lie in the left half-plane for all $(\chi,\tau)$ in the interior of $\shelves$. We recall from the beginning of Section~\ref{sec:zero-level-curve} that $\mathrm{Re}(\ii h(\lambda;\chi,\tau))=0$ holds for all $\lambda\in\mathbb{R} \setminus \Sigma_g$. Therefore, both $h_{-}(a(\chi,\tau);\chi,\tau)$ and $h(b(\chi,\tau);\chi,\tau)$ are real-valued. 
We also note the facts
\begin{equation}
h''_-(a(\chi,\tau);\chi,\tau) < 0 \quad\text{and}\quad h''(b(\chi,\tau);\chi,\tau) > 0,
\label{eq:h-double-prime-a-b-signs}
\end{equation}
which follow from the formula \eqref{eq:hprime-formula} since $\tau>0$ and $a(\chi,\tau),b(\chi,\tau)$ are real roots of the quadratic in the numerator of \eqref{eq:hprime-formula} for $(\chi,\tau)$ in the interior of $\shelves$.
%given in the introduction, taking into account that \tau>0 and that the quadratic has real roots at a and b on \shelves.

%Note that the unique real root $u=u(\chi,\tau)$ of $P(u;\chi,\tau)$ with odd multiplicity (simple except at a set of finitely many points) satisfies $u(\chi,0)=\chi$ and since $P(0;\chi,\tau)=-16 \tau^2 \tau^3 \neq 0$ in the open fist quadrant of the $(\chi,\tau)$-plane, $u(\chi,\tau)>0$ by continuity for $(\chi,\tau)\in \mathbb{R}_{>0} \times \mathbb{R}_{>0}$. As a matter of fact, $u(\chi,\tau)>\chi/3$ for $(\chi,\tau)$ in the open first quadrant by continuity since $P(\chi/3;\chi,\tau)=-(32/27)\chi^3\tau^4 \neq 0$ off the axes. Thus, the roots $a(\chi,\tau)$ and $b(\chi,\tau)$ of the quadratic \eqref{eq:quadratic-a-b} satisfy $a(\chi,\tau)+b(\chi,\tau)<0$, hence $a(\chi,\tau)<0$ for $(\chi,\tau)$ in $\shelves$ in the first quadrant. However, $b(\chi,\tau)$ is not sign-definite, hence the reason for allowing $\Sigma_\mathrm{c}$ to be a general Schwarz symmetric arc connecting the points $\lambda=\pm \ii$.  


Recall from Section~\ref{sec:zero-level-curve} that there is a Schwarz-symmetric arc of the zero level curve of $\lambda\mapsto \Re(\ii h(\lambda;\chi,\tau))$ that connects $\lambda_0(\chi,\tau)$ to $\lambda_0(\chi,\tau)^*$ and passes through the point $a(\chi,\tau)\in\mathbb{R}$. We place the branch cut $\Sigma_g$ on this curve, denote by $\Sigma_g^{\pm}$ its subarcs that lie in the half-planes $\mathbb{C}^{\pm}$, and orient $\Sigma_g$ from $\lambda_0^*$ to $\lambda_0$. The other bounded trajectory of the zero level curve in the upper half plane is one that connects $\lambda_0(\chi,\tau)$ to $b(\chi,\tau)$. We denote this arc by $\Gamma^+ = \Gamma^+(\chi,\tau)$, and its Schwarz reflection by $\Gamma^- = \Gamma^-(\chi,\tau)$, both with downward orientation. We also set $\Gamma\defeq  \Gamma^+ \cup \Gamma^- \cup \{  b(\chi,\tau) \}$. We set $I\defeq [a(\chi,\tau), b(\chi,\tau)]$ to
denote the only remaining bounded component of $\Re(\ii h(\lambda;\chi,\tau))=0$.
For the analysis that follows we take $\Sigma_\circ$ to be the clockwise-oriented loop $\Sigma_{\circ} = \Sigma_g \cup \Gamma$. We choose $\Sigma_\mathrm{c}$  to be a Schwarz-symmetric arc that connects the points $\lambda=\pm \ii$ while passing through the point $\lambda=\tfrac{1}{2}(a(\chi,\tau)+b(\chi,\tau))$, say (any point in $(a(\chi,\tau),b(\chi,\tau))$ would suffice), with upward orientation. See the left-hand panel of Figure~\ref{fig:SB1} for an illustration of these arcs.




%For $(\chi,\tau)\in \shelves$, we begin with deforming the circle $\Sigma_\circ$ to a loop which is a subset of the level curve 



%\textcolor{red}{Describe the properties of $h(\lambda;\chi,\tau)\defeq  g(\lambda;\chi,\tau) + \ii \vartheta(\lambda;\chi, \tau)$ here once more is in the regions with the neck contour $N$. $\lambda_0(\chi,\tau)$ and $\lambda_0(\chi,\tau)^*$ are the two distinct simple roots of $R(\lambda;\chi,\tau)^2$ for $(\chi,\tau)$ in the burger bun. Denote by $a(\chi,\tau) < b(\chi,\tau)$ the two remaining distinct (real) \emph{double} roots of $h'(\lambda;\chi,\tau)^2$, which are the two simple real roots of
%\begin{equation}
%2\tau \lambda^2 + u(\chi,\tau) \lambda + v(\chi,\tau) =0.
%\label{eq:quadratic-a-b}
%\end{equation}
%Then go on to describe the contours $\Gamma=\Gamma^+ \cup \Gamma^{-}$, and $\Sigma^{\pm}_g$ with the regions they separate, set $I\defeq [a(\chi,\tau), b(\chi,\tau)]$. One needs an argument to guarantee that the branch points $\lambda \pm \ii$ are contained in $\Omega^{\pm}$, which can be done if $\lambda_0(\chi,\tau)\neq \ii$.
%}
%In this section, we prove Theorem~\ref{}. As indicated in the statement of theorem, our analysis is valid for all $M\in \mathbb{Z}_{>0}$, with $\mathbf{G}=\mathbf{Q}^{-s}$, $s=(-1)^k$ in contrast to the result in the preceding section. \textcolor{red}{[Rogue waves and solitons being special cases mentioned here or in the introduction?]}. 
%For $(\chi,\tau)\in \shelves$, we begin with deforming the circle $\Sigma_\circ$ to a loop which is a subset of the level curve 


%For the calculation in this region, we assume that initially the Jordan curve $\Sigma_\circ$ contains $\Gamma \cup \Sigma_g$ in its interior and we make the substitution
%\begin{equation}
%\tilde{\mathbf{S}}(\lambda ; \chi, \tau,\mathbf{Q}^{-s},M)\defeq \mathbf{S}(\lambda ; \chi, \tau,\mathbf{Q}^{-s},M) \ee^{-\ii M \vartheta(\lambda ; \chi, \tau) \sigma_3} \mathbf{Q}^{-s} \ee^{ \ii M \vartheta (\lambda ; \chi, \tau) \sigma_3}
%\end{equation}
%for $\lambda$ between $\Sigma_0$ and $\Gamma \cup \Sigma_g$, and we set $\tilde{\mathbf{S}}(\lambda ; \chi, \tau,\mathbf{Q}^{-s},M)\defeq \mathbf{S}(\lambda ; \chi, \tau,\mathbf{Q}^{-s},M)$ everywhere else, i.e., in the exterior of $\Sigma_\circ$ and in the interior of $\Gamma \cup \Sigma_g$; and we drop the tilde. Then the jump contour for $\mathbf{S}(\lambda ; \chi, \tau,\mathbf{Q}^{-s},M)$ becomes $\Gamma \cup \Sigma_g$ and along this contour $\mathbf{S}(\lambda ; \chi, \tau,\mathbf{Q}^{-s},M)$ satisfies the same jump condition \eqref{eq:S-jump} as on $\Sigma_\circ$ prior to the substitution.

\begin{figure}[h]
\begin{center}
\includegraphics{SB1.pdf}
\end{center}
\caption{Left:  the initial jump contour and sign chart of $\mathrm{Re}(\ii h(\lambda;\chi,\tau))$ (shaded for negative, unshaded for positive) for $(\chi,\tau)=(2,0.8)\in \shelves$, showing also the regions where explicit transformations are made.  Right:  the resulting jump contour after the transformations.}
\label{fig:SB1}
\end{figure}

\subsection{Introduction of $g$ and steepest descent deformation of the Riemann-Hilbert problem}
With contours chosen this way, we have $\Sigma_g \cup \Sigma_\mathrm{c} = \emptyset$; therefore, we need the version of the construction of $g(\lambda;\chi,\tau)$ described in Section~\ref{sec:g-function-loop}. We introduce the $g$-function and the matrix function $\mathbf{T}(\lambda;\chi,\tau,\mathbf{Q}^{-s}, M)$ by the global substitution \eqref{eq:T-to-S}.
We let $\Omega^+$ denote the domain enclosed by $\Sigma_g^+ \cup \Gamma^+ \cup I $ which contains $\lambda=\ii$ and $\Sigma_\mathrm{c} \cap \mathbb{C}^+$, and let $\Omega^{-}$ be its Schwarz reflection. 
We let $L_\Sigma^\pm$ and $L_\Gamma^\pm$ (resp., $R_\Sigma^\pm$ and $R_\Gamma^\pm$) denote lens-shaped regions lying to the left (resp., right) of $\Sigma_g^\pm$ and $\Gamma^\pm$ with respect to orientation, as depicted in the left-hand panel of Figure~\ref{fig:SB1}. 
The lens-shaped regions are chosen to be so thin as to exclude the points $\lambda=\pm \ii$ and $\Sigma_\mathrm{c}$ while supporting a fixed sign of $\Re(\ii h(\lambda;\chi,\tau))$. 
On $\Gamma^\pm$, we will use the two-factor factorizations of the central factor $\mathbf{Q}^{-s}$, exactly as written in \eqref{eq:Q-factorizations}.
%\begin{equation}
%\mathbf{Q}^{-s} = 2^{\frac{1}{2}\sigma_3} \begin{bmatrix} 1 & \frac{s}{2} \\ 0 & 1 \end{bmatrix} \begin{bmatrix} 1 & 0 \\ -s & 1 \end{bmatrix},
%\end{equation}
%which we exploit in the part $\Gamma^+$ of $\Gamma$ in the upper half-plane, and
%\begin{equation}
%\mathbf{Q}^{-s} = 2^{-\frac{1}{2}\sigma_3} \begin{bmatrix} 1 & 0 \\ -\frac{s}{2} & 1 \end{bmatrix} \begin{bmatrix} 1 & s \\ 0 & 1 \end{bmatrix}
%\end{equation}
%which we exploit in the part $\Gamma^-$ of $\Gamma$ in the lower half-plane.
However, along $\Sigma_g$, we will employ the following additional factorizations
\begin{equation}
\mathbf{Q}^{-s} = 
\begin{cases}
2^{\frac{1}{2}\sigma_3} \begin{bmatrix} 1 & -\frac{1}{2}s \\ 0 & 1 \end{bmatrix} \begin{bmatrix} 0 & s \\ -s & 0 \end{bmatrix} \begin{bmatrix} 1 & -s \\ 0 & 1 \end{bmatrix}, \quad& \lambda\in \Sigma_g^+,\vspace{0.33em}\\
 2^{-\frac{1}{2}\sigma_3} \begin{bmatrix} 1 & 0 \\ \frac{1}{2}s & 1 \end{bmatrix} \begin{bmatrix} 0 & s \\ -s & 0 \end{bmatrix} \begin{bmatrix} 1 & 0 \\ s & 1 \end{bmatrix},\quad& \lambda\in \Sigma_g^-.
\end{cases}
\label{eq:Q-triple-factorization}
\end{equation}
%On the other hand, $\mathbf{Q}^{-s}$ also admits the following two three-factor factorizations
%\begin{equation}
%\mathbf{Q}^{-s} = 2^{\sigma_3/2} \begin{bmatrix} 1 & -\frac{s}{2} \\ 0 & 1 \end{bmatrix} \begin{bmatrix} 0 & s \\ -s & 0 \end{bmatrix} \begin{bmatrix} 1 & -s \\ 0 & 1 \end{bmatrix},
%\end{equation}
%which we will employ on the subarc $\Sigma_g^+$ of $\Sigma_g$ lying in the upper half-plane, and
%\begin{equation}
%\mathbf{Q}^{-s}  =  2^{-\sigma_3/2} \begin{bmatrix} 1 & 0 \\ \frac{s}{2} & 1 \end{bmatrix} \begin{bmatrix} 0 & s \\ -s & 0 \end{bmatrix} \begin{bmatrix} 1 & 0 \\ s & 1 \end{bmatrix},
%\end{equation}
%which we will employ on the subarc $\Sigma_g^{-}$ of $\Sigma_g$ lying in the lower half-plane. \textcolor{red}{[These have been used before, so cf. them.]}
%Taking advantage of the two-factor matrix factorizations \eqref{eq:Q-factorizations} along with the domains in which $\Re(\ii h(\lambda;\chi,\tau))$ is sign-definite, we define $\mathbf{W}(\lambda)=\mathbf{W}(\lambda;\chi,\tau,\mathbf{Q}^{-s},M)$ in the lens-shaped regions surrounding $\Gamma_\pm$ by:
Taking advantage of the two-factor matrix factorizations \eqref{eq:Q-factorizations}, we define $\mathbf{W}(\lambda)=\mathbf{W}(\lambda;\chi,\tau,\mathbf{Q}^{-s},M)$ in the lens-shaped regions surrounding $\Gamma^\pm$ by:
%making the following substitutions to control the exponential factors present in \eqref{eq:T-jump}:
%\begin{equation}
%\mathbf{W}^{(k)}(\lambda;\chi,\tau)\defeq \mathbf{T}^{(k)}(\lambda;\chi,\tau) \begin{bmatrix} 1 & 0 \\ s \omega(\lambda)^{-2s}\ee^{2n h(\lambda;\chi,\tau)}& 1 \end{bmatrix},\quad \lambda\in L^+_{\Gamma},
%\label{eq:T-to-W-L-plus-Gamma}
%\end{equation}
%\begin{equation}
%\mathbf{W}^{(k)}(\lambda;\chi,\tau)\defeq \mathbf{T}^{(k)}(\lambda;\chi,\tau)
%2^{\sigma_3/2} \begin{bmatrix} 1 & \frac{s}{2} \omega(\lambda)^{2s}\ee^{-2n h(\lambda;\chi,\tau)} \\ 0 & 1 \end{bmatrix},\quad \lambda \in R^+_{\Gamma},
%\end{equation}

%\begin{equation}
%\mathbf{W}^{(k)}(\lambda;\chi,\tau)\defeq \mathbf{T}^{(k)}(\lambda;\chi,\tau) \begin{bmatrix} 1 & 0 \\ s \ee^{2M h(\lambda;\chi,\tau)}& 1 \end{bmatrix},\quad \lambda\in L^+_{\Gamma},
%\label{eq:T-to-W-L-plus-Gamma}
%\end{equation}
%\begin{equation}
%\mathbf{W}^{(k)}(\lambda;\chi,\tau)\defeq \mathbf{T}^{(k)}(\lambda;\chi,\tau)
%2^{\frac{\sigma_3}{2}} \begin{bmatrix} 1 & \frac{s}{2} \ee^{-2M h(\lambda;\chi,\tau)} \\ 0 & 1 \end{bmatrix},\quad \lambda \in R^+_{\Gamma},
%\end{equation}
\begin{equation}
\mathbf{W}(\lambda)\defeq \mathbf{T}(\lambda;\chi,\tau,\mathbf{Q}^{-s},M) 
\begin{bmatrix} 1 & 0 \\ s \ee^{2\ii M h(\lambda;\chi,\tau)}& 1 \end{bmatrix},\quad \lambda\in L^+_{\Gamma},
\label{eq:T-to-W-L-plus-Gamma}
\end{equation}
\begin{equation}
\mathbf{W}(\lambda)\defeq \mathbf{T}(\lambda;\chi,\tau,\mathbf{Q}^{-s},M) 
2^{\frac{1}{2}\sigma_3} \begin{bmatrix} 1 & \frac{1}{2} s\ee^{-2\ii M h(\lambda;\chi,\tau)} \\ 0 & 1 \end{bmatrix},\quad \lambda \in R^+_{\Gamma},
\end{equation}
%
%\begin{equation}
%\mathbf{W}^{(k)}(\lambda;\chi,\tau)\defeq \mathbf{T}^{(k)}(\lambda;\chi,\tau)
%2^{\frac{1}{2}\sigma_3},\quad \lambda \in \Omega^+,
%\end{equation}
%\begin{equation}
%\mathbf{W}^{(k)}(\lambda;\chi,\tau)\defeq \mathbf{T}^{(k)}(\lambda;\chi,\tau)
%2^{-\frac{1}{2}\sigma_3},\quad \lambda \in \Omega^-,
%\end{equation}
\begin{equation}
\mathbf{W}(\lambda)\defeq \mathbf{T}(\lambda;\chi,\tau,\mathbf{Q}^{-s},M) 
2^{\frac{1}{2}\sigma_3},\quad \lambda \in \Omega^+,
\end{equation}
\begin{equation}
\mathbf{W}(\lambda)\defeq \mathbf{T}(\lambda;\chi,\tau,\mathbf{Q}^{-s},M) 
2^{-\frac{1}{2}\sigma_3},\quad \lambda \in \Omega^-,
\end{equation}
\begin{equation}
\mathbf{W}(\lambda)\defeq \mathbf{T}(\lambda;\chi,\tau,\mathbf{Q}^{-s},M) 
2^{-\sigma_3/2}  \begin{bmatrix} 1 & 0 \\ -\frac{1}{2}s \ee^{2\ii M h(\lambda;\chi,\tau)} & 1 \end{bmatrix} ,\quad \lambda \in R^-_{\Gamma},\quad\text{and}
\end{equation}
\begin{equation}
\mathbf{W}(\lambda)\defeq \mathbf{T}(\lambda;\chi,\tau,\mathbf{Q}^{-s},M) 
 \begin{bmatrix} 1 & - s \ee^{-2\ii M h(\lambda;\chi,\tau)} \\ 0 & 1 \end{bmatrix} ,\quad \lambda \in L^-_{\Gamma}.
\end{equation}
Note how the definitions above compare with \eqref{eq:W-def-Schi-Stau-Lplus-ALT}--\eqref{eq:W-def-Schi-Stau-Lminus-ALT} which use the same factorizations of $\mathbf{Q}^{-s}$: the regions $L^\pm$ and $R^\pm$ are merely replaced with $L_\Gamma^\pm$ and $R_\Gamma^\pm$, respectively. In the lens-shaped regions surrounding $\Sigma_g$ on the other hand, we make use of the factorizations \eqref{eq:Q-triple-factorization} and define:
%\begin{equation}
%\mathbf{W}^{(k)}(\lambda;\chi,\tau)\defeq \mathbf{T}^{(k)}(\lambda;\chi,\tau)
%2^{-\frac{1}{2}\sigma_3}  \begin{bmatrix} 1 & 0 \\ -\frac{s}{2}  \ee^{2M h(\lambda;\chi,\tau)} & 1 \end{bmatrix} ,\quad \lambda \in R^-_{\Gamma},
%\end{equation}
%\begin{equation}
%\mathbf{W}^{(k)}(\lambda;\chi,\tau)\defeq \mathbf{T}^{(k)}(\lambda;\chi,\tau)
% \begin{bmatrix} 1 & - s \ee^{-2M h(\lambda;\chi,\tau)} \\ 0 & 1 \end{bmatrix} ,\quad \lambda \in L^-_{\Gamma},
%\end{equation}
%\begin{equation}
%\mathbf{W}^{(k)}(\lambda;\chi,\tau)\defeq \mathbf{T}^{(k)}(\lambda;\chi,\tau)
% \begin{bmatrix} 1 & s \omega(\lambda)^{2s} \ee^{-2n h(\lambda;\chi,\tau)} \\ 0 & 1 \end{bmatrix},\quad \lambda \in L^+_{\Sigma},
%\end{equation}
%\begin{equation}
%\mathbf{W}^{(k)}(\lambda;\chi,\tau)\defeq \mathbf{T}^{(k)}(\lambda;\chi,\tau)
%2^{\sigma_3/2} \begin{bmatrix} 1 & -\frac{s}{2}\omega(\lambda)^{2 s}\ee^{-2n h(\lambda;\chi,\tau)} \\ 0 & 1 \end{bmatrix},\quad \lambda \in R^+_{\Sigma},
%\end{equation}
%\begin{equation}
%\mathbf{W}^{(k)}(\lambda;\chi,\tau)\defeq \mathbf{T}^{(k)}(\lambda;\chi,\tau)
% \begin{bmatrix} 1 & 0 \\ -s \omega(\lambda)^{-2s} \ee^{2n h(\lambda;\chi,\tau)} & 1 \end{bmatrix},\quad \lambda \in L^-_{\Sigma},
%\end{equation}
%and
%\begin{equation}
%\mathbf{W}^{(k)}(\lambda;\chi,\tau)\defeq \mathbf{T}^{(k)}(\lambda;\chi,\tau)
% 2^{-\sigma_3/2} \begin{bmatrix} 1 & 0 \\ \frac{s}{2}\omega(\lambda)^{-2s}\ee^{2n h(\lambda;\chi,\tau)} & 1 \end{bmatrix}
%,\quad \lambda \in R^-_{\Sigma}.
%\label{eq:T-to-W-R-minus-Sigma}
%\end{equation}
%\begin{equation}
%\mathbf{W}^{(k)}(\lambda;\chi,\tau)\defeq \mathbf{T}^{(k)}(\lambda;\chi,\tau)
% \begin{bmatrix} 1 & s  \ee^{-2M h(\lambda;\chi,\tau)} \\ 0 & 1 \end{bmatrix},\quad \lambda \in L^+_{\Sigma},
%\end{equation}
%\begin{equation}
%\mathbf{W}^{(k)}(\lambda;\chi,\tau)\defeq \mathbf{T}^{(k)}(\lambda;\chi,\tau)
%2^{\frac{1}{2}\sigma_3} \begin{bmatrix} 1 & -\frac{s}{2} \ee^{-2M h(\lambda;\chi,\tau)} \\ 0 & 1 \end{bmatrix},\quad \lambda \in R^+_{\Sigma},
%\end{equation}
%\begin{equation}
%\mathbf{W}^{(k)}(\lambda;\chi,\tau)\defeq \mathbf{T}^{(k)}(\lambda;\chi,\tau)
% \begin{bmatrix} 1 & 0 \\ -s  \ee^{2M h(\lambda;\chi,\tau)} & 1 \end{bmatrix},\quad \lambda \in L^-_{\Sigma},
%\end{equation}
\begin{equation}
\mathbf{W}(\lambda)\defeq \mathbf{T}(\lambda;\chi,\tau,\mathbf{Q}^{-s},M) 
 \begin{bmatrix} 1 & s  \ee^{-2\ii M h(\lambda;\chi,\tau)} \\ 0 & 1 \end{bmatrix},\quad \lambda \in L^+_{\Sigma},
\end{equation}
\begin{equation}
\mathbf{W}(\lambda)\defeq \mathbf{T}(\lambda;\chi,\tau,\mathbf{Q}^{-s},M) 
2^{\frac{1}{2}\sigma_3} \begin{bmatrix} 1 & -\frac{1}{2} s \ee^{-2\ii M h(\lambda;\chi,\tau)} \\ 0 & 1 \end{bmatrix},\quad \lambda \in R^+_{\Sigma},
\end{equation}
\begin{equation}
\mathbf{W}(\lambda)\defeq \mathbf{T}(\lambda;\chi,\tau,\mathbf{Q}^{-s},M) 
 \begin{bmatrix} 1 & 0 \\ -s  \ee^{2\ii M h(\lambda;\chi,\tau)} & 1 \end{bmatrix},\quad \lambda \in L^-_{\Sigma},\quad\text{and}
\end{equation}
\begin{equation}
\mathbf{W}(\lambda)\defeq \mathbf{T}(\lambda;\chi,\tau,\mathbf{Q}^{-s},M) 
 2^{-\frac{1}{2}\sigma_3} \begin{bmatrix} 1 & 0 \\ \frac{1}{2} s \ee^{2\ii M h(\lambda;\chi,\tau)} & 1 \end{bmatrix}
,\quad \lambda \in R^-_{\Sigma}.
\label{eq:T-to-W-R-minus-Sigma-ALT}
\end{equation}
We simply leave $\mathbf{W}(\lambda) \defeq  \mathbf{T}(\lambda;\chi,\tau,\mathbf{Q}^{-s},M) $ elsewhere. It is now easy to see that $\mathbf{W}(\lambda)$ extends to $\lambda\in \Gamma^+ \cup \Gamma^-$ as an analytic function, so that $\mathbf{W}(\lambda)$ is analytic in the complement of the jump contour $C^{+}_{\Sigma,L} \cup C^{+}_{\Sigma,R} \cup C^{-}_{\Sigma,L} \cup C^{-}_{\Sigma,R} \cup \Sigma_g \cup I \cup C^{+}_{\Gamma,L} \cup C^{+}_{\Gamma,R} \cup C^{-}_{\Gamma,L} \cup C^{-}_{\Gamma,R}$, the arcs of which are depicted in the right-hand panel of Figure~\ref{fig:SB1}. Across these arcs $\mathbf{W}(\lambda) $ satisfies the following jump relations:
%See Figure~\ref{fig:SB1} for definitions of the above and the curves that separate them. Having done these substitutions, we see that $\mathbf{W}^{(k)}(\lambda;\chi,\tau) $ satisfies the following jump relations
%\begin{equation}
%\mathbf{W}^{(k)}_{+}(\lambda) = \mathbf{W}^{(k)}_{-}(\lambda)
%\begin{bmatrix} 1 & 0 \\ -s \omega(\lambda)^{-2s}\ee^{2n h(\lambda;\chi,\tau)}& 1 \end{bmatrix},\quad \lambda\in C^+_{\Gamma,L},
%\label{eq:W-jump-Gamma-plus-L}
%\end{equation}
%\begin{equation}
%\mathbf{W}^{(k)}_{+}(\lambda) = \mathbf{W}^{(k)}_{-}(\lambda)
%\begin{bmatrix} 1 & \frac{s}{2} \omega(\lambda)^{2s}\ee^{-2n h(\lambda;\chi,\tau)} \\ 0 & 1 \end{bmatrix},\quad \lambda\in C^+_{\Gamma,R},
%\end{equation}
%\begin{equation}
%\mathbf{W}^{(k)}_{+}(\lambda) = \mathbf{W}^{(k)}_{-}(\lambda)
%2^{\sigma_3},\quad \lambda\in I ,
%\end{equation}
%\begin{equation}
%\mathbf{W}^{(k)}_{+}(\lambda) = \mathbf{W}^{(k)}_{-}(\lambda)
%\begin{bmatrix} 1 & 0 \\ -\frac{s}{2} \omega(\lambda)^{-2s} \ee^{2n h(\lambda;\chi,\tau)} & 1 \end{bmatrix} ,\quad \lambda\in C^-_{\Gamma,R},
%\end{equation}
%\begin{equation}
%\mathbf{W}^{(k)}_{+}(\lambda) = \mathbf{W}^{(k)}_{-}(\lambda)
%\begin{bmatrix} 1 &  s  \omega(\lambda)^{2s} \ee^{-2n h(\lambda;\chi,\tau)} \\ 0 & 1 \end{bmatrix} ,\quad \lambda\in C^-_{\Gamma,L},
%\end{equation}
%\begin{equation}
%\mathbf{W}^{(k)}_{+}(\lambda) = \mathbf{W}^{(k)}_{-}(\lambda)
%\begin{bmatrix} 1 & -s \omega(\lambda)^{2s} \ee^{-2n h(\lambda;\chi,\tau)} \\ 0 & 1 \end{bmatrix},\quad \lambda\in C^+_{\Sigma,L},
%\end{equation}
%\begin{equation}
%\mathbf{W}^{(k)}_{+}(\lambda) = \mathbf{W}^{(k)}_{-}(\lambda)
%\begin{bmatrix} 1 & -\frac{s}{2}\omega(\lambda)^{2 s}\ee^{-2n h(\lambda;\chi,\tau)} \\ 0 & 1 \end{bmatrix},\quad \lambda\in C^+_{\Sigma,R},
%\end{equation}
%\begin{equation}
%\mathbf{W}^{(k)}_{+}(\lambda) = \mathbf{W}^{(k)}_{-}(\lambda)
%\begin{bmatrix} 1 & 0 \\ s \omega(\lambda)^{-2s} \ee^{2n h(\lambda;\chi,\tau)} & 1 \end{bmatrix},\quad \lambda\in C^-_{\Sigma,L},
%\end{equation}
%\begin{equation}
%\mathbf{W}^{(k)}_{+}(\lambda) = \mathbf{W}^{(k)}_{-}(\lambda)
%\begin{bmatrix} 1 & 0 \\ \frac{s}{2} \omega(\lambda)^{-2s} \ee^{2n h(\lambda;\chi,\tau)} & 1 \end{bmatrix},\quad \lambda\in C^-_{\Sigma,R},
%\label{eq:W-jump-Sigma-minus-R}
%\end{equation}
%\begin{equation}
%\mathbf{W}^{(k)}_{+}(\lambda;\chi,\tau) = \mathbf{W}^{(k)}_{-}(\lambda;\chi,\tau)
%\begin{bmatrix} 1 & 0 \\ -s \ee^{2M h(\lambda;\chi,\tau)}& 1 \end{bmatrix},\quad \lambda\in C^+_{\Gamma,L},
%\label{eq:W-jump-Gamma-plus-L}
%\end{equation}
%\begin{equation}
%\mathbf{W}^{(k)}_{+}(\lambda;\chi,\tau) = \mathbf{W}^{(k)}_{-}(\lambda;\chi,\tau)
%\begin{bmatrix} 1 & \frac{s}{2} \ee^{-2M h(\lambda;\chi,\tau)} \\ 0 & 1 \end{bmatrix},\quad \lambda\in C^+_{\Gamma,R},
%\end{equation}
%\begin{equation}
%\mathbf{W}^{(k)}_{+}(\lambda;\chi,\tau) = \mathbf{W}^{(k)}_{-}(\lambda;\chi,\tau)
%2^{\sigma_3},\quad \lambda\in I ,
%\label{eq:W-jump-I}
%\end{equation}
%\begin{equation}
%\mathbf{W}^{(k)}_{+}(\lambda;\chi,\tau) = \mathbf{W}^{(k)}_{-}(\lambda;\chi,\tau)
%\begin{bmatrix} 1 & 0 \\ -\frac{s}{2}  \ee^{2M h(\lambda;\chi,\tau)} & 1 \end{bmatrix} ,\quad \lambda\in C^-_{\Gamma,R},
%\end{equation}
%\begin{equation}
%\mathbf{W}^{(k)}_{+}(\lambda;\chi,\tau) = \mathbf{W}^{(k)}_{-}(\lambda;\chi,\tau)
%\begin{bmatrix} 1 &  s \ee^{-2M h(\lambda;\chi,\tau)} \\ 0 & 1 \end{bmatrix} ,\quad \lambda\in C^-_{\Gamma,L},
%\end{equation}
%\begin{equation}
%\mathbf{W}^{(k)}_{+}(\lambda;\chi,\tau) = \mathbf{W}^{(k)}_{-}(\lambda;\chi,\tau)
%\begin{bmatrix} 1 & -s  \ee^{-2M h(\lambda;\chi,\tau)} \\ 0 & 1 \end{bmatrix},\quad \lambda\in C^+_{\Sigma,L},
%\end{equation}
%\begin{equation}
%\mathbf{W}^{(k)}_{+}(\lambda;\chi,\tau) = \mathbf{W}^{(k)}_{-}(\lambda;\chi,\tau)
%\begin{bmatrix} 1 & -\frac{s}{2} \ee^{-2M h(\lambda;\chi,\tau)} \\ 0 & 1 \end{bmatrix},\quad \lambda\in C^+_{\Sigma,R},
%\end{equation}
%\begin{equation}
%\mathbf{W}^{(k)}_{+}(\lambda;\chi,\tau)= \mathbf{W}^{(k)}_{-}(\lambda;\chi,\tau)
%\begin{bmatrix} 1 & 0 \\ s  \ee^{2M h(\lambda;\chi,\tau)} & 1 \end{bmatrix},\quad \lambda\in C^-_{\Sigma,L},
%\end{equation}
%\begin{equation}
%\mathbf{W}^{(k)}_{+}(\lambda;\chi,\tau) = \mathbf{W}^{(k)}_{-}(\lambda;\chi,\tau)
%\begin{bmatrix} 1 & 0 \\ \frac{s}{2}  \ee^{2M h(\lambda;\chi,\tau)} & 1 \end{bmatrix},\quad \lambda\in C^-_{\Sigma,R},
%\label{eq:W-jump-Sigma-minus-R-ALT}
%\end{equation}
\begin{equation}
\mathbf{W}_{+}(\lambda) = \mathbf{W}_{-}(\lambda)
\begin{bmatrix} 1 & 0 \\ -s \ee^{2\ii M h(\lambda;\chi,\tau)}& 1 \end{bmatrix},\quad \lambda\in C^+_{\Gamma,L},
\label{eq:W-jump-Gamma-plus-L}
\end{equation}
\begin{equation}
\mathbf{W}_{+}(\lambda) = \mathbf{W}_{-}(\lambda)
\begin{bmatrix} 1 & \frac{1}{2} s\ee^{-2\ii M h(\lambda;\chi,\tau)} \\ 0 & 1 \end{bmatrix},\quad \lambda\in C^+_{\Gamma,R},
\end{equation}
\begin{equation}
\mathbf{W}_{+}(\lambda) = \mathbf{W}_{-}(\lambda)
\begin{bmatrix} 1 & 0 \\ -\frac{1}{2}s  \ee^{2\ii M h(\lambda;\chi,\tau)} & 1 \end{bmatrix} ,\quad \lambda\in C^-_{\Gamma,R},\quad\text{and}
\end{equation}
\begin{equation}
\mathbf{W}_{+}(\lambda) = \mathbf{W}_{-}(\lambda)
\begin{bmatrix} 1 &  s \ee^{-2\ii M h(\lambda;\chi,\tau)} \\ 0 & 1 \end{bmatrix} ,\quad \lambda\in C^-_{\Gamma,L},
\end{equation}
which are again in parallel with the jump conditions \eqref{eq:Wjump-exterior-CLplus}--\eqref{eq:Wjump-exterior-CLminus}, and
\begin{equation}
\mathbf{W}_{+}(\lambda) = \mathbf{W}_{-}(\lambda)
2^{\sigma_3},\quad \lambda\in I ,
\label{eq:W-jump-I}
\end{equation}
\begin{equation}
\mathbf{W}_{+}(\lambda) = \mathbf{W}_{-}(\lambda)
\begin{bmatrix} 1 & -s  \ee^{-2\ii M h(\lambda;\chi,\tau)} \\ 0 & 1 \end{bmatrix},\quad \lambda\in C^+_{\Sigma,L},
\end{equation}
\begin{equation}
\mathbf{W}_{+}(\lambda) = \mathbf{W}_{-}(\lambda)
\begin{bmatrix} 1 & -\frac{1}{2} s \ee^{-2\ii M h(\lambda;\chi,\tau)} \\ 0 & 1 \end{bmatrix},\quad \lambda\in C^+_{\Sigma,R},
\end{equation}
\begin{equation}
\mathbf{W}_{+}(\lambda) = \mathbf{W}_{-}(\lambda)
\begin{bmatrix} 1 & 0 \\ s  \ee^{2\ii M h(\lambda;\chi,\tau)} & 1 \end{bmatrix},\quad \lambda\in C^-_{\Sigma,L},\quad\text{and}
\end{equation}
\begin{equation}
\mathbf{W}_{+}(\lambda) = \mathbf{W}_{-}(\lambda)
\begin{bmatrix} 1 & 0 \\ \frac{1}{2} s  \ee^{2\ii M h(\lambda;\chi,\tau)} & 1 \end{bmatrix},\quad \lambda\in C^-_{\Sigma,R}.
\label{eq:W-jump-Sigma-minus-R-ALT}
\end{equation}
Finally, along the branch cut $\Sigma_g$ we have
%\begin{equation}
%\mathbf{W}^{(k)}_{+}(\lambda) = \mathbf{W}^{(k)}_{-}(\lambda)
%\begin{bmatrix} 0 & s \omega(\lambda)^{2s} \ee^{-2\ii n \kappa(\chi,\tau))} \\ -s \omega(\lambda)^{-2s} \ee^{2 \ii n \kappa(\chi,\tau)} & 0 \end{bmatrix},\quad \lambda\in \Sigma_g = \Sigma_g^+ \cup \Sigma_g^-,
%\label{eq:W-jump-Sigma}
%\end{equation}
\begin{equation}
\mathbf{W}_{+}(\lambda) = \mathbf{W}_{-}(\lambda)
\begin{bmatrix} 0 & s  \ee^{-2\ii M \kappa(\chi,\tau))} \\ -s  \ee^{2 \ii M \kappa(\chi,\tau)} & 0 \end{bmatrix},\quad \lambda\in \Sigma_g = \Sigma_g^+ \cup \Sigma_g^-,
\label{eq:W-jump-Sigma-g}
\end{equation}
where $2\kappa(\chi,\tau)$ is the real constant value of $h_{+}(\lambda;\chi,\tau)+h_{-}(\lambda;\chi,\tau)$ for $\lambda\in\Sigma_g$ and $\kappa(\chi,\tau)$ is given in \eqref{eq:kappa-formula}. 
%We reminder the reader that the formula \eqref{eq:kappa-formula} defines the real constant $\kappa(\chi,\tau)$ modulo $2\pi$ as long as the path of integration connecting $\lambda=\pm \ii$ avoids the branch cut $\Sigma_g$ of $R(\lambda;\chi,\tau)$, hence $\ee^{\ii n \kappa(\chi,\tau)}$ is well-defined as long as the path of integration avoids $\Sigma_g$ (see the passage following \eqref{eq:kappa-formula}).
It follows from the sign chart of $\Re(\ii h(\lambda;\chi,\tau))$ as shown in Figure~\ref{fig:SB1} that all of the jump matrices above except for those supported on $I\cup\Sigma_g$ tend to the identity matrix exponentially fast as $M\to +\infty$ for $\lambda$ on the relevant supporting arcs away from the points $\lambda = \lambda_{0}(\chi,\tau)$, $\lambda_{0}(\chi,\tau)^{*}$, $a(\chi,\tau)$ and $b(\chi,\tau)$. In sufficiently small neighborhoods of these points, we will construct local parametrices that satisfy the jump conditions exactly.


%Note that the unique real root $u=u(\chi,\tau)$ of $P(u;\chi,\tau)$ with odd multiplicity (simple except at a set of finitely many points) satisfies $u(\chi,0)=\chi$ and since $P(0;\chi,\tau)=-16 \tau^2 \tau^3 \neq 0$ in the open fist quadrant of the $(\chi,\tau)$-plane, $u(\chi,\tau)>0$ by continuity for $(\chi,\tau)\in \mathbb{R}_{>0} \times \mathbb{R}_{>0}$. \textcolor{red}{[As a matter of fact, $u(\chi,\tau)>\chi/3$ for $(\chi,\tau)$ in the open first quadrant by continuity since $P(\chi/3;\chi,\tau)=-(32/27)\chi^3\tau^4 \neq 0$ off the axes.]} Thus, the roots $a(\chi,\tau)$ and $b(\chi,\tau)$ of the quadratic \eqref{eq:quadratic-a-b} satisfy $a(\chi,\tau)+b(\chi,\tau)<0$, hence $a(\chi,\tau)<0$ for $(\chi,\tau)$ in $\shelves$ in the first quadrant. On the other hand, it is possible that $b(\chi,\tau)=0$ or $b(\chi,\tau)<0$ for $(\chi,\tau) \in \shelves$ in the first quadrant. Indeed, as $a(\chi,\tau)<b(\chi,\tau)$, we see from \eqref{eq:quadratic-a-b} that $b(\chi,\tau)=0$ if and only if $v=v(\chi,\tau)=0$, which is by \eqref{eq:eliminate-v} equivalent to the condition $u(\chi,\tau)=\chi/2$. We observe that when $u=\chi/2$, $P(u;\chi,\tau)$ factors
%\begin{equation}
%P(\chi/2; \chi,\tau) = -\frac{1}{128}\chi^3 (\chi^2 +16 \tau + 16\tau^2) (\chi^2 -16 \tau + 16\tau^2),
%\end{equation}
%and the condition relevant for $\tau>0$ upon enforcing $P(\chi/2;\chi,\tau)=0$ is $(\chi^2 -16 \tau + 16\tau^2)=0$, which is the ellipse
%\begin{equation}
%\chi^2 + 16\left(\tau-\tfrac{1}{2}\right)^2 = 4.
%\end{equation}
%The branch of this ellipse in the first quadrant can be expressed as:
%\begin{equation}
%\chi = 4 \sqrt{\tau(1-\tau)},
%\label{eq:b-is-zero}
%\end{equation}
%and note that we necessarily have $0\leq \tau \leq 1$. We claim that this curve lies below the light blue curve plotted in Figure~\ref{fig:RegionsPlot}. 
%To see this, recall that this curve is given by 
%\begin{equation}
%Z(\chi,\tau) \defeq   H_{10}(\chi,\tau) +  H_{8}(\chi,\tau) +  H_{6}(\chi,\tau) +  H_{4}(\chi,\tau) = 0,
%\end{equation}
%where $H_{k}(\chi,\tau)$ are the homogenous polynomials defined in \eqref{eq:H-polynomials}.
%%To see this, first recall that the light blue cure in Figure~\ref{fig:RegionsPlot} is precisely the vanishing of the discriminant of the quadratic \eqref{eq:quadratic-a-b}, and the branch of it that enters the quadrant from the point $(\chi,\tau)=(0,1)$ is the locus $a(\chi,\tau)=b(\chi,\tau)\in\mathbb{R}$ \textcolor{red}{[The other branch is also this locus if we correctly label the points there, two complex roots collide on the real line there.]}. Upon eliminating $v$ via \eqref{eq:eliminate-v} the vanishing of this discriminant is the condition
%%\begin{equation}
%%u^2 = 16 \tau^2\frac{2u-\chi}{3u-\chi}.
%%\end{equation}
%%Recalling that $u>\chi/3$ in the first quadrant and clearing the denominator gives the condition
%%\begin{equation}
%%3u^3 -u^2 \chi -32u \tau^2 + 16\tau^2 \chi =0.
%%\label{eq:condition-real-double-root}
%%\end{equation}
%%Taking the resultant of \eqref{eq:condition-real-double-root} with $P(u;\chi,\tau)=0$ with respect to $u$ yields
%%\begin{multline}
%%Z(\chi,\tau)\defeq 800000000 \tau^{10} - 68000000 \tau ^8 \chi ^2 - 3520000 \tau^6 \chi^4 - 56800 \tau^4 \chi^6 - 392 \tau^2 \chi^8 -\chi ^{10} \\
%%-730880000 \tau^8 + 125516800 \tau ^6 \chi^2 -1741728 \tau ^4 \chi ^4 - 1040 \tau^2 \chi^6 - 2 \chi^8 \\
%%- 67627008 \tau^6 - 3103488 \tau ^4 \chi^2  - 504 \tau^2 \chi^4 - \chi^6 -1492992 \tau^4  
%%= 0,
%%\label{eq:light-blue-curve}
%%\end{multline}
%Substituting \eqref{eq:b-is-zero} in $Z(\chi,\tau)$ and enforcing $Z(4\sqrt{\tau(1-\tau)},\tau) = 0$ yields the condition
%\begin{multline}
%2048 \tau^3 (\tau - 1)( 583443 \tau^6 + 639009 \tau^5 - 1083510 \tau^4 + 257994 \tau^3 + 27314 \tau^2 + 852 \tau + 2) = 0.
%\end{multline}
%Applying the theory of Sturm sequences (see Theorem~\ref{t:Sturm}) for the sextic polynomial in the last factor above shows that it has no positive roots. Thus, $Z(4\sqrt{\tau(1-\tau)},\tau) = 0$ only at $\tau=0$ or $\tau=1$.
%Furthermore, $Z(4\sqrt{\tau(1-\tau)},\tau)<0$ for $0<\tau<1$, and this proves the claim that the curve \eqref{eq:b-is-zero} lies below the curve $Z(\chi,\tau)=0$, while having $(\chi,\tau)=(0,1)$ to be its only point of intersection with this curve. On the other hand, the restriction of the ellipse $(\chi^2 -16 \tau + 16\tau^2)=0$ to the first quadrant can also be viewed as the union of the two branches
%\begin{equation}
%\tau - \frac{1}{2} = %\pm \frac{1}{4}\sqrt{4-\chi^2}=
%\pm \frac{1}{2}\sqrt{1-\left(\frac{\chi}{2}\right)^2},
%\label{eq:b-is-zero-branches}
%\end{equation}
%which are symmetric with respect to the horizontal line $\tau=\frac{1}{2}$. Note that the point $(\chi,\tau)=(2,\tfrac{1}{2})$ is the junction point of these branches, and it is easy to verify that this point also lies on the curve \eqref{eq:boundary-curve}, the boundary curve of the region $C$ which is shown in light red in Figure~\ref{fig:RegionsPlot}. It follows that the branch in \eqref{eq:b-is-zero-branches} with the top sign is contained in the region $B$ and the branch with the bottom sign is contained in the region $C$ \textcolor{red}{[I can give a proof of this if necessary]}, see Figure~\ref{fig:b-is-zero}.
%\begin{figure}
%\includegraphics[width=0.5\linewidth]{b-is-zero-plot.pdf}
%\caption{The first quadrant in the $(\chi,\tau)$-plane; horizontal axis:  $\chi$, vertical axis:  $\tau$. The branch $\tau-\tfrac{1}{2} = \frac{1}{2}\sqrt{1-({\chi}/{2})^2}$ (dashed orange) of $b(\chi,\tau)=0$ lying in the region $B$ and the branch $\tau-\tfrac{1}{2} = -\frac{1}{2}\sqrt{1-({\chi}/{2})^2}$ (dashed green) of $b(\chi,\tau)=0$ lying in the region $C$. The purple colored dot is the point $(2,\tfrac{1}{2})$.}
%\label{fig:b-is-zero}
%\end{figure}
%This implies in particular that for given $\tau\in(\frac{1}{2},1)$, there exists a value of $\chi$ for which $b(\chi,\tau)=0$. For given $\tau\in(0,\frac{1}{2}]$, on the other hand, we have $b(\chi,\tau)>0$ for $(\chi,\tau)\in B$.
%whose union is equal to \eqref{eq:b-is-zero}.

%
%by setting t$(\chi,\tau)$ in the open first quadrant gives the following two branches for the curve $ (\chi^2 -16 \tau + 16\tau^2)=0$
%\begin{equation}
%\tau - \frac{1}{2} = \pm \frac{1}{4}\sqrt{4-\chi^2}
%\end{equation}
%\begin{align}
%\tau &= \frac{1}{4}(2-\sqrt{4-\chi^2})\\
%\tau &= \frac{1}{4}(2+\sqrt{4-\chi^2})
%\end{align}


  


%The analysis in this region involves construction of local parametrices inside disks centered at the points $a(\chi,\tau)$ and $b(\chi,\tau)$, and the facts established above about the configuration of the points $a(\chi,\tau)$ and $b(\chi,\tau)$ play a role in placing the branch cut $\Sigma_\omega$ of $\omega(\lambda)$ connecting the branch points $\lambda=\pm \ii$. In order to equip those local parametrices with the desired analyticity properties we need $\Sigma_\omega$ to avoid the above-mentioned neighborhoods of the points $a(\chi,\tau)$ and $b(\chi,\tau)$.
%In principle we may take $\Sigma_\omega$ to be the line segment $\Sigma_\mathrm{c}$ for $0<\tau\leq \frac{1}{2}$ since $a(\chi,\tau)< 0 < b(\chi,\tau)$ in this case, but this is no longer possible for all $\chi$ if $\tau>\frac{1}{2}$. 
%To treat these situations uniformly, we simply take $\Sigma_\omega$ to be a Schwarz-symmetric curve that connects $\lambda=\ii$ to (the midpoint) $\lambda=(a(\chi,\tau)+b(\chi,\tau))/2$ within $\Omega^+$ while avoiding $\Sigma_g$.


%\textcolor{red}{[What's written in this last paragraph will probably move to the beginning of this section.]}

%As in Section~\ref{sec:Schi-Stau}, the last step before constructing parametrices is to remove the $\lambda$-dependence from the jump matrix on $\Sigma_g$ given in \eqref{eq:W-jump-Sigma} with the help of a suitable Szeg\H{o} function.  We define $S(\lambda;\chi,\tau)$ as a function analytic for $\lambda\in\mathbb{C}\setminus\Sigma_g$ almost exactly as in Section~\ref{sec:Schi-Stau}, simply replacing $\omega_+(\lambda)$ in \eqref{eq:Szego-def-Schi-Stau} and \eqref{eq:gamma-def-Schi-Stau} by $\omega(\lambda)$.  In fact, since for $(\chi,\tau)\in \shelves$ the jump contour for $\omega(\lambda)$ is taken to lie to the right of $\Sigma_g$ whereas the latter is contained in the former for $(\chi,\tau)\in S_\chi\cup S_\tau$, what is called $\omega_+(\lambda)=\tilde{\omega}(\lambda)$ on $\Sigma_g$ in Section~\ref{sec:Schi-Stau} coincides exactly with $\omega(\lambda)$ in the present context.  Likewise, in the simplified formula \eqref{eq:gamma-explicit-Schi-Stau} for $\gamma(\chi,\tau)$ the integration contour can be written  as $\Sigma_\omega$ in the present context of $(\chi,\tau)\in \shelves$.  In particular, $\gamma(\chi,\tau)$ is a real analytic function of $(\chi,\tau)\in \shelves$, i.e., it is analytic on the whole exterior domain.  

%A preliminary step we take before construction of parametrices is to convert the jump condition \eqref{eq:W-jump-Sigma} on $\Sigma_g$ to one that does not depend on $\lambda$. To accomplish this, we introduce a scalar-valued function $S(\lambda)$ which is analytic in $\mathbb{C}\setminus\Sigma_g$ and bounded at the endpoints $\lambda=A(\chi,\tau)\pm \ii B(\chi,\tau)$ of $\Sigma_g$, satisfying $S(\lambda)\to 0$ as $\lambda\to\infty$, and admitting continuous boundary values on $\Sigma_g$ related by 
%\begin{equation}
%S_+(\lambda) + S_-(\lambda) - 2s \log(\omega(\lambda)) = 2 \ii {s}\gamma,\quad \lambda\in \Sigma_g, 
%\label{eq:Szego-jump}
%\end{equation}
%for some constant $\gamma\in\mathbb{C}$ which may depend on the parameters $\chi$ and $\tau$. 
%Here $\log(\cdot)$ is taken to be the principal branch, and we have $\omega(\lambda)^{2s} = \ee^{2s \log(\omega(\lambda))}$ with the right-hand side being well-defined for all $\lambda\in\Sigma_g$ including the endpoints since $\omega(\lambda)$ is nonzero, continuous and bounded (in fact, analytic) in a suitably small\footnote{Suitably small in the sense that the neighborhood leaves $\Sigma_\omega$ in its exterior.} neighborhood containing $\Sigma_g$. As the endpoints $\lambda_0(\chi,\tau)$ and $\lambda_0(\chi,\tau)^*$ of the jump contour $\Sigma_g$ are already determined in the construction of $g(\lambda;\chi,\tau)$, the only degree of freedom left to ensure the existence of such a function $S(\lambda)$ is the constant $\gamma$, which we will choose appropriately to satisfy the normalization.
%It is easy to verify that
%\begin{equation}
%S(\lambda)=S(\lambda;\chi,\tau)\defeq \frac{R(\lambda;\chi,\tau)}{2\pi \ii} \int_{\Sigma_g} \frac{2s \log(\omega(\eta))+  2\ii{s} \gamma}{R_+(\eta; \chi,\tau)(\eta-\lambda)}\dd \eta
%\label{eq:Szego-def}
%\end{equation}
%is analytic in $\mathbb{C}\setminus\Sigma_g$, satisfies the jump condition given in \eqref{eq:Szego-jump}, and it is bounded as $\lambda \to A(\chi,\tau) \pm \ii B(\chi,\tau)$. Moreover, $S(\lambda)\to 0$ as $\lambda\to\infty$ provided that the constant $\gamma=\gamma(\chi,\tau)$ is chosen to ensure
%\begin{equation}
%\int_{\Sigma_g} \frac{\log(\omega(\eta))+ \ii \gamma(\chi,\tau)}{R_+(\eta;\chi,\tau)}\, \dd \eta = 0.
%\label{eq:gamma-condition}
%\end{equation}
%Recalling from \eqref{eq:integral-R-plus} that
%\begin{equation}
%\int_{\Sigma_g} \frac{\dd \eta}{R_+(\eta;\chi,\tau)} = -\ii \pi
%\end{equation}
%is nonzero, we see that the condition \eqref{eq:gamma-condition} indeed determines $\gamma=\gamma(\chi,\tau)$ via the ratio:
%\begin{equation}
%\gamma(\chi,\tau) \defeq \ii  \frac{\displaystyle \int_{\Sigma_g} \frac{ \log(\omega(\eta)) }{R_+(\eta;\chi,\tau)}\dd \eta}{\displaystyle\int_{\Sigma_g}  \frac{\dd \eta}{R_+(\eta;\chi,\tau)}} = -\frac{1}{\pi} \displaystyle \int_{\Sigma_g} \frac{ \log(\omega(\eta)) }{R_+(\eta;\chi,\tau)}\dd \eta.
%\label{eq:gamma-def-bun}
%\end{equation}
%To compute $\gamma(\chi,\tau)$, we let $L$ to be a clockwise-oriented loop surrounding the branch cut $\Sigma_g$ of $R(\lambda;\chi,\tau)$ excluding the branch cut $\Sigma_\omega$ of $\omega(\lambda)$ which connects the points $\lambda=\pm \ii$. Taking $L'$ to be a counter-clockwise-oriented contour that encircles $\Sigma_\omega$ but excludes $\Sigma_g$ and using the fact that the integrand in the last integral in \eqref{eq:gamma-def-bun} is integrable at $\lambda=\infty$, we obtain:
%\begin{equation}
%\gamma(\chi,\tau) = -\frac{1}{2\pi} \oint_L \frac{\log(\omega(\eta))}{R(\eta;\chi,\tau)}\dd \eta
%= -\frac{1}{2\pi} \oint_{L'} \frac{\log(\omega(\eta))}{R(\eta;\chi,\tau)}\dd \eta
%= -\frac{1}{8\pi} \oint_{L'} \log\left(\frac{\eta-\ii}{\eta+\ii}\right) \frac{1}{R(\eta;\chi,\tau)}\dd \eta,
%\end{equation}
%where we used the fact that $\omega(\lambda)^4 = ( (\lambda-\ii)/(\lambda+\ii))$ with the logarithm having $\Sigma_\omega$ to be its branch cut. We may now collapse $L'$ to both sides of the branch cut $\Sigma_{\omega}$, where $R(\lambda;\chi,\tau)$ is analytic but the boundary values of the logarithm differ by $2\pi \ii$, and see that
%\begin{equation}
%\gamma(\chi,\tau) = -\frac{1}{4\ii} \int_{\Sigma_\omega} \frac{1}{R(\eta;\chi,\tau)}\dd \eta =  \frac{\ii}{4} \int_{-\ii}^{\ii} \frac{1}{R(\eta;\chi,\tau)}\dd \eta,
%\label{eq:gamma-explicit-bun}
%\end{equation}
%where the latter equality follows because the line segment from $-\ii$ to $\ii$ remains to the right of $\Sigma_\omega$, which  remains to the right of $\Sigma_g$ by definition. We see from \eqref{eq:gamma-explicit-bun} that $\gamma(\chi,\tau)$ has a purely imaginary value.

%With the Szeg\H{o} function defined as indicated, we now make a global substitution by setting
%\begin{equation}
%\mathbf{X}^{(k)}(\lambda;\chi,\tau) \defeq  \mathbf{W}^{(k)}(\lambda;\chi,\tau)\ee^{S(\lambda;\chi,\tau)\sigma_3}
%\end{equation}
%in the whole domain of analyticity $\mathbf{W}^{(k)}(\lambda;\chi,\tau)$. The boundary values of $\mathbf{X}^{(k)}(\lambda)=\mathbf{X}^{(k)}(\lambda;\chi,\tau)$ along $\Sigma_g$ are related by a constant jump matrix:
%\begin{equation}
%\mathbf{X}^{(k)}_+(\lambda) = \mathbf{X}^{(k)}_-(\lambda) 
%\begin{bmatrix} 0 &  s \ee^{-2\ii (n \kappa(\chi,\tau) + {s}\gamma(\chi,\tau))} \\ - s \ee^{2\ii (n \kappa(\chi,\tau) + {s}\gamma(\chi,\tau))} &  0 \end{bmatrix},\quad \lambda\in \Sigma_g= \Sigma_g^+ \cup \Sigma_g^-,
%\label{eq:X-twist-jump}
%\end{equation}
%The remaining jump conditions \eqref{eq:W-jump-Gamma-plus-L}--\eqref{eq:W-jump-Sigma-minus-R} are modified via conjugations by the diagonal matrix $\ee^{S(z;w,t)\sigma_3}$ and take the form:
%\begin{equation}
%\mathbf{X}^{(k)}_{+}(\lambda) = \mathbf{X}^{(k)}_{-}(\lambda)
%\begin{bmatrix} 1 & 0 \\ -s \omega(\lambda)^{-2s}\ee^{2S(\lambda;\chi,\tau)}\ee^{2n h(\lambda;\chi,\tau)}& 1 \end{bmatrix},\quad \lambda\in C^+_{\Gamma,L},
%\label{eq:X-jump-Gamma-plus-L}
%\end{equation}
%\begin{equation}
%\mathbf{X}^{(k)}_{+}(\lambda) = \mathbf{X}^{(k)}_{-}(\lambda)
%\begin{bmatrix} 1 & \frac{s}{2} \omega(\lambda)^{2s}\ee^{-2S(\lambda;\chi,\tau)}\ee^{-2n h(\lambda;\chi,\tau)} \\ 0 & 1 \end{bmatrix},\quad \lambda\in C^+_{\Gamma,R},
%\end{equation}
%\begin{equation}
%\mathbf{X}^{(k)}_{+}(\lambda) = \mathbf{X}^{(k)}_{-}(\lambda)
%2^{\sigma_3},\quad \lambda\in I ,
%\label{eq:X-I-jump}
%\end{equation}
%\begin{equation}
%\mathbf{X}^{(k)}_{+}(\lambda) = \mathbf{X}^{(k)}_{-}(\lambda)
%\begin{bmatrix} 1 & 0 \\ -\frac{s}{2} \omega(\lambda)^{-2s} \ee^{2S(\lambda;\chi,\tau)} \ee^{2n h(\lambda;\chi,\tau)} & 1 \end{bmatrix} ,\quad \lambda\in C^-_{\Gamma,R},
%\end{equation}
%\begin{equation}
%\mathbf{X}^{(k)}_{+}(\lambda) = \mathbf{X}^{(k)}_{-}(\lambda)
%\begin{bmatrix} 1 &  s  \omega(\lambda)^{2s}\ee^{-2S(\lambda;\chi,\tau)} \ee^{-2n h(\lambda;\chi,\tau)} \\ 0 & 1 \end{bmatrix} ,\quad \lambda\in C^-_{\Gamma,L},
%\end{equation}
%\begin{equation}
%\mathbf{X}^{(k)}_{+}(\lambda) = \mathbf{X}^{(k)}_{-}(\lambda)
%\begin{bmatrix} 1 & -s \omega(\lambda)^{2s} \ee^{-2S(\lambda;\chi,\tau)} \ee^{-2n h(\lambda;\chi,\tau)} \\ 0 & 1 \end{bmatrix},\quad \lambda\in C^+_{\Sigma,L},
%\end{equation}
%\begin{equation}
%\mathbf{X}^{(k)}_{+}(\lambda) = \mathbf{X}^{(k)}_{-}(\lambda)
%\begin{bmatrix} 1 & -\frac{s}{2}\omega(\lambda)^{2 s}\ee^{-2S(\lambda;\chi,\tau)}\ee^{-2n h(\lambda;\chi,\tau)} \\ 0 & 1 \end{bmatrix},\quad \lambda\in C^+_{\Sigma,R},
%\end{equation}
%\begin{equation}
%\mathbf{X}^{(k)}_{+}(\lambda) = \mathbf{X}^{(k)}_{-}(\lambda)
%\begin{bmatrix} 1 & 0 \\ s \omega(\lambda)^{-2s}\ee^{2S(\lambda;\chi,\tau)} \ee^{2n h(\lambda;\chi,\tau)} & 1 \end{bmatrix},\quad \lambda\in C^-_{\Sigma,L},
%\end{equation}
%and, finally,
%\begin{equation}
%\mathbf{X}^{(k)}_{+}(\lambda) = \mathbf{X}^{(k)}_{-}(\lambda)
%\begin{bmatrix} 1 & 0 \\ \frac{s}{2} \omega(\lambda)^{-2s}\ee^{2S(\lambda;\chi,\tau)} \ee^{2n h(\lambda;\chi,\tau)} & 1 \end{bmatrix},\quad \lambda\in C^-_{\Sigma,R}.
%\label{eq:X-jump-Sigma-minus-R}
%\end{equation}
%Note that all of the jump matrices above except for those supported on $I\cup\Sigma_g$ tend to the identity matrix as $n\to +\infty$ for $\lambda$ on the relevant supporting arcs away from the points $\lambda_{0}(\chi,\tau)$, $\lambda_{0}(\chi,\tau)^{*}$, $a(\chi,\tau)$ and $b(\chi,\tau)$.

\subsection{Parametrix construction}
\subsubsection{Outer parametrix construction} 
We start with construction of an outer parametrix denoted $\dot{\mathbf{W}}^\mathrm{out}(\lambda)\defeq \dot{\mathbf{W}}^\mathrm{out}(\lambda;\chi,\tau,\mathbf{Q}^{-s},M)$ satisfying exactly the jump conditions on $I$ and $\Sigma_g$ (cf., \eqref{eq:W-jump-I} and \eqref{eq:W-jump-Sigma-g}) that do not become asymptotically trivial as $M\to +\infty$. The procedure follows closely the construction in \cite[Section 4.2.2]{BilmanLM20} and it can be viewed as a combination of the outer parametrices constructed in Section~\ref{sec:channels} and in Section~\ref{sec:Schi-Stau} together with a new diagonal factor which is intrinsic to \shelves. Indeed, the jump condition \eqref{eq:W-jump-I} on $I$ is identical for $(\chi,\tau)\in\channels$ and $(\chi,\tau)\in\shelves$, hence we employ the outer parametrix \eqref{eq:Channels-Tout} to write $\dot{\mathbf{W}}^\mathrm{out}(\lambda)$ as
 %Recalling from Section~\ref{s:channels} the corresponding outer parametrix that satisfies the jump condition on $I$ 
%We write $\dot{\mathbf{W}}^\mathrm{out}(\lambda)$ in the for
\begin{equation}
%\dot{\mathbf{W}}^\mathrm{out}(\lambda;\chi,\tau)=\mathbf{G}(\lambda;\chi,\tau)\left(\frac{\lambda-a(\chi,\tau)}{\lambda-b(\chi,\tau)}\right)^{\ii p\sigma_3},\quad p\defeq \frac{\ln(2)}{2\pi},
\dot{\mathbf{W}}^\mathrm{out}(\lambda)=\mathbf{J}(\lambda)\left(\frac{\lambda-a(\chi,\tau)}{\lambda-b(\chi,\tau)}\right)^{\ii p\sigma_3},
\label{eq:W-out-G}
\end{equation}
where the power function is defined as the principal branch and $p>0$ was defined in \eqref{eq:Channels-Tout}.
Then, $\mathbf{J}(\lambda)=\mathbf{J}(\lambda;\chi,\tau,\mathbf{Q}^{-s},M)$ extends analytically to $I$, and we will assume that it is bounded near $\lambda=a(\chi,\tau),b(\chi,\tau)$ in particular making it analytic at $\lambda=b(\chi,\tau)$.  Therefore, $\mathbf{J}(\lambda)$ is analytic for $\lambda\in\mathbb{C}\setminus\Sigma_g$ and tends to the identity as $\lambda\to\infty$.
Across $\Sigma_g$, the constant jump condition \eqref{eq:W-jump-Sigma-g} required of $\dot{\mathbf{W}}^\mathrm{out}(\lambda)$ becomes modified for $\mathbf{J}(\lambda)$:
\begin{equation}
\mathbf{J}_+(\lambda)=\mathbf{J}_-(\lambda)\left(\frac{\lambda-a(\chi,\tau)}{\lambda-b(\chi,\tau)}\right)^{\ii p\sigma_3}
\begin{bmatrix}0 & s \ee^{-2\ii M\kappa(\chi,\tau)}\\
- s \ee^{ 2\ii M\kappa(\chi,\tau)} & 0\end{bmatrix}
\left(\frac{\lambda-a(\chi,\tau)}{\lambda-b(\chi,\tau)}\right)^{-\ii p\sigma_3},\quad \lambda\in \Sigma_g,
\end{equation}
and we will convert this back into a constant jump condition on $\Sigma_g$ alone by introducing a \emph{Szeg{\H o} function} $K(\lambda;\chi,\tau)$, which we define by
\begin{equation}
%K(\lambda;\chi,\tau)\defeq \ii p\log\left(\frac{\lambda-a(\chi,\tau)}{\lambda-b(\chi,\tau)}\right)+\ii p R(\lambda;\chi,\tau) \int_{a(\chi,\tau)}^{b(\chi,\tau)}\frac{\dd \eta}{R(\eta;\chi,\tau)(\eta-\lambda)}+\ii\mu(\chi,\tau),
K(\lambda;\chi,\tau)\defeq  p\log\left(\frac{\lambda-a(\chi,\tau)}{\lambda-b(\chi,\tau)}\right)+ p R(\lambda;\chi,\tau) \int_{a(\chi,\tau)}^{b(\chi,\tau)}\frac{\dd \eta}{R(\eta;\chi,\tau)(\eta-\lambda)}, %+ \mu(\chi,\tau),
\label{eq:K-def}
\end{equation}
in which the logarithm is taken to be the principal branch, $-\pi<\mathrm{Im}(\log(\cdot))<\pi$. 
It is straightforward to confirm that $K(\lambda;\chi,\tau)$ has the following properties.
Recalling the definition \eqref{eq:mu-formula-intro} of the constant $\mu(\chi,\tau)$, $K(\lambda;\chi,\tau) = -\mu(\chi,\tau) + O(\lambda^{-1})$ as $\lambda\to\infty$. Despite appearances, $K(\lambda;\chi,\tau)$ does not have a jump across $I$ as is easily confirmed by comparing the boundary values of the logarithm and using the Plemelj formula.  
The apparent singularities at $\lambda=a(\chi,\tau),b(\chi,\tau)$ are removable, so the domain of analyticity for $K(\lambda;\chi,\tau)$ is $\lambda\in\mathbb{C}\setminus\Sigma_g$, and $K(\lambda;\chi,\tau)$ takes continuous boundary values on $\Sigma_g$, including at the endpoints.  These boundary values are related by the jump condition
\begin{equation}
%K_+(\lambda;\chi,\tau)+K_-(\lambda;\chi,\tau)=2\ii p\log\left(\frac{\lambda-a(\chi,\tau)}{\lambda-b(\chi,\tau)}\right)+2\ii\mu(\chi,\tau),\quad \lambda\in\Sigma_g.
K_+(\lambda;\chi,\tau)+K_-(\lambda;\chi,\tau)=2 p\log\left(\frac{\lambda-a(\chi,\tau)}{\lambda-b(\chi,\tau)}\right),\quad \lambda\in\Sigma_g.
\label{eq:K-jump}
\end{equation}
One can indeed check that $K(\lambda;\chi,\tau)$ has the alternate representation obtained by the Plemelj formula:
\begin{equation}
K(\lambda;\chi,\tau) =  \frac{R(\lambda;\chi,\tau)}{2\pi \ii} \int_{\Sigma_g} \log\left( \frac{\eta - a(\chi,\tau)}{\eta - b(\chi,\tau)}\right) \frac{2 p  \dd \eta}{R_+(\eta; \chi,\tau)(\eta - \lambda)},
\label{eq:K-Plemelj}
\end{equation}
which confirms the properties stated above. The values of $K(\lambda;\chi,\tau)$ at $\lambda=a(\chi,\tau),b(\chi,\tau)$ will be useful in obtaining the asymptotic formula for $q( M \chi, M\tau;\mathbf{Q}^{-s},M)$, and they can easily be computed from the representation \eqref{eq:K-Plemelj}. 
%Considering an upward oriented arc $C^\sharp$ that lies to the left of $\Sigma_g$ connecting the branch points $\eta=A(\chi,\tau) \pm \ii B(\chi,\tau)$, and 
Using
\begin{equation}
R(b(\chi,\tau);\chi,\tau) = |b(\chi,\tau)-\lambda_0(\chi,\tau)|\quad\text{and}\quad R_-(a(\chi,\tau);\chi,\tau) = |a(\chi,\tau)-\lambda_0(\chi,\tau)|,
\label{eq:R-a-b}
\end{equation}
we arrive at the formul\ae\ \eqref{eq:intro-Ka}--\eqref{eq:intro-Kb} for $K_a(\chi,\tau)\defeq K_-(a(\chi,\tau);\chi,\tau)$ and $K_b(\chi,\tau)\defeq K(b(\chi,\tau);\chi,\tau)$ respectively.
%while passing from a point at a finite-distance away from (and to the left of) $\eta = a(\chi,\tau)$, 
%we obtain
%\begin{align}
%K_b(\chi,\tau) &\defeq  K(b(\chi,\tau); \chi,\tau) = 
%\frac{|b(\chi,\tau) - \lambda_0(\chi,\tau)|}{2\pi \ii} \int_{C^{\sharp}}  \log\left( \frac{\eta - a(\chi,\tau)}{\eta - b(\chi,\tau)}\right)  \frac{2 p \dd\eta}{R(\eta; \chi,\tau)(\eta - b(\chi,\tau))},\label{eq:K-at-b}\\
%K_a(\chi,\tau) &\defeq   K_{-}(a(\chi,\tau); \chi,\tau) = 
%\frac{|a(\chi,\tau) - \lambda_0(\chi,\tau)|}{2\pi \ii} \int_{C^{\sharp}} \log\left( \frac{\eta - a(\chi,\tau)}{\eta - b(\chi,\tau)}\right) \frac{2 p  \dd\eta }{R(\eta; \chi,\tau)(\eta - a(\chi,\tau))}, \label{eq:K-at-a}
%\end{align}

%Using $R(a(\chi,\tau);\chi,\tau)=...$ ... we arrive at the formul\ae\ \eqref{eq:intro-Ka}--\eqref{eq:intro-Kb} for $K_a(\chi,\tau)\defeq K_-(a(\chi,\tau);\chi,\tau)$ and $K_b(\chi,\tau)\defeq K(b(\chi,\tau);\chi,\tau)$ respectively.
%
%where we have substituted 
Preserving the normalization at infinity, we introduce $K(\lambda;\chi,\tau)$ in the construction of $\dot{\mathbf{W}}^\text{out}(\lambda)$ by
\begin{equation}
%\mathbf{G}(\lambda;\chi,\tau)=\mathbf{H}(\lambda;\chi,\tau)\ee^{-K(\lambda;\chi,\tau)\sigma_3},
\mathbf{J}(\lambda)=\mathbf{L}(\lambda)\ee^{-\ii (K(\lambda;\chi,\tau)+ \mu(\chi,\tau) )\sigma_3}.
\label{eq:G-H}
\end{equation}
It then follows that $\mathbf{L}(\lambda)=\mathbf{L}(\lambda;\chi,\tau,\mathbf{Q}^{-s},M)$ is a matrix function analytic for $\lambda\in\mathbb{C}\setminus\Sigma_g$ that tends to $\mathbb{I}$ as $\lambda\to\infty$, and that satisfies the jump condition
%\begin{equation}
%\mathbf{H}_+(\lambda)=\mathbf{H}_-(\lambda)\begin{bmatrix}0 & s \ee^{-2\ii (n\kappa(\chi,\tau)+ {s}\gamma(\chi,\tau)+\mu(\chi,\tau))}\\
%-s \ee^{2\ii (n\kappa(\chi,\tau)+ {s}\gamma(\chi,\tau)+\mu(\chi,\tau))} & 0\end{bmatrix},\quad \lambda\in\Sigma_g.
%\end{equation}
%\begin{equation}
%\mathbf{L}_+(\lambda)=\mathbf{L}_-(\lambda)
%\ee^{-\ii \mu(\chi,\tau)\sigma_3}
%\begin{bmatrix}0 & s \ee^{-2\ii M\kappa(\chi,\tau)}\\
%-s \ee^{2\ii M\kappa(\chi,\tau)} & 0\end{bmatrix}
%\ee^{\ii \mu(\chi,\tau)\sigma_3}
%,\quad \lambda\in\Sigma_g.
%\label{eq:H-jump-Sigma-g}
%\end{equation}
\begin{equation}
\mathbf{L}_+(\lambda)=\mathbf{L}_-(\lambda)
%\ee^{-\ii \mu(\chi,\tau)\sigma_3}
\begin{bmatrix}0 & s \ee^{-2\ii (M\kappa(\chi,\tau) + \mu(\chi,\tau) )}\\
-s \ee^{2\ii (M\kappa(\chi,\tau) + \mu(\chi,\tau) )} & 0\end{bmatrix}
%\ee^{\ii \mu(\chi,\tau)\sigma_3}
,\quad \lambda\in\Sigma_g.
\label{eq:H-jump-Sigma-g}
\end{equation}
%Notice how similar the central factor in this jump matrix is to the one in \eqref{eq:T-jump-N-Schi-Stau-ALT}. Indeed, factoring out the exponential factors in \eqref{eq:H-jump-Sigma-g} and relating the remaining off-diagonal matrix to the one in \eqref{eq:H-jump-Sigma-g} shows that $\mathbf{L}(\lambda)$ is given by
$\mathbf{L}(\lambda)$ is given by
\begin{equation}
\mathbf{L}(\lambda)\defeq 
\ee^{-\ii \frac{1}{4}s \pi \sigma_3}
\ee^{-\ii (M\kappa(\chi,\tau) + \mu(\chi,\tau) )\sigma_3}
\mathbf{Q}y(\lambda;\chi,\tau)^{\sigma_3}\mathbf{Q}^{-1}
\ee^{\ii (M\kappa(\chi,\tau)+  \mu(\chi,\tau) )\sigma_3}
\ee^{\ii \frac{1}{4}s \pi \sigma_3},
\label{eq:H-def}
\end{equation}
where $y(\lambda;\chi,\tau)$ is defined in \eqref{eq:y-def}.
This solution clearly relates to (232) in exactly the same way that \eqref{eq:outer-parametrix-Schi-Stau-ALT} relates to \eqref{eq:T-jump-N-Schi-Stau-ALT} with $\mathbf{T}$ replaced by $\mathbf{W}$, and obviously $\mathbf{L}(\lambda)\to\mathbb{I}$ as $\lambda\to\infty$.
%Observe that $\mathbf{L}(\lambda)$ relates to \eqref{eq:outer-parametrix-Schi-Stau-ALT} through conjugation by a diagonal constant matrix other than the difference in the phases $M\gamma(\chi,\tau)$ versus $M\kappa(\chi,\tau) + \mu(\chi,\tau)$, and the leftmost factor in \eqref{eq:H-def} is to ensure $\mathbf{L}(\lambda)\to \mathbb{I}$ as $\lambda\to\infty$.
% in the form
%\begin{equation}
%%\mathbf{G}(\lambda;\chi,\tau)=\mathbf{H}(\lambda;\chi,\tau)\ee^{-K(\lambda;\chi,\tau)\sigma_3},
%\mathbf{J}(\lambda)=\mathbf{H}(\lambda)\ee^{-\ii K(\lambda;\chi,\tau)\sigma_3},
%\label{eq:G-H}
%\end{equation}
%where $K(\lambda;\chi,\tau)$ is given, as in \cite[Section 4.2.2]{BilmanLM20} modulo a factor of $\ii$, by 
%\begin{equation}
%%K(\lambda;\chi,\tau)\defeq \ii p\log\left(\frac{\lambda-a(\chi,\tau)}{\lambda-b(\chi,\tau)}\right)+\ii p R(\lambda;\chi,\tau) \int_{a(\chi,\tau)}^{b(\chi,\tau)}\frac{\dd \eta}{R(\eta;\chi,\tau)(\eta-\lambda)}+\ii\mu(\chi,\tau),
%K(\lambda;\chi,\tau)\defeq  p\log\left(\frac{\lambda-a(\chi,\tau)}{\lambda-b(\chi,\tau)}\right)+ p R(\lambda;\chi,\tau) \int_{a(\chi,\tau)}^{b(\chi,\tau)}\frac{\dd \eta}{R(\eta;\chi,\tau)(\eta-\lambda)}+ \mu(\chi,\tau),
%\label{eq:K-def}
%\end{equation}
%in which the logarithm is taken to be the principal branch, $-\pi<\mathrm{Im}(\log(\cdot))<\pi$, and where the constant $\mu(\chi,\tau)$ is given by
%\begin{equation}
%\mu(\chi,\tau)\defeq p\int_{a(\chi,\tau)}^{b(\chi,\tau)}\frac{\dd \eta}{R(\eta;\chi,\tau)}>0.
%\label{eq:mu-def-bun}
%\end{equation}
%It is straightforward to confirm that $K(\lambda;\chi,\tau)$ has the following properties.  By definition of $\mu(\chi,\tau)$, it satisfies $K(\lambda;\chi,\tau)=O(\lambda^{-1})$ as $\lambda\to\infty$.  
%Despite appearances, $K$ does not have a jump across $(a(\chi,\tau),b(\chi,\tau))$ as is easily confirmed by comparing the boundary values of the logarithm and using the Plemelj formula.  
%The apparent singularities at $\lambda=a(\chi,\tau),b(\chi,\tau)$ are removable, so the domain of analyticity for $K(\lambda;\chi,\tau)$ is $\lambda\in\mathbb{C}\setminus\Sigma_g$, and $K(\lambda;\chi,\tau)$ takes continuous boundary values on $\Sigma_g$, including at the endpoints.  These boundary values are related by the jump condition
%\begin{equation}
%%K_+(\lambda;\chi,\tau)+K_-(\lambda;\chi,\tau)=2\ii p\log\left(\frac{\lambda-a(\chi,\tau)}{\lambda-b(\chi,\tau)}\right)+2\ii\mu(\chi,\tau),\quad \lambda\in\Sigma_g.
%K_+(\lambda;\chi,\tau)+K_-(\lambda;\chi,\tau)=2 p\log\left(\frac{\lambda-a(\chi,\tau)}{\lambda-b(\chi,\tau)}\right)+2\mu(\chi,\tau),\quad \lambda\in\Sigma_g.
%\label{eq:K-jump}
%\end{equation}
%It then follows that $\mathbf{H}(\lambda)=\mathbf{H}(\lambda;\chi,\tau,\mathbf{Q}^{-s},M)$ is a matrix function analytic for $\lambda\in\mathbb{C}\setminus\Sigma_g$ that tends to $\mathbb{I}$ as $\lambda\to\infty$, and that satisfies the jump condition
%%\begin{equation}
%%\mathbf{H}_+(\lambda)=\mathbf{H}_-(\lambda)\begin{bmatrix}0 & s \ee^{-2\ii (n\kappa(\chi,\tau)+ {s}\gamma(\chi,\tau)+\mu(\chi,\tau))}\\
%%-s \ee^{2\ii (n\kappa(\chi,\tau)+ {s}\gamma(\chi,\tau)+\mu(\chi,\tau))} & 0\end{bmatrix},\quad \lambda\in\Sigma_g.
%%\end{equation}
%\begin{equation}
%\mathbf{H}_+(\lambda)=\mathbf{H}_-(\lambda)
%\begin{bmatrix}0 & s \ee^{-2\ii (M\kappa(\chi,\tau)+ \mu(\chi,\tau))}\\
%-s \ee^{2\ii (M\kappa(\chi,\tau)+\mu(\chi,\tau))} & 0\end{bmatrix},\quad \lambda\in\Sigma_g.
%\end{equation}
%Comparing with the jump condition \eqref{eq:X-jump-N-Schi-Stau}, 
%It is now straightforward to solve for $\mathbf{H}(\lambda)$ by 
%%adaptation of the formula \eqref{eq:outer-parametrix-Schi-Stau} to augment the phase $n\kappa(\chi,\tau)+s\gamma(\chi,\tau)$ with $\mu(\chi,\tau)$. \textcolor{red}{(The previous sentence needs to be modified according to how the outer parametrix is constructed in $S_\chi\cup S_\tau$.)} The resulting formula reads
%diagonalizing the constant jump matrix, which has eigenvalues $\pm \ii $. All solutions of the jump condition for $\mathbf{H}(\lambda)$ have singularities at the endpoints of $\Sigma_g$, and we select the unique solution with the mildest rate of growth as $\lambda\to \lambda_0(\chi,\tau),\lambda_0(\chi,\tau)^{*}$:
%%\begin{multline}
%%\mathbf{H}(\lambda;\chi,\tau)\defeq\ee^{-\ii (M\kappa(\chi,\tau)+ \mu(\chi,\tau) )\sigma_3}
%%\ii^{\frac{1}{2}(1-s)\sigma_3}
%%\mathbf{O}
%%\left(\frac{\lambda-\lambda_0(\chi,\tau)}{\lambda-\lambda_0(\chi,\tau)^*}\right)^{ \frac{1}{4} \sigma_3}
%%\mathbf{O}^{-1}
%%\ii^{-\frac{1}{2}(1-s)\sigma_3}
%%\\
%%\cdot
%%\ee^{\ii (n\kappa(\chi,\tau)+  {s} \gamma(\chi,\tau) + \mu(\chi,\tau) )\sigma_3},\\
%%\label{eq:H-def}
%%\end{multline}
%%\begin{multline}
%%\mathbf{H}(\lambda;\chi,\tau)\defeq\ee^{-\ii (M\kappa(\chi,\tau) + \mu(\chi,\tau) )\sigma_3}
%%\ii^{\frac{1}{2}(1-s)\sigma_3}
%%\mathbf{O}
%%\left(\frac{\lambda-\lambda_0(\chi,\tau)}{\lambda-\lambda_0(\chi,\tau)^*}\right)^{ \frac{1}{4} \sigma_3}
%%\\
%%\cdot
%%\mathbf{O}^{-1}
%%\ii^{-\frac{1}{2}(1-s)\sigma_3}
%%\ee^{\ii (M\kappa(\chi,\tau)+  \mu(\chi,\tau) )\sigma_3},
%%\label{eq:H-def}
%%\end{multline}
%\begin{equation}
%\mathbf{H}(\lambda)\defeq\ee^{-\ii (M\kappa(\chi,\tau) + \mu(\chi,\tau) )\sigma_3}
%\ii^{\frac{1}{2}(1-s)\sigma_3}
%\mathbf{O}
%q(\lambda;\chi,\tau)^{\sigma_3}
%\mathbf{O}^{-1}
%\ii^{-\frac{1}{2}(1-s)\sigma_3}
%\ee^{\ii (M\kappa(\chi,\tau)+  \mu(\chi,\tau) )\sigma_3},
%\label{eq:H-def}
%\end{equation}
%where $\mathbf{O}$ is defined in \eqref{eq:O-def-Schi-Stau}, and where $y(\lambda;\chi,\tau)$ is the function analytic for $\lambda\in\mathbb{C}\setminus\Sigma_g$ determined by the conditions that $y(\lambda;\chi,\tau)\to 1$ as $\lambda\to\infty$ and 
%\begin{equation}
%y(\lambda;\chi,\tau)^4=\frac{\lambda-\lambda_0(\chi,\tau)}{\lambda-\lambda_0(\chi,\tau)^*}.
%\end{equation}
%%\begin{equation}
%%\mathbf{O}\defeq \frac{1}{\sqrt{2}}\begin{bmatrix}1 & \ii \\ \ii &  1\end{bmatrix},\quad {\det(\mathbf{U})=1}.
%%\end{equation}
%%\textcolor{PineGreen}{
%%[TO-BE-REMOVED] There are various other ways to write $\mathbf{H}(\lambda)$ exploiting the fact that $s^2=1$. For instance, avoid the conjugation by $\ii^{\frac{1}{2}(1-s)\sigma_3}$ in the central factor since $s=1/s$ and write
%%\begin{equation}
%%\mathbf{H}(\lambda;\chi,\tau)=\ee^{-\ii (n\kappa(\chi,\tau)+  {s} \gamma(\chi,\tau) + \mu(\chi,\tau) )\sigma_3}
%%\mathbf{U}\left(\frac{\lambda-\lambda_0(\chi,\tau)}{\lambda-\lambda_0(\chi,\tau)^*}\right)^{ s \sigma_3/4}\mathbf{U}^{-1}
%%\ee^{\ii (n\kappa(\chi,\tau)+  {s} \gamma(\chi,\tau) + \mu(\chi,\tau) )\sigma_3},\\
%%%\label{eq:H-def}
%%\end{equation}
%%where
%%\begin{equation}
%%\mathbf{U}\defeq \frac{1}{\sqrt{2}}\begin{bmatrix}1 & \ii \\ \ii &  1\end{bmatrix},\quad {\det(\mathbf{U})=1}.
%%\end{equation}
%%or can write:
%%\begin{equation}
%%\mathbf{H}(\lambda;\chi,\tau)=\ee^{-\ii (n\kappa(\chi,\tau)+  {s} \gamma(\chi,\tau) + \mu(\chi,\tau) )\sigma_3}
%%\mathbf{U}\left(\frac{\lambda-\lambda_0(\chi,\tau)}{\lambda-\lambda_0(\chi,\tau)^*}\right)^{  \sigma_3/4}\mathbf{U}^{-1}
%%\ee^{\ii (n\kappa(\chi,\tau)+  {s} \gamma(\chi,\tau) + \mu(\chi,\tau) )\sigma_3},\\
%%\end{equation}
%%where
%%\begin{equation}
%%\mathbf{U}\defeq \frac{1}{\sqrt{2}}\begin{bmatrix}1 & \ii {s} \\ \ii  {s} &  1\end{bmatrix},\quad {\det(\mathbf{U})=1},
%%\end{equation}
%%depending on where we would like $s$ to appear.
%%}
%%Here, the power function in the central factor is defined to be analytic for $\lambda\in\mathbb{C}\setminus\Sigma_g$ and to tend to $\mathbb{I}$ as $\lambda\to\infty$.  
Combining \eqref{eq:W-out-G}, \eqref{eq:G-H}, and \eqref{eq:H-def} completes the construction of the outer parametrix $\dot{\mathbf{W}}^\mathrm{out}(\lambda)$:
%\begin{equation}
%\dot{\mathbf{X}}^\mathrm{out}(\lambda;\chi,\tau) = \mathbf{H}(\lambda;\chi,\tau) \ee^{-K(\lambda;\chi,\tau)\sigma_3}\left(\frac{\lambda-a(\chi,\tau)}{\lambda-b(\chi,\tau)}\right)^{\ii p\sigma_3}
%\label{eq:X-out-full}
%\end{equation}
%\textcolor{red}{or in the alternate approach:
\begin{equation}
\dot{\mathbf{W}}^\mathrm{out}(\lambda) = \mathbf{L}(\lambda) \ee^{-\ii ( K(\lambda;\chi,\tau) + \mu(\chi,\tau) )\sigma_3}\left(\frac{\lambda-a(\chi,\tau)}{\lambda-b(\chi,\tau)}\right)^{\ii p\sigma_3}.
\label{eq:W-out-full}
\end{equation}
%}
Note that the only dependence on $M$ enters via the oscillatory factors $\ee^{\pm \ii M \kappa(\chi,\tau)\sigma_3}$ in $\mathbf{L}(\lambda)$. Thus, $\dot{\mathbf{W}}^{\text{out}}(\lambda)=\dot{\mathbf{W}}^{\text{out}}(\lambda;\chi,\tau,\mathbf{Q}^{-s},M)$ is bounded as $M\to+\infty$, provided that $\lambda$ is bounded away from $\lambda_0(\chi,\tau)$ and $\lambda_0(\chi,\tau)^*$.


While the outer parametrix exactly satisfies the same jump conditions satisfied by $\mathbf{W}(\lambda)$ on $\Sigma_g$ and $I$, it is discontinuous near the endpoints of these arcs. Thus, the problem at hand
%The problem at hand in this section 
requires four inner parametrices $\dot{\mathbf{W}}^a(\lambda)$, $\dot{\mathbf{W}}^b(\lambda)$, $\dot{\mathbf{W}}^{\lambda_0}(\lambda)$, and $\dot{\mathbf{W}}^{\lambda_0^*}(\lambda)$ to be defined in disks $D_{\lambda}(\delta)$, centered at the points $\lambda=a(\chi,\tau)$, $b(\chi,\tau)$, $\lambda_0(\chi,\tau)$, $\lambda_0^*(\chi,\tau)$, respectively, where $\delta=\delta(\chi,\tau)>0$ is chosen sufficiently small but independent of $M$. We take the circular boundaries of these disks to have clockwise orientation.

\subsubsection{Inner parametrix construction near the points $a(\chi,\tau)$ and $b(\chi,\tau)$} 
%For the purposes of the local analysis that follows it is convenient to define 
%\begin{equation}
%\tilde{h}(\lambda;\chi,\tau)\defeq  -\ii h(\lambda;\chi,\tau).
%\label{eq:h-tilde}
%\end{equation}
%%For the purposes of local analysis near the points $\lambda=a,b$ it is convenient to define $h(\lambda;\chi,\tau)=:\ii \tilde{h}(\lambda;\chi,\tau)$.
%We see from \eqref{eq:hprime-formula} and \eqref{eq:h-tilde} that the derivative of $\tilde{h}$ is given by
%\begin{equation}
%\tilde{h}'(\lambda;\chi,\tau)=\frac{\left(2 \tau \lambda^{2}+u(\chi, \tau) \lambda+v(\chi, \tau)\right) R(\lambda ; \chi, \tau) }{\lambda^{2}+1},
%\label{eq:h-tilde-prime-formula}
%\end{equation}
%%We start with a direct calculation from \eqref{eq:hprime-formula} and obtain \textcolor{red}{[Where will this go?]}
%%\begin{equation}
%%\begin{split}
%%{h}''(\lambda;\chi,\tau) =& \frac{1}{\lambda^2+1}\left[(4 \tau \lambda + u(\chi,\tau) )R(\lambda;\chi,\tau) +
%%\left(2 \tau \lambda^{2}+u(\chi, \tau) \lambda+v(\chi, \tau)\right) \left(\frac{\lambda- A(\chi,\tau)}{R(\lambda;\chi,\tau)}\right)
%% \right]\\
%% & - \frac{2 \lambda {h}'(\lambda;\chi,\tau)}{\lambda^2 + 1}.
%% \end{split}
%%\end{equation}
%%Noting that $u(\chi,\tau) = - 2 \tau (a(\chi,\tau)+b(\chi,\tau))$ and the fact that the quadratic factor in the numerator in \eqref{eq:hprime-formula} and hence ${h}'(\lambda;\chi,\tau)$ vanishes at $\lambda=a,b$, we see that
%%\begin{align}
%%\tilde{h}''(b(\chi,\tau);\chi,\tau) &= \frac{2\tau(b(\chi,\tau) - a(\chi,\tau))}{b(\chi,\tau)^2+1}R(b(\chi,\tau);\chi,\tau)>0 \label{eq:h-double-prime-b}, \\
%%\tilde{h}_{-}''(a(\chi,\tau);\chi,\tau) &= \frac{2\tau(a(\chi,\tau) - b(\chi,\tau))}{a(\chi,\tau)^2+1}R_{-}(a(\chi,\tau);\chi,\tau)<0 
%%\label{eq:h-double-prime-a},
%%\end{align}
%\begin{align}
%{h}''(b(\chi,\tau);\chi,\tau) &= \frac{2\tau(b(\chi,\tau) - a(\chi,\tau))}{b(\chi,\tau)^2+1}|b(\chi,\tau) - \lambda_0(\chi,\tau)|>0 \label{eq:h-double-prime-b}, \\
%{h}_{-}''(a(\chi,\tau);\chi,\tau) &= \frac{- 2\tau(b(\chi,\tau) - a(\chi,\tau))}{a(\chi,\tau)^2+1} | a(\chi,\tau) - \lambda_0(\chi,\tau)|<0 
%\label{eq:h-double-prime-a},
%\end{align}
%where the inequalities follow because $b(\chi,\tau)>a(\chi,\tau)$ and where we have used the identities \eqref{eq:R-a-b}.

To define an inner parametrix $\dot{\mathbf{W}}^b(\lambda)=\dot{\mathbf{W}}^b(\lambda;\chi,\tau,\mathbf{Q}^{-s},M)$ in $D_b(\delta)$, first note that the properties of $h(\lambda;\chi,\tau)$ summarized at the beginning of this section imply that
${h}(\lambda;\chi,\tau)-{h}(b(\chi,\tau);\chi,\tau)$ is an analytic function of $\lambda$ that vanishes precisely to second order as $\lambda \to b(\chi,\tau)$. 
%first note from the positivity of \eqref{eq:h-double-prime-b} that ${h}(\lambda;\chi,\tau)-{h}(b(\chi,\tau);\chi,\tau)$ vanishes precisely to second order as $\lambda \to b(\chi,\tau)$. 
We introduce an $M$-independent conformal coordinate $f_b$ by setting
\begin{equation}
f_b(\lambda;\chi,\tau)^2 = 2({h}(\lambda;\chi,\tau)-{h}_b(\chi,\tau)),\quad \lambda\in D_b(\delta),
\label{eq:fb-def}
\end{equation}
where ${h}_b(\chi,\tau)\defeq {h}(b(\chi,\tau);\chi,\tau)$,
and choose the solution with $f_b'(b(\chi,\tau);\chi,\tau)>0$. To see why this choice is possible, note that repeated differentiation in \eqref{eq:fb-def} results in the relation
\begin{equation}
f_b'(b(\chi,\tau);\chi,\tau)^2 = h''(b(\chi,\tau);\chi,\tau) > 0.
\label{eq:fb-prime-h-double-prime}
\end{equation}
With this choice
%, which is again possible by the positivity of \eqref{eq:h-double-prime-b}, so that 
the arc $I\cap D_b(\delta)$ is mapped by $\lambda \mapsto f_b(\lambda;\chi,\tau)$ locally to the negative real axis. Then, in the rescaled conformal coordinate $\zeta_b \defeq  M^\frac{1}{2} f_b$, the jump conditions satisfied by the matrix function
\begin{equation}
\mathbf{U}^b(\lambda) \defeq  \mathbf{W}(\lambda) \ii^{\frac{1}{2}(1-s)\sigma_3}
%s^{\frac{1}{2}\sigma_3} 
\ee^{-\ii M {h}_b(\chi,\tau)\sigma_3},\quad \lambda\in D_b(\delta)
\label{eq:W-transformation-b}
\end{equation}
coincide exactly with those of $\mathbf{U}(\zeta)$ described right before \eqref{eq:PCU-asymp} when expressed in terms of the variable $\zeta=\zeta_b$ and the jump contours are locally taken to coincide with the five rays $\arg(\zeta)=\pm \tfrac{1}{4}\pi$, $\arg(\zeta)=\pm \tfrac{3}{4}\pi$, and $\arg(-\zeta)=0$ as shown in \cite[Figure 9]{BilmanLM20}. Therefore, the construction of $\dot{\mathbf{W}}^b(\lambda)$ follows \emph{mutatis mutandis} that of the local parametrix near $b$ in Section~\ref{sec:channels-parametrix}. Indeed, replacing $\vartheta_b$ with $h_b$ and taking into account the fact that $\dot{\mathbf{W}}^\text{out}(\lambda)$ in this section differs from the outer parametrix \eqref{eq:Channels-Tout} in Section~\ref{sec:channels-parametrix} by multiplication on the left by $\mathbf{J}(\lambda)$, one obtains (compare with \eqref{eq:Channels-Tb-ALT})
\begin{equation}
\dot{\mathbf{W}}^b(\lambda)\defeq \mathbf{Y}^b(\lambda)\mathbf{U}(\zeta_b) \ii^{-\frac{1}{2}(1-s)\sigma_3} \ee^{\ii M{h}_b(\chi,\tau)\sigma_3},% s^{-\sigma_3/2},
\end{equation}
where $\mathbf{Y}^b(\lambda)$ is the prefactor that is holomorphic in the disk $D_b(\delta)$ and is given by
\begin{equation}
\mathbf{Y}^b(\lambda)\defeq 
\mathbf{J}(\lambda)
%s^{\sigma_3/2}
M^{\frac{1}{2} \ii p \sigma_3}
 \ee^{-\ii M {h}_b(\chi,\tau)\sigma_3} \ii^{\frac{1}{2}(1-s)\sigma_3}  \mathbf{H}^{b}(\lambda),
%\mathbf{Y}^b(z;M,w,t)\defeq \ee^{-\ii M \kappa(w)\sigma_3/2} \mathbf{X}^b(z;w,t)e^{\ii M \kappa(w)\sigma_3/2}\ee^{-\ii M h(b(w);w)\sigma_3}M^{\ii p \sigma_3/2}
\label{eq:A-b}
\end{equation}
in which $\mathbf{H}^b(\lambda)$ is given \emph{exactly} as in \eqref{eq:Channels-Hb} with the conformal map $f_b(\lambda;\chi,\tau)$ being based on $h$ as in \eqref{eq:fb-def} rather than on $\vartheta$ as in Section~\ref{sec:channels}. $\mathbf{H}^b(\lambda)$ is holomorphic in the disk $D_b(\delta)$.
It is now easy to verify (by drawing a comparison with the construction in Section~\ref{sec:channels-parametrix}) that $\dot{\mathbf{W}}^b(\lambda)$ exactly satisfies the jump conditions for $\mathbf{W}(\lambda)$ in $D_b(\delta)$.
%As $h_b(\chi,\tau)$, $p$, and $\kappa(\chi,\tau)$ are all real valued, $\dot{\mathbf{W}}^b(\lambda)$ remains bounded as $M\to +\infty$ in $D_b(\delta)$. 
Comparing the this parametrix with the outer parametrix (which is unimodular) on the boundary of $D_b(\delta)$, we see that
%with those given 
%in \cite[Riemann-Hilbert Problem 5]{BilmanLM20} when expressed in terms of the variable $\zeta=\zeta_b$ and the jump contours are locally taken to coincide with the five rays $\arg(\zeta)=\pm \pi/4$, $\arg(\zeta)=\pm 3\pi/4$, and $\arg(-\zeta)=0$ as shown in \cite[Figure 9]{BilmanLM20}, which are satisfied exactly by a special case $\mathbf{U}$
%of the parabolic cylinder parametrix. \textcolor{red}{[Link this to the parametrix in the Channels.]} In light of the transformation \eqref{eq:W-transformation-b}, for $\mathbf{W}(\lambda)\dot{\mathbf{W}}^{b}(\lambda)^{-1}$ to be analytic in $D_{b}(\delta)$ we take the inner parametrix $\dot{\mathbf{W}}^{b}(\lambda)=\dot{\mathbf{W}}^{b}(\lambda;\chi,\tau,\mathbf{Q}^{-s},M)$  to be of the form
%\begin{equation}
%\dot{\mathbf{W}}^{b}(\lambda) = \mathbf{A}^{b} (\lambda) \mathbf{U}(M^\frac{1}{2} f_{b}(\lambda;\chi,\tau))\ee^{\ii M \tilde{h}_b(\chi,\tau)\sigma_3} \ii^{-\frac{1}{2}(1-s)\sigma_3},
%\label{eq:Wb-ansatz}
%\end{equation}
%where $\mathbf{A}^{b} (\lambda)$ is a matrix function that is holomorphic in $D_b(\delta)$, to be determined to ensure that $\dot{\mathbf{W}}^{b}(\lambda) \dot{\mathbf{W}}^{\mathrm{out}}(\lambda)^{-1}= \mathbb{I}+o(1)$ for $\lambda\in\partial D_b(\delta)$ as $M\to+\infty$. For $\lambda$ near $b(\chi,\tau)$ the outer parametrix \eqref{eq:W-out-G} may be expressed as
%\begin{equation}
% \dot{\mathbf{W}}^{\mathrm{out}}(\lambda)
% %s^{\sigma_3/2} 
%\ii^{\frac{1}{2}(1-s)\sigma_3} \ee^{- \ii M \tilde{h}_b(\chi,\tau)\sigma_3}
% =\mathbf{G}(\lambda)
%% s^{\sigma_3/2}
% \ii^{\frac{1}{2}(1-s)\sigma_3} \ee^{-\ii M \tilde{h}_b(\chi,\tau)} M^{\frac{1}{2}\ii p \sigma_3} \mathbf{H}^{b}(\lambda) \zeta_b^{-\ii p \sigma_3},
%\end{equation}
%where
%\begin{equation}
%\mathbf{H}^b(\lambda) \defeq  (\lambda-a(\chi,\tau))^{\ii p \sigma_3} \left( \frac{f_b(\lambda;\chi,\tau)}{\lambda-b(\chi,\tau)} \right)^{\ii p \sigma_3},
%\end{equation}
%with both power functions taken to be the principal branch. Thus, the diagonal matrix function $\mathbf{H}^b(\lambda;\chi,\tau)$ is holomorphic in the disk $D_b(\delta)$. The inner parametrix $\dot{\mathbf{W}}^b$ is now defined as 
%%\begin{equation}
%%\dot{\mathbf{X}}^b(\lambda;\chi,\tau)\defeq \mathbf{A}^b(\lambda;\chi,\tau)\mathbf{U}(n^{\frac{1}{2}}f_b(\lambda;\chi,\tau))\ee^{\ii n \tilde{h}_b(\chi,\tau)\sigma_3}\ee^{S(\lambda;\chi,\tau)\sigma_3} \ii^{-\frac{1}{2}(1-s)\sigma_3} \omega(\lambda)^{-s\sigma_3},% s^{-\sigma_3/2},
%%\end{equation}
%%\textcolor{red}{or in the alternate approach:
%\begin{equation}
%\dot{\mathbf{W}}^b(\lambda)\defeq \mathbf{A}^b(\lambda)\mathbf{U}(M^{\frac{1}{2}}f_b(\lambda;\chi,\tau))\ee^{\ii M \tilde{h}_b(\chi,\tau)\sigma_3}  \ii^{-\frac{1}{2}(1-s)\sigma_3},% s^{-\sigma_3/2},
%\end{equation}
%%}
%where
%%\begin{equation}
%%\mathbf{A}^b(\lambda;\chi,\tau)\defeq 
%%\mathbf{G}(\lambda;\chi,\tau)
%%%s^{\sigma_3/2}
%%\omega(\lambda)^{s\sigma_3} \ii^{\frac{1}{2}(1-s)\sigma_3} \ee^{ - S(\lambda;\chi,\tau) \sigma_3}\ee^{-\ii n \tilde{h}_b(\chi,\tau)} n^{\frac{1}{2} \ii p \sigma_3} \mathbf{H}^{b}(\lambda;\chi,\tau) 
%%%\mathbf{Y}^b(z;M,w,t)\defeq \ee^{-\ii M \kappa(w)\sigma_3/2} \mathbf{X}^b(z;w,t)e^{\ii M \kappa(w)\sigma_3/2}\ee^{-\ii M h(b(w);w)\sigma_3}M^{\ii p \sigma_3/2}
%%\label{eq:A-b}
%%\end{equation}
%%\textcolor{red}{or in the alternate approach,
%\begin{equation}
%\mathbf{A}^b(\lambda)\defeq 
%\mathbf{G}(\lambda)
%%s^{\sigma_3/2}
%\ii^{\frac{1}{2}(1-s)\sigma_3} \ee^{-\ii M \tilde{h}_b(\chi,\tau)} M^{\frac{1}{2} \ii p \sigma_3} \mathbf{H}^{b}(\lambda) 
%%\mathbf{Y}^b(z;M,w,t)\defeq \ee^{-\ii M \kappa(w)\sigma_3/2} \mathbf{X}^b(z;w,t)e^{\ii M \kappa(w)\sigma_3/2}\ee^{-\ii M h(b(w);w)\sigma_3}M^{\ii p \sigma_3/2}
%\label{eq:A-b}
%\end{equation}
%%}
%is holomorphic in the disk $D_b(\delta)$. 
%Noting that $-\ii g(\lambda;\chi,\tau) = \tilde{h}(\lambda;\chi,\tau) - \vartheta(\lambda;\chi,\tau)$ and recalling the independence of \eqref{eq:g-integral} from the path of integration in the domain $\mathbb{C}\setminus \Sigma_g$, we may write
%\begin{equation}
%-\ii g_b(\chi,\tau)\defeq -\ii g(b(\chi,\tau);\chi,\tau) = \int_{+\infty}^{b(\chi,\tau)} \left( \tilde{h}'(\lambda;\chi,\tau) -\chi -2\tau\lambda + \frac{2}{\lambda^2+1} \right) \dd \lambda,
%\label{eq:g-at-b}
%\end{equation}
%and since the integrand is real valued on the path $[b(\chi,\tau),+\infty)$ we have $-\ii g_b(\chi,\tau) \in \mathbb{R}$. Then
%\begin{equation}
%\begin{split}
%\tilde{h}_b(\chi,\tau) &= - \ii g_b(\chi,\tau) + \vartheta_b(\chi,\tau)\\
%&= - \ii g_b(\chi,\tau) + \chi b(\chi,\tau) + \tau b(\chi,\tau)^2 + \ii \log\left(\frac{b(\chi,\tau) - \ii}{b(\chi,\tau)+\ii} \right),
%\end{split}
%\label{eq:h-tilde-at-b}
%\end{equation}
%in which the last term is real valued, and this implies that $\ee^{\pm \ii n \tilde{h}_b(\chi,\tau)}$ is bounded as $M\to +\infty$.
%Since $p,\kappa(\chi,\tau)\in\mathbb{R}$ as well, we conclude that the matrix function $\mathbf{A}^b(\lambda)$ remains bounded as $M\to +\infty$ in addition to being holomorphic in $D_b(\delta)$.
%It then follows that for $\lambda\in D_b(\delta)$
%\begin{equation}
%\begin{aligned}
%\dot{\mathbf{X}}^b(\lambda;\chi,\tau)\mathbf{X}^{\mathrm{out}}(\lambda;\chi,\tau)^{-1}&\defeq \mathbf{A}^b(\lambda;\chi,\tau)\mathbf{U}(n^{\frac{1}{2}}f_b(\lambda;\chi,\tau)) \zeta_b^{\ii p \sigma_3}\mathbf{A}^b(\lambda;\chi,\tau)^{-1}\\
%&=\mathbf{A}^b(\lambda;\chi,\tau)\mathbf{U}(\zeta_b) \zeta_b^{\ii p \sigma_3}\mathbf{A}^b(\lambda;\chi,\tau)^{-1},
%\end{aligned}
%\label{eq:error-PC-b}
%\end{equation}
%\textcolor{red}{or in the alternate approach:
%\begin{equation}
%\begin{aligned}
%\dot{\mathbf{W}}^b(\lambda)\mathbf{W}^{\mathrm{out}}(\lambda)^{-1}
%\defeq &\mathbf{A}^b(\lambda)\mathbf{U}(M^{\frac{1}{2}}f_b(\lambda)) \zeta_b^{\ii p \sigma_3}\mathbf{A}^b(\lambda)^{-1}\\
%=&\mathbf{A}^b(\lambda)\mathbf{U}(\zeta_b) \zeta_b^{\ii p \sigma_3}\mathbf{A}^b(\lambda)^{-1},
%\end{aligned}
%\label{eq:error-PC-b}
%\end{equation}
\begin{equation}
\dot{\mathbf{W}}^b(\lambda)\dot{\mathbf{W}}^{\mathrm{out}}(\lambda)^{-1}
=\mathbf{Y}^b(\lambda)\mathbf{U}(\zeta_b) \zeta_b^{\ii p \sigma_3}\mathbf{Y}^b(\lambda)^{-1},
\label{eq:error-PC-b}
\end{equation}
%}
and using the asymptotic expansion \eqref{eq:PCU-asymp} in \eqref{eq:error-PC-b} yields the estimate
\begin{equation}
\sup_{\lambda\in \partial D_b(\delta)}\| \dot{\mathbf{W}}^b(\lambda)\dot{\mathbf{W}}^{\mathrm{out}}(\lambda)^{-1} -\mathbb{I} \| = O(M^{-\frac{1}{2}}),\quad M\to+\infty,
\label{eq:error-PC-disk-b-large-M}
\end{equation}
%}
where $\| \cdot \|$ denotes the matrix norm induced from an arbitrary vector norm on $\mathbb{C}^2$.

%since the parabolic cylinder parametrix $\mathbf{U}$ has the asymptotic expansion (see \cite[Section 4.1.2, Riemann-Hilbert Problem 5 and Eqn.\@ 141]{BilmanLM20}) \textcolor{red}{[Again, modify this by referring to what's been done in the Channels.]}
%\begin{equation}
%\mathbf{U}(\zeta)\zeta^{\ii p \sigma_3} = \mathbb{I} + \frac{1}{2\ii \zeta}\begin{bmatrix} 0 & \alpha \\ -\beta & 0\end{bmatrix} + O(\zeta^{-2}),\quad \zeta\to\infty,
%\label{eq:PC-asymptotics}
%\end{equation}
%where $\alpha$ is defined as in \eqref{eq:alpha-beta-def} and $\beta = -\alpha^*$, we have 
%%\begin{equation}
%%\sup_{\lambda\in \partial D_b(\delta)}\| \dot{\mathbf{X}}^b(\lambda;\chi,\tau) \dot{\mathbf{X}}^{\mathrm{out}}(\lambda;\chi,\tau)^{-1} -\mathbb{I} \| = O(n^{-\frac{1}{2}}),\quad n\to+\infty,
%%\label{eq:error-PC-disk-b-large-n}
%%\end{equation}
%%\textcolor{red}{
%%or in the alternate approach
%\begin{equation}
%\sup_{\lambda\in \partial D_b(\delta)}\| \dot{\mathbf{W}}^b(\lambda)\dot{\mathbf{W}}^{\mathrm{out}}(\lambda)^{-1} -\mathbb{I} \| = O(M^{-\frac{1}{2}}),\quad M\to+\infty,
%\label{eq:error-PC-disk-b-large-M}
%\end{equation}
%%}
%where $\| \cdot \|$ denotes the matrix norm induced from an arbitrary vector norm on $\mathbb{C}^2$.

Constructing an inner parametrix $\dot{\mathbf{W}}^a(\lambda)=\dot{\mathbf{W}}^a(\lambda)(\lambda;\chi,\tau,\mathbf{Q}^{-s},M)$ in the disk $D_a(\delta)$ requires a bit more work due to the presence of the cut $\Sigma_g$ inside $D_a(\delta)$. Note that for $\lambda\in D_a$, ${h}(\lambda;\chi,\tau)$ comprises two different functions that are both analytic in the entire disk $D_a(\delta)$. We will use $\pm$ subscripts to denote these functions: ${h}_{-}(\lambda;\chi,\tau)$ (resp., ${h}_{+}(\lambda;\chi,\tau)$) coincides with ${h}(\lambda;\chi,\tau)$ for $\lambda$ to the right (resp., left) of $\Sigma_g$ with respect to (upward) orientation. These functions are of course related by ${h}_+(\lambda;\chi,\tau) + {h}_-(\lambda;\chi,\tau) = 2 \kappa(\chi,\tau)$ for $ \lambda\in D_{a}(\delta)$,
%${h}_{R}(\lambda;\chi,\tau)$ (respectively ${h}_{L}(\lambda;\chi,\tau)$), which coincides with ${h}(\lambda;\chi,\tau)$ for $\lambda$ to the right (respectively left) of $\Sigma_g$ with respect to the (upward) orientation. The values of these functions for $\lambda \in \Sigma_g$ agree with the boundary values of ${h}(\lambda;\chi,\tau)$ taken on $\Sigma_g$ in the following way: ${h}_{R}(\lambda;\chi,\tau) = 
%{h}_{-}(\lambda;\chi,\tau)$ and ${h}_{L}(\lambda;\chi,\tau) = {h}_{+}(\lambda;\chi,\tau)$ for $\lambda\in\Sigma_g$, and these two functions are of course related by
%\begin{equation}
%{h}_R(\lambda;\chi,\tau) + {h}_L(\lambda;\chi,\tau) = 2 \kappa(\chi,\tau),\quad \lambda\in D_{a}(\delta),
%\end{equation}
where the (real-valued) constant $\kappa(\chi,\tau)$ is given in \eqref{eq:kappa-formula}. By analogy, we denote by $D_{a,-}(\delta)$ (resp., $D_{a,+}(\delta)$) the part of $D_a(\delta)$ that lies to the right (resp., left) of $\Sigma_g$ with respect to orientation. We use the same notational convention for the boundaries of these half-disks: $\partial D_{a,\pm}(\delta)$ denotes the circular boundary of $D_{a,\pm}(\delta)$ (omitting $\Sigma_g$).

We base the definition of a conformal mapping on the analytic function $\lambda\mapsto {h}_-(\lambda;\chi,\tau)$. Again by the properties of $h$ summarized at the beginning of this section, ${h}_-(\lambda;\chi,\tau)-{h}_-(a(\chi,\tau);\chi,\tau)$ vanishes to second order as $\lambda\to a(\chi,\tau)$. We introduce an $M$-independent conformal coordinate $f_a$ by setting
\begin{equation}
%f_a(\lambda;\chi,\tau)^2 = 2(h_R(a(\chi,\tau);\chi,\tau) - h_R(\lambda;\chi,\tau)),\quad \lambda\in D_a,
f_a(\lambda;\chi,\tau)^2 = 2(h_a(\chi,\tau) - h_-(\lambda;\chi,\tau)),\quad \lambda\in D_a(\delta),
\label{eq:fa-def}
\end{equation}
where $h_a(\chi,\tau)\defeq h_{-}(a(\chi,\tau);\chi,\tau)$,
and choose the solution with $f'_a(a(\chi,\tau);\chi,\tau)<0$. This is again possible as one obtains by repeated differentiation in \eqref{eq:fa-def} the relation
\begin{equation}
f_a'(a(\chi,\tau);\chi,\tau)^2= - h_-''(a(\chi,\tau);\chi,\tau) >0,
\label{eq:fa-prime-h-double-prime}
\end{equation}
see \eqref{eq:h-double-prime-a-b-signs}.
With this choice,
%which is possible by the non-vanishing of \eqref{eq:h-double-prime-a}, so that 
the arc $I \cap D_a(\delta)$ is mapped by $\lambda \mapsto f_a(\lambda ;\chi,\tau)$ locally to the negative real axis. In the rescaled conformal coordinate $\zeta_a \defeq M^\frac{1}{2} f_a$, the jump conditions satisfied by the piecewise-defined matrix function
\begin{equation}
\mathbf{U}^a(\lambda) \defeq \begin{cases} 
\mathbf{W} (\lambda)
%s^{\sigma_3/2} 
\ii^{\frac{1}{2}(1-s)\sigma_3}  \ii^{\sigma_3} \ee^{-\ii M {h}_a(\chi,\tau)\sigma_3}(\ii \sigma_2),\quad& \lambda\in D_{a,-}(\delta),\\
\mathbf{W}(\lambda) \ii^{\frac{1}{2}(1-s)\sigma_3} \ee^{-\ii M \kappa(\chi,\tau)\sigma_3}(-\ii \sigma_2)  \ee^{\ii M \kappa(\chi,\tau)\sigma_3}  \ii^{\sigma_3} \ee^{-\ii M {h}_a(\chi,\tau)\sigma_3}(\ii \sigma_2), \quad & \lambda \in D_{a,+}(\delta)
\end{cases}
\label{eq:W-transformation-a}
\end{equation}
coincide exactly with those of $\mathbf{U}(\zeta)$ described right before \eqref{eq:PCU-asymp} again when expressed in terms of the variable $\zeta=\zeta_a$ and the jump contours are locally taken to coincide with the five rays $\arg(\zeta)=\pm \tfrac{1}{4}\pi$, $\arg(\zeta)=\pm \tfrac{3}{4}\pi$, and $\arg(-\zeta)=0$ as shown in \cite[Figure 9]{BilmanLM20}. 
%See Figure~\ref{fig:D-a} for the local jump contours for $\mathbf{W}(\lambda)$ near $\lambda=a(\chi,\tau)$ and the boundaries of the half-disks.
%In the rescaled conformal coordinate $\zeta_a \defeq M^\frac{1}{2} f_a$, the jump conditions satisfied by the piecewise-defined matrix function
%%\begin{equation}
%%\mathbf{U}^a \defeq \begin{cases} 
%%\mathbf{X}^{(k)} 
%%%s^{\sigma_3/2} 
%%\omega(\lambda)^{s\sigma_3} \ii^{\frac{1}{2}(1-s)\sigma_3} \ee^{ - S(\lambda;\chi,\tau) \sigma_3}  \ii^{\sigma_3} \ee^{-\ii n \tilde{h}_a(\chi,\tau)\sigma_3}(\ii \sigma_2),& \lambda\in\Omega_{\pm 2,\pm 3}\\
%%\mathbf{X}^{(k)}  \omega(\lambda)^{s\sigma_3} \ii^{\frac{1}{2}(1-s)\sigma_3} \ee^{ - S(\lambda;\chi,\tau) \sigma_3} \left[ \ee^{-\ii n \kappa(\chi,\tau)\sigma_3}(-\ii \sigma_2)  \ee^{\ii n \kappa(\chi,\tau)\sigma_3} \right] \ii^{\sigma_3} \ee^{-\ii n \tilde{h}_a(\chi,\tau)\sigma_3}(\ii \sigma_2),& \lambda \in\Omega_{0,\pm 1}
%%\end{cases}
%%\label{eq:X-transformation-a}
%%\end{equation}
%%\textcolor{red}{or in the alternate approach:
%\begin{equation}
%\mathbf{U}^a \defeq \begin{cases} 
%\mathbf{W} 
%%s^{\sigma_3/2} 
%\ii^{\frac{1}{2}(1-s)\sigma_3}  \ii^{\sigma_3} \ee^{-\ii M \tilde{h}_a(\chi,\tau)\sigma_3}(\ii \sigma_2),& \lambda\in\Omega_{\pm 2,\pm 3}\\
%\mathbf{W} \ii^{\frac{1}{2}(1-s)\sigma_3} \left[ \ee^{-\ii M \kappa(\chi,\tau)\sigma_3}(-\ii \sigma_2)  \ee^{\ii M \kappa(\chi,\tau)\sigma_3} \right] \ii^{\sigma_3} \ee^{-\ii M \tilde{h}_a(\chi,\tau)\sigma_3}(\ii \sigma_2),& \lambda \in\Omega_{0,\pm 1}
%\end{cases}
%\label{eq:W-transformation-a}
%\end{equation}
%%}
%in $D_a(\delta)$ coincide again with those given in \cite[Rieman-Hilbert Problem 5]{BilmanLM20} when expressed in terms of the variable $\zeta=\zeta_a$ and the jump contours are locally taken to coincide with the five rays $\arg(\zeta)=\pm \pi/4$, $\arg(\zeta)=\pm 3\pi/4$, and $\arg(-\zeta)=0$ (see \cite[Figure 9]{BilmanLM20}). These jump conditions are satisfied again by a special case of the parabolic cylinder parametrix, $\mathbf{U}(\zeta)$.  \textcolor{red}{[Again refer to Channels or whichever appearance of the PC parametrix Channels is currently citing.]}
%See Figure~\ref{fig:D-a} for the definitions of the regions $\Omega_j$, $j=0,\pm 1, \pm 2, \pm 3$, and the jump contours for $\mathbf{W}^{(k)}$.
%\begin{figure}
%\includegraphics{contours-near-a}
%\caption{Local jump contours for $\mathbf{W}(\lambda)$ inside the disk $D_a(\delta)$, the left half-disk boundary $\partial D_{a,L}(\delta)$ (purple) and the right half-disk boundary $\partial D_{a,L}(\delta)$ (green).}
%\label{fig:D-a}
%\end{figure}
In light of the transformation \eqref{eq:W-transformation-a}, for $\mathbf{W}(\lambda)\dot{\mathbf{W}}^{a}(\lambda)^{-1}$ to be analytic in $D_{a}(\delta)$ we take the parametrix $\dot{\mathbf{W}}^{a}(\lambda)$=$\dot{\mathbf{W}}^{a}(\lambda;\chi,\tau,\mathbf{Q}^{-s},M)$ to be of the form
%\begin{equation}
%\dot{\mathbf{X}}^{a}(\lambda)\defeq \begin{cases}
%\mathbf{A}^{a}(\lambda) \mathbf{U}(n^\frac{1}{2}f_a(\lambda;\chi,\tau))(-\ii \sigma_2) \ee^{\ii n \tilde{h}_a(\chi,\tau)\sigma_3} (\ii^{-\sigma_3}) \ee^{S(\lambda;\chi,\tau)\sigma_3} \ii^{-\frac{1}{2}(1-s)\sigma_3}  \omega(\lambda)^{-s\sigma_3} ,& \lambda\in \Omega_{\pm 2, \pm 3}\\
%\mathbf{A}^{a}(\lambda) \mathbf{U}(n^\frac{1}{2} f_a(\lambda;\chi,\tau))(-\ii \sigma_2) \ee^{\ii n \tilde{h}_a(\chi,\tau)\sigma_3}(\ii^{-\sigma_3})\mathbf{K}\ee^{S(\lambda;\chi,\tau)\sigma_3}  \ii^{-\frac{1}{2}(1-s)\sigma_3} \omega(\lambda)^{-s\sigma_3},& \lambda\in \Omega_{0, \pm 1}
%\end{cases}
%\label{eq:X-a}
%\end{equation}
%\textcolor{red}{or, in the alternate approach:
\begin{equation}
\dot{\mathbf{W}}^{a}(\lambda)\defeq \begin{cases}
\mathbf{Y}^{a}(\lambda) \mathbf{U}(\zeta_a)(-\ii \sigma_2) \ee^{\ii M {h}_a(\chi,\tau)\sigma_3} (\ii^{-\sigma_3})  \ii^{-\frac{1}{2}(1-s)\sigma_3}, \quad & \lambda\in D_{a,-}(\delta),\\
\mathbf{Y}^{a}(\lambda) \mathbf{U}(\zeta_a)(-\ii \sigma_2) \ee^{\ii M {h}_a(\chi,\tau)\sigma_3}(\ii^{-\sigma_3})\mathbf{K}(\chi,\tau) \ii^{-\frac{1}{2}(1-s)\sigma_3} , \quad & \lambda\in D_{a,+}(\delta),
\end{cases}
\label{eq:W-a}
\end{equation}
%\begin{equation}
%\dot{\mathbf{W}}^{a}(\lambda)\defeq \begin{cases}
%\mathbf{Y}^{a}(\lambda) \mathbf{U}(\zeta_a)(-\ii \sigma_2) \ee^{\ii M {h}_a(\chi,\tau)\sigma_3} (\ii^{-\sigma_3})  \ii^{-\frac{1}{2}(1-s)\sigma_3}, \quad & \lambda\in D_{a,L}(\delta)\\
%\mathbf{Y}^{a}(\lambda) \mathbf{U}(\zeta_a)(-\ii \sigma_2) \ee^{\ii M {h}_a(\chi,\tau)\sigma_3}(\ii^{-\sigma_3}) \ee^{-\ii M\kappa(\chi,\tau)\sigma_3}(\ii \sigma_2)  \ee^{\ii M \kappa(\chi,\tau)\sigma_3} \ii^{-\frac{1}{2}(1-s)\sigma_3} , \quad & \lambda\in D_{a,R}(\delta)
%\end{cases}
%\label{eq:W-a}
%\end{equation}
%}
where we have set
%\begin{equation}
%\mathbf{K}\defeq \mathbf{K}(n,\chi,\tau) = \ee^{-\ii n\kappa(\chi,\tau)\sigma_3}(\ii \sigma_2)  \ee^{\ii n \kappa(\chi,\tau)\sigma_3}
%\label{eq:K-mat}
%\end{equation}
%\textcolor{red}{or, in the alternate approach:
\begin{equation}
\mathbf{K}(\chi,\tau) \defeq   \ee^{-\ii M\kappa(\chi,\tau)\sigma_3}(\ii \sigma_2)  \ee^{\ii M \kappa(\chi,\tau)\sigma_3}
\label{eq:K-mat}
\end{equation}
%}
for brevity in the expressions, and where $\mathbf{Y}^a(\lambda)$ is a matrix function that is holomorphic in $D_a(\delta)$, to be determined by requiring $\dot{\mathbf{W}}^{a}(\lambda)\dot{\mathbf{W}}^{\mathrm{out}}(\lambda)^{-1}=\mathbb{I}+o(1)$ for $\lambda\in \partial D_a(\delta)$ as $M\to+\infty$.
%Before comparing $\dot{\mathbf{W}}^a(\lambda)$ and $\dot{\mathbf{W}}^{\mathrm{out}}(\lambda)$ on $\partial D_a(\delta)$, 
We note that $\mathbf{J}(\lambda)$ given in \eqref{eq:W-out-G} defines two functions that are analytic in the entire disk $D_a(\delta)$, and we again use $\pm$ subscripts consistent with their boundary values taken on $\Sigma_g$ to label them:
%and we again use the $\pm$ subscripts in labeling them:
%$S_{R}(\lambda;\chi,\tau)$ (respectively $S_L(\lambda;\chi,\tau)$), which coincides with $S(\lambda;\chi,\tau)$ for $\lambda$ to the right (respectively left) of $\Sigma_g$ with respect to the upward orientation; and 
%$\dot{\mathbf{W}}^{\mathrm{out}}_{L}(\lambda)$ (respectively $\dot{\mathbf{W}}^{\mathrm{out}}_{R}(\lambda)$) coincides with $\dot{\mathbf{W}}^{\mathrm{out}}(\lambda)$ for $\lambda$ to the left (respectively right) of $\Sigma_g$ with respect to the orientation. 
$\mathbf{J}_\pm(\lambda)$ coincides with $\mathbf{J}(\lambda)$ for $\lambda\in D_{a,\pm}(\delta)$.
%$\mathbf{J}_+(\lambda)$ (respectively $\mathbf{J}_-(\lambda)$) coincides with $\mathbf{J}(\lambda)$ for $\lambda$ to the left (respectively right) of $\Sigma_g$ in $D_a(\delta)$with respect to the orientation. 
%Before comparing $\dot{\mathbf{W}}^a$ and $\dot{\mathbf{W}}^{\mathrm{out}}$ on $\partial D_a(\delta)$, we note that $\dot{\mathbf{W}}^{\mathrm{out}}(\lambda)$ defines two functions that are analytic in the entire disk $D_a(\delta)$:
%%and we again use the $\pm$ subscripts in labeling them:
%%$S_{R}(\lambda;\chi,\tau)$ (respectively $S_L(\lambda;\chi,\tau)$), which coincides with $S(\lambda;\chi,\tau)$ for $\lambda$ to the right (respectively left) of $\Sigma_g$ with respect to the upward orientation; and 
%%$\dot{\mathbf{W}}^{\mathrm{out}}_{L}(\lambda)$ (respectively $\dot{\mathbf{W}}^{\mathrm{out}}_{R}(\lambda)$) coincides with $\dot{\mathbf{W}}^{\mathrm{out}}(\lambda)$ for $\lambda$ to the left (respectively right) of $\Sigma_g$ with respect to the orientation. 
%$\dot{\mathbf{W}}^{\mathrm{out}}_{L}(\lambda)$ (respectively $\dot{\mathbf{W}}^{\mathrm{out}}_{R}(\lambda)$) coincides with $\dot{\mathbf{W}}^{\mathrm{out}}(\lambda)$ for $\lambda$ to the left (respectively right) of $\Sigma_g$ with respect to the orientation. 
%The values of these functions coincide with the boundary values of $\dot{\mathbf{W}}^\mathrm{out}(\lambda)$ on $\Sigma_g \cap D_a(\delta)$ in the following way: 
%%$S_L(\lambda;\chi,\tau) = S_+(\lambda;\chi,\tau) $ and $S_R(\lambda;\chi,\tau) = S_-(\lambda;\chi,\tau)$ for $\lambda\in \Sigma_g\cap D_a(\delta)$, and 
%$\dot{\mathbf{W}}^{\mathrm{out}}_L(\lambda) = \dot{\mathbf{W}}^{\mathrm{out}}_+(\lambda)$ and $\dot{\mathbf{W}}^{\mathrm{out}}_R(\lambda) = \dot{\mathbf{W}}^{\mathrm{out}}_-(\lambda)$ for $\lambda\in \Sigma_g\cap D_a(\delta)$. 
%It follows from \eqref{eq:W-a} that for $\lambda\in \partial D_{a,-}(\delta)$ we have
It follows from \eqref{eq:W-out-G} and \eqref{eq:W-a} that for $\lambda\in \partial D_{a,-}(\delta)$ we have
%\begin{equation}
%\begin{aligned}
%\dot{\mathbf{X}}^a(\lambda)\dot{\mathbf{X}}^\mathrm{out}(\lambda)^{-1} &= \dot{\mathbf{X}}^a(\lambda)\dot{\mathbf{X}}^\mathrm{out}_R(\lambda)^{-1}\\
%&=\mathbf{A}^{a}(\lambda) \mathbf{U}^{\text{PC}}(\zeta_a)(-\ii\sigma_2) \ee^{\ii n \tilde{h}_a(\chi,\tau)\sigma_3}(\ii^{-\sigma_3})\ee^{S(\lambda;\chi,\tau)\sigma_3}\ii^{-\frac{1}{2}(1-s)\sigma_3}\omega(\lambda)^{-s\sigma_3} \dot{\mathbf{X}}^\mathrm{out}_R(\lambda)^{-1}\\
%&=\mathbf{A}^{a}(\lambda) \mathbf{U}^{\text{PC}}(\zeta_a)\hat{\mathbf{X}}^\mathrm{out}_R(\lambda)^{-1},
%\end{aligned}
%\label{eq:error-right-half-disk-Xhat}
%\end{equation}
%\textcolor{red}{or, in the alternate approach:
%\begin{equation}
%%\begin{aligned}
%\dot{\mathbf{W}}^a(\lambda)\dot{\mathbf{W}}^\mathrm{out}(\lambda)^{-1} 
%%&= \dot{\mathbf{W}}^a(\lambda)\dot{\mathbf{W}}^\mathrm{out}_R(\lambda)^{-1}\\
%=\mathbf{Y}^{a}(\lambda) \mathbf{U}(\zeta_a)(-\ii\sigma_2) \ee^{\ii M {h}_a(\chi,\tau)\sigma_3}(\ii^{-\sigma_3})\ii^{-\frac{1}{2}(1-s)\sigma_3} \dot{\mathbf{W}}^\mathrm{out}_R(\lambda)^{-1}.
%%&=\mathbf{Y}^{a}(\lambda) \mathbf{U}^{\text{PC}}(\zeta_a)\hat{\mathbf{W}}^\mathrm{out}_R(\lambda)^{-1},
%%\end{aligned}
%\label{eq:error-right-half-disk-What}
%\end{equation}
\begin{equation}
%\begin{aligned}
\dot{\mathbf{W}}^a(\lambda)\dot{\mathbf{W}}^\mathrm{out}(\lambda)^{-1} 
%&= \dot{\mathbf{W}}^a(\lambda)\dot{\mathbf{W}}^\mathrm{out}_R(\lambda)^{-1}\\
=\mathbf{Y}^{a}(\lambda) \mathbf{U}(\zeta_a)(-\ii\sigma_2) \ee^{\ii M {h}_a(\chi,\tau)\sigma_3}(\ii^{-\sigma_3})\ii^{-\frac{1}{2}(1-s)\sigma_3} \left(\frac{\lambda-a(\chi,\tau)}{\lambda-b(\chi,\tau)}\right)^{-\ii p\sigma_3} \mathbf{J}_-(\lambda)^{-1}.
%&=\mathbf{Y}^{a}(\lambda) \mathbf{U}^{\text{PC}}(\zeta_a)\hat{\mathbf{W}}^\mathrm{out}_R(\lambda)^{-1},
%\end{aligned}
\label{eq:error-right-half-disk-What}
\end{equation}
On the other hand, the definition \eqref{eq:W-out-G} also yields for $\lambda\in D_{a,-}(\delta)\setminus I$
%\begin{equation}
%\dot{\mathbf{W}}^\mathrm{out}_R(\lambda)  \ii^{\frac{1}{2}(1-s)\sigma_3}  \ii^{\sigma_3}  \ee^{- \ii M {h}_a(\chi,\tau)\sigma_3}(\ii\sigma_2) = 
%\mathbf{J}_R(\lambda)   \ii^{\frac{1}{2}(1-s)\sigma_3} \ii^{\sigma_3}  \ee^{- \ii M {h}_a(\chi,\tau)\sigma_3}
%M^{-\frac{1}{2}\ii p \sigma_3}\mathbf{H}^a(\lambda) \zeta_a^{-\ii p \sigma_3},
%\label{eq:W-hat-R-out}
%\end{equation}
\begin{equation}
\dot{\mathbf{W}}^\mathrm{out}(\lambda)  \ii^{\frac{1}{2}(1-s)\sigma_3}  \ii^{\sigma_3}  \ee^{- \ii M {h}_a(\chi,\tau)\sigma_3}(\ii\sigma_2) = 
\mathbf{J}_{-}(\lambda)   \ii^{\frac{1}{2}(1-s)\sigma_3} \ii^{\sigma_3}  \ee^{- \ii M {h}_a(\chi,\tau)\sigma_3}
M^{-\frac{1}{2}\ii p \sigma_3}\mathbf{H}^a(\lambda) \zeta_a^{-\ii p \sigma_3},
\label{eq:W-hat-R-out}
\end{equation}
where $\mathbf{H}^{a}(\lambda)$ is given \emph{exactly} by the formula \eqref{eq:Channels-Ha} except with the different conformal map $f_a(\lambda;\chi,\tau)$ whose construction \eqref{eq:fa-def} is based on $h_{-}(\lambda;\chi,\tau)$ rather than $\vartheta(\lambda;\chi,\tau)$ as in Section~\ref{sec:channels}. 
%Here $\mathbf{J}_R(\lambda)$ is the matrix function which is holomorphic in $D_a(\delta)$ and which coincides with $\mathbf{J}(\lambda)$ in $D_{a,R}(\delta)$. 
$\mathbf{H}^{a}(\lambda)$ is holomorphic in the entire disk $D_{a}(\delta)$.
%(compare with \eqref{eq:Channels-Ha} noting the different conformal map based on $h$ instead of $\vartheta$)
%\begin{equation}
%\mathbf{H}^{a}(\lambda)\defeq (b(\chi,\tau)-\lambda)^{-\mathrm{i} p \sigma_{3}}\left(\frac{a(\chi,\tau)-\lambda}{f_{a}(\lambda ; \chi,\tau)}\right)^{\mathrm{i} p \sigma_{3}}\left(\mathrm{i} \sigma_{2}\right)
%\end{equation}
%with the power functions taken to be the principle branch, making it holomorphic in the entire disk $D_{a}(\delta)$. 
Using \eqref{eq:W-hat-R-out} in \eqref{eq:error-right-half-disk-What} guides us to choose the prefactor $\mathbf{Y}^a(\lambda)$ to be
\begin{equation}
\mathbf{Y}^a(\lambda)\defeq \mathbf{J}_{-}(\lambda)  M^{-\frac{1}{2}\ii p \sigma_3} \ee^{- \ii M {h}_a(\chi,\tau)\sigma_3} \ii^{\frac{1}{2}(1-s)\sigma_3} \ii^{\sigma_3}  
\mathbf{H}^a(\lambda),
\label{eq:A-a}
\end{equation}
which is holomorphic in the entire disk $D_a(\delta)$ and unimodular.
Since $\mathbf{Y}^a(\lambda)$ is now determined, $\dot{\mathbf{W}}^a(\lambda)$ is determined according to \eqref{eq:W-a} and it follows that the mismatch \eqref{eq:error-right-half-disk-What} 
between the inner parametrix and the outer parametrix along $\partial D_{a,-}(\delta)$ reads:
\begin{equation}
\dot{\mathbf{W}}^{a}(\lambda) \dot{\mathbf{W}}^{\text {out }}(\lambda)^{-1}=\mathbf{Y}^{a}(\lambda)\mathbf{U}\left(\zeta_{a}\right) \zeta_{a}^{\mathrm{i} p \sigma_{3}} \mathbf{Y}^{a}(\lambda )^{-1},\quad \lambda \in \partial D_{a,-}(\delta).
\label{eq:mismatch-right-half-disk}
\end{equation}
However, we have used only the information in the right half-disk $D_{a,R}(\delta)$ to construct $\dot{\mathbf{W}}^a(\lambda)$, and one needs to check whether \eqref{eq:mismatch-right-half-disk} actually holds on the entire disk boundary $\partial D_{a}(\delta)$. This can be verified by a direct calculation using the relation \eqref{eq:W-jump-Sigma-g},
%between $\dot{\mathbf{W}}^{\mathrm{out}}_L(\lambda)$ and $\dot{\mathbf{W}}^{\mathrm{out}}_R(\lambda)$,
%\begin{equation}
%\dot{\mathbf{W}}^{\mathrm{out}}_L(\lambda)  =\dot{\mathbf{W}}^{\mathrm{out}}_R(\lambda) 
%%\ee^{-\ii M\kappa(\chi,\tau)\sigma_3} \ii^{\frac{1}{2}(1-s)\sigma_3} (\ii\sigma_2) \ii^{-\frac{1}{2}(1-s)\sigma_3} \ee^{\ii M\kappa(\chi,\tau)\sigma_3}.
%\ii^{\frac{1}{2}(1-s)\sigma_3} \mathbf{K}(\chi,\tau) \ii^{-\frac{1}{2}(1-s)\sigma_3}, \quad \lambda \in D_a(\delta),
%\label{eq:f-W-out-L-to-R}
%\end{equation}
and we indeed have
 \begin{equation}
\dot{\mathbf{W}}^a(\lambda)\dot{\mathbf{W}}^\mathrm{out}(\lambda)^{-1} = \mathbf{Y}^a(\lambda)\mathbf{U}(\zeta_a) \zeta_a^{\ii p \sigma_3} \mathbf{Y}^a(\lambda)^{-1},\quad \lambda\in\partial D_a(\delta).
\label{eq:error-PC-a}
\end{equation}
By construction, $\dot{\mathbf{W}}^a(\lambda)$ exactly satisfies the jump conditions for $\mathbf{W}^a(\lambda)$ in $D_a(\delta)$.
%Just as in the construction of $\dot{\mathbf{W}}^b(\lambda)$, because $a(\chi,\tau)\in \mathbb{R}$ implies that $h_a(\chi,\tau)$ is real valued, it follows that $\dot{\mathbf{W}}^a(\lambda)$ remains bounded as $M \to +\infty$. 
Then
using the asymptotic expansion \eqref{eq:PCU-asymp} in \eqref{eq:error-PC-a}, we obtain the estimate
\begin{equation}
\sup_{\lambda \in \partial D_a(\delta)}\| \dot{\mathbf{W}}^a(\lambda) \dot{\mathbf{W}}^{\mathrm{out}}(\lambda)^{-1} -\mathbb{I} \| = O(M^{-\frac{1}{2}}),\quad M\to+\infty.
\label{eq:error-PC-disk-a-large-M}
\end{equation}

%Using the asymptotic expansion \eqref{eq:PC-asymptotics} for the (parabolic cylinder) matrix function $\mathbf{U}^\mathrm{PC}$ we obtain as in \eqref{eq:error-PC-disk-b-large-M} the asymptotic estimate \textcolor{red}{[Again refer to Channels.]}
%\begin{equation}
%\sup_{\lambda \in \partial D_a(\delta)}\| \dot{\mathbf{W}}^a(\lambda) \dot{\mathbf{W}}^{\mathrm{out}}(\lambda)^{-1} -\mathbb{I} \| = O(M^{-\frac{1}{2}}),\quad M\to+\infty.
%\label{eq:error-PC-disk-a-large-M}
%\end{equation}






%Thus,
%%Thus, we have the following relation which hold for $\lambda \in D_a(\delta)$
%%\begin{equation}
%%\begin{aligned}
%%S_L(\lambda;\chi,\tau)  + S_R(\lambda;\chi,\tau)   &= 2 \ii s \gamma(\chi,\tau) + 2s \log(\omega(\lambda)),\\
%%\dot{\mathbf{X}}^{\mathrm{out}}_L(\lambda; n,\chi,\tau)  &=\dot{\mathbf{X}}^{\mathrm{out}}_R(\lambda;\chi,\tau)  \ee^{-\ii(n\kappa(\chi,\tau)+s \gamma(\chi,\tau))\sigma_3} \ii^{\frac{1}{2}(1-s)\sigma_3} (\ii\sigma_2) \ii^{-\frac{1}{2}(1-s)\sigma_3} \ee^{\ii(n\kappa(\chi,\tau)+s \gamma(\chi,\tau))\sigma_3}.
%%\end{aligned}
%%\label{eq:f-X-out-L-to-R}
%%\end{equation}
%%\textcolor{red}{or, in the alternate approach we only have:
%\begin{equation}
%\dot{\mathbf{W}}^{\mathrm{out}}_L(\lambda)  =\dot{\mathbf{W}}^{\mathrm{out}}_R(\lambda) 
%%\ee^{-\ii M\kappa(\chi,\tau)\sigma_3} \ii^{\frac{1}{2}(1-s)\sigma_3} (\ii\sigma_2) \ii^{-\frac{1}{2}(1-s)\sigma_3} \ee^{\ii M\kappa(\chi,\tau)\sigma_3}.
%\ii^{\frac{1}{2}(1-s)\sigma_3} \mathbf{K}(\chi,\tau) \ii^{-\frac{1}{2}(1-s)\sigma_3}, \quad \lambda \in D_a(\delta).
%\label{eq:f-W-out-L-to-R}
%\end{equation}
%%}
%
%%We denote by $\partial D_{a,L}(\delta)$ (respectively $\partial D_{a,R}(\delta)$) the part of $\partial D_a(\delta)$ that lies to the left (respectively right) of $\Sigma_g$ with respect to orientation.
%For $\lambda\in \partial D_{a,R}(\delta)$ we have
%%\begin{equation}
%%\begin{aligned}
%%\dot{\mathbf{X}}^a(\lambda)\dot{\mathbf{X}}^\mathrm{out}(\lambda)^{-1} &= \dot{\mathbf{X}}^a(\lambda)\dot{\mathbf{X}}^\mathrm{out}_R(\lambda)^{-1}\\
%%&=\mathbf{A}^{a}(\lambda) \mathbf{U}^{\text{PC}}(\zeta_a)(-\ii\sigma_2) \ee^{\ii n \tilde{h}_a(\chi,\tau)\sigma_3}(\ii^{-\sigma_3})\ee^{S(\lambda;\chi,\tau)\sigma_3}\ii^{-\frac{1}{2}(1-s)\sigma_3}\omega(\lambda)^{-s\sigma_3} \dot{\mathbf{X}}^\mathrm{out}_R(\lambda)^{-1}\\
%%&=\mathbf{A}^{a}(\lambda) \mathbf{U}^{\text{PC}}(\zeta_a)\hat{\mathbf{X}}^\mathrm{out}_R(\lambda)^{-1},
%%\end{aligned}
%%\label{eq:error-right-half-disk-Xhat}
%%\end{equation}
%%\textcolor{red}{or, in the alternate approach:
%\begin{equation}
%\begin{aligned}
%\dot{\mathbf{W}}^a(\lambda)\dot{\mathbf{W}}^\mathrm{out}(\lambda)^{-1} &= \dot{\mathbf{W}}^a(\lambda)\dot{\mathbf{W}}^\mathrm{out}_R(\lambda)^{-1}\\
%&=\mathbf{A}^{a}(\lambda) \mathbf{U}^{\text{PC}}(\zeta_a)(-\ii\sigma_2) \ee^{\ii M \tilde{h}_a(\chi,\tau)\sigma_3}(\ii^{-\sigma_3})\ii^{-\frac{1}{2}(1-s)\sigma_3} \dot{\mathbf{W}}^\mathrm{out}_R(\lambda)^{-1}\\
%&=\mathbf{A}^{a}(\lambda) \mathbf{U}^{\text{PC}}(\zeta_a)\hat{\mathbf{W}}^\mathrm{out}_R(\lambda)^{-1},
%\end{aligned}
%\label{eq:error-right-half-disk-What}
%\end{equation}
%%}
%where we have set
%%\begin{equation}
%%\hat{\mathbf{X}}^\mathrm{out}_R(\lambda)=\hat{\mathbf{X}}^\mathrm{out}_R(\lambda;\chi,\tau)\defeq \dot{\mathbf{X}}^\mathrm{out}_R(\lambda;\chi,\tau)  \omega(\lambda)^{s\sigma_3} \ii^{\frac{1}{2}(1-s)\sigma_3} \ee^{-S_R(\lambda;\chi,\tau))\sigma_3}\ii^{\sigma_3}  \ee^{- \ii n \tilde{h}_a(\chi,\tau)\sigma_3}(\ii\sigma_2).
%%\end{equation}
%%\textcolor{red}{or, in the alternate approach:
%\begin{equation}
%\hat{\mathbf{W}}^\mathrm{out}_R(\lambda)\defeq \dot{\mathbf{W}}^\mathrm{out}_R(\lambda)  \ii^{\frac{1}{2}(1-s)\sigma_3}  \ii^{\sigma_3}  \ee^{- \ii M \tilde{h}_a(\chi,\tau)\sigma_3}(\ii\sigma_2).
%\label{eq:W-hat-R-out}
%\end{equation}
%%}
%for brevity. Using the definition \eqref{eq:W-out-G} of the outer parametrix enables us to express this as
%%\begin{equation}
%%\begin{split}
%%\hat{\mathbf{X}}^\mathrm{out}_R(\lambda) &= \mathbf{G}_R(\lambda) \omega(\lambda)^{s\sigma_3}  \ii^{\frac{1}{2}(1-s)\sigma_3} \ee^{-S_R(\lambda;\chi,\tau))\sigma_3}\ii^{\sigma_3}  \ee^{- \ii n \tilde{h}_a(\chi,\tau)\sigma_3}
%%\left(\frac{\lambda-a(\chi,\tau)}{\lambda-b(\chi,\tau)}\right)^{\mathrm{i} p \sigma_{3}}\left(\mathrm{i} \sigma_{2}\right)\\
%%&=\mathbf{G}_R(\lambda) \omega(\lambda)^{s\sigma_3}  \ii^{\frac{1}{2}(1-s)\sigma_3} \ee^{-S_R(\lambda;\chi,\tau))\sigma_3}\ii^{\sigma_3}  \ee^{- \ii n \tilde{h}_a(\chi,\tau)\sigma_3}
%%n^{-\frac{1}{2}\ii p \sigma_3}\mathbf{H}^a(\lambda) \zeta_a^{-\ii p \sigma_3},
%%\end{split}
%%\end{equation}
%%\textcolor{red}{or in the alternate approach:
%\begin{equation}
%\begin{split}
%\hat{\mathbf{W}}^\mathrm{out}_R(\lambda) &= \mathbf{G}_R(\lambda) \ii^{\frac{1}{2}(1-s)\sigma_3} \ii^{\sigma_3}  \ee^{- \ii M \tilde{h}_a(\chi,\tau)\sigma_3}
%\left(\frac{\lambda-a(\chi,\tau)}{\lambda-b(\chi,\tau)}\right)^{\mathrm{i} p \sigma_{3}}\left(\mathrm{i} \sigma_{2}\right)\\
%&=\mathbf{G}_R(\lambda)   \ii^{\frac{1}{2}(1-s)\sigma_3} \ii^{\sigma_3}  \ee^{- \ii M \tilde{h}_a(\chi,\tau)\sigma_3}
%M^{-\frac{1}{2}\ii p \sigma_3}\mathbf{H}^a(\lambda) \zeta_a^{-\ii p \sigma_3},
%\end{split}
%\end{equation}
%%}
%where $\mathbf{G}_R(\lambda)$ is the matrix function that is analytic in $D_a(\delta)$ satisfying $\mathbf{G}_{R}(\lambda)=\mathbf{G}(\lambda)$ for $z \in D_{a}(\delta)$ to the right of $\Sigma_g$ with respect to the (upward) orientation, and $\mathbf{H}^{a}(\lambda ) $ is the matrix function
%\begin{equation}
%\mathbf{H}^{a}(\lambda)\defeq (b(\chi,\tau)-\lambda)^{-\mathrm{i} p \sigma_{3}}\left(\frac{a(\chi,\tau)-\lambda}{f_{a}(\lambda ; \chi,\tau)}\right)^{\mathrm{i} p \sigma_{3}}\left(\mathrm{i} \sigma_{2}\right)
%\end{equation}
%with the power functions taken to be the principle branch, making it holomorphic in the disk $D_{a}(\delta)$. 
%%It is easy to see that $\mathbf{H}^{a}(\lambda;\chi,\tau)$ is holomorphic in the disk $D_a(\delta)$, and 
%We now determine the holomorphic prefactor matrix $\mathbf{A}^a(\lambda)$ in \eqref{eq:W-a} to be
%%\begin{equation}
%%\mathbf{A}^a(\lambda;\chi,\tau)\defeq \mathbf{G}_R(\lambda;\chi,\tau) \omega(\lambda)^{s\sigma_3} \ii^{\frac{1}{2}(1-s)\sigma_3} \ee^{-S_R(\lambda;\chi,\tau))\sigma_3}\ii^{\sigma_3}  \ee^{- \ii n \tilde{h}_a(\chi,\tau)\sigma_3}
%%n^{-\frac{1}{2}\ii p \sigma_3}\mathbf{H}^a(\lambda;\chi,\tau)
%%\label{eq:A-a}
%%\end{equation}
%%\textcolor{red}{or in the alternate approach:
%\begin{equation}
%\mathbf{A}^a(\lambda)\defeq \mathbf{G}_R(\lambda)  \ii^{\frac{1}{2}(1-s)\sigma_3} \ii^{\sigma_3}  \ee^{- \ii M \tilde{h}_a(\chi,\tau)\sigma_3}
%M^{-\frac{1}{2}\ii p \sigma_3}\mathbf{H}^a(\lambda)
%\label{eq:A-a}
%\end{equation}
%%}
%A calculation analogous to one given in \eqref{eq:g-at-b}--\eqref{eq:h-tilde-at-b} shows that 
%%$\ee^{\pm \ii n \tilde{h}_a(\chi,\tau)}$ \textcolor{red}{(alternately, 
%$\ee^{\pm \ii M \tilde{h}_a(\chi,\tau)}$ has modulus $1$, hence remains bounded as $M\to+\infty$. Since $p,\kappa(\chi,\tau)\in\mathbb{R}$ as well, it follows that the matrix function $\mathbf{A}^a(\lambda)$ is bounded as $M\to+\infty$. The mismatch between the inner parametrix and the outer parametrix on the right component $\partial D_{a,R}(\delta)$ of the boundary of the disk $D_a(\delta)$ is expressed as:
%%\begin{equation}
%%\mathbf{X}^{a}(\lambda ; \chi, \tau) \dot{\mathbf{X}}^{\text {out }}(\lambda ; \chi, \tau)^{-1}=\mathbf{A}^{a}(\lambda ; \chi, \tau)\mathbf{U}\left(\zeta_{a}\right) \zeta_{a}^{\mathrm{i} p \sigma_{3}} \mathbf{A}^{a}(\lambda ; \chi, \tau)^{-1},\quad \lambda \in \partial D_{a,R}.
%%\label{eq:mismatch-right-half-disk}
%%\end{equation}
%%\textcolor{red}{in the alternate approach:
%\begin{equation}
%\mathbf{W}^{a}(\lambda) \dot{\mathbf{W}}^{\text {out }}(\lambda)^{-1}=\mathbf{A}^{a}(\lambda)\mathbf{U}\left(\zeta_{a}\right) \zeta_{a}^{\mathrm{i} p \sigma_{3}} \mathbf{A}^{a}(\lambda )^{-1},\quad \lambda \in \partial D_{a,R}.
%\label{eq:mismatch-right-half-disk}
%\end{equation}
%%}
%We will now show that this formula holds on the entire boundary $\partial D_{a}(\delta)$. For $\lambda\in \partial D_{a,L}(\delta)$ we have
%%\textcolor{red}{Go on consistently.}
%%\begin{equation}
%%\begin{split}
%%\dot{\mathbf{X}}^{a}(\lambda) \dot{\mathbf{X}}^{\text {out }}(\lambda)^{-1} &= \dot{\mathbf{X}}^{a}(\lambda) \dot{\mathbf{X}}_{L}^{\text {out }}(\lambda)^{-1}\\
%%&=\mathbf{A}^{a}(\lambda) \mathbf{U}^{\text{PC}}(\zeta_a)(-\ii\sigma_2) \ee^{\ii n \tilde{h}_a(\chi,\tau)\sigma_3}(\ii^{-\sigma_3})\mathbf{K}\ee^{S_L(\lambda;\chi,\tau)\sigma_3} \ii^{-\frac{1}{2}(1-s)\sigma_3} \omega(\lambda)^{-s \sigma_3}  \dot{\mathbf{X}}^\mathrm{out}_L(\lambda)^{-1}\\
%%&=\mathbf{A}^{a}(\lambda) \mathbf{U}\left(\zeta_{a}\right) \hat{\mathbf{X}}_{L}^{\text {out }}(\lambda)^{-1}
%%\end{split}
%%\label{eq:error-left-half-disk-Xhat}
%%\end{equation}
%%\textcolor{red}{or in the alternate approach:
%\begin{equation}
%\begin{split}
%\dot{\mathbf{W}}^{a}(\lambda;\chi,\tau) \dot{\mathbf{W}}^{\text {out }}(\lambda;\chi,\tau)^{-1} &= \dot{\mathbf{W}}^{a}(\lambda;\chi,\tau)\dot{\mathbf{W}}_{L}^{\text {out }}(\lambda;\chi,\tau)^{-1}\\
%&=\mathbf{A}^{a}(\lambda;\chi,\tau) \mathbf{U}^{\text{PC}}(\zeta_a)(-\ii\sigma_2) \ee^{\ii M \tilde{h}_a(\chi,\tau)\sigma_3}(\ii^{-\sigma_3})\mathbf{K}(\chi,\tau)\ii^{-\frac{1}{2}(1-s)\sigma_3}  \dot{\mathbf{W}}^\mathrm{out}_L(\lambda;\chi,\tau)^{-1}\\
%&=\mathbf{A}^{a}(\lambda;\chi,\tau) \mathbf{U}\left(\zeta_{a}\right) \hat{\mathbf{W}}_{L}^{\text {out }}(\lambda;\chi,\tau)^{-1},
%\end{split}
%\label{eq:error-left-half-disk-What}
%\end{equation}
%%}
%where in analogy with \eqref{eq:error-right-half-disk-What} we have set
%%\begin{equation}
%%\hat{\mathbf{W}}^\mathrm{out}_L(\lambda;\chi,\tau)=\hat{\mathbf{X}}^\mathrm{out}_L(\lambda;\chi,\tau)\defeq \dot{\mathbf{X}}^\mathrm{out}_L(\lambda;\chi,\tau) 
%% \omega(\lambda;\chi,\tau)^{s\sigma_3} \ii^{\frac{1}{2}(1-s)\sigma_3} \ee^{-S_L(\lambda;\chi,\tau)\sigma_3} \mathbf{K}^{-1} \ii^{\sigma_3}  \ee^{- \ii n \tilde{h}_a(\chi,\tau)\sigma_3}(\ii\sigma_2).
%%\end{equation}
%%\textcolor{red}{or in the alternate approach:
%\begin{equation}
%\hat{\mathbf{W}}^\mathrm{out}_L(\lambda)\defeq \dot{\mathbf{W}}^\mathrm{out}_L(\lambda) 
% \ii^{\frac{1}{2}(1-s)\sigma_3} \mathbf{K}(\chi,\tau)^{-1} \ii^{\sigma_3}  \ee^{- \ii M \tilde{h}_a(\chi,\tau)\sigma_3}(\ii\sigma_2).
%\end{equation}
%%}
%Recalling the definitions \eqref{eq:W-a} and \eqref{eq:K-mat}, the relations \eqref{eq:f-W-out-L-to-R}, we see that
%%\begin{equation}
%%\begin{split}
%%\hat{\mathbf{W}}^{\mathrm{out}}_L(\lambda;\chi,\tau) 
%%&= \dot{\mathbf{W}}^{\mathrm{out}}_L(\lambda;\chi,\tau)  \ii^{\frac{1}{2}(1-s)\sigma_3}
%%\mathbf{K}(\chi,\tau)^{-1} \ii^{\sigma_3} \ee^{-\ii M \tilde{h}_a\sigma_3}(\ii \sigma_2)\\
%%&= \dot{\mathbf{W}}^{\mathrm{out}}_R(\lambda;\chi,\tau) \left[ \ii^{\frac{1}{2}(1-s)\sigma_3} \mathbf{K}(\chi,\tau) \ii^{-\frac{1}{2}(1-s)\sigma_3} \right]
%%\ii^{\frac{1}{2}(1-s)\sigma_3} \mathbf{K}(\chi,\tau)^{-1} \ii^{\sigma_3} \ee^{-\ii M \tilde{h}_a\sigma_3}(\ii \sigma_2)\\
%%&= \dot{\mathbf{W}}^{\mathrm{out}}_R(\lambda;\chi,\tau) \ii^{\frac{1}{2}(1-s)\sigma_3} \mathbf{K} \ee^{2 \ii s \gamma \sigma_3} \omega(\lambda;\chi,\tau)^{2s\sigma_3} \ee^{- S_L(\lambda;\chi,\tau)\sigma_3} \omega(\lambda;\chi,\tau)^{-s\sigma_3}\mathbf{K}^{-1} \ii^{\sigma_3} \ee^{-\ii n \tilde{h}_a\sigma_3}(\ii \sigma_2)\\
%%&=\dot{\mathbf{X}}^{\mathrm{out}}_R(\lambda;\chi,\tau) \ii^{\frac{1}{2}(1-s)\sigma_3} \mathbf{K}\ee^{S_R(\lambda;\chi,\tau)\sigma_3} \omega(\lambda;\chi,\tau)^{-s\sigma_3}\mathbf{K}^{-1} \ii^{\sigma_3} \ee^{-\ii n \tilde{h}_a\sigma_3}(\ii \sigma_2)\\
%%&=\dot{\mathbf{X}}^{\mathrm{out}}_R(\lambda;\chi,\tau) \omega(\lambda;\chi,\tau)^{s\sigma_3}  \ii^{\frac{1}{2}(1-s)\sigma_3}  \ee^{-S_R(\lambda;\chi,\tau)\sigma_3} \ii^{\sigma_3} \ee^{-\ii n \tilde{h}_a\sigma_3}(\ii \sigma_2)\\
%%&=\hat{\mathbf{X}}^{\mathrm{out}}_R(\lambda;\chi,\tau).
%%\end{split}
%%\end{equation}
%\begin{equation}
%\begin{split}
%\hat{\mathbf{W}}^{\mathrm{out}}_L(\lambda) 
%&= \dot{\mathbf{W}}^{\mathrm{out}}_L(\lambda)  \ii^{\frac{1}{2}(1-s)\sigma_3}
%\mathbf{K}(\chi,\tau)^{-1} \ii^{\sigma_3} \ee^{-\ii M \tilde{h}_a(\chi,\tau)\sigma_3}(\ii \sigma_2)\\
%&= \dot{\mathbf{W}}^{\mathrm{out}}_R(\lambda) \left[ \ii^{\frac{1}{2}(1-s)\sigma_3} \mathbf{K}(\chi,\tau) \ii^{-\frac{1}{2}(1-s)\sigma_3} \right]
%\ii^{\frac{1}{2}(1-s)\sigma_3} \mathbf{K}(\chi,\tau)^{-1} \ii^{\sigma_3} \ee^{-\ii M \tilde{h}_a(\chi,\tau)\sigma_3}(\ii \sigma_2)\\
%&= \dot{\mathbf{W}}^{\mathrm{out}}_R(\lambda) \ii^{\frac{1}{2}(1-s)\sigma_3} \ii^{\sigma_3} \ee^{-\ii M \tilde{h}_a(\chi,\tau)\sigma_3}(\ii \sigma_2)\\
%%&=\dot{\mathbf{X}}^{\mathrm{out}}_R(\lambda;\chi,\tau) \ii^{\frac{1}{2}(1-s)\sigma_3} \mathbf{K}\ee^{S_R(\lambda;\chi,\tau)\sigma_3} \omega(\lambda;\chi,\tau)^{-s\sigma_3}\mathbf{K}^{-1} \ii^{\sigma_3} \ee^{-\ii n \tilde{h}_a\sigma_3}(\ii \sigma_2)\\
%%&=\dot{\mathbf{X}}^{\mathrm{out}}_R(\lambda;\chi,\tau) \omega(\lambda;\chi,\tau)^{s\sigma_3}  \ii^{\frac{1}{2}(1-s)\sigma_3}  \ee^{-S_R(\lambda;\chi,\tau)\sigma_3} \ii^{\sigma_3} \ee^{-\ii n \tilde{h}_a\sigma_3}(\ii \sigma_2)\\
%&=\hat{\mathbf{W}}^{\mathrm{out}}_R(\lambda),
%\end{split}
%\end{equation}
%where we recalled the definition \eqref{eq:W-hat-R-out}.
%It now follows from the formul\ae{} \eqref{eq:error-right-half-disk-What} and \eqref{eq:error-left-half-disk-What} that the formula \eqref{eq:mismatch-right-half-disk} expresses the mismatch between $\dot{\mathbf{W}}^a(\lambda;\chi,\tau)$ and $\dot{\mathbf{W}}^\mathrm{out}(\lambda;\chi,\tau)$ on the entire boundary $\partial D_a(\delta)$, that is, we have
% \begin{equation}
%\dot{\mathbf{W}}^a(\lambda)\dot{\mathbf{W}}^\mathrm{out}(\lambda)^{-1} = \mathbf{A}^a(\lambda)\mathbf{U}(\zeta_a) \zeta_a^{\ii p \sigma_3} \mathbf{A}^a(\lambda)^{-1},\quad \lambda\in\partial D_a(\delta).
%\label{eq:error-PC-a}
%\end{equation}
%Using the asymptotic expansion \eqref{eq:PC-asymptotics} for the (parabolic cylinder) matrix function $\mathbf{U}^\mathrm{PC}$ we obtain as in \eqref{eq:error-PC-disk-b-large-M} the asymptotic estimate \textcolor{red}{[Again refer to Channels.]}
%\begin{equation}
%\sup_{\lambda \in \partial D_a(\delta)}\| \dot{\mathbf{W}}^a(\lambda) \dot{\mathbf{W}}^{\mathrm{out}}(\lambda)^{-1} -\mathbb{I} \| = O(M^{-\frac{1}{2}}),\quad M\to+\infty.
%\label{eq:error-PC-disk-a-large-M}
%\end{equation}

\subsubsection{Inner parametrix construction near the points $\lambda_0(\chi,\tau)$ and $\lambda_0(\chi,\tau)^*$} We now let $D_{\lambda_0}(\delta)$ and $D_{\lambda_0^*}(\delta) = D_{\lambda_0}(\delta)^*$ denote disks of small radius $\delta$ independent of $M$ centered at $\lambda=\lambda_0(\chi,\tau)$ and $\lambda=\lambda_0(\chi,\tau)^*$ respectively. Recalling that $h(\lambda_0(\chi,\tau);\chi,\tau) = \kappa(\chi,\tau)$ and $h'(\lambda;\chi,\tau)$ vanishes like a square root as $\lambda\to\lambda_0(\chi,\tau)$, a procedure almost exactly like the one following \eqref{eq:Airy-map-Schi-Stau-ALT} in Section~\ref{sec:Airy-parametrix} (replacing both of the boundary values $h_\pm$ in \eqref{eq:Airy-map-Schi-Stau-ALT} with $h$) leads to the construction of an inner parametrix $\dot{\mathbf{W}}^{\lambda_0}(\lambda)$ on $D_{\lambda_0}(\delta)$ in terms of Airy functions which takes continuous boundary values and satisfies exactly the same jump conditions within $D_{\lambda_0}(\delta)$ as $\mathbf{W}(\lambda)$. Moreover, across the boundary $\partial D_{\lambda_0}(\delta)$ this inner parametrix satisfies
\begin{equation}
\sup_{\lambda \in \partial D_{\lambda_0} } \| \dot{\mathbf{W}}^{\lambda_0}(\lambda)\dot{\mathbf{W}}^{\mathrm{out}}(\lambda)^{-1} -\mathbb{I} \| = O(M^{-1}),\quad M\to+\infty.
\label{eq:error-Airy-disk-plus}
\end{equation}
Since the matrix $\mathbf{W}(\lambda)$ satisfies $\mathbf{W}(\lambda^*)=\sigma_2\mathbf{W}(\lambda)^*\sigma_2$, we may define as in Section~\ref{sec:Airy-parametrix} a second inner parametrix for $\lambda\in D_{\lambda_0^*}(\delta)$ to respect this symmetry, which, of course, satisfies
\begin{equation}
\sup_{\lambda \in \partial D_{\lambda_0^*} } \| \dot{\mathbf{W}}^{\lambda_0^*}(\lambda)\dot{\mathbf{W}}^{\mathrm{out}}(\lambda)^{-1} -\mathbb{I} \| = O(M^{-1}),\quad M\to+\infty.
\label{eq:error-Airy-disk-minus}
\end{equation}

A \emph{global parametrix} $\dot{\mathbf{W}}(\lambda)=\dot{\mathbf{W}}(\lambda;\chi,\tau,\mathbf{Q}^{-s},M)$ is finally constructed by assembling the outer and inner parametrices as follows:
\begin{equation}
\dot{\mathbf{W}}(\lambda)\defeq 
\begin{cases}
\dot{\mathbf{W}}^{\lambda_0} (\lambda),&\quad \lambda\in D_{\lambda_0}(\delta),\\
\dot{\mathbf{W}}^{\lambda_0^*} (\lambda),&\quad \lambda\in D_{\lambda_0^*}(\delta),\\
\dot{\mathbf{W}}^a (\lambda),&\quad \lambda\in D_a(\delta),\\
\dot{\mathbf{W}}^b (\lambda),&\quad \lambda\in D_b(\delta),\\
\dot{\mathbf{W}}^\mathrm{out} (\lambda),&\quad \lambda\in \mathbb{C}\setminus(\Sigma_g \cup I \cup \overline{D_{\lambda_0}(\delta)\cup D_{\lambda_0^*}(\delta) \cup D_a(\delta) \cup D_b(\delta)}).
\end{cases}
\label{eq:W-dot-bun}
\end{equation}

\subsection{Small norm problem for the error and large-$M$ expansion}
To analyze the accuracy of the global parametrix $\dot{\mathbf{W}}(\lambda)$ for $(\chi,\tau)\in \shelves$, we define the \emph{error}
\begin{equation}
\mathbf{F}(\lambda) \defeq  \mathbf{W}(\lambda) \dot{\mathbf{W}}(\lambda)^{-1}.
\label{eq:F-bun}
\end{equation} 
As $\dot{\mathbf{W}}(\lambda)$ satisfies exactly the same jump conditions as $\mathbf{W}(\lambda)$ inside the disks $D_\lambda(\delta)$, $\lambda=a,b,\lambda_0,\lambda_0^*$, and on portions of the arcs $\Sigma_g$ and $I$ exterior to these disks, $\mathbf{F}(\lambda)$ can be taken as an analytic function of $\lambda\in\mathbb{C}$ with the exception the contour system $\Sigma_\mathbf{F}$, which consists of the portions of the arcs $C^\pm_{\Gamma, L}, C^\pm_{\Gamma, R}, C^\pm_{\Sigma, L}, C^\pm_{\Sigma, R}$ lying outside the disks $D_{a,b}(\delta)$ and $D_{\lambda_0,\lambda_0^*}(\delta)$ along with the four disk boundaries $\partial D_{a,b}(\delta)$ and $\partial D_{\lambda_0,\lambda_0^*}(\delta)$. We denote by $\mathbf{V}^{\mathbf{F}}(\lambda)$ the jump matrix for $\mathbf{F}(\lambda)$, which is supported on $\Sigma_\mathbf{F}$. On the arcs $C^\pm_{\Gamma, L}, C^\pm_{\Gamma, R}, C^\pm_{\Sigma, L}, C^\pm_{\Sigma, R}$ \emph{outside} the four disks, we can express $\mathbf{V}^{\mathbf{F}}(\lambda)$ as
\begin{equation}
\begin{split}
\mathbf{V}^{\mathbf{F}}(\lambda) &= \mathbf{F}_-(\lambda)^{-1}\mathbf{F}_+ (\lambda)\\
&=\dot{\mathbf{W}}^\mathrm{out}(\lambda)\mathbf{W}_-(\lambda)^{-1}\mathbf{W}_+ (\lambda)\dot{\mathbf{W}}^\mathrm{out}(\lambda)^{-1}.
\end{split}
\end{equation}
Since $\dot{\mathbf{W}}^\mathrm{out}(\lambda)$ remains bounded with unit determinant as $M\to+\infty$ and $\delta$ is fixed, there exists a positive constant $\nu>0$ such that $\mathbf{V}^\mathbf{F}(\lambda) - \mathbb{I} = O(\ee^{-\nu M})$ holds uniformly on the jump contour $\Sigma_\mathbf{F}$ for $\mathbf{F}(\lambda)$ except on the circles $\partial D_{a,b}(\delta)$ and $\partial D_{\lambda_0,\lambda_0^*}(\delta)$. On the circles, the jump matrix for $\mathbf{F}(\lambda)$ takes the form:
\begin{alignat}{2}
 \mathbf{F}_+(\lambda) &= \mathbf{F}_-(\lambda) \cdot \dot{\mathbf{W}}^{a,b}(\lambda)\dot{\mathbf{W}}^\mathrm{out}(\lambda)^{-1},&&\quad \lambda \in \partial D_{a,b}(\delta),\label{eq:F-jump-bun-a-b}\\
  \mathbf{F}_+(\lambda) &= \mathbf{F}_+(\lambda) \cdot \dot{\mathbf{W}}^{\lambda_0,\lambda_0^*}(\lambda)\dot{\mathbf{W}}^\mathrm{out}(\lambda)^{-1},&&\quad \lambda \in \partial D_{\lambda_0,\lambda_0^*}(\delta),
  \label{eq:F-jump-bun-lambda0}
\end{alignat}
because $\mathbf{W}(\lambda)$ is continuous across each of the four circles. Recalling the estimates \eqref{eq:error-PC-disk-b-large-M} and \eqref{eq:error-PC-disk-a-large-M}, it is seen from \eqref{eq:F-jump-bun-a-b} that $\mathbf{V}^\mathbf{F}(\lambda) - \mathbb{I} = O(M^{-\frac{1}{2}})$ holds uniformly on the circles $\partial D_{a,b}(\delta)$. Similarly, we see from \eqref{eq:error-Airy-disk-plus}-\eqref{eq:error-Airy-disk-minus} and \eqref{eq:F-jump-bun-lambda0} that $\mathbf{V}^\mathbf{F}(\lambda) - \mathbb{I} = O(M^{-1})$ holds uniformly on the circles $\partial D_{\lambda_0, \lambda_0^*}(\delta)$. Thus, it follows that $ \mathbf{F}_+(\lambda) = \mathbf{F}_-(\lambda)( \mathbb{I} + O(M^{-\frac{1}{2}}))$ holds uniformly as $M\to+\infty$ on the compact jump contour $\Sigma_\mathbf{F}$. Standard small-norm theory for such Riemann-Hilbert problems implies that $\mathbf{F}_-(\lambda) = \mathbb{I} + O(M^{-\frac{1}{2}})$ holds in the $L^2$ sense on $\Sigma_\mathbf{F}$, in the limit $M\to +\infty$.

\subsection{Asymptotic formula for $q(x,t;\mathbf{Q}^{-s},M)$ and fundamental rogue waves for $(\chi,\tau)\in \shelves$} We note that for the matrix function $\mathbf{W}(\lambda)=\mathbf{W}(\lambda;\chi,\tau,\mathbf{Q}^{-s},M)$
%\begin{equation}
%\mathbf{X}^{(k)} (\lambda;\chi,\tau) = \mathbf{W}^{(k)} (\lambda;\chi,\tau) \ee^{S(\lambda;\chi,\tau)\sigma_3} = \mathbf{T}^{(k)} (\lambda;\chi,\tau) \ee^{S(\lambda)\sigma_3} = \mathbf{S}^{(k)} (\lambda;\chi,\tau) \ee^{n g(\lambda;\chi,\tau)\sigma_3} \ee^{S(\lambda;\chi,\tau)\sigma_3}
%\end{equation}
\begin{equation}
\mathbf{W} (\lambda)  = \mathbf{T} (\lambda) = \mathbf{S} (\lambda) \ee^{M g(\lambda;\chi,\tau)\sigma_3} 
\end{equation}
holds for $|\lambda|$ sufficiently large; therefore, from \eqref{eq:q-S} we have the formula
\begin{equation}
%\psi_k(n\chi, n\tau) = 2\ii \ee^{-\ii n \tau} \lim_{\lambda\to\infty} \left( \lambda X^{(k)}_{12}(\lambda;\chi,\tau) \ee^{n g(\lambda;\chi,\tau)} \ee^{S(\lambda;\chi,\tau)}\right).
q(M\chi, M\tau; \mathbf{Q}^{-s}, M ) = 2\ii  \lim_{\lambda\to\infty} \left( \lambda W_{12}(\lambda) \ee^{\ii M g(\lambda;\chi,\tau)}\right).
\label{eq:psi-k-W}
\end{equation}
On the other hand, we see from the definitions \eqref{eq:W-dot-bun} and \eqref{eq:F-bun} that 
\begin{equation}
\mathbf{W}(\lambda) = \mathbf{F}(\lambda) \dot{\mathbf{W}}^{\mathrm{out}}(\lambda)
\end{equation}
also holds for $|\lambda|$ sufficiently large; therefore, \eqref{eq:psi-k-W} is expressed as:
\begin{equation}
q(M\chi, M\tau; \mathbf{Q}^{-s}, M ) = 2\ii  \lim_{\lambda\to\infty} \left( \lambda F_{11}(\lambda) \dot{W}^{\mathrm{out}}_{12}(\lambda) +  \lambda F_{12}(\lambda) \dot{W}^{\mathrm{out}}_{22}(\lambda) \right).
\label{eq:psi-k-F}
\end{equation}
%As $\dot{\mathbf{W}}^{\mathrm{out}}(\lambda)$ is a full matrix that tends to identity as $\lambda\to\infty$, this formula simplifies as
Now $\dot{\mathbf{W}}^\mathrm{out}(\lambda)$ tends to the identity as $\lambda\to\infty$, and from \eqref{eq:F-bun} so does $\mathbf{F}(\lambda)$.  Therefore,
\begin{equation}
q(M\chi, M\tau; \mathbf{Q}^{-s}, M ) = 2\ii  \lim_{\lambda\to\infty} \left( \lambda  \dot{W}^{\mathrm{out}}_{12}(\lambda) +   \lambda F_{12}(\lambda) \right).
\label{eq:psi-k-F-simp}
\end{equation}
Recalling that $K(\lambda;\chi,\tau) +\mu(\chi,\tau)= O(\lambda^{-1})$ as $\lambda\to\infty$, it is easily seen from the definitions \eqref{eq:H-def} and \eqref{eq:W-out-full} that
%\begin{equation}
%%\lim_{\lambda\to\infty} \lambda \dot{W}^{\mathrm{out}}_{12}(\lambda;\chi,\tau)= \frac{-s B(\chi,\tau)}{2} \ee^{-2\ii \Theta_0(\chi,\tau;M)}, 
%\lim_{\lambda\to\infty} \lambda \dot{W}^{\mathrm{out}}_{12}(\lambda)= \frac{-s B(\chi,\tau)}{2} \ee^{-2\ii (M \kappa(\chi,\tau) +  \mu(\chi,\tau) )}, 
%\label{eq:psi-k-bun-outer}
%\end{equation}
%\begin{equation}
%%\begin{split}
%%\lim_{\lambda\to\infty} \lambda \dot{W}^{\mathrm{out}}_{12}(\lambda;\chi,\tau)= \frac{-s B(\chi,\tau)}{2} \ee^{-2\ii \Theta_0(\chi,\tau;M)}, 
%\lim_{\lambda\to\infty} \lambda \dot{W}^{\mathrm{out}}_{12}(\lambda) 
%=\frac{-\ii B(\chi,\tau)}{2} \ee^{-2\ii (M \kappa(\chi,\tau) +  \mu(\chi,\tau) + \frac{1}{4} s \pi)}.
%%&= \frac{-s B(\chi,\tau)}{2} \ee^{-2\ii (M \kappa(\chi,\tau) +  \mu(\chi,\tau) )}, 
%%\end{split}
%\label{eq:psi-k-bun-outer}
%\end{equation}
\begin{equation}
%\begin{split}
%\lim_{\lambda\to\infty} \lambda \dot{W}^{\mathrm{out}}_{12}(\lambda;\chi,\tau)= \frac{-s B(\chi,\tau)}{2} \ee^{-2\ii \Theta_0(\chi,\tau;M)}, 
2\ii \lim_{\lambda\to\infty} \lambda \dot{W}^{\mathrm{out}}_{12}(\lambda) 
=B(\chi,\tau) \ee^{-2\ii (M \kappa(\chi,\tau) +  \mu(\chi,\tau) + \frac{1}{4} s \pi)} = \mathfrak{L}_s^{[\shelves]}(\chi,\tau;M),
%&= \frac{-s B(\chi,\tau)}{2} \ee^{-2\ii (M \kappa(\chi,\tau) +  \mu(\chi,\tau) )}, 
%\end{split}
\label{eq:psi-k-bun-outer}
\end{equation}
producing the leading term for $q(M\chi,M\tau; \mathbf{Q}^{-s},M)$ given in \eqref{eq:leading-term-shelves-q}.
%where we have introduced the (real-valued) angle
%\begin{equation}
%\Theta_0(\chi,\tau;M)\defeq  n \kappa(\chi,\tau) + s \gamma(\chi,\tau) + \mu(\chi,\tau).
%\label{eq:Theta-0}
%\end{equation}

It now remains to compute the contribution in \eqref{eq:psi-k-F} coming from $\lambda F_{12}(\lambda;\chi,\tau)$ as $\lambda\to\infty$. Formulating the jump condition for $\mathbf{F}(\lambda)$ in the form $\mathbf{F}_+ - \mathbf{F}_- = \mathbf{F}_- (\mathbf{V}^{\mathbf{F}}-\mathbb{I})$ and using the fact that $\mathbf{F}(\lambda)\to\mathbb{I}$ as $\lambda\to\infty$, we obtain from the Plemelj formula
the same representation as in \eqref{eq:F-Cauchy-channels} for $\mathbf{F}(\lambda)$. It then follows that
%\begin{equation}
%\mathbf{F}(\lambda) = \mathbb{I} + \frac{1}{2\pi \ii} \int_{\Sigma_\mathbf{F}} \frac{\mathbf{F}_{-}(\eta)(\mathbf{V}^{\mathbf{F}}(\eta) - \mathbb{I})}{\eta-\lambda}\dd \eta,\quad \lambda\in\mathbb{C}\setminus\Sigma_\mathbf{F}.
%\label{eq:F-Plemelj}
%\end{equation}
%Thus, 
$\mathbf{F}(\lambda)$ has the Laurent series expansion which is convergent for sufficiently large $|\lambda|$:
\begin{equation}
\mathbf{F}(\lambda) = \mathbb{I} - \frac{1}{2\pi \ii} \sum_{m=1}^{\infty} \lambda^{-m} \int_{\Sigma_\mathbf{F}} \mathbf{F}_-(\eta)(\mathbf{V}^{\mathbf{F}}(\eta)-\mathbb{I}) \eta^{m-1}\, \dd \eta,\quad |\lambda|>|\Sigma_\mathbf{F}|\defeq \sup_{\eta\in\Sigma_\mathbf{F}}|\eta|.
\label{eq:F-Laurent-bun}
\end{equation}
We obtain from this expansion the integral representation 
%(after adding and subtracting $1$ to $F_{11-}(\eta)$)
%\begin{equation}
%\lim_{\lambda\to\infty} \lambda F_{12}(\lambda) =
%-\frac{1}{2\pi \ii} \left \lbrace \int_{\Sigma_\mathbf{F}} F_{11-}(\eta)V^{\mathbf{F}}_{12}(\eta)\dd \eta
%+ \int_{\Sigma_\mathbf{F}} F_{12-}(\eta)(V^{\mathbf{F}}_{22}(\eta) -1 ) \dd \eta
% \right\rbrace,
%\end{equation}
%which we express as
\begin{multline}
\lim_{\lambda\to\infty} \lambda F_{12}(\lambda) =
-\frac{1}{2\pi \ii} \left \lbrace \int_{\Sigma_\mathbf{F}} ( F_{11-}(\eta) - 1)V^{\mathbf{F}}_{12}(\eta)\dd \eta\right. \\
\left. +\int_{\Sigma_\mathbf{F}} V^{\mathbf{F}}_{12}(\eta)\dd \eta
+ \int_{\Sigma_\mathbf{F}} F_{12-}(\eta)(V^{\mathbf{F}}_{22}(\eta) -1 ) \dd \eta
 \right\rbrace.
 \label{eq:lambda-F-12-bun}
\end{multline}
We recall that $\mathbf{F}(\lambda)-\mathbb{I} = O(M^{-\frac{1}{2}})$ in the $L^2$ sense and $\mathbf{V}^{\mathbf{F}}(\lambda)-\mathbb{I} = O(M^{-\frac{1}{2}})$ in the $L^\infty$ sense on $\Sigma^{\mathbf{F}}$, in the limit $M\to+\infty$. As the $L^1$ norm is subordinate to the $L^2$ norm on the compact contour $\Sigma^\mathbf{F}$, direct application of Cauchy-Schwarz inequality shows that the first and the last integrals in \eqref{eq:lambda-F-12-bun} are both of size $O(M^{-1})$ as $M\to+\infty$. Combining this fact with \eqref{eq:psi-k-bun-outer} in the formula \eqref{eq:psi-k-F-simp} yields
%\begin{multline}
%%\psi_k(M\chi,M\tau) =  2\ii \ee^{-\ii n \tau} \left( \frac{-s B(\chi,\tau)}{2}\ee^{-2\ii \Theta_0(\chi,\tau;M)}-\frac{1}{2\pi \ii}\int_{\Sigma_\mathbf{F}} V^{\mathbf{F}}_{12}(\eta;\chi,\tau)\dd \eta \right) + O(n^{-1}), \quad n\to+\infty.
%q(M\chi, M\tau; \mathbf{Q}^{-s},M) =  2\ii  \left( \frac{-s B(\chi,\tau)}{2}\ee^{-2\ii (M \kappa(\chi,\tau)+\mu(\chi,\tau) )}-\frac{1}{2\pi \ii}\int_{\Sigma_\mathbf{F}} V^{\mathbf{F}}_{12}(\eta)\dd \eta \right) + O(M^{-1}), \\
%\quad M\to+\infty.
%\end{multline}
%\begin{multline}
%%\psi_k(M\chi,M\tau) =  2\ii \ee^{-\ii n \tau} \left( \frac{-s B(\chi,\tau)}{2}\ee^{-2\ii \Theta_0(\chi,\tau;M)}-\frac{1}{2\pi \ii}\int_{\Sigma_\mathbf{F}} V^{\mathbf{F}}_{12}(\eta;\chi,\tau)\dd \eta \right) + O(n^{-1}), \quad n\to+\infty.
%q(M\chi, M\tau; \mathbf{Q}^{-s},M) =  2\ii  \left( \frac{-\ii B(\chi,\tau)}{2}\ee^{-2\ii (M \kappa(\chi,\tau)+\mu(\chi,\tau) +\frac{1}{4}s \pi)}-\frac{1}{2\pi \ii}\int_{\Sigma_\mathbf{F}} V^{\mathbf{F}}_{12}(\eta)\dd \eta \right) + O(M^{-1}), \\
%\quad M\to+\infty.
%\end{multline}
\begin{equation}
%\psi_k(M\chi,M\tau) =  2\ii \ee^{-\ii n \tau} \left( \frac{-s B(\chi,\tau)}{2}\ee^{-2\ii \Theta_0(\chi,\tau;M)}-\frac{1}{2\pi \ii}\int_{\Sigma_\mathbf{F}} V^{\mathbf{F}}_{12}(\eta;\chi,\tau)\dd \eta \right) + O(n^{-1}), \quad n\to+\infty.
q(M\chi, M\tau; \mathbf{Q}^{-s},M) =  \mathfrak{L}_s^{[\shelves]}(\chi,\tau;M) -\frac{1}{\pi}\int_{\Sigma_\mathbf{F}} V^{\mathbf{F}}_{12}(\eta)\dd \eta  + O(M^{-1}),
\quad M\to+\infty.
\end{equation}
Note that $V^\mathbf{F}_{12}(\lambda)$ is $O(M^{-1})$ on the circles $\partial D_{\lambda_0,\lambda_0^*}(\delta)$ as $M\to+\infty$ and it is  $O(\ee^{-\nu M})$ on the portions of the arcs $C^\pm_{\Gamma, L}, C^\pm_{\Gamma, R}, C^\pm_{\Sigma, L}, C^\pm_{\Sigma, R}$ lying outside the four disks, whereas $V^\mathbf{F}_{12}(\lambda)$ is $O(M^{-\frac{1}{2}})$ on the circles $\partial D_{a,b}(\delta)$. Therefore, the same formula as above holds with a different error of the same size when the integration contour $\Sigma_{\mathbf{F}}$ is replaced with $\partial D_{a}(\delta) \cup \partial D_{b}(\delta)$:
%\begin{multline}
%q(M\chi, M\tau; \mathbf{Q}^{-s},M)=  \ee^{-\ii M \tau} \left( -\ii s B(\chi,\tau) \ee^{-2\ii (M \kappa(\chi,\tau)+\mu(\chi,\tau) )}-\frac{1}{\pi}\int_{\partial D_{a}(\delta) \cup \partial D_{b}(\delta)} V^{\mathbf{F}}_{12}(\eta)\dd \eta \right)\\
% + O(M^{-1}), \quad M\to+\infty.
%\label{eq:psi-k-V-12-bun}
%\end{multline}
%\begin{multline}
%q(M\chi, M\tau; \mathbf{Q}^{-s},M)=B(\chi,\tau) \ee^{-2\ii (M \kappa(\chi,\tau)+\mu(\chi,\tau) +\frac{1}{4}s \pi )}-\frac{1}{\pi}\int_{\partial D_{a}(\delta) \cup \partial D_{b}(\delta)} V^{\mathbf{F}}_{12}(\eta)\dd \eta \\
% + O(M^{-1}), \quad M\to+\infty.
%\label{eq:psi-k-V-12-bun}
%\end{multline}
\begin{equation}
q(M\chi, M\tau; \mathbf{Q}^{-s},M)=\mathfrak{L}_s^{[\shelves]}(\chi,\tau;M) - \frac{1}{\pi}\int_{\partial D_{a}(\delta) \cup \partial D_{b}(\delta)} V^{\mathbf{F}}_{12}(\eta)\dd \eta 
 + O(M^{-1}), \quad M\to+\infty.
\label{eq:psi-k-V-12-bun}
\end{equation}
Using the asymptotic expansion \eqref{eq:PCU-asymp} in the formul\ae{} \eqref{eq:error-PC-b} and \eqref{eq:error-PC-a} and recalling that $\mathbf{V}^{\mathbf{F}}(\lambda) = \dot{\mathbf{W}}^{a,b}(\lambda)\dot{\mathbf{W}}^{\mathrm{out}}(\lambda)^{-1}$ for $\lambda\in \partial D_{a,b}(\delta)$, we see that
\begin{equation}
V^{\mathbf{F}}_{12}(\lambda) = \frac{1}{2\ii M^{\frac{1}{2}}} 
\left( 
\frac{\alpha Y_{11}^{a,b}(\lambda)^2 + \beta Y_{12}^{a,b}(\lambda)^2}{f_{a,b}(\lambda;\chi,\tau)}  
\right)
+ O(M^{-1}),\quad M\to+\infty, \quad \lambda\in\partial D_{a,b}(\delta).
\label{eq:VF-12-bun}
\end{equation}
The definition \eqref{eq:A-b} for $\mathbf{Y}^b(\lambda)$ together with the fact that $\mathbf{H}^b(\lambda)$ is a diagonal matrix directly gives
\begin{align}
Y^b_{11}(\lambda)^2 &= s L_{11}(\lambda)^2 \ee^{-2 \ii (K(\lambda;\chi,\tau) + \mu(\chi,\tau) )} \ee^{-2\ii M {h}_b(\chi,\tau)} M^{\ii p} (\lambda-a(\chi,\tau))^{2\ii p} \left( \frac{f_b(\lambda;\chi,\tau)}{\lambda-b(\chi,\tau)} \right)^{2\ii p},
\label{eq:A-b-11}\\
Y^b_{12}(\lambda)^2 &= s L_{12}(\lambda)^2 \ee^{2 \ii( K(\lambda;\chi,\tau) +\mu(\chi,\tau) )} \ee^{2\ii M {h}_b(\chi,\tau)} M^{-\ii p}(\lambda-a(\chi,\tau))^{-2\ii p} \left( \frac{f_b(\lambda;\chi,\tau)}{\lambda-b(\chi,\tau)} \right)^{-2\ii p},
\label{eq:A-b-12}
\end{align}
%\begin{align}
%Y^b_{11}(\lambda)^2 &= s H_{11}(\lambda)^2 \ee^{-2 \ii (K(\lambda;\chi,\tau) + \mu(\chi,\tau) )} \ee^{-2\ii M {h}_b(\chi,\tau)} M^{\ii p} (\lambda-a(\chi,\tau))^{2\ii p} \left( \frac{f_b(\lambda;\chi,\tau)}{\lambda-b(\chi,\tau)} \right)^{2\ii p},
%\label{eq:A-b-11}\\
%Y^b_{12}(\lambda)^2 &= s H_{12}(\lambda)^2 \ee^{2 \ii( K(\lambda;\chi,\tau) +\mu(\chi,\tau) )} \ee^{2\ii M {h}_b(\chi,\tau)} M^{-\ii p}(\lambda-a(\chi,\tau))^{-2\ii p} \left( \frac{f_b(\lambda;\chi,\tau)}{\lambda-b(\chi,\tau)} \right)^{-2\ii p},
%\label{eq:A-b-12}
%\end{align}
%\begin{equation}
%%A^b_{11}(\lambda;\chi,\tau)^2 = s H_{11}(\lambda;\chi,\tau)^2 \ee^{-2(K(\lambda;\chi,\tau) + S(\lambda;\chi,\tau))}\omega(\lambda)^{2s} \ee^{-2\ii n \tilde{h}_b(\chi,\tau)} n^{\ii p}\\ \cdot (\lambda-a(\chi,\tau))^{2\ii p} \left( \frac{f_b(\lambda;\chi,\tau)}{\lambda-b(\chi,\tau)} \right)^{2\ii p}
%%A^b_{11}(\lambda;\chi,\tau)^2 = s H_{11}(\lambda;\chi,\tau)^2 \ee^{-2(\ii K(\lambda;\chi,\tau) + S(\lambda;\chi,\tau))}\omega(\lambda)^{2s} \ee^{-2\ii n \tilde{h}_b(\chi,\tau)} n^{\ii p}\\ \cdot (\lambda-a(\chi,\tau))^{2\ii p} \left( \frac{f_b(\lambda;\chi,\tau)}{\lambda-b(\chi,\tau)} \right)^{2\ii p}
%%A^b_{11}(\lambda)^2 = s H_{11}(\lambda)^2 \ee^{-2 \ii K(\lambda;\chi,\tau) } \ee^{-2\ii M \tilde{h}_b(\chi,\tau)} M^{\ii p} (\lambda-a(\chi,\tau))^{2\ii p} \left( \frac{f_b(\lambda;\chi,\tau)}{\lambda-b(\chi,\tau)} \right)^{2\ii p}
%%\label{eq:A-b-11}
%Y^b_{11}(\lambda)^2 = s H_{11}(\lambda)^2 \ee^{-2 \ii (K(\lambda;\chi,\tau) + \mu(\chi,\tau) )} \ee^{-2\ii M {h}_b(\chi,\tau)} M^{\ii p} (\lambda-a(\chi,\tau))^{2\ii p} \left( \frac{f_b(\lambda;\chi,\tau)}{\lambda-b(\chi,\tau)} \right)^{2\ii p}
%\label{eq:A-b-11}
%\end{equation}
%and
%\begin{equation}
%%A^b_{12}(\lambda;\chi,\tau)^2 = s H_{12}(\lambda;\chi,\tau)^2 \ee^{2(K(\lambda;\chi,\tau) + S(\lambda;\chi,\tau))}\omega(\lambda)^{-2s} \ee^{2\ii n \tilde{h}_b(\chi,\tau)} n^{-\ii p}\\ \cdot (\lambda-a(\chi,\tau))^{-2\ii p} \left( \frac{f_b(\lambda;\chi,\tau)}{\lambda-b(\chi,\tau)} \right)^{-2\ii p},
%%A^b_{12}(\lambda)^2 = s H_{12}(\lambda)^2 \ee^{2 \ii K(\lambda;\chi,\tau) } \ee^{2\ii M \tilde{h}_b(\chi,\tau)} M^{-\ii p}(\lambda-a(\chi,\tau))^{-2\ii p} \left( \frac{f_b(\lambda;\chi,\tau)}{\lambda-b(\chi,\tau)} \right)^{-2\ii p},
%%\label{eq:A-b-12}
%Y^b_{12}(\lambda)^2 = s H_{12}(\lambda)^2 \ee^{2 \ii( K(\lambda;\chi,\tau) +\mu(\chi,\tau) )} \ee^{2\ii M {h}_b(\chi,\tau)} M^{-\ii p}(\lambda-a(\chi,\tau))^{-2\ii p} \left( \frac{f_b(\lambda;\chi,\tau)}{\lambda-b(\chi,\tau)} \right)^{-2\ii p},
%\label{eq:A-b-12}
%\end{equation}
where to arrive at the latter formula we have used $s=\pm 1$. Similarly, the definition \eqref{eq:A-a} for $\mathbf{Y}^{a}(\lambda)$, this time together with the fact that $\mathbf{H}^a(\lambda)$ is off-diagonal and with perhaps more tedious arithmetic gives
%\begin{align}
%Y^a_{11}(\lambda)^2 &= - s H_{R,12}(\lambda)^2 \ee^{2\ii (K_R(\lambda;\chi,\tau) + \mu(\chi,\tau) )} \ee^{2\ii M {h}_a(\chi,\tau)} M^{\ii p} (b(\chi,\tau)-\lambda)^{2\ii p} \left( \frac{a(\chi,\tau) - \lambda}{f_a(\lambda;\chi,\tau)} \right)^{-2\ii p}
%\label{eq:A-a-11}\\
%Y^a_{12}(\lambda)^2 &= - s H_{R,11}(\lambda)^2 \ee^{-2 \ii (K_R(\lambda;\chi,\tau) + \mu(\chi,\tau) ) } \ee^{-2\ii M {h}_a(\chi,\tau)} M^{-\ii p}(b(\chi,\tau)-\lambda)^{-2\ii p} \left( \frac{a(\chi,\tau) - \lambda}{f_a(\lambda;\chi,\tau)} \right)^{2\ii p},
%\label{eq:A-a-12}
%\end{align}
\begin{align}
Y^a_{11}(\lambda)^2 &= - s L_{-,12}(\lambda)^2 \ee^{2\ii (K_-(\lambda;\chi,\tau) + \mu(\chi,\tau) )} \ee^{2\ii M {h}_a(\chi,\tau)} M^{\ii p} (b(\chi,\tau)-\lambda)^{2\ii p} \left( \frac{a(\chi,\tau) - \lambda}{f_a(\lambda;\chi,\tau)} \right)^{-2\ii p},
\label{eq:A-a-11}\\
Y^a_{12}(\lambda)^2 &= - s L_{-,11}(\lambda)^2 \ee^{-2 \ii (K_-(\lambda;\chi,\tau) + \mu(\chi,\tau) ) } \ee^{-2\ii M {h}_a(\chi,\tau)} M^{-\ii p}(b(\chi,\tau)-\lambda)^{-2\ii p} \left( \frac{a(\chi,\tau) - \lambda}{f_a(\lambda;\chi,\tau)} \right)^{2\ii p},
\label{eq:A-a-12}
\end{align}
%\begin{equation}
%%A^a_{11}(\lambda;\chi,\tau)^2 = - s H_{R,12}(\lambda;\chi,\tau)^2 \ee^{2(K_R(\lambda;\chi,\tau) + S_R(\lambda;\chi,\tau))}\omega(\lambda)^{-2s} \ee^{2\ii n \tilde{h}_a(\chi,\tau)} n^{\ii p}\\ \cdot (b(\chi,\tau)-\lambda)^{2\ii p} \left( \frac{a(\chi,\tau) - \lambda}{f_a(\lambda;\chi,\tau)} \right)^{-2\ii p}
%%A^a_{11}(\lambda;\chi,\tau)^2 = - s H_{R,12}(\lambda;\chi,\tau)^2 \ee^{2(\ii K_R(\lambda;\chi,\tau) + S_R(\lambda;\chi,\tau))}\omega(\lambda)^{-2s} \ee^{2\ii n \tilde{h}_a(\chi,\tau)} n^{\ii p}\\ \cdot (b(\chi,\tau)-\lambda)^{2\ii p} \left( \frac{a(\chi,\tau) - \lambda}{f_a(\lambda;\chi,\tau)} \right)^{-2\ii p}
%%\label{eq:A-a-11}
%A^a_{11}(\lambda)^2 = - s H_{R,12}(\lambda)^2 \ee^{2\ii K_R(\lambda;\chi,\tau) } \ee^{2\ii M \tilde{h}_a(\chi,\tau)} M^{\ii p} (b(\chi,\tau)-\lambda)^{2\ii p} \left( \frac{a(\chi,\tau) - \lambda}{f_a(\lambda;\chi,\tau)} \right)^{-2\ii p}
%\label{eq:A-a-11}
%\end{equation}
%and
%\begin{equation}
%%A^a_{12}(\lambda;\chi,\tau)^2 = - s H_{R,11}(\lambda;\chi,\tau)^2 \ee^{-2(K_R(\lambda;\chi,\tau) + S_R\lambda;\chi,\tau))}\omega(\lambda)^{2s} \ee^{-2\ii n \tilde{h}_a(\chi,\tau)} n^{-\ii p}\\ \cdot (b(\chi,\tau)-\lambda)^{-2\ii p} \left( \frac{a(\chi,\tau) - \lambda}{f_a(\lambda;\chi,\tau)} \right)^{2\ii p},
%A^a_{12}(\lambda)^2 = - s H_{R,11}(\lambda)^2 \ee^{-2 \ii K_R(\lambda;\chi,\tau) } \ee^{-2\ii M \tilde{h}_a(\chi,\tau)} M^{-\ii p}(b(\chi,\tau)-\lambda)^{-2\ii p} \left( \frac{a(\chi,\tau) - \lambda}{f_a(\lambda;\chi,\tau)} \right)^{2\ii p},
%\label{eq:A-a-12}
%\end{equation}
and to obtain the former formula we have again used $s=\pm 1$. 
Here $\mathbf{L}_-(\lambda)$ and $K_-(\lambda;\chi,\tau)$ are the functions analytic for $\lambda\in D_a(\delta)$ coinciding with $\mathbf{L}(\lambda)=\mathbf{L}(\lambda;\chi,\tau,\mathbf{Q}^{-s},M)$ and  $K(\lambda;\chi,\tau)$, respectively, in $D_{a,-}(\delta)$. 
%Here $\mathbf{L}_R(\lambda)$ and $K_R(\lambda;\chi,\tau)$ are the functions analytic for $\lambda\in D_a(\delta)$ coinciding with $\mathbf{L}(\lambda)=\mathbf{L}(\lambda;\chi,\tau,\mathbf{Q}^{-s},M)$ and  $K(\lambda;\chi,\tau)$, respectively, to the right of $\Sigma_g$ in $D_a(\delta)$. 
Thus, with \eqref{eq:A-b-11} and \eqref{eq:A-b-12}, we see from \eqref{eq:VF-12-bun} that $V^{\mathbf{F}}_{12}(\lambda)$ on the circle $\partial D_b(\delta)$ is given by:
%\begin{equation}
%\begin{split}
%V^{\mathbf{F}}_{12}(\lambda;\chi,\tau) &=
%%\! \begin{multlined}[t]
%%\frac{s \alpha n^{\ii p} \ee^{-2\ii n \tilde{h}_b(\chi,\tau)} }{2\ii n^{\frac{1}{2}} f_b(\lambda;\chi,\tau)} H_{11}(\lambda;\chi,\tau)^2 \ee^{-2(K(\lambda;\chi,\tau) + S(\lambda;\chi,\tau))}     \omega(\lambda)^{2s} (\lambda-a(\chi,\tau))^{2\ii p}  \\
%%\cdot\left( \frac{f_b(\lambda;\chi,\tau)}{\lambda-b(\chi,\tau)} \right)^{2\ii p}
%%\end{multlined}\\
%%%********** Here \! is to get the correct spacing after the = sign **************
%%&\quad - \! \begin{multlined}[t]
%%\frac{s \alpha^* n^{-\ii p} \ee^{2\ii n \tilde{h}_b(\chi,\tau)} }{2\ii n^{\frac{1}{2}} f_b(\lambda;\chi,\tau)} H_{12}(\lambda;\chi,\tau)^2 \ee^{2(K(\lambda;\chi,\tau) + S(\lambda;\chi,\tau))}     \omega(\lambda)^{-2s} (\lambda-a(\chi,\tau))^{-2\ii p}  \\
%%\cdot\left( \frac{f_b(\lambda;\chi,\tau)}{\lambda-b(\chi,\tau)} \right)^{-2\ii p},
%%\end{multlined}
%\! \begin{multlined}[t]
%\frac{s \alpha n^{\ii p} \ee^{-2\ii n \tilde{h}_b(\chi,\tau)} }{2\ii n^{\frac{1}{2}} f_b(\lambda;\chi,\tau)} H_{11}(\lambda;\chi,\tau)^2 \ee^{-2(\ii K(\lambda;\chi,\tau) + S(\lambda;\chi,\tau))}     \omega(\lambda)^{2s} (\lambda-a(\chi,\tau))^{2\ii p}  \\
%\cdot\left( \frac{f_b(\lambda;\chi,\tau)}{\lambda-b(\chi,\tau)} \right)^{2\ii p}
%\end{multlined}\\
%%********** Here \! is to get the correct spacing after the = sign **************
%&\quad - \! \begin{multlined}[t]
%\frac{s \alpha^* n^{-\ii p} \ee^{2\ii n \tilde{h}_b(\chi,\tau)} }{2\ii n^{\frac{1}{2}} f_b(\lambda;\chi,\tau)} H_{12}(\lambda;\chi,\tau)^2 \ee^{2(\ii K(\lambda;\chi,\tau) + S(\lambda;\chi,\tau))}     \omega(\lambda)^{-2s} (\lambda-a(\chi,\tau))^{-2\ii p}  \\
%\cdot\left( \frac{f_b(\lambda;\chi,\tau)}{\lambda-b(\chi,\tau)} \right)^{-2\ii p},
%\end{multlined}
%\end{split}
%\label{eq:V-F-12-b}
%\end{equation}
%\begin{equation}
%\begin{split}
%V^{\mathbf{F}}_{12}(\lambda) &=
%\frac{s \alpha M^{\ii p} \ee^{-2\ii M \tilde{h}_b(\chi,\tau)} }{2\ii M^{\frac{1}{2}} f_b(\lambda;\chi,\tau)} H_{11}(\lambda)^2 \ee^{-2\ii K(\lambda;\chi,\tau) }    (\lambda-a(\chi,\tau))^{2\ii p}  \left( \frac{f_b(\lambda;\chi,\tau)}{\lambda-b(\chi,\tau)} \right)^{2\ii p}\\
%&\quad - 
%\frac{s \alpha^* M^{-\ii p} \ee^{2\ii M \tilde{h}_b(\chi,\tau)} }{2\ii M^{\frac{1}{2}} f_b(\lambda;\chi,\tau)} H_{12}(\lambda)^2 \ee^{2\ii K(\lambda;\chi,\tau)}  (\lambda-a(\chi,\tau))^{-2\ii p} \left( \frac{f_b(\lambda;\chi,\tau)}{\lambda-b(\chi,\tau)} \right)^{-2\ii p},
%\end{split}
%\label{eq:V-F-12-b}
%\end{equation}
%\begin{equation}
%\begin{split}
%V^{\mathbf{F}}_{12}(\lambda) &=
%\frac{s \alpha M^{\ii p} \ee^{-2\ii M {h}_b(\chi,\tau)} }{2\ii M^{\frac{1}{2}} f_b(\lambda;\chi,\tau)} H_{11}(\lambda)^2 \ee^{-2\ii ( K(\lambda;\chi,\tau) +\mu(\chi,\tau) ) }    (\lambda-a(\chi,\tau))^{2\ii p}  \left( \frac{f_b(\lambda;\chi,\tau)}{\lambda-b(\chi,\tau)} \right)^{2\ii p}\\
%&\quad - 
%\frac{s \alpha^* M^{-\ii p} \ee^{2\ii M {h}_b(\chi,\tau)} }{2\ii M^{\frac{1}{2}} f_b(\lambda;\chi,\tau)} H_{12}(\lambda)^2 \ee^{2\ii (K(\lambda;\chi,\tau) + \mu(\chi,\tau) )}  (\lambda-a(\chi,\tau))^{-2\ii p} \left( \frac{f_b(\lambda;\chi,\tau)}{\lambda-b(\chi,\tau)} \right)^{-2\ii p},
%\end{split}
%\label{eq:V-F-12-b}
%\end{equation}
\begin{equation}
\begin{split}
V^{\mathbf{F}}_{12}(\lambda) &=
\frac{s \alpha M^{\ii p} \ee^{-2\ii M {h}_b(\chi,\tau)} }{2\ii M^{\frac{1}{2}} f_b(\lambda;\chi,\tau)} L_{11}(\lambda)^2 \ee^{-2\ii ( K(\lambda;\chi,\tau) +\mu(\chi,\tau) ) }    (\lambda-a(\chi,\tau))^{2\ii p}  \left( \frac{f_b(\lambda;\chi,\tau)}{\lambda-b(\chi,\tau)} \right)^{2\ii p}\\
&\quad - 
\frac{s \alpha^* M^{-\ii p} \ee^{2\ii M {h}_b(\chi,\tau)} }{2\ii M^{\frac{1}{2}} f_b(\lambda;\chi,\tau)} L_{12}(\lambda)^2 \ee^{2\ii (K(\lambda;\chi,\tau) + \mu(\chi,\tau) )}  (\lambda-a(\chi,\tau))^{-2\ii p} \left( \frac{f_b(\lambda;\chi,\tau)}{\lambda-b(\chi,\tau)} \right)^{-2\ii p} \\
&\quad+ O(M^{-1}),\quad \text{in $L^\infty(\partial D_b(\delta))$ as $M\to+\infty$},
\end{split}
\label{eq:V-F-12-b}
\end{equation}
where we have used the property $\beta=-\alpha^*$. Similarly, we see from \eqref{eq:A-a-11} and \eqref{eq:A-a-12} that $V^{\mathbf{F}}_{12}(\lambda)$ on the circle $\partial D_a(\delta)$ is given by:
%\begin{equation}
%\begin{split}
%V^{\mathbf{F}}_{12}(\lambda;\chi,\tau) &=
%%\! \begin{multlined}[t]
%%\frac{-s \alpha n^{\ii p} \ee^{2\ii n \tilde{h}_a(\chi,\tau)} }{2\ii n^{\frac{1}{2}} f_a(\lambda;\chi,\tau)} H_{R,12}(\lambda;\chi,\tau)^2 \ee^{2(K_R(\lambda;\chi,\tau) + S_R(\lambda;\chi,\tau))}     \omega(\lambda)^{-2s} (b(\chi,\tau)-\lambda)^{2\ii p}  \\
%%\cdot\left( \frac{a(\chi,\tau)-\lambda}{f_a(\lambda;\chi,\tau)} \right)^{-2\ii p}
%%\end{multlined}\\
%%%********** Here \! is to get the correct spacing after the = sign **************
%%&\quad + \! \begin{multlined}[t]
%%\frac{s \alpha^* n^{-\ii p} \ee^{-2\ii n \tilde{h}_a(\chi,\tau)} }{2\ii n^{\frac{1}{2}} f_a(\lambda;\chi,\tau)} H_{R,11}(\lambda;\chi,\tau)^2 \ee^{-2(K_R(\lambda;\chi,\tau) + S_R(\lambda;\chi,\tau))}     \omega(\lambda)^{2s} (b(\chi,\tau)-\lambda)^{-2\ii p}  \\
%%\cdot\left( \frac{a(\chi,\tau)-\lambda}{f_a(\lambda;\chi,\tau)} \right)^{2\ii p}.
%%\end{multlined}
%\! \begin{multlined}[t]
%\frac{-s \alpha n^{\ii p} \ee^{2\ii n \tilde{h}_a(\chi,\tau)} }{2\ii n^{\frac{1}{2}} f_a(\lambda;\chi,\tau)} H_{R,12}(\lambda;\chi,\tau)^2 \ee^{2(\ii K_R(\lambda;\chi,\tau) + S_R(\lambda;\chi,\tau))}     \omega(\lambda)^{-2s} (b(\chi,\tau)-\lambda)^{2\ii p}  \\
%\cdot\left( \frac{a(\chi,\tau)-\lambda}{f_a(\lambda;\chi,\tau)} \right)^{-2\ii p}
%\end{multlined}\\
%%********** Here \! is to get the correct spacing after the = sign **************
%&\quad + \! \begin{multlined}[t]
%\frac{s \alpha^* n^{-\ii p} \ee^{-2\ii n \tilde{h}_a(\chi,\tau)} }{2\ii n^{\frac{1}{2}} f_a(\lambda;\chi,\tau)} H_{R,11}(\lambda;\chi,\tau)^2 \ee^{-2(\ii K_R(\lambda;\chi,\tau) + S_R(\lambda;\chi,\tau))}     \omega(\lambda)^{2s} (b(\chi,\tau)-\lambda)^{-2\ii p}  \\
%\cdot\left( \frac{a(\chi,\tau)-\lambda}{f_a(\lambda;\chi,\tau)} \right)^{2\ii p}.
%\end{multlined}
%\end{split}
%\label{eq:V-F-12-a}
%\end{equation}
%\begin{equation}
%\begin{split}
%V^{\mathbf{F}}_{12}(\lambda) &=
%\frac{-s \alpha M^{\ii p} \ee^{2\ii M \tilde{h}_a(\chi,\tau)} }{2\ii M^{\frac{1}{2}} f_a(\lambda;\chi,\tau)} H_{R,12}(\lambda)^2 \ee^{2\ii K_R(\lambda;\chi,\tau) } (b(\chi,\tau)-\lambda)^{2\ii p} 
%\left( \frac{a(\chi,\tau)-\lambda}{f_a(\lambda;\chi,\tau)} \right)^{-2\ii p}\\
%&\quad + 
%\frac{s \alpha^* M^{-\ii p} \ee^{-2\ii M \tilde{h}_a(\chi,\tau)} }{2\ii M^{\frac{1}{2}} f_a(\lambda;\chi,\tau)} H_{R,11}(\lambda)^2 \ee^{-2\ii K_R(\lambda;\chi,\tau)}  (b(\chi,\tau)-\lambda)^{-2\ii p} 
%\left( \frac{a(\chi,\tau)-\lambda}{f_a(\lambda;\chi,\tau)} \right)^{2\ii p}.
%\end{split}
%\label{eq:V-F-12-a}
%\end{equation}
%\begin{equation}
%\begin{split}
%V^{\mathbf{F}}_{12}(\lambda) &=
%\frac{-s \alpha M^{\ii p} \ee^{2\ii M {h}_a(\chi,\tau)} }{2\ii M^{\frac{1}{2}} f_a(\lambda;\chi,\tau)} H_{R,12}(\lambda)^2 \ee^{2\ii (K_R(\lambda;\chi,\tau) +\mu(\chi,\tau) )} (b(\chi,\tau)-\lambda)^{2\ii p} 
%\left( \frac{a(\chi,\tau)-\lambda}{f_a(\lambda;\chi,\tau)} \right)^{-2\ii p}\\
%&\quad + 
%\frac{s \alpha^* M^{-\ii p} \ee^{-2\ii M {h}_a(\chi,\tau)} }{2\ii M^{\frac{1}{2}} f_a(\lambda;\chi,\tau)} H_{R,11}(\lambda)^2 \ee^{-2\ii (K_R(\lambda;\chi,\tau)+\mu(\chi,\tau) )}  (b(\chi,\tau)-\lambda)^{-2\ii p} 
%\left( \frac{a(\chi,\tau)-\lambda}{f_a(\lambda;\chi,\tau)} \right)^{2\ii p}.
%\end{split}
%\label{eq:V-F-12-a}
%\end{equation}
\begin{equation}
\begin{split}
V^{\mathbf{F}}_{12}(\lambda) &=
\frac{-s \alpha M^{\ii p} \ee^{2\ii M {h}_a(\chi,\tau)} }{2\ii M^{\frac{1}{2}} f_a(\lambda;\chi,\tau)} L_{-,12}(\lambda)^2 \ee^{2\ii (K_-(\lambda;\chi,\tau) +\mu(\chi,\tau) )} (b(\chi,\tau)-\lambda)^{2\ii p} 
\left( \frac{a(\chi,\tau)-\lambda}{f_a(\lambda;\chi,\tau)} \right)^{-2\ii p}\\
&\quad + 
\frac{s \alpha^* M^{-\ii p} \ee^{-2\ii M {h}_a(\chi,\tau)} }{2\ii M^{\frac{1}{2}} f_a(\lambda;\chi,\tau)} L_{-,11}(\lambda)^2 \ee^{-2\ii (K_-(\lambda;\chi,\tau)+\mu(\chi,\tau) )}  (b(\chi,\tau)-\lambda)^{-2\ii p} 
\left( \frac{a(\chi,\tau)-\lambda}{f_a(\lambda;\chi,\tau)} \right)^{2\ii p}\\
&\quad+ O(M^{-1})\quad \text{in $L^\infty(\partial D_a(\delta))$ as $M\to+\infty$}.
\end{split}
\label{eq:V-F-12-a}
\end{equation}
Note that $f_b(\lambda;\chi,\tau)$ in the leftmost factor of \eqref{eq:V-F-12-b} has a simple zero at $\lambda=b(\chi,\tau)$ and $f_a(\lambda;\chi,\tau)$ in the leftmost factor of \eqref{eq:V-F-12-a} has a simple zero at $\lambda=a(\chi,\tau)$, while the rest of the factors in \eqref{eq:V-F-12-b} and \eqref{eq:V-F-12-a} are holomorphic within the relevant disks.
Recalling the clockwise orientation of the circles $\partial D_{a,b}(\delta)$ and using 
\begin{align}
f_a'(a(\chi,\tau);\chi,\tau)& = - \left(-h_a''(\chi,\tau)\right)^{\frac{1}{2}}  \defeq  -\sqrt{-h''_-(a(\chi,\tau);\chi,\tau)}<0,\\
f_b'(b(\chi,\tau);\chi,\tau)& = h_b''(\chi,\tau)^{\frac{1}{2}}  \defeq  \sqrt{h''(b(\chi,\tau);\chi,\tau)}>0,
\end{align}
a simple residue calculation in \eqref{eq:V-F-12-b}--\eqref{eq:V-F-12-a} yields
\begin{multline}
-\frac{1}{\pi} \int_{D_{a}(\delta)} V^{\mathbf{F}}_{12}(\eta)\dd \eta = \frac{s}{M^{\frac{1}{2}} \left(-h_a''(\chi,\tau)\right)^{\frac{1}{2}}}
\left[ 
\alpha M^{\ii p} \ee^{2\ii M {h}_a(\chi,\tau)} L_{a,12}(\chi,\tau)^2 \ee^{2\ii (K_a(\chi,\tau) + \mu(\chi,\tau) )} X_a(\chi,\tau)^{\ii p}
\right.\\
\left. -\alpha^* M^{-\ii p} \ee^{-2\ii M {h}_a(\chi,\tau)}  
L_{a,11}(\chi,\tau)^2 \ee^{-2 \ii (K_a(\chi,\tau) + \mu(\chi,\tau) )}  X_a(\chi,\tau)^{-\ii p}
\right] + O(M^{-1})\quad\text{and}
\label{eq:V-F-12-a-residue}
\end{multline}
\begin{multline}
-\frac{1}{\pi} \int_{D_{b}(\delta)} V^{\mathbf{F}}_{12}(\eta)\dd \eta =
\frac{s}{M^{\frac{1}{2}}  h_b''(\chi,\tau)^{\frac{1}{2}}}
 \left[ 
\alpha M^{\ii p} \ee^{-2\ii M {h}_b(\chi,\tau)} 
L_{b,11}(\chi,\tau)^2 \ee^{-2 \ii (K_b(\chi,\tau) + \mu(\chi,\tau) )}  X_b(\chi,\tau)^{\ii p}
%(b(\chi,\tau)-a(\chi,\tau))^{2\ii p} h_b''(\chi,\tau)^{\ii p}
\right.\\
\left. - \alpha^{*} M^{-\ii p} \ee^{2\ii M {h}_b(\chi,\tau)}
L_{b,12}(\chi,\tau)^2 \ee^{2\ii (K_b(\chi,\tau) + \mu(\chi,\tau) )}  X_b(\chi,\tau)^{-\ii p}
%(b(\chi,\tau)-a(\chi,\tau))^{-2\ii p} h_b''(\chi,\tau)^{-\ii p}
\right] + O(M^{-1}),
\label{eq:V-F-12-b-residue}
\end{multline}
%\begin{equation}
%\begin{split}
%-\frac{1}{\pi} \int_{D_{a}(\delta)} V^{\mathbf{F}}_{12}(\eta)\dd \eta &=
%%\! \begin{multlined}[t]
%%\frac{-s \alpha n^{\ii p} \ee^{2\ii n \tilde{h}_a(\chi,\tau)} }{n^{\frac{1}{2}} f'_a(a(\chi,\tau);\chi,\tau)} H_{a,12}(\chi,\tau)^2 \ee^{2(K_a(\chi,\tau) + S_a(\chi,\tau))}   \\ \cdot  \omega_a(\chi,\tau)^{-2s} (b(\chi,\tau)-a(\chi,\tau))^{2\ii p} 
%%\left( \frac{-1}{f'_a(a(\chi,\tau);\chi,\tau)} \right)^{-2\ii p}
%%\end{multlined}\\
%%&\quad + \! \begin{multlined}[t]
%%\frac{s \alpha^* n^{-\ii p} \ee^{-2\ii n \tilde{h}_a(\chi,\tau)} }{n^{\frac{1}{2}} f'_a(a(\chi,\tau);\chi,\tau)} H_{a,11}(\lambda;\chi,\tau)^2 \ee^{-2(K_a(\chi,\tau) + S_a(\chi,\tau))}  \\ \cdot  \omega_a(\chi,\tau)^{2s}  (b(\chi,\tau)-a(\chi,\tau))^{-2\ii p}  \left( \frac{-1}{f'_a(a(\chi,\tau);\chi,\tau)} \right)^{2\ii p},
%%\end{multlined}
%\! \begin{multlined}[t]
%\frac{s}{M^{\frac{1}{2}} }\cdot \frac{\alpha M^{\ii p} \ee^{2\ii M {h}_a(\chi,\tau)} }{ \left(-h_a''(\chi,\tau)\right)^{\frac{1}{2}} } L_{a,12}(\chi,\tau)^2 \ee^{2\ii (K_a(\chi,\tau) + \mu(\chi,\tau) )}  (b(\chi,\tau)-a(\chi,\tau))^{2\ii p} \\ \cdot
% \left(-h_a''(\chi,\tau)\right)^{\ii p}
%\end{multlined}\\
%&\quad + \! \begin{multlined}[t]
%\frac{s}{M^{\frac{1}{2}} } \cdot \frac{-\alpha^* M^{-\ii p} \ee^{-2\ii M {h}_a(\chi,\tau)} }{ \left(-h_a''(\chi,\tau)\right)^{\frac{1}{2}} } L_{a,11}(\chi,\tau)^2 \ee^{-2 \ii (K_a(\chi,\tau) + \mu(\chi,\tau) )}   (b(\chi,\tau)-a(\chi,\tau))^{-2\ii p}   \\ \cdot  
% \left(-h_a''(\chi,\tau)\right)^{-\ii p}
%\end{multlined}
%\end{split}
%\label{eq:V-F-12-a-residue}
%\end{equation}
%and
%
%
%\begin{equation}
%\begin{split}
%-\frac{1}{\pi} \int_{D_{b}(\delta)} V^{\mathbf{F}}_{12}(\eta)\dd \eta &=
%%\! \begin{multlined}[t]
%%\frac{s \alpha n^{\ii p} \ee^{-2\ii n \tilde{h}_b(\chi,\tau)} }{n^{\frac{1}{2}} f'_b(b(\chi,\tau);\chi,\tau)} H_{b,11}(\chi,\tau)^2 \ee^{-2(K_b(\chi,\tau) + S_b(\chi,\tau))}   \\ \cdot  \omega_b(\chi,\tau)^{2s} (b(\chi,\tau)-a(\chi,\tau))^{2\ii p} 
%%\left( f'_b(b(\chi,\tau);\chi,\tau) \right)^{2\ii p}
%%\end{multlined}\\
%%&\quad - \! \begin{multlined}[t]
%%\frac{s \alpha^{*} n^{-\ii p} \ee^{2\ii n \tilde{h}_b(\chi,\tau)} }{n^{\frac{1}{2}} f'_b(b(\chi,\tau);\chi,\tau)} H_{b,12}(\chi,\tau)^2 \ee^{2(K_b(\chi,\tau) + S_b(\chi,\tau))}   \\ \cdot  \omega_b(\chi,\tau)^{-2s} (b(\chi,\tau)-a(\chi,\tau))^{-2\ii p} 
%%\left( f'_b(b(\chi,\tau);\chi,\tau) \right)^{-2\ii p},
%%\end{multlined}
%\! \begin{multlined}[t]
%\frac{s}{M^{\frac{1}{2}}}\cdot \frac{\alpha M^{\ii p} \ee^{-2\ii M {h}_b(\chi,\tau)} }{ h_b''(\chi,\tau)^{\frac{1}{2}}} 
%L_{b,11}(\chi,\tau)^2 \ee^{-2 \ii (K_b(\chi,\tau) + \mu(\chi,\tau) )}  (b(\chi,\tau)-a(\chi,\tau))^{2\ii p} \\ \cdot 
%h_b''(\chi,\tau)^{\ii p}
%\end{multlined}\\
%&\quad - \! \begin{multlined}[t]
%\frac{s}{M^{-\frac{1}{2}}} \cdot \frac{\alpha^{*} M^{-\ii p} \ee^{2\ii M {h}_b(\chi,\tau)} }{ h_b''(\chi,\tau)^{\frac{1}{2}}} 
%L_{b,12}(\chi,\tau)^2 \ee^{2\ii (K_b(\chi,\tau) + \mu(\chi,\tau) )}  (b(\chi,\tau)-a(\chi,\tau))^{-2\ii p} \\ \cdot  
%h_b''(\chi,\tau)^{-\ii p},
%\end{multlined}
%\end{split}
%\label{eq:V-F-12-b-residue}
%\end{equation}
where we have set
\begin{equation}
X_a(\chi,\tau) \defeq - (b(\chi,\tau)-a(\chi,\tau))^2 h_a''(\chi,\tau),\quad\text{and}\quad X_b(\chi,\tau) \defeq  (b(\chi,\tau)-a(\chi,\tau))^2 h_b''(\chi,\tau),
\end{equation}
\begin{equation}
%K_a(\chi,\tau) \defeq  K_-(a(\chi,\tau);\chi,\tau)  \quad&\text{and}\quad K_b(\chi,\tau) \defeq  K(b(\chi,\tau);\chi,\tau),\\
\mathbf{L}_a(\chi,\tau) \defeq  \mathbf{L}_-(a(\chi,\tau);\chi,\tau,\mathbf{Q}^{-s},M), \quad \text{and}\quad \mathbf{L}_b(\chi,\tau) \defeq  \mathbf{L}(b(\chi,\tau);\chi,\tau,\mathbf{Q}^{-s},M),
\end{equation}
and used the notation in \eqref{eq:intro-Ka} and \eqref{eq:intro-Kb}. Recalling the definitions of the four positive modulation factors in \eqref{eq:m-a-b-shelves}, we obtain from \eqref{eq:H-def} the (well-defined) expressions
\begin{equation}
%\begin{split}
%H_{b,11}(\chi,\tau)^2 = \frac{1}{2} + \frac{1}{4} \big( y(b(\chi,\tau);\chi,\tau)^{-2} + y(b(\chi,\tau);\chi,\tau)^{2} \big)
%=\frac{1}{2}\left[ 1+ \cos\left(\arg(b(\chi,\tau) - \lambda_0(\chi,\tau) ) \right)\right],
L_{b,11}(\chi,\tau)^2 = \frac{1}{2} + \frac{1}{4} \big( y(b(\chi,\tau);\chi,\tau)^{-2} + y(b(\chi,\tau);\chi,\tau)^{2} \big)
=m^+_b(\chi,\tau),
%\frac{1}{2}\left[\cos\left(\arg(b(\chi,\tau) - \lambda_0(\chi,\tau) ) \right) + 1 \right],
%\end{split}
\label{eq:H-b-11}
\end{equation}
and similarly
\begin{align}
L_{b,12}(\chi,\tau)^2 
&= - m^+_b(\chi,\tau) \ee^{-4\ii (M\kappa(\chi,\tau)+ \mu(\chi,\tau) + \frac{1}{4}s \pi )},
\label{eq:H-b-12}\\
L_{a,11}(\chi,\tau)^2 &=
%\frac{1}{2}\left[  \cos\left(\arg(a(\chi,\tau) - \lambda_0(\chi,\tau) ) \right) + 1\right],
m^+_a(\chi,\tau),\quad\text{and}
\label{eq:H-a-11}\\
L_{a,12}(\chi,\tau)^2 &= - m^-_a(\chi,\tau) \ee^{-4\ii(M\kappa(\chi,\tau)+ \mu(\chi,\tau) + \frac{1}{4}s \pi )}.\label{eq:H-a-12}
%\frac{1}{2}\left[  \cos\left(\arg(a(\chi,\tau) - \lambda_0(\chi,\tau) ) \right) - 1\right]\ee^{-4\ii(M\kappa(\chi,\tau)+ \mu(\chi,\tau) + \frac{1}{4}s \pi )}.\label{eq:H-a-12}
\end{align}
Recalling that $p=\tfrac{\ln(2)}{2\pi}$ and $a(\chi,\tau)<b(\chi,\tau)$ for $(\chi,\tau) \in \shelves$ together with the signs \eqref{eq:fb-prime-h-double-prime} and \eqref{eq:fa-prime-h-double-prime}, we write
%\begin{align}
%M^{\pm \ii p} &= \ee^{\pm \ii \ln(M) \frac{\ln(2)}{2\pi}}\\
%( b(\chi,\tau) - a(\chi,\tau) )^{\pm 2 \ii p} &= \ee^{\pm \ii p \ln\left( (b(\chi,\tau) - a(\chi,\tau) )^2\right)}.
%\label{eq:log-b-a}
%\end{align}
\begin{align}
M^{\pm \ii p} &= \ee^{\pm \ii \ln(M) \frac{\ln(2)}{2\pi}},\\
X_a(\chi,\tau)^{\pm \ii p} &= \ee^{\pm \ii \frac{\ln(2)}{2\pi} \ln\left( - (b(\chi,\tau) - a(\chi,\tau) )^2h''_a(\chi,\tau) \right)},\quad\text{and}
\label{eq:log-b-a-1}\\
X_b(\chi,\tau)^{\pm \ii p} &= \ee^{\pm \ii \frac{\ln(2)}{2\pi} \ln\left( (b(\chi,\tau) - a(\chi,\tau) )^2h''_b(\chi,\tau) \right)}.
\label{eq:log-b-a-2}
\end{align}
Now substituting \eqref{eq:H-b-11}--\eqref{eq:log-b-a-2} along with \eqref{eq:Channels-alpha-beta} for $\alpha$ in \eqref{eq:V-F-12-a-residue} and \eqref{eq:V-F-12-b-residue} yields
\begin{equation}
- \frac{1}{\pi}\int_{\partial D_{a}(\delta) \cup \partial D_{b}(\delta)} V^{\mathbf{F}}_{12}(\eta)\dd \eta  = \mathfrak{S}^{[\shelves]}_s(\chi,\tau;M) + O(M^{-1}), \quad M\to+\infty,
\label{eq:subleading-term-shelves-q-proof}
\end{equation}
in which $\mathfrak{S}_s^{[\shelves]}(\chi,\tau;M)$ is the sub-leading term defined in \eqref{eq:subleading-term-shelves-q}. Substituting \eqref{eq:subleading-term-shelves-q-proof} back in \eqref{eq:psi-k-V-12-bun} 
%gives $q(M\chi,M\tau;\mathbf{Q}^{-s};M)=\mathfrak{L}_s^{[\shelves]}(\chi,\tau;M)+\mathfrak{S}_s^{[\shelves]}(\chi,\tau;M) + O(M^{-1})$, as $M\to+\infty$. This 
finishes the proof of Theorem~\ref{thm:shelves}. 
\begin{remark} In practice, one needs to evaluate the derivatives $h''_a(\chi,\tau)=h''_-(a(\chi,\tau);\chi,\tau)$ and $h''_b(\chi,\tau)=h''(b(\chi,\tau);\chi,\tau)$ to 
use the approximation given in Theorem~\ref{thm:shelves}. These can be computed in a straightforward manner from \eqref{eq:hprime-formula} and using \eqref{eq:R-a-b}:
%\begin{align}
%\tilde{h}''(b(\chi,\tau);\chi,\tau) &= \frac{2\tau(b(\chi,\tau) - a(\chi,\tau))}{b(\chi,\tau)^2+1}R(b(\chi,\tau);\chi,\tau)>0 \label{eq:h-double-prime-b}, \\
%\tilde{h}_{-}''(a(\chi,\tau);\chi,\tau) &= \frac{2\tau(a(\chi,\tau) - b(\chi,\tau))}{a(\chi,\tau)^2+1}R_{-}(a(\chi,\tau);\chi,\tau)<0 
%\label{eq:h-double-prime-a},
%\end{align}
\begin{align}
{h}''(b(\chi,\tau);\chi,\tau) &= \frac{2\tau(b(\chi,\tau) - a(\chi,\tau))}{b(\chi,\tau)^2+1}|b(\chi,\tau) - \lambda_0(\chi,\tau)|, \label{eq:h-double-prime-b} \\
{h}_{-}''(a(\chi,\tau);\chi,\tau) &= \frac{- 2\tau(b(\chi,\tau) - a(\chi,\tau))}{a(\chi,\tau)^2+1} | a(\chi,\tau) - \lambda_0(\chi,\tau)|.
\label{eq:h-double-prime-a}
\end{align}
\label{rem:h-double-prime}
\end{remark}

%\begin{equation}
%q(M\chi, M\tau; \mathbf{Q}^{-s},M)=\mathfrak{L}_s^{[\shelves]}(\chi,\tau;M) - \frac{1}{\pi}\int_{\partial D_{a}(\delta) \cup \partial D_{b}(\delta)} V^{\mathbf{F}}_{12}(\eta)\dd \eta 
% + O(M^{-1}), \quad M\to+\infty.
%\label{eq:psi-k-V-12-bun}
%\end{equation}
%\textcolor{red}{DONE UNTIL HERE.}
%%\begin{equation}
%%\begin{split}
%%-\frac{1}{\pi} \int_{D_{a}(\delta)} V^{\mathbf{F}}_{12}(\eta)\dd \eta &=
%%%\! \begin{multlined}[t]
%%%\frac{-s \alpha n^{\ii p} \ee^{2\ii n \tilde{h}_a(\chi,\tau)} }{n^{\frac{1}{2}} f'_a(a(\chi,\tau);\chi,\tau)} H_{a,12}(\chi,\tau)^2 \ee^{2(K_a(\chi,\tau) + S_a(\chi,\tau))}   \\ \cdot  \omega_a(\chi,\tau)^{-2s} (b(\chi,\tau)-a(\chi,\tau))^{2\ii p} 
%%%\left( \frac{-1}{f'_a(a(\chi,\tau);\chi,\tau)} \right)^{-2\ii p}
%%%\end{multlined}\\
%%%&\quad + \! \begin{multlined}[t]
%%%\frac{s \alpha^* n^{-\ii p} \ee^{-2\ii n \tilde{h}_a(\chi,\tau)} }{n^{\frac{1}{2}} f'_a(a(\chi,\tau);\chi,\tau)} H_{a,11}(\lambda;\chi,\tau)^2 \ee^{-2(K_a(\chi,\tau) + S_a(\chi,\tau))}  \\ \cdot  \omega_a(\chi,\tau)^{2s}  (b(\chi,\tau)-a(\chi,\tau))^{-2\ii p}  \left( \frac{-1}{f'_a(a(\chi,\tau);\chi,\tau)} \right)^{2\ii p},
%%%\end{multlined}
%%\! \begin{multlined}[t]
%%\frac{-s \alpha M^{\ii p} \ee^{2\ii M {h}_a(\chi,\tau)} }{M^{\frac{1}{2}} f'_a(a(\chi,\tau);\chi,\tau)} L_{a,12}(\chi,\tau)^2 \ee^{2\ii (K_a(\chi,\tau) + \mu(\chi,\tau) )}  (b(\chi,\tau)-a(\chi,\tau))^{2\ii p} \\ \cdot
%%\left( \frac{-1}{f'_a(a(\chi,\tau);\chi,\tau)} \right)^{-2\ii p}
%%\end{multlined}\\
%%&\quad + \! \begin{multlined}[t]
%%\frac{s \alpha^* M^{-\ii p} \ee^{-2\ii M {h}_a(\chi,\tau)} }{M^{\frac{1}{2}} f'_a(a(\chi,\tau);\chi,\tau)} L_{a,11}(\chi,\tau)^2 \ee^{-2 \ii (K_a(\chi,\tau) + \mu(\chi,\tau) )}   (b(\chi,\tau)-a(\chi,\tau))^{-2\ii p}   \\ \cdot  \left( \frac{-1}{f'_a(a(\chi,\tau);\chi,\tau)} \right)^{2\ii p},
%%\end{multlined}
%%\end{split}
%%\label{eq:V-F-12-a-residue}
%%\end{equation}
%\begin{equation}
%\begin{split}
%-\frac{1}{\pi} \int_{D_{a}(\delta)} V^{\mathbf{F}}_{12}(\eta)\dd \eta &=
%%\! \begin{multlined}[t]
%%\frac{-s \alpha n^{\ii p} \ee^{2\ii n \tilde{h}_a(\chi,\tau)} }{n^{\frac{1}{2}} f'_a(a(\chi,\tau);\chi,\tau)} H_{a,12}(\chi,\tau)^2 \ee^{2(K_a(\chi,\tau) + S_a(\chi,\tau))}   \\ \cdot  \omega_a(\chi,\tau)^{-2s} (b(\chi,\tau)-a(\chi,\tau))^{2\ii p} 
%%\left( \frac{-1}{f'_a(a(\chi,\tau);\chi,\tau)} \right)^{-2\ii p}
%%\end{multlined}\\
%%&\quad + \! \begin{multlined}[t]
%%\frac{s \alpha^* n^{-\ii p} \ee^{-2\ii n \tilde{h}_a(\chi,\tau)} }{n^{\frac{1}{2}} f'_a(a(\chi,\tau);\chi,\tau)} H_{a,11}(\lambda;\chi,\tau)^2 \ee^{-2(K_a(\chi,\tau) + S_a(\chi,\tau))}  \\ \cdot  \omega_a(\chi,\tau)^{2s}  (b(\chi,\tau)-a(\chi,\tau))^{-2\ii p}  \left( \frac{-1}{f'_a(a(\chi,\tau);\chi,\tau)} \right)^{2\ii p},
%%\end{multlined}
%\! \begin{multlined}[t]
%\frac{-s \alpha M^{\ii p} \ee^{2\ii M {h}_a(\chi,\tau)} }{M^{\frac{1}{2}} f'_a(a(\chi,\tau);\chi,\tau)} L_{a,12}(\chi,\tau)^2 \ee^{2\ii (K_a(\chi,\tau) + \mu(\chi,\tau) )}  (b(\chi,\tau)-a(\chi,\tau))^{2\ii p} \\ \cdot
%\left( \frac{-1}{f'_a(a(\chi,\tau);\chi,\tau)} \right)^{-2\ii p}
%\end{multlined}\\
%&\quad + \! \begin{multlined}[t]
%\frac{s \alpha^* M^{-\ii p} \ee^{-2\ii M {h}_a(\chi,\tau)} }{M^{\frac{1}{2}} f'_a(a(\chi,\tau);\chi,\tau)} L_{a,11}(\chi,\tau)^2 \ee^{-2 \ii (K_a(\chi,\tau) + \mu(\chi,\tau) )}   (b(\chi,\tau)-a(\chi,\tau))^{-2\ii p}   \\ \cdot  \left( \frac{-1}{f'_a(a(\chi,\tau);\chi,\tau)} \right)^{2\ii p},
%\end{multlined}
%\end{split}
%\label{eq:V-F-12-a-residue}
%\end{equation}
%%\begin{equation}
%%\begin{split}
%%-\frac{1}{\pi} \int_{D_{a}(\delta)} V^{\mathbf{F}}_{12}(\eta)\dd \eta &=
%%%\! \begin{multlined}[t]
%%%\frac{-s \alpha n^{\ii p} \ee^{2\ii n \tilde{h}_a(\chi,\tau)} }{n^{\frac{1}{2}} f'_a(a(\chi,\tau);\chi,\tau)} H_{a,12}(\chi,\tau)^2 \ee^{2(K_a(\chi,\tau) + S_a(\chi,\tau))}   \\ \cdot  \omega_a(\chi,\tau)^{-2s} (b(\chi,\tau)-a(\chi,\tau))^{2\ii p} 
%%%\left( \frac{-1}{f'_a(a(\chi,\tau);\chi,\tau)} \right)^{-2\ii p}
%%%\end{multlined}\\
%%%&\quad + \! \begin{multlined}[t]
%%%\frac{s \alpha^* n^{-\ii p} \ee^{-2\ii n \tilde{h}_a(\chi,\tau)} }{n^{\frac{1}{2}} f'_a(a(\chi,\tau);\chi,\tau)} H_{a,11}(\lambda;\chi,\tau)^2 \ee^{-2(K_a(\chi,\tau) + S_a(\chi,\tau))}  \\ \cdot  \omega_a(\chi,\tau)^{2s}  (b(\chi,\tau)-a(\chi,\tau))^{-2\ii p}  \left( \frac{-1}{f'_a(a(\chi,\tau);\chi,\tau)} \right)^{2\ii p},
%%%\end{multlined}
%%\! \begin{multlined}[t]
%%\frac{-s \alpha M^{\ii p} \ee^{2\ii M {h}_a(\chi,\tau)} }{M^{\frac{1}{2}} f'_a(a(\chi,\tau);\chi,\tau)} H_{a,12}(\chi,\tau)^2 \ee^{2\ii (K_a(\chi,\tau) + \mu(\chi,\tau) )}  (b(\chi,\tau)-a(\chi,\tau))^{2\ii p} \\ \cdot
%%\left( \frac{-1}{f'_a(a(\chi,\tau);\chi,\tau)} \right)^{-2\ii p}
%%\end{multlined}\\
%%&\quad + \! \begin{multlined}[t]
%%\frac{s \alpha^* M^{-\ii p} \ee^{-2\ii M {h}_a(\chi,\tau)} }{M^{\frac{1}{2}} f'_a(a(\chi,\tau);\chi,\tau)} H_{a,11}(\lambda)^2 \ee^{-2 \ii (K_a(\chi,\tau) + \mu(\chi,\tau) )}   (b(\chi,\tau)-a(\chi,\tau))^{-2\ii p}   \\ \cdot  \left( \frac{-1}{f'_a(a(\chi,\tau);\chi,\tau)} \right)^{2\ii p},
%%\end{multlined}
%%\end{split}
%%\label{eq:V-F-12-a-residue}
%%\end{equation}
%and
%\begin{equation}
%\begin{split}
%-\frac{1}{\pi} \int_{D_{b}(\delta)} V^{\mathbf{F}}_{12}(\eta)\dd \eta &=
%%\! \begin{multlined}[t]
%%\frac{s \alpha n^{\ii p} \ee^{-2\ii n \tilde{h}_b(\chi,\tau)} }{n^{\frac{1}{2}} f'_b(b(\chi,\tau);\chi,\tau)} H_{b,11}(\chi,\tau)^2 \ee^{-2(K_b(\chi,\tau) + S_b(\chi,\tau))}   \\ \cdot  \omega_b(\chi,\tau)^{2s} (b(\chi,\tau)-a(\chi,\tau))^{2\ii p} 
%%\left( f'_b(b(\chi,\tau);\chi,\tau) \right)^{2\ii p}
%%\end{multlined}\\
%%&\quad - \! \begin{multlined}[t]
%%\frac{s \alpha^{*} n^{-\ii p} \ee^{2\ii n \tilde{h}_b(\chi,\tau)} }{n^{\frac{1}{2}} f'_b(b(\chi,\tau);\chi,\tau)} H_{b,12}(\chi,\tau)^2 \ee^{2(K_b(\chi,\tau) + S_b(\chi,\tau))}   \\ \cdot  \omega_b(\chi,\tau)^{-2s} (b(\chi,\tau)-a(\chi,\tau))^{-2\ii p} 
%%\left( f'_b(b(\chi,\tau);\chi,\tau) \right)^{-2\ii p},
%%\end{multlined}
%\! \begin{multlined}[t]
%\frac{s \alpha M^{\ii p} \ee^{-2\ii M {h}_b(\chi,\tau)} }{M^{\frac{1}{2}} f'_b(b(\chi,\tau);\chi,\tau)} L_{b,11}(\chi,\tau)^2 \ee^{-2 \ii (K_b(\chi,\tau) + \mu(\chi,\tau) )}  (b(\chi,\tau)-a(\chi,\tau))^{2\ii p} \\ \cdot 
%\left( f'_b(b(\chi,\tau);\chi,\tau) \right)^{2\ii p}
%\end{multlined}\\
%&\quad - \! \begin{multlined}[t]
%\frac{s \alpha^{*} M^{-\ii p} \ee^{2\ii M {h}_b(\chi,\tau)} }{M^{\frac{1}{2}} f'_b(b(\chi,\tau);\chi,\tau)} L_{b,12}(\chi,\tau)^2 \ee^{2\ii (K_b(\chi,\tau) + \mu(\chi,\tau) )}  (b(\chi,\tau)-a(\chi,\tau))^{-2\ii p} \\ \cdot  
%\left( f'_b(b(\chi,\tau);\chi,\tau) \right)^{-2\ii p},
%\end{multlined}
%\end{split}
%\label{eq:V-F-12-b-residue}
%\end{equation}
%%\begin{equation}
%%\begin{split}
%%-\frac{1}{\pi} \int_{D_{b}(\delta)} V^{\mathbf{F}}_{12}(\eta)\dd \eta &=
%%%\! \begin{multlined}[t]
%%%\frac{s \alpha n^{\ii p} \ee^{-2\ii n \tilde{h}_b(\chi,\tau)} }{n^{\frac{1}{2}} f'_b(b(\chi,\tau);\chi,\tau)} H_{b,11}(\chi,\tau)^2 \ee^{-2(K_b(\chi,\tau) + S_b(\chi,\tau))}   \\ \cdot  \omega_b(\chi,\tau)^{2s} (b(\chi,\tau)-a(\chi,\tau))^{2\ii p} 
%%%\left( f'_b(b(\chi,\tau);\chi,\tau) \right)^{2\ii p}
%%%\end{multlined}\\
%%%&\quad - \! \begin{multlined}[t]
%%%\frac{s \alpha^{*} n^{-\ii p} \ee^{2\ii n \tilde{h}_b(\chi,\tau)} }{n^{\frac{1}{2}} f'_b(b(\chi,\tau);\chi,\tau)} H_{b,12}(\chi,\tau)^2 \ee^{2(K_b(\chi,\tau) + S_b(\chi,\tau))}   \\ \cdot  \omega_b(\chi,\tau)^{-2s} (b(\chi,\tau)-a(\chi,\tau))^{-2\ii p} 
%%%\left( f'_b(b(\chi,\tau);\chi,\tau) \right)^{-2\ii p},
%%%\end{multlined}
%%\! \begin{multlined}[t]
%%\frac{s \alpha M^{\ii p} \ee^{-2\ii M {h}_b(\chi,\tau)} }{M^{\frac{1}{2}} f'_b(b(\chi,\tau);\chi,\tau)} L_{b,11}(\chi,\tau)^2 \ee^{-2 \ii (K_b(\chi,\tau) + \mu(\chi,\tau) )}  (b(\chi,\tau)-a(\chi,\tau))^{2\ii p} \\ \cdot 
%%\left( f'_b(b(\chi,\tau);\chi,\tau) \right)^{2\ii p}
%%\end{multlined}\\
%%&\quad - \! \begin{multlined}[t]
%%\frac{s \alpha^{*} M^{-\ii p} \ee^{2\ii M {h}_b(\chi,\tau)} }{M^{\frac{1}{2}} f'_b(b(\chi,\tau);\chi,\tau)} L_{b,12}(\chi,\tau)^2 \ee^{2\ii (K_b(\chi,\tau) + \mu(\chi,\tau) )}  (b(\chi,\tau)-a(\chi,\tau))^{-2\ii p} \\ \cdot  
%%\left( f'_b(b(\chi,\tau);\chi,\tau) \right)^{-2\ii p},
%%\end{multlined}
%%\end{split}
%%\label{eq:V-F-12-b-residue}
%%\end{equation}
%%\begin{equation}
%%\begin{split}
%%-\frac{1}{\pi} \int_{D_{b}(\delta)} V^{\mathbf{F}}_{12}(\eta)\dd \eta &=
%%%\! \begin{multlined}[t]
%%%\frac{s \alpha n^{\ii p} \ee^{-2\ii n \tilde{h}_b(\chi,\tau)} }{n^{\frac{1}{2}} f'_b(b(\chi,\tau);\chi,\tau)} H_{b,11}(\chi,\tau)^2 \ee^{-2(K_b(\chi,\tau) + S_b(\chi,\tau))}   \\ \cdot  \omega_b(\chi,\tau)^{2s} (b(\chi,\tau)-a(\chi,\tau))^{2\ii p} 
%%%\left( f'_b(b(\chi,\tau);\chi,\tau) \right)^{2\ii p}
%%%\end{multlined}\\
%%%&\quad - \! \begin{multlined}[t]
%%%\frac{s \alpha^{*} n^{-\ii p} \ee^{2\ii n \tilde{h}_b(\chi,\tau)} }{n^{\frac{1}{2}} f'_b(b(\chi,\tau);\chi,\tau)} H_{b,12}(\chi,\tau)^2 \ee^{2(K_b(\chi,\tau) + S_b(\chi,\tau))}   \\ \cdot  \omega_b(\chi,\tau)^{-2s} (b(\chi,\tau)-a(\chi,\tau))^{-2\ii p} 
%%%\left( f'_b(b(\chi,\tau);\chi,\tau) \right)^{-2\ii p},
%%%\end{multlined}
%%\! \begin{multlined}[t]
%%\frac{s \alpha M^{\ii p} \ee^{-2\ii M {h}_b(\chi,\tau)} }{M^{\frac{1}{2}} f'_b(b(\chi,\tau);\chi,\tau)} H_{b,11}(\chi,\tau)^2 \ee^{-2 \ii K_b(\chi,\tau)}  (b(\chi,\tau)-a(\chi,\tau))^{2\ii p} \\ \cdot 
%%\left( f'_b(b(\chi,\tau);\chi,\tau) \right)^{2\ii p}
%%\end{multlined}\\
%%&\quad - \! \begin{multlined}[t]
%%\frac{s \alpha^{*} M^{-\ii p} \ee^{2\ii M {h}_b(\chi,\tau)} }{M^{\frac{1}{2}} f'_b(b(\chi,\tau);\chi,\tau)} H_{b,12}(\chi,\tau)^2 \ee^{2\ii K_b(\chi,\tau)}  (b(\chi,\tau)-a(\chi,\tau))^{-2\ii p} \\ \cdot  
%%\left( f'_b(b(\chi,\tau);\chi,\tau) \right)^{-2\ii p},
%%\end{multlined}
%%\end{split}
%%\label{eq:V-F-12-b-residue}
%%\end{equation}
%%\begin{equation}
%%\begin{split}
%%-\frac{1}{\pi} \int_{D_{b}(\delta)} V^{\mathbf{F}}_{12}(\eta)\dd \eta &=
%%%\! \begin{multlined}[t]
%%%\frac{s \alpha n^{\ii p} \ee^{-2\ii n \tilde{h}_b(\chi,\tau)} }{n^{\frac{1}{2}} f'_b(b(\chi,\tau);\chi,\tau)} H_{b,11}(\chi,\tau)^2 \ee^{-2(K_b(\chi,\tau) + S_b(\chi,\tau))}   \\ \cdot  \omega_b(\chi,\tau)^{2s} (b(\chi,\tau)-a(\chi,\tau))^{2\ii p} 
%%%\left( f'_b(b(\chi,\tau);\chi,\tau) \right)^{2\ii p}
%%%\end{multlined}\\
%%%&\quad - \! \begin{multlined}[t]
%%%\frac{s \alpha^{*} n^{-\ii p} \ee^{2\ii n \tilde{h}_b(\chi,\tau)} }{n^{\frac{1}{2}} f'_b(b(\chi,\tau);\chi,\tau)} H_{b,12}(\chi,\tau)^2 \ee^{2(K_b(\chi,\tau) + S_b(\chi,\tau))}   \\ \cdot  \omega_b(\chi,\tau)^{-2s} (b(\chi,\tau)-a(\chi,\tau))^{-2\ii p} 
%%%\left( f'_b(b(\chi,\tau);\chi,\tau) \right)^{-2\ii p},
%%%\end{multlined}
%%\! \begin{multlined}[t]
%%\frac{s \alpha M^{\ii p} \ee^{-2\ii M {h}_b(\chi,\tau)} }{M^{\frac{1}{2}} f'_b(b(\chi,\tau);\chi,\tau)} H_{b,11}(\chi,\tau)^2 \ee^{-2 \ii (K_b(\chi,\tau) + \mu(\chi,\tau) )}  (b(\chi,\tau)-a(\chi,\tau))^{2\ii p} \\ \cdot 
%%\left( f'_b(b(\chi,\tau);\chi,\tau) \right)^{2\ii p}
%%\end{multlined}\\
%%&\quad - \! \begin{multlined}[t]
%%\frac{s \alpha^{*} M^{-\ii p} \ee^{2\ii M {h}_b(\chi,\tau)} }{M^{\frac{1}{2}} f'_b(b(\chi,\tau);\chi,\tau)} H_{b,12}(\chi,\tau)^2 \ee^{2\ii (K_b(\chi,\tau) + \mu(\chi,\tau) )}  (b(\chi,\tau)-a(\chi,\tau))^{-2\ii p} \\ \cdot  
%%\left( f'_b(b(\chi,\tau);\chi,\tau) \right)^{-2\ii p},
%%\end{multlined}
%%\end{split}
%%\label{eq:V-F-12-b-residue}
%%\end{equation}
%%where we have set $\omega_a(\chi,\tau)\defeq \omega(a(\chi,\tau))$ and $\omega_b(\chi,\tau)\defeq \omega(b(\chi,\tau))$ along with
%%\begin{alignat}{2}
%%S_a(\chi,\tau) &\defeq  S_-(a(\chi,\tau);\chi,\tau), \qquad S_b(\chi,\tau) &&\defeq  S(b(\chi,\tau);\chi,\tau),\\
%%K_a(\chi,\tau) &\defeq  K_-(a(\chi,\tau);\chi,\tau), \qquad K_b(\chi,\tau) &&\defeq  K(b(\chi,\tau);\chi,\tau),\\
%%\mathbf{H}_a(\chi,\tau) &\defeq  \mathbf{H}_-(a(\chi,\tau);\chi,\tau), \qquad \mathbf{H}_b(\chi,\tau) &&\defeq  \mathbf{H}(b(\chi,\tau);\chi,\tau).
%%\end{alignat}
%where we have set
%\begin{equation}
%%K_a(\chi,\tau) \defeq  K_-(a(\chi,\tau);\chi,\tau)  \quad&\text{and}\quad K_b(\chi,\tau) \defeq  K(b(\chi,\tau);\chi,\tau),\\
%\mathbf{L}_a(\chi,\tau) \defeq  \mathbf{L}_-(a(\chi,\tau);\chi,\tau,\mathbf{Q}^{-s},M), \quad \text{and}\quad \mathbf{L}_b(\chi,\tau) \defeq  \mathbf{L}(b(\chi,\tau);\chi,\tau,\mathbf{Q}^{-s},M),
%\end{equation}
%and used the notation in \eqref{eq:intro-Ka} and \eqref{eq:intro-Kb}. Using 
%\begin{align}
%f_a'(a(\chi,\tau);\chi,\tau)& = - \left(-h_a''(\chi,\tau)\right)^{\frac{1}{2}}  \defeq  -\sqrt{-h_-(a(\chi,\tau);\chi,\tau)}<0\\
%f_b'(b(\chi,\tau);\chi,\tau)& = h_b''(\chi,\tau)^{\frac{1}{2}}  \defeq  \sqrt{h_-(b(\chi,\tau);\chi,\tau)}>0,
%\end{align}
%
%
%Recalling the definitions of the four positive modulation factors in \eqref{eq:m-a-b-shelves}, we obtain from the definition \eqref{eq:H-def} the (well-defined) expressions
%\begin{equation}
%%\begin{split}
%%H_{b,11}(\chi,\tau)^2 = \frac{1}{2} + \frac{1}{4} \big( y(b(\chi,\tau);\chi,\tau)^{-2} + y(b(\chi,\tau);\chi,\tau)^{2} \big)
%%=\frac{1}{2}\left[ 1+ \cos\left(\arg(b(\chi,\tau) - \lambda_0(\chi,\tau) ) \right)\right],
%L_{b,11}(\chi,\tau)^2 = \frac{1}{2} + \frac{1}{4} \big( y(b(\chi,\tau);\chi,\tau)^{-2} + y(b(\chi,\tau);\chi,\tau)^{2} \big)
%=m^+_b(\chi,\tau)
%%\frac{1}{2}\left[\cos\left(\arg(b(\chi,\tau) - \lambda_0(\chi,\tau) ) \right) + 1 \right],
%%\end{split}
%\label{eq:H-b-11}
%\end{equation}
%and similarly
%%\begin{equation}
%%%%\begin{split}
%%%%H_{b,12}(\chi,\tau)^2 &= \ee^{-4\ii \Theta_0(\chi,\tau;M)}\left( \frac{1}{2} - \frac{1}{4} \left[ \left(\frac{b(\chi,\tau) - \lambda_0(\chi,\tau)}{b(\chi,\tau) - \lambda_0(\chi,\tau)^*} \right)^{-\frac{1}{2}} + \left(\frac{b(\chi,\tau) - \lambda_0(\chi,\tau)}{b(\chi,\tau) - \lambda_0(\chi,\tau)^*} \right)^{\frac{1}{2}} \right]\right)\\
%%%%&=\frac{1}{2}\left[ 1- \cos\left(\arg(b(\chi,\tau) - \lambda_0(\chi,\tau) ) \right)\right]\ee^{-4\ii \Theta_0(\chi,\tau;M)},
%%%%\end{split}
%%\begin{split}
%%H_{b,12}(\chi,\tau)^2 &= \ee^{-4\ii (M\kappa(\chi,\tau)+ \mu(\chi,\tau))}\left( \frac{1}{2} - \frac{1}{4} \left[ \left(\frac{b(\chi,\tau) - \lambda_0(\chi,\tau)}{b(\chi,\tau) - \lambda_0(\chi,\tau)^*} \right)^{-\frac{1}{2}} + \left(\frac{b(\chi,\tau) - \lambda_0(\chi,\tau)}{b(\chi,\tau) - \lambda_0(\chi,\tau)^*} \right)^{\frac{1}{2}} \right]\right)\\
%%&=\frac{1}{2}\left[ 1- \cos\left(\arg(b(\chi,\tau) - \lambda_0(\chi,\tau) ) \right)\right]\ee^{-4\ii (M\kappa(\chi,\tau)+ \mu(\chi,\tau))},
%%\end{split}
%%\label{eq:H-b-12}
%%\end{equation}
%\begin{equation}
%%%\begin{split}
%%%H_{b,12}(\chi,\tau)^2 &= \ee^{-4\ii \Theta_0(\chi,\tau;M)}\left( \frac{1}{2} - \frac{1}{4} \left[ \left(\frac{b(\chi,\tau) - \lambda_0(\chi,\tau)}{b(\chi,\tau) - \lambda_0(\chi,\tau)^*} \right)^{-\frac{1}{2}} + \left(\frac{b(\chi,\tau) - \lambda_0(\chi,\tau)}{b(\chi,\tau) - \lambda_0(\chi,\tau)^*} \right)^{\frac{1}{2}} \right]\right)\\
%%%&=\frac{1}{2}\left[ 1- \cos\left(\arg(b(\chi,\tau) - \lambda_0(\chi,\tau) ) \right)\right]\ee^{-4\ii \Theta_0(\chi,\tau;M)},
%%%\end{split}
%%\begin{split}
%%H_{b,12}(\chi,\tau)^2 &= \left( - \frac{1}{2} + \frac{1}{4} \big( y(b(\chi,\tau);\chi,\tau)^{-2} + y(b(\chi,\tau);\chi,\tau)^{2} \big)\right)\ee^{-4\ii (M\kappa(\chi,\tau)+ \mu(\chi,\tau) + \frac{1}{4}s \pi )}\\
%%&=\frac{1}{2}\left[- 1 + \cos\left(\arg(b(\chi,\tau) - \lambda_0(\chi,\tau) ) \right)\right]\ee^{-4\ii (M\kappa(\chi,\tau)+ \mu(\chi,\tau) + \frac{1}{4}s \pi )}.
%%\end{split}
%L_{b,12}(\chi,\tau)^2 
%= - m^+_b(\chi,\tau) \ee^{-4\ii (M\kappa(\chi,\tau)+ \mu(\chi,\tau) + \frac{1}{4}s \pi )},
%%\frac{1}{2}\left[ \cos\left(\arg(b(\chi,\tau) - \lambda_0(\chi,\tau) ) \right) - 1 \right]\ee^{-4\ii (M\kappa(\chi,\tau)+ \mu(\chi,\tau) + \frac{1}{4}s \pi )},
%\label{eq:H-b-12}
%\end{equation}
%%\textcolor{red}{[There's a new minus sign coming from using $\mathbf{Q}$ instead of $\mathbf{O}$ above in the $(1,2)$-element.]}
%%\begin{equation}
%%%\begin{split}
%%%H_{b,12}(\chi,\tau)^2 &= \ee^{-4\ii \Theta_0(\chi,\tau;M)}\left( \frac{1}{2} - \frac{1}{4} \left[ \left(\frac{b(\chi,\tau) - \lambda_0(\chi,\tau)}{b(\chi,\tau) - \lambda_0(\chi,\tau)^*} \right)^{-\frac{1}{2}} + \left(\frac{b(\chi,\tau) - \lambda_0(\chi,\tau)}{b(\chi,\tau) - \lambda_0(\chi,\tau)^*} \right)^{\frac{1}{2}} \right]\right)\\
%%%&=\frac{1}{2}\left[ 1- \cos\left(\arg(b(\chi,\tau) - \lambda_0(\chi,\tau) ) \right)\right]\ee^{-4\ii \Theta_0(\chi,\tau;M)},
%%%\end{split}
%%%\begin{split}
%%H_{b,12}(\chi,\tau)^2 = 
%%%\ee^{-4\ii (M\kappa(\chi,\tau)+ \mu(\chi,\tau))}\left( \frac{1}{2} - \frac{1}{4} \left[ \left(\frac{b(\chi,\tau) - \lambda_0(\chi,\tau)}{b(\chi,\tau) - \lambda_0(\chi,\tau)^*} \right)^{-\frac{1}{2}} + \left(\frac{b(\chi,\tau) - \lambda_0(\chi,\tau)}{b(\chi,\tau) - \lambda_0(\chi,\tau)^*} \right)^{\frac{1}{2}} \right]\right)\\
%%\frac{1}{2}\left[ 1- \cos\left(\arg(b(\chi,\tau) - \lambda_0(\chi,\tau) ) \right)\right]\ee^{-4\ii (M\kappa(\chi,\tau)+ \mu(\chi,\tau) + \frac{1}{4}s \pi )},
%%%\end{split}
%%\label{eq:H-b-12}
%%\end{equation}
%together with
%%\begin{align}
%%H_{a,11}(\chi,\tau)^2 &=\frac{1}{2}\left[ 1+ \cos\left(\arg(a(\chi,\tau) - \lambda_0(\chi,\tau) ) \right)\right],
%%\label{eq:H-a-11}\\
%%H_{a,12}(\chi,\tau)^2 &=\frac{1}{2}\left[ 1- \cos\left(\arg(a(\chi,\tau) - \lambda_0(\chi,\tau) ) \right)\right]\ee^{-4\ii(M\kappa(\chi,\tau)+ \mu(\chi,\tau))},\label{eq:H-a-12}
%%\end{align}
%\begin{align}
%L_{a,11}(\chi,\tau)^2 &=
%%\frac{1}{2}\left[  \cos\left(\arg(a(\chi,\tau) - \lambda_0(\chi,\tau) ) \right) + 1\right],
%m^+_a(\chi,\tau)
%\label{eq:H-a-11}\\
%L_{a,12}(\chi,\tau)^2 &= - m^-_a(\chi,\tau) \ee^{-4\ii(M\kappa(\chi,\tau)+ \mu(\chi,\tau) + \frac{1}{4}s \pi )}.\label{eq:H-a-12}
%%\frac{1}{2}\left[  \cos\left(\arg(a(\chi,\tau) - \lambda_0(\chi,\tau) ) \right) - 1\right]\ee^{-4\ii(M\kappa(\chi,\tau)+ \mu(\chi,\tau) + \frac{1}{4}s \pi )}.\label{eq:H-a-12}
%\end{align}
%Next, by repeated differentiation in \eqref{eq:fb-def} and \eqref{eq:fa-def}, and using \eqref{eq:h-double-prime-b} and \eqref{eq:h-double-prime-a} along with the identities \eqref{eq:R-a-b}, we obtain
%%\begin{align}
%%f'(b(\chi,\tau);\chi,\tau)^2 &= \tilde{h}''(b(\chi,\tau);\chi,\tau) = \frac{2\tau(b(\chi,\tau) - a(\chi,\tau) ) |b(\chi,\tau)-\lambda_0(\chi,\tau)| }{b(\chi,\tau)^2+1}>0\\
%%f'(a(\chi,\tau);\chi,\tau)^2 &= - \tilde{h}_-''(a(\chi,\tau);\chi,\tau) = \frac{2\tau(b(\chi,\tau) - a(\chi,\tau) ) |a(\chi,\tau)-\lambda_0(\chi,\tau)| }{a(\chi,\tau)^2+1}>0,
%%\end{align}
%%recalling \eqref{eq:h-double-prime-b} and \eqref{eq:h-double-prime-a} along with the identities \eqref{eq:R-a-b}.
%%%\begin{equation}
%%%R(b(\chi,\tau);\chi,\tau) = |b(\chi,\tau)-\lambda_0(\chi,\tau)|\quad\text{and}\quad R_-(a(\chi,\tau);\chi,\tau) = |a(\chi,\tau)-\lambda_0(\chi,\tau)|.
%%%\label{eq:R-a-b}
%%%\end{equation}
%%As $f'(b(\chi,\tau);\chi,\tau)>0$ and $f'(a(\chi,\tau);\chi,\tau)<0$, we find that
%\begin{align}
%f_b'(b(\chi,\tau);\chi,\tau) &= \frac{\sqrt{2\tau}(b(\chi,\tau) - a(\chi,\tau) )^{\frac{1}{2}} | b(\chi,\tau)-\lambda_0(\chi,\tau)|^{\frac{1}{2}} }{\sqrt{b(\chi,\tau)^2+1}},
%\label{eq:fb-prime-at-b}\\
%f_a'(a(\chi,\tau);\chi,\tau) &=  \frac{- \sqrt{2\tau} (b(\chi,\tau) - a(\chi,\tau) )^{\frac{1}{2}} | a(\chi,\tau) - \lambda_0(\chi,\tau) |^{\frac{1}{2}} }{\sqrt{a(\chi,\tau)^2+1}}.
%\label{eq:fa-prime-at-a}
%\end{align}
%We also write $M^{\pm \ii p} = \ee^{\pm \ii \ln(M) p}$ because $p\in\mathbb{R}$, and 
%\begin{equation}
%( b(\chi,\tau) - a(\chi,\tau) )^{\pm 2 \ii p} = \ee^{\pm 2\ii p \ln( b(\chi,\tau) - a(\chi,\tau) )}
%\label{eq:log-b-a}
%\end{equation}
%because $a(\chi,\tau)<b(\chi,\tau)$ for $(\chi,\tau) \in \shelves$. 
%
%We now substitute \eqref{eq:H-b-11}--\eqref{eq:H-a-12} and \eqref{eq:fb-prime-at-b}--\eqref{eq:log-b-a} in \eqref{eq:V-F-12-a-residue} and \eqref{eq:V-F-12-b-residue}. In doing so, we write $-1 = \ee^{\ii \pi s}$ for the factors of $-1$ in the first line of \eqref{eq:V-F-12-a-residue} and in the second line of \eqref{eq:V-F-12-b-residue} to arrive at the formula 
%\begin{multline}
%q(M\chi,M\tau;\mathbf{Q}^{-s},M) = B(\chi,\tau)  \ee^{-2\ii (M\kappa(\chi,\tau)+\mu(\chi,\tau) + \frac{1}{4}s \pi)}
%  + M^{-\frac{1}{2}} s \ee^{-2\ii (M\kappa(\chi,\tau)+\mu(\chi,\tau) )} \\
% \cdot \left[ F_a^+(\chi,\tau)\ee^{\ii \Theta_a(\chi,\tau;M)}    + F_a^-(\chi,\tau)\ee^{- \ii \Theta_a(\chi,\tau;M)} \right. \\
%  \left. + F_b^+(\chi,\tau)\ee^{\ii \Theta_b(\chi,\tau;M)}   + F_b^-(\chi,\tau)\ee^{- \ii \Theta_b(\chi,\tau;M)} \right] + O(M^{-1}),
%  \label{eq:q-bun}
%\end{multline}
%which can be further simplified using $s=\pm 1$ to read
%\begin{multline}
%%\psi_k(M\chi, M\tau) = 
%q(M\chi,M\tau;\mathbf{Q}^{-s},M)=
%s  \ee^{-2\ii (M\kappa(\chi,\tau)+\mu(\chi,\tau))} 
%\left[-\ii B(\chi,\tau) + M^{-\frac{1}{2}} \left( F_a^+(\chi,\tau)\ee^{\ii \Theta_a(\chi,\tau;M)}  \right. \right.\\ + F_a^-(\chi,\tau)\ee^{- \ii \Theta_a(\chi,\tau;M)}
%\left.\left. 
%+ F_b^+(\chi,\tau)\ee^{\ii \Theta_b(\chi,\tau;M)} + F_b^-(\chi,\tau)\ee^{- \ii \Theta_b(\chi,\tau;M)} 
%\right) \right]
%+ O(M^{-1}).
%\label{eq:q-bun-alt}
%\end{multline}
%Here
%\begin{equation}
%\begin{aligned}
%%F^{\pm}_a(\chi,\tau) &\defeq  \pm \frac{\sqrt{\ln(2)}\left(1 \pm \cos( \arg( a(\chi,\tau)-\lambda_0(\chi,\tau) ) ) \right) }{2 \sqrt{\pi} f_a'(a(\chi,\tau);\chi,\tau)},\\
%F^{\pm}_a(\chi,\tau) &\defeq  \frac{\sqrt{\ln(2)}\left(\cos( \arg( a(\chi,\tau)-\lambda_0(\chi,\tau) ) ) \pm 1  \right) }{2 \sqrt{\pi} f_a'(a(\chi,\tau);\chi,\tau)},\\
%F^{\pm}_b(\chi,\tau) &\defeq   \frac{\sqrt{\ln(2)} \left(\cos( \arg( b(\chi,\tau)-\lambda_0(\chi,\tau) ) ) \pm 1  \right) }{2 \sqrt{\pi} f_b'(b(\chi,\tau);\chi,\tau)}
%\end{aligned}
%\label{eq:shelves-amplitudes-bun}
%\end{equation}
%are the real-valued amplitudes in which we have incorporated $|\alpha|=\sqrt{\ln(2)/\pi}$ from \eqref{eq:Channels-alpha-beta}, and
%\begin{align}
%\Theta_a(\chi,\tau;M) &\defeq  M \Phi_a(\chi,\tau) - \ln(M) \frac{\ln(2)}{2\pi} + \eta_a(\chi,\tau),\label{eq:Theta-a}\\
%\Theta_b(\chi,\tau;M) &\defeq  M \Phi_b(\chi,\tau)  + \ln(M) \frac{\ln(2)}{2\pi} + \eta_b(\chi,\tau),\label{eq:Theta-b}
%\end{align}
%are the phase factors, in which
%\begin{align}
%\Phi_a(\chi,\tau) &\defeq  2\left( \kappa(\chi,\tau) - {h}_a(\chi,\tau) \right),\\
%\Phi_b(\chi,\tau) &\defeq  2\left( \kappa(\chi,\tau) - {h}_b(\chi,\tau) \right),
%\end{align}
%constitute the \emph{fast} components and
%%\begin{align}
%%\eta_a(\chi,\tau) &\defeq 
%%\! \begin{multlined}[t]
%% 2\left((-1)^k \gamma(\chi,\tau) + \mu(\chi,\tau) +\ii K_a(\chi,\tau) + \ii S_a(\chi,\tau) \right) + (-1)^k[ \arg(a(\chi,\tau)-\ii)+\pi ]\\
%%+ 2p \left[ \ln\left(\frac{\sqrt{a(\chi,\tau)^2 + 1}}{\sqrt{2\tau}(b(\chi,\tau)-a(\chi,\tau))^{\frac{1}{2}} |a(\chi,\tau)-\lambda_0(\chi,\tau)|^{\frac{1}{2}}} \right) - \ln(b(\chi,\tau) - a(\chi,\tau) ) \right]\\
%%- \frac{1}{4}\pi - 2\pi p^2 + \arg(\Gamma(\ii p)),
%%\end{multlined}\\
%%\eta_b(\chi,\tau) &\defeq   
%%\! \begin{multlined}[t]
%%2\left((-1)^k \gamma(\chi,\tau) + \mu(\chi,\tau) +\ii K_b(\chi,\tau) + \ii S_b(\chi,\tau) \right)  + (-1)^k \arg(b(\chi,\tau)-\ii)\\
%%+ 2p \left[ \ln\left(\frac{\sqrt{2\tau}(b(\chi,\tau)-a(\chi,\tau))^{\frac{1}{2}} |b(\chi,\tau)-\lambda_0(\chi,\tau)|^{\frac{1}{2}}}{\sqrt{b(\chi,\tau)^2 + 1}} \right) + \ln(b(\chi,\tau) - a(\chi,\tau) ) \right]\\
%%+\frac{1}{4}\pi+2\pi p^2-\arg(\Gamma(\ii p)),
%%\end{multlined}
%%\eta_a(\chi,\tau) &\defeq 
%%\! \begin{multlined}[t]
%% 2\left((-1)^k \gamma(\chi,\tau) + \mu(\chi,\tau) - K_a(\chi,\tau) + \ii S_a(\chi,\tau) \right) + (-1)^k[ \arg(a(\chi,\tau)-\ii)+\pi ]\\
%%+ 2p \left[ \ln\left(\frac{\sqrt{a(\chi,\tau)^2 + 1}}{\sqrt{2\tau}(b(\chi,\tau)-a(\chi,\tau))^{\frac{1}{2}} |a(\chi,\tau)-\lambda_0(\chi,\tau)|^{\frac{1}{2}}} \right) - \ln(b(\chi,\tau) - a(\chi,\tau) ) \right]\\
%%- \frac{1}{4}\pi - 2\pi p^2 + \arg(\Gamma(\ii p)),
%%\end{multlined}\\
%%\eta_b(\chi,\tau) &\defeq   
%%\! \begin{multlined}[t]
%%2\left((-1)^k \gamma(\chi,\tau) + \mu(\chi,\tau) - K_b(\chi,\tau) + \ii S_b(\chi,\tau) \right)  + (-1)^k \arg(b(\chi,\tau)-\ii)\\
%%+ 2p \left[ \ln\left(\frac{\sqrt{2\tau}(b(\chi,\tau)-a(\chi,\tau))^{\frac{1}{2}} |b(\chi,\tau)-\lambda_0(\chi,\tau)|^{\frac{1}{2}}}{\sqrt{b(\chi,\tau)^2 + 1}} \right) + \ln(b(\chi,\tau) - a(\chi,\tau) ) \right]\\
%%+\frac{1}{4}\pi+2\pi p^2-\arg(\Gamma(\ii p)),
%%\end{multlined}
%%\end{align}
%%constitute the \emph{slow} components, where we have used $\arg(\alpha) =\frac{1}{4}\pi+2\pi p^2-\arg(\Gamma(\ii p)) $ from \eqref{eq:Channels-alpha-beta}. 
%%Using the expressions \eqref{eq:S-at-b}--\eqref{eq:S-tilde-at-a} and \eqref{eq:K-at-b}--\eqref{eq:K-at-a} results in further simplification of the slow phase components $\eta_a(\chi,\tau)$ and $\eta_b(\chi,\tau)$, and we obtain
%%\begin{align}
%%%%\eta_a(\chi,\tau) &\defeq 
%%%%\! \begin{multlined}[t]
%%%% 2\ii\left( \tilde{K}_a(\chi,\tau) + (-1)^k \tilde{S}_a(\chi,\tau) \right) - \frac{1}{4}\pi - 2\pi p^2 + \arg(\Gamma(\ii p)) \\
%%%%+ 2p \ln\left(\frac{\sqrt{a(\chi,\tau)^2 + 1}}{\sqrt{2\tau}(b(\chi,\tau)-a(\chi,\tau))^{\frac{3}{2}} |a(\chi,\tau)-\lambda_0(\chi,\tau)|^{\frac{1}{2}}} \right),
%%%%\end{multlined}\\
%%%\eta_b(\chi,\tau) &\defeq   
%%%\! \begin{multlined}[t]
%%% 2\ii\left( \tilde{K}_b(\chi,\tau) + (-1)^k \tilde{S}_b(\chi,\tau) \right) +\frac{1}{4}\pi+2\pi p^2-\arg(\Gamma(\ii p)) \\
%%%+ 2p  \ln\left(\frac{\sqrt{2\tau}(b(\chi,\tau)-a(\chi,\tau))^{\frac{3}{2}} |b(\chi,\tau)-\lambda_0(\chi,\tau)|^{\frac{1}{2}}}{\sqrt{b(\chi,\tau)^2 + 1}} \right).
%%%\end{multlined}
%%\eta_a(\chi,\tau) &\defeq 
%%\! \begin{multlined}[t]
%% 2\left( -\tilde{K}_a(\chi,\tau) + (-1)^k \ii \tilde{S}_a(\chi,\tau) \right) - \frac{1}{4}\pi - 2\pi p^2 + \arg(\Gamma(\ii p)) \\
%%+ 2p \ln\left(\frac{\sqrt{a(\chi,\tau)^2 + 1}}{\sqrt{2\tau}(b(\chi,\tau)-a(\chi,\tau))^{\frac{3}{2}} |a(\chi,\tau)-\lambda_0(\chi,\tau)|^{\frac{1}{2}}} \right),
%%\end{multlined}\\
%%\eta_b(\chi,\tau) &\defeq   
%%\! \begin{multlined}[t]
%% 2 \left(- \tilde{K}_b(\chi,\tau) + (-1)^k \ii \tilde{S}_b(\chi,\tau) \right) +\frac{1}{4}\pi+2\pi p^2-\arg(\Gamma(\ii p)) \\
%%+ 2p  \ln\left(\frac{\sqrt{2\tau}(b(\chi,\tau)-a(\chi,\tau))^{\frac{3}{2}} |b(\chi,\tau)-\lambda_0(\chi,\tau)|^{\frac{1}{2}}}{\sqrt{b(\chi,\tau)^2 + 1}} \right).
%%\end{multlined}
%%\end{align}
%\begin{align}
%\eta_a(\chi,\tau) &\defeq 
%\! \begin{multlined}[t]
% -2{K}_a(\chi,\tau)  - \frac{1}{4}\pi - 2\pi p^2 + \arg\left(\Gamma\left( \frac{\ii\ln(2)}{2\pi}\right)\right) \\
%+ \frac{\ln(2)}{\pi}  \ln\left(\frac{\sqrt{a(\chi,\tau)^2 + 1}}{\sqrt{2\tau}(b(\chi,\tau)-a(\chi,\tau))^{\frac{3}{2}} |a(\chi,\tau)-\lambda_0(\chi,\tau)|^{\frac{1}{2}}} \right),
%\end{multlined}\\
%\eta_b(\chi,\tau) &\defeq   
%\! \begin{multlined}[t]
%- 2{K}_b(\chi,\tau) +\frac{1}{4}\pi+2\pi p^2-\arg\left(\Gamma\left( \frac{\ii \ln(2)}{2\pi}\right)\right) \\
%+ \frac{\ln(2)}{\pi}  \ln\left(\frac{\sqrt{2\tau}(b(\chi,\tau)-a(\chi,\tau))^{\frac{3}{2}} |b(\chi,\tau)-\lambda_0(\chi,\tau)|^{\frac{1}{2}}}{\sqrt{b(\chi,\tau)^2 + 1}} \right).
%\end{multlined}
%\end{align}
%constitute the \emph{slow} components, where we have used $\arg(\alpha) =\frac{1}{4}\pi+2\pi p^2-\arg(\Gamma(\ii p))$ from \eqref{eq:Channels-alpha-beta} with $p=\tfrac{\ln(2)}{2\pi}$. \textcolor{red}{This completes the proof of Theorem~\ref{}.}


%\subsubsection{Interference pattern}
%\label{sec:interference}
% We now provide a description of the interference pattern observed in the density plots for the amplitude of fundamental rogue waves $\psi_{k}(x,t)$ in Figure 2 in \cite{BilmanLM20} away from the peak of the wave field which occurs at $(x,t)=(0,0)$. The region in which this pattern formation occurs coincides with $\shelves$. We show that the phenomenon is not intrinsic to fundamental rogue waves, it rather is produced for the more general family of solutions $q (M\chi, M \tau; \mathbf{Q}^{-s}, M)$ with an arbitrary unbounded and increasing sequence $\{M_k\}_{k=1}^\infty$ and $s=\pm 1$, which also includes the multi-pole $2M$-solitons studied in \cite{BilmanB19,BilmanBW19} as special case.  
% 
% Observe that taking the squared modulus of $q (M\chi, M \tau; \mathbf{Q}^{-s}, M)$ in \eqref{eq:q-bun} gives:
% \begin{multline}
% | q (M\chi, M \tau; \mathbf{Q}^{-s}, M) |^2 = B(\chi,\tau)^2
% - M^{-\frac{1}{2}} 2 B(\chi,\tau)\sqrt{\frac{\ln(2)}{\pi}}\\
% \cdot \left( \frac{1}{f_a'(a(\chi,\tau);\chi,\tau)} \sin(\Theta_a(\chi,\tau;M)) + \frac{1}{f_b'(b(\chi,\tau);\chi,\tau)} \sin(\Theta_b(\chi,\tau;M)) \right) + O(M^{-1}),\quad M \to+\infty.
% \label{eq:interference}
% \end{multline}
%This shows that the squared modulus of $q (M\chi, M \tau; \mathbf{Q}^{-s}, M)$ has a slowly varying ``shelf'' of size $O(1)$ and a rapidly varying perturbation of this self proportional to $M^{-\frac{1}{2}}$. This perturbation is a superposition of two sine functions with different phases $\Theta_a(\chi,\tau;M)$ and $\Theta_b(\chi,\tau;M)$ that are large functions of $(\chi,\tau)$ for $M\gg 1$ due to the presence of the terms $M\Phi_a(\chi,\tau)$ and $M\Phi_b(\chi,\tau)$, see \eqref{eq:Theta-a} and \eqref{eq:Theta-b}. These fast oscillations produce the interference pattern visible in Figure 2 in \cite{BilmanLM20} \textcolor{red}{and in Figure~\ref{}} for the special case of fundamental rogue waves.  \textcolor{red}{[Let's emphasize the universal nature of this wave pattern in the introduction, or here?]}


%\subsection{Wave theoretic interpretation of the asymptotic formula for $\psi_k(M\chi, M\tau)$ in $\shelves$.}
\subsection{Wave-theoretic interpretation of the asymptotic formula for $q(M\chi, M\tau; \mathbf{Q}^{-s}, M)$ in $\shelves$.} 
\label{sec:wave-theoretic-interpretation}
In this subsection we prove Corollary~\ref{cor:shelves-local}.
%\textcolor{red}{[Checking this. Redefining $h$ changes things.]} 
%Recall that $q(x,t) = q(x, t; \mathbf{Q}^{-s}, M)$ solves the focusing nonlinear Schr\"odinger equation in the form \eqref{eq:NLS-ZBC}. 
As we will be working in a relative perturbation regime of the leading term in the large-$M$ asymptotic expansion of $q(M\chi,M\tau; \mathbf{Q}^{-s}, M)$, we compare with the formula \eqref{eq:leading-and-subleading-shelves-rewritten} and write the leading term in the form
\begin{equation}
\mathfrak{L}_s^{[\shelves]}(\chi,\tau;M) =  -\ii s B(\chi,\tau) \ee^{-2\ii \phi(\chi,\tau;M)}.
\label{eq:q-0-bun}
\end{equation}
We then fix $(\chi_0,\tau_0)\in\shelves$, and write $\chi = \chi_0 + \Delta \chi$ and $ \tau = \tau_0 + \Delta \tau$.
%\begin{equation}
%\chi = \chi_0 + \Delta \chi\quad\text{and}\quad \tau = \tau_0 + \Delta \tau.
%\end{equation}
Noting that $\Delta \chi = M^{-1} \Delta x$ and $\Delta \tau = M^{-1} \Delta t$, and recalling the assumptions $\Delta x = O(1)$ and $\Delta t = O(1)$, it is easy to see that the phase $\phi(\chi,\tau;M)$ (see \eqref{eq:symmetrical-phases}) admits the following Taylor series expansion about $(\chi,\tau) = (\chi_0, \tau_0)$
\begin{equation}
\begin{split}
\phi(\chi,\tau;M) 
%&= M \left[\kappa(\chi_0,\tau_0) + \kappa_\chi (\chi_0,\tau_0)\Delta \chi + \kappa_\tau (\chi_0,\tau_0)\Delta \tau + O(\Delta \chi^2) + O(\Delta \chi \Delta \tau) + O(\Delta \tau^2) \right]\\
%&\quad+ \mu(\chi_0,\tau_0) + O(\Delta \chi) + O(\Delta \tau)\\
&= M \kappa(\chi_0,\tau_0) + \kappa_\chi (\chi_0,\tau_0)\Delta x + \kappa_\tau (\chi_0,\tau_0)\Delta t + O(M^{-1}\Delta x^2) + O(M^{-1}\Delta x \Delta t) + O(M^{-1}\Delta t^2) \\
&\quad+ \mu(\chi_0,\tau_0) + O(M^{-1}\Delta x) + O(M^{-1}\Delta t)\\
&=\phi(\chi_0,\tau_0; M) + \kappa_\chi (\chi_0,\tau_0)\Delta x + \kappa_\tau (\chi_0,\tau_0)\Delta t + O(M^{-1}),\quad M\to +\infty,
\end{split}
\label{eq:Omega-0-expand-1}
\end{equation}
which implies
\begin{equation}
%\ee^{-2\ii \Theta_0(\chi,\tau;M)} = \ee^{-2\ii \Theta_0 (\chi_0,\tau_0)} \ee^{\ii(\xi_0\Delta x - \Omega_0 \Delta t)} + o(n^{-2}),\quad n\to +\infty,
\ee^{-2\ii \phi(\chi,\tau;M)} = \ee^{-2\ii \phi (\chi_0,\tau_0;M)} \ee^{\ii(\xi_0\Delta x - \Omega_0 \Delta t)} + O(M^{-1}),\quad M\to +\infty,
\label{eq:Omega-0-Taylor-bun}
\end{equation}
in which real local wavenumber $\xi_0$ and real local frequency $\Omega_0$ are defined in \eqref{eq:wavenumbers-intro}--\eqref{eq:frequencies-intro}. On the other hand, Taylor expansion of $B(\chi,\tau)$ in \eqref{eq:q-0-bun} around the same point $(\chi_0,\tau_0)$ gives
\begin{equation}
%B(\chi,\tau) = B(\chi_0,\tau_0) + o(n^{-2}),\quad n\to +\infty.
B(\chi,\tau) = B(\chi_0,\tau_0) + O(M^{-1}),\quad M\to +\infty.
\label{eq:B-Taylor-bun}
\end{equation}
Combining \eqref{eq:Omega-0-Taylor-bun} and \eqref{eq:B-Taylor-bun} in \eqref{eq:q-0-bun} yields the expansion
\begin{equation}
%q_k^{(0)}(x,t) = -\ii \ee^{\ii \beta_0} B(\chi_0,\tau_0)  \ee^{\ii(\xi_0 \Delta x - \Omega_0 \Delta t)} + o(n^{-2}),\quad n\to +\infty,
%q_k^{(0)}(x,t) = -\ii \ee^{\ii \beta_0} B(\chi_0,\tau_0)  \ee^{\ii(\xi_0 \Delta x - \Omega_0 \Delta t)} + O(M^{-1}),\quad M\to +\infty,
%q^{(0)}(M \chi_0 + \Delta x, M\tau_0 +\Delta t) 
\mathfrak{L}_s^{[\shelves]}(\chi_0+M^{-1}\Delta x,\tau_0+M^{-1}\Delta t;M)
=Q(\Delta x, \Delta t) + O(M^{-1}),\quad M\to +\infty,
\label{eq:Q-0-expand}
\end{equation}
valid uniformly for $(\Delta x,\Delta t)$ bounded, where $Q(\Delta x, \Delta t)$ is given in \eqref{eq:leading-plane-wave-intro}. 
%We now set 
%\begin{multline}
%q^{(1)}(x,t) \defeq m_a^+(\chi,\tau)F_a^{[\shelves]}(\chi,\tau)\ee^{\ii\phi_a(\chi,\tau;M)} -
%m_a^-(\chi,\tau)F_a^{[\shelves]}(\chi,\tau)\ee^{-\ii\phi_a(\chi,\tau;M)} \\
%{}+
%m_b^+(\chi,\tau)F_b^{[\shelves]}(\chi,\tau)\ee^{\ii\phi_b(\chi,\tau;M)} -
%m_b^-(\chi,\tau)F_b^{[\shelves]}(\chi,\tau)\ee^{-\ii\phi_b(\chi,\tau;M)}
%\end{multline}
%to label the coefficient of the term proportional to $M^{-\frac{1}{2}}$ in \eqref{eq:leading-and-subleading-shelves-rewritten}.
We proceed in a similar manner and obtain Taylor series expansions of the terms in the sub-leading term in  \eqref{eq:leading-and-subleading-shelves-rewritten} around the same fixed point $(\chi_0,\tau) \in \shelves$. Recall the definitions \eqref{eq:symmetrical-phases} of the symmetrical phases $\phi_a$ and $\phi_b$. For bounded $(\Delta x, \Delta t)$ as before, we have
\begin{multline}
\phi_a(\chi,\tau;M) = \phi_a(\chi_0,\tau_0; M) + \left(\Phi^{[\shelves]}_{a,\chi}(\chi_0,\tau_0) + 2\kappa_\chi(\chi_0,\tau_0) \right)\Delta x \\ + \left( \Phi^{[\shelves]}_{a,\chi}(\chi_0,\tau_0) + 2\kappa_\tau(\chi_0,\tau_0) \right) \Delta t + O(M^{-1}).
\end{multline}
Substituting \eqref{eq:Phis-shelves} in this expression and recalling the definitions  \eqref{eq:wavenumbers-intro}--\eqref{eq:frequencies-intro} for the real local wavenumber $\xi_a$ and real local frequency $\Omega_a$ gives
\begin{equation}
%\ee^{\pm \ii \Theta_a(\chi,\tau;M)} = \ee^{\pm \ii \beta_a} \ee^{\pm \ii (\xi_a \Delta x - \Omega_a \Delta t)} + o(n^{-2}),\quad n\to +\infty,
\ee^{\pm \ii \phi_a(\chi,\tau;M)} = \ee^{\pm \ii \phi_a(\chi_0,\tau_0;M)} \ee^{\pm \ii (\xi_a \Delta x - \Omega_a \Delta t)} + O(M^{-1}),\quad M\to +\infty.
\label{eq:expand-phi-a}
\end{equation}
An identical calculation for the phase $\phi_b(\chi,\tau;M)$ gives
\begin{equation}
%\ee^{\pm \ii \Theta_a(\chi,\tau;M)} = \ee^{\pm \ii \beta_a} \ee^{\pm \ii (\xi_a \Delta x - \Omega_a \Delta t)} + o(n^{-2}),\quad n\to +\infty,
\ee^{\pm \ii \phi_b(\chi,\tau;M)} = \ee^{\pm \ii \phi_b(\chi_0,\tau_0;M)} \ee^{\pm \ii (\xi_b \Delta x - \Omega_b \Delta t)} + O(M^{-1}),\quad M\to +\infty.
\label{eq:expand-phi-b}
\end{equation}
On the other hand, for the amplitude factors in \eqref{eq:leading-and-subleading-shelves-rewritten} we have the expansions 
\begin{align}
m_a^{\pm}(\chi,\tau)F_a^{[\shelves]}(\chi,\tau)&=  m_a^{\pm}(\chi_0,\tau_0)F_a^{[\shelves]}(\chi_0,\tau_0) + O(M^{-1}),\label{eq:expand-m-F-a}\\
m_b^{\pm}(\chi,\tau)F_b^{[\shelves]}(\chi,\tau)&=  m_b^{\pm}(\chi_0,\tau_0)F_b^{[\shelves]}(\chi_0,\tau_0) + O(M^{-1})\label{eq:expand-m-F-b}.
\end{align}
Using \eqref{eq:expand-phi-a}--\eqref{eq:expand-m-F-b} in the sub-leading term $\mathfrak{S}_s^{[\shelves]}(\chi,\tau;M)$ written in the form \eqref{eq:leading-and-subleading-shelves-rewritten}, taking into account the overall multiplicative factor $M^{-\frac{1}{2}}$ in \eqref{eq:leading-and-subleading-shelves-rewritten} for the error terms in \eqref{eq:expand-phi-a}--\eqref{eq:expand-m-F-b}, and factoring out $Q(\Delta x, \Delta t)$ to express the sub-leading term as a relative perturbation results in the expansion \eqref{eq:Q-perturbation-shelves}, which proves the first statement in Corollary~\ref{cor:shelves-local}.

To show that $Q(\Delta x, \Delta t)$ is a plane-wave solution of \eqref{eq:NLS-Deltas}, we need the following lemma concerning the partial derivatives $g_\chi(\lambda;\chi,\tau)$ and $g_\tau(\lambda;\chi,\tau)$.
%\label{eq:NLS-Deltas}
%we have set $\ee^{\ii\beta_0}\defeq  s \ee^{-2\ii \Theta_0(\chi_0,\tau_0;M)}$ and $\beta_0$ is real since $s=\pm 1$ and $\Theta_0(\chi_0,\tau_0;M)$ is real.
%Recalling the definition \eqref{eq:symmetrical-phases} of $\phi(\chi,\tau;M)$ 
%
%
%
%and

%We begin the analysis with removing the rotating frame factor $\ee^{-\ii t}$ and consider
%\begin{equation}
%%q_{k}(x,t)\defeq  \ee^{\ii t}\psi_{k}(x,t)
%Q(x,t)=Q(x,t;\mathbf{Q}^{-s},M)\defeq  \ee^{\ii t}q(x,t;\mathbf{Q}^{-s},M)
%\label{eq:psi-to-q}
%\end{equation}
%%which satisfies the NLS equation in the form $\ii q_t + \tfrac{1}{2}q_{xx} + |q|^2 q =0$. We also set
%which satisfies the NLS equation in the form $\ii Q_t + \tfrac{1}{2}Q_{xx} + |Q|^2 Q =0$. 
%set
%\begin{equation}
%\Theta_0(\chi,\tau;M) \defeq  M\kappa(\chi,\tau) + \mu(\chi,\tau)
%\end{equation}
%to denote the overall phase factor which appears in \eqref{eq:q-bun-alt}. 
%Now fix $(\chi_0, \tau_0)\in \shelves$. We write 
%\begin{equation}
%\chi = \chi_0 + \Delta \chi\quad\text{and}\quad \tau = \tau_0 + \Delta \tau,
%\end{equation}
%with $\Delta \chi , \Delta \tau \ll 1$, whose sizes compared to $M \gg 1$ are to be determined. Recalling that $\chi = x/M$ and $\tau = t/M$, we have
%\begin{equation}
%\Delta \chi = \frac{1}{M} \Delta x \quad\text{and}\quad \Delta \tau = \frac{1}{M} \Delta t.
%\end{equation}
%%The leading order term in the asymptotic formula for $q_k(x,t)$ as $M\to+\infty$ is read from \eqref{eq:q-bun} as:
%%\begin{equation}
%%q_k^{(0)}(x,t) \defeq  -\ii s B(\chi,\tau) \ee^{-2\ii \Theta_0(\chi,\tau;M)}.
%%\label{eq:q-0-bun}
%%\end{equation}
%The leading order term in the asymptotic formula for $q(x,t;\mathbf{Q}^{-s},M)$ as $M\to+\infty$ is read from \eqref{eq:q-bun-alt} as:
%\begin{equation}
%q^{(0)}(x,t) \defeq  -\ii s B(\chi,\tau) \ee^{-2\ii \Theta_0(\chi,\tau;M)}.
%\label{eq:q-0-bun}
%\end{equation}
%Expanding the phase $\Theta_0(\chi,\tau;M)$ in Taylor series about $(\chi,\tau) = (\chi_0, \tau_0)$ yields
%\begin{equation}
%\begin{split}
%\Theta_0(\chi,\tau;M) &= M \left[\kappa(\chi_0,\tau_0) + \kappa_\chi (\chi_0,\tau_0)\Delta \chi + \kappa_\tau (\chi_0,\tau_0)\Delta \tau + O(\Delta \chi^2) + O(\Delta \chi \Delta \tau) + O(\Delta \tau^2) \right]\\
%&\quad+ \mu(\chi_0,\tau_0) + O(\Delta \chi) + O(\Delta \tau)\\
%&= M \kappa(\chi_0,\tau_0) + \kappa_\chi (\chi_0,\tau_0)\Delta x + \kappa_\tau (\chi_0,\tau_0)\Delta t + O(M^{-1}\Delta x^2) + O(M^{-1}\Delta x \Delta t) + O(M^{-1}\Delta t^2) \\
%&\quad+ \mu(\chi_0,\tau_0) + O(M^{-1}\Delta x) + O(M^{-1}\Delta t).
%\end{split}
%\label{eq:Omega-0-expand-1}
%\end{equation}
%We assume that $\Delta x =O(1)$ and $\Delta t = O(1)$, which guarantees that the error terms $M^{-1}\Delta x^2$, $M^{-1}\Delta x \Delta t$, $M^{-1}\Delta t^2$, $M^{-1}\Delta x$, and $M^{-1}\Delta t$ above are \emph{all} of size $O(M^{-1})$ as $M\to+\infty$.
%Consequently, we may write
%%To control the errors we require that $M^{-1}\Delta x^2$, $M^{-1}\Delta x \Delta t$, $M^{-1}\Delta t^2$, $M^{-1}\Delta x$, and $M^{-1}\Delta t$ \emph{all} to be of size $O(M^{-1})$ as $M\to+\infty$. This is guaranteed if $\Delta x =O(1)$ and $\Delta t = O(1)$, and consequently, we may write
%\begin{equation}
%\Theta_0(\chi,\tau;M) = \Theta_0(\chi_0,\tau_0;M) + \kappa_\chi (\chi_0,\tau_0)\Delta x + \kappa_\tau (\chi_0,\tau_0)\Delta t + O(M^{-1}),\quad M\to +\infty,
%\label{eq:Omega-0-expand-2}
%\end{equation}
%which implies
%\begin{equation}
%%\ee^{-2\ii \Theta_0(\chi,\tau;M)} = \ee^{-2\ii \Theta_0 (\chi_0,\tau_0)} \ee^{\ii(\xi_0\Delta x - \Omega_0 \Delta t)} + o(n^{-2}),\quad n\to +\infty,
%\ee^{-2\ii \Theta_0(\chi,\tau;M)} = \ee^{-2\ii \Theta_0 (\chi_0,\tau_0;M)} \ee^{\ii(\xi_0\Delta x - \Omega_0 \Delta t)} + O(M^{-1}),\quad M\to +\infty,
%\label{eq:Omega-0-Taylor-bun}
%\end{equation}
%where 
%\begin{align}
%%\beta_0 &\defeq  -2\Theta_0(\chi_0,\tau_0;M),\\
%\xi_0 &\defeq   -2  \kappa_\chi (\chi_0,\tau_0)\label{eq:xi-0-bun}\\
%\Omega_0 &\defeq   2  \kappa_\tau (\chi_0,\tau_0)\label{eq:Omega-0-bun}
%\end{align}
%are all real-valued and constant in $(\Delta x, \Delta t)$. On the other hand, Taylor expansion of $B(\chi,\tau)$ in \eqref{eq:q-0-bun} around the same point $(\chi_0,\tau_0)$ gives
%\begin{equation}
%%B(\chi,\tau) = B(\chi_0,\tau_0) + o(n^{-2}),\quad n\to +\infty.
%B(\chi,\tau) = B(\chi_0,\tau_0) + O(M^{-1}),\quad M\to +\infty.
%\label{eq:B-Taylor-bun}
%\end{equation}
%Combining \eqref{eq:Omega-0-Taylor-bun} and \eqref{eq:B-Taylor-bun} in \eqref{eq:q-0-bun} yields the expansion
%\begin{equation}
%%q_k^{(0)}(x,t) = -\ii \ee^{\ii \beta_0} B(\chi_0,\tau_0)  \ee^{\ii(\xi_0 \Delta x - \Omega_0 \Delta t)} + o(n^{-2}),\quad n\to +\infty,
%%q_k^{(0)}(x,t) = -\ii \ee^{\ii \beta_0} B(\chi_0,\tau_0)  \ee^{\ii(\xi_0 \Delta x - \Omega_0 \Delta t)} + O(M^{-1}),\quad M\to +\infty,
%q^{(0)}(x,t) = -\ii \ee^{\ii \beta_0} B(\chi_0,\tau_0)  \ee^{\ii(\xi_0 \Delta x - \Omega_0 \Delta t)} + O(M^{-1}),\quad M\to +\infty,
%\label{eq:Q-0-expand}
%\end{equation}
%where we have set $\ee^{\ii\beta_0}\defeq  s \ee^{-2\ii \Theta_0(\chi_0,\tau_0;M)}$ and $\beta_0$ is real since $s=\pm 1$ and $\Theta_0(\chi_0,\tau_0;M)$ is real.
%Going forward, we denote the plane wave in the leading-order term above by
%\begin{equation}
%%Q(\Delta x, \Delta t)\defeq  -\ii \ee^{\ii \beta_0} B(\chi_0,\tau_0)  \ee^{\ii(\xi_0 \Delta x - \Omega_0 \Delta t)},
%Q(\Delta x, \Delta t)\defeq  -\ii \ee^{\ii \beta_0} B(\chi_0,\tau_0)  \ee^{\ii(\xi_0 \Delta x - \Omega_0 \Delta t)},
%\label{eq:Q0-bun}
%\end{equation}
%where the amplitude factor $B(\chi_0,\tau_0)$ is real.
%\begin{proposition}
%$Q(\Delta x, \Delta t)$ defined in \eqref{eq:Q-bun} is a plane wave solution of the focusing nonlinear Schr\"odinger equation
%\begin{equation}
%\ii q_{\Delta t}(\Delta x, \Delta t) + \frac{1}{2} q_{\Delta x \Delta x} (\Delta x, \Delta t) + |q(\Delta x, \Delta t)|^2 q(\Delta x, \Delta t) = 0.
%\label{eq:nls-Delta-bun}
%\end{equation}
%\label{p:leading-order-plane-wave}
%\end{proposition}
%\begin{proposition}
%$Q(\Delta x, \Delta t)$ defined in \eqref{eq:Q0-bun} is a plane wave solution of the focusing nonlinear Schr\"odinger equation
%\begin{equation}
%\ii Q_{\Delta t}(\Delta x, \Delta t) + \frac{1}{2} Q_{\Delta x \Delta x} (\Delta x, \Delta t) + |Q(\Delta x, \Delta t)|^2 Q(\Delta x, \Delta t) = 0.
%\label{eq:nls-Delta-bun}
%\end{equation}
%\label{p:leading-order-plane-wave}
%\end{proposition}
%Before we give the proof of this proposition, we have the following lemma concerning the functions $g_{\chi}(\lambda;\chi,\tau)$ and $g_{\tau}(\lambda;\chi,\tau)$.
\begin{lemma}
The partial derivatives $g_{\chi}(\lambda;\chi,\tau)$ and $g_{\tau}(\lambda;\chi,\tau)$ are given for $(\chi,\tau)\in \shelves$ by
\begin{align}
%g_{\chi}(\lambda;\chi,\tau) &= \ii (A(\chi,\tau) - \lambda) + \ii R(\lambda;\chi,\tau),\label{eq:g-chi}\\
g_{\chi}(\lambda;\chi,\tau) &=  A(\chi,\tau) - \lambda + R(\lambda;\chi,\tau),\label{eq:g-chi}\\
%g_{\tau}(\lambda;\chi,\tau) &= \ii \left( A(\chi,\tau)^2 - \frac{1}{2}B(\chi,\tau)^2 - \lambda^2\right) + \ii(A(\chi,\tau) + \lambda) R(\lambda;\chi,\tau),\label{eq:g-tau}.
g_{\tau}(\lambda;\chi,\tau) &= A(\chi,\tau)^2 - \frac{1}{2}B(\chi,\tau)^2 - \lambda^2 + (A(\chi,\tau) + \lambda) R(\lambda;\chi,\tau)\label{eq:g-tau}.
\end{align}
Also, the partial derivatives $\kappa_\chi(\chi,\tau)$ and $\kappa_\tau(\chi,\tau)$ are given by
\begin{align}
\kappa_\chi(\chi,\tau) &= A(\chi,\tau)\label{eq:kappa-chi},\\
\kappa_\tau(\chi,\tau) &= A(\chi,\tau)^2 - \frac{1}{2} B(\chi,\tau)^2\label{eq:kappa-tau}.
\end{align}
\label{lemma:g-derivatives}
\end{lemma}
\begin{proof} 
As $\lambda_0(\chi,\tau)$ is a real analytic function of $\chi$ and $\tau$ for $(\chi,\tau)\in \shelves$, it follows from Morera's theorem that $g_\chi(\lambda;\chi,\tau)$ and $g_\tau(\lambda;\chi,\tau)$ are functions that are analytic for $\lambda\in \mathbb{C}\setminus \Sigma_g$. 
Recall the definition of $\vartheta(\lambda;\chi,\tau)$ from \eqref{eq:vartheta}, and also recall that $g(\lambda;\chi,\tau)$ behaves like the sum of a constant and the product of $(\lambda-\lambda_0)^{\frac{3}{2}}$ with an analytic function in a neighborhood of $\lambda_0$ in $\mathbb{C}\setminus \Sigma_g$ (with the same behavior near $\lambda_0^*$ by symmetry). 
Now it is seen from \eqref{eq:hpm-kappa} that $g_\chi$ can be obtained as the function analytic for $\lambda\in\mathbb{C}\setminus\Sigma_g$ satisfying the jump condition
\begin{equation}
%m_+(\lambda) + m_-(\lambda) = 2\ii \kappa_\chi(\chi,\tau) - 2\ii\theta_\chi(\lambda;\chi,\tau),\quad \lambda\in\mathbb{C}\setminus\Sigma_g,
g_{\chi+}(\lambda;\chi,\tau) + g_{\chi-}(\lambda;\chi,\tau) = 2 \kappa_\chi(\chi,\tau) - 2\vartheta_\chi(\lambda;\chi,\tau),\quad \lambda\in\Sigma_g,
\label{eq:g-chi-jump}
\end{equation}
that is bounded at the endpoints $\lambda=\lambda_0(\chi,\tau), \lambda_0(\chi,\tau)^*$ and is normalized as $g_\chi(\lambda;\chi,\tau)=O(\lambda^{-1})$ as $\lambda\to\infty$. Similarly, $g_\tau$ can be obtained as the function analytic in the same domain satisfying the jump condition
\begin{equation}
%m_+(\lambda) + m_-(\lambda) = 2\ii \kappa_\tau(\chi,\tau) - 2\ii\theta_\tau(\lambda;\chi,\tau),\quad \lambda\in\mathbb{C}\setminus\Sigma_g,
%\label{eq:g-tau-jump}
g_{\tau+}(\lambda;\chi,\tau) + g_{\tau-}(\lambda;\chi,\tau) = 2 \kappa_\tau(\chi,\tau) - 2 \vartheta_\tau(\lambda;\chi,\tau),\quad \lambda\in\Sigma_g,
\label{eq:g-tau-jump}
\end{equation}
and that is again bounded at the endpoints and normalized as $g_\tau(\lambda;\chi,\tau)=O(\lambda^{-1})$ as $\lambda\to\infty$. 
It is easy to see that the unique functions satisfying the analyticity, jump conditions, and boundedness conditions alone are
\begin{align}
g_\chi (\lambda;\chi,\tau) &=  \kappa_{\chi}(\chi,\tau)-\lambda + R(\lambda;\chi,\tau),\quad\lambda\in\mathbb{C}\setminus\Sigma_g,\label{eq:g-chi-integrated}\\
g_\tau(\lambda;\chi,\tau) &=  \kappa_{\tau}(\chi,\tau) - \lambda^{2}  + ( A(\chi,\tau) + \lambda ) R(\lambda;\chi,\tau),\quad\lambda\in\mathbb{C}\setminus\Sigma_g,\label{eq:g-tau-integrated}
\end{align}
for instance by writing $g_\chi$ and $g_\tau$ as $R(\lambda;\chi,\tau)$ times an unknown function in \eqref{eq:g-chi-jump} and \eqref{eq:g-tau-jump} and solving the resulting jump conditions for the new unknowns by a Cauchy integral that can be evaluated by residues. Enforcing the heretofore neglected normalization conditions 
by using the expansion $R(\lambda;\chi,\tau) = \lambda - A(\chi,\tau) + \tfrac{1}{2}B(\chi,\tau)^2 \lambda^{-1} + O(\lambda^{-2})$
as $\lambda\to\infty$ in \eqref{eq:g-chi-integrated} and \eqref{eq:g-tau-integrated} results in the formul\ae\ \eqref{eq:kappa-chi} and \eqref{eq:kappa-tau}.
%the jump conditions by writing the derivatives as $R(\lambda;\chi,\tau)$ times an unknown function that is solved for by a Cauchy integral that can be evaluated by residues leading to the final formulae.
%\begin{equation}
%g_\chi (\lambda;\chi,\tau) = \frac{R(\lambda;\chi,\tau)}{2\pi \ii} \int_{\Sigma_g} \frac{2\ii (\kappa_{\chi}(\chi,\tau) - \eta)}{R_+(\eta;\chi,\tau)(\eta-\lambda)}\dd \eta
%\end{equation}
%\begin{equation}
%g_\chi (\lambda;\chi,\tau) = \frac{R(\lambda;\chi,\tau)}{2\pi \ii} \int_{\Sigma_g} \frac{2 (\kappa_{\chi}(\chi,\tau) - \eta)}{R_+(\eta;\chi,\tau)(\eta-\lambda)}\dd \eta
%\end{equation}
%satisfies \eqref{eq:g-chi-jump} and defines an analytic function in the complement of $\Sigma_g$ that is bounded at the endpoints of $\Sigma_g$, where $\kappa_\chi(\chi,\tau)$ is to be determined to ensure the normalization as $\lambda\to\infty$. Then, for $\lambda\notin \Sigma_g$ a residue calculation at $\eta=\lambda$ and at $\eta=\infty$ using the  expansion $R(\eta;\chi,\tau) = \lambda^{-1} + A(\chi,\tau)\eta^{-2} + \eta^{-2}(A(\chi,\tau)^{2} - \frac{1}{2}B(\chi,\eta)^2) + O(\eta^{-3})$ as $\eta\to\infty$ yields
%\begin{equation}
%%g_\chi (\lambda;\chi,\tau)= \ii \left(\kappa_{x}(\chi,\tau)-\lambda \right)+\ii R(\lambda;\chi,\tau).
%g_\chi (\lambda;\chi,\tau)=  \left(\kappa_{\chi}(\chi,\tau)-\lambda \right)+ R(\lambda;\chi,\tau).
%\label{eq:g-chi-integrated}
%\end{equation}
%Now requiring $g_\chi (\lambda;\chi,\tau)=O(\lambda^{-1})$ as $\lambda\to\infty$, again the expansion of $R(\lambda;\chi,\tau)$ at infinity determines $\kappa_\chi(\chi,\tau)=A(\chi,\tau)$, proving \eqref{eq:kappa-chi}, and hence \eqref{eq:g-chi} by substituting $\kappa_\chi = A$ in \eqref{eq:g-chi-integrated}. 
%
%In an analogous manner, recalling that $\theta_\tau(\lambda;\chi,\tau)=\lambda^2$, we see that 
%\begin{equation}
%%g_\tau (\lambda;\chi,\tau) = \frac{R(\lambda;\chi,\tau)}{2\pi \ii} \int_{\Sigma_g} \frac{2\ii (\kappa_{\tau}(\chi,\tau) - \eta^2)}{R_+(\eta;\chi,\tau)(\eta-\lambda)}\dd \eta
%g_\tau (\lambda;\chi,\tau) = \frac{R(\lambda;\chi,\tau)}{2\pi \ii} \int_{\Sigma_g} \frac{2 (\kappa_{\tau}(\chi,\tau) - \eta^2)}{R_+(\eta;\chi,\tau)(\eta-\lambda)}\dd \eta
%\end{equation}
%satisfies \eqref{eq:g-tau-jump} and defines an analytic function in the complement of $\Sigma_g$ that is bounded at the endpoints of $\Sigma_g$. A similar residue calculation, this time using more terms in the expansion of $R(\eta;\chi,\tau)$ as $\lambda\to \infty$ due to having a quadratic in the numerator of the integrand gives
%%\begin{equation}
%%g_\tau(\lambda;\chi,\tau) = \ii \left( \kappa_{\tau}(\chi,\tau) - \lambda^{2} \right) + \ii( A(\chi,\tau) + \lambda ) R(\lambda;\chi,\tau).
%%\label{eq:g-tau-integrated}
%%\end{equation}
%\begin{equation}
%g_\tau(\lambda;\chi,\tau) =  \kappa_{\tau}(\chi,\tau) - \lambda^{2}  + ( A(\chi,\tau) + \lambda ) R(\lambda;\chi,\tau).
%\label{eq:g-tau-integrated}
%\end{equation}
%Again demanding that $g_\tau(\lambda;\chi,\tau)  = O(\lambda^{-1})$ as $\lambda\to\infty$ determines $\kappa_\tau(\chi,\tau)$ from the $R(\lambda;\chi,\tau)$ at infinity to be $\kappa_\tau(\chi,\tau) = A(\chi,\tau)^2 - \tfrac{1}{2}B(\chi,\tau)^2$, proving the claim \eqref{eq:kappa-tau}. Substituting this in \eqref{eq:g-tau-integrated} establishes \eqref{eq:g-tau} and finishes the proof.
\end{proof}
\begin{remark}
Since $\kappa(\chi,\tau)$ is a smooth function of both variables, its first order partial derivatives with respect to $\chi$ and $\tau$ commute, which in light of the expressions \eqref{eq:kappa-chi} and \eqref{eq:kappa-tau} implies the partial differential equation
\begin{equation}
\frac{\partial A}{\partial\tau}=\frac{\partial}{\partial\chi}(A^2-\tfrac{1}{2}B^2).
\label{eq:Whitham1}
\end{equation}
Similarly, taking the coefficients $c^{(\chi)}(\chi,\tau)$ and $c^{(\tau)}(\chi,\tau)$ of the (leading) term proportional to $\lambda^{-1}$ in $g_\chi(\lambda;\chi,\tau)$ and $g_\tau(\lambda;\chi,\tau)$ respectively, we obtain the consistency relation $c^{(\chi)}_\tau=c^{(\tau)}_\chi$ which is equivalent to the partial differential equation
\begin{equation}
\frac{\partial}{\partial\tau}B^2=\frac{\partial}{\partial\chi}(2AB^2).
\label{eq:Whitham2}
\end{equation}
Under the identifications $\rho=B^2$ and $U=-2A$, the two equations \eqref{eq:Whitham1}--\eqref{eq:Whitham2} are equivalent to the dispersionless nonlinear Schr\"odinger system (or genus-zero Whitham system) written in \eqref{eq:dispersionless-NLS}.  As a $2\times 2$ quasilinear system, it can be written in Riemann invariant (diagonal) form; 
in particular, the variables $\lambda_0=A+\ii B$ and $\lambda_0^*=A-\ii B$ are Riemann invariants for this system, in terms of which it becomes
\begin{equation}
\begin{split}
\lambda_{0,\tau} + \left(-\tfrac{3}{2}\lambda_0-\tfrac{1}{2}\lambda_0^*\right)\lambda_{0,\chi}^*&=0\\
\lambda_{0,\tau}^* + \left(-\tfrac{1}{2}\lambda_0-\tfrac{3}{2}\lambda_0^*\right)\lambda_{0,\chi}^*&=0.
\end{split}
\end{equation}
Since $\kappa(\chi,\tau)$ and $\gamma(\chi,\tau)$ differ by a constant according to \eqref{eq:kappa-gamma}, this proves Corollary~\ref{cor:Whitham}.
\label{rem:Whitham}
\end{remark}
Substituting \eqref{eq:kappa-chi} and \eqref{eq:kappa-tau} in \eqref{eq:wavenumbers-intro} and \eqref{eq:frequencies-intro}, we see that
\begin{align}
\xi_0 &= -2 A(\chi_0,\tau_0)\label{eq:xi-0-explicit},\\
\Omega_0 &= 2 A(\chi_0,\tau_0)^2 - B(\chi_0,\tau_0)^2\label{eq:Omega-0-explicit}.
\end{align}
Using these expressions and noting from \eqref{eq:leading-plane-wave-intro} that $|\mathcal{A}| = B(\chi_0,\tau_0)$, it is straightforward to show that  the wavenumber $\xi_0$, the frequency $\Omega_0$, and the modulus $|\mathcal{A}|$ of the complex amplitude for $Q(\Delta x, \Delta t)$ satisfy the nonlinear dispersion relation
\begin{equation}
\Omega_0 - \tfrac{1}{2}\xi_0^2 +|\mathcal{A}|^2 = 0.
\label{eq:nls-dispersion}
\end{equation}
This proves that $Q(\Delta x, \Delta t)$ is a plane-wave solution of \eqref{eq:NLS-Deltas}.


%\begin{proof}[Proof of Proposition~\ref{p:leading-order-plane-wave}]
%As the product $-\ii \ee^{\ii\beta_0}$ in \eqref{eq:Q0-bun} has unit modulus, it suffices to show that the mapping $(\Delta x, \Delta t)\mapsto B(\chi_0,\tau_0)  \exp[{\ii(\xi_0 \Delta x - \Omega_0 \Delta t)}]$ defines a plane wave solution of \eqref{eq:nls-Delta-bun}. To this end,
%we will show that the wave number $\xi_0$, the frequency $\Omega_0$, and the real amplitude $B(\chi_0,\tau_0)$ satisfy the nonlinear dispersion relation 
%\begin{equation}
%\Omega_0 - \frac{1}{2}\xi_0^2 + B(\chi_0,\tau_0)^2 = 0.
%\label{eq:nls-dispersion}
%\end{equation}
%for the NLS equation \eqref{eq:nls-Delta-bun} in $(\Delta x, \Delta t)$ coordinates. To verify this we use the results \eqref{eq:kappa-chi} and \eqref{eq:kappa-tau} of Lemma~\ref{lemma:g-derivatives} in the definitions \eqref{eq:xi-0-bun} and \eqref{eq:Omega-0-bun} to see that
%\begin{align}
%\xi_0 &= -2 A(\chi_0,\tau_0)\label{eq:xi-0-explicit}\\
%\Omega_0 &= 2 A(\chi_0,\tau_0)^2 - B(\chi_0,\tau_0)^2\label{eq:Omega-0-explicit}.
%\end{align}
%It is then straightforward to verify that \eqref{eq:nls-dispersion} holds.
%%Verifying this requires calculation of $\kappa_\chi (\chi,\tau)$ and $\kappa_\tau (\chi,\tau)$ as they appear in the definitions \eqref{eq:xi-0-bun} and \eqref{eq:Omega-0-bun}
%%
%%
%%With the aid of \eqref{eq:kappa-chi} and \eqref{eq:kappa-tau} that was computed in Lemma~\ref{lemma:g-derivatives}
%%
%%
%%Verifying this requires calculation of $\kappa_\chi (\chi,\tau)$ and $\kappa_\tau (\chi,\tau)$ as they appear in the definitions \eqref{eq:xi-0-bun} and \eqref{eq:Omega-0-bun}. Recalling the jump condition \eqref{eq:hpm-kappa} for $g(\lambda;\chi,\tau)$ and the definition \eqref{eq:vartheta} of $\vartheta(\lambda;\chi,\tau)$, differentiation in \eqref{eq:hpm-kappa} yields the relations
%%\begin{alignat}{2}
%%g_{\chi+}(\lambda;\chi,\tau) + g_{\chi-}(\lambda;\chi,\tau) &= 2 \ii \kappa_\chi(\chi,\tau) - 2\ii \lambda,&&\quad \lambda\in\Sigma_g,\label{eq:g-chi-jump}\\
%%g_{\tau+}(\lambda;\chi,\tau) + g_{\tau-}(\lambda;\chi,\tau) &= 2 \ii \kappa_\tau(\chi,\tau) - 2\ii \lambda^2,&&\quad \lambda\in\Sigma_g.\label{eq:g-tau-jump}
%%\end{alignat}
%%A calculation mimicking the one that follows \eqref{eq:hpm-kappa} shows that
%%\begin{align}
%%\kappa_\chi(\chi,\tau) &= \frac{\ii}{\pi} \int_{\Sigma_g} \frac{\lambda}{R_+(\lambda;\chi,\tau)}\dd \lambda =  \frac{\ii}{2\pi} \oint_{C} \frac{\lambda}{R(\lambda;\chi,\tau)}\dd \lambda,\\
%%\kappa_\tau(\chi,\tau) &= \frac{\ii}{\pi} \int_{\Sigma_g} \frac{\lambda^2 }{R_+(\lambda;\chi,\tau)}\dd \lambda=  \frac{\ii}{2\pi} \oint_{C} \frac{\lambda^2 }{R(\lambda;\chi,\tau)}\dd \lambda,
%%\end{align}
%%for a clockwise-oriented loop $C$ surrounding $\Sigma_g$. Using the expansion $R(\lambda;\chi,\tau) = \lambda^{-1} + A(\chi,\tau)\lambda^{-2} + \lambda^{-2}(A(\chi,\tau)^{2} - \frac{1}{2}B(\chi,\tau)^2) + O(\lambda^{-3})$ as $\lambda\to\infty$, a residue calculation at infinity gives
%%\begin{align}
%%\kappa_\chi(\chi,\tau) &= A(\chi,\tau)\label{eq:kappa-chi}\\
%%\kappa_\tau(\chi,\tau) &= A(\chi,\tau)^2 - \frac{1}{2} B(\chi,\tau)^2\label{eq:kappa-tau},
%%\end{align}
%%and hence
%%\begin{align}
%%\xi_0 &= -2 A(\chi_0,\tau_0)\label{eq:xi-0-AB}\\
%%\Omega_0 &= 2 A(\chi_0,\tau_0)^2 - B(\chi_0,\tau_0)^2\label{eq:Omega-0-AB}.
%%\end{align}
%%It is now straightforward to verify that \eqref{eq:nls-dispersion} holds.
%\end{proof}
%This proposition along with the fact that the product $-\ii \ee^{\ii\beta_0}$ in the leading term in \eqref{eq:q-k-0-expand} has unit modulus implies that 
%the mapping $(\Delta x, \Delta t) \mapsto  -\ii (-1)^k \ee^{\ii \beta_0} B(\chi_0,\tau_0)  \exp[{\ii(\xi_0 \Delta x - \Omega_0 \Delta t)}]$ also defines a plane wave solution of \eqref{eq:nls-Delta}.

%We carry out a similar analysis for the subleading term in the asymptotic expansion for $q(x,t)=q(x,t;\mathbf{Q}^{-s},M)$ obtained from \eqref{eq:q-bun-alt}, which is
%%\begin{multline}
%%q^{(1)}_k(x,t) \defeq   s \ee^{-2 \ii \Theta_0(\chi,\tau;M)} \left( F_a^+(\chi,\tau) \ee^{\ii \Theta_a(\chi,\tau;M)} +F_a^-(\chi,\tau) \ee^{-\ii \Theta_a(\chi,\tau;M)}\right.\\
%%\left. + F_b^+(\chi,\tau) \ee^{\ii \Theta_b(\chi,\tau;M)} + F_b^-(\chi,\tau) \ee^{-\ii \Theta_b(\chi,\tau;M)} \right)
%%\end{multline}
%\begin{multline}
%q^{(1)}(x,t) \defeq   s \ee^{-2 \ii \Theta_0(\chi,\tau;M)} \left( F_a^+(\chi,\tau) \ee^{\ii \Theta_a(\chi,\tau;M)} +F_a^-(\chi,\tau) \ee^{-\ii \Theta_a(\chi,\tau;M)}\right.\\
%\left. + F_b^+(\chi,\tau) \ee^{\ii \Theta_b(\chi,\tau;M)} + F_b^-(\chi,\tau) \ee^{-\ii \Theta_b(\chi,\tau;M)} \right).
%\end{multline}
%Recalling that $\Delta \chi, \Delta \tau = O(M^{-1})$ for $M\gg 1$ and $\chi = x/M$, $\tau = t/M$, a Taylor expansion completely analogous to what we have done above in \eqref{eq:Omega-0-expand-1}-\eqref{eq:Omega-0-expand-2} for $\Theta_0(\chi,\tau;M)$ around the same fixed point $(\chi_0, \tau_0)$ yields
%\begin{equation}
%%\Theta_a (\chi,\tau) = \Theta_a (\chi_0, \tau_0) + 2\left( \kappa_\chi(\chi_0, \tau_0) - 2\tilde{h}_{a\chi}(\chi_0,\tau_0) \right)\Delta x +  + 2\left( \kappa_\tau(\chi_0, \tau_0) - 2\tilde{h}_{a\tau}(\chi_0,\tau_0) \right)\Delta t + o(n^{-2})
%%\Theta_a (\chi,\tau) = \Theta_a (\chi_0, \tau_0) + \Phi_{a\chi}(\chi_0,\tau_0)\Delta x +   \Phi_{a\tau}(\chi_0,\tau_0) \Delta t + o(n^{-2}), \quad n\to +\infty.
%\Theta_a (\chi,\tau;M) = \Theta_a (\chi_0, \tau_0;M) + \Phi_{a\chi}(\chi_0,\tau_0)\Delta x +   \Phi_{a\tau}(\chi_0,\tau_0) \Delta t + O(M^{-1}), \quad M\to +\infty.
%\label{eq:Theta-a-expand}
%%2 n \left( \kappa(\chi_0, \tau_0) - 2\tilde{h}_a(\chi_0,\tau_0) )\right) - \ln(n) p + \eta_a(\chi_0, \tau_0) 
%\end{equation}
%This implies
%\begin{equation}
%%\ee^{\pm \ii \Theta_a(\chi,\tau;M)} = \ee^{\pm \ii \beta_a} \ee^{\pm \ii (\xi_a \Delta x - \Omega_a \Delta t)} + o(n^{-2}),\quad n\to +\infty,
%\ee^{\pm \ii \Theta_a(\chi,\tau;M)} = \ee^{\pm \ii \beta_a} \ee^{\pm \ii (\xi_a \Delta x - \Omega_a \Delta t)} + O(M^{-1}),\quad M\to +\infty,
%\end{equation}
%where
%\begin{align}
%\beta_a &\defeq  \Theta_a(\chi_0,\tau_0;M) = M\Phi_a(\chi_0,\tau_0) - \ln(M) p + \eta_a(\chi_0,\tau_0),\\
%\xi_a &\defeq   \Phi_{a\chi}(\chi_0,\tau_0) = 2(\kappa_\chi(\chi_0, \tau_0) - {h}_{a\chi}(\chi_0,\tau_0)),\label{eq:xi-a-bun}\\
%\Omega_a &\defeq   - \Phi_{a\tau}(\chi_0,\tau_0) = - 2(\kappa_\tau(\chi_0, \tau_0) - {h}_{a\tau}(\chi_0,\tau_0)).\label{eq:Omega-a-bun}
%\end{align}
%Here the derivatives $ {h}_{a\chi}(\chi_0,\tau_0)$ and $ {h}_{a\tau}(\chi_0,\tau_0)$ are meant to denote the quantities
%\begin{align}
%{h}_{a\chi}(\chi_0,\tau_0) &=\left. \left[ \frac{\partial}{\partial \chi}{h}_R(a(\chi, \tau); \chi,\tau)\right]\right \vert_{(\chi,\tau)=(\chi_,\tau_0)},\\
%{h}_{a\tau}(\chi_0,\tau_0) &=\left. \left[ \frac{\partial}{\partial \tau}{h}_R(a(\chi, \tau); \chi,\tau)\right]\right \vert_{(\chi,\tau)=(\chi_,\tau_0)}.
%\end{align}
%As a matter of fact, the use of this notation does not give rise to ambiguity. Indeed,
%\begin{equation}
%\begin{split}
%\frac{\partial}{\partial \chi}{h}_R(a(\chi, \tau); \chi,\tau) &= {h}'_R((a(\chi, \tau); \chi,\tau))\cdot \left( \frac{\partial}{\partial \chi} a(\chi,\tau)\right)+ \left. \frac{\partial}{\partial \chi}{h}_R(\lambda; \chi,\tau))\right \rvert_{\lambda = a(\chi,\tau)}\\
%&=\left. \frac{\partial}{\partial \chi}{h}_R(\lambda; \chi,\tau))\right \rvert_{\lambda = a(\chi,\tau)},
%\end{split}
%\label{eq:h-chi-derivative-chain-rule}
%\end{equation}
%because $\lambda=a(\chi,\tau)$ is a (simple) root of ${h}_R'(\lambda;\chi,\tau)$ for all $(\chi,\tau)\in \shelves$. Thus,
%\begin{equation}
%{h}_{a\chi}(\chi_0,\tau_0) = {h}_{\chi-}(a(\chi_0, \tau_0); \chi_0,\tau_0),
%\label{eq:h-chi-derivative-a}
%\end{equation}
%and similarly,
%\begin{equation}
%{h}_{a\tau}(\chi_0,\tau_0) = {h}_{\tau-}(a(\chi_0, \tau_0); \chi_0,\tau_0).
%\label{eq:h-tau-derivative-a}
%\end{equation}
%In the same manner as in \eqref{eq:Theta-a-expand} we obtain
%\begin{equation}
%\ee^{\pm \ii \Theta_b(\chi,\tau;M)} =  \ee^{\pm \ii \beta_b} \ee^{\pm \ii (\xi_b \Delta x - \Omega_b \Delta t)} + O(M^{-1}),\quad M\to +\infty,
%\end{equation}
%where
%\begin{align}
%\beta_b &\defeq  \Theta_b(\chi_0,\tau_0;M) = M\Phi_b(\chi_0,\tau_0) + \ln(M) p + \eta_b(\chi_0,\tau_0),,\\
%\xi_b &\defeq   \Phi_{b\chi}(\chi_0,\tau_0) = 2(\kappa_\chi(\chi_0, \tau_0) - {h}_{b\chi}(\chi_0,\tau_0)),\label{eq:xi-b-bun}\\
%\Omega_b &\defeq   - \Phi_{b\tau}(\chi_0,\tau_0) = - 2(\kappa_\tau(\chi_0, \tau_0) - {h}_{b\tau}(\chi_0,\tau_0)),\label{eq:Omega-b-bun}
%\end{align}
%in which the derivatives $ {h}_{b\chi}(\chi_0,\tau_0)$ and $ {h}_{b\tau}(\chi_0,\tau_0)$ as before are meant to denote the quantities
%\begin{align}
%{h}_{b\chi}(\chi_0,\tau_0) &=\left. \left[ \frac{\partial}{\partial \chi}{h}(b(\chi, \tau); \chi,\tau)\right]\right \vert_{(\chi,\tau)=(\chi_,\tau_0)},\\
%{h}_{b\tau}(\chi_0,\tau_0) &=\left. \left[ \frac{\partial}{\partial \tau}{h}(b(\chi, \tau); \chi,\tau)\right]\right \vert_{(\chi,\tau)=(\chi_,\tau_0)},
%\end{align}
%and a calculation analogous to \eqref{eq:h-chi-derivative-chain-rule} this time using the fact that $\lambda=b(\chi,\tau)$ is a (simple) root of ${h}'(\lambda;\chi,\tau)$ for all $(\chi,\tau)\in \shelves$ yields
%\begin{align}
%{h}_{b\chi}(\chi_0,\tau_0) &= {h}_{\chi}(b(\chi_0, \tau_0); \chi_0,\tau_0),\label{eq:h-chi-derivative-b}\\
%{h}_{b\tau}(\chi_0,\tau_0) &= {h}_{\tau}(b(\chi_0, \tau_0); \chi_0,\tau_0).
%\label{eq:h-tau-derivative-b}
%\end{align}
%On the other hand, for the amplitude terms in \eqref{eq:q-bun}, the analogous Taylor expansion gives
%\begin{align}
%F_a^{\pm}(\chi,\tau) &= F_a^{\pm}(\chi_0,\tau_0) + o(M^{-1}),\quad M\to+\infty,\\
%F_b^{\pm}(\chi,\tau) &= F_b^{\pm}(\chi_0,\tau_0) + o(M^{-1}),\quad M\to+\infty.
%\end{align}
%Thus, we have arrived at the expansion
%%\begin{multline}
%%%q_k^{(1)}(x,t) = \ee^{\ii\beta_0} \ee^{\ii(\xi_0 \Delta x - \Omega_0 \Delta t)}
%%%\left( \ee^{\ii \beta_a} F_a^+ (\chi_0,\tau_0) \ee^{\ii(\xi_a \Delta x - \Omega_a \Delta t)} +  \ee^{-\ii\beta_a} F_a^- (\chi_0,\tau_0) \ee^{-\ii(\xi_a \Delta x - \Omega_a \Delta t)} \right.\\
%%%\left. \ee^{\ii \beta_b} F_b^+ (\chi_0,\tau_0) \ee^{\ii(\xi_b \Delta x - \Omega_b \Delta t)} +  \ee^{-\ii \beta_b} F_b^- (\chi_0,\tau_0) \ee^{-\ii(\xi_b \Delta x - \Omega_b \Delta t)} 
%%%\right) + o(n^{-2}),\quad n\to +\infty.
%%q_k^{(1)}(x,t) = \ee^{\ii\beta_0} \ee^{\ii(\xi_0 \Delta x - \Omega_0 \Delta t)}
%%\left( \ee^{\ii \beta_a} F_a^+ (\chi_0,\tau_0) \ee^{\ii(\xi_a \Delta x - \Omega_a \Delta t)} +  \ee^{-\ii\beta_a} F_a^- (\chi_0,\tau_0) \ee^{-\ii(\xi_a \Delta x - \Omega_a \Delta t)} \right.\\
%%\left. \ee^{\ii \beta_b} F_b^+ (\chi_0,\tau_0) \ee^{\ii(\xi_b \Delta x - \Omega_b \Delta t)} +  \ee^{-\ii \beta_b} F_b^- (\chi_0,\tau_0) \ee^{-\ii(\xi_b \Delta x - \Omega_b \Delta t)} 
%%\right) \\
%%+ O(M^{-1}),\quad M\to +\infty.
%%\label{eq:q-k-1expand}
%%\end{multline}
%\begin{multline}
%q^{(1)}(x,t) = \ee^{\ii\beta_0} \ee^{\ii(\xi_0 \Delta x - \Omega_0 \Delta t)}
%\left( \ee^{\ii \beta_a} F_a^+ (\chi_0,\tau_0) \ee^{\ii(\xi_a \Delta x - \Omega_a \Delta t)} +  \ee^{-\ii\beta_a} F_a^- (\chi_0,\tau_0) \ee^{-\ii(\xi_a \Delta x - \Omega_a \Delta t)} \right.\\
%\left. \ee^{\ii \beta_b} F_b^+ (\chi_0,\tau_0) \ee^{\ii(\xi_b \Delta x - \Omega_b \Delta t)} +  \ee^{-\ii \beta_b} F_b^- (\chi_0,\tau_0) \ee^{-\ii(\xi_b \Delta x - \Omega_b \Delta t)} 
%\right) \\
%+ O(M^{-1}),\quad M\to +\infty.
%\label{eq:Q-1expand}
%\end{multline}
%Combining \eqref{eq:Q-0-expand} and \eqref{eq:Q-1expand} and factoring out the plane wave $Q(\Delta x, \Delta t)$ lets us obtain from \eqref{eq:q-bun-alt} the expansion
%%\begin{equation}
%%q_k(x,t) = Q(\Delta x, \Delta t) \left( 1 + M^{-\frac{1}{2}} \left (p_a(\Delta x, \Delta t)+ p_b(\Delta x, \Delta t) \right) \right) + O(M^{-1}),\quad M\to +\infty,
%%\label{eq:q-k-bun-perturbation}
%%\end{equation} 
%\begin{equation}
%q(x,t) = Q(\Delta x, \Delta t) \left( 1 + M^{-\frac{1}{2}} \left (p_a(\Delta x, \Delta t)+ p_b(\Delta x, \Delta t) \right) \right) + O(M^{-1}),\quad M\to +\infty,
%\label{eq:Q-bun-perturbation}
%\end{equation} 
%where we have set
%\begin{align}
%p_a(\Delta x, \Delta t) &\defeq    \frac{\ii F_a^+(\chi_0, \tau_0)}{B(\chi_0, \tau_0)} \ee^{\ii \beta_a} \ee^{\ii (\xi_a \Delta x - \Omega_a \Delta t)} + \frac{\ii F_a^-(\chi_0, \tau_0)}{B(\chi_0, \tau_0)}\ee^{-\ii \beta_a} \ee^{-\ii (\xi_a \Delta x - \Omega_a \Delta t)},\label{eq:p-a-bun}\\
%p_b(\Delta x, \Delta t) &\defeq  \frac{\ii F_b^+(\chi_0, \tau_0)}{B(\chi_0, \tau_0)}\ee^{\ii \beta_b} \ee^{\ii (\xi_b \Delta x - \Omega_b \Delta t)} + \frac{\ii F_b^-(\chi_0, \tau_0)}{B(\chi_0, \tau_0)}\ee^{-\ii \beta_b} \ee^{-\ii (\xi_b \Delta x - \Omega_b \Delta t)},\label{eq:p-b-bun}
%\end{align}
%and $B(\chi_0,\tau_0)\neq 0$ for any $(\chi_0,\tau_0)\in \shelves$. 
We will now prove the claim that each of the functions $p_{a}(\Delta x, \Delta t)$ and $p_b(\Delta x, \Delta t)$ in the expansion \eqref{eq:Q-perturbation-shelves} defines a solution of the linearization \eqref{eq:linearization-intro} of \eqref{eq:NLS-Deltas} about the plane-wave solution $Q(\Delta x,\Delta t)$.
Observe that the expansion \eqref{eq:Q-perturbation-shelves} is of the form \eqref{eq:q-perturb-appendix} in the treatment given in Appendix \ref{A:perturbations} and hence gives a relative perturbation expansion of $Q(\Delta x,\Delta t)$ for $M\gg 1$. We let $r_{a,b}(\Delta x, \Delta t)$ and $s_{a,b}(\Delta x, \Delta t)$ denote the real and imaginary parts of $p_{a,b}(\Delta x, \Delta t)$, respectively. For convenience and brevity in the calculations to come, we set
%\begin{equation}
%Z_a^{+}  \defeq  F_a^{[\shelves]}(\chi_0,\tau_0) m_a^+(\chi_0,\tau_0),\qquad
%Z_a^{-}  \defeq   - F_a^{[\shelves]}(\chi_0,\tau_0) m_a^-(\chi_0,\tau_0),
%\label{eq:Z-a}
%\end{equation}
%and
%\begin{equation}
%Z_b^{+}  \defeq  F_b^{[\shelves]}(\chi_0,\tau_0) m_b^+(\chi_0,\tau_0),\qquad
%Z_b^{-}  \defeq   - F_b^{[\shelves]}(\chi_0,\tau_0) m_b^-(\chi_0,\tau_0).
%\label{eq:Z-b}
%\end{equation}
\begin{equation}
Z_{a,b}^\pm \defeq \pm F_{a,b}^{[\shelves]}(\chi_0,\tau_0)m_{a,b}^\pm(\chi_0,\tau_0).
\label{eq:Z-a-b}
\end{equation}
Then $r_{a,b}(\Delta x, \Delta t)$ and $s_{a,b}(\Delta x, \Delta t)$ are expressed in terms of these quantities as
\begin{align}
r_{a,b}(\Delta x,\Delta t) &= \frac{Z_{a,b}^- - Z_{a,b}^+}{B(\chi_0,\tau_0)}\sin\left(\phi_{a,b}(\chi_0,\tau_0;M)+\xi_{a,b}\Delta x -\Omega_{a,b}\Delta t\right),\\
s_{a,b}(\Delta x,\Delta t) &= \frac{Z_{a,b}^- + Z_{a,b}^+}{B(\chi_0,\tau_0)}\cos\left(\phi_{a,b}(\chi_0,\tau_0;M)+\xi_{a,b}\Delta x -\Omega_{a,b}\Delta t\right).
\end{align}
%\begin{align}
%r_{a}(\Delta x, \Delta t)&= \left(\frac{Z_a^{-}-Z_a^{+}}{B(\chi_0,\tau_0)}\right) \sin \left(\phi_a(\chi_0,\tau_0;M) + \xi_{a} \Delta x-\Omega_{a} \Delta t\right),\\
%s_{a}(\Delta x, \Delta t)&= \left(\frac{Z_a^{-}+Z_a^{+}}{B(\chi_0,\tau_0)}\right) \cos \left(\phi_a(\chi_0,\tau_0;M)+\xi_{a} \Delta x-\Omega_{a} \Delta t\right),
%\end{align}
%and
%\begin{align}
%r_{b}(\Delta x, \Delta t)&= \left(\frac{Z_b^{-}-Z_b^{+}}{B(\chi_0,\tau_0)}\right) \sin \left(\phi_b(\chi_0,\tau_0;M)+\xi_{b} \Delta x-\Omega_{b} \Delta t\right),\\
%s_{b}(\Delta x, \Delta t)&\defeq \left(\frac{Z_b^{-} + Z_b^{+}}{B(\chi_0,\tau_0)}\right) \cos \left(\phi_b(\chi_0,\tau_0;M)+\xi_{b} \Delta x-\Omega_{b} \Delta t\right).
%\end{align}
In view of Appendix~\ref{A:perturbations}, to prove that $p_a(\Delta x, \Delta t)$ and $p_b(\Delta x,\Delta t)$ solve \eqref{eq:linearization-intro},
it suffices to show that the pairs $(r_{a}(\Delta x, \Delta t), s_{a}(\Delta x, \Delta t))$ and $(r_{b}(\Delta x, \Delta t), s_{b}(\Delta x, \Delta t))$ satisfy \eqref{eq:linearized-NLS-r-s} (written in the variables $(\Delta x,\Delta t)$ instead of $(x,t)$). Suppressing the dependencies on the fixed point $(\chi_0,\tau_0)$ for brevity, it is easy to see using $|\mathcal{A}|^2=B^2=B(\chi_0,\tau_0)^2$ that $(r_{a,b}(\Delta x, \Delta t), s_{a,b}(\Delta x, \Delta t))$ satisfies \eqref{eq:linearized-NLS-r-s} if and only if 
\begin{align}
\left( \xi_{a,b}^2 + 2(\xi_0 \xi_{a,b} - \Omega_{a,b}) \right)Z_{a,b}^+ +\left( \xi_{a,b}^2 - 2(\xi_0 \xi_{a,b} - \Omega_{a,b}) \right) Z_{a,b}^- &= 0,\label{eq:F-a-b-pm-sys-1}\\%\label{eq:A-pm-sys-1}\\
\left( \xi_{a,b}^2 + 2(\xi_0 \xi_{a,b} - \Omega_{a,b})  - 4 B^2 \right) Z_{a,b}^+ +\left( - \xi_{a,b}^2 + 2(\xi_0 \xi_{a,b} - \Omega_{a,b})  + 4 B^2 \right)Z_{a,b}^- &= 0.\label{eq:F-a-b-pm-sys-2}%\label{eq:A-pm-sys-2},
\end{align}
%\begin{align}
%\left( \xi_a^2 + 2(\xi_0 \xi_a - \Omega_a) \right)Z_a^+ +\left( \xi_a^2 - 2(\xi_0 \xi_a - \Omega_a) \right) Z_a^- &= 0,\label{eq:F-a-pm-sys-1}\\%\label{eq:A-pm-sys-1}\\
%\left( \xi_a^2 + 2(\xi_0 \xi_a - \Omega_a)  - 4 B^2 \right) Z_a^+ +\left( - \xi_a^2 + 2(\xi_0 \xi_a - \Omega_a)  + 4 B^2 \right)Z_a^- &= 0\label{eq:F-a-pm-sys-2},%\label{eq:A-pm-sys-2},
%\end{align}
%and similarly, the pair $(r_{b}(\Delta x, \Delta t), s_{b}(\Delta x, \Delta t))$ satisfies \eqref{eq:linearized-NLS-r-s} if and only if 
%\begin{align}
%\left( \xi_b^2 + 2(\xi_0 \xi_b - \Omega_b) \right) Z_b^+ +\left( \xi_b^2 - 2(\xi_0 \xi_b - \Omega_b) \right )Z_b^- &= 0,\label{eq:F-b-pm-sys-1}\\%\label{eq:B-pm-sys-1}\\
%\left( \xi_b^2 + 2(\xi_0 \xi_b - \Omega_b)  - 4 B^2 \right) Z_b^+ +\left( - \xi_b^2 + 2(\xi_0 \xi_b - \Omega_b)  + 4 B^2 \right ) Z_b^- &= 0.\label{eq:F-b-pm-sys-2}%\label{eq:B-pm-sys-2}
%\end{align}
Note that \eqref{eq:F-a-b-pm-sys-1}--\eqref{eq:F-a-b-pm-sys-2} 
%and \eqref{eq:F-b-pm-sys-1}--\eqref{eq:F-b-pm-sys-2} 
constitute two homogeneous systems of linear equations for $(Z_{a,b}^+, Z_{a,b}^-)$, one for each choice of subscript $a$, $b$.
%and $(Z_b^+, Z_b^-)$, respectively. 
These systems have nontrivial solutions if and only if they are singular, which amount to the conditions
%\begin{align}
%4 (\xi_0 \xi_a-\Omega_a )^2 &= \xi_a^2 \left(\xi_a^2 - 4 B(\chi_0,\tau_0)^2 \right),\label{eq:linearized-dispersion-a-bun}\\
%4 (\xi_0 \xi_b- \Omega_b )^2 &= \xi_b^2 \left(\xi_b^2 - 4 B(\chi_0,\tau_0)^2 \right).\label{eq:linearized-dispersion-b-bun}
%\end{align}
\begin{equation}
4 (\xi_0 \xi_{a,b}-\Omega_{a,b} )^2 = \xi_{a,b}^2 \left(\xi_{a,b}^2 - 4 B^2 \right).\label{eq:linearized-dispersion-a-b-bun}
\end{equation}
Again recalling that $B^2=|\mathcal{A}|^2$, these are precisely two instances of the linearized dispersion relation \eqref{eq:linearized-dispersion} to be satisfied by the pairs $(\xi_{a,b}, \Omega_{a,b})$ of relative local wavenumbers and frequencies. We will first show that the conditions \eqref{eq:linearized-dispersion-a-b-bun} hold, and then show that the pairs $(Z_{a,b}^+, Z_{a,b}^-)$
%and $(Z_{a,b}^+, Z_{a,b}^-)$ 
lie in the (nontrivial) nullspaces of the coefficient matrices for the systems \eqref{eq:F-a-b-pm-sys-1}--\eqref{eq:F-a-b-pm-sys-2}.
% and \eqref{eq:F-b-pm-sys-1}--\eqref{eq:F-b-pm-sys-2}, respectively. 
 To prove \eqref{eq:linearized-dispersion-a-b-bun},
% --\eqref{eq:linearized-dispersion-a-b-bun}, 
we refer back to Lemma~\ref{lemma:g-derivatives} and use the expression \eqref{eq:g-chi} for $g_\chi(\lambda;\chi,\tau)$ together with $\vartheta_\chi(\lambda;\chi,\tau)=\lambda$ and \eqref{eq:kappa-chi} in the definitions \eqref{eq:wavenumbers-intro} of $\xi_{a,b}$ to see that
\begin{equation}
\xi_a = -2 R_-(a(\chi_0;\tau_0);\chi_0,\tau_0)\quad \text{and}\quad
\xi_b = -2 R(b(\chi_0;\tau_0);\chi_0,\tau_0).
\label{eq:xi-a-b-explicit}
\end{equation}
Similarly, using the expression \eqref{eq:g-tau} for $g_\tau(\chi,\tau)$ together with $\vartheta_\tau(\lambda;\chi,\tau)=\lambda^2$ and \eqref{eq:kappa-chi} in the definitions \eqref{eq:frequencies-intro} of $\Omega_{a,b}$ yields
\begin{equation}
\begin{split}
\Omega_a &= 2(A(\chi_0,\tau_0) + a(\chi_0,\tau_0)) R_-(a(\chi_0;\tau_0);\chi_0,\tau_0),\\
\Omega_b &= 2(A(\chi_0,\tau_0) + b(\chi_0,\tau_0))  R(b(\chi_0;\tau_0);\chi_0,\tau_0).
\label{eq:Omega-a-b-explicit}
\end{split}
\end{equation}
Now, to show that \eqref{eq:linearized-dispersion-a-b-bun} holds, we recall the definition of $\xi_0$ in \eqref{eq:wavenumbers-intro}, and use \eqref{eq:xi-a-b-explicit} and \eqref{eq:Omega-a-b-explicit} to observe that
\begin{equation}
\begin{split}
4 (\xi_0 \xi_a-\Omega_a )^2 
&= 4\left(4 A(\chi_0,\tau_0) R(a(\chi_0,\tau_0); \chi_0, \tau_0) - 2(A(\chi_0, \tau_0)+a(\chi_0, \tau_0)) R_-(a(\chi_0, \tau_0); \chi_0, \tau_0)\right)^{2}\\
&=16 R_-(a(\chi_0, \tau_0); \chi_0, \tau_0)^2 \left(a(\chi_0, \tau_0) - A(\chi_0, \tau_0) \right)^2.%\\
%&=16\left( \left(a(\chi_0, \tau_0) - A(\chi_0, \tau_0) \right)^2 + B(\chi_0,\tau_0)^2 \right) \left(a(\chi_0, \tau_0) - A(\chi_0, \tau_0) \right)^2,
\end{split}
\label{eq:linearized-dispersion-a-LHS}
\end{equation}
Next, the right-hand side of \eqref{eq:linearized-dispersion-a-b-bun} reads
\begin{equation}
\begin{split}
\xi_a^2 \left(\xi_a^2 - 4 B(\chi_0,\tau_0)^2 \right)
&= 4 R_-(a(\chi_0,\tau_0);\chi_0,\tau_0)^2 \left(4 R_-(a(\chi_0,\tau_0);\chi_0,\tau_0)^2 - 4 B(\chi_0,\tau_0)^2 \right)\\
&= 16 R_-(a(\chi_0,\tau_0);\chi_0,\tau_0)^2 \left( a(\chi_0,\tau_0) - A(\chi_0,\tau_0) \right)^2,
\end{split}
\label{eq:linearized-dispersion-a-RHS}
\end{equation}
since $R(\lambda;\chi,\tau)^2 = (\lambda- A(\chi,\tau))^2 + B(\chi,\tau)^2$. The identities \eqref{eq:linearized-dispersion-a-LHS}--\eqref{eq:linearized-dispersion-a-RHS} prove that the linearized dispersion relation \eqref{eq:linearized-dispersion-a-b-bun} holds for $(\xi_a, \Omega_a)$. A completely analogous calculation having the point $b(\chi_0,\tau_0)$ in place of $a(\chi_0,\tau_0)$ shows that \eqref{eq:linearized-dispersion-a-b-bun} holds for $(\xi_b, \Omega_b)$.

As we have now established that the linear systems \eqref{eq:F-a-b-pm-sys-1}--\eqref{eq:F-a-b-pm-sys-2}
% and \eqref{eq:F-b-pm-sys-1}--\eqref{eq:F-b-pm-sys-2} 
 are both singular, it remains to show that the pairs of quantities $(Z_{a,b}^{+}, Z_{a,b}^{-})$
%and $(Z_b^{+}, Z_b^{-})$ 
lie in the corresponding nullspaces. 
To do so, it suffices to verify that \eqref{eq:F-a-b-pm-sys-1} 
%and \eqref{eq:F-b-pm-sys-1} 
holds. 
Using the definitions \eqref{eq:m-a-b-shelves} in \eqref{eq:Z-a-b}
%--\eqref{eq:Z-b} 
and noting that $F^{[\shelves]}_a(\chi_0,\tau_0)$ and $F^{[\shelves]}_b(\chi_0,\tau_0)$ are nonzero, it is seen that verifying \eqref{eq:F-a-b-pm-sys-1}
% and \eqref{eq:F-b-pm-sys-1} 
amounts to showing that
\begin{align}
\cos( \arg(a(\chi_0,\tau_0)-\lambda_0(\chi_0,\tau_0)) ) &= \frac{-2(\xi_0\xi_a - \Omega_a)}{\xi_a^2},\label{eq:show-cos-arg-a-bun}\\
\cos( \arg(b(\chi_0,\tau_0)-\lambda_0(\chi_0,\tau_0)) ) &= \frac{-2(\xi_0\xi_b - \Omega_b)}{\xi_b^2}.\label{eq:show-cos-arg-b-bun}
\end{align}
However, according to 
\eqref{eq:xi-0-explicit}, \eqref{eq:xi-a-b-explicit}, and \eqref{eq:Omega-a-b-explicit}, we obtain for the right-hand side of the purported identities \eqref{eq:show-cos-arg-a-bun}--\eqref{eq:show-cos-arg-b-bun} that
\begin{align}
\frac{-2(\xi_0\xi_a - \Omega_a)}{\xi_a^2}&=\frac{a(\chi_0,\tau_0) - A(\chi_0,\tau_0)}{R_-(a(\chi_0,\tau_0);\chi_0,\tau_0)},\label{eq:show-cos-arg-a-simp}\\
\frac{-2(\xi_0\xi_b - \Omega_b)}{\xi_b^2}&=\frac{b(\chi_0,\tau_0) - A(\chi_0,\tau_0)}{R(b(\chi_0,\tau_0);\chi_0,\tau_0)}.\label{eq:show-cos-arg-b-simp}
\end{align}
%The right-hand sides above can be simplified by directly using the expressions \eqref{eq:xi-0-explicit}, \eqref{eq:xi-a-explicit}-\eqref{eq:xi-b-explicit}, and \eqref{eq:Omega-a-explicit}-\eqref{eq:Omega-b-explicit}, and it is seen that showing \eqref{eq:show-cos-arg-a-bun} and \eqref{eq:show-cos-arg-b-bun} is equivalent to verifying the identities
%\begin{align}
%\cos( \arg(a(\chi_0,\tau_0)-\lambda_0(\chi_0,\tau_0)) ) &= \frac{a(\chi_0,\tau_0) - A(\chi_0,\tau_0)}{R_-(a(\chi_0,\tau_0);\chi,\tau)},\label{eq:show-cos-arg-a-simp}\\
%\cos( \arg(b(\chi_0,\tau_0)-\lambda_0(\chi_0,\tau_0)) ) &= \frac{b(\chi_0,\tau_0) - A(\chi_0,\tau_0)}{R(b(\chi_0,\tau_0);\chi,\tau)}.\label{eq:show-cos-arg-b-simp}
%\end{align}
%Recalling that $a(\chi_0,\tau_0) < \Re (\lambda_0(\chi_0,\tau_0))<b(\chi_0,\tau_0)$, it is now straightforward to see that \eqref{eq:show-cos-arg-a-simp}--\eqref{eq:show-cos-arg-b-simp} are indeed true:
%%and $\Im(\lambda_0(\chi_0,\tau_0))=B(\chi_0,\tau_0)>0$, we have
%\begin{align}
%\cos( \arg(a(\chi_0,\tau_0) - \lambda_0(\chi_0,\tau_0)) )
%&= \frac{\Re( a(\chi_0,\tau_0) -\lambda_0(\chi_0,\tau_0) )}{| \lambda_0(\chi_0,\tau_0)-a(\chi_0,\tau_0)|}=\frac{a(\chi_0,\tau_0) - A(\chi_0,\tau_0)}{R_-(a(\chi_0,\tau_0);\chi_0,\tau_0)}\\
%\cos( \arg(b(\chi_0,\tau_0) - \lambda_0(\chi_0,\tau_0)) )
%&= \frac{\Re( b(\chi_0,\tau_0) -\lambda_0(\chi_0,\tau_0) )}{| b(\chi_0,\tau_0) - \lambda_0(\chi_0,\tau_0) |}=\frac{b(\chi_0,\tau_0) - A(\chi_0,\tau_0)}{R(b(\chi_0,\tau_0);\chi_0,\tau_0)}.
%\end{align}
Since $R_-(a(\chi_0,\tau_0);\chi_0,\tau_0)$ and $R(b(\chi_0,\tau_0);\chi_0,\tau_0)$ are both positive, while $a(\chi_0,\tau_0)$ and $b(\chi_0,\tau_0)$ are real and $A(\chi_0,\tau_0)=\mathrm{Re}(\lambda_0(\chi_0,\tau_0))$, 
we see that the identities \eqref{eq:show-cos-arg-a-bun}--\eqref{eq:show-cos-arg-b-bun} indeed both hold:
\begin{align}
\frac{-2(\xi_0\xi_a - \Omega_a)}{\xi_a^2}&=\frac{\mathrm{Re}(a(\chi_0,\tau_0)-\lambda_0(\chi_0,\tau_0))}{|a(\chi_0,\tau_0)-\lambda_0(\chi_0,\tau_0)|}=\cos(\arg(a(\chi_0,\tau_0)-\lambda_0(\chi_0,\tau_0))),\\
\frac{-2(\xi_0\xi_b - \Omega_b)}{\xi_b^2}&=\frac{\mathrm{Re}(b(\chi_0,\tau_0)-\lambda_0(\chi_0,\tau_0))}{|b(\chi_0,\tau_0)-\lambda_0(\chi_0,\tau_0)|}=\cos(\arg(b(\chi_0,\tau_0)-\lambda_0(\chi_0,\tau_0))).
\end{align}
Thus, we have shown that $(Z_a^+,Z_a^-)$ and $(Z_b^+,Z_b^-)$ are non-trivial solutions of the linear homogeneous systems \eqref{eq:F-a-b-pm-sys-1}--\eqref{eq:F-a-b-pm-sys-2}.
%  and \eqref{eq:F-b-pm-sys-1}--\eqref{eq:F-b-pm-sys-2}, respectively. 
This implies that  $p_a(\Delta x, \Delta t)$ and  $p_b(\Delta x, \Delta t)$ are solutions of \eqref{eq:linearization-intro}.
% $p_a(\Delta x, \Delta t)$ is a solution of \eqref{eq:linearized-NLS-Delta-bun}.
% establishes that $p_b(\Delta x, \Delta t)$ is a solution of \eqref{eq:linearized-NLS-Delta-bun}.
%our claim is that each of the subdominant functions $p_{a}(\Delta x, \Delta t)$ and $p_b(\Delta x, \Delta t)$ in the expansion above define a solution of the linearization of the NLS equation about the plane wave $Q(x,t)$.
%\begin{proposition}
%Each of the quantities $p_a(\Delta x, \Delta t)$, and $p_b(\Delta x, \Delta t)$ defines a solution of the linearization
%\begin{equation}
%\ii p_{\Delta t}(\Delta x,\Delta t) + \ii \xi_0 p_{\Delta x}(\Delta x,\Delta t) + \frac{1}{2} p_{\Delta x \Delta x}(\Delta x,\Delta t)+ B(\chi_0,\tau_0)^2 (p(\Delta x,\Delta t) + p(\Delta x,\Delta t)^*)=0
%\label{eq:linearized-NLS-Delta-bun}
%\end{equation}
%of the nonlinear Schr\"odinger equation \eqref{eq:nls-Delta-bun} about the plane wave $Q(\Delta x,\Delta t)$ given in \eqref{eq:Q0-bun}, in which we have substituted $|Q(\chi_0,\tau_0)| = |-\ii \ee^{\ii\beta_0}B(\chi_0,\tau_0) | =B(\chi_0,\tau_0)$.
%\end{proposition}
%\begin{proof}
%We let $r_{a,b}(\Delta x, \Delta t)$ and $s_{a,b}(\Delta x, \Delta t)$ denote the real and imaginary parts of $p_{a,b}(\Delta x, \Delta t)$, respectively. Explicitly,
%\begin{align}
%r_{a}(\Delta x, \Delta t)&\defeq \left(\frac{F_a^{-}(\chi_0,\tau_0)-F_a^{+}(\chi_0,\tau_0)}{B(\chi_0,\tau_0)}\right) \sin \left(\beta_a+\xi_{a} \Delta x-\Omega_{a} \Delta t\right),\\
%s_{a}(\Delta x, \Delta t)&\defeq \left(\frac{F_a^{-}(\chi_0,\tau_0)+F_a^{+}(\chi_0,\tau_0)}{B(\chi_0,\tau_0)}\right) \cos \left(\beta_a+\xi_{a} \Delta x-\Omega_{a} \Delta t\right),
%\end{align}
%and
%\begin{align}
%r_{b}(\Delta x, \Delta t)&\defeq \left(\frac{F_b^{-}(\chi_0,\tau_0)-F_b^{+}(\chi_0,\tau_0)}{B(\chi_0,\tau_0)}\right) \sin \left(\beta_b+\xi_{b} \Delta x-\Omega_{b} \Delta t\right),\\
%s_{b}(\Delta x, \Delta t)&\defeq \left(\frac{F_b^{-}(\chi_0,\tau_0)+F_b^{+}(\chi_0,\tau_0)}{B(\chi_0,\tau_0)}\right) \cos \left(\beta_b+\xi_{b} \Delta x-\Omega_{b} \Delta t\right).
%\end{align}
%In view of Appendix \ref{A:perturbations}, to prove that $p_{a,b}$ solve \eqref{eq:linearized-NLS-Delta-bun}, it suffices to show that the pairs $(r_{a}, s_{a})$ and $(r_{b}, s_{b})$ satisfy \eqref{eq:linearized-NLS-r-s}. $(r_{a}, s_{a})$ satisfies \eqref{eq:linearized-NLS-r-s} if and only if 
%\begin{align}
%\left( \xi_a^2 + 2(\xi_0 \xi_a - \Omega_a) \right)F_a^+ +\left( \xi_a^2 - 2(\xi_0 \xi_a - \Omega_a) \right) F_a^- &= 0,\label{eq:F-a-pm-sys-1}\\%\label{eq:A-pm-sys-1}\\
%\left( \xi_a^2 + 2(\xi_0 \xi_a - \Omega_a)  - 4 B^2 \right) F_a^+ +\left( - \xi_a^2 + 2(\xi_0 \xi_a - \Omega_a)  + 4 B|^2 \right)F_a^- &= 0\label{eq:F-a-pm-sys-2},%\label{eq:A-pm-sys-2},
%\end{align}
%where we dropped the dependencies on the constants $(\chi_0,\tau_0)$, and similarly, $(r_{b}, s_{b})$ satisfies \eqref{eq:linearized-NLS-r-s} if and only if 
%\begin{align}
%\left( \xi_b^2 + 2(\xi_0 \xi_b - \Omega_b) \right) F_b^+ +\left( \xi_b^2 - 2(\xi_0 \xi_b - \Omega_b) \right )F_b^- &= 0,\label{eq:F-b-pm-sys-1}\\%\label{eq:B-pm-sys-1}\\
%\left( \xi_b^2 + 2(\xi_0 \xi_b - \Omega_b)  - 4 B^2 \right) F_b^+ +\left( - \xi_b^2 + 2(\xi_0 \xi_b - \Omega_b)  + 4 B^2 \right ) F_b^- &= 0.\label{eq:F-b-pm-sys-2}%\label{eq:B-pm-sys-2}
%\end{align}
%Note that the pairs \eqref{eq:F-a-pm-sys-1}-\eqref{eq:F-a-pm-sys-2} and \eqref{eq:F-b-pm-sys-1}-\eqref{eq:F-b-pm-sys-2} constitute homogeneous systems of linear equations for $(F_a^+(\chi_0,\tau_0), F_a^-(\chi_0,\tau_0))$ and $(F_b^+(\chi_0,\tau_0), F_b^-(\chi_0,\tau_0))$, respectively. These systems have nontrivial solutions $(F_a^+(\chi_0,\tau_0), F_a^-(\chi_0,\tau_0))$ and $(F_b^+(\chi_0,\tau_0), F_b^-(\chi_0,\tau_0))$ if and only if they are singular, which amount to the conditions
%\begin{align}
%4 (\xi_0 \xi_a-\Omega_a )^2 &= \xi_a^2 \left(\xi_a^2 - 4 B(\chi_0,\tau_0)^2 \right),\label{eq:linearized-dispersion-a-bun}\\
%4 (\xi_0 \xi_b- \Omega_b )^2 &= \xi_b^2 \left(\xi_b^2 - 4 B(\chi_0,\tau_0)^2 \right).\label{eq:linearized-dispersion-b-bun}
%\end{align}
%These are precisely the linearized dispersion relation \eqref{eq:linearized-dispersion} satisfied by the pairs $(\xi_a, \Omega_a)$ and $(\xi_b, \Omega_b)$ of relative wave numbers and frequencies. We will first show that \eqref{eq:linearized-dispersion-a-bun} and \eqref{eq:linearized-dispersion-b-bun} hold, and then show that the pairs $(F_a^+(\chi_0,\tau_0), F_a^-(\chi_0,\tau_0))$ and $(F_b^+(\chi_0,\tau_0), F_b^-(\chi_0,\tau_0))$ defined in \eqref{eq:shelves-amplitudes-bun} lie in the (nontrivial) nullspaces of the coefficient matrices for the systems \eqref{eq:F-a-pm-sys-1}-\eqref{eq:F-a-pm-sys-2} and \eqref{eq:F-b-pm-sys-1}-\eqref{eq:F-b-pm-sys-2}, respectively.
%
%We refer back to Lemma~\ref{lemma:g-derivatives} and use the expression \eqref{eq:g-chi} for $g_\chi(\lambda;\chi,\tau)$ together with $\theta_\chi(\lambda;\chi,\tau)=\lambda$ to see that
%%\begin{equation}
%%%\begin{split}
%%{h}_{a\chi}(\chi_0,\tau_0) 
%%%&= -\ii h_{\chi-}(a(\chi_0; \tau_0),\chi_0,\tau_0) \\
%%%&=  -\ii g_{\chi-}(a(\chi_0;\tau_0),\chi_0,\tau_0)  + \vartheta_\chi(a(\chi_0; \tau_0);\chi_0,\tau_0)  \\
%%= A(\chi_0,\tau_0)  + R_-(a(\chi_0,\tau_0);\chi_0,\tau_0).
%%%\end{split}
%%\label{eq:h-tilde-a-chi}
%%\end{equation}
%%Similarly,
%%\begin{equation}
%%{h}_{b\chi}(\chi_0,\tau_0) = -\ii h_{\chi}(b(\chi_0; \tau_0),\chi_0,\tau_0) = A(\chi_0,\tau_0)  + R(b(\chi_0,\tau_0);\chi_0,\tau_0).
%%\label{eq:h-tilde-b-chi}
%%\end{equation}
%\begin{align}
%{h}_{a\chi}(\chi_0,\tau_0)  &= A(\chi_0,\tau_0)  + R_-(a(\chi_0,\tau_0);\chi_0,\tau_0) \label{eq:h-tilde-a-chi} \\
%{h}_{b\chi}(\chi_0,\tau_0)  &= A(\chi_0,\tau_0)  + R(b(\chi_0,\tau_0);\chi_0,\tau_0) \label{eq:h-tilde-b-chi}
%\end{align}
%Now, combining \eqref{eq:h-tilde-a-chi} and \eqref{eq:h-tilde-b-chi} together with \eqref{eq:kappa-chi} in \eqref{eq:xi-a-bun} and \eqref{eq:xi-b-bun} gives
%\begin{align}
%\xi_a &= -2 R_-(a(\chi_0;\tau_0);\chi_0,\tau_0),\label{eq:xi-a-explicit}\\
%\xi_b &= -2 R(b(\chi_0;\tau_0);\chi_0,\tau_0).\label{eq:xi-b-explicit}
%\end{align}
%On the other hand, using the expression \eqref{eq:g-tau} for $g_\tau$ together with $\theta_\tau(\lambda;\chi,\tau)=\lambda^2$ lets us write
%\begin{align}
%{h}_{a\tau}(\chi_0,\tau_0) &= 
%A(\chi_0,\tau_0)^2 - \frac{1}{2}B(\chi_0,\tau_0)^2 + (A(\chi_0,\tau_0) + a(\chi_0;\tau_0))R_-(a(\chi_0;\tau_0);\chi_0, \tau_0)\label{eq:h-tilde-a-tau}\\
%{h}_{b\tau}(\chi_0,\tau_0) &=
%A(\chi_0,\tau_0)^2 - \frac{1}{2}B(\chi_0,\tau_0)^2 + (A(\chi_0,\tau_0) + b(\chi_0;\tau_0))R(b(\chi_0;\tau_0);\chi_0, \tau_0)\label{eq:h-tilde-b-tau}
%\end{align}
%%\begin{equation}
%%\begin{split}
%%{h}_{a\tau}(\chi_0,\tau_0) &= -\ii h_{\tau-}(a(\chi_0; \tau_0),\chi_0,\tau_0) \\
%%&=  -\ii g_{\tau-}(a(\chi_0;\tau_0),\chi_0,\tau_0)  + \vartheta_\tau(a(\chi_0; \tau_0);\chi_0,\tau_0)  \\
%%&=  A(\chi_0,\tau_0)^2 - \frac{1}{2}B(\chi_0,\tau_0)^2 + (A(\chi_0,\tau_0) + a(\chi_0;\tau_0))R_-(a(\chi_0;\tau_0);\chi_0, \tau_0)
%%\end{split}
%%\label{eq:h-tilde-a-tau}
%%\end{equation}
%%and, similarly,
%%\begin{equation}
%%\begin{split}
%%{h}_{b\tau}(\chi_0,\tau_0) &= -\ii h_{\tau}(b(\chi_0; \tau_0),\chi_0,\tau_0) \\
%%%&=  -\ii g_{\tau-}(b(\chi_0;\tau_0),\chi_0,\tau_0)  + \vartheta_\tau(b(\chi_0; \tau_0);\chi_0,\tau_0)  \\
%%&=  A(\chi_0,\tau_0)^2 - \frac{1}{2}B(\chi_0,\tau_0)^2 + (A(\chi_0,\tau_0) + b(\chi_0;\tau_0))R(b(\chi_0;\tau_0);\chi_0, \tau_0)
%%\end{split}
%%\label{eq:h-tilde-b-tau}
%%\end{equation}
%We then combine \eqref{eq:h-tilde-a-tau} and \eqref{eq:h-tilde-b-tau} together with \eqref{eq:kappa-tau} in \eqref{eq:Omega-a-bun} and \eqref{eq:Omega-b-bun} and obtain
%\begin{align}
%\Omega_a &= 2(A(\chi_0,\tau_0) + a(\chi_0,\tau_0)) R_-(a(\chi_0;\tau_0);\chi_0,\tau_0),\label{eq:Omega-a-explicit}\\
%\Omega_b &= 2(A(\chi_0,\tau_0) + b(\chi_0,\tau_0))  R(b(\chi_0;\tau_0);\chi_0,\tau_0).\label{eq:Omega-b-explicit}
%\end{align}
%Now, to verify that \eqref{eq:linearized-dispersion-a-bun} holds, we use \eqref{eq:xi-a-explicit} and \eqref{eq:Omega-a-explicit} to observe that
%\begin{equation}
%\begin{split}
%4 (\xi_0 \xi_a-\Omega_a )^2 
%&= 4\left(4 A(\chi_0,\tau_0) R(a(\chi_0,\tau_0); \chi_0, \tau_0) - 2(A(\chi_0, \tau_0)+a(\chi_0, \tau_0)) R_-(a(\chi_0, \tau_0); \chi_0, \tau_0)\right)^{2}\\
%&=16 R_-(a(\chi_0, \tau_0); \chi_0, \tau_0)^2 \left(a(\chi_0, \tau_0) - A(\chi_0, \tau_0) \right)^2,%\\
%%&=16\left( \left(a(\chi_0, \tau_0) - A(\chi_0, \tau_0) \right)^2 + B(\chi_0,\tau_0)^2 \right) \left(a(\chi_0, \tau_0) - A(\chi_0, \tau_0) \right)^2,
%\end{split}
%\label{eq:linearized-dispersion-a-LHS}
%\end{equation}
%where we also recalled \eqref{eq:xi-0-bun}. Next,
%\begin{equation}
%\begin{split}
%\xi_a^2 \left(\xi_a^2 - 4 B(\chi_0,\tau_0)^2 \right)
%&= 4 R_-(a(\chi_0,\tau_0);\chi_0,\tau_0)^2 \left(4 R_-(a(\chi_0,\tau_0);\chi_0,\tau_0)^2 - 4 B(\chi_0,\tau_0)^2 \right)\\
%&= 16 R_-(a(\chi_0,\tau_0);\chi_0,\tau_0)^2 \left( a(\chi_0,\tau_0) - A(\chi_0,\tau_0) \right)^2
%\end{split}
%\label{eq:linearized-dispersion-a-RHS}
%\end{equation}
%since $R(\lambda;\chi,\tau)^2 = (\lambda- A(\chi,\tau))^2 + B(\chi,\tau)^2$. We see from the identities \eqref{eq:linearized-dispersion-a-LHS} and \eqref{eq:linearized-dispersion-a-RHS} that the linearized dispersion relation \eqref{eq:linearized-dispersion-a-bun} holds. A completely analogous calculation having the point $b(\chi_0,\tau_0)$ in place of $a(\chi_0,\tau_0)$ and using \eqref{eq:xi-b-explicit} and \eqref{eq:Omega-b-explicit} shows that the relation \eqref{eq:linearized-dispersion-b-bun} also holds.
%
%As we have now established that the linear systems \eqref{eq:F-a-pm-sys-1}-\eqref{eq:F-a-pm-sys-2} and \eqref{eq:F-b-pm-sys-1}-\eqref{eq:F-b-pm-sys-2} are both singular, it remains to show that the amplitude pairs $(F_a^{+}(\chi_0,\tau_0), F_a^{-}(\chi_0,\tau_0))$
%and $(F_b^{+}(\chi_0,\tau_0), F_b^{-}(\chi_0,\tau_0))$ lie in the corresponding nullspaces. 
%To do so, it suffices to verify that \eqref{eq:F-a-pm-sys-1} and \eqref{eq:F-b-pm-sys-1} hold. Taking the definitions \eqref{eq:shelves-amplitudes-bun} of $F_a^{\pm}$ and $F_b^{\pm}$ into account, verifying these amount to showing that
%\begin{align}
%\cos( \arg(a(\chi_0,\tau_0)-\lambda_0(\chi_0,\tau_0)) ) &= \frac{-2(\xi_0\xi_a - \Omega_a)}{\xi_a^2},\label{eq:show-cos-arg-a-bun}\\
%\cos( \arg(b(\chi_0,\tau_0)-\lambda_0(\chi_0,\tau_0)) ) &= \frac{-2(\xi_0\xi_b - \Omega_b)}{\xi_b^2}.\label{eq:show-cos-arg-b-bun}
%\end{align}
%The right-hand sides above can be simplified by directly using the expressions \eqref{eq:xi-0-explicit}, \eqref{eq:xi-a-explicit}-\eqref{eq:xi-b-explicit}, and \eqref{eq:Omega-a-explicit}-\eqref{eq:Omega-b-explicit}, and it is seen that showing \eqref{eq:show-cos-arg-a-bun} and \eqref{eq:show-cos-arg-b-bun} is equivalent to verifying the identities
%\begin{align}
%\cos( \arg(a(\chi_0,\tau_0)-\lambda_0(\chi_0,\tau_0)) ) &= \frac{a(\chi_0,\tau_0) - A(\chi_0,\tau_0)}{R_-(a(\chi_0,\tau_0))},\label{eq:show-cos-arg-a-simp}\\
%\cos( \arg(b(\chi_0,\tau_0)-\lambda_0(\chi_0,\tau_0)) ) &= \frac{b(\chi_0,\tau_0) - A(\chi_0,\tau_0)}{R(b(\chi_0,\tau_0))}.\label{eq:show-cos-arg-b-simp}
%\end{align}
%Now note that since $a(\chi_0,\tau_0) < \Re (\lambda_0(\chi_0,\tau_0))<0$ and $\Im(\lambda_0(\chi_0,\tau_0))=B(\chi_0,\tau_0)>0$ in $\shelves$, it follows that
%\begin{equation}
%\begin{split}
%\cos( \arg(a(\chi_0,\tau_0) - \lambda_0(\chi_0,\tau_0)) )
% &= - \cos( \arg(\lambda_0(\chi_0,\tau_0) - a(\chi_0,\tau_0)) )\\
% &= - \frac{\Re(\lambda_0(\chi_0,\tau_0))-a(\chi_0,\tau_0)}{| \lambda_0(\chi_0,\tau_0)-a(\chi_0,\tau_0)|}\\
% & = \frac{a(\chi_0,\tau_0) -A(\chi_0,\tau_0)}{\sqrt{\left(a(\chi_0,\tau_0) - A(\chi_0,\tau_0)\right)^2 + B(\chi_0,\tau_0)^2} } <0,
%\end{split}
%\label{eq:cos-arg-a-simp-bun}
%\end{equation}
%where we have used $\Re(\lambda_0(\chi_0,\tau_0)) = A(\chi_0,\tau_0)$. On the other hand, $R(\lambda;\chi,\tau) = \lambda + O(1)$ as $\lambda\to\infty$ and $R(\lambda;\chi,\tau)^2 = (\lambda - A(\chi,\tau))^2 + B(\chi,\tau)^2$, hence
%\begin{equation}
%R_-(a(\chi_0,\tau_0);\chi_0,\tau_0) = \sqrt{\left(a(\chi_0,\tau_0) - A(\chi_0,\tau_0)\right)^2 + B(\chi_0,\tau_0)^2}.
%\end{equation}
% Thus, the last line of \eqref{eq:cos-arg-a-simp-bun} verifies \eqref{eq:show-cos-arg-a-simp} and this establishes that $p_a(\Delta x, \Delta t)$ is a solution of \eqref{eq:linearized-NLS-Delta-bun}.
% Similarly, since $ \Re (\lambda_0(\chi_0,\tau_0))<b(\chi_0,\tau_0)$, it is seen that
%\begin{equation}
%\begin{split}
%\cos( \arg(b(\chi_0,\tau_0) - \lambda_0(\chi_0,\tau_0)) ) &= 
%\frac{b(\chi_0,\tau_0) - \Re(\lambda_0(\chi_0,\tau_0)) }{| \lambda_0(\chi_0,\tau_0)-b(\chi_0,\tau_0)|}\\
%& =  \frac{b(\chi_0,\tau_0) -A(\chi_0,\tau_0)}{\sqrt{\left(b(\chi_0,\tau_0) - A(\chi_0,\tau_0)\right)^2 + B(\chi_0,\tau_0)^2} } >0,
%\end{split}
%\label{eq:cos-arg-b-simp-bun}
%\end{equation}
%and
%\begin{equation}
%R(b(\chi_0,\tau_0);\chi_0,\tau_0) = \sqrt{\left(b(\chi_0,\tau_0) - A(\chi_0,\tau_0)\right)^2 + B(\chi_0,\tau_0)^2}
%\end{equation}
%since $b(\chi_0,\tau_0)$ lies to the right of the branch cut $\Sigma_g$ of $R(\lambda;\chi,\tau)$. Therefore, the last line of \eqref{eq:cos-arg-b-simp-bun} verifies \eqref{eq:show-cos-arg-b-simp} and this establishes that $p_b(\Delta x, \Delta t)$ is a solution of \eqref{eq:linearized-NLS-Delta-bun}.
%\end{proof}
\begin{remark}
Requiring instead any of the individual plane waves 
\begin{equation}
\begin{split}
(\Delta x, \Delta t) &\mapsto \frac{\ii F_a^{[\shelves]}(\chi_0,\tau_0)}{B(\chi_0,\tau_0)}m_a^{\pm}(\chi_0,\tau_0)\ee^{\pm \ii\phi_a(\chi_0,\tau_0;M)} \ee^{\pm \ii(\xi_a \Delta x - \Omega_a \Delta t)}\quad\text{or}\\
(\Delta x, \Delta t) &\mapsto \frac{\ii F_b^{[\shelves]}(\chi_0,\tau_0)}{B(\chi_0,\tau_0)}m_b^{\pm}(\chi_0,\tau_0)\ee^{\pm \ii\phi_b(\chi_0,\tau_0;M)} \ee^{\pm \ii(\xi_b \Delta x - \Omega_b \Delta t)}
\end{split}
\end{equation}
in \eqref{eq:p-a-b-shelves} to be solutions of \eqref{eq:linearization-intro} forces $B(\chi_0,\tau_0)=0$, which is a contradiction. Therefore, one indeed needs to form the combinations $p_a(\Delta x, \Delta t)$ and $p_b(\Delta x, \Delta t)$ as in \eqref{eq:p-a-b-shelves}.
\end{remark}
%\begin{remark}
%Requiring instead any of the individual plane waves 
%\begin{equation}
%(\Delta x, \Delta t)\mapsto \frac{\ii F_a^{\pm}(\chi_0,\tau_0)}{B(\chi_0,\tau_0)}\ee^{\pm \ii(\xi_a \Delta x - \Omega_a \Delta t)},\quad 
%(\Delta x, \Delta t)\mapsto \frac{\ii F_b^{\pm}(\chi_0,\tau_0)}{B(\chi_0,\tau_0)}\ee^{\pm \ii(\xi_b \Delta x - \Omega_b \Delta t)}
%\end{equation}
% in \eqref{eq:p-a-bun} and \eqref{eq:p-b-bun} to be solutions of \eqref{eq:linearized-NLS-Delta-bun} forces $B(\chi_0,\tau_0)=0$, which is a contradiction. Therefore, one indeed needs to take the combinations $p_a$ and $p_b$ as in \eqref{eq:p-a-bun} and \eqref{eq:p-b-bun}.
%\end{remark}
Finally, we show that the relative wavenumbers $\xi_{a,b}$ do not lie in the band of modulational instability $(-2 B(\chi_0,\tau_0), 2 B(\chi_0,\tau_0))$ in view of the well-known theory summarized in Appendix~\ref{A:perturbations}. This result follows from the identities \eqref{eq:xi-a-b-explicit} in a straightforward manner. Indeed,
\begin{equation}
\begin{alignedat}{3}
\xi_a^2 &= 4 \left( a(\chi_0,\tau_0) - A(\chi_0,\tau_0) \right)^2 + 4 B(\chi_0,\tau_0)^2 &&> 4 B(\chi_0,\tau_0)^2,
\\
\xi_b^2 &= 4 \left( b(\chi_0,\tau_0) - A(\chi_0,\tau_0) \right)^2 + 4 B(\chi_0,\tau_0)^2 &&> 4 B(\chi_0,\tau_0)^2.
\end{alignedat}
\end{equation}


%As we have now established that $p_a(\Delta x, \Delta t)$ and $p_b(\Delta x, \Delta t)$ define plane wave solutions of the linearization of the NLS equation \eqref{eq:linearized-NLS-Delta-bun} about the plane wave $Q(\Delta x, \Delta t)$, we would like to investigate whether the relative wave numbers $\xi_a$ and $\xi_b$ lie in the bands of modulational instability $(-2 B(\chi_0,\tau_0), 2 B(\chi_0,\tau_0))$ in view of the calculation done in Appendix~\ref{A:perturbations}. The following proposition tells us that this is not the case, and concludes the analysis in this section.
%
%\begin{proposition}
%The relative wave numbers $\xi_a=\xi_a(\chi_0,\tau_0)$ and $\xi_b=\xi_b(\chi_0,\tau_0)$ satisfy
%\begin{equation}
%\xi_a^2 > 4 B(\chi_0,\tau_0)^2\quad\text{and}\quad  \xi_b^2 > 4 B(\chi_0,\tau_0)^2
%\end{equation}
%for $(\chi_0,\tau_0)\in \shelves$.
%\end{proposition}
%\begin{proof} This result follows from the identities \eqref{eq:xi-a-explicit} and \eqref{eq:xi-b-explicit} in a straightforward manner. Indeed,
%\begin{equation}
%\xi_a^2 = 4 \left( a(\chi_0,\tau_0) - A(\chi_0,\tau_0) \right)^2 + 4 B(\chi_0,\tau_0)^2 > 4 B(\chi_0,\tau_0)^2,
%\end{equation}
%and
%\begin{equation}
%\xi_b^2 = 4 \left( b(\chi_0,\tau_0) - A(\chi_0,\tau_0) \right)^2 + 4 B(\chi_0,\tau_0)^2 > 4 B(\chi_0,\tau_0)^2,
%\end{equation}
%which prove the claim.
%\end{proof}

%\subsection{Special case of fundamental rogue waves} \textcolor{red}{[Will there be a corollary?]} To see how the results established in this section apply to the fundamental rogue waves $\psi_k(x,t)$, $k\in \mathbb{Z}>0$, it suffices to (i) restrict $M$ to the sequence $M = \tfrac{1}{2}k + \tfrac{1}{4}$ and tie $s$ to the order $k\in \mathbb{Z}>0$ by $s=(-1)^k$, and (ii) take into account the exponential factor $\ee^{-\ii M \tau}$ mediating between \eqref{eq:q-S} and \eqref{eq:psi-k-S}. Doing so in \eqref{eq:q-bun} yields
%\begin{multline}
%\psi_k(M\chi,M\tau) = B(\chi,\tau)  \ee^{-\ii M \tau} \ee^{-2\ii \left(M \kappa(\chi,\tau) + \mu(\chi,\tau) + \frac{1}{4}(-1)^k \pi\right)}
%+ M^{-\frac{1}{2}} (-1)^k  \ee^{-\ii M \tau}  \ee^{-2\ii (M\kappa(\chi,\tau)+\mu(\chi,\tau) )} \\
% \cdot \left[ F_a^+(\chi,\tau)\ee^{\ii \Theta_a(\chi,\tau;M)}    + F_a^-(\chi,\tau)\ee^{- \ii \Theta_a(\chi,\tau;M)} \right. \\
%  \left. + F_b^+(\chi,\tau)\ee^{\ii \Theta_b(\chi,\tau;M)}   + F_b^-(\chi,\tau)\ee^{- \ii \Theta_b(\chi,\tau;M)} \right] + O(M^{-1}),
%  \label{eq:psi-k-bun}
%\end{multline}
%as $M\to +\infty$. This may also be written in the form
%
%\begin{multline}
%%\psi_k(M\chi, M\tau) = 
%\psi_k(M\chi,M\tau)=
%(-1)^k  \ee^{-2\ii (M\kappa(\chi,\tau)+\mu(\chi,\tau))} 
%\left[-\ii B(\chi,\tau) + M^{-\frac{1}{2}} \left( F_a^+(\chi,\tau)\ee^{\ii \Theta_a(\chi,\tau;M)}  \right. \right.\\ + F_a^-(\chi,\tau)\ee^{- \ii \Theta_a(\chi,\tau;M)}
%\left.\left. 
%+ F_b^+(\chi,\tau)\ee^{\ii \Theta_b(\chi,\tau;M)} + F_b^-(\chi,\tau)\ee^{- \ii \Theta_b(\chi,\tau;M)} 
%\right) \right]
%+ O(M^{-1}).
%\label{eq:psi-k-bun-alt}
%\end{multline}
%
%Notice that as $(\chi,\tau)$ approaches from within $\shelves$ the boundary curve separating $\shelves$ and $\exterior$, $a(\chi,\tau)$ and $b(\chi,\tau)$ coalesce and we see from the formula \eqref{eq:mu-formula-intro} that $\mu(\chi,\tau)$ tends to $0$ since the domain of integration contracts to a point. In this case the overall phase $M \kappa(\chi, \tau)+ \mu(\chi,\tau) + \tfrac{1}{4}(-1)^k \pi$ in \eqref{eq:psi-k-bun} becomes $M \kappa(\chi,\tau)+\tfrac{1}{4}(-1)^k \pi$. On the other hand, we see from \eqref{eq:kappa-gamma} that $M \kappa(\chi,\tau) = M \gamma(\chi,\tau) - n\pi - \tfrac{1}{4}(-1)^k \pi$ with $n\in\mathbb{Z}_>0$, hence the simplification
%\begin{equation}
%\ee^{-2\ii \left(M \kappa(\chi,\tau) + \tfrac{1}{4}(-1)^k \pi \right)} = \ee^{-2\ii M \gamma(\chi,\tau)}.
%\end{equation}
%This shows us that the leading term in \eqref{eq:psi-k-bun} matches the leading term in \eqref{eq:psi-k-shelves-chi-tau-ALT} for $(\chi,\tau)$ on the boundary curve separating $\shelves$ and $\exterior$ through the vanishing of $\mu(\chi,\tau)$. 



%\begin{remark}
%As $(\chi,\tau)$ approaches from within $\shelves$ the boundary curve separating $\shelves$ and $\exterior$, $a(\chi,\tau)$ and $b(\chi,\tau)$ coalesce and we see from the formula \eqref{eq:mu-def-bun} that $\mu(\chi,\tau)$ tends to $0$ since to domain of integration contracts to a point. In this case the overall phase factor $M \kappa(\chi, \tau)+ \mu(\chi,\tau)$ in \eqref{eq:q-bun} becomes equal to $M \gamma(\chi,\tau)$.
%%, see the definition \eqref{eq:Theta-0}. 
%Thus, we see that the leading term in \eqref{eq:q-bun} matches the leading term in \eqref{eq:q-M-shelves-chi-tau} for $(\chi,\tau)$ on the boundary curve separating $\shelves$ and $\exterior$ through the vanishing of $\mu(\chi,\tau)$. \textcolor{red}{[This should be linked to the construction of $g$ earlier in $\shelves$ vs. $\exterior$. Also we should give this remark only after the subsection "special case of fundamental rogue waves". But where would that section go? The following material is also valid for $M>0$ untied to $s=\pm 1$.]}
%\end{remark}

%Next, we compute the absolute errors measured in the sup-norm

%To verify the accuracy of the asymptotic formula \eqref{eq:q-bun} we first compare the  with the exact solution $\psi_k(\chi,\tau)$

%\sqrt{\frac{\ln(2)}{\pi}}\ee^{\ii(\frac{1}{4}\pi+2\pi p^2-\arg(\Gamma(\ii p)))}
%In view of Remark~\ref{rem:omega} we see that 
%\begin{equation}
%\omega(\lambda) = \left( \frac{\lambda-\ii }{ \lambda+\ii } \right)^{1/4}
%\end{equation}
%where the right-hand side defined to have its branch cut to be $\Sigma_g$ 
%complex power function is chosen as the principal branch, ensuring that $\omega(\lambda)\to 1$ as $\lambda\to\infty$, and the branch cut is chosen t
%\section{\textcolor{red}{Far-Field Asymptotics of Rogue Waves in the ``Channels''}}
%We make the following substitution into Riemann-Hilbert Problem~\ref{rhp:rogue-wave}:
%\begin{equation}
%\mathbf{P}^{(k)}(\lambda;x,t)\defeq \ee^{\ii t\sigma_3/2}\mathbf{M}^{(k)}(\lambda;x,t)\begin{cases}
%\ee^{-\ii\rho(\lambda)(x+\lambda t)\sigma_3}\mathbf{Q}\ee^{\ii\lambda(x+\lambda t)\sigma_3},&\quad\text{$\lambda$ inside $\Sigma_\circ$}
%\\
%\ee^{-\ii (t+x\lambda^{-1})\sigma_3/2}\ee^{\pm 2\ii n(\lambda^{-1}-\lambda^{-3}/3)\sigma_3},&\quad\text{$\lambda$ exterior to $\Sigma_\circ$,}
%\end{cases}
%\end{equation}
%where the top (bottom) sign refers to the case $k=2n$ ($k=2n-1$).  Since $\ee^{\ii\rho(\lambda)(x+\lambda t)\sigma_3}$ satisfies the jump condition of $\mathbf{M}^{(k)}(\lambda;x,t)$ across the contour $\Sigma_\mathrm{c}$, it follows that $\mathbf{P}^{(k)}(\lambda;x,t)$ can be considered to be analytic in the interior of $\Sigma_\circ$.  It is not hard to check that moreover, $\mathbf{P}^{(k)}(\lambda;x,t)$ is the solution of the following modified problem.
%\begin{rhp}[Modified problem for rogue waves]
%Let $(x,t)\in\mathbb{R}^2$ be arbitrary parameters, and let $k\in\mathbb{Z}_{\ge 0}$.  Find a $2\times 2$ matrix $\mathbf{P}^{(k)}(\lambda;x,t)$ with the following properties:
%\begin{itemize}
%\item[]\textbf{Analyticity:}  $\mathbf{P}^{(k)}(\lambda;x,t)$ is analytic in $\lambda$ for $\lambda\in\mathbb{C}\setminus\Sigma_\circ$, and it takes continuous boundary values on $\Sigma_\circ$.
%\item[]\textbf{Jump conditions:}  The boundary values on the jump contour $\Sigma_\circ$ are related as follows:
%\begin{multline}
%\mathbf{P}_+^{(k)}(\lambda;x,t)=\mathbf{P}_-^{(k)}(\lambda;x,t)\ee^{-\ii\lambda(x+\lambda t)\sigma_3}\left(\frac{\lambda-\ii}{\lambda+\ii}\right)^{\pm n\sigma_3}\mathbf{Q}^{-1}\mathbf{E}(\lambda)\\
%{}\cdot\ee^{\pm 2\ii n(\lambda^{-1}-\tfrac{1}{3}\lambda^{-3})\sigma_3}\ee^{\ii[\rho(\lambda)(x+\lambda t)-\tfrac{1}{2}(t+x\lambda^{-1})]\sigma_3},\quad
%\lambda\in\Sigma_\circ,
%\label{eq:P-jump}
%\end{multline}
%where the top (bottom) sign in the exponents corresponds to $k=2n$ ($k=2n-1$).
%\item[]\textbf{Normalization:}  $\mathbf{P}^{(k)}(\lambda;x,t)\to\mathbb{I}$ as $\lambda\to\infty$. 
%\end{itemize}
%\label{rhp:rogue-wave-modified}
%\end{rhp}
%The formula for $\psi_k(x,t)$ corresponding to \eqref{eq:psi-from-M} now becomes
%\begin{equation}
%\psi_k(x,t)=2\ii\ee^{-\ii t}\lim_{\lambda\to\infty}\lambda P^{(k)}_{12}(\lambda;x,t).
%\label{eq:psi-from-P}
%\end{equation}
%Now, the radius of $\Sigma_\circ$ can be taken to be arbitrarily large.  Note the elementary expansions
%\begin{equation}
%\left(\frac{\lambda-\ii}{\lambda+\ii}\right)^{\pm n\sigma_3} = \ee^{\mp 2\ii n(\lambda^{-1}-\tfrac{1}{3}\lambda^{-3})\sigma_3}\ee^{O(n\lambda^{-5})\sigma_3},\quad\lambda\to\infty,
%\label{eq:large-lambda-1}
%\end{equation}
%\begin{equation}
%\mathbf{E}(\lambda)=\mathbb{I}-\frac{1}{2}\ii\lambda^{-1}\sigma_1 +\begin{bmatrix}O(\lambda^{-2}) & O(\lambda^{-3})\\
%O(\lambda^{-3}) & O(\lambda^{-2})\end{bmatrix},\quad\lambda\to\infty,
%\label{eq:large-lambda-2}
%\end{equation}
%and
%\begin{equation}
%\ee^{\ii[\rho(\lambda)(x+\lambda t)-\tfrac{1}{2}(t+x\lambda^{-1})]\sigma_3}=\ee^{\ii\lambda(x+\lambda t)\sigma_3}\ee^{O(t\lambda^{-2})\sigma_3}\ee^{O(x\lambda^{-3})\sigma_3},\quad\lambda\to\infty.
%\label{eq:large-lambda-3}
%\end{equation}