\documentclass[11pt]{amsart}
\usepackage{geometry}                % See geometry.pdf to learn the layout options. There are lots.
%\geometry{letterpaper}                   % ... or a4paper or a5paper or ...
\geometry{margin=1in}          % ... or a4paper or a5paper or ...
\usepackage[latin1]{inputenc}   %MAC
\usepackage{mathpazo}  %Palatino fonts
\usepackage[english]{babel}
\usepackage[usenames,dvipsnames,cmyk]{xcolor}
\usepackage{url}
\usepackage{amsmath,amssymb,latexsym,mathrsfs,amssymb,amsthm}
\usepackage{enumerate}
\usepackage{mathtools} %Needed for \defeq and \eqdef
%\usepackage{epstopdf}
%\usepackage{graphicx}
\usepackage[all]{xy}
\usepackage{setspace}
\usepackage{tikz, tkz-euclide}
\onehalfspacing

\usepackage[colorlinks=true, linkcolor=black, citecolor=DarkOrchid, urlcolor=DarkOrchid]{hyperref}

%\usepackage[inline]{showlabels}
%\usepackage[notref,notcite]{showkeys}
%\usepackage{showlabels}


%\parindent 0mm

\usetikzlibrary{angles,quotes}
%%%%%%%%%  TIKZ %%%%%%%%%%%%%%%%%%%%%%%%%%
%\usepackage{graphicx,tikz, tkz-euclide}
%\usepackage{float, pgfplots, overpic}
%%\usepackage{subfigure}
%\usetikzlibrary{arrows,calc, decorations.markings, positioning, fixedpointarithmetic}
%%\pgfplotset{compat=1.9}
%%\usepgfplotslibrary{fillbetween}
%\usetkzobj{all}
%%%%%%%%%%%%%%%%%

%%%%%%%%%COUNTER%%%%%%%%%%%%%%%%%%%%
\setcounter{section}{0}
\setcounter{tocdepth}{2}

%%%%%%%%%ENVIRONMENTS%%%%%%%%%%%%%%%%%
%\theoremstyle{theorem}
\newtheorem{theorem}{Theorem}[section]
\newtheorem{lemma}[theorem]{Lemma}
\newtheorem{proposition}[theorem]{Proposition}
\newtheorem{corollary}[theorem]{Corollary}

\newtheorem{conjecture}{Conjecture}
\newtheorem{question}{Question}
\newtheorem{exercise}{Exercise}
\newtheorem{solution}{Solution}
\newtheorem*{notation}{Notation}

\theoremstyle{definition}
\newtheorem{definition}[theorem]{Definition}

\theoremstyle{remark}
\newtheorem{remark}[theorem]{Remark}
\newtheorem{example}[theorem]{Example}%[chapter]
\newtheorem{discussion}[theorem]{Discussion}%[chapter]
\newtheorem{rhp}{Riemann-Hilbert Problem}
\newcommand{\rhref}[1]{Riemann-Hilbert Problem~\ref{#1}}

%%%%%%%%%%%%%%%OPERATORS%%%%%%%%%%%%%%%%%%%
%\DeclareGraphicsRule{.tif}{png}{.png}{`convert #1 `dirname #1`/`basename #1 .tif`.png}
\let\Re=\undefined\DeclareMathOperator{\Re}{Re}
\let\Im=\undefined\DeclareMathOperator{\Im}{Im}
\DeclareMathOperator*{\res}{Res}
\DeclareMathOperator{\sech}{sech}
\DeclareMathOperator{\rem}{Rem}

\DeclareMathOperator{\diag}{diag}
\DeclareMathOperator{\sign}{sign}
\DeclareMathOperator{\wron}{Wron}
\newcommand{\channels}{\ensuremath{\mathcal{C}}}
\newcommand{\shelves}{\ensuremath{\mathcal{S}}}
\newcommand{\exterior}{\ensuremath{\mathcal{E}}}
\newcommand{\dd}{\ensuremath{\,\mathrm{d}}}
\newcommand{\ii}{\ensuremath{\mathrm{i}}}
\newcommand{\ee}{\ensuremath{\,\mathrm{e}}}
\newcommand{\defeq}{\vcentcolon=}
\newcommand{\eqdef}{=\vcentcolon}

%%-----Counter
%\setcounter{section}{1}
%\setcounter{tocdepth}{2}

%%%%%%%%%%%%%%%%%%%%%%%%%%%%%%%%%%%%
%%LINESPACE FOR MATRICES
\makeatletter
\renewcommand*\env@matrix[1][\arraystretch]{%
  \edef\arraystretch{#1}%
  \hskip -\arraycolsep
  \let\@ifnextchar\new@ifnextchar
  \array{*\c@MaxMatrixCols c}}
\makeatother

%%%%%%%%%%%%%%%%%%%%%%%%%%%%%%%%%%%%

%%%%%%%%%%%%%%%%%%%%%%%%%%%%%%%%
%FIX EXTRA SPACE ADDED BY \LEFT and \RIGHT PARENTHESES
\let\originalleft\left
\let\originalright\right
\renewcommand{\left}{\mathopen{}\mathclose\bgroup\originalleft}
\renewcommand{\right}{\aftergroup\egroup\originalright}
%%%%%%%%%%%%%%%%%%%%%%%%%%%%%%%%

%opening
\title{Extreme Superposition:  High-Order Fundamental Rogue Waves in the Far-Field Regime}

\author{Deniz Bilman}
\address{Deniz Bilman:  Department of Mathematical Sciences, University of Cincinnati, Cincinnati, OH, USA}
\email{bilman@uc.edu}
%\author{Liming Ling}
\author{Peter D.~Miller}
\address{Peter D. Miller:  Department of Mathematics, University of Michigan, Ann Arbor, MI, USA}
\email{millerpd@umich.edu}
%\author{Alexander Tovbis}

%\address{
%University of Michigan, Ann Arbor, MI 48109, USA.
%}
%\email{bilman@umich.edu}
%\email{millerpd@umich.edu}


\thanks{The authors wish to thank Liming Ling and Alex Tovbis for useful discussions during the early stages of this project.
Bilman's work was partially supported by a research fellowship from Charles Phelps Taft Research Center.
Miller was supported by the National Science Foundation under grant number DMS-1812625.
}

%\keywords{Rogue waves;  Inverse scattering transform;  Nonlinear Schr\"odinger equation.}
%\subjclass[2010]{35C05, 35Q15, 35Q55, 37K10}
\date{\today}

\begin{document}

\begin{abstract}
We study fundamental rogue-wave solutions of the focusing nonlinear Schr\"odinger equation in the limit that the order of the rogue wave is large and the independent variables $(x,t)$ are proportional to the order (the far-field limit).  We first formulate a Riemann-Hilbert representation of these solutions that allows the order to vary continuously rather than by integer increments.  The intermediate solutions in this continuous family include also soliton solutions for zero boundary conditions spectrally encoded by  a single complex-conjugate pair of poles of arbitrary order, as well as other solutions having nonzero boundary conditions matching those of the rogue waves albeit with far slower decay as $x\to\pm\infty$.  The large-order far-field asymptotic behavior of the solution depends on which of three disjoint regions $\channels$, $\shelves$, and $\exterior$ contains the rescaled variables.  On the regions $\channels$ and $\shelves$ we show that the asymptotic behavior is the same for all continuous orders, while in the region $\exterior$ the discrete sequence of rogue-wave orders produces distinctive asymptotic behavior that is different from other cases.
\end{abstract}

\maketitle

\section{Introduction}
\section{Introduction}
\label{sec:Introduction}


The goal in top-$\size$ recommendation is to recommend to each
consumer a small set of $\size$ items from a large collection of
items~\cite{cremonesi2010performance}.  For example, Netflix may want
to recommend $\size$ appealing movies to each consumer.  Collaborative
Filtering (CF)~\cite{herlocker2002empirical,lee2012comparative} is a
common top-$\size$ recommendation method.  CF infers user interests by
analyzing partially observed user-item interaction data, such as user
ratings on movies or historical purchase
logs~\cite{kanagal2012supercharging}. The main assumption in CF is that
users with similar interaction patterns have similar interests.


Standard CF methods for top-$\size$ recommendation focus on making  suggestions  that accurately reflect the user's preference history. However, as  observed in previous work,  CF recommendations are generally biased toward  popular items, leading to a rich get richer effect~\cite{vargas2014improving,steck2011item}.  The major reasons for this are \textit{popularity bias} and \textit{sparsity} of CF interaction data (detailed in Section~\ref{sec:related-work}). In a nutshell, to maintain  accuracy, recommendations are generated from the dense regions of the data,  where the popular items lie.  

However,  accurately suggesting popular items, may not be satisfactory for the consumers. For example, in Netflix, an accuracy-focused movie recommender may recommend ``Star Wars: The Force Awakens'' to users who have seen ``Star Wars: Rogue One''.  But, those users are probably already aware of ``The Force Awakens''. Considering additional factors, such as novelty of recommendations,  can lead to more effective suggestions~\cite{cremonesi2010performance,Castells2015,zhang2008avoiding,ziegler2005improving,zhang2012auralist}. 
%Second, accuracy-focused models typically achieve a   overall item-space coverage across their recommendations,  whereas high item-space coverage helps providers of the items increase revenue
%, users satisfaction since they are  likely already aware of or can find these items on their own.  

Focusing on popular items also adversely affects the satisfaction of  the providers of the items. This is because  accuracy-focused models typically achieve a  low overall item space coverage across their recommendations, whereas   high item space coverage helps providers of the items increase their revenue~\cite{vargas2014improving,Castells2015,adomavicius2011maximizing,anderson2006thelongtail, yin2012challenging,adomavicius2012improving}.
%accuracy-focused models typically achieve a

In contrast to the relatively small number of popular items, there are copious  {\it long-tail\/} items that have fewer observations (e.g., ratings) available. More precisely,  using the Pareto  principle (i.e.,~the $80/20$ rule),  long-tail items can be defined as items that generate the lower $20\%$ of observations~\cite{yin2012challenging}. Experimentally we found that these items correspond to almost $85\%$ of the items in several datasets (Sections~\ref{sec:Notation} and \ref{sec:Experiments}). %Table~\ref{tab:DatasetStatsticsSmall})


As previously shown, one way to improve the novelty of top-$\size$ sets is to recommend interesting long-tail items~\cite{cremonesi2010performance,ge2010beyond}.  The intuition  is that since they have fewer observations available,  they are more likely to be unseen~\cite{Kaminskas:2016:DSN:3028254.2926720}.  
 %For example, in online commerce,  newly added items are long-tail items that are yet to be discovered.  
Moreover, long-tail item promotion also results in higher overall coverage of the item space%, which increases profits for providers of the items
~\cite{vargas2014improving,Castells2015,zhang2008avoiding,zhang2012auralist,adomavicius2011maximizing,anderson2006thelongtail,yin2012challenging,jambor2010optimizing}. Because long-tail promotion reduces accuracy~\cite{steck2011item}, there are trade-offs to be explored.


%original submitted to ICDE
%This work studies three aspects of top-$\size$ recommendation: accuracy, novelty, and item-space coverage, and examines their trade-offs. In most previous work, predictions of a base recommendation system are re-ranked to handle their trade-offs~\cite{adomavicius2012improving,jambor2010optimizing,zhang2013personalize,wang2009portfolio}. Due to performance considerations, however, these techniques are not customized per user. For example,  parameters that balance the trade-off between novelty and accuracy are cross-validated at a global level.  This can be detrimental since users have varying preferences for  objectives such as long-tail novelty. We explore how to  automatically infer  user  preference for long-tail novelty, and how to leverage  it to correct  the popularity bias in standard recommender models. Our work does not rely on any additional contextual data, although such data, if available, can help promote newly-added long-tail items~\cite{agarwal2009regression,Saveski:2014:ICR:2645710.2645751}.

This work studies three aspects of top-$\size$ recommendation: accuracy, novelty, and item space coverage, and examines their trade-offs. In most previous work, predictions of a base recommendation algorithm are \textit{re-ranked} to handle these trade-offs~\cite{adomavicius2012improving,jambor2010optimizing,zhang2013personalize,wang2009portfolio}. The re-ranking models are computationally efficient but suffer from two drawbacks. First, due to performance considerations,  parameters that balance the trade-off between novelty and accuracy  are not customized per user. Instead they are cross-validated at a global level.  This can be detrimental since users have varying preferences for  objectives such as long-tail novelty. Second,  the re-ranking methods are often limited to a specific base recommender  that may be sensitive to dataset density. 
As a result, the datasets are pruned and the problem is studied in dense settings~\cite{adomavicius2012improving,ho2014likes}; but real world  scenarios are often sparse~\cite{kanagal2012supercharging,liu2017experimental}.   
% Because  dataset density can impact the performance of most base recommenders (like R-SVD), which in turn affects the performance of the re-ranking model, 

\iffalse
We address these limitations by directly inferring  user  preference for long-tail novelty  from interaction data.  This  allows us to customize the re-ranking  per user, and design a \textit{generic} framework, which resolves the second problem. In particular, since the long-tail novelty preferences are estimated independently of any base  recommender model, we can  plug-in an appropriate base recommender w.r.t. the dataset sparsity.% including ones that are more suitable for sparse settings.  

Modelling  user  preference for  long-tail novelty using only item popularity statistics, e.g., the average popularity of rated items as in~\cite{jugovac2017efficient}, disregards additional information like whether the user found the item interesting and the long-tail preferences of other users  of the items. \iffalse To incorporate them, we introduce the notion of  \emph{item long-tail importance}. Both  user long-tail preferences and item long-tail importance are dependent:  a user has high preference for discovering long-tail items if she is interested in important long-tail items, and an item that is associated with many of these kinds of users is likely to be more important.  We propose a joint optimization framework to directly learn,  from interaction data, both the users' long-tail preferences and the  items' long-tail importance. \fi
We propose an optimization approach that  incorporates  this information and  directly learns,  from interaction data, the users' long-tail novelty preferences.

Next, we use these learned preferences  to design a  top-$\size$ recommendation framework thats is generic, and provides customized balance between accuracy, novelty, and coverage. We refer to it as framework as GANC.  Using GANC, we design a novel algorithm, {\it Ordered Sampling-based Locally Greedy (OSLG)\/}, that relies on the learned long-tail novelty preferences  to scalably correct for popularity bias. Our work does not rely on any additional contextual data, although such data, if available, can help promote newly-added long-tail items~\cite{agarwal2009regression,Saveski:2014:ICR:2645710.2645751}. In summary:
\fi

We address the first limitation by directly inferring  user  preference for long-tail novelty  from interaction data.   Estimating these  preferences  using only item popularity statistics, e.g., the average popularity of rated items as in~\cite{jugovac2017efficient}, disregards additional information, like whether the user found the item interesting or the long-tail preferences of other users  of the items. We propose an approach that  incorporates  this information and  learns the users' long-tail novelty preferences from interaction data.

This approach allows us to customize the re-ranking  per user, and  design a \textit{generic} re-ranking framework, which resolves the second limitation of prior work. In particular, since the long-tail novelty preferences are estimated independently of any base recommender, we can  plug-in an appropriate one w.r.t. different factors, such as the dataset sparsity.

Our top-$\size$ recommendation framework, \textbf{GANC}, is \textbf{G}eneric, and provides customized balance between \textbf{A}ccuracy, \textbf{N}ovelty, and \textbf{C}overage. % Moreover, based on the learned long-tail novelty preferences, we also design a novel algorithm, {\it Ordered Sampling-based Locally Greedy (OSLG)\/}, that relies on the learned long-tail novelty preferences  to scalably correct for popularity bias. 
Our work does not rely on any additional contextual data, although such data, if available, can help promote newly-added long-tail items~\cite{agarwal2009regression,Saveski:2014:ICR:2645710.2645751}. In summary:

%Consider  the following toy example:
\vspace{-0.2cm}
\begin{table}[htb]
\centering
\scriptsize
%\small
\begin{tabular}{ccccccc} 
%\toprule
%&\multirow{2}{*}{}&\multicolumn{7}{c}{Ratings}\\
& & \cellcolor{blue!35}$w_1$ &\cellcolor{blue!18} $w_2$ & $\dots$ &\cellcolor{blue!8} $w_{89}$  &\cellcolor{blue!8} $w_{99}$   
\\
&   &$i_1$&$i_2$&$\dots$&$i_{89}$&$i_{90}$\\ 
\cmidrule(r){3-7} 	 
%\midrule
\cellcolor{red!35}$\theta_1$  &$u_1 $   &5 &   & $\dots$ &  &   \\
\cellcolor{red!28}$\theta_2$  &$u_2$     &5 &    & $\dots$ &  &  \\
 $\theta_3=?$  &$\bf u_3$  &5 &  &   $\dots$ &  &  \\
\cellcolor{red!10}$\theta_4$ & $u_4$  &  &5   & $\dots$ & &\\ 
\cellcolor{red!10}$\theta_5$ & $u_5$  &  & 5  & $\dots$ & &\\ 
$\theta_6=?$  & $\bf u_6$ & &5  &      $\dots$& &  \\ 
 & & $\hdots$  &$\hdots$   &$\hdots$   &$\hdots$   &$\hdots$  \\
%\midrule 
\cmidrule(r){3-7} 	 
\multicolumn{2}{c}{item pop.}  & 3  & 3  & $\dots$ &50&60\\  
%\bottomrule
%$ f_i$    &3  &3  &1  &3  &1  &2  \\  \hline
\end{tabular}
%#.
\caption{Simplified user-item interaction data. The user long-tail novelty preference ($\theta_u$), item long-tail importance weight ($w_i$) are highlighted. Darker colors indicate larger values. } \label{tab:example}
\end{table} 
\vspace{-0.2cm}
\begin{example}  
In Table~\ref{tab:example}, we are interested in estimating $\theta_3$ and $\theta_6$,  the long-tail preference of users $u_3$ and $u_6$ who have each rated a single movie. Additional ratings for other users  are not included here.  Considering only rating information, we observe $i_1$ and $i_2$ are  equally popular $|\mathcal{U}_{i_1}^{\trainset}| = |\mathcal{U}_{i_2}^{\trainset}|=3$, and $r_{31}=5$ and $r_{62}=5$. Using Eq.~\ref{eq:tfidf-risk}  we have $\theta_3 = \theta_6$. However, if we were given the long-tail preferences of the each item's user set, specifically that $u_1$ and $u_2$ have high long-tail preference (darker red), while $u_4$ and $u_5$ have lower long-tail preference (lighter red), we could conclude $i_1$ is a more important long-tail item compared to $i_2$ (indicated by a darker blue shade for $w_1$), and we expect  $\theta_3 \geq \theta_6$.

% On the other hand, if we knew that $u_4$ and $u_5$ have lower long-tail preference, we could conclude $i_2$ is a  less significant long-tail item. Therefore, However, if we  consider the long-tail preferences of other users, we may reason differently.    We need another variable $w_i$ which captures this information. 
%we would conclude that $u_3$ has higher long-tail preference compared to $u_6$, since the users $i_1$ is a more prominent long-tail item. 

% Relying only  on item popularity information, we would  conclude   $u_3$ and $u_6$ have equal long-tail preference, since $i_1$ and $i_2$ are  equally popular. However, considering  the second column,  long-tail preference of users,  long-tail importance for each item,  which captures the long-tail preference of its users. Since  that  both users of $i_1$ have high long-tail preference while  the users of $i_2$ have lower preference,  we may conclude $i_1$ is a more important long-tail item compared to $i_2$. Therefore, $u_3$'s long-tail preference should be at least as large as $u_6$'s preference. Specifically, consider two  items $i_1$ and $i_2$, with the following rating data: $i_1=\{u_1:5, u_2:5, u_3:5 \}$, $i_2=\{u_4:5, u_5:5, u_6:5\}$.  

%Table~\ref{tab:example} shows  simplified rating data. We want an estimate of the long-tail preference of $u_3$ and $u_6$, who have each  rated a single movie.  Relying only  on movie popularity information, we would  conclude   $u_3$ and $u_6$ have similar long-tail preference, since $m_1$ and $m_2$ are  equally popular. However, considering the long-tail preferences of other users of those movies, we may reason differently: since $u_1$ and $u_2$ have high long-tail preference, and $u_4$ and $u_5$ have low long-tail preference, $m_1$ is a more prominent long-tail item compared to $m_2$. Therefore, it is likely that $u_3$ has higher long-tail preference compared to $u_6$.considering the long-tail preferences of other users of those movies, we may reason differently.  For example, 
\label{ex:running}
\end{example}



%------------------------------

\iffalse
\begin{example}
Table~\ref{tab:example} shows rating data for a simplified system. %Note the user-item interaction matrix is sparse.
For this example, we define popular movies as those that have received  three or more ratings; $\{m_1, m_2, m_4\}$ are popular and  $\{m_3, m_5, m_6\}$ are niche movies. We observe $u_1$ and $u_3$  have rated relatively popular movies (risk-averse) while $u_2$ and $u_4$ have rated niche movies (risk-loving). 
\label{ex:running}
\end{example}

\begin{table}[htb]
\centering
\scriptsize
\begin{tabular}{ccccccc} 
\toprule
			&$m_1$ &$m_2$   &$m_3$    &$m_4$   &$m_5$ &$m_6$  \\ \hline 
$u_1 $ &5  &4  & - &-  &-  &-   \\
$u_2$  &-  &-  &-  &-  &5  &5   \\
$u_3$  &-  &4  &-  &5  &-  &-   \\
$u_4$  &-  &-  &3  &-  &-  &4   \\ 
$u_5$  &5  &-  &-  &3  &-  &-   \\ 
$u_6$  &4  &2  &-  &4  &-  &-   \\ 
\bottomrule
%$ f_i$    &3  &3  &1  &3  &1  &2  \\  \hline
\end{tabular}
\caption{User-Movie rating data} \label{tab:example}
\end{table}

It is essential to consider consumer characteristics in designing recommender systems so that they promote long-tail items to the right group of users and spread demand evenly between hit and niche items.  

\fi





%------------------------------
\iffalse
\begin{table}[htb]
\centering
\scriptsize
\begin{tabular}{ccccccc} 
\toprule
			&$m_1$ &$m_2$   &$m_3$    &$m_4$   &$m_5$ &$m_6$  \\ \hline 
$u_1 $ &\textbf{5}  & \textbf{4}  &\textcolor{gray}{ 1.2} &-  &-  &-   \\
$u_2$  &-  &-  &-  &-  & \textbf{5}  &\textbf{5}   \\
$u_3$  &-  &\textbf{4}  &-  &\textbf{5}  &-  &-   \\
$u_4$  &-  &-  &\textbf{3}  &-  &-  &\textbf{4}   \\ 
$u_5$  &\textbf{5}  &-  &-  &\textbf{3}  &-  &-   \\ 
$u_6$  &\textbf{4}  &\textbf{2}  &-  &\textbf{4}  &-  &-   \\ 
\bottomrule
%$ f_i$    &3  &3  &1  &3  &1  &2  \\  \hline
\end{tabular}
\caption{User-Movie rating data} \label{tab:example}
\end{table}
% $\mathcal{P}^1= \{ \mathcal{P}_1^1 \{i_1,i_2,i_3\}, \mathcal{P}_2^1:\{i_2,i_3,i_5\}  \}$
 %$\mathcal{P}^2= \{ \mathcal{P}_1^2: \{i_1,i_2,i_3\}, \mathcal{P}_2^2:\{i_2,i_5,i_6\}  \}$
 %$\mathcal{P}^3= \{ \mathcal{P}_1^3: \{i_7,i_8,i_9\}, \mathcal{P}_2^3:\{i_{10},i_{11},i_{12}\}  \}$
\begin{table}[htb]
\centering
\tiny
\begin{tabular}{ccc} 
\toprule
		&$u_1$&$u_2$  \\ \hline 
$\mathcal{P}^1 $ & $\{i_1,i_2,i_3\}$ & $\{i_2,i_3,i_5\} $ \\
$\mathcal{P}^2$ & $\{i_1,i_2,i_3\}$ & $\{i_2,i_5,i_6\} $ \\
$\mathcal{P}^3$ & $\{i_7,i_8,i_9\}$ & $\{i_{10},i_{11},i_{12} \}$ \\
\bottomrule
%$ f_i$    &3  &3  &1  &3  &1  &2  \\  \hline
\end{tabular}
\caption{Top-$\size$ allocations to users.} \label{tab:paretoExamples}
\end{table}
\fi


\iffalse
When considering long-tail items, it is important to consider consumers' willingness  to explore niche or unpopular items and their propensity towards similar items. In particular, they can be characterized by their  {\it risk degree\/} and {\it focusing degree\/}, respectively.  We compute these estimates  based on historical rating information. The following example further describes these notions in the context of movie rating data. 

\begin{example}  
Table~\ref{tab:example} shows rating data for a simplified system with $6$ users, $6$ movies, and $3$ genres. $m_i^{j}$ implies that movie $m_i$ belongs to genre $j$. Note the user-item interaction matrix is sparse. 
  For this setting, we define popular movies as those that have received  three or more ratings; $\{m_1, m_2, m_4\}$ are popular and  $\{m_3, m_5, m_6\}$ are niche movies. We now profile the users according to their risk and focusing degree. E.g., $u_1$ has rated relatively popular movies belonging to the same genre (risk-averse, high focusing degree); $u_2$ has rated niches movies in the same genre (risk-loving, high focusing degree); $u_3$ has rated popular movies in two different genres (risk-averse, low focusing degree), and $u_4$ has rated niches movies in two different genres (risk-loving, low focusing degree). 
\label{ex:running}
\end{example}
\begin{table}[htb]
\centering
\tiny
\begin{tabular}{ccccccc} 
\toprule
			&$m_1^{1}$ &$m_2^{1}$   &$m_3^{2}$    &$m_4^{3}$   &$m_5^{3}$ &$m_6^{3}$  \\ \hline 
$u_1 $ &5  &4  &-  &-  &-  &-   \\
$u_2$  &-  &-  &-  &-  &5  &5   \\
$u_3$  &-  &4  &-  &5  &-  &-   \\
$u_4$  &-  &-  &3  &-  &-  &4   \\ 
$u_5$  &5  &-  &-  &3  &-  &-   \\ 
$u_6$  &4  &2  &-  &4  &-  &-   \\ 
\bottomrule
%$ f_i$    &3  &3  &1  &3  &1  &2  \\  \hline
\end{tabular}
\caption{User-Movie rating data} \label{tab:example}
\end{table}
It is essential to consider these consumer characteristics in designing recommender systems so that they promote long-tail items to the right group of users and spread demand evenly between the hit and niche items.  
\fi
\iffalse
\begin{center}
\begin{figure*}[tp]
%\scalebox{0.5}{%
\resizebox{1\textwidth}{!}{%
%\small%\addtolength{\tabcolsep}{5pt}% below sums to 8
\begin{tabularx}{1.5\textwidth}{>{\hsize=2.5\hsize}X>{\hsize=2.5\hsize}X>{\hsize=0.5\hsize}X>{\hsize=0.5\hsize}X>{\hsize=0.5\hsize}X>{\hsize=0.5\hsize}X>{\hsize=0.5\hsize}X>{\hsize=0.5\hsize}X}
    \multirow{12}{*}{\includegraphics[scale=0.3]{codeForExample/popularity-movie.png}} & \multirow{12}{*}{\includegraphics[scale=0.3]{codeForExample/scatterplot.png}} & & & & & & \\
%   & &               &       &       &       &       &       \\
    & &\multicolumn{1}{l|}{}               &$m_1^{g1}$   	&$m_2^{g1}$    	&$m_3^{g2}$    &$m_4^{g2}$      &$m_5^{g3}$    \\ \cline{3-8}%\hline
    & &\multicolumn{1}{l|}{u1}          &5  &5  &-  &-   &-  \\
    & &\multicolumn{1}{l|}{u2}    		&-  &-  &4  &4  &5  \\
    & &\multicolumn{1}{l|}{u3}   			&1  &2  &1  &-  &-   \\
    & &\multicolumn{1}{l|}{u4}     		&1  &-  &-  &-  &-  \\
    & &               &       &       &       &       &       \\
    & &               &       &       &       &       &       \\
    & &               &       &       &       &       &       \\
    & &               &       &       &       &       &	\\
    \\
\end{tabularx}}
\caption{User-Movie interaction data a) Popularity-Movie histogram b)Movie genres/clusters c) User-Movie rating data} \label{fig:example}
\end{figure*}
\end{center}
\fi



%We propose a novel approach that allows us to  promote long-tail items in a targeted manner, thereby improving the novelty of top-$\size$ sets, the overall item-space coverage across recommendations, while maintaining reasonable levels of accuracy.

%Next, we integrate these learned preferences  in a generic  top-$\size$ recommendation framework to provide customized balance between accuracy and coverage.

%sequentially make recommendations, while adjusting its parameters with regard to the set of top-$\size$ recommendations made so far. However, since  sequential parameter updates  cause  scalability issues, we propose a sampling based algorithm. This variant of our framework, called {\it Ordered Sampling-based Locally Greedy (OSLG)\/},  allows us to  correct for the popularity bias in recommendations with regard to individual user long-tail preferences. 

%ICDE submission
%Our framework differs with  prior work in the following aspects:  unlike~\cite{adomavicius2011maximizing,adomavicius2012improving,zhang2013personalize,ho2014likes},  the long-tail preference personalization in our framework is learned rather than optimized using cross-validation or parameter tuning. In other words, our personalization method is independent of the underlying base  recommendation models.  Moreover, our framework is  generic. This enables us to  plug-in several base recommenders, and evaluate their  effectiveness without requiring  extensive tuning for the accuracy and coverage trade-off. 


%\vspace{-2.8pt}
\begin{itemize}

\item  We examine various measures for estimating user long-tail novelty preference in Section~\ref{sec:lt-pref} and formulate an optimization problem  to directly learn users' preferences for long-tail  items from interaction data in Section~\ref{sec:learning-lt-pref}. %In addition, we introduce several heuristics for measuring the user preference for less common items from historical rating data.% 

\item  We integrate the user preference estimates into GANC %, a generic re-ranking framework that provides customized balance between accuracy, novelty, and coverage 
(Section~\ref{sec:RiskbasedReranking}), and  introduce {\it Ordered Sampling-based Locally Greedy (OSLG)\/}, a scalable algorithm that relies  on user long-tail preferences to correct the popularity bias (Section~\ref{sec:optimizationAlgorithm}).
%We introduce OSLG, a scalable algorithm that relies  on user long-tail preferences to  maximize item space coverage \textcolor{red}{while maintaining acceptable levels of accuracy} (Section~\ref{sec:optimizationAlgorithm}).

\item   We conduct an extensive empirical study and evaluate performance from  accuracy, novelty, and coverage perspectives (Section~\ref{sec:Experiments}).  We use five  datasets with varying density and difficulty levels. %:  Netflix, MovieTweetings, and MovieLens (100K, 1M, 10M). 
  In contrast to most related work,  our evaluation considers realistic settings that include a large number of infrequent  items and users. %This enables us to study the impact of  data density on the performance trade-offs of several  state of the art top-$\size$ recommendation algorithms. %   %,  and use the all-items ranking protocol~\cite{steck2013evaluation,vargas2014improving}, where performance is measured using all items with train data. to evaluate the performance of several  state of the art top-$\size$ recommendation algorithms 
 
\item Our empirical results confirm that the performance of re-ranking models is impacted by the underlying   base recommender and the dataset density. Our generic approach enables us to easily incorporate a suitable base recommender to devise an effective solution for both dense and sparse settings. In dense settings, we use the same base recommender as existing re-ranking approaches, and we outperform them in accuracy and coverage metrics. For sparse settings, we plug-in a more suitable base recommender, and devise an effective solution that is competitive with existing top-$\size$ recommendation methods in accuracy and novelty. 

%Directly estimating the long-tail novelty preferences allows us to customize re-ranking per user, and  devise a generic framework.   
 
\end{itemize}

Section~\ref{sec:related-work} describes related work. Section~\ref{sec:conclusion} concludes.


%\section{Reformulation Suitable for Bulk Asymptotics}
%\input{ReformulationForBulk}

\section{Far-Field Asymptotic Behavior in the Domain $\channels$}
\label{sec:channels}
In this section we prove Theorem~\ref{thm:channels}.  
Our analysis is driven by the sign chart of $\mathrm{Re}(\ii\vartheta(\lambda;\chi,\tau))$ in the $\lambda$-plane.  The function $\lambda\mapsto\mathrm{Re}(\ii\vartheta(\lambda;\chi,\tau))$ is odd with respect to Schwarz reflection $\lambda\mapsto \lambda^*$.  It follows that the whole real $\lambda$-axis is a component of the zero level curve $\mathrm{Re}(\ii\vartheta(\lambda;\chi,\tau))=0$, so all three critical points lie on the zero level.  Since each critical point is simple when $(\chi,\tau)\in\channels$, from each of them a unique  arc of the zero level curve emanates locally into the upper half-plane with a vertical tangent.  From \eqref{eq:vartheta} one sees easily that as $\chi>0$ holds in $\channels$, $\mathrm{Re}(\ii\vartheta(\lambda;\chi,0))$ is negative (resp., positive) for sufficiently large $\lambda$ in the upper (resp., lower) half-plane; on the other hand $\mathrm{Re}(\ii\vartheta(\lambda;\chi,\tau))$ is always positive (resp., negative) near $\lambda=\ii$ (resp., near $\lambda=-\ii$).  From this and the fact that $\mathrm{Re}(\ii\vartheta(\lambda;\chi,\tau))$ is harmonic away from $\lambda=\pm\ii$ it follows that when $\tau=0$ the two arcs of the zero level curve emanating into the upper half-plane from $\lambda=a(\chi,0)$ and $\lambda=b(\chi,0)$ actually coincide and close around the singularity at $\lambda=\ii$.  This structure persists under perturbation for $\tau\neq 0$, as the arc of the zero level curve emanating into the upper half-plane from the newly-born large critical point must tend to $\lambda=\infty$ vertically without intersecting the arc we denote by $\Gamma^+=\Gamma^+(\chi,\tau)$ joining $a(\chi,\tau)$ and $b(\chi,\tau)$ in the upper half-plane. Therefore, for all $(\chi,\tau)\in \channels$, the zero level curve of $\lambda\mapsto\mathrm{Re}(\ii\vartheta(\lambda;\chi,\tau))$ is the disjoint union $\mathbb{R}\sqcup\Gamma^+\sqcup\Gamma^-\sqcup \ell^+\sqcup \ell^-$, where $\ell^+$ denotes the unbounded arc in the upper half-plane emanating from the third critical point that is large when $\tau\neq 0$ is small (we take $\ell^+=\emptyset$ when $\tau=0$) and $\Gamma^-$ and $\ell^-$ are the Schwarz reflections of $\Gamma^+$ and $\ell^+$ respectively.

\subsection{Steepest descent deformation of the Riemann-Hilbert problem}
Since $\overline{\Gamma^+\cup\Gamma^-}$ is a simple closed curve with the points $\lambda=\pm\ii$ in its interior, we use this curve as $\Sigma_\circ$ in the formulation of Riemann-Hilbert Problem~\ref{rhp:rogue-wave-reformulation}.  As $\Sigma_\circ$ has clockwise orientation, we assume that $\Gamma^+$ is oriented from $a$ to $b$ while $\Gamma^-$ is oriented from $b$ to $a$ in the lower half-plane. In the jump condition \eqref{eq:S-jump} for the matrix $\mathbf{S}(\lambda;\chi,\tau,\mathbf{Q}^{-s},M)$ equivalent to $\mathbf{P}(\lambda;x,t,\mathbf{Q}^{-s},M)$ by \eqref{eq:S-from-P}, we factor the matrix $\mathbf{Q}^{-s}$, $s=\pm 1$, as
\begin{equation}
\mathbf{Q}^{-s}=\begin{cases}
2^{\frac{1}{2}\sigma_3}\begin{bmatrix}1 & \tfrac{1}{2}s\\0 & 1\end{bmatrix}\begin{bmatrix}1 & 0\\-s & 1\end{bmatrix},\quad& \lambda\in\Gamma^+,\\
2^{-\frac{1}{2}\sigma_3}\begin{bmatrix}1 & 0\\-\tfrac{1}{2}s & 1\end{bmatrix}\begin{bmatrix}1 & s\\0 & 1\end{bmatrix},\quad& \lambda\in\Gamma^-.
\end{cases}
\label{eq:Q-factorizations}
\end{equation}
Based on these two factorizations, we define a new unknown $\mathbf{W}(\lambda)=\mathbf{W}(\lambda;\chi,\tau,\mathbf{Q}^{-s},M)$ related to $\mathbf{S}(\lambda;\chi,\tau,\mathbf{Q}^{-s},M)$ by first introducing ``lens'' domains $L^\pm$ and $R^\pm$ to the left and right respectively of $\Gamma^\pm$ (so thin as to exclude the points $\pm\ii$ and to support a fixed sign of $\mathrm{Re}(\ii\vartheta(\lambda;\chi,\tau))$) and we let $\Omega^\pm$ denote the domain between $R^\pm$ and the real line.  See Figure~\ref{fig:Channels1}, left panel.  
\begin{figure}[h]
\begin{center}
\includegraphics{Channels1.pdf}
\end{center}
\caption{Left:  for $(\chi,\tau)=(1,0.145)\in \channels$, the regions in the $\lambda$-plane where $\mathrm{Re}(\ii\vartheta(\lambda;\chi,\tau))<0$ (shaded) and $\mathrm{Re}(\ii\vartheta(\lambda;\chi,\tau))>0$ (unshaded), and the curve $\Sigma_\circ=\Gamma^+\cup\Gamma^-$.  The jump contour $\Sigma_\mathrm{c}$ for $\vartheta(\lambda;\chi,\tau)$ is indicated with a red dashed line terminating at the endpoints $\lambda=\pm\ii$.  Critical points of $\vartheta(\lambda;\chi,\tau)$ are shown with black dots.  Also shown are the ``lens'' regions $L^\pm$ and $R^\pm$ lying to the left and right respectively of $\Gamma^\pm$, and the domains $\Omega^\pm$ lying between $R^\pm$ and the real axis and containing the points $\lambda=\pm\ii$.  Right:  the jump contour for $\mathbf{W}(\lambda)$.}
\label{fig:Channels1}
\end{figure}
Then, we define $\mathbf{W}(\lambda)$ by 
%\textcolor{red}{(later, replace $\mathbf{T}$ in this section by $\mathbf{W}$ to match the notation from other sections)}
%\begin{equation}
%\mathbf{T}^{(k)}(\lambda;\chi,\tau):=\mathbf{S}^{(k)}(\lambda;\chi,\tau)\begin{bmatrix}1 & 0\\
%s\omega(\lambda)^{-2s}\ee^{2\ii n\vartheta(\lambda;\chi,\tau)} & 1\end{bmatrix},\quad\lambda\in L^+,
%\label{eq:T-S-L-plus}
%\end{equation}
%\begin{equation}
%\mathbf{T}^{(k)}(\lambda;\chi,\tau):=\mathbf{S}^{(k)}(\lambda;\chi,\tau)2^{\frac{1}{2}\sigma_3}\begin{bmatrix}1 & \tfrac{1}{2}s\omega(\lambda)^{2s}\ee^{-2\ii n\vartheta(\lambda;\chi,\tau)}\\ 0 & 1\end{bmatrix},\quad\lambda\in R^+,
%\label{eq:T-S-R-plus}
%\end{equation}
%\textcolor{red}{The alternate versions of these read
\begin{equation}
\mathbf{W}(\lambda)\defeq\mathbf{S}(\lambda;\chi,\tau,\mathbf{Q}^{-s},M)\begin{bmatrix}1 & 0\\
s\ee^{2\ii M\vartheta(\lambda;\chi,\tau)} & 1\end{bmatrix},\quad\lambda\in L^+,
\label{eq:T-S-L-plus-ALT}
\end{equation}
\begin{equation}
\mathbf{W}(\lambda)\defeq\mathbf{S}(\lambda;\chi,\tau,\mathbf{Q}^{-s},M)2^{\frac{1}{2}\sigma_3}\begin{bmatrix}1 & \tfrac{1}{2}s\ee^{-2\ii M\vartheta(\lambda;\chi,\tau)}\\ 0 & 1\end{bmatrix},\quad\lambda\in R^+,
\label{eq:T-S-R-plus-ALT}
\end{equation}
%}
\begin{equation}
\mathbf{W}(\lambda)\defeq\mathbf{S}(\lambda;\chi,\tau,\mathbf{Q}^{-s},M)2^{\frac{1}{2}\sigma_3},\quad
\lambda\in\Omega^+,
\end{equation}
\begin{equation}
\mathbf{W}(\lambda)\defeq\mathbf{S}(\lambda;\chi,\tau,\mathbf{Q}^{-s},M)2^{-\frac{1}{2}\sigma_3},\quad
\lambda\in\Omega^-,
\end{equation}
%\begin{equation}
%\mathbf{T}^{(k)}(\lambda;\chi,\tau):=\mathbf{S}^{(k)}(\lambda;\chi,\tau)2^{-\frac{1}{2}\sigma_3}\begin{bmatrix} 1 & 0\\-\tfrac{1}{2}s\omega(\lambda)^{-2s}\ee^{2\ii n\vartheta(\lambda;\chi,\tau)} & 1\end{bmatrix},\quad\lambda\in R^-,
%\label{eq:T-S-R-minus}
%\end{equation}
%\begin{equation}
%\mathbf{T}^{(k)}(\lambda;\chi,\tau):=\mathbf{S}^{(k)}(\lambda;\chi,\tau)\begin{bmatrix}
%1 & -s\omega(\lambda)^{2s}\ee^{-2\ii n\vartheta(\lambda;\chi,\tau)}\\ 0 & 1\end{bmatrix},\quad
%\lambda\in L^-,
%\label{eq:T-S-L-minus}
%\end{equation}
%\textcolor{red}{The alternate versions of these read
\begin{equation}
\mathbf{W}(\lambda)\defeq\mathbf{S}(\lambda;\chi,\tau,\mathbf{Q}^{-s},M)2^{-\frac{1}{2}\sigma_3}\begin{bmatrix} 1 & 0\\-\tfrac{1}{2}s\ee^{2\ii M\vartheta(\lambda;\chi,\tau)} & 1\end{bmatrix},\quad\lambda\in R^-,\quad\text{and}
\label{eq:T-S-R-minus-ALT}
\end{equation}
\begin{equation}
\mathbf{W}(\lambda)\defeq\mathbf{S}(\lambda;\chi,\tau,\mathbf{Q}^{-s},M)\begin{bmatrix}
1 & -s\ee^{-2\ii M\vartheta(\lambda;\chi,\tau)}\\ 0 & 1\end{bmatrix},\quad
\lambda\in L^-,
\label{eq:T-S-L-minus-ALT}
\end{equation}
%}
and we take $\mathbf{W}(\lambda)\defeq\mathbf{S}(\lambda;\chi,\tau,\mathbf{Q}^{-s},M)$ whenever $\lambda\in \mathbb{C}\setminus\overline{L^+\cup R^+\cup\Omega^+\cup\Omega^-\cup R^-\cup L^-}$.  Then it is easy to check that $\mathbf{W}(\lambda)$ may be defined for $\lambda\in\Gamma^+\cup\Gamma^-$ to be analytic there, so that $\mathbf{W}(\lambda)$ is analytic in the complement of the jump contour $C_L^+\cup C_R^+\cup I\cup C_R^-\cup C_L^-$ shown in Figure~\ref{fig:Channels1}, right panel.  The jump conditions satisfied by $\mathbf{W}(\lambda)$ on the arcs of this jump contour are then:
%\begin{equation}
%\mathbf{T}^{(k)}_+(\lambda;\chi,\tau)=\mathbf{T}^{(k)}_-(\lambda;\chi,\tau)\begin{bmatrix}1 & 0\\
%-s\omega(\lambda)^{-2s}\ee^{2\ii n\vartheta(\lambda;\chi,\tau)} & 1\end{bmatrix},\quad\lambda\in C_L^+,
%\label{eq:Tjump-channels-CLplus}
%\end{equation}
%\begin{equation}
%\mathbf{T}^{(k)}_+(\lambda;\chi,\tau)=\mathbf{T}^{(k)}_-(\lambda;\chi,\tau)\begin{bmatrix}1 & \tfrac{1}{2}s\omega(\lambda)^{2s}\ee^{-2\ii n\vartheta(\lambda;\chi,\tau)} \\ 0 & 1\end{bmatrix},\quad\lambda\in C_R^+,
%\label{eq:Tjump-channels-CRplus}
%\end{equation}
%\textcolor{red}{The alternate versions of these read
\begin{equation}
\mathbf{W}_+(\lambda)=\mathbf{W}_-(\lambda)\begin{bmatrix}1 & 0\\
-s\ee^{2\ii M\vartheta(\lambda;\chi,\tau)} & 1\end{bmatrix},\quad\lambda\in C_L^+,
\label{eq:Tjump-channels-CLplus-ALT}
\end{equation}
\begin{equation}
\mathbf{W}_+(\lambda)=\mathbf{W}_-(\lambda)\begin{bmatrix}1 & \tfrac{1}{2}s\ee^{-2\ii M\vartheta(\lambda;\chi,\tau)} \\ 0 & 1\end{bmatrix},\quad\lambda\in C_R^+,
\label{eq:Tjump-channels-CRplus-ALT}
\end{equation}
%}
\begin{equation}
\mathbf{W}_+(\lambda)=\mathbf{W}_-(\lambda)2^{\sigma_3},\quad\lambda\in I,
\label{eq:Channels-T-I-jump}
\end{equation}
%\begin{equation}
%\mathbf{T}^{(k)}_+(\lambda;\chi,\tau)=\mathbf{T}^{(k)}_-(\lambda;\chi,\tau)\begin{bmatrix}1 & 0\\
%-\tfrac{1}{2}s\omega(\lambda)^{-2s}\ee^{2\ii n\vartheta(\lambda;\chi,\tau)} & 1\end{bmatrix},\quad
%\lambda\in C_R^-,\quad\text{and}
%\label{eq:Tjump-channels-CRminus}
%\end{equation}
%\begin{equation}
%\mathbf{T}^{(k)}_+(\lambda;\chi,\tau)=\mathbf{T}^{(k)}_-(\lambda;\chi,\tau)\begin{bmatrix} 1 & s\omega(\lambda)^{2s}\ee^{-2\ii n\vartheta(\lambda;\chi,\tau)}\\ 0 & 1\end{bmatrix},\quad\lambda\in C_L^-.
%\label{eq:Tjump-channels-CLminus}
%\end{equation}
%\textcolor{red}{The alternate versions of these read
\begin{equation}
\mathbf{W}_+(\lambda)=\mathbf{W}_-(\lambda)\begin{bmatrix}1 & 0\\
-\tfrac{1}{2}s\ee^{2\ii M\vartheta(\lambda;\chi,\tau)} & 1\end{bmatrix},\quad
\lambda\in C_R^-,\quad\text{and}
\label{eq:Tjump-channels-CRminus-ALT}
\end{equation}
\begin{equation}
\mathbf{W}_+(\lambda)=\mathbf{W}_-(\lambda)\begin{bmatrix} 1 & s\ee^{-2\ii M\vartheta(\lambda;\chi,\tau)}\\ 0 & 1\end{bmatrix},\quad\lambda\in C_L^-.
\label{eq:Tjump-channels-CLminus-ALT}
\end{equation}
%}
It follows from the sign chart of $\mathrm{Re}(\ii\vartheta(\lambda;\chi,\tau))$ as shown in Figure~\ref{fig:Channels1} that as $n\to+\infty$, the jump matrix for $\mathbf{W}(\lambda)$ is an exponentially small perturbation of the identity everywhere on the jump contour except on the interval $I=[a,b]$ and in neighborhoods of its endpoints.  

\subsection{Parametrix construction}
\label{sec:channels-parametrix}
To deal with those jump matrices that are not near-identity, we first construct an \emph{outer parametrix}
$\dot{\mathbf{W}}^{\mathrm{out}}(\lambda)$ by setting
\begin{equation}
\dot{\mathbf{W}}^{\mathrm{out}}(\lambda)\defeq\left(\frac{\lambda-b(\chi,\tau)}{\lambda-a(\chi,\tau)}\right)^{-\ii p\sigma_3},\quad p\defeq\frac{\ln(2)}{2\pi}>0,\quad\lambda\in\mathbb{C}\setminus I.
\label{eq:Channels-Tout}
\end{equation}
Here, the power function is the principal branch, making $\dot{\mathbf{W}}^\mathrm{out}(\lambda)$ analytic in the indicated domain.  Furthermore it is clear that $\dot{\mathbf{W}}_+^\mathrm{out}(\lambda)=\dot{\mathbf{W}}_-^\mathrm{out}(\lambda)2^{\sigma_3}$ holds for $\lambda\in I$, so the jump condition in \eqref{eq:Channels-T-I-jump} is satisfied exactly by the outer parametrix, which also tends to the identity as $\lambda\to\infty$.  However, $\dot{\mathbf{W}}^\mathrm{out}(\lambda)$ is discontinuous near the endpoints of $I$, making the outer parametrix a poor model for $\mathbf{W}(\lambda)$ near these points.  

We can construct \emph{inner parametrices} near $\lambda=a,b$ that locally satisfy the jump conditions for $\mathbf{W}(\lambda)$ exactly.  Let $D_a(\delta)$ and $D_b(\delta)$ be disks of radius $\delta$ centered at $\lambda=a,b$ respectively, where $\delta>0$ is sufficiently small but independent of $n$.  We first define conformal coordinates $f_a(\lambda;\chi,\tau)$ and $f_b(\lambda;\chi,\tau)$ in these disks by setting 
\begin{equation}
f_a(\lambda;\chi,\tau)^2=2[\vartheta_a(\chi,\tau)-\vartheta(\lambda;\chi,\tau)]\quad\text{and}\quad
f_b(\lambda;\chi,\tau)^2=2[\vartheta(\lambda;\chi,\tau)-\vartheta_b(\chi,\tau)],
\end{equation}
where $\vartheta_a(\chi,\tau)\defeq\vartheta(a(\chi,\tau);\chi,\tau)$ and $\vartheta_b(\chi,\tau)\defeq\vartheta(b(\chi,\tau);\chi,\tau)$, 
and then taking analytic square roots in each case so that the inequalities $f_a'(a(\chi,\tau);\chi,\tau)<0$ and $f_b'(b(\chi,\tau);\chi,\tau)>0$ both hold.  This is possible because $a$ and $b$ are simple critical points of $\vartheta(\lambda;\chi,\tau)$, with $\vartheta''_a(\chi,\tau)\defeq\vartheta''(a(\chi,\tau);\chi,\tau)<0$ and $\vartheta''_b(\chi,\tau)\defeq\vartheta''(b(\chi,\tau);\chi,\tau)>0$.  In fact, one has the formul\ae\
\begin{equation}
f_a'(a(\chi,\tau);\chi,\tau)=-\sqrt{-\vartheta''_a(\chi,\tau)}\quad\text{and}\quad
f_b'(b(\chi,\tau);\chi,\tau)=\sqrt{\vartheta''_b(\chi,\tau)}.
\label{eq:Channels-fafb-Derivs}
\end{equation}
Next, define $M$-independent holomorphic matrix valued functions in $D_a(\delta)$ and $D_b(\delta)$ by
\begin{equation}
\mathbf{H}^a(\lambda)\defeq\left(\frac{f_a(\lambda;\chi,\tau)}{a(\chi,\tau)-\lambda}\right)^{-\ii p\sigma_3}(b(\chi,\tau)-\lambda)^{-\ii p\sigma_3}(\ii\sigma_2),\quad\lambda\in D_a(\delta)
\label{eq:Channels-Ha}
\end{equation}
and
\begin{equation}
\mathbf{H}^b(\lambda)\defeq\left(\frac{f_b(\lambda;\chi,\tau)}{\lambda-b(\chi,\tau)}\right)^{\ii p\sigma_3}(\lambda-a(\chi,\tau))^{\ii p\sigma_3},\quad\lambda\in D_b(\delta).  
\label{eq:Channels-Hb}
\end{equation}
Note that in both cases, the diagonal prefactor is an analytic function nonvanishing in the relevant disk for $\delta$ sufficiently small.  In particular,
\begin{equation}
\begin{split}
\mathbf{H}^a(a(\chi,\tau))&=\left(-f_a'(a(\chi,\tau);\chi,\tau)\right)^{-\ii p\sigma_3}(b(\chi,\tau)-a(\chi,\tau))^{-\ii p\sigma_3}(\ii\sigma_2)\\
&=\left(-\vartheta_a''(\chi,\tau)\right)^{-\frac{1}{2}\ii p\sigma_3}(b(\chi,\tau)-a(\chi,\tau))^{-\ii p\sigma_3}(\ii\sigma_2)
\end{split}
\label{eq:Channels-Ha-center}
\end{equation}
and
\begin{equation}
\begin{split}
\mathbf{H}^b(b(\chi,\tau))&=f_b'(b(\chi,\tau);\chi,\tau)^{\ii p\sigma_3}(b(\chi,\tau)-a(\chi,\tau))^{\ii p\sigma_3}\\
&=\vartheta''_b(\chi,\tau)^{\frac{1}{2}\ii p\sigma_3}(b(\chi,\tau)-a(\chi,\tau))^{\ii p\sigma_3},
\end{split}
\label{eq:Channels-Hb-center}
\end{equation}
where on the second line in each case we used \eqref{eq:Channels-fafb-Derivs}.
Letting 
%$\zeta_{a,b}=n^\frac{1}{2}f_{a,b}(\lambda;\chi,\tau)$ 
%\textcolor{red}{(alternately 
$\zeta_{a,b}=M^{\frac{1}{2}}f_{a,b}(\lambda;\chi,\tau)$
%)}
denote rescalings of the conformal coordinates, 
we then define the inner parametrices by setting
%\begin{multline}
%\dot{\mathbf{T}}^a(\lambda;\chi,\tau):=n^{-\frac{1}{2}\ii p\sigma_3}\ee^{-\ii n\vartheta_a(\chi,\tau)\sigma_3}\omega(\lambda)^{s\sigma_3}\ii^{\frac{1}{2}(1-s)\sigma_3}\mathbf{H}^a(\lambda;\chi,\tau)\\
%{}\cdot\mathbf{U}(\zeta_a)(\ii\sigma_2)^{-1}\ii^{-\frac{1}{2}(1-s)\sigma_3}\omega(\lambda)^{-s\sigma_3}\ee^{\ii n\vartheta_a(\chi,\tau)\sigma_3},\quad\lambda\in D_a
%\label{eq:Channels-Ta}
%\end{multline}
%and
%\begin{multline}
%\dot{\mathbf{T}}^b(\lambda;\chi,\tau):=n^{\frac{1}{2}\ii p\sigma_3}\ee^{-\ii n\vartheta_b(\chi,\tau)\sigma_3}\omega(\lambda)^{s\sigma_3}\ii^{\frac{1}{2}(1-s)\sigma_3}\mathbf{H}^b(\lambda;\chi,\tau)\\
%{}\cdot
%\mathbf{U}(\zeta_b)\ii^{-\frac{1}{2}(1-s)\sigma_3}\omega(\lambda)^{-s\sigma_3}\ee^{\ii n\vartheta_b(\chi,\tau)\sigma_3},\quad\lambda\in D_b(\delta).
%\label{eq:Channels-Tb}
%\end{multline}
%\textcolor{red}{The alternate versions of these read
\begin{equation}
\dot{\mathbf{W}}^a(\lambda)\defeq M^{-\frac{1}{2}\ii p\sigma_3}\ee^{-\ii M\vartheta_a(\chi,\tau)\sigma_3}\ii^{\frac{1}{2}(1-s)\sigma_3}\mathbf{H}^a(\lambda)
\mathbf{U}(\zeta_a)(\ii\sigma_2)^{-1}\ii^{-\frac{1}{2}(1-s)\sigma_3}\ee^{\ii M\vartheta_a(\chi,\tau)\sigma_3},\quad\lambda\in D_a(\delta)
\label{eq:Channels-Ta-ALT}
\end{equation}
and
\begin{equation}
\dot{\mathbf{W}}^b(\lambda)\defeq M^{\frac{1}{2}\ii p\sigma_3}\ee^{-\ii M\vartheta_b(\chi,\tau)\sigma_3}\ii^{\frac{1}{2}(1-s)\sigma_3}\mathbf{H}^b(\lambda)
\mathbf{U}(\zeta_b)\ii^{-\frac{1}{2}(1-s)\sigma_3}\ee^{\ii M\vartheta_b(\chi,\tau)\sigma_3},\quad\lambda\in D_b(\delta).
\label{eq:Channels-Tb-ALT}
\end{equation}
%}
Here the factors to the left of $\mathbf{U}(\zeta_{a,b})$ in each case are analytic on the relevant disk and therefore have no effect on the jump conditions, and the matrix function $\mathbf{U}(\zeta)$ is defined in terms of parabolic cylinder functions as the solution of Riemann-Hilbert Problem 5 in \cite{BilmanLM20} (for example; a development of the solution of this problem is given in \cite[Appendix A]{Miller18} taking $\tau=1$ in the notation of that reference).  The main properties of $\mathbf{U}(\zeta)$ that we need to refer to here are
\begin{itemize}
\item $\mathbf{U}(\zeta)$ is analytic for $|\arg(\zeta)|<\tfrac{1}{4}\pi$, $\tfrac{1}{4}\pi<|\arg(\zeta)|<\tfrac{3}{4}\pi$, and $\tfrac{3}{4}\pi<|\arg(\zeta)|<\pi$ (five sectors);
\item $\mathbf{U}(\zeta)$ takes continuous boundary values from each of the five sectors related by jump conditions $\mathbf{U}_+(\zeta)=\mathbf{U}_-(\zeta)\mathbf{V}^\mathrm{PC}(\zeta)$, where $\mathbf{V}^\mathrm{PC}(\zeta)$ is defined in terms of the exponentials $\ee^{\pm\ii\zeta^2}$ on the five complementary oriented boundary rays as shown in \cite[Figure 9]{BilmanLM20};
\item $\mathbf{U}(\zeta)$ has uniform asymptotics in all directions of the complex plane given by
\begin{equation}
\mathbf{U}(\zeta)\zeta^{\ii p\sigma_3}=\mathbb{I}+\frac{1}{2\ii\zeta}\begin{bmatrix}0 & \alpha\\-\beta & 0\end{bmatrix}+\begin{bmatrix}O(\zeta^{-2}) & O(\zeta^{-3})\\O(\zeta^{-3}) & O(\zeta^{-2})\end{bmatrix},\quad\zeta\to\infty,
\label{eq:PCU-asymp}
\end{equation}
where
\begin{equation}
\alpha\defeq\frac{2^\frac{3}{4}\sqrt{2\pi}}{\Gamma(\ii p)}\ee^{\frac{1}{4}\ii\pi}\ee^{2\pi\ii p^2}
= \sqrt{\frac{\ln(2)}{\pi}}\ee^{\ii(\frac{1}{4}\pi+2\pi p^2-\arg(\Gamma(\ii p)))},\quad\beta\defeq -\alpha^*.
\label{eq:Channels-alpha-beta}
\end{equation}
\end{itemize}
In particular, the analyticity and jump conditions satisfied by $\mathbf{U}(\zeta)$ imply that the inner parametrices $\dot{\mathbf{W}}^a(\lambda)$ and $\dot{\mathbf{W}}^b(\lambda)$ exactly satisfy the jump conditions for $\mathbf{W}(\lambda)$ within their respective disks of definition (here we assume that the jump contours for $\mathbf{W}(\lambda)$ within each disk have been deformed to agree with preimages under $\lambda\mapsto \zeta_{a,b}$ of the straight rays across which $\mathbf{U}(\zeta)$ has jump discontinuities).  

A \emph{global parametrix} is then constructed from the outer and inner parametrices as follows:
\begin{equation}
\dot{\mathbf{W}}(\lambda)\defeq\begin{cases}
\dot{\mathbf{W}}^a(\lambda),&\quad\lambda\in D_a(\delta),\\
\dot{\mathbf{W}}^b(\lambda),&\quad\lambda\in D_b(\delta),\\
\dot{\mathbf{W}}^\mathrm{out}(\lambda),&\quad\lambda\in\mathbb{C}\setminus (I\cup D_a(\delta)\cup D_b(\delta)).
\end{cases}
\end{equation}

\subsection{Small norm problem for the error and large-$M$ expansion}
\label{sec:small-norm-channels}
We now compare the (unknown) matrix $\mathbf{W}(\lambda)=\mathbf{W}(\lambda;\chi,\tau,\mathbf{Q}^{-s},M)$ with its global parametrix by defining the \emph{error} as
\begin{equation}
\mathbf{F}(\lambda)\defeq\mathbf{W}(\lambda)\dot{\mathbf{W}}(\lambda)^{-1}.
\end{equation}
Since the parametrix is an exact solution of the Riemann-Hilbert jump conditions for $\mathbf{W}(\lambda)$ within the disks $D_{a,b}(\delta)$ and across the part of $I=[a,b]$ exterior to these disks, 
$\mathbf{F}(\lambda)$ can be extended to an analytic function of $\lambda\in\mathbb{C}$ with the exception of the arcs of $C_L^\pm$ and $C_R^\pm$ lying outside of the disks $D_{a,b}(\delta)$, and the boundaries $\partial D_{a,b}(\delta)$, which we take to have clockwise orientation.  Because $\delta$ is fixed 
%as $n\to+\infty$ 
%\textcolor{red}{(alternately 
as $M\to+\infty$,
%)}, 
and since $\dot{\mathbf{W}}^\mathrm{out}(\lambda)$ is independent of 
%$n$ \textcolor{red}{(alternately 
$M$,
%)}, 
there is a positive constant $\nu>0$ such that 
%$\mathbf{F}^{(k)}_+(\lambda;\chi,\tau)=\mathbf{F}^{(k)}_-(\lambda;\chi,\tau)(\mathbb{I}+O(\ee^{-Kn}))$ \textcolor{red}{(alternately, replace with $O(\ee^{-KM})$)} 
$\mathbf{F}_+(\lambda)=\mathbf{F}_-(\lambda)(\mathbb{I}+O(\ee^{-\nu M}))$ holds uniformly on the jump contour for $\mathbf{F}(\lambda)$ except on the circles $\partial D_{a,b}(\delta)$.  On the circles, we calculate the jump matrix for $\mathbf{F}(\lambda)$ as follows:
\begin{equation}
\mathbf{F}_+(\lambda)=\mathbf{F}_-(\lambda)\cdot\dot{\mathbf{W}}^{a,b}(\lambda)\dot{\mathbf{W}}^\mathrm{out}(\lambda)^{-1},\quad\lambda\in\partial D_{a,b}(\delta),
\end{equation}
because $\mathbf{W}(\lambda)$ is continuous across $\partial D_{a,b}(\delta)$.  Now we use the fact that by comparing the definition \eqref{eq:Channels-Tout} of the outer parametrix 
$\dot{\mathbf{W}}^\mathrm{out}(\lambda)$ with the definitions \eqref{eq:Channels-Ha}--\eqref{eq:Channels-Hb} of $\mathbf{H}^a(\lambda)$ and $\mathbf{H}^b(\lambda)$, we have
%\begin{multline}
%\dot{\mathbf{T}}^\mathrm{out}(\lambda;\chi,\tau)\ee^{-\ii n\vartheta_a(\chi,\tau)\sigma_3}\omega(\lambda)^{s\sigma_3}\ii^{\frac{1}{2}(1-s)\sigma_3}(\ii\sigma_2)\\
%{}=n^{-\frac{1}{2}\ii p\sigma_3}\ee^{-\ii n\vartheta_a(\chi,\tau)\sigma_3}\omega(\lambda)^{s\sigma_3}\ii^{\frac{1}{2}(1-s)\sigma_3}\mathbf{H}^a(\lambda;\chi,\tau)\zeta_a^{-\ii p\sigma_3},\quad\lambda\in D_a(\delta)\setminus I,
%\end{multline}
%and
%\begin{multline}
%\dot{\mathbf{T}}^\mathrm{out}(\lambda;\chi,\tau)\ee^{-\ii n\vartheta_b(\chi,\tau)\sigma_3}\omega(\lambda)^{s\sigma_3}\ii^{\frac{1}{2}(1-s)\sigma_3}\\
%{}=n^{\frac{1}{2}\ii p\sigma_3}\ee^{-\ii n\vartheta_b(\chi,\tau)\sigma_3}\omega(\lambda)^{s\sigma_3}\ii^{\frac{1}{2}(1-s)\sigma_3}\mathbf{H}^b(\lambda;\chi,\tau)\zeta_b^{-\ii p\sigma_3},\quad\lambda\in D_b(\delta)\setminus I.
%\end{multline}
%\textcolor{red}{The alternate versions of these read:
\begin{multline}
\dot{\mathbf{W}}^\mathrm{out}(\lambda)\ee^{-\ii M\vartheta_a(\chi,\tau)\sigma_3}\ii^{\frac{1}{2}(1-s)\sigma_3}(\ii\sigma_2)
=M^{-\frac{1}{2}\ii p\sigma_3}\ee^{-\ii M\vartheta_a(\chi,\tau)\sigma_3}\ii^{\frac{1}{2}(1-s)\sigma_3}\mathbf{H}^a(\lambda)\zeta_a^{-\ii p\sigma_3},\\
\lambda\in D_a(\delta)\setminus I
\end{multline}
and
\begin{equation}
\dot{\mathbf{W}}^\mathrm{out}(\lambda)\ee^{-\ii M\vartheta_b(\chi,\tau)\sigma_3}\ii^{\frac{1}{2}(1-s)\sigma_3}
=M^{\frac{1}{2}\ii p\sigma_3}\ee^{-\ii M\vartheta_b(\chi,\tau)\sigma_3}\ii^{\frac{1}{2}(1-s)\sigma_3}\mathbf{H}^b(\lambda)\zeta_b^{-\ii p\sigma_3},\quad
\lambda\in D_b(\delta)\setminus I.
\end{equation}
%}
Therefore, using \eqref{eq:Channels-Ta-ALT} and \eqref{eq:PCU-asymp} and the fact that 
%$\zeta_a=n^\frac{1}{2}f_a(\lambda;\chi,\tau)$ \textcolor{red}{(alternately $\zeta_a=M^\frac{1}{2}f_a(\lambda;\chi,\tau)$)} 
$\zeta_a=M^\frac{1}{2}f_a(\lambda;\chi,\tau)$
while $f_a(\lambda;\chi,\tau)$ is bounded away from zero on $\partial D_a(\delta)$ for $\delta$ sufficiently small independent of 
%$n$ \textcolor{red}{($M$)},
$M$,
%\begin{multline}
%\mathbf{F}^{(k)}_+(\lambda;\chi,\tau)=\mathbf{F}^{(k)}_-(\lambda;\chi,\tau)
%n^{-\frac{1}{2}\ii p\sigma_3}\ee^{-\ii n\vartheta_a(\chi,\tau)\sigma_3}\omega(\lambda)^{s\sigma_3}\ii^{\frac{1}{2}(1-s)\sigma_3}\mathbf{H}^a(\lambda;\chi,\tau)\\
%\cdot 
%\left(\mathbb{I}+\frac{1}{2\ii n^{\frac{1}{2}}f_a(\lambda;\chi,\tau)}\begin{bmatrix}0 & \alpha\\-\beta & 0\end{bmatrix}+\begin{bmatrix}O(n^{-1}) & O(n^{-\frac{3}{2}})\\O(n^{-\frac{3}{2}}) & O(n^{-1})\end{bmatrix}\right)\\
%\cdot\mathbf{H}^a(\lambda;\chi,\tau)^{-1}\ii^{-\frac{1}{2}(1-s)\sigma_3}\omega(\lambda)^{-s\sigma_3}\ee^{\ii n\vartheta_a(\chi,\tau)\sigma_3}n^{\frac{1}{2}\ii p\sigma_3},\quad\lambda\in\partial D_a(\delta).
%\label{eq:Channels-VF-partialDa}
%\end{multline}
%\textcolor{red}{The alternate version of this reads:
\begin{multline}
\mathbf{F}_+(\lambda)=\mathbf{F}_-(\lambda)
M^{-\frac{1}{2}\ii p\sigma_3}\ee^{-\ii M\vartheta_a(\chi,\tau)\sigma_3}\ii^{\frac{1}{2}(1-s)\sigma_3}\mathbf{H}^a(\lambda)\\
\cdot 
\left(\mathbb{I}+\frac{1}{2\ii M^{\frac{1}{2}}f_a(\lambda;\chi,\tau)}\begin{bmatrix}0 & \alpha\\-\beta & 0\end{bmatrix}+\begin{bmatrix}O(M^{-1}) & O(M^{-\frac{3}{2}})\\O(M^{-\frac{3}{2}}) & O(M^{-1})\end{bmatrix}\right)\\
\cdot\mathbf{H}^a(\lambda)^{-1}\ii^{-\frac{1}{2}(1-s)\sigma_3}\ee^{\ii M\vartheta_a(\chi,\tau)\sigma_3}M^{\frac{1}{2}\ii p\sigma_3},\quad\lambda\in\partial D_a(\delta).
\label{eq:Channels-VF-partialDa-ALT}
\end{multline}
%}
Likewise, using \eqref{eq:Channels-Tb-ALT} and the fact that 
%$\zeta_b=n^\frac{1}{2}f_b(\lambda;\chi,\tau)$ \textcolor{red}{(or 
$\zeta_b=M^\frac{1}{2}f_b(\lambda;\chi,\tau)$
%)} 
with $f_b(\lambda;\chi,\tau)$ bounded away from zero on $\partial D_b(\delta)$, 
%\begin{multline}
%\mathbf{F}^{(k)}_+(\lambda;\chi,\tau)=\mathbf{F}^{(k)}_-(\lambda;\chi,\tau)
%n^{\frac{1}{2}\ii p\sigma_3}\ee^{-\ii n\vartheta_b(\chi,\tau)\sigma_3}\omega(\lambda)^{s\sigma_3}\ii^{\frac{1}{2}(1-s)\sigma_3}\mathbf{H}^b(\lambda;\chi,\tau)\\
%\cdot 
%\left(\mathbb{I}+\frac{1}{2\ii n^{\frac{1}{2}}f_b(\lambda;\chi,\tau)}\begin{bmatrix}0 & \alpha\\-\beta & 0\end{bmatrix}+\begin{bmatrix}O(n^{-1}) & O(n^{-\frac{3}{2}})\\O(n^{-\frac{3}{2}}) & O(n^{-1})\end{bmatrix}\right)\\
%\cdot\mathbf{H}^b(\lambda;\chi,\tau)^{-1}\ii^{-\frac{1}{2}(1-s)\sigma_3}\omega(\lambda)^{-s\sigma_3}\ee^{\ii n\vartheta_b(\chi,\tau)\sigma_3}n^{-\frac{1}{2}\ii p\sigma_3},\quad\lambda\in\partial D_b(\delta).
%\label{eq:Channels-VF-partialDb}
%\end{multline}
%\textcolor{red}{The alternate version of this reads:
\begin{multline}
\mathbf{F}_+(\lambda)=\mathbf{F}_-(\lambda)
M^{\frac{1}{2}\ii p\sigma_3}\ee^{-\ii M\vartheta_b(\chi,\tau)\sigma_3}\ii^{\frac{1}{2}(1-s)\sigma_3}\mathbf{H}^b(\lambda)\\
\cdot 
\left(\mathbb{I}+\frac{1}{2\ii M^{\frac{1}{2}}f_b(\lambda;\chi,\tau)}\begin{bmatrix}0 & \alpha\\-\beta & 0\end{bmatrix}+\begin{bmatrix}O(M^{-1}) & O(M^{-\frac{3}{2}})\\O(M^{-\frac{3}{2}}) & O(M^{-1})\end{bmatrix}\right)\\
\cdot\mathbf{H}^b(\lambda)^{-1}\ii^{-\frac{1}{2}(1-s)\sigma_3}\ee^{\ii M\vartheta_b(\chi,\tau)\sigma_3}M^{-\frac{1}{2}\ii p\sigma_3},\quad\lambda\in\partial D_b(\delta).
\label{eq:Channels-VF-partialDb-ALT}
\end{multline}
%}
In particular, it follows that 
%$\mathbf{F}^{(k)}_+(\lambda;\chi,\tau)=\mathbf{F}^{(k)}_-(\lambda;\chi,\tau)(\mathbb{I}+O(n^{-\frac{1}{2}}))$ \textcolor{red}{(or $O(M^{-\frac{1}{2}})$)} 
$\mathbf{F}_+(\lambda)=\mathbf{F}_-(\lambda)(\mathbb{I}+O(M^{-\frac{1}{2}}))$ 
holds uniformly on the compact jump contour for $\mathbf{F}(\lambda)$, which otherwise is analytic and tends to $\mathbb{I}$ as $\lambda\to\infty$.  By small-norm theory for such Riemann-Hilbert problems, it follows that 
%$\mathbf{F}^{(k)}_-(\lambda;\chi,\tau)=\mathbb{I}+O(n^{-\frac{1}{2}})$ 
%\textcolor{red}{(or $O(M^{-\frac{1}{2}})$)} 
$\mathbf{F}_-(\cdot)=\mathbb{I}+O(M^{-\frac{1}{2}})$ 
holds in the $L^2$ sense on the jump contour, in the limit 
%$n\to+\infty$ \textcolor{red}{(or $M\to+\infty$)}. 
$M\to+\infty$.

\subsection{Asymptotic formula for 
%$\psi_k(n\chi,n\tau)$ \textcolor{red}{(or 
$q(M\chi,M\tau;\mathbf{Q}^{-s},M)$
%)} 
for $(\chi,\tau)\in \channels$}
Beginning with \eqref{eq:q-S} and using the facts that $\mathbf{S}(\lambda;\chi,\tau,\mathbf{Q}^{-s},M)=\mathbf{W}(\lambda)=\mathbf{W}(\lambda;\chi,\tau,\mathbf{Q}^{-s},M)$ and $\dot{\mathbf{W}}(\lambda)=\dot{\mathbf{W}}^\mathrm{out}(\lambda)$ both hold for $|\lambda|$ sufficiently large, we obtain the exact formula
%\begin{equation}
%\begin{split}
%\psi_k(n\chi,n\tau)&=2\ii\ee^{-\ii n\tau}\lim_{\lambda\to\infty}\lambda T_{12}^{(k)}(\lambda;\chi,\tau)\\
%&=
%2\ii\ee^{-\ii n\tau}\lim_{\lambda\to\infty}\lambda\left[F^{(k)}_{11}(\lambda;\chi,\tau)\dot{T}^\mathrm{out}_{12}(\lambda;\chi,\tau)+F^{(k)}_{12}(\lambda;\chi,\tau)\dot{T}^\mathrm{out}_{22}(\lambda;\chi,\tau)\right].
%\end{split}
%\label{eq:psi-k-exact-channels}
%\end{equation}
%\textcolor{red}{The alternate version of this formula reads:
\begin{equation}
\begin{split}
q(M\chi,M\tau;\mathbf{Q}^{-s},M)&=2\ii\lim_{\lambda\to\infty}\lambda W_{12}(\lambda)\\
&=
2\ii\lim_{\lambda\to\infty}\lambda\left[F_{11}(\lambda)\dot{W}^\mathrm{out}_{12}(\lambda)+F_{12}(\lambda)\dot{W}^\mathrm{out}_{22}(\lambda)\right].
\end{split}
\label{eq:psi-k-exact-channels-ALT}
\end{equation}
%}
Since $\dot{\mathbf{W}}^\mathrm{out}(\lambda)$ is a diagonal matrix tending to $\mathbb{I}$ as $\lambda\to\infty$, this formula simplifies to
%\begin{equation}
%\psi_k(n\chi,n\tau)=
%2\ii\ee^{-\ii n\tau}\lim_{\lambda\to\infty}\lambda F^{(k)}_{12}(\lambda;\chi,\tau)
%\end{equation}
%\textcolor{red}{The alternate version of this reads:
\begin{equation}
q(M\chi,M\tau;\mathbf{Q}^{-s},M)=
2\ii\lim_{\lambda\to\infty}\lambda F_{12}(\lambda).
\end{equation}
%}
If $\mathbf{V}^\mathbf{F}(\lambda)$ denotes the jump matrix for $\mathbf{F}(\lambda)$, i.e., $\mathbf{F}_+(\lambda)=\mathbf{F}_-(\lambda)\mathbf{V}^\mathbf{F}(\lambda)$ holds on the jump contour $\Sigma_\mathbf{F}$, then it follows from the Plemelj formula that
\begin{equation}
\mathbf{F}(\lambda)=\mathbb{I}+\frac{1}{2\pi\ii}\int_{\Sigma_\mathbf{F}}\frac{\mathbf{F}_-(\eta)(\mathbf{V}^\mathbf{F}(\eta)-\mathbb{I})}{\eta-\lambda}\,\dd\eta,\quad
\lambda\in\mathbb{C}\setminus\Sigma_\mathbf{F},
\label{eq:F-Cauchy-channels}
\end{equation}
and therefore
%\begin{equation}
%\psi_k(n\chi,n\tau)=-\frac{1}{\pi}\ee^{-\ii n\tau}\int_{\Sigma_\mathbf{F}}\left[F_{11-}^{(k)}(\mu;\chi,\tau)V^\mathbf{F}_{12}(\mu;\chi,\tau)+F_{12-}^{(k)}(\mu;\chi,\tau)(V^\mathbf{F}_{22}(\mu;\chi,\tau)-1)\right]\,\dd\mu.
%\end{equation}
%\textcolor{red}{The alternate version of this formula reads:
\begin{equation}
q(M\chi,M\tau;\mathbf{Q}^{-s},M)=-\frac{1}{\pi}\int_{\Sigma_\mathbf{F}}\left[F_{11-}(\eta)V^\mathbf{F}_{12}(\eta)+F_{12-}(\eta)(V^\mathbf{F}_{22}(\eta)-1)\right]\,\dd\eta.
\end{equation}
%}
Since 
%$V^\mathbf{F}_{22}(\lambda;\chi,\tau)-1=O(n^{-1})$ \textcolor{red}{(or $O(M^{-1})$)} 
$V^\mathbf{F}_{22}(\cdot)-1=O(M^{-1})$
holds uniformly on $\Sigma_\mathbf{F}$, as $\Sigma_\mathbf{F}$ is compact we also have 
%$V^\mathbf{F}_{22}(\lambda;\chi,\tau)-1=O(n^{-1})$ \textcolor{red}{(or $O(M^{-1})$)} 
$V^\mathbf{F}_{22}(\cdot)-1=O(M^{-1})$
in $L^2(\Sigma_\mathbf{F})$.  Using that 
%$F_{12-}^{(k)}(\lambda;\chi,\tau)=O(n^{-\frac{1}{2}})$ \textcolor{red}{(or $O(M^{-\frac{1}{2}})$)} 
$F_{12-}(\cdot)=O(M^{-\frac{1}{2}})$
in $L^2(\Sigma_\mathbf{F})$ as well, by Cauchy-Schwarz,
%\begin{equation}
%\psi_k(n\chi,n\tau)=-\frac{1}{\pi}\ee^{-\ii n\tau}\int_{\Sigma_\mathbf{F}}F_{11-}^{(k)}(\mu;\chi,\tau)V^\mathbf{F}_{12}(\mu;\chi,\tau)\,\dd\mu + O(n^{-\frac{3}{2}}).
%\end{equation}
%\textcolor{red}{The alternate version reads:
\begin{equation}
q(M\chi,M\tau;\mathbf{Q}^{-s},M)=-\frac{1}{\pi}\int_{\Sigma_\mathbf{F}}F_{11-}(\eta)V^\mathbf{F}_{12}(\eta)\,\dd\eta + O(M^{-\frac{3}{2}}).
\end{equation}
%}
A similar argument allows $F_{11-}(\eta)$ to be replaced with $1$ at the cost of an error term of the same order.  Indeed, taking a boundary value on $\Sigma_\mathbf{F}$ in \eqref{eq:F-Cauchy-channels} gives for $\varphi(\lambda)\defeq F_{11-}(\lambda)-1$ the integral equation
\begin{equation}
\varphi(\lambda)-\frac{1}{2\pi\ii}\int_{\Sigma_\mathbf{F}}\frac{\varphi(\eta)(V^\mathbf{F}_{11}(\eta)-1)}{\eta-\lambda_-}\,\dd\eta = f(\lambda),\quad\lambda\in\Sigma_\mathbf{F},
\label{eq:phi-integral-equation}
\end{equation}
where 
\begin{equation}
f(\lambda)\defeq \frac{1}{2\pi\ii}\int_{\Sigma_\mathbf{F}}\frac{V_{11}^\mathbf{F}(\eta)-1}{\eta-\lambda_-}\,\dd\eta + \frac{1}{2\pi\ii}\int_{\Sigma_\mathbf{F}}\frac{F_{12-}(\eta)V_{21}^\mathbf{F}(\eta)}{\eta-\lambda_-}\,\dd\eta,\quad\lambda\in\Sigma_\mathbf{F}.
\label{eq:phi-integral-equation-RHS}
\end{equation}
The small-norm theory is fundamentally based on the fact that the Cauchy projection operator 
\begin{equation}
m(\lambda)\mapsto\frac{1}{2\pi\ii}\int_{\Sigma_\mathbf{F}}\frac{m(\eta)\,\dd\eta}{\eta-\lambda_-},\quad\lambda\in\Sigma_\mathbf{F}
\end{equation}
is bounded on $L^2(\Sigma_\mathbf{F})$ with norm depending only on the geometry of the contour $\Sigma_\mathbf{F}$, which is independent of any large parameter.  Since 
%$V_{11}^\mathbf{F}(\lambda;\chi,\tau)-1=O(n^{-1})$ \textcolor{red}{(or $O(M^{-1})$)} 
$V_{11}^\mathbf{F}(\cdot)-1=O(M^{-1})$
in $L^\infty(\Sigma_\mathbf{F})$ it follows easily from \eqref{eq:phi-integral-equation} that $\varphi(\cdot)=O(f(\cdot))$ in $L^2(\Sigma_\mathbf{F})$ as 
%$n\to\infty$ \textcolor{red}{(or as $M\to\infty$)}.  
$M\to\infty$.
Likewise, from \eqref{eq:phi-integral-equation-RHS} we see that $f(\cdot)=O(V^\mathbf{F}_{11}(\cdot)-1)+O(F_{12-}(\cdot)V_{21}^\mathbf{F}(\cdot))$ in $L^2(\Sigma_\mathbf{F})$.  Since 
%$V^\mathbf{F}_{11}(\lambda;\chi,\tau)-1=O(n^{-1})$ \textcolor{red}{(or $O(M^{-1})$)} 
$V^\mathbf{F}_{11}(\cdot)-1=O(M^{-1})$
in $L^\infty(\Sigma_\mathbf{F})$, compactness of $\Sigma_\mathbf{F}$ implies that 
%$V^\mathbf{F}_{11}(\lambda;\chi,\tau)-1=O(n^{-1})$ \textcolor{red}{(or $O(M^{-1})$)} 
$V^\mathbf{F}_{11}(\cdot)-1=O(M^{-1})$
in $L^2(\Sigma_\mathbf{F})$.  Also, since 
%$V_{21}^\mathbf{F}(\lambda;\chi,\tau)=O(n^{-\frac{1}{2}})$ \textcolor{red}{(or $O(M^{-\frac{1}{2}})$)} 
$V_{21}^\mathbf{F}(\cdot)=O(M^{-\frac{1}{2}})$
in $L^\infty(\Sigma_\mathbf{F})$ while 
%$F^{(k)}_{12-}(\lambda;\chi,\tau)=O(n^{-\frac{1}{2}})$ \textcolor{red}{(or $O(M^{-\frac{1}{2}})$)} 
$F_{12-}(\cdot)=O(M^{-\frac{1}{2}})$
in $L^2(\Sigma_\mathbf{F})$, we consequently have 
%$F_{12-}^{(k)}(\lambda;\chi,\tau)V_{21}^\mathbf{F}(\lambda;\chi,\tau)=O(n^{-1})$ \textcolor{red}{(or $O(M^{-1})$)} 
$F_{12-}(\cdot)V_{21}^\mathbf{F}(\cdot)=O(M^{-1})$
in $L^2(\Sigma_\mathbf{F})$ as well.  Therefore 
%$\varphi(\lambda)=F^{(k)}_{11-}(\lambda;\chi,\tau)-1=O(n^{-1})$ \textcolor{red}{(or $O(M^{-1})$)} 
$\varphi(\cdot)=F_{11-}(\cdot)-1=O(M^{-1})$
in $L^2(\Sigma_\mathbf{F})$.  As 
%$V_{12}^\mathbf{F}(\lambda;\chi,\tau)=O(n^{-\frac{1}{2}})$ \textcolor{red}{(or $O(M^{-\frac{1}{2}})$)} 
$V_{12}^\mathbf{F}(\cdot)=O(M^{-\frac{1}{2}})$
in $L^\infty(\Sigma_\mathbf{F})$ and hence also in $L^2(\Sigma_\mathbf{F})$ it then follows by Cauchy-Schwarz that
%\begin{equation}
%\psi_k(n\chi,n\tau)=-\frac{1}{\pi}\ee^{-\ii n\tau}\int_{\Sigma_\mathbf{F}}V^\mathbf{F}_{12}(\mu;\chi,\tau)\,\dd\mu + O(n^{-\frac{3}{2}}).
%\end{equation}
%\textcolor{red}{The alternate version reads:
\begin{equation}
q(M\chi,M\tau;\mathbf{Q}^{-s},M)=-\frac{1}{\pi}\int_{\Sigma_\mathbf{F}}V^\mathbf{F}_{12}(\eta)\,\dd\eta + O(M^{-\frac{3}{2}}).
\end{equation}
%}
The dominant contribution to the integral comes from $\partial D_a(\delta)\cup\partial D_b(\delta)$ where $V^\mathbf{F}_{12}(\cdot)$ is proportional to 
%$n^{-\frac{1}{2}}$ \textcolor{red}{(or $M^{-\frac{1}{2}}$)}, 
$M^{-\frac{1}{2}}$,
while contributions from the rest of $\Sigma_\mathbf{F}$ are uniformly exponentially small.  Therefore, we may modify the integration contour to consist of just two small circles:
%\begin{equation}
%\psi_k(n\chi,n\tau)=-\frac{1}{\pi}\ee^{-\ii n\tau}\int_{\partial D_a(\delta)\cup\partial D_b(\delta)}V^\mathbf{F}_{12}(\mu;\chi,\tau)\,\dd\mu + O(n^{-\frac{3}{2}}).
%\end{equation}
%\textcolor{red}{The alternate version reads:
\begin{equation}
q(M\chi,M\tau;\mathbf{Q}^{-s},M)=-\frac{1}{\pi}\int_{\partial D_a(\delta)\cup\partial D_b(\delta)}V^\mathbf{F}_{12}(\eta)\,\dd\eta + O(M^{-\frac{3}{2}}).
\end{equation}
%}
Now, using the jump conditions \eqref{eq:Channels-VF-partialDa-ALT}--\eqref{eq:Channels-VF-partialDb-ALT} and the fact that $\mathbf{H}^a(\cdot)$ is off-diagonal while $\mathbf{H}^b(\cdot)$ is diagonal, one easily finds that
%\begin{equation}
%V_{12}^\mathbf{F}(\mu;\chi,\tau)=\frac{n^{-\ii p}\ee^{-2\ii n\vartheta_a(\chi,\tau)}(-1)^{\frac{1}{2}(1-s)}}{2\ii n^\frac{1}{2}f_a(\mu;\chi,\tau)}
%\beta\omega(\mu)^{2s}H^a_{12}(\mu;\chi,\tau)^2 + O(n^{-\frac{3}{2}}),\quad
%\mu\in\partial D_a(\delta),
%\end{equation}
%\begin{equation}
%V_{12}^\mathbf{F}(\mu;\chi,\tau)=\frac{n^{\ii p}\ee^{-2\ii n\vartheta_b(\chi,\tau)}(-1)^{\frac{1}{2}(1-s)}}{2\ii n^\frac{1}{2}f_b(\mu;\chi,\tau)}
%\alpha\omega(\mu)^{2s}H^b_{11}(\mu;\chi,\tau)^2 + O(n^{-\frac{3}{2}}),\quad
%\mu\in\partial D_b(\delta).
%\end{equation}
%\textcolor{red}{The alternate versions of these read:
\begin{equation}
V_{12}^\mathbf{F}(\eta)=\frac{M^{-\ii p}\ee^{-2\ii M\vartheta_a(\chi,\tau)}(-1)^{\frac{1}{2}(1-s)}}{2\ii M^\frac{1}{2}f_a(\eta;\chi,\tau)}
\beta H^a_{12}(\eta)^2 + O(M^{-\frac{3}{2}}),\quad
\eta\in\partial D_a(\delta),
\end{equation}
\begin{equation}
V_{12}^\mathbf{F}(\eta)=\frac{M^{\ii p}\ee^{-2\ii M\vartheta_b(\chi,\tau)}(-1)^{\frac{1}{2}(1-s)}}{2\ii M^\frac{1}{2}f_b(\eta;\chi,\tau)}
\alpha H^b_{11}(\eta)^2 + O(M^{-\frac{3}{2}}),\quad
\eta\in\partial D_b(\delta).
\end{equation}
%}
Therefore, since $f_{a,b}(\cdot;\chi,\tau)$ are analytic functions with simple zeros at $a$ and $b$ respectively, a residue calculation gives
%\begin{multline}
%\psi_k(n\chi,n\tau)=\ee^{-\ii n\tau}\frac{(-1)^{\frac{1}{2}(1-s)}}{n^\frac{1}{2}}\left[n^{-\ii p}\ee^{-2\ii n\vartheta_a(\chi,\tau)}\frac{\beta\omega(a(\chi,\tau))^{2s}H_{12}^a(a(\chi,\tau);\chi,\tau)^2}{f_a'(a(\chi,\tau);\chi,\tau)}\right.\\
%\left. {}+n^{\ii p}\ee^{-2\ii n\vartheta_b(\chi,\tau)}\frac{\alpha\omega(b(\chi,\tau))^{2s}H_{11}^b(b(\chi,\tau);\chi,\tau)^2}{f_b'(b(\chi,\tau);\chi,\tau)}\right]+O(n^{-\frac{3}{2}}).
%\end{multline}
%\textcolor{red}{The alternate version reads:
\begin{multline}
q(M\chi,M\tau;\mathbf{Q}^{-s},M)=\frac{(-1)^{\frac{1}{2}(1-s)}}{M^\frac{1}{2}}\left[M^{-\ii p}\ee^{-2\ii M\vartheta_a(\chi,\tau)}\frac{\beta H_{12}^a(a(\chi,\tau))^2}{f_a'(a(\chi,\tau);\chi,\tau)}\right.\\
\left. {}+M^{\ii p}\ee^{-2\ii M\vartheta_b(\chi,\tau)}\frac{\alpha H_{11}^b(b(\chi,\tau))^2}{f_b'(b(\chi,\tau);\chi,\tau)}\right]+O(M^{-\frac{3}{2}}).
\end{multline}
%}
Since $s=\pm 1$,
%Recalling that $s$ is the parity index of $k$ ($s=1$ for $k$ even and $s=-1$ for $k$ odd), 
we then use \eqref{eq:Channels-fafb-Derivs}, \eqref{eq:Channels-Ha-center}--\eqref{eq:Channels-Hb-center}, and \eqref{eq:Channels-alpha-beta} to obtain
%\begin{multline}
%\psi_k(n\chi,n\tau)=\ee^{-\ii n\tau}\frac{(-1)^k}{n^\frac{1}{2}}\sqrt{\frac{\ln(2)}{\pi}}
%\left[
%\ee^{\ii\phi}
%\frac{\ee^{-2\ii n\vartheta_a(\chi,\tau)}\omega(a(\chi,\tau))^{2s}(-\vartheta''_a(\chi,\tau))^{-\ii p}}{(-\vartheta''_a(\chi,\tau))^\frac{1}{2}}\right.\\
%\left.{}+\ee^{-\ii\phi}
%\frac{\ee^{-2\ii n\vartheta_b(\chi,\tau)}\omega(b(\chi,\tau))^{2s}\vartheta''_b(\chi,\tau)^{\ii p}}{\vartheta''_b(\chi,\tau)^\frac{1}{2}}\right]+O(n^{-\frac{3}{2}}),
%\end{multline}
%where $\phi$ is a real angle defined by
%\begin{equation}
%\phi:=-p\ln(n) - 2p\ln(b(\chi,\tau)-a(\chi,\tau))-2\pi p^2-\tfrac{1}{4}\pi+\arg(\Gamma(\ii p)).
%\end{equation}
%\textcolor{red}{The alternate version of these formul\ae\ reads:
\begin{multline}
q(M\chi,M\tau;\mathbf{Q}^{-s},M)=\frac{s}{M^\frac{1}{2}}\sqrt{\frac{\ln(2)}{\pi}}
\left[
\ee^{\ii\phi}
\frac{\ee^{-2\ii M\vartheta_a(\chi,\tau)}(-\vartheta''_a(\chi,\tau))^{-\ii p}}{(-\vartheta''_a(\chi,\tau))^\frac{1}{2}}\right.\\
\left.{}+\ee^{-\ii\phi}
\frac{\ee^{-2\ii M\vartheta_b(\chi,\tau)}\vartheta''_b(\chi,\tau)^{\ii p}}{\vartheta''_b(\chi,\tau)^\frac{1}{2}}\right]+O(M^{-\frac{3}{2}}),
\end{multline}
where, recalling the value of $p$ from \eqref{eq:Channels-Tout}, a real angle $\phi$ is defined by 
\begin{equation}
\phi\defeq -\frac{\ln(2)}{2\pi}\ln(M)-\frac{\ln(2)}{\pi}\ln(b(\chi,\tau)-a(\chi,\tau))-\frac{\ln(2)^2}{2\pi}-\frac{1}{4}\pi+\arg\left(\Gamma\left(\frac{\ii\ln(2)}{2\pi}\right)\right).
\label{eq:phi-def}
\end{equation}
%\eqref{eq:intro-phi-def}.
%\begin{equation}
%\phi:=-p\ln(M) - 2p\ln(b(\chi,\tau)-a(\chi,\tau))-2\pi p^2-\tfrac{1}{4}\pi+\arg(\Gamma(\ii p)).
%\end{equation}
%}
We may further observe that the numerator of each of the fractions in square brackets above has unit modulus, so upon identifying the angles of those phase factors
%introducing two additional real angles angles by \eqref{eq:intro-Phia-Phib},
%\begin{equation}
%\begin{split}
%\Phi_a&:=-2n\vartheta_a(\chi,\tau)+2s\arg(\omega(a(\chi,\tau)))-p\ln(-\vartheta_a''(\chi,\tau))\\
%\Phi_b&:=-2n\vartheta_b(\chi,\tau)+2s\arg(\omega(b(\chi,\tau)))+p\ln(\vartheta_b''(\chi,\tau))
%\end{split}
%\end{equation}
%and then 
%\begin{equation}
%\psi_k(n\chi,n\tau)=\ee^{-\ii n\tau}\frac{(-1)^k}{n^\frac{1}{2}}\sqrt{\frac{\ln(2)}{\pi}}\left[\frac{\ee^{\ii(\Phi_a+\phi)}}{(-\vartheta''_a(\chi,\tau))^{\frac{1}{2}}}+\frac{\ee^{\ii(\Phi_b-\phi)}}{\vartheta_b''(\chi,\tau)^\frac{1}{2}}\right]+O(n^{-\frac{3}{2}}).
%\label{eq:psi-k-channels}
%\end{equation}
%\textcolor{red}{The alternate versions read:
%\begin{equation}
%\begin{split}
%\Phi_a&:=-2M\vartheta_a(\chi,\tau)-p\ln(-\vartheta_a''(\chi,\tau))\\
%\Phi_b&:=-2M\vartheta_b(\chi,\tau)+p\ln(\vartheta_b''(\chi,\tau))
%\end{split}
%\end{equation}
%and then 
%\begin{equation}
%\psi_k(M\chi,M\tau)=\ee^{-\ii M\tau}\frac{(-1)^k}{M^\frac{1}{2}}\sqrt{\frac{\ln(2)}{\pi}}\left[\frac{\ee^{\ii(\Phi_a+\phi)}}{(-\vartheta''_a(\chi,\tau))^{\frac{1}{2}}}+\frac{\ee^{\ii(\Phi_b-\phi)}}{\vartheta_b''(\chi,\tau)^\frac{1}{2}}\right]+O(M^{-\frac{3}{2}}).
%\label{eq:psi-k-channels-ALT}
%\end{equation}
%}
the proof of Theorem~\ref{thm:channels} is complete, with a standard argument to supply the local uniformity of the error estimate for $(\chi,\tau)$ in compact subsets of $\channels$ (which can include points on the positive $\chi$-axis).

\subsection{Simplification for $\tau=0$}
%The proof of Corollary~\ref{cor:rogue-wave-channels} merely amounts to (i) restricting $M>0$ to values corresponding to rogue waves $M=\tfrac{1}{2}k+\tfrac{1}{4}$ and then tie $s$ to the order $k\in\mathbb{Z}_{>0}$ by $s=(-1)^k$, and (ii) accounting for the exponential factor $\ee^{-\ii M\tau}$ mediating between \eqref{eq:q-S} and \eqref{eq:psi-k-S}.
%
The further simplification mentioned at the end of Section~\ref{sec:results-channels}, so that $(\chi,\tau)\in \channels$ with $\tau=0$ means $0<\chi<2$, is accomplished by noting that the phase function $\vartheta(\lambda;\chi,0)$ defined by \eqref{eq:vartheta} is an odd function of $\lambda$ for each $\chi\in (0,2)$, and we recall that the critical points $\lambda=a,b$ in this case are given by \eqref{eq:tau-zero-critical-points}: 
\begin{equation}
b(\chi,0)=\sqrt{\frac{2}{\chi}-1},\quad a(\chi,0)=-b(\chi,0).
\end{equation}
A computation then shows that
\begin{equation}
\vartheta_b(\chi,0)=\vartheta(b(\chi,0);\chi,0)=\chi\sqrt{\frac{2}{\chi}-1}+\pi-2\tan^{-1}\left(\sqrt{\frac{2}{\chi}-1}\right),\quad\vartheta_a(\chi,0)=-\vartheta_b(\chi,0),
\end{equation}
and that
\begin{equation}
\vartheta_b''(\chi,0)=\vartheta''(b(\chi,0);\chi,0)=\chi^2\sqrt{\frac{2}{\chi}-1},\quad\vartheta_a''(\chi,0)=-\vartheta_b''(\chi,0).
\end{equation}
%Similarly,
%\begin{equation}
%2\arg(\omega(b(\chi,0)))=\tan^{-1}\left(\sqrt{\frac{2}{\chi}-1}\right)-\frac{1}{2}\pi,\quad
%2\arg(\omega(a(\chi,0)))=-2\arg(\omega(b(\chi,0))).
%\end{equation}
Therefore, in this special case, the leading term denoted $L^{[\channels]}_k(\chi,\tau)$ in \eqref{eq:leading-term-channels} reduces for $\tau=0$ and $0<\chi<2$ to \eqref{eq:leading-term-channels-tau-zero}.
%\begin{equation}
%\psi_k(n\chi,0)=\frac{1}{n^\frac{1}{2}}A(\chi)\cos(2n\vartheta_b(\chi,0)-p\ln(n)-\Omega(\chi) + \Phi_0)+O(n^{-\frac{3}{2}}),\quad 0<\chi<2,
%\end{equation}
%where
%\begin{equation}
%A(\chi):=\sqrt{\frac{\ln(2)}{\pi}}\frac{2}{\chi^{\frac{3}{4}}(2-\chi)^\frac{1}{4}},\quad 
%\Omega(\chi):=2p\ln(\chi)+3p\ln(b(\chi,0))+s\tan^{-1}(b(\chi,0))
%\end{equation}
%and where $\Phi_0$ is a constant phase given by
%\begin{equation}
%\Phi_0:=\frac{1}{4}\pi-\frac{3\ln(2)^2}{2\pi}+\arg(\Gamma(\ii p)).
%\end{equation}
%\textcolor{red}{The alternate version of the final asymptotic formula can be written as (but note that the expressions for $2\arg(\omega(\lambda))$ at $\lambda=a(\chi,0)$ and $\lambda=b(\chi,0)$ are not needed):
%\begin{equation}
%\psi_k(M\chi,0)=\frac{1}{M^\frac{1}{2}}A(\chi)\cos(2M\vartheta_b(\chi,0)-p\ln(M)-\Omega(\chi) + \Phi_0)+O(M^{-\frac{3}{2}}),\quad 0<\chi<2,
%\end{equation}
%where
%\begin{equation}
%A(\chi):=\sqrt{\frac{\ln(2)}{\pi}}\frac{2}{\chi^{\frac{3}{4}}(2-\chi)^\frac{1}{4}},\quad 
%\Omega(\chi):=2p\ln(\chi)+3p\ln(b(\chi,0))
%\end{equation}
%and where $\Phi_0$ is a constant phase given by
%\begin{equation}
%\Phi_0:=\left(k-\frac{1}{4}\right)\pi-\frac{3\ln(2)^2}{2\pi}+\arg(\Gamma(\ii p)).
%\end{equation}
%}


\section{Properties of $h(\lambda;\chi,\tau)$ for $(\chi,\tau)\in\exterior\cup\shelves$}
\label{sec:GenusZeroModification}

\subsection{Unique determination of $h'(\lambda;\chi,\tau)$ for $(\chi,\tau)\in\exterior\cup\shelves$}
\label{sec:g-function}
Here we show how $(\chi,\tau)\in\overline{\exterior\cup\shelves}$ determines a unique function $h'(\lambda;\chi,\tau)$ of the form \eqref{eq:hprime-formula} that satisfies the residue and asymptotic conditions \eqref{eq:hprime-residues} and \eqref{eq:hprime-expansion} respectively.  
 
We first use \eqref{eq:hprime-expansion} with \eqref{eq:hprime-formula} to explicitly eliminate $A(\chi,\tau)$ and $B(\chi,\tau)^2$ in favor of $u(\chi,\tau)$ and $v(\chi,\tau)$:
\begin{equation}
\begin{split}
A(\chi,\tau)&=\frac{u(\chi,\tau)-\chi}{2\tau},\\
A(\chi,\tau)^2+B(\chi,\tau)^2&=\frac{3u(\chi,\tau)^2}{4\tau^2}-\frac{v(\chi,\tau)}{\tau}+2-\frac{\chi u(\chi,\tau)}{\tau^2}+\frac{\chi^2}{4\tau^2}.
\end{split}
\label{eq:eliminate-AB}
\end{equation}
Then, the residue conditions \eqref{eq:hprime-residues} become $(-2\tau\pm\ii u(\chi,\tau)+v(\chi,\tau))R(\pm\ii;\chi,\tau)=-2$.  Imposing instead the \emph{squares} of these conditions\footnote{Later, getting the signs right for the residues is accomplished by choosing the location of the branch cut $\Sigma_g$ in relation to the points $\lambda=\pm\ii$.  See Remark~\ref{rem:Sigma_g}.} one arrives at two complex-conjugate equations, which amount to two real equations by taking real and imaginary parts.  The real part equation reads $\mathcal{R}=0$, where
\begin{multline}
\mathcal{R}\defeq 3u^4-3u^2v^2-4\chi u^3+4\tau v^3+4\chi uv^2+\chi^2u^2-\chi^2v^2-8\chi\tau uv+8\tau^2u^2-20\tau^2v^2\\{}+4\chi^2\tau v+32\tau^3v-16\tau^4+16\tau^2-4\chi^2\tau^2,
\end{multline}
and the imaginary part equation reads $\mathcal{I}_1\mathcal{I}_2=0$, where
\begin{equation}
\mathcal{I}_1\defeq u^2-\chi u-2\tau v+4\tau^2\quad\text{and}\quad
\mathcal{I}_2\defeq 3uv-4\tau u-\chi v+2\chi\tau.
\end{equation}
Note that for $\tau=0$,
\begin{equation}
\left.\mathcal{R}\right|_{\tau=0}=(u+v)(u-v)(3u-\chi)(u-\chi),\quad
\left.\mathcal{I}_1\right|_{\tau=0}=(u-\chi)u,\quad\text{and}\quad
\left.\mathcal{I}_2\right|_{\tau=0}=(3u-\chi)v,
\end{equation}
so one solution is to choose $u(\chi,0)=\chi$ and $v(\chi,0)=0$.  In order to apply the implicit function theorem to continue this solution to $\tau\neq 0$, it is necessary to discard the factor $\mathcal{I}_1$ and enforce only the conditions $\mathcal{R}=0$ and $\mathcal{I}_2=0$.  Then a calculation shows that the Jacobian is
\begin{equation}
\left.\det\begin{bmatrix} \mathcal{R}_u & \mathcal{R}_v\\
\mathcal{I}_{2u} & \mathcal{I}_{2v}\end{bmatrix}\right|_{\tau=0,u=\chi,v=0}=4\chi^4,
\end{equation}
which is nonzero for $\chi>2$.  Moreover, the equation $\mathcal{I}_2=0$ can be used to explicitly eliminate $v$ by
\begin{equation}
\mathcal{I}_2=0\quad\Leftrightarrow\quad v=2\tau\frac{2u-\chi}{3u-\chi}.
\label{eq:eliminate-v}
\end{equation}
With $v$ eliminated, the equation $\mathcal{R}=0$ reads $P(u;\chi,\tau)=0$, where $P(u;\chi,\tau)$ is the septic polynomial
\begin{multline}
P(u;\chi,\tau)\defeq 81u^7-189\chi u^6 + (162\chi^2+72\tau^2)u^5-(66\chi^2+120\tau^2)\chi u^4 \\
{}+ (13\chi^4+56\chi^2\tau^2+16\tau^4+432\tau^2)u^3 -(\chi^4+8\chi^2\tau^2+16\tau^4+432\tau^2)\chi u^2 \\
{}+ 144\chi^2\tau^2u-16\chi^3\tau^2.
\end{multline}
Note that $P(u;\chi,0)=(u-\chi)(3u-\chi)^4u^2$, so if $\chi>0$, $u(\chi,0)=\chi$ is a simple root hence continuable to $\tau>0$ (sufficiently small, given $\chi>0$) by the implicit function theorem.  In the limit $\tau\downarrow 0$ we can compute as many terms in the Taylor expansion of $u(\chi,\tau)$ about $u(\chi,0)=\chi$ as we like; in particular it is easy to see that
\begin{equation}
u(\chi,\tau)=\chi -\frac{8\tau^2}{\chi^3} + O(\tau^4),\quad\tau\downarrow 0,\quad \chi>2,
\end{equation}
which implies via \eqref{eq:eliminate-v} that 
\begin{equation}
v(\chi,\tau)=\tau-\frac{4\tau^3}{\chi^4}+O(\tau^5),\quad\tau\downarrow 0,\quad\chi>2.
\end{equation}
From \eqref{eq:eliminate-AB} we then also find that
\begin{equation}
A(\chi,\tau)=-\frac{4\tau}{\chi^3}+O(\tau^3)\quad\text{and}\quad
B(\chi,\tau)^2=1-\frac{4}{\chi^2}+O(\tau^2),\quad\tau\downarrow 0,\quad \chi>2.
\label{eq:AB-tau-small}
\end{equation}
In the special case that $\tau=0$ and $\chi>2$, it follows that $u(\chi,0)=\chi$, $v(\chi,0)=0$, $A(\chi,0)=0$ and $B(\chi,0)^2=1-4/\chi^2<1$.
We claim that this solution can be uniquely continued not just locally near $\tau=0$ but also to the entire unbounded exterior region $\exterior$ as well as through its common boundary with the bounded region $\shelves$ into that entire region.  We have the following result, the proof of which can be found in Appendix~\ref{A:Proofs}.
%We will prove the stronger statement that if $\tau>0$ and $\chi>0$ and $(\chi,\tau)\in \shelves$ then there exists a unique real and generically simple root of $P(u;\chi,\tau)$.  
\begin{proposition}
Let $(\chi,\tau)\in \overline{\exterior\cup\shelves}$.  Then $P(u;\chi,\tau)$ has a unique real root of odd multiplicity, denoted $u=u(\chi,\tau)$ with $u(0,\tau)=0$ for $\tau>0$ and $u(\chi,0)=\chi$ for $\chi>2$.  There exists a value $\tau_1>0$ such that except for $\chi=0$ and possibly three or fewer points $(\chi,\tau)$ with $\chi>0$ and $\tau=\tau_1$, $u(\chi,\tau)$ is the only real root of $P(u;\chi,\tau)$ and it is simple.  
\label{prop:u}
\end{proposition}

\begin{remark}
In the case $\tau=0$ and $\chi>2$, $\Sigma_g$ is an arc connecting the two points $\lambda=\pm\ii B(\chi,0)$.  If we take $\Sigma_g$ to be the purely imaginary straight-line segment connecting these points, then from the prescribed large-$\lambda$ asymptotic behavior of $R(\lambda;\chi,0)$ we find that $R(\pm\ii;\chi,0)=\pm 2\ii/\chi$, from which it follows directly via the formula \eqref{eq:hprime-formula} that the residue conditions \eqref{eq:hprime-residues} hold, so the signs of the residues which had been conflated in squaring the residue conditions are indeed correctly resolved with the indicated choice of $\Sigma_g$.  To ensure that this successful resolution is maintained upon continuation of the solution from $\tau=0$ it is then sufficient that $\Sigma_g$ deform continuously with $(\chi,\tau)$ without ever contacting the poles $\lambda=\pm\ii$.  This is feasible because the endpoints $A(\chi,\tau)\pm\ii B(\chi,\tau)$ lie in the left half-plane for all $(\chi,\tau)\in\overline{\exterior\cup\shelves}$ with $\chi>0$ and $\tau>0$; indeed from \eqref{eq:eliminate-AB}, $A(\chi,\tau)=0$ holds if and only if $u(\chi,\tau)=\chi$ and $P(\chi;\chi,\tau)=128\tau^2\chi^3$ which vanishes only on the coordinate axes.  Therefore $A(\chi,\tau)$ has one sign on the interior of $\overline{\exterior\cup\shelves}$, and by \eqref{eq:AB-tau-small} we see that $A(\chi,\tau)<0$.  Note that when $\chi\downarrow 0$ for given $\tau>0$, the proof of Proposition~\ref{prop:u} shows that $B(\chi,\tau)^2$ tends to a value strictly greater than $1$ while $A(\chi,\tau)\to 0$, so in this limiting situation we should choose $\Sigma_g$ to lie in the left half-plane except for its endpoints.
\label{rem:Sigma_g}
\end{remark}

We are now in a position to show that, as claimed in Section~\ref{sec:h-intro}, $B(\chi,\tau)^2>0$ holds for all $(\chi,\tau)\in\exterior\cup\shelves\cup(\partial\exterior\cap\partial\shelves)$ and $B(\chi,\tau)^2=0$ holds on the common boundary of the union with $\channels$. 
Indeed,
expressing $B^2$ explicitly in terms of $u$, $\chi$, and $\tau$ using \eqref{eq:eliminate-AB} and \eqref{eq:eliminate-v}, one finds that $B^2=0$ implies $3u^3-4\chi u^2+(4\tau^2+\chi^2) u=0$.  The resultant between this equation and $P(u;\chi,\tau)$ vanishes on the open quadrant $(\chi,\tau)\in\mathbb{R}_{>0}\times\mathbb{R}_{>0}$ exactly where \eqref{eq:boundary-curve} holds.  



\subsubsection{Critical points of $h'(\lambda;\chi,\tau)$ for $(\chi,\tau)\in\exterior\cup\shelves$}
\label{sec:critical-points}
Observe that while the coefficients $u(\chi,\tau)$ and $v(\chi,\tau)$ in the quadratic factor in the numerator of $h'(\lambda;\chi,\tau)$ defined in \eqref{eq:hprime-formula} depend real-analytically on $(\chi,\tau)\in (\mathbb{R}_{\ge 0}\times\mathbb{R}_{\ge 0})\setminus\overline{\channels}$, the quadratic discriminant vanishes to first order along two curves in this domain so the roots undergo bifurcation upon crossing these curves.  Eliminating $v$ via \eqref{eq:eliminate-v}, the quadratic discriminant $u^2-8\tau v$ is seen to vanish only if $3u^3-\chi u^2-32\tau^2 u+16\chi\tau^2=0$.  The resultant of this cubic polynomial with $P(u;\chi,\tau)$ vanishes for $(\chi,\tau)\in\mathbb{R}_{>0}\times\mathbb{R}_{>0}$ exactly where $D(\chi,\tau)\defeq H_{10}(\chi,\tau)+H_8(\chi,\tau)+H_6(\chi,\tau) + H_4(\tau)=0$, in which the $H_j$ are homogeneous polynomials
\begin{equation}
\begin{split}
H_{10}(\chi,\tau)&\defeq -(8\tau^2-\chi^2)(100\tau^2+\chi^2)^4,\\
H_8(\chi,\tau)&\defeq 2\chi^8+1040\chi^6\tau^2+1741728\chi^4\tau^4-125516800\chi^2\tau^6+730880000\tau^8,\\
H_6(\chi,\tau)&\defeq \chi^6+504\chi^4\tau^2+3103488\chi^2\tau^4+67627008\tau^6,\\
H_4(\tau)&\defeq 1492992\tau^4.
\end{split}
\label{eq:H-polynomials}
\end{equation}
There is one unbounded arc in the first quadrant where this condition holds (see the dotted blue curve in Figure~\ref{fig:RegionsPlot}) and it is governed far from the origin by the highest-order homogeneous terms $H_{10}(\chi,\tau)\approx 0$; this arc is therefore asymptotic to the line $\chi=\sqrt{8}\tau$ (see the dotted gray line in Figure~\ref{fig:RegionsPlot}).  The cusp point $(\chi,\tau)=(\chi^\sharp,\tau^\sharp)$ (see \eqref{eq:corner-point}) on the boundary of $\channels$ is a non-smooth point on the locus $D(\chi,\tau)=0$ because the gradient vector vanishes there as well.  
In fact, setting $(\chi,\tau)=(\chi^\sharp+\Delta\chi,\tau^\sharp+\Delta\tau)$, one computes that
\begin{equation}
D(\chi,\tau)=\frac{3948901875}{256}\left(\frac{2}{\sqrt{3}}\Delta \tau-\Delta \chi\right)^2 + O((\Delta\chi^2+\Delta\tau^2)^\frac{3}{2}).
\end{equation}
The leading terms describe two curves tangent to the line $\Delta\chi=\tfrac{2}{\sqrt{3}}\Delta\tau$, and along this line the cubic correction terms are proportional to $\Delta\tau^3$ by a negative coefficient.  Therefore the two curves both emanate from the cusp point $(\Delta\chi,\Delta\tau)=(0,0)$ along this tangent line in the direction $\Delta\tau>0$, entering the exterior of $\channels$ from the cusp point.  
Since $D(0,\tau)=2048(1-\tau^2)\tau^4(625\tau^2+27)^2$, an arc along which this condition holds exits the quadrant $(\chi,\tau)\in \mathbb{R}_{>0}\times\mathbb{R}_{>0}$ on the $\tau$-axis at the point $(\chi,\tau)=(0,1)$.  This is the arc separating $\shelves$ from $\exterior$, and is shown as a solid blue curve in Figure~\ref{fig:RegionsPlot}.

%\textcolor{red}{[This is not a complete picture, but it gives the most important details.]}

\subsubsection{Construction of $g(\lambda;\chi,\tau)$ when $\Sigma_g\cap\Sigma_\mathrm{c}=\emptyset$}
%\textcolor{red}{(Move to beginning of $\shelves$ proof section?)}
\label{sec:g-function-loop}
Since $h'(\lambda;\chi,\tau)$ is now well-defined for all $(\chi,\tau)\in \exterior\cup\shelves$, we have $g'(\lambda;\chi,\tau)=h'(\lambda;\chi,\tau)-\vartheta'(\lambda;\chi,\tau)$ which has removable singularities at $\lambda=\pm\ii$ according to \eqref{eq:hprime-residues}
and hence is an analytic function for $\lambda\in\mathbb{C}\setminus\Sigma_g$ with, according to \eqref{eq:hprime-expansion}, the asymptotic behavior $g'(\lambda;\chi,\tau)=O(\lambda^{-2})$ as $\lambda\to\infty$.  Since $g'(\lambda;\chi,\tau)$ is integrable at $\lambda=\infty$, the contour integral
\begin{equation}
g(\lambda;\chi,\tau)\defeq\int_\infty^\lambda g'(\eta;\chi,\tau)\,\dd\eta
\label{eq:g-integral}
\end{equation}
is independent of path in the domain $\mathbb{C}\setminus\Sigma_g$ and defines the unique antiderivative analytic in the same domain that satisfies the condition $g(\infty;\chi,\tau)=0$.  It is easy to check that $g(\lambda^*;\chi,\tau)=g(\lambda;\chi,\tau)^*$ holds for each $\lambda\in\Sigma_g$ and $(\chi,\tau)\in \exterior\cup\shelves$.  Obtaining $g(\lambda;\chi,\tau)$ from \eqref{eq:g-integral} is a bit of a subtle calculation, because the integrability at $\eta=\infty$ and $\eta=\pm\ii$ relies on cancellations arising from the equations satisfied by the parameters $u,v,A,B$.  Another approach is to assume that $\Sigma_g$ is determined by solving those equations, and then to note that $g(\lambda;\chi,\tau)$ is a function analytic for $\lambda\in\mathbb{C}\setminus\Sigma_g$ with $g(\lambda;\chi,\tau)=O(\lambda^{-1})$ as $\lambda\to\infty$ and whose boundary values on $\Sigma_g$ satisfy 
\begin{equation}
\begin{split}
g_+(\lambda;\chi,\tau)+g_-(\lambda;\chi,\tau)&=h_+(\lambda;\chi,\tau)+h_-(\lambda;\chi,\tau)-2\vartheta(\lambda;\chi,\tau)\\
&=2 \kappa(\chi,\tau)-2\vartheta(\lambda;\chi,\tau),\quad \lambda\in\Sigma_g,
\end{split}
\label{eq:hpm-kappa}
\end{equation}
for some integration constant $\kappa(\chi,\tau)$ (because the sum of boundary values of $h$ is constant along $\Sigma_g$).  This formula \eqref{eq:hpm-kappa} assumes that $\vartheta(\lambda;\chi,\tau)$ is analytic on the subset $\Sigma_g$ of the jump contour for $\mathbf{S}(\lambda;\chi,\tau,\mathbf{G},M)$.  As the jump contour for $\vartheta(\lambda;\chi,\tau)$ is $\Sigma_\mathrm{c}$, we are assuming that the latter is contained in the interior of the Jordan curve $\Sigma_\circ$, which guarantees that $\Sigma_g\cap\Sigma_\mathrm{c}=\emptyset$.  Another situation, in which $\Sigma_\circ$ is deformed into a dumbbell-shaped jump contour with a Schwarz-symmetric neck that is necessarily a subset of $\Sigma_\mathrm{c}$, will be required to prove Theorem~\ref{thm:exterior}.  We will describe how the procedure needs to be modified for that case in Section~\ref{sec:g-function-dumbbell} below.

Returning to \eqref{eq:hpm-kappa}, to determine the constant $\kappa(\chi,\tau)$ and simultaneously obtain $g(\lambda;\chi,\tau)$ without using \eqref{eq:g-integral}, we represent $g(\lambda;\chi,\tau)$ in the form $g(\lambda;\chi,\tau)=R(\lambda;\chi,\tau)k(\lambda;\chi,\tau)$, from which it follows that $k(\lambda;\chi,\tau)$ is analytic for $\lambda\in\mathbb{C}\setminus\Sigma_g$, with bounded boundary values except at the branch points $\lambda=A\pm\ii B$ where it is only required that the product $R(\lambda;\chi,\tau)k(\lambda;\chi,\tau)$ is bounded.  We also require that $k(\lambda;\chi,\tau)=O(\lambda^{-2})$ as $\lambda\to\infty$, and that the boundary values taken by $k(\lambda;\chi,\tau)$ along $\Sigma_g$ are related by
\begin{equation}
k_+(\lambda;\chi,\tau)-k_-(\lambda;\chi,\tau)=\frac{2 \kappa(\chi,\tau)-2\vartheta(\lambda;\chi,\tau)}{R_+(\lambda;\chi,\tau)},\quad\lambda\in\Sigma_g,
\end{equation}
as implied by \eqref{eq:hpm-kappa}.
It follows that $k(\lambda;\chi,\tau)$ is necessarily given by the Plemelj formula:
\begin{equation}
k(\lambda;\chi,\tau)=\frac{1}{\ii\pi}\int_{\Sigma_g}\frac{\kappa(\chi,\tau)-\vartheta(\eta;\chi,\tau)}{R_+(\eta;\chi,\tau)(\eta-\lambda)}\,\dd\eta,\quad\lambda\in\mathbb{C}\setminus\Sigma_g.
\label{eq:k-formula}
\end{equation}
It is not hard to see that this formula automatically gives the condition that $R(\lambda;\chi,\tau)k(\lambda;\chi,\tau)$ is bounded at the branch points $\lambda=A\pm\ii B$.  However the condition $k(\lambda;\chi,\tau)=O(\lambda^{-2})$ as $\lambda\to\infty$ remains to be enforced, and this will determine the integration constant $\kappa(\chi,\tau)$.  Indeed, the coefficient of the leading term proportional to $\lambda^{-1}$ in the Laurent expansion of $k(\lambda;\chi,\tau)$ about $\lambda=\infty$ must vanish, i.e.,
\begin{equation}
\int_{\Sigma_g}\frac{\kappa(\chi,\tau)-\vartheta(\lambda;\chi,\tau)}{R_+(\lambda;\chi,\tau)}\,\dd\lambda = 0.
\label{eq:K-integral}
\end{equation}
Note that letting $L$ denote any clockwise-oriented loop surrounding the branch cut $\Sigma_g$ of $R(\lambda;\chi,\tau)$, a residue computation at $\lambda=\infty$ where $R(\lambda;\chi,\tau)=\lambda+O(1)$ shows that
\begin{equation}
\int_{\Sigma_g}\frac{\dd\lambda}{R_+(\lambda;\chi,\tau)} = \frac{1}{2}\oint_L\frac{\dd\lambda}{R(\lambda;\chi,\tau)} =-\ii\pi.
\label{eq:integral-R-plus}
\end{equation}
This being nonzero shows that $\kappa(\chi,\tau)$ will indeed be determined by the condition \eqref{eq:K-integral}.  Then,
\begin{equation}
\int_{\Sigma_g}\frac{\vartheta(\lambda;\chi,\tau)}{R_+(\lambda;\chi,\tau)}\,\dd\lambda = I_1(\chi,\tau)+I_2(\chi,\tau),
\end{equation}
where
\begin{equation}
I_1(\chi,\tau)\defeq\int_{\Sigma_g}\frac{\chi\lambda+\tau\lambda^2}{R_+(\lambda;\chi,\tau)}\,\dd\lambda\quad\text{and}\quad
I_2(\chi,\tau)\defeq\ii\int_{\Sigma_g}
%\log\left(\frac{\lambda-\ii}{\lambda+\ii}\right)
\frac{\log(B(\lambda))}{R_+(\lambda;\chi,\tau)}\,\dd\lambda.
\label{eq:I1-I2}
\end{equation}
A similar residue calculation using two more terms in the large-$\lambda$ expansion of $R(\lambda;\chi,\tau)$, specifically that $R(\lambda;\chi,\tau)^{-1}=\lambda^{-1}+A\lambda^{-2}+(A^2-\tfrac{1}{2}B^2)\lambda^{-3}+O(\lambda^{-4})$ shows that
\begin{equation}
I_1(\chi,\tau)=-\ii\pi (\chi A +\tau(A^2-\tfrac{1}{2}B^2)),\quad A=A(\chi,\tau),\quad B^2=B(\chi,\tau)^2.
\end{equation}
Now assuming that the loop $L$ excludes the branch cut $\Sigma_\mathrm{c}$ of the logarithm and that $L'$ is a counter-clockwise oriented contour that encircles $\Sigma_\mathrm{c}$ but that excludes $\Sigma_g$, we use the fact that the integrand for $I_2$ is integrable at $\lambda=\infty$ to obtain
\begin{equation}
I_2(\chi,\tau)=\frac{1}{2}\ii\oint_L
%\log\left(\frac{\lambda-\ii}{\lambda+\ii}\right)
\frac{\log(B(\lambda))}{R(\lambda;\chi,\tau)}\,\dd\lambda = \frac{1}{2}\ii\oint_{L'}%\log\left(\frac{\lambda-\ii}{\lambda+\ii}\right)
\frac{\log(B(\lambda))}{R(\lambda;\chi,\tau)}\,\dd\lambda.
\label{eq:I2-identities}
\end{equation}
Then, collapsing $L'$ to both sides of $\Sigma_\mathrm{c}$, where $R(\lambda;\chi,\tau)$ is analytic but the boundary values of the logarithm differ by $2\pi\ii$, 
\begin{equation}
I_2(\chi,\tau)=\pi\int_{\Sigma_\mathrm{c}}\frac{\dd\lambda}{R(\lambda;\chi,\tau)},
\end{equation}
where we recall that $\Sigma_\mathrm{c}$ is a Schwarz-symmetric arc oriented from $-\ii$ to $\ii$.  Therefore, $I_2(\chi,\tau)$ is
purely imaginary and is computable in terms of $A(\chi,\tau)$ and $B(\chi,\tau)^2$ via hyperbolic functions.  We have therefore obtained a formula for the integration constant $\kappa(\chi,\tau)$ in the form \eqref{eq:kappa-formula} written in Section~\ref{sec:Results-Shelves}.
%\begin{equation}
%\kappa(\chi,\tau)=\chi A+\tau(A^2-\tfrac{1}{2}B^2) +\ii\int_{\Sigma_\mathrm{c}}\frac{\dd\lambda}{R(\lambda;\chi,\tau)}.
%\label{eq:kappa-formula}
%\end{equation}
According to \eqref{eq:k-formula} and $g(\lambda;\chi,\tau)=R(\lambda;\chi,\tau)k(\lambda;\chi,\tau)$ we have (evaluating the term proportional to $\kappa(\chi,\tau)$ by residues):
%Note that the ambiguity in the path of integration in the domain $\mathbb{C}\setminus\Sigma_g$ implies by a residue calculation that $\kappa(\chi,\tau)$ is well-defined by \eqref{eq:kappa-formula} modulo $2\pi$.  This ambiguity will not be important as,
\begin{equation}
g(\lambda;\chi,\tau)= \kappa(\chi,\tau)-\frac{R(\lambda;\chi,\tau)}{\ii\pi}\int_{\Sigma_g}\frac{\vartheta(\eta;\chi,\tau)\,\dd\eta}{R_+(\eta;\chi,\tau)(\eta-\lambda)}.
\end{equation}
%so $g(\lambda;\chi,\tau)$ is determined modulo $2\pi\ii$.  Therefore the matrix factor $\ee^{ng(\lambda;\chi,\tau)\sigma_3}$ appearing in the transformation \eqref{eq:T-to-S} is well defined for $n\in\mathbb{Z}$.
%\textcolor{red}{This discussion needs to be modified in the alternative approach, because $M$ is not an integer, so an ambiguity mod $2\pi$ could be a problem in making $\ee^{Mg(\lambda;\chi,\tau)\sigma_3}$ well defined.  I think the right approach here is just to stick with $\Sigma_\mathrm{c}$ as the path of integration in the formula for $\kappa$ and not worry about whether the integral from $-\ii$ to $\ii$ would make sense without additional interpretation of the path of integration.}

\subsubsection{Construction of $g(\lambda;\chi,\tau)$ when $\Sigma_g\subset\Sigma_\mathrm{c}$}
\label{sec:g-function-dumbbell}
%\textcolor{red}{(Move to beginning of $\exterior$ proof section?)}
If the Schwarz-symmetric jump contour $\Sigma_g$ is to be taken as a subset of $\Sigma_\mathrm{c}$,
then some modification of the construction of $g(\lambda;\chi,\tau)$ is needed.  Indeed, in this situation the phase function $\vartheta(\lambda;\chi,\tau)$ defined in \eqref{eq:vartheta} takes two distinct boundary values at each point of $\Sigma_g$, so instead of \eqref{eq:hpm-kappa} the condition that the sum of boundary values of $h(\lambda;\chi,\tau)$ is constant along $\Sigma_g$ now reads
\begin{equation}
\begin{split}
g_+(\lambda;\chi,\tau)+g_-(\lambda;\chi,\tau)&=h_+(\lambda;\chi,\tau)+h_-(\lambda;\chi,\tau)-\vartheta_+(\lambda;\chi,\tau)-\vartheta_-(\lambda;\chi,\tau)\\
&=2\gamma(\chi,\tau)-\vartheta_+(\lambda;\chi,\tau)-\vartheta_-(\lambda;\chi,\tau),\quad
\lambda\in\Sigma_g\subset\Sigma_\mathrm{c},
\end{split}
\label{eq:hpm-gamma}
\end{equation}
where the constant value of $h_++h_-$ is now denoted $2\gamma(\chi,\tau)$.  As before, we write $g(\lambda;\chi,\tau)=R(\lambda;\chi,\tau)k(\lambda;\chi,\tau)$ and find that the analogue of \eqref{eq:k-formula} reads
\begin{equation}
k(\lambda;\chi,\tau)=\frac{1}{\ii\pi}\int_{\Sigma_g}\frac{\gamma(\chi,\tau)-\tfrac{1}{2}\vartheta_+(\eta;\chi,\tau)-\tfrac{1}{2}\vartheta_-(\eta;\chi,\tau)}{R_+(\eta;\chi,\tau)(\eta-\lambda)}\,\dd\eta,\quad\lambda\in\mathbb{C}\setminus\Sigma_g,
\label{eq:k-formula-gamma}
\end{equation}
where $\gamma(\chi,\tau)$ is to be chosen to enforce the condition analogous to \eqref{eq:K-integral}: 
\begin{equation}
\int_{\Sigma_g}\frac{\gamma(\chi,\tau)-\tfrac{1}{2}\vartheta_+(\lambda;\chi,\tau)-\tfrac{1}{2}\vartheta_-(\lambda;\chi,\tau)}{R_+(\lambda;\chi,\tau)}\,\dd\lambda = 0.
\label{eq:K-integral-gamma}
\end{equation}
By residue calculations, this can be written in the form
\begin{equation}
\gamma(\chi,\tau)=\chi A(\chi,\tau)+\tau(A(\chi,\tau)^2-\tfrac{1}{2}B(\chi,\tau)^2)-\frac{1}{\ii\pi}I_2(\chi,\tau),
\end{equation}
where now $I_2(\chi,\tau)$ is given by a modification of the formula in \eqref{eq:I1-I2}:
\begin{equation}
I_2(\chi,\tau)\defeq\ii\int_{\Sigma_g}\frac{\tfrac{1}{2}\log_+(B(\lambda))+\tfrac{1}{2}\log_-(B(\lambda))}{R_+(\lambda;\chi,\tau)}\,\dd\lambda.
\end{equation}
Taking $L$ to be a clockwise-oriented loop passing through the endpoints $A\pm\ii B$ of $\Sigma_g$ and enclosing $\Sigma_g$ but with the two arcs of $\Sigma_\mathrm{c}\setminus\Sigma_g$ in its exterior, and taking $L'$ to be a pair of counterclockwise-oriented loops each enclosing one of the arcs of $\Sigma_\mathrm{c}\setminus\Sigma_g$ and passing through the corresponding endpoint of $\Sigma_g$, we again arrive at the identities \eqref{eq:I2-identities}.  Then collapsing the loops of $L'$ to both sides of the arcs of $\Sigma_\mathrm{c}\setminus\Sigma_g$ we obtain
\begin{equation}
I_2(\chi,\tau)=\pi\int_{\Sigma_\mathrm{c}\setminus\Sigma_g}\frac{\dd\lambda}{R(\lambda;\chi,\tau)},
\end{equation}
leading to the analogue of \eqref{eq:kappa-formula}:
\begin{equation}
\gamma(\chi,\tau)=\chi A+\tau(A^2-\tfrac{1}{2}B^2) +\ii\int_{\Sigma_\mathrm{c}\setminus\Sigma_g}\frac{\dd\lambda}{R(\lambda;\chi,\tau)}.
\label{eq:gamma-formula}
\end{equation}
This is equivalent to the form written in \eqref{eq:gamma-formula-intro} in Section~\ref{sec:Results-Exterior}.

\subsubsection{Structure of the zero level curve $\mathrm{Re}(\ii h(\lambda;\chi,\tau))=0$}
\label{sec:zero-level-curve}
A consequence of the choice of integration constant to ensure that $g(\lambda;\chi,\tau)\to 0$ as $\lambda\to\infty$ is that both $g$ and $h$ have even Schwarz symmetry for all $(\chi,\tau)\in \overline{\exterior\cup\shelves}$:
\begin{equation}
g(\lambda^*;\chi,\tau)^*=g(\lambda;\chi,\tau)\quad\text{and}\quad h(\lambda^*;\chi,\tau)^*=h(\lambda;\chi,\tau),\quad (\chi,\tau)\in \overline{\exterior\cup\shelves}.
\label{eq:g-h-Schwarz}
\end{equation}
It follows that $\mathrm{Re}(\ii h(\lambda;\chi,\tau))=0$ holds for all $\lambda\in\mathbb{R}$, $\lambda\not\in\Sigma_g$.  We also have the following.
\begin{lemma}
For all $(\chi,\tau)\in \overline{\exterior\cup\shelves}$, $\mathrm{Re}(\ii h(A\pm\ii B;\chi,\tau))=0$, where $A\pm\ii B=A(\chi,\tau)\pm\ii B(\chi,\tau)$ are the complex-conjugate endpoints of $\Sigma_g$.
\label{lem:h-imaginary-at-branch-points}
\end{lemma}
\begin{proof}
Let $\lambda_\mathbb{R}\in\mathbb{R}$ with $\lambda_\mathbb{R}\not\in\Sigma_g$.
Since $h(\lambda_\mathbb{R};\chi,\tau)$ is purely real,
\begin{equation}
\begin{split}
\mathrm{Re}(\ii h(A+\ii B;\chi,\tau))&=\mathrm{Re}\left(\ii \int_{\lambda_\mathbb{R}}^{A+\ii B}h'(\lambda;\chi,\tau)\,\dd\lambda\right)\\ &= \frac{1}{2}\ii \int_{\lambda_\mathbb{R}}^{A+\ii B}h'(\lambda;\chi,\tau)\,\dd\lambda -\frac{1}{2}\ii\left[\int_{\lambda_\mathbb{R}}^{A+\ii B}h'(\lambda;\chi,\tau)\,\dd\lambda\right]^*,
\end{split}
\end{equation}
where due to \eqref{eq:hprime-residues} the path of integration $L:\lambda_\mathbb{R}\to A+\ii B$ is arbitrary in the upper half plane, except that it is chosen so that the pole at $\lambda=\ii$ does not lie between $L$ and $\Sigma_g$.  Using the even Schwarz symmetry of $h'(\lambda;\chi,\tau)$ and a contour integral reparametrization,
\begin{equation}
\begin{split}
\left[\int_{\lambda_\mathbb{R}}^{A+\ii B}h'(\lambda;\chi,\tau)\,\dd\lambda\right]^*&=\int_{\lambda_\mathbb{R}}^{A+\ii B}h'(\lambda;\chi,\tau)^*\,\dd\lambda^*\\
&=\int_{\lambda_\mathbb{R}}^{A+\ii B}h'(\lambda^*;\chi,\tau)\,\dd\lambda^*\\
&=-\int_{A-\ii B}^{\lambda_\mathbb{R}}h'(\lambda;\chi,\tau)\,\dd\lambda,
\end{split}
\end{equation}
where in the final integral the path of integration is $L^*$ but with opposite orientation.  Combining these results, and taking into account that $h'(\lambda;\chi,\tau)$ changes sign across $\Sigma_g$, we have
\begin{equation}
\mathrm{Re}(\ii h(A+\ii B;\chi,\tau))=\frac{1}{4}\ii \oint_O h'(\lambda;\chi,\tau)\,\dd\lambda,
\end{equation}
where $O$ is a simple closed contour enclosing $\Sigma_g$ but excluding $\lambda=\pm\ii$, the orientation of which depends on whether $\lambda_\mathbb{R}$ lies to the left or right of the point where $\Sigma_g$ intersects the real axis.  Using \eqref{eq:hprime-residues}, without changing the value of the integral we may replace $O$ by another contour surrounding $\Sigma_g$ with the same orientation but now also enclosing $\lambda=\pm\ii$.  Since there are no longer any singularities of $h'(\lambda;\chi,\tau)$ outside of $O$, we may evaluate the integral over $O$ by residues at $\lambda=\infty$.  Using \eqref{eq:hprime-expansion} one sees that the residue of $h'(\lambda;\chi,\tau)$ at $\lambda=\infty$ vanishes, so we conclude that $\mathrm{Re}(\ii h(A+\ii B;\chi,\tau))=0$.  Using \eqref{eq:g-h-Schwarz} then gives also $\mathrm{Re}(\ii h(A-\ii B;\chi,\tau))=0$.
\end{proof}
This result implies that the level curve $\mathrm{Re}(\ii h(\lambda;\chi,\tau))=0$ does not depend substantially on the choice of branch cut $\Sigma_g$.  Indeed, the differential $\ii h'(\lambda;\chi,\tau)\,\dd\lambda$ can be extended from $\lambda\in\mathbb{C}\setminus\Sigma_g$ to the hyperelliptic Riemann surface $\mathcal{R}$ of the equation $R^2=(\lambda-A)^2+B^2$ just by adding a second copy of $\mathbb{C}\setminus\Sigma_g$ on which $R(\lambda;\chi,\tau)$ is replaced with $-R(\lambda;\chi,\tau)$.  Since $\mathcal{R}$ has genus zero and hence has trivial homology, and since the residues of $h'(\lambda;\chi,\tau)$ (see \eqref{eq:hprime-residues}--\eqref{eq:hprime-expansion}) are imaginary, the real part of an antiderivative of $\ii h'(\lambda;\chi,\tau)\,\dd\lambda$ is well defined up to a constant as a harmonic function on $\mathcal{R}$ with the four points corresponding to $\lambda=\pm\ii$ omitted.  By 
Lemma~\ref{lem:h-imaginary-at-branch-points}, if the constant of integration is determined by fixing the base point of integration to be one of the two branch points, the real part vanishes at both branch points and on the principal sheet of $\mathcal{R}$ this function coincides with $\mathrm{Re}(\ii h(\lambda;\chi,\tau))$ while on the auxiliary sheet it coincides with $-\mathrm{Re}(\ii h(\lambda;\chi,\tau))$.  It follows that the projection from each sheet of $\mathcal{R}$ to $\mathbb{C}$ of the zero level is exactly the same.  Since the choice of branch cut $\Sigma_g$ for $R(\lambda;\chi,\tau)$ only affects the value of $\mathrm{Re}(\ii h(\lambda;\chi,\tau))$ up to a sign, the zero level curve is essentially independent of the location of $\Sigma_g$ (technically, $\mathrm{Re}(\ii h(\lambda;\chi,\tau))$ is undefined on $\Sigma_g$, but the zero level curve can be extended unambiguously to $\Sigma_g$). 

As noted above, the zero level set $\mathrm{Re}(\ii h(\lambda;\chi,\tau))=0$ always contains the real axis as a proper subset, as well as the branch points $\lambda=A\pm\ii B$.  Since $\ii h(\lambda;\chi,\tau)=\ii \vartheta(\lambda;\chi,\tau)+\ii g(\lambda;\chi,\tau)=\ii \vartheta(\lambda;\chi,\tau)+O(\lambda^{-1})=\ii \chi\lambda+\ii \tau\lambda^2+O(\lambda^{-1})$ as $\lambda\to\infty$, for $\tau\neq 0$ in $\overline{\exterior\cup\shelves}$ there is exactly one Schwarz-symmetric pair of arcs of the zero level set that are asymptotically vertical, one in each half-plane.  All other arcs of the level set in $\mathbb{C}\setminus\mathbb{R}$ are bounded.  These arcs are necessarily ``horizontal'' trajectories of the rational quadratic differential $h'(\lambda;\chi,\tau)^2\,\dd\lambda^2$, i.e., curves along which $h'(\lambda;\chi,\tau)^2\,\dd\lambda^2>0$.  By Lemma~\ref{lem:h-imaginary-at-branch-points}, some of the arcs of the zero level set are so-called critical trajectories, i.e., those emanating from zeros of $h'(\lambda;\chi,\tau)^2$.  By Jenkins' three-pole theorem \cite[Theorem 3.6]{Jenkins58} and the basic structure theorem \cite[Theorem 3.5]{Jenkins58}, the union of critical trajectories of $h'(\lambda;\chi,\tau)^2\,\dd\lambda^2$ has empty interior and divides the complex $\lambda$-plane into a finite number of domains.  Two of these domains, one in each half-plane, are so-called \emph{circle domains} each containing one of the poles $\lambda=\pm\ii$ and each having at least one of the zeros of $h'(\lambda;\chi,\tau)^2$ on its boundary.  Furthermore, from each of the simple roots $\lambda=A\pm\ii B$ of $h'(\lambda;\chi,\tau)^2$ emanate locally exactly three critical trajectories, and from each of the double roots of $h'(\lambda;\chi,\tau)^2$ (i.e., the roots of $2\tau\lambda^2+u(\chi,\tau)\lambda+v(\chi,\tau)$) emanate locally exactly four critical trajectories.

Suppose first that $(\chi,\tau)\in \exterior_\chi\cup \shelves$.  Then the double roots of $h'(\lambda;\chi,\tau)$ (two for $\tau\neq 0$ and one for $\tau=0$) are real, and therefore two of the four trajectories emanating from each coincide with intervals of $\mathbb{R}$ (that are contained in the level set $\mathrm{Re}(\ii h(\lambda;\chi,\tau))=0$, and the closure of the union of which is exactly $\mathbb{R}$).  In this case, by Lemma~\ref{lem:h-imaginary-at-branch-points} all critical trajectories are included in the level set $\mathrm{Re}(\ii h(\lambda;\chi,\tau))=0$.  The level curves entering the upper and lower half-planes vertically from $\lambda=\infty$ for $\tau\neq 0$ can only terminate at one of the roots of $h'(\lambda;\chi,\tau)^2$.  These trajectories either terminate at one of the real double roots, or at the conjugate pair of simple roots $\lambda=A\pm\ii B$.  
\begin{itemize}
\item
If they terminate at one of the two real double roots, then the non-real trajectories emanating from the other real double root can only terminate at the simple roots $\lambda=A\pm\ii B$.  It follows that the two additional trajectories emanating from each of these simple roots must coincide and form a closed curve in each half-plane.  By Teichm\"uller's lemma \cite[Theorem 14.1]{Strebel84}, this curve must be the boundary of the circle domain containing the pole $\lambda=\pm\ii$. If $\tau=0$ and hence there are no unbounded arcs of the level set in the open upper and lower half-planes, then by the same arguments the non-real trajectories emanating from the unique real double root terminate at the simple roots $\lambda=A\pm\ii B$, and the remaining two trajectories from each of these coincide and enclose the poles at $\lambda=\pm\ii$.  The zero level $\mathrm{Re}(\ii h(\lambda;\chi,\tau))=0$ consists of the real line, a Schwarz-symmetric pair of arcs connecting a real double root with the conjugate pair of simple roots $\lambda=A\pm\ii B$, a Schwarz-symmetric pair of loops joining each simple root $\lambda=A\pm\ii B$ to itself and enclosing the poles at $\lambda=\pm\ii$, and (if $\tau\neq 0$) a Schwarz-symmetric pair of unbounded arcs emanating from the second real double root and tending vertically to $\lambda=\infty$.  This topological configuration of the zero level set holds on the domain $\exterior_\chi$ (as one can see from the limiting case of $\tau=0$, where the zero level set acquires additional Schwarz reflection symmetry in the imaginary axis).
\item
If they terminate at the conjugate pair of simple roots $\lambda=A\pm \ii B$, then the remaining two trajectories emanating from each simple root terminate at the two real double roots, and the boundary of the circle domain in each half-plane consists of three distinct trajectories, one of which is the interval of the real axis between the two real double roots and is common to the boundaries of both circle domains.  (The other apparent possibility, that the two additional trajectories emanating from $\lambda=A\pm\ii B$ coincide and that the two trajectories emanating into each half-plane from the two real double roots also coincide, can be ruled out by Teichm\"uller's lemma since two closed curves formed by critical trajectories would appear in each half-plane, only one of which can contain a pole.)  The zero level set $\mathrm{Re}(\ii h(\lambda;\chi,\tau))=0$ consists of the real line, a Schwarz-symmetric pair of arcs from each of the two real double roots to the conjugate pair of simple roots, and a Schwarz-symmetric pair of unbounded arcs emanating from the conjugate pair of simple roots and tending vertically to $\lambda=\infty$.  This topological configuration of the zero level set holds on the domain $\shelves$ (as one can see from the limiting case of $\chi=0$, where again the zero level set acquires additional Schwarz reflection symmetry in the imaginary axis).
\end{itemize}

Next suppose that $(\chi,\tau)\in \exterior_\tau$.  Then the double roots of $h'(\lambda;\chi,\tau)^2$ form a conjugate pair that we denote by $\lambda=C\pm\ii D$.  By Lemma~\ref{lem:h-imaginary-at-branch-points}, the simple roots $\lambda=A\pm\ii B$ are on the zero level of $\mathrm{Re}(\ii h(\lambda;\chi,\tau))$ and therefore at most one trajectory from each can be unbounded.  If none of the three trajectories emanating from $\lambda=A\pm\ii B$ is unbounded, then at least one of them must terminate at $\lambda=C\pm\ii D$ implying that $\mathrm{Re}(\ii h(C\pm\ii D;\chi,\tau))=0$ and hence the unbounded arc of the level curve in each half-plane terminates at this point as well.  If it is exactly one trajectory from $\lambda=A\pm\ii B$ that terminates at $\lambda=C\pm\ii D$, then the other two coincide forming a loop, and the remaining two bounded trajectories emanating from the latter must coincide forming a second loop; however only one of these loops can contain the pole at $\lambda=\pm\ii$ so the existence of both is ruled out by Teichm\"uller's lemma.  If it is exactly two trajectories from $\lambda=A\pm\ii B$ that terminate at $\lambda=C\pm\ii D$, then the third trajectory would have to be unbounded contradicting the assumption that all trajectories from $A\pm\ii B$ are bounded.  If all three trajectories from $\lambda=A\pm\ii B$ terminate at $\lambda=C\pm\ii D$, then we again form two domains bounded by trajectories only one of which can contain a pole leading to a contradiction with Teichm\"uller's lemma.  We conclude that exactly one of the trajectories emanating from each simple root $\lambda=A\pm\ii B$ is unbounded.  
It then follows that the other two trajectories emanating from $\lambda=A\pm\ii B$ must coincide.  Indeed, otherwise they must both terminate at the double root $\lambda=C\pm\ii D$ in the same half-plane from which we learn that $\mathrm{Re}(\ii h(C\pm\ii D;\chi,\tau))=0$ which implies that neither of the remaining two trajectories from $\lambda=C\pm\ii D$ can be unbounded or terminate at $A\pm\ii B$, so they must coincide.  It is then apparent that each half-plane contains a domain bounded by the two curves connecting $A\pm\ii B$ with $C\pm\ii D$ and a domain bounded by the trajectory joining $C\pm\ii D$ to itself; however the pole $\lambda=\pm\ii$ can only lie in one of these two domains, so the existence of the other leads to a contradiction with Teichm\"uller's lemma. The zero level set $\mathrm{Re}(\ii h(\lambda;\chi,\tau))=0$ is then the disjoint union of three components:  the real line and a Schwarz-symmetric pair of components each consisting of a loop trajectory joining $\lambda=A\pm\ii B$ to itself and surrounding $\lambda=\pm\ii$ and an unbounded trajectory emanating from $\lambda=A\pm\ii B$.  In this case, the double roots $\lambda=C\pm\ii D$ do not lie on the zero level of $\mathrm{Re}(\ii h(\lambda;\chi,\tau))$, and the level set is not connected. 

\section{Far-Field Asymptotic Behavior of Rogue Waves in the Domain $\exterior$}
\label{sec:Schi-Stau}
In this section, we prove Theorem~\ref{thm:exterior}.  Since that result is specialized to the case of fundamental rogue waves of order $k\in\mathbb{Z}_{>0}$ for which $\mathbf{G}=\mathbf{Q}^{-s}$ with $s=(-1)^k$ and $M=\tfrac{1}{2}k+\tfrac{1}{4}$ (assumptions that are essential to the proof), in this section we will write $\mathbf{S}^{(k)}(\lambda;\chi,\tau)=\mathbf{S}(\lambda;\chi,\tau,\mathbf{Q}^{-s},M)$.
\subsection{Deformation to a dumbbell-shaped contour}
\label{sec:dumbbell}
When $(\chi,\tau)\in \exterior$, we will find it useful to begin by replacing the Jordan jump contour $\Sigma_\circ$ for $\mathbf{S}^{(k)}(\lambda;\chi,\tau)$ with a dumbbell-shaped contour consisting of a closed loop $\Gamma^+$ in the upper half-plane surrounding the point $\lambda=\ii$ in the clockwise sense, its reflection $\Gamma^-$ in the real axis (also oriented in the clockwise sense), and a ``neck'' $N$ consisting of an upward-oriented arc against the left side of the branch cut $\Sigma_\mathrm{c}$ for $\vartheta(\lambda;\chi,\tau)$ and a downward-oriented arc against the right side of the same cut.  Combining these two jump conditions with the jump discontinuity of the function $\vartheta(\lambda;\chi,\tau)$ across the central arc $\Sigma_\mathrm{c}$ of the neck, we can write a single jump condition for $\mathbf{S}^{(k)}(\lambda;\chi,\tau)$ across $N$, which we take to be oriented in the upward direction.  For this calculation, we assume that initially the Jordan curve $\Sigma_\circ$ contains $\Gamma^+\cup N\cup \Gamma^-$ in its interior and we introduce a substitution by setting
%\begin{equation}
%\tilde{\mathbf{S}}^{(k)}(\lambda;\chi,\tau)\defeq
%\mathbf{S}^{(k)}(\lambda;\chi,\tau)\ee^{-\ii n\vartheta(\lambda;\chi,\tau)\sigma_3}\omega(\lambda)^{s\sigma_3}\mathbf{Q}^{-s}\omega(\lambda)^{-s\sigma_3}\ee^{\ii n\vartheta(\lambda;\chi,\tau)\sigma_3},
%\label{eq:S-Stilde}
%\end{equation}
%\textcolor{red}{The alternate version of this reads:
\begin{equation}
\tilde{\mathbf{S}}^{(k)}(\lambda;\chi,\tau)\defeq
\mathbf{S}^{(k)}(\lambda;\chi,\tau)\ee^{-\ii M\vartheta(\lambda;\chi,\tau)\sigma_3}\mathbf{Q}^{-s}\ee^{\ii M\vartheta(\lambda;\chi,\tau)\sigma_3},
\label{eq:S-Stilde-ALT}
\end{equation}
%}
for $\lambda$ between $\Sigma_\circ$ and $\Gamma^+\cup N\cup \Gamma^-$, and we set $\tilde{\mathbf{S}}^{(k)}(\lambda;\chi,\tau)\defeq\mathbf{S}^{(k)}(\lambda;\chi,\tau)$ elsewhere, i.e., in the exterior of $\Sigma_\circ$ and in the interior of $\Gamma^+$ and of $\Gamma^-$.  
Dropping the tilde, the jump contour for $\mathbf{S}^{(k)}(\lambda;\chi,\tau)$ becomes $\Gamma^+\cup N\cup \Gamma^-$.  The jump condition for $\mathbf{S}^{(k)}(\lambda;\chi,\tau)$ across $\Gamma^+$ and $\Gamma^-$ reads exactly the same as the original jump condition \eqref{eq:S-jump} across $\Sigma_\circ$.  
%To compute the jump condition across $N$, we recall that $\ee^{\ii n\vartheta(\lambda;\chi,\tau)}$ is single-valued and analytic for $\lambda\in N$, and then obtain from \eqref{eq:S-Stilde} by distinguishing the boundary values of $\omega(\lambda)$ and using the fact that the original version of $\mathbf{S}^{(k)}(\lambda;\chi,\tau)$ is analytic on $N$ that
%\begin{multline}
%\mathbf{S}_+^{(k)}(\lambda;\chi,\tau)=\\
%\mathbf{S}_-^{(k)}(\lambda;\chi,\tau)\ee^{-\ii n\vartheta(\lambda;\chi,\tau)\sigma_3}
%\omega_-(\lambda)^{s\sigma_3}\mathbf{Q}^s\omega_-(\lambda)^{-s\sigma_3}\omega_+(\lambda)^{s\sigma_3}\mathbf{Q}^{-s}
%\omega_+(\lambda)^{-s\sigma_3}\ee^{\ii n\vartheta(\lambda;\chi,\tau)\sigma_3},\\
%\lambda\in N.
%\end{multline}
%We will now simplify this jump across $N$, showing that the jump matrix is off-diagonal.  Indeed, we first observe that regardless of the value of the parity index $s=\pm 1$, 
%\begin{equation}
%\mathbf{Q}^s\omega_-(\lambda)^{-s\sigma_3}\omega_+(\lambda)^{s\sigma_3}\mathbf{Q}^{-s}=\frac{1}{2}
%\begin{bmatrix}\displaystyle \frac{\omega_+(\lambda)}{\omega_-(\lambda)}+\frac{\omega_-(\lambda)}{\omega_+(\lambda)} & 
%\displaystyle \frac{\omega_+(\lambda)}{\omega_-(\lambda)}-\frac{\omega_-(\lambda)}{\omega_+(\lambda)} \\\\
%\displaystyle \frac{\omega_+(\lambda)}{\omega_-(\lambda)}-\frac{\omega_-(\lambda)}{\omega_+(\lambda)} &
%\displaystyle \frac{\omega_+(\lambda)}{\omega_-(\lambda)}+\frac{\omega_-(\lambda)}{\omega_+(\lambda)}
%\end{bmatrix}.
%\end{equation}
%Then, using \eqref{eq:omega-jump} we find that
%%
%%Now, by the definition \eqref{eq:omega-def} of $\omega(\lambda)$, 
%%\begin{equation}
%%\begin{split}
%%\frac{\omega_+(\lambda)}{\omega_-(\lambda)}\pm\frac{\omega_-(\lambda)}{\omega_+(\lambda)}&=\frac{\omega_+(\lambda)^2\pm\omega_-(\lambda)^2}{\omega_+(\lambda)\omega_-(\lambda)}\\
%%&=\frac{f_+(\lambda)^2(1+\ii(\lambda-\rho_+(\lambda)))^2\pm f_-(\lambda)^2(1+\ii(\lambda-\rho_-(\lambda)))^2}{f_+(\lambda)f_-(\lambda)(1+\ii(\lambda-\rho_+(\lambda)))(1+\ii(\lambda-\rho_-(\lambda)))}.
%%\end{split}
%%\end{equation}
%%Then, using \eqref{eq:f-squared} we have
%%\begin{equation}
%%\begin{split}
%%[f_+(\lambda)f_-(\lambda)]^2&=\frac{(\lambda+\rho_+(\lambda))(\lambda+\rho_-(\lambda))}{4\rho_+(\lambda)\rho_-(\lambda)}\\
%%&=\frac{(\lambda+\rho_+(\lambda))(\lambda-\rho_+(\lambda))}{-4\rho_+(\lambda)^2}\\
%%&=\frac{\rho_+(\lambda)^2-\lambda^2}{4\rho_+(\lambda)^2}\\
%%&=\frac{1}{4(\lambda^2+1)}.
%%\end{split}
%%\end{equation}
%%Using the fact that $f(\lambda)\to 1$ as $\lambda\to\infty$ and that $f(\lambda)$ is analytic for $\lambda\in\mathbb{C}\setminus\Sigma_\mathrm{c}$, we take the appropriate square root and obtain
%%\begin{equation}
%%f_+(\lambda)f_-(\lambda)=\frac{1}{2\rho_-(\lambda)}=-\frac{1}{2\rho_+(\lambda)}.
%%\end{equation}
%%Using this along with $\rho_-(\lambda)=-\rho_+(\lambda)$ and $\rho_\pm(\lambda)^2=\lambda^2+1$ then shows that 
%%\begin{equation}
%%\frac{\omega_+(\lambda)}{\omega_-(\lambda)}\pm\frac{\omega_-(\lambda)}{\omega_+(\lambda)}=\ii(1\mp 1).
%%\end{equation}
%%It follows that regardless of the parity index $s=\pm 1$, 
%\begin{equation}
%\mathbf{Q}^s\omega_-(\lambda)^{-s\sigma_3}\omega_+(\lambda)^{s\sigma_3}\mathbf{Q}^{-s}=\ii\sigma_1,\quad\lambda\in N.
%\end{equation}
%Moreover, it then follows easily from \eqref{eq:omega-jump} again that
%\begin{equation}
%\begin{split}
%\mathbf{S}^{(k)}_+(\lambda;\chi,\tau)&=\mathbf{S}^{(k)}_-(\lambda;\chi,\tau)
%\begin{bmatrix}
%0 & s\omega_+(\lambda)^{2s}\ee^{-2\ii n\vartheta(\lambda;\chi,\tau)}\\
%-s\omega_+(\lambda)^{-2s}\ee^{2\ii n\vartheta(\lambda;\chi,\tau)} & 0\end{bmatrix}\\
%&=
%\mathbf{S}^{(k)}_-(\lambda;\chi,\tau)
%\begin{bmatrix}
%0 & -s\omega_-(\lambda)^{2s}\ee^{-2\ii n\vartheta(\lambda;\chi,\tau)}\\
%s\omega_-(\lambda)^{-2s}\ee^{2\ii n\vartheta(\lambda;\chi,\tau)} & 0\end{bmatrix},\quad\lambda\in N.
%\end{split}
%\label{eq:S-N-jump}
%\end{equation}
%\textcolor{red}{In the alternate approach, the way to compute the jump across $N$ is easier.  
To compute the jump of the redefined $\mathbf{S}^{(k)}(\lambda;\chi,\tau)$ across $N$,
we start from its definition and using the fact that $\vartheta(\lambda;\chi,\tau)$ takes distinct boundary values on $N$ from either side we get
\begin{multline}
\mathbf{S}^{(k)}_+(\lambda;\chi,\tau)=\mathbf{S}_-^{(k)}(\lambda;\chi,\tau)\ee^{-\ii M\vartheta_-(\lambda;\chi,\tau)\sigma_3}\mathbf{Q}^s\ee^{\ii M\vartheta_-(\lambda;\chi,\tau)\sigma_3}
\ee^{-\ii M\vartheta_+(\lambda;\chi,\tau)\sigma_3}\mathbf{Q}^{-s}\ee^{\ii M\vartheta_+(\lambda;\chi,\tau)\sigma_3}\\
{}=\mathbf{S}_-^{(k)}(\lambda;\chi,\tau)\frac{1}{2}
\begin{bmatrix} 1+\ee^{2\ii M(\vartheta_+(\lambda;\chi,\tau)-\vartheta_-(\lambda;\chi,\tau))} & 
s\left(\ee^{-2\ii M\vartheta_+(\lambda;\chi,\tau)}-\ee^{-2\ii M\vartheta_-(\lambda;\chi,\tau)}\right)\\
-s\left(\ee^{2\ii M\vartheta_+(\lambda;\chi,\tau)}-\ee^{2\ii M\vartheta_-(\lambda;\chi,\tau)}\right) & 
1+\ee^{-2\ii M(\vartheta_+(\lambda;\chi,\tau)-\vartheta_-(\lambda;\chi,\tau))}\end{bmatrix},\\
\quad\quad\quad\quad\quad\quad\lambda\in N.
\end{multline}
But by \eqref{eq:vartheta} we have $\vartheta_+(\lambda;\chi,\tau)-\vartheta_-(\lambda;\chi,\tau)=-2\pi$.  Since $M=\tfrac{1}{2}k+\tfrac{1}{4}$ for $k\in\mathbb{Z}_{>0}$, this easily reduces to
\begin{equation}
\begin{split}
\mathbf{S}^{(k)}_+(\lambda;\chi,\tau)&=\mathbf{S}_-^{(k)}(\lambda;\chi,\tau)\begin{bmatrix}0 & s\ee^{-2\ii M\vartheta_+(\lambda;\chi,\tau)}\\-s\ee^{2\ii M\vartheta_+(\lambda;\chi,\tau)} & 0\end{bmatrix}\\
&=\mathbf{S}_-^{(k)}(\lambda;\chi,\tau)\begin{bmatrix}0 & -s\ee^{-2\ii M\vartheta_-(\lambda;\chi,\tau)}\\
s\ee^{2\ii M\vartheta_-(\lambda;\chi,\tau)} & 0\end{bmatrix},\quad\lambda\in N.
\end{split}
\label{eq:S-N-jump-ALT}
\end{equation}
%}

\begin{remark}
The fact that the jump matrix on $N$ is off-diagonal is a consequence of the quantization of $M>0$ via $M=\tfrac{1}{2}k+\tfrac{1}{4}$, $k\in\mathbb{Z}_{>0}$, and the choice of ``core'' matrix $\mathbf{G}=\mathbf{Q}^{-s}$ for $s=(-1)^k$.  
More generally, if we express $M\ge 0$ in the modular form $M=\tfrac{1}{2}k+r$ with $k\in\mathbb{Z}_{\ge 0}$ and $0\le r<\tfrac{1}{2}$, then for $\mathbf{G}=\mathbf{Q}^{-s}$ with $s=\pm 1$ arbitrary we obtain
\begin{equation}
\mathbf{S}_+(\lambda;\chi,\tau,\mathbf{Q}^{-s},M)=
\mathbf{S}_-(\lambda;\chi,\tau,\mathbf{Q}^{-s},M)
\ee^{-\ii M\vartheta_-(\lambda;\chi,\tau)\sigma_3}\mathbf{Z}
\ee^{\ii M\vartheta_+(\lambda;\chi,\tau)\sigma_3},\quad\lambda\in N
\end{equation}
in place of \eqref{eq:S-N-jump-ALT}, where $\mathbf{Z}$ is the constant matrix
\begin{equation}
\mathbf{Z}\defeq\begin{bmatrix}
(-1)^k\cos(2\pi r) & s (-1)^k\ii\sin(2\pi r)\\ s(-1)^k\ii\sin(2\pi r) & (-1)^k\cos(2\pi r)\end{bmatrix}.
\end{equation}
It is then clear that the \emph{only} values of $M\ge 0$ for which $\mathbf{Z}$ is off-diagonal are those corresponding to rogue waves.  
This is the reason why fundamental rogue waves behave differently for $(\chi,\tau)\in \exterior$ than other solutions obtained from Riemann-Hilbert Problem~\ref{rhp:rogue-wave-reformulation} for different parameters as described in Section~\ref{sec:M-arbitrary}, such as the high-order multiple-pole solitons for which $M\in\tfrac{1}{2}\mathbb{Z}_{\ge 0}$.  The latter solutions are special once again, in that they are precisely the solutions for which $\mathbf{Z}$ is diagonal (in fact $\mathbf{Z}=(-1)^k\mathbb{I}$).
In the general case, all four entries of $\mathbf{Z}$ are nonzero and hence available for use as pivots in matrix factorizations, and this distinguishes the asymptotic behavior on $\exterior$ from both special cases as described in Section~\ref{sec:general}.
\label{rem:M-quantum}
\end{remark}

Next, we explain how the contours $\Gamma^+$ and $N$ should be chosen (recall that $\Gamma^-$ is the Schwarz reflection of $\Gamma^+$ with clockwise orientation).  Recall from Section~\ref{sec:zero-level-curve} that as $(\chi,\tau)$ ranges over $\exterior$, there exists a simple closed curve surrounding the point $\lambda=\ii$ and passing through the point $\lambda=A+\ii B$ such that all roots of $2\tau\lambda^2+u\lambda+v=0$ are in the exterior of this curve, and importantly, such that $h'(\lambda;\chi,\tau)\,\dd\lambda$ is purely real along the curve.  In other words, the circle domain for the rational quadratic differential $h'(\lambda;\chi,\tau)^2\,\dd\lambda^2$ containing the pole $\lambda=\ii$ (reality of the residue due to the condition \eqref{eq:hprime-residues} guarantees that this point is indeed contained in a circle domain) has only the critical point $\lambda=A+\ii B$ on its boundary.  We take the boundary curve, which is a critical trajectory for $h'(\lambda;\chi,\tau)^2\,\dd\lambda^2$, to be the loop $\Gamma^+$.  Then we choose $N$ to be any Schwarz-symmetric arc from $\lambda=A-\ii B$ to $\lambda=A+\ii B$ that lies in the exterior of both loops $\Gamma^+\cup \Gamma^-$.  Later we will fix its direction near the endpoints of $N$.  See the left-hand panels of Figures~\ref{fig:Schi1}--\ref{fig:Stau1}.

\begin{figure}[h]
\begin{center}
\includegraphics{Schi1.pdf}
\end{center}
\caption{Left:  for $(\chi,\tau)=(2.5,0.7)\in \exterior_\chi$, the regions in the $\lambda$-plane where $\mathrm{Re}(\ii h(\lambda;\chi,\tau))<0$ (shaded) and $\mathrm{Re}(\ii h(\lambda;\chi,\tau))>0$ (unshaded), and the modified jump contour $\Gamma^+\cup N\cup\Gamma^-$.  The jump contour $\Sigma_\mathrm{c}$ for $\vartheta(\lambda;\chi,\tau)$ consists of the union of $N$ and the dashed red arcs terminating at $\lambda=\pm\ii$ (red dots).  Critical points of $h(\lambda;\chi,\tau)$ are shown with black dots.  Also shown are the ``lens'' regions $L^\pm$ and $R^\pm$ lying to the left and right respectively of $\Gamma^\pm$.  Right:  the jump contour for $\mathbf{W}^{(k)}(\lambda;\chi,\tau)$.  Note that for $(\chi,\tau)\in \exterior_\chi$ we may choose the branch cut $N=\Sigma_g$ (highlighted in orange) to coincide with a level curve of $\mathrm{Re}(\ii h(\lambda;\chi,\tau))$ and with this choice $\mathrm{Re}(\ii h(\lambda;\chi,\tau))$ is a continuous function with the exception of the points $\lambda=\pm \ii$.}
\label{fig:Schi1}
\end{figure}

\begin{figure}[h]
\begin{center}
\includegraphics{Stau1.pdf}
\end{center}
\caption{As in Figure~\ref{fig:Schi1} but now for $(\chi,\tau)=(2.0,1.2)\in \exterior_\tau$.  In this case a pair of real critical points of $h(\lambda;\chi,\tau)$ on $\exterior_\chi$ have merged and split into a conjugate pair that is necessarily on a nonzero level of $\mathrm{Re}(\ii h(\lambda;\chi,\tau))$.  The consequence is that it is no longer possible on $\exterior_\tau$ to choose the branch cut $N=\Sigma_g$ to be a level curve of $\mathrm{Re}(\ii h(\lambda;\chi,\tau))$ which therefore experiences a jump discontinuity across the cut.  We illustrate this fact in this figure by taking $N$ as a somewhat arbitrary union of straight line segments instead of any natural trajectory of $h'(\lambda;\chi,\tau)^2\,\dd\lambda^2$.  Crucially, this issue plays no role in the subsequent analysis, because it is never necessary to factor the jump matrix carried by $N$.}
\label{fig:Stau1}
\end{figure}

%\textcolor{red}{Prove this claim\dots}
\subsection{Introduction of $g$ and steepest descent deformation of the Riemann-Hilbert problem}
Now with the contours $\Gamma^\pm$ and $N$ set up in this way, we introduce the $g$-function via the transformation \eqref{eq:T-to-S} 
%\textcolor{red}{(or its alternate version)} 
taking $\mathbf{S}^{(k)}(\lambda;\chi,\tau)$ to $\mathbf{T}^{(k)}(\lambda;\chi,\tau)$.  We assume that the Schwarz-symmetric arc $\Sigma_g$ where $g(\lambda;\chi,\tau)$ fails to be analytic coincides with $N$, which in turn is a sub-arc of $\Sigma_\mathrm{c}$.  Therefore, we need the version of the construction of $g(\lambda;\chi,\tau)$ described in Section~\ref{sec:g-function-dumbbell}.  Recalling from \eqref{eq:hpm-gamma} that $h_+(\lambda;\chi,\tau)+h_-(\lambda;\chi,\tau)=2\gamma(\chi,\tau)$ for $\lambda\in\Sigma_g=N$, where 
%$\kappa(\chi,\tau)$ 
$\gamma(\chi,\tau)$
is a real quantity 
given by \eqref{eq:gamma-formula}
%determined mod $2\pi$ by \eqref{eq:kappa-formula} 
%\textcolor{red}{(we need the form that is the average over two paths from $-\ii$ to $\ii$ on either side of $\Sigma_g$, or just the sum of an integral from $-\ii$ to $\lambda_0(\chi,\tau)^*$ and an integral from $\lambda_0(\chi,\tau)$ to $\ii$)}, 
we obtain from \eqref{eq:S-N-jump-ALT} the jump condition for $\mathbf{T}^{(k)}(\lambda;\chi,\tau)$ along $N$ in the form
%\begin{equation}
%\begin{split}
%\mathbf{T}^{(k)}_+(\lambda;\chi,\tau)&=\mathbf{T}^{(k)}_-(\lambda;\chi,\tau)\begin{bmatrix}
%0 & s\omega_+(\lambda)^{2s}\ee^{-2\ii n\kappa(\chi,\tau)}\\-s\omega_+(\lambda)^{-2s}\ee^{2\ii n\kappa(\chi,\tau)} & 0\end{bmatrix}\\
%&=\mathbf{T}^{(k)}_-(\lambda;\chi,\tau)\begin{bmatrix}
%0 & -s\omega_-(\lambda)^{2s}\ee^{-2\ii n\kappa(\chi,\tau)}\\s\omega_-(\lambda)^{-2s}\ee^{2\ii n\kappa(\chi,\tau)} & 0\end{bmatrix},\quad \lambda\in N.
%\end{split}
%\label{eq:T-jump-N-Schi-Stau}
%\end{equation}  
%\textcolor{red}{The alternate version of this formula reads
\begin{equation}
\mathbf{T}^{(k)}_+(\lambda;\chi,\tau)=\mathbf{T}^{(k)}_-(\lambda;\chi,\tau)\begin{bmatrix}
0 & \ii\ee^{-2\ii M\gamma(\chi,\tau)}\\\ii\ee^{2\ii M\gamma(\chi,\tau)} & 0\end{bmatrix},\quad\lambda\in N.
\label{eq:T-jump-N-Schi-Stau-ALT}
\end{equation}
%}
We next take advantage of the fact that $\mathrm{Re}(\ii h(\lambda;\chi,\tau))=0$ on $\Gamma^\pm$ to transform $\mathbf{T}^{(k)}(\lambda;\chi,\tau)$ explicitly into $\mathbf{W}^{(k)}(\lambda;\chi,\tau)$ by a substitution based on the same elementary factorization \eqref{eq:Q-factorizations} used in Section~\ref{sec:channels}.  Let $\Omega^\pm$ denote small disks centered at $\lambda=\pm\ii$ and enclosed by $\Gamma^\pm$ respectively, let $R^\pm$ denote the interior of $\Gamma^\pm$ with the closure of $\Omega^\pm$ excluded, and let $L^\pm$ denote lens-shaped regions on the exterior of $\Gamma^\pm$ as shown in the left-hand panels of Figures~\ref{fig:Schi1}--\ref{fig:Stau1}.  Then, we make the following definition (compare with \eqref{eq:T-S-L-plus-ALT}--\eqref{eq:T-S-L-minus-ALT}):
%\begin{equation}
%\mathbf{W}^{(k)}(\lambda;\chi,\tau)\defeq\mathbf{T}^{(k)}(\lambda;\chi,\tau)\begin{bmatrix}
%1 & 0\\ s\omega(\lambda)^{-2s}\ee^{2nh(\lambda;\chi,\tau)} & 1\end{bmatrix},\quad\lambda\in L^+,
%\end{equation}
%\begin{equation}
%\mathbf{W}^{(k)}(\lambda;\chi,\tau)\defeq\mathbf{T}^{(k)}(\lambda;\chi,\tau)2^{\frac{1}{2}\sigma_3}\begin{bmatrix}1 & \tfrac{1}{2}s\omega(\lambda)^{2s}\ee^{-2 n h(\lambda;\chi,\tau)}\\ 0 & 1\end{bmatrix},\quad\lambda\in R^+,
%\label{eq:W-def-Schi-Stau-Rplus}
%\end{equation}
%\textcolor{red}{with alternate versions:
\begin{equation}
\mathbf{W}^{(k)}(\lambda;\chi,\tau)\defeq\mathbf{T}^{(k)}(\lambda;\chi,\tau)\begin{bmatrix}
1 & 0\\ s\ee^{2\ii Mh(\lambda;\chi,\tau)} & 1\end{bmatrix},\quad\lambda\in L^+,
\label{eq:W-def-Schi-Stau-Lplus-ALT}
\end{equation}
\begin{equation}
\mathbf{W}^{(k)}(\lambda;\chi,\tau)\defeq\mathbf{T}^{(k)}(\lambda;\chi,\tau)2^{\frac{1}{2}\sigma_3}\begin{bmatrix}1 & \tfrac{1}{2}s\ee^{-2\ii M h(\lambda;\chi,\tau)}\\ 0 & 1\end{bmatrix},\quad\lambda\in R^+,
\label{eq:W-def-Schi-Stau-Rplus-ALT}
\end{equation}
%}
\begin{equation}
\mathbf{W}^{(k)}(\lambda;\chi,\tau)\defeq\mathbf{T}^{(k)}(\lambda;\chi,\tau)2^{\frac{1}{2}\sigma_3},\quad
\lambda\in\Omega^+,
\end{equation}
\begin{equation}
\mathbf{W}^{(k)}(\lambda;\chi,\tau)\defeq\mathbf{T}^{(k)}(\lambda;\chi,\tau)2^{-\frac{1}{2}\sigma_3},\quad
\lambda\in\Omega^-,
\end{equation}
%\begin{equation}
%\mathbf{W}^{(k)}(\lambda;\chi,\tau)\defeq\mathbf{T}^{(k)}(\lambda;\chi,\tau)2^{-\frac{1}{2}\sigma_3}\begin{bmatrix} 1 & 0\\-\tfrac{1}{2}s\omega(\lambda)^{-2s}\ee^{2 nh(\lambda;\chi,\tau)} & 1\end{bmatrix},\quad\lambda\in R^-,
%\label{eq:W-def-Schi-Stau-Rminus}
%\end{equation}
%\begin{equation}
%\mathbf{W}^{(k)}(\lambda;\chi,\tau)\defeq\mathbf{T}^{(k)}(\lambda;\chi,\tau)\begin{bmatrix}
%1 & -s\omega(\lambda)^{2s}\ee^{-2 nh(\lambda;\chi,\tau)}\\ 0 & 1\end{bmatrix},\quad
%\lambda\in L^-,
%\end{equation}
%\textcolor{red}{with alternate versions:
\begin{equation}
\mathbf{W}^{(k)}(\lambda;\chi,\tau)\defeq\mathbf{T}^{(k)}(\lambda;\chi,\tau)2^{-\frac{1}{2}\sigma_3}\begin{bmatrix} 1 & 0\\-\tfrac{1}{2}s\ee^{2\ii Mh(\lambda;\chi,\tau)} & 1\end{bmatrix},\quad\lambda\in R^-,\quad\text{and}
\label{eq:W-def-Schi-Stau-Rminus-ALT}
\end{equation}
\begin{equation}
\mathbf{W}^{(k)}(\lambda;\chi,\tau)\defeq\mathbf{T}^{(k)}(\lambda;\chi,\tau)\begin{bmatrix}
1 & -s\ee^{-2\ii Mh(\lambda;\chi,\tau)}\\ 0 & 1\end{bmatrix},\quad
\lambda\in L^-,
 \label{eq:W-def-Schi-Stau-Lminus-ALT}
\end{equation}
%}
and elsewhere that $\mathbf{T}^{(k)}(\lambda;\chi,\tau)$ is defined we set $\mathbf{W}^{(k)}(\lambda;\chi,\tau)\defeq\mathbf{T}^{(k)}(\lambda;\chi,\tau)$.  One can check that $\mathbf{W}^{(k)}(\lambda;\chi,\tau)$ can be defined on $\Gamma^\pm$ to be analytic there.  Taking into account that 
%$\omega(\lambda)$ \textcolor{red}{(or 
$\ee^{\pm 2\ii Mh(\lambda;\chi,\tau)}$
%)} 
has jump discontinuities across arcs $N^\pm$ within the annular domains $R^\pm$, 
%(although $\ee^{\pm 2nh(\lambda;\chi,\tau)}$ are both single-valued \textcolor{red}{[omit parenthetical remark for alternate approach]}), 
the jump contour for $\mathbf{W}^{(k)}(\lambda;\chi,\tau)$ is as shown in the right-hand panels of Figures~\ref{fig:Schi1}--\ref{fig:Stau1}.
The jump conditions satisfied by $\mathbf{W}^{(k)}(\lambda;\chi,\tau)$ are then the following.  Firstly, since $\mathbf{W}^{(k)}(\lambda;\chi,\tau)=\mathbf{T}^{(k)}(\lambda;\chi,\tau)$ holds for both boundary values taken along $N$, the same jump condition \eqref{eq:T-jump-N-Schi-Stau-ALT} holds for $\mathbf{W}^{(k)}(\lambda;\chi,\tau)$ also.  Next, comparing with \eqref{eq:Tjump-channels-CLplus-ALT}--\eqref{eq:Tjump-channels-CRplus-ALT} and \eqref{eq:Tjump-channels-CRminus-ALT}--\eqref{eq:Tjump-channels-CLminus-ALT}, we have
%\begin{equation}
%\mathbf{W}^{(k)}_+(\lambda;\chi,\tau)=\mathbf{W}^{(k)}_-(\lambda;\chi,\tau)\begin{bmatrix}1 & 0\\
%-s\omega(\lambda)^{-2s}\ee^{2 nh(\lambda;\chi,\tau)} & 1\end{bmatrix},\quad\lambda\in C_L^+,
%\end{equation}
%\begin{equation}
%\mathbf{W}^{(k)}_+(\lambda;\chi,\tau)=\mathbf{W}^{(k)}_-(\lambda;\chi,\tau)\begin{bmatrix}1 & \tfrac{1}{2}s\omega(\lambda)^{2s}\ee^{-2 nh(\lambda;\chi,\tau)} \\ 0 & 1\end{bmatrix},\quad\lambda\in C_R^+,
%\end{equation}
%\begin{equation}
%\mathbf{W}^{(k)}_+(\lambda;\chi,\tau)=\mathbf{W}^{(k)}_-(\lambda;\chi,\tau)\begin{bmatrix}1 & 0\\
%-\tfrac{1}{2}s\omega(\lambda)^{-2s}\ee^{2 nh(\lambda;\chi,\tau)} & 1\end{bmatrix},\quad
%\lambda\in C_R^-,\quad\text{and}
%\end{equation}
%\begin{equation}
%\mathbf{W}^{(k)}_+(\lambda;\chi,\tau)=\mathbf{W}^{(k)}_-(\lambda;\chi,\tau)\begin{bmatrix} 1 & s\omega(\lambda)^{2s}\ee^{-2 nh(\lambda;\chi,\tau)}\\ 0 & 1\end{bmatrix},\quad\lambda\in C_L^-.
%\end{equation}
%\textcolor{red}{The alternate versions of these read:
\begin{equation}
\mathbf{W}^{(k)}_+(\lambda;\chi,\tau)=\mathbf{W}^{(k)}_-(\lambda;\chi,\tau)\begin{bmatrix}1 & 0\\
-s\ee^{2\ii Mh(\lambda;\chi,\tau)} & 1\end{bmatrix},\quad\lambda\in C_L^+,
\label{eq:Wjump-exterior-CLplus}
\end{equation}
\begin{equation}
\mathbf{W}^{(k)}_+(\lambda;\chi,\tau)=\mathbf{W}^{(k)}_-(\lambda;\chi,\tau)\begin{bmatrix}1 & \tfrac{1}{2}s\ee^{-2\ii Mh(\lambda;\chi,\tau)} \\ 0 & 1\end{bmatrix},\quad\lambda\in C_R^+,
\end{equation}
\begin{equation}
\mathbf{W}^{(k)}_+(\lambda;\chi,\tau)=\mathbf{W}^{(k)}_-(\lambda;\chi,\tau)\begin{bmatrix}1 & 0\\
-\tfrac{1}{2}s\ee^{2\ii Mh(\lambda;\chi,\tau)} & 1\end{bmatrix},\quad
\lambda\in C_R^-,\quad\text{and}
\end{equation}
\begin{equation}
\mathbf{W}^{(k)}_+(\lambda;\chi,\tau)=\mathbf{W}^{(k)}_-(\lambda;\chi,\tau)\begin{bmatrix} 1 & s\ee^{-2\ii Mh(\lambda;\chi,\tau)}\\ 0 & 1\end{bmatrix},\quad\lambda\in C_L^-.
\label{eq:Wjump-exterior-CLminus}
\end{equation}
%}
Finally, for $\lambda\in N^\pm$ we compute 
%the jump condition from \eqref{eq:W-def-Schi-Stau-Rplus} and 
%\eqref{eq:W-def-Schi-Stau-Rminus}:
%\begin{equation}
%\mathbf{W}_+^{(k)}(\lambda;\chi,\tau)=\mathbf{W}_-^{(k)}(\lambda;\chi,\tau)\begin{bmatrix}1 & 
%\tfrac{1}{2}s(\omega_+(\lambda)^{2s}-\omega_-(\lambda)^{2s})\ee^{-2nh(\lambda;\chi,\tau)}\\0 & 1\end{bmatrix},\quad\lambda\in N^+,\quad\text{and}
%\end{equation}
%\begin{equation}
%\mathbf{W}^{(k)}_+(\lambda;\chi,\tau)=\mathbf{W}^{(k)}_-(\lambda;\chi,\tau)\begin{bmatrix}
%1 & 0\\-\tfrac{1}{2}s(\omega_+(\lambda)^{-2s}-\omega_-(\lambda)^{-2s})\ee^{2nh(\lambda;\chi,\tau)} & 1
%\end{bmatrix},\quad\lambda\in N^-.
%\end{equation}
%Note that according to \eqref{eq:omega-jump}, $\omega_+(\lambda)^{\pm 2s}-\omega_-(\lambda)^{\pm 2s}=2\omega_+(\lambda)^{\pm 2s}=-2\omega_-(\lambda)^{\pm 2s}$.  
%\textcolor{red}{The alternate version of this calculation reads:
\begin{equation}
\mathbf{W}^{(k)}_+(\lambda;\chi,\tau)=\mathbf{W}_-^{(k)}(\lambda;\chi,\tau)\begin{bmatrix}1 & \tfrac{1}{2}s\left(\ee^{-2\ii Mh_+(\lambda;\chi,\tau)}-\ee^{-2\ii Mh_-(\lambda;\chi,\tau)}\right)\\0 & 1\end{bmatrix},\quad
\lambda\in N^+,\quad\text{and}
\end{equation}
\begin{equation}
\mathbf{W}^{(k)}_+(\lambda;\chi,\tau)=\mathbf{W}^{(k)}_-(\lambda;\chi,\tau)\begin{bmatrix}1 & 0\\
-\tfrac{1}{2}s\left(\ee^{2\ii Mh_+(\lambda;\chi,\tau)}-\ee^{2\ii Mh_-(\lambda;\chi,\tau)}\right) & 1\end{bmatrix},\quad\lambda\in N^-.
\end{equation}
Then, since for $\lambda\in N^\pm$, $g(\lambda;\chi,\tau)$ has no jump discontinuity and $\vartheta_+(\lambda;\chi,\tau)-\vartheta_-(\lambda;\chi,\tau)=-2\pi$, and since $M=\tfrac{1}{2}k+\tfrac{1}{4}$ for $k\in\mathbb{Z}_{>0}$, these simplify to
\begin{equation}
\mathbf{W}^{(k)}_+(\lambda;\chi,\tau)=\mathbf{W}^{(k)}_-(\lambda;\chi,\tau)\begin{bmatrix}1 & \ii\ee^{-\ii M(h_+(\lambda;\chi,\tau)+h_-(\lambda;\chi,\tau))}\\0 & 1\end{bmatrix},\quad\lambda\in N^+,\quad\text{and}
\end{equation}
\begin{equation}
\mathbf{W}^{(k)}_+(\lambda;\chi,\tau)=\mathbf{W}^{(k)}_-(\lambda;\chi,\tau)\begin{bmatrix}1 & 0\\
\ii\ee^{\ii M(h_+(\lambda;\chi,\tau)+h_-(\lambda;\chi,\tau))} & 1\end{bmatrix},\quad\lambda\in N^-.
\end{equation}
%}

%As a final step to prepare for the construction of parametrices, we introduce a \emph{Szeg\H{o} function} $S(\lambda;\chi,\tau)$ whose purpose is to remove the non-constant factors involving boundary values of $\omega(\lambda)^{\pm 2}$ from the jump condition along $N$ (see \eqref{eq:T-jump-N-Schi-Stau}).  
%Recalling that $N$ is chosen to coincide with $\Sigma_g$, $S:\mathbb{C}\setminus\Sigma_g\to\mathbb{C}$ is required to be analytic and bounded at the endpoints $\lambda=A(\chi,\tau)\pm \ii B(\chi,\tau)$ of $\Sigma_g$, to satisfy the normalization condition $S(\lambda;\chi,\tau)\to 0$ as $\lambda\to\infty$, and to admit continuous boundary values $S_\pm(\lambda;\chi,\tau)$ on $\Sigma_g$ related by 
%\begin{equation}
%S_+(\lambda;\chi,\tau) + S_-(\lambda;\chi,\tau) - 2s \log(\omega_+(\lambda)) = 2 \ii {s}\gamma,\quad \lambda\in \Sigma_g, 
%\label{eq:Szego-jump-Schi-Stau}
%\end{equation}
%for some constant $\gamma=\gamma(\chi,\tau)$ which may depend on the parameters $\chi$ and $\tau$. 
%Unlike in the construction of $g(\lambda;\chi,\tau)$ described in Section~\ref{sec:g-function} where the endpoints $A(\chi,\tau)\pm\ii B(\chi,\tau)$ of $\Sigma_g$ were used as free parameters chosen to guarantee existence, these have now been determined and are no longer available; here the constant $\gamma$ will play a similar role of a parameter to be chosen so that $S$ exists.  In the jump condition \eqref{eq:Szego-jump-Schi-Stau}, $\log(\cdot)$ is taken to be the principal branch, so that $\log(\omega(\lambda))$ is an analytic function on the domain $\mathbb{C}\setminus\overline{N\cup N^+\cup N^-}$ that vanishes as $\lambda\to\infty$.  The boundary values $\log(\omega_\pm(\lambda))$ are analytic for $\lambda\in\Sigma_g$ (in fact, on a neighborhood of this arc) but unequal, and we have $\omega_+(\lambda)^{2s} = \ee^{2s \log(\omega_+(\lambda))}$.
%% with the right-hand side being well-defined for all $\lambda\in\Sigma_g$ including the end points since $\omega(\lambda)$ is nonzero, continuous and bounded (in fact, analytic) in a suitably small\footnote{Suitably small in the sense that the neighborhood leaves $\Sigma_\omega$ in its exterior.} neighborhood containing $\Sigma_g$. 
%It is easy to verify that
%\begin{equation}
%S(\lambda;\chi,\tau)\defeq \frac{R(\lambda;\chi,\tau)}{2\pi \ii} \int_{\Sigma_g} \frac{2s \log(\omega_+(\eta))+  2\ii{s} \gamma}{R_+(\eta; \chi,\tau)(\eta-\lambda)}\,\dd \eta
%\label{eq:Szego-def-Schi-Stau}
%\end{equation}
%is analytic in $\mathbb{C}\setminus\Sigma_g$, satisfies the jump condition given in \eqref{eq:Szego-jump-Schi-Stau}, and it is bounded as $\lambda \to A(\chi,\tau) \pm \ii B(\chi,\tau)$. Since $R(\lambda;\chi,\tau)=\lambda+O(1)$ as $\lambda\to\infty$, enforcing the normalization condition $S(\lambda;\chi,\tau)\to 0$ as $\lambda\to\infty$ requires 
%\begin{equation}
%\int_{\Sigma_g} \frac{\log(\omega_+(\eta))+ \ii \gamma(\chi,\tau)}{R_+(\eta;\chi,\tau)}\, \dd \eta = 0.
%\label{eq:gamma-condition-Schi-Stau}
%\end{equation}
%This condition determines the constant $\gamma=\gamma(\chi,\tau)$; indeed,
%recalling \eqref{eq:integral-R-plus} 
%we obtain from \eqref{eq:gamma-condition-Schi-Stau} that
%\begin{equation}
%\gamma(\chi,\tau)  = -\frac{1}{\pi} \displaystyle \int_{\Sigma_g} \frac{ \log(\omega_+(\eta)) }{R_+(\eta;\chi,\tau)}\,\dd \eta.
%\label{eq:gamma-def-Schi-Stau}
%\end{equation}
%We can simplify this formula as follows.  Firstly, let $\Sigma_{\tilde{\omega}}$ be an arc with the same endpoints ($\lambda=\pm\ii$) as but lying to the right of $\overline{N\cup N^+\cup N^-}$, and let  $\log(\tilde{\omega}(\lambda))$ denote the analytic continuation to $\mathbb{C}\setminus\Sigma_{\tilde{\omega}}$ from $\Sigma_g$ of $\log(\omega_+(\lambda))$.  Then, let $C$ be a clockwise-oriented loop surrounding the branch cut $\Sigma_g$ of $R(\lambda;\chi,\tau)$ excluding the branch cut $\Sigma_{\tilde{\omega}}$ of $\tilde{\omega}(\lambda)$. Taking $C'$ to be a counter-clockwise-oriented contour that encircles $\Sigma_{\tilde{\omega}}$ but excludes $\Sigma_g$ and using the fact that the integrand in \eqref{eq:gamma-def-Schi-Stau} is integrable at $\eta=\infty$, we obtain:
%\begin{equation}
%\gamma(\chi,\tau) = -\frac{1}{2\pi} \oint_C \frac{\log(\tilde{\omega}(\eta))}{R(\eta;\chi,\tau)}\,\dd \eta
%= -\frac{1}{2\pi} \oint_{C'} \frac{\log(\tilde{\omega}(\eta))}{R(\eta;\chi,\tau)}\,\dd \eta
%= -\frac{1}{8\pi} \oint_{C'} \log\left(\frac{\eta-\ii}{\eta+\ii}\right) \frac{\dd\eta}{R(\eta;\chi,\tau)},
%\end{equation}
%where we used the identity \eqref{eq:omega-fourth-power} and where the logarithm has $\Sigma_{\tilde{\omega}}$ as its branch cut. We may now collapse $C'$ to both sides of $\Sigma_{\tilde{\omega}}$, where $R(\lambda;\chi,\tau)$ is analytic but the boundary values of the logarithm differ by $2\pi \ii$, and see that
%\begin{equation}
%\gamma(\chi,\tau) = -\frac{1}{4\ii} \int_{\Sigma_{\tilde{\omega}}} \frac{\dd\eta}{R(\eta;\chi,\tau)},
%\label{eq:gamma-explicit-Schi-Stau}
%\end{equation}
%and since $\Sigma_{\tilde{\omega}}$ is a Schwarz-symmetric contour, it follows that $\gamma(\chi,\tau)\in\mathbb{R}$.  This completes the construction of the Szeg\H{o} function $S(\lambda;\chi,\tau)$.
%
%We now make a global substitution by setting
%\begin{equation}
%\mathbf{X}^{(k)}(\lambda;\chi,\tau) \defeq \mathbf{W}^{(k)}(\lambda;\chi,\tau)\ee^{S(\lambda;\chi,\tau)\sigma_3}
%\end{equation}
%in the whole domain of analyticity of $\mathbf{W}^{(k)}(\lambda;\chi,\tau)$.  From the jump condition \eqref{eq:Szego-jump-Schi-Stau}, the jump matrix on $\Sigma_g=N$ for $\mathbf{X}^{(k)}(\lambda;\chi,\tau)$
%becomes constant as desired:
%\begin{equation}
%\mathbf{X}_+^{(k)}(\lambda;\chi,\tau)=\mathbf{X}_-^{(k)}(\lambda;\chi,\tau)\begin{bmatrix}
%0 & s\ee^{-2\ii (n\kappa(\chi,\tau)+s\gamma(\chi,\tau))}\\
%-s\ee^{2\ii (n\kappa(\chi,\tau)+s\gamma(\chi,\tau))} & 0\end{bmatrix},\quad \lambda\in \Sigma_g=N.
%\label{eq:X-jump-N-Schi-Stau}
%\end{equation}
%The remaining jump conditions for $\mathbf{W}^{(k)}(\lambda;\chi,\tau)$ are merely modified by conjugation of the jump matrix, since $S_+(\lambda;\chi,\tau)=S_-(\lambda;\chi,\tau)$ on all other contour arcs:
%\begin{equation}
%\mathbf{X}^{(k)}_+(\lambda;\chi,\tau)=\mathbf{X}^{(k)}_-(\lambda;\chi,\tau)\begin{bmatrix}1 & 0\\
%-s\omega(\lambda)^{-2s}\ee^{2S(\lambda;\chi,\tau)}\ee^{2 nh(\lambda;\chi,\tau)} & 1\end{bmatrix},\quad\lambda\in C_L^+,
%\end{equation}
%\begin{equation}
%\mathbf{X}^{(k)}_+(\lambda;\chi,\tau)=\mathbf{X}^{(k)}_-(\lambda;\chi,\tau)\begin{bmatrix}1 & \tfrac{1}{2}s\omega(\lambda)^{2s}\ee^{-2S(\lambda;\chi,\tau)}\ee^{-2 nh(\lambda;\chi,\tau)} \\ 0 & 1\end{bmatrix},\quad\lambda\in C_R^+,
%\end{equation}
%\begin{equation}
%\mathbf{X}^{(k)}_+(\lambda;\chi,\tau)=\mathbf{X}^{(k)}_-(\lambda;\chi,\tau)\begin{bmatrix}1 & 0\\
%-\tfrac{1}{2}s\omega(\lambda)^{-2s}\ee^{2S(\lambda;\chi,\tau)}\ee^{2 nh(\lambda;\chi,\tau)} & 1\end{bmatrix},\quad
%\lambda\in C_R^-,\quad\text{and}
%\end{equation}
%\begin{equation}
%\mathbf{X}^{(k)}_+(\lambda;\chi,\tau)=\mathbf{X}^{(k)}_-(\lambda;\chi,\tau)\begin{bmatrix} 1 & s\omega(\lambda)^{2s}\ee^{-2S(\lambda;\chi,\tau)}\ee^{-2 nh(\lambda;\chi,\tau)}\\ 0 & 1\end{bmatrix},\quad\lambda\in C_L^-,
%\end{equation}
%\begin{equation}
%\mathbf{X}_+^{(k)}(\lambda;\chi,\tau)=\mathbf{X}_-^{(k)}(\lambda;\chi,\tau)\begin{bmatrix}1 & 
%\tfrac{1}{2}s(\omega_+(\lambda)^{2s}-\omega_-(\lambda)^{2s})\ee^{-2S(\lambda;\chi,\tau)}\ee^{-2nh(\lambda;\chi,\tau)}\\0 & 1\end{bmatrix},\quad\lambda\in N^+,\quad\text{and}
%\end{equation}
%\begin{equation}
%\mathbf{X}^{(k)}_+(\lambda;\chi,\tau)=\mathbf{X}^{(k)}_-(\lambda;\chi,\tau)\begin{bmatrix}
%1 & 0\\-\tfrac{1}{2}s(\omega_+(\lambda)^{-2s}-\omega_-(\lambda)^{-2s})\ee^{2S(\lambda;\chi,\tau)}\ee^{2nh(\lambda;\chi,\tau)} & 1
%\end{bmatrix},\quad\lambda\in N^-.
%\end{equation}
%\textcolor{red}{In the alternative approach, the entire discussion about the Szeg\H{o} function and the transformation from $\mathbf{W}^{(k)}(\lambda;\chi,\tau)$ to $\mathbf{X}^{(k)}(\lambda;\chi,\tau)$ is to be removed, and we just stick with $\mathbf{W}^{(k)}(\lambda;\chi,\tau)$ going forward (no need for $\mathbf{X}^{(k)}(\lambda;\chi,\tau)$ at all).}

\subsection{Parametrix construction}
\label{sec:Airy-parametrix}
From the sign structure of $\mathrm{Re}(\ii h(\lambda;\chi,\tau))$ as indicated with shading in Figures~\ref{fig:Schi1}--\ref{fig:Stau1}, it is then clear that the jump matrices are exponentially small perturbations of the identity matrix except when $\lambda\in N=\Sigma_g$ and in small neighborhoods of the branch points $\lambda=A\pm \ii B$.  To deal with these, we first construct an \emph{outer parametrix} denoted 
%$\dot{\mathbf{X}}^{(k),\mathrm{out}}(\lambda;\chi,\tau)$ \textcolor{red}{or, 
$\dot{\mathbf{W}}^{(k),\mathrm{out}}(\lambda;\chi,\tau)$
%)} 
designed to solve the jump condition %\eqref{eq:X-jump-N-Schi-Stau} 
%\textcolor{red}{(equation for jump of $\mathbf{W}^{(k)}$ across $N$)} 
\eqref{eq:T-jump-N-Schi-Stau-ALT} for $\lambda\in\Sigma_g$ exactly, to be analytic for $\lambda\in\mathbb{C}\setminus\Sigma_g$, and to tend to the identity as $\lambda\to\infty$.  This is easily accomplished simply by diagonalization of the constant jump matrix, the eigenvalues of which are $\pm \ii$.  All solutions of the jump condition 
%\eqref{eq:X-jump-N-Schi-Stau} 
%\textcolor{red}{(equation for jump of $\mathbf{W}^{(k)}$ across $N$)} 
\eqref{eq:T-jump-N-Schi-Stau-ALT}
have singularities at the endpoints of $\Sigma_g$, and we select the unique solution with the mildest rate of growth at these two points:
%\begin{multline}
%\dot{\mathbf{X}}^{(k),\mathrm{out}}(\lambda;\chi,\tau)\defeq\\
%\ee^{-\ii(n\kappa(\chi,\tau)+s\gamma(\chi,\tau))\sigma_3}\ee^{\frac{1}{4}(1-s)\ii\pi\sigma_3}\mathbf{O}\left(\frac{\lambda-\lambda_0(\chi,\tau)}{\lambda-\lambda_0(\chi,\tau)^*}\right)^{\frac{1}{4}\sigma_3}\mathbf{O}^{-1}\ee^{-\frac{1}{4}(1-s)\ii\pi\sigma_3}\ee^{\ii(n\kappa(\chi,\tau)+s\gamma(\chi,\tau))\sigma_3},
%\label{eq:outer-parametrix-Schi-Stau}
%\end{multline}
%where 
%\begin{equation}
%\mathbf{O}\defeq\frac{1}{\sqrt{2}}\begin{bmatrix} 1 & \ii\\\ii & 1\end{bmatrix},\quad\det(\mathbf{O})=1,
%\label{eq:O-def-Schi-Stau}
%\end{equation}
%and where the central factor is uniquely determined to be analytic for $\lambda\in\mathbb{C}\setminus\Sigma_g$ and to tend to the identity matrix as $\lambda\to\infty$.  
%\textcolor{red}{The alternative version of this formula reads:
\begin{equation}
\dot{\mathbf{W}}^{(k),\mathrm{out}}(\lambda;\chi,\tau)\defeq\\
\ee^{-\ii M\gamma(\chi,\tau)\sigma_3}\mathbf{Q}y(\lambda;\chi,\tau)^{\sigma_3}\mathbf{Q}^{-1}\ee^{\ii M\gamma(\chi,\tau)\sigma_3},
\label{eq:outer-parametrix-Schi-Stau-ALT}
\end{equation}
where $\mathbf{Q}$ is the matrix defined in \eqref{eq:Q-def}, and where $y(\lambda;\chi,\tau)$ is the function analytic for $\lambda\in\mathbb{C}\setminus\Sigma_g$ determined by the conditions
\begin{equation}
y(\lambda;\chi,\tau)^4=\frac{\lambda-\lambda_0(\chi,\tau)}{\lambda-\lambda_0(\chi,\tau)^*},\quad\text{and $y(\lambda;\chi,\tau)\to 1$ as $\lambda\to\infty$}.
\label{eq:y-def}
\end{equation}
%}
Note that the only dependence on 
%$n$ \textcolor{red}{(or, 
$M$
%)} 
enters via the oscillatory factors 
%$\ee^{\pm \ii n\kappa(\chi,\tau)\sigma_3}$ 
%\textcolor{red}{(or, $\ee^{\pm\ii M\kappa(\chi,\tau)\sigma_3}$)}, 
$\ee^{\pm\ii M\gamma(\chi,\tau)\sigma_3}$,
so the outer parametrix
%$\dot{\mathbf{X}}^{(k),\mathrm{out}}(\lambda;\chi,\tau)$ 
%\textcolor{red}{(or, $\dot{\mathbf{W}}^{(k),\mathrm{out}}(\lambda;\chi,\tau)$)} 
$\dot{\mathbf{W}}^{(k),\mathrm{out}}(\lambda;\chi,\tau)$
is bounded as 
%$n\to\infty$ \textcolor{red}{(or, as $M\to\infty$)}, 
$M\to\infty$,
provided that $\lambda$ is bounded away from $\lambda_0(\chi,\tau)$ and $\lambda_0(\chi,\tau)^*$.

Next, we let $D_{\lambda_0}(\delta)$ and $D_{\lambda_0^*}(\delta)=D_{\lambda_0}(\delta)^*$ be disks of small radius $\delta$ independent of 
%$n$ 
$M$
centered at $\lambda=\lambda_0(\chi,\tau)=A(\chi,\tau)+\ii B(\chi,\tau)$ and $\lambda=\lambda_0(\chi,\tau)^*$ respectively.  
%Since $h'(\lambda;\chi,\tau)$ vanishes like a square root as $\lambda\to \lambda_0(\chi,\tau)$ and $h(\lambda_0(\chi,\tau);\chi,\tau)=\ii\kappa(\chi,\tau)$, there is a univalent function $f^+(\lambda;\chi,\tau)$ defined on $D^+(\delta)$ with $f^+(\lambda_0(\chi,\tau);\chi,\tau)=0$ such that 
%\begin{equation}
%f^+(\lambda;\chi,\tau)^3=(2h(\lambda;\chi,\tau)-2\ii\kappa(\chi,\tau))^2,\quad \lambda\in D^+(\delta).
%\label{eq:Airy-map-Schi-Stau}
%\end{equation}
%Moreover, the univalent solution of \eqref{eq:Airy-map-Schi-Stau} and the jump contours $N\cap D^+(\delta)$, $N^+\cap D^+(\delta)$, and $C_L^+\cap D^+(\delta)$ can be chosen so that $\lambda\in N\cap D^+(\delta)$ implies $f^+(\lambda;\chi,\tau)<0$, $\lambda\in N^+\cap D^+(\delta)$ implies $f^+(\lambda;\chi,\tau)>0$, and $\lambda\in C_L^+\cap D^+(\delta)$ implies that either $\arg(f^+(\lambda;\chi,\tau))=\tfrac{2}{3}\pi$ or $\arg(f^+(\lambda;\chi,\tau))=-\tfrac{2}{3}\pi$.  Let $D^+_+(\delta)$ (resp., $D^+_-(\delta)$) denote the part of $D^+(\delta)$ to the left (resp., right) of $N\cup N^+$.  Define a matrix $\mathbf{Y}^{(k)}(\lambda;\chi,\tau)$ within $D^+(\delta)$ by
%\begin{equation}
%\mathbf{Y}^{(k)}(\lambda;\chi,\tau)\defeq\mathbf{X}^{(k)}(\lambda;\chi,\tau)\omega(\lambda)^{s\sigma_3}\ee^{-S(\lambda;\chi,\tau)\sigma_3}\ee^{-\ii n\kappa(\chi,\tau)\sigma_3}\begin{cases}
%\ee^{\frac{1}{4}(1-s)\ii\pi\sigma_3},&\lambda\in D^+_+(\delta)\\
%\ee^{\frac{1}{4}(1+s)\ii\pi\sigma_3},&\lambda\in D^+_-(\delta).
%\end{cases}
%\label{eq:Y-from-X-Schi-Stau}
%\end{equation}
%\textcolor{red}{The alternate version of this construction is as follows:
Since $h'(\lambda;\chi,\tau)$ vanishes like a square root as $\lambda\to \lambda_0(\chi,\tau)$ and $h_+(\lambda_0(\chi,\tau);\chi,\tau)+h_-(\lambda_0(\chi,\tau);\chi,\tau)=2\gamma(\chi,\tau)$, there is a univalent function $f_{\lambda_0}(\lambda;\chi,\tau)$ defined on $D_{\lambda_0}(\delta)$ with $f_{\lambda_0}(\lambda_0(\chi,\tau);\chi,\tau)=0$ such that 
\begin{equation}
f_{\lambda_0}(\lambda;\chi,\tau)^3=-(h_+(\lambda;\chi,\tau)+h_-(\lambda;\chi,\tau)-2\gamma(\chi,\tau))^2,\quad \lambda\in D_{\lambda_0}(\delta),
\label{eq:Airy-map-Schi-Stau-ALT}
\end{equation}
in which the sum of boundary values of $h$ is analytically continued from $N^+$ to $D_{\lambda_0}(\delta)\setminus N$ by means of the identity $h_+(\lambda;\chi,\tau)-h_-(\lambda;\chi,\tau)=-2\pi$ for $\lambda\in N^+$.
Moreover, the univalent solution of \eqref{eq:Airy-map-Schi-Stau-ALT} and the jump contours $N\cap D_{\lambda_0}(\delta)$, $N^+\cap D_{\lambda_0}(\delta)$, and $C_L^+\cap D_{\lambda_0}(\delta)$ can be chosen so that $\lambda\in N\cap D_{\lambda_0}(\delta)$ implies $f_{\lambda_0}(\lambda;\chi,\tau)<0$, $\lambda\in N^+\cap D_{\lambda_0}(\delta)$ implies $f_{\lambda_0}(\lambda;\chi,\tau)>0$, and $\lambda\in C_L^+\cap D_{\lambda_0}(\delta)$ implies that either $\arg(f_{\lambda_0}(\lambda;\chi,\tau))=\tfrac{2}{3}\pi$ or $\arg(f_{\lambda_0}(\lambda;\chi,\tau))=-\tfrac{2}{3}\pi$.  
Define a matrix $\mathbf{X}^{(k)}(\lambda;\chi,\tau)$ within $D_{\lambda_0}(\delta)$ by
\begin{equation}
\mathbf{X}^{(k)}(\lambda;\chi,\tau)\defeq\mathbf{W}^{(k)}(\lambda;\chi,\tau)\ee^{-\ii M\gamma(\chi,\tau)\sigma_3}\ee^{\frac{1}{4}\ii\pi\sigma_3},\quad\lambda\in D_{\lambda_0}(\delta).
\label{eq:Y-from-X-Schi-Stau-ALT}
\end{equation}
%}
Then, using again $M=\tfrac{1}{2}k+\tfrac{1}{4}$ for $k\in\mathbb{Z}_{>0}$, the jump conditions satisfied by 
%$\mathbf{Y}^{(k)}(\lambda;\chi,\tau)$ 
%\textcolor{red}{(or, $\mathbf{X}^{(k)}(\lambda;\chi,\tau)$)} 
$\mathbf{X}^{(k)}(\lambda;\chi,\tau)$
can be written in a simple form, in terms of the variable (rescaled conformal coordinate on $D_{\lambda_0}(\delta)$) 
%$\zeta\defeq n^{\frac{2}{3}}f^+(\lambda;\chi,\tau)$ 
%\textcolor{red}{(or, $\zeta\defeq M^\frac{2}{3}f^+(\lambda;\chi,\tau)$)}:
$\zeta\defeq M^\frac{2}{3}f_{\lambda_0}(\lambda;\chi,\tau)$:
\begin{equation}
\mathbf{X}^{(k)}_+(\lambda;\chi,\tau)=\mathbf{X}^{(k)}_-(\lambda;\chi,\tau)\begin{bmatrix}1 &\ee^{-\zeta^\frac{3}{2}} \\ 0 & 1\end{bmatrix},\quad \arg(\zeta)=0,
\label{eq:Airy-jump-first}
\end{equation}
\begin{equation}
\mathbf{X}^{(k)}_+(\lambda;\chi,\tau)=\mathbf{X}^{(k)}_-(\lambda;\chi,\tau)\begin{bmatrix}1 & 0\\-\ee^{\zeta^\frac{3}{2}} & 1\end{bmatrix},\quad\arg(\zeta)=\pm\tfrac{2}{3}\pi,\quad\text{and}
\end{equation}
\begin{equation}
\mathbf{X}^{(k)}_+(\lambda;\chi,\tau)=\mathbf{X}^{(k)}_-(\lambda;\chi,\tau)\begin{bmatrix}0 & -1\\
1 & 0\end{bmatrix},\quad\arg(-\zeta)=0,
\label{eq:Airy-jump-last}
\end{equation}
where for uniformity all four rays are taken to be oriented away from the origin in the $\zeta$-plane.
%\textcolor{red}{(Alternately, with $\mathbf{Y}^{(k)}$ replaced by $\mathbf{X}^{(k)}$ in \eqref{eq:Airy-jump-first}--\eqref{eq:Airy-jump-last}.)}
There exists a unique matrix function $\mathbf{A}(\zeta)$ with the following properties:
\begin{itemize}
\item
$\mathbf{A}(\zeta)$ is analytic for $0<|\arg(\zeta)|<\tfrac{2}{3}\pi$ and $\tfrac{2}{3}\pi<|\arg(\zeta)|<\pi$ (four sectors);
\item $\mathbf{A}(\zeta)$ takes continuous boundary values from each sector satisfying the same jump conditions written in \eqref{eq:Airy-jump-first}--\eqref{eq:Airy-jump-last};
\item $\mathbf{A}(\zeta)$ has uniform asymptotics in all directions of the complex plane given by 
%\begin{equation}
%\mathbf{A}(\zeta)\mathbf{O}\zeta^{-\frac{1}{4}\sigma_3}=\begin{bmatrix}1+O(\zeta^{-3}) & O(\zeta^{-1})\\O(\zeta^{-2}) & 1+O(\zeta^{-3})\end{bmatrix},\quad\zeta\to\infty,
%\end{equation}
%\textcolor{red}{Alternately (equivalantly):
\begin{equation}
\mathbf{A}(\zeta)\ee^{-\frac{1}{4}\ii\pi\sigma_3}\mathbf{Q}\ee^{\frac{1}{4}\ii\pi\sigma_3}\zeta^{-\frac{1}{4}\sigma_3}=\begin{bmatrix}1+O(\zeta^{-3}) & O(\zeta^{-1})\\O(\zeta^{-2}) & 1+O(\zeta^{-3})\end{bmatrix},\quad\zeta\to\infty,
\end{equation}
%}
%where $\mathbf{O}$ is the matrix defined in \eqref{eq:O-def-Schi-Stau}.
where $\mathbf{Q}$ is the matrix defined in \eqref{eq:Q-def}.
\end{itemize}
It is well-known that the unique solution of these Riemann-Hilbert conditions can be written explicitly in terms of Airy functions, and the reader can find a complete development of the solution in \cite[Appendix B]{BothnerM20}.
%We make a similar transformation of the outer parametrix as in \eqref{eq:Y-from-X-Schi-Stau}:
%\begin{equation}
%\dot{\mathbf{Y}}^{(k),\mathrm{out}}(\lambda;\chi,\tau)\defeq\dot{\mathbf{X}}^{(k),\mathrm{out}}(\lambda;\chi,\tau)\omega(\lambda)^{s\sigma_3}\ee^{-S(\lambda;\chi,\tau)\sigma_3}\ee^{-\ii n\kappa(\chi,\tau)\sigma_3}\begin{cases}\ee^{\frac{1}{4}(1-s)\ii\pi\sigma_3},& \lambda\in D^+_+(\delta)\\
%\ee^{\frac{1}{4}(1+s)\ii\pi\sigma_3},&\lambda\in D^+_-(\delta).
%\end{cases}
%\end{equation}
%It is straightforward to check that $\dot{\mathbf{Y}}^{(k),\mathrm{out}}(\lambda;\chi,\tau)$ satisfies the jump condition \eqref{eq:Airy-jump-last} but otherwise is analytic within $D^+$; it then follows that the matrix 
%\begin{equation}
%\mathbf{H}(\lambda;\chi,\tau)\defeq\ee^{\ii n\kappa(\chi,\tau)\sigma_3}\dot{\mathbf{Y}}^{(k),\mathrm{out}}(\lambda;\chi,\tau)\mathbf{O}f^+(\lambda;\chi,\tau)^{-\frac{1}{4}\sigma_3}
%\end{equation}
%is analytic for $\lambda\in D^+$ and is independent of $n$ (although it depends on the parity index $s$).
%\textcolor{red}{Alternatively, we simply define the matrix
Next, we define the matrix function
\begin{equation}
\mathbf{H}(\lambda;\chi,\tau)\defeq\ee^{\ii M\gamma(\chi,\tau)\sigma_3}\dot{\mathbf{W}}^{(k),\mathrm{out}}(\lambda;\chi,\tau)\ee^{-\ii M\gamma(\chi,\tau)\sigma_3}\mathbf{Q}f_{\lambda_0}(\lambda;\chi,\tau)^{-\frac{1}{4}\sigma_3}\ee^{\frac{1}{4}\ii\pi\sigma_3}
\end{equation}
and note that it follows from the definition of the conformal map $\lambda\mapsto f_{\lambda_0}(\lambda;\chi,\tau)$ and the definition \eqref{eq:outer-parametrix-Schi-Stau-ALT} of $\dot{\mathbf{W}}^{(k),\mathrm{out}}(\lambda;\chi,\tau)$ that $\mathbf{H}(\lambda;\chi,\tau)$ is analytic for $\lambda\in D_{\lambda_0}(\delta)$ and is independent of $M$.
%}
We use $\mathbf{H}(\lambda;\chi,\tau)$ and $\mathbf{A}(\zeta)$ to define an \emph{inner parametrix} on $D_{\lambda_0}(\delta)$ as follows:
%\begin{multline}
%\dot{\mathbf{X}}^{(k),\mathrm{in}}(\lambda;\chi,\tau)\defeq\\
%\ee^{-\ii n\kappa(\chi,\tau)\sigma_3}\mathbf{H}(\lambda;\chi,\tau)n^{-\frac{1}{6}\sigma_3}\mathbf{A}(n^\frac{2}{3}f^+(\lambda;\chi,\tau))\ee^{\ii n\kappa(\chi,\tau)\sigma_3}\ee^{S(\lambda;\chi,\tau)\sigma_3}\omega(\lambda)^{-s\sigma_3}\begin{cases}
%\ee^{-\frac{1}{4}(1-s)\ii\pi\sigma_3},&\lambda\in D_+^+(\delta)\\
%\ee^{-\frac{1}{4}(1+s)\ii\pi\sigma_3},&\lambda\in D_-^+(\delta).
%\end{cases}
%\end{multline}
%\textcolor{red}{The alternate version reads:
\begin{equation}
\dot{\mathbf{W}}^{(k),\lambda_0}(\lambda;\chi,\tau):=\ee^{-\ii M\gamma(\chi,\tau)\sigma_3}\mathbf{H}(\lambda;\chi,\tau)M^{-\frac{1}{6}\sigma_3}\mathbf{A}(M^\frac{2}{3}f_{\lambda_0}(\lambda;\chi,\tau))\ee^{-\frac{1}{4}\ii\pi\sigma_3}\ee^{\ii M\gamma(\chi,\tau)\sigma_3},\quad\lambda\in D_{\lambda_0}(\delta).
\end{equation}
%}
It is easy to check that 
%$\dot{\mathbf{X}}^{(k),\mathrm{in}}(\lambda;\chi,\tau)$ 
%\textcolor{red}{(or, $\dot{\mathbf{W}}^{(k),\mathrm{in}}(\lambda;\chi,\tau)$)} 
$\dot{\mathbf{W}}^{(k),\lambda_0}(\lambda;\chi,\tau)$
takes continuous boundary values that satisfy exactly the same jump conditions within $D_{\lambda_0}(\delta)$ as do those of 
%$\mathbf{X}^{(k)}(\lambda;\chi,\tau)$ \textcolor{red}{(or, $\mathbf{W}^{(k)}(\lambda;\chi,\tau)$)} 
$\mathbf{W}^{(k)}(\lambda;\chi,\tau)$
itself. Also, since $\zeta$ is large of size 
%$n^\frac{2}{3}$ \textcolor{red}{(or, 
$M^\frac{2}{3}$
%)} 
when $\lambda\in \partial D_{\lambda_0}(\delta)$, 
%\begin{multline}
%\dot{\mathbf{X}}^{(k),\mathrm{in}}(\lambda;\chi,\tau)\dot{\mathbf{X}}^{(k),\mathrm{out}}(\lambda;\chi,\tau)^{-1} \\
%\begin{aligned}
%&= \ee^{-\ii n\kappa(\chi,\tau)\sigma_3}\mathbf{H}(\lambda;\chi,\tau)n^{-\frac{1}{6}\sigma_3}\mathbf{A}(\zeta)\mathbf{O}\zeta^{-\frac{1}{4}\sigma_3}n^{\frac{1}{6}\sigma_3}\mathbf{H}(\lambda;\chi,\tau)^{-1}\ee^{\ii n\kappa(\chi,\tau)\sigma_3}\\
%&=\ee^{-\ii n\kappa(\chi,\tau)\sigma_3}\mathbf{H}(\lambda;\chi,\tau)\begin{bmatrix}1+O(\zeta^{-3}) & O(n^{-\frac{1}{3}}\zeta^{-1})\\O(n^\frac{1}{3}\zeta^{-2}) & 1+O(\zeta^{-3})\end{bmatrix}
%\mathbf{H}(\lambda;\chi,\tau)^{-1}\ee^{\ii n\kappa(\chi,\tau)\sigma_3}\\
%&=\mathbb{I}+O(n^{-1}),\quad\lambda\in\partial D^+(\delta).
%\end{aligned}
%\label{eq:Dplus-mismatch-Schi-Stau}
%\end{multline}
%\textcolor{red}{The alternative version reads:
\begin{multline}
\dot{\mathbf{W}}^{(k),\lambda_0}(\lambda;\chi,\tau)\dot{\mathbf{W}}^{(k),\mathrm{out}}(\lambda;\chi,\tau)^{-1} \\
\begin{aligned}
&= \ee^{-\ii M\gamma(\chi,\tau)\sigma_3}\mathbf{H}(\lambda;\chi,\tau)M^{-\frac{1}{6}\sigma_3}\mathbf{A}(\zeta)\ee^{-\frac{1}{4}\ii\pi\sigma_3}\mathbf{Q}\ee^{\frac{1}{4}\ii\pi\sigma_3}\zeta^{-\frac{1}{4}\sigma_3}M^{\frac{1}{6}\sigma_3}\mathbf{H}(\lambda;\chi,\tau)^{-1}\ee^{\ii M\gamma(\chi,\tau)\sigma_3}\\
&=\ee^{-\ii M\gamma(\chi,\tau)\sigma_3}\mathbf{H}(\lambda;\chi,\tau)\begin{bmatrix}1+O(\zeta^{-3}) & O(M^{-\frac{1}{3}}\zeta^{-1})\\O(M^\frac{1}{3}\zeta^{-2}) & 1+O(\zeta^{-3})\end{bmatrix}
\mathbf{H}(\lambda;\chi,\tau)^{-1}\ee^{\ii M\gamma(\chi,\tau)\sigma_3}\\
&=\mathbb{I}+O(M^{-1}),\quad\lambda\in\partial D_{\lambda_0}(\delta).
\end{aligned}
\label{eq:Dplus-mismatch-Schi-Stau-ALT}
\end{multline}
%}
Since the matrix $\mathbf{W}^{(k)}(\lambda;\chi,\tau)$ satisfies $\mathbf{W}^{(k)}(\lambda^*;\chi,\tau)=\sigma_2\mathbf{W}^{(k)}(\lambda;\chi,\tau)^*\sigma_2$, 
%\textcolor{red}{(or, replacing $\mathbf{X}\mapsto\mathbf{W}$)}, 
we may define a second inner parametrix for $\lambda\in D_{\lambda_0^*}(\delta)$ to respect this symmetry.

We combine the inner and outer parametrices into a \emph{global parametrix} by setting
%\begin{equation}
%\dot{\mathbf{X}}^{(k)}(\lambda;\chi,\tau)\defeq\begin{cases}
%\dot{\mathbf{X}}^{(k),\mathrm{in}}(\lambda;\chi,\tau),&\quad\lambda\in D^+(\delta)\\
%\sigma_2\dot{\mathbf{X}}^{(k),\mathrm{in}}(\lambda^*;\chi,\tau)^*\sigma_2,&\quad\lambda\in D^-(\delta)\\
%\dot{\mathbf{X}}^{(k),\mathrm{out}}(\lambda;\chi,\tau),&\quad\lambda\in\mathbb{C}\setminus(D^+(\delta)\cup D^-(\delta)\cup\Sigma_g).
%\end{cases}
%\label{eq:global-parametrix-Schi-Stau}
%\end{equation}
%\textcolor{red}{Alternatively, this reads:
\begin{equation}
\dot{\mathbf{W}}^{(k)}(\lambda;\chi,\tau)\defeq\begin{cases}
\dot{\mathbf{W}}^{(k),\lambda_0}(\lambda;\chi,\tau),&\quad\lambda\in D_{\lambda_0}(\delta),\\
\sigma_2\dot{\mathbf{W}}^{(k),\lambda_0}(\lambda^*;\chi,\tau)^*\sigma_2,&\quad\lambda\in D_{\lambda_0^*}(\delta),\\
\dot{\mathbf{W}}^{(k),\mathrm{out}}(\lambda;\chi,\tau),&\quad\lambda\in\mathbb{C}\setminus(\overline{D_{\lambda_0}(\delta)\cup D_{\lambda_0^*}(\delta)}\cup\Sigma_g).
\end{cases}
\label{eq:global-parametrix-Schi-Stau-ALT}
\end{equation}
%}

\subsection{Error analysis and asymptotic formula for $\psi_k(M\chi,M\tau)$ for $(\chi,\tau)\in \exterior$.}
As in Section~\ref{sec:small-norm-channels}, we define an error matrix to compare $\mathbf{W}^{(k)}(\lambda;\chi,\tau)$ with its global parametrix defined in \eqref{eq:global-parametrix-Schi-Stau-ALT}:
\begin{equation}
\mathbf{F}^{(k)}(\lambda;\chi,\tau):=\mathbf{W}^{(k)}(\lambda;\chi,\tau)\dot{\mathbf{W}}^{(k)}(\lambda;\chi,\tau)^{-1}.
\end{equation}
This matrix can be considered to be analytic in $\lambda$ except on a contour $\Sigma_\mathbf{F}$ consisting of the union of (i) those arcs of the jump contour for $\mathbf{W}^{(k)}(\lambda;\chi,\tau)$ other than $\Sigma_g$ outside the disks $D_{\lambda_0}(\delta)$ and $D_{\lambda_0^*}(\delta)$ and (ii) the disk boundaries $\partial D_{\lambda_0}(\delta)$ and $\partial D_{\lambda_0^*}(\delta)$, which we take to have clockwise orientation.  Also, $\mathbf{F}^{(k)}(\lambda;\chi,\tau)$ takes continuous boundary values on $\Sigma_\mathbf{F}$ from each connected component of $\mathbb{C}\setminus\Sigma_\mathbf{F}$, and $\mathbf{F}^{(k)}(\lambda;\chi,\tau)\to\mathbb{I}$ as $\lambda\to\infty$. Because $\delta>0$ is held fixed as $M\to\infty$ and because the outer parametrix is uniformly bounded on arcs of type (i), there is a constant $\nu>0$ such that on those arcs we have the uniform estimate $\mathbf{F}^{(k)}_+(\lambda;\chi,\tau)=\mathbf{F}^{(k)}_-(\lambda;\chi,\tau)(\mathbb{I}+O(\ee^{-\nu M}))$.  On the circular arcs of type (ii), the estimate \eqref{eq:Dplus-mismatch-Schi-Stau-ALT} and its Schwarz reflection guarantee that on those arcs we have the uniform estimate $\mathbf{F}^{(k)}_+(\lambda;\chi,\tau)=\mathbf{F}^{(k)}_-(\lambda;\chi,\tau)(\mathbb{I}+O(M^{-1}))$. By small-norm theory it then follows that $\mathbf{F}^{(k)}_-(\lambda;\chi,\tau)=\mathbb{I}+O(M^{-1})$ holds in the $L^2$ sense on the union of arcs of types (i) and (ii) as $M\to+\infty$.  Using the Cauchy integral representation \eqref{eq:F-Cauchy-channels} then shows that $\mathbf{F}^{(k)}(\lambda;\chi,\tau)=\mathbb{I}+\lambda^{-1}\mathbf{F}_1^{(k)}(\chi,\tau) + O(\lambda^{-2})$ as $\lambda\to\infty$ where $\mathbf{F}_1^{(k)}(\chi,\tau)=O(M^{-1})$ holds uniformly for $(\chi,\tau)$ in compact subsets of $\exterior$.

Using the fact that for $|\lambda|$ sufficiently large, 
%$\mathbf{S}^{(k)}(\lambda;\chi,\tau)=\mathbf{X}^{(k)}(\lambda;\chi,\tau)\ee^{-ng(\lambda;\chi,\tau)\sigma_3}\ee^{-S(\lambda;\chi,\tau)\sigma_3}$ while $\dot{\mathbf{X}}^{(k)}(\lambda;\chi,\tau)=\dot{\mathbf{X}}^{(k),\mathrm{out}}(\lambda;\chi,\tau)$ 
%\textcolor{red}{(or, 
$\mathbf{S}^{(k)}(\lambda;\chi,\tau)=\mathbf{W}^{(k)}(\lambda;\chi,\tau)\ee^{-\ii Mg(\lambda;\chi,\tau)\sigma_3}$ while $\dot{\mathbf{W}}^{(k)}(\lambda;\chi,\tau)=\dot{\mathbf{W}}^{(k),\mathrm{out}}(\lambda;\chi,\tau)$,
%)}, 
%the analogue of \eqref{eq:psi-k-exact-channels} is 
from \eqref{eq:psi-k-S} we have the following exact formula
%\begin{multline}
%\psi_k(n\chi,n\tau)=2\ii\ee^{-\ii n\tau}\lim_{\lambda\to\infty}\lambda X^{(k)}_{12}(\lambda;\chi,\tau)\ee^{ng(\lambda;\chi,\tau)}\ee^{S(\lambda;\chi,\tau)}\\
%{}=2\ii\ee^{-\ii n\tau}\lim_{\lambda\to\infty}\lambda\left[F^{(k)}_{11}(\lambda;\chi,\tau)\dot{X}^{(k),\mathrm{out}}_{12}(\lambda;\chi,\tau)+F^{(k)}_{12}(\lambda;\chi,\tau)\dot{X}^{(k),\mathrm{out}}_{22}(\lambda;\chi,\tau)\right]\ee^{ng(\lambda;\chi,\tau)}\ee^{S(\lambda;\chi,\tau)}.
%\end{multline}
%\textcolor{red}{Alternatively, this reads:
\begin{equation}
\begin{split}
\psi_k(M\chi,M\tau)&=2\ii\ee^{-\ii M\tau}\lim_{\lambda\to\infty}\lambda W^{(k)}_{12}(\lambda;\chi,\tau)\ee^{\ii Mg(\lambda;\chi,\tau)}\\
&=2\ii\ee^{-\ii M\tau}\lim_{\lambda\to\infty}\lambda\left[F^{(k)}_{11}(\lambda;\chi,\tau)\dot{W}^{(k),\mathrm{out}}_{12}(\lambda;\chi,\tau)\right.\\
&\qquad\qquad\qquad\qquad\qquad\qquad{}\left.+F^{(k)}_{12}(\lambda;\chi,\tau)\dot{W}^{(k),\mathrm{out}}_{22}(\lambda;\chi,\tau)\right]\ee^{\ii Mg(\lambda;\chi,\tau)}.
\end{split}
\end{equation}
%}
%Since $\mathbf{F}^{(k)}(\lambda;\chi,\tau)\to\mathbb{I}$, $\dot{\mathbf{X}}^{(k),\mathrm{out}}(\lambda;\chi,\tau)\to\mathbb{I}$, $g(\lambda;\chi,\tau)\to 0$, and $S(\lambda;\chi,\tau)\to 0$ as $\lambda\to\infty$, this simplifies to
%\begin{equation}
%\psi_k(n\chi,n\tau)=2\ii\ee^{-\ii n\tau}\lim_{\lambda\to\infty}\lambda\left[\dot{X}^{(k),\mathrm{out}}_{12}(\lambda;\chi,\tau) +F^{(k)}_{12}(\lambda;\chi,\tau)\right].
%\end{equation}
%\textcolor{red}{Alternately, since 
Since $\mathbf{F}^{(k)}(\lambda;\chi,\tau)\to\mathbb{I}$, $\dot{\mathbf{W}}^{(k),\mathrm{out}}(\lambda;\chi,\tau)\to\mathbb{I}$, and $g(\lambda;\chi,\tau)\to 0$ as $\lambda\to\infty$, this simplifies to
\begin{equation}
\psi_k(M\chi,M\tau)=2\ii\ee^{-\ii M\tau}\lim_{\lambda\to\infty}\lambda\left[\dot{W}^{(k),\mathrm{out}}_{12}(\lambda;\chi,\tau) +F^{(k)}_{12}(\lambda;\chi,\tau)\right].
\end{equation}
%}
%Using \eqref{eq:outer-parametrix-Schi-Stau} and that $\mathbf{F}_1^{(k)}(\chi,\tau)=O(n^{-1})$,
%recalling $B(\chi,\tau)=\mathrm{Im}(\lambda_0(\chi,\tau))>0$ we obtain
%\begin{equation}
%\psi_k(n\chi,n\tau)=-\ii s\ee^{-\ii n\tau}\ee^{-2\ii (n\kappa(\chi,\tau)+s\gamma(\chi,\tau))}B(\chi,\tau) + O(n^{-1}).
%\label{eq:psi-k-shelves-chi-tau}
%\end{equation}
%\textcolor{red}{Alternately, using 
Using \eqref{eq:outer-parametrix-Schi-Stau-ALT} and that $\mathbf{F}_1^{(k)}(\chi,\tau)=O(M^{-1})$, recalling $B(\chi,\tau)=\mathrm{Im}(\lambda_0(\chi,\tau))>0$ we obtain 
\begin{equation}
\psi_k(M\chi,M\tau)=B(\chi,\tau)\ee^{-\ii M\tau}\ee^{-2\ii M\gamma(\chi,\tau)} + O(M^{-1}),
\label{eq:psi-k-shelves-chi-tau-ALT}
\end{equation}
which completes the proof of Theorem~\ref{thm:exterior}.
%}

\section{Far-Field Asymptotic Behavior in the Domain $\shelves$}
In this section, we prove Theorem~\ref{thm:shelves} and its corollaries. 
%As indicated in the statement of theorem, 
Our analysis is valid for all $M\in \mathbb{Z}_{>0}$, with $\mathbf{G}=\mathbf{Q}^{-s}$, $s=\pm 1$ in contrast to that in the preceding section. The analysis will be guided by the sign chart of $\Re(\ii h(\lambda;\chi,\tau))$, $h(\lambda;\chi,\tau) = g(\lambda;\chi,\tau) +  \vartheta(\lambda;\chi,\tau)$.
Recall from the discussion in Section~\ref{sec:GenusZeroModification} that for $(\chi,\tau)\in\shelves$, $h'(\lambda;\chi,\tau)^2$ has 2 real double roots denoted by $a(\chi,\tau)<b(\chi,\tau)$, and two simple roots $A(\chi,\tau)\pm \ii B(\chi,\tau)$ for which we write $\lambda_0(\chi,\tau)\defeq  A(\chi,\tau) + \ii B(\chi,\tau)$, where $B(\chi,\tau)>0$, and we also have $A(\chi,\tau)<0$ because the endpoints $A(\chi,\tau) \pm \ii B(\chi,\tau)$ of $\Sigma_g$, along which $g(\lambda;\chi,\tau)$ has a jump discontinuity, lie in the left half-plane for all $(\chi,\tau)$ in the interior of $\shelves$. We recall from the beginning of Section~\ref{sec:zero-level-curve} that $\mathrm{Re}(\ii h(\lambda;\chi,\tau))=0$ holds for all $\lambda\in\mathbb{R} \setminus \Sigma_g$. Therefore, both $h_{-}(a(\chi,\tau);\chi,\tau)$ and $h(b(\chi,\tau);\chi,\tau)$ are real-valued. 
We also note the facts
\begin{equation}
h''_-(a(\chi,\tau);\chi,\tau) < 0 \quad\text{and}\quad h''(b(\chi,\tau);\chi,\tau) > 0,
\label{eq:h-double-prime-a-b-signs}
\end{equation}
which follow from the formula \eqref{eq:hprime-formula} since $\tau>0$ and $a(\chi,\tau),b(\chi,\tau)$ are real roots of the quadratic in the numerator of \eqref{eq:hprime-formula} for $(\chi,\tau)$ in the interior of $\shelves$.
%given in the introduction, taking into account that \tau>0 and that the quadratic has real roots at a and b on \shelves.

%Note that the unique real root $u=u(\chi,\tau)$ of $P(u;\chi,\tau)$ with odd multiplicity (simple except at a set of finitely many points) satisfies $u(\chi,0)=\chi$ and since $P(0;\chi,\tau)=-16 \tau^2 \tau^3 \neq 0$ in the open fist quadrant of the $(\chi,\tau)$-plane, $u(\chi,\tau)>0$ by continuity for $(\chi,\tau)\in \mathbb{R}_{>0} \times \mathbb{R}_{>0}$. As a matter of fact, $u(\chi,\tau)>\chi/3$ for $(\chi,\tau)$ in the open first quadrant by continuity since $P(\chi/3;\chi,\tau)=-(32/27)\chi^3\tau^4 \neq 0$ off the axes. Thus, the roots $a(\chi,\tau)$ and $b(\chi,\tau)$ of the quadratic \eqref{eq:quadratic-a-b} satisfy $a(\chi,\tau)+b(\chi,\tau)<0$, hence $a(\chi,\tau)<0$ for $(\chi,\tau)$ in $\shelves$ in the first quadrant. However, $b(\chi,\tau)$ is not sign-definite, hence the reason for allowing $\Sigma_\mathrm{c}$ to be a general Schwarz symmetric arc connecting the points $\lambda=\pm \ii$.  


Recall from Section~\ref{sec:zero-level-curve} that there is a Schwarz-symmetric arc of the zero level curve of $\lambda\mapsto \Re(\ii h(\lambda;\chi,\tau))$ that connects $\lambda_0(\chi,\tau)$ to $\lambda_0(\chi,\tau)^*$ and passes through the point $a(\chi,\tau)\in\mathbb{R}$. We place the branch cut $\Sigma_g$ on this curve, denote by $\Sigma_g^{\pm}$ its subarcs that lie in the half-planes $\mathbb{C}^{\pm}$, and orient $\Sigma_g$ from $\lambda_0^*$ to $\lambda_0$. The other bounded trajectory of the zero level curve in the upper half plane is one that connects $\lambda_0(\chi,\tau)$ to $b(\chi,\tau)$. We denote this arc by $\Gamma^+ = \Gamma^+(\chi,\tau)$, and its Schwarz reflection by $\Gamma^- = \Gamma^-(\chi,\tau)$, both with downward orientation. We also set $\Gamma\defeq  \Gamma^+ \cup \Gamma^- \cup \{  b(\chi,\tau) \}$. We set $I\defeq [a(\chi,\tau), b(\chi,\tau)]$ to
denote the only remaining bounded component of $\Re(\ii h(\lambda;\chi,\tau))=0$.
For the analysis that follows we take $\Sigma_\circ$ to be the clockwise-oriented loop $\Sigma_{\circ} = \Sigma_g \cup \Gamma$. We choose $\Sigma_\mathrm{c}$  to be a Schwarz-symmetric arc that connects the points $\lambda=\pm \ii$ while passing through the point $\lambda=\tfrac{1}{2}(a(\chi,\tau)+b(\chi,\tau))$, say (any point in $(a(\chi,\tau),b(\chi,\tau))$ would suffice), with upward orientation. See the left-hand panel of Figure~\ref{fig:SB1} for an illustration of these arcs.




%For $(\chi,\tau)\in \shelves$, we begin with deforming the circle $\Sigma_\circ$ to a loop which is a subset of the level curve 



%\textcolor{red}{Describe the properties of $h(\lambda;\chi,\tau)\defeq  g(\lambda;\chi,\tau) + \ii \vartheta(\lambda;\chi, \tau)$ here once more is in the regions with the neck contour $N$. $\lambda_0(\chi,\tau)$ and $\lambda_0(\chi,\tau)^*$ are the two distinct simple roots of $R(\lambda;\chi,\tau)^2$ for $(\chi,\tau)$ in the burger bun. Denote by $a(\chi,\tau) < b(\chi,\tau)$ the two remaining distinct (real) \emph{double} roots of $h'(\lambda;\chi,\tau)^2$, which are the two simple real roots of
%\begin{equation}
%2\tau \lambda^2 + u(\chi,\tau) \lambda + v(\chi,\tau) =0.
%\label{eq:quadratic-a-b}
%\end{equation}
%Then go on to describe the contours $\Gamma=\Gamma^+ \cup \Gamma^{-}$, and $\Sigma^{\pm}_g$ with the regions they separate, set $I\defeq [a(\chi,\tau), b(\chi,\tau)]$. One needs an argument to guarantee that the branch points $\lambda \pm \ii$ are contained in $\Omega^{\pm}$, which can be done if $\lambda_0(\chi,\tau)\neq \ii$.
%}
%In this section, we prove Theorem~\ref{}. As indicated in the statement of theorem, our analysis is valid for all $M\in \mathbb{Z}_{>0}$, with $\mathbf{G}=\mathbf{Q}^{-s}$, $s=(-1)^k$ in contrast to the result in the preceding section. \textcolor{red}{[Rogue waves and solitons being special cases mentioned here or in the introduction?]}. 
%For $(\chi,\tau)\in \shelves$, we begin with deforming the circle $\Sigma_\circ$ to a loop which is a subset of the level curve 


%For the calculation in this region, we assume that initially the Jordan curve $\Sigma_\circ$ contains $\Gamma \cup \Sigma_g$ in its interior and we make the substitution
%\begin{equation}
%\tilde{\mathbf{S}}(\lambda ; \chi, \tau,\mathbf{Q}^{-s},M)\defeq \mathbf{S}(\lambda ; \chi, \tau,\mathbf{Q}^{-s},M) \ee^{-\ii M \vartheta(\lambda ; \chi, \tau) \sigma_3} \mathbf{Q}^{-s} \ee^{ \ii M \vartheta (\lambda ; \chi, \tau) \sigma_3}
%\end{equation}
%for $\lambda$ between $\Sigma_0$ and $\Gamma \cup \Sigma_g$, and we set $\tilde{\mathbf{S}}(\lambda ; \chi, \tau,\mathbf{Q}^{-s},M)\defeq \mathbf{S}(\lambda ; \chi, \tau,\mathbf{Q}^{-s},M)$ everywhere else, i.e., in the exterior of $\Sigma_\circ$ and in the interior of $\Gamma \cup \Sigma_g$; and we drop the tilde. Then the jump contour for $\mathbf{S}(\lambda ; \chi, \tau,\mathbf{Q}^{-s},M)$ becomes $\Gamma \cup \Sigma_g$ and along this contour $\mathbf{S}(\lambda ; \chi, \tau,\mathbf{Q}^{-s},M)$ satisfies the same jump condition \eqref{eq:S-jump} as on $\Sigma_\circ$ prior to the substitution.

\begin{figure}[h]
\begin{center}
\includegraphics{SB1.pdf}
\end{center}
\caption{Left:  the initial jump contour and sign chart of $\mathrm{Re}(\ii h(\lambda;\chi,\tau))$ (shaded for negative, unshaded for positive) for $(\chi,\tau)=(2,0.8)\in \shelves$, showing also the regions where explicit transformations are made.  Right:  the resulting jump contour after the transformations.}
\label{fig:SB1}
\end{figure}

\subsection{Introduction of $g$ and steepest descent deformation of the Riemann-Hilbert problem}
With contours chosen this way, we have $\Sigma_g \cup \Sigma_\mathrm{c} = \emptyset$; therefore, we need the version of the construction of $g(\lambda;\chi,\tau)$ described in Section~\ref{sec:g-function-loop}. We introduce the $g$-function and the matrix function $\mathbf{T}(\lambda;\chi,\tau,\mathbf{Q}^{-s}, M)$ by the global substitution \eqref{eq:T-to-S}.
We let $\Omega^+$ denote the domain enclosed by $\Sigma_g^+ \cup \Gamma^+ \cup I $ which contains $\lambda=\ii$ and $\Sigma_\mathrm{c} \cap \mathbb{C}^+$, and let $\Omega^{-}$ be its Schwarz reflection. 
We let $L_\Sigma^\pm$ and $L_\Gamma^\pm$ (resp., $R_\Sigma^\pm$ and $R_\Gamma^\pm$) denote lens-shaped regions lying to the left (resp., right) of $\Sigma_g^\pm$ and $\Gamma^\pm$ with respect to orientation, as depicted in the left-hand panel of Figure~\ref{fig:SB1}. 
The lens-shaped regions are chosen to be so thin as to exclude the points $\lambda=\pm \ii$ and $\Sigma_\mathrm{c}$ while supporting a fixed sign of $\Re(\ii h(\lambda;\chi,\tau))$. 
On $\Gamma^\pm$, we will use the two-factor factorizations of the central factor $\mathbf{Q}^{-s}$, exactly as written in \eqref{eq:Q-factorizations}.
%\begin{equation}
%\mathbf{Q}^{-s} = 2^{\frac{1}{2}\sigma_3} \begin{bmatrix} 1 & \frac{s}{2} \\ 0 & 1 \end{bmatrix} \begin{bmatrix} 1 & 0 \\ -s & 1 \end{bmatrix},
%\end{equation}
%which we exploit in the part $\Gamma^+$ of $\Gamma$ in the upper half-plane, and
%\begin{equation}
%\mathbf{Q}^{-s} = 2^{-\frac{1}{2}\sigma_3} \begin{bmatrix} 1 & 0 \\ -\frac{s}{2} & 1 \end{bmatrix} \begin{bmatrix} 1 & s \\ 0 & 1 \end{bmatrix}
%\end{equation}
%which we exploit in the part $\Gamma^-$ of $\Gamma$ in the lower half-plane.
However, along $\Sigma_g$, we will employ the following additional factorizations
\begin{equation}
\mathbf{Q}^{-s} = 
\begin{cases}
2^{\frac{1}{2}\sigma_3} \begin{bmatrix} 1 & -\frac{1}{2}s \\ 0 & 1 \end{bmatrix} \begin{bmatrix} 0 & s \\ -s & 0 \end{bmatrix} \begin{bmatrix} 1 & -s \\ 0 & 1 \end{bmatrix}, \quad& \lambda\in \Sigma_g^+,\vspace{0.33em}\\
 2^{-\frac{1}{2}\sigma_3} \begin{bmatrix} 1 & 0 \\ \frac{1}{2}s & 1 \end{bmatrix} \begin{bmatrix} 0 & s \\ -s & 0 \end{bmatrix} \begin{bmatrix} 1 & 0 \\ s & 1 \end{bmatrix},\quad& \lambda\in \Sigma_g^-.
\end{cases}
\label{eq:Q-triple-factorization}
\end{equation}
%On the other hand, $\mathbf{Q}^{-s}$ also admits the following two three-factor factorizations
%\begin{equation}
%\mathbf{Q}^{-s} = 2^{\sigma_3/2} \begin{bmatrix} 1 & -\frac{s}{2} \\ 0 & 1 \end{bmatrix} \begin{bmatrix} 0 & s \\ -s & 0 \end{bmatrix} \begin{bmatrix} 1 & -s \\ 0 & 1 \end{bmatrix},
%\end{equation}
%which we will employ on the subarc $\Sigma_g^+$ of $\Sigma_g$ lying in the upper half-plane, and
%\begin{equation}
%\mathbf{Q}^{-s}  =  2^{-\sigma_3/2} \begin{bmatrix} 1 & 0 \\ \frac{s}{2} & 1 \end{bmatrix} \begin{bmatrix} 0 & s \\ -s & 0 \end{bmatrix} \begin{bmatrix} 1 & 0 \\ s & 1 \end{bmatrix},
%\end{equation}
%which we will employ on the subarc $\Sigma_g^{-}$ of $\Sigma_g$ lying in the lower half-plane. \textcolor{red}{[These have been used before, so cf. them.]}
%Taking advantage of the two-factor matrix factorizations \eqref{eq:Q-factorizations} along with the domains in which $\Re(\ii h(\lambda;\chi,\tau))$ is sign-definite, we define $\mathbf{W}(\lambda)=\mathbf{W}(\lambda;\chi,\tau,\mathbf{Q}^{-s},M)$ in the lens-shaped regions surrounding $\Gamma_\pm$ by:
Taking advantage of the two-factor matrix factorizations \eqref{eq:Q-factorizations}, we define $\mathbf{W}(\lambda)=\mathbf{W}(\lambda;\chi,\tau,\mathbf{Q}^{-s},M)$ in the lens-shaped regions surrounding $\Gamma^\pm$ by:
%making the following substitutions to control the exponential factors present in \eqref{eq:T-jump}:
%\begin{equation}
%\mathbf{W}^{(k)}(\lambda;\chi,\tau)\defeq \mathbf{T}^{(k)}(\lambda;\chi,\tau) \begin{bmatrix} 1 & 0 \\ s \omega(\lambda)^{-2s}\ee^{2n h(\lambda;\chi,\tau)}& 1 \end{bmatrix},\quad \lambda\in L^+_{\Gamma},
%\label{eq:T-to-W-L-plus-Gamma}
%\end{equation}
%\begin{equation}
%\mathbf{W}^{(k)}(\lambda;\chi,\tau)\defeq \mathbf{T}^{(k)}(\lambda;\chi,\tau)
%2^{\sigma_3/2} \begin{bmatrix} 1 & \frac{s}{2} \omega(\lambda)^{2s}\ee^{-2n h(\lambda;\chi,\tau)} \\ 0 & 1 \end{bmatrix},\quad \lambda \in R^+_{\Gamma},
%\end{equation}

%\begin{equation}
%\mathbf{W}^{(k)}(\lambda;\chi,\tau)\defeq \mathbf{T}^{(k)}(\lambda;\chi,\tau) \begin{bmatrix} 1 & 0 \\ s \ee^{2M h(\lambda;\chi,\tau)}& 1 \end{bmatrix},\quad \lambda\in L^+_{\Gamma},
%\label{eq:T-to-W-L-plus-Gamma}
%\end{equation}
%\begin{equation}
%\mathbf{W}^{(k)}(\lambda;\chi,\tau)\defeq \mathbf{T}^{(k)}(\lambda;\chi,\tau)
%2^{\frac{\sigma_3}{2}} \begin{bmatrix} 1 & \frac{s}{2} \ee^{-2M h(\lambda;\chi,\tau)} \\ 0 & 1 \end{bmatrix},\quad \lambda \in R^+_{\Gamma},
%\end{equation}
\begin{equation}
\mathbf{W}(\lambda)\defeq \mathbf{T}(\lambda;\chi,\tau,\mathbf{Q}^{-s},M) 
\begin{bmatrix} 1 & 0 \\ s \ee^{2\ii M h(\lambda;\chi,\tau)}& 1 \end{bmatrix},\quad \lambda\in L^+_{\Gamma},
\label{eq:T-to-W-L-plus-Gamma}
\end{equation}
\begin{equation}
\mathbf{W}(\lambda)\defeq \mathbf{T}(\lambda;\chi,\tau,\mathbf{Q}^{-s},M) 
2^{\frac{1}{2}\sigma_3} \begin{bmatrix} 1 & \frac{1}{2} s\ee^{-2\ii M h(\lambda;\chi,\tau)} \\ 0 & 1 \end{bmatrix},\quad \lambda \in R^+_{\Gamma},
\end{equation}
%
%\begin{equation}
%\mathbf{W}^{(k)}(\lambda;\chi,\tau)\defeq \mathbf{T}^{(k)}(\lambda;\chi,\tau)
%2^{\frac{1}{2}\sigma_3},\quad \lambda \in \Omega^+,
%\end{equation}
%\begin{equation}
%\mathbf{W}^{(k)}(\lambda;\chi,\tau)\defeq \mathbf{T}^{(k)}(\lambda;\chi,\tau)
%2^{-\frac{1}{2}\sigma_3},\quad \lambda \in \Omega^-,
%\end{equation}
\begin{equation}
\mathbf{W}(\lambda)\defeq \mathbf{T}(\lambda;\chi,\tau,\mathbf{Q}^{-s},M) 
2^{\frac{1}{2}\sigma_3},\quad \lambda \in \Omega^+,
\end{equation}
\begin{equation}
\mathbf{W}(\lambda)\defeq \mathbf{T}(\lambda;\chi,\tau,\mathbf{Q}^{-s},M) 
2^{-\frac{1}{2}\sigma_3},\quad \lambda \in \Omega^-,
\end{equation}
\begin{equation}
\mathbf{W}(\lambda)\defeq \mathbf{T}(\lambda;\chi,\tau,\mathbf{Q}^{-s},M) 
2^{-\sigma_3/2}  \begin{bmatrix} 1 & 0 \\ -\frac{1}{2}s \ee^{2\ii M h(\lambda;\chi,\tau)} & 1 \end{bmatrix} ,\quad \lambda \in R^-_{\Gamma},\quad\text{and}
\end{equation}
\begin{equation}
\mathbf{W}(\lambda)\defeq \mathbf{T}(\lambda;\chi,\tau,\mathbf{Q}^{-s},M) 
 \begin{bmatrix} 1 & - s \ee^{-2\ii M h(\lambda;\chi,\tau)} \\ 0 & 1 \end{bmatrix} ,\quad \lambda \in L^-_{\Gamma}.
\end{equation}
Note how the definitions above compare with \eqref{eq:W-def-Schi-Stau-Lplus-ALT}--\eqref{eq:W-def-Schi-Stau-Lminus-ALT} which use the same factorizations of $\mathbf{Q}^{-s}$: the regions $L^\pm$ and $R^\pm$ are merely replaced with $L_\Gamma^\pm$ and $R_\Gamma^\pm$, respectively. In the lens-shaped regions surrounding $\Sigma_g$ on the other hand, we make use of the factorizations \eqref{eq:Q-triple-factorization} and define:
%\begin{equation}
%\mathbf{W}^{(k)}(\lambda;\chi,\tau)\defeq \mathbf{T}^{(k)}(\lambda;\chi,\tau)
%2^{-\frac{1}{2}\sigma_3}  \begin{bmatrix} 1 & 0 \\ -\frac{s}{2}  \ee^{2M h(\lambda;\chi,\tau)} & 1 \end{bmatrix} ,\quad \lambda \in R^-_{\Gamma},
%\end{equation}
%\begin{equation}
%\mathbf{W}^{(k)}(\lambda;\chi,\tau)\defeq \mathbf{T}^{(k)}(\lambda;\chi,\tau)
% \begin{bmatrix} 1 & - s \ee^{-2M h(\lambda;\chi,\tau)} \\ 0 & 1 \end{bmatrix} ,\quad \lambda \in L^-_{\Gamma},
%\end{equation}
%\begin{equation}
%\mathbf{W}^{(k)}(\lambda;\chi,\tau)\defeq \mathbf{T}^{(k)}(\lambda;\chi,\tau)
% \begin{bmatrix} 1 & s \omega(\lambda)^{2s} \ee^{-2n h(\lambda;\chi,\tau)} \\ 0 & 1 \end{bmatrix},\quad \lambda \in L^+_{\Sigma},
%\end{equation}
%\begin{equation}
%\mathbf{W}^{(k)}(\lambda;\chi,\tau)\defeq \mathbf{T}^{(k)}(\lambda;\chi,\tau)
%2^{\sigma_3/2} \begin{bmatrix} 1 & -\frac{s}{2}\omega(\lambda)^{2 s}\ee^{-2n h(\lambda;\chi,\tau)} \\ 0 & 1 \end{bmatrix},\quad \lambda \in R^+_{\Sigma},
%\end{equation}
%\begin{equation}
%\mathbf{W}^{(k)}(\lambda;\chi,\tau)\defeq \mathbf{T}^{(k)}(\lambda;\chi,\tau)
% \begin{bmatrix} 1 & 0 \\ -s \omega(\lambda)^{-2s} \ee^{2n h(\lambda;\chi,\tau)} & 1 \end{bmatrix},\quad \lambda \in L^-_{\Sigma},
%\end{equation}
%and
%\begin{equation}
%\mathbf{W}^{(k)}(\lambda;\chi,\tau)\defeq \mathbf{T}^{(k)}(\lambda;\chi,\tau)
% 2^{-\sigma_3/2} \begin{bmatrix} 1 & 0 \\ \frac{s}{2}\omega(\lambda)^{-2s}\ee^{2n h(\lambda;\chi,\tau)} & 1 \end{bmatrix}
%,\quad \lambda \in R^-_{\Sigma}.
%\label{eq:T-to-W-R-minus-Sigma}
%\end{equation}
%\begin{equation}
%\mathbf{W}^{(k)}(\lambda;\chi,\tau)\defeq \mathbf{T}^{(k)}(\lambda;\chi,\tau)
% \begin{bmatrix} 1 & s  \ee^{-2M h(\lambda;\chi,\tau)} \\ 0 & 1 \end{bmatrix},\quad \lambda \in L^+_{\Sigma},
%\end{equation}
%\begin{equation}
%\mathbf{W}^{(k)}(\lambda;\chi,\tau)\defeq \mathbf{T}^{(k)}(\lambda;\chi,\tau)
%2^{\frac{1}{2}\sigma_3} \begin{bmatrix} 1 & -\frac{s}{2} \ee^{-2M h(\lambda;\chi,\tau)} \\ 0 & 1 \end{bmatrix},\quad \lambda \in R^+_{\Sigma},
%\end{equation}
%\begin{equation}
%\mathbf{W}^{(k)}(\lambda;\chi,\tau)\defeq \mathbf{T}^{(k)}(\lambda;\chi,\tau)
% \begin{bmatrix} 1 & 0 \\ -s  \ee^{2M h(\lambda;\chi,\tau)} & 1 \end{bmatrix},\quad \lambda \in L^-_{\Sigma},
%\end{equation}
\begin{equation}
\mathbf{W}(\lambda)\defeq \mathbf{T}(\lambda;\chi,\tau,\mathbf{Q}^{-s},M) 
 \begin{bmatrix} 1 & s  \ee^{-2\ii M h(\lambda;\chi,\tau)} \\ 0 & 1 \end{bmatrix},\quad \lambda \in L^+_{\Sigma},
\end{equation}
\begin{equation}
\mathbf{W}(\lambda)\defeq \mathbf{T}(\lambda;\chi,\tau,\mathbf{Q}^{-s},M) 
2^{\frac{1}{2}\sigma_3} \begin{bmatrix} 1 & -\frac{1}{2} s \ee^{-2\ii M h(\lambda;\chi,\tau)} \\ 0 & 1 \end{bmatrix},\quad \lambda \in R^+_{\Sigma},
\end{equation}
\begin{equation}
\mathbf{W}(\lambda)\defeq \mathbf{T}(\lambda;\chi,\tau,\mathbf{Q}^{-s},M) 
 \begin{bmatrix} 1 & 0 \\ -s  \ee^{2\ii M h(\lambda;\chi,\tau)} & 1 \end{bmatrix},\quad \lambda \in L^-_{\Sigma},\quad\text{and}
\end{equation}
\begin{equation}
\mathbf{W}(\lambda)\defeq \mathbf{T}(\lambda;\chi,\tau,\mathbf{Q}^{-s},M) 
 2^{-\frac{1}{2}\sigma_3} \begin{bmatrix} 1 & 0 \\ \frac{1}{2} s \ee^{2\ii M h(\lambda;\chi,\tau)} & 1 \end{bmatrix}
,\quad \lambda \in R^-_{\Sigma}.
\label{eq:T-to-W-R-minus-Sigma-ALT}
\end{equation}
We simply leave $\mathbf{W}(\lambda) \defeq  \mathbf{T}(\lambda;\chi,\tau,\mathbf{Q}^{-s},M) $ elsewhere. It is now easy to see that $\mathbf{W}(\lambda)$ extends to $\lambda\in \Gamma^+ \cup \Gamma^-$ as an analytic function, so that $\mathbf{W}(\lambda)$ is analytic in the complement of the jump contour $C^{+}_{\Sigma,L} \cup C^{+}_{\Sigma,R} \cup C^{-}_{\Sigma,L} \cup C^{-}_{\Sigma,R} \cup \Sigma_g \cup I \cup C^{+}_{\Gamma,L} \cup C^{+}_{\Gamma,R} \cup C^{-}_{\Gamma,L} \cup C^{-}_{\Gamma,R}$, the arcs of which are depicted in the right-hand panel of Figure~\ref{fig:SB1}. Across these arcs $\mathbf{W}(\lambda) $ satisfies the following jump relations:
%See Figure~\ref{fig:SB1} for definitions of the above and the curves that separate them. Having done these substitutions, we see that $\mathbf{W}^{(k)}(\lambda;\chi,\tau) $ satisfies the following jump relations
%\begin{equation}
%\mathbf{W}^{(k)}_{+}(\lambda) = \mathbf{W}^{(k)}_{-}(\lambda)
%\begin{bmatrix} 1 & 0 \\ -s \omega(\lambda)^{-2s}\ee^{2n h(\lambda;\chi,\tau)}& 1 \end{bmatrix},\quad \lambda\in C^+_{\Gamma,L},
%\label{eq:W-jump-Gamma-plus-L}
%\end{equation}
%\begin{equation}
%\mathbf{W}^{(k)}_{+}(\lambda) = \mathbf{W}^{(k)}_{-}(\lambda)
%\begin{bmatrix} 1 & \frac{s}{2} \omega(\lambda)^{2s}\ee^{-2n h(\lambda;\chi,\tau)} \\ 0 & 1 \end{bmatrix},\quad \lambda\in C^+_{\Gamma,R},
%\end{equation}
%\begin{equation}
%\mathbf{W}^{(k)}_{+}(\lambda) = \mathbf{W}^{(k)}_{-}(\lambda)
%2^{\sigma_3},\quad \lambda\in I ,
%\end{equation}
%\begin{equation}
%\mathbf{W}^{(k)}_{+}(\lambda) = \mathbf{W}^{(k)}_{-}(\lambda)
%\begin{bmatrix} 1 & 0 \\ -\frac{s}{2} \omega(\lambda)^{-2s} \ee^{2n h(\lambda;\chi,\tau)} & 1 \end{bmatrix} ,\quad \lambda\in C^-_{\Gamma,R},
%\end{equation}
%\begin{equation}
%\mathbf{W}^{(k)}_{+}(\lambda) = \mathbf{W}^{(k)}_{-}(\lambda)
%\begin{bmatrix} 1 &  s  \omega(\lambda)^{2s} \ee^{-2n h(\lambda;\chi,\tau)} \\ 0 & 1 \end{bmatrix} ,\quad \lambda\in C^-_{\Gamma,L},
%\end{equation}
%\begin{equation}
%\mathbf{W}^{(k)}_{+}(\lambda) = \mathbf{W}^{(k)}_{-}(\lambda)
%\begin{bmatrix} 1 & -s \omega(\lambda)^{2s} \ee^{-2n h(\lambda;\chi,\tau)} \\ 0 & 1 \end{bmatrix},\quad \lambda\in C^+_{\Sigma,L},
%\end{equation}
%\begin{equation}
%\mathbf{W}^{(k)}_{+}(\lambda) = \mathbf{W}^{(k)}_{-}(\lambda)
%\begin{bmatrix} 1 & -\frac{s}{2}\omega(\lambda)^{2 s}\ee^{-2n h(\lambda;\chi,\tau)} \\ 0 & 1 \end{bmatrix},\quad \lambda\in C^+_{\Sigma,R},
%\end{equation}
%\begin{equation}
%\mathbf{W}^{(k)}_{+}(\lambda) = \mathbf{W}^{(k)}_{-}(\lambda)
%\begin{bmatrix} 1 & 0 \\ s \omega(\lambda)^{-2s} \ee^{2n h(\lambda;\chi,\tau)} & 1 \end{bmatrix},\quad \lambda\in C^-_{\Sigma,L},
%\end{equation}
%\begin{equation}
%\mathbf{W}^{(k)}_{+}(\lambda) = \mathbf{W}^{(k)}_{-}(\lambda)
%\begin{bmatrix} 1 & 0 \\ \frac{s}{2} \omega(\lambda)^{-2s} \ee^{2n h(\lambda;\chi,\tau)} & 1 \end{bmatrix},\quad \lambda\in C^-_{\Sigma,R},
%\label{eq:W-jump-Sigma-minus-R}
%\end{equation}
%\begin{equation}
%\mathbf{W}^{(k)}_{+}(\lambda;\chi,\tau) = \mathbf{W}^{(k)}_{-}(\lambda;\chi,\tau)
%\begin{bmatrix} 1 & 0 \\ -s \ee^{2M h(\lambda;\chi,\tau)}& 1 \end{bmatrix},\quad \lambda\in C^+_{\Gamma,L},
%\label{eq:W-jump-Gamma-plus-L}
%\end{equation}
%\begin{equation}
%\mathbf{W}^{(k)}_{+}(\lambda;\chi,\tau) = \mathbf{W}^{(k)}_{-}(\lambda;\chi,\tau)
%\begin{bmatrix} 1 & \frac{s}{2} \ee^{-2M h(\lambda;\chi,\tau)} \\ 0 & 1 \end{bmatrix},\quad \lambda\in C^+_{\Gamma,R},
%\end{equation}
%\begin{equation}
%\mathbf{W}^{(k)}_{+}(\lambda;\chi,\tau) = \mathbf{W}^{(k)}_{-}(\lambda;\chi,\tau)
%2^{\sigma_3},\quad \lambda\in I ,
%\label{eq:W-jump-I}
%\end{equation}
%\begin{equation}
%\mathbf{W}^{(k)}_{+}(\lambda;\chi,\tau) = \mathbf{W}^{(k)}_{-}(\lambda;\chi,\tau)
%\begin{bmatrix} 1 & 0 \\ -\frac{s}{2}  \ee^{2M h(\lambda;\chi,\tau)} & 1 \end{bmatrix} ,\quad \lambda\in C^-_{\Gamma,R},
%\end{equation}
%\begin{equation}
%\mathbf{W}^{(k)}_{+}(\lambda;\chi,\tau) = \mathbf{W}^{(k)}_{-}(\lambda;\chi,\tau)
%\begin{bmatrix} 1 &  s \ee^{-2M h(\lambda;\chi,\tau)} \\ 0 & 1 \end{bmatrix} ,\quad \lambda\in C^-_{\Gamma,L},
%\end{equation}
%\begin{equation}
%\mathbf{W}^{(k)}_{+}(\lambda;\chi,\tau) = \mathbf{W}^{(k)}_{-}(\lambda;\chi,\tau)
%\begin{bmatrix} 1 & -s  \ee^{-2M h(\lambda;\chi,\tau)} \\ 0 & 1 \end{bmatrix},\quad \lambda\in C^+_{\Sigma,L},
%\end{equation}
%\begin{equation}
%\mathbf{W}^{(k)}_{+}(\lambda;\chi,\tau) = \mathbf{W}^{(k)}_{-}(\lambda;\chi,\tau)
%\begin{bmatrix} 1 & -\frac{s}{2} \ee^{-2M h(\lambda;\chi,\tau)} \\ 0 & 1 \end{bmatrix},\quad \lambda\in C^+_{\Sigma,R},
%\end{equation}
%\begin{equation}
%\mathbf{W}^{(k)}_{+}(\lambda;\chi,\tau)= \mathbf{W}^{(k)}_{-}(\lambda;\chi,\tau)
%\begin{bmatrix} 1 & 0 \\ s  \ee^{2M h(\lambda;\chi,\tau)} & 1 \end{bmatrix},\quad \lambda\in C^-_{\Sigma,L},
%\end{equation}
%\begin{equation}
%\mathbf{W}^{(k)}_{+}(\lambda;\chi,\tau) = \mathbf{W}^{(k)}_{-}(\lambda;\chi,\tau)
%\begin{bmatrix} 1 & 0 \\ \frac{s}{2}  \ee^{2M h(\lambda;\chi,\tau)} & 1 \end{bmatrix},\quad \lambda\in C^-_{\Sigma,R},
%\label{eq:W-jump-Sigma-minus-R-ALT}
%\end{equation}
\begin{equation}
\mathbf{W}_{+}(\lambda) = \mathbf{W}_{-}(\lambda)
\begin{bmatrix} 1 & 0 \\ -s \ee^{2\ii M h(\lambda;\chi,\tau)}& 1 \end{bmatrix},\quad \lambda\in C^+_{\Gamma,L},
\label{eq:W-jump-Gamma-plus-L}
\end{equation}
\begin{equation}
\mathbf{W}_{+}(\lambda) = \mathbf{W}_{-}(\lambda)
\begin{bmatrix} 1 & \frac{1}{2} s\ee^{-2\ii M h(\lambda;\chi,\tau)} \\ 0 & 1 \end{bmatrix},\quad \lambda\in C^+_{\Gamma,R},
\end{equation}
\begin{equation}
\mathbf{W}_{+}(\lambda) = \mathbf{W}_{-}(\lambda)
\begin{bmatrix} 1 & 0 \\ -\frac{1}{2}s  \ee^{2\ii M h(\lambda;\chi,\tau)} & 1 \end{bmatrix} ,\quad \lambda\in C^-_{\Gamma,R},\quad\text{and}
\end{equation}
\begin{equation}
\mathbf{W}_{+}(\lambda) = \mathbf{W}_{-}(\lambda)
\begin{bmatrix} 1 &  s \ee^{-2\ii M h(\lambda;\chi,\tau)} \\ 0 & 1 \end{bmatrix} ,\quad \lambda\in C^-_{\Gamma,L},
\end{equation}
which are again in parallel with the jump conditions \eqref{eq:Wjump-exterior-CLplus}--\eqref{eq:Wjump-exterior-CLminus}, and
\begin{equation}
\mathbf{W}_{+}(\lambda) = \mathbf{W}_{-}(\lambda)
2^{\sigma_3},\quad \lambda\in I ,
\label{eq:W-jump-I}
\end{equation}
\begin{equation}
\mathbf{W}_{+}(\lambda) = \mathbf{W}_{-}(\lambda)
\begin{bmatrix} 1 & -s  \ee^{-2\ii M h(\lambda;\chi,\tau)} \\ 0 & 1 \end{bmatrix},\quad \lambda\in C^+_{\Sigma,L},
\end{equation}
\begin{equation}
\mathbf{W}_{+}(\lambda) = \mathbf{W}_{-}(\lambda)
\begin{bmatrix} 1 & -\frac{1}{2} s \ee^{-2\ii M h(\lambda;\chi,\tau)} \\ 0 & 1 \end{bmatrix},\quad \lambda\in C^+_{\Sigma,R},
\end{equation}
\begin{equation}
\mathbf{W}_{+}(\lambda) = \mathbf{W}_{-}(\lambda)
\begin{bmatrix} 1 & 0 \\ s  \ee^{2\ii M h(\lambda;\chi,\tau)} & 1 \end{bmatrix},\quad \lambda\in C^-_{\Sigma,L},\quad\text{and}
\end{equation}
\begin{equation}
\mathbf{W}_{+}(\lambda) = \mathbf{W}_{-}(\lambda)
\begin{bmatrix} 1 & 0 \\ \frac{1}{2} s  \ee^{2\ii M h(\lambda;\chi,\tau)} & 1 \end{bmatrix},\quad \lambda\in C^-_{\Sigma,R}.
\label{eq:W-jump-Sigma-minus-R-ALT}
\end{equation}
Finally, along the branch cut $\Sigma_g$ we have
%\begin{equation}
%\mathbf{W}^{(k)}_{+}(\lambda) = \mathbf{W}^{(k)}_{-}(\lambda)
%\begin{bmatrix} 0 & s \omega(\lambda)^{2s} \ee^{-2\ii n \kappa(\chi,\tau))} \\ -s \omega(\lambda)^{-2s} \ee^{2 \ii n \kappa(\chi,\tau)} & 0 \end{bmatrix},\quad \lambda\in \Sigma_g = \Sigma_g^+ \cup \Sigma_g^-,
%\label{eq:W-jump-Sigma}
%\end{equation}
\begin{equation}
\mathbf{W}_{+}(\lambda) = \mathbf{W}_{-}(\lambda)
\begin{bmatrix} 0 & s  \ee^{-2\ii M \kappa(\chi,\tau))} \\ -s  \ee^{2 \ii M \kappa(\chi,\tau)} & 0 \end{bmatrix},\quad \lambda\in \Sigma_g = \Sigma_g^+ \cup \Sigma_g^-,
\label{eq:W-jump-Sigma-g}
\end{equation}
where $2\kappa(\chi,\tau)$ is the real constant value of $h_{+}(\lambda;\chi,\tau)+h_{-}(\lambda;\chi,\tau)$ for $\lambda\in\Sigma_g$ and $\kappa(\chi,\tau)$ is given in \eqref{eq:kappa-formula}. 
%We reminder the reader that the formula \eqref{eq:kappa-formula} defines the real constant $\kappa(\chi,\tau)$ modulo $2\pi$ as long as the path of integration connecting $\lambda=\pm \ii$ avoids the branch cut $\Sigma_g$ of $R(\lambda;\chi,\tau)$, hence $\ee^{\ii n \kappa(\chi,\tau)}$ is well-defined as long as the path of integration avoids $\Sigma_g$ (see the passage following \eqref{eq:kappa-formula}).
It follows from the sign chart of $\Re(\ii h(\lambda;\chi,\tau))$ as shown in Figure~\ref{fig:SB1} that all of the jump matrices above except for those supported on $I\cup\Sigma_g$ tend to the identity matrix exponentially fast as $M\to +\infty$ for $\lambda$ on the relevant supporting arcs away from the points $\lambda = \lambda_{0}(\chi,\tau)$, $\lambda_{0}(\chi,\tau)^{*}$, $a(\chi,\tau)$ and $b(\chi,\tau)$. In sufficiently small neighborhoods of these points, we will construct local parametrices that satisfy the jump conditions exactly.


%Note that the unique real root $u=u(\chi,\tau)$ of $P(u;\chi,\tau)$ with odd multiplicity (simple except at a set of finitely many points) satisfies $u(\chi,0)=\chi$ and since $P(0;\chi,\tau)=-16 \tau^2 \tau^3 \neq 0$ in the open fist quadrant of the $(\chi,\tau)$-plane, $u(\chi,\tau)>0$ by continuity for $(\chi,\tau)\in \mathbb{R}_{>0} \times \mathbb{R}_{>0}$. \textcolor{red}{[As a matter of fact, $u(\chi,\tau)>\chi/3$ for $(\chi,\tau)$ in the open first quadrant by continuity since $P(\chi/3;\chi,\tau)=-(32/27)\chi^3\tau^4 \neq 0$ off the axes.]} Thus, the roots $a(\chi,\tau)$ and $b(\chi,\tau)$ of the quadratic \eqref{eq:quadratic-a-b} satisfy $a(\chi,\tau)+b(\chi,\tau)<0$, hence $a(\chi,\tau)<0$ for $(\chi,\tau)$ in $\shelves$ in the first quadrant. On the other hand, it is possible that $b(\chi,\tau)=0$ or $b(\chi,\tau)<0$ for $(\chi,\tau) \in \shelves$ in the first quadrant. Indeed, as $a(\chi,\tau)<b(\chi,\tau)$, we see from \eqref{eq:quadratic-a-b} that $b(\chi,\tau)=0$ if and only if $v=v(\chi,\tau)=0$, which is by \eqref{eq:eliminate-v} equivalent to the condition $u(\chi,\tau)=\chi/2$. We observe that when $u=\chi/2$, $P(u;\chi,\tau)$ factors
%\begin{equation}
%P(\chi/2; \chi,\tau) = -\frac{1}{128}\chi^3 (\chi^2 +16 \tau + 16\tau^2) (\chi^2 -16 \tau + 16\tau^2),
%\end{equation}
%and the condition relevant for $\tau>0$ upon enforcing $P(\chi/2;\chi,\tau)=0$ is $(\chi^2 -16 \tau + 16\tau^2)=0$, which is the ellipse
%\begin{equation}
%\chi^2 + 16\left(\tau-\tfrac{1}{2}\right)^2 = 4.
%\end{equation}
%The branch of this ellipse in the first quadrant can be expressed as:
%\begin{equation}
%\chi = 4 \sqrt{\tau(1-\tau)},
%\label{eq:b-is-zero}
%\end{equation}
%and note that we necessarily have $0\leq \tau \leq 1$. We claim that this curve lies below the light blue curve plotted in Figure~\ref{fig:RegionsPlot}. 
%To see this, recall that this curve is given by 
%\begin{equation}
%Z(\chi,\tau) \defeq   H_{10}(\chi,\tau) +  H_{8}(\chi,\tau) +  H_{6}(\chi,\tau) +  H_{4}(\chi,\tau) = 0,
%\end{equation}
%where $H_{k}(\chi,\tau)$ are the homogenous polynomials defined in \eqref{eq:H-polynomials}.
%%To see this, first recall that the light blue cure in Figure~\ref{fig:RegionsPlot} is precisely the vanishing of the discriminant of the quadratic \eqref{eq:quadratic-a-b}, and the branch of it that enters the quadrant from the point $(\chi,\tau)=(0,1)$ is the locus $a(\chi,\tau)=b(\chi,\tau)\in\mathbb{R}$ \textcolor{red}{[The other branch is also this locus if we correctly label the points there, two complex roots collide on the real line there.]}. Upon eliminating $v$ via \eqref{eq:eliminate-v} the vanishing of this discriminant is the condition
%%\begin{equation}
%%u^2 = 16 \tau^2\frac{2u-\chi}{3u-\chi}.
%%\end{equation}
%%Recalling that $u>\chi/3$ in the first quadrant and clearing the denominator gives the condition
%%\begin{equation}
%%3u^3 -u^2 \chi -32u \tau^2 + 16\tau^2 \chi =0.
%%\label{eq:condition-real-double-root}
%%\end{equation}
%%Taking the resultant of \eqref{eq:condition-real-double-root} with $P(u;\chi,\tau)=0$ with respect to $u$ yields
%%\begin{multline}
%%Z(\chi,\tau)\defeq 800000000 \tau^{10} - 68000000 \tau ^8 \chi ^2 - 3520000 \tau^6 \chi^4 - 56800 \tau^4 \chi^6 - 392 \tau^2 \chi^8 -\chi ^{10} \\
%%-730880000 \tau^8 + 125516800 \tau ^6 \chi^2 -1741728 \tau ^4 \chi ^4 - 1040 \tau^2 \chi^6 - 2 \chi^8 \\
%%- 67627008 \tau^6 - 3103488 \tau ^4 \chi^2  - 504 \tau^2 \chi^4 - \chi^6 -1492992 \tau^4  
%%= 0,
%%\label{eq:light-blue-curve}
%%\end{multline}
%Substituting \eqref{eq:b-is-zero} in $Z(\chi,\tau)$ and enforcing $Z(4\sqrt{\tau(1-\tau)},\tau) = 0$ yields the condition
%\begin{multline}
%2048 \tau^3 (\tau - 1)( 583443 \tau^6 + 639009 \tau^5 - 1083510 \tau^4 + 257994 \tau^3 + 27314 \tau^2 + 852 \tau + 2) = 0.
%\end{multline}
%Applying the theory of Sturm sequences (see Theorem~\ref{t:Sturm}) for the sextic polynomial in the last factor above shows that it has no positive roots. Thus, $Z(4\sqrt{\tau(1-\tau)},\tau) = 0$ only at $\tau=0$ or $\tau=1$.
%Furthermore, $Z(4\sqrt{\tau(1-\tau)},\tau)<0$ for $0<\tau<1$, and this proves the claim that the curve \eqref{eq:b-is-zero} lies below the curve $Z(\chi,\tau)=0$, while having $(\chi,\tau)=(0,1)$ to be its only point of intersection with this curve. On the other hand, the restriction of the ellipse $(\chi^2 -16 \tau + 16\tau^2)=0$ to the first quadrant can also be viewed as the union of the two branches
%\begin{equation}
%\tau - \frac{1}{2} = %\pm \frac{1}{4}\sqrt{4-\chi^2}=
%\pm \frac{1}{2}\sqrt{1-\left(\frac{\chi}{2}\right)^2},
%\label{eq:b-is-zero-branches}
%\end{equation}
%which are symmetric with respect to the horizontal line $\tau=\frac{1}{2}$. Note that the point $(\chi,\tau)=(2,\tfrac{1}{2})$ is the junction point of these branches, and it is easy to verify that this point also lies on the curve \eqref{eq:boundary-curve}, the boundary curve of the region $C$ which is shown in light red in Figure~\ref{fig:RegionsPlot}. It follows that the branch in \eqref{eq:b-is-zero-branches} with the top sign is contained in the region $B$ and the branch with the bottom sign is contained in the region $C$ \textcolor{red}{[I can give a proof of this if necessary]}, see Figure~\ref{fig:b-is-zero}.
%\begin{figure}
%\includegraphics[width=0.5\linewidth]{b-is-zero-plot.pdf}
%\caption{The first quadrant in the $(\chi,\tau)$-plane; horizontal axis:  $\chi$, vertical axis:  $\tau$. The branch $\tau-\tfrac{1}{2} = \frac{1}{2}\sqrt{1-({\chi}/{2})^2}$ (dashed orange) of $b(\chi,\tau)=0$ lying in the region $B$ and the branch $\tau-\tfrac{1}{2} = -\frac{1}{2}\sqrt{1-({\chi}/{2})^2}$ (dashed green) of $b(\chi,\tau)=0$ lying in the region $C$. The purple colored dot is the point $(2,\tfrac{1}{2})$.}
%\label{fig:b-is-zero}
%\end{figure}
%This implies in particular that for given $\tau\in(\frac{1}{2},1)$, there exists a value of $\chi$ for which $b(\chi,\tau)=0$. For given $\tau\in(0,\frac{1}{2}]$, on the other hand, we have $b(\chi,\tau)>0$ for $(\chi,\tau)\in B$.
%whose union is equal to \eqref{eq:b-is-zero}.

%
%by setting t$(\chi,\tau)$ in the open first quadrant gives the following two branches for the curve $ (\chi^2 -16 \tau + 16\tau^2)=0$
%\begin{equation}
%\tau - \frac{1}{2} = \pm \frac{1}{4}\sqrt{4-\chi^2}
%\end{equation}
%\begin{align}
%\tau &= \frac{1}{4}(2-\sqrt{4-\chi^2})\\
%\tau &= \frac{1}{4}(2+\sqrt{4-\chi^2})
%\end{align}


  


%The analysis in this region involves construction of local parametrices inside disks centered at the points $a(\chi,\tau)$ and $b(\chi,\tau)$, and the facts established above about the configuration of the points $a(\chi,\tau)$ and $b(\chi,\tau)$ play a role in placing the branch cut $\Sigma_\omega$ of $\omega(\lambda)$ connecting the branch points $\lambda=\pm \ii$. In order to equip those local parametrices with the desired analyticity properties we need $\Sigma_\omega$ to avoid the above-mentioned neighborhoods of the points $a(\chi,\tau)$ and $b(\chi,\tau)$.
%In principle we may take $\Sigma_\omega$ to be the line segment $\Sigma_\mathrm{c}$ for $0<\tau\leq \frac{1}{2}$ since $a(\chi,\tau)< 0 < b(\chi,\tau)$ in this case, but this is no longer possible for all $\chi$ if $\tau>\frac{1}{2}$. 
%To treat these situations uniformly, we simply take $\Sigma_\omega$ to be a Schwarz-symmetric curve that connects $\lambda=\ii$ to (the midpoint) $\lambda=(a(\chi,\tau)+b(\chi,\tau))/2$ within $\Omega^+$ while avoiding $\Sigma_g$.


%\textcolor{red}{[What's written in this last paragraph will probably move to the beginning of this section.]}

%As in Section~\ref{sec:Schi-Stau}, the last step before constructing parametrices is to remove the $\lambda$-dependence from the jump matrix on $\Sigma_g$ given in \eqref{eq:W-jump-Sigma} with the help of a suitable Szeg\H{o} function.  We define $S(\lambda;\chi,\tau)$ as a function analytic for $\lambda\in\mathbb{C}\setminus\Sigma_g$ almost exactly as in Section~\ref{sec:Schi-Stau}, simply replacing $\omega_+(\lambda)$ in \eqref{eq:Szego-def-Schi-Stau} and \eqref{eq:gamma-def-Schi-Stau} by $\omega(\lambda)$.  In fact, since for $(\chi,\tau)\in \shelves$ the jump contour for $\omega(\lambda)$ is taken to lie to the right of $\Sigma_g$ whereas the latter is contained in the former for $(\chi,\tau)\in S_\chi\cup S_\tau$, what is called $\omega_+(\lambda)=\tilde{\omega}(\lambda)$ on $\Sigma_g$ in Section~\ref{sec:Schi-Stau} coincides exactly with $\omega(\lambda)$ in the present context.  Likewise, in the simplified formula \eqref{eq:gamma-explicit-Schi-Stau} for $\gamma(\chi,\tau)$ the integration contour can be written  as $\Sigma_\omega$ in the present context of $(\chi,\tau)\in \shelves$.  In particular, $\gamma(\chi,\tau)$ is a real analytic function of $(\chi,\tau)\in \shelves$, i.e., it is analytic on the whole exterior domain.  

%A preliminary step we take before construction of parametrices is to convert the jump condition \eqref{eq:W-jump-Sigma} on $\Sigma_g$ to one that does not depend on $\lambda$. To accomplish this, we introduce a scalar-valued function $S(\lambda)$ which is analytic in $\mathbb{C}\setminus\Sigma_g$ and bounded at the endpoints $\lambda=A(\chi,\tau)\pm \ii B(\chi,\tau)$ of $\Sigma_g$, satisfying $S(\lambda)\to 0$ as $\lambda\to\infty$, and admitting continuous boundary values on $\Sigma_g$ related by 
%\begin{equation}
%S_+(\lambda) + S_-(\lambda) - 2s \log(\omega(\lambda)) = 2 \ii {s}\gamma,\quad \lambda\in \Sigma_g, 
%\label{eq:Szego-jump}
%\end{equation}
%for some constant $\gamma\in\mathbb{C}$ which may depend on the parameters $\chi$ and $\tau$. 
%Here $\log(\cdot)$ is taken to be the principal branch, and we have $\omega(\lambda)^{2s} = \ee^{2s \log(\omega(\lambda))}$ with the right-hand side being well-defined for all $\lambda\in\Sigma_g$ including the endpoints since $\omega(\lambda)$ is nonzero, continuous and bounded (in fact, analytic) in a suitably small\footnote{Suitably small in the sense that the neighborhood leaves $\Sigma_\omega$ in its exterior.} neighborhood containing $\Sigma_g$. As the endpoints $\lambda_0(\chi,\tau)$ and $\lambda_0(\chi,\tau)^*$ of the jump contour $\Sigma_g$ are already determined in the construction of $g(\lambda;\chi,\tau)$, the only degree of freedom left to ensure the existence of such a function $S(\lambda)$ is the constant $\gamma$, which we will choose appropriately to satisfy the normalization.
%It is easy to verify that
%\begin{equation}
%S(\lambda)=S(\lambda;\chi,\tau)\defeq \frac{R(\lambda;\chi,\tau)}{2\pi \ii} \int_{\Sigma_g} \frac{2s \log(\omega(\eta))+  2\ii{s} \gamma}{R_+(\eta; \chi,\tau)(\eta-\lambda)}\dd \eta
%\label{eq:Szego-def}
%\end{equation}
%is analytic in $\mathbb{C}\setminus\Sigma_g$, satisfies the jump condition given in \eqref{eq:Szego-jump}, and it is bounded as $\lambda \to A(\chi,\tau) \pm \ii B(\chi,\tau)$. Moreover, $S(\lambda)\to 0$ as $\lambda\to\infty$ provided that the constant $\gamma=\gamma(\chi,\tau)$ is chosen to ensure
%\begin{equation}
%\int_{\Sigma_g} \frac{\log(\omega(\eta))+ \ii \gamma(\chi,\tau)}{R_+(\eta;\chi,\tau)}\, \dd \eta = 0.
%\label{eq:gamma-condition}
%\end{equation}
%Recalling from \eqref{eq:integral-R-plus} that
%\begin{equation}
%\int_{\Sigma_g} \frac{\dd \eta}{R_+(\eta;\chi,\tau)} = -\ii \pi
%\end{equation}
%is nonzero, we see that the condition \eqref{eq:gamma-condition} indeed determines $\gamma=\gamma(\chi,\tau)$ via the ratio:
%\begin{equation}
%\gamma(\chi,\tau) \defeq \ii  \frac{\displaystyle \int_{\Sigma_g} \frac{ \log(\omega(\eta)) }{R_+(\eta;\chi,\tau)}\dd \eta}{\displaystyle\int_{\Sigma_g}  \frac{\dd \eta}{R_+(\eta;\chi,\tau)}} = -\frac{1}{\pi} \displaystyle \int_{\Sigma_g} \frac{ \log(\omega(\eta)) }{R_+(\eta;\chi,\tau)}\dd \eta.
%\label{eq:gamma-def-bun}
%\end{equation}
%To compute $\gamma(\chi,\tau)$, we let $L$ to be a clockwise-oriented loop surrounding the branch cut $\Sigma_g$ of $R(\lambda;\chi,\tau)$ excluding the branch cut $\Sigma_\omega$ of $\omega(\lambda)$ which connects the points $\lambda=\pm \ii$. Taking $L'$ to be a counter-clockwise-oriented contour that encircles $\Sigma_\omega$ but excludes $\Sigma_g$ and using the fact that the integrand in the last integral in \eqref{eq:gamma-def-bun} is integrable at $\lambda=\infty$, we obtain:
%\begin{equation}
%\gamma(\chi,\tau) = -\frac{1}{2\pi} \oint_L \frac{\log(\omega(\eta))}{R(\eta;\chi,\tau)}\dd \eta
%= -\frac{1}{2\pi} \oint_{L'} \frac{\log(\omega(\eta))}{R(\eta;\chi,\tau)}\dd \eta
%= -\frac{1}{8\pi} \oint_{L'} \log\left(\frac{\eta-\ii}{\eta+\ii}\right) \frac{1}{R(\eta;\chi,\tau)}\dd \eta,
%\end{equation}
%where we used the fact that $\omega(\lambda)^4 = ( (\lambda-\ii)/(\lambda+\ii))$ with the logarithm having $\Sigma_\omega$ to be its branch cut. We may now collapse $L'$ to both sides of the branch cut $\Sigma_{\omega}$, where $R(\lambda;\chi,\tau)$ is analytic but the boundary values of the logarithm differ by $2\pi \ii$, and see that
%\begin{equation}
%\gamma(\chi,\tau) = -\frac{1}{4\ii} \int_{\Sigma_\omega} \frac{1}{R(\eta;\chi,\tau)}\dd \eta =  \frac{\ii}{4} \int_{-\ii}^{\ii} \frac{1}{R(\eta;\chi,\tau)}\dd \eta,
%\label{eq:gamma-explicit-bun}
%\end{equation}
%where the latter equality follows because the line segment from $-\ii$ to $\ii$ remains to the right of $\Sigma_\omega$, which  remains to the right of $\Sigma_g$ by definition. We see from \eqref{eq:gamma-explicit-bun} that $\gamma(\chi,\tau)$ has a purely imaginary value.

%With the Szeg\H{o} function defined as indicated, we now make a global substitution by setting
%\begin{equation}
%\mathbf{X}^{(k)}(\lambda;\chi,\tau) \defeq  \mathbf{W}^{(k)}(\lambda;\chi,\tau)\ee^{S(\lambda;\chi,\tau)\sigma_3}
%\end{equation}
%in the whole domain of analyticity $\mathbf{W}^{(k)}(\lambda;\chi,\tau)$. The boundary values of $\mathbf{X}^{(k)}(\lambda)=\mathbf{X}^{(k)}(\lambda;\chi,\tau)$ along $\Sigma_g$ are related by a constant jump matrix:
%\begin{equation}
%\mathbf{X}^{(k)}_+(\lambda) = \mathbf{X}^{(k)}_-(\lambda) 
%\begin{bmatrix} 0 &  s \ee^{-2\ii (n \kappa(\chi,\tau) + {s}\gamma(\chi,\tau))} \\ - s \ee^{2\ii (n \kappa(\chi,\tau) + {s}\gamma(\chi,\tau))} &  0 \end{bmatrix},\quad \lambda\in \Sigma_g= \Sigma_g^+ \cup \Sigma_g^-,
%\label{eq:X-twist-jump}
%\end{equation}
%The remaining jump conditions \eqref{eq:W-jump-Gamma-plus-L}--\eqref{eq:W-jump-Sigma-minus-R} are modified via conjugations by the diagonal matrix $\ee^{S(z;w,t)\sigma_3}$ and take the form:
%\begin{equation}
%\mathbf{X}^{(k)}_{+}(\lambda) = \mathbf{X}^{(k)}_{-}(\lambda)
%\begin{bmatrix} 1 & 0 \\ -s \omega(\lambda)^{-2s}\ee^{2S(\lambda;\chi,\tau)}\ee^{2n h(\lambda;\chi,\tau)}& 1 \end{bmatrix},\quad \lambda\in C^+_{\Gamma,L},
%\label{eq:X-jump-Gamma-plus-L}
%\end{equation}
%\begin{equation}
%\mathbf{X}^{(k)}_{+}(\lambda) = \mathbf{X}^{(k)}_{-}(\lambda)
%\begin{bmatrix} 1 & \frac{s}{2} \omega(\lambda)^{2s}\ee^{-2S(\lambda;\chi,\tau)}\ee^{-2n h(\lambda;\chi,\tau)} \\ 0 & 1 \end{bmatrix},\quad \lambda\in C^+_{\Gamma,R},
%\end{equation}
%\begin{equation}
%\mathbf{X}^{(k)}_{+}(\lambda) = \mathbf{X}^{(k)}_{-}(\lambda)
%2^{\sigma_3},\quad \lambda\in I ,
%\label{eq:X-I-jump}
%\end{equation}
%\begin{equation}
%\mathbf{X}^{(k)}_{+}(\lambda) = \mathbf{X}^{(k)}_{-}(\lambda)
%\begin{bmatrix} 1 & 0 \\ -\frac{s}{2} \omega(\lambda)^{-2s} \ee^{2S(\lambda;\chi,\tau)} \ee^{2n h(\lambda;\chi,\tau)} & 1 \end{bmatrix} ,\quad \lambda\in C^-_{\Gamma,R},
%\end{equation}
%\begin{equation}
%\mathbf{X}^{(k)}_{+}(\lambda) = \mathbf{X}^{(k)}_{-}(\lambda)
%\begin{bmatrix} 1 &  s  \omega(\lambda)^{2s}\ee^{-2S(\lambda;\chi,\tau)} \ee^{-2n h(\lambda;\chi,\tau)} \\ 0 & 1 \end{bmatrix} ,\quad \lambda\in C^-_{\Gamma,L},
%\end{equation}
%\begin{equation}
%\mathbf{X}^{(k)}_{+}(\lambda) = \mathbf{X}^{(k)}_{-}(\lambda)
%\begin{bmatrix} 1 & -s \omega(\lambda)^{2s} \ee^{-2S(\lambda;\chi,\tau)} \ee^{-2n h(\lambda;\chi,\tau)} \\ 0 & 1 \end{bmatrix},\quad \lambda\in C^+_{\Sigma,L},
%\end{equation}
%\begin{equation}
%\mathbf{X}^{(k)}_{+}(\lambda) = \mathbf{X}^{(k)}_{-}(\lambda)
%\begin{bmatrix} 1 & -\frac{s}{2}\omega(\lambda)^{2 s}\ee^{-2S(\lambda;\chi,\tau)}\ee^{-2n h(\lambda;\chi,\tau)} \\ 0 & 1 \end{bmatrix},\quad \lambda\in C^+_{\Sigma,R},
%\end{equation}
%\begin{equation}
%\mathbf{X}^{(k)}_{+}(\lambda) = \mathbf{X}^{(k)}_{-}(\lambda)
%\begin{bmatrix} 1 & 0 \\ s \omega(\lambda)^{-2s}\ee^{2S(\lambda;\chi,\tau)} \ee^{2n h(\lambda;\chi,\tau)} & 1 \end{bmatrix},\quad \lambda\in C^-_{\Sigma,L},
%\end{equation}
%and, finally,
%\begin{equation}
%\mathbf{X}^{(k)}_{+}(\lambda) = \mathbf{X}^{(k)}_{-}(\lambda)
%\begin{bmatrix} 1 & 0 \\ \frac{s}{2} \omega(\lambda)^{-2s}\ee^{2S(\lambda;\chi,\tau)} \ee^{2n h(\lambda;\chi,\tau)} & 1 \end{bmatrix},\quad \lambda\in C^-_{\Sigma,R}.
%\label{eq:X-jump-Sigma-minus-R}
%\end{equation}
%Note that all of the jump matrices above except for those supported on $I\cup\Sigma_g$ tend to the identity matrix as $n\to +\infty$ for $\lambda$ on the relevant supporting arcs away from the points $\lambda_{0}(\chi,\tau)$, $\lambda_{0}(\chi,\tau)^{*}$, $a(\chi,\tau)$ and $b(\chi,\tau)$.

\subsection{Parametrix construction}
\subsubsection{Outer parametrix construction} 
We start with construction of an outer parametrix denoted $\dot{\mathbf{W}}^\mathrm{out}(\lambda)\defeq \dot{\mathbf{W}}^\mathrm{out}(\lambda;\chi,\tau,\mathbf{Q}^{-s},M)$ satisfying exactly the jump conditions on $I$ and $\Sigma_g$ (cf., \eqref{eq:W-jump-I} and \eqref{eq:W-jump-Sigma-g}) that do not become asymptotically trivial as $M\to +\infty$. The procedure follows closely the construction in \cite[Section 4.2.2]{BilmanLM20} and it can be viewed as a combination of the outer parametrices constructed in Section~\ref{sec:channels} and in Section~\ref{sec:Schi-Stau} together with a new diagonal factor which is intrinsic to \shelves. Indeed, the jump condition \eqref{eq:W-jump-I} on $I$ is identical for $(\chi,\tau)\in\channels$ and $(\chi,\tau)\in\shelves$, hence we employ the outer parametrix \eqref{eq:Channels-Tout} to write $\dot{\mathbf{W}}^\mathrm{out}(\lambda)$ as
 %Recalling from Section~\ref{s:channels} the corresponding outer parametrix that satisfies the jump condition on $I$ 
%We write $\dot{\mathbf{W}}^\mathrm{out}(\lambda)$ in the for
\begin{equation}
%\dot{\mathbf{W}}^\mathrm{out}(\lambda;\chi,\tau)=\mathbf{G}(\lambda;\chi,\tau)\left(\frac{\lambda-a(\chi,\tau)}{\lambda-b(\chi,\tau)}\right)^{\ii p\sigma_3},\quad p\defeq \frac{\ln(2)}{2\pi},
\dot{\mathbf{W}}^\mathrm{out}(\lambda)=\mathbf{J}(\lambda)\left(\frac{\lambda-a(\chi,\tau)}{\lambda-b(\chi,\tau)}\right)^{\ii p\sigma_3},
\label{eq:W-out-G}
\end{equation}
where the power function is defined as the principal branch and $p>0$ was defined in \eqref{eq:Channels-Tout}.
Then, $\mathbf{J}(\lambda)=\mathbf{J}(\lambda;\chi,\tau,\mathbf{Q}^{-s},M)$ extends analytically to $I$, and we will assume that it is bounded near $\lambda=a(\chi,\tau),b(\chi,\tau)$ in particular making it analytic at $\lambda=b(\chi,\tau)$.  Therefore, $\mathbf{J}(\lambda)$ is analytic for $\lambda\in\mathbb{C}\setminus\Sigma_g$ and tends to the identity as $\lambda\to\infty$.
Across $\Sigma_g$, the constant jump condition \eqref{eq:W-jump-Sigma-g} required of $\dot{\mathbf{W}}^\mathrm{out}(\lambda)$ becomes modified for $\mathbf{J}(\lambda)$:
\begin{equation}
\mathbf{J}_+(\lambda)=\mathbf{J}_-(\lambda)\left(\frac{\lambda-a(\chi,\tau)}{\lambda-b(\chi,\tau)}\right)^{\ii p\sigma_3}
\begin{bmatrix}0 & s \ee^{-2\ii M\kappa(\chi,\tau)}\\
- s \ee^{ 2\ii M\kappa(\chi,\tau)} & 0\end{bmatrix}
\left(\frac{\lambda-a(\chi,\tau)}{\lambda-b(\chi,\tau)}\right)^{-\ii p\sigma_3},\quad \lambda\in \Sigma_g,
\end{equation}
and we will convert this back into a constant jump condition on $\Sigma_g$ alone by introducing a \emph{Szeg{\H o} function} $K(\lambda;\chi,\tau)$, which we define by
\begin{equation}
%K(\lambda;\chi,\tau)\defeq \ii p\log\left(\frac{\lambda-a(\chi,\tau)}{\lambda-b(\chi,\tau)}\right)+\ii p R(\lambda;\chi,\tau) \int_{a(\chi,\tau)}^{b(\chi,\tau)}\frac{\dd \eta}{R(\eta;\chi,\tau)(\eta-\lambda)}+\ii\mu(\chi,\tau),
K(\lambda;\chi,\tau)\defeq  p\log\left(\frac{\lambda-a(\chi,\tau)}{\lambda-b(\chi,\tau)}\right)+ p R(\lambda;\chi,\tau) \int_{a(\chi,\tau)}^{b(\chi,\tau)}\frac{\dd \eta}{R(\eta;\chi,\tau)(\eta-\lambda)}, %+ \mu(\chi,\tau),
\label{eq:K-def}
\end{equation}
in which the logarithm is taken to be the principal branch, $-\pi<\mathrm{Im}(\log(\cdot))<\pi$. 
It is straightforward to confirm that $K(\lambda;\chi,\tau)$ has the following properties.
Recalling the definition \eqref{eq:mu-formula-intro} of the constant $\mu(\chi,\tau)$, $K(\lambda;\chi,\tau) = -\mu(\chi,\tau) + O(\lambda^{-1})$ as $\lambda\to\infty$. Despite appearances, $K(\lambda;\chi,\tau)$ does not have a jump across $I$ as is easily confirmed by comparing the boundary values of the logarithm and using the Plemelj formula.  
The apparent singularities at $\lambda=a(\chi,\tau),b(\chi,\tau)$ are removable, so the domain of analyticity for $K(\lambda;\chi,\tau)$ is $\lambda\in\mathbb{C}\setminus\Sigma_g$, and $K(\lambda;\chi,\tau)$ takes continuous boundary values on $\Sigma_g$, including at the endpoints.  These boundary values are related by the jump condition
\begin{equation}
%K_+(\lambda;\chi,\tau)+K_-(\lambda;\chi,\tau)=2\ii p\log\left(\frac{\lambda-a(\chi,\tau)}{\lambda-b(\chi,\tau)}\right)+2\ii\mu(\chi,\tau),\quad \lambda\in\Sigma_g.
K_+(\lambda;\chi,\tau)+K_-(\lambda;\chi,\tau)=2 p\log\left(\frac{\lambda-a(\chi,\tau)}{\lambda-b(\chi,\tau)}\right),\quad \lambda\in\Sigma_g.
\label{eq:K-jump}
\end{equation}
One can indeed check that $K(\lambda;\chi,\tau)$ has the alternate representation obtained by the Plemelj formula:
\begin{equation}
K(\lambda;\chi,\tau) =  \frac{R(\lambda;\chi,\tau)}{2\pi \ii} \int_{\Sigma_g} \log\left( \frac{\eta - a(\chi,\tau)}{\eta - b(\chi,\tau)}\right) \frac{2 p  \dd \eta}{R_+(\eta; \chi,\tau)(\eta - \lambda)},
\label{eq:K-Plemelj}
\end{equation}
which confirms the properties stated above. The values of $K(\lambda;\chi,\tau)$ at $\lambda=a(\chi,\tau),b(\chi,\tau)$ will be useful in obtaining the asymptotic formula for $q( M \chi, M\tau;\mathbf{Q}^{-s},M)$, and they can easily be computed from the representation \eqref{eq:K-Plemelj}. 
%Considering an upward oriented arc $C^\sharp$ that lies to the left of $\Sigma_g$ connecting the branch points $\eta=A(\chi,\tau) \pm \ii B(\chi,\tau)$, and 
Using
\begin{equation}
R(b(\chi,\tau);\chi,\tau) = |b(\chi,\tau)-\lambda_0(\chi,\tau)|\quad\text{and}\quad R_-(a(\chi,\tau);\chi,\tau) = |a(\chi,\tau)-\lambda_0(\chi,\tau)|,
\label{eq:R-a-b}
\end{equation}
we arrive at the formul\ae\ \eqref{eq:intro-Ka}--\eqref{eq:intro-Kb} for $K_a(\chi,\tau)\defeq K_-(a(\chi,\tau);\chi,\tau)$ and $K_b(\chi,\tau)\defeq K(b(\chi,\tau);\chi,\tau)$ respectively.
%while passing from a point at a finite-distance away from (and to the left of) $\eta = a(\chi,\tau)$, 
%we obtain
%\begin{align}
%K_b(\chi,\tau) &\defeq  K(b(\chi,\tau); \chi,\tau) = 
%\frac{|b(\chi,\tau) - \lambda_0(\chi,\tau)|}{2\pi \ii} \int_{C^{\sharp}}  \log\left( \frac{\eta - a(\chi,\tau)}{\eta - b(\chi,\tau)}\right)  \frac{2 p \dd\eta}{R(\eta; \chi,\tau)(\eta - b(\chi,\tau))},\label{eq:K-at-b}\\
%K_a(\chi,\tau) &\defeq   K_{-}(a(\chi,\tau); \chi,\tau) = 
%\frac{|a(\chi,\tau) - \lambda_0(\chi,\tau)|}{2\pi \ii} \int_{C^{\sharp}} \log\left( \frac{\eta - a(\chi,\tau)}{\eta - b(\chi,\tau)}\right) \frac{2 p  \dd\eta }{R(\eta; \chi,\tau)(\eta - a(\chi,\tau))}, \label{eq:K-at-a}
%\end{align}

%Using $R(a(\chi,\tau);\chi,\tau)=...$ ... we arrive at the formul\ae\ \eqref{eq:intro-Ka}--\eqref{eq:intro-Kb} for $K_a(\chi,\tau)\defeq K_-(a(\chi,\tau);\chi,\tau)$ and $K_b(\chi,\tau)\defeq K(b(\chi,\tau);\chi,\tau)$ respectively.
%
%where we have substituted 
Preserving the normalization at infinity, we introduce $K(\lambda;\chi,\tau)$ in the construction of $\dot{\mathbf{W}}^\text{out}(\lambda)$ by
\begin{equation}
%\mathbf{G}(\lambda;\chi,\tau)=\mathbf{H}(\lambda;\chi,\tau)\ee^{-K(\lambda;\chi,\tau)\sigma_3},
\mathbf{J}(\lambda)=\mathbf{L}(\lambda)\ee^{-\ii (K(\lambda;\chi,\tau)+ \mu(\chi,\tau) )\sigma_3}.
\label{eq:G-H}
\end{equation}
It then follows that $\mathbf{L}(\lambda)=\mathbf{L}(\lambda;\chi,\tau,\mathbf{Q}^{-s},M)$ is a matrix function analytic for $\lambda\in\mathbb{C}\setminus\Sigma_g$ that tends to $\mathbb{I}$ as $\lambda\to\infty$, and that satisfies the jump condition
%\begin{equation}
%\mathbf{H}_+(\lambda)=\mathbf{H}_-(\lambda)\begin{bmatrix}0 & s \ee^{-2\ii (n\kappa(\chi,\tau)+ {s}\gamma(\chi,\tau)+\mu(\chi,\tau))}\\
%-s \ee^{2\ii (n\kappa(\chi,\tau)+ {s}\gamma(\chi,\tau)+\mu(\chi,\tau))} & 0\end{bmatrix},\quad \lambda\in\Sigma_g.
%\end{equation}
%\begin{equation}
%\mathbf{L}_+(\lambda)=\mathbf{L}_-(\lambda)
%\ee^{-\ii \mu(\chi,\tau)\sigma_3}
%\begin{bmatrix}0 & s \ee^{-2\ii M\kappa(\chi,\tau)}\\
%-s \ee^{2\ii M\kappa(\chi,\tau)} & 0\end{bmatrix}
%\ee^{\ii \mu(\chi,\tau)\sigma_3}
%,\quad \lambda\in\Sigma_g.
%\label{eq:H-jump-Sigma-g}
%\end{equation}
\begin{equation}
\mathbf{L}_+(\lambda)=\mathbf{L}_-(\lambda)
%\ee^{-\ii \mu(\chi,\tau)\sigma_3}
\begin{bmatrix}0 & s \ee^{-2\ii (M\kappa(\chi,\tau) + \mu(\chi,\tau) )}\\
-s \ee^{2\ii (M\kappa(\chi,\tau) + \mu(\chi,\tau) )} & 0\end{bmatrix}
%\ee^{\ii \mu(\chi,\tau)\sigma_3}
,\quad \lambda\in\Sigma_g.
\label{eq:H-jump-Sigma-g}
\end{equation}
%Notice how similar the central factor in this jump matrix is to the one in \eqref{eq:T-jump-N-Schi-Stau-ALT}. Indeed, factoring out the exponential factors in \eqref{eq:H-jump-Sigma-g} and relating the remaining off-diagonal matrix to the one in \eqref{eq:H-jump-Sigma-g} shows that $\mathbf{L}(\lambda)$ is given by
$\mathbf{L}(\lambda)$ is given by
\begin{equation}
\mathbf{L}(\lambda)\defeq 
\ee^{-\ii \frac{1}{4}s \pi \sigma_3}
\ee^{-\ii (M\kappa(\chi,\tau) + \mu(\chi,\tau) )\sigma_3}
\mathbf{Q}y(\lambda;\chi,\tau)^{\sigma_3}\mathbf{Q}^{-1}
\ee^{\ii (M\kappa(\chi,\tau)+  \mu(\chi,\tau) )\sigma_3}
\ee^{\ii \frac{1}{4}s \pi \sigma_3},
\label{eq:H-def}
\end{equation}
where $y(\lambda;\chi,\tau)$ is defined in \eqref{eq:y-def}.
This solution clearly relates to (232) in exactly the same way that \eqref{eq:outer-parametrix-Schi-Stau-ALT} relates to \eqref{eq:T-jump-N-Schi-Stau-ALT} with $\mathbf{T}$ replaced by $\mathbf{W}$, and obviously $\mathbf{L}(\lambda)\to\mathbb{I}$ as $\lambda\to\infty$.
%Observe that $\mathbf{L}(\lambda)$ relates to \eqref{eq:outer-parametrix-Schi-Stau-ALT} through conjugation by a diagonal constant matrix other than the difference in the phases $M\gamma(\chi,\tau)$ versus $M\kappa(\chi,\tau) + \mu(\chi,\tau)$, and the leftmost factor in \eqref{eq:H-def} is to ensure $\mathbf{L}(\lambda)\to \mathbb{I}$ as $\lambda\to\infty$.
% in the form
%\begin{equation}
%%\mathbf{G}(\lambda;\chi,\tau)=\mathbf{H}(\lambda;\chi,\tau)\ee^{-K(\lambda;\chi,\tau)\sigma_3},
%\mathbf{J}(\lambda)=\mathbf{H}(\lambda)\ee^{-\ii K(\lambda;\chi,\tau)\sigma_3},
%\label{eq:G-H}
%\end{equation}
%where $K(\lambda;\chi,\tau)$ is given, as in \cite[Section 4.2.2]{BilmanLM20} modulo a factor of $\ii$, by 
%\begin{equation}
%%K(\lambda;\chi,\tau)\defeq \ii p\log\left(\frac{\lambda-a(\chi,\tau)}{\lambda-b(\chi,\tau)}\right)+\ii p R(\lambda;\chi,\tau) \int_{a(\chi,\tau)}^{b(\chi,\tau)}\frac{\dd \eta}{R(\eta;\chi,\tau)(\eta-\lambda)}+\ii\mu(\chi,\tau),
%K(\lambda;\chi,\tau)\defeq  p\log\left(\frac{\lambda-a(\chi,\tau)}{\lambda-b(\chi,\tau)}\right)+ p R(\lambda;\chi,\tau) \int_{a(\chi,\tau)}^{b(\chi,\tau)}\frac{\dd \eta}{R(\eta;\chi,\tau)(\eta-\lambda)}+ \mu(\chi,\tau),
%\label{eq:K-def}
%\end{equation}
%in which the logarithm is taken to be the principal branch, $-\pi<\mathrm{Im}(\log(\cdot))<\pi$, and where the constant $\mu(\chi,\tau)$ is given by
%\begin{equation}
%\mu(\chi,\tau)\defeq p\int_{a(\chi,\tau)}^{b(\chi,\tau)}\frac{\dd \eta}{R(\eta;\chi,\tau)}>0.
%\label{eq:mu-def-bun}
%\end{equation}
%It is straightforward to confirm that $K(\lambda;\chi,\tau)$ has the following properties.  By definition of $\mu(\chi,\tau)$, it satisfies $K(\lambda;\chi,\tau)=O(\lambda^{-1})$ as $\lambda\to\infty$.  
%Despite appearances, $K$ does not have a jump across $(a(\chi,\tau),b(\chi,\tau))$ as is easily confirmed by comparing the boundary values of the logarithm and using the Plemelj formula.  
%The apparent singularities at $\lambda=a(\chi,\tau),b(\chi,\tau)$ are removable, so the domain of analyticity for $K(\lambda;\chi,\tau)$ is $\lambda\in\mathbb{C}\setminus\Sigma_g$, and $K(\lambda;\chi,\tau)$ takes continuous boundary values on $\Sigma_g$, including at the endpoints.  These boundary values are related by the jump condition
%\begin{equation}
%%K_+(\lambda;\chi,\tau)+K_-(\lambda;\chi,\tau)=2\ii p\log\left(\frac{\lambda-a(\chi,\tau)}{\lambda-b(\chi,\tau)}\right)+2\ii\mu(\chi,\tau),\quad \lambda\in\Sigma_g.
%K_+(\lambda;\chi,\tau)+K_-(\lambda;\chi,\tau)=2 p\log\left(\frac{\lambda-a(\chi,\tau)}{\lambda-b(\chi,\tau)}\right)+2\mu(\chi,\tau),\quad \lambda\in\Sigma_g.
%\label{eq:K-jump}
%\end{equation}
%It then follows that $\mathbf{H}(\lambda)=\mathbf{H}(\lambda;\chi,\tau,\mathbf{Q}^{-s},M)$ is a matrix function analytic for $\lambda\in\mathbb{C}\setminus\Sigma_g$ that tends to $\mathbb{I}$ as $\lambda\to\infty$, and that satisfies the jump condition
%%\begin{equation}
%%\mathbf{H}_+(\lambda)=\mathbf{H}_-(\lambda)\begin{bmatrix}0 & s \ee^{-2\ii (n\kappa(\chi,\tau)+ {s}\gamma(\chi,\tau)+\mu(\chi,\tau))}\\
%%-s \ee^{2\ii (n\kappa(\chi,\tau)+ {s}\gamma(\chi,\tau)+\mu(\chi,\tau))} & 0\end{bmatrix},\quad \lambda\in\Sigma_g.
%%\end{equation}
%\begin{equation}
%\mathbf{H}_+(\lambda)=\mathbf{H}_-(\lambda)
%\begin{bmatrix}0 & s \ee^{-2\ii (M\kappa(\chi,\tau)+ \mu(\chi,\tau))}\\
%-s \ee^{2\ii (M\kappa(\chi,\tau)+\mu(\chi,\tau))} & 0\end{bmatrix},\quad \lambda\in\Sigma_g.
%\end{equation}
%Comparing with the jump condition \eqref{eq:X-jump-N-Schi-Stau}, 
%It is now straightforward to solve for $\mathbf{H}(\lambda)$ by 
%%adaptation of the formula \eqref{eq:outer-parametrix-Schi-Stau} to augment the phase $n\kappa(\chi,\tau)+s\gamma(\chi,\tau)$ with $\mu(\chi,\tau)$. \textcolor{red}{(The previous sentence needs to be modified according to how the outer parametrix is constructed in $S_\chi\cup S_\tau$.)} The resulting formula reads
%diagonalizing the constant jump matrix, which has eigenvalues $\pm \ii $. All solutions of the jump condition for $\mathbf{H}(\lambda)$ have singularities at the endpoints of $\Sigma_g$, and we select the unique solution with the mildest rate of growth as $\lambda\to \lambda_0(\chi,\tau),\lambda_0(\chi,\tau)^{*}$:
%%\begin{multline}
%%\mathbf{H}(\lambda;\chi,\tau)\defeq\ee^{-\ii (M\kappa(\chi,\tau)+ \mu(\chi,\tau) )\sigma_3}
%%\ii^{\frac{1}{2}(1-s)\sigma_3}
%%\mathbf{O}
%%\left(\frac{\lambda-\lambda_0(\chi,\tau)}{\lambda-\lambda_0(\chi,\tau)^*}\right)^{ \frac{1}{4} \sigma_3}
%%\mathbf{O}^{-1}
%%\ii^{-\frac{1}{2}(1-s)\sigma_3}
%%\\
%%\cdot
%%\ee^{\ii (n\kappa(\chi,\tau)+  {s} \gamma(\chi,\tau) + \mu(\chi,\tau) )\sigma_3},\\
%%\label{eq:H-def}
%%\end{multline}
%%\begin{multline}
%%\mathbf{H}(\lambda;\chi,\tau)\defeq\ee^{-\ii (M\kappa(\chi,\tau) + \mu(\chi,\tau) )\sigma_3}
%%\ii^{\frac{1}{2}(1-s)\sigma_3}
%%\mathbf{O}
%%\left(\frac{\lambda-\lambda_0(\chi,\tau)}{\lambda-\lambda_0(\chi,\tau)^*}\right)^{ \frac{1}{4} \sigma_3}
%%\\
%%\cdot
%%\mathbf{O}^{-1}
%%\ii^{-\frac{1}{2}(1-s)\sigma_3}
%%\ee^{\ii (M\kappa(\chi,\tau)+  \mu(\chi,\tau) )\sigma_3},
%%\label{eq:H-def}
%%\end{multline}
%\begin{equation}
%\mathbf{H}(\lambda)\defeq\ee^{-\ii (M\kappa(\chi,\tau) + \mu(\chi,\tau) )\sigma_3}
%\ii^{\frac{1}{2}(1-s)\sigma_3}
%\mathbf{O}
%q(\lambda;\chi,\tau)^{\sigma_3}
%\mathbf{O}^{-1}
%\ii^{-\frac{1}{2}(1-s)\sigma_3}
%\ee^{\ii (M\kappa(\chi,\tau)+  \mu(\chi,\tau) )\sigma_3},
%\label{eq:H-def}
%\end{equation}
%where $\mathbf{O}$ is defined in \eqref{eq:O-def-Schi-Stau}, and where $y(\lambda;\chi,\tau)$ is the function analytic for $\lambda\in\mathbb{C}\setminus\Sigma_g$ determined by the conditions that $y(\lambda;\chi,\tau)\to 1$ as $\lambda\to\infty$ and 
%\begin{equation}
%y(\lambda;\chi,\tau)^4=\frac{\lambda-\lambda_0(\chi,\tau)}{\lambda-\lambda_0(\chi,\tau)^*}.
%\end{equation}
%%\begin{equation}
%%\mathbf{O}\defeq \frac{1}{\sqrt{2}}\begin{bmatrix}1 & \ii \\ \ii &  1\end{bmatrix},\quad {\det(\mathbf{U})=1}.
%%\end{equation}
%%\textcolor{PineGreen}{
%%[TO-BE-REMOVED] There are various other ways to write $\mathbf{H}(\lambda)$ exploiting the fact that $s^2=1$. For instance, avoid the conjugation by $\ii^{\frac{1}{2}(1-s)\sigma_3}$ in the central factor since $s=1/s$ and write
%%\begin{equation}
%%\mathbf{H}(\lambda;\chi,\tau)=\ee^{-\ii (n\kappa(\chi,\tau)+  {s} \gamma(\chi,\tau) + \mu(\chi,\tau) )\sigma_3}
%%\mathbf{U}\left(\frac{\lambda-\lambda_0(\chi,\tau)}{\lambda-\lambda_0(\chi,\tau)^*}\right)^{ s \sigma_3/4}\mathbf{U}^{-1}
%%\ee^{\ii (n\kappa(\chi,\tau)+  {s} \gamma(\chi,\tau) + \mu(\chi,\tau) )\sigma_3},\\
%%%\label{eq:H-def}
%%\end{equation}
%%where
%%\begin{equation}
%%\mathbf{U}\defeq \frac{1}{\sqrt{2}}\begin{bmatrix}1 & \ii \\ \ii &  1\end{bmatrix},\quad {\det(\mathbf{U})=1}.
%%\end{equation}
%%or can write:
%%\begin{equation}
%%\mathbf{H}(\lambda;\chi,\tau)=\ee^{-\ii (n\kappa(\chi,\tau)+  {s} \gamma(\chi,\tau) + \mu(\chi,\tau) )\sigma_3}
%%\mathbf{U}\left(\frac{\lambda-\lambda_0(\chi,\tau)}{\lambda-\lambda_0(\chi,\tau)^*}\right)^{  \sigma_3/4}\mathbf{U}^{-1}
%%\ee^{\ii (n\kappa(\chi,\tau)+  {s} \gamma(\chi,\tau) + \mu(\chi,\tau) )\sigma_3},\\
%%\end{equation}
%%where
%%\begin{equation}
%%\mathbf{U}\defeq \frac{1}{\sqrt{2}}\begin{bmatrix}1 & \ii {s} \\ \ii  {s} &  1\end{bmatrix},\quad {\det(\mathbf{U})=1},
%%\end{equation}
%%depending on where we would like $s$ to appear.
%%}
%%Here, the power function in the central factor is defined to be analytic for $\lambda\in\mathbb{C}\setminus\Sigma_g$ and to tend to $\mathbb{I}$ as $\lambda\to\infty$.  
Combining \eqref{eq:W-out-G}, \eqref{eq:G-H}, and \eqref{eq:H-def} completes the construction of the outer parametrix $\dot{\mathbf{W}}^\mathrm{out}(\lambda)$:
%\begin{equation}
%\dot{\mathbf{X}}^\mathrm{out}(\lambda;\chi,\tau) = \mathbf{H}(\lambda;\chi,\tau) \ee^{-K(\lambda;\chi,\tau)\sigma_3}\left(\frac{\lambda-a(\chi,\tau)}{\lambda-b(\chi,\tau)}\right)^{\ii p\sigma_3}
%\label{eq:X-out-full}
%\end{equation}
%\textcolor{red}{or in the alternate approach:
\begin{equation}
\dot{\mathbf{W}}^\mathrm{out}(\lambda) = \mathbf{L}(\lambda) \ee^{-\ii ( K(\lambda;\chi,\tau) + \mu(\chi,\tau) )\sigma_3}\left(\frac{\lambda-a(\chi,\tau)}{\lambda-b(\chi,\tau)}\right)^{\ii p\sigma_3}.
\label{eq:W-out-full}
\end{equation}
%}
Note that the only dependence on $M$ enters via the oscillatory factors $\ee^{\pm \ii M \kappa(\chi,\tau)\sigma_3}$ in $\mathbf{L}(\lambda)$. Thus, $\dot{\mathbf{W}}^{\text{out}}(\lambda)=\dot{\mathbf{W}}^{\text{out}}(\lambda;\chi,\tau,\mathbf{Q}^{-s},M)$ is bounded as $M\to+\infty$, provided that $\lambda$ is bounded away from $\lambda_0(\chi,\tau)$ and $\lambda_0(\chi,\tau)^*$.


While the outer parametrix exactly satisfies the same jump conditions satisfied by $\mathbf{W}(\lambda)$ on $\Sigma_g$ and $I$, it is discontinuous near the endpoints of these arcs. Thus, the problem at hand
%The problem at hand in this section 
requires four inner parametrices $\dot{\mathbf{W}}^a(\lambda)$, $\dot{\mathbf{W}}^b(\lambda)$, $\dot{\mathbf{W}}^{\lambda_0}(\lambda)$, and $\dot{\mathbf{W}}^{\lambda_0^*}(\lambda)$ to be defined in disks $D_{\lambda}(\delta)$, centered at the points $\lambda=a(\chi,\tau)$, $b(\chi,\tau)$, $\lambda_0(\chi,\tau)$, $\lambda_0^*(\chi,\tau)$, respectively, where $\delta=\delta(\chi,\tau)>0$ is chosen sufficiently small but independent of $M$. We take the circular boundaries of these disks to have clockwise orientation.

\subsubsection{Inner parametrix construction near the points $a(\chi,\tau)$ and $b(\chi,\tau)$} 
%For the purposes of the local analysis that follows it is convenient to define 
%\begin{equation}
%\tilde{h}(\lambda;\chi,\tau)\defeq  -\ii h(\lambda;\chi,\tau).
%\label{eq:h-tilde}
%\end{equation}
%%For the purposes of local analysis near the points $\lambda=a,b$ it is convenient to define $h(\lambda;\chi,\tau)=:\ii \tilde{h}(\lambda;\chi,\tau)$.
%We see from \eqref{eq:hprime-formula} and \eqref{eq:h-tilde} that the derivative of $\tilde{h}$ is given by
%\begin{equation}
%\tilde{h}'(\lambda;\chi,\tau)=\frac{\left(2 \tau \lambda^{2}+u(\chi, \tau) \lambda+v(\chi, \tau)\right) R(\lambda ; \chi, \tau) }{\lambda^{2}+1},
%\label{eq:h-tilde-prime-formula}
%\end{equation}
%%We start with a direct calculation from \eqref{eq:hprime-formula} and obtain \textcolor{red}{[Where will this go?]}
%%\begin{equation}
%%\begin{split}
%%{h}''(\lambda;\chi,\tau) =& \frac{1}{\lambda^2+1}\left[(4 \tau \lambda + u(\chi,\tau) )R(\lambda;\chi,\tau) +
%%\left(2 \tau \lambda^{2}+u(\chi, \tau) \lambda+v(\chi, \tau)\right) \left(\frac{\lambda- A(\chi,\tau)}{R(\lambda;\chi,\tau)}\right)
%% \right]\\
%% & - \frac{2 \lambda {h}'(\lambda;\chi,\tau)}{\lambda^2 + 1}.
%% \end{split}
%%\end{equation}
%%Noting that $u(\chi,\tau) = - 2 \tau (a(\chi,\tau)+b(\chi,\tau))$ and the fact that the quadratic factor in the numerator in \eqref{eq:hprime-formula} and hence ${h}'(\lambda;\chi,\tau)$ vanishes at $\lambda=a,b$, we see that
%%\begin{align}
%%\tilde{h}''(b(\chi,\tau);\chi,\tau) &= \frac{2\tau(b(\chi,\tau) - a(\chi,\tau))}{b(\chi,\tau)^2+1}R(b(\chi,\tau);\chi,\tau)>0 \label{eq:h-double-prime-b}, \\
%%\tilde{h}_{-}''(a(\chi,\tau);\chi,\tau) &= \frac{2\tau(a(\chi,\tau) - b(\chi,\tau))}{a(\chi,\tau)^2+1}R_{-}(a(\chi,\tau);\chi,\tau)<0 
%%\label{eq:h-double-prime-a},
%%\end{align}
%\begin{align}
%{h}''(b(\chi,\tau);\chi,\tau) &= \frac{2\tau(b(\chi,\tau) - a(\chi,\tau))}{b(\chi,\tau)^2+1}|b(\chi,\tau) - \lambda_0(\chi,\tau)|>0 \label{eq:h-double-prime-b}, \\
%{h}_{-}''(a(\chi,\tau);\chi,\tau) &= \frac{- 2\tau(b(\chi,\tau) - a(\chi,\tau))}{a(\chi,\tau)^2+1} | a(\chi,\tau) - \lambda_0(\chi,\tau)|<0 
%\label{eq:h-double-prime-a},
%\end{align}
%where the inequalities follow because $b(\chi,\tau)>a(\chi,\tau)$ and where we have used the identities \eqref{eq:R-a-b}.

To define an inner parametrix $\dot{\mathbf{W}}^b(\lambda)=\dot{\mathbf{W}}^b(\lambda;\chi,\tau,\mathbf{Q}^{-s},M)$ in $D_b(\delta)$, first note that the properties of $h(\lambda;\chi,\tau)$ summarized at the beginning of this section imply that
${h}(\lambda;\chi,\tau)-{h}(b(\chi,\tau);\chi,\tau)$ is an analytic function of $\lambda$ that vanishes precisely to second order as $\lambda \to b(\chi,\tau)$. 
%first note from the positivity of \eqref{eq:h-double-prime-b} that ${h}(\lambda;\chi,\tau)-{h}(b(\chi,\tau);\chi,\tau)$ vanishes precisely to second order as $\lambda \to b(\chi,\tau)$. 
We introduce an $M$-independent conformal coordinate $f_b$ by setting
\begin{equation}
f_b(\lambda;\chi,\tau)^2 = 2({h}(\lambda;\chi,\tau)-{h}_b(\chi,\tau)),\quad \lambda\in D_b(\delta),
\label{eq:fb-def}
\end{equation}
where ${h}_b(\chi,\tau)\defeq {h}(b(\chi,\tau);\chi,\tau)$,
and choose the solution with $f_b'(b(\chi,\tau);\chi,\tau)>0$. To see why this choice is possible, note that repeated differentiation in \eqref{eq:fb-def} results in the relation
\begin{equation}
f_b'(b(\chi,\tau);\chi,\tau)^2 = h''(b(\chi,\tau);\chi,\tau) > 0.
\label{eq:fb-prime-h-double-prime}
\end{equation}
With this choice
%, which is again possible by the positivity of \eqref{eq:h-double-prime-b}, so that 
the arc $I\cap D_b(\delta)$ is mapped by $\lambda \mapsto f_b(\lambda;\chi,\tau)$ locally to the negative real axis. Then, in the rescaled conformal coordinate $\zeta_b \defeq  M^\frac{1}{2} f_b$, the jump conditions satisfied by the matrix function
\begin{equation}
\mathbf{U}^b(\lambda) \defeq  \mathbf{W}(\lambda) \ii^{\frac{1}{2}(1-s)\sigma_3}
%s^{\frac{1}{2}\sigma_3} 
\ee^{-\ii M {h}_b(\chi,\tau)\sigma_3},\quad \lambda\in D_b(\delta)
\label{eq:W-transformation-b}
\end{equation}
coincide exactly with those of $\mathbf{U}(\zeta)$ described right before \eqref{eq:PCU-asymp} when expressed in terms of the variable $\zeta=\zeta_b$ and the jump contours are locally taken to coincide with the five rays $\arg(\zeta)=\pm \tfrac{1}{4}\pi$, $\arg(\zeta)=\pm \tfrac{3}{4}\pi$, and $\arg(-\zeta)=0$ as shown in \cite[Figure 9]{BilmanLM20}. Therefore, the construction of $\dot{\mathbf{W}}^b(\lambda)$ follows \emph{mutatis mutandis} that of the local parametrix near $b$ in Section~\ref{sec:channels-parametrix}. Indeed, replacing $\vartheta_b$ with $h_b$ and taking into account the fact that $\dot{\mathbf{W}}^\text{out}(\lambda)$ in this section differs from the outer parametrix \eqref{eq:Channels-Tout} in Section~\ref{sec:channels-parametrix} by multiplication on the left by $\mathbf{J}(\lambda)$, one obtains (compare with \eqref{eq:Channels-Tb-ALT})
\begin{equation}
\dot{\mathbf{W}}^b(\lambda)\defeq \mathbf{Y}^b(\lambda)\mathbf{U}(\zeta_b) \ii^{-\frac{1}{2}(1-s)\sigma_3} \ee^{\ii M{h}_b(\chi,\tau)\sigma_3},% s^{-\sigma_3/2},
\end{equation}
where $\mathbf{Y}^b(\lambda)$ is the prefactor that is holomorphic in the disk $D_b(\delta)$ and is given by
\begin{equation}
\mathbf{Y}^b(\lambda)\defeq 
\mathbf{J}(\lambda)
%s^{\sigma_3/2}
M^{\frac{1}{2} \ii p \sigma_3}
 \ee^{-\ii M {h}_b(\chi,\tau)\sigma_3} \ii^{\frac{1}{2}(1-s)\sigma_3}  \mathbf{H}^{b}(\lambda),
%\mathbf{Y}^b(z;M,w,t)\defeq \ee^{-\ii M \kappa(w)\sigma_3/2} \mathbf{X}^b(z;w,t)e^{\ii M \kappa(w)\sigma_3/2}\ee^{-\ii M h(b(w);w)\sigma_3}M^{\ii p \sigma_3/2}
\label{eq:A-b}
\end{equation}
in which $\mathbf{H}^b(\lambda)$ is given \emph{exactly} as in \eqref{eq:Channels-Hb} with the conformal map $f_b(\lambda;\chi,\tau)$ being based on $h$ as in \eqref{eq:fb-def} rather than on $\vartheta$ as in Section~\ref{sec:channels}. $\mathbf{H}^b(\lambda)$ is holomorphic in the disk $D_b(\delta)$.
It is now easy to verify (by drawing a comparison with the construction in Section~\ref{sec:channels-parametrix}) that $\dot{\mathbf{W}}^b(\lambda)$ exactly satisfies the jump conditions for $\mathbf{W}(\lambda)$ in $D_b(\delta)$.
%As $h_b(\chi,\tau)$, $p$, and $\kappa(\chi,\tau)$ are all real valued, $\dot{\mathbf{W}}^b(\lambda)$ remains bounded as $M\to +\infty$ in $D_b(\delta)$. 
Comparing the this parametrix with the outer parametrix (which is unimodular) on the boundary of $D_b(\delta)$, we see that
%with those given 
%in \cite[Riemann-Hilbert Problem 5]{BilmanLM20} when expressed in terms of the variable $\zeta=\zeta_b$ and the jump contours are locally taken to coincide with the five rays $\arg(\zeta)=\pm \pi/4$, $\arg(\zeta)=\pm 3\pi/4$, and $\arg(-\zeta)=0$ as shown in \cite[Figure 9]{BilmanLM20}, which are satisfied exactly by a special case $\mathbf{U}$
%of the parabolic cylinder parametrix. \textcolor{red}{[Link this to the parametrix in the Channels.]} In light of the transformation \eqref{eq:W-transformation-b}, for $\mathbf{W}(\lambda)\dot{\mathbf{W}}^{b}(\lambda)^{-1}$ to be analytic in $D_{b}(\delta)$ we take the inner parametrix $\dot{\mathbf{W}}^{b}(\lambda)=\dot{\mathbf{W}}^{b}(\lambda;\chi,\tau,\mathbf{Q}^{-s},M)$  to be of the form
%\begin{equation}
%\dot{\mathbf{W}}^{b}(\lambda) = \mathbf{A}^{b} (\lambda) \mathbf{U}(M^\frac{1}{2} f_{b}(\lambda;\chi,\tau))\ee^{\ii M \tilde{h}_b(\chi,\tau)\sigma_3} \ii^{-\frac{1}{2}(1-s)\sigma_3},
%\label{eq:Wb-ansatz}
%\end{equation}
%where $\mathbf{A}^{b} (\lambda)$ is a matrix function that is holomorphic in $D_b(\delta)$, to be determined to ensure that $\dot{\mathbf{W}}^{b}(\lambda) \dot{\mathbf{W}}^{\mathrm{out}}(\lambda)^{-1}= \mathbb{I}+o(1)$ for $\lambda\in\partial D_b(\delta)$ as $M\to+\infty$. For $\lambda$ near $b(\chi,\tau)$ the outer parametrix \eqref{eq:W-out-G} may be expressed as
%\begin{equation}
% \dot{\mathbf{W}}^{\mathrm{out}}(\lambda)
% %s^{\sigma_3/2} 
%\ii^{\frac{1}{2}(1-s)\sigma_3} \ee^{- \ii M \tilde{h}_b(\chi,\tau)\sigma_3}
% =\mathbf{G}(\lambda)
%% s^{\sigma_3/2}
% \ii^{\frac{1}{2}(1-s)\sigma_3} \ee^{-\ii M \tilde{h}_b(\chi,\tau)} M^{\frac{1}{2}\ii p \sigma_3} \mathbf{H}^{b}(\lambda) \zeta_b^{-\ii p \sigma_3},
%\end{equation}
%where
%\begin{equation}
%\mathbf{H}^b(\lambda) \defeq  (\lambda-a(\chi,\tau))^{\ii p \sigma_3} \left( \frac{f_b(\lambda;\chi,\tau)}{\lambda-b(\chi,\tau)} \right)^{\ii p \sigma_3},
%\end{equation}
%with both power functions taken to be the principal branch. Thus, the diagonal matrix function $\mathbf{H}^b(\lambda;\chi,\tau)$ is holomorphic in the disk $D_b(\delta)$. The inner parametrix $\dot{\mathbf{W}}^b$ is now defined as 
%%\begin{equation}
%%\dot{\mathbf{X}}^b(\lambda;\chi,\tau)\defeq \mathbf{A}^b(\lambda;\chi,\tau)\mathbf{U}(n^{\frac{1}{2}}f_b(\lambda;\chi,\tau))\ee^{\ii n \tilde{h}_b(\chi,\tau)\sigma_3}\ee^{S(\lambda;\chi,\tau)\sigma_3} \ii^{-\frac{1}{2}(1-s)\sigma_3} \omega(\lambda)^{-s\sigma_3},% s^{-\sigma_3/2},
%%\end{equation}
%%\textcolor{red}{or in the alternate approach:
%\begin{equation}
%\dot{\mathbf{W}}^b(\lambda)\defeq \mathbf{A}^b(\lambda)\mathbf{U}(M^{\frac{1}{2}}f_b(\lambda;\chi,\tau))\ee^{\ii M \tilde{h}_b(\chi,\tau)\sigma_3}  \ii^{-\frac{1}{2}(1-s)\sigma_3},% s^{-\sigma_3/2},
%\end{equation}
%%}
%where
%%\begin{equation}
%%\mathbf{A}^b(\lambda;\chi,\tau)\defeq 
%%\mathbf{G}(\lambda;\chi,\tau)
%%%s^{\sigma_3/2}
%%\omega(\lambda)^{s\sigma_3} \ii^{\frac{1}{2}(1-s)\sigma_3} \ee^{ - S(\lambda;\chi,\tau) \sigma_3}\ee^{-\ii n \tilde{h}_b(\chi,\tau)} n^{\frac{1}{2} \ii p \sigma_3} \mathbf{H}^{b}(\lambda;\chi,\tau) 
%%%\mathbf{Y}^b(z;M,w,t)\defeq \ee^{-\ii M \kappa(w)\sigma_3/2} \mathbf{X}^b(z;w,t)e^{\ii M \kappa(w)\sigma_3/2}\ee^{-\ii M h(b(w);w)\sigma_3}M^{\ii p \sigma_3/2}
%%\label{eq:A-b}
%%\end{equation}
%%\textcolor{red}{or in the alternate approach,
%\begin{equation}
%\mathbf{A}^b(\lambda)\defeq 
%\mathbf{G}(\lambda)
%%s^{\sigma_3/2}
%\ii^{\frac{1}{2}(1-s)\sigma_3} \ee^{-\ii M \tilde{h}_b(\chi,\tau)} M^{\frac{1}{2} \ii p \sigma_3} \mathbf{H}^{b}(\lambda) 
%%\mathbf{Y}^b(z;M,w,t)\defeq \ee^{-\ii M \kappa(w)\sigma_3/2} \mathbf{X}^b(z;w,t)e^{\ii M \kappa(w)\sigma_3/2}\ee^{-\ii M h(b(w);w)\sigma_3}M^{\ii p \sigma_3/2}
%\label{eq:A-b}
%\end{equation}
%%}
%is holomorphic in the disk $D_b(\delta)$. 
%Noting that $-\ii g(\lambda;\chi,\tau) = \tilde{h}(\lambda;\chi,\tau) - \vartheta(\lambda;\chi,\tau)$ and recalling the independence of \eqref{eq:g-integral} from the path of integration in the domain $\mathbb{C}\setminus \Sigma_g$, we may write
%\begin{equation}
%-\ii g_b(\chi,\tau)\defeq -\ii g(b(\chi,\tau);\chi,\tau) = \int_{+\infty}^{b(\chi,\tau)} \left( \tilde{h}'(\lambda;\chi,\tau) -\chi -2\tau\lambda + \frac{2}{\lambda^2+1} \right) \dd \lambda,
%\label{eq:g-at-b}
%\end{equation}
%and since the integrand is real valued on the path $[b(\chi,\tau),+\infty)$ we have $-\ii g_b(\chi,\tau) \in \mathbb{R}$. Then
%\begin{equation}
%\begin{split}
%\tilde{h}_b(\chi,\tau) &= - \ii g_b(\chi,\tau) + \vartheta_b(\chi,\tau)\\
%&= - \ii g_b(\chi,\tau) + \chi b(\chi,\tau) + \tau b(\chi,\tau)^2 + \ii \log\left(\frac{b(\chi,\tau) - \ii}{b(\chi,\tau)+\ii} \right),
%\end{split}
%\label{eq:h-tilde-at-b}
%\end{equation}
%in which the last term is real valued, and this implies that $\ee^{\pm \ii n \tilde{h}_b(\chi,\tau)}$ is bounded as $M\to +\infty$.
%Since $p,\kappa(\chi,\tau)\in\mathbb{R}$ as well, we conclude that the matrix function $\mathbf{A}^b(\lambda)$ remains bounded as $M\to +\infty$ in addition to being holomorphic in $D_b(\delta)$.
%It then follows that for $\lambda\in D_b(\delta)$
%\begin{equation}
%\begin{aligned}
%\dot{\mathbf{X}}^b(\lambda;\chi,\tau)\mathbf{X}^{\mathrm{out}}(\lambda;\chi,\tau)^{-1}&\defeq \mathbf{A}^b(\lambda;\chi,\tau)\mathbf{U}(n^{\frac{1}{2}}f_b(\lambda;\chi,\tau)) \zeta_b^{\ii p \sigma_3}\mathbf{A}^b(\lambda;\chi,\tau)^{-1}\\
%&=\mathbf{A}^b(\lambda;\chi,\tau)\mathbf{U}(\zeta_b) \zeta_b^{\ii p \sigma_3}\mathbf{A}^b(\lambda;\chi,\tau)^{-1},
%\end{aligned}
%\label{eq:error-PC-b}
%\end{equation}
%\textcolor{red}{or in the alternate approach:
%\begin{equation}
%\begin{aligned}
%\dot{\mathbf{W}}^b(\lambda)\mathbf{W}^{\mathrm{out}}(\lambda)^{-1}
%\defeq &\mathbf{A}^b(\lambda)\mathbf{U}(M^{\frac{1}{2}}f_b(\lambda)) \zeta_b^{\ii p \sigma_3}\mathbf{A}^b(\lambda)^{-1}\\
%=&\mathbf{A}^b(\lambda)\mathbf{U}(\zeta_b) \zeta_b^{\ii p \sigma_3}\mathbf{A}^b(\lambda)^{-1},
%\end{aligned}
%\label{eq:error-PC-b}
%\end{equation}
\begin{equation}
\dot{\mathbf{W}}^b(\lambda)\dot{\mathbf{W}}^{\mathrm{out}}(\lambda)^{-1}
=\mathbf{Y}^b(\lambda)\mathbf{U}(\zeta_b) \zeta_b^{\ii p \sigma_3}\mathbf{Y}^b(\lambda)^{-1},
\label{eq:error-PC-b}
\end{equation}
%}
and using the asymptotic expansion \eqref{eq:PCU-asymp} in \eqref{eq:error-PC-b} yields the estimate
\begin{equation}
\sup_{\lambda\in \partial D_b(\delta)}\| \dot{\mathbf{W}}^b(\lambda)\dot{\mathbf{W}}^{\mathrm{out}}(\lambda)^{-1} -\mathbb{I} \| = O(M^{-\frac{1}{2}}),\quad M\to+\infty,
\label{eq:error-PC-disk-b-large-M}
\end{equation}
%}
where $\| \cdot \|$ denotes the matrix norm induced from an arbitrary vector norm on $\mathbb{C}^2$.

%since the parabolic cylinder parametrix $\mathbf{U}$ has the asymptotic expansion (see \cite[Section 4.1.2, Riemann-Hilbert Problem 5 and Eqn.\@ 141]{BilmanLM20}) \textcolor{red}{[Again, modify this by referring to what's been done in the Channels.]}
%\begin{equation}
%\mathbf{U}(\zeta)\zeta^{\ii p \sigma_3} = \mathbb{I} + \frac{1}{2\ii \zeta}\begin{bmatrix} 0 & \alpha \\ -\beta & 0\end{bmatrix} + O(\zeta^{-2}),\quad \zeta\to\infty,
%\label{eq:PC-asymptotics}
%\end{equation}
%where $\alpha$ is defined as in \eqref{eq:alpha-beta-def} and $\beta = -\alpha^*$, we have 
%%\begin{equation}
%%\sup_{\lambda\in \partial D_b(\delta)}\| \dot{\mathbf{X}}^b(\lambda;\chi,\tau) \dot{\mathbf{X}}^{\mathrm{out}}(\lambda;\chi,\tau)^{-1} -\mathbb{I} \| = O(n^{-\frac{1}{2}}),\quad n\to+\infty,
%%\label{eq:error-PC-disk-b-large-n}
%%\end{equation}
%%\textcolor{red}{
%%or in the alternate approach
%\begin{equation}
%\sup_{\lambda\in \partial D_b(\delta)}\| \dot{\mathbf{W}}^b(\lambda)\dot{\mathbf{W}}^{\mathrm{out}}(\lambda)^{-1} -\mathbb{I} \| = O(M^{-\frac{1}{2}}),\quad M\to+\infty,
%\label{eq:error-PC-disk-b-large-M}
%\end{equation}
%%}
%where $\| \cdot \|$ denotes the matrix norm induced from an arbitrary vector norm on $\mathbb{C}^2$.

Constructing an inner parametrix $\dot{\mathbf{W}}^a(\lambda)=\dot{\mathbf{W}}^a(\lambda)(\lambda;\chi,\tau,\mathbf{Q}^{-s},M)$ in the disk $D_a(\delta)$ requires a bit more work due to the presence of the cut $\Sigma_g$ inside $D_a(\delta)$. Note that for $\lambda\in D_a$, ${h}(\lambda;\chi,\tau)$ comprises two different functions that are both analytic in the entire disk $D_a(\delta)$. We will use $\pm$ subscripts to denote these functions: ${h}_{-}(\lambda;\chi,\tau)$ (resp., ${h}_{+}(\lambda;\chi,\tau)$) coincides with ${h}(\lambda;\chi,\tau)$ for $\lambda$ to the right (resp., left) of $\Sigma_g$ with respect to (upward) orientation. These functions are of course related by ${h}_+(\lambda;\chi,\tau) + {h}_-(\lambda;\chi,\tau) = 2 \kappa(\chi,\tau)$ for $ \lambda\in D_{a}(\delta)$,
%${h}_{R}(\lambda;\chi,\tau)$ (respectively ${h}_{L}(\lambda;\chi,\tau)$), which coincides with ${h}(\lambda;\chi,\tau)$ for $\lambda$ to the right (respectively left) of $\Sigma_g$ with respect to the (upward) orientation. The values of these functions for $\lambda \in \Sigma_g$ agree with the boundary values of ${h}(\lambda;\chi,\tau)$ taken on $\Sigma_g$ in the following way: ${h}_{R}(\lambda;\chi,\tau) = 
%{h}_{-}(\lambda;\chi,\tau)$ and ${h}_{L}(\lambda;\chi,\tau) = {h}_{+}(\lambda;\chi,\tau)$ for $\lambda\in\Sigma_g$, and these two functions are of course related by
%\begin{equation}
%{h}_R(\lambda;\chi,\tau) + {h}_L(\lambda;\chi,\tau) = 2 \kappa(\chi,\tau),\quad \lambda\in D_{a}(\delta),
%\end{equation}
where the (real-valued) constant $\kappa(\chi,\tau)$ is given in \eqref{eq:kappa-formula}. By analogy, we denote by $D_{a,-}(\delta)$ (resp., $D_{a,+}(\delta)$) the part of $D_a(\delta)$ that lies to the right (resp., left) of $\Sigma_g$ with respect to orientation. We use the same notational convention for the boundaries of these half-disks: $\partial D_{a,\pm}(\delta)$ denotes the circular boundary of $D_{a,\pm}(\delta)$ (omitting $\Sigma_g$).

We base the definition of a conformal mapping on the analytic function $\lambda\mapsto {h}_-(\lambda;\chi,\tau)$. Again by the properties of $h$ summarized at the beginning of this section, ${h}_-(\lambda;\chi,\tau)-{h}_-(a(\chi,\tau);\chi,\tau)$ vanishes to second order as $\lambda\to a(\chi,\tau)$. We introduce an $M$-independent conformal coordinate $f_a$ by setting
\begin{equation}
%f_a(\lambda;\chi,\tau)^2 = 2(h_R(a(\chi,\tau);\chi,\tau) - h_R(\lambda;\chi,\tau)),\quad \lambda\in D_a,
f_a(\lambda;\chi,\tau)^2 = 2(h_a(\chi,\tau) - h_-(\lambda;\chi,\tau)),\quad \lambda\in D_a(\delta),
\label{eq:fa-def}
\end{equation}
where $h_a(\chi,\tau)\defeq h_{-}(a(\chi,\tau);\chi,\tau)$,
and choose the solution with $f'_a(a(\chi,\tau);\chi,\tau)<0$. This is again possible as one obtains by repeated differentiation in \eqref{eq:fa-def} the relation
\begin{equation}
f_a'(a(\chi,\tau);\chi,\tau)^2= - h_-''(a(\chi,\tau);\chi,\tau) >0,
\label{eq:fa-prime-h-double-prime}
\end{equation}
see \eqref{eq:h-double-prime-a-b-signs}.
With this choice,
%which is possible by the non-vanishing of \eqref{eq:h-double-prime-a}, so that 
the arc $I \cap D_a(\delta)$ is mapped by $\lambda \mapsto f_a(\lambda ;\chi,\tau)$ locally to the negative real axis. In the rescaled conformal coordinate $\zeta_a \defeq M^\frac{1}{2} f_a$, the jump conditions satisfied by the piecewise-defined matrix function
\begin{equation}
\mathbf{U}^a(\lambda) \defeq \begin{cases} 
\mathbf{W} (\lambda)
%s^{\sigma_3/2} 
\ii^{\frac{1}{2}(1-s)\sigma_3}  \ii^{\sigma_3} \ee^{-\ii M {h}_a(\chi,\tau)\sigma_3}(\ii \sigma_2),\quad& \lambda\in D_{a,-}(\delta),\\
\mathbf{W}(\lambda) \ii^{\frac{1}{2}(1-s)\sigma_3} \ee^{-\ii M \kappa(\chi,\tau)\sigma_3}(-\ii \sigma_2)  \ee^{\ii M \kappa(\chi,\tau)\sigma_3}  \ii^{\sigma_3} \ee^{-\ii M {h}_a(\chi,\tau)\sigma_3}(\ii \sigma_2), \quad & \lambda \in D_{a,+}(\delta)
\end{cases}
\label{eq:W-transformation-a}
\end{equation}
coincide exactly with those of $\mathbf{U}(\zeta)$ described right before \eqref{eq:PCU-asymp} again when expressed in terms of the variable $\zeta=\zeta_a$ and the jump contours are locally taken to coincide with the five rays $\arg(\zeta)=\pm \tfrac{1}{4}\pi$, $\arg(\zeta)=\pm \tfrac{3}{4}\pi$, and $\arg(-\zeta)=0$ as shown in \cite[Figure 9]{BilmanLM20}. 
%See Figure~\ref{fig:D-a} for the local jump contours for $\mathbf{W}(\lambda)$ near $\lambda=a(\chi,\tau)$ and the boundaries of the half-disks.
%In the rescaled conformal coordinate $\zeta_a \defeq M^\frac{1}{2} f_a$, the jump conditions satisfied by the piecewise-defined matrix function
%%\begin{equation}
%%\mathbf{U}^a \defeq \begin{cases} 
%%\mathbf{X}^{(k)} 
%%%s^{\sigma_3/2} 
%%\omega(\lambda)^{s\sigma_3} \ii^{\frac{1}{2}(1-s)\sigma_3} \ee^{ - S(\lambda;\chi,\tau) \sigma_3}  \ii^{\sigma_3} \ee^{-\ii n \tilde{h}_a(\chi,\tau)\sigma_3}(\ii \sigma_2),& \lambda\in\Omega_{\pm 2,\pm 3}\\
%%\mathbf{X}^{(k)}  \omega(\lambda)^{s\sigma_3} \ii^{\frac{1}{2}(1-s)\sigma_3} \ee^{ - S(\lambda;\chi,\tau) \sigma_3} \left[ \ee^{-\ii n \kappa(\chi,\tau)\sigma_3}(-\ii \sigma_2)  \ee^{\ii n \kappa(\chi,\tau)\sigma_3} \right] \ii^{\sigma_3} \ee^{-\ii n \tilde{h}_a(\chi,\tau)\sigma_3}(\ii \sigma_2),& \lambda \in\Omega_{0,\pm 1}
%%\end{cases}
%%\label{eq:X-transformation-a}
%%\end{equation}
%%\textcolor{red}{or in the alternate approach:
%\begin{equation}
%\mathbf{U}^a \defeq \begin{cases} 
%\mathbf{W} 
%%s^{\sigma_3/2} 
%\ii^{\frac{1}{2}(1-s)\sigma_3}  \ii^{\sigma_3} \ee^{-\ii M \tilde{h}_a(\chi,\tau)\sigma_3}(\ii \sigma_2),& \lambda\in\Omega_{\pm 2,\pm 3}\\
%\mathbf{W} \ii^{\frac{1}{2}(1-s)\sigma_3} \left[ \ee^{-\ii M \kappa(\chi,\tau)\sigma_3}(-\ii \sigma_2)  \ee^{\ii M \kappa(\chi,\tau)\sigma_3} \right] \ii^{\sigma_3} \ee^{-\ii M \tilde{h}_a(\chi,\tau)\sigma_3}(\ii \sigma_2),& \lambda \in\Omega_{0,\pm 1}
%\end{cases}
%\label{eq:W-transformation-a}
%\end{equation}
%%}
%in $D_a(\delta)$ coincide again with those given in \cite[Rieman-Hilbert Problem 5]{BilmanLM20} when expressed in terms of the variable $\zeta=\zeta_a$ and the jump contours are locally taken to coincide with the five rays $\arg(\zeta)=\pm \pi/4$, $\arg(\zeta)=\pm 3\pi/4$, and $\arg(-\zeta)=0$ (see \cite[Figure 9]{BilmanLM20}). These jump conditions are satisfied again by a special case of the parabolic cylinder parametrix, $\mathbf{U}(\zeta)$.  \textcolor{red}{[Again refer to Channels or whichever appearance of the PC parametrix Channels is currently citing.]}
%See Figure~\ref{fig:D-a} for the definitions of the regions $\Omega_j$, $j=0,\pm 1, \pm 2, \pm 3$, and the jump contours for $\mathbf{W}^{(k)}$.
%\begin{figure}
%\includegraphics{contours-near-a}
%\caption{Local jump contours for $\mathbf{W}(\lambda)$ inside the disk $D_a(\delta)$, the left half-disk boundary $\partial D_{a,L}(\delta)$ (purple) and the right half-disk boundary $\partial D_{a,L}(\delta)$ (green).}
%\label{fig:D-a}
%\end{figure}
In light of the transformation \eqref{eq:W-transformation-a}, for $\mathbf{W}(\lambda)\dot{\mathbf{W}}^{a}(\lambda)^{-1}$ to be analytic in $D_{a}(\delta)$ we take the parametrix $\dot{\mathbf{W}}^{a}(\lambda)$=$\dot{\mathbf{W}}^{a}(\lambda;\chi,\tau,\mathbf{Q}^{-s},M)$ to be of the form
%\begin{equation}
%\dot{\mathbf{X}}^{a}(\lambda)\defeq \begin{cases}
%\mathbf{A}^{a}(\lambda) \mathbf{U}(n^\frac{1}{2}f_a(\lambda;\chi,\tau))(-\ii \sigma_2) \ee^{\ii n \tilde{h}_a(\chi,\tau)\sigma_3} (\ii^{-\sigma_3}) \ee^{S(\lambda;\chi,\tau)\sigma_3} \ii^{-\frac{1}{2}(1-s)\sigma_3}  \omega(\lambda)^{-s\sigma_3} ,& \lambda\in \Omega_{\pm 2, \pm 3}\\
%\mathbf{A}^{a}(\lambda) \mathbf{U}(n^\frac{1}{2} f_a(\lambda;\chi,\tau))(-\ii \sigma_2) \ee^{\ii n \tilde{h}_a(\chi,\tau)\sigma_3}(\ii^{-\sigma_3})\mathbf{K}\ee^{S(\lambda;\chi,\tau)\sigma_3}  \ii^{-\frac{1}{2}(1-s)\sigma_3} \omega(\lambda)^{-s\sigma_3},& \lambda\in \Omega_{0, \pm 1}
%\end{cases}
%\label{eq:X-a}
%\end{equation}
%\textcolor{red}{or, in the alternate approach:
\begin{equation}
\dot{\mathbf{W}}^{a}(\lambda)\defeq \begin{cases}
\mathbf{Y}^{a}(\lambda) \mathbf{U}(\zeta_a)(-\ii \sigma_2) \ee^{\ii M {h}_a(\chi,\tau)\sigma_3} (\ii^{-\sigma_3})  \ii^{-\frac{1}{2}(1-s)\sigma_3}, \quad & \lambda\in D_{a,-}(\delta),\\
\mathbf{Y}^{a}(\lambda) \mathbf{U}(\zeta_a)(-\ii \sigma_2) \ee^{\ii M {h}_a(\chi,\tau)\sigma_3}(\ii^{-\sigma_3})\mathbf{K}(\chi,\tau) \ii^{-\frac{1}{2}(1-s)\sigma_3} , \quad & \lambda\in D_{a,+}(\delta),
\end{cases}
\label{eq:W-a}
\end{equation}
%\begin{equation}
%\dot{\mathbf{W}}^{a}(\lambda)\defeq \begin{cases}
%\mathbf{Y}^{a}(\lambda) \mathbf{U}(\zeta_a)(-\ii \sigma_2) \ee^{\ii M {h}_a(\chi,\tau)\sigma_3} (\ii^{-\sigma_3})  \ii^{-\frac{1}{2}(1-s)\sigma_3}, \quad & \lambda\in D_{a,L}(\delta)\\
%\mathbf{Y}^{a}(\lambda) \mathbf{U}(\zeta_a)(-\ii \sigma_2) \ee^{\ii M {h}_a(\chi,\tau)\sigma_3}(\ii^{-\sigma_3}) \ee^{-\ii M\kappa(\chi,\tau)\sigma_3}(\ii \sigma_2)  \ee^{\ii M \kappa(\chi,\tau)\sigma_3} \ii^{-\frac{1}{2}(1-s)\sigma_3} , \quad & \lambda\in D_{a,R}(\delta)
%\end{cases}
%\label{eq:W-a}
%\end{equation}
%}
where we have set
%\begin{equation}
%\mathbf{K}\defeq \mathbf{K}(n,\chi,\tau) = \ee^{-\ii n\kappa(\chi,\tau)\sigma_3}(\ii \sigma_2)  \ee^{\ii n \kappa(\chi,\tau)\sigma_3}
%\label{eq:K-mat}
%\end{equation}
%\textcolor{red}{or, in the alternate approach:
\begin{equation}
\mathbf{K}(\chi,\tau) \defeq   \ee^{-\ii M\kappa(\chi,\tau)\sigma_3}(\ii \sigma_2)  \ee^{\ii M \kappa(\chi,\tau)\sigma_3}
\label{eq:K-mat}
\end{equation}
%}
for brevity in the expressions, and where $\mathbf{Y}^a(\lambda)$ is a matrix function that is holomorphic in $D_a(\delta)$, to be determined by requiring $\dot{\mathbf{W}}^{a}(\lambda)\dot{\mathbf{W}}^{\mathrm{out}}(\lambda)^{-1}=\mathbb{I}+o(1)$ for $\lambda\in \partial D_a(\delta)$ as $M\to+\infty$.
%Before comparing $\dot{\mathbf{W}}^a(\lambda)$ and $\dot{\mathbf{W}}^{\mathrm{out}}(\lambda)$ on $\partial D_a(\delta)$, 
We note that $\mathbf{J}(\lambda)$ given in \eqref{eq:W-out-G} defines two functions that are analytic in the entire disk $D_a(\delta)$, and we again use $\pm$ subscripts consistent with their boundary values taken on $\Sigma_g$ to label them:
%and we again use the $\pm$ subscripts in labeling them:
%$S_{R}(\lambda;\chi,\tau)$ (respectively $S_L(\lambda;\chi,\tau)$), which coincides with $S(\lambda;\chi,\tau)$ for $\lambda$ to the right (respectively left) of $\Sigma_g$ with respect to the upward orientation; and 
%$\dot{\mathbf{W}}^{\mathrm{out}}_{L}(\lambda)$ (respectively $\dot{\mathbf{W}}^{\mathrm{out}}_{R}(\lambda)$) coincides with $\dot{\mathbf{W}}^{\mathrm{out}}(\lambda)$ for $\lambda$ to the left (respectively right) of $\Sigma_g$ with respect to the orientation. 
$\mathbf{J}_\pm(\lambda)$ coincides with $\mathbf{J}(\lambda)$ for $\lambda\in D_{a,\pm}(\delta)$.
%$\mathbf{J}_+(\lambda)$ (respectively $\mathbf{J}_-(\lambda)$) coincides with $\mathbf{J}(\lambda)$ for $\lambda$ to the left (respectively right) of $\Sigma_g$ in $D_a(\delta)$with respect to the orientation. 
%Before comparing $\dot{\mathbf{W}}^a$ and $\dot{\mathbf{W}}^{\mathrm{out}}$ on $\partial D_a(\delta)$, we note that $\dot{\mathbf{W}}^{\mathrm{out}}(\lambda)$ defines two functions that are analytic in the entire disk $D_a(\delta)$:
%%and we again use the $\pm$ subscripts in labeling them:
%%$S_{R}(\lambda;\chi,\tau)$ (respectively $S_L(\lambda;\chi,\tau)$), which coincides with $S(\lambda;\chi,\tau)$ for $\lambda$ to the right (respectively left) of $\Sigma_g$ with respect to the upward orientation; and 
%%$\dot{\mathbf{W}}^{\mathrm{out}}_{L}(\lambda)$ (respectively $\dot{\mathbf{W}}^{\mathrm{out}}_{R}(\lambda)$) coincides with $\dot{\mathbf{W}}^{\mathrm{out}}(\lambda)$ for $\lambda$ to the left (respectively right) of $\Sigma_g$ with respect to the orientation. 
%$\dot{\mathbf{W}}^{\mathrm{out}}_{L}(\lambda)$ (respectively $\dot{\mathbf{W}}^{\mathrm{out}}_{R}(\lambda)$) coincides with $\dot{\mathbf{W}}^{\mathrm{out}}(\lambda)$ for $\lambda$ to the left (respectively right) of $\Sigma_g$ with respect to the orientation. 
%The values of these functions coincide with the boundary values of $\dot{\mathbf{W}}^\mathrm{out}(\lambda)$ on $\Sigma_g \cap D_a(\delta)$ in the following way: 
%%$S_L(\lambda;\chi,\tau) = S_+(\lambda;\chi,\tau) $ and $S_R(\lambda;\chi,\tau) = S_-(\lambda;\chi,\tau)$ for $\lambda\in \Sigma_g\cap D_a(\delta)$, and 
%$\dot{\mathbf{W}}^{\mathrm{out}}_L(\lambda) = \dot{\mathbf{W}}^{\mathrm{out}}_+(\lambda)$ and $\dot{\mathbf{W}}^{\mathrm{out}}_R(\lambda) = \dot{\mathbf{W}}^{\mathrm{out}}_-(\lambda)$ for $\lambda\in \Sigma_g\cap D_a(\delta)$. 
%It follows from \eqref{eq:W-a} that for $\lambda\in \partial D_{a,-}(\delta)$ we have
It follows from \eqref{eq:W-out-G} and \eqref{eq:W-a} that for $\lambda\in \partial D_{a,-}(\delta)$ we have
%\begin{equation}
%\begin{aligned}
%\dot{\mathbf{X}}^a(\lambda)\dot{\mathbf{X}}^\mathrm{out}(\lambda)^{-1} &= \dot{\mathbf{X}}^a(\lambda)\dot{\mathbf{X}}^\mathrm{out}_R(\lambda)^{-1}\\
%&=\mathbf{A}^{a}(\lambda) \mathbf{U}^{\text{PC}}(\zeta_a)(-\ii\sigma_2) \ee^{\ii n \tilde{h}_a(\chi,\tau)\sigma_3}(\ii^{-\sigma_3})\ee^{S(\lambda;\chi,\tau)\sigma_3}\ii^{-\frac{1}{2}(1-s)\sigma_3}\omega(\lambda)^{-s\sigma_3} \dot{\mathbf{X}}^\mathrm{out}_R(\lambda)^{-1}\\
%&=\mathbf{A}^{a}(\lambda) \mathbf{U}^{\text{PC}}(\zeta_a)\hat{\mathbf{X}}^\mathrm{out}_R(\lambda)^{-1},
%\end{aligned}
%\label{eq:error-right-half-disk-Xhat}
%\end{equation}
%\textcolor{red}{or, in the alternate approach:
%\begin{equation}
%%\begin{aligned}
%\dot{\mathbf{W}}^a(\lambda)\dot{\mathbf{W}}^\mathrm{out}(\lambda)^{-1} 
%%&= \dot{\mathbf{W}}^a(\lambda)\dot{\mathbf{W}}^\mathrm{out}_R(\lambda)^{-1}\\
%=\mathbf{Y}^{a}(\lambda) \mathbf{U}(\zeta_a)(-\ii\sigma_2) \ee^{\ii M {h}_a(\chi,\tau)\sigma_3}(\ii^{-\sigma_3})\ii^{-\frac{1}{2}(1-s)\sigma_3} \dot{\mathbf{W}}^\mathrm{out}_R(\lambda)^{-1}.
%%&=\mathbf{Y}^{a}(\lambda) \mathbf{U}^{\text{PC}}(\zeta_a)\hat{\mathbf{W}}^\mathrm{out}_R(\lambda)^{-1},
%%\end{aligned}
%\label{eq:error-right-half-disk-What}
%\end{equation}
\begin{equation}
%\begin{aligned}
\dot{\mathbf{W}}^a(\lambda)\dot{\mathbf{W}}^\mathrm{out}(\lambda)^{-1} 
%&= \dot{\mathbf{W}}^a(\lambda)\dot{\mathbf{W}}^\mathrm{out}_R(\lambda)^{-1}\\
=\mathbf{Y}^{a}(\lambda) \mathbf{U}(\zeta_a)(-\ii\sigma_2) \ee^{\ii M {h}_a(\chi,\tau)\sigma_3}(\ii^{-\sigma_3})\ii^{-\frac{1}{2}(1-s)\sigma_3} \left(\frac{\lambda-a(\chi,\tau)}{\lambda-b(\chi,\tau)}\right)^{-\ii p\sigma_3} \mathbf{J}_-(\lambda)^{-1}.
%&=\mathbf{Y}^{a}(\lambda) \mathbf{U}^{\text{PC}}(\zeta_a)\hat{\mathbf{W}}^\mathrm{out}_R(\lambda)^{-1},
%\end{aligned}
\label{eq:error-right-half-disk-What}
\end{equation}
On the other hand, the definition \eqref{eq:W-out-G} also yields for $\lambda\in D_{a,-}(\delta)\setminus I$
%\begin{equation}
%\dot{\mathbf{W}}^\mathrm{out}_R(\lambda)  \ii^{\frac{1}{2}(1-s)\sigma_3}  \ii^{\sigma_3}  \ee^{- \ii M {h}_a(\chi,\tau)\sigma_3}(\ii\sigma_2) = 
%\mathbf{J}_R(\lambda)   \ii^{\frac{1}{2}(1-s)\sigma_3} \ii^{\sigma_3}  \ee^{- \ii M {h}_a(\chi,\tau)\sigma_3}
%M^{-\frac{1}{2}\ii p \sigma_3}\mathbf{H}^a(\lambda) \zeta_a^{-\ii p \sigma_3},
%\label{eq:W-hat-R-out}
%\end{equation}
\begin{equation}
\dot{\mathbf{W}}^\mathrm{out}(\lambda)  \ii^{\frac{1}{2}(1-s)\sigma_3}  \ii^{\sigma_3}  \ee^{- \ii M {h}_a(\chi,\tau)\sigma_3}(\ii\sigma_2) = 
\mathbf{J}_{-}(\lambda)   \ii^{\frac{1}{2}(1-s)\sigma_3} \ii^{\sigma_3}  \ee^{- \ii M {h}_a(\chi,\tau)\sigma_3}
M^{-\frac{1}{2}\ii p \sigma_3}\mathbf{H}^a(\lambda) \zeta_a^{-\ii p \sigma_3},
\label{eq:W-hat-R-out}
\end{equation}
where $\mathbf{H}^{a}(\lambda)$ is given \emph{exactly} by the formula \eqref{eq:Channels-Ha} except with the different conformal map $f_a(\lambda;\chi,\tau)$ whose construction \eqref{eq:fa-def} is based on $h_{-}(\lambda;\chi,\tau)$ rather than $\vartheta(\lambda;\chi,\tau)$ as in Section~\ref{sec:channels}. 
%Here $\mathbf{J}_R(\lambda)$ is the matrix function which is holomorphic in $D_a(\delta)$ and which coincides with $\mathbf{J}(\lambda)$ in $D_{a,R}(\delta)$. 
$\mathbf{H}^{a}(\lambda)$ is holomorphic in the entire disk $D_{a}(\delta)$.
%(compare with \eqref{eq:Channels-Ha} noting the different conformal map based on $h$ instead of $\vartheta$)
%\begin{equation}
%\mathbf{H}^{a}(\lambda)\defeq (b(\chi,\tau)-\lambda)^{-\mathrm{i} p \sigma_{3}}\left(\frac{a(\chi,\tau)-\lambda}{f_{a}(\lambda ; \chi,\tau)}\right)^{\mathrm{i} p \sigma_{3}}\left(\mathrm{i} \sigma_{2}\right)
%\end{equation}
%with the power functions taken to be the principle branch, making it holomorphic in the entire disk $D_{a}(\delta)$. 
Using \eqref{eq:W-hat-R-out} in \eqref{eq:error-right-half-disk-What} guides us to choose the prefactor $\mathbf{Y}^a(\lambda)$ to be
\begin{equation}
\mathbf{Y}^a(\lambda)\defeq \mathbf{J}_{-}(\lambda)  M^{-\frac{1}{2}\ii p \sigma_3} \ee^{- \ii M {h}_a(\chi,\tau)\sigma_3} \ii^{\frac{1}{2}(1-s)\sigma_3} \ii^{\sigma_3}  
\mathbf{H}^a(\lambda),
\label{eq:A-a}
\end{equation}
which is holomorphic in the entire disk $D_a(\delta)$ and unimodular.
Since $\mathbf{Y}^a(\lambda)$ is now determined, $\dot{\mathbf{W}}^a(\lambda)$ is determined according to \eqref{eq:W-a} and it follows that the mismatch \eqref{eq:error-right-half-disk-What} 
between the inner parametrix and the outer parametrix along $\partial D_{a,-}(\delta)$ reads:
\begin{equation}
\dot{\mathbf{W}}^{a}(\lambda) \dot{\mathbf{W}}^{\text {out }}(\lambda)^{-1}=\mathbf{Y}^{a}(\lambda)\mathbf{U}\left(\zeta_{a}\right) \zeta_{a}^{\mathrm{i} p \sigma_{3}} \mathbf{Y}^{a}(\lambda )^{-1},\quad \lambda \in \partial D_{a,-}(\delta).
\label{eq:mismatch-right-half-disk}
\end{equation}
However, we have used only the information in the right half-disk $D_{a,R}(\delta)$ to construct $\dot{\mathbf{W}}^a(\lambda)$, and one needs to check whether \eqref{eq:mismatch-right-half-disk} actually holds on the entire disk boundary $\partial D_{a}(\delta)$. This can be verified by a direct calculation using the relation \eqref{eq:W-jump-Sigma-g},
%between $\dot{\mathbf{W}}^{\mathrm{out}}_L(\lambda)$ and $\dot{\mathbf{W}}^{\mathrm{out}}_R(\lambda)$,
%\begin{equation}
%\dot{\mathbf{W}}^{\mathrm{out}}_L(\lambda)  =\dot{\mathbf{W}}^{\mathrm{out}}_R(\lambda) 
%%\ee^{-\ii M\kappa(\chi,\tau)\sigma_3} \ii^{\frac{1}{2}(1-s)\sigma_3} (\ii\sigma_2) \ii^{-\frac{1}{2}(1-s)\sigma_3} \ee^{\ii M\kappa(\chi,\tau)\sigma_3}.
%\ii^{\frac{1}{2}(1-s)\sigma_3} \mathbf{K}(\chi,\tau) \ii^{-\frac{1}{2}(1-s)\sigma_3}, \quad \lambda \in D_a(\delta),
%\label{eq:f-W-out-L-to-R}
%\end{equation}
and we indeed have
 \begin{equation}
\dot{\mathbf{W}}^a(\lambda)\dot{\mathbf{W}}^\mathrm{out}(\lambda)^{-1} = \mathbf{Y}^a(\lambda)\mathbf{U}(\zeta_a) \zeta_a^{\ii p \sigma_3} \mathbf{Y}^a(\lambda)^{-1},\quad \lambda\in\partial D_a(\delta).
\label{eq:error-PC-a}
\end{equation}
By construction, $\dot{\mathbf{W}}^a(\lambda)$ exactly satisfies the jump conditions for $\mathbf{W}^a(\lambda)$ in $D_a(\delta)$.
%Just as in the construction of $\dot{\mathbf{W}}^b(\lambda)$, because $a(\chi,\tau)\in \mathbb{R}$ implies that $h_a(\chi,\tau)$ is real valued, it follows that $\dot{\mathbf{W}}^a(\lambda)$ remains bounded as $M \to +\infty$. 
Then
using the asymptotic expansion \eqref{eq:PCU-asymp} in \eqref{eq:error-PC-a}, we obtain the estimate
\begin{equation}
\sup_{\lambda \in \partial D_a(\delta)}\| \dot{\mathbf{W}}^a(\lambda) \dot{\mathbf{W}}^{\mathrm{out}}(\lambda)^{-1} -\mathbb{I} \| = O(M^{-\frac{1}{2}}),\quad M\to+\infty.
\label{eq:error-PC-disk-a-large-M}
\end{equation}

%Using the asymptotic expansion \eqref{eq:PC-asymptotics} for the (parabolic cylinder) matrix function $\mathbf{U}^\mathrm{PC}$ we obtain as in \eqref{eq:error-PC-disk-b-large-M} the asymptotic estimate \textcolor{red}{[Again refer to Channels.]}
%\begin{equation}
%\sup_{\lambda \in \partial D_a(\delta)}\| \dot{\mathbf{W}}^a(\lambda) \dot{\mathbf{W}}^{\mathrm{out}}(\lambda)^{-1} -\mathbb{I} \| = O(M^{-\frac{1}{2}}),\quad M\to+\infty.
%\label{eq:error-PC-disk-a-large-M}
%\end{equation}






%Thus,
%%Thus, we have the following relation which hold for $\lambda \in D_a(\delta)$
%%\begin{equation}
%%\begin{aligned}
%%S_L(\lambda;\chi,\tau)  + S_R(\lambda;\chi,\tau)   &= 2 \ii s \gamma(\chi,\tau) + 2s \log(\omega(\lambda)),\\
%%\dot{\mathbf{X}}^{\mathrm{out}}_L(\lambda; n,\chi,\tau)  &=\dot{\mathbf{X}}^{\mathrm{out}}_R(\lambda;\chi,\tau)  \ee^{-\ii(n\kappa(\chi,\tau)+s \gamma(\chi,\tau))\sigma_3} \ii^{\frac{1}{2}(1-s)\sigma_3} (\ii\sigma_2) \ii^{-\frac{1}{2}(1-s)\sigma_3} \ee^{\ii(n\kappa(\chi,\tau)+s \gamma(\chi,\tau))\sigma_3}.
%%\end{aligned}
%%\label{eq:f-X-out-L-to-R}
%%\end{equation}
%%\textcolor{red}{or, in the alternate approach we only have:
%\begin{equation}
%\dot{\mathbf{W}}^{\mathrm{out}}_L(\lambda)  =\dot{\mathbf{W}}^{\mathrm{out}}_R(\lambda) 
%%\ee^{-\ii M\kappa(\chi,\tau)\sigma_3} \ii^{\frac{1}{2}(1-s)\sigma_3} (\ii\sigma_2) \ii^{-\frac{1}{2}(1-s)\sigma_3} \ee^{\ii M\kappa(\chi,\tau)\sigma_3}.
%\ii^{\frac{1}{2}(1-s)\sigma_3} \mathbf{K}(\chi,\tau) \ii^{-\frac{1}{2}(1-s)\sigma_3}, \quad \lambda \in D_a(\delta).
%\label{eq:f-W-out-L-to-R}
%\end{equation}
%%}
%
%%We denote by $\partial D_{a,L}(\delta)$ (respectively $\partial D_{a,R}(\delta)$) the part of $\partial D_a(\delta)$ that lies to the left (respectively right) of $\Sigma_g$ with respect to orientation.
%For $\lambda\in \partial D_{a,R}(\delta)$ we have
%%\begin{equation}
%%\begin{aligned}
%%\dot{\mathbf{X}}^a(\lambda)\dot{\mathbf{X}}^\mathrm{out}(\lambda)^{-1} &= \dot{\mathbf{X}}^a(\lambda)\dot{\mathbf{X}}^\mathrm{out}_R(\lambda)^{-1}\\
%%&=\mathbf{A}^{a}(\lambda) \mathbf{U}^{\text{PC}}(\zeta_a)(-\ii\sigma_2) \ee^{\ii n \tilde{h}_a(\chi,\tau)\sigma_3}(\ii^{-\sigma_3})\ee^{S(\lambda;\chi,\tau)\sigma_3}\ii^{-\frac{1}{2}(1-s)\sigma_3}\omega(\lambda)^{-s\sigma_3} \dot{\mathbf{X}}^\mathrm{out}_R(\lambda)^{-1}\\
%%&=\mathbf{A}^{a}(\lambda) \mathbf{U}^{\text{PC}}(\zeta_a)\hat{\mathbf{X}}^\mathrm{out}_R(\lambda)^{-1},
%%\end{aligned}
%%\label{eq:error-right-half-disk-Xhat}
%%\end{equation}
%%\textcolor{red}{or, in the alternate approach:
%\begin{equation}
%\begin{aligned}
%\dot{\mathbf{W}}^a(\lambda)\dot{\mathbf{W}}^\mathrm{out}(\lambda)^{-1} &= \dot{\mathbf{W}}^a(\lambda)\dot{\mathbf{W}}^\mathrm{out}_R(\lambda)^{-1}\\
%&=\mathbf{A}^{a}(\lambda) \mathbf{U}^{\text{PC}}(\zeta_a)(-\ii\sigma_2) \ee^{\ii M \tilde{h}_a(\chi,\tau)\sigma_3}(\ii^{-\sigma_3})\ii^{-\frac{1}{2}(1-s)\sigma_3} \dot{\mathbf{W}}^\mathrm{out}_R(\lambda)^{-1}\\
%&=\mathbf{A}^{a}(\lambda) \mathbf{U}^{\text{PC}}(\zeta_a)\hat{\mathbf{W}}^\mathrm{out}_R(\lambda)^{-1},
%\end{aligned}
%\label{eq:error-right-half-disk-What}
%\end{equation}
%%}
%where we have set
%%\begin{equation}
%%\hat{\mathbf{X}}^\mathrm{out}_R(\lambda)=\hat{\mathbf{X}}^\mathrm{out}_R(\lambda;\chi,\tau)\defeq \dot{\mathbf{X}}^\mathrm{out}_R(\lambda;\chi,\tau)  \omega(\lambda)^{s\sigma_3} \ii^{\frac{1}{2}(1-s)\sigma_3} \ee^{-S_R(\lambda;\chi,\tau))\sigma_3}\ii^{\sigma_3}  \ee^{- \ii n \tilde{h}_a(\chi,\tau)\sigma_3}(\ii\sigma_2).
%%\end{equation}
%%\textcolor{red}{or, in the alternate approach:
%\begin{equation}
%\hat{\mathbf{W}}^\mathrm{out}_R(\lambda)\defeq \dot{\mathbf{W}}^\mathrm{out}_R(\lambda)  \ii^{\frac{1}{2}(1-s)\sigma_3}  \ii^{\sigma_3}  \ee^{- \ii M \tilde{h}_a(\chi,\tau)\sigma_3}(\ii\sigma_2).
%\label{eq:W-hat-R-out}
%\end{equation}
%%}
%for brevity. Using the definition \eqref{eq:W-out-G} of the outer parametrix enables us to express this as
%%\begin{equation}
%%\begin{split}
%%\hat{\mathbf{X}}^\mathrm{out}_R(\lambda) &= \mathbf{G}_R(\lambda) \omega(\lambda)^{s\sigma_3}  \ii^{\frac{1}{2}(1-s)\sigma_3} \ee^{-S_R(\lambda;\chi,\tau))\sigma_3}\ii^{\sigma_3}  \ee^{- \ii n \tilde{h}_a(\chi,\tau)\sigma_3}
%%\left(\frac{\lambda-a(\chi,\tau)}{\lambda-b(\chi,\tau)}\right)^{\mathrm{i} p \sigma_{3}}\left(\mathrm{i} \sigma_{2}\right)\\
%%&=\mathbf{G}_R(\lambda) \omega(\lambda)^{s\sigma_3}  \ii^{\frac{1}{2}(1-s)\sigma_3} \ee^{-S_R(\lambda;\chi,\tau))\sigma_3}\ii^{\sigma_3}  \ee^{- \ii n \tilde{h}_a(\chi,\tau)\sigma_3}
%%n^{-\frac{1}{2}\ii p \sigma_3}\mathbf{H}^a(\lambda) \zeta_a^{-\ii p \sigma_3},
%%\end{split}
%%\end{equation}
%%\textcolor{red}{or in the alternate approach:
%\begin{equation}
%\begin{split}
%\hat{\mathbf{W}}^\mathrm{out}_R(\lambda) &= \mathbf{G}_R(\lambda) \ii^{\frac{1}{2}(1-s)\sigma_3} \ii^{\sigma_3}  \ee^{- \ii M \tilde{h}_a(\chi,\tau)\sigma_3}
%\left(\frac{\lambda-a(\chi,\tau)}{\lambda-b(\chi,\tau)}\right)^{\mathrm{i} p \sigma_{3}}\left(\mathrm{i} \sigma_{2}\right)\\
%&=\mathbf{G}_R(\lambda)   \ii^{\frac{1}{2}(1-s)\sigma_3} \ii^{\sigma_3}  \ee^{- \ii M \tilde{h}_a(\chi,\tau)\sigma_3}
%M^{-\frac{1}{2}\ii p \sigma_3}\mathbf{H}^a(\lambda) \zeta_a^{-\ii p \sigma_3},
%\end{split}
%\end{equation}
%%}
%where $\mathbf{G}_R(\lambda)$ is the matrix function that is analytic in $D_a(\delta)$ satisfying $\mathbf{G}_{R}(\lambda)=\mathbf{G}(\lambda)$ for $z \in D_{a}(\delta)$ to the right of $\Sigma_g$ with respect to the (upward) orientation, and $\mathbf{H}^{a}(\lambda ) $ is the matrix function
%\begin{equation}
%\mathbf{H}^{a}(\lambda)\defeq (b(\chi,\tau)-\lambda)^{-\mathrm{i} p \sigma_{3}}\left(\frac{a(\chi,\tau)-\lambda}{f_{a}(\lambda ; \chi,\tau)}\right)^{\mathrm{i} p \sigma_{3}}\left(\mathrm{i} \sigma_{2}\right)
%\end{equation}
%with the power functions taken to be the principle branch, making it holomorphic in the disk $D_{a}(\delta)$. 
%%It is easy to see that $\mathbf{H}^{a}(\lambda;\chi,\tau)$ is holomorphic in the disk $D_a(\delta)$, and 
%We now determine the holomorphic prefactor matrix $\mathbf{A}^a(\lambda)$ in \eqref{eq:W-a} to be
%%\begin{equation}
%%\mathbf{A}^a(\lambda;\chi,\tau)\defeq \mathbf{G}_R(\lambda;\chi,\tau) \omega(\lambda)^{s\sigma_3} \ii^{\frac{1}{2}(1-s)\sigma_3} \ee^{-S_R(\lambda;\chi,\tau))\sigma_3}\ii^{\sigma_3}  \ee^{- \ii n \tilde{h}_a(\chi,\tau)\sigma_3}
%%n^{-\frac{1}{2}\ii p \sigma_3}\mathbf{H}^a(\lambda;\chi,\tau)
%%\label{eq:A-a}
%%\end{equation}
%%\textcolor{red}{or in the alternate approach:
%\begin{equation}
%\mathbf{A}^a(\lambda)\defeq \mathbf{G}_R(\lambda)  \ii^{\frac{1}{2}(1-s)\sigma_3} \ii^{\sigma_3}  \ee^{- \ii M \tilde{h}_a(\chi,\tau)\sigma_3}
%M^{-\frac{1}{2}\ii p \sigma_3}\mathbf{H}^a(\lambda)
%\label{eq:A-a}
%\end{equation}
%%}
%A calculation analogous to one given in \eqref{eq:g-at-b}--\eqref{eq:h-tilde-at-b} shows that 
%%$\ee^{\pm \ii n \tilde{h}_a(\chi,\tau)}$ \textcolor{red}{(alternately, 
%$\ee^{\pm \ii M \tilde{h}_a(\chi,\tau)}$ has modulus $1$, hence remains bounded as $M\to+\infty$. Since $p,\kappa(\chi,\tau)\in\mathbb{R}$ as well, it follows that the matrix function $\mathbf{A}^a(\lambda)$ is bounded as $M\to+\infty$. The mismatch between the inner parametrix and the outer parametrix on the right component $\partial D_{a,R}(\delta)$ of the boundary of the disk $D_a(\delta)$ is expressed as:
%%\begin{equation}
%%\mathbf{X}^{a}(\lambda ; \chi, \tau) \dot{\mathbf{X}}^{\text {out }}(\lambda ; \chi, \tau)^{-1}=\mathbf{A}^{a}(\lambda ; \chi, \tau)\mathbf{U}\left(\zeta_{a}\right) \zeta_{a}^{\mathrm{i} p \sigma_{3}} \mathbf{A}^{a}(\lambda ; \chi, \tau)^{-1},\quad \lambda \in \partial D_{a,R}.
%%\label{eq:mismatch-right-half-disk}
%%\end{equation}
%%\textcolor{red}{in the alternate approach:
%\begin{equation}
%\mathbf{W}^{a}(\lambda) \dot{\mathbf{W}}^{\text {out }}(\lambda)^{-1}=\mathbf{A}^{a}(\lambda)\mathbf{U}\left(\zeta_{a}\right) \zeta_{a}^{\mathrm{i} p \sigma_{3}} \mathbf{A}^{a}(\lambda )^{-1},\quad \lambda \in \partial D_{a,R}.
%\label{eq:mismatch-right-half-disk}
%\end{equation}
%%}
%We will now show that this formula holds on the entire boundary $\partial D_{a}(\delta)$. For $\lambda\in \partial D_{a,L}(\delta)$ we have
%%\textcolor{red}{Go on consistently.}
%%\begin{equation}
%%\begin{split}
%%\dot{\mathbf{X}}^{a}(\lambda) \dot{\mathbf{X}}^{\text {out }}(\lambda)^{-1} &= \dot{\mathbf{X}}^{a}(\lambda) \dot{\mathbf{X}}_{L}^{\text {out }}(\lambda)^{-1}\\
%%&=\mathbf{A}^{a}(\lambda) \mathbf{U}^{\text{PC}}(\zeta_a)(-\ii\sigma_2) \ee^{\ii n \tilde{h}_a(\chi,\tau)\sigma_3}(\ii^{-\sigma_3})\mathbf{K}\ee^{S_L(\lambda;\chi,\tau)\sigma_3} \ii^{-\frac{1}{2}(1-s)\sigma_3} \omega(\lambda)^{-s \sigma_3}  \dot{\mathbf{X}}^\mathrm{out}_L(\lambda)^{-1}\\
%%&=\mathbf{A}^{a}(\lambda) \mathbf{U}\left(\zeta_{a}\right) \hat{\mathbf{X}}_{L}^{\text {out }}(\lambda)^{-1}
%%\end{split}
%%\label{eq:error-left-half-disk-Xhat}
%%\end{equation}
%%\textcolor{red}{or in the alternate approach:
%\begin{equation}
%\begin{split}
%\dot{\mathbf{W}}^{a}(\lambda;\chi,\tau) \dot{\mathbf{W}}^{\text {out }}(\lambda;\chi,\tau)^{-1} &= \dot{\mathbf{W}}^{a}(\lambda;\chi,\tau)\dot{\mathbf{W}}_{L}^{\text {out }}(\lambda;\chi,\tau)^{-1}\\
%&=\mathbf{A}^{a}(\lambda;\chi,\tau) \mathbf{U}^{\text{PC}}(\zeta_a)(-\ii\sigma_2) \ee^{\ii M \tilde{h}_a(\chi,\tau)\sigma_3}(\ii^{-\sigma_3})\mathbf{K}(\chi,\tau)\ii^{-\frac{1}{2}(1-s)\sigma_3}  \dot{\mathbf{W}}^\mathrm{out}_L(\lambda;\chi,\tau)^{-1}\\
%&=\mathbf{A}^{a}(\lambda;\chi,\tau) \mathbf{U}\left(\zeta_{a}\right) \hat{\mathbf{W}}_{L}^{\text {out }}(\lambda;\chi,\tau)^{-1},
%\end{split}
%\label{eq:error-left-half-disk-What}
%\end{equation}
%%}
%where in analogy with \eqref{eq:error-right-half-disk-What} we have set
%%\begin{equation}
%%\hat{\mathbf{W}}^\mathrm{out}_L(\lambda;\chi,\tau)=\hat{\mathbf{X}}^\mathrm{out}_L(\lambda;\chi,\tau)\defeq \dot{\mathbf{X}}^\mathrm{out}_L(\lambda;\chi,\tau) 
%% \omega(\lambda;\chi,\tau)^{s\sigma_3} \ii^{\frac{1}{2}(1-s)\sigma_3} \ee^{-S_L(\lambda;\chi,\tau)\sigma_3} \mathbf{K}^{-1} \ii^{\sigma_3}  \ee^{- \ii n \tilde{h}_a(\chi,\tau)\sigma_3}(\ii\sigma_2).
%%\end{equation}
%%\textcolor{red}{or in the alternate approach:
%\begin{equation}
%\hat{\mathbf{W}}^\mathrm{out}_L(\lambda)\defeq \dot{\mathbf{W}}^\mathrm{out}_L(\lambda) 
% \ii^{\frac{1}{2}(1-s)\sigma_3} \mathbf{K}(\chi,\tau)^{-1} \ii^{\sigma_3}  \ee^{- \ii M \tilde{h}_a(\chi,\tau)\sigma_3}(\ii\sigma_2).
%\end{equation}
%%}
%Recalling the definitions \eqref{eq:W-a} and \eqref{eq:K-mat}, the relations \eqref{eq:f-W-out-L-to-R}, we see that
%%\begin{equation}
%%\begin{split}
%%\hat{\mathbf{W}}^{\mathrm{out}}_L(\lambda;\chi,\tau) 
%%&= \dot{\mathbf{W}}^{\mathrm{out}}_L(\lambda;\chi,\tau)  \ii^{\frac{1}{2}(1-s)\sigma_3}
%%\mathbf{K}(\chi,\tau)^{-1} \ii^{\sigma_3} \ee^{-\ii M \tilde{h}_a\sigma_3}(\ii \sigma_2)\\
%%&= \dot{\mathbf{W}}^{\mathrm{out}}_R(\lambda;\chi,\tau) \left[ \ii^{\frac{1}{2}(1-s)\sigma_3} \mathbf{K}(\chi,\tau) \ii^{-\frac{1}{2}(1-s)\sigma_3} \right]
%%\ii^{\frac{1}{2}(1-s)\sigma_3} \mathbf{K}(\chi,\tau)^{-1} \ii^{\sigma_3} \ee^{-\ii M \tilde{h}_a\sigma_3}(\ii \sigma_2)\\
%%&= \dot{\mathbf{W}}^{\mathrm{out}}_R(\lambda;\chi,\tau) \ii^{\frac{1}{2}(1-s)\sigma_3} \mathbf{K} \ee^{2 \ii s \gamma \sigma_3} \omega(\lambda;\chi,\tau)^{2s\sigma_3} \ee^{- S_L(\lambda;\chi,\tau)\sigma_3} \omega(\lambda;\chi,\tau)^{-s\sigma_3}\mathbf{K}^{-1} \ii^{\sigma_3} \ee^{-\ii n \tilde{h}_a\sigma_3}(\ii \sigma_2)\\
%%&=\dot{\mathbf{X}}^{\mathrm{out}}_R(\lambda;\chi,\tau) \ii^{\frac{1}{2}(1-s)\sigma_3} \mathbf{K}\ee^{S_R(\lambda;\chi,\tau)\sigma_3} \omega(\lambda;\chi,\tau)^{-s\sigma_3}\mathbf{K}^{-1} \ii^{\sigma_3} \ee^{-\ii n \tilde{h}_a\sigma_3}(\ii \sigma_2)\\
%%&=\dot{\mathbf{X}}^{\mathrm{out}}_R(\lambda;\chi,\tau) \omega(\lambda;\chi,\tau)^{s\sigma_3}  \ii^{\frac{1}{2}(1-s)\sigma_3}  \ee^{-S_R(\lambda;\chi,\tau)\sigma_3} \ii^{\sigma_3} \ee^{-\ii n \tilde{h}_a\sigma_3}(\ii \sigma_2)\\
%%&=\hat{\mathbf{X}}^{\mathrm{out}}_R(\lambda;\chi,\tau).
%%\end{split}
%%\end{equation}
%\begin{equation}
%\begin{split}
%\hat{\mathbf{W}}^{\mathrm{out}}_L(\lambda) 
%&= \dot{\mathbf{W}}^{\mathrm{out}}_L(\lambda)  \ii^{\frac{1}{2}(1-s)\sigma_3}
%\mathbf{K}(\chi,\tau)^{-1} \ii^{\sigma_3} \ee^{-\ii M \tilde{h}_a(\chi,\tau)\sigma_3}(\ii \sigma_2)\\
%&= \dot{\mathbf{W}}^{\mathrm{out}}_R(\lambda) \left[ \ii^{\frac{1}{2}(1-s)\sigma_3} \mathbf{K}(\chi,\tau) \ii^{-\frac{1}{2}(1-s)\sigma_3} \right]
%\ii^{\frac{1}{2}(1-s)\sigma_3} \mathbf{K}(\chi,\tau)^{-1} \ii^{\sigma_3} \ee^{-\ii M \tilde{h}_a(\chi,\tau)\sigma_3}(\ii \sigma_2)\\
%&= \dot{\mathbf{W}}^{\mathrm{out}}_R(\lambda) \ii^{\frac{1}{2}(1-s)\sigma_3} \ii^{\sigma_3} \ee^{-\ii M \tilde{h}_a(\chi,\tau)\sigma_3}(\ii \sigma_2)\\
%%&=\dot{\mathbf{X}}^{\mathrm{out}}_R(\lambda;\chi,\tau) \ii^{\frac{1}{2}(1-s)\sigma_3} \mathbf{K}\ee^{S_R(\lambda;\chi,\tau)\sigma_3} \omega(\lambda;\chi,\tau)^{-s\sigma_3}\mathbf{K}^{-1} \ii^{\sigma_3} \ee^{-\ii n \tilde{h}_a\sigma_3}(\ii \sigma_2)\\
%%&=\dot{\mathbf{X}}^{\mathrm{out}}_R(\lambda;\chi,\tau) \omega(\lambda;\chi,\tau)^{s\sigma_3}  \ii^{\frac{1}{2}(1-s)\sigma_3}  \ee^{-S_R(\lambda;\chi,\tau)\sigma_3} \ii^{\sigma_3} \ee^{-\ii n \tilde{h}_a\sigma_3}(\ii \sigma_2)\\
%&=\hat{\mathbf{W}}^{\mathrm{out}}_R(\lambda),
%\end{split}
%\end{equation}
%where we recalled the definition \eqref{eq:W-hat-R-out}.
%It now follows from the formul\ae{} \eqref{eq:error-right-half-disk-What} and \eqref{eq:error-left-half-disk-What} that the formula \eqref{eq:mismatch-right-half-disk} expresses the mismatch between $\dot{\mathbf{W}}^a(\lambda;\chi,\tau)$ and $\dot{\mathbf{W}}^\mathrm{out}(\lambda;\chi,\tau)$ on the entire boundary $\partial D_a(\delta)$, that is, we have
% \begin{equation}
%\dot{\mathbf{W}}^a(\lambda)\dot{\mathbf{W}}^\mathrm{out}(\lambda)^{-1} = \mathbf{A}^a(\lambda)\mathbf{U}(\zeta_a) \zeta_a^{\ii p \sigma_3} \mathbf{A}^a(\lambda)^{-1},\quad \lambda\in\partial D_a(\delta).
%\label{eq:error-PC-a}
%\end{equation}
%Using the asymptotic expansion \eqref{eq:PC-asymptotics} for the (parabolic cylinder) matrix function $\mathbf{U}^\mathrm{PC}$ we obtain as in \eqref{eq:error-PC-disk-b-large-M} the asymptotic estimate \textcolor{red}{[Again refer to Channels.]}
%\begin{equation}
%\sup_{\lambda \in \partial D_a(\delta)}\| \dot{\mathbf{W}}^a(\lambda) \dot{\mathbf{W}}^{\mathrm{out}}(\lambda)^{-1} -\mathbb{I} \| = O(M^{-\frac{1}{2}}),\quad M\to+\infty.
%\label{eq:error-PC-disk-a-large-M}
%\end{equation}

\subsubsection{Inner parametrix construction near the points $\lambda_0(\chi,\tau)$ and $\lambda_0(\chi,\tau)^*$} We now let $D_{\lambda_0}(\delta)$ and $D_{\lambda_0^*}(\delta) = D_{\lambda_0}(\delta)^*$ denote disks of small radius $\delta$ independent of $M$ centered at $\lambda=\lambda_0(\chi,\tau)$ and $\lambda=\lambda_0(\chi,\tau)^*$ respectively. Recalling that $h(\lambda_0(\chi,\tau);\chi,\tau) = \kappa(\chi,\tau)$ and $h'(\lambda;\chi,\tau)$ vanishes like a square root as $\lambda\to\lambda_0(\chi,\tau)$, a procedure almost exactly like the one following \eqref{eq:Airy-map-Schi-Stau-ALT} in Section~\ref{sec:Airy-parametrix} (replacing both of the boundary values $h_\pm$ in \eqref{eq:Airy-map-Schi-Stau-ALT} with $h$) leads to the construction of an inner parametrix $\dot{\mathbf{W}}^{\lambda_0}(\lambda)$ on $D_{\lambda_0}(\delta)$ in terms of Airy functions which takes continuous boundary values and satisfies exactly the same jump conditions within $D_{\lambda_0}(\delta)$ as $\mathbf{W}(\lambda)$. Moreover, across the boundary $\partial D_{\lambda_0}(\delta)$ this inner parametrix satisfies
\begin{equation}
\sup_{\lambda \in \partial D_{\lambda_0} } \| \dot{\mathbf{W}}^{\lambda_0}(\lambda)\dot{\mathbf{W}}^{\mathrm{out}}(\lambda)^{-1} -\mathbb{I} \| = O(M^{-1}),\quad M\to+\infty.
\label{eq:error-Airy-disk-plus}
\end{equation}
Since the matrix $\mathbf{W}(\lambda)$ satisfies $\mathbf{W}(\lambda^*)=\sigma_2\mathbf{W}(\lambda)^*\sigma_2$, we may define as in Section~\ref{sec:Airy-parametrix} a second inner parametrix for $\lambda\in D_{\lambda_0^*}(\delta)$ to respect this symmetry, which, of course, satisfies
\begin{equation}
\sup_{\lambda \in \partial D_{\lambda_0^*} } \| \dot{\mathbf{W}}^{\lambda_0^*}(\lambda)\dot{\mathbf{W}}^{\mathrm{out}}(\lambda)^{-1} -\mathbb{I} \| = O(M^{-1}),\quad M\to+\infty.
\label{eq:error-Airy-disk-minus}
\end{equation}

A \emph{global parametrix} $\dot{\mathbf{W}}(\lambda)=\dot{\mathbf{W}}(\lambda;\chi,\tau,\mathbf{Q}^{-s},M)$ is finally constructed by assembling the outer and inner parametrices as follows:
\begin{equation}
\dot{\mathbf{W}}(\lambda)\defeq 
\begin{cases}
\dot{\mathbf{W}}^{\lambda_0} (\lambda),&\quad \lambda\in D_{\lambda_0}(\delta),\\
\dot{\mathbf{W}}^{\lambda_0^*} (\lambda),&\quad \lambda\in D_{\lambda_0^*}(\delta),\\
\dot{\mathbf{W}}^a (\lambda),&\quad \lambda\in D_a(\delta),\\
\dot{\mathbf{W}}^b (\lambda),&\quad \lambda\in D_b(\delta),\\
\dot{\mathbf{W}}^\mathrm{out} (\lambda),&\quad \lambda\in \mathbb{C}\setminus(\Sigma_g \cup I \cup \overline{D_{\lambda_0}(\delta)\cup D_{\lambda_0^*}(\delta) \cup D_a(\delta) \cup D_b(\delta)}).
\end{cases}
\label{eq:W-dot-bun}
\end{equation}

\subsection{Small norm problem for the error and large-$M$ expansion}
To analyze the accuracy of the global parametrix $\dot{\mathbf{W}}(\lambda)$ for $(\chi,\tau)\in \shelves$, we define the \emph{error}
\begin{equation}
\mathbf{F}(\lambda) \defeq  \mathbf{W}(\lambda) \dot{\mathbf{W}}(\lambda)^{-1}.
\label{eq:F-bun}
\end{equation} 
As $\dot{\mathbf{W}}(\lambda)$ satisfies exactly the same jump conditions as $\mathbf{W}(\lambda)$ inside the disks $D_\lambda(\delta)$, $\lambda=a,b,\lambda_0,\lambda_0^*$, and on portions of the arcs $\Sigma_g$ and $I$ exterior to these disks, $\mathbf{F}(\lambda)$ can be taken as an analytic function of $\lambda\in\mathbb{C}$ with the exception the contour system $\Sigma_\mathbf{F}$, which consists of the portions of the arcs $C^\pm_{\Gamma, L}, C^\pm_{\Gamma, R}, C^\pm_{\Sigma, L}, C^\pm_{\Sigma, R}$ lying outside the disks $D_{a,b}(\delta)$ and $D_{\lambda_0,\lambda_0^*}(\delta)$ along with the four disk boundaries $\partial D_{a,b}(\delta)$ and $\partial D_{\lambda_0,\lambda_0^*}(\delta)$. We denote by $\mathbf{V}^{\mathbf{F}}(\lambda)$ the jump matrix for $\mathbf{F}(\lambda)$, which is supported on $\Sigma_\mathbf{F}$. On the arcs $C^\pm_{\Gamma, L}, C^\pm_{\Gamma, R}, C^\pm_{\Sigma, L}, C^\pm_{\Sigma, R}$ \emph{outside} the four disks, we can express $\mathbf{V}^{\mathbf{F}}(\lambda)$ as
\begin{equation}
\begin{split}
\mathbf{V}^{\mathbf{F}}(\lambda) &= \mathbf{F}_-(\lambda)^{-1}\mathbf{F}_+ (\lambda)\\
&=\dot{\mathbf{W}}^\mathrm{out}(\lambda)\mathbf{W}_-(\lambda)^{-1}\mathbf{W}_+ (\lambda)\dot{\mathbf{W}}^\mathrm{out}(\lambda)^{-1}.
\end{split}
\end{equation}
Since $\dot{\mathbf{W}}^\mathrm{out}(\lambda)$ remains bounded with unit determinant as $M\to+\infty$ and $\delta$ is fixed, there exists a positive constant $\nu>0$ such that $\mathbf{V}^\mathbf{F}(\lambda) - \mathbb{I} = O(\ee^{-\nu M})$ holds uniformly on the jump contour $\Sigma_\mathbf{F}$ for $\mathbf{F}(\lambda)$ except on the circles $\partial D_{a,b}(\delta)$ and $\partial D_{\lambda_0,\lambda_0^*}(\delta)$. On the circles, the jump matrix for $\mathbf{F}(\lambda)$ takes the form:
\begin{alignat}{2}
 \mathbf{F}_+(\lambda) &= \mathbf{F}_-(\lambda) \cdot \dot{\mathbf{W}}^{a,b}(\lambda)\dot{\mathbf{W}}^\mathrm{out}(\lambda)^{-1},&&\quad \lambda \in \partial D_{a,b}(\delta),\label{eq:F-jump-bun-a-b}\\
  \mathbf{F}_+(\lambda) &= \mathbf{F}_+(\lambda) \cdot \dot{\mathbf{W}}^{\lambda_0,\lambda_0^*}(\lambda)\dot{\mathbf{W}}^\mathrm{out}(\lambda)^{-1},&&\quad \lambda \in \partial D_{\lambda_0,\lambda_0^*}(\delta),
  \label{eq:F-jump-bun-lambda0}
\end{alignat}
because $\mathbf{W}(\lambda)$ is continuous across each of the four circles. Recalling the estimates \eqref{eq:error-PC-disk-b-large-M} and \eqref{eq:error-PC-disk-a-large-M}, it is seen from \eqref{eq:F-jump-bun-a-b} that $\mathbf{V}^\mathbf{F}(\lambda) - \mathbb{I} = O(M^{-\frac{1}{2}})$ holds uniformly on the circles $\partial D_{a,b}(\delta)$. Similarly, we see from \eqref{eq:error-Airy-disk-plus}-\eqref{eq:error-Airy-disk-minus} and \eqref{eq:F-jump-bun-lambda0} that $\mathbf{V}^\mathbf{F}(\lambda) - \mathbb{I} = O(M^{-1})$ holds uniformly on the circles $\partial D_{\lambda_0, \lambda_0^*}(\delta)$. Thus, it follows that $ \mathbf{F}_+(\lambda) = \mathbf{F}_-(\lambda)( \mathbb{I} + O(M^{-\frac{1}{2}}))$ holds uniformly as $M\to+\infty$ on the compact jump contour $\Sigma_\mathbf{F}$. Standard small-norm theory for such Riemann-Hilbert problems implies that $\mathbf{F}_-(\lambda) = \mathbb{I} + O(M^{-\frac{1}{2}})$ holds in the $L^2$ sense on $\Sigma_\mathbf{F}$, in the limit $M\to +\infty$.

\subsection{Asymptotic formula for $q(x,t;\mathbf{Q}^{-s},M)$ and fundamental rogue waves for $(\chi,\tau)\in \shelves$} We note that for the matrix function $\mathbf{W}(\lambda)=\mathbf{W}(\lambda;\chi,\tau,\mathbf{Q}^{-s},M)$
%\begin{equation}
%\mathbf{X}^{(k)} (\lambda;\chi,\tau) = \mathbf{W}^{(k)} (\lambda;\chi,\tau) \ee^{S(\lambda;\chi,\tau)\sigma_3} = \mathbf{T}^{(k)} (\lambda;\chi,\tau) \ee^{S(\lambda)\sigma_3} = \mathbf{S}^{(k)} (\lambda;\chi,\tau) \ee^{n g(\lambda;\chi,\tau)\sigma_3} \ee^{S(\lambda;\chi,\tau)\sigma_3}
%\end{equation}
\begin{equation}
\mathbf{W} (\lambda)  = \mathbf{T} (\lambda) = \mathbf{S} (\lambda) \ee^{M g(\lambda;\chi,\tau)\sigma_3} 
\end{equation}
holds for $|\lambda|$ sufficiently large; therefore, from \eqref{eq:q-S} we have the formula
\begin{equation}
%\psi_k(n\chi, n\tau) = 2\ii \ee^{-\ii n \tau} \lim_{\lambda\to\infty} \left( \lambda X^{(k)}_{12}(\lambda;\chi,\tau) \ee^{n g(\lambda;\chi,\tau)} \ee^{S(\lambda;\chi,\tau)}\right).
q(M\chi, M\tau; \mathbf{Q}^{-s}, M ) = 2\ii  \lim_{\lambda\to\infty} \left( \lambda W_{12}(\lambda) \ee^{\ii M g(\lambda;\chi,\tau)}\right).
\label{eq:psi-k-W}
\end{equation}
On the other hand, we see from the definitions \eqref{eq:W-dot-bun} and \eqref{eq:F-bun} that 
\begin{equation}
\mathbf{W}(\lambda) = \mathbf{F}(\lambda) \dot{\mathbf{W}}^{\mathrm{out}}(\lambda)
\end{equation}
also holds for $|\lambda|$ sufficiently large; therefore, \eqref{eq:psi-k-W} is expressed as:
\begin{equation}
q(M\chi, M\tau; \mathbf{Q}^{-s}, M ) = 2\ii  \lim_{\lambda\to\infty} \left( \lambda F_{11}(\lambda) \dot{W}^{\mathrm{out}}_{12}(\lambda) +  \lambda F_{12}(\lambda) \dot{W}^{\mathrm{out}}_{22}(\lambda) \right).
\label{eq:psi-k-F}
\end{equation}
%As $\dot{\mathbf{W}}^{\mathrm{out}}(\lambda)$ is a full matrix that tends to identity as $\lambda\to\infty$, this formula simplifies as
Now $\dot{\mathbf{W}}^\mathrm{out}(\lambda)$ tends to the identity as $\lambda\to\infty$, and from \eqref{eq:F-bun} so does $\mathbf{F}(\lambda)$.  Therefore,
\begin{equation}
q(M\chi, M\tau; \mathbf{Q}^{-s}, M ) = 2\ii  \lim_{\lambda\to\infty} \left( \lambda  \dot{W}^{\mathrm{out}}_{12}(\lambda) +   \lambda F_{12}(\lambda) \right).
\label{eq:psi-k-F-simp}
\end{equation}
Recalling that $K(\lambda;\chi,\tau) +\mu(\chi,\tau)= O(\lambda^{-1})$ as $\lambda\to\infty$, it is easily seen from the definitions \eqref{eq:H-def} and \eqref{eq:W-out-full} that
%\begin{equation}
%%\lim_{\lambda\to\infty} \lambda \dot{W}^{\mathrm{out}}_{12}(\lambda;\chi,\tau)= \frac{-s B(\chi,\tau)}{2} \ee^{-2\ii \Theta_0(\chi,\tau;M)}, 
%\lim_{\lambda\to\infty} \lambda \dot{W}^{\mathrm{out}}_{12}(\lambda)= \frac{-s B(\chi,\tau)}{2} \ee^{-2\ii (M \kappa(\chi,\tau) +  \mu(\chi,\tau) )}, 
%\label{eq:psi-k-bun-outer}
%\end{equation}
%\begin{equation}
%%\begin{split}
%%\lim_{\lambda\to\infty} \lambda \dot{W}^{\mathrm{out}}_{12}(\lambda;\chi,\tau)= \frac{-s B(\chi,\tau)}{2} \ee^{-2\ii \Theta_0(\chi,\tau;M)}, 
%\lim_{\lambda\to\infty} \lambda \dot{W}^{\mathrm{out}}_{12}(\lambda) 
%=\frac{-\ii B(\chi,\tau)}{2} \ee^{-2\ii (M \kappa(\chi,\tau) +  \mu(\chi,\tau) + \frac{1}{4} s \pi)}.
%%&= \frac{-s B(\chi,\tau)}{2} \ee^{-2\ii (M \kappa(\chi,\tau) +  \mu(\chi,\tau) )}, 
%%\end{split}
%\label{eq:psi-k-bun-outer}
%\end{equation}
\begin{equation}
%\begin{split}
%\lim_{\lambda\to\infty} \lambda \dot{W}^{\mathrm{out}}_{12}(\lambda;\chi,\tau)= \frac{-s B(\chi,\tau)}{2} \ee^{-2\ii \Theta_0(\chi,\tau;M)}, 
2\ii \lim_{\lambda\to\infty} \lambda \dot{W}^{\mathrm{out}}_{12}(\lambda) 
=B(\chi,\tau) \ee^{-2\ii (M \kappa(\chi,\tau) +  \mu(\chi,\tau) + \frac{1}{4} s \pi)} = \mathfrak{L}_s^{[\shelves]}(\chi,\tau;M),
%&= \frac{-s B(\chi,\tau)}{2} \ee^{-2\ii (M \kappa(\chi,\tau) +  \mu(\chi,\tau) )}, 
%\end{split}
\label{eq:psi-k-bun-outer}
\end{equation}
producing the leading term for $q(M\chi,M\tau; \mathbf{Q}^{-s},M)$ given in \eqref{eq:leading-term-shelves-q}.
%where we have introduced the (real-valued) angle
%\begin{equation}
%\Theta_0(\chi,\tau;M)\defeq  n \kappa(\chi,\tau) + s \gamma(\chi,\tau) + \mu(\chi,\tau).
%\label{eq:Theta-0}
%\end{equation}

It now remains to compute the contribution in \eqref{eq:psi-k-F} coming from $\lambda F_{12}(\lambda;\chi,\tau)$ as $\lambda\to\infty$. Formulating the jump condition for $\mathbf{F}(\lambda)$ in the form $\mathbf{F}_+ - \mathbf{F}_- = \mathbf{F}_- (\mathbf{V}^{\mathbf{F}}-\mathbb{I})$ and using the fact that $\mathbf{F}(\lambda)\to\mathbb{I}$ as $\lambda\to\infty$, we obtain from the Plemelj formula
the same representation as in \eqref{eq:F-Cauchy-channels} for $\mathbf{F}(\lambda)$. It then follows that
%\begin{equation}
%\mathbf{F}(\lambda) = \mathbb{I} + \frac{1}{2\pi \ii} \int_{\Sigma_\mathbf{F}} \frac{\mathbf{F}_{-}(\eta)(\mathbf{V}^{\mathbf{F}}(\eta) - \mathbb{I})}{\eta-\lambda}\dd \eta,\quad \lambda\in\mathbb{C}\setminus\Sigma_\mathbf{F}.
%\label{eq:F-Plemelj}
%\end{equation}
%Thus, 
$\mathbf{F}(\lambda)$ has the Laurent series expansion which is convergent for sufficiently large $|\lambda|$:
\begin{equation}
\mathbf{F}(\lambda) = \mathbb{I} - \frac{1}{2\pi \ii} \sum_{m=1}^{\infty} \lambda^{-m} \int_{\Sigma_\mathbf{F}} \mathbf{F}_-(\eta)(\mathbf{V}^{\mathbf{F}}(\eta)-\mathbb{I}) \eta^{m-1}\, \dd \eta,\quad |\lambda|>|\Sigma_\mathbf{F}|\defeq \sup_{\eta\in\Sigma_\mathbf{F}}|\eta|.
\label{eq:F-Laurent-bun}
\end{equation}
We obtain from this expansion the integral representation 
%(after adding and subtracting $1$ to $F_{11-}(\eta)$)
%\begin{equation}
%\lim_{\lambda\to\infty} \lambda F_{12}(\lambda) =
%-\frac{1}{2\pi \ii} \left \lbrace \int_{\Sigma_\mathbf{F}} F_{11-}(\eta)V^{\mathbf{F}}_{12}(\eta)\dd \eta
%+ \int_{\Sigma_\mathbf{F}} F_{12-}(\eta)(V^{\mathbf{F}}_{22}(\eta) -1 ) \dd \eta
% \right\rbrace,
%\end{equation}
%which we express as
\begin{multline}
\lim_{\lambda\to\infty} \lambda F_{12}(\lambda) =
-\frac{1}{2\pi \ii} \left \lbrace \int_{\Sigma_\mathbf{F}} ( F_{11-}(\eta) - 1)V^{\mathbf{F}}_{12}(\eta)\dd \eta\right. \\
\left. +\int_{\Sigma_\mathbf{F}} V^{\mathbf{F}}_{12}(\eta)\dd \eta
+ \int_{\Sigma_\mathbf{F}} F_{12-}(\eta)(V^{\mathbf{F}}_{22}(\eta) -1 ) \dd \eta
 \right\rbrace.
 \label{eq:lambda-F-12-bun}
\end{multline}
We recall that $\mathbf{F}(\lambda)-\mathbb{I} = O(M^{-\frac{1}{2}})$ in the $L^2$ sense and $\mathbf{V}^{\mathbf{F}}(\lambda)-\mathbb{I} = O(M^{-\frac{1}{2}})$ in the $L^\infty$ sense on $\Sigma^{\mathbf{F}}$, in the limit $M\to+\infty$. As the $L^1$ norm is subordinate to the $L^2$ norm on the compact contour $\Sigma^\mathbf{F}$, direct application of Cauchy-Schwarz inequality shows that the first and the last integrals in \eqref{eq:lambda-F-12-bun} are both of size $O(M^{-1})$ as $M\to+\infty$. Combining this fact with \eqref{eq:psi-k-bun-outer} in the formula \eqref{eq:psi-k-F-simp} yields
%\begin{multline}
%%\psi_k(M\chi,M\tau) =  2\ii \ee^{-\ii n \tau} \left( \frac{-s B(\chi,\tau)}{2}\ee^{-2\ii \Theta_0(\chi,\tau;M)}-\frac{1}{2\pi \ii}\int_{\Sigma_\mathbf{F}} V^{\mathbf{F}}_{12}(\eta;\chi,\tau)\dd \eta \right) + O(n^{-1}), \quad n\to+\infty.
%q(M\chi, M\tau; \mathbf{Q}^{-s},M) =  2\ii  \left( \frac{-s B(\chi,\tau)}{2}\ee^{-2\ii (M \kappa(\chi,\tau)+\mu(\chi,\tau) )}-\frac{1}{2\pi \ii}\int_{\Sigma_\mathbf{F}} V^{\mathbf{F}}_{12}(\eta)\dd \eta \right) + O(M^{-1}), \\
%\quad M\to+\infty.
%\end{multline}
%\begin{multline}
%%\psi_k(M\chi,M\tau) =  2\ii \ee^{-\ii n \tau} \left( \frac{-s B(\chi,\tau)}{2}\ee^{-2\ii \Theta_0(\chi,\tau;M)}-\frac{1}{2\pi \ii}\int_{\Sigma_\mathbf{F}} V^{\mathbf{F}}_{12}(\eta;\chi,\tau)\dd \eta \right) + O(n^{-1}), \quad n\to+\infty.
%q(M\chi, M\tau; \mathbf{Q}^{-s},M) =  2\ii  \left( \frac{-\ii B(\chi,\tau)}{2}\ee^{-2\ii (M \kappa(\chi,\tau)+\mu(\chi,\tau) +\frac{1}{4}s \pi)}-\frac{1}{2\pi \ii}\int_{\Sigma_\mathbf{F}} V^{\mathbf{F}}_{12}(\eta)\dd \eta \right) + O(M^{-1}), \\
%\quad M\to+\infty.
%\end{multline}
\begin{equation}
%\psi_k(M\chi,M\tau) =  2\ii \ee^{-\ii n \tau} \left( \frac{-s B(\chi,\tau)}{2}\ee^{-2\ii \Theta_0(\chi,\tau;M)}-\frac{1}{2\pi \ii}\int_{\Sigma_\mathbf{F}} V^{\mathbf{F}}_{12}(\eta;\chi,\tau)\dd \eta \right) + O(n^{-1}), \quad n\to+\infty.
q(M\chi, M\tau; \mathbf{Q}^{-s},M) =  \mathfrak{L}_s^{[\shelves]}(\chi,\tau;M) -\frac{1}{\pi}\int_{\Sigma_\mathbf{F}} V^{\mathbf{F}}_{12}(\eta)\dd \eta  + O(M^{-1}),
\quad M\to+\infty.
\end{equation}
Note that $V^\mathbf{F}_{12}(\lambda)$ is $O(M^{-1})$ on the circles $\partial D_{\lambda_0,\lambda_0^*}(\delta)$ as $M\to+\infty$ and it is  $O(\ee^{-\nu M})$ on the portions of the arcs $C^\pm_{\Gamma, L}, C^\pm_{\Gamma, R}, C^\pm_{\Sigma, L}, C^\pm_{\Sigma, R}$ lying outside the four disks, whereas $V^\mathbf{F}_{12}(\lambda)$ is $O(M^{-\frac{1}{2}})$ on the circles $\partial D_{a,b}(\delta)$. Therefore, the same formula as above holds with a different error of the same size when the integration contour $\Sigma_{\mathbf{F}}$ is replaced with $\partial D_{a}(\delta) \cup \partial D_{b}(\delta)$:
%\begin{multline}
%q(M\chi, M\tau; \mathbf{Q}^{-s},M)=  \ee^{-\ii M \tau} \left( -\ii s B(\chi,\tau) \ee^{-2\ii (M \kappa(\chi,\tau)+\mu(\chi,\tau) )}-\frac{1}{\pi}\int_{\partial D_{a}(\delta) \cup \partial D_{b}(\delta)} V^{\mathbf{F}}_{12}(\eta)\dd \eta \right)\\
% + O(M^{-1}), \quad M\to+\infty.
%\label{eq:psi-k-V-12-bun}
%\end{multline}
%\begin{multline}
%q(M\chi, M\tau; \mathbf{Q}^{-s},M)=B(\chi,\tau) \ee^{-2\ii (M \kappa(\chi,\tau)+\mu(\chi,\tau) +\frac{1}{4}s \pi )}-\frac{1}{\pi}\int_{\partial D_{a}(\delta) \cup \partial D_{b}(\delta)} V^{\mathbf{F}}_{12}(\eta)\dd \eta \\
% + O(M^{-1}), \quad M\to+\infty.
%\label{eq:psi-k-V-12-bun}
%\end{multline}
\begin{equation}
q(M\chi, M\tau; \mathbf{Q}^{-s},M)=\mathfrak{L}_s^{[\shelves]}(\chi,\tau;M) - \frac{1}{\pi}\int_{\partial D_{a}(\delta) \cup \partial D_{b}(\delta)} V^{\mathbf{F}}_{12}(\eta)\dd \eta 
 + O(M^{-1}), \quad M\to+\infty.
\label{eq:psi-k-V-12-bun}
\end{equation}
Using the asymptotic expansion \eqref{eq:PCU-asymp} in the formul\ae{} \eqref{eq:error-PC-b} and \eqref{eq:error-PC-a} and recalling that $\mathbf{V}^{\mathbf{F}}(\lambda) = \dot{\mathbf{W}}^{a,b}(\lambda)\dot{\mathbf{W}}^{\mathrm{out}}(\lambda)^{-1}$ for $\lambda\in \partial D_{a,b}(\delta)$, we see that
\begin{equation}
V^{\mathbf{F}}_{12}(\lambda) = \frac{1}{2\ii M^{\frac{1}{2}}} 
\left( 
\frac{\alpha Y_{11}^{a,b}(\lambda)^2 + \beta Y_{12}^{a,b}(\lambda)^2}{f_{a,b}(\lambda;\chi,\tau)}  
\right)
+ O(M^{-1}),\quad M\to+\infty, \quad \lambda\in\partial D_{a,b}(\delta).
\label{eq:VF-12-bun}
\end{equation}
The definition \eqref{eq:A-b} for $\mathbf{Y}^b(\lambda)$ together with the fact that $\mathbf{H}^b(\lambda)$ is a diagonal matrix directly gives
\begin{align}
Y^b_{11}(\lambda)^2 &= s L_{11}(\lambda)^2 \ee^{-2 \ii (K(\lambda;\chi,\tau) + \mu(\chi,\tau) )} \ee^{-2\ii M {h}_b(\chi,\tau)} M^{\ii p} (\lambda-a(\chi,\tau))^{2\ii p} \left( \frac{f_b(\lambda;\chi,\tau)}{\lambda-b(\chi,\tau)} \right)^{2\ii p},
\label{eq:A-b-11}\\
Y^b_{12}(\lambda)^2 &= s L_{12}(\lambda)^2 \ee^{2 \ii( K(\lambda;\chi,\tau) +\mu(\chi,\tau) )} \ee^{2\ii M {h}_b(\chi,\tau)} M^{-\ii p}(\lambda-a(\chi,\tau))^{-2\ii p} \left( \frac{f_b(\lambda;\chi,\tau)}{\lambda-b(\chi,\tau)} \right)^{-2\ii p},
\label{eq:A-b-12}
\end{align}
%\begin{align}
%Y^b_{11}(\lambda)^2 &= s H_{11}(\lambda)^2 \ee^{-2 \ii (K(\lambda;\chi,\tau) + \mu(\chi,\tau) )} \ee^{-2\ii M {h}_b(\chi,\tau)} M^{\ii p} (\lambda-a(\chi,\tau))^{2\ii p} \left( \frac{f_b(\lambda;\chi,\tau)}{\lambda-b(\chi,\tau)} \right)^{2\ii p},
%\label{eq:A-b-11}\\
%Y^b_{12}(\lambda)^2 &= s H_{12}(\lambda)^2 \ee^{2 \ii( K(\lambda;\chi,\tau) +\mu(\chi,\tau) )} \ee^{2\ii M {h}_b(\chi,\tau)} M^{-\ii p}(\lambda-a(\chi,\tau))^{-2\ii p} \left( \frac{f_b(\lambda;\chi,\tau)}{\lambda-b(\chi,\tau)} \right)^{-2\ii p},
%\label{eq:A-b-12}
%\end{align}
%\begin{equation}
%%A^b_{11}(\lambda;\chi,\tau)^2 = s H_{11}(\lambda;\chi,\tau)^2 \ee^{-2(K(\lambda;\chi,\tau) + S(\lambda;\chi,\tau))}\omega(\lambda)^{2s} \ee^{-2\ii n \tilde{h}_b(\chi,\tau)} n^{\ii p}\\ \cdot (\lambda-a(\chi,\tau))^{2\ii p} \left( \frac{f_b(\lambda;\chi,\tau)}{\lambda-b(\chi,\tau)} \right)^{2\ii p}
%%A^b_{11}(\lambda;\chi,\tau)^2 = s H_{11}(\lambda;\chi,\tau)^2 \ee^{-2(\ii K(\lambda;\chi,\tau) + S(\lambda;\chi,\tau))}\omega(\lambda)^{2s} \ee^{-2\ii n \tilde{h}_b(\chi,\tau)} n^{\ii p}\\ \cdot (\lambda-a(\chi,\tau))^{2\ii p} \left( \frac{f_b(\lambda;\chi,\tau)}{\lambda-b(\chi,\tau)} \right)^{2\ii p}
%%A^b_{11}(\lambda)^2 = s H_{11}(\lambda)^2 \ee^{-2 \ii K(\lambda;\chi,\tau) } \ee^{-2\ii M \tilde{h}_b(\chi,\tau)} M^{\ii p} (\lambda-a(\chi,\tau))^{2\ii p} \left( \frac{f_b(\lambda;\chi,\tau)}{\lambda-b(\chi,\tau)} \right)^{2\ii p}
%%\label{eq:A-b-11}
%Y^b_{11}(\lambda)^2 = s H_{11}(\lambda)^2 \ee^{-2 \ii (K(\lambda;\chi,\tau) + \mu(\chi,\tau) )} \ee^{-2\ii M {h}_b(\chi,\tau)} M^{\ii p} (\lambda-a(\chi,\tau))^{2\ii p} \left( \frac{f_b(\lambda;\chi,\tau)}{\lambda-b(\chi,\tau)} \right)^{2\ii p}
%\label{eq:A-b-11}
%\end{equation}
%and
%\begin{equation}
%%A^b_{12}(\lambda;\chi,\tau)^2 = s H_{12}(\lambda;\chi,\tau)^2 \ee^{2(K(\lambda;\chi,\tau) + S(\lambda;\chi,\tau))}\omega(\lambda)^{-2s} \ee^{2\ii n \tilde{h}_b(\chi,\tau)} n^{-\ii p}\\ \cdot (\lambda-a(\chi,\tau))^{-2\ii p} \left( \frac{f_b(\lambda;\chi,\tau)}{\lambda-b(\chi,\tau)} \right)^{-2\ii p},
%%A^b_{12}(\lambda)^2 = s H_{12}(\lambda)^2 \ee^{2 \ii K(\lambda;\chi,\tau) } \ee^{2\ii M \tilde{h}_b(\chi,\tau)} M^{-\ii p}(\lambda-a(\chi,\tau))^{-2\ii p} \left( \frac{f_b(\lambda;\chi,\tau)}{\lambda-b(\chi,\tau)} \right)^{-2\ii p},
%%\label{eq:A-b-12}
%Y^b_{12}(\lambda)^2 = s H_{12}(\lambda)^2 \ee^{2 \ii( K(\lambda;\chi,\tau) +\mu(\chi,\tau) )} \ee^{2\ii M {h}_b(\chi,\tau)} M^{-\ii p}(\lambda-a(\chi,\tau))^{-2\ii p} \left( \frac{f_b(\lambda;\chi,\tau)}{\lambda-b(\chi,\tau)} \right)^{-2\ii p},
%\label{eq:A-b-12}
%\end{equation}
where to arrive at the latter formula we have used $s=\pm 1$. Similarly, the definition \eqref{eq:A-a} for $\mathbf{Y}^{a}(\lambda)$, this time together with the fact that $\mathbf{H}^a(\lambda)$ is off-diagonal and with perhaps more tedious arithmetic gives
%\begin{align}
%Y^a_{11}(\lambda)^2 &= - s H_{R,12}(\lambda)^2 \ee^{2\ii (K_R(\lambda;\chi,\tau) + \mu(\chi,\tau) )} \ee^{2\ii M {h}_a(\chi,\tau)} M^{\ii p} (b(\chi,\tau)-\lambda)^{2\ii p} \left( \frac{a(\chi,\tau) - \lambda}{f_a(\lambda;\chi,\tau)} \right)^{-2\ii p}
%\label{eq:A-a-11}\\
%Y^a_{12}(\lambda)^2 &= - s H_{R,11}(\lambda)^2 \ee^{-2 \ii (K_R(\lambda;\chi,\tau) + \mu(\chi,\tau) ) } \ee^{-2\ii M {h}_a(\chi,\tau)} M^{-\ii p}(b(\chi,\tau)-\lambda)^{-2\ii p} \left( \frac{a(\chi,\tau) - \lambda}{f_a(\lambda;\chi,\tau)} \right)^{2\ii p},
%\label{eq:A-a-12}
%\end{align}
\begin{align}
Y^a_{11}(\lambda)^2 &= - s L_{-,12}(\lambda)^2 \ee^{2\ii (K_-(\lambda;\chi,\tau) + \mu(\chi,\tau) )} \ee^{2\ii M {h}_a(\chi,\tau)} M^{\ii p} (b(\chi,\tau)-\lambda)^{2\ii p} \left( \frac{a(\chi,\tau) - \lambda}{f_a(\lambda;\chi,\tau)} \right)^{-2\ii p},
\label{eq:A-a-11}\\
Y^a_{12}(\lambda)^2 &= - s L_{-,11}(\lambda)^2 \ee^{-2 \ii (K_-(\lambda;\chi,\tau) + \mu(\chi,\tau) ) } \ee^{-2\ii M {h}_a(\chi,\tau)} M^{-\ii p}(b(\chi,\tau)-\lambda)^{-2\ii p} \left( \frac{a(\chi,\tau) - \lambda}{f_a(\lambda;\chi,\tau)} \right)^{2\ii p},
\label{eq:A-a-12}
\end{align}
%\begin{equation}
%%A^a_{11}(\lambda;\chi,\tau)^2 = - s H_{R,12}(\lambda;\chi,\tau)^2 \ee^{2(K_R(\lambda;\chi,\tau) + S_R(\lambda;\chi,\tau))}\omega(\lambda)^{-2s} \ee^{2\ii n \tilde{h}_a(\chi,\tau)} n^{\ii p}\\ \cdot (b(\chi,\tau)-\lambda)^{2\ii p} \left( \frac{a(\chi,\tau) - \lambda}{f_a(\lambda;\chi,\tau)} \right)^{-2\ii p}
%%A^a_{11}(\lambda;\chi,\tau)^2 = - s H_{R,12}(\lambda;\chi,\tau)^2 \ee^{2(\ii K_R(\lambda;\chi,\tau) + S_R(\lambda;\chi,\tau))}\omega(\lambda)^{-2s} \ee^{2\ii n \tilde{h}_a(\chi,\tau)} n^{\ii p}\\ \cdot (b(\chi,\tau)-\lambda)^{2\ii p} \left( \frac{a(\chi,\tau) - \lambda}{f_a(\lambda;\chi,\tau)} \right)^{-2\ii p}
%%\label{eq:A-a-11}
%A^a_{11}(\lambda)^2 = - s H_{R,12}(\lambda)^2 \ee^{2\ii K_R(\lambda;\chi,\tau) } \ee^{2\ii M \tilde{h}_a(\chi,\tau)} M^{\ii p} (b(\chi,\tau)-\lambda)^{2\ii p} \left( \frac{a(\chi,\tau) - \lambda}{f_a(\lambda;\chi,\tau)} \right)^{-2\ii p}
%\label{eq:A-a-11}
%\end{equation}
%and
%\begin{equation}
%%A^a_{12}(\lambda;\chi,\tau)^2 = - s H_{R,11}(\lambda;\chi,\tau)^2 \ee^{-2(K_R(\lambda;\chi,\tau) + S_R\lambda;\chi,\tau))}\omega(\lambda)^{2s} \ee^{-2\ii n \tilde{h}_a(\chi,\tau)} n^{-\ii p}\\ \cdot (b(\chi,\tau)-\lambda)^{-2\ii p} \left( \frac{a(\chi,\tau) - \lambda}{f_a(\lambda;\chi,\tau)} \right)^{2\ii p},
%A^a_{12}(\lambda)^2 = - s H_{R,11}(\lambda)^2 \ee^{-2 \ii K_R(\lambda;\chi,\tau) } \ee^{-2\ii M \tilde{h}_a(\chi,\tau)} M^{-\ii p}(b(\chi,\tau)-\lambda)^{-2\ii p} \left( \frac{a(\chi,\tau) - \lambda}{f_a(\lambda;\chi,\tau)} \right)^{2\ii p},
%\label{eq:A-a-12}
%\end{equation}
and to obtain the former formula we have again used $s=\pm 1$. 
Here $\mathbf{L}_-(\lambda)$ and $K_-(\lambda;\chi,\tau)$ are the functions analytic for $\lambda\in D_a(\delta)$ coinciding with $\mathbf{L}(\lambda)=\mathbf{L}(\lambda;\chi,\tau,\mathbf{Q}^{-s},M)$ and  $K(\lambda;\chi,\tau)$, respectively, in $D_{a,-}(\delta)$. 
%Here $\mathbf{L}_R(\lambda)$ and $K_R(\lambda;\chi,\tau)$ are the functions analytic for $\lambda\in D_a(\delta)$ coinciding with $\mathbf{L}(\lambda)=\mathbf{L}(\lambda;\chi,\tau,\mathbf{Q}^{-s},M)$ and  $K(\lambda;\chi,\tau)$, respectively, to the right of $\Sigma_g$ in $D_a(\delta)$. 
Thus, with \eqref{eq:A-b-11} and \eqref{eq:A-b-12}, we see from \eqref{eq:VF-12-bun} that $V^{\mathbf{F}}_{12}(\lambda)$ on the circle $\partial D_b(\delta)$ is given by:
%\begin{equation}
%\begin{split}
%V^{\mathbf{F}}_{12}(\lambda;\chi,\tau) &=
%%\! \begin{multlined}[t]
%%\frac{s \alpha n^{\ii p} \ee^{-2\ii n \tilde{h}_b(\chi,\tau)} }{2\ii n^{\frac{1}{2}} f_b(\lambda;\chi,\tau)} H_{11}(\lambda;\chi,\tau)^2 \ee^{-2(K(\lambda;\chi,\tau) + S(\lambda;\chi,\tau))}     \omega(\lambda)^{2s} (\lambda-a(\chi,\tau))^{2\ii p}  \\
%%\cdot\left( \frac{f_b(\lambda;\chi,\tau)}{\lambda-b(\chi,\tau)} \right)^{2\ii p}
%%\end{multlined}\\
%%%********** Here \! is to get the correct spacing after the = sign **************
%%&\quad - \! \begin{multlined}[t]
%%\frac{s \alpha^* n^{-\ii p} \ee^{2\ii n \tilde{h}_b(\chi,\tau)} }{2\ii n^{\frac{1}{2}} f_b(\lambda;\chi,\tau)} H_{12}(\lambda;\chi,\tau)^2 \ee^{2(K(\lambda;\chi,\tau) + S(\lambda;\chi,\tau))}     \omega(\lambda)^{-2s} (\lambda-a(\chi,\tau))^{-2\ii p}  \\
%%\cdot\left( \frac{f_b(\lambda;\chi,\tau)}{\lambda-b(\chi,\tau)} \right)^{-2\ii p},
%%\end{multlined}
%\! \begin{multlined}[t]
%\frac{s \alpha n^{\ii p} \ee^{-2\ii n \tilde{h}_b(\chi,\tau)} }{2\ii n^{\frac{1}{2}} f_b(\lambda;\chi,\tau)} H_{11}(\lambda;\chi,\tau)^2 \ee^{-2(\ii K(\lambda;\chi,\tau) + S(\lambda;\chi,\tau))}     \omega(\lambda)^{2s} (\lambda-a(\chi,\tau))^{2\ii p}  \\
%\cdot\left( \frac{f_b(\lambda;\chi,\tau)}{\lambda-b(\chi,\tau)} \right)^{2\ii p}
%\end{multlined}\\
%%********** Here \! is to get the correct spacing after the = sign **************
%&\quad - \! \begin{multlined}[t]
%\frac{s \alpha^* n^{-\ii p} \ee^{2\ii n \tilde{h}_b(\chi,\tau)} }{2\ii n^{\frac{1}{2}} f_b(\lambda;\chi,\tau)} H_{12}(\lambda;\chi,\tau)^2 \ee^{2(\ii K(\lambda;\chi,\tau) + S(\lambda;\chi,\tau))}     \omega(\lambda)^{-2s} (\lambda-a(\chi,\tau))^{-2\ii p}  \\
%\cdot\left( \frac{f_b(\lambda;\chi,\tau)}{\lambda-b(\chi,\tau)} \right)^{-2\ii p},
%\end{multlined}
%\end{split}
%\label{eq:V-F-12-b}
%\end{equation}
%\begin{equation}
%\begin{split}
%V^{\mathbf{F}}_{12}(\lambda) &=
%\frac{s \alpha M^{\ii p} \ee^{-2\ii M \tilde{h}_b(\chi,\tau)} }{2\ii M^{\frac{1}{2}} f_b(\lambda;\chi,\tau)} H_{11}(\lambda)^2 \ee^{-2\ii K(\lambda;\chi,\tau) }    (\lambda-a(\chi,\tau))^{2\ii p}  \left( \frac{f_b(\lambda;\chi,\tau)}{\lambda-b(\chi,\tau)} \right)^{2\ii p}\\
%&\quad - 
%\frac{s \alpha^* M^{-\ii p} \ee^{2\ii M \tilde{h}_b(\chi,\tau)} }{2\ii M^{\frac{1}{2}} f_b(\lambda;\chi,\tau)} H_{12}(\lambda)^2 \ee^{2\ii K(\lambda;\chi,\tau)}  (\lambda-a(\chi,\tau))^{-2\ii p} \left( \frac{f_b(\lambda;\chi,\tau)}{\lambda-b(\chi,\tau)} \right)^{-2\ii p},
%\end{split}
%\label{eq:V-F-12-b}
%\end{equation}
%\begin{equation}
%\begin{split}
%V^{\mathbf{F}}_{12}(\lambda) &=
%\frac{s \alpha M^{\ii p} \ee^{-2\ii M {h}_b(\chi,\tau)} }{2\ii M^{\frac{1}{2}} f_b(\lambda;\chi,\tau)} H_{11}(\lambda)^2 \ee^{-2\ii ( K(\lambda;\chi,\tau) +\mu(\chi,\tau) ) }    (\lambda-a(\chi,\tau))^{2\ii p}  \left( \frac{f_b(\lambda;\chi,\tau)}{\lambda-b(\chi,\tau)} \right)^{2\ii p}\\
%&\quad - 
%\frac{s \alpha^* M^{-\ii p} \ee^{2\ii M {h}_b(\chi,\tau)} }{2\ii M^{\frac{1}{2}} f_b(\lambda;\chi,\tau)} H_{12}(\lambda)^2 \ee^{2\ii (K(\lambda;\chi,\tau) + \mu(\chi,\tau) )}  (\lambda-a(\chi,\tau))^{-2\ii p} \left( \frac{f_b(\lambda;\chi,\tau)}{\lambda-b(\chi,\tau)} \right)^{-2\ii p},
%\end{split}
%\label{eq:V-F-12-b}
%\end{equation}
\begin{equation}
\begin{split}
V^{\mathbf{F}}_{12}(\lambda) &=
\frac{s \alpha M^{\ii p} \ee^{-2\ii M {h}_b(\chi,\tau)} }{2\ii M^{\frac{1}{2}} f_b(\lambda;\chi,\tau)} L_{11}(\lambda)^2 \ee^{-2\ii ( K(\lambda;\chi,\tau) +\mu(\chi,\tau) ) }    (\lambda-a(\chi,\tau))^{2\ii p}  \left( \frac{f_b(\lambda;\chi,\tau)}{\lambda-b(\chi,\tau)} \right)^{2\ii p}\\
&\quad - 
\frac{s \alpha^* M^{-\ii p} \ee^{2\ii M {h}_b(\chi,\tau)} }{2\ii M^{\frac{1}{2}} f_b(\lambda;\chi,\tau)} L_{12}(\lambda)^2 \ee^{2\ii (K(\lambda;\chi,\tau) + \mu(\chi,\tau) )}  (\lambda-a(\chi,\tau))^{-2\ii p} \left( \frac{f_b(\lambda;\chi,\tau)}{\lambda-b(\chi,\tau)} \right)^{-2\ii p} \\
&\quad+ O(M^{-1}),\quad \text{in $L^\infty(\partial D_b(\delta))$ as $M\to+\infty$},
\end{split}
\label{eq:V-F-12-b}
\end{equation}
where we have used the property $\beta=-\alpha^*$. Similarly, we see from \eqref{eq:A-a-11} and \eqref{eq:A-a-12} that $V^{\mathbf{F}}_{12}(\lambda)$ on the circle $\partial D_a(\delta)$ is given by:
%\begin{equation}
%\begin{split}
%V^{\mathbf{F}}_{12}(\lambda;\chi,\tau) &=
%%\! \begin{multlined}[t]
%%\frac{-s \alpha n^{\ii p} \ee^{2\ii n \tilde{h}_a(\chi,\tau)} }{2\ii n^{\frac{1}{2}} f_a(\lambda;\chi,\tau)} H_{R,12}(\lambda;\chi,\tau)^2 \ee^{2(K_R(\lambda;\chi,\tau) + S_R(\lambda;\chi,\tau))}     \omega(\lambda)^{-2s} (b(\chi,\tau)-\lambda)^{2\ii p}  \\
%%\cdot\left( \frac{a(\chi,\tau)-\lambda}{f_a(\lambda;\chi,\tau)} \right)^{-2\ii p}
%%\end{multlined}\\
%%%********** Here \! is to get the correct spacing after the = sign **************
%%&\quad + \! \begin{multlined}[t]
%%\frac{s \alpha^* n^{-\ii p} \ee^{-2\ii n \tilde{h}_a(\chi,\tau)} }{2\ii n^{\frac{1}{2}} f_a(\lambda;\chi,\tau)} H_{R,11}(\lambda;\chi,\tau)^2 \ee^{-2(K_R(\lambda;\chi,\tau) + S_R(\lambda;\chi,\tau))}     \omega(\lambda)^{2s} (b(\chi,\tau)-\lambda)^{-2\ii p}  \\
%%\cdot\left( \frac{a(\chi,\tau)-\lambda}{f_a(\lambda;\chi,\tau)} \right)^{2\ii p}.
%%\end{multlined}
%\! \begin{multlined}[t]
%\frac{-s \alpha n^{\ii p} \ee^{2\ii n \tilde{h}_a(\chi,\tau)} }{2\ii n^{\frac{1}{2}} f_a(\lambda;\chi,\tau)} H_{R,12}(\lambda;\chi,\tau)^2 \ee^{2(\ii K_R(\lambda;\chi,\tau) + S_R(\lambda;\chi,\tau))}     \omega(\lambda)^{-2s} (b(\chi,\tau)-\lambda)^{2\ii p}  \\
%\cdot\left( \frac{a(\chi,\tau)-\lambda}{f_a(\lambda;\chi,\tau)} \right)^{-2\ii p}
%\end{multlined}\\
%%********** Here \! is to get the correct spacing after the = sign **************
%&\quad + \! \begin{multlined}[t]
%\frac{s \alpha^* n^{-\ii p} \ee^{-2\ii n \tilde{h}_a(\chi,\tau)} }{2\ii n^{\frac{1}{2}} f_a(\lambda;\chi,\tau)} H_{R,11}(\lambda;\chi,\tau)^2 \ee^{-2(\ii K_R(\lambda;\chi,\tau) + S_R(\lambda;\chi,\tau))}     \omega(\lambda)^{2s} (b(\chi,\tau)-\lambda)^{-2\ii p}  \\
%\cdot\left( \frac{a(\chi,\tau)-\lambda}{f_a(\lambda;\chi,\tau)} \right)^{2\ii p}.
%\end{multlined}
%\end{split}
%\label{eq:V-F-12-a}
%\end{equation}
%\begin{equation}
%\begin{split}
%V^{\mathbf{F}}_{12}(\lambda) &=
%\frac{-s \alpha M^{\ii p} \ee^{2\ii M \tilde{h}_a(\chi,\tau)} }{2\ii M^{\frac{1}{2}} f_a(\lambda;\chi,\tau)} H_{R,12}(\lambda)^2 \ee^{2\ii K_R(\lambda;\chi,\tau) } (b(\chi,\tau)-\lambda)^{2\ii p} 
%\left( \frac{a(\chi,\tau)-\lambda}{f_a(\lambda;\chi,\tau)} \right)^{-2\ii p}\\
%&\quad + 
%\frac{s \alpha^* M^{-\ii p} \ee^{-2\ii M \tilde{h}_a(\chi,\tau)} }{2\ii M^{\frac{1}{2}} f_a(\lambda;\chi,\tau)} H_{R,11}(\lambda)^2 \ee^{-2\ii K_R(\lambda;\chi,\tau)}  (b(\chi,\tau)-\lambda)^{-2\ii p} 
%\left( \frac{a(\chi,\tau)-\lambda}{f_a(\lambda;\chi,\tau)} \right)^{2\ii p}.
%\end{split}
%\label{eq:V-F-12-a}
%\end{equation}
%\begin{equation}
%\begin{split}
%V^{\mathbf{F}}_{12}(\lambda) &=
%\frac{-s \alpha M^{\ii p} \ee^{2\ii M {h}_a(\chi,\tau)} }{2\ii M^{\frac{1}{2}} f_a(\lambda;\chi,\tau)} H_{R,12}(\lambda)^2 \ee^{2\ii (K_R(\lambda;\chi,\tau) +\mu(\chi,\tau) )} (b(\chi,\tau)-\lambda)^{2\ii p} 
%\left( \frac{a(\chi,\tau)-\lambda}{f_a(\lambda;\chi,\tau)} \right)^{-2\ii p}\\
%&\quad + 
%\frac{s \alpha^* M^{-\ii p} \ee^{-2\ii M {h}_a(\chi,\tau)} }{2\ii M^{\frac{1}{2}} f_a(\lambda;\chi,\tau)} H_{R,11}(\lambda)^2 \ee^{-2\ii (K_R(\lambda;\chi,\tau)+\mu(\chi,\tau) )}  (b(\chi,\tau)-\lambda)^{-2\ii p} 
%\left( \frac{a(\chi,\tau)-\lambda}{f_a(\lambda;\chi,\tau)} \right)^{2\ii p}.
%\end{split}
%\label{eq:V-F-12-a}
%\end{equation}
\begin{equation}
\begin{split}
V^{\mathbf{F}}_{12}(\lambda) &=
\frac{-s \alpha M^{\ii p} \ee^{2\ii M {h}_a(\chi,\tau)} }{2\ii M^{\frac{1}{2}} f_a(\lambda;\chi,\tau)} L_{-,12}(\lambda)^2 \ee^{2\ii (K_-(\lambda;\chi,\tau) +\mu(\chi,\tau) )} (b(\chi,\tau)-\lambda)^{2\ii p} 
\left( \frac{a(\chi,\tau)-\lambda}{f_a(\lambda;\chi,\tau)} \right)^{-2\ii p}\\
&\quad + 
\frac{s \alpha^* M^{-\ii p} \ee^{-2\ii M {h}_a(\chi,\tau)} }{2\ii M^{\frac{1}{2}} f_a(\lambda;\chi,\tau)} L_{-,11}(\lambda)^2 \ee^{-2\ii (K_-(\lambda;\chi,\tau)+\mu(\chi,\tau) )}  (b(\chi,\tau)-\lambda)^{-2\ii p} 
\left( \frac{a(\chi,\tau)-\lambda}{f_a(\lambda;\chi,\tau)} \right)^{2\ii p}\\
&\quad+ O(M^{-1})\quad \text{in $L^\infty(\partial D_a(\delta))$ as $M\to+\infty$}.
\end{split}
\label{eq:V-F-12-a}
\end{equation}
Note that $f_b(\lambda;\chi,\tau)$ in the leftmost factor of \eqref{eq:V-F-12-b} has a simple zero at $\lambda=b(\chi,\tau)$ and $f_a(\lambda;\chi,\tau)$ in the leftmost factor of \eqref{eq:V-F-12-a} has a simple zero at $\lambda=a(\chi,\tau)$, while the rest of the factors in \eqref{eq:V-F-12-b} and \eqref{eq:V-F-12-a} are holomorphic within the relevant disks.
Recalling the clockwise orientation of the circles $\partial D_{a,b}(\delta)$ and using 
\begin{align}
f_a'(a(\chi,\tau);\chi,\tau)& = - \left(-h_a''(\chi,\tau)\right)^{\frac{1}{2}}  \defeq  -\sqrt{-h''_-(a(\chi,\tau);\chi,\tau)}<0,\\
f_b'(b(\chi,\tau);\chi,\tau)& = h_b''(\chi,\tau)^{\frac{1}{2}}  \defeq  \sqrt{h''(b(\chi,\tau);\chi,\tau)}>0,
\end{align}
a simple residue calculation in \eqref{eq:V-F-12-b}--\eqref{eq:V-F-12-a} yields
\begin{multline}
-\frac{1}{\pi} \int_{D_{a}(\delta)} V^{\mathbf{F}}_{12}(\eta)\dd \eta = \frac{s}{M^{\frac{1}{2}} \left(-h_a''(\chi,\tau)\right)^{\frac{1}{2}}}
\left[ 
\alpha M^{\ii p} \ee^{2\ii M {h}_a(\chi,\tau)} L_{a,12}(\chi,\tau)^2 \ee^{2\ii (K_a(\chi,\tau) + \mu(\chi,\tau) )} X_a(\chi,\tau)^{\ii p}
\right.\\
\left. -\alpha^* M^{-\ii p} \ee^{-2\ii M {h}_a(\chi,\tau)}  
L_{a,11}(\chi,\tau)^2 \ee^{-2 \ii (K_a(\chi,\tau) + \mu(\chi,\tau) )}  X_a(\chi,\tau)^{-\ii p}
\right] + O(M^{-1})\quad\text{and}
\label{eq:V-F-12-a-residue}
\end{multline}
\begin{multline}
-\frac{1}{\pi} \int_{D_{b}(\delta)} V^{\mathbf{F}}_{12}(\eta)\dd \eta =
\frac{s}{M^{\frac{1}{2}}  h_b''(\chi,\tau)^{\frac{1}{2}}}
 \left[ 
\alpha M^{\ii p} \ee^{-2\ii M {h}_b(\chi,\tau)} 
L_{b,11}(\chi,\tau)^2 \ee^{-2 \ii (K_b(\chi,\tau) + \mu(\chi,\tau) )}  X_b(\chi,\tau)^{\ii p}
%(b(\chi,\tau)-a(\chi,\tau))^{2\ii p} h_b''(\chi,\tau)^{\ii p}
\right.\\
\left. - \alpha^{*} M^{-\ii p} \ee^{2\ii M {h}_b(\chi,\tau)}
L_{b,12}(\chi,\tau)^2 \ee^{2\ii (K_b(\chi,\tau) + \mu(\chi,\tau) )}  X_b(\chi,\tau)^{-\ii p}
%(b(\chi,\tau)-a(\chi,\tau))^{-2\ii p} h_b''(\chi,\tau)^{-\ii p}
\right] + O(M^{-1}),
\label{eq:V-F-12-b-residue}
\end{multline}
%\begin{equation}
%\begin{split}
%-\frac{1}{\pi} \int_{D_{a}(\delta)} V^{\mathbf{F}}_{12}(\eta)\dd \eta &=
%%\! \begin{multlined}[t]
%%\frac{-s \alpha n^{\ii p} \ee^{2\ii n \tilde{h}_a(\chi,\tau)} }{n^{\frac{1}{2}} f'_a(a(\chi,\tau);\chi,\tau)} H_{a,12}(\chi,\tau)^2 \ee^{2(K_a(\chi,\tau) + S_a(\chi,\tau))}   \\ \cdot  \omega_a(\chi,\tau)^{-2s} (b(\chi,\tau)-a(\chi,\tau))^{2\ii p} 
%%\left( \frac{-1}{f'_a(a(\chi,\tau);\chi,\tau)} \right)^{-2\ii p}
%%\end{multlined}\\
%%&\quad + \! \begin{multlined}[t]
%%\frac{s \alpha^* n^{-\ii p} \ee^{-2\ii n \tilde{h}_a(\chi,\tau)} }{n^{\frac{1}{2}} f'_a(a(\chi,\tau);\chi,\tau)} H_{a,11}(\lambda;\chi,\tau)^2 \ee^{-2(K_a(\chi,\tau) + S_a(\chi,\tau))}  \\ \cdot  \omega_a(\chi,\tau)^{2s}  (b(\chi,\tau)-a(\chi,\tau))^{-2\ii p}  \left( \frac{-1}{f'_a(a(\chi,\tau);\chi,\tau)} \right)^{2\ii p},
%%\end{multlined}
%\! \begin{multlined}[t]
%\frac{s}{M^{\frac{1}{2}} }\cdot \frac{\alpha M^{\ii p} \ee^{2\ii M {h}_a(\chi,\tau)} }{ \left(-h_a''(\chi,\tau)\right)^{\frac{1}{2}} } L_{a,12}(\chi,\tau)^2 \ee^{2\ii (K_a(\chi,\tau) + \mu(\chi,\tau) )}  (b(\chi,\tau)-a(\chi,\tau))^{2\ii p} \\ \cdot
% \left(-h_a''(\chi,\tau)\right)^{\ii p}
%\end{multlined}\\
%&\quad + \! \begin{multlined}[t]
%\frac{s}{M^{\frac{1}{2}} } \cdot \frac{-\alpha^* M^{-\ii p} \ee^{-2\ii M {h}_a(\chi,\tau)} }{ \left(-h_a''(\chi,\tau)\right)^{\frac{1}{2}} } L_{a,11}(\chi,\tau)^2 \ee^{-2 \ii (K_a(\chi,\tau) + \mu(\chi,\tau) )}   (b(\chi,\tau)-a(\chi,\tau))^{-2\ii p}   \\ \cdot  
% \left(-h_a''(\chi,\tau)\right)^{-\ii p}
%\end{multlined}
%\end{split}
%\label{eq:V-F-12-a-residue}
%\end{equation}
%and
%
%
%\begin{equation}
%\begin{split}
%-\frac{1}{\pi} \int_{D_{b}(\delta)} V^{\mathbf{F}}_{12}(\eta)\dd \eta &=
%%\! \begin{multlined}[t]
%%\frac{s \alpha n^{\ii p} \ee^{-2\ii n \tilde{h}_b(\chi,\tau)} }{n^{\frac{1}{2}} f'_b(b(\chi,\tau);\chi,\tau)} H_{b,11}(\chi,\tau)^2 \ee^{-2(K_b(\chi,\tau) + S_b(\chi,\tau))}   \\ \cdot  \omega_b(\chi,\tau)^{2s} (b(\chi,\tau)-a(\chi,\tau))^{2\ii p} 
%%\left( f'_b(b(\chi,\tau);\chi,\tau) \right)^{2\ii p}
%%\end{multlined}\\
%%&\quad - \! \begin{multlined}[t]
%%\frac{s \alpha^{*} n^{-\ii p} \ee^{2\ii n \tilde{h}_b(\chi,\tau)} }{n^{\frac{1}{2}} f'_b(b(\chi,\tau);\chi,\tau)} H_{b,12}(\chi,\tau)^2 \ee^{2(K_b(\chi,\tau) + S_b(\chi,\tau))}   \\ \cdot  \omega_b(\chi,\tau)^{-2s} (b(\chi,\tau)-a(\chi,\tau))^{-2\ii p} 
%%\left( f'_b(b(\chi,\tau);\chi,\tau) \right)^{-2\ii p},
%%\end{multlined}
%\! \begin{multlined}[t]
%\frac{s}{M^{\frac{1}{2}}}\cdot \frac{\alpha M^{\ii p} \ee^{-2\ii M {h}_b(\chi,\tau)} }{ h_b''(\chi,\tau)^{\frac{1}{2}}} 
%L_{b,11}(\chi,\tau)^2 \ee^{-2 \ii (K_b(\chi,\tau) + \mu(\chi,\tau) )}  (b(\chi,\tau)-a(\chi,\tau))^{2\ii p} \\ \cdot 
%h_b''(\chi,\tau)^{\ii p}
%\end{multlined}\\
%&\quad - \! \begin{multlined}[t]
%\frac{s}{M^{-\frac{1}{2}}} \cdot \frac{\alpha^{*} M^{-\ii p} \ee^{2\ii M {h}_b(\chi,\tau)} }{ h_b''(\chi,\tau)^{\frac{1}{2}}} 
%L_{b,12}(\chi,\tau)^2 \ee^{2\ii (K_b(\chi,\tau) + \mu(\chi,\tau) )}  (b(\chi,\tau)-a(\chi,\tau))^{-2\ii p} \\ \cdot  
%h_b''(\chi,\tau)^{-\ii p},
%\end{multlined}
%\end{split}
%\label{eq:V-F-12-b-residue}
%\end{equation}
where we have set
\begin{equation}
X_a(\chi,\tau) \defeq - (b(\chi,\tau)-a(\chi,\tau))^2 h_a''(\chi,\tau),\quad\text{and}\quad X_b(\chi,\tau) \defeq  (b(\chi,\tau)-a(\chi,\tau))^2 h_b''(\chi,\tau),
\end{equation}
\begin{equation}
%K_a(\chi,\tau) \defeq  K_-(a(\chi,\tau);\chi,\tau)  \quad&\text{and}\quad K_b(\chi,\tau) \defeq  K(b(\chi,\tau);\chi,\tau),\\
\mathbf{L}_a(\chi,\tau) \defeq  \mathbf{L}_-(a(\chi,\tau);\chi,\tau,\mathbf{Q}^{-s},M), \quad \text{and}\quad \mathbf{L}_b(\chi,\tau) \defeq  \mathbf{L}(b(\chi,\tau);\chi,\tau,\mathbf{Q}^{-s},M),
\end{equation}
and used the notation in \eqref{eq:intro-Ka} and \eqref{eq:intro-Kb}. Recalling the definitions of the four positive modulation factors in \eqref{eq:m-a-b-shelves}, we obtain from \eqref{eq:H-def} the (well-defined) expressions
\begin{equation}
%\begin{split}
%H_{b,11}(\chi,\tau)^2 = \frac{1}{2} + \frac{1}{4} \big( y(b(\chi,\tau);\chi,\tau)^{-2} + y(b(\chi,\tau);\chi,\tau)^{2} \big)
%=\frac{1}{2}\left[ 1+ \cos\left(\arg(b(\chi,\tau) - \lambda_0(\chi,\tau) ) \right)\right],
L_{b,11}(\chi,\tau)^2 = \frac{1}{2} + \frac{1}{4} \big( y(b(\chi,\tau);\chi,\tau)^{-2} + y(b(\chi,\tau);\chi,\tau)^{2} \big)
=m^+_b(\chi,\tau),
%\frac{1}{2}\left[\cos\left(\arg(b(\chi,\tau) - \lambda_0(\chi,\tau) ) \right) + 1 \right],
%\end{split}
\label{eq:H-b-11}
\end{equation}
and similarly
\begin{align}
L_{b,12}(\chi,\tau)^2 
&= - m^+_b(\chi,\tau) \ee^{-4\ii (M\kappa(\chi,\tau)+ \mu(\chi,\tau) + \frac{1}{4}s \pi )},
\label{eq:H-b-12}\\
L_{a,11}(\chi,\tau)^2 &=
%\frac{1}{2}\left[  \cos\left(\arg(a(\chi,\tau) - \lambda_0(\chi,\tau) ) \right) + 1\right],
m^+_a(\chi,\tau),\quad\text{and}
\label{eq:H-a-11}\\
L_{a,12}(\chi,\tau)^2 &= - m^-_a(\chi,\tau) \ee^{-4\ii(M\kappa(\chi,\tau)+ \mu(\chi,\tau) + \frac{1}{4}s \pi )}.\label{eq:H-a-12}
%\frac{1}{2}\left[  \cos\left(\arg(a(\chi,\tau) - \lambda_0(\chi,\tau) ) \right) - 1\right]\ee^{-4\ii(M\kappa(\chi,\tau)+ \mu(\chi,\tau) + \frac{1}{4}s \pi )}.\label{eq:H-a-12}
\end{align}
Recalling that $p=\tfrac{\ln(2)}{2\pi}$ and $a(\chi,\tau)<b(\chi,\tau)$ for $(\chi,\tau) \in \shelves$ together with the signs \eqref{eq:fb-prime-h-double-prime} and \eqref{eq:fa-prime-h-double-prime}, we write
%\begin{align}
%M^{\pm \ii p} &= \ee^{\pm \ii \ln(M) \frac{\ln(2)}{2\pi}}\\
%( b(\chi,\tau) - a(\chi,\tau) )^{\pm 2 \ii p} &= \ee^{\pm \ii p \ln\left( (b(\chi,\tau) - a(\chi,\tau) )^2\right)}.
%\label{eq:log-b-a}
%\end{align}
\begin{align}
M^{\pm \ii p} &= \ee^{\pm \ii \ln(M) \frac{\ln(2)}{2\pi}},\\
X_a(\chi,\tau)^{\pm \ii p} &= \ee^{\pm \ii \frac{\ln(2)}{2\pi} \ln\left( - (b(\chi,\tau) - a(\chi,\tau) )^2h''_a(\chi,\tau) \right)},\quad\text{and}
\label{eq:log-b-a-1}\\
X_b(\chi,\tau)^{\pm \ii p} &= \ee^{\pm \ii \frac{\ln(2)}{2\pi} \ln\left( (b(\chi,\tau) - a(\chi,\tau) )^2h''_b(\chi,\tau) \right)}.
\label{eq:log-b-a-2}
\end{align}
Now substituting \eqref{eq:H-b-11}--\eqref{eq:log-b-a-2} along with \eqref{eq:Channels-alpha-beta} for $\alpha$ in \eqref{eq:V-F-12-a-residue} and \eqref{eq:V-F-12-b-residue} yields
\begin{equation}
- \frac{1}{\pi}\int_{\partial D_{a}(\delta) \cup \partial D_{b}(\delta)} V^{\mathbf{F}}_{12}(\eta)\dd \eta  = \mathfrak{S}^{[\shelves]}_s(\chi,\tau;M) + O(M^{-1}), \quad M\to+\infty,
\label{eq:subleading-term-shelves-q-proof}
\end{equation}
in which $\mathfrak{S}_s^{[\shelves]}(\chi,\tau;M)$ is the sub-leading term defined in \eqref{eq:subleading-term-shelves-q}. Substituting \eqref{eq:subleading-term-shelves-q-proof} back in \eqref{eq:psi-k-V-12-bun} 
%gives $q(M\chi,M\tau;\mathbf{Q}^{-s};M)=\mathfrak{L}_s^{[\shelves]}(\chi,\tau;M)+\mathfrak{S}_s^{[\shelves]}(\chi,\tau;M) + O(M^{-1})$, as $M\to+\infty$. This 
finishes the proof of Theorem~\ref{thm:shelves}. 
\begin{remark} In practice, one needs to evaluate the derivatives $h''_a(\chi,\tau)=h''_-(a(\chi,\tau);\chi,\tau)$ and $h''_b(\chi,\tau)=h''(b(\chi,\tau);\chi,\tau)$ to 
use the approximation given in Theorem~\ref{thm:shelves}. These can be computed in a straightforward manner from \eqref{eq:hprime-formula} and using \eqref{eq:R-a-b}:
%\begin{align}
%\tilde{h}''(b(\chi,\tau);\chi,\tau) &= \frac{2\tau(b(\chi,\tau) - a(\chi,\tau))}{b(\chi,\tau)^2+1}R(b(\chi,\tau);\chi,\tau)>0 \label{eq:h-double-prime-b}, \\
%\tilde{h}_{-}''(a(\chi,\tau);\chi,\tau) &= \frac{2\tau(a(\chi,\tau) - b(\chi,\tau))}{a(\chi,\tau)^2+1}R_{-}(a(\chi,\tau);\chi,\tau)<0 
%\label{eq:h-double-prime-a},
%\end{align}
\begin{align}
{h}''(b(\chi,\tau);\chi,\tau) &= \frac{2\tau(b(\chi,\tau) - a(\chi,\tau))}{b(\chi,\tau)^2+1}|b(\chi,\tau) - \lambda_0(\chi,\tau)|, \label{eq:h-double-prime-b} \\
{h}_{-}''(a(\chi,\tau);\chi,\tau) &= \frac{- 2\tau(b(\chi,\tau) - a(\chi,\tau))}{a(\chi,\tau)^2+1} | a(\chi,\tau) - \lambda_0(\chi,\tau)|.
\label{eq:h-double-prime-a}
\end{align}
\label{rem:h-double-prime}
\end{remark}

%\begin{equation}
%q(M\chi, M\tau; \mathbf{Q}^{-s},M)=\mathfrak{L}_s^{[\shelves]}(\chi,\tau;M) - \frac{1}{\pi}\int_{\partial D_{a}(\delta) \cup \partial D_{b}(\delta)} V^{\mathbf{F}}_{12}(\eta)\dd \eta 
% + O(M^{-1}), \quad M\to+\infty.
%\label{eq:psi-k-V-12-bun}
%\end{equation}
%\textcolor{red}{DONE UNTIL HERE.}
%%\begin{equation}
%%\begin{split}
%%-\frac{1}{\pi} \int_{D_{a}(\delta)} V^{\mathbf{F}}_{12}(\eta)\dd \eta &=
%%%\! \begin{multlined}[t]
%%%\frac{-s \alpha n^{\ii p} \ee^{2\ii n \tilde{h}_a(\chi,\tau)} }{n^{\frac{1}{2}} f'_a(a(\chi,\tau);\chi,\tau)} H_{a,12}(\chi,\tau)^2 \ee^{2(K_a(\chi,\tau) + S_a(\chi,\tau))}   \\ \cdot  \omega_a(\chi,\tau)^{-2s} (b(\chi,\tau)-a(\chi,\tau))^{2\ii p} 
%%%\left( \frac{-1}{f'_a(a(\chi,\tau);\chi,\tau)} \right)^{-2\ii p}
%%%\end{multlined}\\
%%%&\quad + \! \begin{multlined}[t]
%%%\frac{s \alpha^* n^{-\ii p} \ee^{-2\ii n \tilde{h}_a(\chi,\tau)} }{n^{\frac{1}{2}} f'_a(a(\chi,\tau);\chi,\tau)} H_{a,11}(\lambda;\chi,\tau)^2 \ee^{-2(K_a(\chi,\tau) + S_a(\chi,\tau))}  \\ \cdot  \omega_a(\chi,\tau)^{2s}  (b(\chi,\tau)-a(\chi,\tau))^{-2\ii p}  \left( \frac{-1}{f'_a(a(\chi,\tau);\chi,\tau)} \right)^{2\ii p},
%%%\end{multlined}
%%\! \begin{multlined}[t]
%%\frac{-s \alpha M^{\ii p} \ee^{2\ii M {h}_a(\chi,\tau)} }{M^{\frac{1}{2}} f'_a(a(\chi,\tau);\chi,\tau)} L_{a,12}(\chi,\tau)^2 \ee^{2\ii (K_a(\chi,\tau) + \mu(\chi,\tau) )}  (b(\chi,\tau)-a(\chi,\tau))^{2\ii p} \\ \cdot
%%\left( \frac{-1}{f'_a(a(\chi,\tau);\chi,\tau)} \right)^{-2\ii p}
%%\end{multlined}\\
%%&\quad + \! \begin{multlined}[t]
%%\frac{s \alpha^* M^{-\ii p} \ee^{-2\ii M {h}_a(\chi,\tau)} }{M^{\frac{1}{2}} f'_a(a(\chi,\tau);\chi,\tau)} L_{a,11}(\chi,\tau)^2 \ee^{-2 \ii (K_a(\chi,\tau) + \mu(\chi,\tau) )}   (b(\chi,\tau)-a(\chi,\tau))^{-2\ii p}   \\ \cdot  \left( \frac{-1}{f'_a(a(\chi,\tau);\chi,\tau)} \right)^{2\ii p},
%%\end{multlined}
%%\end{split}
%%\label{eq:V-F-12-a-residue}
%%\end{equation}
%\begin{equation}
%\begin{split}
%-\frac{1}{\pi} \int_{D_{a}(\delta)} V^{\mathbf{F}}_{12}(\eta)\dd \eta &=
%%\! \begin{multlined}[t]
%%\frac{-s \alpha n^{\ii p} \ee^{2\ii n \tilde{h}_a(\chi,\tau)} }{n^{\frac{1}{2}} f'_a(a(\chi,\tau);\chi,\tau)} H_{a,12}(\chi,\tau)^2 \ee^{2(K_a(\chi,\tau) + S_a(\chi,\tau))}   \\ \cdot  \omega_a(\chi,\tau)^{-2s} (b(\chi,\tau)-a(\chi,\tau))^{2\ii p} 
%%\left( \frac{-1}{f'_a(a(\chi,\tau);\chi,\tau)} \right)^{-2\ii p}
%%\end{multlined}\\
%%&\quad + \! \begin{multlined}[t]
%%\frac{s \alpha^* n^{-\ii p} \ee^{-2\ii n \tilde{h}_a(\chi,\tau)} }{n^{\frac{1}{2}} f'_a(a(\chi,\tau);\chi,\tau)} H_{a,11}(\lambda;\chi,\tau)^2 \ee^{-2(K_a(\chi,\tau) + S_a(\chi,\tau))}  \\ \cdot  \omega_a(\chi,\tau)^{2s}  (b(\chi,\tau)-a(\chi,\tau))^{-2\ii p}  \left( \frac{-1}{f'_a(a(\chi,\tau);\chi,\tau)} \right)^{2\ii p},
%%\end{multlined}
%\! \begin{multlined}[t]
%\frac{-s \alpha M^{\ii p} \ee^{2\ii M {h}_a(\chi,\tau)} }{M^{\frac{1}{2}} f'_a(a(\chi,\tau);\chi,\tau)} L_{a,12}(\chi,\tau)^2 \ee^{2\ii (K_a(\chi,\tau) + \mu(\chi,\tau) )}  (b(\chi,\tau)-a(\chi,\tau))^{2\ii p} \\ \cdot
%\left( \frac{-1}{f'_a(a(\chi,\tau);\chi,\tau)} \right)^{-2\ii p}
%\end{multlined}\\
%&\quad + \! \begin{multlined}[t]
%\frac{s \alpha^* M^{-\ii p} \ee^{-2\ii M {h}_a(\chi,\tau)} }{M^{\frac{1}{2}} f'_a(a(\chi,\tau);\chi,\tau)} L_{a,11}(\chi,\tau)^2 \ee^{-2 \ii (K_a(\chi,\tau) + \mu(\chi,\tau) )}   (b(\chi,\tau)-a(\chi,\tau))^{-2\ii p}   \\ \cdot  \left( \frac{-1}{f'_a(a(\chi,\tau);\chi,\tau)} \right)^{2\ii p},
%\end{multlined}
%\end{split}
%\label{eq:V-F-12-a-residue}
%\end{equation}
%%\begin{equation}
%%\begin{split}
%%-\frac{1}{\pi} \int_{D_{a}(\delta)} V^{\mathbf{F}}_{12}(\eta)\dd \eta &=
%%%\! \begin{multlined}[t]
%%%\frac{-s \alpha n^{\ii p} \ee^{2\ii n \tilde{h}_a(\chi,\tau)} }{n^{\frac{1}{2}} f'_a(a(\chi,\tau);\chi,\tau)} H_{a,12}(\chi,\tau)^2 \ee^{2(K_a(\chi,\tau) + S_a(\chi,\tau))}   \\ \cdot  \omega_a(\chi,\tau)^{-2s} (b(\chi,\tau)-a(\chi,\tau))^{2\ii p} 
%%%\left( \frac{-1}{f'_a(a(\chi,\tau);\chi,\tau)} \right)^{-2\ii p}
%%%\end{multlined}\\
%%%&\quad + \! \begin{multlined}[t]
%%%\frac{s \alpha^* n^{-\ii p} \ee^{-2\ii n \tilde{h}_a(\chi,\tau)} }{n^{\frac{1}{2}} f'_a(a(\chi,\tau);\chi,\tau)} H_{a,11}(\lambda;\chi,\tau)^2 \ee^{-2(K_a(\chi,\tau) + S_a(\chi,\tau))}  \\ \cdot  \omega_a(\chi,\tau)^{2s}  (b(\chi,\tau)-a(\chi,\tau))^{-2\ii p}  \left( \frac{-1}{f'_a(a(\chi,\tau);\chi,\tau)} \right)^{2\ii p},
%%%\end{multlined}
%%\! \begin{multlined}[t]
%%\frac{-s \alpha M^{\ii p} \ee^{2\ii M {h}_a(\chi,\tau)} }{M^{\frac{1}{2}} f'_a(a(\chi,\tau);\chi,\tau)} H_{a,12}(\chi,\tau)^2 \ee^{2\ii (K_a(\chi,\tau) + \mu(\chi,\tau) )}  (b(\chi,\tau)-a(\chi,\tau))^{2\ii p} \\ \cdot
%%\left( \frac{-1}{f'_a(a(\chi,\tau);\chi,\tau)} \right)^{-2\ii p}
%%\end{multlined}\\
%%&\quad + \! \begin{multlined}[t]
%%\frac{s \alpha^* M^{-\ii p} \ee^{-2\ii M {h}_a(\chi,\tau)} }{M^{\frac{1}{2}} f'_a(a(\chi,\tau);\chi,\tau)} H_{a,11}(\lambda)^2 \ee^{-2 \ii (K_a(\chi,\tau) + \mu(\chi,\tau) )}   (b(\chi,\tau)-a(\chi,\tau))^{-2\ii p}   \\ \cdot  \left( \frac{-1}{f'_a(a(\chi,\tau);\chi,\tau)} \right)^{2\ii p},
%%\end{multlined}
%%\end{split}
%%\label{eq:V-F-12-a-residue}
%%\end{equation}
%and
%\begin{equation}
%\begin{split}
%-\frac{1}{\pi} \int_{D_{b}(\delta)} V^{\mathbf{F}}_{12}(\eta)\dd \eta &=
%%\! \begin{multlined}[t]
%%\frac{s \alpha n^{\ii p} \ee^{-2\ii n \tilde{h}_b(\chi,\tau)} }{n^{\frac{1}{2}} f'_b(b(\chi,\tau);\chi,\tau)} H_{b,11}(\chi,\tau)^2 \ee^{-2(K_b(\chi,\tau) + S_b(\chi,\tau))}   \\ \cdot  \omega_b(\chi,\tau)^{2s} (b(\chi,\tau)-a(\chi,\tau))^{2\ii p} 
%%\left( f'_b(b(\chi,\tau);\chi,\tau) \right)^{2\ii p}
%%\end{multlined}\\
%%&\quad - \! \begin{multlined}[t]
%%\frac{s \alpha^{*} n^{-\ii p} \ee^{2\ii n \tilde{h}_b(\chi,\tau)} }{n^{\frac{1}{2}} f'_b(b(\chi,\tau);\chi,\tau)} H_{b,12}(\chi,\tau)^2 \ee^{2(K_b(\chi,\tau) + S_b(\chi,\tau))}   \\ \cdot  \omega_b(\chi,\tau)^{-2s} (b(\chi,\tau)-a(\chi,\tau))^{-2\ii p} 
%%\left( f'_b(b(\chi,\tau);\chi,\tau) \right)^{-2\ii p},
%%\end{multlined}
%\! \begin{multlined}[t]
%\frac{s \alpha M^{\ii p} \ee^{-2\ii M {h}_b(\chi,\tau)} }{M^{\frac{1}{2}} f'_b(b(\chi,\tau);\chi,\tau)} L_{b,11}(\chi,\tau)^2 \ee^{-2 \ii (K_b(\chi,\tau) + \mu(\chi,\tau) )}  (b(\chi,\tau)-a(\chi,\tau))^{2\ii p} \\ \cdot 
%\left( f'_b(b(\chi,\tau);\chi,\tau) \right)^{2\ii p}
%\end{multlined}\\
%&\quad - \! \begin{multlined}[t]
%\frac{s \alpha^{*} M^{-\ii p} \ee^{2\ii M {h}_b(\chi,\tau)} }{M^{\frac{1}{2}} f'_b(b(\chi,\tau);\chi,\tau)} L_{b,12}(\chi,\tau)^2 \ee^{2\ii (K_b(\chi,\tau) + \mu(\chi,\tau) )}  (b(\chi,\tau)-a(\chi,\tau))^{-2\ii p} \\ \cdot  
%\left( f'_b(b(\chi,\tau);\chi,\tau) \right)^{-2\ii p},
%\end{multlined}
%\end{split}
%\label{eq:V-F-12-b-residue}
%\end{equation}
%%\begin{equation}
%%\begin{split}
%%-\frac{1}{\pi} \int_{D_{b}(\delta)} V^{\mathbf{F}}_{12}(\eta)\dd \eta &=
%%%\! \begin{multlined}[t]
%%%\frac{s \alpha n^{\ii p} \ee^{-2\ii n \tilde{h}_b(\chi,\tau)} }{n^{\frac{1}{2}} f'_b(b(\chi,\tau);\chi,\tau)} H_{b,11}(\chi,\tau)^2 \ee^{-2(K_b(\chi,\tau) + S_b(\chi,\tau))}   \\ \cdot  \omega_b(\chi,\tau)^{2s} (b(\chi,\tau)-a(\chi,\tau))^{2\ii p} 
%%%\left( f'_b(b(\chi,\tau);\chi,\tau) \right)^{2\ii p}
%%%\end{multlined}\\
%%%&\quad - \! \begin{multlined}[t]
%%%\frac{s \alpha^{*} n^{-\ii p} \ee^{2\ii n \tilde{h}_b(\chi,\tau)} }{n^{\frac{1}{2}} f'_b(b(\chi,\tau);\chi,\tau)} H_{b,12}(\chi,\tau)^2 \ee^{2(K_b(\chi,\tau) + S_b(\chi,\tau))}   \\ \cdot  \omega_b(\chi,\tau)^{-2s} (b(\chi,\tau)-a(\chi,\tau))^{-2\ii p} 
%%%\left( f'_b(b(\chi,\tau);\chi,\tau) \right)^{-2\ii p},
%%%\end{multlined}
%%\! \begin{multlined}[t]
%%\frac{s \alpha M^{\ii p} \ee^{-2\ii M {h}_b(\chi,\tau)} }{M^{\frac{1}{2}} f'_b(b(\chi,\tau);\chi,\tau)} L_{b,11}(\chi,\tau)^2 \ee^{-2 \ii (K_b(\chi,\tau) + \mu(\chi,\tau) )}  (b(\chi,\tau)-a(\chi,\tau))^{2\ii p} \\ \cdot 
%%\left( f'_b(b(\chi,\tau);\chi,\tau) \right)^{2\ii p}
%%\end{multlined}\\
%%&\quad - \! \begin{multlined}[t]
%%\frac{s \alpha^{*} M^{-\ii p} \ee^{2\ii M {h}_b(\chi,\tau)} }{M^{\frac{1}{2}} f'_b(b(\chi,\tau);\chi,\tau)} L_{b,12}(\chi,\tau)^2 \ee^{2\ii (K_b(\chi,\tau) + \mu(\chi,\tau) )}  (b(\chi,\tau)-a(\chi,\tau))^{-2\ii p} \\ \cdot  
%%\left( f'_b(b(\chi,\tau);\chi,\tau) \right)^{-2\ii p},
%%\end{multlined}
%%\end{split}
%%\label{eq:V-F-12-b-residue}
%%\end{equation}
%%\begin{equation}
%%\begin{split}
%%-\frac{1}{\pi} \int_{D_{b}(\delta)} V^{\mathbf{F}}_{12}(\eta)\dd \eta &=
%%%\! \begin{multlined}[t]
%%%\frac{s \alpha n^{\ii p} \ee^{-2\ii n \tilde{h}_b(\chi,\tau)} }{n^{\frac{1}{2}} f'_b(b(\chi,\tau);\chi,\tau)} H_{b,11}(\chi,\tau)^2 \ee^{-2(K_b(\chi,\tau) + S_b(\chi,\tau))}   \\ \cdot  \omega_b(\chi,\tau)^{2s} (b(\chi,\tau)-a(\chi,\tau))^{2\ii p} 
%%%\left( f'_b(b(\chi,\tau);\chi,\tau) \right)^{2\ii p}
%%%\end{multlined}\\
%%%&\quad - \! \begin{multlined}[t]
%%%\frac{s \alpha^{*} n^{-\ii p} \ee^{2\ii n \tilde{h}_b(\chi,\tau)} }{n^{\frac{1}{2}} f'_b(b(\chi,\tau);\chi,\tau)} H_{b,12}(\chi,\tau)^2 \ee^{2(K_b(\chi,\tau) + S_b(\chi,\tau))}   \\ \cdot  \omega_b(\chi,\tau)^{-2s} (b(\chi,\tau)-a(\chi,\tau))^{-2\ii p} 
%%%\left( f'_b(b(\chi,\tau);\chi,\tau) \right)^{-2\ii p},
%%%\end{multlined}
%%\! \begin{multlined}[t]
%%\frac{s \alpha M^{\ii p} \ee^{-2\ii M {h}_b(\chi,\tau)} }{M^{\frac{1}{2}} f'_b(b(\chi,\tau);\chi,\tau)} H_{b,11}(\chi,\tau)^2 \ee^{-2 \ii K_b(\chi,\tau)}  (b(\chi,\tau)-a(\chi,\tau))^{2\ii p} \\ \cdot 
%%\left( f'_b(b(\chi,\tau);\chi,\tau) \right)^{2\ii p}
%%\end{multlined}\\
%%&\quad - \! \begin{multlined}[t]
%%\frac{s \alpha^{*} M^{-\ii p} \ee^{2\ii M {h}_b(\chi,\tau)} }{M^{\frac{1}{2}} f'_b(b(\chi,\tau);\chi,\tau)} H_{b,12}(\chi,\tau)^2 \ee^{2\ii K_b(\chi,\tau)}  (b(\chi,\tau)-a(\chi,\tau))^{-2\ii p} \\ \cdot  
%%\left( f'_b(b(\chi,\tau);\chi,\tau) \right)^{-2\ii p},
%%\end{multlined}
%%\end{split}
%%\label{eq:V-F-12-b-residue}
%%\end{equation}
%%\begin{equation}
%%\begin{split}
%%-\frac{1}{\pi} \int_{D_{b}(\delta)} V^{\mathbf{F}}_{12}(\eta)\dd \eta &=
%%%\! \begin{multlined}[t]
%%%\frac{s \alpha n^{\ii p} \ee^{-2\ii n \tilde{h}_b(\chi,\tau)} }{n^{\frac{1}{2}} f'_b(b(\chi,\tau);\chi,\tau)} H_{b,11}(\chi,\tau)^2 \ee^{-2(K_b(\chi,\tau) + S_b(\chi,\tau))}   \\ \cdot  \omega_b(\chi,\tau)^{2s} (b(\chi,\tau)-a(\chi,\tau))^{2\ii p} 
%%%\left( f'_b(b(\chi,\tau);\chi,\tau) \right)^{2\ii p}
%%%\end{multlined}\\
%%%&\quad - \! \begin{multlined}[t]
%%%\frac{s \alpha^{*} n^{-\ii p} \ee^{2\ii n \tilde{h}_b(\chi,\tau)} }{n^{\frac{1}{2}} f'_b(b(\chi,\tau);\chi,\tau)} H_{b,12}(\chi,\tau)^2 \ee^{2(K_b(\chi,\tau) + S_b(\chi,\tau))}   \\ \cdot  \omega_b(\chi,\tau)^{-2s} (b(\chi,\tau)-a(\chi,\tau))^{-2\ii p} 
%%%\left( f'_b(b(\chi,\tau);\chi,\tau) \right)^{-2\ii p},
%%%\end{multlined}
%%\! \begin{multlined}[t]
%%\frac{s \alpha M^{\ii p} \ee^{-2\ii M {h}_b(\chi,\tau)} }{M^{\frac{1}{2}} f'_b(b(\chi,\tau);\chi,\tau)} H_{b,11}(\chi,\tau)^2 \ee^{-2 \ii (K_b(\chi,\tau) + \mu(\chi,\tau) )}  (b(\chi,\tau)-a(\chi,\tau))^{2\ii p} \\ \cdot 
%%\left( f'_b(b(\chi,\tau);\chi,\tau) \right)^{2\ii p}
%%\end{multlined}\\
%%&\quad - \! \begin{multlined}[t]
%%\frac{s \alpha^{*} M^{-\ii p} \ee^{2\ii M {h}_b(\chi,\tau)} }{M^{\frac{1}{2}} f'_b(b(\chi,\tau);\chi,\tau)} H_{b,12}(\chi,\tau)^2 \ee^{2\ii (K_b(\chi,\tau) + \mu(\chi,\tau) )}  (b(\chi,\tau)-a(\chi,\tau))^{-2\ii p} \\ \cdot  
%%\left( f'_b(b(\chi,\tau);\chi,\tau) \right)^{-2\ii p},
%%\end{multlined}
%%\end{split}
%%\label{eq:V-F-12-b-residue}
%%\end{equation}
%%where we have set $\omega_a(\chi,\tau)\defeq \omega(a(\chi,\tau))$ and $\omega_b(\chi,\tau)\defeq \omega(b(\chi,\tau))$ along with
%%\begin{alignat}{2}
%%S_a(\chi,\tau) &\defeq  S_-(a(\chi,\tau);\chi,\tau), \qquad S_b(\chi,\tau) &&\defeq  S(b(\chi,\tau);\chi,\tau),\\
%%K_a(\chi,\tau) &\defeq  K_-(a(\chi,\tau);\chi,\tau), \qquad K_b(\chi,\tau) &&\defeq  K(b(\chi,\tau);\chi,\tau),\\
%%\mathbf{H}_a(\chi,\tau) &\defeq  \mathbf{H}_-(a(\chi,\tau);\chi,\tau), \qquad \mathbf{H}_b(\chi,\tau) &&\defeq  \mathbf{H}(b(\chi,\tau);\chi,\tau).
%%\end{alignat}
%where we have set
%\begin{equation}
%%K_a(\chi,\tau) \defeq  K_-(a(\chi,\tau);\chi,\tau)  \quad&\text{and}\quad K_b(\chi,\tau) \defeq  K(b(\chi,\tau);\chi,\tau),\\
%\mathbf{L}_a(\chi,\tau) \defeq  \mathbf{L}_-(a(\chi,\tau);\chi,\tau,\mathbf{Q}^{-s},M), \quad \text{and}\quad \mathbf{L}_b(\chi,\tau) \defeq  \mathbf{L}(b(\chi,\tau);\chi,\tau,\mathbf{Q}^{-s},M),
%\end{equation}
%and used the notation in \eqref{eq:intro-Ka} and \eqref{eq:intro-Kb}. Using 
%\begin{align}
%f_a'(a(\chi,\tau);\chi,\tau)& = - \left(-h_a''(\chi,\tau)\right)^{\frac{1}{2}}  \defeq  -\sqrt{-h_-(a(\chi,\tau);\chi,\tau)}<0\\
%f_b'(b(\chi,\tau);\chi,\tau)& = h_b''(\chi,\tau)^{\frac{1}{2}}  \defeq  \sqrt{h_-(b(\chi,\tau);\chi,\tau)}>0,
%\end{align}
%
%
%Recalling the definitions of the four positive modulation factors in \eqref{eq:m-a-b-shelves}, we obtain from the definition \eqref{eq:H-def} the (well-defined) expressions
%\begin{equation}
%%\begin{split}
%%H_{b,11}(\chi,\tau)^2 = \frac{1}{2} + \frac{1}{4} \big( y(b(\chi,\tau);\chi,\tau)^{-2} + y(b(\chi,\tau);\chi,\tau)^{2} \big)
%%=\frac{1}{2}\left[ 1+ \cos\left(\arg(b(\chi,\tau) - \lambda_0(\chi,\tau) ) \right)\right],
%L_{b,11}(\chi,\tau)^2 = \frac{1}{2} + \frac{1}{4} \big( y(b(\chi,\tau);\chi,\tau)^{-2} + y(b(\chi,\tau);\chi,\tau)^{2} \big)
%=m^+_b(\chi,\tau)
%%\frac{1}{2}\left[\cos\left(\arg(b(\chi,\tau) - \lambda_0(\chi,\tau) ) \right) + 1 \right],
%%\end{split}
%\label{eq:H-b-11}
%\end{equation}
%and similarly
%%\begin{equation}
%%%%\begin{split}
%%%%H_{b,12}(\chi,\tau)^2 &= \ee^{-4\ii \Theta_0(\chi,\tau;M)}\left( \frac{1}{2} - \frac{1}{4} \left[ \left(\frac{b(\chi,\tau) - \lambda_0(\chi,\tau)}{b(\chi,\tau) - \lambda_0(\chi,\tau)^*} \right)^{-\frac{1}{2}} + \left(\frac{b(\chi,\tau) - \lambda_0(\chi,\tau)}{b(\chi,\tau) - \lambda_0(\chi,\tau)^*} \right)^{\frac{1}{2}} \right]\right)\\
%%%%&=\frac{1}{2}\left[ 1- \cos\left(\arg(b(\chi,\tau) - \lambda_0(\chi,\tau) ) \right)\right]\ee^{-4\ii \Theta_0(\chi,\tau;M)},
%%%%\end{split}
%%\begin{split}
%%H_{b,12}(\chi,\tau)^2 &= \ee^{-4\ii (M\kappa(\chi,\tau)+ \mu(\chi,\tau))}\left( \frac{1}{2} - \frac{1}{4} \left[ \left(\frac{b(\chi,\tau) - \lambda_0(\chi,\tau)}{b(\chi,\tau) - \lambda_0(\chi,\tau)^*} \right)^{-\frac{1}{2}} + \left(\frac{b(\chi,\tau) - \lambda_0(\chi,\tau)}{b(\chi,\tau) - \lambda_0(\chi,\tau)^*} \right)^{\frac{1}{2}} \right]\right)\\
%%&=\frac{1}{2}\left[ 1- \cos\left(\arg(b(\chi,\tau) - \lambda_0(\chi,\tau) ) \right)\right]\ee^{-4\ii (M\kappa(\chi,\tau)+ \mu(\chi,\tau))},
%%\end{split}
%%\label{eq:H-b-12}
%%\end{equation}
%\begin{equation}
%%%\begin{split}
%%%H_{b,12}(\chi,\tau)^2 &= \ee^{-4\ii \Theta_0(\chi,\tau;M)}\left( \frac{1}{2} - \frac{1}{4} \left[ \left(\frac{b(\chi,\tau) - \lambda_0(\chi,\tau)}{b(\chi,\tau) - \lambda_0(\chi,\tau)^*} \right)^{-\frac{1}{2}} + \left(\frac{b(\chi,\tau) - \lambda_0(\chi,\tau)}{b(\chi,\tau) - \lambda_0(\chi,\tau)^*} \right)^{\frac{1}{2}} \right]\right)\\
%%%&=\frac{1}{2}\left[ 1- \cos\left(\arg(b(\chi,\tau) - \lambda_0(\chi,\tau) ) \right)\right]\ee^{-4\ii \Theta_0(\chi,\tau;M)},
%%%\end{split}
%%\begin{split}
%%H_{b,12}(\chi,\tau)^2 &= \left( - \frac{1}{2} + \frac{1}{4} \big( y(b(\chi,\tau);\chi,\tau)^{-2} + y(b(\chi,\tau);\chi,\tau)^{2} \big)\right)\ee^{-4\ii (M\kappa(\chi,\tau)+ \mu(\chi,\tau) + \frac{1}{4}s \pi )}\\
%%&=\frac{1}{2}\left[- 1 + \cos\left(\arg(b(\chi,\tau) - \lambda_0(\chi,\tau) ) \right)\right]\ee^{-4\ii (M\kappa(\chi,\tau)+ \mu(\chi,\tau) + \frac{1}{4}s \pi )}.
%%\end{split}
%L_{b,12}(\chi,\tau)^2 
%= - m^+_b(\chi,\tau) \ee^{-4\ii (M\kappa(\chi,\tau)+ \mu(\chi,\tau) + \frac{1}{4}s \pi )},
%%\frac{1}{2}\left[ \cos\left(\arg(b(\chi,\tau) - \lambda_0(\chi,\tau) ) \right) - 1 \right]\ee^{-4\ii (M\kappa(\chi,\tau)+ \mu(\chi,\tau) + \frac{1}{4}s \pi )},
%\label{eq:H-b-12}
%\end{equation}
%%\textcolor{red}{[There's a new minus sign coming from using $\mathbf{Q}$ instead of $\mathbf{O}$ above in the $(1,2)$-element.]}
%%\begin{equation}
%%%\begin{split}
%%%H_{b,12}(\chi,\tau)^2 &= \ee^{-4\ii \Theta_0(\chi,\tau;M)}\left( \frac{1}{2} - \frac{1}{4} \left[ \left(\frac{b(\chi,\tau) - \lambda_0(\chi,\tau)}{b(\chi,\tau) - \lambda_0(\chi,\tau)^*} \right)^{-\frac{1}{2}} + \left(\frac{b(\chi,\tau) - \lambda_0(\chi,\tau)}{b(\chi,\tau) - \lambda_0(\chi,\tau)^*} \right)^{\frac{1}{2}} \right]\right)\\
%%%&=\frac{1}{2}\left[ 1- \cos\left(\arg(b(\chi,\tau) - \lambda_0(\chi,\tau) ) \right)\right]\ee^{-4\ii \Theta_0(\chi,\tau;M)},
%%%\end{split}
%%%\begin{split}
%%H_{b,12}(\chi,\tau)^2 = 
%%%\ee^{-4\ii (M\kappa(\chi,\tau)+ \mu(\chi,\tau))}\left( \frac{1}{2} - \frac{1}{4} \left[ \left(\frac{b(\chi,\tau) - \lambda_0(\chi,\tau)}{b(\chi,\tau) - \lambda_0(\chi,\tau)^*} \right)^{-\frac{1}{2}} + \left(\frac{b(\chi,\tau) - \lambda_0(\chi,\tau)}{b(\chi,\tau) - \lambda_0(\chi,\tau)^*} \right)^{\frac{1}{2}} \right]\right)\\
%%\frac{1}{2}\left[ 1- \cos\left(\arg(b(\chi,\tau) - \lambda_0(\chi,\tau) ) \right)\right]\ee^{-4\ii (M\kappa(\chi,\tau)+ \mu(\chi,\tau) + \frac{1}{4}s \pi )},
%%%\end{split}
%%\label{eq:H-b-12}
%%\end{equation}
%together with
%%\begin{align}
%%H_{a,11}(\chi,\tau)^2 &=\frac{1}{2}\left[ 1+ \cos\left(\arg(a(\chi,\tau) - \lambda_0(\chi,\tau) ) \right)\right],
%%\label{eq:H-a-11}\\
%%H_{a,12}(\chi,\tau)^2 &=\frac{1}{2}\left[ 1- \cos\left(\arg(a(\chi,\tau) - \lambda_0(\chi,\tau) ) \right)\right]\ee^{-4\ii(M\kappa(\chi,\tau)+ \mu(\chi,\tau))},\label{eq:H-a-12}
%%\end{align}
%\begin{align}
%L_{a,11}(\chi,\tau)^2 &=
%%\frac{1}{2}\left[  \cos\left(\arg(a(\chi,\tau) - \lambda_0(\chi,\tau) ) \right) + 1\right],
%m^+_a(\chi,\tau)
%\label{eq:H-a-11}\\
%L_{a,12}(\chi,\tau)^2 &= - m^-_a(\chi,\tau) \ee^{-4\ii(M\kappa(\chi,\tau)+ \mu(\chi,\tau) + \frac{1}{4}s \pi )}.\label{eq:H-a-12}
%%\frac{1}{2}\left[  \cos\left(\arg(a(\chi,\tau) - \lambda_0(\chi,\tau) ) \right) - 1\right]\ee^{-4\ii(M\kappa(\chi,\tau)+ \mu(\chi,\tau) + \frac{1}{4}s \pi )}.\label{eq:H-a-12}
%\end{align}
%Next, by repeated differentiation in \eqref{eq:fb-def} and \eqref{eq:fa-def}, and using \eqref{eq:h-double-prime-b} and \eqref{eq:h-double-prime-a} along with the identities \eqref{eq:R-a-b}, we obtain
%%\begin{align}
%%f'(b(\chi,\tau);\chi,\tau)^2 &= \tilde{h}''(b(\chi,\tau);\chi,\tau) = \frac{2\tau(b(\chi,\tau) - a(\chi,\tau) ) |b(\chi,\tau)-\lambda_0(\chi,\tau)| }{b(\chi,\tau)^2+1}>0\\
%%f'(a(\chi,\tau);\chi,\tau)^2 &= - \tilde{h}_-''(a(\chi,\tau);\chi,\tau) = \frac{2\tau(b(\chi,\tau) - a(\chi,\tau) ) |a(\chi,\tau)-\lambda_0(\chi,\tau)| }{a(\chi,\tau)^2+1}>0,
%%\end{align}
%%recalling \eqref{eq:h-double-prime-b} and \eqref{eq:h-double-prime-a} along with the identities \eqref{eq:R-a-b}.
%%%\begin{equation}
%%%R(b(\chi,\tau);\chi,\tau) = |b(\chi,\tau)-\lambda_0(\chi,\tau)|\quad\text{and}\quad R_-(a(\chi,\tau);\chi,\tau) = |a(\chi,\tau)-\lambda_0(\chi,\tau)|.
%%%\label{eq:R-a-b}
%%%\end{equation}
%%As $f'(b(\chi,\tau);\chi,\tau)>0$ and $f'(a(\chi,\tau);\chi,\tau)<0$, we find that
%\begin{align}
%f_b'(b(\chi,\tau);\chi,\tau) &= \frac{\sqrt{2\tau}(b(\chi,\tau) - a(\chi,\tau) )^{\frac{1}{2}} | b(\chi,\tau)-\lambda_0(\chi,\tau)|^{\frac{1}{2}} }{\sqrt{b(\chi,\tau)^2+1}},
%\label{eq:fb-prime-at-b}\\
%f_a'(a(\chi,\tau);\chi,\tau) &=  \frac{- \sqrt{2\tau} (b(\chi,\tau) - a(\chi,\tau) )^{\frac{1}{2}} | a(\chi,\tau) - \lambda_0(\chi,\tau) |^{\frac{1}{2}} }{\sqrt{a(\chi,\tau)^2+1}}.
%\label{eq:fa-prime-at-a}
%\end{align}
%We also write $M^{\pm \ii p} = \ee^{\pm \ii \ln(M) p}$ because $p\in\mathbb{R}$, and 
%\begin{equation}
%( b(\chi,\tau) - a(\chi,\tau) )^{\pm 2 \ii p} = \ee^{\pm 2\ii p \ln( b(\chi,\tau) - a(\chi,\tau) )}
%\label{eq:log-b-a}
%\end{equation}
%because $a(\chi,\tau)<b(\chi,\tau)$ for $(\chi,\tau) \in \shelves$. 
%
%We now substitute \eqref{eq:H-b-11}--\eqref{eq:H-a-12} and \eqref{eq:fb-prime-at-b}--\eqref{eq:log-b-a} in \eqref{eq:V-F-12-a-residue} and \eqref{eq:V-F-12-b-residue}. In doing so, we write $-1 = \ee^{\ii \pi s}$ for the factors of $-1$ in the first line of \eqref{eq:V-F-12-a-residue} and in the second line of \eqref{eq:V-F-12-b-residue} to arrive at the formula 
%\begin{multline}
%q(M\chi,M\tau;\mathbf{Q}^{-s},M) = B(\chi,\tau)  \ee^{-2\ii (M\kappa(\chi,\tau)+\mu(\chi,\tau) + \frac{1}{4}s \pi)}
%  + M^{-\frac{1}{2}} s \ee^{-2\ii (M\kappa(\chi,\tau)+\mu(\chi,\tau) )} \\
% \cdot \left[ F_a^+(\chi,\tau)\ee^{\ii \Theta_a(\chi,\tau;M)}    + F_a^-(\chi,\tau)\ee^{- \ii \Theta_a(\chi,\tau;M)} \right. \\
%  \left. + F_b^+(\chi,\tau)\ee^{\ii \Theta_b(\chi,\tau;M)}   + F_b^-(\chi,\tau)\ee^{- \ii \Theta_b(\chi,\tau;M)} \right] + O(M^{-1}),
%  \label{eq:q-bun}
%\end{multline}
%which can be further simplified using $s=\pm 1$ to read
%\begin{multline}
%%\psi_k(M\chi, M\tau) = 
%q(M\chi,M\tau;\mathbf{Q}^{-s},M)=
%s  \ee^{-2\ii (M\kappa(\chi,\tau)+\mu(\chi,\tau))} 
%\left[-\ii B(\chi,\tau) + M^{-\frac{1}{2}} \left( F_a^+(\chi,\tau)\ee^{\ii \Theta_a(\chi,\tau;M)}  \right. \right.\\ + F_a^-(\chi,\tau)\ee^{- \ii \Theta_a(\chi,\tau;M)}
%\left.\left. 
%+ F_b^+(\chi,\tau)\ee^{\ii \Theta_b(\chi,\tau;M)} + F_b^-(\chi,\tau)\ee^{- \ii \Theta_b(\chi,\tau;M)} 
%\right) \right]
%+ O(M^{-1}).
%\label{eq:q-bun-alt}
%\end{multline}
%Here
%\begin{equation}
%\begin{aligned}
%%F^{\pm}_a(\chi,\tau) &\defeq  \pm \frac{\sqrt{\ln(2)}\left(1 \pm \cos( \arg( a(\chi,\tau)-\lambda_0(\chi,\tau) ) ) \right) }{2 \sqrt{\pi} f_a'(a(\chi,\tau);\chi,\tau)},\\
%F^{\pm}_a(\chi,\tau) &\defeq  \frac{\sqrt{\ln(2)}\left(\cos( \arg( a(\chi,\tau)-\lambda_0(\chi,\tau) ) ) \pm 1  \right) }{2 \sqrt{\pi} f_a'(a(\chi,\tau);\chi,\tau)},\\
%F^{\pm}_b(\chi,\tau) &\defeq   \frac{\sqrt{\ln(2)} \left(\cos( \arg( b(\chi,\tau)-\lambda_0(\chi,\tau) ) ) \pm 1  \right) }{2 \sqrt{\pi} f_b'(b(\chi,\tau);\chi,\tau)}
%\end{aligned}
%\label{eq:shelves-amplitudes-bun}
%\end{equation}
%are the real-valued amplitudes in which we have incorporated $|\alpha|=\sqrt{\ln(2)/\pi}$ from \eqref{eq:Channels-alpha-beta}, and
%\begin{align}
%\Theta_a(\chi,\tau;M) &\defeq  M \Phi_a(\chi,\tau) - \ln(M) \frac{\ln(2)}{2\pi} + \eta_a(\chi,\tau),\label{eq:Theta-a}\\
%\Theta_b(\chi,\tau;M) &\defeq  M \Phi_b(\chi,\tau)  + \ln(M) \frac{\ln(2)}{2\pi} + \eta_b(\chi,\tau),\label{eq:Theta-b}
%\end{align}
%are the phase factors, in which
%\begin{align}
%\Phi_a(\chi,\tau) &\defeq  2\left( \kappa(\chi,\tau) - {h}_a(\chi,\tau) \right),\\
%\Phi_b(\chi,\tau) &\defeq  2\left( \kappa(\chi,\tau) - {h}_b(\chi,\tau) \right),
%\end{align}
%constitute the \emph{fast} components and
%%\begin{align}
%%\eta_a(\chi,\tau) &\defeq 
%%\! \begin{multlined}[t]
%% 2\left((-1)^k \gamma(\chi,\tau) + \mu(\chi,\tau) +\ii K_a(\chi,\tau) + \ii S_a(\chi,\tau) \right) + (-1)^k[ \arg(a(\chi,\tau)-\ii)+\pi ]\\
%%+ 2p \left[ \ln\left(\frac{\sqrt{a(\chi,\tau)^2 + 1}}{\sqrt{2\tau}(b(\chi,\tau)-a(\chi,\tau))^{\frac{1}{2}} |a(\chi,\tau)-\lambda_0(\chi,\tau)|^{\frac{1}{2}}} \right) - \ln(b(\chi,\tau) - a(\chi,\tau) ) \right]\\
%%- \frac{1}{4}\pi - 2\pi p^2 + \arg(\Gamma(\ii p)),
%%\end{multlined}\\
%%\eta_b(\chi,\tau) &\defeq   
%%\! \begin{multlined}[t]
%%2\left((-1)^k \gamma(\chi,\tau) + \mu(\chi,\tau) +\ii K_b(\chi,\tau) + \ii S_b(\chi,\tau) \right)  + (-1)^k \arg(b(\chi,\tau)-\ii)\\
%%+ 2p \left[ \ln\left(\frac{\sqrt{2\tau}(b(\chi,\tau)-a(\chi,\tau))^{\frac{1}{2}} |b(\chi,\tau)-\lambda_0(\chi,\tau)|^{\frac{1}{2}}}{\sqrt{b(\chi,\tau)^2 + 1}} \right) + \ln(b(\chi,\tau) - a(\chi,\tau) ) \right]\\
%%+\frac{1}{4}\pi+2\pi p^2-\arg(\Gamma(\ii p)),
%%\end{multlined}
%%\eta_a(\chi,\tau) &\defeq 
%%\! \begin{multlined}[t]
%% 2\left((-1)^k \gamma(\chi,\tau) + \mu(\chi,\tau) - K_a(\chi,\tau) + \ii S_a(\chi,\tau) \right) + (-1)^k[ \arg(a(\chi,\tau)-\ii)+\pi ]\\
%%+ 2p \left[ \ln\left(\frac{\sqrt{a(\chi,\tau)^2 + 1}}{\sqrt{2\tau}(b(\chi,\tau)-a(\chi,\tau))^{\frac{1}{2}} |a(\chi,\tau)-\lambda_0(\chi,\tau)|^{\frac{1}{2}}} \right) - \ln(b(\chi,\tau) - a(\chi,\tau) ) \right]\\
%%- \frac{1}{4}\pi - 2\pi p^2 + \arg(\Gamma(\ii p)),
%%\end{multlined}\\
%%\eta_b(\chi,\tau) &\defeq   
%%\! \begin{multlined}[t]
%%2\left((-1)^k \gamma(\chi,\tau) + \mu(\chi,\tau) - K_b(\chi,\tau) + \ii S_b(\chi,\tau) \right)  + (-1)^k \arg(b(\chi,\tau)-\ii)\\
%%+ 2p \left[ \ln\left(\frac{\sqrt{2\tau}(b(\chi,\tau)-a(\chi,\tau))^{\frac{1}{2}} |b(\chi,\tau)-\lambda_0(\chi,\tau)|^{\frac{1}{2}}}{\sqrt{b(\chi,\tau)^2 + 1}} \right) + \ln(b(\chi,\tau) - a(\chi,\tau) ) \right]\\
%%+\frac{1}{4}\pi+2\pi p^2-\arg(\Gamma(\ii p)),
%%\end{multlined}
%%\end{align}
%%constitute the \emph{slow} components, where we have used $\arg(\alpha) =\frac{1}{4}\pi+2\pi p^2-\arg(\Gamma(\ii p)) $ from \eqref{eq:Channels-alpha-beta}. 
%%Using the expressions \eqref{eq:S-at-b}--\eqref{eq:S-tilde-at-a} and \eqref{eq:K-at-b}--\eqref{eq:K-at-a} results in further simplification of the slow phase components $\eta_a(\chi,\tau)$ and $\eta_b(\chi,\tau)$, and we obtain
%%\begin{align}
%%%%\eta_a(\chi,\tau) &\defeq 
%%%%\! \begin{multlined}[t]
%%%% 2\ii\left( \tilde{K}_a(\chi,\tau) + (-1)^k \tilde{S}_a(\chi,\tau) \right) - \frac{1}{4}\pi - 2\pi p^2 + \arg(\Gamma(\ii p)) \\
%%%%+ 2p \ln\left(\frac{\sqrt{a(\chi,\tau)^2 + 1}}{\sqrt{2\tau}(b(\chi,\tau)-a(\chi,\tau))^{\frac{3}{2}} |a(\chi,\tau)-\lambda_0(\chi,\tau)|^{\frac{1}{2}}} \right),
%%%%\end{multlined}\\
%%%\eta_b(\chi,\tau) &\defeq   
%%%\! \begin{multlined}[t]
%%% 2\ii\left( \tilde{K}_b(\chi,\tau) + (-1)^k \tilde{S}_b(\chi,\tau) \right) +\frac{1}{4}\pi+2\pi p^2-\arg(\Gamma(\ii p)) \\
%%%+ 2p  \ln\left(\frac{\sqrt{2\tau}(b(\chi,\tau)-a(\chi,\tau))^{\frac{3}{2}} |b(\chi,\tau)-\lambda_0(\chi,\tau)|^{\frac{1}{2}}}{\sqrt{b(\chi,\tau)^2 + 1}} \right).
%%%\end{multlined}
%%\eta_a(\chi,\tau) &\defeq 
%%\! \begin{multlined}[t]
%% 2\left( -\tilde{K}_a(\chi,\tau) + (-1)^k \ii \tilde{S}_a(\chi,\tau) \right) - \frac{1}{4}\pi - 2\pi p^2 + \arg(\Gamma(\ii p)) \\
%%+ 2p \ln\left(\frac{\sqrt{a(\chi,\tau)^2 + 1}}{\sqrt{2\tau}(b(\chi,\tau)-a(\chi,\tau))^{\frac{3}{2}} |a(\chi,\tau)-\lambda_0(\chi,\tau)|^{\frac{1}{2}}} \right),
%%\end{multlined}\\
%%\eta_b(\chi,\tau) &\defeq   
%%\! \begin{multlined}[t]
%% 2 \left(- \tilde{K}_b(\chi,\tau) + (-1)^k \ii \tilde{S}_b(\chi,\tau) \right) +\frac{1}{4}\pi+2\pi p^2-\arg(\Gamma(\ii p)) \\
%%+ 2p  \ln\left(\frac{\sqrt{2\tau}(b(\chi,\tau)-a(\chi,\tau))^{\frac{3}{2}} |b(\chi,\tau)-\lambda_0(\chi,\tau)|^{\frac{1}{2}}}{\sqrt{b(\chi,\tau)^2 + 1}} \right).
%%\end{multlined}
%%\end{align}
%\begin{align}
%\eta_a(\chi,\tau) &\defeq 
%\! \begin{multlined}[t]
% -2{K}_a(\chi,\tau)  - \frac{1}{4}\pi - 2\pi p^2 + \arg\left(\Gamma\left( \frac{\ii\ln(2)}{2\pi}\right)\right) \\
%+ \frac{\ln(2)}{\pi}  \ln\left(\frac{\sqrt{a(\chi,\tau)^2 + 1}}{\sqrt{2\tau}(b(\chi,\tau)-a(\chi,\tau))^{\frac{3}{2}} |a(\chi,\tau)-\lambda_0(\chi,\tau)|^{\frac{1}{2}}} \right),
%\end{multlined}\\
%\eta_b(\chi,\tau) &\defeq   
%\! \begin{multlined}[t]
%- 2{K}_b(\chi,\tau) +\frac{1}{4}\pi+2\pi p^2-\arg\left(\Gamma\left( \frac{\ii \ln(2)}{2\pi}\right)\right) \\
%+ \frac{\ln(2)}{\pi}  \ln\left(\frac{\sqrt{2\tau}(b(\chi,\tau)-a(\chi,\tau))^{\frac{3}{2}} |b(\chi,\tau)-\lambda_0(\chi,\tau)|^{\frac{1}{2}}}{\sqrt{b(\chi,\tau)^2 + 1}} \right).
%\end{multlined}
%\end{align}
%constitute the \emph{slow} components, where we have used $\arg(\alpha) =\frac{1}{4}\pi+2\pi p^2-\arg(\Gamma(\ii p))$ from \eqref{eq:Channels-alpha-beta} with $p=\tfrac{\ln(2)}{2\pi}$. \textcolor{red}{This completes the proof of Theorem~\ref{}.}


%\subsubsection{Interference pattern}
%\label{sec:interference}
% We now provide a description of the interference pattern observed in the density plots for the amplitude of fundamental rogue waves $\psi_{k}(x,t)$ in Figure 2 in \cite{BilmanLM20} away from the peak of the wave field which occurs at $(x,t)=(0,0)$. The region in which this pattern formation occurs coincides with $\shelves$. We show that the phenomenon is not intrinsic to fundamental rogue waves, it rather is produced for the more general family of solutions $q (M\chi, M \tau; \mathbf{Q}^{-s}, M)$ with an arbitrary unbounded and increasing sequence $\{M_k\}_{k=1}^\infty$ and $s=\pm 1$, which also includes the multi-pole $2M$-solitons studied in \cite{BilmanB19,BilmanBW19} as special case.  
% 
% Observe that taking the squared modulus of $q (M\chi, M \tau; \mathbf{Q}^{-s}, M)$ in \eqref{eq:q-bun} gives:
% \begin{multline}
% | q (M\chi, M \tau; \mathbf{Q}^{-s}, M) |^2 = B(\chi,\tau)^2
% - M^{-\frac{1}{2}} 2 B(\chi,\tau)\sqrt{\frac{\ln(2)}{\pi}}\\
% \cdot \left( \frac{1}{f_a'(a(\chi,\tau);\chi,\tau)} \sin(\Theta_a(\chi,\tau;M)) + \frac{1}{f_b'(b(\chi,\tau);\chi,\tau)} \sin(\Theta_b(\chi,\tau;M)) \right) + O(M^{-1}),\quad M \to+\infty.
% \label{eq:interference}
% \end{multline}
%This shows that the squared modulus of $q (M\chi, M \tau; \mathbf{Q}^{-s}, M)$ has a slowly varying ``shelf'' of size $O(1)$ and a rapidly varying perturbation of this self proportional to $M^{-\frac{1}{2}}$. This perturbation is a superposition of two sine functions with different phases $\Theta_a(\chi,\tau;M)$ and $\Theta_b(\chi,\tau;M)$ that are large functions of $(\chi,\tau)$ for $M\gg 1$ due to the presence of the terms $M\Phi_a(\chi,\tau)$ and $M\Phi_b(\chi,\tau)$, see \eqref{eq:Theta-a} and \eqref{eq:Theta-b}. These fast oscillations produce the interference pattern visible in Figure 2 in \cite{BilmanLM20} \textcolor{red}{and in Figure~\ref{}} for the special case of fundamental rogue waves.  \textcolor{red}{[Let's emphasize the universal nature of this wave pattern in the introduction, or here?]}


%\subsection{Wave theoretic interpretation of the asymptotic formula for $\psi_k(M\chi, M\tau)$ in $\shelves$.}
\subsection{Wave-theoretic interpretation of the asymptotic formula for $q(M\chi, M\tau; \mathbf{Q}^{-s}, M)$ in $\shelves$.} 
\label{sec:wave-theoretic-interpretation}
In this subsection we prove Corollary~\ref{cor:shelves-local}.
%\textcolor{red}{[Checking this. Redefining $h$ changes things.]} 
%Recall that $q(x,t) = q(x, t; \mathbf{Q}^{-s}, M)$ solves the focusing nonlinear Schr\"odinger equation in the form \eqref{eq:NLS-ZBC}. 
As we will be working in a relative perturbation regime of the leading term in the large-$M$ asymptotic expansion of $q(M\chi,M\tau; \mathbf{Q}^{-s}, M)$, we compare with the formula \eqref{eq:leading-and-subleading-shelves-rewritten} and write the leading term in the form
\begin{equation}
\mathfrak{L}_s^{[\shelves]}(\chi,\tau;M) =  -\ii s B(\chi,\tau) \ee^{-2\ii \phi(\chi,\tau;M)}.
\label{eq:q-0-bun}
\end{equation}
We then fix $(\chi_0,\tau_0)\in\shelves$, and write $\chi = \chi_0 + \Delta \chi$ and $ \tau = \tau_0 + \Delta \tau$.
%\begin{equation}
%\chi = \chi_0 + \Delta \chi\quad\text{and}\quad \tau = \tau_0 + \Delta \tau.
%\end{equation}
Noting that $\Delta \chi = M^{-1} \Delta x$ and $\Delta \tau = M^{-1} \Delta t$, and recalling the assumptions $\Delta x = O(1)$ and $\Delta t = O(1)$, it is easy to see that the phase $\phi(\chi,\tau;M)$ (see \eqref{eq:symmetrical-phases}) admits the following Taylor series expansion about $(\chi,\tau) = (\chi_0, \tau_0)$
\begin{equation}
\begin{split}
\phi(\chi,\tau;M) 
%&= M \left[\kappa(\chi_0,\tau_0) + \kappa_\chi (\chi_0,\tau_0)\Delta \chi + \kappa_\tau (\chi_0,\tau_0)\Delta \tau + O(\Delta \chi^2) + O(\Delta \chi \Delta \tau) + O(\Delta \tau^2) \right]\\
%&\quad+ \mu(\chi_0,\tau_0) + O(\Delta \chi) + O(\Delta \tau)\\
&= M \kappa(\chi_0,\tau_0) + \kappa_\chi (\chi_0,\tau_0)\Delta x + \kappa_\tau (\chi_0,\tau_0)\Delta t + O(M^{-1}\Delta x^2) + O(M^{-1}\Delta x \Delta t) + O(M^{-1}\Delta t^2) \\
&\quad+ \mu(\chi_0,\tau_0) + O(M^{-1}\Delta x) + O(M^{-1}\Delta t)\\
&=\phi(\chi_0,\tau_0; M) + \kappa_\chi (\chi_0,\tau_0)\Delta x + \kappa_\tau (\chi_0,\tau_0)\Delta t + O(M^{-1}),\quad M\to +\infty,
\end{split}
\label{eq:Omega-0-expand-1}
\end{equation}
which implies
\begin{equation}
%\ee^{-2\ii \Theta_0(\chi,\tau;M)} = \ee^{-2\ii \Theta_0 (\chi_0,\tau_0)} \ee^{\ii(\xi_0\Delta x - \Omega_0 \Delta t)} + o(n^{-2}),\quad n\to +\infty,
\ee^{-2\ii \phi(\chi,\tau;M)} = \ee^{-2\ii \phi (\chi_0,\tau_0;M)} \ee^{\ii(\xi_0\Delta x - \Omega_0 \Delta t)} + O(M^{-1}),\quad M\to +\infty,
\label{eq:Omega-0-Taylor-bun}
\end{equation}
in which real local wavenumber $\xi_0$ and real local frequency $\Omega_0$ are defined in \eqref{eq:wavenumbers-intro}--\eqref{eq:frequencies-intro}. On the other hand, Taylor expansion of $B(\chi,\tau)$ in \eqref{eq:q-0-bun} around the same point $(\chi_0,\tau_0)$ gives
\begin{equation}
%B(\chi,\tau) = B(\chi_0,\tau_0) + o(n^{-2}),\quad n\to +\infty.
B(\chi,\tau) = B(\chi_0,\tau_0) + O(M^{-1}),\quad M\to +\infty.
\label{eq:B-Taylor-bun}
\end{equation}
Combining \eqref{eq:Omega-0-Taylor-bun} and \eqref{eq:B-Taylor-bun} in \eqref{eq:q-0-bun} yields the expansion
\begin{equation}
%q_k^{(0)}(x,t) = -\ii \ee^{\ii \beta_0} B(\chi_0,\tau_0)  \ee^{\ii(\xi_0 \Delta x - \Omega_0 \Delta t)} + o(n^{-2}),\quad n\to +\infty,
%q_k^{(0)}(x,t) = -\ii \ee^{\ii \beta_0} B(\chi_0,\tau_0)  \ee^{\ii(\xi_0 \Delta x - \Omega_0 \Delta t)} + O(M^{-1}),\quad M\to +\infty,
%q^{(0)}(M \chi_0 + \Delta x, M\tau_0 +\Delta t) 
\mathfrak{L}_s^{[\shelves]}(\chi_0+M^{-1}\Delta x,\tau_0+M^{-1}\Delta t;M)
=Q(\Delta x, \Delta t) + O(M^{-1}),\quad M\to +\infty,
\label{eq:Q-0-expand}
\end{equation}
valid uniformly for $(\Delta x,\Delta t)$ bounded, where $Q(\Delta x, \Delta t)$ is given in \eqref{eq:leading-plane-wave-intro}. 
%We now set 
%\begin{multline}
%q^{(1)}(x,t) \defeq m_a^+(\chi,\tau)F_a^{[\shelves]}(\chi,\tau)\ee^{\ii\phi_a(\chi,\tau;M)} -
%m_a^-(\chi,\tau)F_a^{[\shelves]}(\chi,\tau)\ee^{-\ii\phi_a(\chi,\tau;M)} \\
%{}+
%m_b^+(\chi,\tau)F_b^{[\shelves]}(\chi,\tau)\ee^{\ii\phi_b(\chi,\tau;M)} -
%m_b^-(\chi,\tau)F_b^{[\shelves]}(\chi,\tau)\ee^{-\ii\phi_b(\chi,\tau;M)}
%\end{multline}
%to label the coefficient of the term proportional to $M^{-\frac{1}{2}}$ in \eqref{eq:leading-and-subleading-shelves-rewritten}.
We proceed in a similar manner and obtain Taylor series expansions of the terms in the sub-leading term in  \eqref{eq:leading-and-subleading-shelves-rewritten} around the same fixed point $(\chi_0,\tau) \in \shelves$. Recall the definitions \eqref{eq:symmetrical-phases} of the symmetrical phases $\phi_a$ and $\phi_b$. For bounded $(\Delta x, \Delta t)$ as before, we have
\begin{multline}
\phi_a(\chi,\tau;M) = \phi_a(\chi_0,\tau_0; M) + \left(\Phi^{[\shelves]}_{a,\chi}(\chi_0,\tau_0) + 2\kappa_\chi(\chi_0,\tau_0) \right)\Delta x \\ + \left( \Phi^{[\shelves]}_{a,\chi}(\chi_0,\tau_0) + 2\kappa_\tau(\chi_0,\tau_0) \right) \Delta t + O(M^{-1}).
\end{multline}
Substituting \eqref{eq:Phis-shelves} in this expression and recalling the definitions  \eqref{eq:wavenumbers-intro}--\eqref{eq:frequencies-intro} for the real local wavenumber $\xi_a$ and real local frequency $\Omega_a$ gives
\begin{equation}
%\ee^{\pm \ii \Theta_a(\chi,\tau;M)} = \ee^{\pm \ii \beta_a} \ee^{\pm \ii (\xi_a \Delta x - \Omega_a \Delta t)} + o(n^{-2}),\quad n\to +\infty,
\ee^{\pm \ii \phi_a(\chi,\tau;M)} = \ee^{\pm \ii \phi_a(\chi_0,\tau_0;M)} \ee^{\pm \ii (\xi_a \Delta x - \Omega_a \Delta t)} + O(M^{-1}),\quad M\to +\infty.
\label{eq:expand-phi-a}
\end{equation}
An identical calculation for the phase $\phi_b(\chi,\tau;M)$ gives
\begin{equation}
%\ee^{\pm \ii \Theta_a(\chi,\tau;M)} = \ee^{\pm \ii \beta_a} \ee^{\pm \ii (\xi_a \Delta x - \Omega_a \Delta t)} + o(n^{-2}),\quad n\to +\infty,
\ee^{\pm \ii \phi_b(\chi,\tau;M)} = \ee^{\pm \ii \phi_b(\chi_0,\tau_0;M)} \ee^{\pm \ii (\xi_b \Delta x - \Omega_b \Delta t)} + O(M^{-1}),\quad M\to +\infty.
\label{eq:expand-phi-b}
\end{equation}
On the other hand, for the amplitude factors in \eqref{eq:leading-and-subleading-shelves-rewritten} we have the expansions 
\begin{align}
m_a^{\pm}(\chi,\tau)F_a^{[\shelves]}(\chi,\tau)&=  m_a^{\pm}(\chi_0,\tau_0)F_a^{[\shelves]}(\chi_0,\tau_0) + O(M^{-1}),\label{eq:expand-m-F-a}\\
m_b^{\pm}(\chi,\tau)F_b^{[\shelves]}(\chi,\tau)&=  m_b^{\pm}(\chi_0,\tau_0)F_b^{[\shelves]}(\chi_0,\tau_0) + O(M^{-1})\label{eq:expand-m-F-b}.
\end{align}
Using \eqref{eq:expand-phi-a}--\eqref{eq:expand-m-F-b} in the sub-leading term $\mathfrak{S}_s^{[\shelves]}(\chi,\tau;M)$ written in the form \eqref{eq:leading-and-subleading-shelves-rewritten}, taking into account the overall multiplicative factor $M^{-\frac{1}{2}}$ in \eqref{eq:leading-and-subleading-shelves-rewritten} for the error terms in \eqref{eq:expand-phi-a}--\eqref{eq:expand-m-F-b}, and factoring out $Q(\Delta x, \Delta t)$ to express the sub-leading term as a relative perturbation results in the expansion \eqref{eq:Q-perturbation-shelves}, which proves the first statement in Corollary~\ref{cor:shelves-local}.

To show that $Q(\Delta x, \Delta t)$ is a plane-wave solution of \eqref{eq:NLS-Deltas}, we need the following lemma concerning the partial derivatives $g_\chi(\lambda;\chi,\tau)$ and $g_\tau(\lambda;\chi,\tau)$.
%\label{eq:NLS-Deltas}
%we have set $\ee^{\ii\beta_0}\defeq  s \ee^{-2\ii \Theta_0(\chi_0,\tau_0;M)}$ and $\beta_0$ is real since $s=\pm 1$ and $\Theta_0(\chi_0,\tau_0;M)$ is real.
%Recalling the definition \eqref{eq:symmetrical-phases} of $\phi(\chi,\tau;M)$ 
%
%
%
%and

%We begin the analysis with removing the rotating frame factor $\ee^{-\ii t}$ and consider
%\begin{equation}
%%q_{k}(x,t)\defeq  \ee^{\ii t}\psi_{k}(x,t)
%Q(x,t)=Q(x,t;\mathbf{Q}^{-s},M)\defeq  \ee^{\ii t}q(x,t;\mathbf{Q}^{-s},M)
%\label{eq:psi-to-q}
%\end{equation}
%%which satisfies the NLS equation in the form $\ii q_t + \tfrac{1}{2}q_{xx} + |q|^2 q =0$. We also set
%which satisfies the NLS equation in the form $\ii Q_t + \tfrac{1}{2}Q_{xx} + |Q|^2 Q =0$. 
%set
%\begin{equation}
%\Theta_0(\chi,\tau;M) \defeq  M\kappa(\chi,\tau) + \mu(\chi,\tau)
%\end{equation}
%to denote the overall phase factor which appears in \eqref{eq:q-bun-alt}. 
%Now fix $(\chi_0, \tau_0)\in \shelves$. We write 
%\begin{equation}
%\chi = \chi_0 + \Delta \chi\quad\text{and}\quad \tau = \tau_0 + \Delta \tau,
%\end{equation}
%with $\Delta \chi , \Delta \tau \ll 1$, whose sizes compared to $M \gg 1$ are to be determined. Recalling that $\chi = x/M$ and $\tau = t/M$, we have
%\begin{equation}
%\Delta \chi = \frac{1}{M} \Delta x \quad\text{and}\quad \Delta \tau = \frac{1}{M} \Delta t.
%\end{equation}
%%The leading order term in the asymptotic formula for $q_k(x,t)$ as $M\to+\infty$ is read from \eqref{eq:q-bun} as:
%%\begin{equation}
%%q_k^{(0)}(x,t) \defeq  -\ii s B(\chi,\tau) \ee^{-2\ii \Theta_0(\chi,\tau;M)}.
%%\label{eq:q-0-bun}
%%\end{equation}
%The leading order term in the asymptotic formula for $q(x,t;\mathbf{Q}^{-s},M)$ as $M\to+\infty$ is read from \eqref{eq:q-bun-alt} as:
%\begin{equation}
%q^{(0)}(x,t) \defeq  -\ii s B(\chi,\tau) \ee^{-2\ii \Theta_0(\chi,\tau;M)}.
%\label{eq:q-0-bun}
%\end{equation}
%Expanding the phase $\Theta_0(\chi,\tau;M)$ in Taylor series about $(\chi,\tau) = (\chi_0, \tau_0)$ yields
%\begin{equation}
%\begin{split}
%\Theta_0(\chi,\tau;M) &= M \left[\kappa(\chi_0,\tau_0) + \kappa_\chi (\chi_0,\tau_0)\Delta \chi + \kappa_\tau (\chi_0,\tau_0)\Delta \tau + O(\Delta \chi^2) + O(\Delta \chi \Delta \tau) + O(\Delta \tau^2) \right]\\
%&\quad+ \mu(\chi_0,\tau_0) + O(\Delta \chi) + O(\Delta \tau)\\
%&= M \kappa(\chi_0,\tau_0) + \kappa_\chi (\chi_0,\tau_0)\Delta x + \kappa_\tau (\chi_0,\tau_0)\Delta t + O(M^{-1}\Delta x^2) + O(M^{-1}\Delta x \Delta t) + O(M^{-1}\Delta t^2) \\
%&\quad+ \mu(\chi_0,\tau_0) + O(M^{-1}\Delta x) + O(M^{-1}\Delta t).
%\end{split}
%\label{eq:Omega-0-expand-1}
%\end{equation}
%We assume that $\Delta x =O(1)$ and $\Delta t = O(1)$, which guarantees that the error terms $M^{-1}\Delta x^2$, $M^{-1}\Delta x \Delta t$, $M^{-1}\Delta t^2$, $M^{-1}\Delta x$, and $M^{-1}\Delta t$ above are \emph{all} of size $O(M^{-1})$ as $M\to+\infty$.
%Consequently, we may write
%%To control the errors we require that $M^{-1}\Delta x^2$, $M^{-1}\Delta x \Delta t$, $M^{-1}\Delta t^2$, $M^{-1}\Delta x$, and $M^{-1}\Delta t$ \emph{all} to be of size $O(M^{-1})$ as $M\to+\infty$. This is guaranteed if $\Delta x =O(1)$ and $\Delta t = O(1)$, and consequently, we may write
%\begin{equation}
%\Theta_0(\chi,\tau;M) = \Theta_0(\chi_0,\tau_0;M) + \kappa_\chi (\chi_0,\tau_0)\Delta x + \kappa_\tau (\chi_0,\tau_0)\Delta t + O(M^{-1}),\quad M\to +\infty,
%\label{eq:Omega-0-expand-2}
%\end{equation}
%which implies
%\begin{equation}
%%\ee^{-2\ii \Theta_0(\chi,\tau;M)} = \ee^{-2\ii \Theta_0 (\chi_0,\tau_0)} \ee^{\ii(\xi_0\Delta x - \Omega_0 \Delta t)} + o(n^{-2}),\quad n\to +\infty,
%\ee^{-2\ii \Theta_0(\chi,\tau;M)} = \ee^{-2\ii \Theta_0 (\chi_0,\tau_0;M)} \ee^{\ii(\xi_0\Delta x - \Omega_0 \Delta t)} + O(M^{-1}),\quad M\to +\infty,
%\label{eq:Omega-0-Taylor-bun}
%\end{equation}
%where 
%\begin{align}
%%\beta_0 &\defeq  -2\Theta_0(\chi_0,\tau_0;M),\\
%\xi_0 &\defeq   -2  \kappa_\chi (\chi_0,\tau_0)\label{eq:xi-0-bun}\\
%\Omega_0 &\defeq   2  \kappa_\tau (\chi_0,\tau_0)\label{eq:Omega-0-bun}
%\end{align}
%are all real-valued and constant in $(\Delta x, \Delta t)$. On the other hand, Taylor expansion of $B(\chi,\tau)$ in \eqref{eq:q-0-bun} around the same point $(\chi_0,\tau_0)$ gives
%\begin{equation}
%%B(\chi,\tau) = B(\chi_0,\tau_0) + o(n^{-2}),\quad n\to +\infty.
%B(\chi,\tau) = B(\chi_0,\tau_0) + O(M^{-1}),\quad M\to +\infty.
%\label{eq:B-Taylor-bun}
%\end{equation}
%Combining \eqref{eq:Omega-0-Taylor-bun} and \eqref{eq:B-Taylor-bun} in \eqref{eq:q-0-bun} yields the expansion
%\begin{equation}
%%q_k^{(0)}(x,t) = -\ii \ee^{\ii \beta_0} B(\chi_0,\tau_0)  \ee^{\ii(\xi_0 \Delta x - \Omega_0 \Delta t)} + o(n^{-2}),\quad n\to +\infty,
%%q_k^{(0)}(x,t) = -\ii \ee^{\ii \beta_0} B(\chi_0,\tau_0)  \ee^{\ii(\xi_0 \Delta x - \Omega_0 \Delta t)} + O(M^{-1}),\quad M\to +\infty,
%q^{(0)}(x,t) = -\ii \ee^{\ii \beta_0} B(\chi_0,\tau_0)  \ee^{\ii(\xi_0 \Delta x - \Omega_0 \Delta t)} + O(M^{-1}),\quad M\to +\infty,
%\label{eq:Q-0-expand}
%\end{equation}
%where we have set $\ee^{\ii\beta_0}\defeq  s \ee^{-2\ii \Theta_0(\chi_0,\tau_0;M)}$ and $\beta_0$ is real since $s=\pm 1$ and $\Theta_0(\chi_0,\tau_0;M)$ is real.
%Going forward, we denote the plane wave in the leading-order term above by
%\begin{equation}
%%Q(\Delta x, \Delta t)\defeq  -\ii \ee^{\ii \beta_0} B(\chi_0,\tau_0)  \ee^{\ii(\xi_0 \Delta x - \Omega_0 \Delta t)},
%Q(\Delta x, \Delta t)\defeq  -\ii \ee^{\ii \beta_0} B(\chi_0,\tau_0)  \ee^{\ii(\xi_0 \Delta x - \Omega_0 \Delta t)},
%\label{eq:Q0-bun}
%\end{equation}
%where the amplitude factor $B(\chi_0,\tau_0)$ is real.
%\begin{proposition}
%$Q(\Delta x, \Delta t)$ defined in \eqref{eq:Q-bun} is a plane wave solution of the focusing nonlinear Schr\"odinger equation
%\begin{equation}
%\ii q_{\Delta t}(\Delta x, \Delta t) + \frac{1}{2} q_{\Delta x \Delta x} (\Delta x, \Delta t) + |q(\Delta x, \Delta t)|^2 q(\Delta x, \Delta t) = 0.
%\label{eq:nls-Delta-bun}
%\end{equation}
%\label{p:leading-order-plane-wave}
%\end{proposition}
%\begin{proposition}
%$Q(\Delta x, \Delta t)$ defined in \eqref{eq:Q0-bun} is a plane wave solution of the focusing nonlinear Schr\"odinger equation
%\begin{equation}
%\ii Q_{\Delta t}(\Delta x, \Delta t) + \frac{1}{2} Q_{\Delta x \Delta x} (\Delta x, \Delta t) + |Q(\Delta x, \Delta t)|^2 Q(\Delta x, \Delta t) = 0.
%\label{eq:nls-Delta-bun}
%\end{equation}
%\label{p:leading-order-plane-wave}
%\end{proposition}
%Before we give the proof of this proposition, we have the following lemma concerning the functions $g_{\chi}(\lambda;\chi,\tau)$ and $g_{\tau}(\lambda;\chi,\tau)$.
\begin{lemma}
The partial derivatives $g_{\chi}(\lambda;\chi,\tau)$ and $g_{\tau}(\lambda;\chi,\tau)$ are given for $(\chi,\tau)\in \shelves$ by
\begin{align}
%g_{\chi}(\lambda;\chi,\tau) &= \ii (A(\chi,\tau) - \lambda) + \ii R(\lambda;\chi,\tau),\label{eq:g-chi}\\
g_{\chi}(\lambda;\chi,\tau) &=  A(\chi,\tau) - \lambda + R(\lambda;\chi,\tau),\label{eq:g-chi}\\
%g_{\tau}(\lambda;\chi,\tau) &= \ii \left( A(\chi,\tau)^2 - \frac{1}{2}B(\chi,\tau)^2 - \lambda^2\right) + \ii(A(\chi,\tau) + \lambda) R(\lambda;\chi,\tau),\label{eq:g-tau}.
g_{\tau}(\lambda;\chi,\tau) &= A(\chi,\tau)^2 - \frac{1}{2}B(\chi,\tau)^2 - \lambda^2 + (A(\chi,\tau) + \lambda) R(\lambda;\chi,\tau)\label{eq:g-tau}.
\end{align}
Also, the partial derivatives $\kappa_\chi(\chi,\tau)$ and $\kappa_\tau(\chi,\tau)$ are given by
\begin{align}
\kappa_\chi(\chi,\tau) &= A(\chi,\tau)\label{eq:kappa-chi},\\
\kappa_\tau(\chi,\tau) &= A(\chi,\tau)^2 - \frac{1}{2} B(\chi,\tau)^2\label{eq:kappa-tau}.
\end{align}
\label{lemma:g-derivatives}
\end{lemma}
\begin{proof} 
As $\lambda_0(\chi,\tau)$ is a real analytic function of $\chi$ and $\tau$ for $(\chi,\tau)\in \shelves$, it follows from Morera's theorem that $g_\chi(\lambda;\chi,\tau)$ and $g_\tau(\lambda;\chi,\tau)$ are functions that are analytic for $\lambda\in \mathbb{C}\setminus \Sigma_g$. 
Recall the definition of $\vartheta(\lambda;\chi,\tau)$ from \eqref{eq:vartheta}, and also recall that $g(\lambda;\chi,\tau)$ behaves like the sum of a constant and the product of $(\lambda-\lambda_0)^{\frac{3}{2}}$ with an analytic function in a neighborhood of $\lambda_0$ in $\mathbb{C}\setminus \Sigma_g$ (with the same behavior near $\lambda_0^*$ by symmetry). 
Now it is seen from \eqref{eq:hpm-kappa} that $g_\chi$ can be obtained as the function analytic for $\lambda\in\mathbb{C}\setminus\Sigma_g$ satisfying the jump condition
\begin{equation}
%m_+(\lambda) + m_-(\lambda) = 2\ii \kappa_\chi(\chi,\tau) - 2\ii\theta_\chi(\lambda;\chi,\tau),\quad \lambda\in\mathbb{C}\setminus\Sigma_g,
g_{\chi+}(\lambda;\chi,\tau) + g_{\chi-}(\lambda;\chi,\tau) = 2 \kappa_\chi(\chi,\tau) - 2\vartheta_\chi(\lambda;\chi,\tau),\quad \lambda\in\Sigma_g,
\label{eq:g-chi-jump}
\end{equation}
that is bounded at the endpoints $\lambda=\lambda_0(\chi,\tau), \lambda_0(\chi,\tau)^*$ and is normalized as $g_\chi(\lambda;\chi,\tau)=O(\lambda^{-1})$ as $\lambda\to\infty$. Similarly, $g_\tau$ can be obtained as the function analytic in the same domain satisfying the jump condition
\begin{equation}
%m_+(\lambda) + m_-(\lambda) = 2\ii \kappa_\tau(\chi,\tau) - 2\ii\theta_\tau(\lambda;\chi,\tau),\quad \lambda\in\mathbb{C}\setminus\Sigma_g,
%\label{eq:g-tau-jump}
g_{\tau+}(\lambda;\chi,\tau) + g_{\tau-}(\lambda;\chi,\tau) = 2 \kappa_\tau(\chi,\tau) - 2 \vartheta_\tau(\lambda;\chi,\tau),\quad \lambda\in\Sigma_g,
\label{eq:g-tau-jump}
\end{equation}
and that is again bounded at the endpoints and normalized as $g_\tau(\lambda;\chi,\tau)=O(\lambda^{-1})$ as $\lambda\to\infty$. 
It is easy to see that the unique functions satisfying the analyticity, jump conditions, and boundedness conditions alone are
\begin{align}
g_\chi (\lambda;\chi,\tau) &=  \kappa_{\chi}(\chi,\tau)-\lambda + R(\lambda;\chi,\tau),\quad\lambda\in\mathbb{C}\setminus\Sigma_g,\label{eq:g-chi-integrated}\\
g_\tau(\lambda;\chi,\tau) &=  \kappa_{\tau}(\chi,\tau) - \lambda^{2}  + ( A(\chi,\tau) + \lambda ) R(\lambda;\chi,\tau),\quad\lambda\in\mathbb{C}\setminus\Sigma_g,\label{eq:g-tau-integrated}
\end{align}
for instance by writing $g_\chi$ and $g_\tau$ as $R(\lambda;\chi,\tau)$ times an unknown function in \eqref{eq:g-chi-jump} and \eqref{eq:g-tau-jump} and solving the resulting jump conditions for the new unknowns by a Cauchy integral that can be evaluated by residues. Enforcing the heretofore neglected normalization conditions 
by using the expansion $R(\lambda;\chi,\tau) = \lambda - A(\chi,\tau) + \tfrac{1}{2}B(\chi,\tau)^2 \lambda^{-1} + O(\lambda^{-2})$
as $\lambda\to\infty$ in \eqref{eq:g-chi-integrated} and \eqref{eq:g-tau-integrated} results in the formul\ae\ \eqref{eq:kappa-chi} and \eqref{eq:kappa-tau}.
%the jump conditions by writing the derivatives as $R(\lambda;\chi,\tau)$ times an unknown function that is solved for by a Cauchy integral that can be evaluated by residues leading to the final formulae.
%\begin{equation}
%g_\chi (\lambda;\chi,\tau) = \frac{R(\lambda;\chi,\tau)}{2\pi \ii} \int_{\Sigma_g} \frac{2\ii (\kappa_{\chi}(\chi,\tau) - \eta)}{R_+(\eta;\chi,\tau)(\eta-\lambda)}\dd \eta
%\end{equation}
%\begin{equation}
%g_\chi (\lambda;\chi,\tau) = \frac{R(\lambda;\chi,\tau)}{2\pi \ii} \int_{\Sigma_g} \frac{2 (\kappa_{\chi}(\chi,\tau) - \eta)}{R_+(\eta;\chi,\tau)(\eta-\lambda)}\dd \eta
%\end{equation}
%satisfies \eqref{eq:g-chi-jump} and defines an analytic function in the complement of $\Sigma_g$ that is bounded at the endpoints of $\Sigma_g$, where $\kappa_\chi(\chi,\tau)$ is to be determined to ensure the normalization as $\lambda\to\infty$. Then, for $\lambda\notin \Sigma_g$ a residue calculation at $\eta=\lambda$ and at $\eta=\infty$ using the  expansion $R(\eta;\chi,\tau) = \lambda^{-1} + A(\chi,\tau)\eta^{-2} + \eta^{-2}(A(\chi,\tau)^{2} - \frac{1}{2}B(\chi,\eta)^2) + O(\eta^{-3})$ as $\eta\to\infty$ yields
%\begin{equation}
%%g_\chi (\lambda;\chi,\tau)= \ii \left(\kappa_{x}(\chi,\tau)-\lambda \right)+\ii R(\lambda;\chi,\tau).
%g_\chi (\lambda;\chi,\tau)=  \left(\kappa_{\chi}(\chi,\tau)-\lambda \right)+ R(\lambda;\chi,\tau).
%\label{eq:g-chi-integrated}
%\end{equation}
%Now requiring $g_\chi (\lambda;\chi,\tau)=O(\lambda^{-1})$ as $\lambda\to\infty$, again the expansion of $R(\lambda;\chi,\tau)$ at infinity determines $\kappa_\chi(\chi,\tau)=A(\chi,\tau)$, proving \eqref{eq:kappa-chi}, and hence \eqref{eq:g-chi} by substituting $\kappa_\chi = A$ in \eqref{eq:g-chi-integrated}. 
%
%In an analogous manner, recalling that $\theta_\tau(\lambda;\chi,\tau)=\lambda^2$, we see that 
%\begin{equation}
%%g_\tau (\lambda;\chi,\tau) = \frac{R(\lambda;\chi,\tau)}{2\pi \ii} \int_{\Sigma_g} \frac{2\ii (\kappa_{\tau}(\chi,\tau) - \eta^2)}{R_+(\eta;\chi,\tau)(\eta-\lambda)}\dd \eta
%g_\tau (\lambda;\chi,\tau) = \frac{R(\lambda;\chi,\tau)}{2\pi \ii} \int_{\Sigma_g} \frac{2 (\kappa_{\tau}(\chi,\tau) - \eta^2)}{R_+(\eta;\chi,\tau)(\eta-\lambda)}\dd \eta
%\end{equation}
%satisfies \eqref{eq:g-tau-jump} and defines an analytic function in the complement of $\Sigma_g$ that is bounded at the endpoints of $\Sigma_g$. A similar residue calculation, this time using more terms in the expansion of $R(\eta;\chi,\tau)$ as $\lambda\to \infty$ due to having a quadratic in the numerator of the integrand gives
%%\begin{equation}
%%g_\tau(\lambda;\chi,\tau) = \ii \left( \kappa_{\tau}(\chi,\tau) - \lambda^{2} \right) + \ii( A(\chi,\tau) + \lambda ) R(\lambda;\chi,\tau).
%%\label{eq:g-tau-integrated}
%%\end{equation}
%\begin{equation}
%g_\tau(\lambda;\chi,\tau) =  \kappa_{\tau}(\chi,\tau) - \lambda^{2}  + ( A(\chi,\tau) + \lambda ) R(\lambda;\chi,\tau).
%\label{eq:g-tau-integrated}
%\end{equation}
%Again demanding that $g_\tau(\lambda;\chi,\tau)  = O(\lambda^{-1})$ as $\lambda\to\infty$ determines $\kappa_\tau(\chi,\tau)$ from the $R(\lambda;\chi,\tau)$ at infinity to be $\kappa_\tau(\chi,\tau) = A(\chi,\tau)^2 - \tfrac{1}{2}B(\chi,\tau)^2$, proving the claim \eqref{eq:kappa-tau}. Substituting this in \eqref{eq:g-tau-integrated} establishes \eqref{eq:g-tau} and finishes the proof.
\end{proof}
\begin{remark}
Since $\kappa(\chi,\tau)$ is a smooth function of both variables, its first order partial derivatives with respect to $\chi$ and $\tau$ commute, which in light of the expressions \eqref{eq:kappa-chi} and \eqref{eq:kappa-tau} implies the partial differential equation
\begin{equation}
\frac{\partial A}{\partial\tau}=\frac{\partial}{\partial\chi}(A^2-\tfrac{1}{2}B^2).
\label{eq:Whitham1}
\end{equation}
Similarly, taking the coefficients $c^{(\chi)}(\chi,\tau)$ and $c^{(\tau)}(\chi,\tau)$ of the (leading) term proportional to $\lambda^{-1}$ in $g_\chi(\lambda;\chi,\tau)$ and $g_\tau(\lambda;\chi,\tau)$ respectively, we obtain the consistency relation $c^{(\chi)}_\tau=c^{(\tau)}_\chi$ which is equivalent to the partial differential equation
\begin{equation}
\frac{\partial}{\partial\tau}B^2=\frac{\partial}{\partial\chi}(2AB^2).
\label{eq:Whitham2}
\end{equation}
Under the identifications $\rho=B^2$ and $U=-2A$, the two equations \eqref{eq:Whitham1}--\eqref{eq:Whitham2} are equivalent to the dispersionless nonlinear Schr\"odinger system (or genus-zero Whitham system) written in \eqref{eq:dispersionless-NLS}.  As a $2\times 2$ quasilinear system, it can be written in Riemann invariant (diagonal) form; 
in particular, the variables $\lambda_0=A+\ii B$ and $\lambda_0^*=A-\ii B$ are Riemann invariants for this system, in terms of which it becomes
\begin{equation}
\begin{split}
\lambda_{0,\tau} + \left(-\tfrac{3}{2}\lambda_0-\tfrac{1}{2}\lambda_0^*\right)\lambda_{0,\chi}^*&=0\\
\lambda_{0,\tau}^* + \left(-\tfrac{1}{2}\lambda_0-\tfrac{3}{2}\lambda_0^*\right)\lambda_{0,\chi}^*&=0.
\end{split}
\end{equation}
Since $\kappa(\chi,\tau)$ and $\gamma(\chi,\tau)$ differ by a constant according to \eqref{eq:kappa-gamma}, this proves Corollary~\ref{cor:Whitham}.
\label{rem:Whitham}
\end{remark}
Substituting \eqref{eq:kappa-chi} and \eqref{eq:kappa-tau} in \eqref{eq:wavenumbers-intro} and \eqref{eq:frequencies-intro}, we see that
\begin{align}
\xi_0 &= -2 A(\chi_0,\tau_0)\label{eq:xi-0-explicit},\\
\Omega_0 &= 2 A(\chi_0,\tau_0)^2 - B(\chi_0,\tau_0)^2\label{eq:Omega-0-explicit}.
\end{align}
Using these expressions and noting from \eqref{eq:leading-plane-wave-intro} that $|\mathcal{A}| = B(\chi_0,\tau_0)$, it is straightforward to show that  the wavenumber $\xi_0$, the frequency $\Omega_0$, and the modulus $|\mathcal{A}|$ of the complex amplitude for $Q(\Delta x, \Delta t)$ satisfy the nonlinear dispersion relation
\begin{equation}
\Omega_0 - \tfrac{1}{2}\xi_0^2 +|\mathcal{A}|^2 = 0.
\label{eq:nls-dispersion}
\end{equation}
This proves that $Q(\Delta x, \Delta t)$ is a plane-wave solution of \eqref{eq:NLS-Deltas}.


%\begin{proof}[Proof of Proposition~\ref{p:leading-order-plane-wave}]
%As the product $-\ii \ee^{\ii\beta_0}$ in \eqref{eq:Q0-bun} has unit modulus, it suffices to show that the mapping $(\Delta x, \Delta t)\mapsto B(\chi_0,\tau_0)  \exp[{\ii(\xi_0 \Delta x - \Omega_0 \Delta t)}]$ defines a plane wave solution of \eqref{eq:nls-Delta-bun}. To this end,
%we will show that the wave number $\xi_0$, the frequency $\Omega_0$, and the real amplitude $B(\chi_0,\tau_0)$ satisfy the nonlinear dispersion relation 
%\begin{equation}
%\Omega_0 - \frac{1}{2}\xi_0^2 + B(\chi_0,\tau_0)^2 = 0.
%\label{eq:nls-dispersion}
%\end{equation}
%for the NLS equation \eqref{eq:nls-Delta-bun} in $(\Delta x, \Delta t)$ coordinates. To verify this we use the results \eqref{eq:kappa-chi} and \eqref{eq:kappa-tau} of Lemma~\ref{lemma:g-derivatives} in the definitions \eqref{eq:xi-0-bun} and \eqref{eq:Omega-0-bun} to see that
%\begin{align}
%\xi_0 &= -2 A(\chi_0,\tau_0)\label{eq:xi-0-explicit}\\
%\Omega_0 &= 2 A(\chi_0,\tau_0)^2 - B(\chi_0,\tau_0)^2\label{eq:Omega-0-explicit}.
%\end{align}
%It is then straightforward to verify that \eqref{eq:nls-dispersion} holds.
%%Verifying this requires calculation of $\kappa_\chi (\chi,\tau)$ and $\kappa_\tau (\chi,\tau)$ as they appear in the definitions \eqref{eq:xi-0-bun} and \eqref{eq:Omega-0-bun}
%%
%%
%%With the aid of \eqref{eq:kappa-chi} and \eqref{eq:kappa-tau} that was computed in Lemma~\ref{lemma:g-derivatives}
%%
%%
%%Verifying this requires calculation of $\kappa_\chi (\chi,\tau)$ and $\kappa_\tau (\chi,\tau)$ as they appear in the definitions \eqref{eq:xi-0-bun} and \eqref{eq:Omega-0-bun}. Recalling the jump condition \eqref{eq:hpm-kappa} for $g(\lambda;\chi,\tau)$ and the definition \eqref{eq:vartheta} of $\vartheta(\lambda;\chi,\tau)$, differentiation in \eqref{eq:hpm-kappa} yields the relations
%%\begin{alignat}{2}
%%g_{\chi+}(\lambda;\chi,\tau) + g_{\chi-}(\lambda;\chi,\tau) &= 2 \ii \kappa_\chi(\chi,\tau) - 2\ii \lambda,&&\quad \lambda\in\Sigma_g,\label{eq:g-chi-jump}\\
%%g_{\tau+}(\lambda;\chi,\tau) + g_{\tau-}(\lambda;\chi,\tau) &= 2 \ii \kappa_\tau(\chi,\tau) - 2\ii \lambda^2,&&\quad \lambda\in\Sigma_g.\label{eq:g-tau-jump}
%%\end{alignat}
%%A calculation mimicking the one that follows \eqref{eq:hpm-kappa} shows that
%%\begin{align}
%%\kappa_\chi(\chi,\tau) &= \frac{\ii}{\pi} \int_{\Sigma_g} \frac{\lambda}{R_+(\lambda;\chi,\tau)}\dd \lambda =  \frac{\ii}{2\pi} \oint_{C} \frac{\lambda}{R(\lambda;\chi,\tau)}\dd \lambda,\\
%%\kappa_\tau(\chi,\tau) &= \frac{\ii}{\pi} \int_{\Sigma_g} \frac{\lambda^2 }{R_+(\lambda;\chi,\tau)}\dd \lambda=  \frac{\ii}{2\pi} \oint_{C} \frac{\lambda^2 }{R(\lambda;\chi,\tau)}\dd \lambda,
%%\end{align}
%%for a clockwise-oriented loop $C$ surrounding $\Sigma_g$. Using the expansion $R(\lambda;\chi,\tau) = \lambda^{-1} + A(\chi,\tau)\lambda^{-2} + \lambda^{-2}(A(\chi,\tau)^{2} - \frac{1}{2}B(\chi,\tau)^2) + O(\lambda^{-3})$ as $\lambda\to\infty$, a residue calculation at infinity gives
%%\begin{align}
%%\kappa_\chi(\chi,\tau) &= A(\chi,\tau)\label{eq:kappa-chi}\\
%%\kappa_\tau(\chi,\tau) &= A(\chi,\tau)^2 - \frac{1}{2} B(\chi,\tau)^2\label{eq:kappa-tau},
%%\end{align}
%%and hence
%%\begin{align}
%%\xi_0 &= -2 A(\chi_0,\tau_0)\label{eq:xi-0-AB}\\
%%\Omega_0 &= 2 A(\chi_0,\tau_0)^2 - B(\chi_0,\tau_0)^2\label{eq:Omega-0-AB}.
%%\end{align}
%%It is now straightforward to verify that \eqref{eq:nls-dispersion} holds.
%\end{proof}
%This proposition along with the fact that the product $-\ii \ee^{\ii\beta_0}$ in the leading term in \eqref{eq:q-k-0-expand} has unit modulus implies that 
%the mapping $(\Delta x, \Delta t) \mapsto  -\ii (-1)^k \ee^{\ii \beta_0} B(\chi_0,\tau_0)  \exp[{\ii(\xi_0 \Delta x - \Omega_0 \Delta t)}]$ also defines a plane wave solution of \eqref{eq:nls-Delta}.

%We carry out a similar analysis for the subleading term in the asymptotic expansion for $q(x,t)=q(x,t;\mathbf{Q}^{-s},M)$ obtained from \eqref{eq:q-bun-alt}, which is
%%\begin{multline}
%%q^{(1)}_k(x,t) \defeq   s \ee^{-2 \ii \Theta_0(\chi,\tau;M)} \left( F_a^+(\chi,\tau) \ee^{\ii \Theta_a(\chi,\tau;M)} +F_a^-(\chi,\tau) \ee^{-\ii \Theta_a(\chi,\tau;M)}\right.\\
%%\left. + F_b^+(\chi,\tau) \ee^{\ii \Theta_b(\chi,\tau;M)} + F_b^-(\chi,\tau) \ee^{-\ii \Theta_b(\chi,\tau;M)} \right)
%%\end{multline}
%\begin{multline}
%q^{(1)}(x,t) \defeq   s \ee^{-2 \ii \Theta_0(\chi,\tau;M)} \left( F_a^+(\chi,\tau) \ee^{\ii \Theta_a(\chi,\tau;M)} +F_a^-(\chi,\tau) \ee^{-\ii \Theta_a(\chi,\tau;M)}\right.\\
%\left. + F_b^+(\chi,\tau) \ee^{\ii \Theta_b(\chi,\tau;M)} + F_b^-(\chi,\tau) \ee^{-\ii \Theta_b(\chi,\tau;M)} \right).
%\end{multline}
%Recalling that $\Delta \chi, \Delta \tau = O(M^{-1})$ for $M\gg 1$ and $\chi = x/M$, $\tau = t/M$, a Taylor expansion completely analogous to what we have done above in \eqref{eq:Omega-0-expand-1}-\eqref{eq:Omega-0-expand-2} for $\Theta_0(\chi,\tau;M)$ around the same fixed point $(\chi_0, \tau_0)$ yields
%\begin{equation}
%%\Theta_a (\chi,\tau) = \Theta_a (\chi_0, \tau_0) + 2\left( \kappa_\chi(\chi_0, \tau_0) - 2\tilde{h}_{a\chi}(\chi_0,\tau_0) \right)\Delta x +  + 2\left( \kappa_\tau(\chi_0, \tau_0) - 2\tilde{h}_{a\tau}(\chi_0,\tau_0) \right)\Delta t + o(n^{-2})
%%\Theta_a (\chi,\tau) = \Theta_a (\chi_0, \tau_0) + \Phi_{a\chi}(\chi_0,\tau_0)\Delta x +   \Phi_{a\tau}(\chi_0,\tau_0) \Delta t + o(n^{-2}), \quad n\to +\infty.
%\Theta_a (\chi,\tau;M) = \Theta_a (\chi_0, \tau_0;M) + \Phi_{a\chi}(\chi_0,\tau_0)\Delta x +   \Phi_{a\tau}(\chi_0,\tau_0) \Delta t + O(M^{-1}), \quad M\to +\infty.
%\label{eq:Theta-a-expand}
%%2 n \left( \kappa(\chi_0, \tau_0) - 2\tilde{h}_a(\chi_0,\tau_0) )\right) - \ln(n) p + \eta_a(\chi_0, \tau_0) 
%\end{equation}
%This implies
%\begin{equation}
%%\ee^{\pm \ii \Theta_a(\chi,\tau;M)} = \ee^{\pm \ii \beta_a} \ee^{\pm \ii (\xi_a \Delta x - \Omega_a \Delta t)} + o(n^{-2}),\quad n\to +\infty,
%\ee^{\pm \ii \Theta_a(\chi,\tau;M)} = \ee^{\pm \ii \beta_a} \ee^{\pm \ii (\xi_a \Delta x - \Omega_a \Delta t)} + O(M^{-1}),\quad M\to +\infty,
%\end{equation}
%where
%\begin{align}
%\beta_a &\defeq  \Theta_a(\chi_0,\tau_0;M) = M\Phi_a(\chi_0,\tau_0) - \ln(M) p + \eta_a(\chi_0,\tau_0),\\
%\xi_a &\defeq   \Phi_{a\chi}(\chi_0,\tau_0) = 2(\kappa_\chi(\chi_0, \tau_0) - {h}_{a\chi}(\chi_0,\tau_0)),\label{eq:xi-a-bun}\\
%\Omega_a &\defeq   - \Phi_{a\tau}(\chi_0,\tau_0) = - 2(\kappa_\tau(\chi_0, \tau_0) - {h}_{a\tau}(\chi_0,\tau_0)).\label{eq:Omega-a-bun}
%\end{align}
%Here the derivatives $ {h}_{a\chi}(\chi_0,\tau_0)$ and $ {h}_{a\tau}(\chi_0,\tau_0)$ are meant to denote the quantities
%\begin{align}
%{h}_{a\chi}(\chi_0,\tau_0) &=\left. \left[ \frac{\partial}{\partial \chi}{h}_R(a(\chi, \tau); \chi,\tau)\right]\right \vert_{(\chi,\tau)=(\chi_,\tau_0)},\\
%{h}_{a\tau}(\chi_0,\tau_0) &=\left. \left[ \frac{\partial}{\partial \tau}{h}_R(a(\chi, \tau); \chi,\tau)\right]\right \vert_{(\chi,\tau)=(\chi_,\tau_0)}.
%\end{align}
%As a matter of fact, the use of this notation does not give rise to ambiguity. Indeed,
%\begin{equation}
%\begin{split}
%\frac{\partial}{\partial \chi}{h}_R(a(\chi, \tau); \chi,\tau) &= {h}'_R((a(\chi, \tau); \chi,\tau))\cdot \left( \frac{\partial}{\partial \chi} a(\chi,\tau)\right)+ \left. \frac{\partial}{\partial \chi}{h}_R(\lambda; \chi,\tau))\right \rvert_{\lambda = a(\chi,\tau)}\\
%&=\left. \frac{\partial}{\partial \chi}{h}_R(\lambda; \chi,\tau))\right \rvert_{\lambda = a(\chi,\tau)},
%\end{split}
%\label{eq:h-chi-derivative-chain-rule}
%\end{equation}
%because $\lambda=a(\chi,\tau)$ is a (simple) root of ${h}_R'(\lambda;\chi,\tau)$ for all $(\chi,\tau)\in \shelves$. Thus,
%\begin{equation}
%{h}_{a\chi}(\chi_0,\tau_0) = {h}_{\chi-}(a(\chi_0, \tau_0); \chi_0,\tau_0),
%\label{eq:h-chi-derivative-a}
%\end{equation}
%and similarly,
%\begin{equation}
%{h}_{a\tau}(\chi_0,\tau_0) = {h}_{\tau-}(a(\chi_0, \tau_0); \chi_0,\tau_0).
%\label{eq:h-tau-derivative-a}
%\end{equation}
%In the same manner as in \eqref{eq:Theta-a-expand} we obtain
%\begin{equation}
%\ee^{\pm \ii \Theta_b(\chi,\tau;M)} =  \ee^{\pm \ii \beta_b} \ee^{\pm \ii (\xi_b \Delta x - \Omega_b \Delta t)} + O(M^{-1}),\quad M\to +\infty,
%\end{equation}
%where
%\begin{align}
%\beta_b &\defeq  \Theta_b(\chi_0,\tau_0;M) = M\Phi_b(\chi_0,\tau_0) + \ln(M) p + \eta_b(\chi_0,\tau_0),,\\
%\xi_b &\defeq   \Phi_{b\chi}(\chi_0,\tau_0) = 2(\kappa_\chi(\chi_0, \tau_0) - {h}_{b\chi}(\chi_0,\tau_0)),\label{eq:xi-b-bun}\\
%\Omega_b &\defeq   - \Phi_{b\tau}(\chi_0,\tau_0) = - 2(\kappa_\tau(\chi_0, \tau_0) - {h}_{b\tau}(\chi_0,\tau_0)),\label{eq:Omega-b-bun}
%\end{align}
%in which the derivatives $ {h}_{b\chi}(\chi_0,\tau_0)$ and $ {h}_{b\tau}(\chi_0,\tau_0)$ as before are meant to denote the quantities
%\begin{align}
%{h}_{b\chi}(\chi_0,\tau_0) &=\left. \left[ \frac{\partial}{\partial \chi}{h}(b(\chi, \tau); \chi,\tau)\right]\right \vert_{(\chi,\tau)=(\chi_,\tau_0)},\\
%{h}_{b\tau}(\chi_0,\tau_0) &=\left. \left[ \frac{\partial}{\partial \tau}{h}(b(\chi, \tau); \chi,\tau)\right]\right \vert_{(\chi,\tau)=(\chi_,\tau_0)},
%\end{align}
%and a calculation analogous to \eqref{eq:h-chi-derivative-chain-rule} this time using the fact that $\lambda=b(\chi,\tau)$ is a (simple) root of ${h}'(\lambda;\chi,\tau)$ for all $(\chi,\tau)\in \shelves$ yields
%\begin{align}
%{h}_{b\chi}(\chi_0,\tau_0) &= {h}_{\chi}(b(\chi_0, \tau_0); \chi_0,\tau_0),\label{eq:h-chi-derivative-b}\\
%{h}_{b\tau}(\chi_0,\tau_0) &= {h}_{\tau}(b(\chi_0, \tau_0); \chi_0,\tau_0).
%\label{eq:h-tau-derivative-b}
%\end{align}
%On the other hand, for the amplitude terms in \eqref{eq:q-bun}, the analogous Taylor expansion gives
%\begin{align}
%F_a^{\pm}(\chi,\tau) &= F_a^{\pm}(\chi_0,\tau_0) + o(M^{-1}),\quad M\to+\infty,\\
%F_b^{\pm}(\chi,\tau) &= F_b^{\pm}(\chi_0,\tau_0) + o(M^{-1}),\quad M\to+\infty.
%\end{align}
%Thus, we have arrived at the expansion
%%\begin{multline}
%%%q_k^{(1)}(x,t) = \ee^{\ii\beta_0} \ee^{\ii(\xi_0 \Delta x - \Omega_0 \Delta t)}
%%%\left( \ee^{\ii \beta_a} F_a^+ (\chi_0,\tau_0) \ee^{\ii(\xi_a \Delta x - \Omega_a \Delta t)} +  \ee^{-\ii\beta_a} F_a^- (\chi_0,\tau_0) \ee^{-\ii(\xi_a \Delta x - \Omega_a \Delta t)} \right.\\
%%%\left. \ee^{\ii \beta_b} F_b^+ (\chi_0,\tau_0) \ee^{\ii(\xi_b \Delta x - \Omega_b \Delta t)} +  \ee^{-\ii \beta_b} F_b^- (\chi_0,\tau_0) \ee^{-\ii(\xi_b \Delta x - \Omega_b \Delta t)} 
%%%\right) + o(n^{-2}),\quad n\to +\infty.
%%q_k^{(1)}(x,t) = \ee^{\ii\beta_0} \ee^{\ii(\xi_0 \Delta x - \Omega_0 \Delta t)}
%%\left( \ee^{\ii \beta_a} F_a^+ (\chi_0,\tau_0) \ee^{\ii(\xi_a \Delta x - \Omega_a \Delta t)} +  \ee^{-\ii\beta_a} F_a^- (\chi_0,\tau_0) \ee^{-\ii(\xi_a \Delta x - \Omega_a \Delta t)} \right.\\
%%\left. \ee^{\ii \beta_b} F_b^+ (\chi_0,\tau_0) \ee^{\ii(\xi_b \Delta x - \Omega_b \Delta t)} +  \ee^{-\ii \beta_b} F_b^- (\chi_0,\tau_0) \ee^{-\ii(\xi_b \Delta x - \Omega_b \Delta t)} 
%%\right) \\
%%+ O(M^{-1}),\quad M\to +\infty.
%%\label{eq:q-k-1expand}
%%\end{multline}
%\begin{multline}
%q^{(1)}(x,t) = \ee^{\ii\beta_0} \ee^{\ii(\xi_0 \Delta x - \Omega_0 \Delta t)}
%\left( \ee^{\ii \beta_a} F_a^+ (\chi_0,\tau_0) \ee^{\ii(\xi_a \Delta x - \Omega_a \Delta t)} +  \ee^{-\ii\beta_a} F_a^- (\chi_0,\tau_0) \ee^{-\ii(\xi_a \Delta x - \Omega_a \Delta t)} \right.\\
%\left. \ee^{\ii \beta_b} F_b^+ (\chi_0,\tau_0) \ee^{\ii(\xi_b \Delta x - \Omega_b \Delta t)} +  \ee^{-\ii \beta_b} F_b^- (\chi_0,\tau_0) \ee^{-\ii(\xi_b \Delta x - \Omega_b \Delta t)} 
%\right) \\
%+ O(M^{-1}),\quad M\to +\infty.
%\label{eq:Q-1expand}
%\end{multline}
%Combining \eqref{eq:Q-0-expand} and \eqref{eq:Q-1expand} and factoring out the plane wave $Q(\Delta x, \Delta t)$ lets us obtain from \eqref{eq:q-bun-alt} the expansion
%%\begin{equation}
%%q_k(x,t) = Q(\Delta x, \Delta t) \left( 1 + M^{-\frac{1}{2}} \left (p_a(\Delta x, \Delta t)+ p_b(\Delta x, \Delta t) \right) \right) + O(M^{-1}),\quad M\to +\infty,
%%\label{eq:q-k-bun-perturbation}
%%\end{equation} 
%\begin{equation}
%q(x,t) = Q(\Delta x, \Delta t) \left( 1 + M^{-\frac{1}{2}} \left (p_a(\Delta x, \Delta t)+ p_b(\Delta x, \Delta t) \right) \right) + O(M^{-1}),\quad M\to +\infty,
%\label{eq:Q-bun-perturbation}
%\end{equation} 
%where we have set
%\begin{align}
%p_a(\Delta x, \Delta t) &\defeq    \frac{\ii F_a^+(\chi_0, \tau_0)}{B(\chi_0, \tau_0)} \ee^{\ii \beta_a} \ee^{\ii (\xi_a \Delta x - \Omega_a \Delta t)} + \frac{\ii F_a^-(\chi_0, \tau_0)}{B(\chi_0, \tau_0)}\ee^{-\ii \beta_a} \ee^{-\ii (\xi_a \Delta x - \Omega_a \Delta t)},\label{eq:p-a-bun}\\
%p_b(\Delta x, \Delta t) &\defeq  \frac{\ii F_b^+(\chi_0, \tau_0)}{B(\chi_0, \tau_0)}\ee^{\ii \beta_b} \ee^{\ii (\xi_b \Delta x - \Omega_b \Delta t)} + \frac{\ii F_b^-(\chi_0, \tau_0)}{B(\chi_0, \tau_0)}\ee^{-\ii \beta_b} \ee^{-\ii (\xi_b \Delta x - \Omega_b \Delta t)},\label{eq:p-b-bun}
%\end{align}
%and $B(\chi_0,\tau_0)\neq 0$ for any $(\chi_0,\tau_0)\in \shelves$. 
We will now prove the claim that each of the functions $p_{a}(\Delta x, \Delta t)$ and $p_b(\Delta x, \Delta t)$ in the expansion \eqref{eq:Q-perturbation-shelves} defines a solution of the linearization \eqref{eq:linearization-intro} of \eqref{eq:NLS-Deltas} about the plane-wave solution $Q(\Delta x,\Delta t)$.
Observe that the expansion \eqref{eq:Q-perturbation-shelves} is of the form \eqref{eq:q-perturb-appendix} in the treatment given in Appendix \ref{A:perturbations} and hence gives a relative perturbation expansion of $Q(\Delta x,\Delta t)$ for $M\gg 1$. We let $r_{a,b}(\Delta x, \Delta t)$ and $s_{a,b}(\Delta x, \Delta t)$ denote the real and imaginary parts of $p_{a,b}(\Delta x, \Delta t)$, respectively. For convenience and brevity in the calculations to come, we set
%\begin{equation}
%Z_a^{+}  \defeq  F_a^{[\shelves]}(\chi_0,\tau_0) m_a^+(\chi_0,\tau_0),\qquad
%Z_a^{-}  \defeq   - F_a^{[\shelves]}(\chi_0,\tau_0) m_a^-(\chi_0,\tau_0),
%\label{eq:Z-a}
%\end{equation}
%and
%\begin{equation}
%Z_b^{+}  \defeq  F_b^{[\shelves]}(\chi_0,\tau_0) m_b^+(\chi_0,\tau_0),\qquad
%Z_b^{-}  \defeq   - F_b^{[\shelves]}(\chi_0,\tau_0) m_b^-(\chi_0,\tau_0).
%\label{eq:Z-b}
%\end{equation}
\begin{equation}
Z_{a,b}^\pm \defeq \pm F_{a,b}^{[\shelves]}(\chi_0,\tau_0)m_{a,b}^\pm(\chi_0,\tau_0).
\label{eq:Z-a-b}
\end{equation}
Then $r_{a,b}(\Delta x, \Delta t)$ and $s_{a,b}(\Delta x, \Delta t)$ are expressed in terms of these quantities as
\begin{align}
r_{a,b}(\Delta x,\Delta t) &= \frac{Z_{a,b}^- - Z_{a,b}^+}{B(\chi_0,\tau_0)}\sin\left(\phi_{a,b}(\chi_0,\tau_0;M)+\xi_{a,b}\Delta x -\Omega_{a,b}\Delta t\right),\\
s_{a,b}(\Delta x,\Delta t) &= \frac{Z_{a,b}^- + Z_{a,b}^+}{B(\chi_0,\tau_0)}\cos\left(\phi_{a,b}(\chi_0,\tau_0;M)+\xi_{a,b}\Delta x -\Omega_{a,b}\Delta t\right).
\end{align}
%\begin{align}
%r_{a}(\Delta x, \Delta t)&= \left(\frac{Z_a^{-}-Z_a^{+}}{B(\chi_0,\tau_0)}\right) \sin \left(\phi_a(\chi_0,\tau_0;M) + \xi_{a} \Delta x-\Omega_{a} \Delta t\right),\\
%s_{a}(\Delta x, \Delta t)&= \left(\frac{Z_a^{-}+Z_a^{+}}{B(\chi_0,\tau_0)}\right) \cos \left(\phi_a(\chi_0,\tau_0;M)+\xi_{a} \Delta x-\Omega_{a} \Delta t\right),
%\end{align}
%and
%\begin{align}
%r_{b}(\Delta x, \Delta t)&= \left(\frac{Z_b^{-}-Z_b^{+}}{B(\chi_0,\tau_0)}\right) \sin \left(\phi_b(\chi_0,\tau_0;M)+\xi_{b} \Delta x-\Omega_{b} \Delta t\right),\\
%s_{b}(\Delta x, \Delta t)&\defeq \left(\frac{Z_b^{-} + Z_b^{+}}{B(\chi_0,\tau_0)}\right) \cos \left(\phi_b(\chi_0,\tau_0;M)+\xi_{b} \Delta x-\Omega_{b} \Delta t\right).
%\end{align}
In view of Appendix~\ref{A:perturbations}, to prove that $p_a(\Delta x, \Delta t)$ and $p_b(\Delta x,\Delta t)$ solve \eqref{eq:linearization-intro},
it suffices to show that the pairs $(r_{a}(\Delta x, \Delta t), s_{a}(\Delta x, \Delta t))$ and $(r_{b}(\Delta x, \Delta t), s_{b}(\Delta x, \Delta t))$ satisfy \eqref{eq:linearized-NLS-r-s} (written in the variables $(\Delta x,\Delta t)$ instead of $(x,t)$). Suppressing the dependencies on the fixed point $(\chi_0,\tau_0)$ for brevity, it is easy to see using $|\mathcal{A}|^2=B^2=B(\chi_0,\tau_0)^2$ that $(r_{a,b}(\Delta x, \Delta t), s_{a,b}(\Delta x, \Delta t))$ satisfies \eqref{eq:linearized-NLS-r-s} if and only if 
\begin{align}
\left( \xi_{a,b}^2 + 2(\xi_0 \xi_{a,b} - \Omega_{a,b}) \right)Z_{a,b}^+ +\left( \xi_{a,b}^2 - 2(\xi_0 \xi_{a,b} - \Omega_{a,b}) \right) Z_{a,b}^- &= 0,\label{eq:F-a-b-pm-sys-1}\\%\label{eq:A-pm-sys-1}\\
\left( \xi_{a,b}^2 + 2(\xi_0 \xi_{a,b} - \Omega_{a,b})  - 4 B^2 \right) Z_{a,b}^+ +\left( - \xi_{a,b}^2 + 2(\xi_0 \xi_{a,b} - \Omega_{a,b})  + 4 B^2 \right)Z_{a,b}^- &= 0.\label{eq:F-a-b-pm-sys-2}%\label{eq:A-pm-sys-2},
\end{align}
%\begin{align}
%\left( \xi_a^2 + 2(\xi_0 \xi_a - \Omega_a) \right)Z_a^+ +\left( \xi_a^2 - 2(\xi_0 \xi_a - \Omega_a) \right) Z_a^- &= 0,\label{eq:F-a-pm-sys-1}\\%\label{eq:A-pm-sys-1}\\
%\left( \xi_a^2 + 2(\xi_0 \xi_a - \Omega_a)  - 4 B^2 \right) Z_a^+ +\left( - \xi_a^2 + 2(\xi_0 \xi_a - \Omega_a)  + 4 B^2 \right)Z_a^- &= 0\label{eq:F-a-pm-sys-2},%\label{eq:A-pm-sys-2},
%\end{align}
%and similarly, the pair $(r_{b}(\Delta x, \Delta t), s_{b}(\Delta x, \Delta t))$ satisfies \eqref{eq:linearized-NLS-r-s} if and only if 
%\begin{align}
%\left( \xi_b^2 + 2(\xi_0 \xi_b - \Omega_b) \right) Z_b^+ +\left( \xi_b^2 - 2(\xi_0 \xi_b - \Omega_b) \right )Z_b^- &= 0,\label{eq:F-b-pm-sys-1}\\%\label{eq:B-pm-sys-1}\\
%\left( \xi_b^2 + 2(\xi_0 \xi_b - \Omega_b)  - 4 B^2 \right) Z_b^+ +\left( - \xi_b^2 + 2(\xi_0 \xi_b - \Omega_b)  + 4 B^2 \right ) Z_b^- &= 0.\label{eq:F-b-pm-sys-2}%\label{eq:B-pm-sys-2}
%\end{align}
Note that \eqref{eq:F-a-b-pm-sys-1}--\eqref{eq:F-a-b-pm-sys-2} 
%and \eqref{eq:F-b-pm-sys-1}--\eqref{eq:F-b-pm-sys-2} 
constitute two homogeneous systems of linear equations for $(Z_{a,b}^+, Z_{a,b}^-)$, one for each choice of subscript $a$, $b$.
%and $(Z_b^+, Z_b^-)$, respectively. 
These systems have nontrivial solutions if and only if they are singular, which amount to the conditions
%\begin{align}
%4 (\xi_0 \xi_a-\Omega_a )^2 &= \xi_a^2 \left(\xi_a^2 - 4 B(\chi_0,\tau_0)^2 \right),\label{eq:linearized-dispersion-a-bun}\\
%4 (\xi_0 \xi_b- \Omega_b )^2 &= \xi_b^2 \left(\xi_b^2 - 4 B(\chi_0,\tau_0)^2 \right).\label{eq:linearized-dispersion-b-bun}
%\end{align}
\begin{equation}
4 (\xi_0 \xi_{a,b}-\Omega_{a,b} )^2 = \xi_{a,b}^2 \left(\xi_{a,b}^2 - 4 B^2 \right).\label{eq:linearized-dispersion-a-b-bun}
\end{equation}
Again recalling that $B^2=|\mathcal{A}|^2$, these are precisely two instances of the linearized dispersion relation \eqref{eq:linearized-dispersion} to be satisfied by the pairs $(\xi_{a,b}, \Omega_{a,b})$ of relative local wavenumbers and frequencies. We will first show that the conditions \eqref{eq:linearized-dispersion-a-b-bun} hold, and then show that the pairs $(Z_{a,b}^+, Z_{a,b}^-)$
%and $(Z_{a,b}^+, Z_{a,b}^-)$ 
lie in the (nontrivial) nullspaces of the coefficient matrices for the systems \eqref{eq:F-a-b-pm-sys-1}--\eqref{eq:F-a-b-pm-sys-2}.
% and \eqref{eq:F-b-pm-sys-1}--\eqref{eq:F-b-pm-sys-2}, respectively. 
 To prove \eqref{eq:linearized-dispersion-a-b-bun},
% --\eqref{eq:linearized-dispersion-a-b-bun}, 
we refer back to Lemma~\ref{lemma:g-derivatives} and use the expression \eqref{eq:g-chi} for $g_\chi(\lambda;\chi,\tau)$ together with $\vartheta_\chi(\lambda;\chi,\tau)=\lambda$ and \eqref{eq:kappa-chi} in the definitions \eqref{eq:wavenumbers-intro} of $\xi_{a,b}$ to see that
\begin{equation}
\xi_a = -2 R_-(a(\chi_0;\tau_0);\chi_0,\tau_0)\quad \text{and}\quad
\xi_b = -2 R(b(\chi_0;\tau_0);\chi_0,\tau_0).
\label{eq:xi-a-b-explicit}
\end{equation}
Similarly, using the expression \eqref{eq:g-tau} for $g_\tau(\chi,\tau)$ together with $\vartheta_\tau(\lambda;\chi,\tau)=\lambda^2$ and \eqref{eq:kappa-chi} in the definitions \eqref{eq:frequencies-intro} of $\Omega_{a,b}$ yields
\begin{equation}
\begin{split}
\Omega_a &= 2(A(\chi_0,\tau_0) + a(\chi_0,\tau_0)) R_-(a(\chi_0;\tau_0);\chi_0,\tau_0),\\
\Omega_b &= 2(A(\chi_0,\tau_0) + b(\chi_0,\tau_0))  R(b(\chi_0;\tau_0);\chi_0,\tau_0).
\label{eq:Omega-a-b-explicit}
\end{split}
\end{equation}
Now, to show that \eqref{eq:linearized-dispersion-a-b-bun} holds, we recall the definition of $\xi_0$ in \eqref{eq:wavenumbers-intro}, and use \eqref{eq:xi-a-b-explicit} and \eqref{eq:Omega-a-b-explicit} to observe that
\begin{equation}
\begin{split}
4 (\xi_0 \xi_a-\Omega_a )^2 
&= 4\left(4 A(\chi_0,\tau_0) R(a(\chi_0,\tau_0); \chi_0, \tau_0) - 2(A(\chi_0, \tau_0)+a(\chi_0, \tau_0)) R_-(a(\chi_0, \tau_0); \chi_0, \tau_0)\right)^{2}\\
&=16 R_-(a(\chi_0, \tau_0); \chi_0, \tau_0)^2 \left(a(\chi_0, \tau_0) - A(\chi_0, \tau_0) \right)^2.%\\
%&=16\left( \left(a(\chi_0, \tau_0) - A(\chi_0, \tau_0) \right)^2 + B(\chi_0,\tau_0)^2 \right) \left(a(\chi_0, \tau_0) - A(\chi_0, \tau_0) \right)^2,
\end{split}
\label{eq:linearized-dispersion-a-LHS}
\end{equation}
Next, the right-hand side of \eqref{eq:linearized-dispersion-a-b-bun} reads
\begin{equation}
\begin{split}
\xi_a^2 \left(\xi_a^2 - 4 B(\chi_0,\tau_0)^2 \right)
&= 4 R_-(a(\chi_0,\tau_0);\chi_0,\tau_0)^2 \left(4 R_-(a(\chi_0,\tau_0);\chi_0,\tau_0)^2 - 4 B(\chi_0,\tau_0)^2 \right)\\
&= 16 R_-(a(\chi_0,\tau_0);\chi_0,\tau_0)^2 \left( a(\chi_0,\tau_0) - A(\chi_0,\tau_0) \right)^2,
\end{split}
\label{eq:linearized-dispersion-a-RHS}
\end{equation}
since $R(\lambda;\chi,\tau)^2 = (\lambda- A(\chi,\tau))^2 + B(\chi,\tau)^2$. The identities \eqref{eq:linearized-dispersion-a-LHS}--\eqref{eq:linearized-dispersion-a-RHS} prove that the linearized dispersion relation \eqref{eq:linearized-dispersion-a-b-bun} holds for $(\xi_a, \Omega_a)$. A completely analogous calculation having the point $b(\chi_0,\tau_0)$ in place of $a(\chi_0,\tau_0)$ shows that \eqref{eq:linearized-dispersion-a-b-bun} holds for $(\xi_b, \Omega_b)$.

As we have now established that the linear systems \eqref{eq:F-a-b-pm-sys-1}--\eqref{eq:F-a-b-pm-sys-2}
% and \eqref{eq:F-b-pm-sys-1}--\eqref{eq:F-b-pm-sys-2} 
 are both singular, it remains to show that the pairs of quantities $(Z_{a,b}^{+}, Z_{a,b}^{-})$
%and $(Z_b^{+}, Z_b^{-})$ 
lie in the corresponding nullspaces. 
To do so, it suffices to verify that \eqref{eq:F-a-b-pm-sys-1} 
%and \eqref{eq:F-b-pm-sys-1} 
holds. 
Using the definitions \eqref{eq:m-a-b-shelves} in \eqref{eq:Z-a-b}
%--\eqref{eq:Z-b} 
and noting that $F^{[\shelves]}_a(\chi_0,\tau_0)$ and $F^{[\shelves]}_b(\chi_0,\tau_0)$ are nonzero, it is seen that verifying \eqref{eq:F-a-b-pm-sys-1}
% and \eqref{eq:F-b-pm-sys-1} 
amounts to showing that
\begin{align}
\cos( \arg(a(\chi_0,\tau_0)-\lambda_0(\chi_0,\tau_0)) ) &= \frac{-2(\xi_0\xi_a - \Omega_a)}{\xi_a^2},\label{eq:show-cos-arg-a-bun}\\
\cos( \arg(b(\chi_0,\tau_0)-\lambda_0(\chi_0,\tau_0)) ) &= \frac{-2(\xi_0\xi_b - \Omega_b)}{\xi_b^2}.\label{eq:show-cos-arg-b-bun}
\end{align}
However, according to 
\eqref{eq:xi-0-explicit}, \eqref{eq:xi-a-b-explicit}, and \eqref{eq:Omega-a-b-explicit}, we obtain for the right-hand side of the purported identities \eqref{eq:show-cos-arg-a-bun}--\eqref{eq:show-cos-arg-b-bun} that
\begin{align}
\frac{-2(\xi_0\xi_a - \Omega_a)}{\xi_a^2}&=\frac{a(\chi_0,\tau_0) - A(\chi_0,\tau_0)}{R_-(a(\chi_0,\tau_0);\chi_0,\tau_0)},\label{eq:show-cos-arg-a-simp}\\
\frac{-2(\xi_0\xi_b - \Omega_b)}{\xi_b^2}&=\frac{b(\chi_0,\tau_0) - A(\chi_0,\tau_0)}{R(b(\chi_0,\tau_0);\chi_0,\tau_0)}.\label{eq:show-cos-arg-b-simp}
\end{align}
%The right-hand sides above can be simplified by directly using the expressions \eqref{eq:xi-0-explicit}, \eqref{eq:xi-a-explicit}-\eqref{eq:xi-b-explicit}, and \eqref{eq:Omega-a-explicit}-\eqref{eq:Omega-b-explicit}, and it is seen that showing \eqref{eq:show-cos-arg-a-bun} and \eqref{eq:show-cos-arg-b-bun} is equivalent to verifying the identities
%\begin{align}
%\cos( \arg(a(\chi_0,\tau_0)-\lambda_0(\chi_0,\tau_0)) ) &= \frac{a(\chi_0,\tau_0) - A(\chi_0,\tau_0)}{R_-(a(\chi_0,\tau_0);\chi,\tau)},\label{eq:show-cos-arg-a-simp}\\
%\cos( \arg(b(\chi_0,\tau_0)-\lambda_0(\chi_0,\tau_0)) ) &= \frac{b(\chi_0,\tau_0) - A(\chi_0,\tau_0)}{R(b(\chi_0,\tau_0);\chi,\tau)}.\label{eq:show-cos-arg-b-simp}
%\end{align}
%Recalling that $a(\chi_0,\tau_0) < \Re (\lambda_0(\chi_0,\tau_0))<b(\chi_0,\tau_0)$, it is now straightforward to see that \eqref{eq:show-cos-arg-a-simp}--\eqref{eq:show-cos-arg-b-simp} are indeed true:
%%and $\Im(\lambda_0(\chi_0,\tau_0))=B(\chi_0,\tau_0)>0$, we have
%\begin{align}
%\cos( \arg(a(\chi_0,\tau_0) - \lambda_0(\chi_0,\tau_0)) )
%&= \frac{\Re( a(\chi_0,\tau_0) -\lambda_0(\chi_0,\tau_0) )}{| \lambda_0(\chi_0,\tau_0)-a(\chi_0,\tau_0)|}=\frac{a(\chi_0,\tau_0) - A(\chi_0,\tau_0)}{R_-(a(\chi_0,\tau_0);\chi_0,\tau_0)}\\
%\cos( \arg(b(\chi_0,\tau_0) - \lambda_0(\chi_0,\tau_0)) )
%&= \frac{\Re( b(\chi_0,\tau_0) -\lambda_0(\chi_0,\tau_0) )}{| b(\chi_0,\tau_0) - \lambda_0(\chi_0,\tau_0) |}=\frac{b(\chi_0,\tau_0) - A(\chi_0,\tau_0)}{R(b(\chi_0,\tau_0);\chi_0,\tau_0)}.
%\end{align}
Since $R_-(a(\chi_0,\tau_0);\chi_0,\tau_0)$ and $R(b(\chi_0,\tau_0);\chi_0,\tau_0)$ are both positive, while $a(\chi_0,\tau_0)$ and $b(\chi_0,\tau_0)$ are real and $A(\chi_0,\tau_0)=\mathrm{Re}(\lambda_0(\chi_0,\tau_0))$, 
we see that the identities \eqref{eq:show-cos-arg-a-bun}--\eqref{eq:show-cos-arg-b-bun} indeed both hold:
\begin{align}
\frac{-2(\xi_0\xi_a - \Omega_a)}{\xi_a^2}&=\frac{\mathrm{Re}(a(\chi_0,\tau_0)-\lambda_0(\chi_0,\tau_0))}{|a(\chi_0,\tau_0)-\lambda_0(\chi_0,\tau_0)|}=\cos(\arg(a(\chi_0,\tau_0)-\lambda_0(\chi_0,\tau_0))),\\
\frac{-2(\xi_0\xi_b - \Omega_b)}{\xi_b^2}&=\frac{\mathrm{Re}(b(\chi_0,\tau_0)-\lambda_0(\chi_0,\tau_0))}{|b(\chi_0,\tau_0)-\lambda_0(\chi_0,\tau_0)|}=\cos(\arg(b(\chi_0,\tau_0)-\lambda_0(\chi_0,\tau_0))).
\end{align}
Thus, we have shown that $(Z_a^+,Z_a^-)$ and $(Z_b^+,Z_b^-)$ are non-trivial solutions of the linear homogeneous systems \eqref{eq:F-a-b-pm-sys-1}--\eqref{eq:F-a-b-pm-sys-2}.
%  and \eqref{eq:F-b-pm-sys-1}--\eqref{eq:F-b-pm-sys-2}, respectively. 
This implies that  $p_a(\Delta x, \Delta t)$ and  $p_b(\Delta x, \Delta t)$ are solutions of \eqref{eq:linearization-intro}.
% $p_a(\Delta x, \Delta t)$ is a solution of \eqref{eq:linearized-NLS-Delta-bun}.
% establishes that $p_b(\Delta x, \Delta t)$ is a solution of \eqref{eq:linearized-NLS-Delta-bun}.
%our claim is that each of the subdominant functions $p_{a}(\Delta x, \Delta t)$ and $p_b(\Delta x, \Delta t)$ in the expansion above define a solution of the linearization of the NLS equation about the plane wave $Q(x,t)$.
%\begin{proposition}
%Each of the quantities $p_a(\Delta x, \Delta t)$, and $p_b(\Delta x, \Delta t)$ defines a solution of the linearization
%\begin{equation}
%\ii p_{\Delta t}(\Delta x,\Delta t) + \ii \xi_0 p_{\Delta x}(\Delta x,\Delta t) + \frac{1}{2} p_{\Delta x \Delta x}(\Delta x,\Delta t)+ B(\chi_0,\tau_0)^2 (p(\Delta x,\Delta t) + p(\Delta x,\Delta t)^*)=0
%\label{eq:linearized-NLS-Delta-bun}
%\end{equation}
%of the nonlinear Schr\"odinger equation \eqref{eq:nls-Delta-bun} about the plane wave $Q(\Delta x,\Delta t)$ given in \eqref{eq:Q0-bun}, in which we have substituted $|Q(\chi_0,\tau_0)| = |-\ii \ee^{\ii\beta_0}B(\chi_0,\tau_0) | =B(\chi_0,\tau_0)$.
%\end{proposition}
%\begin{proof}
%We let $r_{a,b}(\Delta x, \Delta t)$ and $s_{a,b}(\Delta x, \Delta t)$ denote the real and imaginary parts of $p_{a,b}(\Delta x, \Delta t)$, respectively. Explicitly,
%\begin{align}
%r_{a}(\Delta x, \Delta t)&\defeq \left(\frac{F_a^{-}(\chi_0,\tau_0)-F_a^{+}(\chi_0,\tau_0)}{B(\chi_0,\tau_0)}\right) \sin \left(\beta_a+\xi_{a} \Delta x-\Omega_{a} \Delta t\right),\\
%s_{a}(\Delta x, \Delta t)&\defeq \left(\frac{F_a^{-}(\chi_0,\tau_0)+F_a^{+}(\chi_0,\tau_0)}{B(\chi_0,\tau_0)}\right) \cos \left(\beta_a+\xi_{a} \Delta x-\Omega_{a} \Delta t\right),
%\end{align}
%and
%\begin{align}
%r_{b}(\Delta x, \Delta t)&\defeq \left(\frac{F_b^{-}(\chi_0,\tau_0)-F_b^{+}(\chi_0,\tau_0)}{B(\chi_0,\tau_0)}\right) \sin \left(\beta_b+\xi_{b} \Delta x-\Omega_{b} \Delta t\right),\\
%s_{b}(\Delta x, \Delta t)&\defeq \left(\frac{F_b^{-}(\chi_0,\tau_0)+F_b^{+}(\chi_0,\tau_0)}{B(\chi_0,\tau_0)}\right) \cos \left(\beta_b+\xi_{b} \Delta x-\Omega_{b} \Delta t\right).
%\end{align}
%In view of Appendix \ref{A:perturbations}, to prove that $p_{a,b}$ solve \eqref{eq:linearized-NLS-Delta-bun}, it suffices to show that the pairs $(r_{a}, s_{a})$ and $(r_{b}, s_{b})$ satisfy \eqref{eq:linearized-NLS-r-s}. $(r_{a}, s_{a})$ satisfies \eqref{eq:linearized-NLS-r-s} if and only if 
%\begin{align}
%\left( \xi_a^2 + 2(\xi_0 \xi_a - \Omega_a) \right)F_a^+ +\left( \xi_a^2 - 2(\xi_0 \xi_a - \Omega_a) \right) F_a^- &= 0,\label{eq:F-a-pm-sys-1}\\%\label{eq:A-pm-sys-1}\\
%\left( \xi_a^2 + 2(\xi_0 \xi_a - \Omega_a)  - 4 B^2 \right) F_a^+ +\left( - \xi_a^2 + 2(\xi_0 \xi_a - \Omega_a)  + 4 B|^2 \right)F_a^- &= 0\label{eq:F-a-pm-sys-2},%\label{eq:A-pm-sys-2},
%\end{align}
%where we dropped the dependencies on the constants $(\chi_0,\tau_0)$, and similarly, $(r_{b}, s_{b})$ satisfies \eqref{eq:linearized-NLS-r-s} if and only if 
%\begin{align}
%\left( \xi_b^2 + 2(\xi_0 \xi_b - \Omega_b) \right) F_b^+ +\left( \xi_b^2 - 2(\xi_0 \xi_b - \Omega_b) \right )F_b^- &= 0,\label{eq:F-b-pm-sys-1}\\%\label{eq:B-pm-sys-1}\\
%\left( \xi_b^2 + 2(\xi_0 \xi_b - \Omega_b)  - 4 B^2 \right) F_b^+ +\left( - \xi_b^2 + 2(\xi_0 \xi_b - \Omega_b)  + 4 B^2 \right ) F_b^- &= 0.\label{eq:F-b-pm-sys-2}%\label{eq:B-pm-sys-2}
%\end{align}
%Note that the pairs \eqref{eq:F-a-pm-sys-1}-\eqref{eq:F-a-pm-sys-2} and \eqref{eq:F-b-pm-sys-1}-\eqref{eq:F-b-pm-sys-2} constitute homogeneous systems of linear equations for $(F_a^+(\chi_0,\tau_0), F_a^-(\chi_0,\tau_0))$ and $(F_b^+(\chi_0,\tau_0), F_b^-(\chi_0,\tau_0))$, respectively. These systems have nontrivial solutions $(F_a^+(\chi_0,\tau_0), F_a^-(\chi_0,\tau_0))$ and $(F_b^+(\chi_0,\tau_0), F_b^-(\chi_0,\tau_0))$ if and only if they are singular, which amount to the conditions
%\begin{align}
%4 (\xi_0 \xi_a-\Omega_a )^2 &= \xi_a^2 \left(\xi_a^2 - 4 B(\chi_0,\tau_0)^2 \right),\label{eq:linearized-dispersion-a-bun}\\
%4 (\xi_0 \xi_b- \Omega_b )^2 &= \xi_b^2 \left(\xi_b^2 - 4 B(\chi_0,\tau_0)^2 \right).\label{eq:linearized-dispersion-b-bun}
%\end{align}
%These are precisely the linearized dispersion relation \eqref{eq:linearized-dispersion} satisfied by the pairs $(\xi_a, \Omega_a)$ and $(\xi_b, \Omega_b)$ of relative wave numbers and frequencies. We will first show that \eqref{eq:linearized-dispersion-a-bun} and \eqref{eq:linearized-dispersion-b-bun} hold, and then show that the pairs $(F_a^+(\chi_0,\tau_0), F_a^-(\chi_0,\tau_0))$ and $(F_b^+(\chi_0,\tau_0), F_b^-(\chi_0,\tau_0))$ defined in \eqref{eq:shelves-amplitudes-bun} lie in the (nontrivial) nullspaces of the coefficient matrices for the systems \eqref{eq:F-a-pm-sys-1}-\eqref{eq:F-a-pm-sys-2} and \eqref{eq:F-b-pm-sys-1}-\eqref{eq:F-b-pm-sys-2}, respectively.
%
%We refer back to Lemma~\ref{lemma:g-derivatives} and use the expression \eqref{eq:g-chi} for $g_\chi(\lambda;\chi,\tau)$ together with $\theta_\chi(\lambda;\chi,\tau)=\lambda$ to see that
%%\begin{equation}
%%%\begin{split}
%%{h}_{a\chi}(\chi_0,\tau_0) 
%%%&= -\ii h_{\chi-}(a(\chi_0; \tau_0),\chi_0,\tau_0) \\
%%%&=  -\ii g_{\chi-}(a(\chi_0;\tau_0),\chi_0,\tau_0)  + \vartheta_\chi(a(\chi_0; \tau_0);\chi_0,\tau_0)  \\
%%= A(\chi_0,\tau_0)  + R_-(a(\chi_0,\tau_0);\chi_0,\tau_0).
%%%\end{split}
%%\label{eq:h-tilde-a-chi}
%%\end{equation}
%%Similarly,
%%\begin{equation}
%%{h}_{b\chi}(\chi_0,\tau_0) = -\ii h_{\chi}(b(\chi_0; \tau_0),\chi_0,\tau_0) = A(\chi_0,\tau_0)  + R(b(\chi_0,\tau_0);\chi_0,\tau_0).
%%\label{eq:h-tilde-b-chi}
%%\end{equation}
%\begin{align}
%{h}_{a\chi}(\chi_0,\tau_0)  &= A(\chi_0,\tau_0)  + R_-(a(\chi_0,\tau_0);\chi_0,\tau_0) \label{eq:h-tilde-a-chi} \\
%{h}_{b\chi}(\chi_0,\tau_0)  &= A(\chi_0,\tau_0)  + R(b(\chi_0,\tau_0);\chi_0,\tau_0) \label{eq:h-tilde-b-chi}
%\end{align}
%Now, combining \eqref{eq:h-tilde-a-chi} and \eqref{eq:h-tilde-b-chi} together with \eqref{eq:kappa-chi} in \eqref{eq:xi-a-bun} and \eqref{eq:xi-b-bun} gives
%\begin{align}
%\xi_a &= -2 R_-(a(\chi_0;\tau_0);\chi_0,\tau_0),\label{eq:xi-a-explicit}\\
%\xi_b &= -2 R(b(\chi_0;\tau_0);\chi_0,\tau_0).\label{eq:xi-b-explicit}
%\end{align}
%On the other hand, using the expression \eqref{eq:g-tau} for $g_\tau$ together with $\theta_\tau(\lambda;\chi,\tau)=\lambda^2$ lets us write
%\begin{align}
%{h}_{a\tau}(\chi_0,\tau_0) &= 
%A(\chi_0,\tau_0)^2 - \frac{1}{2}B(\chi_0,\tau_0)^2 + (A(\chi_0,\tau_0) + a(\chi_0;\tau_0))R_-(a(\chi_0;\tau_0);\chi_0, \tau_0)\label{eq:h-tilde-a-tau}\\
%{h}_{b\tau}(\chi_0,\tau_0) &=
%A(\chi_0,\tau_0)^2 - \frac{1}{2}B(\chi_0,\tau_0)^2 + (A(\chi_0,\tau_0) + b(\chi_0;\tau_0))R(b(\chi_0;\tau_0);\chi_0, \tau_0)\label{eq:h-tilde-b-tau}
%\end{align}
%%\begin{equation}
%%\begin{split}
%%{h}_{a\tau}(\chi_0,\tau_0) &= -\ii h_{\tau-}(a(\chi_0; \tau_0),\chi_0,\tau_0) \\
%%&=  -\ii g_{\tau-}(a(\chi_0;\tau_0),\chi_0,\tau_0)  + \vartheta_\tau(a(\chi_0; \tau_0);\chi_0,\tau_0)  \\
%%&=  A(\chi_0,\tau_0)^2 - \frac{1}{2}B(\chi_0,\tau_0)^2 + (A(\chi_0,\tau_0) + a(\chi_0;\tau_0))R_-(a(\chi_0;\tau_0);\chi_0, \tau_0)
%%\end{split}
%%\label{eq:h-tilde-a-tau}
%%\end{equation}
%%and, similarly,
%%\begin{equation}
%%\begin{split}
%%{h}_{b\tau}(\chi_0,\tau_0) &= -\ii h_{\tau}(b(\chi_0; \tau_0),\chi_0,\tau_0) \\
%%%&=  -\ii g_{\tau-}(b(\chi_0;\tau_0),\chi_0,\tau_0)  + \vartheta_\tau(b(\chi_0; \tau_0);\chi_0,\tau_0)  \\
%%&=  A(\chi_0,\tau_0)^2 - \frac{1}{2}B(\chi_0,\tau_0)^2 + (A(\chi_0,\tau_0) + b(\chi_0;\tau_0))R(b(\chi_0;\tau_0);\chi_0, \tau_0)
%%\end{split}
%%\label{eq:h-tilde-b-tau}
%%\end{equation}
%We then combine \eqref{eq:h-tilde-a-tau} and \eqref{eq:h-tilde-b-tau} together with \eqref{eq:kappa-tau} in \eqref{eq:Omega-a-bun} and \eqref{eq:Omega-b-bun} and obtain
%\begin{align}
%\Omega_a &= 2(A(\chi_0,\tau_0) + a(\chi_0,\tau_0)) R_-(a(\chi_0;\tau_0);\chi_0,\tau_0),\label{eq:Omega-a-explicit}\\
%\Omega_b &= 2(A(\chi_0,\tau_0) + b(\chi_0,\tau_0))  R(b(\chi_0;\tau_0);\chi_0,\tau_0).\label{eq:Omega-b-explicit}
%\end{align}
%Now, to verify that \eqref{eq:linearized-dispersion-a-bun} holds, we use \eqref{eq:xi-a-explicit} and \eqref{eq:Omega-a-explicit} to observe that
%\begin{equation}
%\begin{split}
%4 (\xi_0 \xi_a-\Omega_a )^2 
%&= 4\left(4 A(\chi_0,\tau_0) R(a(\chi_0,\tau_0); \chi_0, \tau_0) - 2(A(\chi_0, \tau_0)+a(\chi_0, \tau_0)) R_-(a(\chi_0, \tau_0); \chi_0, \tau_0)\right)^{2}\\
%&=16 R_-(a(\chi_0, \tau_0); \chi_0, \tau_0)^2 \left(a(\chi_0, \tau_0) - A(\chi_0, \tau_0) \right)^2,%\\
%%&=16\left( \left(a(\chi_0, \tau_0) - A(\chi_0, \tau_0) \right)^2 + B(\chi_0,\tau_0)^2 \right) \left(a(\chi_0, \tau_0) - A(\chi_0, \tau_0) \right)^2,
%\end{split}
%\label{eq:linearized-dispersion-a-LHS}
%\end{equation}
%where we also recalled \eqref{eq:xi-0-bun}. Next,
%\begin{equation}
%\begin{split}
%\xi_a^2 \left(\xi_a^2 - 4 B(\chi_0,\tau_0)^2 \right)
%&= 4 R_-(a(\chi_0,\tau_0);\chi_0,\tau_0)^2 \left(4 R_-(a(\chi_0,\tau_0);\chi_0,\tau_0)^2 - 4 B(\chi_0,\tau_0)^2 \right)\\
%&= 16 R_-(a(\chi_0,\tau_0);\chi_0,\tau_0)^2 \left( a(\chi_0,\tau_0) - A(\chi_0,\tau_0) \right)^2
%\end{split}
%\label{eq:linearized-dispersion-a-RHS}
%\end{equation}
%since $R(\lambda;\chi,\tau)^2 = (\lambda- A(\chi,\tau))^2 + B(\chi,\tau)^2$. We see from the identities \eqref{eq:linearized-dispersion-a-LHS} and \eqref{eq:linearized-dispersion-a-RHS} that the linearized dispersion relation \eqref{eq:linearized-dispersion-a-bun} holds. A completely analogous calculation having the point $b(\chi_0,\tau_0)$ in place of $a(\chi_0,\tau_0)$ and using \eqref{eq:xi-b-explicit} and \eqref{eq:Omega-b-explicit} shows that the relation \eqref{eq:linearized-dispersion-b-bun} also holds.
%
%As we have now established that the linear systems \eqref{eq:F-a-pm-sys-1}-\eqref{eq:F-a-pm-sys-2} and \eqref{eq:F-b-pm-sys-1}-\eqref{eq:F-b-pm-sys-2} are both singular, it remains to show that the amplitude pairs $(F_a^{+}(\chi_0,\tau_0), F_a^{-}(\chi_0,\tau_0))$
%and $(F_b^{+}(\chi_0,\tau_0), F_b^{-}(\chi_0,\tau_0))$ lie in the corresponding nullspaces. 
%To do so, it suffices to verify that \eqref{eq:F-a-pm-sys-1} and \eqref{eq:F-b-pm-sys-1} hold. Taking the definitions \eqref{eq:shelves-amplitudes-bun} of $F_a^{\pm}$ and $F_b^{\pm}$ into account, verifying these amount to showing that
%\begin{align}
%\cos( \arg(a(\chi_0,\tau_0)-\lambda_0(\chi_0,\tau_0)) ) &= \frac{-2(\xi_0\xi_a - \Omega_a)}{\xi_a^2},\label{eq:show-cos-arg-a-bun}\\
%\cos( \arg(b(\chi_0,\tau_0)-\lambda_0(\chi_0,\tau_0)) ) &= \frac{-2(\xi_0\xi_b - \Omega_b)}{\xi_b^2}.\label{eq:show-cos-arg-b-bun}
%\end{align}
%The right-hand sides above can be simplified by directly using the expressions \eqref{eq:xi-0-explicit}, \eqref{eq:xi-a-explicit}-\eqref{eq:xi-b-explicit}, and \eqref{eq:Omega-a-explicit}-\eqref{eq:Omega-b-explicit}, and it is seen that showing \eqref{eq:show-cos-arg-a-bun} and \eqref{eq:show-cos-arg-b-bun} is equivalent to verifying the identities
%\begin{align}
%\cos( \arg(a(\chi_0,\tau_0)-\lambda_0(\chi_0,\tau_0)) ) &= \frac{a(\chi_0,\tau_0) - A(\chi_0,\tau_0)}{R_-(a(\chi_0,\tau_0))},\label{eq:show-cos-arg-a-simp}\\
%\cos( \arg(b(\chi_0,\tau_0)-\lambda_0(\chi_0,\tau_0)) ) &= \frac{b(\chi_0,\tau_0) - A(\chi_0,\tau_0)}{R(b(\chi_0,\tau_0))}.\label{eq:show-cos-arg-b-simp}
%\end{align}
%Now note that since $a(\chi_0,\tau_0) < \Re (\lambda_0(\chi_0,\tau_0))<0$ and $\Im(\lambda_0(\chi_0,\tau_0))=B(\chi_0,\tau_0)>0$ in $\shelves$, it follows that
%\begin{equation}
%\begin{split}
%\cos( \arg(a(\chi_0,\tau_0) - \lambda_0(\chi_0,\tau_0)) )
% &= - \cos( \arg(\lambda_0(\chi_0,\tau_0) - a(\chi_0,\tau_0)) )\\
% &= - \frac{\Re(\lambda_0(\chi_0,\tau_0))-a(\chi_0,\tau_0)}{| \lambda_0(\chi_0,\tau_0)-a(\chi_0,\tau_0)|}\\
% & = \frac{a(\chi_0,\tau_0) -A(\chi_0,\tau_0)}{\sqrt{\left(a(\chi_0,\tau_0) - A(\chi_0,\tau_0)\right)^2 + B(\chi_0,\tau_0)^2} } <0,
%\end{split}
%\label{eq:cos-arg-a-simp-bun}
%\end{equation}
%where we have used $\Re(\lambda_0(\chi_0,\tau_0)) = A(\chi_0,\tau_0)$. On the other hand, $R(\lambda;\chi,\tau) = \lambda + O(1)$ as $\lambda\to\infty$ and $R(\lambda;\chi,\tau)^2 = (\lambda - A(\chi,\tau))^2 + B(\chi,\tau)^2$, hence
%\begin{equation}
%R_-(a(\chi_0,\tau_0);\chi_0,\tau_0) = \sqrt{\left(a(\chi_0,\tau_0) - A(\chi_0,\tau_0)\right)^2 + B(\chi_0,\tau_0)^2}.
%\end{equation}
% Thus, the last line of \eqref{eq:cos-arg-a-simp-bun} verifies \eqref{eq:show-cos-arg-a-simp} and this establishes that $p_a(\Delta x, \Delta t)$ is a solution of \eqref{eq:linearized-NLS-Delta-bun}.
% Similarly, since $ \Re (\lambda_0(\chi_0,\tau_0))<b(\chi_0,\tau_0)$, it is seen that
%\begin{equation}
%\begin{split}
%\cos( \arg(b(\chi_0,\tau_0) - \lambda_0(\chi_0,\tau_0)) ) &= 
%\frac{b(\chi_0,\tau_0) - \Re(\lambda_0(\chi_0,\tau_0)) }{| \lambda_0(\chi_0,\tau_0)-b(\chi_0,\tau_0)|}\\
%& =  \frac{b(\chi_0,\tau_0) -A(\chi_0,\tau_0)}{\sqrt{\left(b(\chi_0,\tau_0) - A(\chi_0,\tau_0)\right)^2 + B(\chi_0,\tau_0)^2} } >0,
%\end{split}
%\label{eq:cos-arg-b-simp-bun}
%\end{equation}
%and
%\begin{equation}
%R(b(\chi_0,\tau_0);\chi_0,\tau_0) = \sqrt{\left(b(\chi_0,\tau_0) - A(\chi_0,\tau_0)\right)^2 + B(\chi_0,\tau_0)^2}
%\end{equation}
%since $b(\chi_0,\tau_0)$ lies to the right of the branch cut $\Sigma_g$ of $R(\lambda;\chi,\tau)$. Therefore, the last line of \eqref{eq:cos-arg-b-simp-bun} verifies \eqref{eq:show-cos-arg-b-simp} and this establishes that $p_b(\Delta x, \Delta t)$ is a solution of \eqref{eq:linearized-NLS-Delta-bun}.
%\end{proof}
\begin{remark}
Requiring instead any of the individual plane waves 
\begin{equation}
\begin{split}
(\Delta x, \Delta t) &\mapsto \frac{\ii F_a^{[\shelves]}(\chi_0,\tau_0)}{B(\chi_0,\tau_0)}m_a^{\pm}(\chi_0,\tau_0)\ee^{\pm \ii\phi_a(\chi_0,\tau_0;M)} \ee^{\pm \ii(\xi_a \Delta x - \Omega_a \Delta t)}\quad\text{or}\\
(\Delta x, \Delta t) &\mapsto \frac{\ii F_b^{[\shelves]}(\chi_0,\tau_0)}{B(\chi_0,\tau_0)}m_b^{\pm}(\chi_0,\tau_0)\ee^{\pm \ii\phi_b(\chi_0,\tau_0;M)} \ee^{\pm \ii(\xi_b \Delta x - \Omega_b \Delta t)}
\end{split}
\end{equation}
in \eqref{eq:p-a-b-shelves} to be solutions of \eqref{eq:linearization-intro} forces $B(\chi_0,\tau_0)=0$, which is a contradiction. Therefore, one indeed needs to form the combinations $p_a(\Delta x, \Delta t)$ and $p_b(\Delta x, \Delta t)$ as in \eqref{eq:p-a-b-shelves}.
\end{remark}
%\begin{remark}
%Requiring instead any of the individual plane waves 
%\begin{equation}
%(\Delta x, \Delta t)\mapsto \frac{\ii F_a^{\pm}(\chi_0,\tau_0)}{B(\chi_0,\tau_0)}\ee^{\pm \ii(\xi_a \Delta x - \Omega_a \Delta t)},\quad 
%(\Delta x, \Delta t)\mapsto \frac{\ii F_b^{\pm}(\chi_0,\tau_0)}{B(\chi_0,\tau_0)}\ee^{\pm \ii(\xi_b \Delta x - \Omega_b \Delta t)}
%\end{equation}
% in \eqref{eq:p-a-bun} and \eqref{eq:p-b-bun} to be solutions of \eqref{eq:linearized-NLS-Delta-bun} forces $B(\chi_0,\tau_0)=0$, which is a contradiction. Therefore, one indeed needs to take the combinations $p_a$ and $p_b$ as in \eqref{eq:p-a-bun} and \eqref{eq:p-b-bun}.
%\end{remark}
Finally, we show that the relative wavenumbers $\xi_{a,b}$ do not lie in the band of modulational instability $(-2 B(\chi_0,\tau_0), 2 B(\chi_0,\tau_0))$ in view of the well-known theory summarized in Appendix~\ref{A:perturbations}. This result follows from the identities \eqref{eq:xi-a-b-explicit} in a straightforward manner. Indeed,
\begin{equation}
\begin{alignedat}{3}
\xi_a^2 &= 4 \left( a(\chi_0,\tau_0) - A(\chi_0,\tau_0) \right)^2 + 4 B(\chi_0,\tau_0)^2 &&> 4 B(\chi_0,\tau_0)^2,
\\
\xi_b^2 &= 4 \left( b(\chi_0,\tau_0) - A(\chi_0,\tau_0) \right)^2 + 4 B(\chi_0,\tau_0)^2 &&> 4 B(\chi_0,\tau_0)^2.
\end{alignedat}
\end{equation}


%As we have now established that $p_a(\Delta x, \Delta t)$ and $p_b(\Delta x, \Delta t)$ define plane wave solutions of the linearization of the NLS equation \eqref{eq:linearized-NLS-Delta-bun} about the plane wave $Q(\Delta x, \Delta t)$, we would like to investigate whether the relative wave numbers $\xi_a$ and $\xi_b$ lie in the bands of modulational instability $(-2 B(\chi_0,\tau_0), 2 B(\chi_0,\tau_0))$ in view of the calculation done in Appendix~\ref{A:perturbations}. The following proposition tells us that this is not the case, and concludes the analysis in this section.
%
%\begin{proposition}
%The relative wave numbers $\xi_a=\xi_a(\chi_0,\tau_0)$ and $\xi_b=\xi_b(\chi_0,\tau_0)$ satisfy
%\begin{equation}
%\xi_a^2 > 4 B(\chi_0,\tau_0)^2\quad\text{and}\quad  \xi_b^2 > 4 B(\chi_0,\tau_0)^2
%\end{equation}
%for $(\chi_0,\tau_0)\in \shelves$.
%\end{proposition}
%\begin{proof} This result follows from the identities \eqref{eq:xi-a-explicit} and \eqref{eq:xi-b-explicit} in a straightforward manner. Indeed,
%\begin{equation}
%\xi_a^2 = 4 \left( a(\chi_0,\tau_0) - A(\chi_0,\tau_0) \right)^2 + 4 B(\chi_0,\tau_0)^2 > 4 B(\chi_0,\tau_0)^2,
%\end{equation}
%and
%\begin{equation}
%\xi_b^2 = 4 \left( b(\chi_0,\tau_0) - A(\chi_0,\tau_0) \right)^2 + 4 B(\chi_0,\tau_0)^2 > 4 B(\chi_0,\tau_0)^2,
%\end{equation}
%which prove the claim.
%\end{proof}

%\subsection{Special case of fundamental rogue waves} \textcolor{red}{[Will there be a corollary?]} To see how the results established in this section apply to the fundamental rogue waves $\psi_k(x,t)$, $k\in \mathbb{Z}>0$, it suffices to (i) restrict $M$ to the sequence $M = \tfrac{1}{2}k + \tfrac{1}{4}$ and tie $s$ to the order $k\in \mathbb{Z}>0$ by $s=(-1)^k$, and (ii) take into account the exponential factor $\ee^{-\ii M \tau}$ mediating between \eqref{eq:q-S} and \eqref{eq:psi-k-S}. Doing so in \eqref{eq:q-bun} yields
%\begin{multline}
%\psi_k(M\chi,M\tau) = B(\chi,\tau)  \ee^{-\ii M \tau} \ee^{-2\ii \left(M \kappa(\chi,\tau) + \mu(\chi,\tau) + \frac{1}{4}(-1)^k \pi\right)}
%+ M^{-\frac{1}{2}} (-1)^k  \ee^{-\ii M \tau}  \ee^{-2\ii (M\kappa(\chi,\tau)+\mu(\chi,\tau) )} \\
% \cdot \left[ F_a^+(\chi,\tau)\ee^{\ii \Theta_a(\chi,\tau;M)}    + F_a^-(\chi,\tau)\ee^{- \ii \Theta_a(\chi,\tau;M)} \right. \\
%  \left. + F_b^+(\chi,\tau)\ee^{\ii \Theta_b(\chi,\tau;M)}   + F_b^-(\chi,\tau)\ee^{- \ii \Theta_b(\chi,\tau;M)} \right] + O(M^{-1}),
%  \label{eq:psi-k-bun}
%\end{multline}
%as $M\to +\infty$. This may also be written in the form
%
%\begin{multline}
%%\psi_k(M\chi, M\tau) = 
%\psi_k(M\chi,M\tau)=
%(-1)^k  \ee^{-2\ii (M\kappa(\chi,\tau)+\mu(\chi,\tau))} 
%\left[-\ii B(\chi,\tau) + M^{-\frac{1}{2}} \left( F_a^+(\chi,\tau)\ee^{\ii \Theta_a(\chi,\tau;M)}  \right. \right.\\ + F_a^-(\chi,\tau)\ee^{- \ii \Theta_a(\chi,\tau;M)}
%\left.\left. 
%+ F_b^+(\chi,\tau)\ee^{\ii \Theta_b(\chi,\tau;M)} + F_b^-(\chi,\tau)\ee^{- \ii \Theta_b(\chi,\tau;M)} 
%\right) \right]
%+ O(M^{-1}).
%\label{eq:psi-k-bun-alt}
%\end{multline}
%
%Notice that as $(\chi,\tau)$ approaches from within $\shelves$ the boundary curve separating $\shelves$ and $\exterior$, $a(\chi,\tau)$ and $b(\chi,\tau)$ coalesce and we see from the formula \eqref{eq:mu-formula-intro} that $\mu(\chi,\tau)$ tends to $0$ since the domain of integration contracts to a point. In this case the overall phase $M \kappa(\chi, \tau)+ \mu(\chi,\tau) + \tfrac{1}{4}(-1)^k \pi$ in \eqref{eq:psi-k-bun} becomes $M \kappa(\chi,\tau)+\tfrac{1}{4}(-1)^k \pi$. On the other hand, we see from \eqref{eq:kappa-gamma} that $M \kappa(\chi,\tau) = M \gamma(\chi,\tau) - n\pi - \tfrac{1}{4}(-1)^k \pi$ with $n\in\mathbb{Z}_>0$, hence the simplification
%\begin{equation}
%\ee^{-2\ii \left(M \kappa(\chi,\tau) + \tfrac{1}{4}(-1)^k \pi \right)} = \ee^{-2\ii M \gamma(\chi,\tau)}.
%\end{equation}
%This shows us that the leading term in \eqref{eq:psi-k-bun} matches the leading term in \eqref{eq:psi-k-shelves-chi-tau-ALT} for $(\chi,\tau)$ on the boundary curve separating $\shelves$ and $\exterior$ through the vanishing of $\mu(\chi,\tau)$. 



%\begin{remark}
%As $(\chi,\tau)$ approaches from within $\shelves$ the boundary curve separating $\shelves$ and $\exterior$, $a(\chi,\tau)$ and $b(\chi,\tau)$ coalesce and we see from the formula \eqref{eq:mu-def-bun} that $\mu(\chi,\tau)$ tends to $0$ since to domain of integration contracts to a point. In this case the overall phase factor $M \kappa(\chi, \tau)+ \mu(\chi,\tau)$ in \eqref{eq:q-bun} becomes equal to $M \gamma(\chi,\tau)$.
%%, see the definition \eqref{eq:Theta-0}. 
%Thus, we see that the leading term in \eqref{eq:q-bun} matches the leading term in \eqref{eq:q-M-shelves-chi-tau} for $(\chi,\tau)$ on the boundary curve separating $\shelves$ and $\exterior$ through the vanishing of $\mu(\chi,\tau)$. \textcolor{red}{[This should be linked to the construction of $g$ earlier in $\shelves$ vs. $\exterior$. Also we should give this remark only after the subsection "special case of fundamental rogue waves". But where would that section go? The following material is also valid for $M>0$ untied to $s=\pm 1$.]}
%\end{remark}

%Next, we compute the absolute errors measured in the sup-norm

%To verify the accuracy of the asymptotic formula \eqref{eq:q-bun} we first compare the  with the exact solution $\psi_k(\chi,\tau)$

%\sqrt{\frac{\ln(2)}{\pi}}\ee^{\ii(\frac{1}{4}\pi+2\pi p^2-\arg(\Gamma(\ii p)))}
%In view of Remark~\ref{rem:omega} we see that 
%\begin{equation}
%\omega(\lambda) = \left( \frac{\lambda-\ii }{ \lambda+\ii } \right)^{1/4}
%\end{equation}
%where the right-hand side defined to have its branch cut to be $\Sigma_g$ 
%complex power function is chosen as the principal branch, ensuring that $\omega(\lambda)\to 1$ as $\lambda\to\infty$, and the branch cut is chosen t
%\section{\textcolor{red}{Far-Field Asymptotics of Rogue Waves in the ``Channels''}}
%We make the following substitution into Riemann-Hilbert Problem~\ref{rhp:rogue-wave}:
%\begin{equation}
%\mathbf{P}^{(k)}(\lambda;x,t)\defeq \ee^{\ii t\sigma_3/2}\mathbf{M}^{(k)}(\lambda;x,t)\begin{cases}
%\ee^{-\ii\rho(\lambda)(x+\lambda t)\sigma_3}\mathbf{Q}\ee^{\ii\lambda(x+\lambda t)\sigma_3},&\quad\text{$\lambda$ inside $\Sigma_\circ$}
%\\
%\ee^{-\ii (t+x\lambda^{-1})\sigma_3/2}\ee^{\pm 2\ii n(\lambda^{-1}-\lambda^{-3}/3)\sigma_3},&\quad\text{$\lambda$ exterior to $\Sigma_\circ$,}
%\end{cases}
%\end{equation}
%where the top (bottom) sign refers to the case $k=2n$ ($k=2n-1$).  Since $\ee^{\ii\rho(\lambda)(x+\lambda t)\sigma_3}$ satisfies the jump condition of $\mathbf{M}^{(k)}(\lambda;x,t)$ across the contour $\Sigma_\mathrm{c}$, it follows that $\mathbf{P}^{(k)}(\lambda;x,t)$ can be considered to be analytic in the interior of $\Sigma_\circ$.  It is not hard to check that moreover, $\mathbf{P}^{(k)}(\lambda;x,t)$ is the solution of the following modified problem.
%\begin{rhp}[Modified problem for rogue waves]
%Let $(x,t)\in\mathbb{R}^2$ be arbitrary parameters, and let $k\in\mathbb{Z}_{\ge 0}$.  Find a $2\times 2$ matrix $\mathbf{P}^{(k)}(\lambda;x,t)$ with the following properties:
%\begin{itemize}
%\item[]\textbf{Analyticity:}  $\mathbf{P}^{(k)}(\lambda;x,t)$ is analytic in $\lambda$ for $\lambda\in\mathbb{C}\setminus\Sigma_\circ$, and it takes continuous boundary values on $\Sigma_\circ$.
%\item[]\textbf{Jump conditions:}  The boundary values on the jump contour $\Sigma_\circ$ are related as follows:
%\begin{multline}
%\mathbf{P}_+^{(k)}(\lambda;x,t)=\mathbf{P}_-^{(k)}(\lambda;x,t)\ee^{-\ii\lambda(x+\lambda t)\sigma_3}\left(\frac{\lambda-\ii}{\lambda+\ii}\right)^{\pm n\sigma_3}\mathbf{Q}^{-1}\mathbf{E}(\lambda)\\
%{}\cdot\ee^{\pm 2\ii n(\lambda^{-1}-\tfrac{1}{3}\lambda^{-3})\sigma_3}\ee^{\ii[\rho(\lambda)(x+\lambda t)-\tfrac{1}{2}(t+x\lambda^{-1})]\sigma_3},\quad
%\lambda\in\Sigma_\circ,
%\label{eq:P-jump}
%\end{multline}
%where the top (bottom) sign in the exponents corresponds to $k=2n$ ($k=2n-1$).
%\item[]\textbf{Normalization:}  $\mathbf{P}^{(k)}(\lambda;x,t)\to\mathbb{I}$ as $\lambda\to\infty$. 
%\end{itemize}
%\label{rhp:rogue-wave-modified}
%\end{rhp}
%The formula for $\psi_k(x,t)$ corresponding to \eqref{eq:psi-from-M} now becomes
%\begin{equation}
%\psi_k(x,t)=2\ii\ee^{-\ii t}\lim_{\lambda\to\infty}\lambda P^{(k)}_{12}(\lambda;x,t).
%\label{eq:psi-from-P}
%\end{equation}
%Now, the radius of $\Sigma_\circ$ can be taken to be arbitrarily large.  Note the elementary expansions
%\begin{equation}
%\left(\frac{\lambda-\ii}{\lambda+\ii}\right)^{\pm n\sigma_3} = \ee^{\mp 2\ii n(\lambda^{-1}-\tfrac{1}{3}\lambda^{-3})\sigma_3}\ee^{O(n\lambda^{-5})\sigma_3},\quad\lambda\to\infty,
%\label{eq:large-lambda-1}
%\end{equation}
%\begin{equation}
%\mathbf{E}(\lambda)=\mathbb{I}-\frac{1}{2}\ii\lambda^{-1}\sigma_1 +\begin{bmatrix}O(\lambda^{-2}) & O(\lambda^{-3})\\
%O(\lambda^{-3}) & O(\lambda^{-2})\end{bmatrix},\quad\lambda\to\infty,
%\label{eq:large-lambda-2}
%\end{equation}
%and
%\begin{equation}
%\ee^{\ii[\rho(\lambda)(x+\lambda t)-\tfrac{1}{2}(t+x\lambda^{-1})]\sigma_3}=\ee^{\ii\lambda(x+\lambda t)\sigma_3}\ee^{O(t\lambda^{-2})\sigma_3}\ee^{O(x\lambda^{-3})\sigma_3},\quad\lambda\to\infty.
%\label{eq:large-lambda-3}
%\end{equation}

%\section{Numerical Verification of the Asymptotic Formul\ae{}}
%\input{NumericalVerification}

%\section{\textcolor{red}{Far-Field Asymptotics of Rogue Waves in the ``Channels''}}
%\input{OLD-Channels}
%
%
%\section{\textcolor{red}{Far-Field Asymptotics of Rogue Waves in the ``Shelves''}}
%\input{OLD-Shelves}
%
%\section{Numerical Verification of the Asymptotic Formul\ae{}}
%\input{OLD-NumericalVerification}


\newpage
\appendix
%\section{OLD --- Evaluation of Certain Quantities for Computational Purposes}
%\input{OLD-Appendix-Evaluations}
%
%
\section{Relative Perturbations of Plane Waves and Linear Instability Bands}
\label{A:perturbations}
In this section of the Appendix we consider relative perturbations of a plane-wave solution $q=Q(x,t):= \mathcal{A} \ee^{\ii (\xi_0 x - \Omega_0 t)}$ of the focusing nonlinear Schr\"odinger equation in the form \eqref{eq:NLS-ZBC}, 
%\begin{equation}
%\ii q_t + \frac{1}{2} q_{xx} + |q|^2 q = 0,
%\label{eq:q-nls-appendix}
%\end{equation}
having complex-valued amplitude $\mathcal{A}$, wavenumber $\xi_0\in\mathbb{R}$, and frequency $\Omega_0\in\mathbb{R}$ necessarily linked by the nonlinear dispersion relation $\Omega_0 -\tfrac{1}{2}\xi_0^2 + |\mathcal{A}|^2=0$. Requiring more generally that 
\begin{equation}
q(x,t) = Q(x,t)(1 + \varepsilon p(x,t) + o(\varepsilon)),\quad 0< \varepsilon \ll 1,
\label{eq:q-perturb-appendix}
\end{equation}
also solves \eqref{eq:NLS-ZBC} 
%to be a solution of \eqref{eq:q-nls-appendix} 
and formally retaining terms up to $o(\varepsilon)$ yields the differential equation
\begin{equation}
\ii p_t + \ii \xi_0 p_x + \tfrac{1}{2}p_{xx} |\mathcal{A}|^2(p + p^*) = 0,
\label{eq:linearized-NLS}
\end{equation} 
which is real-linear, but not complex-linear. We split $p(x,t)$ into its real and imaginary parts:
\begin{equation}
r(x,t) := \tfrac{1}{2}(p(x,t) + p(x,t)^*)\quad \text{and} \quad s(x,t) := -\ii\tfrac{1}{2}(p(x,t) - p(x,t)^*),
\end{equation}
giving rise to the following system of coupled linear differential equations with real-valued coefficients for $(r,s)$:
\begin{equation}
\begin{aligned}
r_t + \xi_0 r_x + \tfrac{1}{2} s_{xx} &=0,\\
s_t + \xi_0 s_x - \tfrac{1}{2} r_{xx} - 2 |\mathcal{A}|^2 r &=0.\\
\end{aligned}
\label{eq:linearized-NLS-r-s}
\end{equation}
We will now carry out a Fourier analysis to determine the instability bands for relative perturbations $p(x,t)$ of $Q(x,t)$. To this end, we suppose $p(x,t)$ is a plane-wave solution of \eqref{eq:linearized-NLS}. For convenience we drop for the moment the reality condition for $(r,s)$ and work with the ansatz
\begin{equation}
\begin{bmatrix}
r(x,t) \\ s(x,t)
\end{bmatrix}
:= \begin{bmatrix}
\alpha \\ \beta
\end{bmatrix}\ee^{\ii (\xi x - \Omega t)}
\label{eq:r-s-ansatz}
\end{equation}
for some complex constants $\alpha$ and $\beta$, and $\xi\in\mathbb{R}$, $\Omega\in\mathbb{R}$. Substituting \eqref{eq:r-s-ansatz} in \eqref{eq:linearized-NLS-r-s} yields the homogenous linear algebraic system
\begin{equation}
\begin{bmatrix}
-\ii(\Omega -\xi_0 \xi) & -\frac{1}{2}\xi^2 \\  -\frac{1}{2}\xi^2 + 2 |\mathcal{A}|^2 & \ii(\Omega - \xi_0 \xi)
\end{bmatrix}
\begin{bmatrix}
\alpha \\ \beta
\end{bmatrix}
=
\begin{bmatrix}
0 \\ 0
\end{bmatrix}.
\end{equation}
This system has a nontrivial solution $(\alpha,\beta)$ if and only if 
\begin{equation}
4 (\Omega - \xi_0 \xi)^2 = \xi^2 \left(\xi^2 - 4|\mathcal{A}|^2 \right),
\label{eq:linearized-dispersion}
\end{equation}
which is the linearized dispersion relation for the relative wavenumber $\xi$ and the relative frequency $\Omega$. As $\xi,\xi_0\in\mathbb{R}$, we see that $\Im(\Omega) \neq 0$ if $\xi^2< 4|\mathcal{A}|^2$. Therefore, the plane-wave solutions of \eqref{eq:linearized-NLS-r-s} with relative wavenumbers $\xi$ lying in the band $(-2 |\mathcal{A}|, 2 |\mathcal{A}|)$ exhibit exponential growth in time $t>0$.  This is the well-known modulational (or sideband, or Benjamin-Feir) instability of plane-wave solutions for the focusing nonlinear Schr\"odinger equation.

\section{Proofs of Some Elementary Results}
\label{A:Proofs}
\subsection{Symmetries of $q(x,t;\mathbf{Q}^{-s},M)$:  Proof of Proposition~\ref{prop:symmetry}}
\begin{proof}[Proof of Proposition~\ref{prop:symmetry}]
For this proof, we assume without loss of generality that $\Sigma_\circ$ is a circle centered at the origin of arbitrary radius $r$ greater than $1$ with clockwise orientation.
Taking $\mathbf{G}=\mathbf{Q}^{-s}$ for $s=\pm 1$, the jump condition \eqref{eq:P-bulk-jump} in Riemann-Hilbert Problem~\ref{rhp:rogue-wave-reformulation} for $\mathbf{P}(\lambda;x,t)=\mathbf{P}(\lambda;x,t,\mathbf{Q}^{-s},M)$ can be written as 
\begin{equation}
\mathbf{P}_+(\lambda;x,t)=\mathbf{P}_-(\lambda;x,t)\ee^{-\ii\theta(\lambda;x,t)\sigma_3}%B_k(\lambda)^{\sigma_3}
%\omega(\lambda)^{N\sigma_3}
B(\lambda)^{M\sigma_3}
\mathbf{Q}^{-s}%B_k(\lambda)^{-\sigma_3}
%\omega(\lambda)^{-N\sigma_3}
B(\lambda)^{-M\sigma_3}
\ee^{\ii\theta(\lambda;x,t)\sigma_3},
\label{eq:P-bulk-jump-rewrite}
\end{equation}
where $\theta(\lambda;x,t)\defeq\lambda x+\lambda^2t$. 
%and where $B_k(\lambda)=\omega(\lambda)^{s}((\lambda-\ii)/(\lambda+\ii))^{n}$.  
Define $\mathbf{R}(\lambda;x,t)$ in terms of $\mathbf{P}(\lambda;x,t)$ by
\begin{equation}
\mathbf{R}(\lambda;x,t)\defeq\begin{cases}
-s\sigma_3\mathbf{P}(\lambda;x,t)\ee^{-2\ii\theta(\lambda;x,t)\sigma_3}\sigma_1,&\quad |\lambda|<r,
\\
\sigma_3\mathbf{P}(\lambda;x,t)%B_k(\lambda)^{2\sigma_3}
%\omega(\lambda)^{2N\sigma_3}
B(\lambda)^{2M\sigma_3}
\sigma_3,&\quad |\lambda|>r.
\end{cases}
\end{equation} 
$\mathbf{R}(\lambda;x,t)$ is obviously analytic for $\lambda\in\mathbb{C}\setminus\Sigma_\circ$, and since 
%$f(\lambda)\to 1$ and $\rho(\lambda)-\lambda\to 0$ 
%$\omega(\lambda)\to 1$ 
powers of $B(\lambda)$ tend to $1$ 
as $\lambda\to\infty$ we have $\mathbf{R}(\lambda;x,t)\to\mathbb{I}$ as $\lambda\to\infty$.  To compute the jump across the circle $\Sigma_\circ$, we use the jump condition \eqref{eq:P-bulk-jump} for $\mathbf{P}(\lambda;x,t)$ to obtain, using \eqref{eq:Q-def} in the last step,
\begin{equation}
\begin{split}
\mathbf{R}_+(\lambda;x,t)&=\sigma_3\mathbf{P}_+(\lambda;x,t)
%B_k(\lambda)^{2\sigma_3}
%\omega(\lambda)^{2N\sigma_3}
B(\lambda)^{2M\sigma_3}
\sigma_3\\
&=\sigma_3\mathbf{P}_-(\lambda;x,t)\ee^{-\ii\theta(\lambda;x,t)\sigma_3}
%B_k(\lambda)^{\sigma_3}
%\omega(\lambda)^{N\sigma_3}
B(\lambda)^{M\sigma_3}
\mathbf{Q}^{-s}\sigma_3 
%B_k(\lambda)^{\sigma_3}
%\omega(\lambda)^{N\sigma_3}
B(\lambda)^{M\sigma_3}
\ee^{\ii\theta(\lambda;x,t)\sigma_3}\\
&=-s\mathbf{R}_-(\lambda;x,t)\sigma_1\ee^{\ii\theta(\lambda;x,t)\sigma_3}
%B_k(\lambda)^{\sigma_3}
%\omega(\lambda)^{N\sigma_3}
B(\lambda)^{M\sigma_3}
\mathbf{Q}^{-s}\sigma_3
%B_k(\lambda)^{\sigma_3}
%\omega(\lambda)^{N\sigma_3}
B(\lambda)^{M\sigma_3}
\ee^{\ii\theta(\lambda;x,t)\sigma_3}\\
&=\mathbf{R}_-(\lambda;x,t)\ee^{-\ii\theta(\lambda;x,t)\sigma_3}
%B_k(\lambda)^{-\sigma_3}
%\omega(\lambda)^{-N\sigma_3}
B(\lambda)^{-M\sigma_3}
\left[-s\sigma_1\mathbf{Q}^{-s}\sigma_3\right]
%B_k(\lambda)^{\sigma_3}
%\omega(\lambda)^{N\sigma_3}
B(\lambda)^{M\sigma_3}
\ee^{\ii\theta(\lambda;x,t)\sigma_3}\\
&=\mathbf{R}_-(\lambda;x,t)\ee^{-\ii\theta(\lambda;x,t)\sigma_3}
%B_k(\lambda)^{-\sigma_3}
%\omega(\lambda)^{-N\sigma_3}
B(\lambda)^{-M\sigma_3}
\mathbf{Q}^{-s}
%B_k(\lambda)^{\sigma_3}
%\omega(\lambda)^{N\sigma_3}
B(\lambda)^{M\sigma_3}
\ee^{\ii\theta(\lambda;x,t)\sigma_3}.
\end{split}
\end{equation}
Now $\theta(\lambda;-x,t)=\theta(-\lambda;x,t)$
%, and 
%since $f(-\lambda)=f(\lambda)$, $\rho(-\lambda)=-\rho(\lambda)$, and $\rho^2=\lambda^2+1$, $B_k(\lambda)=B_k(-\lambda)^{-1}$.  
%and $\omega(\lambda)=\omega(-\lambda)^{-1}$.
and $B(\lambda)=B(-\lambda)^{-1}$.
Therefore, we see that $\mathbf{P}(\lambda;-x,t)$ and $\mathbf{R}(-\lambda;x,t)$ satisfy exactly the same analyticity, jump, and normalization conditions and therefore by uniqueness $\mathbf{P}(\lambda;-x,t)=\mathbf{R}(-\lambda;x,t)$.  Thus,
\begin{equation}
\begin{split}
q(-x,t;\mathbf{Q}^{-s},M)&=2\ii\lim_{\lambda\to\infty}\lambda P_{12}(\lambda;-x,t)\\
&=2\ii\lim_{\lambda\to\infty}\lambda R_{12}(-\lambda;x,t)\\
&=-2\ii\lim_{\lambda\to\infty}\lambda R_{12}(\lambda;x,t)\\
&=2\ii\lim_{\lambda\to\infty}\lambda P_{12}(\lambda;x,t)\\
&=q(x,t;\mathbf{Q}^{-s},M).
\end{split}
\end{equation}
Since 
%$B_k(-\lambda^*)=B_k(\lambda)^*$, 
%$\omega(-\lambda^*)=\omega(\lambda)^*$
$B(-\lambda^*)=B(\lambda)^*$,
it is even easier to see that $\mathbf{P}(\lambda;x,-t)$ and $\mathbf{P}(-\lambda^*;x,t)^*$ solve the same Riemann-Hilbert problem and hence are equal.  Therefore
\begin{equation}
\begin{split}
q(x,-t;\mathbf{Q}^{-s},M)&=2\ii\lim_{\lambda\to\infty}\lambda P_{12}(\lambda;x,-t)\\
&=2\ii\lim_{\lambda\to\infty}\lambda P_{12}(-\lambda^*;x,t)^*\\
&=-2\ii\lim_{\lambda\to\infty}\left[\lambda^* P_{12}(\lambda^*;x,t)\right]^*\\
&=\left[2\ii\lim_{\lambda\to\infty}\lambda P_{12}(\lambda;x,t)\right]^*\\
&=q(x,t;\mathbf{Q}^{-s},M)^*.
\end{split}
\end{equation}
This completes the proof of Proposition~\ref{prop:symmetry}.
\end{proof}

\subsection{Continuation of $u(\chi,\tau)$ to $\overline{\exterior\cup\shelves}$:  Proof of Proposition~\ref{prop:u}}
\begin{proof}[Proof of Proposition~\ref{prop:u}]
We first examine $P(u;\chi,\tau)$ near the positive $\chi$ and $\tau$ axes.  
\begin{lemma}
Fix $\chi>0$.  Then for $\tau>0$ sufficiently small there exists a unique and simple real root of $P(u;\chi,\tau)$.
\label{lem:tau-small}
\end{lemma}
\begin{proof}
Since the simple root at $u=\chi$ for $\tau=0$ persists for small $\tau$, we need to show that the four roots of $P(u;\chi,0)$ near $u=\tfrac{1}{3}\chi$ and the two roots of $P(u;\chi,0)$ near $u=0$ become complex roots of $P(u;\chi,\tau)$ for $\tau\neq 0$ small.  

To study the roots of $P(u;\chi,\tau)$ near $u=\tfrac{1}{3}\chi$, we set $u=\tfrac{1}{3}\chi + \Delta u$ and then express $P(u;\chi,\tau)$ in terms of $\Delta u$:
\begin{multline}
27 P(\tfrac{1}{3}\chi+\Delta u;\chi,\tau)=2187\Delta u^7 - 243(3\chi^2-8\tau^2)\Delta u^5-162\chi^3\Delta u^4 \\{}+ 216\tau^2(54-3\chi^2+2\tau^2)\Delta u^3
-144\chi^3\tau^2\Delta u^2-144\chi^2\tau^4\Delta u-32\chi^3\tau^4.
\end{multline}
For small $\tau$, the dominant balance in which $\Delta u$ is also small is $\Delta u=\tau\zeta$ for a new unknown $\zeta=O(1)$ as $\tau\to 0$.  Then we find that we can divide by $\tau^4$ for $\tau\neq 0$ and obtain
\begin{equation}
27 \tau^{-4}P(\tfrac{1}{3}\chi +\tau\zeta;\chi,\tau)=-2\chi^3 (9\zeta^2+4)^2 + O(\tau),\quad\tau\to 0,\quad\tau\neq 0.
\end{equation}
So, to leading order, we have double purely imaginary roots at $\zeta=\pm\tfrac{2}{3}\ii$, which can split apart at higher order in $\tau$.  This shows that the four roots of $P(u;\chi,\tau)$ near $u=\tfrac{1}{3}\chi$ for small $\tau$ have nonzero imaginary parts.  

To study the roots of $P(u;\chi,\tau)$ near $u=0$ we observe that the dominant balance in $P(u;\chi,\tau)=0$ in which $u$ and $\tau$ are both small for $\chi>0$ fixed occurs with $u=\tau\zeta$ with new unknown $\zeta=O(1)$ as $\tau\to 0$.  Dividing by $\tau^2$ after the substitution yields
\begin{equation}
\tau^{-2}P(\tau\zeta;\chi,\tau)=-(\chi^5+ 8\chi(54+\chi^2)\tau^2+16\chi\tau^4)\zeta^2-16\chi^3 + O(\tau),\quad\tau\to 0,\quad\tau\neq 0.
\end{equation}
So, to leading order, we have a purely imaginary pair of simple roots at $\zeta=\pm 4\ii\chi^{-1}$.  This shows that the two roots of $P(u;\chi,\tau)$ near $u=0$ for small $\tau$ have nonzero imaginary parts.
\end{proof}

\begin{lemma}
Fix $\tau>0$.  Then for $\chi>0$ sufficiently small there exists a unique and simple real root of $P(u;\chi,\tau)$.
\label{lem:chi-small}
\end{lemma}
\begin{proof}
It is easy to see that since $P(u;0,\tau)=u^3(81u^4+72\tau^2u^2 + 432\tau^2+16\tau^4)$, for all $\tau>0$, $P(u;0,\tau)$ has a triple root at $u=0$ and no other real roots.  To unfold the triple root for small $\chi$, set $u=\chi \zeta$ and assume that the new unknown $\zeta$ is bounded as $\chi\downarrow 0$.  Thus one finds that one may divide by $\chi^3$ for $\chi\neq 0$ and obtain
\begin{equation}
\chi^{-3}P(\chi \zeta;\chi,\tau)=P_0(\zeta;\tau) + O(\chi^2),\quad\chi\downarrow 0,\quad \chi\neq 0,
\end{equation}
where $P_0$ is a cubic polynomial in $\zeta$:
\begin{equation}
P_0(\zeta;\tau)\defeq (432\tau^2+16\tau^4)\zeta^3-(432\tau^2+16\tau^4)\zeta^2+144\tau^2\zeta-16\tau^2.
\end{equation}
The discriminant of $P_0(\zeta;\tau)$ is proportional to $27\tau^{12}+\tau^{14}$ which vanishes for no $\tau>0$.  Therefore as $\tau$ varies between $\tau=0$ and $\tau=+\infty$, the root configuration of $P_0(\zeta;\tau)$ (i.e., three real roots or one real root with a complex-conjugate pair with nonzero imaginary part) persists for all $\tau>0$.  In the limit $\tau\to\infty$, the dominant terms in $P_0(\zeta;\tau)$ are $16\tau^4(\zeta^3-\zeta^2)$, so there is one real root near $\zeta=1$ and two small roots of size $\zeta=O(\tau^{-1})$.  We may write $\zeta=\tau^{-1}\zeta_1$ to separate them:
\begin{equation}
P_0(\tau^{-1}\zeta_1;\tau)=-16\tau^2(\zeta_1^2+1)+O(\tau),\quad\tau\to\infty,
\end{equation}
so $\zeta_1=\pm\ii + O(\tau^{-1})$ as $\tau\to\infty$.  Therefore, $P_0(\zeta;\tau)$ has a unique simple real root denoted $\zeta(\tau)$ and a conjugate pair of complex roots for all $\tau>0$.  It is easy to see that
\begin{equation}
\zeta(\tau)=\frac{1}{3}+O(\tau^2),\quad\tau\downarrow 0\quad\text{and}\quad
\zeta(\tau)=1+O(\tau^{-2}),\quad\tau\uparrow\infty.
\end{equation}
Since neither $P_0(\tfrac{1}{3};\tau)=-\tfrac{32}{27}\tau^4$ nor $P_0(1;\tau)=128\tau^2$ can vanish for any $\tau>0$, it then follows that $\tfrac{1}{3}<\zeta(\tau)<1$ holds for all $\tau>0$.  
This proves that the triple root of $P(u;0,\tau)$ originates in the limit $\chi\downarrow 0$ as the collision of a conjugate pair of roots and a real simple root having the expansion
\begin{equation}
u(\chi,\tau)=\chi \zeta(\tau)+O(\chi^3),\quad\chi\downarrow 0,\quad\tau>0.
\end{equation}
Since the remaining quartet of complex roots of $P(u;\chi,\tau)$ for $\chi=0$ remains complex for small $\chi$, the proof is finished.
\end{proof}
It then follows via \eqref{eq:eliminate-v}
that 
\begin{equation}
v(\chi,\tau)=2\tau\frac{2\zeta(\tau)-1}{3\zeta(\tau)-1}+O(\chi^2),\quad\chi\downarrow 0,\quad\tau>0,
\end{equation}
and then from \eqref{eq:eliminate-AB},
\begin{equation}
A(\chi,\tau)=O(\chi)\quad\text{and}\quad B(\chi,\tau)^2=\frac{2\zeta(\tau)}{3\zeta(\tau)-1} + O(\chi^2),\quad\chi\downarrow 0,\quad \tau>0.
\end{equation}
Note that $B(0,\tau)^2>1$ for all $\tau>0$, and that $B(0,\tau)^2\to 1$ as $\tau\uparrow +\infty$ while $B(0,\tau)^2\to +\infty$ as $\tau\downarrow 0$.

By Lemma~\ref{lem:tau-small} and Lemma~\ref{lem:chi-small}, $P(u;\chi,\tau)$ has a unique simple real root $u=u(\chi,\tau)$ for $(\chi,\tau)$ in the open first quadrant near each of the coordinate axes.  Next we show that this situation persists throughout $(\mathbb{R}_{>0}\times\mathbb{R}_{>0})\setminus\overline{\channels}$ by studying the resultant $\delta(\chi,\tau)$ of $P(u;\chi,\tau)$ and $P'(u;\chi,\tau)$ with respect to $u$, a polynomial in $(\chi,\tau)$ the zero locus of which detects repeated roots $u$ of $P(u;\chi,\tau)$. 
We consider the renormalized resultant
\begin{equation}
\delta^\mathrm{R}\defeq\frac{\delta(\chi,\tau)}{c \tau^{16}\chi^{6}},
\end{equation}
where $c \defeq 539122498937926189056$ is a constant that factors out of $\delta(\chi,\tau)$ along with the product $\tau^{16}\chi^{6}$. The renormalized resultant $\delta^\mathrm{R}$ is even in $\chi$ and $\tau$ and so can be expressed as
\begin{equation}
\begin{aligned}
\delta^\mathrm{R}(X,T) =& 16 X^{7} + 304 T X^{6} + 24 T (98 T + 1011 ) X^{5} + T (9488 T^2  - 380376 T -19683) X^4\\
&+64 T^2 (332 T^2 - 18009 T + 57645)X^3+ 
384T^2 (68 T^3 - 2553 T^2 + 159246 T - 59049 )X^2\\
&+16384  T^3  (T-54)(T+27)^2 X + 4096 T^3 (T+27)^4,
\end{aligned}
\end{equation}
where $X=\chi^2$ and $T=\tau^2$.
Since we have already shown that $P(u;\chi,\tau)$ has a unique and simple real root for $(\chi,\tau)$ near the coordinate axes, we can study the equation $\delta^\mathrm{R}(\chi^2,\tau^2)=0$ rather than $\delta(\chi,\tau)=0$.  If, as $(\chi,\tau)$ is taken out of one or the other region near the axes where it is known that $P(u;\chi,\tau)$ has a unique real and simple root, $P(u;\chi,\tau)$ does not acquire any repeated roots then in particular it does not acquire any repeated real roots and hence the number of real roots cannot change.  

Therefore, it would be sufficient to prove that the renormalized resultant $\delta^\mathrm{R}(\chi^2,\tau^2)$ does not vanish in the unbounded region $(\chi,\tau)\in(\mathbb{R}_+\times\mathbb{R}_+)\setminus\overline{\channels}$.  With this goal in mind, we view $\delta^\mathrm{R}(X,T)$ as a polynomial in $X$ with coefficients polynomial in $T$.
Then it is easy to see that for $T=\tau^2$ sufficiently large, all of the coefficients of powers of $X$ in $\delta^\mathrm{R}(X,T)$ are positive, so there are no nonnegative roots $X\ge 0$ of $\delta^\mathrm{R}(X,T)$.  Looking on the $T$-axis, we see that $\delta^\mathrm{R}(0,T)=4096T^3(T+27)^4$, which does not vanish for any $T=\tau^2>0$.  Therefore, as $\tau$ is decreased, the only way a positive value of $X>0$ for which $\delta^\mathrm{R}(X,T)=0$ can occur is if first there is a positive repeated root, i.e., a positive value $X>0$ for which both $\delta^\mathrm{R}(X,T)=0$ and $\delta^\mathrm{R}_X(X,T)=0$.  Setting to zero the resultant of the latter two polynomial equations with respect to $X$ gives the condition on $T=\tau^2$ for which there exist repeated roots $X$ (possibly negative or complex) of $\delta^\mathrm{R}(X,T)$.  This condition factors as:
\begin{equation}
T^{16}(T+27)^5(243T+1)^3(64T-27)^3Q_5(T)^2=0,
\label{eq:resultant-of-resultant}
\end{equation}
where $Q_5(T)$ is a quintic polynomial:
\begin{multline}
Q_5(T)\defeq 663552T^5+954511200T^4+829508109289T^3-14696124806763T^2\\
{}+82617806699739T-205891132094649.
\end{multline}
For $T=\tau^2>0$, the first three factors on the left-hand side of \eqref{eq:resultant-of-resultant} are nonzero, and the fourth factor vanishes exactly for $T=T^\sharp\defeq (\tau^\sharp)^2$, i.e., for $\tau=\pm\tau^\sharp$, where $\tau^\sharp$ is defined in \eqref{eq:corner-point}.  So it remains to determine whether $Q_5(T)=0$ holds for any $T>0$.  In fact $Q_5(0)\neq 0$ and $Q_5(T^\sharp)\neq 0$ by direct computation, so we will apply the theory of Sturm sequences (see Definition~\ref{def:Sturm-sequence}) to count the number of real roots $T$ in the intervals $(0,T^\sharp)$ and $(T^\sharp,+\infty)$.
We thus obtain the following sign sequences at the points $T=0$, $T=T^\sharp=(\tau^\sharp)^2$, and $T=\infty$:
\begin{equation}
\begin{split}
\Xi[Q_5](0)&=(-,+,+,-,-,+),\\
\Xi[Q_5](T^\sharp)&=(-,+,+,-,-,+),\\
\Xi[Q_5](+\infty)&=(+,+,-,-,-,+).
\end{split}
\end{equation}
Since $\#(\Xi[Q_5](0))- \#(\Xi[Q_5](T^\sharp))=3-3=0$, by Sturm's theorem (see Theorem~\ref{t:Sturm}) $Q_5(T)$ has no real root in $[0,T^\sharp]$. We also see that $\#(\Xi[Q_5](T^\sharp))- \#(\Xi[Q_5](+\infty))=3-2=1$, which similarly proves that there exists exactly one real root of $Q_5(T)$ in the interval 
$(T^\sharp,+\infty)$; we denote it by $T_1$. One can easily check numerically that $T_1\approx 10.232235>T^\sharp$, which gives $\tau_1\defeq \sqrt{T_1} \approx 3.198786$.  So as $T$ decreases from $T=+\infty$, the first possible bifurcation point at which positive solutions $X$ of $\delta^\mathrm{R}(X,T)=0$ might appear is $T=T_1$.  
%However, the existence of a double root of $X\mapsto\tilde{\rho}(X,T_1)$ is not relevant because this double root is not a positive number.  Indeed, the Sturm sequences of $\tilde{\rho}(X,T_1)$ at $X=0$ and $X=+\infty$ are
%\begin{equation}
%\begin{split}
%\Xi[\tilde{\rho}(\cdot,T_1)](0)&=(+,-,-,+,+,-,-,-),\\
%\Xi[\tilde{\rho}(\cdot,T_1)](+\infty)&=(+,+,-,-,-,-,+,-),
%\end{split}
%\end{equation}
%so since $\#(\Xi[\tilde{\rho}(\cdot,T_1)](0))-\#(\Xi[\tilde{\rho}(\cdot,T_1)](+\infty))=0$ there are no positive solutions $X>0$ of $\tilde{\rho}(X,T_1)=0$ by Sturm's Theorem.  Together with the above analysis for $T>T_1$ and for $T_0<T<T_1$, this proves that there are no solutions to $\tilde{\rho}(X,T)$ for any $(X,T)$ with $X\ge 0$ and $T>T_0$.  

Next, one checks directly that $\delta^\mathrm{R}(X,T^\sharp)$ factors as the product of a quintic polynomial in $X$ with strictly positive coefficients and $(16X-81)^2$.  Referring to \eqref{eq:corner-point}, this means that $X^\sharp=(\chi^\sharp)^2$ is a positive double root of $X\mapsto\delta^\mathrm{R}(X,T^\sharp)$, and that there are no other positive roots.  We now show that this double root splits into a pair of real simple roots as $T$ decreases from $T^\sharp$ and into a pair of complex-conjugate simple roots as $T$ increases from $T^\sharp$.  Indeed, if we write $X=X^\sharp+\Delta X$ and $T=T^\sharp+\Delta T$ for $\Delta X$ and $\Delta T$ small, then the dominant terms in $\delta^\mathrm{R}(X,T)$ are those homogeneous in $(\Delta X,\Delta T)$ of degree $2$ and these terms turn out to be proportional to a perfect square:  $(\Delta X-4\Delta T)^2$.  Therefore $\Delta X=4\Delta T+o(\Delta T)$ as $\Delta T\to 0$.  To split the double root present for $\Delta T=0$ therefore requires continuing the calculation to higher order; for this purpose we write $X=X^\sharp+\Delta X$ with $\Delta X=4\Delta T + \zeta$ and discover that the dominant terms in $\delta^\mathrm{R}(X,T)$ are now proportional to $3645\zeta^2+8192\Delta T^3$.  Setting these to zero gives distinct real solutions for $\zeta$ only if $\Delta T<0$.  This perturbative analysis proves that near $T=T^\sharp$ there only exist positive real solutions $X$ of $\delta^\mathrm{R}(X,T)=0$ for $T\le T^\sharp$, and these roots satisfy
\begin{equation}
X=X^\sharp+4(T-T^\sharp)\pm\sqrt{\tfrac{8192}{3645}}(T^\sharp-T)^\frac{3}{2} + o((T^\sharp-T)^\frac{3}{2}),\quad T\uparrow T^\sharp.
\label{eq:discriminant-roots-near-T0}
\end{equation}
Then, since we have already shown that there can be no repeated roots of $X\mapsto\delta^\mathrm{R}(X,T)$ for $T^\sharp<T<T_1$, there are no positive roots $X$ at all for $T$ in this range.  Therefore, for $T>T^\sharp$, only for $T=T_1$ is it possible for there to be any positive roots $X$ of $X\mapsto\delta^\mathrm{R}(X,T)$, and no such root can be simple.  Since $\delta^\mathrm{R}(X,T_1)$ is a polynomial in $X$ of degree $7$, for this special value of $T=T_1$ there are at most finitely many positive and necessarily repeated roots $X=X_i>0$, $i\le 3$, corresponding to $\chi_i=\sqrt{X_i}$.  Numerically, one sees that in fact the only repeated root of $X\mapsto\delta^\mathrm{R}(X,T_1)$ (recall that this map must have one or more repeated roots, possibly negative real or complex, by choice of $T_1$) is a positive number $X=X_1\approx31.8597$ corresponding to $\chi_1=\sqrt{X_1}\approx5.64444$ and that there are no other positive roots.

Finally, we consider the range $T<T^\sharp$.  The two simple roots of $X\mapsto\delta^\mathrm{R}(X,T)$ with the expansions \eqref{eq:discriminant-roots-near-T0} cannot coalesce, nor can any new roots appear, for $0<T<T^\sharp$ as has already been shown.  We will show that the two simple roots with the expansions \eqref{eq:discriminant-roots-near-T0} are contained within the domain $\channels$ for all $T\in (0,T^\sharp)$. To show this, we look for simultaneous solutions of the condition \eqref{eq:boundary-curve} describing the boundary of $\channels$, expressed as a polynomial condition in $(X,T)$, and $\delta^\mathrm{R}(X,T)=0$ by computing the resultant with respect to $X$.  The latter resultant is proportional to $T^9(64T-27)^6Q_9(T)$ where $Q_9(T)$ is a ninth-degree polynomial having the Sturm sequences
\begin{equation}
\Xi[Q_9](0)=\Xi[Q_9](T^\sharp)=(+,+,+,-,-,+,+,-,+,+)
\end{equation}
from which it follows by Sturm's theorem that there are no values of $T\in (0,T^\sharp)$ for which roots $X$ of $X\mapsto\delta^\mathrm{R}(X,T)$ can coincide with points of the boundary of $\channels$.  It therefore remains to determine whether the expansions \eqref{eq:discriminant-roots-near-T0} give values of $X$ that lie in the interior of $\channels$.  But near $(X,T)=(X^\sharp,T^\sharp)$ a similar local analysis of the condition \eqref{eq:boundary-curve} as already performed for the condition $\delta^\mathrm{R}(X,T)=0$ shows that \eqref{eq:boundary-curve} only has real solutions $X$ for $T\le T^\sharp$ and that these solutions have the expansions
\begin{equation}
X=X^\sharp+4(T-T^\sharp)\pm\sqrt{\tfrac{8192}{729}}(T^\sharp-T)^\frac{3}{2}+o((T^\sharp-T)^\frac{3}{2}),\quad T\uparrow T^\sharp.
\label{eq:ChannelsBoundaryNearX0T0}
\end{equation}
For $T^\sharp-T$ small and positive, the interior of $\channels$ lies between these latter two curves.
Comparing with \eqref{eq:discriminant-roots-near-T0} we then see that locally the roots $X$ of $X\mapsto\delta^\mathrm{R}(X,T)$ are indeed contained within $\channels$, and this necessarily persists throughout the whole interval $T\in (0,T^\sharp)$.

Therefore, the only points $(\chi,\tau)\in\mathbb{R}_{> 0}\times\mathbb{R}_{> 0}$ in the exterior of $\overline{\channels}$ where the resultant $\delta(\chi,\tau)$ of $P(u;\chi,\tau)$ and $P'(u;\chi,\tau)$ vanishes are $(\chi_i,\tau_1)$, $i\le 3$.  Since by Lemmas~\ref{lem:tau-small} and ~\ref{lem:chi-small} it is known that $P(u;\chi,\tau)$ has a unique real and simple root for points $(\chi,\tau)$ in the exterior sufficiently close to the coordinate axes, and since complex-conjugate roots of $P(u;\chi,\tau)$ are prevented from bifurcating onto the real axis under continuation in $(\chi,\tau)$ unless $\rho(\chi,\tau)$ vanishes, it follows that $P(u;\chi,\tau)$ has a unique real and simple root for all $(\chi,\tau)$ in the part of the open first quadrant exterior to $\overline{\channels}$ with the possible exception of only the points $(\chi_i,\tau_1)$, $i\le 3$. For these exceptional isolated points it can in principle happen that one or more complex-conjugate pairs of roots of $P(u;\chi,\tau)$ coalesce on the real axis, but these are either roots of even multiplicity or in the case of a collision with the simple root they may add an even number to its multiplicity.

Letting $u(\chi,\tau)$ denote the unique real root of odd multiplicity, we extend $u(\chi,\tau)$ to the coordinate axes within $(\mathbb{R}_{>0}\times\mathbb{R}_{>0})\setminus\overline{\channels}$ by continuity:  $u(0,\tau)=0$ for $\tau>0$ and $u(\chi,0)=\chi$ for $\chi>2$.  Note that $u(0,\tau)$ is non-simple root of $P(u;0,\tau)$, but $u(\chi,0)$ is a simple root of $P(u;\chi,0)$.  This completes the proof of Proposition~\ref{prop:u}.
\end{proof}

\begin{remark}
As pointed out earlier, numerics suggest that there is only one positive value of $\chi=\chi_1$ for which there are repeated roots of $P(u;\chi,\tau)$ for $\tau=\tau_1$.  Numerical calculations also show that there are two repeated roots of $P(u;\chi_1,\tau_1)$ forming a complex-conjugate pair.  Therefore $P(u;\chi_1,\tau_1)$ also has just one real root and it is simple.  Thus apparently there is just one exceptional point, and in fact it is not really exceptional after all.
\end{remark}

%\textcolor{red}{[That's the end of the discussion about the root $u(\chi,\tau)$.  ]}
%
%In fact, the same is true throughout the open first quadrant provided that $\chi^2+\tau^2$ is sufficiently large:
%\begin{lemma}
%Given $\delta>0$ arbitrarily small, there exists $r>0$ that $P(u;\chi,\tau)$ has exactly one real and simple root if $\chi>0$, $\tau>\delta\chi$, and $\chi^2+\tau^2>r^2$.
%\label{lem:large-radius}
%\end{lemma}
%\begin{proof}
%We express the unknown $u$ in the form $u=\chi \zeta$.  Then we express $(\chi,\tau)$ in polar coordinates via $\chi=r\cos(\theta)$ and $\tau=r\sin(\theta)$ and compute:
%\begin{equation}
%\frac{P(r\cos(\theta) \zeta;r\cos(\theta),r\sin(\theta))}{r^7\cos^3(\theta)}=Q_0(\zeta;\theta)+\frac{\sin^2(\theta)}{r^2}Q_1(\zeta), 
%\label{eq:P-polar}
%\end{equation}
%in which 
%\begin{equation}
%\begin{split}
%Q_0(\zeta;\theta)&:=(\zeta-1)\zeta^2\left((9\zeta^2-6\zeta-3)\cos^2(\theta)+4\right)^2\\
%Q_1(\zeta)&:=16(3\zeta-1)^3.
%\end{split}
%\end{equation}
%%In continuing $U$ from $U=1$ for $\theta=0$ with $r>2$, we equate to zero only the factor in square brackets in \eqref{eq:P-polar}.  
%We see that in the limit $r\to\infty$, there exists a unique simple root $\zeta=1$, and double roots at $\zeta=0$ and at the roots of the quadratic $(9\zeta^2-6\zeta-3)\cos^2(\theta)+4$.  The roots of this quadratic have imaginary parts $\pm\tfrac{2}{3}\tan(\theta)$ which are bounded away from zero uniformly for $\tan(\theta)=\tau/\chi\ge\delta>0$, so the only issue in allowing $r$ to decrease for given positive angle $\theta$ is that the double root at $\zeta=0$ might split into a pair of real roots.  
%
%To study the limit $r\to\infty$ with $\zeta$ small, one sees that a dominant balance arises by scaling $\zeta=r^{-1}\sin(\theta)\zeta_1$ for $\zeta_1=O(1)$, and then from \eqref{eq:P-polar} we find that
%\begin{equation}
%\frac{P(\cos(\theta)\sin(\theta)\zeta_1;r\cos(\theta),r\sin(\theta)}{r^5\cos^3(\theta)\sin^2(\theta)}=
%-(4-3\cos^2(\theta))^2\zeta_1^2-16 + O(r^{-1}\sin(\theta)),\quad r\to\infty,
%\end{equation}
%so to leading order $\zeta_1$ is a purely imaginary conjugate pair:  $\zeta_1=\pm 4\ii(4-3\cos^2(\theta))^{-1} + O(r^{-1}\sin(\theta))$.  This proves that the double root at $u=0$ for $r=\infty$ becomes purely imaginary as $r$ is decreased.  
%
%Therefore, the lemma is proved.  We can easily find from this calculation that 
%as $r\to\infty$, the simple root $\zeta$ of \eqref{eq:P-polar} has an asymptotic expansion in ascending powers of $\sin^2(\theta)/r^2$:
%\begin{equation}
%\zeta=1-8r^{-2}\sin^2(\theta) + O(r^{-4}\sin^4(\theta)),\quad r\to\infty
%\end{equation}
%which is uniformly valid for $0\le\theta\le\tfrac{1}{2}\pi$.
%\end{proof}
%
%It follows from Lemma~\ref{lem:large-radius} that also
%\begin{equation}
%\begin{split}
%u=Ur\cos(\theta)&=r\cos(\theta)-8r^{-1}\sin^2(\theta)\cos(\theta)+O(r^{-3}\cos(\theta)\sin^4(\theta))\\
%v=2r\sin(\theta)\frac{2u-r\cos(\theta)}{3u-r\cos(\theta)}&=r\sin(\theta)\left(1-4r^{-2}\sin^2(\theta)+O(r^{-4}\sin^4(\theta))\right)\\
%A=\frac{u-r\cos(\theta)}{2r\sin(\theta)}&=-2r^{-2}\sin(2\theta)+O(r^{-4}\cos(\theta)\sin^3(\theta))\\
%B^2&=1-4r^{-2}\cos(2\theta)+O(r^{-4}\sin^2(\theta))
%\end{split}
%\end{equation}
%in the same uniform sense of convergence for large $r$.  It also follows that
%\begin{equation}
%\lambda_0=A+\ii B = \ii - 2\ii r^{-2}\ee^{-2\ii\theta} + O(r^{-4}),\quad r\to\infty.
%\end{equation}
%This shows that for large $r$ we have $A=\mathrm{Re}(\lambda_0)<0$ for $0<\theta<\tfrac{1}{2}\pi$.
%
%\textcolor{red}{[The following calculation is probably made obsolete by the preceding calculation.]}
%In particular, one can see easily that similar expansions as indicated above for small $\tau$ are also valid for $\chi^2+\tau^2$ large.  Indeed, if one writes $u=\chi U$ and assumes that $U$ is bounded as $\tau\to\infty$ with $\xi:=\chi/\tau>0$ fixed, then 
%\begin{equation}
%P(\chi U;\chi,\tau)=P(\xi\tau U;\xi\tau,\tau)=(U-1)U^2\xi^3(4+(9U^2-6U+1)\xi^2)^2\tau^7 + O(\tau^5)
%\end{equation}
%so $U=1+O((\chi^2+\tau^2)^{-1})$, or equivalently
%\begin{equation}
%u(\chi,\tau)=\chi + O(\tau^{-1}),\quad\tau\to\infty,\quad \xi=\frac{\chi}{\tau}\ge\delta>0.
%\end{equation}
%Using this result in \eqref{eq:eliminate-v} then gives
%\begin{equation}
%v(\chi,\tau)=\tau + O(\tau^{-1}),\quad\tau\to\infty,\quad\xi=\frac{\chi}{\tau}\ge\delta>0.
%\end{equation}
%From \eqref{eq:eliminate-AB} we then also find that
%\begin{equation}
%A(\chi,\tau)=O(\tau^{-2})\quad\text{and}\quad B(\chi,\tau)^2=1+O(\tau^{-2}),\quad\tau\to\infty,\quad
%\xi=\frac{\chi}{\tau}\ge\delta>0.
%\end{equation}
%Finally, since the above results used division by $\xi$, we should investigate the solution in the limit $\chi\downarrow 0$.  
%
%\textcolor{red}{These are the small-$\chi$ expansions for $v$, $A$, and $B^2$ following from the proof of Lemma~\ref{lem:chi-small}.}


\section{Some Useful Facts About Polynomials with Real Coefficients}
\label{A:Sturm}
We remind the reader that the discriminant $\Delta_f$ of a polynomial $f(z)= a_n z^n + a_{n-1} z^{n-1} + \cdots + a z+ a_0$, $n\geq 1$, $a_n\neq 0$, with roots (counted with multiplicity) $\xi_1, \xi_2,\dots,\xi_n\in\mathbb{C}$ can be expressed as
\begin{equation}
\Delta_f = a_n^{2n-2}\prod_{1 \leq j < k \leq n}\left(\xi_{j}-\xi_{k}\right)^{2}.
\label{eq:discriminant-roots}
\end{equation}
We assume that the coefficients $a_k$, $k=1,\dots,n$ of the polynomial $f$ are real in the rest of this appendix. In this case, the representation \eqref{eq:discriminant-roots} provides information about the number of non-real roots of $f$. Since the non-real roots of $f$ come in complex conjugate pairs, it is seen from \eqref{eq:discriminant-roots} that $\Delta_f>0$ if and only if $f$ has all distinct real roots or the number of non-real roots are a multiple of $4$. On the other hand, in case $n\geq 2$, $\Delta_f<0$ if and only if the number of non-real roots of $f$ is $2~\mathrm{mod}(4)$. 

The following method is useful for obtaining information about the real roots of a univariate polynomial $f(z)$. We first give a definition (\cite{Sturm1829}, see also \cite[Section 1.3]{Sturmfels02}).
\begin{definition}[Sturm sequence]
Given a polynomial $f(z)$ of degree $n$, define polynomials $f_k(z)$, $k=0,1,2,\dots$ by
\begin{equation}
\begin{split}
f_0(z) &:= f(z),\\
f_1(z) &:= f'(z),\\
f_k(z) &:= -\rem(f_{k-2}(z), f_{k-1}(z)),\quad \text{for $k\geq 2$},
\end{split}
\end{equation}
where $\rem(f_{k-2}(z), f_{k-1}(z))$ denotes the remainder arising in the division of $f_{k-2}(z)$ by $f_{k-1}(z)$. For sufficiently large $k$ we have $f_k(z)\equiv 0$, so let $m$ be the index of the last non-trivial polynomial $f_m(z)$. The \emph{Sturm sequence} of $f(z)$ is the finite sequence of polynomials $(f_0(z), f_1(z),\ldots, f_m(z))$, where necessarily $m\leq n = \deg(f)$.
\label{def:Sturm-sequence}
\end{definition}
We denote by $\Xi[f](a)$ the sequence of \emph{signs} of the Sturm sequence of $f(z)$ evaluated at a point $a\in\mathbb{R}$:
\begin{equation}
\Xi[f](a) := (\sign(f_0(a)),\sign(f_1(a)),\sign(f_2(a)),\ldots, \sign(f_m(a)) ),
\end{equation}
and we let $\#(\Xi[f](a))$ denote the number of sign variations in $\Xi[f](a)$, i.e., the number of sign changes ignoring any zeros when counting. For instance, for $f(z)=4z^3 + z^2 -2$, we have the Sturm sequence
\begin{align}
f_0(z)=4z^3 + z^2 -2,\quad
f_1(z):=12 z^2 +2z,\quad
f_2(z):= \frac{1}{18}z +2,\quad
f_3(z):=-15480,\quad
f_4(z):=0,
\end{align}
and hence at $z=4$, for example, we have
\begin{equation}
 \Xi[f](4) = (\sign(-30), \sign(44), \sign(17/9), \sign(-15480))=(-,+,+,-),
\end{equation}
which gives $\#(\Xi[f](-2))=2$. As a more complicated example, we obtain $\#(\Xi[f](a))= 3$ if $\Xi[f](a)=(+,+,0,+,-,0,+,+,0,-)$. The following theorem (\cite{Sturm1829}, see also \cite[Theorem 1.4]{Sturmfels02}) gives an \emph{exact} count of real zeros of $f(z)$ weighted by multiplicity in an interval using $\#(\Xi[f](\cdot))$.
\begin{theorem}[Sturm's Theorem] Suppose that $a<b$ and neither $a$ nor $b$ is a zero of $f(z)$. Then $\#(\Xi[f](a))\geq \#(\Xi[f](b))$, and the number of real zeros, weighted by multiplicity, of the polynomial $f(z)$ in the interval $[a,b]$ is equal to $\#(\Xi[f](a)) - \#(\Xi[f](b))$.
\label{t:Sturm}
\end{theorem}
The theorem also applies to the case where $a=-\infty$ or $b=+\infty$ by considering the asymptotic behavior of the polynomials in the Sturm sequence, which amounts to looking at the signs of the leading coefficients of the polynomials $f_k(z)$ in the Sturm sequence of $f$.

Another result on the real roots of a polynomial is the following:

\begin{theorem}[D\'escartes' Rule of Signs] Let $f(z)=a_n z^n + a_{n-1} z^{n-1} + \cdots + a_1 z + a_0$, $a_n\neq 0$. The number of positive real roots of $f$ is at most the number of sign variations in its coefficient sequence $(a_n,a_{n-1},\ldots, a_1,a_0)$. Moreover, the number of positive real roots of $f$ differs from the the number of sign variations of the coefficients sequence by an even (nonnegative) integer.
\label{t:Descartes}
\end{theorem}


\begin{thebibliography}{99}
\bibitem{AAS09}
N.\@ Akhmediev, A.\@ Ankiewicz, and J.\@ M.\@ Soto-Crespo, ``Rogue waves and rational solutions of the nonlinear Schr\"odinger equation,'' \textit{Phys.\@ Rev.\@ E} \textbf{80}, art.ID 026601, 2009.

\bibitem{BilmanB19}
D. Bilman and R. J. Buckingham, 
``Large-order asymptotics for multiple-pole solitons of the focusing nonlinear Schr\"odinger equation,'' 
\textit{J. Nonlinear Sci.\@} \textbf{29}, 2185--2229, 2019. 

\bibitem{BilmanBW19} D. Bilman, R. J. Buckingham, and D.-S. Wang, ``Large-order asymptotics for multiple-pole solitons of the focusing nonlinear Schr\"odinger equation II: far-field behavior,'' \texttt{arXiv:1911.04327}, 2019. 

\bibitem{BilmanM19} D. Bilman and P. D. Miller, ``A robust inverse scattering transform for the focusing nonlinear Schr\"odinger equation,'' \textit{Comm.\@ Pure Appl.\@ Math.\@} \textbf{72}, 1722--1805, 2019.

\bibitem{BilmanLM20} D. Bilman, L. Ling and P. D. Miller, ``Extreme superposition:  rogue waves of infinite order and the Painlev\'e-III hierarchy,'' \textit{Duke Math.\@ J.\@} \textbf{169}, 671--760, 2020.

\bibitem{BiondiniM17} G. Biondini and D. Mantzavinos, 
``Long-time asymptotics for the focusing nonlinear Schr\"odinger equation with nonzero boundary conditions at infinity and asymptotic stage of modulational instability,'' 
\textit{Comm.\@ Pure Appl.\@ Math.\@} \textbf{70}, 2300--2365, 2017.

\bibitem{BothnerM20} T.\@ Bothner and P.\@ D.\@ Miller, ``Rational solutions of the Painlev\'e-III equation: Large parameter asymptotics,'' \textit{Constr.\@ Approx.\@} \textbf{51}, 123--225, 2020.

\bibitem{BuckinghamJM21} R. J. Buckingham, R. M. Jenkins, and P. D. Miller, ``Talanov self-focusing and its non-generic character,'' in preparation, 2021.

\bibitem{Jenkins58} J. A. Jenkins, \textit{Univalent Functions and Conformal Mapping}, Springer-Verlag, Berlin, 1958.

\bibitem{LiM21} S. Li and P. D. Miller, ``On the Maxwell-Bloch system in the sharp-line limit without solitons,'' in preparation, 2021.

\bibitem{Miller18} P.\@ D.\@ Miller, ``On the increasing tritronqu\'ee solutions of the Painlev\'e-II equation,'' \textit{SIGMA} \textbf{14}, 125, 38 pages, 2018.

\bibitem{Strebel84} K. Strebel, \textit{Quadratic Differentials}, Springer-Verlag, Berlin, 1984.

\bibitem{Sturm1829}
J.\@ C.\@ F.\@ Sturm, ``Analyse d'un m\'emoire sur la r\'esolution des \'equations num\'eriques,'' \textit{Bulletin des Sciences de F\'erussac} \textbf{11}, 419--422, 1928.

\bibitem{Sturmfels02}
B.\@ Sturmfels, \textit{Solving Systems of Polynomial Equations}, \textit{CBMS Regional Conference Series in Mathematics} \textbf{97}, 152 pp., Published for the Conference Board of the Mathematical Sciences, Washington, DC; by the American Mathematical Society, Providence, RI, 2002. ISBN:978-0-8218-3251-6.

\bibitem{Suleimanov17}
B. I. Suleimanov, ``Effect of a small dispersion on self-focusing in a spatially one-dimensional case,''
\textit{JETP Lett.\@} \textbf{106}, 400--405, 2017.

\bibitem{WangYWH17}
L.\@ Wang, C.\@ Yang, J.\@ Wang, and J.\@ He, ``The height of an $n$th-order fundamental rogue wave for the nonlinear Schr\"odinger equation,'' \textit{Phys.\@ Lett.\@ A} \textbf{381},  1714--1718, 2017.

\bibitem{Zhou89}
X. Zhou, ``The Riemann-Hilbert problem and inverse scattering,'' \textit{SIAM J. Math.\@ Anal.\@}, \textbf{20}, 966--986, 1989.
\end{thebibliography}
\end{document}
