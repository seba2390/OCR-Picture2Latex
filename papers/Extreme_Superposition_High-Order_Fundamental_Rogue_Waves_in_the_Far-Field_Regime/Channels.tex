\label{sec:channels}
In this section we prove Theorem~\ref{thm:channels}.  
Our analysis is driven by the sign chart of $\mathrm{Re}(\ii\vartheta(\lambda;\chi,\tau))$ in the $\lambda$-plane.  The function $\lambda\mapsto\mathrm{Re}(\ii\vartheta(\lambda;\chi,\tau))$ is odd with respect to Schwarz reflection $\lambda\mapsto \lambda^*$.  It follows that the whole real $\lambda$-axis is a component of the zero level curve $\mathrm{Re}(\ii\vartheta(\lambda;\chi,\tau))=0$, so all three critical points lie on the zero level.  Since each critical point is simple when $(\chi,\tau)\in\channels$, from each of them a unique  arc of the zero level curve emanates locally into the upper half-plane with a vertical tangent.  From \eqref{eq:vartheta} one sees easily that as $\chi>0$ holds in $\channels$, $\mathrm{Re}(\ii\vartheta(\lambda;\chi,0))$ is negative (resp., positive) for sufficiently large $\lambda$ in the upper (resp., lower) half-plane; on the other hand $\mathrm{Re}(\ii\vartheta(\lambda;\chi,\tau))$ is always positive (resp., negative) near $\lambda=\ii$ (resp., near $\lambda=-\ii$).  From this and the fact that $\mathrm{Re}(\ii\vartheta(\lambda;\chi,\tau))$ is harmonic away from $\lambda=\pm\ii$ it follows that when $\tau=0$ the two arcs of the zero level curve emanating into the upper half-plane from $\lambda=a(\chi,0)$ and $\lambda=b(\chi,0)$ actually coincide and close around the singularity at $\lambda=\ii$.  This structure persists under perturbation for $\tau\neq 0$, as the arc of the zero level curve emanating into the upper half-plane from the newly-born large critical point must tend to $\lambda=\infty$ vertically without intersecting the arc we denote by $\Gamma^+=\Gamma^+(\chi,\tau)$ joining $a(\chi,\tau)$ and $b(\chi,\tau)$ in the upper half-plane. Therefore, for all $(\chi,\tau)\in \channels$, the zero level curve of $\lambda\mapsto\mathrm{Re}(\ii\vartheta(\lambda;\chi,\tau))$ is the disjoint union $\mathbb{R}\sqcup\Gamma^+\sqcup\Gamma^-\sqcup \ell^+\sqcup \ell^-$, where $\ell^+$ denotes the unbounded arc in the upper half-plane emanating from the third critical point that is large when $\tau\neq 0$ is small (we take $\ell^+=\emptyset$ when $\tau=0$) and $\Gamma^-$ and $\ell^-$ are the Schwarz reflections of $\Gamma^+$ and $\ell^+$ respectively.

\subsection{Steepest descent deformation of the Riemann-Hilbert problem}
Since $\overline{\Gamma^+\cup\Gamma^-}$ is a simple closed curve with the points $\lambda=\pm\ii$ in its interior, we use this curve as $\Sigma_\circ$ in the formulation of Riemann-Hilbert Problem~\ref{rhp:rogue-wave-reformulation}.  As $\Sigma_\circ$ has clockwise orientation, we assume that $\Gamma^+$ is oriented from $a$ to $b$ while $\Gamma^-$ is oriented from $b$ to $a$ in the lower half-plane. In the jump condition \eqref{eq:S-jump} for the matrix $\mathbf{S}(\lambda;\chi,\tau,\mathbf{Q}^{-s},M)$ equivalent to $\mathbf{P}(\lambda;x,t,\mathbf{Q}^{-s},M)$ by \eqref{eq:S-from-P}, we factor the matrix $\mathbf{Q}^{-s}$, $s=\pm 1$, as
\begin{equation}
\mathbf{Q}^{-s}=\begin{cases}
2^{\frac{1}{2}\sigma_3}\begin{bmatrix}1 & \tfrac{1}{2}s\\0 & 1\end{bmatrix}\begin{bmatrix}1 & 0\\-s & 1\end{bmatrix},\quad& \lambda\in\Gamma^+,\\
2^{-\frac{1}{2}\sigma_3}\begin{bmatrix}1 & 0\\-\tfrac{1}{2}s & 1\end{bmatrix}\begin{bmatrix}1 & s\\0 & 1\end{bmatrix},\quad& \lambda\in\Gamma^-.
\end{cases}
\label{eq:Q-factorizations}
\end{equation}
Based on these two factorizations, we define a new unknown $\mathbf{W}(\lambda)=\mathbf{W}(\lambda;\chi,\tau,\mathbf{Q}^{-s},M)$ related to $\mathbf{S}(\lambda;\chi,\tau,\mathbf{Q}^{-s},M)$ by first introducing ``lens'' domains $L^\pm$ and $R^\pm$ to the left and right respectively of $\Gamma^\pm$ (so thin as to exclude the points $\pm\ii$ and to support a fixed sign of $\mathrm{Re}(\ii\vartheta(\lambda;\chi,\tau))$) and we let $\Omega^\pm$ denote the domain between $R^\pm$ and the real line.  See Figure~\ref{fig:Channels1}, left panel.  
\begin{figure}[h]
\begin{center}
\includegraphics{Channels1.pdf}
\end{center}
\caption{Left:  for $(\chi,\tau)=(1,0.145)\in \channels$, the regions in the $\lambda$-plane where $\mathrm{Re}(\ii\vartheta(\lambda;\chi,\tau))<0$ (shaded) and $\mathrm{Re}(\ii\vartheta(\lambda;\chi,\tau))>0$ (unshaded), and the curve $\Sigma_\circ=\Gamma^+\cup\Gamma^-$.  The jump contour $\Sigma_\mathrm{c}$ for $\vartheta(\lambda;\chi,\tau)$ is indicated with a red dashed line terminating at the endpoints $\lambda=\pm\ii$.  Critical points of $\vartheta(\lambda;\chi,\tau)$ are shown with black dots.  Also shown are the ``lens'' regions $L^\pm$ and $R^\pm$ lying to the left and right respectively of $\Gamma^\pm$, and the domains $\Omega^\pm$ lying between $R^\pm$ and the real axis and containing the points $\lambda=\pm\ii$.  Right:  the jump contour for $\mathbf{W}(\lambda)$.}
\label{fig:Channels1}
\end{figure}
Then, we define $\mathbf{W}(\lambda)$ by 
%\textcolor{red}{(later, replace $\mathbf{T}$ in this section by $\mathbf{W}$ to match the notation from other sections)}
%\begin{equation}
%\mathbf{T}^{(k)}(\lambda;\chi,\tau):=\mathbf{S}^{(k)}(\lambda;\chi,\tau)\begin{bmatrix}1 & 0\\
%s\omega(\lambda)^{-2s}\ee^{2\ii n\vartheta(\lambda;\chi,\tau)} & 1\end{bmatrix},\quad\lambda\in L^+,
%\label{eq:T-S-L-plus}
%\end{equation}
%\begin{equation}
%\mathbf{T}^{(k)}(\lambda;\chi,\tau):=\mathbf{S}^{(k)}(\lambda;\chi,\tau)2^{\frac{1}{2}\sigma_3}\begin{bmatrix}1 & \tfrac{1}{2}s\omega(\lambda)^{2s}\ee^{-2\ii n\vartheta(\lambda;\chi,\tau)}\\ 0 & 1\end{bmatrix},\quad\lambda\in R^+,
%\label{eq:T-S-R-plus}
%\end{equation}
%\textcolor{red}{The alternate versions of these read
\begin{equation}
\mathbf{W}(\lambda)\defeq\mathbf{S}(\lambda;\chi,\tau,\mathbf{Q}^{-s},M)\begin{bmatrix}1 & 0\\
s\ee^{2\ii M\vartheta(\lambda;\chi,\tau)} & 1\end{bmatrix},\quad\lambda\in L^+,
\label{eq:T-S-L-plus-ALT}
\end{equation}
\begin{equation}
\mathbf{W}(\lambda)\defeq\mathbf{S}(\lambda;\chi,\tau,\mathbf{Q}^{-s},M)2^{\frac{1}{2}\sigma_3}\begin{bmatrix}1 & \tfrac{1}{2}s\ee^{-2\ii M\vartheta(\lambda;\chi,\tau)}\\ 0 & 1\end{bmatrix},\quad\lambda\in R^+,
\label{eq:T-S-R-plus-ALT}
\end{equation}
%}
\begin{equation}
\mathbf{W}(\lambda)\defeq\mathbf{S}(\lambda;\chi,\tau,\mathbf{Q}^{-s},M)2^{\frac{1}{2}\sigma_3},\quad
\lambda\in\Omega^+,
\end{equation}
\begin{equation}
\mathbf{W}(\lambda)\defeq\mathbf{S}(\lambda;\chi,\tau,\mathbf{Q}^{-s},M)2^{-\frac{1}{2}\sigma_3},\quad
\lambda\in\Omega^-,
\end{equation}
%\begin{equation}
%\mathbf{T}^{(k)}(\lambda;\chi,\tau):=\mathbf{S}^{(k)}(\lambda;\chi,\tau)2^{-\frac{1}{2}\sigma_3}\begin{bmatrix} 1 & 0\\-\tfrac{1}{2}s\omega(\lambda)^{-2s}\ee^{2\ii n\vartheta(\lambda;\chi,\tau)} & 1\end{bmatrix},\quad\lambda\in R^-,
%\label{eq:T-S-R-minus}
%\end{equation}
%\begin{equation}
%\mathbf{T}^{(k)}(\lambda;\chi,\tau):=\mathbf{S}^{(k)}(\lambda;\chi,\tau)\begin{bmatrix}
%1 & -s\omega(\lambda)^{2s}\ee^{-2\ii n\vartheta(\lambda;\chi,\tau)}\\ 0 & 1\end{bmatrix},\quad
%\lambda\in L^-,
%\label{eq:T-S-L-minus}
%\end{equation}
%\textcolor{red}{The alternate versions of these read
\begin{equation}
\mathbf{W}(\lambda)\defeq\mathbf{S}(\lambda;\chi,\tau,\mathbf{Q}^{-s},M)2^{-\frac{1}{2}\sigma_3}\begin{bmatrix} 1 & 0\\-\tfrac{1}{2}s\ee^{2\ii M\vartheta(\lambda;\chi,\tau)} & 1\end{bmatrix},\quad\lambda\in R^-,\quad\text{and}
\label{eq:T-S-R-minus-ALT}
\end{equation}
\begin{equation}
\mathbf{W}(\lambda)\defeq\mathbf{S}(\lambda;\chi,\tau,\mathbf{Q}^{-s},M)\begin{bmatrix}
1 & -s\ee^{-2\ii M\vartheta(\lambda;\chi,\tau)}\\ 0 & 1\end{bmatrix},\quad
\lambda\in L^-,
\label{eq:T-S-L-minus-ALT}
\end{equation}
%}
and we take $\mathbf{W}(\lambda)\defeq\mathbf{S}(\lambda;\chi,\tau,\mathbf{Q}^{-s},M)$ whenever $\lambda\in \mathbb{C}\setminus\overline{L^+\cup R^+\cup\Omega^+\cup\Omega^-\cup R^-\cup L^-}$.  Then it is easy to check that $\mathbf{W}(\lambda)$ may be defined for $\lambda\in\Gamma^+\cup\Gamma^-$ to be analytic there, so that $\mathbf{W}(\lambda)$ is analytic in the complement of the jump contour $C_L^+\cup C_R^+\cup I\cup C_R^-\cup C_L^-$ shown in Figure~\ref{fig:Channels1}, right panel.  The jump conditions satisfied by $\mathbf{W}(\lambda)$ on the arcs of this jump contour are then:
%\begin{equation}
%\mathbf{T}^{(k)}_+(\lambda;\chi,\tau)=\mathbf{T}^{(k)}_-(\lambda;\chi,\tau)\begin{bmatrix}1 & 0\\
%-s\omega(\lambda)^{-2s}\ee^{2\ii n\vartheta(\lambda;\chi,\tau)} & 1\end{bmatrix},\quad\lambda\in C_L^+,
%\label{eq:Tjump-channels-CLplus}
%\end{equation}
%\begin{equation}
%\mathbf{T}^{(k)}_+(\lambda;\chi,\tau)=\mathbf{T}^{(k)}_-(\lambda;\chi,\tau)\begin{bmatrix}1 & \tfrac{1}{2}s\omega(\lambda)^{2s}\ee^{-2\ii n\vartheta(\lambda;\chi,\tau)} \\ 0 & 1\end{bmatrix},\quad\lambda\in C_R^+,
%\label{eq:Tjump-channels-CRplus}
%\end{equation}
%\textcolor{red}{The alternate versions of these read
\begin{equation}
\mathbf{W}_+(\lambda)=\mathbf{W}_-(\lambda)\begin{bmatrix}1 & 0\\
-s\ee^{2\ii M\vartheta(\lambda;\chi,\tau)} & 1\end{bmatrix},\quad\lambda\in C_L^+,
\label{eq:Tjump-channels-CLplus-ALT}
\end{equation}
\begin{equation}
\mathbf{W}_+(\lambda)=\mathbf{W}_-(\lambda)\begin{bmatrix}1 & \tfrac{1}{2}s\ee^{-2\ii M\vartheta(\lambda;\chi,\tau)} \\ 0 & 1\end{bmatrix},\quad\lambda\in C_R^+,
\label{eq:Tjump-channels-CRplus-ALT}
\end{equation}
%}
\begin{equation}
\mathbf{W}_+(\lambda)=\mathbf{W}_-(\lambda)2^{\sigma_3},\quad\lambda\in I,
\label{eq:Channels-T-I-jump}
\end{equation}
%\begin{equation}
%\mathbf{T}^{(k)}_+(\lambda;\chi,\tau)=\mathbf{T}^{(k)}_-(\lambda;\chi,\tau)\begin{bmatrix}1 & 0\\
%-\tfrac{1}{2}s\omega(\lambda)^{-2s}\ee^{2\ii n\vartheta(\lambda;\chi,\tau)} & 1\end{bmatrix},\quad
%\lambda\in C_R^-,\quad\text{and}
%\label{eq:Tjump-channels-CRminus}
%\end{equation}
%\begin{equation}
%\mathbf{T}^{(k)}_+(\lambda;\chi,\tau)=\mathbf{T}^{(k)}_-(\lambda;\chi,\tau)\begin{bmatrix} 1 & s\omega(\lambda)^{2s}\ee^{-2\ii n\vartheta(\lambda;\chi,\tau)}\\ 0 & 1\end{bmatrix},\quad\lambda\in C_L^-.
%\label{eq:Tjump-channels-CLminus}
%\end{equation}
%\textcolor{red}{The alternate versions of these read
\begin{equation}
\mathbf{W}_+(\lambda)=\mathbf{W}_-(\lambda)\begin{bmatrix}1 & 0\\
-\tfrac{1}{2}s\ee^{2\ii M\vartheta(\lambda;\chi,\tau)} & 1\end{bmatrix},\quad
\lambda\in C_R^-,\quad\text{and}
\label{eq:Tjump-channels-CRminus-ALT}
\end{equation}
\begin{equation}
\mathbf{W}_+(\lambda)=\mathbf{W}_-(\lambda)\begin{bmatrix} 1 & s\ee^{-2\ii M\vartheta(\lambda;\chi,\tau)}\\ 0 & 1\end{bmatrix},\quad\lambda\in C_L^-.
\label{eq:Tjump-channels-CLminus-ALT}
\end{equation}
%}
It follows from the sign chart of $\mathrm{Re}(\ii\vartheta(\lambda;\chi,\tau))$ as shown in Figure~\ref{fig:Channels1} that as $n\to+\infty$, the jump matrix for $\mathbf{W}(\lambda)$ is an exponentially small perturbation of the identity everywhere on the jump contour except on the interval $I=[a,b]$ and in neighborhoods of its endpoints.  

\subsection{Parametrix construction}
\label{sec:channels-parametrix}
To deal with those jump matrices that are not near-identity, we first construct an \emph{outer parametrix}
$\dot{\mathbf{W}}^{\mathrm{out}}(\lambda)$ by setting
\begin{equation}
\dot{\mathbf{W}}^{\mathrm{out}}(\lambda)\defeq\left(\frac{\lambda-b(\chi,\tau)}{\lambda-a(\chi,\tau)}\right)^{-\ii p\sigma_3},\quad p\defeq\frac{\ln(2)}{2\pi}>0,\quad\lambda\in\mathbb{C}\setminus I.
\label{eq:Channels-Tout}
\end{equation}
Here, the power function is the principal branch, making $\dot{\mathbf{W}}^\mathrm{out}(\lambda)$ analytic in the indicated domain.  Furthermore it is clear that $\dot{\mathbf{W}}_+^\mathrm{out}(\lambda)=\dot{\mathbf{W}}_-^\mathrm{out}(\lambda)2^{\sigma_3}$ holds for $\lambda\in I$, so the jump condition in \eqref{eq:Channels-T-I-jump} is satisfied exactly by the outer parametrix, which also tends to the identity as $\lambda\to\infty$.  However, $\dot{\mathbf{W}}^\mathrm{out}(\lambda)$ is discontinuous near the endpoints of $I$, making the outer parametrix a poor model for $\mathbf{W}(\lambda)$ near these points.  

We can construct \emph{inner parametrices} near $\lambda=a,b$ that locally satisfy the jump conditions for $\mathbf{W}(\lambda)$ exactly.  Let $D_a(\delta)$ and $D_b(\delta)$ be disks of radius $\delta$ centered at $\lambda=a,b$ respectively, where $\delta>0$ is sufficiently small but independent of $n$.  We first define conformal coordinates $f_a(\lambda;\chi,\tau)$ and $f_b(\lambda;\chi,\tau)$ in these disks by setting 
\begin{equation}
f_a(\lambda;\chi,\tau)^2=2[\vartheta_a(\chi,\tau)-\vartheta(\lambda;\chi,\tau)]\quad\text{and}\quad
f_b(\lambda;\chi,\tau)^2=2[\vartheta(\lambda;\chi,\tau)-\vartheta_b(\chi,\tau)],
\end{equation}
where $\vartheta_a(\chi,\tau)\defeq\vartheta(a(\chi,\tau);\chi,\tau)$ and $\vartheta_b(\chi,\tau)\defeq\vartheta(b(\chi,\tau);\chi,\tau)$, 
and then taking analytic square roots in each case so that the inequalities $f_a'(a(\chi,\tau);\chi,\tau)<0$ and $f_b'(b(\chi,\tau);\chi,\tau)>0$ both hold.  This is possible because $a$ and $b$ are simple critical points of $\vartheta(\lambda;\chi,\tau)$, with $\vartheta''_a(\chi,\tau)\defeq\vartheta''(a(\chi,\tau);\chi,\tau)<0$ and $\vartheta''_b(\chi,\tau)\defeq\vartheta''(b(\chi,\tau);\chi,\tau)>0$.  In fact, one has the formul\ae\
\begin{equation}
f_a'(a(\chi,\tau);\chi,\tau)=-\sqrt{-\vartheta''_a(\chi,\tau)}\quad\text{and}\quad
f_b'(b(\chi,\tau);\chi,\tau)=\sqrt{\vartheta''_b(\chi,\tau)}.
\label{eq:Channels-fafb-Derivs}
\end{equation}
Next, define $M$-independent holomorphic matrix valued functions in $D_a(\delta)$ and $D_b(\delta)$ by
\begin{equation}
\mathbf{H}^a(\lambda)\defeq\left(\frac{f_a(\lambda;\chi,\tau)}{a(\chi,\tau)-\lambda}\right)^{-\ii p\sigma_3}(b(\chi,\tau)-\lambda)^{-\ii p\sigma_3}(\ii\sigma_2),\quad\lambda\in D_a(\delta)
\label{eq:Channels-Ha}
\end{equation}
and
\begin{equation}
\mathbf{H}^b(\lambda)\defeq\left(\frac{f_b(\lambda;\chi,\tau)}{\lambda-b(\chi,\tau)}\right)^{\ii p\sigma_3}(\lambda-a(\chi,\tau))^{\ii p\sigma_3},\quad\lambda\in D_b(\delta).  
\label{eq:Channels-Hb}
\end{equation}
Note that in both cases, the diagonal prefactor is an analytic function nonvanishing in the relevant disk for $\delta$ sufficiently small.  In particular,
\begin{equation}
\begin{split}
\mathbf{H}^a(a(\chi,\tau))&=\left(-f_a'(a(\chi,\tau);\chi,\tau)\right)^{-\ii p\sigma_3}(b(\chi,\tau)-a(\chi,\tau))^{-\ii p\sigma_3}(\ii\sigma_2)\\
&=\left(-\vartheta_a''(\chi,\tau)\right)^{-\frac{1}{2}\ii p\sigma_3}(b(\chi,\tau)-a(\chi,\tau))^{-\ii p\sigma_3}(\ii\sigma_2)
\end{split}
\label{eq:Channels-Ha-center}
\end{equation}
and
\begin{equation}
\begin{split}
\mathbf{H}^b(b(\chi,\tau))&=f_b'(b(\chi,\tau);\chi,\tau)^{\ii p\sigma_3}(b(\chi,\tau)-a(\chi,\tau))^{\ii p\sigma_3}\\
&=\vartheta''_b(\chi,\tau)^{\frac{1}{2}\ii p\sigma_3}(b(\chi,\tau)-a(\chi,\tau))^{\ii p\sigma_3},
\end{split}
\label{eq:Channels-Hb-center}
\end{equation}
where on the second line in each case we used \eqref{eq:Channels-fafb-Derivs}.
Letting 
%$\zeta_{a,b}=n^\frac{1}{2}f_{a,b}(\lambda;\chi,\tau)$ 
%\textcolor{red}{(alternately 
$\zeta_{a,b}=M^{\frac{1}{2}}f_{a,b}(\lambda;\chi,\tau)$
%)}
denote rescalings of the conformal coordinates, 
we then define the inner parametrices by setting
%\begin{multline}
%\dot{\mathbf{T}}^a(\lambda;\chi,\tau):=n^{-\frac{1}{2}\ii p\sigma_3}\ee^{-\ii n\vartheta_a(\chi,\tau)\sigma_3}\omega(\lambda)^{s\sigma_3}\ii^{\frac{1}{2}(1-s)\sigma_3}\mathbf{H}^a(\lambda;\chi,\tau)\\
%{}\cdot\mathbf{U}(\zeta_a)(\ii\sigma_2)^{-1}\ii^{-\frac{1}{2}(1-s)\sigma_3}\omega(\lambda)^{-s\sigma_3}\ee^{\ii n\vartheta_a(\chi,\tau)\sigma_3},\quad\lambda\in D_a
%\label{eq:Channels-Ta}
%\end{multline}
%and
%\begin{multline}
%\dot{\mathbf{T}}^b(\lambda;\chi,\tau):=n^{\frac{1}{2}\ii p\sigma_3}\ee^{-\ii n\vartheta_b(\chi,\tau)\sigma_3}\omega(\lambda)^{s\sigma_3}\ii^{\frac{1}{2}(1-s)\sigma_3}\mathbf{H}^b(\lambda;\chi,\tau)\\
%{}\cdot
%\mathbf{U}(\zeta_b)\ii^{-\frac{1}{2}(1-s)\sigma_3}\omega(\lambda)^{-s\sigma_3}\ee^{\ii n\vartheta_b(\chi,\tau)\sigma_3},\quad\lambda\in D_b(\delta).
%\label{eq:Channels-Tb}
%\end{multline}
%\textcolor{red}{The alternate versions of these read
\begin{equation}
\dot{\mathbf{W}}^a(\lambda)\defeq M^{-\frac{1}{2}\ii p\sigma_3}\ee^{-\ii M\vartheta_a(\chi,\tau)\sigma_3}\ii^{\frac{1}{2}(1-s)\sigma_3}\mathbf{H}^a(\lambda)
\mathbf{U}(\zeta_a)(\ii\sigma_2)^{-1}\ii^{-\frac{1}{2}(1-s)\sigma_3}\ee^{\ii M\vartheta_a(\chi,\tau)\sigma_3},\quad\lambda\in D_a(\delta)
\label{eq:Channels-Ta-ALT}
\end{equation}
and
\begin{equation}
\dot{\mathbf{W}}^b(\lambda)\defeq M^{\frac{1}{2}\ii p\sigma_3}\ee^{-\ii M\vartheta_b(\chi,\tau)\sigma_3}\ii^{\frac{1}{2}(1-s)\sigma_3}\mathbf{H}^b(\lambda)
\mathbf{U}(\zeta_b)\ii^{-\frac{1}{2}(1-s)\sigma_3}\ee^{\ii M\vartheta_b(\chi,\tau)\sigma_3},\quad\lambda\in D_b(\delta).
\label{eq:Channels-Tb-ALT}
\end{equation}
%}
Here the factors to the left of $\mathbf{U}(\zeta_{a,b})$ in each case are analytic on the relevant disk and therefore have no effect on the jump conditions, and the matrix function $\mathbf{U}(\zeta)$ is defined in terms of parabolic cylinder functions as the solution of Riemann-Hilbert Problem 5 in \cite{BilmanLM20} (for example; a development of the solution of this problem is given in \cite[Appendix A]{Miller18} taking $\tau=1$ in the notation of that reference).  The main properties of $\mathbf{U}(\zeta)$ that we need to refer to here are
\begin{itemize}
\item $\mathbf{U}(\zeta)$ is analytic for $|\arg(\zeta)|<\tfrac{1}{4}\pi$, $\tfrac{1}{4}\pi<|\arg(\zeta)|<\tfrac{3}{4}\pi$, and $\tfrac{3}{4}\pi<|\arg(\zeta)|<\pi$ (five sectors);
\item $\mathbf{U}(\zeta)$ takes continuous boundary values from each of the five sectors related by jump conditions $\mathbf{U}_+(\zeta)=\mathbf{U}_-(\zeta)\mathbf{V}^\mathrm{PC}(\zeta)$, where $\mathbf{V}^\mathrm{PC}(\zeta)$ is defined in terms of the exponentials $\ee^{\pm\ii\zeta^2}$ on the five complementary oriented boundary rays as shown in \cite[Figure 9]{BilmanLM20};
\item $\mathbf{U}(\zeta)$ has uniform asymptotics in all directions of the complex plane given by
\begin{equation}
\mathbf{U}(\zeta)\zeta^{\ii p\sigma_3}=\mathbb{I}+\frac{1}{2\ii\zeta}\begin{bmatrix}0 & \alpha\\-\beta & 0\end{bmatrix}+\begin{bmatrix}O(\zeta^{-2}) & O(\zeta^{-3})\\O(\zeta^{-3}) & O(\zeta^{-2})\end{bmatrix},\quad\zeta\to\infty,
\label{eq:PCU-asymp}
\end{equation}
where
\begin{equation}
\alpha\defeq\frac{2^\frac{3}{4}\sqrt{2\pi}}{\Gamma(\ii p)}\ee^{\frac{1}{4}\ii\pi}\ee^{2\pi\ii p^2}
= \sqrt{\frac{\ln(2)}{\pi}}\ee^{\ii(\frac{1}{4}\pi+2\pi p^2-\arg(\Gamma(\ii p)))},\quad\beta\defeq -\alpha^*.
\label{eq:Channels-alpha-beta}
\end{equation}
\end{itemize}
In particular, the analyticity and jump conditions satisfied by $\mathbf{U}(\zeta)$ imply that the inner parametrices $\dot{\mathbf{W}}^a(\lambda)$ and $\dot{\mathbf{W}}^b(\lambda)$ exactly satisfy the jump conditions for $\mathbf{W}(\lambda)$ within their respective disks of definition (here we assume that the jump contours for $\mathbf{W}(\lambda)$ within each disk have been deformed to agree with preimages under $\lambda\mapsto \zeta_{a,b}$ of the straight rays across which $\mathbf{U}(\zeta)$ has jump discontinuities).  

A \emph{global parametrix} is then constructed from the outer and inner parametrices as follows:
\begin{equation}
\dot{\mathbf{W}}(\lambda)\defeq\begin{cases}
\dot{\mathbf{W}}^a(\lambda),&\quad\lambda\in D_a(\delta),\\
\dot{\mathbf{W}}^b(\lambda),&\quad\lambda\in D_b(\delta),\\
\dot{\mathbf{W}}^\mathrm{out}(\lambda),&\quad\lambda\in\mathbb{C}\setminus (I\cup D_a(\delta)\cup D_b(\delta)).
\end{cases}
\end{equation}

\subsection{Small norm problem for the error and large-$M$ expansion}
\label{sec:small-norm-channels}
We now compare the (unknown) matrix $\mathbf{W}(\lambda)=\mathbf{W}(\lambda;\chi,\tau,\mathbf{Q}^{-s},M)$ with its global parametrix by defining the \emph{error} as
\begin{equation}
\mathbf{F}(\lambda)\defeq\mathbf{W}(\lambda)\dot{\mathbf{W}}(\lambda)^{-1}.
\end{equation}
Since the parametrix is an exact solution of the Riemann-Hilbert jump conditions for $\mathbf{W}(\lambda)$ within the disks $D_{a,b}(\delta)$ and across the part of $I=[a,b]$ exterior to these disks, 
$\mathbf{F}(\lambda)$ can be extended to an analytic function of $\lambda\in\mathbb{C}$ with the exception of the arcs of $C_L^\pm$ and $C_R^\pm$ lying outside of the disks $D_{a,b}(\delta)$, and the boundaries $\partial D_{a,b}(\delta)$, which we take to have clockwise orientation.  Because $\delta$ is fixed 
%as $n\to+\infty$ 
%\textcolor{red}{(alternately 
as $M\to+\infty$,
%)}, 
and since $\dot{\mathbf{W}}^\mathrm{out}(\lambda)$ is independent of 
%$n$ \textcolor{red}{(alternately 
$M$,
%)}, 
there is a positive constant $\nu>0$ such that 
%$\mathbf{F}^{(k)}_+(\lambda;\chi,\tau)=\mathbf{F}^{(k)}_-(\lambda;\chi,\tau)(\mathbb{I}+O(\ee^{-Kn}))$ \textcolor{red}{(alternately, replace with $O(\ee^{-KM})$)} 
$\mathbf{F}_+(\lambda)=\mathbf{F}_-(\lambda)(\mathbb{I}+O(\ee^{-\nu M}))$ holds uniformly on the jump contour for $\mathbf{F}(\lambda)$ except on the circles $\partial D_{a,b}(\delta)$.  On the circles, we calculate the jump matrix for $\mathbf{F}(\lambda)$ as follows:
\begin{equation}
\mathbf{F}_+(\lambda)=\mathbf{F}_-(\lambda)\cdot\dot{\mathbf{W}}^{a,b}(\lambda)\dot{\mathbf{W}}^\mathrm{out}(\lambda)^{-1},\quad\lambda\in\partial D_{a,b}(\delta),
\end{equation}
because $\mathbf{W}(\lambda)$ is continuous across $\partial D_{a,b}(\delta)$.  Now we use the fact that by comparing the definition \eqref{eq:Channels-Tout} of the outer parametrix 
$\dot{\mathbf{W}}^\mathrm{out}(\lambda)$ with the definitions \eqref{eq:Channels-Ha}--\eqref{eq:Channels-Hb} of $\mathbf{H}^a(\lambda)$ and $\mathbf{H}^b(\lambda)$, we have
%\begin{multline}
%\dot{\mathbf{T}}^\mathrm{out}(\lambda;\chi,\tau)\ee^{-\ii n\vartheta_a(\chi,\tau)\sigma_3}\omega(\lambda)^{s\sigma_3}\ii^{\frac{1}{2}(1-s)\sigma_3}(\ii\sigma_2)\\
%{}=n^{-\frac{1}{2}\ii p\sigma_3}\ee^{-\ii n\vartheta_a(\chi,\tau)\sigma_3}\omega(\lambda)^{s\sigma_3}\ii^{\frac{1}{2}(1-s)\sigma_3}\mathbf{H}^a(\lambda;\chi,\tau)\zeta_a^{-\ii p\sigma_3},\quad\lambda\in D_a(\delta)\setminus I,
%\end{multline}
%and
%\begin{multline}
%\dot{\mathbf{T}}^\mathrm{out}(\lambda;\chi,\tau)\ee^{-\ii n\vartheta_b(\chi,\tau)\sigma_3}\omega(\lambda)^{s\sigma_3}\ii^{\frac{1}{2}(1-s)\sigma_3}\\
%{}=n^{\frac{1}{2}\ii p\sigma_3}\ee^{-\ii n\vartheta_b(\chi,\tau)\sigma_3}\omega(\lambda)^{s\sigma_3}\ii^{\frac{1}{2}(1-s)\sigma_3}\mathbf{H}^b(\lambda;\chi,\tau)\zeta_b^{-\ii p\sigma_3},\quad\lambda\in D_b(\delta)\setminus I.
%\end{multline}
%\textcolor{red}{The alternate versions of these read:
\begin{multline}
\dot{\mathbf{W}}^\mathrm{out}(\lambda)\ee^{-\ii M\vartheta_a(\chi,\tau)\sigma_3}\ii^{\frac{1}{2}(1-s)\sigma_3}(\ii\sigma_2)
=M^{-\frac{1}{2}\ii p\sigma_3}\ee^{-\ii M\vartheta_a(\chi,\tau)\sigma_3}\ii^{\frac{1}{2}(1-s)\sigma_3}\mathbf{H}^a(\lambda)\zeta_a^{-\ii p\sigma_3},\\
\lambda\in D_a(\delta)\setminus I
\end{multline}
and
\begin{equation}
\dot{\mathbf{W}}^\mathrm{out}(\lambda)\ee^{-\ii M\vartheta_b(\chi,\tau)\sigma_3}\ii^{\frac{1}{2}(1-s)\sigma_3}
=M^{\frac{1}{2}\ii p\sigma_3}\ee^{-\ii M\vartheta_b(\chi,\tau)\sigma_3}\ii^{\frac{1}{2}(1-s)\sigma_3}\mathbf{H}^b(\lambda)\zeta_b^{-\ii p\sigma_3},\quad
\lambda\in D_b(\delta)\setminus I.
\end{equation}
%}
Therefore, using \eqref{eq:Channels-Ta-ALT} and \eqref{eq:PCU-asymp} and the fact that 
%$\zeta_a=n^\frac{1}{2}f_a(\lambda;\chi,\tau)$ \textcolor{red}{(alternately $\zeta_a=M^\frac{1}{2}f_a(\lambda;\chi,\tau)$)} 
$\zeta_a=M^\frac{1}{2}f_a(\lambda;\chi,\tau)$
while $f_a(\lambda;\chi,\tau)$ is bounded away from zero on $\partial D_a(\delta)$ for $\delta$ sufficiently small independent of 
%$n$ \textcolor{red}{($M$)},
$M$,
%\begin{multline}
%\mathbf{F}^{(k)}_+(\lambda;\chi,\tau)=\mathbf{F}^{(k)}_-(\lambda;\chi,\tau)
%n^{-\frac{1}{2}\ii p\sigma_3}\ee^{-\ii n\vartheta_a(\chi,\tau)\sigma_3}\omega(\lambda)^{s\sigma_3}\ii^{\frac{1}{2}(1-s)\sigma_3}\mathbf{H}^a(\lambda;\chi,\tau)\\
%\cdot 
%\left(\mathbb{I}+\frac{1}{2\ii n^{\frac{1}{2}}f_a(\lambda;\chi,\tau)}\begin{bmatrix}0 & \alpha\\-\beta & 0\end{bmatrix}+\begin{bmatrix}O(n^{-1}) & O(n^{-\frac{3}{2}})\\O(n^{-\frac{3}{2}}) & O(n^{-1})\end{bmatrix}\right)\\
%\cdot\mathbf{H}^a(\lambda;\chi,\tau)^{-1}\ii^{-\frac{1}{2}(1-s)\sigma_3}\omega(\lambda)^{-s\sigma_3}\ee^{\ii n\vartheta_a(\chi,\tau)\sigma_3}n^{\frac{1}{2}\ii p\sigma_3},\quad\lambda\in\partial D_a(\delta).
%\label{eq:Channels-VF-partialDa}
%\end{multline}
%\textcolor{red}{The alternate version of this reads:
\begin{multline}
\mathbf{F}_+(\lambda)=\mathbf{F}_-(\lambda)
M^{-\frac{1}{2}\ii p\sigma_3}\ee^{-\ii M\vartheta_a(\chi,\tau)\sigma_3}\ii^{\frac{1}{2}(1-s)\sigma_3}\mathbf{H}^a(\lambda)\\
\cdot 
\left(\mathbb{I}+\frac{1}{2\ii M^{\frac{1}{2}}f_a(\lambda;\chi,\tau)}\begin{bmatrix}0 & \alpha\\-\beta & 0\end{bmatrix}+\begin{bmatrix}O(M^{-1}) & O(M^{-\frac{3}{2}})\\O(M^{-\frac{3}{2}}) & O(M^{-1})\end{bmatrix}\right)\\
\cdot\mathbf{H}^a(\lambda)^{-1}\ii^{-\frac{1}{2}(1-s)\sigma_3}\ee^{\ii M\vartheta_a(\chi,\tau)\sigma_3}M^{\frac{1}{2}\ii p\sigma_3},\quad\lambda\in\partial D_a(\delta).
\label{eq:Channels-VF-partialDa-ALT}
\end{multline}
%}
Likewise, using \eqref{eq:Channels-Tb-ALT} and the fact that 
%$\zeta_b=n^\frac{1}{2}f_b(\lambda;\chi,\tau)$ \textcolor{red}{(or 
$\zeta_b=M^\frac{1}{2}f_b(\lambda;\chi,\tau)$
%)} 
with $f_b(\lambda;\chi,\tau)$ bounded away from zero on $\partial D_b(\delta)$, 
%\begin{multline}
%\mathbf{F}^{(k)}_+(\lambda;\chi,\tau)=\mathbf{F}^{(k)}_-(\lambda;\chi,\tau)
%n^{\frac{1}{2}\ii p\sigma_3}\ee^{-\ii n\vartheta_b(\chi,\tau)\sigma_3}\omega(\lambda)^{s\sigma_3}\ii^{\frac{1}{2}(1-s)\sigma_3}\mathbf{H}^b(\lambda;\chi,\tau)\\
%\cdot 
%\left(\mathbb{I}+\frac{1}{2\ii n^{\frac{1}{2}}f_b(\lambda;\chi,\tau)}\begin{bmatrix}0 & \alpha\\-\beta & 0\end{bmatrix}+\begin{bmatrix}O(n^{-1}) & O(n^{-\frac{3}{2}})\\O(n^{-\frac{3}{2}}) & O(n^{-1})\end{bmatrix}\right)\\
%\cdot\mathbf{H}^b(\lambda;\chi,\tau)^{-1}\ii^{-\frac{1}{2}(1-s)\sigma_3}\omega(\lambda)^{-s\sigma_3}\ee^{\ii n\vartheta_b(\chi,\tau)\sigma_3}n^{-\frac{1}{2}\ii p\sigma_3},\quad\lambda\in\partial D_b(\delta).
%\label{eq:Channels-VF-partialDb}
%\end{multline}
%\textcolor{red}{The alternate version of this reads:
\begin{multline}
\mathbf{F}_+(\lambda)=\mathbf{F}_-(\lambda)
M^{\frac{1}{2}\ii p\sigma_3}\ee^{-\ii M\vartheta_b(\chi,\tau)\sigma_3}\ii^{\frac{1}{2}(1-s)\sigma_3}\mathbf{H}^b(\lambda)\\
\cdot 
\left(\mathbb{I}+\frac{1}{2\ii M^{\frac{1}{2}}f_b(\lambda;\chi,\tau)}\begin{bmatrix}0 & \alpha\\-\beta & 0\end{bmatrix}+\begin{bmatrix}O(M^{-1}) & O(M^{-\frac{3}{2}})\\O(M^{-\frac{3}{2}}) & O(M^{-1})\end{bmatrix}\right)\\
\cdot\mathbf{H}^b(\lambda)^{-1}\ii^{-\frac{1}{2}(1-s)\sigma_3}\ee^{\ii M\vartheta_b(\chi,\tau)\sigma_3}M^{-\frac{1}{2}\ii p\sigma_3},\quad\lambda\in\partial D_b(\delta).
\label{eq:Channels-VF-partialDb-ALT}
\end{multline}
%}
In particular, it follows that 
%$\mathbf{F}^{(k)}_+(\lambda;\chi,\tau)=\mathbf{F}^{(k)}_-(\lambda;\chi,\tau)(\mathbb{I}+O(n^{-\frac{1}{2}}))$ \textcolor{red}{(or $O(M^{-\frac{1}{2}})$)} 
$\mathbf{F}_+(\lambda)=\mathbf{F}_-(\lambda)(\mathbb{I}+O(M^{-\frac{1}{2}}))$ 
holds uniformly on the compact jump contour for $\mathbf{F}(\lambda)$, which otherwise is analytic and tends to $\mathbb{I}$ as $\lambda\to\infty$.  By small-norm theory for such Riemann-Hilbert problems, it follows that 
%$\mathbf{F}^{(k)}_-(\lambda;\chi,\tau)=\mathbb{I}+O(n^{-\frac{1}{2}})$ 
%\textcolor{red}{(or $O(M^{-\frac{1}{2}})$)} 
$\mathbf{F}_-(\cdot)=\mathbb{I}+O(M^{-\frac{1}{2}})$ 
holds in the $L^2$ sense on the jump contour, in the limit 
%$n\to+\infty$ \textcolor{red}{(or $M\to+\infty$)}. 
$M\to+\infty$.

\subsection{Asymptotic formula for 
%$\psi_k(n\chi,n\tau)$ \textcolor{red}{(or 
$q(M\chi,M\tau;\mathbf{Q}^{-s},M)$
%)} 
for $(\chi,\tau)\in \channels$}
Beginning with \eqref{eq:q-S} and using the facts that $\mathbf{S}(\lambda;\chi,\tau,\mathbf{Q}^{-s},M)=\mathbf{W}(\lambda)=\mathbf{W}(\lambda;\chi,\tau,\mathbf{Q}^{-s},M)$ and $\dot{\mathbf{W}}(\lambda)=\dot{\mathbf{W}}^\mathrm{out}(\lambda)$ both hold for $|\lambda|$ sufficiently large, we obtain the exact formula
%\begin{equation}
%\begin{split}
%\psi_k(n\chi,n\tau)&=2\ii\ee^{-\ii n\tau}\lim_{\lambda\to\infty}\lambda T_{12}^{(k)}(\lambda;\chi,\tau)\\
%&=
%2\ii\ee^{-\ii n\tau}\lim_{\lambda\to\infty}\lambda\left[F^{(k)}_{11}(\lambda;\chi,\tau)\dot{T}^\mathrm{out}_{12}(\lambda;\chi,\tau)+F^{(k)}_{12}(\lambda;\chi,\tau)\dot{T}^\mathrm{out}_{22}(\lambda;\chi,\tau)\right].
%\end{split}
%\label{eq:psi-k-exact-channels}
%\end{equation}
%\textcolor{red}{The alternate version of this formula reads:
\begin{equation}
\begin{split}
q(M\chi,M\tau;\mathbf{Q}^{-s},M)&=2\ii\lim_{\lambda\to\infty}\lambda W_{12}(\lambda)\\
&=
2\ii\lim_{\lambda\to\infty}\lambda\left[F_{11}(\lambda)\dot{W}^\mathrm{out}_{12}(\lambda)+F_{12}(\lambda)\dot{W}^\mathrm{out}_{22}(\lambda)\right].
\end{split}
\label{eq:psi-k-exact-channels-ALT}
\end{equation}
%}
Since $\dot{\mathbf{W}}^\mathrm{out}(\lambda)$ is a diagonal matrix tending to $\mathbb{I}$ as $\lambda\to\infty$, this formula simplifies to
%\begin{equation}
%\psi_k(n\chi,n\tau)=
%2\ii\ee^{-\ii n\tau}\lim_{\lambda\to\infty}\lambda F^{(k)}_{12}(\lambda;\chi,\tau)
%\end{equation}
%\textcolor{red}{The alternate version of this reads:
\begin{equation}
q(M\chi,M\tau;\mathbf{Q}^{-s},M)=
2\ii\lim_{\lambda\to\infty}\lambda F_{12}(\lambda).
\end{equation}
%}
If $\mathbf{V}^\mathbf{F}(\lambda)$ denotes the jump matrix for $\mathbf{F}(\lambda)$, i.e., $\mathbf{F}_+(\lambda)=\mathbf{F}_-(\lambda)\mathbf{V}^\mathbf{F}(\lambda)$ holds on the jump contour $\Sigma_\mathbf{F}$, then it follows from the Plemelj formula that
\begin{equation}
\mathbf{F}(\lambda)=\mathbb{I}+\frac{1}{2\pi\ii}\int_{\Sigma_\mathbf{F}}\frac{\mathbf{F}_-(\eta)(\mathbf{V}^\mathbf{F}(\eta)-\mathbb{I})}{\eta-\lambda}\,\dd\eta,\quad
\lambda\in\mathbb{C}\setminus\Sigma_\mathbf{F},
\label{eq:F-Cauchy-channels}
\end{equation}
and therefore
%\begin{equation}
%\psi_k(n\chi,n\tau)=-\frac{1}{\pi}\ee^{-\ii n\tau}\int_{\Sigma_\mathbf{F}}\left[F_{11-}^{(k)}(\mu;\chi,\tau)V^\mathbf{F}_{12}(\mu;\chi,\tau)+F_{12-}^{(k)}(\mu;\chi,\tau)(V^\mathbf{F}_{22}(\mu;\chi,\tau)-1)\right]\,\dd\mu.
%\end{equation}
%\textcolor{red}{The alternate version of this formula reads:
\begin{equation}
q(M\chi,M\tau;\mathbf{Q}^{-s},M)=-\frac{1}{\pi}\int_{\Sigma_\mathbf{F}}\left[F_{11-}(\eta)V^\mathbf{F}_{12}(\eta)+F_{12-}(\eta)(V^\mathbf{F}_{22}(\eta)-1)\right]\,\dd\eta.
\end{equation}
%}
Since 
%$V^\mathbf{F}_{22}(\lambda;\chi,\tau)-1=O(n^{-1})$ \textcolor{red}{(or $O(M^{-1})$)} 
$V^\mathbf{F}_{22}(\cdot)-1=O(M^{-1})$
holds uniformly on $\Sigma_\mathbf{F}$, as $\Sigma_\mathbf{F}$ is compact we also have 
%$V^\mathbf{F}_{22}(\lambda;\chi,\tau)-1=O(n^{-1})$ \textcolor{red}{(or $O(M^{-1})$)} 
$V^\mathbf{F}_{22}(\cdot)-1=O(M^{-1})$
in $L^2(\Sigma_\mathbf{F})$.  Using that 
%$F_{12-}^{(k)}(\lambda;\chi,\tau)=O(n^{-\frac{1}{2}})$ \textcolor{red}{(or $O(M^{-\frac{1}{2}})$)} 
$F_{12-}(\cdot)=O(M^{-\frac{1}{2}})$
in $L^2(\Sigma_\mathbf{F})$ as well, by Cauchy-Schwarz,
%\begin{equation}
%\psi_k(n\chi,n\tau)=-\frac{1}{\pi}\ee^{-\ii n\tau}\int_{\Sigma_\mathbf{F}}F_{11-}^{(k)}(\mu;\chi,\tau)V^\mathbf{F}_{12}(\mu;\chi,\tau)\,\dd\mu + O(n^{-\frac{3}{2}}).
%\end{equation}
%\textcolor{red}{The alternate version reads:
\begin{equation}
q(M\chi,M\tau;\mathbf{Q}^{-s},M)=-\frac{1}{\pi}\int_{\Sigma_\mathbf{F}}F_{11-}(\eta)V^\mathbf{F}_{12}(\eta)\,\dd\eta + O(M^{-\frac{3}{2}}).
\end{equation}
%}
A similar argument allows $F_{11-}(\eta)$ to be replaced with $1$ at the cost of an error term of the same order.  Indeed, taking a boundary value on $\Sigma_\mathbf{F}$ in \eqref{eq:F-Cauchy-channels} gives for $\varphi(\lambda)\defeq F_{11-}(\lambda)-1$ the integral equation
\begin{equation}
\varphi(\lambda)-\frac{1}{2\pi\ii}\int_{\Sigma_\mathbf{F}}\frac{\varphi(\eta)(V^\mathbf{F}_{11}(\eta)-1)}{\eta-\lambda_-}\,\dd\eta = f(\lambda),\quad\lambda\in\Sigma_\mathbf{F},
\label{eq:phi-integral-equation}
\end{equation}
where 
\begin{equation}
f(\lambda)\defeq \frac{1}{2\pi\ii}\int_{\Sigma_\mathbf{F}}\frac{V_{11}^\mathbf{F}(\eta)-1}{\eta-\lambda_-}\,\dd\eta + \frac{1}{2\pi\ii}\int_{\Sigma_\mathbf{F}}\frac{F_{12-}(\eta)V_{21}^\mathbf{F}(\eta)}{\eta-\lambda_-}\,\dd\eta,\quad\lambda\in\Sigma_\mathbf{F}.
\label{eq:phi-integral-equation-RHS}
\end{equation}
The small-norm theory is fundamentally based on the fact that the Cauchy projection operator 
\begin{equation}
m(\lambda)\mapsto\frac{1}{2\pi\ii}\int_{\Sigma_\mathbf{F}}\frac{m(\eta)\,\dd\eta}{\eta-\lambda_-},\quad\lambda\in\Sigma_\mathbf{F}
\end{equation}
is bounded on $L^2(\Sigma_\mathbf{F})$ with norm depending only on the geometry of the contour $\Sigma_\mathbf{F}$, which is independent of any large parameter.  Since 
%$V_{11}^\mathbf{F}(\lambda;\chi,\tau)-1=O(n^{-1})$ \textcolor{red}{(or $O(M^{-1})$)} 
$V_{11}^\mathbf{F}(\cdot)-1=O(M^{-1})$
in $L^\infty(\Sigma_\mathbf{F})$ it follows easily from \eqref{eq:phi-integral-equation} that $\varphi(\cdot)=O(f(\cdot))$ in $L^2(\Sigma_\mathbf{F})$ as 
%$n\to\infty$ \textcolor{red}{(or as $M\to\infty$)}.  
$M\to\infty$.
Likewise, from \eqref{eq:phi-integral-equation-RHS} we see that $f(\cdot)=O(V^\mathbf{F}_{11}(\cdot)-1)+O(F_{12-}(\cdot)V_{21}^\mathbf{F}(\cdot))$ in $L^2(\Sigma_\mathbf{F})$.  Since 
%$V^\mathbf{F}_{11}(\lambda;\chi,\tau)-1=O(n^{-1})$ \textcolor{red}{(or $O(M^{-1})$)} 
$V^\mathbf{F}_{11}(\cdot)-1=O(M^{-1})$
in $L^\infty(\Sigma_\mathbf{F})$, compactness of $\Sigma_\mathbf{F}$ implies that 
%$V^\mathbf{F}_{11}(\lambda;\chi,\tau)-1=O(n^{-1})$ \textcolor{red}{(or $O(M^{-1})$)} 
$V^\mathbf{F}_{11}(\cdot)-1=O(M^{-1})$
in $L^2(\Sigma_\mathbf{F})$.  Also, since 
%$V_{21}^\mathbf{F}(\lambda;\chi,\tau)=O(n^{-\frac{1}{2}})$ \textcolor{red}{(or $O(M^{-\frac{1}{2}})$)} 
$V_{21}^\mathbf{F}(\cdot)=O(M^{-\frac{1}{2}})$
in $L^\infty(\Sigma_\mathbf{F})$ while 
%$F^{(k)}_{12-}(\lambda;\chi,\tau)=O(n^{-\frac{1}{2}})$ \textcolor{red}{(or $O(M^{-\frac{1}{2}})$)} 
$F_{12-}(\cdot)=O(M^{-\frac{1}{2}})$
in $L^2(\Sigma_\mathbf{F})$, we consequently have 
%$F_{12-}^{(k)}(\lambda;\chi,\tau)V_{21}^\mathbf{F}(\lambda;\chi,\tau)=O(n^{-1})$ \textcolor{red}{(or $O(M^{-1})$)} 
$F_{12-}(\cdot)V_{21}^\mathbf{F}(\cdot)=O(M^{-1})$
in $L^2(\Sigma_\mathbf{F})$ as well.  Therefore 
%$\varphi(\lambda)=F^{(k)}_{11-}(\lambda;\chi,\tau)-1=O(n^{-1})$ \textcolor{red}{(or $O(M^{-1})$)} 
$\varphi(\cdot)=F_{11-}(\cdot)-1=O(M^{-1})$
in $L^2(\Sigma_\mathbf{F})$.  As 
%$V_{12}^\mathbf{F}(\lambda;\chi,\tau)=O(n^{-\frac{1}{2}})$ \textcolor{red}{(or $O(M^{-\frac{1}{2}})$)} 
$V_{12}^\mathbf{F}(\cdot)=O(M^{-\frac{1}{2}})$
in $L^\infty(\Sigma_\mathbf{F})$ and hence also in $L^2(\Sigma_\mathbf{F})$ it then follows by Cauchy-Schwarz that
%\begin{equation}
%\psi_k(n\chi,n\tau)=-\frac{1}{\pi}\ee^{-\ii n\tau}\int_{\Sigma_\mathbf{F}}V^\mathbf{F}_{12}(\mu;\chi,\tau)\,\dd\mu + O(n^{-\frac{3}{2}}).
%\end{equation}
%\textcolor{red}{The alternate version reads:
\begin{equation}
q(M\chi,M\tau;\mathbf{Q}^{-s},M)=-\frac{1}{\pi}\int_{\Sigma_\mathbf{F}}V^\mathbf{F}_{12}(\eta)\,\dd\eta + O(M^{-\frac{3}{2}}).
\end{equation}
%}
The dominant contribution to the integral comes from $\partial D_a(\delta)\cup\partial D_b(\delta)$ where $V^\mathbf{F}_{12}(\cdot)$ is proportional to 
%$n^{-\frac{1}{2}}$ \textcolor{red}{(or $M^{-\frac{1}{2}}$)}, 
$M^{-\frac{1}{2}}$,
while contributions from the rest of $\Sigma_\mathbf{F}$ are uniformly exponentially small.  Therefore, we may modify the integration contour to consist of just two small circles:
%\begin{equation}
%\psi_k(n\chi,n\tau)=-\frac{1}{\pi}\ee^{-\ii n\tau}\int_{\partial D_a(\delta)\cup\partial D_b(\delta)}V^\mathbf{F}_{12}(\mu;\chi,\tau)\,\dd\mu + O(n^{-\frac{3}{2}}).
%\end{equation}
%\textcolor{red}{The alternate version reads:
\begin{equation}
q(M\chi,M\tau;\mathbf{Q}^{-s},M)=-\frac{1}{\pi}\int_{\partial D_a(\delta)\cup\partial D_b(\delta)}V^\mathbf{F}_{12}(\eta)\,\dd\eta + O(M^{-\frac{3}{2}}).
\end{equation}
%}
Now, using the jump conditions \eqref{eq:Channels-VF-partialDa-ALT}--\eqref{eq:Channels-VF-partialDb-ALT} and the fact that $\mathbf{H}^a(\cdot)$ is off-diagonal while $\mathbf{H}^b(\cdot)$ is diagonal, one easily finds that
%\begin{equation}
%V_{12}^\mathbf{F}(\mu;\chi,\tau)=\frac{n^{-\ii p}\ee^{-2\ii n\vartheta_a(\chi,\tau)}(-1)^{\frac{1}{2}(1-s)}}{2\ii n^\frac{1}{2}f_a(\mu;\chi,\tau)}
%\beta\omega(\mu)^{2s}H^a_{12}(\mu;\chi,\tau)^2 + O(n^{-\frac{3}{2}}),\quad
%\mu\in\partial D_a(\delta),
%\end{equation}
%\begin{equation}
%V_{12}^\mathbf{F}(\mu;\chi,\tau)=\frac{n^{\ii p}\ee^{-2\ii n\vartheta_b(\chi,\tau)}(-1)^{\frac{1}{2}(1-s)}}{2\ii n^\frac{1}{2}f_b(\mu;\chi,\tau)}
%\alpha\omega(\mu)^{2s}H^b_{11}(\mu;\chi,\tau)^2 + O(n^{-\frac{3}{2}}),\quad
%\mu\in\partial D_b(\delta).
%\end{equation}
%\textcolor{red}{The alternate versions of these read:
\begin{equation}
V_{12}^\mathbf{F}(\eta)=\frac{M^{-\ii p}\ee^{-2\ii M\vartheta_a(\chi,\tau)}(-1)^{\frac{1}{2}(1-s)}}{2\ii M^\frac{1}{2}f_a(\eta;\chi,\tau)}
\beta H^a_{12}(\eta)^2 + O(M^{-\frac{3}{2}}),\quad
\eta\in\partial D_a(\delta),
\end{equation}
\begin{equation}
V_{12}^\mathbf{F}(\eta)=\frac{M^{\ii p}\ee^{-2\ii M\vartheta_b(\chi,\tau)}(-1)^{\frac{1}{2}(1-s)}}{2\ii M^\frac{1}{2}f_b(\eta;\chi,\tau)}
\alpha H^b_{11}(\eta)^2 + O(M^{-\frac{3}{2}}),\quad
\eta\in\partial D_b(\delta).
\end{equation}
%}
Therefore, since $f_{a,b}(\cdot;\chi,\tau)$ are analytic functions with simple zeros at $a$ and $b$ respectively, a residue calculation gives
%\begin{multline}
%\psi_k(n\chi,n\tau)=\ee^{-\ii n\tau}\frac{(-1)^{\frac{1}{2}(1-s)}}{n^\frac{1}{2}}\left[n^{-\ii p}\ee^{-2\ii n\vartheta_a(\chi,\tau)}\frac{\beta\omega(a(\chi,\tau))^{2s}H_{12}^a(a(\chi,\tau);\chi,\tau)^2}{f_a'(a(\chi,\tau);\chi,\tau)}\right.\\
%\left. {}+n^{\ii p}\ee^{-2\ii n\vartheta_b(\chi,\tau)}\frac{\alpha\omega(b(\chi,\tau))^{2s}H_{11}^b(b(\chi,\tau);\chi,\tau)^2}{f_b'(b(\chi,\tau);\chi,\tau)}\right]+O(n^{-\frac{3}{2}}).
%\end{multline}
%\textcolor{red}{The alternate version reads:
\begin{multline}
q(M\chi,M\tau;\mathbf{Q}^{-s},M)=\frac{(-1)^{\frac{1}{2}(1-s)}}{M^\frac{1}{2}}\left[M^{-\ii p}\ee^{-2\ii M\vartheta_a(\chi,\tau)}\frac{\beta H_{12}^a(a(\chi,\tau))^2}{f_a'(a(\chi,\tau);\chi,\tau)}\right.\\
\left. {}+M^{\ii p}\ee^{-2\ii M\vartheta_b(\chi,\tau)}\frac{\alpha H_{11}^b(b(\chi,\tau))^2}{f_b'(b(\chi,\tau);\chi,\tau)}\right]+O(M^{-\frac{3}{2}}).
\end{multline}
%}
Since $s=\pm 1$,
%Recalling that $s$ is the parity index of $k$ ($s=1$ for $k$ even and $s=-1$ for $k$ odd), 
we then use \eqref{eq:Channels-fafb-Derivs}, \eqref{eq:Channels-Ha-center}--\eqref{eq:Channels-Hb-center}, and \eqref{eq:Channels-alpha-beta} to obtain
%\begin{multline}
%\psi_k(n\chi,n\tau)=\ee^{-\ii n\tau}\frac{(-1)^k}{n^\frac{1}{2}}\sqrt{\frac{\ln(2)}{\pi}}
%\left[
%\ee^{\ii\phi}
%\frac{\ee^{-2\ii n\vartheta_a(\chi,\tau)}\omega(a(\chi,\tau))^{2s}(-\vartheta''_a(\chi,\tau))^{-\ii p}}{(-\vartheta''_a(\chi,\tau))^\frac{1}{2}}\right.\\
%\left.{}+\ee^{-\ii\phi}
%\frac{\ee^{-2\ii n\vartheta_b(\chi,\tau)}\omega(b(\chi,\tau))^{2s}\vartheta''_b(\chi,\tau)^{\ii p}}{\vartheta''_b(\chi,\tau)^\frac{1}{2}}\right]+O(n^{-\frac{3}{2}}),
%\end{multline}
%where $\phi$ is a real angle defined by
%\begin{equation}
%\phi:=-p\ln(n) - 2p\ln(b(\chi,\tau)-a(\chi,\tau))-2\pi p^2-\tfrac{1}{4}\pi+\arg(\Gamma(\ii p)).
%\end{equation}
%\textcolor{red}{The alternate version of these formul\ae\ reads:
\begin{multline}
q(M\chi,M\tau;\mathbf{Q}^{-s},M)=\frac{s}{M^\frac{1}{2}}\sqrt{\frac{\ln(2)}{\pi}}
\left[
\ee^{\ii\phi}
\frac{\ee^{-2\ii M\vartheta_a(\chi,\tau)}(-\vartheta''_a(\chi,\tau))^{-\ii p}}{(-\vartheta''_a(\chi,\tau))^\frac{1}{2}}\right.\\
\left.{}+\ee^{-\ii\phi}
\frac{\ee^{-2\ii M\vartheta_b(\chi,\tau)}\vartheta''_b(\chi,\tau)^{\ii p}}{\vartheta''_b(\chi,\tau)^\frac{1}{2}}\right]+O(M^{-\frac{3}{2}}),
\end{multline}
where, recalling the value of $p$ from \eqref{eq:Channels-Tout}, a real angle $\phi$ is defined by 
\begin{equation}
\phi\defeq -\frac{\ln(2)}{2\pi}\ln(M)-\frac{\ln(2)}{\pi}\ln(b(\chi,\tau)-a(\chi,\tau))-\frac{\ln(2)^2}{2\pi}-\frac{1}{4}\pi+\arg\left(\Gamma\left(\frac{\ii\ln(2)}{2\pi}\right)\right).
\label{eq:phi-def}
\end{equation}
%\eqref{eq:intro-phi-def}.
%\begin{equation}
%\phi:=-p\ln(M) - 2p\ln(b(\chi,\tau)-a(\chi,\tau))-2\pi p^2-\tfrac{1}{4}\pi+\arg(\Gamma(\ii p)).
%\end{equation}
%}
We may further observe that the numerator of each of the fractions in square brackets above has unit modulus, so upon identifying the angles of those phase factors
%introducing two additional real angles angles by \eqref{eq:intro-Phia-Phib},
%\begin{equation}
%\begin{split}
%\Phi_a&:=-2n\vartheta_a(\chi,\tau)+2s\arg(\omega(a(\chi,\tau)))-p\ln(-\vartheta_a''(\chi,\tau))\\
%\Phi_b&:=-2n\vartheta_b(\chi,\tau)+2s\arg(\omega(b(\chi,\tau)))+p\ln(\vartheta_b''(\chi,\tau))
%\end{split}
%\end{equation}
%and then 
%\begin{equation}
%\psi_k(n\chi,n\tau)=\ee^{-\ii n\tau}\frac{(-1)^k}{n^\frac{1}{2}}\sqrt{\frac{\ln(2)}{\pi}}\left[\frac{\ee^{\ii(\Phi_a+\phi)}}{(-\vartheta''_a(\chi,\tau))^{\frac{1}{2}}}+\frac{\ee^{\ii(\Phi_b-\phi)}}{\vartheta_b''(\chi,\tau)^\frac{1}{2}}\right]+O(n^{-\frac{3}{2}}).
%\label{eq:psi-k-channels}
%\end{equation}
%\textcolor{red}{The alternate versions read:
%\begin{equation}
%\begin{split}
%\Phi_a&:=-2M\vartheta_a(\chi,\tau)-p\ln(-\vartheta_a''(\chi,\tau))\\
%\Phi_b&:=-2M\vartheta_b(\chi,\tau)+p\ln(\vartheta_b''(\chi,\tau))
%\end{split}
%\end{equation}
%and then 
%\begin{equation}
%\psi_k(M\chi,M\tau)=\ee^{-\ii M\tau}\frac{(-1)^k}{M^\frac{1}{2}}\sqrt{\frac{\ln(2)}{\pi}}\left[\frac{\ee^{\ii(\Phi_a+\phi)}}{(-\vartheta''_a(\chi,\tau))^{\frac{1}{2}}}+\frac{\ee^{\ii(\Phi_b-\phi)}}{\vartheta_b''(\chi,\tau)^\frac{1}{2}}\right]+O(M^{-\frac{3}{2}}).
%\label{eq:psi-k-channels-ALT}
%\end{equation}
%}
the proof of Theorem~\ref{thm:channels} is complete, with a standard argument to supply the local uniformity of the error estimate for $(\chi,\tau)$ in compact subsets of $\channels$ (which can include points on the positive $\chi$-axis).

\subsection{Simplification for $\tau=0$}
%The proof of Corollary~\ref{cor:rogue-wave-channels} merely amounts to (i) restricting $M>0$ to values corresponding to rogue waves $M=\tfrac{1}{2}k+\tfrac{1}{4}$ and then tie $s$ to the order $k\in\mathbb{Z}_{>0}$ by $s=(-1)^k$, and (ii) accounting for the exponential factor $\ee^{-\ii M\tau}$ mediating between \eqref{eq:q-S} and \eqref{eq:psi-k-S}.
%
The further simplification mentioned at the end of Section~\ref{sec:results-channels}, so that $(\chi,\tau)\in \channels$ with $\tau=0$ means $0<\chi<2$, is accomplished by noting that the phase function $\vartheta(\lambda;\chi,0)$ defined by \eqref{eq:vartheta} is an odd function of $\lambda$ for each $\chi\in (0,2)$, and we recall that the critical points $\lambda=a,b$ in this case are given by \eqref{eq:tau-zero-critical-points}: 
\begin{equation}
b(\chi,0)=\sqrt{\frac{2}{\chi}-1},\quad a(\chi,0)=-b(\chi,0).
\end{equation}
A computation then shows that
\begin{equation}
\vartheta_b(\chi,0)=\vartheta(b(\chi,0);\chi,0)=\chi\sqrt{\frac{2}{\chi}-1}+\pi-2\tan^{-1}\left(\sqrt{\frac{2}{\chi}-1}\right),\quad\vartheta_a(\chi,0)=-\vartheta_b(\chi,0),
\end{equation}
and that
\begin{equation}
\vartheta_b''(\chi,0)=\vartheta''(b(\chi,0);\chi,0)=\chi^2\sqrt{\frac{2}{\chi}-1},\quad\vartheta_a''(\chi,0)=-\vartheta_b''(\chi,0).
\end{equation}
%Similarly,
%\begin{equation}
%2\arg(\omega(b(\chi,0)))=\tan^{-1}\left(\sqrt{\frac{2}{\chi}-1}\right)-\frac{1}{2}\pi,\quad
%2\arg(\omega(a(\chi,0)))=-2\arg(\omega(b(\chi,0))).
%\end{equation}
Therefore, in this special case, the leading term denoted $L^{[\channels]}_k(\chi,\tau)$ in \eqref{eq:leading-term-channels} reduces for $\tau=0$ and $0<\chi<2$ to \eqref{eq:leading-term-channels-tau-zero}.
%\begin{equation}
%\psi_k(n\chi,0)=\frac{1}{n^\frac{1}{2}}A(\chi)\cos(2n\vartheta_b(\chi,0)-p\ln(n)-\Omega(\chi) + \Phi_0)+O(n^{-\frac{3}{2}}),\quad 0<\chi<2,
%\end{equation}
%where
%\begin{equation}
%A(\chi):=\sqrt{\frac{\ln(2)}{\pi}}\frac{2}{\chi^{\frac{3}{4}}(2-\chi)^\frac{1}{4}},\quad 
%\Omega(\chi):=2p\ln(\chi)+3p\ln(b(\chi,0))+s\tan^{-1}(b(\chi,0))
%\end{equation}
%and where $\Phi_0$ is a constant phase given by
%\begin{equation}
%\Phi_0:=\frac{1}{4}\pi-\frac{3\ln(2)^2}{2\pi}+\arg(\Gamma(\ii p)).
%\end{equation}
%\textcolor{red}{The alternate version of the final asymptotic formula can be written as (but note that the expressions for $2\arg(\omega(\lambda))$ at $\lambda=a(\chi,0)$ and $\lambda=b(\chi,0)$ are not needed):
%\begin{equation}
%\psi_k(M\chi,0)=\frac{1}{M^\frac{1}{2}}A(\chi)\cos(2M\vartheta_b(\chi,0)-p\ln(M)-\Omega(\chi) + \Phi_0)+O(M^{-\frac{3}{2}}),\quad 0<\chi<2,
%\end{equation}
%where
%\begin{equation}
%A(\chi):=\sqrt{\frac{\ln(2)}{\pi}}\frac{2}{\chi^{\frac{3}{4}}(2-\chi)^\frac{1}{4}},\quad 
%\Omega(\chi):=2p\ln(\chi)+3p\ln(b(\chi,0))
%\end{equation}
%and where $\Phi_0$ is a constant phase given by
%\begin{equation}
%\Phi_0:=\left(k-\frac{1}{4}\right)\pi-\frac{3\ln(2)^2}{2\pi}+\arg(\Gamma(\ii p)).
%\end{equation}
%}
