
\subsection{Unique determination of $h'(\lambda;\chi,\tau)$ for $(\chi,\tau)\in\exterior\cup\shelves$}
\label{sec:g-function}
Here we show how $(\chi,\tau)\in\overline{\exterior\cup\shelves}$ determines a unique function $h'(\lambda;\chi,\tau)$ of the form \eqref{eq:hprime-formula} that satisfies the residue and asymptotic conditions \eqref{eq:hprime-residues} and \eqref{eq:hprime-expansion} respectively.  
 
We first use \eqref{eq:hprime-expansion} with \eqref{eq:hprime-formula} to explicitly eliminate $A(\chi,\tau)$ and $B(\chi,\tau)^2$ in favor of $u(\chi,\tau)$ and $v(\chi,\tau)$:
\begin{equation}
\begin{split}
A(\chi,\tau)&=\frac{u(\chi,\tau)-\chi}{2\tau},\\
A(\chi,\tau)^2+B(\chi,\tau)^2&=\frac{3u(\chi,\tau)^2}{4\tau^2}-\frac{v(\chi,\tau)}{\tau}+2-\frac{\chi u(\chi,\tau)}{\tau^2}+\frac{\chi^2}{4\tau^2}.
\end{split}
\label{eq:eliminate-AB}
\end{equation}
Then, the residue conditions \eqref{eq:hprime-residues} become $(-2\tau\pm\ii u(\chi,\tau)+v(\chi,\tau))R(\pm\ii;\chi,\tau)=-2$.  Imposing instead the \emph{squares} of these conditions\footnote{Later, getting the signs right for the residues is accomplished by choosing the location of the branch cut $\Sigma_g$ in relation to the points $\lambda=\pm\ii$.  See Remark~\ref{rem:Sigma_g}.} one arrives at two complex-conjugate equations, which amount to two real equations by taking real and imaginary parts.  The real part equation reads $\mathcal{R}=0$, where
\begin{multline}
\mathcal{R}\defeq 3u^4-3u^2v^2-4\chi u^3+4\tau v^3+4\chi uv^2+\chi^2u^2-\chi^2v^2-8\chi\tau uv+8\tau^2u^2-20\tau^2v^2\\{}+4\chi^2\tau v+32\tau^3v-16\tau^4+16\tau^2-4\chi^2\tau^2,
\end{multline}
and the imaginary part equation reads $\mathcal{I}_1\mathcal{I}_2=0$, where
\begin{equation}
\mathcal{I}_1\defeq u^2-\chi u-2\tau v+4\tau^2\quad\text{and}\quad
\mathcal{I}_2\defeq 3uv-4\tau u-\chi v+2\chi\tau.
\end{equation}
Note that for $\tau=0$,
\begin{equation}
\left.\mathcal{R}\right|_{\tau=0}=(u+v)(u-v)(3u-\chi)(u-\chi),\quad
\left.\mathcal{I}_1\right|_{\tau=0}=(u-\chi)u,\quad\text{and}\quad
\left.\mathcal{I}_2\right|_{\tau=0}=(3u-\chi)v,
\end{equation}
so one solution is to choose $u(\chi,0)=\chi$ and $v(\chi,0)=0$.  In order to apply the implicit function theorem to continue this solution to $\tau\neq 0$, it is necessary to discard the factor $\mathcal{I}_1$ and enforce only the conditions $\mathcal{R}=0$ and $\mathcal{I}_2=0$.  Then a calculation shows that the Jacobian is
\begin{equation}
\left.\det\begin{bmatrix} \mathcal{R}_u & \mathcal{R}_v\\
\mathcal{I}_{2u} & \mathcal{I}_{2v}\end{bmatrix}\right|_{\tau=0,u=\chi,v=0}=4\chi^4,
\end{equation}
which is nonzero for $\chi>2$.  Moreover, the equation $\mathcal{I}_2=0$ can be used to explicitly eliminate $v$ by
\begin{equation}
\mathcal{I}_2=0\quad\Leftrightarrow\quad v=2\tau\frac{2u-\chi}{3u-\chi}.
\label{eq:eliminate-v}
\end{equation}
With $v$ eliminated, the equation $\mathcal{R}=0$ reads $P(u;\chi,\tau)=0$, where $P(u;\chi,\tau)$ is the septic polynomial
\begin{multline}
P(u;\chi,\tau)\defeq 81u^7-189\chi u^6 + (162\chi^2+72\tau^2)u^5-(66\chi^2+120\tau^2)\chi u^4 \\
{}+ (13\chi^4+56\chi^2\tau^2+16\tau^4+432\tau^2)u^3 -(\chi^4+8\chi^2\tau^2+16\tau^4+432\tau^2)\chi u^2 \\
{}+ 144\chi^2\tau^2u-16\chi^3\tau^2.
\end{multline}
Note that $P(u;\chi,0)=(u-\chi)(3u-\chi)^4u^2$, so if $\chi>0$, $u(\chi,0)=\chi$ is a simple root hence continuable to $\tau>0$ (sufficiently small, given $\chi>0$) by the implicit function theorem.  In the limit $\tau\downarrow 0$ we can compute as many terms in the Taylor expansion of $u(\chi,\tau)$ about $u(\chi,0)=\chi$ as we like; in particular it is easy to see that
\begin{equation}
u(\chi,\tau)=\chi -\frac{8\tau^2}{\chi^3} + O(\tau^4),\quad\tau\downarrow 0,\quad \chi>2,
\end{equation}
which implies via \eqref{eq:eliminate-v} that 
\begin{equation}
v(\chi,\tau)=\tau-\frac{4\tau^3}{\chi^4}+O(\tau^5),\quad\tau\downarrow 0,\quad\chi>2.
\end{equation}
From \eqref{eq:eliminate-AB} we then also find that
\begin{equation}
A(\chi,\tau)=-\frac{4\tau}{\chi^3}+O(\tau^3)\quad\text{and}\quad
B(\chi,\tau)^2=1-\frac{4}{\chi^2}+O(\tau^2),\quad\tau\downarrow 0,\quad \chi>2.
\label{eq:AB-tau-small}
\end{equation}
In the special case that $\tau=0$ and $\chi>2$, it follows that $u(\chi,0)=\chi$, $v(\chi,0)=0$, $A(\chi,0)=0$ and $B(\chi,0)^2=1-4/\chi^2<1$.
We claim that this solution can be uniquely continued not just locally near $\tau=0$ but also to the entire unbounded exterior region $\exterior$ as well as through its common boundary with the bounded region $\shelves$ into that entire region.  We have the following result, the proof of which can be found in Appendix~\ref{A:Proofs}.
%We will prove the stronger statement that if $\tau>0$ and $\chi>0$ and $(\chi,\tau)\in \shelves$ then there exists a unique real and generically simple root of $P(u;\chi,\tau)$.  
\begin{proposition}
Let $(\chi,\tau)\in \overline{\exterior\cup\shelves}$.  Then $P(u;\chi,\tau)$ has a unique real root of odd multiplicity, denoted $u=u(\chi,\tau)$ with $u(0,\tau)=0$ for $\tau>0$ and $u(\chi,0)=\chi$ for $\chi>2$.  There exists a value $\tau_1>0$ such that except for $\chi=0$ and possibly three or fewer points $(\chi,\tau)$ with $\chi>0$ and $\tau=\tau_1$, $u(\chi,\tau)$ is the only real root of $P(u;\chi,\tau)$ and it is simple.  
\label{prop:u}
\end{proposition}

\begin{remark}
In the case $\tau=0$ and $\chi>2$, $\Sigma_g$ is an arc connecting the two points $\lambda=\pm\ii B(\chi,0)$.  If we take $\Sigma_g$ to be the purely imaginary straight-line segment connecting these points, then from the prescribed large-$\lambda$ asymptotic behavior of $R(\lambda;\chi,0)$ we find that $R(\pm\ii;\chi,0)=\pm 2\ii/\chi$, from which it follows directly via the formula \eqref{eq:hprime-formula} that the residue conditions \eqref{eq:hprime-residues} hold, so the signs of the residues which had been conflated in squaring the residue conditions are indeed correctly resolved with the indicated choice of $\Sigma_g$.  To ensure that this successful resolution is maintained upon continuation of the solution from $\tau=0$ it is then sufficient that $\Sigma_g$ deform continuously with $(\chi,\tau)$ without ever contacting the poles $\lambda=\pm\ii$.  This is feasible because the endpoints $A(\chi,\tau)\pm\ii B(\chi,\tau)$ lie in the left half-plane for all $(\chi,\tau)\in\overline{\exterior\cup\shelves}$ with $\chi>0$ and $\tau>0$; indeed from \eqref{eq:eliminate-AB}, $A(\chi,\tau)=0$ holds if and only if $u(\chi,\tau)=\chi$ and $P(\chi;\chi,\tau)=128\tau^2\chi^3$ which vanishes only on the coordinate axes.  Therefore $A(\chi,\tau)$ has one sign on the interior of $\overline{\exterior\cup\shelves}$, and by \eqref{eq:AB-tau-small} we see that $A(\chi,\tau)<0$.  Note that when $\chi\downarrow 0$ for given $\tau>0$, the proof of Proposition~\ref{prop:u} shows that $B(\chi,\tau)^2$ tends to a value strictly greater than $1$ while $A(\chi,\tau)\to 0$, so in this limiting situation we should choose $\Sigma_g$ to lie in the left half-plane except for its endpoints.
\label{rem:Sigma_g}
\end{remark}

We are now in a position to show that, as claimed in Section~\ref{sec:h-intro}, $B(\chi,\tau)^2>0$ holds for all $(\chi,\tau)\in\exterior\cup\shelves\cup(\partial\exterior\cap\partial\shelves)$ and $B(\chi,\tau)^2=0$ holds on the common boundary of the union with $\channels$. 
Indeed,
expressing $B^2$ explicitly in terms of $u$, $\chi$, and $\tau$ using \eqref{eq:eliminate-AB} and \eqref{eq:eliminate-v}, one finds that $B^2=0$ implies $3u^3-4\chi u^2+(4\tau^2+\chi^2) u=0$.  The resultant between this equation and $P(u;\chi,\tau)$ vanishes on the open quadrant $(\chi,\tau)\in\mathbb{R}_{>0}\times\mathbb{R}_{>0}$ exactly where \eqref{eq:boundary-curve} holds.  



\subsubsection{Critical points of $h'(\lambda;\chi,\tau)$ for $(\chi,\tau)\in\exterior\cup\shelves$}
\label{sec:critical-points}
Observe that while the coefficients $u(\chi,\tau)$ and $v(\chi,\tau)$ in the quadratic factor in the numerator of $h'(\lambda;\chi,\tau)$ defined in \eqref{eq:hprime-formula} depend real-analytically on $(\chi,\tau)\in (\mathbb{R}_{\ge 0}\times\mathbb{R}_{\ge 0})\setminus\overline{\channels}$, the quadratic discriminant vanishes to first order along two curves in this domain so the roots undergo bifurcation upon crossing these curves.  Eliminating $v$ via \eqref{eq:eliminate-v}, the quadratic discriminant $u^2-8\tau v$ is seen to vanish only if $3u^3-\chi u^2-32\tau^2 u+16\chi\tau^2=0$.  The resultant of this cubic polynomial with $P(u;\chi,\tau)$ vanishes for $(\chi,\tau)\in\mathbb{R}_{>0}\times\mathbb{R}_{>0}$ exactly where $D(\chi,\tau)\defeq H_{10}(\chi,\tau)+H_8(\chi,\tau)+H_6(\chi,\tau) + H_4(\tau)=0$, in which the $H_j$ are homogeneous polynomials
\begin{equation}
\begin{split}
H_{10}(\chi,\tau)&\defeq -(8\tau^2-\chi^2)(100\tau^2+\chi^2)^4,\\
H_8(\chi,\tau)&\defeq 2\chi^8+1040\chi^6\tau^2+1741728\chi^4\tau^4-125516800\chi^2\tau^6+730880000\tau^8,\\
H_6(\chi,\tau)&\defeq \chi^6+504\chi^4\tau^2+3103488\chi^2\tau^4+67627008\tau^6,\\
H_4(\tau)&\defeq 1492992\tau^4.
\end{split}
\label{eq:H-polynomials}
\end{equation}
There is one unbounded arc in the first quadrant where this condition holds (see the dotted blue curve in Figure~\ref{fig:RegionsPlot}) and it is governed far from the origin by the highest-order homogeneous terms $H_{10}(\chi,\tau)\approx 0$; this arc is therefore asymptotic to the line $\chi=\sqrt{8}\tau$ (see the dotted gray line in Figure~\ref{fig:RegionsPlot}).  The cusp point $(\chi,\tau)=(\chi^\sharp,\tau^\sharp)$ (see \eqref{eq:corner-point}) on the boundary of $\channels$ is a non-smooth point on the locus $D(\chi,\tau)=0$ because the gradient vector vanishes there as well.  
In fact, setting $(\chi,\tau)=(\chi^\sharp+\Delta\chi,\tau^\sharp+\Delta\tau)$, one computes that
\begin{equation}
D(\chi,\tau)=\frac{3948901875}{256}\left(\frac{2}{\sqrt{3}}\Delta \tau-\Delta \chi\right)^2 + O((\Delta\chi^2+\Delta\tau^2)^\frac{3}{2}).
\end{equation}
The leading terms describe two curves tangent to the line $\Delta\chi=\tfrac{2}{\sqrt{3}}\Delta\tau$, and along this line the cubic correction terms are proportional to $\Delta\tau^3$ by a negative coefficient.  Therefore the two curves both emanate from the cusp point $(\Delta\chi,\Delta\tau)=(0,0)$ along this tangent line in the direction $\Delta\tau>0$, entering the exterior of $\channels$ from the cusp point.  
Since $D(0,\tau)=2048(1-\tau^2)\tau^4(625\tau^2+27)^2$, an arc along which this condition holds exits the quadrant $(\chi,\tau)\in \mathbb{R}_{>0}\times\mathbb{R}_{>0}$ on the $\tau$-axis at the point $(\chi,\tau)=(0,1)$.  This is the arc separating $\shelves$ from $\exterior$, and is shown as a solid blue curve in Figure~\ref{fig:RegionsPlot}.

%\textcolor{red}{[This is not a complete picture, but it gives the most important details.]}

\subsubsection{Construction of $g(\lambda;\chi,\tau)$ when $\Sigma_g\cap\Sigma_\mathrm{c}=\emptyset$}
%\textcolor{red}{(Move to beginning of $\shelves$ proof section?)}
\label{sec:g-function-loop}
Since $h'(\lambda;\chi,\tau)$ is now well-defined for all $(\chi,\tau)\in \exterior\cup\shelves$, we have $g'(\lambda;\chi,\tau)=h'(\lambda;\chi,\tau)-\vartheta'(\lambda;\chi,\tau)$ which has removable singularities at $\lambda=\pm\ii$ according to \eqref{eq:hprime-residues}
and hence is an analytic function for $\lambda\in\mathbb{C}\setminus\Sigma_g$ with, according to \eqref{eq:hprime-expansion}, the asymptotic behavior $g'(\lambda;\chi,\tau)=O(\lambda^{-2})$ as $\lambda\to\infty$.  Since $g'(\lambda;\chi,\tau)$ is integrable at $\lambda=\infty$, the contour integral
\begin{equation}
g(\lambda;\chi,\tau)\defeq\int_\infty^\lambda g'(\eta;\chi,\tau)\,\dd\eta
\label{eq:g-integral}
\end{equation}
is independent of path in the domain $\mathbb{C}\setminus\Sigma_g$ and defines the unique antiderivative analytic in the same domain that satisfies the condition $g(\infty;\chi,\tau)=0$.  It is easy to check that $g(\lambda^*;\chi,\tau)=g(\lambda;\chi,\tau)^*$ holds for each $\lambda\in\Sigma_g$ and $(\chi,\tau)\in \exterior\cup\shelves$.  Obtaining $g(\lambda;\chi,\tau)$ from \eqref{eq:g-integral} is a bit of a subtle calculation, because the integrability at $\eta=\infty$ and $\eta=\pm\ii$ relies on cancellations arising from the equations satisfied by the parameters $u,v,A,B$.  Another approach is to assume that $\Sigma_g$ is determined by solving those equations, and then to note that $g(\lambda;\chi,\tau)$ is a function analytic for $\lambda\in\mathbb{C}\setminus\Sigma_g$ with $g(\lambda;\chi,\tau)=O(\lambda^{-1})$ as $\lambda\to\infty$ and whose boundary values on $\Sigma_g$ satisfy 
\begin{equation}
\begin{split}
g_+(\lambda;\chi,\tau)+g_-(\lambda;\chi,\tau)&=h_+(\lambda;\chi,\tau)+h_-(\lambda;\chi,\tau)-2\vartheta(\lambda;\chi,\tau)\\
&=2 \kappa(\chi,\tau)-2\vartheta(\lambda;\chi,\tau),\quad \lambda\in\Sigma_g,
\end{split}
\label{eq:hpm-kappa}
\end{equation}
for some integration constant $\kappa(\chi,\tau)$ (because the sum of boundary values of $h$ is constant along $\Sigma_g$).  This formula \eqref{eq:hpm-kappa} assumes that $\vartheta(\lambda;\chi,\tau)$ is analytic on the subset $\Sigma_g$ of the jump contour for $\mathbf{S}(\lambda;\chi,\tau,\mathbf{G},M)$.  As the jump contour for $\vartheta(\lambda;\chi,\tau)$ is $\Sigma_\mathrm{c}$, we are assuming that the latter is contained in the interior of the Jordan curve $\Sigma_\circ$, which guarantees that $\Sigma_g\cap\Sigma_\mathrm{c}=\emptyset$.  Another situation, in which $\Sigma_\circ$ is deformed into a dumbbell-shaped jump contour with a Schwarz-symmetric neck that is necessarily a subset of $\Sigma_\mathrm{c}$, will be required to prove Theorem~\ref{thm:exterior}.  We will describe how the procedure needs to be modified for that case in Section~\ref{sec:g-function-dumbbell} below.

Returning to \eqref{eq:hpm-kappa}, to determine the constant $\kappa(\chi,\tau)$ and simultaneously obtain $g(\lambda;\chi,\tau)$ without using \eqref{eq:g-integral}, we represent $g(\lambda;\chi,\tau)$ in the form $g(\lambda;\chi,\tau)=R(\lambda;\chi,\tau)k(\lambda;\chi,\tau)$, from which it follows that $k(\lambda;\chi,\tau)$ is analytic for $\lambda\in\mathbb{C}\setminus\Sigma_g$, with bounded boundary values except at the branch points $\lambda=A\pm\ii B$ where it is only required that the product $R(\lambda;\chi,\tau)k(\lambda;\chi,\tau)$ is bounded.  We also require that $k(\lambda;\chi,\tau)=O(\lambda^{-2})$ as $\lambda\to\infty$, and that the boundary values taken by $k(\lambda;\chi,\tau)$ along $\Sigma_g$ are related by
\begin{equation}
k_+(\lambda;\chi,\tau)-k_-(\lambda;\chi,\tau)=\frac{2 \kappa(\chi,\tau)-2\vartheta(\lambda;\chi,\tau)}{R_+(\lambda;\chi,\tau)},\quad\lambda\in\Sigma_g,
\end{equation}
as implied by \eqref{eq:hpm-kappa}.
It follows that $k(\lambda;\chi,\tau)$ is necessarily given by the Plemelj formula:
\begin{equation}
k(\lambda;\chi,\tau)=\frac{1}{\ii\pi}\int_{\Sigma_g}\frac{\kappa(\chi,\tau)-\vartheta(\eta;\chi,\tau)}{R_+(\eta;\chi,\tau)(\eta-\lambda)}\,\dd\eta,\quad\lambda\in\mathbb{C}\setminus\Sigma_g.
\label{eq:k-formula}
\end{equation}
It is not hard to see that this formula automatically gives the condition that $R(\lambda;\chi,\tau)k(\lambda;\chi,\tau)$ is bounded at the branch points $\lambda=A\pm\ii B$.  However the condition $k(\lambda;\chi,\tau)=O(\lambda^{-2})$ as $\lambda\to\infty$ remains to be enforced, and this will determine the integration constant $\kappa(\chi,\tau)$.  Indeed, the coefficient of the leading term proportional to $\lambda^{-1}$ in the Laurent expansion of $k(\lambda;\chi,\tau)$ about $\lambda=\infty$ must vanish, i.e.,
\begin{equation}
\int_{\Sigma_g}\frac{\kappa(\chi,\tau)-\vartheta(\lambda;\chi,\tau)}{R_+(\lambda;\chi,\tau)}\,\dd\lambda = 0.
\label{eq:K-integral}
\end{equation}
Note that letting $L$ denote any clockwise-oriented loop surrounding the branch cut $\Sigma_g$ of $R(\lambda;\chi,\tau)$, a residue computation at $\lambda=\infty$ where $R(\lambda;\chi,\tau)=\lambda+O(1)$ shows that
\begin{equation}
\int_{\Sigma_g}\frac{\dd\lambda}{R_+(\lambda;\chi,\tau)} = \frac{1}{2}\oint_L\frac{\dd\lambda}{R(\lambda;\chi,\tau)} =-\ii\pi.
\label{eq:integral-R-plus}
\end{equation}
This being nonzero shows that $\kappa(\chi,\tau)$ will indeed be determined by the condition \eqref{eq:K-integral}.  Then,
\begin{equation}
\int_{\Sigma_g}\frac{\vartheta(\lambda;\chi,\tau)}{R_+(\lambda;\chi,\tau)}\,\dd\lambda = I_1(\chi,\tau)+I_2(\chi,\tau),
\end{equation}
where
\begin{equation}
I_1(\chi,\tau)\defeq\int_{\Sigma_g}\frac{\chi\lambda+\tau\lambda^2}{R_+(\lambda;\chi,\tau)}\,\dd\lambda\quad\text{and}\quad
I_2(\chi,\tau)\defeq\ii\int_{\Sigma_g}
%\log\left(\frac{\lambda-\ii}{\lambda+\ii}\right)
\frac{\log(B(\lambda))}{R_+(\lambda;\chi,\tau)}\,\dd\lambda.
\label{eq:I1-I2}
\end{equation}
A similar residue calculation using two more terms in the large-$\lambda$ expansion of $R(\lambda;\chi,\tau)$, specifically that $R(\lambda;\chi,\tau)^{-1}=\lambda^{-1}+A\lambda^{-2}+(A^2-\tfrac{1}{2}B^2)\lambda^{-3}+O(\lambda^{-4})$ shows that
\begin{equation}
I_1(\chi,\tau)=-\ii\pi (\chi A +\tau(A^2-\tfrac{1}{2}B^2)),\quad A=A(\chi,\tau),\quad B^2=B(\chi,\tau)^2.
\end{equation}
Now assuming that the loop $L$ excludes the branch cut $\Sigma_\mathrm{c}$ of the logarithm and that $L'$ is a counter-clockwise oriented contour that encircles $\Sigma_\mathrm{c}$ but that excludes $\Sigma_g$, we use the fact that the integrand for $I_2$ is integrable at $\lambda=\infty$ to obtain
\begin{equation}
I_2(\chi,\tau)=\frac{1}{2}\ii\oint_L
%\log\left(\frac{\lambda-\ii}{\lambda+\ii}\right)
\frac{\log(B(\lambda))}{R(\lambda;\chi,\tau)}\,\dd\lambda = \frac{1}{2}\ii\oint_{L'}%\log\left(\frac{\lambda-\ii}{\lambda+\ii}\right)
\frac{\log(B(\lambda))}{R(\lambda;\chi,\tau)}\,\dd\lambda.
\label{eq:I2-identities}
\end{equation}
Then, collapsing $L'$ to both sides of $\Sigma_\mathrm{c}$, where $R(\lambda;\chi,\tau)$ is analytic but the boundary values of the logarithm differ by $2\pi\ii$, 
\begin{equation}
I_2(\chi,\tau)=\pi\int_{\Sigma_\mathrm{c}}\frac{\dd\lambda}{R(\lambda;\chi,\tau)},
\end{equation}
where we recall that $\Sigma_\mathrm{c}$ is a Schwarz-symmetric arc oriented from $-\ii$ to $\ii$.  Therefore, $I_2(\chi,\tau)$ is
purely imaginary and is computable in terms of $A(\chi,\tau)$ and $B(\chi,\tau)^2$ via hyperbolic functions.  We have therefore obtained a formula for the integration constant $\kappa(\chi,\tau)$ in the form \eqref{eq:kappa-formula} written in Section~\ref{sec:Results-Shelves}.
%\begin{equation}
%\kappa(\chi,\tau)=\chi A+\tau(A^2-\tfrac{1}{2}B^2) +\ii\int_{\Sigma_\mathrm{c}}\frac{\dd\lambda}{R(\lambda;\chi,\tau)}.
%\label{eq:kappa-formula}
%\end{equation}
According to \eqref{eq:k-formula} and $g(\lambda;\chi,\tau)=R(\lambda;\chi,\tau)k(\lambda;\chi,\tau)$ we have (evaluating the term proportional to $\kappa(\chi,\tau)$ by residues):
%Note that the ambiguity in the path of integration in the domain $\mathbb{C}\setminus\Sigma_g$ implies by a residue calculation that $\kappa(\chi,\tau)$ is well-defined by \eqref{eq:kappa-formula} modulo $2\pi$.  This ambiguity will not be important as,
\begin{equation}
g(\lambda;\chi,\tau)= \kappa(\chi,\tau)-\frac{R(\lambda;\chi,\tau)}{\ii\pi}\int_{\Sigma_g}\frac{\vartheta(\eta;\chi,\tau)\,\dd\eta}{R_+(\eta;\chi,\tau)(\eta-\lambda)}.
\end{equation}
%so $g(\lambda;\chi,\tau)$ is determined modulo $2\pi\ii$.  Therefore the matrix factor $\ee^{ng(\lambda;\chi,\tau)\sigma_3}$ appearing in the transformation \eqref{eq:T-to-S} is well defined for $n\in\mathbb{Z}$.
%\textcolor{red}{This discussion needs to be modified in the alternative approach, because $M$ is not an integer, so an ambiguity mod $2\pi$ could be a problem in making $\ee^{Mg(\lambda;\chi,\tau)\sigma_3}$ well defined.  I think the right approach here is just to stick with $\Sigma_\mathrm{c}$ as the path of integration in the formula for $\kappa$ and not worry about whether the integral from $-\ii$ to $\ii$ would make sense without additional interpretation of the path of integration.}

\subsubsection{Construction of $g(\lambda;\chi,\tau)$ when $\Sigma_g\subset\Sigma_\mathrm{c}$}
\label{sec:g-function-dumbbell}
%\textcolor{red}{(Move to beginning of $\exterior$ proof section?)}
If the Schwarz-symmetric jump contour $\Sigma_g$ is to be taken as a subset of $\Sigma_\mathrm{c}$,
then some modification of the construction of $g(\lambda;\chi,\tau)$ is needed.  Indeed, in this situation the phase function $\vartheta(\lambda;\chi,\tau)$ defined in \eqref{eq:vartheta} takes two distinct boundary values at each point of $\Sigma_g$, so instead of \eqref{eq:hpm-kappa} the condition that the sum of boundary values of $h(\lambda;\chi,\tau)$ is constant along $\Sigma_g$ now reads
\begin{equation}
\begin{split}
g_+(\lambda;\chi,\tau)+g_-(\lambda;\chi,\tau)&=h_+(\lambda;\chi,\tau)+h_-(\lambda;\chi,\tau)-\vartheta_+(\lambda;\chi,\tau)-\vartheta_-(\lambda;\chi,\tau)\\
&=2\gamma(\chi,\tau)-\vartheta_+(\lambda;\chi,\tau)-\vartheta_-(\lambda;\chi,\tau),\quad
\lambda\in\Sigma_g\subset\Sigma_\mathrm{c},
\end{split}
\label{eq:hpm-gamma}
\end{equation}
where the constant value of $h_++h_-$ is now denoted $2\gamma(\chi,\tau)$.  As before, we write $g(\lambda;\chi,\tau)=R(\lambda;\chi,\tau)k(\lambda;\chi,\tau)$ and find that the analogue of \eqref{eq:k-formula} reads
\begin{equation}
k(\lambda;\chi,\tau)=\frac{1}{\ii\pi}\int_{\Sigma_g}\frac{\gamma(\chi,\tau)-\tfrac{1}{2}\vartheta_+(\eta;\chi,\tau)-\tfrac{1}{2}\vartheta_-(\eta;\chi,\tau)}{R_+(\eta;\chi,\tau)(\eta-\lambda)}\,\dd\eta,\quad\lambda\in\mathbb{C}\setminus\Sigma_g,
\label{eq:k-formula-gamma}
\end{equation}
where $\gamma(\chi,\tau)$ is to be chosen to enforce the condition analogous to \eqref{eq:K-integral}: 
\begin{equation}
\int_{\Sigma_g}\frac{\gamma(\chi,\tau)-\tfrac{1}{2}\vartheta_+(\lambda;\chi,\tau)-\tfrac{1}{2}\vartheta_-(\lambda;\chi,\tau)}{R_+(\lambda;\chi,\tau)}\,\dd\lambda = 0.
\label{eq:K-integral-gamma}
\end{equation}
By residue calculations, this can be written in the form
\begin{equation}
\gamma(\chi,\tau)=\chi A(\chi,\tau)+\tau(A(\chi,\tau)^2-\tfrac{1}{2}B(\chi,\tau)^2)-\frac{1}{\ii\pi}I_2(\chi,\tau),
\end{equation}
where now $I_2(\chi,\tau)$ is given by a modification of the formula in \eqref{eq:I1-I2}:
\begin{equation}
I_2(\chi,\tau)\defeq\ii\int_{\Sigma_g}\frac{\tfrac{1}{2}\log_+(B(\lambda))+\tfrac{1}{2}\log_-(B(\lambda))}{R_+(\lambda;\chi,\tau)}\,\dd\lambda.
\end{equation}
Taking $L$ to be a clockwise-oriented loop passing through the endpoints $A\pm\ii B$ of $\Sigma_g$ and enclosing $\Sigma_g$ but with the two arcs of $\Sigma_\mathrm{c}\setminus\Sigma_g$ in its exterior, and taking $L'$ to be a pair of counterclockwise-oriented loops each enclosing one of the arcs of $\Sigma_\mathrm{c}\setminus\Sigma_g$ and passing through the corresponding endpoint of $\Sigma_g$, we again arrive at the identities \eqref{eq:I2-identities}.  Then collapsing the loops of $L'$ to both sides of the arcs of $\Sigma_\mathrm{c}\setminus\Sigma_g$ we obtain
\begin{equation}
I_2(\chi,\tau)=\pi\int_{\Sigma_\mathrm{c}\setminus\Sigma_g}\frac{\dd\lambda}{R(\lambda;\chi,\tau)},
\end{equation}
leading to the analogue of \eqref{eq:kappa-formula}:
\begin{equation}
\gamma(\chi,\tau)=\chi A+\tau(A^2-\tfrac{1}{2}B^2) +\ii\int_{\Sigma_\mathrm{c}\setminus\Sigma_g}\frac{\dd\lambda}{R(\lambda;\chi,\tau)}.
\label{eq:gamma-formula}
\end{equation}
This is equivalent to the form written in \eqref{eq:gamma-formula-intro} in Section~\ref{sec:Results-Exterior}.

\subsubsection{Structure of the zero level curve $\mathrm{Re}(\ii h(\lambda;\chi,\tau))=0$}
\label{sec:zero-level-curve}
A consequence of the choice of integration constant to ensure that $g(\lambda;\chi,\tau)\to 0$ as $\lambda\to\infty$ is that both $g$ and $h$ have even Schwarz symmetry for all $(\chi,\tau)\in \overline{\exterior\cup\shelves}$:
\begin{equation}
g(\lambda^*;\chi,\tau)^*=g(\lambda;\chi,\tau)\quad\text{and}\quad h(\lambda^*;\chi,\tau)^*=h(\lambda;\chi,\tau),\quad (\chi,\tau)\in \overline{\exterior\cup\shelves}.
\label{eq:g-h-Schwarz}
\end{equation}
It follows that $\mathrm{Re}(\ii h(\lambda;\chi,\tau))=0$ holds for all $\lambda\in\mathbb{R}$, $\lambda\not\in\Sigma_g$.  We also have the following.
\begin{lemma}
For all $(\chi,\tau)\in \overline{\exterior\cup\shelves}$, $\mathrm{Re}(\ii h(A\pm\ii B;\chi,\tau))=0$, where $A\pm\ii B=A(\chi,\tau)\pm\ii B(\chi,\tau)$ are the complex-conjugate endpoints of $\Sigma_g$.
\label{lem:h-imaginary-at-branch-points}
\end{lemma}
\begin{proof}
Let $\lambda_\mathbb{R}\in\mathbb{R}$ with $\lambda_\mathbb{R}\not\in\Sigma_g$.
Since $h(\lambda_\mathbb{R};\chi,\tau)$ is purely real,
\begin{equation}
\begin{split}
\mathrm{Re}(\ii h(A+\ii B;\chi,\tau))&=\mathrm{Re}\left(\ii \int_{\lambda_\mathbb{R}}^{A+\ii B}h'(\lambda;\chi,\tau)\,\dd\lambda\right)\\ &= \frac{1}{2}\ii \int_{\lambda_\mathbb{R}}^{A+\ii B}h'(\lambda;\chi,\tau)\,\dd\lambda -\frac{1}{2}\ii\left[\int_{\lambda_\mathbb{R}}^{A+\ii B}h'(\lambda;\chi,\tau)\,\dd\lambda\right]^*,
\end{split}
\end{equation}
where due to \eqref{eq:hprime-residues} the path of integration $L:\lambda_\mathbb{R}\to A+\ii B$ is arbitrary in the upper half plane, except that it is chosen so that the pole at $\lambda=\ii$ does not lie between $L$ and $\Sigma_g$.  Using the even Schwarz symmetry of $h'(\lambda;\chi,\tau)$ and a contour integral reparametrization,
\begin{equation}
\begin{split}
\left[\int_{\lambda_\mathbb{R}}^{A+\ii B}h'(\lambda;\chi,\tau)\,\dd\lambda\right]^*&=\int_{\lambda_\mathbb{R}}^{A+\ii B}h'(\lambda;\chi,\tau)^*\,\dd\lambda^*\\
&=\int_{\lambda_\mathbb{R}}^{A+\ii B}h'(\lambda^*;\chi,\tau)\,\dd\lambda^*\\
&=-\int_{A-\ii B}^{\lambda_\mathbb{R}}h'(\lambda;\chi,\tau)\,\dd\lambda,
\end{split}
\end{equation}
where in the final integral the path of integration is $L^*$ but with opposite orientation.  Combining these results, and taking into account that $h'(\lambda;\chi,\tau)$ changes sign across $\Sigma_g$, we have
\begin{equation}
\mathrm{Re}(\ii h(A+\ii B;\chi,\tau))=\frac{1}{4}\ii \oint_O h'(\lambda;\chi,\tau)\,\dd\lambda,
\end{equation}
where $O$ is a simple closed contour enclosing $\Sigma_g$ but excluding $\lambda=\pm\ii$, the orientation of which depends on whether $\lambda_\mathbb{R}$ lies to the left or right of the point where $\Sigma_g$ intersects the real axis.  Using \eqref{eq:hprime-residues}, without changing the value of the integral we may replace $O$ by another contour surrounding $\Sigma_g$ with the same orientation but now also enclosing $\lambda=\pm\ii$.  Since there are no longer any singularities of $h'(\lambda;\chi,\tau)$ outside of $O$, we may evaluate the integral over $O$ by residues at $\lambda=\infty$.  Using \eqref{eq:hprime-expansion} one sees that the residue of $h'(\lambda;\chi,\tau)$ at $\lambda=\infty$ vanishes, so we conclude that $\mathrm{Re}(\ii h(A+\ii B;\chi,\tau))=0$.  Using \eqref{eq:g-h-Schwarz} then gives also $\mathrm{Re}(\ii h(A-\ii B;\chi,\tau))=0$.
\end{proof}
This result implies that the level curve $\mathrm{Re}(\ii h(\lambda;\chi,\tau))=0$ does not depend substantially on the choice of branch cut $\Sigma_g$.  Indeed, the differential $\ii h'(\lambda;\chi,\tau)\,\dd\lambda$ can be extended from $\lambda\in\mathbb{C}\setminus\Sigma_g$ to the hyperelliptic Riemann surface $\mathcal{R}$ of the equation $R^2=(\lambda-A)^2+B^2$ just by adding a second copy of $\mathbb{C}\setminus\Sigma_g$ on which $R(\lambda;\chi,\tau)$ is replaced with $-R(\lambda;\chi,\tau)$.  Since $\mathcal{R}$ has genus zero and hence has trivial homology, and since the residues of $h'(\lambda;\chi,\tau)$ (see \eqref{eq:hprime-residues}--\eqref{eq:hprime-expansion}) are imaginary, the real part of an antiderivative of $\ii h'(\lambda;\chi,\tau)\,\dd\lambda$ is well defined up to a constant as a harmonic function on $\mathcal{R}$ with the four points corresponding to $\lambda=\pm\ii$ omitted.  By 
Lemma~\ref{lem:h-imaginary-at-branch-points}, if the constant of integration is determined by fixing the base point of integration to be one of the two branch points, the real part vanishes at both branch points and on the principal sheet of $\mathcal{R}$ this function coincides with $\mathrm{Re}(\ii h(\lambda;\chi,\tau))$ while on the auxiliary sheet it coincides with $-\mathrm{Re}(\ii h(\lambda;\chi,\tau))$.  It follows that the projection from each sheet of $\mathcal{R}$ to $\mathbb{C}$ of the zero level is exactly the same.  Since the choice of branch cut $\Sigma_g$ for $R(\lambda;\chi,\tau)$ only affects the value of $\mathrm{Re}(\ii h(\lambda;\chi,\tau))$ up to a sign, the zero level curve is essentially independent of the location of $\Sigma_g$ (technically, $\mathrm{Re}(\ii h(\lambda;\chi,\tau))$ is undefined on $\Sigma_g$, but the zero level curve can be extended unambiguously to $\Sigma_g$). 

As noted above, the zero level set $\mathrm{Re}(\ii h(\lambda;\chi,\tau))=0$ always contains the real axis as a proper subset, as well as the branch points $\lambda=A\pm\ii B$.  Since $\ii h(\lambda;\chi,\tau)=\ii \vartheta(\lambda;\chi,\tau)+\ii g(\lambda;\chi,\tau)=\ii \vartheta(\lambda;\chi,\tau)+O(\lambda^{-1})=\ii \chi\lambda+\ii \tau\lambda^2+O(\lambda^{-1})$ as $\lambda\to\infty$, for $\tau\neq 0$ in $\overline{\exterior\cup\shelves}$ there is exactly one Schwarz-symmetric pair of arcs of the zero level set that are asymptotically vertical, one in each half-plane.  All other arcs of the level set in $\mathbb{C}\setminus\mathbb{R}$ are bounded.  These arcs are necessarily ``horizontal'' trajectories of the rational quadratic differential $h'(\lambda;\chi,\tau)^2\,\dd\lambda^2$, i.e., curves along which $h'(\lambda;\chi,\tau)^2\,\dd\lambda^2>0$.  By Lemma~\ref{lem:h-imaginary-at-branch-points}, some of the arcs of the zero level set are so-called critical trajectories, i.e., those emanating from zeros of $h'(\lambda;\chi,\tau)^2$.  By Jenkins' three-pole theorem \cite[Theorem 3.6]{Jenkins58} and the basic structure theorem \cite[Theorem 3.5]{Jenkins58}, the union of critical trajectories of $h'(\lambda;\chi,\tau)^2\,\dd\lambda^2$ has empty interior and divides the complex $\lambda$-plane into a finite number of domains.  Two of these domains, one in each half-plane, are so-called \emph{circle domains} each containing one of the poles $\lambda=\pm\ii$ and each having at least one of the zeros of $h'(\lambda;\chi,\tau)^2$ on its boundary.  Furthermore, from each of the simple roots $\lambda=A\pm\ii B$ of $h'(\lambda;\chi,\tau)^2$ emanate locally exactly three critical trajectories, and from each of the double roots of $h'(\lambda;\chi,\tau)^2$ (i.e., the roots of $2\tau\lambda^2+u(\chi,\tau)\lambda+v(\chi,\tau)$) emanate locally exactly four critical trajectories.

Suppose first that $(\chi,\tau)\in \exterior_\chi\cup \shelves$.  Then the double roots of $h'(\lambda;\chi,\tau)$ (two for $\tau\neq 0$ and one for $\tau=0$) are real, and therefore two of the four trajectories emanating from each coincide with intervals of $\mathbb{R}$ (that are contained in the level set $\mathrm{Re}(\ii h(\lambda;\chi,\tau))=0$, and the closure of the union of which is exactly $\mathbb{R}$).  In this case, by Lemma~\ref{lem:h-imaginary-at-branch-points} all critical trajectories are included in the level set $\mathrm{Re}(\ii h(\lambda;\chi,\tau))=0$.  The level curves entering the upper and lower half-planes vertically from $\lambda=\infty$ for $\tau\neq 0$ can only terminate at one of the roots of $h'(\lambda;\chi,\tau)^2$.  These trajectories either terminate at one of the real double roots, or at the conjugate pair of simple roots $\lambda=A\pm\ii B$.  
\begin{itemize}
\item
If they terminate at one of the two real double roots, then the non-real trajectories emanating from the other real double root can only terminate at the simple roots $\lambda=A\pm\ii B$.  It follows that the two additional trajectories emanating from each of these simple roots must coincide and form a closed curve in each half-plane.  By Teichm\"uller's lemma \cite[Theorem 14.1]{Strebel84}, this curve must be the boundary of the circle domain containing the pole $\lambda=\pm\ii$. If $\tau=0$ and hence there are no unbounded arcs of the level set in the open upper and lower half-planes, then by the same arguments the non-real trajectories emanating from the unique real double root terminate at the simple roots $\lambda=A\pm\ii B$, and the remaining two trajectories from each of these coincide and enclose the poles at $\lambda=\pm\ii$.  The zero level $\mathrm{Re}(\ii h(\lambda;\chi,\tau))=0$ consists of the real line, a Schwarz-symmetric pair of arcs connecting a real double root with the conjugate pair of simple roots $\lambda=A\pm\ii B$, a Schwarz-symmetric pair of loops joining each simple root $\lambda=A\pm\ii B$ to itself and enclosing the poles at $\lambda=\pm\ii$, and (if $\tau\neq 0$) a Schwarz-symmetric pair of unbounded arcs emanating from the second real double root and tending vertically to $\lambda=\infty$.  This topological configuration of the zero level set holds on the domain $\exterior_\chi$ (as one can see from the limiting case of $\tau=0$, where the zero level set acquires additional Schwarz reflection symmetry in the imaginary axis).
\item
If they terminate at the conjugate pair of simple roots $\lambda=A\pm \ii B$, then the remaining two trajectories emanating from each simple root terminate at the two real double roots, and the boundary of the circle domain in each half-plane consists of three distinct trajectories, one of which is the interval of the real axis between the two real double roots and is common to the boundaries of both circle domains.  (The other apparent possibility, that the two additional trajectories emanating from $\lambda=A\pm\ii B$ coincide and that the two trajectories emanating into each half-plane from the two real double roots also coincide, can be ruled out by Teichm\"uller's lemma since two closed curves formed by critical trajectories would appear in each half-plane, only one of which can contain a pole.)  The zero level set $\mathrm{Re}(\ii h(\lambda;\chi,\tau))=0$ consists of the real line, a Schwarz-symmetric pair of arcs from each of the two real double roots to the conjugate pair of simple roots, and a Schwarz-symmetric pair of unbounded arcs emanating from the conjugate pair of simple roots and tending vertically to $\lambda=\infty$.  This topological configuration of the zero level set holds on the domain $\shelves$ (as one can see from the limiting case of $\chi=0$, where again the zero level set acquires additional Schwarz reflection symmetry in the imaginary axis).
\end{itemize}

Next suppose that $(\chi,\tau)\in \exterior_\tau$.  Then the double roots of $h'(\lambda;\chi,\tau)^2$ form a conjugate pair that we denote by $\lambda=C\pm\ii D$.  By Lemma~\ref{lem:h-imaginary-at-branch-points}, the simple roots $\lambda=A\pm\ii B$ are on the zero level of $\mathrm{Re}(\ii h(\lambda;\chi,\tau))$ and therefore at most one trajectory from each can be unbounded.  If none of the three trajectories emanating from $\lambda=A\pm\ii B$ is unbounded, then at least one of them must terminate at $\lambda=C\pm\ii D$ implying that $\mathrm{Re}(\ii h(C\pm\ii D;\chi,\tau))=0$ and hence the unbounded arc of the level curve in each half-plane terminates at this point as well.  If it is exactly one trajectory from $\lambda=A\pm\ii B$ that terminates at $\lambda=C\pm\ii D$, then the other two coincide forming a loop, and the remaining two bounded trajectories emanating from the latter must coincide forming a second loop; however only one of these loops can contain the pole at $\lambda=\pm\ii$ so the existence of both is ruled out by Teichm\"uller's lemma.  If it is exactly two trajectories from $\lambda=A\pm\ii B$ that terminate at $\lambda=C\pm\ii D$, then the third trajectory would have to be unbounded contradicting the assumption that all trajectories from $A\pm\ii B$ are bounded.  If all three trajectories from $\lambda=A\pm\ii B$ terminate at $\lambda=C\pm\ii D$, then we again form two domains bounded by trajectories only one of which can contain a pole leading to a contradiction with Teichm\"uller's lemma.  We conclude that exactly one of the trajectories emanating from each simple root $\lambda=A\pm\ii B$ is unbounded.  
It then follows that the other two trajectories emanating from $\lambda=A\pm\ii B$ must coincide.  Indeed, otherwise they must both terminate at the double root $\lambda=C\pm\ii D$ in the same half-plane from which we learn that $\mathrm{Re}(\ii h(C\pm\ii D;\chi,\tau))=0$ which implies that neither of the remaining two trajectories from $\lambda=C\pm\ii D$ can be unbounded or terminate at $A\pm\ii B$, so they must coincide.  It is then apparent that each half-plane contains a domain bounded by the two curves connecting $A\pm\ii B$ with $C\pm\ii D$ and a domain bounded by the trajectory joining $C\pm\ii D$ to itself; however the pole $\lambda=\pm\ii$ can only lie in one of these two domains, so the existence of the other leads to a contradiction with Teichm\"uller's lemma. The zero level set $\mathrm{Re}(\ii h(\lambda;\chi,\tau))=0$ is then the disjoint union of three components:  the real line and a Schwarz-symmetric pair of components each consisting of a loop trajectory joining $\lambda=A\pm\ii B$ to itself and surrounding $\lambda=\pm\ii$ and an unbounded trajectory emanating from $\lambda=A\pm\ii B$.  In this case, the double roots $\lambda=C\pm\ii D$ do not lie on the zero level of $\mathrm{Re}(\ii h(\lambda;\chi,\tau))$, and the level set is not connected. 