\subsection{Symmetries of $q(x,t;\mathbf{Q}^{-s},M)$:  Proof of Proposition~\ref{prop:symmetry}}
\begin{proof}[Proof of Proposition~\ref{prop:symmetry}]
For this proof, we assume without loss of generality that $\Sigma_\circ$ is a circle centered at the origin of arbitrary radius $r$ greater than $1$ with clockwise orientation.
Taking $\mathbf{G}=\mathbf{Q}^{-s}$ for $s=\pm 1$, the jump condition \eqref{eq:P-bulk-jump} in Riemann-Hilbert Problem~\ref{rhp:rogue-wave-reformulation} for $\mathbf{P}(\lambda;x,t)=\mathbf{P}(\lambda;x,t,\mathbf{Q}^{-s},M)$ can be written as 
\begin{equation}
\mathbf{P}_+(\lambda;x,t)=\mathbf{P}_-(\lambda;x,t)\ee^{-\ii\theta(\lambda;x,t)\sigma_3}%B_k(\lambda)^{\sigma_3}
%\omega(\lambda)^{N\sigma_3}
B(\lambda)^{M\sigma_3}
\mathbf{Q}^{-s}%B_k(\lambda)^{-\sigma_3}
%\omega(\lambda)^{-N\sigma_3}
B(\lambda)^{-M\sigma_3}
\ee^{\ii\theta(\lambda;x,t)\sigma_3},
\label{eq:P-bulk-jump-rewrite}
\end{equation}
where $\theta(\lambda;x,t)\defeq\lambda x+\lambda^2t$. 
%and where $B_k(\lambda)=\omega(\lambda)^{s}((\lambda-\ii)/(\lambda+\ii))^{n}$.  
Define $\mathbf{R}(\lambda;x,t)$ in terms of $\mathbf{P}(\lambda;x,t)$ by
\begin{equation}
\mathbf{R}(\lambda;x,t)\defeq\begin{cases}
-s\sigma_3\mathbf{P}(\lambda;x,t)\ee^{-2\ii\theta(\lambda;x,t)\sigma_3}\sigma_1,&\quad |\lambda|<r,
\\
\sigma_3\mathbf{P}(\lambda;x,t)%B_k(\lambda)^{2\sigma_3}
%\omega(\lambda)^{2N\sigma_3}
B(\lambda)^{2M\sigma_3}
\sigma_3,&\quad |\lambda|>r.
\end{cases}
\end{equation} 
$\mathbf{R}(\lambda;x,t)$ is obviously analytic for $\lambda\in\mathbb{C}\setminus\Sigma_\circ$, and since 
%$f(\lambda)\to 1$ and $\rho(\lambda)-\lambda\to 0$ 
%$\omega(\lambda)\to 1$ 
powers of $B(\lambda)$ tend to $1$ 
as $\lambda\to\infty$ we have $\mathbf{R}(\lambda;x,t)\to\mathbb{I}$ as $\lambda\to\infty$.  To compute the jump across the circle $\Sigma_\circ$, we use the jump condition \eqref{eq:P-bulk-jump} for $\mathbf{P}(\lambda;x,t)$ to obtain, using \eqref{eq:Q-def} in the last step,
\begin{equation}
\begin{split}
\mathbf{R}_+(\lambda;x,t)&=\sigma_3\mathbf{P}_+(\lambda;x,t)
%B_k(\lambda)^{2\sigma_3}
%\omega(\lambda)^{2N\sigma_3}
B(\lambda)^{2M\sigma_3}
\sigma_3\\
&=\sigma_3\mathbf{P}_-(\lambda;x,t)\ee^{-\ii\theta(\lambda;x,t)\sigma_3}
%B_k(\lambda)^{\sigma_3}
%\omega(\lambda)^{N\sigma_3}
B(\lambda)^{M\sigma_3}
\mathbf{Q}^{-s}\sigma_3 
%B_k(\lambda)^{\sigma_3}
%\omega(\lambda)^{N\sigma_3}
B(\lambda)^{M\sigma_3}
\ee^{\ii\theta(\lambda;x,t)\sigma_3}\\
&=-s\mathbf{R}_-(\lambda;x,t)\sigma_1\ee^{\ii\theta(\lambda;x,t)\sigma_3}
%B_k(\lambda)^{\sigma_3}
%\omega(\lambda)^{N\sigma_3}
B(\lambda)^{M\sigma_3}
\mathbf{Q}^{-s}\sigma_3
%B_k(\lambda)^{\sigma_3}
%\omega(\lambda)^{N\sigma_3}
B(\lambda)^{M\sigma_3}
\ee^{\ii\theta(\lambda;x,t)\sigma_3}\\
&=\mathbf{R}_-(\lambda;x,t)\ee^{-\ii\theta(\lambda;x,t)\sigma_3}
%B_k(\lambda)^{-\sigma_3}
%\omega(\lambda)^{-N\sigma_3}
B(\lambda)^{-M\sigma_3}
\left[-s\sigma_1\mathbf{Q}^{-s}\sigma_3\right]
%B_k(\lambda)^{\sigma_3}
%\omega(\lambda)^{N\sigma_3}
B(\lambda)^{M\sigma_3}
\ee^{\ii\theta(\lambda;x,t)\sigma_3}\\
&=\mathbf{R}_-(\lambda;x,t)\ee^{-\ii\theta(\lambda;x,t)\sigma_3}
%B_k(\lambda)^{-\sigma_3}
%\omega(\lambda)^{-N\sigma_3}
B(\lambda)^{-M\sigma_3}
\mathbf{Q}^{-s}
%B_k(\lambda)^{\sigma_3}
%\omega(\lambda)^{N\sigma_3}
B(\lambda)^{M\sigma_3}
\ee^{\ii\theta(\lambda;x,t)\sigma_3}.
\end{split}
\end{equation}
Now $\theta(\lambda;-x,t)=\theta(-\lambda;x,t)$
%, and 
%since $f(-\lambda)=f(\lambda)$, $\rho(-\lambda)=-\rho(\lambda)$, and $\rho^2=\lambda^2+1$, $B_k(\lambda)=B_k(-\lambda)^{-1}$.  
%and $\omega(\lambda)=\omega(-\lambda)^{-1}$.
and $B(\lambda)=B(-\lambda)^{-1}$.
Therefore, we see that $\mathbf{P}(\lambda;-x,t)$ and $\mathbf{R}(-\lambda;x,t)$ satisfy exactly the same analyticity, jump, and normalization conditions and therefore by uniqueness $\mathbf{P}(\lambda;-x,t)=\mathbf{R}(-\lambda;x,t)$.  Thus,
\begin{equation}
\begin{split}
q(-x,t;\mathbf{Q}^{-s},M)&=2\ii\lim_{\lambda\to\infty}\lambda P_{12}(\lambda;-x,t)\\
&=2\ii\lim_{\lambda\to\infty}\lambda R_{12}(-\lambda;x,t)\\
&=-2\ii\lim_{\lambda\to\infty}\lambda R_{12}(\lambda;x,t)\\
&=2\ii\lim_{\lambda\to\infty}\lambda P_{12}(\lambda;x,t)\\
&=q(x,t;\mathbf{Q}^{-s},M).
\end{split}
\end{equation}
Since 
%$B_k(-\lambda^*)=B_k(\lambda)^*$, 
%$\omega(-\lambda^*)=\omega(\lambda)^*$
$B(-\lambda^*)=B(\lambda)^*$,
it is even easier to see that $\mathbf{P}(\lambda;x,-t)$ and $\mathbf{P}(-\lambda^*;x,t)^*$ solve the same Riemann-Hilbert problem and hence are equal.  Therefore
\begin{equation}
\begin{split}
q(x,-t;\mathbf{Q}^{-s},M)&=2\ii\lim_{\lambda\to\infty}\lambda P_{12}(\lambda;x,-t)\\
&=2\ii\lim_{\lambda\to\infty}\lambda P_{12}(-\lambda^*;x,t)^*\\
&=-2\ii\lim_{\lambda\to\infty}\left[\lambda^* P_{12}(\lambda^*;x,t)\right]^*\\
&=\left[2\ii\lim_{\lambda\to\infty}\lambda P_{12}(\lambda;x,t)\right]^*\\
&=q(x,t;\mathbf{Q}^{-s},M)^*.
\end{split}
\end{equation}
This completes the proof of Proposition~\ref{prop:symmetry}.
\end{proof}

\subsection{Continuation of $u(\chi,\tau)$ to $\overline{\exterior\cup\shelves}$:  Proof of Proposition~\ref{prop:u}}
\begin{proof}[Proof of Proposition~\ref{prop:u}]
We first examine $P(u;\chi,\tau)$ near the positive $\chi$ and $\tau$ axes.  
\begin{lemma}
Fix $\chi>0$.  Then for $\tau>0$ sufficiently small there exists a unique and simple real root of $P(u;\chi,\tau)$.
\label{lem:tau-small}
\end{lemma}
\begin{proof}
Since the simple root at $u=\chi$ for $\tau=0$ persists for small $\tau$, we need to show that the four roots of $P(u;\chi,0)$ near $u=\tfrac{1}{3}\chi$ and the two roots of $P(u;\chi,0)$ near $u=0$ become complex roots of $P(u;\chi,\tau)$ for $\tau\neq 0$ small.  

To study the roots of $P(u;\chi,\tau)$ near $u=\tfrac{1}{3}\chi$, we set $u=\tfrac{1}{3}\chi + \Delta u$ and then express $P(u;\chi,\tau)$ in terms of $\Delta u$:
\begin{multline}
27 P(\tfrac{1}{3}\chi+\Delta u;\chi,\tau)=2187\Delta u^7 - 243(3\chi^2-8\tau^2)\Delta u^5-162\chi^3\Delta u^4 \\{}+ 216\tau^2(54-3\chi^2+2\tau^2)\Delta u^3
-144\chi^3\tau^2\Delta u^2-144\chi^2\tau^4\Delta u-32\chi^3\tau^4.
\end{multline}
For small $\tau$, the dominant balance in which $\Delta u$ is also small is $\Delta u=\tau\zeta$ for a new unknown $\zeta=O(1)$ as $\tau\to 0$.  Then we find that we can divide by $\tau^4$ for $\tau\neq 0$ and obtain
\begin{equation}
27 \tau^{-4}P(\tfrac{1}{3}\chi +\tau\zeta;\chi,\tau)=-2\chi^3 (9\zeta^2+4)^2 + O(\tau),\quad\tau\to 0,\quad\tau\neq 0.
\end{equation}
So, to leading order, we have double purely imaginary roots at $\zeta=\pm\tfrac{2}{3}\ii$, which can split apart at higher order in $\tau$.  This shows that the four roots of $P(u;\chi,\tau)$ near $u=\tfrac{1}{3}\chi$ for small $\tau$ have nonzero imaginary parts.  

To study the roots of $P(u;\chi,\tau)$ near $u=0$ we observe that the dominant balance in $P(u;\chi,\tau)=0$ in which $u$ and $\tau$ are both small for $\chi>0$ fixed occurs with $u=\tau\zeta$ with new unknown $\zeta=O(1)$ as $\tau\to 0$.  Dividing by $\tau^2$ after the substitution yields
\begin{equation}
\tau^{-2}P(\tau\zeta;\chi,\tau)=-(\chi^5+ 8\chi(54+\chi^2)\tau^2+16\chi\tau^4)\zeta^2-16\chi^3 + O(\tau),\quad\tau\to 0,\quad\tau\neq 0.
\end{equation}
So, to leading order, we have a purely imaginary pair of simple roots at $\zeta=\pm 4\ii\chi^{-1}$.  This shows that the two roots of $P(u;\chi,\tau)$ near $u=0$ for small $\tau$ have nonzero imaginary parts.
\end{proof}

\begin{lemma}
Fix $\tau>0$.  Then for $\chi>0$ sufficiently small there exists a unique and simple real root of $P(u;\chi,\tau)$.
\label{lem:chi-small}
\end{lemma}
\begin{proof}
It is easy to see that since $P(u;0,\tau)=u^3(81u^4+72\tau^2u^2 + 432\tau^2+16\tau^4)$, for all $\tau>0$, $P(u;0,\tau)$ has a triple root at $u=0$ and no other real roots.  To unfold the triple root for small $\chi$, set $u=\chi \zeta$ and assume that the new unknown $\zeta$ is bounded as $\chi\downarrow 0$.  Thus one finds that one may divide by $\chi^3$ for $\chi\neq 0$ and obtain
\begin{equation}
\chi^{-3}P(\chi \zeta;\chi,\tau)=P_0(\zeta;\tau) + O(\chi^2),\quad\chi\downarrow 0,\quad \chi\neq 0,
\end{equation}
where $P_0$ is a cubic polynomial in $\zeta$:
\begin{equation}
P_0(\zeta;\tau)\defeq (432\tau^2+16\tau^4)\zeta^3-(432\tau^2+16\tau^4)\zeta^2+144\tau^2\zeta-16\tau^2.
\end{equation}
The discriminant of $P_0(\zeta;\tau)$ is proportional to $27\tau^{12}+\tau^{14}$ which vanishes for no $\tau>0$.  Therefore as $\tau$ varies between $\tau=0$ and $\tau=+\infty$, the root configuration of $P_0(\zeta;\tau)$ (i.e., three real roots or one real root with a complex-conjugate pair with nonzero imaginary part) persists for all $\tau>0$.  In the limit $\tau\to\infty$, the dominant terms in $P_0(\zeta;\tau)$ are $16\tau^4(\zeta^3-\zeta^2)$, so there is one real root near $\zeta=1$ and two small roots of size $\zeta=O(\tau^{-1})$.  We may write $\zeta=\tau^{-1}\zeta_1$ to separate them:
\begin{equation}
P_0(\tau^{-1}\zeta_1;\tau)=-16\tau^2(\zeta_1^2+1)+O(\tau),\quad\tau\to\infty,
\end{equation}
so $\zeta_1=\pm\ii + O(\tau^{-1})$ as $\tau\to\infty$.  Therefore, $P_0(\zeta;\tau)$ has a unique simple real root denoted $\zeta(\tau)$ and a conjugate pair of complex roots for all $\tau>0$.  It is easy to see that
\begin{equation}
\zeta(\tau)=\frac{1}{3}+O(\tau^2),\quad\tau\downarrow 0\quad\text{and}\quad
\zeta(\tau)=1+O(\tau^{-2}),\quad\tau\uparrow\infty.
\end{equation}
Since neither $P_0(\tfrac{1}{3};\tau)=-\tfrac{32}{27}\tau^4$ nor $P_0(1;\tau)=128\tau^2$ can vanish for any $\tau>0$, it then follows that $\tfrac{1}{3}<\zeta(\tau)<1$ holds for all $\tau>0$.  
This proves that the triple root of $P(u;0,\tau)$ originates in the limit $\chi\downarrow 0$ as the collision of a conjugate pair of roots and a real simple root having the expansion
\begin{equation}
u(\chi,\tau)=\chi \zeta(\tau)+O(\chi^3),\quad\chi\downarrow 0,\quad\tau>0.
\end{equation}
Since the remaining quartet of complex roots of $P(u;\chi,\tau)$ for $\chi=0$ remains complex for small $\chi$, the proof is finished.
\end{proof}
It then follows via \eqref{eq:eliminate-v}
that 
\begin{equation}
v(\chi,\tau)=2\tau\frac{2\zeta(\tau)-1}{3\zeta(\tau)-1}+O(\chi^2),\quad\chi\downarrow 0,\quad\tau>0,
\end{equation}
and then from \eqref{eq:eliminate-AB},
\begin{equation}
A(\chi,\tau)=O(\chi)\quad\text{and}\quad B(\chi,\tau)^2=\frac{2\zeta(\tau)}{3\zeta(\tau)-1} + O(\chi^2),\quad\chi\downarrow 0,\quad \tau>0.
\end{equation}
Note that $B(0,\tau)^2>1$ for all $\tau>0$, and that $B(0,\tau)^2\to 1$ as $\tau\uparrow +\infty$ while $B(0,\tau)^2\to +\infty$ as $\tau\downarrow 0$.

By Lemma~\ref{lem:tau-small} and Lemma~\ref{lem:chi-small}, $P(u;\chi,\tau)$ has a unique simple real root $u=u(\chi,\tau)$ for $(\chi,\tau)$ in the open first quadrant near each of the coordinate axes.  Next we show that this situation persists throughout $(\mathbb{R}_{>0}\times\mathbb{R}_{>0})\setminus\overline{\channels}$ by studying the resultant $\delta(\chi,\tau)$ of $P(u;\chi,\tau)$ and $P'(u;\chi,\tau)$ with respect to $u$, a polynomial in $(\chi,\tau)$ the zero locus of which detects repeated roots $u$ of $P(u;\chi,\tau)$. 
We consider the renormalized resultant
\begin{equation}
\delta^\mathrm{R}\defeq\frac{\delta(\chi,\tau)}{c \tau^{16}\chi^{6}},
\end{equation}
where $c \defeq 539122498937926189056$ is a constant that factors out of $\delta(\chi,\tau)$ along with the product $\tau^{16}\chi^{6}$. The renormalized resultant $\delta^\mathrm{R}$ is even in $\chi$ and $\tau$ and so can be expressed as
\begin{equation}
\begin{aligned}
\delta^\mathrm{R}(X,T) =& 16 X^{7} + 304 T X^{6} + 24 T (98 T + 1011 ) X^{5} + T (9488 T^2  - 380376 T -19683) X^4\\
&+64 T^2 (332 T^2 - 18009 T + 57645)X^3+ 
384T^2 (68 T^3 - 2553 T^2 + 159246 T - 59049 )X^2\\
&+16384  T^3  (T-54)(T+27)^2 X + 4096 T^3 (T+27)^4,
\end{aligned}
\end{equation}
where $X=\chi^2$ and $T=\tau^2$.
Since we have already shown that $P(u;\chi,\tau)$ has a unique and simple real root for $(\chi,\tau)$ near the coordinate axes, we can study the equation $\delta^\mathrm{R}(\chi^2,\tau^2)=0$ rather than $\delta(\chi,\tau)=0$.  If, as $(\chi,\tau)$ is taken out of one or the other region near the axes where it is known that $P(u;\chi,\tau)$ has a unique real and simple root, $P(u;\chi,\tau)$ does not acquire any repeated roots then in particular it does not acquire any repeated real roots and hence the number of real roots cannot change.  

Therefore, it would be sufficient to prove that the renormalized resultant $\delta^\mathrm{R}(\chi^2,\tau^2)$ does not vanish in the unbounded region $(\chi,\tau)\in(\mathbb{R}_+\times\mathbb{R}_+)\setminus\overline{\channels}$.  With this goal in mind, we view $\delta^\mathrm{R}(X,T)$ as a polynomial in $X$ with coefficients polynomial in $T$.
Then it is easy to see that for $T=\tau^2$ sufficiently large, all of the coefficients of powers of $X$ in $\delta^\mathrm{R}(X,T)$ are positive, so there are no nonnegative roots $X\ge 0$ of $\delta^\mathrm{R}(X,T)$.  Looking on the $T$-axis, we see that $\delta^\mathrm{R}(0,T)=4096T^3(T+27)^4$, which does not vanish for any $T=\tau^2>0$.  Therefore, as $\tau$ is decreased, the only way a positive value of $X>0$ for which $\delta^\mathrm{R}(X,T)=0$ can occur is if first there is a positive repeated root, i.e., a positive value $X>0$ for which both $\delta^\mathrm{R}(X,T)=0$ and $\delta^\mathrm{R}_X(X,T)=0$.  Setting to zero the resultant of the latter two polynomial equations with respect to $X$ gives the condition on $T=\tau^2$ for which there exist repeated roots $X$ (possibly negative or complex) of $\delta^\mathrm{R}(X,T)$.  This condition factors as:
\begin{equation}
T^{16}(T+27)^5(243T+1)^3(64T-27)^3Q_5(T)^2=0,
\label{eq:resultant-of-resultant}
\end{equation}
where $Q_5(T)$ is a quintic polynomial:
\begin{multline}
Q_5(T)\defeq 663552T^5+954511200T^4+829508109289T^3-14696124806763T^2\\
{}+82617806699739T-205891132094649.
\end{multline}
For $T=\tau^2>0$, the first three factors on the left-hand side of \eqref{eq:resultant-of-resultant} are nonzero, and the fourth factor vanishes exactly for $T=T^\sharp\defeq (\tau^\sharp)^2$, i.e., for $\tau=\pm\tau^\sharp$, where $\tau^\sharp$ is defined in \eqref{eq:corner-point}.  So it remains to determine whether $Q_5(T)=0$ holds for any $T>0$.  In fact $Q_5(0)\neq 0$ and $Q_5(T^\sharp)\neq 0$ by direct computation, so we will apply the theory of Sturm sequences (see Definition~\ref{def:Sturm-sequence}) to count the number of real roots $T$ in the intervals $(0,T^\sharp)$ and $(T^\sharp,+\infty)$.
We thus obtain the following sign sequences at the points $T=0$, $T=T^\sharp=(\tau^\sharp)^2$, and $T=\infty$:
\begin{equation}
\begin{split}
\Xi[Q_5](0)&=(-,+,+,-,-,+),\\
\Xi[Q_5](T^\sharp)&=(-,+,+,-,-,+),\\
\Xi[Q_5](+\infty)&=(+,+,-,-,-,+).
\end{split}
\end{equation}
Since $\#(\Xi[Q_5](0))- \#(\Xi[Q_5](T^\sharp))=3-3=0$, by Sturm's theorem (see Theorem~\ref{t:Sturm}) $Q_5(T)$ has no real root in $[0,T^\sharp]$. We also see that $\#(\Xi[Q_5](T^\sharp))- \#(\Xi[Q_5](+\infty))=3-2=1$, which similarly proves that there exists exactly one real root of $Q_5(T)$ in the interval 
$(T^\sharp,+\infty)$; we denote it by $T_1$. One can easily check numerically that $T_1\approx 10.232235>T^\sharp$, which gives $\tau_1\defeq \sqrt{T_1} \approx 3.198786$.  So as $T$ decreases from $T=+\infty$, the first possible bifurcation point at which positive solutions $X$ of $\delta^\mathrm{R}(X,T)=0$ might appear is $T=T_1$.  
%However, the existence of a double root of $X\mapsto\tilde{\rho}(X,T_1)$ is not relevant because this double root is not a positive number.  Indeed, the Sturm sequences of $\tilde{\rho}(X,T_1)$ at $X=0$ and $X=+\infty$ are
%\begin{equation}
%\begin{split}
%\Xi[\tilde{\rho}(\cdot,T_1)](0)&=(+,-,-,+,+,-,-,-),\\
%\Xi[\tilde{\rho}(\cdot,T_1)](+\infty)&=(+,+,-,-,-,-,+,-),
%\end{split}
%\end{equation}
%so since $\#(\Xi[\tilde{\rho}(\cdot,T_1)](0))-\#(\Xi[\tilde{\rho}(\cdot,T_1)](+\infty))=0$ there are no positive solutions $X>0$ of $\tilde{\rho}(X,T_1)=0$ by Sturm's Theorem.  Together with the above analysis for $T>T_1$ and for $T_0<T<T_1$, this proves that there are no solutions to $\tilde{\rho}(X,T)$ for any $(X,T)$ with $X\ge 0$ and $T>T_0$.  

Next, one checks directly that $\delta^\mathrm{R}(X,T^\sharp)$ factors as the product of a quintic polynomial in $X$ with strictly positive coefficients and $(16X-81)^2$.  Referring to \eqref{eq:corner-point}, this means that $X^\sharp=(\chi^\sharp)^2$ is a positive double root of $X\mapsto\delta^\mathrm{R}(X,T^\sharp)$, and that there are no other positive roots.  We now show that this double root splits into a pair of real simple roots as $T$ decreases from $T^\sharp$ and into a pair of complex-conjugate simple roots as $T$ increases from $T^\sharp$.  Indeed, if we write $X=X^\sharp+\Delta X$ and $T=T^\sharp+\Delta T$ for $\Delta X$ and $\Delta T$ small, then the dominant terms in $\delta^\mathrm{R}(X,T)$ are those homogeneous in $(\Delta X,\Delta T)$ of degree $2$ and these terms turn out to be proportional to a perfect square:  $(\Delta X-4\Delta T)^2$.  Therefore $\Delta X=4\Delta T+o(\Delta T)$ as $\Delta T\to 0$.  To split the double root present for $\Delta T=0$ therefore requires continuing the calculation to higher order; for this purpose we write $X=X^\sharp+\Delta X$ with $\Delta X=4\Delta T + \zeta$ and discover that the dominant terms in $\delta^\mathrm{R}(X,T)$ are now proportional to $3645\zeta^2+8192\Delta T^3$.  Setting these to zero gives distinct real solutions for $\zeta$ only if $\Delta T<0$.  This perturbative analysis proves that near $T=T^\sharp$ there only exist positive real solutions $X$ of $\delta^\mathrm{R}(X,T)=0$ for $T\le T^\sharp$, and these roots satisfy
\begin{equation}
X=X^\sharp+4(T-T^\sharp)\pm\sqrt{\tfrac{8192}{3645}}(T^\sharp-T)^\frac{3}{2} + o((T^\sharp-T)^\frac{3}{2}),\quad T\uparrow T^\sharp.
\label{eq:discriminant-roots-near-T0}
\end{equation}
Then, since we have already shown that there can be no repeated roots of $X\mapsto\delta^\mathrm{R}(X,T)$ for $T^\sharp<T<T_1$, there are no positive roots $X$ at all for $T$ in this range.  Therefore, for $T>T^\sharp$, only for $T=T_1$ is it possible for there to be any positive roots $X$ of $X\mapsto\delta^\mathrm{R}(X,T)$, and no such root can be simple.  Since $\delta^\mathrm{R}(X,T_1)$ is a polynomial in $X$ of degree $7$, for this special value of $T=T_1$ there are at most finitely many positive and necessarily repeated roots $X=X_i>0$, $i\le 3$, corresponding to $\chi_i=\sqrt{X_i}$.  Numerically, one sees that in fact the only repeated root of $X\mapsto\delta^\mathrm{R}(X,T_1)$ (recall that this map must have one or more repeated roots, possibly negative real or complex, by choice of $T_1$) is a positive number $X=X_1\approx31.8597$ corresponding to $\chi_1=\sqrt{X_1}\approx5.64444$ and that there are no other positive roots.

Finally, we consider the range $T<T^\sharp$.  The two simple roots of $X\mapsto\delta^\mathrm{R}(X,T)$ with the expansions \eqref{eq:discriminant-roots-near-T0} cannot coalesce, nor can any new roots appear, for $0<T<T^\sharp$ as has already been shown.  We will show that the two simple roots with the expansions \eqref{eq:discriminant-roots-near-T0} are contained within the domain $\channels$ for all $T\in (0,T^\sharp)$. To show this, we look for simultaneous solutions of the condition \eqref{eq:boundary-curve} describing the boundary of $\channels$, expressed as a polynomial condition in $(X,T)$, and $\delta^\mathrm{R}(X,T)=0$ by computing the resultant with respect to $X$.  The latter resultant is proportional to $T^9(64T-27)^6Q_9(T)$ where $Q_9(T)$ is a ninth-degree polynomial having the Sturm sequences
\begin{equation}
\Xi[Q_9](0)=\Xi[Q_9](T^\sharp)=(+,+,+,-,-,+,+,-,+,+)
\end{equation}
from which it follows by Sturm's theorem that there are no values of $T\in (0,T^\sharp)$ for which roots $X$ of $X\mapsto\delta^\mathrm{R}(X,T)$ can coincide with points of the boundary of $\channels$.  It therefore remains to determine whether the expansions \eqref{eq:discriminant-roots-near-T0} give values of $X$ that lie in the interior of $\channels$.  But near $(X,T)=(X^\sharp,T^\sharp)$ a similar local analysis of the condition \eqref{eq:boundary-curve} as already performed for the condition $\delta^\mathrm{R}(X,T)=0$ shows that \eqref{eq:boundary-curve} only has real solutions $X$ for $T\le T^\sharp$ and that these solutions have the expansions
\begin{equation}
X=X^\sharp+4(T-T^\sharp)\pm\sqrt{\tfrac{8192}{729}}(T^\sharp-T)^\frac{3}{2}+o((T^\sharp-T)^\frac{3}{2}),\quad T\uparrow T^\sharp.
\label{eq:ChannelsBoundaryNearX0T0}
\end{equation}
For $T^\sharp-T$ small and positive, the interior of $\channels$ lies between these latter two curves.
Comparing with \eqref{eq:discriminant-roots-near-T0} we then see that locally the roots $X$ of $X\mapsto\delta^\mathrm{R}(X,T)$ are indeed contained within $\channels$, and this necessarily persists throughout the whole interval $T\in (0,T^\sharp)$.

Therefore, the only points $(\chi,\tau)\in\mathbb{R}_{> 0}\times\mathbb{R}_{> 0}$ in the exterior of $\overline{\channels}$ where the resultant $\delta(\chi,\tau)$ of $P(u;\chi,\tau)$ and $P'(u;\chi,\tau)$ vanishes are $(\chi_i,\tau_1)$, $i\le 3$.  Since by Lemmas~\ref{lem:tau-small} and ~\ref{lem:chi-small} it is known that $P(u;\chi,\tau)$ has a unique real and simple root for points $(\chi,\tau)$ in the exterior sufficiently close to the coordinate axes, and since complex-conjugate roots of $P(u;\chi,\tau)$ are prevented from bifurcating onto the real axis under continuation in $(\chi,\tau)$ unless $\rho(\chi,\tau)$ vanishes, it follows that $P(u;\chi,\tau)$ has a unique real and simple root for all $(\chi,\tau)$ in the part of the open first quadrant exterior to $\overline{\channels}$ with the possible exception of only the points $(\chi_i,\tau_1)$, $i\le 3$. For these exceptional isolated points it can in principle happen that one or more complex-conjugate pairs of roots of $P(u;\chi,\tau)$ coalesce on the real axis, but these are either roots of even multiplicity or in the case of a collision with the simple root they may add an even number to its multiplicity.

Letting $u(\chi,\tau)$ denote the unique real root of odd multiplicity, we extend $u(\chi,\tau)$ to the coordinate axes within $(\mathbb{R}_{>0}\times\mathbb{R}_{>0})\setminus\overline{\channels}$ by continuity:  $u(0,\tau)=0$ for $\tau>0$ and $u(\chi,0)=\chi$ for $\chi>2$.  Note that $u(0,\tau)$ is non-simple root of $P(u;0,\tau)$, but $u(\chi,0)$ is a simple root of $P(u;\chi,0)$.  This completes the proof of Proposition~\ref{prop:u}.
\end{proof}

\begin{remark}
As pointed out earlier, numerics suggest that there is only one positive value of $\chi=\chi_1$ for which there are repeated roots of $P(u;\chi,\tau)$ for $\tau=\tau_1$.  Numerical calculations also show that there are two repeated roots of $P(u;\chi_1,\tau_1)$ forming a complex-conjugate pair.  Therefore $P(u;\chi_1,\tau_1)$ also has just one real root and it is simple.  Thus apparently there is just one exceptional point, and in fact it is not really exceptional after all.
\end{remark}

%\textcolor{red}{[That's the end of the discussion about the root $u(\chi,\tau)$.  ]}
%
%In fact, the same is true throughout the open first quadrant provided that $\chi^2+\tau^2$ is sufficiently large:
%\begin{lemma}
%Given $\delta>0$ arbitrarily small, there exists $r>0$ that $P(u;\chi,\tau)$ has exactly one real and simple root if $\chi>0$, $\tau>\delta\chi$, and $\chi^2+\tau^2>r^2$.
%\label{lem:large-radius}
%\end{lemma}
%\begin{proof}
%We express the unknown $u$ in the form $u=\chi \zeta$.  Then we express $(\chi,\tau)$ in polar coordinates via $\chi=r\cos(\theta)$ and $\tau=r\sin(\theta)$ and compute:
%\begin{equation}
%\frac{P(r\cos(\theta) \zeta;r\cos(\theta),r\sin(\theta))}{r^7\cos^3(\theta)}=Q_0(\zeta;\theta)+\frac{\sin^2(\theta)}{r^2}Q_1(\zeta), 
%\label{eq:P-polar}
%\end{equation}
%in which 
%\begin{equation}
%\begin{split}
%Q_0(\zeta;\theta)&:=(\zeta-1)\zeta^2\left((9\zeta^2-6\zeta-3)\cos^2(\theta)+4\right)^2\\
%Q_1(\zeta)&:=16(3\zeta-1)^3.
%\end{split}
%\end{equation}
%%In continuing $U$ from $U=1$ for $\theta=0$ with $r>2$, we equate to zero only the factor in square brackets in \eqref{eq:P-polar}.  
%We see that in the limit $r\to\infty$, there exists a unique simple root $\zeta=1$, and double roots at $\zeta=0$ and at the roots of the quadratic $(9\zeta^2-6\zeta-3)\cos^2(\theta)+4$.  The roots of this quadratic have imaginary parts $\pm\tfrac{2}{3}\tan(\theta)$ which are bounded away from zero uniformly for $\tan(\theta)=\tau/\chi\ge\delta>0$, so the only issue in allowing $r$ to decrease for given positive angle $\theta$ is that the double root at $\zeta=0$ might split into a pair of real roots.  
%
%To study the limit $r\to\infty$ with $\zeta$ small, one sees that a dominant balance arises by scaling $\zeta=r^{-1}\sin(\theta)\zeta_1$ for $\zeta_1=O(1)$, and then from \eqref{eq:P-polar} we find that
%\begin{equation}
%\frac{P(\cos(\theta)\sin(\theta)\zeta_1;r\cos(\theta),r\sin(\theta)}{r^5\cos^3(\theta)\sin^2(\theta)}=
%-(4-3\cos^2(\theta))^2\zeta_1^2-16 + O(r^{-1}\sin(\theta)),\quad r\to\infty,
%\end{equation}
%so to leading order $\zeta_1$ is a purely imaginary conjugate pair:  $\zeta_1=\pm 4\ii(4-3\cos^2(\theta))^{-1} + O(r^{-1}\sin(\theta))$.  This proves that the double root at $u=0$ for $r=\infty$ becomes purely imaginary as $r$ is decreased.  
%
%Therefore, the lemma is proved.  We can easily find from this calculation that 
%as $r\to\infty$, the simple root $\zeta$ of \eqref{eq:P-polar} has an asymptotic expansion in ascending powers of $\sin^2(\theta)/r^2$:
%\begin{equation}
%\zeta=1-8r^{-2}\sin^2(\theta) + O(r^{-4}\sin^4(\theta)),\quad r\to\infty
%\end{equation}
%which is uniformly valid for $0\le\theta\le\tfrac{1}{2}\pi$.
%\end{proof}
%
%It follows from Lemma~\ref{lem:large-radius} that also
%\begin{equation}
%\begin{split}
%u=Ur\cos(\theta)&=r\cos(\theta)-8r^{-1}\sin^2(\theta)\cos(\theta)+O(r^{-3}\cos(\theta)\sin^4(\theta))\\
%v=2r\sin(\theta)\frac{2u-r\cos(\theta)}{3u-r\cos(\theta)}&=r\sin(\theta)\left(1-4r^{-2}\sin^2(\theta)+O(r^{-4}\sin^4(\theta))\right)\\
%A=\frac{u-r\cos(\theta)}{2r\sin(\theta)}&=-2r^{-2}\sin(2\theta)+O(r^{-4}\cos(\theta)\sin^3(\theta))\\
%B^2&=1-4r^{-2}\cos(2\theta)+O(r^{-4}\sin^2(\theta))
%\end{split}
%\end{equation}
%in the same uniform sense of convergence for large $r$.  It also follows that
%\begin{equation}
%\lambda_0=A+\ii B = \ii - 2\ii r^{-2}\ee^{-2\ii\theta} + O(r^{-4}),\quad r\to\infty.
%\end{equation}
%This shows that for large $r$ we have $A=\mathrm{Re}(\lambda_0)<0$ for $0<\theta<\tfrac{1}{2}\pi$.
%
%\textcolor{red}{[The following calculation is probably made obsolete by the preceding calculation.]}
%In particular, one can see easily that similar expansions as indicated above for small $\tau$ are also valid for $\chi^2+\tau^2$ large.  Indeed, if one writes $u=\chi U$ and assumes that $U$ is bounded as $\tau\to\infty$ with $\xi:=\chi/\tau>0$ fixed, then 
%\begin{equation}
%P(\chi U;\chi,\tau)=P(\xi\tau U;\xi\tau,\tau)=(U-1)U^2\xi^3(4+(9U^2-6U+1)\xi^2)^2\tau^7 + O(\tau^5)
%\end{equation}
%so $U=1+O((\chi^2+\tau^2)^{-1})$, or equivalently
%\begin{equation}
%u(\chi,\tau)=\chi + O(\tau^{-1}),\quad\tau\to\infty,\quad \xi=\frac{\chi}{\tau}\ge\delta>0.
%\end{equation}
%Using this result in \eqref{eq:eliminate-v} then gives
%\begin{equation}
%v(\chi,\tau)=\tau + O(\tau^{-1}),\quad\tau\to\infty,\quad\xi=\frac{\chi}{\tau}\ge\delta>0.
%\end{equation}
%From \eqref{eq:eliminate-AB} we then also find that
%\begin{equation}
%A(\chi,\tau)=O(\tau^{-2})\quad\text{and}\quad B(\chi,\tau)^2=1+O(\tau^{-2}),\quad\tau\to\infty,\quad
%\xi=\frac{\chi}{\tau}\ge\delta>0.
%\end{equation}
%Finally, since the above results used division by $\xi$, we should investigate the solution in the limit $\chi\downarrow 0$.  
%
%\textcolor{red}{These are the small-$\chi$ expansions for $v$, $A$, and $B^2$ following from the proof of Lemma~\ref{lem:chi-small}.}
