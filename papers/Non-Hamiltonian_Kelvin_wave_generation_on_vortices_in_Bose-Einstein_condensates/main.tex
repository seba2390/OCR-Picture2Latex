% ****** Start of file apssamp.tex ******
%
%   This file is part of the APS files in the REVTeX 4.1 distribution.
%   Version 4.1r of REVTeX, August 2010
%
%   Copyright (c) 2009, 2010 The American Physical Society.
%
%   See the REVTeX 4 README file for restrictions and more information.
%
% TeX'ing this file requires that you have AMS-LaTeX 2.0 installed
% as well as the rest of the prerequisites for REVTeX 4.1
%
% See the REVTeX 4 README file
% It also requires running BibTeX. The commands are as follows:
%
%  1)  latex apssamp.tex
%  2)  bibtex apssamp
%  3)  latex apssamp.tex
%  4)  latex apssamp.tex
%onecolumn,12pt,tightenlines,onecolumn
\documentclass[aps,graphicx,reprint,onecolumn,12pt,tightenlines,longbibliography]{revtex4-1}
\usepackage{amssymb,amsbsy,times,fancyhdr,color}
\usepackage{amsmath}
\usepackage{mathrsfs,amsfonts}
%\usepackage[dvips]{epsfig,graphics,lscape}
%\usepackage{natbib}
\usepackage{latexsym,afterpage}
\usepackage{graphicx,subfigure,multirow}
\usepackage[colorlinks=true, urlcolor=blue, linkcolor=blue, citecolor=blue]{hyperref}


\usepackage{float}



%\documentclass[aip,reprint]{revtex4-1}

\usepackage{outlines}

\usepackage[T1]{fontenc}

%\usepackage{graphicx}% Include figure files
\usepackage{bm}% bold math


\usepackage{amsmath}
\DeclareMathOperator{\Tr}{Tr}
%\usepackage{hyperref}% add hypertext capabilities
%\usepackage[mathlines]{lineno}% Enable numbering of text and display math
%\linenumbers\relax % Commence numbering lines

%\usepackage[showframe,%Uncomment any one of the following lines to test 
%%scale=0.7, marginratio={1:1, 2:3}, ignoreall,% default settings
%%text={7in,10in},centering,
%%margin=1.5in,
%%total={6.5in,8.75in}, top=1.2in, left=0.9in, includefoot,
%%height=10in,a5paper,hmargin={3cm,0.8in},
%]{geometry}

\DeclareMathOperator{\arcsinh}{arcsinh}

\usepackage{soul}
% Edit Mode
%\newcommand{\sas}[2]{{\color{red}\st{#1}\,}{\color{blue}#2}}
% Take only the second statement
\newcommand{\sas}[2]{{#2}}


\begin{document}

\preprint{APS/123-QED}

\title{Non-Hamiltonian Kelvin wave generation on vortices in Bose-Einstein condensates}% Force line breaks with \\
%\thanks{A footnote to the article title}%

\author{Scott A.~Strong}
 \email{sstrong@mines.edu}
\author{Lincoln D.~Carr}%
\affiliation{%
 \sas{Physics Department, Colorado School of Mines}{Department of Physics, Colorado School of Mines, 1523 Illinois St., Golden, CO 80401, USA}\\
% This line break forced with \textbackslash\textbackslash
}%



%\collaboration{MUSO Collaboration}%\noaffiliation

% \author{Charlie Author}
%  \homepage{http://www.Second.institution.edu/~Charlie.Author}
% \affiliation{
%  Second institution and/or address\\
%  This line break forced% with \\
% }%
% \affiliation{
%  Third institution, the second for Charlie Author
% }%
% \author{Delta Author}
% \affiliation{%
%  Authors' institution and/or address\\
%  This line break forced with \textbackslash\textbackslash
% }%

%\collaboration{CLEO Collaboration}%\noaffiliation

\date{\today}% It is always \today, today,
             %  but any date may be explicitly specified

\begin{abstract}
Ultra-cold quantum turbulence is expected to decay through a cascade of Kelvin waves. These helical excitations couple vorticity to the quantum fluid causing long wavelength phonon fluctuations in a Bose-Einstein condensate. This interaction is hypothesized to be the route to relaxation for turbulent tangles in quantum hydrodynamics. The local induction approximation is the lowest order approximation to the Biot-Savart velocity field induced by a vortex line and, because of its integrability, is thought to prohibit energy transfer by Kelvin waves. Using the Biot-Savart description, we derive a generalization to the local induction approximation which predicts that regions of large curvature can reconfigure themselves as Kelvin wave packets. While this generalization preserves the arclength metric, a quantity conserved under the Eulerian flow of vortex lines, \sas{it does introduce}{it also introduces} a non-Hamiltonian structure on the geometric properties of the vortex line. It is this non-Hamiltonian evolution of curvature and torsion which provides a resolution to the missing Kelvin wave motion. In this work, we derive corrections to the local induction approximation in powers of curvature and state them for utilization in vortex filament methods. Using the Hasimoto transformation, we arrive at a nonlinear integro-differential equation which reduces to a \sas{fully}{modified} nonlinear Schr\"odinger type evolution of the curvature and torsion on the vortex line. We show that this \sas{dynamic}{modification} seeks to disperse localized curvature profiles. At the same time, the non-Hamiltonian break in integrability bolsters the deforming curvature profile and simulations show that this dynamic results in Kelvin wave propagation along the dispersive vortex medium.    
\end{abstract}

\pacs{Valid PACS appear here}% PACS, the Physics and Astronomy
                             % Classification Scheme.
%\keywords{Suggested keywords}%Use showkeys class option if keyword
                              %display desired
\maketitle

%\tableofcontents

% \section{\label{sec:introduction}Introduction}
% \section{\label{sec:VFW}Biot-Savart Asymptotics}
%  \begin{outline}
%   \1 Dirac formulate of vorticity distribution
%   \1 Biot-Savart on R3 to BS on R and statement of perturbed binormal flow
%   \1 Biot-Savart for plane circular arc
%   \1 Biot-Savart for quadratic
%  \end{outline}
% \section{\label{sec:Hasimoto transform}Hasimoto's Transform}
%  \begin{outline}
%   \1 Perturbed binormal flow 
%  \end{outline}
% \section{\label{sec:Sim}Simulations}

\section{\label{sec:Intro} Introduction}

Quantized vortex lines provide the simplest scaffolding for three-dimensional fluid turbulence. While vortex lines and filaments are fundamental to quantum fluids, they also appear as the geometric primitives in a variety of hydrodynamic settings including atmospheric,  aerodynamic, oceanographic \sas{hydrodynamic}{} phenomenon and astrophysical plasmas.~[\onlinecite{Andersen2014IntroductionEquilibrium}] Through an analogy with the Euler elastica, the twist and writhe of filament structures  appear in models of biological soft-matter and are used to explain the dynamics of DNA supercoiling and self-assembly of bacterial fibers.~[\onlinecite{Mesirov2012MathematicalDynamics},\onlinecite{Scott2006EncyclopediaScience},\onlinecite{Shi1999TheSupercoiling}] Recent research utilizes filaments in less exotic settings where a turbulent un-mixing provides motility to phytoplankton that simultaneously protects them from predation and enhances their seasonal survival.~[\onlinecite{Lindemann2017DynamicsCells},\onlinecite{Durham2013TurbulencePhytoplankton}] In quantum liquids, turbulent tangles of vorticity are known to undergo various changes of state. Transition to the ultra-cold regime is marked by a subsiding Richardson cascade resulting in a randomized tangle with no discernible large-scale structure. In this state, turbulent energy is driven by vortex-vortex interactions where reconnection events trigger a cascade of wave motion along the lines. When these interactions become infrequent, the turbulence begins to relax to a configuration where the mobile vortices spend most of their time in isolation. In this paper, we show that vortex lines in perfect barotropic fluids seek to transport bending along their length in an effort to disperse localized regions of curvature. In mean-field Bose-Einstein condensates, this dynamic predicts the generation of Kelvin waves which are capable of insonifying the Bose gas, providing a pathway to relaxation of ultra-cold quantum turbulence.~[\onlinecite{Leadbeater2001SoundReconnections},  \onlinecite{VanGorder2015TheFluid}] 

Our result derives from the Hasimoto transformation of the Biot-Savart description of the flow induced by a vortex line. We begin with a natural parameterization of the vortex about an arbitrary reference point. In this setting, the Biot-Savart integral (BSI) can be calculated exactly, and asymptotic formulae may be applied to get an accurate description of the velocity field in terms of elementary functions. \sas{}{In the analogous setting of electromagnetism, a steady line current plays the role of vorticity and literally induces a field, in this case magnetic, which has a representation given by the BSI.} \sas{This}{Our analysis yields an induced velocity field generating a vortex} dynamic \sas{}{that} preserves the Hamiltonian structure associated with the Eulerian flow of vortex lines\sas{}{.} \sas{and a}{Additionally, the velocity field a}dmits a Hasimoto transformation \sas{that}{which} describes \sas{}{the evolution of} vorticity through its curvature and torsion. The geometric dynamics are provided by a nonlinear, non-local integro-differential equation of Schr\"odinger type that can be reduced to a local differential equation, which is fully nonlinear in powers of curvature. An analytic analysis assisted by symbolic computational tools  indicates that \sas{all}{} higher order contributions reinforce the changes described by the first correction to the lowest order integrable structure. When simulated, we find that  regions of localized curvature disperse\sas{}{,} causing the production of traveling Kelvin waves. In addition, a gain mechanism emerges to support the dispersion process keeping the helical waves from being absorbed into \sas{the}{an} otherwise straight \sas{background}{vortex line}. 

The local induction approximation (LIA) is the lowest order truncation of BSI representations of the velocity field induced by a vortex line and is used to describe the local flow about vortex points.  The Hasimoto transformation (HT) shows that this is an integrable theory of the curvature and torsion dynamics and, it is argued, incapable of modeling energy transfer through Kelvin waves. It is expected that to solicit Kelvin waves models must undermine or, at least, reallocate the conserved quantities of the flow. There are currently two distinct ways to adapt the LIA to accommodate the study of Kelvin waves. One option is to consider perturbations of the Hasimoto transformation which are known to introduce non-locality and non-integrability into the geometric evolution.~[\onlinecite{Majda2002VorticityFlow}] This technique can study wave motion on wavelengths much smaller than the radius of curvature and much larger than the core thickness. Here the focus is on  kink and hairpin formations in classical vortices capable of self-stretching.~[\onlinecite{Klein1991Self-stretchingLine},\onlinecite{Klein1991Self-stretchingSolutions}] The second method is to consider approximations of the Hamiltonian structure within the LIA and is capable of modeling resonant interactions between Kelvin waves. Currently, there are two derivations of this result which predict cascade features and time scalings arising from the kinetics of Kelvin waves.~[\onlinecite{VanGorder2017MotionApproximation},\onlinecite{Laurie2010InteractionSuperfluids},\onlinecite{Boffetta2009ModelingHelium}]  While the exact source of driven Kelvin wave motion is an open topic~[\onlinecite{White2014VorticesCondensates.},\onlinecite{Kozik2010CommentAndLvov},\onlinecite{Lebedev2010Reply:Turbulence}], there is little doubt that vortex plucking through reconnection generates helical wave motion.~[\onlinecite{Fonda2014DirectReconnection.}] Our results do not seek to model the inception of localized regions of curvature and are instead focused on how the fluid responds to curved abnormalities on the vortex\sas{.}{ and are based on the interplay between the characteristic length scales of curvature, arclength and core size, which are highly constrained in the LIA.} Where the LIA asserts that the curvature, $\kappa$, and torsion, $\tau$, obey a cubic focusing nonlinear Schr\"odinger equation, \sas{}{both Hamiltonian and integrable}, our generalization predicts that the function $\psi = \kappa \mbox{Exp}[\int_{0}^{s}ds' \tau]$ obeys the non-Hamiltonian evolution  $i\psi_{t}+\psi_{ss} + |\psi|^{2} \psi/2 + \left[\tilde{\alpha} \psi\right]_{ss} + f(|\psi|)|\psi|^{2}\psi =0$ where $\tilde{\alpha}$ and $f$ are even functions of curvature with related coefficient structures. \sas{}{An evolution is Hamiltonian if it corresponds to a flow induced by a highly structured vector field. For a completely integrable evolution, one has the ability to utilize linearizations throughout the phase space of solutions associated with a Hamiltonian evolution.  The relationship between the flow and the vector field \sas{which induces}{inducing} it demands that the operator of the Hamiltonian evolution be self-adjoint. Our modified evolution is incapable of supporting such a structure and is, consequently, non-Hamiltonian.} \sas{This}{Moreover, its} fully nonlinear correction allows for energy transfer between helical modes of the vortex line, which significantly alters the evolution of the solitons predicted by the \sas{integrable cubic base}{LIA}.  
% To gain intuition over the resulting non-Hamiltonian structure found in curvature driven arclength conserving perturbations of LIA, we simulate the dynamics of an initially localized vortex soliton predicted by Hasimoto. Such a vortex configuration is a natural consequence of the tent or hairpin formations forming through reconnection events~[\onlinecite{deWaele1994RouteReconnection}]  whose scaling and dynamical properties have been further incorporated into the Biot-Savart model.~[\onlinecite{Kimura2018ScalingEvolution},\onlinecite{Kimura2018AEvolution}] It is expected that a KW cascade, driven at the length scale of reconnection, couples the turbulent energy to the mass field causing phonon excitations permitting an ultracold turbulent quantum fluid to dissipate energy through sound emission.~[\onlinecite{Leadbeater2001SoundReconnections},  \onlinecite{VanGorder2015TheFluid}] It has been argued that the integrability of LIA is incapable of modeling KW dynamics and that non-locality must be introduced to overcome this issue.~[\onlinecite{VanGorder2017MotionApproximation}, \onlinecite{Boffetta2009ModelingHelium}, \onlinecite{Laurie2010InteractionSuperfluids}] Under simulation we see that the enhanced dispersion and gain/loss mechanism manifest in the non-Hamiltonian term is responsible for creating helical excitations of the otherwise straight vortex line and are similar to results of two-dimensional restrictions of LIA.~[\onlinecite{VanGorder2016SolitonsApproximation}] At the same time, when the evolution is applied to KW perturbed vortex rings we see speed modifications, with minimal(?) distortion of the curvature profile, which are analogous to results found in previous studies.~[!] This body of results indicates that isolated vortex lines may prefer to store bending in the form of harmonic coils and that non-integrable flows seek to transport bending to such states via KW cascade. 

The body of this paper is organized into three sections. First, in Sec.~(\ref{sec:BSI}), we derive an exact expression for the BSI representation of the velocity field induced by a vortex line which requires the use of incomplete elliptic integrals. Application of asymptotic formulae makes accessible an expansion of the field in powers of curvature. The coefficients in this expansion depend on the characteristic arclength in ratio with the vortex core size and can be tuned to numerical meshes used in vortex filament methods. Second, in Sec.~(\ref{sec:Hasimoto transform}), we apply the Hasimoto transform to describe the effect of the vector evolution in terms of the curvature and torsion. We show that the corrected dynamic is non-Hamiltonian and allows for helical wave dispersion supported by a gain mechanism. Lastly, we simulate the evolution of soliton, breathing\sas{}{,} and ring dynamics. Specifically, we consider the bright curvature soliton that produces Hasimoto's vortex soliton and find that at various degrees of non-Hamiltonian correction the corrective terms seek to disperse bending in the form of helical wave generation along the vortex line. After this we consider breathing and ring states to find that the intuition given by changes to Hasimoto's soliton carries over and that both cases breakdown into helical wave motions. A notable example is that of a helically perturbed vortex ring which experiences far less dispersive deformation and, in this way, appears stable under simulation. \sas{Perhaps this is an indication that vortex line structures prefer to store bending in helical configurations.}{}      
% An ultra-cold quantum tangle is a collection of vortex lines where the Richardson cascade, which is responsible for energy transport from large scales to smaller ones, no longer dominants the dynamics. After this, the randomly tangled dense web of vorticity is driven by vortex-vortex reconnections which trigger a cascade of wave motion along the lines. As the interactions become less frequent, the turbulence begins to relax to a state where the dynamic vortices spend most of their time in isolation. The simplest model of vortex line dynamics is given by the local induction approximation (LIA) which is an integrable theory considered incapable of modeling KW without compromising some amount of its  conserved quantities.~[\onlinecite{VanGorder2017MotionApproximation}, \onlinecite{Boffetta2009ModelingHelium}, \onlinecite{Laurie2010InteractionSuperfluids}] For arbitrary perturbations away from LIA one finds a non-local, non-integrable evolution of curvature and torsion on the vortex line which is capable of modeling wave motion that is small with respect to the characteristic arclength and large with respect to the radius of curvature.~[\onlinecite{Majda2002VorticityFlow}] More recent adaptations seeking to  accommodate helical Kelvin waves have sought to truncate expansions of Hamiltonian formulations of the Biot-Savart integral (BSI) in powers of wave amplitude gradients and permits the study of resonant interactions of KW. Our result compliments the prior research by deriving a KW producing evolution from the BSI which maintains the arclength conservation of Eulerian evolution of lines and describes the vortex as a dispersive medium capable of gain and loss. 
% is to generalize the asymptotic arguments leading to the LIA which then gives a non-Hamiltonian evolution of the curvature and torsion on an isolated vortex line under the Hasimoto map. While our result introduces non-locality to the geometric evolution it can be recast into a differential equation. On the other hand, We show that while this non-Hamiltonian flow preserves the arclength metric along the vortex, it introduces a gain/loss mechanism in conjunction with plane wave dispersion and a nonlinear convex flux condition. Altogether these mechanisms support the decomposition of localized curvature into packets of Kelvin waves.
% Instead of expanding the Hamiltonian structure of  However, in our generalization of the model, the non-locality can be mediated and offers the simplest route to breaking the integrability that prevents Kelvin wave generation. 
% i our generalization 
% only mechanism for energy transport  and reconnect exists a state where the vortices are   
% in ultra-cold quantu    realized in a perfect bar a Recent experimentation 
% In this paper we show that bending helix 
% A vortex filament is the idealization of a vortex tube whose dynamics are characterized by the behavior of the vortex line representing its centerline. The Biot-Savart integral (BSI) is frequently used to study centerline dynamics that ignore core dynamics in the transverse direction. When these results are coupled to the Frenet-Serret (FS) apparatus, a geometric analysis can be conducted under the Hasimoto transformation (Hasimoto transform) which describes the evolution of the filament through the curvature and torsion dynamics of the centerline. Approximations of the full BSI reveal primitive flows endowed to isolated vortical structures and, through the Bustamante-Nazarenko correspondence~[\onlinecite{Bustamante2015DerivationEquation}], can provide a window into the dynamics of a quantum liquid at length-scales intermediate to the mean-field and Biot-Savart descriptions. While this follow-up to our previous Letter~[\onlinecite{Strong2017Non-HamiltonianCondensates}] contextualizes the application of BSI and Hasimoto transform to vortical motion in Bose-Einstein condensates (BEC), we would like to note that the geometric generality of this procedure finds application to a wide variety of natural phenomenon. Typical examples include atmospheric,  aerodynamic, oceanographic hydrodynamic phenomenon and astrophysical plasmas.~[\onlinecite{Andersen2014IntroductionEquilibrium}] Recently,  filamented vortex structures present in Langmuir cells have been applied to modeling phytoplankton motility where a turbulent un-mixing simultaneously protects them from predation and, at the same time, enhances seasonal survival.~[\onlinecite{McKeown2017TheCollisions},\onlinecite{Lindemann2017DynamicsCells},\onlinecite{Durham2013TurbulencePhytoplankton}] Additionally, through an analogy with the Euler elastica, modeling of biological soft-matter, whose dynamics are dominated by the twist and writhe of filament structures, is used to explain the dynamics of DNA supercoiling and self-assembly of bacterial fibers.~[\onlinecite{Mesirov2012MathematicalDynamics},\onlinecite{Scott2006EncyclopediaScience},\onlinecite{Shi1999TheSupercoiling}] Thus, a deeper understand of those dynamics fundamental to filamented vortical structures is broadly applicable to a variety of physical models. 
% Our previous Letter reviews the consequences associated with moving past the integrable first-order approximation to the ambient flow given by the local induction approximation (LIA) to BSI. This flow is capable of deforming and transporting an initial vortex geometry embedded into a perfect barotropic incompressible fluid which is consistent with mean-field models of BEC.~[\onlinecite{Salman2013BreathersVortices}]  It was found that initial solitons configurations of the lowest-order  integrable theory are evolved according to a non-Hamiltonian equation that enhances dispersion and introduces nonlinear gain/loss mechanisms which act to decompose the initially localized curvature distribution into helical Kelvin wave (KW) packets traveling along the vortex line. While integrability of the geometric variables from the FS apparatus is compromised is a non-Hamiltonian way, pure binormal flow conserves arclength regardless of modifications to its magnitude. Such a flow is therefore consistent with the view that the Euler equation defines a Hamiltonian flow acting on vortex filaments and that the evolution conserves the arclength metric. In this work, we outline the technical details of the BSI analysis leading to the Hasimoto transform that endows vortex lines with dynamics beyond LIA. We corroborate this symbolic analysis with illustrative simulations. As a procedure, this vortex filament workflow (VFW) provides a connection between the vector analysis of hydrodynamic problems to the the differential geometry of a vortex line.  % and interpreted in the context of initiation of wave turbulence in a BEC.[CheckBlueBook]
% The results are useful in informing numerical procedures known as vortex filament methods~[\onlinecite{Hanninen2014VortexTurbulence.}]  but also provide a description of vorticity as a wave supporting dispersive medium.~[\onlinecite{Salman2013BreathersVortices}, \onlinecite{VanGorder2017MotionApproximation}]
% %[BLUE BOOK STATEMENT AND CITATION] 
% The VFW begins with the Biot-Savart integral (BSI) which results from the study of a vector Poisson equation through Green's function methods and yields a representation for the velocity field corresponding to a specific vortical structure.[!]  In this sense, the BSI can be thought of as the unique left-inverse of the curl operator.~[\onlinecite{Parsley2012TheThree-sphere},\onlinecite{DeTurck2008Electrodynamics3-space}] As a volume integral over the fluid space, it reduces to a line integral when vorticity is ideally localized to a topological defect, in the sense of Helmholtz, of co-dimension two. [!Pisman, ArnoldCo-Author and his student] The LIA, or vortex filament equation and local induction equation, is the leading-order term in an asymptotic approximation of the BSI about an arbitrary reference point on the vortex line and asserts that the induced non-circulatory, non-tangential flow is in the binormal direction.~[!Batch,RiccaOverview,] Under a time rescaling, at lowest-order the magnitude of this binormal flow (BF) has a linear dependence on the local curvature and explains why vortex rings with smaller radius propagate faster than those with a larger radius. That said, it is possible to exactly recover the field induced by a plane circular arc of vortex~[\onlinecite{Strong2012GeneralizedTurbulence}] and, from this, a general expansion of the flow field in powers of curvature with lowest-order LIA as a consequence. This asymptotic approximation of a regularized BSI straightforward to incorporate into pre-existing filament methods and make non-Hamiltonian evolution resulting in KW generation possible. Adapting such schemes to include higher-order curvature effects, manifesting from lengths nearing the mean-field scales, relies on information already present in the numerical vortex mesh and are significantly more efficient than their continuum counterparts. With recent progress in experimental observation of vortex-vortex interactions~[\onlinecite{Serafini2017VortexCondensates}] and the construction of BEC experiments focused on large space-time scales[!], the is an increasing importance multiscale modeling of vortex lines. However, to understand the general predictions on the vortex wave motion, we must transform the local vector flow to an evolution of the local curvature and torsion. 
% Next paper set up. In the introduction of our previous paper, which details the extension of LIA to the curvature driven binormal flow, we discuss Hasimoto transform as a manifestation of MT and thus a fundamental connection to the geometric perspective of hydrodynamic flows. In this paper we make clear the connection between continuum hydrodynamics and the study of fluid spaces with topological defects of co-dimension one.    previous paper we 
%
% Co-dimension 2 is accessible through generalizations of Hasimoto transform given by MT. 
%
% Make a graph of the VFW connections.
%
% To gain intuition over the resulting non-Hamiltonian structure found in curvature driven arclength conserving perturbations of LIA, we simulate the dynamics of an initially localized vortex soliton predicted by Hasimoto. Such a vortex configuration is a natural consequence of the tent or hairpin formations forming through reconnection events~[\onlinecite{deWaele1994RouteReconnection}]  whose scaling and dynamical properties have been further incorporated into the Biot-Savart model.~[\onlinecite{Kimura2018ScalingEvolution},\onlinecite{Kimura2018AEvolution}] It is expected that a KW cascade, driven at the length scale of reconnection, couples the turbulent energy to the mass field causing phonon excitations permitting an ultracold turbulent quantum fluid to dissipate energy through sound emission.~[\onlinecite{Leadbeater2001SoundReconnections},  \onlinecite{VanGorder2015TheFluid}] It has been argued that the integrability of LIA is incapable of modeling KW dynamics and that non-locality must be introduced to overcome this issue.~[\onlinecite{VanGorder2017MotionApproximation}, \onlinecite{Boffetta2009ModelingHelium}, \onlinecite{Laurie2010InteractionSuperfluids}] Under simulation we see that the enhanced dispersion and gain/loss mechanism manifest in the non-Hamiltonian term is responsible for creating helical excitations of the otherwise straight vortex line and are similar to results of two-dimensional restrictions of LIA.~[\onlinecite{VanGorder2016SolitonsApproximation}] At the same time, when the evolution is applied to KW perturbed vortex rings we see speed modifications, with minimal(?) distortion of the curvature profile, which are analogous to results found in previous studies.~[!] This body of results indicates that isolated vortex lines may prefer to store bending in the form of harmonic coils and that non-integrable flows seek to transport bending to such states via KW cascade. 
%%%CONCLUSION NOTE%%% We cannot expect these mechansims to go on unabated and so we should seek connections to the length scale censored by BN regularization.  
% what could be called the Vortex Filament Workflow (VFW). The VFW is a procedure which seeks to understand the wave motion of a vortex medium by characterization of the ambient flow through vector analysis and transformation of these results to evolutionary problems on the intrinsic differential geometric variables of the centerline space curve. The results of VFW give us an analytic understanding of the wave motion as well as prescriptions for computational routines. 
% [old] Vortex dominated flows are often associated with turbulent fluids. Perhaps the simplest case is that of an isolated vortex line. According to Helmholtz's theorems this vortex must either close in on itself or terminate on the boundaries, which is to say that it must be a topological defect that undermines the connectedness of the space. Furthermore, Kelvin's circulation theorem implies that the vortex defect moves as if it were frozen into the background flow. Thus, understanding the vortex dynamics is tantamount to providing a representation to the induced velocity field. For the case of a vortex line, it is possible to explicitly calculate the induced field in terms of Jacobi elliptic functions which, in turn, can be mapped directly onto the evolution of the vortex geometry through the Hasimoto transform. 
%
\section{\label{sec:BSI} The Biot-Savart Integral and Local Induction Models}
%
The special orthogonal group acts to transform real three-dimensional space by committing rigid rotations about a specific axis which is given by the curl operator.~[\onlinecite{Lax2007LinearApplications}] If we define the instantaneous velocity of a fluid continuum over this spatial domain, then curl represents the axis about which a fluid element rotates. Curves that are parallel to the vorticity vector are called {vortex lines}.~[\onlinecite{Saffman1993VortexDynamics}] A vortex filament is the idealization of a vortex tube whose dynamics are characterized by the behavior of the vortex line. The Biot-Savart integral (BSI) is frequently used to model line and filament dynamics which ignore transverse variations to the vortex core. The BSI provides the unique velocity field such that $\nabla \times \bm{v}=\boldsymbol{\omega}$ defines the vorticity~[\onlinecite{Berselli2002SomeEquation},\onlinecite{Callegari1978MotionVelocity}] and can be thought of as the unique left-inverse of the curl operator.~[\onlinecite{Parsley2012TheThree-sphere},\onlinecite{DeTurck2008Electrodynamics3-space}] In accordance with Helmholtz decomposition of $\mathbb{R}^{3}$, the BSI treats the  velocity field as being sourced by vortical elements convolved with Poisson's formula for the Green's function of a stream reformulation of the hydrodynamic problem.~[\onlinecite{Batchelor2000AnDynamics},\onlinecite{Andersen2014IntroductionEquilibrium}] If the evolution of the vortex is given by the Euler equations, then it is known to be arclength conserving Hamiltonian flow and, in this case, the Biot-Savart volume integral reduces to an integral over the vortex line.~[\onlinecite{Khesin2012SymplecticMembranes}]
%
\begin{figure}[h]
\includegraphics[width=\textwidth]{vortex.png}
%\includegraphics[width=2.5 in]{images/OsculatingPlaneDark_Rot.png}
%\includegraphics[width=2.5 in]{images/TangentBundle.png}
\caption{Vortex Line Geometry. (a) The local orthogonal frame, tangent (red, $\textbf{T}$), normal (blue, $\textbf{N}$) and binormal (green, $\textbf{B}$) vectors, at the initial, middle and terminal points (brown) of the vortex line, $\bm{\gamma}(s,t)$, with embellished width. At the reference point $P$ we have a local description of the vortex given by a blue semi-circle. Changes to the Frenet frame from point to point are described by the curvature, $\kappa$, and torsion, $\tau$. (b) View\sas{ing}{} down the long axis of $\bm{\gamma}$ where we see the osculating plane spanned by the tangent and normal vectors. The local geometry at $P$ is defined by the radius of curvature $R$, which is related to the curvature by $\kappa=R^{-1}$. The angle $\theta$ sweeps out an arclength from $0$ to $L$ providing a local parameterization to $\bm{\gamma}$ about $P$. The core scale is defined as $\xi$\sas{.}{; in Bose-Einstein condensates taken as the superfluid healing length.} (c) The curvature distribution associated with the vortex line in (a) with unit torsion.  (d) The normal plane is spanned by $\textbf{N}$ and $\textbf{B}$\sas{. An}{, with} observation point, $\textbf{r}=(0,x_{2},x_{3})$, placed in this plane and defined by the polar angle $\phi$.
% In (c) we plot the binormal vector at 35 evenly spaced sample points over the vortex line. For each vector the opacity and arrowhead size is scaled according to the local curvature, which indicates the velocity field associated with the local induction approximation, $\bm{\gamma}_{t} \propto \kappa \hat{\textbf{b}}$.
}
\label{fig:vortex}
\end{figure}
An isolated vortex line is depicted in Fig.~(\ref{fig:vortex}a) and can be defined distributionally for a vortex with homogeneous vorticity density, $\Omega$,
%
\begin{align}\label{eqn:vortex}
 \bm{\omega}(\textbf{r},t) = \Omega \int_{0}^{L} \delta(\textbf{r}-\bm{\gamma}) \, d\bm{\gamma}, 
\end{align}
%
where $\bm{\gamma}: (0,L) \times \mathbb{R}^{+} \to \mathbb{R}$, is the dynamic parametric representation of the one-dimensional sub-region on which the vorticity is supported.~[{\onlinecite{Andersen2014IntroductionEquilibrium}}] Additionally, if $\bm{\gamma}=\bm{\gamma}(s,t)$ is parameterized in the natural gauge, then ${\textbf{T}}= \bm{\gamma}_{s}$ is the unit-tangent vector.~[\onlinecite{Burns1991ModernFields}]  Such a distribution of vorticity reduces the BSI to an integral over the vortex line, 
%
\begin{align}\label{eqn:vortex}
 \bm{v}(\textbf{r},t) = \frac{1}{4\pi} \iiint_{\mathbb{R}^{3}} \frac{\bm{\omega}(\bm{y},t) \times (\textbf{r}-\bm{y})}{|\textbf{r}-\textbf{y}|^{3}}d\bm{y} = - \frac{\Gamma}{4\pi} \int_{0}^{L} \frac{(\textbf{r}-\bm{\gamma})\times d \bm{\gamma}}{|\textbf{r}-\bm{\gamma}^{3}|},
\end{align}
%
where the circulation/strength $\Gamma$ is the product of $\Omega$ and the characteristic volume resulting from the ideal concentration of vorticity to $\bm{\gamma}$\sas{}{,} and $\textbf{r}$ is the location at which the velocity field is observed in the normal plane, Fig.~(\ref{fig:vortex}d). \sas{}{In the context of electromagnetism, the velocity field plays the role of the magnetic field induced by a steady charge concentrated on the line, $\bm{\gamma}$.} Analogous to problems in electromagnetism, the BSI diverges on the vortex line which is an ideally localized source of the ambient velocity field.  \sas{}{We seek to characterize the flow predcited by Eq.~(\ref{eqn:vortex}). To arrive at a beyond leading order asymptotic representation of the flow predicted by the vortex, multiple layers of analytic work will be needed. To assist the reader we provide an overview of the process.} 

\sas{}{The analysis which we apply to Eq.~(\ref{eqn:vortex}) is as follows.} 
After stating a parameterization for $\bm{\gamma}$ one must consider field points asymptotically close to the vortex line with $|\textbf{r}|$ on the order of vortex core size defined by $\xi$, see Fig.~(\ref{fig:vortex}b). \sas{}{Since the field diverges at the vortex source,  we must regularize the BSI which is tantamount to eliminating high frequency oscillations along the vortex.} The local induction approximation (LIA) is the reduction of BSI to its logarithmic singularity. A classic treatment can be found in Batchelor~[\onlinecite{Batchelor2000AnDynamics}] who derives the result by formally setting the ratio of the observation point \sas{}{magnitude}, $|\textbf{r}|$, with the local radius of curvature, $R$, to zero. While this is the most expedient route to the lowest order kinematics, it quickly loses accuracy at moments where the local curvature becomes large, see Fig.~(\ref{fig:error}a). \sas{}{While our exact elliptic representation of the regularized BSI can certainly resolve the field at moments of large curvature, they obstruct our understanding of primitive  wave motions along the vortex understood through Hasimoto's transform. Thus, we apply asymptotic approximations to the exact field to get simpler representations in powers of curvature.} \sas{Such events}{Moments of large curvature} are a \sas{natural}{} consequence of vortex-vortex interactions leading to tent/cusp like formations and, when pushed far enough, reconnection. In LIA the ratio of arclength to vortex core radius is required to be large, $L\gg |\textbf{r}|\sim \xi$, which is incompatible with reconnection where the vorticity local to the event drives the dynamics. \sas{}{We now correct LIA so that we can describe the dynamics in this regime of interest.}
% For a Bose-Einstein condensate, this radius is known as the healing length, $\xi$ in Fig.~(\ref{fig:vortex}b), and is the characteristic distance needed for the density field to recover from a localized disturbance.

The dynamics nearing reconnection remain unresolved by LIA, and we seek to rectify this issue by retaining curvature dependence in the BSI integrand. While our approximation recovers LIA in the standard limit, it also allows for an interplay between the characteristic length scales of curvature, arclength and core size that is forbidden by LIA and allows for accuracy in scenarios applicable to modeling wave motions post-reconnection. \sas{}{The end result will be that the speed of the local velocity field is given by $|\bm{v}| \propto \alpha(\kappa)$, where}
%
\begin{align}\label{eqn:alpha}
 \alpha(\kappa) = \sum_{n=0}^{\infty} a_{2n}\kappa^{2n},
\end{align}
\sas{}{such that restricting the series to $a_{0}$ results in the LIA. Prior to regularization and asymptotic approximation we extract the divergent component of BSI.} First, we restrict the parameterization of the vortex line at an arbitrary point, $P$, in the osculating plane, see  Fig.~(\ref{fig:vortex}b), and consider only the binormal component of the local velocity field to get 
%
\begin{align}\label{eqn:binormal1}
 v_{3}(\textbf{r})  \propto \int_{0}^{L} \frac{\epsilon_{3jk}(x_{j}-\gamma_{j})d\bm{\gamma}_{k}}{|\textbf{r}-\bm{\gamma}|^{3}} = -x_{2}\int_{0}^{L} \frac{\gamma_{1}'}{|\textbf{r}-\bm{\gamma}|^{3}}\, ds   +\int_{0}^{L} \frac{\gamma_{2}\gamma_{1}'-\gamma_{1}\gamma_{2}'}{|\textbf{r}-\bm{\gamma}|^{3}}\, ds,
\end{align}
%
where we have made use of the Levi-Civita permutation tensor\sas{}{, $\epsilon_{ijk}$,} in conjunction with the Einstein summation convention to compute the cross-product integrand in terms of the components of the parameterization, $\bm{\gamma} = (\gamma_{1}, \gamma_{2},0)$ and their first partials with respect to arclength, $\partial_{s}\bm{\gamma} = (\gamma_{1}', \gamma_{2}',0)$.  Additionally, we may omit the first\sas{-}{}component in the location of the observer, $\textbf{r}=(0,x_2,x_3)$, who is restricted to the normal plane, see Fig.~(\ref{fig:vortex}d). A quick computation of ${v}_{2}$ reveals the circulatory counterpart to the $x_{2}$ term of the velocity field. Thus, the last term in Eq.~(\ref{eqn:binormal1}) defines a non-circulatory binormal flow, which is understood as a non-stretching dynamic capable of producing  geometric alterations to the vortex line.

To derive a locally induced flow from Eq.~(\ref{eqn:binormal1}) one must consider field points asymptotically close to the vortex line and also regularize the divergence by excising a portion of the vortex line neighboring the reference point. The length of the excised domain of integration in Eq.~(\ref{eqn:binormal1}) is often decided in an ad hoc manner.~[\onlinecite{Schwarz1985Three-dimensionalInteractions}]  However, the recent work of Bustamante and Nazarenko provides a regularization cutoff consistent with the mean field vortices of a Bose-Einstein condensate.~[\onlinecite{Bustamante2015DerivationEquation}] To make use of this we specify a parameterization of the vortex line local to the reference point and explicitly  process the binormal flow in Eq.~(\ref{eqn:binormal1}).  In generalization to Batchelor's work~[\onlinecite{Batchelor2000AnDynamics}], we consider a dynamic element of vorticity given by $\bm{\gamma}(s,t) = (R \sin(\kappa s), R - R \cos(\kappa s),0)$ where $\kappa^{-1} = R=R(s,t)$ and $\kappa s= \theta \in (-\pi,\pi)$, see Fig.~(\ref{fig:vortex}b), whose quadratic approximation is consistent with the parameterization given in ~[\onlinecite{Batchelor2000AnDynamics}]. In either case, the  exact representation of the induced field is given in terms of incomplete elliptic integrals. Since our derivation relies on differentiation of the integral with respect to an internal parameter, elliptic integrals of both first and second kind appear.~[\onlinecite{Strong2012GeneralizedTurbulence}] Integrating the final term in Eq.~(\ref{eqn:binormal1}) over the angle $\theta$\sas{}{,} which is related to arclength by $s=R\theta$, gives the following representation for the binormal flow induced by a plane circular arc, 
%
\begin{align}\label{eqn:binormal2}
\bm{v}_{\textbf{B}} = -\frac{\Gamma \kappa}{4\pi} \int_{0}^{L} \frac{\cos(\theta) - 1}{(c_{1} + c_{2} \cos(\theta))^{3/2}} \, d\theta, \quad c_{1} = \epsilon^{2}  -2 \epsilon \cos(\phi)+2, \quad c_{2} = 2\epsilon \cos(\phi)-2,
\end{align}
%
% where 
%   \begin{align}
%    c_{1} &= \epsilon^{2} + 2 -2 \epsilon \cos(\phi),\\
%    c_{2} &= 2\epsilon \cos(\phi)-2,
%   \end{align}
where $\phi$ is the polar angle of the field point in the normal plane, see Fig.~(\ref{fig:vortex}d), and $\epsilon = |\bm{r}| \kappa$ is the product of the distance of the field point and the local curvature. Except at moments of reconnection where cusps form on the vortex line, this parameter is small, though not formally zero as in Batchelor's derivation. As was perhaps first witnessed with the theory of boundary layers, the predictions in the asymptotic regime of $\epsilon \to 0$ differ significantly from those stemming from $\epsilon=0$, which prohibits the existence of large curvatures. The corresponding indefinite integral can be evaluated to the form

\begin{align}\label{eqn:IntegralEllipticCircle}
\int \frac{\cos(\theta) - 1}{(c_{1} + c_{2} \cos(\theta))^{3/2}} \, d\theta = F  \, C_{-}-E  \, C_{+}+\frac{c_{2} \sin (\theta)}{c_{2} (c_{1}-c_{2}) \sqrt{c_{1}+c_{2} \cos (\theta )}}
\end{align}
%
where
%
\begin{align}\label{eqn:coefficients}
&F=F\left(\left.\frac{\theta }{2}\right|\frac{2 c_{2}}{c_{1}+c_{2}}\right), \\
&E=E\left(\left.\frac{\theta }{2}\right|\frac{2 c_{2}}{c_{1}+c_{2}}\right), \\
&C_{\mp}=\frac{2(c_{1}\mp c_{2}) \sqrt{\displaystyle\frac{c_{1}+c_{2} \cos (\theta )}{c_{1}+c_{2}}}}{c_{2} (c_{1}-c_{2}) \sqrt{c_{1}+c_{2} \cos (\theta )}}.
\end{align}
%
\sas{where}{Here} we have used the traditional notation for the elliptic integrals native to Mathematica, which are related to the standard notations by $F(z,m) = F(z|m^{2})$ and $E(z,m) = E(z|m^{2})=\int_{0}^{z} \sqrt{1-m^{2} \sin^{2}(t)}dt$.  Of the terms in Eq.~(\ref{eqn:IntegralEllipticCircle}), only the first is divergent as $\epsilon \to 0$.  There are various asymptotic formulae for elliptic integrals.~[\onlinecite{ToshioFukushima2012SeriesIntegrals}] The result of Karp and Sitnik was found to be more accurate than the prior result of Carlson and Gustafson, in the sense of average absolute and relative errors, over a wider range of parameters.~[\onlinecite{Karp2007AsymptoticSingularity},\onlinecite{Carlson1994}] As $\epsilon\to 0$, the Karp and Sitnik representation of the divergent term in Eq.~(\ref{eqn:IntegralEllipticCircle}) is given by
%
\begin{align}\label{eqn:KS}
 F\left(\left.\frac{\theta }{2}\right|\frac{2 c_{2}}{c_{1}+c_{2}}\right)C_{-}\sim \mathcal{A} \left[ \ln\left(\frac{\epsilon  \sin\left(\theta/2\right)}{\sin(\theta/2)\sqrt{A_{3}}+\sqrt{A_{2}}}\right) -\frac{2A_{3}}{\sqrt{A_{2}}\sin(\theta/2)} \ln\left(\frac{\sqrt{4A_{1}}}{\sqrt{A_{1}}+2}\right)\right],
\end{align}	
%
where
\begin{align}\label{eqn:KSAux}
A_{1} &= \cos(\theta) +1 ,\\
A_{2} &= c_{2} \cos(\theta) - c_{1},\\
A_{3} &= 2c_{1} - \epsilon^{2},\\
D_{1} &= A_{3} (A_{3}-c_{1})/2,\\ 
\mathcal{A} &= \frac{A_{2}}{2D_{1}A_{3}^{3/2}}.
\end{align}
%
Comparing this approximation to the elliptic form gives an average absolute and relative error of less than 1.5\% over the parameter domain $(\epsilon, \theta) \in (0,1) \times (10^{-9},\pi)$, which gets significantly better away from the boundaries in $\theta$ and away from the upper bound in $\epsilon$. Noting that \sas{as $\epsilon{\to} 0$}{for $\epsilon{\ll} 0$} , $c_{1} \sim 2$, $c_{2} \sim -2$, $A_{2}\sim -2A_{1}$ and $A_{3} \sim 4$, implies the divergence manifests from the $\ln(\epsilon)$ term in Eq.~(\ref{eqn:KS}). However, discarding the remaining terms in Eq.~(\ref{eqn:KS}) raises both error measures to roughly 30\%. Thus, our interest is in how terms other than $\ln(\epsilon)$ temper the divergence away from the singularity\sas{.}{and how we can incorporate their effects into our expansion of the field strength as a function of curvature given by Eq.~(\ref{eqn:alpha}). }However, before we proceed, we must regularize the integral by omitting a portion of vortex neighboring the reference point.     
%
% \begin{align}\label{eqn:KS1}
% \frac{2 \left(\csc \left(\frac{\theta }{2}\right) \log \left(\frac{2 \sqrt{2} \sqrt{\cos (\theta )+1}}{\sqrt{2} \sqrt{\cos (\theta )+1}+2}\right) \sqrt{\frac{2 \cos (\theta ) (\text{x2} \epsilon -1)-2 \text{x2} \epsilon +\epsilon ^2+2}{-4 \text{x2} \epsilon +\epsilon ^2+4}}+\frac{\left(16 \text{x2} \epsilon -5 \epsilon ^2-16\right) \log \left(\frac{\sec \left(\frac{\theta }{2}\right) \left(\sin \left(\frac{\theta }{2}\right) \sqrt{-4 \text{x2} \epsilon +\epsilon ^2+4}+\sqrt{2 \cos (\theta ) (\text{x2} \epsilon -1)-2 \text{x2} \epsilon +\epsilon ^2+2}\right)}{\epsilon }\right)}{4 \left(4 \text{x2} \epsilon -\epsilon ^2-4\right)}\right)}{(2 \text{x2} \epsilon -2) \sqrt{-4 \text{x2} \epsilon +\epsilon ^2+4}}.
% \end{align}
%

Previously, regularization of the BSI for a vortex line were conducted ad hoc with cutoffs tuned to experimental observation.~[\onlinecite{Moore1972TheFlow}] However, a recent derivation of BSI from the Gross-Pitaevskii (GP) equation, which models mean--field Bose--Einstein condensates, provides an a priori regularization of high-frequency spatial modes.~[\onlinecite{Bustamante2015DerivationEquation}] Adapting the results to our parameterization defines a domain of integration, ${D} = (a \epsilon, \epsilon \delta)$, such that $\pi > \epsilon \delta = \kappa L$ where $a\approx 0.3416293$ and $L$ is the length of half  of the symmetric vortex arc. \sas{}{It is important to note that the cutoff parameter, $a$, of Bustamante and Nazarenko is not phenomenological as in prior regularization techniques. Instead, by reformulating the Hamiltonian structure of the Gross-Pitaevskii equation to be consistent with vortex lines, the authors were able to numerically approximate a cutoff value under the assumption that density fluctuations ceased far from the vortex. Interestingly, their derivation of a self-consistent cutoff relies on incorporating the leading order contributions of quantum pressure and the mean-field potential, in addition to the kinetic term of the Bose-Einstein condensate Hamiltonian. Thus, our generated Kelvin waves arise from a regularization procedure that takes into account a non-trivial portion of vortex core dynamics.} 

Assisted by Mathematica, we compute the coefficients of a power-series expansion of Eq.~(\ref{eqn:KS}) up through the first twenty terms. Due to the complicated dependence on $\epsilon$, computation of higher order coefficients requires more sophisticated handling of system memory. Restricting our field observation to the normal plane and averaging over $\phi$, we find that the first ten odd coefficients vanish, which is consistent with the symmetry properties of known matched asymptotic expansion.~[\onlinecite{Svidzinsky2000DynamicsCondensate}] \sas{For t}{T}he \sas{}{two-}parameter regime, \sas{$(\epsilon,\delta)\in(0,0.35)\times(a,10)$}{$0<\epsilon<0.35$ and $a < \delta < 10$}, \sas{}{corresponds to vortex lines whose radius of curvature is up to 35\% of the core radius that is integrated from the Bustamante-Nazerenko cutoff through an order of magnitude greater than the core radius.} the following regularized two-term approximation has an average absolute error that is roughly $3.1\%$ different than the exact elliptic form, 
%
\begin{align}
\label{eqn:LIA2}\alpha^{*} &= \int_{a\epsilon}^{\delta \epsilon} \frac{\cos(\theta) -1 }{(c_1 + c_2 \cos(\theta))^{3/2}}d\theta \sim a_{0} + a_{2} \epsilon^{2},\\
a_{0}(\delta,a)& = g(a)-g(\delta),\\
 a_{2}(\delta,a)& = \frac{h(a)}{48}-\frac{h(\delta)}{48}+ \left(\frac{3}{4}\right)^{2}a_{0},  
\end{align}
%
where $g(\eta)= \arcsinh(\eta)$ and $ h(\eta) =(\eta^5-12\eta^3-11\eta) (1+\eta^{2})^{-3/2} $. \sas{}{We define $\alpha$, from Eq.~(\ref{eqn:alpha}), by normalizing all coefficients of Eq.~(\ref{eqn:LIA2}) by $a_{0}$. That is, $ \alpha=\alpha^{*}/a_{0}$.}

\sas{}{Prior to taking Hasimoto's transformation, we would like to make sure that our expansion recovers LIA. Also, we would like to understand how well the beyond lowest order LIA terms in Eq.~(\ref{eqn:alpha}) approximate the BSI defined by Eq.~(\ref{eqn:binormal2}).} Neglecting the quadratic term and noting that $\arcsinh(\delta) \sim \ln(\delta)$ \sas{as $\delta{\to} \infty$}{for $\delta{\gg} 1$} and  yields $\alpha^{*} \sim -\ln(\delta) + O(1)$\sas{,}{} \sas{i.e.~}{That is, from Eq.~(\ref{eqn:LIA2}) we  recover} the standard LIA with $\delta = L/|\textbf{r}|$. In Fig.~(\ref{fig:error}a) we depict the absolute percent error, averaged over $0.1<\delta<10$ for the first, second and third corrections as a function of $\epsilon$. \sas{}{Specifically, $\alpha^{*} \approx a_{0}$ gives the LIA, while the higher order corrections correspond to  $\alpha^{*} \approx a_{0} + a_{2} \epsilon^{2}$ (First correction), $\alpha^{*} \approx a_{0} + a_{2} \epsilon^{2}+a_{4} \epsilon^{4}$ (Second), and $\alpha^{*} \approx a_{0} + a_{2} \epsilon^{2}+a_{4} \epsilon^{4}+a_{6} \epsilon^{6}$ (Third).}  We see that the terms in Eq.~(\ref{eqn:KS}) tempering the logarithmic singularity significantly improves the lowest order approximation   for regions with larger curvature. Figure (\ref{fig:error}b) provides a logarithmic plot of this error now as a function of $\delta$ and averaged over $0.05<\epsilon<1$. We see that LIA is  an inaccurate approximation of the binormal speed of the vortex when curvature is large. In Figs.~(\ref{fig:error}c) and (\ref{fig:error}d) we plot information about the coefficients in our expansion of the asymptotic representation of the local velocity field. The Domb-Sykes plot~[\onlinecite{Hinch1991PerturbationMethods},\onlinecite{Georgescu1995AsymptoticEquations}] in Fig.~(\ref{fig:error}d) shows that we expect a mean radius of convergence for the series of approximately $\epsilon=0.94$, which is consistent with the assumptions of our approximation. Altogether, we find that the LIA demands the lowest order term dominates the representation and requires the scale separation,  $\epsilon \ll \epsilon\delta \ll \delta=L/|\bm{r}|$\sas{,}{In contrast,} our generalized local induction equation permits the study of flows where the scale separation is much less restrictive. \sas{Specifically, it requires {the} vortex arcs {have} a small enough central angle, $\theta$, such that $\epsilon < \epsilon \delta < \pi$ where $\epsilon < 1$.}{Specifically, it requires that the vortex arc has a small central angle, $\theta$, and a radius of curvature bounded by vortex core size,}
%
\begin{align}
 \epsilon \delta &< \pi, \\ \epsilon &< 1.
\end{align}
%

%
\begin{figure}
\includegraphics[width=\textwidth]{ErrorChart.pdf}
\caption{Absolute errors and convergence analysis. (a) The average absolute error, $E_{\mbox{Abs}}$, of LIA and the first, second, and third corrections given by expanding the asymptotic representation of the Biot-Savart integral in powers of curvature. The error is calculated against   numerical quadrature converged to six digits of accuracy. While the accuracy improves with higher order corrections, what is noteworthy is how quickly LIA loses  its accuracy for large curvature. (b) The error as a function of $\delta$ shows that LIA is a generally inaccurate approximation.  (c) \sas{Plots t}{T}he first four non-trivial coefficients in \sas{the}{our} expansion \sas{}{correcting LIA} as a function of $\delta$\sas{ which are positive and increasing}. (d)  A Domb-Sykes plot where the data are given by a uniform sampling of the coefficients over $a + 0.01< \delta < 2$\sas{.}{ where $a$ is the Bustamante-Nazarenko cutoff.}  We see that $a_{2}/a_{1}$ varies the most over this sampling. The data for each $\delta$ in the sampling are fitted to linear models with an average square residual of $0.98$. The red curve depicts a line whose vertical intercept and slope are given by averaging over the system of linear fits. The resulting vertical intercept predicts a radius of convergence in $\epsilon$ of roughly $0.94$.}
\label{fig:error}
\end{figure}
%
% Where the LIA requires the scale separation defined by  $\epsilon \ll \epsilon\delta \ll \delta=L/|\bm{x}|$, This leads to the following scale separation, $\epsilon \ll \epsilon \delta < \delta = L/|\textbf{x}|$. Upon ultraviolet regularization and application of the asymptotic formula, one can prove the existence of a convergent expansion of $\alpha$ in powers of $\epsilon$, about $\epsilon=0$, with coefficients depending only on $\delta$ and $x_{2}$. [cite formal power series] 

% We focus our attention on the logarithmic singularity of the previous term,
% %
% \begin{align}\label{eqn:KSLog}
% \log \left(\frac{\sec \left(\frac{\theta }{2}\right) \left(\sin \left(\frac{\theta }{2}\right) \sqrt{-4 \text{x2} \epsilon +\epsilon ^2+4}+\sqrt{2 \cos (\theta ) (\text{x2} \epsilon -1)-2 \text{x2} \epsilon +\epsilon ^2+2}\right)}{\epsilon }\right)
% \end{align}
% %
% so that $v_b = (\kappa F(\epsilon) + O(1))\hat{\textbf{b}}$.[NOTE THIS IS TOO ERROR PRONE] Next we introduce the bounds for the integral consistent with the cutoff of Bustamante and Nazarenko, 
% %
% \begin{align}\label{eqn:KSLogCutOff}
% \frac{2 \left(\csc \left(\frac{\text{Upper}}{2}\right) \log \left(\frac{2 \sqrt{2} \sqrt{\cos (\text{Upper})+1}}{\sqrt{2} \sqrt{\cos (\text{Upper})+1}+2}\right) \sqrt{\frac{2 \cos (\text{Upper}) (\text{x2} \epsilon -1)-2 \text{x2} \epsilon +\epsilon ^2+2}{-4 \text{x2} \epsilon +\epsilon ^2+4}}+\frac{\left(16 \text{x2} \epsilon -5 \epsilon ^2-16\right) \log \left(\frac{\sec \left(\frac{\text{Upper}}{2}\right) \left(\sin \left(\frac{\text{Upper}}{2}\right) \sqrt{-4 \text{x2} \epsilon +\epsilon ^2+4}+\sqrt{2 \cos (\text{Upper}) (\text{x2} \epsilon -1)-2 \text{x2} \epsilon +\epsilon ^2+2}\right)}{\epsilon }\right)}{4 \left(4 \text{x2} \epsilon -\epsilon ^2-4\right)}\right)}{(2 \text{x2} \epsilon -2) \sqrt{-4 \text{x2} \epsilon +\epsilon ^2+4}}-\frac{2 \left(\csc \left(\frac{\text{Lower}}{2}\right) \log \left(\frac{2 \sqrt{2} \sqrt{\cos (\text{Lower})+1}}{\sqrt{2} \sqrt{\cos (\text{Lower})+1}+2}\right) \sqrt{\frac{2 \cos (\text{Lower}) (\text{x2} \epsilon -1)-2 \text{x2} \epsilon +\epsilon ^2+2}{-4 \text{x2} \epsilon +\epsilon ^2+4}}+\frac{\left(16 \text{x2} \epsilon -5 \epsilon ^2-16\right) \log \left(\frac{\sec \left(\frac{\text{Lower}}{2}\right) \left(\sin \left(\frac{\text{Lower}}{2}\right) \sqrt{-4 \text{x2} \epsilon +\epsilon ^2+4}+\sqrt{2 \cos (\text{Lower}) (\text{x2} \epsilon -1)-2 \text{x2} \epsilon +\epsilon ^2+2}\right)}{\epsilon }\right)}{4 \left(4 \text{x2} \epsilon -\epsilon ^2-4\right)}\right)}{(2 \text{x2} \epsilon -2) \sqrt{-4 \text{x2} \epsilon +\epsilon ^2+4}}
% \end{align}
% %
% to find that this is to within less than a percent average and absolute error of the divergent term in the binormal flow BSI. 
% [Right now what I've found is that the two-term approximation gets us to within 3\% of the actual divergent term for $\epsilon = 0.11$, which implies that $|x| = 11\% R$ for the a ring we have something like $\epsilon = 0.058$, which implies that $|x| = 5.5\%R$ and so our shape can be about as ``twice as bent'' as a typical ring in the wild.]
% % This section justifies Eq.(1) from PRL
% \begin{outline}[enumerate]
% %  \1 Vorticity defined distributionally correctly and with the right units
% %  \1 Reduction of Biot-Savart to a line integral
% %    \2 Pre-statements: Where/why it comes from
% %    	\3 curl inversion
% %     \3 PDE 
% %     \3 vector analysis
% %    \2 post statements: see old PRL drafts for ideas
%  \1 Asymptotic analysis comparisons for parabolas and circles
%    \2 Decompose Mathematica
%    \2 Plan for this to go into the thesis appendix and make available
%    \2 Comments about issues associated with parabola
% \end{outline}
For a barotropic inviscid fluid, Kelvin's circulation theorem tells us that a vortex line flows as if frozen into the ambient fluid flow. Thus, the contortions it undergoes result from the flow which it induces. Furthermore, if the fluid is incompressible and of infinite extent, then its autonomous dynamics are completely determined by the Biot-Savart integral, which provides a representation of the ambient flow. Since the vortex line inherits the velocity of the fluid background, we have the following evolution law for the vortex, 
%
\begin{align}\label{eqn:BNF}
\frac{\partial \bm{\gamma}}{\partial t} = \frac{\Gamma a_{0} }{4\pi} \alpha(\kappa) \frac{\partial \bm{\gamma}}{\partial s}\times \frac{\partial^{2} \bm{\gamma}}{\partial s^{2}},
\end{align}
%
where $\alpha = \alpha^{*}/a_{0}$, given by Eq.~(\ref{eqn:LIA2}), is even in the curvature variable and $\bm{\gamma}_{s}\times\bm{\gamma}_{ss}={\textbf{B}}$. The local induction approximation is then the linear approximation to $\alpha$, in $\kappa$, where $a_{0} = \ln(\delta)$ and the starting point of Hasimoto's mapping. In the next section, we apply this transformation to a generalization of Eq.~(\ref{eqn:BNF}) and show that higher order curvature effects break the fragile integrability and allow the vortex medium to support a wider array of nonlinear waves. 

% While we emphasize the case of parameterizing locally with a plane circular arc, we do state the results for an exact treatment of Batchelor's approximation. However, we note that Hasimoto's global analysis of the kinematic results derived from BSI necessitates solving the Frenet-Serret equations. For a natural gauge, these equations form a non-autonomous system of linear ordinary differential equations. is When stated in the natural gauge  a choice of gauge so that the parameter  apparatus. 
% However, when either representation is treated exactly, the resulting This generalization   quadratic approximation of $\bm{\gamma}$
% As is often the case, our field diverges as we approach the source. It is well known that for a vortex line, this divergence is realized as an increasing circulatory velocity and a drift in the local binormal direction.[Batchelor, Saffman] To see this we can parameterize our vortex in the osculating, $\bm{\gamma} = (f,g,0)$, to get
% %
% \begin{align}\label{eqn:vortex}
% \int = ,
% \end{align}
% %
% where the first two integrals in the decomposition are associated with the circulation and the last with binormal drift. This binormal flow has interesting and novel consequences on the geometric properties of the vortex line. To resolve the integral we must assume a local parameterization. There are, of course, many possible parameterizations to choose. We will explore our previous result associated with a plane circular arc [StrongCarr2012] and adapt these results to a filament with quadratic parameterization. 
% From our work in 2012 it is clear that the non-circulatory and non-axial flow induced by a plane circular arc has an exact representation through the elliptic integrals.
% The affect a local flow has on the global geometry can be described by the Hasimoto transformation (Hasimoto transform).~[\onlinecite{Hasimoto1972}] The Hasimoto transform is a coordinate change which can decouples the evolution of an extrinsic shape from that of its intrinsic geometry, with the Frenet-Serret equations operating as the intermediary. Shortly after Hasimoto's discovery, it was realized that LIA of the velocity field were Killing on $\mathbb{R}^{3}$ and that Hasimoto transform forms a connection to the sequence of commuting Hamiltonian flows associated with the integrable cubic focusing nonlinear Schr\"odinger equation. [!,!,!] The consequences to the geometric evolution complicates itself substantially when perturbing off the Killing structure, leading to a quasilinear integro-differential equation of Schr\"odinger type, it can also lead to other mixtures Schr\"odinger and Korteweg–de Vries hierarchies.~[\onlinecite{Majda2002VorticityFlow},\onlinecite{Fukumoto1991Three-dimensionalVelocity}] Generalizing BF in orders of curvature maintains the Killing structure but introduces a non-Hamiltonian structure to the Hasimoto transform geometric problem. A modern interpretation is that the Hasimoto transform is a one-dimensional manifestation of the Madelung transformation which is an inversion of volume preserving morphisms of the phase space defined by the solutions of the Euler equations, to the projective space of non-vanishing complex functions.~[!] In which case, we see the VFW as a systematic study of Euler fluids whose phase space is topologically distinct from the isotropic state possessing trivial vorticity. Furthermore, BF is a realization of VFW for such phase spaces where the arclength metric is preserved under the evolution which is different than perturbations off the Killing structure that are capable of inducing vortex stretching. While we maintain a perspective that considers the effects BF has on isolated vortices in a BEC, in light of Onsager's conjecture, it is important to press the contextualization of these descriptions into the language of geometric hydrodynamics laid out by Arnold.~[!,!Co-author statement] Specifically, it is becoming apparent that the relaxation of ultra cold quantum turbulence through KW cascade in inexorably tied to the underlying geometric problem. 
\section{\label{sec:Hasimoto transform} Hasimoto's Transformation of Binormal Flows}
%
A space curve is defined by how the local tangent, normal and binormal frame, $(\textbf{T},\textbf{N},\textbf{B})$, changes between points, see Fig.~(\ref{fig:vortex}). The Frenet-Serret equations, in the natural arclength parameterization, is a system of first-order skew-symmetric ordinary differential equations that recovers the local frame based on how a curve fails to be straight (curvature, $\kappa$) and planar (torsion, $\tau$) along its arclength.~[\onlinecite{Kuhnel2006DifferentialManifolds}]
%A parametric curve is then recovered by integrating  the local tangent vector. 
How Eq.~(\ref{eqn:BNF}) affects the global geometry is described by the Hasimoto transformation.~[\onlinecite{Hasimoto1972}] The Hasimoto transform is a coordinate change which decouples the evolution of the extrinsic shape defined by the parametric curve from its intrinsic curvature and torsion, with the Frenet-Serret equations acting as the intermediary. The modern perspective is that the Hasimoto transform is a scalar manifestation of the Madelung transformation, which is the inversion of volume preserving mappings from the Euler equation phase space to the projective space of non-vanishing complex functions.~[\onlinecite{Khesin2017GeometricTransform}] In light of Bustamante and Nazarenko's work, the geometric analysis provided by Hasimoto transform applied to BSI flows is formally a study of isolated vortex lines in Bose-Einstein condensates. This analysis also describes Eulerian fluids whose phase space is made topologically distinct from the isotropic state with trivial vorticity through the presence of vortex lines.

The Euler evolution of vortex lines is known to be a Hamiltonian flow of the arclength metric. Shortly after Hasimoto's discovery, it was recognized that the velocity fields defined by LIA were Killing, or arclength preserving, on $\mathbb{R}^{3}$ and that Hasimoto transform connects  them to the sequence of commuting Hamiltonian flows of the integrable cubic focusing nonlinear Schr\"odinger equation.~[\onlinecite{Langer1991PoissonEquation},\onlinecite{Langer1990TheCurves}]  The Hasimoto evolution complicates itself substantially when perturbing off the Killing structure, leading to a quasilinear integro-differential equation of Schr\"odinger type, it also produces other mixtures \sas{}{of} Schr\"odinger and Korteweg-de Vries hierarchies.~[\onlinecite{Arnold1999TopologicalHydrodynamics}, \onlinecite{Majda2002VorticityFlow},\onlinecite{Fukumoto1991Three-dimensionalVelocity}] Generalizing LIA in powers of curvature maintains the Killing structure and, consequently, the Hamiltonian of the Euler equations. The cost, however, is that it introduces a non-Hamiltonian evolution to the  geometric variables of curvature and torsion. In other words, nonlinear curvature dependent binormal flow is an arclength preserving non-Hamiltonian flow on the vortex geometry that gives rise  to the \sas{non-Hamiltonian}{dispersive} bending \sas{}{generation} along the vortex.  
% It is important to note that under a different choice of gauge, the FS equations are non-linear in the tangent variable and while a natural choice is always theoretically possible, finding an arclength parameterization for a given curve is often non-trivial. For this reason, we have chosen to work with a plane circular arc, where the natural parameterization is known. 

If the Madelung transformation~[\onlinecite{Madelung1926EineSchrodinger}] describes a mean-field \sas{GP}{Gross-Pitaevskii} \sas{BEC}{Bose-Einstein condensate} as a perfect and incompressible fluid in which rotation must manifest through circulation about a topological defect known as a vortex line, then the Biot-Savart integral provides its evolution, in the absence of boundary effects. We consider the Hasimoto transformation on a perturbation of our Biot-Savart derived binormal flow, Eq.~(\ref{eqn:BNF}),
%
\begin{align}\label{eqn:Hasimoto transform1}
\frac{\partial \bm{\gamma}}{\partial t} =  A \alpha(\kappa) \kappa \textbf{B}+\mu \nu
\end{align}
%
where $A=\Gamma a_{0} / 4\pi$, $\bm{\nu}\in\mathbb{R}^{3}$ and $\mu \ll 1$. Perturbations of this form were first considered by Klein and Majda and will be a useful contrast to our result.~[\onlinecite{Majda2002VorticityFlow}] Specifically, we see that arclength metric preserving modifications of LIA generally result in gain/loss mechanisms. First, however, we express two key quantities in Hasimoto's work \sas{}{that track the frame changes from point to point and through time,}
%
\begin{align}
\label{eqn:HasNormal} \bm{\aleph}(s,t) &= ({\textbf{N}}(s,t) + i {\textbf{B}}(s,t))e^{i\phi(s,t)},\\
\label{eqn:HasWave} \psi(s,t) &= \kappa(s,t)\,  e^{i \phi(s,t)},
\end{align}
%
\sas{}{which are written in} in terms of the dimensionless phase, $\phi(s,t) = \int_{0}^{s} ds' \tau(s',t)$. The Hasimoto frame, $({\textbf{T}},\bm{\aleph}, \bar{\bm{\aleph}})$, constitutes an orthogonal coordinate system for $\mathbb{R}\times\mathbb{C}^{2}$ with respect to the Hermitian inner-product. Our derivation is greatly simplified by using the following modifications of the standard commutator and anti-commutator operators, $\left[A,B\right]= A\bar{B}-\bar{A}B$ and $\left\{A,B\right\}= A\bar{B}+B\bar{A}$, which takes into account the way complex conjugation \sas{}{, $\bar{\bm{\aleph}}=({\textbf{N}} - i {\textbf{B}})e^{-i\phi},$} appears in our adaptation of Hasimoto's transformation. \sas{}{Additionally, we will use subscript notation to denote partial differentiation, $\partial_{s} \psi = \psi_{s}$.}
% While the Hasimoto transform is an outcropping of geometric hydrodynamics ~[!,\onlinecite{Khesin2017GeometricTransform}], it seems to be a stroke of luck as $\psi$ manifests when considering dynamics of the complexified frame in conjunction with BSI induced flow. 
\sas{The first notable change to the process, due to Eq.~(), occurs when considering the dynamics of the local tangent vector,}{The first notable change to Hasimoto's process occurs when trying to express the derivative of Eq.~(\ref{eqn:Hasimoto transform1}) in terms of $\bm{\aleph}$. In Sec.~(\ref{sec:simulations}) we introduce the Frenet-Serret equations, Eq.~(\ref{eqn:FS}). For now, we note that $\textbf{B}_{s}=-\tau \textbf{N}$ which, in conjunction with Eq.~(\ref{eqn:Hasimoto transform1}) and the relation  $\left(\bm{\gamma}_{t}\right)_{s}=\textbf{T}_{t}$, gives}
%
\begin{align}\label{eqn:Hasimoto transform2}
{\textbf{T}}_{t} = \frac{i A}{2}\left[\eta, \bm{\aleph}\right] + \mu \bm{\nu}_{s}, \quad \eta = \frac{\partial}{\partial s}\left(\alpha \psi\right)=\left(\alpha \psi\right)_{s},
\end{align}
%
\sas{}{where we have assumed continuity of the second order mixed space-time partial derivatives so that $\partial_{st}=\partial_{ts}$.}
 This alteration affects a critical step in the transformation where two definitions of the mixed partial derivative of $\bm{\aleph}$ are equated. Specifically, we have the two equations
%
\begin{align}
\label{eqn:Hasimoto transform3} \bm{\aleph}_{st}&=-\psi_{t}{\textbf{T}} - \psi {\textbf{T}}_{t},\\
\label{eqn:Hasimoto transform4} \bm{\aleph}_{ts}&=iR_{s} \bm{\aleph}-i\left( R \psi +   A \eta_{s}\right) {\textbf{T}} - \frac{iA\eta}{2} \left\{ \psi,\bm{\aleph} \right\},
\end{align}
%
where the first expression derives from the definition of the frame coupled to Eq.~(\ref{eqn:Hasimoto transform2}) and the second from the orthogonal decomposition of ${\bm{\aleph}_{t}}$. Projecting out the coefficients using the local tangent gives,
%
\begin{align}
\label{eqn:Hasimoto transform5} i\psi_{t} + A(\alpha \psi)_{ss} + R \psi + i \psi\mu (\bm{\nu}\cdot {\textbf{T}}) =0,
\end{align}
%
while using the Hasimoto normal vector gives
%
\begin{align}
\label{eqn:Hasimoto transform6} R_{s} = \frac{A}{2}\left\{ \psi,\eta \right\}  + \frac{i \mu}{2} \left[\psi, \bm{\aleph}\right]\cdot \bm{\nu}_{s}.
\end{align}
%
Letting $\alpha=1$ implies that $\eta=\psi_{s}$ and if $\mu=0$, we have Hasimoto's original transformation where the first-term becomes the exact derivative of $A|\psi|^{2}/2$. However, after integrating by parts to find $R$ we have that, up to constants of integration, Eq.~(\ref{eqn:Hasimoto transform5}) is generally given by\sas{,}{the integro-differential equation,}
%
\begin{align}
\label{eqn:Hasimoto transform7} i\psi_{t} + A(\alpha \psi)_{ss} +  \frac{A\psi}{2} \int_{0}^{s} \alpha_{s'}|\psi|^{2} ds'  + \mu \left\{ i[(\bm{\aleph}\cdot \bm{\nu}_{s})_{s}-\psi\nu_{s}\cdot {\textbf{T}}] +\psi \int_{0}^{s} \mbox{Im}[\psi \bar{\bm{\aleph}}] \cdot \bm{\nu}_{s'}\,ds'\right\}=0.
\end{align}
%
\sas{which is a fully nonlinear integro-differential equation of Schr\"odinger type.}{As this evolution contains nonlinearities in the highest order derivative, it is fully nonlinear. Additionally, we say that it is of Schr\"odinger type since $\alpha\to 1$ and $\mu \to 0$ produces the cubic focusing nonlinear Schr\"odinger equation consistent with LIA.} The terms associated with $\mu$ were found to model perturbations whose  wavelength was small with respect to the radius of curvature, but long compared to core thickness. Letting $\alpha=1$ and $\bm{\nu}=\mu \textbf{B}$ reduces Eq.~(\ref{eqn:Hasimoto transform7}) to a complex Ginzburg-Landau type equation with a torsion driven gain/loss term and shows that even the simplest arclength preserving alteration to LIA is capable of breaking its fragile integrability. Focusing now on ambient flows completely characterized by Eq.~(\ref{eqn:BNF}), we let $\mu=0$ and expand $\alpha$ in powers of curvature to calculate the first integral in Eq.~(\ref{eqn:Hasimoto transform7}) explicitly. Doing so, under an appropriate time rescaling, reduces Eq.~(\ref{eqn:Hasimoto transform7}) to the fully nonlinear differential equation,
%
\begin{align}\label{eqn:Hasimoto transform8}
i\psi_{t}+\psi_{ss} + \frac{1}{2} |\psi|^{2} \psi + \left(\tilde{\alpha} \psi\right)_{ss} + f(|\psi|)|\psi|^{2}\psi =0,
\end{align}
%
such that $\alpha = 1 + \tilde{\alpha}$ and 
%
\begin{align}\label{eqn:Hasimoto transform8aux1}
f(|\psi|) = \sum_{n=1}^{\infty} a_{2n} \frac{2n+1}{2n+2} |\psi|^{2n},
\end{align}
%
where $a_{n}$ are the coefficients in the \sas{continued expansion}{series} \sas{of}{ Eq.~(\ref{eqn:alpha}) defined by the Taylor expansion of Eq.~(\ref{eqn:LIA2})}. Thus, when $\tilde{\alpha}=0$ we have the LIA, which is an integrable theory on the geometric variables from the Frenet frame. As we will see, any amount of curvature correction to the integrable theory yields a non-Hamiltonian evolution. 

Our even expansions of $\alpha$ correct the cubic focusing nonlinear Schr\"odinger equation of LIA\sas{}{, to fourth order in $\kappa$,}  in the following way 
%
\begin{align}\label{eqn:GNVC}
i\psi_{t}+\psi_{ss} + \frac{1}{2} |\psi|^{2} \psi +
   a_2 \left(\left[|\psi|^{2} \psi\right]_{ss} + \frac{3}{4}|\psi|^{4}\psi\right)+a_4 \left(\left[|\psi|^{4} \psi\right]_{ss} + \frac{5}{6}|\psi|^{6}\psi\right) =0.
\end{align}
%
\sas{and indicates a straightforward pattern to arbitrary order, i.e., $a_{n}([|\psi|^{n} \psi]_{ss} + (n+1) |\psi|^{n+2}/(n+2))$.}{} While the \sas{power}{} nonlinearities \sas{}{due to powers of $|\psi|$} can adapt to the typical Hamiltonian structure of the integrable theory, the fully nonlinear \sas{terms}{derivative terms} cannot. Specifically, the question of whether the Hermitian inner-product on the Hilbert space of complex-valued square integrable functions on the real line induces a symplectic form such that one can identify a self-adjoint Hamiltonian whose variational derivative defines a Hamiltonian vector field consistent with Eq.~(\ref{eqn:GNVC}) has a negative answer.~[\onlinecite{Holmer2008GeometricEvolution}] Consequently, Noether's theorem is inapplicable and known symmetries need not generate conserved quantities.~[\onlinecite{user153764https://mathoverflow.net/users/110090/user153764InfinitesimalEvolution}] Our inability to formulate Eq.~(\ref{eqn:GNVC}) as an infinite-dimensional Hamiltonian flow is rooted to the fully nonlinear \sas{}{derivative} terms \sas{given by, $(|\psi|^{2n}\psi)_{ss}$}{}. Considering a \sas{linear}{smooth compactly supported} perturbation\sas{}{, $ \varepsilon \xi$,} of the \sas{lowest non-trivial order}{ fully nonlinear derivative term, $(|\psi|^{2}\psi)_{ss}$} we find \sas{}{the linearization}, 
%
\begin{align}\label{eqn:Ham1}
\left( |\psi + \varepsilon \xi |^{2} (\psi + \varepsilon \xi)\right)_{ss} = \left(|\psi|^{2} \psi\right)_{ss}+ \varepsilon \left(2|\psi|^{2} \xi + \bar{\xi}\psi^{2}\right)_{ss}+O(\varepsilon^{2}).
\end{align}
%
Assuming a smooth compactly supported perturbation and test function $u$ the functional given by the induced symplectic form yields~[\onlinecite{Buhler2006AMechanics}]
%
\begin{align}\label{eqn:Ham2}
\int  \bar{u}\left[2|\psi|^{2} \xi + \bar{\xi}\psi^{2}\right]_{ss}   ds = \int \bar{\xi}\left[2|\psi|^{2} u_{ss} + \bar{u}_{ss} \psi^{2}\right]   ds \neq \int \bar{\xi}\left[2|\psi|^{2} u + \bar{u} \psi^{2}\right]_{ss}   ds,
\end{align}
%
implying that a formal self-adjointness condition cannot be satisfied.~[\onlinecite{Olver2000ApplicationsEquations}] It can be verified that the linear derivative term and higher order power terms obey \sas{and}{a} Hamiltonian structure. Thus, our break from Hamiltonian structure is due to the full nonlinearity. Though our existing space-time symmetries do not yield the ``total energy'' and ``total momentum'' conservation typically associated with Schr\"odinger evolutions, this does not preclude the existence of conserved quantities nor additional non-obvious symmetries. However, application of the SYM symmetry software package~[\onlinecite{Dimas2006APDEs}] to the $a_{2}$ correction of Eq.~(\ref{eqn:GNVC}) found no additional continuous symmetries. Also, a Mathematica package that symbolically calculates conservation laws found no low-order conserved densities.~[\onlinecite{Poole2011SymbolicDimensions}]

While this result speaks to the wholesale loss of Hamiltonian structure that appears as we move away from LIA, it tells us nothing about the wave motions of the vortex line. A useful perspective is given by decomposing the system into its real and imaginary components via Madelung's transformation to get, 
%
\begin{align}\label{eqn:Conservation}
\displaystyle \frac{\partial}{\partial t}\begin{bmatrix}
\rho \\ \tau
\end{bmatrix}
+ \frac{\partial}{\partial s}\begin{bmatrix}
2\alpha \rho \tau  \\ 
\displaystyle \left( \alpha \sqrt{\rho}\right)_{ss}/\sqrt{\rho} + f(\rho)\rho + \rho/2 - \alpha \tau^{2} 
\end{bmatrix} = 
\begin{bmatrix}
-2 \alpha_{s} \rho \tau  \\ 0 
\end{bmatrix},
\end{align}
%
where $\rho = \kappa^{2} = |\psi|^{2}$, which we call the {bending density}. If $\alpha=1$, then $a_{2}=0$ and we recover the standard hydrodynamic reformulation of cubic focusing NLS, which asserts that $\rho$ is conserved and $\tau$ obeys an Euler equation. Furthermore, curves of constant torsion define a Jacobian matrix whose spectrum reveals a single traveling wave solution which is the Hasimoto soliton or its generalization to elliptic representations of soliton trains.~[\onlinecite{Ludu2012NonlinearSurfaces}] The Jabcobian matrix in the general case, or for systems with non-constant torsion, are too complicated to analyze directly and we cannot make an assertion of hyperbolicity for the system. However, this reformulation does highlight the emergence of a source term in the bending density for non-constant $\alpha$. For compactly supported functions or those with suitable decay, we can integrate the first equation to find that \sas{}{for a segment of vortex line, parameterized by the arclength $s_0<s<s_1$,} the total bending \sas{across the vortex line} obeys 
%
\begin{align}\label{eqn:totalBending1}
\frac{d}{dt}\int_{s_0}^{s_1} \kappa^{2}ds = -2 \int_{s_0}^{s_1} \alpha_{s} \rho \tau ds = -2 \int_{s_0}^{s_1} (a_{2}+2a_{4}\rho + 3 a_{6} \rho^{2} +\cdots )\rho_{s}\rho \tau ds . 
\end{align}
%
Given that this non-conservation of the total norm can be traced back \sas{}{to} the fully nonlinear term in Eq.~(\ref{eqn:GNVC}), the previous loss of Hamiltonian structure is, perhaps, not surprising. It is interesting to note that if the coefficients $a_{n}$ are non-negative, then higher order corrections enter this formula additively and reinforce the gain/loss mechanism supplied by the first correction to LIA. \sas{}{Hasimoto originally considered a class of solitons defined by traveling curvature waves with constant torsion given by $\kappa = 2 \tau \mbox{sech}(\tau(s-ct))$ with $\tau = c/2$, which we refer to Hasimoto vortex solitons.} \sas{Considering  Hasimoto vortex soliton defined by $\kappa(s,t) = 2 \mbox{sech}(s-2t)$ with $\tau=1$,}{Setting $\tau=1$,} we calculate the \sas{}{parenthesis of the} last integrand\sas{}{, $\alpha_{\rho}$,} in Eq.~(\ref{eqn:totalBending1}) and find it to be non-negative, see Fig.~(\ref{fig:dispersion}a). This implies that, at least initially, the higher order contributions reinforce the gain/loss emergent in the first correction to LIA. Additionally, we plot the integrand, \sas{}{$\alpha_{s}\rho \tau$}, in Fig.~(\ref{fig:dispersion}a) and see that the vortex line should experience curvature gain \sas{to the right of}{ahead and loss behind} the soliton peak \sas{ and loss to the left}{which is propagating in the positive $s$ direction.} If we consider the system as a scalar conservation equation on $\rho$, for fixed $\tau$, then one can state the approximate characteristic speed as $c=[2\alpha \rho \tau]_{\rho} = 2\alpha_{\rho} \rho \tau + 2\alpha \tau$.~[\onlinecite{Toro2009NotionsEquations}] For $\alpha=1$, we have the LIA and a predicted speed of $c=2$ for the Hasimoto vortex soliton. Using our expansion for $\alpha$ provides the new approximate speed, $c= 2\tau( 1+ 2 a_{2} \rho + 3a_{4}\rho^{2} + 4a_{6} \rho^{3}+ \dots)$. As with Eq.~(\ref{eqn:totalBending1}), we see an additive influence of higher order corrections. In Sec.~(\ref{sec:simulations}) we consider simulations of Hasimoto's vortex soliton under a first correction to LIA. We find that the curvature peak, $\kappa_{\mbox{max}}$, has a strong linear relationship with the first correction, $\kappa_{\mbox{max}}(a_{2}) = 2.02810 -4.42020
a_{2}$, with a square residual of $0.9929$. Using this and a first correction of $c$ we find that the approximation quickly loses its accuracy with greater than a $3\%$ underestimation of the simulated peak speed for a $1\%$ correction strength. The implication is that the dynamics of gain/loss and torsion non-trivially affect the speed of the peak. Additionally, we see that if $\kappa \propto \mbox{sech}(s)$, then the approximate speed has a Gaussian-like profile which defines a non-convex/concave flux. \sas{}{That is, if the speed of a point on the traveling curvature wave is dependent on the value of curvature at that point, then the associated flux given by the first derivative of the speed with respect to $\rho$ shows that the speeds of points on the curvature distribution are neither strictly increasing nor decreasing functions of curvature. If, for example, the waves were strictly increasing with respect to curvature, then we would expect the soliton profile to undergo a wave steepening dynamic but this is not our case.} The simplest analog here is the Buckley-Leverett equation~[\onlinecite{Buckley1942MechanismSands}] which predicts \sas{}{a} shock front followed by a rarefaction wave. In our simulations, one can see a wake of helicity behind the propagating curvature profile. This structure is supported by the gain/loss mechanism which acts to evolve the soliton curvature profile to a step. However, this shock formation is tempered by other dynamics. In particular, there is a gross deformation of the curvature profile due to the nonlinear dispersion of helical/Fourier modes. 

A single mode helix, $\psi = \mathcal{A} e^{i(k s- \omega t)}$, which is a Fourier mode of the soliton state, initially obeys the nonlinear dispersion relation~[\onlinecite{Newton1987StabilityWaves}] associated with Eq.~(\ref{eqn:GNVC}),
 %
 \begin{align}\label{eqn:Dispersion}
  \omega(k,\mathcal{A},\lambda) = k^{2}(1+ a_{2} \mathcal{A}^{2}+a_{4}\mathcal{A}^{4}+a_{6}\mathcal{A}^{6}) - \frac{\mathcal{A}^{2}}{2} - \frac{3}{4} a_{2} \mathcal{A}^{4} -\frac{5}{6}a_{4}\mathcal{A}^{6}  - \frac{7}{8} a_{6}\mathcal{A}^{8},
 \end{align}
 % 
where again we see the corrections alter the LIA dispersion relation additively. The Hasimoto vortex soliton given by $\psi = 2 \mbox{sech}(s) e^{is}$ contains  $95\%$ of its total Fourier \sas{mass}{energy} contained between wave numbers $k\in[-2,2]$.  Figure (\ref{fig:dispersion}b) plots the group velocity, where wave amplitude, $\mathcal{A}$, is related to wavenumber $k$ via Fourier transform, for the second correction to LIA. We find that these corrective terms seek to enhance the propagation speed of long wavelength modes, which will cause the initial curvature profile to distort. Thus, together with Eq.~(\ref{eqn:totalBending1}), the $a_{4}$ and $a_{6}$ corrections reinforce the gain/loss and dispersion mechanisms seen in the $a_{2}$ correction, a pattern which \sas{}{we checked} holds for the first 10 nontrivial corrections. As we reported in a previous work [\onlinecite{Strong2017Non-HamiltonianCondensates}] for a perturbative correction, the original peak can maintain localization even under the enhanced dispersion. Thus, when the kink is discernible, it is reasonable to consider it  a dissipative soliton.~[\onlinecite{Akhmediev2005DissipativeSolitons}]
%
\begin{figure}[h]
\includegraphics[width=\textwidth]{DispersionGraph_Final.pdf}
\caption{Non-Hamiltonian gain/loss and dispersion of helical modes. Corrections to LIA increase/decrease line bending to right/left of a hyperbolic secant curvature profile with higher order corrections reinforcing this effect additively. Also, long wavelength modes propagate faster under correction causing a deformation of the initial curvature profile.  (a)  The integrand of Eq.~(\ref{eqn:totalBending1}), up through the first ten coefficients, for Hasimoto's vortex soliton (black.) The thinner and lighter black curves indicate how the integrand changes as we increase $\delta$. As we accumulate more vorticity with the Biot-Savart integral, the amplitude of the integrand increases while not distorting the basic shape. In addition, we plot the contribution due to the $\alpha_{\rho}\rho$ expansion of the integrand and see that the quantity is strictly positive. (b) The dispersion relation for helical modes of the initial state for the second correction. We see that the low wavenumber modes experience an enhanced dispersion, which grows as $\delta$ increases.}
\label{fig:dispersion}
\end{figure}
%
\section{\label{sec:simulations} Simulating Binormal Vortex Motion}

Together, continuum mechanics, vector analysis,\sas{}{and} Helmholtz's and Kelvin's theorems from fluid mechanics assert that the motion of a vortex line is prescribed by the flow of the ambient field in which it is embedded. Past the lowest order approximation, the dynamics are sufficiently complicated \sas{and}{to} necessitate\sas{}{s} the use of numerical tools. The previous sections imply two distinct simulation procedures. The first is clear cut and relies on the approximation of solutions to initial-boundary value problems evolved according to an approximation to the vector evolution, Eq.~(\ref{eqn:vortex}). With the existence of efficient routines to evaluate incomplete elliptic integrals~[\onlinecite{Fukushima2011PreciseTransformations}] it appears possible to simulate the binormal evolution outright without approximation, however, such an investigation has never been attempted. Instead people work through the full BSI over an interpolated mesh or approximations via LIA at mesh points.  We call simulations stemming from Eq.~(\ref{eqn:binormal2}), {vector \sas{evolutions/}{}simulations} and consider first and second corrections to LIA given by Eq.~(\ref{eqn:BNF})\sas{}{.} 

On the other hand, the Hasimoto transform works by separating the parameterization of the vortex line from the evolution of its intrinsic geometric description given by the curvature and torsion variables. We will call a simulation of the vortex through the \sas{of the}{} geometric variables a {Hasimoto \sas{evolution/}{}simulation}. Naturally, this procedure introduces a post-processing step, which reconstructs the curve through the Frenet-Serret equations.~[\onlinecite{Bloch1997AGeometry}] Specifically, we must find the tangent vector by solving the following non-autonomous linear system of equations 
%
\begin{align}\label{eqn:FS}
\frac{d}{ds}\begin{bmatrix}{\textbf{T}} \\ {\textbf{N}}\\ {\textbf{B}} \end{bmatrix}  = \begin{bmatrix} 0 & \kappa & 0 \\ -\kappa & 0 & \tau \\ 0 & -\tau  & 0 \end{bmatrix} \begin{bmatrix}{\textbf{T}} \\ {\textbf{N}}\\ {\textbf{B}}. \end{bmatrix} 
\end{align}
%
From the tangent vector the curve's parameterization is found. The coefficient matrix in Eq.~(\ref{eqn:FS}) is skew-symmetric and thus an infinitesimal generator of the rotations mapping the Frenet frame from one point on the vortex line to the next. More importantly, these group elements of $SO(3)$ have spinor representations in the general linear group of two-by-two complex valued matrices under the isometric mapping,  $-2 x_{k} =\Tr({X}{\sigma}_{k}),\, \,  k=1,2,3,$ where ${X} = i(x_{1} \mathbf{\sigma}_{z}-x_{2}{\sigma}_{y}-x_{3}{\sigma}_{x})$ are defined through the standard Pauli spin matrices. In this representation, the Frenet-Serret equations are implicitly defined by the lower dimensional form,
%
\begin{align}\label{eqn:FSSpinor}
 \frac{d{U}}{ds} = {F}(s) {U}, \quad 2{F}(s) = \begin{bmatrix} 0 & i\psi/2 \\ i \bar{\psi}/2 & 0 \end{bmatrix},
\end{align}
%
where $\psi = \kappa \, \mbox{Exp}\left[ \int_{0}^{s} ds' \tau \right]$ and ${U}\in\mathbb{C}^{2 \times 2}$ such that $\bar{{U}}^{\textsc{t}}  {U}={I}$, which defines the spinor tangent vector ${E}_{1}=i{U}^{-1}{\sigma}_{z} {U}$.~[\onlinecite{Grinevich1997ClosedEquation},\onlinecite{Burns1991ModernFields}]  Interestingly, the spinor representation yields a Frenet-Serret coefficient matrix where the curvature and torsion are cast into the form of Hasimoto's wave function. Thus, the second process is to simulate the evolution of vortex line configurations through  Eq.~(\ref{eqn:GNVC}) and then recover the curve geometry by the application of quadrature to the traced out tangent vector created by the numerical approximation of Eq.~(\ref{eqn:FSSpinor}). It is also possible to numerically differentiate rectified phase data to recover the torsion, which can then be used in Eq.~(\ref{eqn:FS}).
% Each simulation procedure has its own advantages and disadvantaged. Perhaps the two most important quantities to understand are the vortex line configuration and its associated curvature. With this is mind, and understanding that post-processing introduces another point of numerical error generation, we will use vector simulations to evolve the vortex geometry and Hasimoto simulations to understand curvature dynamics. [!!More about BC and Pro/Cons !!] Both simulation procedures are implemented through the \texttt{NDSolve} routine of Mathematica, set to work at 10 digits of internal precision and interpolating at the order of the underlying method chosen to  numerically integrate the system of ordinary differential equations arrived at by application of the method of lines. 

We consider the $a_{2}$ correction to the evolution of a Hasimoto vortex soliton given by the initial state $\psi(s,0) = 2 \mbox{sech}(s) e^{is}$. To gain intuition over how a perturbation affects the solitonic evolution, we consider three cases\sas{}{,} $a_{2} \in \left\{0.11,0.19,0.23\right\}$. When $a_{2}=0.11$ a small number of low wavenumber curvature modes, $13.40$\% of the total Fourier energy, begin to propagate faster than those modes responsible for 95\% of the total initial bending. At $a_{2}=0.19$, $38.55$\% of the total Fourier energy is contained in the low wavenumber modes propagating faster than the remaining modes. The final value is chosen so that 47.73\% of the initial bending is propagating faster than the remaining modes. Density plots of these three cases are given in Fig.~(\ref{fig:DPHasimoto transform}). In each \sas{}{case}, we see an asymmetric evolution consistent with the gain/loss mechanism described in Sec.~{\ref{sec:Hasimoto transform}. We also\sas{,}{} see that \sas{as}{} dispersion of low wavenumber modes smears the distribution out\sas{}{.} \sas{and the peak is not completely eroded, which is due to the support it receives from the non-Hamiltonian gain mechanism.}{Additionally, the non-Hamiltonian gain mechanism keeps the peak from being completely eroded by dispersion.} \sas{Additionally, we see that f}{F}or smaller corrections, there is a breathing feature, which causes pockets of small curvature to form tightening the localization of the curvature peak. This feature, which was confirmed with an analysis of the power spectrum, is short-lived  under strong dispersion. In Fig.~(\ref{fig:vortexconfigA2}) we depict vortex lines produced from this correction to LIA at $t=10$. Figures (\ref{fig:vortexconfigA2}a)-(\ref{fig:vortexconfigA2}c) depict vortex lines corresponding to  the density plots in Fig.~(\ref{fig:DPHasimoto transform}). The remaining two are the result of higher order correction and we see that the Hasimoto vortex soliton decomposes itself into helical excitations of the vortex line. In light of the way corrections appear additively in speed, bending and dispersion calculations, it is reasonable that the evolution to a Kelvin wave cascade is more profound at higher order.  
%
\begin{figure}
%  \includegraphics[width=0.33\textwidth]{images/DP_Hasimoto transform_011.png}
% \includegraphics[width=0.33\textwidth]{images/DP_Hasimoto transform_017.png}
% \includegraphics[width=0.33\textwidth]{images/DP_Hasimoto transform_023.png}
% \includegraphics[width=0.33\textwidth]{images/DP_Hasimoto transform_023.pdf}
% \includegraphics[width=0.33\textwidth]{images/DP_Hasimoto transform_023.pdf}
 \includegraphics[width=\textwidth]{DP_HT.png}
\caption{Density plots of Hasimoto vortex soliton under first correction. The non-Hamiltonian binormal evolution of the bright soliton state maintains the kink feature despite dispersive deformations to the curvature distribution. (a) Evolution of the initial soliton state for $a_{2}=0.11$ produces a tightly confined peak with limited dispersion to the right. A slight breathing fluctuation is present. (b) The first correction is now $a_{2}=0.19$, and we see increased dispersion and a less frequent but more pronounced breathing dynamic. (c) Lastly, we have $a_{2}=0.23$ and see that under strong dispersion the breathing abates but a propagating peak, supported by the non-Hamiltonian gain mechanism, remains. }
\label{fig:DPHasimoto transform}
\end{figure}
%
%
\begin{figure}[h]
\includegraphics[width=\textwidth]{Tube.png}
\caption{Vortex configurations. Subplots (a)-(c) correspond to the evolutions in Fig.~(\ref{fig:DPHasimoto transform}) at $t=10$. We see that the increased dispersion corresponds to Kelvin wave generation, the largest in amplitude of which is an artifact of the original kink. (d) A Hasimoto soliton under the second correction where the coefficients, $a_{2} = 0.352625$ and  $a_{4}= 0.213401$, are chosen so that $\delta \to a$, which represents the smallest length of regularized vortex permitted under the Bustamante-Nazarenko cutoff. The quick decomposition of the soliton into a Kelvin wave cascade pronounced. (e) Hasimoto soliton evolved with $a_{2} = 0.402559$ and $a_{4}=0.249431$ which corresponds to $\delta =1$. We see a minimal change to the dynamics even though $\delta \approx 3 a$ indicating the corrections to LIA imply a rapid cascade dynamic.}
\label{fig:vortexconfigA2}
\end{figure}
%
% In Fig.~(\ref{fig:error}) we see that the coefficients have well defined limits as we approach the Bustamante-Nazarenko cutoff. In this limit we have that $a_{2}=0.352626$, $a_{4}=0.213402$ and $a_{6}=0.152592$ and in Fig.~(\ref{fig:DPCorrections}) we show the plot of $|w(s,t)|^{2}$ where  
% %
% \begin{figure}[H]
%  \includegraphics[width=0.33\textwidth]{images/a2_DP.png}
%  \includegraphics[width=0.33\textwidth]{images/a4_DP.png}
%  \includegraphics[width=0.33\textwidth]{images/a6_DP.png}
% \includegraphics[width=0.33\textwidth]{images/DP_Hasimoto transform_023.pdf}
% \caption{Density plots of binormal flow $y^{2}+z^{2}$: Left to right, (a) $a_{2}=0.352626, a_4=a_6=0$, (b) $a_{2}=0.352626, a_{4}=0.213402, a_6=0$ (c) $a_{2}=0.352626$, $a_{4}=0.213402, a_{6}=0.152592$ (it is unlikely that I can achieve the resolution needed to smooth the last plot.}
% \label{fig:DPCorrections}
% \end{figure}
%
%
%  \begin{figure}[H]
% % \includegraphics[width=\textwidth]{images/a2_10sec.png}\\
% % \includegraphics[width=\textwidth]{images/a4_10sec.png}\\
% % \includegraphics[width=\textwidth]{images/a6_10sec.png}\\
% \caption{Vortex Configurations @ $t=10$: Top to bottom, (a) $a_{2}=0.352626, a_4=a_6=0$, (b) $a_{2}=0.352626, a_{4}=0.213402, a_6=0$ (c) $a_{2}=0.352626$, $a_{4}=0.213402, a_{6}=0.152592$ (it is unlikely that I can achieve the resolution needed to smooth the last plot.}
% \label{fig:vortexconfigA2}
% \end{figure}
%
%
% \begin{figure}[H]
% %  \includegraphics[width=0.33\textwidth]{images/DP_a2D.png}
% %  \includegraphics[width=0.33\textwidth]{images/DP_a2a4D.png}
% %  \includegraphics[width=0.33\textwidth]{images/a6_DP.png}
% % \includegraphics[width=0.33\textwidth]{images/DP_Hasimoto transform_023.pdf}
% \caption{Density plots of binormal flow $y^{2}+z^{2}$: Left to right, (a) $a_{2}=0.402560, a_4=a_6=0$, (b) $a_{2}=0.402560, a_{4}=0.0.249432, a_6=0$}
% \label{fig:DPCorrections}
% \end{figure}
%
%
%  \begin{figure}[H]
% % \includegraphics[width=\textwidth]{images/a2_10Dsec.png}\\
% % \includegraphics[width=\textwidth]{images/a4_10Dsec.png}\\
% % \includegraphics[width=\textwidth]{images/a6_10Dsec.png}\\
% \caption{Vortex Configurations @ $t=10$: Top to bottom, (a) $a_{2}=0.352626, a_4=a_6=0$, (b) $a_{2}=0.352626, a_{4}=0.213402, a_6=0$ (c) $a_{2}=0.352626$, $a_{4}=0.213402, a_{6}=0.152592$ (it is unlikely that I can achieve the resolution needed to smooth the last plot.}
% \label{fig:vortexconfigA2}
% \end{figure}
%
%

Prior to Hayder's work of 2014~[\onlinecite{Salman2013BreathersVortices}], a comparison between Hasimoto, vector, and mean-field simulations had not been conducted. His work showed a qualitative agreement between the methods, except at points of reconnection which the mean-field model handled natively. In Fig.~(\ref{fig:HayderCurvatures}) we depict the $a_{2}$ correction to LIA, at various strengths, applied to the Akhmediev breather considered in~[\onlinecite{Salman2013BreathersVortices}]. We see that the correction not only increases the frequency of breathing, but the dispersion retards the relaxation to a non-peaked state. That the peaks are still maintained through several cycles is due to the non-Hamiltonian gain/loss mechanism.  The appearance of small-scale structures, caused by  wave interference across the periodic boundary, resulted in inefficient simulations for larger corrections. Using the spinor representation of the Frenet-Serret equations we were able to reconstruct the vortex line and found results consistent with the Hasimoto vortex solitons applied to each loop formed by the twisting and bending of the breathing dynamic. Specifically, the traveling curvature events, emerging from \sas{a breath}{one period of the breathing}, jettison helical excitations cascading Kelvin waves \sas{energy} away from the regions of highly localized curvature making it impossible to achieve a full exhalation, see from Fig.~(\ref{fig:HayderCurvatures}b). 
%
 \begin{figure}
\includegraphics[width=\textwidth]{BreatherDiagram.png}
\caption{Evolution of space-time periodic Akhmediev breather. (a) We plot the LIA evolution of the breather first considered in~[\onlinecite{Salman2013BreathersVortices}]. (b) The first correction to LIA, where $a_{2}=0.01$, increases the frequency of breathing. (c) Increasing the strength of correction to $a_{2} = 0.05$ sees two major changes. First, the frequency of breathing continues to increase with correction strength. Second, the dispersion tends to erode the relatively flat period \sas{in-between breaths}{occurring during one period of breathing}.}
\label{fig:HayderCurvatures}
\end{figure}
%

Lastly, we conducted simulations on vortex rings perturbed by Kelvin waves and vortex rings with initially localized \sas{bumps}{out of plane perturbation to a vortex ring}, similar to those seen after reconnection in classical hydrodynamics. Under LIA, \sas{rings with bumps}{perturbed rings} oscillated about the plane normal to their direction of motion as the \sas{bump}{perturbation} release\sas{d}{s} its bending into the ring in the form of \sas{wavy perturbations}{smaller amplitude traveling helical waves.} \sas{of the ring structure.}{} Initial testing indicates that \sas{a bump will}{the perturbation} create\sas{}{s} two curvature disturbances that are similar to a Hasimoto soliton traveling around the ring. However, these features were not true soltions and lost their shape as they traveled, even under LIA. A first correction to LIA increased the speed of propagation of the ring and helical decomposition of any kinks formed on the ring, see Fig.~(\ref{fig:ringDiagram}). On the other hand, the Kelvin rings which were tested appeared stable under LIA and corrected LIA. Specifically, while the speed of rotation and propagation was enhanced, the shape was relatively un-deformed when compared against LIA. While further testing is necessary, the possible decomposition of \sas{bumps}{perturbations} into Kelvin rings may provide stability to closed vortex structures propagating through mean-field simulated BEC.   
%
\begin{figure}[h]
\includegraphics[width=\textwidth]{RingDiagram.png}
\caption{Perturbed Rings. (a) A \sas{bump}{non-planar perturbation} on a vortex \sas{}{ring}, whose height is one-third of the ring radius, and its curvature profile. (b) Curvature profiles for the three evolutions, LIA where $a_{2}=0$, $a_{2}=0.01$, and $a_{2}=0.05$ at $t=0.0005$ shows that the \sas{bump}{perturbation} jettisons curved regions away from the initial deformation.   (c) Plots of the initial vortex state under LIA and corrections. We see that the corrections greatly increases the speed of the dynamic. For LIA we see the emergence of two curvature peaks propagating \sas{of}{away from} the initial peak. The corrections produce a similar state in shorter time. Fast moving waves vibrate the ring and cause noise in the curvature distributions. That said, in the vortex shows the existence of two counter propagating kink formations. All of these dynamics seek to distribute the bending along the vortex in a way similar to that which is seen in interacting bubble rings in classical hydrodynamic settings.}
\label{fig:ringDiagram}
\end{figure}
%

\section{\label{sec:conclusions} Discussion and Conclusions}

In this paper we present\sas{}{ed} an integrability breaking modification to the local induction approximation that maintains the \sas{Killing symmetry}{arclength preserving Hamiltonian structure} of the Eulerian flow induced by a vortex line while enhancing dispersion and introducing a non-Hamiltonian gain/loss mechanism affecting the geometric properties of the vortex medium.  This correction allows localized curvature distributions to decompose into Kelvin wave packets. In fact, we derive a hierarchy of non-Hamiltonian vortex cascade evolutions, which limit to an integro-differential equation resulting from the Hasimoto transformation of arbitrary curvature dependent binormal flows defined by the Biot-Savart representation of the velocity field induced by a vortex line. While we are unable to prove positivity in the coefficient structure of our expansion, which is an open and hard mathematical problem, the first ten non-trivial coefficients do not contradict this  conjecture and a Domb-Sykes analysis predicts a radius of curvature for our expansion which is consistent with the assumptions of our derivation. If the coefficients are positive to all orders, then the higher order terms in the asymptotic expansion of the local field additively reinforce the emergent non-Hamiltonian dynamics given by the first correction to the lowest order integrable theory. In other words, all curvature driven non-stretching Eulerian evolutions of vortex lines seek to disperse locally bent regions by the generation of helical waves. \sas{}{Additionally, in the setting of Kolmogorov-Arnold-Moser theory, the implication of the break in LIA integrability contrasted with the conservation law maintained by binormal flow is unclear since not every Euler evolution is an integrable one. Future connections to the differential geometry of fluid flows may shed light on this interesting and open question.}

The coefficient formulae depend on the characteristic length scales defined by arclength, vortex core size and local curvature and are ready for integration into established vortex filament methods.~[\onlinecite{Hanninen2014VortexTurbulence.}] These methods are used to simulate quantum fluids with a dense arrangement of quantized vortices and gain efficiency by approximation of the Biot-Savart integral with locally induced flows. They result in qualitatively similar dynamics for systems where vortex-vortex and self-interaction \sas{interaction}{} is weak.~[\onlinecite{Salman2013BreathersVortices}] When these dynamics dominate the flow, mean-field models that take into account \sas{the}{additional} physics of the vortex core must be included. That said, our analysis is appropriate for flows induced by lengths of vorticity nearing these scales and may represent as far a regulated Biot-Savart line integral can take the model into a core structure. With the emergence of experiments at both larger~[\onlinecite{Note1}] and small scale\sas{}{s}, focused on tangle behavior and primitive vortex interactions~[\onlinecite{Serafini2017VortexCondensates}], the importance of efficient multi-scale models for quantum turbulence is clear.~[\onlinecite{Hanninen2014VortexTurbulence.}]   

Preliminary analysis and simulations show that local but non-integrable induction models permit the excitation of Kelvin waves and indicate that a vortex line may attempt to find stability by storing bending in helical coils arrived at by a curvature cascade. It is expected that this energy transfer process couples to the fluid so that the bending can be relaxed through long wavelength acoustic fluctuations of the mass density. Incorporation of this effect into the geometric picture supports a connection between geometric hydrodynamics and the analysis of anomalous dissipation  conjectured by Onsager.~[\onlinecite{Eyink2006OnsagerTurbulence},\onlinecite{Onsager1931ReciprocalI.},\onlinecite{Onsager1931ReciprocalII.}] The current understanding is that classical Navier-Stokes solutions exhibiting anomalous kinetic energy dissipation in the inviscid limit correspond to weak Euler solutions, the singular fields defined by vortex lines being one such example.~[\onlinecite{Drivas2017AnEquations}] Furthermore, this mechanism is expected to be non-existent above a certain degree of solution regularity. The Biot-Savart perspective may be compatible with the regularity analysis associated with Onsager's conjecture and provide a geometric connection between anomalous dissipation and the relaxation of  ultracold quantum turbulence observable through the wave motion of the vortical substructure. 

A less theoretical application can be found in the recent high-resolution imaging of vortex ring breakdown where a self-similar decomposition of toroidal rings into vortex filaments occurs and is very much in the spirit of the predictions of Richardson and da Vinci. High-speed and high-resolution imagery~[\onlinecite{McKeown2017TheCollisions}] shows that interactions leading to tent formations~[\onlinecite{Kimura2018ScalingEvolution},\onlinecite{Kimura2018AEvolution}] and the flattening of tubes leads to filament generation at fine scales. In other words, the smoke we see as classical rings collide hides a discernible vortical skeleton comprised of bent filaments. At the finest scales, vortex lines should be the most appropriate model for the dynamics of the bent classical filaments and offer an opportunity to provide experimental corroboration of line models. In a similar thread, recent experiments seek to induce vortex line interactions in Bose-Einstein condensates whose vortical structure is not dense.~[\onlinecite{Serafini2017VortexCondensates}] This offers the clearest picture of vortex line dynamics, post tent formation and will certainly be an important apparatus in understanding primitive vortex dynamics. As our understanding of Kelvin wave generation on vortex lines and filaments is still in its early stages, such theoretical/experimental crossovers will be important for the continued development of future theories.


% [Open questions: What is the relationship between bump, helical excitations and reconnection/breakdown events?  Relationship with Tsubota Barenghi work. Both together, reference recent imaging of self-similar filamentation. ]



% The Euler equation predicts that vortex lines flow with a conserved arclength. Inverting the Madelung transformation maps such flows to the Gross-Pitaevskii model of mean-field Bose-Einstein condensates. Through the work of Bustamante and Nazarenko a regularized Biot-Savart integral can be defined consistent with mean-field models. In this paper we derive an asymptotic approximation of the induced velocity field for a region of vorticity described by a plane circular arc. This non-circulatory and non-axial local motion is in the binormal direction with a magnitude determined by the local curvature. All binormal flows preserve the arclength metric and when the magnitude is linear in the curvature variable, then the local induction approximation is the result. For general curvature dependencies, the Hasimoto transformation results in a fully nonlinear integro-differential equation. Through expansions of the asymptotic approximations in powers of curvature, the integral operator can be recast as power nonlinearities. The fully nonlinear term, however, results in a non-Hamiltonian structure that gives rise to a gain/loss mechanism that bolsters the nonlinear dispersion of curvature resulting in traveling Kelvin wave packets along the vortex line. 



 


% Need to introduce smaller GP length scales to dissipate the energy away. Contextualize in terms of helical storage of bending energy. Imply a connection to biological structures. What we learn in this simplest system could inform more complicated systems found in solid mechanics. That is, if this is the simplest system, then those found in solid mechanics. A second reason to think about this is to quantify the scales omitted in this description. It is, perhaps, in these regions that the loss in regularity associated with theoretical investigations of Onsager's conjecture may be realized. 



% ] 

% Our previous Letter reviews the consequences associated with moving past the integrable first-order approximation to the ambient flow given by the local induction approximation (LIA) to BSI. This flow is capable of deforming and transporting an initial vortex geometry embedded into a perfect barotropic incompressible fluid which is consistent with mean-field models of BEC.~[\onlinecite{Salman2013BreathersVortices}]  It was found that initial solitons configurations of the lowest-order  integrable theory are evolved according to a non-Hamiltonian equation that enhances dispersion and introduces nonlinear gain/loss mechanisms which act to decompose the initially localized curvature distribution into helical Kelvin wave (KW) packets traveling along the vortex line. While integrability of the geometric variables from the FS apparatus is compromised is a non-Hamiltonian way, pure binormal flow conserves arclength regardless of modifications to its magnitude. Such a flow is therefore consistent with the view that the Euler equation defines a Hamiltonian flow acting on vortex filaments and that the evolution conserves the arclength metric. In this work, we outline the technical details of the BSI analysis leading to the Hasimoto transform that endows vortex lines with dynamics beyond LIA. We corroborate this symbolic analysis with illustrative simulations. As a procedure, this vortex filament workflow (VFW) provides a connection between the vector analysis of hydrodynamic problems to the the differential geometry of a vortex line.  % and interpreted in the context of initiation of wave turbulence in a BEC.[CheckBlueBook]
% The results are useful in informing numerical procedures known as vortex filament methods~[\onlinecite{Hanninen2014VortexTurbulence.}]  but also provide a description of vorticity as a wave supporting dispersive medium.~[\onlinecite{Salman2013BreathersVortices}, \onlinecite{VanGorder2017MotionApproximation}]
% %[BLUE BOOK STATEMENT AND CITATION] 

% \begin{outline}
%  \1 In this paper we report on the non-Hamiltonian corrections to the local induction approximation and present a workflow that begins at Biot-Savart and ends with  simulation of vortices in Eulerian flows.
%  \2 We present an analysis of BSI that yields systematic corrections to LIA that can be incorporated into vortex filament methods to break integrability, which a known limitation of LIA, and allow for KW generation without the use of continuum models or full BSI integration.
%   \2 We derive an effective PDE that evolves the FS apparatus according to a non-Hamiltonian flow, which enhances propagation speed of waves, and their nonlinear dispersion, in vortex media which are balanced against an emergent gain/loss mechanism.   
%   \2 We simulate a variety of vortex states found in previous literature to confirm the predictions stemming from our non-Hamiltonian flow. 
%    \3 For vortex rings, note that it would be interesting to re-evaluate the work of Tsubota and Barenghi under correction. 
%    \3 Also for rings, it would be interesting to correlate the bump amplitude to ring radius and spectrum. 
%    \2 A simple test of this could be in ultra small ring experimentation.
% \1 Discussion of QT and the relationship with Onsager's conjecture and how there is a need for relaxation of bending energy.
% \1 Hypothesize that with our initial inspection of vortex rings, specifically Kelvin rings, that there may be a preference for vortex lines to hold their geometric energy in the form of a helix. This seems reasonable in light of the solid mechanics of biological structures. 
% \end{outline}



% \begin{outline}

%  \1 Simulations
%   \2 Various settings of Hasimoto soliton
%   \2 Next term correction
%   \2 Three corrections at same time side by side. 
%   \2 Kelvin wave perturbed rings
%   \2 Pergrine breather
%   \2 Power spectrum analysis?
% \end{outline}

The authors would like to acknowledge Willy Hereman, Igor Khavkine, Randy LeVeque, Stephen Pankavich,  Barbara Prinari and David Sommer for useful discussions.  The authors acknowledge support from the US National Science Foundation under grant numbers PHY-1306638, PHY-1520915, OAC-1740130, and the US Air Force Office of Scientific Research grant number FA9550-14-1-0287. This work was performed in part at the Aspen Center for Physics, which is supported by the US National Science Foundation grant PHY-1607611. 

%\bibliographystyle{apsrev4-1}

%\bibliographystyle{abbrv}
%\bibliographystyle{short,cvpubs}
%\bibliography{Paper3}
%merlin.mbs apsrev4-1.bst 2010-07-25 4.21a (PWD, AO, DPC) hacked
%Control: key (0)
%Control: author (72) initials jnrlst
%Control: editor formatted (1) identically to author
%Control: production of article title (-1) disabled
%Control: page (0) single
%Control: year (1) truncated
%Control: production of eprint (0) enabled
\begin{thebibliography}{69}%
\makeatletter
\providecommand \@ifxundefined [1]{%
 \@ifx{#1\undefined}
}%
\providecommand \@ifnum [1]{%
 \ifnum #1\expandafter \@firstoftwo
 \else \expandafter \@secondoftwo
 \fi
}%
\providecommand \@ifx [1]{%
 \ifx #1\expandafter \@firstoftwo
 \else \expandafter \@secondoftwo
 \fi
}%
\providecommand \natexlab [1]{#1}%
\providecommand \enquote  [1]{#1}%
\providecommand \bibnamefont  [1]{#1}%
\providecommand \bibfnamefont [1]{#1}%
\providecommand \citenamefont [1]{#1}%
\providecommand \href@noop [0]{\@secondoftwo}%
\providecommand \href [0]{\begingroup \@sanitize@url \@href}%
\providecommand \@href[1]{\@@startlink{#1}\@@href}%
\providecommand \@@href[1]{\endgroup#1\@@endlink}%
\providecommand \@sanitize@url [0]{\catcode `\\12\catcode `\$12\catcode
  `\&12\catcode `\#12\catcode `\^12\catcode `\_12\catcode `\%12\relax}%
\providecommand \@@startlink[1]{}%
\providecommand \@@endlink[0]{}%
\providecommand \url  [0]{\begingroup\@sanitize@url \@url }%
\providecommand \@url [1]{\endgroup\@href {#1}{\urlprefix }}%
\providecommand \urlprefix  [0]{URL }%
\providecommand \Eprint [0]{\href }%
\providecommand \doibase [0]{http://dx.doi.org/}%
\providecommand \selectlanguage [0]{\@gobble}%
\providecommand \bibinfo  [0]{\@secondoftwo}%
\providecommand \bibfield  [0]{\@secondoftwo}%
\providecommand \translation [1]{[#1]}%
\providecommand \BibitemOpen [0]{}%
\providecommand \bibitemStop [0]{}%
\providecommand \bibitemNoStop [0]{.\EOS\space}%
\providecommand \EOS [0]{\spacefactor3000\relax}%
\providecommand \BibitemShut  [1]{\csname bibitem#1\endcsname}%
\let\auto@bib@innerbib\@empty
%</preamble>
\bibitem [{\citenamefont {Andersen}\ and\ \citenamefont
  {Lim}(2014)}]{Andersen2014IntroductionEquilibrium}%
  \BibitemOpen
  \bibfield  {author} {\bibinfo {author} {\bibfnamefont {T.~D.}\ \bibnamefont
  {Andersen}}\ and\ \bibinfo {author} {\bibfnamefont {C.~C.}\ \bibnamefont
  {Lim}},\ }\href {http://www.springer.com/us/book/9781493919376} {\emph
  {\bibinfo {title} {{Introduction to Vortex Filaments in Equilibrium}}}},\
   (\bibinfo  {publisher} {Springer-Verlag, New
  York},\ \bibinfo {year} {2014})\BibitemShut {NoStop}%
\bibitem [{\citenamefont {Mesirov}\ \emph {et~al.}(2012)\citenamefont
  {Mesirov}, \citenamefont {Schulten},\ and\ \citenamefont
  {Sumners}}]{Mesirov2012MathematicalDynamics}%
  \BibitemOpen
  \bibfield  {author} {\bibinfo {author} {\bibfnamefont {J.~P.}\ \bibnamefont
  {Mesirov}}, \bibinfo {author} {\bibfnamefont {K.}~\bibnamefont {Schulten}}, \
  and\ \bibinfo {author} {\bibfnamefont {D.~W.}\ \bibnamefont {Sumners}},\
  }\href {http://www.springer.com/us/book/9780387948386} {\emph {\bibinfo
  {title} {{Mathematical Approaches to Biomolecular Structure and
  Dynamics}}}},\ (\bibinfo
   {publisher} {Springer-Verlag, New York},\ \bibinfo {year} {2012})\BibitemShut
  {NoStop}%
\bibitem [{\citenamefont {Scott}\ and\ \citenamefont
  {Scott}(2006)}]{Scott2006EncyclopediaScience}%
  \BibitemOpen
  \bibfield  {author} {\bibinfo {author} {\bibfnamefont {A.}~\bibnamefont
  {Scott}}\ and\ \bibinfo {author} {\bibfnamefont {E.}~\bibnamefont {Scott}},\
  }\href {https://www.routledge.com/Encyclopedia-of-Nonlinear-Science/Scott/p/book/9781579583859} {\emph {\bibinfo
  {title} {{Encyclopedia of Nonlinear Science}}}}\ (\bibinfo  {publisher}
  {Routledge, New York},\ \bibinfo {year} {2006})\BibitemShut {NoStop}%
\bibitem [{\citenamefont {Shi}\ and\ \citenamefont
  {Hearst}(1999)}]{Shi1999TheSupercoiling}%
  \BibitemOpen
  \bibfield  {author} {\bibinfo {author} {\bibfnamefont {Y.}~\bibnamefont
  {Shi}}\ and\ \bibinfo {author} {\bibfnamefont {J.~E.}\ \bibnamefont
  {Hearst}}\, }  \bibfield  {title} {\emph {\enquote {\bibinfo {title} {{The Kirchhoff elastic rod, the nonlinear Schr\"odinger equation, and DNA supercoiling}},}\ }}\href {\doibase
  10.1063/1.468506} {\bibfield  {journal} {\bibinfo  {journal} {J. Chem. Phys.
   }\ }\textbf {\bibinfo {volume} {101}} (\bibinfo
  {year} {1999})}\BibitemShut {NoStop}%
\bibitem [{\citenamefont {Lindemann}\ \emph {et~al.}(2017)\citenamefont
  {Lindemann}, \citenamefont {Visser},\ and\ \citenamefont
  {Mariani}}]{Lindemann2017DynamicsCells}%
  \BibitemOpen
  \bibfield  {author} {\bibinfo {author} {\bibfnamefont {C.}~\bibnamefont
  {Lindemann}}, \bibinfo {author} {\bibfnamefont {A.}~\bibnamefont {Visser}}, \
  and\ \bibinfo {author} {\bibfnamefont {P.}~\bibnamefont {Mariani}},\
  }\bibfield  {title} {\emph {\enquote {\bibinfo {title} {{Dynamics of
  phytoplankton blooms in turbulent vortex cells}},}\ }}\href {\doibase
  10.1098/rsif.2017.0453} {\bibfield  {journal} {\bibinfo  {journal} {J. R. Soc. Interface
   }\ }\textbf {\bibinfo {volume} {14}} (\bibinfo
  {year} {2017})}\BibitemShut {NoStop}%
\bibitem [{\citenamefont {Durham}\ \emph {et~al.}(2013)\citenamefont {Durham},
  \citenamefont {Climent}, \citenamefont {Barry}, \citenamefont {De~Lillo},
  \citenamefont {Boffetta}, \citenamefont {Cencini},\ and\ \citenamefont
  {Stocker}}]{Durham2013TurbulencePhytoplankton}%
  \BibitemOpen
  \bibfield  {author} {\bibinfo {author} {\bibfnamefont {W.~M.}\ \bibnamefont
  {Durham}}, \bibinfo {author} {\bibfnamefont {E.}~\bibnamefont {Climent}},
  \bibinfo {author} {\bibfnamefont {M.}~\bibnamefont {Barry}}, \bibinfo
  {author} {\bibfnamefont {F.}~\bibnamefont {De~Lillo}}, \bibinfo {author}
  {\bibfnamefont {G.}~\bibnamefont {Boffetta}}, \bibinfo {author}
  {\bibfnamefont {M.}~\bibnamefont {Cencini}}, \ and\ \bibinfo {author}
  {\bibfnamefont {R.}~\bibnamefont {Stocker}},\ }\bibfield  {title} {\emph
  {\enquote {\bibinfo {title} {{Turbulence drives microscale patches of motile
  phytoplankton}},}\ }}\href {http://dx.doi.org/10.1038/ncomms3148
  10.1038/ncomms3148
  https://www.nature.com/articles/ncomms3148#supplementary-information}
  {\bibfield  {journal} {\bibinfo  {journal} {Nature Commun.}\ }\textbf
  {\bibinfo {volume} {4}},\ \bibinfo {pages} {2148} (\bibinfo {year}
  {2013})}\BibitemShut {NoStop}%
\bibitem [{\citenamefont {Leadbeater}\ \emph {et~al.}(2001)\citenamefont
  {Leadbeater}, \citenamefont {Winiecki}, \citenamefont {Samuels},
  \citenamefont {Barenghi},\ and\ \citenamefont
  {Adams}}]{Leadbeater2001SoundReconnections}%
  \BibitemOpen
  \bibfield  {author} {\bibinfo {author} {\bibfnamefont {M.}~\bibnamefont
  {Leadbeater}}, \bibinfo {author} {\bibfnamefont {T.}~\bibnamefont
  {Winiecki}}, \bibinfo {author} {\bibfnamefont {D.~C.}\ \bibnamefont
  {Samuels}}, \bibinfo {author} {\bibfnamefont {C.~F.}\ \bibnamefont
  {Barenghi}}, \ and\ \bibinfo {author} {\bibfnamefont {C.~S.}\ \bibnamefont
  {Adams}},\ }\bibfield  {title} {\emph {\enquote {\bibinfo {title} {{Sound
  Emission due to Superfluid Vortex Reconnections}},}\ }}\href {\doibase
  10.1103/PhysRevLett.86.1410} {\bibfield  {journal} {\bibinfo  {journal}
  {Phys. Rev. Lett.}\ }\textbf {\bibinfo {volume} {86}},\ \bibinfo {pages}
  {1410} (\bibinfo {year} {2001})}\BibitemShut {NoStop}%
\bibitem [{\citenamefont {Van~Gorder}(2015)}]{VanGorder2015TheFluid}%
  \BibitemOpen
  \bibfield  {author} {\bibinfo {author} {\bibfnamefont {R.~A.}\ \bibnamefont
  {Van~Gorder}},\ }\bibfield  {title} {\emph {\enquote {\bibinfo {title} {{The
  Biot–Savart description of Kelvin waves on a quantum vortex filament in the
  presence of mutual friction and a driving fluid}},}\ }}\href
  {http://rspa.royalsocietypublishing.org/content/471/2179/20150149.abstract}
  {\bibfield  {journal} {\bibinfo  {journal} {Proc. R. Soc. Ser. A. }\ }\textbf {\bibinfo
  {volume} {471}} (\bibinfo {year} {2015})}\BibitemShut {NoStop}%
\bibitem [{\citenamefont {Majda}\ and\ \citenamefont
  {Bertozzi}(2002)}]{Majda2002VorticityFlow}%
  \BibitemOpen
  \bibfield  {author} {\bibinfo {author} {\bibfnamefont {A.~J.}\ \bibnamefont
  {Majda}}\ and\ \bibinfo {author} {\bibfnamefont {A.~L.}\ \bibnamefont
  {Bertozzi}},\ }\href {http://www.cambridge.org/gb/academic/subjects/mathematics/fluid-dynamics-and-solid-mechanics/vorticity-and-incompressible-flow} {\emph
  {\bibinfo {title} {{Vorticity and Incompressible Flow}}}},\  (\bibinfo  {publisher} {Cambridge University Press, Cambridge},\
  \bibinfo {year} {2002})\BibitemShut {NoStop}%
\bibitem [{\citenamefont {Klein}\ and\ \citenamefont
  {Majda}(1991{\natexlab{a}})}]{Klein1991Self-stretchingLine}%
  \BibitemOpen
  \bibfield  {author} {\bibinfo {author} {\bibfnamefont {R.}~\bibnamefont
  {Klein}}\ and\ \bibinfo {author} {\bibfnamefont {A.~J.}\ \bibnamefont
  {Majda}},\ }\bibfield  {title} {\emph {\enquote {\bibinfo {title}
  {{Self-stretching of a perturbed vortex filament I. The asymptotic equation
  for deviations from a straight line}},}\ }}\href {\doibase
  10.1016/0167-2789(91)90151-X} {\bibfield  {journal} {\bibinfo  {journal}
  {Physica D}\ }\textbf {\bibinfo {volume} {49}},\
  \bibinfo {pages} {323} (\bibinfo {year} {1991}{\natexlab{a}})}\BibitemShut
  {NoStop}%
\bibitem [{\citenamefont {Klein}\ and\ \citenamefont
  {Majda}(1991{\natexlab{b}})}]{Klein1991Self-stretchingSolutions}%
  \BibitemOpen
  \bibfield  {author} {\bibinfo {author} {\bibfnamefont {R.}~\bibnamefont
  {Klein}}\ and\ \bibinfo {author} {\bibfnamefont {A.~J.}\ \bibnamefont
  {Majda}},\ }\bibfield  {title} {\emph {\enquote {\bibinfo {title}
  {{Self-stretching of perturbed vortex filaments: II. Structure of
  solutions}},}\ }}\href {\doibase 10.1016/0167-2789(91)90066-I} {\bibfield
  {journal} {\bibinfo  {journal} {Physica D}\ }\textbf
  {\bibinfo {volume} {53}},\ \bibinfo {pages} {267} (\bibinfo {year}
  {1991}{\natexlab{b}})}\BibitemShut {NoStop}%
\bibitem [{\citenamefont
  {Van~Gorder}(2017)}]{VanGorder2017MotionApproximation}%
  \BibitemOpen
  \bibfield  {author} {\bibinfo {author} {\bibfnamefont {R.~A.}\ \bibnamefont
  {Van~Gorder}},\ }\bibfield  {title} {\emph {\enquote {\bibinfo {title}
  {{Motion of isolated open vortex filaments evolving under the truncated local
  induction approximation}},}\ }}\href {\doibase 10.1063/1.5005113} {\bibfield
  {journal} {\bibinfo  {journal} {Phys. Fluids}\ }\textbf {\bibinfo
  {volume} {29}},\ \bibinfo {pages} {115105} (\bibinfo {year}
  {2017})}\BibitemShut {NoStop}%
\bibitem [{\citenamefont {Laurie}\ \emph {et~al.}(2010)\citenamefont {Laurie},
  \citenamefont {L'vov}, \citenamefont {Nazarenko},\ and\ \citenamefont
  {Rudenko}}]{Laurie2010InteractionSuperfluids}%
  \BibitemOpen
  \bibfield  {author} {\bibinfo {author} {\bibfnamefont {J.}~\bibnamefont
  {Laurie}}, \bibinfo {author} {\bibfnamefont {V.~S.}\ \bibnamefont {L'vov}},
  \bibinfo {author} {\bibfnamefont {S.}~\bibnamefont {Nazarenko}}, \ and\
  \bibinfo {author} {\bibfnamefont {O.}~\bibnamefont {Rudenko}},\ }\bibfield
  {title} {\emph {\enquote {\bibinfo {title} {{Interaction of Kelvin waves and
  nonlocality of energy transfer in superfluids}},}\ }}\href {\doibase
  10.1103/PhysRevB.81.104526} {\bibfield  {journal} {\bibinfo  {journal} {Phys.
  Rev. B}\ }\textbf {\bibinfo {volume} {81}},\ \bibinfo {pages} {104526}
  (\bibinfo {year} {2010})}\BibitemShut {NoStop}%
\bibitem [{\citenamefont {Boffetta}\ \emph {et~al.}(2009)\citenamefont
  {Boffetta}, \citenamefont {Celani}, \citenamefont {Dezzani}, \citenamefont
  {Laurie},\ and\ \citenamefont {Nazarenko}}]{Boffetta2009ModelingHelium}%
  \BibitemOpen
  \bibfield  {author} {\bibinfo {author} {\bibfnamefont {G.}~\bibnamefont
  {Boffetta}}, \bibinfo {author} {\bibfnamefont {A.}~\bibnamefont {Celani}},
  \bibinfo {author} {\bibfnamefont {D.}~\bibnamefont {Dezzani}}, \bibinfo
  {author} {\bibfnamefont {J.}~\bibnamefont {Laurie}}, \ and\ \bibinfo {author}
  {\bibfnamefont {S.}~\bibnamefont {Nazarenko}},\ }\bibfield  {title} {\emph
  {\enquote {\bibinfo {title} {{Modeling Kelvin Wave Cascades in Superfluid
  Helium}},}\ }}\href {\doibase 10.1007/s10909-009-9895-x} {\bibfield
  {journal} {\bibinfo  {journal} {J. Low Temp. Phys.}\ }\textbf
  {\bibinfo {volume} {156}},\ \bibinfo {pages} {193} (\bibinfo {year}
  {2009})}\BibitemShut {NoStop}%
\bibitem [{\citenamefont {White}\ \emph {et~al.}(2014)\citenamefont {White},
  \citenamefont {Anderson},\ and\ \citenamefont
  {Bagnato}}]{White2014VorticesCondensates.}%
  \BibitemOpen
  \bibfield  {author} {\bibinfo {author} {\bibfnamefont {A.~C.}\ \bibnamefont
  {White}}, \bibinfo {author} {\bibfnamefont {B.~P.}\ \bibnamefont {Anderson}},
  \ and\ \bibinfo {author} {\bibfnamefont {V.~S.}\ \bibnamefont {Bagnato}},\
  }\bibfield  {title} {\emph {\enquote {\bibinfo {title} {{Vortices and
  turbulence in trapped atomic condensates.}}}\ }}\href {\doibase
  10.1073/pnas.1312737110} {\bibfield  {journal} {\bibinfo  {journal}
  {Proc. Natl. Acad. Sci. USA}\ }\textbf {\bibinfo {volume} {111 Suppl 1}},\ \bibinfo {pages}
  {4719} (\bibinfo {year} {2014})}\BibitemShut {NoStop}%
\bibitem [{\citenamefont {Kozik}\ and\ \citenamefont
  {Svistunov}(2010)}]{Kozik2010CommentAndLvov}%
  \BibitemOpen
  \bibfield  {author} {\bibinfo {author} {\bibfnamefont {E.~V.}\ \bibnamefont
  {Kozik}}\ and\ \bibinfo {author} {\bibfnamefont {B.~V.}\ \bibnamefont
  {Svistunov}},\ }\bibfield  {title} {\emph {\enquote {\bibinfo {title}
  {{Comment on ``Symmetries and Interaction Coefficients of Kelvin waves''
  by Lebedev and L'vov}},}\ }}\href {\doibase 10.1007/s10909-010-0242-z}
  {\bibfield  {journal} {\bibinfo  {journal} {J. Low Temp. Phys.}\ }\textbf {\bibinfo {volume} {161}},\ \bibinfo {pages} {603}
  (\bibinfo {year} {2010})}\BibitemShut {NoStop}%
\bibitem [{\citenamefont {Lebedev}\ \emph {et~al.}(2010)\citenamefont
  {Lebedev}, \citenamefont {L'vov},\ and\ \citenamefont
  {Nazarenko}}]{Lebedev2010Reply:Turbulence}%
  \BibitemOpen
  \bibfield  {author} {\bibinfo {author} {\bibfnamefont {V.~V.}\ \bibnamefont
  {Lebedev}}, \bibinfo {author} {\bibfnamefont {V.~S.}\ \bibnamefont {L'vov}},
  \ and\ \bibinfo {author} {\bibfnamefont {S.~V.}\ \bibnamefont {Nazarenko}},\
  }\bibfield  {title} {\emph {\enquote {\bibinfo {title} {{Reply: On Role of
  Symmetries in Kelvin Wave Turbulence}},}\ }}\href {\doibase
  10.1007/s10909-010-0240-1} {\bibfield  {journal} {\bibinfo  {journal}
  {J. Low Temp. Phys.}\ }\textbf {\bibinfo {volume} {161}},\
  \bibinfo {pages} {606} (\bibinfo {year} {2010})}\BibitemShut {NoStop}%
\bibitem [{\citenamefont {Fonda}\ \emph {et~al.}(2014)\citenamefont {Fonda},
  \citenamefont {Meichle}, \citenamefont {Ouellette}, \citenamefont {Hormoz},\
  and\ \citenamefont {Lathrop}}]{Fonda2014DirectReconnection.}%
  \BibitemOpen
  \bibfield  {author} {\bibinfo {author} {\bibfnamefont {E.}~\bibnamefont
  {Fonda}}, \bibinfo {author} {\bibfnamefont {D.~P.}\ \bibnamefont {Meichle}},
  \bibinfo {author} {\bibfnamefont {N.~T.}\ \bibnamefont {Ouellette}}, \bibinfo
  {author} {\bibfnamefont {S.}~\bibnamefont {Hormoz}}, \ and\ \bibinfo {author}
  {\bibfnamefont {D.~P.}\ \bibnamefont {Lathrop}},\ }\bibfield  {title} {\emph
  {\enquote {\bibinfo {title} {{Direct observation of Kelvin waves excited by
  quantized vortex reconnection.}}}\ }}\href {\doibase 10.1073/pnas.1312536110}
  {\bibfield  {journal} {\bibinfo  {journal} {Proc. Natl.  Acad.  Sci. USA}\ }\textbf {\bibinfo
  {volume} {111 Suppl 1}},\ \bibinfo {pages} {4707} (\bibinfo {year}
  {2014})}\BibitemShut {NoStop}%
\bibitem [{\citenamefont {Lax}(2007)}]{Lax2007LinearApplications}%
  \BibitemOpen
  \bibfield  {author} {\bibinfo {author} {\bibfnamefont {P.~D.}\ \bibnamefont
  {Lax}},\ }\href {https://www.wiley.com/en-us/Linear+Algebra+and+Its+Applications%2C+2nd+Edition-p-9780471751564} {\emph
  {\bibinfo {title} {{Linear Algebra and Its Applications}}}},\  (\bibinfo  {publisher} {Wiley, New Jersey},\ \bibinfo {year} {2007})\BibitemShut
  {NoStop}%
\bibitem [{\citenamefont {Saffman}(1993)}]{Saffman1993VortexDynamics}%
  \BibitemOpen
  \bibfield  {author} {\bibinfo {author} {\bibfnamefont {P.~G.}\ \bibnamefont
  {Saffman}},\ }\href {\doibase 10.1017/CBO9780511624063} {\emph {\bibinfo
  {title} {{Vortex Dynamics}}}},\ (\bibinfo
  {publisher} {Cambridge University Press, Cambridge},\ \bibinfo {year}
  {1993})\BibitemShut {NoStop}%
\bibitem [{\citenamefont {Berselli}\ and\ \citenamefont
  {Bessaih}(2002)}]{Berselli2002SomeEquation}%
  \BibitemOpen
  \bibfield  {author} {\bibinfo {author} {\bibfnamefont {L.~C.}\ \bibnamefont
  {Berselli}}\ and\ \bibinfo {author} {\bibfnamefont {H.}~\bibnamefont
  {Bessaih}},\ }\bibfield  {title} {\emph {\enquote {\bibinfo {title} {{Some
  results for the line vortex equation}},}\ }}\href {\doibase
  10.1088/0951-7715/15/6/301} {\bibfield  {journal} {\bibinfo  {journal}
  {Nonlinearity}\ }\textbf {\bibinfo {volume} {15}},\ \bibinfo {pages} {1729}
  (\bibinfo {year} {2002})}\BibitemShut {NoStop}%
\bibitem [{\citenamefont {Callegari}\ and\ \citenamefont
  {Ting}(1978)}]{Callegari1978MotionVelocity}%
  \BibitemOpen
  \bibfield  {author} {\bibinfo {author} {\bibfnamefont {A.~J.}\ \bibnamefont
  {Callegari}}\ and\ \bibinfo {author} {\bibfnamefont {L.}~\bibnamefont
  {Ting}},\ }\bibfield  {title} {\emph {\enquote {\bibinfo {title} {{Motion of
  a Curved Vortex Filament with Decaying Vortical Core and Axial Velocity}},}\
  }}\href {http://www.jstor.org/stable/2101038} {\bibfield  {journal} {\bibinfo
   {journal} {SIAM J. Appl. Math.}\ }\textbf {\bibinfo {volume}
  {35}},\ \bibinfo {pages} {148} (\bibinfo {year} {1978})}\BibitemShut
  {NoStop}%
\bibitem [{\citenamefont {Parsley}(2012)}]{Parsley2012TheThree-sphere}%
  \BibitemOpen
  \bibfield  {author} {\bibinfo {author} {\bibfnamefont {J.}~\bibnamefont
  {Parsley}},\ }\bibfield  {title} {\emph {\enquote {\bibinfo {title} {{The
  Biot-Savart operator and electrodynamics on subdomains of the
  three-sphere}},}\ }}\href {\doibase 10.1063/1.3673788} {\bibfield  {journal}
  {\bibinfo  {journal} {J. Math Phys.}\ }\textbf {\bibinfo
  {volume} {53}},\ \bibinfo {pages} {013102} (\bibinfo {year}
  {2012})}\BibitemShut {NoStop}%
\bibitem [{\citenamefont {DeTurck}\ and\ \citenamefont
  {Gluck}(2008)}]{DeTurck2008Electrodynamics3-space}%
  \BibitemOpen
  \bibfield  {author} {\bibinfo {author} {\bibfnamefont {D.}~\bibnamefont
  {DeTurck}}\ and\ \bibinfo {author} {\bibfnamefont {H.}~\bibnamefont
  {Gluck}},\ }\bibfield  {title} {\emph {\enquote {\bibinfo {title}
  {{Electrodynamics and the Gauss linking integral on the 3-sphere and in
  hyperbolic 3-space}},}\ }}\href {\doibase 10.1063/1.2827467} {\bibfield
  {journal} {\bibinfo  {journal} {J. Math Phys.}\ }\textbf
  {\bibinfo {volume} {49}},\ \bibinfo {pages} {023504} (\bibinfo {year}
  {2008})}\BibitemShut {NoStop}%
\bibitem [{\citenamefont {Batchelor}(2000)}]{Batchelor2000AnDynamics}%
  \BibitemOpen
  \bibfield  {author} {\bibinfo {author} {\bibfnamefont {G.~K.}\ \bibnamefont
  {Batchelor}},\ }\href {\doibase 10.1017/CBO9780511800955} {\emph {\bibinfo
  {title} {{An Introduction to Fluid Dynamics}}}}\ (\bibinfo  {publisher}
  {Cambridge University Press},\ \bibinfo {address} {Cambridge},\ \bibinfo
  {year} {2000})\BibitemShut {NoStop}%
\bibitem [{\citenamefont {Khesin}(2012)}]{Khesin2012SymplecticMembranes}%
  \BibitemOpen
  \bibfield  {author} {\bibinfo {author} {\bibfnamefont {B.}~\bibnamefont
  {Khesin}},\ }\bibfield  {title} {\emph {\enquote {\bibinfo {title}
  {{Symplectic structures and dynamics on vortex membranes}},}\ }}\href {http://www.ams.org/distribution/mmj/vol12-2-2012/abst12-2-2012.html#khesin}
  {\bibfield  {journal} {\bibinfo  {journal} {Mosc. Math. J.}\ }\textbf
  {\bibinfo {volume} {12}},\ \bibinfo {pages} {413} (\bibinfo {year}
  {2012})}\BibitemShut {NoStop}%
\bibitem [{\citenamefont {Burns}\ \emph {et~al.}(1991)\citenamefont {Burns},
  \citenamefont {Dubrovin}, \citenamefont {Fomenko},\ and\ \citenamefont
  {Novikov}}]{Burns1991ModernFields}%
  \BibitemOpen
  \bibfield  {author} {\bibinfo {author} {\bibfnamefont {R.~G.}\ \bibnamefont
  {Burns}}, \bibinfo {author} {\bibfnamefont {B.~A.}\ \bibnamefont {Dubrovin}},
  \bibinfo {author} {\bibfnamefont {A.~T.}\ \bibnamefont {Fomenko}}, \ and\
  \bibinfo {author} {\bibfnamefont {S.~P.}\ \bibnamefont {Novikov}},\ }\href
  {https://www.springer.com/la/book/9781468499469} {\emph {\bibinfo {title}
  {{Modern Geometry — Methods and Applications: Part I: The Geometry of
  Surfaces, Transformation Groups, and Fields}}}},\  (\bibinfo  {publisher} {Springer-Verlag, New York},\ \bibinfo {year}
  {1991})\BibitemShut {NoStop}%
\bibitem [{\citenamefont
  {Schwarz}(1985)}]{Schwarz1985Three-dimensionalInteractions}%
  \BibitemOpen
  \bibfield  {author} {\bibinfo {author} {\bibfnamefont {K.~W.}\ \bibnamefont
  {Schwarz}},\ }\bibfield  {title} {\emph {\enquote {\bibinfo {title}
  {{Three-dimensional vortex dynamics in superfluid {\^{}}4He: Line-line and
  line-boundary interactions}},}\ }}\href {\doibase 10.1103/PhysRevB.31.5782}
  {\bibfield  {journal} {\bibinfo  {journal} {Phys. Rev. B}\ }\textbf {\bibinfo
  {volume} {31}},\ \bibinfo {pages} {5782} (\bibinfo {year}
  {1985})}\BibitemShut {NoStop}%
\bibitem [{\citenamefont {Bustamante}\ and\ \citenamefont
  {Nazarenko}(2015)}]{Bustamante2015DerivationEquation}%
  \BibitemOpen
  \bibfield  {author} {\bibinfo {author} {\bibfnamefont {M.~D.}\ \bibnamefont
  {Bustamante}}\ and\ \bibinfo {author} {\bibfnamefont {S.}~\bibnamefont
  {Nazarenko}},\ }\bibfield  {title} {\emph {\enquote {\bibinfo {title}
  {{Derivation of the Biot-Savart equation from the nonlinear Schr{\"{o}}dinger
  equation}},}\ }}\href {\doibase 10.1103/PhysRevE.92.053019} {\bibfield
  {journal} {\bibinfo  {journal} {Phys. Rev. E}\ }\textbf {\bibinfo {volume}
  {92}},\ \bibinfo {pages} {53019} (\bibinfo {year} {2015})}\BibitemShut
  {NoStop}%
\bibitem [{\citenamefont {Strong}\ and\ \citenamefont
  {Carr}(2012)}]{Strong2012GeneralizedTurbulence}%
  \BibitemOpen
  \bibfield  {author} {\bibinfo {author} {\bibfnamefont {S.~A.}\ \bibnamefont
  {Strong}}\ and\ \bibinfo {author} {\bibfnamefont {L.~D.}\ \bibnamefont
  {Carr}},\ }\bibfield  {title} {\emph {\enquote {\bibinfo {title}
  {{Generalized local induction equation, elliptic asymptotics, and simulating
  superfluid turbulence}},}\ }}\href {\doibase 10.1063/1.3696689} {\bibfield
  {journal} {\bibinfo  {journal} {J. Math Phys.}\ }\textbf
  {\bibinfo {volume} {53}} (\bibinfo {year} {2012}),\
  10.1063/1.3696689}\BibitemShut {NoStop}%
\bibitem [{\citenamefont {{Toshio
  Fukushima}}(2012)}]{ToshioFukushima2012SeriesIntegrals}%
  \BibitemOpen
  \bibfield  {author} {\bibinfo {author} {\bibnamefont {{T. Fukushima}}},\
  }\bibfield  {title} {\emph {\enquote {\bibinfo {title} {{Series expansions of
  symmetric elliptic integrals}},}\ }}\href {\doibase
  10.1090/S0025-5718-2011-02531-7} {\bibfield  {journal} {\bibinfo  {journal}
  {Math. Comput.}\ }\textbf {\bibinfo {volume} {81}},\ \bibinfo
  {pages} {957} (\bibinfo {year} {2012})}\BibitemShut {NoStop}%
\bibitem [{\citenamefont {Karp}\ and\ \citenamefont
  {Sitnik}(2007)}]{Karp2007AsymptoticSingularity}%
  \BibitemOpen
  \bibfield  {author} {\bibinfo {author} {\bibfnamefont {D.}~\bibnamefont
  {Karp}}\ and\ \bibinfo {author} {\bibfnamefont {S.}~\bibnamefont {Sitnik}},\
  }\bibfield  {title} {\emph {\enquote {\bibinfo {title} {{Asymptotic
  approximations for the first incomplete elliptic integral near logarithmic
  singularity}},}\ }}\href {\doibase 10.1016/j.cam.2006.04.053} {\bibfield
  {journal} {\bibinfo  {journal} {J. Comput. Appl.
  Math.}\ }\textbf {\bibinfo {volume} {205}},\ \bibinfo {pages} {186}
  (\bibinfo {year} {2007})}\BibitemShut {NoStop}%
\bibitem [{\citenamefont {Carlson}\ and\ \citenamefont
  {Gustafsoni}(1994)}]{Carlson1994}%
  \BibitemOpen
  \bibfield  {author} {\bibinfo {author} {\bibfnamefont {B.~C.}\ \bibnamefont
  {Carlson}}\ and\ \bibinfo {author} {\bibfnamefont {J.~L.}\ \bibnamefont
  {Gustafsoni}},\ }\bibfield  {title} {\emph {\enquote {\bibinfo {title}
  {{Asymptotic approximations for symmetric elliptic integrals}},}\
  }}\href@noop {} {\bibfield  {journal} {\bibinfo  {journal} {SIAM J. Math.
  Anal.}\ }\textbf {\bibinfo {volume} {25}},\ \bibinfo {pages} {288} (\bibinfo
  {year} {1994})}\BibitemShut {NoStop}%
\bibitem [{\citenamefont {Moore}\ and\ \citenamefont
  {Saffman}(1972)}]{Moore1972TheFlow}%
  \BibitemOpen
  \bibfield  {author} {\bibinfo {author} {\bibfnamefont {D.~W.}\ \bibnamefont
  {Moore}}\ and\ \bibinfo {author} {\bibfnamefont {P.~G.}\ \bibnamefont
  {Saffman}},\ }\bibfield  {title} {\emph {\enquote {\bibinfo {title} {{The
  Motion of a Vortex Filament with Axial Flow}},}\ }}\href
  {http://rsta.royalsocietypublishing.org/content/272/1226/403.abstract}
  {\bibfield  {journal} {\bibinfo  {journal} {Phil. Tran. 
  R. Soc. London, Ser. A}\
  }\textbf {\bibinfo {volume} {272}},\ \bibinfo {pages} {403} (\bibinfo {year}
  {1972})}\BibitemShut {NoStop}%
\bibitem [{\citenamefont {Svidzinsky}\ and\ \citenamefont
  {Fetter}(2000)}]{Svidzinsky2000DynamicsCondensate}%
  \BibitemOpen
  \bibfield  {author} {\bibinfo {author} {\bibfnamefont {A.~A.}\ \bibnamefont
  {Svidzinsky}}\ and\ \bibinfo {author} {\bibfnamefont {A.~L.}\ \bibnamefont
  {Fetter}},\ }\bibfield  {title} {\emph {\enquote {\bibinfo {title} {{Dynamics
  of a vortex in a trapped Bose-Einstein condensate}},}\ }}\href {\doibase
  10.1103/PhysRevA.62.063617}  {\bibfield  {journal} {\bibinfo  {journal} {Phys. Rev. A}\
  }\textbf {\bibinfo {volume} {62}},\ \bibinfo {pages} {063617}\ (\bibinfo {year}
  {2000})}\BibitemShut {NoStop}%
\bibitem [{\citenamefont {Hinch}(1991)}]{Hinch1991PerturbationMethods}%
  \BibitemOpen
  \bibfield  {author} {\bibinfo {author} {\bibfnamefont {E.~J.}\ \bibnamefont
  {Hinch}},\ }\href {https://books.google.com/books?id=mqY4ZM0BWwIC} {\emph
  {\bibinfo {title} {{Perturbation Methods}}}},\  (\bibinfo  {publisher} {Cambridge University Press, Cambridge},\ \bibinfo
  {year} {1991})\BibitemShut {NoStop}%
\bibitem [{\citenamefont {Georgescu}(1995)}]{Georgescu1995AsymptoticEquations}%
  \BibitemOpen
  \bibfield  {author} {\bibinfo {author} {\bibfnamefont {A.}~\bibnamefont
  {Georgescu}},\ }\href {https://www.crcpress.com/Asymptotic-Treatment-of-Differential-Equations/Georgescu/p/book/9780412558603} {\emph
  {\bibinfo {title} {{Asymptotic Treatment of Differential Equations}}}},\
   (\bibinfo  {publisher} {Chapman and Hall/CRC, Boca Raton},\ \bibinfo
  {year} {1995})\BibitemShut {NoStop}%
\bibitem [{\citenamefont
  {K{\"{u}}hnel}(2006)}]{Kuhnel2006DifferentialManifolds}%
  \BibitemOpen
  \bibfield  {author} {\bibinfo {author} {\bibfnamefont {W.}~\bibnamefont
  {K{\"{u}}hnel}},\ }\href {https://bookstore.ams.org/stml-77/}
  {\emph {\bibinfo {title} {{Differential Geometry: Curves - Surfaces -
  Manifolds}}}},\  (\bibinfo  {publisher}
  {American Mathematical Society, Providence},\ \bibinfo {year} {2006})\BibitemShut
  {NoStop}%
\bibitem [{\citenamefont {Hasimoto}(1972)}]{Hasimoto1972}%
  \BibitemOpen
  \bibfield  {author} {\bibinfo {author} {\bibfnamefont {H.}~\bibnamefont
  {Hasimoto}},\ }\bibfield  {title} {\emph {\enquote {\bibinfo {title} {{A
  soliton on a vortex filament}},}\ }}\href {\doibase
  10.1017/S0022112072002307} {\bibfield  {journal} {\bibinfo  {journal}
  {J.  Fluid Mech.}\ }\textbf {\bibinfo {volume} {51}},\ \bibinfo
  {pages} {477} (\bibinfo {year} {1972})}\BibitemShut {NoStop}%
\bibitem [{\citenamefont {Khesin}\ \emph {et~al.}(2017)\citenamefont {Khesin},
  \citenamefont {Misiolek},\ and\ \citenamefont
  {Modin}}]{Khesin2017GeometricTransform}%
  \BibitemOpen
  \bibfield  {author} {\bibinfo {author} {\bibfnamefont {B.}~\bibnamefont
  {Khesin}}, \bibinfo {author} {\bibfnamefont {G.}~\bibnamefont {Misiolek}}, \
  and\ \bibinfo {author} {\bibfnamefont {K.}~\bibnamefont {Modin}},\ }\bibfield
   {title} {\emph {\enquote {\bibinfo {title} {{Geometric Hydrodynamics via
  Madelung Transform}},}\ }}\href {http://arxiv.org/abs/1711.00321}  {\bibfield  {journal} {\bibinfo  {journal}
  {arXiv:1711.00321}\ }
  (\bibinfo {year} {2017})}\BibitemShut {NoStop}%
\bibitem [{\citenamefont {Langer}\ and\ \citenamefont
  {Perline}(1991)}]{Langer1991PoissonEquation}%
  \BibitemOpen
  \bibfield  {author} {\bibinfo {author} {\bibfnamefont {J.}~\bibnamefont
  {Langer}}\ and\ \bibinfo {author} {\bibfnamefont {R.}~\bibnamefont
  {Perline}},\ }\bibfield  {title} {\emph {\enquote {\bibinfo {title} {{Poisson
  geometry of the filament equation}},}\ }}\href {\doibase 10.1007/BF01209148}
  {\bibfield  {journal} {\bibinfo  {journal} {J. Nonlinear Sci.}\
  }\textbf {\bibinfo {volume} {1}},\ \bibinfo {pages} {71} (\bibinfo {year}
  {1991})}\BibitemShut {NoStop}%
\bibitem [{\citenamefont {Langer}\ and\ \citenamefont
  {Perline}(1990)}]{Langer1990TheCurves}%
  \BibitemOpen
  \bibfield  {author} {\bibinfo {author} {\bibfnamefont {J.}~\bibnamefont
  {Langer}}\ and\ \bibinfo {author} {\bibfnamefont {R.}~\bibnamefont
  {Perline}},\ }\bibfield  {title} {\emph {\enquote {\bibinfo {title} {{The
  Hasimoto transformation and integrable flows on curves}},}\ }}\href {\doibase
  10.1016/0893-9659(90)90015-4} {\bibfield  {journal} {\bibinfo  {journal}
  {Appl. Math. Lett.}\ }\textbf {\bibinfo {volume} {3}},\ \bibinfo
  {pages} {61} (\bibinfo {year} {1990})}\BibitemShut {NoStop}%
\bibitem [{\citenamefont {Arnold}\ and\ \citenamefont
  {Khesin}(1999)}]{Arnold1999TopologicalHydrodynamics}%
  \BibitemOpen
  \bibfield  {author} {\bibinfo {author} {\bibfnamefont {V.~I.}\ \bibnamefont
  {Arnold}}\ and\ \bibinfo {author} {\bibfnamefont {B.~A.}\ \bibnamefont
  {Khesin}},\ }\href {http://www.springer.com/us/book/9780387949475} {\emph
  {\bibinfo {title} {{Topological Methods in Hydrodynamics}}}}, (\bibinfo  {publisher} {Springer-Verlag New York},\ \bibinfo
  {year} {1999})\BibitemShut {NoStop}%
\bibitem [{\citenamefont {Fukumoto}\ and\ \citenamefont
  {Miyazaki}(1991)}]{Fukumoto1991Three-dimensionalVelocity}%
  \BibitemOpen
  \bibfield  {author} {\bibinfo {author} {\bibfnamefont {Y.}~\bibnamefont
  {Fukumoto}}\ and\ \bibinfo {author} {\bibfnamefont {T.}~\bibnamefont
  {Miyazaki}},\ }\bibfield  {title} {\emph {\enquote {\bibinfo {title}
  {{Three-dimensional distortions of a vortex filament with axial velocity}},}\
  }}\href {\doibase 10.1017/S0022112091001143} {\bibfield  {journal} {\bibinfo
  {journal} {J. Fluid Mech.}\ }\textbf {\bibinfo {volume} {222}},\
  \bibinfo {pages} {369} (\bibinfo {year} {1991})}\BibitemShut {NoStop}%
\bibitem [{\citenamefont {Madelung}(1926)}]{Madelung1926EineSchrodinger}%
  \BibitemOpen
  \bibfield  {author} {\bibinfo {author} {\bibfnamefont {E.}~\bibnamefont
  {Madelung}},\ }\bibfield  {title} {\emph {\enquote {\bibinfo {title} {{Eine
  anschauliche Deutung der Gleichung von Schr{\"{o}}dinger}},}\ }}\href
  {\doibase 10.1007/BF01504657} {\bibfield  {journal} {\bibinfo  {journal}
  {Naturwissenschaften}\ }\textbf {\bibinfo {volume} {14}},\ \bibinfo {pages}
  {1004} (\bibinfo {year} {1926})}\BibitemShut {NoStop}%
\bibitem [{\citenamefont {Holmer}\ and\ \citenamefont
  {Zworski}(2008)}]{Holmer2008GeometricEvolution}%
  \BibitemOpen
  \bibfield  {author} {\bibinfo {author} {\bibfnamefont {J.}~\bibnamefont
  {Holmer}}\ and\ \bibinfo {author} {\bibfnamefont {M.}~\bibnamefont
  {Zworski}},\ }\bibfield  {title} {\emph {\enquote {\bibinfo {title}
  {{Geometric structure of NLS evolution}},}\ }}\href
  {http://arxiv.org/abs/0809.1844} {\bibfield  {journal} {\bibinfo  {journal}
  {arXiv:0809.1844}\ }
  (\bibinfo {year} {2008})}\BibitemShut
  {NoStop}%
\bibitem [{\citenamefont {{user153764
  (https://mathoverflow.net/users/110090/user153764)}}()}]{user153764https://mathoverflow.net/users/110090/user153764InfinitesimalEvolution}%
  \BibitemOpen
  \bibfield  {author} {\bibinfo {author} {\bibnamefont {{user153764
  (https://mathoverflow.net/users/110090/user153764)}}},\ }\href
  {https://mathoverflow.net/q/270157} {\enquote {\bibinfo {title}
  {{Infinitesimal generators and conserved quantities (Schrodinger type
  evolution)}},}\ }\bibinfo {howpublished} {MathOverflow}\BibitemShut {NoStop}%
\bibitem [{\citenamefont {B{\"{u}}hler}(2006)}]{Buhler2006AMechanics}%
  \BibitemOpen
  \bibfield  {author} {\bibinfo {author} {\bibfnamefont {O.}~\bibnamefont
  {B{\"{u}}hler}},\ }\href {https://bookstore.ams.org/cln-13}
  {\emph {\bibinfo {title} {{A Brief Introduction to Classical, Statistical,
  and Quantum Mechanics}}}},\  (\bibinfo
  {publisher} {Courant Institute of Mathematical Sciences, New York},\ \bibinfo {year} {2006})\BibitemShut {NoStop}%
\bibitem [{\citenamefont {Olver}(2000)}]{Olver2000ApplicationsEquations}%
  \BibitemOpen
  \bibfield  {author} {\bibinfo {author} {\bibfnamefont {P.~J.}\ \bibnamefont
  {Olver}},\ }\href {https://www.springer.com/us/book/9781468402742} {\emph
  {\bibinfo {title} {{Applications of Lie Groups to Differential
  Equations}}}},
  (\bibinfo  {publisher} {Springer-Verlag, New York},\ \bibinfo {year}
  {2000})\BibitemShut {NoStop}%
\bibitem [{\citenamefont {Dimas}\ and\ \citenamefont
  {Tsoubelis}(2006)}]{Dimas2006APDEs}%
  \BibitemOpen
  \bibfield  {author} {\bibinfo {author} {\bibfnamefont {S.}~\bibnamefont
  {Dimas}}\ and\ \bibinfo {author} {\bibfnamefont {D.}~\bibnamefont
  {Tsoubelis}},\ } \bibfield  {title} { {\enquote {\bibinfo {title}
  {{SYM: A new symmetry-finding package for Mathematica. }}}\ }In: Ibrahimov, N. H., Sophocleous, C. and P. A. Damianou, P. A. (eds), \emph{Proceedings Of 10th International Conference in MOdern GRoup ANalysis,}} \href {https://www.researchgate.net/profile/Dimitri_Tsoubelis/publication/228773733_SYM_A_new_symmetry-finding_package_for_Mathematica/links/0deec52119d34052bd000000/SYM-A-new-symmetry-finding-package-for-Mathematica.pdf}   {\bibfield ( University of Cyprus, Nicosia, \bibinfo {year} {2005})}\BibitemShut
  {NoStop}%
\bibitem [{\citenamefont {Poole}\ and\ \citenamefont
  {Hereman}(2011)}]{Poole2011SymbolicDimensions}%
  \BibitemOpen
  \bibfield  {author} {\bibinfo {author} {\bibfnamefont {D.}~\bibnamefont
  {Poole}}\ and\ \bibinfo {author} {\bibfnamefont {W.}~\bibnamefont
  {Hereman}},\ }\bibfield  {title} {\emph {\enquote {\bibinfo {title}
  {{Symbolic computation of conservation laws for nonlinear partial
  differential equations in multiple space dimensions}},}\ }}\href {\doibase
  10.1016/j.jsc.2011.08.014} {\bibfield  {journal} {\bibinfo  {journal}
  {J. Symb. Comput.}\ }\textbf {\bibinfo {volume} {46}},\
  \bibinfo {pages} {1355} (\bibinfo {year} {2011})}\BibitemShut {NoStop}%
\bibitem [{\citenamefont {Ludu}(2012)}]{Ludu2012NonlinearSurfaces}%
  \BibitemOpen
  \bibfield  {author} {\bibinfo {author} {\bibfnamefont {A.}~\bibnamefont
  {Ludu}},\ }\href {http://www.springer.com/us/book/9783642228940} {\emph
  {\bibinfo {title} {{Nonlinear Waves and Solitons on Contours and Closed
  Surfaces}}}},\  (\bibinfo  {publisher}
  {Springer-Verlag, Berlin, Heidelberg},\ \bibinfo {year} {2012})\BibitemShut {NoStop}%
\bibitem [{\citenamefont {Toro}(2009)}]{Toro2009NotionsEquations}%
  \BibitemOpen
  \bibfield  {author} {\bibinfo {author} {\bibfnamefont {E.~F.}\ \bibnamefont
  {Toro}},\ }in\ \href {\doibase 10.1007/b79761{\_}2} {\emph {\bibinfo
  {booktitle} {Riemann Solvers and Numerical Methods for Fluid Dynamics: A
  Practical Introduction}}}\ (\bibinfo  {publisher} {Springer-Verlag, Berlin, Heidelberg},\  \bibinfo {year}
  {2009})\BibitemShut {NoStop}%
\bibitem [{\citenamefont {Buckley}\ and\ \citenamefont
  {Leverett}(1942)}]{Buckley1942MechanismSands}%
  \BibitemOpen
  \bibfield  {author} {\bibinfo {author} {\bibfnamefont {S.~E.}\ \bibnamefont
  {Buckley}}\ and\ \bibinfo {author} {\bibfnamefont {M.~C.}\ \bibnamefont
  {Leverett}},\ }\bibfield  {title} {\emph {\enquote {\bibinfo {title}
  {{Mechanism of Fluid Displacement in Sands}},}\ }}\href {\doibase
  10.2118/942107-G}  {\bibfield  {journal} {\bibinfo  {journal}
  {Transactions of the AIME}\ }\textbf {\bibinfo {volume} {146}},\
   (\bibinfo {year} {1942})}\BibitemShut {NoStop}%
\bibitem [{\citenamefont {Newton}\ and\ \citenamefont
  {Keller}(1987)}]{Newton1987StabilityWaves}%
  \BibitemOpen
  \bibfield  {author} {\bibinfo {author} {\bibfnamefont {P.~K.}\ \bibnamefont
  {Newton}}\ and\ \bibinfo {author} {\bibfnamefont {J.~B.}\ \bibnamefont
  {Keller}},\ }\bibfield  {title} {\emph {\enquote {\bibinfo {title}
  {{Stability of Periodic Plane Waves}},}\ }}\href {\doibase 10.1137/0147063}
  {\bibfield  {journal} {\bibinfo  {journal} {SIAM J. Appl.
  Math.}\ }\textbf {\bibinfo {volume} {47}},\ \bibinfo {pages} {959}
  (\bibinfo {year} {1987})}\BibitemShut {NoStop}%
\bibitem [{\citenamefont {Strong}\ and\ \citenamefont
  {Carr}(2017)}]{Strong2017Non-HamiltonianCondensates}%
  \BibitemOpen
  \bibfield  {author} {\bibinfo {author} {\bibfnamefont {S.~A.}\ \bibnamefont
  {Strong}}\ and\ \bibinfo {author} {\bibfnamefont {L.~D.}\ \bibnamefont
  {Carr}},\ }\bibfield  {title} {\emph {\enquote {\bibinfo {title}
  {{Non-Hamiltonian Dynamics of Quantized Vortices in Bose-Einstein
  Condensates}},}\ }}\href {http://arxiv.org/abs/1712.05885}  {\bibfield  {journal} {\bibinfo  {journal}
  {arXiv:1712.05885}\ }
  (\bibinfo {year} {2017})}\BibitemShut {NoStop}%
\bibitem [{\citenamefont {Akhmediev}\ and\ \citenamefont
  {Ankiewicz}(2005)}]{Akhmediev2005DissipativeSolitons}%
  \BibitemOpen
  \bibfield  {author} {\bibinfo {author} {\bibfnamefont {N.}~\bibnamefont
  {Akhmediev}}\ and\ \bibinfo {author} {\bibfnamefont {A.}~\bibnamefont
  {Ankiewicz}},\ }\href {http://www.springer.com/us/book/9783540233732} {\textit
  {\bibinfo {title} {{Dissipative Solitons}}}},
  (\bibinfo  {publisher} {Springer},\ \bibinfo {address} {Berlin, Heidelberg},\ \bibinfo {year}
  {2005})\BibitemShut {NoStop}%  
\bibitem [{\citenamefont
  {Fukushima}(2011)}]{Fukushima2011PreciseTransformations}%
  \BibitemOpen
  \bibfield  {author} {\bibinfo {author} {\bibfnamefont {T.}~\bibnamefont
  {Fukushima}},\ }\bibfield  {title} {\emph {\enquote {\bibinfo {title}
  {{Precise and fast computation of a general incomplete elliptic integral of
  second kind by half and double argument transformations}},}\ }}\href
  {\doibase 10.1016/J.CAM.2011.03.004} {\bibfield  {journal} {\bibinfo
  {journal} {J. Comput. Appl. Math.}\ }\textbf
  {\bibinfo {volume} {235}},\ \bibinfo {pages} {4140} (\bibinfo {year}
  {2011})}\BibitemShut {NoStop}%
\bibitem [{\citenamefont {Bloch}(1997)}]{Bloch1997AGeometry}%
  \BibitemOpen
  \bibfield  {author} {\bibinfo {author} {\bibfnamefont {E.~D.}\ \bibnamefont
  {Bloch}},\ }\href {http://www.springer.com/us/book/9780817681227} {\emph
  {\bibinfo {title} {{A First Course in Geometric Topology and Differential
  Geometry}}}},\  (\bibinfo  {publisher}
  {Birkh{\"{a}}user Boston},\ \bibinfo {year} {1997})\BibitemShut {NoStop}%
\bibitem [{\citenamefont {Grinevich}\ and\ \citenamefont
  {Schmidt}(1997)}]{Grinevich1997ClosedEquation}%
  \BibitemOpen
  \bibfield  {author} {\bibinfo {author} {\bibfnamefont {P.~G.}\ \bibnamefont
  {Grinevich}}\ and\ \bibinfo {author} {\bibfnamefont {M.~U.}\ \bibnamefont
  {Schmidt}},\ }\bibfield  {title} {\emph {\enquote {\bibinfo {title} {{Closed
  curves in $\mathbb{R}^3$: a characterization in terms of curvature and torsion, the
  Hasimoto map and periodic solutions of the Filament Equation}},}\ }}\href
  {http://arxiv.org/abs/dg-ga/9703020} {\bibfield  {journal} {\bibinfo  {journal}
  {arXiv:9703020}\ }
  (\bibinfo {year} {1997})}\BibitemShut {NoStop}%
\bibitem [{\citenamefont {Salman}(2013)}]{Salman2013BreathersVortices}%
  \BibitemOpen
  \bibfield  {author} {\bibinfo {author} {\bibfnamefont {H.}~\bibnamefont
  {Salman}},\ }\bibfield  {title} {\emph {\enquote {\bibinfo {title}
  {{Breathers on Quantized Superfluid Vortices}},}\ }}\href {\doibase
  10.1103/PhysRevLett.111.165301} {\bibfield  {journal} {\bibinfo  {journal}
  {Phys. Rev. Lett.}\ }\textbf {\bibinfo {volume} {111}},\ \bibinfo {pages}
  {165301} (\bibinfo {year} {2013})}\BibitemShut {NoStop}%
  \bibitem [{Note1()}]{Note1}%
  \BibitemOpen
   \href@noop {} {}\bibinfo {note} 
{Mark Raizen, UT Austin, private communication (2017)}\BibitemShut 
{NoStop}%
\bibitem [{\citenamefont {H{\"{a}}nninen}\ and\ \citenamefont
  {Baggaley}(2014)}]{Hanninen2014VortexTurbulence.}%
  \BibitemOpen
  \bibfield  {author} {\bibinfo {author} {\bibfnamefont {R.}~\bibnamefont
  {H{\"{a}}nninen}}\ and\ \bibinfo {author} {\bibfnamefont {A.~W.}\
  \bibnamefont {Baggaley}},\ }\bibfield  {title} {\emph {\enquote {\bibinfo
  {title} {{Vortex filament method as a tool for computational visualization of
  quantum turbulence.}}}\ }}\href {\doibase 10.1073/pnas.1312535111} {\bibfield
   {journal} {\bibinfo  {journal} {Proc. Natl. Acad.
  Sci. USA}\ }\textbf {\bibinfo {volume} {111
  Suppl 1}},\ \bibinfo {pages} {4667} (\bibinfo {year} {2014})}\BibitemShut
  {NoStop}%
\bibitem [{\citenamefont {Eyink}\ and\ \citenamefont
  {Sreenivasan}(2006)}]{Eyink2006OnsagerTurbulence}%
  \BibitemOpen
  \bibfield  {author} {\bibinfo {author} {\bibfnamefont {G.~L.}\ \bibnamefont
  {Eyink}}\ and\ \bibinfo {author} {\bibfnamefont {K.~R.}\ \bibnamefont
  {Sreenivasan}},\ }\bibfield  {title} {\emph {\enquote {\bibinfo {title}
  {{Onsager and the theory of hydrodynamic turbulence}},}\ }}\href {\doibase
  10.1103/RevModPhys.78.87} {\bibfield  {journal} {\bibinfo  {journal} {Rev.
  Mod. Phys.}\ }\textbf {\bibinfo {volume} {78}},\ \bibinfo {pages} {87}
  (\bibinfo {year} {2006})}\BibitemShut {NoStop}%
\bibitem [{\citenamefont
  {Onsager}(1931{\natexlab{a}})}]{Onsager1931ReciprocalI.}%
  \BibitemOpen
  \bibfield  {author} {\bibinfo {author} {\bibfnamefont {L.}~\bibnamefont
  {Onsager}},\ }\bibfield  {title} {\emph {\enquote {\bibinfo {title}
  {{Reciprocal Relations in Irreversible Processes. I.}}}\ }}\href {\doibase
  10.1103/PhysRev.37.405} {\bibfield  {journal} {\bibinfo  {journal} {Phys.
  Rev.}\ }\textbf {\bibinfo {volume} {37}},\ \bibinfo {pages} {405} (\bibinfo
  {year} {1931}{\natexlab{a}})}\BibitemShut {NoStop}%
\bibitem [{\citenamefont
  {Onsager}(1931{\natexlab{b}})}]{Onsager1931ReciprocalII.}%
  \BibitemOpen
  \bibfield  {author} {\bibinfo {author} {\bibfnamefont {L.}~\bibnamefont
  {Onsager}},\ }\bibfield  {title} {\emph {\enquote {\bibinfo {title}
  {{Reciprocal Relations in Irreversible Processes. II.}}}\ }}\href {\doibase
  10.1103/PhysRev.38.2265} {\bibfield  {journal} {\bibinfo  {journal} {Phys.
  Rev.}\ }\textbf {\bibinfo {volume} {38}},\ \bibinfo {pages} {2265}
  (\bibinfo {year} {1931}{\natexlab{b}})}\BibitemShut {NoStop}%
\bibitem [{\citenamefont {Drivas}\ and\ \citenamefont
  {Eyink}(2017)}]{Drivas2017AnEquations}%
  \BibitemOpen
  \bibfield  {author} {\bibinfo {author} {\bibfnamefont {T.~D.}\ \bibnamefont
  {Drivas}}\ and\ \bibinfo {author} {\bibfnamefont {G.~L.}\ \bibnamefont
  {Eyink}},\ }\bibfield  {title} {\emph {\enquote {\bibinfo {title} {{An
  Onsager Singularity Theorem for Turbulent Solutions of Compressible Euler
  Equations}},}\ }}\href {\doibase 10.1007/s00220-017-3078-4} {\bibfield
  {journal} {\bibinfo  {journal} {Commun. Math. Phys.}\ }
  (\bibinfo {year} {2017})}\BibitemShut {NoStop}%
\bibitem [{\citenamefont {McKeown}\ \emph {et~al.}(2017)\citenamefont
  {McKeown}, \citenamefont {Monico}, \citenamefont {Pumir}, \citenamefont
  {Brenner},\ and\ \citenamefont {Rubinstein}}]{McKeown2017TheCollisions}%
  \BibitemOpen
  \bibfield  {author} {\bibinfo {author} {\bibfnamefont {R.}~\bibnamefont
  {McKeown}}, \bibinfo {author} {\bibfnamefont {R.~O.}\ \bibnamefont {Monico}},
  \bibinfo {author} {\bibfnamefont {A.}~\bibnamefont {Pumir}}, \bibinfo
  {author} {\bibfnamefont {M.~P.}\ \bibnamefont {Brenner}}, \ and\ \bibinfo
  {author} {\bibfnamefont {S.~M.}\ \bibnamefont {Rubinstein}},\ }  \bibfield  {title} { {\enquote {\bibinfo {title}
  {{The Emergence of Small Scales in Vortex Ring Collisions. }}}\ } In: \emph{70th Annual Meeting Of The APS Division Of Fluid Dynamics,}} \href {https://gfm.aps.org/meetings/dfd-2017/59b7fbecb8ac316d38841c31}   {\bibfield {publisher} (APS, Denver, \bibinfo {year} {2017})}\BibitemShut {NoStop}%
\bibitem [{\citenamefont {Kimura}\ and\ \citenamefont
  {Moffatt}(2018{\natexlab{a}})}]{Kimura2018ScalingEvolution}%
  \BibitemOpen
  \bibfield  {author} {\bibinfo {author} {\bibfnamefont {Y.}~\bibnamefont
  {Kimura}}\ and\ \bibinfo {author} {\bibfnamefont {H.~K.}\ \bibnamefont
  {Moffatt}},\ }\bibfield  {title} {\emph {\enquote {\bibinfo {title} {{Scaling
  properties towards vortex reconnection under Biot–Savart evolution}},}\
  }}\href {\doibase 10.1088/1873-7005/aa710c} {\bibfield  {journal} {\bibinfo
  {journal} {Fluid Dyn. Res.}\ }\textbf {\bibinfo {volume} {50}},\
  \bibinfo {pages} {011409} (\bibinfo {year} {2018}{\natexlab{a}})}\BibitemShut
  {NoStop}%
\bibitem [{\citenamefont {Kimura}\ and\ \citenamefont
  {Moffatt}(2018{\natexlab{b}})}]{Kimura2018AEvolution}%
  \BibitemOpen
  \bibfield  {author} {\bibinfo {author} {\bibfnamefont {Y.}~\bibnamefont
  {Kimura}}\ and\ \bibinfo {author} {\bibfnamefont {H.~K.}\ \bibnamefont
  {Moffatt}},\ }\bibfield  {title} {\emph {\enquote {\bibinfo {title} {{A tent
  model of vortex reconnection under Biot–Savart evolution}},}\ }}\href
  {\doibase 10.1017/jfm.2017.769} {\bibfield  {journal} {\bibinfo  {journal}
  {J. Fluid Mech.}\ }\textbf {\bibinfo {volume} {834}},\ \bibinfo
  {pages} {R1} (\bibinfo {year} {2018}{\natexlab{b}})}\BibitemShut {NoStop}%
\bibitem [{\citenamefont {Serafini}\ \emph {et~al.}(2017)\citenamefont
  {Serafini}, \citenamefont {Galantucci}, \citenamefont {Iseni}, \citenamefont
  {Bienaim{\'{e}}}, \citenamefont {Bisset}, \citenamefont {Barenghi},
  \citenamefont {Dalfovo}, \citenamefont {Lamporesi},\ and\ \citenamefont
  {Ferrari}}]{Serafini2017VortexCondensates}%
  \BibitemOpen
  \bibfield  {author} {\bibinfo {author} {\bibfnamefont {S.}~\bibnamefont
  {Serafini}}, \bibinfo {author} {\bibfnamefont {L.}~\bibnamefont
  {Galantucci}}, \bibinfo {author} {\bibfnamefont {E.}~\bibnamefont {Iseni}},
  \bibinfo {author} {\bibfnamefont {T.}~\bibnamefont {Bienaim{\'{e}}}},
  \bibinfo {author} {\bibfnamefont {R.~N.}\ \bibnamefont {Bisset}}, \bibinfo
  {author} {\bibfnamefont {C.~F.}\ \bibnamefont {Barenghi}}, \bibinfo {author}
  {\bibfnamefont {F.}~\bibnamefont {Dalfovo}}, \bibinfo {author} {\bibfnamefont
  {G.}~\bibnamefont {Lamporesi}}, \ and\ \bibinfo {author} {\bibfnamefont
  {G.}~\bibnamefont {Ferrari}},\ }\bibfield  {title} {\emph {\enquote {\bibinfo
  {title} {{Vortex Reconnections and Rebounds in Trapped Atomic Bose-Einstein
  Condensates}},}\ }}\href {\doibase 10.1103/PhysRevX.7.021031} {\bibfield
  {journal} {\bibinfo  {journal} {Phys. Rev. X}\ }\textbf {\bibinfo {volume}
  {7}},\ \bibinfo {pages} {21031} (\bibinfo {year} {2017})}\BibitemShut
  {NoStop}%
\end{thebibliography}%



\end{document}
%
% ****** End of file apssamp.tex ******