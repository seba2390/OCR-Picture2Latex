\section{Evaluation Suite}
%We design a comprehensive evaluation suite to assess the performance of instructions along 5 metrics relevant to practical ICL. To ensure a fair analysis, we test each instruction selection method across 9 tasks and 13 autoregressive models from 4 model families.

\subsection{Prompt format}

\begin{figure}[t]
    \centering
    %\begin{tikzpicture}[
  node distance = 0.2cm and 0.2cm,
  single node/.style = {rectangle, fill=#1, fill opacity=0.5, text opacity=1, align=left, font=\fontfamily{FiraSans-LF}\selectfont\footnotesize, text width=0.40\textwidth, inner xsep=3mm, inner ysep=3mm, rounded corners=1mm},
]

\node[single node=lightyellow] (instruction) 
{\normalFont{Is this review positive or negative?}};

\node[single node=lightblue, below=of instruction] (example1)
{\normalFont{Review:} \ultraLight{Whoever wrote the screenplay for this movie obviously never consulted…} \\ \normalFont{Sentiment:} \ultraLight{Negative}};

\node[single node=lightblue, below=of example1] (example2)
{\normalFont{Review:} \ultraLight{The story centers around Barry McKenzie who must go to England…} \\ \normalFont{Sentiment:} \ultraLight{Positive}};

\node[single node=lightblue, below=of example2] (example3)
{\normalFont{Review:} \ultraLight{This film is just plain horrible. John Ritter doing pratt falls, 75\% of the actors…} \\ \normalFont{Sentiment:} \ultraLight{Negative}};


\node[single node=lightpink, below=of example3] (test) 
{\normalFont{Review:} \ultraLight{BLACK WATER has to be one of the best Australian movies I've seen in many…} \\ \normalFont{Sentiment:} };

\draw [black, line width=0.5pt, rounded corners=1mm] ([shift={(-2mm,-2mm)}]current bounding box.south west) rectangle ([shift={(2mm,2mm)}]current bounding box.north east);

\end{tikzpicture}
    \includegraphics[width=\linewidth]{figures/prompt_example.png}
    \caption{
        An example of a prompt following the template we use for IMDB. By  `prompt' we refer to the concatenation of the \textcolor{darkyellow}{instruction}, solved \textcolor{blue}{demonstrations} and an unsolved \textcolor{red}{test example}.
    }
    \label{fig:prompt_example}
\end{figure}
We define a `prompt' as the full textual input provided to an LLM. Our evaluation suite supports the use of any number of demonstrations, arbitrary demonstration templates and the inclusion of custom strings anywhere within the prompt. Since the instructions used can be set to any arbitrary strings, users are free to use any external means to select instructions and have them evaluated by our suite.

For consistency, we conduct all experiments in this work using prompts that begin with an instruction, continue with a sequence of annotated training demonstrations, and conclude with an unsolved test example\footnote{Instructions are omitted during `Null instruction' evaluations. Demonstrations are omitted in zero-shot evaluations.} (Figure~\ref{fig:prompt_example}), and express each example in a minimal, task-specific key-value format (Table~\ref{table:templates}) that reflects task semantics.

\subsection{Metrics}
\begin{figure*}
  \begin{subfigure}[b]{0.32\textwidth}
    \centering
    \resizebox{\linewidth}{!}{
    \iffalse

\begin{tikzpicture}[
  node distance = 0.2cm and 0.2cm,
  single node/.style = {rectangle, fill=#1, text opacity=1, opacity=0.5, align=center, font=\footnotesize, text width=2cm, inner sep=1.5mm, rounded corners=0.5mm},
  bigbox/.style = {draw, thick, rounded corners, rectangle},
  arrow/.style={thick, -latex},
]

\node[single node=lightyellow] (instruction) {instruction};

\node[single node=lightblue, right=of instruction] (example1) {$(x_1^{tr}, y_1^{tr})$};
\node[single node=lightblue, right=of example1] (example2) {$(x_2^{tr}, y_2^{tr})$};
\node[single node=lightblue, right=of example2] (example3) {$(x_3^{tr}, y_3^{tr})$};

\node[single node=lightpink, right=of example3] (test) {$x_1^{test}$};

\node[bigbox, fit=(instruction) (example1) (example2) (example3) (test)] (box1) {};

\node[single node=lightyellow, below=1cm of instruction] (instruction2) {instruction};

\node[single node=lightblue, right=of instruction2] (example12) {$(x_1^{tr}, y_1^{tr})$};
\node[single node=lightblue, right=of example12] (example22) {$(x_2^{tr}, y_2^{tr})$};
\node[single node=lightblue, right=of example22] (example32) {$(x_3^{tr}, y_3^{tr})$};

\node[single node=lightpink, right=of example32] (test2) {\textsc{Perturb}$(x_1^{test})$};

\node[bigbox, fit=(instruction2) (example12) (example22) (example32) (test2)] (box2) {};

\draw [arrow] (box1.south) -- (box2.north);

\end{tikzpicture}

\fi


\begin{tikzpicture}[
  node distance = 0.2cm and 1cm, % Increase horizontal distance between nodes
  single node/.style = {rectangle, fill=#1, text opacity=1, opacity=0.5, align=center, font=\footnotesize, text width=2cm, inner sep=1.5mm, rounded corners=0.5mm},
  bigbox/.style = {draw, thick, rounded corners, rectangle, inner sep=1.5mm}, % Reduce inner sep for less padding
  arrow/.style={thick, -latex},
]

\node[single node=lightyellow] (instruction1) {instruction};
\node[single node=lightblue, below=of instruction1] (example1) {$(x_1^{tr}, y_1^{tr})$};
\node[single node=lightblue, below=of example1] (example2) {$(x_2^{tr}, y_2^{tr})$};
\node[single node=lightblue, below=of example2] (example3) {$(x_3^{tr}, y_3^{tr})$};
\node[single node=lightpink, below=of example3] (test1) {$x_1^{te}$};

\node[bigbox, fit=(instruction1) (example1) (example2) (example3) (test1)] (box1) {};

\node[single node=lightyellow, right=of instruction1] (instruction2) {instruction};
\node[single node=lightblue, below=of instruction2] (example12) {$(x_1^{tr}, y_1^{tr})$};
\node[single node=lightblue, below=of example12] (example22) {$(x_2^{tr}, y_2^{tr})$};
\node[single node=lightblue, below=of example22] (example32) {$(x_3^{tr}, y_3^{tr})$};
\node[single node=lightpink, below=of example32] (test2) {\textsc{Perturb}$(x_1^{te})$};

\node[bigbox, fit=(instruction2) (example12) (example22) (example32) (test2)] (box2) {};

\draw [arrow] (box1.east) -- (box2.west);
\end{tikzpicture}

    }
    \caption{Perturbation accuracy}
  \end{subfigure}\hfill
  \begin{subfigure}[b]{0.32\textwidth}
    \centering
    \resizebox{\linewidth}{!}{
    \input{figures/selection_example_tikz}
    }
    \caption{Selectional sensitivity}
  \end{subfigure}\hfill
  \begin{subfigure}[b]{0.32\textwidth}
    \centering
    \resizebox{\linewidth}{!}{
    \iffalse
\begin{tikzpicture}[
  node distance = 0.2cm and 0.2cm,
  single node/.style = {rectangle, fill=#1, text opacity=1, opacity=0.5, align=center, font=\footnotesize, text width=2cm, inner sep=1.5mm, rounded corners=0.5mm},
  bigbox/.style = {draw, thick, rounded corners, rectangle},
  arrow/.style={thick, -latex},
]

\node[single node=lightyellow] (instruction) {instruction};

\node[single node=lightblue, right=of instruction] (example1) {$(x_1^{tr}, y_1^{tr})$};
\node[single node=lightblue, right=of example1] (example2) {$(x_2^{tr}, y_2^{tr})$};
\node[single node=lightblue, right=of example2] (example3) {$(x_3^{tr}, y_3^{tr})$};

\node[single node=lightpink, right=of example3] (test) {$x_1^{test}$};

\node[bigbox, fit=(instruction) (example1) (example2) (example3) (test)] (box1) {};

\node[single node=lightyellow, below=1cm of instruction] (instruction2) {instruction};

\node[single node=lightblue, right=of instruction2] (example12) {$(x_2^{tr}, y_2^{tr})$};
\node[single node=lightblue, right=of example12] (example22) {$(x_1^{tr}, y_1^{tr})$};
\node[single node=lightblue, right=of example22] (example32) {$(x_3^{tr}, y_3^{tr})$};

\node[single node=lightpink, right=of example32] (test2) {$x_1^{test}$};

\node[bigbox, fit=(instruction2) (example12) (example22) (example32) (test2)] (box2) {};

\draw [arrow] (box1.south) -- (box2.north);

\node[single node=lightyellow, below=0.5cm of instruction2] (instruction3) {instruction};

\node[single node=lightblue, right=of instruction3] (example13) {$(x_3^{tr}, y_3^{tr})$};
\node[single node=lightblue, right=of example13] (example23) {$(x_2^{tr}, y_2^{tr})$};
\node[single node=lightblue, right=of example23] (example33) {$(x_1^{tr}, y_1^{tr})$};

\node[single node=lightpink, right=of example33] (test3) {$x_1^{test}$};

\node[bigbox, fit=(instruction3) (example13) (example23) (example33) (test3)] (box3) {};


\end{tikzpicture}
\fi



\begin{tikzpicture}[
  node distance = 0.2cm and 0.2cm,
  single node/.style = {rectangle, fill=#1, text opacity=1, opacity=0.5, align=center, font=\footnotesize, text width=2cm, inner sep=1.5mm, rounded corners=0.5mm},
  bigbox/.style = {draw, thick, rounded corners, rectangle},
  arrow/.style={thick, -latex},
  bigbox opaque/.style = {draw, thick, rounded corners, rectangle, opacity=1, fill=white},
  arrow/.style={thick, -latex},
]

\node[single node=lightyellow] (instruction) {instruction};

\node[single node=lightblue, below=of instruction] (example1) {$(x_1^{tr}, y_1^{tr})$};
\node[single node=lightblue, below=of example1] (example2) {$(x_2^{tr}, y_2^{tr})$};
\node[single node=lightblue, below=of example2] (example3) {$(x_3^{tr}, y_3^{tr})$};

\node[single node=lightpink, below=of example3] (test) {$x_1^{te}$};

\node[bigbox, fit=(instruction) (example1) (example2) (example3) (test)] (box1) {};

\begin{pgfonlayer}{layer1}
\node[single node=lightyellow, right=1cm of instruction] (instruction2) {instruction};

\node[single node=lightblue, below=of instruction2] (example12) {$(x_3^{tr}, y_3^{tr})$};
\node[single node=lightblue, below=of example12] (example22) {$(x_1^{tr}, y_1^{tr})$};
\node[single node=lightblue, below=of example22] (example32) {$(x_2^{tr}, y_2^{tr})$};

\node[single node=lightpink, below=of example32] (test2) {$x_1^{te}$};

\node[bigbox, fit=(instruction2) (example12) (example22) (example32) (test2)] (box2) {};
\end{pgfonlayer}

\begin{pgfonlayer}{layer2}
\node[bigbox opaque, fit=(instruction2) (example12) (example22) (example32) (test2)] (box2) {};
\end{pgfonlayer}

\begin{pgfonlayer}{layer3}
\node[single node=lightyellow, below=-0.25cm of instruction2, xshift=0.25cm] (instruction3) {instruction};

\node[single node=lightblue, below=of instruction3] (example13) {$(x_1^{tr}, y_1^{tr})$};
\node[single node=lightblue, below=of example13] (example23) {$(x_2^{tr}, y_2^{tr})$};
\node[single node=lightblue, below=of example23] (example33) {$(x_3^{tr}, y_3^{tr})$};

\node[single node=lightpink, below=of example33] (test3) {$x_1^{test}$};

\node[bigbox, fit=(instruction3) (example13) (example23) (example33) (test3)] (box3) {};
\end{pgfonlayer}

\begin{pgfonlayer}{layer4}
\node[bigbox opaque, fit=(instruction3) (example13) (example23) (example33) (test3)] (box3) {};
\end{pgfonlayer}

\begin{pgfonlayer}{layer5}
\node[single node=lightyellow, below=-0.25cm of instruction3, xshift=0.25cm] (instruction4) {instruction};

\node[single node=lightblue, below=of instruction4] (example14) {$(x_1^{tr}, y_1^{tr})$};
\node[single node=lightblue, below=of example14] (example24) {$(x_2^{tr}, y_2^{tr})$};
\node[single node=lightblue, below=of example24] (example34) {$(x_3^{tr}, y_3^{tr})$};

\node[single node=lightpink, below=of example34] (test4) {$x_1^{test}$};

\node[bigbox, fit=(instruction4) (example14) (example24) (example34) (test4)] (box4) {};
\end{pgfonlayer}

\begin{pgfonlayer}{layer6}
\node[bigbox, fit=(instruction4) (example14) (example24) (example34) (test4)] (box4) {};
\end{pgfonlayer}

\draw [arrow] (box1.east) -- (box2.west);

\end{tikzpicture}
    }
    \caption{Permutational sensitivity}
  \end{subfigure}
  \caption{
        We provide schematic diagrams that show prompts are modified to measure \textit{perturbation accuracy}, \textit{selectional sensitivity} and \textit{permutational sensitivity}. We perturb the test input to measure perturbation accuracy, and demonstration selection and permutation respectively while measuring selectional and permutational sensitivity.
    }
    \label{fig:sensitivity-metrics}
\end{figure*}

\label{sec:metrics}
\paragraph{Accuracy metrics} Accuracy is typically the primary metric of interest in ICL. While ICL is most commonly performed in few-shot settings where a handful of annotated demonstrations are included in the prompt, models are also prompted zero-shot without the use of such demonstrations. Since real-world scenarios can often contain grammatical errors and misspellings in the test input, it is desirable to find prompts robust to these perturbations. Hence, we measure \textit{zero-shot accuracy}, \textit{few-shot accuracy}, and \textit{perturbation accuracy}\footnote{We choose to treat this as an accuracy metric rather than a sensitivity metric since it is not meaningful to measure sensitivity to such perturbations in situations where a prompt only elicits near random-chance task performance from a model.} in our evaluations. Following \citet{helm}, we measure perturbation accuracy by introducing random capitalization, spacing, contractions and common misspellings in the test input.

\paragraph{Sensitivity metrics} Previous work has shown that the accuracy obtained using a prompt template can fluctuate significantly as a function of the set of demonstrations included in the prompt ~\cite{Liu2021WhatMG, Su2022SelectiveAM, rubin2022learning, Wang2023LargeLM} and the order they are presented in \cite{fantasticallyorderedprompts}. It may be desirable in practice to identify prompt templates and instructions that offer consistent performance regardless of the choice of demonstrations and their arrangement. Hence, we introduce \textit{selectional sensitivity} and \textit{permutational sensitivity} metrics to measure the sensitivity of chosen instructions respectively to selected demonstrations, and the order in which they are arranged. We quantify the sensitivity of an instruction (given a model and task) using the standard deviation of accuracies obtained on varying  the selection or permutation of the demonstrations used, each across 16 random choices.

\subsection{Aggregating metrics across Models}

Each instruction selection method being tested across $N$ models and $M$ datasets yields $NM$ values per metric. Comparing these $NM$-dimensional vectors directly is complex. It can be challenging to reduce them to a single representative scalar. Simple approaches such as computing the mean of these $NM$ values can prove inadequate since the resulting scores would tend to be heavily influenced by metric values that exhibit a high variance across different inspected methods. 

We opt against using aggregation techniques used by previous works \cite{helm, bigbench} due to their drawbacks (Section~\ref{app:scoring}) and instead adopt `mean relative gain' as a means to aggregate accuracy metrics across multiple models. We rely on simple averaging for sensitivity metrics, partly because we observe that these quantities do not show much variation across methods.

\subsubsection{Accuracy metrics}
Considering the range of models and datasets in our evaluation suite, we unsurprisingly observe substantial variation in accuracy magnitudes across model scales and tasks. However, we notice that the degree of variation in accuracy due to instruction choice is usually considerably smaller than the degree of variation due to model and task choice.

To meaningfully compare and aggregate the relative performance of different instruction selection methods across models, we use a measure called \textit{mean relative gain}. First, we define the \textit{relative gain} for a value $x$ from a population $P$ as the percentage by which $x$ exceeds the mean value of $P$:

$$\text{r-gain}_P(x) = 100 \times \dfrac{x-\mu_P}{\mu_P}$$

Consider a collection of models $\mathcal{M}$ and instructions $\mathcal{I}$ for a task $t$. Given a model $m$, we calculate the raw accuracy scores $s_{tmi}$ for each instruction $i \in \mathcal{I}$. Taking this set $S_{tm}$ to be the population, we compare the performances of the instructions against each other by computing their corresponding relative gains $r_{tmi} = \text{r-gain}_{S_{tm}}(s_{tmi})$. Each $r_{tmi}$ represents the degree by which method $i$ outperforms the average performance along the metric on task $t$ for model $m$.

We now define the mean relative gain as 
$$\overline{r}_{ti} = \dfrac{1}{|\mathcal{M}|} \sum_{m \in \mathcal{M}} r_{tmi}$$

These $\overline{r}_{ti}$ values, tabulated and analyzed in \secref{sec:results}, capture not only the ordinal information about each method's performance on a given task but also provide an intuitive sense of the magnitude by which these methods outperform others. Specifically, if an induction method $i$ has a mean relative gain $\overline{r}_{ti}$ on task $t$, this means that method $i$ exceeds average performance (across $\mathcal{I}$) on task $t$ by $\overline{r}_{ti}$ percent when averaged across models $\mathcal{M}$. 

\subsubsection{Sensitivity metrics}
To aggregate the sensitivity of an instruction selection/induction method $i$ over all models for a task $t$, we simply compute the average of the raw sensitivity scores (described in \secref{sec:metrics}). Specifically, if $\sigma_{tmi}$ is the raw sensitivity score obtained for model $m$ and task $t$ when using instruction $i$, then the aggregated sensitivity score $\overline{\sigma}_{ti}$ is given by 

$$\overline{\sigma}_{ti} = \dfrac{1}{|\mathcal{M}|} \sum_{m \in \mathcal{M}} \sigma_{tmi}$$

We choose to avoid more sophisticated aggregation strategies like relative gain for sensitivity metrics since standard deviations are already secondary metrics making it unintuitive to discuss the relative gain of the standard deviation obtained using a method over the average. %\mz{Do we still average the stds? It seems that in table 5 we didn't do it?}

\iffalse
To fairly compare instruction induction methods across various models and tasks, we adopt mean z-scores as a measure of relative performance along a specific metric. This choice allows us to capture the magnitude of variation in the underlying metric without requiring expert knowledge about its specific range of variation.~\footnote{Please refer to Appendix \ref{app:scoring} for more detailed information on why we opted not to use head-to-head win rates from HELM~\cite{helm} and normalized accuracy from BigBench~\cite{bigbench} as alternative scoring methods.}

A z-score measures the relationship between a value and the population it is drawn from. It quantifies how many standard deviations a value $x$ is away from the mean of the population $P$:
$$\text{z-score}_{P}(x) = \dfrac{x - \mu_P}{\sigma_P},$$
Let $s_{pmt}$ represent the raw scores of each instruction induction method $p$ for model $m$ and task $t$. We denote the subset of these metric scores associated with each pair $(m,t)$ as $$S(m,t) = \{s_{pmt}: p \in \mathcal{P}\}.$$ We compute the z-score for metric scores within each $S(m, t)$ and group them by a method $p$:
$$Z_p = \{\text{z-score}_{S(m,t)}(s_{pmt}) : (m,t) \in \mathcal{M} \times \mathcal{T}\}.$$
We then take an average and derive the mean z-score for $p$ across tasks and models:
$$\overline{z}_p = \sum_{z_p \in Z_p} z_p / |Z_p|$$

These mean z-score values are indicative of the degree to which each instruction induction method outperforms all other method under consideration along a certain metric, aggregated over all possible choices of models in $\mathcal{M}$ and tasks in $\mathcal{T}$.  Although they do communicate the magnitude by which certain methods outperform others, they are indeed bound by the limitation that they have to be recomputed for all $p$ whenever a new method $p'$ is inserted into $\mathcal{P}$. However, this limitation is shown by scoring systems used in previous works \cite{helm} as well and is not a cause for concern in practice when evaluating the relative performances of instruction induction methods.
\fi


\subsection{Tasks}
\begin{table}[t]
\centering
\resizebox{\linewidth}{!}{
\begin{tabular}{ll}
\toprule
\textbf{Task Type} & \textbf{Tasks} \\ 
\midrule
\multirow{5}{*}{Classification (CLS)}      & AG News \cite{agnews} \\ 
                                            & ANLI \cite{anli} \\
                                            & BoolQ \cite{boolq} \\ 
                                            & IMDB \cite{imdb} \\ 
                                            & TweetEval Emotion \cite{emotion} \\
\midrule                                            
\multirow{2}{*}{Multiple-choice (MCQ)}                      & CosmosQA \cite{cosmosqa} \\ 
& HellaSwag \cite{hellaswag} \\
\midrule
\multirow{2}{*}{Generative QA (GQA)}                         & NQ-Open \cite{nqopen} \\ 
& TriviaQA \cite{triviaqa} \\ \bottomrule
\end{tabular}}
\caption{Tasks included in our evaluation suite.}
\label{table:tasks}
\end{table} 
While previous instruction induction \cite{ape, rlprompt} work has tended to focus mostly on classification tasks, we include 9 tasks (\tableref{table:tasks}) in our evaluation suite spanning classification (CLS), multiple-choice question-answering (MCQ) and generative question-answering (GQA) to assess the applicability of instruction selection and induction methods to other task-types as well. We concentrate on tasks that are challenging to contemporary language models, and yet are not so demanding that the performance of these models does not exceed random chance. We exclude certain generative tasks, like summarization, which are challenging to assess objectively.~\footnote{Standard summarization metrics correlate poorly with human preferences~\cite{helm, goyal2023news}.}

\begin{table}[t]
\centering
\resizebox{\linewidth}{!}{
\begin{tabular}{@{}ll@{}}
\toprule
\textbf{Model Family} & \textbf{Size} \\ \midrule
BLOOM~\cite{bloom}                 & 1.1B, 1.7B, 3B, 7.1B              \\
GPT Neo   ~\cite{gptneo, gptneox}            & 1.3B, 2.7B, 20B                   \\
LLaMA~\cite{llama}                 & 7B, 13B                           \\
OPT    ~\cite{opt}               & 1.3B, 2.7B, 6.7B, 13B             \\ \bottomrule
\end{tabular}}
\caption{Model families and corresponding model scales included in our evaluation suite.}
\label{table:models}
\end{table}

\subsection{Models}
\section{Methods}\label{sec:methods}

Given a single facial image of an individual, our objective is to endow the pre-trained T2I model with the ability to efficiently re-contextualize this unique identity under various textual prompts. These prompts may include variations in clothing, accessories, styles, or backgrounds.


The overall framework is shown in Fig.\ref{fig:pipeline}, given a pre-trained T2I model, 
to achieve fast and identity-preserved image generation, we first directly encode the target identity into the word embedding  space (represented as the pseudo word $S*$) with the proposed $M^2$ ID encoder. Afterward,
$S*$ is integrated with the input template prompt for 
generating the text-guided image. To empower the editability for the target identity, a novel \emph{self-augmented editability learning} is further introduced to train the $M^2$ ID encoder with the editability objective.


In the following parts, we first briefly introduce the pre-trained diffusion-based text-to-image model used in our work, then describe our proposed  $M^2$ ID encoder and self-augmented editability learning in detail, respectively.

\subsection{Preliminary}
In this work, we adopt the open-sourced Stable Diffusion 2.1-base (SD) as our text-to-image model, which has been trained on billions of images and shows amazing image generation quality and prompt understanding. 

SD is a kind of Latent Diffusion Model (LDM) \cite{rombach2022high}. LDM firstly represents the input image $x$ in a lower resolution latent space $z$ via a Variational Auto-Encoder (VAE) \cite{kingma2013auto}. Then a text-conditioned diffusion model is trained to generate the latent code of the target image from text input $c$. The loss function of this diffusion model can be formulated as:
\begin{equation}
    \mathcal{L}_{diffusion} = \mathbb{E}_{\epsilon,z,c,t}[\lVert{\epsilon - \epsilon_{\theta}(z_t,c,t)}\rVert_2^2],
\end{equation}
where $\epsilon_{\theta}$ is the noise predicted by the model with learnable parameters $\theta$, $\epsilon$ is noise sampled from standard normal distribution, $t$ is the time step, and $z_t$ is noisy latent at the time step $t$.

During inference, the image is generated by two stages: latent code is first generated by the diffusion model, then the decoder is employed to map the latent code to image space. 

\subsection{$M^{2}$ ID Encoder}

To accurately represent the input face identity, we propose a novel Multi-word Multi-scale embedding ID encoder ($M^2$ ID encoder) for an accurate mapping, which is achieved by the multi-scale ID features extracted from a dedicated backbone then followed by multiple word embedding projection.




\myparagraph{Backbone.} We argue that an accurate representation of the face identity is crucial, while common image encoder CLIP (which is adopted by \emph{all} existing works) fails for that purpose since it can not capture the identity feature as accurately as the face ID encoder which has been trained for face identification tasks on the large-scale face dataset. As \cite{bhat2023face} shows, the current best CLIP VIT-L/14 model is still much worse than the face recognition model on top-1 face identification accuracy ($80.95\%$ vs $87.61\%$). Therefore, we employ a ViT backbone \cite{dosovitskiy2020image} pre-trained on a large-scale face recognition dataset to faithfully extract ID-aware features for input face.


\myparagraph{Multi-scale Feature.}  However, naively mapping the final layer's output identity vector $v_{final}$ could only bring sub-optimal identity preservation. The reason lies in that $v_{final}$ mainly contains the high-level semantics which is suitable for discriminative tasks (\eg, face identification) rather than generative tasks. For example, the same identity with different expressions should share similar representation under the face recognition training loss, while the generation requests more detailed information like facial expressions. Hence, only mapping the last layer representation could become an information bottleneck for the image generation task. To deal with the above problem, we propose to utilize multi-scale features from the face encoder to represent an identity more faithfully. Specifically, the identity vector is augmented by four CLS embeddings ($v_3$, $v_6$, $v_{12}$, $v_{12}$) from the 3rd, 6th, 9th, and 12th layer, respectively. Formally, the multi-scale feature from the ID encoder is depicted as follows:
\begin{equation}
% \setlength\abovedisplayskip{1pt}
% \setlength\belowdisplayskip{1pt}
V = [v_3, v_6, v_9, v_{12}, v_{final}].
\end{equation}

\myparagraph{Multi-word Embeddings.} The multi-scale feature is further projected into the word embedding domain. To maintain the original large-scale T2I model's generalization and editability, we leave all its parameters and structure unchanged. As a result, it raises the problem that a single word embedding is hard to faithfully represent the face's identity. Therefore, we further propose a multi-word projection mechanism to represent a face with multi-word embedding:
\begin{equation}
\begin{aligned}
s_{i} = MLP_i(V), \text{for } i = 1, ..., k,
\end{aligned}
\end{equation}
where $k$ is the number of embeddings . Experimentally, we set $k=2$ as depicted in Fig.\ref{fig:pipeline}.  Following \cite{gal2023designing}, $l_2$ regularization is further adapted to constrain the output embedding:

\begin{equation}
    \mathcal{L}_{reg} = \sum_{i=1}^k{\lVert{s_{i}}\rVert}.
\end{equation}

Benefiting from the above-dedicated ID feature, we can facilitate highly identity-preservation control in the embedding space only, without sacrificing pre-trained T2I model's editability caused by feature injection. 

\subsection{Self-Augmented Editability  
 Learning}
Current efficient methods are trained under the reconstruction paradigm, which is given an input face image $I$, the objective to learn a unique word $S*$ such that the $S*$ can reconstruct $I$. However, in real-world applications, we wish to generate a set of new images, such as "watercolor style of $S*$ face", "$S*$ as a police". As a result, there exists a huge inconsistency between training and testing. We hope we can rely on the inductive bias in the word embedding space to achieve editability, but in reality, as Fig.\ref{fig:exp_self_aug} shows, the generated image doesn't always follow the text prompt if we only train encoder under the reconstruction objective. 

To deal with the inconsistency between training and testing, in this paper, we propose a novel \emph{self-augmented editability learning} to take the editing task into the training phase. However, collecting such pair data for the editing task is difficult. Experimentally,  we notice that the current state-of-the-art general text-to-image models can generate celebrity (\eg, Boris Johnson, Emma Watson) in different contexts with good identity preservation and text-coherence. With this insight, The \emph{self-augmented editability learning} utilizes the pre-trained model itself to construct a self-augmented dataset by generating various celebrity faces along with the target edited celebrity images, which will be used to train the $M^2$ ID encoder with the editability objective. Formally, the construction of the dataset includes the following four steps:


\myparagraph {Step 1: Celebrity List Generation.} Firstly we collect a candidate celebrity list. The large language model (\ie, ChatGPT) is used to generate the most famous 400 names in four fields (\ie, sports players, singers, actors, and politicians). After filtering duplicate ones, we finally get 1015 celebrity names.

\myparagraph {Step 2: Celebrity Face Generation.} We propose to use generated face images rather than real images because the model has its own understanding of celebrity. Specifically, the celebrities who appeared less frequently in the Stable Diffusion training dataset are not very similar to the real person while these generated faces maintain a high level of identity resemblance. We use the prompt template "<celebrity-name> face, looking at the camera" to produce the source images, then followed by face crop and alignment operation to get face-only images. A face-only image is kept if its short size is larger than 128 pixels. 

\myparagraph {Step 3: Edit Prompts and Edited Images Generation.} We manually design a variety of prompts that contain images of celebrities in different jobs, styles, and accessories (\eg, "<celebrity-name> as a chef", "oil painting style, <celebrity-name> face"). Then these prompts are transformed to images by the T2I model as edited images, and the <celebrity-name> in prompts is replaced by the pseudo word $S^*$ as Editing Prompts.

\myparagraph {Step 4: Data Cleaning.} After the above procedures, we can now get the initial self-augmented dataset consisting of a set of triplets, <identity face, editing prompt, edited images>. Due to the instability of the current diffusion model, the edited images don't always follow the edit instructions. 
Therefore, we need to filter out the noise data in the self-augmented dataset. We employ ID Loss and CLIP score which reflect identity similarity and text-image consistency as the metrics to rank the edited images for every prompt, then the top $25\%$ triplets at kept as the final training set. 

Finally, we construct a high-quality self-augmented dataset from the pre-trained T2I model itself, which is then used for edit-oriented training.


\subsection{Training}
We combine the FFHQ \cite{karras2019style} and the self-augmented dataset to train our proposed $M^2$ ID encoder. The total loss consists of noise prediction loss of diffusion and the embedding regularization loss:  
\begin{equation}
\mathcal{L}_{total} = \mathcal{L}_{diffusion} + \lambda \mathcal{L}_{reg} ,
\end{equation} 
where $\lambda$ is the embedding regularization weight.



We include a diverse range of 13 autoregressive LLMs (\tableref{table:models}) from 4 model families of sizes ranging from 1.1 billion to 20 billion parameters in our evaluation suite. We choose contemporary models that span different architectures and training paradigms which are known to show good ICL performance. This diversity bolsters the generalizability of insights obtained using our evaluation suite while mitigating potential bias towards any specific model family. Moreover, we select open-source models which are large enough to show non-trivial ICL performance while still being small enough to run on reasonable consumer hardware to ensure the practical significance of our findings.
