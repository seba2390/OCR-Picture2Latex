\section{Evaluation Suite}
%We design a comprehensive evaluation suite to assess the performance of instructions along 5 metrics relevant to practical ICL. To ensure a fair analysis, we test each instruction selection method across 9 tasks and 13 autoregressive models from 4 model families.

\subsection{Prompt format}

\begin{figure}[t]
    \centering
    %\begin{tikzpicture}[
  node distance = 0.2cm and 0.2cm,
  single node/.style = {rectangle, fill=#1, fill opacity=0.5, text opacity=1, align=left, font=\fontfamily{FiraSans-LF}\selectfont\footnotesize, text width=0.40\textwidth, inner xsep=3mm, inner ysep=3mm, rounded corners=1mm},
]

\node[single node=lightyellow] (instruction) 
{\normalFont{Is this review positive or negative?}};

\node[single node=lightblue, below=of instruction] (example1)
{\normalFont{Review:} \ultraLight{Whoever wrote the screenplay for this movie obviously never consulted…} \\ \normalFont{Sentiment:} \ultraLight{Negative}};

\node[single node=lightblue, below=of example1] (example2)
{\normalFont{Review:} \ultraLight{The story centers around Barry McKenzie who must go to England…} \\ \normalFont{Sentiment:} \ultraLight{Positive}};

\node[single node=lightblue, below=of example2] (example3)
{\normalFont{Review:} \ultraLight{This film is just plain horrible. John Ritter doing pratt falls, 75\% of the actors…} \\ \normalFont{Sentiment:} \ultraLight{Negative}};


\node[single node=lightpink, below=of example3] (test) 
{\normalFont{Review:} \ultraLight{BLACK WATER has to be one of the best Australian movies I've seen in many…} \\ \normalFont{Sentiment:} };

\draw [black, line width=0.5pt, rounded corners=1mm] ([shift={(-2mm,-2mm)}]current bounding box.south west) rectangle ([shift={(2mm,2mm)}]current bounding box.north east);

\end{tikzpicture}
    \includegraphics[width=\linewidth]{figures/prompt_example.png}
    \caption{
        An example of a prompt following the template we use for IMDB. By  `prompt' we refer to the concatenation of the \textcolor{darkyellow}{instruction}, solved \textcolor{blue}{demonstrations} and an unsolved \textcolor{red}{test example}.
    }
    \label{fig:prompt_example}
\end{figure}
We define a `prompt' as the full textual input provided to an LLM. Our evaluation suite supports the use of any number of demonstrations, arbitrary demonstration templates and the inclusion of custom strings anywhere within the prompt. Since the instructions used can be set to any arbitrary strings, users are free to use any external means to select instructions and have them evaluated by our suite.

For consistency, we conduct all experiments in this work using prompts that begin with an instruction, continue with a sequence of annotated training demonstrations, and conclude with an unsolved test example\footnote{Instructions are omitted during `Null instruction' evaluations. Demonstrations are omitted in zero-shot evaluations.} (Figure~\ref{fig:prompt_example}), and express each example in a minimal, task-specific key-value format (Table~\ref{table:templates}) that reflects task semantics.

\subsection{Metrics}
\begin{figure*}
  \begin{subfigure}[b]{0.32\textwidth}
    \centering
    \resizebox{\linewidth}{!}{
    \iffalse

\begin{tikzpicture}[
  node distance = 0.2cm and 0.2cm,
  single node/.style = {rectangle, fill=#1, text opacity=1, opacity=0.5, align=center, font=\footnotesize, text width=2cm, inner sep=1.5mm, rounded corners=0.5mm},
  bigbox/.style = {draw, thick, rounded corners, rectangle},
  arrow/.style={thick, -latex},
]

\node[single node=lightyellow] (instruction) {instruction};

\node[single node=lightblue, right=of instruction] (example1) {$(x_1^{tr}, y_1^{tr})$};
\node[single node=lightblue, right=of example1] (example2) {$(x_2^{tr}, y_2^{tr})$};
\node[single node=lightblue, right=of example2] (example3) {$(x_3^{tr}, y_3^{tr})$};

\node[single node=lightpink, right=of example3] (test) {$x_1^{test}$};

\node[bigbox, fit=(instruction) (example1) (example2) (example3) (test)] (box1) {};

\node[single node=lightyellow, below=1cm of instruction] (instruction2) {instruction};

\node[single node=lightblue, right=of instruction2] (example12) {$(x_1^{tr}, y_1^{tr})$};
\node[single node=lightblue, right=of example12] (example22) {$(x_2^{tr}, y_2^{tr})$};
\node[single node=lightblue, right=of example22] (example32) {$(x_3^{tr}, y_3^{tr})$};

\node[single node=lightpink, right=of example32] (test2) {\textsc{Perturb}$(x_1^{test})$};

\node[bigbox, fit=(instruction2) (example12) (example22) (example32) (test2)] (box2) {};

\draw [arrow] (box1.south) -- (box2.north);

\end{tikzpicture}

\fi


\begin{tikzpicture}[
  node distance = 0.2cm and 1cm, % Increase horizontal distance between nodes
  single node/.style = {rectangle, fill=#1, text opacity=1, opacity=0.5, align=center, font=\footnotesize, text width=2cm, inner sep=1.5mm, rounded corners=0.5mm},
  bigbox/.style = {draw, thick, rounded corners, rectangle, inner sep=1.5mm}, % Reduce inner sep for less padding
  arrow/.style={thick, -latex},
]

\node[single node=lightyellow] (instruction1) {instruction};
\node[single node=lightblue, below=of instruction1] (example1) {$(x_1^{tr}, y_1^{tr})$};
\node[single node=lightblue, below=of example1] (example2) {$(x_2^{tr}, y_2^{tr})$};
\node[single node=lightblue, below=of example2] (example3) {$(x_3^{tr}, y_3^{tr})$};
\node[single node=lightpink, below=of example3] (test1) {$x_1^{te}$};

\node[bigbox, fit=(instruction1) (example1) (example2) (example3) (test1)] (box1) {};

\node[single node=lightyellow, right=of instruction1] (instruction2) {instruction};
\node[single node=lightblue, below=of instruction2] (example12) {$(x_1^{tr}, y_1^{tr})$};
\node[single node=lightblue, below=of example12] (example22) {$(x_2^{tr}, y_2^{tr})$};
\node[single node=lightblue, below=of example22] (example32) {$(x_3^{tr}, y_3^{tr})$};
\node[single node=lightpink, below=of example32] (test2) {\textsc{Perturb}$(x_1^{te})$};

\node[bigbox, fit=(instruction2) (example12) (example22) (example32) (test2)] (box2) {};

\draw [arrow] (box1.east) -- (box2.west);
\end{tikzpicture}

    }
    \caption{Perturbation accuracy}
  \end{subfigure}\hfill
  \begin{subfigure}[b]{0.32\textwidth}
    \centering
    \resizebox{\linewidth}{!}{
    \begin{tikzpicture}[
  node distance = 0.2cm and 0.2cm,
  single node/.style = {rectangle, fill=#1, text opacity=1, opacity=0.5, align=center, font=\footnotesize, text width=2cm, inner sep=1.5mm, rounded corners=0.5mm},
  bigbox/.style = {draw, thick, rounded corners, rectangle},
  arrow/.style={thick, -latex},
  bigbox opaque/.style = {draw, thick, rounded corners, rectangle, opacity=1, fill=white},
  arrow/.style={thick, -latex},
]

\node[single node=lightyellow] (instruction) {instruction};

\node[single node=lightblue, below=of instruction] (example1) {$(x_1^{tr}, y_1^{tr})$};
\node[single node=lightblue, below=of example1] (example2) {$(x_2^{tr}, y_2^{tr})$};
\node[single node=lightblue, below=of example2] (example3) {$(x_3^{tr}, y_3^{tr})$};

\node[single node=lightpink, below=of example3] (test) {$x_1^{te}$};

\node[bigbox, fit=(instruction) (example1) (example2) (example3) (test)] (box1) {};

\begin{pgfonlayer}{layer1}
\node[single node=lightyellow, right=1cm of instruction] (instruction2) {instruction};

\node[single node=lightblue, below=of instruction2] (example12) {$(x_4^{tr}, y_4^{tr})$};
\node[single node=lightblue, below=of example12] (example22) {$(x_5^{tr}, y_5^{tr})$};
\node[single node=lightblue, below=of example22] (example32) {$(x_6^{tr}, y_6^{tr})$};

\node[single node=lightpink, below=of example32] (test2) {$x_1^{te}$};

\node[bigbox, fit=(instruction2) (example12) (example22) (example32) (test2)] (box2) {};
\end{pgfonlayer}

\begin{pgfonlayer}{layer2}
\node[bigbox opaque, fit=(instruction2) (example12) (example22) (example32) (test2)] (box2) {};
\end{pgfonlayer}

\begin{pgfonlayer}{layer3}
\node[single node=lightyellow, below=-0.25cm of instruction2, xshift=0.25cm] (instruction3) {instruction};

\node[single node=lightblue, below=of instruction3] (example13) {$(x_1^{tr}, y_1^{tr})$};
\node[single node=lightblue, below=of example13] (example23) {$(x_2^{tr}, y_2^{tr})$};
\node[single node=lightblue, below=of example23] (example33) {$(x_3^{tr}, y_3^{tr})$};

\node[single node=lightpink, below=of example33] (test3) {$x_1^{test}$};

\node[bigbox, fit=(instruction3) (example13) (example23) (example33) (test3)] (box3) {};
\end{pgfonlayer}

\begin{pgfonlayer}{layer4}
\node[bigbox opaque, fit=(instruction3) (example13) (example23) (example33) (test3)] (box3) {};
\end{pgfonlayer}

\begin{pgfonlayer}{layer5}
\node[single node=lightyellow, below=-0.25cm of instruction3, xshift=0.25cm] (instruction4) {instruction};

\node[single node=lightblue, below=of instruction4] (example14) {$(x_1^{tr}, y_1^{tr})$};
\node[single node=lightblue, below=of example14] (example24) {$(x_2^{tr}, y_2^{tr})$};
\node[single node=lightblue, below=of example24] (example34) {$(x_3^{tr}, y_3^{tr})$};

\node[single node=lightpink, below=of example34] (test4) {$x_1^{test}$};

\node[bigbox, fit=(instruction4) (example14) (example24) (example34) (test4)] (box4) {};
\end{pgfonlayer}

\begin{pgfonlayer}{layer6}
\node[bigbox, fit=(instruction4) (example14) (example24) (example34) (test4)] (box4) {};
\end{pgfonlayer}

\draw [arrow] (box1.east) -- (box2.west);

\end{tikzpicture}

    }
    \caption{Selectional sensitivity}
  \end{subfigure}\hfill
  \begin{subfigure}[b]{0.32\textwidth}
    \centering
    \resizebox{\linewidth}{!}{
    \iffalse
\begin{tikzpicture}[
  node distance = 0.2cm and 0.2cm,
  single node/.style = {rectangle, fill=#1, text opacity=1, opacity=0.5, align=center, font=\footnotesize, text width=2cm, inner sep=1.5mm, rounded corners=0.5mm},
  bigbox/.style = {draw, thick, rounded corners, rectangle},
  arrow/.style={thick, -latex},
]

\node[single node=lightyellow] (instruction) {instruction};

\node[single node=lightblue, right=of instruction] (example1) {$(x_1^{tr}, y_1^{tr})$};
\node[single node=lightblue, right=of example1] (example2) {$(x_2^{tr}, y_2^{tr})$};
\node[single node=lightblue, right=of example2] (example3) {$(x_3^{tr}, y_3^{tr})$};

\node[single node=lightpink, right=of example3] (test) {$x_1^{test}$};

\node[bigbox, fit=(instruction) (example1) (example2) (example3) (test)] (box1) {};

\node[single node=lightyellow, below=1cm of instruction] (instruction2) {instruction};

\node[single node=lightblue, right=of instruction2] (example12) {$(x_2^{tr}, y_2^{tr})$};
\node[single node=lightblue, right=of example12] (example22) {$(x_1^{tr}, y_1^{tr})$};
\node[single node=lightblue, right=of example22] (example32) {$(x_3^{tr}, y_3^{tr})$};

\node[single node=lightpink, right=of example32] (test2) {$x_1^{test}$};

\node[bigbox, fit=(instruction2) (example12) (example22) (example32) (test2)] (box2) {};

\draw [arrow] (box1.south) -- (box2.north);

\node[single node=lightyellow, below=0.5cm of instruction2] (instruction3) {instruction};

\node[single node=lightblue, right=of instruction3] (example13) {$(x_3^{tr}, y_3^{tr})$};
\node[single node=lightblue, right=of example13] (example23) {$(x_2^{tr}, y_2^{tr})$};
\node[single node=lightblue, right=of example23] (example33) {$(x_1^{tr}, y_1^{tr})$};

\node[single node=lightpink, right=of example33] (test3) {$x_1^{test}$};

\node[bigbox, fit=(instruction3) (example13) (example23) (example33) (test3)] (box3) {};


\end{tikzpicture}
\fi



\begin{tikzpicture}[
  node distance = 0.2cm and 0.2cm,
  single node/.style = {rectangle, fill=#1, text opacity=1, opacity=0.5, align=center, font=\footnotesize, text width=2cm, inner sep=1.5mm, rounded corners=0.5mm},
  bigbox/.style = {draw, thick, rounded corners, rectangle},
  arrow/.style={thick, -latex},
  bigbox opaque/.style = {draw, thick, rounded corners, rectangle, opacity=1, fill=white},
  arrow/.style={thick, -latex},
]

\node[single node=lightyellow] (instruction) {instruction};

\node[single node=lightblue, below=of instruction] (example1) {$(x_1^{tr}, y_1^{tr})$};
\node[single node=lightblue, below=of example1] (example2) {$(x_2^{tr}, y_2^{tr})$};
\node[single node=lightblue, below=of example2] (example3) {$(x_3^{tr}, y_3^{tr})$};

\node[single node=lightpink, below=of example3] (test) {$x_1^{te}$};

\node[bigbox, fit=(instruction) (example1) (example2) (example3) (test)] (box1) {};

\begin{pgfonlayer}{layer1}
\node[single node=lightyellow, right=1cm of instruction] (instruction2) {instruction};

\node[single node=lightblue, below=of instruction2] (example12) {$(x_3^{tr}, y_3^{tr})$};
\node[single node=lightblue, below=of example12] (example22) {$(x_1^{tr}, y_1^{tr})$};
\node[single node=lightblue, below=of example22] (example32) {$(x_2^{tr}, y_2^{tr})$};

\node[single node=lightpink, below=of example32] (test2) {$x_1^{te}$};

\node[bigbox, fit=(instruction2) (example12) (example22) (example32) (test2)] (box2) {};
\end{pgfonlayer}

\begin{pgfonlayer}{layer2}
\node[bigbox opaque, fit=(instruction2) (example12) (example22) (example32) (test2)] (box2) {};
\end{pgfonlayer}

\begin{pgfonlayer}{layer3}
\node[single node=lightyellow, below=-0.25cm of instruction2, xshift=0.25cm] (instruction3) {instruction};

\node[single node=lightblue, below=of instruction3] (example13) {$(x_1^{tr}, y_1^{tr})$};
\node[single node=lightblue, below=of example13] (example23) {$(x_2^{tr}, y_2^{tr})$};
\node[single node=lightblue, below=of example23] (example33) {$(x_3^{tr}, y_3^{tr})$};

\node[single node=lightpink, below=of example33] (test3) {$x_1^{test}$};

\node[bigbox, fit=(instruction3) (example13) (example23) (example33) (test3)] (box3) {};
\end{pgfonlayer}

\begin{pgfonlayer}{layer4}
\node[bigbox opaque, fit=(instruction3) (example13) (example23) (example33) (test3)] (box3) {};
\end{pgfonlayer}

\begin{pgfonlayer}{layer5}
\node[single node=lightyellow, below=-0.25cm of instruction3, xshift=0.25cm] (instruction4) {instruction};

\node[single node=lightblue, below=of instruction4] (example14) {$(x_1^{tr}, y_1^{tr})$};
\node[single node=lightblue, below=of example14] (example24) {$(x_2^{tr}, y_2^{tr})$};
\node[single node=lightblue, below=of example24] (example34) {$(x_3^{tr}, y_3^{tr})$};

\node[single node=lightpink, below=of example34] (test4) {$x_1^{test}$};

\node[bigbox, fit=(instruction4) (example14) (example24) (example34) (test4)] (box4) {};
\end{pgfonlayer}

\begin{pgfonlayer}{layer6}
\node[bigbox, fit=(instruction4) (example14) (example24) (example34) (test4)] (box4) {};
\end{pgfonlayer}

\draw [arrow] (box1.east) -- (box2.west);

\end{tikzpicture}
    }
    \caption{Permutational sensitivity}
  \end{subfigure}
  \caption{
        We provide schematic diagrams that show prompts are modified to measure \textit{perturbation accuracy}, \textit{selectional sensitivity} and \textit{permutational sensitivity}. We perturb the test input to measure perturbation accuracy, and demonstration selection and permutation respectively while measuring selectional and permutational sensitivity.
    }
    \label{fig:sensitivity-metrics}
\end{figure*}

\label{sec:metrics}
\paragraph{Accuracy metrics} Accuracy is typically the primary metric of interest in ICL. While ICL is most commonly performed in few-shot settings where a handful of annotated demonstrations are included in the prompt, models are also prompted zero-shot without the use of such demonstrations. Since real-world scenarios can often contain grammatical errors and misspellings in the test input, it is desirable to find prompts robust to these perturbations. Hence, we measure \textit{zero-shot accuracy}, \textit{few-shot accuracy}, and \textit{perturbation accuracy}\footnote{We choose to treat this as an accuracy metric rather than a sensitivity metric since it is not meaningful to measure sensitivity to such perturbations in situations where a prompt only elicits near random-chance task performance from a model.} in our evaluations. Following \citet{helm}, we measure perturbation accuracy by introducing random capitalization, spacing, contractions and common misspellings in the test input.

\paragraph{Sensitivity metrics} Previous work has shown that the accuracy obtained using a prompt template can fluctuate significantly as a function of the set of demonstrations included in the prompt ~\cite{Liu2021WhatMG, Su2022SelectiveAM, rubin2022learning, Wang2023LargeLM} and the order they are presented in \cite{fantasticallyorderedprompts}. It may be desirable in practice to identify prompt templates and instructions that offer consistent performance regardless of the choice of demonstrations and their arrangement. Hence, we introduce \textit{selectional sensitivity} and \textit{permutational sensitivity} metrics to measure the sensitivity of chosen instructions respectively to selected demonstrations, and the order in which they are arranged. We quantify the sensitivity of an instruction (given a model and task) using the standard deviation of accuracies obtained on varying  the selection or permutation of the demonstrations used, each across 16 random choices.

\subsection{Aggregating metrics across Models}

Each instruction selection method being tested across $N$ models and $M$ datasets yields $NM$ values per metric. Comparing these $NM$-dimensional vectors directly is complex. It can be challenging to reduce them to a single representative scalar. Simple approaches such as computing the mean of these $NM$ values can prove inadequate since the resulting scores would tend to be heavily influenced by metric values that exhibit a high variance across different inspected methods. 

We opt against using aggregation techniques used by previous works \cite{helm, bigbench} due to their drawbacks (Section~\ref{app:scoring}) and instead adopt `mean relative gain' as a means to aggregate accuracy metrics across multiple models. We rely on simple averaging for sensitivity metrics, partly because we observe that these quantities do not show much variation across methods.

\subsubsection{Accuracy metrics}
Considering the range of models and datasets in our evaluation suite, we unsurprisingly observe substantial variation in accuracy magnitudes across model scales and tasks. However, we notice that the degree of variation in accuracy due to instruction choice is usually considerably smaller than the degree of variation due to model and task choice.

To meaningfully compare and aggregate the relative performance of different instruction selection methods across models, we use a measure called \textit{mean relative gain}. First, we define the \textit{relative gain} for a value $x$ from a population $P$ as the percentage by which $x$ exceeds the mean value of $P$:

$$\text{r-gain}_P(x) = 100 \times \dfrac{x-\mu_P}{\mu_P}$$

Consider a collection of models $\mathcal{M}$ and instructions $\mathcal{I}$ for a task $t$. Given a model $m$, we calculate the raw accuracy scores $s_{tmi}$ for each instruction $i \in \mathcal{I}$. Taking this set $S_{tm}$ to be the population, we compare the performances of the instructions against each other by computing their corresponding relative gains $r_{tmi} = \text{r-gain}_{S_{tm}}(s_{tmi})$. Each $r_{tmi}$ represents the degree by which method $i$ outperforms the average performance along the metric on task $t$ for model $m$.

We now define the mean relative gain as 
$$\overline{r}_{ti} = \dfrac{1}{|\mathcal{M}|} \sum_{m \in \mathcal{M}} r_{tmi}$$

These $\overline{r}_{ti}$ values, tabulated and analyzed in \secref{sec:results}, capture not only the ordinal information about each method's performance on a given task but also provide an intuitive sense of the magnitude by which these methods outperform others. Specifically, if an induction method $i$ has a mean relative gain $\overline{r}_{ti}$ on task $t$, this means that method $i$ exceeds average performance (across $\mathcal{I}$) on task $t$ by $\overline{r}_{ti}$ percent when averaged across models $\mathcal{M}$. 

\subsubsection{Sensitivity metrics}
To aggregate the sensitivity of an instruction selection/induction method $i$ over all models for a task $t$, we simply compute the average of the raw sensitivity scores (described in \secref{sec:metrics}). Specifically, if $\sigma_{tmi}$ is the raw sensitivity score obtained for model $m$ and task $t$ when using instruction $i$, then the aggregated sensitivity score $\overline{\sigma}_{ti}$ is given by 

$$\overline{\sigma}_{ti} = \dfrac{1}{|\mathcal{M}|} \sum_{m \in \mathcal{M}} \sigma_{tmi}$$

We choose to avoid more sophisticated aggregation strategies like relative gain for sensitivity metrics since standard deviations are already secondary metrics making it unintuitive to discuss the relative gain of the standard deviation obtained using a method over the average. %\mz{Do we still average the stds? It seems that in table 5 we didn't do it?}

\iffalse
To fairly compare instruction induction methods across various models and tasks, we adopt mean z-scores as a measure of relative performance along a specific metric. This choice allows us to capture the magnitude of variation in the underlying metric without requiring expert knowledge about its specific range of variation.~\footnote{Please refer to Appendix \ref{app:scoring} for more detailed information on why we opted not to use head-to-head win rates from HELM~\cite{helm} and normalized accuracy from BigBench~\cite{bigbench} as alternative scoring methods.}

A z-score measures the relationship between a value and the population it is drawn from. It quantifies how many standard deviations a value $x$ is away from the mean of the population $P$:
$$\text{z-score}_{P}(x) = \dfrac{x - \mu_P}{\sigma_P},$$
Let $s_{pmt}$ represent the raw scores of each instruction induction method $p$ for model $m$ and task $t$. We denote the subset of these metric scores associated with each pair $(m,t)$ as $$S(m,t) = \{s_{pmt}: p \in \mathcal{P}\}.$$ We compute the z-score for metric scores within each $S(m, t)$ and group them by a method $p$:
$$Z_p = \{\text{z-score}_{S(m,t)}(s_{pmt}) : (m,t) \in \mathcal{M} \times \mathcal{T}\}.$$
We then take an average and derive the mean z-score for $p$ across tasks and models:
$$\overline{z}_p = \sum_{z_p \in Z_p} z_p / |Z_p|$$

These mean z-score values are indicative of the degree to which each instruction induction method outperforms all other method under consideration along a certain metric, aggregated over all possible choices of models in $\mathcal{M}$ and tasks in $\mathcal{T}$.  Although they do communicate the magnitude by which certain methods outperform others, they are indeed bound by the limitation that they have to be recomputed for all $p$ whenever a new method $p'$ is inserted into $\mathcal{P}$. However, this limitation is shown by scoring systems used in previous works \cite{helm} as well and is not a cause for concern in practice when evaluating the relative performances of instruction induction methods.
\fi


\subsection{Tasks}
\begin{table}[t]
\centering
\resizebox{\linewidth}{!}{
\begin{tabular}{ll}
\toprule
\textbf{Task Type} & \textbf{Tasks} \\ 
\midrule
\multirow{5}{*}{Classification (CLS)}      & AG News \cite{agnews} \\ 
                                            & ANLI \cite{anli} \\
                                            & BoolQ \cite{boolq} \\ 
                                            & IMDB \cite{imdb} \\ 
                                            & TweetEval Emotion \cite{emotion} \\
\midrule                                            
\multirow{2}{*}{Multiple-choice (MCQ)}                      & CosmosQA \cite{cosmosqa} \\ 
& HellaSwag \cite{hellaswag} \\
\midrule
\multirow{2}{*}{Generative QA (GQA)}                         & NQ-Open \cite{nqopen} \\ 
& TriviaQA \cite{triviaqa} \\ \bottomrule
\end{tabular}}
\caption{Tasks included in our evaluation suite.}
\label{table:tasks}
\end{table} 
While previous instruction induction \cite{ape, rlprompt} work has tended to focus mostly on classification tasks, we include 9 tasks (\tableref{table:tasks}) in our evaluation suite spanning classification (CLS), multiple-choice question-answering (MCQ) and generative question-answering (GQA) to assess the applicability of instruction selection and induction methods to other task-types as well. We concentrate on tasks that are challenging to contemporary language models, and yet are not so demanding that the performance of these models does not exceed random chance. We exclude certain generative tasks, like summarization, which are challenging to assess objectively.~\footnote{Standard summarization metrics correlate poorly with human preferences~\cite{helm, goyal2023news}.}

\section{Model and Optimization}
\vspace{-2mm}
\label{sec:model}
\setlength{\textfloatsep}{\textfloatsepsave}
We use the {\em attentional neural \encdec} model \cite{2014arXiv1409.0473B} as a basis for both \proposed and \mle.
The model takes (possibly ungrammatical) source sentences $x \in X$ as an input, and predicts grammatical and fluent output sentences $y \in Y$ according to the model parameter $\theta$.
The model consists of two sub-modules, {\em encoder} and {\em decoder}. 
The encoder transforms $x$ into a sequence of vector representations (hidden states) using a bidirectional gated recurrent neural network (GRU) \cite{2014arXiv1412.3555C}.
The decoder predicts a word $y_t$ at a time, using previous token $y_{t-1}$ and linear combination of encoder information as attention.
%With the encoder representation, the decoder predicts a word $y_t$ with previous information at each time step.
%% MJP: I think you could just delete the rest of the subsection (the equation and next two sentences)
%\todo{definition of $\theta$}
%Thus, the entire \encdec model is formalized as follows:
%\begin{eqnarray}
%p(y|x;\theta) = \sum_{t=1}^{T}p(y_{t}|x, y_1^{t-1};\theta)
%\end{eqnarray}
%
%For more technical details such as attention mechanism, refer to \newcite{2014arXiv1409.0473B}.
%\todo{or should I describe it in the Appendix?}

%%%%%%%%%%%%%%%%%%%%%%%%%%%%%%%
\vspace{-1mm}
\subsection{Maximum Likelihood Estimation}
\vspace{-1mm}
\label{sec:mle}
Maximum Likelihood Estimation training (\mle) is a standard optimization method for \encdec models.
In \mle, the objective is to maximize the log likelihood of the correct sequence for a given sequence for the entire training data.
\begin{eqnarray}
L(\theta) = \sum_{\langle X,Y \rangle} \sum_{t=1}^{T}\log p(y_{t}|x, y_1^{t-1};\theta)
\end{eqnarray}
The gradient of $L(\theta)$ is as follows:
\begin{eqnarray}
\pderivl = \sum_{\langle X,Y \rangle} \sum_{t=1}^{T} \frac{\nabla p(y_t|x, y_1^{t-1};\theta)}{p(y_t|x,y_1^{t-1};\theta)}
\end{eqnarray}

One drawback of \mle is the {\em exposure bias} \cite{2015arXiv151106732R}. 
The decoder predicts a word conditioned on the correct word sequence ($y_{1}^{t-1}$) during training, whereas it does with the predicted word sequence ($\hat{y}_{1}^{t-1}$) at test time. 
Namely, the model is not exposed to the predicted words in training time.
This is problematic, because once the model fails to predict a correct word at test time, it falls off the right track and does not come back to it easily.
%% MJP: this should state that it not exposed to the predicted words at training time, right?
%% KS: Correct. I added the sentence explaining it to make sure.
Furthermore, in most sentence generation tasks, the \mle objective does not necessarily correlate with our final evaluation metrics, such as BLEU \cite{papineni-EtAl:2002:ACL} in machine translation and ROUGE \cite{lin:2004:ACLsummarization} in summarization.
This is because \mle optimizes word level predictions at each time step instead of evaluating sentences as a whole.

\gec is no exception.
It depends on sentence-level evaluation that considers grammaticality and fluency.
For this purpose, it is natural to use \metric \cite{napoles-EtAl:2015:ACL-IJCNLP}, which has been used as a fluency-oriented GEC metric. 
We explain more details of this metric in \S\ref{sec:gleu}.

%To address these issues, we directly optimize the neural \encdec model toward our final evaluation metric for GEC using reinforcement learning.
%%\todo{MJP: what metric? GLEU? You should say right here.}
%For this purpose, it is natural to use GLEU \cite{napoles-EtAl:2015:ACL-IJCNLP}, which a fluency-oriented metric. 
%We explain the details of the metric in \S\ref{sec:gleu}.
%MJP: I think presenting GLEU as *the* GEC metric is misleading --- GLEU is one metric, but it isn't the default the way BLEU is for MT

%%%%%%%%%%%%%%%%%%%%%%%%%%%%%%%
\vspace{-1mm}
\subsection{Neural Reinforcement Learning}
\vspace{-1mm}
\setlength{\abovedisplayskip}{4.0pt} % top margin
\setlength{\belowdisplayskip}{4.0pt} % bottom margin
To address the issues in \mle, we directly optimize the neural \encdec model toward our final objective for GEC using reinforcement learning.
In reinforcement learning, {\em agents} aim to maximize expected {\em rewards} by taking {\em actions} and updating the {\em policy} under a given {\em state}.
In the neural \encdec model, we treat the \encdec as an agent which predicts a word from a fixed vocabulary at each time step (the action), given the hidden states of the neural \encdec representation.
The key difference from \mle is that the reward is not restricted to token-level accuracy. 
Namely, any arbitrary metric is applicable as the reward.\footnote{The reward is given at the end of the decoder output (i.e., delayed reward).}
%As mentioned, we use \metric as the reward to maximize. 
%For GEC, it is natural to use GLEU \cite{napoles-EtAl:2015:ACL-IJCNLP}, which a fluency-oriented metric. 
%We explain the details of the metric in \S\ref{sec:gleu}.
%\footnote{In theory, it is possible to use document-level reward, but we focus sentence-level reward in the paper for the sake of simplicity.} 
%The score is given by the evaluation metric of the task in general.
%(e.g., GLEU score for \gec, BLEU score for Machine Translation).

Since we use \metric as the final evaluation metric, the objective of \proposed is to maximize the expected \metric by learning the model parameter. 
%Formally, the objective in \proposed is defined as follows.
\begin{align}
\label{eq:j}
J(\theta) &= \mathbb{E}[\nrlreward] \nonumber \\
    &= \sum_{\sampledata} \mrtp \nrlreward
\end{align}
where $S(x)$ is a sampling function that produces $k$ samples $\hat{y}_1, ... \hat{y}_k$, $\mrtp$ is a probability of the output sentence, and $\nrlreward$ is the reward for $\hat{y}_k$ given a reference set $y$.
As described in Algorithm \ref{alg:nrl}, given a pair of source sentence and the reference $(x, y)$, \proposed takes $k$ sample outputs ($\hat{y}_1$, ... $\hat{y}_k$) and their probabilities ($p(\hat{y}_1)$, ... $p(\hat{y}_k)$).
Then, the expected reward is computed by multiplying the probability and metric score for each sample $\hat{y}_i$.

In the \encdec, the parameters $\theta$ are updated through back-propagation and the number of parameter updates is determined by the partial derivative of $J(\theta)$, called the {\em policy gradient} \cite{williams1992simple,sutton1999policy} in reinforcement learning:
\begin{align}
\label{eq:partialj}
\pderivj = \alpha \mathbb{E} \left[\nabla \log \mrtpsimple \{\nrlreward - b \}  \right]
\end{align}
where $\alpha$ is a learning rate and $b$ is an arbitrary baseline reward to reduce the variance.
The sample mean reward is often used for $b$ \cite{williams1992simple}, and we follow it in \proposed.
%of gradients.

It is reasonable to compare \proposed to minimum risk training (\mrt) \cite{shen-EtAl:2016:P16-1}.
In fact, \proposed with a {\em negative expected reward} can be regarded as \mrt.
%with mini-batch size being 1,
The gradient of \mrt objective is a special case of {\em policy gradient} in \proposed.
We show mathematical details about the relevance between \proposed and \mrt in the supplemental material.
%\ref{sec:appendix}.

%%%%%%%%%%%%%%%%%%%%%%%%%%%%%%%
%\vspace{-1mm}
\subsection{Reward in Grammatical Error Correction}
\label{sec:gleu}
%In \proposed for the \gec task, we need an appropriate reward function $\nrlreward$.
%% MJP: what is "the fluency-oriented GEC metric"? I suggest deleting this sentence and merging the following paragraph with this one
To capture fluency as well as grammaticality in evaluation on such references, we use \metric as the reward.
\metric has been shown to be more strongly preferred than other GEC metrics by native speakers \cite{TACL800}. 
Similar to BLEU in machine translation, \metric computes $n$-gram precision between the system hypothesis ($H$) and the reference ($R$).
In \metric, however, $n$-grams in source ($S$) are also considered. The precision is penalized when the $n$-gram in $H$ overlaps with the source and not with the reference.
Formally, 
\begin{align}
\text{\metric} &= \text{BP}\cdot \exp \left( \sum_{n=1}^4 \frac{1}{n} \log p'_n \right) \nonumber \\
p'_n &= \frac{ N(H,R) - \left[ N(H,S)-N(H,S,R) \right] }{N(H)} \nonumber
\end{align}
where  $N(A,B,C,...)$ is the number of overlapped $n$-grams among the sets, and BP is the same {\em brevity penalty} as in BLEU.

\subsection{Models}
\section{SYSTEM OVERVIEW}
\begin{figure}
\centering

\def\picScale{0.08}    % define variable for scaling all pictures evenly
\def\colWidth{0.5\linewidth}

\begin{tikzpicture}
\matrix [row sep=0.25cm, column sep=0cm, style={align=center}] (my matrix) at (0,0) %(2,1)
{
\node[style={anchor=center}] (FREEhand) {\includegraphics[width=0.85\linewidth]{figures/FREEhand.pdf}}; %\fill[blue] (0,0) circle (2pt);
\\
\node[style={anchor=center}] (rigid_v_soft) {\includegraphics[width=0.75\linewidth]{figures/FREE_vs_rigid-v8.pdf}}; %\fill[blue] (0,0) circle (2pt);
\\
};
\node[above] (FREEhand) at ($ (FREEhand.south west)  !0.05! (FREEhand.south east) + (0, 0.1)$) {(a)};
\node[below] (a) at ($ (rigid_v_soft.south west) !0.20! (rigid_v_soft.south east) $) {(b)};
\node[below] (b) at ($ (rigid_v_soft.south west) !0.75! (rigid_v_soft.south east) $) {(c)};
\end{tikzpicture}


% \begin{tikzpicture} %[every node/.style={draw=black}]
% % \draw[help lines] (0,0) grid (4,2);
% \matrix [row sep=0cm, column sep=0cm, style={align=center}] (my matrix) at (0,0) %(2,1)
% {
% \node[style={anchor=center}] {\includegraphics[width=\colWidth]{figures/photos/labFREEs3.jpg}}; %\fill[blue] (0,0) circle (2pt)
% &
% \node[style={anchor=center}] {\includegraphics[width=\colWidth, height=160pt]{figures/stewartRender.png}}; %\fill[blue] (0,0) circle (2pt);
% \\
% };

% %\node[style={anchor=center}] at (0,-5) (FREEstate) {\includegraphics[width=0.7\linewidth]{figures/FREEstate_noLabels2.pdf}};

% \end{tikzpicture}

\caption{\revcomment{2.3}{(a) A fiber-reinforced elastomerc enclosure (FREE) is a soft fluid-driven actuator composed of an elastomer tube with fibers wound around it to impose specific deformations under an increase in volume, such as extension and torsion. (b) A linear actuator and motor combined in \emph{series} has the ability to generate 2 dimensional forces at the end effector (shown in red), but is constrained to motions only in the directions of these forces. (b) Three FREEs combined in \emph{parallel} can generate the same 2 dimensional forces at the end effector (shown in red), without imposing kinematic constraints that prohibit motion in other directions (shown in blue).}}

% \caption{A fiber-reinforced elastomeric enclosure (FREE) (top) is a soft fluid-driven actuator composed of an elastomer tube with fibers wound around it to impose deformation in specific directions upon pressurization, such as extension and torsion. \revcomment{2.3}{In this paper we explore the potential of combining multiple FREEs in parallel to generate fully controllable multi-dimensional spacial forces}, such as in a parallel arrangement around a flexible spine element (bottom-left), or a Stewart Platform arrangement (bottom-right).}

\label{fig:overview}
\end{figure}


We now give an overview of our learning framework as illustrated in Figure~\ref{fig:overview}. Our framework splits athletic jumps into two phases: a run-up phase and a jump phase. The {\em take-off state} marks the transition between these two phases, and consists of a time instant midway through the last support phase before becoming airborne. The take-off state is key to our exploration strategy, as it is a strong determinant of the resulting jump strategy. We characterize the take-off state by a feature vector that captures key aspects of the state, such as the net angular velocity and body orientation. This defines a low-dimensional take-off feature space that we can sample in order to explore and evaluate a variety of motion strategies. While random sampling of take-off state features is straightforward, it is computationally impractical as evaluating one sample involves an expensive DRL learning process that takes hours even on modern machines. Therefore, we introduce a sample-efficient Bayesian Diversity Search (BDS) algorithm as a key part of our Stage~1 optimization process.

Given a specific sampled take-off state, we then need to produce an optimized run-up controller and a jump controller that result in the best possible corresponding jumps. This process has several steps. We first train a {\em }run-up controller, using deep reinforcement learning, that imitates a single generic run-up motion capture clip while also targeting the desired take-off state. For simplicity, the run-up controller and its training are not shown in Figure~\ref{fig:overview}. These are discussed in Section~\ref{sec:Experiments-Runup}. The main challenge lies with the synthesis of the actual jump controller which governs the remainder of the motion, and for which we wish to discover strategies without any recourse to known solutions.

The jump controller begins from the take-off state and needs to control the body during take-off, over the bar, and to prepare for landing. This poses a challenging learning problem because of the demanding nature of the task, the sparse fail/success rewards, and the difficulty of also achieving natural human-like movement. We apply two key insights to make this task learnable using deep reinforcement learning. First, we employ an action space defined by a subspace of natural human poses as modeled with a Pose Variational Autoencoder (P-VAE). Given an action parameterized as a target body pose, individual joint torques are then realized using PD-controllers. We additionally allow for regularized {\em offset} PD-targets that are added to the P-VAE targets to enable strong takeoff forces. Second, we employ a curriculum that progressively increases the task difficulty, i.e., the height of the bar, based on current performance.

A diverse set of strategies can already emerge after the Stage 1 BDS optimization. To achieve further strategy variations, we reuse the take-off states of the existing discovered strategies for another stage of optimization. The diversity is explicitly incentivized during this Stage 2 optimization via a novelty reward, which is focused specifically on features of the body pose at the peak height of the jump. As shown in Figure~\ref{fig:overview}, Stage~2 makes use of the same overall DRL learning procedure as in Stage~1, albeit with a slightly different reward structure.



We include a diverse range of 13 autoregressive LLMs (\tableref{table:models}) from 4 model families of sizes ranging from 1.1 billion to 20 billion parameters in our evaluation suite. We choose contemporary models that span different architectures and training paradigms which are known to show good ICL performance. This diversity bolsters the generalizability of insights obtained using our evaluation suite while mitigating potential bias towards any specific model family. Moreover, we select open-source models which are large enough to show non-trivial ICL performance while still being small enough to run on reasonable consumer hardware to ensure the practical significance of our findings.
