%====================================================================%
%                  MORIOND.TEX                                       %
%====================================================================%

\documentclass{moriond}

\usepackage{amssymb}
%\usepackage{wrapfig}
\usepackage{xspace}

\bibliographystyle{unsrt}    
% for BibTeX - sorted numerical labels by order of
% first citation.

% A useful Journal macro
\def\Journal#1#2#3#4{{#1} {\bf #2}, #3 (#4)}

% Some useful journal names
\def\NCA{\em Nuovo Cimento}
\def\NIM{\em Nucl. Instrum. Methods}
\def\NIMA{{\em Nucl. Instrum. Methods} A}
\def\NPB{{\em Nucl. Phys.} B}
\def\PLB{{\em Phys. Lett.}  B}
\def\PRL{\em Phys. Rev. Lett.}
\def\PRD{{\em Phys. Rev.} D}
\def\ZPC{{\em Z. Phys.} C}

% Some other macros used in the sample text
\def\st{\scriptstyle}
\def\sst{\scriptscriptstyle}
\def\mco{\multicolumn}
\def\epp{\epsilon^{\prime}}
\def\vep{\varepsilon}
\def\ra{\rightarrow}
%\def\ppg{\pi^+\pi^-\gamma}
%\def\vp{{\bf p}}
%\def\ko{K^0}
%\def\kb{\bar{K^0}}
%\def\al{\alpha}
%\def\ab{\bar{\alpha}}
\def\be{\begin{equation}}
\def\ee{\end{equation}}
\def\bea{\begin{eqnarray}}
\def\eea{\end{eqnarray}}
%\def\CPbar{\hbox{{\rm CP}\hskip-1.80em{/}}}

\newcommand{\alphas}{\alpha_{\rm s}}
\newcommand{\alphasmZ}{\alphas(\rm m^2_{_{\rm Z}})}
\newcommand{\sqrts}{\sqrt{\rm s}}
\newcommand{\sqrtsnn}{\sqrt{\rm s_{_{\mathrm{NN}}}}}
\newcommand{\lqcd}{\Lambda_{_{\rm QCD}}}
\newcommand{\lqcdms}{\Lambda_{_{{\rm \overline{MS}}}}}
\newcommand{\MSbar}{\overline{\rm MS}}
\newcommand{\pp}{$p$--$p$\ }
\newcommand{\epem}{e^+e^-}
\newcommand{\meff}{m{_{\rm eff}}}
\newcommand{\ximax}{\xi{_{\rm max}}}
\providecommand{\ffbar}{f\overline{f}}
\providecommand{\bbbar}{b\overline{b}}
\providecommand{\ccbar}{c\overline{c}}
\providecommand{\ttbar}{t\overline{t}}
\providecommand{\QQbar}{\rm Q\overline{Q}}

\newcommand{\AFBb}  {A_{_{\textsc{fb}}}^{0,b}}
\newcommand{\AFBbb}  {A_{_{\textsc{fb}}}^{b}}
\newcommand{\AFBobs}  {A_{_{\textsc{fb}}}^{obs,b}}
\newcommand{\weakang}  {\sin^2\theta_{\rm W}}
\newcommand{\weakeff}  {\sin^2\theta_{_{\rm eff}}^f}
\newcommand{\pythia}{{\sc pythia}}
\newcommand{\vincia}{{\sc vincia}}
\newcommand{\jetset}{{\sc jetset}}

\newcommand*{\eg}{e.g.\@\xspace}
\newcommand*{\ie}{i.e.\@\xspace}
\def\ttt#1{\texttt{\small #1}}
\def\mean#1{\ensuremath{\left<#1\right>}}

%temp replacement due to no font
%%%%%%%%%%%%%%%%%%%%%%%%%%%%%%%%%%%%%%%%%%%%%%%%%%
%                                                %
%    BEGINNING OF TEXT                           %
%                                                %
%%%%%%%%%%%%%%%%%%%%%%%%%%%%%%%%%%%%%%%%%%%%%%%%%%

\newcommand{\Photo}{\includegraphics[height=30mm]{dde_pic1.jpg}}

\begin{document}
%\vspace*{4.cm}
\title{Forward-backward $b$-quark asymmetry at the Z pole: QCD uncertainties redux}

%\scrollmode

\author{\underline{David d'Enterria}$^1$, and Cynthia Yan$^{1,2}$}
\address{$^1$ CERN, EP Department, CH-1211 Geneva 23, Switzerland\\
$^2$ Harvey Mudd College, Department of Physics, Claremont CA91711, USA}

\maketitle\abstracts{
The forward-backward asymmetry of $b$-quarks measured at LEP in $\epem$ collisions at the Z pole, 
$\AFBb|^{\rm exp} = 0.0992\pm0.0016$, remains today the electroweak precision observable with the largest 
disagreement (2.8$\sigma$) with the Standard Model theoretical 
prediction, $\AFBb|^{\rm th} = 0.1037\pm0.0008$. The dominant systematic uncertainties 
are due to QCD effects --- $b,c$-quark showering and fragmentation, and $B,D$ meson 
decay models --- which have not been revisited in the last 20 years. We reassess the QCD uncertainties
of the eight original LEP measurements of $\AFBb$, using modern parton shower simulations
based on \pythia~8 and \pythia\,8\,$+$\,\vincia\ with different tunes of soft and collinear radiation 
as well as of hadronization. Our analysis indicates QCD uncertainties, of order $\pm$0.4\% and $\pm$1\%
for the jet-charge and lepton-charge based analyses, that are overall slightly smaller but still 
consistent with the original ones. Using the updated QCD systematic uncertainties, we obtain $\AFBb = 0.0996\pm0.0016$.
}

%%%%%%%%%%%%%%%%%%%%%%%%%%%%%%%%%%%%%%%%%%%%%%%%%%%%%%%%%%%%%%%%%%%%%%%%%%%%%%%%%%%%%%%%%%%%%%%%%
\section{Introduction}

In the Standard Model (SM), the Z boson mediates weak neutral currents between fermions of
the same generation. The Z couples to both left- and right-handed chiral states with different 
strengths depending on weak-isospin and electromagnetic charges. The vector 
and axial-vector Z couplings for a fermion of type $f$ are $g^f_V = (g_L^f + g_R^f) = I_3^f - 2Q^f\weakang$ 
and $g^f_A = (g_L^f -g_R^f) = I_3^f$ respectively, where $I_3$ is the third component of the weak isospin of the fermion, 
$Q^f$ its charge (related to the former via the hypercharge $Y^f$: $Q^f=I_3^f+Y^f/2$), and $\weakang\approx 0.23$ 
is the weak mixing angle %that unifies the electroweak interaction couplings: $g\sin\theta_W=g'\cos\theta_W=e$ 
that controls the $\gamma$--Z mixing and provides a relationship between
the coupling constants of the electroweak theory: $g\sin\theta_W=g'\cos\theta_W=e$.
From the expressions above, the varying strengths of the Z-fermion couplings for the
$(\nu_e,\nu_\mu,\nu_\tau)$, $(e,\mu,\tau)$, $(u,c,t)$, and $(d,s,b)$ lepton/quark groups are explained.
%In the Standard Model (SM), the weak neutral current between fermions is mediated by the 
%Z boson which features mixed weak-isospin and electromagnetic couplings. The Z boson couplings
%to right- and left-handed fermions are given by: $g_L = I_3 - Q\weakang$, $g_R = -Q\weakang$, 
%where $I_3$ is the third component of weak isospin of the fermion, $Q$ its charge, and $\weakang$ 
%the weak mixing angle that unifies the electroweak interaction couplings: $$
The mixed Z vector and axial-vector couplings not only affect the total $\epem\to\ffbar$ cross section but
induce asymmetries in the angular distributions of the final-state fermions produced 
in the process. %$\epem\to\ffbar$ -- beyond those issuing from the incoming $e^\pm$ helicity, and from the polarisation of the produced particles -- 
%\frac{}{}=\frac{}{}\left[(1-\mathcal{P}_e\mathcal{A}_e)(1+cos^2\theta)+\right]
Angular asymmetries in the $\epem\to\ffbar$ final-state are ultimately driven by the fermions' charge $Q$ and the weak mixing angle:
\begin{eqnarray}
\mathcal{A}_f=\frac{(g_{L}^f)^2-(g_{R}^f)^2}{(g_{L}^f)^2+(g_{R}^f)^2}=2\frac{g_{V}^f/g_{A}^f}{1+(g_{V}^f/g_{A}^f)^2}\,,
%\mbox{ with }\;\;\frac{g_{V}^f}{g_{A}^f}=1-\frac{2Q_f}{I_3^f}\sin^2\theta^f_{\text{eff}}=1-4|Q_f|\sin^2\theta^f_{\text{eff}}
\mbox{ with }\;\;\frac{g_{V}^f}{g_{A}^f}=1-4|Q_f|\sin^2\theta^f_{\mbox{eff}}\,.
\end{eqnarray}
Experimentally, forward-backward asymmetries in $\epem\to\ffbar$ are determined from the ratio of the number of forward- (backward-)going
(anti)fermions measured in the hemisphere defined by the direction of the $e^+$ ($e^-$) beams: %, as follows:
\begin{eqnarray}
A_{\rm FB}^f=\frac{N_F-N_B}{N_F+N_B},\;\;\mbox{ where }\;F=\int_0^1 \frac{d\sigma}{d\Omega}d\Omega,\;\;B=\int_{-1}^0 \frac{d\sigma}{d\Omega}d\Omega,\;
\end{eqnarray}
The forward-backward asymmetry of $b$ quarks ($\AFBb$) in the process $\epem\to Z \to\bbbar$ at $\sqrts = m_Z$
is the one most accurately measured among all quarks at LEP, given that $b$-quarks are the easiest jets to identify. The 
value $\AFBb|^{\rm exp} = 0.0992\pm0.0016$, obtained from the combination of eight measurements at $\sqrts = 91.21$--91.26~GeV 
using two different (lepton- and jet-charge based) methods, shows today the largest discrepancy ($2.8\sigma$) 
with respect to the theoretical SM prediction, $\AFBb|^{\rm th} = 0.1037\pm0.0008$
(and so does the value of $\weakang$ derived from them)~\cite{ALEPH:2005ab}. %It is the purpose of this work to 
We reanalyze here the original studies to see if such a discrepancy could be explained by a potential 
underestimation of the associated systematic uncertainties.

% The interaction of $Z$ boson with fermions is given by the vector and axial-vector couplings, which satisfy the following relationship with the effective electroweak mixing angle
% $$\frac{g_{Vf}}{g_{Af}}=1-\frac{2Q_f}{T_3^f}\sin^2\theta^f_{\text{eff}}=1-4|Q_f|\sin^2\theta^f_{\text{eff}}$$
% where $Q_f$ is charge and $T_3^f$ is the third component of weak-isospin. The dependence of differential cross-section on the fermion couplings can be incorporated into \textbf{asymmetry parameters} $\mathcal{A}_f$
% $$\mathcal{A}_f=\frac{g_{Lf}^2-g_{Rf}^2}{g_{Lf}^2+g_{Rf}^2}=\frac{2g_{Vf}g_{Af}}{g_{Vf}^2+g_{Af}^2}=2\frac{g_{Vf}/g_{Af}}{1+(g_{Vf}/g_{Af})^2}$$
% Forward/backward asymmetries of the produced $b$ and $\overline{b}$ quarks are in turn directly related to the relevant asymmetry parameters
% $$A_{FB}^{0,f}=\frac{3}{4}\mathcal{A}_e\mathcal{A}_f$$
% On the other hand, forward backward asymmetry is given by
% $$A_{FB}^f=\frac{N_F-N_B}{N_F+N_B}$$
% where $N_F$ is the number of forward events and $N_B$ is the number of backward events. Here, forward (backward) means the fermion (as opposed to anti-fermion) is produced in the hemisphere defined by the direction of the electron (positron) beam. Combining the above expressions together, we can extract the effective electroweak mixing angle $\sin\theta^f_{\text{eff}}$ by measuring the forward-backward asymmetry.


%%%%%%%%%%%%%%%%%%%%%%%%%%%%%%%%%%%%%%%%%%%%%%%%%%%%%%%%%%%%%%%%%%%%%%%%%%%%%%%%%%%%%%%%%%%%%%%%%
\section{LEP $b$-quark forward-backward asymmetry data}
\label{sec:}

Table~\ref{tab:AFBb} lists the eight $\AFBb$ measurements with the breakdown of their uncertainties.
In four measurements, the original $b$,~$\bar{b}$ quarks are identified from the charge of the leading 
lepton $\ell$ inside each $b$-jet (through the fragmentation $b\to B, b\to c\to D$ and subsequent 
$B,D\to\ell$ decay), whereas in the other four, the $b$ charge is reconstructed from 
the jet constituent particles.
%, four of which are lepton-based (ALEPH-2002~\cite{leptonALEPH}
%with $\rm \pm4.1(stat)\pm1.7\%(syst)$,
% DELPHI-1995~\cite{leptonDELPHI95}, DELPHI-2004~\cite{leptonDELPHI04}, L3-1992~\cite{leptonL392},
%L3-1999~\cite{leptonL399}, OPAL-2003~\cite{leptonOPAL}), and four of which jet-charge-based (ALEPH-2001~\cite{jetqALEPH}, 
%DELPHI-2005~\cite{jetqDELPHI}, L3-1998~\cite{jetqL3}, OPAL-2002~\cite{jetqOPAL}). 
The statistical uncertainties of $\AFBb$ dominate, being twice bigger than the systematic ones, while the 
QCD uncertainties account for about half of the latter (and are assumed to be fully-correlated among measurements). 
The QCD-related biases on $\AFBb$ depend strongly on the experimental selection procedure and are related to: (i) hard gluon radiation, and
(ii) smearing of the $b$-jet (thrust) axis due to $b$ and $(b\to)c$ soft radiation and hadronization,
and subsequent $B$ and $D$ hadron decay models. Whereas the first bias is theoretically well controlled
through next-to-next-to-leading-order perturbative QCD (plus massive $b$-quark) corrections~\cite{Bernreuther:2016ccf}, 
the uncertainties of the latter were estimated using Monte Carlo (MC) parton shower 
simulations~\cite{Abbaneo:1998xt} that have not been revisited in 20 years.
%\renewcommandrraystretch{1.2}% Tabular row height (1.0 is for standard spacing)
\begin{table}[htbp]
\caption[]{LEP measurements of $\AFBb$  and associated statistical, total systematic, and QCD-systematic uncertainties
(with the newly-computed QCD systematics quoted in parentheses).\label{tab:AFBb}}
%\vspace{0.1cm}
%\vspace{0.4cm}
\begin{center}
\tabcolsep=1.1mm
\begin{tabular}{lcccc}\hline
Measurement &  $\AFBb$ &  & uncertainties & \\
            &          &  stat. & total syst. & QCD syst. (new)\\\hline
ALEPH lepton (2002)~\cite{leptonALEPH} & $0.1003 \pm 0.0038 \pm 0.0017$ &  $4.1\%$ & $1.7\%$ & $0.6\%\,(0.8\%)$ \\
DELPHI lepton (2004-5)~\cite{leptonDELPHI95} & $0.1025 \pm 0.0051 \pm 0.0024$ & $6.4\%$ & $2.4\%$ & $1.5\%\,(1.3\%)$ \\ 
L3 lepton (1999)~\cite{leptonL392} & $0.1001 \pm 0.0060 \pm 0.0035$ & $6.9\%$ & $3.4\%$ & $1.8\%\,(0.8\%)$\\
OPAL lepton (2003)~\cite{leptonOPAL}& $0.0977 \pm 0.0038 \pm 0.0018$ & $4.3\%$ & $1.5\%$ & $1.1\%\,(1.4\%)$ \\\hline
ALEPH jet-charge (2001)~\cite{jetqALEPH} & $0.1010 \pm 0.0025 \pm 0.0012$ & $2.7\%$ & $1.1\%$ & $0.5\%\,(0.5\%)$ \\
DELPHI jet-charge (2005)~\cite{jetqDELPHI} & $0.0978 \pm 0.0030 \pm 0.0015$ & $3.3\%$ & $1.5\%$& $0.5\%\,(0.4\%)$\\
L3 jet-charge (1998)~\cite{jetqL3} & $0.0948 \pm 0.0101 \pm 0.0056$ & $10.8\%$ & $5.9\%$& $4.1\%\,(0.4\%)$\\ 
OPAL jet-charge (2002)~\cite{jetqOPAL}& $0.0994 \pm 0.0034 \pm 0.0018 $& $3.7\%$ & $1.8\%$& $1.5\%\,(0.3\%)$\\\hline
\end{tabular}
\end{center}
\end{table}
At future high-luminosity $\epem$ machines, such as the FCC-ee with $10^5$ times more data collected
at the Z pole than at LEP~\cite{FCCee}, statistical uncertainties will be totally negligible, and 
the latter QCD effects will dominate the systematics of the $\AFBb$ measurement. 
%In the next Section we reassess the
%QCD-related uncertainties of $A_{FB}^b$ using modern MC with up-to-date models of parton shower and 
%hadronization.

% \noindent Uncertainties of lepton-based extractions: % of $A_{FB}^b$ ($\pm$stat.$\pm$syst.):
% \begin{itemize}
% \item ALEPH-2002~\cite{leptonALEPH}  $\pm4.1\%\,\mbox{(stat)}\,\pm1.7\%$ (syst., QCD-related: $\pm0.6\%$)
% \item DELPHI-2004~\cite{leptonDELPHI04} $\pm6.4\%\pm2.4\%$ (QCD-related: $\pm1.5\%$)
% DELPHI-1995~\cite{leptonDELPHI95}: $\pm5\%$ (QCD: $\pm2.5\%$)
% \item L3-1999~\cite{leptonL392,leptonL399}  $\pm6.9\%\pm3.4\%$ (QCD-related: $\pm1.8\%$)
% \item OPAL-2003~\cite{leptonOPAL}  $\pm4.3\%\pm1.5\%$ (QCD-related: $\pm1.1\%$)
% \end{itemize}
% Uncertainties of jet-charge-based extraction of $A_{FB}^b$ ($\pm$stat.$\pm$syst.):
% \begin{itemize}
% \item ALEPH-2001~\cite{jetqALEPH}  $\pm2.7\%\pm1.1\%$ (QCD-related: $\pm0.2\%$, $0.5\%$ including flavour prod.)
% \item DELPHI-2005~\cite{jetqDELPHI}  $\pm3.3\%\pm1.5\%$ (QCD-related: $\pm0.5\%$)
% \item L3-1998~\cite{jetqL3}  $\pm10.8\%\pm5.9\%$ (QCD-related: $\pm4.1\%$)
% \item OPAL-2002~\cite{jetqOPAL}  $\pm3.7\%\pm1.8\%$ (QCD-related: $\pm1.5\%$)
% \end{itemize}


%%%%%%%%%%%%%%%%%%%%%%%%%%%%%%%%%%%%%%%%%%%%%%%%%%%%%%%%%%%%%%%%%%%%%%%%%%%%%%%%%%%%%%%%%%%%%%%%%
%\section{QCD uncertainties of the LEP $b$-quark forward-backward asymmetry data}
\section{Simulation of the LEP $b$-quark forward-backward asymmetry measurements}
\label{sec:}

The eight original LEP measurements of $\AFBb$ %(4 lepton-based, and 4 jet-charge-based) 
have been implemented in a MC event simulation based on \pythia~8.226~\cite{pythia8} with seven 
different parton-shower and hadronization tunes, as well as based on two alternative (dipole antenna) shower
approaches from \pythia~8.210 combined with \vincia~1.1 and 2.2 (with uncertainties given by
12 variations of the \vincia\ parameter set)~\cite{vincia}. Ten million $\epem\to Z(\bbbar)$ events are thereby generated
at $\sqrts = 92.4$~GeV with QED radiation on, %(in order to match closely the original experimental conditions), 
and analysed as done in the original experiments. The whole MC setup effectively corresponds 
to nine different modelings of the underlying QCD effects (bottom- and charm-quark gluon radiation and 
fragmentation functions, and $B,D$ semileptonic decays). Tune-7 and \vincia~2.2 include proton-proton 
data whereas all other models are based on LEP data alone.
For all analyses, the $b$-jets are first reconstructed with the JADE algorithm %~\cite{jade} 
from the list of final-state particles and the thrust axis of the event is computed as a proxy of the $\bbbar$ 
direction. Each original $y_{\rm cut}$ and $M_{\rm jet}$ jet selection criteria, and %of %as well as the 
(transverse) momenta ($p_T$) $p$ cuts on the final electron and muons, are applied. On the one hand, %At this stage,
the lepton-based analyses determine the $b$-quark charge from that of the hardest charged lepton in the
event, and then extract $\AFBobs$ by fitting the corresponding distribution of polar angles $\theta$ between 
the $e^-$ and the thrust axis, $dN/d\cos\theta = 3/8\,\;[1+\cos^2\theta+8/3\,\AFBobs(1-2\chi_B)\cos\theta]$, where
$\chi_B\approx 0.12$ is the $B^0\overline{B^0}$ effective mixing parameter.
On the other, in the jet-charge-based analyses, $b,\bar{b}$-quarks are identified via their
measured jet charge $Q_{\rm jet}=\sum p_L^\kappa\,Q/\sum p_L^\kappa$ (where $p_L$
is the longitudinal momentum of the final-state particles, with charge $Q$, with respect
to the thrust axis, and the power $\kappa$ varies between 0.4 and 0.6), and $\AFBobs$ is derived 
by fitting the distribution
$\mean{Q_F-Q_B}/\mean{Q_b-Q_{\bar{b}}}=8/3\,\AFBobs(1+C)\cos\theta/(1+\cos^2\theta)$,
where $Q_F\;(Q_B)$ are the jet charges in the forward (backward) hemisphere, and the
$C$ factor is a $\sim$3.5\% correction for missing higher-order QCD terms and for the 
difference between the thrust axis and the $b$-quark direction~\cite{ALEPH:2005ab,Abbaneo:1998xt}. 
%Examples of the fitted distributions in the MC samples are shown in Fig.~\ref{fig:fits}.

% \begin{figure}[htpb!]
% \centerline{
% \includegraphics[width=0.48\linewidth]{costheta_fits.png}
% \includegraphics[width=0.52\linewidth]{bjet_charge_fits.png}
% }
% \caption[]{Examples of the $\AFBb$ extraction from fits to the angular distributions
% in the lepton-based (left) and jet-charge-based (right) MC analyses.}
% \label{fig:fits}
% \end{figure}

%%%%%%%%%%%%%%%%%%%%%%%%%%%%%%%%%%%%%%%%%%%%%%%%%%%%%%%%%%%%%%%%%%%%%%%%%%%%%%%%%%%%%%%%%%%%%%%%%
\section{Results and conclusions}
\label{sec:}

Through the procedure describe above, we extract 9 different MC values of $\AFBobs$ 
for each one of the eight experimental setups, which we compare among themselves and against 
the experimental data in Fig.~\ref{fig:lepton_AFBobs_vs_MC} and~\ref{fig:jet_AFBobs_vs_MC}
for lepton- and jet-charge analyses. % (for the lepton-based analyses) and Fig.~\ref{fig:jet_AFBobs_vs_MC} (jet-charge based). 
The central $\AFBobs$ values plotted differ slightly 
from the $\AFBb$ values quoted in Table~\ref{tab:AFBb}, since the latter are obtained correcting for radiative 
effects, $\gamma$ exchange, Z-$\gamma$ interference, and shifted to the pole $m_Z = 91.187$~GeV mass. 
The first (leftmost) MC point corresponds to the
\pythia~8 tune-1 result obtained with the 1990 \jetset\ parameter set, very similar to the
one used to obtain the original LEP QCD %parton-shower and hadronization 
uncertainties~\cite{Abbaneo:1998xt}. 
The red band around the MC points is the standard deviation of the predictions, 
which we take as indicative of the associated QCD systematic uncertainty for each measurement. It amounts to
about 1\% (0.4\%) for the lepton (jet) charge-based measurements, and is found to be overall slightly smaller but
still fully consistent with the original QCD uncertainties (last column of Table~\ref{tab:AFBb}).
Using the updated QCD systematics, we obtain~\cite{DdEYan} a new weighted-average $b$-quark 
forward-backward asymmetry, $\AFBb = 0.0996\pm0.0016$, very similar to the current one.

\begin{figure}[htpb!]
%\centerline{
\includegraphics[width=0.47\linewidth,height=4.8cm]{leptonALEPH.pdf}
\includegraphics[width=0.47\linewidth,height=4.8cm]{leptonDELPHI.pdf}\\
\includegraphics[width=0.47\linewidth,height=4.8cm]{leptonL3.pdf}
\includegraphics[width=0.47\linewidth,height=4.8cm]{leptonOPAL.pdf}
%}
\caption[]{$b$-quark forward-backward asymmetry %$\AFBobs$ 
extracted from lepton-charge
analyses of $\epem\to\bbbar$ simulations based on seven \pythia~8 and 
two \pythia~8+\vincia\ tunes %parton-shower+hadronization models 
(squares with red band), compared to the corresponding
experimental results  (rightmost data point, with QCD, in red, and uncorrelated, in blue, systematic uncertainty bands) 
measured by ALEPH (top left)~\cite{leptonALEPH}, DELPHI 
(top right)~\cite{leptonDELPHI95}, L3 (bottom left)~\cite{leptonL392},
and OPAL (bottom right)~\cite{leptonOPAL}.} %The band around the MC predictions indicate their
%associated QCD systematic uncertainty. The error bands around the experimental data point indicate
%QCD systematic and total systematic uncertainties, whereas the error bar shows the statistical uncertainty.}
\label{fig:lepton_AFBobs_vs_MC}
\end{figure}

\begin{figure}[htpb!]
%\centerline{
\includegraphics[width=0.47\linewidth,height=4.8cm]{jetqALEPH.pdf}
\includegraphics[width=0.47\linewidth,height=4.8cm]{jetqDELPHI.pdf}\\
\includegraphics[width=0.47\linewidth,height=4.8cm]{jetqL3.pdf}
\includegraphics[width=0.47\linewidth,height=4.8cm]{jetqOPAL.pdf}
%}
\caption[]{$b$-quark forward-backward asymmetry %$\AFBobs$ 
extracted from jet-charge
analyses of $\epem\to\bbbar$ simulations based on seven \pythia~8 and 
two \pythia~8+\vincia\ tunes %parton-shower+hadronization models 
(squares with red band), compared to the corresponding
experimental results (rightmost data point, with QCD, in red, and uncorrelated, in blue, systematic uncertainty bands) 
measured by ALEPH (top left)~\cite{jetqALEPH}, DELPHI 
(top right)~\cite{jetqDELPHI}, L3 (bottom left)~\cite{jetqL3},
and OPAL (bottom right)~\cite{jetqOPAL}.} %The band around the MC predictions indicate their
%associated QCD systematic uncertainty. The error bands around the experimental data point indicate
%QCD systematic and total systematic uncertainties, whereas the error bar shows the statistical uncertainty.}
\label{fig:jet_AFBobs_vs_MC}
\end{figure}


%\noindent {\bf Acknowledgments} I am grateful to ... for .

%%%%%%%%%%%%%%%%%%%%%%%%%%%%%%%%%%%%%%%%%%%%%%%%%%%%%%%%%%%%%%%%%%%%%%%%%%%%%%%%%%%%%%%%%%%%%%%%%
\section*{References}

\begin{thebibliography}{99}

%\cite{ALEPH:2005ab}%``Precision electroweak measurements on the $Z$ resonance,''
\bibitem{ALEPH:2005ab}S.~Schael {\it et al.} [LEP/SLD Electroweak Working Group], Phys.\ Rept.\  {\bf 427} (2006) 257
%  doi:10.1016/j.physrep.2005.12.006 [hep-ex/0509008].  %%CITATION = doi:10.1016/j.physrep.2005.12.006;%%
%\cite{Bernreuther:2016ccf} %``The forward-backward asymmetry for massive bottom quarks at the $Z$ peak at next-to-next-to-leading order QCD,''
\bibitem{Bernreuther:2016ccf}W.~Bernreuther, L.~Chen, O.~Dekkers, T.~Gehrmann, and D.~Heisler, JHEP {\bf 01} (2017) 053
%  doi:10.1007/JHEP01(2017)053  [arXiv:1611.07942 [hep-ph]].  %%CITATION = doi:10.1007/JHEP01(2017)053;%%
%\cite{Abbaneo:1998xt}%``QCD corrections to the forward - backward asymmetries of c and b quarks at the Z pole,''
\bibitem{Abbaneo:1998xt}D.~Abbaneo {\it et al.} [LEP Heavy Flavor Working Group], Eur.\ Phys.\ J.\ C {\bf 4} (1998) 185
%  doi:10.1007/s100520050196  %%CITATION = doi:10.1007/s100520050196;%%

%\cite{Heister:2001my}  %``Measurement of the forward backward asymmetry in Z --> b anti-b and Z --> c anti-c decays with leptons,''
\bibitem{leptonALEPH} A.~Heister {\it et al.} [ALEPH Collaboration], Eur.\ Phys.\ J.\ C {\bf 24} (2002) 177
%  doi:10.1007/s100520200950  %%CITATION = doi:10.1007/s100520200950;%%
%\cite{Abreu:1994vn} %``Measurement of the forward - backward asymmetry of e+ e- ---> Z ---> b anti-b using prompt leptons and a lifetime tag,''
\bibitem{leptonDELPHI95}  P.~Abreu {\it et al.} [DELPHI Collaboration],   Z.\ Phys.\ C {\bf 65} (1995) 569;
%  doi:10.1007/BF01578667 %%CITATION = doi:10.1007/BF01578667;%%
%\cite{Abdallah:2003gp}  %``Measurement of the forward backward asymmetries of e+ e- ---> Z ---> b anti-b and e+ e- ---> Z -> c anti-c using prompt leptons,''
%\bibitem{leptonDELPHI04} 
J.~Abdallah {\it et al.} [DELPHI Collaboration], Eur.\ Phys.\ J.\ C {\bf 34} (2004) 109
%  doi:10.1140/epjc/s2004-01708-6 [hep-ex/0403041].  %%CITATION = doi:10.1140/epjc/s2004-01708-6;%%
%\cite{Adriani:1992zj} %``Measurement of the e+ e- --> b anti-b and e+ e- --> c anti-c forward backward asymmetries at the Z0 resonance,''
\bibitem{leptonL392} O.~Adriani {\it et al.} [L3 Collaboration], Phys.\ Lett.\ B {\bf 292} (1992) 454;
%  doi:10.1016/0370-2693(92)91203-L %%CITATION = doi:10.1016/0370-2693(92)91203-L;%%
%\cite{Acciarri:1998uy}  %``Measurement of the e+ e- --> Z ---> b anti-b forward - backward asymmetry and the B0 anti-B0 mixing parameter using prompt leptons,''
%\bibitem{leptonL399}
M.~Acciarri {\it et al.} [L3 Collaboration], Phys.\ Lett.\ B {\bf 448} (1999) 152
%  doi:10.1016/S0370-2693(98)01601-3 %%CITATION = doi:10.1016/S0370-2693(98)01601-3;%%
%\cite{Abbiendi:2003pq}%``Measurement of heavy quark forward backward asymmetries and average B mixing using leptons in hadronic Z decays,''
\bibitem{leptonOPAL}  G.~Abbiendi {\it et al.} [OPAL Collaboration],  Phys.\ Lett.\ B {\bf 577} (2003) 18
%  doi:10.1016/j.physletb.2003.10.022 [hep-ex/0308051].   %%CITATION = doi:10.1016/j.physletb.2003.10.022;%%

%\cite{Heister:2001ya} %``Measurement of A**b(FB) using inclusive b hadron decays,''
\bibitem{jetqALEPH} A.~Heister {\it et al.} [ALEPH Collaboration], Eur.\ Phys.\ J.\ C {\bf 22} (2001) 201
%  doi:10.1007/s100520100812 [hep-ex/0107033].  %%CITATION = doi:10.1007/s100520100812;%%
%\cite{Abdallah:2004nh}  %``Determination of A**b(FB) at the Z pole using inclusive charge reconstruction and lifetime tagging,''
\bibitem{jetqDELPHI} J.~Abdallah {\it et al.} [DELPHI Collaboration], Eur.\ Phys.\ J.\ C {\bf 40} (2005) 1
%  doi:10.1140/epjc/s2004-02104-0 [hep-ex/0412004].  %%CITATION = doi:10.1140/epjc/s2004-02104-0;%%
%\cite{Acciarri:1998ef}  %``Measurement of the effective weak mixing angle by jet charge asymmetry in hadronic decays of the Z boson,''
\bibitem{jetqL3} M.~Acciarri {\it et al.} [L3 Collaboration],  Phys.\ Lett.\ B {\bf 439} (1998) 225
%  doi:10.1016/S0370-2693(98)01174-5 %%CITATION = doi:10.1016/S0370-2693(98)01174-5;%%
%\cite{Abbiendi:2002bx}  %``Measurement of the b quark forward backward asymmetry around the Z0 peak using an inclusive tag,''
\bibitem{jetqOPAL} G.~Abbiendi {\it et al.} [OPAL Collaboration],  Phys.\ Lett.\ B {\bf 546} (2002) 29;
%  doi:10.1016/S0370-2693(02)02594-7 [hep-ex/0209076].  %%CITATION = doi:10.1016/S0370-2693(02)02594-7;%%
%\cite{Ackerstaff:1997ke}  %``Measurements of the b quark forward - backward asymmetry around the Z0 peak using jet charge and vertex charge,''
%\bibitem{Ackerstaff:1997ke}
K.~Ackerstaff {\it et al.} [OPAL Collaboration],  Z.\ Phys.\ C {\bf 75} (1997) 385
%  doi:10.1007/s002880050482 %%CITATION = doi:10.1007/s002880050482;%%

%\cite{Gomez-Ceballos:2013zzn}%``First Look at the Physics Case of TLEP,''
\bibitem{FCCee}M.~Bicer {\it et al.} [TLEP Design Study Working Group], JHEP {\bf 01} (2014) 164;
%  doi:10.1007/JHEP01(2014)164 [arXiv:1308.6176 [hep-ex]].  %%CITATION = doi:10.1007/JHEP01(2014)164;%%
%\cite{dEnterria:2016sca}\bibitem{dEnterria:2016sca}%``Physics at the FCC-ee,''
and D.~d'Enterria, doi:10.1142/9789813224568\_0028; arXiv:1602.05043 [hep-ex] %%CITATION = doi:10.1142/9789813224568_0028;%%

%\cite{Sjostrand:2014zea} %``An Introduction to PYTHIA 8.2,''
\bibitem{pythia8}T.~Sj\"ostrand {\it et al.}, Comput.\ Phys.\ Commun.\  {\bf 191} (2015) 159
%  doi:10.1016/j.cpc.2015.01.024 [arXiv:1410.3012 [hep-ph]].  %%CITATION = doi:10.1016/j.cpc.2015.01.024;%%
%\cite{Fischer:2016vfv}  %``Vincia for Hadron Colliders,''
\bibitem{vincia}N.~Fischer, S.~Prestel, M.~Ritzmann and P.~Skands,  Eur.\ Phys.\ J.\ C {\bf 76} (2016) 589
%  doi:10.1140/epjc/s10052-016-4429-6 [arXiv:1605.06142 [hep-ph]].  %%CITATION = doi:10.1140/epjc/s10052-016-4429-6;%%
%\cite{Bartel:1986ua} %``Experimental Studies on Multi-Jet Production in e+ e- Annihilation at PETRA Energies,''
%\bibitem{jade}W.~Bartel {\it et al.} [JADE Collaboration], Z.\ Phys.\ C {\bf 33} (1986) 23
%  doi:10.1007/BF01410449 %%CITATION = doi:10.1007/BF01410449;%%
\bibitem{DdEYan}D.~d'Enterria and C.~Yan, in preparation
\end{thebibliography}

\end{document}

%%%%%%%%%%%%%%%%%%%%%%
% End of moriond.tex  %
%%%%%%%%%%%%%%%%%%%%%%


