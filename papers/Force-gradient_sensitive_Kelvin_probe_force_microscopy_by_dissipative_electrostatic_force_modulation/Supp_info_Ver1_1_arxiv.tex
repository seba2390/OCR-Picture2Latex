%\documentclass[aip,graphicx]{revtex4-1}
\documentclass[preprint]{revtex4-1}

\usepackage{graphicx} % or graphicx
\usepackage{epstopdf}
\usepackage{float}
\usepackage{dcolumn}
\usepackage{mathtools}
\usepackage{amsmath}
\usepackage[T1]{fontenc}

\newcommand{\Ecanti}{E_\mathrm{canti}}
\newcommand{\dEloss}{\Delta E_\mathrm{loss}}
\newcommand{\Ploss}{P_\mathrm{loss}}
\newcommand{\Ftot}{F_\mathrm{tot}}
\newcommand{\Fchem}{F_\mathrm{chem}}
\newcommand{\FvdW}{F_\mathrm{vdW}}
\newcommand{\Fres}{F_\mathrm{res}}
\newcommand{\Fcap}{F_\mathrm{cap}}
\newcommand{\Felec}{F_\mathrm{elec}}
\newcommand{\Fexc}{F_\mathrm{exc}}
\newcommand{\Fts}{F_\mathrm{ts}}
\newcommand{\Vbias}{V_\mathrm{bias}}
\newcommand{\Vcpd}{V_\mathrm{cpd}}
\newcommand{\Fother}{F_\mathrm{other}}
\newcommand{\dcdz}{\frac{\partial C}{\partial z}}
\newcommand{\qsub}{q_\mathrm{sub}}
\newcommand{\qtip}{q_\mathrm{tip}}
\newcommand{\qO}{q_{0}}
\newcommand{\Ctip}{C_\mathrm{tip}}
\newcommand{\Csub}{C_\mathrm{sub}}
\newcommand{\Csem}{C_\mathrm{sem}}
\newcommand{\Ctot}{C_\mathrm{tot}}
\newcommand{\Cpal}{C_{\parallel}}
\newcommand{\Cgap}{C_\mathrm{gap}}
\newcommand{\Vtip}{V_\mathrm{tip}}
\newcommand{\Vsub}{V_\mathrm{sub}}
\newcommand{\Vdot}{V_\mathrm{dot}}
\newcommand{\Vdc}{V_\mathrm{dc}}
\newcommand{\Vac}{V_\mathrm{ac}}
\newcommand{\eO}{\epsilon_{0}}
\newcommand{\es}{\epsilon_\mathrm{s}}
\newcommand{\zO}{z_\mathrm{0}}
\newcommand{\Fext}{F_\mathrm{ext}}
\newcommand{\zw}{z_\mathrm{\omega}}
\newcommand{\iw}{i\omega}
\newcommand{\Edc}{E_\mathrm{dc}}
\newcommand{\wO}{\omega _\mathrm{0}}
\newcommand{\df}{\Delta f}
\newcommand{\kB}{k_\mathrm{B}}
\newcommand{\Gre}{G_\mathrm{re}}
\newcommand{\Gim}{G_\mathrm{im}}
\newcommand{\dw}{\Delta \omega}
\newcommand{\qdot}{q_\mathrm{dot}}
\newcommand{\delinvQ}{\Delta \left(\frac{1}{Q}\right)}
\newcommand{\kts}{k_\mathrm{ts}}
\newcommand{\gts}{\gamma_\mathrm{ts}}
\newcommand{\alphaO}{\alpha_\mathrm{0}}
\newcommand{\Fdc}{F_\mathrm{0}}
\newcommand{\Fw}{F_\mathrm{\omega}}
\newcommand{\Fww}{F_\mathrm{2\omega}}
\newcommand{\Fwww}{F_\mathrm{3\omega}}
\newcommand{\Fin}{F_\mathrm{in}}
%\newcommand{\Fout}{F_\mathrm{out}}
\newcommand{\Fout}{F_\mathrm{quad}}
\newcommand{\wm}{\omega_\mathrm{m}}
\newcommand{\Fwm}{F_\mathrm{\wm}}
\newcommand{\wel}{\omega_\mathrm{el}}
\newcommand{\fm}{f_\mathrm{m}}


\begin{document}

%Title of paper
\title{Supplemental information for\\
''Force-gradient sensitive Kelvin probe force microscopy 
by dissipative electrostatic force modulation''}

\author{Yoichi Miyahara}%
\email[Corresponding author: ]{yoichi.miyahara@mcgill.ca}
\author{Peter Grutter}%
\affiliation{Department of Physics, Faculty of Science, McGill University, Montreal, Quebec, Canada H3A 2T8}

\date{\today}

\maketitle
\section{Derivation of Eq.~2 and 3}
In the absence of localized charges, 
the electrostatic force between a conducting tip and sample under dc and ac bias voltage
is described as follows:
\begin{eqnarray}
      \Felec (t) & = & \frac{1}{2}\dcdz \{\Vbias +\Vac \cos (\wel t + \phi) - \Vcpd\}^2 \\
      & = & \Fdc + \Fw + \Fww \nonumber \\
       \Fdc & = & \frac{1}{2} \dcdz \left[(\Vbias - \Vcpd)^2 + \frac{\Vac ^2}{2}\right] \label{Fdc} 
       = \alpha \left(\Vdc ^2 + \frac{\Vac^2}{2} \right)\\ 
       \Fw &= &  \dcdz (\Vbias - \Vcpd)\Vac \cos (\wel t + \phi) 
       = 2\alpha \Vdc \Vac \cos (\wel t + \phi)  \label{Fw} \\ 
       \Fww & = & \frac{1}{4} \dcdz \Vac ^2 \cos (2(\wel t + \phi))  \label{F2w} 
       = \frac{1}{2}\alpha \Vac^2 \cos (2(\wel t + \phi))
\end{eqnarray}
where
\[ 
  \label{eq:alpha}
         \alpha \equiv \frac{1}{2}\dcdz \nonumber, \,\,\,  \Vdc \equiv \Vbias - \Vcpd
\]


\begin{tabular}{l l l}
  $C$ & : & tip-sample capacitance\\
  $\Vbias$ & : &applied dc bias voltage\\
  $\Vcpd$ & : & contact potential difference\\
  $\Vac$ & : & amplitude of applied ac voltage\\
  $\wel$ & : & angular frequency of applied ac voltage\\
  $\phi$ & : & phase of applied ac voltage with respect to tip oscillation\\
  $z$ & : & position of the tip measured from the sample surface
\end{tabular}
\vspace*{5mm}

In general, $\alpha \equiv \frac{1}{2}\dcdz(z(t))$ is 
also a function of time through time-dependent $z(t)=A\cos(\wm t)$
with $\wm$ and $A$ being its oscillation angular frequency and amplitude,
respectively.

By expanding $\alpha$ around the mean position $z_0$,
and taking the first order,

\begin{equation}
  \alpha (z) \approx \alpha_0 + \alpha ' (z-z_0) = \alpha_0 + \alpha ' A \cos(\wm t)
\end{equation}


Substituting $\alpha$ into eq.\ref{Fdc} yields 
\begin{eqnarray}
       \Fdc(t)  &= & \{\alpha_0 + \alpha' A \cos(\wm t)\} \left(\Vdc ^2 + \frac{\Vac^2}{2}\right) \nonumber \\
       &=& \alpha_0 \left(\Vdc ^2 + \frac{\Vac^2}{2}\right)  
       + \alpha' A\left(\Vdc ^2 + \frac{\Vac^2}{2}\right) \cos (\wm t)
      \label{Fdc2} 
\end{eqnarray}

The first term in the above formula yields the static deflection of the cantilever 
wheares the second one is an alternating force whose frequency is $\wm/2\pi$.
As this force is in phase with $z(t)$, it will lead to the shift in the resonance frequency.
\\\\
Next, we look at $\Fw$ term (eq.~\ref{Fw}).
% Omega term
\begin{eqnarray*}
       \Fw(t) &= & 2 \{\alpha_0 + \alpha' A \cos(\wm t)\} \Vdc\Vac \cos (\wel t + \phi) \\
       & = & 2\alpha_0 \Vdc\Vac \cos(\wel t + \phi) 
       + 2\alpha' A\Vdc\Vac \cos(\wel t + \phi) \cos(\wm t) \nonumber \\
       & = & 2\alpha_0 \Vdc\Vac \cos(\wel t + \phi) 
       + 2\alpha' A\Vdc\Vac \frac{1}{2}[\cos\{(\wel + \wm) t + \phi)\} + \cos\{(\wel - \wm)t +\phi\}] \nonumber \\
       & = & 2\alpha_0 \Vdc\Vac \{\cos(\wel t)\cos(\phi)- \sin(\wel t)\sin(\phi) \} \nonumber \\
       & & + \alpha' A\Vdc\Vac [\cos\{(\wel + \wm) t + \phi)\} + \cos\{(\wel - \wm)t +\phi\}]       \label{Fw2} 
\end{eqnarray*}

When $\wel =2 \wm$, 
\begin{eqnarray}
       \Fw(t) 
       & = & 2\alpha_0 \Vdc\Vac \{\cos(2\wm t)\cos(\phi)- \sin(2\wm t)\sin(\phi) \} \nonumber \\
       & & + \alpha' A\Vdc\Vac [\cos\{(3\wm) t + \phi)\} + \cos(\wm t +\phi)]       \label{Fw2} 
\end{eqnarray}
The only component that contributes to the frequency shift and dissipation is the last term
in Eq.~\ref{Fw2}.

Looking at $\Fww$ term in the same way,
\begin{eqnarray}
       \Fww(t) &=& \frac{1}{2} \{\alpha_0 +  \alpha' A \cos(\wm t)\} \Vac ^2 \cos (2(\wel t + \phi)) \nonumber \\ 
       & = & \frac{1}{2}  [\alpha_0 \Vac ^2 \cos (2(\wel t + \phi)) 
       + \frac{1}{2} \alpha' A \Vac ^2 \{ \cos \{(2\wel - \wm)t + 2\phi)\} 
       +  \cos \{(2\wel + \wm) t + 2\phi)\}]  \nonumber \\
       & = & \frac{1}{2}  \alpha_0 \Vac ^2 \cos (2(\wel t + \phi)) 
       + \frac{1}{4} \alpha' A \Vac ^2 \{ \cos ((2\wel - \wm)t + 2\phi)
       +  \cos ((2\wel + \wm) t + 2\phi)\} \nonumber \\
       & = & \frac{1}{2}  \alpha_0 \Vac ^2 \cos (4\wm t + 2\phi) 
       + \frac{1}{4} \alpha' A \Vac ^2 \{ \cos (3\wm t + 2\phi)
       +  \cos (5\wm t + 2\phi)\}   \label{eq:F2w2}
\end{eqnarray}

 No term in Eq.~\ref{eq:F2w2} contributes to the frequency shift and dissipation.

 Gathering the terms with $\wm$ from Eq.~\ref{Fdc2} and \ref{eq:F2w2}, 
we obtain the following result shown in Eq~2 and 3 in the body text.

 \begin{eqnarray*}
  \Fwm(t) & = &   \alpha' A \left\{ \left(\Vdc ^2 + \frac{\Vac^2}{2}\right) \cos (\wm t)
      +  \Vdc \Vac  \cos (\wm t + \phi)\right\}\\
       & = & \alpha' A \left\{\Vdc ^2 +  \cos(\phi) \Vdc\Vac +\frac{\Vac^2}{2} \right\}\cos(\wm t) \\ 
  & &    -\alpha' A \Vdc \Vac  \sin(\phi) \sin(\wm t)\\
    & = & \Fin \cos (\wm t) + \Fout \sin(\wm t)
\end{eqnarray*}
where
\begin{equation}
  \Fin =  \alpha'A \left(\Vdc + \frac{\Vac}{2} \cos\phi   \right)^2
  + \frac{\alpha'A }{2}\left(1 - \frac{\cos^2\phi}{2} \right) \Vac^2
  \label{eq:Fin}
\end{equation}

\begin{equation}
  \Fout = - \alpha'A  \Vdc\Vac  \sin\phi 
    \label{eq:Fquad}
\end{equation}


\section{Fitting result of frequency shift offset as a function of phase}
\begin{figure}[h]
  \centering
  \includegraphics[width=80mm]{figureS1.pdf}
  \caption{Frequency shift offset as a function of phase. 
    Solid line is the fitted curve with the second term of Eq.~2 in the body text.
\label{fig:DKPFM_image}}
\end{figure}


\section{Detail of the sample}
\subsection{Optical micrograph of MoS$_2$ flakes}
\begin{figure}[h]
  \centering
  \includegraphics[width=80mm]{figureS2.pdf}
  \caption{Optical micrograph of the MoS$_2$ flakes exfoliated on SiO$_2$/Si substrate. 
The flake in the circle is the one imaged by KPFM. Scale bar is 20~$\mu$m.
\label{fig:DKPFM_image}}
\end{figure}

\subsection{AFM topography images of MoS$_2$ flake}
\begin{figure}[h]
  \centering
  \includegraphics[width=170mm]{figureS3.pdf}
  \caption{Topography images of patterned MoS$_2$ on SiO$_2$/Si substrate 
        simultaneously taken with each corresponding KPFM image shown in 
        Fig.~4 in the body text. 
        (a) D-KPFM ($1\omega$) ($\df = -10.3$~Hz, $A=5$~nm$_\text{p-p}$, 
        $\Vac=50$~mV) and 
        (b) D-KPFM ($2\omega$) ($\df = -8.3$~Hz, $A=5$~nm$_\text{p-p}$, 
        $\Vac=1$~V) and 
        (b) FM-KPFM ($\df = -7.0$~Hz, $A=5$~nm$_\text{p-p}$, 
        $\Vac=1$~V). 
        The line profiles are shown for the same location on the sample.
\label{fig:DKPFM_image}}
\end{figure}




\bibliography {../library}

\end{document}

