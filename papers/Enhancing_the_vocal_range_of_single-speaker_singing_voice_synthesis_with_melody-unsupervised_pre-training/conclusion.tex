\section{Conclusion}
% \label{sec:conclusion}

This paper proposes a melody-unsupervised pre-training method conducted on a multi-singer dataset to enhance the vocal range of the single-speaker, while not degrading the timbre similarity.
Moreover, it also contributes to improving the sound quality and rhythm naturalness of the synthesized singing voices. It is the first to introduce a differentiable duration regulator to improve the rhythm naturalness, and a bi-directional flow model to improve the sound quality.
Experiments on Opencpop show that the proposed method outperforms the baseline in both subjective and objective evaluations.


% To improve the pitch robustness of SVS model, we propose a pre-training strategy that can be used for a large-scale multi-singer dataset containing only audio-and-lyrics pairs, without temporal alignment information. We adopt a speaker encoder base on the ECAPA-TDNN to model the different timbre information of the multi-singer dataset. We further proposed a differentiable-up-sampling layer and a bi-directional flow model to improve the naturalness of synthesized singing voice. Experiments on Opencpop show that the proposed system outperforms the baseline model in both subjective evaluation and objective indicators.

% In the future, we will continue to explore ways to improve the quality of our synthesized singing voice to further bridge the gap between synthesized voice and recorded singing voice performed by humans.