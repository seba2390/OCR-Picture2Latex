 \documentclass[a4paper,11pt]{amsart}
\usepackage{amsmath,amsfonts,amsthm,amssymb,enumerate}
\usepackage{graphicx}
%\graphicspath{ {./picture1/}} 

\numberwithin{equation}{section}
%\numberwithin{figure}{section}
% Theorem environment

\newtheorem{theorem}{Theorem}[section]
\newtheorem{corollary}[theorem]{Corollary}
\newtheorem{proposition}[theorem]{Proposition}
\newtheorem{conjecture}[theorem]{Conjecture}
\newtheorem{claim}[theorem]{Claim}
\newtheorem{lemma}[theorem]{Lemma}
%\newtheorem{remark}[theorem]{Remark}
\newtheorem{question}[theorem]{Question}
{
\theoremstyle{definition}
\newtheorem{definition}[theorem]{Definition}
\newtheorem{example}[theorem]{Example}
\newtheorem{remark}[theorem]{Remark}
}
 

% Definition with Roman Letters.
%\newenvironment{df}{\begin{definition}\rm}{\end{definition}}
\newenvironment{rem}{\begin{remark}\rm}{\end{remark}}
%\newenvironment{ex}{\begin{example}\rm}{\end{example}}
 
\title{Transformations of lattice polytopes and their associated dotted graphs}
\author{Inasa Nakamura}
\address{Department of Mathematics, Information Science and  Engineering, \newline
Saga University, \newline 
1 Honjomachi, Saga, 840-1153, Japan}
\email{inasa@cc.saga-u.ac.jp}
\subjclass[2020]{Primary: 05C10, Secondary: 57K10}
\keywords{graph; deformation; partial matching}

\pagestyle{plain}
\begin{document}  
\begin{abstract}
We consider a lattice polytope in the $xy$-plane such that each edge is parallel to the $x$-axis or the $y$-axis. In \cite{N}, we investigated transformations of certain lattice polytopes, and we considered the reduced graph that is obtained from deformations of a graph  associated with a lattice polytope. In this paper, we refine the notion of the reduced graph by introducing the notion of a dotted graph. A lattice polytope is presented by an admissible dotted graph.
 We investigate deformations of dotted graphs, and we investigate relation between deformations of admissible dotted graphs and transformations of lattice polytopes, giving results that are refined and corrected versions of \cite[Lemma 6.2, Theorem 6.3]{N}. % with minimal area. 
\end{abstract}
\maketitle

\section{Introduction}\label{sec1}


 
Let $\mathbb{R}^2$ be the $xy$-plane. We say that a segment is in the {\it $x$-direction} or in the {\it $y$-direction} if it is parallel to the $x$-axis or the $y$-axis, respectively.  
 
 

\begin{definition}
A {\it lattice polytope in the lattice}, or simply a {\it lattice polytope}, is a polytope $P$ in $\mathbb{R}^2$ consisting of a finite number of vertices of degree zero (called {\it isolated vertices}) with multiplicity $2$, vertices of degree $2$ with multiplicity $1$, edges, and faces satisfying the following conditions. 

\begin{enumerate}[$(1)$]
\item
Each edge is either in the $x$-direction or in the $y$-direction. 

\item
The $x$-components (respectively $y$-components) of isolated vertices and edges in the $x$-direction (respectively $y$-direction)  are distinct. 

\item
The boundary $\partial P$ is equipped with a coherent orientation as an immersion of a union of several circles, where we call the union of edges of $P$ the {\it boundary} of $P$, denoted by $\partial P$. 
\end{enumerate}
 
The set of vertices are divided into two sets $X_0$ and $X_1$ 
such that each edge in the $x$-direction is oriented from a vertex of $X_0$ to a vertex of $X_1$, and isolated vertices are both in $X_0$ and $X_1$ with multiplicity 1. We denote $X_0$ (respectively $X_1$) by $\mathrm{Ver}_0(P)$ (respectively $\mathrm{Ver}_1(P)$), and we call it the set of {\it initial vertices} (respectively {\it terminal vertices}). 
When we draw figures, we denote an initial vertex (respectively a terminal vertex) by a small black disk (respectively an $X$ mark). And in our figures, the $x$-direction is the vertical direction, in order to be coherent with figures in \cite{N}. 
\end{definition}
%
\begin{figure}[ht]
\centering
\includegraphics*[height=4cm]{A-fig1}
\caption{A lattice polytope in the lattice (left figure) and the result of a transformation along a rectangle (right figure). The rectangle is denoted by the shadowed area. We remark that the $x$-direction is the vertical direction, and we denote an initial vertex (respectively a terminal vertex) by a small black disk (respectively an $X$ mark). }
\label{Fig1}
\end{figure}
%

 See the left figure in Figure \ref{Fig1} for an example of a lattice polytope. We remark that in graph theory, a lattice polytope is defined as a polytope whose vertices are lattice points, where a {\it lattice point} is a point $(x,y)$ of $\mathbb{R}^2$ such that $x$ and $y$ are integers \cite{Barvinok2, Diestel}. However, here, though we don't require that the vertices are lattice points, we call our polytope a lattice polytope. 

In \cite{N}, we investigated partial matchings, using certain lattice polytopes whose vertices are lattice points. 
A partial matching is a finite graph consisting of edges and vertices of degree zero or one, with the vertex set $\{1,2,\ldots,m\}$ for some positive integer $m$; see \cite{Reidys} for researches on structures of partial matchings using chord diagrams. 
In \cite{N}, with a motivation to investigate partial matchings, we introduced the lattice presentation of a partial matching, and the lattice polytope associated with a pair of partial matchings. Further, in \cite{N}, we treated transformations of lattice presentations and lattice polytopes, and the area of a transformation, and the reduced graph of a lattice polytope. A transformation of a lattice polytope is a sequence of transformations of lattice polytopes such that each is described by a rectangle as in Figure \ref{Fig1}. 

In this paper, we refine the notion of a reduced graph by introducing the notion called a dotted graph (Definition \ref{def2-1}). 
A lattice polytope is presented by an admissible dotted graph.
%
We introduce a refined version of deformations of dotted graphs, and 
 %We investigate deformations of dotted graphs, and 
show that a reduced graph of a dotted graph obtained by a certain sequence of deformations is uniquely determined up to certain local moves (Theorem \ref{prop3-5}), and that for a dotted graph, any sequence consisting of certain deformations is finite (Theorem \ref{r-prop3-5}). These theorems provide a refined and corrected version of \cite[Lemma 6.2]{N}. 
%
Further, we give a refined and corrected version of \cite[Theorem 6.3]{N}: 
for an admissible dotted graph $\Gamma$, its associated lattice polytope $P$ admits a transformation with minimal area if it is deformed to the empty graph by specific deformations, and if $P$ admits a transformation with minimal area, then $\Gamma$ is deformed to the empty graph by certain deformations (Theorem \ref{thm3-7}); see Remark \ref{rem0904}. Further, we show that if a dotted graph has \lq\lq many'' dots, then it is admissible (Proposition \ref{prop3-8}), and a dotted graph with ``many'' dots is reduced to the empty graph (Proposition \ref{prop3-9}). 
%Further, we investigate the reducibility of admissible dotted graphs in certain simple forms (Propositions \ref{prop4-1}, \ref{prop4-4}, \ref{prop4-5}). 


The paper is organized as follows. In Section \ref{sec1}, we review transformations of lattice polytopes and the areas of transformations. 
In Section \ref{sec2}, we give the definitions of a dotted graph and the deformations of dotted graphs, and Theorems \ref{prop3-5} and \ref{r-prop3-5}. 
In section \ref{sec-proof}, we show Theorems \ref{prop3-5} and \ref{r-prop3-5}. 
In section \ref{deformation}, we investigate relation between deformations of admissible dotted graphs and transformations of lattice polytopes, and we give Theorem \ref{thm3-7} and Propositions \ref{prop3-8} and \ref{prop3-9}. 
Section \ref{sec-lemma} is devoted to showing lemmas. 
%In Section \ref{sec3}, we show Propositions \ref{prop4-1}, \ref{prop4-4}, \ref{prop4-5}, using combinatorial arguments. 


\section{Transformations of lattice polytopes and the areas of  transformations}\label{sec1}
In this section, we review transformations of lattice polytopes and the areas of transformations \cite{N}. 
\subsection{Transformations of lattice polytopes}
  
For a point $v=(x_1,y_1)$ of $\mathbb{R}^2$, we call $x_1$ (respectively $y_1$) the {\it $x$-component} (respectively the {\it $y$-component}) of $v$. 
For a set of points of $\mathbb{R}^2$, $\{ v_1, \ldots, v_n\}$  with $v_j=(x_j, y_j)$ ($j=1,\dots, n$), we call the set $\{ x_1, \ldots, x_n, y_1,\ldots, y_n\}$ the {\it set of $x$, $y$ components} of $\{ v_1,\ldots, v_n\}$. 

\begin{definition}
Let $\Delta$ be a set of points in $\mathbb{R}^2$. For a point $u$ of $\mathbb{R}^2$, we denote by $x(u)$ and $y(u)$ the $x$ and $y$-components of $u$, respectively.  For two distinct points $v, w$ of $\Delta$, 
let $R(v, w)$ be the rectangle one pair of whose diagonal vertices are $v$ and $w$. Put $\tilde{v}=(x(w), y(v))$ and $\tilde{w}=(x(v), y(w))$, which form the other pair of diagonal vertices of $R(v,w)$. Then, consider a new set of points obtained from $\Delta$, from removing $v,w$ and adding $\tilde{v}, \tilde{w}$. 
We call the new set of points the result of the {\it transformation of $\Delta$ by the rectangle} $R(v, w)$, denoted by $t( \Delta; R(v, w))$. 

For two sets of points $\Delta$ and $\Delta'$ in $\mathbb{R}^2$ with the same set of $x, y$-components, we define a {\it transformation from $\Delta$ to $\Delta'$}, denoted by $\Delta \to \Delta'$, as a sequence of transformations by rectangles such that the initial and the terminal points are $\Delta$ and $\Delta'$, respectively. 

For a lattice polytope $P$, we define a {\it transformation of $P$} as a transformation from $\mathrm{Ver}_0(P)$ to $\mathrm{Ver}_1(P)$. We recall that $\mathrm{Ver}_0(P)$ and $\mathrm{Ver}_1(P)$ are the set of initial vertices and terminal vertices of $P$, respectively. 

For a lattice polytope $P'$, let $R$ be a rectangle that contains vertices of $\mathrm{Ver}_0(P')$ as a diagonal pair. 
We define the {\it transformation of $P'$ along the rectangle $R$} as 
the transformation from $P'$ to the lattice polytope whose initial vertices  are $t(\mathrm{Ver}_0(P'); R)$ and terminal vertices are $\mathrm{Ver}_1(P')$, respectively. 
\end{definition}

A transformation of a lattice polytope $P$ is described as a sequence of transformations of lattice polytopes along rectangles such that each rectangle contains initial vertices of each lattice polytope as a diagonal pair. See Figure \ref{Fig1} for an example of a transformation of a lattice polytope along a rectangle. 




\subsection{Areas of transformations}\label{sec4-2}

For a lattice polytope $P$, we define the area of $P$ and the area of $|P|$, denoted by $\mathrm{Area}(P)$ and $\mathrm{Area}|P|$, respectively, as follows. 
Recall that for a lattice polytope $P$, 
we give an orientation of $\partial P$ by giving each edge in the $x$-direction (respectively in the $y$-direction) the orientation from a vertex of $\mathrm{Ver}_0(P)$ to a vertex of $\mathrm{Ver}_1(P)$ (respectively from a vertex of $\mathrm{Ver}_1(P)$ to a vertex of $\mathrm{Ver}_0(P)$). Then, $\partial P$ has a coherent orientation as an immersion of a union of several circles. 
The space $\mathbb{R}^2$, which contains $P$, is divided into several regions $A_1,\ldots, A_m$ by $\partial P$. For each region $A_i$ ($i=1,\dots,m$), let $\omega(A_i)$ be the {\it rotation number} of $P$ with respect to $A_i$, which is the sum of rotation numbers of the connected components (as immersed circles) 
of $P$, with respect to $A_i$. Here, the {\it rotation number} of a connected lattice polytope $Q$ (that is regarded as an immersed circle) with respect to a region $A$ of $\mathbb{R}^2 \backslash\partial Q$ is the rotation number of a map $f$ from $\partial Q=\mathbb{R}/2\pi \mathbb{Z}$ to $\mathbb{R}/2\pi\mathbb{Z}$ which maps $x \in \partial Q$ to the argument of the vector from a fixed interior point of $A$ to $x$. Here, the {\it rotation number} of the map $f$ is defined by 
$(F(x)-x)/2\pi$, where $F: \mathbb{R} \to \mathbb{R}$ is the lift of $f$ and $x \in \partial Q$.
We define $\mathrm{Area}(A_i)$ for a region $A_i$ by the area induced from $\mathrm{Area}(R)=|x_2-x_1||y_2-y_1|$ for a rectangle $R$ whose diagonal vertices are $(x_1, y_1)$ and  $(x_2, y_2)$. 
Then, we define the {\it area} of $P$, denoted by $\mathrm{Area}(P)$, by $\mathrm{Area}(P)=\sum_{i=1}^m \omega(A_i)\mathrm{Area}(A_i)$, and the {\it area} of $|P|$, denoted by $\mathrm{Area}|P|$, by $\mathrm{Area}|P|=\sum_{i=1}^m |\omega(A_i)\mathrm{Area}(A_i)|$. 
 
 
\begin{definition}
 

Let $P$ be a lattice polytope. We consider a transformation of $P$,  
$\mathrm{Ver}_0(P)=\Delta_0 \to \Delta_1\to \cdots \to \Delta_k=\mathrm{Ver}_1(P)$, such that each $\Delta_{j-1} \to \Delta_j$ is a transformation by a rectangle $R_j$ 
($j=1,2,\ldots,k$). 
Then, we call $\sum_{j=1}^k |\mathrm{Area}(R_j)|$ the {\it area of the transformation of $P$}.

\end{definition} 
 

Theorem 5.9 in \cite{N} implies the following. 

\begin{theorem}\label{thm3-10}
Let $P$ be a lattice polytope. We consider a transformation of $P$,  
$\mathrm{Ver}_0(P)=\Delta_0 \to \Delta_1\to \cdots \to \Delta_k=\mathrm{Ver}_1(P)$, such that each $\Delta_{j-1} \to \Delta_j$ is a transformation by a rectangle $R_j$ 
($j=1,2,\ldots, k)$. 

Then
\begin{equation}\label{eq0}
\sum _{j=1}^k \mathrm{Area}(R_j) =\mathrm{Area}(P) 
\end{equation}
and 
\begin{equation}\label{eq2}
\sum_{j=1}^k |\mathrm{Area}(R_j)|\geq \mathrm{Area}|P|. 
\end{equation}

Further, when $P$ satisfies either the condition $(1)$ or $(2)$ in \cite[Theorem 5.9]{N}, there exist transformations which realize the equality of $(\ref{eq2})$.  

 \end{theorem}

We call a transformation which realizes the equality of $(\ref{eq2})$ a transformation {\it with minimal area}.  

\section{Dotted graphs and their reduced graphs}\label{sec2}
In this section, we introduce the notion of reduced graphs of a lattice polytope. And we give theorems that a reduced graph of a dotted graph obtained by a certain sequence of deformations is uniquely determined up to certain local moves (Theorem \ref{prop3-5}), and that for a dotted graph, any sequence consisting of certain deformations is finite (Theorem \ref{r-prop3-5}). 


\begin{remark}\label{rem0904}
 

Theorems \ref{prop3-5} and \ref{r-prop3-5} provide a refined and corrected version of \cite[Lemma 6.2]{N}; we remark that the set of deformations given in \cite{N} does not include the cases that have \lq\lq overlapping regions'', and for deformations that delete circle/loop components, the condition for the label of the complement region is necessary to show the uniqueness; 
the original version of \cite[Lemma 6.2]{N} holds true when we add the condition of labels for deformations that delete circle/loop components, and any sequence of deformations satisfies the condition (A) given in Theorem \ref{prop3-5}. 

Theorem \ref{thm3-7} is a refined and corrected version of \cite[Theorem 6.3]{N}; we remark that \cite[Theorem 6.3]{N} does not hold for lattice polytopes that admit transformations other than those described by the given deformations. 
\end{remark}
 

We reformulate the notion of a reduced graph. 
\begin{definition}\label{def2-1}
Let $\Gamma$ be a finite graph in $\mathbb{R}^2$. Then $\Gamma$ is a {\it dotted graph} if each edge has an orientation and each vertex is of degree 2 or degree 4, satisfying the following conditions. 
\begin{enumerate}
\item
Around each vertex of degree 2, the edges has a coherent orientation. We denote the vertex by a small black disk, called a {\it dot}.  

\item
Around each vertex of degree 4, each pair of diagonal edges has a coherent orientation. We call the vertex a {\it crossing}.

\end{enumerate}

We regard edges connected by vertices of degree 2 as an edge  equipped with several dots. We call an edge or a part of an edge an {\it arc} (with/without dots). The complement of a dotted graph $\Gamma \subset \mathbb{R}^2$ consists of a finite number of connected components. We call each component a {\it region}. We regard $\Gamma$ as a finite number of immersed oriented circles with transverse intersection points, and we assign each region with an integral label denoting the rotation number. 
%
We also call $\Gamma$ equipped with integral labels on regions a {\it dotted graph}. 

Two dotted graphs are the {\it same} if they are related by an ambient isotopy of $\mathbb{R}^2$. 

\end{definition}

 

Let $P$ be a lattice polytope in the lattice. 
Let $\partial P$ be the boundary of $P$ equipped with the initial vertices $\mathrm{Ver}_0(P)$, which are one of the two types of vertices $\mathrm{Ver}_0(P)$ and $\mathrm{Ver}_1(P)$. 
Then $\partial P$ is a dotted graph, which will be called the {\it dotted graph associated with a lattice polytope $P$}; see Figure \ref{Fig2}.
We say that a dotted graph $\Gamma$ is {\it admissible} if there exists a lattice polytope $P$ with which $\Gamma$ is associated. 

\begin{figure}[ht]
\centering
\includegraphics*[height=4cm]{A-fig2}
\caption{(a) A lattice polytope and (b) the associated dotted graph.}
\label{Fig2}
\end{figure}


Let $\Gamma$ be a dotted graph. 
We call a set of arcs with dots and connecting crossings a {\it circle component} if it bounds an embedded disk and any pair of arcs connected by a crossing is a pair of diagonal arcs. And we call a set of arcs with dots and connecting crossings a {\it loop component} if it bounds an embedded disk and one pair of arcs connected by a crossing $c$ is a pair of adjacent arcs and 
any other pair of arcs connected by a crossing is a pair of diagonal arcs, and moreover around the crossing $c$ connecting the adjacent arcs forming the component, the bounded disk does not contain the other two arcs. 


Let $\Gamma_0, \Gamma_1$ be dotted graphs such that $\Gamma_0 \cap \Gamma_1$ consists of a finite number of transverse intersection points of arcs, avoiding dots. Regarding the intersection points as crossings, we have a dotted graph 
$\Gamma_0 \cup \Gamma_1$. 
Each region $R$ of $\Gamma_0 \cup \Gamma_1$ is described as $R_0 \cap R_1$, where $R_0$ (respectively $R_1$) is the region of $\Gamma_0$ (respectively $\Gamma_1$) containing an inner point of $R$, and the label of $R$ is the sum of the labels of $R_0$ and $R_1$. 
Let $X_1$ be the union of regions of $\Gamma_1$ such that any region  in $\mathbb{R}^2\backslash X_1$ has the label zero. 
We call the dotted graph $\Gamma_0 \cup \Gamma_1$ the dotted graph $\Gamma_0$ {\it overlapped by the regions $X_1$}, 
and we call $X_1$ {\it overlapping regions}. 
And for a region $R_0$ of $\Gamma_0$ such that $R_0 \cap X_1 \neq \emptyset$, we say that {\it $R_0$ is overlapped by $X_1$}, 
and we call $R_0$ the {\it overlapped region}. 
Similarly, by induction, we call the dotted graph $\Gamma=\Gamma_0  \cup \Gamma_1 \cdots \cup \Gamma_m$ the dotted graph $\Gamma_0$ {\it overlapped by regions $X_1, \ldots, X_m$}, where $X_i$ is the union of regions of $\Gamma_i$ such that any region  in $\mathbb{R}^2\backslash X_i$ has the label zero $(i=1, \ldots, m)$. 
The label of a region $R$ of $\Gamma$ is the sum of the labels of the corresponding regions in the overlapped/overlapping regions. 
When we want to distinguish the label of $R$ of $\Gamma$ from that of an overlapped/overlapping region, we call the former the {\it total label} of $R$. 

%$X_i$ is the union of overlapping regions of 
 
%When a union of regions of $\Gamma_2$ 
%For a region $R_1$ of $\Gamma_1$ and a region $R_2$ of $\Gamma_2$, we say that $R_1$ and $R_2$ have {\it overlapping} if 
%$R_1 \cap R_2 \neq \emptyset$. 

Let $\Gamma$ be a dotted graph. 
Let $D$ be a disk in $\mathbb{R}^2$. 
If there exists a dotted graph $\Gamma_0$ overlapped by some regions $X$, denoted by $\Gamma'$, such that $\Gamma \cap D=\Gamma' \cap D$ and $X \cap D \neq \emptyset$, then  
we say that $\Gamma \cap D$ is {\it overlapped by regions $X \cap D$}. 
And we call the part of regions $X \cap D$ the {\it overlapping regions}. 
 %$R_i$ is a region (or a union of several regions) of $\Gamma_i$ with $R_i \cap D \neq \emptyset$ $(i=1,2)$, then, 
%we say that the regions (or parts of regions) $R_1 \cap D$ and $R_2 \cap D$ have {\it overlapping} or $R_1 \cap D$ is overlapped by $R_2 \cap D$. 
%We call $R_1 \cap D$ or $R_2 \cap D$ an {\it overlapping region}. 
%Further, when $\Gamma \cap (R_1 \cap D)$ is an arc, we also say that the arc is {\it overlapped} by a region  $R_2 \cap D$. 


 

\begin{definition}\label{def3-2}
Let $\Gamma$ and $\Gamma'$ be dotted graphs. 
Then, 
$\Gamma'$ is obtained from $\Gamma$ by 
a {\it dotted graph deformation} or simply a  {\it deformation} if $\Gamma$ is changed to $\Gamma'$ by a local move in a disk $D\subset \mathbb{R}^2$, satisfying the following. 

\begin{enumerate}
\item
The boundary $\partial D$ does not contain vertices (dots and crossings) of $\Gamma$ and $\Gamma'$, and $\Gamma \cap \partial D$ (respectively $\Gamma' \cap \partial D$) consists of transverse intersection points of edges of $\Gamma$ (respectively $\Gamma'$) and $\partial D$. 


\item
The dotted graphs are identical in the complement of $D$: 
$\Gamma \cap (\mathbb{R}^2\backslash D)=\Gamma' \cap (\mathbb{R}^2\backslash D)$. 

\item
The dotted graph in $D$, $\Gamma \cap D$, is deformed to $\Gamma' \cap D$, by one of the following.
 
\begin{enumerate}

\item[(I)]
Reduce several dots on an arc to one dot on the arc; see Figure \ref{Fig3} (I).

\item[(II)]
Remove a circle component whose bounding disk has a nonzero label $\epsilon i$, where $\epsilon \in \{+1, -1\}$ and $i$ is a positive integer, and in the neighborhood of the disk, the complement has the  label $\epsilon (i-1)$, where we don't give a condition concerning the number of dots in the circle component: it can have dots, and we also include the case it has no dots; see Figure \ref{Fig3} (II).  

\item[(III)]
Remove a loop component whose bounding disk has a nonzero label  $\epsilon i$, where $\epsilon \in \{+1, -1\}$ and $i$ is a positive integer, and in the neighborhood of the disk, the region in the complement whose closure's boundary contains the loop component has the label $\epsilon (i-1)$, such that we deform the pair of adjacent arcs to one arc with a dot  (respectively, with no dots) if the loop component has dots (respectively, does not have dots); see Figure \ref{Fig3} (III). 

\item[(IV)]
A local move as illustrated in Figure \ref{Fig3} (IV), where we require that there is a dot on each arc, and the label $\epsilon i$ is not zero, and the arcs admit induced orientations. 
We call a part of the region before the deformation with the label $\epsilon i$ the {\it middle region}. 

\end{enumerate}
Further, 
a deformation II/III/IV can be applied when the concerning regions are overlapped by several regions such that each overlapping region has the label $\epsilon $. 
See Remark \ref{rem902}. 
\end{enumerate}

Further, we say that $\Gamma'$ is obtained from $\Gamma$ by 
a {\it local move} or a {\it deformation} $\mathcal{E}$ if $\Gamma$ is changed to $\Gamma'$ by a local move in a disk $D\subset \mathbb{R}^2$, satisfying the following. 
We have the situation (1) and (2), and that $\Gamma \cap D$ is an arc as in the left figure of Figure \ref{Fig3} (I), 
where we don't give a condition for the number of dots. 
Further, we assume that the concerning regions are overlapped by several regions $R$ such that each overlapping region has the label $\epsilon$. 
Then, $\Gamma \cap D$ is deformed to $\Gamma'\cap D$ by moving the arc 
 by an ambient isotopy of $D$ ignoring $R$ such that the result does not create loop components. 

\begin{enumerate}

\item[($\mathcal{E}$)] 
Move an arc with/without dots  
 by an ambient isotopy of $D$ ignoring overlapping regions $R$ such that the result does not create loop components; where $R$ satisfies the condition given above. 
In particular, move a dot ignoring $R$. 
\end{enumerate}


\end{definition}

\begin{figure}[ht]
\centering
\includegraphics*[height=5cm]{B-fig16-3}
\caption{Local deformations I--IV, where $\epsilon \in \{+1, -1\}$ and $i$ is a positive integer, and we omit the orientations of the arcs and some of the labels of the regions. A deformation IV is  applicable when the arcs admit induced orientations. 
Deformations II, III, IV can be applied including the case when the regions are overlapped by several regions such that each overlapping region has the label $\epsilon$.}  %Fig3(label), B-fig16-3}
\label{Fig3}
\end{figure}



\begin{definition}\label{def3-3}
Let $\Gamma$ be a dotted graph. We say that $\Gamma$ is  {\it reducible} if we cannot apply deformations I--IV to $\Gamma$. 
And we call a dotted graph $\Gamma'$ a {\it reduced graph} of 
$\Gamma$ if $\Gamma'$ is not reducible and $\Gamma'$ is obtained from $\Gamma$ by a finite sequence of deformations I--IV. 


Further, we consider a specific deformation IV, called a {\it deformation IVa}, that satisfies one of the following conditions (a1) and  (a2):

\begin{enumerate}

\item[(a1)]
The arcs involved in $\mathcal{R}$ are adjacent arcs of a crossing, where we ignore overlapping regions, such that $\mathcal{R}$ creates a loop component applicable of a deformation III. 

\item[(a2)]
The deformation $\mathcal{R}$ creates a circle component $C$ from two concentric circle components such that a deformation II is applicable to $C$. 
\end{enumerate}

We call deformations I, II, III and IVa {\it good deformations}. 
And we say that a sequence of deformations I, II, III, IVa is {\it in  good order} if it satisfies the following rule: 
\begin{enumerate}
\item[(1)]
After a deformation IVa satisfying (a1), next comes the deformation III that deletes the created loop component. 

\item[(2)]
After a deformation IVa satisfying (a2), next comes the deformation II that deletes the created circle component. 
\end{enumerate}

And we call a dotted graph $\Gamma'$ a {\it good reduced graph} of 
$\Gamma$ if $\Gamma'$ is not applicable of good deformations I, II, III, IVa, and $\Gamma'$ is obtained from $\Gamma$ by a finite sequence of good deformations I, II, III, IVa in good order.  


\end{definition}

See \cite[Figure 6.2]{N} for an example of a non-empty reduced graph. 



\begin{proposition}\label{prop919}
Let $P$ be a lattice polytope, and let $\Gamma$ be the dotted graph associated with $P$. 
Then, if $\Gamma$ is deformed to a dotted graph $\Gamma'$ by a sequence of good deformations I, II, III, IVa in good order, then, 
there exists a sequence of normal/reversed transformations (see Definition \ref{def818}) of lattice polytopes from $P$ to a lattice polytope $P'$ whose associated dotted graph is $\Gamma'$, such that each transformation along a rectangle changes the area of the rectangle from $\epsilon i$ to $\epsilon (i-1)$ for some positive integer $i$ and $\epsilon\in \{+1, -1\}$. 
\end{proposition}

\begin{proof}
The argument in the proof of Theorem \ref{thm3-7} implies the required result. See the proof of Theorem \ref{thm3-7}. 
\end{proof}
 
We remark that transformations of a lattice polytope along a rectangle satisfying the condition given in Proposition \ref{prop919} form a transformation of a lattice polytope with minimal area.  \\


Let $\mathcal{R}$ be a deformation IV and let $p_1$ and $p_2$ be the involved dots. The result of $\mathcal{R}$ is described as the result of band surgery \cite{Kawauchi} along an untwisted band, and the band is presented by a core between $p_1$ and $p_2$, that is a simple arc in the middle region connecting $p_1$ and $p_2$; see the proof of Lemma \ref{lem826} for the precise definition of a core. When the middle region has overlapping regions, we have choices of the band and the core, but the result of $\mathcal{R}$ is determined up to local moves $\mathcal{E}$. 

For a dotted graph $\Gamma$, we consider the following condition. 
Let $p_1$ and $p_2$ be a pair of dots of $\Gamma$, between which a deformations IV is applicable. 
We say that $\Gamma$ {\it satisfies the condition $\mathrm{(A)}$ with respect to $(p_1, p_2)$} if it satisfies the following condition: 
\begin{enumerate}
\item[(A)]
%Let $p_1$ and $p_2$ be a pair of dots of $\Gamma$. 
Let $\rho$ and $\rho'$ be cores of bands connecting $p_1$ and $p_2$, 
and let $\Gamma'$ (respectively $\Gamma''$) be the dotted graph obtained from $\Gamma$ by the deformation IV associated with $\rho$ (respectively $\rho'$). 
Then, $\Gamma'$ and $\Gamma''$ are related by local moves $\mathcal{E}$, for all possible cores $\rho, \rho'$. 
\end{enumerate}

And we say that $\Gamma$ {\it satisfies the condition} (A) if it satisfies the condition (A) with respect to all pairs of dots. 
Further, we say that a sequence of deformations {\it satisfies the condition} (A) if any appearing dotted graph satisfies the condition (A). 

We remark in an imprecise expression that if the \lq\lq middle region'' contains connected components that cannot be regarded as overlapping regions, then the dotted graph does not satisfy the condition (A).  
 
\begin{theorem}\label{prop3-5}
Let $\Gamma$ be a dotted graph, and let $\Gamma'$ be a reduced graph of $\Gamma$ whose associated sequence of deformations  satisfies the condition $\mathrm{(A)}$, and the sequence does not contain a deformation IV applied between a circle/loop component $C$ applicable of a deformation II/III and an arc of an overlapping region of $C$. 
Then, $\Gamma'$ is uniquely determined up to local moves $\mathcal{E}$ and deformations I. 
\end{theorem}


 
 
 

Let $\mathcal{R}$ be a deformation consisting of deformations I--IV satisfying the following property. 
\begin{enumerate}
\item[(X)]
Let $\Gamma$ be a dotted graph and let $\Gamma'$ be the dotted graph obtained from $\mathcal{R}$. Then, the number of dots of $\Gamma'$ is equal to or less than that of $\Gamma$, and the number of crossings of $\Gamma'$ is equal to or less than $\Gamma$, for any dotted graph $\Gamma$ applicable of $\mathcal{R}$. 
\end{enumerate}
We call a deformation satisfying property (X) a {\it deformation X}. Then we have the following theorem; see Remark \ref{rem920}. 

\begin{theorem}\label{r-prop3-5}
For a dotted graph $\Gamma$, any sequence of deformations consisting of deformations X is finite. In particular, there exists a good reduced graph of $\Gamma$.
\end{theorem}


 

\section{Proofs of Theorems \ref{prop3-5} and \ref{r-prop3-5}}\label{sec-proof}

 
We modify the proof Lemma 6.2 in \cite{N}. See also Remark  \ref{rem902}. 

 
\begin{proof}[
Proof of Theorem \ref{prop3-5}]
By Lemma \ref{rem915}, 
it suffices to show that 
(1) when we have graphs $\Gamma_1$ and $\Gamma_2$ obtained from  $\Gamma$ by a finite sequence of deformations I--IV and local moves $\mathcal{E}$ satisfying the condition (A), $\Gamma_1$ and $\Gamma_2$ can be deformed by deformations I--IV and local moves $\mathcal{E}$ to the same dotted graph, where we don't apply a deformation IV between the arc used in a local move $\mathcal{E}$ and an arc of its overlapping region.  
If there are two dots $p_1$ and $p_2$ on an arc $\alpha$, we can apply a deformation IV between $p_1$ and $p_2$. By the condition of labels for the overlapping regions of the middle region and the condition (A), we see that any deformation IV between $p_1$ and $p_2$ deforms $\alpha$ into an arc $\alpha'$ and a circle component $C$, where $\alpha'$ is an arc with a dot obtained from  $\alpha$ by a deformation $\mathcal{E}$ and a deformation I, and $C$ is a circle component with a dot applicable of a deformation II; we remark that by the former condition, we see that any core associated with the deformation IV, which is an arc connecting $p_1$ and $p_2$ with no self-intersections, does not intersect with $\alpha$ except at the endpoints $p_1$ and $p_2$. 
And if there is a pair of arcs $\alpha, \alpha'$ with dots where deformations IV are applicable, 
then, no matter what times we apply deformations IV, the resulting dotted graphs can be deformed to the same dotted graph as when we consider $\alpha, \alpha'$ with one dot on each arc; see the left figure of Figure \ref{B-fig11}. 
Similarly, if there is an arc $\alpha$ with dots and several arcs where deformations IV are applicable between $\alpha$ and the other arcs, 
then, no matter what times we apply deformations IV between $\alpha$ and the other arcs, the resulting dotted graphs can be deformed to the same dotted graph as when we consider $\alpha$ with one dot; see the right figure of Figure \ref{B-fig11}. 
Further, deformations II and III are not effected by the number of dots. 
So we can assume that there is at most one dot in each arc. See also the proof of Lemma \ref{lem826}. 
 
If we have arcs where both deformations II and IV are applicable, then, since by assumption we don't have a deformation IV applied between a circle component $C$ applicable of a deformation II and an arc of an overlapping region of $C$, the resulting dotted graphs can be deformed to the same dotted graph; see Figure \ref{B-fig12}. We remark that we don't have a deformation IV between the arc used in the local move $\mathcal{E}$ and an arc of its overlapping region. 
And by Lemma \ref{lem913}, there do not exist arcs where both deformations III and IV are applicable. 

When a deformations II or III can be applied, and the interior of the disk $D$ bounded by the circle/loop component of the deformations II or III contains arcs where a deformation IV is applicable, since any region in $D$ 
has a nonzero total label, by Claim \ref{rem2-6}, after the deformation IV the regions in $D$ also have nonzero total labels and the deformation II or III is applicable after the deformation IV. 
Further, by Lemma \ref{lem912}, when we have a circle/loop component $C$ applicable a deformation II/III  such that the interior of the disk bounding $C$ contains another circle/loop component $C'$ applicable of a deformation II or III, the result of the deformations II/III to $C$ and then $C'$ is the same with that of the deformations II/III to $C'$ and then $C$, up to local moves $\mathcal{E}$. 
Hence we see the following. We consider a dotted graph $\Gamma$ that deforms to $\Gamma'$ by a deformation II or III. 
We denote by $G$ the circle/loop components and arcs of $\Gamma$ which are applicable of deformations I--IV in $\Gamma$ and $G \cap \Gamma'=G$. Then $G$ in $\Gamma'$ are applicable of the same deformations I--IV, where two deformations are the {\it same} if they are the same as deformations of graphs; we remark that the labels of the regions might differ.  

Further, if there are plural circle/loop components applicable of deformations II or III that have overlapping with each other, since the regions of loop/circle components have all positive/negative labels, we can delete all components applying deformations II or III, and the result is independent of the order of deformations. 

Next we consider deformations IV. 
By taking the middle region to be sufficiently thin, we assume that the middle region does not contain circle/loop components. 
We recall that the sequence of deformations satisfies the condition (A). 
%When the middle region $R$ is overlapped by regions in $\mathrm{Int}(R)$ associated with a connected component of the dotted graph, we have choices of the core connecting the two dots of the deformation IV such that the results might not be related by local moves $\mathcal{E}$. However, this situation does not occur by the assumption that the sequence of deformations satisfies the condition (A). 

%From now on, we assume that middle regions do not contain in their interiors connected components. 
We consider the case when we have several choices of deformations IV. 
Suppose that there are three dots $p_1, p_2, p_3$ and we 
have two choices of applying a deformation IV: between $(p_1, p_2)$ or $(p_1, p_3)$. Then we can apply a deformation IV also to the other pair of dots $(p_2, p_3)$, and the result of each choice can be deformed to the same dotted graph (up to local moves $\mathcal{E}$), as shown in the top and middle rows of figures in Figure \ref{C-fig2};  
 we remark that in the middle figure of the top row in Figure \ref{C-fig2}, we cannot apply a deformation IV to a pair of dots other than given in the figure, by seeing the labels of the regions. 

Thus, we see that if we have several possible parts of a dotted graph $\Gamma$ to each of which a deformation I--IV, denoted by $\mathcal{R}_i$, is applicable, then, for any $i$, the dotted graph $\Gamma'_i$ obtained by $\mathcal{R}_i$ can be deformed to the same dotted graph by deformations I--IV and $\mathcal{E}$. 
This implies (1), and we have the required result. 
\end{proof}

\begin{figure}[ht]
\centering
\includegraphics*[height=4.5cm]{E-fig3}
\caption{If there is a pair of arcs $\alpha, \alpha'$ with dots where deformations IV are applicable, 
then, no matter what times we apply deformations IV, the resulting dotted graphs can be deformed to the same dotted graph as when we consider $\alpha, \alpha'$ with one dot on each arc (left figure). If there is an arc $\alpha$ with dots and several arcs where deformations IV are applicable between $\alpha$ and the other arcs, 
then, no matter what times we apply deformations IV between $\alpha$ and the other arcs, the resulting dotted graphs can be deformed to the same dotted graph as when we consider $\alpha$ with one dot (right figure). We omit orientations of arcs and labels of regions.} %E-fig3 (label B-fig11)}
\label{B-fig11}
\end{figure}

\begin{figure}[ht]
\centering
\includegraphics*[height=6.5cm]{F-fig3}
\caption{If we have arcs where both deformations II and IV are applicable, then the resulting dotted graphs can be deformed to the same dotted graph, where we omit the orientations of arcs and labels of regions; we remark that in the lower figure, the shadowed region is the overlapping region. We also remark that the upper figure is \cite[Figure 6.4]{N}.}% F-fig3 (label B-fig12)} 
\label{B-fig12}
\end{figure}

 

\begin{figure}[ht]
\centering
\includegraphics*[height=7cm]{C-fig2}
\caption{A deformation IV can be regarded as the result of band surgery along an untwisted band (see the proof of Lemma \ref{lem826}). In the top and the bottom rows of figures, we denote by dotted arcs cores of bands. We assume that each appearing dotted graph satisfies the condition (A) (see Theorem \ref{prop3-5}). 
If there are $n$ dots between which there are several possible sequences of $n-1$ deformations IV, then the result is independent of the choice of sequence up to local moves $\mathcal{E}$ (the middle row). The deformation sequence can be regarded as band surgery along mutually disjoint bands such that the endpoints of cores are in the neighborhoods of dots, where we omit dots in the figures (the bottom row). }% C-fig2 (label);C-fig2}
\label{C-fig2}
\end{figure}


 

\begin{proof}[
Proof of Theorem \ref{r-prop3-5}]

 

Let $\Gamma=\Gamma_0, \Gamma_1, \Gamma_2, \ldots, \Gamma_n$ be dotted graphs that appear in a sequence of deformations X. 
We denote by $\mathcal{R}_i$ the deformation such that $\Gamma_{i+1}$ is obtained from $\Gamma_i$ by $\mathcal{R}_i$. 
 
When $\mathcal{R}_i$ is a deformation I, then the number of dots in ${\Gamma}_{i+1}$ is less than that of ${\Gamma}_{i}$, and the number of crossings of ${\Gamma}_{i+1}$ equals that of ${\Gamma}_{i}$. 
When $\mathcal{R}_i$ is a deformation III, then the number of dots of  ${\Gamma}_{i+1}$ is equal to or less than that of ${\Gamma}_{i}$, and the number of crossings of ${\Gamma}_{i+1}$ is less than that of ${\Gamma}_{i}$. 
Thus, in any sequence of deformations I--IV, the number of deformations I and III is finite. 

Since a deformation II decreases the number of circle components, 
if there exists a sequence consisting of deformations II, the number of deformations is finite. 
And if a deformation IV creates a circle component applicable of a deformation II, then the circle component has a dot; hence the resulting graph can be deformed by a deformation II to a dotted graph the number of whose dots is less than that of the initial dotted graph, and the number of whose crossings is equal to or less than that of the initial dotted graph. And we have a similar situation if a deformation III creates a circle component applicable of a deformation II. Thus, in any sequence of deformations I--IV, the number of deformations II is also finite. 

Hence, since a deformation X consists of deformations I--IV, 
it suffices to show that for an arbitrary dotted graph $\Gamma$, there does not exist a sequence of deformations consisting of 
infinite number of deformations IV. 
This is shown in Lemma \ref{lem826}. Thus, for a dotted graph $\Gamma$, any sequence of deformations X applicable to $\Gamma$ is finite. 

Now we show that good deformations I, II, III, and IVa satisfy the property (X). We see that deformations I, II, and III satisfy the property (X). From now on, we show that a good deformation IVa satisfies the property (X). 
We call a deformation IVa satisfying the condition (a1) (respectively (a2)) a deformation IV (a1) and (respectively IV (a2)). For a deformation IV (a1), we take the core sufficiently near the adjacent edges of the crossing. 
When $\mathcal{R}_i$ is a deformation IV (a1), then, the number of dots of $\Gamma_{i+2}$ is equal to or less than that of ${\Gamma}_{i}$, and the number of crossings of $\Gamma_{i+2}$ 
is less than that of $\Gamma_i$. When $\mathcal{R}_i$ is a deformation IV (a2), then, the number of dots of $\Gamma_{i+2}$ is less than that of ${\Gamma}_{i}$, and the number of crossings of $\Gamma_{i+2}$ 
is less than or equal to that of $\Gamma_i$. 
Thus a deformation IVa is a deformation X. 
Hence, for a dotted graph $\Gamma$, there exists a good reduced graph of $\Gamma$, which is the required result. 
\end{proof}


 
\begin{remark}\label{rem920}
We remark that by the same argument in the proof of Theorem \ref{r-prop3-5}, we can see the following. 
We denote by a deformation IVb a deformation IV that does not have overlapping regions. 
Then, for a dotted graph $\Gamma$, we can construct a sequence of deformations consisting of a finite number of deformations I, II, III, and IVb, so that the result is a dotted graph to which we cannot apply deformations I, II, III, and IVb. 
\end{remark}

 
 

\section{Deformations of admissible dotted graphs and transformations of lattice polytopes}\label{deformation}

In this section, we investigate relation between deformations of admissible dotted graphs and transformations of lattice polytopes. We show that if an admissible dotted graph is deformed to the empty graph by a sequence of good deformations in good order, then its associated lattice polytope has a transformation with minimal area, and we obtain deformations of dotted graphs describing a transformation of lattice polytopes with minimal area (Theorem \ref{thm3-7}). 
Further, we show that if a dotted graph has \lq\lq many'' dots, then it is admissible (Proposition \ref{prop3-8}), and a dotted graph with ``many'' dots is reduced to the empty graph (Proposition \ref{prop3-9}). 

We give deformation I*--IV* with a weaker condition for application than given in Definition \ref{def3-2}, as follows. %and a new deformation V, as follow. 

\begin{definition}\label{def3-7}
We use the notations in Definition \ref{def3-2}. 
For a dotted graph, 
we define {\it deformations I*, II*, III*, IV*}, as 
deformations I, II, III, IV, respectively, 
such that we allow overlapping regions for deformations I and for the other deformations change the condition for overlapping regions as follows.  


\begin{enumerate}


\item[(Y*)]
A deformation Y* (Y=I, II, III, IV) can be applied when the concerning regions are overlapped by some regions 
such that each region in the region with the label $\epsilon i$ in Figure \ref{Fig3} has a nonzero total label. 
Further, we give an additional condition that for a deformation II*/III* (respectively IV*), the union of the closures of overlapping regions does not contain in its interior the disk bounded by the circle/loop component (respectively the middle region). 

%\item[(V)]
%A local move as illustrated in Figure \ref{Fig3} (V), where we require that there is a dot on each arc, and the label $\epsilon i$ is not zero, and the arcs admit induced orientations. For this deformation, we don't allow overlapping regions. 
 
\end{enumerate}
\end{definition}

\begin{claim}\label{rem2-6}
%We consider a deformation II/III 
Let $D$ be the disk bounded by a circle/loop component or the middle region where we can apply a deformation II*, III*, IV*: before applying the deformation, among the overlapping regions 
in $D$ there does not exist a region with the total label zero; further, 
the union of overlapping region does not contain in its interior the disk or the middle region $D$. 

Then, the total labels of the regions in the intersection of overlapping regions and $D$ are all positive or all negative, and the total label of the region does not change (respectively changes from $\epsilon j$ to $\epsilon (j-1)$ for some positive integer $j$ and $\epsilon \in \{+1, -1\}$) by the deformation if it is in the overlapped region that does not change (respectively changes from $\epsilon i$ to $\epsilon (i-1)$ for some positive integer $i$) by the deformation. 
\end{claim}
 
\begin{proof}
We show the case for a deformation IV; the other cases are shown similarly. 
Let $\epsilon i$ be the label of $D$, and let $x$ be a point in $D$ such that $x$ is not in the overlapping regions. 
Let $\rho$ be a path in $D$ connecting $x$ and a point in $D$. Then, the labels of the regions crossed by $\rho$ form a sequence beginning from $\epsilon i$ such that the adjacent labels differ by $\pm 1$. Since there does not exist a region with the total label zero among the overlapping regions in $D$, and the total label of the region containing $x$ is $\epsilon i$, the sequence does not contain zero, and hence all positive if $\epsilon=+1$, or negative if $\epsilon=-1$. 
Since we can consider $\rho$ that intersects with any region in $D$, we have the result for the sign of the labels of regions in $D$. 
Further, we can see the total labels satisfy the required condition. 
Hence we have the required result. 
\end{proof}

Besides deformations I*--IV* given in Definition \ref{def3-7},  we consider other new types of deformations: we call a deformation as in Figure \ref{C-fig9} (V) a {\it deformation V}, and we give a {\it local move $\mathcal{E}^*$} as a local move induced from the definition of $\mathcal{E}$ by changing the condition of overlapping regions to that of I* and moreover removing the condition that it does not create loop components.   
See Remark \ref{rem902} (4). 



\begin{theorem}\label{thm3-7}
Let $\Gamma$ be an admissible dotted graph associated with a lattice polytope $P$. 
Then we have the following. 
\begin{enumerate}[$(1)$]


\item
If $\Gamma$ has a good reduced graph that is the empty graph, then there exists a transformation of $P$ with minimal area, that is, a transformation satisfying the equality of (\ref{eq2}). 

\item
If there exists a transformation of $P$ with minimal area, then $\Gamma$ is deformed to the empty graph by a sequence of deformations II*, III*, IV*, V and $\mathcal{E}^*$.  

\end{enumerate}
\end{theorem}

\begin{proof}


First we show (1). 
Suppose that $\Gamma$ has a good reduced graph that is the empty graph. 

By an argument as in the proof of \cite[Theorem 5.9]{N}, for each deletion of a circle/loop component, there exists a corresponding transformation of the lattice polytope realizing the minimal area; here we remark that in the case of a loop component, the transformation of the lattice polytope might consists of normal transformations and reversed transformations; see Definition \ref{def818} for these notions. 


When we have a deformation IVa satisfying the condition (a2) that creates a circle component $C$ from two concentric circle components and then a deformation II that deletes $C$, the lattice polytope satisfies the condition (2) in  \cite[Theorem 5.9]{N}, and hence there exists a transformation of the lattice polytope that realizes the deformations II and III. 
For each deformation IVa satisfying the condition (a1) that is applied between adjacent arcs of a crossing, there exists a normal/reversed transformation of the lattice polytope that realizes the deformation IVa by Lemma \ref{lem-822}, that is the transformation as shown in Figure \ref{F-fig1}. We denote the resulting lattice polytope by $P$. Then, as shown in Figure \ref{F-fig1}, $P$ is not the result of the transformation of the dotted graph, denoted by $G$, but 
since the deformation sequence is in good order, we delete the loop component of $G$ by a deformation III, and the result is the dotted graph associated with $P$. 

Hence, by Lemma \ref{lem-821}, we have a transformation of the lattice polytope with minimal area, consisting of only normal transformations. Thus we have the required result. 

Next we show (2). Suppose that there exists a transformation of $P$ with minimal area. Then translate the transformation by its associated dotted graphs. 
Since the transformation of $P$ has minimal area, each transformation along a rectangle $R$ satisfies that any region in $R$ has a nonzero label and in the resulting lattice polytope, each region in $R$ has a label whose absolute value is less than that of the former region by one. 
We see that each transformation of a lattice polytope along a rectangle is presented by one of deformations I*, II*, III*, IV*, V, or deformations as in Figure \ref{F-fig1} or Figure \ref{C-fig9} (VI) (and their mirror images). 
We draw in Figure \ref{B-fig17} transformations of a lattice polytope whose associated dotted graphs are related by deformations I*--IV*. 
Since we have a sequence of dotted graphs from $\Gamma$ to the empty graph, we can ignore deformations I*. The dots will be deleted by deformations II* and III*. 
A deformation as in Figure \ref{F-fig1} is realized by a sequence of deformations IV* and III*, 
A deformation VI is realized by a sequence of deformations I*, III* and $\mathcal{E}^*$; see Figure \ref{B-fig18}. 

Thus, we have a sequence of deformations from $\Gamma$ to the empty graph, consisting of deformations II*, III*, IV*, V and $\mathcal{E}^*$; which is the required result. 
\end{proof}

\begin{figure}[ht]
\centering
\includegraphics*[height=4cm]{F-fig1}
\caption{Transformations of a lattice polytope along a rectangle   and the corresponding sequence of deformations IVa and III in good order, where we omit orientations of edges and labels of regions. In the figure, (IV) can be (IVa) or (IV*), and (III) can be (III*).}%  F-fig1 (label); F-fig1}
\label{F-fig1}
\end{figure}

\begin{figure}[ht]
\centering
\includegraphics*[height=5cm]{C-fig9}
\caption{Transformations V and VI of lattice polytopes along a rectangle, where we omit orientations of edges and labels of regions; in the bracket we draw a deformation V of a dotted graph, where $\epsilon \in \{+1, -1\}$ and $i$ is a positive integer. }%C-fig9 (label); C-fig9}
\label{C-fig9}
\end{figure}




\begin{figure}[ht]
\centering
\includegraphics*[height=4cm]{C-fig7}
\caption{Transformations of a lattice polytope along a rectangle corresponding to deformations I*--IV*, where in the figures we omit the index \lq\lq $*$'', and we omit orientations of edges and labels of regions.}%  B-fig17 (label); C-fig7}
\label{B-fig17}
\end{figure}



\begin{figure}[ht]
\centering
\includegraphics*[height=6cm]{E-fig1-2}
\caption{A transformation VI is represented by a sequence consisting of deformations I*, III*, $\mathcal{E}^*$, where in the figures we omit the index \lq\lq $*$'' and we omit orientations of edges and labels of regions. }%B-fig18(label); E-fig1-2}
\label{B-fig18}
\end{figure}


 
A dotted graph is admissible when it has \lq\lq many'' dots. 

\begin{proposition}\label{prop3-8}
Let $\Gamma$ be a dotted graph such that each arc has at least two dots. Then, $\Gamma$ is admissible. 
\end{proposition}

\begin{proof}
For a lattice polytope $P$, we call the part of $P$ corresponding to an arc (respectively a crossing) of its associated dotted graph an {\it arc} (respectively a {\it crossing}) of $P$. %a set of edges connected by vertices of degree $2$ an {\it arc}, and if a pair of edges has an intersection point in the interior of edges, we call the intersection point a {\it crossing}, and we separate each edge forming the crossing into two arcs connected by the crossing. 

When we consider an arc of a lattice polytope with at least two initial vertices, we can construct an arc $\alpha$, connecting a given pair of distinct points, such that the edges connecting to the points are in given directions, that is, $\alpha$ consists of edges such that the initial edge is in the $a$-direction with orientation coherent/incoherent with the $a$-axis, and the terminal edge is in the $b$-direction with orientation coherent/incoherent with the $b$-axis, for any given pair $(a,b)$ and orientations ($a,b \in \{x,y\}$). Hence, by fixing each crossing of $\Gamma$ to be a crossing consisting of edges in the $x$-direction and $y$-direction, 
we can construct the corresponding lattice polytope. Thus $\Gamma$ is admissible. 
\end{proof}

\begin{proposition}\label{prop3-9}
Let $\Gamma$ be a dotted graph such that each arc has at least one dot. Then, the reduced graph of $\Gamma$ is the empty graph. 
Further, if $\Gamma$ is admissible, then any lattice polytope associated with $\Gamma$ has a transformation with minimal area. 
%
%$\Gamma$ has a good reduced graph that is the empty graph. In particular, 
\end{proposition}

\begin{proof}
We show that (1) we can deform $\Gamma$ by good deformations I, II, III, IVa in good order to a dotted graph $\Gamma'$ consisting only of mutually disjoint circle components such that each circle component has at least one dot, and (2) we can deform the dotted graph $\Gamma'$ by deformations I--IV to the empty graph, and (3) if $\Gamma$ is admissible, then, for a lattice polytope whose associated dotted graph is $\Gamma'$, there exists a transformation with minimal area. Then, together with Proposition \ref{prop919}, we have the required result. 

We show (1). 
Since each arc has a dot, 
We see that around each crossing, we have a pair of adjacent arcs between which a deformation IVa is applicable, if they don't form a loop component. 
So we apply deformations IVa for such arcs, and then we apply deformations III to delete the created loop components. Then we have a dotted graph consisting of circle components that are mutually disjoint and each circle component has at least one dot. 

We show (2). 
From now on, we call a circle component simply a {\it circle}. 
We consider a set of concentric circles in $\Gamma'$ from an outermost circle to an innermost circle. Then, let $\rho$ be a path from the outermost region to the innermost disk such that $\rho$ crosses each concentric circle exactly once. 
Then, the labels of the regions crossed by $\rho$ form a sequence of integers from zero to some $\epsilon n$ for a positive integer $n$ and $\epsilon \in \{+1, -1\}$. If the sequence does not contain zero other than the initial zero, then, $n$ is not zero and the sequence ends with $\epsilon (n-1), \epsilon n$; hence we apply a deformation II and delete the innermost circle. 



If the sequence contains zero other than the initial zero, then we have a subsequence $0, \epsilon 1, \epsilon 2, \ldots, \epsilon (n-1), \epsilon n, \epsilon (n-1), \ldots, \epsilon 2, \epsilon 1, 0$, for a positive integer $n$ and  $\epsilon \in \{+1, -1\}$. Let $C_1$ and $C_2$ be the concentric circles  whose arcs bounds the region with the label $\epsilon n$ such that $C_1$ is the outer circle. We denote by $R$ the region with the label $\epsilon n$. 
The boundary of the closure $R$, denoted by $\partial R$, consists of $C_1$, $C_2$ and several circles, denoted by $C_3, \ldots, C_m$. 

(Case 1)
If the circles other than $C_1$ have the same orientations, then the regions adjacent to $R$ in the disks bounded by $C_2, \ldots, C_m$ have the label $n-1$, and we apply deformations IV between $C_1, \ldots, C_m$ to make $\partial R$ into one circle and then we apply a deformation II to delete $R$. 

 

(Case 2)
If there exists a circle $C_j$ $(j=1, \ldots, m)$ with orientation opposite to that of $C_2$, then, the region adjacent to $R$ in the disk bounded by $C_j$ has the label $\epsilon(n+1)$, and we take a new path $\rho$ from $R$ to an innermost circle in the disk bounded by $C_j$. If the sequence of the labels does not contain $\epsilon n$ other than the initial $\epsilon n$, then, we can delete the innermost circle by a deformation II. And if the sequence of the labels contains $\epsilon n$ other than the initial $\epsilon n$, then, we consider the circles bounding the region the absolute value of whose label is the largest, and repeat the same process as above. 
Since the circles of $\Gamma'$ are finite, 
by repeating this process, we can delete circles by the method described in (Case 1) or by a deformation II, until we have the empty graph. 

We show (3). 
Suppose that $\Gamma'$ is admissible. 
We recall that for a lattice polytope $P$ whose dotted graph contains a circle component $C$ applicable of a deformation II, there exists a sequence of transformations of the lattice polytopes from $P$ to $P\backslash C$ realizing the minimal area, where we denote by the same notation $C$ the part of $P$ corresponding to the circle component $C$. 

We consider the deformation described in (Case 1) in the above argument. 
Let $P$ be a lattice polytope whose dotted graph is $\Gamma'$. We denote by the same notation $\partial R$ the part of $P$ corresponding to $\partial R$ in $\Gamma'$. 
Then, by \cite[Theorem 5.9]{N}, there exists a sequence of transformations of the lattice polytopes from $P$ to $P\backslash \partial R$ realizing the minimal area. 
Thus, by the argument concerning (2), we see that there exists a transformation of $P$ with minimal area. 
Thus we have the required result. 
\end{proof}

By the proof of Proposition \ref{prop3-9}, we have the following. 
\begin{proposition}
Let $\Gamma$ be a dotted graph such that each arc has at least one dot. Then, $\Gamma$ is deformed by a sequence of good deformations I, II, III, IVa in good order to a dotted graph $\Gamma'$ consisting only of mutually disjoint circle components such that each circle component has at least one dot.  
\end{proposition}


 

\section{Remark and Lemmas}\label{sec-lemma}



\begin{remark}\label{rem902}
Let $\Gamma$ be a dotted graph. 
\begin{enumerate}[(1)]

\item
If we use deformations II* and III* instead of deformations II and III in Theorem \ref{prop3-5}, then deformations $\mathcal{E}$ might effect the results of deformations II/III and IV. 
Similarly, 
if we use deformations IV* instead of deformations IV in Theorem \ref{prop3-5}, then the commutativity of the order of deformations IV  and Lemma \ref{lem0901} do not hold. 

\item
Suppose that we introduce a deformation III' that deletes a loop component with no condition for the labels of regions. Then, an application of a deformation IV to a loop component $C$ is possible where $C$ is applicable of a deformation III', and the result of a deformation IV is the result of a deformation III' and a deformation V as in Figure \ref{C-fig9}; see Figure \ref{B-fig14}, and we cannot use the argument in the proof of Theorem \ref{prop3-5}. 

\item
If we include deformations V and $\mathcal{E}$ besides deformations I--IV, there exists a dotted graph $\Gamma$ and a sequence of deformations whose result is $\Gamma$ itself; there exists an infinite loop of deformations, and we cannot have a reduced graph, and we cannot use the argument in the proof of Theorem \ref{r-prop3-5}. For example, see Figure \ref{C-fig4}. 

\item
Concerning Theorem \ref{thm3-7} (2), 
by a more specified argument, we can take a new deformation sequence consisting of deformations I*--IV*, V, $\mathcal{E}^{**}$ such that each local move $\mathcal{E}^{**}$ satisfies the condition that it does not create loop components. 
\end{enumerate}
\end{remark}

 


\begin{figure}[ht]
\centering
\includegraphics*[height=4.5cm]{E-fig2}
\caption{If we have arcs where both deformations III' and IV are applicable, then the resulting dotted graphs need a deformation V to be deformed locally to the same dotted graph, where we omit the orientations of arcs. We remark that by the condition for labels of regions, a deformation III is not applicable. }%(label B-fig14)E-fig2}
\label{B-fig14}
\end{figure}

\begin{figure}[ht]
\centering
\includegraphics*[height=3.5cm]{C-fig4}
\caption{There exist a dotted graph $\Gamma$ admitting a sequence of deformations V and $\mathcal{E}$ whose result is $\Gamma$ itself.}% C-fig4 (label);C-fig4}
\label{C-fig4}
\end{figure}

\begin{lemma}\label{rem915}
 In the proof of Theorem \ref{prop3-5}, (1) implies the uniqueness of the reduced graph up to local moves $\mathcal{E}$ and deformations I. 
\end{lemma}
\begin{proof}

It suffices to show that the possibilities of application of deformations II--IV are the same before and after the application of a local move $\mathcal{E}$, where we don't apply deformations IV between the arc used in the local move $\mathcal{E}$ and an arc of its overlapping region. 
Let $\Gamma$ be a dotted graph and let $\Gamma'$ be the dotted graph obtained from $\Gamma$ by applying a local move $\mathcal{E}$. 
We denote the local move $\mathcal{E}$ by $\mathcal{R}$, 
and we denote the concerning arcs before and after the application by $\alpha$ and $\alpha'$, respectively.  
Further, we denote by $R_0$ the region with the label $\epsilon i$ in Figure \ref{Fig3} (I) (with/without dots) in Definition \ref{def3-2}, and we denote by $X_1, \ldots, X_m$ the overlapping regions such that each $X_i$ $(i=1,\ldots,m)$ has the label $\epsilon$, and we put $R=R_0\cap (X_1 \cup \ldots \cup X_m)$.  

By seeing the labels of regions, we see that 
 if a deformation IV is applicable to the arc $\alpha$ (respectively $\alpha'$) and some arc $\alpha''$,  then a corresponding deformation IV between the arc $\alpha'$ and $\alpha''$ (respectively $\alpha$ and $\alpha''$) is also applicable. 
And if $\mathrm{Int}(R)$ contains a middle region applicable of a deformation IV, denoted by $D$, we can also apply a deformation IV to $D$ after the application of $\mathcal{R}$. 
Thus, 
$\Gamma'$ (respectively $\Gamma$) does not admit a deformation IV that is not induced from one of those applicable to $\Gamma$ (respectively $\Gamma'$). 
 
Further, $\mathcal{R}$ does not create a circle component. 
And by assumption, $\mathcal{R}$ does not create a loop component. 
Thus, $\mathcal{R}$ does not create a circle/loop component, so $\Gamma'$ does not admit deformations II/III that are not induced from those applicable to $\Gamma$. 
%
And if the closure of $R$ intersects a circle/loop component applicable of a deformation II/III, denoted by $C$, we can also apply a deformation II/III to $C$ after the application of $\mathcal{R}$. 
Thus, 
$\Gamma'$ (respectively $\Gamma$) does not admit a deformation II/III that is not induced from one of those applicable to $\Gamma$ (respectively $\Gamma'$). 


Thus, local moves $\mathcal{E}$ do not effect deformations II, III  and IV.  
And local moves $\mathcal{E}$ might change arcs with dots to one arc with dots (or move a dot into another arc with a dot). Hence 
we see that (1) in the proof of Proposition \ref{prop3-5} implies the uniqueness of the reduced graph up to local moves $\mathcal{E}$ and deformations I. 
\end{proof}

\begin{lemma}\label{lem913}
There do not exist arcs where both deformations III and IV are applicable, under the assumption that we don't apply a deformation IV between a loop component $C$ applicable of a deformation III and an arc of an overlapping region of $C$. 
\end{lemma}
\begin{proof}
By seeing the labels of the regions, we have the required result. 
See Figure \ref{B-fig14}. 
\end{proof}

\begin{lemma}\label{lem912}
When we have a circle/loop component $C$ applicable a deformation II/III  such that the interior of the disk bounding $C$ contains another circle/loop component $C'$ applicable of a deformation II/III, the result of the deformations II/III to $C$ and then $C'$ is the same with that of the deformations II/III to $C'$ and then $C$, up to local moves $\mathcal{E}$. 
\end{lemma}

\begin{proof}
Let $D$ and $D'$ be the disks bounded by $C$ and $C'$, respectively. 
By seeing the labels of the regions, we see that the total labels in $D$ are all positive or all negative, and $C$ and $C'$ has a coherent orientations as concentric circles. Hence the result of the deformations II/III to $C$ and then $C'$ is the same with that of the deformations II/III to $C'$ and then $C$, up to local moves $\mathcal{E}$. 
\end{proof}

\begin{lemma}\label{lem826}
 Let  $\Gamma$ be a dotted graph. Then, there does not exist a sequence of deformations consisting of 
infinite number of deformations IV.  
\end{lemma}


\begin{proof}
Assume that there is a sequence of deformations IV, such that the number of deformations is greater than an arbitrarily large number $n$. 
 
 

A deformation IV can be regarded as the result of a band surgery along an untwisted band \cite{Kawauchi}, as follows. 
Let $\Gamma'$ be a dotted graph where a deformation IV is applied, and let $\Gamma''$ be the resulting dotted graph. Let $B$ be the disk that is the intersection of the region with the label $\epsilon i$ of $\Gamma'$ and the region with the label $\epsilon (i-1)$ of $\Gamma''$. We assume that $B$ does not contain the pair of dots. Then, $B$ is a band attached to $\Gamma'$, and the deformation IV is the result of the band surgery along $B$. The {\it core} of the band $B$ is a simple arc connecting points of arcs of $\Gamma'$ that is a retraction of $B$. 
By the condition for the labels of the overlapping regions, the core is determined up to local moves $\mathcal{E}$. 

Let $\Gamma'$ be a dotted graph where a deformation IV is applied, and let $\Gamma''$ be the resulting dotted graph. Let $B$ be the band attached to $\Gamma'$ such that $\Gamma''$ is the result of band surgery along $B$.  
We call arcs $\Gamma' \cap B$ {\it ends} of the band $B$ attaching to $\Gamma'$. We also often assume that the ends are in the neighborhoods of the pair of dots, and we present the core by an arc connecting the dots. We remark that when we deform the band (and the core) by local moves $\mathcal{E}$, the result of the band surgery along the deformed band is related with the result of the original  deformation IV by local moves $\mathcal{E}$. Thus, a deformation IV is determined by the core of the band, up to local moves $\mathcal{E}$. 



When we have a sequence of deformations IV from $\Gamma$ to some dotted graph, the sequence is presented as a set of the bands of the deformations attached to $\Gamma$. 
By applying local moves $\mathcal{E}$ if necessary, 
we assume that the bands are sufficiently thin, and hence they are presented by the cores. 
 For each dot $p_i$ in $\Gamma$, we denote by the same notation $p_i$ the dot coming from $p_i$ in the resulting dotted graphs of the deformations IV; we remark that we have two choices of $p_i$ for each deformation IV. 
Let $n_i$ be the number of deformations IV applied to $p_i$. Let $N$ be the number of dots $p_i$ with $n_i>0$. By changing the indices, we arrange that $p_1, \ldots, p_N$ are the dots with $n_i>0$ $(i=1, \ldots, N)$; see Figure \ref{C-fig2}. 
 Then, the cores of bands form a graph $G$ in $\mathbb{R}^2$ consisting of edges and vertices of degree one called {\it endpoints}, and vertices $v_i$ of degree $n_i+1$  such that the endpoints are dots $p_i$ of $\Gamma$ $(i=1, \ldots, N$). 
The result of the deformations IV is obtained from $\Gamma$ by the surgery along the regular neighborhood $N(G)$ of $G$, that is defined as the closure of $(\Gamma \cup \partial N(G))\backslash (a_1 \cup \cdots \cup a_N)$, where $N(G)$ is the union of bands associated with cores, and $a_i$ is the end of the band containing $p_i$ in $\Gamma$ $(i=1, \ldots, N)$. 
By regarding the vertex $v_i$ of degree $n_i+1$ as the dot $p_i$, we regard that the result of deformations IV is the result of band surgery along bands such that for each $p_i$, the associated cores are mutually disjoint and the endpoints are $n_i$ points in the neighborhood of the dot $p_i$. We call the endpoints the endpoints {\it associated with $p_i$}. We remark that since a deformation IV satisfies the condition of labels for overlapping regions, together with Lemma \ref{lem0901}, we see that the cores whose endpoints are associated with $p_i$ don't have intersecting points. 

Since we can take the number of deformations to be greater than an arbitrarily large number $n$, 
there are 
cores each of whose endpoints are associated with a fixed pair of dots. The cores are mutually disjoint, and in the neighborhood of the associated dot, near the endpoints, the cores are parallel by Lemma \ref{lem0901}. 
Since the bands have no twists, when we have $m$ such cores, the resulting dotted graph has at least $m-2$ circle components more than before the deformations, and each circle component has a dot. 
Since the number $m$ can be made arbitrarily large, the resulting graph has more dots than $\Gamma$. Since deformations IV do not change the number of dots, this is a contradiction. 
Thus a dotted graph does not admit a sequence of arbitrarily large number of deformations IV. 
\end{proof}

\begin{lemma}\label{lem0901}
Let $\Gamma$ be a dotted graph and let $p$ be a dot of $\Gamma$. Let $D$ be a small disk around $p$ such that the arc of $\Gamma$ divides $D$ into two disks $D_1$ and $D_2$. Let $\mathcal{X}$ be the set of  deformations  IV applicable between $p$ and other dots. Then, the middle regions of deformations in $\mathcal{X}$ intersect 
with $D$ in one of the disks $D_1$ and $D_2$. 
\end{lemma}

\begin{proof}
Assume that we can apply a deformation IV, denoted by $\mathcal{R}_1$ (respectively $\mathcal{R}_2$), between $p$ and a dot $p_1$ (respectively $p$ and $p_2$) such that the middle region intersects with $D$ in $D_1$ (respectively $D_2$). 
Since $\mathcal{R}_1$ is applicable, we see that when $D_1$ is not overlapped, the label of $D_1$ (respectively $D_2$) is $\epsilon i$ (respectively $\epsilon (i-1)$), where $\epsilon \in \{+1, -1\}$ and $i$ is a positive integer: 
the absolute value of the label of $D_1$ is greater than that of $D_2$. 
And when the middle region containing $D_1$ is overlapped by some regions, since the total label of each region in the middle region are  $\epsilon j$ for some $j\geq i$, the case does not occur that the absolute value of the total label of a region in $D_1$ is smaller than that of $D_2$. 

By the same argument, since $\mathcal{R}_2$ is applicable, we see that the absolute value of the total label of a region in $D_2$ is greater than that of $D_1$, which is a contradiction. 
Hence we have the required result. 
\end{proof}

         
  


When we have a lattice polytope $P$ with initial vertices $\mathrm{Ver}_0(P)$ and terminal vertices $\mathrm{Ver}_1(P)$, 
we have given a transformation along a rectangle $R$ that has vertices in $\mathrm{Ver}_0(P)$ as its diagonal vertices. 
The transformation of $P$ along the rectangle $R$ is 
the transformation from $P$ to the lattice polytope whose initial vertices  are $t(\mathrm{Ver}_0(P); R)$ and terminal vertices are $\mathrm{Ver}_1(P)$. 
We call such a transformation of $P$ a {\it normal transformation}. 
Now, we define a reversed transformation as follows. 
\begin{definition}\label{def818}
Let $P$ be a lattice polytope, and let $R$ be 
a rectangle that has vertices in $\mathrm{Ver}_1(P)$ as its diagonal vertices. 
Then, we define the {\it reversed transformation} of $P$ along the rectangle $R$ as
the transformation from $P$ to the lattice polytope whose initial vertices  are $\mathrm{Ver}_0(P)$ and terminal vertices are $t(\mathrm{Ver}_1(P); R)$. 
\end{definition}


\begin{lemma}\label{lem-822}
Let $P$ be a lattice polytope and 
let $\Gamma$ be its associated dotted graph. 
When we can apply a deformation IVa that satisfies the condition (a1) given in Definition \ref{def3-3}, then 
there exists a normal/reversed transformation of $P$ that realizes the deformation IVa. 
\end{lemma}

\begin{proof}
Suppose that we can apply to $\Gamma$ a deformation IV in question. We recall that the condition (a1) is that the arcs involved in the deformation are adjacent arcs of a crossing. 


We denote by $D$ the region of $P$ associated with the middle region. 
Then, the pair of dots involved in the deformation are on adjacent arcs connected by a crossing such that around the crossing, the other arcs are in the complement of $D$. Let $e$ and $e'$ be the pair of edges of $P$ in the $x$-direction and $y$-direction that form the crossing, and let $v$ and $v'$ be the pair of vertices that are endpoints of $e$ and $e'$ contained in the boundary of the closure of $D$. Let $R$ be the rectangle whose pair of diagonal vertices are $v$ and $v'$. We can see that $v$ and $v'$ are both initial vertices or both terminal vertices of $P$. The transformation of $P$ along $R$ is the required transformation: it realizes the deformation IV, and it is a normal (respectively reversed) transformation if $v$ and $v'$ are the initial (respectively terminal) vertices. 
\end{proof}


\begin{lemma}\label{lem-821}
Let $P$ be a lattice polytope. 
When $P$ admits a sequence of  transformations consisting of normal transformations and reversed transformations realizing minimal area, 
then there is a sequence of  transformations consisting only of normal transformations realizing minimal area. 
\end{lemma}

\begin{proof}
We denote the sequence of transformations by  
\[
(\mathcal{R}^1_1 \mathcal{R}^1_2 \cdots \mathcal{R}^1_{m_1}) (\mathcal{L}^1_1 \mathcal{L}^1_2 \cdots \mathcal{L}^1_{n_1})
(\mathcal{R}^2_1 \mathcal{R}^2_2 \ldots \mathcal{R}^2_{m_2}) \cdots (\mathcal{L}^r_1 \mathcal{L}^r_2 \ldots, \mathcal{L}^r_{n_r}), 
\]
where $\mathcal{R}_s^t$ (respectively $\mathcal{L}_s^t$) is a normal  transformation (respectively a reversed transformation) along a rectangle, and we apply transformations from left to right. 
For a reversed transformation $\mathcal{L}$ along a rectangle $R$ from a lattice polytope $P_1$ to a lattice polytope $P_2$, we denote by $\mathcal{\bar{L}}$ the normal transformation along $R$ from $P_2$ to $P_1$. Then, 
the sequence of transformations 
\begin{eqnarray*}&&
(\mathcal{R}^1_1 \mathcal{R}^1_2 \cdots \mathcal{R}^1_{m_1}) 
(\mathcal{R}^2_1 \mathcal{R}^2_2 \cdots \mathcal{R}^2_{m_2}) (\mathcal{\bar{L}}^1_{n_1} \mathcal{\bar{L}}^1_{n_1-1} \cdots \mathcal{\bar{L}}^1_{1})\\
&&
\ \cdot (\mathcal{R}^3_1 \mathcal{R}^3_2 \cdots \mathcal{R}^3_{m_3})
%
(\mathcal{\bar{L}}^2_{n_2}\mathcal{\bar{L}}^2_{n_2-1}\cdots \mathcal{\bar{L}}^2_{1}) \cdots\\
&&
\ \cdot (\mathcal{R}^r_1 \mathcal{R}^r_2 \cdots \mathcal{R}^r_{m_r})
%
(\mathcal{\bar{L}}^{r-1}_{n_{r-1}}\mathcal{\bar{L}}^{r-1}_{n_{r-1}-1}\cdots \mathcal{\bar{L}}^{r-1}_{1})
(\mathcal{\bar{L}}^r_{n_r} \mathcal{\bar{L}}^r_{n_r-1} \cdots \mathcal{\bar{L}}^r_{1})
\end{eqnarray*}
is the required transformation of $P$. 
\end{proof}

\section*{Acknowledgements}
The author was partially supported by JST FOREST Program, Grant Number JPMJFR202U. 

\begin{thebibliography}{0}

\bibitem{Barvinok2}
Barvinok, A. {\em Integer Points in Polyhedra}. Zurich Lectures in Advanced Mathematics, European Mathematical Society, 2008.


 \bibitem{Diestel}
Diestel, R. {\em Graph Theory}; Graduate Texts in Mathematics 173, American Mathematical Society, Springer, 2010. 

\bibitem{Kawauchi}
A. Kawauchi, {\it A Survey of Knot Theory}, 
Birkh\"{a}user Verlag, Basel, 1996. 

\bibitem {N}
Nakamura, I. {\em Transformations of partial matchings}; Kyungpook Math. J. 61 (2021), No.2, 409-439. 

 
\bibitem{Reidys} 
Reidys, C. M. {\em Combinatorial Computational Biology of RNA. Pseudoknots and neutral networks}; Springer, New York, 2011.


\end{thebibliography}

\end{document}