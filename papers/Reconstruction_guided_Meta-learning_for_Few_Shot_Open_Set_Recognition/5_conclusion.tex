\section{Conclusion}
In this work, we present a novel strategy for addressing few-shot open-set recognition. We frame the few-shot open-set classification task as a meta-learning problem similar to \cite{peeler}, but unlike their strategy, we do not solely rely on thresholding softmax scores to indicate the openness of a sample. We argue that existing thresholding type FSOSR methods \cite{peeler,snatcher} rely heavily on the choice of a carefully tuned threshold to achieve good performance. Additionally, the proclivity of softmax to overfit to unseen classes makes it an unreliable choice as an open-set indicator, especially when there is a dearth of samples. Instead, we propose to use a reconstruction of exemplar images as a key signal to detect out-of-distribution samples. 
The learned embedding which is used to classify the sample is further modulated to ensure a proficient gap between the seen and unseen class clusters in the feature space. Finally, the modulated embedding, the softmax score, and the quality reconstructed exemplar are jointly utilized to cognize if the sample is in-distribution or out-of-distribution. 
The enhanced performance of our framework is verified empirically over a wide variety of few-shot tasks and the results establish it as the new state-of-the-art. In the future, we would like to extend this approach to more cross-domain few-shot tasks, including videos.
\vspace{-2em}
\section{Acknowledgement}
This work was partially supported by US National Science Foundation grant 2008020 and US Office of Naval Research grants N00014-19-1-2264 and N00014-18-1-2252.
\vspace{-1em}