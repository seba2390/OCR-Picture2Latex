\documentclass[10pt,journal,compsoc]{IEEEtran}

\usepackage{url}
\usepackage{microtype}
	
\usepackage{color, colortbl}
% *** CITATION PACKAGES ***
% \usepackage[moderate]{savetrees}
\ifCLASSOPTIONcompsoc
  % IEEE Computer Society needs nocompress option
  % requires cite.sty v4.0 or later (November 2003)
  \usepackage[nocompress]{cite}
\else
  % normal IEEE
  \usepackage{cite}
\fi

% *** GRAPHICS RELATED PACKAGES ***
%
\ifCLASSINFOpdf
  % \usepackage[pdftex]{graphicx}
  % declare the path(s) where your graphic files are
  % \graphicspath{{../pdf/}{../jpeg/}}
  % and their extensions so you won't have to specify these with
  % every instance of \includegraphics
  % \DeclareGraphicsExtensions{.pdf,.jpeg,.png}
\else
  % or other class option (dvipsone, dvipdf, if not using dvips). graphicx
  % will default to the driver specified in the system graphics.cfg if no
  % driver is specified.
  % \usepackage[dvips]{graphicx}
  % declare the path(s) where your graphic files are
  % \graphicspath{{../eps/}}
  % and their extensions so you won't have to specify these with
  % every instance of \includegraphics
  % \DeclareGraphicsExtensions{.eps}
\fi

% *** MATH PACKAGES ***
%


% *** SPECIALIZED LIST PACKAGES ***
%
%\usepackage{algorithmic}
% algorithmic.sty was written by Peter Williams and Rogerio Brito.
% This package provides an algorithmic environment fo describing algorithms.
% You can use the algorithmic environment in-text or within a figure
% environment to provide for a floating algorithm. Do NOT use the algorithm
% floating environment provided by algorithm.sty (by the same authors) or
% algorithm2e.sty (by Christophe Fiorio) as the IEEE does not use dedicated
% algorithm float types and packages that provide these will not provide
% correct IEEE style captions. The latest version and documentation of
% algorithmic.sty can be obtained at:
% http://www.ctan.org/pkg/algorithms
% Also of interest may be the (relatively newer and more customizable)
% algorithmicx.sty package by Szasz Janos:
% http://www.ctan.org/pkg/algorithmicx

% *** ALIGNMENT PACKAGES ***
%
%\usepackage{array}
% Frank Mittelbach's and David Carlisle's array.sty patches and improves
% the standard LaTeX2e array and tabular environments to provide better
% appearance and additional user controls. As the default LaTeX2e table
% generation code is lacking to the point of almost being broken with
% respect to the quality of the end results, all users are strongly
% advised to use an enhanced (at the very least that provided by array.sty)
% set of table tools. array.sty is already installed on most systems. The
% latest version and documentation can be obtained at:
% http://www.ctan.org/pkg/array


% IEEEtran contains the IEEEeqnarray family of commands that can be used to
% generate multiline equations as well as matrices, tables, etc., of high
% quality.

% *** SUBFIGURE PACKAGES ***
%\ifCLASSOPTIONcompsoc
%  \usepackage[caption=false,font=footnotesize,labelfont=sf,textfont=sf]{subfig}
%\else
%  \usepackage[caption=false,font=footnotesize]{subfig}
%\fi
% subfig.sty, written by Steven Douglas Cochran, is the modern replacement
% for subfigure.sty, the latter of which is no longer maintained and is
% incompatible with some LaTeX packages including fixltx2e. However,
% subfig.sty requires and automatically loads Axel Sommerfeldt's caption.sty
% which will override IEEEtran.cls' handling of captions and this will result
% in non-IEEE style figure/table captions. To prevent this problem, be sure
% and invoke subfig.sty's "caption=false" package option (available since
% subfig.sty version 1.3, 2005/06/28) as this is will preserve IEEEtran.cls
% handling of captions.
% Note that the Computer Society format requires a sans serif font rather
% than the serif font used in traditional IEEE formatting and thus the need
% to invoke different subfig.sty package options depending on whether
% compsoc mode has been enabled.
%
% The latest version and documentation of subfig.sty can be obtained at:
% http://www.ctan.org/pkg/subfig

% *** FLOAT PACKAGES ***
%
%\usepackage{fixltx2e}
% fixltx2e, the successor to the earlier fix2col.sty, was written by
% Frank Mittelbach and David Carlisle. This package corrects a few problems
% in the LaTeX2e kernel, the most notable of which is that in current
% LaTeX2e releases, the ordering of single and double column floats is not
% guaranteed to be preserved. Thus, an unpatched LaTeX2e can allow a
% single column figure to be placed prior to an earlier double column
% figure.
% Be aware that LaTeX2e kernels dated 2015 and later have fixltx2e.sty's
% corrections already built into the system in which case a warning will
% be issued if an attempt is made to load fixltx2e.sty as it is no longer
% needed.
% The latest version and documentation can be found at:
% http://www.ctan.org/pkg/fixltx2e


%\usepackage{stfloats}
% stfloats.sty was written by Sigitas Tolusis. This package gives LaTeX2e
% the ability to do double column floats at the bottom of the page as well
% as the top. (e.g., "\begin{figure*}[!b]" is not normally possible in
% LaTeX2e). It also provides a command:
%\fnbelowfloat
% to enable the placement of footnotes below bottom floats (the standard
% LaTeX2e kernel puts them above bottom floats). This is an invasive package
% which rewrites many portions of the LaTeX2e float routines. It may not work
% with other packages that modify the LaTeX2e float routines. The latest
% version and documentation can be obtained at:
% http://www.ctan.org/pkg/stfloats
% Do not use the stfloats baselinefloat ability as the IEEE does not allow
% \baselineskip to stretch. Authors submitting work to the IEEE should note
% that the IEEE rarely uses double column equations and that authors should try
% to avoid such use. Do not be tempted to use the cuted.sty or midfloat.sty
% packages (also by Sigitas Tolusis) as the IEEE does not format its papers in
% such ways.
% Do not attempt to use stfloats with fixltx2e as they are incompatible.
% Instead, use Morten Hogholm'a dblfloatfix which combines the features
% of both fixltx2e and stfloats:
%
% \usepackage{dblfloatfix}
% The latest version can be found at:
% http://www.ctan.org/pkg/dblfloatfix




%\ifCLASSOPTIONcaptionsoff
%  \usepackage[nomarkers]{endfloat}
% \let\MYoriglatexcaption\caption
% \renewcommand{\caption}[2][\relax]{\MYoriglatexcaption[#2]{#2}}
%\fi
% endfloat.sty was written by James Darrell McCauley, Jeff Goldberg and 
% Axel Sommerfeldt. This package may be useful when used in conjunction with 
% IEEEtran.cls'  captionsoff option. Some IEEE journals/societies require that
% submissions have lists of figures/tables at the end of the paper and that
% figures/tables without any captions are placed on a page by themselves at
% the end of the document. If needed, the draftcls IEEEtran class option or
% \CLASSINPUTbaselinestretch interface can be used to increase the line
% spacing as well. Be sure and use the nomarkers option of endfloat to
% prevent endfloat from "marking" where the figures would have been placed
% in the text. The two hack lines of code above are a slight modification of
% that suggested by in the endfloat docs (section 8.4.1) to ensure that
% the full captions always appear in the list of figures/tables - even if
% the user used the short optional argument of \caption[]{}.
% IEEE papers do not typically make use of \caption[]'s optional argument,
% so this should not be an issue. A similar trick can be used to disable
% captions of packages such as subfig.sty that lack options to turn off
% the subcaptions:
% For subfig.sty:
% \let\MYorigsubfloat\subfloat
% \renewcommand{\subfloat}[2][\relax]{\MYorigsubfloat[]{#2}}
% However, the above trick will not work if both optional arguments of
% the \subfloat command are used. Furthermore, there needs to be a
% description of each subfigure *somewhere* and endfloat does not add
% subfigure captions to its list of figures. Thus, the best approach is to
% avoid the use of subfigure captions (many IEEE journals avoid them anyway)
% and instead reference/explain all the subfigures within the main caption.
% The latest version of endfloat.sty and its documentation can obtained at:
% http://www.ctan.org/pkg/endfloat
%
% The IEEEtran \ifCLASSOPTIONcaptionsoff conditional can also be used
% later in the document, say, to conditionally put the References on a 
% page by themselves.




% *** PDF, URL AND HYPERLINK PACKAGES ***
%
%\usepackage{url}
% url.sty was written by Donald Arseneau. It provides better support for
% handling and breaking URLs. url.sty is already installed on most LaTeX
% systems. The latest version and documentation can be obtained at:
% http://www.ctan.org/pkg/url
% Basically, \url{my_url_here}.





% *** Do not adjust lengths that control margins, column widths, etc. ***
% *** Do not use packages that alter fonts (such as pslatex).         ***
% There should be no need to do such things with IEEEtran.cls V1.6 and later.
% (Unless specifically asked to do so by the journal or conference you plan
% to submit to, of course. )


% correct bad hyphenation here
\hyphenation{op-tical net-works semi-conduc-tor}

\usepackage{graphicx}
\usepackage{amsmath}
\usepackage{amssymb,amsfonts}
\usepackage{enumitem}
\usepackage{microtype}
\usepackage{bbm}
\usepackage{subfig}
\usepackage[
singlelinecheck=false % <-- important
]{caption}
\usepackage{booktabs}
\usepackage{multirow}
\usepackage[normalem]{ulem}
\usepackage[dvipsnames]{xcolor}
\usepackage[subtle]{savetrees}
\newcommand{\sayak}[1]{\textcolor{blue}{#1}}
\newcommand{\amit}[1]{\textcolor{red}{#1}}
\begin{document}
%
% paper title
% Titles are generally capitalized except for words such as a, an, and, as,
% at, but, by, for, in, nor, of, on, or, the, to and up, which are usually
% not capitalized unless they are the first or last word of the title.
% Linebreaks \\ can be used within to get better formatting as desired.
% Do not put math or special symbols in the title.
% \title{Learning to Identify Unseen Data \\ in a Few-shot Setting}
\title{Reconstruction guided Meta-learning for Few Shot Open Set Recognition}
%
%
% author names and IEEE memberships
% note positions of commas and nonbreaking spaces ( ~ ) LaTeX will not break
% a structure at a ~ so this keeps an author's name from being broken across
% two lines.
% use \thanks{} to gain access to the first footnote area
% a separate \thanks must be used for each paragraph as LaTeX2e's \thanks
% was not built to handle multiple paragraphs
%
%
%\IEEEcompsocitemizethanks is a special \thanks that produces the bulleted
% lists the Computer Society journals use for "first footnote" author
% affiliations. Use \IEEEcompsocthanksitem which works much like \item
% for each affiliation group. When not in compsoc mode,
% \IEEEcompsocitemizethanks becomes like \thanks and
% \IEEEcompsocthanksitem becomes a line break with idention. This
% facilitates dual compilation, although admittedly the differences in the
% desired content of \author between the different types of papers makes a
% one-size-fits-all approach a daunting prospect. For instance, compsoc 
% journal papers have the author affiliations above the "Manuscript
% received ..."  text while in non-compsoc journals this is reversed. Sigh.

\author{Sayak~Nag,
        Dripta~S.~Raychaudhuri,
        Sujoy~Paul,
        and~Amit~K.~Roy-Chowdhury,~\IEEEmembership{Fellow,~IEEE}}% <-this % stops a space
% \IEEEcompsocitemizethanks{\IEEEcompsocthanksitem M. Shell was with the Department
% of Electrical and Computer Engineering, Georgia Institute of Technology, Atlanta,
% GA, 30332.\protect\\
% % note need leading \protect in front of \\ to get a newline within \thanks as
% % \\ is fragile and will error, could use \hfil\break instead.
% E-mail: see http://www.michaelshell.org/contact.html
% \IEEEcompsocthanksitem J. Doe and J. Doe are with Anonymous University.}% <-this % stops an unwanted space
% \thanks{Manuscript received April 19, 2005; revised August 26, 2015.}}

% note the % following the last \IEEEmembership and also \thanks - 
% these prevent an unwanted space from occurring between the last author name
% and the end of the author line. i.e., if you had this:
% 
% \author{....lastname \thanks{...} \thanks{...} }
%                     ^------------^------------^----Do not want these spaces!
%
% a space would be appended to the last name and could cause every name on that
% line to be shifted left slightly. This is one of those "LaTeX things". For
% instance, "\textbf{A} \textbf{B}" will typeset as "A B" not "AB". To get
% "AB" then you have to do: "\textbf{A}\textbf{B}"
% \thanks is no different in this regard, so shield the last } of each \thanks
% that ends a line with a % and do not let a space in before the next \thanks.
% Spaces after \IEEEmembership other than the last one are OK (and needed) as
% you are supposed to have spaces between the names. For what it is worth,
% this is a minor point as most people would not even notice if the said evil
% space somehow managed to creep in.



% The paper headers
% \markboth{Journal of \LaTeX\ Class Files,~Vol.~14, No.~8, August~2015}%
% {Shell \MakeLowercase{\textit{et al.}}: Bare Demo of IEEEtran.cls for Computer Society Journals}

\begin{abstract}

Visual perception tasks often require vast amounts of labelled data, including 3D poses and image space segmentation masks. The process of creating such training data sets can prove difficult or time-intensive to scale up to efficacy for general use. Consider the task of pose estimation for rigid objects. Deep neural network based approaches have shown good performance when trained on large, public datasets. However, adapting these networks for other novel objects, or fine-tuning existing models for different environments, requires significant time investment to generate newly labelled instances. Towards this end, we propose ProgressLabeller as a method for more efficiently generating large amounts of 6D pose training data from color images sequences for custom scenes in a scalable manner. ProgressLabeller is intended to also support transparent or translucent objects, for which the previous methods based on depth dense reconstruction will fail.
We demonstrate the effectiveness of ProgressLabeller by rapidly create a dataset of over 1M samples with which we fine-tune a state-of-the-art pose estimation network in order to markedly improve the downstream robotic grasp success rates. Progresslabeller is open-source at \href{https://github.com/huijieZH/ProgressLabeller}{https://github.com/huijieZH/ProgressLabeller}

\end{abstract}

% make the title area
\maketitle

\IEEEdisplaynontitleabstractindextext

\IEEEpeerreviewmaketitle

\begin{figure}[t]
\begin{center}
   \includegraphics[width=1.0\linewidth]{figures/nas_comp_v3}
\end{center}
   \vspace{-4mm}
   \caption{The comparison between NetAdaptV2 and related works. The number above a marker is the corresponding total search time measured on NVIDIA V100 GPUs.}
\label{fig:nas_comparison}
\end{figure}

\section{Introduction}
\label{sec:introduction}

Neural architecture search (NAS) applies machine learning to automatically discover deep neural networks (DNNs) with better performance (e.g., better accuracy-latency trade-offs) by sampling the search space, which is the union of all discoverable DNNs. The search time is one key metric for NAS algorithms, which accounts for three steps: 1) training a \emph{super-network}, whose weights are shared by all the DNNs in the search space and trained by minimizing the loss across them, 2) training and evaluating sampled DNNs (referred to as \emph{samples}), and 3) training the discovered DNN. Another important metric for NAS is whether it supports non-differentiable search metrics such as hardware metrics (e.g., latency and energy). Incorporating hardware metrics into NAS is the key to improving the performance of the discovered DNNs~\cite{eccv2018-netadapt, Tan2018MnasNetPN, cai2018proxylessnas, Chen2020MnasFPNLL, chamnet}.


There is usually a trade-off between the time spent for the three steps and the support of non-differentiable search metrics. For example, early reinforcement-learning-based NAS methods~\cite{zoph2017nasreinforcement, zoph2018nasnet, Tan2018MnasNetPN} suffer from the long time for training and evaluating samples. Using a super-network~\cite{yu2018slimmable, Yu_2019_ICCV, autoslim_arxiv, cai2020once, yu2020bignas, Bender2018UnderstandingAS, enas, tunas, Guo2020SPOS} solves this problem, but super-network training is typically time-consuming and becomes the new time bottleneck. The gradient-based methods~\cite{gordon2018morphnet, liu2018darts, wu2018fbnet, fbnetv2, cai2018proxylessnas, stamoulis2019singlepath, stamoulis2019singlepathautoml, Mei2020AtomNAS, Xu2020PC-DARTS} reduce the time for training a super-network and training and evaluating samples at the cost of sacrificing the support of non-differentiable search metrics. In summary, many existing works either have an unbalanced reduction in the time spent per step (i.e., optimizing some steps at the cost of a significant increase in the time for other steps), which still leads to a long \emph{total} search time, or are unable to support non-differentiable search metrics, which limits the performance of the discovered DNNs.

In this paper, we propose an efficient NAS algorithm, NetAdaptV2, to significantly reduce the \emph{total} search time by introducing three innovations to \emph{better balance} the reduction in the time spent per step while supporting non-differentiable search metrics:

\textbf{Channel-level bypass connections (mainly reduce the time for training and evaluating samples, Sec.~\ref{subsec:channel_level_bypass_connections})}: Early NAS works only search for DNNs with different numbers of filters (referred to as \emph{layer widths}). To improve the performance of the discovered DNN, more recent works search for DNNs with different numbers of layers (referred to as \emph{network depths}) in addition to different layer widths at the cost of training and evaluating more samples because network depths and layer widths are usually considered independently. In NetAdaptV2, we propose \emph{channel-level bypass connections} to merge network depth and layer width into a single search dimension, which requires only searching for layer width and hence reduces the number of samples.

\textbf{Ordered dropout (mainly reduces the time for training a super-network, Sec.~\ref{subsec:ordered_droput})}: We adopt the idea of super-network to reduce the time for training and evaluating samples. In previous works, \emph{each} DNN in the search space requires one forward-backward pass to train. As a result, training multiple DNNs in the search space requires multiple forward-backward passes, which results in a long training time. To address the problem, we propose \emph{ordered dropout} to jointly train multiple DNNs in a \emph{single} forward-backward pass, which decreases the required number of forward-backward passes for a given number of DNNs and hence the time for training a super-network.

\textbf{Multi-layer coordinate descent optimizer (mainly reduces the time for training and evaluating samples and supports non-differentiable search metrics, Sec.~\ref{subsec:optimizer}):} NetAdaptV1~\cite{eccv2018-netadapt} and MobileNetV3~\cite{Howard_2019_ICCV}, which utilizes NetAdaptV1, have demonstrated the effectiveness of the single-layer coordinate descent (SCD) optimizer~\cite{book2020sze} in discovering high-performance DNN architectures. The SCD optimizer supports both differentiable and non-differentiable search metrics and has only a few interpretable hyper-parameters that need to be tuned, such as the per-iteration resource reduction. However, there are two shortcomings of the SCD optimizer. First, it only considers one layer per optimization iteration. Failing to consider the joint effect of multiple layers may lead to a worse decision and hence sub-optimal performance. Second, the per-iteration resource reduction (e.g., latency reduction) is limited by the layer with the smallest resource consumption (e.g., latency). It may take a large number of iterations to search for a very deep network because the per-iteration resource reduction is relatively small compared with the network resource consumption. To address these shortcomings,  we propose the \emph{multi-layer coordinate descent (MCD) optimizer} that considers multiple layers per optimization iteration to improve performance while reducing search time and preserving the support of non-differentiable search metrics.

Fig.~\ref{fig:nas_comparison} (and Table~\ref{tab:nas_result}) compares NetAdaptV2 with related works. NetAdaptV2 can reduce the search time by up to $5.8\times$ and $2.4\times$ on ImageNet~\cite{imagenet_cvpr09} and NYU Depth V2~\cite{nyudepth} respectively and discover DNNs with better performance than state-of-the-art NAS works. Moreover, compared to NAS-discovered MobileNetV3~\cite{Howard_2019_ICCV}, the discovered DNN has $1.8\%$ higher accuracy with the same latency.


\section{Related Work}
Our work draws from, and improves upon, several research threads.

\textbf{Sustainability.}~\citet{srba2016stack} conducted a case study on why StackOverflow, the largest and oldest of the sites in \CQA{StackExchange} network, is failing. They shed some insights into knowledge market failure such as novice and negligent users generating low quality content perpetuating the decline of the market. However, they do not provide a systematic way to understand and prevent failures in these markets.~\citet{wu2016} introduced a framework for understanding the user strategies in a knowledge market---revealing the importance of diverse user strategies for sustainable markets. In this paper, we present an alternative model that provides many interesting insights including knowledge market sustainability.

\textbf{Activity Dynamics.}~\citet{walk2016} modeled user-level activity dynamics in \CQA{StackExchange} using two factors: intrinsic activity decay, and positive peer influence. However, the model proposed there does not reveal the collective platform dynamics, and the eventual success or failure of a platform.~\citet{abufouda2017} developed two models for predicting the interaction decay of community members in online social communities. Similar to~\citet{walk2016}, these models accommodate user-level dynamics, whereas we concentrate on the collective platform dynamics.~\citet{wu2011} proposed a discrete generalized beta distribution (DGBD) model that reveals several insights into the collective platform dynamics, notably the concept of a size-dependent distribution. In this paper, we improve upon the concept of a size-dependent distribution.  

\textbf{Economic Perspective.} \citet{Kumar2010} proposed an economic view of CQA platforms, where they concentrated on the growth of two types of users in a market setting: users who provide questions, and users who provide answers. In this paper, we concentrate on a subsequent problem---the ``relation'' between user growth and content generation in a knowledge market.~\citet{butler2001} proposed a resource-based theory of sustainable social structures. While they treat members as resources, like we do, our model differs in that it concentrates on a market setting, instead of a network setting, and takes the complex content dependency of the platform into consideration. Furthermore, our model provides a systematic way to understand successes and failures of knowledge markets, which none of these models provide.  

\textbf{Scale Study.}~\citet{lin2017} examined Reddit communities to characterize the effect of user growth in voting patterns, linguistic patterns, and community network patterns. Their study reveals that these patterns do not change much after a massive growth in the size of the user community.~\citet{tausczik2017} investigated the effects of crowd size on solution quality in StackExchange communities. Their study uncovers three distinct levels of group size in the crowd that affect solution quality: topic audience size, question audience size, and number of contributors. In this paper, we examine the consequence of scale on knowledge markets from a different perspective by using a set of health metrics.

\textbf{Stability.} Successes and failures of platforms have been studied from the perspective of user retention and stability~\cite{patil2013, garcia2013, kapoor2014, ellis2016}. Notably,~\citet{patil2013} studied the dynamics of group stability based on the average increase or decrease in member growth. Our paper examines stability in a different manner---namely, by considering the relative exchangeability of users as a function of scale.

\textbf{User Growth.} Successes and failures of user communities have also been widely studied from the perspective of user growth~\cite{Kumar2006, Backstrom2006, kairam2012, Ribeiro2014, zang2016}.~\citet{kairam2012} examined diffusion and non-diffusion growth to design models that predict the longevity of social groups.~\citet{Ribeiro2014} proposed a daily active user prediction model which classifies membership based websites as sustainable and unsustainable. While this perspective is important, we argue that studying the successes and failures of communities based on content production can perhaps be more meaningful~\cite{kraut2014, zhu2014, zhu2014niche}.

\textbf{Modeling CQA Websites.} There is a rich body of work that extensively analyzed CQA websites~\cite{Adamic2008, chen2010, anderson2012, wang2013, srba2016}, along with user behavior~\cite{zhang2007, liu2011, pal2012, hanrahan2012, upadhyay2017}, roles~\cite{furtado2013, kumar2016}, and content generation~\cite{baezaYates2015, Yang2015, ferrara2017}. Notably,~\citet{Yang2015} noted the \emph{scalability problem} of CQA---namely, the volume of questions eventually subsumes the capacity of the answerers within the community. Understanding and modeling this phenomenon is one of the goals of this paper.


\section{Proposed Method: SyMFM6D}

We propose a deep multi-directional fusion approach called SyMFM6D that estimates the 6D object poses of all objects in a cluttered scene based on multiple RGB-D images while considering object symmetries. 
In this section, we define the task of multi-view 6D object pose estimation and present our multi-view deep fusion architecture.

\begin{figure*}[tbh]
  \vspace{2mm}
  \centering
  \includegraphics[page=1, trim = 5mm 40mm 5mm 42mm, clip,  width=1.0\linewidth]{figures/SyMFM6D_architecture4_2.pdf}
   \caption{Network architecture of SyMFM6D which fuses $N$ RGB-D input images. Our method converts the $N$ depth images to a single point cloud which is processed by an encoder-decoder point cloud network. The $N$ RGB images are processed by an encoder-decoder CNN. Every hierarchy contains a point-to-pixel fusion module and a pixel-to-point fusion module for deep multi-directional multi-view fusion. We utilize three MLPs with four layers each to regress 3D keypoint offsets, center point offsets, and semantic labels based on the final features. The 6D object poses are computed as in \cite{pvn3d} based on mean shift clustering and least-squares fitting. We train our network by minimizing our proposed symmetry-aware multi-task loss function using precomputed object symmetries. $N_p$ is the number of points in the point cloud. $H$ and $W$ are height and width of the RGB images.}
   \label{fig_architecture}
   \vspace{-2mm}
\end{figure*}


6D object pose estimation describes the task of predicting a rigid transformation $\boldsymbol p = [\boldsymbol R |  \boldsymbol t] \in SE(3)$ which transforms the coordinates of an observed object from the object coordinate system into the camera coordinate system. This transformation is called 6D object pose because it is composed of a 3D rotation $\boldsymbol R \in SO(3)$ and a 3D translation $\boldsymbol t \in \mathbb{R}^3$. 
The designated aim of our approach is to jointly estimate the 6D poses of all objects in a given cluttered scene using multiple RGB-D images which depict the scene from multiple perspectives. We assume the 3D models of the objects and the camera poses to be known as proposed by \cite{mv6d}.



\subsection{Network Overview}

Our symmetry-aware multi-view network consists of three stages which are visualized in \cref{fig_architecture}. 
The first stage receives one or multiple RGB-D images and extracts visual features as well as geometric features which are fused to a joint representation of the scene. 
The second stage performs a detection of predefined 3D keypoints and an instance semantic segmentation.
Based on the keypoints and the information to which object the keypoints belong, we compute the 6D object poses with a least-squares fitting algorithm \cite{leastSquares} in the third stage.



\subsection{Multi-View Feature Extraction}

To efficiently predict keypoints and semantic labels, the first stage of our approach learns a compact representation of the given scene by extracting and merging features from all available RGB-D images in a deep multi-directional fusion manner. For that, we first separate the set of RGB images $\text{RGB}_1, ..., \text{RGB}_N$ from their corresponding depth images $\text{Dpt}_1$, ..., $\text{Dpt}_N$. The $N$ depth images are converted into point clouds, transformed into the coordinate system of the first camera, and merged to a single point cloud using the known camera poses as in \cite{mv6d}. 
Unlike \cite{mv6d}, we employ a point cloud network based on RandLA-Net \cite{hu2020randla} with an encoder-decoder architecture using skip connections.
The point cloud network learns geometric features from the fused point cloud and considers visual features from the multi-directional point-to-pixel fusion modules as described in \cref{sec_multi_view_fusion}.

The $N$ RGB images are independently processed by a CNN with encoder-decoder architecture using the same weights for all $N$ views. The CNN learns visual features while considering geometric features from the multi-directional pixel-to-point fusion modules. We followed \cite{ffb6d} and build the encoder upon a ResNet-34 \cite{resnet} pretrained on ImageNet~\cite{imagenet} and the decoder upon a PSPNet \cite{pspnet}. 

After the encoding and decoding procedures including several multi-view feature fusions, we collect the visual features from each view corresponding to the final geometric feature map and concatenate them. The output is a compact feature tensor containing the relevant information about the entire scene which is used for keypoint detection and instance semantic segmentation as described in \cref{sec_keypoint_detection_and_segmentation}.


\begin{figure*}[tbh]
  \vspace{2mm} 
  \centering  
\begin{subfigure}[b]{0.48\textwidth}
  \includegraphics[page=1, trim = 1mm 6mm 6mm 6mm, clip,  width=1.0\linewidth]{figures/p2r_8.pdf}
   \caption{Point-to-pixel fusion module.~~~~}
   \label{fig_pt2px_fusion}
\end{subfigure}
\begin{subfigure}[b]{0.48\textwidth}
  \centering  
  \includegraphics[page=1, trim = 1mm 6mm 6mm 6mm, clip,  width=1.0\linewidth]{figures/r2p_8.pdf}
   \caption{Pixel-to-point fusion module.~~~~~}
   \label{fig_px2pt_fusion}
   \end{subfigure}
      \caption{Overview of our proposed multi-directional multi-view fusion modules. They combine pixel-wise visual features and point-wise geometric features by exploiting the correspondence between pixels and points using the nearest neighbor algorithm. We compute the resulting features using multiple shared MLPs with a single layer and max-pooling.
      For simplification, we depict an example with $N=2$ views and $K_\text{i}=K_\text{p}=3$ nearest neighbors. The points of ellipsis (...) illustrate the generalization for an arbitrary number of views $N$. Please refer to \cite{ffb6d} for better understanding the basic operations.
      }
   \label{fig_fusion_modules}
   \vspace{-1mm}
\end{figure*}



\subsection{Multi-View Feature Fusion}
\label{sec_multi_view_fusion}
In order to efficiently fuse the visual and geometric features from multiple views, we extend the fusion modules of FFB6D~\cite{ffb6d} from bi-directional fusion to \emph{multi-directional fusion}. We present two types of multi-directional fusion modules which are illustrated in \cref{fig_fusion_modules}.
Both types of fusion modules take the pixel-wise visual feature maps and the point-wise geometric feature maps from each view, combine them, and compute a new feature map.
This process requires a correspondence between pixel-wise and point-wise features which we obtain by computing an XYZ map for each RGB feature map based on the depth data of each pixel using the camera intrinsic matrix as in \cite{ffb6d}. To deal with the changing dimensions at different layers, we use the centers of the convolutional kernels as new coordinates of the feature maps and resize the XYZ map to the same size using nearest interpolation as proposed in \cite{ffb6d}.

The \emph{point-to-pixel} fusion module in \cref{fig_pt2px_fusion} computes a 
fused feature map $\bb F_\text{f}$ based on the image features $\bb F_{\text{i}}(v)$ of all views $v \in \{1, \ldots, N\}$.
We collect the $K_\text{p}$ nearest point features $\bb F_{\text{p}_k}(v)$ with $k \in \{1, \ldots, K_\text{p}\}$ from the point cloud for each pixel-wise feature and each view independently by computing the nearest neighbors according to the Euclidean distance in the XYZ map. Subsequently, we process them by a shared MLP before aggregating them by max-pooling, i.e.,
\begin{align} 
    \widetilde{\bb F}_{\text{p}}(v) = \max_{k \in \{1, \ldots, K_\text{p}\}} 
    \Big( \text{MLP}_\text{p}(\bb F_{\text{p}_k}(v)) \Big).
    \label{eq_p2r}
\end{align}
Finally, we apply a second shared MLP to fuse all features $\bb F_\text{i}$ and 
$\widetilde{\bb F}_{\text{p}}$ as 
$\bb F_{\text{f}} = \text{MLP}_\text{fp}(\widetilde{\bb F}_{\text{p}} \oplus \bb F_\text{i})$ where $\oplus$ denotes the concatenate operation.


The \emph{pixel-to-point} fusion module in \cref{fig_px2pt_fusion} collects the $K_\text{i}$ nearest image features $\bb F_{\text{i}_k}(\textrm{i2v}(i_k))$ with $k\in\{1, ..., K_\text{i}\}$. $\textrm{i2v}(i_k)$ is a mapping that maps the index of an image feature to its corresponding view. This procedure is performed for each point feature vector $\bb F_\text{p}(n)$.
We aggregate the collected image features by max-pooling and apply a shared MLP, i.e.,
\begin{align}
    \widetilde{\bb F}_{\text{i}} = \text{MLP}_\text{i} 
    \left( \max_{k \in \{1, \ldots, K_\text{i}\}} 
    \Big( \bb F_{\text{i}_k}(\textrm{i2v}(i_k)) \Big)  
    \right).
    \label{eq_r2p}
\end{align}
One more shared MLP fuses the resulting image features $\widetilde{\bb F}_{\text{i}}$ with the point features $\bb F_\text{p}$ as 
$\bb F_{\text{f}} = \text{MLP}_\text{fi}(\widetilde{\bb F}_{\text{i}} \oplus \bb F_\text{p})$.




\subsection{Keypoint Detection and Segmentation}
\label{sec_keypoint_detection_and_segmentation}
The second stage of our SyMFM6D network contains modules for 3D keypoint detection and instance semantic segmentation following \cite{mv6d}. However, unlike \cite{mv6d}, we use the SIFT-FPS algorithm \cite{lowe1999sift} as proposed by FFB6D \cite{ffb6d} to define eight target keypoints for each object class. SIFT-FPS yields keypoints with salient features which are easier to detect.
Based on the extracted features, we apply two shared MLPs to estimate the translation offsets from each point of the fused point cloud to each target keypoint and to each object center.
We obtain the actual point proposals by adding the translation offsets to the respective points of the fused point cloud. 
Applying the mean shift clustering algorithm \cite{cheng1995meanshift} results in predictions for the keypoints and the object centers.
We employ one more shared MLP 
for estimating the object class of each point in the fused point cloud as in \cite{pvn3d}.



\subsection{6D Pose Computation via Least-Squares Fitting}

Following \cite{pvn3d}, we use the least-squares fitting algorithm \cite{leastSquares} to compute the 6D poses of all objects based on the estimated keypoints. As the $M$ estimated keypoints $\boldsymbol{\widehat{k}}_1, ..., \boldsymbol{\widehat{k}}_M$ are in the coordinate system of the first camera and the target keypoints $\boldsymbol k_1, ..., \boldsymbol k_M$ are in the object coordinate system, least-squares fitting calculates the rotation matrix $\boldsymbol R$ and the translation vector $\boldsymbol t$ of the 6D pose by minimizing the squared loss
\begin{equation}
    L_\text{Least-squares} = \sum_{i=1}^M \norm{\boldsymbol{\widehat{k}_i} - (\boldsymbol R \boldsymbol k_i + \boldsymbol t)}_2^2.
\end{equation}



\subsection{Symmetry-aware Keypoint Detection}

Most related work, including \cite{pvn3d, ffb6d}, and \cite{mv6d} does not specifically consider object symmetries. 
However, symmetries lead to ambiguities in the predicted keypoints as multiple 6D poses can have the same visual and geometric appearance. 
Therefore, we introduce a novel symmetry-aware training procedure for the 3D keypoint detection including a novel symmetry-aware objective function to make the network predicting either the original set of target keypoints for an object or a rotated version of the set corresponding to one object symmetry. Either way, we can still apply the least-squares fitting which efficiently computes an estimate of the target 6D pose or a rotated version corresponding to an object symmetry. To do so, we precompute the set $\boldsymbol{S}_I$ of all rotational symmetric transformations for the given object instance $I$ with a stochastic gradient
descent algorithm \cite{sgdr}.
Given the known mesh of an object and an initial estimate for the symmetry axis, we transform the object mesh along the symmetry axis estimate and optimize the symmetry axis iteratively by minimizing the ADD-S metric \cite{hinterstoisser2012model}.
Reflectional symmetries which can be represented as rotational symmetries are handled as rotational symmetries. 
Other reflectional symmetries are ignored, since the reflection cannot be expressed as an Euclidean transformation.
To consider continuous rotational symmetries, we discretize them into 16 discrete rotational symmetry transformations.

We extend the keypoints loss function of \cite{pvn3d} to become symmetry-aware such that it predicts the keypoints of the closest symmetric transformation, i.e. 
\begin{equation}
    L_\text{kp}(\mathcal{I}) = \frac{1}{N_I} 
    \min_{\boldsymbol{S} \in \boldsymbol{S}_I} 
    \sum_{i \in \mathcal{I}} \sum_{j=1}^M 
    \norm{\boldsymbol{x}_{ij} - \boldsymbol{S}\boldsymbol{\widehat{x}}_{ij}}_2, 
\label{eq_keypoint_loss}
\end{equation}
where $N_I$ is the number of points in the point cloud for object instance $I$, $M$ is the number of target keypoints per object, and $\mathcal{I}$ is the set of all point indices that belong to object instance $I$.  
The vector $\boldsymbol{\widehat{x}}_{ij}$ is the predicted keypoint offset for the $i$-th point and the $j$-th keypoint while $\boldsymbol{x}_{ij}$ is the corresponding ground truth. 



\subsection{Objective Function}

We train our network by minimizing the multi-task loss function
\begin{equation}
 \label{eq_total_loss}
    L_\text{multi-task} = \lambda_1 L_\text{kp} 
    + \lambda_2 L_\text{semantic}  
    +  \lambda_3 L_\text{cp},
\end{equation}
where $L_\text{kp}$ is our symmetry-aware keypoint loss from \cref{eq_keypoint_loss}.
$L_\text{cp}$ is an L1 loss for the center point prediction, $L_\text{semantic}$ is a Focal loss \cite{focalLoss} for the instance semantic segmentation, and $\lambda_1=2$, $\lambda_2=1$, and $\lambda_3=1$ are the weights for the individual loss functions as in \cite{ffb6d}.


\begin{figure*}
    \centering
    \includegraphics[width=1.0\linewidth]{Figures/imgs/tsne_motivation.pdf}
    \caption{$t$-SNE~\cite{tsne} visualizations of features learned from ER and \frameworkName on the test set of CIFAR-10.
    When learning new classes, ER suffers serious class confusion probably because shortcut learning. In contrast, \frameworkName significantly mitigates the forgetting.
    }
    \label{fig:tsne_motivation}
\end{figure*}
\begin{table*}[ht]
\small
\begin{center}
\resizebox{\linewidth}{!}{
\begin{tabular}{rrrrrrrrrrrr}
\shline
\multirow{2}{*}{Method}  & \multicolumn{3}{c}{CIFAR-10}   && \multicolumn{3}{c}{CIFAR-100}  && \multicolumn{3}{c}{TinyImageNet} \\ \cline{2-4}\cline{6-8}\cline{10-12}
       & $M=0.1k$   & $M=0.2k$   & $M=0.5k$     && $M=0.5k$     & $M=1k$     & $M=2k$     && $M=1k$      & $M=2k$ & $M=4k$   \\ \midrule
iCaRL~\cite{iCaRL}    & 31.0\std{$\pm$1.2} & 33.9\std{$\pm$0.9} & 42.0\std{$\pm$0.9} && 12.8\std{$\pm$0.4}  & 16.5\std{$\pm$0.4}  & 17.6\std{$\pm$0.5} && 5.0\std{$\pm$0.3}   & 6.6\std{$\pm$0.4} & 7.8\std{$\pm$0.4} \\ 
DER++~\cite{DER++}   & 31.5\std{$\pm$2.9} & 39.7\std{$\pm$2.7} & 50.9\std{$\pm$1.8} && 16.0\std{$\pm$0.6}  & 21.4\std{$\pm$0.9}  & 23.9\std{$\pm$1.0} && 3.7\std{$\pm$0.4} & 5.1\std{$\pm$0.8} & 6.8\std{$\pm$0.6} \\ 
PASS~\cite{protoAug}    & 33.7\std{$\pm$2.2} & 33.7\std{$\pm$2.2} & 33.7\std{$\pm$2.2} && 7.5\std{$\pm$0.7}  & 7.5\std{$\pm$0.7}  & 7.5\std{$\pm$0.7} && 0.5\std{$\pm$0.1}   & 0.5\std{$\pm$0.1} & 0.5\std{$\pm$0.1} \\ 
\hline
AGEM~\cite{AGEM}    & 17.7\std{$\pm$0.3} & 17.5\std{$\pm$0.3} & 17.5\std{$\pm$0.2} && 5.8\std{$\pm$0.1}  & 5.9\std{$\pm$0.1}  & 5.8\std{$\pm$0.1} && 0.8\std{$\pm$0.1}   & 0.8\std{$\pm$0.1} & 0.8\std{$\pm$0.1} \\ 
GSS~\cite{GSS}     & 18.4\std{$\pm$0.2} & 19.4\std{$\pm$0.7} & 25.2\std{$\pm$0.9} && 8.1\std{$\pm$0.2}  & 9.4\std{$\pm$0.5}  & 10.1\std{$\pm$0.8} && 1.1\std{$\pm$0.1}   & 1.5\std{$\pm$0.1} & 2.4\std{$\pm$0.4} \\ 
ER~\cite{ER}      & 19.4\std{$\pm$0.6} & 20.9\std{$\pm$0.9} & 26.0\std{$\pm$1.2} && 8.7\std{$\pm$0.3}  & 9.9\std{$\pm$0.5}  & 10.7\std{$\pm$0.8} && 1.2\std{$\pm$0.1}   & 1.5\std{$\pm$0.2} & 2.0\std{$\pm$0.2} \\ 
MIR~\cite{MIR}     & 20.7\std{$\pm$0.7} & 23.5\std{$\pm$0.8} & 29.9\std{$\pm$1.2} && 9.7\std{$\pm$0.3}  & 11.2\std{$\pm$0.4}  & 13.0\std{$\pm$0.7} && 1.4\std{$\pm$0.1}   & 1.9\std{$\pm$0.2} & 2.9\std{$\pm$0.3} \\ 
GDumb~\cite{GDumb}   & 23.3\std{$\pm$1.3} & 27.1\std{$\pm$0.7} & 34.0\std{$\pm$0.8} && 8.2\std{$\pm$0.2}  & 11.0\std{$\pm$0.4}  & 15.3\std{$\pm$0.3} && 4.6\std{$\pm$0.3}   & 6.6\std{$\pm$0.2} & 10.0\std{$\pm$0.3} \\ 
ASER~\cite{ASER}   & 20.0\std{$\pm$1.0} & 22.8\std{$\pm$0.6} & 31.6\std{$\pm$1.1} && 11.0\std{$\pm$0.3}  & 13.5\std{$\pm$0.3}  & 17.6\std{$\pm$0.4} && 2.2\std{$\pm$0.1}   & 4.2\std{$\pm$0.6} & 8.4\std{$\pm$0.7} \\ 
SCR~\cite{SCR}     & 40.2\std{$\pm$1.3} & 48.5\std{$\pm$1.5} & 59.1\std{$\pm$1.3} && 19.3\std{$\pm$0.6}  & 26.5\std{$\pm$0.5}  & 32.7\std{$\pm$0.3} && 8.9\std{$\pm$0.3}   & 14.7\std{$\pm$0.3} & 19.5\std{$\pm$0.3} \\ 
CoPE~\cite{online_pro_ema}  & 33.5\std{$\pm$3.2} & 37.3\std{$\pm$2.2} & 42.9\std{$\pm$3.5} && 11.6\std{$\pm$0.7}  & 14.6\std{$\pm$1.3}  & 16.8\std{$\pm$0.9} && 2.1\std{$\pm$0.3}   & 2.3\std{$\pm$0.4} & 2.5\std{$\pm$0.3} \\
DVC~\cite{DVC} & 35.2\std{$\pm$1.7}  & 41.6\std{$\pm$2.7} & 53.8\std{$\pm$2.2} &&  15.4\std{$\pm$0.7} & 20.3\std{$\pm$1.0} & 25.2\std{$\pm$1.6} && 4.9\std{$\pm$0.6} &  7.5\std{$\pm$0.5} & 10.9\std{$\pm$1.1} \\ 
OCM~\cite{OCM} & 47.5\std{$\pm$1.7}  & 59.6\std{$\pm$0.4} & 70.1\std{$\pm$1.5} && 19.7\std{$\pm$0.5} & 27.4\std{$\pm$0.3} & 34.4\std{$\pm$0.5} && 10.8\std{$\pm$0.4} & 15.4\std{$\pm$0.4} & 20.9\std{$\pm$0.7} \\ 
\hline
\frameworkName (\textbf{ours}) & \textbf{57.8}\std{$\pm$1.1} & \textbf{65.5}\std{$\pm$1.0} & \textbf{72.6}\std{$\pm$0.8} && \textbf{22.7}\std{$\pm$0.7} & \textbf{30.0}\std{$\pm$0.4} & \textbf{35.9}\std{$\pm$0.6} && \textbf{11.9}\std{$\pm$0.3} & \textbf{16.9}\std{$\pm$0.4} &  \textbf{22.1}\std{$\pm$0.4}
\\ 
\shline
\end{tabular}
}
\end{center}
\caption{Average Accuracy~(higher is better) on three benckmark datasets with different memory bank sizes $M$. All results are the average and standard deviation of 15 runs.}
\label{tab:acc}
\end{table*}

\section{Experiments}
\subsection{Experimental Setup}
\paragraph{Datasets.}
We use three image classification benchmark datasets, including \textbf{CIFAR-10}~\cite{cifar10_100}, \textbf{CIFAR-100}~\cite{cifar10_100}, and \textbf{TinyImageNet}~\cite{tinyImageNet}, to evaluate the performance of online CIL methods. 
Following~\cite{ASER, SCR, DVC}, we split CIFAR-10 into 5 disjoint tasks, where each task has 2 disjoint classes, 10,000 samples for training, and 2,000 samples for testing, and split CIFAR-100 into 10 disjoint tasks, where each task has 10 disjoint classes, 5,000 samples for training, and 1,000 samples for testing.
Following~\cite{OCM}, we split TinyImageNet into 100 disjoint tasks, where each task has 2 disjoint classes, 1,000 samples for training, and 100 samples for testing.
Note that the order of tasks is fixed in all experimental settings.

\paragraph{Baselines.}
We compare our \frameworkName with 13 baselines, including 10 replay-based online CL baselines: {AGEM}~\cite{AGEM}, {MIR}~\cite{MIR}, {GSS}~\cite{GSS}, {ER}~\cite{ER}, {GDumb}~\cite{GDumb}, {ASER}~\cite{ASER}, {SCR}~\cite{SCR}, {CoPE}~\cite{online_pro_ema}, {DVC}~\cite{DVC}, and {OCM}~\cite{OCM}; 3 offline CL baselines that use knowledge distillation by running them in one epoch: {iCaRL}~\cite{iCaRL}, {DER++}~\cite{DER++}, and PASS~\cite{protoAug}. Note that PASS is a non-exemplar method.

\paragraph{Evaluation metrics.}
We use Average Accuracy and Average Forgetting~\cite{ASER, DVC} to measure the performance of our framework in online CIL. Average Accuracy evaluates the accuracy of the test sets from all seen tasks, defined as $\text {Average Accuracy} =\frac{1}{T} \sum_{j=1}^T a_{T, j},$
where $a_{i, j}$ is the accuracy on task $j$ after the model is trained from task $1$ to $i$.
Average Forgetting represents how much the model forgets about each task after being trained on the final task, defined as
$\text { Average Forgetting } =\frac{1}{T-1} \sum_{j=1}^{T-1} f_{T, j}, 
\text { where } f_{i, j}=\max _{k \in\{1, \ldots, i-1\}} a_{k, j}-a_{i, j}.$

\paragraph{Implementation details.}
We use ResNet18~\cite{ResNet} as the backbone $f$ and a linear layer as the projection head $g$ like~\cite{SCR, OCM, Co2L}; the hidden dim in $g$ is set to 128 as~\cite{SimCLR}. We also employ a linear layer as the classifier $\varphi$. We train the model from scratch with Adam optimizer and an initial learning rate of $5\times10^{-4}$ for all datasets. The weight decay is set to $1.0\times10^{-4}$. Following~\cite{ASER, DVC}, we set the batch size $N$ as 10, and following~\cite{OCM} the replay batch size $m$ is set to 64. 
For CIFAR-10, we set the ratio of \dataaugname $\alpha = 0.25$. For CIFAR-100 and TinyImageNet, $\alpha$ is set to $0.1$. The temperature $\tau = 0.5$ and $\tau^{\prime} = 0.07$.
For baselines, we also use ResNet18 as their backbone and set the same batch size and replay batch size for fair comparisons.
We reproduce all baselines in the same environment with their source code and default settings; see Appendix~\ref{appendix:baselines} for implementation details about all baselines.
We report the average results across 15 runs for all experiments.



\paragraph{Data augmentation.}
Similar to data augmentations used in SimCLR~\cite{SimCLR}, we use resized-crop, horizontal-flip, and gray-scale as our data augmentations. For all baselines, we also use these augmentations. In addition, for DER++\cite{DER++}, SCR~\cite{SCR}, and DVC~\cite{DVC}, we follow their default settings and use their own extra data augmentations. OCM~\cite{OCM} uses extra rotation augmentations, which are also used in \frameworkName.


\subsection{Motivation Justification}
\label{pre_exp}
\paragraph{Shortcut learning in online CL.}
Shortcut learning is severe in online CL since the model cannot learn sufficient representative features due to the single-pass data stream. To intuitively demonstrate this issue,  
we conduct GradCAM++~\cite{Grad-cam++} on the training set of CIFAR-10 ($M=0.2k$) after the model is trained incrementally, as shown in Fig.~\ref{fig:heatmap}.
Each row in Fig.~\ref{fig:heatmap} represents a task with two classes.
We can observe that although ER and DVC predict the correct class, the models actually take shortcuts and focus on some object-unrelated features. 
An interesting phenomenon is that ER tends to take shortcuts in each task. For example, ER learns the sky on both the airplane class in task 1 (the first row) and the bird class in task 2 (the second row) . Thus, ER forgets almost all the knowledge of the old classes.  
DVC maximizes the mutual information between instances like contrastive learning~\cite{SimCLR, MoCo}, which only partially alleviates shortcut learning in online CL. 
In contrast, \frameworkName focuses on the representative features of the objects themselves. The results confirm that learning representative features is crucial against shortcut learning; see Appendix~\ref{appendix:more_visual} for more visual explanations.


\begin{table*}[htbp]
\small
\begin{center}
\resizebox{\linewidth}{!}{
\begin{tabular}{rrrrrrrrrrrr}
\shline
\multirow{2}{*}{Method}  & \multicolumn{3}{c}{CIFAR-10}   && \multicolumn{3}{c}{CIFAR-100}  && \multicolumn{3}{c}{TinyImageNet} \\ \cline{2-4}\cline{6-8}\cline{10-12}
       &  $M=0.1k$   &  $M=0.2k$   &  $M=0.5k$     &&  $M=0.5k$     &  $M=1k$     &  $M=2k$    &&  $M=1k$      &  $M=2k$ &  $M=4k$    \\ \midrule
iCaRL~\cite{iCaRL}    & 52.7\std{$\pm$1.0} & 49.3\std{$\pm$0.8} & 38.3\std{$\pm$0.9} && 16.5\std{$\pm$1.0}  & 11.2\std{$\pm$0.4}  & 10.4\std{$\pm$0.4} && 9.9\std{$\pm$0.5}   & 10.1\std{$\pm$0.5} & 9.7\std{$\pm$0.6} \\ 
DER++~\cite{DER++}   & 57.8\std{$\pm$4.1} & 46.7\std{$\pm$3.6} & 33.6\std{$\pm$3.5} && 41.0\std{$\pm$1.1} & 34.8\std{$\pm$1.1} & 33.2\std{$\pm$1.2} && 77.8\std{$\pm$1.0} & 74.9\std{$\pm$0.6} & 73.2\std{$\pm$0.8}  \\ 
PASS~\cite{protoAug}    & 21.2\std{$\pm$2.2} & 21.2\std{$\pm$2.2} & 21.2\std{$\pm$2.2} && 10.6\std{$\pm$0.9}  & 10.6\std{$\pm$0.9}  & 10.6\std{$\pm$0.9} && 27.0\std{$\pm$2.4}   & 27.0\std{$\pm$2.4} & 27.0\std{$\pm$2.4} \\ 
\hline
AGEM~\cite{AGEM}    & 64.8\std{$\pm$0.7} & 64.8\std{$\pm$0.7} & 64.5\std{$\pm$0.5} && 41.7\std{$\pm$0.8} & 41.8\std{$\pm$0.7} & 41.7\std{$\pm$0.6} && 73.9\std{$\pm$0.7} & 73.1\std{$\pm$0.7} & 72.9\std{$\pm$0.5} \\ 
GSS~\cite{GSS}     & 67.1\std{$\pm$0.6} & 65.8\std{$\pm$0.6} & 61.2\std{$\pm$1.2} && 48.7\std{$\pm$0.8} & 46.7\std{$\pm$1.3} & 44.7\std{$\pm$1.1} && 78.9\std{$\pm$0.7} & 77.0\std{$\pm$0.5} & 75.2\std{$\pm$0.7} \\ 
ER~\cite{ER}      & 64.7\std{$\pm$1.1} & 62.9\std{$\pm$1.0} & 57.5\std{$\pm$1.8} && 47.0\std{$\pm$1.0} & 46.4\std{$\pm$0.8} & 44.7\std{$\pm$1.5} && 79.1\std{$\pm$0.6} & 77.7\std{$\pm$0.6} & 76.3\std{$\pm$0.5} \\ 
MIR~\cite{MIR}     & 62.6\std{$\pm$1.0} & 58.5\std{$\pm$1.4} & 51.1\std{$\pm$1.1} && 45.7\std{$\pm$0.9} & 44.2\std{$\pm$1.3} & 42.3\std{$\pm$1.0} && 75.3\std{$\pm$0.9} & 71.5\std{$\pm$1.0} & 66.8\std{$\pm$0.8} \\ 
GDumb~\cite{GDumb}   & 28.5\std{$\pm$1.4} & 28.4\std{$\pm$1.0} & 28.1\std{$\pm$1.0} && 25.0\std{$\pm$0.4} & 23.2\std{$\pm$0.4} & 20.7\std{$\pm$0.3}  && 22.7\std{$\pm$0.3} & 18.4\std{$\pm$0.2} & 17.0\std{$\pm$0.2} \\
ASER~\cite{ASER}    & 64.8\std{$\pm$1.0} & 62.6\std{$\pm$1.1} & 53.2\std{$\pm$1.5} && 52.8\std{$\pm$0.8} & 50.4\std{$\pm$0.9} & 46.8\std{$\pm$0.7} && 78.9\std{$\pm$0.5} & 75.4\std{$\pm$0.7} & 68.2\std{$\pm$1.1} \\ 
SCR~\cite{SCR}     & 43.2\std{$\pm$1.5} & 35.5\std{$\pm$1.8} & 24.1\std{$\pm$1.0} && 29.3\std{$\pm$0.9} & 20.4\std{$\pm$0.6} & 11.5\std{$\pm$0.6} && 44.8\std{$\pm$0.6} & 26.8\std{$\pm$0.5} & 20.1\std{$\pm$0.4} \\ 
CoPE~\cite{online_pro_ema}  & 49.7\std{$\pm$1.6} & 45.7\std{$\pm$1.5} & 39.4\std{$\pm$1.8} && 25.6\std{$\pm$0.9}  & 17.8\std{$\pm$1.3}  & 14.4\std{$\pm$0.8} && 11.9\std{$\pm$0.6}   & 10.9\std{$\pm$0.4} & 9.7\std{$\pm$0.4} \\
DVC~\cite{DVC} & 40.2\std{$\pm$2.6} & 31.4\std{$\pm$4.1} & 21.2\std{$\pm$2.8} && 32.0\std{$\pm$0.9} & 32.7\std{$\pm$2.0} & 28.0\std{$\pm$2.2} && 59.8\std{$\pm$2.2} & 52.9\std{$\pm$1.3} & 45.1\std{$\pm$1.9} \\
OCM~\cite{OCM} & 35.5\std{$\pm$2.4} & 23.9\std{$\pm$1.4} & 13.5\std{$\pm$1.5} && 18.3\std{$\pm$0.9} & 15.2\std{$\pm$1.0} & 10.8\std{$\pm$0.6} && 23.6\std{$\pm$0.5} & 26.2\std{$\pm$0.5}  & 23.8\std{$\pm$1.0} \\ 
\hline
{\frameworkName} (\textbf{ours})   & 23.2\std{$\pm$1.3} & 17.6\std{$\pm$1.4} & 12.5\std{$\pm$0.7} && 
15.0\std{$\pm$0.8} & 10.4\std{$\pm$0.5} & 6.1\std{$\pm$0.6} && 21.3\std{$\pm$0.5} & 17.4\std{$\pm$0.4} & 16.8\std{$\pm$0.4} \\
\shline
\end{tabular}
}
\end{center}
\caption{Average Forgetting~(lower is better) on three benckmark datasets. All results are the average and standard deviation of 15 runs.}
\label{tab:forget}
\end{table*}

\begin{figure*}[htp]
  \centering
  \subfloat[Average incremental performance]{
    \includegraphics[width=0.55\linewidth]{Figures/imgs/incremental_step_acc.pdf}
    \label{fig:incrementalAcc}
  }
  \subfloat[Confusion matrix of OCM and \frameworkName]{
    \includegraphics[width=0.42\linewidth]{Figures/imgs/confusion_matrix.pdf}
    \label{fig:confusionMatrix}
  }
  \caption{Incremental accuracy on tasks observed so far and confusion matrix of accuracy (\%) in the {test set} of CIFAR-10.}
  \label{fig:incrementalAcc_confusionMatrix}
\end{figure*}



\paragraph{Class confusion in online CL.}
Fig.~\ref{fig:tsne_motivation} provides the $t$-SNE~\cite{tsne} visualization results for ER and \frameworkName on the test set of CIFAR-10 ($M=0.2k$). 
We can draw intuitive observations as follows. 
(1) There is serious class confusion in ER.
When the new task (task 2) arrives, features learned in task 1 are not discriminative for task 2, leading to class confusion and decreased performance in old classes.
(2) Shortcut learning may cause class confusion. For example, the performance of ER decreases more on airplanes compared to automobiles, probably because birds in the new task have more similar backgrounds to airplanes, as shown in Fig.~\ref{fig:heatmap}.
(3) \frameworkName achieves better discrimination both on task 1 and task 2. The results demonstrate that \frameworkName can maintain discrimination of all seen classes and significantly mitigate forgetting by 
combining the proposed \methodname and \dataaugname.






\subsection{Results and Analysis}
\label{result}
\paragraph{Performance of average accuracy.}
Table~\ref{tab:acc} presents the results of average accuracy with different memory bank sizes ($M$) on three benchmark datasets. Our \frameworkName consistently outperforms all baselines on three datasets.
Remarkably, the performance improvement of \frameworkName is more significant when the memory bank size is relatively small; this is critical for online CL with limited resources. For example, compared to the second-best method OCM, \frameworkName achieves about 10$\%$ and 6$\%$ improvement on CIFAR-10 when $M$ is 100 and 200, respectively. 
The results show that our \frameworkName can learn more representative and discriminative features with a limited memory bank.
Compared to baselines that use knowledge distillation (iCaRL, DER++, PASS, OCM), our \frameworkName achieves better performance by leveraging the feedback of online prototypes.  
Besides, \frameworkName significantly outperforms PASS and CoPE that also use prototypes, showing that online prototypes are more suitable for online CL. 


We find that the performance improvement tends to be gentle when $M$ increases.
The reason is that as $M$ increases, the samples in the memory bank become more diverse, and the model can extract sufficient information from massive samples to distinguish seen classes. 
In addition, many baselines perform poorly on CIFAR-100 and TinyImageNet due to a dramatic increase in the number of tasks. In contrast, \frameworkName still performs well and improves accuracy over the second best.



\paragraph{Performance of average forgetting.}
We report the Average Forgetting results of our \frameworkName and all baselines on three benchmark datasets in Table~\ref{tab:forget}. The results confirm that \frameworkName can effectively mitigate catastrophic forgetting. 
For CIFAR-10 and CIFAR-100, \frameworkName achieves the lowest average forgetting compared to all replay-based baselines. 
For TinyImageNet, our result is a little higher than iCaRL and CoPE but better than the latest methods DVC and OCM. 
The reason is that iCaRL uses a nearest class mean classifier, but we use softmax and FC layer during the test phase, and CoPE slowly updates prototypes with a high momentum.
However, as shown in Table~\ref{tab:acc}, \frameworkName provides more accurate classification results than iCaRL and CoPE. 
It is a fact that when the maximum accuracy of a task is small, the forgetting on this task is naturally rare, even if the model completely forgets what it learned.





\paragraph{Performance of each incremental step.}
We evaluate the average incremental performance~\cite{DER++, DVC} on CIFAR-10 ($M=0.1k$) and CIFAR-100 ($M=0.5k$), which indicates the accuracy over all seen tasks at each incremental step. 
Fig.~\ref{fig:incrementalAcc} shows that \frameworkName achieves better accuracy and effectively mitigates forgetting while the performance of most baselines degrades rapidly with the arrival of new classes.

\paragraph{Confusion matrices at the end of learning.}
We report the confusion matrices of our \frameworkName and the second-best method OCM, as shown in Fig.~\ref{fig:confusionMatrix}. 
After learning the last task (\ie, the last two classes), OCM forgets the knowledge of early tasks (classes 0 to 3). 
In contrast, \frameworkName performs relatively well in all classes, especially in the first task (classes 0 and 1), outperforming OCM by 27.8\% average improvements.
The results show that learning representative and discriminative features is crucial to mitigate catastrophic forgetting; see Appendix~\ref{appendix:extra_exp} for extra experimental results.  




\subsection{Ablation Studies}
\label{ablation}

\begin{table}[t]
\small
\begin{center}
\begin{tabular}{ccccc}
\shline
\multirow{2}{*}{{Method}} & {CIFAR-10}&{CIFAR-100} \\
& Acc $\uparrow$(Forget $\downarrow$) & Acc $\uparrow$(Forget $\downarrow$) \\ 
\midrule
baseline & 46.4\std{$\pm$1.2}(36.0\std{$\pm$}2.1) & 18.8\std{$\pm$0.8}(18.5\std{$\pm$}0.7) \\
w/o \methodname & 53.1\std{$\pm$1.4}(24.7\std{$\pm$2.0}) & 19.3\std{$\pm$0.7}(15.9\std{$\pm$0.9}) \\
w/o \dataaugname & 52.0\std{$\pm$1.5}(34.6\std{$\pm$2.4}) & 21.5\std{$\pm$0.5}(16.3\std{$\pm$0.8}) \\ 
\hline
w/o $\mathcal{L}^{\mathrm{new}}_{\mathrm{pro}}$ & 54.8\std{$\pm$1.2}(\textbf{22.1}\std{$\pm$3.0}) & 19.6\std{$\pm$0.8}(19.9\std{$\pm$0.7}) \\
w/o $\mathcal{L}^{\mathrm{seen}}_{\mathrm{pro}}$ & 55.7\std{$\pm$1.4}(25.5\std{$\pm$1.5}) & 20.1\std{$\pm$0.4}(16.2\std{$\pm$0.6}) \\ 
$\mathcal{L}^{\mathrm{seen}}_{\mathrm{pro}}$ w/o $\mathcal{C}^\mathrm{new}$ & 56.2\std{$\pm$1.2}(26.4\std{$\pm$2.3}) & 20.8\std{$\pm$0.6}(17.9\std{$\pm$0.7}) \\ 
\hline
{\frameworkName} (\textbf{ours}) & \textbf{57.8}\std{$\pm$1.1}(23.2\std{$\pm$1.3}) & \textbf{22.7}\std{$\pm$0.7}(\textbf{15.0}\std{$\pm$0.8}) \\ 
\shline 
\end{tabular}
\end{center}
\caption{Ablation studies on CIFAR-10 ($M=0.1k$) and CIFAR-100 ($M=0.5k$). 
``baseline'' means $\mathcal{L}_\mathrm{INS}+\mathcal{L}_\mathrm{CE}$.
``$\mathcal{L}^{\mathrm{seen}}_{\mathrm{pro}}$ w/o $\mathcal{C}^\mathrm{new}$'' means $\mathcal{L}^{\mathrm{seen}}_{\mathrm{pro}}$ do not consider new classes in current task.
}
\label{tab:ablation}
\end{table}

\paragraph{Effects of each component.} Table~\ref{tab:ablation} presents the ablation results of each component. Obviously, \methodname and \dataaugname can consistently improve the average accuracy of classification. 
We can observe that the effect of \methodname is more significant on more tasks while \dataaugname plays a crucial role when the memory bank size is limited. Moreover, when combining \methodname and \dataaugname, the performance is further improved, which indicates that both can benefit from each other. For example, \dataaugname boosts \methodname by about 6$\%$ improvements on CIFAR-10 ($M=0.1k$), and the performance of \dataaugname is improved by about 3$\%$ on CIFAR-100 ($M=0.5k$) by combining \methodname.


\paragraph{Equilibrium in \methodname.}
When learning new classes, the data of new classes is involved in both $\mathcal{L}^{\mathrm{new}}_{\mathrm{pro}}$ and $\mathcal{L}^{\mathrm{seen}}_{\mathrm{pro}}$ of \methodname, where $\mathcal{L}^{\mathrm{new}}_{\mathrm{pro}}$ only focuses on learning new knowledge while $\mathcal{L}^{\mathrm{seen}}_{\mathrm{pro}}$ tends to alleviate forgetting on seen classes.
To explore the best way of learning new classes, we consider three scenarios for \methodname in Table~\ref{tab:ablation}.
The results show that only learning new knowledge (w/o $\mathcal{L}^{\mathrm{seen}}_{\mathrm{pro}}$) or only consolidating the previous knowledge (w/o $\mathcal{L}^{\mathrm{new}}_{\mathrm{pro}}$) can significantly degrade the performance, which indicates that both are indispensable for online CL.
Furthermore, when $\mathcal{L}^{\mathrm{seen}}_{\mathrm{pro}}$ only considers old classes and ignores new classes ($\mathcal{L}^{\mathrm{seen}}_{\mathrm{pro}}$ w/o $\mathcal{C}^\mathrm{new}$), the performance also decreases. These results show that the equilibrium of all seen classes (\methodname) can achieve the best performance and is crucial for online CL.


\paragraph{Effects of \dataaugname.} 
To verify the advantage of \dataaugname, we compare it with the completely random mixup
in Table~\ref{tab:ablation_mixup}.
\begin{table}
\small
\begin{center}
\begin{tabular}{c|rrr}
\shline
\multicolumn{1}{c|}{Method}       & ${M=0.1k}$   & ${M=0.2k}$   & ${M=0.5k}$     \\ \hline
Random & 53.5\std{$\pm$2.7} & 62.9\std{$\pm$2.5} & 70.8\std{$\pm$2.2} \\
\dataaugname (\textbf{ours})  & \textbf{57.8}\std{$\pm$1.1} & \textbf{65.5}\std{$\pm$1.0} & \textbf{72.6}\std{$\pm$0.8} \\ 
\shline
\end{tabular}
\end{center}
\caption{Comparison of Random Mixup and \dataaugname on CIFAR-10. 
}
\label{tab:ablation_mixup}
\end{table}
\dataaugname outperforms random mixup in all three scenarios. Notably, \dataaugname works significantly when the memory bank size is small, which shows that the feedback can prevent class confusion due to a restricted memory bank; see Appendix~\ref{appendix:ablations} for extra ablation studies.



\subsection{Validation of Online Prototypes}
\label{prove_onlinePrototypes}
\begin{figure}
    \centering
    \includegraphics[width=1.0\linewidth]{Figures/imgs/cosine_similarity.pdf}
    \caption{The cosine similarity between online prototypes and prototypes of the entire memory bank.}
    \label{fig:cosine_similarity}
\end{figure}
Fig.~\ref{fig:cosine_similarity} shows the cosine similarity between online prototypes and global prototypes (prototypes of the entire memory bank) at each time step.
For the first mini-batch of each task, online prototypes are equal to global prototypes (similarity is 1, omitted in Fig.~\ref{fig:cosine_similarity}).
In the first task, online and global prototypes are updated synchronously with the model updates, resulting in high similarity. 
In subsequent tasks, the model initially learns inadequate features of new classes, causing online prototypes to be inconsistent with global prototypes and low similarity, which shows that accumulating early features as prototypes may be harmful to new tasks. However, the similarity will improve as the model learns, because the model gradually learns representative features of new classes.
Furthermore, the similarity on old classes is only slightly lower, showing that online prototypes are resistant to forgetting. 

\section{Conclusion}
In this work, we present a novel strategy for addressing few-shot open-set recognition. We frame the few-shot open-set classification task as a meta-learning problem similar to \cite{peeler}, but unlike their strategy, we do not solely rely on thresholding softmax scores to indicate the openness of a sample. We argue that existing thresholding type FSOSR methods \cite{peeler,snatcher} rely heavily on the choice of a carefully tuned threshold to achieve good performance. Additionally, the proclivity of softmax to overfit to unseen classes makes it an unreliable choice as an open-set indicator, especially when there is a dearth of samples. Instead, we propose to use a reconstruction of exemplar images as a key signal to detect out-of-distribution samples. 
The learned embedding which is used to classify the sample is further modulated to ensure a proficient gap between the seen and unseen class clusters in the feature space. Finally, the modulated embedding, the softmax score, and the quality reconstructed exemplar are jointly utilized to cognize if the sample is in-distribution or out-of-distribution. 
The enhanced performance of our framework is verified empirically over a wide variety of few-shot tasks and the results establish it as the new state-of-the-art. In the future, we would like to extend this approach to more cross-domain few-shot tasks, including videos.
\vspace{-2em}
\section{Acknowledgement}
This work was partially supported by US National Science Foundation grant 2008020 and US Office of Naval Research grants N00014-19-1-2264 and N00014-18-1-2252.
\vspace{-1em}
\ifCLASSOPTIONcaptionsoff
  \newpage
\fi

\bibliographystyle{IEEEtran}
\bibliography{IEEEabrv,ref}
% \begin{thebibliography}{1}
% \bibliography{ref.bib}
% \bibitem{IEEEhowto:kopka}
% H.~Kopka and P.~W. Daly, \emph{A Guide to \LaTeX}, 3rd~ed.\hskip 1em plus
%   0.5em minus 0.4em\relax Harlow, England: Addison-Wesley, 1999.

% \end{thebibliography}


% \appendices
% \section{Hyperparameters}
% Fig. \ref{fig:hyper_params} shows an how the open-set AUROC is affected by the hyperparameters $\lambda_1$ or $\lambda_3$ (from Eq. \ref{loss_func}) for the GTSRB$\rightarrow$TT100K task. In both cases, the classification term $\lambda_2$ is fixed at $10$. In Fig. \ref{subfig:lmbda_1}, when $\lambda_1$ is very low, the open-set detection is hampered due to poor quality of the reconstruction and when it is increased beyond $10^{-4}$ reconstruction becomes the sole objective of the model, thus the open-set detection again degrades. From Fig. \ref{subfig:lmbda_3} we can see that for both the $5$-shot and $1$-shot cases increasing $\lambda_3$ causes the AUROC to improve the knee point of $\lambda_3=10$, after which it starts to degrade.

% \begin{figure}[t]
% \centering
% \captionsetup[subfigure]{justification=centering}
% \subfloat[ ]{
% 	\label{subfig:lmbda_1}
% 	\includegraphics[width=0.22\textwidth]{images/lmbda_1.pdf} } 
% \hfill
% \subfloat[ ]{
% 	\label{subfig:lmbda_3}
% 	\includegraphics[width=0.22\textwidth]{images/lmbda_3.pdf} } 
% \caption{\textbf{Hyperparameter Analysis}. (a) $\lambda_2$ \& $\lambda_3$ are fixed at $5$ and $10$,  changing only the reconstruction loss term. (b) $\lambda_1$ and $\lambda_2$ are fixed at $10^{-4}$ and $10$ and $\lambda_3$ is varied.}
% % \vskip -0.1in
% \label{fig:hyper_params}
% \end{figure}

% % you can choose not to have a title for an appendix
% % if you want by leaving the argument blank
% \section{}
% Appendix two text goes here.


% use section* for acknowledgment
% \ifCLASSOPTIONcompsoc
%   % The Computer Society usually uses the plural form
%   \section*{Acknowledgments}
% \else
%   % regular IEEE prefers the singular form
%   \section*{Acknowledgment}
% \fi


% The authors would like to thank...




\vskip -2\baselineskip plus -1fil
\begin{IEEEbiography}[{\includegraphics[width=1in,clip,keepaspectratio]{bio/sayak.pdf}}]{Sayak Nag}
received his Bachelor’s degree in Instrumentation and Electronics Engineering engineering from Jadavpur University, Kolkata, India. Currently, he is pursuing a Ph.D. in the Department of Electrical and Computer Engineering at the University of California, Riverside. His broad research interests include computer vision and machine learning with a focus on few-shot learning, meta-learning, open-set recognition, and weakly-supervised learning.
\end{IEEEbiography}
\vskip -2\baselineskip plus -1fil
\begin{IEEEbiography}[{\includegraphics[width=1in,clip,keepaspectratio]{bio/dripta.pdf}}]{Dripta S. Raychaudhuri} 
received his Ph.D. in Electrical and Computer Engineering from the University of California, Riverside, and his Bachelor’s degree in Electrical and Telecommunication engineering from Jadavpur University, Kolkata, India. He is currently an Applied Scientist at Amazon AWS, USA. His broad research interests include computer vision and machine learning with a focus on multi-task learning, domain adaptation, and imitation learning.
\end{IEEEbiography}
\vskip -2\baselineskip plus -1fil
\begin{IEEEbiography}
[{\includegraphics[width=1in,height=1.25in,clip,keepaspectratio]{bio/sujoy.pdf}}]{Sujoy Paul} received his PhD in Electrical and Computer Engineering from the University of California, Riverside, and his Bachelor’s degree in Electronics and Telecommunication Engineering from Jadavpur University. 
He is currently a Research Scientist at Google Research, India. His broad research interest includes Computer Vision and Machine Learning, focusing on semantic segmentation, human action recognition, domain adaptation, weak supervision, active learning, reinforcement learning, and so on. 
\end{IEEEbiography}
\vskip -2\baselineskip plus -1fil
\begin{IEEEbiography}
[{\includegraphics[width=1in,height=1.25in,clip,keepaspectratio]{bio/amit.pdf}}]{Amit K. Roy-Chowdhury} 
received his PhD from the University of Maryland, College Park (UMCP) in 2002 and joined the University of California, Riverside (UCR) in 2004 where he is a Professor and Bourns Family Faculty Fellow of Electrical and Computer Engineering, Director of the Center for Robotics and Intelligent Systems, and Cooperating Faculty in the department of Computer Science and Engineering. He leads the Video Computing Group at UCR, working on foundational principles of computer vision, image processing, and statistical learning, with applications in cyber-physical, autonomous, and intelligent systems. He has published over 200 papers in peer-reviewed journals and conferences. He has also published two monographs on camera networks and wide-area tracking. He is on the editorial boards of major journals and program committees of the main conferences in his area. He is a Fellow of the IEEE and IAPR, received the Doctoral Dissertation Advising/Mentoring Award 2019 from UCR, and the ECE Distinguished Alumni Award from UMCP.
\end{IEEEbiography}
%
% that's all folks
% \appendices
% % CVPR 2023 Paper Template
% based on the CVPR template provided by Ming-Ming Cheng (https://github.com/MCG-NKU/CVPR_Template)
% modified and extended by Stefan Roth (stefan.roth@NOSPAMtu-darmstadt.de)

\documentclass[10pt,twocolumn,letterpaper]{article}

%%%%%%%%% PAPER TYPE  - PLEASE UPDATE FOR THE FINAL VERSION
% \usepackage[review]{cvpr}      % To produce the REVIEW version
\usepackage{cvpr}              % To produce the CAMERA-READY version
% \usepackage[pagenumbers]{cvpr} % To force page numbers, e.g. for an arXiv version

% Include other packages here, before hyperref.
\usepackage{graphicx}
\usepackage{amsmath}
\usepackage{amssymb}
\usepackage{booktabs}

% My additional packages
\usepackage{mismath}
\usepackage{tabularray}
\usepackage{multirow}
\usepackage[noline]{algorithm2e}
\usepackage[hang,flushmargin]{footmisc}


% It is strongly recommended to use hyperref, especially for the review version.
% hyperref with option pagebackref eases the reviewers' job.
% Please disable hyperref *only* if you encounter grave issues, e.g. with the
% file validation for the camera-ready version.
%
% If you comment hyperref and then uncomment it, you should delete
% ReviewTempalte.aux before re-running LaTeX.
% (Or just hit 'q' on the first LaTeX run, let it finish, and you
%  should be clear).
\usepackage[pagebackref,breaklinks,colorlinks]{hyperref}


% Support for easy cross-referencing
\usepackage[capitalize]{cleveref}
\crefname{section}{Sec.}{Secs.}
\Crefname{section}{Section}{Sections}
\Crefname{table}{Table}{Tables}
\crefname{table}{Tab.}{Tabs.}


%%%%%%%%% PAPER ID  - PLEASE UPDATE
\def\cvprPaperID{42} % *** Enter the CVPR Paper ID here
\def\confName{ECV}
\def\confYear{2023}


\newcommand\manualfootnote[1]{%
  \begingroup
  \renewcommand\thefootnote{}\footnote{#1}%
  \addtocounter{footnote}{-1}%
  \endgroup
}

\begin{document}

%%%%%%%%% TITLE - PLEASE UPDATE
\title{Vision Transformers with Mixed-Resolution Tokenization}

\author{
Tomer Ronen\\
Tel Aviv University\\
{\tt\small tomer.ronen34@gmail.com}
% For a paper whose authors are all at the same institution,
% omit the following lines up until the closing ``}''.
% Additional authors and addresses can be added with ``\and'',
% just like the second author.
% To save space, use either the email address or home page, not both
\and
Omer Levy\\
Tel Aviv University\\
\and
Avram Golbert\\
Google Research$^\text{*}$
% Google Research\thanks{Parts of the work were done while the author was affiliated with Alibaba Group}
% Google Research\thanks{The author was affiliated with Alibaba Group during parts of the work.}
% Google Research\thanks{The author was affiliated with Alibaba Group for some of the work.}
}

\appendix

\renewcommand{\thetable}{A\arabic{table}}
\setcounter{table}{0}


\begin{center}
\LARGE
\textbf{Supplementary Material}
\end{center}

\hfill \break

\section{Full results}
We report ImageNet-1k top-1 accuracy and various cost indicators for every model configuration that appears in the figures of the main text (see Table \ref{table:results_vit_small}, Table \ref{table:results_vit_base}, Table \ref{table:results_vit_large}). Throughput is measured on a single GeForce RTX 3090 GPU in mixed precision.


\section{More implementation details}
\paragraph{Hyperparameters.} We train all of our models using the timm library~\cite{rw2019timm} with the following hyperparameters: learning rate warmup for 5 epochs, learning rate cooldown for 10 epochs, cosine learning rate scheduler~\cite{Loshchilov2016SGDRSG}, weight decay 0.025, DropPath~\cite{Huang2016DeepNW} rate 0.1, AdamW~\cite{Loshchilov2017DecoupledWD} optimizer with epsilon $1\text{e-}8$, AutoAugment~\cite{Cubuk2018AutoAugmentLA} image augmentations with configuration \verb|rand-m9-mstd0.5-inc1|, mixup~\cite{Zhang2017mixupBE} alpha 0.8, cutmix~\cite{Yun2019CutMixRS} alpha 1.0, label smoothing 0.1. Unless otherwise specified, we use base learning rate $5\text{e-}5$.

We fine-tune ViT-Small models for 130 epochs with batch size 1024, ViT-Base models for 60 epochs with batch size 400, and ViT-Large models for 20 epochs with batch size 192. For evaluation, we use exponential moving average (EMA)~\cite{Polyak1992AccelerationOS} with decay 0.99996. We use the default values in timm for all other hyperparameters.


\hfill \break

\begin{table}[h!]
\centering
\textbf{ViT-Small}
\resizebox{0.95\linewidth}{!}{
\begingroup
\renewcommand{\arraystretch}{1.1}
\begin{tabular}{ | c | c c c c | c | }
\hline
\multirow{2}{*}{Method}   &   \multirow{2}{*}{\#Patches}  &  \multirow{2}{*}{GMACs}   &  Throughput  & Runtime  & ImageNet-1k   \\
   &  &  & ims/sec   & $\mu$-secs/im   & Top-1 Acc.   \\
\hline
\multirow{5}{*}{Vanilla ViT}    & 64    & 1.44   & 6489   & 154   & 74.55   \\
 & 81    & 1.83   & 5208   & 192   & 76.36   \\
 & 100   & 2.28   & 4212   & 237   & 77.55   \\
 & 121   & 2.78   & 3460   & 289   & 78.26   \\
 & 169   & 3.94   & 2315   & 432   & 79.84   \\
 & 196   & 4.62   & 1975   & 506   & 80.28   \\
\hline
\multirow{5}{*}{\shortstack[c]{Quadformer \\ \small{Feature-based scorer}}}   & 64    & 1.54   & 3611   & 277   & 76.53   \\
 & 79    & 1.88   & 3204   & 312   & 77.53   \\
 & 100   & 2.37   & 2766   & 362   & 78.64   \\
 & 121   & 2.87   & 2419   & 413   & 79.35   \\
 & 169   & 4.04   & 1792   & 558   & 80.43   \\
 & 196   & 4.71   & 1576   & 635   & 80.84   \\
\hline
\multirow{5}{*}{\shortstack[c]{Quadformer \\ \small{Pixel-blur scorer}}}    & 64    & 1.45   & 5150   & 194   & 74.97   \\
 & 79    & 1.79   & 4362   & 229   & 76.27   \\
 & 100   & 2.28   & 3590   & 279   & 77.47   \\
 & 121   & 2.78   & 3022   & 331   & 78.58   \\
 & 169   & 3.95   & 2104   & 475   & 80.01   \\
 & 196   & 4.62   & 1813   & 552   & 80.4   \\
\hline
\end{tabular}
\endgroup
}% close resizebox
\vspace*{0.2cm}
\caption{Full results - ViT Small.}
\label{table:results_vit_small}
\end{table}

\hfill \break

\hfill \break

\hfill \break

\hfill \break


\begin{table}[h!]
\centering
\textbf{ViT-Base}
\resizebox{0.95\linewidth}{!}{
\begingroup
\renewcommand{\arraystretch}{1.1}
\begin{tabular}{ | c | c c c c | c | }
\hline
\multirow{2}{*}{Method}   &   \multirow{2}{*}{\#Patches}  &  \multirow{2}{*}{GMACs}   &  Throughput  & Runtime  & ImageNet-1k   \\
   &  &  & ims/sec   & $\mu$-secs/im   & Top-1 Acc.   \\
\hline
\multirow{5}{*}{Vanilla ViT}   & 64  &  5.6 & 2676   & 374   & 80.78   \\
   & 81  &  7.2  & 2155   & 464    & 81.73   \\
   & 100 &  8.8  & 1739   & 575    & 82.31   \\
   & 121 &  10.7  & 1429   & 700    & 82.71   \\
   & 169 &  15.1  & 966    & 1035   & 83.74   \\
   & 196 &  17.6  & 823    & 1215   & 84.07   \\
\hline
\multirow{5}{*}{\shortstack[c]{Quadformer \\ \small{Feature-based scorer}}}   &   64    & 5.7  &   2019   &   495   &   81.52   \\
   &   79   & 7.1  &   1732   &   577   &   82.34   \\
   &   100  & 8.9  &   1435   &   697   &   83.05   \\
   &   121  & 10.8 &   1218   &   821   &   83.50   \\
   &   169  & 15.2 &   864    &   1157  &   84.23   \\
   &   196  & 17.7 &   750    &   1333  &   84.38   \\
\hline
\multirow{5}{*}{\shortstack[c]{Quadformer \\ \small{Pixel-blur scorer}}}   & 64 &  5.7  & 2424   & 413   & 80.78   \\
   & 79  & 7.0  & 2021   & 495   & 81.68  \\
   & 100 & 8.8  & 1630   & 613   & 82.57  \\
   & 121 & 10.7 & 1354   & 739   & 83.06   \\
   & 169 & 15.1 & 931    & 1074  & 83.87   \\
   & 196 & 17.6 & 800    & 1250  & 84.23   \\
\hline
\multirow{5}{*}{\shortstack[c]{Quadformer \\ \small{Oracle scorer}}}   & 64   &  ---   &  ---   & ---   & 84.76   \\
   & 79    &  ---   &  ---   & ---   & 85.19   \\
   & 100   &  ---   &  ---   & ---   & 85.40   \\
   & 121   &  ---   &  ---   & ---   & 85.67   \\
   & 169   &  ---   &  ---   & ---   & 85.40   \\
   & 196   &  ---   &  ---   & ---   & 85.25   \\
\hline
\end{tabular}
\endgroup
}% close resizebox
\vspace*{0.2cm}
\caption{Full results - ViT Base.}
\label{table:results_vit_base}
\end{table}



\begin{table}[h!]
% \vspace*{10pt}
\centering
\textbf{ViT-Large}
\resizebox{0.95\linewidth}{!}{
\begingroup
\renewcommand{\arraystretch}{1.1}
\begin{tabular}{ | c | c c c c | c | }
\hline
\multirow{2}{*}{Method}   &   \multirow{2}{*}{\#Patches}  &  \multirow{2}{*}{GMACs}   &  Throughput  & Runtime  & ImageNet-1k   \\
   &  &  & ims/sec   & $\mu$-secs/im   & Top-1 Acc.   \\
\hline
\multirow{5}{*}{Vanilla ViT}   & 64 &  19.9   & 900   & 1111   & 82.00   \\
   & 81  & 25.2   & 720   & 1389   & 83.02   \\
   & 100 & 31.1   & 580   & 1724   & 83.86   \\
   & 121 & 37.7   & 478   & 2092   & 84.46   \\
   & 169 & 53.0   & 323   & 3096   & 85.42   \\
   & 196 & 61.7   & 277   & 3610   & 85.74   \\
\hline
\multirow{5}{*}{\shortstack[c]{Quadformer \\ \small{Feature-based scorer}}}   & 64   & 20.1   & 777   & 1287   & 82.88   \\
   & 79  &  24.7  & 649   & 1541   & 83.67   \\
   & 100 &  31.3  & 527   & 1898   & 84.41   \\
   & 121 &  37.9  & 440   & 2273   & 85.03   \\
   & 169 &  53.1  & 306   & 3268   & 85.65   \\
   & 196 &  61.8  & 265   & 3774   & 85.79   \\
\hline
\multirow{5}{*}{\shortstack[c]{Quadformer \\ \small{Pixel-blur scorer}}}  & 64  & 19.9   & 869   & 1151  & 81.66  \\
 & 79  & 24.6   & 712   & 1404  & 82.69  \\
 & 100 & 31.1   & 568   & 1761  & 83.61  \\
 & 121 & 37.7   & 470   & 2128  & 84.3   \\
 & 169 & 53.0   & 320   & 3125  & 85.22  \\
 & 196 & 61.7   & 275   & 3636  & 85.56  \\
\hline
\multirow{5}{*}{\shortstack[c]{Quadformer \\ \small{Oracle scorer}}}   & 64   &  ---   &  ---   & ---   & 85.89   \\
   & 79    &  ---   &  ---   & ---   & 86.33   \\
   & 100   &  ---   &  ---   & ---   & 86.5   \\
   & 121   &  ---   &  ---   & ---   & 86.7   \\
   & 169   &  ---   &  ---   & ---   & 86.52   \\
   & 196   &  ---   &  ---   & ---   & 86.54   \\
\hline
\end{tabular}
\endgroup
}% close resizebox
\vspace*{0.2cm}
\caption{Full results - ViT-Large.}
\label{table:results_vit_large}
\end{table}


\clearpage

%%%%%%%%% REFERENCES
{\small
\bibliographystyle{ieee_fullname}
\bibliography{egbib}
}


\end{document}


\end{document}


