\IEEEtitleabstractindextext{%
\begin{abstract}
In many applications, we are constrained to learn classifiers from very limited data (few-shot classification). The task becomes even more challenging if it is also required to identify samples from unknown categories (open-set classification). Learning a good abstraction for a class with very few samples is extremely difficult, especially under open-set settings. As a result, open-set recognition has received limited attention in the few-shot setting. However, it is a critical task in many applications like environmental monitoring, where the number of labeled examples for each class is limited. Existing few-shot open-set recognition (FSOSR) methods rely on thresholding schemes, with some considering uniform probability for open-class samples. However, this approach is often inaccurate, especially for fine-grained categorization, and makes them highly sensitive to the choice of a threshold. To address these concerns, we propose Reconstructing Exemplar-based Few-shot Open-set ClaSsifier (ReFOCS). By using a novel exemplar reconstruction-based meta-learning strategy ReFOCS streamlines FSOSR eliminating the need for a carefully tuned threshold by learning to be self-aware of the openness of a sample. The exemplars, act as class representatives and can be either provided in the training dataset or estimated in the feature domain. By testing on a wide variety of datasets, we show ReFOCS to outperform multiple state-of-the-art methods.
\end{abstract}

% Note that keywords are not normally used for peer review papers.
\begin{IEEEkeywords}
Few-Shot Learning, Open-Set Recognition, Out-of-distribution detection, Meta-learning
\end{IEEEkeywords}}


