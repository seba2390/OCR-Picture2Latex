\begin{figure}[ht]
\centering
\vspace{-1em}
\captionsetup[subfigure]{justification=centering}
\subfloat[ ]{
	\label{subfig:exemplars}
	\includegraphics[width=0.5\textwidth]{images/mini_recon/templates.png} } 
\hfill
\subfloat[ ]{
	\label{subfig:in_query}
	\includegraphics[width=0.5\textwidth]{images/mini_recon/in_q.png} } 
\hfill
\subfloat[ ]{
	\label{subfig:out_query}
	\includegraphics[width=0.5\textwidth]{images/mini_recon/out_q.png} } 
\hfill
\subfloat[ ]{
	\label{subfig:inq_recon}
	\includegraphics[width=0.5\textwidth]{images/mini_recon/in_recon.png} } 
	\hfill
\subfloat[ ]{
	\label{subfig:outq_recon}
	\includegraphics[width=0.5\textwidth]{images/mini_recon/out_recon.png} } 
	
\caption{\textbf{Sample exemplar reconstructions for \textit{mini}ImageNet$\rightarrow$\textit{mini}ImageNet.} Exemplar reconstructions of a few query samples from one test episode. (a) Class-wise exemplars are provided in the support set. (b) In-distribution query samples. (c) Out-of-distribution query samples. (d) Reconstructed exemplars from the in-distribution queries. (e) Reconstructed Exemplars from the out-of-distribution queries, as hypothesized when out-of-distribution samples are fed into ReFOCS it fails to reconstruct the in-distribution class-wise exemplars provided in the support.}
 % \vskip -0.15in
\label{fig:exemp_recon_mini}
\end{figure}