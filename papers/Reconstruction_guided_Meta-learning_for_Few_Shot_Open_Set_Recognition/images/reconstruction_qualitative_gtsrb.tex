\begin{figure}[!ht]
\centering
\captionsetup[subfigure]{justification=centering}
\subfloat[ ]{
	\label{subfig:exemplars}
	\includegraphics[width=0.47\textwidth]{images/gtsrb_recon/templates_cropped.pdf} } 
\hfill
\subfloat[ ]{
	\label{subfig:in_query}
	\includegraphics[width=0.47\textwidth]{images/gtsrb_recon/inq_cropped.pdf} } 
\hfill
\subfloat[ ]{
	\label{subfig:out_query}
	\includegraphics[width=0.47\textwidth]{images/gtsrb_recon/outq_cropped.pdf} } 
\hfill
\subfloat[ ]{
	\label{subfig:inq_recon}
	\includegraphics[width=0.47\textwidth]{images/gtsrb_recon/inq_recon_cropped.pdf} } 
	\hfill
\subfloat[ ]{
	\label{subfig:outq_recon}
	\includegraphics[width=0.47\textwidth]{images/gtsrb_recon/outq_recon_cropped.pdf} } 
	
\caption{\textbf{Sample exemplar reconstructions for GTSRB$\rightarrow$GTSRB.} Exemplar reconstructions of a few query samples from $1$ test episode. (a) Class-wise exemplars provided in the support set. (b) In-distribution query samples. (c) Out-of-distribution query samples. (d) Reconstructed exemplars from the in-distribution queries. (e) Reconstructed Exemplars from the out-of-distribution queries, as hypothesized when out-of-distribution samples are fed into ReFOCS it fails to reconstruct the in-distribution class-wise exemplars provided in the support.}
 % \vskip -0.15in
\label{fig:exemp_recon}
\end{figure}