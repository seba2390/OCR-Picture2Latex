A single multi-bounce ray path can be represented as a polygonal chain where TX, the interaction points (e.g. reflection points or diffraction points) and RX are defined as the vertexes of the chain (see Fig. \ref{ray_path}). The evolution of this chain in a dynamic environment can be described only if we know how its vertexes will move in time. While the motion of the terminals is independent, the motion of the interaction points depends on the motion of the terminals and of objects generating these interaction points as well as on their instantaneous positions. Furthermore, the motion of the interaction points is not constant even if the terminals motion is constant. \par

\begin{figure}[h!]
	\centering
	\includegraphics[width=3in]{figures/multi_bounce_path.png}
	\caption{Representation of single multi-bounce ray path.}
	\label{ray_path}
\end{figure}

The computation of the dynamic evolution of paths can of course be accomplished by multiple RT runs on multiple environment descriptions corresponding to successive "snapshots" of the environment in time. Nonetheless, this technique requires the creation of a very large number of environment databases as well as an equally large number of RT runs making it impractical if an accurate description is required and moreover it will require a huge CPU time. \par
The proposed DRT algorithm is based on the combination of a classical 3D image-based RT approach with an analytical extrapolation of multipath evolution using a proper description of the dynamic environment \cite{bilibashi}, \cite{bilibashi2}. A dynamic environment database is introduced that describes the geometry of the environment also including the motion characteristics of both the radio terminals and the rigid bodies that can move, such as vehicles or machines. The database is divided into two parts. The first one provides a geometrical description of the environment for a given time instant $t_0$. Objects are described here - as in most RT models - as polyhedrons with flat surfaces and right edges, although an extension to curved surfaces is possible. In the second part, the dynamic parameters of every terminal and object are provided. In particular, the objects are modelled as "rigid bodies" with a roto-translational motion, that can be described in a complete way by providing translation velocity, rotation axis, angular velocity and the corresponding accelerations. \par
%
%  put here a figure with an example of a record of the dynamic environment database ?  Only if there is space, we already have many figures...
%
A traditional RT prediction is carried out for the initial time $t_0$, then the positions of all interaction points are extracted. From the initial positions, the motion of each vertex of the chain within $T_C$ must be computed. Our assumption is that the complete motion of the radio terminals is a-priori known. If this is not true, we assume at least that the the terminals' instant velocity and acceleration at $t_0$ is known, so that the positions $\overline{r}_P(t)$ of TX or RX on a time instant $t=t_{0}+\Delta t$ can be calculated using the Taylor series formula:
\begin{equation}
\begin{gathered}
 \overline{r}_P(t_{0}+\Delta t) = \overline{r}_P(t_{0}) + \overline{v}_{P}(t_0) \Delta t +  \\
 \frac{1}{2} \overline{a}_{P}(t_0) \Delta t^2 + O(\Delta t^3)
\end{gathered}
\label{Taylor}
\end{equation}
where $\overline{v}_{P}(t_0)$ and $\overline{a}_{P}(t_0)$ are the velocity and acceleration of P, respectively, at $t_0$, and the symbol $O(\Delta t^3)$ means that all the variations higher than $2^{nd}$ order, i.e. variations of accelerations, are neglected.

Now the problem is that the interaction points' positions, velocities and accelerations is unknown and needs to be determined for every time instant $t_{0}+\Delta t$ on the base of the current positions, velocities and accelerations of Tx, Rx and of the obstacles generating such interactions. For this purpose, we developed closed form formulas as explained in Section III.

The analytical extrapolation procedure is valid as long as the multipath structure remains the same, i.e., no path disappears and no new path appears to significantly change such structure. This time interval will be referred to in the following as \textit{multipath~lifetime}~$(T_C)$. As long as $\Delta t < T_C$ , DRT allows to extrapolate the multipath - and therefore the total field, the time-variant channel's transfer function, Doppler's shift frequencies, time delays, etc. - for every time instant $t_{0}+\Delta t$ without re-running the RT engine, and therefore at only a fraction of the computation time. In fact, the computationally most expensive part of a RT algorithm consists in determining the geometry of the rays, which requires checking the visibility and obstructions between all objects in the database, in order to establish the existence of each of the traced rays. With the DRT approach, this is done only once at time $t_0$, while in the subsequent time instants within $T_C$ we rely on analytical prolongation of the same rays, which is computationally much faster.

Also, we believe that it is a reasonable assumption to neglect the temporal variations of acceleration during the time $T_C$, in accordance with eq. (\ref{Taylor}): in fact, in vehicular scenarios the multipath structure usually varies on a faster time scale than acceleration, as discussed in Section IV.

Since DRT naturally accounts for speeds and accelerations, another advantage is that dynamic channel parameters, such as Doppler information, are derived in closed form, without resorting to finite-difference computation. A more detailed explanation about Doppler frequency calculation is described in Appendix A.