The presented algorithm is validated over two main case studies. In the first case the results are obtained in a simple vehicular environment. In the second case, a realistic scenario is considered, and results are compared with measurements carried out in an intersection in the city of Lund, Sweden \cite{abbas}. 
\subsection{Case Study 1: Ideal Street Canyon}

In this section the results are obtained using the DRT approach in a vehicular environment composed of a street canyon, a moving parallelepiped made of metal representing a bus, and two moving radio terminals. The propagation environment consists of an ideal street canyon 1km long and 30m large, with building walls on both sides and no intersections. Two reflecting walls are also present at its end sections. 
We first consider a vehicle-to-vehicle (V2V) communication scenario, where the two vehicles carrying TX and RX terminals drive in opposite lanes and the bus is moving on a middle lane between the two terminals as shown in Fig. \ref{bus_middle}. Therefore, the bus can generate a reflection from its front. We assume that TX is moving toward the end of the street at a constant speed of 50 km/h, RX and the bus are moving in the opposite direction at constant speeds of 36 km/h and 30 km/h, respectively. In the simulations, we considered reflections up to the second order and constant velocities for terminals and moving object. \par
To perform the simulation of the above-mentioned scenario, omnidirectional antennas were chosen with a transmit power of 1 W operating at a carrier frequency of 3 GHz. The antennas were placed on top of the TX and RX terminals at a height of 1.75m.  

\begin{figure}
    \captionsetup[subfigure]{font=scriptsize,labelfont=scriptsize}
    \centering
    \subfloat[\label{bus_middle}]{\includegraphics[width=1.4in]{figures/street_canyon_bus_middle.png}}
    \quad
    \subfloat[\label{bus_aside}]{\includegraphics[width=1.41in]{figures/street_canyon_bus_aside.png}}
    \caption{Ideal street canyon with a moving parallelepiped representing a bus and two moving terminals (TX and RX, not to scale).  The initial positions of the terminals and the bus are depicted with solid lines,  the final positions are shown with dashed lines. The bus is driving in the middle lane in (a) and in the side lane in (b).}
    \label{street_canyon}
\end{figure}

\subsubsection{Power Doppler Profile}
Figure \ref{PDfP} shows the evolution of Power-Doppler frequency profile (PDfP) obtained through DRT with a total simulation time of 5 seconds and discretized into "bins" for display purposes, with a time step of 200 ms and a Doppler frequency step of 14.34 Hz. The Doppler shift is computed for each ray using eq. (\ref{doppler_freq}) (see Appendix A). For each bin, an incoherent sum of the power contribution of each ray whose time of arrival and Doppler shift fall within the bin is performed. 

Since the scenario is simple and the obstructions or abrupt changes are minimal, we perform DRT simulation with only two $T_C$ with length 3 s and 2 s. The reason to use two $T_C$ is related to the contributions from the parallelepiped representing a bus: since the bus passes through the LoS line in the middle of the considered time span, the reflection from the front of the bus disappears and therefore we must consider the first $T_C$ expired and refresh the multipath structure with a new RT simulation.\par 
The other main contributions to PDfP (LoS, single and double reflections) are tagged in the plot and are present mainly throughout the whole simulation period. The most interesting trend in the contributions can be seen in the direct ray which from a positive Doppler shift in the first part of the simulation becomes negative with an inflection when TX and RX passes by each other. The same trend can be seen even for the reflections from the side walls but with lower values and less abrupt transitions. 

\begin{figure}[h!]
	\centering
	\includegraphics[width=3.4in]{figures/PDfP.png}
	\caption{PDfP evolution in a V2V scenario with a simulation time of 5 seconds.}
	\label{PDfP}
\end{figure} 

\subsubsection{Absolute value of error}
We have also compared the DRT analytical approach with the classical approach based on "snapshots", where the RT simulation is repeated at every discrete time instant. In Fig. \ref{absolute_error}, we show the absolute value of the error of the estimated power by DRT w.r.t RT in each of the time-Doppler bins of fig \ref{PDfP}. The absolute error is computed as:
\begin{equation*}
e_{ij}^{ABS}(dB)=|P_{ij}^{DRT}(dBm)-P_{ij}^{RT}(dBm)|
\end{equation*}
where $i,j$ are the indices associated to the time and Doppler bins, respectively.
Fig. \ref{absolute_error} shows that the error is virtually zero, as even in the worst case its value is several order of magnitudes lower than 1 dB. The non-zero values in Fig. \ref{absolute_error} are essentially caused by numerical inaccuracies in floating-point operations, and propagation of the errors when applying several successive reference systems transformations.
However, it is interesting to observe that the error becomes more evident in cases where there is a strong acceleration of the reflection point, e.g. when the reflecting wall is perpendicular to motion direction of the vehicles. This acceleration becomes more relevant when TX gets nearer to the wall, and also when the wall is part of a moving object. 

\begin{figure}[h!]
	\centering
	\includegraphics[width=3in]{figures/absolute_error_PDfP.png}
	\caption{Absolute value of error between standard multiple-run RT and DRT analytical approach}
	\label{absolute_error}
\end{figure}

\subsubsection{Bus moving on a side lane and then rotating}
We consider now the case of TX and RX moving in the same direction with constant speed at 28 km/h and 30 km/h, respectively, while the bus is moving in the opposite direction at 30 km/h. We consider a time instant when a reflection on the bus side wall is present, and the bus deviates from its straight path toward the center lane, by rotating counterclockwise with an instantaneous angular velocity $\omega=\pi/6$ [rad/s] (see Fig. \ref{bus_aside}). A comparison of the PDfP is shown in Fig. \ref{bus_rotation} for the case of the bus driving straight (blue) and the bus swerving toward the center lane (red), respectively.
In the first case, the LoS ray and the ray reflected from the bus side are overlapped in the figure with Doppler shift close to zero, as both are almost perpendicular to the TX's, RX's and reflection point's velocities.
In the latter case the Doppler's shift of bus reflection becomes positive (37 Hz) due to the reflection point's movement toward the central lane. \par 
This behaviour is important from the applications point of view: when Doppler frequency of a major multipath component abruptly changes, this could indicate a potentially dangerous situation, e.g. a vehicle swerving from its lane. 

\begin{figure}[h!]
	\centering
	\includegraphics[width=2.3in]{figures/bus_rotation.png}
	\caption{PDfP for a snapshot of the V2V scenario with reflection from bus side (see Fig. \ref{bus_aside}). Here TX and RX are moving in the same direction at 28 km/ and 30 km/h, respectively, while bus is moving at 30 km/h in the opposite direction. Bus moving straight without (blue) and with (red) rotation.}
	\label{bus_rotation}
\end{figure}

\subsection{Case Study 2 : Comparison with measurements in a street intersection}

\begin{figure}[h!]
    \centering
    \includegraphics[width=2.1in]{figures/lund_scenario2.png}
    \caption{Intersection scenario in the city of Lund, Sweden}
    \label{lund}
\end{figure}

In this section we perform a validation of DRT simulations results for a V2V scenario by means of a comparison with measured channel data \cite{abbas} and conventional, multiple-snapshot RT simulation using the same 3D RT algorithm used in our DRT model for initial simulation \cite{fuschini2015}. Both RT and DRT are performed with a maximum of 2 specular reflections, 1 diffraction and 1 diffuse scattering. \par
The measurements described in \cite{abbas} were performed using the RUSK LUND channel sounder at a carrier frequency of 5.9 GHz with a bandwidth of 240 MHz and TX power of 1 W. The antenna modules, identical for Tx and Rx, consist of 4 identical arrays mounted at 90° from each other and properly driven in order to simulate a quasi omnidirectional antenna. Two hatchback cars were used to carry TX and RX and the antenna modules were mounted on the roof-top \cite{paier2010}.  
Based on these information, we used  omni-directional half-wavelength dipole antennas for all RT and DRT simulations. 

\subsubsection{Power delay profile}
The scenario is a narrow urban intersection as shown in Fig. \ref{lund}. The measurements are divided into two parts: $M1)$ when TX and RX are driving from the streets $TX-M1$ and $RX-M1$, and $M2)$ when TX and RX are driving from $TX-M2$ and $RX-M2$, respectively, towards the intersection with a speed of approximately $10$ m/s. The RT simulations are performed using the classical approach based on snapshots repeated every $\Delta t=10$ms. The overall time-span is $t_S=10$s. Since the scenario evolution within $t_S$ includes a transition between NLoS and LoS then certainly $t_S>T_C$. Therefore DRT simulation is divided into several $T_C$, namely into 25 and 20\,$T_C$s for $M1$ and $M2$, respectively, with larger $T_C$s at the beginning of the routes and shorter ones close to the NLoS/LoS transition. Such a subdivision was optimized after a short trial-and-error procedure in order for DRT results to become very similar to RT results. During each $T_C$ the acceleration of the terminals is considered constant. 
In the first seconds, TX and RX are far from the street intersection and LOS between them is blocked by the buildings situated at the corners for both scenarios. In this period of time, we used only 2\,$T_C$. In the following time span which stretches till the appearance of LOS components (C) and (F) at approximately $6.8$s and $7.9$s in $M1$ and $M2$, respectively, new contributions appear that make the multipath structure change rapidly. At this stage the number of $T_C$ need to be increased and their duration reduced. In the last $0.4$s leading to the LOS, 4\,$T_C$ with a duration of $100$ms, are used in both scenarios. By doing so, we were able to capture most of the contributions in the time span. A further discussion on the criteria for the choice of $T_C$ is provided in the next sub-section. \par 

\begin{figure*} [t]
    \captionsetup[subfigure]{font=scriptsize,labelfont=scriptsize}
    \centering
    \subfloat[Measured \cite{abbas} \label{measure_M1}]{\includegraphics[width=2.85in]{figures/PDP_M1_measurements.png}}
    \quad \quad \quad \quad
    \subfloat[Measured \cite{abbas}\label{measure_M2}]{\includegraphics[width=2.8in]{figures/PDP_M2_measurements.PNG}}
    \\
    \subfloat[RT simulation\label{RT_M1}]{\includegraphics[width=2.8in]{figures/PDP_M1_RT.png}}
    \quad \quad \quad \quad
    \subfloat[RT simulation\label{RT_M2}]{\includegraphics[width=2.8in]{figures/PDP_M2_RT.png}}
    \\
    \subfloat[DRT simulation\label{DRT_M1}]{\includegraphics[width=2.85in]{figures/PDP_M1_DRT.png}}
    \quad \quad \quad \quad
    \subfloat[DRT simulation\label{DRT_M2}]{\includegraphics[width=2.8in]{figures/PDP_M2_DRT.png}}
    \caption{Power Delay Profile obtained from measurements (a) and (b), from RT simulations (c) and (d) and from DRT prediction (e) and (f)}
    \label{PDP}
\end{figure*}

The measured, RT- and DRT-simulated power delay profiles (PDP) are depicted in Fig. \ref{measure_M1}, \ref{RT_M1}, \ref{DRT_M1} for $M1$ and Fig. \ref{measure_M2}, \ref{RT_M2}, \ref{DRT_M2} for $M2$. In general, there is good agreement between simulations and measurements with several similar contributions in the RT/DRT simulation and in the measurements. The group of arrows (D), (G) and (B) in Fig. \ref{PDP} point at contributions that probably originated from nearby buildings. There is also good agreement between our simulations and RT simulations performed in \cite{abbas} although our simulations appears to capture some more contributions in the final 2 seconds of $t_S$, probably because we considered a larger variety of interactions, including diffuse scattering from building walls and cars parked along the streets. However, there are several contributions in the measured PDP that are missing in the simulations. One of the reasons could be the incomplete building database used in RT/DRT. For example some of the contributions like (E) and (I), in \cite{abbas} are said to be from a building which has a metallic structure on the walls, which is not present in the building database.  This shows the necessity of a very detailed scenario description in order to get a really good match between simulations and measurements. \par

\subsubsection{Impact of the choice of $T_C$}
The number of $T_C$ and the right choice of them is very important in  DRT simulation, that has a strong impact on both accuracy of results and computation time. In Fig. \ref{3Tc}, DRT simulation PDP for $M1$ is presented with only 3\,$T_C$. It is evident that results are incomplete if compared to Fig. \ref{DRT_M1} because dynamic changes in the multipath structure are under-sampled in this case and several paths aree missed. In Fig. \ref{10Tc}, 10\,$T_C$ has been used. The PDP in this case is more complete and many new contributions can be seen. Nevertheless a larger number of $T_C$ close to the NLoS/LoS transition is needed in order to achieve the more complete results of Fig. \ref{DRT_M1}.  

\begin{figure*}[t]
    
    \captionsetup[subfigure]{font=scriptsize,labelfont=scriptsize}
    \centering
    \subfloat[\label{3Tc}]{\includegraphics[width=2.8in]{figures/PDP_3Tc.png}}
    \quad \quad \quad \quad
    \subfloat[\label{10Tc}]{\includegraphics[width=2.9in]{figures/PDP_10Tc.png}}
    \caption{PDP obtained from simulation with DRT. In (a) 3\,$T_C$ were used, in (b) 10\,$T_C$}
\end{figure*}


\subsection{Computational Gain vs. traditional RT}
To compare the computation time of DRT vs. RT, we recorded computation times with the two approaches in $M1$ and $M2$ on the same Intel(R) Core(TM) i5-8265U CPU @1.60GHz 1.80 GHz platform. In DRT, one traditional RT run is included for multipath initialisation at the beginning of each $T_C$. To achieve the desired graphical resolution 1000 runs are performed with both DRT and RT, but a large fraction of them only requires the computation of analytical extrapolation formulas when using DRT, with a great computation time gain. \par
From Table \ref{time_gain}, we can observe a CPU-time  gain of about 48 times for the  $M2$ scenario while the computational gain is even higher in the $M2$ scenario because the number of $T_C$ is smaller.

\begin{table} [h!]
    \caption{Execution time for RT and DRT in M1 and M2}
    \label{time_gain}
    \centering
    \begin{tabular}{c|c|c|c}
         Scenario&\makecell{Standard RT\\(1000 snapshots)} &\makecell{Analytical approach\\(DRT)} &\makecell{Speed-up\\Factor} \\
         \hline
      M1 &4486 seconds &97.9 seconds  & 45.8x \\
      \hline
      M2 &4384 seconds &88.8 seconds  & 48.2x\\
      \hline
    \end{tabular}
\end{table}

\subsection{Future prospect: Predictive Radio Awareness}
The  DRT approach to predict “ahead-of-time” (or anticipate) the environment and/or the radio channel characteristics in highly dynamic, industrial or vehicular applications and realize the so-called \emph{predictive radio awareness} or \emph{location aware communications} \cite{Kuerner2018,DiTaranto2014} is quite attractive. Exploiting such capabilities could be of paramount importance to guarantee reliable connectivity in critical application such as automated and connected driving and to foster interesting safety applications to detect dangerous situations in advance. Two kinds of applications are possible: 
\begin{enumerate}
    \item DRT-based radio channel anticipative prediction
    \item Environment configuration anticipative prediction
\end{enumerate}
In both cases, accurate localization of radio terminals and moving objects, which is likely to be available in future mobile radio systems \cite{Witrisal2016,Koivisto2017}, is a prerequisite. Accurate localization, together with the availability of a local environment database is used to build a dynamic environment database for the current time $T_0$.
In 1), DRT is used to extrapolate multipath characteristics, and therefore to estimate the Channel State Information for $t>T_0$. In 2) kinematics theory is used to extrapolate the environment configuration for $t>T_0$ and therefore detect possible hazards or collisions.  These applications will be fully addressed in follow-on studies.
It is worth noting that, in turn, the availability of techniques 1) and 2)  would be of great help to realize multipath-exploiting localization techniques  \cite{Thomae2018}. Therefore, the two goals of anticipative channel prediction and localization might be achieved in synergy to realize an environment-aware system and enhance both connectivity and safety, as depicted in Fig. \ref{scheme_prediction}.

\begin{figure}[h!]
    \centering
    \includegraphics[width=3.1in]{figures/EnvironmentAware.jpg}
    \caption{Scheme of an environment-aware system including both channel prediction and localization}
    \label{scheme_prediction}
\end{figure}

