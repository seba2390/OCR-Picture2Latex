Originally conceived in the seventies and eighties of the last century for optical propagation problems, Ray tracing (RT) algorithms have been developed and used in the last three decades to simulate multipath radio propagation in man-made environments where obstacles, such as buildings, vehicles and machines, can be represented as polyhedrons of a given material \cite{iskander,fuschini2019}. Ray tracing applications have encompassed radio coverage prediction, the simulation of the radio channel, including its multidimensional characteristics, such as time-dispersion and angle-dispersion, which are important for MIMO system design and performance evaluation.
In the last years, as carrier frequencies are extending towards mm-wave and sub-THz frequencies in search for free bandwidth, where ray optics approximations become more valid, RT has become increasingly popular and its performance has improved considerably.

At the same time, public transportation and vehicular applications of wireless systems aimed at improving safety, connectivity, and enabling Connected, Cooperative, Autonomous Mobility (CCAM) technologies are becoming more and more important \cite{Bhat2018}. Due to the high mobility, vehicular channels are characterized by severe time-variability, especially at mm-wave frequencies where Line of Sight (LoS) to Non- Line of Sight (NLoS) transitions become more abrupt due to vanishing diffraction and transmission contributions and Doppler shifts become larger due to the higher carrier frequency \cite{boban1}. Real-time estimation of the channel state, which is crucial to achieve good performance, especially in MIMO and beamforming applications, becomes difficult in dynamic environment. Therefore, new solutions based on the so called "location awareness" are being proposed, including the use of artificial intelligence techniques \cite{Xing2020} or the real-time use of RT within the radio network \cite{iskander,VDE2021}. 

Unfortunately, the application of traditional RT models to vehicular environments, or dynamic environments in general, requires the solution of complex problems. Due to the high variability, realistic, spatially-consistent simulation of propagation in such cases requires to consider a high number of successive environment configurations (or “snapshots”): for each one of them, a new RT simulation needs to be carried out from scratch, which can result in an overwhelming computation time, not to mention the problems related to handling a huge number of environment description files and output files. The availability of fast ray tracing algorithms is therefore critical to future wireless applications in dynamic environments. 

Methods to predict the evolution of radio visibility prior to the actual ray tracing run, based on geometric techniques \cite{he2019,hussain2019}, methods to interpolate the evolution of the channel’s coefficients based on the speed of the moving transmitter (TX) and receiver (RX) \cite{nuckelt2015} as well as fast techniques to estimate Doppler profiles \cite{azpilicueta} have been proposed.   
Only recently, a new paradigm where RT is redesigned specifically for dynamic environments, such as vehicular and smart-factory environments, has been proposed using the definition Dynamic Ray Tracing (DRT) \cite{bilibashi,bilibashi2,qua}. The DRT concept implies the use of a dynamic environment database, where all the moving objects, including the radio terminals, vehicles and machine’s moving parts, are described in terms of their position, geometric shape, and their speed and acceleration at a given reference time instant $t_0$. Based on such a description and on a traditional RT run for the environment configuration at $t_0$, DRT predicts the dynamic evolution of the multipath geometry using an analytic extrapolation technique for any successive instant in time $t_{0}+\Delta t$, as long as $\Delta t < T_C$, where $T_C$ is the multipath coherence time (or lifetime), i.e. the time during which no major path should either appear or disappear.  Such a technique allows analytical field prediction for any number of snapshots within $T_C$ using a single RT run, with a reduction of the computation time of several orders of magnitude. Of course, the estimation of $T_C$ is a key point that deserves investigation: a discussion on $T_C$ can be found in the results section of this paper. 

Differently to previous work, a complete description of moving objects as solid bodies is considered in DRT, which allows the correct sliding of reflection and diffraction points on the bodys' surfaces to be taken into account. Therefore a high degree of accuracy can be achieved even when obstacles and radio terminals are relatively close to each other, which is common in both vehicular and industrial environments. Note that, if DRT is used in real-time and instantaneous speeds and accelerations for all environment’s moving parts are available for $t=t_0$, then even anticipative prediction for $t>t_0$ is possible within $T_C$: this is a very attractive possibility to enable a timely selection of the proper channel coding and/or beam for transmission and realize truly dependable wireless communications in dynamic environments.  When prediction for $\Delta t > T_C$ is needed, a conventional RT run must be carried out again based on the new geometric environment configuration that might be predicted using kinematics theory \cite{Martin1968,taylor2005}.

In the present work, the DRT concept is described and developed into a full-3D approach that includes multiple-bounce reflection, diffraction and diffuse scattering. With respect to the work presented in \cite{qua} the method is fully extended to edge diffraction and the rotation of solid bodies is also taken into account to achieve a general approach for dynamic environments. The DRT concept is described in more detail in section II, the algorithm formulation is presented in section III and applied to both ideal cases and real-world cases where it is checked against literature results including measured power-delay profiles. Using comparison with measurements and different choices of the maximum extrapolation time, the multipath coherence time $T_C$ is estimated and the possibility of anticipative channel prediction is presented. 
