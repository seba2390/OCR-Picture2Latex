In this Section, the DRT algorithm formulation is described for single and multiple bounce rays, including specular reflection and edge diffraction, based on the outcome of a single RT simulation at the initial time $t_0$, and on the knowledge of the dynamic parameters of TX/RX, and of the objects in the propagation environment. \par 


\subsection{Reflection Points' Calculation}
In this subsection, the DRT algorithm is initially explained for the single reflection case. Starting from this, further extensions of DRT to double reflection and multiple reflections, are presented in the following subsection. 

\subsubsection{Single Reflection Procedure}
A simple case is considered with a reflecting wall laying on a plane $\Pi^I$, and transmitter (TX) and receiver (RX) located at different distances from $\Pi^I$. As a start, we consider the case where TX and RX move with instantaneous speeds $\overline{v}_{TX}$ and $\overline{v}_{RX}$, while the reflecting wall is at rest. Without loss of generality, we can assume a proper reference system so that the wall lies on the plane of equation $y=0$, while TX and RX lie on the xy plane. Then, by using the image method and through simple geometric considerations we can derive the reflection point position ($Q_R$) at the considered time instant, which is located at the intersection between the reflecting plane and the line passing through RX and the image of TX:

\begin{equation}
\begin{gathered}
\begin{cases}
x_{Q_{R}}(t) = x_{TX}(t) + \frac{x_{RX}(t)-x_{TX}(t)}{y_{RX}(t)+y_{TX}(t)}~y_{TX}(t) \\
y_{Q_{R}}(t) = 0 \\
z_{Q_{R}}(t) = z_{TX}(t) + \frac{z_{RX}(t)-z_{TX}(t)}{y_{RX}(t)+y_{TX}(t)}~y_{TX}(t)
\end{cases}
\end{gathered}
\label{ref_point}
\end{equation}
When TX/RX move with a certain speed, the corresponding reflection point "slides" along the plane surface: therefore, by deriving $Q_R$ coordinates with respect to time, the instantaneous velocity of the reflection point ($\overline{v}_{Q_{R}}=v_{Q_{R,x}}\hat{x}+v_{Q_{R,z}}\hat{z}$) can be determined, by using the derivative chain rule. 

For example, the x-component of $\overline{v}_{Q_{R}}$ is:
\begin{equation}
\begin{gathered}
v_{Q_{R,x}} = \frac{\partial x_{Q_R}}{\partial t} = \frac{\partial x_{Q_R}}{\partial x_{TX}}~v_{TX,x} + \frac{\partial x_{Q_R}}{\partial y_{TX}}~v_{TX,y}  \\ + \frac{\partial x_{Q_R}}{\partial x_{RX}}~v_{RX,x} + \frac{\partial x_{Q_R}}{\partial y_{RX}}~v_{RX,y}.
\end{gathered}
\label{ref_point_velo}
\end{equation}
The closed-form expressions of the derivatives in eq. (\ref{ref_point_velo}) are reported in Appendix B.


This approach is similar to the one adopted in \cite{qua}, but in the following it will be extended to a more general case, where the reflecting wall has a roto-translational motion, and TX, RX, and the reflecting wall can vary their velocities during time, so their motion is accelerated.
The basic idea is to use a local reference system integral with the wall (local frame) so that we can resort to the previous case where the wall is at rest and apply equations (\ref{ref_point}) and (\ref{ref_point_velo}) again, as shown below. \par

%When TX/RX and the reflecting plane move with a certain speed, the corresponding reflection point "slides" along the plane surface:% 
Moreover, in the present work we make a complete characterization of the reflection point motion, including acceleration: in fact, except very particular cases, $Q_R$ velocity is not constant in time, i.e. the reflection point has an accelerated motion. Therefore, with a similar method as in (\ref{ref_point_velo}), the acceleration of the reflection point ($\overline{a}_{Q_{R}}=a_{Q_{R,x}}\hat{x}+a_{Q_{R,z}}\hat{z}$) can be calculated by deriving $\overline{v}_{Q_{R}}$ with respect to time. For instance, the x-component of the acceleration will be:  
\begin{equation}
a_{Q_{R,x}} = \frac{\partial v_{Q_{R,x}}}{\partial t} = \frac{\partial}{\partial t} \frac{\partial x_{Q_{R}}}{\partial t} 
\label{ref_point_acc}
\end{equation}
The complete expression of $\overline{a}_{Q_{R}}$, and the detailed computation of $\overline{v}_{Q_{R}}$ and $\overline{a}_{Q_{R}}$ are presented in Appendix B. \par 

In Fig. \ref{scenario_before_after}, an example of a single reflection scenario is presented. For the sake of simplicity, we refer to a bi-dimensional case where TX/RX are located in a horizontal plane $z=z_0$ and moving with arbitrary speeds, meanwhile the reflecting wall is located along a vertical plane $y=y_0$, and is then represented with a straight line. Moreover, the wall rotates around a vertical axis, and the intersection of the rotation axis with the plane $z=z_0$ provides the rotation center $O^I$. The local reference system $O^Ix^Iy^Iz^I$ centered in $O^I$ (local frame), is also represented in the figure. The axes of the local frame are oriented so that at each time instant the reflecting wall is laying on the $z^Ix^I$ plane of equation $y=0$, in order to apply eq. (\ref{ref_point}).
The scenario represented in the figure is simplified for ease of readability, but the presented method is general, so the TX/RX positions and velocities, the wall plane and the rotation axis can be arbitrarily oriented with respect to the global reference system $Oxyz$.

\begin{figure}[!ht]
	\centering
	\includegraphics[width=3.5in]{scenario_t0_and_t0+delta}
	\caption{Example of a single reflection scenario with the plane $\Pi^I$ that rotates w.r.t. to the global frame. a) Scenario at $t=t_0$ and instantaneous velocities of TX, image-TX, RX, and reflection point $Q_R$. b) Scenario at $t=t_0+\Delta t$.}
	\label{scenario_before_after}
\end{figure}

Fig. \ref{scenario_before_after}a) shows the configuration of the system at a certain time $t=t_0$, when the wall plane $\Pi^{I}$ is parallel to the plane $xz$ of the global frame $Oxyz$. The instantaneous velocities of TX ($\overline{v}_{TX}$), image-TX ($\overline{v}_{TX'}$), reflection point ($\overline{v}_{Q_R}$) and RX ($\overline{v}_{RX}$) at $t=t_0$ are depicted as well. The instantaneous angular velocity of the wall plane is expressed by the vector $\overline{\omega}=\omega \hat{k}$, where $\omega=\frac{d \alpha}{dt}$ is the scalar angular velocity, and $\hat{k}$ is a unit vector parallel to the rotation axis, and properly oriented according to the right-hand rule. In the example of Fig. \ref{scenario_before_after}, we have $\hat{k}=-\hat{z}^I$, as the wall is rotating clockwise around the z-axis of the local frame.

Fig. \ref{scenario_before_after}b) shows the configuration of the system in a subsequent time instant $t=t_0+\Delta t$, when the wall plane has rotated clockwise by an angle $\theta$, and TX/RX have moved to different positions: the result of this motion is a shift of the reflection point $Q_R$ along the reflecting wall.  

In Fig. \ref{scenario_before_after} a wall rotating around a fixed rotation axis is shown, but in general, a translational motion of the wall plane can be also present, in addition to the rotational motion.
In the general case of roto-translational motion, any point Q of the wall will have a different speed, given by \cite{Martin1968}:
\begin{equation}
\overline{v}_Q=\overline{v}_{\Pi^I}+\overline{\omega} \times \overline{O^IQ}
\label{RigidBodyMotion}
\end{equation}
where the symbol "$\times$" stands for the cross vector product, $\overline{v}_{\Pi^I}$ is the translation velocity, common to all the points of the plane, and $\overline{O^IQ}$ is the position vector of the considered point Q w.r.t. the origin $O^I$ of the local frame, positioned on the (instantaneous) rotation axis.
\par

The first step of the DRT procedure consists in the computation of the (instantaneous) position of the reflecting point $Q_R$. Thanks to the adoption of the local frame, this can be accomplished through eq. (\ref{ref_point}), but in order to do that, we need to transform the coordinates of TX and RX into the local frame associated with the wall. To this end, we observe that the positions of TX/RX with respect to the global and local frames are related each other through the following \textit{coordinate transformation} \cite{Martin1968}:
\begin{equation}
\begin{gathered}
    \overline{r}_{TX}^0(t) = \Bar{\Bar{R}}(t) \cdot \overline{r}_{TX}^{I}(t) + \overline{r}_{0^I}(t) \\
    \overline{r}_{RX}^0(t) = \Bar{\Bar{R}}(t) \cdot \overline{r}_{RX}^{I}(t) + \overline{r}_{0^I}(t)
\end{gathered}
\label{coordinate_transf}
\end{equation}
with
\begin{center}
\begin{small}
$\overline{r}_{TX}^0(t)=[x_{TX}(t)~y_{TX}(t)~z_{TX}(t)]^T$ \\
$\overline{r}_{RX}^0(t)=[x_{RX}(t)~y_{RX}(t)~z_{RX}(t)]^T$
\end{small}
\end{center} 
and 
\begin{center}
\begin{small}
$\overline{r}_{TX}^I(t)=[x_{TX}^I(t)~y_{TX}^I(t)~z_{TX}^I(t)]^T$ \\
$\overline{r}_{RX}^I(t)=[x_{RX}^I(t)~y_{RX}^I(t)~z_{RX}^I(t)]^T$ 
\end{small} 
\end{center}
being the position vectors of TX/RX w.r.t. the global and local frame, respectively, where $\Bar{\Bar{R}}(t)$ is the (instantaneous) rotation matrix and $\overline{r}_{0^I}(t)=[x_{O^I}(t)~y_{O^I}(t)~z_{O^I}(t)]^T$ is the position vector associated with the origin point $O^{I}$ of the local frame. By inverting eq. (\ref{coordinate_transf}), we can determine the positions $\overline{r}_{TX}^I$, $\overline{r}_{RX}^I$ of TX and RX w.r.t. to the local frame, and then apply eq. (\ref{ref_point}) to find the coordinates of $Q_R$.

In the simple example of Fig. \ref{scenario_before_after} (clockwise rotation around the z-axis), the rotation matrix at the time instant $t=t_0+\Delta t$ is given by:
\begin{equation*}
\Bar{\Bar{R}}(t_0+\Delta t)=
\begin{pmatrix}
cos\theta & sin\theta & 0\\
-sin\theta & cos\theta & 0\\
0 & 0 & 1
\end{pmatrix}
\end{equation*}
In a generic time instant $t=t_0+\Delta t$ within the multipath lifetime $T_C$, the instantaneous rotation angle $\theta(t)$ can be obtained in the following way, similarly to eq.(\ref{Taylor}) and neglecting time variations of angular acceleration:
\begin{equation*}
\theta(t)=\theta(t_0)+\omega\Delta t+\frac{1}{2}\frac{d\omega}{dt}\Delta t^2
\end{equation*}
Of course, \textit{coordinate transformation} must be applied to the components of $\overline{v}_{TX}$, $\overline{v}_{RX}$ and $\overline{a}_{TX}$, $\overline{a}_{RX}$ as well.

The next step of the DRT procedure consists in the computation of the velocity and acceleration of $Q_R$ which can be obtained through eq. (\ref{ref_point_velo}) and (\ref{ref_point_acc}). In fact, one of the main advantages of the DRT approach is that we can derive $\overline{v}_{Q_R}$ and $\overline{a}_{Q_R}$ analytically, without need to resort to time differences methods. However, projecting the components of $\overline{v}_{TX}$, $\overline{v}_{RX}$ and $\overline{a}_{TX}$, $\overline{a}_{RX}$ on the local frame is not enough to apply eq. (\ref{ref_point_velo}) and (\ref{ref_point_acc}), which have been obtained from eq. (\ref{ref_point}), i.e. assuming that the reflecting wall is at rest.  Since now the reflecting wall is moving, and then the local frame is in motion w.r.t. the global reference system, the velocities and accelerations of TX/RX must be preliminary transformed according to the \textit{relative motion transformations}, to obtain "relative" velocities and accelerations w.r.t. an observer located in the origin of the local frame \cite{taylor2005}. For instance, the velocity and acceleration of TX must be transformed in the following way: 
\begin{gather}
\overline{v}_{TX}^{I} = \overline{v}_{TX}^{0} - \overline{v}_{\Pi^{I}} - \overline{\omega} \times \overline{r}_{TX}^{I} \label{relative_vel} \\
\overline{a}_{TX}^{I} = \overline{a}_{TX}^{0} - \overline{a}_{\Pi^{I}} - \dot{\overline{\omega}} \times \overline{r}_{TX}^{I} - 2 \overline{\omega} \times \overline{v}_{TX}^{I} \label{relative_acc} \\ \nonumber
-\overline{\omega} \times (\overline{\omega} \times \overline{r}_{TX}^{I})
\end{gather}
where the headers $"I"$ and $"0"$ are associated with the velocities and accelerations seen by an observer located in the origin of the local and global frame, respectively, $\overline{r}_{TX}^{I}=\overline{r}_{TX}^{0}-\overline{r}_{O^{I}}$ is the position vector of TX in the local frame, $\overline{v}_{\Pi^{I}}$, $\overline{\omega}$ are the translation and angular velocities of the wall plane $\Pi^{I}$, and $\overline{a}_{\Pi^{I}}$, $\dot{\overline{\omega}}=\frac{d\overline{\omega}}{dt}$ are the corresponding translation and angular accelerations, respectively.
The terms $\dot{\overline{\omega}} \times \overline{r}_{TX}^{I}$, $2 \overline{\omega} \times \overline{v}_{TX}^{I}$ and $\overline{\omega} \times (\overline{\omega} \times \overline{r}_{TX}^{I})$ in eq. (\ref{relative_acc}) are also known as Euler's, Coriolis', and centrifugal acceleration, respectively.

It is worth noting that the accelerations $\overline{a}_{\Pi^{I}}$, $\dot{\overline{\omega}}$ are assumed to be constant in the time interval $[t_0~t_0+T_C]$, while $\overline{v}_{\Pi^{I}}$, $\overline{\omega}$ in eq. (\ref{relative_vel}) are \textit{instantaneous} velocities, computed as:
\begin{equation*}
\begin{gathered}
\overline{v}_{\Pi^{I}}(t)=\overline{v}_{\Pi^{I}}(t_0)+\overline{a}_{\Pi^{I}}\Delta t \\
\overline{\omega}(t)=\overline{\omega}(t_0)+\dot{\overline{\omega}} \Delta t
\end{gathered}
\end{equation*}

In practice, with the \textit{relative motion transformations} (\ref{relative_vel}) and (\ref{relative_acc}) we turn the original problem into an equivalent problem, where the wall plane is at rest and the TX/RX velocities and accelerations are modified, according to the point of view of an observer located in the origin of the local frame. It is worth noting that, even in the simple case of TX/RX moving with constant speed, TX and RX are accelerated in the equivalent problem: this acceleration is caused by the angular rotation ${\overline{\omega}}$ of the wall plane, according to eq. (\ref{relative_vel}) and (\ref{relative_acc}).
\par

Once the reflection point position, velocity and acceleration have been determined in the local frame using eq. (\ref{ref_point}), (\ref{ref_point_velo}), and (\ref{ref_point_acc}), the following inverse transformations need to be applied: 
\begin{itemize}
	\item back-transformation to get $Q_R$ coordinates, and $\overline{v}_{Q_{R}}$ and $\overline{a}_{Q_{R}}$ components w.r.t. the global reference system (inverse \textit{coordinate  transformation}).
	\item back-transformation to get the velocity and acceleration ($\overline{v}_{Q_{R}}$ and $\overline{a}_{Q_{R}}$) relative to an observer located in the origin of the global reference system (inverse \textit{relative motion transformation}):
	\begin{gather}
	    \overline{v}_{Q_R}^{0} = \overline{v}_{Q_R}^{I} + \overline{v}_{\Pi^{I}} + \overline{\omega} \times \overline{r}_{Q_R}^{I} \label{global_vel_ref_point} \\
        \overline{a}_{Q_R}^{0} = \overline{a}_{Q_R}^{I} + \overline{a}_{\Pi^{I}} + \dot{\overline{\omega}} \times \overline{r}_{Q_R}^{I} + 2 \overline{\omega} \times \overline{v}_{Q_R}^{I} \label{global_acc_ref_point} \\ \nonumber
        +\overline{\omega} \times (\overline{\omega} \times \overline{r}_{Q_R}^{I})   
	\end{gather}
\end{itemize}  


\subsubsection{Generalization to Multiple Reflections}
In the case of multiple reflections, the motion of a certain reflection point is influenced by the motion of the previous or latter reflection points.  
However, the method discussed above can be extended in a straightforward way to a multiple-bounce case. 

Let's consider for simplicity a double reflection case: instead of using TX, we can resort to the use of the image-TX ($TX^{'}$) with respect to the first wall plane, and after that we can analyze the reflection on the second wall: in practice, we replace $TX$ with $TX^{'}$ and we bring the computation back to the single-reflection scenario analyzed in the previous section. This means that in a case with two reflecting walls, $TX^{'}$ and $RX$ are used to compute the motion of the reflection point along the second reflecting wall ($Q_{R2}$), by applying the same procedure as for the single-bounce scenario. Then, once $Q_{R2}$ is known, it is used as a new virtual receiver to compute, in addition to the TX location, the position and the motion of the reflection point along the first reflecting wall ($Q_{R1}$). 

This approach can be iterated in a similar way for the case of more than 2 reflections. In practice, we compute the image-TX ($TX^{'}$), the image of the image ($TX^{''}$), etc., until we reach the last reflecting wall, then a "back-tracking" procedure is used: we start from RX and the last reflecting wall, we apply all the equations of the previous section to compute the last reflection point, and then we move back towards TX, to trace the motion of all the remaining reflection points.  

In order to apply all the equations of the previous section for the determination of the reflection points motion, we need to apply all the necessary transformations from the global to the local frame, and vice-versa, for each of the reflecting walls.

Moreover, a preliminary computation is needed to compute the position and the instantaneous velocity for each of the image-transmitters. This can be done in a straightforward way by relying on the local frame, and using the image principle. For example, the image-TX with respect to the first wall ($TX^{'}$), will have the same $x^I$ and $z^I$ coordinates as TX, and opposite $y^I$ coordinate. Similarly, the velocity components will be:

\begin{equation}
\begin{gathered}
\begin{cases}
v_{TX^{'},x}^{I} = v_{TX,x}^{I} \\
v_{TX^{'},y}^{I} = -v_{TX,y}^{I} \\
v_{TX^{'},z}^{I} = v_{TX,z}^{I}
\end{cases}
\end{gathered} 
\end{equation}

Once $\overline{v}_{TX^{'}}$ is computed in the local frame, it can be expressed w.r.t. to an observer located in the origin of the global frame, according to the relative motion transformation:
\begin{equation}
\overline{v}_{TX^{'}}^{0}=\overline{v}_{TX^{'}}^{I}+\overline{\omega} \times \overline{r}_{TX^{'}}^{I}
\end{equation}
In a similar way, the acceleration of the image-TX, $\overline{a}_{TX^{'}}$, can be also found.
The DRT algorithm then proceeds according to steps described above, to find the coordinates, velocities and accelerations for each of the reflection points.
\par

The whole DRT algorithm is summarized by the flowchart depicted in Fig. \ref{flowchart}, with reference to a double-bounce case, for the sake of simplicity. 

Once the geometric part is done, i.e. the analytical prolongation of all the rays in the considered time instants within $T_C$ has been completed, the very last step of DRT consists in the re-computation of the field associated to each ray. This is done in a straightforward way as the geometry of the rays is known, by applying the Fresnel's reflection coefficients and the ray divergence factor, as usually done in standard RT algorithms \cite{fuschini2015,vitucci2019}. It is worth noting however, that ray's field computation is based on analytical formulas and is therefore orders of magnitude faster than ray's geometry computation \cite{fuschini2015}.

\begin{figure}[h!]
	\centering
	\includegraphics[width=2.5in]{flowchart_2RFL}
	\caption{Flowchart showing the DRT algorithm for a double reflection case.}
	\label{flowchart}
\end{figure}

\subsection{Diffraction Points' Calculation}
The DRT algorithm can be further extended to diffraction, modeled with a ray-based approach according the Uniform Theory of Diffraction (UTD) \cite{UTD}.
A method to track the motion of diffraction points in an analytical way, as previously done for the reflection points, is outlined in this section. We present only the single-diffraction case for the sake of brevity, but the procedure can be extended in a straightforward way to multiple diffractions, as well as to combinations of multiple reflections and diffractions.

In Fig. \ref{diffraction} diffraction from an edge formed by two adjacent walls is illustrated, where the unit edge vector $\hat{e}$ is chosen to be aligned with the $z$-axis of the reference system $Oxyz$, with no loss of generality. For the sake of simplicity and with no limitation - as the diffracted rays lay on the Keller's cone and share the same geometric properties \cite{Keller} - we represent in Fig. \ref{diffraction} an "unfolded" diffracted ray, i.e. the diffraction plane has been rotated to be coincident with the incidence plane. 

\begin{figure}[!ht]
	\centering
	\includegraphics[width=3.2in]{diffraction_scenario}
	\caption{Example of edge diffraction, and related geometry for the computation of the diffraction point.}
	\label{diffraction}
\end{figure}

Since the diffraction point ($Q_D$) is constrained to move along the edge, the $x$ and $y$ coordinates of $Q_D$ are known, then only $z_{Q_D}$ remains to be computed: this can be done in a simple way by using the similar triangles properties.  \par
Looking at Fig. \ref{diffraction}, we see that two similar triangles are formed. The sides of these triangles are proportional each other, so the following relation holds: 
\begin{equation}
z_{TX}-z_{RX} : d_{TX}+d_{RX} = z_{Q_D}-z_{RX} : d_{RX}
\end{equation} 
where 
\begin{equation}
\begin{gathered}
d_{TX}(t) = \sqrt{(x_{TX}(t)-x_{Q_D})^2+(y_{TX}(t)-y_{Q_D})^2} \\[1ex]
d_{RX}(t) = \sqrt{(x_{RX}(t)-x_{Q_D})^2+(y_{RX}(t)-y_{Q_D})^2}
\end{gathered}
\label{dist2D_TXRX}
\end{equation}
are the 2D distances of TX and RX from the edge, respectively. Hence, the z-coordinate of $Q_D$ is given by: 
\begin{equation}
z_{Q_D}(t) = z_{RX}(t) + \frac{d_{RX}(t)\cdot[z_{TX}(t)-z_{RX}(t)]}{d_{TX}(t)+d_{RX}(t)}
\label{diff_point}
\end{equation}
The instantaneous velocity of $Q_D$ ($\overline{v}_{Q_D}=v_{Q_{D,z}}\hat{z}$) can be computed by time deriving $z_{Q_D}$. The detailed calculation of $v_{Q_{D,z}}$ is presented in Appendix C.\par
Similarly, by further deriving $\overline{v}_{Q_D}$, we can find the acceleration of the diffraction point, $\overline{a}_{Q_D}=a_{Q_{D,z}}\hat{z}$, not reported here for the sake of brevity.\par
The expression in (\ref{diff_point}) and the related velocity and acceleration $\overline{v}_{Q_D}$, $\overline{a}_{Q_D}$ are valid and do not require any further computation in case the terminals are moving but the edge is at rest. However, in general an edge might be part of a moving object and then might be moving with a certain roto-translational velocity. In particular, we assume that the edge has a rotational motion with instantaneous angular velocity $\overline{\omega}$, and is also translating according to the instantaneous velocity $\overline{v}_{edge}$. In the example of Fig. \ref{diffraction}, the edge is rotating clockwise around the x-axis. 

As for reflections, we can compute the instantaneous position and the motion of the diffraction point if we assume a proper local reference $O^Ix^Iy^Iz^I$, with the origin located in the rotation center $O^I$, and the z-axis parallel to the edge. By doing so, the same procedure used for reflections, can be followed to transform velocities and accelerations, and finally find $z_{Q_D}$ as well as $v_{Q_{D,z}}$ and $a_{Q_{D,z}}$.  \par 
The final step of DRT is, as usual, the computation of the updated UTD coefficients, and then, of the total diffracted field, at the considered time instant.

\subsection{Diffuse scattering}
Diffuse scattering is modeled according to the Effective Roughness approach \cite{vitucci2019}, which is based on a subdivision of each surface into tiles and on the application of a virtual scattering source to the centroid of each tile. Therefore, the  calculation of scattering points' position and speed is straightforward, as it boils down to the application of basic kinematic equations for the motion of rigid bodies' surface points. For instance, equation (\ref{RigidBodyMotion}) can be used to compute each scattering point's velocity if the rototranslation speed of the body is known.

