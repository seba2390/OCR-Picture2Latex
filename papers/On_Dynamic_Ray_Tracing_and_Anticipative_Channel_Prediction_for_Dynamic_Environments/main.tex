\documentclass[lettersize,journal]{IEEEtran}
\usepackage{amsmath,amsfonts}
\usepackage{mathtools}
\usepackage{algorithmic}
\usepackage{algorithm}
\usepackage{array}
\usepackage[caption=false,font=normalsize,labelfont=sf,textfont=sf]{subfig}
\usepackage{textcomp}
\usepackage{stfloats}
\usepackage{url}
\usepackage{verbatim}
\usepackage{graphicx}
\graphicspath{ {figures/} }
\usepackage{cite}
\usepackage{makecell}
\usepackage{resizegather}
%\usepackage{subfig}
%\usepackage{subcaption}
\hyphenation{op-tical net-works semi-conduc-tor IEEE-Xplore}
%\usepackage{appendix}
\usepackage{xcolor}

\begin{document}

\title{On Dynamic Ray Tracing and Anticipative Channel Prediction for Dynamic Environments}

\author{Denis Bilibashi, Enrico M. Vitucci, and Vittorio Degli-Esposti
        % <-this % stops a space

\thanks{This work was funded in part by the Italian Ministry of University and Research (MUR) through the programme ”Dipartimenti di Eccellenza (2018–2022) — Precision Cyberphysical Systems Project (P-CPS),” and in part by the Eu COST Action INTERACT (Intelligence-Enabling Radio Communications for Seamless Inclusive Interactions) under Grant CA20120.}
\thanks{Denis Bilibashi, Enrico~M.~Vitucci, and V.~Degli-Esposti are with the Department of Electrical, Electronic, and Information Engineering "Guglielmo Marconi" (DEI), CNIT, University of Bologna, 40126 Bologna, Italy (e-mail: denis.bilibashi2, enricomaria.vitucci, v.degliesposti @unibo.it)}
}% <-this % stops a space



%\IEEEpubid{0000--0000/00\$00.00~\copyright~2021 IEEE}
% Remember, if you use this you must call \IEEEpubidadjcol in the second
% column for its text to clear the IEEEpubid mark.

\maketitle

\begin{abstract}
Ray tracing algorithms, that can simulate multipath radio propagation in presence of geometric obstacles such as buildings, objects or vehicles, are becoming quite popular, due to the increasing availability of digital environment databases and  high-performance computation platforms, such as multi-core computers and cloud computing services. When objects or vehicles are moving, which is the case of industrial or vehicular environments, multiple successive representations of the environment ("snapshots") and multiple ray tracing runs are often necessary, which require a great human effort and a great deal of computation resources.
Recently, the Dynamic Ray Tracing (DRT) approach has been proposed to predict the multipath evolution within a given time lapse on the base of the current multipath geometry, assuming constant speeds and/or accelerations for moving objects, using  analytical extrapolation formulas.  This is done without re-running a full ray tracing for every "snapshot" of the environment, therefore with a great computation time saving. When DRT is embedded in a mobile radio system and used in real-time, ahead-of-time (or anticipative) field prediction is possible that opens the way to interesting applications. In the present work, a full-3D DRT algorithm is presented that allows to account for multiple reflections, edge diffraction and diffuse scattering for the general case where moving objects can translate and rotate. For the purpose of validation, the model is first applied to some ideal cases and then to realistic cases where results are compared with conventional ray tracing simulation and measurements available in the literature.
\end{abstract}

\begin{IEEEkeywords}
Ray Tracing, Dynamic ray tracing, Radiowave Propagation, Millimeter wave propagation, Vehicular Ad Hoc Networks, Doppler Effect, 6G Mobile Communications
\end{IEEEkeywords}

\section{Introduction}
% \leavevmode
% \\
% \\
% \\
% \\
% \\
\section{Introduction}
\label{introduction}

AutoML is the process by which machine learning models are built automatically for a new dataset. Given a dataset, AutoML systems perform a search over valid data transformations and learners, along with hyper-parameter optimization for each learner~\cite{VolcanoML}. Choosing the transformations and learners over which to search is our focus.
A significant number of systems mine from prior runs of pipelines over a set of datasets to choose transformers and learners that are effective with different types of datasets (e.g. \cite{NEURIPS2018_b59a51a3}, \cite{10.14778/3415478.3415542}, \cite{autosklearn}). Thus, they build a database by actually running different pipelines with a diverse set of datasets to estimate the accuracy of potential pipelines. Hence, they can be used to effectively reduce the search space. A new dataset, based on a set of features (meta-features) is then matched to this database to find the most plausible candidates for both learner selection and hyper-parameter tuning. This process of choosing starting points in the search space is called meta-learning for the cold start problem.  

Other meta-learning approaches include mining existing data science code and their associated datasets to learn from human expertise. The AL~\cite{al} system mined existing Kaggle notebooks using dynamic analysis, i.e., actually running the scripts, and showed that such a system has promise.  However, this meta-learning approach does not scale because it is onerous to execute a large number of pipeline scripts on datasets, preprocessing datasets is never trivial, and older scripts cease to run at all as software evolves. It is not surprising that AL therefore performed dynamic analysis on just nine datasets.

Our system, {\sysname}, provides a scalable meta-learning approach to leverage human expertise, using static analysis to mine pipelines from large repositories of scripts. Static analysis has the advantage of scaling to thousands or millions of scripts \cite{graph4code} easily, but lacks the performance data gathered by dynamic analysis. The {\sysname} meta-learning approach guides the learning process by a scalable dataset similarity search, based on dataset embeddings, to find the most similar datasets and the semantics of ML pipelines applied on them.  Many existing systems, such as Auto-Sklearn \cite{autosklearn} and AL \cite{al}, compute a set of meta-features for each dataset. We developed a deep neural network model to generate embeddings at the granularity of a dataset, e.g., a table or CSV file, to capture similarity at the level of an entire dataset rather than relying on a set of meta-features.
 
Because we use static analysis to capture the semantics of the meta-learning process, we have no mechanism to choose the \textbf{best} pipeline from many seen pipelines, unlike the dynamic execution case where one can rely on runtime to choose the best performing pipeline.  Observing that pipelines are basically workflow graphs, we use graph generator neural models to succinctly capture the statically-observed pipelines for a single dataset. In {\sysname}, we formulate learner selection as a graph generation problem to predict optimized pipelines based on pipelines seen in actual notebooks.

%. This formulation enables {\sysname} for effective pruning of the AutoML search space to predict optimized pipelines based on pipelines seen in actual notebooks.}
%We note that increasingly, state-of-the-art performance in AutoML systems is being generated by more complex pipelines such as Directed Acyclic Graphs (DAGs) \cite{piper} rather than the linear pipelines used in earlier systems.  
 
{\sysname} does learner and transformation selection, and hence is a component of an AutoML systems. To evaluate this component, we integrated it into two existing AutoML systems, FLAML \cite{flaml} and Auto-Sklearn \cite{autosklearn}.  
% We evaluate each system with and without {\sysname}.  
We chose FLAML because it does not yet have any meta-learning component for the cold start problem and instead allows user selection of learners and transformers. The authors of FLAML explicitly pointed to the fact that FLAML might benefit from a meta-learning component and pointed to it as a possibility for future work. For FLAML, if mining historical pipelines provides an advantage, we should improve its performance. We also picked Auto-Sklearn as it does have a learner selection component based on meta-features, as described earlier~\cite{autosklearn2}. For Auto-Sklearn, we should at least match performance if our static mining of pipelines can match their extensive database. For context, we also compared {\sysname} with the recent VolcanoML~\cite{VolcanoML}, which provides an efficient decomposition and execution strategy for the AutoML search space. In contrast, {\sysname} prunes the search space using our meta-learning model to perform hyperparameter optimization only for the most promising candidates. 

The contributions of this paper are the following:
\begin{itemize}
    \item Section ~\ref{sec:mining} defines a scalable meta-learning approach based on representation learning of mined ML pipeline semantics and datasets for over 100 datasets and ~11K Python scripts.  
    \newline
    \item Sections~\ref{sec:kgpipGen} formulates AutoML pipeline generation as a graph generation problem. {\sysname} predicts efficiently an optimized ML pipeline for an unseen dataset based on our meta-learning model.  To the best of our knowledge, {\sysname} is the first approach to formulate  AutoML pipeline generation in such a way.
    \newline
    \item Section~\ref{sec:eval} presents a comprehensive evaluation using a large collection of 121 datasets from major AutoML benchmarks and Kaggle. Our experimental results show that {\sysname} outperforms all existing AutoML systems and achieves state-of-the-art results on the majority of these datasets. {\sysname} significantly improves the performance of both FLAML and Auto-Sklearn in classification and regression tasks. We also outperformed AL in 75 out of 77 datasets and VolcanoML in 75  out of 121 datasets, including 44 datasets used only by VolcanoML~\cite{VolcanoML}.  On average, {\sysname} achieves scores that are statistically better than the means of all other systems. 
\end{itemize}


%This approach does not need to apply cleaning or transformation methods to handle different variances among datasets. Moreover, we do not need to deal with complex analysis, such as dynamic code analysis. Thus, our approach proved to be scalable, as discussed in Sections~\ref{sec:mining}.

\section{The DRT Concept}
A single multi-bounce ray path can be represented as a polygonal chain where TX, the interaction points (e.g. reflection points or diffraction points) and RX are defined as the vertexes of the chain (see Fig. \ref{ray_path}). The evolution of this chain in a dynamic environment can be described only if we know how its vertexes will move in time. While the motion of the terminals is independent, the motion of the interaction points depends on the motion of the terminals and of objects generating these interaction points as well as on their instantaneous positions. Furthermore, the motion of the interaction points is not constant even if the terminals motion is constant. \par

\begin{figure}[h!]
	\centering
	\includegraphics[width=3in]{figures/multi_bounce_path.png}
	\caption{Representation of single multi-bounce ray path.}
	\label{ray_path}
\end{figure}

The computation of the dynamic evolution of paths can of course be accomplished by multiple RT runs on multiple environment descriptions corresponding to successive "snapshots" of the environment in time. Nonetheless, this technique requires the creation of a very large number of environment databases as well as an equally large number of RT runs making it impractical if an accurate description is required and moreover it will require a huge CPU time. \par
The proposed DRT algorithm is based on the combination of a classical 3D image-based RT approach with an analytical extrapolation of multipath evolution using a proper description of the dynamic environment \cite{bilibashi}, \cite{bilibashi2}. A dynamic environment database is introduced that describes the geometry of the environment also including the motion characteristics of both the radio terminals and the rigid bodies that can move, such as vehicles or machines. The database is divided into two parts. The first one provides a geometrical description of the environment for a given time instant $t_0$. Objects are described here - as in most RT models - as polyhedrons with flat surfaces and right edges, although an extension to curved surfaces is possible. In the second part, the dynamic parameters of every terminal and object are provided. In particular, the objects are modelled as "rigid bodies" with a roto-translational motion, that can be described in a complete way by providing translation velocity, rotation axis, angular velocity and the corresponding accelerations. \par
%
%  put here a figure with an example of a record of the dynamic environment database ?  Only if there is space, we already have many figures...
%
A traditional RT prediction is carried out for the initial time $t_0$, then the positions of all interaction points are extracted. From the initial positions, the motion of each vertex of the chain within $T_C$ must be computed. Our assumption is that the complete motion of the radio terminals is a-priori known. If this is not true, we assume at least that the the terminals' instant velocity and acceleration at $t_0$ is known, so that the positions $\overline{r}_P(t)$ of TX or RX on a time instant $t=t_{0}+\Delta t$ can be calculated using the Taylor series formula:
\begin{equation}
\begin{gathered}
 \overline{r}_P(t_{0}+\Delta t) = \overline{r}_P(t_{0}) + \overline{v}_{P}(t_0) \Delta t +  \\
 \frac{1}{2} \overline{a}_{P}(t_0) \Delta t^2 + O(\Delta t^3)
\end{gathered}
\label{Taylor}
\end{equation}
where $\overline{v}_{P}(t_0)$ and $\overline{a}_{P}(t_0)$ are the velocity and acceleration of P, respectively, at $t_0$, and the symbol $O(\Delta t^3)$ means that all the variations higher than $2^{nd}$ order, i.e. variations of accelerations, are neglected.

Now the problem is that the interaction points' positions, velocities and accelerations is unknown and needs to be determined for every time instant $t_{0}+\Delta t$ on the base of the current positions, velocities and accelerations of Tx, Rx and of the obstacles generating such interactions. For this purpose, we developed closed form formulas as explained in Section III.

The analytical extrapolation procedure is valid as long as the multipath structure remains the same, i.e., no path disappears and no new path appears to significantly change such structure. This time interval will be referred to in the following as \textit{multipath~lifetime}~$(T_C)$. As long as $\Delta t < T_C$ , DRT allows to extrapolate the multipath - and therefore the total field, the time-variant channel's transfer function, Doppler's shift frequencies, time delays, etc. - for every time instant $t_{0}+\Delta t$ without re-running the RT engine, and therefore at only a fraction of the computation time. In fact, the computationally most expensive part of a RT algorithm consists in determining the geometry of the rays, which requires checking the visibility and obstructions between all objects in the database, in order to establish the existence of each of the traced rays. With the DRT approach, this is done only once at time $t_0$, while in the subsequent time instants within $T_C$ we rely on analytical prolongation of the same rays, which is computationally much faster.

Also, we believe that it is a reasonable assumption to neglect the temporal variations of acceleration during the time $T_C$, in accordance with eq. (\ref{Taylor}): in fact, in vehicular scenarios the multipath structure usually varies on a faster time scale than acceleration, as discussed in Section IV.

Since DRT naturally accounts for speeds and accelerations, another advantage is that dynamic channel parameters, such as Doppler information, are derived in closed form, without resorting to finite-difference computation. A more detailed explanation about Doppler frequency calculation is described in Appendix A.

\section{DRT Algorithm Description}
In this Section, the DRT algorithm formulation is described for single and multiple bounce rays, including specular reflection and edge diffraction, based on the outcome of a single RT simulation at the initial time $t_0$, and on the knowledge of the dynamic parameters of TX/RX, and of the objects in the propagation environment. \par 


\subsection{Reflection Points' Calculation}
In this subsection, the DRT algorithm is initially explained for the single reflection case. Starting from this, further extensions of DRT to double reflection and multiple reflections, are presented in the following subsection. 

\subsubsection{Single Reflection Procedure}
A simple case is considered with a reflecting wall laying on a plane $\Pi^I$, and transmitter (TX) and receiver (RX) located at different distances from $\Pi^I$. As a start, we consider the case where TX and RX move with instantaneous speeds $\overline{v}_{TX}$ and $\overline{v}_{RX}$, while the reflecting wall is at rest. Without loss of generality, we can assume a proper reference system so that the wall lies on the plane of equation $y=0$, while TX and RX lie on the xy plane. Then, by using the image method and through simple geometric considerations we can derive the reflection point position ($Q_R$) at the considered time instant, which is located at the intersection between the reflecting plane and the line passing through RX and the image of TX:

\begin{equation}
\begin{gathered}
\begin{cases}
x_{Q_{R}}(t) = x_{TX}(t) + \frac{x_{RX}(t)-x_{TX}(t)}{y_{RX}(t)+y_{TX}(t)}~y_{TX}(t) \\
y_{Q_{R}}(t) = 0 \\
z_{Q_{R}}(t) = z_{TX}(t) + \frac{z_{RX}(t)-z_{TX}(t)}{y_{RX}(t)+y_{TX}(t)}~y_{TX}(t)
\end{cases}
\end{gathered}
\label{ref_point}
\end{equation}
When TX/RX move with a certain speed, the corresponding reflection point "slides" along the plane surface: therefore, by deriving $Q_R$ coordinates with respect to time, the instantaneous velocity of the reflection point ($\overline{v}_{Q_{R}}=v_{Q_{R,x}}\hat{x}+v_{Q_{R,z}}\hat{z}$) can be determined, by using the derivative chain rule. 

For example, the x-component of $\overline{v}_{Q_{R}}$ is:
\begin{equation}
\begin{gathered}
v_{Q_{R,x}} = \frac{\partial x_{Q_R}}{\partial t} = \frac{\partial x_{Q_R}}{\partial x_{TX}}~v_{TX,x} + \frac{\partial x_{Q_R}}{\partial y_{TX}}~v_{TX,y}  \\ + \frac{\partial x_{Q_R}}{\partial x_{RX}}~v_{RX,x} + \frac{\partial x_{Q_R}}{\partial y_{RX}}~v_{RX,y}.
\end{gathered}
\label{ref_point_velo}
\end{equation}
The closed-form expressions of the derivatives in eq. (\ref{ref_point_velo}) are reported in Appendix B.


This approach is similar to the one adopted in \cite{qua}, but in the following it will be extended to a more general case, where the reflecting wall has a roto-translational motion, and TX, RX, and the reflecting wall can vary their velocities during time, so their motion is accelerated.
The basic idea is to use a local reference system integral with the wall (local frame) so that we can resort to the previous case where the wall is at rest and apply equations (\ref{ref_point}) and (\ref{ref_point_velo}) again, as shown below. \par

%When TX/RX and the reflecting plane move with a certain speed, the corresponding reflection point "slides" along the plane surface:% 
Moreover, in the present work we make a complete characterization of the reflection point motion, including acceleration: in fact, except very particular cases, $Q_R$ velocity is not constant in time, i.e. the reflection point has an accelerated motion. Therefore, with a similar method as in (\ref{ref_point_velo}), the acceleration of the reflection point ($\overline{a}_{Q_{R}}=a_{Q_{R,x}}\hat{x}+a_{Q_{R,z}}\hat{z}$) can be calculated by deriving $\overline{v}_{Q_{R}}$ with respect to time. For instance, the x-component of the acceleration will be:  
\begin{equation}
a_{Q_{R,x}} = \frac{\partial v_{Q_{R,x}}}{\partial t} = \frac{\partial}{\partial t} \frac{\partial x_{Q_{R}}}{\partial t} 
\label{ref_point_acc}
\end{equation}
The complete expression of $\overline{a}_{Q_{R}}$, and the detailed computation of $\overline{v}_{Q_{R}}$ and $\overline{a}_{Q_{R}}$ are presented in Appendix B. \par 

In Fig. \ref{scenario_before_after}, an example of a single reflection scenario is presented. For the sake of simplicity, we refer to a bi-dimensional case where TX/RX are located in a horizontal plane $z=z_0$ and moving with arbitrary speeds, meanwhile the reflecting wall is located along a vertical plane $y=y_0$, and is then represented with a straight line. Moreover, the wall rotates around a vertical axis, and the intersection of the rotation axis with the plane $z=z_0$ provides the rotation center $O^I$. The local reference system $O^Ix^Iy^Iz^I$ centered in $O^I$ (local frame), is also represented in the figure. The axes of the local frame are oriented so that at each time instant the reflecting wall is laying on the $z^Ix^I$ plane of equation $y=0$, in order to apply eq. (\ref{ref_point}).
The scenario represented in the figure is simplified for ease of readability, but the presented method is general, so the TX/RX positions and velocities, the wall plane and the rotation axis can be arbitrarily oriented with respect to the global reference system $Oxyz$.

\begin{figure}[!ht]
	\centering
	\includegraphics[width=3.5in]{scenario_t0_and_t0+delta}
	\caption{Example of a single reflection scenario with the plane $\Pi^I$ that rotates w.r.t. to the global frame. a) Scenario at $t=t_0$ and instantaneous velocities of TX, image-TX, RX, and reflection point $Q_R$. b) Scenario at $t=t_0+\Delta t$.}
	\label{scenario_before_after}
\end{figure}

Fig. \ref{scenario_before_after}a) shows the configuration of the system at a certain time $t=t_0$, when the wall plane $\Pi^{I}$ is parallel to the plane $xz$ of the global frame $Oxyz$. The instantaneous velocities of TX ($\overline{v}_{TX}$), image-TX ($\overline{v}_{TX'}$), reflection point ($\overline{v}_{Q_R}$) and RX ($\overline{v}_{RX}$) at $t=t_0$ are depicted as well. The instantaneous angular velocity of the wall plane is expressed by the vector $\overline{\omega}=\omega \hat{k}$, where $\omega=\frac{d \alpha}{dt}$ is the scalar angular velocity, and $\hat{k}$ is a unit vector parallel to the rotation axis, and properly oriented according to the right-hand rule. In the example of Fig. \ref{scenario_before_after}, we have $\hat{k}=-\hat{z}^I$, as the wall is rotating clockwise around the z-axis of the local frame.

Fig. \ref{scenario_before_after}b) shows the configuration of the system in a subsequent time instant $t=t_0+\Delta t$, when the wall plane has rotated clockwise by an angle $\theta$, and TX/RX have moved to different positions: the result of this motion is a shift of the reflection point $Q_R$ along the reflecting wall.  

In Fig. \ref{scenario_before_after} a wall rotating around a fixed rotation axis is shown, but in general, a translational motion of the wall plane can be also present, in addition to the rotational motion.
In the general case of roto-translational motion, any point Q of the wall will have a different speed, given by \cite{Martin1968}:
\begin{equation}
\overline{v}_Q=\overline{v}_{\Pi^I}+\overline{\omega} \times \overline{O^IQ}
\label{RigidBodyMotion}
\end{equation}
where the symbol "$\times$" stands for the cross vector product, $\overline{v}_{\Pi^I}$ is the translation velocity, common to all the points of the plane, and $\overline{O^IQ}$ is the position vector of the considered point Q w.r.t. the origin $O^I$ of the local frame, positioned on the (instantaneous) rotation axis.
\par

The first step of the DRT procedure consists in the computation of the (instantaneous) position of the reflecting point $Q_R$. Thanks to the adoption of the local frame, this can be accomplished through eq. (\ref{ref_point}), but in order to do that, we need to transform the coordinates of TX and RX into the local frame associated with the wall. To this end, we observe that the positions of TX/RX with respect to the global and local frames are related each other through the following \textit{coordinate transformation} \cite{Martin1968}:
\begin{equation}
\begin{gathered}
    \overline{r}_{TX}^0(t) = \Bar{\Bar{R}}(t) \cdot \overline{r}_{TX}^{I}(t) + \overline{r}_{0^I}(t) \\
    \overline{r}_{RX}^0(t) = \Bar{\Bar{R}}(t) \cdot \overline{r}_{RX}^{I}(t) + \overline{r}_{0^I}(t)
\end{gathered}
\label{coordinate_transf}
\end{equation}
with
\begin{center}
\begin{small}
$\overline{r}_{TX}^0(t)=[x_{TX}(t)~y_{TX}(t)~z_{TX}(t)]^T$ \\
$\overline{r}_{RX}^0(t)=[x_{RX}(t)~y_{RX}(t)~z_{RX}(t)]^T$
\end{small}
\end{center} 
and 
\begin{center}
\begin{small}
$\overline{r}_{TX}^I(t)=[x_{TX}^I(t)~y_{TX}^I(t)~z_{TX}^I(t)]^T$ \\
$\overline{r}_{RX}^I(t)=[x_{RX}^I(t)~y_{RX}^I(t)~z_{RX}^I(t)]^T$ 
\end{small} 
\end{center}
being the position vectors of TX/RX w.r.t. the global and local frame, respectively, where $\Bar{\Bar{R}}(t)$ is the (instantaneous) rotation matrix and $\overline{r}_{0^I}(t)=[x_{O^I}(t)~y_{O^I}(t)~z_{O^I}(t)]^T$ is the position vector associated with the origin point $O^{I}$ of the local frame. By inverting eq. (\ref{coordinate_transf}), we can determine the positions $\overline{r}_{TX}^I$, $\overline{r}_{RX}^I$ of TX and RX w.r.t. to the local frame, and then apply eq. (\ref{ref_point}) to find the coordinates of $Q_R$.

In the simple example of Fig. \ref{scenario_before_after} (clockwise rotation around the z-axis), the rotation matrix at the time instant $t=t_0+\Delta t$ is given by:
\begin{equation*}
\Bar{\Bar{R}}(t_0+\Delta t)=
\begin{pmatrix}
cos\theta & sin\theta & 0\\
-sin\theta & cos\theta & 0\\
0 & 0 & 1
\end{pmatrix}
\end{equation*}
In a generic time instant $t=t_0+\Delta t$ within the multipath lifetime $T_C$, the instantaneous rotation angle $\theta(t)$ can be obtained in the following way, similarly to eq.(\ref{Taylor}) and neglecting time variations of angular acceleration:
\begin{equation*}
\theta(t)=\theta(t_0)+\omega\Delta t+\frac{1}{2}\frac{d\omega}{dt}\Delta t^2
\end{equation*}
Of course, \textit{coordinate transformation} must be applied to the components of $\overline{v}_{TX}$, $\overline{v}_{RX}$ and $\overline{a}_{TX}$, $\overline{a}_{RX}$ as well.

The next step of the DRT procedure consists in the computation of the velocity and acceleration of $Q_R$ which can be obtained through eq. (\ref{ref_point_velo}) and (\ref{ref_point_acc}). In fact, one of the main advantages of the DRT approach is that we can derive $\overline{v}_{Q_R}$ and $\overline{a}_{Q_R}$ analytically, without need to resort to time differences methods. However, projecting the components of $\overline{v}_{TX}$, $\overline{v}_{RX}$ and $\overline{a}_{TX}$, $\overline{a}_{RX}$ on the local frame is not enough to apply eq. (\ref{ref_point_velo}) and (\ref{ref_point_acc}), which have been obtained from eq. (\ref{ref_point}), i.e. assuming that the reflecting wall is at rest.  Since now the reflecting wall is moving, and then the local frame is in motion w.r.t. the global reference system, the velocities and accelerations of TX/RX must be preliminary transformed according to the \textit{relative motion transformations}, to obtain "relative" velocities and accelerations w.r.t. an observer located in the origin of the local frame \cite{taylor2005}. For instance, the velocity and acceleration of TX must be transformed in the following way: 
\begin{gather}
\overline{v}_{TX}^{I} = \overline{v}_{TX}^{0} - \overline{v}_{\Pi^{I}} - \overline{\omega} \times \overline{r}_{TX}^{I} \label{relative_vel} \\
\overline{a}_{TX}^{I} = \overline{a}_{TX}^{0} - \overline{a}_{\Pi^{I}} - \dot{\overline{\omega}} \times \overline{r}_{TX}^{I} - 2 \overline{\omega} \times \overline{v}_{TX}^{I} \label{relative_acc} \\ \nonumber
-\overline{\omega} \times (\overline{\omega} \times \overline{r}_{TX}^{I})
\end{gather}
where the headers $"I"$ and $"0"$ are associated with the velocities and accelerations seen by an observer located in the origin of the local and global frame, respectively, $\overline{r}_{TX}^{I}=\overline{r}_{TX}^{0}-\overline{r}_{O^{I}}$ is the position vector of TX in the local frame, $\overline{v}_{\Pi^{I}}$, $\overline{\omega}$ are the translation and angular velocities of the wall plane $\Pi^{I}$, and $\overline{a}_{\Pi^{I}}$, $\dot{\overline{\omega}}=\frac{d\overline{\omega}}{dt}$ are the corresponding translation and angular accelerations, respectively.
The terms $\dot{\overline{\omega}} \times \overline{r}_{TX}^{I}$, $2 \overline{\omega} \times \overline{v}_{TX}^{I}$ and $\overline{\omega} \times (\overline{\omega} \times \overline{r}_{TX}^{I})$ in eq. (\ref{relative_acc}) are also known as Euler's, Coriolis', and centrifugal acceleration, respectively.

It is worth noting that the accelerations $\overline{a}_{\Pi^{I}}$, $\dot{\overline{\omega}}$ are assumed to be constant in the time interval $[t_0~t_0+T_C]$, while $\overline{v}_{\Pi^{I}}$, $\overline{\omega}$ in eq. (\ref{relative_vel}) are \textit{instantaneous} velocities, computed as:
\begin{equation*}
\begin{gathered}
\overline{v}_{\Pi^{I}}(t)=\overline{v}_{\Pi^{I}}(t_0)+\overline{a}_{\Pi^{I}}\Delta t \\
\overline{\omega}(t)=\overline{\omega}(t_0)+\dot{\overline{\omega}} \Delta t
\end{gathered}
\end{equation*}

In practice, with the \textit{relative motion transformations} (\ref{relative_vel}) and (\ref{relative_acc}) we turn the original problem into an equivalent problem, where the wall plane is at rest and the TX/RX velocities and accelerations are modified, according to the point of view of an observer located in the origin of the local frame. It is worth noting that, even in the simple case of TX/RX moving with constant speed, TX and RX are accelerated in the equivalent problem: this acceleration is caused by the angular rotation ${\overline{\omega}}$ of the wall plane, according to eq. (\ref{relative_vel}) and (\ref{relative_acc}).
\par

Once the reflection point position, velocity and acceleration have been determined in the local frame using eq. (\ref{ref_point}), (\ref{ref_point_velo}), and (\ref{ref_point_acc}), the following inverse transformations need to be applied: 
\begin{itemize}
	\item back-transformation to get $Q_R$ coordinates, and $\overline{v}_{Q_{R}}$ and $\overline{a}_{Q_{R}}$ components w.r.t. the global reference system (inverse \textit{coordinate  transformation}).
	\item back-transformation to get the velocity and acceleration ($\overline{v}_{Q_{R}}$ and $\overline{a}_{Q_{R}}$) relative to an observer located in the origin of the global reference system (inverse \textit{relative motion transformation}):
	\begin{gather}
	    \overline{v}_{Q_R}^{0} = \overline{v}_{Q_R}^{I} + \overline{v}_{\Pi^{I}} + \overline{\omega} \times \overline{r}_{Q_R}^{I} \label{global_vel_ref_point} \\
        \overline{a}_{Q_R}^{0} = \overline{a}_{Q_R}^{I} + \overline{a}_{\Pi^{I}} + \dot{\overline{\omega}} \times \overline{r}_{Q_R}^{I} + 2 \overline{\omega} \times \overline{v}_{Q_R}^{I} \label{global_acc_ref_point} \\ \nonumber
        +\overline{\omega} \times (\overline{\omega} \times \overline{r}_{Q_R}^{I})   
	\end{gather}
\end{itemize}  


\subsubsection{Generalization to Multiple Reflections}
In the case of multiple reflections, the motion of a certain reflection point is influenced by the motion of the previous or latter reflection points.  
However, the method discussed above can be extended in a straightforward way to a multiple-bounce case. 

Let's consider for simplicity a double reflection case: instead of using TX, we can resort to the use of the image-TX ($TX^{'}$) with respect to the first wall plane, and after that we can analyze the reflection on the second wall: in practice, we replace $TX$ with $TX^{'}$ and we bring the computation back to the single-reflection scenario analyzed in the previous section. This means that in a case with two reflecting walls, $TX^{'}$ and $RX$ are used to compute the motion of the reflection point along the second reflecting wall ($Q_{R2}$), by applying the same procedure as for the single-bounce scenario. Then, once $Q_{R2}$ is known, it is used as a new virtual receiver to compute, in addition to the TX location, the position and the motion of the reflection point along the first reflecting wall ($Q_{R1}$). 

This approach can be iterated in a similar way for the case of more than 2 reflections. In practice, we compute the image-TX ($TX^{'}$), the image of the image ($TX^{''}$), etc., until we reach the last reflecting wall, then a "back-tracking" procedure is used: we start from RX and the last reflecting wall, we apply all the equations of the previous section to compute the last reflection point, and then we move back towards TX, to trace the motion of all the remaining reflection points.  

In order to apply all the equations of the previous section for the determination of the reflection points motion, we need to apply all the necessary transformations from the global to the local frame, and vice-versa, for each of the reflecting walls.

Moreover, a preliminary computation is needed to compute the position and the instantaneous velocity for each of the image-transmitters. This can be done in a straightforward way by relying on the local frame, and using the image principle. For example, the image-TX with respect to the first wall ($TX^{'}$), will have the same $x^I$ and $z^I$ coordinates as TX, and opposite $y^I$ coordinate. Similarly, the velocity components will be:

\begin{equation}
\begin{gathered}
\begin{cases}
v_{TX^{'},x}^{I} = v_{TX,x}^{I} \\
v_{TX^{'},y}^{I} = -v_{TX,y}^{I} \\
v_{TX^{'},z}^{I} = v_{TX,z}^{I}
\end{cases}
\end{gathered} 
\end{equation}

Once $\overline{v}_{TX^{'}}$ is computed in the local frame, it can be expressed w.r.t. to an observer located in the origin of the global frame, according to the relative motion transformation:
\begin{equation}
\overline{v}_{TX^{'}}^{0}=\overline{v}_{TX^{'}}^{I}+\overline{\omega} \times \overline{r}_{TX^{'}}^{I}
\end{equation}
In a similar way, the acceleration of the image-TX, $\overline{a}_{TX^{'}}$, can be also found.
The DRT algorithm then proceeds according to steps described above, to find the coordinates, velocities and accelerations for each of the reflection points.
\par

The whole DRT algorithm is summarized by the flowchart depicted in Fig. \ref{flowchart}, with reference to a double-bounce case, for the sake of simplicity. 

Once the geometric part is done, i.e. the analytical prolongation of all the rays in the considered time instants within $T_C$ has been completed, the very last step of DRT consists in the re-computation of the field associated to each ray. This is done in a straightforward way as the geometry of the rays is known, by applying the Fresnel's reflection coefficients and the ray divergence factor, as usually done in standard RT algorithms \cite{fuschini2015,vitucci2019}. It is worth noting however, that ray's field computation is based on analytical formulas and is therefore orders of magnitude faster than ray's geometry computation \cite{fuschini2015}.

\begin{figure}[h!]
	\centering
	\includegraphics[width=2.5in]{flowchart_2RFL}
	\caption{Flowchart showing the DRT algorithm for a double reflection case.}
	\label{flowchart}
\end{figure}

\subsection{Diffraction Points' Calculation}
The DRT algorithm can be further extended to diffraction, modeled with a ray-based approach according the Uniform Theory of Diffraction (UTD) \cite{UTD}.
A method to track the motion of diffraction points in an analytical way, as previously done for the reflection points, is outlined in this section. We present only the single-diffraction case for the sake of brevity, but the procedure can be extended in a straightforward way to multiple diffractions, as well as to combinations of multiple reflections and diffractions.

In Fig. \ref{diffraction} diffraction from an edge formed by two adjacent walls is illustrated, where the unit edge vector $\hat{e}$ is chosen to be aligned with the $z$-axis of the reference system $Oxyz$, with no loss of generality. For the sake of simplicity and with no limitation - as the diffracted rays lay on the Keller's cone and share the same geometric properties \cite{Keller} - we represent in Fig. \ref{diffraction} an "unfolded" diffracted ray, i.e. the diffraction plane has been rotated to be coincident with the incidence plane. 

\begin{figure}[!ht]
	\centering
	\includegraphics[width=3.2in]{diffraction_scenario}
	\caption{Example of edge diffraction, and related geometry for the computation of the diffraction point.}
	\label{diffraction}
\end{figure}

Since the diffraction point ($Q_D$) is constrained to move along the edge, the $x$ and $y$ coordinates of $Q_D$ are known, then only $z_{Q_D}$ remains to be computed: this can be done in a simple way by using the similar triangles properties.  \par
Looking at Fig. \ref{diffraction}, we see that two similar triangles are formed. The sides of these triangles are proportional each other, so the following relation holds: 
\begin{equation}
z_{TX}-z_{RX} : d_{TX}+d_{RX} = z_{Q_D}-z_{RX} : d_{RX}
\end{equation} 
where 
\begin{equation}
\begin{gathered}
d_{TX}(t) = \sqrt{(x_{TX}(t)-x_{Q_D})^2+(y_{TX}(t)-y_{Q_D})^2} \\[1ex]
d_{RX}(t) = \sqrt{(x_{RX}(t)-x_{Q_D})^2+(y_{RX}(t)-y_{Q_D})^2}
\end{gathered}
\label{dist2D_TXRX}
\end{equation}
are the 2D distances of TX and RX from the edge, respectively. Hence, the z-coordinate of $Q_D$ is given by: 
\begin{equation}
z_{Q_D}(t) = z_{RX}(t) + \frac{d_{RX}(t)\cdot[z_{TX}(t)-z_{RX}(t)]}{d_{TX}(t)+d_{RX}(t)}
\label{diff_point}
\end{equation}
The instantaneous velocity of $Q_D$ ($\overline{v}_{Q_D}=v_{Q_{D,z}}\hat{z}$) can be computed by time deriving $z_{Q_D}$. The detailed calculation of $v_{Q_{D,z}}$ is presented in Appendix C.\par
Similarly, by further deriving $\overline{v}_{Q_D}$, we can find the acceleration of the diffraction point, $\overline{a}_{Q_D}=a_{Q_{D,z}}\hat{z}$, not reported here for the sake of brevity.\par
The expression in (\ref{diff_point}) and the related velocity and acceleration $\overline{v}_{Q_D}$, $\overline{a}_{Q_D}$ are valid and do not require any further computation in case the terminals are moving but the edge is at rest. However, in general an edge might be part of a moving object and then might be moving with a certain roto-translational velocity. In particular, we assume that the edge has a rotational motion with instantaneous angular velocity $\overline{\omega}$, and is also translating according to the instantaneous velocity $\overline{v}_{edge}$. In the example of Fig. \ref{diffraction}, the edge is rotating clockwise around the x-axis. 

As for reflections, we can compute the instantaneous position and the motion of the diffraction point if we assume a proper local reference $O^Ix^Iy^Iz^I$, with the origin located in the rotation center $O^I$, and the z-axis parallel to the edge. By doing so, the same procedure used for reflections, can be followed to transform velocities and accelerations, and finally find $z_{Q_D}$ as well as $v_{Q_{D,z}}$ and $a_{Q_{D,z}}$.  \par 
The final step of DRT is, as usual, the computation of the updated UTD coefficients, and then, of the total diffracted field, at the considered time instant.

\subsection{Diffuse scattering}
Diffuse scattering is modeled according to the Effective Roughness approach \cite{vitucci2019}, which is based on a subdivision of each surface into tiles and on the application of a virtual scattering source to the centroid of each tile. Therefore, the  calculation of scattering points' position and speed is straightforward, as it boils down to the application of basic kinematic equations for the motion of rigid bodies' surface points. For instance, equation (\ref{RigidBodyMotion}) can be used to compute each scattering point's velocity if the rototranslation speed of the body is known.



\section{Results}
%!TEX ROOT = ../../centralized_vs_distributed.tex

\section{{\titlecap{the centralized-distributed trade-off}}}\label{sec:numerical-results}

\revision{In the previous sections we formulated the optimal control problem for a given controller architecture
(\ie the number of links) parametrized by $ n $
and showed how to compute minimum-variance objective function and the corresponding constraints.
In this section, we present our main result:
%\red{for a ring topology with multiple options for the parameter $ n $},
we solve the optimal control problem for each $ n $ and compare the best achievable closed-loop performance with different control architectures.\footnote{
\revision{Recall that small (large) values of $ n $ mean sparse (dense) architectures.}}
For delays that increase linearly with $n$,
\ie $ f(n) \propto n $, 
we demonstrate that distributed controllers with} {few communication links outperform controllers with larger number of communication links.}

\textcolor{subsectioncolor}{Figure~\ref{fig:cont-time-single-int-opt-var}} shows the steady-state variances
obtained with single-integrator dynamics~\eqref{eq:cont-time-single-int-variance-minimization}
%where we compare the standard multi-parameter design 
%with a simplified version \tcb{that utilizes spatially-constant feedback gains
and the quadratic approximation~\eqref{eq:quadratic-approximation} for \revision{ring topology}
with $ N = 50 $ nodes. % and $ n\in\{1,\dots,10\} $.
%with $ N = 50 $, $ f(n) = n $ and $ \tau_{\textit{min}} = 0.1 $.
%\autoref{fig:cont-time-single-int-err} shows the relative error, defined as
%\begin{equation}\label{eq:relative-error}
%	e \doteq \dfrac{\optvarx-\optvar}{\optvar}
%\end{equation}
%where $ \optvar $ and $ \optvarx $ denote the the optimal and sub-optimal scalar variances, respectively.
%The performance gap is small
%and becomes negligible for large $ n $.
{The best performance is achieved for a sparse architecture with  $ n = 2 $ 
in which each agent communicates with the two closest pairs of neighboring nodes. 
This should be compared and contrasted to nearest-neighbor and all-to-all 
communication topologies which induce higher closed-loop variances. 
Thus, 
the advantage of introducing additional communication links diminishes 
beyond}
{a certain threshold because of communication delays.}

%For a linear increase in the delay,
\textcolor{subsectioncolor}{Figure~\ref{fig:cont-time-double-int-opt-var}} shows that the use of approximation~\eqref{eq:cont-time-double-int-min-var-simplified} with $ \tilde{\gvel}^* = 70 $
identifies nearest-neighbor information exchange as the {near-optimal} architecture for a double-integrator model
with ring topology. 
This can be explained by noting that the variance of the process noise $ n(t) $
in the reduced model~\eqref{eq:x-dynamics-1st-order-approximation}
is proportional to $ \nicefrac{1}{\gvel} $ and thereby to $ \taun $,
according to~\eqref{eq:substitutions-4-normalization},
making the variance scale with the delay.

%\mjmargin{i feel that we need to comment about different results that we obtained for CT and DT double-intergrator dynamics (monotonic deterioration of performance for the former and oscillations for the latter)}
\revision{\textcolor{subsectioncolor}{Figures~\ref{fig:disc-time-single-int-opt-var}--\ref{fig:disc-time-double-int-opt-var}}
show the results obtained by solving the optimal control problem for discrete-time dynamics.
%which exhibit similar trade-offs.
The oscillations about the minimum in~\autoref{fig:disc-time-double-int-opt-var}
are compatible with the investigated \tradeoff~\eqref{eq:trade-off}:
in general, 
the sum of two monotone functions does not have a unique local minimum.
Details about discrete-time systems are deferred to~\autoref{sec:disc-time}.
Interestingly,
double integrators with continuous- (\autoref{fig:cont-time-double-int-opt-var}) ad discrete-time (\autoref{fig:disc-time-double-int-opt-var}) dynamics
exhibits very different trade-off curves,
whereby performance monotonically deteriorates for the former and oscillates for the latter.
While a clear interpretation is difficult because there is no explicit expression of the variance as a function of $ n $,
one possible explanation might be the first-order approximation used to compute gains in the continuous-time case.
%which reinforce our thesis exposed in~\autoref{sec:contribution}.

%\begin{figure}
%	\centering
%	\includegraphics[width=.6\linewidth]{cont-time-double-int-opt-var-n}
%	\caption{Steady-state scalar variance for continuous-time double integrators with $ \taun = 0.1n $.
%		Here, the \tradeoff is optimized by nearest-neighbor interaction.
%	}
%	\label{fig:cont-time-double-int-opt-var-lin}
%\end{figure}
}

\begin{figure}
	\centering
	\begin{minipage}[l]{.5\linewidth}
		\centering
		\includegraphics[width=\linewidth]{random-graph}
	\end{minipage}%
	\begin{minipage}[r]{.5\linewidth}
		\centering
		\includegraphics[width=\linewidth]{disc-time-single-int-random-graph-opt-var}
	\end{minipage}
	\caption{Network topology and its optimal {closed-loop} variance.}
	\label{fig:general-graph}
\end{figure}

Finally,
\autoref{fig:general-graph} shows the optimization results for a random graph topology with discrete-time single integrator agents. % with a linear increase in the delay, $ \taun = n $.
Here, $ n $ denotes the number of communication hops in the ``original" network, shown in~\autoref{fig:general-graph}:
as $ n $ increases, each agent can first communicate with its nearest neighbors,
then with its neighbors' neighbors, and so on. For a control architecture that utilizes different feedback gains for each communication link
	(\ie we only require $ K = K^\top $) we demonstrate that, in this case, two communication hops provide optimal closed-loop performance. % of the system.}

Additional computational experiments performed with different rates $ f(\cdot) $ show that the optimal number of links increases for slower rates: 
for example, 
the optimal number of links is larger for $ f(n) = \sqrt{n} $ than for $ f(n) = n $. 
\revision{These results are not reported because of space limitations.}

\section{Conclusions}
\section{Conclusions}
\label{sec:conclusions}

In this paper, we apply shared-workload techniques at the \sql level for
improving the throughput of \qaasl systems without incurring in additional
query execution costs. Our approach is based on query rewriting for grouping
multiple queries together into a single query to be executed in one go. This
results in a significant reduction of the aggregated data access done by the
shared execution compared to executing queries independently.

%execution times and costs of the shared scan operator when
%varying query selectivity and predicate evaluation. We observed that for
%\athena, although the cost only depends on the amount of data read, it is
%conditioned to its ability to use its statistics about the data. In some cases
%a wrong query execution plan leads to higher query execution costs, which the
%end-user has to pay. 

%\bigquery's minimum query execution cost is determined by
%the input size of a query.  However, the query cost can increase depending not
%just in the amount of computation it requires, but in the mix of resources the
%query requires.  

We presented a cost and runtime evaluation of the shared operator driving data access costs. 
Our experimental study using the TPC-H benchmark confirmed the benefits of our
query rewrite approach. Using a shared execution approach reduces significantly
the execution costs. For \athena, we are able to make it 107x cheaper and for
\bigquery, 16x cheaper taking into account Query 10 which we cannot execute,
but 128x if it is not taken into account. Moreover, when having queries that do
not share their entire execution plan, i.e., using a single global plan, we
demonstrated that it is possible to improve throughput and obtain a 10x cost
reduction in \bigquery.

%We followed the TPC systems pricing guideline for
%computing how expensive is to have a TPC-H workload working on the evaluated
%\qaasl systems. The result is that even though we are able to reduce overall
%costs a TPC-H workload in 15x for \bigquery (128x excluding query 10 which we
%could not optimize) and in 107x for \athena, the overall price is at least 10x
%more expensive than the cheapest system price published by the TPC.

There are multiple ways to extend our work. The first is
to implement a full \sql-to-\sql translation layer to encapsulate the proposed
per-operator rewrites.  Another one is to incorporate the initial work on
building a cost-based optimizer for shared execution
\cite{Giannikis:2014:SWO:2732279.2732280} as an external component for \qaasl
systems.  Moreover, incorporating different lines of work (e.g., adding
provenance computation \cite{GA09} capabilities) also based on query
rewriting is part of our future work to enhance our system.


%\section*{Acknowledgements}

\appendices
\section{Doppler Frequency Calculation}
One of the advantages of the DRT approach is the computation of Doppler information online in the algorithm with the aid of simple formulas. In such a way, there is no need to consider successive “snapshots” of the environment with slightly different displacements of the objects, and then to calculate the Doppler shifts with a “finite difference” computation method. \par 
	When TX and RX are both moving, the resulting apparent frequency $f^{'}$, including the Doppler frequency shift $f_D$ is computed for the LoS ray using the following equation \cite{ChenDoppler}: 
	\begin{equation}
	f^{'} = f_{0} + f_{D} = f_{0} \left (\frac{c-\overline{v}_{RX}\cdot \hat{k}}{c-\overline{v}_{TX}\cdot \hat{k}} \right) 
	\end{equation}
	where $f_0$ is the carrier frequency of the transmitted signal, $\hat{k}$ is the unit vector of the ray's direction from TX towards RX, and $\overline{v}_{TX}$, $\overline{v}_{RX}$ are the velocities of transmitter and receiver, respectively. \par
	This formula can be extended to rays with multiple bounces, where we have $n$ scattering points, each one moving with different speed (see Fig. \ref{doppler}): 
	\begin{equation}
	f^{'} = f_{0} + f_{D} = f_{0} \prod_{i=1}^{n+1} \left(\frac{c-\overline{v}_{i}\cdot \hat{k}_i}{c-\overline{v}_{i-1}\cdot \hat{k}_i}  \right)
	\label{doppler_freq}
	\end{equation}	
    where $\overline{v}_i$, $i=1,2,..,n$ is the velocity of the i-th interaction point, $\overline{v}_0=\overline{v}_{TX}$, and $\overline{v}_{n+1}=\overline{v}_{RX}$, respectively.\par
	\begin{figure}[h!]
		\centering
		\includegraphics[width=3in]{doppler_freq}
		\caption{Representation of the multiple scatterers Doppler model}
		\label{doppler}
	\end{figure}
	Equation (\ref{doppler_freq}) assumes that the velocities of the interaction points on the objects are known. In the simplified case of small scattering objects that can be approximated as "point scatterers", no further processing is needed, and we can directly apply eq. (\ref{doppler_freq}). \par 
	Instead, in the case of large objects we need to compute the velocity of the interaction points on the object surface as explained in section III. 

\section{Reflection Point's Velocity and Acceleration Calculation}
	Given the reflection point position ($Q_R$), its velocity ($\overline{v}_{Q_{R}}$) can be determined by deriving eq. (\ref{ref_point}) with respect to time. The x-component of $\overline{v}_{Q_{R}}$ can be calculated as: 
	
	\begin{equation}
	\begin{gathered}
	v_{Q_{R,x}} = \frac{\partial x_{Q_R}}{\partial t} = \frac{\partial x_{Q_R}}{\partial x_{TX}}~\frac{\partial x_{TX}}{\partial t} + \frac{\partial x_{Q_R}}{\partial y_{TX}}~\frac{\partial y_{TX}}{\partial t} \\ 
	+ \frac{\partial x_{Q_R}}{\partial x_{RX}}~\frac{\partial x_{RX}}{\partial t} + \frac{\partial x_{Q_R}}{\partial y_{RX}}~\frac{\partial y_{RX}}{\partial t}  \\ 
	= f_{x_{TX}} + f_{y_{TX}} + f_{x_{RX}} + f_{y_{RX}}, 
	\end{gathered}
	\label{vQRx}
	\end{equation}
	with 
	\begin{equation*}
	\begin{gathered}
	    \begin{cases*}
	        f_{x_{TX}} = \frac{\partial x_{Q_R}}{\partial x_{TX}}~v_{TX,x} \\
	        f_{y_{TX}} = \frac{\partial x_{Q_R}}{\partial y_{TX}}~v_{TX,y} \\ 
	        f_{x_{RX}} = \frac{\partial x_{Q_R}}{\partial x_{RX}}~v_{RX,x} \\ 
	        f_{y_{RX}} = \frac{\partial x_{Q_R}}{\partial y_{RX}}~v_{RX,y}.
	    \end{cases*}
	\end{gathered}
   \end{equation*}
	
The partial derivatives in (\ref{vQRx}) can be easily computed by deriving $x_{Q_R}$ w.r.t. the $x$,$y$ coordinates of TX and RX: 
\begin{equation*}
	\begin{gathered}
	\begin{cases*}
	\frac{\partial x_{Q_R}}{\partial x_{TX}} = 1 - \frac{y_{TX}}{y_{TX}+y_{RX}}\\ 
	\frac{\partial x_{Q_R}}{\partial y_{TX}} = \frac{y_{RX} (x_{RX}-x_{TX})}{(y_{RX}+y_{TX})^2}\\
	\frac{\partial x_{Q_R}}{\partial x_{RX}} = \frac{y_{TX}}{y_{TX}+y_{RX}} \\ 
	\frac{\partial x_{Q_R}}{\partial y_{RX}} = \frac{-y_{RX} (x_{RX}-x_{TX})}{(y_{RX}+y_{TX})^2}
	\end{cases*}
	\end{gathered} 
\end{equation*}
and substituting these expressions in (\ref{vQRx}) we finally get $v_{Q_R,x}$. \par
By following the same method as above, we can obtain the z-component of the velocity, $v_{Q_R,z}$, by time deriving the z-coordinate of $Q_R$ ($z_{Q_R}$). \par 
A similar procedure is used to calculate $\overline{a}_{Q_{R}}$ by deriving $\overline{v}_{Q_{R}}$ with respect to time:
\begin{equation}
	\begin{gathered}
	a_{Q_R,x}=a_{Q_R,x}^{(1)}+a_{Q_R,x}^{(2)}+a_{Q_R,x}^{(3)}+a_{Q_R,x}^{(4)} \\
	= \frac{\partial}{\partial t} f_{x_{TX}}+\frac{\partial}{\partial t} f_{y_{TX}}+\frac{\partial}{\partial t} f_{x_{RX}}+\frac{\partial}{\partial t} f_{y_{RX}}
	\end{gathered}
	\label{a_QRx}
\end{equation}
For the sake of brevity, we report below only the computation of the first partial derivative in eq. (\ref{a_QRx}): 
\begin{equation}
	\begin{gathered}
	a_{Q_R,x}^{(1)} = \frac{\partial}{\partial t} f_{x_{TX}} = \frac{\partial}{\partial t} \frac{\partial x_{Q_R}}{\partial x_{TX}} \frac{\partial x_{TX}}{\partial t} + \frac{\partial x_{Q_R}}{\partial x_{TX}} \frac{v_{TX,x}}{\partial t} \\ 
	= \left(\frac{\partial^2 x_{Q_R}}{\partial x_{TX} \partial y_{TX}} \frac{\partial y_{TX}}{\partial t} + \frac{\partial^2 x_{Q_R}}{\partial x_{TX} \partial y_{RX}} \frac{\partial y_{RX}}{\partial t} \right) v_{TX,x} \\
	+ \frac{\partial x_{Q_R}}{\partial x_{TX}} a_{TX,x}\\ 
	= \left(\frac{\partial^2 x_{Q_R}}{\partial x_{TX} \partial y_{TX}} v_{TX,y}  + \frac{\partial^2 x_{Q_R}}{\partial x_{TX} \partial y_{RX}} v_{RX,y} \right) v_{TX,x} \\ 
	+ \frac{\partial x_{Q_R}}{\partial x_{TX}} a_{TX,x}
	\end{gathered}
	\label{aQRx1}
\end{equation}
	where the partial derivatives in (\ref{aQRx1}) are expressed by:
	\begin{equation*}
	\begin{gathered}
	\begin{cases*}
	\frac{\partial x_{Q_R}}{\partial x_{TX}} = 1 - \frac{y_{TX}}{y_{TX}+y_{RX}}\\ 
	\frac{\partial^2 x_{Q_R}}{\partial x_{TX} \partial y_{TX}} = \frac{-y_{RX}}{\left( y_{TX}+y_{RX}\right)^2}\\ 
	\frac{\partial^2 x_{Q_R}}{\partial x_{TX} \partial y_{RX}} = \frac{y_{TX}}{\left( y_{TX}+y_{RX}\right)^2}.
	\end{cases*}
	\end{gathered}
	\end{equation*}
	By substituting these expressions in (\ref{aQRx1}) we obtain $a_{Q_R,x}^{(1)}$. \par
	By repeating the same procedure for the remaining components of $a_{Q_R,x}$, we finally get: 
	\begin{equation}
			\begin{gathered}
			a_{Q_R,x} = \left(\frac{\partial^2 x_{Q_R}}{\partial x_{TX} \partial y_{TX}} v_{TX,y}  + \frac{\partial^2 x_{Q_R}}{\partial x_{TX} \partial y_{RX}} v_{RX,y} \right) v_{TX,x}  \\
			+\frac{\partial x_{Q_R}}{\partial x_{TX}} a_{TX,x} + \left(\frac{\partial^2 x_{Q_R}}{\partial y_{TX} \partial x_{TX}} v_{TX,x} + \frac{\partial^2 x_{Q_R}}{\partial y_{TX} \partial x_{RX}} v_{RX,x} \right) v_{TX,y} \\ 
			+ \left (\frac{\partial^2 x_{Q_R}}{\partial y_{TX}^2 } v_{TX,y} + \frac{\partial^2 x_{Q_R}}{\partial y_{TX} \partial y_{RX} } v_{RX,y}  \right) v_{TX,y} + \frac{\partial x_{Q_R}}{\partial y_{TX}} a_{TX,y} \\ 
			+ \left(\frac{\partial^2 x_{Q_R}}{\partial x_{RX} \partial y_{TX}} v_{TX,y} + \frac{\partial^2 x_{Q_R}}{\partial x_{RX} \partial y_{RX}} v_{RX,y} \right) v_{RX,x} + \frac{\partial x_{Q_R}}{\partial x_{RX}} a_{RX,y} \\ 
			+ \left( \frac{\partial^2 x_{Q_R}}{\partial y_{RX} \partial x_{TX}} v_{TX,x} +  \frac{\partial^2 x_{Q_R}}{\partial y_{RX} \partial x_{RX}} v_{RX,x}  \right) v_{RX,y} \\ 
			+ \left(  \frac{\partial^2 x_{Q_R}}{\partial y_{RX} \partial y_{TX}} v_{TX,y} +  \frac{\partial^2 x_{Q_R}}{\partial y_{RX}^2} v_{RX,x}\right) v_{RX,x} + \frac{\partial x_{Q_R}}{\partial y_{RX}} a_{RX,y}.
			\end{gathered}
    \end{equation}
	
	A similar approach can be adopted to compute the z-component of  $\overline{a}_{Q_{R}}$.

\section{Diffraction Point's Velocity Calculation}
Time derivation of (\ref{diff_point}), gives us the z-component of the diffraction point instantaneous velocity: 
	\begin{equation}
	\begin{gathered}
	v_{Q_{D,z}} = \frac{\partial z_{Q_D}}{\partial t} = \frac{\partial z_{RX}}{\partial t} + \frac{\partial }{\partial t} \left(\frac{d_{RX}}{d_{RX}+d_{TX}} (z_{TX}-z_{RX}) \right) \\
	= v_{RX,z} + \frac{d_{RX}}{d_{RX}+d_{TX}} \left(v_{TX,z}-v_{RX,z} \right)\\ 
	+ (z_{TX}-z_{RX}) \frac{\partial}{\partial t} \left(\frac{d_{RX}}{d_{RX}+d_{TX}} \right)
	\end{gathered}
	\label{diff_point_velo}
	\end{equation}
	where \\
	\begin{equation}
	\begin{multlined}
	\frac{\partial}{\partial t} \left(\frac{d_{RX}}{d_{TX}+d_{RX}} \right)= \\[2ex]
	=\frac{\frac{\partial d_{RX}}{\partial t} (d_{RX}+d_{TX})-\left(\frac{\partial d_{RX}}{\partial t} + \frac{\partial d_{TX}}{\partial t} \right) d_{RX}}{(d_{TX}+d_{RX})^2}
	\end{multlined}
	\label{diff_point_velo2}
	\end{equation}
    \par 
	The derivative of $d_{RX}$ with respect to time can be computed by applying the derivative chain rule to eq. (\ref{dist2D_TXRX}):
	\begin{equation}
	\begin{gathered}
	\frac{\partial d_{RX}}{\partial t} = \frac{\partial d_{RX}}{\partial x_{RX}} \frac{\partial x_{RX}}{\partial t} + \frac{\partial d_{RX}}{\partial y_{RX}} \frac{\partial y_{RX}}{\partial t} \\ 
	= \frac{1}{d_{RX}}\left[ \left(x_{RX}-x_{Q_D}\right) v_{RX,x}+  \left(y_{RX}-y_{Q_D}\right) v_{RX,y}\right].
	\end{gathered}
	\label{diff_point_velo3}
	\end{equation}
	In similar way the derivative of $d_{TX}$ can be calculated: 
	\begin{equation}
	\frac{\partial d_{TX}}{\partial t} = \frac{1}{d_{TX}}\left[ \left(x_{TX}-x_{Q_D}\right) v_{TX,x}+  \left(y_{TX}-y_{Q_D}\right) v_{TX,y}\right]
	\label{diff_point_velo4}
	\end{equation}
By substituting (\ref{diff_point_velo3}) and (\ref{diff_point_velo4}) into (\ref{diff_point_velo2}) and (\ref{diff_point_velo}), we finally get $v_{Q_D,x}$.\par
The diffraction point's acceleration ($a_{Q_{D,z}}$) can be computed in a similar way by time deriving (\ref{diff_point_velo}). 


\section*{}
%%%%%%%%%%%%%%%%%%%%%%%% referenc.tex %%%%%%%%%%%%%%%%%%%%%%%%%%%%%%
% sample references
% %
% Use this file as a template for your own input.
%
%%%%%%%%%%%%%%%%%%%%%%%% Springer-Verlag %%%%%%%%%%%%%%%%%%%%%%%%%%
%
% BibTeX users please use
% \bibliographystyle{}
% \bibliography{}
%

\biblstarthook{References may be \textit{cited} in the text either by number (preferred) or by author/year.\footnote{Make sure that all references from the list are cited in the text. Those not cited should be moved to a separate \textit{Further Reading} section or chapter.} The reference list should ideally be \textit{sorted} in alphabetical order -- even if reference numbers are used for the their citation in the text. If there are several works by the same author, the following order should be used: 
\begin{enumerate}
\item all works by the author alone, ordered chronologically by year of publication
\item all works by the author with a coauthor, ordered alphabetically by coauthor
\item all works by the author with several coauthors, ordered chronologically by year of publication.
\end{enumerate}
The \textit{styling} of references\footnote{Always use the standard abbreviation of a journal's name according to the ISSN \textit{List of Title Word Abbreviations}, see \url{http://www.issn.org/en/node/344}} depends on the subject of your book:
\begin{itemize}
\item The \textit{two} recommended styles for references in books on \textit{mathematical, physical, statistical and computer sciences} are depicted in ~\cite{science-contrib, science-online, science-mono, science-journal, science-DOI} and ~\cite{phys-online, phys-mono, phys-journal, phys-DOI, phys-contrib}.
\item Examples of the most commonly used reference style in books on \textit{Psychology, Social Sciences} are~\cite{psysoc-mono, psysoc-online,psysoc-journal, psysoc-contrib, psysoc-DOI}.
\item Examples for references in books on \textit{Humanities, Linguistics, Philosophy} are~\cite{humlinphil-journal, humlinphil-contrib, humlinphil-mono, humlinphil-online, humlinphil-DOI}.
\item Examples of the basic Springer style used in publications on a wide range of subjects such as \textit{Computer Science, Economics, Engineering, Geosciences, Life Sciences, Medicine, Biomedicine} are ~\cite{basic-contrib, basic-online, basic-journal, basic-DOI, basic-mono}. 
\end{itemize}
}

\begin{thebibliography}{99.}%
% and use \bibitem to create references.
%
% Use the following syntax and markup for your references if 
% the subject of your book is from the field 
% "Mathematics, Physics, Statistics, Computer Science"
%
% Contribution 
\bibitem{science-contrib} Broy, M.: Software engineering --- from auxiliary to key technologies. In: Broy, M., Dener, E. (eds.) Software Pioneers, pp. 10-13. Springer, Heidelberg (2002)
%
% Online Document
\bibitem{science-online} Dod, J.: Effective substances. In: The Dictionary of Substances and Their Effects. Royal Society of Chemistry (1999) Available via DIALOG. \\
\url{http://www.rsc.org/dose/title of subordinate document. Cited 15 Jan 1999}
%
% Monograph
\bibitem{science-mono} Geddes, K.O., Czapor, S.R., Labahn, G.: Algorithms for Computer Algebra. Kluwer, Boston (1992) 
%
% Journal article
\bibitem{science-journal} Hamburger, C.: Quasimonotonicity, regularity and duality for nonlinear systems of partial differential equations. Ann. Mat. Pura. Appl. \textbf{169}, 321--354 (1995)
%
% Journal article by DOI
\bibitem{science-DOI} Slifka, M.K., Whitton, J.L.: Clinical implications of dysregulated cytokine production. J. Mol. Med. (2000) doi: 10.1007/s001090000086 
%
\bigskip

% Use the following (APS) syntax and markup for your references if 
% the subject of your book is from the field 
% "Mathematics, Physics, Statistics, Computer Science"
%
% Online Document
\bibitem{phys-online} J. Dod, in \textit{The Dictionary of Substances and Their Effects}, Royal Society of Chemistry. (Available via DIALOG, 1999), 
\url{http://www.rsc.org/dose/title of subordinate document. Cited 15 Jan 1999}
%
% Monograph
\bibitem{phys-mono} H. Ibach, H. L\"uth, \textit{Solid-State Physics}, 2nd edn. (Springer, New York, 1996), pp. 45-56 
%
% Journal article
\bibitem{phys-journal} S. Preuss, A. Demchuk Jr., M. Stuke, Appl. Phys. A \textbf{61}
%
% Journal article by DOI
\bibitem{phys-DOI} M.K. Slifka, J.L. Whitton, J. Mol. Med., doi: 10.1007/s001090000086
%
% Contribution 
\bibitem{phys-contrib} S.E. Smith, in \textit{Neuromuscular Junction}, ed. by E. Zaimis. Handbook of Experimental Pharmacology, vol 42 (Springer, Heidelberg, 1976), p. 593
%
\bigskip
%
% Use the following syntax and markup for your references if 
% the subject of your book is from the field 
% "Psychology, Social Sciences"
%
%
% Monograph
\bibitem{psysoc-mono} Calfee, R.~C., \& Valencia, R.~R. (1991). \textit{APA guide to preparing manuscripts for journal publication.} Washington, DC: American Psychological Association.
%
% Online Document
\bibitem{psysoc-online} Dod, J. (1999). Effective substances. In: The dictionary of substances and their effects. Royal Society of Chemistry. Available via DIALOG. \\
\url{http://www.rsc.org/dose/Effective substances.} Cited 15 Jan 1999.
%
% Journal article
\bibitem{psysoc-journal} Harris, M., Karper, E., Stacks, G., Hoffman, D., DeNiro, R., Cruz, P., et al. (2001). Writing labs and the Hollywood connection. \textit{J Film} Writing, 44(3), 213--245.
%
% Contribution 
\bibitem{psysoc-contrib} O'Neil, J.~M., \& Egan, J. (1992). Men's and women's gender role journeys: Metaphor for healing, transition, and transformation. In B.~R. Wainrig (Ed.), \textit{Gender issues across the life cycle} (pp. 107--123). New York: Springer.
%
% Journal article by DOI
\bibitem{psysoc-DOI}Kreger, M., Brindis, C.D., Manuel, D.M., Sassoubre, L. (2007). Lessons learned in systems change initiatives: benchmarks and indicators. \textit{American Journal of Community Psychology}, doi: 10.1007/s10464-007-9108-14.
%
%
% Use the following syntax and markup for your references if 
% the subject of your book is from the field 
% "Humanities, Linguistics, Philosophy"
%
\bigskip
%
% Journal article
\bibitem{humlinphil-journal} Alber John, Daniel C. O'Connell, and Sabine Kowal. 2002. Personal perspective in TV interviews. \textit{Pragmatics} 12:257--271
%
% Contribution 
\bibitem{humlinphil-contrib} Cameron, Deborah. 1997. Theoretical debates in feminist linguistics: Questions of sex and gender. In \textit{Gender and discourse}, ed. Ruth Wodak, 99--119. London: Sage Publications.

\end{thebibliography}



\end{document}
