\documentclass[lettersize,journal]{IEEEtran}
\usepackage{amsmath,amsfonts}
\usepackage{mathtools}
\usepackage{algorithmic}
\usepackage{algorithm}
\usepackage{array}
\usepackage[caption=false,font=normalsize,labelfont=sf,textfont=sf]{subfig}
\usepackage{textcomp}
\usepackage{stfloats}
\usepackage{url}
\usepackage{verbatim}
\usepackage{graphicx}
\graphicspath{ {figures/} }
\usepackage{cite}
\usepackage{makecell}
\usepackage{resizegather}
%\usepackage{subfig}
%\usepackage{subcaption}
\hyphenation{op-tical net-works semi-conduc-tor IEEE-Xplore}
%\usepackage{appendix}
\usepackage{xcolor}

\begin{document}

\title{On Dynamic Ray Tracing and Anticipative Channel Prediction for Dynamic Environments}

\author{Denis Bilibashi, Enrico M. Vitucci, and Vittorio Degli-Esposti
        % <-this % stops a space

\thanks{This work was funded in part by the Italian Ministry of University and Research (MUR) through the programme ”Dipartimenti di Eccellenza (2018–2022) — Precision Cyberphysical Systems Project (P-CPS),” and in part by the Eu COST Action INTERACT (Intelligence-Enabling Radio Communications for Seamless Inclusive Interactions) under Grant CA20120.}
\thanks{Denis Bilibashi, Enrico~M.~Vitucci, and V.~Degli-Esposti are with the Department of Electrical, Electronic, and Information Engineering "Guglielmo Marconi" (DEI), CNIT, University of Bologna, 40126 Bologna, Italy (e-mail: denis.bilibashi2, enricomaria.vitucci, v.degliesposti @unibo.it)}
}% <-this % stops a space



%\IEEEpubid{0000--0000/00\$00.00~\copyright~2021 IEEE}
% Remember, if you use this you must call \IEEEpubidadjcol in the second
% column for its text to clear the IEEEpubid mark.

\maketitle

\begin{abstract}
Ray tracing algorithms, that can simulate multipath radio propagation in presence of geometric obstacles such as buildings, objects or vehicles, are becoming quite popular, due to the increasing availability of digital environment databases and  high-performance computation platforms, such as multi-core computers and cloud computing services. When objects or vehicles are moving, which is the case of industrial or vehicular environments, multiple successive representations of the environment ("snapshots") and multiple ray tracing runs are often necessary, which require a great human effort and a great deal of computation resources.
Recently, the Dynamic Ray Tracing (DRT) approach has been proposed to predict the multipath evolution within a given time lapse on the base of the current multipath geometry, assuming constant speeds and/or accelerations for moving objects, using  analytical extrapolation formulas.  This is done without re-running a full ray tracing for every "snapshot" of the environment, therefore with a great computation time saving. When DRT is embedded in a mobile radio system and used in real-time, ahead-of-time (or anticipative) field prediction is possible that opens the way to interesting applications. In the present work, a full-3D DRT algorithm is presented that allows to account for multiple reflections, edge diffraction and diffuse scattering for the general case where moving objects can translate and rotate. For the purpose of validation, the model is first applied to some ideal cases and then to realistic cases where results are compared with conventional ray tracing simulation and measurements available in the literature.
\end{abstract}

\begin{IEEEkeywords}
Ray Tracing, Dynamic ray tracing, Radiowave Propagation, Millimeter wave propagation, Vehicular Ad Hoc Networks, Doppler Effect, 6G Mobile Communications
\end{IEEEkeywords}

\section{Introduction}
\section{Introduction}  \label{sec:introduction}

\newcommand\inexpIntro[3]{#1?(#2,#3).}
\newcommand\rinexpIntro[3]{*#1?(#2,#3).}
\newcommand\outexpIntro[3]{#1!(#2,#3).}
\newcommand\outatomIntro[3]{#1!(#2,#3)}

We propose a fully automated method for proving termination of \(\pi\)-calculus processes.
Although there have been a lot of studies on termination analysis for the \(\pi\)-calculus
and related calculi~\cite{Deng06IC,Demangeon07,SangiorgiTermination,KobayashiHybrid,Yoshida04IC,DBLP:journals/jlp/DemangeonHS10,Venet98SAS}, most of them have been rather theoretical,
and there have been surprisingly little efforts in developing  fully automated termination
verification methods and tools based on them. To our knowledge,
Kobayashi's \typical{}~\cite{TyPiCal,KobayashiHybrid} is the only exception that
can prove termination of \(\pi\)-calculus processes (extended with natural numbers)
fully automatically, but its termination analysis is quite limited (see Section~\ref{sec:relatedwork}).

Our method is based on a reduction to termination analysis for sequential programs:
we translate a \(\pi\)-calculus process \(P\) to a sequential program \(S_P\), so that
if \(S_P\) is terminating, so is \(P\). The reduction allows us to use
powerful, mature methods and tools
for termination analysis of sequential programs~\cite{heizmann2016ultimate,freqterm,DBLP:conf/lics/PodelskiR04,Kuwahara2014Termination,DBLP:journals/cacm/CookPR11}.

The idea of the translation is to convert a chain of communications on replicated input
channels to a chain of recursive function calls of the target sequential program.
Let us consider the following Fibonacci process:
\begin{align*}
    & \rinexpIntro{\fib}{n}{r}
        \ifexp{n<2}{ \soutatom{r}{1} \\ &\quad}
                   { \nuexp{s_1} \nuexp{s_2} (\outatomIntro{\fib}{n-1}{s_1} \PAR \outatomIntro{\fib}{n-2}{s_2} \PAR \sinexp{s_1}{x}\sinexp{s_2}{y}\soutatom{r}{x+y}) \\}
    & \PAR \outatomIntro{\fib}{m}{r}
\end{align*}
Here, the process
$\rinexpIntro{\fib}{n}{r} \ldots$ is a function server that computes the \(n\)-th Fibonacci number
in parallel and returns the result to \(r\),
and $\outatom{\fib}{m}{r}$ sends a request for computing the \(m\)-th Fibonacci number;
those who are not familiar with the syntax of the \(\pi\)-calculus may wish to consult
Section~\ref{sec:targetlanguage} first.
To prove that the process above is terminating for any integer \(m\),
it suffices to show that there is no infinite chain of communications on $\fib$:
\[
    \fib(m,r) \to \fib(m_1,r_1) \to \fib(m_2,r_2) \to \cdots.
\]
We convert the process above to the following program:\footnote{The actual translation
  given later is a little more complex.}
\begin{verbatim}
 let rec fib(n) = if n<2 then () else (fib(n-1) [] fib(n-2)) in
 fib(m)
\end{verbatim}
Here, \texttt{[]} represents the non-deterministic choice.
Note that, although the calculation of Fibonacci numbers is not preserved,
for each chain of communications on \texttt{fib}, there is a corresponding
sequence of recursive calls:
\[
\mathtt{fib}(m) \to \mathtt{fib}(m_1) \to \mathtt{fib}(m_2) \to \cdots.
\]
Thus, the termination of the sequential program above implies the termination of
the original process.
As shown in the example above, (i) each communication on a replicated input channel
is converted to a function call, (ii) each communication on a non-replicated input
channel is just removed (or, in the actual translation, replaced by a call of
a trivial function defined by \(f(\seq{x})=(\,)\)), and (iii) parallel composition
is replaced by a non-deterministic choice.
We formalize the translation outlined above and prove its correctness.

The basic translation sketched above sometimes loses too much information.
For example, consider the following process:
\begin{align*}
    & \rinexpIntro{\pre}{n}{r} \soutatom{r}{n-1} \\
    & \PAR \rinexpIntro{f}{n}{r} \ifexp{n<0}{ \soutatom{r}{1} }
                                       { \nuexp{s} (\outatomIntro{\pre}{n}{s} \PAR \sinexp{s}{x}\outatomIntro{f}{x}{r}) } \\
    & \PAR \outatomIntro{f}{m}{r}
\end{align*}
The translation sketched above would yield:
\begin{verbatim}
  let pred(n) = n-1 in
  let rec f(n) = if n<0 then () else (pred(n) [] f(*)) in
  f(m)
\end{verbatim}
Here, \texttt{*} represents a non-deterministic integer: since we have removed
the input $\sinatom{s}{x}$, we do not have information about the value of \( x \).
As a result, the sequential program above is non-terminating, although the original
process is terminating.
To remedy this problem, we also refine the basic translation above by using a refinement
type system for the \(\pi\)-calculus. Using the refinement type system,
we can infer that the value of \(x\) in the original process is less than \(n\),
so that we can refine the definition of \texttt{f} to:
\begin{verbatim}
 let rec f(n) = ... else (pred(n) [] let x=* in assume(x<n);f(x))
\end{verbatim}
The target program is now terminating, from which
we can deduce that the original process is also terminating.
We have implemented an automated tool based on the refined translation above.

The contributions of this paper are summarized as follows.
\begin{itemize}
\item The formalization of the basic translation from the \(\pi\)-calculus
  (extended with integers) to sequential programs, and a proof of its correctness.
\item The formalization of a refined translation based on a refinement type system.
\item An implementation of the refined translation, including automated refinement type
  inference based on CHC solving, and experiments to evaluate the effectiveness of
  our method.
\end{itemize}

The rest of this paper is structured as follows.
Section~\ref{sec:targetlanguage} introduces the source and target languages
of our translation.
Section~\ref{sec:approach} 
formalizes the basic translation, and proves its correctness.
Section~\ref{sec:refinement} refines the basic translation by using a refinement type system.
Section~\ref{sec:implementation} reports an implementation and experiments.
Section~\ref{sec:relatedwork} discusses related work,
and Section~\ref{sec:conclusion} concludes the paper.


\section{The DRT Concept}
A single multi-bounce ray path can be represented as a polygonal chain where TX, the interaction points (e.g. reflection points or diffraction points) and RX are defined as the vertexes of the chain (see Fig. \ref{ray_path}). The evolution of this chain in a dynamic environment can be described only if we know how its vertexes will move in time. While the motion of the terminals is independent, the motion of the interaction points depends on the motion of the terminals and of objects generating these interaction points as well as on their instantaneous positions. Furthermore, the motion of the interaction points is not constant even if the terminals motion is constant. \par

\begin{figure}[h!]
	\centering
	\includegraphics[width=3in]{figures/multi_bounce_path.png}
	\caption{Representation of single multi-bounce ray path.}
	\label{ray_path}
\end{figure}

The computation of the dynamic evolution of paths can of course be accomplished by multiple RT runs on multiple environment descriptions corresponding to successive "snapshots" of the environment in time. Nonetheless, this technique requires the creation of a very large number of environment databases as well as an equally large number of RT runs making it impractical if an accurate description is required and moreover it will require a huge CPU time. \par
The proposed DRT algorithm is based on the combination of a classical 3D image-based RT approach with an analytical extrapolation of multipath evolution using a proper description of the dynamic environment \cite{bilibashi}, \cite{bilibashi2}. A dynamic environment database is introduced that describes the geometry of the environment also including the motion characteristics of both the radio terminals and the rigid bodies that can move, such as vehicles or machines. The database is divided into two parts. The first one provides a geometrical description of the environment for a given time instant $t_0$. Objects are described here - as in most RT models - as polyhedrons with flat surfaces and right edges, although an extension to curved surfaces is possible. In the second part, the dynamic parameters of every terminal and object are provided. In particular, the objects are modelled as "rigid bodies" with a roto-translational motion, that can be described in a complete way by providing translation velocity, rotation axis, angular velocity and the corresponding accelerations. \par
%
%  put here a figure with an example of a record of the dynamic environment database ?  Only if there is space, we already have many figures...
%
A traditional RT prediction is carried out for the initial time $t_0$, then the positions of all interaction points are extracted. From the initial positions, the motion of each vertex of the chain within $T_C$ must be computed. Our assumption is that the complete motion of the radio terminals is a-priori known. If this is not true, we assume at least that the the terminals' instant velocity and acceleration at $t_0$ is known, so that the positions $\overline{r}_P(t)$ of TX or RX on a time instant $t=t_{0}+\Delta t$ can be calculated using the Taylor series formula:
\begin{equation}
\begin{gathered}
 \overline{r}_P(t_{0}+\Delta t) = \overline{r}_P(t_{0}) + \overline{v}_{P}(t_0) \Delta t +  \\
 \frac{1}{2} \overline{a}_{P}(t_0) \Delta t^2 + O(\Delta t^3)
\end{gathered}
\label{Taylor}
\end{equation}
where $\overline{v}_{P}(t_0)$ and $\overline{a}_{P}(t_0)$ are the velocity and acceleration of P, respectively, at $t_0$, and the symbol $O(\Delta t^3)$ means that all the variations higher than $2^{nd}$ order, i.e. variations of accelerations, are neglected.

Now the problem is that the interaction points' positions, velocities and accelerations is unknown and needs to be determined for every time instant $t_{0}+\Delta t$ on the base of the current positions, velocities and accelerations of Tx, Rx and of the obstacles generating such interactions. For this purpose, we developed closed form formulas as explained in Section III.

The analytical extrapolation procedure is valid as long as the multipath structure remains the same, i.e., no path disappears and no new path appears to significantly change such structure. This time interval will be referred to in the following as \textit{multipath~lifetime}~$(T_C)$. As long as $\Delta t < T_C$ , DRT allows to extrapolate the multipath - and therefore the total field, the time-variant channel's transfer function, Doppler's shift frequencies, time delays, etc. - for every time instant $t_{0}+\Delta t$ without re-running the RT engine, and therefore at only a fraction of the computation time. In fact, the computationally most expensive part of a RT algorithm consists in determining the geometry of the rays, which requires checking the visibility and obstructions between all objects in the database, in order to establish the existence of each of the traced rays. With the DRT approach, this is done only once at time $t_0$, while in the subsequent time instants within $T_C$ we rely on analytical prolongation of the same rays, which is computationally much faster.

Also, we believe that it is a reasonable assumption to neglect the temporal variations of acceleration during the time $T_C$, in accordance with eq. (\ref{Taylor}): in fact, in vehicular scenarios the multipath structure usually varies on a faster time scale than acceleration, as discussed in Section IV.

Since DRT naturally accounts for speeds and accelerations, another advantage is that dynamic channel parameters, such as Doppler information, are derived in closed form, without resorting to finite-difference computation. A more detailed explanation about Doppler frequency calculation is described in Appendix A.

\section{DRT Algorithm Description}
In this Section, the DRT algorithm formulation is described for single and multiple bounce rays, including specular reflection and edge diffraction, based on the outcome of a single RT simulation at the initial time $t_0$, and on the knowledge of the dynamic parameters of TX/RX, and of the objects in the propagation environment. \par 


\subsection{Reflection Points' Calculation}
In this subsection, the DRT algorithm is initially explained for the single reflection case. Starting from this, further extensions of DRT to double reflection and multiple reflections, are presented in the following subsection. 

\subsubsection{Single Reflection Procedure}
A simple case is considered with a reflecting wall laying on a plane $\Pi^I$, and transmitter (TX) and receiver (RX) located at different distances from $\Pi^I$. As a start, we consider the case where TX and RX move with instantaneous speeds $\overline{v}_{TX}$ and $\overline{v}_{RX}$, while the reflecting wall is at rest. Without loss of generality, we can assume a proper reference system so that the wall lies on the plane of equation $y=0$, while TX and RX lie on the xy plane. Then, by using the image method and through simple geometric considerations we can derive the reflection point position ($Q_R$) at the considered time instant, which is located at the intersection between the reflecting plane and the line passing through RX and the image of TX:

\begin{equation}
\begin{gathered}
\begin{cases}
x_{Q_{R}}(t) = x_{TX}(t) + \frac{x_{RX}(t)-x_{TX}(t)}{y_{RX}(t)+y_{TX}(t)}~y_{TX}(t) \\
y_{Q_{R}}(t) = 0 \\
z_{Q_{R}}(t) = z_{TX}(t) + \frac{z_{RX}(t)-z_{TX}(t)}{y_{RX}(t)+y_{TX}(t)}~y_{TX}(t)
\end{cases}
\end{gathered}
\label{ref_point}
\end{equation}
When TX/RX move with a certain speed, the corresponding reflection point "slides" along the plane surface: therefore, by deriving $Q_R$ coordinates with respect to time, the instantaneous velocity of the reflection point ($\overline{v}_{Q_{R}}=v_{Q_{R,x}}\hat{x}+v_{Q_{R,z}}\hat{z}$) can be determined, by using the derivative chain rule. 

For example, the x-component of $\overline{v}_{Q_{R}}$ is:
\begin{equation}
\begin{gathered}
v_{Q_{R,x}} = \frac{\partial x_{Q_R}}{\partial t} = \frac{\partial x_{Q_R}}{\partial x_{TX}}~v_{TX,x} + \frac{\partial x_{Q_R}}{\partial y_{TX}}~v_{TX,y}  \\ + \frac{\partial x_{Q_R}}{\partial x_{RX}}~v_{RX,x} + \frac{\partial x_{Q_R}}{\partial y_{RX}}~v_{RX,y}.
\end{gathered}
\label{ref_point_velo}
\end{equation}
The closed-form expressions of the derivatives in eq. (\ref{ref_point_velo}) are reported in Appendix B.


This approach is similar to the one adopted in \cite{qua}, but in the following it will be extended to a more general case, where the reflecting wall has a roto-translational motion, and TX, RX, and the reflecting wall can vary their velocities during time, so their motion is accelerated.
The basic idea is to use a local reference system integral with the wall (local frame) so that we can resort to the previous case where the wall is at rest and apply equations (\ref{ref_point}) and (\ref{ref_point_velo}) again, as shown below. \par

%When TX/RX and the reflecting plane move with a certain speed, the corresponding reflection point "slides" along the plane surface:% 
Moreover, in the present work we make a complete characterization of the reflection point motion, including acceleration: in fact, except very particular cases, $Q_R$ velocity is not constant in time, i.e. the reflection point has an accelerated motion. Therefore, with a similar method as in (\ref{ref_point_velo}), the acceleration of the reflection point ($\overline{a}_{Q_{R}}=a_{Q_{R,x}}\hat{x}+a_{Q_{R,z}}\hat{z}$) can be calculated by deriving $\overline{v}_{Q_{R}}$ with respect to time. For instance, the x-component of the acceleration will be:  
\begin{equation}
a_{Q_{R,x}} = \frac{\partial v_{Q_{R,x}}}{\partial t} = \frac{\partial}{\partial t} \frac{\partial x_{Q_{R}}}{\partial t} 
\label{ref_point_acc}
\end{equation}
The complete expression of $\overline{a}_{Q_{R}}$, and the detailed computation of $\overline{v}_{Q_{R}}$ and $\overline{a}_{Q_{R}}$ are presented in Appendix B. \par 

In Fig. \ref{scenario_before_after}, an example of a single reflection scenario is presented. For the sake of simplicity, we refer to a bi-dimensional case where TX/RX are located in a horizontal plane $z=z_0$ and moving with arbitrary speeds, meanwhile the reflecting wall is located along a vertical plane $y=y_0$, and is then represented with a straight line. Moreover, the wall rotates around a vertical axis, and the intersection of the rotation axis with the plane $z=z_0$ provides the rotation center $O^I$. The local reference system $O^Ix^Iy^Iz^I$ centered in $O^I$ (local frame), is also represented in the figure. The axes of the local frame are oriented so that at each time instant the reflecting wall is laying on the $z^Ix^I$ plane of equation $y=0$, in order to apply eq. (\ref{ref_point}).
The scenario represented in the figure is simplified for ease of readability, but the presented method is general, so the TX/RX positions and velocities, the wall plane and the rotation axis can be arbitrarily oriented with respect to the global reference system $Oxyz$.

\begin{figure}[!ht]
	\centering
	\includegraphics[width=3.5in]{scenario_t0_and_t0+delta}
	\caption{Example of a single reflection scenario with the plane $\Pi^I$ that rotates w.r.t. to the global frame. a) Scenario at $t=t_0$ and instantaneous velocities of TX, image-TX, RX, and reflection point $Q_R$. b) Scenario at $t=t_0+\Delta t$.}
	\label{scenario_before_after}
\end{figure}

Fig. \ref{scenario_before_after}a) shows the configuration of the system at a certain time $t=t_0$, when the wall plane $\Pi^{I}$ is parallel to the plane $xz$ of the global frame $Oxyz$. The instantaneous velocities of TX ($\overline{v}_{TX}$), image-TX ($\overline{v}_{TX'}$), reflection point ($\overline{v}_{Q_R}$) and RX ($\overline{v}_{RX}$) at $t=t_0$ are depicted as well. The instantaneous angular velocity of the wall plane is expressed by the vector $\overline{\omega}=\omega \hat{k}$, where $\omega=\frac{d \alpha}{dt}$ is the scalar angular velocity, and $\hat{k}$ is a unit vector parallel to the rotation axis, and properly oriented according to the right-hand rule. In the example of Fig. \ref{scenario_before_after}, we have $\hat{k}=-\hat{z}^I$, as the wall is rotating clockwise around the z-axis of the local frame.

Fig. \ref{scenario_before_after}b) shows the configuration of the system in a subsequent time instant $t=t_0+\Delta t$, when the wall plane has rotated clockwise by an angle $\theta$, and TX/RX have moved to different positions: the result of this motion is a shift of the reflection point $Q_R$ along the reflecting wall.  

In Fig. \ref{scenario_before_after} a wall rotating around a fixed rotation axis is shown, but in general, a translational motion of the wall plane can be also present, in addition to the rotational motion.
In the general case of roto-translational motion, any point Q of the wall will have a different speed, given by \cite{Martin1968}:
\begin{equation}
\overline{v}_Q=\overline{v}_{\Pi^I}+\overline{\omega} \times \overline{O^IQ}
\label{RigidBodyMotion}
\end{equation}
where the symbol "$\times$" stands for the cross vector product, $\overline{v}_{\Pi^I}$ is the translation velocity, common to all the points of the plane, and $\overline{O^IQ}$ is the position vector of the considered point Q w.r.t. the origin $O^I$ of the local frame, positioned on the (instantaneous) rotation axis.
\par

The first step of the DRT procedure consists in the computation of the (instantaneous) position of the reflecting point $Q_R$. Thanks to the adoption of the local frame, this can be accomplished through eq. (\ref{ref_point}), but in order to do that, we need to transform the coordinates of TX and RX into the local frame associated with the wall. To this end, we observe that the positions of TX/RX with respect to the global and local frames are related each other through the following \textit{coordinate transformation} \cite{Martin1968}:
\begin{equation}
\begin{gathered}
    \overline{r}_{TX}^0(t) = \Bar{\Bar{R}}(t) \cdot \overline{r}_{TX}^{I}(t) + \overline{r}_{0^I}(t) \\
    \overline{r}_{RX}^0(t) = \Bar{\Bar{R}}(t) \cdot \overline{r}_{RX}^{I}(t) + \overline{r}_{0^I}(t)
\end{gathered}
\label{coordinate_transf}
\end{equation}
with
\begin{center}
\begin{small}
$\overline{r}_{TX}^0(t)=[x_{TX}(t)~y_{TX}(t)~z_{TX}(t)]^T$ \\
$\overline{r}_{RX}^0(t)=[x_{RX}(t)~y_{RX}(t)~z_{RX}(t)]^T$
\end{small}
\end{center} 
and 
\begin{center}
\begin{small}
$\overline{r}_{TX}^I(t)=[x_{TX}^I(t)~y_{TX}^I(t)~z_{TX}^I(t)]^T$ \\
$\overline{r}_{RX}^I(t)=[x_{RX}^I(t)~y_{RX}^I(t)~z_{RX}^I(t)]^T$ 
\end{small} 
\end{center}
being the position vectors of TX/RX w.r.t. the global and local frame, respectively, where $\Bar{\Bar{R}}(t)$ is the (instantaneous) rotation matrix and $\overline{r}_{0^I}(t)=[x_{O^I}(t)~y_{O^I}(t)~z_{O^I}(t)]^T$ is the position vector associated with the origin point $O^{I}$ of the local frame. By inverting eq. (\ref{coordinate_transf}), we can determine the positions $\overline{r}_{TX}^I$, $\overline{r}_{RX}^I$ of TX and RX w.r.t. to the local frame, and then apply eq. (\ref{ref_point}) to find the coordinates of $Q_R$.

In the simple example of Fig. \ref{scenario_before_after} (clockwise rotation around the z-axis), the rotation matrix at the time instant $t=t_0+\Delta t$ is given by:
\begin{equation*}
\Bar{\Bar{R}}(t_0+\Delta t)=
\begin{pmatrix}
cos\theta & sin\theta & 0\\
-sin\theta & cos\theta & 0\\
0 & 0 & 1
\end{pmatrix}
\end{equation*}
In a generic time instant $t=t_0+\Delta t$ within the multipath lifetime $T_C$, the instantaneous rotation angle $\theta(t)$ can be obtained in the following way, similarly to eq.(\ref{Taylor}) and neglecting time variations of angular acceleration:
\begin{equation*}
\theta(t)=\theta(t_0)+\omega\Delta t+\frac{1}{2}\frac{d\omega}{dt}\Delta t^2
\end{equation*}
Of course, \textit{coordinate transformation} must be applied to the components of $\overline{v}_{TX}$, $\overline{v}_{RX}$ and $\overline{a}_{TX}$, $\overline{a}_{RX}$ as well.

The next step of the DRT procedure consists in the computation of the velocity and acceleration of $Q_R$ which can be obtained through eq. (\ref{ref_point_velo}) and (\ref{ref_point_acc}). In fact, one of the main advantages of the DRT approach is that we can derive $\overline{v}_{Q_R}$ and $\overline{a}_{Q_R}$ analytically, without need to resort to time differences methods. However, projecting the components of $\overline{v}_{TX}$, $\overline{v}_{RX}$ and $\overline{a}_{TX}$, $\overline{a}_{RX}$ on the local frame is not enough to apply eq. (\ref{ref_point_velo}) and (\ref{ref_point_acc}), which have been obtained from eq. (\ref{ref_point}), i.e. assuming that the reflecting wall is at rest.  Since now the reflecting wall is moving, and then the local frame is in motion w.r.t. the global reference system, the velocities and accelerations of TX/RX must be preliminary transformed according to the \textit{relative motion transformations}, to obtain "relative" velocities and accelerations w.r.t. an observer located in the origin of the local frame \cite{taylor2005}. For instance, the velocity and acceleration of TX must be transformed in the following way: 
\begin{gather}
\overline{v}_{TX}^{I} = \overline{v}_{TX}^{0} - \overline{v}_{\Pi^{I}} - \overline{\omega} \times \overline{r}_{TX}^{I} \label{relative_vel} \\
\overline{a}_{TX}^{I} = \overline{a}_{TX}^{0} - \overline{a}_{\Pi^{I}} - \dot{\overline{\omega}} \times \overline{r}_{TX}^{I} - 2 \overline{\omega} \times \overline{v}_{TX}^{I} \label{relative_acc} \\ \nonumber
-\overline{\omega} \times (\overline{\omega} \times \overline{r}_{TX}^{I})
\end{gather}
where the headers $"I"$ and $"0"$ are associated with the velocities and accelerations seen by an observer located in the origin of the local and global frame, respectively, $\overline{r}_{TX}^{I}=\overline{r}_{TX}^{0}-\overline{r}_{O^{I}}$ is the position vector of TX in the local frame, $\overline{v}_{\Pi^{I}}$, $\overline{\omega}$ are the translation and angular velocities of the wall plane $\Pi^{I}$, and $\overline{a}_{\Pi^{I}}$, $\dot{\overline{\omega}}=\frac{d\overline{\omega}}{dt}$ are the corresponding translation and angular accelerations, respectively.
The terms $\dot{\overline{\omega}} \times \overline{r}_{TX}^{I}$, $2 \overline{\omega} \times \overline{v}_{TX}^{I}$ and $\overline{\omega} \times (\overline{\omega} \times \overline{r}_{TX}^{I})$ in eq. (\ref{relative_acc}) are also known as Euler's, Coriolis', and centrifugal acceleration, respectively.

It is worth noting that the accelerations $\overline{a}_{\Pi^{I}}$, $\dot{\overline{\omega}}$ are assumed to be constant in the time interval $[t_0~t_0+T_C]$, while $\overline{v}_{\Pi^{I}}$, $\overline{\omega}$ in eq. (\ref{relative_vel}) are \textit{instantaneous} velocities, computed as:
\begin{equation*}
\begin{gathered}
\overline{v}_{\Pi^{I}}(t)=\overline{v}_{\Pi^{I}}(t_0)+\overline{a}_{\Pi^{I}}\Delta t \\
\overline{\omega}(t)=\overline{\omega}(t_0)+\dot{\overline{\omega}} \Delta t
\end{gathered}
\end{equation*}

In practice, with the \textit{relative motion transformations} (\ref{relative_vel}) and (\ref{relative_acc}) we turn the original problem into an equivalent problem, where the wall plane is at rest and the TX/RX velocities and accelerations are modified, according to the point of view of an observer located in the origin of the local frame. It is worth noting that, even in the simple case of TX/RX moving with constant speed, TX and RX are accelerated in the equivalent problem: this acceleration is caused by the angular rotation ${\overline{\omega}}$ of the wall plane, according to eq. (\ref{relative_vel}) and (\ref{relative_acc}).
\par

Once the reflection point position, velocity and acceleration have been determined in the local frame using eq. (\ref{ref_point}), (\ref{ref_point_velo}), and (\ref{ref_point_acc}), the following inverse transformations need to be applied: 
\begin{itemize}
	\item back-transformation to get $Q_R$ coordinates, and $\overline{v}_{Q_{R}}$ and $\overline{a}_{Q_{R}}$ components w.r.t. the global reference system (inverse \textit{coordinate  transformation}).
	\item back-transformation to get the velocity and acceleration ($\overline{v}_{Q_{R}}$ and $\overline{a}_{Q_{R}}$) relative to an observer located in the origin of the global reference system (inverse \textit{relative motion transformation}):
	\begin{gather}
	    \overline{v}_{Q_R}^{0} = \overline{v}_{Q_R}^{I} + \overline{v}_{\Pi^{I}} + \overline{\omega} \times \overline{r}_{Q_R}^{I} \label{global_vel_ref_point} \\
        \overline{a}_{Q_R}^{0} = \overline{a}_{Q_R}^{I} + \overline{a}_{\Pi^{I}} + \dot{\overline{\omega}} \times \overline{r}_{Q_R}^{I} + 2 \overline{\omega} \times \overline{v}_{Q_R}^{I} \label{global_acc_ref_point} \\ \nonumber
        +\overline{\omega} \times (\overline{\omega} \times \overline{r}_{Q_R}^{I})   
	\end{gather}
\end{itemize}  


\subsubsection{Generalization to Multiple Reflections}
In the case of multiple reflections, the motion of a certain reflection point is influenced by the motion of the previous or latter reflection points.  
However, the method discussed above can be extended in a straightforward way to a multiple-bounce case. 

Let's consider for simplicity a double reflection case: instead of using TX, we can resort to the use of the image-TX ($TX^{'}$) with respect to the first wall plane, and after that we can analyze the reflection on the second wall: in practice, we replace $TX$ with $TX^{'}$ and we bring the computation back to the single-reflection scenario analyzed in the previous section. This means that in a case with two reflecting walls, $TX^{'}$ and $RX$ are used to compute the motion of the reflection point along the second reflecting wall ($Q_{R2}$), by applying the same procedure as for the single-bounce scenario. Then, once $Q_{R2}$ is known, it is used as a new virtual receiver to compute, in addition to the TX location, the position and the motion of the reflection point along the first reflecting wall ($Q_{R1}$). 

This approach can be iterated in a similar way for the case of more than 2 reflections. In practice, we compute the image-TX ($TX^{'}$), the image of the image ($TX^{''}$), etc., until we reach the last reflecting wall, then a "back-tracking" procedure is used: we start from RX and the last reflecting wall, we apply all the equations of the previous section to compute the last reflection point, and then we move back towards TX, to trace the motion of all the remaining reflection points.  

In order to apply all the equations of the previous section for the determination of the reflection points motion, we need to apply all the necessary transformations from the global to the local frame, and vice-versa, for each of the reflecting walls.

Moreover, a preliminary computation is needed to compute the position and the instantaneous velocity for each of the image-transmitters. This can be done in a straightforward way by relying on the local frame, and using the image principle. For example, the image-TX with respect to the first wall ($TX^{'}$), will have the same $x^I$ and $z^I$ coordinates as TX, and opposite $y^I$ coordinate. Similarly, the velocity components will be:

\begin{equation}
\begin{gathered}
\begin{cases}
v_{TX^{'},x}^{I} = v_{TX,x}^{I} \\
v_{TX^{'},y}^{I} = -v_{TX,y}^{I} \\
v_{TX^{'},z}^{I} = v_{TX,z}^{I}
\end{cases}
\end{gathered} 
\end{equation}

Once $\overline{v}_{TX^{'}}$ is computed in the local frame, it can be expressed w.r.t. to an observer located in the origin of the global frame, according to the relative motion transformation:
\begin{equation}
\overline{v}_{TX^{'}}^{0}=\overline{v}_{TX^{'}}^{I}+\overline{\omega} \times \overline{r}_{TX^{'}}^{I}
\end{equation}
In a similar way, the acceleration of the image-TX, $\overline{a}_{TX^{'}}$, can be also found.
The DRT algorithm then proceeds according to steps described above, to find the coordinates, velocities and accelerations for each of the reflection points.
\par

The whole DRT algorithm is summarized by the flowchart depicted in Fig. \ref{flowchart}, with reference to a double-bounce case, for the sake of simplicity. 

Once the geometric part is done, i.e. the analytical prolongation of all the rays in the considered time instants within $T_C$ has been completed, the very last step of DRT consists in the re-computation of the field associated to each ray. This is done in a straightforward way as the geometry of the rays is known, by applying the Fresnel's reflection coefficients and the ray divergence factor, as usually done in standard RT algorithms \cite{fuschini2015,vitucci2019}. It is worth noting however, that ray's field computation is based on analytical formulas and is therefore orders of magnitude faster than ray's geometry computation \cite{fuschini2015}.

\begin{figure}[h!]
	\centering
	\includegraphics[width=2.5in]{flowchart_2RFL}
	\caption{Flowchart showing the DRT algorithm for a double reflection case.}
	\label{flowchart}
\end{figure}

\subsection{Diffraction Points' Calculation}
The DRT algorithm can be further extended to diffraction, modeled with a ray-based approach according the Uniform Theory of Diffraction (UTD) \cite{UTD}.
A method to track the motion of diffraction points in an analytical way, as previously done for the reflection points, is outlined in this section. We present only the single-diffraction case for the sake of brevity, but the procedure can be extended in a straightforward way to multiple diffractions, as well as to combinations of multiple reflections and diffractions.

In Fig. \ref{diffraction} diffraction from an edge formed by two adjacent walls is illustrated, where the unit edge vector $\hat{e}$ is chosen to be aligned with the $z$-axis of the reference system $Oxyz$, with no loss of generality. For the sake of simplicity and with no limitation - as the diffracted rays lay on the Keller's cone and share the same geometric properties \cite{Keller} - we represent in Fig. \ref{diffraction} an "unfolded" diffracted ray, i.e. the diffraction plane has been rotated to be coincident with the incidence plane. 

\begin{figure}[!ht]
	\centering
	\includegraphics[width=3.2in]{diffraction_scenario}
	\caption{Example of edge diffraction, and related geometry for the computation of the diffraction point.}
	\label{diffraction}
\end{figure}

Since the diffraction point ($Q_D$) is constrained to move along the edge, the $x$ and $y$ coordinates of $Q_D$ are known, then only $z_{Q_D}$ remains to be computed: this can be done in a simple way by using the similar triangles properties.  \par
Looking at Fig. \ref{diffraction}, we see that two similar triangles are formed. The sides of these triangles are proportional each other, so the following relation holds: 
\begin{equation}
z_{TX}-z_{RX} : d_{TX}+d_{RX} = z_{Q_D}-z_{RX} : d_{RX}
\end{equation} 
where 
\begin{equation}
\begin{gathered}
d_{TX}(t) = \sqrt{(x_{TX}(t)-x_{Q_D})^2+(y_{TX}(t)-y_{Q_D})^2} \\[1ex]
d_{RX}(t) = \sqrt{(x_{RX}(t)-x_{Q_D})^2+(y_{RX}(t)-y_{Q_D})^2}
\end{gathered}
\label{dist2D_TXRX}
\end{equation}
are the 2D distances of TX and RX from the edge, respectively. Hence, the z-coordinate of $Q_D$ is given by: 
\begin{equation}
z_{Q_D}(t) = z_{RX}(t) + \frac{d_{RX}(t)\cdot[z_{TX}(t)-z_{RX}(t)]}{d_{TX}(t)+d_{RX}(t)}
\label{diff_point}
\end{equation}
The instantaneous velocity of $Q_D$ ($\overline{v}_{Q_D}=v_{Q_{D,z}}\hat{z}$) can be computed by time deriving $z_{Q_D}$. The detailed calculation of $v_{Q_{D,z}}$ is presented in Appendix C.\par
Similarly, by further deriving $\overline{v}_{Q_D}$, we can find the acceleration of the diffraction point, $\overline{a}_{Q_D}=a_{Q_{D,z}}\hat{z}$, not reported here for the sake of brevity.\par
The expression in (\ref{diff_point}) and the related velocity and acceleration $\overline{v}_{Q_D}$, $\overline{a}_{Q_D}$ are valid and do not require any further computation in case the terminals are moving but the edge is at rest. However, in general an edge might be part of a moving object and then might be moving with a certain roto-translational velocity. In particular, we assume that the edge has a rotational motion with instantaneous angular velocity $\overline{\omega}$, and is also translating according to the instantaneous velocity $\overline{v}_{edge}$. In the example of Fig. \ref{diffraction}, the edge is rotating clockwise around the x-axis. 

As for reflections, we can compute the instantaneous position and the motion of the diffraction point if we assume a proper local reference $O^Ix^Iy^Iz^I$, with the origin located in the rotation center $O^I$, and the z-axis parallel to the edge. By doing so, the same procedure used for reflections, can be followed to transform velocities and accelerations, and finally find $z_{Q_D}$ as well as $v_{Q_{D,z}}$ and $a_{Q_{D,z}}$.  \par 
The final step of DRT is, as usual, the computation of the updated UTD coefficients, and then, of the total diffracted field, at the considered time instant.

\subsection{Diffuse scattering}
Diffuse scattering is modeled according to the Effective Roughness approach \cite{vitucci2019}, which is based on a subdivision of each surface into tiles and on the application of a virtual scattering source to the centroid of each tile. Therefore, the  calculation of scattering points' position and speed is straightforward, as it boils down to the application of basic kinematic equations for the motion of rigid bodies' surface points. For instance, equation (\ref{RigidBodyMotion}) can be used to compute each scattering point's velocity if the rototranslation speed of the body is known.



\section{Results}
\begin{table}[t!]
\centering
\caption{Voice conversion \& F0 manipulation results. MOS results are reported with 95\% confidence interval. VDE, and FFE are reported for F0 manipulation while PER, WER, EER, and MOS are reported for voice conversion. Notice, for VDE, and FFE higher is the better since F0 was flattened.}
\label{tab:conv}

\resizebox{1\columnwidth}{!}{
\begin{tabular}{c@{~} | c@{~} | c@{~}c@{~} | c@{~} | c@{~} ||  c@{~}c@{~} }
\toprule
\multirow{2}{*}{Dataset} & \multirow{2}{*}{Method} & \multicolumn{4}{c||}{Voice Conversion} & \multicolumn{2}{c}{F0 Manipulation} \\
\cmidrule{3-8}
& & PER~$\downarrow$ & WER~$\downarrow$ & EER~$\downarrow$ & MOS~$\uparrow$ & VDE~$\uparrow$ & FFE~$\uparrow$ \\
\midrule
VCTK & GT  & 17.16 & 4.32 & 3.25 & 4.11$\pm$0.29 & -- & -- \\
\midrule 
\multirow{3}{*}{LJ}
% & ASR-TTS   & 50.74  & --     & 66.08 & 32.96 & 1.46 \\
& CPC       & 22.22 	& 16.11 		& 0.46 		& 3.57$\pm$0.15 		& \bf 46.68 & \bf 48.71\\
& HuBERT    & \bf 19.09 & \bf 12.23 & \bf 0.31  & \bf 3.71$\pm$0.24 & 39.20 		& 48.42\\
& VQ-VAE    & 40.88 	& 36.96 		& 9.65 		& 2.90$\pm$0.17 		& 10.54 	& 12.08 \\
\midrule 
\multirow{3}{*}{VCTK} 
% & ASR-TTS   & 68.88  & --    & 41.77 & 13.55 & 6.48 \\
& CPC       &  23.58 		& 15.98 		& \bf 4.83  &  3.42 $\pm$ 0.24 		& \bf 25.29 & \bf 26.97 \\
& HuBERT    &  \bf 20.85 	& \bf 12.72 & 6.01  		& \bf  3.58 $\pm$ 0.28 	& 23.46 	& 26.67 \\
& VQ-VAE    & 36.88  		& 29.44 		& 11.56 		& 3.08 $\pm$ 0.34 		& 7.03  	& 7.80  \\
\bottomrule
\end{tabular}}
\vspace{-0.4cm}
\end{table}

\vspace{-0.1cm}
\section{Results}
\vspace{-0.1cm}
Our results cover
% We report results for 
three different settings: (i) speech reconstruction experiments; (ii) speaker conversion and F0 manipulation; (iii) bitrate analysis with subjective tests for speech codec evaluation. We employ two datasets: LJ~\cite{ljspeech17} single speaker dataset and VCTK~\cite{vctk} multi-speaker dataset. All datasets were resampled to a 16kHz sample rate.

% \paragraph*{Implementation Details.}
% \smallskip
\noindent{\bf Implementation Details\quad} 
\label{sec:impl}
We follow the same setup as in~\cite{lakhotia2021generative}. For CPC, we used the model from~\cite{Riviere2020}, which was trained on a ``clean'' 6k hour sub-sample of the LibriLight dataset~\cite{Kahn2020,Riviere2020}. We extract a downsampled representation from an intermediate layer with a 256-dimensional embedding and a hop size of 160 audio samples. For HuBERT we used a \textsc{Base} 12 transformer-layer model trained for two iterations~\cite{hsu2020hubert} on 960 hours of LibriSpeech corpus~\cite{Panayotov2015}. 
% This model encodes every 320 raw audio samples into a 768-dimensional vector. 
This model downsamples the raw audio $\times320$ into a sequence of 768-dimensional vectors. Similarly to~\cite{lakhotia2021generative}, activations were extracted from the sixth layer.

%CPC: We use a dictionary of 100 units, leading to a bitrate of 700bps.
%HuBERT: A dictionary of 100 units is used, leading to a bitrate of 350bps. 
%VQVE: The VQ-VAE discrete code operates at a bitrate of 800bps.
% For both CPC and HuBERT, the k-means algorithm is applied to convert continuous frames to discrete codes, using the LibriSpeech clean-100h~\cite{Panayotov2015} dataset. 
For CPC and HuBERT, the k-means algorithm is trained on LibriSpeech clean-100h~\cite{Panayotov2015} dataset to convert continuous frames to discrete codes. We quantize both learned representations with $K=100$ centroids. Leading to a bitrate of 700bps for CPC and 350bps for HuBERT.

% VQ-VAE
Similarly to CPC models, we trained the VQ-VAE content encoder model on the ``clean'' 6K hours subset from the LibriLight dataset. We use an encoder operating on the raw signal to extract discrete units, similar to~\cite{jukebox}. In addition, ``random restarts'' were performed when the mean usage of a codebook vector fell below a predetermined threshold. Finally, we used HiFiGAN (architecture and objective) as the decoder instead of a simple convolutional decoder, as it improved the overall audio quality. This model encodes the raw audio into a sequence of discrete tokens from 256 possible tokens~\cite{garbacea2019low} with a hop size of 160 raw audio samples. The VQ-VAE discrete code operates at a bitrate of 800bps. We additionally experimented with 100 discrete units for VQ-VAE, however results were the best for 256. This finding is consistent with~\cite{garbacea2019low}.

% verification model
The speaker verification network uses the architecture proposed in~\cite{heigold2016end}. It was trained on the VoxCeleb2~\cite{voxceleb2} dataset, achieving a 7.4\% Equal Error Rate (EER) for speaker verification on the test split of the VoxCeleb1~\cite{Nagrani17} dataset.

% pitch
Only a single F0 representation is considered across all evaluated models, trained on the VCTK dataset.
% The F0 is extracted from the raw audio using YAAPT~\cite{yaapt} algorithm, using a window size of 20ms and a 5ms hop. 
The F0 is extracted from the raw audio using a window size of 20ms and a 5ms hop. 
As a result, the F0 sequence is sampled at 200Hz. 
% We apply the quantization described at Sec.~\ref{sec:method}, using a pitch codebook of $K'=20$ tokens and an encoder that downsamples the pitch by $\times16$. 
The quantization described at Sec.~\ref{sec:method}, is applied using an F0 codebook of $K'=20$ tokens and an encoder that downsamples the signal by $\times16$. Hence, the discrete F0 representation is sampled at 12.5Hz, leading to a bitrate of 65bps. The final bitrate of the evaluated codecs is the sum of the pitch code bitrate with the content code bitrate.

% \paragraph*{Evaluation Metrics}
% \smallskip
\noindent{\bf Evaluation Metrics\quad} 
We consider both subjective and objective evaluation metrics. For subjective tests, we report the Mean Opinion Scores (MOS). In which human evaluators rate the naturalness of audio samples on a scale of 1--5. Each experiment, included 50 randomly selected samples rated by 30 raters. For objective evaluation, we consider: (i) Equal Error Rate~(EER) as an automatic speaker verification metric obtained using a pre-trained speaker verification network. We report EER between test utterances and enrolled speakers; (ii) Voicing Decision Error (VDE)~\cite{nakatani2008method}, which measures the portion of frames with voicing decision error; (iii) F0 Frame Error (FFE)~\cite{chu2009reducing}, measures the percentage of frames that contain a deviation of more than 20\% in pitch value or have a voicing decision error; (iv) Word Error Rate (WER) and Phoneme Error Rate (PER), proxy metrics to the intelligibility of the generated audio. We used a pre-trained ASR network~\cite{baevski2020wav2vec} on both reconstructed and converted samples to calculate both metrics. %To generate target phonemes, the g2p-en~\cite{g2pE2019} Grapheme2Phoneme module was used.

% \vspace{-0.1cm}
% \smallskip
\noindent{\bf Reconstruction \& Conversion}
% \vspace{-0.1cm}
We start by reporting the reconstruction performance. Results are summarized in Table~\ref{tab:recon}. When considering the intelligibility of the reconstructed signal HuBERT reaches the lowest PER and WER scores across all models, where both CPC and HuBERT are superior to VQ-VAE. However, when considering F0 reconstruction VQ-VAE outperforms both HuBERT and CPC by a significant margin. This results are somewhat intuitive, bearing in mind VQ-VAE objective is to fully reconstruct the input signal. In terms of subjective evaluation, all models reach similar MOS scores, with one exception of CPC on LJ. 

%Notice, since the same F0 units are used for each method, this result implies the VQ-VAE units contain some information about the F0 of the signal, enabling better reconstruction. Regarding speaker information, the CPC gets the lowest EER. 

To better evaluate the disentanglement properties of each method with respect to speaker identity and F0, we conducted an additional set of experiments aiming at speaker conversion and F0 manipulation. For voice conversion, we converted each test utterance into five random target speakers. Next, we employed a speaker verification network, which extracts \emph{d-vector} representation to evaluate speaker-converted utterances' similarity to real speaker utterances (low error-rate indicates good conversion), providing measurement to the speaker identity's disentanglement from the evaluated coding method. The error-rate is reported between converted test utterances and enrolled speakers. For the LJ speech single speaker dataset, we converted samples from the VCTK dataset to the single speaker and enrolled all VCTK speakers together with the single speaker. Results are summarized in Table~\ref{tab:conv} (left). Unlike resynthesis results, on voice conversion CPC and HuBERT outperform VQ-VAE on both LJ and VCTK datasets, indicating VQ-VAE contains more information about the speaker in the encoded units, hence producing more artifacts. Notice, this also affects WER, PER, and the overall subjective quality (MOS). 

Next, to evaluate the presence of F0 in the discrete units, we flattened the F0 units before synthesizing the signal and calculated VDE and FFE with respect to the original F0 values. F0 flattening was done by setting the speakers' mean F0 value across all voiced frames. In this experiment, we expected units that contain F0 information to be better at F0 reconstruction over disentangled units. Results are summarized in Table~\ref{tab:conv} (right). Notice VQ-VAE can still reconstruct the F0 almost at the same level as when using the original F0 as conditioning (5.2 vs 7.03, and 5.59 vs 7.8), in contrast to CPC and HuBERT.

\begin{figure}[t!]
\centering
\includegraphics[width=0.65\columnwidth, trim={50 20 70 20}]{figures/codec_2.pdf}
% \caption{MUSHRA subjective listening test results as a function of bitrate per second for various methods. Purple dots denote the baseline methods, and green dots the proposed SSL based method.} 
\caption{MUSHRA subjective quality results as a function of bitrate per second. Purple dots denote the baseline methods, and green dots the proposed SSL based method.} 
\label{fig:codec}
\vspace{-0.5cm}
\end{figure}

% \vspace{-0.1cm}
% \smallskip
\noindent{\bf Speech Codec}
Our final experiment evaluates the obtained speech units as a low bitrate speech codec. 
% Therefore, we evaluate how the performance varies as a function of the number of discrete units. Changing the number of units is equivalent to varying the bitrate of the encoded signal. 
We use a subjective MUSHRA-type listening test~\cite{series2014method} to measure the perceived quality of the proposed speech codec with regard to its bitrate constraints. In MUSHRA evaluations, listeners are presented with a labeled uncompressed signal for reference, a set of test samples to rate, a copy of the uncompressed reference, and a low-quality anchor. Listeners are asked to rate each test utterance and the copy of the uncompressed reference with respect to the labeled reference in a scale of 1-100.

The experiment is performed on the VCTK dataset~\cite{vctk}. For evaluation, we used 20 utterances from 5 speakers. The set of speakers in the test data is disjoint with those in the training data. For this experiment, HuBERT models with 50, 100, and 200 units were trained as described in Sec.~\ref{sec:impl}. For comparison, we included other speech codecs in our evaluation: Opus~\cite{valin2012definition} wideband at 9 kbps VBR, Codec2~\cite{rowe2011codec} at 2.4 kbps and LPCNet~\cite{valin2019real} operating at 1.6 kbps. The LPCNet model was trained from scratch on the VCTK dataset following the experimental setup in~\cite{valin2019real}. The VQ-VAE model employs the HiFiGAN decoder trained on the LibriLight dataset to match the amount of data reported in~\cite{garbacea2019low}. We compressed the anchor sample with Speex~\cite{valin2016speex} at 4 kbps as a low anchor. Fig.~\ref{fig:codec} depicts the results. HuBERT with 50 units reaches the best MUSHRA score while its bitrate is only 365bps, which is significantly lower than the baseline methods.

\section{Conclusions}

\section{Discussion and Conclusions}



Our method based on stabilizing forward and backward pass, resulted in improved accuracy over the baseline and it was able to predict optimal dampening, sharpness and tail-fatness before training. 
Our findings are coherent with the line of research that has established that stabilizing gradients and representations at initialization results in better performance \cite{glorot2010understanding, orthogonal_initialization, he2015delving, roberts2022principles, defazio2022scaling, bengio1994learning, hochreiter1997long, hochreiter2001gradient, arjovsky2016unitary, pascanu2013difficulty}. Moreover it gives an initial reply to the question raised by
\cite{surrogate2019, zenke2021remarkable}, which asked  for a theoretical justification of initialization and SG choice for Spiking Neural Networks. With a similar intention, \cite{rossbroich2022fluctuation} proposed an approach that guarantees sparsity of activity at initialization to pick the weights distribution at initialization, resulting in improved accuracy. Our method differs from theirs in that it starts from a principle of stability to derive constraints, instead of a principle of sparsity. It differs also in that we use it to define the SG shape at initialization, not only the weights distribution, and we can show mathematically how weights initialization is intertwined to the SG shape choice. Our results suggest that a tedious hyper-parameter grid-search can be often avoided by making use of sound and established principles of learning.

One of the conditions was designed to hit the most sensitive part of an SG, its center, which resulted in a low sparsity requirement at initialization. This is very uncommon in the Neuromorphic literature, since sparsity brings large energy gains \cite{henderson2020towards,blouw2019benchmarking, 9395703,taulsnn, rossbroich2022fluctuation}.
However, the energy gains of SNNs also come from their binary activity. A matrix-vector multiplication, with a $\mathbb{R}^{m\times n}$ matrix, has an energy cost of $mnE_{MAC}$ for a real vector, and of $mn\rho E_{AC}$ for a binary vector, where $\rho$ is the Bernouilli probability of the binary vector, and in our case the neuron firing rate, and $E_{AC}, E_{MAC}$ are the energies of an accumulate and a multiply-accumulate operation \cite{yin2021accurate, hunger2005floating}. Since MAC are more costly than AC, 31 times on a $45$nm complementary metal–oxide–semiconductor \cite{yin2021accurate, horowitz20141}, we have energy savings with any $\rho$, e.g., when all neurons fire ($\rho=1$) and when they fire half of the time steps ($\rho=1/2$). This gain does not depend on the simulation speed, since it compares a spiking and an analogue computation, at the same computation speed.
Typically requiring more sparsity through a sparsity encouraging loss term, leads to a measurable decrease in performance \cite{zenke2021remarkable, rossbroich2022fluctuation}. However we observed that it is actually possible to achieve higher performance with higher sparsity, by starting with a strong firing rate at initialization, since their synergy acts as a regularization mechanism. This was possible also because the sparsity encouraging loss term was introduced gradually, and because its contribution was kept comparable to the task loss towards the end of training.

We observed that the more complex the task is and the more complex the network to train is, the more drastic is the difference in performance of different SG shapes. It is known that learning is possible with a wide variety of SG shapes \cite{zenke2021remarkable} and the community has not yet settled for one shape or one method to reliably choose which SG to use in each case \cite{surrogate2019}. We showed how to apply a well known stability principle to the forward and backward pass of the simplest Spiking Neural Network, the LIF, as a starting point, but we think that the principles of good Neuromorphic initialization can be further elaborated, in order to tackle more complex tasks and networks.




%\section*{Acknowledgements}

\appendices
\section{Doppler Frequency Calculation}
One of the advantages of the DRT approach is the computation of Doppler information online in the algorithm with the aid of simple formulas. In such a way, there is no need to consider successive “snapshots” of the environment with slightly different displacements of the objects, and then to calculate the Doppler shifts with a “finite difference” computation method. \par 
	When TX and RX are both moving, the resulting apparent frequency $f^{'}$, including the Doppler frequency shift $f_D$ is computed for the LoS ray using the following equation \cite{ChenDoppler}: 
	\begin{equation}
	f^{'} = f_{0} + f_{D} = f_{0} \left (\frac{c-\overline{v}_{RX}\cdot \hat{k}}{c-\overline{v}_{TX}\cdot \hat{k}} \right) 
	\end{equation}
	where $f_0$ is the carrier frequency of the transmitted signal, $\hat{k}$ is the unit vector of the ray's direction from TX towards RX, and $\overline{v}_{TX}$, $\overline{v}_{RX}$ are the velocities of transmitter and receiver, respectively. \par
	This formula can be extended to rays with multiple bounces, where we have $n$ scattering points, each one moving with different speed (see Fig. \ref{doppler}): 
	\begin{equation}
	f^{'} = f_{0} + f_{D} = f_{0} \prod_{i=1}^{n+1} \left(\frac{c-\overline{v}_{i}\cdot \hat{k}_i}{c-\overline{v}_{i-1}\cdot \hat{k}_i}  \right)
	\label{doppler_freq}
	\end{equation}	
    where $\overline{v}_i$, $i=1,2,..,n$ is the velocity of the i-th interaction point, $\overline{v}_0=\overline{v}_{TX}$, and $\overline{v}_{n+1}=\overline{v}_{RX}$, respectively.\par
	\begin{figure}[h!]
		\centering
		\includegraphics[width=3in]{doppler_freq}
		\caption{Representation of the multiple scatterers Doppler model}
		\label{doppler}
	\end{figure}
	Equation (\ref{doppler_freq}) assumes that the velocities of the interaction points on the objects are known. In the simplified case of small scattering objects that can be approximated as "point scatterers", no further processing is needed, and we can directly apply eq. (\ref{doppler_freq}). \par 
	Instead, in the case of large objects we need to compute the velocity of the interaction points on the object surface as explained in section III. 

\section{Reflection Point's Velocity and Acceleration Calculation}
	Given the reflection point position ($Q_R$), its velocity ($\overline{v}_{Q_{R}}$) can be determined by deriving eq. (\ref{ref_point}) with respect to time. The x-component of $\overline{v}_{Q_{R}}$ can be calculated as: 
	
	\begin{equation}
	\begin{gathered}
	v_{Q_{R,x}} = \frac{\partial x_{Q_R}}{\partial t} = \frac{\partial x_{Q_R}}{\partial x_{TX}}~\frac{\partial x_{TX}}{\partial t} + \frac{\partial x_{Q_R}}{\partial y_{TX}}~\frac{\partial y_{TX}}{\partial t} \\ 
	+ \frac{\partial x_{Q_R}}{\partial x_{RX}}~\frac{\partial x_{RX}}{\partial t} + \frac{\partial x_{Q_R}}{\partial y_{RX}}~\frac{\partial y_{RX}}{\partial t}  \\ 
	= f_{x_{TX}} + f_{y_{TX}} + f_{x_{RX}} + f_{y_{RX}}, 
	\end{gathered}
	\label{vQRx}
	\end{equation}
	with 
	\begin{equation*}
	\begin{gathered}
	    \begin{cases*}
	        f_{x_{TX}} = \frac{\partial x_{Q_R}}{\partial x_{TX}}~v_{TX,x} \\
	        f_{y_{TX}} = \frac{\partial x_{Q_R}}{\partial y_{TX}}~v_{TX,y} \\ 
	        f_{x_{RX}} = \frac{\partial x_{Q_R}}{\partial x_{RX}}~v_{RX,x} \\ 
	        f_{y_{RX}} = \frac{\partial x_{Q_R}}{\partial y_{RX}}~v_{RX,y}.
	    \end{cases*}
	\end{gathered}
   \end{equation*}
	
The partial derivatives in (\ref{vQRx}) can be easily computed by deriving $x_{Q_R}$ w.r.t. the $x$,$y$ coordinates of TX and RX: 
\begin{equation*}
	\begin{gathered}
	\begin{cases*}
	\frac{\partial x_{Q_R}}{\partial x_{TX}} = 1 - \frac{y_{TX}}{y_{TX}+y_{RX}}\\ 
	\frac{\partial x_{Q_R}}{\partial y_{TX}} = \frac{y_{RX} (x_{RX}-x_{TX})}{(y_{RX}+y_{TX})^2}\\
	\frac{\partial x_{Q_R}}{\partial x_{RX}} = \frac{y_{TX}}{y_{TX}+y_{RX}} \\ 
	\frac{\partial x_{Q_R}}{\partial y_{RX}} = \frac{-y_{RX} (x_{RX}-x_{TX})}{(y_{RX}+y_{TX})^2}
	\end{cases*}
	\end{gathered} 
\end{equation*}
and substituting these expressions in (\ref{vQRx}) we finally get $v_{Q_R,x}$. \par
By following the same method as above, we can obtain the z-component of the velocity, $v_{Q_R,z}$, by time deriving the z-coordinate of $Q_R$ ($z_{Q_R}$). \par 
A similar procedure is used to calculate $\overline{a}_{Q_{R}}$ by deriving $\overline{v}_{Q_{R}}$ with respect to time:
\begin{equation}
	\begin{gathered}
	a_{Q_R,x}=a_{Q_R,x}^{(1)}+a_{Q_R,x}^{(2)}+a_{Q_R,x}^{(3)}+a_{Q_R,x}^{(4)} \\
	= \frac{\partial}{\partial t} f_{x_{TX}}+\frac{\partial}{\partial t} f_{y_{TX}}+\frac{\partial}{\partial t} f_{x_{RX}}+\frac{\partial}{\partial t} f_{y_{RX}}
	\end{gathered}
	\label{a_QRx}
\end{equation}
For the sake of brevity, we report below only the computation of the first partial derivative in eq. (\ref{a_QRx}): 
\begin{equation}
	\begin{gathered}
	a_{Q_R,x}^{(1)} = \frac{\partial}{\partial t} f_{x_{TX}} = \frac{\partial}{\partial t} \frac{\partial x_{Q_R}}{\partial x_{TX}} \frac{\partial x_{TX}}{\partial t} + \frac{\partial x_{Q_R}}{\partial x_{TX}} \frac{v_{TX,x}}{\partial t} \\ 
	= \left(\frac{\partial^2 x_{Q_R}}{\partial x_{TX} \partial y_{TX}} \frac{\partial y_{TX}}{\partial t} + \frac{\partial^2 x_{Q_R}}{\partial x_{TX} \partial y_{RX}} \frac{\partial y_{RX}}{\partial t} \right) v_{TX,x} \\
	+ \frac{\partial x_{Q_R}}{\partial x_{TX}} a_{TX,x}\\ 
	= \left(\frac{\partial^2 x_{Q_R}}{\partial x_{TX} \partial y_{TX}} v_{TX,y}  + \frac{\partial^2 x_{Q_R}}{\partial x_{TX} \partial y_{RX}} v_{RX,y} \right) v_{TX,x} \\ 
	+ \frac{\partial x_{Q_R}}{\partial x_{TX}} a_{TX,x}
	\end{gathered}
	\label{aQRx1}
\end{equation}
	where the partial derivatives in (\ref{aQRx1}) are expressed by:
	\begin{equation*}
	\begin{gathered}
	\begin{cases*}
	\frac{\partial x_{Q_R}}{\partial x_{TX}} = 1 - \frac{y_{TX}}{y_{TX}+y_{RX}}\\ 
	\frac{\partial^2 x_{Q_R}}{\partial x_{TX} \partial y_{TX}} = \frac{-y_{RX}}{\left( y_{TX}+y_{RX}\right)^2}\\ 
	\frac{\partial^2 x_{Q_R}}{\partial x_{TX} \partial y_{RX}} = \frac{y_{TX}}{\left( y_{TX}+y_{RX}\right)^2}.
	\end{cases*}
	\end{gathered}
	\end{equation*}
	By substituting these expressions in (\ref{aQRx1}) we obtain $a_{Q_R,x}^{(1)}$. \par
	By repeating the same procedure for the remaining components of $a_{Q_R,x}$, we finally get: 
	\begin{equation}
			\begin{gathered}
			a_{Q_R,x} = \left(\frac{\partial^2 x_{Q_R}}{\partial x_{TX} \partial y_{TX}} v_{TX,y}  + \frac{\partial^2 x_{Q_R}}{\partial x_{TX} \partial y_{RX}} v_{RX,y} \right) v_{TX,x}  \\
			+\frac{\partial x_{Q_R}}{\partial x_{TX}} a_{TX,x} + \left(\frac{\partial^2 x_{Q_R}}{\partial y_{TX} \partial x_{TX}} v_{TX,x} + \frac{\partial^2 x_{Q_R}}{\partial y_{TX} \partial x_{RX}} v_{RX,x} \right) v_{TX,y} \\ 
			+ \left (\frac{\partial^2 x_{Q_R}}{\partial y_{TX}^2 } v_{TX,y} + \frac{\partial^2 x_{Q_R}}{\partial y_{TX} \partial y_{RX} } v_{RX,y}  \right) v_{TX,y} + \frac{\partial x_{Q_R}}{\partial y_{TX}} a_{TX,y} \\ 
			+ \left(\frac{\partial^2 x_{Q_R}}{\partial x_{RX} \partial y_{TX}} v_{TX,y} + \frac{\partial^2 x_{Q_R}}{\partial x_{RX} \partial y_{RX}} v_{RX,y} \right) v_{RX,x} + \frac{\partial x_{Q_R}}{\partial x_{RX}} a_{RX,y} \\ 
			+ \left( \frac{\partial^2 x_{Q_R}}{\partial y_{RX} \partial x_{TX}} v_{TX,x} +  \frac{\partial^2 x_{Q_R}}{\partial y_{RX} \partial x_{RX}} v_{RX,x}  \right) v_{RX,y} \\ 
			+ \left(  \frac{\partial^2 x_{Q_R}}{\partial y_{RX} \partial y_{TX}} v_{TX,y} +  \frac{\partial^2 x_{Q_R}}{\partial y_{RX}^2} v_{RX,x}\right) v_{RX,x} + \frac{\partial x_{Q_R}}{\partial y_{RX}} a_{RX,y}.
			\end{gathered}
    \end{equation}
	
	A similar approach can be adopted to compute the z-component of  $\overline{a}_{Q_{R}}$.

\section{Diffraction Point's Velocity Calculation}
Time derivation of (\ref{diff_point}), gives us the z-component of the diffraction point instantaneous velocity: 
	\begin{equation}
	\begin{gathered}
	v_{Q_{D,z}} = \frac{\partial z_{Q_D}}{\partial t} = \frac{\partial z_{RX}}{\partial t} + \frac{\partial }{\partial t} \left(\frac{d_{RX}}{d_{RX}+d_{TX}} (z_{TX}-z_{RX}) \right) \\
	= v_{RX,z} + \frac{d_{RX}}{d_{RX}+d_{TX}} \left(v_{TX,z}-v_{RX,z} \right)\\ 
	+ (z_{TX}-z_{RX}) \frac{\partial}{\partial t} \left(\frac{d_{RX}}{d_{RX}+d_{TX}} \right)
	\end{gathered}
	\label{diff_point_velo}
	\end{equation}
	where \\
	\begin{equation}
	\begin{multlined}
	\frac{\partial}{\partial t} \left(\frac{d_{RX}}{d_{TX}+d_{RX}} \right)= \\[2ex]
	=\frac{\frac{\partial d_{RX}}{\partial t} (d_{RX}+d_{TX})-\left(\frac{\partial d_{RX}}{\partial t} + \frac{\partial d_{TX}}{\partial t} \right) d_{RX}}{(d_{TX}+d_{RX})^2}
	\end{multlined}
	\label{diff_point_velo2}
	\end{equation}
    \par 
	The derivative of $d_{RX}$ with respect to time can be computed by applying the derivative chain rule to eq. (\ref{dist2D_TXRX}):
	\begin{equation}
	\begin{gathered}
	\frac{\partial d_{RX}}{\partial t} = \frac{\partial d_{RX}}{\partial x_{RX}} \frac{\partial x_{RX}}{\partial t} + \frac{\partial d_{RX}}{\partial y_{RX}} \frac{\partial y_{RX}}{\partial t} \\ 
	= \frac{1}{d_{RX}}\left[ \left(x_{RX}-x_{Q_D}\right) v_{RX,x}+  \left(y_{RX}-y_{Q_D}\right) v_{RX,y}\right].
	\end{gathered}
	\label{diff_point_velo3}
	\end{equation}
	In similar way the derivative of $d_{TX}$ can be calculated: 
	\begin{equation}
	\frac{\partial d_{TX}}{\partial t} = \frac{1}{d_{TX}}\left[ \left(x_{TX}-x_{Q_D}\right) v_{TX,x}+  \left(y_{TX}-y_{Q_D}\right) v_{TX,y}\right]
	\label{diff_point_velo4}
	\end{equation}
By substituting (\ref{diff_point_velo3}) and (\ref{diff_point_velo4}) into (\ref{diff_point_velo2}) and (\ref{diff_point_velo}), we finally get $v_{Q_D,x}$.\par
The diffraction point's acceleration ($a_{Q_{D,z}}$) can be computed in a similar way by time deriving (\ref{diff_point_velo}). 


\section*{}
@book{meyerOffin,
  author =  {K. R. Meyer and D. C. Offin},
  title =   {Introduction to Hamiltonian Dynamical Systems and the N-Body Problem},
  editor =    {},
  publisher = {Springer International Publishing},
  year =      {2017},
  volume =    {90},
  number =    {},
  series =    {Applied Mathematical Sciences},
  address =   {},
  edition =   {3rd},
  month =     {},
}

@article{Llibre2021,
  author =  {J. Llibre and D. Pa?ca and C. Valls},
  title =   {The circular restricted 4-body problem with three equal primaries in the collinear central configuration of the 3-body problem},
  journal = {Celestial Mechanics and Dynamical Astronomy},
  year =    {2021},
  volume =  {133},
  number =  {53},
  pages =   {},
  doi =     {https://doi.org/10.1007/s10569-021-10052-6},
}

@article{AlvarezRamirez2015,
  author =  {M. Alvarez-Ram�rez and J.E.F. Skea and T.J. Stuchi},
  title =   {Nonlinear stability analysis in a equilateral restricted four-body problem},
  journal = {Astrophys Space Sci},
  year =    {2015},
  volume =  {358},
  number =  {3},
  pages =   {},
  doi =     {https://doi.org/10.1007/s10509-015-2333-4},
}

@article{Corbera2014,
  author =  {M. Corbera and J. Llibre},
  title =   {Central configurations of the $4$-body problem with masses $m_1=m_2>m_3=m_4>0$ and small},
  journal = {Appl Math Comput},
  year =    {2014},
  volume =  {246},
  number =  {},
  pages =   {121-147},
  doi =     {https://doi.org/10.1016/j.amc.2014.07.109},
}

@article{AlvarezRamirez2013,
  author =  {M. Alvarez-Ram\'irez and J. Llibre},
  title =   {The symmetric central configuration of the $4$-body problem with masses $m_1=m_2\ne m_3=m_4$},
  journal = {Appl Math Comput},
  year =    {2013},
  volume =  {219},
  number =  {},
  pages =   {5996-6001},
  doi =     {https://doi.org/10.1016/j.amc.2012.12.036},
}

@article{Long2002,
  author =  {Y. Long and S. Sun},
  title =   {Four-Body Central Configurations with some Equal Masses},
  journal = {Arch Rational Mech Anal},
  year =    {2002},
  volume =  {162},
  number =  {},
  pages =   {25-44},
  doi =     {https//doi.org/10.1007/s002050100183},
}

@article{Simo,
  author =  {C. Sim\'o},
  title =   {Relative Equilibrium Solutions in the Four Body Problem},
  journal = {Celestial Mechanics},
  year =    {1978},
  volume =  {18},
  number =  {},
  pages =   {165-184},
  doi =     {},
}

@article{Roy,
  author =  {A. E. Roy and B. A. Steves},
  title =   {Some Special restricted four-body problems-{\rm II}: From Caledonia to Copenhagen},
  journal = {Planet Space Sci},
  year =    {1998},
  volume =  {46},
  number =  {11-12},
  pages =   {1475-1486},
  doi =     {},
}

@article{Shoaib,
  author =  {M. Shoaib and I. Faye},
  title =   {Collinear Equilibrium Solutions of Four-Body Problems},
  journal = {J Astrophys Astr},
  volume =  {32},
  year =    {2011},
  number =  {},
  pages =   {411-423},
  doi =     {},
}

@article{Davis,
  author =  {J. Davis},
  title =   {Planetary Society asteroid hunters help find rare type of double asteroid},
  journal = {The Planetary Society},
  year =    {2018},
  number =  {},
  pages =   {},
  note =     {Online article at https://www.planetary.org/articles/shoemaker-winners-2017-ye5},
}

@manual{nasa,
  author =  {C. Cofield and J. Wendel},
  title =   {Observatories Team Up To Reveal Rare Double Asteroid},
  organization = {NASA},
  year =    {2018},
  note =    {Online article at https://www.nasa.gov/feature/jpl/observatories-team-up-to-reveal-rare-double-asteroid},
}

@misc{Johnston,
  author =  {W. R. Johnston},
  title =   {Asteroids with Satellites},
  year =    {2023},
  note =    {Online Database at https://www.johnstonsarchive.net/astro/asteroidmoons.html},
}

@article{Monteiro,
  author =  {F. Monteiro and E. Rond\'on and D. Lazzaro and J. Oey and M Evangelista-Santana and P. Avcoverde and M DeCicco and J. S. Silva-Cabrera and T. Rodrigues and L. B. Santos},
  title =   {Physical characterization of equal-mass binary near-Earth asteroid 2017 YE5: a possible dormant Jupiter-family comet},
  journal = {Monthly Notices of the Royal Astronomical Society},
  year =    {2021},
  volume =  {507},
  number =  {},
  pages =   {5403-5414},
  doi =     {https://doi.org/10.1093/mnras/stab2408},
}

@article{Scheeres,
  author =  {D. J. Scheeres and J. Bellerose},
  title =   {The Restricted Hill Full $4$-Body Problem: application to spacecraft motion about binary asteroids},
  journal = {Dyn Syst},
  year =    {2005},
  volume =  {20},
  number =  {1},
  pages =   {23-44},
  doi =     {https://doi.org/10.1080/1468936042000281321},
}

@article{Wang,
  author =  {H.S. Wang and X. Y. Hou},
  title =   {Forced Hovering orbit above the primary in the binary asteroid system},
  journal = {Celest Mech Dyn Astron},
  year =    {2022},
  volume =  {134},
  number =  {},
  pages =   {},
  doi =     {https://doi.org/10.1007/s10569-022-10098-0},
}

@article{Lu,
  author =  {J. Lu H. Shang and B. Wei},
  title =   {Accelerating binary asteroid system propagation via nested interpolation method},
  journal = {Celest Mech Dyn Astron},
  year =    {2023},
  volume =  {135},
  number =  {},
  pages  =  {},
  doi    =  {https://doi.org/10.1007/s10569-023-10123-w},
}

@article{Espitia,
  author =  {D. Espitia and E. A. Quintero and I. D. Arellano-Ram\'irez},
  title =   {Determination of Orbital Elements and Ephemerides using the Geometric Laplace's Method},
  journal = {J Astron Space Sci},
  year =    {2020},
  volume =  {37},
  number =  {3},
  pages =   {171-185},
  doi =     {https://doi.org/10.5140/JASS.2020.37.3.171},
}

@article{Stoica,
  author =  {J. Llibre and C. Stoica},
  title =   {Comet- and Hill-type periodic orbits in restricted $(N+1)$-body problems},
  journal = {J Differential Equations},
  year =    {2011},
  volume =  {250},
  number =  {},
  pages =   {1747-1766},
  doi =     {doi:10.1016/j.jde.2010.08.005},
}

@unpublished{Bakker,
  author =  {L. F. Bakker and J. Murri and S. Simmons},
  title =   {A model for the Binary Asteroid 2017 YE5},
  year =    {2023},
  Note =    {In preparation},
}

@unpublished{Cochran,
  author =  {L. F. Bakker and S. Cochran},
  title =   {Periodic solutions in the planar $5$-body problem with two free bodies},
  year =    {2021},
  note =    {Undergraduate Research},
}

@misc{Freeman,
  author = {N. J. Freeman},
  title  = {Investigations of a Binary Asteroid Dynamical Model},
  year   = {2023},
  month  = {April},
  note   = {Senior Thesis, Department of Physics and Astronomy, Brigham Young University},
}

@book{Szebehely,
  author    = {V. Szebehely},
  title     = {The Theory of Orbits},
  publisher = {Academic Press},
  address   = {New York},
  year      = {1967},
}





\end{document}
