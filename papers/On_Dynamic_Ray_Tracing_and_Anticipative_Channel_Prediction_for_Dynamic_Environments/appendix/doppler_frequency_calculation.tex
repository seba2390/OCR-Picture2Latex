One of the advantages of the DRT approach is the computation of Doppler information online in the algorithm with the aid of simple formulas. In such a way, there is no need to consider successive “snapshots” of the environment with slightly different displacements of the objects, and then to calculate the Doppler shifts with a “finite difference” computation method. \par 
	When TX and RX are both moving, the resulting apparent frequency $f^{'}$, including the Doppler frequency shift $f_D$ is computed for the LoS ray using the following equation \cite{ChenDoppler}: 
	\begin{equation}
	f^{'} = f_{0} + f_{D} = f_{0} \left (\frac{c-\overline{v}_{RX}\cdot \hat{k}}{c-\overline{v}_{TX}\cdot \hat{k}} \right) 
	\end{equation}
	where $f_0$ is the carrier frequency of the transmitted signal, $\hat{k}$ is the unit vector of the ray's direction from TX towards RX, and $\overline{v}_{TX}$, $\overline{v}_{RX}$ are the velocities of transmitter and receiver, respectively. \par
	This formula can be extended to rays with multiple bounces, where we have $n$ scattering points, each one moving with different speed (see Fig. \ref{doppler}): 
	\begin{equation}
	f^{'} = f_{0} + f_{D} = f_{0} \prod_{i=1}^{n+1} \left(\frac{c-\overline{v}_{i}\cdot \hat{k}_i}{c-\overline{v}_{i-1}\cdot \hat{k}_i}  \right)
	\label{doppler_freq}
	\end{equation}	
    where $\overline{v}_i$, $i=1,2,..,n$ is the velocity of the i-th interaction point, $\overline{v}_0=\overline{v}_{TX}$, and $\overline{v}_{n+1}=\overline{v}_{RX}$, respectively.\par
	\begin{figure}[h!]
		\centering
		\includegraphics[width=3in]{doppler_freq}
		\caption{Representation of the multiple scatterers Doppler model}
		\label{doppler}
	\end{figure}
	Equation (\ref{doppler_freq}) assumes that the velocities of the interaction points on the objects are known. In the simplified case of small scattering objects that can be approximated as "point scatterers", no further processing is needed, and we can directly apply eq. (\ref{doppler_freq}). \par 
	Instead, in the case of large objects we need to compute the velocity of the interaction points on the object surface as explained in section III. 