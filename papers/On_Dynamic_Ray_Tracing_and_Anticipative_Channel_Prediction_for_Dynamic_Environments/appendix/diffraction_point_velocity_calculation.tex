Time derivation of (\ref{diff_point}), gives us the z-component of the diffraction point instantaneous velocity: 
	\begin{equation}
	\begin{gathered}
	v_{Q_{D,z}} = \frac{\partial z_{Q_D}}{\partial t} = \frac{\partial z_{RX}}{\partial t} + \frac{\partial }{\partial t} \left(\frac{d_{RX}}{d_{RX}+d_{TX}} (z_{TX}-z_{RX}) \right) \\
	= v_{RX,z} + \frac{d_{RX}}{d_{RX}+d_{TX}} \left(v_{TX,z}-v_{RX,z} \right)\\ 
	+ (z_{TX}-z_{RX}) \frac{\partial}{\partial t} \left(\frac{d_{RX}}{d_{RX}+d_{TX}} \right)
	\end{gathered}
	\label{diff_point_velo}
	\end{equation}
	where \\
	\begin{equation}
	\begin{multlined}
	\frac{\partial}{\partial t} \left(\frac{d_{RX}}{d_{TX}+d_{RX}} \right)= \\[2ex]
	=\frac{\frac{\partial d_{RX}}{\partial t} (d_{RX}+d_{TX})-\left(\frac{\partial d_{RX}}{\partial t} + \frac{\partial d_{TX}}{\partial t} \right) d_{RX}}{(d_{TX}+d_{RX})^2}
	\end{multlined}
	\label{diff_point_velo2}
	\end{equation}
    \par 
	The derivative of $d_{RX}$ with respect to time can be computed by applying the derivative chain rule to eq. (\ref{dist2D_TXRX}):
	\begin{equation}
	\begin{gathered}
	\frac{\partial d_{RX}}{\partial t} = \frac{\partial d_{RX}}{\partial x_{RX}} \frac{\partial x_{RX}}{\partial t} + \frac{\partial d_{RX}}{\partial y_{RX}} \frac{\partial y_{RX}}{\partial t} \\ 
	= \frac{1}{d_{RX}}\left[ \left(x_{RX}-x_{Q_D}\right) v_{RX,x}+  \left(y_{RX}-y_{Q_D}\right) v_{RX,y}\right].
	\end{gathered}
	\label{diff_point_velo3}
	\end{equation}
	In similar way the derivative of $d_{TX}$ can be calculated: 
	\begin{equation}
	\frac{\partial d_{TX}}{\partial t} = \frac{1}{d_{TX}}\left[ \left(x_{TX}-x_{Q_D}\right) v_{TX,x}+  \left(y_{TX}-y_{Q_D}\right) v_{TX,y}\right]
	\label{diff_point_velo4}
	\end{equation}
By substituting (\ref{diff_point_velo3}) and (\ref{diff_point_velo4}) into (\ref{diff_point_velo2}) and (\ref{diff_point_velo}), we finally get $v_{Q_D,x}$.\par
The diffraction point's acceleration ($a_{Q_{D,z}}$) can be computed in a similar way by time deriving (\ref{diff_point_velo}). 