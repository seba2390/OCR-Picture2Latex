	Given the reflection point position ($Q_R$), its velocity ($\overline{v}_{Q_{R}}$) can be determined by deriving eq. (\ref{ref_point}) with respect to time. The x-component of $\overline{v}_{Q_{R}}$ can be calculated as: 
	
	\begin{equation}
	\begin{gathered}
	v_{Q_{R,x}} = \frac{\partial x_{Q_R}}{\partial t} = \frac{\partial x_{Q_R}}{\partial x_{TX}}~\frac{\partial x_{TX}}{\partial t} + \frac{\partial x_{Q_R}}{\partial y_{TX}}~\frac{\partial y_{TX}}{\partial t} \\ 
	+ \frac{\partial x_{Q_R}}{\partial x_{RX}}~\frac{\partial x_{RX}}{\partial t} + \frac{\partial x_{Q_R}}{\partial y_{RX}}~\frac{\partial y_{RX}}{\partial t}  \\ 
	= f_{x_{TX}} + f_{y_{TX}} + f_{x_{RX}} + f_{y_{RX}}, 
	\end{gathered}
	\label{vQRx}
	\end{equation}
	with 
	\begin{equation*}
	\begin{gathered}
	    \begin{cases*}
	        f_{x_{TX}} = \frac{\partial x_{Q_R}}{\partial x_{TX}}~v_{TX,x} \\
	        f_{y_{TX}} = \frac{\partial x_{Q_R}}{\partial y_{TX}}~v_{TX,y} \\ 
	        f_{x_{RX}} = \frac{\partial x_{Q_R}}{\partial x_{RX}}~v_{RX,x} \\ 
	        f_{y_{RX}} = \frac{\partial x_{Q_R}}{\partial y_{RX}}~v_{RX,y}.
	    \end{cases*}
	\end{gathered}
   \end{equation*}
	
The partial derivatives in (\ref{vQRx}) can be easily computed by deriving $x_{Q_R}$ w.r.t. the $x$,$y$ coordinates of TX and RX: 
\begin{equation*}
	\begin{gathered}
	\begin{cases*}
	\frac{\partial x_{Q_R}}{\partial x_{TX}} = 1 - \frac{y_{TX}}{y_{TX}+y_{RX}}\\ 
	\frac{\partial x_{Q_R}}{\partial y_{TX}} = \frac{y_{RX} (x_{RX}-x_{TX})}{(y_{RX}+y_{TX})^2}\\
	\frac{\partial x_{Q_R}}{\partial x_{RX}} = \frac{y_{TX}}{y_{TX}+y_{RX}} \\ 
	\frac{\partial x_{Q_R}}{\partial y_{RX}} = \frac{-y_{RX} (x_{RX}-x_{TX})}{(y_{RX}+y_{TX})^2}
	\end{cases*}
	\end{gathered} 
\end{equation*}
and substituting these expressions in (\ref{vQRx}) we finally get $v_{Q_R,x}$. \par
By following the same method as above, we can obtain the z-component of the velocity, $v_{Q_R,z}$, by time deriving the z-coordinate of $Q_R$ ($z_{Q_R}$). \par 
A similar procedure is used to calculate $\overline{a}_{Q_{R}}$ by deriving $\overline{v}_{Q_{R}}$ with respect to time:
\begin{equation}
	\begin{gathered}
	a_{Q_R,x}=a_{Q_R,x}^{(1)}+a_{Q_R,x}^{(2)}+a_{Q_R,x}^{(3)}+a_{Q_R,x}^{(4)} \\
	= \frac{\partial}{\partial t} f_{x_{TX}}+\frac{\partial}{\partial t} f_{y_{TX}}+\frac{\partial}{\partial t} f_{x_{RX}}+\frac{\partial}{\partial t} f_{y_{RX}}
	\end{gathered}
	\label{a_QRx}
\end{equation}
For the sake of brevity, we report below only the computation of the first partial derivative in eq. (\ref{a_QRx}): 
\begin{equation}
	\begin{gathered}
	a_{Q_R,x}^{(1)} = \frac{\partial}{\partial t} f_{x_{TX}} = \frac{\partial}{\partial t} \frac{\partial x_{Q_R}}{\partial x_{TX}} \frac{\partial x_{TX}}{\partial t} + \frac{\partial x_{Q_R}}{\partial x_{TX}} \frac{v_{TX,x}}{\partial t} \\ 
	= \left(\frac{\partial^2 x_{Q_R}}{\partial x_{TX} \partial y_{TX}} \frac{\partial y_{TX}}{\partial t} + \frac{\partial^2 x_{Q_R}}{\partial x_{TX} \partial y_{RX}} \frac{\partial y_{RX}}{\partial t} \right) v_{TX,x} \\
	+ \frac{\partial x_{Q_R}}{\partial x_{TX}} a_{TX,x}\\ 
	= \left(\frac{\partial^2 x_{Q_R}}{\partial x_{TX} \partial y_{TX}} v_{TX,y}  + \frac{\partial^2 x_{Q_R}}{\partial x_{TX} \partial y_{RX}} v_{RX,y} \right) v_{TX,x} \\ 
	+ \frac{\partial x_{Q_R}}{\partial x_{TX}} a_{TX,x}
	\end{gathered}
	\label{aQRx1}
\end{equation}
	where the partial derivatives in (\ref{aQRx1}) are expressed by:
	\begin{equation*}
	\begin{gathered}
	\begin{cases*}
	\frac{\partial x_{Q_R}}{\partial x_{TX}} = 1 - \frac{y_{TX}}{y_{TX}+y_{RX}}\\ 
	\frac{\partial^2 x_{Q_R}}{\partial x_{TX} \partial y_{TX}} = \frac{-y_{RX}}{\left( y_{TX}+y_{RX}\right)^2}\\ 
	\frac{\partial^2 x_{Q_R}}{\partial x_{TX} \partial y_{RX}} = \frac{y_{TX}}{\left( y_{TX}+y_{RX}\right)^2}.
	\end{cases*}
	\end{gathered}
	\end{equation*}
	By substituting these expressions in (\ref{aQRx1}) we obtain $a_{Q_R,x}^{(1)}$. \par
	By repeating the same procedure for the remaining components of $a_{Q_R,x}$, we finally get: 
	\begin{equation}
			\begin{gathered}
			a_{Q_R,x} = \left(\frac{\partial^2 x_{Q_R}}{\partial x_{TX} \partial y_{TX}} v_{TX,y}  + \frac{\partial^2 x_{Q_R}}{\partial x_{TX} \partial y_{RX}} v_{RX,y} \right) v_{TX,x}  \\
			+\frac{\partial x_{Q_R}}{\partial x_{TX}} a_{TX,x} + \left(\frac{\partial^2 x_{Q_R}}{\partial y_{TX} \partial x_{TX}} v_{TX,x} + \frac{\partial^2 x_{Q_R}}{\partial y_{TX} \partial x_{RX}} v_{RX,x} \right) v_{TX,y} \\ 
			+ \left (\frac{\partial^2 x_{Q_R}}{\partial y_{TX}^2 } v_{TX,y} + \frac{\partial^2 x_{Q_R}}{\partial y_{TX} \partial y_{RX} } v_{RX,y}  \right) v_{TX,y} + \frac{\partial x_{Q_R}}{\partial y_{TX}} a_{TX,y} \\ 
			+ \left(\frac{\partial^2 x_{Q_R}}{\partial x_{RX} \partial y_{TX}} v_{TX,y} + \frac{\partial^2 x_{Q_R}}{\partial x_{RX} \partial y_{RX}} v_{RX,y} \right) v_{RX,x} + \frac{\partial x_{Q_R}}{\partial x_{RX}} a_{RX,y} \\ 
			+ \left( \frac{\partial^2 x_{Q_R}}{\partial y_{RX} \partial x_{TX}} v_{TX,x} +  \frac{\partial^2 x_{Q_R}}{\partial y_{RX} \partial x_{RX}} v_{RX,x}  \right) v_{RX,y} \\ 
			+ \left(  \frac{\partial^2 x_{Q_R}}{\partial y_{RX} \partial y_{TX}} v_{TX,y} +  \frac{\partial^2 x_{Q_R}}{\partial y_{RX}^2} v_{RX,x}\right) v_{RX,x} + \frac{\partial x_{Q_R}}{\partial y_{RX}} a_{RX,y}.
			\end{gathered}
    \end{equation}
	
	A similar approach can be adopted to compute the z-component of  $\overline{a}_{Q_{R}}$.