\section{Introduction}
\label{sec:introduction}

Inverse problems arise in many applications in science and engineering. The term ``inverse problem'' is generally understood as the problem of finding a specific physical property, or properties, of the medium under investigation, using indirect measurements. This is a highly important field of applied mathematics and scientific computing, as to a great extent, it forms the backbone of modern science and engineering. Examples of inverse problems can be found in various fields within medical imaging \cite{arridge1999optical,arridge1997optical,bertero2010introduction,rundell1997inverse,louis1992medical} and several areas of geophysics including mineral and oil exploration \cite{menke2012geophysical,aster2013parameter,bunks1995multiscale,russell1988introduction}. 

In general, an inverse problem aims at recovering the unknown underlying parameters
of a physical system which produces the available observations/measurements.
Such problems are generally ill-posed \cite{hadamard1902sur}.
This is often solved via two approaches: 
a Bayesian approach which computes a posterior distribution
of the models given prior knowledge and the data, or a regularized data
fitting approach which chooses an optimal model by minimizing an objective that
takes into account both fitness to data and prior knowledge.
The Bayesian approach can be used for a variety of downstream inference tasks, such as credible intervals for the
parameters; it is generally more computationally expensive than the data fitting approach. The computational attractiveness of data fitting comes at a cost: it can only produce a ``point'' estimate of the unknown parameters. However, in many applications, such a point estimate can be more than adequate.  

In this review, we focus on the data fitting approach.
Optimization algorithms are central in this approach as the recovery of the unknown parameters is formulated as an optimization problem.
While numerous works have been done on the subject, there are still many
challenges remaining, including scaling up to large-scale problems, dealing with
non-convexity.
Optimization constitutes a backbone of many machine learning applications~\cite{bottou2016optimization,domingos2012few}. Consequently, there are many related developments in optimization from the
machine learning community. However, thus far and rather independently, the machine learning and the inverse problems communities have largely developed their own sets of tools and algorithms to address their respective optimization challenges.
It only stands to reason that many of the recent advances by machine learning can be potentially applicable for addressing challenges in solving inverse problems.
We aim to bring out this connection and encourage permeation of ideas across these two communities.

In \Cref{sec:inv_prob}, we present general formulations for the inverse
problem, some typical inverse problems, and optimization algorithms commonly
used to solve the data fitting problem.
We discuss recent advances in optimization in \Cref{sec:advances}.
We then discuss areas in which cross-fertilization of optimization and inverse
problems can be beneficial in \Cref{sec:discussion}.
We conclude in \Cref{sec:conclusion}.
