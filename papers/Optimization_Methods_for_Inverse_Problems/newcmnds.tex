\usepackage{graphicx}
\usepackage{amsmath}
\usepackage{amssymb}
\usepackage{latexsym}
\usepackage{subfigure}
\usepackage{crop}
\usepackage{algorithmic}
\usepackage{algorithm}
\usepackage{multirow}
%\usepackage{algorithm,algorithmic}
%\usepackage[section,subsection,subsubsection]{placeins}
\usepackage{bm}
\usepackage{bbm}
\usepackage{enumerate}
\usepackage{framed} % or, "mdframed"
\usepackage[framed]{ntheorem}
%\usepackage{datetime}
\usepackage{url}
\usepackage[colorlinks = true, pdfstartview = FitV, linkcolor = blue, citecolor = blue, urlcolor = blue]{hyperref}
\usepackage{array}
\usepackage{paralist}
\usepackage{authblk}

\usepackage{cleveref}
\crefname{equation}{Eq.}{Eqs.}
\crefname{figure}{Fig.}{Figs.}
\usepackage{relsize}

%%%%%%%%%%%%%%%%%%%%%%%%%%%%%%%%%%%%%%%%%%%%%%%%%%%%%%%%%%%%%%%%%%%%%%%%%%%%%%%%%%%%%%%%%%%%%
\usepackage{fullpage}
%\def\spacingset#1{\renewcommand{\baselinestretch}{#1}\small\normalsize}
%\setlength{\topmargin}{-.50in}
%\setlength{\leftmargin}{0.0in}
%\setlength{\evensidemargin}{0.25in}
%\setlength{\oddsidemargin}{0.25in}
%\setlength{\textheight}{8.5in}
%\setlength{\textwidth}{6.0in}
%%%%%%%%%%%%%%%%%%%%%%%%%%%%%%%%%%%%%%%%%%%%%%%%%%%%%%%%%%%%%%%%%%%%%%%%%%%%%%%%%%%%%%%%%%%%%


%%%%%%%%%%%%%%%%%%%%%%%%%%%%%%%%%%%%%%%%%%%%%%%%%%%%%%%%%%%%%%%%%%%%%%%%%%%%%%%%%%%%%%%%%%%%%
\usepackage[sort,nocompress]{cite}
%\usepackage[sort,nocompress,space]{cite}
%%%%%%%%%%%%%%%%%%%%%%%%%%%%%%%%%%%%%%%%%%%%%%%%%%%%%%%%%%%%%%%%%%%%%%%%%%%%%%%%%%%%%%%%%%%%%


%%%%%%%%%%%%%%%%%%%%%%%%%%%%%%%%%%%%%%%%%%%%%%%%%%%%%%%%%%%%%%%%%%%%%%%%%%%%%%%%%%%%%%%%%%%%%
\usepackage{enumitem}
%% The following can be uncommented instead of using paralist package...but cannot use together.
%\newlist{compactenum}{enumerate}{4}
%\setlist[compactenum,1]{nolistsep}
\setlist[enumerate,1]{leftmargin=*,wide=0em, noitemsep,nolistsep, label = {\bfseries \arabic*.}}
\setlist[itemize,1]{leftmargin=*,wide=0em, noitemsep,nolistsep}
%%%%%%%%%%%%%%%%%%%%%%%%%%%%%%%%%%%%%%%%%%%%%%%%%%%%%%%%%%%%%%%%%%%%%%%%%%%%%%%%%%%%%%%%%%%%%


%%%%%%%%%%%%%%%%%%%%%%%%%%%%%%%%%%%%%%%%%%%%%%%%%%%%%%%%%%%%%%%%%%%%%%%%%%%%%%%%%%%%%%%%%%%%%
\usepackage{titlesec}
\titleformat*{\section}{\large\bfseries}
\titleformat*{\subsection}{\large\bfseries}
\titleformat*{\subsubsection}{\large\bfseries}
\titleformat*{\paragraph}{\normalsize\bfseries}
\titleformat*{\subparagraph}{\normalsize\bfseries}
%%%%%%%%%%%%%%%%%%%%%%%%%%%%%%%%%%%%%%%%%%%%%%%%%%%%%%%%%%%%%%%%%%%%%%%%%%%%%%%%%%%%%%%%%%%%%


%\newcommand {\ourname} {{\bf OurName\ }}

\def\Mm#1{\mbox{\boldmath$\scriptstyle #1$\unboldmath}} % Math-bold in subscript
\def\MM#1{\mbox{\boldmath$#1$\unboldmath}} % Math-bold


%%%%%%%%%%%%%%%%%%%%%%%%%%%%%%%%%%%%%%%%%%%%%%%%%%%%%%%%%%%%%%%%%%%%%%%%%%%%%%%%%%%%%%%%%%%%%
\newcommand {\uu}  { {\bf u} }
\newcommand {\zz}  { {\bf z} }
\newcommand {\bgg}  { {\bf g} }
\newcommand {\NN}  { {\cal N} }
\newcommand {\xx}  { {\bf x} }
\renewcommand {\aa}  { {\bf a} }
\newcommand {\PP}  { {\bf P} }
\newcommand {\ii}  { {\bf i} }
\newcommand {\jj}  { {\bf j} }
\newcommand {\yy}  { {\bf y} }
\newcommand {\hh}  { {\bf h} }
\newcommand {\kk}  { {\bf k} }\newcommand {\rr}  { {\bf r} }
\newcommand {\EE}  { {\bf E} }
\newcommand {\HH}  { {\bf H} }
\newcommand {\JJ}  { {\bf J} }
\newcommand{\R}{{\rm I\!R}}
\newcommand{\im}{{\cal I}{\rm m} }
\newcommand {\cS}  { {\cal S} }
\newcommand {\cC}  { {\cal C} }
\newcommand {\cK}  { {\cal K} }
\newcommand {\cX}  { {\cal X} }
\newcommand {\KH}  { {\mathcal K}_H }
\newcommand {\KG}  { {\mathcal K}_G }
\newcommand {\KU}  { {\mathcal K}_U }
\newcommand\argmin[1]  {\underset{#1}{\operatorname{arg\ min}}}
%\newcommand {\grad}  { {\rm grad} \,}
%\newcommand {\curl}  { {\rm curl} \,}
\newcommand {\veps} {\bm \epsilon}
\newcommand {\Ro} { {\cal R}}
\newcommand {\Ex} { {\mathbb E} }
\newcommand {\lag} { {\cal L}}
\newcommand {\sci} { {\rm i}}
\newcommand {\qq}  { {\bf q} }
\newcommand {\QQ}  { {\bf Q} }
\newcommand {\Ab}  { {\bf A} }
\newcommand {\pp}  { {\bf p} }
\newcommand {\mm}  { {\bf m} }
\newcommand {\vv}  { {\bf v} }
\newcommand {\ww}  { {\bf w} }
\newcommand {\ff}  { {\bf f} }
\newcommand {\FF}  { {\bf F} }
\newcommand {\BB}  { {\bf B} }
\newcommand {\DD}  { {\bf D} }
\newcommand {\ZZ}  { {\bf Z} }
\newcommand {\bb}  { {\bf b} }
\newcommand {\cc}  { {\bf c} }
\newcommand {\dd}  { {\bf d} }
\newcommand {\ee}  { {\bf e} }
\newcommand {\sa}  { {\bf s} }
\newcommand {\blambda}  { {\boldsymbol \lambda} }
\newcommand {\bmu}  { {\boldsymbol \mu} }
\newcommand {\zero}  { {\bf 0} }
\newcommand {\one}  { {\bf 1} }
\newcommand {\gb}  { {\bf g} }
\newcommand {\bnabla} { { \boldsymbol \nabla} }
\newcommand {\btheta} { { \boldsymbol \theta} }
\newcommand {\bdelta} { { \boldsymbol \delta} }
\newcommand {\bepsilon} { { \boldsymbol \epsilon} }
\newcommand {\bxi} { { \boldsymbol \xi} }
\newcommand {\fuu}  { {\frac{\partial (f_u^T\lambda)}{\partial u}} }
\newcommand {\fum}  { {\frac{\partial (f_u^T\lambda)}{\partial m}} }
\newcommand {\fmm}  { {\frac{\partial (f_m^T\lambda)}{\partial m}} }
\newcommand {\sg}{{\sigma}}
\newcommand{\hf}{\frac12}
\newcommand{\hx}[1]{{\ensuremath{h^x_{\scriptscriptstyle #1}}}}
\newcommand{\hy}[1]{{\ensuremath{h^y_{\scriptscriptstyle #1}}}}
\newcommand{\hz}[1]{{\ensuremath{h^z_{\scriptscriptstyle #1}}}}
\newcommand{\x}[1]{\ensuremath{x_{\scriptscriptstyle #1}}}
\newcommand{\y}[1]{\ensuremath{y_{\scriptscriptstyle #1}}}
\newcommand{\z}[1]{\ensuremath{z_{\scriptscriptstyle #1}}}
\renewcommand{\vec}[1]{\ensuremath{\mathbf{#1}}}
\newcommand{\A}{\vec{A}}
\renewcommand{\H}{\vec{H}}
\newcommand{\J}{\vec{J}}
\newcommand{\F}{\vec{F}}
\newcommand{\s}{\vec{s}}
%\newcommand{\curl}{\ensuremath{\nabla\times\,}}
%\newcommand{\grad}{\ensuremath{\bnabla}}
\newcommand{\sig}{\sigma}
\newcommand{\hsig}{\widehat \sigma}
\newcommand{\hJ}{\widehat{\vec{J}}}
%\newcommand {\bnabla} { { \boldsymbol \nabla} }
\newcommand{\nn}{\vec{n}}
\renewcommand{\div}{\nabla\cdot\,}
\newcommand{\grad}{\ensuremath {\vec \nabla}}
\newcommand{\curl}{\ensuremath{{\vec \nabla}\times\,}}
\newcommand\CP[2]{#1\times#2}         % cross product
\newcommand\DP[2]{(#1\cdot#2)}        % dot product
%\newcommand\grad{\nabla}              % gradient symbol
\newcommand\interval[2]{[#1\dots#2]}  % intervals
\newcommand\M[1]{{\bf#1}}             % matrix
\newcommand\Pt[1]{{\bf#1}}            % point
\newcommand\V[1]{\vec{#1}}            % vector
\newcommand\neighbor{\mathcal{N}}
\newcommand{\defeq}{\mathrel{\mathop:}=}
\newcommand{\defeqr}{=\mathrel{\mathop:}}
\newcommand{\opnsubset}{\mathrel{\ooalign{$\subset$\cr
  \hidewidth\hbox{$\circ\mkern.5mu$}\cr}}}
\renewcommand{\Pr}{\hbox{\bf{Pr}}}


%%%%%%%%%%%%%%%%%%%%%%%%%%%%%%%%%%%%%%%%%%%%%%%%%%%%%%%%%%%%%%%%%%%%%%%%%%%%%%%%%%%%%%%%%%%%%  

\newcommand{\fred}[1]{\textcolor{red}{Fred:\ #1}}

%%%%%%%%%%%%%%%%%%%%%%%%%%%%%%%%%%%%%%%%%%%%%%%%%%%%%%%%%%%%%%%%%%%%%%%%%%%%%%%%%%%%%%%%%%%%%  
\newcommand{\changeurlcolor}[1]{\hypersetup{urlcolor=#1}}   

\newcommand{\red}[1]{\textcolor{red}{#1}}

\newcommand{\blue}[1]{\textcolor{blue}{#1}}

\definecolor{forestgreen}{rgb}{0.13, 0.55, 0.13}
\newcommand{\forestgreen}[1]{\textcolor{forestgreen}{#1}}

%%%%%%%%%%%%%%%%%%%%%%%%%%%%%%%%%%%%%%%%%%%%%%%%%%%%%%%%%%%%%%%%%%%%%%%%%%%%%%%%%%%%%%%%%%%%%  
\newcounter{cmt}
\setcounter{cmt}{0}
%\newcommand{\Comment}[1]{\refstepcounter{cmt}\textbf{Comment \arabic{cmt}: }#1}
\marginparwidth=0.75in
\newcommand{\Comment}[2]{\addtocounter{cmt}{1}{\color{brown}#1}\marginpar{\smaller\noindent{\raggedright{\color{brown}[\arabic{cmt}]}\newline\color{brown}{#2}\par}}}

%%%%%%%%%%%%%%%%%%%%%%%%%%%%%%%%%%%%%%%%%%%%%%%%%%%%%%%%%%%%%%%%%%%%%%%%%%%%%%%%%%%%%%%%%%%%%  

%\theoremclass{Theorem}
%\theoremstyle{break}
%\newframedtheorem{theorem}{Theorem}
%\newframedtheorem{corollary}{Corollary}
%\newframedtheorem{lemma}{Lemma}
%\newframedtheorem{definition}{Definition}
%\newframedtheorem{proposition}{Proposition}
%\newframedtheorem{assumption}{Assumption}
%%\newtheorem{theorem}{Theorem}
%%\newtheorem{conjecture}[theorem]{Conjecture}
%%\newtheorem{corollary}[theorem]{Corollary}
%%\newtheorem{proposition}[theorem]{Proposition}
%%\newtheorem{lemma}[theorem]{Lemma}
%%\newtheorem{definition}{Definition}
%\newtheorem{example}{Example}
%%\newtheorem{experiment}{Experiment}

%\newenvironment{proof}[1][Proof]{\begin{trivlist}
%\item[\hskip \labelsep {\bfseries #1}]}{\end{trivlist}}
%\newenvironment{remark}[1][Remark]{\begin{trivlist}
%\item[\hskip \labelsep {\bfseries #1}]}{\end{trivlist}}

%\newcommand{\qed}{\nobreak \ifvmode \relax \else
%      \ifdim\lastskip<1.5em \hskip-\lastskip
%      \hskip1.5em plus0em minus0.5em \fi \nobreak
%      \vrule height0.75em width0.5em depth0.25em\fi}
%%%%%%%%%%%%%%%%%%%%%%%%%%%%%%%%%%%%%%%%%%%%%%%%%%%%%%%%%%%%%%%%%%%%%%%%%%%%%%%%%%%%%%%%%%%%%  

\usepackage{tcolorbox} % for boxed text
\tcbuselibrary{breakable}
\tcbuselibrary{skins}
% use by 
% \begin{tcolorbox}[breakable,enhanced]
%\end{tcolorbox}

%%%%%%%%%%%%%%%%%%%%%%%%%%%%%%%%%%%%%%%%%%%%%%%%%%%%%%%%%%%%%%%%%%%%%%%%%%%%%%%%%%%%%%%%%%%%
\usepackage{listings} % to inser code

\definecolor{mygreen}{rgb}{0,0.6,0}
\definecolor{mygray}{rgb}{0.5,0.5,0.5}
\definecolor{mymauve}{rgb}{0.58,0,0.82}

\lstset{ %
  backgroundcolor=\color{white},   % choose the background color; you must add \usepackage{color} or \usepackage{xcolor}; should come as last argument
  basicstyle=\footnotesize,        % the size of the fonts that are used for the code
  breakatwhitespace=false,         % sets if automatic breaks should only happen at whitespace
  breaklines=true,                 % sets automatic line breaking
  captionpos=b,                    % sets the caption-position to bottom
  commentstyle=\color{mygreen},    % comment style
  deletekeywords={...},            % if you want to delete keywords from the given language
  escapeinside={\%*}{*)},          % if you want to add LaTeX within your code
  extendedchars=true,              % lets you use non-ASCII characters; for 8-bits encodings only, does not work with UTF-8
  frame=single,	                   % adds a frame around the code
  keepspaces=true,                 % keeps spaces in text, useful for keeping indentation of code (possibly needs columns=flexible)
  keywordstyle=\color{blue},       % keyword style
  language=Octave,                 % the language of the code
  morekeywords={*,...},           % if you want to add more keywords to the set
  numbers=left,                    % where to put the line-numbers; possible values are (none, left, right)
  numbersep=5pt,                   % how far the line-numbers are from the code
  numberstyle=\tiny\color{mygray}, % the style that is used for the line-numbers
  rulecolor=\color{black},         % if not set, the frame-color may be changed on line-breaks within not-black text (e.g. comments (green here))
  showspaces=false,                % show spaces everywhere adding particular underscores; it overrides 'showstringspaces'
  showstringspaces=false,          % underline spaces within strings only
  showtabs=false,                  % show tabs within strings adding particular underscores
  stepnumber=2,                    % the step between two line-numbers. If it's 1, each line will be numbered
  stringstyle=\color{mymauve},     % string literal style
  tabsize=2,	                   % sets default tabsize to 2 spaces
  title=\lstname                   % show the filename of files included with \lstinputlisting; also try caption instead of title
}

%%%%%%%%%%%%%%%%%%%%%%%%%%%%%%%%%%%%%%%%%%%%%%%%%%%%%%%%%%%%%%%%%%%%%%%%%%%%%%%%%%%5
\newcommand\modelspace{{\cal M}}
\newcommand*\lin[1]{\langle #1\rangle}
