%%%%%%%%%%%%%%%%%%%%%%%%%%%%%%%%%%%%%%%%%%%%%%%%%%%
\documentclass[graybox]{svmult}
% choose options for [] as required from the list  in the Reference Guide
\usepackage{mathptmx}       % selects Times Roman as basic font
\usepackage{helvet}         % selects Helvetica as sans-serif font
\usepackage{courier}        % selects Courier as typewriter font
%\usepackage{type1cm}        % activate if the above 3 fonts are not available on your system
\usepackage{makeidx}         % allows index generation
\usepackage{graphicx}        % standard LaTeX graphics tool
\usepackage{multicol}        % used for the two-column index
\usepackage[bottom]{footmisc}% places footnotes at page bottom
\usepackage{amsmath,amssymb,amsfonts}        % AMS math packages

\usepackage{graphicx}
\usepackage{amsmath}
\usepackage{amssymb}
\usepackage{latexsym}
\usepackage{subfigure}
\usepackage{crop}
\usepackage{algorithmic}
\usepackage{algorithm}
\usepackage{multirow}
%\usepackage{algorithm,algorithmic}
%\usepackage[section,subsection,subsubsection]{placeins}
\usepackage{bm}
\usepackage{bbm}
\usepackage{enumerate}
\usepackage{framed} % or, "mdframed"
\usepackage[framed]{ntheorem}
%\usepackage{datetime}
\usepackage{url}
\usepackage[colorlinks = true, pdfstartview = FitV, linkcolor = blue, citecolor = blue, urlcolor = blue]{hyperref}
\usepackage{array}
\usepackage{paralist}
\usepackage{authblk}

\usepackage{cleveref}
\crefname{equation}{Eq.}{Eqs.}
\crefname{figure}{Fig.}{Figs.}
\usepackage{relsize}

%%%%%%%%%%%%%%%%%%%%%%%%%%%%%%%%%%%%%%%%%%%%%%%%%%%%%%%%%%%%%%%%%%%%%%%%%%%%%%%%%%%%%%%%%%%%%
\usepackage{fullpage}
%\def\spacingset#1{\renewcommand{\baselinestretch}{#1}\small\normalsize}
%\setlength{\topmargin}{-.50in}
%\setlength{\leftmargin}{0.0in}
%\setlength{\evensidemargin}{0.25in}
%\setlength{\oddsidemargin}{0.25in}
%\setlength{\textheight}{8.5in}
%\setlength{\textwidth}{6.0in}
%%%%%%%%%%%%%%%%%%%%%%%%%%%%%%%%%%%%%%%%%%%%%%%%%%%%%%%%%%%%%%%%%%%%%%%%%%%%%%%%%%%%%%%%%%%%%


%%%%%%%%%%%%%%%%%%%%%%%%%%%%%%%%%%%%%%%%%%%%%%%%%%%%%%%%%%%%%%%%%%%%%%%%%%%%%%%%%%%%%%%%%%%%%
\usepackage[sort,nocompress]{cite}
%\usepackage[sort,nocompress,space]{cite}
%%%%%%%%%%%%%%%%%%%%%%%%%%%%%%%%%%%%%%%%%%%%%%%%%%%%%%%%%%%%%%%%%%%%%%%%%%%%%%%%%%%%%%%%%%%%%


%%%%%%%%%%%%%%%%%%%%%%%%%%%%%%%%%%%%%%%%%%%%%%%%%%%%%%%%%%%%%%%%%%%%%%%%%%%%%%%%%%%%%%%%%%%%%
\usepackage{enumitem}
%% The following can be uncommented instead of using paralist package...but cannot use together.
%\newlist{compactenum}{enumerate}{4}
%\setlist[compactenum,1]{nolistsep}
\setlist[enumerate,1]{leftmargin=*,wide=0em, noitemsep,nolistsep, label = {\bfseries \arabic*.}}
\setlist[itemize,1]{leftmargin=*,wide=0em, noitemsep,nolistsep}
%%%%%%%%%%%%%%%%%%%%%%%%%%%%%%%%%%%%%%%%%%%%%%%%%%%%%%%%%%%%%%%%%%%%%%%%%%%%%%%%%%%%%%%%%%%%%


%%%%%%%%%%%%%%%%%%%%%%%%%%%%%%%%%%%%%%%%%%%%%%%%%%%%%%%%%%%%%%%%%%%%%%%%%%%%%%%%%%%%%%%%%%%%%
\usepackage{titlesec}
\titleformat*{\section}{\large\bfseries}
\titleformat*{\subsection}{\large\bfseries}
\titleformat*{\subsubsection}{\large\bfseries}
\titleformat*{\paragraph}{\normalsize\bfseries}
\titleformat*{\subparagraph}{\normalsize\bfseries}
%%%%%%%%%%%%%%%%%%%%%%%%%%%%%%%%%%%%%%%%%%%%%%%%%%%%%%%%%%%%%%%%%%%%%%%%%%%%%%%%%%%%%%%%%%%%%


%\newcommand {\ourname} {{\bf OurName\ }}

\def\Mm#1{\mbox{\boldmath$\scriptstyle #1$\unboldmath}} % Math-bold in subscript
\def\MM#1{\mbox{\boldmath$#1$\unboldmath}} % Math-bold


%%%%%%%%%%%%%%%%%%%%%%%%%%%%%%%%%%%%%%%%%%%%%%%%%%%%%%%%%%%%%%%%%%%%%%%%%%%%%%%%%%%%%%%%%%%%%
\newcommand {\uu}  { {\bf u} }
\newcommand {\zz}  { {\bf z} }
\newcommand {\bgg}  { {\bf g} }
\newcommand {\NN}  { {\cal N} }
\newcommand {\xx}  { {\bf x} }
\renewcommand {\aa}  { {\bf a} }
\newcommand {\PP}  { {\bf P} }
\newcommand {\ii}  { {\bf i} }
\newcommand {\jj}  { {\bf j} }
\newcommand {\yy}  { {\bf y} }
\newcommand {\hh}  { {\bf h} }
\newcommand {\kk}  { {\bf k} }\newcommand {\rr}  { {\bf r} }
\newcommand {\EE}  { {\bf E} }
\newcommand {\HH}  { {\bf H} }
\newcommand {\JJ}  { {\bf J} }
\newcommand{\R}{{\rm I\!R}}
\newcommand{\im}{{\cal I}{\rm m} }
\newcommand {\cS}  { {\cal S} }
\newcommand {\cC}  { {\cal C} }
\newcommand {\cK}  { {\cal K} }
\newcommand {\cX}  { {\cal X} }
\newcommand {\KH}  { {\mathcal K}_H }
\newcommand {\KG}  { {\mathcal K}_G }
\newcommand {\KU}  { {\mathcal K}_U }
\newcommand\argmin[1]  {\underset{#1}{\operatorname{arg\ min}}}
%\newcommand {\grad}  { {\rm grad} \,}
%\newcommand {\curl}  { {\rm curl} \,}
\newcommand {\veps} {\bm \epsilon}
\newcommand {\Ro} { {\cal R}}
\newcommand {\Ex} { {\mathbb E} }
\newcommand {\lag} { {\cal L}}
\newcommand {\sci} { {\rm i}}
\newcommand {\qq}  { {\bf q} }
\newcommand {\QQ}  { {\bf Q} }
\newcommand {\Ab}  { {\bf A} }
\newcommand {\pp}  { {\bf p} }
\newcommand {\mm}  { {\bf m} }
\newcommand {\vv}  { {\bf v} }
\newcommand {\ww}  { {\bf w} }
\newcommand {\ff}  { {\bf f} }
\newcommand {\FF}  { {\bf F} }
\newcommand {\BB}  { {\bf B} }
\newcommand {\DD}  { {\bf D} }
\newcommand {\ZZ}  { {\bf Z} }
\newcommand {\bb}  { {\bf b} }
\newcommand {\cc}  { {\bf c} }
\newcommand {\dd}  { {\bf d} }
\newcommand {\ee}  { {\bf e} }
\newcommand {\sa}  { {\bf s} }
\newcommand {\blambda}  { {\boldsymbol \lambda} }
\newcommand {\bmu}  { {\boldsymbol \mu} }
\newcommand {\zero}  { {\bf 0} }
\newcommand {\one}  { {\bf 1} }
\newcommand {\gb}  { {\bf g} }
\newcommand {\bnabla} { { \boldsymbol \nabla} }
\newcommand {\btheta} { { \boldsymbol \theta} }
\newcommand {\bdelta} { { \boldsymbol \delta} }
\newcommand {\bepsilon} { { \boldsymbol \epsilon} }
\newcommand {\bxi} { { \boldsymbol \xi} }
\newcommand {\fuu}  { {\frac{\partial (f_u^T\lambda)}{\partial u}} }
\newcommand {\fum}  { {\frac{\partial (f_u^T\lambda)}{\partial m}} }
\newcommand {\fmm}  { {\frac{\partial (f_m^T\lambda)}{\partial m}} }
\newcommand {\sg}{{\sigma}}
\newcommand{\hf}{\frac12}
\newcommand{\hx}[1]{{\ensuremath{h^x_{\scriptscriptstyle #1}}}}
\newcommand{\hy}[1]{{\ensuremath{h^y_{\scriptscriptstyle #1}}}}
\newcommand{\hz}[1]{{\ensuremath{h^z_{\scriptscriptstyle #1}}}}
\newcommand{\x}[1]{\ensuremath{x_{\scriptscriptstyle #1}}}
\newcommand{\y}[1]{\ensuremath{y_{\scriptscriptstyle #1}}}
\newcommand{\z}[1]{\ensuremath{z_{\scriptscriptstyle #1}}}
\renewcommand{\vec}[1]{\ensuremath{\mathbf{#1}}}
\newcommand{\A}{\vec{A}}
\renewcommand{\H}{\vec{H}}
\newcommand{\J}{\vec{J}}
\newcommand{\F}{\vec{F}}
\newcommand{\s}{\vec{s}}
%\newcommand{\curl}{\ensuremath{\nabla\times\,}}
%\newcommand{\grad}{\ensuremath{\bnabla}}
\newcommand{\sig}{\sigma}
\newcommand{\hsig}{\widehat \sigma}
\newcommand{\hJ}{\widehat{\vec{J}}}
%\newcommand {\bnabla} { { \boldsymbol \nabla} }
\newcommand{\nn}{\vec{n}}
\renewcommand{\div}{\nabla\cdot\,}
\newcommand{\grad}{\ensuremath {\vec \nabla}}
\newcommand{\curl}{\ensuremath{{\vec \nabla}\times\,}}
\newcommand\CP[2]{#1\times#2}         % cross product
\newcommand\DP[2]{(#1\cdot#2)}        % dot product
%\newcommand\grad{\nabla}              % gradient symbol
\newcommand\interval[2]{[#1\dots#2]}  % intervals
\newcommand\M[1]{{\bf#1}}             % matrix
\newcommand\Pt[1]{{\bf#1}}            % point
\newcommand\V[1]{\vec{#1}}            % vector
\newcommand\neighbor{\mathcal{N}}
\newcommand{\defeq}{\mathrel{\mathop:}=}
\newcommand{\defeqr}{=\mathrel{\mathop:}}
\newcommand{\opnsubset}{\mathrel{\ooalign{$\subset$\cr
  \hidewidth\hbox{$\circ\mkern.5mu$}\cr}}}
\renewcommand{\Pr}{\hbox{\bf{Pr}}}


%%%%%%%%%%%%%%%%%%%%%%%%%%%%%%%%%%%%%%%%%%%%%%%%%%%%%%%%%%%%%%%%%%%%%%%%%%%%%%%%%%%%%%%%%%%%%  

\newcommand{\fred}[1]{\textcolor{red}{Fred:\ #1}}

%%%%%%%%%%%%%%%%%%%%%%%%%%%%%%%%%%%%%%%%%%%%%%%%%%%%%%%%%%%%%%%%%%%%%%%%%%%%%%%%%%%%%%%%%%%%%  
\newcommand{\changeurlcolor}[1]{\hypersetup{urlcolor=#1}}   

\newcommand{\red}[1]{\textcolor{red}{#1}}

\newcommand{\blue}[1]{\textcolor{blue}{#1}}

\definecolor{forestgreen}{rgb}{0.13, 0.55, 0.13}
\newcommand{\forestgreen}[1]{\textcolor{forestgreen}{#1}}

%%%%%%%%%%%%%%%%%%%%%%%%%%%%%%%%%%%%%%%%%%%%%%%%%%%%%%%%%%%%%%%%%%%%%%%%%%%%%%%%%%%%%%%%%%%%%  
\newcounter{cmt}
\setcounter{cmt}{0}
%\newcommand{\Comment}[1]{\refstepcounter{cmt}\textbf{Comment \arabic{cmt}: }#1}
\marginparwidth=0.75in
\newcommand{\Comment}[2]{\addtocounter{cmt}{1}{\color{brown}#1}\marginpar{\smaller\noindent{\raggedright{\color{brown}[\arabic{cmt}]}\newline\color{brown}{#2}\par}}}

%%%%%%%%%%%%%%%%%%%%%%%%%%%%%%%%%%%%%%%%%%%%%%%%%%%%%%%%%%%%%%%%%%%%%%%%%%%%%%%%%%%%%%%%%%%%%  

%\theoremclass{Theorem}
%\theoremstyle{break}
%\newframedtheorem{theorem}{Theorem}
%\newframedtheorem{corollary}{Corollary}
%\newframedtheorem{lemma}{Lemma}
%\newframedtheorem{definition}{Definition}
%\newframedtheorem{proposition}{Proposition}
%\newframedtheorem{assumption}{Assumption}
%%\newtheorem{theorem}{Theorem}
%%\newtheorem{conjecture}[theorem]{Conjecture}
%%\newtheorem{corollary}[theorem]{Corollary}
%%\newtheorem{proposition}[theorem]{Proposition}
%%\newtheorem{lemma}[theorem]{Lemma}
%%\newtheorem{definition}{Definition}
%\newtheorem{example}{Example}
%%\newtheorem{experiment}{Experiment}

%\newenvironment{proof}[1][Proof]{\begin{trivlist}
%\item[\hskip \labelsep {\bfseries #1}]}{\end{trivlist}}
%\newenvironment{remark}[1][Remark]{\begin{trivlist}
%\item[\hskip \labelsep {\bfseries #1}]}{\end{trivlist}}

%\newcommand{\qed}{\nobreak \ifvmode \relax \else
%      \ifdim\lastskip<1.5em \hskip-\lastskip
%      \hskip1.5em plus0em minus0.5em \fi \nobreak
%      \vrule height0.75em width0.5em depth0.25em\fi}
%%%%%%%%%%%%%%%%%%%%%%%%%%%%%%%%%%%%%%%%%%%%%%%%%%%%%%%%%%%%%%%%%%%%%%%%%%%%%%%%%%%%%%%%%%%%%  

\usepackage{tcolorbox} % for boxed text
\tcbuselibrary{breakable}
\tcbuselibrary{skins}
% use by 
% \begin{tcolorbox}[breakable,enhanced]
%\end{tcolorbox}

%%%%%%%%%%%%%%%%%%%%%%%%%%%%%%%%%%%%%%%%%%%%%%%%%%%%%%%%%%%%%%%%%%%%%%%%%%%%%%%%%%%%%%%%%%%%
\usepackage{listings} % to inser code

\definecolor{mygreen}{rgb}{0,0.6,0}
\definecolor{mygray}{rgb}{0.5,0.5,0.5}
\definecolor{mymauve}{rgb}{0.58,0,0.82}

\lstset{ %
  backgroundcolor=\color{white},   % choose the background color; you must add \usepackage{color} or \usepackage{xcolor}; should come as last argument
  basicstyle=\footnotesize,        % the size of the fonts that are used for the code
  breakatwhitespace=false,         % sets if automatic breaks should only happen at whitespace
  breaklines=true,                 % sets automatic line breaking
  captionpos=b,                    % sets the caption-position to bottom
  commentstyle=\color{mygreen},    % comment style
  deletekeywords={...},            % if you want to delete keywords from the given language
  escapeinside={\%*}{*)},          % if you want to add LaTeX within your code
  extendedchars=true,              % lets you use non-ASCII characters; for 8-bits encodings only, does not work with UTF-8
  frame=single,	                   % adds a frame around the code
  keepspaces=true,                 % keeps spaces in text, useful for keeping indentation of code (possibly needs columns=flexible)
  keywordstyle=\color{blue},       % keyword style
  language=Octave,                 % the language of the code
  morekeywords={*,...},           % if you want to add more keywords to the set
  numbers=left,                    % where to put the line-numbers; possible values are (none, left, right)
  numbersep=5pt,                   % how far the line-numbers are from the code
  numberstyle=\tiny\color{mygray}, % the style that is used for the line-numbers
  rulecolor=\color{black},         % if not set, the frame-color may be changed on line-breaks within not-black text (e.g. comments (green here))
  showspaces=false,                % show spaces everywhere adding particular underscores; it overrides 'showstringspaces'
  showstringspaces=false,          % underline spaces within strings only
  showtabs=false,                  % show tabs within strings adding particular underscores
  stepnumber=2,                    % the step between two line-numbers. If it's 1, each line will be numbered
  stringstyle=\color{mymauve},     % string literal style
  tabsize=2,	                   % sets default tabsize to 2 spaces
  title=\lstname                   % show the filename of files included with \lstinputlisting; also try caption instead of title
}

%%%%%%%%%%%%%%%%%%%%%%%%%%%%%%%%%%%%%%%%%%%%%%%%%%%%%%%%%%%%%%%%%%%%%%%%%%%%%%%%%%%5
\newcommand\modelspace{{\cal M}}
\newcommand*\lin[1]{\langle #1\rangle}

% see the list of further useful packages in the Reference Guide
\makeindex             % used for the subject index
% please use the style svind.ist with your makeindex program

% \usepackage{natbib}
%%%%%%%%%%%%%%%%%%%%%%%%%%%%%%%%%%%%%%%%%%%%%%%%%%%%%%%%%%%%%%%%%%%%
\begin{document}
\title*{Optimization Methods for Inverse Problems}
% Use \titlerunning{Short Title} for an abbreviated version of
% your contribution title if the original one is too long
\author{Nan Ye and Farbod Roosta-Khorasani and Tiangang Cui}
% Use \authorrunning{Short Title} for an abbreviated version of
% your contribution title if the original one is too long
\institute{Nan Ye \at ACEMS \& Queensland University of TechnologyInstitute, \email{n.ye@qut.edu.au}
\and Farbod Roosta-Khorasani \at University of Queensland \email{fred.roostauq.edu.au}
\and Tiangang Cui \at Monash University \email{tiangang.cui@monash.edu}}
\maketitle{}

\abstract{
Optimization plays an important role in solving many inverse problems. Indeed, the task of inversion often either involves or is fully cast as a solution of an optimization problem. In this light, the mere non-linear, non-convex, and large-scale nature of many of these inversions gives rise to some very challenging optimization problems.
The inverse problem community has long been developing various techniques for
solving such optimization tasks. However, other, seemingly disjoint communities,
such as that of machine learning, have developed, almost in parallel, interesting alternative methods which might have stayed under the radar of the inverse problem community. In this survey, we aim to change that. In doing so, we first discuss current state-of-the-art optimization methods widely used in inverse problems. We then survey recent related advances in addressing similar challenges in problems faced by the machine learning community, and discuss their potential advantages for solving inverse problems. 
By highlighting the similarities among the optimization challenges faced by the inverse problem and the machine learning communities, we hope that this survey can serve as a bridge in bringing together these two communities and encourage cross fertilization of ideas.}

\section{Introduction}  \label{sec:introduction}

\newcommand\inexpIntro[3]{#1?(#2,#3).}
\newcommand\rinexpIntro[3]{*#1?(#2,#3).}
\newcommand\outexpIntro[3]{#1!(#2,#3).}
\newcommand\outatomIntro[3]{#1!(#2,#3)}

We propose a fully automated method for proving termination of \(\pi\)-calculus processes.
Although there have been a lot of studies on termination analysis for the \(\pi\)-calculus
and related calculi~\cite{Deng06IC,Demangeon07,SangiorgiTermination,KobayashiHybrid,Yoshida04IC,DBLP:journals/jlp/DemangeonHS10,Venet98SAS}, most of them have been rather theoretical,
and there have been surprisingly little efforts in developing  fully automated termination
verification methods and tools based on them. To our knowledge,
Kobayashi's \typical{}~\cite{TyPiCal,KobayashiHybrid} is the only exception that
can prove termination of \(\pi\)-calculus processes (extended with natural numbers)
fully automatically, but its termination analysis is quite limited (see Section~\ref{sec:relatedwork}).

Our method is based on a reduction to termination analysis for sequential programs:
we translate a \(\pi\)-calculus process \(P\) to a sequential program \(S_P\), so that
if \(S_P\) is terminating, so is \(P\). The reduction allows us to use
powerful, mature methods and tools
for termination analysis of sequential programs~\cite{heizmann2016ultimate,freqterm,DBLP:conf/lics/PodelskiR04,Kuwahara2014Termination,DBLP:journals/cacm/CookPR11}.

The idea of the translation is to convert a chain of communications on replicated input
channels to a chain of recursive function calls of the target sequential program.
Let us consider the following Fibonacci process:
\begin{align*}
    & \rinexpIntro{\fib}{n}{r}
        \ifexp{n<2}{ \soutatom{r}{1} \\ &\quad}
                   { \nuexp{s_1} \nuexp{s_2} (\outatomIntro{\fib}{n-1}{s_1} \PAR \outatomIntro{\fib}{n-2}{s_2} \PAR \sinexp{s_1}{x}\sinexp{s_2}{y}\soutatom{r}{x+y}) \\}
    & \PAR \outatomIntro{\fib}{m}{r}
\end{align*}
Here, the process
$\rinexpIntro{\fib}{n}{r} \ldots$ is a function server that computes the \(n\)-th Fibonacci number
in parallel and returns the result to \(r\),
and $\outatom{\fib}{m}{r}$ sends a request for computing the \(m\)-th Fibonacci number;
those who are not familiar with the syntax of the \(\pi\)-calculus may wish to consult
Section~\ref{sec:targetlanguage} first.
To prove that the process above is terminating for any integer \(m\),
it suffices to show that there is no infinite chain of communications on $\fib$:
\[
    \fib(m,r) \to \fib(m_1,r_1) \to \fib(m_2,r_2) \to \cdots.
\]
We convert the process above to the following program:\footnote{The actual translation
  given later is a little more complex.}
\begin{verbatim}
 let rec fib(n) = if n<2 then () else (fib(n-1) [] fib(n-2)) in
 fib(m)
\end{verbatim}
Here, \texttt{[]} represents the non-deterministic choice.
Note that, although the calculation of Fibonacci numbers is not preserved,
for each chain of communications on \texttt{fib}, there is a corresponding
sequence of recursive calls:
\[
\mathtt{fib}(m) \to \mathtt{fib}(m_1) \to \mathtt{fib}(m_2) \to \cdots.
\]
Thus, the termination of the sequential program above implies the termination of
the original process.
As shown in the example above, (i) each communication on a replicated input channel
is converted to a function call, (ii) each communication on a non-replicated input
channel is just removed (or, in the actual translation, replaced by a call of
a trivial function defined by \(f(\seq{x})=(\,)\)), and (iii) parallel composition
is replaced by a non-deterministic choice.
We formalize the translation outlined above and prove its correctness.

The basic translation sketched above sometimes loses too much information.
For example, consider the following process:
\begin{align*}
    & \rinexpIntro{\pre}{n}{r} \soutatom{r}{n-1} \\
    & \PAR \rinexpIntro{f}{n}{r} \ifexp{n<0}{ \soutatom{r}{1} }
                                       { \nuexp{s} (\outatomIntro{\pre}{n}{s} \PAR \sinexp{s}{x}\outatomIntro{f}{x}{r}) } \\
    & \PAR \outatomIntro{f}{m}{r}
\end{align*}
The translation sketched above would yield:
\begin{verbatim}
  let pred(n) = n-1 in
  let rec f(n) = if n<0 then () else (pred(n) [] f(*)) in
  f(m)
\end{verbatim}
Here, \texttt{*} represents a non-deterministic integer: since we have removed
the input $\sinatom{s}{x}$, we do not have information about the value of \( x \).
As a result, the sequential program above is non-terminating, although the original
process is terminating.
To remedy this problem, we also refine the basic translation above by using a refinement
type system for the \(\pi\)-calculus. Using the refinement type system,
we can infer that the value of \(x\) in the original process is less than \(n\),
so that we can refine the definition of \texttt{f} to:
\begin{verbatim}
 let rec f(n) = ... else (pred(n) [] let x=* in assume(x<n);f(x))
\end{verbatim}
The target program is now terminating, from which
we can deduce that the original process is also terminating.
We have implemented an automated tool based on the refined translation above.

The contributions of this paper are summarized as follows.
\begin{itemize}
\item The formalization of the basic translation from the \(\pi\)-calculus
  (extended with integers) to sequential programs, and a proof of its correctness.
\item The formalization of a refined translation based on a refinement type system.
\item An implementation of the refined translation, including automated refinement type
  inference based on CHC solving, and experiments to evaluate the effectiveness of
  our method.
\end{itemize}

The rest of this paper is structured as follows.
Section~\ref{sec:targetlanguage} introduces the source and target languages
of our translation.
Section~\ref{sec:approach} 
formalizes the basic translation, and proves its correctness.
Section~\ref{sec:refinement} refines the basic translation by using a refinement type system.
Section~\ref{sec:implementation} reports an implementation and experiments.
Section~\ref{sec:relatedwork} discusses related work,
and Section~\ref{sec:conclusion} concludes the paper.

\section{Inverse Problems}
\label{sec:inv_prob}

An inverse problem can be seen as the reverse process of a forward problem,
which concerns with predicting the outcome of some measurements given a complete
description of a physical system.
Mathematically, a physical system is often specified using a set of model
parameters $\mm$ whose values completely characterize the system.
The model space $\modelspace$ is the set of possible values of $\mm$.
While $\mm$ is usually arise as a parameter function, in practice it is often
discretized as a parameter vector for the ease of computation, typically using
the finite element method, the finite volume method, or the finite difference
method.
The forward problem can be denoted as 
\begin{equation} \label{eq:forward_prob}
	\mm \to \dd = \ff(\mm),
\end{equation}
where $\dd$ are the error-free predictions, and the above notation is a
shorthand for 
$\dd 
= (\dd_1, \ldots, \dd_s) 
= (\ff_1(\mm), \ldots, \ff_s(\mm))$,
with $\dd_{i} \in \mathbb{R}^{l}$ being the $i$-th measurement.
The function $\ff$ represents the physical theory used for the prediction and is
called the forward operator.
%The forward operator can be linear or, more interestingly, non-linear in parameter $\mm$. 
The observed outcomes contain noises and relate to the system via the following
the observation equation
\begin{equation} \label{eq:obs}
	\dd = \ff(\mm) + \bm{\eta},
\end{equation}
where $\bm{\eta}$ are the noises occurred in the measurements.
The inverse problem aims to recover the model parameters $\mm$ from such noisy
measurements.

The inverse problem is almost always ill-posed, because the same measurements
can often be predicted by different models. 
There are two main approaches to deal with this issue.
The Bayesian approach assumes a prior distribution $P(\mm)$ on the model and a
conditional distribution $P(\bm{\eta} \mid \mm)$ on noise given the model.
The latter is equivalent to a conditional distribution $P(\dd \mid \mm)$ on
measurements given the model.
Given some measurements $\dd$, a posterior distribution 
$P(\mm \mid \dd)$ on the models is then computed using the Bayes rule
\begin{equation}
	P(\mm \mid \dd) 
		\propto P(\mm) P(\dd \mid \mm).
\end{equation}
Another approach sees the inverse problem as a data fitting problem that finds
an parameter vector $\mm$ that gives predictions $\ff(\mm)$ that best fit the
observed outcomes $\dd$ in some sense.
This is often cast as an optimization problem
\begin{equation} \label{eq:invopt}
	\min_{\mm \in \modelspace}\quad 
		\psi(\mm, \dd),
\end{equation}
where the misfit function $\psi$ measures how well the model $\mm$ fits the data
$\dd$.
When there is a probabilistic model of $\dd$ given $\mm$, a typical choice of
$\psi(\mm, \dd)$ is the negative log-likelihood. 
Regularization is often used to address the issue of multiple solutions, and 
additionally has the benefit of stabilizing the solution, that is, the
solution is less likely to change significantly in the presence of outliers
\cite{vogelbook,ehn1, archer1995some}. 
Regularization incorporats some \textit{a priori} information on $\mm$ in the
form of a regularizer $R(\mm)$ and solves the regularized optimization
problem
\begin{equation} \label{eq:reg_invopt1}
	\min_{\mm \in \modelspace}\quad 
		\psi_{R,\alpha}(\mm, \dd) 
			\defeq \psi(\mm, \dd) + \alpha R(\mm),
\end{equation}
where $\alpha > 0$ is a constant that controls the tradeoff between prior
knowledge and the fitness to data.
The regularizer $R(\mm)$ encodes a preference over the models, with preferred
models having smaller $R$ values.
The formulation in \Cref{eq:reg_invopt1} can often be given a \textit{maximum a
posteriori (MAP)} interpretation within the Bayesian framework~\cite{scharf1991statistical}. 
Implicit regularization also exists in which there is no explicit term $R(\mm)$
in the objective~\cite{hansen1998,doas, doas5,rieder2005,rieder2010,hanke1}.

The misfit function often has the form $\phi(\ff(\mm), \dd)$, which
measures the difference between the prediction $\ff(\mm)$ and the observation
$\dd$.
For example, $\phi$ may be chosen to be the Euclidean distance between
$\ff(\mm)$ and $\dd$.
In this case, the regularized problem takes the form
\begin{equation} \label{eq:reg_invopt2}
	\min_{\mm \in \modelspace}\quad 
		\phi_{R,\alpha}(\mm, \dd) 
			\defeq \phi(\ff(\mm), \dd) + \alpha R(\mm),
\end{equation}
This can also be equivalently formulated as choosing the most preferred model
satisfying constraints on its predictions
\begin{equation} \label{eq:reg_invopt3}
	\min_{\mm \in \modelspace}\quad R(\mm), \quad
	 \text{ s.t. }\quad \phi(\ff(\mm), \dd) \le \rho.
\end{equation}
The constant $\rho$ usually relates to noise and the maximum discrepancy between
the measured and the predicted data, and can be more intuitive than $\alpha$.

\subsection{PDE-Contrained Inverse Problems}
\label{sec:application}
For many inverse problems in science and engineering, the forward model is not
given explicitly via a forward operator $\ff(\mm)$, but often conveniently
specified via a set of partial differential equations (PDEs).
For such problems, \Cref{eq:reg_invopt2} has the form
\begin{equation} \label{eq:pde-invopt}
	\min_{\mm \in \modelspace, \uu}\quad 
		\phi(P \cdot \uu, \dd) + \alpha R(\mm), \quad
			\text{ s.t. }\quad c_{i}(\mm, \uu_{i}) = 0, \quad i=1, \ldots, s,
\end{equation}
where $P \cdot \uu 
= (P_1, \ldots, P_s) \cdot (\uu_1, \ldots, \uu_s)
= (P_1 \uu_1, \ldots, P_s \uu_s)$ with $\uu_{i}$ being the field in the $i$-th
experiment, $P_{i}$ being the projection operator that selects fields at
measurement locations in $\dd_{i}$ (that is, $P_{i} \uu_{i}$ are the predicted values
at locations measured in $\dd_{i}$), and $c_{i}(\mm, \uu_{i}) = 0$ corresponds to the
forward model in the $i$-th experiment.
In practice, the forward model can often be written as 
\begin{equation} \label{eq:lin-model} 
	{\cal L}_{i}(\mm) \uu_{i} = \qq_{i}, \quad i = 1, \ldots, s,
\end{equation}
where ${\cal L}_{i}(\mm)$ is a differential operator, and $\qq_{i}$ is a term that
incorporates source terms and boundary values.

The fields $\uu_1, \ldots, \uu_s$ in \Cref{eq:pde-invopt} and
\Cref{eq:lin-model} are generally functions in two or three dimensional spaces,
and finding closed-form solutions is usually not possible.
Instead, the PDE-constrained inverse problem is often solved numerically by
discretizing \Cref{eq:pde-invopt} and \Cref{eq:lin-model} using the finite
element method, the finite volume method, or the finite difference method.
Often the discretized PDE-constrained inverse problem takes the form
\begin{equation} \label{eq:discretized-constrained}
	\min_{\mm \in \modelspace, \uu}\quad 
		\phi(P \uu, \dd) + \alpha R(\mm), \quad
			\text{ s.t. }\quad L_{i}(\mm) \uu_{i} = \qq_{i}, \quad i=1, \ldots, s,
\end{equation}
where $P$ is a block-diagonal matrix consisting of diagonal blocks 
$P_{1}, \ldots, P_{s}$ representing the discretized projection operators,
$\uu$ is the concatenation of the vectors $\uu_1, \ldots, \uu_s$ representing
the discretized fields,
and each $L_{i}(\mm)$ is a square, non-singular matrix representing the
differential operator ${\cal L}_{i}(\mm)$.
Each $L_{i}(\mm)$ is typically large and sparse.
We abuse the notations $P$, $\uu$ to represent both functions and their
discretized versions, but the meanings of these notations will be clear from
context.

The constrained problem in \Cref{eq:discretized-constrained} can be written in
an unconstrained form by eliminating $\uu$ using $\uu_{i} = L_{i}^{-1} \qq_{i}$,
\begin{equation} \label{eq:discretized-unconstrained}
	\min_{\mm \in \modelspace}\quad 
		\phi(P L^{-1}(\mm) \qq, \dd) + \alpha R(\mm), 
\end{equation}
where $L$ is the block-diagonal matrix with $L_1, \ldots, L_s$ as the diagonal
blocks, and $\qq$ is the concatenation of $\qq_1, \ldots, \qq_s$. Note that, as in the case of~\eqref{eq:reg_invopt2}, here we have $ \ff(\mm) = P L^{-1}(\mm) \qq $.

Both the constrained and unconstrained formulations are used in practice.
The constrained formulation can be solved using the method of Lagrangian
multipliers.
This does not require explicitly solving the forward problem as in the
unconstrained formulation.
However, the problem size increases, and the problem becomes one of finding a
saddle point of the Lagrangian, instead of finding a minimum as in the
constrained formulation.

%\subsection{Assumptions on the Noise}
%\label{noise_assumption}
%The developments of the methods and algorithms presented in this thesis are done under one of the following assumptions on the noise. In what follows $\mathcal{N}$ denotes the normal distribution.
%\begin{enumerate}[label = ({N}.\arabic*)]
%	\item \label{noise_iid} 
%	The noise is independent and identically distributed (i.i.d) as 
%	$\bm{\eta}_{i} \sim \mathcal{N}(0,\Sigma ), \forall i$, where $\Sigma \in \mathbb{R}^{l \times l}$ is the 
%	symmetric positive definite covariance matrix. 
%	\item \label{noise_inid} 
%	The noise is independent but \textit{not} necessarily identically distributed,
%	satisfying instead $\bm{\eta}_{i} \sim \mathcal{N}(0,\sigma^{2}_{i} \mathbb{I}), i = 1,2,\ldots,s$, where
%	$\sigma_{i} > 0$ are the standard deviations. 
%\end{enumerate}
%
%Henceforth, for notational simplicity, most of the algorithms and methods are presented for the special case of Assumption~\hyperref[noise_iid]{(N.1)} with $\Sigma = \sigma \mathbb{I}$. However, all of these methods and algorithms can be readily extended to the more general cases in a completely straightforward manner.


\subsection{Image Reconstruction}
\label{sec:image_reconstruction}
Image reconstruction studies the creation of 2-D and 3-D images from sets
of 1-D projections. 
The 1-D projections are generally line integrals of a function representing the
image to be reconstructed.
In the 2-D case, given an image function $f(x, y)$, the integral along the line
at a distance of $s$ away from the origin and having a normal which forms an
angle $\phi$ with the $x$-axis is given by the Randon transform
\begin{equation}
	p(s, \phi) 
		= \int_{-\infty}^{\infty} f(z \sin \phi + s \cos \phi,
			-z \cos \phi + s \sin \phi) dz.
\end{equation}

Reconstruction is often done via back projection, filtered back projection, or
iterative methods \cite{natterer2001mathematical,herman2009fundamentals}.
Back projection is the simplest but often results in a blurred reconstruction.
Filtered back projection (FBP) is the analytical inversion of the Radon transform and
generally yields reconstructions of much better quality than back projection.
However, FBP may be infeasible in the presence of discontinuities or noise.
Iterative methods generally takes the noise into account and assumes a
distribution on the noise.
The objective function is often chosen to be a regularized likelihood of the
observation, which is then iteratively optimized using the expectation
maximization (EM) algorithm.

\subsection{Objective Function}
\label{sec:obj}
One of the most commonly used objective function is the least squares criterion,
which uses a quadratic loss and a quadratic regularizer.
Assume that the noise for each experiment in~\eqref{eq:obs} is independently but
normally distributed, i.e.,
$\bm{\eta}_{i} \sim \mathcal{N}(0,\Sigma_{i} ), \forall i$, where $\Sigma_{i}
\in \mathbb{R}^{l \times l}$ is the covariance matrix.
Let $\Sigma$ be the block-diagonal matrix with $\Sigma_{1}, \ldots, \Sigma_{s}$
as the diagonal blocks.
The standard \textit{maximum likelihood} (ML) approach~\cite{scharf1991statistical}, leads
to minimizing the least squares (LS) misfit function 
\begin{equation} \label{eq:l2}
	\phi(\mm) \defeq \|\ff(\mm) - \dd\|_{\Sigma^{-1}}^2,
\end{equation}
where the norm $\|x\|_{A} = \sqrt{x^{\top} A x}$ is a generalization of the Euclidean
norm (assuming the matrix $A$ is positive definite, which is true in the case of
$\Sigma_{i}^{-1}$). In the above equation, we simply write the general misfit function
$\phi(\ff(\mm), \dd)$ as $\phi(\mm)$ by taking the measurements $\dd$ as fixed
and omitting it from the notation.
As previously discussed, we often minimize a regularized misfit function
\begin{equation} \label{eq:reg_l2}
	\phi_{R,\alpha}(\mm) \defeq  \phi(\mm) + \alpha R(\mm).
\end{equation}
The prior $R(\mm)$ is often chosen as a Gaussian regularizer
$R(\mm) 
= (\mm - \mm_{\text{prior}})^{\top} \Sigma_{m}^{-1} (\mm - \mm_{\text{prior}})$.
We can also write the above optimization problem as minimizing $R(\mm)$ under
the constraints
\begin{equation} \label{eq:objective_eq}
	\sum_{i=1}^{s} \|\ff_{i}(\mm) - \dd_{i}\| \leq \rho.
\end{equation}

The least-squares criterion belongs to the class of $\ell_p$-norm
criteria, which contain two other commonly used criteria:
the least-absolute-values criterion and the minimax criterion
\cite{tarantola2005inverse}.
These correspond to the use of the $\ell_{1}$-norm and the $\ell_{\infty}$-norm
for the misfit function, while the least squares criterion uses the
$\ell_{2}$-norm.
Specifically, the least-absolute-values criterion takes 
$\phi(\mm) \defeq \|\ff(\mm) - \dd\|_1$, 
and the minimax criterion takes
$\phi(\mm) \defeq \|\ff(\mm) - \dd\|_{\infty}$.
More generally, each coordinate in the difference may be weighted.
The $\ell_{1}$ solution is more robust (that is, less sensitive to outliers) than the
$\ell_{2}$ solution, which is in turn more robust than the $\ell_{\infty}$ solution
\cite{claerbout1973robust}.
The $\ell_{\infty}$ norm is desirable when outliers are uncommon but the data
are corrupted by uniform noise such as the quantization errors
\cite{clason2012fitting}.

Besides the $\ell_{2}$ regularizer discussed above, the $\ell_{1}$-norm is
often used too.
The $\ell_{1}$ regularizer induces sparsity in the model parameters, that is,
heavier $\ell_{1}$ regularization leads to fewer non-zero model parameters.

\subsection{Optimization Algorithms}
\label{sec:optimization}

Various optimization techniques can be used to solve the regularized data
fitting problem.
We focus on iterative algorithms for nonlinear optimization below as the
objective functions are generally nonlinear.
In some cases, the optimization problem can be transformed to a linear program.
For example, linear programming can be used to solve the least-absolute-values
criterion or the minimax criterion.
However, linear programming are considered to have no advantage over
gradient-based methods (see Section 4.4.2 in \cite{tarantola2005inverse}), and
thus we do not discuss such methods here.
Nevertheless, there are still many optimization algorithms that can be covered
here, and we refer the readers to
\cite{bjorck1996numerical,nocedal2006numerical}.

For simplicity of presentation, we consider the problem of minimizing a
function $g(\mm)$.
We consider iterative algorithms which start with an iterate $\mm_0$, and
compute new iterates using
\begin{equation}
	\mm_{k+1} = \mm_{k} + \lambda_{k} p_{k},
\end{equation}
where $p_{k}$ is a search direction, and $\lambda_{k}$ a step size.
Unless otherwise stated, we focus on unconstrained optimization.
These algorithms can be used to directly solve the inverse problem in
\Cref{eq:reg_invopt1}. 
We only present a selected subset of the algorithms available and have to omit
many other interesting algorithms. 
% For example, we do not discuss optimization on level set \cite{doas1,doas5,of}.

\medskip\noindent{\bf Newton-type methods.}
The classical Newton's method starts with an initial iterate $\mm_{0}$, and computes new
iterates using 
\begin{equation}
	\mm_{k+1} = \mm_{k} - \left(\grad^2 g(\mm_{k})\right)^{-1} \grad g(\mm_{k}),
\end{equation}
that is, the search direction is 
$p_{k} = - \left(\grad^2 g(\mm_{k})\right)^{-1} \grad g(\mm_{k})$, and the step
length is $\lambda_{k} = 1$.
The basic Newton's method has quadratic local convergence rate at a small
neighborhood of a local minimum.
However, computing the search direction $p_{k}$ can be very expensive, and thus
many variants have been developed. In addition, in non-convex problems, classical Newton direction might not exist (if the Hessian matrix is not invertible) or it might not be an appropriate direction for descent (if the Hessian matrix is not positive definite). 

For non-linear least squares problems, where the objective function $g(\mm)$ is
a sum of squares of nonlinear functions, the Gauss-Newton (GN) method is often
used~\cite{sun2006optimization}.
Extensions to more general objective functions as in \Cref{eq:l2} with covariance matrix $ \Sigma $ and arbitrary regularization as in \Cref{eq:reg_l2} is considered in~\cite{roszas}.
Without loss of generality, assume 
$g(\mm) = \sum_{i=1}^{s} (\ff_{i}(\mm) - \dd_{i})^{2}$.
At iteration $k$, the GN search direction $p_{k}$ is given by
\begin{equation} 
	\left( \sum_{i=1}^{s} {J}_{i}^{\top} {J}_{i} \right) p_{k}  = -\grad g,
\end{equation}
where the sensitivity matrix $J_i$ and the gradient $\grad g$ are given by
\begin{align}
	J_i &= \frac{\partial \ff_{i}}{\partial \mm}(\mm_{k}), \quad i = 1, \ldots , s,\\
	\grad g &= 2 \sum_{i=1}^s J_{i}^T(\ff_{i}(\mm_{k}) - \dd_{i}),
\end{align}
The Gauss-Newton method can be seen as an approximation of the basic Newton's
method obtained by replacing $\grad^2 g$ by $\sum_{i=1}^{s} J_i^{\top} J_{i}$.
The step length $\lambda_{k} \in [0, 1]$ can be determined by a weak line
search~\cite{nocedal2006numerical} (using, say, the Armijo algorithm starting
with $\lambda_{k} = 1$) ensuring sufficient decrease in $g(\mm_{k+1})$ as
compared to $g(\mm_{k})$.

Often several nontrivial modifications are required to adapt this prototype method
for different applications, e.g., dynamic regularization~\cite{doas1,hanke1,rieder2005,rieder2010} and more general~\emph{stabilized GN} studied~\cite{rodoas1,doas12}.
This method replaces the solution of the linear systems defining $p_{k}$ by $r$
preconditioned conjugate gradient (PCG) inner iterations, which costs $2r$
solutions of the forward problem per iteration, for a moderate integer value $r$. 
Thus, if $K$ outer iterations are required to obtain an acceptable solution then
the total work estimate (in terms of the number of PDE solves) is approximated
{\em from below} by $2(r+1) K s$. 

Though Gauss-Newton is arguable the method of choice within the inverse problem community, other Newton-type methods exist which have been designed to suitably deal with the non-convex nature of the underlying optimization problem include Trust Region~\cite{conn2000trust,xu2017newton} and the Cubic Regularization~\cite{xu2017newton,cartis2012evaluation}. These methods have recently found applications in machine learning~\cite{xu2017second}. Studying the advantages/disadvantages of these non-convex methods for solving inverse problems can be indeed a useful undertaking.

\medskip\noindent{\bf Quasi-Newton methods.}
An alternative method to the above Newton-type methods is the quasi-Newton
variants including the celebrated limited memory BFGS (L-BFGS)
\cite{liu1989limited,nocedal1980updating}.
BFGS iteration is closely related to conjugate gradient (CG) iteration. In particular, BFGS applied to a strongly convex 
quadratic objective, with exact line search as well as initial Hessian $ P $, is equivalent to preconditioned CG with preconditioner $ P $. However, as the objective function departs from being a simple quadratic, the number of iterations of L-BFGS could be significantly higher than that of GN or trust region. In addition, it has been shown that the performance of BFGS and its limited memory version is greatly negatively affected by the high degree if ill-conditioning present in such problems~\cite{romassn1, romassn2,pyrrm_ssn_nonuni}. These two factor are among the main reasons why BFGS (and L-BFGS) can be less effective compared with other Newton-type alternatives in many inversion applications~\cite{haber2004quasi}.

\medskip\noindent{\bf Krylov subspace method.}
A Krylov subspace method iteratively finds the optimal solution to an
optimization in a larger subspace by making use of the previous solution in a
smaller subspace.
One of the most commonly used Krylov subspace method is the conjugate gradient
(CG) method.
CG was originally designed to solve convex quadratic minimization problems of
the form $g(\mm) = \frac{1}{2} \mm^{\top} A \mm - b^{\top} \mm$.
Equivalently, this solves the positive definite linear system $A\mm = b$.
It computes a sequence of iterates $\mm_0, \mm_1, \ldots$ converging to the minimum
through the following two set of equations.
\begin{align}
	\mm_0 &=0, & r_0 &= b, & p_0 &= r_0, & \\
  \mm_{k+1} &= \mm_{k} + \frac{||r_k||_2^2}{p_{k}^{\top} A p_{k}} p_{k}, 
    &
  r_{k+1} &= r_{k} - \frac{||r_k||_2^2}{p_{k}^{\top} A p_{k}} A p_k,
    &
	p_{k+1} &= r_{k+1} + \frac{||r_{k+1}||_2^2}{||r_{k}||_2^2} p_{k}, & k \ge 0.
\end{align}
This can be used to solve the forward problem of the form 
$L_{i}(\mm) \uu_{i} = \qq_{i}$, provided that $L_{i}(\mm)$ is positive definite, which
is true in many cases.

CG can be used to solve the linear system for the basic Newton direction.
However, the Hessian is not necessarily positive definite and modification is
needed \cite{nocedal2006numerical}.

In general, CG can be generalized to minimize a nonlinear function $g(\mm)$ 
\cite{fletcher2013practical,dai2011nonlinear}.
It starts with an arbitrary $\mm_0$, and $p_1 = - \grad g(\mm_0)$, and computes
a sequence of iterates $\mm_1, \mm_2, \ldots$ using the equations below:
for $k \ge 0$,
\begin{align}
	\mm_{k+1} 
		&= \arg\min_{\mm \in \{\mm_k + \lambda p_k, \lambda \in \mathbb{R}\}} g(\mm), \\
	p_{k+1}
		&= -\grad g(\mm_{k+1}) + \beta_{k} p_k, 
		\qquad\text{ where }
		\beta_k = \frac{||\grad g(\mm_{k+1})||_2^2}{||\grad g(\mm_{k})||_2^2}.
\end{align}
The above formula for $\beta_k$ is known as the Fletcher-Reeves formula.
Other choices of $\beta_k$ exist.
The following two formula are known as the Polak-Ribiere and Hestenes-Stiefel
formula respectively.
\begin{align}
    \beta_{k}
		&= \frac{\langle \grad g(\mm_{k+1}) - \grad g(\mm_{k}),  \grad g(\mm_{k+1}) \rangle}
        {||\grad g(\mm_{k})||_2^2}, \\
    \beta_{k}
		&= \frac{\langle \grad g(\mm_{k+1}) - \grad g(\mm_{k}),  \grad g(\mm_{k+1}) \rangle}
        {p_{k}^{\top} (\grad g(\mm_{k+1}) - \grad g(\mm_{k}))}.
\end{align}
In practice, nonlinear CG does not seem to work well, and is mainly used
together with other methods, such as in the Newton CG method
\cite{nocedal2006numerical}.

\medskip\noindent{\bf Lagrangian method of multipliers.}
The above discussion focuses on unconstrained optimization algorithms, which are
suitable for unconstrained formulations of inverse problems, or unconstrained
auxiliary optimization problems in methods which solves the constrained
formulations directly.
The Lagrangian method of multipliers is often used to directly solve the
constrained version.
Algorithms have been developed to offset the heavier computational cost and
slow convergence rates of standard algorthms observed on the Lagrangian, which
is a larger problem than the constrained problem.
For example, such algorithm may reduce the problem to a smaller one, such as
working with the reduced Hessian of the Lagrangian~\cite{haber2000optimization},
or preconditioning~\cite{haber2001preconditioned,benzi2011preconditioning}.
These methods have shown some success in certain PDE-constrained optimization
problems.

Augmented Lagrangian methods have also been developed
(e.g.~\cite{ito1990augmented,abdoulaev2005optical}).
These methods constructs a series of penalized Lagrangians with vanishing
penalty, and finds an optimizer of the Lagrangian by successively optimizing the
penalized Lagrangians.

\subsection{Challenges}
\label{sec:challenges}

\noindent{\bf Scaling up to large problems.}
The discretized version of an inverse problem is usually of very large scale,
and working with fine resolution or discretized problems in high dimension is
still an active area of research.

Another challenge is to scale up to large number of measurements, which is
widely believed to be helpful for quality reconstruction of the model in
practice, with some theoretical support.
While recent technological advances makes many big datasets available, existing
algorithms cannot efficiently cope with such datasets.
Examples of such problems include electromagnetic data inversion in mining
exploration \cite{na,dmr,haasol,olhash}, seismic data inversion in oil
exploration \cite{fichtner,hel,rnkkda}, diffuse optical tomography (DOT)
\cite{arridge1999optical,boas}, quantitative photo-acoustic tomography (QPAT)
\cite{gaooscher,yuan}, direct current (DC) resistivity
\cite{smvoz,pihakn,haheas,HaberChungHermann2010,doas12}, 
and electrical impedance tomography (EIT) \cite{bbp,cin,van2013lost}.

It has been suggested that many well-placed experiments yield practical advantage in order to obtain reconstructions of acceptable quality. 
For the special case where the measurement locations as well as the discretization matrices do not change from one experiment to another, various approximation techniques have been proposed to reduce the effective number of measurements, which in turn implies a smaller scale optimization problem, under the unifying category of ``simultaneous sources inversion'' \cite{rodoas1,roszas,roosta2015randomized,haber2014simultaneous,kumar2014GEOPemc}. Under certain circumstances, even if the $P_{i}$'s are different across experiments (but $ L_{i} $'s are fixed), there are methods to transform the existing data set 
into the one where all sources share the same receivers, \cite{rodoas2}. 

\medskip\noindent{\bf Dealing with non-convexity.}
Another major source of difficulty in solving many inverse problems, is the high-degree of non-linearity and non-convexity in~\eqref{eq:forward_prob}. This is most often encountered in problems involving PDE-constrained optimization where each $ \ff_{i} $ corresponds to the solution of a PDE. Even if the output of the PDE model itself, i.e., the ``right-hand side'', is linear in the sought-after parameter, the solution of the PDE, i.e., the forward problem, shows a great deal of non-linearity. This coupled with a great amount of non-convexity can have significant consequences in the quality of inversion and the obtained parameter. Indeed, in presence of non-convexity, the large-scale computational challenges are exacerbated, multiple folds over, by the difficulty of avoiding (possibly degenerate) \emph{saddle-points} as well as finding (at least) a \emph{local minimum}. 

\medskip\noindent{\bf Dealing with discontinuity.}
While the parameter function of the model is often smooth, the parameter
function can be discontinuous in some cases.
Such discontinuities arise very naturally as a result of the physical properties
of the underlying physical system, e.g., EIT and DC resistivity, and require
non-trivial modifications to optimization algorithms, e.g.,~\cite{rodoas1,
doas12}. 
Ignoring such discontinuities can lead to unsatisfactory recovery results
\cite{tali,doas12,doasleit2010}.
The level set method \cite{osse} is often used to model discontinuous parameter
function.
This reparametrizes the discontinuous parameter function as a differentiable
one, and thus enabling more stable optimization \cite{doasleit2010}.

\section{Recent Advances in Optimization}
\label{sec:advances}

Recent successes in using machine learning to deal with challenging perception
and natural language understanding problems have spurred many advances in the
study of optimization algorithms as optimization is a building block in machine
learning. 
These new developments include efficient methods for large-scale optimization, 
methods designed to handle non-convex problems, methods incorporating the structural
constraints, and finally the revival of second-order methods.
While these developments address a different set of applications in machine
learning, they address similar issues as encountered in inverse optimization and
could be useful.
We highlight some of the works below.
We keep the discussion brief because numerous works have been done behind these
developments and an indepth and comprehensive discussion is beyond the scope of
this review.
Our objective is thus to delineate the general trends and ideas, and provide
references for interested readers to dig on relevant topics.

\medskip\noindent{\bf Stochastic optimization.}
The development in large-scale optimization methods is driven by the
availability of many large datasets, which are made possible by the rapid
development and extensive use of IT technology. 
In machine learning, a model is generally built by optimizing a sum of misfit on
the examples.
This finite-sum structure naturally invites the application of stochastic optimization algorithms. This is mainly due to the fact that stochastic algorithms recover the sought-after models more efficiently by employing small batches of data in each iteration, as opposed to the whole data-set.
The most well-known stochastic gradient based algorithm is the stochastic gradient descent (SGD).
To minimize a finite-sum objective function 
\begin{align}
\label{eq:finite_sum}
g(\mm) = \frac{1}{n} \sum_{i=1}^{n} g_{i}(\mm),
\end{align} 
in the big data regime where $ n \gg 1 $, 
the vanilla SGD performs an update 
\begin{equation}
	\mm_{k+1} = \mm_{k} - \lambda_{k} \grad g_{i_{k}}(\mm_{k}),
\end{equation}
where $i_{k}$ is randomly sampled from $1, \ldots, n$.
As compared to gradient descent, SGD replaces the full gradient 
$\grad g(\mm)$ by a stochastic gradient $g_{i_{k}}(\mm_{k})$ with its
expectation being the full gradient.
The batch version of SGD constructs a stochastic gradient by taking the average
of several stochastic gradients.

SGD is inexpensive per iteration, but suffers from a slow rate of convergence.
For example, while full gradient descent achieves a linear convergence rate for
smooth strongly convex problems, SGD only converges at a sublinear rate.
The slow convergence rate can be partly accounted by the variance in the
stochastic gradient.
Recently, variance reduction techniques have been developed, e.g.
SVRG \cite{johnson2013accelerating} and SDCA \cite{shalev2013stochastic}.
Perhaps surprisingly, such variants can achieve linear convergence rates on
convex smooth problems as full gradient descent does, instead of sublinear rates
achieved by the vanilla SGD.
There are also a number of variants with no known linear rates but have fast
convergence rates for non-convex problems in practice, e.g.,
AdaGrad~\cite{duchi2011adaptive}, RMSProp~\cite{tijmen2012rmsprop},
ESGD~\cite{dauphin2015equilibrated}, Adam~\cite{kingma2014adam}, and
Adadelta~\cite{zeiler2012adadelta}.
Indeed, besides efficiency, stochastic optimization algorithms also seem to be able to
cope with the nonconvex objective functions well, and play a key role in the
revival of neural networks as deep learning~\cite{jin2017escape,ge2015escaping, levy2016power}.

\medskip\noindent{\bf Nonconvex optimization.}
There is also an increasing interest in non-convex optimization in the machine
learning community recently.
Nonconvex objectives not only naturally occur in deep learning, but also occur
in problems such as tensor decomposition, variable selection, low-rank matrix
completion, e.g. see \cite{ge2015escaping,mazumder2011sparsenet,jain2013low}
and references therein.

As discussed above, stochastic algorithms have been found to be capable of
effectively escaping local minima.
There are also a number of studies which adapt well-known acceleration
techniques for convex optimization to accelerate the convergence rates of both
stochastic and non-stochastic optimization algorithms for nonconvex problems,
e.g.,~\cite{li2015accelerated,allen2016variance,reddi2016stochastic,sutskever2013importance}.

\medskip\noindent{\bf Dealing with structural constraints.}
Many problems in machine learning come with complex structural constraints.
The Frank-Wolfe algorithm (a.k.a. conditional gradient)
\cite{frank1956algorithm} is an algorithm for optimizing over a convex domain.
It has gained a revived interest due to its ability to deal with many structural
constraints efficiently.
It requires solving a linear minimization problem over the feasible set, instead
of a quadratic program as in the case of proximal gradient algorithms or
projected gradient descent.
Domains suitable for the Frank-Wolfe algorithm include simplices,
$\ell_p$-balls, matrix nuclear norm ball, matrix operator norm ball
\cite{jaggi2013revisiting}.

The Frank-Wolfe algorithm belongs to the class of linear-optimization-based
algorithms \cite{lan2016conditional,lan2017conditional}.
These algorithms share with the Frank-Wolfe algorithm the characteristic of
requiring a first-order oracle for gradient computation and an oracle for
solving a linear optimization problem over the constraint set.

\medskip\noindent{\bf Second-order methods.}
The great appeal of the second-order methods lies mainly in the observed empirical performance as well as some very appealing theoretical properties. For example, it has been shown that stochastic Newton-type methods in general, and Gauss-Newton in particular, can not only be made scalable and have low per-iteration cost~\cite{rodoas1,rodoas2,roszas,haber2000optimization,haber2012effective,doas12}, but more importantly, and unlike first-order methods, are very \emph{resilient} to many adversarial effects such as \emph{ill-conditioning}~\cite{romassn1,romassn2,pyrrm_ssn_nonuni}. As a result, for moderately to very ill-conditioned problems, commonly found in scientific computing, while first-order methods make effectively no progress at all, second-order counterparts are not affected by the degree of ill-conditioning. 
A more subtle, yet potentially more severe draw-back in using first-order methods, is that their success is tightly intertwined with \emph{fine-tunning} (often many) \emph{hyper-parameters}, most importantly, the step-size~\cite{berahas2017investigation}. In fact, it is highly unlikely that many of these methods exhibit acceptable performance on first try, and it often takes many trials and errors before one can see reasonable results. In contrast, second-order optimization algorithms involve much less parameter tuning and are less sensitive to the choice of hyper-parameters~\cite{berahas2017investigation, xu2017second}. 

Since for the finite-sum problem~\eqref{eq:finite_sum} with $ n \gg 1 $, the operations with the Hessian/gradient constitute major computational bottlenecks, a rather more recent line of research is to construct the inexact Hessian information using the application of \emph{randomized methods}. Specifically, for convex optimization, the stochastic approximation of the full Hessian matrix in the classical Newton's method has been recently considered in~\cite{byrd2011use, byrd2012sample, wang2015subsampled,pilanci2015newton, erdogdu2015convergence, romassn1, romassn2, pyrrm_ssn_nonuni, Agarwal2016SecondOS, mutny2016stochastic, ye2016revisiting, bollapragada2016exact, mutny2017parallel, berahas2017investigation,eisen2017large}. In addition to inexact Hessian, a few of these methods study the fully stochastic case in which the gradient is also approximated, e.g.,~\cite{romassn1, romassn2,bollapragada2016exact}. For non-convex problems, however, the literature on methods that employ randomized Hessian approximation is significantly less developed than that of convex problems. A few recent examples include the stochastic trust region~\cite{xu2017newton}, stochastic cubic regularization~\cite{xu2017newton,tripuraneni2017stochastic}, and noisy negative curvature method~\cite{liu2017noisy}. Empirical performance of many of these methods for some non-convex machine learning applications has been considered in~\cite{xu2017second}.


In this paper, 2D and 3D CNN models were used to generate pelvic sCTs from T1-weighted MR images. Our sCT generation methods were fully automated, requiring no deformable registration or manual segmentation of bone tissues. As shown in Figure~\ref{fig3}, the 2D and 3D CNN models generated high quality sCTs. MAE curves shown in Figure~\ref{fig4} indicated that both models could precisely estimate soft-tissue HU values but had difficulty in reproducing air and high-density bone tissues. 

The MAEs within the body contour across all patients were 40.5 $\pm$ 5.4 HU and 37.6 $\pm$ 5.1 HU for the 2D and 3D models, respectively. The time required for generating a pelvic sCT using our CNN models was about 5.5 s. Our MAE results are comparable to previous studies. Kim $et \ al.$\cite{RN41} presented a voxel-based weighted summation method that produced an MAE of 74.3 $\pm$ 3.9 HU. However, manual contouring of bone tissues required for this method can be tedious and time-consuming. An MAE of 40.5 $\pm$ 8.2 HU was achieved by Dowling $et \ al.$\cite{RN11} using an average MRI-CT atlas from 38 patients. Andreasen $et \ al.$\cite{RN42} reported an MAE of 54 $\pm$ 8 HU using an atlas-based method with pattern recognition, and its prediction time was about 20.8 min. Another random forest model proposed by Andreasen $et \ al.$\cite{RN43} generated sCTs with an MAE of 58 $pm$ 9 HU. A hybrid method suggested by Siversson $et \ al.$ \cite{RN45} obtained an MAE of 36.5 $\pm$ 4.1 HU when ignoring errors introduced by gas cavities. This hybrid method was implemented in the cloud-based commercial software MriPlanner (Spectronic Medical AB, Helsingborg, Sweden), which required 50 to 80 min to generate a sCT.\cite{RN45} The patch-based 3D context-aware generative adversarial network presented by Nie $et \ al.$\cite{RN26} achieved an MAE of 39.0 $\pm$ 4.6 HU. 

Our CNN models reproduced low-density bone as shown in Figure ~\ref{fig4}. The bone-region DSCs were 0.81 $\pm$ 0.04 and 0.82 $\pm$ 0.04 from the 2D and 3D models, respectively. These results are comparable to reported DSC results of 0.79 $\pm$ 0.12\cite{RN10} and 0.91$\pm$0.03{\cite{RN11}}, where the authors compared bone contours manually drawn on the sCT and CT.

It was feasible to train the proposed 3D model with 16 image volumes from scratch. Results of the Wilcoxon signed-rank tests shown in Table~\ref{tab1} demonstrated a statistically significant improvement in overall MAE, bone DSC, and bone precision of the 3D model compared to the 2D model. However, as shown in Figure~\ref{fig4}, the 2D model seemed to perform better in estimating the high-density bone HU values. It should be noted that smaller overall MAEs do not guarantee improved sCT dose calculation and patient positioning performance. While the models performed well, we will continue to acquire more patient data to potentially improve model accuracy and further test model differences.

As this was a retrospective study, the MR image voxel sizes were not matched, resulting in different voxel intensities between images. This may have affected the sCT generation accuracy although we applied intensity normalization. A potential study could examine how voxel size variations affects sCT estimation. 

The proposed 3D model can be implemented on a 12 GB GPU to process volumetric images with dimensions of 256 $\times$ 256 $\times$ 30. More GPU memory would be required to process higher resolution 3D images. Considering the limited access to multi-GPU systems, a 3D architecture with fewer convolutional layers could be considered to deal with higher resolutions. However, the performance could be affected by the reduced parameters and smaller receptive fields of the less complex model. Another approach would be to extract 30-slice sub-volumes from CT and MR images for training the 3D model. The sCT could then be generated by averaging 30-slice sCT sub-volumes produced by the model. 

A number of techniques could be investigated for improving model performance.  Nie $et \ al.$\cite{RN26} showed that introducing an additional adversarial discriminator improved overall sCT quality. The same approach could be adapted in our proposed 2D and 3D CNN models.  Non-rigid deformation\cite{RN44} could also be applied to both CT and MR images in the process of the on-the-fly data augmentation to produce more training pairs. Multiple MR images acquired with different sequences could be fed into models to provide more information for distinguishing different tissues. Multi-GPU systems with more memory would enable the exploration of larger batch sizes for training CNN models, which could reduce variances in gradient estimation and accelerate the training. 



\begin{comment}
\begin{figure}
\includegraphics[width=\linewidth]{figs/beyond_tss_lesion.pdf}
\caption[]{End-to-End runtime lesion study of the entire MNIST dataset and the FMA featurized music dataset. Each of DROP's contributions provides a runtime improvement.}
\label{fig:beyond_lesion}
\end{figure}
\end{comment}



\section{Conclusion}
\label{sec:conclusion}

Advanced data analytics techniques must scale to rising data volumes. 
DR techniques offer a powerful toolkit when processing these datasets, with PCA frequently outperforming popular techniques in exchange for high computational cost. 
In response, we propose DROP, a new dimensionality reduction optimizer. 
DROP combines progressive sampling, progress estimation, and online aggregation to identify high quality low dimensional bases via PCA without processing the entire dataset by balancing the runtime of downstream tasks and achieved dimensionality. 
Thus, DROP provides a first step in bridging the gap between quality and efficiency in end-to-end DR for downstream \red{analytics}. 

%We revisit canonical operators for time series dimensionality reduction and the measurement study of~\cite{keogh-study}, and show that PCA is more effective than popular alternatives in the data mining literature often by a margin of over $2\times$ on average on gold-standard time series benchmark data sets with respect to output data dimension. More surprisingly, we empirically demonstrate that a small number of samples are sufficient to accurately characterize directions of maximum variance and obtain a high-quality low-dimensional transformation.




%\nocite{*}
\bibliographystyle{plain}
\bibliography{ref_Nan,ref_Fred}
\end{document}
