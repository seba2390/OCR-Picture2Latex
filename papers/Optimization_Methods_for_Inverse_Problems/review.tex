%%%%%%%%%%%%%%%%%%%%%%%%%%%%%%%%%%%%%%%%%%%%%%%%%%%
\documentclass[graybox]{svmult}
% choose options for [] as required from the list  in the Reference Guide
\usepackage{mathptmx}       % selects Times Roman as basic font
\usepackage{helvet}         % selects Helvetica as sans-serif font
\usepackage{courier}        % selects Courier as typewriter font
%\usepackage{type1cm}        % activate if the above 3 fonts are not available on your system
\usepackage{makeidx}         % allows index generation
\usepackage{graphicx}        % standard LaTeX graphics tool
\usepackage{multicol}        % used for the two-column index
\usepackage[bottom]{footmisc}% places footnotes at page bottom
\usepackage{amsmath,amssymb,amsfonts}        % AMS math packages

\usepackage{graphicx}
\usepackage{amsmath}
\usepackage{amssymb}
\usepackage{latexsym}
\usepackage{subfigure}
\usepackage{crop}
\usepackage{algorithmic}
\usepackage{algorithm}
\usepackage{multirow}
%\usepackage{algorithm,algorithmic}
%\usepackage[section,subsection,subsubsection]{placeins}
\usepackage{bm}
\usepackage{bbm}
\usepackage{enumerate}
\usepackage{framed} % or, "mdframed"
\usepackage[framed]{ntheorem}
%\usepackage{datetime}
\usepackage{url}
\usepackage[colorlinks = true, pdfstartview = FitV, linkcolor = blue, citecolor = blue, urlcolor = blue]{hyperref}
\usepackage{array}
\usepackage{paralist}
\usepackage{authblk}

\usepackage{cleveref}
\crefname{equation}{Eq.}{Eqs.}
\crefname{figure}{Fig.}{Figs.}
\usepackage{relsize}

%%%%%%%%%%%%%%%%%%%%%%%%%%%%%%%%%%%%%%%%%%%%%%%%%%%%%%%%%%%%%%%%%%%%%%%%%%%%%%%%%%%%%%%%%%%%%
\usepackage{fullpage}
%\def\spacingset#1{\renewcommand{\baselinestretch}{#1}\small\normalsize}
%\setlength{\topmargin}{-.50in}
%\setlength{\leftmargin}{0.0in}
%\setlength{\evensidemargin}{0.25in}
%\setlength{\oddsidemargin}{0.25in}
%\setlength{\textheight}{8.5in}
%\setlength{\textwidth}{6.0in}
%%%%%%%%%%%%%%%%%%%%%%%%%%%%%%%%%%%%%%%%%%%%%%%%%%%%%%%%%%%%%%%%%%%%%%%%%%%%%%%%%%%%%%%%%%%%%


%%%%%%%%%%%%%%%%%%%%%%%%%%%%%%%%%%%%%%%%%%%%%%%%%%%%%%%%%%%%%%%%%%%%%%%%%%%%%%%%%%%%%%%%%%%%%
\usepackage[sort,nocompress]{cite}
%\usepackage[sort,nocompress,space]{cite}
%%%%%%%%%%%%%%%%%%%%%%%%%%%%%%%%%%%%%%%%%%%%%%%%%%%%%%%%%%%%%%%%%%%%%%%%%%%%%%%%%%%%%%%%%%%%%


%%%%%%%%%%%%%%%%%%%%%%%%%%%%%%%%%%%%%%%%%%%%%%%%%%%%%%%%%%%%%%%%%%%%%%%%%%%%%%%%%%%%%%%%%%%%%
\usepackage{enumitem}
%% The following can be uncommented instead of using paralist package...but cannot use together.
%\newlist{compactenum}{enumerate}{4}
%\setlist[compactenum,1]{nolistsep}
\setlist[enumerate,1]{leftmargin=*,wide=0em, noitemsep,nolistsep, label = {\bfseries \arabic*.}}
\setlist[itemize,1]{leftmargin=*,wide=0em, noitemsep,nolistsep}
%%%%%%%%%%%%%%%%%%%%%%%%%%%%%%%%%%%%%%%%%%%%%%%%%%%%%%%%%%%%%%%%%%%%%%%%%%%%%%%%%%%%%%%%%%%%%


%%%%%%%%%%%%%%%%%%%%%%%%%%%%%%%%%%%%%%%%%%%%%%%%%%%%%%%%%%%%%%%%%%%%%%%%%%%%%%%%%%%%%%%%%%%%%
\usepackage{titlesec}
\titleformat*{\section}{\large\bfseries}
\titleformat*{\subsection}{\large\bfseries}
\titleformat*{\subsubsection}{\large\bfseries}
\titleformat*{\paragraph}{\normalsize\bfseries}
\titleformat*{\subparagraph}{\normalsize\bfseries}
%%%%%%%%%%%%%%%%%%%%%%%%%%%%%%%%%%%%%%%%%%%%%%%%%%%%%%%%%%%%%%%%%%%%%%%%%%%%%%%%%%%%%%%%%%%%%


%\newcommand {\ourname} {{\bf OurName\ }}

\def\Mm#1{\mbox{\boldmath$\scriptstyle #1$\unboldmath}} % Math-bold in subscript
\def\MM#1{\mbox{\boldmath$#1$\unboldmath}} % Math-bold


%%%%%%%%%%%%%%%%%%%%%%%%%%%%%%%%%%%%%%%%%%%%%%%%%%%%%%%%%%%%%%%%%%%%%%%%%%%%%%%%%%%%%%%%%%%%%
\newcommand {\uu}  { {\bf u} }
\newcommand {\zz}  { {\bf z} }
\newcommand {\bgg}  { {\bf g} }
\newcommand {\NN}  { {\cal N} }
\newcommand {\xx}  { {\bf x} }
\renewcommand {\aa}  { {\bf a} }
\newcommand {\PP}  { {\bf P} }
\newcommand {\ii}  { {\bf i} }
\newcommand {\jj}  { {\bf j} }
\newcommand {\yy}  { {\bf y} }
\newcommand {\hh}  { {\bf h} }
\newcommand {\kk}  { {\bf k} }\newcommand {\rr}  { {\bf r} }
\newcommand {\EE}  { {\bf E} }
\newcommand {\HH}  { {\bf H} }
\newcommand {\JJ}  { {\bf J} }
\newcommand{\R}{{\rm I\!R}}
\newcommand{\im}{{\cal I}{\rm m} }
\newcommand {\cS}  { {\cal S} }
\newcommand {\cC}  { {\cal C} }
\newcommand {\cK}  { {\cal K} }
\newcommand {\cX}  { {\cal X} }
\newcommand {\KH}  { {\mathcal K}_H }
\newcommand {\KG}  { {\mathcal K}_G }
\newcommand {\KU}  { {\mathcal K}_U }
\newcommand\argmin[1]  {\underset{#1}{\operatorname{arg\ min}}}
%\newcommand {\grad}  { {\rm grad} \,}
%\newcommand {\curl}  { {\rm curl} \,}
\newcommand {\veps} {\bm \epsilon}
\newcommand {\Ro} { {\cal R}}
\newcommand {\Ex} { {\mathbb E} }
\newcommand {\lag} { {\cal L}}
\newcommand {\sci} { {\rm i}}
\newcommand {\qq}  { {\bf q} }
\newcommand {\QQ}  { {\bf Q} }
\newcommand {\Ab}  { {\bf A} }
\newcommand {\pp}  { {\bf p} }
\newcommand {\mm}  { {\bf m} }
\newcommand {\vv}  { {\bf v} }
\newcommand {\ww}  { {\bf w} }
\newcommand {\ff}  { {\bf f} }
\newcommand {\FF}  { {\bf F} }
\newcommand {\BB}  { {\bf B} }
\newcommand {\DD}  { {\bf D} }
\newcommand {\ZZ}  { {\bf Z} }
\newcommand {\bb}  { {\bf b} }
\newcommand {\cc}  { {\bf c} }
\newcommand {\dd}  { {\bf d} }
\newcommand {\ee}  { {\bf e} }
\newcommand {\sa}  { {\bf s} }
\newcommand {\blambda}  { {\boldsymbol \lambda} }
\newcommand {\bmu}  { {\boldsymbol \mu} }
\newcommand {\zero}  { {\bf 0} }
\newcommand {\one}  { {\bf 1} }
\newcommand {\gb}  { {\bf g} }
\newcommand {\bnabla} { { \boldsymbol \nabla} }
\newcommand {\btheta} { { \boldsymbol \theta} }
\newcommand {\bdelta} { { \boldsymbol \delta} }
\newcommand {\bepsilon} { { \boldsymbol \epsilon} }
\newcommand {\bxi} { { \boldsymbol \xi} }
\newcommand {\fuu}  { {\frac{\partial (f_u^T\lambda)}{\partial u}} }
\newcommand {\fum}  { {\frac{\partial (f_u^T\lambda)}{\partial m}} }
\newcommand {\fmm}  { {\frac{\partial (f_m^T\lambda)}{\partial m}} }
\newcommand {\sg}{{\sigma}}
\newcommand{\hf}{\frac12}
\newcommand{\hx}[1]{{\ensuremath{h^x_{\scriptscriptstyle #1}}}}
\newcommand{\hy}[1]{{\ensuremath{h^y_{\scriptscriptstyle #1}}}}
\newcommand{\hz}[1]{{\ensuremath{h^z_{\scriptscriptstyle #1}}}}
\newcommand{\x}[1]{\ensuremath{x_{\scriptscriptstyle #1}}}
\newcommand{\y}[1]{\ensuremath{y_{\scriptscriptstyle #1}}}
\newcommand{\z}[1]{\ensuremath{z_{\scriptscriptstyle #1}}}
\renewcommand{\vec}[1]{\ensuremath{\mathbf{#1}}}
\newcommand{\A}{\vec{A}}
\renewcommand{\H}{\vec{H}}
\newcommand{\J}{\vec{J}}
\newcommand{\F}{\vec{F}}
\newcommand{\s}{\vec{s}}
%\newcommand{\curl}{\ensuremath{\nabla\times\,}}
%\newcommand{\grad}{\ensuremath{\bnabla}}
\newcommand{\sig}{\sigma}
\newcommand{\hsig}{\widehat \sigma}
\newcommand{\hJ}{\widehat{\vec{J}}}
%\newcommand {\bnabla} { { \boldsymbol \nabla} }
\newcommand{\nn}{\vec{n}}
\renewcommand{\div}{\nabla\cdot\,}
\newcommand{\grad}{\ensuremath {\vec \nabla}}
\newcommand{\curl}{\ensuremath{{\vec \nabla}\times\,}}
\newcommand\CP[2]{#1\times#2}         % cross product
\newcommand\DP[2]{(#1\cdot#2)}        % dot product
%\newcommand\grad{\nabla}              % gradient symbol
\newcommand\interval[2]{[#1\dots#2]}  % intervals
\newcommand\M[1]{{\bf#1}}             % matrix
\newcommand\Pt[1]{{\bf#1}}            % point
\newcommand\V[1]{\vec{#1}}            % vector
\newcommand\neighbor{\mathcal{N}}
\newcommand{\defeq}{\mathrel{\mathop:}=}
\newcommand{\defeqr}{=\mathrel{\mathop:}}
\newcommand{\opnsubset}{\mathrel{\ooalign{$\subset$\cr
  \hidewidth\hbox{$\circ\mkern.5mu$}\cr}}}
\renewcommand{\Pr}{\hbox{\bf{Pr}}}


%%%%%%%%%%%%%%%%%%%%%%%%%%%%%%%%%%%%%%%%%%%%%%%%%%%%%%%%%%%%%%%%%%%%%%%%%%%%%%%%%%%%%%%%%%%%%  

\newcommand{\fred}[1]{\textcolor{red}{Fred:\ #1}}

%%%%%%%%%%%%%%%%%%%%%%%%%%%%%%%%%%%%%%%%%%%%%%%%%%%%%%%%%%%%%%%%%%%%%%%%%%%%%%%%%%%%%%%%%%%%%  
\newcommand{\changeurlcolor}[1]{\hypersetup{urlcolor=#1}}   

\newcommand{\red}[1]{\textcolor{red}{#1}}

\newcommand{\blue}[1]{\textcolor{blue}{#1}}

\definecolor{forestgreen}{rgb}{0.13, 0.55, 0.13}
\newcommand{\forestgreen}[1]{\textcolor{forestgreen}{#1}}

%%%%%%%%%%%%%%%%%%%%%%%%%%%%%%%%%%%%%%%%%%%%%%%%%%%%%%%%%%%%%%%%%%%%%%%%%%%%%%%%%%%%%%%%%%%%%  
\newcounter{cmt}
\setcounter{cmt}{0}
%\newcommand{\Comment}[1]{\refstepcounter{cmt}\textbf{Comment \arabic{cmt}: }#1}
\marginparwidth=0.75in
\newcommand{\Comment}[2]{\addtocounter{cmt}{1}{\color{brown}#1}\marginpar{\smaller\noindent{\raggedright{\color{brown}[\arabic{cmt}]}\newline\color{brown}{#2}\par}}}

%%%%%%%%%%%%%%%%%%%%%%%%%%%%%%%%%%%%%%%%%%%%%%%%%%%%%%%%%%%%%%%%%%%%%%%%%%%%%%%%%%%%%%%%%%%%%  

%\theoremclass{Theorem}
%\theoremstyle{break}
%\newframedtheorem{theorem}{Theorem}
%\newframedtheorem{corollary}{Corollary}
%\newframedtheorem{lemma}{Lemma}
%\newframedtheorem{definition}{Definition}
%\newframedtheorem{proposition}{Proposition}
%\newframedtheorem{assumption}{Assumption}
%%\newtheorem{theorem}{Theorem}
%%\newtheorem{conjecture}[theorem]{Conjecture}
%%\newtheorem{corollary}[theorem]{Corollary}
%%\newtheorem{proposition}[theorem]{Proposition}
%%\newtheorem{lemma}[theorem]{Lemma}
%%\newtheorem{definition}{Definition}
%\newtheorem{example}{Example}
%%\newtheorem{experiment}{Experiment}

%\newenvironment{proof}[1][Proof]{\begin{trivlist}
%\item[\hskip \labelsep {\bfseries #1}]}{\end{trivlist}}
%\newenvironment{remark}[1][Remark]{\begin{trivlist}
%\item[\hskip \labelsep {\bfseries #1}]}{\end{trivlist}}

%\newcommand{\qed}{\nobreak \ifvmode \relax \else
%      \ifdim\lastskip<1.5em \hskip-\lastskip
%      \hskip1.5em plus0em minus0.5em \fi \nobreak
%      \vrule height0.75em width0.5em depth0.25em\fi}
%%%%%%%%%%%%%%%%%%%%%%%%%%%%%%%%%%%%%%%%%%%%%%%%%%%%%%%%%%%%%%%%%%%%%%%%%%%%%%%%%%%%%%%%%%%%%  

\usepackage{tcolorbox} % for boxed text
\tcbuselibrary{breakable}
\tcbuselibrary{skins}
% use by 
% \begin{tcolorbox}[breakable,enhanced]
%\end{tcolorbox}

%%%%%%%%%%%%%%%%%%%%%%%%%%%%%%%%%%%%%%%%%%%%%%%%%%%%%%%%%%%%%%%%%%%%%%%%%%%%%%%%%%%%%%%%%%%%
\usepackage{listings} % to inser code

\definecolor{mygreen}{rgb}{0,0.6,0}
\definecolor{mygray}{rgb}{0.5,0.5,0.5}
\definecolor{mymauve}{rgb}{0.58,0,0.82}

\lstset{ %
  backgroundcolor=\color{white},   % choose the background color; you must add \usepackage{color} or \usepackage{xcolor}; should come as last argument
  basicstyle=\footnotesize,        % the size of the fonts that are used for the code
  breakatwhitespace=false,         % sets if automatic breaks should only happen at whitespace
  breaklines=true,                 % sets automatic line breaking
  captionpos=b,                    % sets the caption-position to bottom
  commentstyle=\color{mygreen},    % comment style
  deletekeywords={...},            % if you want to delete keywords from the given language
  escapeinside={\%*}{*)},          % if you want to add LaTeX within your code
  extendedchars=true,              % lets you use non-ASCII characters; for 8-bits encodings only, does not work with UTF-8
  frame=single,	                   % adds a frame around the code
  keepspaces=true,                 % keeps spaces in text, useful for keeping indentation of code (possibly needs columns=flexible)
  keywordstyle=\color{blue},       % keyword style
  language=Octave,                 % the language of the code
  morekeywords={*,...},           % if you want to add more keywords to the set
  numbers=left,                    % where to put the line-numbers; possible values are (none, left, right)
  numbersep=5pt,                   % how far the line-numbers are from the code
  numberstyle=\tiny\color{mygray}, % the style that is used for the line-numbers
  rulecolor=\color{black},         % if not set, the frame-color may be changed on line-breaks within not-black text (e.g. comments (green here))
  showspaces=false,                % show spaces everywhere adding particular underscores; it overrides 'showstringspaces'
  showstringspaces=false,          % underline spaces within strings only
  showtabs=false,                  % show tabs within strings adding particular underscores
  stepnumber=2,                    % the step between two line-numbers. If it's 1, each line will be numbered
  stringstyle=\color{mymauve},     % string literal style
  tabsize=2,	                   % sets default tabsize to 2 spaces
  title=\lstname                   % show the filename of files included with \lstinputlisting; also try caption instead of title
}

%%%%%%%%%%%%%%%%%%%%%%%%%%%%%%%%%%%%%%%%%%%%%%%%%%%%%%%%%%%%%%%%%%%%%%%%%%%%%%%%%%%5
\newcommand\modelspace{{\cal M}}
\newcommand*\lin[1]{\langle #1\rangle}

% see the list of further useful packages in the Reference Guide
\makeindex             % used for the subject index
% please use the style svind.ist with your makeindex program

% \usepackage{natbib}
%%%%%%%%%%%%%%%%%%%%%%%%%%%%%%%%%%%%%%%%%%%%%%%%%%%%%%%%%%%%%%%%%%%%
\begin{document}
\title*{Optimization Methods for Inverse Problems}
% Use \titlerunning{Short Title} for an abbreviated version of
% your contribution title if the original one is too long
\author{Nan Ye and Farbod Roosta-Khorasani and Tiangang Cui}
% Use \authorrunning{Short Title} for an abbreviated version of
% your contribution title if the original one is too long
\institute{Nan Ye \at ACEMS \& Queensland University of TechnologyInstitute, \email{n.ye@qut.edu.au}
\and Farbod Roosta-Khorasani \at University of Queensland \email{fred.roostauq.edu.au}
\and Tiangang Cui \at Monash University \email{tiangang.cui@monash.edu}}
\maketitle{}

\abstract{
Optimization plays an important role in solving many inverse problems. Indeed, the task of inversion often either involves or is fully cast as a solution of an optimization problem. In this light, the mere non-linear, non-convex, and large-scale nature of many of these inversions gives rise to some very challenging optimization problems.
The inverse problem community has long been developing various techniques for
solving such optimization tasks. However, other, seemingly disjoint communities,
such as that of machine learning, have developed, almost in parallel, interesting alternative methods which might have stayed under the radar of the inverse problem community. In this survey, we aim to change that. In doing so, we first discuss current state-of-the-art optimization methods widely used in inverse problems. We then survey recent related advances in addressing similar challenges in problems faced by the machine learning community, and discuss their potential advantages for solving inverse problems. 
By highlighting the similarities among the optimization challenges faced by the inverse problem and the machine learning communities, we hope that this survey can serve as a bridge in bringing together these two communities and encourage cross fertilization of ideas.}

% \leavevmode
% \\
% \\
% \\
% \\
% \\
\section{Introduction}
\label{introduction}

AutoML is the process by which machine learning models are built automatically for a new dataset. Given a dataset, AutoML systems perform a search over valid data transformations and learners, along with hyper-parameter optimization for each learner~\cite{VolcanoML}. Choosing the transformations and learners over which to search is our focus.
A significant number of systems mine from prior runs of pipelines over a set of datasets to choose transformers and learners that are effective with different types of datasets (e.g. \cite{NEURIPS2018_b59a51a3}, \cite{10.14778/3415478.3415542}, \cite{autosklearn}). Thus, they build a database by actually running different pipelines with a diverse set of datasets to estimate the accuracy of potential pipelines. Hence, they can be used to effectively reduce the search space. A new dataset, based on a set of features (meta-features) is then matched to this database to find the most plausible candidates for both learner selection and hyper-parameter tuning. This process of choosing starting points in the search space is called meta-learning for the cold start problem.  

Other meta-learning approaches include mining existing data science code and their associated datasets to learn from human expertise. The AL~\cite{al} system mined existing Kaggle notebooks using dynamic analysis, i.e., actually running the scripts, and showed that such a system has promise.  However, this meta-learning approach does not scale because it is onerous to execute a large number of pipeline scripts on datasets, preprocessing datasets is never trivial, and older scripts cease to run at all as software evolves. It is not surprising that AL therefore performed dynamic analysis on just nine datasets.

Our system, {\sysname}, provides a scalable meta-learning approach to leverage human expertise, using static analysis to mine pipelines from large repositories of scripts. Static analysis has the advantage of scaling to thousands or millions of scripts \cite{graph4code} easily, but lacks the performance data gathered by dynamic analysis. The {\sysname} meta-learning approach guides the learning process by a scalable dataset similarity search, based on dataset embeddings, to find the most similar datasets and the semantics of ML pipelines applied on them.  Many existing systems, such as Auto-Sklearn \cite{autosklearn} and AL \cite{al}, compute a set of meta-features for each dataset. We developed a deep neural network model to generate embeddings at the granularity of a dataset, e.g., a table or CSV file, to capture similarity at the level of an entire dataset rather than relying on a set of meta-features.
 
Because we use static analysis to capture the semantics of the meta-learning process, we have no mechanism to choose the \textbf{best} pipeline from many seen pipelines, unlike the dynamic execution case where one can rely on runtime to choose the best performing pipeline.  Observing that pipelines are basically workflow graphs, we use graph generator neural models to succinctly capture the statically-observed pipelines for a single dataset. In {\sysname}, we formulate learner selection as a graph generation problem to predict optimized pipelines based on pipelines seen in actual notebooks.

%. This formulation enables {\sysname} for effective pruning of the AutoML search space to predict optimized pipelines based on pipelines seen in actual notebooks.}
%We note that increasingly, state-of-the-art performance in AutoML systems is being generated by more complex pipelines such as Directed Acyclic Graphs (DAGs) \cite{piper} rather than the linear pipelines used in earlier systems.  
 
{\sysname} does learner and transformation selection, and hence is a component of an AutoML systems. To evaluate this component, we integrated it into two existing AutoML systems, FLAML \cite{flaml} and Auto-Sklearn \cite{autosklearn}.  
% We evaluate each system with and without {\sysname}.  
We chose FLAML because it does not yet have any meta-learning component for the cold start problem and instead allows user selection of learners and transformers. The authors of FLAML explicitly pointed to the fact that FLAML might benefit from a meta-learning component and pointed to it as a possibility for future work. For FLAML, if mining historical pipelines provides an advantage, we should improve its performance. We also picked Auto-Sklearn as it does have a learner selection component based on meta-features, as described earlier~\cite{autosklearn2}. For Auto-Sklearn, we should at least match performance if our static mining of pipelines can match their extensive database. For context, we also compared {\sysname} with the recent VolcanoML~\cite{VolcanoML}, which provides an efficient decomposition and execution strategy for the AutoML search space. In contrast, {\sysname} prunes the search space using our meta-learning model to perform hyperparameter optimization only for the most promising candidates. 

The contributions of this paper are the following:
\begin{itemize}
    \item Section ~\ref{sec:mining} defines a scalable meta-learning approach based on representation learning of mined ML pipeline semantics and datasets for over 100 datasets and ~11K Python scripts.  
    \newline
    \item Sections~\ref{sec:kgpipGen} formulates AutoML pipeline generation as a graph generation problem. {\sysname} predicts efficiently an optimized ML pipeline for an unseen dataset based on our meta-learning model.  To the best of our knowledge, {\sysname} is the first approach to formulate  AutoML pipeline generation in such a way.
    \newline
    \item Section~\ref{sec:eval} presents a comprehensive evaluation using a large collection of 121 datasets from major AutoML benchmarks and Kaggle. Our experimental results show that {\sysname} outperforms all existing AutoML systems and achieves state-of-the-art results on the majority of these datasets. {\sysname} significantly improves the performance of both FLAML and Auto-Sklearn in classification and regression tasks. We also outperformed AL in 75 out of 77 datasets and VolcanoML in 75  out of 121 datasets, including 44 datasets used only by VolcanoML~\cite{VolcanoML}.  On average, {\sysname} achieves scores that are statistically better than the means of all other systems. 
\end{itemize}


%This approach does not need to apply cleaning or transformation methods to handle different variances among datasets. Moreover, we do not need to deal with complex analysis, such as dynamic code analysis. Thus, our approach proved to be scalable, as discussed in Sections~\ref{sec:mining}.
\section{Inverse Problems}
\label{sec:inv_prob}

An inverse problem can be seen as the reverse process of a forward problem,
which concerns with predicting the outcome of some measurements given a complete
description of a physical system.
Mathematically, a physical system is often specified using a set of model
parameters $\mm$ whose values completely characterize the system.
The model space $\modelspace$ is the set of possible values of $\mm$.
While $\mm$ is usually arise as a parameter function, in practice it is often
discretized as a parameter vector for the ease of computation, typically using
the finite element method, the finite volume method, or the finite difference
method.
The forward problem can be denoted as 
\begin{equation} \label{eq:forward_prob}
	\mm \to \dd = \ff(\mm),
\end{equation}
where $\dd$ are the error-free predictions, and the above notation is a
shorthand for 
$\dd 
= (\dd_1, \ldots, \dd_s) 
= (\ff_1(\mm), \ldots, \ff_s(\mm))$,
with $\dd_{i} \in \mathbb{R}^{l}$ being the $i$-th measurement.
The function $\ff$ represents the physical theory used for the prediction and is
called the forward operator.
%The forward operator can be linear or, more interestingly, non-linear in parameter $\mm$. 
The observed outcomes contain noises and relate to the system via the following
the observation equation
\begin{equation} \label{eq:obs}
	\dd = \ff(\mm) + \bm{\eta},
\end{equation}
where $\bm{\eta}$ are the noises occurred in the measurements.
The inverse problem aims to recover the model parameters $\mm$ from such noisy
measurements.

The inverse problem is almost always ill-posed, because the same measurements
can often be predicted by different models. 
There are two main approaches to deal with this issue.
The Bayesian approach assumes a prior distribution $P(\mm)$ on the model and a
conditional distribution $P(\bm{\eta} \mid \mm)$ on noise given the model.
The latter is equivalent to a conditional distribution $P(\dd \mid \mm)$ on
measurements given the model.
Given some measurements $\dd$, a posterior distribution 
$P(\mm \mid \dd)$ on the models is then computed using the Bayes rule
\begin{equation}
	P(\mm \mid \dd) 
		\propto P(\mm) P(\dd \mid \mm).
\end{equation}
Another approach sees the inverse problem as a data fitting problem that finds
an parameter vector $\mm$ that gives predictions $\ff(\mm)$ that best fit the
observed outcomes $\dd$ in some sense.
This is often cast as an optimization problem
\begin{equation} \label{eq:invopt}
	\min_{\mm \in \modelspace}\quad 
		\psi(\mm, \dd),
\end{equation}
where the misfit function $\psi$ measures how well the model $\mm$ fits the data
$\dd$.
When there is a probabilistic model of $\dd$ given $\mm$, a typical choice of
$\psi(\mm, \dd)$ is the negative log-likelihood. 
Regularization is often used to address the issue of multiple solutions, and 
additionally has the benefit of stabilizing the solution, that is, the
solution is less likely to change significantly in the presence of outliers
\cite{vogelbook,ehn1, archer1995some}. 
Regularization incorporats some \textit{a priori} information on $\mm$ in the
form of a regularizer $R(\mm)$ and solves the regularized optimization
problem
\begin{equation} \label{eq:reg_invopt1}
	\min_{\mm \in \modelspace}\quad 
		\psi_{R,\alpha}(\mm, \dd) 
			\defeq \psi(\mm, \dd) + \alpha R(\mm),
\end{equation}
where $\alpha > 0$ is a constant that controls the tradeoff between prior
knowledge and the fitness to data.
The regularizer $R(\mm)$ encodes a preference over the models, with preferred
models having smaller $R$ values.
The formulation in \Cref{eq:reg_invopt1} can often be given a \textit{maximum a
posteriori (MAP)} interpretation within the Bayesian framework~\cite{scharf1991statistical}. 
Implicit regularization also exists in which there is no explicit term $R(\mm)$
in the objective~\cite{hansen1998,doas, doas5,rieder2005,rieder2010,hanke1}.

The misfit function often has the form $\phi(\ff(\mm), \dd)$, which
measures the difference between the prediction $\ff(\mm)$ and the observation
$\dd$.
For example, $\phi$ may be chosen to be the Euclidean distance between
$\ff(\mm)$ and $\dd$.
In this case, the regularized problem takes the form
\begin{equation} \label{eq:reg_invopt2}
	\min_{\mm \in \modelspace}\quad 
		\phi_{R,\alpha}(\mm, \dd) 
			\defeq \phi(\ff(\mm), \dd) + \alpha R(\mm),
\end{equation}
This can also be equivalently formulated as choosing the most preferred model
satisfying constraints on its predictions
\begin{equation} \label{eq:reg_invopt3}
	\min_{\mm \in \modelspace}\quad R(\mm), \quad
	 \text{ s.t. }\quad \phi(\ff(\mm), \dd) \le \rho.
\end{equation}
The constant $\rho$ usually relates to noise and the maximum discrepancy between
the measured and the predicted data, and can be more intuitive than $\alpha$.

\subsection{PDE-Contrained Inverse Problems}
\label{sec:application}
For many inverse problems in science and engineering, the forward model is not
given explicitly via a forward operator $\ff(\mm)$, but often conveniently
specified via a set of partial differential equations (PDEs).
For such problems, \Cref{eq:reg_invopt2} has the form
\begin{equation} \label{eq:pde-invopt}
	\min_{\mm \in \modelspace, \uu}\quad 
		\phi(P \cdot \uu, \dd) + \alpha R(\mm), \quad
			\text{ s.t. }\quad c_{i}(\mm, \uu_{i}) = 0, \quad i=1, \ldots, s,
\end{equation}
where $P \cdot \uu 
= (P_1, \ldots, P_s) \cdot (\uu_1, \ldots, \uu_s)
= (P_1 \uu_1, \ldots, P_s \uu_s)$ with $\uu_{i}$ being the field in the $i$-th
experiment, $P_{i}$ being the projection operator that selects fields at
measurement locations in $\dd_{i}$ (that is, $P_{i} \uu_{i}$ are the predicted values
at locations measured in $\dd_{i}$), and $c_{i}(\mm, \uu_{i}) = 0$ corresponds to the
forward model in the $i$-th experiment.
In practice, the forward model can often be written as 
\begin{equation} \label{eq:lin-model} 
	{\cal L}_{i}(\mm) \uu_{i} = \qq_{i}, \quad i = 1, \ldots, s,
\end{equation}
where ${\cal L}_{i}(\mm)$ is a differential operator, and $\qq_{i}$ is a term that
incorporates source terms and boundary values.

The fields $\uu_1, \ldots, \uu_s$ in \Cref{eq:pde-invopt} and
\Cref{eq:lin-model} are generally functions in two or three dimensional spaces,
and finding closed-form solutions is usually not possible.
Instead, the PDE-constrained inverse problem is often solved numerically by
discretizing \Cref{eq:pde-invopt} and \Cref{eq:lin-model} using the finite
element method, the finite volume method, or the finite difference method.
Often the discretized PDE-constrained inverse problem takes the form
\begin{equation} \label{eq:discretized-constrained}
	\min_{\mm \in \modelspace, \uu}\quad 
		\phi(P \uu, \dd) + \alpha R(\mm), \quad
			\text{ s.t. }\quad L_{i}(\mm) \uu_{i} = \qq_{i}, \quad i=1, \ldots, s,
\end{equation}
where $P$ is a block-diagonal matrix consisting of diagonal blocks 
$P_{1}, \ldots, P_{s}$ representing the discretized projection operators,
$\uu$ is the concatenation of the vectors $\uu_1, \ldots, \uu_s$ representing
the discretized fields,
and each $L_{i}(\mm)$ is a square, non-singular matrix representing the
differential operator ${\cal L}_{i}(\mm)$.
Each $L_{i}(\mm)$ is typically large and sparse.
We abuse the notations $P$, $\uu$ to represent both functions and their
discretized versions, but the meanings of these notations will be clear from
context.

The constrained problem in \Cref{eq:discretized-constrained} can be written in
an unconstrained form by eliminating $\uu$ using $\uu_{i} = L_{i}^{-1} \qq_{i}$,
\begin{equation} \label{eq:discretized-unconstrained}
	\min_{\mm \in \modelspace}\quad 
		\phi(P L^{-1}(\mm) \qq, \dd) + \alpha R(\mm), 
\end{equation}
where $L$ is the block-diagonal matrix with $L_1, \ldots, L_s$ as the diagonal
blocks, and $\qq$ is the concatenation of $\qq_1, \ldots, \qq_s$. Note that, as in the case of~\eqref{eq:reg_invopt2}, here we have $ \ff(\mm) = P L^{-1}(\mm) \qq $.

Both the constrained and unconstrained formulations are used in practice.
The constrained formulation can be solved using the method of Lagrangian
multipliers.
This does not require explicitly solving the forward problem as in the
unconstrained formulation.
However, the problem size increases, and the problem becomes one of finding a
saddle point of the Lagrangian, instead of finding a minimum as in the
constrained formulation.

%\subsection{Assumptions on the Noise}
%\label{noise_assumption}
%The developments of the methods and algorithms presented in this thesis are done under one of the following assumptions on the noise. In what follows $\mathcal{N}$ denotes the normal distribution.
%\begin{enumerate}[label = ({N}.\arabic*)]
%	\item \label{noise_iid} 
%	The noise is independent and identically distributed (i.i.d) as 
%	$\bm{\eta}_{i} \sim \mathcal{N}(0,\Sigma ), \forall i$, where $\Sigma \in \mathbb{R}^{l \times l}$ is the 
%	symmetric positive definite covariance matrix. 
%	\item \label{noise_inid} 
%	The noise is independent but \textit{not} necessarily identically distributed,
%	satisfying instead $\bm{\eta}_{i} \sim \mathcal{N}(0,\sigma^{2}_{i} \mathbb{I}), i = 1,2,\ldots,s$, where
%	$\sigma_{i} > 0$ are the standard deviations. 
%\end{enumerate}
%
%Henceforth, for notational simplicity, most of the algorithms and methods are presented for the special case of Assumption~\hyperref[noise_iid]{(N.1)} with $\Sigma = \sigma \mathbb{I}$. However, all of these methods and algorithms can be readily extended to the more general cases in a completely straightforward manner.


\subsection{Image Reconstruction}
\label{sec:image_reconstruction}
Image reconstruction studies the creation of 2-D and 3-D images from sets
of 1-D projections. 
The 1-D projections are generally line integrals of a function representing the
image to be reconstructed.
In the 2-D case, given an image function $f(x, y)$, the integral along the line
at a distance of $s$ away from the origin and having a normal which forms an
angle $\phi$ with the $x$-axis is given by the Randon transform
\begin{equation}
	p(s, \phi) 
		= \int_{-\infty}^{\infty} f(z \sin \phi + s \cos \phi,
			-z \cos \phi + s \sin \phi) dz.
\end{equation}

Reconstruction is often done via back projection, filtered back projection, or
iterative methods \cite{natterer2001mathematical,herman2009fundamentals}.
Back projection is the simplest but often results in a blurred reconstruction.
Filtered back projection (FBP) is the analytical inversion of the Radon transform and
generally yields reconstructions of much better quality than back projection.
However, FBP may be infeasible in the presence of discontinuities or noise.
Iterative methods generally takes the noise into account and assumes a
distribution on the noise.
The objective function is often chosen to be a regularized likelihood of the
observation, which is then iteratively optimized using the expectation
maximization (EM) algorithm.

\subsection{Objective Function}
\label{sec:obj}
One of the most commonly used objective function is the least squares criterion,
which uses a quadratic loss and a quadratic regularizer.
Assume that the noise for each experiment in~\eqref{eq:obs} is independently but
normally distributed, i.e.,
$\bm{\eta}_{i} \sim \mathcal{N}(0,\Sigma_{i} ), \forall i$, where $\Sigma_{i}
\in \mathbb{R}^{l \times l}$ is the covariance matrix.
Let $\Sigma$ be the block-diagonal matrix with $\Sigma_{1}, \ldots, \Sigma_{s}$
as the diagonal blocks.
The standard \textit{maximum likelihood} (ML) approach~\cite{scharf1991statistical}, leads
to minimizing the least squares (LS) misfit function 
\begin{equation} \label{eq:l2}
	\phi(\mm) \defeq \|\ff(\mm) - \dd\|_{\Sigma^{-1}}^2,
\end{equation}
where the norm $\|x\|_{A} = \sqrt{x^{\top} A x}$ is a generalization of the Euclidean
norm (assuming the matrix $A$ is positive definite, which is true in the case of
$\Sigma_{i}^{-1}$). In the above equation, we simply write the general misfit function
$\phi(\ff(\mm), \dd)$ as $\phi(\mm)$ by taking the measurements $\dd$ as fixed
and omitting it from the notation.
As previously discussed, we often minimize a regularized misfit function
\begin{equation} \label{eq:reg_l2}
	\phi_{R,\alpha}(\mm) \defeq  \phi(\mm) + \alpha R(\mm).
\end{equation}
The prior $R(\mm)$ is often chosen as a Gaussian regularizer
$R(\mm) 
= (\mm - \mm_{\text{prior}})^{\top} \Sigma_{m}^{-1} (\mm - \mm_{\text{prior}})$.
We can also write the above optimization problem as minimizing $R(\mm)$ under
the constraints
\begin{equation} \label{eq:objective_eq}
	\sum_{i=1}^{s} \|\ff_{i}(\mm) - \dd_{i}\| \leq \rho.
\end{equation}

The least-squares criterion belongs to the class of $\ell_p$-norm
criteria, which contain two other commonly used criteria:
the least-absolute-values criterion and the minimax criterion
\cite{tarantola2005inverse}.
These correspond to the use of the $\ell_{1}$-norm and the $\ell_{\infty}$-norm
for the misfit function, while the least squares criterion uses the
$\ell_{2}$-norm.
Specifically, the least-absolute-values criterion takes 
$\phi(\mm) \defeq \|\ff(\mm) - \dd\|_1$, 
and the minimax criterion takes
$\phi(\mm) \defeq \|\ff(\mm) - \dd\|_{\infty}$.
More generally, each coordinate in the difference may be weighted.
The $\ell_{1}$ solution is more robust (that is, less sensitive to outliers) than the
$\ell_{2}$ solution, which is in turn more robust than the $\ell_{\infty}$ solution
\cite{claerbout1973robust}.
The $\ell_{\infty}$ norm is desirable when outliers are uncommon but the data
are corrupted by uniform noise such as the quantization errors
\cite{clason2012fitting}.

Besides the $\ell_{2}$ regularizer discussed above, the $\ell_{1}$-norm is
often used too.
The $\ell_{1}$ regularizer induces sparsity in the model parameters, that is,
heavier $\ell_{1}$ regularization leads to fewer non-zero model parameters.

\subsection{Optimization Algorithms}
\label{sec:optimization}

Various optimization techniques can be used to solve the regularized data
fitting problem.
We focus on iterative algorithms for nonlinear optimization below as the
objective functions are generally nonlinear.
In some cases, the optimization problem can be transformed to a linear program.
For example, linear programming can be used to solve the least-absolute-values
criterion or the minimax criterion.
However, linear programming are considered to have no advantage over
gradient-based methods (see Section 4.4.2 in \cite{tarantola2005inverse}), and
thus we do not discuss such methods here.
Nevertheless, there are still many optimization algorithms that can be covered
here, and we refer the readers to
\cite{bjorck1996numerical,nocedal2006numerical}.

For simplicity of presentation, we consider the problem of minimizing a
function $g(\mm)$.
We consider iterative algorithms which start with an iterate $\mm_0$, and
compute new iterates using
\begin{equation}
	\mm_{k+1} = \mm_{k} + \lambda_{k} p_{k},
\end{equation}
where $p_{k}$ is a search direction, and $\lambda_{k}$ a step size.
Unless otherwise stated, we focus on unconstrained optimization.
These algorithms can be used to directly solve the inverse problem in
\Cref{eq:reg_invopt1}. 
We only present a selected subset of the algorithms available and have to omit
many other interesting algorithms. 
% For example, we do not discuss optimization on level set \cite{doas1,doas5,of}.

\medskip\noindent{\bf Newton-type methods.}
The classical Newton's method starts with an initial iterate $\mm_{0}$, and computes new
iterates using 
\begin{equation}
	\mm_{k+1} = \mm_{k} - \left(\grad^2 g(\mm_{k})\right)^{-1} \grad g(\mm_{k}),
\end{equation}
that is, the search direction is 
$p_{k} = - \left(\grad^2 g(\mm_{k})\right)^{-1} \grad g(\mm_{k})$, and the step
length is $\lambda_{k} = 1$.
The basic Newton's method has quadratic local convergence rate at a small
neighborhood of a local minimum.
However, computing the search direction $p_{k}$ can be very expensive, and thus
many variants have been developed. In addition, in non-convex problems, classical Newton direction might not exist (if the Hessian matrix is not invertible) or it might not be an appropriate direction for descent (if the Hessian matrix is not positive definite). 

For non-linear least squares problems, where the objective function $g(\mm)$ is
a sum of squares of nonlinear functions, the Gauss-Newton (GN) method is often
used~\cite{sun2006optimization}.
Extensions to more general objective functions as in \Cref{eq:l2} with covariance matrix $ \Sigma $ and arbitrary regularization as in \Cref{eq:reg_l2} is considered in~\cite{roszas}.
Without loss of generality, assume 
$g(\mm) = \sum_{i=1}^{s} (\ff_{i}(\mm) - \dd_{i})^{2}$.
At iteration $k$, the GN search direction $p_{k}$ is given by
\begin{equation} 
	\left( \sum_{i=1}^{s} {J}_{i}^{\top} {J}_{i} \right) p_{k}  = -\grad g,
\end{equation}
where the sensitivity matrix $J_i$ and the gradient $\grad g$ are given by
\begin{align}
	J_i &= \frac{\partial \ff_{i}}{\partial \mm}(\mm_{k}), \quad i = 1, \ldots , s,\\
	\grad g &= 2 \sum_{i=1}^s J_{i}^T(\ff_{i}(\mm_{k}) - \dd_{i}),
\end{align}
The Gauss-Newton method can be seen as an approximation of the basic Newton's
method obtained by replacing $\grad^2 g$ by $\sum_{i=1}^{s} J_i^{\top} J_{i}$.
The step length $\lambda_{k} \in [0, 1]$ can be determined by a weak line
search~\cite{nocedal2006numerical} (using, say, the Armijo algorithm starting
with $\lambda_{k} = 1$) ensuring sufficient decrease in $g(\mm_{k+1})$ as
compared to $g(\mm_{k})$.

Often several nontrivial modifications are required to adapt this prototype method
for different applications, e.g., dynamic regularization~\cite{doas1,hanke1,rieder2005,rieder2010} and more general~\emph{stabilized GN} studied~\cite{rodoas1,doas12}.
This method replaces the solution of the linear systems defining $p_{k}$ by $r$
preconditioned conjugate gradient (PCG) inner iterations, which costs $2r$
solutions of the forward problem per iteration, for a moderate integer value $r$. 
Thus, if $K$ outer iterations are required to obtain an acceptable solution then
the total work estimate (in terms of the number of PDE solves) is approximated
{\em from below} by $2(r+1) K s$. 

Though Gauss-Newton is arguable the method of choice within the inverse problem community, other Newton-type methods exist which have been designed to suitably deal with the non-convex nature of the underlying optimization problem include Trust Region~\cite{conn2000trust,xu2017newton} and the Cubic Regularization~\cite{xu2017newton,cartis2012evaluation}. These methods have recently found applications in machine learning~\cite{xu2017second}. Studying the advantages/disadvantages of these non-convex methods for solving inverse problems can be indeed a useful undertaking.

\medskip\noindent{\bf Quasi-Newton methods.}
An alternative method to the above Newton-type methods is the quasi-Newton
variants including the celebrated limited memory BFGS (L-BFGS)
\cite{liu1989limited,nocedal1980updating}.
BFGS iteration is closely related to conjugate gradient (CG) iteration. In particular, BFGS applied to a strongly convex 
quadratic objective, with exact line search as well as initial Hessian $ P $, is equivalent to preconditioned CG with preconditioner $ P $. However, as the objective function departs from being a simple quadratic, the number of iterations of L-BFGS could be significantly higher than that of GN or trust region. In addition, it has been shown that the performance of BFGS and its limited memory version is greatly negatively affected by the high degree if ill-conditioning present in such problems~\cite{romassn1, romassn2,pyrrm_ssn_nonuni}. These two factor are among the main reasons why BFGS (and L-BFGS) can be less effective compared with other Newton-type alternatives in many inversion applications~\cite{haber2004quasi}.

\medskip\noindent{\bf Krylov subspace method.}
A Krylov subspace method iteratively finds the optimal solution to an
optimization in a larger subspace by making use of the previous solution in a
smaller subspace.
One of the most commonly used Krylov subspace method is the conjugate gradient
(CG) method.
CG was originally designed to solve convex quadratic minimization problems of
the form $g(\mm) = \frac{1}{2} \mm^{\top} A \mm - b^{\top} \mm$.
Equivalently, this solves the positive definite linear system $A\mm = b$.
It computes a sequence of iterates $\mm_0, \mm_1, \ldots$ converging to the minimum
through the following two set of equations.
\begin{align}
	\mm_0 &=0, & r_0 &= b, & p_0 &= r_0, & \\
  \mm_{k+1} &= \mm_{k} + \frac{||r_k||_2^2}{p_{k}^{\top} A p_{k}} p_{k}, 
    &
  r_{k+1} &= r_{k} - \frac{||r_k||_2^2}{p_{k}^{\top} A p_{k}} A p_k,
    &
	p_{k+1} &= r_{k+1} + \frac{||r_{k+1}||_2^2}{||r_{k}||_2^2} p_{k}, & k \ge 0.
\end{align}
This can be used to solve the forward problem of the form 
$L_{i}(\mm) \uu_{i} = \qq_{i}$, provided that $L_{i}(\mm)$ is positive definite, which
is true in many cases.

CG can be used to solve the linear system for the basic Newton direction.
However, the Hessian is not necessarily positive definite and modification is
needed \cite{nocedal2006numerical}.

In general, CG can be generalized to minimize a nonlinear function $g(\mm)$ 
\cite{fletcher2013practical,dai2011nonlinear}.
It starts with an arbitrary $\mm_0$, and $p_1 = - \grad g(\mm_0)$, and computes
a sequence of iterates $\mm_1, \mm_2, \ldots$ using the equations below:
for $k \ge 0$,
\begin{align}
	\mm_{k+1} 
		&= \arg\min_{\mm \in \{\mm_k + \lambda p_k, \lambda \in \mathbb{R}\}} g(\mm), \\
	p_{k+1}
		&= -\grad g(\mm_{k+1}) + \beta_{k} p_k, 
		\qquad\text{ where }
		\beta_k = \frac{||\grad g(\mm_{k+1})||_2^2}{||\grad g(\mm_{k})||_2^2}.
\end{align}
The above formula for $\beta_k$ is known as the Fletcher-Reeves formula.
Other choices of $\beta_k$ exist.
The following two formula are known as the Polak-Ribiere and Hestenes-Stiefel
formula respectively.
\begin{align}
    \beta_{k}
		&= \frac{\langle \grad g(\mm_{k+1}) - \grad g(\mm_{k}),  \grad g(\mm_{k+1}) \rangle}
        {||\grad g(\mm_{k})||_2^2}, \\
    \beta_{k}
		&= \frac{\langle \grad g(\mm_{k+1}) - \grad g(\mm_{k}),  \grad g(\mm_{k+1}) \rangle}
        {p_{k}^{\top} (\grad g(\mm_{k+1}) - \grad g(\mm_{k}))}.
\end{align}
In practice, nonlinear CG does not seem to work well, and is mainly used
together with other methods, such as in the Newton CG method
\cite{nocedal2006numerical}.

\medskip\noindent{\bf Lagrangian method of multipliers.}
The above discussion focuses on unconstrained optimization algorithms, which are
suitable for unconstrained formulations of inverse problems, or unconstrained
auxiliary optimization problems in methods which solves the constrained
formulations directly.
The Lagrangian method of multipliers is often used to directly solve the
constrained version.
Algorithms have been developed to offset the heavier computational cost and
slow convergence rates of standard algorthms observed on the Lagrangian, which
is a larger problem than the constrained problem.
For example, such algorithm may reduce the problem to a smaller one, such as
working with the reduced Hessian of the Lagrangian~\cite{haber2000optimization},
or preconditioning~\cite{haber2001preconditioned,benzi2011preconditioning}.
These methods have shown some success in certain PDE-constrained optimization
problems.

Augmented Lagrangian methods have also been developed
(e.g.~\cite{ito1990augmented,abdoulaev2005optical}).
These methods constructs a series of penalized Lagrangians with vanishing
penalty, and finds an optimizer of the Lagrangian by successively optimizing the
penalized Lagrangians.

\subsection{Challenges}
\label{sec:challenges}

\noindent{\bf Scaling up to large problems.}
The discretized version of an inverse problem is usually of very large scale,
and working with fine resolution or discretized problems in high dimension is
still an active area of research.

Another challenge is to scale up to large number of measurements, which is
widely believed to be helpful for quality reconstruction of the model in
practice, with some theoretical support.
While recent technological advances makes many big datasets available, existing
algorithms cannot efficiently cope with such datasets.
Examples of such problems include electromagnetic data inversion in mining
exploration \cite{na,dmr,haasol,olhash}, seismic data inversion in oil
exploration \cite{fichtner,hel,rnkkda}, diffuse optical tomography (DOT)
\cite{arridge1999optical,boas}, quantitative photo-acoustic tomography (QPAT)
\cite{gaooscher,yuan}, direct current (DC) resistivity
\cite{smvoz,pihakn,haheas,HaberChungHermann2010,doas12}, 
and electrical impedance tomography (EIT) \cite{bbp,cin,van2013lost}.

It has been suggested that many well-placed experiments yield practical advantage in order to obtain reconstructions of acceptable quality. 
For the special case where the measurement locations as well as the discretization matrices do not change from one experiment to another, various approximation techniques have been proposed to reduce the effective number of measurements, which in turn implies a smaller scale optimization problem, under the unifying category of ``simultaneous sources inversion'' \cite{rodoas1,roszas,roosta2015randomized,haber2014simultaneous,kumar2014GEOPemc}. Under certain circumstances, even if the $P_{i}$'s are different across experiments (but $ L_{i} $'s are fixed), there are methods to transform the existing data set 
into the one where all sources share the same receivers, \cite{rodoas2}. 

\medskip\noindent{\bf Dealing with non-convexity.}
Another major source of difficulty in solving many inverse problems, is the high-degree of non-linearity and non-convexity in~\eqref{eq:forward_prob}. This is most often encountered in problems involving PDE-constrained optimization where each $ \ff_{i} $ corresponds to the solution of a PDE. Even if the output of the PDE model itself, i.e., the ``right-hand side'', is linear in the sought-after parameter, the solution of the PDE, i.e., the forward problem, shows a great deal of non-linearity. This coupled with a great amount of non-convexity can have significant consequences in the quality of inversion and the obtained parameter. Indeed, in presence of non-convexity, the large-scale computational challenges are exacerbated, multiple folds over, by the difficulty of avoiding (possibly degenerate) \emph{saddle-points} as well as finding (at least) a \emph{local minimum}. 

\medskip\noindent{\bf Dealing with discontinuity.}
While the parameter function of the model is often smooth, the parameter
function can be discontinuous in some cases.
Such discontinuities arise very naturally as a result of the physical properties
of the underlying physical system, e.g., EIT and DC resistivity, and require
non-trivial modifications to optimization algorithms, e.g.,~\cite{rodoas1,
doas12}. 
Ignoring such discontinuities can lead to unsatisfactory recovery results
\cite{tali,doas12,doasleit2010}.
The level set method \cite{osse} is often used to model discontinuous parameter
function.
This reparametrizes the discontinuous parameter function as a differentiable
one, and thus enabling more stable optimization \cite{doasleit2010}.

\section{Recent Advances in Optimization}
\label{sec:advances}

Recent successes in using machine learning to deal with challenging perception
and natural language understanding problems have spurred many advances in the
study of optimization algorithms as optimization is a building block in machine
learning. 
These new developments include efficient methods for large-scale optimization, 
methods designed to handle non-convex problems, methods incorporating the structural
constraints, and finally the revival of second-order methods.
While these developments address a different set of applications in machine
learning, they address similar issues as encountered in inverse optimization and
could be useful.
We highlight some of the works below.
We keep the discussion brief because numerous works have been done behind these
developments and an indepth and comprehensive discussion is beyond the scope of
this review.
Our objective is thus to delineate the general trends and ideas, and provide
references for interested readers to dig on relevant topics.

\medskip\noindent{\bf Stochastic optimization.}
The development in large-scale optimization methods is driven by the
availability of many large datasets, which are made possible by the rapid
development and extensive use of IT technology. 
In machine learning, a model is generally built by optimizing a sum of misfit on
the examples.
This finite-sum structure naturally invites the application of stochastic optimization algorithms. This is mainly due to the fact that stochastic algorithms recover the sought-after models more efficiently by employing small batches of data in each iteration, as opposed to the whole data-set.
The most well-known stochastic gradient based algorithm is the stochastic gradient descent (SGD).
To minimize a finite-sum objective function 
\begin{align}
\label{eq:finite_sum}
g(\mm) = \frac{1}{n} \sum_{i=1}^{n} g_{i}(\mm),
\end{align} 
in the big data regime where $ n \gg 1 $, 
the vanilla SGD performs an update 
\begin{equation}
	\mm_{k+1} = \mm_{k} - \lambda_{k} \grad g_{i_{k}}(\mm_{k}),
\end{equation}
where $i_{k}$ is randomly sampled from $1, \ldots, n$.
As compared to gradient descent, SGD replaces the full gradient 
$\grad g(\mm)$ by a stochastic gradient $g_{i_{k}}(\mm_{k})$ with its
expectation being the full gradient.
The batch version of SGD constructs a stochastic gradient by taking the average
of several stochastic gradients.

SGD is inexpensive per iteration, but suffers from a slow rate of convergence.
For example, while full gradient descent achieves a linear convergence rate for
smooth strongly convex problems, SGD only converges at a sublinear rate.
The slow convergence rate can be partly accounted by the variance in the
stochastic gradient.
Recently, variance reduction techniques have been developed, e.g.
SVRG \cite{johnson2013accelerating} and SDCA \cite{shalev2013stochastic}.
Perhaps surprisingly, such variants can achieve linear convergence rates on
convex smooth problems as full gradient descent does, instead of sublinear rates
achieved by the vanilla SGD.
There are also a number of variants with no known linear rates but have fast
convergence rates for non-convex problems in practice, e.g.,
AdaGrad~\cite{duchi2011adaptive}, RMSProp~\cite{tijmen2012rmsprop},
ESGD~\cite{dauphin2015equilibrated}, Adam~\cite{kingma2014adam}, and
Adadelta~\cite{zeiler2012adadelta}.
Indeed, besides efficiency, stochastic optimization algorithms also seem to be able to
cope with the nonconvex objective functions well, and play a key role in the
revival of neural networks as deep learning~\cite{jin2017escape,ge2015escaping, levy2016power}.

\medskip\noindent{\bf Nonconvex optimization.}
There is also an increasing interest in non-convex optimization in the machine
learning community recently.
Nonconvex objectives not only naturally occur in deep learning, but also occur
in problems such as tensor decomposition, variable selection, low-rank matrix
completion, e.g. see \cite{ge2015escaping,mazumder2011sparsenet,jain2013low}
and references therein.

As discussed above, stochastic algorithms have been found to be capable of
effectively escaping local minima.
There are also a number of studies which adapt well-known acceleration
techniques for convex optimization to accelerate the convergence rates of both
stochastic and non-stochastic optimization algorithms for nonconvex problems,
e.g.,~\cite{li2015accelerated,allen2016variance,reddi2016stochastic,sutskever2013importance}.

\medskip\noindent{\bf Dealing with structural constraints.}
Many problems in machine learning come with complex structural constraints.
The Frank-Wolfe algorithm (a.k.a. conditional gradient)
\cite{frank1956algorithm} is an algorithm for optimizing over a convex domain.
It has gained a revived interest due to its ability to deal with many structural
constraints efficiently.
It requires solving a linear minimization problem over the feasible set, instead
of a quadratic program as in the case of proximal gradient algorithms or
projected gradient descent.
Domains suitable for the Frank-Wolfe algorithm include simplices,
$\ell_p$-balls, matrix nuclear norm ball, matrix operator norm ball
\cite{jaggi2013revisiting}.

The Frank-Wolfe algorithm belongs to the class of linear-optimization-based
algorithms \cite{lan2016conditional,lan2017conditional}.
These algorithms share with the Frank-Wolfe algorithm the characteristic of
requiring a first-order oracle for gradient computation and an oracle for
solving a linear optimization problem over the constraint set.

\medskip\noindent{\bf Second-order methods.}
The great appeal of the second-order methods lies mainly in the observed empirical performance as well as some very appealing theoretical properties. For example, it has been shown that stochastic Newton-type methods in general, and Gauss-Newton in particular, can not only be made scalable and have low per-iteration cost~\cite{rodoas1,rodoas2,roszas,haber2000optimization,haber2012effective,doas12}, but more importantly, and unlike first-order methods, are very \emph{resilient} to many adversarial effects such as \emph{ill-conditioning}~\cite{romassn1,romassn2,pyrrm_ssn_nonuni}. As a result, for moderately to very ill-conditioned problems, commonly found in scientific computing, while first-order methods make effectively no progress at all, second-order counterparts are not affected by the degree of ill-conditioning. 
A more subtle, yet potentially more severe draw-back in using first-order methods, is that their success is tightly intertwined with \emph{fine-tunning} (often many) \emph{hyper-parameters}, most importantly, the step-size~\cite{berahas2017investigation}. In fact, it is highly unlikely that many of these methods exhibit acceptable performance on first try, and it often takes many trials and errors before one can see reasonable results. In contrast, second-order optimization algorithms involve much less parameter tuning and are less sensitive to the choice of hyper-parameters~\cite{berahas2017investigation, xu2017second}. 

Since for the finite-sum problem~\eqref{eq:finite_sum} with $ n \gg 1 $, the operations with the Hessian/gradient constitute major computational bottlenecks, a rather more recent line of research is to construct the inexact Hessian information using the application of \emph{randomized methods}. Specifically, for convex optimization, the stochastic approximation of the full Hessian matrix in the classical Newton's method has been recently considered in~\cite{byrd2011use, byrd2012sample, wang2015subsampled,pilanci2015newton, erdogdu2015convergence, romassn1, romassn2, pyrrm_ssn_nonuni, Agarwal2016SecondOS, mutny2016stochastic, ye2016revisiting, bollapragada2016exact, mutny2017parallel, berahas2017investigation,eisen2017large}. In addition to inexact Hessian, a few of these methods study the fully stochastic case in which the gradient is also approximated, e.g.,~\cite{romassn1, romassn2,bollapragada2016exact}. For non-convex problems, however, the literature on methods that employ randomized Hessian approximation is significantly less developed than that of convex problems. A few recent examples include the stochastic trust region~\cite{xu2017newton}, stochastic cubic regularization~\cite{xu2017newton,tripuraneni2017stochastic}, and noisy negative curvature method~\cite{liu2017noisy}. Empirical performance of many of these methods for some non-convex machine learning applications has been considered in~\cite{xu2017second}.


\mySection{Related Works and Discussion}{}
\label{chap3:sec:discussion}

In this section we briefly discuss the similarities and differences of the model presented in this chapter, comparing it with some related work presented earlier (Chapter \ref{chap1:artifact-centric-bpm}). We will mention a few related studies and discuss directly; a more formal comparative study using qualitative and quantitative metrics should be the subject of future work.

Hull et al. \citeyearpar{hull2009facilitating} provide an interoperation framework in which, data are hosted on central infrastructures named \textit{artifact-centric hubs}. As in the work presented in this chapter, they propose mechanisms (including user views) for controlling access to these data. Compared to choreography-like approach as the one presented in this chapter, their settings has the advantage of providing a conceptual rendezvous point to exchange status information. The same purpose can be replicated in this chapter's approach by introducing a new type of agent called "\textit{monitor}", which will serve as a rendezvous point; the behaviour of the agents will therefore have to be slightly adapted to take into account the monitor and to preserve as much as possible the autonomy of agents.

Lohmann and Wolf \citeyearpar{lohmann2010artifact} abandon the concept of having a single artifact hub \cite{hull2009facilitating} and they introduce the idea of having several agents which operate on artifacts. Some of those artifacts are mobile; thus, the authors provide a systematic approach for modelling artifact location and its impact on the accessibility of actions using a Petri net. Even though we also manipulate mobile artifacts, we do not model artifact location; rather, our agents are equipped with capabilities that allow them to manipulate the artifacts appropriately (taking into account their location). Moreover, our approach considers that artifacts can not be remotely accessed, this increases the autonomy of agents.

The process design approach presented in this chapter, has some conceptual similarities with the concept of \textit{proclets} proposed by Wil M. P. van der Aalst et al. \citeyearpar{van2001proclets, van2009workflow}: they both split the process when designing it. In the model presented in this chapter, the process is split into execution scenarios and its specification consists in the diagramming of each of them. Proclets \cite{van2001proclets, van2009workflow} uses the concept of \textit{proclet-class} to model different levels of granularity and cardinality of processes. Additionally, proclets act like agents and are autonomous enough to decide how to interact with each other.

The model presented in this chapter uses an attributed grammar as its mathematical foundation. This is also the case of the AWGAG model by Badouel et al. \citeyearpar{badouel14, badouel2015active}. However, their model puts stress on modelling process data and users as first class citizens and it is designed for Adaptive Case Management.

To summarise, the proposed approach in this chapter allows the modelling and decentralized execution of administrative processes using autonomous agents. In it, process management is very simply done in two steps. The designer only needs to focus on modelling the artifacts in the form of task trees and the rest is easily deduced. Moreover, we propose a simple but powerful mechanism for securing data based on the notion of accreditation; this mechanism is perfectly composed with that of artifacts. The main strengths of our model are therefore : 
\begin{itemize}
	\item The simplicity of its syntax (process specification language), which moreover (well helped by the accreditation model), is suitable for administrative processes;
	\item The simplicity of its execution model; the latter is very close to the blockchain's execution model \cite{hull2017blockchain, mendling2018blockchains}. On condition of a formal study, the latter could possess the same qualities (fault tolerance, distributivity, security, peer autonomy, etc.) that emanate from the blockchain;
	\item Its formal character, which makes it verifiable using appropriate mathematical tools;
	\item The conformity of its execution model with the agent paradigm and service technology.
\end{itemize}
In view of all these benefits, we can say that the objectives set for this thesis have indeed been achieved. However, the proposed model is perfectible. For example, it can be modified to permit agents to respond incrementally to incoming requests as soon as any prefix of the extension of a bud is produced. This makes it possible to avoid the situation observed on figure \ref{chap3:fig:execution-figure-4} where the associated editor is informed of the evolution of the subtree resulting from $C$ only when this one is closed. All the criticisms we can make of the proposed model in particular, and of this thesis in general, have been introduced in the general conclusion (page \pageref{chap5:general-conclusion}) of this manuscript.




% \vspace{-0.5em}
\section{Conclusion}
% \vspace{-0.5em}
Recent advances in multimodal single-cell technology have enabled the simultaneous profiling of the transcriptome alongside other cellular modalities, leading to an increase in the availability of multimodal single-cell data. In this paper, we present \method{}, a multimodal transformer model for single-cell surface protein abundance from gene expression measurements. We combined the data with prior biological interaction knowledge from the STRING database into a richly connected heterogeneous graph and leveraged the transformer architectures to learn an accurate mapping between gene expression and surface protein abundance. Remarkably, \method{} achieves superior and more stable performance than other baselines on both 2021 and 2022 NeurIPS single-cell datasets.

\noindent\textbf{Future Work.}
% Our work is an extension of the model we implemented in the NeurIPS 2022 competition. 
Our framework of multimodal transformers with the cross-modality heterogeneous graph goes far beyond the specific downstream task of modality prediction, and there are lots of potentials to be further explored. Our graph contains three types of nodes. While the cell embeddings are used for predictions, the remaining protein embeddings and gene embeddings may be further interpreted for other tasks. The similarities between proteins may show data-specific protein-protein relationships, while the attention matrix of the gene transformer may help to identify marker genes of each cell type. Additionally, we may achieve gene interaction prediction using the attention mechanism.
% under adequate regulations. 
% We expect \method{} to be capable of much more than just modality prediction. Note that currently, we fuse information from different transformers with message-passing GNNs. 
To extend more on transformers, a potential next step is implementing cross-attention cross-modalities. Ideally, all three types of nodes, namely genes, proteins, and cells, would be jointly modeled using a large transformer that includes specific regulations for each modality. 

% insight of protein and gene embedding (diff task)

% all in one transformer

% \noindent\textbf{Limitations and future work}
% Despite the noticeable performance improvement by utilizing transformers with the cross-modality heterogeneous graph, there are still bottlenecks in the current settings. To begin with, we noticed that the performance variations of all methods are consistently higher in the ``CITE'' dataset compared to the ``GEX2ADT'' dataset. We hypothesized that the increased variability in ``CITE'' was due to both less number of training samples (43k vs. 66k cells) and a significantly more number of testing samples used (28k vs. 1k cells). One straightforward solution to alleviate the high variation issue is to include more training samples, which is not always possible given the training data availability. Nevertheless, publicly available single-cell datasets have been accumulated over the past decades and are still being collected on an ever-increasing scale. Taking advantage of these large-scale atlases is the key to a more stable and well-performing model, as some of the intra-cell variations could be common across different datasets. For example, reference-based methods are commonly used to identify the cell identity of a single cell, or cell-type compositions of a mixture of cells. (other examples for pretrained, e.g., scbert)


%\noindent\textbf{Future work.}
% Our work is an extension of the model we implemented in the NeurIPS 2022 competition. Now our framework of multimodal transformers with the cross-modality heterogeneous graph goes far beyond the specific downstream task of modality prediction, and there are lots of potentials to be further explored. Our graph contains three types of nodes. while the cell embeddings are used for predictions, the remaining protein embeddings and gene embeddings may be further interpreted for other tasks. The similarities between proteins may show data-specific protein-protein relationships, while the attention matrix of the gene transformer may help to identify marker genes of each cell type. Additionally, we may achieve gene interaction prediction using the attention mechanism under adequate regulations. We expect \method{} to be capable of much more than just modality prediction. Note that currently, we fuse information from different transformers with message-passing GNNs. To extend more on transformers, a potential next step is implementing cross-attention cross-modalities. Ideally, all three types of nodes, namely genes, proteins, and cells, would be jointly modeled using a large transformer that includes specific regulations for each modality. The self-attention within each modality would reconstruct the prior interaction network, while the cross-attention between modalities would be supervised by the data observations. Then, The attention matrix will provide insights into all the internal interactions and cross-relationships. With the linearized transformer, this idea would be both practical and versatile.

% \begin{acks}
% This research is supported by the National Science Foundation (NSF) and Johnson \& Johnson.
% \end{acks}

%\nocite{*}
\bibliographystyle{plain}
\bibliography{ref_Nan,ref_Fred}
\end{document}
