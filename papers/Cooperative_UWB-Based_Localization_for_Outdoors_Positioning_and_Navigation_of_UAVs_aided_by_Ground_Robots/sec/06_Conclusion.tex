%%%%%%%%%%%%%%%%%%%%%%%%%%%%%%%%%%%%%%%%%%%%%%
%%                                          %%
%%              CONCLUSION                  %%
%%                                          %%
%%%%%%%%%%%%%%%%%%%%%%%%%%%%%%%%%%%%%%%%%%%%%%

\section{Conclusion}\label{sec:conclusion}

We have presented an analysis on how UWB-based relative localization between a UAV and a companion ground robot can improve the accuracy of autonomous flights outdoors. In particular, we have simulated different scenarios to assess the accuracy of the UWB-based relative positioning method. We have then validated this with robots in outdoor experiments, in an urban area where GNSS receivers do not perform optimally. Our analysis includes VIO estimation, which is more accurate at first but loses the reference when the UAV starts gaining altitude, presumably due to the lack of reference points.

In summary, we can conclude that UWB-based positioning systems can provide an alternative to RTK-GNSS when the accuracy of standalone GNSS is not enough for gathering aerial data. Moreover, we have proved that even when the transceivers are placed near each other in the ground, mounted on a mobile platform, the accuracy is enough to enable autonomous flight. 

Future work will focus on integrating UWB with GNSS data and performing experiments in more varied environments. We will also analyze different anchor configurations and consider additional robots.
