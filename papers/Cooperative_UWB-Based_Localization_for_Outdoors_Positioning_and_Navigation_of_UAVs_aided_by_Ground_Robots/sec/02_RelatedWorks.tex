
%%%%%%%%%%%%%%%%%%%%%%%%%%%%%%%%%%%%%%%%%%%%%%
%%                                          %%
%%              RELATED WORKS               %%
%%                                          %%
%%%%%%%%%%%%%%%%%%%%%%%%%%%%%%%%%%%%%%%%%%%%%%
% \newpage

\section{Background} \label{sec:related_work}

This section reviews the literature in the area of outdoors positioning and navigation methods for multi-robot systems. 

\subsection{Limitations of standalone GNSS}

The long-term operation of autonomous robots outdoors is often a reliance on GNSS~\cite{qingqing2019multi}.  However, the positioning accuracy of GNSS can be easily influenced by multipath when satellites are not in line-of-sight. This is a typical problem in urban environments or partly covered environments such as forests~\cite{groves2012intelligent, li2020localization}. Additional sensors are thus used in practice, from IMUs at the lowest level~\cite{jiang2017board} to odometry estimation from lidars~\cite{chang2019gnss} or visual sensors~\cite{li2019tight}. It is worth mentioning, nonetheless, that more recent receivers exploiting multi-constellation signals (e.g., GPS, GLONASS, BEIDOU, or GALILEO) are able to deliver significantly higher positioning accuracy~\cite{li2019triple}.

% To meet the localization accuracy needs in a challenging environment, additional sensors are utilized and fused with GNSS to enhance the positioning accuracy, such as Lidar, IMU, camera.
% % -- Drones with multiple receivers
% When large numbers of robots are deployed in swarms for collaboration tasks, accurate relative state estimation is an essential issue. 
% % To realize relative state estimation in GNSS available environment, one intuitive method is integrating GNSS to each robot. 
% However, the meters-level accuracy limitation of GNSS hinders the dense swarm collaboration performance. 
% % -- Multi-constellation (GPS+GLONASS, +BEIDOU, +GALILEO)
% With more commercial GNSS system become available, such as GPS, GLONASS, BEIDOU, multi-constellation GNSS and multi-frequency signals open new prospects for fast ambiguity resolution of precise point positioning~\cite{li2019triple}. Fusing multi-GNSS to improve the position estimation has been studied in ~\cite{li2015accuracy}.

\subsection{GNSS-RTK}
% RTK
% As the accuracy of satellite-based positioning systems is easily being influenced by multi-path effect in clustered environment,  
High-accuracy GNSS positioning is possible with real-time kinematic (RTK) systems. RTK positioning leverages measurements of the phase of the signal's carrier wave in addition to the information content of the signal and relies on a reference station or interpolated virtual station to provide real-time corrections, providing up to centimeter-level accuracy~\cite{tomavstik2019uav}. These systems, however, are costly and require calibration and setup for each different location.
% -- Papers where is has been used for multi-robot systems
% Accuracy assessment of GNSS-RTK measurements on UAVs for direct geo-referencing has been studied in ~\cite{ekaso2020accuracy}.% -- Benefits and drawbacks
% However, the reference base station add more cost and also the system complexity which require addition effort at preparation stage e.g. system calibration.

\subsection{Onboard navigation}

With the increasing adoption of UAVs in recent years, the maturity of onboard estate estimation and localization has reached a point where it is standard in commercial systems. Onboard odometry and positioning are typically based on monocular or stereo vision (e.g., VINS-mono~\cite{qin2018vins}, vins-fusion~\cite{qin2019general}), but lidars are also effective in larger UAVs~\cite{zhang2019maximum}. Passive visual sensors, however, have evident limitations in terms of environmental conditions (e.g., night operation) and in situations where there is a lack of features~\cite{xiao2017uav, qingqing2020towards}

% Owning to less dependence on external equipment, less system complexity and high scalability, on-board navigation system has attracted much attention in recent years. The on-board navigation approaches can be divided into two groups: camera-based, e.g. VINS-mono~\cite{qin2018vins}, vins-fusion~\cite{qin2019general} and Lidar based, e.g LOAM~\cite{zhang2014loam}, Lego-loam~\cite{shan2018lego}. 
% % VIO outdoors
% The use of vision sensors in the inertial navigation of multi-robot systems outdoor has been investigated in~\cite{xiao2017uav,qingqing2020towards} 
% % 
% However, passive vision systems suffer from disadvantages such as night vision incapability, the inability of direct depth perception and illumination variations, which limits their use in realistic outdoor scenarios. By contrast, lidar-based approaches are more robust against the aforementioned problems.
% % % 
% 2D lidar has been widely adopted as a core sensor for mapping and navigation tasks in structured indoor environments in various commercial products, e.g. cleaning robots,  due to its cost-effective, lightweight and power efficiency\cite{catapang2016obstacle}. Combined with RGB-D sensors, 2D lidar also has been utilized to detect obstacles and navigate for UAV in DARPA subterranean challenge\cite{rouvcek2019darpa}. 
% 3D lidar able to accurately measure and sample the distance to the surface of objects, so it has been widely deployed in autonomous system~\cite{qingqing2020towards}. However, high-definition 3D lidar proved to be a prominent performance for obstacle detection~\cite{qingqing2020towards}, object recognition, SLAM~\cite{shan2018lego}, but suffer from the high cost and the heavyweight which make it hard to implement on load and power limited device such as micro UAVs.
% % Load and power sensible device


\subsection{UWB Localization}

Ultra-wideband (UWB) positioning systems are being increasingly adopted for autonomous systems ~\cite{yu2021applications}.UWB positioning systems based on a series of fixed nodes in known locations (or anchors), and ranging measurements between these and mobile nodes (or tags), can be used for consistent, long-term localization of mobile robots~\cite{macoir2019uwb, queralta2020uwb}.%, provides a competitive alternative to MOCAP systems that is accurate and stable enough by itself whenever centimeter-level accuracy suffices~\cite{shule2020uwb}.
% -- Talk about fixed anchor systems, competitive compared to RTK and MOCAP
Compared to RTK-based localization systems, UWB systems can be utilized both indoors and outdoors, can be automatically calibrated~\cite{almansa2020autocalibration} for ad-hoc deployment, and offer similar accuracies at much lower prices. UWB sensors also have a small form factor and are generally considered more energy efficient than other wireless solutions. Finally, UWB ranging is often combined with other sensors to add orientation estimation and increase the overall localization performance. Different approaches in the literature include fusion of UWB with IMU~\cite{yao2017integrated}, VIO estimators~\cite{nguyen2019integrated}, GPS~\cite{zhang2019combined}, or lidar~\cite{song2019uwb}.

% a significant advantage as its ability to penetrate walls in buildings and mitigate individual multipath components due to its large bandwidth~\cite{ruiz2017comparing}, so it can be utilized both indoor and outdoor environment. Furthermore, UWB sensor is small, power-efficient and highly portable, it is more flexible than MOCAP system.
 
% -- Give a few examples of applications of drones
% As a radio-based sensor, the multipath interference, NLOS transmission will degrade the UWB positioning performance. Therefore, UWB ranging data is often fused with different sensors or sources of location, inertial, or odometry data in real applications, such as raw IMU~\cite{yao2017integrated}, VIO estimators~\cite{nguyen2019integrated}, GPS~\cite{zhang2019combined}, and Lidar~\cite{song2019uwb}. 
% -- Talk about accuracy and saying that it is better than Wifi/Bluetooth
% Multiple examples in the literature have investigated that UWB positioning systems are more robust and accurate than their Wi-Fi or Bluetooth counterparts~~\cite{woo2011application, karaagac2017evaluation}. Specifically, UWB positioning has been proven significantly more accurate in harsh and complex environments. 

\subsection{UWB for relative estimation}

UWB ranging has been widely used for relative localization within multi-robot systems. For instance, in~\cite{nguyen2018robust}, the authors demonstrate a system where relative positioning between and UAV and a UGV is designed based on UWB transceivers installed on both robots. In subsequent works~\cite{nguyen2019integrated}, a similar system is employed during docking maneuvers. Combined with vision sensors for the final docking, the autonomous approach of the UAV to the UGV relied on UWB ranging between transceivers in both robots. Relative localization between UAVs and UGVs has also been shown within the context of collaborative dense scene reconstruction~\cite{queralta2020vio}. In multi-UAV systems and UAV swarms, UWB ranging has been leveraged for swarm-level decentralized estate estimation~\cite{xu2020decentralized, qi2020cooperative}.

In general, terms, while UWB systems including those for relative localization have been widely studied in the literature, we see a lack of studies that quantitatively analyze how UWB-based relative estate estimation can improve GNSS positioning and navigation outdoors.  


% Relative positioning play an essential role in cooperative heterogeneous multi-robot systems~\cite{shule2020uwb}.
% % -- Swarm relative estate estimation
% \red{Relative localization between two equipped with multiple transceivers~\cite{nguyen2018robust} can be leveraged for, e.g., autonomous docking~\cite{nguyen2019integrated}, or for collaborative scene reconstruction~\cite{queralta2020vio}. When large numbers of robots are deployed in swarms, UWB ranging and other sensor data can be leveraged for decentralized system-level estate estimation~\cite{xu2020decentralized, qi2020cooperative}.}
% % -- Landing and relative localization
% % -- Mostly related to the paper topic
% % One of the most similar work is ~\cite{nguyen2019integrated} in which the authors proposed a positioning method combining UWB ranging sensor with vision-based techniques to achieve both autonomous approaching and landing capabilities in GPS-denied environments.


% \red{TODO} However there is a lack of studies in the literature on quantitatively analyzing how UWB-based relative estate estimation can improve GNSS positioning and navigation ourdoors. 

 