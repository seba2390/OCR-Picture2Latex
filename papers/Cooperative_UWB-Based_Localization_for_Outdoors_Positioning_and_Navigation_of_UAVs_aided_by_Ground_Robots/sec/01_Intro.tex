%%%%%%%%%%%%%%%%%%%%%%%%%%%%%%%%%%%%%%%%%%%%%%
%%                                          %%
%%              INTRODUCTION                %%
%%                                          %%
%%%%%%%%%%%%%%%%%%%%%%%%%%%%%%%%%%%%%%%%%%%%%%

\section{Introduction}\label{sec:introduction}

Multiple industrial use cases benefit from the deployment of Unmanned aerial vehicles (UAVs)~\cite{shakhatreh2019unmanned}. When accurate localization is needed, GNSS-RTK is the de-facto standard for gathering aerial data with UAVs~\cite{li2018high}. For example, high-accuracy photogrammetry~\cite{lee2018assessment}, civil infrastructure monitoring~\cite{kim2018structural}, or in urban environments where GNSS signals suffer more degradation~\cite{li2018high}. As UAVs become ubiquitous across different domains and application areas~\cite{queralta2020collaborative}, having access to more flexible and lower-cost solutions to precise UAV navigation can aid in accelerating adoption and widespread use. In this paper, we consider the problem of UAV navigation through relative localization to a companion unmanned ground vehicle (UGV). We consider a ground robot as a more flexible platform from the point of view of deployment, but in simulations, we also consider localization based on fixed beacons in the environment, closer to how GNSS-RTK systems are deployed.

Within the different approaches that can be used for cooperative relative localization, from visual sensors~\cite{hui2013autonomous} to cooperative SLAM~\cite{kim2019uav}, wireless ranging technologies offer high performance with low system complexity~\cite{queralta2020uwb}. In particular, ultra-wideband (UWB) wireless ranging offers unparalleled localization performance within the different radio technologies in unlicensed bands~\cite{shule2020uwb}. Other benefits of UWB include resilience to multipath, high time resolution, and low interference with other radio technologies~\cite{yu2021applications}.

\begin{figure}
    \centering
    \includegraphics[width=0.49\textwidth]{fig/cooperative_v23.pdf}
    \caption{Cooperative localization approach based on UWB ranging measurements from multiple transceivers in different robots}
    \label{fig:concept}
\end{figure}

The system we analyze in this paper consists of a UGV equipped with four UWB transceivers and a UAV equipped with two transceivers. The UAV transceivers act as initiators, taking turns in sending signals to each of the UGV transceivers. When these respond, the time of flight of the signal is calculated and the distance between each pair of transceivers is calculated. This process is illustrated in Fig.~\ref{fig:concept}. The main contribution of this paper is thus on evaluating how UWB-based relative localization can improve the positioning of UAVs when supported by ground robots. We simulate different trajectories to evaluate the performance of the system and compare the accuracy of the GNSS, UWB, and VIO approach to localization with field tests in an urban environment. In the simulations, we consider different configurations of transceivers in the ground to compare the localization and navigation performance.

The remainder of this document is organized as follows. Section II introduces absolute and relative positioning approaches relevant to the presented approach. Section III then describes the cooperative localization approach. In Section IV we introduce the methodology for simulations and experiments, with results presented in Section V. Section VI concludes the work and outlines future research directions.