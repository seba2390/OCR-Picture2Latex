\documentclass[prd,aps,twocolumn,preprintnumbers, showpacs, nofootinbib,superscriptaddress,notitlepage]{revtex4-1}
\usepackage{amssymb,amsthm,amsmath}
\usepackage{graphicx}   % figures
\usepackage{color}      % color is used in text
\usepackage{slashed}    % Feynman slash
\usepackage{verbatim}
\usepackage[normalem]{ulem}

\usepackage{rotating}   % to rotate tables
\usepackage{multirow}   % multicolumn and multirow
\usepackage[normalem]{ulem}
 
  \newcommand{\red}[1]{\textcolor{red}{#1}} %% QA
  \newcommand{\blue}[1]{\textcolor{blue}{#1}} %% MH

  \newcommand\bl{\color{blue}}  

\begin{document}

\title{Nonperturbative Determination of Collins-Soper Kernel from  Quasi Transverse-Momentum Dependent Wave Functions}

\collaboration{\bf{Lattice Parton Collaboration ($\rm {\bf LPC}$)}}


\author{Min-Huan Chu}
\affiliation{INPAC, Key Laboratory for Particle Astrophysics and Cosmology (MOE),  Shanghai Key Laboratory for Particle Physics and Cosmology, School of Physics and Astronomy, Shanghai Jiao Tong University, Shanghai 200240, China}

\affiliation{Yang Yuanqing Scientific Computering Center, 
Tsung-Dao Lee Institute, Shanghai Jiao Tong University, Shanghai 200240, China}

\author{Zhi-Fu Deng}
\affiliation{INPAC, Key Laboratory for Particle Astrophysics and Cosmology (MOE),  Shanghai Key Laboratory for Particle Physics and Cosmology, School of Physics and Astronomy, Shanghai Jiao Tong University, Shanghai 200240, China}

\author{Jun Hua}
\affiliation{Guangdong Provincial Key Laboratory of Nuclear Science, Institute of Quantum Matter, South China Normal University, Guangzhou 510006, China}
\affiliation{Guangdong-Hong Kong Joint Laboratory of Quantum Matter, Southern Nuclear Science Computing Center, South China Normal University, Guangzhou 510006, China}

\affiliation{INPAC, Key Laboratory for Particle Astrophysics and Cosmology (MOE),  Shanghai Key Laboratory for Particle Physics and Cosmology, School of Physics and Astronomy, Shanghai Jiao Tong University, Shanghai 200240, China}

\author{Xiangdong Ji}
\affiliation{Department of Physics, University of Maryland, College Park, MD 20742, USA}
\affiliation{Center for Nuclear Femtography,
1201 New York Ave., NW, Washington DC, 20005, USA}
\affiliation{Key Laboratory for Particle Astrophysics and Cosmology (MOE),
Shanghai Key Laboratory for Particle Physics and Cosmology,
Tsung-Dao Lee Institute, Shanghai Jiao Tong University, Shanghai 200240, China}

\author{Andreas Sch\"afer}
\affiliation{Institut f\"ur Theoretische Physik, Universit\"at Regensburg, D-93040 Regensburg, Germany}

\author{Yushan Su}
\affiliation{Department of Physics, University of Maryland, College Park, MD 20742, USA}

\author{Peng Sun}
\affiliation{Nanjing Normal University, Nanjing, Jiangsu, 210023, China}

\author{Wei Wang}
\affiliation{INPAC, Key Laboratory for Particle Astrophysics and Cosmology (MOE),  Shanghai Key Laboratory for Particle Physics and Cosmology, School of Physics and Astronomy, Shanghai Jiao Tong University, Shanghai 200240, China}

\author{Yi-Bo Yang}
\affiliation{CAS Key Laboratory of Theoretical Physics, Institute of Theoretical Physics, Chinese Academy of Sciences, Beijing 100190, China}
\affiliation{School of Fundamental Physics and Mathematical Sciences, Hangzhou Institute for Advanced Study, UCAS, Hangzhou 310024, China}
\affiliation{International Centre for Theoretical Physics Asia-Pacific, Beijing/Hangzhou, China}
\affiliation{School of Physical Sciences, University of Chinese Academy of Sciences,
Beijing 100049, China}

\author{Jun Zeng}
\affiliation{INPAC, Key Laboratory for Particle Astrophysics and Cosmology (MOE),  Shanghai Key Laboratory for Particle Physics and Cosmology, School of Physics and Astronomy, Shanghai Jiao Tong University, Shanghai 200240, China}

\author{Jialu Zhang}
\affiliation{INPAC, Key Laboratory for Particle Astrophysics and Cosmology (MOE),  Shanghai Key Laboratory for Particle Physics and Cosmology, School of Physics and Astronomy, Shanghai Jiao Tong University, Shanghai 200240, China}

\author{Jian-Hui Zhang} 
\affiliation{Center of Advanced Quantum Studies, Department of Physics, Beijing Normal University, Beijing 100875, China}

\author{Qi-An Zhang}
\affiliation{
School of Physics, Beihang University, Beijing 102206, China}
\affiliation{Key Laboratory for Particle Astrophysics and Cosmology (MOE),
Shanghai Key Laboratory for Particle Physics and Cosmology,
Tsung-Dao Lee Institute, Shanghai Jiao Tong University, Shanghai 200240, China}


\begin{abstract}
  In the framework of large-momentum effective theory, we present a first-principle determination of the Collins-Soper kernel which governs the rapidity evolution of  transverse-momentum dependent (TMD)  distributions.  We simulate the quasi TMD wave functions on a Euclidean lattice, and remove the pertinent linear divergences with  Wilson loop renormalization. By implementing {\bl one-loop matching of} the light-cone wave functions, we determine the Collins-Soper kernel with {\bl transverse} separation up to 0.6 fm.  We also discuss the systematic uncertainties from {\bl operator mixing} and scale {\bl dependence. The} impact from higher power corrections {\bl is} also taken into account. Our results allow for a more precise first-principle determination of the soft function and other transverse-momentum dependent quantities from lattice QCD. 
\end{abstract}






\maketitle
%%%%%%%%%%%%%%%%%%%%%%%%%%%%%%%%%%%%%%%%%%%%%%%%%%%%%%%%%%%%%%%%%%%%
%% Section I: introduction
%%%%%%%%%%%%%%%%%%%%%%%%%%%%%%%%%%%%%%%%%%%%%%%%%%%%%%%%%%%%%%%%%%%%
\section{Introduction}

Understanding the internal, {\bl three-dimensional structure of hadrons is} a primary goal of nuclear and particle physics. In this regard, the transverse momentum-dependent (TMD) parton distribution functions (TMDPDFs)~\cite{Collins:1981uk,Collins:1981va} play an important role as they characterize the intrinsic transverse 
partonic structure of protons. They are also essential ingredients in the description of multi-scale and non-inclusive processes, such as Drell-Yan production of electroweak gauge bosons or Higgs bosons or semi-inclusive deep-inelastic scattering with small transverse momentum, in the context of QCD factorization theorems. As a result, they have received considerable attention in the past few decades (for a review, see Ref.~\cite{Angeles-Martinez:2015sea}). More accurate experimental measurements are expected in the coming decades from JLab 12 GeV and the Electron-Ion {\bl Colliders} in the US~\cite{Accardi:2012qut,AbdulKhalek:2021gbh} and in China~\cite{Anderle:2021wcy}.


In contrast to the TMDPDFs that characterize the probability density of parton momenta in hadrons, the transverse momentum-dependent wave functions (TMDWFs) offer a probability amplitude description of the partonic structure of hadrons, from which one can extract all quark/gluon densities. In {\bl the} TMD factorization scheme, they are the most important ingredients to predict physical observables in exclusive processes, for instance, weak decays of heavy $B$ meson~\cite{Keum:2000wi,Lu:2000em} which are valuable to extract the CKM matrix element and probe new physics beyond the standard model. However, due to the lack of precise knowledge of TMDWFs, the one-dimensional lightcone distribution amplitudes (LCDAs) are used instead in most analyses of $B$ decays \cite{Keum:2000wi,Lu:2000em,Nagashima:2002ia}, resulting in uncontrollable errors.  The unprecedented  precision of experimental measurements  of $B$ decays~\cite{Cerri:2018ypt} strongly requests a reliable theoretical  knowledge of TMDWFs. 



A common feature of TMDPDFs and TMDWFs is that they depend both on the longitudinal momentum fraction $x$ and on the transverse spatial separation of partons.  Considerable theoretical efforts have been devoted in recent years to determine these quantities by fitting the pertinent experimental data~\cite{Landry:1999an,Landry:2002ix,DAlesio:2014mrz,Sun:2014dqm,Konychev:2005iy,Bacchetta:2017gcc,Scimemi:2017etj,Scimemi:2019cmh,Bacchetta:2019sam}, which, however, %inevitably introduces systematic uncertainties. Thus 
is limited by the imprecise knowledge of the nonperturbative behaviour of TMDPDFs and TMDWFs. Thus, it is highly desirable to find a method to calculate them from first-principle approaches such as lattice QCD.


This has been realized in the framework of large momentum effective theory\cite{Ji:2013dva,Ji:2014gla}, which offers a systematic way to calculate light-cone correlations by simulating time-independent Euclidean correlations on the lattice. Significant progress has been made {\bl in} calculating various parton quantities from LaMET. For recent reviews, see Ref.~\cite{Cichy:2018mum,Ji:2020ect}. 

A very important  {\bl property of} LaMET is that the TMDPDFs and TMDWFs can be accessed through the so-called quasi-TMDPDFs and quasi-TMDWFs~\cite{Ebert:2019okf,Ji:2019sxk,Ji:2019ewn,Ji:2021znw}. In Ref.~\cite{Ji:2019sxk}, it has been {\bl noticed for the first time} that  a four-quark form factor can be directly calculated on the lattice,  and {\bl that} TMD factorization ensures that this quantity  is factorized into  TMDWFs,  soft factor and the matching kernel.  The matching kernel is perturbative, but  obtaining the full TMD distribution requires detailed  simulations  of both four-quark form factors and quasi TMDWFs on the lattice~\cite{Ji:2019sxk,Ji:2019ewn}.  On the other hand, one can make use of the multiplicative factorization to obtain {\bl the} CS kernel {\bl from} quasi TMDWFs. The first {\bl results for the CS kernel based on} these proposals {\bl have been published recently~\cite{Shanahan:2020zxr,LatticeParton:2020uhz,Schlemmer:2021aij,Li:2021wvl,Shanahan:2021tst}.} 
%In addition, the one-loop perturbative contribution to the matching from quasi TMDWFs to TMDWFs is also presented in Ref.~\cite{Ji:2021znw}.  

In this work, we present a state-of-the-art calculation of the CS kernel, based on a lattice QCD analysis of quasi-TMDWFs with {\bl  $N_f=2+1+1$  valence clover fermions on a staggered quark sea ???.} A single ensemble with lattice spacing {\bl $a\simeq0.12$fm,  volume $n_s^3\times n_t=48^3\times64$, and  physical sea-quark masses is used}. In order to improve the signal-to-noise ratio, we tune the light-valence quark {\bl masses such that} $m_\pi = 670$ MeV.  The CS  kernel is then extracted through the ratios of the quasi-TMDWFs and the perturbative matching kernels at different momenta, $P^z=2\pi/n_s\times\{8,10,12\}=\{1.72,~2.15,~2.58\}$GeV. This  corresponds to the boost factor  $\gamma=\{2.57,~3.21,~3.85\}$, respectively.
%\red{[What's the point of this? No need to do chiral extrapolation? The pion mass here is not too small compared with the momentum, anyway.]} 
This analysis improves the previous ones~\cite{LatticeParton:2020uhz,Li:2021wvl} by taking into account  the one-loop perturbative contributions, and analyzing  systematic uncertainties from {\bl operator} mixing,  higher-order corrections from the scale dependence, and higher power corrections in terms of $1/P^z$.  
 

The rest of this paper is organized as follows. 
In Sec.~\ref{sec:framework}, we give the theoretical framework to extract the CS kernel from  quasi-TMDWFs. Numerical results for quasi-TMDWFs and CS kernel are {\bl presented in} Sec.~\ref{sec:numerics}. A brief  summary of this work is given in Sec.~\ref{sec:summary}.  More details about the analysis are collected in the appendix. 


%%%%%%%%%%%%%%%%%%%%%%%%%%%%%%%%%%%%%%%%%%%%%%%%%%%%%%%%%%%%%%%%%%%%
%%% Section II: Theoretical framework
%%%%%%%%%%%%%%%%%%%%%%%%%%%%%%%%%%%%%%%%%%%%%%%%%%%%%%%%%%%%%%%%%%%%

  
\section{Theoretical framework}
\label{sec:framework}
 

   
\subsection{Collins-Soper Kernel and Rapidity Evolution}


Unlike the  lightcone  PDFs and  distribution amplitudes,  the TMDPDFs and TMDWFs depend on both the renormalization scale $\mu$ and an additional rapidity renormalization scale. The latter arises because the matrix elements also suffer from {\bl so-called} rapidity divergences that require a dedicated regulator~\cite{Collins:1981uk,Becher:2010tm,Chiu:2011qc}.  In TMD factorization, the {\bl ??? contributions of hard, i.e. highly virtual,  modes to the Born process are usually calculated in} the dimensional regularization scheme. {\bl ??? Collinear modes in distinct  directions and soft modes share the same virtuality ???} , %and are connected by Lorentz transformation. 
and are only distinguishable by their rapidity. In calculations using {\bl regularisation} schemes such as dimensional {\bl regularisation}, which only regulate virtualities, one will encounter additional rapidity divergences that arise in soft and collinear matrix elements when integrating {\bl over rapidity}, and have to be resolved using a dedicated regulator. After  {\bl regularization, TMDPDFs and TMDWFs} acquire an additional rapidity scale dependence. This dependence should cancel in theoretical predictions for  physical observables. 


The CS kernel $K\left(b_{\perp}, \mu\right)$, known as the rapidity anomalous dimension, encodes the rapidity dependence  of the TMD distributions~\cite{Collins:1981va,Collins:1981uk}:
\begin{align}
	2 \zeta \frac{d}{d \zeta} \ln f^{\mathrm{TMD}}\left(x, b_{\perp}, \mu, \zeta\right)=K\left(b_{\perp}, \mu\right), \label{eq:TMDsrapidityevolution}
\end{align}
where $f^{\mathrm{TMD}}$ denotes {\bl any leading twist ??? TMDPDF or TMDWF. The} TMD distributions depend on the longitudinal momentum fraction $x$, transverse spatial separation $b_{\perp}$, which is the Fourier-conjugate to the transverse momentum $k_{\perp}$, as well as the renormalization scale $\mu$ and rapidity scale $\zeta$ {which is related to the hadron momentum}. The $\mu$-dependence of CS kernel $K\left(b_{\perp}, \mu\right)$ satisfies the following renormalization group equation (RGE):
\begin{align}
	\mu^{2} \frac{d}{d \mu^{2}} K\left(b_{\perp}, \mu\right)=-\Gamma_{\text {cusp}}\left(\alpha_{s}\right).  \label{eq:cuspanomalousdimension}
\end{align}
Here $\Gamma_{\text {cusp}}\left(\alpha_{s}\right)=\alpha_sC_F/\pi+\mathcal{O}(\alpha_s^2)$ is the cusp anomalous dimension, {\bl which} has been calculated  in  perturbation theory  up to 2-loop in Ref.~\cite{Li:2016ctv}, and 3-loop in Ref.~\cite{Moch:2017uml}. The solution to  the RGE  can be expressed as: 
\begin{align}
	K\left(b_{\perp}, \mu\right)=-2 \int_{1 / b_{\perp}}^{\mu} \frac{\mathrm{d} \mu^{\prime}}{\mu^{\prime}} \Gamma_{\text {cusp}}\left(\alpha_{s}\left(\mu^{\prime}\right)\right)+K\left(\alpha_{s}\left(1 / b_{\perp}\right)\right). 
	\label{eq:nonper-per-CS-kernel}
\end{align} 
{\bl For large $b_{\perp}$}, where $b_{\perp}^{-1}\lesssim\Lambda_{\mathrm{QCD}}$, the CS kernel becomes nonperturbative, which is  represented by the non-cusp anomalous dimension $K\left(\alpha_{s}\left(1/b_{\perp}\right)\right)$ in the above equation. 

In the past decades, the CS kernel has  been widely studied in global fits of  TMD parton distributions~\cite{Landry:1999an,Landry:2002ix,DAlesio:2014mrz,Sun:2014dqm,Konychev:2005iy,Bacchetta:2017gcc,Scimemi:2017etj,Scimemi:2019cmh,Bacchetta:2019sam}. The explicit form in the nonperturbative region can only be parametrized by extending the perturbative expressions at  small {\bl $b_{\perp}$}, which inevitably introduces systematic uncertainties. A direct calculation on the lattice  of TMDPDFs and the relevant CS kernel {\bl was an almost unsurmountable hurdle} until the establishment  of LaMET~\cite{Ji:2013dva,Ji:2014gla}.  A remarkable {\bl recent development} in LaMET is that  these quantities  can    be accessed through the corresponding quasi observables~\cite{Ebert:2019okf,Ji:2019sxk,Ji:2019ewn,Ji:2021znw}.
%\red{[The last few sentences are repeating some previous paragraph.]}

%%%%%%%%%%%%%%%%%%%%%%%%%%%%%%%


\subsection{Quasi TMD Wave Functions}
As stated above, one can define the quasi TMDWFs for a highly-boosted pion along the $z$-direction with large momentum $P^z$ as: 
\begin{align}
	\tilde{\Psi}^{\pm}&\left(x,b_{\perp},\mu,\zeta_z\right)=\lim_{L\to\infty}\int\frac{dz}{2\pi}e^{ixzP^z}\frac{\tilde{\Phi}^{\pm0}\left(z,b_{\perp},P^z,a,L\right)}{\sqrt{Z_E(2L,b_{\perp},\mu, a)}},\label{eq:quasiWFinmomentumspace}
\end{align}
where the unsubtracted quasi TMDWF is defined as an equal-time correlator containing nonlocal quark bilinear operator with staple-shaped gauge link:
\begin{align}
	\tilde{\Phi}^{\pm0}&\left(z,b_{\perp},P^z,a,L\right) =\left\langle 0\right|\bar{\psi}\left(z \hat{n}_{z}/2+b_{\perp} \hat{n}_{\perp}\right) \Gamma \nonumber\\
	&\times U_{\sqsupset, \pm L}\left(z \hat{n}_{z}/2+b_{\perp} \hat{n}_{\perp},-z \hat{n}_{z}/2\right) \psi\left(-z \hat{n}_{z}/2\right)\left| P^{z}\right\rangle. \label{eq:quasiWFincoordinatespace}
\end{align}
For a pseudoscalar mesonic state, the Dirac structure $\Gamma$ can be chosen as $\gamma^z\gamma_5$ or $\gamma^t\gamma_5$, which approaches the leading-twist structure $\gamma^+\gamma_5$ in {\bl the} light-cone limit. With a large but finite $P^z$, the difference between $\gamma^z\gamma_5$ and $\gamma^t\gamma_5$  is power suppressed by $P^z$. Technically, one can also use a combination of them, such as $\left(\gamma^z\gamma_5+\gamma^t\gamma_5\right)/2$ to diminish  power corrections, and more details can be found  in Sec.~\ref{sec:operator}.  Various combinations were also explored in Ref.~\cite{Li:2021wvl}.  The {\bl superscript ``0''} in $\tilde{\Phi}^{\pm0}$ indicates bare quantities. The linear divergences come from the radiation of collinear gluons {\bl ??? by} the gauge-link,  and show up as singularities in loop momentum integrals in the gauge-link's self-energy. Therefore, this divergence appears in the staple-shaped gauge link,  
\begin{align}
	&U_{\sqsupset, \pm L}\left(z \hat{n}_{z}/2+b_{\perp} \hat{n}_{\perp},-z \hat{n}_{z}/2\right) \equiv\nonumber\\
	&	U_z^{\dagger}\left(z \hat{n}_{z}/2+b_{\perp} \hat{n}_{\perp}; L \right) U_{\perp}\big((L-z /2)\hat{n}_{z};{b}_T \big) U_z\left( -z \hat{n}_{z}/2; L \right), \label{eq:stapleshapedUlink}
\end{align}
and cannot be regularized by the standard UV regulator in dimensional regularization. The Euclidean gauge link in $U_{\sqsupset, \pm L}$ is defined as
\begin{align}
	U_z(\xi,\pm L)=\mathcal{P}	\exp \left[-i g \int_{\xi^{z}}^{\pm L} d \lambda n_{z} \cdot A\left(\vec{\xi}_{\perp}+n_{z} \lambda\right)\right],
\end{align}
where $\xi^{z}=-\xi \cdot n_{z}$. The $\pm L$  corresponds to the farthest difference that the gauge link can reach in positive or negative $n_z$  {\bl direction}  on a finite Euclidean lattice. This is depicted as the blue and red lines {\bl in} Fig.~\ref{fig:definitionofquasiTMDWF}.



%%%%%%%%%%%%%%%%%%%%%%%%%%%%%%%%%%
\begin{figure}
\centering
\includegraphics[scale=0.45]{plots/quasi-TMDWF.eps}
\caption{Illustration of the staple-shaped gauge-link included in unsubtracted quasi-TMDWFs  and related Wilson loop. The blue and red double lines in the upper panel represent the $L$-shift direction on the Euclidean lattice, and the lower panel shows the correspondingly Wilson loop, which will subtract   UV logarithmic  and linear divergences in  quasi-TMDWFs . }
\label{fig:definitionofquasiTMDWF}
\end{figure}
%%%%%%%%%%%%%%%%%%%%%%%%%%%%%%%%%%

Since the  linear divergence is associated with the gauge-link, it can be removed by a similar gauge-link with the same total length. An  optional choice  is to make use the Wilson loop,  denoted as $Z_E$. The Wilson loop can be chosen  as the vacuum expectation of a flat rectangular Euclidean Wilson-loop {\bl in the} $z$-$\perp$ plane:
\begin{align}\label{eq:WilsonLoop}
Z_E\left(2L, b_{\perp}, \mu,a\right)=\frac{1}{N_{c}} \operatorname{Tr}\left\langle 0\left|U_{\perp}(0;b_{\perp}) U_{z}\left(b_{\perp}\hat{n}_{\perp}; 2L\right)\right| 0\right\rangle. 
\end{align}
%here $2L$ is the twice length of $U_z^{(\dagger)}$ in the staple-shaped gauge-link, so the square root of $Z_E\left(2L, b_{\perp}, \mu\right)$ will cancel the linear divergence in gauge-link and guarantee the existence of the $L\to\infty$ limit, as well as remove additional co ntributions from the transverse gauge link from the finite ``infinity" on the Euclidean lattice. Apart from this, the Wilson loop will also performs as an UV subtractor for the unsubtracted quasi-WF matrix element. The UV divergence is regulated by lattice spacing $a$ and after subtraction, the logarithmic UV divergences are still present and formulated as the $\mu$ dependence. \red{(How this subtraction worked?)}
Here  the length of $Z_E$ is twice {\bl that} of the staple-shaped gauge-link $U_z^{(\dagger)}$ in the $z$ direction, and thus it is anticipated that the square root of $Z_E\left(2L, b_{\perp}, \mu\right)$ {\bl ??? cancels} the linear divergence and heavy quark potential in {\bl the} gauge-link.  There are residual logarithmic divergences from the vertex of   Wilson line and light quark, which can be removed by {\bl forming an appropriate ratio with a by short distance quasi-PDF matrix element~\cite{Ji:2021uvr}. As} these logarithmic divergences are independent of $z$, $b_{\perp}$, $P_{z}$ and $L$, they will explicitly cancel out when  the ratio of quasi TMDWFs is studied. 



%%%%%%%%%%%%%%%%%%%%%%%%%%%%%%%%%%%%%%%%
\subsection{Factorization of Quasi TMDWFs}\label{sec:factorizationformula}

With the help of soft functions, the infrared contributions in the subtracted quasi TMDWFs can be properly accounted {\bl for} such that the same infrared structures {\bl for} the   quasi TMDWFs and light-cone ones are guaranteed. {\bl This implies} a multiplicative factorization theorem in the framework of LaMET~\cite{Ebert:2019okf,Ji:2019sxk,Ji:2019ewn,Ji:2021znw}:
\begin{align}
&\tilde \Psi^{\pm}(x, b_{\perp}, \mu, \zeta_z) S_r^{1/2}(b_{\perp}, \mu) \nonumber\\
&= H^{\pm}\left(\zeta_z, \overline\zeta_z, \mu^2\right) \exp\left[\frac{1}{2}K(b_{\perp},\mu)\ln\frac{\mp\zeta_z- i\epsilon}{\zeta} \right]  \nonumber\\
& \qquad\times \Psi^{\pm}(x, b_\perp, \mu, \zeta)+\mathcal{O}\left(\frac{\Lambda_{QCD}^2}{\zeta_z}, M^2\zeta_z,\frac{1}{b_{\perp}^2\zeta_z}\right),
\label{eq:matching}
\end{align}
where the superscript $\pm$ in Eq.~\eqref{eq:matching} corresponds to the direction in the Wilson line, $\Psi^{\pm}$ is the TMDWFs defined in the infinite momentum frame. The reduced soft function $S^{1/2}_r(b_{\perp}, \mu)$ emerges from the different soft gluon radiation effects  {\bl for} $\tilde \Psi^{\pm}$ and $\Psi^{\pm}$, and {\bl is defined as a combination of non-cancelling soft functions ???}~\cite{Ji:2021znw}. The mismatch  of the rapidity scale $\zeta$ and $\zeta_z$ can be compensated by the CS kernel $K(b_{\perp},\mu)$. Both $S$ and $K$ are independent of the $\pm$ choice.  $H^{\pm}$ is the 1-loop perturbative matching kernel~\cite{Ji:2021znw}:
\begin{align}
& H^{\pm}(\zeta_z, \overline{\zeta}_z,\mu) \nonumber\\
& = 1+  \frac{ \alpha_s C_F}{4\pi} \bigg(-\frac{5\pi^2}{6}-4 +{\ell}_{\pm} + \overline \ell_{\pm} - \frac{1}{2} ({\ell}_{\pm}^2 + \overline \ell_{\pm}^2) \bigg).  \label{eq:1loophardkernel}
\end{align}
{\bl With the abbreviations} ${\ell}_{\pm} = \ln \left[(-\zeta_z \pm i\epsilon)/\mu^2\right]$, and $\overline {\ell}_{\pm} = \ln \left[(-\overline \zeta_z \pm i\epsilon)/\mu^2\right]$, {\bl the rapidity scales $\zeta_z = (2x P^z)^2$ and $\overline\zeta_z = \left(2\bar{x} P^z\right)^2$, and}  $\bar{x}=1-x$. It should be noticed that $H^{\pm}$ contains nonzero imaginary parts in $\ell_{\pm}$ and $\bar{\ell}_{\pm}$. While the imaginary parts in  $\ell_{\pm}$  are constants, the  ones in the double logarithms $\bar{\ell}_{\pm}$ are momentum-dependent.

%\red{xxx $\ell\sim\i\pi$, $\ell^2$ momentum dependent}

A characteristic behavior of Eq.(\ref{eq:matching}) is that this factorization is multiplicative {\bl \em Note: A factor is by definiton multiplicative ???}, which indicates {\bl that} hard gluon contributions are local. This is due to the fact that hard gluon exchange between  the quark and anti-quark  sectors in quasi TMDWFs {\bl is} power suppressed: if there were such a hard gluon, the spatial separation between its attachments  is much smaller than $b_{\perp}$, resulting {\bl in} power suppression compared to the typical hard mode contributions.  
Thus, at leading power the factorization of quasi TMDWFs  are multiplicative  {\bl \em Note: A factor is by definiton multiplicative ???}.  This feature  is illustrated in Fig.\ref{fig:reduced_graph}, in which the collinear, soft and hard sub-diagrams represent the pertinent  contributions. Further, $\zeta_z$ arising from   Lorentz-invariant combinations of collinear {\bl momentum modes}, will provide the natural hard scale of the hard sub-diagram. More detailed explanations for the factorization of quasi TMDPDFs in LaMET are given in the recent review~ \cite{Ji:2019ewn}. 


%%%%%%%%%%%%%%%%%%%%%%%%%%%%%%%%%%
\begin{figure}
\centering
\includegraphics[scale=0.4]{plots/reduced_graph.eps}
\caption{Leading-power reduced graph for pion quasi TMDWFs. Here $C, ~S$ denote the collinear and soft sectors of the infra-red structure, while $H$ denote the hard contributions. Since the hard-gluon exchange between the quark and anti-quark is power suppressed, the hard parts are disconnected with each others, and therefore the factorization of quasi-WF amplitude is multiplicative.  }
\label{fig:reduced_graph}
\end{figure}
%%%%%%%%%%%%%%%%%%%%%%%%%%%%%%%%%%




\subsection{Collins-Soper Kernel From Quasi TMDWFs}\label{sec:extractingCSkernel}


From Eq.(\ref{eq:matching}), one can see that the momentum dependence in   quasi TMDWFs {\bl provides} an option to determine the CS kernel. This can be written  in a way similar {\bl to} Eq.(\ref{eq:TMDsrapidityevolution})~\cite{Ji:2021znw}: 
\begin{align}
	2 \zeta_z \frac{d}{d \zeta_z} \ln \tilde \Psi^{\pm}\left(x, b_{\perp}, \mu, \zeta_z\right)=&K\left(b_{\perp}, \mu\right) +\frac{1}{2}\mathcal{G}^{\pm}\left(x^2\zeta_z,\mu\right) \nonumber\\
	+& \frac{1}{2}\mathcal{G}^{\pm}\left(\bar{x}^2\zeta_z,\mu\right)+\mathcal{O}\left(\frac{1}{\zeta_z}\right), \label{eq:quasiTMDsrapidityevolution}
\end{align}
where $K\left(b_{\perp}, \mu\right)$ denotes the same kernel as in Eq.(\ref{eq:TMDsrapidityevolution}), and {\bl does not depend} on the hard scale $\zeta_z$ for large $P^z$. Unlike TMDWFs, the  quasi distributions also contain hard contributions, whose  rapidity dependence is  represented by the perturbative  $\mathcal{G}^{\pm}$ as a function of hard scale $\zeta_z$. 
{\bl From the $\zeta_z$ dependence} of quasi TMDWFs, we can see that when $P^z\to\infty$, the large logarithms in $P^z$ are partially absorbed {\bl into} $K\left(b_{\perp}, \mu\right)$  and the remanent is incorporated in the perturbative matching kernel. Therefore,  both the matching kernel $H^{\pm}$  and an exponential of  CS kernel $K\left(b_{\perp}, \mu\right)$ are needed to describe the dependence on $\zeta_z$ of quasi TMDWFs. 

In order to extract the CS kernel $K\left(b_{\perp}, \mu\right)$ explicitly, one  can make use of    Eq.(\ref{eq:matching}) with  two different large {\bl momenta}  $P_1^z\neq P_2^z\gg1/b_{\perp}$ but the same   scale $\zeta_z$. Taking a  ratio of these two quantities  gives 
\begin{align}
	\frac{\tilde \Psi^{\pm}(x, b_{\perp}, \mu,P_1^z)}{\tilde \Psi^{\pm}(x, b_{\perp}, \mu, P_2^z)}=\frac{H^{\pm}\left(xP_1^z,\mu \right)}{H^{\pm}\left(xP_2^z,\mu \right)}\exp\left[K(b_{\perp},\mu) \ln\frac{P_1^z}{P_2^z}\right], \label{eq:ratioofquasiWF}
\end{align}
where the reduced soft function $S_r(b_{\perp},\mu)$ and TMDWFs $\Psi^{\pm}(x,b_{\perp},\mu,\zeta)$  have been canceled in the ratio. Therefore, the CS kernel $K\left(b_{\perp}, \mu\right)$ can be extracted through 
\begin{align}
	K\left(b_{\perp}, \mu\right)=\frac{1}{\ln(P_1^z/P_2^z)} \ln\frac{H^{\pm}(xP_2^z,\mu)\tilde{\Psi}^{\pm}(x,b_{\perp},\mu,P_1^z)}{H^{\pm}(xP_1^z,\mu)\tilde{\Psi}^{\pm}(x,b_{\perp},\mu,P_2^z)}. \label{eq:extractingCSkernel}
\end{align}
Note that the extracted result is formally independent of $x$ and $P_{1/2}^z$ at leading power,  and both $\tilde{\Psi}^{+}$ and $\tilde{\Psi}^{-}$ can be used to {\bl extract} $K\left(b_{\perp}, \mu\right)$.  Such behavior is derived at the leading power in the factorization scheme and  might be undermined by power corrections. Accordingly, in order  to diminish  the systematic uncertainties,  we {\bl take} the average: 
 \begin{align}
	K\left(b_{\perp}, \mu\right)=&\frac{1}{2\ln(P_1^z/P_2^z)} \left[\ln\frac{H^{+}(xP_2^z,\mu)\tilde{\Psi}^{+}(x,b_{\perp},\mu,P_1^z)}{H^{+}(xP_1^z,\mu)\tilde{\Psi}^{+}(x,b_{\perp},\mu,P_2^z)}\right.  \nonumber\\
	&+\left.\ln\frac{H^{-}(xP_2^z,\mu)\tilde{\Psi}^{-}(x,b_{\perp},\mu,P_1^z)}{H^{-}(xP_1^z,\mu)\tilde{\Psi}^{-}(x,b_{\perp},\mu,P_2^z)}\right].
\end{align}
The details will be discussed in Sec. \ref{subsec:csresults}.
%%%%%%%%%%%%%%%%%%%%%%%%%%%%%%%%%%%%%%%

\section{Numerical Simulations and Results}
\label{sec:numerics}

\subsection{Lattice setup }

The numerical simulation {\bl ??? use $N_f=2+1+1$ valence clover fermions on a  highly improved staggered quark (HISQ) sea ???~\cite{Follana:2006rc} and a 1-loop} Symanzik improved gauge action~\cite{Symanzik:1983dc}, generated by the MILC collaboration~\cite{MILC:2012znn} using periodic boundary conditions. In the calculation, we use a single ensemble with the lattice spacing $a\simeq0.12$fm and the {\bl volume} $n_s^3\times n_t=48^3\times64$ at physical sea-quark masses. In order to increase the signal-to-noise ratio in the numerical simulation, we tune the light-valence quark masses to the strange one, namely $m_{\pi}^{\mathrm{sea}}=130$MeV and  $m_{\pi}^{\mathrm{val}}=670$MeV. {\bl (The Collins-Soper kernel should only depend weakly on quark mass.) ???}  To further improve the statistical signals, we adopt {\bl  hypercubic (HYP) smeared fat links}~\cite{Hasenfratz:2001hp} for the gauge ensembles. To access  the infinite momentum limit for the CS kernel, we employ three different hadron momenta as $P^z=2\pi/n_s\times\{8,10,12\}=\{1.72,~2.15,~2.58\}$GeV  corresponding  to the boost factor $\gamma=\{2.57,~3.21,~3.85\}$. 

%%%%%%%%%%%%%%%%%%%%%%%%%%


\subsection{Quasi TMDWFs From Two Point Correlation Functions}
\label{sec:quasi-WF_from_2pt}
In order to calculate the quasi TMDWFs  defined in Eq.(\ref{eq:quasiWFinmomentumspace}), we generate {\bl Coulomb gauge fixed wall source propagators}
\begin{align}
S_{w}\left(x, t, t^{\prime} ; \vec{p}\right)=\sum_{\vec{y}} S\left(t, \vec{x} ; t^{\prime}, \vec{y}\right) e^{i \vec{p} \cdot(\vec{y}-\vec{x})}, \label{eq:wallsource}
\end{align}
where $(t^{\prime},\vec{y})$ and $(t,\vec{x})$ denote the {\bl space-time} positions of source and sink. Then one  can construct the two-point function (2pt) related to the quasi TMDWFs  in Eq.(\ref{eq:quasiWFinmomentumspace}):
\begin{align}
C_2^{\pm}&	\left(z,b_{\perp},P^z; p^z,L,t \right)=\frac{1}{n_s^3}\sum_{\vec{x}}\mathrm{tr} e^{i\vec{P}\cdot\vec{x}}\left\langle S_w^{\dagger}\left(\vec{x}_1,t,0;-\vec{p}\right)\right. \nonumber\\
&\;\;\;\;\;\;\;\;\;\; \;\;\;\;\;\;\;\;\;\;    \times \left. \Gamma U_{\sqsupset, \pm L}\left(\vec{x}_1,\vec{x}_2\right) S_w\left(\vec{x}_2,t,0;\vec{p} \right)  \right\rangle
\label{eq:nonlocal2pt}
\end{align}
with $\vec{x}_1=\vec{x}+z\hat{n}_z/2+b_{\perp}\hat{n}_{\perp}$ and  $\vec{x}_2=\vec{x}-z \hat{n}_{z}/2$. The quark momentum $\vec{p}=(\vec{0}_{\perp},p^z)$ is along the $z$-direction, and each of two quarks carries {\bl half of the} hadron momentum. Thereby the hadron momentum satisfies  $\vec{P}=2\vec{p}$. The anti-quark propagator can be obtained from Eq.(\ref{eq:1loophardkernel}) by applying $\gamma_5$-{\bl hermiticity} $S_w^{\dagger}(x,y)=\gamma_5S_w(y,x)\gamma_5$. As mentioned above, the Dirac structures are chosen as $\Gamma=\gamma^z\gamma_5$ and $\gamma^t\gamma_5$ that can be projected onto the leading twist light-cone contributions in the large $P^z$ limit.
 

By generating the wall source propagators with quark momentum $p^z=\pm\{4,5,6\}\times2\pi/(n_sa)$, and three segments of gauge links {\bl following} Eq.(\ref{eq:stapleshapedUlink}), one  can construct  the 2pt {\bl function} on the  lattice. With the help of {\bl reduction formulas}, the $C_2^{\pm}$ can be parametrized as
 \begin{align}
C_2^{\pm}&\left(z,b_{\perp},P^z; p^z,L,t \right)=\frac{A_w(p^z)A_p}{2E}\tilde{\Phi}^{\pm0}\left(z,b_{\perp},P^z,L\right) \nonumber\\
& \times e^{-Et}\left[1+c_0\left(z,b_{\perp},P^z,L\right)e^{-\Delta Et} \right], \label{eq:C2parametrization}
 \end{align}
where $A_w(p^z)$ is the matrix element of {\bl the pion interpolating} field with Coulomb gauge fixed wall source and $A_p$ is the one {\bl for a} point source (sink). These terms as well as the {\bl factor $E^{-1}=1/\sqrt{m_{\pi}^2+\left(P^z\right)^2}$ are cancelled} by the local  two-point function $C_2^{\pm}\left(0,0,P^z; p^z,0,t \right)$ at the same time slice. Thus the remaining  ground-state matrix element $\tilde{\Phi}^{\pm0}$ is normalized. The ratio of nonlocal and local 2pt {\bl functions} can be parametrized  as
\begin{align}
&R^{\pm}\left(z,b_{\perp},P^z,L,t \right)=\frac{C_2^{\pm}\left(z,b_{\perp},P^z,L,t \right)}{C_2\left(0,0,P^z,0,t \right)} \nonumber\\
=&\tilde{\Phi}^{\pm0}\left(z,b_{\perp},P^z,L\right)\left[1+c_0\left(z,b_{\perp},P^z,L\right)e^{-\Delta Et} \right].
\label{eq:two-state-fit}
\end{align}
 %}
 
 
 %%%%%%%%%%%%%%%%%%%%%%%%%%%%%%%%%%
\begin{figure}
\centering
\includegraphics[scale=0.55]{plots/t_dep_b2z0_r.eps}
\caption{Comparison of  two-state fit  and  one-state fit to {\bl extract} $\tilde{\Phi}^{\pm0}\left(z,b_{\perp},P^z,L\right)$. {\bl Taking} $\{z,b_{\perp},P^z,L\}=\{0a, 2a, 24\pi/n_s, 6a\}$ {\bl as} example, we can see that  the two-state fit works {\bl for} $t\in[2a, 8a]$ while the one-state fit works {\bl for} $t\in[5a, 8a]$. The fitted  results are consistent with each other, while  the one-state fit is more conservative. }
\label{fig:tdependenceofratio}
\end{figure}
%%%%%%%%%%%%%%%%%%%%%%%%%%%%%%%%%%
 
 
In the above parametrization,  the excited-state contributions are collected into the $c_0$ term, and $\Delta E$ denotes the mass gap between the ground and first excited state.  With {\bl increasing Euclidean time}, contributions from the excited state {\bl decay} and the plateau {\bl obtained for}  $R^{\pm}\left(z,b_{\perp},P^z,L,t \right)$ at  large {\bl times} reflects the ground-state contribution $\tilde{\Phi}^{\pm0}$. {\bl We} employ two methods to extract {\bl $\tilde{\Phi}^{\pm0}$,} namely the two-state fit directly using Eq.~\eqref{eq:two-state-fit}, and the one-state fit {\bl setting} $c_0=0$. Fig.~\ref{fig:tdependenceofratio} {\bl compares the results for both} methods. From this figure,  one can see that {\bl these results are consistent within error. While the two-state fit works for $t\in[2a, 8a]$, the one-state fit works for $t\in[5a, 8a]$}.   More results  are given {\bl in Appendix}~\ref{appsec:C2}. In most cases,  we find that the one-state fit results are consistent with and   more conservative than {\bl the two-state fit ones. In the following analysis we chose} the more conservative results from the one-state fit as the central results. 



%%%%%%%%%%%%%%%%%%%%%%%%%%%%%%%%%%
\subsection{Wilson Loop}
\label{subsec:numerical_wl}
%%%%%%%%%%%%%%%%%%%%%%%%%%%%%%%%%%

%The unsubtracted quasi-WF matrix elements extracted from the joint fit of 2pt data will suffer linear divergence, which comes from the staple-shaped gauge link. \red{Discuss how the linear divergence and UV divergence are canceled...}

The unsubtracted quasi TMDWF matrix elements $\tilde{\Phi}^{\pm0}\left(z,b_{\perp},P^z,a,L\right)$ extracted from the joint fit of the  two-point function {\bl contain the factor $e^{-\delta \bar{m}(2L+b_{\perp})}$ from the linear divergence, the } heavy quark effective potential factor $e^{-V(b_{\perp})L}$ and logarithmic divergences $Z_{O}$:
\begin{align}
\tilde{\Phi}^{\pm0}\left(z,b_{\perp},P^z,a,L\right) \propto e^{-\delta \bar{m}(2L+b_{\perp})}e^{-V(b_{\perp})L}Z_{O}.
\end{align} 

The linear divergence {\bl in} $e^{-\delta \bar{m}(2L+b_{\perp})}$ comes from the self-energy of  {\bl the}  Wilson line~\cite{Ji:2017oey,Ishikawa:2017faj,Green:2017xeu, Ji:2020brr}, where $\delta\bar m$ {\bl contains a term  proportional to} $1/a$  and {\bl a  non-perturbative renormalon contribution} $m_{0}$:
\begin{align}
\delta\bar m = \frac{m_{-1}(a)}{a} - m_0
\end{align}
{\bl Note} that the exponent {\bl of the linear divergence term} is proportional to the total length of {\bl the} Wilson link, e.g. $2L+b_{\perp}$ for the staple link. Due to this factor,  the numerical value for {\bl a} Wilson loop dramatically {\bl decreases}  for small $a$ and large $L$. 
%Since this factor is not of our physical interest, and it breaks the continuum limit and large distance behavior (see the  unsubtracted case in Fig.~\ref{fig:WilsonLooprenormalization}), we need to get rid of it.

The heavy quark effective potential term $e^{-V(b_{\perp})L}$ comes from {\bl interactions} between the two Wilson lines along the $z$ direction in the staple link. The heavy quark effective potential $V(b_{\perp})$ is often used to determine the lattice spacing of an ensemble.  

\begin{comment}
One can study the Wilson line with heavy quark effective theory and calculate the logarithmic  divergence $Z_{O}$  from the {\bl light-heavy} quark vertex perturbatively at leading order~\cite{Ji:2020ect,Constantinou:2017sej}:
\begin{align}
Z_{O} = 1+\frac{3 C_{F} \alpha_{s}}{2 \pi} \frac{1}{4-D}, 
\end{align}
with $D$ being the space-time dimension. 
The logarithmic  divergence up to next-to-leading order, resumed by renormalization group equation, and matched to lattice is~\cite{Ji:1991pr, LatticePartonCollaborationLPC:2021xdx} 
\begin{align}\label{eq:ZO_RS_higher}
Z_{O}(1/a, \mu)=\left(\frac{\ln [1 /(a \Lambda_{\rm QCD})]}{\ln [\mu / \Lambda_{\rm QCD}]}\right)^{\frac{3 C_{F}}{b_0}}\left(1+\frac{d}{\ln[a \Lambda_{\rm QCD}]}\right),\nonumber\\
\end{align}
where $\Lambda_{\rm QCD}$   is different from that of {\bl the} $\overline{\rm MS}$ scheme and lattice perturbation theory. One can use it to effectively absorb higher order contributions~\cite{Lepage:1992xa}. $d$ is the next to leading order anomalous dimension which contains scheme conversion constants. Both $\Lambda_{\rm QCD}$ and $d$ depend on the specific lattice action.  
\end{comment}

The logarithmic divergence $Z_{O}$ comes from the vertices
involving the Wilson line and light quark. The logarithmic  divergence up to leading order, resumed by renormalization group equation, and matched to lattice is~\cite{Ji:1991pr, LatticePartonCollaborationLPC:2021xdx} 
\begin{align}\label{eq:ZO_RS_higher}
Z_{O}(1/a, \mu)=\left(\frac{\ln [1 /(a \Lambda_{\rm QCD})]}{\ln [\mu / \Lambda_{\rm QCD}]}\right)^{\frac{3 C_{F}}{b_0}},\nonumber\\
\end{align}
where $\Lambda_{\rm QCD}$  is different from that of $\overline{\rm MS}$ scheme and lattice perturbation theory. One can use it to effectively absorb higher order contributions~\cite{Lepage:1992xa}.

In this work,  the Wilson loop renormalization method~\cite{Chen:2016fxx,Zhang:2017bzy,Musch:2010ka,Green:2017xeu,Zhang:2017zfe}  is adopted, in which the Wilson loop $Z_E$ defined in Eq.(\ref{eq:WilsonLoop})  contains linear divergence, heavy quark effective potential and logarithimic divergence:
\begin{align}
Z_E\left(2L, b_{\perp}, a\right) \propto e^{-\delta \bar{m}(4L+2b_{\perp})}e^{-V(b_{\perp})2L}.
\end{align}
According to Ref.~\cite{LatticePartonCollaborationLPC:2021xdx}, $\delta \bar{m}$ in {\bl the Wilson loop is the same as that in hadron matrix elements, and thus it is anticipated that the linear divergence is removed when dividing by} $\sqrt{Z_E\left(2L, b_{\perp}, a\right)}$:
\begin{align}
\tilde{\Phi}^{\pm}(z,b_{\perp},P^z,L)=\frac{\tilde{\Phi}^{\pm0}\left(z,b_{\perp},P^z,a,L\right)}{\sqrt{Z_E\left(2L, b_{\perp}, a\right)}}.
\end{align}
As shown in Fig.~\ref{fig:WilsonLooprenormalization}, the subtracted quasi TMDWFs tend  to  be a constant when  $L\ge 0.4$ fm. We then use the subtracted quasi TMDWFs {\bl defined} as
\begin{align}\label{eq:WLRQuasi}
\tilde{\Phi}^{\pm}(z,b_{\perp},P^z)=\lim_{L\to\infty}\frac{\tilde{\Phi}^{\pm0}\left(z,b_{\perp},P^z,a,L\right)}{\sqrt{Z_E\left(2L, b_{\perp}, a\right)}}.
\end{align}
{\bl However,} it is anticipated that there is a residual logarithmic divergence $Z_{O}$:
\begin{align}
\frac{\tilde{\Phi}^{\pm0}\left(z,b_{\perp},P^z,a,L\right)}{\sqrt{Z_E\left(2L, b_{\perp}, a\right)}}
\propto \frac{e^{-\delta \bar{m}(2L+b_{\perp})}e^{-V(b_{\perp})L}Z_{O}}{e^{-\delta \bar{m}(2L+b_{\perp})}e^{-V(b_{\perp})L}}
= Z_{O}.
\end{align}
In the extraction of {\bl the} CS kernel, a ratio of quasi TMDWFs is adopted and accordingly the residual logarithmic divergence $Z_{O}$ is canceled.




 %%%%%%%%%%%%%%%%%%%%%%%%%%%%%%%%%%
\begin{figure}
\centering
\includegraphics[scale=0.55]{plots/l_dep_8_21_r.eps}
\includegraphics[scale=0.55]{plots/l_dep_8_21_i.eps}
\caption{Results for the $L$-dependence of  unsubtracted and subtracted quasi TMDWFs:  real part (upper panel) and imaginary part (lower panel), as well as the square root of the Wilson loop. The {\bl case} $\Gamma=\gamma^z\gamma_5$ and $\{P^z,b_{\perp}, z\}=\{16\pi/n_s,2a,2a\}$ is used for illustration. }
\label{fig:WilsonLooprenormalization}
\end{figure}
%%%%%%%%%%%%%%%%%%%%%%%%%%%%%%%%%%
 
\subsection{Operator Mixing Effects}
\label{sec:operator}


%%%%%%%%%%%%%%%%%%%%%%%%%%%%%%%%%%%%%%
\begin{figure}
\centering
\includegraphics[scale=0.55]{plots/qwf_co_b3_r_p12.eps}
\includegraphics[scale=0.55]{plots/qwf_co_b3_i_p12.eps}
\caption{$\lambda$-dependence of quasi-WF matrix elements with different Dirac structures. Here we take the case of $\{P^z,b_{\perp}\}=\{24\pi/n_s,3a\}$ as an example, the deviation between these two cases mainly comes from {\bl power corrections. Results for} more sets of $\{P^z,b_{\perp}\}$ are shown in appendix~\ref{sec:app_operator_mixing}.}
    \label{fig:qwf_coordinate}
\end{figure}
%%%%%%%%%%%%%%%%%%%%%%%%%%%%%%%%%%%%%%


For a pseudoscalar meson on a Euclidean lattice, both $\Gamma=\gamma^t\gamma_5$ and $\gamma^z\gamma_5$ {\bl  project}  onto the leading twist light-cone distribution amplitude, i.e. $\gamma^+\gamma_5$ in {\bl the} large $P^z$ limit. The differences  between them arises from power corrections in terms of $M^2/\left(P^z\right)^2$.

Fig.~\ref{fig:qwf_coordinate} shows the comparison of the $\lambda= z   P^z$-dependence of quasi TMDWFs with $\Gamma=\gamma^t\gamma_5$ and $\gamma^z\gamma_5$ at $P^z=24\pi/n_s\simeq2.58$GeV. It can be seen from the plots that there are some differences between the two sets of results for the real part in  the small $\lambda$ region. The differences are expected to decrease with {\bl increasing $P^z$}, and the {\bl correlators} with $\gamma^t\gamma_5$ and $\gamma^z\gamma_5$ will gradually converge to the light-cone from opposite {\bl directions}. Besides, in light-cone coordinate, $\gamma^t$ and $\gamma^z$ can be represented by $\gamma^+$ and $\gamma^-$,
\begin{align}
\gamma^t\gamma_5=\frac{1}{\sqrt{2}}(\gamma^++\gamma^-)\gamma_5,\nonumber\\
\gamma^z\gamma_5=\frac{1}{\sqrt{2}}(\gamma^+-\gamma^-)\gamma_5,
\end{align}
Due to the momentum along the light-cone, {\bl operators} with $\gamma^-\gamma_5$ {\bl correspond to higher order terms} of TMDWFs. Therefore, power corrections {\bl arising} from finite $P^z$ are likely to be eliminated in the average of these two terms:
\begin{align}
\tilde{\Phi}^{\pm}=\frac{1}{2}\left[\tilde{\Phi}^{\pm}\left(\Gamma=\gamma^t\gamma_5\right) + \tilde{\Phi}^{\pm}\left(\Gamma=\gamma^z\gamma_5\right)\right].
\end{align}
%(Could we use P=8 and P=12 to show the deviation will decrease with momentum increasing?)
{\bl For a quantitative analysis see the appendix of~\cite{LatticeParton:2020uhz}. The} operator mixing effect {\bl reaches order 5\%, which is} much smaller than the systematic uncertainties discussed in next subsection. 

According to our numerical simulations, the subtracted quasi TMDWFs in coordinate space $\tilde{\Psi}^{\pm}(x,b_{\perp},P^z)$ as a function of $\lambda=zP^z$ are complex, which {\bl is} shown in Fig.~\ref{fig:qwf_co}. The examples are {\bl the} real and imaginary part of $\tilde{\Psi}^{\pm}(x,b_{\perp},P^z)$ with $P^z=24\pi/n_s$, $b_{\perp}=2a$ and $4a$. %\red{The differences between different momentum cases are related to the CS kernel, which governs the momentum evolution behavior of quasi TMDWFs.} 
To determine quasi TMDWFs in momentum space $\tilde{\Psi}^{\pm}(x,b_{\perp},P^z)$, {\bl we use a brute-force} Fourier transformation (FT)
\begin{align}
\tilde{\Psi}^{\pm}(x,b_{\perp},P^z)=\frac{1}{2\pi}\sum_{z_{min}}^{z_{max}}e^{ixzP^z}\tilde{\Phi}^{\pm}(z,b_{\perp},P^z).
\label{eq:FT}
\end{align}
Due to the imaginary {\bl ??? part of $\tilde{\Phi}^{\pm}(z,b_{\perp},P^z)$, also $\tilde{\Psi}^{\pm}(x,b_{\perp},P^z)$ has an ???}  imaginary part. We obtain the quasi TMDWFs in {\bl  momentum space} for both real and imaginary part of $\tilde{\Psi}^{\pm}(x,b_{\perp},P^z)$ shown in Fig.\ref{fig:qwf_momenta}, by taking $P^z=24\pi/n_s$ and $b_{\perp}=2a$ and $4a$ as examples. {\bl ??? We truncate the FT at $z_{min}$ and $z_{max}$. The deviation of
  $\tilde{\Phi}^{\pm}(z_{min},b_{\perp},P^z)$ and $\tilde{\Phi}^{\pm}(z_{max},b_{\perp},P^z)$ from zero is a measure of the resulting truncation error. For the largest range of $z$ values we could realize numerical,  $z_{min}=-1.44$fm, $z_{max}=1.44$fm, this error is ??? still noticable. This brute-force truncation of the FT leads to an oscillatory behavior of TMDWFs. ???}%\red{[Fig. 7 is 
\newpage
\begin{widetext}

%%%%%%%%%%%%%%%%%%%%%%%%%%%%%%%%%%%%%%
\begin{figure}
\centering
\includegraphics[scale=0.55]{plots/qwf_co_b2.eps}
\includegraphics[scale=0.55]{plots/qwf_co_b4.eps}
\includegraphics[scale=0.55]{plots/qwf_co_i_b2.eps}
\includegraphics[scale=0.55]{plots/qwf_co_i_b4.eps}
\caption{Examples for subtracted quasi TMDWFs in coordinate space. Here we take the cases of $P^z=24\pi/n_s$. According to our results, the subtracted quasi TMDWFs $\tilde{\Psi}^+(z,b_{\perp},P^z)$ are complex, thus the {\bl real and the imaginary part both} need to be investigated. The upper four figures are for subtracted quasi TMDWFs $\tilde{\Psi}^+(z,b_{\perp},P^z)$ as a function of $\lambda=zP^z$ after the average {\bl over the two Dirac structures ($\gamma^t\gamma_5$ and $\gamma^z\gamma_5$) is taken}. The upper left figure shows the real part of $\tilde{\Psi}^+(z,b_{\perp},P^z)$ with $b_{\perp}=2a$, {\bl while} the upper right one is for $\tilde{\Psi}^+(z,b_{\perp},P^z)$ with $b_{\perp}=4a$. {\bl The lower two figures show the} corresponding imaginary parts with $b_{\perp}=2a$ and $4a$.}
    \label{fig:qwf_co}
\end{figure}
%%%%%%%%%%%%%%%%%%%%%%%%%%%%%%%%%%%%%%


%%%%%%%%%%%%%%%%%%%%%%%%%%%%%%%%%%%%%%
\begin{figure}
\centering
\includegraphics[scale=0.55]{plots/qwf_mom_b2.eps}
\includegraphics[scale=0.55]{plots/qwf_mom_b4.eps}
\includegraphics[scale=0.55]{plots/qwf_mom_i_b2.eps}
\includegraphics[scale=0.55]{plots/qwf_mom_i_b4.eps}
\caption{Examples for subtracted quasi TMDWFs in momentum space with {\bl hadron} momentum $P^z=24\pi/n_s$. The subtracted quasi TMDWFs in momentum space $\tilde{\Psi}^+(x,b_{\perp},P^z)$ are Fourier transformed from {\bl $\tilde{\Phi}^+(z,b_{\perp},P^z)$}, which have real and imaginary {\bl part. Both have to be investigated.} Shown in Eq. (\ref{eq:FT}), {\bl a brute-force} FT is used to determine $\tilde{\Psi}^+(x,b_{\perp},P^z)$, where $z_{min}=-1.44$fm, $z_{max}=1.44$fm. The upper left figure shows the real part for $\tilde{\Psi}^+(x,b_{\perp},P^z)$ with $b_{\perp}=2a$ case, while the upper right one is for $\tilde{\Psi}^+(x,b_{\perp},P^z)$ with $b_{\perp}=4a$ case. Correspondingly, the lower two figures shows the imaginary part for $\tilde{\Psi}^+(x,b_{\perp},P^z)$ with $b_{\perp}=2a$ and $4a$.}
    \label{fig:qwf_momenta}
\end{figure}
%%%%%%%%%%%%%%%%%%%%%%%%%%%%%%%%%%%%%%


\end{widetext}

%%%%%%%%%%%%%%%%%%%%%%%%%%%%%%%%%%%%%%
%\begin{figure}
%\centering
%\includegraphics[scale=0.55]{plots/qwf_co_b2.eps}
%\includegraphics[scale=0.55]{plots/qwf_co_b4.eps}
%\caption{Real parts of subtracted quasi TMDWFs $\tilde{\Psi}^+(z,b_{\perp},P^z)$ as a function of $\lambda=zP^z$ after the average of two Dirac structures ($\gamma^t\gamma_5$ and $\gamma^z\gamma_5$). Here we take the cases of $P^z=24\pi/n_s$, and $b_{\perp}=2a$ for the upper one, $b_{\perp}=4a$ for the lower one.}
%    \label{fig:qwf_co}
%\end{figure}
%%%%%%%%%%%%%%%%%%%%%%%%%%%%%%%%%%%%%%

%%%%%%%%%%%%%%%%%%%%%%%%%%%%%%%%%%%%%%
%\begin{figure}
%\centering
%\includegraphics[scale=0.55]{plots/qwf_mom_b2.eps}
%\includegraphics[scale=0.55]{plots/qwf_mom_b4.eps}
%\caption{Real parts of subtracted quasi TMDWFs in momentum space $\tilde{\Psi}^+(x,b_{\perp},P^z)$ Fourier transformed from $\tilde{\Psi}^+(z,b_{\perp},P^z)$. Shown in Eq. (\ref{eq:FT}), the brutal-force FT is used to determine $\tilde{\Psi}^+(x,b_{\perp},P^z)$, where $z_{min}=-1.44$fm, $z_{max}=1.44$fm. The upper figure is for $b_{\perp}=2a$, and the lower one is for $b_{\perp}=4a$.}
%    \label{fig:qwf_momenta}
%\end{figure}
%%%%%%%%%%%%%%%%%%%%%%%%%%%%%%%%%%%%%%

\begin{comment}
\subsection{\red{Scale Dependence of Quasi TMDWFs}}
\label{subsec:scale-dependence}

Moreover, in order to match the $\overline{\text{MS}}$
 renormalization scheme, one  needs to consider the scale dependence of subtracted quasi-TMDWFs $\tilde{\Psi}(x,b_{\perp},P^z)$. For $\tilde{\Psi}(x,b_{\perp},P^z)$, the renormalization scale $\mu$-dependence satisfies the renormalization group equation\cite{Ji:2020ect}:
\begin{eqnarray}
&&\mu^2\frac{d^2}{d\mu^2}\ln\tilde{\Psi}^{\pm}_{\text{Re}}(x,b_{\perp},\mu,\zeta_z)=\gamma_F\left(\alpha_s(\mu)\right),\\
&&\mu^2\frac{d^2}{d\mu^2}\ln\tilde{\Psi}^{\pm}_{\text{Im}}(x,b_{\perp},\mu,\zeta_z)=0.\label{eq:scale}
\end{eqnarray}
For our calculation, the hadron momenta for three cases are $P^z = \{1.72, 2.15, 2.58\}$ GeV, and the results of $\tilde{\Psi}^{\pm}(x,b_{\perp},\mu_0=P^z,P^z)$ can be evolved to $\overline{\text{MS}}$ scheme at $\mu=2$GeV by Eq.(\ref{eq:scale-ev})
\begin{eqnarray}
\tilde{\Psi}^{\pm}_{\mathrm{Re}}(x,b_{\perp},\mu,P_z)&=&\tilde{\Psi}_{\mathrm{Re}}(x,b_{\perp},\mu_0,P_z)\nonumber\\
&&\times\exp\int^{\mu}_{\mu_0}\frac{4d\mu^2}{\beta_0\mu^2(\ln \mu^2-\ln\Lambda^2)},\nonumber\\
\tilde{\Psi}^{\pm}_{\mathrm{Im}}(x,b_{\perp},\mu_0,P_z)&=&\tilde{\Psi}^{\pm}_{\mathrm{Im}}(x,b_{\perp},\mu,P_z). \label{eq:scale-ev}
\end{eqnarray}
Here we adopt the 1-loop result of $\alpha_s(\mu)=\frac{4\pi}{\beta_0\ln\frac{\mu^2}{\Lambda^2}}$, $\beta_0=11-\frac{2}{3}N_f$. The comparison of $\tilde{\Psi}^{+}(x,b_{\perp},\mu_0=P^z,P^z)$ and $\tilde{\Psi}^{+}(x,b_{\perp},\mu,P^z)$ is performed in Fig.~\ref{fig:qwf_ev}, in which not much difference is found. In addition, according to the expression in Eq.(\ref{eq:scale-ev}), the scale factors between $\mu=2$GeV and $\mu_0=P^z$ for $P^z=\{24\pi/n_s,20\pi/n_s,16\pi/n_s\}$ are $\{0.960,0.988,1.027\}$, which are very close to $1$. This calculation reveals quasi TMDWFs is insensitive to the scale. In the following discussion, we choose $\mu=2$GeV as the renormalization scale, and do not distinguish $\Psi^{\pm}(x,b_{\perp},\mu,P^z)$ and $\Psi^{\pm}(x,b_{\perp},P^z)$.

%%%%%%%%%%%%%%%%%%%%%%%%%%%%%%%%%%%%%%
\begin{figure}
\centering
\includegraphics[scale=0.55]{plots/qwf_ev_p12_b3.eps}
\caption{Comparison of numerical results for $\tilde{\Psi}^+_{\text{Re}}(x,b_{\perp},\mu_0=P^z,\zeta_z)$ and $\tilde{\Psi}^+_{\mathrm{Re}}(x,b_{\perp},\mu=2\text{GeV},\zeta_z)$, take the case of $\{P^z,b_{\perp}\}=\{24\pi/L,3a\}$ as a example.}
    \label{fig:qwf_ev}
\end{figure}
%%%%%%%%%%%%%%%%%%%%%%%%%%%%%%%%%%%%%%
\end{comment}
\newpage
\subsection{Collins-Soper Kernel From Quasi TMDWFs}
\label{subsec:csresults}

{\bl The} CS kernel governs the rapidity evolution and thus is independent of the momentum fraction of the involved parton.  But as indicated in Eq.(\ref{eq:matching}), the factorization formula  works {\bl only when} $xP^z\gg \Lambda_{\rm QCD}$, and could be invalid  in the end-point {\bl regions $x\to0,1$. Power corrections are likely of the form $1/\left(xP^z\right)^2$ or $1/\left(\bar{x}P^z\right)^2$. Therefore,  the numerical CS kernel is fitted by a function of $x$, $P_1^z$ and $P_2^z$ and is written} as $K(b_{\perp},\mu,x,P^z_1,P^z_2)$,
\begin{align}
&K(b_{\perp},\mu,x,P^z_1,P^z_2) =\nonumber\\
&\frac{1}{2\ln(P_1^z/P_2^z)} \left[\ln\frac{H^{+}(xP_2^z,\mu)\tilde{\Psi}^{+}(x,b_{\perp},\mu,P_1^z)}{H^{+}(xP_1^z,\mu)\tilde{\Psi}^{+}(x,b_{\perp},\mu,P_2^z)}\right.  \nonumber\\
	&+\left.\ln\frac{H^{-}(xP_2^z,\mu)\tilde{\Psi}^{-}(x,b_{\perp},\mu,P_1^z)}{H^{-}(xP_1^z,\mu)\tilde{\Psi}^{-}(x,b_{\perp},\mu,P_2^z)}\right].
\end{align}
Here $K(b_{\perp},\mu,x,P^z_1,P^z_2)$ are extracted from the perturbative matching kernels and quasi TMDWFs {\bl using 1-loop matching. They will have power corrections of teh form} $\mathcal{O}\left(1/\left(xP^z\right)^2\right)$ and $\mathcal{O}\left(1/\left(\bar{x}P^z\right)^2\right)$. In order to extract the leading power contributions, we adopt  the following {\bl parametrization}
\begin{align}
&K(b_{\perp},\mu,x,P^z_1,P^z_2) = K(b_{\perp},\mu) \nonumber\\
&\quad+A\left[\frac{1}{x^2(1-x)^2(P^z_1)^2}-\frac{1}{x^2(1-x)^2(P^z_2)^2}\right], \label{eq:parametrizationofCSkernel}
\end{align}
where $A$ is the coefficient accounting for  the magnitude of higher power contributions, and can be determined through a joint fit of different lattice data.

Fig.~\ref{fig:K_b} presents {\bl the physical} CS kernel $K(b_{\perp},\mu)$ at different $b_{\perp}$. By employing three cases of quasi TMDWFs  with $P^z=\{8,10,12\}\times2\pi/n_s$, one can extract $K(b_{\perp},\mu,x,P^z_1,P^z_2)$ with $P_1^z/P_2^z=10/8$ and $12/8$, shown as the different colored bands.  Except in the end-point regions ($x<0.2$ or $x>0.8$), the {\bl lattice data is flat} and reflects the leading power contribution, which {\bl conforms with } expectations. Using the parametrization {\bl formula} Eq.(\ref{eq:parametrizationofCSkernel}), the physical CS kernel $K(b_{\perp},\mu)$ can be determined by fitting the data, shown as the {\bl green} band. As mentioned before, at large {\bl $b_{\perp}$, the quasi TMDWFs   show oscillations due to the  truncation of} the Fourier transformation, which also affect the extracted $K(b_{\perp},\mu,x,P^z_1,P^z_2)$, as shown in the lower panel of Fig.\ref{fig:K_b}. 

%\red{(How to deal with these?)}

%%%%%%%%%%%%%%%%%%%%%%%%%%%%%%%%%%%%%%
\begin{figure}
\centering
\includegraphics[scale=0.55]{plots/K_b2.eps}
\includegraphics[scale=0.55]{plots/K_b4.eps}
\caption{The fit results of $K(b_{\perp},\mu,x,P^z_1,P^z_2)$ extracted {\bl from the} quasi-WFs $\tilde{\Psi}^{\pm}$. {\bl The chosen momentum pairs $\{P_1^z,P^z_2\}$ are denoted by $P^z_1/P_2^z$ in the legend}. The upper figure is for $b_{\perp}=2a$ and the lower one is for $b_{\perp}=4a$. The horizontal shaded band {\bl shows} the central value and uncertainty of $K(b_{\perp},\mu)$, as well as the fit range of $x$, as described in the text. The strong oscillation in the shaded area {\bl at both edges, where LaMET approach is invalid, is caused by power corrections.}}
    \label{fig:K_b}
\end{figure}
%%%%%%%%%%%%%%%%%%%%%%%%%%%%%%%%%%%%%%

Theoretically the physical CS kernel is purely real, however, there still exists a residual imaginary part at 1-loop matching. As discussed above, this imaginary term comes from the perturbative matching kernel. It is easily to prove that $H^{\pm}(zP_2^z,\mu)/H^{\pm}(zP_1^z,\mu)$ contains imaginary part while the lattice calculated $\tilde{\Psi}^{\pm}(x,b_{\perp},\mu,P_1^z)/\tilde{\Psi}^{\pm}(x,b_{\perp},\mu,P_2^z)$ is nearly real, shown as \ref{fig:ratio_phi}. Therefore, we consider this imaginary part as systematic uncertainty from factorization theorem. This uncertainty can be expressed as

\begin{align}
&\sigma_{\mathrm{sys}}=\sqrt{K(b_{\perp},\mu)+\text{Im}^2\left[K^+(b_{\perp},\mu)\right]}-K(b_{\perp},\mu)
\label{eq:systematical_error}
\end{align}
where $\text{Im}\left[K^+(b_{\perp},\mu)\right]$ represents the numerical imaginary part of extracting $K(b_{\perp},\mu)$ only by $\tilde{\Psi}^+$:
\begin{align}
K^+(b_{\perp},\mu)=&\frac{1}{\ln (P^z_1/P^z_2)}\ln\frac{H^+(xP^z_2,\mu)\tilde \Psi^+(x, b_\perp, \mu, P^z_1)}{H^+(xP^z_1,\mu)\tilde \Psi^+(x, b_\perp, \mu, P^z_2)}.
\label{eq:K+}
\end{align}

It should be noticed that the perturbative matching kernel $H^+(zP^z,\mu)$ is the complex conjugate of $H^-(zP^z,\mu)$, that is the imaginary parts in these terms can be cancelled each other when we employ the average of $H^+(zP_2^z,\mu)/H^+(zP_1^z,\mu)$ and $H^-(zP_2^z,\mu)/H^-(zP_1^z,\mu)$. Therefore, as the final result, we adopt $K(b_{\perp},\mu)=\left[K^+(b_{\perp},\mu)+K^-(b_{\perp},\mu)\right]/2$ to reserve the real part, and regard the imaginary contributions as our systematic uncertainty.


%According to Eq. (\ref{eq:K+}), this unexpected imaginary part comes from the logarithm term of $K^+(b_{\perp},\mu)$. Thus, the ratio of perturbative matching kernel $\frac{H^+(zP_2^z,\mu)}{H^+(zP_1^z,\mu)}$ and quasi TMDWFs $\frac{\tilde{\Psi}^+(x,b_{\perp},\mu,P_1^z)}{\tilde{\Psi}^+(x,b_{\perp},\mu,P_2^z)}$ need to be investigated.

%As is simply seen from Eq.(\ref{eq:1loophardkernel}), $\frac{H^+(zP_2^z,\mu)}{H^+(zP_1^z,\mu)}$ is complex. 

%Besides, Fig. \ref{fig:ratio_phi} as one example of numerical results of the imaginary part of the ratio for $\Psi^+$ at different $P^z$ with the set of $b_{\perp}=3$a, shows the information that the the imaginary part of ratio $\text{Im}[\frac{\tilde{\Psi}^+(x,b_{\perp},\mu,P_1^z)}{\tilde{\Psi}^+(x,b_{\perp},\mu,P_2^z)}]$ is very close to zero. Thus, we can make a statement that $\frac{\tilde{\Psi}^+(x,b_{\perp},\mu,P_1^z)}{\tilde{\Psi}^+(x,b_{\perp},\mu,P_2^z)}$ is real.

%Consequently, the imaginary part of $K^+(b_{\perp},\mu)$ is mainly caused by the ratio of perturbative matching kernel. In addition, Eq.(\ref{eq:1loophardkernel}) also shows $H^+(zP^z,\mu)$ is the complex conjugate of $H^-(zP^z,\mu)$, then it leads the average of $\frac{H^+(zP_2^z,\mu)}{H^+(zP_1^z,\mu)}$ and $\frac{H^-(zP_2^z,\mu)}{H^-(zP_1^z,\mu)}$ to a real quantity. As a result, we employ the average $K^+(b_{\perp},\mu)$ and $K^-(b_{\perp},\mu)$ (defined like $K^+(b_{\perp},\mu)$) as our final result.

%%%%%%%%%%%%%%%%%%%%%%%%%%%%%%%%%%%%%%
\begin{figure}
\centering
\includegraphics[scale=0.55]{plots/imag_ratio_b3.eps}
\caption{{\bl Example of numerical results for} the imaginary part of the ratio for quasi TMDWFs at different $P^z$, $\frac{\tilde{\Psi}^+(x,b_{\perp},\mu,P_1^z)}{\tilde{\Psi}^+(x,b_{\perp},\mu,P_2^z)}$, with $b_{\perp}=3$a as a function of momentum fraction $x$. The imaginary parts of both cases are close to zero.}
    \label{fig:ratio_phi}
\end{figure}
%%%%%%%%%%%%%%%%%%%%%%%%%%%%%%%%%%%%%%

%%%%%%%%%%%%%%%%%%%%%%%%%%%%%%%%%%%%%%

\subsection{Discussions}

One should notice that the Wilson loop renormalized quasi TMDWF on {\bl the} lattice (Eq.(\ref{eq:WLRQuasi})) has no dependence on scale $\mu$. A {\bl natural} choice for the scale is $1/a$, but it does not have the same meaning as the scale $\mu$ in {\bl the} $\overline{\rm MS}$ scheme.  If one converts it to {\bl the} $\overline{\rm MS}$ scheme through dividing it by $Z_{O}$ (Eq.(\ref{eq:ZO_RS_higher})), {\bl the} scale $\mu$ is introduced. In principle, one should convert the Wilson loop renormalized quasi TMDWF to {\bl the} $\overline{\rm MS}$ scheme since our factorization formula works {\bl there.} However, since $Z_{O}$ has no dependence on momentum $P_{z}$, it {\bl cancels} in the ratio of quasi TMDWFs, so {\bl does the} scale dependence. {\bl So,} one doesn't need to do the scheme and scale conversion {\bl of} the quasi TMDWF during the extraction of CS kernel.

The extracted CS kernel from the combined fit of the ratios of quasi TMDWFs  with different {\bl momenta} are shown as the red data points in Fig. \ref{fig:K_com}, and the two kinds of errors {\bl ???it is unclear which error bar corresponds to which error???} for each {\bl result} are the statistical and systematical uncertainties. In the small-$b_{\perp}$ region,   systematical  uncertainties are dominant due to the large power {\bl and the} nonzero imaginary part. 

As a comparison, we also give the tree-level matching  result for the CS kernel.  With the leading order matching kernel $H(xP^z,\mu)=1+\mathcal{O}(\alpha_s)$, Eq. (\ref{eq:extractingCSkernel}) {\bl simplifies to} the ratio of quasi TMDWFs $\tilde{\Phi}$ at $z=0$ {\bl with} momentum $P_1^z/P^z_2$. Blue dots in Fig.~\ref{fig:K_com} denote the results obtained {\bl for tree-level matching, for which only} statistical uncertainties are shown. 

%%%%%%%%%%%%%%%%%%%%%%%%%%%%%%%%%%%%%%
\begin{figure}
\centering
\includegraphics[scale=0.55]{plots/K_com.eps}
\includegraphics[scale=0.55]{plots/K_pheno.eps}
\caption{{The upper panel shows the comparison of our results $K(b_{\perp},\mu)$ and $K_0(b_{\perp},\mu)$ with the lattice calculations by SWZ~\cite{Shanahan:2021tst}, LPC~\cite{LatticeParton:2020uhz}, ETMC/PKU~\cite{Li:2021wvl} and SVZES\cite{Schlemmer:2021aij}, as well as the perturbative calculations up to 3-loop. $K(b_{\perp},\mu)$ denotes the CS kernel extracted through 1-loop matching, and whose uncertainties correspond to the statistical errors and the systematic ones from the non-zero imaginary part. $K_0(b_{\perp},\mu)$ denotes our tree-level results, only with statistical uncertainties. The lower panel shows the comparison of our result with phenomenological extractions: SV19~\cite{Scimemi:2019cmh}, Pavia19~\cite{Bacchetta:2019sam} and SIYY15~\cite{Sun:2014dqm} give phenomenological parameterizations of CS kernel {\bl fitted to data from} high energy collision processes like Drell-Yan.}}
\label{fig:K_com}
\end{figure}
%%%%%%%%%%%%%%%%%%%%%%%%%%%%%%%%%%%%%%

Besides, we also show {\bl the CS kernel from} perturbative calculations, phenomenological extractions as well as the lattice results determined by other collaborations. The black solid and dashed lines in the upper panel of Fig.~\ref{fig:K_com} indicate the perturbative results  up to 3-loops, with coupling constant $\alpha_s(\mu=1/b_{\perp})$. Similar {\bl as in} this work, the results of LPC~\cite{LatticeParton:2020uhz} and ETMC/PKU~\cite{Li:2021wvl} are also extracted 
from {\bl quasi} TMDWFs amplitudes through tree-level {\bl matching}. While the SWZ~\cite{Shanahan:2021tst} and SVZES\cite{Schlemmer:2021aij} results are obtained {\bl using } the factorization formula of quasi-TMDPDF.  In the small $b_{\perp}$ region, our results are consistent with the perturbative ones, which {\bl adds confidence in this} first-principle calculation of {\bl the} CS kernel. In {\bl the} large $b_{\perp}$ region, our results are consistent with other lattice calculations within uncertainties. The lower panel shows the comparison with phenomenological results. {\bl SV19}~\cite{Scimemi:2019cmh} and SIYY15~\cite{Sun:2014dqm} {\bl use  a parameterization} with perturbative and nonperturbative parts, however Pavia19~\cite{Bacchetta:2019sam} {\bl obtained} their result with the factorization of TMDPDFs, {\bl obtaining the} CS kernel {\bl from} the rapidity derivative. {\bl ??? In addition they fit parameters from the Drell-Yan data to obtain their phenomenological CS kernel ???}.

%\section{Discussions}


\section{Summary and Outlook}
\label{sec:summary}

In this work, we have calculated the CS kernel in the large momentum effective theory framework. In particular  the one-loop matching kernel has been adopted, and several momenta {\bl were} used to extract the CS kernel. We found that in the small $b_{\perp}$ region, our results are consistent with  the perturbative ones, which {\bl adds confidence in this} first-principle calculation of {\bl the} CS kernel %(It seems strange the perturbative calculation give evidence to first-principle calculation)}.
In large $b_{\perp}$ region, our results are consistent with other lattice calculations within uncertainties. 

Our work shows that the methodology proposed in Ref.~\cite{Ji:2019sxk,Ji:2019ewn} is indeed valid for the study of {\bl the} CS kernel from  {\bl first principles and that LaMET provides important insights into transverse momentum-dependent hadron structure.} 


\section*{Acknowledgement}
We thank Xu Feng,   Yizhuang Liu,  and Feng Yuan for useful discussions.  This work is supported in part by Natural Science Foundation of China under grant No. 11735010, 11911530088, U2032102, 11653003, 11975127, 11975051, 12005130, 12147140. MC, JH, WW is also supported  by Natural Science Foundation of Shanghai under grant No. 15DZ2272100.  PS is also supported by Jiangsu Specially Appointed Professor Program. YBY is also supported by the Strategic Priority Research Program of Chinese Academy of Sciences, Grant No. XDB34030303. AS, PS, WW, YBY and JHZ are also supported by the NSFC-DFG joint grant under grant No. 12061131006 and SCHA~~458/22.









 

\appendix

\section{Euclidean Time $t$ Dependence of Normalized $C_2$}
\label{appsec:C2}


In Sec. \ref{sec:quasi-WF_from_2pt} the ratio of nonlocal and local  two-point {\bl functions is parametrized in} Eq.(\ref{eq:two-state-fit})
\begin{align}
&R^{\pm}\left(z,b_{\perp},P^z,L,t \right)=\frac{C_2^{\pm}\left(z,b_{\perp},P^z,L,t \right)}{C_2\left(0,0,P^z,0,t \right)} \nonumber\\
=&\tilde{\Phi}^{\pm0}\left(z,b_{\perp},P^z,L\right)\left[1+c_0\left(z,b_{\perp},P^z,L\right)e^{-\Delta Et} \right].
\end{align}
From the above equation,  one can see {\bl that} $R^{\pm}\left(z,b_{\perp},P^z,L,t \right)$ decays exponentially with $t$. As discussed in Sec. \ref{sec:quasi-WF_from_2pt}, the one-state and two-state fits are both used to extract $\tilde{\Phi}^{\pm0}\left(z,b_{\perp},P^z,L\right)$. Fig.~\ref{fig:t_dep} shows four examples of fitted results with different $\{z,b_{\perp}\}=\{0a,1a\},\{0a,3a\},\{2a,2a\},\{2a,3a\}$, and same $\{P^z,L\}=\{24\pi/n_s,6a\}$, demonstrating that the one-state and two-state fits are consistent with each other, but the one-state fit is more conservative.


\section{Gauge Line Length $L$ Dependence of Quasi TMDWFs}
\label{sec:gauge_line}

In Sec. \ref{subsec:numerical_wl}, the Wilson loop is used to renormalize quasi TMDWFs, which removes the linear divergence. Similar with the discussion in Sec.~\ref{subsec:numerical_wl}, we give results with the different $\{P^z,b_{\perp},z\}$ in Fig.~\ref{fig:multi_l_dep} to show the Wilson line length $L$-dependence of Wilson loop, unsubtracted quasi TMDWFs and subtracted quasi TMDWFs. At large $L$, $\tilde{\Phi}^{+0}$ decays at the same speed of $\sqrt{Z_E}$, so Wilson loop cancels the linear divergence in unsubtracted quasi TMDWFs.

\section{Operator Mixing Effects}
\label{sec:app_operator_mixing}

As described in Sec. \ref{sec:operator}, quasi TMDWFs for a pseudoscalar meson require  the projectors $\Gamma=\gamma^t\gamma_5$ or $\Gamma=\gamma^z\gamma_5$ onto the leading twist light-cone distribution amplitude, {\it i.e.} $\gamma^+\gamma_5$ in large $P^z$ limit. Fig. \ref{fig:multi_qwf_co} shows examples with different $\{P^z,b_{\perp}\}$ for comparing quasi TMDWFs as functions of $\lambda=zP^z$ of two Dirac matrices $\Gamma=\gamma^t\gamma_5$ and $\Gamma=\gamma^z\gamma_5$. In small $\lambda$ area, the behaviors of quasi TMDWFs for two Dirac matrices are a little different, which is expected to decrease with the increase of $P^z$. So the average of these two cases is likely to eliminate the power corrections. 

\section{$x$-dependence in Collins-Soper kernel}

As Sec.~\ref{subsec:csresults} describes,   there  are power corrections to quasi TMDWFs with the form of $1/(xP^z)^2$ or $1/(\bar{x}P^z)^2$. The $x$-dependence analysis  is needed for extracting CS kernel from quasi TMDWFs. Eq.~\ref{eq:parametrizationofCSkernel} gives parametrization form of CS kernel as a function of $x$, which is used for $x$-dependence fit. Fig.~\ref{fig:x_dep_fit_CS_kernel} gives fit results with $b_{\perp}=\{0.12,0.36,0.6\}$fm of CS kernel of $x$-dependence. When the $b_{\perp}$ get larger, the oscillation becomes more severe  due to the the finite truncation in the Fourier transformation as described in Sec.~\ref{sec:operator}. 

\newpage
\begin{widetext}


%%%%%%%%%%%%%%%%%%%%%%%%%%%%%%%%%%%%%%
\begin{figure}
    \centering
    \includegraphics[scale=0.55]{plots/t_dep_b1z0_r.eps}
    \includegraphics[scale=0.55]{plots/t_dep_b3z0_r.eps}
    \includegraphics[scale=0.55]{plots/t_dep_b2z1_r.eps}
    \includegraphics[scale=0.55]{plots/t_dep_b3z1_r.eps}
    \caption{Four examples for comparing two-state fit and one-state fit to extract the $\tilde{\Phi}^{\pm0}(z,b_{\perp},P^z,L)$ from $R^{\pm}(z,b_{\perp},P^z,L,t)$ as described in Sec. \ref{sec:quasi-WF_from_2pt} with $\{z,b_{\perp}\}=\{0a,1a\},\{0a,3a\},\{2a,2a\},\{2a,3a\}$, and $\{P^z,L\}=\{24\pi/n_s,6a\}$. The fit range for two-state fit is $t\in[2a,8a]$, which for one-state fit is $t\in[5a,8a]$.}
    \label{fig:t_dep}
\end{figure}
%%%%%%%%%%%%%%%%%%%%%%%%%%%%%%%%%%%%%%


%%%%%%%%%%%%%%%%%%%%%%%%%%%%%%%%%%%%%%
\begin{figure}
    \centering
    \includegraphics[scale=0.55]{plots/l_dep_10_21_r.eps}
    \includegraphics[scale=0.55]{plots/l_dep_10_21_i.eps}
    \includegraphics[scale=0.55]{plots/l_dep_12_21_r.eps}
    \includegraphics[scale=0.55]{plots/l_dep_12_21_i.eps}
    \includegraphics[scale=0.55]{plots/l_dep_8_31_r.eps}
    \includegraphics[scale=0.55]{plots/l_dep_8_31_i.eps}
    \caption{Results {\bl showing the dependence  on the gauge line length $L$} of unsubtracted {\bl and} subtracted quasi TMDWFs {\bl as well as the} Wilson loop with $\{P^z,b_{\perp},z\}$ shown in each figure. {\bl These results are for  $\Gamma=\gamma^z\gamma_5$.} }
    \label{fig:multi_l_dep}
\end{figure}
%%%%%%%%%%%%%%%%%%%%%%%%%%%%%%%%%%%%%%

%%%%%%%%%%%%%%%%%%%%%%%%%%%%%%%%%%%%%%
\begin{figure}
    \centering
    \includegraphics[scale=0.55]{plots/qwf_co_b2_r_p12.eps}
    \includegraphics[scale=0.55]{plots/qwf_co_b2_i_p12.eps}
    \includegraphics[scale=0.55]{plots/qwf_co_b4_r_p12.eps}
    \includegraphics[scale=0.55]{plots/qwf_co_b4_i_p12.eps}
    \includegraphics[scale=0.55]{plots/qwf_co_b3_r_p8.eps}
    \includegraphics[scale=0.55]{plots/qwf_co_b3_i_p8.eps}
    \caption{Examples of comparisons for $\lambda$-dependence of quasi-WF matrix elements with two Dirac matrices: $\Gamma=\gamma^t\gamma_5$ and $\Gamma=\gamma^z\gamma_5$, with $\{P^z,b_{\perp}\}$ shown in each figure. Power corrections cause the deviation between {\bl both} cases.}
    \label{fig:multi_qwf_co}
\end{figure}
%%%%%%%%%%%%%%%%%%%%%%%%%%%%%%%%%%%%%%

%%%%%%%%%%%%%%%%%%%%%%%%%%%%%%%%%%%%%%
\begin{figure}
    \centering
    \includegraphics[scale=0.55]{plots/K_b1.eps}
    \includegraphics[scale=0.55]{plots/K_b3.eps}
    \includegraphics[scale=0.55]{plots/K_b5.eps}
    \caption{The fit results of $K(b_{\perp},\mu,x,P^z_1,P^z_2)$ extracted of quasi-WFs $\tilde{\Psi}^{\pm}$ for the cases $b_{\perp}=\{0.12,0.36,0.6\}$fm.}
    \label{fig:x_dep_fit_CS_kernel}
\end{figure}
%%%%%%%%%%%%%%%%%%%%%%%%%%%%%%%%%%%%%%

\end{widetext}

%%%%%%%%%%%%%%%%%%%%%%%%

\begin{thebibliography}{11}


\input{main-references.tex}

\end{thebibliography}
\end{document}


\iffalse %%%%%%%%%%%%%%%%%%% 整段注释开始 %%%%%%%%%%%%%%%%%%%
\clearpage


%%%%%%%%%%%%%%%%%%%%%%%%%%%%%%%%%%%%%%%%%%%%%%%%%%%%%%%%%%%%%%%%%%%%
%%% Section II written by mh, has been commented
%%%%%%%%%%%%%%%%%%%%%%%%%%%%%%%%%%%%%%%%%%%%%%%%%%%%%%%%%%%%%%%%%%%%
\begin{comment}
 \section{Theoretical framework}

 \subsection{Overview historical progresses on calculating CS kernel}
\label{sec:overview_CS_kernel}
CS kernel $K\left(b_{\perp}, \mu\right)$, known as the rapidity anomalous dimension, encodes the rapidity dependence, in other words, the momentum of hadron or equivalently the hard scale of scattering processes, of the TMD distributions. This dependence is governed by the CS equation~\cite{Collins:1981va,Collins:1981uk}:
\begin{align}
	2 \zeta \frac{d}{d \zeta} \ln f^{\mathrm{TMD}}\left(x, b_{\perp}, \mu, \zeta\right)=K\left(b_{\perp}, \mu\right),
\end{align}
\red{(add more discussions and physical pictures of CS kernel, like Fig.4.1 in TMD handbook)}\\
where $f^{\mathrm{TMD}}$ denotes the TMD distributions like TMD parton distributions, TMD wave functions and so on. It depends on the longitudinal momentum fraction $x$, transverse space separation $b_{\perp}$, which is Fourier-conjugate to transverse momentum $q_{\perp}$, as well as the renormalization scale $\mu$ and rapidity scale $\zeta$. The $\mu$-dependence of CS kernel $K\left(b_{\perp}, \mu\right)$ satisfies the following renormalization group equation (RGE):
\begin{align}
	\mu^{2} \frac{d}{d \mu^{2}} K\left(b_{\perp}, \mu\right)=-\Gamma_{\text {cusp}}\left(\alpha_{s}\right), \label{eq:cuspanomalousdimension}
\end{align}
where $\Gamma_{\text {cusp}}\left(\alpha_{s}\right)=\alpha_sC_F/\pi+\mathcal{O}(\alpha_s^2)$ is the cusp anomalous dimension, has been perturbative calculated up to 2-loop at Ref.~\cite{Li:2016ctv}, and 3-loop at Ref.\cite{Moch:2017uml}. From the RGE, the all-order structures of $K\left(b_{\perp}, \mu\right)$ can be expressed as
\begin{align}
	K\left(b_{\perp}, \mu\right)=-2 \int_{1 / b_{\perp}}^{\mu} \frac{\mathrm{d} \mu^{\prime}}{\mu^{\prime}} \Gamma_{\text {cusp}}\left(\alpha_{s}\left(\mu^{\prime}\right)\right)+K\left(\alpha_{s}\left(1 / b_{\perp}\right)\right),
\end{align} 
where $K\left(\alpha_{s}\left(1 / b_{\perp}\right)\right)$ is the non-cusp anomalous dimension, is introduced for the single logarithmic evolution of rapidity logarithms, which  is also known perturbatively to 3-loops in QCD~\cite{Li:2016ctv,Vladimirov:2016dll,Lubbert:2016rku,Gehrmann:2014yya}. With the increasing of $b_{\perp}$, the CS kernel becomes nonperturbative, and independent of the renormalization scale $\mu$ when $b_{\perp}^{-1}\lesssim\Lambda_{\mathrm{QCD}}$. The non-cusp anomalous dimension $K\left(\alpha_{s}\left(1 / b_{\perp}\right)\right)$, which encodes the long-distance informations at large $b_{\perp}$, contains the nonperturbative part of the CS kernel.

Generally the nonperturbative content of CS kernel can be directively determined by global fits~\cite{Scimemi:2019cmh}(\red{add more references}), or extracted from the fit results of TMD distributions~\cite{Bacchetta:2019sam}(\red{add more references}). The parametrization form usually be obtained from extending the perturbative expressions at small-$b_{\perp}$ region, \red{discuss the pros and cons of fit results.}.

The absence of QCD dynamic informations in the global fits  motivated an experiment independent determination of the nonperturbative part of CS kernel from lattice QCD. Recently a newly raised approach, named large-momentum effective theory (LaMET), was proposed to determine the partonic observables from the first principle of QCD~\red{(Ji,2013)}. The key idea of this approach is replace the standard formalism of parton distributions with light-like operators by an equivalent one with large-momentum states and time-like operators, which can be calculated from lattice QCD directly. With LaMET, people have systematically studied the 1-dimensional parton distributions like unpolarized and transverly PDFs~\red{(refs)}, pion/kaon and $\rho/K^*$ light-cone distribution amplitudes~\red{(refs)}, and so on. Inspired by the first attempt at TMD soft function~\red{(QAZ,2020)}, people started to pay attention to simulate the 3-dimensional distributions on the Euclidean lattice~\red{(TMD works)}.

Very recently, the applications of LaMET to extract CS kernel from lattice has attracted more and more attentions~\red{(refs)}. \red{Discuss the pros and cons of  SWZ, ETMC/PKU, LPC, SVZES results. And list the necessities of calculating CS kernel from TMDWF up to 1-loop.}

\subsection{Quasi TMDWFs in Large Momentum Effective Theory}
\label{sec:definition_quasi_TMDWF}
We define the quasi-TMDWF for a highly boosted pion along the $z$-direction with large momentum $P^z$ as
\begin{align}
	\tilde{\Psi}^{\pm}&\left(x,b_T,\mu,\zeta_z\right)=\lim_{a\to0 \atop L\to\infty}\int\frac{dz}{2\pi}e^{iz\cdot xP^z}\tilde{Z}_{\overline{\mathrm{MS}}}\left(z,\mu,\tilde{\mu} \right) \nonumber\\
	&\times \tilde{Z}_{\mathrm{NP}}\left(z,\tilde{\mu},a \right)\tilde{\Phi}^{\pm0}\left(z,b_T,P^z,a,L\right) \tilde{Z}_E^{-1/2}(r,b_T,\tilde{\mu}),
\end{align}
where the bare, or unsubtracted quasi-TMDWF amplitude in coordinate space is defined as an equal-time correlator containing nonlocal quark bilinear operator with staple-shaped gauge link:
\begin{align}
	\tilde{\Phi}^{\pm0}&\left(z,b_T,P^z,a,L\right) =\left\langle 0\right|\bar{\psi}\left(z \hat{n}_{z}/2+b_{T} \hat{n}_{\perp}\right) \Gamma \nonumber\\
	&U_{\sqsupset, \pm L}\left(z \hat{n}_{z}/2+b_{T} \hat{n}_{\perp},-z \hat{n}_{z}/2\right) \psi\left(-z \hat{n}_{z}/2\right)\left| P^{z}\right\rangle, \label{eq:oridefinitionofquasiWF}
\end{align}
the Dirac structure $\Gamma$ can be chosen as $\gamma^z\gamma_5$ or $\gamma^t\gamma_5$, which is related to the leading twist structure $\gamma^-\gamma_5$ in the light-cone correlator. With large but finite $P^z$, the deviation between $\gamma^z\gamma_5/\gamma^t\gamma_5$  and $\gamma^-\gamma_5$ is power suppressed by $M^2/\left(P^z\right)^2$. Technically, we can also use a combination of them, such as $\left(\gamma^z\gamma_5+\gamma^t\gamma_5\right)/2$ to cancel out part of power correction contributions. The superscript 0 in $\tilde{\Phi}^{\pm0}$ indicates that the UV and rapidity divergences have not been regularized. The rapidity divergence, or called end-point singularity in some references \red{(references)}, comes from the radiation of gluons collinear to the gauge-link,  and performs as the singularities in loop momentum integrals at end-point region. Therefore, this divergence appears in the staple-shaped gauge-link, like
\begin{align}
	&U_{\sqsupset, \pm L}\left(z \hat{n}_{z}/2+b_{T} \hat{n}_{\perp},-z \hat{n}_{z}/2\right) \equiv\nonumber\\
	&	U_z^{\dagger}\left(z \hat{n}_{z}/2+b_{T} \hat{n}_{\perp}; L \right) U_{\perp}\big((L-z /2)\hat{n}_{z};{b}_T \big) U_z\left( -z \hat{n}_{z}/2; L \right),
\end{align}
and cannot be regularized by the standard UV regulator. The Euclidean gauge link in $U_{\sqsupset, \pm L}$ is defined as
\begin{align}
	U_z(\xi,\pm L)=\mathcal{P}	\exp \left[-i g \int_{\xi^{z}}^{\pm L} d \lambda n_{z} \cdot A\left(\vec{\xi}_{\perp}+n_{z} \lambda\right)\right],
\end{align}
here $\xi^{z}=-\xi \cdot n_{z}$,  and $\pm L$ pointing out the positive or negative infinity the gauge link can reach on a finite Euclidean lattice, such like the blue and red cases illustrated in Fig.(\ref{fig:Wilson_line}).

%%%%%%%%%%%%%%%%%%%%%%%%%%%%%%%%%%
\begin{figure}
\centering
%\includegraphics[scale=0.45]{plots/quasi-TMDWF.eps}
\includegraphics[scale=0.13]{link_dir_2.eps}
\caption{Diagrammatic representation of the Wilson line included in quasi TMDWFs and the corresponding Wilson loop. In the upper panel, $L$ is taken as positive corresponding to the $\psi^+$, while this Wilson link can also go to the negative $n^z$ direction, corresponding to the $\psi^-$. The lower panel shows the length and width of Wilson loop matrix element.}
\label{fig:Wilson_line}
%\label{fig:definitionofquasiTMDWF}
\end{figure}
%%%%%%%%%%%%%%%%%%%%%%%%%%%%%%%%%%

The rapidity divergence will canceled between the unsubtracted quasi-TMDWF amplitude and half of quasi-TMD soft function $\tilde{Z}_E^{-1/2}(r,b_T,\tilde{\mu})$, and leading to the explicit dependence of amplitude on a set of rapidity scales $\zeta_z=(2xP^z)^2$. The quasi-TMD soft function is expressed as the vacuum expectation of a flat rectangular Euclidean Wilson-loop on $z$-$\perp$ plane:
\begin{align}
\tilde{Z}_{E}\left(r, b_T, \tilde{\mu}\right)=\frac{1}{N_{c}} \operatorname{Tr}\left\langle 0\left|U_{\perp}(0;b_T) U_{z}\left(b_T\hat{n}_{\perp}; r\right)\right| 0\right\rangle.
\end{align}
Here $r=2L$ is the twice length of $U_z^{(\dagger)}$ in the staple-shaped gauge-link, so the square root of $\tilde{Z}_{E}\left(r, b_T, \tilde{\mu}\right)$ will cancel the divergences in gauge-link and guarantee the existence of the $L\to\infty$ limit, as well as remove additional contributions from the transverse gauge link from the finite ``infinity" on the Euclidean lattice. It should be mentioned that after subtraction, the logarithmic UV divergences are still present and formulated as the $\tilde{\mu}$ dependence.

\subsection{Factorization of quasi TMDWFs and extracting CS kernel from 1-loop matching}
As we discussed in Sec. \ref{sec:overview_CS_kernel}, CS kernel describe the rapidity evolution of TMD distributions. But for quasi TMDWFs, the evolution equation includes a perturbative terms and higher powers:\cite{Ji:2021znw}

\begin{eqnarray}
&&2\zeta^z\frac{d}{d\zeta^z}\ln\tilde{\psi}^{\pm}(x,b_{\perp},\mu,\zeta^z)\nonumber\\
&&=K(b_{\perp},\mu)+\mathcal{G}^{\pm}(\zeta^z,\mu)+O((\frac{1}{P^z})^2),
\label{eq:quasi_ev_CS}
\end{eqnarray}
where $\zeta_z=(2xP^z)^2$ and $\mathcal{G}^{\pm}(x^2\zeta^z,\mu)$ are perturbative kernel. One can clearly see that, from this rapidity evolution of quasi TMDWFs, as $P^z\to\infty$, $\mathcal{G}^{\pm}(x^2\zeta^z,\mu)$ includes large logarithms, part of it being non-perturbative, part of it being perturbative. Thus, to match the quasi TMDWFs to light-cone ones, we need a hard kernel to represent the perturbative part, and an exponential form of $K(b_{\perp},\mu)$ is needed for non-perturbative rapidity logarithms. 

%%%%%%%%%%%%%%%%%%%%%%%%%%%%%%%%%%
\begin{figure}
\centering
\includegraphics[scale=0.6]{reduced_graph.eps}
\caption{Leading-order reduced graph for quasi TMDWFs of a pseudo-scalar meson. $C$ denotes a collinear sector, and $S$ corresponds to the soft sector, while the $H$ is the hard-core contribution. Since the hard-gluon exchange between the quark and anti-quark is power suppressed, the hard-cores are disconnected with each others, and therefore the factorization of quasi TMDWFs is multiplicative, and the momentum fractions of the TMDWFs and quasi TMDWFs are the same.}
\label{fig:reduced_graph}
\end{figure}
%%%%%%%%%%%%%%%%%%%%%%%%%%%%%%%%%%

The matching relates the quasi TMDWFs, which are defined to match them to light-cone TMDWFs through a multiplicative factorization theorem including a non-perturbative soft factor.
\begin{eqnarray}
&&\tilde \psi^{\pm}(x, b_\perp, \mu, \zeta_z) S_{r}^{1/2}(b_\perp, \mu)\\
&&= H_1^{\pm}(\zeta_z/\mu^2, \overline\zeta_z/\mu^2) e^{\frac{1}{2} \ln \frac{\mp \zeta_z-i\epsilon}{\zeta}K(b_\perp,\mu)} \psi^{\pm}(x, b_\perp, \mu, \zeta). \nonumber
\label{eq:matching}
\end{eqnarray}
The power-corrections terms are omitted of the order $\mathcal{O}(\Lambda_{QCD}^2/\zeta_z,M^2\zeta_z,1/(b_{\perp}^2\zeta_z))$. The one-loop perturbative hard function is given as:
\begin{eqnarray}
&& H_1^{\pm}(\zeta_z)\\
&& = 1+ \alpha_s  \frac{C_F}{4\pi} \bigg(-\frac{5\pi^2}{6}-4 +{\ell}_{\pm} + \overline \ell_{\pm} - \frac{1}{2} ({\ell}_{\pm}^2 + \overline \ell_{\pm}^2) \bigg),\nonumber
\end{eqnarray}
with ${\ell}_{\pm} = \ln (-\zeta_z \pm i\epsilon)-\ln \mu^2$, and $\overline {\ell}_{\pm} = \ln (-\overline \zeta_z \pm i\epsilon)-\ln \mu^2$. $\zeta_z = (2x P\cdot n_z)^2$ and $\overline\zeta_z = (2\overline x P\cdot n_z)^2$.

The fact that the factorization of the quasi TMDWFs is multiplicative can be understood as follows. In the quasi TMDWFs, the collinear contributions are absorbed into light-cone TMDWFs, while the soft contribution is partially absorbed into the soft-function. The hard contributions between the quark and anti-quark are power suppressed: if there is a hard momentum flowing from quark sector to the anti-quark sector, it contains a factor $\int d^4l\cdot e^{i\vec l\cdot \vec b}$. When $b\gg 1/(xP^z)$, the transverse momentum of this hard mode is much smaller than the hard scale, resulting a power suppression compared to the typical hard scale. Thus at leading power the factorization of quasi TMDWFs is multiplicative. This is the sharp contrast for the transverse momentum of parton $k_{\perp}$ in quasi TMDWFs, where it allows the hard momenta to flow between the vertices when $k_{\perp}$ is comparable to $P^z$. The review paper \cite{Ji:2019ewn} gives more detailed improvements for the factorization form of quasi TMDWFs. The leading order reduced graph is shown in Fig.~\ref{fig:reduced_graph}. The collinear, soft and hard sub-diagrams are respectively responsible for collinear, soft and ultraviolet (or hard) contributions. The off-light-cone soft function is able to represent the effect of soft radiations between the fast-moving color-changes and the staple-shaped Wilson lines. Further, $\zeta_z$ arising from the Lorentz-invariant combinations of collinear modes momentum, along $n^z$ in the operator, provides the natural hard scale of hard sub-diagram.

Reconsidering the physics in the factorization of the quasi TMDWFs, it indicates the soft function required in defining the light-cone TMDWFs $\psi^{\pm}$ is different from which in quasi TMDWFs $\tilde{\psi}^{\pm}$, although the rapidity-dependent term is the same. The reduced soft function $S_{r}$ is exactly the difference between the rapidity-independent part for quasi TMDWFs and light-cone ones. In the momentum evolution equation of quasi TMDWFs shown in Eq.(\ref{eq:quasi_ev_CS}), there are logarithms of form $K(b_{\perp},\mu)\ln(\frac{\mp\zeta_z-i\epsilon}{\mu^2})$. It indicates a factor $\exp[\frac{1}{2}\ln(\frac{\mp\zeta_z-i\epsilon}{\zeta})K(b_{\perp},\mu)]$ to compensate the difference of matching quasi TMDWFs to light-cone TMDWFs at arbitrary rapidity $\zeta$. This leads the matching formula involving a exponential term of CS kernel. In addition, the perturbative part of rapidity evolution $\mathcal{G}^{\pm}$ make the hard contributions represented by the hard kernel $H^{\pm}$ as a function of hard scale $\zeta_z$ and the renormalization scale $\mu$. Comparing the Eqs. (\ref{eq:quasi_ev_CS}) and (\ref{eq:matching}), the relation of perturbative matching kernel $H^{\pm}$ and perturbative evolution factor $\mathcal{G}^{\pm}$ is:
\begin{eqnarray}
2\frac{d\ln H^{\pm}(\zeta_z,\mu)}{d\zeta_z}=\mathcal{G}^{\pm}(\zeta_z,\mu).
\end{eqnarray}

Notice that for the matching formula Eq.(\ref{eq:matching}), by taking the ratio of two quasi TMDWFs in different momentum, the reduced soft function $S_{r}^{1/2}(b_{\perp},\mu)$ cancels. Moreover, the rapidity of light-cone TMDWFs $\zeta$ does not depend on rapidity of quasi TMDWFs, then it will also cancel. This allows that quark CS kernel $K(b_{\perp},\mu)$ can be computed in lattice QCD approach through a ratio of quasi TMDWFs $\tilde{\psi}(x,b_{\perp},\mu,\zeta_z)$ at different hadron momenta(taken in z-direction) $P^z\gg \Lambda_{\text{QCD}}$\cite{Ji:2021znw}: 
\begin{eqnarray}
K(b_\perp, \mu) = \frac{2}{\ln (\zeta_z/\zeta_z')} \ln\frac{H_1^{\pm}(\zeta_z') \tilde \psi^{\pm}(x_i, b_\perp, \mu, \zeta_z)}{H_1^{\pm}(\zeta_z)\tilde \psi^{\pm}(x_i, b_\perp, \mu, \zeta_z')}.
\label{eq:definition_CS-kernel}
\end{eqnarray}
\end{comment}


%%%%%%%%%%%%%%%%%%%%%%%%%%%%%%%%%%%%%%%%%%%%%%%%%%%%%%%%%%%%%%%%%%%%
%% Section III: Numerical results and discussions
%%%%%%%%%%%%%%%%%%%%%%%%%%%%%%%%%%%%%%%%%%%%%%%%%%%%%%%%%%%%%%%%%%%%

\section{Numerical results}
\label{sec:numerics}

\subsection{Lattice Setup}
For the present study, we use the configurations based on clover fermion action with 2+1+1 flavors of highly improved staggered quarks (HISQ) action~\cite{Follana:2006rc}, generated by MILC collaboration~\cite{MILC:2012znn}. A sngle ensemble is used, with lattice spacing $a=0.12$fm and volume as $L^3\times T=48^3\times64$. Sea quark masses on this ensemble have been tuned to match the physical quark masses ~\cite{MILC:2012znn}, while the valance light quark masses are chosen as strange quark masses to improve the signal-to-noise ratio. 
After such settings, the corresponding pion masses on this ensemble are $m_{\pi}^{\mathrm{sea}}=130$MeV and  $m_{\pi}^{\mathrm{val}}=670$MeV respectively.  
To improve the signal-to-noise ratio of numerical simulations, we use the hypercubic (HYP) smeared fat link ~\cite{Hasenfratz:2001hp}.
To achieve the infinite momentum limit with a proper signal, we employ three hadron momenta as $P^z=2\pi/L\times\{8,10,12\}=\{1.72,~2.15,~2.58\}$GeV, which related boost factor $\gamma=\{2.57,~3.21,~3.85\}$.  \\

\subsection{Quasi TMDWFs and Two Point Correlation Functions}
%The quasi TMDWF  $\tilde{\psi}^{\pm}(z,b_{\perp},\zeta^z)$ of $\pi$ meson is defined as the matrix element of a nonlocal quark bilinear operator with a staple-shaped Wilson line in a boosted hadron state:
%\begin{eqnarray}
%&&\tilde{\psi}^{\pm}(z,b_{\perp},\zeta^z)\nonumber\\
%&&= \langle 0|\bar{q}(zn^z/2+b_{\perp})\Gamma W_{\pm n^z}q(-zn^z/2)|\pi \rangle,
%\label{eq:psi_def}
%\end{eqnarray}
%where $|\pi\rangle$ denotes the state of $\pi$ meson with four-momentum $P^{\mu}=(E_{\vec{P}},0,0,P^z)$. The operator Wilson line make the gauge-invariance for operator $\bar{q}\Gamma W q$. It has three components, two along longitudinal direction $n^z$ (extended to length $L$), one along transverse direction with length $b_{\perp}$ in the infinity area.
%\begin{eqnarray}
%&&W_{\pm n^z}=W_{\pm n^z}^{\dagger}(zn^z/2+b_{\perp})W_{\perp}W_{\pm n^z}(zn^z/2),\nonumber\\
%&&W_{\pm n^z}(\xi)=\mathcal{P}\exp\bigg[-ig\int_{\xi}^{\pm \infty}ds\; n^z\cdot A(\xi+n^zs)\bigg],\nonumber\\
%&&W_{\perp}=\mathcal{P}\exp\bigg[-ig\int_0^{b_{\perp}}ds\; n^{\perp}\cdot A(n^{\perp}s)\bigg],
%\end{eqnarray}
%where $A^{\mu}$ relates the gauge potential in QCD, and $g$ is the constant coupling. These Wilson lines extends to $\pm L$ corresponding to $\pm$ choices.

%With two spinor operators, the Wilson lines combine to a staple one in Fig. \ref{fig:Wilson_line}. The renormalization factor of matrix element needs to satisfy the gauge-invariance, then the Wilson loop factor $Z_E(2L,b_{\perp})$ with length $L$ and width $b_{\perp}$ is constructed for the quasi TMDWFs.
%\begin{eqnarray}
%Z_E(2L,b_{\perp},\mu)&=&\langle0|\mathcal{P}\exp\int_{-L}^{L}dzA_z(0,z)\nonumber\\
%&&+\mathcal{P}\exp\int_{0}^{b}dxA_x(L,x)\nonumber\\
%&&+\mathcal{P}\exp\int_{L}^{-L}dzA_z(b,z)\nonumber\\
%&&+\mathcal{P}\exp\int_{b}^{0}dxA_x(x,-L)|0\rangle.
%\end{eqnarray}
%Staple shaped Wilson loop can naively be divided into four sides of a rectangle, shown in Fig. \ref{fig:Wilson_line}. The purpose is to cancel the self-energy of the Wilson line between field operators $\bar{q}$ and $q$. A set of space-like Wilson-line along longitudinal direction $z$ shapes as $Z_E(2L,b_{\perp})$, which is the vacuum expansion, separating in the transverse plane. 


%%%%%%%%%%%%%%%%%%%%%%%%%%%%%%%%%%
%\begin{figure}
%\centering
%\includegraphics[scale=0.13]{link_dir_2.eps}
%\caption{Diagrammatic representation of the Wilson line included in quasi TMDWFs and the corresponding Wilson loop. In the upper panel, $L$ is taken as positive corresponding to the $\psi^+$, while this Wilson link can also go to the negative $n^z$ direction, corresponding to the $\psi^-$. The lower panel shows the length and width of Wilson loop matrix element.}
%\label{fig:Wilson_line}
%\end{figure}
%%%%%%%%%%%%%%%%%%%%%%%%%%%%%%%%%%


According to the definition of quasi TMDWFs in Eq. \ref{eq:oridefinitionofquasiWF}, it is useful to define the subtracted quasi-TMD wave functions $\tilde{\Psi}^{\pm}$ in a $\pi$ meson external state, which can be obtained from ratios of nonlocal and local correlation functions:
\begin{eqnarray}
&&C_2^{\pm}(L,b,z,t,P^z)\\
&&=\int d^3xd^3ye^{-i\vec{P}\vec{x}}e^{i\vec{P}\vec{y}}\langle0|\bar{\psi}(\vec{x}+\frac{z}{2}n^z+b_{\perp},t)\Gamma U_{\sqsupset, \pm L}\nonumber\\
&&\psi(\vec{x}-\frac{z}{2}n^z,t)\bar{\psi}(\vec{y},0)\Gamma^{\prime}\psi(\vec{y},0)|0\rangle\nonumber\\
&&=\frac{L^3}{2E_{P^z}}e^{-iE_{P^z}t}\langle 0|\overline \psi(\frac{z}{2}n^z+b_\perp)\Gamma U_{\sqsupset, \pm L}\psi(-\frac{z}{2}n^z)|\pi(P^z) \rangle,\nonumber\\
&&C_2(0,0,0,t,P^z)\nonumber\\
&&=\frac{L^3}{2E_{P^z}}e^{-iE_{P^z}t}\langle 0|\overline \psi(0)\Gamma \psi(0)|\pi(P^z)\rangle,
\label{eq:nonlocal2pt}
\end{eqnarray}
where the $\pi$ meson effective energy $E_{P^z}=\sqrt{m_{\pi}^2+(P^z)^2}$. In the nonlocal two point function, the staple shaped Wilson line $U_{\sqsupset, \pm L}$ is described in Sec. \ref{sec:definition_quasi-TMDWFs}. Therefore the renormalized $\tilde{\psi}_N$ can be obtained as:
\begin{eqnarray}
\tilde{\Phi}^{\pm}(z,b_{\perp},\zeta^z)_{rn}=\lim_{L\to\infty}\frac{C_2^{\pm}(L,b,z,t,P^z)}{C_2(0,0,0,t,P^z)\sqrt{Z_E(2L,b)}}.\nonumber\\
\label{eq:2pt_wf}
\end{eqnarray}

According to LaMET, calculation of quasi TMDWFs start with the nonlocal and local two-point correlation functions. In order to get a good enough signal-to-noise ratio for the lattice QCD calcualtion, we generate the wall source propagators:
\begin{eqnarray}
S_{w}\left(x, t, t_0 ; \vec{p}\right)=\sum_{\vec{y}} S\left(t, \vec{x} ; t_0, \vec{y}\right) e^{i \vec{p} \cdot(\vec{y}-\vec{x})},
\end{eqnarray}
on the Coulomb gauge fixed configurations, where $(t_0, \vec{y})$ and $(t, \vec{x})$ denote the source and sink positions of  the quark propagator $S_w$ on a Euclidean lattice. Then we can construct the two-point correlation functions:
\begin{eqnarray}
&&C_2^{\pm}(L, z, b_{\perp},P^z; p^z, t)=\frac{1}{L^3}\sum_{\vec{x}}\mathrm{Tr}e^{i\vec{P}\cdot\vec{x}}\nonumber\\
&&\Big\langle S_w^{\dagger}\left(\vec{x}-\frac{z}{2}\hat{n}_z,t;\vec{y},0;-\vec{p}\right)U_{\sqsupset, \pm L}\gamma_5\Gamma\nonumber\\
&&\times S_w\left(\vec{x}+\frac{z}{2}\hat{n}_z+b_{\perp}\hat{n}_{\perp},t;\vec{y},0;-\vec{p} \right) \Big\rangle,
\end{eqnarray}
the quark momentum $\vec{p}=(\vec{0}_{\perp},p^z)$ and hadron momentum $\vec{P}=2\vec{p}$ due to the momentum conservation. $\Gamma=\gamma^{\mu}\gamma_5$ is the leading twist Dirac structure of the nonlocal  quark bilinear operators with pion external state, and because of universality, $\Gamma$ can be $\gamma^z\gamma_5$ or $\gamma^t\gamma_5$, or or any linear combination of them except $\gamma^t\gamma_5-\gamma^z\gamma_5$. The comparison of these two structures is collected in Sec. \ref{sec:oper}, corresponding to the field operators. 

%%%%%%%%%%%%%%%%%%%%%%%%%
\begin{figure}
    \centering
    \includegraphics[scale=0.55]{l_dep_r.pdf}
    \includegraphics[scale=0.55]{l_dep_i.pdf}
    \caption{Results of $L$-dependence of the subtracted and unsubtracted quasi TMDWFs, where we present the results on $\Gamma=\gamma^z\gamma_5$ as an example, and also the square root of the Wilson loop which is used for the subtraction. We take the $\{P^z,b_{\perp}, z\}=\{2.58\text{GeV},2a,2a\}$ as an example, and some other cases with different $P^z$, $b$ and $z$ can be found in the appendix.}
    \label{fig:l_dep}
\end{figure}
%%%%%%%%%%%%%%%%%%%%%%%%%

According to Eq.(\ref{eq:2pt_wf}), it is useful to define subtracted and unsubtracted quasi TMDWFs:
\begin{eqnarray}
&&\tilde{\Phi}^{\pm}(z,b_{\perp},P^z)=\lim_{L\to\infty}\frac{\tilde{\Psi}^{\pm0}(L,z,b_{\perp},P^z)}{\sqrt{Z_E(2L,b_{\perp},\mu)}}, \\
&&\tilde{\Phi}^{\pm0}(L,z,b_{\perp},P^z)=\frac{C_2^{\pm}(L,z,b_{\perp},t,P^z)}{C_2(0,0,t,P^z)},
\end{eqnarray}
where $\sqrt{Z_E(2L,b_{\perp},\mu)}$ is the square root of the vacuum expectation value of a flat rectangular Euclidean
Wilson loop along the $\hat{n}_z$ direction with length $2L$ and width $b_{\perp}$. The self-interactions of gauge links,  expressed as the pinch-pole singularity, in the unsubtracted quasi-TMDWF $\tilde{\Phi}^{\pm0}$ can be subtracted by $\sqrt{Z_E}$~\cite{Ji:2019sxk}.


As the exponential decay of two point correlation functions, we do a fit for normalized one as the function of $t$ to eliminate the excited state effect. Further details are described in appendix. Then just as shown in Fig.\ref{fig:l_dep}, both the real part and imaginary part of unsubtracted quasi-TMDWF $\tilde{\Psi}^{\pm0}$ and the square root of the Wilson loop $Z_E$ decay exponentially with length $L$ at the same speed. However the subtracted one is length independent for large-$L$. In the following discussion, we choose a fixed $L=0.72$fm as asymptotic results for all cases.

\subsection{Operator Mixing Effect}
\label{sec:oper}
The coordinate quasi TMDWFs have quark bilinear operators forming as $\bar{\psi}(x)\Gamma W(x,y)\psi(y)$. Shown in Eq.(\ref{eq:nonlocal2pt}), the operators with staple-shaped Wilson lines are used to study the quasi TMDWFs on lattice. This Lorentz structure based on the matrix $\Gamma$ has $16$ linear independence forms, thus the operator mixing effect may occurs in our quasi TMDWFs. To investigate which of them are able to be put into pion matrix element make useful help for our lattice calculation. 
\subsubsection{Discrete Symmetry}
In quantum field theory, the discrete symmetry plays an important role in the transformation properties. It usually contains parity transformation ($\mathcal{P}$), time reversal ($\mathcal{T}$), charge conjugation ($\mathcal{C}$). In the following discussion, we display the result of transformations for the components, spinor field $\psi$ and $\bar{\psi}$, vector field $A_{\mu}$, and give the summary of quark bilinear form after three discrete transformations.

For parity transformation, it changes the space components into the minus side, and remain time component, noted as $\mathcal{P}$.
\begin{eqnarray}
\mathcal{P}\psi(x^0,\vec{x})\mathcal{P}^{-1}&=&\gamma^0\psi(x^0,-\vec{x}),\\
\mathcal{P}\bar{\psi}(x^0,\vec{x})\mathcal{P}^{-1}&=&\bar{\psi}(x^0,-\vec{x})\gamma^0,\\
\mathcal{P}A^{\mu}(t,\vec{x})\mathcal{P}^{-1}&=&A_{\mu}(x^0,-\vec{x}),\\
\mathcal{P}W(x^0,x^i;y^0,y^i)\mathcal{P}^{-1}&=&W(x^0,-x^i;y^0,-y^i).
\end{eqnarray}
Time reversal does not change the space components, and give a minus sign for time component, noted as $\mathcal{T}$.
\begin{eqnarray}
\mathcal{T}\psi(x^0,\vec{x})\mathcal{T}^{-1}&=&\gamma^1\gamma^3\psi(-x^0,\vec{x}),\\
\mathcal{T}\bar{\psi}(x^0,\vec{x})\mathcal{T}^{-1}&=&\bar{\psi}(-x^0,\vec{x})(-\gamma^1\gamma^3),\\
\mathcal{T}A^{\mu}(t,\vec{x})\mathcal{T}^{-1}&=&A_{\mu}(-x^0,\vec{x}),\\
\mathcal{T}W(x^0,x^i;y^0,y^i)\mathcal{T}^{-1}&=&W(-x^0,x^i;-y^0,y^i).
\end{eqnarray}
Charge conjugation is firstly introduced by Kramer in 1937. It means to change positive particle into negative one, and turn negative particle into positive one, noted as $\mathcal{C}$.
\begin{eqnarray}
\mathcal{C}\psi(x^0,\vec{x})\mathcal{C}^{-1}&=&-i(\bar{\psi}(x^0,\vec{x})\gamma^0\gamma^2)^T,\\
\mathcal{C}\bar{\psi}(x^0,\vec{x})\mathcal{C}^{-1}&=&(-i\gamma^0\gamma^3\psi(x^0,\vec{x}))^T,\\
\mathcal{C}A^{\mu}(t,\vec{x})\mathcal{C}^{-1}&=&-A_{\mu}(x^0,\vec{x}),\\
\mathcal{C}W(x^0,x^i;y^0,y^i)\mathcal{C}^{-1}&=&W^*(x^0,x^i;y^0,y^i).
\end{eqnarray}
The linear independent $16$ Dirac matrix is written as:
\begin{eqnarray}
\Gamma\in\{I,i\gamma_5,\gamma_{\mu},\gamma^{\mu}\gamma_5,\sigma_{\mu\nu},\partial_{\mu}\}.
\end{eqnarray}
In the following, Table. \ref{CPT_Dirac_matrix} gives the summary of three discrete symmetries of these $16$ Dirac matrices.
\begin{table}[]
\begin{tabular}{|c|c|c|c|c|c|c|}
\hline
 & $\bar{\psi}\psi$ & $i\bar{\psi}\gamma_5\psi$ & $\bar{\psi}\gamma^{\mu}\psi$ & $\bar{\psi}\gamma^{\mu}\gamma_5\psi$ & $\bar{\psi}\sigma^{\mu\nu}\psi$ & $\partial^{\mu}$ \\ \hline
$\mathcal{P}$ & $+1$ & $-1$ & $(-1)^{\mu}$ & $-(-1)^{\mu}$ & $(-1)^{\mu}(-1)^{\nu}$ & $(-1)^{\mu}$ \\ \hline
$\mathcal{T}$ & $+1$ & $-1$ & $(-1)^{\mu}$ & $(-1)^{\mu}$ & $-(-1)^{\mu}(-1)^{\nu}$ & $-(-1)^{\mu}$ \\ \hline
$\mathcal{C}$ & $+1$ & $+1$ & $-1$ & $+1$ & $-1$ & $+1$ \\ \hline
\end{tabular}
\caption{The $\mathcal{C}$, $\mathcal{P}$, $\mathcal{T}$ symmetries for the $16$ linear independent Dirac matrices. In which, $(-1)^{\mu}$ equals 1 when $\mu=0$, and equals $-1$ for $\mu=1,2,3$. One example is taken as $i\bar{\psi}\gamma_5\psi$. For parity transformation, the index in the table for $i\bar{\psi}\gamma_5\psi$ is $-1$, which means $\mathcal{P}i\bar{\psi}(t,x)\gamma_5\psi(t,x)\mathcal{P}^{-1}=-i\bar{\psi}(t,-x)\gamma_5\psi(t,-x)$.}
\label{CPT_Dirac_matrix}
\end{table}

In our calculation, the used hadron state is pion state. As a pseudoscalar particle, if the charge is indistinguishable, pion has $\mathcal{P}$ parity $-1$, and $\mathcal{C}$ parity $+1$, then $\mathcal{T}$ parity $-1$~\cite{Plano:1959zz}.
To use this $\mathcal{C}$, $\mathcal{P}$, $\mathcal{T}$ features of pion state $|\pi(P)\rangle$, many matrix elements of Dirac Gamma matrix for quasi TMDWFs are determined as zeros.
For charge conjugation, the pion state has $\mathcal{C}$ parity +1, thus only the bilinear operators of $+1$ for $\mathcal{C}$ parity bring the matrix element non-zero. An example is taken for $\bar{\psi}\gamma^{\mu}\psi$:
\begin{eqnarray}
&&\langle0|\bar{\psi}(x)\gamma^{\mu}W(x,y)\psi(y)|\pi\rangle\nonumber\\
&&=\langle0|\mathcal{C}^{-1}\mathcal{C}\bar{\psi}(x)\gamma^{\mu}W(x,y)\psi(y)\mathcal{C}^{-1}\mathcal{C}|\pi\rangle\nonumber\\
&&=\langle0|\bar{\psi}(y)\gamma^{\mu}W^*(x,y)\psi(x)(-1)(+1)|\pi\rangle\nonumber\\
&&=-\langle0|\bar{\psi}(x)\gamma^{\mu}W(x,y)\psi(y)|\pi\rangle=0.
\end{eqnarray}
Similar for $\mathcal{P}$ and $\mathcal{T}$ transformations, for pion state matrix element, there are only $i\bar{\psi}\gamma_5\psi$ and $\bar{\psi}\gamma^{\mu}\gamma_5\psi$ which is non-zero.
\subsubsection{Twist Expansion}
To get more further, considering the twist expansion, the Light-cone matrix element based on the hadron external state along $+$ direction indicates the leading twist Lorentz structures. For spinor, the equation of motion indicates:
\begin{eqnarray}
&&n\!\!\!\slash_+ u(p)=0\to n\!\!\!\slash_+ \psi(0)=0\nonumber\\
&&\bar{u}(p)n\!\!\!\slash_+=0\to \bar{\psi}(\xi)n\!\!\!\slash_+=0,
\end{eqnarray}
for light-cone unit vector, they have the relation
\begin{eqnarray}
\frac{1}{2}(n\!\!\!\slash_+n\!\!\!\slash_-+n\!\!\!\slash_-n\!\!\!\slash_+)=1,
\end{eqnarray}
then the leading order of $\gamma_5$ and $\gamma^{\mu}\gamma_5$ are
\begin{eqnarray}
&&\bar{\psi}(\xi)\gamma_5\psi(0)=\frac{1}{2}\bar{\psi}(\xi)(n\!\!\!\slash_+n\!\!\!\slash_-+n\!\!\!\slash_-n\!\!\!\slash_+)\gamma_5\psi(0)=0,\\
&&\bar{\psi}(\xi)\gamma^{\mu}\gamma_5\psi(0)=\frac{1}{2}\bar{\psi}(\xi)(n\!\!\!\slash_+n_-^{\mu}+n\!\!\!\slash_-n_+^{\mu}+\gamma_{\perp})\gamma_5\psi(0)\ne0.\nonumber
\end{eqnarray}
So only $\gamma^{\mu}\gamma_5=\gamma^+\gamma_5$ is the leading twist for light-cone pion matrix element. Therefore, for quasi TMDWFs, the Lorentz structure $\Gamma=\gamma^{\mu}\gamma_5$ can reduced to this leading twist of Light-cone ones.

In order to increase the signal of the calculation, we apply the leading twist operators $\bar{\psi}(x)\gamma^{\mu}\gamma_5\psi(y)$ and the corresponding matrix element is proportional to the momentum of external state:
\begin{eqnarray}
\langle0|\bar{\psi}(x)\gamma^{\mu}\gamma_5\psi(y)|\pi(P)\rangle\propto\ if_{\pi}P^{\mu}.
\end{eqnarray}
Because the pion momentum is set along the $z$-direction, there are two different leading twist Dirac structures $\gamma^t\gamma_5$ and $\gamma^z\gamma_5$ which have identical contributions theoretically.

\subsubsection{Quasi TMDWFs results}
%%%%%%%%%%%%%%%%%%%%%%%%%%%%%%%%%%%%%%
\begin{figure}
    \centering
    \includegraphics[scale=0.55]{qwf_co_r_p12.pdf}
    \includegraphics[scale=0.55]{qwf_co_i_p12.pdf}
    \caption{Numerical results for $\tilde{\Phi}(z,b_{\perp},P^z)$ with $\{P^z,b_{\perp}\}=\{2.58\text{GeV},3a\}$. Further cases for different momentum and $b_{\perp}$ are summarized in appendix.}
    \label{fig:qwf_coordinate}
\end{figure}
%%%%%%%%%%%%%%%%%%%%%%%%%%%%%%%%%%%%%%

In Fig.\ref{fig:qwf_coordinate}, we compare the quasi-TMDWF $\tilde{\Phi}^+(z,b_{\perp},P^z,\Gamma)$ as the function of $zP^z$ with $\Gamma=\gamma^t\gamma_5$ and $\gamma^z\gamma_5$ to investigate the difference of these two cases on lattice calculation. From the comparison we can see that the difference of these leading twist cases are much smaller than the absolute values, in indicates that the contributions from higher twist terms are not significant. So that we adopt the average of case $\Gamma=\gamma^z\gamma_5$ and $\Gamma=\gamma^t\gamma_5$ for the following analysis.
\begin{eqnarray}
&&\tilde{\Phi}^{\pm} = \frac{1}{2}[\tilde{\Phi}^{\pm}(\Gamma=\gamma^t\gamma_5)+ \tilde{\Phi}^{\pm}(\Gamma=\gamma^z\gamma_5)].\end{eqnarray}
Fig. \ref{fig:qwf_co}. shows the real and imaginary parts of subtracted quasi TMDWFs in coordinate space for the average of two Lorentz structures.  

%%%%%%%%%%%%%%%%%%%%%%%%%%%%%%%%%%%%%%
\begin{figure}
    \centering
    \includegraphics[scale=0.55]{qwf_co.pdf}
    \caption{An example of subtracted quasi TMDWFs $\tilde{\Phi}^{\pm}(z,b_{\perp},P^z,\mu)$ with $\{P^z,b_{\perp}\}=\{2.58\text{GeV},3a\}$ as a function of $\lambda=zP^z$.}
    \label{fig:qwf_co}
\end{figure}
%%%%%%%%%%%%%%%%%%%%%%%%%%%%%%%%%%%%%%

The quasi TMDWFs in momentum space is defined as:
\begin{eqnarray}
&&\tilde \Psi^{\pm}(x, b_\perp,\mu, P^z) \\
&&= \lim\limits_{L\to \infty}\int \frac{P^zdz}{4\pi} e^{i(x-1/2) P^z z} \frac{1}{\sqrt{Z_E(2L,b_\perp,\mu)}}\tilde{\Psi}^{\pm}(z,b_{\perp},P^z),\nonumber
\end{eqnarray}
This equation gives the Fourier transformation factor $e^{i(x-1/2)zP^z}$ between quasi TMDWFs in coordinate $\tilde{\Phi}^{\pm}(z,b_\perp,\mu,P^z)$ and momentum space $\tilde\Psi^{\pm}(x,b_\perp,\mu,P^z)$. Corresponding to $\tilde{\Phi}^+(z,b_\perp,\mu,P^z)$ shown in Fig. \ref{fig:qwf_co}, we obtain the $\tilde{\Psi}^+(x,b_\perp,\mu,P^z)$ shown in Fig. \ref{fig:qwf_momenta}.

%%%%%%%%%%%%%%%%%%%%%%%%%%%%%%%%%%%%%%
\begin{figure}
    \centering
    \includegraphics[scale=0.55]{qwf_mom.pdf}
    \caption{Numerical results for subtracted quasi-TMDWF in momentum space $\tilde{\Psi}^+(x,b_{\perp},P^z)$ with $\{P^z,b_{\perp}\}=\{2.58\text{GeV},3a\}$.}
    \label{fig:qwf_momenta}
\end{figure}
%%%%%%%%%%%%%%%%%%%%%%%%%%%%%%%%%%%%%%

\subsection{Collins-Soper kernel extracted from quasi TMDWFs}

As the CS kernel is defined as the ratio of different momenta and perturbative kernels in Eq. \ref{eq:definition_CS-kernel}, an average over $\tilde{\Psi}^\pm$ is taken to reduce the systematic errors in the calculation,
\begin{align}
	K\left(b_{\perp}, \mu\right)&=\frac{1}{2\ln(P_1^z/P_2^z)} \left[\ln\frac{H^{+}(xP_2^z,\mu)\tilde{\Psi}^{+}(x,b_{\perp},\mu,P_1^z)}{H^{+}(xP_1^z,\mu)\tilde{\Psi}^{+}(x,b_{\perp},\mu,P_2^z)}\right.  \nonumber\\
	&+\left.\ln\frac{H^{-}(xP_2^z,\mu)\tilde{\Psi}^{-}(x,b_{\perp},\mu,P_1^z)}{H^{-}(xP_1^z,\mu)\tilde{\Psi}^{-}(x,b_{\perp},\mu,P_2^z)}\right].
\label{eq:CS-kernel}
\end{align}

However, in numerical analysis, the momentum fraction $x$ of $K(b_{\perp},\mu)$ should be eliminated. In some area when $x$ is close to $0$ or $1$, the high order correction of factorization is clearly displayed. To include the information of $x\sim(0,1)$, doing an appropriate fit for $K(b_{\perp},\mu)$ seems reasonable. The high order correction of $K(b_{\perp},\mu)$ is mainly comes from $\tilde{\Psi}(x,b_\perp,\mu,P^z)$ as term of $\frac{1}{x(P^z)^2}$ or $\frac{1}{(1-x)(P^z)^2}$,  so one approach to factorize this effect is in the following equations:

\begin{eqnarray}
&&K(b_{\perp},\mu,x,P^z_1,P^z_2) = K(b_{\perp},\mu)\nonumber\\
&&+A\bigg[\frac{1}{x^2(1-x)^2(P^z_1)^2}-\frac{1}{x^2(1-x)^2(P^z_2)^2}\bigg]
\end{eqnarray}

Our data shows $K(b_{\perp},\mu)$ is real to a high extent, so only the significantly non-zero real part is considered for this fit, the detailed information is shown the appendix. The result of central value of $K(b_{\perp},\mu)$ is generated by the average of bootstrap fit results, and the uncertainty is the standard deviation at 68\% confidence interval of them. Figure\ref{fig:K_b} shows the bootstrap joint fit can give the plateau of $x$ for $K(b_{\perp},\mu,x,P^z_1,P^z_2)$, and of different momentum pairs are not all that dissimilar.

\begin{widetext}

%%%%%%%%%%%%%%%%%%%%%%%%%%%%%%%%%%%%%%
\begin{figure}
    \centering
    \includegraphics[scale=0.5]{K_b1.pdf}
    \includegraphics[scale=0.5]{K_b2.pdf}
    \includegraphics[scale=0.5]{K_b3.pdf}
   \includegraphics[scale=0.5]{K_b4.pdf}
    \includegraphics[scale=0.5]{K_b5.pdf}
    \caption{The fit results of $K(b_{\perp},\mu,x,P^z_1,P^z_2)$ extracted of quasi TMDWFs $\tilde{\Psi}^{\pm}$. We choosing the momentum pairs $\{P_1^z,P^z_2\}$ denoted by $P^z_1/P_z^2$ in the legend. The horizontal shaded band show the central value and uncertainty of $K(b_{\perp},\mu)$, as well as the fit range of $x$, as described in the text.}
    \label{fig:K_b}
\end{figure}
%%%%%%%%%%%%%%%%%%%%%%%%%%%%%%%%%%%%%%

\end{widetext}

In addition, as a comparison, the tree level result of $K(b_{\perp},\mu)$ is computed by coordinate quasi TMDWFs $\tilde{\Psi}(z,b,P^z,\mu)$, in which $H_1^{\pm}=1$ 
\begin{eqnarray}
&&K_0(b_\perp, \mu) \nonumber\\
&&= \frac{1}{2} \bigg(\frac{1}{\ln (P^z_1/P^z_2)} \ln\frac{\tilde \Phi^+(z=0, b_\perp, \mu, P^z_1)}{\tilde \Phi^+(z=0, b_\perp, \mu, P^z_2)}\nonumber\\
&&+ \frac{1}{\ln (P^z_1/P^z_2)}\ln\frac{\tilde \Phi^-(z=0, b_\perp, \mu, P^z_1)}{\tilde \Phi^-(z=0, b_\perp, \mu, P^z_2)}\bigg),
\end{eqnarray}
Also, the systematical uncertainty needs to be considered to determine the final result $K(b_{\perp},\mu)$. It is formed as the following equation Eq. (\ref{eq:systematical_error}).
\begin{eqnarray}
&&\sigma_{sys}\nonumber\\
&&=\sqrt{K(b_{\perp},\mu)+\text{Im}^2[K^+(b_{\perp},\mu)]}-K(b_{\perp},\mu)
\label{eq:systematical_error}
\end{eqnarray}
$K(b_{\perp},\mu)$ is related to the Eq.(\ref{eq:CS-kernel}),  
We compared this result with other groups' lattice results and a perturbative calculation in Fig. \ref{fig:K_com}, which shows in small $b_{\perp}$ area, our result reaching to one loop matching kernel for TMDWFs is very close to the perturbative calculation. Beside, it is consist of our result with others', and is more precise in large $b_{\perp}$ area for the uncertainty is quite small.


%%%%%%%%%%%%%%%%%%%%%%%%%%%%%%%%%%%%%%
\begin{figure}
    \centering
    \includegraphics[scale=0.45]{K_com.pdf}
    \caption{The comparison of our final result of $K(b_{\perp},\mu)$ and the results of SWZ~\cite{Shanahan:2021tst}, LPC collaborations~\cite{LatticeParton:2020uhz} as well as ETMC/PKU~\cite{Li:2021wvl}, SVZES\cite{Schlemmer:2021aij} and the perturbative calculations (at small $b_{\perp}$) with the strong coupling $\alpha_s$ up to three loop. The uncertainty of the $K(b_{\perp},\mu)$ is related to statistic one plus systematical one.}
    \label{fig:K_com}
\end{figure}
%%%%%%%%%%%%%%%%%%%%%%%%%%%%%%%%%%%%%%
Additionally, we obtain the phenomenological result from Drell-Yan process to extract CS kernel by the parameterization for the anomalous dimension in SV19 ~\cite{Scimemi:2019cmh}, and for TMDPDFs in Pavia19 ~\cite{Bacchetta:2019sam}. For SV19, $K(b_{\perp},\mu)$ is consist of a perturbative part and a non-perturbative part fitted by Drell-Yan data. For Pavia19, TMDPDFs is separated by perturbative and non-perturbative part. Thus CS kernel is obtained by the derivative of TMDPDFs to the energy scale. Fig. \ref{fig:K_pheno} shows the comparison of our results and these two phenomenological ones.

%%%%%%%%%%%%%%%%%%%%%%%%%%%%%%%%%%%%%%
\begin{figure}
    \centering
    \includegraphics[scale=0.45]{K_pheno.pdf}
    \caption{Comparison with the SV19 ~\cite{Scimemi:2019cmh} and Pavia19 ~\cite{Bacchetta:2019sam} phenomenological parameterizations. The uncertainty of the $K(b_{\perp},\mu)$ is related to statistic one plus systematical one.}
    \label{fig:K_pheno}
\end{figure}
%%%%%%%%%%%%%%%%%%%%%%%%%%%%%%%%%%%%%%



\section{Proof of whether TMDWF is real or not}
If we choose $\gamma^z\gamma_5$ as the current, the definition of TMDWF and its factorization are written 
\begin{eqnarray}
&&\langle0|\bar{\psi}(-\frac{b}{2},-\frac{z}{2})U(-\frac{z}{2},-\frac{b}{2}\leftarrow B)U(B\leftarrow A)U(A\leftarrow\frac{b}{2},\frac{z}{2})\nonumber\\
&&\gamma^z\gamma_5\psi(\frac{b}{2},\frac{z}{2})|\pi(P_z)\rangle=if_{\pi}P^0\phi(zP_z+m_{\pi}\sqrt{z^2+b^2}),
\end{eqnarray}
and Wilson line is written
\begin{eqnarray}
&&U(-\frac{z}{2},-\frac{b}{2}\leftarrow B)=exp\{ig\int_{+\infty}^{-\frac{z}{2}}dx^3A_3(x)\}=U_1,\nonumber\\
&&U(B\leftarrow A)=exp\{ig\int_{\frac{b}{2}}^{-\frac{b}{2}}dx^1A_1(x)\}=U_2,\nonumber\\
&&U(A\leftarrow\frac{b}{2},\frac{z}{2})=exp\{ig\int_{\frac{z}{2}}^{+\infty}dx^3A_3(x)\}=U_3,\end{eqnarray}


%%%%%%%%%%%%%%%%%%%%%%%%%%%%%%%%%%%%%%
\begin{figure}
    \centering
    \includegraphics[scale=0.25]{shift.pdf}
    \caption{Wilson line in coordinate space}
    \label{fig:Fit_to_MC}
\end{figure}
%%%%%%%%%%%%%%%%%%%%%%%%%%%%%%%%%%%%%%



first we do time reversal of it
\begin{eqnarray}
&&TT^{\dagger}=I,\;\;TA^{\mu}(x)T^{\dagger}=A_{\mu}(\vec{x},-t)\nonumber\\
&&\to TU(x)T^{\dagger}=U(\vec{x},-t),\\
&&T\psi(x)T^{\dagger}=i\gamma^x\gamma^z\psi(\vec{x},-t)\nonumber\\
&&T\bar{\psi}(x)T^{\dagger}=-i\psi(\vec{x},-t)\gamma^z\gamma^x,
\end{eqnarray}
then we can insert the identity operator $I=TT^{\dagger}$ and then do a complex conjugate
\begin{eqnarray}
&&-if_{\pi}P^0\phi^*(zP_z+m_{\pi}\sqrt{z^2+b^2})\\
&&=\langle\pi(P_z)|\bar{\psi}(-\frac{b}{2},-\frac{z}{2},)U_1U_2U_3(\vec{x},-t)\gamma^t\gamma_5\psi(\frac{b}{2},\frac{z}{2})|0\rangle^*\nonumber\\
&&=-\langle0|\bar{\psi}(\frac{b}{2},\frac{z}{2})U_3^*(x)U_2^*(x)U_1^*(x)\gamma^t\gamma_5\psi(-\frac{b}{2},-\frac{z}{2})|\pi(P_z)\rangle\nonumber
\end{eqnarray}
now to do a parity transformation
\begin{eqnarray}
&&-if_{\pi}P^0\phi^*(zP_z+m_{\pi}\sqrt{z^2+b^2}\\
&&=-\langle0|\bar{\psi}(-\frac{b}{2},-\frac{z}{2})U_1^{\prime}(x)U_2^{\prime}(x)U_3^{\prime}(x)\gamma^t\gamma_5\psi(\frac{b}{2},\frac{z}{2})|\pi(-P_z)\rangle\nonumber\\
&&=-if_{\pi}P^0\phi^{\prime}(zP_z+m_{\pi}\sqrt{z^2+b^2}),\nonumber
\end{eqnarray}
\begin{eqnarray}
&&U_1^{\prime}=U(-\frac{z}{2},-\frac{b}{2}\leftarrow B^{\prime})=exp\{ig\int_{-\infty}^{-\frac{z}{2}}dx^3A_3(x)\},\\
&&U_2^{\prime}=U(B^{\prime}\leftarrow A^{\prime})=exp\{ig\int_{\frac{b}{2}}^{-\frac{b}{2}}dx^1A_1(x)\},\\
&&U_3^{\prime}=U(A^{\prime}\leftarrow\frac{z}{2},\frac{b}{2})=exp\{ig\int_{\frac{z}{2}}^{-\infty}dx^3A_3(x)\},
\end{eqnarray}
the difference between $\phi$ and $\phi^{\prime}$ is the Wilson lines, and there're some inferences	
\begin{eqnarray}
&&\text{if $b=0$:}\\
&&U_1^{\prime}U_2^{\prime}U_3^{\prime}=exp\{ig\int_{
\frac{z}{2}}^{-\frac{z}{2}}dx^3A_3(x)\}=U_1U_2U_3\nonumber\\
&&\to\phi^{\prime}=\phi\to\phi^*=\phi\nonumber\\
&&\text{if $(-\frac{b}{2},-\frac{z}{2})=B$ and $z=0$:}\\
&&U_1^{\prime}U_2^{\prime}U_3^{\prime}=exp\{ig\int_{
\frac{b}{2}}^{-\frac{b}{2}}dx^3A_3(x)\}=U_1U_2U_3\nonumber\\
&&\to\phi^{\prime}=\phi\to\phi^*=\phi\nonumber
\end{eqnarray}
In total, TMDWF($\phi(l,b,z)$) is not real, except for $b=0$ or $(-\frac{b}{2},-\frac{z}{2})=B\;and\;z=0$.

