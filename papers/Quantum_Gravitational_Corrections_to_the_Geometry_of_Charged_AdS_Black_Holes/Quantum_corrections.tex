
\documentclass[10pt,a4paper]{article}
%\usepackage{jcappub}
\usepackage{jheppub}
\usepackage{pdflscape}
\usepackage{amsmath}
\usepackage{amssymb}
\usepackage{dcolumn}
\usepackage{bm}
\usepackage{color}
\usepackage{epsfig}
\usepackage{amsfonts}
\usepackage{graphicx}
\usepackage{subfigure}
\usepackage{dcolumn}
\usepackage{color}
\newcommand{\tcb}{\textcolor{blue}}
\newcommand{\tcr}{\textcolor{red}}
\newcommand{\tcg}{\textcolor{green}}
 \newcommand{\tcc}{\textcolor{cyan}}
\newcommand{\tcp}{\textcolor{purple}}			
\newcommand{\tco}{\textcolor{orange}}

\begin{document}
\title{ Quantum Gravitational Corrections to the Geometry of Charged AdS Black Holes}

\author[a]{Behnam Pourhassan,}
\author[c]{Ruben Campos Delgado}

\affiliation[a]{School of Physics, Damghan University, P. O. Box 3671641167, Damghan, Iran.}
\affiliation[c]{Bethe Center for Theoretical Physics and Physikalisches Institut der Universit\"at Bonn,
Nussallee 12, 53115 Bonn, Germany}

\emailAdd{b.pourhassan@du.ac.ir, b.pourhassan@candqrc.ca}

\emailAdd{ruben.camposdelgado@gmail.com}

\abstract{We study the quantum gravitational corrections to the geometry of a four-dimensional charged (Reissner-Nordstr\"om) Anti de Sitter black hole starting from an effective field theory approach to quantum gravity. We use the expression of the modified horizon radius to compute the quantum corrected Wald entropy, whose expression reproduces the logarithmic behaviour found by other methods. We perform a thermodynamics analysis and compute the quantum gravitational corrections to the temperature, pressure, specific heat and Helmholtz free energy. All these quantities are renormalisation group invariant. We find that a quantum charged AdS black hole can exist only for a bounded range of masses and that it can undergo a second order phase transition as it moves from a state with positive specific heat to a negative one. }

\keywords{Black Hole; Quantum Corrections; Thermodynamics.}

\maketitle
%%%%%%%%%%%%%%%%%%%%%%%%
\section{Introduction}
A classical black hole never vanishes and is eternal. However, from the quantum mechanical point of view, a black hole emits Hawking radiation \cite{Page:2004xp}  which reduces its size. In that case, there are two scenarios for a black hole at the final stage. In the first one, the black hole evaporates and vanishes. In the second one, it stops evaporating and becomes a stable remnant. The reason for the second scenario is attributable to perturbative \cite{Das:2001ic, Pourhassan:2015cga} or non-perturbative \cite{Chatterjee:2020iuf, Pourhassan:2020yei} corrections, which are due to thermal fluctuations or quantum effects. The area entropy formula \cite{Bardeen:1973gs} for the black holes gets corrected due to the quantum effects. Some leading order corrections appear as a logarithmic term in the black hole entropy \cite{Sadeghi:2016dvc}, which may be related to conformal anomaly \cite{Xiao:2021zly}. Non-perturbative corrections can also lead to an exponential corrected entropy \cite{Pourhassan:2021mhb}. In general, any modification to the black hole entropy and temperature leads to a modification of the whole black hole thermodynamics \cite{Rostami:2019dom}. As a consequence, the black hole stability may be altered and a black hole can become stable at the quantum scales \cite{Upadhyay:2019hyw}. This line of research has been attracted a wide interest over the past recent years, resulting in the study of several types of black holes. For example, singly spinning Kerr-AdS black holes were investigated to find that thermal fluctuations, contributing with a logarithm term, are important only for small black holes \cite{Pourhassan:2016zzc}. However, as soon as the size reduces due to Hawking radiation, non-perturbative corrections become dominant \cite{Pourhassan:2020bzu}. Logarithmic corrections to the black hole entropy were also considered in \cite{Castro:2018hsc}.

As the black hole entropy is obtained using the black hole horizon area and hence the black hole event horizon radius, it is natural to expect that the quantum corrections modify the black hole metric as to reproduce the corrected entropy. Effective field theory may be used to obtain the quantum corrected metric. This requires to write down a general action as a curvature expansion, then solve the corresponding equations of motion. For the case of a Schwarzschild black hole, it was shown that the metric receives quantum corrections at third order in curvature \cite{Calmet:2021lny}. For a Reissner-Nordstr\"om black hole, the metric receives corrections already at second order \cite{Delgado:2022pcc}. Although there are several other ways to obtain the quantum corrected entropy \cite{Cano:2019ycn,PhysRevD.62.024001,Yoon:2007aj,Akbar:2003mv, Sadeghi:2014zna}, we followed the same approach used in \cite{Calmet:2021lny, Delgado:2022pcc} to obtain the modified geometry that produces the corrected entropy given by the Wald formula \cite{Wald:1993nt}. There are several works in the literature where the black hole entropy gets corrected but the temperature does not (see for example \cite{Pourhassan:2016qoz, Khan:2021tzv}). However, the first law of black hole thermodynamics suggests that the black hole temperature should also receive some corrections (otherwise, one should modify the first law by adding new thermodynamic variables \cite{Dehghani:2021qzm}). Hence, the best way to obtain both the modified entropy and temperature of a black hole is to obtain the quantum effects of the black hole geometry.

In this paper we consider four-dimensional Anti de Sitter Reissner-Nordstr\"om (AdS RN) black holes. Such type of black holes are important from an AdS/CFT correspondence point of view. We determine the modified black hole metric due to the quantum effects and compute the quantum gravitational corrections to the entropy. Our expression of the entropy reproduces the logarithmic behaviour for charged AdS black holes already found in the literature (see for example \cite{Das:2001ic, Sadeghi:2016dvc}). We then proceed to compute the quantum corrections of several thermodynamic quantities: temperature, pressure, specific heat and Helmholz free energy. We find that a quantum AdS RN black hole can exists only within a bounded range of masses, and that a second order phase transition can occur. Moreover, all quantities are renormalisation group (RG) invariant, meaning that they do not depend on the arbitrary energy scale of the effective field theory. 
The advantage of the method used in this paper is that it can be used as a general formalism for all black hole solutions. It is also extendable to all orders in perturbation theory. 

This paper is organized as follows. In Section \ref{sec:corrections_metric} we introduce the effective field theory approach to quantum gravity and obtain the solution of the equations of motion that represents the quantum corrected metric of a charged AdS black hole. In Section \ref{sec:entropy} we calculate the quantum corrections to the black hole entropy. The behaviour of several quantum corrected thermodynamic quantities is discussed in section \ref{sec:thermodynamics}. Finally, in section \ref{sec:conclusions} we summarise our results and discuss possible future directions. 
%%%%%%%%%%%%%%%%%%%%%%%%%%%%
%%%%%%%%%%%%%%%%%%%%%%%%%%%%
\section{Quantum gravitational corrections to the classical metric}\label{sec:corrections_metric}
We start by reviewing some known facts about the effective field theory approach to quantum gravity. For a good introduction to this topic, see for example \cite{Donoghue:2017pgk}. A possible way to study quantum effects in gravity is to modify the pure Einstein-Hilbert action by including additional terms normally suppressed at low energies. The main possibility, at second order in curvature, is the local action
\begin{equation}\label{eq:local_action}
    \Gamma_{L}=\int d^4x\, \sqrt{-g}\,\left(\frac{R+2\Lambda}{16\pi G_N} +c_1(\mu)R^2+c_2(\mu)R_{\mu\nu}R^{\mu\nu}+c_3(\mu)R_{\mu\nu\rho\sigma}R^{\mu\nu\rho\sigma}\right),
\end{equation}
where $\Lambda$ is the cosmological constant and $\mu$ is an energy scale. The exact values of the constants $c_1$, $c_2$, $c_3$ are unknown, as they depend on the UV completion of quantum gravity.
By integrating out fluctuations of the graviton and of any matter field irrelevant to the problem under consideration, one also gets a non-local (NL) effective action \cite{Weinberg:1980gg, Starobinky:1981, Barvinsky:1983vpp, Barvinsky:1985an, Barvinsky:1987uw, Barvinsky:1990up, Donoghue:1994dn}, which, at second order in curvature, is
\begin{equation}
    \Gamma_{NL}=-\int d^4 x \sqrt{-g}\left[\alpha R\ln\left(\frac{\Box}{\mu^2}\right)R+\beta R_{\mu\nu}\ln\left(\frac{\Box}{\mu^2}\right)R^{\mu\nu} + \gamma R_{\mu\nu\rho\sigma}\ln\left(\frac{\Box}{\mu^2}\right)R^{\mu\nu\rho\sigma}\right],
\end{equation}
where $\alpha$, $\beta$ and $\gamma$ are known constants \cite{Donoghue:2014yha}.
The operator $\ln\left(\Box/\mu^2\right)$ has the integral representation \cite{Donoghue:2015nba}
\begin{equation}
    \ln\left(\frac{\Box}{\mu^2}\right)=\int_0^{+\infty}ds\, \left(\frac{1}{\mu^2+s}-\frac{1}{\Box+s}\right).
\end{equation}
Using both the local and non-local Gauss-Bonnet identities \cite{Calmet:2018elv}, it is possible to eliminate the Riemann tensor by redefining the coefficients as
\begin{equation}
\begin{split}
   c_1 &\to\bar{c}_1=c_1-c_3, \hspace{4mm}c_2\to\bar{c}_2=c_2+4c_3, \hspace{4mm} c_3\to  \bar{c}_3=0,\\
    &\hspace{3mm}\alpha\to\bar{\alpha}=\alpha-\gamma, \hspace{4mm} \beta\to\bar{\beta}=\beta+4\gamma, \hspace{4mm} \gamma\to\bar{\gamma}=0.
\end{split}
\end{equation}
We can couple gravity to electromagnetism by adding the Maxwell action
\begin{equation}
    \Gamma_{M}=-\frac{1}{4}\int d^4x\,\sqrt{-g}\,F_{\mu\nu}F^{\mu\nu},
\end{equation}
where $F_{\mu\nu}=\partial_{\mu}A_{\nu}-\partial_{\nu}A_{\mu}$ is the electromagnetic tensor and $A_{\mu}$ is the electromagnetic potential.
In this paper we then consider the full action
\begin{equation}\label{eq:total_action}
    \Gamma=\Gamma_{L}+\Gamma_{NL}+\Gamma_{M}\equiv\int d^4x\,\sqrt{-g}\,\mathcal{L}.
\end{equation}
The Maxwell equations, obtained by varying \eqref{eq:total_action} with respect to $A_{\mu}$, are
\begin{equation}\label{eq:maxwell_equations}
    g^{\mu\nu}\nabla_{\mu}F_{\nu\tau}=0.
\end{equation}
The quantum corrected Einstein equations, obtained by varying \eqref{eq:total_action} with respect to the metric, are
\begin{equation}\label{eq:einstein_equations}
    \frac{1}{8\pi G_N}\left(G_{\mu\nu}-\Lambda g_{\mu\nu}\right)+2\left(H_{\mu\nu}+K_{\mu\nu}\right)=T_{\mu\nu},
\end{equation}
where $T_{\mu\nu}$ and $G_{\mu\nu}$ are respectively the energy-momentum tensor and Einstein tensor, which are given by
\begin{equation}
    T_{\mu\nu}=\frac{1}{4\pi}\left(F_{\mu\rho}{F_{\nu}}^{\rho}-\frac{1}{4}g_{\mu\nu}F_{\rho\sigma}F^{\rho\sigma}\right),
\end{equation}
\begin{equation}
    G_{\mu\nu}=R_{\mu\nu}-\frac{1}{2}R g_{\mu\nu}.
\end{equation}
$H_{\mu\nu}$ and $K_{\mu\nu}$ encode the quantum effects. The local contribution is
\begin{equation}
\begin{split}
    &H_{\mu\nu}=\bar{c}_1\left(2R R_{\mu\nu}-\frac{1}{2}g_{\mu\nu}R^2-2\nabla_{\mu}\nabla_{\nu}R+2g_{\mu\nu}\Box R\right)\\
    +\bar{c}_2 &\left(-\frac{1}{2}g_{\mu\nu}R_{\rho\sigma}R^{\rho\sigma}+2R^{\rho\sigma}R_{\mu\rho\nu\sigma}-\nabla_{\mu}\nabla_{\nu}R+\Box R_{\mu\nu}+\frac{1}{2}g_{\mu\nu}\Box R\right),
\end{split}
\end{equation}
while the non-local contribution is
\begin{equation}
\begin{split}
    K_{\mu\nu}&= -2\bar{\alpha}\left(R_{\mu\nu}-\frac{1}{4}g_{\mu\nu}R+g_{\mu\nu}\Box-\nabla_{\mu}\nabla_{\nu}\right)\ln\left(\frac{\Box}{\mu^2}\right)R
    -\bar{\beta}\bigg({\delta^{\rho}}_{\mu}R_{\nu\sigma}+{\delta^{\rho}}_{\nu}R_{\mu\sigma}\\&-\frac{1}{2}g_{\mu\nu}{R^{\rho}}_{\sigma}+{\delta^{\rho}}_{\mu}g_{\nu\sigma}\Box+
   g_{\mu\nu}\nabla^{\rho}\nabla_{\sigma}-{\delta^{\rho}}_{\mu}\nabla_{\sigma}\nabla_{\nu}-{\delta^{\rho}}_{\nu}\nabla_{\sigma}\nabla_{\mu}\bigg)\ln\left(\frac{\Box}{\mu^2}\right){R^{\sigma}}_{\rho}.
\end{split}
\end{equation}
We now solve the quantum corrected Einstein and Maxwell equations using perturbation theory in the gravitational coupling $G_N$. Classically, a four-dimensional charged AdS black hole is described by its mass $M$, its charge $Q$ and the AdS radius $L$, which is related to the cosmological constant via
\begin{equation}
    L=\sqrt{\frac{3}{\vert \Lambda \rvert}}.
\end{equation}
In turn, the cosmological constant is proportional to the vacuum energy density $\rho$,  $\Lambda=8\pi G \rho$. For our future calculations we find convenient to keep the classical metric written in terms of $\rho$.
We now consider a small perturbation of order $\mathcal{O}\left(G^2_N\right)$ around the classical solution:
\begin{equation}\label{eq:expansion}
    g_{\mu\nu}=g^{AdS}_{\mu\nu}+g^{q}_{\mu\nu},
\end{equation}
where
\begin{equation}
\begin{gathered}
    ds^2_{AdS}=g^{AdS}_{\mu\nu}dx^{\mu}dx^{\nu}=-\left(1-\frac{2G_N M}{r}+\frac{G_N Q^2}{r^2}+\frac{1}{3}8\pi\rho G_N r^2\right)dt^2\\
    +\left(1-\frac{2G_N M}{r}+\frac{G_N Q^2}{r^2}+\frac{1}{3}8\pi\rho G_N r^2\right)^{-1}dr^2+
    r^2d\theta^2+r^2\sin^2\theta d\phi^2.
\end{gathered}
\end{equation}
The quantum effects are encoded in $g^q_{\mu\nu}$. We set $g_{\theta\theta}=g_{\phi\phi}=0$ and introduce two functions $\Sigma(r)$ and $\Omega(r)$ such that
\begin{equation}\label{eq:quantum_metric}
    ds^2_{q}=g^{q}_{\mu\nu}dx^{\mu}dx^{\nu}=G^2_N\Delta(r)dt^2+G^2_N\Sigma(r)dr^2.
\end{equation}
The full metric is then
\begin{equation}\label{eq:full_metric}
\begin{gathered}
    ds^2=g_{\mu\nu}dx^{\mu}dx^{\nu}=\left[-\left(1-\frac{2G_N M}{r}+\frac{G_N Q^2}{r^2}+\frac{1}{3}8\pi\rho G_N r^2\right)+G^2_N\Delta(r)\right]dt^2\\
    +\left[\left(1-\frac{2G_N M}{r}+\frac{G_N Q^2}{r^2}+\frac{1}{3}8\pi\rho G_N r^2\right)^{-1}+G^2_N\Sigma(r)\right]dr^2+
    r^2d\theta^2+r^2\sin^2\theta d\phi^2.
\end{gathered}
\end{equation}
Quantum effects can also shift the classical value of the non-vanishing components of the electromagnetic tensor $F_{\mu\nu}$. Thus, we introduce a function $\Omega(r)$ such that
\begin{equation}\label{eq:expanded_F}
    F_{tr}=-F_{rt}=\frac{Q}{r^2}+G^2_N\Omega(r).
\end{equation}
Next, we plug the metric \eqref{eq:full_metric} and the components of the electromagnetic tensor \eqref{eq:expanded_F} into the equations \eqref{eq:maxwell_equations}, \eqref{eq:einstein_equations}, keeping only terms up to order $\mathcal{O}\left(G^2_N\right)$.
Since $H_{\mu\nu}$ and $K_{\mu\nu}$ are quadratic in curvature, they just need to be evaluated on the classical charged AdS solution, i.e. \eqref{eq:einstein_equations} can be expressed schematically as
\begin{equation}\label{eq:einstein_quantum}
\frac{1}{8\pi G_N}\left(G_{\mu\nu}-\Lambda g_{\mu\nu}\right)\left[g^{AdS}+g^{q}\right]+2 \left(H_{\mu\nu}\left[g^{AdS}\right]+K_{\mu\nu}\left[g^{AdS}\right]\right)=T_{\mu\nu}\left[g^{AdS}+g^{q}\right].
\end{equation}
The results of the action of $\ln\left(\Box/\mu^2\right)$ on several radial functions were collected in \cite{Delgado:2022pcc}. However, unlike the pure Reissner-Nordstr\"om solution, the charged AdS spacetime has the non-vanishing Ricci curvature $R=-32G\pi\rho$. Hence, we need to additionally compute the action of $\ln\left(\Box/\mu^2\right)$ on a constant. Using the formula \cite{Calmet:2019eof}
\begin{equation}\label{eq:formula_actionlog}
\begin{split}
    \ln\left(\frac{\Box}{\mu^2}\right)f(r)&=\frac{1}{r}\int_0^{+\infty}dr'\,\frac{r'}{r+r'}f(r')-\lim_{\epsilon\to 0^{+}}\bigg\{ \frac{1}{r}\int_0^{r-\epsilon}dr'\,\frac{r'}{r-r'}f(r')\\
    &+\frac{1}{r}\int_{r+\epsilon}^{+\infty}dr'\,\frac{r'}{r'-r}f(r')
    +2f(r)\left[\gamma_E+\ln\left(\mu\epsilon\right)\right]\bigg\},
\end{split}
\end{equation}
the result is
\begin{equation}
    \ln\left(\frac{\Box}{\mu^2}\right)\cdot 1=-2\left(\ln(\mu r) + \gamma_E -1 \right),
\end{equation}
where $\gamma_E$ is the Euler-Mascheroni constant.
The corresponding calculation is collected in Appendix \ref{sec:appendix}, where we show how to regularise the integrals.
The final solution of the quantum corrected equations of motion \eqref{eq:einstein_quantum} and \eqref{eq:maxwell_equations} is
\begin{equation}\label{eq:final_metric}
\begin{split}
    &ds^2=-f(r)dt^2+\frac{1}{g(r)}dr^2+r^2d\theta^2+r^2\sin^2\theta d\phi^2,\\
    F_{tr}=-F_{rt}=&\frac{Q}{r^2}+\frac{16\pi G^2_N Q^3}{r^6}\left[c_2+4c_3+\left(\beta+4\gamma\right)\left(2\ln\left(\mu r\right)+2\gamma_E-5\right)\right]\\
    &+\frac{256\pi^2G^2_NQ\rho}{r^2}(4\alpha+\beta)\ln(r)
\end{split}
\end{equation}
with
\begin{equation}
\begin{split}
    f(r)=&1-\frac{2G_N M}{r}+\frac{G_N Q^2}{r^2}+\frac{1}{3}8\pi\rho G_N r^2+512\pi^2G^2_N\rho(4\alpha+\beta)\ln(r)\\&-\frac{32\pi G^2_N Q^2}{r^4} \Big[c_2+4c_3+\left(\beta+4\gamma\right)\left(2\ln\left(\mu r\right)+2\gamma_E-3\right)\Big],
\end{split}
\end{equation}
\begin{equation}
\begin{split}
    g(r)=&1-\frac{2G_N M}{r}+\frac{G_N Q^2}{r^2}+\frac{1}{3}8\pi\rho G_N r^2+512\pi^2G^2_N\rho(4\alpha+\beta)\\&-\frac{64\pi G^2_N Q^2}{r^4}\Big[c_2+4c_3+2\left(\beta+4\gamma\right)\left(\ln\left(\mu r\right)+\gamma_E-2\right)\Big].
\end{split}
\end{equation}
In the limit $\rho\to 0$ (or equivalently $L\to\infty$) one correctly recovers the solution obtained in \cite{Delgado:2022pcc} for the pure Reissner-Nordstr\"om black hole.
Although the metric seems to depend on the arbitrary energy scale $\mu$, the renormalised constants $c_1$, $c_2$, and $c_3$ also carry an explicit scale dependence \cite{El-Menoufi:2015cqw}:
\begin{equation}\label{eq:coefficients_scale}
\begin{split}
    c_1(\mu)=c_1(\mu_*)-\alpha\ln\left(\frac{\mu^2}{\mu^2_*}\right),\\
    c_2(\mu)=c_2(\mu_*)-\beta\ln\left(\frac{\mu^2}{\mu^2_*}\right),\\
    c_3(\mu)=c_3(\mu_*)-\gamma\ln\left(\frac{\mu^2}{\mu^2_*}\right),\\
\end{split}
\end{equation}
where $\mu_*$ is some fixed scale where the effective theory is matched onto the full theory. Taking into account \eqref{eq:coefficients_scale}, one sees that the metric ultimately does not depend on $\mu$, as it must be.

In Fig. \ref{fig1} we plot the function $g(r)$ for both the classical and the quantum corrected metric. We choose units in which $G=\mu_{*}=Q=1$ and $M=2$. 
With a red line and a dashed green line we plot $g(r)$ for the classical Reissner-Nordstr\"om ($\rho=0$) and AdS Reissner-Nordstr\"om metric (with $\rho=0.05$), respectively. Since $GM>Q^2$, both cases present two distinct horizons. We denote the outer Reissner-Nordstr\"om horizon as $r_{RN}$ and the outer AdS Reissner-Nordstr\"om horizon as $r_{AdS}$. If we switch on the quantum corrections, then only one horizon exists. The numerical values of the coefficients $\alpha$, $\beta$, $\gamma$ can be explicitly determined and depend on which type of field is coupled to gravity. For the case of scalar, fermion, vector and graviton fields, the corresponding values were collected in  \cite{Donoghue:2014yha}. In our analysis we found that a unique quantum corrected horizon $r_h$ exists, independently of the values of $M$, $Q$, $c_{1}$ and $c_{2}$. In particular, $r_{AdS}<r_{h}<r_{RN}$. This is shown in Fig. \ref{fig1} for the choice  $G=\mu_{*}=Q=1$, $M=2$, $\rho=0.05$ and $c_2=c_3=0.1$. In the next section we present an analytical expression for $r_h$. 
\begin{figure}
\begin{center}$
\begin{array}{cccc}
\includegraphics[width=100 mm]{horizon_structure.eps}
\end{array}$
\end{center}
\caption{Horizon structure of AdS RN black holes with $G=Q=\mu_{*}=1$ and $M=2$. Solid red: classical RN. Dashed green: classical AdS RN with $\rho=0.05$. For the quantum corrected metric we set everywhere $\rho=0.05$ and $c_{2}=c_{3}=0.1$. Dotted black: quantum corrected AdS RN with a scalar field ($\alpha=0.0010994, \beta=-0.00001759$, $\gamma=0.00001759$). Dashed orange: quantum corrected AdS RN with a fermionic field ($\alpha=-0.000043976, \beta=0.000070361$, $\gamma=0.000061566$). Dashed cyan: quantum corrected AdS RN with a vector field ($\alpha=-0.00043976, \beta=0.0015480$, $\gamma=-0.00022867$). Dashed dotted blue: quantum corrected AdS RN with a graviton field ($\alpha=0.0037819, \beta=-0.012700$, $\gamma=0.0037292$). The numerical values for $\alpha$, $\beta$ and $\gamma$ were taken from \cite{Donoghue:2014yha}.}
\label{fig1}
\end{figure}
%%%%%%%%%%%%%%%%%%%%%%%%%
%%%%%%%%%%%%%%%%%%%%%%%%%
\section{Corrected entropy}\label{sec:entropy}
In this section we compute the quantum gravitational corrections to the entropy. Analogously to what done in \cite{Delgado:2022pcc}, we consider a small electric charge. In addition, we assume the cosmological constant to be small, or, equivalently, the AdS radius to be large. We then compute the entropy up to orders $\mathcal{O}\left(Q^2\right),\mathcal{O}\left(\rho\right)$.
The corrections to the metric imply a shift to the classical outer horizon radius:
\begin{equation}\label{eq:new_radius}
\begin{gathered}
    r_h=2G_NM-\frac{Q^2}{2M}+\frac{8\pi Q^2}{G_NM^3}\Big[c_2+4c_3+2\left(\beta+4\gamma\right)\left(\ln\left(2G_N M\mu \right)+\gamma_E-2\right)\Big]\\
    +\frac{8192\pi^3G_NQ^2\rho}{M^3}(4\alpha+\beta)\Big[2c_2+8c_3+(\beta+4\gamma)\left(4\ln(2GM\mu)+4\gamma_E-9\right)\Big]\\
    +\frac{512\pi^2G^2_NQ^2\rho}{3M}\Big[c_2+4c_3+(\beta+4\gamma)\left(2\ln(2G_N\mu)+2\gamma_E-5\right)\Big].
\end{gathered}
\end{equation}
We introduce the totally antisymmetric symbol $\epsilon_{\mu\nu}$
\begin{equation}
\epsilon_{\mu\nu}=
\begin{cases}
\sqrt{f(r)/g(r)} \hspace{10mm} \text{if} \hspace{1 mm} (\mu,\nu)=(t,r) \\
-\sqrt{f(r)/g(r)} \hspace{7.5 mm} \text{if} \hspace{1 mm} (\mu,\nu)=(r,t)\\
0 \hspace{30 mm} \text{otherwise},
\end{cases}
\end{equation}
so that the Wald formula for the entropy is \cite{Wald:1993nt}
\begin{equation}\label{eq:wald}
\begin{split}
    S_{\text{Wald}}=&-2\pi \int\limits_{r=r_h} d\Sigma\, \epsilon_{\mu\nu}\epsilon_{\rho\sigma}\frac{\partial \mathcal{L}}{\partial R_{\mu\nu\rho\sigma}}= -8\pi\sqrt{\frac{f(r_h)}{g(r_h)}}\int\limits_{r=r_h}d\Sigma\, \frac{\partial \mathcal{L}}{\partial R_{rtrt}}\\
    &=-8\pi\sqrt{\frac{f(r_h)}{g(r_h)}}\frac{\partial \mathcal{L}}{\partial R_{rtrt}}\bigg\rvert_{r=r_h}4\pi r^2_h.
\end{split}
\end{equation}
All the terms in the Lagrangian have to be considered without invoking the Gauss-Bonnet theorem. Furthermore, the following relations are useful:
\begin{equation}
    \frac{\partial R}{\partial R_{\mu\nu\rho\sigma}}=\frac{1}{2}\left(g^{\mu\rho}g^{\nu\sigma}-g^{\mu\sigma}g^{\nu\rho}\right),
\end{equation}
\begin{equation}
    \frac{\partial\left(R_{\alpha\beta}R^{\alpha\beta}\right)}{\partial R_{\mu\nu\rho\sigma}}=g^{\mu\rho}R^{\nu\sigma}-g^{\nu\rho}R^{\mu\sigma},
\end{equation}
\begin{equation}
    \frac{\partial\left(R_{\alpha\beta\gamma\delta}R^{\alpha\beta\gamma\delta}\right)}{\partial R_{\mu\nu\rho\sigma}}=2R^{\mu\nu\rho\sigma}.
\end{equation}
In applying the formula one also needs the results of the action of  $\ln\left(\Box/\mu^2\right)$ on some radial functions. These were computed in \cite{Delgado:2022pcc}. However, the Riemann and Ricci tensors obtained from \eqref{eq:final_metric} contain new terms proportional to $\rho$.  In particular, one needs to know the action of  $\ln\left(\Box/\mu^2\right)$ on $1/r$ and $1/r^2$. The corresponding calculations are collected in Appendix \ref{sec:appendix}.
The final result for the entropy of a charged AdS black hole is
\begin{equation}\label{eq:corrected_entropy}
    S_{\text{Wald}}=\frac{A}{4G}-2\pi Q^2+S_{Sch}+S_{RN}+S_{AdS}+\mathcal{O}\left(Q^4,\rho^2\right),
\end{equation}
where $A=16\pi G^2M^2$ is the classical area of the event horizon of a Schwarzschild black
hole, $S_{Sch}$ is the quantum correction to the entropy of a Schwarzschild black hole, $S_{RN}$ represents the corrections to the entropy of a Reissner-Nordstr\"om black hole and $S_{AdS}$ are the new corrections containing the vacuum energy density $\rho$. The expression of $S_{Sch}$ is \cite{Delgado:2022pcc, Calmet:2021lny}
\begin{equation}
    S_{Sch}=64\pi^2c_3+64\pi^2\gamma\Big[4\ln\left(2G_NM\mu\right)+2\gamma_E-2\Big].
\end{equation}
$S_{RN}$ is \cite{Delgado:2022pcc}
\begin{multline}\label{eq:S_2}
    S_{RN}=\frac{4\pi^2Q^2}{G_NM^2}\Big[5c_2+20c_3+\beta\left(10\gamma_E-21\right)+8\gamma\left(5\gamma_E-11\right)+10\left(\beta+4\gamma\right)\ln\left(2G_NM\mu\right)\Big]\\
   +\frac{64\pi^3Q^2}{9G^2_NM^4}\bigg\{54(\beta+4\gamma)\Big[c_1+2\alpha\ln\left(2G_NM\mu\right)\Big]
   +6(36\gamma_E-75)\beta\Big[c_2+2\beta\ln\left(2G_NM\mu\right)\Big]\\
   +48(48\gamma_E-97)\gamma\Big[c_3+2\gamma\ln\left(2G_NM\mu\right)\Big]
   +6\Big[c_2\gamma(120\gamma_E-251)+3c_3\beta(40\gamma_E-81)\\
   +4\beta\gamma(120\gamma_E-247)\ln\left(2G_NM\mu\right)\Big]
   +54\Big[c^2_2+4c_2\beta\ln\left(2G_NM\mu\right)+4\beta^2\ln^2\left(2G_NM\mu\right)\Big]\\
   +576\Big[c^2_3+4c_3\gamma\ln\left(2G_NM\mu\right)+4\gamma^2\ln^2\left(2G_NM\mu\right)\Big]
   +360\Big[c_2c_3+2c_2\gamma\ln\left(2G_NM\mu\right)\\+2c_3\beta\ln\left(2G_NM\mu\right)
   +4\beta\gamma\ln^2\left(2G_NM\mu\right)\Big]+(\beta+4\gamma)\Big[ 9\alpha(12\gamma_E-25)
   \\+18\beta\left(\gamma_E(6\gamma_E-25)+35+\pi^2\right)
   +4\gamma\big(6\gamma_E\left(24\gamma_E-97\right)+331+30\pi^2\big)\Big]
\bigg\}.
\end{multline}
The expression of $S_{AdS}$ is quite lengthy:
\begin{multline}
   S_{AdS}=-8192\pi^4G^2_N\rho(4\alpha+\beta)\Big[6c_1+c_2-2c_3+2(6\alpha+\beta-2\gamma)\left(\ln(2G_NM\mu)+2\gamma_E\right)\Big]\\
   +\frac{256\pi^3G^2_NQ^2\rho}{3}\Big[2(12c_1+7c_2+18c_3)+4(12\alpha+7\beta+18\gamma)\ln(2G_NM\mu)\\
   +12\alpha(4\gamma_E-3)+7\beta(4\gamma_E-7)+4\gamma(18\gamma_E-41)+6(4\alpha+\beta)\ln(2G_NM)\Big]\\+\frac{4096\pi^4 GQ^2\rho}{3M^2}\bigg\{12\Big[4c_1\beta(\gamma_E-2)+c_2\alpha(4\gamma_E-19)+2\alpha\beta(8\gamma_E-27)\ln(2G_NM\mu)\\
   +2\alpha\ln(2G_NM)\left(c_2+2\beta\ln(2G_NM\mu)\right)\Big]+24\Big[c_1c_2+2c_1\beta\ln(2G_NM\mu)+2c_2\alpha\ln(2GM\mu)\\
   +4\alpha\beta\ln^2(2G_NM\mu)\Big]+48\Big[4c_1\gamma(\gamma_E-2)+c_3\alpha(4\gamma_E-19)+2\alpha\gamma(8\gamma_E-27)\ln(2G_NM\mu)\\
   +2\alpha\ln(2G_NM)\left(c_3+2\gamma\ln(2GM\mu)\right)\Big]+96\Big[c_1c_3+2c_1\gamma\ln(2G_NM\mu)+2c_3\alpha\ln(2G_NM\mu)\\
   +4\alpha\gamma\ln^2(2G_NM\mu)\Big]+2\Big[c_2\gamma(32\gamma_E-58)+4c_3\beta (8\gamma_E-33)+4\beta\gamma(32\gamma_E-95)\ln(2G_NM\mu)\\
   +12\beta\ln(2G_NM)(c_3+2\ln(2GM\mu))\Big]+32\Big[c_2c_3+2c_2\gamma\ln(2G_NM\mu)+2c_3\beta\ln(2G_NM\mu)\\
   +4\beta\gamma\ln^2(2GM\mu)\Big]+6\Big[c^2_2+4c_2\beta\ln(2G_NM\mu)+4\beta^2\ln^2(2G_NM\mu)\Big]\\
   +3\beta\Big[8\gamma_E-27+2\ln(2G_NM)\Big]\Big[c_2+2\beta\ln(2G_NM\mu)\Big]+32\Big[c^2_3+4c_3\gamma\ln(2G_NM\mu)\\
   +4\gamma^2\ln(2G_NM\mu)\Big]+32(4\gamma_E-7)\gamma\Big[c_3+2\gamma\ln(2G_NM\mu)\Big]+6\alpha\beta\Big[163+16\gamma_E(\gamma_E-18)\\
   +8(\gamma_E-2)\ln(2G_NM)\Big]+24\alpha\gamma\Big[169+4\gamma_E(4\gamma_E-27)+8(\gamma_E-2)\ln(2G_NM)\Big]\\
   +2\beta\gamma\Big[555+4\gamma_E(16\gamma_E-95)+24(\gamma_E-2)\ln(2G_NM)\Big]-72\alpha^2\\
   +\beta^2\Big[249+6\gamma_E(4\gamma_E-27)+12(\gamma_E-2)\ln(2G_NM)\Big]+64\gamma^2(\gamma_E-2)(2\gamma_E-3)\bigg\}\\
   +\frac{65536\pi^5Q^2\rho}{M^4}(4\alpha+\beta)\bigg\{2\Big[c_2\gamma(8\gamma_E-17)+8c_3\beta(\gamma_E-2)+2\beta\gamma(16\gamma_E-33)\ln(2G_NM\mu)\Big]\\
   +8\Big[c_2c_3+2c_2\gamma\ln(2G_NM\mu)+2c_3\beta\ln(2G_NM\mu)+4\beta\gamma\ln^2(2G_NM\mu)\Big]+8\Big[c_3\gamma(16\gamma_E-35)\\
   +2\gamma^2(16\gamma_E-35)\ln(2G_NM\mu)\Big]+32\Big[c^2_3+4c_3\gamma\ln(2G_NM\mu)+4\gamma^2\ln^2(2G_NM\mu)\Big]\\
   +(6\alpha+\beta)\Big[c_2+2\beta\ln(2G_NM\mu)\Big]+24\alpha\Big[c_3+2\gamma\ln(2G_NM\mu)\Big]+12\alpha\beta(\gamma_E-2)\\
   +4\beta\gamma(\gamma_E-2)(8\gamma_E-17)+48\alpha\gamma(\gamma_E-2)
   +2\beta^2(\gamma_E-2)+16\gamma^2(\gamma_E-2)(8\gamma_E-19)\bigg\}.
\end{multline}
Inserting the explicit scale dependence of the coefficients according to \eqref{eq:coefficients_scale}, one can see that
\begin{equation}
    \frac{\partial S_{\text{Wald}}}{\partial \mu}=0,
\end{equation}
i.e. the entropy is RG invariant. The $\mu$-independent entropy is then effectively obtained with the substitutions $\mu\to\mu_*$ and $c_i\to c_i(\mu_*)$ everywhere.

The typical behavior of the entropy of a AdS Reissner-Nordstr\"om black hole as a function of $M$ is shown in Fig. \ref{fig2}. The entropy of a classical Reissner-Nordstr\"om ($\rho=0$) and AdS Reissner-Nordstr\"om black hole (with $\rho=0.05$) is represented, respectively, by a solid red line and a dashed green line. The values of the entropy of gravitational quantum corrected AdS Reissner-Nordstr\"om black holes (with the choice $\rho=0.05$, $c_1=c_2=c_3=0.1$) coupled to a scalar, fermion, vector or graviton field have approximately the same behavior. This becomes more transparent by looking at Fig. \ref{fig2} (b). Furthermore, we see that there exists an upper bound for the black hole mass, past which the entropy becomes negative.  If we set $S_{0}\equiv4\pi G M^2$ and assume $\mu_{*}=2\pi M$, then the corrected entropy \eqref{eq:corrected_entropy} can be simplified as
\begin{equation}\label{model}
S=S_{0}+\alpha_{1}\ln{S_{0}}+\alpha_{2}(\ln{S_{0}})^{2}+\alpha_{3},
\end{equation}
where $\alpha_{i}$ ($i=1,2,3$) are some parameters depending on $G_{N}$, $M$, $\rho$, $Q$ and the quantum coefficients ($\alpha, \beta, \gamma, c_{1}, c_{2}, c_{3}$). The first two terms are leading order corrections to the black hole entropy: they were obtained in several gravitational theories for various black objects \cite{Das:2001ic, Sadeghi:2016dvc, Castro:2018hsc}. The last two terms correspond to higher-order corrections.
\begin{figure}
\begin{center}$
\begin{array}{cccc}
\includegraphics[width=70 mm]{entropy_a.eps}\includegraphics[width=70 mm]{entropy_b.eps}
\end{array}$
\end{center}
\caption{Entropy of AdS RN black holes with $G=Q=\mu_{*}=1$. (a) Solid red: classical RN. Dashed green: classical AdS RN with $\rho=0.05$. For the quantum corrected black holes we set everywhere $\rho=0.05$ and $c_{2}=c_{3}=0.1$. Dotted black: quantum corrected AdS RN with a scalar field ($\alpha=0.0010994, \beta=-0.00001759$, $\gamma=0.00001759$). Dashed orange: quantum corrected AdS RN with a fermionic field ($\alpha=-0.000043976, \beta=0.000070361$, $\gamma=0.000061566$). Dashed cyan: quantum corrected AdS RN with a vector field ($\alpha=-0.00043976, \beta=0.0015480$, $\gamma=-0.00022867$). Dashed dotted blue: quantum corrected AdS RN with a graviton field ($\alpha=0.0037819, \beta=-0.012700$, $\gamma=0.0037292$). The numerical values of $\alpha$, $\beta$, $\gamma$ were taken from \cite{Donoghue:2014yha}. (b) Quantum corrected AdS RN entropy at large scale: all fields give rise to approximately the same behaviour.}
\label{fig2}
\end{figure}
%%%%%%%%%%%%%%%%%%%%%%
%%%%%%%%%%%%%%%%%%%%%%
\section{Thermodynamics}\label{sec:thermodynamics}
In this section we compute the quantum gravitational corrections to relevant thermodynamics quantities, namely temperature, pressure, specific heat and Helmholtz free energy. The first law of thermodynamics for a black hole with mass $M$ and charge $Q$ is given by
\begin{equation}
    dM=TdS-Qd\Phi+PdV,
\end{equation}
where $P$, $T$, $V$, $\Phi$ are the pressure, the temperature, the volume and the electric potential, respectively.
The electric potential is 
\begin{equation}
\begin{split}
\Phi=&\int_{r_h}^{+\infty} dr'\, F_{tr}=\int_{r_h}^{+\infty} dr'\, \left(\frac{Q}{{r'}^2}+G^2_N\Omega(r')\right)=\frac{Q}{2G_N M}\\
&+\frac{128\pi^2GQ\rho}{M}(4\alpha+\beta)\left[1+\ln(2G_NM)\right]+\mathcal{O}\left(Q^3,\rho^2\right).
\end{split}
\end{equation}
The temperature is
\begin{eqnarray}
    T&=&\frac{1}{4\pi}\sqrt{\frac{df(r)}{dr}\frac{dg(r)}{dr}}\bigg\rvert_{r=r_h}\nonumber\\
    &=&\frac{1}{8\pi G_N M}+\frac{8G_N^2M\rho}{3}+\frac{2G_N\rho}{3M}+T_{RN}+T_{AdS}+\mathcal{O}\left(Q^4,\rho^2\right),
\end{eqnarray}
where $T_{RN}$ represents the quantum gravitational corrections to a Reissner-Nordstr\"om black hole and $T_{AdS}$ contains the new corrections proportional to $\rho$. The corresponding expressions are
\begin{equation}
   T_{RN}=\frac{Q^2}{4G^3_NM^5}\Big[2(c_2+4c_3)+(\beta+4\gamma)\left(4\gamma_E-9+4\ln(2G_NM\mu)\right)\Big]
\end{equation}
and
\begin{equation}
\begin{gathered}
    T_{AdS}=\frac{32\pi G_N\rho}{M}(4\alpha+\beta)+\frac{8\pi Q^2\rho}{3M^3}\Big[4(c_2+4c_3)-12\alpha+\beta(8\gamma_E-27)+32\gamma(\gamma_E-3)\\
    +8(\beta+4\gamma)\ln(2G_NM\mu)\Big]+\frac{128\pi^2Q^2\rho}{G_NM^5}(4\alpha+\beta)\Big[17(c_2+4c_3)\\
    +2(\beta+4\gamma)\left(17\gamma_E-38+17\ln(2G_NM\mu)\right)\Big].
\end{gathered}
\end{equation}
The typical behavior of the temperature of a AdS Reissner-Nordstr\"om black hole as a function of $M$ is depicted in Fig. \ref{fig3}. The temperature of a classical Reissner-Nordstr\"om ($\rho=0$) and AdS Reissner-Nordstr\"om black hole (with $\rho=0.05$) is represented, respectively, by a solid red line and a dashed green line. 
In the case of the gravitational quantum corrected AdS Reissner-Nordstr\"om black hole with a scalar field (dotted black line), we see that the temperature is an increasing function of the black hole mass. The same happens for the gravitational quantum corrected AdS Reissner-Nordstr\"om black hole with a graviton field (dashed dotted blue line). On the other hand, in the case of the gravitational quantum corrected AdS Reissner-Nordstr\"om black hole with a fermionic field (dashed orange) we see that the temperature diverges at infinitesimal mass, has a negative minimum, then grows and increases with $M$. The same happens for the gravitational quantum corrected AdS Reissner-Nordstr\"om black hole with a vector field (dashed cyan line). The upshot is that there exists a lower bound for the mass of a quantum AdS Reissner-Nordstr\"om black hole, below which the temperature becomes negative. Combining this result with the previous one obtained from the entropy analysis, we conclude that a quantum corrected black hole exists only within a bounded range of masses, i.e. $M_{min}\leq M\leq M_{max}$.
Moreover, at larger $M$ all curves asymptotically approach the classical AdS Reissner-Nordstr\"om temperature, which shows that quantum effects are important when the black black hole mass is small.
\begin{figure}
\begin{center}$
\begin{array}{cccc}
\includegraphics[width=80 mm]{temperature.eps}
\end{array}$
\end{center}
\caption{Temperature of AdS RN black holes with $G=Q=\mu_{*}=1$. (a) Solid red: classical RN. Dashed green: classical AdS RN with $\rho=0.05$. For the quantum corrected black holes we set everywhere $\rho=0.05$ and $c_{2}=c_{3}=0.1$. Dotted black: quantum corrected AdS RN with a scalar field ($\alpha=0.0010994, \beta=-0.00001759$, $\gamma=0.00001759$). Dashed orange: quantum corrected AdS RN with a fermionic field ($\alpha=-0.000043976, \beta=0.000070361$, $\gamma=0.000061566$). Dashed cyan: quantum corrected AdS RN with a vector field ($\alpha=-0.00043976, \beta=0.0015480$, $\gamma=-0.00022867$). Dashed dotted blue: quantum corrected AdS RN with a graviton field ($\alpha=0.0037819, \beta=-0.012700$, $\gamma=0.0037292$). The numerical values of $\alpha$, $\beta$, $\gamma$ were taken from \cite{Donoghue:2014yha}. (b) Quantum corrected AdS RN entropy at large scale: all fields give rise to approximately the same behaviour. }
\label{fig3}
\end{figure}

Quantum effects also produce corrections to the pressure \cite{Calmet:2021lny, Delgado:2022pcc}. For a charged AdS$_4$ black hole, the pressure is
\begin{equation}
    P=-\frac{T\frac{dS}{dM}-Q\frac{d\Phi}{dM}-1}{\frac{dV}{dM}}=-\frac{Q^2}{64\pi G^4_N M^4}-\frac{2}{3}\rho+P_{sch}+P_{RN}+P_{AdS}+\mathcal{O}\left(Q^4,\rho^2\right),
\end{equation}
where $V=4/3\pi r^3_h$ is the volume, $P_{sch}=-\gamma/(G^4_NM^4)$ is the quantum correction for a Schwarzschild black hole, $P_{RN}$ corresponds to the corrections for a Reissner-Nordstr\"om black hole and $P_{AdS}$ represents the new contributions containing the AdS$_4$ radius.
The expression of $P_{RN}$ is \cite{Delgado:2022pcc}
\begin{multline}
    P_{RN}=\frac{Q^2}{32G^5_NM^6}\Big[c_2+4c_3+2\beta(\gamma_E-4)+8\gamma(\gamma_E-5)+2(\beta+4\gamma)\ln(2G_NM\mu)\Big]
    \\+\frac{\pi Q^2}{9G^6_NM^8}\bigg\{54(\beta+4\gamma)\Big[c_1+2\alpha\ln\left(2G_NM\mu\right)\Big]
    +72(3\gamma_E-7)\beta\Big[c_2+2\beta\ln\left(2G_NM\mu\right)\Big]\\
    +768(3\gamma_E-7)\gamma\Big[c_3+2\gamma\ln\left(2G_NM\mu\right)\Big]
    +6\Big[c_2\gamma(120\gamma_E-287)+3c_3\beta(40\gamma_E-91)\\
    +160\beta\gamma(3\gamma_E-7)\ln\left(2G_NM\mu\right)\Big]
     +54\Big[c^2_2+4c_2\beta\ln\left(2G_NM\mu\right)+4\beta^2\ln^2\left(2G_NM\mu\right)\Big]\\
   +576\Big[c^2_3+4c_3\gamma\ln\left(2G_NM\mu\right)+4\gamma^2\ln^2\left(2G_NM\mu\right)\Big]
   +360\Big[c_2c_3+2c_2\gamma\ln\left(2G_NM\mu\right)\\+2c_3\beta\ln\left(2G_NM\mu\right)
    +4\beta\gamma\ln^2\left(2G_NM\mu\right)\Big]+(\beta+4\gamma)\Big[36\alpha(3\gamma_E-7) \\+9\beta\big(8\gamma_E(3\gamma_E-14)+95+4\pi^2\big)
    +\gamma\big(192\gamma_E(3\gamma_E-14)+2095+120\pi^2\big)\Big]
    \bigg\}.
    \end{multline}
The expression of $P_{AdS}$ is
\begin{multline}
    P_{AdS}=-\frac{8\pi\rho}{3G_NM^2}(12\alpha+3\beta+4\gamma)+\frac{64\pi^2\rho}{G_N^2M^4}(4\alpha+\beta)(6\alpha+\beta-4\gamma)
    +\frac{2\pi Q^2\rho}{3G_N^2M^4}\Big[5(c_2+4c_3)-60\alpha\\+\beta(10\gamma_E-49)+8\gamma(5\gamma_E-18)-6(4\alpha+\beta)\ln(2G_NM)+10(\beta+4\gamma)\ln(2G_NM\mu)\Big]\\
    -\frac{8\pi^2Q^2\rho}{27G_N^3M^6}\bigg\{108\Big[4c_1\beta(4\gamma_E-11)+c_2\alpha(16\gamma_E-157)+2\alpha\beta(32\gamma_E-201)\ln(2G_NM\mu)\\
    +8\alpha\ln(2G_NM)\left(c_2+2\beta\ln(2G_NM\mu)\right)\Big]+864\Big[c_1c_2+2c_1\beta\ln(2G_NM\mu)+2c_2\alpha\ln(2G_NM\mu)\\
    +4\alpha\beta\ln^2(2G_NM\mu)\Big]+432\Big[4c_1\gamma(4\gamma_E-11)+c_3\alpha(16\gamma_E-157)+2\alpha\gamma(32\gamma_E-201)\ln(2G_NM\mu)\\
    +8\alpha\ln(2G_NM)\left(c_3+2\gamma\ln(2G_NM\mu)\right)\Big]+3456\Big[c_1c_3+2c_1\gamma\ln(2G_NM\mu)+2c_3\alpha\ln(2G_NM\mu)\\
    +4\alpha\gamma\ln^2(2G_NM\mu)\Big]-12\Big[4c_2\gamma(72\gamma_E-167)+3c_3\beta(96\gamma_E-151)+2\beta\gamma(576\gamma_E-215)\ln(2G_NM\mu)\\
    -72\beta\ln(2G_NM)\left(c_3+2\gamma\ln(2G_NM\mu)\right)\Big]-1728\Big[c_2c_3+2c_2\gamma\ln(2G_NM\mu)+2c_3\beta\ln(2G_NM\mu)\\
    +4\beta\gamma\ln^2(2G_NM\mu)\Big]-216\Big[c^2_2+4c_2\beta\ln(2G_NM\mu)+4\beta^2\ln^2(2G_NM\mu)\Big]\\
    +\beta\Big[-9(96\gamma_E+143)+216\ln(2G_NM)\Big]\Big[c_2+2\beta\ln(2G_NM\mu)\Big]-3456\Big[c^2_3+4c_3\gamma\ln(2G_NM\mu)\\
    +4\gamma^2\ln^2(2G_NM\mu)\Big]-192\gamma(72\gamma_E-161)\Big[c_3+2\gamma\ln(2G_NM\mu)\Big]+36\alpha\beta\Big[2309+6\gamma_E(16\gamma_E-201)\\
    24(2\gamma_E-5)\ln(2G_NM)\Big]+18\alpha\gamma\Big[18977+48\gamma_E(16\gamma_E-201)+192(2\gamma_E-5)\ln(2G_NM)\Big]\\
    +8\beta\gamma\Big[5834-264\pi^2-3\gamma_E(288\gamma_E-215)+108(2\gamma_E-5)\ln(2G_NM)\Big]-3888\alpha^2\\
    -18\beta^2\Big[-760+16\pi^2+\gamma_E(48\gamma_E+143)+12(2\gamma_E-5)\ln(2G_NM)\Big]\\
    -128\gamma^2\Big[292+30\pi^2+3\gamma_E(36\gamma_E-161)\Big]\bigg\}+\frac{128\pi^3Q^2\rho}{9G_N^4M^8}(4\alpha+\beta)\bigg\{108\Big[c_1\beta-2c_2\alpha\\
    -2\alpha\beta\ln(2G_NM\mu)\Big]+432\Big[c_1\gamma-2c_3\alpha-2\alpha\gamma\ln(2G_NM\mu)\Big]+144\Big[c_2c_3+2c_2\gamma\ln(2G_NM\mu)\\
    +2c_3\beta\ln(2G_NM\mu)+4\beta\gamma\ln^2(2G_NM\mu)\Big]+12\Big[c_2\gamma(24\gamma_E-35)+c_3\beta(24\gamma_E-45)\\
    +32\beta\gamma(3\gamma_E-5)\ln(2G_NM\mu)\Big]+36\beta(12\gamma_E-29)\Big[c_2+2\beta\ln(2G_NM\mu)\Big]+108\Big[c_2^2\\
    +4c_2\beta\ln(2G_NM\mu)+4\beta^2\ln^2(2G_NM\mu)\Big]-192\gamma(24\gamma_E-67)\Big[c_3+2\gamma\ln(2G_NM\mu)\Big]\\
    -1152\Big[c^2_3+4\gamma\ln(2G_NM\mu)+4\gamma^2\ln^2(2G_NM\mu)\Big]-18\alpha\beta(12\gamma_E-17)\\-72\alpha\gamma(12\gamma_E-17)
    +2\beta\gamma\Big[-1271+264\pi^2+96\gamma_E(3\gamma_E-10)\Big]+9\beta^2\Big[205+8\pi^2\\+8\gamma_E(6\gamma_E-29)\Big]
    -8\gamma^2\Big[4961-120\pi^2+48\gamma_E(12\gamma_E-67)\Big]\bigg\}.
\end{multline}
Plugging the explicit scale dependence \eqref{eq:coefficients_scale} it is easy to verify that the pressure is RG invariant,
\begin{equation}
    \frac{\partial P}{\partial \mu}=0.
\end{equation}
Next, we compute the specific heat which is an important quantity related to the thermodynamics stability. The black hole is stable if $C\geq0$ (see Fig. \ref{fig4}). One obtains
\begin{equation}
    C=\frac{1}{M}\frac{TdS}{dT}=\frac{T}{M}\frac{1}{\frac{dT}{dM}}\frac{dS}{dM}=-8\pi GM+C_{Sch}+C_{RN}+C_{AdS}+\mathcal{O}\left(Q^4,\rho^2\right),
\end{equation}
where
\begin{equation}
   C_{Sch}=-\frac{128\pi^2\gamma}{M}
\end{equation}
is the quantum contribution to a Schwarzschild black hole, while
\begin{multline}
   C_{RN}=\frac{8\pi^2Q^2}{G_NM^3}\Big[21(c_2+4c_3)+2\beta(21\gamma_E-53)+4\gamma(42\gamma_E-107)+42(\beta+4\gamma)\ln(2GM\mu)\Big]\\
   +\frac{256\pi^3}{9G^2M^5}\bigg\{54\beta\Big[c_1+2\alpha\ln(2G_NM\mu)\Big]+216\gamma\Big[c_1+2\alpha\ln(2G_NM\mu)\Big]+6\Big[c_2\gamma(120\gamma_E-269)\\
   3c_3\beta(40\gamma_E-91)+4\beta\gamma(120\gamma_E+271)\ln(2G_NM\mu)\Big]+360\Big[c_2c_3+2c_2\gamma\ln(2G_NM\mu)\\
   +2c_3\beta\ln(2G_NM\mu)+4\beta\gamma\ln^2(2G_NM\mu)\Big]+72(3\gamma_E-7)\beta\Big[c_2+2\beta\ln(2G_NM\mu)\Big]\\
   +54\Big[c_2^2+4c_2\beta\ln(2G_NM\mu)+4\beta^2\ln^2(2G_NM\mu)\Big]+48(48\gamma_E-103)\gamma\Big[c_3+2\gamma\ln(2G_NM\mu)\Big]\\
   576\Big[c^2_3+4c_3\gamma\ln(2G_NM\mu)+4\gamma^2\ln^2(2G_NM\mu)\Big]+36\alpha\beta(3\gamma_E-7)\\+144\alpha\gamma(3\gamma_E-7)
   +2\beta\gamma\Big[2483+132\pi^2+12\gamma_E(60\gamma_E-271)\Big]+9\beta^2\Big[95+4\pi^2\\+8\gamma_E(3\gamma_E-14)\Big]
   +8\gamma^2\Big[773+60\pi^2+12\gamma_E(24\gamma_E-103)\Big]\bigg\}
\end{multline}
represent the quantum correction to a Reissner-Nordstr\"om black hole. Finally, the purely AdS part contributes as
\begin{figure}
\begin{center}$
\begin{array}{cccc}
\includegraphics[width=75 mm]{specific_heat.eps}
\end{array}$
\end{center}
\caption{Specific heat of AdS RN black holes with $G=Q=\mu_{*}=1$ and $M=2$. Solid red: classical RN. Dashed green: classical AdS RN with $\rho=0.05$. For the quantum corrected metric we set everywhere $\rho=0.05$ and $c_1=c_{2}=c_{3}=0.1$. Dotted black: quantum corrected AdS RN with a scalar field ($\alpha=0.0010994, \beta=-0.00001759$, $\gamma=0.00001759$). Dashed orange: quantum corrected AdS RN with a fermionic field ($\alpha=-0.000043976, \beta=0.000070361$, $\gamma=0.000061566$). Dashed cyan: quantum corrected AdS RN with a vector field ($\alpha=-0.00043976, \beta=0.0015480$, $\gamma=-0.00022867$). Dashed dotted blue: quantum corrected AdS RN with a graviton field ($\alpha=0.0037819, \beta=-0.012700$, $\gamma=0.0037292$). The numerical values for $\alpha$, $\beta$ and $\gamma$ were taken from \cite{Donoghue:2014yha}.}
\label{fig4}
\end{figure}
\begin{multline}
    C_{AdS}=\frac{16384\pi^4G^2_N}{M}(4\alpha+\beta)(6\alpha+\beta-2\gamma)+\frac{512\pi^3G^2_NQ^2\rho}{3M}\Big[58(c_2+4c_3)-12\alpha\\
    +(\beta+4\gamma)\left(116\gamma_E-283+116\ln(2G_NM\mu)\right)\Big]-\frac{16384\pi^4G_NQ^2\rho}{27M^3}\bigg\{108\Big[2c_1\beta(\gamma_E-3)\\
    +c_2\alpha(2\gamma_E+25)+2\alpha\beta(4\gamma_E+19)\ln(2G_NM\mu)+\alpha\ln(2G_NM)\left(c_2+2\beta\ln(2G_NM\mu)\right)\Big]
    \\+108\Big[c_1c_2+2c_1\beta\ln(2G_NM\mu)+2c_2\alpha\ln(2G_NM\mu)+4\alpha\beta\ln^2(2G_NM\mu)\Big]+432\Big[2c_1\gamma(\gamma_E-3)\\
    +c_3\alpha(2\gamma_E+25)+2\alpha\gamma(4\gamma_E+19)\ln(2G_NM\mu)
    +\alpha\ln(2G_NM)\left(c_3+2\gamma\ln(2G_NM\mu)\right)\Big]\\
    432\Big[c_1c_3+2c_1\gamma\ln(2G_NM\mu)+2c_3\alpha\ln(2G_NM\mu)+4\alpha\gamma\ln^2(2G_NM\mu)\Big]-6\Big[c_2\gamma(192\gamma_E-415)\\
    +3c_3\beta(64\gamma_E-321)+4\beta\gamma(192\gamma_E-689)\ln(2G_NM\mu)+18\beta\ln(2G_NM)\left(c_3+2\gamma\ln(2G_NM\mu)\right)\Big]\\
    -576\Big[c_2c_3+2c_2\gamma\ln(2G_NM\mu)+2c_3\beta\ln(2G_NM\mu)+4\beta\gamma\ln^2(2G_NM\mu)\Big]\\
    +9\beta\Big[c_2+2\beta\ln(2G_NM\mu)\Big]\Big[3\ln(2G_NM)-4(9\gamma_E-43)\Big]-81\Big[c^2_2+4c_2\beta\ln(2G_NM\mu)\\
    +4\beta^2\ln^2(2G_NM\mu)\Big]-48\gamma(84\gamma_E-173)\Big[c_3+2\gamma\ln(2G_NM\mu)\Big]-1008\Big[c^2_3\\
    +4c_3\gamma\ln(2G_NM\mu)+4\gamma^2(2G_NM\mu)\Big]+9\alpha\beta\Big[-1417+24\gamma_E(2\gamma_E+19)+12(2\gamma_E-5)\ln(2G_NM)\Big]\\
    +72\alpha\gamma\Big[-701+864(2\gamma_E+19)+6(2\gamma_E-5)\ln(2G_NM)\Big]-4\beta\gamma\Big[5525+132\pi^2+6\gamma_E(96\gamma_E-689)\\
    -27(2\gamma_E-5)\ln(2G_NM)\Big]+9\beta^2\Big[-4\left(139+2\pi^2+\gamma_E(9\gamma_E-86)\right)\\
    +3(2\gamma_E-5)\ln(2G_NM)\Big]-324\alpha^2-8\gamma^2\Big[1069+120\pi^2+12\gamma_E(42\gamma_E-173)\Big]\bigg\}\\
    -\frac{131072\pi^5Q^2\rho}{M^5}(4\alpha+\beta)\bigg\{24\alpha\Big[c_2+2\beta\ln(2G_NM\mu)\Big]+96\alpha\Big[c_3+2\gamma\ln(2G_NM\mu)\Big]\\
    +8\Big[c_2\gamma(4\gamma_E-1)+4c_3\beta(\gamma_E-2)+2\beta\gamma(8\gamma_E-9)\ln(2G_NM\mu)\Big]+16\Big[c_2c_3+2c_2\gamma\ln(2G_NM\mu)\\
    +2c_3\beta\ln(2G_NM\mu)+4\beta\gamma\ln^2(2G_NM\mu)\Big]+4\beta\Big[c_2+2\beta\ln(2G_NM\mu)\Big]\\
    +32\gamma(8\gamma_E-11)\Big[c_3+2\gamma\ln(2G_NM\mu)\Big]+64\Big[c^2_3+4c_3\gamma\ln(2G_NM\mu)\\+4\gamma^2\ln^2(2G_NM\mu)\Big]
    +6\alpha\beta(8\gamma_E-19)+24\alpha\gamma(8\gamma_E-19)\\+8\beta\gamma(2\gamma_E-5)(4\gamma_E+1)
    +\beta^2(8\gamma_E-19)+16\gamma^2\Big[9+4\gamma_E(4\gamma_E-11)\Big]\bigg\}.
\end{multline}
In Fig. \ref{fig4} we provide a graphical analysis of the specific heat. In the case of a classical Reissner-Nordstr\"om black hole there exists an unstable/stable phase transition (see the solid red line). For large $M$, the specific heat is negative. Decreasing the mass due to the Hawking radiation, the specific heat reaches a point where it is positive, meaning that the black hole becomes stable. The case of a classical AdS Reissner-Nordstr\"om black hole (depicted with the dashed green line) is opposite. Here, the black hole is in the stable phase at large $M$, but becomes unstable for small mass. The latter form of instability also happens for the quantum corrected cases with the scalar and graviton fields. On the other hand, fermionic and vector fields lead to a stable phase for small mass. It means that a stable black remnant may form due to the quantum corrections.

Last but not least, the Helmholtz free energy is
\begin{equation}
    F=E-TS=M+Q\Phi-TS=\frac{M}{2}+\frac{3Q^2}{4G_NM}+F_{Sch}+F_{RN}+F_{AdS}+\mathcal{O}\left(Q^3,\rho^2\right),
\end{equation}
\begin{equation}
\begin{split}
    F_{Sch}=-\frac{8\pi}{G_NM}\Big[c_3+2\gamma\left(\ln(2G_NM\mu)+\gamma_E-1\right)\Big]
\end{split}
\end{equation}
and
\begin{multline}
F_{RN}=-\frac{\pi Q^2}{2G^2_NM^3}\Big[9(c_2+4c_3)+3\beta(6\gamma_E-13)+8\gamma(9\gamma_E-20)+18(\beta+4\gamma)\ln(2G_NM\mu)\Big]\\
+\frac{8\pi^2Q^2}{9G^3_NM^5}\bigg\{54\beta\Big[c_1+2\alpha\ln(2G_NM\mu)\big]+216\gamma\Big[c_1+2\alpha\ln(2G_NM\mu)\Big]+6\Big[c_2\gamma(132\gamma_E-263)\\
+6c_3\beta(22\gamma_E-45)+2\beta\gamma(264\gamma_E-533)\ln(2G_NM\mu)\Big]+396\Big[c_2c_3+2c_2\gamma\ln(2G_NM\mu)\\
+2c_3\beta\ln(2G_NM\mu)+4\beta\gamma\ln^2(2G_NM\mu)\Big]+6\beta(36\gamma_E-75)\Big[c_2+2\beta\ln(2G_NM\mu)\Big]\\
+54\Big[c_2^2+4c_2\beta\ln(2G_NM\mu)+4\beta^2\ln(2G_NM\mu)\Big]+24\gamma(120\gamma_E-233)\Big[c_3+2\gamma\ln(2G_NM\mu)\Big]\\
+720\Big[c^2_3+4c_3\gamma\ln(2G_NM\mu)+4\gamma^2\ln^2(2G_NM\mu)\Big]
+9\alpha\beta(12\gamma_E-25)\\
+36\alpha\gamma(12\gamma_E-25)+4\beta\gamma\Big[1042+66\pi^2+3\gamma_E(132\gamma_E-533)\Big]+18\beta^2\Big[35+2\pi^2\\+2\gamma(6\gamma_E-25)\Big]
+16\gamma^2\Big[412+30\pi^2+3\gamma_E(60\gamma_E-233)\Big]\bigg\},
\end{multline}
and
\begin{multline}
F_{AdS}=-\frac{128\pi^2G^2_NM\rho}{3}\Big[4c_3+8\gamma\ln(2G_NM\mu)+3(4\alpha+\beta)+8\gamma(\gamma_E-1)\Big]\\
+\frac{1024\pi^3G_N\rho}{M}(4\alpha+\beta)\Big[6c_1+c_2-4c_3+2(6\alpha+\beta-4\gamma)\ln(2G_NM\mu)+2\gamma_E(6\alpha+\beta-4\gamma)\Big]\\
+8\pi G^2M Q^2\rho-\frac{32\pi^2 G Q^2\rho}{3M}\Big[3(8c_1+5c_2+12c_3)+48\alpha(\gamma_E-2)+\beta(30\gamma_E-61)\\
+\gamma(18\gamma_E-37)-6(4\alpha+\beta)\ln(2G_NM)+6(8\alpha+5\beta+12\gamma)\ln(2G_NM\mu)\Big]\\
+\frac{128\pi^3Q^2\rho}{27M^3}\bigg\{108\Big[2c_1\beta(8\gamma_E-17)+c_2\alpha(16\gamma_E-13)+2\alpha\beta(32\gamma_E-47)\ln(2G_NM\mu)\\
+8\alpha\ln(2G_NM)\left(c_2+2\beta\ln(2G_NM\mu)\right)\Big]+864\Big[c_1c_2+2c_1\beta\ln(2G_NM\mu)+2c_2\alpha\ln(2G_NM\mu)\\
+4\alpha\beta\ln^2(2G_NM\mu)\Big]+864\Big[(8\gamma_E-17)(c_1\gamma+c_3\alpha)+16\alpha\gamma(2\gamma_E-3)\ln(2G_NM\mu)\\
+4\alpha\ln(2G_NM)\left(c_3+2\gamma\ln(2G_NM\mu)\right)\Big]+3456\Big[c_1c_3+2c_1\gamma\ln(2G_NM\mu)+2c_3\alpha\ln(2G_NM\mu)\\
+4\alpha\gamma\ln^2(2G_NM\mu)\Big]-24\Big[c_2\gamma(12\gamma_E-65)+6c_3\beta(2\gamma_E-15)+2\beta\gamma(24\gamma_E+55)\ln(2G_NM\mu)\\
-36\beta\ln(2G_NM)\left(c_3+2\gamma\ln(2G_NM\mu)\right)\Big]-144\Big[c_2c_3+2c_2\gamma\ln(2G_NM\mu)+2c_3\beta\ln(2G_NM\mu)\\
+4\beta\gamma\ln^2(2G_NM\mu)\Big]+\beta\Big[585+216\ln(2G_NM)\Big]\Big[c_2+2\beta\ln(2G_NM\mu)\Big]\\
-192\gamma(12\gamma_E-31)\Big[c_3+2\gamma\ln(2G_NM\mu)\Big]-576\Big[c^2_3+4c_3\gamma\ln(2G_NM\mu)+4\gamma^2\ln^2(2G_NM\mu)\Big]\\
+72\alpha\beta\Big[77+3\gamma_E(16\gamma_E-47)+24(\gamma_E-2)\ln(2G_NM)\Big]+144\alpha\gamma\Big[199+96\gamma_E(\gamma_E-3)\\
+48(\gamma_E-2)\ln(2G_NM)\Big]-8\Big[491+132\pi^2+6\gamma_E(12\gamma_E-155)+216(\gamma_E-2)\ln(2G_NM)\Big]\\
+9\beta^2\Big[130\gamma_E-133-16\pi^2+48(\gamma_E-2)\ln(2G_NM)\Big]-64\gamma^2\Big[7+30\pi^2+6\gamma_E(6\gamma_E-31)\Big]\bigg\}\\
+\frac{2048\pi^4Q^2\rho}{9G_NM^5}(4\alpha+\beta)\bigg\{108\Big[c_1\beta(2\gamma_E-5)+2c_2\alpha(\gamma_E+1)+2(4\gamma_E-3)\ln(2G_NM\mu)\Big]\\
+108\Big[c_1c_2+2c_1\beta\ln(2G_NM\mu)+c_2\alpha\ln(2G_NM\mu)+4\alpha\beta\ln^2(2G_NM\mu)+432\Big[c_1\gamma(2\gamma_E-5)\\
+2c_3\alpha(\gamma_E+1)+2(4\gamma_E-3)\ln(2G_NM\mu)\Big]+432\Big[c_1c_3+2c_1\gamma\ln(2G_NM\mu)\\
+2c_3\alpha\ln(2G_NM\mu)+4\alpha\gamma\ln^2(2G_NM\mu)\Big]+6\Big[c_2\gamma(192\gamma_E-211)+6c_3\beta(32\gamma_E-63)\\
+2\beta\gamma(384\gamma_E-589)\ln(2G_NM\mu)\Big]+576\Big[c_2c_3+2c_2\gamma\ln(2G_NM\mu)+2c_3\beta\ln(2G_NM\mu)\\
+4\beta\gamma\ln^2(2G_NM\mu)\Big]-9\beta(16\gamma_E-45)\Big[c_2+2\beta\ln(2G_NM\mu)\Big]-36\Big[c^2_2+4c_2\beta\ln(2G_NM\mu)\\
+4\beta^2\ln(2G_NM\mu)\Big]+24\gamma(480\gamma_E-859)\Big[c_3+2\ln(2G_NM\mu)\Big]+2880\Big[c^2_3\\+4c_3\gamma\ln(2G_NM\mu)
+4\gamma^2\ln^2(2G_NM\mu)\Big]+9\alpha\beta\Big[-71+24\gamma_E(2\gamma_E-3)\Big]+36\alpha\gamma\Big[-71\\+24\gamma_E(2\gamma_E-3)\Big]+4\beta\gamma\Big[1631-66\pi^2
+3\gamma_E(192\gamma_E-589)\Big]-18\beta^2\Big[43+2\pi^2\\+\gamma_E(8\gamma_E-45)\Big]
+16\gamma^2\Big[2405-30\pi^2+3\gamma_E(240\gamma_E-859)\Big]\bigg\}.
\end{multline}
Again, one can verify that both specific heat and Helmholtz free energy are RG invariant:
\begin{equation}
    \frac{\partial C}{\partial \mu}=0, \hspace{5mm} \frac{\partial F}{\partial \mu}=0.
\end{equation}
%%%%%%%%%%%%%%%%%%%%%%%
%%%%%%%%%%%%%%%%%%%%%%%
\section{Conclusions}\label{sec:conclusions}
In this paper we considered a charged AdS black hole in four dimensions and calculated the quantum gravitational corrections to the black hole geometry. We obtained the modified black hole metric and provided a numerical and analytical horizon structure analysis. We found that, even though classically there exist two distinct horizons, a quantum corrected AdS RN black hole has only one horizon. 
Using the quantum gravitational corrected horizon radius and applying the Wald entropy formula, we obtained the quantum corrected, RG invariant entropy of the black hole. 
We showed that our results confirm the logarithmic corrections of the black hole entropy previously obtained in the literature. 
Finally, we computed the quantum gravitational corrections to several thermodynamic quantities: temperature, pressure, specific heat and Helmholtz free energy. All the quantities are RG invariant and satisfy the first law of thermodynamics. We found that a quantum AdS RN black can exists only for a bounded range of masses. Moreover, we showed that a second order phase transition can happen in presence of quantum corrections. 

There are different future directions that one could pursue. An important question to answer is the following. How do the quantum corrections that we found fit into the framework of the AdS/CFT correspondence? Is it possible to define a theory of gravity on a spacetime defined by the quantum corrected AdS RN metric such that it is dual to some conformal field theory? Furthermore, it was shown that a charged AdS black hole behaves like the Van der Waals system \cite{Kubiznak:2012wp}. It would be interesting to see whether there is such a holographic dual behavior in presence of quantum corrections or not.
%%%%%%%%%%%%%%%%%%%
%%%%%%%%%%%%%%%%%%%
\appendix
\section{Action of \texorpdfstring{$\ln\left(\frac{\Box}{\mu^2}\right)$}{TEXT} on radial functions}\label{sec:appendix}
We show the result of the action of $\ln\left(\Box/\mu^2\right)$ on some radial functions. Let us start with the constant function. We regularize the integrals in \eqref{eq:formula_actionlog} as
\begin{equation}
\begin{split}
    \ln\left(\frac{\Box}{\mu^2}\right)1&=\lim_{\epsilon\to 0}\bigg[\frac{1}{r}\int_0^{1/\sqrt{\epsilon/r^3}}dr'\,\frac{r'}{r+r'}- \frac{1}{r}\int_0^{r-\sqrt{\epsilon r}}dr'\,\frac{r'}{r-r'}\\
    &-\frac{1}{r}\int_{r+\sqrt{\epsilon r}}^{+\infty}dr'\,\frac{r'}{r'-r}
    -2\gamma_E-2\ln\left(\mu\epsilon\right)\bigg],
\end{split}
\end{equation}
where we have added the $r$ in the integral limits for dimensional reasons (they must have the dimension of a length). The divergences cancel out:
\begin{equation}
\begin{gathered}
     \ln\left(\frac{\Box}{\mu^2}\right)1=\lim_{\epsilon\to 0}\Big[2+\ln(\epsilon r)-\ln\left(-r+\sqrt{r^3/\epsilon}\right)-\ln\left(r+\sqrt{r^3/\epsilon}\right)\\-2\gamma_E-2\ln(\mu\epsilon)\Big]=-2(\ln(\mu r)+\gamma_E-1).
\end{gathered}
\end{equation}
Let us consider now $f(r)=1/r$. We regularize the integrals in \eqref{eq:formula_actionlog} as
\begin{equation}
\begin{split}
    \ln\left(\frac{\Box}{\mu^2}\right)\frac{1}{r}=&\lim_{M\to\infty}\lim_{\epsilon\to0}\bigg\{\frac{1}{r}\int_0^M dr^{'}\frac{r'}{r+r'}\frac{1}{r'}-\frac{1}{r}\int_0^{r-\epsilon}dr'\frac{r'}{r-r'}\frac{1}{r'}\\
    &-\frac{1}{r}\int_{r+\epsilon}^M dr'\frac{r'}{r'-r}\frac{1}{r'}-\frac{2}{r}\left[\gamma_E+\ln\left(\mu\epsilon\right)\right]\bigg\}.
\end{split}
\end{equation}
Again, the divergences cancel:
\begin{equation}
\begin{gathered}
\ln\left(\frac{\Box}{\mu^2}\right)\frac{1}{r}=\lim_{M\to\infty}\lim_{\epsilon\to0}\frac{1}{r}\bigg[\ln\left(\frac{M}{r}\right)-\ln\left(\frac{M}{\epsilon}\right)-\ln\left(\frac{r}{\epsilon}\right)\\
-2\gamma_E-2\ln(\mu\epsilon)\bigg]=-\frac{2}{r}\left(\ln(\mu r)+\gamma_E\right).
\end{gathered}
\end{equation}
Analogously,
\begin{equation}
\begin{split}
    \ln\left(\frac{\Box}{\mu^2}\right)\frac{1}{r^2}=&\text{Re}\lim_{\epsilon\to0}\bigg\{\frac{1}{r}\int_0^\infty dr^{'}\frac{r'}{r+r'}\frac{1}{{(r'+i\epsilon})^2}-\frac{1}{r}\int_0^{r-\epsilon}dr'\frac{r'}{r-r'}\frac{1}{{(r'+i\epsilon)}^2}\\
    &-\frac{1}{r}\int_{r+\epsilon}^\infty dr'\frac{r'}{r'-r}\frac{1}{{r'}^2}-\frac{2}{r^2}\left[\gamma_E+\ln\left(\mu\epsilon\right)\right]\bigg\}=-\frac{2}{r^2}\left(\ln\left(\mu r\right)+\gamma_E\right),
\end{split}
\end{equation}
%%%%%%%%%%%%%%%%%%%%%%%%%%%%%%%%%%%%%%%%%%%%%%%%%%%%%%%%%%%%%%%%%%%%%
%%%%%%%%%%%%%%%%%%%%%%%%%%%%%%%%%%%%%%%%%%%%%%%%%%%%%%%%%%%%%%%%%5
\bibliography{references}
\bibliographystyle{JHEP}


\end{document}
