Our experimental results revealed the potential of using hybrid models. The `hybrid model' experiment showed an increase in performance of all models when prior classification is added, even when the classification is not perfect. The `perfect classification' experiment showed that promising results can be achieved when a perfect classification method can be approached or reached. It also showed that results will keep increasing as the classification becomes more accurate. 
 
U-net seems to have benefited more from the hybrid solution than M-net. This indicates that the M-net variant can better handle the large number of images without an annotated BP region, which can be attributed to the more densely supervised nature of the architecture \cite{Abraham2019}.

The difference between the performance on the non-filtered data and the filtered data increased for the hybrid variant of both U-net and M-net. The incoherent data may have caused errors in the learning process of the models. In the hybrid model, this would happen in both the classification and segmentation models, which would have caused the error to propagate. This propagating error could have been the cause of the increased difference in performance between filtered and non-filtered data.
This difference decreased in the third experiment. Since a perfect classification model was simulated, no images with a fully negative ground truth were sent to the segmentation models. This means that no incoherent data was being fed to the segmentation models when using the non-filtered dataset, since only incoherent negatives were removed in the filtered dataset. This can explain the reduction in performance difference between filtered and non-filtered data in the second experiment.

For the U-net model, the variance decreased when using the hybrid model, indicating a more stable model. Since most images without an annotated mask are filtered out by the CNN, more homogeneous data is fed to the segmentation model. This increases the likelihood for the segmentation model to recognize the BP regions. 

The results of M-net in our first experiment do not correspond with the results presented in \cite{Abraham2019}, with an average DSC of 0.59 in this study, compared to the reported 0.88 in \cite{Abraham2019}. Discrepancies between results can be expected since there are differences in methods. For one, the images used for testing are unknown, since each study created an own test set from the original data. Furthermore, the authors did not provide information about their test methods, making it harder to compare results. 

In future research, improving the classification model performance should be considered as an important step, since the perfect classification experiment showed the potential for much better segmentation performance. Moreover, merging the models into a single `chain' could improve time performance.

