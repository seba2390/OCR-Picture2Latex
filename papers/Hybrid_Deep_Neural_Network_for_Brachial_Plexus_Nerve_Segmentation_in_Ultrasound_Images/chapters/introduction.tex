\begin{figure*}[!tb] 
    \centering
    \subfloat[Classification model (CNN) \label{subfig:ClassModel}]{\includegraphics[width=0.65\linewidth]{figures/CNNModel.png}} \\
    \subfloat[Segmentation model (U-net/M-net)  \label{subfig:SegModel}]{\includegraphics[width=0.85\linewidth]{figures/SegModel.png}}
    \caption{The architecture of (a) CNN for the classification and (b) U-net and M-net for the segmentation.}
    \label{fig:Hybrid_Model} 
\end{figure*}

The brachial plexus (BP) is a part of the cervical nerves which originates from the spinal cord. It is partially located in the neck and partially in the axilla. It is an important part of the nervous system, as it innervates the upper limb. The BP contains cervical nerves C\textsubscript{5}-C\textsubscript{8} and most of the thoracic T\textsubscript{1} nerves, often along with fibers from C\textsubscript{4} and/or T\textsubscript{2} \cite{Marieb}.
The segmentation of BP regions is an important consideration in treatment planning for lung or head-and-neck cancer patients, as radiotherapy used to treat cancer patients can induce brachial plexopathy when the BP is overburdened with radiation energy, causing severe and irreversible effects \cite{Yang2013}.
Furthermore, the BP is used more for regional anesthesia (RA) in clavicular surgeries, in the form of interscalene BP blocks. It can replace general anesthesia, allowing better pain control, decreasing opioid consumption, and reducing recovery time \cite{Pincus2019}, \cite{Banerjee2019}. The classic RA procedure is to inject the anesthesia into the target nerve region rather blindly. This method poses risks as block failure, nerve trauma, and local anesthesia toxicity \cite{Kakade2018}. Ultrasound-guided RA (UGRA) has become a popular method to visualize this procedure. UGRA is often faster, requires fewer attempts, and in some cases provides a better sensory block compared to other RA techniques \cite{Liu2009}.  

In general, tissue segmentation in ultrasound (US) images is a challenging task due to low contrast between background and the tissue, compared to other modalities like MRI and CT scans.
Especially the segmentation of nerve tissue can be challenging due to speckled noise coming with the US modality and the fact that the nerve region does not form a salient structure in the images \cite{Abraham2019}. The need for trained experts, capable of recognizing the region of interest (ROI), limits the applicability of the UGRA procedure. Deep learning algorithms could provide a solution by automating the recognition of the ROI. This would make UGRA better available for more extensive use.




Deep learning methods have shown promising results in image segmentation applications\cite{Wang2019}. Deep learning algorithms, and more specifically convolutional neural networks (CNNs) and U-net are widely used in medical applications such as tumor region identification and metastasis detection \cite{Seetha2018, Kong2017}.
CNNs are suitable to be applied to data that have a grid-like topology such as time-series and images \cite{Goodfellow-et-al-2016}. A typical CNN has a hierarchical architecture that alternates a convolution layer, a rectified linear unit (ReLU) activation function, and a pooling layer to summarize the large input spaces into a lower-dimensional feature space. CNN solutions are among the best-performing systems on pattern recognition systems\cite{Goodfellow-et-al-2016}.

U-net \cite{UNetArticle} is the leading model architecture for medical image segmentation. The model consists of an encoder and a decoder part. The encoder part, the left side of the U-shape, resembles a normal CNN. It consists of subsequent convolution, activation, and pooling layers. The decoder part, the right side of the U-shape, is symmetrical to the encoder part. It consists of upsampling layers, a concatenating layer which adds the feature map of the encoder layer, and subsequent convolutional layers. The U-net architecture has proven to be able to accurately localize ROIs, even when only trained on relatively small training sets \cite{UNetArticle}. Over the years, a variety of adaptations have been made to the U-net structure in attempts to improve its performance on specific datasets. A promising architecture is M-net \cite{MNetArticle}. It uses multiple scaled inputs and multiple outputs coming from the different layers, in order to densely supervise extracted feature maps. 

The goal of this paper is to investigate the effect of implementing a hybrid model in segmenting the BP nerve region from US images. We hypothesize that a combination of CNN and a segmentation model provides more accurate BP segmentation compared to using only a segmentation model.

The rest of this paper is structured as follows: Section \ref{section:methods} presents our hybrid deep neural network architecture. The experimental materials and the procedures are described in Section \ref{section:experiments}. The results of experiments are described and discussed in Sections \ref{section:results} and \ref{section:discussion}. Section \ref{section:conclusion} concludes this paper by summarizing our achievements.

