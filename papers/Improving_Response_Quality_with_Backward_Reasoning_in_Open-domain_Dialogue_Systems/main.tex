\documentclass[sigconf]{acmart}

% \acmSubmissionID{1281}

\AtBeginDocument{%
  \providecommand\BibTeX{{%
    \normalfont B\kern-0.5em{\scshape i\kern-0.25em b}\kern-0.8em\TeX}}}

\copyrightyear{2021}
\acmYear{2021}
\setcopyright{acmlicensed}\acmConference[SIGIR '21]{Proceedings of the 44th International ACM SIGIR Conference on Research and Development in Information Retrieval}{July 11--15, 2021}{Virtual Event, Canada}
\acmBooktitle{Proceedings of the 44th International ACM SIGIR Conference on Research and Development in Information Retrieval (SIGIR '21), July 11--15, 2021, Virtual Event, Canada}
\acmPrice{15.00}
\acmDOI{10.1145/3404835.3463004}
\acmISBN{978-1-4503-8037-9/21/07}

% \settopmatter{printfolios=false}

\usepackage{graphicx} % DO NOT CHANGE THIS
\usepackage{latexsym}
\usepackage{url}
\usepackage{booktabs}       % professional-quality tables
\usepackage{amsfonts}       % blackboard math symbols
\usepackage{nicefrac}       % compact symbols for 1/2, etc.
\usepackage{microtype}      % microtypography
\usepackage{algorithmic}
\usepackage{algorithm}
\usepackage{wrapfig}
\usepackage{lipsum}
\usepackage{xspace}
\usepackage{bm}
%\usepackage[compact]{titlesec}
% For figures
\usepackage{subfigure} 
\usepackage[skip=0pt]{caption}
\usepackage{paralist}
\usepackage{float}
\usepackage{tabularx}
\usepackage{multirow}
\usepackage{diagbox}
\usepackage{tikz}
\usepackage{array, makecell} 
\usepackage{amsmath}
% \usepackage{amssymb}
\usepackage[inline]{enumitem}
\usepackage{acronym}
\usepackage{bm}

\newtheorem{defn}{Definition}
 \newcommand\BibTeX{B{\sc ib}\TeX}
\DeclareMathOperator{\E}{\mathbb{E}}
\DeclareMathOperator*{\argmax}{arg\,max}
\newcommand{\hilight}[1]{\colorbox{yellow}{#1}}
\newcounter{todocnt}
\newcommand{\todo}[1]{\textcolor{blue}{#1}\xspace}
\renewcommand{\todo}[1][$\bullet\bullet\bullet$]{%
{\refstepcounter{todocnt}% 
\textcolor{blue}{$\bullet\bullet\bullet$\ \sf To do (\thetodocnt): #1}}\xspace}
\newcommand{\mdr}[1]{\textcolor{red}{\bf #1}}

%Acronyms list%

\acrodef{RL}{Reinforcement Learning}
\acrodef{DRL}{Deep Reinforcement Learning}
\acrodef{IRL}{Inverse Reinforcement Learning}
\acrodef{SERP}{search engine result page}
\acrodef{IR}{Information Retrieval}
\acrodef{MDP}{Markov Decision Process}
\acrodef{MaxEnt-IRL}{Maximum Entropy Inverse Reinforcement Learning}
\acrodef{DM-IRL}{Distance Minimization Inverse Reinforcement Learning}
\acrodef{MMI}{Maximum Mutual Information} 
\acrodef{DNN}{Deep Neural Networks}
\acrodef{RNN}{Recurrent Neural Networks}
\acrodef{MLP}{Multilayer Perceptron}
\acrodef{GRU}{Gated Recurrent Net}

\looseness=-1


\settopmatter{printacmref=true}

\begin{document}
\fancyhead{}
%%
%% The "title" command has an optional parameter,
%% allowing the author to define a "short title" to be used in page headers.
\title[Improving Response Quality with Backward Reasoning]{Improving Response Quality with Backward Reasoning\\ in Open-domain Dialogue Systems}

%%
%% The "author" command and its associated commands are used to define
%% the authors and their affiliations.
%% Of note is the shared affiliation of the first two authors, and the
%% "authornote" and "authornotemark" commands
%% used to denote shared contribution to the research.
\author{Ziming Li}
\orcid{}
\email{z.li@uva.nl}
\affiliation{%
\institution{University of Amsterdam}
\city{Amsterdam}
\country{The Netherlands}
}

\author{Julia Kiseleva}
\orcid{}
\email{julia.kiseleva@microsoft.com}
\affiliation{%
\institution{Microsoft}
\city{Redmond}
\country{United States}
}

\author{Maarten de Rijke}
\orcid{0000-0002-1086-0202}
\email{m.derijke@uva.nl}
\affiliation{%
\institution{University of Amsterdam \& Ahold Delhaize}
\city{Amsterdam}
\country{The Netherlands}
}

\begin{abstract}
Being able to generate informative and coherent dialogue responses is crucial when designing human-like open-domain dialogue systems. Encoder-decoder-based dialogue models tend to produce generic and dull responses during the decoding step because the most predictable response is likely to be a non-informative response instead of the most suitable one. 
To alleviate this problem, we propose to train the generation model in a bidirectional manner by adding a backward reasoning step to the vanilla encoder-decoder training. The proposed backward reasoning step pushes the model to produce more informative and coherent content because the forward generation step's output is used to infer the dialogue context in the backward direction. The advantage of our method is that the forward generation and backward reasoning steps are trained simultaneously through the use of a latent variable to facilitate bidirectional optimization. Our method can improve response quality without introducing side information (e.g., a pre-trained topic model). 
The proposed bidirectional response generation method achieves state-of-the-art performance for response quality. 
\end{abstract}

\begin{CCSXML}
<ccs2012>
   <concept>
       <concept_id>10002951.10003260.10003282.10003286.10003290</concept_id>
       <concept_desc>Information systems~Chat</concept_desc>
       <concept_significance>500</concept_significance>
       </concept>
   <concept>
       <concept_id>10002951.10003317.10003347.10003348</concept_id>
       <concept_desc>Information systems~Question answering</concept_desc>
       <concept_significance>300</concept_significance>
       </concept>
 </ccs2012>
\end{CCSXML}

\ccsdesc[500]{Information systems~Chat}
\ccsdesc[300]{Information systems~Question answering}

\keywords{Open-domain dialogue system; response generation}

\maketitle

% \input{01-intoduction}
% \section{Related Work}
\subsection{Placeholder Translation}
To ensure that certain words appear in the translated sentence, previous studies have explored the method of replacing certain classes of words with special placeholder tokens and restore the words in a post-processing step, which we call {\it placeholder translation} in this paper.

\citet{luong-etal-2015-addressing} and \citet{long-etal-2016-translation} employed placeholder tokens to improve the translation of rare words or technical terms.
However, simply replacing words with a unique placeholder token loses the information on the original words. To alleviate this problem, subsequent studies distinguish different types of placeholders, such as named entity types \citep{Crego2016SYSTRANsPN,post-etal-2019-exploration} or parts-of-speech \citep{michon-etal-2020-integrating}.

Instead of replacing the placeholder token with a dictionary entry, some studies propose generating the content of the placeholder with a character-level sequence-to-sequence model to translate words not covered in the bilingual dictionary. \citet{Li2016NeuralNT} and \citet{wang-etal-2017-sogou} incorporated a named entity translator, which is supposed to learn transliteration of named entities.
As in their work, our proposed model also uses a character-level decoder to generate the content of placeholders, but our focus is to inflect a lemma to the appropriately inflected form given the context.

\subsection{The Code-switching Method}
Another way to introduce terminology constraints is the code-switching method \citep{song-etal-2019-code,dinu-etal-2019-training,exel-etal-2020-terminology}. The model is trained with source sentences where some words are replaced or followed by specific target words and expected to copy the words to the translation.

One advantage of the code-switching method is that, unlike the placeholder methods, it preserves the meaning of the original words, which likely leads to better translation quality.
Also, the model can incorporate the specified terminology in a flexible way: a model trained with the code-switching method not only copies the pre-specified target words but can inflect the words according to the target-side context \citep{dinu-etal-2019-training}.
In parallel to our work, \citet{niehues-2021-continuous} offers a quantitative evaluation of how well the code-switching method handles inflection of a pre-specified terminology when the terminology is given in the lemma form.

Although the code-switching method is flexible, one disadvantage is that it tends to ignore the pre-specified terminology more often than the placeholder method (\cref{sec:results}).
We propose a placeholder method that handles inflection of pre-specified terms, aiming for both flexibility and faithfulness to terminology constraints.

\subsection{Constrained Decoding}
Another approach to ensure that a pre-specified term appears in the translation is constrained decoding \citep{anderson-etal-2017-guided,hokamp-liu-2017-lexically,post-vilar-2018-fast}.
Constrained decoding can be applied to any existing NMT models without modifying its architecture and training regime, but imposes a significant cost on the decoding speed.
It is also unclear how to incorporate lexical inflection into constrained decoding.
Therefore, we focus on the placeholder and code-switching methods in this study.

\subsection{Modeling Morphological Inflection in Neural Machine Translation}
Explicitly modeling morphological inflection into NMT models has been studied mainly to enable effective generalization over morphological variation of words.
\citet{tamchyna-etal-2017-modeling} and \citet{weller-di-marco-fraser-2020-modeling} propose to decompose certain classes of words into its lemma and morphological tags to reduce data sparsity.
At decoding time, the inflected form is restored by a morphological analyzer. \citet{Song_Zhang_Zhang_Luo_2018} proposed a model that only requires a stemmer to alleviate the need for linguistic analyzers. The model decomposes the process of word decoding into stem generation and suffix prediction.

In this work, we propose to model morphological inflection in the process of embedding pre-specified terms into placeholders to improve the flexibility of placeholder translation. Our approach requires no external linguistic analyzer at prediction time; instead, inflection is performed via a neural character-based decoder.

% % \vspace{-3ex}
\section{Preliminary Study}
\vspace{-1ex}
\label{sec:preliminary}
This section first presents the foundational setup of synthetic data generation. 
Then, we provide an in-depth investigation into the pitfalls of existing synthetic data generation methods. 


\vspace{-1ex}
\subsection{Problem Setup}
In this paper, we study the synthetic data generation problem in the few-shot setting.
The input consists of a training set $\cD_{train}=\{(x_i,y_i)\}_{i=1}^K$, where $(x_i, y_i)$ represents the text and its label for the $i$-th example. $K$ denotes the total number of training samples, which is intentionally kept at a very small value (5-shot per label). The primary objective is to harness the capabilities of an LLM $\cM$ to generate a synthetic dataset, denoted as $\cD_{\text{syn}}=\{(\tilde{x_i},\tilde{y_i})\}_{i=1}^N$, where $N$ is the number of generated samples ($N \gg K$). 
% To use  for 
For each downstream task, we fine-tune an additional pre-trained classifier $\cC_{\theta}$ parameterized by $\theta$ on the synthetic dataset $\cD_{\text{syn}}$ for evaluating on the target task\footnote{While In-context Learning~\citep{brown2020language} can also be utilized, it is often hard to fit all generated instances into the context window, especially for datasets with high cardinality.}. 
% This deliberate design is aimed at leveraging $\cM$ to create a substantially augmented dataset for downstream tasks.





% The training set Dtrain = {(x, y)i}
% consists of K training samples per label where x =
% [x1, x2, . . . , xn] is a text sequence with n tokens.
\vspace{-1ex}
\subsection{Limitations of Existing Synthetic Data Generation Methods}
\label{sec:limitations}
Here, we take a closer look at the synthetic text data generated by two representative approaches: ZeroGen~\citep{ye2022zerogen}, which directly instructs LLMs for data generation, and DemoGen~\citep{gpt3mix,meng2023tuning}, which augments the prompt with few-shot demonstrations. 
We observe that these methods often introduce distribution shifts and exhibit limited diversity, which can be suboptimal for improving downstream performance. The illustration is as follows, and we include additional figures in Appendix~\ref{sec:add_prelim}.



\begin{figure}
	\centering
	\vspace{-3ex}
	\subfigure[t-SNE plot]{
		\includegraphics[width=0.28\linewidth]{figures/bc5cdr_disease_sentencebert_bsl.pdf}
		\label{fig:bc5cdr_disease_sentencebert_bsl}
	} %\hfill
         \hspace{-2.3ex}
	% \subfigure[MEDIQA-RQE]{
	% 	\includegraphics[width=0.25\linewidth]{figures/mediqa_rqe_sentencebert_bsl.pdf}
	% 	\label{fig:mediqa_rqe_sentencebert_bsl}
	% }\hspace{-1.5ex}
     \subfigure[Case study of generated examples]{
		\includegraphics[width=0.69\linewidth]{figures/case_study_prelim.pdf}
		\label{fig:case_study_prelim}
	}
	\caption{Preliminary Studies. (a) is from BC5CDR-Disease and (b) is from MEDIQA-RQE.\vspace{-1ex}}
	\vspace{-1ex}
\label{fig:prelim1}
% \vspace{-3ex}
\end{figure}



\textbf{Distribution Shift.} An inherent challenge when adapting LLMs to specific domains for text generation is the issue of \emph{distribution shift}, given that LLMs are primarily trained on vast amounts of web text in general domains. In Figure \ref{fig:bc5cdr_disease_sentencebert_bsl}, we visualize the embeddings\footnote{We employ SentenceBERT~\citep{reimers2019sentence} as the text encoder.} of both the ground truth training data and synthetic datasets generated via two representative methods. Overall, these methods use generic prompts (see Appendix~\ref{sec:prompt_format_bsl} for details) with minimal domain-specific constraints.
This limitation remains evident even when incorporating few-shot demonstrations into the process, with a notable disparity between the embeddings of the ground truth data and synthetic data.
% as there exists a large discrepancy between the embeddings of the ground truth data and the synthetic data. 
% \ran{sentencebert, more in appendix} 

To quantify the data distribution shift, we employ Central Moment Discrepancy (CMD)~\citep{zellinger2017central} to measure the gap between synthetic and real data across six clinical NLP datasets. Particularly, a high CMD value indicates a large gap between the two given distributions. Figure \ref{fig:cmd-baseline} illustrates that both ZeroGen and DemoGen exhibit elevated CMD scores, indicating substantial dissimilarity between the synthetic data and those of the real dataset.

% TSNE-embedding
% Case Study

 \begin{figure}
	\centering
	\vspace{-4.5ex}
	\subfigure[CMD]{
		\includegraphics[width=0.345\linewidth]{figures/cmd-baseline.pdf}
		\label{fig:cmd-baseline}
	} %\hfill
         \hspace{-1.5ex}
	% \subfigure[MEDIQA-RQE]{
	% 	\includegraphics[width=0.25\linewidth]{figures/mediqa_rqe_sentencebert_bsl.pdf}
	% 	\label{fig:mediqa_rqe_sentencebert_bsl}
	% }\hspace{-1.5ex}
     \subfigure[Entity Coverage]{
		\includegraphics[width=0.345\linewidth]{figures/avg-entity-baseline.pdf}
		\label{fig:avg-entity-baseline}
	}
 \hspace{-1.5ex}
      \subfigure[Entity Frequency]{
		\includegraphics[width=0.28\linewidth]{figures/bc5cdr_disease_freq_bsl.pdf}
		\label{fig:bc5cdr_disease_freq_bsl}
	}
 \vspace{-2ex}
	\caption{Preliminary Studies. (c) is from BC5CDR-Disease and is in log scale. \vspace{-3ex}}

\label{fig:prelim2}
\end{figure}

\textbf{Limited Diversity.}
Clinical datasets in real-world scenarios harbor a wealth of valuable knowledge that can be challenging to replicate within synthetically generated data by AI models. We evaluate synthetic dataset diversity by using both entity quantity and their normalized frequencies. The results are illustrated in Figures~\ref{fig:avg-entity-baseline} and \ref{fig:bc5cdr_disease_freq_bsl}. Our analysis reveals that datasets generated by ZeroGen and DemoGen exhibit a limited number of clinical entities, having a substantial discrepancy with the ground truth. 
Furthermore, it is highlighted that only a minority of potential entities and relations are frequently referenced across instances, while the majority are generated infrequently.

% Furthermore, these synthetic entities exhibit a \textit{long-tailed distribution} of normalized frequencies, highlighting that only a minority are frequently referenced across instances, while the majority of potential entities and relations are generated infrequently.

To explicitly illustrate the aforementioned limitations of synthetic datasets created using existing methods, we present a case study in Figure~\ref{fig:case_study_prelim}. In this case study, we randomly select one sample from each class within the training set generated by ZeroGen and DemoGen. These selected samples are compared with the ground truth data from the MEDIQA-RQE dataset, which aims to predict whether a consumer health query can entail an existing Frequently Asked Question (FAQ). The comparison reveals that the samples generated by ZeroGen and DemoGen tend to be more straightforward, lacking the \textit{sufficient details} and \textit{real-case nuances} present in the ground truth data. 
Furthermore, the generated samples adhere to a more uniform style and structure, while the ground truth encompasses various situations and writing styles, including urgent and informal inquiries.
% Furthermore, the generated samples lack the \textit{situation diversity} and \textit{style variability} present in the ground truth. The ground truth encompasses various situations and writing styles, including urgent and informal inquiries, while the generated samples adhere to a more uniform style and structure.


\section{Clinical Knowledge Infused Data Generation}
The revealed insights from the preliminary studies assert the necessity of domain-tailored knowledge for clinical synthetic data generation. In pursuit of efficient, effective, and scalable data generation for clinical domains, we introduce our novel framework, {\ours}, a prior knowledge-informed clinical data generation. The overview of {\ours} is shown in Figure~\ref{fig:overall}. This innovative two-step methodology harnesses the emergent capabilities of LLMs and external knowledge from KGs to facilitate the synthesis of clinical data, even when only presented with few-shot examples. 



\subsection{Clinical knowledge extraction}
Contrary to previous studies~\citep{ye2022zerogen,ye2022progen,meng2022generating,meng2023tuning} which employ generic queries to prompt LLMs for text generation, {\ours} emphasizes refining clinically informed prompts. This approach aims to extract rich clinically relevant knowledge from parametric (\eg LLMs) or nonparametric sources (\eg knowledge graphs) and tailor it to clinical NLP tasks.
% Different from prior research~\citep{ye2022zerogen,ye2022progen,meng2022generating,meng2023tuning} that use the simple task-specific queries to prompt Language Model (LLM) for data generation,
% {\ours}'s primary focus is to optimize clinically informed prompts to better harvest the clinical-relevant knowledge from LLMs and adapt them to clinical NLP tasks.
% emphasize on the initial ground of contextual domain knowledge.
To realize this objective, our modeling contains two dimensions including \emph{clinical topics} and \emph{writing styles}, which are integrated into the original prompts to infuse domain-specific knowledge. 
By dynamically composing different topics and writing styles together, {\ours} can provide a diverse suite of prompts, resulting in a wider spectrum of text produced from LLM.


% harness the synergistic potential of Knowledge Graphs (KGs) and Large Language Models (LLMs) in constructing a candidate set enriched with prior knowledge, specifically (1) the sampling of pertinent entities or relations from external knowledge graphs (KGs), and (2) the extraction of relevant hints through queries to Language Models (LLMs).\ran{style}

\subsubsection{Clinical Topics Generation}
We provide two choices to generate clinical topics -- one is to sample related entities or relations from external KG, and the other is to query relevant knowledge from LLM.

\textbf{Topics sampled from Non-Parametric KGs.} 
Healthcare KGs offer 
% comprehensive view of medical concepts and their relationships.
a rich collection of medical concepts and their complex relationships, and have emerged as a promising tool for organizing medical knowledge in a structured way~\citep{li2022graph,cui2023survey}. 
In our methodology, we employ the iBKH KG~\citep{su2023biomedical} due to its broad coverage over clinical entities. 
To illustrate, for the Disease Recognition task (NCBI)~\citep{ncbi-disease}, we extract all medication nodes from the iBKH to bolster the pharmaceutical information. 
As another example, we retrieve links between drug and disease nodes for the chemical and disease relation extraction (CDR) task~\citep{cdr_dataset}. 
By integrating the information from the clinical KG into our data generation process, we guarantee that our generated samples exhibit a high degree of contextual accuracy, diversity, and semantic richness.


\begin{figure}[t]
    \centering
    \vspace{-4ex}
    \includegraphics[width=0.96\linewidth]{figures/clingen-framework.pdf}
    \caption{The overview of \ours. \vspace{-2.5ex}
    % The left orange panel illustrates the knowledge extraction part. The middle purple panel shows the synthetic data generation module. The right green one is the fine-tuning step.}
    }
    % \vspace{-ex}
    \label{fig:overall}
\end{figure}

\textbf{Topics queried from Parametric Model (LLMs).} 
LLMs provide an alternative method for acquiring domain knowledge, as they are pre-trained on extensive text corpora, including medical literature.
% Large Language Models (LLMs) offer another viable avenue for acquiring foundational domain knowledge, owing to their extensive training on diverse text corpora, including the vast expanse of medical literature. 
Specifically, we aim to harness the rich clinical domain knowledge encoded in ChatGPT (\texttt{gpt-3.5-turbo-0301}) to augment the prompt. 
The incorporated prior knowledge from LLMs is focused on entity categories that hold significant relevance within clinical text datasets, including diseases, drugs, symptoms, and side effects.
For each of these pivotal entity types, we prompt the LLMs by formulating inquiries, \eg, ``\texttt{Suppose you are a clinician and want to collect a set of <Entity Type>. Could you list 100 entities about <Entity Type>?}''. These crafted conversational cues serve as effective prompts, aiding in the retrieval of clinically significant entities from the extensive domain knowledge within LLMs. For each entity type, we generate 300 entities which will be used for synthetic data generation.

\vspace{-0.5ex}
\subsubsection{Writing Styles Suggestion}
\vspace{-0.5ex}
\textbf{Styles suggested by LLMs.} To address the limitations mentioned in Sec~\ref{sec:limitations} and introduce a diverse range of writing styles into the generated samples, we leverage the powerful LLM again by suggesting candidate writing styles for each task. Specifically, we incorporate task names into our prompts (e.g., disease entity recognition, recognizing text entailment, etc.) and integrate few-shot demonstrations. We then engage ChatGPT in suggesting several potential sources, speakers, or authors of the sentences. See Appendix~\ref{sec: style_prompt} for detailed prompt. Responses such as ``\texttt{medical literature}" or ``\texttt{patient-doctor dialogues}" are augmented into the prompts to imitate the writing styles found in real datasets. 
% \joyce{I'm not quite sure what this quite looks like. Is there an example in the appendix/supplemental part? Seems you're downplaying it here otherwise.}
% explicitly query 

\vspace{-0.5ex}
\subsection{Knowledge-infused Synthetic Data Generation}
\vspace{-0.5ex}
% After building the candidate set with one of the previous approaches, we randomly sample one keyword each time and augment it into the prompt. 
% % An example of the prompt on xxx dataset is shown in Figure xxx. 
% For example, a keyword for the NCBI dataset could be ``\texttt{stroke}", then we enrich the prompt for querying ChatGPT by adding an additional sentence ``\texttt{generate a sentence about stroke}" (see Figure/Appendix for full prompt). 
With the generated entities as well as styles, the key challenge becomes how to leverage them to extract rich clinical information from the LLM for improving synthetic data quality.
% hat, if used wisely, can be potentially leveraged for generalizing over limited context.
% if they are guided effectively. 
Directly putting all the elements to enrich the prompt is often infeasible due to the massive size of entities.
% and over-complicated 
To balance informativeness as well as diversity, we propose a  knowledge-infused strategy, where the collected clinical topics and writing styles serve as the base unit. 
% for leading LLMs towards clinical domains.  
% To elucidate, each extracted hinting keyword from the candidate set takes a role in shaping a clinically informed prompt structure. 
In each step, we randomly sample a topic and a writing style from the candidate set to augment the prompts.
For instance, for the Disease Recognition (NCBI) task, consider a clinical entity like ``\texttt{stroke}" . We enrich the prompt query for LLM by appending ``\texttt{generate a sentence about stroke}" as a generation guidance. For a comprehensive view of the prompt formats across various tasks, please refer to Appendix~\ref{sec:prompt_format}. 
Despite its simplicity, this knowledge-infused strategy ensures that the clinical context is incorporated into the prompts while encouraging prompt diversity (via composing different entities and writing styles), 
thereby enhancing the quality and clinical relevance of the generated synthetic data.


\subsection{Language model fine-tuning}
After generating synthetic data $\cD_{\text{syn}}$ through LLMs, we fine-tune a pre-trained classifier $\cC_{\theta}$ to each downstream task. Following \citep{meng2023tuning}, we first fine-tune $\cC_{\theta}$ on $\cD_{\text{train}}$ with standard supervised training objectives (denoted as $\ell(\cdot)$), then on the synthetic data $\cD_{\text{syn}}$ as  
\begin{align}
    \label{eq:stage1}
    \mathrm{Stage~I}: \theta^{(1)} = \min_{\theta}~\mathbb{E}_{(x, y) \sim \cD_{\text{train}}} \ell\left( f(x; \theta), y \right), \\
    \mathrm{Stage~II}: \theta^{(2)} =  \min_{\theta}~\mathbb{E}_{(x, y) \sim \cD_{\text{syn}}} \ell\left( f(x; \theta), y \right),  \theta_{\text{init}} = \theta^{(1)}.
\end{align} 
% \end{equation}
It's important to highlight that we strictly follow a standard fine-tuning process and avoid using any extra techniques: (1) for standard classification tasks, $\ell(\cdot)$ is the cross-entropy loss; (2) for multi-label classification tasks, $\ell(\cdot)$ is the binary cross-entropy loss; 
(3) for token-level classification tasks, we stack an additional linear layer as the classification head and  $\ell(\cdot)$ is the token-level cross-entropy loss. 
% This is to ensure methodological consistency and transparency throughout the evaluation across different methods and tasks. 
The design of \emph{advanced learning objectives} as well as \emph{data mixing strategies}, while important, are orthogonal to the scope of this paper. 






% It is important to underscore that, in pursuit of a rigorous quality evaluation of the synthetic data, we rigorously adhere to a conventional fine-tuning pipeline, without any auxiliary techniques. Our chosen learning objective is the minimization of the cross-entropy loss against the task-specific target, ensuring methodological consistency and transparency throughout the evaluation across different methods and tasks.

% \subsubsection{Style}

% \section{Experimental Setups}
We evaluate the proposed model with several baselines to show how well the model can produce the appropriately inflected form of a given lemma.

\subsection{Corpus}
We conduct experiments in a Japanese-to-English translation task with the ASPEC corpus \citep{nakazawa-etal-2016-aspec}.
This corpus consists of abstracts from scientific articles, which tend to contain many technical terms.
Such words are rare and hard for the model to learn the correct translation, and thus this corpus fits the typical use-case of lexically constrained translation.
We use the initial 1M sentence pairs from the training split for training.

\subsection{Word Dictionary}
In this study, lexical constraints in translation are introduced through a source-to-target word dictionary. We construct the dictionary automatically from the ASPEC corpus through the following procedure.

First, we obtain the word alignment by feeding the first 1M sentence pairs of the training split and validation/test splits to \texttt{GIZA++}.\footnote{\url{https://github.com/moses-smt/giza-pp}} We tokenize Japanese sentences with \texttt{Mecab}\footnote{\url{https://taku910.github.io/mecab/}} and English sentences with \texttt{spaCy}.\footnote{\url{https://spacy.io/}} We then construct a phrase table and extract only those with more than 100 occurrences.
Then, we split the dictionary into noun and verb entries to facilitate the analysis of the results and remove noise. If both the Japanese and English phrases are noun phrases, the entry is registered in the noun dictionary. If the Japanese phrase is a nominal verb\footnote{The nominal verb (\ja{サ変動詞}) is the most productive class of verb in Japanese and many new or technical terms fall into this category ({\it e.g.}, \ja{最適化する}-{\it optimize}, \ja{過学習する}-{\it overfit}).} and English is a verb, the entry is registered in the verb dictionary.
In this study, we evaluate the model's ability to inflect a provided lemma. Lemmas for the target language (English) are obtained with \texttt{spaCy}.

\subsection{Models}
As the baseline, we implement a Transformer \citep{NIPS2017_3f5ee243} translation model based on \texttt{AllenNLP} \citep{Gardner2017AllenNLP}.
We configure the model in the Transformer-base setting and sentences are tokenized using \texttt{sentencepiece} \citep{kudo-2018-subword}, which has a shared source-target vocabulary of about 16k sub-words.
The overviews of lexically constrained models are summarized in \fig{fig:baseline}.

\minisection{Placeholder (PH)}
In the placeholder method, the model is trained to translate sentences with a placeholder token and pass that through to the translation. In our experiments, we use different placeholder tokens \texttt{[NOUN]} and \texttt{[VERB]} for nouns and verbs.
Predicted placeholder tokens are replaced by the pre-specified term in the post-processing step.
We evaluate three types of placeholder baselines, each of which differs in what inflected form the target placeholder token is replaced with: {\bf PH (oracle)}, where the pre-specified term is embedded in the same form as in the reference; {\bf PH (lemma)}, always the lemma form; {\bf PH (common)}, the most common inflected forms in the training data, which are the singular form for \texttt{[NOUN]} and the past tense form for \texttt{[VERB]}. The results of PH (lemma) and PH (common) are provided as naive baselines to give a sense of how difficult predicting the correct inflected form is.

We also provide a baseline that performs word inflection through an external resource ({\bf PH (morph)}).
As in \citet{tamchyna-etal-2017-modeling}, words that need inflection are followed by morphological tags, and word formation is realized through an external resource.
We use \texttt{LemmInflect}\footnote{\url{https://github.com/bjascob/LemmInflect}} to decompose the dictionary entries with their lemma and part-of-speech tags and to recover the inflected word form.
As this model uses an external resource to perform inflection, it is not directly comparable with our proposed models but we provide its results as an oracle baseline.

\minisection{Code-switching (CH)}
The code-switching model replaces a phrase in a source sentence with the corresponding target phrase according to a bilingual dictionary.\footnote{\citet{dinu-etal-2019-training} utilize source factors that indicate which tokens are code-switched, but we observe no significant difference by adding source factors. Therefore, we simply report the results from the model with minimal components.}
{\bf CH (oracle)} uses the same target words as in the reference, and {\bf CH (lemma)} uses the lemma form.

\minisection{Proposed Model}
We implement our proposed model described in \cref{sec:approach} on top of the placeholder baseline model.
Compared to the baseline, our proposed model has three additional modules: the target context encoder, target character embeddings, and character-level decoder.
The embedding and hidden sizes are all set to 512, which is the same as in the Transformer-base model.
The additional encoder and decoder have two layers, and the feedforward dimension is 1024.


Note that, for all the models, we restrict the number of constraints to at most one in each sentence as an initial investigation. This favors the placeholder-based models as handling more than one placeholder introduces additional complexity in the system and tends to degrade the performance, while the code-switching methods suffer less from multiple constraints \citep{song-etal-2019-code}.
We leave experiments with multiple constraints to future work.


\begin{figure}[t]
\centering
\includegraphics[width=14.0cm]{data/baselines.png}
\caption{The preprocessing of the lexically constrained baseline models.}
\label{fig:baseline}
\end{figure}



\subsection{Training with Lexical Constraints}
To apply lexical constraints, the models are trained with data augmentation.
Augmented data is created for all sentences that contain any of the source and target phrases found in the dictionary entries.
To control the amount of augmented data to around 10\% of the original training data, we restrict the dictionary entries to infrequent ones.
The restriction to infrequent phrases also simulates real-word use-cases, where user-specified terms are often rare words that typical NMT models struggle with in translation.
Specifically, we restrict the noun entries to ones with a count at most 20, and the verb entries to 2000.
The threshold is chosen to balance the amount of noun and verb entries in the augmented data.


\subsection{Optimization}
We optimize the models using Adam \citep{Kingma2015AdamAM} with the Noam learning rate scheduler with 8000 warmup steps \citep{NIPS2017_3f5ee243}. The training is stopped when the validation BLEU score does not improve for 3 epochs.

For our proposed model, we found that optimizing the word-level modules and character-level modules separately stabilizes the training process and improves the translation quality.
We first train a normal placeholder model, use the weights to initialize those of our proposed model, and then only update the parameters of the additional modules.
In this second training stage, we use the loss value as validation metric and stop the training when the lowest value is updated for 5 epochs.

% \section{Results}
\label{sec:results}
\subsection{Evaluation}
For each model, we evaluate the overall translation quality with BLEU \citep{papineni-etal-2002-bleu}.\footnote{SacreBLEU\citep{post-2018-call} version string: \\\texttt{case.mixed+numrefs.1812+smooth.exp+tok.13a+version.1.5.1}}
We also evaluate the {\it specified term use rate}, a metric to check if the model correctly includes the specified target term.
Note that this is only an approximate measure of what we want to measure: whether the specified term is used in the correct form in the output translation.
Since a single source sentence can be translated into different grammatical constructions, it is possible that the inflected form in the system output is different from the one in the reference but still correct in the context.
Still, we find a substantial overlap in the inflectional form of the specified term between the reference and the system output, and thus report this metric, followed by a more closely inspected manual evaluation.

Also, we are interested in how well the model generalizes to dictionary entries unseen during training. In typical use cases of lexically constrained translation, the specified terms are new or rare words that are not likely to appear in the training data. We construct two kinds of evaluation dictionaries: {\it seen} and {\it unseen}.
We first construct a dictionary by aggregating only entries that appear in the dev/test set.
Then, we randomly split the entries into {\it seen} and {\it unseen} and remove the {\it unseen} entries from the training dictionary. Thus, the {\it seen} split contains entries that appear in the training data while the {\it unseen} not.
We evaluate the model separately using the noun and verb dictionary, which results in a total of four kinds of evaluation configurations.

\begin{table*}[h]
  \centering
  \begin{tabular}{lllll} \toprule
      & \multicolumn{2}{c}{NOUN}                              & \multicolumn{2}{c}{VERB}                              \\
      & \multicolumn{1}{c}{{\it seen}} & \multicolumn{1}{c}{{\it unseen}} & \multicolumn{1}{c}{{\it seen}} & \multicolumn{1}{c}{{\it unseen}} \\ \midrule
      Baseline & 27.1 / 68.3 & 27.1 / 66.5 & 27.1 / 63.4 & 27.1 / 61.2 \\ \midrule
      CS (oracle) & 27.3 / 86.8 & 27.0 / 79.3 & 27.5 / 91.9 & 27.2 / 43.9 \\
      PH (oracle) & 27.2 / 98.8 & 27.0 / 99.2 & 27.4 / 98.7 & 27.5 / 99.4 \\ \midrule
      PH (lemma) & \multirow{2}{*}{27.1 / 84.7}  & \multirow{2}{*}{26.9 / 84.0} & 26.9 / 9.41 & 27.1 / 11.4 \\
      PH (common) &                            &  & 27.3 / 81.8 & 27.3 / 68.9 \\
      CS (lemma) & 27.4 / 81.7 & 27.1 / 74.6 & 27.6 / 81.7 & 27.3 / 42.1 \\
      Proposed & 27.2 / 89.9 & 26.9 / 79.1 & 27.4 / 88.3 & 27.4 / 73.9 \\
      PH (morph) & 27.9 / 84.7 & 27.8 / 81.2 & 28.5 / 91.1 & 28.4 / 87.9 \\
  \bottomrule
  \end{tabular}
\caption{BLUE scores and the specified term use rate of the different models over different evaluation dictionaries. CS: Code-switching, PH: placeholder. For NOUN, PH (lemma) and PH (common) are the same model because the most common inflection for nouns is their lemma.}
\label{table:results-overall}
\end{table*}



\subsection{Main Results}
The results are shown in Table \ref{table:results-overall}. For each configuration, we report the average of three models trained with different random seeds.

First, the lexically constrained models show BLEU scores not significantly different from the baseline.
The only exception is PH (morph): it consistently improves the BLEU score by from 0.7 to 1.4 points from the baseline.
This indicates the strength of injecting the NMT model with morphological knowledge for better generalization in translation.
In the following discussion, we focus on the comparison of the specified term use rate.

PH (oracle) and CS (oracle) models receive the same inflected form of a specified term as in the reference, and thus offer upper bounds for the specified term use rate.
We observe that PH (oracle) exhibits nearly perfect specified term use rates (more than 98\% with all dictionaries).
Also, it is more successful at incorporating the specified term into translation than CS (oracle) in the setting of one constraint, which is in line with previous observations \citep{song-etal-2019-code}.

As for the models that need to handle inflection, the results are quite mixed for NOUN.
A simple strategy of predicting the most common inflection achieves better specified term use rates than most of the other sophisticated models.
We conjecture that some examples allow either singular or plural form and that makes a proper evaluation difficult. Therefore, we turn to the results from VERB for model comparison.

In terms of both {\it seen} and {\it unseen} of the VERB dictionary, PH (morph) performs the best.
Note, however, that this model is not comparable to our model as it assumes access to a high-quality morphological analyzer at training time to obtain morphological tags and the correct inflectional paradigm of user-specified terms at prediction time.

In a more restricted setting, our proposed model outperforms the comparable code-switching model (CS (lemma)) and the other baselines.
In particular, the proposed model is more robust than CS (lemma) to {\it unseen} specified terms: we observe a consistent tendency that the specified term use rate degrades when the entries are unseen during training especially with CS (lemma) and verb entries (81.7 to 42.1), while this tendency is less pronounced in the placeholder model with lemmas (88.3 to 73.9).
Overall, our model exhibits faithfulness to lexical constraints similar to those of the normal placeholder model while having flexibility, which we examine below.

\subsection{Fine-grained Analysis}
The specified term use rate only checks whether specified terms are used in the same form as in the reference. Now we examine the systems' output more closely by manual inspection.
As the problem of inflection matters more in verbs than in nouns in English, here we focus on the translation with the verb dictionary.

We sample from the system's output of the test set 50 sentences with the {\it seen} and {\it unseen} lexical constraints respectively.
We manually check the sampled sentences and annotate each sentence with one of the three tags: {\bf {\it correct}} --- the specified term is used in the translation in the correct inflected form (not necessarily the same as in the reference); {\bf {\it incorrect}} --- the model produces the specified term in some inflected form but that results in an ungrammatical sentence; {\bf {\it null}} --- the model fails to produce the specified term in any form.
The result is shown in Table \ref{table:results-manual}.

\begin{table}[]
  \centering
  \begin{tabular}{lcc} \toprule
                            & VERB {\it seen}  & VERB {\it unseen} \\
  CS (lemma)            & 49 / 0 / 1 & 26 / 0 / 24 \\
  PH with lemmas (proposed) & 48 / 2 / 0 & 39 / 7 / 4  \\
  PH (morph)          & 50 / 0 / 0 & 47 / 3 / 0 \\ \bottomrule
  \end{tabular}
\caption{The manual evaluation of the 50 sampled sentences. The values in each cell indicate {\it correct} / {\it incorrect} / {\it null}.}
\label{table:results-manual}
\end{table}

Firstly, for the words that are seen in the training data, all the models mostly generate the correct word form in the context.
On the other hand, the evaluation with VERB unseen reveals both the advantages and disadvantages of each model, which we discuss with examples below.

\minisection{The placeholder model with morphological tags can handle inflection well}
The model mostly generates the correct inflectional form of the specified terms.
The only three exceptions from VERB seen are errors in choosing the transitive or intransitive usage of the term (Table \ref{fig:ph_pos_wrong}).

\begin{table}[h]
  \centering
  \begin{tabularx}{\textwidth}{X} \toprule
    {\bf Source}: \ja{特発性肺線維症(IPF)患者14例及びIPF急性増悪で\textcolor{red}{入院}した患者8例を対象として,BALF・血漿に関してウィルス検査・免疫血清学的検査を施行した}\\
    {\bf Reference}: The virus inspection and immunoserologic inspection of BALF and blood plasma were carried out for 14 idiopathic pulmonary fibrosis (IPF) patients and of 8 patients \textcolor{red}{hospitalized} for IPF acute aggravation. \\
    {\bf System Output}: Wils inspection and immunoserologic inspection were enforced on BALF blood and blood in 14 patients with idiopathic pulmonary fibrosis (IPF) and 8 patients who \textcolor{red}{hospitalized} in the IPF acute aggravation. \\ \bottomrule
  \end{tabularx}
\caption{A translation example with the placeholder model with morphological tags. The system output should have generated {\it were hospitalized} in the red part.}
\label{fig:ph_pos_wrong}
\end{table}

\minisection{The code-switching method always produces grammatical inflectional forms}
We observe no {\it incorrect} examples from the code-switching model.
Since the output is determined solely by the word decoder with no additional post-editing performed, if the word decoder is well trained, we can expect the output sentences to be grammatical.

\minisection{The code-switching method tends to fail to observe the constraints}
However, the code-switching methods fail to produce the specified term in 24 examples out of 50, which is notably higher than the other methods.
A typical error is the model ignoring the constraint and producing a synonym, for example, generating {\it conclude} instead of {\it judge}, {\it examine} instead of {\it study}.
This is reasonable given the model architecture.
A well-trained NMT model usually assigns similar vector representations to synonyms.
Even when the specified term is given in the source sentence, it is given a representation similar to other synonyms inside the model, and thus the decoder can generate any words with similar meaning.
We also observe a few character decoding errors: wrongly generating {\it hot-spitalized} instead of {\it hospitalized}, {\it move} instead of {\it remove}.

\minisection{The placeholder method almost always produces the specified term, but sometimes fails to inflect it correctly}
The placeholder method fails to observe the constraint much less frequently than the code-switching method (only 4 examples out of 50).
In most cases (39 examples out of 50), the model can successfully predict the correct form as shown in Table \ref{fig:ph_example_correct}.

\begin{table}[h]
  \centering
  \begin{tabularx}{\textwidth}{X} \toprule
    {\bf Source}: \ja{フローセンサーの原理は浮遊式流量計のテーパー管内フロートの位置を差動トランスで検出し,これの電圧制御により流量を\textcolor{red}{管理}する。}\\
    {\bf Reference}: The sensor controls the flow rate by detecting the position of the float in the tepered tube with a differential transformer and \textcolor{red}{controlling} it with the obtained voltage. \\
    {\bf System Output}: The principle of the flow sensor is that the position of the float in the taper tube of the floating flowmeter is detected by the differential transformer, and the flow rate is \textcolor{red}{controlled} by this voltage control. \\ \bottomrule
  \end{tabularx}
  \caption{A translation example with the placeholder model with a character decoder. The model predicts the correct inflectional form of {\it control} that fits in the context.}
  \label{fig:ph_example_correct}
\end{table}

The failures consist of generalization errors of inflectional form: generating {\it maken} for {\it make}.
It is impossible in principle to correctly predict irregular inflectional forms that are unseen in the training data, but this is usually not much of a problem since the specified term is usually a rare or new word, which tends to have a regular inflectional paradigm.
The other kind of error we observe is the model predicting a well-defined word form that is wrong in the context (Table \ref{fig:ph_example_wrong}). We expect that both error types can be addressed by exploiting additional data, either parallel or monolingual, to learn inflection rules in the target language.

\begin{table}[t]
  \centering
  \begin{tabularx}{\textwidth}{X} \toprule
    {\bf Source}: \ja{国立病院機構関門医療センター(国立下関病院)は2002年9月30日に女性総合診療を\textcolor{red}{開設}した。}\\
    {\bf Reference}: A National Hospital System Kanmon Medical Center (A National Shimonoseki Hospital) \textcolor{red}{opened} the comprehensive woman medical care service on September 30th in 2002. \\
    {\bf System Output}: National Hospital Mechanism Kanmon Medical Center ( the national Shimonoseki Hospital ) \textcolor{red}{opening} the woman general medical care on September 30th, 2002. \\ \bottomrule
  \end{tabularx}
\caption{A translation example with the placeholder model with a character decoder. The model predicts a wrong inflectional form for {\it open}.}
\label{fig:ph_example_wrong}
\end{table}

% \section{Conclusion and Future Work}
In this study, we point out that the traditional placeholder translation method embeds the specified term into the generated translation without considering the context of the placeholder token, which potentially leads to grammatically incorrect translations.
To address this shortcoming, we proposed a flexible placeholder translation model that handles inflection when the specified term is given in the form of a lemma.
In the experiment of the Japanese-to-English translation task, we showed that the proposed model can inflect user-specified terms more accurately than the code-switching method.

Future work includes testing the proposed method on morphologically-rich languages or extending the model to handle more than one placeholder in a sentence.
Also, the proposed model still has room for improvement to learn inflection.
It is possible that we can improve the model by exploiting monolingual corpora in the target language to provide additional training signals for learning the correct inflection in context.


\section{Introduction}
\label{sec:introduction}
\begin{figure}[t]
\centering
   \includegraphics[clip, width=0.8\columnwidth]{./example_3.pdf}
   \caption{A more informative response (Response B in the figure) can provide information that helps to infer the query content given the dialogue context.}
   \vspace{-0.5 cm}
   \label{fig:back-reasoning}
\end{figure}

Recently developed end-to-end dialogue systems are trained using large volumes of human-human dialogues to capture  underlying interaction patterns~\citep{li2015diversity,li2017adversarial,xing2017topic,khatri2018advancing,vinyals2015neural,zhang2019dialogpt,bao2019plato}. A commonly used approach to designing data-driven dialogue systems is to use an encoder-decoder framework: feed the dialogue context to the encoder, and let the decoder output an appropriate response. Building on this foundation, different directions have been explored to design dialogue systems that tend to interact with humans in a coherent and engaging manner~\citep{li2016deep,li2019dialogue,wiseman2016sequence,baheti2018generating,xing2016topic,zhang2018personalizing}. However, despite significant advances, there is still room for improvement in the quality of machine-generated responses.  


An important problem with encoder-decoder dialogue models is their tendency to generate generic and dull responses, such as \emph{``I don't know''} or \emph{``I'm not sure''}~\citep{li2015diversity,baheti2018generating,li2016deep,jiang-why-2018}. 
There are two types of methods for dealing with this problem.
The first introduces updating signals during training, such as modeling future rewards (e.g., ease of answering) by applying reinforcement learning~\citep{li2016deep,li2019dialogue}, or bringing variants or adding constraints to the decoding step~\citep{wiseman2016sequence,li2015diversity,baheti2018generating}. 
The second type holds that, by itself, the dialogue history is not enough for generating high-quality responses, and side information should be taken into account, such as topic information~\citep{xing2016topic,xing2017topic} or personal user profiles~\citep{zhang2018personalizing}. Solutions relying on large pre-trained language models, such as DialoGPT~\citep{zhang2019dialogpt}, can be classified into the second family as well.


In this paper, we propose to train dialogue generation models \emph{bidirectionally} by adding a backward reasoning step to the vanilla encoder-decoder training process. 
We assume that the information flow in a conversation should be coherent and topic-relevant. Given the dialogue history, neighboring turns are supposed to have a tight topical connection to infer the partial content of one turn given the previous turn \emph{and vice versa}. 
Inferring the next turn given the (previous) conversation history and the current turn is the traditional take on the dialogue generation task. 
We extend it by adding one more step: given the dialogue history and the next turn, we aim to infer the content of the current turn. We call the latter step \emph{backward reasoning}. We hypothesize that this can push the generated response to be more informative and coherent: it is unlikely to infer the dialogue topic given a generic and dull response in the backward direction. 
An example is shown in Figure~\ref{fig:back-reasoning}. Given the dialogue context and \emph{query},\footnote{We use \emph{query} to distinguish the current dialogue turn from the context and the response; \emph{query} is not necessarily a real query or question as considered in search or question-answering tasks.} we can predict the reply following a traditional encoder-decoder dialogue generation setup. In contrast, we can infer the content of \emph{query} given the context and reply as long as the reply is informative. 
Inspired by~\citet{zheng2019mirror}, we introduce a latent space as a bridge to simultaneously train the encoder-decoder model from two directions. 
Our experiments demonstrate that the resulting dialogue generation model, called \emph{Mirror}, benefits from this bidirectional training process.

Overall, our work provides the following contributions:
\begin{enumerate}[leftmargin=*,label=\textbf{C\arabic*},nosep]

\item We introduce a dialogue generation model, \emph{Mirror}, for generating high quality responses in open-domain dialogue systems;

\item We define a new way to train dialogue generation models bidirectionally by introducing a latent variable; and

\item We obtain improvements in terms of dialogue generation performance with respect to human evaluation on two datasets.
\end{enumerate}

\section{Related Work}
% In the early years of dialogue system development, the key components of dialogue systems consisted of a set of predefined rules mapping dialogue contexts to predefined responses~\citep{weizenbaum1966eliza}. 
Conversational scenarios being considered today are increasingly complex, going beyond the ability of rule-based dialogue systems~\citep{weizenbaum1966eliza}. \citet{ritter2011data} propose a data-driven approach to generate responses, building on phrase-based statistical machine translation. Neural network-based models have been studied to generate more informative and interesting responses~\citep{sordoni2015neural,vinyals2015neural,serban2016hred}.
\citet{serban2017hierarchical} introduce latent stochastic variables that span a variable number of time steps to facilitate the generation of long outputs. 
Deep reinforcement learning methods have also been applied to generate coherent and interesting responses by modeling the future influence of generated responses \citep{li2016deep,li2019dialogue}.  Retrieval-based methods are also popular in building dialogue systems by learning a matching model between the context and pre-defined response candidates for response selection~\citep{qiu2020if,tao2019multi,wu2016sequential,gu2020speaker}. Our work focuses on response \emph{generation} rather than \emph{selection}.

Since encoder-decoder models tend to generate generic and dull responses, \citet{li2015diversity} propose using maximum mutual information as the objective function in neural models to generate more diverse responses.
\citet{xing2017topic} consider incorporating topic information into the encoder-decoder framework to generate informative and interesting responses. 
To address the dull-response problem, \citet{baheti2018generating} propose incorporating side information in the form of distributional constraints over the generated responses. \citet{su2020diversifying} propose a new perspective to diversify dialogue generation by leveraging non-conversational text. Recently, pre-trained language models, such as GPT-2~\citep{radford2019language}, Bert~\citep{devlin2018bert}, XL-Net~\citep{yang2019xlnet}, have been proved effective for a wide range of natural language processing tasks. Several authors make use of pre-trained transformers to attain performance close to humans both in terms of automatic and human evaluation~\citep{zhang2019dialogpt,wolf2019transfertransfo,golovanov2019large}. Though pre-trained language models can perform well for general dialogue generation, they may become less effective without enough data or resources to support these models' pre-training. In this work, we show the value of developing dialogue generation models with limited data and resources.

The key distinction compared to previous efforts~\citep{li2015diversity,baheti2018generating} is our work is the first to use the original training dataset through a differentiable backward reasoning step, without external information. 



\section{Method: \emph{Mirror}}
\label{sec:method}
\subsection{Problem setting}
In many conversational scenarios, the dialogue context is relatively long and contains a lot of information, while the reply (\emph{Response}) is short (and from a different speaker). This makes it difficult to predict the information in the context by only relying on the response in the backward direction. 
Therefore, we decompose the dialogue context into two different segments: the context $c$ and query $x$ (Figure~\ref{fig:back-reasoning}). 
Assuming that we are predicting the response at turn $t$ in a dialogue, the context $c$ will consist of the dialogue turns from $t-m$ to $t-2$ and the query $x$ corresponds to turn $t-1$. 
Here, we use the term \emph{query} to distinguish the dialogue turn at time step $t-1$ from the context $c$ and response $y$; as explained before, the term \emph{query} should not be confused with a query or question as in search or question-answering tasks. 
The value $m$ indicates how many dialogue turns we keep in the context $c$. 
We use $c_{all}$ to represent the concatenation of $c$ and $x$, which is also the original context before being decomposed. 
Our final goal is to predict the response $y$ given dialogue context $c$ and query $x$. 

\subsection{Mirror-generative dialogue generation}
\label{section:mirror-method}
\citet{shen2017conditional} propose to maximize the conditional log likelihood of generating response $y$ given context $c_{all}$, $\log p(y \mid c_{all})$,  and they introduce a latent variable $z$ to group different valid responses according to the context $c_{all}$. 
The lower bound of $\log p(y \mid c_{all})$ is given as:
%
\begin{equation}
\begin{split}
\log p(y \mid c_{all}) \geq{} &  \mathbb{E}_{z \sim q_\phi(z \mid c_{all}, y)} \log p_\theta (y \mid c_{all}, z)  - {}\\
& D_\mathit{KL}(q_\phi(z \mid c_{all}, y) \| p_\theta(z \mid c_{all})).
\end{split}
\label{eq:loss-vhred}
\end{equation}
In Eq.~\ref{eq:loss-vhred}, $q_\phi(z \mid c_{all}, y)$ is the posterior network while $p_\theta(z \mid c_{all})$ is the prior one.

Instead of maximizing the conditional log likelihood $\log p(y \mid c_{all})$, we propose to maximize $\log p(x,y \mid c)$, representing the conditional likelihood that $\langle x,y\rangle$ appears together given dialogue context $c$. 
The main assumption underlying this change is that in a conversation, the information flow between neighboring turns should be coherent and relevant, and this connection should be bidirectional. 
For example, it is not possible to infer what the query is about when a generic and non-informative reply ``I don't know'' is given as shown in Figure~\ref{fig:back-reasoning}. 
By taking into account the information flow from two different directions, we hypothesize that we can build a closer connection between the response and the dialogue history and generate more coherent and informative responses. Therefore, we propose to optimize $\log p(x,y \mid c)$ instead of $\log p(y\mid c_{all})$. 

Following \cite{kingma2013auto,shen2017conditional}, we choose to maximize the variational lower bound of $\log p(x,y \mid c)$, which is given as:
%
\begin{equation}
\begin{split}
\log p(x,y \mid c) \geq {} & \mathbb{E}_{z \sim q_\phi(z \mid c, x, y)} \log p_\theta (x, y \mid c, z)  - {}\\
& D_\mathit{KL}(q_\phi(z \mid c, x, y) \| p_\theta(z \mid c)),
\end{split}
\label{eq:loss-lb}
\end{equation}
%
where $z$ is a shared latent variable between context $c$, query $x$ and response $y$.  
Next, we explain how we optimize a dialogue system by maximizing the lower bound shown in  Eq.~\ref{eq:loss-lb} from two directions.

\begin{figure}[t]
\centering
   % \includegraphics[clip, width=2.0\columnwidth]{./figures/framework.pdf}
   \includegraphics[clip, width=0.75\linewidth]{./framework_half.pdf}
   \caption{The main architecture of our model, \emph{Mirror}. It consists of three steps: information encoding, latent variable generation, and target decoding. }
   \vspace*{-0.5\baselineskip}
   \label{fig:framework}
\end{figure} 

  
\subsubsection{Forward generation in dialogue generation}
% \todo[looks redundant here]
With respect to the forward dialogue generation, we interpret the conditional likelihood $\log p_\theta (x, y \mid c, z)$ in the forward direction:
%
\begin{equation}
\begin{split}
\mbox{}\hspace*{-3mm}
\log p_\theta (x, y \mid c, z) 
=  \log p_\theta (y \mid c, z, x) + \log p_\theta (x \mid c, z).
\end{split}
\hspace*{-2mm}\mbox{}
\label{eq:cond-backward}
\end{equation}
%
Therefore, we can rewrite Eq.~\ref{eq:loss-lb} in the forward direction as:
%
\begin{equation}
\begin{split}
\log\, & p(x,y \mid c)\\
\geq{} & \mathbb{E}_{z \sim q_\phi(z \mid c, x, y)} [\log p_\theta (y \mid c, x, z) + \log p_\theta (x \mid c, z)] {}\\& - D_\mathit{KL}(q_\phi(z \mid c, x, y) \| p_\theta(z \mid c)).
\end{split}
\label{eq:forward-loss}
\end{equation}
We introduce $q_\phi(z \mid c, x, y)$ as the posterior network, also referred to as the recognition net, and $p_\theta(z \mid c)$ as the prior network.


\subsubsection{Backward reasoning in dialogue generation}
As in the forward direction, if we decompose the conditional likelihood $\log p_\theta (x, y \mid c, z)$ in the backward direction, we can rewrite Eq.~\ref{eq:loss-lb} as:
%
\begin{equation}
\begin{split}
\log\,& p(x,y \mid c) \\
\geq {} & \mathbb{E}_{z \sim q_\phi(z \mid c, x, y)} [\log p_\theta (x \mid c, y, z) + \log p_\theta (y \mid c, z)]  {}\\
&- D_\mathit{KL}(q_\phi(z \mid c, x, y) \| p_\theta(z \mid c)).
\end{split}
\label{eq:backward-loss}
\end{equation}


\subsubsection{Optimizing dialogue systems bidirectionally} Since the variable $z$ is sampled from the shared latent space between forward generation and backward reasoning steps, we can regard $z$ as a bridge to connect the training in two different direction and this opens the possibility to train dialogue models effectively. By merging Eq.~\ref{eq:forward-loss} and Eq~\ref{eq:backward-loss}, we can rewrite the lower bound Eq.~\ref{eq:loss-lb} as:
%
\begin{equation}
\begin{split}
\mbox{}\hspace*{-2mm}
\log {} &p(x,y \mid c) \geq \mathbb{E}_{z \sim q_\phi(z \mid c, x, y)} \left[
 \frac{1}{2} \log p_\theta (x \mid c, z, y)  \right. \\
& \phantom{XX}+ \frac{1}{2} \log p_\theta (y \mid c, z)  + \frac{1}{2}\log p_\theta (y \mid c, z, x) \\
& \phantom{XX}+ \left.\frac{1}{2} \log p_\theta (x \mid c, z) -D_\mathit{KL}(q_\phi(z \mid c, x, y) \| p_\theta(z \mid c))\vphantom{\frac{1}{2}}\right]\\
 ={}&L(c, x, y; \theta, \phi),
\end{split}
\label{eq:loss}
\end{equation}
%
%$L(c, x, y; \theta, \phi)$
which is the final loss function for our dialogue generation model. 

\subsubsection{Model architecture} The complete architecture of the proposed joint training process is shown in Figure~\ref{fig:framework}. 
It consists of three steps: (1) information encoding, (2) latent variable generation, and (3) target decoding. 
With respect to the information encoding step, we utilize a context encoder $Enc_{ctx}$ to compress the dialogue context $c$ while an utterance encoder $Enc_{utt}$ is used to compress the query $x$ and response $y$, respectively.
To model the latent variable $z$, we assume $z$ follows the multivariate normal distribution, the posterior network $q_\phi(z\mid c, x, y) \sim N(\mu, \sigma^2 I)$ and the prior network $p_\theta(z \mid c) \sim N(\mu^\prime, \sigma^{\prime2} I)$. Then, by applying the reparameterization trick \citep{kingma2013auto}, we can sample a latent variable $z$ from the estimated posterior distribution $N(\mu, \sigma^2 \bm{I})$. During testing, we use the prior distribution $N(\mu^\prime, \sigma^{\prime2} \bm{I})$ to generate the variable $z$. The KL-divergence distance is applied to encourage the approximated posterior $N(\mu, \sigma^2 \bm{I})$ to be close to the prior $N(\mu^\prime, \sigma^{\prime2} \bm{I})$. 
According to Eq.~\ref{eq:loss}, the decoding step in the right side of Figure~\ref{fig:framework} consists of four independent decoders, $Dec_1$, $Dec_2$, $Dec_3$, and $Dec_4$, corresponding to $\log p(y\mid c,z,x)$, $\log p(x\mid c,z)$, $\log p(x\mid c,z,y)$ and $\log p(y\mid c,z)$, respectively. Decoder $Dec_1$ is used to generate the final response during the testing stage. To make full use of the variable $z$, we attach it to the input of each decoding step. Since we have the shared latent vector $z$ as a bridge, training for the two directions is not independent, and updating one direction will definitely improve the other direction as well. In the end, both directions will contribute to the final dialogue generation process.


\section{Experimental Setup}
\label{sec:experiments}
\subsection{Datasets}
We use two datasets.
First, the MovieTriples dataset~\citep{serban2016hred} has been developed by expanding and preprocessing the Movie-Dic corpus~\citep{banchs2012movie} of film transcripts and each dialogue consists of 3 turns between two speakers. We regard the first turn as the dialogue context while the second and third one as the query and response, respectively. In the final dataset, there are around 166k dialogues in the training set, 21k in the validation set and 20k in the test set. In terms of the vocabulary table size, we set it to the top 20k most frequent words in the dataset.

Second, the DailyDialog dataset~\citep{li2017dailydialog} is a high-quality multi-turn dialogue dataset. We split the dialogues in the original dataset into shorter dialogues by every three turns as a new dialogue. The last turn is used as the target response and the first as the context and the third one as the query. After preprocessing, we have 65k, 6k, and 6k dialogs in the training, testing and validation sets, respectively. We limit the vocabulary table size to the top 20k most frequent words for the DailyDialog dataset.

\subsection{Baselines}
%For comparison, we consider six baselines.
\begin{description}[leftmargin=\parindent,nosep]
\item[\textbf{Seq2SeqAtt}] This is a LSTM-based~\citep{hochreiter1997long} dialogue generation model with attention mechanism~\citep{bahdanau2014neural}.
\item[\textbf{HRED}] This method~\citep{serban2016hred} uses a hierarchical recurrent encoder-decoder to sequentially generate the tokens in the replies.
\item[\textbf{VHRED}] This extension of HRED incorporates a stochastic latent variable to explicitly model generative processes that possess multiple levels of variability~\citep{serban2017hierarchical}. This is also the model trained with Eq.~\ref{eq:loss-vhred}.
\item[\textbf{MMI}] This method first generates response candidates on a Seq2Seq model trained in the direction of context-to-target, $P(y\mid c,x)$, then re-ranks them using a separately trained Seq2Seq model in the direction of target-to-context, $P(x\mid y)$, to maximize the mutual information~\citep{li2015diversity}.
\item[\textbf{DC}] This method incorporates side information in the form of distributional constraints, including topic constraints and semantic constraints~\citep{baheti2018generating}.
\item[\textbf{DC-MMI}] This method is a combination of \textbf{MMI} and \textbf{DC}, where the decoding step takes into account mutual information together with the proposed distribution constraints in the method \textbf{DC}. 
\end{description}

\subsection{Training details}
We implement our model, \emph{Mirror}\footnote{Codebase: \url{https://github.com/cszmli/mirror-sigir}}, with PyTorch in the OpenNMT framework~\citep{opennmt}. The utterance encoder is a two-layer LSTM~\citep{hochreiter1997long} and the dimension is 1,000. The context encoder has the same architecture as the utterance encoder but the parameters are not shared. The four decoders have the same design but independent parameters, and each one is a two-layer LSTM with 1,000 dimensions. 
In terms of the dimension of the hidden vector $z$, we set it to $160$ for the DailyDialog dataset while $100$ for MovieTriples.  The word embedding size is $200$ for both datasets. We use Adam~\citep{kingma2014adam} as the optimizer. The initial learning rate is $0.001$ and learning rate decay is applied to stabilize the training process.

\subsection{Evaluation}
We conduct a human evaluation 
%of the systems' output 
on Amazon MTurk guided by~\citep{li2019acute}.
For each two-way comparison of dialogue responses (against Mirror), we ask annotators to judge which of two responses is more appropriate given the context. 
For each method pair (Mirror, Baseline) and each dataset, we randomly sample $200$ dialogues from the test datasets; each pair of responses is annotated by $3$ annotators.

\section{Results and Analysis}
\label{sec:results}

In Table~\ref{Table:results}, we show performance comparisons between Mirror and other baselines on two different datasets. According to Table~\ref{Table:results}(top), it is somewhat unexpected to see that HRED can achieve such close performance compared to Mirror on DailyDialog, given its main architecture is a hierarchical encoder-decoder model. We randomly sample some dialogue pairs for which HRED outperforms Mirror to see why annotators prefer HRED over Mirror. For many of these cases, Mirror fails to generate appropriate responses, while HRED returns generic but still acceptable responses given the context. When we have the back reasoning step in Mirror, we expect that it will lead to more informative generations. Still, it also increases the risk of generating responses with incorrect syntax or relevant but inappropriate responses. A possible reason for the latter is that the backward reasoning step has dominated the joint training process, which can degenerate the forward generation performance. 

The performance gap between Mirror and all approaches (including HRED) is large on the DailyDialog dataset (see Table~\ref{Table:results}(bottom)). 
%
\begin{table}[t]
\caption{Human evaluation using the MovieTriple and DailyDialog datasets.} 
\label{Table:results}
  \centering
%  \resizebox{0.85\columnwidth}{!}{
\begin{tabular}{l l*{4}{c}}
\toprule
%\multicolumn{4}{c}{(a) MovieTriple} \\
%\midrule
& \textbf{Method pair} & Wins  & Losses &Ties  \\
\midrule
\parbox[t]{4mm}{\multirow{6}{*}{\rotatebox[origin=c]{90}{\em (a) MovieTriple~}}} 
& Mirror vs. Seq2SeqAttn  &0.53 &0.37 &0.10 \\
& Mirror vs. HRED  &0.41 &0.40 &0.19 \\
& Mirror vs. VHRED  &0.45 &0.38 &0.17 \\
& Mirror vs. MMI  &0.48 &0.42 &0.10 \\
& Mirror vs. DC  &0.50 &0.33 &0.17 \\
& Mirror vs. DC-MMI &0.39 &0.35 &0.26 \\
\midrule
%& \multicolumn{4}{c}{(b) DailyDialog} \\
%\midrule
\parbox[t]{4mm}{\multirow{6}{*}{\rotatebox[origin=c]{90}{\em (b) DailyDialog~}}} 
% & \textbf{Method Pair} & Wins  & Losses &Ties  \\
%\midrule
& Mirror vs. Seq2SeqAttn  &0.50 &0.26 &0.24 \\
& Mirror vs. HRED  &0.49 &0.32 &0.19 \\
& Mirror vs. VHRED  &0.48 &0.37 &0.15 \\
& Mirror vs. MMI  &0.40 &0.34 &0.26 \\
& Mirror vs. DC  &0.45 &0.38 &0.17 \\
& Mirror vs. DC-MMI &0.47 &0.35 &0.18 \\
\bottomrule
\end{tabular} 
 \vspace*{-0.5\baselineskip}
\end{table}
%
Due to space limitations, we only present one dialogue example in Table~\ref{Table:case_table}. The example is a typical case of why the response generated by DC has high embedding scores, but the human evaluation result is not promising. In this example, the response from DC has high semantic similarity with the context because of words like ``ask you'', ``apartment'', and ``questions''. However, it cannot be regarded as an appropriate and meaningful response in the given context. Comparing Mirror with methods that have use MMI (MMI, DC-MMI), the performance gap is relatively small. This is evidence showing the effectiveness of maximizing mutual information in improving the response quality. The Mirror method can be treated as a way to maximize mutual information implicitly. The advantage is that we can train dialogue models in two directions simultaneously.  

%%%%%%%%%%%%%%%%%%%%%%%%%%%%%%%%%
\begin{table}[!ht]
  \centering
\caption{Example generated responses by different models when the dialogue context is given.}
\label{Table:case_table}
\resizebox{0.8\linewidth}{!}{
  \begin{tabular}{c}
    \toprule
    % \textbf{Example 1} & \textbf{Example 2} \\
    % \midrule
    \textbf{Context}\\
    \midrule
    \makecell[lt]{\textbf{Speaker A:} here ' s my license . \\
                   \textbf{Speaker B:} i ' m afraid i ' m going to have to ask you to\\ remain in the apartment . the narcotics squad will be arriving \\any moment now . they want to ask you a few questions . }  \\
    \midrule
    \textbf{Response} \\
    \midrule
    \makecell[lt]{\textbf{Reference:} squad ? what do they want with me ?\\ i don ' t even use aspirin !\\ 
                 \textbf{Seq2Seq:} no .\\
                \textbf{HRED:} i don ' t think so .\\ 
                \textbf{VHRED:} oh , i ' m sorry . \\ 
                \textbf{MMI:} i ' m sorry . i ' m sorry . i don ' t know what you ' re \\talking about . i don ' t know what i ' m afraid of . \\ 
                \textbf{DC:} i ' m not going to ask you . but he will be in the apartment for \\a moment -- and we can have some questions with that one of them !\\
                \textbf{DC-MMI:} i ' m going to ask you . \\
                \textbf{Mirror:} well , i ' m sure they ' ll have to wait . } \\

    \bottomrule
\end{tabular}
}
 \vspace*{-0.5\baselineskip}
\end{table}


\section{Conclusion and Future Work}
\label{sec:conclusion}
We have presented a novel approach to generating informative and coherent responses in open-domain dialogue systems, called \emph{Mirror}. First, we reformulate the original response generation task from two sides: context and response, to three sides: context, query, and response. Given the dialogue context and query, predicting the response is exactly like the traditional dialogue generation setup. Thus, \emph{Mirror} has one more step: inferring the query given the dialogue context and response.
By incorporating the backward reasoning step, we implicitly push the model to generate responses that have closer connections with the dialogue history. 
%
By conducting experiments on two datasets, we have demonstrated that Mirror improves the response quality compared to several competitive baselines without incorporating additional sources of information, which comes with additional computational costs and complexity. 
%
For future work, Mirror's bidirectional training approach can be generalized to other domains, such as task-oriented dialogue systems and question-answering tasks.


% \chapter{Supplementary Material}
\label{appendix}

In this appendix, we present supplementary material for the techniques and
experiments presented in the main text.

\section{Baseline Results and Analysis for Informed Sampler}
\label{appendix:chap3}

Here, we give an in-depth
performance analysis of the various samplers and the effect of their
hyperparameters. We choose hyperparameters with the lowest PSRF value
after $10k$ iterations, for each sampler individually. If the
differences between PSRF are not significantly different among
multiple values, we choose the one that has the highest acceptance
rate.

\subsection{Experiment: Estimating Camera Extrinsics}
\label{appendix:chap3:room}

\subsubsection{Parameter Selection}
\paragraph{Metropolis Hastings (\MH)}

Figure~\ref{fig:exp1_MH} shows the median acceptance rates and PSRF
values corresponding to various proposal standard deviations of plain
\MH~sampling. Mixing gets better and the acceptance rate gets worse as
the standard deviation increases. The value $0.3$ is selected standard
deviation for this sampler.

\paragraph{Metropolis Hastings Within Gibbs (\MHWG)}

As mentioned in Section~\ref{sec:room}, the \MHWG~sampler with one-dimensional
updates did not converge for any value of proposal standard deviation.
This problem has high correlation of the camera parameters and is of
multi-modal nature, which this sampler has problems with.

\paragraph{Parallel Tempering (\PT)}

For \PT~sampling, we took the best performing \MH~sampler and used
different temperature chains to improve the mixing of the
sampler. Figure~\ref{fig:exp1_PT} shows the results corresponding to
different combination of temperature levels. The sampler with
temperature levels of $[1,3,27]$ performed best in terms of both
mixing and acceptance rate.

\paragraph{Effect of Mixture Coefficient in Informed Sampling (\MIXLMH)}

Figure~\ref{fig:exp1_alpha} shows the effect of mixture
coefficient ($\alpha$) on the informed sampling
\MIXLMH. Since there is no significant different in PSRF values for
$0 \le \alpha \le 0.7$, we chose $0.7$ due to its high acceptance
rate.


% \end{multicols}

\begin{figure}[h]
\centering
  \subfigure[MH]{%
    \includegraphics[width=.48\textwidth]{figures/supplementary/camPose_MH.pdf} \label{fig:exp1_MH}
  }
  \subfigure[PT]{%
    \includegraphics[width=.48\textwidth]{figures/supplementary/camPose_PT.pdf} \label{fig:exp1_PT}
  }
\\
  \subfigure[INF-MH]{%
    \includegraphics[width=.48\textwidth]{figures/supplementary/camPose_alpha.pdf} \label{fig:exp1_alpha}
  }
  \mycaption{Results of the `Estimating Camera Extrinsics' experiment}{PRSFs and Acceptance rates corresponding to (a) various standard deviations of \MH, (b) various temperature level combinations of \PT sampling and (c) various mixture coefficients of \MIXLMH sampling.}
\end{figure}



\begin{figure}[!t]
\centering
  \subfigure[\MH]{%
    \includegraphics[width=.48\textwidth]{figures/supplementary/occlusionExp_MH.pdf} \label{fig:exp2_MH}
  }
  \subfigure[\BMHWG]{%
    \includegraphics[width=.48\textwidth]{figures/supplementary/occlusionExp_BMHWG.pdf} \label{fig:exp2_BMHWG}
  }
\\
  \subfigure[\MHWG]{%
    \includegraphics[width=.48\textwidth]{figures/supplementary/occlusionExp_MHWG.pdf} \label{fig:exp2_MHWG}
  }
  \subfigure[\PT]{%
    \includegraphics[width=.48\textwidth]{figures/supplementary/occlusionExp_PT.pdf} \label{fig:exp2_PT}
  }
\\
  \subfigure[\INFBMHWG]{%
    \includegraphics[width=.5\textwidth]{figures/supplementary/occlusionExp_alpha.pdf} \label{fig:exp2_alpha}
  }
  \mycaption{Results of the `Occluding Tiles' experiment}{PRSF and
    Acceptance rates corresponding to various standard deviations of
    (a) \MH, (b) \BMHWG, (c) \MHWG, (d) various temperature level
    combinations of \PT~sampling and; (e) various mixture coefficients
    of our informed \INFBMHWG sampling.}
\end{figure}

%\onecolumn\newpage\twocolumn
\subsection{Experiment: Occluding Tiles}
\label{appendix:chap3:tiles}

\subsubsection{Parameter Selection}

\paragraph{Metropolis Hastings (\MH)}

Figure~\ref{fig:exp2_MH} shows the results of
\MH~sampling. Results show the poor convergence for all proposal
standard deviations and rapid decrease of AR with increasing standard
deviation. This is due to the high-dimensional nature of
the problem. We selected a standard deviation of $1.1$.

\paragraph{Blocked Metropolis Hastings Within Gibbs (\BMHWG)}

The results of \BMHWG are shown in Figure~\ref{fig:exp2_BMHWG}. In
this sampler we update only one block of tile variables (of dimension
four) in each sampling step. Results show much better performance
compared to plain \MH. The optimal proposal standard deviation for
this sampler is $0.7$.

\paragraph{Metropolis Hastings Within Gibbs (\MHWG)}

Figure~\ref{fig:exp2_MHWG} shows the result of \MHWG sampling. This
sampler is better than \BMHWG and converges much more quickly. Here
a standard deviation of $0.9$ is found to be best.

\paragraph{Parallel Tempering (\PT)}

Figure~\ref{fig:exp2_PT} shows the results of \PT sampling with various
temperature combinations. Results show no improvement in AR from plain
\MH sampling and again $[1,3,27]$ temperature levels are found to be optimal.

\paragraph{Effect of Mixture Coefficient in Informed Sampling (\INFBMHWG)}

Figure~\ref{fig:exp2_alpha} shows the effect of mixture
coefficient ($\alpha$) on the blocked informed sampling
\INFBMHWG. Since there is no significant different in PSRF values for
$0 \le \alpha \le 0.8$, we chose $0.8$ due to its high acceptance
rate.



\subsection{Experiment: Estimating Body Shape}
\label{appendix:chap3:body}

\subsubsection{Parameter Selection}
\paragraph{Metropolis Hastings (\MH)}

Figure~\ref{fig:exp3_MH} shows the result of \MH~sampling with various
proposal standard deviations. The value of $0.1$ is found to be
best.

\paragraph{Metropolis Hastings Within Gibbs (\MHWG)}

For \MHWG sampling we select $0.3$ proposal standard
deviation. Results are shown in Fig.~\ref{fig:exp3_MHWG}.


\paragraph{Parallel Tempering (\PT)}

As before, results in Fig.~\ref{fig:exp3_PT}, the temperature levels
were selected to be $[1,3,27]$ due its slightly higher AR.

\paragraph{Effect of Mixture Coefficient in Informed Sampling (\MIXLMH)}

Figure~\ref{fig:exp3_alpha} shows the effect of $\alpha$ on PSRF and
AR. Since there is no significant differences in PSRF values for $0 \le
\alpha \le 0.8$, we choose $0.8$.


\begin{figure}[t]
\centering
  \subfigure[\MH]{%
    \includegraphics[width=.48\textwidth]{figures/supplementary/bodyShape_MH.pdf} \label{fig:exp3_MH}
  }
  \subfigure[\MHWG]{%
    \includegraphics[width=.48\textwidth]{figures/supplementary/bodyShape_MHWG.pdf} \label{fig:exp3_MHWG}
  }
\\
  \subfigure[\PT]{%
    \includegraphics[width=.48\textwidth]{figures/supplementary/bodyShape_PT.pdf} \label{fig:exp3_PT}
  }
  \subfigure[\MIXLMH]{%
    \includegraphics[width=.48\textwidth]{figures/supplementary/bodyShape_alpha.pdf} \label{fig:exp3_alpha}
  }
\\
  \mycaption{Results of the `Body Shape Estimation' experiment}{PRSFs and
    Acceptance rates corresponding to various standard deviations of
    (a) \MH, (b) \MHWG; (c) various temperature level combinations
    of \PT sampling and; (d) various mixture coefficients of the
    informed \MIXLMH sampling.}
\end{figure}


\subsection{Results Overview}
Figure~\ref{fig:exp_summary} shows the summary results of the all the three
experimental studies related to informed sampler.
\begin{figure*}[h!]
\centering
  \subfigure[Results for: Estimating Camera Extrinsics]{%
    \includegraphics[width=0.9\textwidth]{figures/supplementary/camPose_ALL.pdf} \label{fig:exp1_all}
  }
  \subfigure[Results for: Occluding Tiles]{%
    \includegraphics[width=0.9\textwidth]{figures/supplementary/occlusionExp_ALL.pdf} \label{fig:exp2_all}
  }
  \subfigure[Results for: Estimating Body Shape]{%
    \includegraphics[width=0.9\textwidth]{figures/supplementary/bodyShape_ALL.pdf} \label{fig:exp3_all}
  }
  \label{fig:exp_summary}
  \mycaption{Summary of the statistics for the three experiments}{Shown are
    for several baseline methods and the informed samplers the
    acceptance rates (left), PSRFs (middle), and RMSE values
    (right). All results are median results over multiple test
    examples.}
\end{figure*}

\subsection{Additional Qualitative Results}

\subsubsection{Occluding Tiles}
In Figure~\ref{fig:exp2_visual_more} more qualitative results of the
occluding tiles experiment are shown. The informed sampling approach
(\INFBMHWG) is better than the best baseline (\MHWG). This still is a
very challenging problem since the parameters for occluded tiles are
flat over a large region. Some of the posterior variance of the
occluded tiles is already captured by the informed sampler.

\begin{figure*}[h!]
\begin{center}
\centerline{\includegraphics[width=0.95\textwidth]{figures/supplementary/occlusionExp_Visual.pdf}}
\mycaption{Additional qualitative results of the occluding tiles experiment}
  {From left to right: (a)
  Given image, (b) Ground truth tiles, (c) OpenCV heuristic and most probable estimates
  from 5000 samples obtained by (d) MHWG sampler (best baseline) and
  (e) our INF-BMHWG sampler. (f) Posterior expectation of the tiles
  boundaries obtained by INF-BMHWG sampling (First 2000 samples are
  discarded as burn-in).}
\label{fig:exp2_visual_more}
\end{center}
\end{figure*}

\subsubsection{Body Shape}
Figure~\ref{fig:exp3_bodyMeshes} shows some more results of 3D mesh
reconstruction using posterior samples obtained by our informed
sampling \MIXLMH.

\begin{figure*}[t]
\begin{center}
\centerline{\includegraphics[width=0.75\textwidth]{figures/supplementary/bodyMeshResults.pdf}}
\mycaption{Qualitative results for the body shape experiment}
  {Shown is the 3D mesh reconstruction results with first 1000 samples obtained
  using the \MIXLMH informed sampling method. (blue indicates small
  values and red indicates high values)}
\label{fig:exp3_bodyMeshes}
\end{center}
\end{figure*}

\clearpage



\section{Additional Results on the Face Problem with CMP}

Figure~\ref{fig:shading-qualitative-multiple-subjects-supp} shows inference results for reflectance maps, normal maps and lights for randomly chosen test images, and Fig.~\ref{fig:shading-qualitative-same-subject-supp} shows reflectance estimation results on multiple images of the same subject produced under different illumination conditions. CMP is able to produce estimates that are closer to the groundtruth across different subjects and illumination conditions.

\begin{figure*}[h]
  \begin{center}
  \centerline{\includegraphics[width=1.0\columnwidth]{figures/face_cmp_visual_results_supp.pdf}}
  \vspace{-1.2cm}
  \end{center}
	\mycaption{A visual comparison of inference results}{(a)~Observed images. (b)~Inferred reflectance maps. \textit{GT} is the photometric stereo groundtruth, \textit{BU} is the Biswas \etal (2009) reflectance estimate and \textit{Forest} is the consensus prediction. (c)~The variance of the inferred reflectance estimate produced by \MTD (normalized across rows).(d)~Visualization of inferred light directions. (e)~Inferred normal maps.}
	\label{fig:shading-qualitative-multiple-subjects-supp}
\end{figure*}


\begin{figure*}[h]
	\centering
	\setlength\fboxsep{0.2mm}
	\setlength\fboxrule{0pt}
	\begin{tikzpicture}

		\matrix at (0, 0) [matrix of nodes, nodes={anchor=east}, column sep=-0.05cm, row sep=-0.2cm]
		{
			\fbox{\includegraphics[width=1cm]{figures/sample_3_4_X.png}} &
			\fbox{\includegraphics[width=1cm]{figures/sample_3_4_GT.png}} &
			\fbox{\includegraphics[width=1cm]{figures/sample_3_4_BISWAS.png}}  &
			\fbox{\includegraphics[width=1cm]{figures/sample_3_4_VMP.png}}  &
			\fbox{\includegraphics[width=1cm]{figures/sample_3_4_FOREST.png}}  &
			\fbox{\includegraphics[width=1cm]{figures/sample_3_4_CMP.png}}  &
			\fbox{\includegraphics[width=1cm]{figures/sample_3_4_CMPVAR.png}}
			 \\

			\fbox{\includegraphics[width=1cm]{figures/sample_3_5_X.png}} &
			\fbox{\includegraphics[width=1cm]{figures/sample_3_5_GT.png}} &
			\fbox{\includegraphics[width=1cm]{figures/sample_3_5_BISWAS.png}}  &
			\fbox{\includegraphics[width=1cm]{figures/sample_3_5_VMP.png}}  &
			\fbox{\includegraphics[width=1cm]{figures/sample_3_5_FOREST.png}}  &
			\fbox{\includegraphics[width=1cm]{figures/sample_3_5_CMP.png}}  &
			\fbox{\includegraphics[width=1cm]{figures/sample_3_5_CMPVAR.png}}
			 \\

			\fbox{\includegraphics[width=1cm]{figures/sample_3_6_X.png}} &
			\fbox{\includegraphics[width=1cm]{figures/sample_3_6_GT.png}} &
			\fbox{\includegraphics[width=1cm]{figures/sample_3_6_BISWAS.png}}  &
			\fbox{\includegraphics[width=1cm]{figures/sample_3_6_VMP.png}}  &
			\fbox{\includegraphics[width=1cm]{figures/sample_3_6_FOREST.png}}  &
			\fbox{\includegraphics[width=1cm]{figures/sample_3_6_CMP.png}}  &
			\fbox{\includegraphics[width=1cm]{figures/sample_3_6_CMPVAR.png}}
			 \\
	     };

       \node at (-3.85, -2.0) {\small Observed};
       \node at (-2.55, -2.0) {\small `GT'};
       \node at (-1.27, -2.0) {\small BU};
       \node at (0.0, -2.0) {\small MP};
       \node at (1.27, -2.0) {\small Forest};
       \node at (2.55, -2.0) {\small \textbf{CMP}};
       \node at (3.85, -2.0) {\small Variance};

	\end{tikzpicture}
	\mycaption{Robustness to varying illumination}{Reflectance estimation on a subject images with varying illumination. Left to right: observed image, photometric stereo estimate (GT)
  which is used as a proxy for groundtruth, bottom-up estimate of \cite{Biswas2009}, VMP result, consensus forest estimate, CMP mean, and CMP variance.}
	\label{fig:shading-qualitative-same-subject-supp}
\end{figure*}

\clearpage

\section{Additional Material for Learning Sparse High Dimensional Filters}
\label{sec:appendix-bnn}

This part of supplementary material contains a more detailed overview of the permutohedral
lattice convolution in Section~\ref{sec:permconv}, more experiments in
Section~\ref{sec:addexps} and additional results with protocols for
the experiments presented in Chapter~\ref{chap:bnn} in Section~\ref{sec:addresults}.

\vspace{-0.2cm}
\subsection{General Permutohedral Convolutions}
\label{sec:permconv}

A core technical contribution of this work is the generalization of the Gaussian permutohedral lattice
convolution proposed in~\cite{adams2010fast} to the full non-separable case with the
ability to perform back-propagation. Although, conceptually, there are minor
differences between Gaussian and general parameterized filters, there are non-trivial practical
differences in terms of the algorithmic implementation. The Gauss filters belong to
the separable class and can thus be decomposed into multiple
sequential one dimensional convolutions. We are interested in the general filter
convolutions, which can not be decomposed. Thus, performing a general permutohedral
convolution at a lattice point requires the computation of the inner product with the
neighboring elements in all the directions in the high-dimensional space.

Here, we give more details of the implementation differences of separable
and non-separable filters. In the following, we will explain the scalar case first.
Recall, that the forward pass of general permutohedral convolution
involves 3 steps: \textit{splatting}, \textit{convolving} and \textit{slicing}.
We follow the same splatting and slicing strategies as in~\cite{adams2010fast}
since these operations do not depend on the filter kernel. The main difference
between our work and the existing implementation of~\cite{adams2010fast} is
the way that the convolution operation is executed. This proceeds by constructing
a \emph{blur neighbor} matrix $K$ that stores for every lattice point all
values of the lattice neighbors that are needed to compute the filter output.

\begin{figure}[t!]
  \centering
    \includegraphics[width=0.6\columnwidth]{figures/supplementary/lattice_construction}
  \mycaption{Illustration of 1D permutohedral lattice construction}
  {A $4\times 4$ $(x,y)$ grid lattice is projected onto the plane defined by the normal
  vector $(1,1)^{\top}$. This grid has $s+1=4$ and $d=2$ $(s+1)^{d}=4^2=16$ elements.
  In the projection, all points of the same color are projected onto the same points in the plane.
  The number of elements of the projected lattice is $t=(s+1)^d-s^d=4^2-3^2=7$, that is
  the $(4\times 4)$ grid minus the size of lattice that is $1$ smaller at each size, in this
  case a $(3\times 3)$ lattice (the upper right $(3\times 3)$ elements).
  }
\label{fig:latticeconstruction}
\end{figure}

The blur neighbor matrix is constructed by traversing through all the populated
lattice points and their neighboring elements.
% For efficiency, we do this matrix construction recursively with shared computations
% since $n^{th}$ neighbourhood elements are $1^{st}$ neighborhood elements of $n-1^{th}$ neighbourhood elements. \pg{do not understand}
This is done recursively to share computations. For any lattice point, the neighbors that are
$n$ hops away are the direct neighbors of the points that are $n-1$ hops away.
The size of a $d$ dimensional spatial filter with width $s+1$ is $(s+1)^{d}$ (\eg, a
$3\times 3$ filter, $s=2$ in $d=2$ has $3^2=9$ elements) and this size grows
exponentially in the number of dimensions $d$. The permutohedral lattice is constructed by
projecting a regular grid onto the plane spanned by the $d$ dimensional normal vector ${(1,\ldots,1)}^{\top}$. See
Fig.~\ref{fig:latticeconstruction} for an illustration of the 1D lattice construction.
Many corners of a grid filter are projected onto the same point, in total $t = {(s+1)}^{d} -
s^{d}$ elements remain in the permutohedral filter with $s$ neighborhood in $d-1$ dimensions.
If the lattice has $m$ populated elements, the
matrix $K$ has size $t\times m$. Note that, since the input signal is typically
sparse, only a few lattice corners are being populated in the \textit{slicing} step.
We use a hash-table to keep track of these points and traverse only through
the populated lattice points for this neighborhood matrix construction.

Once the blur neighbor matrix $K$ is constructed, we can perform the convolution
by the matrix vector multiplication
\begin{equation}
\ell' = BK,
\label{eq:conv}
\end{equation}
where $B$ is the $1 \times t$ filter kernel (whose values we will learn) and $\ell'\in\mathbb{R}^{1\times m}$
is the result of the filtering at the $m$ lattice points. In practice, we found that the
matrix $K$ is sometimes too large to fit into GPU memory and we divided the matrix $K$
into smaller pieces to compute Eq.~\ref{eq:conv} sequentially.

In the general multi-dimensional case, the signal $\ell$ is of $c$ dimensions. Then
the kernel $B$ is of size $c \times t$ and $K$ stores the $c$ dimensional vectors
accordingly. When the input and output points are different, we slice only the
input points and splat only at the output points.


\subsection{Additional Experiments}
\label{sec:addexps}
In this section, we discuss more use-cases for the learned bilateral filters, one
use-case of BNNs and two single filter applications for image and 3D mesh denoising.

\subsubsection{Recognition of subsampled MNIST}\label{sec:app_mnist}

One of the strengths of the proposed filter convolution is that it does not
require the input to lie on a regular grid. The only requirement is to define a distance
between features of the input signal.
We highlight this feature with the following experiment using the
classical MNIST ten class classification problem~\cite{lecun1998mnist}. We sample a
sparse set of $N$ points $(x,y)\in [0,1]\times [0,1]$
uniformly at random in the input image, use their interpolated values
as signal and the \emph{continuous} $(x,y)$ positions as features. This mimics
sub-sampling of a high-dimensional signal. To compare against a spatial convolution,
we interpolate the sparse set of values at the grid positions.

We take a reference implementation of LeNet~\cite{lecun1998gradient} that
is part of the Caffe project~\cite{jia2014caffe} and compare it
against the same architecture but replacing the first convolutional
layer with a bilateral convolution layer (BCL). The filter size
and numbers are adjusted to get a comparable number of parameters
($5\times 5$ for LeNet, $2$-neighborhood for BCL).

The results are shown in Table~\ref{tab:all-results}. We see that training
on the original MNIST data (column Original, LeNet vs. BNN) leads to a slight
decrease in performance of the BNN (99.03\%) compared to LeNet
(99.19\%). The BNN can be trained and evaluated on sparse
signals, and we resample the image as described above for $N=$ 100\%, 60\% and
20\% of the total number of pixels. The methods are also evaluated
on test images that are subsampled in the same way. Note that we can
train and test with different subsampling rates. We introduce an additional
bilinear interpolation layer for the LeNet architecture to train on the same
data. In essence, both models perform a spatial interpolation and thus we
expect them to yield a similar classification accuracy. Once the data is of
higher dimensions, the permutohedral convolution will be faster due to hashing
the sparse input points, as well as less memory demanding in comparison to
naive application of a spatial convolution with interpolated values.

\begin{table}[t]
  \begin{center}
    \footnotesize
    \centering
    \begin{tabular}[t]{lllll}
      \toprule
              &     & \multicolumn{3}{c}{Test Subsampling} \\
       Method  & Original & 100\% & 60\% & 20\%\\
      \midrule
       LeNet &  \textbf{0.9919} & 0.9660 & 0.9348 & \textbf{0.6434} \\
       BNN &  0.9903 & \textbf{0.9844} & \textbf{0.9534} & 0.5767 \\
      \hline
       LeNet 100\% & 0.9856 & 0.9809 & 0.9678 & \textbf{0.7386} \\
       BNN 100\% & \textbf{0.9900} & \textbf{0.9863} & \textbf{0.9699} & 0.6910 \\
      \hline
       LeNet 60\% & 0.9848 & 0.9821 & 0.9740 & 0.8151 \\
       BNN 60\% & \textbf{0.9885} & \textbf{0.9864} & \textbf{0.9771} & \textbf{0.8214}\\
      \hline
       LeNet 20\% & \textbf{0.9763} & \textbf{0.9754} & 0.9695 & 0.8928 \\
       BNN 20\% & 0.9728 & 0.9735 & \textbf{0.9701} & \textbf{0.9042}\\
      \bottomrule
    \end{tabular}
  \end{center}
\vspace{-.2cm}
\caption{Classification accuracy on MNIST. We compare the
    LeNet~\cite{lecun1998gradient} implementation that is part of
    Caffe~\cite{jia2014caffe} to the network with the first layer
    replaced by a bilateral convolution layer (BCL). Both are trained
    on the original image resolution (first two rows). Three more BNN
    and CNN models are trained with randomly subsampled images (100\%,
    60\% and 20\% of the pixels). An additional bilinear interpolation
    layer samples the input signal on a spatial grid for the CNN model.
  }
  \label{tab:all-results}
\vspace{-.5cm}
\end{table}

\subsubsection{Image Denoising}

The main application that inspired the development of the bilateral
filtering operation is image denoising~\cite{aurich1995non}, there
using a single Gaussian kernel. Our development allows to learn this
kernel function from data and we explore how to improve using a \emph{single}
but more general bilateral filter.

We use the Berkeley segmentation dataset
(BSDS500)~\cite{arbelaezi2011bsds500} as a test bed. The color
images in the dataset are converted to gray-scale,
and corrupted with Gaussian noise with a standard deviation of
$\frac {25} {255}$.

We compare the performance of four different filter models on a
denoising task.
The first baseline model (`Spatial' in Table \ref{tab:denoising}, $25$
weights) uses a single spatial filter with a kernel size of
$5$ and predicts the scalar gray-scale value at the center pixel. The next model
(`Gauss Bilateral') applies a bilateral \emph{Gaussian}
filter to the noisy input, using position and intensity features $\f=(x,y,v)^\top$.
The third setup (`Learned Bilateral', $65$ weights)
takes a Gauss kernel as initialization and
fits all filter weights on the train set to minimize the
mean squared error with respect to the clean images.
We run a combination
of spatial and permutohedral convolutions on spatial and bilateral
features (`Spatial + Bilateral (Learned)') to check for a complementary
performance of the two convolutions.

\label{sec:exp:denoising}
\begin{table}[!h]
\begin{center}
  \footnotesize
  \begin{tabular}[t]{lr}
    \toprule
    Method & PSNR \\
    \midrule
    Noisy Input & $20.17$ \\
    Spatial & $26.27$ \\
    Gauss Bilateral & $26.51$ \\
    Learned Bilateral & $26.58$ \\
    Spatial + Bilateral (Learned) & \textbf{$26.65$} \\
    \bottomrule
  \end{tabular}
\end{center}
\vspace{-0.5em}
\caption{PSNR results of a denoising task using the BSDS500
  dataset~\cite{arbelaezi2011bsds500}}
\vspace{-0.5em}
\label{tab:denoising}
\end{table}
\vspace{-0.2em}

The PSNR scores evaluated on full images of the test set are
shown in Table \ref{tab:denoising}. We find that an untrained bilateral
filter already performs better than a trained spatial convolution
($26.27$ to $26.51$). A learned convolution further improve the
performance slightly. We chose this simple one-kernel setup to
validate an advantage of the generalized bilateral filter. A competitive
denoising system would employ RGB color information and also
needs to be properly adjusted in network size. Multi-layer perceptrons
have obtained state-of-the-art denoising results~\cite{burger12cvpr}
and the permutohedral lattice layer can readily be used in such an
architecture, which is intended future work.

\subsection{Additional results}
\label{sec:addresults}

This section contains more qualitative results for the experiments presented in Chapter~\ref{chap:bnn}.

\begin{figure*}[th!]
  \centering
    \includegraphics[width=\columnwidth,trim={5cm 2.5cm 5cm 4.5cm},clip]{figures/supplementary/lattice_viz.pdf}
    \vspace{-0.7cm}
  \mycaption{Visualization of the Permutohedral Lattice}
  {Sample lattice visualizations for different feature spaces. All pixels falling in the same simplex cell are shown with
  the same color. $(x,y)$ features correspond to image pixel positions, and $(r,g,b) \in [0,255]$ correspond
  to the red, green and blue color values.}
\label{fig:latticeviz}
\end{figure*}

\subsubsection{Lattice Visualization}

Figure~\ref{fig:latticeviz} shows sample lattice visualizations for different feature spaces.

\newcolumntype{L}[1]{>{\raggedright\let\newline\\\arraybackslash\hspace{0pt}}b{#1}}
\newcolumntype{C}[1]{>{\centering\let\newline\\\arraybackslash\hspace{0pt}}b{#1}}
\newcolumntype{R}[1]{>{\raggedleft\let\newline\\\arraybackslash\hspace{0pt}}b{#1}}

\subsubsection{Color Upsampling}\label{sec:color_upsampling}
\label{sec:col_upsample_extra}

Some images of the upsampling for the Pascal
VOC12 dataset are shown in Fig.~\ref{fig:Colour_upsample_visuals}. It is
especially the low level image details that are better preserved with
a learned bilateral filter compared to the Gaussian case.

\begin{figure*}[t!]
  \centering
    \subfigure{%
   \raisebox{2.0em}{
    \includegraphics[width=.06\columnwidth]{figures/supplementary/2007_004969.jpg}
   }
  }
  \subfigure{%
    \includegraphics[width=.17\columnwidth]{figures/supplementary/2007_004969_gray.pdf}
  }
  \subfigure{%
    \includegraphics[width=.17\columnwidth]{figures/supplementary/2007_004969_gt.pdf}
  }
  \subfigure{%
    \includegraphics[width=.17\columnwidth]{figures/supplementary/2007_004969_bicubic.pdf}
  }
  \subfigure{%
    \includegraphics[width=.17\columnwidth]{figures/supplementary/2007_004969_gauss.pdf}
  }
  \subfigure{%
    \includegraphics[width=.17\columnwidth]{figures/supplementary/2007_004969_learnt.pdf}
  }\\
    \subfigure{%
   \raisebox{2.0em}{
    \includegraphics[width=.06\columnwidth]{figures/supplementary/2007_003106.jpg}
   }
  }
  \subfigure{%
    \includegraphics[width=.17\columnwidth]{figures/supplementary/2007_003106_gray.pdf}
  }
  \subfigure{%
    \includegraphics[width=.17\columnwidth]{figures/supplementary/2007_003106_gt.pdf}
  }
  \subfigure{%
    \includegraphics[width=.17\columnwidth]{figures/supplementary/2007_003106_bicubic.pdf}
  }
  \subfigure{%
    \includegraphics[width=.17\columnwidth]{figures/supplementary/2007_003106_gauss.pdf}
  }
  \subfigure{%
    \includegraphics[width=.17\columnwidth]{figures/supplementary/2007_003106_learnt.pdf}
  }\\
  \setcounter{subfigure}{0}
  \small{
  \subfigure[Inp.]{%
  \raisebox{2.0em}{
    \includegraphics[width=.06\columnwidth]{figures/supplementary/2007_006837.jpg}
   }
  }
  \subfigure[Guidance]{%
    \includegraphics[width=.17\columnwidth]{figures/supplementary/2007_006837_gray.pdf}
  }
   \subfigure[GT]{%
    \includegraphics[width=.17\columnwidth]{figures/supplementary/2007_006837_gt.pdf}
  }
  \subfigure[Bicubic]{%
    \includegraphics[width=.17\columnwidth]{figures/supplementary/2007_006837_bicubic.pdf}
  }
  \subfigure[Gauss-BF]{%
    \includegraphics[width=.17\columnwidth]{figures/supplementary/2007_006837_gauss.pdf}
  }
  \subfigure[Learned-BF]{%
    \includegraphics[width=.17\columnwidth]{figures/supplementary/2007_006837_learnt.pdf}
  }
  }
  \vspace{-0.5cm}
  \mycaption{Color Upsampling}{Color $8\times$ upsampling results
  using different methods, from left to right, (a)~Low-resolution input color image (Inp.),
  (b)~Gray scale guidance image, (c)~Ground-truth color image; Upsampled color images with
  (d)~Bicubic interpolation, (e) Gauss bilateral upsampling and, (f)~Learned bilateral
  updampgling (best viewed on screen).}

\label{fig:Colour_upsample_visuals}
\end{figure*}

\subsubsection{Depth Upsampling}
\label{sec:depth_upsample_extra}

Figure~\ref{fig:depth_upsample_visuals} presents some more qualitative results comparing bicubic interpolation, Gauss
bilateral and learned bilateral upsampling on NYU depth dataset image~\cite{silberman2012indoor}.

\subsubsection{Character Recognition}\label{sec:app_character}

 Figure~\ref{fig:nnrecognition} shows the schematic of different layers
 of the network architecture for LeNet-7~\cite{lecun1998mnist}
 and DeepCNet(5, 50)~\cite{ciresan2012multi,graham2014spatially}. For the BNN variants, the first layer filters are replaced
 with learned bilateral filters and are learned end-to-end.

\subsubsection{Semantic Segmentation}\label{sec:app_semantic_segmentation}
\label{sec:semantic_bnn_extra}

Some more visual results for semantic segmentation are shown in Figure~\ref{fig:semantic_visuals}.
These include the underlying DeepLab CNN\cite{chen2014semantic} result (DeepLab),
the 2 step mean-field result with Gaussian edge potentials (+2stepMF-GaussCRF)
and also corresponding results with learned edge potentials (+2stepMF-LearnedCRF).
In general, we observe that mean-field in learned CRF leads to slightly dilated
classification regions in comparison to using Gaussian CRF thereby filling-in the
false negative pixels and also correcting some mis-classified regions.

\begin{figure*}[t!]
  \centering
    \subfigure{%
   \raisebox{2.0em}{
    \includegraphics[width=.06\columnwidth]{figures/supplementary/2bicubic}
   }
  }
  \subfigure{%
    \includegraphics[width=.17\columnwidth]{figures/supplementary/2given_image}
  }
  \subfigure{%
    \includegraphics[width=.17\columnwidth]{figures/supplementary/2ground_truth}
  }
  \subfigure{%
    \includegraphics[width=.17\columnwidth]{figures/supplementary/2bicubic}
  }
  \subfigure{%
    \includegraphics[width=.17\columnwidth]{figures/supplementary/2gauss}
  }
  \subfigure{%
    \includegraphics[width=.17\columnwidth]{figures/supplementary/2learnt}
  }\\
    \subfigure{%
   \raisebox{2.0em}{
    \includegraphics[width=.06\columnwidth]{figures/supplementary/32bicubic}
   }
  }
  \subfigure{%
    \includegraphics[width=.17\columnwidth]{figures/supplementary/32given_image}
  }
  \subfigure{%
    \includegraphics[width=.17\columnwidth]{figures/supplementary/32ground_truth}
  }
  \subfigure{%
    \includegraphics[width=.17\columnwidth]{figures/supplementary/32bicubic}
  }
  \subfigure{%
    \includegraphics[width=.17\columnwidth]{figures/supplementary/32gauss}
  }
  \subfigure{%
    \includegraphics[width=.17\columnwidth]{figures/supplementary/32learnt}
  }\\
  \setcounter{subfigure}{0}
  \small{
  \subfigure[Inp.]{%
  \raisebox{2.0em}{
    \includegraphics[width=.06\columnwidth]{figures/supplementary/41bicubic}
   }
  }
  \subfigure[Guidance]{%
    \includegraphics[width=.17\columnwidth]{figures/supplementary/41given_image}
  }
   \subfigure[GT]{%
    \includegraphics[width=.17\columnwidth]{figures/supplementary/41ground_truth}
  }
  \subfigure[Bicubic]{%
    \includegraphics[width=.17\columnwidth]{figures/supplementary/41bicubic}
  }
  \subfigure[Gauss-BF]{%
    \includegraphics[width=.17\columnwidth]{figures/supplementary/41gauss}
  }
  \subfigure[Learned-BF]{%
    \includegraphics[width=.17\columnwidth]{figures/supplementary/41learnt}
  }
  }
  \mycaption{Depth Upsampling}{Depth $8\times$ upsampling results
  using different upsampling strategies, from left to right,
  (a)~Low-resolution input depth image (Inp.),
  (b)~High-resolution guidance image, (c)~Ground-truth depth; Upsampled depth images with
  (d)~Bicubic interpolation, (e) Gauss bilateral upsampling and, (f)~Learned bilateral
  updampgling (best viewed on screen).}

\label{fig:depth_upsample_visuals}
\end{figure*}

\subsubsection{Material Segmentation}\label{sec:app_material_segmentation}
\label{sec:material_bnn_extra}

In Fig.~\ref{fig:material_visuals-app2}, we present visual results comparing 2 step
mean-field inference with Gaussian and learned pairwise CRF potentials. In
general, we observe that the pixels belonging to dominant classes in the
training data are being more accurately classified with learned CRF. This leads to
a significant improvements in overall pixel accuracy. This also results
in a slight decrease of the accuracy from less frequent class pixels thereby
slightly reducing the average class accuracy with learning. We attribute this
to the type of annotation that is available for this dataset, which is not
for the entire image but for some segments in the image. We have very few
images of the infrequent classes to combat this behaviour during training.

\subsubsection{Experiment Protocols}
\label{sec:protocols}

Table~\ref{tbl:parameters} shows experiment protocols of different experiments.

 \begin{figure*}[t!]
  \centering
  \subfigure[LeNet-7]{
    \includegraphics[width=0.7\columnwidth]{figures/supplementary/lenet_cnn_network}
    }\\
    \subfigure[DeepCNet]{
    \includegraphics[width=\columnwidth]{figures/supplementary/deepcnet_cnn_network}
    }
  \mycaption{CNNs for Character Recognition}
  {Schematic of (top) LeNet-7~\cite{lecun1998mnist} and (bottom) DeepCNet(5,50)~\cite{ciresan2012multi,graham2014spatially} architectures used in Assamese
  character recognition experiments.}
\label{fig:nnrecognition}
\end{figure*}

\definecolor{voc_1}{RGB}{0, 0, 0}
\definecolor{voc_2}{RGB}{128, 0, 0}
\definecolor{voc_3}{RGB}{0, 128, 0}
\definecolor{voc_4}{RGB}{128, 128, 0}
\definecolor{voc_5}{RGB}{0, 0, 128}
\definecolor{voc_6}{RGB}{128, 0, 128}
\definecolor{voc_7}{RGB}{0, 128, 128}
\definecolor{voc_8}{RGB}{128, 128, 128}
\definecolor{voc_9}{RGB}{64, 0, 0}
\definecolor{voc_10}{RGB}{192, 0, 0}
\definecolor{voc_11}{RGB}{64, 128, 0}
\definecolor{voc_12}{RGB}{192, 128, 0}
\definecolor{voc_13}{RGB}{64, 0, 128}
\definecolor{voc_14}{RGB}{192, 0, 128}
\definecolor{voc_15}{RGB}{64, 128, 128}
\definecolor{voc_16}{RGB}{192, 128, 128}
\definecolor{voc_17}{RGB}{0, 64, 0}
\definecolor{voc_18}{RGB}{128, 64, 0}
\definecolor{voc_19}{RGB}{0, 192, 0}
\definecolor{voc_20}{RGB}{128, 192, 0}
\definecolor{voc_21}{RGB}{0, 64, 128}
\definecolor{voc_22}{RGB}{128, 64, 128}

\begin{figure*}[t]
  \centering
  \small{
  \fcolorbox{white}{voc_1}{\rule{0pt}{6pt}\rule{6pt}{0pt}} Background~~
  \fcolorbox{white}{voc_2}{\rule{0pt}{6pt}\rule{6pt}{0pt}} Aeroplane~~
  \fcolorbox{white}{voc_3}{\rule{0pt}{6pt}\rule{6pt}{0pt}} Bicycle~~
  \fcolorbox{white}{voc_4}{\rule{0pt}{6pt}\rule{6pt}{0pt}} Bird~~
  \fcolorbox{white}{voc_5}{\rule{0pt}{6pt}\rule{6pt}{0pt}} Boat~~
  \fcolorbox{white}{voc_6}{\rule{0pt}{6pt}\rule{6pt}{0pt}} Bottle~~
  \fcolorbox{white}{voc_7}{\rule{0pt}{6pt}\rule{6pt}{0pt}} Bus~~
  \fcolorbox{white}{voc_8}{\rule{0pt}{6pt}\rule{6pt}{0pt}} Car~~ \\
  \fcolorbox{white}{voc_9}{\rule{0pt}{6pt}\rule{6pt}{0pt}} Cat~~
  \fcolorbox{white}{voc_10}{\rule{0pt}{6pt}\rule{6pt}{0pt}} Chair~~
  \fcolorbox{white}{voc_11}{\rule{0pt}{6pt}\rule{6pt}{0pt}} Cow~~
  \fcolorbox{white}{voc_12}{\rule{0pt}{6pt}\rule{6pt}{0pt}} Dining Table~~
  \fcolorbox{white}{voc_13}{\rule{0pt}{6pt}\rule{6pt}{0pt}} Dog~~
  \fcolorbox{white}{voc_14}{\rule{0pt}{6pt}\rule{6pt}{0pt}} Horse~~
  \fcolorbox{white}{voc_15}{\rule{0pt}{6pt}\rule{6pt}{0pt}} Motorbike~~
  \fcolorbox{white}{voc_16}{\rule{0pt}{6pt}\rule{6pt}{0pt}} Person~~ \\
  \fcolorbox{white}{voc_17}{\rule{0pt}{6pt}\rule{6pt}{0pt}} Potted Plant~~
  \fcolorbox{white}{voc_18}{\rule{0pt}{6pt}\rule{6pt}{0pt}} Sheep~~
  \fcolorbox{white}{voc_19}{\rule{0pt}{6pt}\rule{6pt}{0pt}} Sofa~~
  \fcolorbox{white}{voc_20}{\rule{0pt}{6pt}\rule{6pt}{0pt}} Train~~
  \fcolorbox{white}{voc_21}{\rule{0pt}{6pt}\rule{6pt}{0pt}} TV monitor~~ \\
  }
  \subfigure{%
    \includegraphics[width=.18\columnwidth]{figures/supplementary/2007_001423_given.jpg}
  }
  \subfigure{%
    \includegraphics[width=.18\columnwidth]{figures/supplementary/2007_001423_gt.png}
  }
  \subfigure{%
    \includegraphics[width=.18\columnwidth]{figures/supplementary/2007_001423_cnn.png}
  }
  \subfigure{%
    \includegraphics[width=.18\columnwidth]{figures/supplementary/2007_001423_gauss.png}
  }
  \subfigure{%
    \includegraphics[width=.18\columnwidth]{figures/supplementary/2007_001423_learnt.png}
  }\\
  \subfigure{%
    \includegraphics[width=.18\columnwidth]{figures/supplementary/2007_001430_given.jpg}
  }
  \subfigure{%
    \includegraphics[width=.18\columnwidth]{figures/supplementary/2007_001430_gt.png}
  }
  \subfigure{%
    \includegraphics[width=.18\columnwidth]{figures/supplementary/2007_001430_cnn.png}
  }
  \subfigure{%
    \includegraphics[width=.18\columnwidth]{figures/supplementary/2007_001430_gauss.png}
  }
  \subfigure{%
    \includegraphics[width=.18\columnwidth]{figures/supplementary/2007_001430_learnt.png}
  }\\
    \subfigure{%
    \includegraphics[width=.18\columnwidth]{figures/supplementary/2007_007996_given.jpg}
  }
  \subfigure{%
    \includegraphics[width=.18\columnwidth]{figures/supplementary/2007_007996_gt.png}
  }
  \subfigure{%
    \includegraphics[width=.18\columnwidth]{figures/supplementary/2007_007996_cnn.png}
  }
  \subfigure{%
    \includegraphics[width=.18\columnwidth]{figures/supplementary/2007_007996_gauss.png}
  }
  \subfigure{%
    \includegraphics[width=.18\columnwidth]{figures/supplementary/2007_007996_learnt.png}
  }\\
   \subfigure{%
    \includegraphics[width=.18\columnwidth]{figures/supplementary/2010_002682_given.jpg}
  }
  \subfigure{%
    \includegraphics[width=.18\columnwidth]{figures/supplementary/2010_002682_gt.png}
  }
  \subfigure{%
    \includegraphics[width=.18\columnwidth]{figures/supplementary/2010_002682_cnn.png}
  }
  \subfigure{%
    \includegraphics[width=.18\columnwidth]{figures/supplementary/2010_002682_gauss.png}
  }
  \subfigure{%
    \includegraphics[width=.18\columnwidth]{figures/supplementary/2010_002682_learnt.png}
  }\\
     \subfigure{%
    \includegraphics[width=.18\columnwidth]{figures/supplementary/2010_004789_given.jpg}
  }
  \subfigure{%
    \includegraphics[width=.18\columnwidth]{figures/supplementary/2010_004789_gt.png}
  }
  \subfigure{%
    \includegraphics[width=.18\columnwidth]{figures/supplementary/2010_004789_cnn.png}
  }
  \subfigure{%
    \includegraphics[width=.18\columnwidth]{figures/supplementary/2010_004789_gauss.png}
  }
  \subfigure{%
    \includegraphics[width=.18\columnwidth]{figures/supplementary/2010_004789_learnt.png}
  }\\
       \subfigure{%
    \includegraphics[width=.18\columnwidth]{figures/supplementary/2007_001311_given.jpg}
  }
  \subfigure{%
    \includegraphics[width=.18\columnwidth]{figures/supplementary/2007_001311_gt.png}
  }
  \subfigure{%
    \includegraphics[width=.18\columnwidth]{figures/supplementary/2007_001311_cnn.png}
  }
  \subfigure{%
    \includegraphics[width=.18\columnwidth]{figures/supplementary/2007_001311_gauss.png}
  }
  \subfigure{%
    \includegraphics[width=.18\columnwidth]{figures/supplementary/2007_001311_learnt.png}
  }\\
  \setcounter{subfigure}{0}
  \subfigure[Input]{%
    \includegraphics[width=.18\columnwidth]{figures/supplementary/2010_003531_given.jpg}
  }
  \subfigure[Ground Truth]{%
    \includegraphics[width=.18\columnwidth]{figures/supplementary/2010_003531_gt.png}
  }
  \subfigure[DeepLab]{%
    \includegraphics[width=.18\columnwidth]{figures/supplementary/2010_003531_cnn.png}
  }
  \subfigure[+GaussCRF]{%
    \includegraphics[width=.18\columnwidth]{figures/supplementary/2010_003531_gauss.png}
  }
  \subfigure[+LearnedCRF]{%
    \includegraphics[width=.18\columnwidth]{figures/supplementary/2010_003531_learnt.png}
  }
  \vspace{-0.3cm}
  \mycaption{Semantic Segmentation}{Example results of semantic segmentation.
  (c)~depicts the unary results before application of MF, (d)~after two steps of MF with Gaussian edge CRF potentials, (e)~after
  two steps of MF with learned edge CRF potentials.}
    \label{fig:semantic_visuals}
\end{figure*}


\definecolor{minc_1}{HTML}{771111}
\definecolor{minc_2}{HTML}{CAC690}
\definecolor{minc_3}{HTML}{EEEEEE}
\definecolor{minc_4}{HTML}{7C8FA6}
\definecolor{minc_5}{HTML}{597D31}
\definecolor{minc_6}{HTML}{104410}
\definecolor{minc_7}{HTML}{BB819C}
\definecolor{minc_8}{HTML}{D0CE48}
\definecolor{minc_9}{HTML}{622745}
\definecolor{minc_10}{HTML}{666666}
\definecolor{minc_11}{HTML}{D54A31}
\definecolor{minc_12}{HTML}{101044}
\definecolor{minc_13}{HTML}{444126}
\definecolor{minc_14}{HTML}{75D646}
\definecolor{minc_15}{HTML}{DD4348}
\definecolor{minc_16}{HTML}{5C8577}
\definecolor{minc_17}{HTML}{C78472}
\definecolor{minc_18}{HTML}{75D6D0}
\definecolor{minc_19}{HTML}{5B4586}
\definecolor{minc_20}{HTML}{C04393}
\definecolor{minc_21}{HTML}{D69948}
\definecolor{minc_22}{HTML}{7370D8}
\definecolor{minc_23}{HTML}{7A3622}
\definecolor{minc_24}{HTML}{000000}

\begin{figure*}[t]
  \centering
  \small{
  \fcolorbox{white}{minc_1}{\rule{0pt}{6pt}\rule{6pt}{0pt}} Brick~~
  \fcolorbox{white}{minc_2}{\rule{0pt}{6pt}\rule{6pt}{0pt}} Carpet~~
  \fcolorbox{white}{minc_3}{\rule{0pt}{6pt}\rule{6pt}{0pt}} Ceramic~~
  \fcolorbox{white}{minc_4}{\rule{0pt}{6pt}\rule{6pt}{0pt}} Fabric~~
  \fcolorbox{white}{minc_5}{\rule{0pt}{6pt}\rule{6pt}{0pt}} Foliage~~
  \fcolorbox{white}{minc_6}{\rule{0pt}{6pt}\rule{6pt}{0pt}} Food~~
  \fcolorbox{white}{minc_7}{\rule{0pt}{6pt}\rule{6pt}{0pt}} Glass~~
  \fcolorbox{white}{minc_8}{\rule{0pt}{6pt}\rule{6pt}{0pt}} Hair~~ \\
  \fcolorbox{white}{minc_9}{\rule{0pt}{6pt}\rule{6pt}{0pt}} Leather~~
  \fcolorbox{white}{minc_10}{\rule{0pt}{6pt}\rule{6pt}{0pt}} Metal~~
  \fcolorbox{white}{minc_11}{\rule{0pt}{6pt}\rule{6pt}{0pt}} Mirror~~
  \fcolorbox{white}{minc_12}{\rule{0pt}{6pt}\rule{6pt}{0pt}} Other~~
  \fcolorbox{white}{minc_13}{\rule{0pt}{6pt}\rule{6pt}{0pt}} Painted~~
  \fcolorbox{white}{minc_14}{\rule{0pt}{6pt}\rule{6pt}{0pt}} Paper~~
  \fcolorbox{white}{minc_15}{\rule{0pt}{6pt}\rule{6pt}{0pt}} Plastic~~\\
  \fcolorbox{white}{minc_16}{\rule{0pt}{6pt}\rule{6pt}{0pt}} Polished Stone~~
  \fcolorbox{white}{minc_17}{\rule{0pt}{6pt}\rule{6pt}{0pt}} Skin~~
  \fcolorbox{white}{minc_18}{\rule{0pt}{6pt}\rule{6pt}{0pt}} Sky~~
  \fcolorbox{white}{minc_19}{\rule{0pt}{6pt}\rule{6pt}{0pt}} Stone~~
  \fcolorbox{white}{minc_20}{\rule{0pt}{6pt}\rule{6pt}{0pt}} Tile~~
  \fcolorbox{white}{minc_21}{\rule{0pt}{6pt}\rule{6pt}{0pt}} Wallpaper~~
  \fcolorbox{white}{minc_22}{\rule{0pt}{6pt}\rule{6pt}{0pt}} Water~~
  \fcolorbox{white}{minc_23}{\rule{0pt}{6pt}\rule{6pt}{0pt}} Wood~~ \\
  }
  \subfigure{%
    \includegraphics[width=.18\columnwidth]{figures/supplementary/000010868_given.jpg}
  }
  \subfigure{%
    \includegraphics[width=.18\columnwidth]{figures/supplementary/000010868_gt.png}
  }
  \subfigure{%
    \includegraphics[width=.18\columnwidth]{figures/supplementary/000010868_cnn.png}
  }
  \subfigure{%
    \includegraphics[width=.18\columnwidth]{figures/supplementary/000010868_gauss.png}
  }
  \subfigure{%
    \includegraphics[width=.18\columnwidth]{figures/supplementary/000010868_learnt.png}
  }\\[-2ex]
  \subfigure{%
    \includegraphics[width=.18\columnwidth]{figures/supplementary/000006011_given.jpg}
  }
  \subfigure{%
    \includegraphics[width=.18\columnwidth]{figures/supplementary/000006011_gt.png}
  }
  \subfigure{%
    \includegraphics[width=.18\columnwidth]{figures/supplementary/000006011_cnn.png}
  }
  \subfigure{%
    \includegraphics[width=.18\columnwidth]{figures/supplementary/000006011_gauss.png}
  }
  \subfigure{%
    \includegraphics[width=.18\columnwidth]{figures/supplementary/000006011_learnt.png}
  }\\[-2ex]
    \subfigure{%
    \includegraphics[width=.18\columnwidth]{figures/supplementary/000008553_given.jpg}
  }
  \subfigure{%
    \includegraphics[width=.18\columnwidth]{figures/supplementary/000008553_gt.png}
  }
  \subfigure{%
    \includegraphics[width=.18\columnwidth]{figures/supplementary/000008553_cnn.png}
  }
  \subfigure{%
    \includegraphics[width=.18\columnwidth]{figures/supplementary/000008553_gauss.png}
  }
  \subfigure{%
    \includegraphics[width=.18\columnwidth]{figures/supplementary/000008553_learnt.png}
  }\\[-2ex]
   \subfigure{%
    \includegraphics[width=.18\columnwidth]{figures/supplementary/000009188_given.jpg}
  }
  \subfigure{%
    \includegraphics[width=.18\columnwidth]{figures/supplementary/000009188_gt.png}
  }
  \subfigure{%
    \includegraphics[width=.18\columnwidth]{figures/supplementary/000009188_cnn.png}
  }
  \subfigure{%
    \includegraphics[width=.18\columnwidth]{figures/supplementary/000009188_gauss.png}
  }
  \subfigure{%
    \includegraphics[width=.18\columnwidth]{figures/supplementary/000009188_learnt.png}
  }\\[-2ex]
  \setcounter{subfigure}{0}
  \subfigure[Input]{%
    \includegraphics[width=.18\columnwidth]{figures/supplementary/000023570_given.jpg}
  }
  \subfigure[Ground Truth]{%
    \includegraphics[width=.18\columnwidth]{figures/supplementary/000023570_gt.png}
  }
  \subfigure[DeepLab]{%
    \includegraphics[width=.18\columnwidth]{figures/supplementary/000023570_cnn.png}
  }
  \subfigure[+GaussCRF]{%
    \includegraphics[width=.18\columnwidth]{figures/supplementary/000023570_gauss.png}
  }
  \subfigure[+LearnedCRF]{%
    \includegraphics[width=.18\columnwidth]{figures/supplementary/000023570_learnt.png}
  }
  \mycaption{Material Segmentation}{Example results of material segmentation.
  (c)~depicts the unary results before application of MF, (d)~after two steps of MF with Gaussian edge CRF potentials, (e)~after two steps of MF with learned edge CRF potentials.}
    \label{fig:material_visuals-app2}
\end{figure*}


\begin{table*}[h]
\tiny
  \centering
    \begin{tabular}{L{2.3cm} L{2.25cm} C{1.5cm} C{0.7cm} C{0.6cm} C{0.7cm} C{0.7cm} C{0.7cm} C{1.6cm} C{0.6cm} C{0.6cm} C{0.6cm}}
      \toprule
& & & & & \multicolumn{3}{c}{\textbf{Data Statistics}} & \multicolumn{4}{c}{\textbf{Training Protocol}} \\

\textbf{Experiment} & \textbf{Feature Types} & \textbf{Feature Scales} & \textbf{Filter Size} & \textbf{Filter Nbr.} & \textbf{Train}  & \textbf{Val.} & \textbf{Test} & \textbf{Loss Type} & \textbf{LR} & \textbf{Batch} & \textbf{Epochs} \\
      \midrule
      \multicolumn{2}{c}{\textbf{Single Bilateral Filter Applications}} & & & & & & & & & \\
      \textbf{2$\times$ Color Upsampling} & Position$_{1}$, Intensity (3D) & 0.13, 0.17 & 65 & 2 & 10581 & 1449 & 1456 & MSE & 1e-06 & 200 & 94.5\\
      \textbf{4$\times$ Color Upsampling} & Position$_{1}$, Intensity (3D) & 0.06, 0.17 & 65 & 2 & 10581 & 1449 & 1456 & MSE & 1e-06 & 200 & 94.5\\
      \textbf{8$\times$ Color Upsampling} & Position$_{1}$, Intensity (3D) & 0.03, 0.17 & 65 & 2 & 10581 & 1449 & 1456 & MSE & 1e-06 & 200 & 94.5\\
      \textbf{16$\times$ Color Upsampling} & Position$_{1}$, Intensity (3D) & 0.02, 0.17 & 65 & 2 & 10581 & 1449 & 1456 & MSE & 1e-06 & 200 & 94.5\\
      \textbf{Depth Upsampling} & Position$_{1}$, Color (5D) & 0.05, 0.02 & 665 & 2 & 795 & 100 & 654 & MSE & 1e-07 & 50 & 251.6\\
      \textbf{Mesh Denoising} & Isomap (4D) & 46.00 & 63 & 2 & 1000 & 200 & 500 & MSE & 100 & 10 & 100.0 \\
      \midrule
      \multicolumn{2}{c}{\textbf{DenseCRF Applications}} & & & & & & & & &\\
      \multicolumn{2}{l}{\textbf{Semantic Segmentation}} & & & & & & & & &\\
      \textbf{- 1step MF} & Position$_{1}$, Color (5D); Position$_{1}$ (2D) & 0.01, 0.34; 0.34  & 665; 19  & 2; 2 & 10581 & 1449 & 1456 & Logistic & 0.1 & 5 & 1.4 \\
      \textbf{- 2step MF} & Position$_{1}$, Color (5D); Position$_{1}$ (2D) & 0.01, 0.34; 0.34 & 665; 19 & 2; 2 & 10581 & 1449 & 1456 & Logistic & 0.1 & 5 & 1.4 \\
      \textbf{- \textit{loose} 2step MF} & Position$_{1}$, Color (5D); Position$_{1}$ (2D) & 0.01, 0.34; 0.34 & 665; 19 & 2; 2 &10581 & 1449 & 1456 & Logistic & 0.1 & 5 & +1.9  \\ \\
      \multicolumn{2}{l}{\textbf{Material Segmentation}} & & & & & & & & &\\
      \textbf{- 1step MF} & Position$_{2}$, Lab-Color (5D) & 5.00, 0.05, 0.30  & 665 & 2 & 928 & 150 & 1798 & Weighted Logistic & 1e-04 & 24 & 2.6 \\
      \textbf{- 2step MF} & Position$_{2}$, Lab-Color (5D) & 5.00, 0.05, 0.30 & 665 & 2 & 928 & 150 & 1798 & Weighted Logistic & 1e-04 & 12 & +0.7 \\
      \textbf{- \textit{loose} 2step MF} & Position$_{2}$, Lab-Color (5D) & 5.00, 0.05, 0.30 & 665 & 2 & 928 & 150 & 1798 & Weighted Logistic & 1e-04 & 12 & +0.2\\
      \midrule
      \multicolumn{2}{c}{\textbf{Neural Network Applications}} & & & & & & & & &\\
      \textbf{Tiles: CNN-9$\times$9} & - & - & 81 & 4 & 10000 & 1000 & 1000 & Logistic & 0.01 & 100 & 500.0 \\
      \textbf{Tiles: CNN-13$\times$13} & - & - & 169 & 6 & 10000 & 1000 & 1000 & Logistic & 0.01 & 100 & 500.0 \\
      \textbf{Tiles: CNN-17$\times$17} & - & - & 289 & 8 & 10000 & 1000 & 1000 & Logistic & 0.01 & 100 & 500.0 \\
      \textbf{Tiles: CNN-21$\times$21} & - & - & 441 & 10 & 10000 & 1000 & 1000 & Logistic & 0.01 & 100 & 500.0 \\
      \textbf{Tiles: BNN} & Position$_{1}$, Color (5D) & 0.05, 0.04 & 63 & 1 & 10000 & 1000 & 1000 & Logistic & 0.01 & 100 & 30.0 \\
      \textbf{LeNet} & - & - & 25 & 2 & 5490 & 1098 & 1647 & Logistic & 0.1 & 100 & 182.2 \\
      \textbf{Crop-LeNet} & - & - & 25 & 2 & 5490 & 1098 & 1647 & Logistic & 0.1 & 100 & 182.2 \\
      \textbf{BNN-LeNet} & Position$_{2}$ (2D) & 20.00 & 7 & 1 & 5490 & 1098 & 1647 & Logistic & 0.1 & 100 & 182.2 \\
      \textbf{DeepCNet} & - & - & 9 & 1 & 5490 & 1098 & 1647 & Logistic & 0.1 & 100 & 182.2 \\
      \textbf{Crop-DeepCNet} & - & - & 9 & 1 & 5490 & 1098 & 1647 & Logistic & 0.1 & 100 & 182.2 \\
      \textbf{BNN-DeepCNet} & Position$_{2}$ (2D) & 40.00  & 7 & 1 & 5490 & 1098 & 1647 & Logistic & 0.1 & 100 & 182.2 \\
      \bottomrule
      \\
    \end{tabular}
    \mycaption{Experiment Protocols} {Experiment protocols for the different experiments presented in this work. \textbf{Feature Types}:
    Feature spaces used for the bilateral convolutions. Position$_1$ corresponds to un-normalized pixel positions whereas Position$_2$ corresponds
    to pixel positions normalized to $[0,1]$ with respect to the given image. \textbf{Feature Scales}: Cross-validated scales for the features used.
     \textbf{Filter Size}: Number of elements in the filter that is being learned. \textbf{Filter Nbr.}: Half-width of the filter. \textbf{Train},
     \textbf{Val.} and \textbf{Test} corresponds to the number of train, validation and test images used in the experiment. \textbf{Loss Type}: Type
     of loss used for back-propagation. ``MSE'' corresponds to Euclidean mean squared error loss and ``Logistic'' corresponds to multinomial logistic
     loss. ``Weighted Logistic'' is the class-weighted multinomial logistic loss. We weighted the loss with inverse class probability for material
     segmentation task due to the small availability of training data with class imbalance. \textbf{LR}: Fixed learning rate used in stochastic gradient
     descent. \textbf{Batch}: Number of images used in one parameter update step. \textbf{Epochs}: Number of training epochs. In all the experiments,
     we used fixed momentum of 0.9 and weight decay of 0.0005 for stochastic gradient descent. ```Color Upsampling'' experiments in this Table corresponds
     to those performed on Pascal VOC12 dataset images. For all experiments using Pascal VOC12 images, we use extended
     training segmentation dataset available from~\cite{hariharan2011moredata}, and used standard validation and test splits
     from the main dataset~\cite{voc2012segmentation}.}
  \label{tbl:parameters}
\end{table*}

\clearpage

\section{Parameters and Additional Results for Video Propagation Networks}

In this Section, we present experiment protocols and additional qualitative results for experiments
on video object segmentation, semantic video segmentation and video color
propagation. Table~\ref{tbl:parameters_supp} shows the feature scales and other parameters used in different experiments.
Figures~\ref{fig:video_seg_pos_supp} show some qualitative results on video object segmentation
with some failure cases in Fig.~\ref{fig:video_seg_neg_supp}.
Figure~\ref{fig:semantic_visuals_supp} shows some qualitative results on semantic video segmentation and
Fig.~\ref{fig:color_visuals_supp} shows results on video color propagation.

\newcolumntype{L}[1]{>{\raggedright\let\newline\\\arraybackslash\hspace{0pt}}b{#1}}
\newcolumntype{C}[1]{>{\centering\let\newline\\\arraybackslash\hspace{0pt}}b{#1}}
\newcolumntype{R}[1]{>{\raggedleft\let\newline\\\arraybackslash\hspace{0pt}}b{#1}}

\begin{table*}[h]
\tiny
  \centering
    \begin{tabular}{L{3.0cm} L{2.4cm} L{2.8cm} L{2.8cm} C{0.5cm} C{1.0cm} L{1.2cm}}
      \toprule
\textbf{Experiment} & \textbf{Feature Type} & \textbf{Feature Scale-1, $\Lambda_a$} & \textbf{Feature Scale-2, $\Lambda_b$} & \textbf{$\alpha$} & \textbf{Input Frames} & \textbf{Loss Type} \\
      \midrule
      \textbf{Video Object Segmentation} & ($x,y,Y,Cb,Cr,t$) & (0.02,0.02,0.07,0.4,0.4,0.01) & (0.03,0.03,0.09,0.5,0.5,0.2) & 0.5 & 9 & Logistic\\
      \midrule
      \textbf{Semantic Video Segmentation} & & & & & \\
      \textbf{with CNN1~\cite{yu2015multi}-NoFlow} & ($x,y,R,G,B,t$) & (0.08,0.08,0.2,0.2,0.2,0.04) & (0.11,0.11,0.2,0.2,0.2,0.04) & 0.5 & 3 & Logistic \\
      \textbf{with CNN1~\cite{yu2015multi}-Flow} & ($x+u_x,y+u_y,R,G,B,t$) & (0.11,0.11,0.14,0.14,0.14,0.03) & (0.08,0.08,0.12,0.12,0.12,0.01) & 0.65 & 3 & Logistic\\
      \textbf{with CNN2~\cite{richter2016playing}-Flow} & ($x+u_x,y+u_y,R,G,B,t$) & (0.08,0.08,0.2,0.2,0.2,0.04) & (0.09,0.09,0.25,0.25,0.25,0.03) & 0.5 & 4 & Logistic\\
      \midrule
      \textbf{Video Color Propagation} & ($x,y,I,t$)  & (0.04,0.04,0.2,0.04) & No second kernel & 1 & 4 & MSE\\
      \bottomrule
      \\
    \end{tabular}
    \mycaption{Experiment Protocols} {Experiment protocols for the different experiments presented in this work. \textbf{Feature Types}:
    Feature spaces used for the bilateral convolutions, with position ($x,y$) and color
    ($R,G,B$ or $Y,Cb,Cr$) features $\in [0,255]$. $u_x$, $u_y$ denotes optical flow with respect
    to the present frame and $I$ denotes grayscale intensity.
    \textbf{Feature Scales ($\Lambda_a, \Lambda_b$)}: Cross-validated scales for the features used.
    \textbf{$\alpha$}: Exponential time decay for the input frames.
    \textbf{Input Frames}: Number of input frames for VPN.
    \textbf{Loss Type}: Type
     of loss used for back-propagation. ``MSE'' corresponds to Euclidean mean squared error loss and ``Logistic'' corresponds to multinomial logistic loss.}
  \label{tbl:parameters_supp}
\end{table*}

% \begin{figure}[th!]
% \begin{center}
%   \centerline{\includegraphics[width=\textwidth]{figures/video_seg_visuals_supp_small.pdf}}
%     \mycaption{Video Object Segmentation}
%     {Shown are the different frames in example videos with the corresponding
%     ground truth (GT) masks, predictions from BVS~\cite{marki2016bilateral},
%     OFL~\cite{tsaivideo}, VPN (VPN-Stage2) and VPN-DLab (VPN-DeepLab) models.}
%     \label{fig:video_seg_small_supp}
% \end{center}
% \vspace{-1.0cm}
% \end{figure}

\begin{figure}[th!]
\begin{center}
  \centerline{\includegraphics[width=0.7\textwidth]{figures/video_seg_visuals_supp_positive.pdf}}
    \mycaption{Video Object Segmentation}
    {Shown are the different frames in example videos with the corresponding
    ground truth (GT) masks, predictions from BVS~\cite{marki2016bilateral},
    OFL~\cite{tsaivideo}, VPN (VPN-Stage2) and VPN-DLab (VPN-DeepLab) models.}
    \label{fig:video_seg_pos_supp}
\end{center}
\vspace{-1.0cm}
\end{figure}

\begin{figure}[th!]
\begin{center}
  \centerline{\includegraphics[width=0.7\textwidth]{figures/video_seg_visuals_supp_negative.pdf}}
    \mycaption{Failure Cases for Video Object Segmentation}
    {Shown are the different frames in example videos with the corresponding
    ground truth (GT) masks, predictions from BVS~\cite{marki2016bilateral},
    OFL~\cite{tsaivideo}, VPN (VPN-Stage2) and VPN-DLab (VPN-DeepLab) models.}
    \label{fig:video_seg_neg_supp}
\end{center}
\vspace{-1.0cm}
\end{figure}

\begin{figure}[th!]
\begin{center}
  \centerline{\includegraphics[width=0.9\textwidth]{figures/supp_semantic_visual.pdf}}
    \mycaption{Semantic Video Segmentation}
    {Input video frames and the corresponding ground truth (GT)
    segmentation together with the predictions of CNN~\cite{yu2015multi} and with
    VPN-Flow.}
    \label{fig:semantic_visuals_supp}
\end{center}
\vspace{-0.7cm}
\end{figure}

\begin{figure}[th!]
\begin{center}
  \centerline{\includegraphics[width=\textwidth]{figures/colorization_visuals_supp.pdf}}
  \mycaption{Video Color Propagation}
  {Input grayscale video frames and corresponding ground-truth (GT) color images
  together with color predictions of Levin et al.~\cite{levin2004colorization} and VPN-Stage1 models.}
  \label{fig:color_visuals_supp}
\end{center}
\vspace{-0.7cm}
\end{figure}

\clearpage

\section{Additional Material for Bilateral Inception Networks}
\label{sec:binception-app}

In this section of the Appendix, we first discuss the use of approximate bilateral
filtering in BI modules (Sec.~\ref{sec:lattice}).
Later, we present some qualitative results using different models for the approach presented in
Chapter~\ref{chap:binception} (Sec.~\ref{sec:qualitative-app}).

\subsection{Approximate Bilateral Filtering}
\label{sec:lattice}

The bilateral inception module presented in Chapter~\ref{chap:binception} computes a matrix-vector
product between a Gaussian filter $K$ and a vector of activations $\bz_c$.
Bilateral filtering is an important operation and many algorithmic techniques have been
proposed to speed-up this operation~\cite{paris2006fast,adams2010fast,gastal2011domain}.
In the main paper we opted to implement what can be considered the
brute-force variant of explicitly constructing $K$ and then using BLAS to compute the
matrix-vector product. This resulted in a few millisecond operation.
The explicit way to compute is possible due to the
reduction to super-pixels, e.g., it would not work for DenseCRF variants
that operate on the full image resolution.

Here, we present experiments where we use the fast approximate bilateral filtering
algorithm of~\cite{adams2010fast}, which is also used in Chapter~\ref{chap:bnn}
for learning sparse high dimensional filters. This
choice allows for larger dimensions of matrix-vector multiplication. The reason for choosing
the explicit multiplication in Chapter~\ref{chap:binception} was that it was computationally faster.
For the small sizes of the involved matrices and vectors, the explicit computation is sufficient and we had no
GPU implementation of an approximate technique that matched this runtime. Also it
is conceptually easier and the gradient to the feature transformations ($\Lambda \mathbf{f}$) is
obtained using standard matrix calculus.

\subsubsection{Experiments}

We modified the existing segmentation architectures analogous to those in Chapter~\ref{chap:binception}.
The main difference is that, here, the inception modules use the lattice
approximation~\cite{adams2010fast} to compute the bilateral filtering.
Using the lattice approximation did not allow us to back-propagate through feature transformations ($\Lambda$)
and thus we used hand-specified feature scales as will be explained later.
Specifically, we take CNN architectures from the works
of~\cite{chen2014semantic,zheng2015conditional,bell2015minc} and insert the BI modules between
the spatial FC layers.
We use superpixels from~\cite{DollarICCV13edges}
for all the experiments with the lattice approximation. Experiments are
performed using Caffe neural network framework~\cite{jia2014caffe}.

\begin{table}
  \small
  \centering
  \begin{tabular}{p{5.5cm}>{\raggedright\arraybackslash}p{1.4cm}>{\centering\arraybackslash}p{2.2cm}}
    \toprule
		\textbf{Model} & \emph{IoU} & \emph{Runtime}(ms) \\
    \midrule

    %%%%%%%%%%%% Scores computed by us)%%%%%%%%%%%%
		\deeplablargefov & 68.9 & 145ms\\
    \midrule
    \bi{7}{2}-\bi{8}{10}& \textbf{73.8} & +600 \\
    \midrule
    \deeplablargefovcrf~\cite{chen2014semantic} & 72.7 & +830\\
    \deeplabmsclargefovcrf~\cite{chen2014semantic} & \textbf{73.6} & +880\\
    DeepLab-EdgeNet~\cite{chen2015semantic} & 71.7 & +30\\
    DeepLab-EdgeNet-CRF~\cite{chen2015semantic} & \textbf{73.6} & +860\\
  \bottomrule \\
  \end{tabular}
  \mycaption{Semantic Segmentation using the DeepLab model}
  {IoU scores on the Pascal VOC12 segmentation test dataset
  with different models and our modified inception model.
  Also shown are the corresponding runtimes in milliseconds. Runtimes
  also include superpixel computations (300 ms with Dollar superpixels~\cite{DollarICCV13edges})}
  \label{tab:largefovresults}
\end{table}

\paragraph{Semantic Segmentation}
The experiments in this section use the Pascal VOC12 segmentation dataset~\cite{voc2012segmentation} with 21 object classes and the images have a maximum resolution of 0.25 megapixels.
For all experiments on VOC12, we train using the extended training set of
10581 images collected by~\cite{hariharan2011moredata}.
We modified the \deeplab~network architecture of~\cite{chen2014semantic} and
the CRFasRNN architecture from~\cite{zheng2015conditional} which uses a CNN with
deconvolution layers followed by DenseCRF trained end-to-end.

\paragraph{DeepLab Model}\label{sec:deeplabmodel}
We experimented with the \bi{7}{2}-\bi{8}{10} inception model.
Results using the~\deeplab~model are summarized in Tab.~\ref{tab:largefovresults}.
Although we get similar improvements with inception modules as with the
explicit kernel computation, using lattice approximation is slower.

\begin{table}
  \small
  \centering
  \begin{tabular}{p{6.4cm}>{\raggedright\arraybackslash}p{1.8cm}>{\raggedright\arraybackslash}p{1.8cm}}
    \toprule
    \textbf{Model} & \emph{IoU (Val)} & \emph{IoU (Test)}\\
    \midrule
    %%%%%%%%%%%% Scores computed by us)%%%%%%%%%%%%
    CNN &  67.5 & - \\
    \deconv (CNN+Deconvolutions) & 69.8 & 72.0 \\
    \midrule
    \bi{3}{6}-\bi{4}{6}-\bi{7}{2}-\bi{8}{6}& 71.9 & - \\
    \bi{3}{6}-\bi{4}{6}-\bi{7}{2}-\bi{8}{6}-\gi{6}& 73.6 &  \href{http://host.robots.ox.ac.uk:8080/anonymous/VOTV5E.html}{\textbf{75.2}}\\
    \midrule
    \deconvcrf (CRF-RNN)~\cite{zheng2015conditional} & 73.0 & 74.7\\
    Context-CRF-RNN~\cite{yu2015multi} & ~~ - ~ & \textbf{75.3} \\
    \bottomrule \\
  \end{tabular}
  \mycaption{Semantic Segmentation using the CRFasRNN model}{IoU score corresponding to different models
  on Pascal VOC12 reduced validation / test segmentation dataset. The reduced validation set consists of 346 images
  as used in~\cite{zheng2015conditional} where we adapted the model from.}
  \label{tab:deconvresults-app}
\end{table}

\paragraph{CRFasRNN Model}\label{sec:deepinception}
We add BI modules after score-pool3, score-pool4, \fc{7} and \fc{8} $1\times1$ convolution layers
resulting in the \bi{3}{6}-\bi{4}{6}-\bi{7}{2}-\bi{8}{6}
model and also experimented with another variant where $BI_8$ is followed by another inception
module, G$(6)$, with 6 Gaussian kernels.
Note that here also we discarded both deconvolution and DenseCRF parts of the original model~\cite{zheng2015conditional}
and inserted the BI modules in the base CNN and found similar improvements compared to the inception modules with explicit
kernel computaion. See Tab.~\ref{tab:deconvresults-app} for results on the CRFasRNN model.

\paragraph{Material Segmentation}
Table~\ref{tab:mincresults-app} shows the results on the MINC dataset~\cite{bell2015minc}
obtained by modifying the AlexNet architecture with our inception modules. We observe
similar improvements as with explicit kernel construction.
For this model, we do not provide any learned setup due to very limited segment training
data. The weights to combine outputs in the bilateral inception layer are
found by validation on the validation set.

\begin{table}[t]
  \small
  \centering
  \begin{tabular}{p{3.5cm}>{\centering\arraybackslash}p{4.0cm}}
    \toprule
    \textbf{Model} & Class / Total accuracy\\
    \midrule

    %%%%%%%%%%%% Scores computed by us)%%%%%%%%%%%%
    AlexNet CNN & 55.3 / 58.9 \\
    \midrule
    \bi{7}{2}-\bi{8}{6}& 68.5 / 71.8 \\
    \bi{7}{2}-\bi{8}{6}-G$(6)$& 67.6 / 73.1 \\
    \midrule
    AlexNet-CRF & 65.5 / 71.0 \\
    \bottomrule \\
  \end{tabular}
  \mycaption{Material Segmentation using AlexNet}{Pixel accuracy of different models on
  the MINC material segmentation test dataset~\cite{bell2015minc}.}
  \label{tab:mincresults-app}
\end{table}

\paragraph{Scales of Bilateral Inception Modules}
\label{sec:scales}

Unlike the explicit kernel technique presented in the main text (Chapter~\ref{chap:binception}),
we didn't back-propagate through feature transformation ($\Lambda$)
using the approximate bilateral filter technique.
So, the feature scales are hand-specified and validated, which are as follows.
The optimal scale values for the \bi{7}{2}-\bi{8}{2} model are found by validation for the best performance which are
$\sigma_{xy}$ = (0.1, 0.1) for the spatial (XY) kernel and $\sigma_{rgbxy}$ = (0.1, 0.1, 0.1, 0.01, 0.01) for color and position (RGBXY)  kernel.
Next, as more kernels are added to \bi{8}{2}, we set scales to be $\alpha$*($\sigma_{xy}$, $\sigma_{rgbxy}$).
The value of $\alpha$ is chosen as  1, 0.5, 0.1, 0.05, 0.1, at uniform interval, for the \bi{8}{10} bilateral inception module.


\subsection{Qualitative Results}
\label{sec:qualitative-app}

In this section, we present more qualitative results obtained using the BI module with explicit
kernel computation technique presented in Chapter~\ref{chap:binception}. Results on the Pascal VOC12
dataset~\cite{voc2012segmentation} using the DeepLab-LargeFOV model are shown in Fig.~\ref{fig:semantic_visuals-app},
followed by the results on MINC dataset~\cite{bell2015minc}
in Fig.~\ref{fig:material_visuals-app} and on
Cityscapes dataset~\cite{Cordts2015Cvprw} in Fig.~\ref{fig:street_visuals-app}.


\definecolor{voc_1}{RGB}{0, 0, 0}
\definecolor{voc_2}{RGB}{128, 0, 0}
\definecolor{voc_3}{RGB}{0, 128, 0}
\definecolor{voc_4}{RGB}{128, 128, 0}
\definecolor{voc_5}{RGB}{0, 0, 128}
\definecolor{voc_6}{RGB}{128, 0, 128}
\definecolor{voc_7}{RGB}{0, 128, 128}
\definecolor{voc_8}{RGB}{128, 128, 128}
\definecolor{voc_9}{RGB}{64, 0, 0}
\definecolor{voc_10}{RGB}{192, 0, 0}
\definecolor{voc_11}{RGB}{64, 128, 0}
\definecolor{voc_12}{RGB}{192, 128, 0}
\definecolor{voc_13}{RGB}{64, 0, 128}
\definecolor{voc_14}{RGB}{192, 0, 128}
\definecolor{voc_15}{RGB}{64, 128, 128}
\definecolor{voc_16}{RGB}{192, 128, 128}
\definecolor{voc_17}{RGB}{0, 64, 0}
\definecolor{voc_18}{RGB}{128, 64, 0}
\definecolor{voc_19}{RGB}{0, 192, 0}
\definecolor{voc_20}{RGB}{128, 192, 0}
\definecolor{voc_21}{RGB}{0, 64, 128}
\definecolor{voc_22}{RGB}{128, 64, 128}

\begin{figure*}[!ht]
  \small
  \centering
  \fcolorbox{white}{voc_1}{\rule{0pt}{4pt}\rule{4pt}{0pt}} Background~~
  \fcolorbox{white}{voc_2}{\rule{0pt}{4pt}\rule{4pt}{0pt}} Aeroplane~~
  \fcolorbox{white}{voc_3}{\rule{0pt}{4pt}\rule{4pt}{0pt}} Bicycle~~
  \fcolorbox{white}{voc_4}{\rule{0pt}{4pt}\rule{4pt}{0pt}} Bird~~
  \fcolorbox{white}{voc_5}{\rule{0pt}{4pt}\rule{4pt}{0pt}} Boat~~
  \fcolorbox{white}{voc_6}{\rule{0pt}{4pt}\rule{4pt}{0pt}} Bottle~~
  \fcolorbox{white}{voc_7}{\rule{0pt}{4pt}\rule{4pt}{0pt}} Bus~~
  \fcolorbox{white}{voc_8}{\rule{0pt}{4pt}\rule{4pt}{0pt}} Car~~\\
  \fcolorbox{white}{voc_9}{\rule{0pt}{4pt}\rule{4pt}{0pt}} Cat~~
  \fcolorbox{white}{voc_10}{\rule{0pt}{4pt}\rule{4pt}{0pt}} Chair~~
  \fcolorbox{white}{voc_11}{\rule{0pt}{4pt}\rule{4pt}{0pt}} Cow~~
  \fcolorbox{white}{voc_12}{\rule{0pt}{4pt}\rule{4pt}{0pt}} Dining Table~~
  \fcolorbox{white}{voc_13}{\rule{0pt}{4pt}\rule{4pt}{0pt}} Dog~~
  \fcolorbox{white}{voc_14}{\rule{0pt}{4pt}\rule{4pt}{0pt}} Horse~~
  \fcolorbox{white}{voc_15}{\rule{0pt}{4pt}\rule{4pt}{0pt}} Motorbike~~
  \fcolorbox{white}{voc_16}{\rule{0pt}{4pt}\rule{4pt}{0pt}} Person~~\\
  \fcolorbox{white}{voc_17}{\rule{0pt}{4pt}\rule{4pt}{0pt}} Potted Plant~~
  \fcolorbox{white}{voc_18}{\rule{0pt}{4pt}\rule{4pt}{0pt}} Sheep~~
  \fcolorbox{white}{voc_19}{\rule{0pt}{4pt}\rule{4pt}{0pt}} Sofa~~
  \fcolorbox{white}{voc_20}{\rule{0pt}{4pt}\rule{4pt}{0pt}} Train~~
  \fcolorbox{white}{voc_21}{\rule{0pt}{4pt}\rule{4pt}{0pt}} TV monitor~~\\


  \subfigure{%
    \includegraphics[width=.15\columnwidth]{figures/supplementary/2008_001308_given.png}
  }
  \subfigure{%
    \includegraphics[width=.15\columnwidth]{figures/supplementary/2008_001308_sp.png}
  }
  \subfigure{%
    \includegraphics[width=.15\columnwidth]{figures/supplementary/2008_001308_gt.png}
  }
  \subfigure{%
    \includegraphics[width=.15\columnwidth]{figures/supplementary/2008_001308_cnn.png}
  }
  \subfigure{%
    \includegraphics[width=.15\columnwidth]{figures/supplementary/2008_001308_crf.png}
  }
  \subfigure{%
    \includegraphics[width=.15\columnwidth]{figures/supplementary/2008_001308_ours.png}
  }\\[-2ex]


  \subfigure{%
    \includegraphics[width=.15\columnwidth]{figures/supplementary/2008_001821_given.png}
  }
  \subfigure{%
    \includegraphics[width=.15\columnwidth]{figures/supplementary/2008_001821_sp.png}
  }
  \subfigure{%
    \includegraphics[width=.15\columnwidth]{figures/supplementary/2008_001821_gt.png}
  }
  \subfigure{%
    \includegraphics[width=.15\columnwidth]{figures/supplementary/2008_001821_cnn.png}
  }
  \subfigure{%
    \includegraphics[width=.15\columnwidth]{figures/supplementary/2008_001821_crf.png}
  }
  \subfigure{%
    \includegraphics[width=.15\columnwidth]{figures/supplementary/2008_001821_ours.png}
  }\\[-2ex]



  \subfigure{%
    \includegraphics[width=.15\columnwidth]{figures/supplementary/2008_004612_given.png}
  }
  \subfigure{%
    \includegraphics[width=.15\columnwidth]{figures/supplementary/2008_004612_sp.png}
  }
  \subfigure{%
    \includegraphics[width=.15\columnwidth]{figures/supplementary/2008_004612_gt.png}
  }
  \subfigure{%
    \includegraphics[width=.15\columnwidth]{figures/supplementary/2008_004612_cnn.png}
  }
  \subfigure{%
    \includegraphics[width=.15\columnwidth]{figures/supplementary/2008_004612_crf.png}
  }
  \subfigure{%
    \includegraphics[width=.15\columnwidth]{figures/supplementary/2008_004612_ours.png}
  }\\[-2ex]


  \subfigure{%
    \includegraphics[width=.15\columnwidth]{figures/supplementary/2009_001008_given.png}
  }
  \subfigure{%
    \includegraphics[width=.15\columnwidth]{figures/supplementary/2009_001008_sp.png}
  }
  \subfigure{%
    \includegraphics[width=.15\columnwidth]{figures/supplementary/2009_001008_gt.png}
  }
  \subfigure{%
    \includegraphics[width=.15\columnwidth]{figures/supplementary/2009_001008_cnn.png}
  }
  \subfigure{%
    \includegraphics[width=.15\columnwidth]{figures/supplementary/2009_001008_crf.png}
  }
  \subfigure{%
    \includegraphics[width=.15\columnwidth]{figures/supplementary/2009_001008_ours.png}
  }\\[-2ex]




  \subfigure{%
    \includegraphics[width=.15\columnwidth]{figures/supplementary/2009_004497_given.png}
  }
  \subfigure{%
    \includegraphics[width=.15\columnwidth]{figures/supplementary/2009_004497_sp.png}
  }
  \subfigure{%
    \includegraphics[width=.15\columnwidth]{figures/supplementary/2009_004497_gt.png}
  }
  \subfigure{%
    \includegraphics[width=.15\columnwidth]{figures/supplementary/2009_004497_cnn.png}
  }
  \subfigure{%
    \includegraphics[width=.15\columnwidth]{figures/supplementary/2009_004497_crf.png}
  }
  \subfigure{%
    \includegraphics[width=.15\columnwidth]{figures/supplementary/2009_004497_ours.png}
  }\\[-2ex]



  \setcounter{subfigure}{0}
  \subfigure[\scriptsize Input]{%
    \includegraphics[width=.15\columnwidth]{figures/supplementary/2010_001327_given.png}
  }
  \subfigure[\scriptsize Superpixels]{%
    \includegraphics[width=.15\columnwidth]{figures/supplementary/2010_001327_sp.png}
  }
  \subfigure[\scriptsize GT]{%
    \includegraphics[width=.15\columnwidth]{figures/supplementary/2010_001327_gt.png}
  }
  \subfigure[\scriptsize Deeplab]{%
    \includegraphics[width=.15\columnwidth]{figures/supplementary/2010_001327_cnn.png}
  }
  \subfigure[\scriptsize +DenseCRF]{%
    \includegraphics[width=.15\columnwidth]{figures/supplementary/2010_001327_crf.png}
  }
  \subfigure[\scriptsize Using BI]{%
    \includegraphics[width=.15\columnwidth]{figures/supplementary/2010_001327_ours.png}
  }
  \mycaption{Semantic Segmentation}{Example results of semantic segmentation
  on the Pascal VOC12 dataset.
  (d)~depicts the DeepLab CNN result, (e)~CNN + 10 steps of mean-field inference,
  (f~result obtained with bilateral inception (BI) modules (\bi{6}{2}+\bi{7}{6}) between \fc~layers.}
  \label{fig:semantic_visuals-app}
\end{figure*}


\definecolor{minc_1}{HTML}{771111}
\definecolor{minc_2}{HTML}{CAC690}
\definecolor{minc_3}{HTML}{EEEEEE}
\definecolor{minc_4}{HTML}{7C8FA6}
\definecolor{minc_5}{HTML}{597D31}
\definecolor{minc_6}{HTML}{104410}
\definecolor{minc_7}{HTML}{BB819C}
\definecolor{minc_8}{HTML}{D0CE48}
\definecolor{minc_9}{HTML}{622745}
\definecolor{minc_10}{HTML}{666666}
\definecolor{minc_11}{HTML}{D54A31}
\definecolor{minc_12}{HTML}{101044}
\definecolor{minc_13}{HTML}{444126}
\definecolor{minc_14}{HTML}{75D646}
\definecolor{minc_15}{HTML}{DD4348}
\definecolor{minc_16}{HTML}{5C8577}
\definecolor{minc_17}{HTML}{C78472}
\definecolor{minc_18}{HTML}{75D6D0}
\definecolor{minc_19}{HTML}{5B4586}
\definecolor{minc_20}{HTML}{C04393}
\definecolor{minc_21}{HTML}{D69948}
\definecolor{minc_22}{HTML}{7370D8}
\definecolor{minc_23}{HTML}{7A3622}
\definecolor{minc_24}{HTML}{000000}

\begin{figure*}[!ht]
  \small % scriptsize
  \centering
  \fcolorbox{white}{minc_1}{\rule{0pt}{4pt}\rule{4pt}{0pt}} Brick~~
  \fcolorbox{white}{minc_2}{\rule{0pt}{4pt}\rule{4pt}{0pt}} Carpet~~
  \fcolorbox{white}{minc_3}{\rule{0pt}{4pt}\rule{4pt}{0pt}} Ceramic~~
  \fcolorbox{white}{minc_4}{\rule{0pt}{4pt}\rule{4pt}{0pt}} Fabric~~
  \fcolorbox{white}{minc_5}{\rule{0pt}{4pt}\rule{4pt}{0pt}} Foliage~~
  \fcolorbox{white}{minc_6}{\rule{0pt}{4pt}\rule{4pt}{0pt}} Food~~
  \fcolorbox{white}{minc_7}{\rule{0pt}{4pt}\rule{4pt}{0pt}} Glass~~
  \fcolorbox{white}{minc_8}{\rule{0pt}{4pt}\rule{4pt}{0pt}} Hair~~\\
  \fcolorbox{white}{minc_9}{\rule{0pt}{4pt}\rule{4pt}{0pt}} Leather~~
  \fcolorbox{white}{minc_10}{\rule{0pt}{4pt}\rule{4pt}{0pt}} Metal~~
  \fcolorbox{white}{minc_11}{\rule{0pt}{4pt}\rule{4pt}{0pt}} Mirror~~
  \fcolorbox{white}{minc_12}{\rule{0pt}{4pt}\rule{4pt}{0pt}} Other~~
  \fcolorbox{white}{minc_13}{\rule{0pt}{4pt}\rule{4pt}{0pt}} Painted~~
  \fcolorbox{white}{minc_14}{\rule{0pt}{4pt}\rule{4pt}{0pt}} Paper~~
  \fcolorbox{white}{minc_15}{\rule{0pt}{4pt}\rule{4pt}{0pt}} Plastic~~\\
  \fcolorbox{white}{minc_16}{\rule{0pt}{4pt}\rule{4pt}{0pt}} Polished Stone~~
  \fcolorbox{white}{minc_17}{\rule{0pt}{4pt}\rule{4pt}{0pt}} Skin~~
  \fcolorbox{white}{minc_18}{\rule{0pt}{4pt}\rule{4pt}{0pt}} Sky~~
  \fcolorbox{white}{minc_19}{\rule{0pt}{4pt}\rule{4pt}{0pt}} Stone~~
  \fcolorbox{white}{minc_20}{\rule{0pt}{4pt}\rule{4pt}{0pt}} Tile~~
  \fcolorbox{white}{minc_21}{\rule{0pt}{4pt}\rule{4pt}{0pt}} Wallpaper~~
  \fcolorbox{white}{minc_22}{\rule{0pt}{4pt}\rule{4pt}{0pt}} Water~~
  \fcolorbox{white}{minc_23}{\rule{0pt}{4pt}\rule{4pt}{0pt}} Wood~~\\
  \subfigure{%
    \includegraphics[width=.15\columnwidth]{figures/supplementary/000008468_given.png}
  }
  \subfigure{%
    \includegraphics[width=.15\columnwidth]{figures/supplementary/000008468_sp.png}
  }
  \subfigure{%
    \includegraphics[width=.15\columnwidth]{figures/supplementary/000008468_gt.png}
  }
  \subfigure{%
    \includegraphics[width=.15\columnwidth]{figures/supplementary/000008468_cnn.png}
  }
  \subfigure{%
    \includegraphics[width=.15\columnwidth]{figures/supplementary/000008468_crf.png}
  }
  \subfigure{%
    \includegraphics[width=.15\columnwidth]{figures/supplementary/000008468_ours.png}
  }\\[-2ex]

  \subfigure{%
    \includegraphics[width=.15\columnwidth]{figures/supplementary/000009053_given.png}
  }
  \subfigure{%
    \includegraphics[width=.15\columnwidth]{figures/supplementary/000009053_sp.png}
  }
  \subfigure{%
    \includegraphics[width=.15\columnwidth]{figures/supplementary/000009053_gt.png}
  }
  \subfigure{%
    \includegraphics[width=.15\columnwidth]{figures/supplementary/000009053_cnn.png}
  }
  \subfigure{%
    \includegraphics[width=.15\columnwidth]{figures/supplementary/000009053_crf.png}
  }
  \subfigure{%
    \includegraphics[width=.15\columnwidth]{figures/supplementary/000009053_ours.png}
  }\\[-2ex]




  \subfigure{%
    \includegraphics[width=.15\columnwidth]{figures/supplementary/000014977_given.png}
  }
  \subfigure{%
    \includegraphics[width=.15\columnwidth]{figures/supplementary/000014977_sp.png}
  }
  \subfigure{%
    \includegraphics[width=.15\columnwidth]{figures/supplementary/000014977_gt.png}
  }
  \subfigure{%
    \includegraphics[width=.15\columnwidth]{figures/supplementary/000014977_cnn.png}
  }
  \subfigure{%
    \includegraphics[width=.15\columnwidth]{figures/supplementary/000014977_crf.png}
  }
  \subfigure{%
    \includegraphics[width=.15\columnwidth]{figures/supplementary/000014977_ours.png}
  }\\[-2ex]


  \subfigure{%
    \includegraphics[width=.15\columnwidth]{figures/supplementary/000022922_given.png}
  }
  \subfigure{%
    \includegraphics[width=.15\columnwidth]{figures/supplementary/000022922_sp.png}
  }
  \subfigure{%
    \includegraphics[width=.15\columnwidth]{figures/supplementary/000022922_gt.png}
  }
  \subfigure{%
    \includegraphics[width=.15\columnwidth]{figures/supplementary/000022922_cnn.png}
  }
  \subfigure{%
    \includegraphics[width=.15\columnwidth]{figures/supplementary/000022922_crf.png}
  }
  \subfigure{%
    \includegraphics[width=.15\columnwidth]{figures/supplementary/000022922_ours.png}
  }\\[-2ex]


  \subfigure{%
    \includegraphics[width=.15\columnwidth]{figures/supplementary/000025711_given.png}
  }
  \subfigure{%
    \includegraphics[width=.15\columnwidth]{figures/supplementary/000025711_sp.png}
  }
  \subfigure{%
    \includegraphics[width=.15\columnwidth]{figures/supplementary/000025711_gt.png}
  }
  \subfigure{%
    \includegraphics[width=.15\columnwidth]{figures/supplementary/000025711_cnn.png}
  }
  \subfigure{%
    \includegraphics[width=.15\columnwidth]{figures/supplementary/000025711_crf.png}
  }
  \subfigure{%
    \includegraphics[width=.15\columnwidth]{figures/supplementary/000025711_ours.png}
  }\\[-2ex]


  \subfigure{%
    \includegraphics[width=.15\columnwidth]{figures/supplementary/000034473_given.png}
  }
  \subfigure{%
    \includegraphics[width=.15\columnwidth]{figures/supplementary/000034473_sp.png}
  }
  \subfigure{%
    \includegraphics[width=.15\columnwidth]{figures/supplementary/000034473_gt.png}
  }
  \subfigure{%
    \includegraphics[width=.15\columnwidth]{figures/supplementary/000034473_cnn.png}
  }
  \subfigure{%
    \includegraphics[width=.15\columnwidth]{figures/supplementary/000034473_crf.png}
  }
  \subfigure{%
    \includegraphics[width=.15\columnwidth]{figures/supplementary/000034473_ours.png}
  }\\[-2ex]


  \subfigure{%
    \includegraphics[width=.15\columnwidth]{figures/supplementary/000035463_given.png}
  }
  \subfigure{%
    \includegraphics[width=.15\columnwidth]{figures/supplementary/000035463_sp.png}
  }
  \subfigure{%
    \includegraphics[width=.15\columnwidth]{figures/supplementary/000035463_gt.png}
  }
  \subfigure{%
    \includegraphics[width=.15\columnwidth]{figures/supplementary/000035463_cnn.png}
  }
  \subfigure{%
    \includegraphics[width=.15\columnwidth]{figures/supplementary/000035463_crf.png}
  }
  \subfigure{%
    \includegraphics[width=.15\columnwidth]{figures/supplementary/000035463_ours.png}
  }\\[-2ex]


  \setcounter{subfigure}{0}
  \subfigure[\scriptsize Input]{%
    \includegraphics[width=.15\columnwidth]{figures/supplementary/000035993_given.png}
  }
  \subfigure[\scriptsize Superpixels]{%
    \includegraphics[width=.15\columnwidth]{figures/supplementary/000035993_sp.png}
  }
  \subfigure[\scriptsize GT]{%
    \includegraphics[width=.15\columnwidth]{figures/supplementary/000035993_gt.png}
  }
  \subfigure[\scriptsize AlexNet]{%
    \includegraphics[width=.15\columnwidth]{figures/supplementary/000035993_cnn.png}
  }
  \subfigure[\scriptsize +DenseCRF]{%
    \includegraphics[width=.15\columnwidth]{figures/supplementary/000035993_crf.png}
  }
  \subfigure[\scriptsize Using BI]{%
    \includegraphics[width=.15\columnwidth]{figures/supplementary/000035993_ours.png}
  }
  \mycaption{Material Segmentation}{Example results of material segmentation.
  (d)~depicts the AlexNet CNN result, (e)~CNN + 10 steps of mean-field inference,
  (f)~result obtained with bilateral inception (BI) modules (\bi{7}{2}+\bi{8}{6}) between
  \fc~layers.}
\label{fig:material_visuals-app}
\end{figure*}


\definecolor{city_1}{RGB}{128, 64, 128}
\definecolor{city_2}{RGB}{244, 35, 232}
\definecolor{city_3}{RGB}{70, 70, 70}
\definecolor{city_4}{RGB}{102, 102, 156}
\definecolor{city_5}{RGB}{190, 153, 153}
\definecolor{city_6}{RGB}{153, 153, 153}
\definecolor{city_7}{RGB}{250, 170, 30}
\definecolor{city_8}{RGB}{220, 220, 0}
\definecolor{city_9}{RGB}{107, 142, 35}
\definecolor{city_10}{RGB}{152, 251, 152}
\definecolor{city_11}{RGB}{70, 130, 180}
\definecolor{city_12}{RGB}{220, 20, 60}
\definecolor{city_13}{RGB}{255, 0, 0}
\definecolor{city_14}{RGB}{0, 0, 142}
\definecolor{city_15}{RGB}{0, 0, 70}
\definecolor{city_16}{RGB}{0, 60, 100}
\definecolor{city_17}{RGB}{0, 80, 100}
\definecolor{city_18}{RGB}{0, 0, 230}
\definecolor{city_19}{RGB}{119, 11, 32}
\begin{figure*}[!ht]
  \small % scriptsize
  \centering


  \subfigure{%
    \includegraphics[width=.18\columnwidth]{figures/supplementary/frankfurt00000_016005_given.png}
  }
  \subfigure{%
    \includegraphics[width=.18\columnwidth]{figures/supplementary/frankfurt00000_016005_sp.png}
  }
  \subfigure{%
    \includegraphics[width=.18\columnwidth]{figures/supplementary/frankfurt00000_016005_gt.png}
  }
  \subfigure{%
    \includegraphics[width=.18\columnwidth]{figures/supplementary/frankfurt00000_016005_cnn.png}
  }
  \subfigure{%
    \includegraphics[width=.18\columnwidth]{figures/supplementary/frankfurt00000_016005_ours.png}
  }\\[-2ex]

  \subfigure{%
    \includegraphics[width=.18\columnwidth]{figures/supplementary/frankfurt00000_004617_given.png}
  }
  \subfigure{%
    \includegraphics[width=.18\columnwidth]{figures/supplementary/frankfurt00000_004617_sp.png}
  }
  \subfigure{%
    \includegraphics[width=.18\columnwidth]{figures/supplementary/frankfurt00000_004617_gt.png}
  }
  \subfigure{%
    \includegraphics[width=.18\columnwidth]{figures/supplementary/frankfurt00000_004617_cnn.png}
  }
  \subfigure{%
    \includegraphics[width=.18\columnwidth]{figures/supplementary/frankfurt00000_004617_ours.png}
  }\\[-2ex]

  \subfigure{%
    \includegraphics[width=.18\columnwidth]{figures/supplementary/frankfurt00000_020880_given.png}
  }
  \subfigure{%
    \includegraphics[width=.18\columnwidth]{figures/supplementary/frankfurt00000_020880_sp.png}
  }
  \subfigure{%
    \includegraphics[width=.18\columnwidth]{figures/supplementary/frankfurt00000_020880_gt.png}
  }
  \subfigure{%
    \includegraphics[width=.18\columnwidth]{figures/supplementary/frankfurt00000_020880_cnn.png}
  }
  \subfigure{%
    \includegraphics[width=.18\columnwidth]{figures/supplementary/frankfurt00000_020880_ours.png}
  }\\[-2ex]



  \subfigure{%
    \includegraphics[width=.18\columnwidth]{figures/supplementary/frankfurt00001_007285_given.png}
  }
  \subfigure{%
    \includegraphics[width=.18\columnwidth]{figures/supplementary/frankfurt00001_007285_sp.png}
  }
  \subfigure{%
    \includegraphics[width=.18\columnwidth]{figures/supplementary/frankfurt00001_007285_gt.png}
  }
  \subfigure{%
    \includegraphics[width=.18\columnwidth]{figures/supplementary/frankfurt00001_007285_cnn.png}
  }
  \subfigure{%
    \includegraphics[width=.18\columnwidth]{figures/supplementary/frankfurt00001_007285_ours.png}
  }\\[-2ex]


  \subfigure{%
    \includegraphics[width=.18\columnwidth]{figures/supplementary/frankfurt00001_059789_given.png}
  }
  \subfigure{%
    \includegraphics[width=.18\columnwidth]{figures/supplementary/frankfurt00001_059789_sp.png}
  }
  \subfigure{%
    \includegraphics[width=.18\columnwidth]{figures/supplementary/frankfurt00001_059789_gt.png}
  }
  \subfigure{%
    \includegraphics[width=.18\columnwidth]{figures/supplementary/frankfurt00001_059789_cnn.png}
  }
  \subfigure{%
    \includegraphics[width=.18\columnwidth]{figures/supplementary/frankfurt00001_059789_ours.png}
  }\\[-2ex]


  \subfigure{%
    \includegraphics[width=.18\columnwidth]{figures/supplementary/frankfurt00001_068208_given.png}
  }
  \subfigure{%
    \includegraphics[width=.18\columnwidth]{figures/supplementary/frankfurt00001_068208_sp.png}
  }
  \subfigure{%
    \includegraphics[width=.18\columnwidth]{figures/supplementary/frankfurt00001_068208_gt.png}
  }
  \subfigure{%
    \includegraphics[width=.18\columnwidth]{figures/supplementary/frankfurt00001_068208_cnn.png}
  }
  \subfigure{%
    \includegraphics[width=.18\columnwidth]{figures/supplementary/frankfurt00001_068208_ours.png}
  }\\[-2ex]

  \subfigure{%
    \includegraphics[width=.18\columnwidth]{figures/supplementary/frankfurt00001_082466_given.png}
  }
  \subfigure{%
    \includegraphics[width=.18\columnwidth]{figures/supplementary/frankfurt00001_082466_sp.png}
  }
  \subfigure{%
    \includegraphics[width=.18\columnwidth]{figures/supplementary/frankfurt00001_082466_gt.png}
  }
  \subfigure{%
    \includegraphics[width=.18\columnwidth]{figures/supplementary/frankfurt00001_082466_cnn.png}
  }
  \subfigure{%
    \includegraphics[width=.18\columnwidth]{figures/supplementary/frankfurt00001_082466_ours.png}
  }\\[-2ex]

  \subfigure{%
    \includegraphics[width=.18\columnwidth]{figures/supplementary/lindau00033_000019_given.png}
  }
  \subfigure{%
    \includegraphics[width=.18\columnwidth]{figures/supplementary/lindau00033_000019_sp.png}
  }
  \subfigure{%
    \includegraphics[width=.18\columnwidth]{figures/supplementary/lindau00033_000019_gt.png}
  }
  \subfigure{%
    \includegraphics[width=.18\columnwidth]{figures/supplementary/lindau00033_000019_cnn.png}
  }
  \subfigure{%
    \includegraphics[width=.18\columnwidth]{figures/supplementary/lindau00033_000019_ours.png}
  }\\[-2ex]

  \subfigure{%
    \includegraphics[width=.18\columnwidth]{figures/supplementary/lindau00052_000019_given.png}
  }
  \subfigure{%
    \includegraphics[width=.18\columnwidth]{figures/supplementary/lindau00052_000019_sp.png}
  }
  \subfigure{%
    \includegraphics[width=.18\columnwidth]{figures/supplementary/lindau00052_000019_gt.png}
  }
  \subfigure{%
    \includegraphics[width=.18\columnwidth]{figures/supplementary/lindau00052_000019_cnn.png}
  }
  \subfigure{%
    \includegraphics[width=.18\columnwidth]{figures/supplementary/lindau00052_000019_ours.png}
  }\\[-2ex]




  \subfigure{%
    \includegraphics[width=.18\columnwidth]{figures/supplementary/lindau00027_000019_given.png}
  }
  \subfigure{%
    \includegraphics[width=.18\columnwidth]{figures/supplementary/lindau00027_000019_sp.png}
  }
  \subfigure{%
    \includegraphics[width=.18\columnwidth]{figures/supplementary/lindau00027_000019_gt.png}
  }
  \subfigure{%
    \includegraphics[width=.18\columnwidth]{figures/supplementary/lindau00027_000019_cnn.png}
  }
  \subfigure{%
    \includegraphics[width=.18\columnwidth]{figures/supplementary/lindau00027_000019_ours.png}
  }\\[-2ex]



  \setcounter{subfigure}{0}
  \subfigure[\scriptsize Input]{%
    \includegraphics[width=.18\columnwidth]{figures/supplementary/lindau00029_000019_given.png}
  }
  \subfigure[\scriptsize Superpixels]{%
    \includegraphics[width=.18\columnwidth]{figures/supplementary/lindau00029_000019_sp.png}
  }
  \subfigure[\scriptsize GT]{%
    \includegraphics[width=.18\columnwidth]{figures/supplementary/lindau00029_000019_gt.png}
  }
  \subfigure[\scriptsize Deeplab]{%
    \includegraphics[width=.18\columnwidth]{figures/supplementary/lindau00029_000019_cnn.png}
  }
  \subfigure[\scriptsize Using BI]{%
    \includegraphics[width=.18\columnwidth]{figures/supplementary/lindau00029_000019_ours.png}
  }%\\[-2ex]

  \mycaption{Street Scene Segmentation}{Example results of street scene segmentation.
  (d)~depicts the DeepLab results, (e)~result obtained by adding bilateral inception (BI) modules (\bi{6}{2}+\bi{7}{6}) between \fc~layers.}
\label{fig:street_visuals-app}
\end{figure*}

\clearpage

\bibliographystyle{ACM-Reference-Format}
\bibliography{bibliography}

\end{document}
