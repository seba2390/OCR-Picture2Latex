% !TEX root =  ../pers_schedules.tex 

\begin{abstract}
Low risk prostate cancer patients enrolled in active surveillance (AS) programs commonly undergo biopsies on a frequent basis for examination of cancer progression. AS programs employ a fixed schedule of biopsies for all patients. Such fixed and frequent schedules, may schedule unnecessary biopsies for the patients. Since biopsies have an associated risk of complications, patients do not always comply with the schedule, which increases the risk of delayed detection of cancer progression. Motivated by the world's largest AS program, Prostate Cancer Research International Active Surveillance (PRIAS), in this paper we present personalized schedules for biopsies to counter these problems. Using joint models for time to event and longitudinal data, our methods combine information from historical prostate-specific antigen (PSA) levels and repeat biopsy results of a patient, to schedule the next biopsy. We also present methods to compare personalized schedules with existing biopsy schedules.
\end{abstract}