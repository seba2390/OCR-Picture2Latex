

\section{Numerical Simulations}
\label{sec:ex}

\subsection{Idealized Experiment I}
We consider an
experiment with $\Nr=20$, $\tau=30$ and $T=400$.
%
The aim is to
reconstruct the true impulse responses $\vecc{g^{0}_{i}}$,
plotted in Figure~\ref{fig:simple_example_bd0}a, 
from the channel outputs generated using a 
Gaussian random source signature $s^{0}$.
%
The impulse responses of 
similar kind  
are of 
particular interest in seismic inversion and room acoustics 
as they reveal the arrival
of 
energy,
propagated from an impulsive source,
at the receivers in the medium.
%
In this case, the arrivals have onsets of 6$\,$s and 10$\,$s at the first channel and they 
curve 
linearly and hyperbolically, respectively.
%
The linear arrival is 
the earliest arrival that doesn't undergo scattering.
%
The hyperbolic arrival is likely to represent a wave that is 
reflected or scattered from an interface between two materials with different acoustic impedances.
%
\subsubsection*{LSBD}

\begin{figure*}
	\begin{center}
		\IfFileExists{./FIG/simple_example_bd0.png}
		{
		\includegraphics[width=0.95\textwidth]{./FIG/simple_example_bd0.png}
			}
		{
	\begin{tikzpicture}
		\input{./codes/simple_example_bd0.tex}
	\end{tikzpicture}
		}
	\end{center}
	\caption{
		Idealized Experiment I.
	 The results are displayed as images that use 
	 the full range of colors in a colormap. 
	 Each pixel of these images corresponds to a time $t$ and a channel index $i$.
	 %
	 Impulse responses: a) true; b)---d) undesired.
	 }
	\label{fig:simple_example_bd0}
\end{figure*}


\begin{figure*}
	\begin{center}
		\IfFileExists{./FIG/simple_example_bd1.png}
		{
		\includegraphics[width=0.95\textwidth]{./FIG/simple_example_bd1.png}
			}
		{
	\begin{tikzpicture}
		\input{./codes/simple_example_bd1.tex}
	\end{tikzpicture}
		}
	\end{center}
	\caption{
		Idealized Experiment I.
		Cross-correlations of impulse responses corresponding to the Figure~\ref{fig:simple_example_bd0}:
	 a) true; b)---d) undesired.
	 }
	\label{fig:simple_example_bd1}
\end{figure*}


\begin{figure*}
	\centering
	\IfFileExists{./FIG/simple_example_fbd.png}
	{
	\includegraphics[width=0.95\textwidth]{./FIG/simple_example_fbd.png}
		}
	{
	\begin{tikzpicture}
		\input{./codes/simple_example_fbd.tex}
	\end{tikzpicture}
	}
	\caption{ Idealized Experiment I. 
		a) FIBD estimated interferometric 
		impulse responses corresponding to the Figure~\ref{fig:simple_example_bd1}a, 
		after fitting the
		interferometric channel outputs.
		b) Same as (a), except after white noise is added to the channel outputs.
		%
		c) Estimated impulse responses from FPR by fitting the FIBD-outcome interferometric impulse responses in (a).
		d) Same as (c), except fitting the FIBD outcome in (b).
	}
	\label{fig:simple_example_fbd}
\end{figure*}




\begin{figure*}
	\begin{center}
		\IfFileExists{./FIG/simple_example_bd2.png}
		{
		\includegraphics[width=0.95\textwidth]{./FIG/simple_example_bd2.png}
			}
		{
	\begin{tikzpicture}
		\input{./codes/simple_example_bd2.tex}
	\end{tikzpicture}
		}
	\end{center}
	\caption{
		Idealized Experiment I.
	 Normalized cumulative energy of: a) true; b)---d) undesired impulse responses corresponding to the Figure~\ref{fig:simple_example_bd0}.
	 }
	\label{fig:simple_example_bd2}
\end{figure*}


To illustrate its non-uniqueness, we use 
three different initial estimates of 
$s$ and $\vecc{g_i}$ to observe the convergence to 
three different solutions that belong to $\mathbb{P}$.
%
The channel responses corresponding to these solutions are plotted in 
Figures~\ref{fig:simple_example_bd0}b--d.
%
At the convergence, the misfit (given in eq.~\ref{eqn:bd1}) 
in all these three cases $U(\hat{s},\vecc{\hat{g}_i})\lessapprox 10^{-6}$, justifying non-uniqueness.
%
Moreover, we notice that none of the 
solutions 
is desirable due to 
insufficient resolution.


\subsubsection*{FIBD}

In order to isolate the indeterminacy due to the 
amplitude spectrum of the unknown filter $\phi(t)$ in eq.~\ref{eqn:decont2}
and 
justify the use of the 
focusing constraint in eq.~\ref{eqn:fibd1},
we plot 
the true and undesirable impulse responses after cross-correlation
in the Figure~\ref{fig:simple_example_bd1}.
%
It can be easily noticed
that the true impulse-response cross-correlations corresponding to the first channel
are more focused at $t=0$ 
than the 
undesirable impulse-response cross-correlations.
%
The defocusing is caused by the ambiguity 
related to the amplitude spectrum of $\phi(t)$.
%
FIBD in Algorithm~\ref{alg:fibd} with $\vec{\alpha}=[\infty,0.0]$ 
resolves this ambiguity 
and
satisfactorily recovers the true
interferometric impulse responses $\vecc{g^0_{ij}}$, as plotted in Figure~\ref{fig:simple_example_fbd}a.
%
We regard the FIBD recovery be satisfactory in Figure~\ref{fig:simple_example_fbd}b
when the 
Gaussian white noise is added to the channel outputs so that the signal-to-noise (SNR) is $1\,$dB.

\subsubsection*{FPR}


In order to motivate the use of the second focusing constraint,
we plotted the 
normalized 
cumulative energy 
of the 
true and undesired impulse responses
in the Figures~\ref{fig:simple_example_bd2}.
%
It can be easily noticed that the fastest 
rate of energy buildup in time occurs in the 
case of the true impulse responses.
%
In other words, the energy of the 
true impulse responses is more front-loaded compared to 
undesired impulse responses, after neglecting an overall translation in time.
%
The FPR in Algorithm~\ref{alg:fpr} with $\vec{\beta}=[\infty, 0]$ 
satisfactorily recovers $\vecc{g^0_{i}}$ that are plotted in: the 
Figure~\ref{fig:simple_example_fbd}c --- utilizing $\vecc{g_{ij}}$ recovered from the noiseless channel outputs (Figure~\ref{fig:simple_example_fbd}a);
the Figure~\ref{fig:simple_example_fbd}d --- utilizing $\vecc{g_{ij}}$ 
recovered from the channel outputs (Figure~\ref{fig:simple_example_fbd}b) with Gaussian white noise.
Note that the overall time translation and scaling cannot be fundamentally determined.
%

\subsection{Idealized Experiment II}
This IBD-benchmark 
experiment with $\Nr=20$ $\tau=30$ and $T=400$
aims to reconstruct simpler  
interferometric impulse responses, plotted in 
Figure~\ref{fig:simple_example_ibd_fail}b, corresponding to the 
true impulse responses in Figure~\ref{fig:simple_example_ibd_fail}a.
%
A satisfactory recovery of $\vecc{g^0_{ij}}$ is not achievable without the focusing constraint ---
the IBD outcome in the Figure~\ref{fig:simple_example_ibd_fail}c
doesn't match the true 
interferometric impulse responses in the Figure~\ref{fig:simple_example_ibd_fail}b, unlike FIBD in the Figure~\ref{fig:simple_example_ibd_fail}d.

\begin{figure*}
	\centering
	\IfFileExists{./FIG/simple_example_ibd_fail.png}
	{
	\includegraphics[width=0.95\textwidth]{./FIG/simple_example_ibd_fail.png}
		}
	{
	\begin{tikzpicture}
		\input{./codes/simple_example_ibd_fail.tex}
	\end{tikzpicture}
	}
	\caption{Idealized Experiment II.
	Interferometric impulse responses:
	a) true; 
	b) estimated using IBD; c) estimated using FIBD.}
	\label{fig:simple_example_ibd_fail}
\end{figure*}




\subsection{Idealized Experiment III}
We consider another 
experiment with $\Nr=20$ and $\tau=30$
to
reconstruct the true impulse responses $\vecc{g^{0}_{i}}$ 
(plotted in Figure~\ref{fig:simple_example_pr_fail}a) 
by fitting their
cross-correlations
$\vecc{g^0_{ij}}$.
%
A satisfactory recovery of $\vecc{g^0_{i}}$ from $\vecc{g^0_{ij}}$ is not achievable without the focusing constraint ---
the outcome of LSPR, in Figure~\ref{fig:simple_example_pr_fail}b,
doesn't match the true impulse responses, in Figure~\ref{fig:simple_example_pr_fail}a, but is 
contaminated by the filter $\chi(t)$ in eq.~\ref{eqn:lspr_unknown}.
On the other hand, FPR results in the outcome (Figure~\ref{fig:simple_example_pr_fail}c) 
that is not contaminated by $\chi(t)$.
\begin{figure*}
	\centering
	\IfFileExists{./FIG/simple_example_pr_fail.png}
	{
	\includegraphics[width=0.95\textwidth]{./FIG/simple_example_pr_fail.png}
		}
	{
	\begin{tikzpicture}
		\input{./codes/simple_example_pr_fail.tex}
	\end{tikzpicture}
	}
	\caption{Idealized Experiment III.
	a) True impulse responses. 
	b) Estimated impulse responses using LSPR.
	c) Estimated impulse responses using FPR.
	}
	\label{fig:simple_example_pr_fail}
\end{figure*}



\subsection{Idealized Experiment IV}
This 
experiment with $\Nr=20$,
$\tau=30$ and $T=400$  
aims to reconstruct the true 
interferometric impulse responses, plotted in 
Figure~\ref{fig:simple_example_fibd_fail}b, corresponding to the 
true impulse responses in Figure~\ref{fig:simple_example_fibd_fail}a.
%
The outcome of 
FIBD with $\vec{\alpha}=[\infty,0]$, plotted in Figure~\ref{fig:simple_example_fibd_fail}c,
doesn't clearly match the true 
interferometric impulse responses
because the channels are \emph{not} sufficiently dissimilar.
%
In this regard,
observe that the Figure-\ref{fig:simple_example_fibd_fail}a
true impulse responses 
at various channels 
$i$ differ only by a fixed time-translation instead 
of curving as in Figure~\ref{fig:simple_example_bd0}a.
%
\begin{figure*}
	\centering
	\IfFileExists{./FIG/simple_example_fibd_fail.png}
	{
	\includegraphics[width=0.95\textwidth]{./FIG/simple_example_fibd_fail.png}
		}
	{
	\begin{tikzpicture}
		\input{./codes/simple_example_fibd_fail.tex}
	\end{tikzpicture}
	}
	\caption{ Idealized Experiment IV. 
		a) True impulse responses of channels that are not sufficiently dissimilar.
		b) True interferometric impulse responses corresponding to (a).
		c) FIBD estimated interferometric 
		impulse responses corresponding to (b), 
		after fitting the
		interferometric channel outputs.
		%
	}
	\label{fig:simple_example_fibd_fail}
\end{figure*}


\subsection{Idealized Experiment V}
We consider another 
experiment with $\Nr=20$ 
and $\tau=30$
to reconstruct the true impulse responses $\vecc{g^{0}_{i}}$ 
(plotted in Figure~\ref{fig:simple_example_fpr_fail}a) that are 
\emph{not} front-loaded, by fitting their
cross-correlations
$\vecc{g^0_{ij}}$.
%
The FPR estimated impulse responses $\vecc{\hat{g}_{i}}$, plotted in 
Figure~\ref{fig:simple_example_fpr_fail}b, 
do not 
clearly depict the arrivals 
because there exists a spurious
$\chi\ne\delta$ obeying 
eq.~\ref{eqn:lspr_unknown}, such that $\vecc{g^0_{i}\ast \chi}$ are
more front-loaded than $\vecc{g^0_{i}}$.
%
We observe that
FPR typically doesn't result in a favorable outcome 
if the impulse responses are not front-loaded.
%
Otherwise, 
the 
front-loaded 
$\vecc{g^{0}_{i}}$, plotted in Figure~\ref{fig:simple_example_fpr_fail}c, 
are successfully reconstructed in Figure~\ref{fig:simple_example_fpr_fail}d,
except 
for an overall translation in time.

\begin{figure*}
	\centering
	\IfFileExists{./FIG/simple_example_fpr_fail.png}
	{
	\includegraphics[width=0.95\textwidth]{./FIG/simple_example_fpr_fail.png}
		}
	{
	\begin{tikzpicture}
		\input{./codes/simple_example_fpr_fail.tex}
	\end{tikzpicture}
	}
	\caption{
		Idealized Experiment V.
		a) True impulse responses that are not front-loaded.
		b) FPR estimated impulse responses corresponding to (a), 
		after fitting the
		true interferometric impulse responses.
		%
		c) Same as (a), but front-loaded.
		d) Same as (b), but corresponding to (c).
	}
	\label{fig:simple_example_fpr_fail}
\end{figure*}




\section{Green's function Retrieval}

\begin{figure*}
	\centering
	\centering
	\IfFileExists{./FIG/wav.png}
	{
	\includegraphics[width=0.5\textwidth]{./FIG/wav.png}
		}
		{
	\input{codes/main_wavobs.tex}
	}
 \caption{
	 Source signature for the seismic experiment.
%	 For the drill-bit source used in the synthetic experiments:
	 (a) auto-correlation that contaminates the interferometric Green's functions in the time domain ---
	 only 5\% of $T$ is plotted;
	 (b) power spectrum, where the Nyquist frequency is $60\,$Hz.
 }
 \label{fig:wavobs}
\end{figure*}
%
%\begin{figure*}
%	\begin{center}
%	\begin{tikzpicture}
%		\input{./codes/reflecvh.tex}
%	\end{tikzpicture}
%	\end{center}
%	\caption{Undesired solutions are shown in black}
%	\label{fig:marmousi}
%\end{figure*}
%
%\begin{figure*}
%	\begin{center}
%	\begin{tikzpicture}
%		\input{./codes/reflech.tex}
%	\end{tikzpicture}
%	\end{center}
%	\caption{Undesired solutions are shown in black}
%	\label{fig:marmousi}
%\end{figure*}


\begin{figure*}
	\centering
	\IfFileExists{./FIG/marmousi.png}
	{
	\includegraphics[width=0.95\textwidth]{./FIG/marmousi.png}
		}
	{
	\begin{tikzpicture}
		\input{./codes/marmousi.tex}
	\end{tikzpicture}
	}
 \caption{
	 Seismic Experiment.
	 a) Acoustic velocity model for wave propagation.
	 b) Acoustic impedance model depicting interfaces that reflect waves.
	 c) True interferometric Green's functions.
	 d) Seismic interferometry by cross-correlation.
	 e) FIBD estimated interferometric Green's functions.
	 f) True Green's functions.
	 g) FBD estimated Green's functions.
 }
 \label{fig:reflec_main_marm}
\end{figure*}



%
Finally, we consider a more realistic scenario 
involving seismic-wave propagation
in a 
complex 2-D structural model, which is 
commonly known as the Marmousi model \citep{brougois1990marmousi} 
in exploration seismology. 
%
The Marmousi 
P-wave velocity and impedance plots are  
in the Figures~\ref{fig:reflec_main_marm}a and \ref{fig:reflec_main_marm}b, respectively.
%
We inject an unknown band-limited source signal, e.g., due to a drill bit,
into this model for $30\,$s, such that $T=3600$. 
%
The signal's auto-correlation 
and power spectrum are plotted in Figures~\ref{fig:wavobs}a and \ref{fig:wavobs}b, respectively.
%
We used an acoustic time-domain 
staggered-grid finite-difference solver for wave-equation modeling.
%
The recorded seismic data 
at twenty %evenly-spaced
receivers spaced roughly $100\,$m apart, placed at a depth of roughly $500\,$m,
can be modeled as
the output of a linear system that convolves the source signature
with the Earth's
impulse response, i.e., its Green's function.
%
We recall that
in the seismic context:
\begin{itemize}

	\item the impulse responses $\vecc{g_i}$ correspond 
	to the 
	unique subsurface Green's function {$g(\vec{x},t)$} 
	evaluated at the receiver locations {$\vecc{\vec{x}_i}$},
	where the seismic-source signals are recorded;

	\item the channel outputs $\vecc{d_i}$ correspond to 
		the noisy subsurface wavefield $d(\vec{x},t)$ 
	recorded at the receivers 
		only for $\{0,\ldots,T\}$ ---  
		we are assuming that the source may be arbitrarily on or off %at any point
	throughout this time interval, just as in usual drilling operations;

	\item $\tau$ denotes the propagation time
	necessary for the 
	seismic energy,
	including multiple scattering, 
	traveling 
	from the source to a total of $\Nr$ receivers,
	to decrease below an ad-hoc threshold.
	%
\end{itemize}
	%
The goal of 
this experiment
is to reconstruct the subsurface Green's function vector $\vecc{g_{i}}$ 
that contains:
\begin{inparaenum}
	\item the direct arrival from the source to the receivers and
%	useful	for 
%	determination of the background velocity; {and}
\item   the scattered waves from
	various interfaces in the model.
%	reflectors due to a 
%	mass-density contrast --- necessary for imaging.
\end{inparaenum}
%
The `true' Green's functions $g^0_{i}$ and the  
interferometric Green's functions $g^0_{ij}$,
in Figures~\ref{fig:reflec_main_marm}f and \ref{fig:reflec_main_marm}c,
are generated following these steps: 
\begin{inparaenum}
\item get data for $1.5\,$s ($\tau=180$) using a Ricker source wavelet (basically a degree-2 Hermite function modulated to a 
	peak frequency of 20$\,$Hz);
\item create cross-correlated data necessary for $\vecc{g^0_{ij}}$; {and}
\item perform a deterministic deconvolution on the data 
	using the Ricker wavelet.
\end{inparaenum}
%
Observe that we have chosen
the 
propagation time to be 
$1.5\,$s, such that $T/\tau=20$.

Seismic interferometry by cross-correlation (see eq.~\ref{eqn:intro1})
fails to retrieve 
direct and the scattered arrivals in the 
true interferometric
Green's functions,
as the  
cross-correlated data $\vecc{d_{ij}}$, plotted in Figure~\ref{fig:reflec_main_marm}d,
is contaminated by the 
auto-correlation of the source signature (Figure~\ref{fig:wavobs}a).
%
Therefore, 
we use FBD to first extract the interferometric Green's functions by FIBD, plotted in 
the Figure~\ref{fig:reflec_main_marm}e, and then recover the 
Green's functions, plotted in the Figure~\ref{fig:reflec_main_marm}g, using FPR.
%
Notice that the FBD estimated Green's functions 
clearly depict 
the direct and the scattered arrivals, confirming that 
our method
doesn't suffer from the complexities in the subsurface models.




