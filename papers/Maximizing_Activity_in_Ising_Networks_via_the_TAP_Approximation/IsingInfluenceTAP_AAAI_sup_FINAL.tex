\def\year{2018}\relax
%File: formatting-instruction.tex
\documentclass[letterpaper]{article} %DO NOT CHANGE THIS
\usepackage{aaai18}  %Required
\usepackage{times}  %Required
\usepackage{helvet}  %Required
\usepackage{courier}  %Required
\usepackage{url}  %Required
\usepackage{graphicx}  %Required

\usepackage{bm}
\usepackage{amsmath}
\usepackage{amssymb}
\usepackage[ruled, linesnumbered]{algorithm2e}
\DeclareMathOperator*{\argmax}{arg\,max}
\DeclareMathOperator{\sign}{sign}

\frenchspacing  %Required
\setlength{\pdfpagewidth}{8.5in}  %Required
\setlength{\pdfpageheight}{11in}  %Required
%PDF Info Is Required:
  \pdfinfo{
/Title (Supplementary Information: Maximizing Activity in Ising Networks via the TAP Approximation)
/Author (Christopher W. Lynn, Daniel D. Lee)}
\setcounter{secnumdepth}{0}  
 \begin{document}
% The file aaai.sty is the style file for AAAI Press 
% proceedings, working notes, and technical reports.
%
\title{Supplementary Information: Maximizing Activity in Ising Networks via the TAP Approximation}
\author{Christopher W. Lynn \\
Department of Physics \& Astronomy \\
University of Pennsylvania \\
Philadelphia, PA 19104, USA \\
\And Daniel D. Lee \\
Department of Electrical \& Systems Engineering \\
University of Pennsylvania \\
Philadelphia, PA 19104, USA}

\maketitle

\section{Appendix A: The Plefka Expansion}

Many complex systems in the biological and social sciences have recently been shown to be quantitatively described by the Ising model from statistical physics \cite{Schneidman-01,Ganmor-01,Lynn-03,Galam-02}. The Ising model is defined by the Boltzmann distribution:
\begin{align}
\label{Boltzmann}
P(\bm{\sigma}) &= \frac{1}{Z}\exp\left(- H(\bm{\sigma})\right), \\
H(\bm{\sigma}) &= -\frac{1}{2}\sum_{ij}J_{ij}\sigma_i\sigma_j - \sum_i b_i\sigma_i, \text{and} \\
Z &= \sum_{\bm{\sigma}\in \{\pm 1\}^n} e^{-\beta H(\bm{\sigma})},
\end{align}
where $H(\bm{\sigma})$ is the energy function, or Hamiltonian, that defines the system, and $Z$ is a normalization constant. The parameters $J = \{J_{ij}\}$ define the network of interactions between elements and the parameters $\bm{b} = \{b_i\}$ represent individual biases, which can be altered by application of targeted external influence.

Since the Ising model has remained unsolved for all but a select number of special cases, tremendous interdisciplinary effort has focused on developing tractable approximation techniques. Here, we present an advanced approximation method known as the Plefka expansion \cite{Yedidia-01}. The Plefka expansion is not an approximation itself, but is rather a principled method for deriving a series of increasingly accurate approximations, the first two orders of which are the na\"{i}ve mean-field (MF) and Thouless-Anderson-Palmer (TAP) approximations.

The Plefka expansion is a small-interaction expansion derived by Georges and Yedidia in \cite{Georges-01}, extending the work of Thouless, Anderson, Palmer, and Plefka \cite{Thouless-01,Plefka-01}. Throughout this section, we assume that the interactions are symmetric (i.e., $J = J^T$) to ease the presentation; however, we note that the Plefka expansion easily generalizes to include asymmetric interactions as well \cite{Lynn-01,Zeng-01}. Furthermore, because we are expanding in the limit $J\ll 1$, it is useful to take $J\rightarrow \beta J$, where $\beta$ parameterizes the strength of interactions in the system. 

Given an Ising system defined by $J$, $\bm{b}$, and $\beta$, we consider the free energy $G = -\ln (Z)/\beta$, which can be thought of as a mathematical tool whose derivative with respect to the external field $b_i$ defines the average activity of node $i$; i.e., $\left<\sigma_i\right> = -\partial G/\partial b_i$.
To derive the Plefka expansion, we consider the approximate free energy:
\begin{equation}
\label{Gibbs}
\beta \tilde{G} = -\ln\sum_{\{\bm{\sigma}\}} \exp\Big(-\beta H(\bm{\sigma}) + \sum_i \lambda_i (\sigma_i-m_i)\Big),
\end{equation}
where $\lambda_i$ are auxiliary fields that impose the constraints $m_i=\left<\sigma_i\right>$ and are eventually set to zero to recover the true free energy. Using (\ref{Gibbs}), we can expand the free energy around the small-interaction limit $\beta=0$. Carrying out this expansion to third-order in $\beta$, we obtain \cite{Yedidia-01}:
\begin{align}
\label{Expansion}
\beta G = &\sum_i \frac{1+m_i}{2}\ln\left(\frac{1+m_i}{2}\right) + \frac{1-m_i}{2}\ln\left(\frac{1-m_i}{2}\right) \nonumber \\
&-\beta\Big(\frac{1}{2} \sum_{ij} J_{ij}m_im_j + \sum_i b_i m_i\Big) \nonumber \\
&-\frac{\beta^2}{4}\sum_{ij} J_{ij}^2(1-m_i^2)(1-m_j^2) \nonumber \\
&-\frac{\beta^3}{3}\sum_{ij} J_{ij}^3 m_i(1-m_i^2)m_j(1-m_j^2) \nonumber \\
&- \frac{\beta^3}{6}\sum_{ijk} J_{ij}J_{jk}J_{ki}(1-m_i^2)(1-m_j^2)(1-m_k^2) \nonumber \\
&+ \hdots.
\end{align}
The zeroth-order term in the expansion corresponds to the mean-field entropy and the first-order term is the mean-field energy. Thus, up to first-order in $\beta$, we recover the MF approximation. The second-order term is known as the Onsager reaction term, and its inclusion yields the TAP approximation \cite{Thouless-01}. Higher-order terms are systematic corrections first presented in \cite{Georges-01} and, in principle, can be carried out to arbitrary order.

The quantities $m_i$ approximate the average activities $\left<\sigma_i\right>$ and are defined by the stationary conditions $\partial \tilde{G}/\partial m_i=0$. Thus, differentiating (\ref{Expansion}) with respect to $m_i$, and only keeping terms to second-order in $\beta$, we arrive at the TAP self-consistency equations for the magnetizations $\bm{m}$:
\begin{align}
\label{TAP}
m_i &= \tanh\Big[b_i + \sum_j J_{ij}m_j - m_i\sum_j J_{ij}^2(1-m_j^2) \Big] \nonumber \\
&\triangleq f^{\text{TAP}}_i(\bm{m}).
\end{align}
Thus, for any order $\alpha$ of the Plefka expansion, the intractable problem of computing exact averages over the Glauber dynamics is replaced by the manageable task of computing a fixed point of the corresponding self-consistency map $\bm{m} = \bm{f}^{(\alpha)}(\bm{m})$.

\section{Appendix B: The Third-Order TAP Approximation}

We now derive the self-consistency equations and response function for the third-order approximation in the Plefka expansion. To third-order in $\beta$, we arrive at the self-consistency equations:
\begin{align}
\label{SC}
m_i &= \tanh\left[b_i + \sum_j J_{ij} m_j - m_i\sum_j J_{ij}^2(1-m_j^2) \right. \nonumber \\
&\quad \quad\quad\quad + \frac{2}{3}(1-3m_i^2)\sum_j J_{ij}^3 m_j(1-m_j^2) \nonumber \\
& \quad\quad\quad\quad \left. - m_i\sum_{jk} J_{ij}J_{jk} J_{ki} (1-m_j^2)(1-m_k^2)\right] \nonumber \\
&\triangleq f^{\text{TAP3}}_i(\bm{m}).
\end{align}

For algorithmic purposes, we are also interested in the response function $\tilde{\chi}^{(\alpha)}_{ij} = \frac{\partial m_i}{\partial b_j}$, which, for any order $\alpha$ of the Plefka expansion, takes the form:
\begin{equation}
\tilde{\chi}^{(\alpha)} = (I-Df^{(\alpha)})^{-1}A,
\end{equation}
where $I$ is the identity matrix, $Df^{(\alpha)}_{ij} = \frac{\partial f^{(\alpha)}_i}{\partial m_j}$ is the Jacobian of the mean-field map, and $A_{ij} = (1-m_i^2)\delta_{ij}$ is a diagonal matrix. Thus, the response function for each extended mean-field approximation is defined by the Jacobian of the corresponding mean-field map. Up to third-order in $\beta$, $Df^{\text{TAP3}}$ takes the form:
\begin{align}
\label{Jacobian}
Df^{\text{TAP3}}_{ij} = (1-m_i^2 &) \Bigg[ J_{ij}  + 2 J_{ij}^2m_im_j \nonumber \\
& - \delta_{ij}\sum_k J_{ik}^2(1-m_k^2) \nonumber \\
& - 4 m_i\delta_{ij}\sum_k J_{ik}^3 m_k(1-m_k^2) \nonumber \\
& + \frac{2}{3} J_{ij}^3(1-3m_i^2)(1-3m_j^2) \nonumber \\
& -  \delta_{ij} \sum_{k\ell} J_{ik}J_{k\ell}J_{\ell i} (1-m_k^2)(1-m_{\ell}^2) \nonumber \\
& + 4 J_{ij}m_i m_j \sum_k J_{jk}J_{ki} (1-m_k^2) \Bigg].
\end{align}

%References and End of Paper
%These lines must be placed at the end of your paper
\bibliography{IsingInfluenceBib.bib}
\bibliographystyle{aaai}

\end{document}
