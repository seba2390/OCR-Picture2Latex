\documentclass[aps,prl,twocolumn,amsmath,amssymb,nofootinbib,superscriptaddress]{revtex4-1}
%\documentclass{nature}


\usepackage{times}
\usepackage[pdftex]{graphicx}
\usepackage{dcolumn}
\usepackage{bm}
\usepackage{amsmath}
\usepackage{indentfirst}
\usepackage{float}
%\usepackage[colorlinks]{hyperref}
\usepackage[dvipsnames]{xcolor}
\newcommand{\comment}[1]{{\color{blue}{\textbf{#1}}}}
\bibliographystyle{apsrev4-1}



%\title{Tests of causal models using Wheeler's delayed-choice experiment}
%\title{Demonstration of a loophole-free delayed-choice experiment}
%\author{XXXXXX}


\begin{document}

\title{A loophole-free Wheeler-delayed-choice experiment}

\author{H.-L. Huang}

\affiliation{Hefei National Laboratory for Physical Sciences at Microscale and Department of Modern Physics,\\
University of Science and Technology of China, Hefei, Anhui 230026, China}
\affiliation{CAS Centre for Excellence and Synergetic Innovation Centre in Quantum Information and Quantum Physics,\\
University of Science and Technology of China, Hefei, Anhui 230026, China}
%\affiliation{CAS-Alibaba Quantum Computing Laboratory, Shanghai 201315, China}
\affiliation{Henan Key Laboratory of Quantum Information and Cryptography, Zhengzhou, Henan 450000, China}

\author{Y.-H. Luo}

\affiliation{Hefei National Laboratory for Physical Sciences at Microscale and Department of Modern Physics,\\
University of Science and Technology of China, Hefei, Anhui 230026, China}
\affiliation{CAS Centre for Excellence and Synergetic Innovation Centre in Quantum Information and Quantum Physics,\\
University of Science and Technology of China, Hefei, Anhui 230026, China}
%\affiliation{CAS-Alibaba Quantum Computing Laboratory, Shanghai 201315, China}

\author{B. Bai}
\author{Y.-H. Deng}
\author{H. Wang}
\author{H.-S. Zhong}
\author{Y.-Q. Nie}
\author{W.-H. Jiang}
\author{X.-L. Wang}
\author{J. Zhang}
\author{Li Li}
\author{Nai-Le Liu}
\affiliation{Hefei National Laboratory for Physical Sciences at Microscale and Department of Modern Physics,\\
University of Science and Technology of China, Hefei, Anhui 230026, China}
\affiliation{CAS Centre for Excellence and Synergetic Innovation Centre in Quantum Information and Quantum Physics,\\
University of Science and Technology of China, Hefei, Anhui 230026, China}
%\affiliation{CAS-Alibaba Quantum Computing Laboratory, Shanghai 201315, China}

\author{Tim Byrnes}
\affiliation{New York University Shanghai, 1555 Century Ave, Pudong, Shanghai 200122, China}
%\affiliation{State Key Laboratory of Precision Spectroscopy, School of Physical and Material Sciences,East China Normal University, Shanghai 200062, China}
\affiliation{NYU-ECNU Institute of Physics at NYU Shanghai, 3663 Zhongshan Road North, Shanghai 200062, China}
%\affiliation{National Institute of Informatics, 2-1-2 Hitotsubashi, Chiyoda-ku, Tokyo 101-8430, Japan}
\affiliation{Department of Physics, New York University, New York, NY 10003, USA}

\author{J. P. Dowling}
\affiliation{Hearne Institute for Theoretical Physics, Department of Physics and Astronomy, Louisiana State University, Baton Rouge, Louisiana 70803, USA}
\affiliation{Hefei National Laboratory for Physical Sciences at Microscale and Department of Modern Physics,\\
University of Science and Technology of China, Hefei, Anhui 230026, China}
\affiliation{NYU-ECNU Institute of Physics at NYU Shanghai, 3663 Zhongshan Road North, Shanghai 200062, China}

\author{Chao-Yang Lu}
\affiliation{Hefei National Laboratory for Physical Sciences at Microscale and Department of Modern Physics,\\
University of Science and Technology of China, Hefei, Anhui 230026, China}
\affiliation{CAS Centre for Excellence and Synergetic Innovation Centre in Quantum Information and Quantum Physics,\\
University of Science and Technology of China, Hefei, Anhui 230026, China}
%\affiliation{CAS-Alibaba Quantum Computing Laboratory, Shanghai 201315, China}

\author{Jian-Wei Pan}
\affiliation{Hefei National Laboratory for Physical Sciences at Microscale and Department of Modern Physics,\\
University of Science and Technology of China, Hefei, Anhui 230026, China}
\affiliation{CAS Centre for Excellence and Synergetic Innovation Centre in Quantum Information and Quantum Physics,\\
University of Science and Technology of China, Hefei, Anhui 230026, China}
%\affiliation{CAS-Alibaba Quantum Computing Laboratory, Shanghai 201315, China}


\date{\today}

\pacs{03.65.Ud, 03.67.Mn, 42.50.Dv, 42.50.Xa}

\begin{abstract}
Wheeler's delayed-choice experiment investigates the indeterminacy of wave-particle duality and the role played by the measurement apparatus in quantum theory. Due to the inconsistency with classical physics, it has been generally believed that it is not possible to reproduce the delayed-choice experiment using a hidden variable theory.  Recently, it was shown that this assumption was incorrect, and in fact Wheeler's delayed-choice experiment can be explained by a causal two dimensional hidden-variable theory [R. Chaves, G. B. Lemos, and J. Pienaar, Phys. Rev. Lett. 120, 190401 (2018)]. Here, we carry out an experiment of a device-independent delayed-choice experiment using photon states that are space-like separated, and demonstrate a loophole-free version of the delayed-choice protocol that is consistent with quantum theory but inconsistent with any causal two-dimensional hidden variable theory. This salvages Wheeler's thought experiment and shows that causality can be used to test quantum theory in a complementary way to the Bell and Leggett-Garg tests.
\end{abstract}

\maketitle
%\section{introduction}

After the two famous Bohr-Einstein debates of 1927 and 1930 on the validity of the quantum theory \cite{bohr1996discussion}, Einstein had to accept that quantum mechanics was correct. However, in his paper with Podolsky and Rosen (EPR) \cite{Einstein1935}, EPR claimed that quantum theory, while not incorrect, was incomplete. That paper showed that quantum-entangled states had elements of nonlocality, un-reality, and uncertainty that no ``sensible'' theory should have. The EPR paper opened the door for a replacement theory for quantum mechanics, now called a hidden variable (HV) theory, that would be more like a classical statistical theory, where the statistics were governed by HVs that were either unknown or unaccessable. Von Neumann then provided a proof that no HV theory could reproduce all the predictions of quantum theory \cite{neumann1932}. Later, Bohm produced a HV theory that reproduced all the predictions of nonrelativistic quantum theory. To solve this apparent paradox, Bell showed that von Neumann had made an explicit assumption that the HV theory was local, but that the HV theory of Bohm was nonlocal --- that is actions at one place could affect outcomes far away in apparent violation of Einstein causality \cite{Bell1965}. Thus, it is important to make the locality requirement explicit and show whether local HV theory could reproduce the predictions of quantum theory.
%Bell was then able to make the locality requirement explicit and was able to show that no local HV theory could reproduce all the predictions of quantum theory, and also provided a simple test that could determine which was correct \cite{Bell1965}. The experimental race was on.

%In 1972, Freedman and Clauser performed and experiment with entangled photon pairs that showed a small Bell violation, ruling quantum in and HV out by a significan margin \cite{Freedman1972}. Then in 1974, Holt and Pipkin circulated a preprint of an experiment finding the reverse. In 1976, Clauser showed that the Holt-Pipkin result was due to a systematic error \cite{Clauser1976}. A third experiment was carried out 1976 by Fry and Thompson, again ruling quantum theory in and HV theory out by several standard deviations. 


In an effort to challenge quantum mechanics, and make quantum predictions consistent with common sense, it was suggested that quantum particles can actually know in advance to which experiment they are going to be confronted through a HV, and thus can determine which behavior to show. For example, the photon could ``decide'' whether it was going to behave as a particle or behave as a wave before it reach the detection device in the double-slit experiment. However, Wheeler published two theory papers, now called  Wheeler's delayed-choice experiments (WDCE), that claimed to exclude the causal link between the experimental setup and a HV that predefines the photons behavior, and point out that complementarity and wave-particle duality alone contained an element of Einstein nonlocality \cite{Wheeler1978, Wheeler1984}. Fig. \ref{fig:1}(a) shows an example of an implementation of WDCE, where a photon enters an Mach-Zehnder interferometer.  Wave-like behavior or particle-like behavior can be dictated according to whether the second beamsplitter (controlled by Bob) is put in place or not.  
%An equivalent procedure is to place a phase shifter with two phase settings just before the such as to tune between the two regimes.  
So far, WDCE has been implemented experimentally in various quantum systems \cite{hellmuth1987delayed, lawson1996delayed, kim2000delayed, zeilinger2005happy, jacques2007experimental, Vedovatoe1701180}. Interestingly, a recent extension, quantum delayed-choice experiment (QDCE), suggested using a quantum beam splitter at the interferometer's output \cite{ionicioiu2011proposal, ionicioiu2014wave}, enabling us to project the test photon into an arbitrary coherent wave-particle superposition, which motivated many QDCE experiments \cite{kaiser2012entanglement, peruzzo2012quantum, rab2017entanglement, Liue1603159}. In short, until recently experimental demonstrations of WDCE (or QDCE) were thought to have perfectly ruled out the possible of quantum behavior induced by HV.

%This simplistic argument was however challenged by Wheeler in his elegant "delayed choice" arrangement \cite{Wheeler1978, Wheeler1984}. By excluding any causal link between the experimental setup and a hidden variable that predefines the photons behavior, delayed-choice experiments are usually about defining and testing wave-particle objectivity models, in which a quantum system is intrinsically either a wave or a particle. 
%These papers made the point that all true experimental EPR tests should have an element of delayed choice, where the experimenter chooses what to measure long after the photon has left the source, in order to rule out HV theories that permit subluminal communication between various parts of the experiment. 

%Over 1981--1982, Aspect $et~al.$ carried out three Bell test experiments with polarization-entangled photons, and in the third experiment (on advice from Bell) implemented a delayed choice protocol, where the choice of the measurement settings was made after left the source and were in flight to the detectors \cite{Aspect1981, Aspect1982a, Aspect1982b, Gilde2009}. However, the Aspect experiment was carried out in one large laboratory and could not rule out HV-like effects from influencing the result. Hence it was important to do the quantum-delayed-choice experiment QDCE over large enough distances so that the two detectors, Alice and Bob, and the source Charlie, were sufficiently far apart that the random choice of measurement settings could not be influenced by subluminal signals from the rest of the experiment. This is matter of not testing over sufficiently large distances is called the locality loophole. The locality loophole was first closed in 1998 over a distance of hundreds of meters by Weihs $et~al.$ \cite{Weihs1998}. It was last closed in 2017 over a distance of 1,200 km by Yin $et~al.$ in an experiment using the Chinese quantum satellite Mozi \cite{Yin2017}. 

%In addition to this loophole, there is a second called the detection loophole. Photon detectors often have poor efficiency and detect only a few percent of the photons incident on them. In Bell tests, one often makes the Fair Sampling Assumption (FSA) that the statistics of the collected photons is a fair sample of those not collected. While this seems reasonable, there are types of local HV theories which allow for hidden variables in the detectors to skew the statistics. The locality and detector loopholes have only been recently closed in a single experiment \cite{Hensen2015, Guistina2015, Shalm2015Strong}. For the standard Bell test the detector efficiency must be quite large, over 80\%, to close the detector loophole. 




Chaves, Lemos, and Pienaar recently revisited WDCE, and showed using methods in causal inference \cite{pearl2009causality, Ringbauere1600162} that the original WDCE and QDCE can in fact be modeled by a causal HV theory \cite{Chaves2018}. The HV theory they suggest follows the same causal structure of the experiment shown in Fig. \ref{fig:1}(a), such that statistics produced by the final detection can be determined from a HV and the information of the type of measurement being performed.  To overcome this shortcoming, the authors suggest a modified version of WDCE (MWDCE, Fig. \ref{fig:1}(b)) that cannot be explained by a causally structured HV theory, assuming a HV dimension of two. In the modified setup, the second beamsplitter is always put in place, and the two types of detection is controlled by the phase shifter $ \beta_j $.  The additional component is that Alice has the ability to prepare a set of states, controlled by her own phase shifter $ \alpha_i $.  Here, an important aspect is that Alice's setting $ \alpha_i $ cannot be directly communicated to Bob.  In Ref. \cite{Chaves2018} it is shown that this new setup can no longer be modeled by a causally structured HV theory, as long as the HV has a dimension of two. Such a causal HV theory can be distinguished from a genuine quantum theory by comparing the statistics to a device-independent prepare and measure (PAM) scenario \cite{Bowles2014, gallego2010device} (Fig. \ref{fig:1}(c)).  Using the statistics of the measurements and the setting values, a device-independent witness \cite{Bowles2014} can be constructed that is capable of distinguishing between a causally structured HV theory. The dimensional witness can, in principle, work with any transmittance strictly larger than zero, thus making MWDCE specially suited to loophole-free experimental implementations \cite{giustina2015significant, shalm2015strong}.




 




%One fix is to greatly enlarge the MZI so that Alice is in control of the single-photon emitter, of what we now call $\text{BS}_\text{A}$, and of her own phase shifter ${\alpha _i}$, as shown in Fig.~\ref{fig:1}(b). Meanwhile Bob is far away and controls what we now call $\text{BS}_\text{B}$, his phase shifter ${\beta _j}$, and the detector. In this new scheme--MWDCE, instead of choosing whether or not to remove the $\text{BS}_\text{B}$, Bob always inserts it in the MZI, and both Alice and Bob make independently random choices of their own phase shifters, where Bob's phase shifter is chosen after the photon passes through the Alice's phase shifter. If their labs are now sufficiently far apart, Alice's random choices of her phase shifter cannot be communicated to Bob before he chooses at random his own phase shift, and then the experiment truly tests the nonlocality of the WDCE. In the MWDCE of Fig.~\ref{fig:1}(b). Bob's measured $b_{ij}$ at his detector now correspond to the choices of both phase shifters. If Alice and Bob are truly space-like separated, and the random numbers are truly random, then we can rule out a local two-dimensional HV at Alice's source from influencing the outcome at Bob's detector \cite{Chaves2018}. 

%In particular, we denote that the gray areas are Alice (Preparer) and Bob (Measurer), respectively, and the MWDCE can be described generically by the $prepare$-$and$-$measure$ (PAM) scenario in Fig.~\ref{fig:1}(c), as described in Ref. \cite{Chaves2018}. In the PAM scenario, shown in Fig.~\ref{fig:1}(c), upon randomly pressing a button labeled by ${\alpha _i}$, Alice's black-box prepares different physical systems, represented by the density matrix $\rho _i^A$, that are then sent to Bob's black-box where the systems are measured upon randomly pressing a button labeled ${\beta _j}$, which produces an outcome labeled by ${b_{ij}}$. Using the two-dimensional witness $W$ in the PAM scenario for the MWDCE, it has been proved, in a device-independent manner, the incompatibility of a causal and local HV model with the generated data predicted by quantum theory \cite{Chaves2018}. This means the only way to explain the data with a local HV is to allow for signals traveling backwards in time, so-called retro-causal effects, which are ruled out by strict Einstein causality.












%Furthermore, Chaves $et~al.$ suggest a modified WDCE (MWDCE) experiment that truly tests quantum nonlocality, and provide new type of test they call a dimensional witness that can rule out several local HV theories in the context of MWDCE. The MWDCE can, in principle, work with any transmittance strictly larger than zero, thus making it specially suited to loophole-free experimental implementations.
%In the simplest version of a QDCE, a photon is sent from a source Alice to a distant Mach Zehnder interferometer (MZI) controlled by Bob, as in Fig.~\ref{fig:1}(a). Before the photon's arrival, Bob randomly decides to insert one of two phase shifters ${\beta _j}$ into the upper arm of his MZI, which then changes the probability of detecting the photon at his detector. This is Bob's delayed choice. It would seem that nothing Alice can do could affect Bob's choice of phase shifter \cite{Wheeler1984}. The problem with this setup is that, since Alice transmits her photon directly to Bob, Alice and Bob are never truly outside of each other's light cone, since her emitted photon is always within her light cone on its way to Bob, and hence the experiment does not truly test the nonlocal nature of quantum mechanics. 
%In fact, Chaves $et~al.$ show that the original WDCE can indeed be modeled a local hidden variable theory, and they therefore suggest a modified experiment that truly tests quantum nonlocality \cite{Chaves2018}. 







In this paper, we carry out a device-independent demonstration of the MWDCE. 
%The basic setup is shown in Fig. \ref{fig:1}(d).  
%Alice's initial state is prepared using EPR photon pair emitter prepared by Charlie, and measured in a choosable basis, such as to project the other photon into the state $ |H \rangle + e^{i \alpha_i} | V \rangle $ (we drop normalization factors $1/\sqrt 2$ for convenience).  This photon then travels to Bob, where another phase shift according to $ \beta_j $ is made, followed by interference with a polarizing beam splitter (PBS) and measurement.  Of particular importance is the causal structure of the experiment, where the choice of the initial state ($ \alpha_i $ in Fig. \ref{fig:1}(c))  is ensured to have no influence upon the 
%final detection.  This is achieved by creating a space-like separation between the initialization and detection steps (Fig. \ref{fig:3}). 
By measuring a device-independent witness we verify that two-dimensional causally structured HV theory can indeed be ruled out.  We furthermore rule out any a priori prepared correlated noise sources that may exist between Alice and Bob by violation of a dimension witness inequality.  Finally, we quantify the degree of retrocausality that would be required to reproduce our experiments, placing bounds on such hypothetical scenarios. Our experiment tests the causal structure in a WDCE, and is complementary to tests of nonlocality in quantum theory such as in a Bell test or a Leggett-Garg test \cite{Leggett1985}. 
% It is also significant in that it closes both the locality and detection loopholes simultaneously in a way that is simpler to achieve than with a full Bell test \cite{giustina2015significant, shalm2015strong}. 




%However, it is experimentally challenging to set up and balance a single large MZI, as required in the theory of the MWDCE of Chaves $et~al.$, in order to insure everybody is outside everybody else's light cone. Instead, we here make a few experimental modifications to the setup, as shown in Fig.~\ref{fig:1}(d), to make the experiment tractable. Firstly, we switch from a path-based interferometer to a polarization-based interferometer, which is much easier to align. Now the horizontal ($H$) and vertical ($V$) polarizations correspond to the upper and lower paths in the MZIs in Fig.~\ref{fig:1}(b). The phase shifts ${\alpha _i}$ and ${\beta _j}$ now correspond to relative rotations of the $H$ polarization with respect to the $V$ polarization, and the two BS in the experiment are replaced by polarizing beam splitters (PBS). 

%To get around the long-distance MZI balancing problem, instead of a single photon emitter, we use an EPR photon pair emitter located at Charlie's location $C$. Charlie is located closer to Alice than to Bob to allow Alice to make her preparation step first by measuring her half of the entangled pair after applying her polarization rotation ${\alpha _i}$ first. This collapses Bob's half of the entangled pair into a single photon to which Alice's polarization rotation ${\alpha _i}$ has been applied. Then Bob applies his polarization rotation ${\beta _j}$, before any influence from Alice can reach him, and performs his detection. We note that this is much different than the usual Bell test where Alice and Bob are symmetric in the protocol and measure correlations between their outcomes as opposed to the current scenario, and Bob measures an outcome condidtioned on Alice's preparation state. Time ordering is now important in the current scheme.

%While the setup appears radically different than the MWDCE in Fig.~\ref{fig:1}(b), it is mathematically equivalent to the PAM procedure in Fig.~\ref{fig:1}(c), the latter upon which all the analysis of Chaves $et~al.$ is made. 








% using photon states that are space-like separated, and demonstrate a loophole-free version of the delayed choice protocol, using entangled photon pairs produced from spontaneous parametric down conversion (SPDC), in a prepare-and-measure protocol. 
%In Fig.~\ref{fig:1}, we show the Wheeler-delayed choice experiment (WDCE) in (a) \cite{Wheeler1984}, the MWDCE (MWDCE) (b), and the prepare-and-measure interpretation in (c) \cite{Chaves2018}. 











\begin{figure}[tb]
\includegraphics[width=\columnwidth]{1}
\caption{Various configurations of Wheeler's delayed choice experiment (WDCE).  
(a) The original WDCE implemented with photons. Wave-like or particle-like behavior can be produced by either putting in or leaving out the lower beamsplitter (BS).  (b) The modified WDCE (MWDCE). The gray areas correspond to Alice (Preparer) and Bob (Measurer), respectively.  In the modified setup, the second beamsplitter is always put in place, Alice has the ability to prepare a set of states, controlled by her own phase shifter ${\alpha_i} \in \{ {\alpha_0} = 0,{\alpha_1} = \pi ,{\alpha_2} =  - \pi /2,{\alpha_3} = \pi /2\}$, and two types of detection is controlled by the phase shifter ${\beta _j} \in \{ {\beta _0} = \pi /2,{\beta _1} = 0\}$ of Bob. The $U$ is the upper path and $D$ is the lower path.  (c) The device-independent prepare-and-measure (PAM) scenario for the MWDCE. An initial black-box prepares different physical systems (upon pressing a button labeled by ${\alpha _i}$) that are then sent to a second black-box where the systems are measured (upon pressing a button labeled by ${\beta _j}$) to produce an outcome labeled by $d_{ij}$. (d) Schematic experimental description performed in this paper. Alice prepares a photon by measuring an EPR state in one of two basis, producing four possible states $ | H \rangle + e^{i \alpha_i} | V \rangle $.  Bob then applies a phase shift and interferes the photon using a polarizing beam splitter (PBS) to obtain the output.} \label{fig:1}
\end{figure}









%\section{Wheeler's delayed-choice experiment without hidden variables}



%In the original WDCE (see Fig.~\ref{fig:1}(a)), Bob randomly chooses to either remove or not remove the second beam splitter (BS)---long after the photon has entered a Mach-Zehnder interferometer (MZI) at Alice's lab---perhaps light years away, thereby testing the particle or wave nature of the photon, accordingly.

%which tests the particle or wave nature of the photon, respectively. 
%This is equivalent to randomly choosing to balance or unbalance, respectively, the MZI by not inserting or not inserting a phase shifter ${\beta _j} \in \{ {\beta _0} = 0,{\beta _1} = \pi \} $, respectively, as shown in Fig.~\ref{fig:1}(a). For example, when the MZI is balanced (${\beta _0} = 0$) then the photon will always exit to the detector after $\text{BS}_2$. If the MZI is maximally unbalanced (${\beta _1} =  \pi $), then the photon always exits the other port. 
%Hence by randomly changing the phase shifter, Bob can randomly choose which port the photon exits---long after the photon has entered a Mach Zehnder interferometer (MZI) at Alice's lab---perhaps light years away. 
%It would at first appear, as Wheeler proposed, that Alice can do nothing to affect the outcome of Bob's experiment. However, 
%in the original setup of Fig.~\ref{fig:1}(a), Alice's preparation and Bob's measurement are never truly space-like separated. This is because any information about the photon source could travel along with the photon and so Bob is inside Alice's light cone, and therefore some hidden information could be encoded in Alice's photon at the source, which Bob is unaware of at the detector. Indeed 
%Chaves $et~al.$ have shown that the WDCE in Fig.~\ref{fig:1}(a) can be completely explained by a causal local two-dimensional HV theory by introducing a local noise for the detector \cite{Chaves2018}. 


\begin{figure*}
\includegraphics[width=1.5\columnwidth]{2}
\caption{Experimental setup. The polarization-entangled photon pairs produced by SPDC are coupled into the single-mode fiber and sent to Alice (preparer) and to Bob (measurer), respectively. The distance between Alice and Bob is 46 meters and the length of the fiber from the entanglement source (Charlie) to Alice is shorter than that to Bob (the fiber lengths of Charlie-Alice and Charlie-Bob are 28 meters and 33 meters, respectively). Alice's photon passes through an EOM, then is measured by the APD which collpases Bob's state to $|H{\rangle _B} + {e^{i{\alpha _j}}}|V{\rangle _B}$ with $ \alpha_j \in \{0,\pi,\pm \pi/2 \} $, thereby preparing the state. Bob's photon then passes through another EOM where a random phase $ \beta_j \in \{ \pi/2, 0 \} $ is applied and is interfered with a PBS, followed by a measurement by the APD. The measurement basis of Alice and Bob are each randomly determined by two independent and space-like separated QRNGs and EOMs. In order to meet the delayed-choice condition, the measurement basis of Bob is chosen much later than that Alice, so that no causal information can reach Bob before his QNRG has fired.} \label{fig:2}
\end{figure*}





%\section{Experiment}



\begin{figure}[t]
\includegraphics[width=\columnwidth]{3}
\caption{Minkowski diagrams for the spacetime events related to Alice, Bob, and the source Charlie. All light cones are shaded gray.  (a) Alice and Bob are space-like separated as the measurement is finished by Alice and Bob before information about the other party's measurement setting could have arrived. (b) The quantum random number generators (QRNGs) at Alice and Bob finish picking a setting outside the light cone of the generation of an entangled photon pair by Charlie. All the events in our experiment are space-like separated.} \label{fig:3}
\end{figure}

%In fact, it is experimentally challenging to set up and balance a single large MZI, as required in the theory of the MWDCE of Chaves $et~al.$, in order to insure everybody is outside everybody else's light cone. Instead, we here make a few experimental modifications to the setup, as shown in Fig.~\ref{fig:1}(d), to make the experiment tractable. Firstly, we switch from a path-based interferometer to a polarization-based interferometer, which is much easier to align. Now the horizontal ($H$) and vertical ($V$) polarizations correspond to the upper and lower paths in the MZIs in Fig.~\ref{fig:1}(b). The phase shifts ${\alpha _i}$ and ${\beta _j}$ now correspond to relative rotations of the $H$ polarization with respect to the $V$ polarization, and the two BS in the experiment are replaced by polarizing beam splitters (PBS). 

%To get around the long-distance MZI balancing problem, instead of a single photon emitter, we use an EPR photon pair emitter located at Charlie's location. Charlie is located closer to Alice than to Bob to allow Alice to make her preparation step first by measuring her half of the entangled pair after applying her polarization rotation ${\alpha _i}$ first. This collapses Bob's half of the entangled pair into a single photon to which Alice's polarization rotation ${\alpha _i}$ has been applied. Then Bob applies his polarization rotation ${\beta _j}$, before any influence from Alice can reach him, and performs his detection. We note that this is much different than the usual Bell test where Alice and Bob are symmetric in the protocol and measure correlations between their outcomes as opposed to the current scenario, and Bob measures an outcome condidtioned on Alice's preparation state. Time ordering is now important in the current scheme. While the setup appears radically different than the MWDCE in Fig.~\ref{fig:1}(b), it is mathematically equivalent to the PAM procedure in Fig.~\ref{fig:1}(c), the latter upon which all the analysis of Chaves $et~al.$ is made. 

Figure \ref{fig:1}(d) shows the schematic experimental implementation for realizing the MWDCE.  The path-based interferometer as shown in Fig. \ref{fig:1}(b) is implemented in our experiment by a polarization-based interferometer.  The horizontal ($H$) and vertical ($V$) polarizations correspond to the upper and lower paths in the MZIs in Fig.~\ref{fig:1}(b). To prepare the various initial states we use an EPR photon pair emitter located at Charlie's location $C$. Charlie is located closer to Alice than to Bob to allow Alice to make her preparation step first by measuring her half of the entangled pair after applying her polarization rotation ${\alpha _i}$ first. This collapses Bob's half of the entangled pair into a single photon to which Alice's polarization rotation ${\alpha _i}$ has been applied. Then Bob applies his polarization rotation ${\beta _j}$, before any influence from Alice can reach him, and performs the interference by passing the photon through a polarizing beam splitter (PBS), followed by a measurement.  We note that this is much different than the usual Bell test where Alice and Bob are symmetric in the protocol and measure correlations between their outcomes as opposed to the current scenario, and Bob measures an outcome condidtioned on Alice's preparation state. Time ordering is now important in the current scheme.

Our experimental setup is shown in Fig. \ref{fig:2}. At Charlie's location, an ultraviolet laser pulse with a central wavelength of 394 nm, pulse duration of 150 fs, and repetition rate of 80 MHz passes through a $\beta$-barium borate (BBO) crystal to produce a polarization-entangled pairs $|{\Psi ^ + }{\rangle _{AB}} = |H{\rangle _A}|V{\rangle _B} + |V{\rangle _A}|H{\rangle _B}$ \cite{wang2016experimental}. A half-wave plate (HWP) is placed at an arm of the entangled pairs to produce the Bell state  $|{\Phi ^ + }{\rangle _{AB}} = |H{\rangle _A}|H{\rangle _B} + |V{\rangle _A}|V{\rangle _B}$. 
The two photons are then coupled to the into single mode fiber and sent to Alice (preparer) and Bob (measurer), respectively. The distance between Alice and Bob is 46 m, and the length of the fiber from the entanglement source (Charlie) to Alice is shorter than that to Bob. The electro-optic modulator (EOM) then applies a phase shift of 0  or $ \pi/2 $ chosen by a quantum random number generator (QRNG) such that Alice's photon is measured as one of four possibilities  $ |H \rangle_A \pm |V \rangle_A $ and $ |H \rangle_A \pm i |V \rangle_A $.  This causes Bob's state to collapse to $ |H \rangle_B \pm |V \rangle_B $ or $ |H \rangle_B \mp i |V \rangle_B $ respectively.  Bob then deploys his delayed-choice of applying two polarization rotations ${\beta _j} \in \{ {\beta _0} = \pi /2,{\beta _1} = 0\}$, chosen by another QRNG.  This is equivalent to measuring his photon one of the two bases $\{ |H{\rangle _B} \pm {e^{i{\beta _j}}}|V{\rangle _B} \}$ chosen by $\beta_j $.  We note that the four preparation states are produced by a combination of Alice's QRNG and the random collapse of the 
EPR state.  Thus no postselection is performed to prepare the state.  


%The problem of aligning a very large MZI is thus solved by instead using quantum remote state preparation. 


In order that there is no causal connection between Alice's choice of settings and Bob's detection, they are space-like separated at large enough distances such that the QRNG and the photon measurements are both outside their mutual lightcone (Fig. \ref{fig:3}(a)).  
In other words, in order to meet the delayed-choice condition, the measurement basis of Bob must be chosen sufficiently later than the measurement basis choice of Alice. Two synchronized signals from a central clock are used to trigger the two QNRGs. In addition, the QNRGs must fire outside the light cone of the EPR source to ensure that Charlie cannot influence the outcome of the QNRGs, as shown in Fig.~\ref{fig:3}(b). Without this additional condition on the EPR source, the experiment does not meet all the assumptions of the PAM scenario.




To measure the causal HV witness, we measure the conditional probability of the detection of the state $ |H{\rangle _B} \pm {e^{i{\beta_j}}}|V{\rangle _B} $ as a function of the settings $ ( \alpha_i, \beta_j ) $.  The matrix elements of the $(2 \times 2)$ witness matrix $W$ is defined as \cite{Bowles2014} 
%
\begin{align}
W_{k,l}= p(d_{2k-2,l-1}) - p(d_{2k-1,l-1}),
\end{align}
%
where $k,l \in \{ 1,2\} $, $ p(d_{i,j}) $ is the probability of the outcome $ d_{ij} $.  For a causally structured two dimensional HV theory one should find that $ \det (W) = 0 $, whereas according to quantum mechanics for an ideal system $ |\det (W)| = 1 $. Figure \ref{fig:4} shows a comparison of the theoretical predictions and  the experimental measurements with and without the fair-sampling assumption (FSA), which performs a postselection only on coincidence events. With the FSA, the two-dimensional witness is calculated as $|\text{det}({W})| = 0.778 \pm 0.005$ according to the measurement results. Experimental errors mainly come from higher-order events in the SPDC and the control accuracy of EOM.  Without the FSA, the two-dimensional witness is measured to be $|\text{det}({W})| = 0.0268 \pm 0.0006$, which is mainly due to the low collection efficiency. However, even without the FSA, the witness of the two dimensional HV,  $|\text{det}({W})| = 0$, is still violated by 44 standard deviations.   This means that we have --- with a great degree of confidence --- shown that our experiment is inconsistent with a causally structured HV theory with two dimensions. Thus, this scheme is highly resilient to detection inefficiencies.



%\section{Discussion}


%\subsection{Hiding in the noise}

We also test for HV models that are dependent upon hidden noise terms that could influence the output of the interferometer.   While our experiment rules out the causal influence of the operations that Alice and Bob perform on each other, it is possible that such a noise variable could be prepared long before the start of the experiment, and would not be forbidden by causality.  We use the dimension witness  \cite{gallego2010device,ahrens2012experimental}
%
\begin{align}
{I_{DW}} = \langle {D_{00}}\rangle  + \langle {D_{01}}\rangle  + \langle {D_{10}}\rangle  - \langle {D_{11}}\rangle  - \langle {D_{20}}\rangle
\end{align}
%
where  $ \langle {D_{ij}}\rangle  = p(e_{i,j}) - p(d_{i,j}) $, $ p(e_{i,j}) $ is the probability of the outcome $ e_{ij} $ (see Fig. \ref{fig:1}(d)), and the measurement settings are  $ \alpha_i \in  \{ \pi/4,  3\pi /4, - \pi /2 \} $ and  $\beta_j \in \{\pi /2,  0 \}$ as before.   Any HV theory that accounts for correlations between Alice and Bob hiding in the noise gives a strict bound of $I_{DW}^{\text{HV}} \le 3$, with quantum theory predicting $I_{DW}^Q = 1 + 2\sqrt 2  \approx 3.828$ \cite{Chaves2018, tavakoli2018self}. The data for this witness is shown in Fig. \ref{fig:5}, and yields, $I_{DW}^\text{Q} = 3.445 \pm 0.043$, with the FSA. The bound in this case is violated by 10 standard deviations. 




%\subsection{Retrocausality}


\begin{figure}[t]
\includegraphics[width=\columnwidth]{4}
\caption{Conditional probabilities for the outcome $ |H{\rangle _B} \pm {e^{i{\beta_j}}}|V{\rangle _B} $ by Bob for various 
settings $ (\alpha_i, \beta_j ) $.  Error bars represent one standard deviation, deduced from propagated Poissonian counting statistics of the raw detection events.} \label{fig:4}
\end{figure}



\begin{figure}[t]
\includegraphics[width=\columnwidth]{5}
\caption{Experimental results of the dimensional witness $I_{DW}$. The $\langle {D_{ij}}\rangle$ for the theoretical predictions and experimental measurement are shown as grey and yellow bars, respectively. Error bars represent one standard deviation, deduced from propagated Poissonian counting statistics of the raw detection events.} \label{fig:5}
\end{figure}





The above shows that a causally structured two-dimensional HV model would be inconsistent with our experimental results.  However, if one allows for the possibility of retrocausality (i.e. signaling backwards in time), it becomes possible to construct a HV theory that can account for statistics consistent with quantum experiment.  Recall if full retrocausality is allowed, then a local HV model cannot be distinguished from quantum theory \cite{lazarovici2015relativistic}. Interestingly, in Ref. \cite{Chaves2018} it is shown that quantum mechanics can be give bounds on types of retrocausal HV models that are allowed, by quantifying the degree of retrocausality contained in them. A measure of retrocausality $R$ is given by 
%
\begin{align}
R = \max \left[ {\frac{{{I_{DW}} - 3}}{4},0} \right],
\end{align}	
%							
where ${I_{DW}}$ is the same dimensional witness used above. Using our experimental estimate we obtain $ R = 0.114 \pm 0003 $, in comparison to the ideal case where ${R^\text{Q}} = (\sqrt{2} - 1)/2 \approx 0.207$.  The meaning of this is that any retrocausal model with $ R \le 0.207 $ would not be able to reproduce the results that quantum theory is capable of producing.  Taking our estimate,  any retrocausal model with $ R \le 0.114 $ could not give consistent results to our experiment.  


%\section{Conclusion}

%In the original WDCE, it has been shown that the experiment can be explained by a two-dimensional, causal, local-HV theory. 
In summary, we have demonstrated a modified WDCE experiment and measured a device-independent witness to evaluate its consistency with a causally structured two-dimensional HV model.  We have found that the witness clearly shows that our experiment is inconsistent with such a HV model.  A key component in our experiment is to preserve the same causal structure as the model provided in Ref. \cite{Chaves2018}, which was achieved by separating Alice (preparer) and Bob (measurer) by space-like distances, which that there was no causal influence on each other.  We have also excluded HV theories which assume noise between Alice and Bob has been correlated in advance of the experiment, and we put a bound on the amount of retrocausality needed to explain our data without quantum mechanics.  Thus, the original WDCE experiment is salvaged, with no more reference to the rather heuristic notion of wave-particle duality, and the evidence for quantum theory against various classes of HV theory seems uncontestable.

This work was supported by the National Natural Science Foundation of China, the Chinese Academy of Sciences, and the National Fundamental Research Program. J. P. D was supported by the National Science Foundation of the United States.

H.-L. Huang and Y.-H. Luo contributed equally to this work.





%\section*{Competing Interests}
%The authors declare no competing financial interests.

%\section*{Author Contributions}
%XX performed the calculations.  XX conceived and guided the work. XX wrote the paper.




%\section*{Methods}


%\subsection*{Device-independent witness}

%The two-dimensional witness is defined in terms of a $(2 \times 2)$ matrix $W$ whose elements are ${W_{k,l}}= p(2k - 2,l - 1) - p(2k - 1,l - 1)$, where $k,l \in \{ 1,2\} $, $p(i,j):= p(d=0|{\alpha _i},{\beta _j})$ is the conditional probability that the detector clicks ($d = 0$ corresponds to one of Bob's two detectors firing and $d = 1$ the other) given that Alice has made one of the four choices  $i \in \{ 0,1,2,3\}$ of phase offsets, ${\alpha_i} \in \{ {\alpha_0} = 0,{\alpha_1} = \pi ,{\alpha_2} =  - \pi /2,{\alpha_3} = \pi /2\}$, and Bob has made one of the two choices ${\beta _i} \in \{ {\beta _0} = \pi /2,{\beta _1} = 0\}$ \cite{Bowles2014}.

%Given the above information, and assuming quantum theory is correct, it is easy to show that $\text{det}|W| = 1$ in the ideal case without noise and loss in the quantum experiment, and $\text{det}|W| = 0$ for any classical system of dimension $ \le 2$. Thus, in the original WDCE, we have instead that the causal and two-dimensional HV theory predicts $ \text{det}  |W| = 0$. Hence, the goal of the experiment is to measure this quantity and show that the results are consistent with quantum theory, but inconsistent with the causal two-dimensional HV theory. We should point out that after all of these changes, we are no longer strictly testing the wave-particle duality of the photons, as in the WDCE, as the results would remain the same if the photon was a particle, a wave, or something in between \cite{Chaves2018}. 

$Note$ $added.$ During the final stages of manuscript preparation, we became aware of a similar work by Polino $et$ $al.$, which was carried out simultaneously and independently.

\bibliographystyle{apsrev4-1}
%\bibliographystyle{naturemag}
\bibliography{dc}


\end{document}
