\section{Introduction}
\label{sec:introduction}
The study of neutrino oscillations has seen remarkable progress in the last three decades. An increasing number of solar, atmospheric and accelerator neutrino experiments have performed precision measurements of the neutrino oscillation parameters \cite{PDG}. The experimental data is consistent with the three weakly-interacting neutrino picture (here referred to as the standard picture). Nevertheless, a number of questions remain unanswered, in particular what is the Neutrino Mass Ordering (NMO) and whether neutrino oscillations violate the CP symmetry. Upcoming experiments such as KM3NeT/ORCA \cite{loi}, SBN \cite{SBN}, DUNE \cite{DUNE}, JUNO \cite{JUNO}, Hyper-K \cite{Hyper-K}, IceCube/Gen2 \cite{ic_gentwo} and INO \cite{INO} aim to resolve these questions over the next decades. 
\\
At the same time, several short baseline (SBL) neutrino experiments have reported anomalous experimental results which are inconsistent with the standard picture. A comprehensive review can be found in Ref. \cite{sterile_review}. Such results could be explained by assuming the existence of an additional neutrino (hereafter SBL neutrino). However, the Z-width measurement \cite{Zwidth} has demonstrated that only three neutrinos can participate to weak interactions, for which they are referred as active neutrinos. Therefore, the SBL neutrino, not being able to participate to weak interactions, is called sterile. The SBL sterile neutrino should be light ($\dmf \sim 1 \,\eV$) and its presence affects the standard neutrino oscillation probabilities via its mixing with active neutrinos, in the so called 3+1 model. 
\\
Specifically, oscillations in the presence of a single sterile neutrino can be modelled by extending the standard picture to include four neutrino eigenstates. In this case, six new parameters are introduced in the model: one additional mass square difference $\dmf$, three active-sterile mixing angles $\theta_{14}$, $\theta_{24}$ and $\theta_{34}$, and two additional CP-violating phases $\delta_{14}$, $\delta_{24}$.
\\
The neutrino evolution in matter can be described by the following effective Hamiltonian:
\begin{equation}
\label{eq:Ham}
H = UH_0U^\dagger + V,
\end{equation}
\noindent where $H_0 = \mbox{diag}(0, \Delta{m^2_{21}}, \Delta{m^2_{31}}, \Delta{m^2_{41}}) / 2E$, and $V = \sqrt{2}G_F\mbox{diag(}N_e, 0, 0, N_n/2)$, with $G_F$ being the Fermi constant and $N_e$, $N_n$ representing the density of electrons and neutrons in the propagation medium. $U$ is an extended $4 \times 4$ unitary matrix relating flavour and mass eigenstates, which can be parametrised such that:
\begin{equation}
U = R_{34} \tilde{R}_{24} \tilde{R}_{14} R_{23} \tilde{R}_{13} R_{12},
\end{equation}
\noindent where $R_{jk}$ is a rotation matrix in the $j\mbox{-}k$ plane and, similarly, $\tilde{R}_{jk}$ is a generalised unitary rotation matrix with an added complex phase.
\\
 In the 3+1 model, the active-sterile mixing elements are expressed by 
\begin{eqnarray}
U_{e4} &=& \sin \theta_{14} e^{-i \delta_{14}}, \\
U_{\mu4} &= & \cos \theta_{14} \sin \theta_{24} e^{-i \delta_{24}}, \\
U_{\tau4} &=& \cos \theta_{14} \cos \theta_{24} \sin \theta_{34}.
\end{eqnarray}
Several experiments have been searching for the SBL sterile neutrino. To date, results are not fully consistent with the 3+1 model: disappearance experiments results are compatible with the standard neutrino scenario while some appearance experiments, such as LSND \cite{LSND} and MiniBooNE \cite{Miniboone}, observed significant $\nu_e$ or $\bar{\nu}_e$ excesses. The global fit of the experimental data with the 3+1 model results in a poor goodness-of-fit, suggesting the need of additional factors in order to explain all data.
\\
Even stronger bounds on the sterile parametric space come from cosmology \cite{strongboundCosmology}, which indirectly constrains the effective number of relativistic species $N_{\rm eff}$ in our Universe. Theoretically, the three active neutrinos give $N_{\rm eff} \sim 3$ \cite{Dolgov}. If a light sterile neutrino with the mixing parameters determined by SBL oscillations is included in the model, it should have been fully thermalised with the active neutrinos \cite{gariazzo2016light}.
This would require $N_{\rm eff} \sim 4$. Cosmological data measure a value of $N_{\rm eff}$ well-compatible with three neutrino species \cite{review_nu_cosmo}, showing a tension with the SBL anomalies. Such a tension is relaxed when cosmological data are combined with astrophysical measurements of cepheids, supernovae and gravitational lensing. In this case, the obtained value of $N_{\rm eff}$ is compatible with four at 68$\%$ C.L. \cite{review_nu_cosmo, gariazzo2016light}.
\\
More generally, cosmological data alone can be compatible with a sterile neutrino with a mass in the eV range only if its contribution to $N_{\rm eff}$ is very small, or with a somewhat larger $N_{\rm eff}$ only if it comes from a nearly massless sterile particle \cite{Planck, Archidiacono_2016}. 
\\
Therefore, more terrestrial and cosmological observations are necessary to understand the origin of the SBL anomalies. Moreover, new observations able to constrain the not-fully-excluded sterile neutrino region from cosmology, at very low sterile mass splittings ($\dmf \ll 1 \, \eV$) can further contribute to testing the sterile neutrino hypothesis.
\\
In this context, the role of next-generation neutrino detectors, such as KM3NeT, is relevant, given their ability to probe the sterile neutrino hypothesis with atmospheric neutrinos \cite{Razzaque_2011, Razzaque_2012}. KM3NeT is a research infrastructure hosting a network of next generation neutrino telescopes currently under construction in the Mediterranean Sea \cite{loi} and built upon the experience from the ANTARES neutrino telescope \cite{antares_loi}. Once completed, KM3NeT will consist of two detectors: (1) ORCA (Oscillation Research with Cosmics in the Abyss) near Toulon, France, optimised for GeV-scale atmospheric neutrino studies, and (2) ARCA (Astroparticle Research with Cosmics in the Abyss), in Sicily, Italy, optimised for the observation of higher-energy ($E_\nu > 1$ TeV) neutrinos from astrophysical sources. 
\\
By exploiting the natural source of atmospheric neutrinos, passing through the Earth and interacting within the detector volume, KM3NeT will perform neutrino oscillation studies over a broad range of energies (from few GeV up to PeV) and baselines (up to the Earth diameter).
Matter effects, experienced by atmospheric neutrinos during their passage through the Earth, are expected to enhance the effect of the presence of a sterile neutrino. Moreover, the wide L/E range available in KM3NeT increases its potential to investigate the existence of a sterile neutrino in the 3+1 model.
\\
This paper is focused on the ORCA capability to search for a light sterile neutrino. It will be shown that ORCA has a high potential to simultaneously constrain the active-sterile mixing angles $\theta_{14}$, $\theta_{24}$, $\theta_{34}$ and the effective angle $\theta_{\mu e}$, with three years of data taking. Particularly, the ORCA sensitivity to such parameters is competitive with other experiments for sterile neutrino mass at the eV scale, indicated by SBL anomalies, and it is able to provide even stronger constraints for extremely low sterile mass splittings ($\dmf$ down to $10^{-5} \, \eV$).
\\
This paper is organised as follows: Section \ref{sec:km3net} describes the KM3NeT/ORCA neutrino telescope. Section \ref{sec:phenomeno} discusses the 3+1 flavour model and oscillation probabilities. Section \ref{sec:analysismethod} describes the sterile neutrino analysis method, including a brief summary of the ORCA Monte Carlo (MC) simulation flow. Results on the ORCA sensitivity are presented in Section~\ref{sec:sensi}. Finally, the results are summarised and discussed in Section~\ref{sec:summary}.