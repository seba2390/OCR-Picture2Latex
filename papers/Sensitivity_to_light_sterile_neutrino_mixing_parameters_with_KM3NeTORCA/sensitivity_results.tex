\section{Sensitivity Results}
\label{sec:sensi}
The ORCA sensitivity to the active-sterile mixing angles is here presented. The Asimov dataset is obtained using the parameters in Tab. \ref{tab:benchmarkoscparam}, assuming no sterile neutrino in NO and IO. No assumption is made on NMO: the fit is marginalised over NMO. This allows to conservatively take into account degeneracies between NMO and the sterile parameters.
\\
At the SBL neutrino mass scale, $\dmf \sim 1$ $\eV$, correlated constraints in the $\theta_{24} -\theta_{34}$ parameter space are obtained. And, for a more general analysis, sensitivities to the mixing elements $\theta_{14}$, $\theta_{24}$, $\theta_{\mu e}$ and $\theta_{34}$ over the range $\dmf \in [10^{-5}, 10] \; \eV$ are presented.

\subsection{Sensitivity to $\theta_{24} -\theta_{34}$ in the large $\dmf$ limit}
As shown in Fig.\ \ref{fig:chi2paperstd3years}, in this sterile mass region, the track channel appears to be the most effective in constraining $\theta_{24}$ and $\theta_{34}$.
\\
As stated in Sec.\ \ref{sec:phenomeno}, %\ref{sec:LSMS}, 
$\delta_{24}$ highly impacts the analysis due to matter effects. Therefore, $\delta_{24}$ is kept free in the fit. Whereas, we  investigated the impact of $\theta_{14}$ and found it to be negligible, therefore $\theta_{14}$ and $\delta_{14}$ are fixed to zero in this part of the analysis.
%%%%%%%%%%%%%%%%%%%%%%%%%%%%%%%%%%%%%%%%%%%%%%%%%%%%%%%%%%%%%%%%%%%
\begin{figure}[h]
\centering
\includegraphics[width=1.01\textwidth]{./figures/sensitivity_results/OffORCA_Umu4_Utau4}
\mycaption{The $90\%$ (left) and $99\%$ C.L. (right) KM3NeT/ORCA sensitivity to the mixing parameters $\theta_{24} - \theta_{34}$, with $\dmf = 1 \,\eV$, for three years of assumed data taking. The obtained sensitivity is compared with current upper limits from ANTARES \cite{ANTARES_Sterile}, IceCube/DeepCore (IC) \cite{DeepCore_Steriles} and SK \cite{SK}. If not explicitly stated, $\delta_{24}$ is free in the fit: this applies to the results from ORCA and ANTARES. The excluded region is the one on the top right of the lines.}
\label{fig:sen-Um-Ut}
\end{figure}
%%%%%%%%%%%%%%%%%%%%%%%%%%%%%%%%%%%%%%%%%%%%%%%%%%%%%%%%%%%%%%%%%%%%%
\\
Fig.\ \ref{fig:sen-Um-Ut} shows the 90$\%$ and 99$\%$ C.L. ORCA sensitivity on $\sin^2 \theta_{24}$ and $\sin^2 \theta_{34} \cos^2 \theta_{24}$ for three years of data taking. The ORCA sensitivity is compared to upper limits from other neutrino experiments, namely ANTARES \cite{ANTARES_Sterile}, IceCube/DeepCore \cite{DeepCore_Steriles} and SK \cite{SK}. In order to highlight the impact of $\delta_{24}$ in the final constraints, ANTARES has presented upper limits \cite{ANTARES_Sterile} with $\delta_{24}$ fixed to 0 and free. Allowing $\delta_{24}$ to be free worsens the constraints on $\theta_{24}$ and $\theta_{34}$ and it needs to be considered as a free parameter by all the analyses in which Earth matter effects are not negligible. Here, only the analysis with $\delta_{24}$ free is presented. The impact of this quantity in the ORCA sensitivity can be found in Ref. \cite{icrc_steriles}: it is maximal when $\sin^2 \theta_{24} = \sin^2 \theta_{34} \cos^2 \theta_{24}$, for which case it worsens the sensitivity by about a factor of two for $\sin^2 \theta_{24}$ and a factor three for $\sin^2 \theta_{34} \cos^2 \theta_{24}$. 
\\
Due to the degeneracy driven by NMO and $\delta_{24}$, discussed in Sec. \ref{sec:phenomeno}, the ORCA Asimov dataset in NO and $\delta_{24}$ free (blue line) can be directly compared with IceCube/DeepCore IO and $\delta_{24}=0$ (red line). For SK, upper limits with IO are not available, therefore the ones with NO and $\delta_{24}=0$ are here reported.
\\
From Fig. \ref{fig:sen-Um-Ut} it can be concluded that ORCA is competitive in constraining the mixing elements $\theta_{24}$ and $\theta_{34}$, and it is expected to improve the sensitivity to $\sin^2 \theta_{34} \cos^2 \theta_{24}$ by over a factor of two with respect to current limits.

\subsection{Sensitivity to $\theta_{24}$ for different $\dmf$ values}
Fig. \ref{fig:sen-Um} shows the $90\%$ and $99\%$ C.L. ORCA sensitivity to $\sin^2\theta_{24}$ assuming three years of data taking. For this analysis, $\theta_{14}$, $\theta_{34}$, $\delta_{14}$ and $\delta_{24}$ are set free in the fit, since their effects on the results of the analysis are expected to be not negligible. 
\begin{figure}[h]
	\centering
	%\includegraphics[width=0.95\textwidth]{./figures/sensitivity_results/ORCA_Umu4_0129.png}
	\includegraphics[width=1.01\textwidth]{./figures/sensitivity_results/Umu4vsDm14_newOfficial_Cosmo.pdf}
	\mycaption{The $90\%$ (left) and $99\%$ C.L. (right) KM3NeT/ORCA sensitivity to the mixing parameter $\theta_{24}$, assuming three years of data taking. The obtained sensitivity is compared with current upper limits from cosmology \cite{strongboundCosmology}, MINOS/MINOS+ \cite{numu_disapp_minos}, IceCube (IC) \cite{IceCube_sterile} and SK \cite{SK}. The excluded region is the one on the right of the lines, for IceCube at $90\%$ C.L. it is the external region to the closed contour line.}
	\label{fig:sen-Um}
\end{figure}
\\
The ORCA sensitivity is compared with upper limits from cosmology \cite{strongboundCosmology} for which only $95\%$ C.L. are available, and upper limits from MINOS/MINOS+ \cite{numu_disapp_minos}, IceCube \cite{IceCube_sterile} and SK \cite{SK}.
\\
Both plots show that ORCA is less competitive than MINOS/MINOS+ and IceCube for HF. KM3NeT/ARCA would be better suited to test $\sin^2\theta_{24}$ in this region. In the LF region, ORCA is able to improve current limits on $\sin^2\theta_{24}$ by more than one order of magnitude.
%%%%%%%%%%%%%%%%%%%%%%%%%%%%%%%%%%%%%%%%%%%%%%%%%%%%%%%%%%%%%%%%%%%%

\subsection{Sensitivity to $\theta_{14}$ for different $\dmf$ values}
Fig. \ref{fig:sen-Ue} shows the $95\%$ C.L. ORCA sensitivity to $\sin^2\theta_{14}$ after three years of data taking. The choice to show the sensitivity at such a level of confidence is motivated by the goal to have a fair comparison with the other experiments, for which the majority of the available upper limits and sensitivity is reported at $95\%$ C.L. For this analysis, $\theta_{24}$, $\theta_{34}$, $\delta_{14}$ and $\delta_{24}$ are free in the fit, since their effects on the results of the analysis are expected to be not negligible.
\begin{figure}[H]
\centering
%\includegraphics[width=0.95\textwidth]{./figures/sensitivity_results/ORCA_Ue4_0117.png}
%\includegraphics[width=0.6\textwidth]{./figures/sensitivity_results/Ue4_Official_v3_withCosmo.pdf}
\includegraphics[width=0.55\textwidth]{./figures/sensitivity_results/OFFICIAL_Ue4_dm41.pdf}
\mycaption{The $95\%$ C.L. KM3NeT/ORCA sensitivity to the mixing parameter $\theta_{14}$, for different values of $\dmf$, for three years of data taking. Sensitivity results are compared with current upper limits from cosmology \cite{strongboundCosmology}, STEREO \cite{stereo_b}, and Daya Bay+Bugey-3 \cite{numu_disapp_minos}. Current anomaly regions are also reported, from Neutrino-4 \cite{neutrino-4}, global fits \cite{sterile_review} and reactors global fits \cite{globesfit}. The excluded region is the one on the right of the lines.}
\label{fig:sen-Ue}
\end{figure}
Fig. \ref{fig:chi2paperue43years} shows that, in the HF region, shower-like events are the most affected by $\theta_{14}$ and in the optimal energy region for ORCA ($E' < 10$ GeV). However, they are concentrated in the nearly-horizontal region ($-0.1 <\cos\theta_{Z} < -0.6$). Nevertheless, ORCA has a competitive sensitivity to Daya Bay+Bugey-3 \cite{numu_disapp_minos} and STEREO \cite{stereo_b} in the HF region. Moreover, ORCA will also be able to test part of the Neutrino-4 allowed region \cite{neutrino-4}. On the contrary, the global fit regions can not be reached with three years of data taking.

%%%%%%%%%%%%%%%%%%%%%%%%%%%%%%%%%%%%%%%%%%%%%%%%%%%%%%%%%%%%%%%%%%%%%
\subsection{Sensitivity to $|U_{\mu e}|^2$ for different $\dmf$ values}
Since ORCA can observe both $\nu_e$ and $\nu_\mu$ disappearance, the effective mixing element $|U_{\mu e}|^2 = \sin^2 2\theta_{\mu e} = 4 |U_{e4}|^2|U_{\mu4}|^2$ can be constrained directly. In this case, $\theta_{14}$ and $\theta_{24}$ are left free in the fit, however, their combination is constrained to match the appropriate $\theta_{\mu e}$ value by introducing a penalty term in the likelihood with a very small prior uncertainty of $10^{-6}$. Fig. \ref{fig:sen-Umue} shows the $90\%$ and $99\%$ C.L. ORCA sensitivity to $|U_{\mu e}|^2$, compared with current upper limits from Daya Bay+Bugey-3+MINOS/MINOS+ \cite{numu_disapp_minos}, KARMEN \cite{karmen}, and NOMAD \cite{nomad}.
\begin{figure}[h]
	\centering
	%\includegraphics[width=0.95\textwidth]{./figures/sensitivity_results/ORCA_Ue4_0117.png}
	%\includegraphics[width=0.6\textwidth]{./figures/sensitivity_results/Ue4_Official_v3_withCosmo.pdf}
	\includegraphics[width=1\textwidth]{./figures/sensitivity_results/UmuevsDm14_newOfficial.pdf}
	\mycaption{The $90\%$ (left) and $99\%$ C.L. (right) KM3NeT/ORCA sensitivity to the mixing parameter $|U_{\mu e}|^2$, assuming three years of data taking. Sensitivity results are compared with current upper limits from Daya Bay+Bugey-3+MINOS/MINOS+\cite{numu_disapp_minos}, KARMEN \cite{karmen} and NOMAD \cite{nomad}. Current anomaly regions from LSND \cite{LSND} and MiniBooNE \cite{Miniboone} are also reported. The excluded region is the one on the right of the lines.}
	\label{fig:sen-Umue}
\end{figure}
\\
Fig.\ \ref{fig:sen-Umue} shows that, after three years of data taking, ORCA will be able to test the majority of the LSND \cite{LSND} and MiniBoone \cite{Miniboone} anomaly region. Moreover, current limits on $\sin^2 2\theta_{\mu e}$ will be improved by 1-2 orders of magnitude in the LF region. 

%%%%%%%%%%%%%%%%%%%%%%%%%%%%%%%%%%%%%%%%%%%%%%%%%%%%%%%%%%%%%%%%%%%

\subsection{Sensitivity to $\theta_{34}$ for different $\dmf$ values}
Fig. \ref{fig:sen-Ut} shows the ORCA sensitivity at $99\%$ C.L. to $\sin^2\theta_{34}$ after three years of data taking. Here, $\theta_{14}$, $\theta_{24}$, $\delta_{14}$ and $\delta_{24}$ are set free in the fit. Upper limits from cosmology \cite{strongboundCosmology}, IceCube/DeepCore \cite{DeepCore_Steriles} and SK \cite{SK} are also reported. In the LF region there are no upper limits on $\theta_{34}$ coming from other experiments. 
\begin{figure}[h!]
\centering
\includegraphics[width=0.55\textwidth]{./figures/sensitivity_results/Utau4_phasesFree99}
\mycaption{The $99\%$ C.L. KM3NeT/ORCA sensitivity to the mixing parameter $\theta_{34}$, for different values of $\dmf$, for three years of data taking. Sensitivity results are compared with current upper limits from cosmology \cite{strongboundCosmology}, IceCube/DeepCore \cite{DeepCore_Steriles} and SK \cite{SK}. The excluded region is the one on the right of the lines.}
\label{fig:sen-Ut}
\end{figure}
\\
ORCA is able to constrain $\theta_{34}$ over a broad range of $\dmf$. In the HF region, consistently with Fig. \ref{fig:sen-Um-Ut}, ORCA can improve current upper limits on $\sin^2\theta_{34}$ by about a factor two. 