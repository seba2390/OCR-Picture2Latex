\newcommand{\UNP}{U^{4\nu}}
\newcommand{\USM}{U^{3\nu}}

\section{Theoretical Background}
\label{sec:phenomeno}
The general solutions to the Hamiltonian in Eq. \ref{eq:Ham} have a rich phenomenology that is difficult to express in analytical form. For the purposes of this analysis, Eq.~(\ref{eq:Ham}) is solved numerically in its full form using the software package OscProb~\cite{OscProb}. Fig.~\ref{fig:massres} shows an example of the impact of the matter potential on the effective values of the squared masses (eigenvalues of Eq.~(\ref{eq:Ham})) as a function of energy, assuming a medium of constant density for illustration purposes. Four resonances can be identified in the sterile neutrino models as regions of minimal distance between consecutive masses: one related to each pair ($s_{1k}$ \footnote{$s_{jk}$, $c_{jk}$ represent $\sin\theta_{jk}$ and $\cos\theta_{jk}$ respectively.}, $\Delta{m^2_{k1}}$) (at $\sim$0.05~GeV, 4~GeV and 3~TeV), and a second-order resonance connecting $s_{23}$, $s_{24}$, $s_{34}$, and $\Delta{m^2_{31}}$ (at $\sim$100~GeV), and with a strong dependence on $\delta_{24}$ as explained in section \ref{sec:high-E}.
\begin{figure}
    \centering
    \subfloat{\includegraphics[width=0.49\textwidth]{figures/theory/res_mass_NH_3vs4.png}}
    \subfloat{\includegraphics[width=0.49\textwidth]{figures/theory/res_mass_IH_3vs4.png}}\\
    \subfloat{\includegraphics[width=0.49\textwidth]{figures/theory/res_mass_NH_0vsPi.png}}
    \subfloat{\includegraphics[width=0.49\textwidth]{figures/theory/res_mass_IH_0vsPi.png}}
    \caption{Effective mass-squared values, representing the eigenvalues of Eq.~(\ref{eq:Ham}) for neutrinos in both normal (left) and inverted (right) orderings, as a function of neutrino energy. Three models are shown: In the upper panels, the standard picture with three active neutrinos (3$\nu$), is compared to a model with one light sterile neutrino where the CP violating phase $\delta_{24}$ is set to $0$. In the lower panels, two sterile neutrino models are compared with $\delta_{24}$ set to either $0$ or $\pi$. The absolute mass scale has been chosen so that the lightest neutrino is massless in vacuum. The oscillation parameters were set to $\Delta{m^2_{21}}=7.5\times10^{-5}~\mathrm{eV}^2$, $|\Delta{m^2_{31}}|=2.5\times10^{-3}~\mathrm{eV}^2$, $|\Delta{m^2_{41}}|=1~\mathrm{eV}^2$, $s_{12}^2=0.3$, $s_{13}^2=0.02$, $s_{23}^2=0.57$, $s_{14}^2=0.01$, $s_{24}^2=s_{34}^2=0.04$, and $\delta_{13}=\delta_{14}=0$. The matter density is set to 8.5~g/cm$^3$ with a ratio $N_n/N_e=1.08$.}
    \label{fig:massres}
\end{figure}
The eigenvectors of Eq.~(\ref{eq:Ham}) define an effective mixing matrix which, in a medium of constant density, can be used to compute oscillation probabilities by direct replacement in the vacuum oscillation formula:
\begin{equation}
    \label{eq:probVac}
    \begin{array}{rl}
    P_{\alpha\beta}=\delta_{\alpha\beta}&-\ \sum_{j>k}4\,\mathrm{Re}[U_{\alpha{j}}U^*_{\beta{j}}U^*_{\alpha{k}}U_{\beta{k}}]\,\sin^2\frac{\Delta_{jk}L}{2}\\
    &-\ \sum_{j>k}2\,\mathrm{Im}[U_{\alpha{j}}U^*_{\beta{j}}U^*_{\alpha{k}}U_{\beta{k}}]\,\sin\Delta_{jk}L,
    \end{array}
\end{equation}
\noindent where $\Delta_{jk}=\Delta{m^2_{jk}}/2E$. Fig.~\ref{fig:mixres} shows examples of effective values of the magnitude of some terms from Eq.~(\ref{eq:probVac}) as a function of neutrino energy. The impact of the aforementioned resonances can be readily identified.
\begin{figure}
    \centering
    \subfloat{\includegraphics[width=0.49\textwidth]{figures/theory/res_nue_3vs4.png}}
    \subfloat{\includegraphics[width=0.49\textwidth]{figures/theory/res_numu_3vs4.png}}\\
    \subfloat{\includegraphics[width=0.49\textwidth]{figures/theory/res_nue_0vsPi.png}}
    \subfloat{\includegraphics[width=0.49\textwidth]{figures/theory/res_numu_0vsPi.png}}
    \caption{Effective oscillation magnitudes associated with $\Delta_{21}$, $\Delta_{31}$, $\Delta_{32}$, and $\Delta_{4k}$ (see Equation~\ref{eq:probVac}) as a function of neutrino energy. The latter is taken as a combination of all three mass-squared difference terms involving the fourth mass state, which are approximately of equal frequency at this scale. Left: Magnitudes associated with $\nu_e$ disappearance probabilities. Right: Magnitudes associated with $\nu_\mu$ disappearance probabilities. Top: Comparison between 3$\nu$ and a sterile neutrino scenario with $\delta_{24}=0$. Bottom: Comparison between sterile neutrino scenarios with $\delta_{24}$ set to either $0$ or $\pi$. All plots apply to neutrinos in normal ordering. The same parameters as in Fig.~\ref{fig:massres} were used.}
    \label{fig:mixres}
\end{figure}
While the full numerical solutions exemplified above can already provide some insight, some exploration of common analytical approximations can be enlightening even if not used in the analysis. They are described in the following subsections.

\subsection{Large $|\Delta{m^2_{41}}|$ limit}

Anomalous oscillation results, such as LSND and MiniBooNE, are commonly interpreted as oscillations in a higher frequency than the solar and atmospheric scales. Under this scenario, the limit $\Delta{m^2_{41}}\rightarrow\infty$ can be considered in which all oscillations driven by $\Delta{m^2_{41}}$ are averaged out and observable only through scaling factors. Hereafter, this will be referred to as the high frequency (HF) region.
\\
Following Ref.~\cite{Maltoni:2007zf}, the mixing matrix $U$ can be split such that $U = \UNP \USM$, with $\USM = R_{23} \tilde{R}_{13} R_{12}$ containing only the active-active mixing elements, and $\UNP = R_{34} \tilde{R}_{24} \tilde{R}_{14}$ representing the active-sterile mixing. If the Hamiltonian is rotated with $\UNP$, it becomes approximately block-diagonal in the limit where $\Delta{m^2_{41}}\rightarrow\infty$:
\begin{equation}
\begin{array}{rl}
    \tilde{H} = \USM H_0 (\USM)^\dagger + (\UNP)^\dagger V \UNP 
    \approx \left(\begin{array}{cc}
        \tilde{H}^{(3)}  & 0 \\
        0  & \Delta_{41}
    \end{array}\right).
\end{array}
\end{equation}
The evolution matrix can then be expressed as:
\begin{equation}
    \label{eq:evol}
    S \approx \UNP \left(\begin{array}{cc}
        e^{-i\tilde{H}^{(3)}L} & 0 \\
        0 & e^{-i\Delta_{41}L}
    \end{array}\right) (\UNP)^\dagger .
\end{equation}
The remaining problem lies in the diagonalisation of $\tilde{H}^{(3)}$. For that, further approximations, which are valid in specific energy regimes, are employed. In general, a scale $\epsilon$ will be used to represent small quantities. The mixing parameters $s_{34}$, $s_{24}$, $s_{14}$, and $s_{13}$ will all be considered of $\mathcal{O}(\epsilon)$. Additionally, $\Delta{m^2_{21}}/\Delta{m^2_{31}}\sim s_{13}^2$ will be treated as $\mathcal{O}(\epsilon^2)$. In this approximation, probabilities can be written to $\mathcal{O}(\epsilon^2)$ as:
\begin{equation}
    P_{ee} \approx P_{ee}^{(3)} \cos2\theta_{14},
\end{equation}
\begin{equation}
    P_{e\mu} \approx c_{14}^2c_{24}^2P^{(3)}_{e\mu}+2c_{14}^2\mathrm{Re}[\UNP_{\mu2}{\UNP_{\mu1}}^*S^{(3)}_{e\mu}{S^{(3)}_{ee}}^*],
\end{equation}
\begin{equation}
    P_{\mu e} \approx c_{14}^2c_{24}^2P^{(3)}_{\mu e}+2c_{14}^2\mathrm{Re}[\UNP_{\mu2}{\UNP_{\mu1}}^*S^{(3)}_{\mu e}{S^{(3)}_{ee}}^*],
\end{equation}
\begin{equation}
    P_{\mu\mu} \approx P_{\mu\mu}^{(3)} \cos2\theta_{24} + 2c_{24}^2\mathrm{Re}[\UNP_{\mu2}\UNP_{\mu1}{}^*S^{(3)}_{\mu\mu}({S^{(3)}_{e\mu}}^*+{S^{(3)}_{\mu e}}^*)],
\end{equation}
\noindent where $\UNP_{\mu1}=-s_{14}s_{24}e^{i\delta_{14}-i\delta_{24}}$, $\UNP_{\mu2}=c_{24}$, and $S_{\alpha\beta}$ correspond to elements of the evolution matrix in Eq.~\ref{eq:evol}. The effect of mixing with sterile neutrinos is given by a scaling of the 3-neutrino submatrix probabilities. Additionally, some interference terms appear if both $s_{14}$ and $s_{24}$ are non-zero.

\subsubsection{The ORCA low energy regime}

The ORCA detector is most sensitive to neutrinos in the energy range of $3-100$ GeV considered in this analysis, crossing the Earth with paths of mean density varying between 3 and 9 g/cm$^3$. In the lower part of this energy range ($E < 10$ GeV), when $V_e=\sqrt{2}G_FN_e\sim\Delta_{31}=\Delta{m^2_{31}}/2E$, to leading order in small quantities, $\tilde{H}^{(3)}$ simplifies to:
\begin{equation}
    \tilde{H}^{(3)} \approx R_{23} \left(\begin{array}{ccc}
        V_e & 0 & \Delta_{31}c_{13}s_{13}e^{-i\delta_{13}} \\
        0 & 0 & 0 \\
        \Delta_{31}c_{13}s_{13}e^{i\delta_{13}} & 0 & \Delta_{31}c_{13}^2
    \end{array}\right) R_{23}^\dagger .
\end{equation}
This approximately 2-flavour form can be readily solved leading to the well-known MSW resonance of $\theta_{13}$:
\begin{equation}
    \tilde{H}^{(3)} \approx R_{23} \tilde{R}_{13}^m.
    \left(\begin{array}{ccc}
        -\Delta_{31}^m/2 & 0 & 0 \\
        0 & 0 & 0 \\
        0 & 0 & \Delta_{31}^m/2
    \end{array}\right) (\tilde{R}_{13}^m)^\dagger R_{23}^\dagger+\mathrm{const.},
\end{equation}
\begin{equation}
    \Delta_{31}^m = \sqrt{(\Delta_{31}\cos2\theta_{13}-V_e)^2+\Delta_{31}^2\sin^22\theta_{13}},
\end{equation}
\begin{equation}
    \sin2\theta_{13}^m = \frac{|\Delta_{31}|}{\Delta_{13}^m}\sin2\theta_{13},
\end{equation}
\noindent where $\tilde{R}_{13}^m$ represents the effective generalised unitary rotation matrix in the $1\mbox{-}3$ plane, parametrised by the effective mixing angle $\theta_{13}^m$ and the unchanged phase $\delta_{13}$.

All effects arising from the presence of sterile neutrino are constrained to vacuum-like mixing through $\UNP$ as in Eq.~(\ref{eq:evol}).

\subsubsection{The ORCA high energy regime}
\label{sec:high-E}
At higher energies ($E\gtrsim10$~GeV), the matter potential starts to dominate. However, a new resonance can still be found when $\Delta_{31}/V_n$ is of $\mathcal{O}(\epsilon^2)$. In this regime, $\tilde{H}^{(3)}$ is expressed in leading order as:
\begin{equation}
    \tilde{H}^{(3)} \approx 
    \left(\begin{array}{ccc}
        V_e & 0 & 0 \\
        0 & \Delta_{31}s_{23}^2+V_n |\UNP_{s2}|^2 & \Delta_{31}s_{23}c_{23}+V_n {\UNP_{s2}}^*\UNP_{s3} \\
        0 & \Delta_{31}s_{23}c_{23}+V_n \UNP_{s2}\UNP_{s3} & \Delta_{31}c_{23}^2+V_n|\UNP_{s3}|^2
    \end{array}\right) ,
\end{equation}
where $\UNP_{s2}=-c_{34}s_{24}e^{i\delta_{24}}$ and $\UNP_{s3}=-s_{34}$. Once again, the Hamiltonian is approximately block diagonal and can be easily solved to give:
\begin{equation}
    \tilde{H}^{(3)} \approx R_{23}^m
    \left(\begin{array}{ccc}
        V_e & 0 & 0 \\
        0 & -\Delta_{32}^m/2 & 0 \\
        0 & 0 & \Delta_{32}^m/2
    \end{array}\right) (R_{23}^m)^\dagger+\mathrm{const.},
\end{equation}
\begin{equation}
    \begin{array}{rl}
    {\Delta_{32}^m}^2 = &[\Delta_{31}\cos2\theta_{23}+(|\UNP_{s3}|^2-|\UNP_{s2}|^2)V_n]^2\ +\\
    &|\Delta_{31}\sin2\theta_{23}+2V_n {\UNP_{s2}}^*\UNP_{s3}|^2 ,
    \end{array}
\end{equation}
\begin{equation}
    \sin2\theta_{23}^m = \frac{1}{\Delta_{32}^m}|\Delta_{31}\sin2\theta_{23}+2V_n{\UNP_{s2}}^*\UNP_{s3}| .
\end{equation}
This new resonance corresponds to a second order effect that couples the $2\mathrm{-}3$ sector indirectly via $s_{24}$ and $s_{34}$. It provides a very rich structure having two main features: a resonance when $\sin2\theta_{23}^m\rightarrow1$ and an antiresonance when $\sin2\theta_{23}^m\rightarrow0$ at finite $V_n$. The resonance conditions are:
\begin{equation}
    \label{eq:vnres}
    V_n = \frac{\Delta_{31}\cos2\theta_{23}}{(|\UNP_{s2}|^2-|\UNP_{s3}|^2)} \Rightarrow  \sin2\theta_{23}^m=1 ,
\end{equation}
\begin{equation}
    \label{eq:antires}
    V_n = -\frac{\Delta_{31}\sin2\theta_{23}}{2{\UNP_{s2}}^*\UNP_{s3}} \Rightarrow  \sin2\theta_{23}^m=0 .
\end{equation}
A pole exists when both conditions are satisfied, as $\Delta_{32}^m\rightarrow0$ and no mixing is possible. The structure of these resonances is shown in Fig.~\ref{fig:23res}.
\begin{figure}
    \centering
    \subfloat{\includegraphics[width=0.49\textwidth]{figures/theory/res_dm32.png}}
    \subfloat{\includegraphics[width=0.49\textwidth]{figures/theory/res_sin23.png}}
    \caption{Effective parameters $\Delta_{32}^m/\Delta_{31}$ (left) and $\sin^22\theta_{23}^m$ (right) as a function of real values of $U_{\mu4}$ and $U_{\tau4}$, for a neutrino energy of 20 GeV. Here, $\Delta{m^2_{31}}=2.5\times10^{-3}~\mathrm{eV}^2$, $s_{23}^2=0.57$, and a matter density of 8.5~g/cm$^3$ with a ratio $N_n/N_e=1.08$ were assumed. The resonance and antiresonance described in Equations (\ref{eq:vnres}) and (\ref{eq:antires}) are visible on the right as regions of maximum and minimum $\sin^22\theta_{23}^m$. At the point where they seem to meet, a pole exists where $\Delta_{32}^m\rightarrow0$ and $\sin^22\theta_{23}^m$ becomes undefined.}
    \label{fig:23res}
\end{figure}
Since $\theta_{23}$ is close to maximal, the antiresonance of Eq.~(\ref{eq:antires}) is the most noticeable effect in this regime. The antiresonance occurs for neutrinos when $\cos\delta_{24}\Delta{m^2_{31}}<0$ or for antineutrinos when $\cos\delta_{24}\Delta{m^2_{31}}>0$, and is only exact for $\delta_{24}=0\mbox{ or }\pi$. Hence, there is a degeneracy between mass ordering and $\mathrm{sign}(\cos\delta_{24})$, enhanced by the maximal value of $\sin2\theta_{23}$, which suppresses NMO contributions from the resonance term in Eq.~(\ref{eq:vnres}).

\subsection{Finite $|\Delta{m^2_{41}}|$ regime}

At values of $\Delta{m^2_{41}}$ for which the associated oscillations cannot be averaged out, no simplifying approximations are known to us at the time of writing. In ORCA, this corresponds to values of $\Delta{m^2_{41}} \lesssim 0.1$ eV$^2$, this regime will be referred to as the low frequency (LF) region. In this case, many interference terms are present and the probability formulas can become exceedingly complex. Nevertheless, a full numerical solution is possible on all regimes considered in the analysis, and it is used to extend the results through six orders of magnitude in $\Delta{m^2_{41}}$. For simplicity, $\Delta{m^2_{41}}$ will be restricted to positive values.
