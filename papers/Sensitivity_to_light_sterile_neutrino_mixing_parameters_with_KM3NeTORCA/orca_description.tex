\section{The KM3NeT/ORCA Detector}
\label{sec:km3net}
KM3NeT/ORCA is a deep water neutrino detector under construction in the Mediterranean Sea. Its location is $42^\circ 48'$ N $06^\circ 02'$ E, about 40 km offshore from Toulon, France, at a depth of about 2450 m. Upon its completion, ORCA will consist of 115 flexible detection units (DUs), 200 m high, each comprising 18 Digital Optical Modules (DOMs). A DOM is a pressure resistant, 17-inch diameter glass sphere containing a total of 31, 3" photomultiplier tubes (PMTs) and their associated electronics. 
\\
The primary goal of ORCA is to determine the neutrino mass ordering and to make neutrino oscillation measurements, such as atmospheric parameters ($\sa$, $\ma$) as well as to search for $\nu_\tau$ appearance \cite{ORCA_NMO_Paper}. Neutrino oscillation studies \cite{loi} have demonstrated the presence of a resonance in neutrino oscillation probabilities for few-GeV ($2-8$ GeV) atmospheric neutrinos passing through the Earth. Such a resonance allows the NMO \cite{loi} measurement. 
\\
The ORCA geometrical configuration is optimised for studies with atmospheric neutrinos in the few GeV range: the horizontal spacing between DUs is $\sim 20$ m, whereas the vertical spacing between DOMs in each DU is $\sim 9$ m, with the first DOM being about 30 m above the seabed. The total instrumented volume is $6.7 \cdot 10^6$ m$^3$ (about 7 Mt of sea water). 
\\
In this energy regime, the events produced by atmospheric neutrinos interacting in water are spatially contained. In particular, two event topologies can be produced: track-like events, characterised by a long muon track, mostly from $\nu_\mu$ charged-current (CC) interactions in water, and shower-like events, characterised by events with no distinguishable tracks, mostly from $\nu_e$-CC and all neutral-current (NC) interactions, but with sizeable contributions from $\nu_\tau$-CC and $\nu_\mu$-CC events with short tracks. A track-like event in water has a length of $\sim 4$ m/GeV, whereas shower-like events have a $\log$(E/GeV) dependence, which corresponds to a size of the order of a few meters.
\\
The ORCA detector is an excellent instrument for the sterile neutrino search due to its dense configuration and to matter effects, whose impact in oscillation probabilities of GeV neutrinos travelling in the Earth is described in the next section.
\\
More details on KM3NeT/ORCA can be found in \cite{loi, ORCA_NMO_Paper}.