\section{Estimation of Visual Observables from Event-Based Optical Flow}
\label{sec:landing_eof}
This section describes our approach for estimating visual observables from event-based optical flow. While optic flow estimation is performed asynchronously, most existing control systems still operate on a periodic basis. Similarly, the proposed algorithm aims to update the estimates of visual observables at a fixed rate. For each periodic iteration, all newly detected optical flow vectors between the current iteration and the previous one form a planar optical flow field, of which the parameters are estimated. 

The algorithm is based on two components. First, newly detected optical flow vectors are grouped per direction and incorporated into a weighted least-squares estimator for the visual observables, as discussed in \cref{sec:vo_directional_flow_fields}. To enable preservation of flow field information over subsequent periodic iterations, a recursive update technique is introduced in \cref{sec:vo_recursive}. In addition, a confidence value is computed and applied to filter the visual observable estimates, as is described in \cref{sec:vo_confidence}. The estimator is evaluated in combination with our event-based optical flow algorithm in \cref{sec:results_scaled_velocity}.

\subsection{Directional Flow Field Parameter Estimation}
\label{sec:vo_directional_flow_fields}
The presented approach is based on techniques introduced in \citet{DeCroon2013} and used in \citet{Alkowatly2015,Ho2016}, in which fully defined optical flow estimates are available. Since our optical flow algorithm provides normal flow output, a regular optical flow field representation as in \cref{eq:planar_flow_field1} leads to inaccurate parameter estimates. However, in planar flow fields, normal flow may already provide sufficient information for computing the visual observables. Along the direction of the flow vector, normal flow does provide accurate information. 

An example diverging flow field with both optical flow and normal flow is sketched in \cref{fig:normalFlowField}. Note that the normal flow in some cases deviates significantly from the optical flow equivalent, which leads to significant errors when computing the flow field parameters. However, when grouped by direction (which is done in \cref{fig:normalFlowField} through the arrow colors), the normal flow vectors indeed show the original pattern of divergence. This idea is central to the proposed directional flow fields approach.

\begin{figure}[!ht]
	\centering
	\includegraphics[width=0.35\linewidth]{normalFlowField}
	\caption{Example of a diverging flow field resulting from several randomly oriented moving edges. The gray vectors indicate the true flow field, while the colored vectors show the normal flow along the edge orientation. Each color indicates a group of normal flow vectors with similar direction.}
	\label{fig:normalFlowField}
\end{figure}

In order to observe flow field divergence along a normal flow direction, at least two separate normal flow vectors are required, whose positions are sufficiently apart. For example, in \cref{fig:normalFlowField} the purple group of normal flow vectors does not, by itself, provide sufficient information for perceiving divergence. Also, if the flow vectors are located in close proximity, errors in normal flow magnitude have a larger influence. In \cref{fig:normalFlowField} the green group is more sensitive to these errors than the red group, since the edges are located closely together. Grouping per direction enables assessment of the reliability of the flow field in each direction, taking the previous issues into account.

A set of $m$ directions $\lbrace\alpha_1,\alpha_2,\ldots,\alpha_N\rbrace$ is defined, where $\alpha_1=0$ and $\alpha_i-\alpha_{i-1} = \pi/m$. In this work, $m=6$ directions are used. For each newly available flow vector, we first determine the closest match of $\alpha_i$ to the flow direction $\alpha_f$. Each direction $\alpha_i$ accommodates both flow in similar and opposite direction, i.e. when $-\pi<\alpha_f<0$, a match is computed for $\alpha_f + \pi$. 

Along the selected direction $\alpha_i$, the projected normal flow position $S$ and magnitude $V$ are computed, hence obtaining a one-dimensional representation of the flow along $\alpha_i$:

\begin{equation}
\label{eq:transform_flow_field}
\left[ {\begin{array}{*{20}{c}}
	S\\
	V
	\end{array}} \right] = \left[ {\begin{array}{*{20}{c}}
	{\hat x}&{\hat y}\\
	{\hat u}&{\hat v}
	\end{array}} \right]\left[ {\begin{array}{*{20}{c}}
	{\cos \alpha_i }\\
	{\sin \alpha_i }
	\end{array}} \right]
\end{equation}

Subsequently, it is corrected for rotational motion by subtracting the normal component of the rotational flow:

\begin{equation}
\begin{aligned}
{V_T} = V &- \cos {\alpha _i}\left( {p - \hat yr - q\hat x\hat y + p{{\hat x}^2}} \right) \\~&+ \sin {\alpha _i}\left( {q - \hat xr - p\hat x\hat y + q{{\hat y}^2}} \right)
\end{aligned}
\end{equation}

For each direction, a one-dimensional flow field is maintained. From \cref{eq:planar_flow_field1} and \cref{eq:transform_flow_field}, the flow field in a single direction is expressed as:

\begin{equation}
\label{eq:flow_field_line}
V_T =  - {\vartheta _x}\cos \alpha_i  - {\vartheta _y}\sin \alpha_i  + {\vartheta _z}S
\end{equation}

To solve \cref{eq:flow_field_line} for the visual observables, a weighted least-squares solution is computed using the flow vectors from all directions. Let $\mathrm{c}_\alpha=\cos\alpha$ and $\mathrm{s}_\alpha=\sin\alpha$. The overdetermined system to be solved is composed as follows:
\begin{equation}
\label{eq:dir_flow_field_system}
\arraycolsep=1.4pt
\left[ {\begin{array}{*{20}{c}}
	{ - {\rm{c}}_{\alpha _1}}&{ - {\rm{s}}_{\alpha _1}}&{{S_{1,1}}}\\
	\vdots & \vdots & \vdots \\
	{ - {\rm{c}}_{\alpha _1}}&{ - {\rm{s}}_{\alpha _1}}&{{S_{1,{n_1}}}}\\
	{ - {\rm{c}}_{\alpha _2}}&{ - {\rm{s}}_{\alpha _2}}&{{S_{2,1}}}\\
	\vdots & \vdots & \vdots \\
	{ - {\rm{c}}_{\alpha _2}}&{ - {\rm{s}}_{\alpha _2}}&{{S_{1,{n_2}}}}\\
	\vdots & \vdots & \vdots \\
	{ - {\rm{c}}_{\alpha _m}}&{ - {\rm{s}}_{\alpha _m}}&{{S_{m,{n_m}}}}
	\end{array}} \right]\left[ {\begin{array}{*{20}{c}}
	{{\vartheta _x}}\\
	{{\vartheta _y}}\\
	{{\vartheta _z}}
	\end{array}} \right] \approx \left[ {\begin{array}{*{20}{c}}
	{{V_{1,1}}}\\
	\vdots \\
	{{V_{1,{n_1}}}}\\
	{{V_{2,1}}}\\
	\vdots \\
	{{V_{2,{n_2}}}}\\
	\vdots \\
	{{V_{m,{n_m}}}}
	\end{array}} \right]
\end{equation}
which has the form $\mathbf{A\Theta}\approx\mathbf{y}$. The weighted least-squares solution is then obtained from the normal equations:

\begin{equation}
\label{eq:weighted_least_squares}
\mathbf{A}^T\mathbf{W}\mathbf{A}\mathbf{\Theta}=\mathbf{A}^T \mathbf{Wy}
\end{equation} 
in which $\mathbf{W}$ a diagonal matrix composed of the weights per direction:

\begin{equation}
\mathbf{W} = \mathrm{diag}\Big(W_1,\cdots,W_1,W_2,\cdots,W_2,\cdots, W_m\Big)
\end{equation}

The weight $W_i$ is used to represent the reliability of normal flow along a direction $i$ based on the spread of $S_i$ along that direction. Its value is determined by the variance $\mathrm{Var}\lbrace S_i\rbrace$. We let $W_i$ scale linearly with $\mathrm{Var}\lbrace S_i\rbrace$, up to a maximum of $\mathrm{Var}\lbrace S\rbrace_{min}$:

\begin{equation}
\label{eq:weight_variance}
{W_{i} = \left\{ {\begin{array}{*{20}{c}}
	0&{\mathrm{Var}\lbrace S_i\rbrace= 0}\\
	{\frac{\mathrm{Var}\lbrace S_i\rbrace}{\mathrm{Var}\lbrace S\rbrace_{min}}}&0<{{\mathrm{Var}\lbrace S_i\rbrace} \le {\mathrm{Var}\lbrace S\rbrace_{min}}}\\
	1&{{\mathrm{Var}\lbrace S_i\rbrace} > {\mathrm{Var}\lbrace S\rbrace_{min}}}
	\end{array}} \right.}
\end{equation}

The minimum variance $\mathrm{Var}\lbrace S\rbrace_{min}$ is set to 600 pixels$^2$. 

Note also that, through the formulation of \cref{eq:dir_flow_field_system}, directions with more normal flow estimates have a larger influence on $\mathbf{\Theta}$. Hence, directions for which more information is available, contribute more to the solution.

\subsection{Recursive Updating of the Flow Field}
\label{sec:vo_recursive}
The solution to \cref{eq:weighted_least_squares} for $\mathbf{\Theta}$ provides the estimate for the visual observables. However, depending on the sampling rate of the estimator, it is possible that, during a single periodic iteration, too few normal flow estimates are available for an accurate fit. This leads to noise peaks in the measurement of $\mathbf{\Theta}$, especially during low speed motion. To limit this effect, the matrices $\mathbf{A}$ and $\mathbf{y}$ are not completely renewed at each iteration. Instead, rows from previous iterations are retained and assigned an exponentially decreasing weight, similar to an exponential moving average filter.

For an efficient implementation of the former, $\mathbf{A}$ and $\mathbf{y}$ are not explicitly composed as shown in \cref{eq:dir_flow_field_system}. Instead, our approach operates on the normal equations in \cref{eq:weighted_least_squares}. For each direction independently, we recursively update parts of the matrices $\mathbf{B}={{\bf{A}}^T}{\bf{WA}}$ and $\mathbf{C}={{\bf{A}}^T}{\bf{Wy}}$. These matrices are composed by the following elements:

\begin{equation}
\bf{B} = \left[ {\begin{array}{*{20}{c}}
	{{b_{11}}}&{{b_{21}}}&{{b_{31}}}\\
	{{b_{21}}}&{{b_{22}}}&{{b_{32}}}\\
	{{b_{31}}}&{{b_{32}}}&{{b_{33}}}
	\end{array}} \right],\;\mathbf{C} = \left[ {\begin{array}{*{20}{c}}
	{{c_1}}\\
	{{c_2}}\\
	{{c_3}}
	\end{array}} \right]
\end{equation}

From \cref{eq:dir_flow_field_system}, it can be shown that the elements of $\mathbf{B}$ are expressed as:

\begin{equation}
\def\arraystretch{2.2}
\begin{array}{c@{\hspace{0.5em}}c@{\hspace{0.5em}}c@{\hspace{0.5em}}c@{\hspace{0.5em}}c@{\hspace{0.5em}}c@{\hspace{0.5em}}c@{\hspace{0.5em}}}
{b_{11}} &=& \sum\limits_{i=1}^m {{W_i}{n_i}{{\left( {{\rm{c}}_{\alpha _i}} \right)}^2}}, &\;& {b_{21}} &=& \sum\limits_{i=1}^m {{W_i}{n_i}{\rm{c}}_{\alpha _i}{\rm{s}}_{\alpha _i}} \\
{b_{22}} &=& \sum\limits_{i=1}^m {{W_i}{n_i}{{\left( {{\rm{s}}_{\alpha _i}} \right)}^2}}, &\;& {b_{31}} &=& \sum\limits_{i=1}^m {{W_i}{\rm{c}}_{\alpha _i}\sum\limits_{j=1}^{n_i} {S_{i,j}}}  \\
{b_{33}} &=& \sum\limits_{i=1}^m {{W_i}\sum\limits_{j=1}^{n_i} {S_{i,j}^2} }, &\;&{b_{32}} &=& \sum\limits_{i=1}^m {{W_i}{\rm{s}}_{\alpha _i}\sum\limits_{j=1}^{n_i} {{S_{i,j}}}} 
\end{array}
\end{equation}

and those of $\mathbf{C}$ are expressed as:

\begin{equation}
\def\arraystretch{2.2}
\begin{array}{c@{\hspace{0.5em}}c@{\hspace{0.5em}}c@{\hspace{0.5em}}}
{c_1} &=& \sum\limits_{i=1}^m {{W_i}{\rm{c}}_{\alpha _i}\sum\limits_{j=1}^{n_i} {{V_{i,j}}} }\\
{c_2} &=& \sum\limits_{i=1}^m {{W_i}{\rm{s}}_{\alpha _i}\sum\limits_{j=1}^{n_i} {{V_{i,j}}} }\\
{c_3} &=& \sum\limits_{i=1}^m {{W_i}\sum\limits_{j=1}^{n_i} {{S_{i,j}}{V_{i,j}}} } 
\end{array}
\end{equation}

We introduce a shorthand notation $\Sigma_S^i = \sum_{j=1}^{n_i}S_{i,j}$ to represent the sums, cross-product sums, and sums of squares of $S$ and $V$ for direction $i$. The unweighted contribution of the associated flow vectors is then contained in $n_i$ and the sums $\Sigma_S^i$, $\Sigma_{S^2}^i$, $\Sigma_{V}^i$, and $\Sigma_{SV}^i$. These values are further referred to as the \emph{flow field statistics}. Hence, a newly detected flow vector is included in the flow field estimate by incrementing these quantities according to the values $S$ and $V$ of the new vector. 

What makes this decomposition interesting, is that the flow field statistics form a compact summary of the flow field, independent of the actual number of flow vectors. Thus, flow field information from a previous iteration can be efficiently included in subsequent ones, without increasing the size of the system in \cref{eq:dir_flow_field_system}. Now, at the start of each iteration, it is possible to include information from the flow field of the previous iteration, simply by preserving a fraction $F$ of the previous flow field statistics. Hence, the estimator accuracy is less dependent on the sampling rate of the algorithm.

The preservation process is illustrated using the statistic $\Sigma_{V}^i$. At the start of iteration $k$, $\Sigma_{V}^i$ is initialized as $\Sigma_{V}^i(k) = F \Sigma_{V}^i(k-1)$. During iteration $k$, $\Sigma_{V}^i$ is then updated using newly available normal flow vectors that are allocated to direction $i$. Hence, the complete update for $\Sigma_{V}^i$ is performed as follows:
\begin{equation}
\Sigma_{V}^i(k) = F \Sigma_{V}^i(k-1) + \sum_{j=1}^{n_i}{V_{i,j}}
\end{equation}

The value of $F$ is computed as:

\begin{equation}
F = 1-\frac{t(k)-t(k-1)}{k_f}
\end{equation}

where the time constant $k_f$ is assigned a value of 0.02 s. This step is similar for all statistics. When all newly available vectors are categorized and processed, the flow field is recomputed using \cref{eq:weighted_least_squares}. 


%Instead, they are updated through an infinite impulse response low-pass filter:
%
%\begin{equation}
%\begin{array}{l}
%{{\bf{B}}_{k + 1}} = {{\bf{B}}_k} + \left( {{{\bf{A}}^T}{\bf{WA}} - {{\bf{B}}_k}} \right)\alpha \\
%{{\bf{C}}_{k + 1}} = {{\bf{C}}_k} + \left( {{{\bf{A}}^T}{\bf{Wy}} - {{\bf{C}}_k}} \right)\alpha 
%\end{array}
%\end{equation}
%
%The update constant $\alpha$ is set to 0.5. The estimate for $\mathbf{\Theta}$ is then the solution to $\mathbf{B\Theta}=\mathbf{C}$.



\subsection{Confidence Estimation and Filtering}
\label{sec:vo_confidence}
In visual sensing, the reliability of motion estimates varies greatly depending on the environment. Factors such as visible texture and scene illumination have an effect on the estimate. With event-based sensing, motion in the scene is another key factor. 

Therefore, a confidence value is computed based on several characteristics of the flow field, in order to quantify the reliability of the estimate. This confidence value is defined as a product of three individual confidence metrics based on the following statistical quantities: 

\begin{itemize}
	\item The flow estimation rate $\rho_F$.
	\item The maximal variance $\mathrm{Var}\lbrace S\rbrace$ of all flow directions.
	\item The coefficient of determination $R^2$ of the solution to \cref{eq:weighted_least_squares}, applied here as a nondimensional measure of the fit quality.
\end{itemize}

$R^2$ is generally computed through the following \cite{Weisberg2005}:

\begin{equation}
\label{eq:R2}
{R^2} = 1 - \frac{\mathit{RSS}}{\mathit{TSS}}
\end{equation}

In this work, the Residual Sum of Squares (RSS) and Total Sum of Squares (TSS) are computed in weighted form as follows:

\begin{equation}
\begin{aligned}
\mathit{RSS} &=& {{\bf{y}}^T}{\bf{Wy}} - {{\bf{\Theta }}^T}{{\bf{A}}^T}{\bf{Wy}}\\
\mathit{TSS} &=& {{\bf{y}}^T}{\bf{Wy}} - \frac{\left({\sum\limits_{i = 1}^m {W\Sigma _V^i} }\right)^2}{{\sum\limits_{i = 1}^m {W{n_i}} }}
\end{aligned}
\end{equation}

For each indicator, a confidence value $k$ is computed ranging from 0 to 1 (higher is better), similar to the variance weight in \cref{eq:weight_variance}. The individual confidence values are thus dependent on settings for $R^2_{min}$, $\mathrm{Var}\lbrace S\rbrace_{min}$, and $\rho_{F_{min}}$ (not to be confused with $\rho_{F_{max}}$). The values of $R^2_{min}$ and $\rho_{F_{min}}$ are set to 1.0 and 500 respectively. Note that, since \cref{eq:weight_variance} already provides individual confidence values per direction in the form of $W$, we simply let $k_{\mathrm{Var}\lbrace S\rbrace}=\mathrm{max}\left({W_i:i=1,\ldots,m}\right)$.

The total confidence value $K$ is then the product of $k_{\rho_F}$, $k_{\mathrm{Var}\lbrace S\rbrace}$, and $k_{R^2}$. Hence, each individual confidence factor needs to be close to 1 in order to obtain a high $K$. For example, when $\rho_F$ and $R^2$ are very large, but the flow is very localized (the maximal value for $\mathrm{Var}\lbrace S\rbrace$ is small), the estimate is still not reliable. In this case, it is likely that a single visual feature causes the normal flow, which is insufficient for computing the visual observables.

The confidence $K$ is useful to monitor the estimate quality of the visual observables during flight. In addition, it is the main component of a \emph{confidence filter} for $\mathbf{\Theta}$. This filter is based on a conventional infinite impulse response low-pass filter, in which $K$ is multiplied with the filter's update constant. The final estimate for the visual observables $\hat{\mathbf{\Theta}}$ is determined through the following update equation at iteration $k$:

\begin{equation}
\mathbf{\hat\Theta} (k)=\mathbf{\hat\Theta}(k-1) + \left(\mathbf{\Theta}(k) - \mathbf{\hat\Theta}(k-1)\right)K \frac{t(k)- t(k-1)}{k_t}
\end{equation}

where $k_t$ is the time constant of the low-pass filter, which is set to 0.02 s. Lastly, a saturation limit is applied that caps the magnitude of the update of each individual value in $\mathbf{\Theta}$ to $\Delta \vartheta_{max}$ in order to reject significant outliers. The value for $\Delta \vartheta_{max}$ is set to 0.3.

\subsection{Results}
\label{sec:results_scaled_velocity}
For evaluating the accuracy of the presented visual observable estimator, we use the measurements generated for evaluating optical flow performance in \cref{sec:results_optical_flow}, which are generated through handheld motion. Optitrack position measurements provide the ground truth estimates for $\vartheta_x$, $\vartheta_y$, and $\vartheta_z$. For each set, normal flow estimates are computed using the C-based implementation discussed in \cref{sec:results_optical_flow}. The flow detection rate cap $\rho_{F_{max}}$ is set to 2500 flow vectors per second and the periodic estimator samples the visual observables at 100 Hz, similar to the on-board implementation in \cref{sec:results}.

In our experiments the main variable of interest is $\vartheta_z$, as it forms the basis for the constant divergence controller. Therefore, this variable is investigated over a wide range of velocities. However, the estimates of $\vartheta_x$ and $\vartheta_y$ are also interesting to assess, since a more elaborate optical flow based controller may also include the horizontal components for hover stabilization. The latter process does require the MAV to perform rolling and pitching motion, inducing rotational normal flow. Therefore, the effectiveness of derotation is evaluated as well.

\subsubsection{Vertical Motion}
For assessment of $\vartheta_z$ estimates, vertical oscillating motion was performed above both texture types. The vertical speed of these oscillations was gradually increased, hence covering a wide range of divergence values. This enables a first-order characterization of the estimator behavior.

\cref{fig:divergence_estimates} shows the resulting estimates compared to ground truth measurements, accompanied by height measurements $h=-Z_{\cal W}$. Detail sections are shown for low and high divergence motion, for which also the confidence value is shown. 

\begin{figure*}[!ht]
	\centering
	\setlength{\fwidth}{0.4\linewidth}
	\begin{framed}
		\subfloat[Checkerboard texture]	{
			% This file was created by matlab2tikz.
%
%The latest updates can be retrieved from
%  http://www.mathworks.com/matlabcentral/fileexchange/22022-matlab2tikz-matlab2tikz
%where you can also make suggestions and rate matlab2tikz.
%
\definecolor{mycolor1}{rgb}{0.00000,0.44700,0.74100}%
\definecolor{mycolor2}{rgb}{0.85000,0.32500,0.09800}%
%
\begin{tikzpicture}

\begin{axis}[%
width=0.951\fwidth,
height=0.189\fwidth,
at={(0\fwidth,1.011\fwidth)},
scale only axis,
xmin=30.0046,
xmax=69.9917,
xlabel={$t$ [s]},
ymin=0.0000,
ymax=1.3680,
ylabel={$h$ [m]},
axis background/.style={fill=white},
title style={font=\labelsize},
xlabel style={font=\labelsize,at={(axis description cs:0.5,\xlabeldist)}},
ylabel style={font=\labelsize,at={(axis description cs:\ylabeldist,0.5)}},
legend style={font=\ticksize},
ticklabel style={font=\ticksize}
]
\addplot [color=mycolor1,solid,forget plot]
  table[row sep=crcr]{%
30.0046	0.9280\\
30.0366	0.9290\\
30.0793	0.9290\\
30.1208	0.9290\\
30.1633	0.9290\\
30.1945	0.9290\\
30.2380	0.9290\\
30.2812	0.9280\\
30.3239	0.9260\\
30.3675	0.9250\\
30.4042	0.9230\\
30.4475	0.9200\\
30.4910	0.9180\\
30.5333	0.9140\\
30.5651	0.9120\\
30.6082	0.9090\\
30.6509	0.9050\\
30.6943	0.9020\\
30.7393	0.8980\\
30.7709	0.8950\\
30.8129	0.8920\\
30.8554	0.8880\\
30.8972	0.8840\\
30.9407	0.8790\\
30.9730	0.8750\\
31.0155	0.8690\\
31.0583	0.8640\\
31.1016	0.8570\\
31.1341	0.8520\\
31.1758	0.8460\\
31.2193	0.8390\\
31.2612	0.8330\\
31.3030	0.8260\\
31.3349	0.8210\\
31.3776	0.8150\\
31.4197	0.8080\\
31.4622	0.8000\\
31.4940	0.7960\\
31.5366	0.7890\\
31.5791	0.7800\\
31.6220	0.7720\\
31.6641	0.7650\\
31.6964	0.7590\\
31.7395	0.7520\\
31.7828	0.7450\\
31.8258	0.7370\\
31.8684	0.7300\\
31.9006	0.7240\\
31.9434	0.7170\\
31.9859	0.7100\\
32.0288	0.7000\\
32.0613	0.6960\\
32.1031	0.6880\\
32.1457	0.6780\\
32.1879	0.6700\\
32.2299	0.6630\\
32.2624	0.6570\\
32.3051	0.6490\\
32.3474	0.6420\\
32.3896	0.6350\\
32.4218	0.6290\\
32.4650	0.6220\\
32.5082	0.6150\\
32.5506	0.6080\\
32.5930	0.6010\\
32.6238	0.5950\\
32.6672	0.5870\\
32.7102	0.5800\\
32.7522	0.5710\\
32.7944	0.5630\\
32.8270	0.5570\\
32.8699	0.5480\\
32.9122	0.5390\\
32.9552	0.5320\\
32.9877	0.5270\\
33.0298	0.5210\\
33.0719	0.5160\\
33.1151	0.5110\\
33.1580	0.5070\\
33.1894	0.5040\\
33.2328	0.5020\\
33.2757	0.5000\\
33.3181	0.4980\\
33.3498	0.4950\\
33.3925	0.4930\\
33.4339	0.4900\\
33.4769	0.4870\\
33.5193	0.4840\\
33.5515	0.4810\\
33.5944	0.4780\\
33.6373	0.4760\\
33.6808	0.4740\\
33.7228	0.4740\\
33.7554	0.4740\\
33.7977	0.4760\\
33.8394	0.4790\\
33.8816	0.4830\\
33.9135	0.4870\\
33.9562	0.4920\\
33.9984	0.4980\\
34.0403	0.5040\\
34.0839	0.5100\\
34.1159	0.5150\\
34.1580	0.5210\\
34.2007	0.5270\\
34.2434	0.5340\\
34.2747	0.5390\\
34.3182	0.5440\\
34.3613	0.5510\\
34.4028	0.5560\\
34.4442	0.5620\\
34.4767	0.5660\\
34.5192	0.5720\\
34.5620	0.5790\\
34.6046	0.5840\\
34.6463	0.5900\\
34.6775	0.5930\\
34.7210	0.5990\\
34.7630	0.6050\\
34.8054	0.6090\\
34.8385	0.6130\\
34.8812	0.6180\\
34.9237	0.6230\\
34.9654	0.6270\\
35.0085	0.6310\\
35.0404	0.6340\\
35.0820	0.6370\\
35.1237	0.6400\\
35.1672	0.6430\\
35.1995	0.6460\\
35.2436	0.6490\\
35.2858	0.6520\\
35.3290	0.6570\\
35.3715	0.6600\\
35.4025	0.6630\\
35.4455	0.6680\\
35.4883	0.6720\\
35.5309	0.6760\\
35.5728	0.6810\\
35.6049	0.6840\\
35.6466	0.6890\\
35.6883	0.6940\\
35.7297	0.6980\\
35.7611	0.7010\\
35.8036	0.7060\\
35.8464	0.7110\\
35.8894	0.7140\\
35.9319	0.7190\\
35.9631	0.7220\\
36.0050	0.7260\\
36.0474	0.7300\\
36.0894	0.7340\\
36.1205	0.7370\\
36.1632	0.7410\\
36.2049	0.7440\\
36.2465	0.7480\\
36.2882	0.7520\\
36.3196	0.7540\\
36.3609	0.7580\\
36.4027	0.7610\\
36.4454	0.7660\\
36.4872	0.7700\\
36.5185	0.7730\\
36.5609	0.7790\\
36.6025	0.7830\\
36.6442	0.7880\\
36.6755	0.7930\\
36.7170	0.7970\\
36.7598	0.8020\\
36.8016	0.8070\\
36.8442	0.8110\\
36.8756	0.8150\\
36.9171	0.8200\\
36.9598	0.8240\\
37.0016	0.8290\\
37.0330	0.8320\\
37.0765	0.8360\\
37.1183	0.8390\\
37.1601	0.8420\\
37.2027	0.8450\\
37.2348	0.8470\\
37.2776	0.8500\\
37.3196	0.8530\\
37.3631	0.8560\\
37.4048	0.8580\\
37.4360	0.8600\\
37.4787	0.8640\\
37.5204	0.8670\\
37.5621	0.8700\\
37.5934	0.8720\\
37.6357	0.8750\\
37.6788	0.8790\\
37.7205	0.8820\\
37.7631	0.8850\\
37.7942	0.8870\\
37.8372	0.8890\\
37.8789	0.8910\\
37.9215	0.8930\\
37.9527	0.8940\\
37.9941	0.8960\\
38.0358	0.8980\\
38.0777	0.9010\\
38.1204	0.9040\\
38.1514	0.9060\\
38.1932	0.9080\\
38.2349	0.9110\\
38.2767	0.9130\\
38.3190	0.9140\\
38.3507	0.9160\\
38.3930	0.9170\\
38.4359	0.9190\\
38.4774	0.9210\\
38.5098	0.9230\\
38.5514	0.9250\\
38.5931	0.9290\\
38.6351	0.9320\\
38.6774	0.9350\\
38.7087	0.9380\\
38.7505	0.9410\\
38.7922	0.9440\\
38.8337	0.9470\\
38.8651	0.9490\\
38.9068	0.9520\\
38.9484	0.9550\\
38.9909	0.9580\\
39.0327	0.9610\\
39.0641	0.9640\\
39.1066	0.9670\\
39.1485	0.9700\\
39.1910	0.9740\\
39.2328	0.9770\\
39.2642	0.9800\\
39.3066	0.9840\\
39.3483	0.9870\\
39.3902	0.9890\\
39.4225	0.9920\\
39.4651	0.9940\\
39.5076	0.9970\\
39.5505	1.0000\\
39.5930	1.0030\\
39.6242	1.0060\\
39.6659	1.0090\\
39.7078	1.0120\\
39.7493	1.0150\\
39.7817	1.0180\\
39.8235	1.0210\\
39.8660	1.0250\\
39.9077	1.0280\\
39.9492	1.0320\\
39.9809	1.0350\\
40.0232	1.0380\\
40.0650	1.0420\\
40.1065	1.0450\\
40.1482	1.0490\\
40.1797	1.0530\\
40.2224	1.0560\\
40.2649	1.0590\\
40.3066	1.0600\\
40.3378	1.0610\\
40.3796	1.0610\\
40.4212	1.0600\\
40.4630	1.0580\\
40.5050	1.0540\\
40.5368	1.0510\\
40.5785	1.0460\\
40.6200	1.0410\\
40.6620	1.0360\\
40.7035	1.0300\\
40.7347	1.0260\\
40.7772	1.0200\\
40.8189	1.0140\\
40.8608	1.0090\\
40.8926	1.0050\\
40.9346	0.9990\\
40.9762	0.9940\\
41.0189	0.9890\\
41.0617	0.9840\\
41.0931	0.9820\\
41.1345	0.9770\\
41.1763	0.9740\\
41.2188	0.9700\\
41.2508	0.9670\\
41.2929	0.9630\\
41.3355	0.9580\\
41.3772	0.9540\\
41.4202	0.9490\\
41.4517	0.9470\\
41.4929	0.9430\\
41.5346	0.9390\\
41.5764	0.9360\\
41.6176	0.9320\\
41.6490	0.9300\\
41.6920	0.9270\\
41.7345	0.9240\\
41.7771	0.9210\\
41.8095	0.9180\\
41.8523	0.9160\\
41.8939	0.9130\\
41.9357	0.9100\\
41.9785	0.9060\\
42.0094	0.9040\\
42.0512	0.9010\\
42.0928	0.8970\\
42.1354	0.8930\\
42.1672	0.8900\\
42.2085	0.8850\\
42.2502	0.8810\\
42.2928	0.8760\\
42.3346	0.8710\\
42.3659	0.8660\\
42.4075	0.8610\\
42.4492	0.8550\\
42.4918	0.8490\\
42.5336	0.8440\\
42.5647	0.8400\\
42.6064	0.8350\\
42.6482	0.8310\\
42.6900	0.8270\\
42.7220	0.8230\\
42.7637	0.8190\\
42.8057	0.8140\\
42.8481	0.8090\\
42.8905	0.8040\\
42.9220	0.8010\\
42.9637	0.7960\\
43.0055	0.7910\\
43.0470	0.7860\\
43.0784	0.7820\\
43.1199	0.7770\\
43.1620	0.7720\\
43.2034	0.7680\\
43.2451	0.7630\\
43.2770	0.7590\\
43.3187	0.7530\\
43.3604	0.7470\\
43.4024	0.7390\\
43.4439	0.7340\\
43.4758	0.7290\\
43.5177	0.7230\\
43.5605	0.7170\\
43.6022	0.7110\\
43.6336	0.7060\\
43.6757	0.7010\\
43.7187	0.6970\\
43.7604	0.6930\\
43.8021	0.6890\\
43.8345	0.6850\\
43.8770	0.6820\\
43.9187	0.6780\\
43.9605	0.6740\\
43.9928	0.6710\\
44.0352	0.6670\\
44.0774	0.6630\\
44.1186	0.6590\\
44.1603	0.6550\\
44.1927	0.6520\\
44.2345	0.6470\\
44.2760	0.6430\\
44.3176	0.6380\\
44.3595	0.6330\\
44.3916	0.6300\\
44.4344	0.6250\\
44.4764	0.6200\\
44.5186	0.6150\\
44.5501	0.6100\\
44.5926	0.6050\\
44.6344	0.5990\\
44.6771	0.5940\\
44.7199	0.5860\\
44.7521	0.5820\\
44.7937	0.5760\\
44.8360	0.5690\\
44.8783	0.5630\\
44.9094	0.5600\\
44.9509	0.5540\\
44.9926	0.5490\\
45.0343	0.5440\\
45.0768	0.5390\\
45.1084	0.5360\\
45.1499	0.5320\\
45.1925	0.5280\\
45.2352	0.5250\\
45.2770	0.5210\\
45.3093	0.5180\\
45.3512	0.5150\\
45.3927	0.5100\\
45.4353	0.5060\\
45.4674	0.5030\\
45.5101	0.4990\\
45.5520	0.4960\\
45.5935	0.4930\\
45.6367	0.4890\\
45.6677	0.4880\\
45.7100	0.4870\\
45.7520	0.4870\\
45.7948	0.4880\\
45.8262	0.4900\\
45.8677	0.4930\\
45.9092	0.4960\\
45.9511	0.5010\\
45.9935	0.5050\\
46.0257	0.5090\\
46.0675	0.5140\\
46.1102	0.5180\\
46.1520	0.5230\\
46.1949	0.5290\\
46.2259	0.5320\\
46.2675	0.5360\\
46.3093	0.5410\\
46.3520	0.5470\\
46.3837	0.5510\\
46.4248	0.5560\\
46.4676	0.5620\\
46.5093	0.5680\\
46.5509	0.5730\\
46.5822	0.5780\\
46.6247	0.5840\\
46.6665	0.5900\\
46.7091	0.5970\\
46.7415	0.6020\\
46.7833	0.6080\\
46.8250	0.6140\\
46.8678	0.6210\\
46.9101	0.6270\\
46.9417	0.6330\\
46.9832	0.6390\\
47.0257	0.6460\\
47.0676	0.6520\\
47.1100	0.6580\\
47.1415	0.6630\\
47.1842	0.6690\\
47.2269	0.6740\\
47.2697	0.6810\\
47.3020	0.6850\\
47.3436	0.6900\\
47.3875	0.6960\\
47.4293	0.7020\\
47.4719	0.7070\\
47.5041	0.7110\\
47.5468	0.7160\\
47.5885	0.7210\\
47.6301	0.7260\\
47.6624	0.7290\\
47.7040	0.7330\\
47.7458	0.7370\\
47.7873	0.7410\\
47.8292	0.7450\\
47.8615	0.7490\\
47.9041	0.7530\\
47.9458	0.7570\\
47.9872	0.7620\\
48.0292	0.7670\\
48.0601	0.7700\\
48.1031	0.7770\\
48.1448	0.7820\\
48.1864	0.7880\\
48.2173	0.7920\\
48.2593	0.7980\\
48.3009	0.8030\\
48.3434	0.8100\\
48.3852	0.8170\\
48.4176	0.8210\\
48.4590	0.8270\\
48.5018	0.8330\\
48.5447	0.8400\\
48.5764	0.8440\\
48.6194	0.8510\\
48.6624	0.8570\\
48.7040	0.8630\\
48.7458	0.8690\\
48.7780	0.8730\\
48.8206	0.8790\\
48.8624	0.8840\\
48.9042	0.8900\\
48.9459	0.8940\\
48.9781	0.8990\\
49.0207	0.9040\\
49.0625	0.9090\\
49.1055	0.9150\\
49.1370	0.9190\\
49.1800	0.9240\\
49.2224	0.9290\\
49.2643	0.9340\\
49.3061	0.9390\\
49.3371	0.9440\\
49.3790	0.9500\\
49.4206	0.9560\\
49.4622	0.9620\\
49.4934	0.9650\\
49.5361	0.9730\\
49.5790	0.9790\\
49.6207	0.9850\\
49.6622	0.9910\\
49.6935	0.9940\\
49.7350	1.0000\\
49.7768	1.0050\\
49.8184	1.0100\\
49.8603	1.0160\\
49.8914	1.0200\\
49.9329	1.0250\\
49.9746	1.0310\\
50.0153	1.0360\\
50.0468	1.0400\\
50.0888	1.0450\\
50.1309	1.0500\\
50.1737	1.0540\\
50.2162	1.0580\\
50.2487	1.0600\\
50.2910	1.0630\\
50.3340	1.0660\\
50.3766	1.0680\\
50.4193	1.0710\\
50.4514	1.0720\\
50.4934	1.0730\\
50.5361	1.0740\\
50.5788	1.0740\\
50.6099	1.0740\\
50.6537	1.0730\\
50.6956	1.0700\\
50.7371	1.0660\\
50.7800	1.0620\\
50.8124	1.0580\\
50.8551	1.0530\\
50.8985	1.0470\\
50.9408	1.0410\\
50.9725	1.0360\\
51.0143	1.0290\\
51.0560	1.0220\\
51.0976	1.0140\\
51.1390	1.0060\\
51.1704	1.0000\\
51.2133	0.9920\\
51.2548	0.9850\\
51.2976	0.9780\\
51.3391	0.9700\\
51.3705	0.9650\\
51.4120	0.9580\\
51.4543	0.9510\\
51.4966	0.9440\\
51.5279	0.9370\\
51.5703	0.9300\\
51.6121	0.9220\\
51.6533	0.9140\\
51.6965	0.9050\\
51.7278	0.8990\\
51.7706	0.8900\\
51.8131	0.8820\\
51.8556	0.8740\\
51.8870	0.8670\\
51.9288	0.8590\\
51.9699	0.8510\\
52.0141	0.8430\\
52.0563	0.8340\\
52.0888	0.8280\\
52.1307	0.8200\\
52.1734	0.8110\\
52.2162	0.8030\\
52.2582	0.7950\\
52.2892	0.7880\\
52.3320	0.7790\\
52.3741	0.7700\\
52.4170	0.7610\\
52.4486	0.7540\\
52.4918	0.7450\\
52.5337	0.7350\\
52.5754	0.7250\\
52.6172	0.7150\\
52.6494	0.7070\\
52.6919	0.6960\\
52.7348	0.6860\\
52.7768	0.6760\\
52.8087	0.6670\\
52.8508	0.6570\\
52.8936	0.6470\\
52.9362	0.6350\\
52.9789	0.6250\\
53.0111	0.6170\\
53.0534	0.6080\\
53.0967	0.5980\\
53.1399	0.5890\\
53.1828	0.5800\\
53.2136	0.5730\\
53.2549	0.5640\\
53.2965	0.5560\\
53.3389	0.5470\\
53.3700	0.5390\\
53.4122	0.5290\\
53.4555	0.5180\\
53.4985	0.5070\\
53.5412	0.4990\\
53.5731	0.4940\\
53.6161	0.4900\\
53.6584	0.4880\\
53.7004	0.4890\\
53.7323	0.4920\\
53.7761	0.4980\\
53.8182	0.5060\\
53.8610	0.5180\\
53.9035	0.5290\\
53.9359	0.5390\\
53.9778	0.5510\\
54.0202	0.5650\\
54.0631	0.5790\\
54.1064	0.5930\\
54.1379	0.6050\\
54.1799	0.6200\\
54.2229	0.6350\\
54.2662	0.6500\\
54.2976	0.6620\\
54.3404	0.6780\\
54.3829	0.6930\\
54.4252	0.7090\\
54.4669	0.7250\\
54.4997	0.7380\\
54.5411	0.7550\\
54.5836	0.7710\\
54.6261	0.7880\\
54.6571	0.8020\\
54.6993	0.8190\\
54.7409	0.8370\\
54.7837	0.8540\\
54.8264	0.8710\\
54.8580	0.8850\\
54.9011	0.9020\\
54.9431	0.9190\\
54.9853	0.9390\\
55.0285	0.9560\\
55.0597	0.9650\\
55.1031	0.9840\\
55.1452	0.9990\\
55.1885	1.0130\\
55.2204	1.0240\\
55.2630	1.0390\\
55.3060	1.0530\\
55.3495	1.0670\\
55.3910	1.0810\\
55.4238	1.0910\\
55.4659	1.1030\\
55.5091	1.1150\\
55.5512	1.1270\\
55.5837	1.1360\\
55.6260	1.1470\\
55.6679	1.1580\\
55.7096	1.1690\\
55.7531	1.1810\\
55.7845	1.1860\\
55.8264	1.1940\\
55.8695	1.2000\\
55.9119	1.2010\\
55.9533	1.1990\\
55.9846	1.1940\\
56.0275	1.1850\\
56.0699	1.1730\\
56.1129	1.1580\\
56.1447	1.1450\\
56.1872	1.1270\\
56.2296	1.1070\\
56.2729	1.0870\\
56.3158	1.0660\\
56.3486	1.0480\\
56.3909	1.0270\\
56.4337	1.0060\\
56.4757	0.9850\\
56.5073	0.9690\\
56.5497	0.9480\\
56.5908	0.9280\\
56.6334	0.9080\\
56.6763	0.8890\\
56.7077	0.8740\\
56.7508	0.8560\\
56.7928	0.8380\\
56.8353	0.8170\\
56.8786	0.7990\\
56.9107	0.7860\\
56.9532	0.7690\\
56.9962	0.7520\\
57.0385	0.7350\\
57.0699	0.7200\\
57.1128	0.7030\\
57.1545	0.6860\\
57.1964	0.6690\\
57.2389	0.6540\\
57.2702	0.6410\\
57.3130	0.6250\\
57.3545	0.6100\\
57.3971	0.5930\\
57.4285	0.5800\\
57.4716	0.5650\\
57.5147	0.5510\\
57.5567	0.5360\\
57.5990	0.5230\\
57.6305	0.5140\\
57.6735	0.5040\\
57.7169	0.4970\\
57.7585	0.4930\\
57.8011	0.4940\\
57.8324	0.4980\\
57.8746	0.5070\\
57.9179	0.5210\\
57.9595	0.5380\\
57.9917	0.5540\\
58.0337	0.5770\\
58.0762	0.6010\\
58.1189	0.6310\\
58.1607	0.6570\\
58.1931	0.6730\\
58.2350	0.7040\\
58.2776	0.7310\\
58.3200	0.7580\\
58.3517	0.7800\\
58.3941	0.8080\\
58.4365	0.8350\\
58.4794	0.8620\\
58.5210	0.8890\\
58.5532	0.9110\\
58.5960	0.9370\\
58.6379	0.9640\\
58.6805	0.9900\\
58.7233	1.0150\\
58.7547	1.0360\\
58.7970	1.0600\\
58.8376	1.0840\\
58.8804	1.1080\\
58.9114	1.1260\\
58.9551	1.1480\\
58.9978	1.1680\\
59.0411	1.1870\\
59.0837	1.2050\\
59.1149	1.2170\\
59.1568	1.2310\\
59.2000	1.2410\\
59.2428	1.2470\\
59.2740	1.2480\\
59.3166	1.2440\\
59.3595	1.2360\\
59.4016	1.2200\\
59.4440	1.2020\\
59.4754	1.1900\\
59.5173	1.1650\\
59.5594	1.1470\\
59.6017	1.1200\\
59.6442	1.0970\\
59.6760	1.0840\\
59.7184	1.0570\\
59.7606	1.0340\\
59.8022	1.0110\\
59.8339	0.9970\\
59.8772	0.9690\\
59.9197	0.9450\\
59.9625	0.9220\\
60.0049	0.8980\\
60.0366	0.8790\\
60.0784	0.8540\\
60.1210	0.8300\\
60.1636	0.8060\\
60.2065	0.7820\\
60.2375	0.7630\\
60.2792	0.7390\\
60.3220	0.7170\\
60.3657	0.6950\\
60.3971	0.6770\\
60.4402	0.6560\\
60.4826	0.6360\\
60.5244	0.6150\\
60.5684	0.5910\\
60.6000	0.5800\\
60.6424	0.5620\\
60.6849	0.5430\\
60.7283	0.5300\\
60.7607	0.5210\\
60.8036	0.5120\\
60.8470	0.5080\\
60.8908	0.5090\\
60.9323	0.5170\\
60.9648	0.5270\\
61.0079	0.5460\\
61.0499	0.5690\\
61.0920	0.5950\\
61.1349	0.6300\\
61.1665	0.6490\\
61.2095	0.6810\\
61.2521	0.7210\\
61.2945	0.7560\\
61.3274	0.7840\\
61.3699	0.8200\\
61.4124	0.8560\\
61.4547	0.8930\\
61.4970	0.9300\\
61.5291	0.9590\\
61.5709	0.9950\\
61.6134	1.0320\\
61.6550	1.0680\\
61.6863	1.0970\\
61.7293	1.1320\\
61.7722	1.1650\\
61.8148	1.1970\\
61.8573	1.2260\\
61.8893	1.2470\\
61.9321	1.2700\\
61.9753	1.2870\\
62.0175	1.2980\\
62.0592	1.3000\\
62.0914	1.2960\\
62.1331	1.2840\\
62.1755	1.2640\\
62.2182	1.2330\\
62.2498	1.2140\\
62.2926	1.1800\\
62.3353	1.1350\\
62.3781	1.0950\\
62.4215	1.0540\\
62.4535	1.0210\\
62.4950	0.9790\\
62.5363	0.9370\\
62.5797	0.8950\\
62.6124	0.8620\\
62.6539	0.8220\\
62.6957	0.7830\\
62.7381	0.7470\\
62.7812	0.7140\\
62.8136	0.6890\\
62.8568	0.6600\\
62.8993	0.6320\\
62.9413	0.6050\\
62.9839	0.5790\\
63.0166	0.5590\\
63.0589	0.5350\\
63.1018	0.5130\\
63.1440	0.5020\\
63.1755	0.4980\\
63.2184	0.4990\\
63.2607	0.5080\\
63.3028	0.5250\\
63.3469	0.5480\\
63.3784	0.5730\\
63.4207	0.6080\\
63.4635	0.6480\\
63.5062	0.6920\\
63.5380	0.7290\\
63.5803	0.7760\\
63.6233	0.8230\\
63.6654	0.8720\\
63.7078	0.9210\\
63.7400	0.9590\\
63.7824	1.0070\\
63.8247	1.0540\\
63.8665	1.0980\\
63.9090	1.1410\\
63.9413	1.1730\\
63.9841	1.2110\\
64.0270	1.2500\\
64.0693	1.2750\\
64.1020	1.2890\\
64.1446	1.2970\\
64.1878	1.2940\\
64.2309	1.2800\\
64.2730	1.2560\\
64.3051	1.2290\\
64.3477	1.1890\\
64.3891	1.1420\\
64.4322	1.0900\\
64.4645	1.0470\\
64.5075	0.9900\\
64.5505	0.9320\\
64.5924	0.8750\\
64.6349	0.8100\\
64.6663	0.7780\\
64.7080	0.7250\\
64.7501	0.6750\\
64.7925	0.6270\\
64.8363	0.5750\\
64.8677	0.5500\\
64.9103	0.5050\\
64.9536	0.4750\\
64.9951	0.4550\\
65.0272	0.4470\\
65.0695	0.4480\\
65.1122	0.4620\\
65.1537	0.4870\\
65.1957	0.5230\\
65.2281	0.5610\\
65.2697	0.6170\\
65.3115	0.6810\\
65.3541	0.7510\\
65.3848	0.8090\\
65.4280	0.8820\\
65.4712	0.9540\\
65.5132	1.0220\\
65.5558	1.0870\\
65.5877	1.1350\\
65.6299	1.1900\\
65.6714	1.2370\\
65.7136	1.2760\\
65.7566	1.3040\\
65.7889	1.3170\\
65.8318	1.3220\\
65.8741	1.3130\\
65.9172	1.2910\\
65.9486	1.2630\\
65.9910	1.2190\\
66.0339	1.1650\\
66.0771	1.0900\\
66.1195	1.0220\\
66.1509	0.9810\\
66.1936	0.8970\\
66.2362	0.8280\\
66.2781	0.7610\\
66.3103	0.7100\\
66.3532	0.6510\\
66.3956	0.6000\\
66.4378	0.5570\\
66.4801	0.5240\\
66.5122	0.5060\\
66.5556	0.4960\\
66.5973	0.5020\\
66.6408	0.5270\\
66.6838	0.5650\\
66.7155	0.6060\\
66.7577	0.6660\\
66.8006	0.7330\\
66.8437	0.8180\\
66.8756	0.8630\\
66.9172	0.9360\\
66.9601	1.0220\\
67.0028	1.0920\\
67.0454	1.1570\\
67.0770	1.2040\\
67.1195	1.2560\\
67.1612	1.2990\\
67.2037	1.3320\\
67.2348	1.3510\\
67.2769	1.3650\\
67.3182	1.3680\\
67.3599	1.3600\\
67.4015	1.3430\\
67.4338	1.3290\\
67.4755	1.3000\\
67.5183	1.2600\\
67.5613	1.2230\\
67.6045	1.1840\\
67.6359	1.1530\\
67.6778	1.1140\\
67.7200	1.0760\\
67.7620	1.0400\\
67.7933	1.0130\\
67.8361	0.9810\\
67.8775	0.9510\\
67.9195	0.9230\\
67.9620	0.8960\\
67.9933	0.8750\\
68.0358	0.8510\\
68.0787	0.8270\\
68.1207	0.8040\\
68.1527	0.7870\\
68.1944	0.7660\\
68.2361	0.7450\\
68.2776	0.7250\\
68.3192	0.7040\\
68.3504	0.6910\\
68.3919	0.6720\\
68.4337	0.6530\\
68.4754	0.6370\\
68.5187	0.6180\\
68.5504	0.6100\\
68.5923	0.5980\\
68.6340	0.5870\\
68.6755	0.5770\\
68.7073	0.5720\\
68.7506	0.5660\\
68.7922	0.5650\\
68.8338	0.5670\\
68.8770	0.5740\\
68.9088	0.5790\\
68.9516	0.5910\\
68.9944	0.6040\\
69.0362	0.6190\\
69.0670	0.6280\\
69.1086	0.6440\\
69.1516	0.6660\\
69.1934	0.6840\\
69.2351	0.7020\\
69.2661	0.7130\\
69.3085	0.7310\\
69.3505	0.7500\\
69.3921	0.7690\\
69.4348	0.7910\\
69.4671	0.8030\\
69.5089	0.8200\\
69.5517	0.8410\\
69.5934	0.8580\\
69.6246	0.8670\\
69.6661	0.8810\\
69.7075	0.8930\\
69.7494	0.9040\\
69.7919	0.9130\\
69.8232	0.9180\\
69.8649	0.9210\\
69.9068	0.9200\\
69.9492	0.9170\\
69.9917	0.9110\\
};
\end{axis}

\begin{axis}[%
width=0.951\fwidth,
height=0.189\fwidth,
at={(0\fwidth,0.688\fwidth)},
scale only axis,
xmin=30.0046,
xmax=69.9995,
xlabel={$t$ [s]},
ymin=-2.9219,
ymax=2.7500,
ylabel={$\vartheta_z$, $\hat{\vartheta}_z$ [1/s]},
axis background/.style={fill=white},
title style={font=\labelsize},
xlabel style={font=\labelsize,at={(axis description cs:0.5,\xlabeldist)}},
ylabel style={font=\labelsize,at={(axis description cs:\ylabeldist,0.5)}},
legend style={font=\ticksize},
ticklabel style={font=\ticksize}
]
\addplot [color=mycolor1,solid,forget plot]
  table[row sep=crcr]{%
30.0046	-0.0165\\
30.0366	-0.0142\\
30.0793	-0.0097\\
30.1208	-0.0061\\
30.1633	0.0000\\
30.1945	0.0054\\
30.2380	0.0145\\
30.2812	0.0299\\
30.3239	0.0472\\
30.3675	0.0536\\
30.4042	0.0786\\
30.4475	0.0868\\
30.4910	0.0932\\
30.5333	0.1017\\
30.5651	0.1014\\
30.6082	0.1089\\
30.6509	0.1205\\
30.6943	0.1135\\
30.7393	0.1168\\
30.7709	0.1169\\
30.8129	0.1147\\
30.8554	0.1230\\
30.8972	0.1246\\
30.9407	0.1660\\
30.9730	0.1818\\
31.0155	0.2007\\
31.0583	0.2117\\
31.1016	0.2280\\
31.1341	0.2210\\
31.1758	0.2221\\
31.2193	0.2423\\
31.2612	0.2297\\
31.3030	0.2317\\
31.3349	0.2387\\
31.3776	0.2375\\
31.4197	0.2384\\
31.4622	0.2675\\
31.4940	0.2785\\
31.5366	0.2732\\
31.5791	0.2911\\
31.6220	0.2913\\
31.6641	0.2926\\
31.6964	0.3031\\
31.7395	0.2960\\
31.7828	0.3031\\
31.8258	0.3119\\
31.8684	0.3023\\
31.9006	0.3059\\
31.9434	0.3168\\
31.9859	0.3277\\
32.0288	0.3672\\
32.0613	0.3551\\
32.1031	0.3512\\
32.1457	0.3733\\
32.1879	0.3569\\
32.2299	0.3657\\
32.2624	0.3734\\
32.3051	0.3568\\
32.3474	0.3632\\
32.3896	0.3706\\
32.4218	0.3521\\
32.4650	0.3618\\
32.5082	0.3787\\
32.5506	0.3721\\
32.5930	0.3937\\
32.6238	0.4134\\
32.6672	0.4083\\
32.7102	0.4543\\
32.7522	0.4793\\
32.7944	0.4739\\
32.8270	0.5045\\
32.8699	0.4898\\
32.9122	0.4958\\
32.9552	0.4345\\
32.9877	0.3980\\
33.0298	0.3740\\
33.0719	0.3424\\
33.1151	0.3069\\
33.1580	0.2522\\
33.1894	0.2019\\
33.2328	0.1587\\
33.2757	0.1551\\
33.3181	0.1538\\
33.3498	0.1643\\
33.3925	0.1793\\
33.4339	0.1852\\
33.4769	0.2171\\
33.5193	0.2535\\
33.5515	0.2416\\
33.5944	0.1847\\
33.6373	0.1496\\
33.6808	0.1087\\
33.7228	0.0474\\
33.7554	-0.0145\\
33.7977	-0.1297\\
33.8394	-0.2150\\
33.8816	-0.2814\\
33.9135	-0.3410\\
33.9562	-0.3888\\
33.9984	-0.3751\\
34.0403	-0.4041\\
34.0839	-0.4095\\
34.1159	-0.4012\\
34.1580	-0.3936\\
34.2007	-0.3894\\
34.2434	-0.4008\\
34.2747	-0.3858\\
34.3182	-0.3545\\
34.3613	-0.3545\\
34.4028	-0.3467\\
34.4442	-0.3420\\
34.4767	-0.3232\\
34.5192	-0.3135\\
34.5620	-0.3158\\
34.6046	-0.3129\\
34.6463	-0.2930\\
34.6775	-0.2845\\
34.7210	-0.2979\\
34.7630	-0.2806\\
34.8054	-0.2535\\
34.8385	-0.2681\\
34.8812	-0.2624\\
34.9237	-0.2243\\
34.9654	-0.2021\\
35.0085	-0.2004\\
35.0404	-0.1882\\
35.0820	-0.1613\\
35.1237	-0.1481\\
35.1672	-0.1605\\
35.1995	-0.1672\\
35.2436	-0.1669\\
35.2858	-0.1746\\
35.3290	-0.1691\\
35.3715	-0.1700\\
35.4025	-0.1893\\
35.4455	-0.2015\\
35.4883	-0.2010\\
35.5309	-0.2106\\
35.5728	-0.2048\\
35.6049	-0.1914\\
35.6466	-0.2162\\
35.6883	-0.2055\\
35.7297	-0.1933\\
35.7611	-0.1867\\
35.8036	-0.1906\\
35.8464	-0.1852\\
35.8894	-0.1745\\
35.9319	-0.1787\\
35.9631	-0.1738\\
36.0050	-0.1646\\
36.0474	-0.1619\\
36.0894	-0.1556\\
36.1205	-0.1614\\
36.1632	-0.1513\\
36.2049	-0.1339\\
36.2465	-0.1576\\
36.2882	-0.1568\\
36.3196	-0.1496\\
36.3609	-0.1456\\
36.4027	-0.1515\\
36.4454	-0.1655\\
36.4872	-0.1664\\
36.5185	-0.1873\\
36.5609	-0.1912\\
36.6025	-0.1860\\
36.6442	-0.1876\\
36.6755	-0.1858\\
36.7170	-0.1778\\
36.7598	-0.1793\\
36.8016	-0.1730\\
36.8442	-0.1710\\
36.8756	-0.1718\\
36.9171	-0.1717\\
36.9598	-0.1643\\
37.0016	-0.1553\\
37.0330	-0.1449\\
37.0765	-0.1337\\
37.1183	-0.1162\\
37.1601	-0.1020\\
37.2027	-0.0962\\
37.2348	-0.0983\\
37.2776	-0.0897\\
37.3196	-0.0907\\
37.3631	-0.0922\\
37.4048	-0.0988\\
37.4360	-0.1015\\
37.4787	-0.1066\\
37.5204	-0.1112\\
37.5621	-0.1049\\
37.5934	-0.1092\\
37.6357	-0.1067\\
37.6788	-0.1031\\
37.7205	-0.0985\\
37.7631	-0.0969\\
37.7942	-0.0896\\
37.8372	-0.0750\\
37.8789	-0.0657\\
37.9215	-0.0607\\
37.9527	-0.0604\\
37.9941	-0.0619\\
38.0358	-0.0692\\
38.0777	-0.0792\\
38.1204	-0.0869\\
38.1514	-0.0880\\
38.1932	-0.0799\\
38.2349	-0.0747\\
38.2767	-0.0644\\
38.3190	-0.0514\\
38.3507	-0.0444\\
38.3930	-0.0469\\
38.4359	-0.0555\\
38.4774	-0.0665\\
38.5098	-0.0810\\
38.5514	-0.0919\\
38.5931	-0.0981\\
38.6351	-0.1042\\
38.6774	-0.0999\\
38.7087	-0.0990\\
38.7505	-0.0964\\
38.7922	-0.0931\\
38.8337	-0.0933\\
38.8651	-0.0895\\
38.9068	-0.0867\\
38.9484	-0.0901\\
38.9909	-0.0905\\
39.0327	-0.0944\\
39.0641	-0.0929\\
39.1066	-0.0941\\
39.1485	-0.1067\\
39.1910	-0.1089\\
39.2328	-0.1025\\
39.2642	-0.1014\\
39.3066	-0.0951\\
39.3483	-0.0874\\
39.3902	-0.0844\\
39.4225	-0.0828\\
39.4651	-0.0732\\
39.5076	-0.0754\\
39.5505	-0.0909\\
39.5930	-0.0943\\
39.6242	-0.0898\\
39.6659	-0.0860\\
39.7078	-0.0881\\
39.7493	-0.0918\\
39.7817	-0.0926\\
39.8235	-0.0946\\
39.8660	-0.1043\\
39.9077	-0.1020\\
39.9492	-0.1006\\
39.9809	-0.0966\\
40.0232	-0.0934\\
40.0650	-0.0951\\
40.1065	-0.1034\\
40.1482	-0.1076\\
40.1797	-0.1044\\
40.2224	-0.0911\\
40.2649	-0.0710\\
40.3066	-0.0473\\
40.3378	-0.0266\\
40.3796	0.0050\\
40.4212	0.0316\\
40.4630	0.0586\\
40.5050	0.0857\\
40.5368	0.1099\\
40.5785	0.1316\\
40.6200	0.1459\\
40.6620	0.1529\\
40.7035	0.1600\\
40.7347	0.1640\\
40.7772	0.1641\\
40.8189	0.1629\\
40.8608	0.1583\\
40.8926	0.1622\\
40.9346	0.1535\\
40.9762	0.1433\\
41.0189	0.1358\\
41.0617	0.1338\\
41.0931	0.1285\\
41.1345	0.1209\\
41.1763	0.1130\\
41.2188	0.1115\\
41.2508	0.1123\\
41.2929	0.1203\\
41.3355	0.1303\\
41.3772	0.1309\\
41.4202	0.1324\\
41.4517	0.1217\\
41.4929	0.1165\\
41.5346	0.1073\\
41.5764	0.1112\\
41.6176	0.1035\\
41.6490	0.1001\\
41.6920	0.1005\\
41.7345	0.0964\\
41.7771	0.0911\\
41.8095	0.0914\\
41.8523	0.0891\\
41.8939	0.0937\\
41.9357	0.0953\\
41.9785	0.1031\\
42.0094	0.1091\\
42.0512	0.1137\\
42.0928	0.1226\\
42.1354	0.1266\\
42.1672	0.1326\\
42.2085	0.1454\\
42.2502	0.1483\\
42.2928	0.1652\\
42.3346	0.1820\\
42.3659	0.1867\\
42.4075	0.1895\\
42.4492	0.1975\\
42.4918	0.2016\\
42.5336	0.1856\\
42.5647	0.1720\\
42.6064	0.1699\\
42.6482	0.1546\\
42.6900	0.1505\\
42.7220	0.1592\\
42.7637	0.1674\\
42.8057	0.1623\\
42.8481	0.1719\\
42.8905	0.1845\\
42.9220	0.1828\\
42.9637	0.1781\\
43.0055	0.1862\\
43.0470	0.1872\\
43.0784	0.1926\\
43.1199	0.1875\\
43.1620	0.1903\\
43.2034	0.1856\\
43.2451	0.2026\\
43.2770	0.2225\\
43.3187	0.2364\\
43.3604	0.2565\\
43.4024	0.2671\\
43.4439	0.2336\\
43.4758	0.2711\\
43.5177	0.2882\\
43.5605	0.2711\\
43.6022	0.2437\\
43.6336	0.2287\\
43.6757	0.2286\\
43.7187	0.2048\\
43.7604	0.1817\\
43.8021	0.1847\\
43.8345	0.1818\\
43.8770	0.1786\\
43.9187	0.1773\\
43.9605	0.1665\\
43.9928	0.1713\\
44.0352	0.1843\\
44.0774	0.1815\\
44.1186	0.1927\\
44.1603	0.1978\\
44.1927	0.1955\\
44.2345	0.2057\\
44.2760	0.2199\\
44.3176	0.2295\\
44.3595	0.2251\\
44.3916	0.2324\\
44.4344	0.2618\\
44.4764	0.2648\\
44.5186	0.2609\\
44.5501	0.2688\\
44.5926	0.3054\\
44.6344	0.3078\\
44.6771	0.3118\\
44.7199	0.3449\\
44.7521	0.3382\\
44.7937	0.3394\\
44.8360	0.3680\\
44.8783	0.3325\\
44.9094	0.3288\\
44.9509	0.3264\\
44.9926	0.2987\\
45.0343	0.2956\\
45.0768	0.2948\\
45.1084	0.2644\\
45.1499	0.2374\\
45.1925	0.2382\\
45.2352	0.2243\\
45.2770	0.2295\\
45.3093	0.2337\\
45.3512	0.2568\\
45.3927	0.2721\\
45.4353	0.2632\\
45.4674	0.2637\\
45.5101	0.2518\\
45.5520	0.2102\\
45.5935	0.1989\\
45.6367	0.1843\\
45.6677	0.1304\\
45.7100	0.0731\\
45.7520	-0.0004\\
45.7948	-0.1047\\
45.8262	-0.1524\\
45.8677	-0.2070\\
45.9092	-0.2402\\
45.9511	-0.2908\\
45.9935	-0.3105\\
46.0257	-0.3199\\
46.0675	-0.3069\\
46.1102	-0.3176\\
46.1520	-0.3082\\
46.1949	-0.2963\\
46.2259	-0.2824\\
46.2675	-0.2924\\
46.3093	-0.3074\\
46.3520	-0.3102\\
46.3837	-0.2747\\
46.4248	-0.3402\\
46.4676	-0.3466\\
46.5093	-0.3452\\
46.5509	-0.3320\\
46.5822	-0.3284\\
46.6247	-0.3212\\
46.6665	-0.3356\\
46.7091	-0.3504\\
46.7415	-0.3334\\
46.7833	-0.3267\\
46.8250	-0.3247\\
46.8678	-0.3372\\
46.9101	-0.3242\\
46.9417	-0.3094\\
46.9832	-0.3119\\
47.0257	-0.3179\\
47.0676	-0.3032\\
47.1100	-0.2828\\
47.1415	-0.2760\\
47.1842	-0.2784\\
47.2269	-0.2702\\
47.2697	-0.2633\\
47.3020	-0.2559\\
47.3436	-0.2419\\
47.3875	-0.2386\\
47.4293	-0.2362\\
47.4719	-0.2406\\
47.5041	-0.2299\\
47.5468	-0.2130\\
47.5885	-0.1970\\
47.6301	-0.1974\\
47.6624	-0.1844\\
47.7040	-0.1715\\
47.7458	-0.1549\\
47.7873	-0.1637\\
47.8292	-0.1689\\
47.8615	-0.1787\\
47.9041	-0.1772\\
47.9458	-0.1842\\
47.9872	-0.1946\\
48.0292	-0.1907\\
48.0601	-0.1996\\
48.1031	-0.2155\\
48.1448	-0.2173\\
48.1864	-0.2137\\
48.2173	-0.2185\\
48.2593	-0.2238\\
48.3009	-0.2281\\
48.3434	-0.2315\\
48.3852	-0.2230\\
48.4176	-0.2204\\
48.4590	-0.2221\\
48.5018	-0.2142\\
48.5447	-0.2192\\
48.5764	-0.2149\\
48.6194	-0.2167\\
48.6624	-0.2075\\
48.7040	-0.1980\\
48.7458	-0.1987\\
48.7780	-0.1957\\
48.8206	-0.1878\\
48.8624	-0.1717\\
48.9042	-0.1697\\
48.9459	-0.1631\\
48.9781	-0.1742\\
49.0207	-0.1703\\
49.0625	-0.1759\\
49.1055	-0.1750\\
49.1370	-0.1684\\
49.1800	-0.1610\\
49.2224	-0.1592\\
49.2643	-0.1587\\
49.3061	-0.1644\\
49.3371	-0.1679\\
49.3790	-0.1748\\
49.4206	-0.1792\\
49.4622	-0.1825\\
49.4934	-0.1831\\
49.5361	-0.1902\\
49.5790	-0.1843\\
49.6207	-0.1761\\
49.6622	-0.1698\\
49.6935	-0.1585\\
49.7350	-0.1555\\
49.7768	-0.1518\\
49.8184	-0.1522\\
49.8603	-0.1527\\
49.8914	-0.1530\\
49.9329	-0.1487\\
49.9746	-0.1436\\
50.0153	-0.1473\\
50.0468	-0.1440\\
50.0888	-0.1323\\
50.1309	-0.1289\\
50.1737	-0.1121\\
50.2162	-0.0961\\
50.2487	-0.0895\\
50.2910	-0.0812\\
50.3340	-0.0729\\
50.3766	-0.0640\\
50.4193	-0.0540\\
50.4514	-0.0474\\
50.4934	-0.0389\\
50.5361	-0.0226\\
50.5788	-0.0021\\
50.6099	0.0130\\
50.6537	0.0415\\
50.6956	0.0697\\
50.7371	0.0962\\
50.7800	0.1249\\
50.8124	0.1370\\
50.8551	0.1471\\
50.8985	0.1609\\
50.9408	0.1699\\
50.9725	0.1744\\
51.0143	0.1894\\
51.0560	0.2025\\
51.0976	0.2198\\
51.1390	0.2225\\
51.1704	0.2252\\
51.2133	0.2332\\
51.2548	0.2181\\
51.2976	0.2137\\
51.3391	0.2133\\
51.3705	0.2104\\
51.4120	0.2142\\
51.4543	0.2176\\
51.4966	0.2376\\
51.5279	0.2328\\
51.5703	0.2334\\
51.6121	0.2474\\
51.6533	0.2668\\
51.6965	0.2834\\
51.7278	0.2763\\
51.7706	0.2823\\
51.8131	0.2846\\
51.8556	0.2753\\
51.8870	0.2768\\
51.9288	0.2718\\
51.9699	0.2598\\
52.0141	0.3101\\
52.0563	0.3199\\
52.0888	0.3130\\
52.1307	0.3042\\
52.1734	0.3081\\
52.2162	0.3110\\
52.2582	0.3115\\
52.2892	0.3237\\
52.3320	0.3445\\
52.3741	0.3540\\
52.4170	0.3700\\
52.4486	0.3738\\
52.4918	0.3847\\
52.5337	0.4015\\
52.5754	0.4188\\
52.6172	0.4379\\
52.6494	0.4558\\
52.6919	0.4699\\
52.7348	0.4629\\
52.7768	0.4806\\
52.8087	0.4933\\
52.8508	0.4854\\
52.8936	0.4857\\
52.9362	0.5158\\
52.9789	0.4945\\
53.0111	0.5371\\
53.0534	0.5046\\
53.0967	0.5142\\
53.1399	0.4952\\
53.1828	0.4970\\
53.2136	0.4996\\
53.2549	0.4889\\
53.2965	0.5010\\
53.3389	0.5394\\
53.3700	0.5745\\
53.4122	0.6508\\
53.4555	0.7112\\
53.4985	0.6555\\
53.5412	0.5335\\
53.5731	0.4121\\
53.6161	0.2553\\
53.6584	0.0742\\
53.7004	-0.1100\\
53.7323	-0.2304\\
53.7761	-0.4048\\
53.8182	-0.5371\\
53.8610	-0.6640\\
53.9035	-0.6931\\
53.9359	-0.7517\\
53.9778	-0.7527\\
54.0202	-0.7593\\
54.0631	-0.7844\\
54.1064	-0.7829\\
54.1379	-0.7869\\
54.1799	-0.7418\\
54.2229	-0.7424\\
54.2662	-0.7341\\
54.2976	-0.7129\\
54.3404	-0.7221\\
54.3829	-0.6844\\
54.4252	-0.7060\\
54.4669	-0.6846\\
54.4997	-0.6552\\
54.5411	-0.6662\\
54.5836	-0.6502\\
54.6261	-0.6372\\
54.6571	-0.6333\\
54.6993	-0.6239\\
54.7409	-0.6313\\
54.7837	-0.6114\\
54.8264	-0.5886\\
54.8580	-0.5861\\
54.9011	-0.5575\\
54.9431	-0.5322\\
54.9853	-0.5312\\
55.0285	-0.4965\\
55.0597	-0.4796\\
55.1031	-0.4688\\
55.1452	-0.4341\\
55.1885	-0.4254\\
55.2204	-0.4098\\
55.2630	-0.4124\\
55.3060	-0.3951\\
55.3495	-0.3841\\
55.3910	-0.3639\\
55.4238	-0.3485\\
55.4659	-0.3154\\
55.5091	-0.3070\\
55.5512	-0.2933\\
55.5837	-0.2902\\
55.6260	-0.2822\\
55.6679	-0.2689\\
55.7096	-0.2545\\
55.7531	-0.2401\\
55.7845	-0.2144\\
55.8264	-0.1716\\
55.8695	-0.1125\\
55.9119	-0.0327\\
55.9533	0.0632\\
55.9846	0.1429\\
56.0275	0.2188\\
56.0699	0.2944\\
56.1129	0.3795\\
56.1447	0.4065\\
56.1872	0.4589\\
56.2296	0.5297\\
56.2729	0.5543\\
56.3158	0.5638\\
56.3486	0.5956\\
56.3909	0.6156\\
56.4337	0.6235\\
56.4757	0.6169\\
56.5073	0.6099\\
56.5497	0.6185\\
56.5908	0.6269\\
56.6334	0.6480\\
56.6763	0.6400\\
56.7077	0.6477\\
56.7508	0.6385\\
56.7928	0.6318\\
56.8353	0.6636\\
56.8786	0.6555\\
56.9107	0.6681\\
56.9532	0.6713\\
56.9962	0.7037\\
57.0385	0.7106\\
57.0699	0.7495\\
57.1128	0.7784\\
57.1545	0.7628\\
57.1964	0.7569\\
57.2389	0.7501\\
57.2702	0.7719\\
57.3130	0.8308\\
57.3545	0.8434\\
57.3971	0.8623\\
57.4285	0.8778\\
57.4716	0.8949\\
57.5147	0.8982\\
57.5567	0.8653\\
57.5990	0.8068\\
57.6305	0.7676\\
57.6735	0.6257\\
57.7169	0.4563\\
57.7585	0.1960\\
57.8011	-0.1111\\
57.8324	-0.3347\\
57.8746	-0.6848\\
57.9179	-0.9083\\
57.9595	-1.0779\\
57.9917	-1.2176\\
58.0337	-1.2391\\
58.0762	-1.2483\\
58.1189	-1.2970\\
58.1607	-1.2062\\
58.1931	-1.1828\\
58.2350	-1.1626\\
58.2776	-1.1044\\
58.3200	-1.0733\\
58.3517	-1.0840\\
58.3941	-1.0098\\
58.4365	-0.9712\\
58.4794	-0.9560\\
58.5210	-0.8848\\
58.5532	-0.8643\\
58.5960	-0.8578\\
58.6379	-0.7908\\
58.6805	-0.7739\\
58.7233	-0.7443\\
58.7547	-0.6944\\
58.7970	-0.6606\\
58.8376	-0.6283\\
58.8804	-0.5942\\
58.9114	-0.5762\\
58.9551	-0.5521\\
58.9978	-0.5040\\
59.0411	-0.4731\\
59.0837	-0.4073\\
59.1149	-0.3597\\
59.1568	-0.3011\\
59.2000	-0.2169\\
59.2428	-0.1094\\
59.2740	-0.0275\\
59.3166	0.1019\\
59.3595	0.2144\\
59.4016	0.3358\\
59.4440	0.4093\\
59.4754	0.4585\\
59.5173	0.5171\\
59.5594	0.4806\\
59.6017	0.5933\\
59.6442	0.6197\\
59.6760	0.6298\\
59.7184	0.6440\\
59.7606	0.6311\\
59.8022	0.6462\\
59.8339	0.6618\\
59.8772	0.7155\\
59.9197	0.7267\\
59.9625	0.7674\\
60.0049	0.7685\\
60.0366	0.7861\\
60.0784	0.8321\\
60.1210	0.8824\\
60.1636	0.9051\\
60.2065	0.9245\\
60.2375	0.9159\\
60.2792	0.9272\\
60.3220	0.9683\\
60.3657	1.0048\\
60.3971	1.0137\\
60.4402	1.0245\\
60.4826	1.0133\\
60.5244	1.0744\\
60.5684	1.1157\\
60.6000	1.0747\\
60.6424	1.0211\\
60.6849	0.9545\\
60.7283	0.8349\\
60.7607	0.7332\\
60.8036	0.5430\\
60.8470	0.2261\\
60.8908	-0.1586\\
60.9323	-0.5326\\
60.9648	-0.8327\\
61.0079	-1.2258\\
61.0499	-1.3587\\
61.0920	-1.4252\\
61.1349	-1.5211\\
61.1665	-1.4991\\
61.2095	-1.4584\\
61.2521	-1.4819\\
61.2945	-1.4115\\
61.3274	-1.4051\\
61.3699	-1.3245\\
61.4124	-1.2942\\
61.4547	-1.1947\\
61.4970	-1.1782\\
61.5291	-1.1549\\
61.5709	-1.0575\\
61.6134	-1.0066\\
61.6550	-0.9612\\
61.6863	-0.9358\\
61.7293	-0.9078\\
61.7722	-0.8325\\
61.8148	-0.7620\\
61.8573	-0.6629\\
61.8893	-0.6047\\
61.9321	-0.4960\\
61.9753	-0.3543\\
62.0175	-0.1953\\
62.0592	-0.0316\\
62.0914	0.1066\\
62.1331	0.3010\\
62.1755	0.4848\\
62.2182	0.6462\\
62.2498	0.7197\\
62.2926	0.8269\\
62.3353	0.9790\\
62.3781	1.0503\\
62.4215	1.1510\\
62.4535	1.1655\\
62.4950	1.2341\\
62.5363	1.3495\\
62.5797	1.4057\\
62.6124	1.4656\\
62.6539	1.4578\\
62.6957	1.4595\\
62.7381	1.4257\\
62.7812	1.4380\\
62.8136	1.4605\\
62.8568	1.4101\\
62.8993	1.3733\\
62.9413	1.4619\\
62.9839	1.4763\\
63.0166	1.4835\\
63.0589	1.3839\\
63.1018	1.1689\\
63.1440	0.7394\\
63.1755	0.3762\\
63.2184	-0.1523\\
63.2607	-0.6052\\
63.3028	-1.0534\\
63.3469	-1.5451\\
63.3784	-1.7351\\
63.4207	-1.8316\\
63.4635	-1.9671\\
63.5062	-1.9954\\
63.5380	-1.8947\\
63.5803	-1.8504\\
63.6233	-1.7672\\
63.6654	-1.6192\\
63.7078	-1.5811\\
63.7400	-1.4997\\
63.7824	-1.3501\\
63.8247	-1.2798\\
63.8665	-1.1567\\
63.9090	-1.0432\\
63.9413	-0.9949\\
63.9841	-0.8728\\
64.0270	-0.7318\\
64.0693	-0.5568\\
64.1020	-0.4078\\
64.1446	-0.1780\\
64.1878	0.0655\\
64.2309	0.3211\\
64.2730	0.5725\\
64.3051	0.7242\\
64.3477	0.9852\\
64.3891	1.1975\\
64.4322	1.3238\\
64.4645	1.4879\\
64.5075	1.7078\\
64.5505	1.7620\\
64.5924	1.8390\\
64.6349	2.0200\\
64.6663	2.0369\\
64.7080	2.0974\\
64.7501	2.1880\\
64.7925	2.2731\\
64.8363	2.4751\\
64.8677	2.4588\\
64.9103	2.4747\\
64.9536	2.0767\\
64.9951	1.4354\\
65.0272	0.8174\\
65.0695	-0.1436\\
65.1122	-1.1127\\
65.1537	-1.7402\\
65.1957	-2.2609\\
65.2281	-2.5346\\
65.2697	-2.7623\\
65.3115	-2.8889\\
65.3541	-2.9219\\
65.3848	-2.6735\\
65.4280	-2.4453\\
65.4712	-2.2435\\
65.5132	-1.9460\\
65.5558	-1.6963\\
65.5877	-1.4681\\
65.6299	-1.2497\\
65.6714	-1.0427\\
65.7136	-0.8271\\
65.7566	-0.5819\\
65.7889	-0.3646\\
65.8318	-0.0765\\
65.8741	0.2370\\
65.9172	0.5154\\
65.9486	0.7209\\
65.9910	1.0820\\
66.0339	1.3618\\
66.0771	1.6835\\
66.1195	1.8466\\
66.1509	1.9641\\
66.1936	2.2839\\
66.2362	2.4067\\
66.2781	2.5779\\
66.3103	2.7500\\
66.3532	2.7126\\
66.3956	2.6335\\
66.4378	2.2930\\
66.4801	1.8759\\
66.5122	1.3969\\
66.5556	0.4429\\
66.5973	-0.5990\\
66.6408	-1.5820\\
66.6838	-2.2084\\
66.7155	-2.4201\\
66.7577	-2.7025\\
66.8006	-2.6783\\
66.8437	-2.5649\\
66.8756	-2.4192\\
66.9172	-2.1892\\
66.9601	-2.0354\\
67.0028	-1.8052\\
67.0454	-1.6163\\
67.0770	-1.3960\\
67.1195	-1.1485\\
67.1612	-0.9169\\
67.2037	-0.6910\\
67.2348	-0.5118\\
67.2769	-0.2758\\
67.3182	-0.0593\\
67.3599	0.1478\\
67.4015	0.3374\\
67.4338	0.4642\\
67.4755	0.6177\\
67.5183	0.7911\\
67.5613	0.8709\\
67.6045	0.9549\\
67.6359	0.9601\\
67.6778	0.9871\\
67.7200	1.0134\\
67.7620	0.9688\\
67.7933	0.9489\\
67.8361	0.9163\\
67.8775	0.8985\\
67.9195	0.8792\\
67.9620	0.8781\\
67.9933	0.8494\\
68.0358	0.8388\\
68.0787	0.8579\\
68.1207	0.8419\\
68.1527	0.7948\\
68.1944	0.7507\\
68.2361	0.8916\\
68.2776	0.9023\\
68.3192	0.8980\\
68.3504	0.8817\\
68.3919	0.9053\\
68.4337	0.8794\\
68.4754	0.8008\\
68.5187	0.7959\\
68.5504	0.7520\\
68.5923	0.6463\\
68.6340	0.6000\\
68.6755	0.5007\\
68.7073	0.4084\\
68.7506	0.2618\\
68.7922	0.0742\\
68.8338	-0.1327\\
68.8770	-0.3257\\
68.9088	-0.4267\\
68.9516	-0.5758\\
68.9944	-0.6503\\
69.0362	-0.7200\\
69.0670	-0.7527\\
69.1086	-0.7967\\
69.1516	-0.8424\\
69.1934	-0.8126\\
69.2351	-0.7767\\
69.2661	-0.7698\\
69.3085	-0.7522\\
69.3505	-0.7406\\
69.3921	-0.7368\\
69.4348	-0.7264\\
69.4671	-0.6861\\
69.5089	-0.6456\\
69.5517	-0.6218\\
69.5934	-0.5604\\
69.6246	-0.5176\\
69.6661	-0.4628\\
69.7075	-0.4066\\
69.7494	-0.3385\\
69.7919	-0.2572\\
69.8232	-0.1854\\
69.8649	-0.0844\\
69.9068	0.0127\\
69.9492	0.1307\\
69.9917	0.2146\\
};
\addplot [color=mycolor2,solid,forget plot]
  table[row sep=crcr]{%
30.0095	0.0572\\
30.0495	0.0913\\
30.0895	0.0909\\
30.1295	0.0899\\
30.1695	0.0757\\
30.2095	0.0433\\
30.2495	0.0207\\
30.2895	0.0310\\
30.3295	0.0251\\
30.3695	0.0022\\
30.4095	0.0022\\
30.4495	-0.0145\\
30.4895	0.0409\\
30.5295	0.0769\\
30.5695	0.1269\\
30.6095	0.1300\\
30.6495	0.1805\\
30.6895	0.1849\\
30.7295	0.1585\\
30.7695	0.1426\\
30.8095	0.1307\\
30.8495	0.1095\\
30.8895	0.0945\\
30.9295	0.0924\\
30.9695	0.1460\\
31.0095	0.1875\\
31.0495	0.1998\\
31.0895	0.1866\\
31.1295	0.1967\\
31.1695	0.2337\\
31.2095	0.3073\\
31.2495	0.2594\\
31.2895	0.2529\\
31.3295	0.2197\\
31.3695	0.2493\\
31.4095	0.2543\\
31.4495	0.2094\\
31.4895	0.2196\\
31.5295	0.2573\\
31.5695	0.3298\\
31.6095	0.3160\\
31.6495	0.2539\\
31.6895	0.3110\\
31.7295	0.3462\\
31.7695	0.3321\\
31.8095	0.3290\\
31.8495	0.2656\\
31.8895	0.3363\\
31.9295	0.3673\\
31.9695	0.3116\\
32.0095	0.2955\\
32.0495	0.3188\\
32.0895	0.3526\\
32.1295	0.3149\\
32.1695	0.2676\\
32.2095	0.3032\\
32.2495	0.3845\\
32.2895	0.5063\\
32.3295	0.5328\\
32.3695	0.4502\\
32.4095	0.4036\\
32.4495	0.3647\\
32.4895	0.3569\\
32.5295	0.4360\\
32.5695	0.3633\\
32.6095	0.3085\\
32.6495	0.2962\\
32.6895	0.3357\\
32.7295	0.3673\\
32.7695	0.4840\\
32.8095	0.4419\\
32.8495	0.5021\\
32.8895	0.4404\\
32.9295	0.3792\\
32.9695	0.3763\\
33.0095	0.2732\\
33.0495	0.3444\\
33.0895	0.4051\\
33.1295	0.4345\\
33.1695	0.4043\\
33.2095	0.6751\\
33.2495	0.3722\\
33.2895	0.3183\\
33.3295	0.1980\\
33.3695	0.1247\\
33.4095	0.1235\\
33.4495	0.1401\\
33.4895	0.1357\\
33.5295	0.1853\\
33.5695	0.2205\\
33.6095	0.2563\\
33.6495	0.3441\\
33.6895	0.2720\\
33.7295	0.1551\\
33.7695	0.1151\\
33.8095	0.0883\\
33.8495	-0.0711\\
33.8895	-0.1841\\
33.9295	-0.5496\\
33.9695	-0.5897\\
34.0095	-0.8044\\
34.0495	-0.6453\\
34.0895	-0.4873\\
34.1295	-0.2016\\
34.1695	-0.3783\\
34.2095	-0.5497\\
34.2495	-0.4901\\
34.2895	-0.4000\\
34.3295	-0.3708\\
34.3695	-0.3288\\
34.4095	-0.3465\\
34.4495	-0.3779\\
34.4895	-0.3320\\
34.5295	-0.3334\\
34.5695	-0.3549\\
34.6095	-0.3277\\
34.6495	-0.3208\\
34.6895	-0.2963\\
34.7295	-0.2651\\
34.7695	-0.2371\\
34.8095	-0.2683\\
34.8495	-0.2704\\
34.8895	-0.2216\\
34.9295	-0.2170\\
34.9695	-0.2733\\
35.0095	-0.2312\\
35.0495	-0.2314\\
35.0895	-0.1926\\
35.1295	-0.2187\\
35.1695	-0.2134\\
35.2095	-0.2326\\
35.2495	-0.2204\\
35.2895	-0.2016\\
35.3295	-0.2249\\
35.3695	-0.2227\\
35.4095	-0.2160\\
35.4495	-0.1811\\
35.4895	-0.2907\\
35.5295	-0.4098\\
35.5695	-0.3028\\
35.6095	-0.3013\\
35.6495	-0.3033\\
35.6895	-0.2297\\
35.7295	-0.1109\\
35.7695	-0.1759\\
35.8095	-0.2393\\
35.8495	-0.2193\\
35.8895	-0.2424\\
35.9295	-0.1958\\
35.9695	-0.3248\\
36.0095	-0.3342\\
36.0495	-0.2035\\
36.0895	-0.2324\\
36.1295	-0.2600\\
36.1695	-0.2488\\
36.2095	-0.1895\\
36.2495	-0.1968\\
36.2895	-0.1792\\
36.3295	-0.2000\\
36.3695	-0.2512\\
36.4095	-0.2282\\
36.4495	-0.2441\\
36.4895	-0.2815\\
36.5295	-0.2581\\
36.5695	-0.2325\\
36.6095	-0.2140\\
36.6495	-0.2133\\
36.6895	-0.2657\\
36.7295	-0.2428\\
36.7695	-0.2182\\
36.8095	-0.2216\\
36.8495	-0.2380\\
36.8895	-0.2221\\
36.9295	-0.2162\\
36.9695	-0.2185\\
37.0095	-0.1860\\
37.0495	-0.1456\\
37.0895	-0.1235\\
37.1295	-0.1192\\
37.1695	-0.1389\\
37.2095	-0.1106\\
37.2495	-0.0729\\
37.2895	-0.1092\\
37.3295	-0.2133\\
37.3695	-0.2514\\
37.4095	-0.2324\\
37.4495	-0.1708\\
37.4895	-0.1889\\
37.5295	-0.1320\\
37.5695	-0.0898\\
37.6095	-0.1298\\
37.6495	-0.1271\\
37.6895	-0.1421\\
37.7295	-0.1273\\
37.7695	-0.1487\\
37.8095	-0.1418\\
37.8495	-0.1376\\
37.8895	-0.1461\\
37.9295	-0.0844\\
37.9695	-0.0799\\
38.0095	-0.0934\\
38.0495	-0.0883\\
38.0895	-0.0955\\
38.1295	-0.1196\\
38.1695	-0.1280\\
38.2095	-0.1804\\
38.2495	-0.2223\\
38.2895	-0.2145\\
38.3295	-0.1412\\
38.3695	-0.0943\\
38.4095	-0.0942\\
38.4495	-0.1726\\
38.4895	-0.1480\\
38.5295	-0.1311\\
38.5695	-0.0609\\
38.6095	-0.0722\\
38.6495	-0.1381\\
38.6895	-0.2349\\
38.7295	-0.2273\\
38.7695	-0.1872\\
38.8095	-0.1392\\
38.8495	-0.1164\\
38.8895	-0.1200\\
38.9295	-0.1135\\
38.9695	-0.0995\\
39.0095	-0.0908\\
39.0495	-0.0908\\
39.0895	0.0084\\
39.1295	-0.0333\\
39.1695	-0.0395\\
39.2095	-0.0559\\
39.2495	-0.1417\\
39.2895	-0.2159\\
39.3295	-0.2804\\
39.3695	-0.2453\\
39.4095	-0.1852\\
39.4495	-0.1957\\
39.4895	-0.1940\\
39.5295	-0.1433\\
39.5695	-0.1025\\
39.6095	-0.1104\\
39.6495	-0.1150\\
39.6895	-0.1142\\
39.7295	-0.1034\\
39.7695	-0.0903\\
39.8095	-0.0841\\
39.8495	-0.0982\\
39.8895	-0.0822\\
39.9295	-0.0858\\
39.9695	-0.0983\\
40.0095	-0.1047\\
40.0495	-0.0382\\
40.0895	-0.0595\\
40.1295	-0.0466\\
40.1695	-0.0238\\
40.2095	-0.0606\\
40.2495	-0.0855\\
40.2895	-0.0955\\
40.3295	-0.0451\\
40.3695	-0.0736\\
40.4095	-0.0720\\
40.4495	-0.0617\\
40.4895	-0.0577\\
40.5295	-0.0345\\
40.5695	0.0422\\
40.6095	0.1156\\
40.6495	0.1463\\
40.6895	0.0890\\
40.7295	0.0936\\
40.7695	0.1394\\
40.8095	0.1163\\
40.8495	0.1171\\
40.8895	0.1483\\
40.9295	0.2131\\
40.9695	0.1899\\
41.0095	0.1764\\
41.0495	0.1250\\
41.0895	-0.0047\\
41.1295	0.0067\\
41.1695	0.0670\\
41.2095	0.0907\\
41.2495	0.1184\\
41.2895	0.0433\\
41.3295	0.1656\\
41.3695	0.1619\\
41.4095	0.2013\\
41.4495	0.1715\\
41.4895	0.1167\\
41.5295	0.1086\\
41.5695	0.1568\\
41.6095	0.1684\\
41.6495	0.1651\\
41.6895	0.1496\\
41.7295	0.1483\\
41.7695	0.1785\\
41.8095	0.1761\\
41.8495	0.1180\\
41.8895	0.0993\\
41.9295	0.1696\\
41.9695	0.1718\\
42.0095	0.1599\\
42.0495	0.1347\\
42.0895	0.1765\\
42.1295	0.2107\\
42.1695	0.1568\\
42.2095	0.1828\\
42.2495	0.1423\\
42.2895	0.1301\\
42.3295	0.1508\\
42.3695	0.1623\\
42.4095	0.1756\\
42.4495	0.1826\\
42.4895	0.1608\\
42.5295	0.1441\\
42.5695	0.1339\\
42.6095	0.1470\\
42.6495	0.1559\\
42.6895	0.1341\\
42.7295	0.1835\\
42.7695	0.1630\\
42.8095	0.1066\\
42.8495	0.1007\\
42.8895	0.1406\\
42.9295	0.1524\\
42.9695	0.1955\\
43.0095	0.2150\\
43.0495	0.2323\\
43.0895	0.3874\\
43.1295	0.3385\\
43.1695	0.2556\\
43.2095	0.2012\\
43.2495	0.2107\\
43.2895	0.2290\\
43.3395	0.2364\\
43.3795	0.2618\\
43.4195	0.2444\\
43.4595	0.2074\\
43.4995	0.2194\\
43.5395	0.1792\\
43.5795	0.2433\\
43.6195	0.2876\\
43.6595	0.3769\\
43.6995	0.3677\\
43.7395	0.2998\\
43.7795	0.2355\\
43.8195	0.2072\\
43.8595	0.1768\\
43.8995	0.2149\\
43.9395	0.1687\\
43.9795	0.2845\\
44.0195	0.2853\\
44.0595	0.2514\\
44.0995	0.2072\\
44.1395	0.2120\\
44.1795	0.1818\\
44.2195	0.1831\\
44.2595	0.1720\\
44.2995	0.2501\\
44.3395	0.2112\\
44.3795	0.1782\\
44.4195	0.1925\\
44.4595	0.2777\\
44.4995	0.2839\\
44.5395	0.4328\\
44.5795	0.3516\\
44.6195	0.2793\\
44.6595	0.2251\\
44.6995	0.2486\\
44.7395	0.3088\\
44.7795	0.3270\\
44.8195	0.3372\\
44.8595	0.2626\\
44.8995	0.3151\\
44.9395	0.2932\\
44.9795	0.2988\\
45.0195	0.4056\\
45.0595	0.4459\\
45.0995	0.3226\\
45.1395	0.2150\\
45.1795	0.2749\\
45.2195	0.3643\\
45.2595	0.3232\\
45.2995	0.4497\\
45.3395	0.3313\\
45.3795	0.3245\\
45.4195	0.3364\\
45.4595	0.5211\\
45.4995	0.3561\\
45.5395	0.3555\\
45.5795	0.3109\\
45.6195	0.3608\\
45.6595	0.4529\\
45.6995	0.3792\\
45.7395	0.1104\\
45.7795	0.0447\\
45.8195	0.1039\\
45.8595	0.0921\\
45.8995	-0.5548\\
45.9395	-0.3856\\
45.9795	-0.4222\\
46.0195	-0.3185\\
46.0595	-0.4220\\
46.0995	-0.4221\\
46.1395	-0.3766\\
46.1795	-0.3768\\
46.2195	-0.3725\\
46.2595	-0.3316\\
46.2995	-0.2641\\
46.3395	-0.2519\\
46.3795	-0.3728\\
46.4195	-0.3275\\
46.4595	-0.2484\\
46.4995	-0.2844\\
46.5395	-0.3086\\
46.5795	-0.3109\\
46.6195	-0.2846\\
46.6595	-0.2737\\
46.6995	-0.2928\\
46.7395	-0.3430\\
46.7795	-0.3339\\
46.8195	-0.3175\\
46.8595	-0.3423\\
46.8995	-0.3429\\
46.9395	-0.2949\\
46.9795	-0.2610\\
47.0195	-0.3630\\
47.0595	-0.3114\\
47.0995	-0.2096\\
47.1395	-0.2220\\
47.1795	-0.3008\\
47.2195	-0.2912\\
47.2595	-0.2741\\
47.2995	-0.2537\\
47.3395	-0.2627\\
47.3795	-0.2244\\
47.4195	-0.2229\\
47.4595	-0.2108\\
47.4995	-0.2465\\
47.5395	-0.2357\\
47.5795	-0.2193\\
47.6195	-0.2248\\
47.6595	-0.1897\\
47.6995	-0.2342\\
47.7395	-0.2016\\
47.7795	-0.2339\\
47.8195	-0.1969\\
47.8595	-0.2095\\
47.8995	-0.2495\\
47.9395	-0.2535\\
47.9795	-0.2447\\
48.0195	-0.2632\\
48.0595	-0.2264\\
48.0995	-0.2158\\
48.1395	-0.2208\\
48.1795	-0.1906\\
48.2195	-0.1905\\
48.2595	-0.1863\\
48.2995	-0.1744\\
48.3395	-0.1875\\
48.3795	-0.2110\\
48.4195	-0.2180\\
48.4595	-0.2007\\
48.4995	-0.2257\\
48.5395	-0.1911\\
48.5795	-0.2263\\
48.6195	-0.1898\\
48.6595	-0.2360\\
48.6995	-0.2608\\
48.7395	-0.2295\\
48.7795	-0.2171\\
48.8195	-0.2045\\
48.8595	-0.2101\\
48.8995	-0.1933\\
48.9395	-0.1747\\
48.9795	-0.1546\\
49.0195	-0.1591\\
49.0595	-0.1529\\
49.0995	-0.1888\\
49.1395	-0.2022\\
49.1795	-0.2050\\
49.2195	-0.1860\\
49.2595	-0.1774\\
49.2995	-0.1998\\
49.3395	-0.1986\\
49.3795	-0.1758\\
49.4195	-0.1612\\
49.4595	-0.1383\\
49.4995	-0.1125\\
49.5395	-0.3267\\
49.5795	-0.1986\\
49.6195	-0.1920\\
49.6595	-0.1178\\
49.6995	-0.1650\\
49.7395	-0.1675\\
49.7795	-0.1515\\
49.8195	-0.1291\\
49.8595	-0.0792\\
49.8995	-0.0785\\
49.9395	-0.1156\\
49.9795	-0.1481\\
50.0195	-0.1663\\
50.0595	-0.2065\\
50.0995	-0.1972\\
50.1395	-0.1633\\
50.1795	-0.1677\\
50.2195	-0.1687\\
50.2595	-0.1723\\
50.2995	-0.2075\\
50.3395	-0.2057\\
50.3795	-0.1948\\
50.4195	-0.1908\\
50.4595	-0.0483\\
50.4995	-0.0455\\
50.5395	-0.0317\\
50.5795	-0.0278\\
50.6195	-0.0461\\
50.6595	-0.0596\\
50.6995	-0.0402\\
50.7395	0.0274\\
50.7795	0.0641\\
50.8195	0.1058\\
50.8595	0.1836\\
50.8995	0.1699\\
50.9395	0.1616\\
50.9795	0.1286\\
51.0195	0.1352\\
51.0595	0.1408\\
51.0995	0.1518\\
51.1395	0.2501\\
51.1795	0.3216\\
51.2195	0.3293\\
51.2595	0.3089\\
51.2995	0.2963\\
51.3395	0.2637\\
51.3795	0.2236\\
51.4195	0.3731\\
51.4595	0.3038\\
51.4995	0.2646\\
51.5395	0.2607\\
51.5795	0.2338\\
51.6195	0.1997\\
51.6595	0.2654\\
51.6995	0.3773\\
51.7395	0.4419\\
51.7795	0.3564\\
51.8195	0.3154\\
51.8595	0.3507\\
51.8995	0.2587\\
51.9395	0.2264\\
51.9795	0.1497\\
52.0195	0.1910\\
52.0595	0.1008\\
52.0995	0.1799\\
52.1395	0.2668\\
52.1795	0.3228\\
52.2195	0.3649\\
52.2595	0.3922\\
52.2995	0.3949\\
52.3395	0.3371\\
52.3795	0.3017\\
52.4195	0.3482\\
52.4595	0.1909\\
52.4995	0.3005\\
52.5395	0.3843\\
52.5795	0.3736\\
52.6195	0.3251\\
52.6595	0.5101\\
52.6995	0.4752\\
52.7395	0.4282\\
52.7795	0.4754\\
52.8195	0.3746\\
52.8595	0.4272\\
52.8995	0.4859\\
52.9395	0.4544\\
52.9795	0.5044\\
53.0195	0.4721\\
53.0595	0.5058\\
53.0995	0.4785\\
53.1395	0.4241\\
53.1795	0.4715\\
53.2195	0.4964\\
53.2595	0.4455\\
53.2995	0.5075\\
53.3395	0.5066\\
53.3795	0.7484\\
53.4195	0.6333\\
53.4595	0.6106\\
53.4995	0.4129\\
53.5395	0.3941\\
53.5795	0.2342\\
53.6195	0.6630\\
53.6595	0.6242\\
53.6995	0.2533\\
53.7395	0.0201\\
53.7795	-0.0792\\
53.8195	-0.1594\\
53.8595	-0.5806\\
53.8995	-0.6571\\
53.9395	-0.5434\\
53.9795	-0.7402\\
54.0195	-0.8455\\
54.0595	-0.7870\\
54.0995	-0.6951\\
54.1395	-0.7582\\
54.1795	-0.7519\\
54.2195	-0.7395\\
54.2595	-0.7478\\
54.2995	-0.7141\\
54.3395	-0.6557\\
54.3795	-0.6870\\
54.4195	-0.7001\\
54.4595	-0.7009\\
54.4995	-0.6369\\
54.5395	-0.6331\\
54.5795	-0.6328\\
54.6195	-0.6334\\
54.6595	-0.6474\\
54.6995	-0.6382\\
54.7395	-0.6837\\
54.7795	-0.7046\\
54.8195	-0.5788\\
54.8595	-0.5554\\
54.8995	-0.5494\\
54.9395	-0.6691\\
54.9795	-0.5332\\
55.0195	-0.4934\\
55.0595	-0.4309\\
55.0995	-0.3528\\
55.1395	-0.4654\\
55.1795	-0.5308\\
55.2195	-0.4674\\
55.2595	-0.4064\\
55.2995	-0.3499\\
55.3395	-0.3958\\
55.3795	-0.3587\\
55.4195	-0.3903\\
55.4595	-0.3357\\
55.4995	-0.3297\\
55.5395	-0.2554\\
55.5795	-0.2276\\
55.6195	-0.3465\\
55.6595	-0.2506\\
55.6995	-0.1285\\
55.7395	-0.2175\\
55.7795	-0.2724\\
55.8195	-0.2169\\
55.8595	-0.1013\\
55.8995	-0.0814\\
55.9395	-0.0270\\
55.9795	0.0573\\
56.0195	0.0354\\
56.0595	0.0553\\
56.0995	0.4369\\
56.1395	0.3206\\
56.1795	0.4046\\
56.2195	0.3642\\
56.2595	0.4238\\
56.2995	0.4419\\
56.3395	0.5184\\
56.3795	0.5219\\
56.4195	0.5605\\
56.4595	0.5906\\
56.4995	0.5828\\
56.5395	0.6126\\
56.5795	0.5866\\
56.6195	0.5812\\
56.6695	0.5659\\
56.7095	0.5316\\
56.7495	0.5426\\
56.7895	0.6066\\
56.8295	0.6321\\
56.8695	0.7377\\
56.9095	0.6816\\
56.9495	0.6502\\
56.9895	0.6739\\
57.0295	0.6116\\
57.0695	0.6118\\
57.1095	0.7399\\
57.1495	0.7001\\
57.1895	0.6870\\
57.2295	0.8065\\
57.2695	0.7757\\
57.3095	0.6742\\
57.3495	0.7751\\
57.3895	0.7773\\
57.4295	0.7198\\
57.4695	0.8054\\
57.5095	0.7333\\
57.5495	0.7650\\
57.5895	0.7184\\
57.6295	0.6782\\
57.6695	0.5366\\
57.7095	0.4853\\
57.7495	0.3008\\
57.7895	0.1127\\
57.8295	-0.0190\\
57.8695	-0.1556\\
57.9095	-0.7001\\
57.9495	-1.0768\\
57.9895	-1.0836\\
58.0295	-1.1487\\
58.0695	-1.2012\\
58.1095	-1.1628\\
58.1495	-1.1866\\
58.1895	-1.1583\\
58.2295	-1.1562\\
58.2695	-1.0703\\
58.3095	-1.0591\\
58.3495	-1.0329\\
58.3895	-1.0053\\
58.4295	-0.9227\\
58.4695	-0.9116\\
58.5095	-0.8758\\
58.5495	-0.8742\\
58.5895	-0.8032\\
58.6295	-0.7912\\
58.6695	-0.7508\\
58.7095	-0.6772\\
58.7495	-0.6884\\
58.7895	-0.6977\\
58.8295	-0.6771\\
58.8695	-0.6387\\
58.9095	-0.5714\\
58.9495	-0.5405\\
58.9895	-0.5242\\
59.0295	-0.6638\\
59.0695	-0.6689\\
59.1095	-0.5134\\
59.1495	-0.3758\\
59.1895	-0.3227\\
59.2295	-0.3319\\
59.2695	-0.2050\\
59.3095	-0.0934\\
59.3495	-0.0373\\
59.3895	0.0338\\
59.4295	0.2931\\
59.4695	0.3515\\
59.5095	0.4227\\
59.5495	0.4504\\
59.5895	0.4727\\
59.6295	0.4496\\
59.6695	0.4624\\
59.7095	0.4673\\
59.7495	0.5437\\
59.7895	0.5681\\
59.8295	0.5899\\
59.8695	0.6061\\
59.9095	0.6279\\
59.9495	0.6537\\
59.9895	0.7420\\
60.0295	0.7456\\
60.0695	0.7644\\
60.1095	0.7692\\
60.1495	0.7916\\
60.1895	0.7565\\
60.2295	0.7965\\
60.2695	0.8850\\
60.3095	0.9665\\
60.3495	0.9243\\
60.3895	0.8822\\
60.4295	0.9120\\
60.4695	0.9212\\
60.5095	0.7939\\
60.5495	0.8901\\
60.5895	0.8559\\
60.6295	0.8109\\
60.6695	1.0227\\
60.7095	0.8864\\
60.7495	0.9032\\
60.7895	0.9371\\
60.8295	0.4765\\
60.8695	0.1547\\
60.9095	0.0122\\
60.9495	-0.2584\\
60.9895	-0.6063\\
61.0295	-1.0636\\
61.0695	-0.9191\\
61.1095	-1.0744\\
61.1495	-1.2742\\
61.1895	-1.3928\\
61.2295	-1.2992\\
61.2695	-1.3156\\
61.3095	-1.4146\\
61.3495	-1.3993\\
61.3895	-1.3827\\
61.4295	-1.2638\\
61.4695	-1.2716\\
61.5095	-1.1708\\
61.5495	-1.1047\\
61.5895	-0.9965\\
61.6295	-0.9841\\
61.6695	-0.9115\\
61.7095	-0.9432\\
61.7495	-0.9415\\
61.7895	-0.7435\\
61.8295	-0.7583\\
61.8695	-0.7733\\
61.9095	-0.6735\\
61.9495	-0.5353\\
61.9895	-0.4614\\
62.0295	-0.3659\\
62.0695	-0.1738\\
62.1095	-0.0694\\
62.1495	-0.0030\\
62.1895	0.1204\\
62.2295	0.4331\\
62.2695	0.6128\\
62.3095	0.6382\\
62.3495	0.8548\\
62.3895	0.9643\\
62.4295	1.1369\\
62.4695	1.0515\\
62.5095	1.1788\\
62.5495	1.1541\\
62.5895	1.2756\\
62.6295	1.2851\\
62.6695	1.3189\\
62.7095	1.3195\\
62.7495	1.3278\\
62.7895	1.3049\\
62.8295	1.2995\\
62.8695	1.3198\\
62.9095	1.1566\\
62.9495	1.1187\\
62.9895	1.1786\\
63.0295	1.5118\\
63.0695	1.2909\\
63.1095	1.0810\\
63.1495	0.7148\\
63.1895	0.4224\\
63.2295	0.1958\\
63.2695	-0.1315\\
63.3095	-0.8327\\
63.3495	-1.1671\\
63.3895	-1.5167\\
63.4295	-1.7548\\
63.4695	-1.7457\\
63.5095	-1.9842\\
63.5495	-2.0127\\
63.5895	-1.7466\\
63.6295	-1.8224\\
63.6695	-1.6792\\
63.7095	-1.5954\\
63.7495	-1.5074\\
63.7895	-1.3359\\
63.8295	-1.2732\\
63.8695	-1.1742\\
63.9095	-1.0933\\
63.9495	-0.9631\\
63.9895	-0.8887\\
64.0295	-0.8769\\
64.0695	-0.7295\\
64.1095	-0.5543\\
64.1495	-0.3404\\
64.1895	-0.2084\\
64.2295	-0.0026\\
64.2695	0.0392\\
64.3095	0.4548\\
64.3495	0.7954\\
64.3895	0.8462\\
64.4295	0.8973\\
64.4695	1.0853\\
64.5095	1.5166\\
64.5495	1.5570\\
64.5895	1.5665\\
64.6295	1.7593\\
64.6695	1.8718\\
64.7095	2.0585\\
64.7495	1.9172\\
64.7895	1.9649\\
64.8295	2.0412\\
64.8695	2.0662\\
64.9095	1.9551\\
64.9495	1.9879\\
64.9895	2.1535\\
65.0295	1.9610\\
65.0695	1.5772\\
65.1095	0.3772\\
65.1495	-0.8228\\
65.1895	-1.9243\\
65.2295	-2.1160\\
65.2695	-2.4497\\
65.3095	-2.7244\\
65.3495	-2.8384\\
65.3895	-2.8067\\
65.4295	-2.3959\\
65.4695	-2.3835\\
65.5095	-2.0900\\
65.5495	-1.7954\\
65.5895	-1.6011\\
65.6295	-1.2806\\
65.6695	-1.3229\\
65.7095	-1.1369\\
65.7495	-0.8205\\
65.7895	-0.4599\\
65.8295	-0.3556\\
65.8695	-0.0727\\
65.9095	0.0581\\
65.9495	0.4284\\
65.9895	0.8531\\
66.0295	1.0432\\
66.0695	1.4168\\
66.1095	1.5537\\
66.1495	1.7038\\
66.1895	1.7858\\
66.2295	1.9124\\
66.2695	2.1661\\
66.3095	2.2752\\
66.3495	2.1713\\
66.3895	1.9828\\
66.4295	1.9420\\
66.4695	1.8012\\
66.5095	1.6032\\
66.5495	0.7479\\
66.5895	0.2159\\
66.6295	-0.8264\\
66.6695	-1.8095\\
66.7095	-2.3668\\
66.7495	-2.6081\\
66.7895	-2.5545\\
66.8295	-2.6670\\
66.8695	-2.3043\\
66.9095	-2.2375\\
66.9495	-2.1241\\
66.9895	-1.9879\\
67.0295	-1.4090\\
67.0695	-1.5309\\
67.1095	-1.4513\\
67.1495	-1.0902\\
67.1895	-0.9196\\
67.2295	-0.6013\\
67.2695	-0.5428\\
67.3095	-0.3104\\
67.3495	-0.1197\\
67.3895	-0.0728\\
67.4295	0.2593\\
67.4695	0.3182\\
67.5095	0.6405\\
67.5495	0.8155\\
67.5895	0.8573\\
67.6295	0.8675\\
67.6695	0.9570\\
67.7095	0.8456\\
67.7495	0.8442\\
67.7895	0.7874\\
67.8295	0.8796\\
67.8695	0.8193\\
67.9095	0.8834\\
67.9495	0.8113\\
67.9895	0.8541\\
68.0295	0.8304\\
68.0695	0.7812\\
68.1095	0.8254\\
68.1495	0.7910\\
68.1895	0.7079\\
68.2295	0.7049\\
68.2695	0.6283\\
68.3095	0.6850\\
68.3495	0.7976\\
68.3895	0.9113\\
68.4295	0.8228\\
68.4695	0.8054\\
68.5095	0.6756\\
68.5495	0.5920\\
68.5895	0.5679\\
68.6295	0.5190\\
68.6695	0.4365\\
68.7095	0.4136\\
68.7495	0.0306\\
68.7895	0.0167\\
68.8295	0.0231\\
68.8695	-0.0131\\
68.9095	-0.3441\\
68.9495	-0.3209\\
68.9895	-0.4381\\
69.0295	-0.6570\\
69.0695	-0.7431\\
69.1095	-0.8542\\
69.1495	-0.9596\\
69.1895	-0.8425\\
69.2295	-0.8562\\
69.2695	-0.7365\\
69.3095	-0.7613\\
69.3495	-0.7938\\
69.3895	-0.6665\\
69.4295	-0.6492\\
69.4695	-0.8209\\
69.5095	-0.7308\\
69.5495	-0.6114\\
69.5895	-0.5270\\
69.6295	-0.3998\\
69.6695	-0.4243\\
69.7095	-0.3989\\
69.7495	-0.3506\\
69.7895	-0.3403\\
69.8295	-0.3216\\
69.8695	-0.2170\\
69.9095	-0.0917\\
69.9495	-0.0557\\
69.9995	-0.0166\\
};
\end{axis}

\begin{axis}[%
width=0.411\fwidth,
height=0.232\fwidth,
at={(0\fwidth,0.323\fwidth)},
scale only axis,
xmin=40.0022,
xmax=44.9995,
ymin=-0.1076,
ymax=0.4430,
ylabel={$\vartheta_z$, $\hat{\vartheta}_z$ [1/s]},
axis background/.style={fill=white},
title style={font=\labelsize},
xlabel style={font=\labelsize,at={(axis description cs:0.5,\xlabeldist)}},
ylabel style={font=\labelsize,at={(axis description cs:\ylabeldist,0.5)}},
legend style={font=\ticksize},
ticklabel style={font=\ticksize}
]
\addplot [color=mycolor1,solid,forget plot]
  table[row sep=crcr]{%
40.0022	-0.0947\\
40.0130	-0.0940\\
40.0232	-0.0934\\
40.0337	-0.0929\\
40.0441	-0.0931\\
40.0545	-0.0938\\
40.0650	-0.0951\\
40.0753	-0.0971\\
40.0857	-0.0993\\
40.0961	-0.1013\\
40.1065	-0.1034\\
40.1168	-0.1053\\
40.1271	-0.1065\\
40.1380	-0.1072\\
40.1482	-0.1076\\
40.1587	-0.1074\\
40.1689	-0.1062\\
40.1797	-0.1044\\
40.1900	-0.1023\\
40.2011	-0.0993\\
40.2119	-0.0954\\
40.2224	-0.0911\\
40.2325	-0.0866\\
40.2435	-0.0816\\
40.2545	-0.0764\\
40.2649	-0.0710\\
40.2752	-0.0654\\
40.2858	-0.0596\\
40.2961	-0.0535\\
40.3066	-0.0473\\
40.3170	-0.0409\\
40.3274	-0.0340\\
40.3378	-0.0266\\
40.3483	-0.0189\\
40.3584	-0.0107\\
40.3688	-0.0026\\
40.3796	0.0050\\
40.3900	0.0122\\
40.4003	0.0188\\
40.4106	0.0251\\
40.4212	0.0316\\
40.4315	0.0383\\
40.4419	0.0452\\
40.4521	0.0520\\
40.4630	0.0586\\
40.4733	0.0651\\
40.4835	0.0714\\
40.4940	0.0781\\
40.5050	0.0857\\
40.5159	0.0939\\
40.5263	0.1020\\
40.5368	0.1099\\
40.5473	0.1167\\
40.5576	0.1225\\
40.5678	0.1271\\
40.5785	0.1316\\
40.5888	0.1358\\
40.5992	0.1398\\
40.6094	0.1430\\
40.6200	0.1459\\
40.6304	0.1481\\
40.6410	0.1499\\
40.6516	0.1512\\
40.6620	0.1529\\
40.6722	0.1547\\
40.6826	0.1564\\
40.6930	0.1582\\
40.7035	0.1600\\
40.7139	0.1615\\
40.7243	0.1631\\
40.7347	0.1640\\
40.7453	0.1647\\
40.7564	0.1648\\
40.7668	0.1645\\
40.7772	0.1641\\
40.7879	0.1638\\
40.7984	0.1635\\
40.8086	0.1632\\
40.8189	0.1629\\
40.8295	0.1621\\
40.8403	0.1608\\
40.8505	0.1591\\
40.8608	0.1583\\
40.8712	0.1590\\
40.8815	0.1609\\
40.8926	0.1622\\
40.9036	0.1632\\
40.9139	0.1617\\
40.9243	0.1582\\
40.9346	0.1535\\
40.9452	0.1499\\
40.9555	0.1469\\
40.9658	0.1448\\
40.9762	0.1433\\
40.9872	0.1418\\
40.9982	0.1398\\
41.0084	0.1376\\
41.0189	0.1358\\
41.0295	0.1348\\
41.0402	0.1344\\
41.0507	0.1341\\
41.0617	0.1338\\
41.0722	0.1327\\
41.0825	0.1309\\
41.0931	0.1285\\
41.1035	0.1264\\
41.1137	0.1244\\
41.1240	0.1226\\
41.1345	0.1209\\
41.1453	0.1190\\
41.1556	0.1170\\
41.1659	0.1149\\
41.1763	0.1130\\
41.1870	0.1120\\
41.1971	0.1114\\
41.2077	0.1111\\
41.2188	0.1115\\
41.2297	0.1119\\
41.2399	0.1121\\
41.2508	0.1123\\
41.2617	0.1125\\
41.2722	0.1125\\
41.2825	0.1165\\
41.2929	0.1203\\
41.3037	0.1241\\
41.3149	0.1275\\
41.3251	0.1308\\
41.3355	0.1303\\
41.3463	0.1300\\
41.3564	0.1300\\
41.3669	0.1302\\
41.3772	0.1309\\
41.3878	0.1321\\
41.3981	0.1333\\
41.4092	0.1332\\
41.4202	0.1324\\
41.4306	0.1299\\
41.4409	0.1258\\
41.4517	0.1217\\
41.4619	0.1189\\
41.4723	0.1172\\
41.4827	0.1166\\
41.4929	0.1165\\
41.5036	0.1154\\
41.5140	0.1131\\
41.5244	0.1099\\
41.5346	0.1073\\
41.5451	0.1063\\
41.5555	0.1070\\
41.5660	0.1092\\
41.5764	0.1112\\
41.5868	0.1117\\
41.5971	0.1102\\
41.6075	0.1071\\
41.6176	0.1035\\
41.6284	0.1012\\
41.6387	0.1000\\
41.6490	0.1001\\
41.6602	0.1009\\
41.6704	0.1013\\
41.6815	0.1012\\
41.6920	0.1005\\
41.7024	0.0994\\
41.7128	0.0984\\
41.7234	0.0975\\
41.7345	0.0964\\
41.7452	0.0954\\
41.7564	0.0939\\
41.7669	0.0923\\
41.7771	0.0911\\
41.7881	0.0907\\
41.7992	0.0910\\
41.8095	0.0914\\
41.8202	0.0917\\
41.8313	0.0912\\
41.8419	0.0901\\
41.8523	0.0891\\
41.8628	0.0891\\
41.8732	0.0900\\
41.8836	0.0918\\
41.8939	0.0937\\
41.9043	0.0950\\
41.9147	0.0956\\
41.9252	0.0954\\
41.9357	0.0953\\
41.9461	0.0962\\
41.9568	0.0979\\
41.9679	0.1002\\
41.9785	0.1031\\
41.9889	0.1056\\
41.9992	0.1075\\
42.0094	0.1091\\
42.0202	0.1105\\
42.0306	0.1117\\
42.0409	0.1127\\
42.0512	0.1137\\
42.0618	0.1156\\
42.0721	0.1177\\
42.0825	0.1202\\
42.0928	0.1226\\
42.1037	0.1247\\
42.1147	0.1260\\
42.1250	0.1265\\
42.1354	0.1266\\
42.1461	0.1269\\
42.1566	0.1289\\
42.1672	0.1326\\
42.1773	0.1364\\
42.1879	0.1403\\
42.1982	0.1438\\
42.2085	0.1454\\
42.2188	0.1454\\
42.2295	0.1459\\
42.2397	0.1469\\
42.2502	0.1483\\
42.2607	0.1512\\
42.2714	0.1552\\
42.2825	0.1601\\
42.2928	0.1652\\
42.3034	0.1707\\
42.3140	0.1754\\
42.3242	0.1793\\
42.3346	0.1820\\
42.3451	0.1845\\
42.3557	0.1859\\
42.3659	0.1867\\
42.3761	0.1870\\
42.3871	0.1876\\
42.3974	0.1884\\
42.4075	0.1895\\
42.4177	0.1909\\
42.4285	0.1933\\
42.4389	0.1954\\
42.4492	0.1975\\
42.4598	0.1991\\
42.4701	0.2008\\
42.4807	0.2014\\
42.4918	0.2016\\
42.5024	0.1998\\
42.5129	0.1967\\
42.5231	0.1919\\
42.5336	0.1856\\
42.5439	0.1794\\
42.5544	0.1750\\
42.5647	0.1720\\
42.5751	0.1706\\
42.5856	0.1709\\
42.5961	0.1714\\
42.6064	0.1699\\
42.6166	0.1667\\
42.6267	0.1629\\
42.6379	0.1591\\
42.6482	0.1546\\
42.6588	0.1514\\
42.6690	0.1503\\
42.6794	0.1502\\
42.6900	0.1505\\
42.7012	0.1523\\
42.7118	0.1556\\
42.7220	0.1592\\
42.7324	0.1633\\
42.7428	0.1663\\
42.7533	0.1679\\
42.7637	0.1674\\
42.7742	0.1656\\
42.7845	0.1633\\
42.7950	0.1624\\
42.8057	0.1623\\
42.8167	0.1632\\
42.8270	0.1657\\
42.8378	0.1687\\
42.8481	0.1719\\
42.8584	0.1755\\
42.8687	0.1789\\
42.8793	0.1821\\
42.8905	0.1845\\
42.9011	0.1850\\
42.9116	0.1845\\
42.9220	0.1828\\
42.9323	0.1799\\
42.9427	0.1777\\
42.9535	0.1772\\
42.9637	0.1781\\
42.9740	0.1805\\
42.9844	0.1835\\
42.9952	0.1855\\
43.0055	0.1862\\
43.0157	0.1858\\
43.0260	0.1852\\
43.0367	0.1858\\
43.0470	0.1872\\
43.0575	0.1895\\
43.0677	0.1916\\
43.0784	0.1926\\
43.0888	0.1921\\
43.0992	0.1903\\
43.1095	0.1882\\
43.1199	0.1875\\
43.1302	0.1875\\
43.1405	0.1887\\
43.1514	0.1898\\
43.1620	0.1903\\
43.1723	0.1892\\
43.1824	0.1871\\
43.1927	0.1856\\
43.2034	0.1856\\
43.2139	0.1874\\
43.2242	0.1908\\
43.2345	0.1963\\
43.2451	0.2026\\
43.2556	0.2093\\
43.2661	0.2163\\
43.2770	0.2225\\
43.2876	0.2278\\
43.2979	0.2315\\
43.3083	0.2342\\
43.3187	0.2364\\
43.3296	0.2401\\
43.3398	0.2444\\
43.3501	0.2502\\
43.3604	0.2565\\
43.3711	0.2634\\
43.3812	0.2695\\
43.3916	0.2705\\
43.4024	0.2671\\
43.4128	0.2579\\
43.4232	0.2436\\
43.4335	0.2341\\
43.4439	0.2336\\
43.4544	0.2429\\
43.4651	0.2556\\
43.4758	0.2711\\
43.4866	0.2808\\
43.4971	0.2860\\
43.5074	0.2862\\
43.5177	0.2882\\
43.5286	0.2888\\
43.5388	0.2863\\
43.5500	0.2796\\
43.5605	0.2711\\
43.5711	0.2617\\
43.5814	0.2536\\
43.5918	0.2479\\
43.6022	0.2437\\
43.6127	0.2395\\
43.6230	0.2342\\
43.6336	0.2287\\
43.6439	0.2246\\
43.6544	0.2244\\
43.6647	0.2273\\
43.6757	0.2286\\
43.6868	0.2289\\
43.6973	0.2242\\
43.7083	0.2157\\
43.7187	0.2048\\
43.7295	0.1962\\
43.7396	0.1892\\
43.7500	0.1837\\
43.7604	0.1817\\
43.7711	0.1812\\
43.7814	0.1817\\
43.7917	0.1835\\
43.8021	0.1847\\
43.8126	0.1851\\
43.8237	0.1836\\
43.8345	0.1818\\
43.8451	0.1795\\
43.8555	0.1783\\
43.8666	0.1782\\
43.8770	0.1786\\
43.8876	0.1793\\
43.8981	0.1793\\
43.9086	0.1790\\
43.9187	0.1773\\
43.9295	0.1756\\
43.9396	0.1727\\
43.9501	0.1691\\
43.9605	0.1665\\
43.9712	0.1665\\
43.9824	0.1680\\
43.9928	0.1713\\
44.0033	0.1761\\
44.0138	0.1797\\
44.0249	0.1830\\
44.0352	0.1843\\
44.0461	0.1852\\
44.0563	0.1844\\
44.0668	0.1827\\
44.0774	0.1815\\
44.0878	0.1822\\
44.0980	0.1843\\
44.1084	0.1883\\
44.1186	0.1927\\
44.1294	0.1963\\
44.1396	0.1982\\
44.1500	0.1985\\
44.1603	0.1978\\
44.1712	0.1973\\
44.1823	0.1961\\
44.1927	0.1955\\
44.2033	0.1967\\
44.2138	0.1986\\
44.2240	0.2017\\
44.2345	0.2057\\
44.2449	0.2095\\
44.2553	0.2133\\
44.2659	0.2167\\
44.2760	0.2199\\
44.2868	0.2233\\
44.2971	0.2259\\
44.3073	0.2284\\
44.3176	0.2295\\
44.3285	0.2299\\
44.3388	0.2288\\
44.3490	0.2265\\
44.3595	0.2251\\
44.3704	0.2254\\
44.3815	0.2279\\
44.3916	0.2324\\
44.4021	0.2393\\
44.4127	0.2474\\
44.4237	0.2556\\
44.4344	0.2618\\
44.4451	0.2666\\
44.4554	0.2684\\
44.4658	0.2676\\
44.4764	0.2648\\
44.4867	0.2631\\
44.4971	0.2614\\
44.5077	0.2603\\
44.5186	0.2609\\
44.5293	0.2627\\
44.5397	0.2655\\
44.5501	0.2688\\
44.5604	0.2753\\
44.5711	0.2850\\
44.5824	0.2964\\
44.5926	0.3054\\
44.6031	0.3129\\
44.6135	0.3156\\
44.6240	0.3130\\
44.6344	0.3078\\
44.6451	0.3053\\
44.6555	0.3046\\
44.6665	0.3060\\
44.6771	0.3118\\
44.6874	0.3209\\
44.6983	0.3307\\
44.7092	0.3381\\
44.7199	0.3449\\
44.7304	0.3459\\
44.7415	0.3428\\
44.7521	0.3382\\
44.7628	0.3352\\
44.7730	0.3335\\
44.7833	0.3338\\
44.7937	0.3394\\
44.8044	0.3477\\
44.8147	0.3578\\
44.8249	0.3640\\
44.8360	0.3680\\
44.8471	0.3639\\
44.8574	0.3533\\
44.8678	0.3409\\
44.8783	0.3325\\
44.8885	0.3275\\
44.8991	0.3266\\
44.9094	0.3288\\
44.9198	0.3303\\
44.9303	0.3310\\
44.9404	0.3300\\
44.9509	0.3264\\
44.9616	0.3219\\
44.9720	0.3148\\
44.9825	0.3063\\
44.9926	0.2987\\
};
\addplot [color=mycolor2,solid,forget plot]
  table[row sep=crcr]{%
40.0095	-0.1047\\
40.0195	-0.0894\\
40.0295	-0.0596\\
40.0395	-0.0457\\
40.0495	-0.0382\\
40.0595	-0.0364\\
40.0695	-0.0398\\
40.0795	-0.0475\\
40.0895	-0.0595\\
40.0995	-0.0659\\
40.1095	-0.0641\\
40.1195	-0.0616\\
40.1295	-0.0466\\
40.1395	-0.0379\\
40.1495	-0.0355\\
40.1595	-0.0297\\
40.1695	-0.0238\\
40.1795	-0.0372\\
40.1895	-0.0443\\
40.1995	-0.0540\\
40.2095	-0.0606\\
40.2195	-0.0674\\
40.2295	-0.0774\\
40.2395	-0.0781\\
40.2495	-0.0855\\
40.2595	-0.0938\\
40.2695	-0.0903\\
40.2795	-0.1071\\
40.2895	-0.0955\\
40.2995	-0.0849\\
40.3095	-0.0625\\
40.3195	-0.0528\\
40.3295	-0.0451\\
40.3395	-0.0565\\
40.3495	-0.0565\\
40.3595	-0.0693\\
40.3695	-0.0736\\
40.3795	-0.0736\\
40.3895	-0.0720\\
40.3995	-0.0720\\
40.4095	-0.0720\\
40.4195	-0.0675\\
40.4295	-0.0653\\
40.4395	-0.0651\\
40.4495	-0.0617\\
40.4595	-0.0603\\
40.4695	-0.0594\\
40.4795	-0.0580\\
40.4895	-0.0577\\
40.4995	-0.0576\\
40.5095	-0.0557\\
40.5195	-0.0507\\
40.5295	-0.0345\\
40.5395	-0.0164\\
40.5495	0.0081\\
40.5595	0.0127\\
40.5695	0.0422\\
40.5795	0.0531\\
40.5895	0.0833\\
40.5995	0.1105\\
40.6095	0.1156\\
40.6195	0.1355\\
40.6295	0.1450\\
40.6395	0.1315\\
40.6495	0.1463\\
40.6595	0.1320\\
40.6695	0.1236\\
40.6795	0.1248\\
40.6895	0.0890\\
40.6995	0.0882\\
40.7095	0.0837\\
40.7195	0.0842\\
40.7295	0.0936\\
40.7395	0.1148\\
40.7495	0.1291\\
40.7595	0.1417\\
40.7695	0.1394\\
40.7795	0.1267\\
40.7895	0.1151\\
40.7995	0.1201\\
40.8095	0.1163\\
40.8195	0.1098\\
40.8295	0.1091\\
40.8395	0.1109\\
40.8495	0.1171\\
40.8595	0.1097\\
40.8695	0.1305\\
40.8795	0.1216\\
40.8895	0.1483\\
40.8995	0.1758\\
40.9095	0.1906\\
40.9195	0.2088\\
40.9295	0.2131\\
40.9395	0.1959\\
40.9495	0.1937\\
40.9595	0.1914\\
40.9695	0.1899\\
40.9795	0.1830\\
40.9895	0.1789\\
40.9995	0.1811\\
41.0095	0.1764\\
41.0195	0.1843\\
41.0295	0.1908\\
41.0395	0.1800\\
41.0495	0.1250\\
41.0595	0.0503\\
41.0695	0.0071\\
41.0795	-0.0018\\
41.0895	-0.0047\\
41.0995	-0.0031\\
41.1095	-0.0032\\
41.1195	-0.0023\\
41.1295	0.0067\\
41.1395	0.0158\\
41.1495	0.0468\\
41.1595	0.0601\\
41.1695	0.0670\\
41.1795	0.0724\\
41.1895	0.0778\\
41.1995	0.0824\\
41.2095	0.0907\\
41.2195	0.1264\\
41.2295	0.1432\\
41.2395	0.1510\\
41.2495	0.1184\\
41.2595	0.0969\\
41.2695	0.0720\\
41.2795	0.0557\\
41.2895	0.0433\\
41.2995	0.0243\\
41.3095	0.0199\\
41.3195	0.1311\\
41.3295	0.1656\\
41.3395	0.1726\\
41.3495	0.1640\\
41.3595	0.1733\\
41.3695	0.1619\\
41.3795	0.1495\\
41.3895	0.1624\\
41.3995	0.1836\\
41.4095	0.2013\\
41.4195	0.1819\\
41.4295	0.2027\\
41.4395	0.1922\\
41.4495	0.1715\\
41.4595	0.1503\\
41.4695	0.1214\\
41.4795	0.1169\\
41.4895	0.1167\\
41.4995	0.1136\\
41.5095	0.1054\\
41.5195	0.1045\\
41.5295	0.1086\\
41.5395	0.1150\\
41.5495	0.1488\\
41.5595	0.1495\\
41.5695	0.1568\\
41.5795	0.1601\\
41.5895	0.1581\\
41.5995	0.1602\\
41.6095	0.1684\\
41.6195	0.1695\\
41.6295	0.1709\\
41.6395	0.1712\\
41.6495	0.1651\\
41.6595	0.1608\\
41.6695	0.1762\\
41.6795	0.1600\\
41.6895	0.1496\\
41.6995	0.1487\\
41.7095	0.1488\\
41.7195	0.1498\\
41.7295	0.1483\\
41.7395	0.1487\\
41.7495	0.1606\\
41.7595	0.1785\\
41.7695	0.1785\\
41.7795	0.1728\\
41.7895	0.1814\\
41.7995	0.1761\\
41.8095	0.1761\\
41.8195	0.1664\\
41.8295	0.1411\\
41.8395	0.1280\\
41.8495	0.1180\\
41.8595	0.1071\\
41.8695	0.1038\\
41.8795	0.0999\\
41.8895	0.0993\\
41.8995	0.0894\\
41.9095	0.0894\\
41.9195	0.1524\\
41.9295	0.1696\\
41.9395	0.1755\\
41.9495	0.1798\\
41.9595	0.1754\\
41.9695	0.1718\\
41.9795	0.1646\\
41.9895	0.1636\\
41.9995	0.1618\\
42.0095	0.1599\\
42.0195	0.1565\\
42.0295	0.1498\\
42.0395	0.1425\\
42.0495	0.1347\\
42.0595	0.1320\\
42.0695	0.1818\\
42.0795	0.1743\\
42.0895	0.1765\\
42.0995	0.1704\\
42.1095	0.1746\\
42.1195	0.1830\\
42.1295	0.2107\\
42.1395	0.2037\\
42.1495	0.1814\\
42.1595	0.1560\\
42.1695	0.1568\\
42.1795	0.1626\\
42.1895	0.1606\\
42.1995	0.1638\\
42.2095	0.1828\\
42.2195	0.1812\\
42.2295	0.1633\\
42.2395	0.1434\\
42.2495	0.1423\\
42.2595	0.1379\\
42.2695	0.1274\\
42.2795	0.1372\\
42.2895	0.1301\\
42.2995	0.1270\\
42.3095	0.1345\\
42.3195	0.1496\\
42.3295	0.1508\\
42.3395	0.1493\\
42.3495	0.1538\\
42.3595	0.1591\\
42.3695	0.1623\\
42.3795	0.1746\\
42.3895	0.1746\\
42.3995	0.1764\\
42.4095	0.1756\\
42.4195	0.1874\\
42.4295	0.1868\\
42.4395	0.1856\\
42.4495	0.1826\\
42.4595	0.1815\\
42.4695	0.1739\\
42.4795	0.1649\\
42.4895	0.1608\\
42.4995	0.1579\\
42.5095	0.1529\\
42.5195	0.1487\\
42.5295	0.1441\\
42.5395	0.1406\\
42.5495	0.1362\\
42.5595	0.1331\\
42.5695	0.1339\\
42.5795	0.1415\\
42.5895	0.1516\\
42.5995	0.1522\\
42.6095	0.1470\\
42.6195	0.1551\\
42.6295	0.1551\\
42.6395	0.1579\\
42.6495	0.1559\\
42.6595	0.1486\\
42.6695	0.1441\\
42.6795	0.1409\\
42.6895	0.1341\\
42.6995	0.1982\\
42.7095	0.2153\\
42.7195	0.2015\\
42.7295	0.1835\\
42.7395	0.1658\\
42.7495	0.1757\\
42.7595	0.1822\\
42.7695	0.1630\\
42.7795	0.1971\\
42.7895	0.1501\\
42.7995	0.1133\\
42.8095	0.1066\\
42.8195	0.1005\\
42.8295	0.0932\\
42.8395	0.0931\\
42.8495	0.1007\\
42.8595	0.0908\\
42.8695	0.1251\\
42.8795	0.1402\\
42.8895	0.1406\\
42.8995	0.1376\\
42.9095	0.1520\\
42.9195	0.1548\\
42.9295	0.1524\\
42.9395	0.1545\\
42.9495	0.1561\\
42.9595	0.1745\\
42.9695	0.1955\\
42.9795	0.2064\\
42.9895	0.1946\\
42.9995	0.1990\\
43.0095	0.2150\\
43.0195	0.2229\\
43.0295	0.2648\\
43.0395	0.2501\\
43.0495	0.2323\\
43.0595	0.3885\\
43.0695	0.4142\\
43.0795	0.4024\\
43.0895	0.3874\\
43.0995	0.3723\\
43.1095	0.3602\\
43.1195	0.3603\\
43.1295	0.3385\\
43.1395	0.3134\\
43.1495	0.2924\\
43.1595	0.2734\\
43.1695	0.2556\\
43.1795	0.2455\\
43.1895	0.2248\\
43.1995	0.1955\\
43.2095	0.2012\\
43.2195	0.2053\\
43.2295	0.2116\\
43.2395	0.2114\\
43.2495	0.2107\\
43.2595	0.2098\\
43.2695	0.2104\\
43.2795	0.2182\\
43.2895	0.2290\\
43.2995	0.2398\\
43.3095	0.2403\\
43.3195	0.2391\\
43.3295	0.2380\\
43.3395	0.2364\\
43.3495	0.2436\\
43.3595	0.2376\\
43.3695	0.2522\\
43.3795	0.2618\\
43.3895	0.2529\\
43.3995	0.2425\\
43.4095	0.2503\\
43.4195	0.2444\\
43.4295	0.2285\\
43.4395	0.2325\\
43.4495	0.2213\\
43.4595	0.2074\\
43.4695	0.2076\\
43.4795	0.2095\\
43.4895	0.2140\\
43.4995	0.2194\\
43.5095	0.1519\\
43.5195	0.0790\\
43.5295	0.1441\\
43.5395	0.1792\\
43.5495	0.1603\\
43.5595	0.2015\\
43.5695	0.2440\\
43.5795	0.2433\\
43.5895	0.2562\\
43.5995	0.2523\\
43.6095	0.2865\\
43.6195	0.2876\\
43.6295	0.2915\\
43.6395	0.2997\\
43.6495	0.3495\\
43.6595	0.3769\\
43.6695	0.3611\\
43.6795	0.3420\\
43.6895	0.3555\\
43.6995	0.3677\\
43.7095	0.3475\\
43.7195	0.3266\\
43.7295	0.3140\\
43.7395	0.2998\\
43.7495	0.2754\\
43.7595	0.2347\\
43.7695	0.2291\\
43.7795	0.2355\\
43.7895	0.2302\\
43.7995	0.2185\\
43.8095	0.2168\\
43.8195	0.2072\\
43.8295	0.1934\\
43.8395	0.1630\\
43.8495	0.1630\\
43.8595	0.1768\\
43.8695	0.1929\\
43.8795	0.1945\\
43.8895	0.1929\\
43.8995	0.2149\\
43.9095	0.1803\\
43.9195	0.1249\\
43.9295	0.1403\\
43.9395	0.1687\\
43.9495	0.1850\\
43.9595	0.2546\\
43.9695	0.2672\\
43.9795	0.2845\\
43.9895	0.2979\\
43.9995	0.2931\\
44.0095	0.2862\\
44.0195	0.2853\\
44.0295	0.2697\\
44.0395	0.2619\\
44.0495	0.2595\\
44.0595	0.2514\\
44.0695	0.2409\\
44.0795	0.2377\\
44.0895	0.2234\\
44.0995	0.2072\\
44.1095	0.2067\\
44.1195	0.2017\\
44.1295	0.2148\\
44.1395	0.2120\\
44.1495	0.2004\\
44.1595	0.1898\\
44.1695	0.1821\\
44.1795	0.1818\\
44.1895	0.1887\\
44.1995	0.1895\\
44.2095	0.1897\\
44.2195	0.1831\\
44.2295	0.1906\\
44.2395	0.1884\\
44.2495	0.1781\\
44.2595	0.1720\\
44.2695	0.1914\\
44.2795	0.1992\\
44.2895	0.2094\\
44.2995	0.2501\\
44.3095	0.2533\\
44.3195	0.2504\\
44.3295	0.2303\\
44.3395	0.2112\\
44.3495	0.2071\\
44.3595	0.1984\\
44.3695	0.1882\\
44.3795	0.1782\\
44.3895	0.1661\\
44.3995	0.1587\\
44.4095	0.1600\\
44.4195	0.1925\\
44.4295	0.2190\\
44.4395	0.2375\\
44.4495	0.2615\\
44.4595	0.2777\\
44.4695	0.2718\\
44.4795	0.2956\\
44.4895	0.2969\\
44.4995	0.2839\\
44.5095	0.3265\\
44.5195	0.3739\\
44.5295	0.4249\\
44.5395	0.4328\\
44.5495	0.4430\\
44.5595	0.4077\\
44.5695	0.3690\\
44.5795	0.3516\\
44.5895	0.3402\\
44.5995	0.3289\\
44.6095	0.3119\\
44.6195	0.2793\\
44.6295	0.2628\\
44.6395	0.2528\\
44.6495	0.2215\\
44.6595	0.2251\\
44.6695	0.2154\\
44.6795	0.2325\\
44.6895	0.2450\\
44.6995	0.2486\\
44.7095	0.2661\\
44.7195	0.2780\\
44.7295	0.2920\\
44.7395	0.3088\\
44.7495	0.3196\\
44.7595	0.3181\\
44.7695	0.3212\\
44.7795	0.3270\\
44.7895	0.3188\\
44.7995	0.3252\\
44.8095	0.3318\\
44.8195	0.3372\\
44.8295	0.3247\\
44.8395	0.3311\\
44.8495	0.2350\\
44.8595	0.2626\\
44.8695	0.2972\\
44.8795	0.3122\\
44.8895	0.3191\\
44.8995	0.3151\\
44.9095	0.3172\\
44.9195	0.3036\\
44.9295	0.3003\\
44.9395	0.2932\\
44.9495	0.3237\\
44.9595	0.3315\\
44.9695	0.3201\\
44.9795	0.2988\\
44.9895	0.3135\\
44.9995	0.3094\\
};
\end{axis}

\begin{axis}[%
width=0.411\fwidth,
height=0.232\fwidth,
at={(0\fwidth,0\fwidth)},
scale only axis,
xmin=40.0095,
xmax=44.9995,
xlabel={$t$ [s]},
ymin=0.0000,
ymax=0.9648,
ylabel={$K$ [-]},
axis background/.style={fill=white},
title style={font=\labelsize},
xlabel style={font=\labelsize,at={(axis description cs:0.5,\xlabeldist)}},
ylabel style={font=\labelsize,at={(axis description cs:\ylabeldist,0.5)}},
legend style={font=\ticksize},
ticklabel style={font=\ticksize}
]
\addplot [color=mycolor1,solid,forget plot]
  table[row sep=crcr]{%
40.0095	0.4928\\
40.0195	0.6633\\
40.0295	0.7605\\
40.0395	0.8189\\
40.0495	0.8455\\
40.0595	0.4963\\
40.0695	0.5032\\
40.0795	0.5975\\
40.0895	0.6595\\
40.0995	0.6723\\
40.1095	0.0456\\
40.1195	0.0453\\
40.1295	0.1596\\
40.1395	0.1321\\
40.1495	0.0571\\
40.1595	0.1616\\
40.1695	0.4483\\
40.1795	0.4692\\
40.1895	0.4693\\
40.1995	0.5767\\
40.2095	0.6374\\
40.2195	0.6690\\
40.2295	0.6120\\
40.2395	0.6649\\
40.2495	0.7810\\
40.2595	0.8406\\
40.2695	0.8121\\
40.2795	0.8138\\
40.2895	0.7975\\
40.2995	0.8626\\
40.3095	0.3855\\
40.3195	0.5557\\
40.3295	0.6502\\
40.3395	0.6213\\
40.3495	0.0000\\
40.3595	0.3230\\
40.3695	0.2581\\
40.3795	0.0000\\
40.3895	0.1874\\
40.3995	0.0000\\
40.4095	0.0000\\
40.4195	0.1088\\
40.4295	0.0924\\
40.4395	0.0044\\
40.4495	0.0584\\
40.4595	0.0408\\
40.4695	0.0253\\
40.4795	0.0930\\
40.4895	0.0417\\
40.4995	0.0037\\
40.5095	0.0411\\
40.5195	0.0981\\
40.5295	0.2049\\
40.5395	0.2677\\
40.5495	0.3781\\
40.5595	0.2412\\
40.5695	0.4166\\
40.5795	0.5060\\
40.5895	0.3700\\
40.5995	0.3476\\
40.6095	0.2616\\
40.6195	0.3838\\
40.6295	0.2685\\
40.6395	0.3498\\
40.6495	0.5936\\
40.6595	0.6507\\
40.6695	0.7677\\
40.6795	0.8379\\
40.6895	0.6817\\
40.6995	0.5339\\
40.7095	0.6214\\
40.7195	0.5733\\
40.7295	0.6598\\
40.7395	0.7933\\
40.7495	0.8827\\
40.7595	0.9156\\
40.7695	0.9375\\
40.7795	0.9078\\
40.7895	0.8969\\
40.7995	0.8117\\
40.8095	0.8722\\
40.8195	0.9191\\
40.8295	0.9206\\
40.8395	0.7514\\
40.8495	0.8848\\
40.8595	0.8767\\
40.8695	0.4754\\
40.8795	0.9372\\
40.8895	0.7824\\
40.8995	0.8414\\
40.9095	0.8336\\
40.9195	0.8187\\
40.9295	0.7166\\
40.9395	0.4411\\
40.9495	0.1143\\
40.9595	0.0868\\
40.9695	0.0532\\
40.9795	0.2603\\
40.9895	0.0865\\
40.9995	0.2053\\
41.0095	0.4415\\
41.0195	0.6292\\
41.0295	0.6647\\
41.0395	0.5957\\
41.0495	0.5192\\
41.0595	0.5233\\
41.0695	0.5165\\
41.0795	0.3448\\
41.0895	0.3307\\
41.0995	0.4961\\
41.1095	0.0220\\
41.1195	0.0551\\
41.1295	0.2953\\
41.1395	0.4060\\
41.1495	0.5076\\
41.1595	0.4771\\
41.1695	0.5042\\
41.1795	0.5044\\
41.1895	0.5761\\
41.1995	0.6343\\
41.2095	0.5856\\
41.2195	0.5751\\
41.2295	0.3116\\
41.2395	0.4152\\
41.2495	0.5040\\
41.2595	0.4612\\
41.2695	0.6512\\
41.2795	0.8346\\
41.2895	0.8582\\
41.2995	0.9514\\
41.3095	0.9135\\
41.3195	0.7267\\
41.3295	0.4857\\
41.3395	0.5062\\
41.3495	0.4386\\
41.3595	0.6605\\
41.3695	0.5633\\
41.3795	0.6095\\
41.3895	0.6231\\
41.3995	0.6921\\
41.4095	0.7660\\
41.4195	0.7443\\
41.4295	0.4516\\
41.4395	0.3550\\
41.4495	0.4184\\
41.4595	0.6215\\
41.4695	0.8415\\
41.4795	0.9128\\
41.4895	0.9317\\
41.4995	0.9296\\
41.5095	0.5605\\
41.5195	0.3767\\
41.5295	0.7635\\
41.5395	0.9648\\
41.5495	0.6053\\
41.5595	0.7395\\
41.5695	0.2905\\
41.5795	0.2814\\
41.5895	0.2580\\
41.5995	0.3999\\
41.6095	0.6448\\
41.6195	0.7010\\
41.6295	0.7808\\
41.6395	0.5067\\
41.6495	0.5698\\
41.6595	0.5655\\
41.6695	0.7241\\
41.6795	0.6008\\
41.6895	0.6680\\
41.6995	0.2376\\
41.7095	0.2301\\
41.7195	0.0340\\
41.7295	0.2862\\
41.7395	0.4553\\
41.7495	0.1928\\
41.7595	0.3341\\
41.7695	0.4105\\
41.7795	0.2282\\
41.7895	0.5250\\
41.7995	0.4465\\
41.8095	0.0000\\
41.8195	0.1667\\
41.8295	0.4457\\
41.8395	0.3142\\
41.8495	0.2901\\
41.8595	0.4998\\
41.8695	0.4900\\
41.8795	0.5206\\
41.8895	0.5912\\
41.8995	0.7100\\
41.9095	0.0000\\
41.9195	0.3232\\
41.9295	0.5006\\
41.9395	0.5166\\
41.9495	0.5021\\
41.9595	0.6448\\
41.9695	0.0788\\
41.9795	0.1697\\
41.9895	0.0136\\
41.9995	0.0430\\
42.0095	0.0579\\
42.0195	0.0865\\
42.0295	0.2675\\
42.0395	0.3285\\
42.0495	0.4375\\
42.0595	0.5099\\
42.0695	0.4793\\
42.0795	0.4007\\
42.0895	0.5371\\
42.0995	0.5968\\
42.1095	0.8004\\
42.1195	0.8203\\
42.1295	0.7577\\
42.1395	0.7318\\
42.1495	0.8239\\
42.1595	0.8215\\
42.1695	0.2747\\
42.1795	0.2219\\
42.1895	0.1202\\
42.1995	0.3125\\
42.2095	0.2809\\
42.2195	0.6124\\
42.2295	0.7478\\
42.2395	0.5334\\
42.2495	0.1127\\
42.2595	0.1756\\
42.2695	0.2354\\
42.2795	0.4058\\
42.2895	0.1225\\
42.2995	0.2910\\
42.3095	0.1482\\
42.3195	0.1840\\
42.3295	0.1073\\
42.3395	0.1535\\
42.3495	0.1594\\
42.3595	0.2147\\
42.3695	0.1630\\
42.3795	0.3047\\
42.3895	0.4885\\
42.3995	0.5117\\
42.4095	0.0275\\
42.4195	0.1924\\
42.4295	0.0542\\
42.4395	0.0521\\
42.4495	0.0665\\
42.4595	0.0264\\
42.4695	0.1450\\
42.4795	0.1797\\
42.4895	0.1285\\
42.4995	0.0907\\
42.5095	0.2907\\
42.5195	0.4124\\
42.5295	0.3923\\
42.5395	0.4909\\
42.5495	0.5335\\
42.5595	0.3298\\
42.5695	0.4248\\
42.5795	0.5613\\
42.5895	0.6840\\
42.5995	0.7497\\
42.6095	0.7313\\
42.6195	0.5260\\
42.6295	0.0000\\
42.6395	0.4597\\
42.6495	0.5010\\
42.6595	0.4815\\
42.6695	0.5448\\
42.6795	0.5303\\
42.6895	0.4470\\
42.6995	0.7935\\
42.7095	0.6872\\
42.7195	0.6749\\
42.7295	0.7433\\
42.7395	0.6869\\
42.7495	0.3572\\
42.7595	0.7195\\
42.7695	0.8307\\
42.7795	0.4300\\
42.7895	0.8337\\
42.7995	0.8187\\
42.8095	0.8676\\
42.8195	0.9199\\
42.8295	0.9353\\
42.8395	0.8854\\
42.8495	0.9191\\
42.8595	0.7588\\
42.8695	0.7508\\
42.8795	0.7459\\
42.8895	0.5502\\
42.8995	0.5561\\
42.9095	0.5179\\
42.9195	0.4881\\
42.9295	0.4810\\
42.9395	0.5420\\
42.9495	0.5634\\
42.9595	0.8058\\
42.9695	0.8107\\
42.9795	0.8303\\
42.9895	0.8746\\
42.9995	0.8941\\
43.0095	0.6942\\
43.0195	0.6610\\
43.0295	0.6520\\
43.0395	0.6180\\
43.0495	0.5130\\
43.0595	0.6492\\
43.0695	0.4903\\
43.0795	0.2356\\
43.0895	0.2232\\
43.0995	0.4115\\
43.1095	0.6721\\
43.1195	0.6641\\
43.1295	0.7321\\
43.1395	0.8210\\
43.1495	0.8329\\
43.1595	0.5107\\
43.1695	0.5151\\
43.1795	0.5063\\
43.1895	0.6945\\
43.1995	0.7501\\
43.2095	0.3821\\
43.2195	0.4013\\
43.2295	0.3311\\
43.2395	0.0020\\
43.2495	0.1979\\
43.2595	0.3248\\
43.2695	0.0328\\
43.2795	0.1379\\
43.2895	0.3283\\
43.2995	0.3426\\
43.3095	0.1906\\
43.3195	0.2756\\
43.3295	0.3885\\
43.3395	0.5772\\
43.3495	0.6531\\
43.3595	0.6930\\
43.3695	0.7155\\
43.3795	0.7268\\
43.3895	0.7530\\
43.3995	0.7627\\
43.4095	0.7988\\
43.4195	0.8721\\
43.4295	0.7826\\
43.4395	0.8075\\
43.4495	0.8209\\
43.4595	0.8380\\
43.4695	0.9150\\
43.4795	0.9025\\
43.4895	0.9182\\
43.4995	0.7970\\
43.5095	0.6131\\
43.5195	0.7110\\
43.5295	0.6090\\
43.5395	0.6834\\
43.5495	0.6061\\
43.5595	0.6007\\
43.5695	0.7390\\
43.5795	0.6630\\
43.5895	0.8029\\
43.5995	0.8696\\
43.6095	0.9083\\
43.6195	0.9064\\
43.6295	0.9100\\
43.6395	0.7477\\
43.6495	0.6949\\
43.6595	0.7980\\
43.6695	0.7981\\
43.6795	0.7959\\
43.6895	0.7234\\
43.6995	0.8708\\
43.7095	0.9051\\
43.7195	0.9281\\
43.7295	0.8141\\
43.7395	0.8247\\
43.7495	0.8736\\
43.7595	0.6364\\
43.7695	0.9002\\
43.7795	0.3071\\
43.7895	0.9370\\
43.7995	0.8048\\
43.8095	0.8424\\
43.8195	0.3478\\
43.8295	0.3074\\
43.8395	0.4254\\
43.8495	0.4434\\
43.8595	0.7836\\
43.8695	0.8565\\
43.8795	0.8769\\
43.8895	0.2818\\
43.8995	0.4180\\
43.9095	0.4933\\
43.9195	0.6262\\
43.9295	0.7509\\
43.9395	0.7960\\
43.9495	0.8369\\
43.9595	0.3544\\
43.9695	0.1552\\
43.9795	0.1969\\
43.9895	0.3009\\
43.9995	0.3678\\
44.0095	0.4447\\
44.0195	0.5364\\
44.0295	0.7384\\
44.0395	0.5111\\
44.0495	0.4187\\
44.0595	0.5633\\
44.0695	0.2593\\
44.0795	0.3265\\
44.0895	0.3140\\
44.0995	0.4010\\
44.1095	0.5294\\
44.1195	0.6787\\
44.1295	0.4843\\
44.1395	0.4757\\
44.1495	0.4494\\
44.1595	0.3817\\
44.1695	0.0987\\
44.1795	0.4596\\
44.1895	0.4995\\
44.1995	0.3806\\
44.2095	0.4753\\
44.2195	0.4656\\
44.2295	0.7724\\
44.2395	0.8869\\
44.2495	0.6177\\
44.2595	0.7447\\
44.2695	0.7569\\
44.2795	0.7444\\
44.2895	0.7549\\
44.2995	0.6297\\
44.3095	0.6572\\
44.3195	0.6921\\
44.3295	0.1846\\
44.3395	0.2056\\
44.3495	0.1901\\
44.3595	0.1782\\
44.3695	0.3216\\
44.3795	0.4407\\
44.3895	0.4583\\
44.3995	0.6873\\
44.4095	0.3612\\
44.4195	0.5811\\
44.4295	0.7357\\
44.4395	0.8089\\
44.4495	0.9235\\
44.4595	0.9440\\
44.4695	0.9492\\
44.4795	0.9041\\
44.4895	0.9232\\
44.4995	0.9516\\
44.5095	0.8914\\
44.5195	0.9119\\
44.5295	0.9453\\
44.5395	0.9125\\
44.5495	0.8177\\
44.5595	0.8592\\
44.5695	0.8761\\
44.5795	0.8257\\
44.5895	0.8718\\
44.5995	0.8789\\
44.6095	0.8556\\
44.6195	0.7505\\
44.6295	0.7939\\
44.6395	0.8206\\
44.6495	0.6437\\
44.6595	0.7189\\
44.6695	0.6917\\
44.6795	0.7505\\
44.6895	0.8423\\
44.6995	0.8840\\
44.7095	0.8955\\
44.7195	0.9107\\
44.7295	0.9128\\
44.7395	0.9351\\
44.7495	0.9492\\
44.7595	0.9359\\
44.7695	0.9316\\
44.7795	0.9305\\
44.7895	0.8961\\
44.7995	0.9435\\
44.8095	0.9132\\
44.8195	0.9326\\
44.8295	0.9504\\
44.8395	0.9328\\
44.8495	0.5774\\
44.8595	0.5017\\
44.8695	0.6229\\
44.8795	0.6917\\
44.8895	0.7697\\
44.8995	0.7797\\
44.9095	0.2257\\
44.9195	0.4094\\
44.9295	0.6114\\
44.9395	0.6549\\
44.9495	0.6595\\
44.9595	0.7420\\
44.9695	0.7700\\
44.9795	0.7518\\
44.9895	0.8272\\
44.9995	0.8813\\
};
\end{axis}

\begin{axis}[%
width=0.411\fwidth,
height=0.232\fwidth,
at={(0.54\fwidth,0.323\fwidth)},
scale only axis,
xmin=62.0075,
xmax=64.9995,
ymin=-2.0832,
ymax=2.5005,
axis background/.style={fill=white},
title style={font=\labelsize},
xlabel style={font=\labelsize,at={(axis description cs:0.5,\xlabeldist)}},
ylabel style={font=\labelsize,at={(axis description cs:\ylabeldist,0.5)}},
legend style={font=\ticksize},
ticklabel style={font=\ticksize}
]
\addplot [color=mycolor1,solid,forget plot]
  table[row sep=crcr]{%
62.0075	-0.2367\\
62.0175	-0.1953\\
62.0280	-0.1547\\
62.0381	-0.1148\\
62.0489	-0.0739\\
62.0592	-0.0316\\
62.0698	0.0121\\
62.0802	0.0583\\
62.0914	0.1066\\
62.1022	0.1561\\
62.1124	0.2053\\
62.1229	0.2537\\
62.1331	0.3010\\
62.1441	0.3485\\
62.1546	0.3941\\
62.1653	0.4398\\
62.1755	0.4848\\
62.1862	0.5312\\
62.1968	0.5741\\
62.2074	0.6111\\
62.2182	0.6462\\
62.2292	0.6743\\
62.2395	0.6990\\
62.2498	0.7197\\
62.2605	0.7467\\
62.2713	0.7709\\
62.2824	0.7965\\
62.2926	0.8269\\
62.3034	0.8681\\
62.3145	0.9072\\
62.3248	0.9423\\
62.3353	0.9790\\
62.3460	1.0005\\
62.3564	1.0128\\
62.3670	1.0233\\
62.3781	1.0503\\
62.3891	1.0771\\
62.3996	1.1023\\
62.4105	1.1369\\
62.4215	1.1510\\
62.4323	1.1555\\
62.4428	1.1528\\
62.4535	1.1655\\
62.4639	1.1703\\
62.4743	1.1791\\
62.4844	1.1970\\
62.4950	1.2341\\
62.5056	1.2651\\
62.5158	1.2932\\
62.5258	1.3329\\
62.5363	1.3495\\
62.5473	1.3551\\
62.5577	1.3594\\
62.5688	1.3844\\
62.5797	1.4057\\
62.5907	1.4237\\
62.6017	1.4582\\
62.6124	1.4656\\
62.6228	1.4628\\
62.6331	1.4528\\
62.6438	1.4617\\
62.6539	1.4578\\
62.6646	1.4571\\
62.6748	1.4560\\
62.6854	1.4696\\
62.6957	1.4595\\
62.7064	1.4465\\
62.7166	1.4317\\
62.7273	1.4332\\
62.7381	1.4257\\
62.7488	1.4185\\
62.7595	1.4149\\
62.7702	1.4321\\
62.7812	1.4380\\
62.7922	1.4410\\
62.8031	1.4619\\
62.8136	1.4605\\
62.8238	1.4521\\
62.8347	1.4507\\
62.8458	1.4327\\
62.8568	1.4101\\
62.8674	1.3903\\
62.8785	1.3897\\
62.8890	1.3775\\
62.8993	1.3733\\
62.9094	1.3803\\
62.9200	1.4121\\
62.9311	1.4371\\
62.9413	1.4619\\
62.9515	1.4928\\
62.9625	1.5003\\
62.9733	1.4913\\
62.9839	1.4763\\
62.9948	1.4851\\
63.0054	1.4876\\
63.0166	1.4835\\
63.0270	1.4938\\
63.0374	1.4721\\
63.0480	1.4319\\
63.0589	1.3839\\
63.0693	1.3430\\
63.0803	1.2920\\
63.0913	1.2315\\
63.1018	1.1689\\
63.1124	1.0806\\
63.1229	0.9753\\
63.1333	0.8560\\
63.1440	0.7394\\
63.1548	0.6194\\
63.1654	0.4998\\
63.1755	0.3762\\
63.1862	0.2465\\
63.1966	0.1118\\
63.2075	-0.0211\\
63.2184	-0.1523\\
63.2290	-0.2750\\
63.2395	-0.3904\\
63.2500	-0.4989\\
63.2607	-0.6052\\
63.2713	-0.7122\\
63.2822	-0.8204\\
63.2923	-0.9393\\
63.3028	-1.0534\\
63.3136	-1.1862\\
63.3247	-1.3174\\
63.3358	-1.4291\\
63.3469	-1.5451\\
63.3577	-1.6368\\
63.3678	-1.6843\\
63.3784	-1.7351\\
63.3893	-1.7802\\
63.3999	-1.8047\\
63.4104	-1.8044\\
63.4207	-1.8316\\
63.4311	-1.8656\\
63.4419	-1.8953\\
63.4529	-1.9124\\
63.4635	-1.9671\\
63.4744	-2.0065\\
63.4851	-2.0013\\
63.4956	-2.0090\\
63.5062	-1.9954\\
63.5165	-1.9701\\
63.5271	-1.9147\\
63.5380	-1.8947\\
63.5489	-1.8778\\
63.5590	-1.8686\\
63.5695	-1.8447\\
63.5803	-1.8504\\
63.5909	-1.8528\\
63.6017	-1.8203\\
63.6125	-1.7983\\
63.6233	-1.7672\\
63.6336	-1.7261\\
63.6446	-1.6697\\
63.6549	-1.6419\\
63.6654	-1.6192\\
63.6756	-1.6096\\
63.6858	-1.5837\\
63.6966	-1.5856\\
63.7078	-1.5811\\
63.7182	-1.5534\\
63.7291	-1.5312\\
63.7400	-1.4997\\
63.7505	-1.4590\\
63.7614	-1.4072\\
63.7721	-1.3758\\
63.7824	-1.3501\\
63.7925	-1.3321\\
63.8037	-1.3040\\
63.8144	-1.2948\\
63.8247	-1.2798\\
63.8351	-1.2500\\
63.8456	-1.2261\\
63.8558	-1.1935\\
63.8665	-1.1567\\
63.8771	-1.1141\\
63.8880	-1.0855\\
63.8989	-1.0616\\
63.9090	-1.0432\\
63.9194	-1.0207\\
63.9302	-1.0099\\
63.9413	-0.9949\\
63.9516	-0.9680\\
63.9625	-0.9427\\
63.9733	-0.9096\\
63.9841	-0.8728\\
63.9948	-0.8332\\
64.0057	-0.8031\\
64.0163	-0.7702\\
64.0270	-0.7318\\
64.0376	-0.6945\\
64.0488	-0.6522\\
64.0591	-0.6058\\
64.0693	-0.5568\\
64.0801	-0.5104\\
64.0913	-0.4609\\
64.1020	-0.4078\\
64.1123	-0.3543\\
64.1232	-0.2987\\
64.1336	-0.2391\\
64.1446	-0.1780\\
64.1554	-0.1158\\
64.1664	-0.0541\\
64.1770	0.0067\\
64.1878	0.0655\\
64.1987	0.1221\\
64.2094	0.1822\\
64.2201	0.2489\\
64.2309	0.3211\\
64.2414	0.3904\\
64.2517	0.4609\\
64.2625	0.5206\\
64.2730	0.5725\\
64.2839	0.6186\\
64.2948	0.6726\\
64.3051	0.7242\\
64.3153	0.7793\\
64.3259	0.8402\\
64.3368	0.9144\\
64.3477	0.9852\\
64.3579	1.0496\\
64.3681	1.1171\\
64.3791	1.1626\\
64.3891	1.1975\\
64.3996	1.2220\\
64.4104	1.2600\\
64.4211	1.2910\\
64.4322	1.3238\\
64.4424	1.3649\\
64.4533	1.4294\\
64.4645	1.4879\\
64.4747	1.5477\\
64.4855	1.6247\\
64.4966	1.6748\\
64.5075	1.7078\\
64.5184	1.7470\\
64.5291	1.7579\\
64.5402	1.7595\\
64.5505	1.7620\\
64.5615	1.7929\\
64.5719	1.8015\\
64.5822	1.8156\\
64.5924	1.8390\\
64.6032	1.8951\\
64.6134	1.9308\\
64.6238	1.9627\\
64.6349	2.0200\\
64.6456	2.0373\\
64.6558	2.0395\\
64.6663	2.0369\\
64.6770	2.0691\\
64.6873	2.0796\\
64.6975	2.0840\\
64.7080	2.0974\\
64.7187	2.1602\\
64.7292	2.1679\\
64.7400	2.1839\\
64.7501	2.1880\\
64.7611	2.2408\\
64.7718	2.2498\\
64.7823	2.2641\\
64.7925	2.2731\\
64.8036	2.3436\\
64.8143	2.3788\\
64.8253	2.4017\\
64.8363	2.4751\\
64.8467	2.4885\\
64.8577	2.4689\\
64.8677	2.4588\\
64.8780	2.5005\\
64.8890	2.4974\\
64.8999	2.4715\\
64.9103	2.4747\\
64.9214	2.4049\\
64.9322	2.3082\\
64.9427	2.1835\\
64.9536	2.0767\\
64.9638	1.9341\\
64.9748	1.7711\\
64.9849	1.6207\\
64.9951	1.4354\\
};
\addplot [color=mycolor2,solid,forget plot]
  table[row sep=crcr]{%
62.0095	-0.4111\\
62.0195	-0.3839\\
62.0295	-0.3659\\
62.0395	-0.3420\\
62.0495	-0.2970\\
62.0595	-0.2317\\
62.0695	-0.1738\\
62.0795	-0.1571\\
62.0895	-0.1283\\
62.0995	-0.0901\\
62.1095	-0.0694\\
62.1195	-0.0490\\
62.1295	-0.0373\\
62.1395	-0.0129\\
62.1495	-0.0030\\
62.1595	0.0022\\
62.1695	0.0113\\
62.1795	0.0293\\
62.1895	0.1204\\
62.1995	0.2297\\
62.2095	0.3292\\
62.2195	0.3863\\
62.2295	0.4331\\
62.2395	0.4636\\
62.2495	0.5317\\
62.2595	0.5743\\
62.2695	0.6128\\
62.2795	0.6167\\
62.2895	0.5978\\
62.2995	0.5967\\
62.3095	0.6382\\
62.3195	0.7665\\
62.3295	0.8148\\
62.3395	0.8449\\
62.3495	0.8548\\
62.3595	0.8952\\
62.3695	0.9205\\
62.3795	0.9541\\
62.3895	0.9643\\
62.3995	1.0116\\
62.4095	1.0400\\
62.4195	1.1346\\
62.4295	1.1369\\
62.4395	1.0953\\
62.4495	1.0406\\
62.4595	1.0324\\
62.4695	1.0515\\
62.4795	1.0750\\
62.4895	1.1084\\
62.4995	1.1397\\
62.5095	1.1788\\
62.5195	1.1745\\
62.5295	1.1980\\
62.5395	1.1807\\
62.5495	1.1541\\
62.5595	1.1826\\
62.5695	1.2160\\
62.5795	1.2294\\
62.5895	1.2756\\
62.5995	1.3003\\
62.6095	1.3340\\
62.6195	1.3083\\
62.6295	1.2851\\
62.6395	1.2424\\
62.6495	1.2460\\
62.6595	1.2705\\
62.6695	1.3189\\
62.6795	1.3474\\
62.6895	1.3771\\
62.6995	1.3598\\
62.7095	1.3195\\
62.7195	1.3691\\
62.7295	1.3866\\
62.7395	1.3361\\
62.7495	1.3278\\
62.7595	1.2918\\
62.7695	1.2930\\
62.7795	1.2854\\
62.7895	1.3049\\
62.7995	1.3010\\
62.8095	1.3072\\
62.8195	1.3013\\
62.8295	1.2995\\
62.8395	1.2832\\
62.8495	1.2862\\
62.8595	1.3069\\
62.8695	1.3198\\
62.8795	1.3532\\
62.8895	1.2774\\
62.8995	1.1881\\
62.9095	1.1566\\
62.9195	1.1117\\
62.9295	1.1147\\
62.9395	1.1512\\
62.9495	1.1187\\
62.9595	1.1099\\
62.9695	1.1345\\
62.9795	1.1836\\
62.9895	1.1786\\
62.9995	1.1676\\
63.0095	1.3149\\
63.0195	1.4140\\
63.0295	1.5118\\
63.0395	1.6240\\
63.0495	1.5159\\
63.0595	1.4297\\
63.0695	1.2909\\
63.0795	1.2188\\
63.0895	1.1592\\
63.0995	1.1090\\
63.1095	1.0810\\
63.1195	0.9876\\
63.1295	0.9174\\
63.1395	0.8144\\
63.1495	0.7148\\
63.1595	0.6362\\
63.1695	0.5584\\
63.1795	0.4998\\
63.1895	0.4224\\
63.1995	0.3651\\
63.2095	0.3073\\
63.2195	0.2518\\
63.2295	0.1958\\
63.2395	0.1448\\
63.2495	0.0777\\
63.2595	-0.0090\\
63.2695	-0.1315\\
63.2795	-0.3155\\
63.2895	-0.4784\\
63.2995	-0.6275\\
63.3095	-0.8327\\
63.3195	-0.8144\\
63.3295	-0.8375\\
63.3395	-1.0036\\
63.3495	-1.1671\\
63.3595	-1.2742\\
63.3695	-1.3815\\
63.3795	-1.4758\\
63.3895	-1.5167\\
63.3995	-1.6254\\
63.4095	-1.6304\\
63.4195	-1.7463\\
63.4295	-1.7548\\
63.4395	-1.7662\\
63.4495	-1.7860\\
63.4595	-1.7749\\
63.4695	-1.7457\\
63.4795	-1.7669\\
63.4895	-1.7881\\
63.4995	-1.8883\\
63.5095	-1.9842\\
63.5195	-1.9895\\
63.5295	-2.0329\\
63.5395	-2.0832\\
63.5495	-2.0127\\
63.5595	-1.9721\\
63.5695	-1.8764\\
63.5795	-1.7953\\
63.5895	-1.7466\\
63.5995	-1.7103\\
63.6095	-1.7623\\
63.6195	-1.7919\\
63.6295	-1.8224\\
63.6395	-1.8062\\
63.6495	-1.7848\\
63.6595	-1.7317\\
63.6695	-1.6792\\
63.6795	-1.6960\\
63.6895	-1.7082\\
63.6995	-1.6405\\
63.7095	-1.5954\\
63.7195	-1.5112\\
63.7295	-1.5236\\
63.7395	-1.5248\\
63.7495	-1.5074\\
63.7595	-1.4741\\
63.7695	-1.4512\\
63.7795	-1.3962\\
63.7895	-1.3359\\
63.7995	-1.2838\\
63.8095	-1.2800\\
63.8195	-1.2800\\
63.8295	-1.2732\\
63.8395	-1.2313\\
63.8495	-1.2495\\
63.8595	-1.2321\\
63.8695	-1.1742\\
63.8795	-1.1300\\
63.8895	-1.1306\\
63.8995	-1.1087\\
63.9095	-1.0933\\
63.9195	-1.1134\\
63.9295	-1.0892\\
63.9395	-1.0481\\
63.9495	-0.9631\\
63.9595	-0.9205\\
63.9695	-0.8667\\
63.9795	-0.8745\\
63.9895	-0.8887\\
63.9995	-0.8905\\
64.0095	-0.9191\\
64.0195	-0.9011\\
64.0295	-0.8769\\
64.0395	-0.8489\\
64.0495	-0.8233\\
64.0595	-0.7849\\
64.0695	-0.7295\\
64.0795	-0.6613\\
64.0895	-0.6130\\
64.0995	-0.5680\\
64.1095	-0.5543\\
64.1195	-0.5115\\
64.1295	-0.4531\\
64.1395	-0.3853\\
64.1495	-0.3404\\
64.1595	-0.3008\\
64.1695	-0.2755\\
64.1795	-0.2366\\
64.1895	-0.2084\\
64.1995	-0.1776\\
64.2095	-0.0701\\
64.2195	-0.0187\\
64.2295	-0.0026\\
64.2395	0.0046\\
64.2495	0.0151\\
64.2595	0.0259\\
64.2695	0.0392\\
64.2795	0.0581\\
64.2895	0.1404\\
64.2995	0.2957\\
64.3095	0.4548\\
64.3195	0.6285\\
64.3295	0.7044\\
64.3395	0.7711\\
64.3495	0.7954\\
64.3595	0.7384\\
64.3695	0.7952\\
64.3795	0.8075\\
64.3895	0.8462\\
64.3995	0.8602\\
64.4095	0.8699\\
64.4195	0.8783\\
64.4295	0.8973\\
64.4395	0.9737\\
64.4495	1.0309\\
64.4595	1.0704\\
64.4695	1.0853\\
64.4795	1.1402\\
64.4895	1.2389\\
64.4995	1.3437\\
64.5095	1.5166\\
64.5195	1.6172\\
64.5295	1.7073\\
64.5395	1.6216\\
64.5495	1.5570\\
64.5595	1.5653\\
64.5695	1.5990\\
64.5795	1.5966\\
64.5895	1.5665\\
64.5995	1.6192\\
64.6095	1.6679\\
64.6195	1.7577\\
64.6295	1.7593\\
64.6395	1.7002\\
64.6495	1.7329\\
64.6595	1.7909\\
64.6695	1.8718\\
64.6795	1.8940\\
64.6895	1.9465\\
64.6995	2.0324\\
64.7095	2.0585\\
64.7195	2.0176\\
64.7295	1.9864\\
64.7395	1.9512\\
64.7495	1.9172\\
64.7595	1.9557\\
64.7695	1.9249\\
64.7795	1.9667\\
64.7895	1.9649\\
64.7995	1.9217\\
64.8095	1.9386\\
64.8195	1.9315\\
64.8295	2.0412\\
64.8395	2.1093\\
64.8495	2.1269\\
64.8595	2.1048\\
64.8695	2.0662\\
64.8795	2.0024\\
64.8895	2.0247\\
64.8995	1.9939\\
64.9095	1.9551\\
64.9195	1.8865\\
64.9295	1.9376\\
64.9395	2.0271\\
64.9495	1.9879\\
64.9595	2.0493\\
64.9695	2.0263\\
64.9795	2.1002\\
64.9895	2.1535\\
64.9995	1.9717\\
};
\end{axis}

\begin{axis}[%
width=0.411\fwidth,
height=0.232\fwidth,
at={(0.54\fwidth,0\fwidth)},
scale only axis,
xmin=62.0095,
xmax=64.9995,
xlabel={$t$ [s]},
ymin=0.0000,
ymax=1.0000,
axis background/.style={fill=white},
title style={font=\labelsize},
xlabel style={font=\labelsize,at={(axis description cs:0.5,\xlabeldist)}},
ylabel style={font=\labelsize,at={(axis description cs:\ylabeldist,0.5)}},
legend style={font=\ticksize},
ticklabel style={font=\ticksize}
]
\addplot [color=mycolor1,solid,forget plot]
  table[row sep=crcr]{%
62.0095	0.1905\\
62.0195	0.2871\\
62.0295	0.1723\\
62.0395	0.2294\\
62.0495	0.4762\\
62.0595	0.5714\\
62.0695	0.7211\\
62.0795	0.8182\\
62.0895	0.7393\\
62.0995	0.7461\\
62.1095	0.7616\\
62.1195	0.8401\\
62.1295	0.1692\\
62.1395	0.1460\\
62.1495	0.0812\\
62.1595	0.0642\\
62.1695	0.1218\\
62.1795	0.1641\\
62.1895	0.5768\\
62.1995	0.6644\\
62.2095	0.8045\\
62.2195	0.8264\\
62.2295	0.8881\\
62.2395	0.8780\\
62.2495	0.8677\\
62.2595	0.9138\\
62.2695	0.9190\\
62.2795	0.9042\\
62.2895	0.9056\\
62.2995	0.8642\\
62.3095	0.9175\\
62.3195	0.9066\\
62.3295	0.9224\\
62.3395	0.9347\\
62.3495	0.8027\\
62.3595	0.9344\\
62.3695	0.9476\\
62.3795	0.9467\\
62.3895	0.9551\\
62.3995	0.9502\\
62.4095	0.9543\\
62.4195	0.9500\\
62.4295	0.9551\\
62.4395	0.9515\\
62.4495	0.9606\\
62.4595	0.9615\\
62.4695	0.9617\\
62.4795	0.9633\\
62.4895	0.9613\\
62.4995	0.9505\\
62.5095	0.8966\\
62.5195	0.9211\\
62.5295	0.9402\\
62.5395	0.9553\\
62.5495	0.9434\\
62.5595	0.9414\\
62.5695	0.9362\\
62.5795	0.9520\\
62.5895	0.9560\\
62.5995	0.9371\\
62.6095	0.9544\\
62.6195	0.9389\\
62.6295	0.9588\\
62.6395	0.9614\\
62.6495	0.9628\\
62.6595	0.9642\\
62.6695	0.9544\\
62.6795	0.9410\\
62.6895	0.9482\\
62.6995	0.9349\\
62.7095	0.9393\\
62.7195	0.9592\\
62.7295	0.9313\\
62.7395	0.9440\\
62.7495	0.9563\\
62.7595	0.9623\\
62.7695	0.9522\\
62.7795	0.9506\\
62.7895	0.9600\\
62.7995	0.9691\\
62.8095	0.9766\\
62.8195	0.9739\\
62.8295	0.9668\\
62.8395	0.9565\\
62.8495	0.9551\\
62.8595	0.9631\\
62.8695	0.9364\\
62.8795	0.9125\\
62.8895	0.7882\\
62.8995	0.7296\\
62.9095	0.6687\\
62.9195	0.7641\\
62.9295	0.8462\\
62.9395	0.8471\\
62.9495	0.8506\\
62.9595	0.8088\\
62.9695	0.8347\\
62.9795	0.8375\\
62.9895	0.8386\\
62.9995	0.8723\\
63.0095	0.8280\\
63.0195	0.8189\\
63.0295	0.7179\\
63.0395	0.6182\\
63.0495	0.8499\\
63.0595	0.8792\\
63.0695	0.8358\\
63.0795	0.8288\\
63.0895	0.8582\\
63.0995	0.8778\\
63.1095	0.7322\\
63.1195	0.7580\\
63.1295	0.7745\\
63.1395	0.8133\\
63.1495	0.8711\\
63.1595	0.8908\\
63.1695	0.8712\\
63.1795	0.8193\\
63.1895	0.7636\\
63.1995	0.7109\\
63.2095	0.7145\\
63.2195	0.7877\\
63.2295	0.8647\\
63.2395	0.8711\\
63.2495	0.8622\\
63.2595	0.4471\\
63.2695	0.5177\\
63.2795	0.5844\\
63.2895	0.6358\\
63.2995	0.7552\\
63.3095	0.7333\\
63.3195	0.4322\\
63.3295	0.4634\\
63.3395	0.8240\\
63.3495	0.9150\\
63.3595	0.9371\\
63.3695	0.9557\\
63.3795	0.9596\\
63.3895	0.8386\\
63.3995	0.9618\\
63.4095	0.6737\\
63.4195	0.8522\\
63.4295	0.5957\\
63.4395	0.4259\\
63.4495	0.7558\\
63.4595	0.8917\\
63.4695	0.9041\\
63.4795	0.9224\\
63.4895	0.9057\\
63.4995	0.9188\\
63.5095	0.6455\\
63.5195	0.7368\\
63.5295	0.8617\\
63.5395	0.9080\\
63.5495	0.9449\\
63.5595	0.9598\\
63.5695	0.9636\\
63.5795	0.9647\\
63.5895	0.9713\\
63.5995	0.9627\\
63.6095	0.9489\\
63.6195	0.9647\\
63.6295	0.9538\\
63.6395	0.9425\\
63.6495	0.9548\\
63.6595	0.9410\\
63.6695	0.9389\\
63.6795	0.9049\\
63.6895	0.9099\\
63.6995	0.9187\\
63.7095	0.9133\\
63.7195	0.8052\\
63.7295	0.9172\\
63.7395	0.9121\\
63.7495	0.9358\\
63.7595	0.9384\\
63.7695	0.9293\\
63.7795	0.9462\\
63.7895	0.9564\\
63.7995	0.9647\\
63.8095	0.9696\\
63.8195	0.9680\\
63.8295	0.9432\\
63.8395	0.9383\\
63.8495	0.8818\\
63.8595	0.8771\\
63.8695	0.8955\\
63.8795	0.9239\\
63.8895	0.9122\\
63.8995	0.8994\\
63.9095	0.9065\\
63.9195	0.9480\\
63.9295	0.9504\\
63.9395	0.9015\\
63.9495	0.8343\\
63.9595	0.8707\\
63.9695	0.8742\\
63.9795	0.8921\\
63.9895	0.8712\\
63.9995	0.8851\\
64.0095	0.9031\\
64.0195	0.8607\\
64.0295	0.8310\\
64.0395	0.8097\\
64.0495	0.7127\\
64.0595	0.1921\\
64.0695	0.2689\\
64.0795	0.4004\\
64.0895	0.5329\\
64.0995	0.4516\\
64.1095	0.1018\\
64.1195	0.2362\\
64.1295	0.3284\\
64.1395	0.4978\\
64.1495	0.6233\\
64.1595	0.6457\\
64.1695	0.6822\\
64.1795	0.6787\\
64.1895	0.7285\\
64.1995	0.7931\\
64.2095	0.7418\\
64.2195	0.7239\\
64.2295	0.1850\\
64.2395	0.1045\\
64.2495	0.1066\\
64.2595	0.1106\\
64.2695	0.1089\\
64.2795	0.1005\\
64.2895	0.2938\\
64.2995	0.5336\\
64.3095	0.6712\\
64.3195	0.7576\\
64.3295	0.8043\\
64.3395	0.8850\\
64.3495	0.8945\\
64.3595	0.8580\\
64.3695	0.8755\\
64.3795	0.8388\\
64.3895	0.7063\\
64.3995	0.6074\\
64.4095	0.7133\\
64.4195	0.7541\\
64.4295	0.8159\\
64.4395	0.8335\\
64.4495	0.7784\\
64.4595	0.7592\\
64.4695	0.7745\\
64.4795	0.8254\\
64.4895	0.8553\\
64.4995	0.9152\\
64.5095	0.9132\\
64.5195	0.9224\\
64.5295	0.9444\\
64.5395	0.9391\\
64.5495	0.9294\\
64.5595	0.9451\\
64.5695	0.9281\\
64.5795	0.9357\\
64.5895	0.9257\\
64.5995	0.9218\\
64.6095	0.9314\\
64.6195	0.8877\\
64.6295	0.9036\\
64.6395	0.9170\\
64.6495	0.9142\\
64.6595	0.8460\\
64.6695	0.9009\\
64.6795	0.8948\\
64.6895	0.9380\\
64.6995	0.9510\\
64.7095	0.9435\\
64.7195	0.9575\\
64.7295	0.9462\\
64.7395	0.9409\\
64.7495	0.9160\\
64.7595	0.9233\\
64.7695	0.8904\\
64.7795	0.9270\\
64.7895	0.8415\\
64.7995	0.8712\\
64.8095	0.8706\\
64.8195	0.8619\\
64.8295	0.5912\\
64.8395	0.6707\\
64.8495	0.7924\\
64.8595	0.8007\\
64.8695	0.8162\\
64.8795	0.8410\\
64.8895	0.7757\\
64.8995	0.8023\\
64.9095	0.8389\\
64.9195	0.8732\\
64.9295	0.8893\\
64.9395	0.9217\\
64.9495	0.9169\\
64.9595	0.9113\\
64.9695	0.8812\\
64.9795	0.9371\\
64.9895	0.9235\\
64.9995	0.8306\\
};
\end{axis}
\end{tikzpicture}%
			\label{fig:divergence_checkerboard}
		}
		\subfloat[Roadmap texture]	{
			% This file was created by matlab2tikz.
%
%The latest updates can be retrieved from
%  http://www.mathworks.com/matlabcentral/fileexchange/22022-matlab2tikz-matlab2tikz
%where you can also make suggestions and rate matlab2tikz.
%
\definecolor{mycolor1}{rgb}{0.00000,0.44700,0.74100}%
\definecolor{mycolor2}{rgb}{0.85000,0.32500,0.09800}%
%
\begin{tikzpicture}

\begin{axis}[%
width=0.951\fwidth,
height=0.189\fwidth,
at={(0\fwidth,1.011\fwidth)},
scale only axis,
xmin=34.0042,
xmax=62.7984,
xlabel={$t$ [s]},
ymin=0.0000,
ymax=1.3040,
ylabel={$h$ [m]},
axis background/.style={fill=white},
title style={font=\labelsize},
xlabel style={font=\labelsize,at={(axis description cs:0.5,\xlabeldist)}},
ylabel style={font=\labelsize,at={(axis description cs:\ylabeldist,0.5)}},
legend style={font=\ticksize},
ticklabel style={font=\ticksize}
]
\addplot [color=mycolor1,solid,forget plot]
  table[row sep=crcr]{%
34.0042	0.8080\\
34.0261	0.8080\\
34.0576	0.8090\\
34.0898	0.8090\\
34.1115	0.8090\\
34.1431	0.8080\\
34.1742	0.8080\\
34.2054	0.8080\\
34.2262	0.8080\\
34.2574	0.8090\\
34.2891	0.8100\\
34.3096	0.8100\\
34.3412	0.8090\\
34.3733	0.8090\\
34.4048	0.8090\\
34.4263	0.8090\\
34.4578	0.8080\\
34.4901	0.8070\\
34.5116	0.8060\\
34.5432	0.8050\\
34.5746	0.8020\\
34.6068	0.8000\\
34.6273	0.7990\\
34.6589	0.7980\\
34.6911	0.7950\\
34.7225	0.7930\\
34.7435	0.7920\\
34.7746	0.7900\\
34.8061	0.7870\\
34.8277	0.7860\\
34.8588	0.7840\\
34.8904	0.7820\\
34.9225	0.7800\\
34.9433	0.7780\\
34.9746	0.7760\\
35.0061	0.7740\\
35.0274	0.7730\\
35.0589	0.7710\\
35.0912	0.7690\\
35.1230	0.7670\\
35.1444	0.7660\\
35.1766	0.7640\\
35.2084	0.7620\\
35.2403	0.7600\\
35.2611	0.7590\\
35.2923	0.7570\\
35.3236	0.7540\\
35.3442	0.7530\\
35.3766	0.7510\\
35.4085	0.7480\\
35.4414	0.7460\\
35.4620	0.7450\\
35.4934	0.7420\\
35.5250	0.7390\\
35.5453	0.7380\\
35.5770	0.7350\\
35.6081	0.7320\\
35.6403	0.7290\\
35.6611	0.7270\\
35.6936	0.7250\\
35.7250	0.7220\\
35.7573	0.7180\\
35.7780	0.7170\\
35.8102	0.7140\\
35.8416	0.7100\\
35.8624	0.7080\\
35.8937	0.7060\\
35.9249	0.7020\\
35.9562	0.6980\\
35.9768	0.6970\\
36.0082	0.6950\\
36.0396	0.6910\\
36.0604	0.6900\\
36.0918	0.6880\\
36.1231	0.6850\\
36.1541	0.6830\\
36.1750	0.6810\\
36.2062	0.6780\\
36.2372	0.6760\\
36.2688	0.6740\\
36.2897	0.6720\\
36.3210	0.6700\\
36.3519	0.6670\\
36.3734	0.6650\\
36.4067	0.6610\\
36.4387	0.6580\\
36.4697	0.6550\\
36.4909	0.6530\\
36.5222	0.6490\\
36.5544	0.6480\\
36.5754	0.6460\\
36.6076	0.6440\\
36.6397	0.6420\\
36.6709	0.6400\\
36.6920	0.6380\\
36.7242	0.6360\\
36.7555	0.6330\\
36.7868	0.6320\\
36.8081	0.6310\\
36.8401	0.6280\\
36.8723	0.6260\\
36.8930	0.6250\\
36.9243	0.6230\\
36.9559	0.6220\\
36.9877	0.6210\\
37.0089	0.6200\\
37.0401	0.6180\\
37.0724	0.6170\\
37.0941	0.6160\\
37.1253	0.6140\\
37.1570	0.6140\\
37.1880	0.6130\\
37.2090	0.6130\\
37.2402	0.6120\\
37.2715	0.6120\\
37.2922	0.6120\\
37.3247	0.6130\\
37.3566	0.6140\\
37.3880	0.6150\\
37.4088	0.6170\\
37.4402	0.6180\\
37.4717	0.6200\\
37.5034	0.6230\\
37.5246	0.6250\\
37.5560	0.6290\\
37.5871	0.6320\\
37.6080	0.6340\\
37.6396	0.6370\\
37.6713	0.6400\\
37.7028	0.6440\\
37.7241	0.6460\\
37.7560	0.6500\\
37.7871	0.6520\\
37.8082	0.6540\\
37.8395	0.6560\\
37.8708	0.6590\\
37.9025	0.6620\\
37.9237	0.6640\\
37.9549	0.6670\\
37.9870	0.6690\\
38.0194	0.6720\\
38.0407	0.6730\\
38.0731	0.6770\\
38.1048	0.6790\\
38.1258	0.6810\\
38.1562	0.6830\\
38.1882	0.6850\\
38.2198	0.6880\\
38.2407	0.6900\\
38.2730	0.6930\\
38.3039	0.6950\\
38.3249	0.6970\\
38.3563	0.6990\\
38.3883	0.7010\\
38.4201	0.7050\\
38.4421	0.7070\\
38.4743	0.7090\\
38.5063	0.7120\\
38.5372	0.7140\\
38.5585	0.7160\\
38.5907	0.7170\\
38.6218	0.7190\\
38.6426	0.7200\\
38.6740	0.7230\\
38.7051	0.7240\\
38.7373	0.7260\\
38.7588	0.7280\\
38.7908	0.7300\\
38.8217	0.7310\\
38.8429	0.7320\\
38.8744	0.7340\\
38.9053	0.7350\\
38.9374	0.7370\\
38.9585	0.7380\\
38.9902	0.7400\\
39.0222	0.7420\\
39.0541	0.7440\\
39.0752	0.7450\\
39.1065	0.7470\\
39.1384	0.7480\\
39.1597	0.7500\\
39.1920	0.7520\\
39.2233	0.7530\\
39.2551	0.7540\\
39.2756	0.7560\\
39.3078	0.7570\\
39.3400	0.7590\\
39.3619	0.7600\\
39.3932	0.7620\\
39.4255	0.7630\\
39.4581	0.7660\\
39.4794	0.7670\\
39.5108	0.7690\\
39.5424	0.7710\\
39.5745	0.7720\\
39.5951	0.7730\\
39.6264	0.7760\\
39.6588	0.7780\\
39.6793	0.7790\\
39.7119	0.7810\\
39.7442	0.7830\\
39.7758	0.7840\\
39.7970	0.7850\\
39.8297	0.7870\\
39.8620	0.7880\\
39.8830	0.7900\\
39.9141	0.7920\\
39.9463	0.7940\\
39.9787	0.7960\\
39.9998	0.7980\\
40.0307	0.7990\\
40.0632	0.8010\\
40.0953	0.8030\\
40.1163	0.8050\\
40.1484	0.8060\\
40.1807	0.8080\\
40.2021	0.8100\\
40.2349	0.8130\\
40.2665	0.8150\\
40.2985	0.8180\\
40.3205	0.8190\\
40.3527	0.8210\\
40.3837	0.8230\\
40.4047	0.8240\\
40.4367	0.8260\\
40.4685	0.8280\\
40.5002	0.8310\\
40.5215	0.8330\\
40.5540	0.8360\\
40.5859	0.8380\\
40.6075	0.8400\\
40.6393	0.8410\\
40.6713	0.8440\\
40.7028	0.8470\\
40.7243	0.8490\\
40.7557	0.8510\\
40.7881	0.8540\\
40.8197	0.8570\\
40.8407	0.8590\\
40.8728	0.8610\\
40.9059	0.8630\\
40.9270	0.8650\\
40.9582	0.8670\\
40.9895	0.8680\\
41.0218	0.8700\\
41.0427	0.8710\\
41.0751	0.8730\\
41.1064	0.8740\\
41.1283	0.8760\\
41.1597	0.8780\\
41.1920	0.8800\\
41.2241	0.8820\\
41.2453	0.8830\\
41.2767	0.8850\\
41.3086	0.8870\\
41.3395	0.8880\\
41.3605	0.8900\\
41.3920	0.8920\\
41.4248	0.8940\\
41.4459	0.8950\\
41.4774	0.8960\\
41.5096	0.8980\\
41.5420	0.9000\\
41.5638	0.9010\\
41.5953	0.9030\\
41.6275	0.9050\\
41.6491	0.9060\\
41.6810	0.9080\\
41.7127	0.9090\\
41.7450	0.9110\\
41.7657	0.9140\\
41.7982	0.9160\\
41.8304	0.9190\\
41.8619	0.9210\\
41.8836	0.9240\\
41.9147	0.9260\\
41.9472	0.9290\\
41.9681	0.9310\\
42.0004	0.9330\\
42.0327	0.9360\\
42.0640	0.9380\\
42.0856	0.9390\\
42.1172	0.9410\\
42.1500	0.9430\\
42.1706	0.9440\\
42.2029	0.9460\\
42.2349	0.9480\\
42.2660	0.9500\\
42.2877	0.9520\\
42.3195	0.9540\\
42.3506	0.9560\\
42.3829	0.9580\\
42.4045	0.9590\\
42.4368	0.9610\\
42.4692	0.9620\\
42.4902	0.9620\\
42.5232	0.9620\\
42.5558	0.9610\\
42.5880	0.9600\\
42.6092	0.9580\\
42.6414	0.9550\\
42.6734	0.9530\\
42.6947	0.9500\\
42.7268	0.9470\\
42.7581	0.9430\\
42.7900	0.9410\\
42.8113	0.9390\\
42.8436	0.9360\\
42.8759	0.9340\\
42.9082	0.9320\\
42.9300	0.9310\\
42.9624	0.9300\\
42.9947	0.9270\\
43.0165	0.9260\\
43.0485	0.9240\\
43.0813	0.9220\\
43.1135	0.9200\\
43.1352	0.9180\\
43.1675	0.9160\\
43.1996	0.9130\\
43.2217	0.9120\\
43.2536	0.9090\\
43.2859	0.9080\\
43.3172	0.9050\\
43.3392	0.9030\\
43.3712	0.9010\\
43.4041	0.8990\\
43.4356	0.8960\\
43.4568	0.8950\\
43.4890	0.8930\\
43.5213	0.8900\\
43.5429	0.8880\\
43.5737	0.8870\\
43.6052	0.8840\\
43.6370	0.8820\\
43.6585	0.8810\\
43.6909	0.8790\\
43.7234	0.8770\\
43.7439	0.8750\\
43.7754	0.8730\\
43.8077	0.8710\\
43.8401	0.8680\\
43.8606	0.8670\\
43.8921	0.8640\\
43.9243	0.8620\\
43.9462	0.8600\\
43.9776	0.8560\\
44.0099	0.8530\\
44.0424	0.8500\\
44.0637	0.8490\\
44.0954	0.8450\\
44.1278	0.8420\\
44.1599	0.8390\\
44.1807	0.8370\\
44.2129	0.8340\\
44.2454	0.8310\\
44.2672	0.8290\\
44.2992	0.8270\\
44.3316	0.8250\\
44.3638	0.8210\\
44.3850	0.8190\\
44.4161	0.8170\\
44.4478	0.8140\\
44.4687	0.8120\\
44.5007	0.8090\\
44.5318	0.8060\\
44.5640	0.8030\\
44.5847	0.8010\\
44.6160	0.7980\\
44.6473	0.7950\\
44.6786	0.7920\\
44.7000	0.7900\\
44.7318	0.7880\\
44.7632	0.7840\\
44.7844	0.7820\\
44.8165	0.7790\\
44.8484	0.7770\\
44.8799	0.7730\\
44.9011	0.7700\\
44.9328	0.7670\\
44.9656	0.7650\\
44.9865	0.7620\\
45.0190	0.7590\\
45.0516	0.7560\\
45.0827	0.7530\\
45.1043	0.7510\\
45.1372	0.7480\\
45.1687	0.7430\\
45.2005	0.7400\\
45.2223	0.7370\\
45.2539	0.7340\\
45.2854	0.7310\\
45.3069	0.7290\\
45.3388	0.7250\\
45.3703	0.7210\\
45.4024	0.7180\\
45.4236	0.7160\\
45.4558	0.7130\\
45.4881	0.7100\\
45.5096	0.7070\\
45.5417	0.7040\\
45.5739	0.7000\\
45.6068	0.6970\\
45.6281	0.6940\\
45.6595	0.6910\\
45.6905	0.6890\\
45.7226	0.6850\\
45.7430	0.6830\\
45.7744	0.6810\\
45.8056	0.6770\\
45.8263	0.6750\\
45.8584	0.6720\\
45.8904	0.6700\\
45.9218	0.6670\\
45.9433	0.6640\\
45.9752	0.6620\\
46.0082	0.6580\\
46.0294	0.6550\\
46.0616	0.6520\\
46.0931	0.6490\\
46.1248	0.6470\\
46.1459	0.6440\\
46.1782	0.6420\\
46.2096	0.6390\\
46.2413	0.6380\\
46.2629	0.6370\\
46.2951	0.6360\\
46.3264	0.6360\\
46.3482	0.6360\\
46.3797	0.6370\\
46.4116	0.6390\\
46.4431	0.6410\\
46.4647	0.6440\\
46.4971	0.6480\\
46.5288	0.6520\\
46.5503	0.6550\\
46.5819	0.6600\\
46.6129	0.6640\\
46.6449	0.6710\\
46.6664	0.6730\\
46.6984	0.6780\\
46.7308	0.6830\\
46.7618	0.6880\\
46.7830	0.6910\\
46.8151	0.6960\\
46.8462	0.7010\\
46.8669	0.7030\\
46.8992	0.7080\\
46.9317	0.7130\\
46.9630	0.7180\\
46.9842	0.7210\\
47.0158	0.7250\\
47.0473	0.7300\\
47.0688	0.7330\\
47.1002	0.7360\\
47.1326	0.7400\\
47.1650	0.7440\\
47.1860	0.7470\\
47.2178	0.7500\\
47.2492	0.7540\\
47.2708	0.7570\\
47.3029	0.7610\\
47.3347	0.7660\\
47.3662	0.7690\\
47.3868	0.7720\\
47.4181	0.7760\\
47.4506	0.7810\\
47.4818	0.7840\\
47.5029	0.7870\\
47.5342	0.7920\\
47.5669	0.7950\\
47.5878	0.7980\\
47.6199	0.8040\\
47.6520	0.8080\\
47.6836	0.8130\\
47.7047	0.8160\\
47.7364	0.8210\\
47.7686	0.8270\\
47.7900	0.8290\\
47.8223	0.8330\\
47.8545	0.8390\\
47.8863	0.8420\\
47.9082	0.8450\\
47.9409	0.8480\\
47.9724	0.8520\\
48.0042	0.8560\\
48.0254	0.8580\\
48.0567	0.8630\\
48.0881	0.8680\\
48.1086	0.8700\\
48.1403	0.8740\\
48.1718	0.8790\\
48.2034	0.8840\\
48.2245	0.8870\\
48.2560	0.8910\\
48.2882	0.8960\\
48.3097	0.8990\\
48.3413	0.9020\\
48.3736	0.9070\\
48.4063	0.9100\\
48.4274	0.9140\\
48.4587	0.9160\\
48.4899	0.9200\\
48.5214	0.9240\\
48.5422	0.9260\\
48.5731	0.9290\\
48.6052	0.9320\\
48.6266	0.9350\\
48.6585	0.9370\\
48.6900	0.9410\\
48.7212	0.9440\\
48.7421	0.9450\\
48.7731	0.9490\\
48.8054	0.9520\\
48.8265	0.9550\\
48.8587	0.9570\\
48.8902	0.9610\\
48.9226	0.9650\\
48.9442	0.9690\\
48.9766	0.9720\\
49.0076	0.9750\\
49.0397	0.9800\\
49.0618	0.9830\\
49.0943	0.9870\\
49.1265	0.9920\\
49.1482	0.9940\\
49.1798	0.9980\\
49.2117	1.0030\\
49.2434	1.0080\\
49.2648	1.0100\\
49.2964	1.0160\\
49.3284	1.0210\\
49.3492	1.0240\\
49.3817	1.0280\\
49.4128	1.0330\\
49.4439	1.0380\\
49.4647	1.0400\\
49.4956	1.0450\\
49.5276	1.0490\\
49.5597	1.0530\\
49.5803	1.0550\\
49.6126	1.0590\\
49.6442	1.0630\\
49.6660	1.0660\\
49.6967	1.0710\\
49.7284	1.0750\\
49.7597	1.0800\\
49.7807	1.0830\\
49.8119	1.0880\\
49.8439	1.0920\\
49.8646	1.0940\\
49.8961	1.0990\\
49.9277	1.1030\\
49.9592	1.1070\\
49.9803	1.1100\\
50.0129	1.1150\\
50.0452	1.1190\\
50.0665	1.1210\\
50.0980	1.1240\\
50.1295	1.1280\\
50.1616	1.1320\\
50.1832	1.1340\\
50.2157	1.1370\\
50.2472	1.1410\\
50.2796	1.1440\\
50.3004	1.1460\\
50.3324	1.1500\\
50.3639	1.1520\\
50.3845	1.1540\\
50.4166	1.1550\\
50.4480	1.1570\\
50.4794	1.1570\\
50.4997	1.1570\\
50.5313	1.1540\\
50.5630	1.1520\\
50.5844	1.1490\\
50.6162	1.1430\\
50.6480	1.1390\\
50.6802	1.1310\\
50.7012	1.1260\\
50.7329	1.1200\\
50.7643	1.1110\\
50.7960	1.1020\\
50.8176	1.0960\\
50.8502	1.0880\\
50.8813	1.0790\\
50.9027	1.0720\\
50.9341	1.0620\\
50.9665	1.0550\\
50.9980	1.0460\\
51.0194	1.0400\\
51.0521	1.0310\\
51.0832	1.0250\\
51.1041	1.0180\\
51.1363	1.0100\\
51.1686	1.0030\\
51.2009	0.9950\\
51.2217	0.9910\\
51.2544	0.9840\\
51.2858	0.9780\\
51.3188	0.9710\\
51.3394	0.9680\\
51.3709	0.9630\\
51.4024	0.9570\\
51.4238	0.9540\\
51.4557	0.9480\\
51.4873	0.9420\\
51.5196	0.9380\\
51.5406	0.9350\\
51.5725	0.9290\\
51.6041	0.9250\\
51.6254	0.9230\\
51.6578	0.9170\\
51.6903	0.9120\\
51.7223	0.9060\\
51.7433	0.9010\\
51.7750	0.8970\\
51.8064	0.8910\\
51.8377	0.8840\\
51.8581	0.8810\\
51.8901	0.8740\\
51.9226	0.8680\\
51.9442	0.8630\\
51.9763	0.8560\\
52.0079	0.8510\\
52.0402	0.8440\\
52.0614	0.8390\\
52.0928	0.8320\\
52.1234	0.8270\\
52.1453	0.8210\\
52.1765	0.8150\\
52.2083	0.8100\\
52.2402	0.8030\\
52.2622	0.7990\\
52.2938	0.7920\\
52.3266	0.7850\\
52.3587	0.7810\\
52.3805	0.7760\\
52.4121	0.7690\\
52.4431	0.7620\\
52.4642	0.7590\\
52.4959	0.7520\\
52.5282	0.7460\\
52.5597	0.7400\\
52.5812	0.7370\\
52.6126	0.7310\\
52.6449	0.7240\\
52.6665	0.7210\\
52.6989	0.7160\\
52.7301	0.7100\\
52.7615	0.7060\\
52.7826	0.7020\\
52.8144	0.6990\\
52.8468	0.6940\\
52.8782	0.6910\\
52.8995	0.6880\\
52.9313	0.6840\\
52.9634	0.6790\\
52.9841	0.6760\\
53.0164	0.6730\\
53.0489	0.6700\\
53.0811	0.6670\\
53.1019	0.6660\\
53.1340	0.6650\\
53.1660	0.6660\\
53.1871	0.6680\\
53.2185	0.6720\\
53.2506	0.6760\\
53.2828	0.6810\\
53.3047	0.6860\\
53.3373	0.6950\\
53.3690	0.7040\\
53.3902	0.7080\\
53.4223	0.7180\\
53.4539	0.7280\\
53.4861	0.7380\\
53.5077	0.7420\\
53.5396	0.7520\\
53.5709	0.7610\\
53.6018	0.7710\\
53.6229	0.7760\\
53.6547	0.7860\\
53.6866	0.7950\\
53.7083	0.8010\\
53.7405	0.8090\\
53.7721	0.8200\\
53.8049	0.8290\\
53.8262	0.8370\\
53.8587	0.8430\\
53.8902	0.8530\\
53.9117	0.8600\\
53.9439	0.8690\\
53.9761	0.8770\\
54.0086	0.8850\\
54.0305	0.8910\\
54.0628	0.9000\\
54.0948	0.9080\\
54.1263	0.9160\\
54.1478	0.9200\\
54.1802	0.9280\\
54.2113	0.9360\\
54.2322	0.9400\\
54.2645	0.9480\\
54.2957	0.9560\\
54.3282	0.9640\\
54.3497	0.9680\\
54.3811	0.9750\\
54.4125	0.9840\\
54.4342	0.9900\\
54.4664	0.9960\\
54.4990	1.0050\\
54.5303	1.0140\\
54.5521	1.0200\\
54.5830	1.0270\\
54.6145	1.0360\\
54.6461	1.0450\\
54.6678	1.0510\\
54.6988	1.0580\\
54.7316	1.0660\\
54.7520	1.0720\\
54.7831	1.0780\\
54.8144	1.0860\\
54.8461	1.0930\\
54.8677	1.0990\\
54.8992	1.1050\\
54.9307	1.1130\\
54.9520	1.1190\\
54.9829	1.1250\\
55.0146	1.1330\\
55.0468	1.1400\\
55.0687	1.1440\\
55.0998	1.1480\\
55.1316	1.1530\\
55.1645	1.1550\\
55.1854	1.1550\\
55.2165	1.1540\\
55.2490	1.1520\\
55.2706	1.1490\\
55.3019	1.1410\\
55.3340	1.1340\\
55.3662	1.1260\\
55.3881	1.1190\\
55.4204	1.1070\\
55.4518	1.0940\\
55.4735	1.0880\\
55.5051	1.0730\\
55.5363	1.0600\\
55.5676	1.0450\\
55.5893	1.0380\\
55.6215	1.0230\\
55.6541	1.0080\\
55.6850	0.9940\\
55.7062	0.9870\\
55.7375	0.9730\\
55.7685	0.9600\\
55.7901	0.9540\\
55.8227	0.9410\\
55.8540	0.9290\\
55.8861	0.9170\\
55.9071	0.9100\\
55.9389	0.8990\\
55.9706	0.8870\\
55.9912	0.8810\\
56.0233	0.8710\\
56.0553	0.8600\\
56.0874	0.8510\\
56.1079	0.8470\\
56.1398	0.8380\\
56.1705	0.8290\\
56.2020	0.8200\\
56.2236	0.8150\\
56.2558	0.8050\\
56.2881	0.7950\\
56.3092	0.7870\\
56.3408	0.7790\\
56.3715	0.7700\\
56.4031	0.7590\\
56.4245	0.7540\\
56.4567	0.7440\\
56.4892	0.7350\\
56.5103	0.7270\\
56.5425	0.7180\\
56.5745	0.7110\\
56.6069	0.7020\\
56.6280	0.6960\\
56.6592	0.6880\\
56.6913	0.6810\\
56.7125	0.6760\\
56.7446	0.6690\\
56.7759	0.6610\\
56.8070	0.6570\\
56.8279	0.6530\\
56.8593	0.6490\\
56.8912	0.6470\\
56.9229	0.6450\\
56.9435	0.6460\\
56.9747	0.6480\\
57.0068	0.6520\\
57.0278	0.6570\\
57.0593	0.6660\\
57.0913	0.6730\\
57.1226	0.6850\\
57.1435	0.6960\\
57.1757	0.7090\\
57.2071	0.7220\\
57.2280	0.7330\\
57.2596	0.7480\\
57.2913	0.7610\\
57.3226	0.7780\\
57.3437	0.7910\\
57.3747	0.8050\\
57.4059	0.8210\\
57.4373	0.8400\\
57.4579	0.8470\\
57.4892	0.8660\\
57.5205	0.8840\\
57.5412	0.8920\\
57.5727	0.9100\\
57.6038	0.9280\\
57.6354	0.9460\\
57.6569	0.9560\\
57.6884	0.9740\\
57.7196	0.9930\\
57.7402	1.0030\\
57.7716	1.0220\\
57.8030	1.0410\\
57.8351	1.0580\\
57.8571	1.0680\\
57.8887	1.0850\\
57.9206	1.1030\\
57.9531	1.1190\\
57.9746	1.1280\\
58.0059	1.1430\\
58.0375	1.1580\\
58.0581	1.1660\\
58.0893	1.1800\\
58.1206	1.1930\\
58.1520	1.2070\\
58.1735	1.2130\\
58.2051	1.2270\\
58.2373	1.2410\\
58.2581	1.2480\\
58.2893	1.2620\\
58.3206	1.2740\\
58.3518	1.2870\\
58.3727	1.2920\\
58.4039	1.2990\\
58.4351	1.3030\\
58.4663	1.3040\\
58.4873	1.3030\\
58.5186	1.2990\\
58.5496	1.2920\\
58.5705	1.2850\\
58.6020	1.2720\\
58.6328	1.2610\\
58.6641	1.2450\\
58.6854	1.2310\\
58.7175	1.2110\\
58.7494	1.1940\\
58.7706	1.1790\\
58.8020	1.1560\\
58.8343	1.1320\\
58.8661	1.1150\\
58.8874	1.0980\\
58.9195	1.0750\\
58.9520	1.0530\\
58.9829	1.0370\\
59.0037	1.0210\\
59.0352	0.9990\\
59.0665	0.9830\\
59.0872	0.9670\\
59.1184	0.9460\\
59.1495	0.9290\\
59.1817	0.9080\\
59.2029	0.8920\\
59.2350	0.8700\\
59.2663	0.8540\\
59.2882	0.8380\\
59.3197	0.8170\\
59.3518	0.7960\\
59.3830	0.7810\\
59.4040	0.7640\\
59.4360	0.7440\\
59.4675	0.7240\\
59.4993	0.7080\\
59.5207	0.6940\\
59.5526	0.6760\\
59.5844	0.6610\\
59.6056	0.6520\\
59.6373	0.6390\\
59.6694	0.6280\\
59.7008	0.6200\\
59.7213	0.6170\\
59.7527	0.6150\\
59.7852	0.6170\\
59.8068	0.6200\\
59.8381	0.6280\\
59.8694	0.6400\\
59.9018	0.6560\\
59.9224	0.6650\\
59.9537	0.6860\\
59.9851	0.7080\\
60.0059	0.7190\\
60.0371	0.7440\\
60.0686	0.7690\\
60.1008	0.7950\\
60.1225	0.8090\\
60.1538	0.8350\\
60.1859	0.8620\\
60.2173	0.8870\\
60.2390	0.9000\\
60.2706	0.9250\\
60.3018	0.9510\\
60.3225	0.9630\\
60.3538	0.9860\\
60.3853	1.0100\\
60.4175	1.0340\\
60.4380	1.0450\\
60.4696	1.0670\\
60.5007	1.0890\\
60.5214	1.1000\\
60.5535	1.1200\\
60.5852	1.1400\\
60.6174	1.1580\\
60.6390	1.1660\\
60.6705	1.1820\\
60.7035	1.1950\\
60.7350	1.2060\\
60.7559	1.2100\\
60.7871	1.2170\\
60.8183	1.2190\\
60.8389	1.2190\\
60.8706	1.2160\\
60.9026	1.2090\\
60.9339	1.1990\\
60.9546	1.1930\\
60.9859	1.1790\\
61.0176	1.1610\\
61.0390	1.1510\\
61.0707	1.1310\\
61.1028	1.1080\\
61.1339	1.0830\\
61.1557	1.0720\\
61.1870	1.0460\\
61.2183	1.0210\\
61.2494	1.0030\\
61.2709	0.9830\\
61.3027	0.9570\\
61.3341	0.9310\\
61.3555	0.9190\\
61.3870	0.8910\\
61.4182	0.8660\\
61.4492	0.8460\\
61.4709	0.8270\\
61.5030	0.8020\\
61.5350	0.7780\\
61.5557	0.7660\\
61.5871	0.7430\\
61.6183	0.7210\\
61.6492	0.7050\\
61.6704	0.6890\\
61.7015	0.6690\\
61.7339	0.6520\\
61.7648	0.6380\\
61.7860	0.6260\\
61.8172	0.6140\\
61.8482	0.6070\\
61.8694	0.6010\\
61.9016	0.6000\\
61.9326	0.6020\\
61.9640	0.6060\\
61.9850	0.6120\\
62.0165	0.6220\\
62.0492	0.6360\\
62.0704	0.6490\\
62.1016	0.6680\\
62.1325	0.6830\\
62.1639	0.7050\\
62.1850	0.7220\\
62.2172	0.7450\\
62.2492	0.7620\\
62.2806	0.7860\\
62.3018	0.8050\\
62.3339	0.8300\\
62.3658	0.8500\\
62.3871	0.8690\\
62.4185	0.8950\\
62.4506	0.9210\\
62.4815	0.9410\\
62.5030	0.9590\\
62.5349	0.9840\\
62.5674	1.0070\\
62.5881	1.0180\\
62.6195	1.0400\\
62.6516	1.0600\\
62.6824	1.0740\\
62.7036	1.0870\\
62.7350	1.1030\\
62.7660	1.1130\\
62.7984	1.1260\\
};
\end{axis}

\begin{axis}[%
width=0.951\fwidth,
height=0.189\fwidth,
at={(0\fwidth,0.688\fwidth)},
scale only axis,
xmin=34.0040,
xmax=62.7984,
xlabel={$t$ [s]},
ymin=-1.5585,
ymax=1.3701,
ylabel={$\vartheta_z$, $\hat{\vartheta}_z$ [1/s]},
axis background/.style={fill=white},
title style={font=\labelsize},
xlabel style={font=\labelsize,at={(axis description cs:0.5,\xlabeldist)}},
ylabel style={font=\labelsize,at={(axis description cs:\ylabeldist,0.5)}},
legend style={font=\ticksize},
ticklabel style={font=\ticksize}
]
\addplot [color=mycolor1,solid,forget plot]
  table[row sep=crcr]{%
34.0042	-0.0312\\
34.0261	-0.0291\\
34.0576	-0.0149\\
34.0898	0.0008\\
34.1115	0.0063\\
34.1431	0.0122\\
34.1742	0.0109\\
34.2054	-0.0016\\
34.2262	-0.0146\\
34.2574	-0.0292\\
34.2891	-0.0257\\
34.3096	-0.0113\\
34.3412	0.0049\\
34.3733	0.0067\\
34.4048	0.0150\\
34.4263	0.0220\\
34.4578	0.0346\\
34.4901	0.0451\\
34.5116	0.0513\\
34.5432	0.0773\\
34.5746	0.1031\\
34.6068	0.1016\\
34.6273	0.0937\\
34.6589	0.0880\\
34.6911	0.0900\\
34.7225	0.0984\\
34.7435	0.1011\\
34.7746	0.1104\\
34.8061	0.1303\\
34.8277	0.1315\\
34.8588	0.1108\\
34.8904	0.0959\\
34.9225	0.1035\\
34.9433	0.1129\\
34.9746	0.1147\\
35.0061	0.1007\\
35.0274	0.0954\\
35.0589	0.0974\\
35.0912	0.0950\\
35.1230	0.0917\\
35.1444	0.0958\\
35.1766	0.0955\\
35.2084	0.0979\\
35.2403	0.1049\\
35.2611	0.1047\\
35.2923	0.1155\\
35.3236	0.1236\\
35.3442	0.1184\\
35.3766	0.1157\\
35.4085	0.1194\\
35.4414	0.1246\\
35.4620	0.1308\\
35.4934	0.1498\\
35.5250	0.1484\\
35.5453	0.1293\\
35.5770	0.1401\\
35.6081	0.1658\\
35.6403	0.1627\\
35.6611	0.1595\\
35.6936	0.1520\\
35.7250	0.1695\\
35.7573	0.1853\\
35.7780	0.1708\\
35.8102	0.1740\\
35.8416	0.1777\\
35.8624	0.1621\\
35.8937	0.1684\\
35.9249	0.1959\\
35.9562	0.1961\\
35.9768	0.1684\\
36.0082	0.1457\\
36.0396	0.1667\\
36.0604	0.1622\\
36.0918	0.1388\\
36.1231	0.1314\\
36.1541	0.1556\\
36.1750	0.1693\\
36.2062	0.1629\\
36.2372	0.1254\\
36.2688	0.1246\\
36.2897	0.1454\\
36.3210	0.1678\\
36.3519	0.1757\\
36.3734	0.1788\\
36.4067	0.1931\\
36.4387	0.2049\\
36.4697	0.2021\\
36.4909	0.2022\\
36.5222	0.1968\\
36.5544	0.1540\\
36.5754	0.1370\\
36.6076	0.1351\\
36.6397	0.1302\\
36.6709	0.1341\\
36.6920	0.1477\\
36.7242	0.1629\\
36.7555	0.1502\\
36.7868	0.1216\\
36.8081	0.1181\\
36.8401	0.1355\\
36.8723	0.1360\\
36.8930	0.1249\\
36.9243	0.1076\\
36.9559	0.0976\\
36.9877	0.0920\\
37.0089	0.1041\\
37.0401	0.1143\\
37.0724	0.0931\\
37.0941	0.0891\\
37.1253	0.0811\\
37.1570	0.0594\\
37.1880	0.0500\\
37.2090	0.0458\\
37.2402	0.0364\\
37.2715	0.0212\\
37.2922	0.0012\\
37.3247	-0.0402\\
37.3566	-0.0746\\
37.3880	-0.0980\\
37.4088	-0.1046\\
37.4402	-0.1076\\
37.4717	-0.1211\\
37.5034	-0.1690\\
37.5246	-0.2010\\
37.5560	-0.2174\\
37.5871	-0.2088\\
37.6080	-0.1998\\
37.6396	-0.1930\\
37.6713	-0.2012\\
37.7028	-0.2137\\
37.7241	-0.2165\\
37.7560	-0.2006\\
37.7871	-0.1679\\
37.8082	-0.1556\\
37.8395	-0.1618\\
37.8708	-0.1827\\
37.9025	-0.1847\\
37.9237	-0.1760\\
37.9549	-0.1669\\
37.9870	-0.1565\\
38.0194	-0.1433\\
38.0407	-0.1473\\
38.0731	-0.1716\\
38.1048	-0.1720\\
38.1258	-0.1603\\
38.1562	-0.1364\\
38.1882	-0.1263\\
38.2198	-0.1511\\
38.2407	-0.1585\\
38.2730	-0.1639\\
38.3039	-0.1584\\
38.3249	-0.1462\\
38.3563	-0.1419\\
38.3883	-0.1591\\
38.4201	-0.1759\\
38.4421	-0.1673\\
38.4743	-0.1375\\
38.5063	-0.1338\\
38.5372	-0.1356\\
38.5585	-0.1326\\
38.5907	-0.1152\\
38.6218	-0.1071\\
38.6426	-0.1154\\
38.6740	-0.1182\\
38.7051	-0.1024\\
38.7373	-0.1043\\
38.7588	-0.1076\\
38.7908	-0.1031\\
38.8217	-0.0869\\
38.8429	-0.0879\\
38.8744	-0.1035\\
38.9053	-0.0910\\
38.9374	-0.0907\\
38.9585	-0.0995\\
38.9902	-0.1053\\
39.0222	-0.1032\\
39.0541	-0.0957\\
39.0752	-0.0928\\
39.1065	-0.0931\\
39.1384	-0.0938\\
39.1597	-0.0994\\
39.1920	-0.1004\\
39.2233	-0.0862\\
39.2551	-0.0840\\
39.2756	-0.0842\\
39.3078	-0.0827\\
39.3400	-0.0820\\
39.3619	-0.0828\\
39.3932	-0.0830\\
39.4255	-0.0939\\
39.4581	-0.1199\\
39.4794	-0.1285\\
39.5108	-0.1029\\
39.5424	-0.0776\\
39.5745	-0.0879\\
39.5951	-0.0982\\
39.6264	-0.1105\\
39.6588	-0.1110\\
39.6793	-0.1047\\
39.7119	-0.0904\\
39.7442	-0.0799\\
39.7758	-0.0762\\
39.7970	-0.0741\\
39.8297	-0.0783\\
39.8620	-0.0940\\
39.8830	-0.1031\\
39.9141	-0.1067\\
39.9463	-0.1043\\
39.9787	-0.0987\\
39.9998	-0.0952\\
40.0307	-0.0903\\
40.0632	-0.0936\\
40.0953	-0.0960\\
40.1163	-0.0929\\
40.1484	-0.0937\\
40.1807	-0.1110\\
40.2021	-0.1285\\
40.2349	-0.1323\\
40.2665	-0.1110\\
40.2985	-0.1056\\
40.3205	-0.0991\\
40.3527	-0.0960\\
40.3837	-0.1057\\
40.4047	-0.1072\\
40.4367	-0.0972\\
40.4685	-0.0967\\
40.5002	-0.1179\\
40.5215	-0.1290\\
40.5540	-0.1289\\
40.5859	-0.1114\\
40.6075	-0.1031\\
40.6393	-0.1114\\
40.6713	-0.1275\\
40.7028	-0.1229\\
40.7243	-0.1142\\
40.7557	-0.1162\\
40.7881	-0.1270\\
40.8197	-0.1281\\
40.8407	-0.1230\\
40.8728	-0.1172\\
40.9059	-0.1069\\
40.9270	-0.0943\\
40.9582	-0.0746\\
40.9895	-0.0658\\
41.0218	-0.0712\\
41.0427	-0.0723\\
41.0751	-0.0754\\
41.1064	-0.0864\\
41.1283	-0.0946\\
41.1597	-0.1011\\
41.1920	-0.0956\\
41.2241	-0.0814\\
41.2453	-0.0724\\
41.2767	-0.0716\\
41.3086	-0.0817\\
41.3395	-0.0846\\
41.3605	-0.0854\\
41.3920	-0.0844\\
41.4248	-0.0772\\
41.4459	-0.0733\\
41.4774	-0.0714\\
41.5096	-0.0725\\
41.5420	-0.0772\\
41.5638	-0.0834\\
41.5953	-0.0824\\
41.6275	-0.0773\\
41.6491	-0.0771\\
41.6810	-0.0711\\
41.7127	-0.0717\\
41.7450	-0.0940\\
41.7657	-0.1105\\
41.7982	-0.1163\\
41.8304	-0.1238\\
41.8619	-0.1155\\
41.8836	-0.1070\\
41.9147	-0.1180\\
41.9472	-0.1162\\
41.9681	-0.1067\\
42.0004	-0.1004\\
42.0327	-0.0984\\
42.0640	-0.0877\\
42.0856	-0.0807\\
42.1172	-0.0794\\
42.1500	-0.0870\\
42.1706	-0.0833\\
42.2029	-0.0753\\
42.2349	-0.0805\\
42.2660	-0.0878\\
42.2877	-0.0877\\
42.3195	-0.0832\\
42.3506	-0.0829\\
42.3829	-0.0855\\
42.4045	-0.0797\\
42.4368	-0.0630\\
42.4692	-0.0406\\
42.4902	-0.0213\\
42.5232	0.0031\\
42.5558	0.0339\\
42.5880	0.0604\\
42.6092	0.0795\\
42.6414	0.1176\\
42.6734	0.1273\\
42.6947	0.1284\\
42.7268	0.1289\\
42.7581	0.1241\\
42.7900	0.1201\\
42.8113	0.1183\\
42.8436	0.1067\\
42.8759	0.0878\\
42.9082	0.0810\\
42.9300	0.0740\\
42.9624	0.0744\\
42.9947	0.0882\\
43.0165	0.0892\\
43.0485	0.0842\\
43.0813	0.0892\\
43.1135	0.0945\\
43.1352	0.0927\\
43.1675	0.0922\\
43.1996	0.1002\\
43.2217	0.1070\\
43.2536	0.0972\\
43.2859	0.0808\\
43.3172	0.0948\\
43.3392	0.1121\\
43.3712	0.1086\\
43.4041	0.0953\\
43.4356	0.1023\\
43.4568	0.1036\\
43.4890	0.1148\\
43.5213	0.1079\\
43.5429	0.0934\\
43.5737	0.0858\\
43.6052	0.0955\\
43.6370	0.0972\\
43.6585	0.0881\\
43.6909	0.0807\\
43.7234	0.0873\\
43.7439	0.0948\\
43.7754	0.1015\\
43.8077	0.1056\\
43.8401	0.1133\\
43.8606	0.1142\\
43.8921	0.1103\\
43.9243	0.1222\\
43.9462	0.1405\\
43.9776	0.1541\\
44.0099	0.1383\\
44.0424	0.1273\\
44.0637	0.1293\\
44.0954	0.1553\\
44.1278	0.1602\\
44.1599	0.1273\\
44.1807	0.1251\\
44.2129	0.1421\\
44.2454	0.1397\\
44.2672	0.1291\\
44.2992	0.1235\\
44.3316	0.1290\\
44.3638	0.1438\\
44.3850	0.1451\\
44.4161	0.1315\\
44.4478	0.1262\\
44.4687	0.1345\\
44.5007	0.1531\\
44.5318	0.1605\\
44.5640	0.1415\\
44.5847	0.1305\\
44.6160	0.1444\\
44.6473	0.1526\\
44.6786	0.1476\\
44.7000	0.1472\\
44.7318	0.1515\\
44.7632	0.1546\\
44.7844	0.1482\\
44.8165	0.1353\\
44.8484	0.1516\\
44.8799	0.1700\\
44.9011	0.1728\\
44.9328	0.1702\\
44.9656	0.1746\\
44.9865	0.1734\\
45.0190	0.1614\\
45.0516	0.1537\\
45.0827	0.1530\\
45.1043	0.1562\\
45.1372	0.1778\\
45.1687	0.2074\\
45.2005	0.2094\\
45.2223	0.1965\\
45.2539	0.1811\\
45.2854	0.1953\\
45.3069	0.2188\\
45.3388	0.2137\\
45.3703	0.1783\\
45.4024	0.1725\\
45.4236	0.1751\\
45.4558	0.1777\\
45.4881	0.1815\\
45.5096	0.1946\\
45.5417	0.2214\\
45.5739	0.2067\\
45.6068	0.1948\\
45.6281	0.2001\\
45.6595	0.1937\\
45.6905	0.1876\\
45.7226	0.1856\\
45.7430	0.1861\\
45.7744	0.1712\\
45.8056	0.1793\\
45.8263	0.1948\\
45.8584	0.1839\\
45.8904	0.1646\\
45.9218	0.1833\\
45.9433	0.1988\\
45.9752	0.1882\\
46.0082	0.2106\\
46.0294	0.2250\\
46.0616	0.2093\\
46.0931	0.1961\\
46.1248	0.1858\\
46.1459	0.1821\\
46.1782	0.1672\\
46.2096	0.1394\\
46.2413	0.1167\\
46.2629	0.0969\\
46.2951	0.0542\\
46.3264	0.0089\\
46.3482	-0.0179\\
46.3797	-0.0597\\
46.4116	-0.1038\\
46.4431	-0.1537\\
46.4647	-0.2036\\
46.4971	-0.2637\\
46.5288	-0.2792\\
46.5503	-0.2814\\
46.5819	-0.2707\\
46.6129	-0.3131\\
46.6449	-0.3394\\
46.6664	-0.2959\\
46.6984	-0.2878\\
46.7308	-0.3119\\
46.7618	-0.2992\\
46.7830	-0.2915\\
46.8151	-0.2791\\
46.8462	-0.2574\\
46.8669	-0.2440\\
46.8992	-0.2679\\
46.9317	-0.2774\\
46.9630	-0.2583\\
46.9842	-0.2581\\
47.0158	-0.2656\\
47.0473	-0.2509\\
47.0688	-0.2308\\
47.1002	-0.2085\\
47.1326	-0.2173\\
47.1650	-0.2100\\
47.1860	-0.1917\\
47.2178	-0.1905\\
47.2492	-0.2137\\
47.2708	-0.2149\\
47.3029	-0.2043\\
47.3347	-0.2238\\
47.3662	-0.2210\\
47.3868	-0.2118\\
47.4181	-0.2052\\
47.4506	-0.2030\\
47.4818	-0.1840\\
47.5029	-0.1915\\
47.5342	-0.2209\\
47.5669	-0.2179\\
47.5878	-0.2255\\
47.6199	-0.2537\\
47.6520	-0.2300\\
47.6836	-0.2329\\
47.7047	-0.2332\\
47.7364	-0.2281\\
47.7686	-0.2464\\
47.7900	-0.2282\\
47.8223	-0.2198\\
47.8545	-0.2272\\
47.8863	-0.1877\\
47.9082	-0.1886\\
47.9409	-0.1950\\
47.9724	-0.1814\\
48.0042	-0.1770\\
48.0254	-0.1798\\
48.0567	-0.1793\\
48.0881	-0.1724\\
48.1086	-0.1777\\
48.1403	-0.2017\\
48.1718	-0.2134\\
48.2034	-0.2234\\
48.2245	-0.2271\\
48.2560	-0.2023\\
48.2882	-0.1762\\
48.3097	-0.1725\\
48.3413	-0.1697\\
48.3736	-0.1628\\
48.4063	-0.1742\\
48.4274	-0.1777\\
48.4587	-0.1511\\
48.4899	-0.1376\\
48.5214	-0.1450\\
48.5422	-0.1480\\
48.5731	-0.1309\\
48.6052	-0.1215\\
48.6266	-0.1309\\
48.6585	-0.1391\\
48.6900	-0.1313\\
48.7212	-0.1196\\
48.7421	-0.1178\\
48.7731	-0.1297\\
48.8054	-0.1349\\
48.8265	-0.1322\\
48.8587	-0.1388\\
48.8902	-0.1476\\
48.9226	-0.1555\\
48.9442	-0.1549\\
48.9766	-0.1473\\
49.0076	-0.1557\\
49.0397	-0.1659\\
49.0618	-0.1622\\
49.0943	-0.1563\\
49.1265	-0.1728\\
49.1482	-0.1678\\
49.1798	-0.1696\\
49.2117	-0.1844\\
49.2434	-0.1716\\
49.2648	-0.1713\\
49.2964	-0.1843\\
49.3284	-0.1900\\
49.3492	-0.1841\\
49.3817	-0.1754\\
49.4128	-0.1647\\
49.4439	-0.1538\\
49.4647	-0.1564\\
49.4956	-0.1572\\
49.5276	-0.1487\\
49.5597	-0.1496\\
49.5803	-0.1415\\
49.6126	-0.1329\\
49.6442	-0.1500\\
49.6660	-0.1672\\
49.6967	-0.1664\\
49.7284	-0.1553\\
49.7597	-0.1633\\
49.7807	-0.1644\\
49.8119	-0.1569\\
49.8439	-0.1465\\
49.8646	-0.1421\\
49.8961	-0.1497\\
49.9277	-0.1487\\
49.9592	-0.1275\\
49.9803	-0.1290\\
50.0129	-0.1458\\
50.0452	-0.1429\\
50.0665	-0.1315\\
50.0980	-0.1264\\
50.1295	-0.1303\\
50.1616	-0.1234\\
50.1832	-0.1157\\
50.2157	-0.1146\\
50.2472	-0.1082\\
50.2796	-0.1029\\
50.3004	-0.1087\\
50.3324	-0.1086\\
50.3639	-0.0897\\
50.3845	-0.0775\\
50.4166	-0.0698\\
50.4480	-0.0369\\
50.4794	0.0004\\
50.4997	0.0230\\
50.5313	0.0641\\
50.5630	0.0865\\
50.5844	0.1159\\
50.6162	0.1623\\
50.6480	0.1866\\
50.6802	0.2260\\
50.7012	0.2538\\
50.7329	0.2673\\
50.7643	0.2713\\
50.7960	0.2892\\
50.8176	0.3037\\
50.8502	0.3107\\
50.8813	0.3109\\
50.9027	0.3184\\
50.9341	0.3373\\
50.9665	0.3304\\
50.9980	0.3107\\
51.0194	0.3058\\
51.0521	0.3164\\
51.0832	0.3111\\
51.1041	0.3037\\
51.1363	0.2888\\
51.1686	0.2815\\
51.2009	0.3057\\
51.2217	0.2975\\
51.2544	0.2582\\
51.2858	0.2390\\
51.3188	0.2494\\
51.3394	0.2406\\
51.3709	0.2177\\
51.4024	0.2194\\
51.4238	0.2314\\
51.4557	0.2367\\
51.4873	0.1985\\
51.5196	0.1824\\
51.5406	0.2067\\
51.5725	0.2115\\
51.6041	0.1941\\
51.6254	0.1966\\
51.6578	0.2161\\
51.6903	0.2345\\
51.7223	0.2391\\
51.7433	0.2393\\
51.7750	0.2398\\
51.8064	0.2507\\
51.8377	0.2611\\
51.8581	0.2639\\
51.8901	0.2929\\
51.9226	0.2957\\
51.9442	0.2863\\
51.9763	0.2997\\
52.0079	0.2980\\
52.0402	0.3033\\
52.0614	0.3108\\
52.0928	0.3131\\
52.1234	0.3082\\
52.1453	0.3018\\
52.1765	0.2926\\
52.2083	0.3039\\
52.2402	0.3126\\
52.2622	0.3110\\
52.2938	0.3167\\
52.3266	0.3357\\
52.3587	0.3158\\
52.3805	0.3201\\
52.4121	0.3424\\
52.4431	0.3358\\
52.4642	0.3286\\
52.4959	0.3323\\
52.5282	0.3342\\
52.5597	0.3225\\
52.5812	0.3264\\
52.6126	0.3425\\
52.6449	0.3355\\
52.6665	0.3161\\
52.6989	0.2906\\
52.7301	0.2717\\
52.7615	0.2673\\
52.7826	0.2741\\
52.8144	0.2600\\
52.8468	0.2322\\
52.8782	0.2256\\
52.8995	0.2304\\
52.9313	0.2586\\
52.9634	0.2539\\
52.9841	0.2307\\
53.0164	0.2238\\
53.0489	0.1881\\
53.0811	0.1424\\
53.1019	0.1097\\
53.1340	0.0247\\
53.1660	-0.0635\\
53.1871	-0.1166\\
53.2185	-0.1992\\
53.2506	-0.2873\\
53.2828	-0.3440\\
53.3047	-0.4016\\
53.3373	-0.4743\\
53.3690	-0.4957\\
53.3902	-0.5184\\
53.4223	-0.5180\\
53.4539	-0.5033\\
53.4861	-0.5141\\
53.5077	-0.5145\\
53.5396	-0.4941\\
53.5709	-0.4743\\
53.6018	-0.4681\\
53.6229	-0.4722\\
53.6547	-0.4939\\
53.6866	-0.4819\\
53.7083	-0.4430\\
53.7405	-0.4423\\
53.7721	-0.4437\\
53.8049	-0.4533\\
53.8262	-0.4656\\
53.8587	-0.4259\\
53.8902	-0.4113\\
53.9117	-0.4107\\
53.9439	-0.3825\\
53.9761	-0.3712\\
54.0086	-0.3824\\
54.0305	-0.3908\\
54.0628	-0.3734\\
54.0948	-0.3420\\
54.1263	-0.3311\\
54.1478	-0.3209\\
54.1802	-0.3139\\
54.2113	-0.3161\\
54.2322	-0.3125\\
54.2645	-0.2981\\
54.2957	-0.3004\\
54.3282	-0.3126\\
54.3497	-0.3101\\
54.3811	-0.2924\\
54.4125	-0.2920\\
54.4342	-0.3033\\
54.4664	-0.3102\\
54.4990	-0.3118\\
54.5303	-0.3140\\
54.5521	-0.3129\\
54.5830	-0.3088\\
54.6145	-0.3050\\
54.6461	-0.3033\\
54.6678	-0.2966\\
54.6988	-0.2778\\
54.7316	-0.2745\\
54.7520	-0.2768\\
54.7831	-0.2655\\
54.8144	-0.2439\\
54.8461	-0.2516\\
54.8677	-0.2653\\
54.8992	-0.2556\\
54.9307	-0.2491\\
54.9520	-0.2508\\
54.9829	-0.2441\\
55.0146	-0.2278\\
55.0468	-0.2220\\
55.0687	-0.2126\\
55.0998	-0.1620\\
55.1316	-0.1277\\
55.1645	-0.0786\\
55.1854	-0.0295\\
55.2165	0.0315\\
55.2490	0.0877\\
55.2706	0.1310\\
55.3019	0.1982\\
55.3340	0.2465\\
55.3662	0.2986\\
55.3881	0.3398\\
55.4204	0.3904\\
55.4518	0.4178\\
55.4735	0.4349\\
55.5051	0.4661\\
55.5363	0.4803\\
55.5676	0.4827\\
55.5893	0.4933\\
55.6215	0.5276\\
55.6541	0.5397\\
55.6850	0.5215\\
55.7062	0.5124\\
55.7375	0.5065\\
55.7685	0.5033\\
55.7901	0.5076\\
55.8227	0.4985\\
55.8540	0.4843\\
55.8861	0.5010\\
55.9071	0.5209\\
55.9389	0.5123\\
55.9706	0.4681\\
55.9912	0.4506\\
56.0233	0.4698\\
56.0553	0.4467\\
56.0874	0.3890\\
56.1079	0.3752\\
56.1398	0.3833\\
56.1705	0.4040\\
56.2020	0.4199\\
56.2236	0.4350\\
56.2558	0.4674\\
56.2881	0.5002\\
56.3092	0.5102\\
56.3408	0.5035\\
56.3715	0.4921\\
56.4031	0.4928\\
56.4245	0.5010\\
56.4567	0.5181\\
56.4892	0.5173\\
56.5103	0.5118\\
56.5425	0.5063\\
56.5745	0.5092\\
56.6069	0.4893\\
56.6280	0.4700\\
56.6592	0.4616\\
56.6913	0.4615\\
56.7125	0.4516\\
56.7446	0.4362\\
56.7759	0.4388\\
56.8070	0.3837\\
56.8279	0.3236\\
56.8593	0.2555\\
56.8912	0.1766\\
56.9229	0.0708\\
56.9435	-0.0126\\
56.9747	-0.1428\\
57.0068	-0.2833\\
57.0278	-0.3702\\
57.0593	-0.4725\\
57.0913	-0.5546\\
57.1226	-0.6612\\
57.1435	-0.7299\\
57.1757	-0.7884\\
57.2071	-0.8054\\
57.2280	-0.7979\\
57.2596	-0.7860\\
57.2913	-0.7917\\
57.3226	-0.8181\\
57.3437	-0.8279\\
57.3747	-0.8113\\
57.4059	-0.7903\\
57.4373	-0.7730\\
57.4579	-0.7638\\
57.4892	-0.7332\\
57.5205	-0.7181\\
57.5412	-0.7263\\
57.5727	-0.7289\\
57.6038	-0.7125\\
57.6354	-0.7035\\
57.6569	-0.7251\\
57.6884	-0.7159\\
57.7196	-0.6810\\
57.7402	-0.6805\\
57.7716	-0.6541\\
57.8030	-0.6174\\
57.8351	-0.6026\\
57.8571	-0.6050\\
57.8887	-0.5907\\
57.9206	-0.5656\\
57.9531	-0.5337\\
57.9746	-0.4963\\
58.0059	-0.4576\\
58.0375	-0.4346\\
58.0581	-0.4172\\
58.0893	-0.3968\\
58.1206	-0.3830\\
58.1520	-0.3748\\
58.1735	-0.3775\\
58.2051	-0.3690\\
58.2373	-0.3618\\
58.2581	-0.3677\\
58.2893	-0.3555\\
58.3206	-0.3285\\
58.3518	-0.2864\\
58.3727	-0.2452\\
58.4039	-0.1741\\
58.4351	-0.1016\\
58.4663	-0.0200\\
58.4873	0.0273\\
58.5186	0.0929\\
58.5496	0.1865\\
58.5705	0.2481\\
58.6020	0.3260\\
58.6328	0.3834\\
58.6641	0.4369\\
58.6854	0.4879\\
58.7175	0.5747\\
58.7494	0.6142\\
58.7706	0.6251\\
58.8020	0.6650\\
58.8343	0.7477\\
58.8661	0.7415\\
58.8874	0.7467\\
58.9195	0.7631\\
58.9520	0.7400\\
58.9829	0.7372\\
59.0037	0.7496\\
59.0352	0.7687\\
59.0665	0.7739\\
59.0872	0.7904\\
59.1184	0.8287\\
59.1495	0.8628\\
59.1817	0.8995\\
59.2029	0.9093\\
59.2350	0.9006\\
59.2663	0.9147\\
59.2882	0.9640\\
59.3197	1.0154\\
59.3518	0.9979\\
59.3830	0.9965\\
59.4040	1.0546\\
59.4360	1.1022\\
59.4675	1.0562\\
59.4993	1.0699\\
59.5207	1.0666\\
59.5526	0.9906\\
59.5844	0.9057\\
59.6056	0.8661\\
59.6373	0.7990\\
59.6694	0.6670\\
59.7008	0.5052\\
59.7213	0.3692\\
59.7527	0.1422\\
59.7852	-0.0947\\
59.8068	-0.2729\\
59.8381	-0.5089\\
59.8694	-0.6838\\
59.9018	-0.8314\\
59.9224	-0.9377\\
59.9537	-1.0568\\
59.9851	-1.1240\\
60.0059	-1.1642\\
60.0371	-1.1885\\
60.0686	-1.2005\\
60.1008	-1.2464\\
60.1225	-1.2296\\
60.1538	-1.1906\\
60.1859	-1.1560\\
60.2173	-1.0834\\
60.2390	-1.0621\\
60.2706	-1.0128\\
60.3018	-0.9353\\
60.3225	-0.9186\\
60.3538	-0.8834\\
60.3853	-0.8338\\
60.4175	-0.8290\\
60.4380	-0.7953\\
60.4696	-0.7356\\
60.5007	-0.7170\\
60.5214	-0.6945\\
60.5535	-0.6380\\
60.5852	-0.5836\\
60.6174	-0.5483\\
60.6390	-0.5083\\
60.6705	-0.4402\\
60.7035	-0.3809\\
60.7350	-0.2995\\
60.7559	-0.2445\\
60.7871	-0.1571\\
60.8183	-0.0698\\
60.8389	-0.0142\\
60.8706	0.0796\\
60.9026	0.1841\\
60.9339	0.2760\\
60.9546	0.3263\\
60.9859	0.4051\\
61.0176	0.4981\\
61.0390	0.5548\\
61.0707	0.6462\\
61.1028	0.7292\\
61.1339	0.7696\\
61.1557	0.7768\\
61.1870	0.8089\\
61.2183	0.8642\\
61.2494	0.8987\\
61.2709	0.9397\\
61.3027	0.9896\\
61.3341	1.0098\\
61.3555	1.0309\\
61.3870	1.0749\\
61.4182	1.0925\\
61.4492	1.1269\\
61.4709	1.1659\\
61.5030	1.1989\\
61.5350	1.1966\\
61.5557	1.1886\\
61.5871	1.1838\\
61.6183	1.1752\\
61.6492	1.1639\\
61.6704	1.1776\\
61.7015	1.1605\\
61.7339	1.0924\\
61.7648	1.0305\\
61.7860	0.9743\\
61.8172	0.8263\\
61.8482	0.6260\\
61.8694	0.4629\\
61.9016	0.1803\\
61.9326	-0.0753\\
61.9640	-0.3077\\
61.9850	-0.4672\\
62.0165	-0.7195\\
62.0492	-0.9517\\
62.0704	-1.0330\\
62.1016	-1.0722\\
62.1325	-1.1211\\
62.1639	-1.1393\\
62.1850	-1.1560\\
62.2172	-1.2093\\
62.2492	-1.1764\\
62.2806	-1.1400\\
62.3018	-1.1463\\
62.3339	-1.1665\\
62.3658	-1.1397\\
62.3871	-1.1267\\
62.4185	-1.1135\\
62.4506	-1.0515\\
62.4815	-0.9794\\
62.5030	-0.9382\\
62.5349	-0.8978\\
62.5674	-0.8435\\
62.5881	-0.8016\\
62.6195	-0.7333\\
62.6516	-0.6712\\
62.6824	-0.6028\\
62.7036	-0.5532\\
62.7350	-0.4872\\
62.7660	-0.4305\\
62.7984	-0.3757\\
};
\addplot [color=mycolor2,solid,forget plot]
  table[row sep=crcr]{%
34.0040	0.1687\\
34.0240	0.1063\\
34.0540	0.0191\\
34.0840	0.0191\\
34.1140	0.0191\\
34.1440	0.0191\\
34.1740	0.0191\\
34.2040	0.0191\\
34.2340	0.0191\\
34.2540	0.0191\\
34.2840	0.0191\\
34.3140	0.0504\\
34.3440	0.0460\\
34.3740	0.0488\\
34.4040	0.0488\\
34.4340	0.0488\\
34.4640	0.0488\\
34.4840	0.0488\\
34.5140	0.0488\\
34.5440	0.0488\\
34.5740	0.0488\\
34.6040	0.0488\\
34.6340	0.1019\\
34.6640	0.1390\\
34.6940	0.1518\\
34.7240	0.1260\\
34.7440	0.1200\\
34.7740	0.0830\\
34.8040	0.0854\\
34.8340	0.0854\\
34.8640	0.1235\\
34.8940	0.1655\\
34.9240	0.1989\\
34.9540	0.1551\\
34.9740	0.1629\\
35.0040	0.1689\\
35.0340	0.1712\\
35.0640	0.1387\\
35.0940	0.1369\\
35.1240	0.1218\\
35.1540	0.1183\\
35.1840	0.0976\\
35.2140	0.0998\\
35.2340	0.0833\\
35.2640	0.1428\\
35.2940	0.1123\\
35.3240	0.0834\\
35.3540	0.1235\\
35.3840	0.1307\\
35.4140	0.1423\\
35.4440	0.1476\\
35.4640	0.1245\\
35.4940	0.0917\\
35.5240	0.0897\\
35.5540	0.1116\\
35.5840	0.0960\\
35.6140	0.0529\\
35.6440	0.1627\\
35.6740	0.2510\\
35.7040	0.1107\\
35.7240	0.1476\\
35.7540	0.1683\\
35.7840	0.1855\\
35.8140	0.1700\\
35.8440	0.1330\\
35.8740	0.1779\\
35.9040	0.2288\\
35.9340	0.2312\\
35.9540	0.2016\\
35.9840	0.1969\\
36.0140	0.2099\\
36.0440	0.1963\\
36.0740	0.2051\\
36.1040	0.2089\\
36.1340	0.1857\\
36.1640	0.1612\\
36.1940	0.1548\\
36.2140	0.1530\\
36.2440	0.1575\\
36.2740	0.1467\\
36.3040	0.1467\\
36.3340	0.2359\\
36.3640	0.2481\\
36.3940	0.3011\\
36.4240	0.2553\\
36.4440	0.2969\\
36.4740	0.2309\\
36.5040	0.1923\\
36.5340	0.1587\\
36.5640	0.1638\\
36.5940	0.1457\\
36.6240	0.1231\\
36.6540	0.1092\\
36.6840	0.1282\\
36.7040	0.2312\\
36.7340	0.3488\\
36.7640	0.1852\\
36.7940	0.1688\\
36.8240	0.3005\\
36.8540	0.3126\\
36.8840	0.2676\\
36.9140	0.2167\\
36.9340	0.1964\\
36.9640	0.1731\\
36.9940	0.1498\\
37.0240	0.0778\\
37.0540	0.1715\\
37.0840	0.1906\\
37.1140	0.1684\\
37.1440	0.1429\\
37.1740	0.1143\\
37.1940	0.1106\\
37.2240	0.0866\\
37.2540	0.0769\\
37.2840	0.0630\\
37.3140	0.0477\\
37.3440	0.0290\\
37.3740	0.0269\\
37.4040	0.0227\\
37.4240	0.0127\\
37.4540	-0.0052\\
37.4840	-0.0207\\
37.5140	-0.0740\\
37.5440	-0.1350\\
37.5740	-0.1771\\
37.6040	-0.1958\\
37.6340	-0.2020\\
37.6540	-0.1947\\
37.6840	-0.2066\\
37.7140	-0.2018\\
37.7440	-0.1670\\
37.7740	-0.1701\\
37.8040	-0.1670\\
37.8340	-0.1945\\
37.8640	-0.2151\\
37.8940	-0.2322\\
37.9140	-0.2105\\
37.9440	-0.2601\\
37.9740	-0.2710\\
38.0040	-0.2019\\
38.0340	-0.2544\\
38.0640	-0.2454\\
38.0940	-0.0719\\
38.1240	-0.1145\\
38.1440	-0.1341\\
38.1740	-0.1193\\
38.2040	-0.1482\\
38.2340	-0.1645\\
38.2640	-0.1578\\
38.2940	-0.1785\\
38.3240	-0.1049\\
38.3540	-0.0396\\
38.3840	-0.0792\\
38.4040	-0.0996\\
38.4340	-0.1962\\
38.4640	-0.2054\\
38.4940	-0.2124\\
38.5240	-0.1559\\
38.5540	-0.1398\\
38.5840	-0.1133\\
38.6140	-0.0737\\
38.6340	-0.0696\\
38.6640	-0.0696\\
38.6940	-0.1287\\
38.7240	-0.1281\\
38.7540	-0.1920\\
38.7840	-0.2326\\
38.8140	-0.1667\\
38.8440	-0.1388\\
38.8740	-0.1696\\
38.8940	-0.1575\\
38.9240	-0.1282\\
38.9540	-0.1147\\
38.9840	-0.0924\\
39.0140	-0.0787\\
39.0440	-0.0627\\
39.0740	-0.0946\\
39.1040	-0.1054\\
39.1240	-0.0991\\
39.1540	-0.0966\\
39.1840	-0.0821\\
39.2140	-0.0988\\
39.2440	-0.1001\\
39.2740	-0.1098\\
39.3040	-0.1129\\
39.3340	-0.1709\\
39.3640	-0.1890\\
39.3840	-0.1378\\
39.4140	-0.1190\\
39.4440	-0.1178\\
39.4740	-0.1524\\
39.5040	-0.1009\\
39.5340	-0.0708\\
39.5640	-0.1162\\
39.5940	-0.1033\\
39.6140	-0.1293\\
39.6440	-0.0956\\
39.6740	-0.0628\\
39.7040	-0.0737\\
39.7340	-0.0823\\
39.7640	-0.0928\\
39.7940	-0.0792\\
39.8240	-0.0787\\
39.8540	-0.0390\\
39.8740	-0.0652\\
39.9040	-0.0999\\
39.9340	-0.1061\\
39.9640	-0.1526\\
39.9940	-0.1971\\
40.0240	-0.2387\\
40.0540	-0.2121\\
40.0840	-0.2258\\
40.1040	-0.2217\\
40.1340	-0.1838\\
40.1640	-0.1532\\
40.1940	-0.1590\\
40.2240	-0.1700\\
40.2540	-0.1708\\
40.2840	-0.1460\\
40.3140	-0.1341\\
40.3440	-0.0893\\
40.3640	-0.0966\\
40.3940	-0.1599\\
40.4240	-0.1481\\
40.4540	-0.1375\\
40.4840	-0.1377\\
40.5140	-0.1319\\
40.5440	-0.1444\\
40.5740	-0.1645\\
40.5940	-0.1696\\
40.6240	-0.1582\\
40.6540	-0.1760\\
40.6840	-0.1646\\
40.7140	-0.1307\\
40.7440	-0.1220\\
40.7740	-0.0928\\
40.8040	-0.0847\\
40.8340	-0.1107\\
40.8540	-0.1062\\
40.8840	-0.1279\\
40.9140	-0.1642\\
40.9440	-0.1832\\
40.9740	-0.1754\\
41.0040	-0.1425\\
41.0340	-0.1394\\
41.0640	-0.1394\\
41.0840	-0.1215\\
41.1140	-0.0741\\
41.1440	-0.0611\\
41.1740	-0.0681\\
41.2040	-0.0601\\
41.2340	-0.0793\\
41.2640	-0.0787\\
41.2940	-0.0638\\
41.3140	-0.0627\\
41.3440	-0.0674\\
41.3740	-0.0703\\
41.4040	-0.1072\\
41.4340	-0.1544\\
41.4640	-0.1404\\
41.4940	-0.1244\\
41.5240	-0.1043\\
41.5540	-0.1111\\
41.5740	-0.1151\\
41.6040	-0.1059\\
41.6340	-0.0924\\
41.6640	-0.0646\\
41.6940	-0.0410\\
41.7240	0.0482\\
41.7540	0.0835\\
41.7840	0.0817\\
41.8040	0.0337\\
41.8340	-0.0189\\
41.8640	-0.0786\\
41.8940	-0.1086\\
41.9240	-0.2080\\
41.9540	-0.1879\\
41.9840	-0.1444\\
42.0140	-0.0977\\
42.0440	-0.0886\\
42.0640	-0.0840\\
42.0940	-0.0929\\
42.1240	-0.1273\\
42.1540	-0.1148\\
42.1840	-0.1035\\
42.2140	-0.1037\\
42.2440	-0.1037\\
42.2740	-0.1047\\
42.2940	-0.1047\\
42.3240	-0.0888\\
42.3540	-0.0707\\
42.3840	-0.0791\\
42.4140	-0.0794\\
42.4440	-0.1111\\
42.4740	-0.0878\\
42.5040	-0.0741\\
42.5340	-0.0628\\
42.5540	-0.0773\\
42.5840	-0.0773\\
42.6140	-0.0710\\
42.6440	-0.0369\\
42.6740	-0.0314\\
42.7040	-0.0319\\
42.7340	-0.0905\\
42.7640	-0.1729\\
42.7840	-0.1507\\
42.8140	-0.1080\\
42.8440	-0.1080\\
42.8740	-0.1080\\
42.9040	-0.1080\\
42.9340	-0.0135\\
42.9640	-0.0139\\
42.9940	-0.0323\\
43.0240	-0.0006\\
43.0440	0.0390\\
43.0740	0.0589\\
43.1040	0.1011\\
43.1340	0.0983\\
43.1640	0.1304\\
43.1940	0.1323\\
43.2240	0.1338\\
43.2540	0.1117\\
43.2740	0.0862\\
43.3040	0.0714\\
43.3340	0.1085\\
43.3640	0.1167\\
43.3940	0.1151\\
43.4240	0.1133\\
43.4540	0.1019\\
43.4840	0.1040\\
43.5140	0.0910\\
43.5340	0.0871\\
43.5640	0.0726\\
43.5940	0.0888\\
43.6240	0.1267\\
43.6540	0.1937\\
43.6840	0.2100\\
43.7140	0.1801\\
43.7440	0.1856\\
43.7640	0.1624\\
43.7940	0.1874\\
43.8240	0.2033\\
43.8540	0.1772\\
43.8840	0.1499\\
43.9140	0.1554\\
43.9440	0.0265\\
43.9740	0.1022\\
44.0040	0.1610\\
44.0240	0.1820\\
44.0540	0.1775\\
44.0840	0.1639\\
44.1140	0.1313\\
44.1440	0.1505\\
44.1740	0.1576\\
44.2040	0.1586\\
44.2340	0.1867\\
44.2540	0.1645\\
44.2840	0.1400\\
44.3140	0.1314\\
44.3440	0.1449\\
44.3740	0.1113\\
44.4040	0.1718\\
44.4340	0.2296\\
44.4640	0.1638\\
44.4940	0.1460\\
44.5140	0.1480\\
44.5440	0.1140\\
44.5740	0.1725\\
44.6040	0.1556\\
44.6340	0.1451\\
44.6640	0.1533\\
44.6940	0.1382\\
44.7240	0.1375\\
44.7440	0.1320\\
44.7740	0.1416\\
44.8040	0.1517\\
44.8340	0.1718\\
44.8640	0.1660\\
44.8940	0.1638\\
44.9240	0.1645\\
44.9540	0.1407\\
44.9740	0.1461\\
45.0040	0.1743\\
45.0340	0.1775\\
45.0640	0.1623\\
45.0940	0.1783\\
45.1240	0.1985\\
45.1540	0.1837\\
45.1840	0.1941\\
45.2140	0.2128\\
45.2340	0.2129\\
45.2640	0.2302\\
45.2940	0.2121\\
45.3240	0.2164\\
45.3540	0.2257\\
45.3840	0.2040\\
45.4140	0.1931\\
45.4440	0.2032\\
45.4640	0.2089\\
45.4940	0.2481\\
45.5240	0.2122\\
45.5540	0.1948\\
45.5840	0.2216\\
45.6140	0.2088\\
45.6440	0.2500\\
45.6740	0.2113\\
45.7040	0.2101\\
45.7240	0.2301\\
45.7540	0.1948\\
45.7840	0.2493\\
45.8140	0.2558\\
45.8440	0.2414\\
45.8740	0.2362\\
45.9040	0.3232\\
45.9340	0.2589\\
45.9540	0.2159\\
45.9840	0.1980\\
46.0140	0.2574\\
46.0440	0.2475\\
46.0740	0.2223\\
46.1040	0.2301\\
46.1340	0.2525\\
46.1640	0.2630\\
46.1940	0.3197\\
46.2140	0.3432\\
46.2440	0.2605\\
46.2740	0.1743\\
46.3040	0.1308\\
46.3340	0.2258\\
46.3640	0.1809\\
46.3940	0.0811\\
46.4240	0.0585\\
46.4440	0.0189\\
46.4740	-0.1061\\
46.5040	-0.1702\\
46.5340	-0.2281\\
46.5640	-0.3334\\
46.5940	-0.3316\\
46.6240	-0.3841\\
46.6540	-0.4134\\
46.6840	-0.4185\\
46.7040	-0.3889\\
46.7340	-0.3590\\
46.7640	-0.3050\\
46.7940	-0.2783\\
46.8240	-0.3028\\
46.8540	-0.3078\\
46.8840	-0.3158\\
46.9140	-0.3331\\
46.9340	-0.3280\\
46.9640	-0.3140\\
46.9940	-0.2858\\
47.0240	-0.3017\\
47.0540	-0.3028\\
47.0840	-0.2971\\
47.1140	-0.2896\\
47.1440	-0.2575\\
47.1740	-0.2398\\
47.1940	-0.2206\\
47.2240	-0.2433\\
47.2540	-0.2304\\
47.2840	-0.2318\\
47.3140	-0.2147\\
47.3440	-0.2295\\
47.3740	-0.2253\\
47.4040	-0.2202\\
47.4240	-0.2246\\
47.4540	-0.2314\\
47.4840	-0.2592\\
47.5140	-0.2437\\
47.5440	-0.2543\\
47.5740	-0.2616\\
47.6040	-0.3319\\
47.6340	-0.2723\\
47.6640	-0.3022\\
47.6840	-0.3606\\
47.7140	-0.4051\\
47.7440	-0.3491\\
47.7740	-0.3181\\
47.8040	-0.3278\\
47.8340	-0.3029\\
47.8640	-0.3109\\
47.8940	-0.2500\\
47.9140	-0.2312\\
47.9440	-0.2206\\
47.9740	-0.2140\\
48.0040	-0.1822\\
48.0340	-0.1238\\
48.0640	-0.1671\\
48.0940	-0.1614\\
48.1240	-0.2310\\
48.1540	-0.2513\\
48.1740	-0.2248\\
48.2040	-0.2060\\
48.2340	-0.2043\\
48.2640	-0.1865\\
48.2940	-0.1571\\
48.3240	-0.1563\\
48.3540	-0.1938\\
48.3840	-0.2066\\
48.4040	-0.2404\\
48.4340	-0.1993\\
48.4640	-0.1728\\
48.4940	-0.1728\\
48.5240	-0.1728\\
48.5540	-0.1648\\
48.5840	-0.1741\\
48.6140	-0.1292\\
48.6340	-0.2014\\
48.6640	-0.1964\\
48.6940	-0.1896\\
48.7240	-0.1751\\
48.7540	-0.1852\\
48.7840	-0.1677\\
48.8140	-0.1480\\
48.8440	-0.1223\\
48.8740	-0.1232\\
48.8940	-0.1366\\
48.9240	-0.1596\\
48.9540	-0.1650\\
48.9840	-0.2117\\
49.0140	-0.1694\\
49.0440	-0.1537\\
49.0740	-0.1442\\
49.1040	-0.1536\\
49.1240	-0.2217\\
49.1540	-0.1843\\
49.1840	-0.2454\\
49.2140	-0.2729\\
49.2440	-0.2402\\
49.2740	-0.2499\\
49.3040	-0.2392\\
49.3340	-0.3104\\
49.3640	-0.2720\\
49.3840	-0.2115\\
49.4140	-0.1802\\
49.4440	-0.1984\\
49.4740	-0.1803\\
49.5040	-0.1841\\
49.5340	-0.1920\\
49.5640	-0.2238\\
49.5940	-0.2027\\
49.6140	-0.1834\\
49.6440	-0.1711\\
49.6740	-0.1991\\
49.7040	-0.2231\\
49.7340	-0.1173\\
49.7640	-0.1259\\
49.7940	-0.1180\\
49.8240	-0.1938\\
49.8540	-0.2319\\
49.8740	-0.2159\\
49.9040	-0.2065\\
49.9340	-0.2060\\
49.9640	-0.1849\\
49.9940	-0.2000\\
50.0240	-0.1994\\
50.0540	-0.3084\\
50.0840	-0.3107\\
50.1040	-0.2825\\
50.1340	-0.2225\\
50.1640	-0.2054\\
50.1940	-0.2097\\
50.2240	-0.1776\\
50.2540	-0.1343\\
50.2840	-0.1197\\
50.3140	-0.1248\\
50.3440	-0.2620\\
50.3640	-0.2688\\
50.3940	-0.2511\\
50.4240	-0.1707\\
50.4540	-0.1624\\
50.4840	-0.1618\\
50.5140	-0.1208\\
50.5440	-0.0960\\
50.5740	-0.1254\\
50.5940	-0.1045\\
50.6240	-0.0999\\
50.6540	-0.0333\\
50.6840	0.0722\\
50.7140	0.1355\\
50.7440	0.4325\\
50.7740	0.3698\\
50.8040	0.2787\\
50.8340	0.2362\\
50.8540	0.2275\\
50.8840	0.3263\\
50.9140	0.3370\\
50.9440	0.3134\\
50.9740	0.3073\\
51.0040	0.2942\\
51.0340	0.3203\\
51.0640	0.2913\\
51.0840	0.2698\\
51.1140	0.2956\\
51.1440	0.2798\\
51.1740	0.3273\\
51.2040	0.3091\\
51.2340	0.2899\\
51.2640	0.2751\\
51.2940	0.2313\\
51.3240	0.2194\\
51.3440	0.2301\\
51.3740	0.2208\\
51.4040	0.2365\\
51.4340	0.2319\\
51.4640	0.2466\\
51.4940	0.2429\\
51.5240	0.2154\\
51.5540	0.2075\\
51.5740	0.2265\\
51.6040	0.2437\\
51.6340	0.2557\\
51.6640	0.2399\\
51.6940	0.2088\\
51.7240	0.2420\\
51.7540	0.2496\\
51.7840	0.2712\\
51.8140	0.2461\\
51.8340	0.2173\\
51.8640	0.2199\\
51.8940	0.2248\\
51.9240	0.2377\\
51.9540	0.2344\\
51.9840	0.2150\\
52.0140	0.2897\\
52.0440	0.2999\\
52.0640	0.3058\\
52.0940	0.3478\\
52.1240	0.3832\\
52.1540	0.3764\\
52.1840	0.3117\\
52.2140	0.2992\\
52.2440	0.2733\\
52.2740	0.2758\\
52.2940	0.3511\\
52.3240	0.3426\\
52.3540	0.3228\\
52.3840	0.3189\\
52.4140	0.3213\\
52.4440	0.3649\\
52.4740	0.3416\\
52.5040	0.3310\\
52.5340	0.3011\\
52.5540	0.3049\\
52.5840	0.3163\\
52.6140	0.3059\\
52.6440	0.3408\\
52.6740	0.4033\\
52.7040	0.4092\\
52.7340	0.3826\\
52.7640	0.3447\\
52.7840	0.2959\\
52.8140	0.2874\\
52.8440	0.3578\\
52.8740	0.3467\\
52.9040	0.2994\\
52.9340	0.2939\\
52.9640	0.2794\\
52.9940	0.3443\\
53.0240	0.3663\\
53.0440	0.3697\\
53.0740	0.3156\\
53.1040	0.2754\\
53.1340	0.2210\\
53.1640	0.1414\\
53.1940	0.0714\\
53.2240	0.0922\\
53.2540	0.0583\\
53.2740	-0.0576\\
53.3040	-0.2304\\
53.3340	-0.4360\\
53.3640	-0.3642\\
53.3940	-0.5123\\
53.4240	-0.6748\\
53.4540	-0.6436\\
53.4840	-0.5630\\
53.5140	-0.5294\\
53.5340	-0.5068\\
53.5640	-0.5264\\
53.5940	-0.5469\\
53.6240	-0.5378\\
53.6540	-0.5212\\
53.6840	-0.4861\\
53.7140	-0.4849\\
53.7440	-0.4804\\
53.7640	-0.4763\\
53.7940	-0.4765\\
53.8240	-0.4947\\
53.8540	-0.4867\\
53.8840	-0.4591\\
53.9140	-0.4670\\
53.9440	-0.4348\\
53.9740	-0.4140\\
54.0040	-0.4512\\
54.0240	-0.4434\\
54.0540	-0.5461\\
54.0840	-0.4809\\
54.1140	-0.3886\\
54.1440	-0.3477\\
54.1740	-0.3547\\
54.2040	-0.3554\\
54.2340	-0.4286\\
54.2540	-0.4008\\
54.2840	-0.3576\\
54.3140	-0.3416\\
54.3440	-0.3261\\
54.3740	-0.3218\\
54.4040	-0.3164\\
54.4340	-0.2035\\
54.4640	-0.2119\\
54.4940	-0.2879\\
54.5140	-0.3110\\
54.5440	-0.3517\\
54.5740	-0.3801\\
54.6040	-0.3644\\
54.6340	-0.3308\\
54.6640	-0.2629\\
54.6940	-0.1851\\
54.7240	-0.4132\\
54.7440	-0.4735\\
54.7740	-0.4183\\
54.8040	-0.3714\\
54.8340	-0.3354\\
54.8640	-0.3465\\
54.8940	-0.2761\\
54.9240	-0.2380\\
54.9540	-0.2433\\
54.9840	-0.3030\\
55.0040	-0.2843\\
55.0340	-0.2459\\
55.0640	-0.3755\\
55.0940	-0.4104\\
55.1240	-0.3259\\
55.1540	-0.2021\\
55.1840	-0.1409\\
55.2140	-0.0913\\
55.2340	-0.0308\\
55.2640	-0.0376\\
55.2940	-0.0006\\
55.3240	0.1759\\
55.3540	0.2583\\
55.3840	0.1964\\
55.4140	0.2669\\
55.4440	0.3390\\
55.4740	0.3948\\
55.4940	0.3830\\
55.5240	0.4461\\
55.5540	0.4561\\
55.5840	0.4526\\
55.6140	0.4596\\
55.6440	0.5215\\
55.6740	0.4793\\
55.7040	0.5232\\
55.7240	0.4926\\
55.7540	0.4663\\
55.7840	0.5207\\
55.8140	0.5952\\
55.8440	0.5346\\
55.8740	0.5381\\
55.9040	0.5388\\
55.9340	0.4546\\
55.9540	0.5960\\
55.9840	0.6062\\
56.0140	0.5449\\
56.0440	0.4985\\
56.0740	0.4744\\
56.1040	0.4391\\
56.1340	0.4635\\
56.1640	0.4512\\
56.1940	0.4875\\
56.2140	0.5051\\
56.2440	0.4196\\
56.2740	0.4382\\
56.3040	0.5418\\
56.3340	0.5425\\
56.3640	0.5234\\
56.3940	0.5571\\
56.4240	0.5264\\
56.4440	0.5341\\
56.4740	0.5436\\
56.5040	0.5727\\
56.5340	0.5088\\
56.5640	0.5279\\
56.5940	0.5260\\
56.6240	0.5404\\
56.6540	0.5154\\
56.6840	0.4414\\
56.7040	0.4077\\
56.7340	0.3943\\
56.7640	0.4225\\
56.7940	0.4902\\
56.8240	0.5744\\
56.8540	0.4556\\
56.8840	0.4016\\
56.9140	0.2949\\
56.9340	0.2289\\
56.9640	0.0165\\
56.9940	-0.2915\\
57.0240	-0.3106\\
57.0540	-0.2940\\
57.0840	-0.4263\\
57.1140	-0.5725\\
57.1440	-0.6566\\
57.1740	-0.8018\\
57.1940	-0.8645\\
57.2240	-0.8807\\
57.2540	-0.9127\\
57.2840	-0.9457\\
57.3140	-0.9254\\
57.3440	-0.9341\\
57.3740	-0.9392\\
57.4040	-0.7822\\
57.4240	-0.8136\\
57.4540	-0.8337\\
57.4840	-0.8107\\
57.5140	-0.8418\\
57.5440	-0.7926\\
57.5740	-0.7433\\
57.6040	-0.7587\\
57.6340	-0.7283\\
57.6640	-0.7439\\
57.6840	-0.7378\\
57.7140	-0.7688\\
57.7440	-0.7917\\
57.7740	-0.8131\\
57.8040	-0.6974\\
57.8340	-0.6654\\
57.8640	-0.6411\\
57.8940	-0.6107\\
57.9140	-0.5922\\
57.9440	-0.5627\\
57.9740	-0.5963\\
58.0040	-0.6317\\
58.0340	-0.5902\\
58.0640	-0.5146\\
58.0940	-0.5200\\
58.1240	-0.5601\\
58.1540	-0.4221\\
58.1740	-0.3317\\
58.2040	-0.2743\\
58.2340	-0.2763\\
58.2640	-0.3233\\
58.2940	-0.5065\\
58.3240	-0.6920\\
58.3540	-0.6086\\
58.3840	-0.4640\\
58.4040	-0.3556\\
58.4340	-0.2562\\
58.4640	-0.1933\\
58.4940	-0.1559\\
58.5240	-0.1377\\
58.5540	-0.0048\\
58.5840	0.0775\\
58.6140	0.1418\\
58.6440	0.2722\\
58.6640	0.3174\\
58.6940	0.3872\\
58.7240	0.4231\\
58.7540	0.3667\\
58.7840	0.5731\\
58.8140	0.6442\\
58.8440	0.7717\\
58.8740	0.8479\\
58.8940	0.7672\\
58.9240	0.8213\\
58.9540	0.8076\\
58.9840	0.7602\\
59.0140	0.7715\\
59.0440	0.7620\\
59.0740	0.7316\\
59.1040	0.7985\\
59.1340	0.7979\\
59.1540	0.8319\\
59.1840	0.8834\\
59.2140	0.8012\\
59.2440	0.9394\\
59.2740	0.9310\\
59.3040	0.9250\\
59.3340	0.9731\\
59.3640	0.9778\\
59.3840	0.9743\\
59.4140	0.9816\\
59.4440	1.1196\\
59.4740	1.2024\\
59.5040	1.2117\\
59.5340	1.0816\\
59.5640	1.0191\\
59.5940	1.0647\\
59.6140	1.0107\\
59.6440	0.8720\\
59.6740	0.7960\\
59.7040	0.7625\\
59.7340	0.6566\\
59.7640	0.6099\\
59.7940	0.3297\\
59.8240	-0.1142\\
59.8540	-0.6651\\
59.8740	-0.8714\\
59.9040	-1.0096\\
59.9340	-1.1251\\
59.9640	-1.2451\\
59.9940	-1.3062\\
60.0240	-1.2398\\
60.0540	-1.4435\\
60.0840	-1.3920\\
60.1040	-1.4277\\
60.1340	-1.4998\\
60.1640	-1.5585\\
60.1940	-1.4465\\
60.2240	-1.2556\\
60.2540	-1.1094\\
60.2840	-1.0495\\
60.3140	-1.0416\\
60.3440	-0.9658\\
60.3640	-0.9222\\
60.3940	-0.9139\\
60.4240	-0.9246\\
60.4540	-0.8929\\
60.4840	-0.9037\\
60.5140	-0.8426\\
60.5440	-0.8061\\
60.5740	-0.8246\\
60.5940	-0.7390\\
60.6240	-0.6441\\
60.6540	-0.5873\\
60.6840	-0.4857\\
60.7140	-0.3541\\
60.7440	-0.3521\\
60.7740	-0.3807\\
60.8040	-0.3443\\
60.8340	-0.2333\\
60.8540	-0.2204\\
60.8840	-0.1634\\
60.9140	-0.1060\\
60.9440	-0.0451\\
60.9740	0.3144\\
61.0040	0.4822\\
61.0340	0.4970\\
61.0640	0.5302\\
61.0840	0.6736\\
61.1140	0.7504\\
61.1440	0.7067\\
61.1740	0.7773\\
61.2040	0.8312\\
61.2340	0.8746\\
61.2640	0.9739\\
61.2940	0.9243\\
61.3240	0.8886\\
61.3440	0.8863\\
61.3740	0.9338\\
61.4040	0.8662\\
61.4340	0.9777\\
61.4640	1.0693\\
61.4940	1.1619\\
61.5240	1.1691\\
61.5540	1.2177\\
61.5740	1.1907\\
61.6040	1.1889\\
61.6340	1.1763\\
61.6640	1.1953\\
61.6940	1.2796\\
61.7240	1.3701\\
61.7540	1.1492\\
61.7840	1.0753\\
61.8140	0.9075\\
61.8340	0.8620\\
61.8640	0.7809\\
61.8940	0.6580\\
61.9240	0.5769\\
61.9540	0.4147\\
61.9840	-0.0020\\
62.0140	-0.5833\\
62.0440	-0.9744\\
62.0640	-1.0711\\
62.0940	-1.0614\\
62.1240	-1.2004\\
62.1540	-1.2129\\
62.1840	-1.3438\\
62.2140	-1.1397\\
62.2440	-1.0924\\
62.2740	-1.2858\\
62.3040	-1.1991\\
62.3240	-1.1513\\
62.3540	-1.2547\\
62.3840	-1.2372\\
62.4140	-1.1783\\
62.4440	-1.0076\\
62.4740	-1.0338\\
62.5040	-1.0018\\
62.5340	-1.0283\\
62.5540	-0.9738\\
62.5840	-0.7941\\
62.6140	-0.7510\\
62.6440	-0.6517\\
62.6740	-0.7263\\
62.7040	-0.6779\\
62.7340	-0.6140\\
62.7640	-0.5400\\
62.7940	-0.5103\\
};
\end{axis}

\begin{axis}[%
width=0.411\fwidth,
height=0.232\fwidth,
at={(0\fwidth,0.323\fwidth)},
scale only axis,
xmin=40.0040,
xmax=46.1885,
ymin=-0.2491,
ymax=0.3232,
ylabel={$\vartheta_z$, $\hat{\vartheta}_z$ [1/s]},
axis background/.style={fill=white},
title style={font=\labelsize},
xlabel style={font=\labelsize,at={(axis description cs:0.5,\xlabeldist)}},
ylabel style={font=\labelsize,at={(axis description cs:\ylabeldist,0.5)}},
legend style={font=\ticksize},
ticklabel style={font=\ticksize}
]
\addplot [color=mycolor1,solid,forget plot]
  table[row sep=crcr]{%
40.0100	-0.0934\\
40.0204	-0.0912\\
40.0307	-0.0903\\
40.0414	-0.0906\\
40.0523	-0.0918\\
40.0632	-0.0936\\
40.0738	-0.0953\\
40.0842	-0.0963\\
40.0953	-0.0960\\
40.1059	-0.0947\\
40.1163	-0.0929\\
40.1267	-0.0916\\
40.1376	-0.0918\\
40.1484	-0.0937\\
40.1593	-0.0975\\
40.1702	-0.1030\\
40.1807	-0.1110\\
40.1913	-0.1193\\
40.2021	-0.1285\\
40.2131	-0.1337\\
40.2238	-0.1350\\
40.2349	-0.1323\\
40.2454	-0.1252\\
40.2558	-0.1168\\
40.2665	-0.1110\\
40.2769	-0.1081\\
40.2878	-0.1063\\
40.2985	-0.1056\\
40.3093	-0.1033\\
40.3205	-0.0991\\
40.3308	-0.0957\\
40.3417	-0.0946\\
40.3527	-0.0960\\
40.3631	-0.0987\\
40.3731	-0.1027\\
40.3837	-0.1057\\
40.3945	-0.1076\\
40.4047	-0.1072\\
40.4156	-0.1053\\
40.4260	-0.1019\\
40.4367	-0.0972\\
40.4475	-0.0940\\
40.4583	-0.0937\\
40.4685	-0.0967\\
40.4789	-0.1024\\
40.4895	-0.1106\\
40.5002	-0.1179\\
40.5114	-0.1247\\
40.5215	-0.1290\\
40.5322	-0.1307\\
40.5428	-0.1307\\
40.5540	-0.1289\\
40.5644	-0.1242\\
40.5749	-0.1182\\
40.5859	-0.1114\\
40.5966	-0.1063\\
40.6075	-0.1031\\
40.6184	-0.1029\\
40.6292	-0.1055\\
40.6393	-0.1114\\
40.6498	-0.1173\\
40.6604	-0.1236\\
40.6713	-0.1275\\
40.6818	-0.1287\\
40.6926	-0.1271\\
40.7028	-0.1229\\
40.7134	-0.1176\\
40.7243	-0.1142\\
40.7349	-0.1127\\
40.7454	-0.1133\\
40.7557	-0.1162\\
40.7665	-0.1197\\
40.7770	-0.1239\\
40.7881	-0.1270\\
40.7988	-0.1290\\
40.8093	-0.1294\\
40.8197	-0.1281\\
40.8303	-0.1255\\
40.8407	-0.1230\\
40.8518	-0.1209\\
40.8623	-0.1193\\
40.8728	-0.1172\\
40.8837	-0.1145\\
40.8948	-0.1113\\
40.9059	-0.1069\\
40.9166	-0.1008\\
40.9270	-0.0943\\
40.9374	-0.0876\\
40.9479	-0.0807\\
40.9582	-0.0746\\
40.9687	-0.0692\\
40.9791	-0.0664\\
40.9895	-0.0658\\
41.0005	-0.0671\\
41.0113	-0.0690\\
41.0218	-0.0712\\
41.0320	-0.0720\\
41.0427	-0.0723\\
41.0537	-0.0723\\
41.0641	-0.0732\\
41.0751	-0.0754\\
41.0853	-0.0789\\
41.0958	-0.0824\\
41.1064	-0.0864\\
41.1173	-0.0905\\
41.1283	-0.0946\\
41.1387	-0.0979\\
41.1488	-0.1000\\
41.1597	-0.1011\\
41.1704	-0.1010\\
41.1813	-0.0989\\
41.1920	-0.0956\\
41.2029	-0.0914\\
41.2138	-0.0867\\
41.2241	-0.0814\\
41.2351	-0.0766\\
41.2453	-0.0724\\
41.2560	-0.0705\\
41.2663	-0.0702\\
41.2767	-0.0716\\
41.2871	-0.0749\\
41.2978	-0.0788\\
41.3086	-0.0817\\
41.3189	-0.0837\\
41.3292	-0.0846\\
41.3395	-0.0846\\
41.3503	-0.0845\\
41.3605	-0.0854\\
41.3711	-0.0857\\
41.3815	-0.0854\\
41.3920	-0.0844\\
41.4030	-0.0828\\
41.4138	-0.0798\\
41.4248	-0.0772\\
41.4355	-0.0751\\
41.4459	-0.0733\\
41.4563	-0.0723\\
41.4666	-0.0717\\
41.4774	-0.0714\\
41.4883	-0.0715\\
41.4991	-0.0720\\
41.5096	-0.0725\\
41.5206	-0.0731\\
41.5314	-0.0747\\
41.5420	-0.0772\\
41.5531	-0.0809\\
41.5638	-0.0834\\
41.5746	-0.0850\\
41.5850	-0.0847\\
41.5953	-0.0824\\
41.6063	-0.0795\\
41.6169	-0.0778\\
41.6275	-0.0773\\
41.6384	-0.0772\\
41.6491	-0.0771\\
41.6597	-0.0758\\
41.6706	-0.0734\\
41.6810	-0.0711\\
41.6913	-0.0701\\
41.7021	-0.0702\\
41.7127	-0.0717\\
41.7230	-0.0767\\
41.7339	-0.0841\\
41.7450	-0.0940\\
41.7552	-0.1028\\
41.7657	-0.1105\\
41.7766	-0.1152\\
41.7876	-0.1165\\
41.7982	-0.1163\\
41.8088	-0.1180\\
41.8193	-0.1217\\
41.8304	-0.1238\\
41.8411	-0.1242\\
41.8513	-0.1215\\
41.8619	-0.1155\\
41.8731	-0.1096\\
41.8836	-0.1070\\
41.8940	-0.1083\\
41.9043	-0.1129\\
41.9147	-0.1180\\
41.9257	-0.1199\\
41.9366	-0.1192\\
41.9472	-0.1162\\
41.9579	-0.1110\\
41.9681	-0.1067\\
41.9789	-0.1031\\
41.9896	-0.1014\\
42.0004	-0.1004\\
42.0111	-0.1002\\
42.0219	-0.0995\\
42.0327	-0.0984\\
42.0432	-0.0960\\
42.0534	-0.0922\\
42.0640	-0.0877\\
42.0746	-0.0838\\
42.0856	-0.0807\\
42.0962	-0.0782\\
42.1067	-0.0780\\
42.1172	-0.0794\\
42.1278	-0.0826\\
42.1389	-0.0853\\
42.1500	-0.0870\\
42.1602	-0.0863\\
42.1706	-0.0833\\
42.1815	-0.0791\\
42.1921	-0.0764\\
42.2029	-0.0753\\
42.2131	-0.0755\\
42.2237	-0.0779\\
42.2349	-0.0805\\
42.2453	-0.0836\\
42.2555	-0.0861\\
42.2660	-0.0878\\
42.2767	-0.0883\\
42.2877	-0.0877\\
42.2983	-0.0861\\
42.3092	-0.0846\\
42.3195	-0.0832\\
42.3300	-0.0821\\
42.3401	-0.0820\\
42.3506	-0.0829\\
42.3614	-0.0846\\
42.3721	-0.0856\\
42.3829	-0.0855\\
42.3935	-0.0835\\
42.4045	-0.0797\\
42.4149	-0.0741\\
42.4258	-0.0685\\
42.4368	-0.0630\\
42.4476	-0.0566\\
42.4582	-0.0491\\
42.4692	-0.0406\\
42.4798	-0.0310\\
42.4902	-0.0213\\
42.5013	-0.0127\\
42.5122	-0.0052\\
42.5232	0.0031\\
42.5337	0.0122\\
42.5448	0.0231\\
42.5558	0.0339\\
42.5660	0.0446\\
42.5768	0.0535\\
42.5880	0.0604\\
42.5982	0.0684\\
42.6092	0.0795\\
42.6202	0.0933\\
42.6308	0.1061\\
42.6414	0.1176\\
42.6518	0.1249\\
42.6623	0.1280\\
42.6734	0.1273\\
42.6842	0.1276\\
42.6947	0.1284\\
42.7057	0.1290\\
42.7161	0.1295\\
42.7268	0.1289\\
42.7373	0.1276\\
42.7475	0.1257\\
42.7581	0.1241\\
42.7687	0.1225\\
42.7798	0.1211\\
42.7900	0.1201\\
42.8003	0.1192\\
42.8113	0.1183\\
42.8219	0.1160\\
42.8327	0.1121\\
42.8436	0.1067\\
42.8546	0.0994\\
42.8652	0.0927\\
42.8759	0.0878\\
42.8867	0.0848\\
42.8975	0.0826\\
42.9082	0.0810\\
42.9193	0.0781\\
42.9300	0.0740\\
42.9405	0.0714\\
42.9513	0.0715\\
42.9624	0.0744\\
42.9736	0.0786\\
42.9838	0.0841\\
42.9947	0.0882\\
43.0056	0.0898\\
43.0165	0.0892\\
43.0272	0.0878\\
43.0378	0.0855\\
43.0485	0.0842\\
43.0593	0.0846\\
43.0704	0.0867\\
43.0813	0.0892\\
43.0917	0.0923\\
43.1026	0.0941\\
43.1135	0.0945\\
43.1242	0.0937\\
43.1352	0.0927\\
43.1458	0.0917\\
43.1566	0.0914\\
43.1675	0.0922\\
43.1782	0.0939\\
43.1893	0.0965\\
43.1996	0.1002\\
43.2107	0.1036\\
43.2217	0.1070\\
43.2319	0.1070\\
43.2429	0.1038\\
43.2536	0.0972\\
43.2646	0.0905\\
43.2751	0.0840\\
43.2859	0.0808\\
43.2961	0.0810\\
43.3070	0.0867\\
43.3172	0.0948\\
43.3283	0.1050\\
43.3392	0.1121\\
43.3497	0.1162\\
43.3609	0.1150\\
43.3712	0.1086\\
43.3821	0.1005\\
43.3931	0.0961\\
43.4041	0.0953\\
43.4144	0.0967\\
43.4244	0.1003\\
43.4356	0.1023\\
43.4458	0.1027\\
43.4568	0.1036\\
43.4676	0.1066\\
43.4782	0.1117\\
43.4890	0.1148\\
43.4999	0.1165\\
43.5105	0.1142\\
43.5213	0.1079\\
43.5321	0.0997\\
43.5429	0.0934\\
43.5532	0.0886\\
43.5634	0.0856\\
43.5737	0.0858\\
43.5846	0.0876\\
43.5948	0.0908\\
43.6052	0.0955\\
43.6156	0.0987\\
43.6263	0.0994\\
43.6370	0.0972\\
43.6480	0.0934\\
43.6585	0.0881\\
43.6695	0.0840\\
43.6802	0.0811\\
43.6909	0.0807\\
43.7017	0.0818\\
43.7127	0.0840\\
43.7234	0.0873\\
43.7336	0.0913\\
43.7439	0.0948\\
43.7545	0.0977\\
43.7651	0.0998\\
43.7754	0.1015\\
43.7862	0.1029\\
43.7970	0.1040\\
43.8077	0.1056\\
43.8185	0.1079\\
43.8295	0.1110\\
43.8401	0.1133\\
43.8505	0.1146\\
43.8606	0.1142\\
43.8710	0.1122\\
43.8812	0.1102\\
43.8921	0.1103\\
43.9032	0.1120\\
43.9136	0.1154\\
43.9243	0.1222\\
43.9352	0.1308\\
43.9462	0.1405\\
43.9564	0.1479\\
43.9674	0.1531\\
43.9776	0.1541\\
43.9885	0.1511\\
43.9991	0.1445\\
44.0099	0.1383\\
44.0206	0.1325\\
44.0318	0.1286\\
44.0424	0.1273\\
44.0527	0.1274\\
44.0637	0.1293\\
44.0744	0.1355\\
44.0851	0.1442\\
44.0954	0.1553\\
44.1064	0.1621\\
44.1174	0.1648\\
44.1278	0.1602\\
44.1384	0.1489\\
44.1493	0.1357\\
44.1599	0.1273\\
44.1700	0.1238\\
44.1807	0.1251\\
44.1911	0.1308\\
44.2019	0.1366\\
44.2129	0.1421\\
44.2236	0.1444\\
44.2345	0.1435\\
44.2454	0.1397\\
44.2561	0.1347\\
44.2672	0.1291\\
44.2778	0.1252\\
44.2887	0.1234\\
44.2992	0.1235\\
44.3098	0.1236\\
44.3207	0.1260\\
44.3316	0.1290\\
44.3424	0.1332\\
44.3533	0.1381\\
44.3638	0.1438\\
44.3743	0.1454\\
44.3850	0.1451\\
44.3954	0.1423\\
44.4058	0.1373\\
44.4161	0.1315\\
44.4265	0.1281\\
44.4370	0.1261\\
44.4478	0.1262\\
44.4579	0.1297\\
44.4687	0.1345\\
44.4795	0.1407\\
44.4898	0.1469\\
44.5007	0.1531\\
44.5112	0.1579\\
44.5214	0.1613\\
44.5318	0.1605\\
44.5424	0.1573\\
44.5533	0.1507\\
44.5640	0.1415\\
44.5744	0.1338\\
44.5847	0.1305\\
44.5955	0.1316\\
44.6058	0.1370\\
44.6160	0.1444\\
44.6266	0.1495\\
44.6372	0.1523\\
44.6473	0.1526\\
44.6580	0.1505\\
44.6683	0.1490\\
44.6786	0.1476\\
44.6893	0.1469\\
44.7000	0.1472\\
44.7103	0.1482\\
44.7209	0.1499\\
44.7318	0.1515\\
44.7425	0.1531\\
44.7527	0.1541\\
44.7632	0.1546\\
44.7738	0.1524\\
44.7844	0.1482\\
44.7955	0.1419\\
44.8061	0.1374\\
44.8165	0.1353\\
44.8269	0.1369\\
44.8378	0.1429\\
44.8484	0.1516\\
44.8591	0.1592\\
44.8697	0.1652\\
44.8799	0.1700\\
44.8908	0.1720\\
44.9011	0.1728\\
44.9122	0.1719\\
44.9228	0.1708\\
44.9328	0.1702\\
44.9437	0.1706\\
44.9548	0.1727\\
44.9656	0.1746\\
44.9760	0.1750\\
44.9865	0.1734\\
44.9974	0.1699\\
45.0080	0.1653\\
45.0190	0.1614\\
45.0298	0.1578\\
45.0405	0.1550\\
45.0516	0.1537\\
45.0622	0.1529\\
45.0726	0.1526\\
45.0827	0.1530\\
45.0936	0.1535\\
45.1043	0.1562\\
45.1152	0.1608\\
45.1262	0.1683\\
45.1372	0.1778\\
45.1474	0.1898\\
45.1579	0.1998\\
45.1687	0.2074\\
45.1790	0.2124\\
45.1902	0.2128\\
45.2005	0.2094\\
45.2115	0.2041\\
45.2223	0.1965\\
45.2329	0.1888\\
45.2436	0.1839\\
45.2539	0.1811\\
45.2644	0.1802\\
45.2746	0.1854\\
45.2854	0.1953\\
45.2959	0.2090\\
45.3069	0.2188\\
45.3173	0.2258\\
45.3280	0.2241\\
45.3388	0.2137\\
45.3490	0.1987\\
45.3597	0.1868\\
45.3703	0.1783\\
45.3812	0.1728\\
45.3917	0.1720\\
45.4024	0.1725\\
45.4127	0.1736\\
45.4236	0.1751\\
45.4342	0.1767\\
45.4448	0.1775\\
45.4558	0.1777\\
45.4664	0.1779\\
45.4771	0.1792\\
45.4881	0.1815\\
45.4989	0.1864\\
45.5096	0.1946\\
45.5200	0.2040\\
45.5310	0.2151\\
45.5417	0.2214\\
45.5527	0.2215\\
45.5635	0.2146\\
45.5739	0.2067\\
45.5850	0.1978\\
45.5958	0.1947\\
45.6068	0.1948\\
45.6174	0.1987\\
45.6281	0.2001\\
45.6388	0.1991\\
45.6493	0.1957\\
45.6595	0.1937\\
45.6697	0.1914\\
45.6800	0.1890\\
45.6905	0.1876\\
45.7012	0.1866\\
45.7116	0.1860\\
45.7226	0.1856\\
45.7328	0.1858\\
45.7430	0.1861\\
45.7533	0.1828\\
45.7636	0.1757\\
45.7744	0.1712\\
45.7845	0.1701\\
45.7953	0.1714\\
45.8056	0.1793\\
45.8158	0.1891\\
45.8263	0.1948\\
45.8370	0.1958\\
45.8481	0.1920\\
45.8584	0.1839\\
45.8692	0.1759\\
45.8794	0.1682\\
45.8904	0.1646\\
45.9005	0.1666\\
45.9109	0.1725\\
45.9218	0.1833\\
45.9323	0.1938\\
45.9433	0.1988\\
45.9539	0.1985\\
45.9646	0.1928\\
45.9752	0.1882\\
45.9862	0.1912\\
45.9971	0.2008\\
46.0082	0.2106\\
46.0189	0.2213\\
46.0294	0.2250\\
46.0398	0.2226\\
46.0509	0.2156\\
46.0616	0.2093\\
46.0718	0.2040\\
46.0824	0.2000\\
46.0931	0.1961\\
46.1039	0.1926\\
46.1145	0.1887\\
46.1248	0.1858\\
46.1353	0.1839\\
46.1459	0.1821\\
46.1570	0.1790\\
46.1676	0.1743\\
46.1782	0.1672\\
46.1885	0.1583\\
};
\addplot [color=mycolor2,solid,forget plot]
  table[row sep=crcr]{%
40.0040	-0.2046\\
40.0140	-0.2491\\
40.0240	-0.2387\\
40.0340	-0.2295\\
40.0440	-0.2190\\
40.0540	-0.2121\\
40.0640	-0.2187\\
40.0740	-0.2318\\
40.0840	-0.2258\\
40.0940	-0.2217\\
40.1040	-0.2217\\
40.1140	-0.2217\\
40.1240	-0.2217\\
40.1340	-0.1838\\
40.1440	-0.1838\\
40.1540	-0.1838\\
40.1640	-0.1532\\
40.1740	-0.1569\\
40.1840	-0.1562\\
40.1940	-0.1590\\
40.2040	-0.1640\\
40.2140	-0.1643\\
40.2240	-0.1700\\
40.2340	-0.1690\\
40.2440	-0.1720\\
40.2540	-0.1708\\
40.2640	-0.1601\\
40.2740	-0.1439\\
40.2840	-0.1460\\
40.2940	-0.1376\\
40.3040	-0.1359\\
40.3140	-0.1341\\
40.3240	-0.1313\\
40.3340	-0.0988\\
40.3440	-0.0893\\
40.3540	-0.0944\\
40.3640	-0.0966\\
40.3740	-0.1079\\
40.3840	-0.1362\\
40.3940	-0.1599\\
40.4040	-0.1592\\
40.4140	-0.1559\\
40.4240	-0.1481\\
40.4340	-0.1457\\
40.4440	-0.1407\\
40.4540	-0.1375\\
40.4640	-0.1363\\
40.4740	-0.1342\\
40.4840	-0.1377\\
40.4940	-0.1364\\
40.5040	-0.1355\\
40.5140	-0.1319\\
40.5240	-0.1335\\
40.5340	-0.1364\\
40.5440	-0.1444\\
40.5540	-0.1508\\
40.5640	-0.1551\\
40.5740	-0.1645\\
40.5840	-0.1653\\
40.5940	-0.1696\\
40.6040	-0.1653\\
40.6140	-0.1662\\
40.6240	-0.1582\\
40.6340	-0.1516\\
40.6440	-0.1734\\
40.6540	-0.1760\\
40.6640	-0.1740\\
40.6740	-0.1741\\
40.6840	-0.1646\\
40.6940	-0.1547\\
40.7040	-0.1391\\
40.7140	-0.1307\\
40.7240	-0.1268\\
40.7340	-0.1268\\
40.7440	-0.1220\\
40.7540	-0.0980\\
40.7640	-0.0929\\
40.7740	-0.0928\\
40.7840	-0.0793\\
40.7940	-0.0835\\
40.8040	-0.0847\\
40.8140	-0.0736\\
40.8240	-0.0729\\
40.8340	-0.1107\\
40.8440	-0.1115\\
40.8540	-0.1062\\
40.8640	-0.1117\\
40.8740	-0.1323\\
40.8840	-0.1279\\
40.8940	-0.1743\\
40.9040	-0.1693\\
40.9140	-0.1642\\
40.9240	-0.1535\\
40.9340	-0.1528\\
40.9440	-0.1832\\
40.9540	-0.1865\\
40.9640	-0.1834\\
40.9740	-0.1754\\
40.9840	-0.1710\\
40.9940	-0.1563\\
41.0040	-0.1425\\
41.0140	-0.1394\\
41.0240	-0.1394\\
41.0340	-0.1394\\
41.0440	-0.1394\\
41.0540	-0.1394\\
41.0640	-0.1394\\
41.0740	-0.1394\\
41.0840	-0.1215\\
41.0940	-0.1109\\
41.1040	-0.1056\\
41.1140	-0.0741\\
41.1240	-0.0685\\
41.1340	-0.0638\\
41.1440	-0.0611\\
41.1540	-0.0681\\
41.1640	-0.0681\\
41.1740	-0.0681\\
41.1840	-0.0681\\
41.1940	-0.0681\\
41.2040	-0.0601\\
41.2140	-0.0687\\
41.2240	-0.0703\\
41.2340	-0.0793\\
41.2440	-0.0793\\
41.2540	-0.0759\\
41.2640	-0.0787\\
41.2740	-0.0823\\
41.2840	-0.0740\\
41.2940	-0.0638\\
41.3040	-0.0632\\
41.3140	-0.0627\\
41.3240	-0.0627\\
41.3340	-0.0654\\
41.3440	-0.0674\\
41.3540	-0.0703\\
41.3640	-0.0703\\
41.3740	-0.0703\\
41.3840	-0.0690\\
41.3940	-0.0815\\
41.4040	-0.1072\\
41.4140	-0.1367\\
41.4240	-0.1596\\
41.4340	-0.1544\\
41.4440	-0.1520\\
41.4540	-0.1520\\
41.4640	-0.1404\\
41.4740	-0.1156\\
41.4840	-0.1098\\
41.4940	-0.1244\\
41.5040	-0.1205\\
41.5140	-0.1146\\
41.5240	-0.1043\\
41.5340	-0.1126\\
41.5440	-0.1126\\
41.5540	-0.1111\\
41.5640	-0.0984\\
41.5740	-0.1151\\
41.5840	-0.1166\\
41.5940	-0.1166\\
41.6040	-0.1059\\
41.6140	-0.0934\\
41.6240	-0.0982\\
41.6340	-0.0924\\
41.6440	-0.0857\\
41.6540	-0.0757\\
41.6640	-0.0646\\
41.6740	-0.0343\\
41.6840	-0.0442\\
41.6940	-0.0410\\
41.7040	-0.0363\\
41.7140	0.0037\\
41.7240	0.0482\\
41.7340	0.0484\\
41.7440	0.0485\\
41.7540	0.0835\\
41.7640	0.0859\\
41.7740	0.0823\\
41.7840	0.0817\\
41.7940	0.0508\\
41.8040	0.0337\\
41.8140	0.0236\\
41.8240	0.0076\\
41.8340	-0.0189\\
41.8440	-0.0331\\
41.8540	-0.0680\\
41.8640	-0.0786\\
41.8740	-0.0790\\
41.8840	-0.0918\\
41.8940	-0.1086\\
41.9040	-0.1518\\
41.9140	-0.1924\\
41.9240	-0.2080\\
41.9340	-0.1957\\
41.9440	-0.1879\\
41.9540	-0.1879\\
41.9640	-0.1588\\
41.9740	-0.1449\\
41.9840	-0.1444\\
41.9940	-0.1295\\
42.0040	-0.1113\\
42.0140	-0.0977\\
42.0240	-0.0865\\
42.0340	-0.0883\\
42.0440	-0.0886\\
42.0540	-0.0907\\
42.0640	-0.0840\\
42.0740	-0.1035\\
42.0840	-0.0980\\
42.0940	-0.0929\\
42.1040	-0.1192\\
42.1140	-0.1228\\
42.1240	-0.1273\\
42.1340	-0.1254\\
42.1440	-0.1209\\
42.1540	-0.1148\\
42.1640	-0.1086\\
42.1740	-0.1054\\
42.1840	-0.1035\\
42.1940	-0.1025\\
42.2040	-0.1037\\
42.2140	-0.1037\\
42.2240	-0.1037\\
42.2340	-0.1037\\
42.2440	-0.1037\\
42.2540	-0.1037\\
42.2640	-0.1037\\
42.2740	-0.1047\\
42.2840	-0.1047\\
42.2940	-0.1047\\
42.3040	-0.1068\\
42.3140	-0.1013\\
42.3240	-0.0888\\
42.3340	-0.0827\\
42.3440	-0.0780\\
42.3540	-0.0707\\
42.3640	-0.0724\\
42.3740	-0.0736\\
42.3840	-0.0791\\
42.3940	-0.0818\\
42.4040	-0.0816\\
42.4140	-0.0794\\
42.4240	-0.0828\\
42.4340	-0.0926\\
42.4440	-0.1111\\
42.4540	-0.1115\\
42.4640	-0.0926\\
42.4740	-0.0878\\
42.4840	-0.0864\\
42.4940	-0.0839\\
42.5040	-0.0741\\
42.5140	-0.0697\\
42.5240	-0.0695\\
42.5340	-0.0628\\
42.5440	-0.0615\\
42.5540	-0.0773\\
42.5640	-0.0860\\
42.5740	-0.0817\\
42.5840	-0.0773\\
42.5940	-0.0709\\
42.6040	-0.0725\\
42.6140	-0.0710\\
42.6240	-0.0601\\
42.6340	-0.0518\\
42.6440	-0.0369\\
42.6540	-0.0365\\
42.6640	-0.0337\\
42.6740	-0.0314\\
42.6840	-0.0314\\
42.6940	-0.0319\\
42.7040	-0.0319\\
42.7140	-0.0319\\
42.7240	-0.0319\\
42.7340	-0.0905\\
42.7440	-0.1578\\
42.7540	-0.1845\\
42.7640	-0.1729\\
42.7740	-0.1683\\
42.7840	-0.1507\\
42.7940	-0.1507\\
42.8040	-0.1507\\
42.8140	-0.1080\\
42.8240	-0.1080\\
42.8340	-0.1080\\
42.8440	-0.1080\\
42.8540	-0.1080\\
42.8640	-0.1080\\
42.8740	-0.1080\\
42.8840	-0.1080\\
42.8940	-0.1080\\
42.9040	-0.1080\\
42.9140	-0.1080\\
42.9240	-0.0415\\
42.9340	-0.0135\\
42.9440	0.0040\\
42.9540	-0.0042\\
42.9640	-0.0139\\
42.9740	-0.0130\\
42.9840	-0.0119\\
42.9940	-0.0323\\
43.0040	-0.0573\\
43.0140	-0.0006\\
43.0240	-0.0006\\
43.0340	0.0039\\
43.0440	0.0390\\
43.0540	0.0480\\
43.0640	0.0589\\
43.0740	0.0589\\
43.0840	0.1011\\
43.0940	0.1011\\
43.1040	0.1011\\
43.1140	0.1006\\
43.1240	0.0973\\
43.1340	0.0983\\
43.1440	0.1103\\
43.1540	0.1239\\
43.1640	0.1304\\
43.1740	0.1338\\
43.1840	0.1323\\
43.1940	0.1323\\
43.2040	0.1323\\
43.2140	0.1475\\
43.2240	0.1338\\
43.2340	0.1223\\
43.2440	0.1211\\
43.2540	0.1117\\
43.2640	0.1044\\
43.2740	0.0862\\
43.2840	0.0717\\
43.2940	0.0714\\
43.3040	0.0714\\
43.3140	0.0715\\
43.3240	0.0854\\
43.3340	0.1085\\
43.3440	0.1109\\
43.3540	0.1208\\
43.3640	0.1167\\
43.3740	0.1162\\
43.3840	0.1156\\
43.3940	0.1151\\
43.4040	0.1151\\
43.4140	0.1373\\
43.4240	0.1133\\
43.4340	0.0913\\
43.4440	0.0806\\
43.4540	0.1019\\
43.4640	0.1054\\
43.4740	0.1083\\
43.4840	0.1040\\
43.4940	0.0988\\
43.5040	0.0996\\
43.5140	0.0910\\
43.5240	0.0861\\
43.5340	0.0871\\
43.5440	0.0843\\
43.5540	0.0799\\
43.5640	0.0726\\
43.5740	0.0735\\
43.5840	0.0816\\
43.5940	0.0888\\
43.6040	0.0900\\
43.6140	0.1068\\
43.6240	0.1267\\
43.6340	0.1335\\
43.6440	0.1552\\
43.6540	0.1937\\
43.6640	0.2186\\
43.6740	0.2223\\
43.6840	0.2100\\
43.6940	0.1928\\
43.7040	0.1835\\
43.7140	0.1801\\
43.7240	0.1861\\
43.7340	0.1875\\
43.7440	0.1856\\
43.7540	0.1722\\
43.7640	0.1624\\
43.7740	0.1624\\
43.7840	0.1624\\
43.7940	0.1874\\
43.8040	0.1876\\
43.8140	0.2033\\
43.8240	0.2033\\
43.8340	0.1833\\
43.8440	0.1772\\
43.8540	0.1772\\
43.8640	0.1613\\
43.8740	0.1472\\
43.8840	0.1499\\
43.8940	0.1406\\
43.9040	0.1406\\
43.9140	0.1554\\
43.9240	0.1450\\
43.9340	0.0271\\
43.9440	0.0265\\
43.9540	0.0471\\
43.9640	0.0978\\
43.9740	0.1022\\
43.9840	0.1187\\
43.9940	0.1293\\
44.0040	0.1610\\
44.0140	0.1785\\
44.0240	0.1820\\
44.0340	0.1755\\
44.0440	0.1800\\
44.0540	0.1775\\
44.0640	0.1728\\
44.0740	0.1667\\
44.0840	0.1639\\
44.0940	0.1601\\
44.1040	0.1442\\
44.1140	0.1313\\
44.1240	0.1255\\
44.1340	0.1298\\
44.1440	0.1505\\
44.1540	0.1563\\
44.1640	0.1653\\
44.1740	0.1576\\
44.1840	0.1883\\
44.1940	0.1641\\
44.2040	0.1586\\
44.2140	0.1527\\
44.2240	0.1527\\
44.2340	0.1867\\
44.2440	0.2314\\
44.2540	0.1645\\
44.2640	0.1578\\
44.2740	0.1503\\
44.2840	0.1400\\
44.2940	0.1435\\
44.3040	0.1335\\
44.3140	0.1314\\
44.3240	0.1338\\
44.3340	0.1294\\
44.3440	0.1449\\
44.3540	0.1077\\
44.3640	0.1045\\
44.3740	0.1113\\
44.3840	0.1161\\
44.3940	0.1393\\
44.4040	0.1718\\
44.4140	0.2415\\
44.4240	0.2681\\
44.4340	0.2296\\
44.4440	0.2188\\
44.4540	0.1994\\
44.4640	0.1638\\
44.4740	0.1460\\
44.4840	0.1431\\
44.4940	0.1460\\
44.5040	0.1517\\
44.5140	0.1480\\
44.5240	0.1239\\
44.5340	0.1155\\
44.5440	0.1140\\
44.5540	0.1367\\
44.5640	0.1622\\
44.5740	0.1725\\
44.5840	0.1633\\
44.5940	0.1615\\
44.6040	0.1556\\
44.6140	0.1573\\
44.6240	0.1595\\
44.6340	0.1451\\
44.6440	0.1581\\
44.6540	0.1634\\
44.6640	0.1533\\
44.6740	0.1440\\
44.6840	0.1429\\
44.6940	0.1382\\
44.7040	0.1396\\
44.7140	0.1434\\
44.7240	0.1375\\
44.7340	0.1405\\
44.7440	0.1320\\
44.7540	0.1452\\
44.7640	0.1433\\
44.7740	0.1416\\
44.7840	0.1434\\
44.7940	0.1443\\
44.8040	0.1517\\
44.8140	0.1621\\
44.8240	0.1694\\
44.8340	0.1718\\
44.8440	0.1645\\
44.8540	0.1645\\
44.8640	0.1660\\
44.8740	0.1754\\
44.8840	0.1754\\
44.8940	0.1638\\
44.9040	0.1489\\
44.9140	0.1507\\
44.9240	0.1645\\
44.9340	0.1688\\
44.9440	0.1500\\
44.9540	0.1407\\
44.9640	0.1315\\
44.9740	0.1461\\
44.9840	0.1715\\
44.9940	0.1698\\
45.0040	0.1743\\
45.0140	0.1794\\
45.0240	0.1935\\
45.0340	0.1775\\
45.0440	0.1696\\
45.0540	0.1669\\
45.0640	0.1623\\
45.0740	0.1657\\
45.0840	0.1658\\
45.0940	0.1783\\
45.1040	0.1850\\
45.1140	0.1920\\
45.1240	0.1985\\
45.1340	0.1907\\
45.1440	0.1944\\
45.1540	0.1837\\
45.1640	0.1793\\
45.1740	0.1846\\
45.1840	0.1941\\
45.1940	0.2085\\
45.2040	0.2088\\
45.2140	0.2128\\
45.2240	0.2091\\
45.2340	0.2129\\
45.2440	0.2111\\
45.2540	0.2267\\
45.2640	0.2302\\
45.2740	0.2173\\
45.2840	0.2132\\
45.2940	0.2121\\
45.3040	0.2168\\
45.3140	0.2108\\
45.3240	0.2164\\
45.3340	0.2205\\
45.3440	0.2216\\
45.3540	0.2257\\
45.3640	0.2108\\
45.3740	0.2150\\
45.3840	0.2040\\
45.3940	0.1982\\
45.4040	0.1912\\
45.4140	0.1931\\
45.4240	0.2048\\
45.4340	0.2092\\
45.4440	0.2032\\
45.4540	0.2174\\
45.4640	0.2089\\
45.4740	0.2217\\
45.4840	0.2471\\
45.4940	0.2481\\
45.5040	0.2320\\
45.5140	0.2233\\
45.5240	0.2122\\
45.5340	0.1998\\
45.5440	0.1980\\
45.5540	0.1948\\
45.5640	0.1861\\
45.5740	0.2007\\
45.5840	0.2216\\
45.5940	0.2245\\
45.6040	0.2126\\
45.6140	0.2088\\
45.6240	0.2243\\
45.6340	0.2335\\
45.6440	0.2500\\
45.6540	0.2603\\
45.6640	0.2360\\
45.6740	0.2113\\
45.6840	0.2190\\
45.6940	0.2085\\
45.7040	0.2101\\
45.7140	0.2246\\
45.7240	0.2301\\
45.7340	0.2252\\
45.7440	0.2040\\
45.7540	0.1948\\
45.7640	0.2020\\
45.7740	0.2299\\
45.7840	0.2493\\
45.7940	0.2519\\
45.8040	0.2550\\
45.8140	0.2558\\
45.8240	0.2615\\
45.8340	0.2462\\
45.8440	0.2414\\
45.8540	0.2295\\
45.8640	0.2272\\
45.8740	0.2362\\
45.8840	0.2506\\
45.8940	0.2555\\
45.9040	0.3232\\
45.9140	0.3105\\
45.9240	0.2855\\
45.9340	0.2589\\
45.9440	0.2340\\
45.9540	0.2159\\
45.9640	0.2110\\
45.9740	0.1989\\
45.9840	0.1980\\
45.9940	0.2171\\
46.0040	0.2599\\
46.0140	0.2574\\
46.0240	0.2598\\
46.0340	0.2583\\
46.0440	0.2475\\
46.0540	0.2558\\
46.0640	0.2451\\
46.0740	0.2223\\
46.0840	0.2292\\
46.0940	0.2256\\
46.1040	0.2301\\
46.1140	0.2364\\
46.1240	0.2698\\
46.1340	0.2525\\
46.1440	0.2369\\
46.1540	0.2321\\
46.1640	0.2630\\
46.1740	0.3135\\
46.1840	0.2676\\
};
\end{axis}

\begin{axis}[%
width=0.411\fwidth,
height=0.232\fwidth,
at={(0\fwidth,0\fwidth)},
scale only axis,
xmin=40.0040,
xmax=46.1840,
xlabel={$t$ [s]},
ymin=0.0000,
ymax=1.0000,
ylabel={$K$ [-]},
axis background/.style={fill=white},
title style={font=\labelsize},
xlabel style={font=\labelsize,at={(axis description cs:0.5,\xlabeldist)}},
ylabel style={font=\labelsize,at={(axis description cs:\ylabeldist,0.5)}},
legend style={font=\ticksize},
ticklabel style={font=\ticksize}
]
\addplot [color=mycolor1,solid,forget plot]
  table[row sep=crcr]{%
40.0040	0.3156\\
40.0140	0.6333\\
40.0240	0.3602\\
40.0340	0.3340\\
40.0440	0.2866\\
40.0540	0.1557\\
40.0640	0.2375\\
40.0740	0.3490\\
40.0840	0.1058\\
40.0940	0.0803\\
40.1040	0.0000\\
40.1140	0.0000\\
40.1240	0.0000\\
40.1340	0.9560\\
40.1440	0.0000\\
40.1540	0.0000\\
40.1640	0.9305\\
40.1740	0.9862\\
40.1840	0.5814\\
40.1940	0.7608\\
40.2040	0.9031\\
40.2140	0.7780\\
40.2240	0.6103\\
40.2340	0.5994\\
40.2440	0.5744\\
40.2540	0.6784\\
40.2640	0.3537\\
40.2740	0.3649\\
40.2840	0.6133\\
40.2940	0.3595\\
40.3040	0.0436\\
40.3140	0.0590\\
40.3240	0.0479\\
40.3340	0.4060\\
40.3440	0.4273\\
40.3540	0.4431\\
40.3640	0.3916\\
40.3740	0.3689\\
40.3840	0.5623\\
40.3940	0.6485\\
40.4040	0.3091\\
40.4140	0.4259\\
40.4240	0.3997\\
40.4340	0.3912\\
40.4440	0.3120\\
40.4540	0.3559\\
40.4640	0.1836\\
40.4740	0.2141\\
40.4840	0.1179\\
40.4940	0.1393\\
40.5040	0.2512\\
40.5140	0.3332\\
40.5240	0.3402\\
40.5340	0.2545\\
40.5440	0.3767\\
40.5540	0.4393\\
40.5640	0.5704\\
40.5740	0.6491\\
40.5840	0.6005\\
40.5940	0.6424\\
40.6040	0.7807\\
40.6140	0.8125\\
40.6240	0.7246\\
40.6340	0.7497\\
40.6440	0.7909\\
40.6540	0.8137\\
40.6640	0.5355\\
40.6740	0.4989\\
40.6840	0.5066\\
40.6940	0.5322\\
40.7040	0.5431\\
40.7140	0.9472\\
40.7240	0.4533\\
40.7340	0.0000\\
40.7440	0.8637\\
40.7540	0.5558\\
40.7640	0.6215\\
40.7740	0.3333\\
40.7840	0.4685\\
40.7940	0.5166\\
40.8040	0.2581\\
40.8140	0.1716\\
40.8240	0.0385\\
40.8340	0.8173\\
40.8440	0.7714\\
40.8540	0.6315\\
40.8640	0.5617\\
40.8740	0.4226\\
40.8840	0.3437\\
40.8940	0.5500\\
40.9040	0.6342\\
40.9140	0.5291\\
40.9240	0.1390\\
40.9340	0.2459\\
40.9440	0.4156\\
40.9540	0.3972\\
40.9640	0.0889\\
40.9740	0.2494\\
40.9840	0.2621\\
40.9940	0.6939\\
41.0040	0.8343\\
41.0140	0.1705\\
41.0240	0.0000\\
41.0340	0.0000\\
41.0440	0.0000\\
41.0540	0.0000\\
41.0640	0.0000\\
41.0740	0.0000\\
41.0840	0.4373\\
41.0940	0.2823\\
41.1040	0.2399\\
41.1140	0.5225\\
41.1240	0.1250\\
41.1340	0.5123\\
41.1440	0.1509\\
41.1540	0.6841\\
41.1640	0.0000\\
41.1740	0.0000\\
41.1840	0.0000\\
41.1940	0.0000\\
41.2040	0.7532\\
41.2140	0.5445\\
41.2240	0.3732\\
41.2340	0.6454\\
41.2440	0.0000\\
41.2540	0.1457\\
41.2640	0.2797\\
41.2740	0.3370\\
41.2840	0.2273\\
41.2940	0.3527\\
41.3040	0.0384\\
41.3140	0.0771\\
41.3240	0.0000\\
41.3340	0.4664\\
41.3440	0.3339\\
41.3540	0.1872\\
41.3640	0.0000\\
41.3740	0.0000\\
41.3840	0.3473\\
41.3940	0.6194\\
41.4040	0.7153\\
41.4140	0.8465\\
41.4240	0.8353\\
41.4340	0.8753\\
41.4440	0.1234\\
41.4540	0.0000\\
41.4640	0.6692\\
41.4740	0.8921\\
41.4840	0.3696\\
41.4940	0.3493\\
41.5040	0.1607\\
41.5140	0.3263\\
41.5240	0.2419\\
41.5340	0.3426\\
41.5440	0.0000\\
41.5540	0.8523\\
41.5640	0.6052\\
41.5740	0.6652\\
41.5840	0.1753\\
41.5940	0.0000\\
41.6040	0.2921\\
41.6140	0.5358\\
41.6240	0.7378\\
41.6340	0.6273\\
41.6440	0.5999\\
41.6540	0.5145\\
41.6640	0.5359\\
41.6740	0.7358\\
41.6840	0.3131\\
41.6940	0.4277\\
41.7040	0.1216\\
41.7140	0.7158\\
41.7240	0.5399\\
41.7340	0.7216\\
41.7440	0.8740\\
41.7540	0.1767\\
41.7640	0.0514\\
41.7740	0.0445\\
41.7840	0.0083\\
41.7940	0.2005\\
41.8040	0.2052\\
41.8140	0.1652\\
41.8240	0.2254\\
41.8340	0.3239\\
41.8440	0.3235\\
41.8540	0.4772\\
41.8640	0.4215\\
41.8740	0.2904\\
41.8840	0.3945\\
41.8940	0.4439\\
41.9040	0.6662\\
41.9140	0.7653\\
41.9240	0.5012\\
41.9340	0.8197\\
41.9440	0.1481\\
41.9540	0.0000\\
41.9640	0.7753\\
41.9740	0.5797\\
41.9840	0.7357\\
41.9940	0.7884\\
42.0040	0.4092\\
42.0140	0.2935\\
42.0240	0.1491\\
42.0340	0.1294\\
42.0440	0.2557\\
42.0540	0.5924\\
42.0640	0.6989\\
42.0740	0.5775\\
42.0840	0.4146\\
42.0940	0.2624\\
42.1040	0.6047\\
42.1140	0.5237\\
42.1240	0.5899\\
42.1340	0.1561\\
42.1440	0.1944\\
42.1540	0.1366\\
42.1640	0.3451\\
42.1740	0.0988\\
42.1840	0.0842\\
42.1940	0.2202\\
42.2040	0.1674\\
42.2140	0.0000\\
42.2240	0.0000\\
42.2340	0.0000\\
42.2440	0.0000\\
42.2540	0.0000\\
42.2640	0.0000\\
42.2740	0.1752\\
42.2840	0.0000\\
42.2940	0.0000\\
42.3040	0.3446\\
42.3140	0.5951\\
42.3240	0.4300\\
42.3340	0.3315\\
42.3440	0.1701\\
42.3540	0.3158\\
42.3640	0.5740\\
42.3740	0.2817\\
42.3840	0.3914\\
42.3940	0.1042\\
42.4040	0.2517\\
42.4140	0.0949\\
42.4240	0.5495\\
42.4340	0.6824\\
42.4440	0.7896\\
42.4540	0.6379\\
42.4640	0.4972\\
42.4740	0.1695\\
42.4840	0.3369\\
42.4940	0.1346\\
42.5040	0.4844\\
42.5140	0.8499\\
42.5240	0.8846\\
42.5340	0.8795\\
42.5440	0.9404\\
42.5540	0.7766\\
42.5640	0.4319\\
42.5740	0.1715\\
42.5840	0.1985\\
42.5940	0.3480\\
42.6040	0.2057\\
42.6140	0.0669\\
42.6240	0.2725\\
42.6340	0.7525\\
42.6440	0.4079\\
42.6540	0.0162\\
42.6640	0.0893\\
42.6740	0.2059\\
42.6840	0.0000\\
42.6940	0.0660\\
42.7040	0.0000\\
42.7140	0.0000\\
42.7240	0.0000\\
42.7340	0.3924\\
42.7440	0.8768\\
42.7540	0.5013\\
42.7640	0.3004\\
42.7740	0.0561\\
42.7840	0.1249\\
42.7940	0.0000\\
42.8040	0.0000\\
42.8140	0.3662\\
42.8240	0.0000\\
42.8340	0.0000\\
42.8440	0.0000\\
42.8540	0.0000\\
42.8640	0.0000\\
42.8740	0.0000\\
42.8840	0.0000\\
42.8940	0.0000\\
42.9040	0.0000\\
42.9140	0.0000\\
42.9240	0.9226\\
42.9340	0.6205\\
42.9440	0.2660\\
42.9540	0.1491\\
42.9640	0.3482\\
42.9740	0.1194\\
42.9840	0.6820\\
42.9940	0.5840\\
43.0040	0.7689\\
43.0140	0.6380\\
43.0240	0.0000\\
43.0340	0.3049\\
43.0440	0.7128\\
43.0540	0.2940\\
43.0640	0.1841\\
43.0740	0.0000\\
43.0840	0.2193\\
43.0940	0.0000\\
43.1040	0.0000\\
43.1140	0.0787\\
43.1240	0.1327\\
43.1340	0.2185\\
43.1440	0.4941\\
43.1540	0.4609\\
43.1640	0.2547\\
43.1740	0.4440\\
43.1840	0.3200\\
43.1940	0.0000\\
43.2040	0.0000\\
43.2140	0.5398\\
43.2240	0.2859\\
43.2340	0.6577\\
43.2440	0.6801\\
43.2540	0.6979\\
43.2640	0.8093\\
43.2740	0.5847\\
43.2840	0.7385\\
43.2940	0.7266\\
43.3040	0.0464\\
43.3140	0.1025\\
43.3240	0.3067\\
43.3340	0.4899\\
43.3440	0.1489\\
43.3540	0.2316\\
43.3640	0.3937\\
43.3740	0.1029\\
43.3840	0.0142\\
43.3940	0.0247\\
43.4040	0.0000\\
43.4140	0.7696\\
43.4240	0.8098\\
43.4340	0.8327\\
43.4440	0.5221\\
43.4540	0.7447\\
43.4640	0.7394\\
43.4740	0.6210\\
43.4840	0.3041\\
43.4940	0.5790\\
43.5040	0.6037\\
43.5140	0.6650\\
43.5240	0.5261\\
43.5340	0.1668\\
43.5440	0.2416\\
43.5540	0.4077\\
43.5640	0.1452\\
43.5740	0.1079\\
43.5840	0.1755\\
43.5940	0.1484\\
43.6040	0.0653\\
43.6140	0.2834\\
43.6240	0.3672\\
43.6340	0.2219\\
43.6440	0.2963\\
43.6540	0.2996\\
43.6640	0.3493\\
43.6740	0.2716\\
43.6840	0.1775\\
43.6940	0.4760\\
43.7040	0.4636\\
43.7140	0.3171\\
43.7240	0.2605\\
43.7340	0.5501\\
43.7440	0.3686\\
43.7540	0.3929\\
43.7640	0.2439\\
43.7740	0.0000\\
43.7840	0.0000\\
43.7940	0.9465\\
43.8040	0.1909\\
43.8140	0.5311\\
43.8240	0.0000\\
43.8340	0.2241\\
43.8440	0.1471\\
43.8540	0.0000\\
43.8640	0.5427\\
43.8740	0.7094\\
43.8840	0.5657\\
43.8940	0.7284\\
43.9040	0.0000\\
43.9140	0.5680\\
43.9240	0.5169\\
43.9340	0.4004\\
43.9440	0.0181\\
43.9540	0.1583\\
43.9640	0.3939\\
43.9740	0.2750\\
43.9840	0.3251\\
43.9940	0.3084\\
44.0040	0.5994\\
44.0140	0.6460\\
44.0240	0.3176\\
44.0340	0.4986\\
44.0440	0.5660\\
44.0540	0.4306\\
44.0640	0.1534\\
44.0740	0.2039\\
44.0840	0.1219\\
44.0940	0.2110\\
44.1040	0.3383\\
44.1140	0.3218\\
44.1240	0.3811\\
44.1340	0.3505\\
44.1440	0.5609\\
44.1540	0.4371\\
44.1640	0.5258\\
44.1740	0.3721\\
44.1840	0.7099\\
44.1940	0.5282\\
44.2040	0.2313\\
44.2140	0.4314\\
44.2240	0.0000\\
44.2340	0.3116\\
44.2440	0.6460\\
44.2540	0.5210\\
44.2640	0.0886\\
44.2740	0.4079\\
44.2840	0.2887\\
44.2940	0.3165\\
44.3040	0.7877\\
44.3140	0.1121\\
44.3240	0.4061\\
44.3340	0.3306\\
44.3440	0.4623\\
44.3540	0.2118\\
44.3640	0.0409\\
44.3740	0.2652\\
44.3840	0.2070\\
44.3940	0.6031\\
44.4040	0.5644\\
44.4140	0.8023\\
44.4240	0.7679\\
44.4340	0.2431\\
44.4440	0.1263\\
44.4540	0.2907\\
44.4640	0.5613\\
44.4740	0.7240\\
44.4840	0.7870\\
44.4940	0.7573\\
44.5040	0.8207\\
44.5140	0.8324\\
44.5240	0.6246\\
44.5340	0.7760\\
44.5440	0.7656\\
44.5540	0.8443\\
44.5640	0.7988\\
44.5740	0.6729\\
44.5840	0.8718\\
44.5940	0.8233\\
44.6040	0.7786\\
44.6140	0.9089\\
44.6240	0.8766\\
44.6340	0.8696\\
44.6440	0.8158\\
44.6540	0.6477\\
44.6640	0.5500\\
44.6740	0.6828\\
44.6840	0.6743\\
44.6940	0.7514\\
44.7040	0.7750\\
44.7140	0.8522\\
44.7240	0.9031\\
44.7340	0.8444\\
44.7440	0.6779\\
44.7540	0.5807\\
44.7640	0.1857\\
44.7740	0.1728\\
44.7840	0.6072\\
44.7940	0.7300\\
44.8040	0.5379\\
44.8140	0.8178\\
44.8240	0.9019\\
44.8340	0.9325\\
44.8440	0.8719\\
44.8540	0.9017\\
44.8640	0.8702\\
44.8740	0.8116\\
44.8840	0.8234\\
44.8940	0.7436\\
44.9040	0.7237\\
44.9140	0.7369\\
44.9240	0.7676\\
44.9340	0.7499\\
44.9440	0.7499\\
44.9540	0.8160\\
44.9640	0.7799\\
44.9740	0.6380\\
44.9840	0.6510\\
44.9940	0.5524\\
45.0040	0.4662\\
45.0140	0.5633\\
45.0240	0.7628\\
45.0340	0.5969\\
45.0440	0.6629\\
45.0540	0.7321\\
45.0640	0.6608\\
45.0740	0.7875\\
45.0840	0.7796\\
45.0940	0.8924\\
45.1040	0.8913\\
45.1140	0.8466\\
45.1240	0.9020\\
45.1340	0.8547\\
45.1440	0.8633\\
45.1540	0.7752\\
45.1640	0.7537\\
45.1740	0.8777\\
45.1840	0.8942\\
45.1940	0.8797\\
45.2040	0.8014\\
45.2140	0.7099\\
45.2240	0.7122\\
45.2340	0.7297\\
45.2440	0.7462\\
45.2540	0.7530\\
45.2640	0.7970\\
45.2740	0.6710\\
45.2840	0.4242\\
45.2940	0.4764\\
45.3040	0.7049\\
45.3140	0.6676\\
45.3240	0.7568\\
45.3340	0.7992\\
45.3440	0.8249\\
45.3540	0.8664\\
45.3640	0.8617\\
45.3740	0.8580\\
45.3840	0.8954\\
45.3940	0.8151\\
45.4040	0.7859\\
45.4140	0.6307\\
45.4240	0.7842\\
45.4340	0.6112\\
45.4440	0.6225\\
45.4540	0.7098\\
45.4640	0.6615\\
45.4740	0.8003\\
45.4840	0.8561\\
45.4940	0.8229\\
45.5040	0.8369\\
45.5140	0.6638\\
45.5240	0.4656\\
45.5340	0.4005\\
45.5440	0.1895\\
45.5540	0.0515\\
45.5640	0.2132\\
45.5740	0.5300\\
45.5840	0.4958\\
45.5940	0.6315\\
45.6040	0.4290\\
45.6140	0.2423\\
45.6240	0.5107\\
45.6340	0.5739\\
45.6440	0.7239\\
45.6540	0.7053\\
45.6640	0.6300\\
45.6740	0.6418\\
45.6840	0.6984\\
45.6940	0.7977\\
45.7040	0.8232\\
45.7140	0.7764\\
45.7240	0.6537\\
45.7340	0.8171\\
45.7440	0.8761\\
45.7540	0.8938\\
45.7640	0.7000\\
45.7740	0.5386\\
45.7840	0.5934\\
45.7940	0.6126\\
45.8040	0.7126\\
45.8140	0.7034\\
45.8240	0.7682\\
45.8340	0.7378\\
45.8440	0.8538\\
45.8540	0.8229\\
45.8640	0.9106\\
45.8740	0.7652\\
45.8840	0.7712\\
45.8940	0.5348\\
45.9040	0.5818\\
45.9140	0.4354\\
45.9240	0.4922\\
45.9340	0.6142\\
45.9440	0.6277\\
45.9540	0.4422\\
45.9640	0.4680\\
45.9740	0.5901\\
45.9840	0.7482\\
45.9940	0.6240\\
46.0040	0.5543\\
46.0140	0.4624\\
46.0240	0.6401\\
46.0340	0.7040\\
46.0440	0.6237\\
46.0540	0.6686\\
46.0640	0.6588\\
46.0740	0.6325\\
46.0840	0.6738\\
46.0940	0.7400\\
46.1040	0.6211\\
46.1140	0.8059\\
46.1240	0.8681\\
46.1340	0.8157\\
46.1440	0.7470\\
46.1540	0.3833\\
46.1640	0.7937\\
46.1740	0.8521\\
46.1840	0.7417\\
};
\end{axis}

\begin{axis}[%
width=0.411\fwidth,
height=0.232\fwidth,
at={(0.54\fwidth,0.323\fwidth)},
scale only axis,
xmin=60.6925,
xmax=62.7984,
ymin=-1.3438,
ymax=1.3701,
axis background/.style={fill=white},
title style={font=\labelsize},
xlabel style={font=\labelsize,at={(axis description cs:0.5,\xlabeldist)}},
ylabel style={font=\labelsize,at={(axis description cs:\ylabeldist,0.5)}},
legend style={font=\ticksize},
ticklabel style={font=\ticksize}
]
\addplot [color=mycolor1,solid,forget plot]
  table[row sep=crcr]{%
60.6925	-0.4011\\
60.7035	-0.3809\\
60.7140	-0.3569\\
60.7243	-0.3288\\
60.7350	-0.2995\\
60.7455	-0.2720\\
60.7559	-0.2445\\
60.7660	-0.2160\\
60.7767	-0.1868\\
60.7871	-0.1571\\
60.7975	-0.1280\\
60.8079	-0.0988\\
60.8183	-0.0698\\
60.8286	-0.0416\\
60.8389	-0.0142\\
60.8493	0.0148\\
60.8599	0.0461\\
60.8706	0.0796\\
60.8816	0.1139\\
60.8923	0.1496\\
60.9026	0.1841\\
60.9128	0.2173\\
60.9236	0.2470\\
60.9339	0.2760\\
60.9444	0.3014\\
60.9546	0.3263\\
60.9652	0.3501\\
60.9756	0.3776\\
60.9859	0.4051\\
60.9964	0.4355\\
61.0072	0.4666\\
61.0176	0.4981\\
61.0286	0.5268\\
61.0390	0.5548\\
61.0493	0.5819\\
61.0599	0.6176\\
61.0707	0.6462\\
61.0818	0.6774\\
61.0924	0.7074\\
61.1028	0.7292\\
61.1130	0.7471\\
61.1234	0.7563\\
61.1339	0.7696\\
61.1446	0.7739\\
61.1557	0.7768\\
61.1660	0.7814\\
61.1767	0.7931\\
61.1870	0.8089\\
61.1973	0.8232\\
61.2076	0.8411\\
61.2183	0.8642\\
61.2288	0.8760\\
61.2392	0.8866\\
61.2494	0.8987\\
61.2600	0.9230\\
61.2709	0.9397\\
61.2817	0.9558\\
61.2921	0.9793\\
61.3027	0.9896\\
61.3128	0.9916\\
61.3235	0.9969\\
61.3341	1.0098\\
61.3445	1.0223\\
61.3555	1.0309\\
61.3659	1.0483\\
61.3767	1.0680\\
61.3870	1.0749\\
61.3976	1.0767\\
61.4078	1.0791\\
61.4182	1.0925\\
61.4288	1.1019\\
61.4391	1.1114\\
61.4492	1.1269\\
61.4601	1.1532\\
61.4709	1.1659\\
61.4817	1.1779\\
61.4925	1.1941\\
61.5030	1.1989\\
61.5140	1.1967\\
61.5243	1.1902\\
61.5350	1.1966\\
61.5455	1.1909\\
61.5557	1.1886\\
61.5661	1.1832\\
61.5766	1.1910\\
61.5871	1.1838\\
61.5974	1.1783\\
61.6077	1.1719\\
61.6183	1.1752\\
61.6285	1.1713\\
61.6390	1.1673\\
61.6492	1.1639\\
61.6599	1.1766\\
61.6704	1.1776\\
61.6805	1.1742\\
61.6913	1.1763\\
61.7015	1.1605\\
61.7121	1.1378\\
61.7231	1.1092\\
61.7339	1.0924\\
61.7443	1.0741\\
61.7546	1.0551\\
61.7648	1.0305\\
61.7755	1.0074\\
61.7860	0.9743\\
61.7963	0.9320\\
61.8066	0.8808\\
61.8172	0.8263\\
61.8277	0.7669\\
61.8379	0.7010\\
61.8482	0.6260\\
61.8589	0.5513\\
61.8694	0.4629\\
61.8805	0.3679\\
61.8909	0.2725\\
61.9016	0.1803\\
61.9120	0.0917\\
61.9222	0.0065\\
61.9326	-0.0753\\
61.9432	-0.1539\\
61.9538	-0.2321\\
61.9640	-0.3077\\
61.9744	-0.3862\\
61.9850	-0.4672\\
61.9951	-0.5511\\
62.0057	-0.6324\\
62.0165	-0.7195\\
62.0275	-0.7997\\
62.0382	-0.8843\\
62.0492	-0.9517\\
62.0598	-0.9940\\
62.0704	-1.0330\\
62.0806	-1.0545\\
62.0911	-1.0640\\
62.1016	-1.0722\\
62.1120	-1.0852\\
62.1223	-1.1017\\
62.1325	-1.1211\\
62.1432	-1.1229\\
62.1536	-1.1305\\
62.1639	-1.1393\\
62.1743	-1.1490\\
62.1850	-1.1560\\
62.1953	-1.1856\\
62.2065	-1.2044\\
62.2172	-1.2093\\
62.2275	-1.2102\\
62.2381	-1.1951\\
62.2492	-1.1764\\
62.2598	-1.1513\\
62.2704	-1.1433\\
62.2806	-1.1400\\
62.2910	-1.1472\\
62.3018	-1.1463\\
62.3129	-1.1618\\
62.3232	-1.1705\\
62.3339	-1.1665\\
62.3445	-1.1644\\
62.3556	-1.1526\\
62.3658	-1.1397\\
62.3767	-1.1255\\
62.3871	-1.1267\\
62.3981	-1.1285\\
62.4083	-1.1195\\
62.4185	-1.1135\\
62.4296	-1.1014\\
62.4400	-1.0818\\
62.4506	-1.0515\\
62.4611	-1.0273\\
62.4712	-1.0047\\
62.4815	-0.9794\\
62.4923	-0.9539\\
62.5030	-0.9382\\
62.5137	-0.9254\\
62.5241	-0.9189\\
62.5349	-0.8978\\
62.5455	-0.8871\\
62.5565	-0.8693\\
62.5674	-0.8435\\
62.5778	-0.8244\\
62.5881	-0.8016\\
62.5983	-0.7801\\
62.6087	-0.7556\\
62.6195	-0.7333\\
62.6305	-0.7137\\
62.6409	-0.6948\\
62.6516	-0.6712\\
62.6618	-0.6501\\
62.6721	-0.6260\\
62.6824	-0.6028\\
62.6932	-0.5764\\
62.7036	-0.5532\\
62.7140	-0.5297\\
62.7244	-0.5095\\
62.7350	-0.4872\\
62.7451	-0.4677\\
62.7556	-0.4500\\
62.7660	-0.4305\\
62.7769	-0.4132\\
62.7878	-0.3955\\
62.7984	-0.3757\\
};
\addplot [color=mycolor2,solid,forget plot]
  table[row sep=crcr]{%
60.6940	-0.4544\\
60.7040	-0.4046\\
60.7140	-0.3541\\
60.7240	-0.3218\\
60.7340	-0.3435\\
60.7440	-0.3521\\
60.7540	-0.3501\\
60.7640	-0.3724\\
60.7740	-0.3807\\
60.7840	-0.3864\\
60.7940	-0.3667\\
60.8040	-0.3443\\
60.8140	-0.3006\\
60.8240	-0.2584\\
60.8340	-0.2333\\
60.8440	-0.2301\\
60.8540	-0.2204\\
60.8640	-0.2009\\
60.8740	-0.1851\\
60.8840	-0.1634\\
60.8940	-0.1403\\
60.9040	-0.1340\\
60.9140	-0.1060\\
60.9240	-0.0885\\
60.9340	-0.0809\\
60.9440	-0.0451\\
60.9540	0.0036\\
60.9640	0.0649\\
60.9740	0.3144\\
60.9840	0.4484\\
60.9940	0.5080\\
61.0040	0.4822\\
61.0140	0.5067\\
61.0240	0.5029\\
61.0340	0.4970\\
61.0440	0.4807\\
61.0540	0.4750\\
61.0640	0.5302\\
61.0740	0.5914\\
61.0840	0.6736\\
61.0940	0.7277\\
61.1040	0.7473\\
61.1140	0.7504\\
61.1240	0.7007\\
61.1340	0.6967\\
61.1440	0.7067\\
61.1540	0.7316\\
61.1640	0.7685\\
61.1740	0.7773\\
61.1840	0.7968\\
61.1940	0.7766\\
61.2040	0.8312\\
61.2140	0.8482\\
61.2240	0.8841\\
61.2340	0.8746\\
61.2440	0.8785\\
61.2540	0.9372\\
61.2640	0.9739\\
61.2740	1.0095\\
61.2840	0.9680\\
61.2940	0.9243\\
61.3040	0.9066\\
61.3140	0.9024\\
61.3240	0.8886\\
61.3340	0.8708\\
61.3440	0.8863\\
61.3540	0.8823\\
61.3640	0.9118\\
61.3740	0.9338\\
61.3840	0.9174\\
61.3940	0.9156\\
61.4040	0.8662\\
61.4140	0.8803\\
61.4240	0.9190\\
61.4340	0.9777\\
61.4440	1.0074\\
61.4540	1.0312\\
61.4640	1.0693\\
61.4740	1.1013\\
61.4840	1.1623\\
61.4940	1.1619\\
61.5040	1.1617\\
61.5140	1.1395\\
61.5240	1.1691\\
61.5340	1.2165\\
61.5440	1.2376\\
61.5540	1.2177\\
61.5640	1.2130\\
61.5740	1.1907\\
61.5840	1.2071\\
61.5940	1.1985\\
61.6040	1.1889\\
61.6140	1.1887\\
61.6240	1.1598\\
61.6340	1.1763\\
61.6440	1.1979\\
61.6540	1.1904\\
61.6640	1.1953\\
61.6740	1.2462\\
61.6840	1.2580\\
61.6940	1.2796\\
61.7040	1.2814\\
61.7140	1.3150\\
61.7240	1.3701\\
61.7340	1.2941\\
61.7440	1.2122\\
61.7540	1.1492\\
61.7640	1.1166\\
61.7740	1.0699\\
61.7840	1.0753\\
61.7940	1.0107\\
61.8040	0.9753\\
61.8140	0.9075\\
61.8240	0.8463\\
61.8340	0.8620\\
61.8440	0.8789\\
61.8540	0.8333\\
61.8640	0.7809\\
61.8740	0.7142\\
61.8840	0.6592\\
61.8940	0.6580\\
61.9040	0.6530\\
61.9140	0.6307\\
61.9240	0.5769\\
61.9340	0.5220\\
61.9440	0.4820\\
61.9540	0.4147\\
61.9640	0.2910\\
61.9740	0.1343\\
61.9840	-0.0020\\
61.9940	-0.1385\\
62.0040	-0.2833\\
62.0140	-0.5833\\
62.0240	-0.8367\\
62.0340	-0.9081\\
62.0440	-0.9744\\
62.0540	-1.0291\\
62.0640	-1.0711\\
62.0740	-1.0879\\
62.0840	-1.0559\\
62.0940	-1.0614\\
62.1040	-1.0980\\
62.1140	-1.1332\\
62.1240	-1.2004\\
62.1340	-1.2390\\
62.1440	-1.2366\\
62.1540	-1.2129\\
62.1640	-1.2298\\
62.1740	-1.2806\\
62.1840	-1.3438\\
62.1940	-1.2468\\
62.2040	-1.1400\\
62.2140	-1.1397\\
62.2240	-1.1204\\
62.2340	-1.0734\\
62.2440	-1.0924\\
62.2540	-1.0540\\
62.2640	-1.2060\\
62.2740	-1.2858\\
62.2840	-1.2413\\
62.2940	-1.2429\\
62.3040	-1.1991\\
62.3140	-1.1704\\
62.3240	-1.1513\\
62.3340	-1.1319\\
62.3440	-1.2451\\
62.3540	-1.2547\\
62.3640	-1.2673\\
62.3740	-1.2587\\
62.3840	-1.2372\\
62.3940	-1.2313\\
62.4040	-1.2309\\
62.4140	-1.1783\\
62.4240	-1.1777\\
62.4340	-1.0815\\
62.4440	-1.0076\\
62.4540	-0.9773\\
62.4640	-0.9677\\
62.4740	-1.0338\\
62.4840	-1.0202\\
62.4940	-1.0115\\
62.5040	-1.0018\\
62.5140	-1.0103\\
62.5240	-1.0232\\
62.5340	-1.0283\\
62.5440	-1.0336\\
62.5540	-0.9738\\
62.5640	-0.9156\\
62.5740	-0.8688\\
62.5840	-0.7941\\
62.5940	-0.7561\\
62.6040	-0.7502\\
62.6140	-0.7510\\
62.6240	-0.6922\\
62.6340	-0.6681\\
62.6440	-0.6517\\
62.6540	-0.6659\\
62.6640	-0.7071\\
62.6740	-0.7263\\
62.6840	-0.7008\\
62.6940	-0.6947\\
62.7040	-0.6779\\
62.7140	-0.6445\\
62.7240	-0.6253\\
62.7340	-0.6140\\
62.7440	-0.6000\\
62.7540	-0.5729\\
62.7640	-0.5400\\
62.7740	-0.5366\\
62.7840	-0.5220\\
62.7940	-0.5103\\
};
\end{axis}

\begin{axis}[%
width=0.411\fwidth,
height=0.232\fwidth,
at={(0.54\fwidth,0\fwidth)},
scale only axis,
xmin=60.6940,
xmax=62.7940,
xlabel={$t$ [s]},
ymin=0.0000,
ymax=1.0000,
axis background/.style={fill=white},
title style={font=\labelsize},
xlabel style={font=\labelsize,at={(axis description cs:0.5,\xlabeldist)}},
ylabel style={font=\labelsize,at={(axis description cs:\ylabeldist,0.5)}},
legend style={font=\ticksize},
ticklabel style={font=\ticksize}
]
\addplot [color=mycolor1,solid,forget plot]
  table[row sep=crcr]{%
60.6940	0.7905\\
60.7040	0.7862\\
60.7140	0.6988\\
60.7240	0.7787\\
60.7340	0.5847\\
60.7440	0.5870\\
60.7540	0.6540\\
60.7640	0.7778\\
60.7740	0.7525\\
60.7840	0.7921\\
60.7940	0.6699\\
60.8040	0.8094\\
60.8140	0.7485\\
60.8240	0.6406\\
60.8340	0.6230\\
60.8440	0.5918\\
60.8540	0.4587\\
60.8640	0.3103\\
60.8740	0.5268\\
60.8840	0.4689\\
60.8940	0.3592\\
60.9040	0.5982\\
60.9140	0.6496\\
60.9240	0.6867\\
60.9340	0.4948\\
60.9440	0.4086\\
60.9540	0.4048\\
60.9640	0.2360\\
60.9740	0.8061\\
60.9840	0.7116\\
60.9940	0.8297\\
61.0040	0.7254\\
61.0140	0.7575\\
61.0240	0.7567\\
61.0340	0.7143\\
61.0440	0.5774\\
61.0540	0.6476\\
61.0640	0.6873\\
61.0740	0.7930\\
61.0840	0.8748\\
61.0940	0.8034\\
61.1040	0.8424\\
61.1140	0.7873\\
61.1240	0.8137\\
61.1340	0.8202\\
61.1440	0.7673\\
61.1540	0.8289\\
61.1640	0.8356\\
61.1740	0.8751\\
61.1840	0.8996\\
61.1940	0.8169\\
61.2040	0.8608\\
61.2140	0.7985\\
61.2240	0.8426\\
61.2340	0.8103\\
61.2440	0.8119\\
61.2540	0.8842\\
61.2640	0.8322\\
61.2740	0.8379\\
61.2840	0.8024\\
61.2940	0.7640\\
61.3040	0.7710\\
61.3140	0.7714\\
61.3240	0.6439\\
61.3340	0.3816\\
61.3440	0.5543\\
61.3540	0.5168\\
61.3640	0.3229\\
61.3740	0.4280\\
61.3840	0.5088\\
61.3940	0.7059\\
61.4040	0.6123\\
61.4140	0.7333\\
61.4240	0.7379\\
61.4340	0.8228\\
61.4440	0.8337\\
61.4540	0.8294\\
61.4640	0.8759\\
61.4740	0.8837\\
61.4840	0.8671\\
61.4940	0.8837\\
61.5040	0.8620\\
61.5140	0.8597\\
61.5240	0.8719\\
61.5340	0.8816\\
61.5440	0.8705\\
61.5540	0.8729\\
61.5640	0.8587\\
61.5740	0.8727\\
61.5840	0.8809\\
61.5940	0.7937\\
61.6040	0.8485\\
61.6140	0.8298\\
61.6240	0.7740\\
61.6340	0.8543\\
61.6440	0.8785\\
61.6540	0.8785\\
61.6640	0.5443\\
61.6740	0.6769\\
61.6840	0.7064\\
61.6940	0.7984\\
61.7040	0.7675\\
61.7140	0.6627\\
61.7240	0.6872\\
61.7340	0.6867\\
61.7440	0.7118\\
61.7540	0.6884\\
61.7640	0.7080\\
61.7740	0.7094\\
61.7840	0.7787\\
61.7940	0.7736\\
61.8040	0.8062\\
61.8140	0.7681\\
61.8240	0.7293\\
61.8340	0.6561\\
61.8440	0.7205\\
61.8540	0.6517\\
61.8640	0.5874\\
61.8740	0.2633\\
61.8840	0.2380\\
61.8940	0.0043\\
61.9040	0.0163\\
61.9140	0.0626\\
61.9240	0.1601\\
61.9340	0.1732\\
61.9440	0.1209\\
61.9540	0.2088\\
61.9640	0.3723\\
61.9740	0.4391\\
61.9840	0.4819\\
61.9940	0.5513\\
62.0040	0.6688\\
62.0140	0.8580\\
62.0240	0.9150\\
62.0340	0.8559\\
62.0440	0.8544\\
62.0540	0.8273\\
62.0640	0.8025\\
62.0740	0.8178\\
62.0840	0.6837\\
62.0940	0.7147\\
62.1040	0.7974\\
62.1140	0.8023\\
62.1240	0.6698\\
62.1340	0.7798\\
62.1440	0.8114\\
62.1540	0.8673\\
62.1640	0.8138\\
62.1740	0.8059\\
62.1840	0.7817\\
62.1940	0.4949\\
62.2040	0.3162\\
62.2140	0.5464\\
62.2240	0.5730\\
62.2340	0.5557\\
62.2440	0.7766\\
62.2540	0.7545\\
62.2640	0.6703\\
62.2740	0.7336\\
62.2840	0.6749\\
62.2940	0.7426\\
62.3040	0.7890\\
62.3140	0.7439\\
62.3240	0.7688\\
62.3340	0.8177\\
62.3440	0.8135\\
62.3540	0.8498\\
62.3640	0.8607\\
62.3740	0.8262\\
62.3840	0.8122\\
62.3940	0.7894\\
62.4040	0.7280\\
62.4140	0.6781\\
62.4240	0.5900\\
62.4340	0.6417\\
62.4440	0.6035\\
62.4540	0.6816\\
62.4640	0.4943\\
62.4740	0.7192\\
62.4840	0.7861\\
62.4940	0.6970\\
62.5040	0.2800\\
62.5140	0.4875\\
62.5240	0.5896\\
62.5340	0.6030\\
62.5440	0.7071\\
62.5540	0.6904\\
62.5640	0.7108\\
62.5740	0.5478\\
62.5840	0.5844\\
62.5940	0.6663\\
62.6040	0.6869\\
62.6140	0.7693\\
62.6240	0.6688\\
62.6340	0.7114\\
62.6440	0.4817\\
62.6540	0.6471\\
62.6640	0.6018\\
62.6740	0.6601\\
62.6840	0.6790\\
62.6940	0.7557\\
62.7040	0.8023\\
62.7140	0.8207\\
62.7240	0.8498\\
62.7340	0.8604\\
62.7440	0.8570\\
62.7540	0.8714\\
62.7640	0.8118\\
62.7740	0.8600\\
62.7840	0.8838\\
62.7940	0.8979\\
};
\end{axis}
\end{tikzpicture}%
			\label{fig:divergence_roadmap}
		}	
		\caption{From top to bottom: Height measurements (top row) and estimates of $\vartheta_z$ (second row, red line) in comparison to ground truth measurements (blue line). The two bottom rows show detail sections of $\vartheta_z$ estimates (third row) at low speed and high speed, as well as the accompanying estimate confidence value $K$ (bottom row). Measurements are shown for \protect\subref{fig:divergence_checkerboard} checkerboard and \protect\subref{fig:divergence_roadmap} roadmap textures separately.}
		\label{fig:divergence_estimates}
	\end{framed}
\end{figure*}

The detail plots show that the estimator is relatively sensitive to local outliers in normal flow at low speeds. In addition, the confidence $K$ is generally low due to lower detection rates of optical flow and low value of $R^2$. At higher speeds, the errors are relatively smaller. Note that $K$ is also generally higher there. Somewhat lower confidence values are seen for the roadmap texture.

Around sign changes, brief moments are present where the confidence value $K$ is low. The result of this is that, due to the confidence filter, the update of $\hat{\vartheta}_z$ at these points is limited, which leads to a local delay with respect to the ground truth. However, when higher confidence estimates are available, the estimate quickly converges back to the ground truth value.

Based on the estimator results in \cref{fig:divergence_estimates} we can assess how the error varies with the ground truth divergence. \cref{fig:divergence_error_dist} shows the variation of the absolute error $\varepsilon_{\vartheta_z}=\lvert\hat{\vartheta}_z-\vartheta_z \rvert$ with the magnitude of $\vartheta_z$\footnote{The dataset used to generate these plots and statistics is currently being prepared to be made public. They are planned to be made public at: \url{https://beta.dataverse.nl/dataverse/mavlab}}. A quadratic model $\varepsilon = p_0 + p_1 \vartheta_z + p_2 \vartheta_z^2$ is fitted to the points, which is represented by the blue line. The values of $p_0$, $p_1$, and $p_2$ are shown in \cref{tab:div_error_parameters}. The errors of both the checkerboard set and the roadmap set are combined, since the estimator shows roughly the same error distribution for both cases with the quadratic fit almost flat for the divergences tested. Interestingly, the largest absolute errors appear to be present at low divergence. This results from the local delay occurring around zero-crossings in \cref{fig:divergence_estimates} where the confidence value of the filter is low. Note, however, that the error increase with the magnitude of $\lvert\vartheta_z\rvert$ is limited, which enables application of the presented pipeline to a wide range of velocities.

\begin{figure}[!ht]
	\centering
	\setlength{\fwidth}{0.6\linewidth}
	\renewcommand{\xlabeldist}{0.0}
	\renewcommand{\ylabeldist}{0.0}
	%\tikzset{external/export next=false}
	% This file was created by matlab2tikz.
%
%The latest updates can be retrieved from
%  http://www.mathworks.com/matlabcentral/fileexchange/22022-matlab2tikz-matlab2tikz
%where you can also make suggestions and rate matlab2tikz.
%
\definecolor{mycolor1}{rgb}{0.00000,0.44700,0.94100}%
\definecolor{mycolor2}{rgb}{0.85000,0.32500,0.09800}%
%
\begin{tikzpicture}

\begin{axis}[%
width=0.951\fwidth,
height=0.5\fwidth,
at={(0\fwidth,0\fwidth)},
scale only axis,
xmin=-1.2500,
xmax=1.2500,
xlabel={$\vartheta_z$ [1/s]},
ymin=0.0000,
ymax=0.7000,
ylabel={$\varepsilon_{\vartheta_z}$ [1/s]},
axis background/.style={fill=white},
legend style={legend cell align=left,align=left,draw=white!15!black},
clip mode=individual,
title style={font=\labelsize},
xlabel style={font=\labelsize,at={(axis description cs:0.5,\xlabeldist)}},
ylabel style={font=\labelsize,at={(axis description cs:\ylabeldist,0.5)}},
legend style={font=\ticksize},
ticklabel style={font=\ticksize}
]
\addplot [color=white!60!black,mark size=0.5pt,only marks,mark=*,mark options={solid},forget plot]
  table[row sep=crcr]{%
0.0572	0.0737\\
0.0913	0.1042\\
0.0950	0.1056\\
0.0865	0.0945\\
0.0899	0.0960\\
0.0909	0.0942\\
0.0757	0.0757\\
0.0482	0.0446\\
0.0396	0.0305\\
0.0274	0.0129\\
0.0218	0.0000\\
0.0251	0.0222\\
0.0096	0.0442\\
0.0022	0.0515\\
-0.0106	0.1029\\
-0.0186	0.1035\\
0.0417	0.0515\\
0.0693	0.0301\\
0.0814	0.0203\\
0.1112	0.0100\\
0.1356	0.0339\\
0.1362	0.0305\\
0.1259	0.0135\\
0.1805	0.0613\\
0.2067	0.0868\\
0.1849	0.0691\\
0.1678	0.0549\\
0.1585	0.0436\\
0.1459	0.0279\\
0.1406	0.0237\\
0.1362	0.0228\\
0.1209	0.0061\\
0.1095	0.0116\\
0.1040	0.0159\\
0.0945	0.0249\\
0.0838	0.0494\\
0.1094	0.0474\\
0.1349	0.0378\\
0.1500	0.0317\\
0.1787	0.0119\\
0.1795	0.0211\\
0.2006	0.0055\\
0.2142	0.0025\\
0.1866	0.0350\\
0.1791	0.0488\\
0.1967	0.0285\\
0.2329	0.0154\\
0.2337	0.0154\\
0.2581	0.0301\\
0.3073	0.0679\\
0.2767	0.0343\\
0.2479	0.0129\\
0.2412	0.0140\\
0.2520	0.0239\\
0.2252	0.0096\\
0.2271	0.0116\\
0.2382	0.0008\\
0.2514	0.0139\\
0.2574	0.0210\\
0.2342	0.0042\\
0.2094	0.0413\\
0.1945	0.0731\\
0.2196	0.0582\\
0.2692	0.0095\\
0.2573	0.0171\\
0.2656	0.0099\\
0.3298	0.0446\\
0.3440	0.0499\\
0.2949	0.0029\\
0.2675	0.0229\\
0.2466	0.0446\\
0.2494	0.0484\\
0.3761	0.0731\\
0.2946	0.0080\\
0.3299	0.0339\\
0.3321	0.0363\\
0.3481	0.0450\\
0.3290	0.0176\\
0.2874	0.0245\\
0.2656	0.0380\\
0.3125	0.0102\\
0.3400	0.0346\\
0.3545	0.0479\\
0.3626	0.0495\\
0.3366	0.0154\\
0.3169	0.0075\\
0.3044	0.0319\\
0.3048	0.0538\\
0.3188	0.0495\\
0.3352	0.0199\\
0.3526	0.0055\\
0.3441	0.0071\\
0.3149	0.0499\\
0.2761	0.0972\\
0.2676	0.0986\\
0.2641	0.0928\\
0.3032	0.0536\\
0.3511	0.0146\\
0.3914	0.0190\\
0.4055	0.0343\\
0.5924	0.2328\\
0.5645	0.2100\\
0.5201	0.1625\\
0.4511	0.0818\\
0.4517	0.0777\\
0.4036	0.0392\\
0.3552	0.0031\\
0.3647	0.0136\\
0.3397	0.0221\\
0.3569	0.0152\\
0.4000	0.0214\\
0.4074	0.0300\\
0.3973	0.0252\\
0.3355	0.0418\\
0.3114	0.0823\\
0.3013	0.1086\\
0.3044	0.1086\\
0.2809	0.1296\\
0.2981	0.1129\\
0.3823	0.0554\\
0.3673	0.1053\\
0.3524	0.1309\\
0.4840	0.0069\\
0.4892	0.0185\\
0.4419	0.0411\\
0.4708	0.0337\\
0.5021	0.0084\\
0.4539	0.0359\\
0.4330	0.0569\\
0.4176	0.0783\\
0.4050	0.0686\\
0.4211	0.0134\\
0.3509	0.0549\\
0.2983	0.0924\\
0.2926	0.0864\\
0.3444	0.0221\\
0.4088	0.0591\\
0.4051	0.0710\\
0.4355	0.1189\\
0.4345	0.1382\\
0.4026	0.1342\\
0.4043	0.1689\\
0.5345	0.3327\\
0.6100	0.4360\\
0.4602	0.3014\\
0.3533	0.1987\\
0.3252	0.1700\\
0.2861	0.1302\\
0.2411	0.0873\\
0.1614	0.0023\\
0.1247	0.0452\\
0.1248	0.0530\\
0.1235	0.0568\\
0.1256	0.0574\\
0.1401	0.0496\\
0.1404	0.0660\\
0.1357	0.0934\\
0.1428	0.1063\\
0.2231	0.0315\\
0.2253	0.0163\\
0.2092	0.0047\\
0.2015	0.0167\\
0.2760	0.1118\\
0.3304	0.1807\\
0.3410	0.2086\\
0.2720	0.1633\\
0.1655	0.0853\\
0.1551	0.1076\\
0.1246	0.1146\\
0.1151	0.1576\\
0.1067	0.2090\\
0.0883	0.2439\\
-0.0194	0.1775\\
-0.1088	0.1234\\
-0.1645	0.0994\\
-0.4594	0.1388\\
-0.6124	0.2235\\
-0.7272	0.3488\\
-0.8565	0.4813\\
-0.7256	0.3364\\
-0.6453	0.2411\\
-0.5922	0.1829\\
-0.4873	0.0779\\
-0.3296	0.0728\\
-0.2016	0.1982\\
-0.2119	0.1849\\
-0.3783	0.0135\\
-0.4416	0.0526\\
-0.6010	0.2093\\
-0.5995	0.1998\\
-0.4397	0.0405\\
-0.4062	0.0204\\
-0.3925	0.0262\\
-0.3766	0.0221\\
-0.3473	0.0051\\
-0.3288	0.0257\\
-0.3141	0.0382\\
-0.3465	0.0002\\
-0.3697	0.0268\\
-0.3779	0.0360\\
-0.3516	0.0217\\
-0.3320	0.0169\\
-0.3123	0.0010\\
-0.3458	0.0313\\
-0.3607	0.0436\\
-0.3587	0.0431\\
-0.3357	0.0215\\
-0.3452	0.0352\\
-0.3090	0.0102\\
-0.3274	0.0396\\
-0.3133	0.0288\\
-0.2834	0.0066\\
-0.2651	0.0328\\
-0.2099	0.0900\\
-0.2371	0.0435\\
-0.2540	0.0070\\
-0.2683	0.0148\\
-0.2804	0.0188\\
-0.2704	0.0025\\
-0.2658	0.0029\\
-0.2024	0.0514\\
-0.1917	0.0418\\
-0.2325	0.0170\\
-0.2464	0.0418\\
-0.2887	0.0873\\
-0.2576	0.0561\\
-0.2207	0.0230\\
-0.2314	0.0432\\
-0.2520	0.0765\\
-0.1926	0.0313\\
-0.2138	0.0640\\
-0.2187	0.0706\\
-0.2153	0.0609\\
-0.2134	0.0529\\
-0.2181	0.0534\\
-0.2357	0.0669\\
-0.2211	0.0533\\
-0.2214	0.0547\\
-0.2070	0.0358\\
-0.2031	0.0278\\
-0.2196	0.0469\\
-0.2218	0.0546\\
-0.2227	0.0558\\
-0.2280	0.0529\\
-0.2160	0.0267\\
-0.1909	0.0081\\
-0.1811	0.0204\\
-0.1972	0.0048\\
-0.2907	0.0898\\
-0.4171	0.2120\\
-0.3022	0.0915\\
-0.3106	0.0975\\
-0.2838	0.0790\\
-0.3256	0.1333\\
-0.2815	0.0866\\
-0.2766	0.0652\\
-0.3149	0.0973\\
-0.2730	0.0626\\
-0.1862	0.0154\\
-0.1381	0.0577\\
-0.1435	0.0473\\
-0.1759	0.0108\\
-0.2406	0.0532\\
-0.2393	0.0487\\
-0.2306	0.0394\\
-0.2193	0.0341\\
-0.2359	0.0581\\
-0.2424	0.0679\\
-0.2183	0.0405\\
-0.1880	0.0094\\
-0.2861	0.1100\\
-0.3681	0.1968\\
-0.3553	0.1890\\
-0.3052	0.1408\\
-0.2460	0.0824\\
-0.1751	0.0159\\
-0.1822	0.0268\\
-0.2556	0.0985\\
-0.2600	0.0987\\
-0.2745	0.1124\\
-0.2488	0.0976\\
-0.2113	0.0757\\
-0.1895	0.0556\\
-0.1997	0.0536\\
-0.1968	0.0392\\
-0.1919	0.0313\\
-0.1792	0.0224\\
-0.1894	0.0375\\
-0.2352	0.0872\\
-0.2568	0.1109\\
-0.2430	0.0977\\
-0.2380	0.0899\\
-0.2170	0.0616\\
-0.2351	0.0712\\
-0.2533	0.0876\\
-0.2772	0.1130\\
-0.2713	0.0997\\
-0.2483	0.0610\\
-0.2565	0.0624\\
-0.2325	0.0412\\
-0.2160	0.0285\\
-0.2080	0.0224\\
-0.1971	0.0102\\
-0.2217	0.0343\\
-0.2628	0.0770\\
-0.2566	0.0743\\
-0.2431	0.0654\\
-0.2400	0.0635\\
-0.2182	0.0388\\
-0.2144	0.0355\\
-0.2216	0.0486\\
-0.2228	0.0529\\
-0.2380	0.0670\\
-0.2495	0.0776\\
-0.2221	0.0506\\
-0.2177	0.0466\\
-0.2162	0.0450\\
-0.2236	0.0565\\
-0.2117	0.0510\\
-0.1988	0.0420\\
-0.1696	0.0169\\
-0.1461	0.0012\\
-0.1370	0.0006\\
-0.1256	0.0081\\
-0.1220	0.0055\\
-0.1193	0.0032\\
-0.1235	0.0165\\
-0.1389	0.0369\\
-0.1399	0.0416\\
-0.1106	0.0145\\
-0.0785	0.0192\\
-0.0729	0.0252\\
-0.0869	0.0027\\
-0.1224	0.0345\\
-0.2133	0.1226\\
-0.2504	0.1579\\
-0.2514	0.1592\\
-0.2372	0.1422\\
-0.2324	0.1336\\
-0.1985	0.0970\\
-0.1708	0.0695\\
-0.1675	0.0629\\
-0.1889	0.0801\\
-0.1556	0.0438\\
-0.0990	0.0108\\
-0.0895	0.0176\\
-0.1298	0.0200\\
-0.1247	0.0163\\
-0.1271	0.0219\\
-0.1258	0.0221\\
-0.1421	0.0400\\
-0.1273	0.0279\\
-0.1307	0.0326\\
-0.1487	0.0518\\
-0.1509	0.0579\\
-0.1418	0.0559\\
-0.1438	0.0653\\
-0.1376	0.0657\\
-0.1513	0.0838\\
-0.1419	0.0779\\
-0.1026	0.0412\\
-0.0791	0.0190\\
-0.0799	0.0193\\
-0.0871	0.0261\\
-0.0934	0.0304\\
-0.0897	0.0231\\
-0.0883	0.0166\\
-0.0877	0.0109\\
-0.0955	0.0139\\
-0.1201	0.0344\\
-0.1159	0.0282\\
-0.1037	0.0157\\
-0.1556	0.0693\\
-0.1663	0.0864\\
-0.2063	0.1315\\
-0.2151	0.1404\\
-0.2273	0.1555\\
-0.2254	0.1610\\
-0.1892	0.1321\\
-0.1512	0.0998\\
-0.1118	0.0652\\
-0.0943	0.0507\\
-0.0817	0.0367\\
-0.0942	0.0446\\
-0.1634	0.1097\\
-0.1726	0.1150\\
-0.1577	0.0949\\
-0.1480	0.0770\\
-0.1407	0.0597\\
-0.1062	0.0175\\
-0.0764	0.0155\\
-0.0609	0.0337\\
-0.0602	0.0378\\
-0.0916	0.0100\\
-0.1246	0.0204\\
-0.1456	0.0423\\
-0.2045	0.1045\\
-0.2437	0.1452\\
-0.2129	0.1134\\
-0.1976	0.0995\\
-0.1872	0.0926\\
-0.1732	0.0802\\
-0.1392	0.0455\\
-0.1165	0.0228\\
-0.1164	0.0242\\
-0.1220	0.0325\\
-0.1200	0.0335\\
-0.1184	0.0317\\
-0.1135	0.0241\\
-0.1064	0.0163\\
-0.0995	0.0104\\
-0.0908	0.0010\\
0.0401	0.1329\\
0.0084	0.1007\\
-0.0314	0.0627\\
-0.0333	0.0659\\
-0.0365	0.0702\\
-0.0474	0.0635\\
-0.0557	0.0532\\
-0.1188	0.0148\\
-0.1409	0.0384\\
-0.1491	0.0470\\
-0.1806	0.0804\\
-0.2415	0.1446\\
-0.2768	0.1834\\
-0.2752	0.1858\\
-0.2582	0.1727\\
-0.2124	0.1286\\
-0.1852	0.1007\\
-0.1997	0.1169\\
-0.1957	0.1185\\
-0.1887	0.1155\\
-0.1940	0.1218\\
-0.1516	0.0762\\
-0.1326	0.0458\\
-0.1025	0.0105\\
-0.0995	0.0056\\
-0.1104	0.0167\\
-0.1126	0.0228\\
-0.1150	0.0288\\
-0.1147	0.0287\\
-0.1142	0.0266\\
-0.1125	0.0244\\
-0.1034	0.0145\\
-0.0938	0.0020\\
-0.0890	0.0042\\
-0.0847	0.0062\\
-0.0873	0.0042\\
-0.0965	0.0016\\
-0.0954	0.0086\\
-0.0849	0.0186\\
-0.0802	0.0219\\
-0.0852	0.0168\\
-0.0868	0.0148\\
-0.0983	0.0011\\
-0.1087	0.0121\\
-0.1047	0.0100\\
-0.0596	0.0338\\
-0.0382	0.0549\\
-0.0398	0.0553\\
-0.0595	0.0399\\
-0.0641	0.0393\\
-0.0466	0.0599\\
-0.0355	0.0722\\
-0.0238	0.0824\\
-0.0540	0.0483\\
-0.0674	0.0280\\
-0.0781	0.0085\\
-0.0938	0.0175\\
-0.1071	0.0417\\
-0.0849	0.0314\\
-0.0528	0.0120\\
-0.0565	0.0299\\
-0.0736	0.0710\\
-0.0675	0.0926\\
-0.0651	0.1035\\
-0.0603	0.1122\\
-0.0580	0.1231\\
-0.0576	0.1357\\
-0.0507	0.1446\\
-0.0164	0.1263\\
0.0127	0.1098\\
0.0531	0.0785\\
0.1105	0.0292\\
0.1450	0.0009\\
0.1463	0.0036\\
0.1236	0.0292\\
0.0890	0.0674\\
0.0837	0.0763\\
0.0936	0.0695\\
0.1291	0.0356\\
0.1394	0.0251\\
0.1151	0.0487\\
0.1163	0.0469\\
0.1091	0.0530\\
0.1097	0.0493\\
0.1216	0.0374\\
0.1758	0.0137\\
0.2088	0.0470\\
0.1959	0.0424\\
0.1914	0.0445\\
0.1830	0.0397\\
0.1811	0.0413\\
0.1843	0.0485\\
0.1250	0.0094\\
0.0071	0.1267\\
-0.0047	0.1356\\
-0.0032	0.1296\\
0.0067	0.1159\\
0.0468	0.0722\\
0.0670	0.0478\\
0.0778	0.0342\\
0.0907	0.0204\\
0.1510	0.0390\\
0.0969	0.0154\\
0.0557	0.0569\\
0.0243	0.0960\\
0.1311	0.0035\\
0.1726	0.0423\\
0.1733	0.0433\\
0.1495	0.0185\\
0.1836	0.0503\\
0.2027	0.0703\\
0.1715	0.0457\\
0.1214	0.0025\\
0.1167	0.0001\\
0.1054	0.0100\\
0.1086	0.0014\\
0.1488	0.0425\\
0.1568	0.0476\\
0.1581	0.0463\\
0.1684	0.0613\\
0.1709	0.0698\\
0.1651	0.0650\\
0.1600	0.0587\\
0.1487	0.0483\\
0.1498	0.0514\\
0.1487	0.0523\\
0.1785	0.0847\\
0.1728	0.0817\\
0.1761	0.0850\\
0.1280	0.0368\\
0.1071	0.0180\\
0.0999	0.0099\\
0.0894	0.0043\\
0.1696	0.0742\\
0.1798	0.0836\\
0.1718	0.0716\\
0.1636	0.0581\\
0.1599	0.0508\\
0.1425	0.0308\\
0.1320	0.0182\\
0.1743	0.0566\\
0.1704	0.0477\\
0.1830	0.0570\\
0.2037	0.0771\\
0.1560	0.0271\\
0.1626	0.0261\\
0.1638	0.0199\\
0.1812	0.0358\\
0.1423	0.0046\\
0.1274	0.0238\\
0.1301	0.0300\\
0.1345	0.0362\\
0.1508	0.0286\\
0.1538	0.0307\\
0.1623	0.0244\\
0.1746	0.0130\\
0.1756	0.0138\\
0.1868	0.0065\\
0.1826	0.0149\\
0.1649	0.0360\\
0.1579	0.0436\\
0.1487	0.0481\\
0.1406	0.0450\\
0.1331	0.0420\\
0.1415	0.0291\\
0.1522	0.0192\\
0.1551	0.0116\\
0.1559	0.0013\\
0.1441	0.0062\\
0.1982	0.0477\\
0.2015	0.0459\\
0.1658	0.0025\\
0.1822	0.0143\\
0.1971	0.0314\\
0.1133	0.0491\\
0.1005	0.0628\\
0.0931	0.0756\\
0.0908	0.0847\\
0.1402	0.0419\\
0.1520	0.0330\\
0.1524	0.0304\\
0.1561	0.0216\\
0.1955	0.0174\\
0.1946	0.0111\\
0.2150	0.0288\\
0.2648	0.0796\\
0.2323	0.0451\\
0.4142	0.2226\\
0.3874	0.1953\\
0.3602	0.1720\\
0.3134	0.1258\\
0.2734	0.0835\\
0.2455	0.0562\\
0.1955	0.0099\\
0.2053	0.0179\\
0.2114	0.0151\\
0.2098	0.0005\\
0.2182	0.0043\\
0.2398	0.0084\\
0.2391	0.0027\\
0.2436	0.0007\\
0.2522	0.0043\\
0.2529	0.0166\\
0.2503	0.0168\\
0.2285	0.0151\\
0.2213	0.0123\\
0.2076	0.0479\\
0.2140	0.0668\\
0.1519	0.1343\\
0.1441	0.1448\\
0.2015	0.0781\\
0.2433	0.0184\\
0.2523	0.0044\\
0.2876	0.0481\\
0.2997	0.0711\\
0.3769	0.1525\\
0.3420	0.1135\\
0.3677	0.1435\\
0.3266	0.1217\\
0.2754	0.0862\\
0.2291	0.0473\\
0.2302	0.0484\\
0.2168	0.0321\\
0.1934	0.0098\\
0.1630	0.0165\\
0.1929	0.0147\\
0.1929	0.0136\\
0.1803	0.0013\\
0.1403	0.0353\\
0.2546	0.0855\\
0.2845	0.1180\\
0.2931	0.1218\\
0.2853	0.1057\\
0.2619	0.0776\\
0.2514	0.0670\\
0.2377	0.0562\\
0.2072	0.0229\\
0.2017	0.0090\\
0.2004	0.0022\\
0.1821	0.0157\\
0.1887	0.0075\\
0.1897	0.0070\\
0.1906	0.0111\\
0.1781	0.0314\\
0.1914	0.0253\\
0.2094	0.0139\\
0.2533	0.0248\\
0.2303	0.0004\\
0.2071	0.0194\\
0.1782	0.0472\\
0.1587	0.0737\\
0.1925	0.0549\\
0.2375	0.0243\\
0.2777	0.0093\\
0.2956	0.0309\\
0.2839	0.0225\\
0.3739	0.1130\\
0.4430	0.1775\\
0.3690	0.0937\\
0.3402	0.0438\\
0.3119	0.0010\\
0.2628	0.0502\\
0.2215	0.0839\\
0.2154	0.0906\\
0.2450	0.0759\\
0.2661	0.0720\\
0.3088	0.0371\\
0.3181	0.0202\\
0.3270	0.0066\\
0.3252	0.0143\\
0.3372	0.0206\\
0.3311	0.0369\\
0.2626	0.0907\\
0.3122	0.0202\\
0.3151	0.0115\\
0.3003	0.0300\\
0.3237	0.0063\\
0.3201	0.0018\\
0.3135	0.0072\\
0.3884	0.0939\\
0.4173	0.1242\\
0.4414	0.1433\\
0.4393	0.1404\\
0.3710	0.0827\\
0.2921	0.0277\\
0.2174	0.0259\\
0.2130	0.0244\\
0.2749	0.0349\\
0.3482	0.1100\\
0.3643	0.1326\\
0.3375	0.1132\\
0.3232	0.0993\\
0.5148	0.2853\\
0.4497	0.2177\\
0.3321	0.0955\\
0.3181	0.0696\\
0.3102	0.0462\\
0.3318	0.0605\\
0.3429	0.0694\\
0.3386	0.0729\\
0.5007	0.2390\\
0.5310	0.2673\\
0.4039	0.1414\\
0.3247	0.1037\\
0.3067	0.1038\\
0.3306	0.1332\\
0.3512	0.1515\\
0.3609	0.1668\\
0.4155	0.2460\\
0.4453	0.3149\\
0.4004	0.3029\\
0.2769	0.2038\\
0.1104	0.0684\\
-0.0043	0.0039\\
0.0447	0.0991\\
0.0838	0.1885\\
0.1039	0.2442\\
0.0958	0.2611\\
0.0913	0.2983\\
-0.5087	0.2825\\
-0.4925	0.2523\\
-0.3856	0.1240\\
-0.3503	0.0595\\
-0.4222	0.1141\\
-0.4314	0.1209\\
-0.3185	0.0026\\
-0.2830	0.0376\\
-0.4220	0.1066\\
-0.4364	0.1330\\
-0.4164	0.1057\\
-0.3784	0.0595\\
-0.3695	0.0550\\
-0.3539	0.0508\\
-0.3722	0.0732\\
-0.3720	0.0795\\
-0.3673	0.0850\\
-0.3481	0.0662\\
-0.3257	0.0333\\
-0.2758	0.0254\\
-0.2456	0.0618\\
-0.2519	0.0639\\
-0.3011	0.0091\\
-0.3728	0.0898\\
-0.3807	0.1007\\
-0.3275	0.0090\\
-0.2669	0.0831\\
-0.2484	0.0983\\
-0.2722	0.0743\\
-0.2844	0.0639\\
-0.2887	0.0515\\
-0.3282	0.0056\\
-0.3184	0.0109\\
-0.3045	0.0239\\
-0.2877	0.0372\\
-0.2823	0.0388\\
-0.2729	0.0497\\
-0.2477	0.0878\\
-0.2759	0.0722\\
-0.3031	0.0473\\
-0.3430	0.0031\\
-0.3361	0.0078\\
-0.3339	0.0097\\
-0.3208	0.0088\\
-0.3175	0.0105\\
-0.3308	0.0078\\
-0.3423	0.0108\\
-0.3357	0.0049\\
-0.3398	0.0064\\
-0.3103	0.0063\\
-0.2687	0.0406\\
-0.2508	0.0596\\
-0.2841	0.0279\\
-0.3593	0.0446\\
-0.3702	0.0523\\
-0.3438	0.0295\\
-0.2838	0.0194\\
-0.2416	0.0495\\
-0.1984	0.0844\\
-0.2220	0.0557\\
-0.2653	0.0104\\
-0.3008	0.0234\\
-0.2948	0.0168\\
-0.2912	0.0160\\
-0.2926	0.0265\\
-0.2741	0.0103\\
-0.2517	0.0103\\
-0.2626	0.0066\\
-0.2543	0.0058\\
-0.2620	0.0201\\
-0.1999	0.0389\\
-0.2334	0.0052\\
-0.2331	0.0038\\
-0.2412	0.0049\\
-0.2108	0.0296\\
-0.2697	0.0290\\
-0.2465	0.0127\\
-0.2486	0.0218\\
-0.2357	0.0173\\
-0.2287	0.0222\\
-0.2193	0.0214\\
-0.2316	0.0333\\
-0.2098	0.0104\\
-0.1880	0.0056\\
-0.2243	0.0399\\
-0.2477	0.0695\\
-0.2178	0.0463\\
-0.2016	0.0404\\
-0.2031	0.0482\\
-0.2094	0.0517\\
-0.2311	0.0674\\
-0.2042	0.0385\\
-0.2017	0.0327\\
-0.2095	0.0336\\
-0.2409	0.0608\\
-0.2495	0.0714\\
-0.2593	0.0818\\
-0.2535	0.0719\\
-0.2459	0.0584\\
-0.2447	0.0510\\
-0.2717	0.0774\\
-0.2632	0.0722\\
-0.2527	0.0607\\
-0.2217	0.0221\\
-0.2177	0.0089\\
-0.2310	0.0155\\
-0.2309	0.0123\\
-0.1881	0.0292\\
-0.1888	0.0258\\
-0.1906	0.0231\\
-0.1900	0.0265\\
-0.1864	0.0338\\
-0.1895	0.0336\\
-0.1709	0.0537\\
-0.1744	0.0525\\
-0.1764	0.0524\\
-0.1875	0.0441\\
-0.2074	0.0225\\
-0.2110	0.0147\\
-0.2159	0.0053\\
-0.2180	0.0024\\
-0.1967	0.0244\\
-0.2007	0.0213\\
-0.2274	0.0087\\
-0.2302	0.0160\\
-0.2118	0.0038\\
-0.1933	0.0260\\
-0.2180	0.0009\\
-0.2042	0.0098\\
-0.1857	0.0301\\
-0.2231	0.0071\\
-0.2360	0.0252\\
-0.2362	0.0317\\
-0.2608	0.0614\\
-0.2554	0.0576\\
-0.2295	0.0307\\
-0.2158	0.0175\\
-0.2171	0.0214\\
-0.2179	0.0245\\
-0.2038	0.0160\\
-0.2069	0.0279\\
-0.2036	0.0320\\
-0.1970	0.0264\\
-0.1862	0.0165\\
-0.1784	0.0143\\
-0.1660	0.0029\\
-0.1612	0.0094\\
-0.1462	0.0293\\
-0.1591	0.0126\\
-0.1528	0.0177\\
-0.1529	0.0214\\
-0.1711	0.0060\\
-0.1888	0.0126\\
-0.2095	0.0362\\
-0.2022	0.0338\\
-0.2151	0.0513\\
-0.1827	0.0217\\
-0.1705	0.0100\\
-0.1941	0.0349\\
-0.1829	0.0250\\
-0.1755	0.0168\\
-0.1990	0.0372\\
-0.1965	0.0321\\
-0.1951	0.0287\\
-0.1980	0.0284\\
-0.1919	0.0183\\
-0.1713	0.0050\\
-0.1612	0.0174\\
-0.1423	0.0380\\
-0.1383	0.0442\\
-0.1242	0.0585\\
-0.1125	0.0705\\
-0.2642	0.0775\\
-0.3267	0.1364\\
-0.2965	0.1070\\
-0.1986	0.0142\\
-0.1887	0.0091\\
-0.1771	0.0010\\
-0.0966	0.0772\\
-0.1280	0.0418\\
-0.1603	0.0019\\
-0.1660	0.0102\\
-0.1645	0.0098\\
-0.1694	0.0141\\
-0.1558	0.0026\\
-0.1481	0.0027\\
-0.1557	0.0044\\
-0.1192	0.0335\\
-0.0792	0.0737\\
-0.0727	0.0796\\
-0.0785	0.0745\\
-0.0873	0.0647\\
-0.1156	0.0331\\
-0.1393	0.0053\\
-0.1481	0.0045\\
-0.1366	0.0085\\
-0.1663	0.0191\\
-0.1937	0.0475\\
-0.2065	0.0658\\
-0.1996	0.0654\\
-0.2262	0.0949\\
-0.1841	0.0533\\
-0.1425	0.0164\\
-0.1422	0.0245\\
-0.1826	0.0750\\
-0.1725	0.0728\\
-0.1762	0.0827\\
-0.1862	0.0967\\
-0.2128	0.1268\\
-0.2075	0.1263\\
-0.2057	0.1328\\
-0.1998	0.1305\\
-0.1948	0.1308\\
-0.1908	0.1320\\
-0.0508	0.0013\\
-0.0478	0.0025\\
-0.0463	0.0049\\
-0.0449	0.0091\\
-0.0329	0.0052\\
-0.0278	0.0103\\
-0.0261	0.0337\\
-0.0477	0.0670\\
-0.0606	0.0942\\
-0.0588	0.1078\\
-0.0402	0.1099\\
-0.0093	0.0916\\
0.0274	0.0688\\
0.0576	0.0539\\
0.0748	0.0500\\
0.0919	0.0417\\
0.1212	0.0186\\
0.1599	0.0150\\
0.1706	0.0209\\
0.1694	0.0129\\
0.1685	0.0040\\
0.1616	0.0080\\
0.1506	0.0195\\
0.1286	0.0458\\
0.1292	0.0532\\
0.1352	0.0542\\
0.1368	0.0576\\
0.1408	0.0616\\
0.1414	0.0714\\
0.1518	0.0680\\
0.1513	0.0704\\
0.2501	0.0276\\
0.3374	0.1146\\
0.3274	0.0989\\
0.3276	0.0940\\
0.3199	0.0900\\
0.3159	0.0945\\
0.3053	0.0899\\
0.2990	0.0850\\
0.2891	0.0750\\
0.2665	0.0524\\
0.2508	0.0391\\
0.2236	0.0133\\
0.2444	0.0323\\
0.3731	0.1589\\
0.3391	0.1260\\
0.3038	0.0863\\
0.2669	0.0374\\
0.2646	0.0269\\
0.2665	0.0323\\
0.2607	0.0294\\
0.2371	0.0056\\
0.2298	0.0061\\
0.2098	0.0332\\
0.2368	0.0147\\
0.2608	0.0004\\
0.2771	0.0034\\
0.3588	0.0761\\
0.3848	0.1038\\
0.4462	0.1699\\
0.4206	0.1435\\
0.3564	0.0741\\
0.3373	0.0519\\
0.3154	0.0308\\
0.3093	0.0300\\
0.3507	0.0754\\
0.3433	0.0696\\
0.2587	0.0217\\
0.2553	0.0239\\
0.2264	0.0329\\
0.1592	0.0912\\
0.2457	0.0272\\
0.2088	0.0937\\
0.1867	0.1264\\
0.1375	0.1817\\
0.0927	0.2259\\
0.1350	0.1780\\
0.2502	0.0570\\
0.2668	0.0374\\
0.3002	0.0034\\
0.3228	0.0147\\
0.3570	0.0446\\
0.3649	0.0540\\
0.3741	0.0646\\
0.3922	0.0807\\
0.4114	0.0921\\
0.3881	0.0597\\
0.3444	0.0050\\
0.3307	0.0177\\
0.3079	0.0455\\
0.3041	0.0511\\
0.3745	0.0094\\
0.3293	0.0455\\
0.2321	0.1417\\
0.2747	0.0993\\
0.3005	0.0843\\
0.3643	0.0337\\
0.3843	0.0172\\
0.3577	0.0503\\
0.3736	0.0451\\
0.3748	0.0547\\
0.3251	0.1128\\
0.4507	0.0013\\
0.5265	0.0644\\
0.4941	0.0248\\
0.4811	0.0117\\
0.4254	0.0390\\
0.4050	0.0605\\
0.4158	0.0594\\
0.4214	0.0669\\
0.3844	0.1089\\
0.3825	0.1092\\
0.4272	0.0582\\
0.4388	0.0438\\
0.4859	0.0002\\
0.5147	0.0101\\
0.4544	0.0615\\
0.4655	0.0333\\
0.5044	0.0100\\
0.4785	0.0439\\
0.4684	0.0699\\
0.4988	0.0125\\
0.5158	0.0126\\
0.5091	0.0039\\
0.4553	0.0562\\
0.4317	0.0685\\
0.4382	0.0540\\
0.4715	0.0227\\
0.4659	0.0336\\
0.4964	0.0032\\
0.4764	0.0161\\
0.4455	0.0435\\
0.4367	0.0523\\
0.5075	0.0066\\
0.5557	0.0369\\
0.5066	0.0328\\
0.6763	0.1168\\
0.8051	0.2166\\
0.6967	0.0719\\
0.5663	0.1113\\
0.5788	0.1312\\
0.6019	0.1027\\
0.4018	0.2702\\
0.4480	0.1852\\
0.3941	0.1782\\
0.3739	0.1158\\
0.2342	0.1779\\
0.7002	0.3627\\
0.6630	0.4078\\
0.6242	0.5041\\
0.3469	0.3658\\
0.1845	0.2945\\
0.0201	0.2505\\
-0.0424	0.2711\\
-0.0792	0.3256\\
-0.0087	0.4718\\
-0.1594	0.3777\\
-0.5110	0.0971\\
-0.6180	0.0460\\
-0.6679	0.0174\\
-0.6516	0.0415\\
-0.6192	0.1159\\
-0.5756	0.1880\\
-0.5788	0.1807\\
-0.8674	0.1184\\
-0.8455	0.0931\\
-0.8478	0.0780\\
-0.7870	0.0046\\
-0.7063	0.0763\\
-0.6951	0.0818\\
-0.7450	0.0437\\
-0.7582	0.0287\\
-0.7842	0.0155\\
-0.7400	0.0018\\
-0.7200	0.0099\\
-0.7670	0.0246\\
-0.7699	0.0170\\
-0.7361	0.0020\\
-0.7165	0.0078\\
-0.6942	0.0258\\
-0.6557	0.0723\\
-0.6512	0.0607\\
-0.6870	0.0041\\
-0.6740	0.0082\\
-0.7001	0.0036\\
-0.6970	0.0109\\
-0.7009	0.0023\\
-0.6677	0.0020\\
-0.6620	0.0069\\
-0.6646	0.0096\\
-0.6297	0.0365\\
-0.6301	0.0346\\
-0.6618	0.0115\\
-0.6366	0.0007\\
-0.6514	0.0141\\
-0.6510	0.0165\\
-0.6332	0.0008\\
-0.6230	0.0023\\
-0.6554	0.0318\\
-0.6837	0.0550\\
-0.7809	0.1511\\
-0.7046	0.0842\\
-0.6308	0.0297\\
-0.5788	0.0114\\
-0.5672	0.0176\\
-0.5554	0.0306\\
-0.5504	0.0257\\
-0.5640	0.0065\\
-0.6488	0.1091\\
-0.6534	0.1212\\
-0.5664	0.0334\\
-0.4808	0.0504\\
-0.5007	0.0173\\
-0.4967	0.0003\\
-0.4996	0.0169\\
-0.3628	0.1137\\
-0.3528	0.1211\\
-0.4151	0.0465\\
-0.4654	0.0240\\
-0.5082	0.0756\\
-0.5308	0.1016\\
-0.5034	0.0850\\
-0.4530	0.0432\\
-0.4318	0.0176\\
-0.3822	0.0303\\
-0.3291	0.0698\\
-0.3504	0.0447\\
-0.3762	0.0181\\
-0.3955	0.0114\\
-0.3587	0.0109\\
-0.4303	0.0664\\
-0.3903	0.0344\\
-0.3467	0.0073\\
-0.3357	0.0133\\
-0.3529	0.0421\\
-0.3297	0.0224\\
-0.2688	0.0368\\
-0.2458	0.0525\\
-0.2318	0.0586\\
-0.2261	0.0642\\
-0.2507	0.0386\\
-0.3679	0.0857\\
-0.3400	0.0655\\
-0.2201	0.0488\\
-0.1405	0.1214\\
-0.1817	0.0728\\
-0.2175	0.0315\\
-0.2250	0.0151\\
-0.2724	0.0476\\
-0.2309	0.0272\\
-0.2169	0.0339\\
-0.1602	0.0012\\
-0.1013	0.0287\\
-0.0815	0.0126\\
-0.0776	0.0234\\
-0.0319	0.0213\\
0.0040	0.0328\\
0.0604	0.0306\\
0.0488	0.0941\\
0.0352	0.1497\\
0.0367	0.1820\\
0.0438	0.2084\\
0.0969	0.1975\\
0.4369	0.0947\\
0.4280	0.0485\\
0.3206	0.0767\\
0.3536	0.0627\\
0.4046	0.0369\\
0.4120	0.0649\\
0.3642	0.1499\\
0.4270	0.1137\\
0.4208	0.1323\\
0.4246	0.1295\\
0.4600	0.1002\\
0.4794	0.0921\\
0.5253	0.0703\\
0.5219	0.0930\\
0.5319	0.0838\\
0.5605	0.0554\\
0.6079	0.0156\\
0.5906	0.0388\\
0.5857	0.0311\\
0.5828	0.0283\\
0.6153	0.0041\\
0.6126	0.0044\\
0.5892	0.0306\\
0.5736	0.0521\\
0.5762	0.0537\\
0.5777	0.0661\\
0.5624	0.0904\\
0.5659	0.0797\\
0.5558	0.0858\\
0.5316	0.1161\\
0.5178	0.1347\\
0.5491	0.0894\\
0.5794	0.0520\\
0.6036	0.0282\\
0.6106	0.0384\\
0.5984	0.0652\\
0.7528	0.0942\\
0.7412	0.0856\\
0.6816	0.0206\\
0.6473	0.0211\\
0.6502	0.0139\\
0.7062	0.0264\\
0.6739	0.0268\\
0.6215	0.0806\\
0.6116	0.0948\\
0.5624	0.1575\\
0.6210	0.1285\\
0.7019	0.0674\\
0.7514	0.0270\\
0.7291	0.0381\\
0.6522	0.1106\\
0.7020	0.0561\\
0.6774	0.0795\\
0.7544	0.0067\\
0.8113	0.0612\\
0.7955	0.0384\\
0.7199	0.0673\\
0.6742	0.1468\\
0.7530	0.0821\\
0.7751	0.0679\\
0.7623	0.0856\\
0.7773	0.0832\\
0.7618	0.1020\\
0.7198	0.1580\\
0.7984	0.0885\\
0.7858	0.1091\\
0.7526	0.1370\\
0.7551	0.1431\\
0.7436	0.1552\\
0.7357	0.1296\\
0.7283	0.0962\\
0.7464	0.0604\\
0.6782	0.1106\\
0.5837	0.1559\\
0.5366	0.1301\\
0.5960	0.0140\\
0.4853	0.0179\\
0.3824	0.0216\\
0.3008	0.0281\\
0.1807	0.0619\\
0.0906	0.1251\\
0.0324	0.2186\\
-0.0672	0.2675\\
-0.1599	0.3417\\
-0.2471	0.4377\\
-0.5260	0.2967\\
-0.8164	0.0919\\
-1.0151	0.0332\\
-1.0736	0.0043\\
-1.0836	0.0915\\
-1.1293	0.1050\\
-1.1487	0.1004\\
-1.1199	0.1041\\
-1.2012	0.0298\\
-1.1522	0.1092\\
-1.1628	0.1377\\
-1.1479	0.1427\\
-1.2065	0.0352\\
-1.1776	0.0158\\
-1.1707	0.0121\\
-1.1780	0.0016\\
-1.1415	0.0212\\
-1.0673	0.0747\\
-1.0419	0.0626\\
-1.0496	0.0321\\
-1.0463	0.0270\\
-1.0329	0.0574\\
-1.0268	0.0500\\
-1.0053	0.0309\\
-0.9301	0.0652\\
-0.9227	0.0567\\
-0.9109	0.0628\\
-0.9116	0.0531\\
-0.9050	0.0348\\
-0.8894	0.0090\\
-0.8798	0.0064\\
-0.8796	0.0153\\
-0.8307	0.0411\\
-0.7987	0.0590\\
-0.7781	0.0466\\
-0.7760	0.0148\\
-0.7533	0.0234\\
-0.7283	0.0456\\
-0.6772	0.0900\\
-0.6566	0.0877\\
-0.6884	0.0183\\
-0.7295	0.0450\\
-0.6977	0.0308\\
-0.6661	0.0126\\
-0.6771	0.0417\\
-0.6725	0.0525\\
-0.6342	0.0323\\
-0.5884	0.0013\\
-0.5720	0.0042\\
-0.5462	0.0237\\
-0.5397	0.0124\\
-0.5352	0.0095\\
-0.4872	0.0169\\
-0.5516	0.0622\\
-0.7020	0.2289\\
-0.6689	0.2260\\
-0.5714	0.1641\\
-0.5134	0.1394\\
-0.4212	0.0750\\
-0.3758	0.0594\\
-0.3524	0.0691\\
-0.3227	0.0818\\
-0.3206	0.1305\\
-0.3033	0.1666\\
-0.2356	0.1534\\
-0.1651	0.1376\\
-0.1092	0.1436\\
-0.0755	0.1774\\
-0.0420	0.2053\\
-0.0251	0.2395\\
0.0338	0.2419\\
0.1261	0.2097\\
0.2931	0.0858\\
0.3675	0.0418\\
0.3515	0.0892\\
0.3605	0.1190\\
0.4227	0.0896\\
0.4641	0.0451\\
0.4504	0.0264\\
0.4420	0.0607\\
0.4840	0.0816\\
0.4897	0.1180\\
0.4328	0.1812\\
0.4484	0.1737\\
0.4805	0.1492\\
0.4807	0.1606\\
0.4948	0.1492\\
0.5395	0.0966\\
0.5566	0.0744\\
0.5681	0.0654\\
0.5867	0.0595\\
0.5899	0.0653\\
0.5983	0.0770\\
0.6061	0.0956\\
0.6197	0.1003\\
0.6582	0.0597\\
0.6357	0.1006\\
0.6548	0.1081\\
0.7305	0.0383\\
0.7491	0.0202\\
0.7464	0.0239\\
0.7512	0.0349\\
0.7645	0.0374\\
0.7759	0.0562\\
0.7692	0.0923\\
0.7872	0.0953\\
0.7916	0.0939\\
0.8046	0.1005\\
0.7565	0.1706\\
0.7901	0.1345\\
0.7965	0.1227\\
0.7954	0.1209\\
0.8850	0.0408\\
0.9903	0.0641\\
0.9356	0.0119\\
0.9152	0.0714\\
0.9181	0.0932\\
0.8936	0.1049\\
0.8808	0.1330\\
0.9127	0.1262\\
0.9000	0.1245\\
0.9212	0.0887\\
0.8595	0.1537\\
0.7939	0.2560\\
0.7958	0.2786\\
0.8901	0.2159\\
0.8958	0.2199\\
0.8559	0.2394\\
0.8525	0.2138\\
0.9455	0.0899\\
1.0727	0.0593\\
0.9666	0.0070\\
0.8633	0.0624\\
0.8943	0.0329\\
0.9207	0.1176\\
0.8475	0.1144\\
0.9371	0.2881\\
0.7887	0.2457\\
0.4765	0.0714\\
0.2794	0.0533\\
0.1547	0.1248\\
0.0392	0.1978\\
-0.0340	0.3081\\
-0.2302	0.3025\\
-0.3048	0.4245\\
-0.5129	0.4273\\
-0.8775	0.2628\\
-1.1243	0.1539\\
-1.0787	0.2719\\
-0.9191	0.4361\\
-0.9233	0.4699\\
-1.0744	0.3682\\
-1.1200	0.3970\\
-1.2742	0.2532\\
-1.3655	0.1337\\
-1.3928	0.0642\\
-1.3525	0.1058\\
-1.3077	0.1685\\
-1.2950	0.1869\\
-1.4161	0.0407\\
-1.4319	0.0204\\
-1.3716	0.0458\\
-1.3810	0.0140\\
-1.3742	0.0246\\
-1.3827	0.0613\\
-1.3393	0.0338\\
-1.2638	0.0071\\
-1.3099	0.0973\\
-1.2716	0.0883\\
-1.1702	0.0018\\
-1.1708	0.0101\\
-1.0971	0.0578\\
-1.0868	0.0188\\
-1.0207	0.0368\\
-0.9978	0.0362\\
-1.0007	0.0059\\
-0.9574	0.0309\\
-0.9281	0.0331\\
-0.8988	0.0470\\
-0.9250	0.0120\\
-0.9507	0.0293\\
-0.9415	0.0536\\
-0.7697	0.0743\\
-0.7435	0.0778\\
-0.7582	0.0271\\
-0.7583	0.0235\\
-0.7789	0.0952\\
-0.7733	0.1287\\
-0.7155	0.1108\\
-0.6460	0.0901\\
-0.5614	0.0655\\
-0.5115	0.0866\\
-0.4732	0.1189\\
-0.4371	0.1604\\
-0.3839	0.1887\\
-0.3420	0.2272\\
-0.2317	0.2001\\
-0.1283	0.1866\\
-0.0694	0.2255\\
-0.0373	0.2910\\
-0.0030	0.3515\\
0.0113	0.4285\\
0.1204	0.4108\\
0.3292	0.2819\\
0.4331	0.2412\\
0.5743	0.1454\\
0.6167	0.1543\\
0.5967	0.2302\\
0.7665	0.1407\\
0.8449	0.1341\\
0.8952	0.1176\\
0.9541	0.0962\\
1.0400	0.0623\\
1.1369	0.0141\\
1.0406	0.1122\\
1.0515	0.1187\\
1.1084	0.0886\\
1.1788	0.0863\\
1.1980	0.1349\\
1.1541	0.2010\\
1.2160	0.1684\\
1.3003	0.1234\\
1.3083	0.1573\\
1.2424	0.2103\\
1.2705	0.1874\\
1.3474	0.1087\\
1.3598	0.0997\\
1.3691	0.0626\\
1.3278	0.0906\\
1.2854	0.1467\\
1.3010	0.1400\\
1.3013	0.1593\\
1.2832	0.1675\\
1.3069	0.1033\\
1.3532	0.0365\\
1.1881	0.1852\\
1.1147	0.2975\\
1.1187	0.3433\\
1.1345	0.3658\\
1.1786	0.2977\\
1.3149	0.1727\\
1.5159	0.0840\\
1.1592	0.1328\\
1.0810	0.0878\\
0.9174	0.0579\\
0.7148	0.0246\\
0.5584	0.0586\\
0.4224	0.1759\\
0.3073	0.3284\\
0.1958	0.4707\\
-0.0090	0.4900\\
-0.3155	0.3967\\
-0.6275	0.3118\\
-0.8144	0.3719\\
-1.0036	0.4255\\
-1.2742	0.3626\\
-1.4758	0.2594\\
-1.6304	0.1743\\
-1.7548	0.0769\\
-1.7860	0.1093\\
-1.7457	0.2214\\
-1.7881	0.2131\\
-1.9895	0.0194\\
-1.9721	0.1035\\
-1.7103	0.1425\\
-1.7919	0.0065\\
-1.7848	0.1151\\
-1.6792	0.0600\\
-1.7082	0.1245\\
-1.5954	0.0142\\
-1.5236	0.0076\\
-1.4741	0.0151\\
-1.3962	0.0203\\
-1.2838	0.0482\\
-1.2800	0.0148\\
-1.2313	0.0187\\
-1.2321	0.0386\\
-1.1300	0.0159\\
-1.1087	0.0470\\
-1.1134	0.0926\\
-0.9631	0.0317\\
-0.8667	0.0760\\
-0.8887	0.0159\\
-0.9191	0.1160\\
-0.8769	0.1451\\
-0.8233	0.1711\\
-0.7295	0.1727\\
-0.5680	0.1070\\
-0.5115	0.1572\\
-0.3853	0.1462\\
-0.3008	0.1850\\
-0.2366	0.2434\\
-0.1776	0.2997\\
-0.0026	0.2514\\
0.0151	0.3753\\
0.0392	0.4814\\
0.1404	0.4782\\
0.4548	0.2694\\
0.7044	0.1358\\
0.7954	0.1898\\
0.7952	0.3218\\
0.8462	0.3513\\
0.8783	0.3816\\
0.9737	0.3500\\
1.0704	0.3591\\
1.1402	0.4075\\
1.3437	0.3312\\
1.7073	0.0506\\
1.5653	0.1967\\
1.7577	0.1731\\
1.7002	0.3198\\
1.7909	0.2486\\
1.8940	0.1751\\
2.0324	0.0516\\
2.0176	0.1427\\
1.9172	0.2667\\
1.9249	0.3159\\
1.9649	0.2991\\
1.9386	0.4050\\
2.0412	0.3605\\
2.0024	0.4981\\
1.9551	0.5164\\
1.9376	0.4673\\
1.9879	0.1957\\
2.0263	0.0922\\
1.7614	0.5222\\
1.9610	1.1436\\
0.9772	1.3894\\
0.3772	1.2782\\
-0.2228	1.0764\\
-0.8228	0.7658\\
-1.4228	0.4672\\
-1.9243	0.2061\\
-1.9347	0.0358\\
-1.7526	0.0564\\
-1.6470	0.1179\\
-1.5256	0.1134\\
-1.3307	0.0315\\
-1.3188	0.1172\\
-1.3229	0.2301\\
-1.2596	0.2690\\
-1.1369	0.2574\\
-0.9673	0.1957\\
-0.8205	0.1737\\
-0.6502	0.1385\\
-0.4599	0.0953\\
-0.3866	0.1696\\
-0.2994	0.2229\\
-0.1640	0.2372\\
-0.0066	0.2436\\
0.0397	0.3496\\
0.1287	0.3867\\
0.2922	0.3581\\
0.5992	0.2008\\
0.8531	0.1388\\
0.9171	0.2617\\
1.0432	0.2651\\
1.2812	0.1660\\
1.5772	0.1575\\
1.5537	0.2387\\
1.7038	0.2099\\
1.7971	0.2582\\
1.7858	0.4030\\
1.8516	0.4695\\
1.9124	0.4360\\
1.9507	0.4738\\
2.0460	0.6270\\
2.0381	0.5954\\
1.9495	0.5073\\
1.8438	0.4492\\
1.8032	0.2836\\
1.7069	0.1691\\
1.3785	0.0184\\
0.9702	0.0063\\
0.5454	0.1025\\
0.3777	0.4793\\
0.0218	0.6208\\
-0.5264	0.5565\\
-1.4264	0.1556\\
-1.8095	0.1481\\
-2.0005	0.0068\\
-1.6461	0.1591\\
-1.4090	0.3213\\
-1.4672	0.1491\\
-1.5309	0.0511\\
-1.6905	0.3580\\
-1.4513	0.2353\\
-1.1485	0.0557\\
-1.0352	0.0556\\
-0.9644	0.1015\\
-0.9013	0.1535\\
-0.7358	0.1026\\
-0.4880	0.0238\\
-0.5462	0.1538\\
-0.4819	0.2062\\
-0.3929	0.2272\\
-0.2419	0.1825\\
-0.1284	0.1747\\
-0.0906	0.2876\\
0.0403	0.2507\\
0.2465	0.1346\\
0.2734	0.1908\\
0.3240	0.2193\\
0.4289	0.1888\\
0.5823	0.1273\\
0.6970	0.0941\\
0.8155	0.0188\\
0.8855	0.0146\\
0.8573	0.0599\\
0.8562	0.0987\\
0.8675	0.0878\\
0.9229	0.0378\\
0.9570	0.0135\\
0.8799	0.1194\\
0.8456	0.1712\\
0.8592	0.1431\\
0.7970	0.1829\\
0.7920	0.1675\\
0.8099	0.1390\\
0.8831	0.0453\\
0.8514	0.0649\\
0.8110	0.0924\\
0.8273	0.0712\\
0.8898	0.0061\\
0.8822	0.0030\\
0.8113	0.0678\\
0.8524	0.0257\\
0.8541	0.0004\\
0.8297	0.0088\\
0.8304	0.0009\\
0.8223	0.0237\\
0.7812	0.0796\\
0.7848	0.0651\\
0.8962	0.0553\\
0.8465	0.0141\\
0.7387	0.0562\\
0.6681	0.0542\\
0.7169	0.0339\\
0.6927	0.1434\\
0.6972	0.1944\\
0.6416	0.2536\\
0.6260	0.2763\\
0.6695	0.2317\\
0.7051	0.1928\\
0.7976	0.0857\\
0.9126	0.0274\\
0.9113	0.0129\\
0.8433	0.0696\\
0.8228	0.0739\\
0.8227	0.0425\\
0.8054	0.0138\\
0.7694	0.0249\\
0.6756	0.1115\\
0.6226	0.1797\\
0.5692	0.1828\\
0.5627	0.1290\\
0.6030	0.0432\\
0.5548	0.0720\\
0.4957	0.1043\\
0.5017	0.0576\\
0.4177	0.0829\\
0.4107	0.0294\\
0.4150	0.0375\\
0.0306	0.2732\\
-0.0059	0.2238\\
0.0167	0.1076\\
0.0269	0.0034\\
0.0231	0.1032\\
-0.0574	0.2682\\
-0.3463	0.0482\\
-0.3397	0.1245\\
-0.3209	0.2227\\
-0.3209	0.2824\\
-0.4381	0.2004\\
-0.5704	0.1006\\
-0.6570	0.0520\\
-0.6295	0.1045\\
-0.7431	0.0096\\
-0.8695	0.0990\\
-0.8542	0.0576\\
-0.9103	0.0849\\
-0.9516	0.1092\\
-0.8168	0.0224\\
-0.8909	0.0784\\
-0.8897	0.0952\\
-0.8067	0.0300\\
-0.7551	0.0152\\
-0.7471	0.0162\\
-0.7480	0.0085\\
-0.7853	0.0416\\
-0.7938	0.0534\\
-0.7242	0.0127\\
-0.6665	0.0699\\
-0.6395	0.0953\\
-0.6492	0.0861\\
-0.8151	0.0983\\
-0.8209	0.1348\\
-0.8085	0.1491\\
-0.7308	0.0852\\
-0.6208	0.0152\\
-0.6295	0.0077\\
-0.5513	0.0452\\
-0.4614	0.0990\\
-0.3867	0.1442\\
-0.4079	0.0948\\
-0.4245	0.0518\\
-0.4144	0.0334\\
-0.4087	0.0122\\
-0.4096	0.0197\\
-0.4102	0.0533\\
-0.3213	0.0028\\
-0.3403	0.0613\\
-0.3487	0.1145\\
-0.3216	0.1362\\
-0.2762	0.1417\\
-0.2170	0.1326\\
-0.1374	0.1020\\
-0.0917	0.1044\\
-0.0691	0.1376\\
-0.0557	0.1864\\
-0.0487	0.2312\\
-0.0166	0.2312\\
0.1063	0.1369\\
0.0191	0.0445\\
0.0504	0.0617\\
0.0460	0.0438\\
0.0488	0.0426\\
0.1311	0.0401\\
0.1454	0.0571\\
0.1518	0.0618\\
0.1379	0.0421\\
0.1242	0.0237\\
0.0945	0.0075\\
0.0856	0.0248\\
0.0854	0.0396\\
0.1797	0.0765\\
0.1989	0.1027\\
0.1551	0.0422\\
0.1400	0.0253\\
0.1689	0.0638\\
0.1844	0.0871\\
0.1357	0.0403\\
0.1387	0.0413\\
0.1169	0.0254\\
0.1237	0.0303\\
0.1100	0.0130\\
0.0976	0.0022\\
0.1212	0.0259\\
0.0924	0.0081\\
0.1311	0.0262\\
0.1266	0.0201\\
0.1123	0.0032\\
0.0948	0.0284\\
0.1026	0.0189\\
0.1288	0.0124\\
0.1307	0.0150\\
0.1351	0.0126\\
0.1405	0.0136\\
0.1272	0.0087\\
0.0917	0.0581\\
0.0897	0.0640\\
0.1048	0.0333\\
0.1207	0.0060\\
0.0960	0.0442\\
0.0701	0.0909\\
0.0826	0.0841\\
0.1627	0.0000\\
0.2337	0.0742\\
0.2439	0.0889\\
0.1107	0.0426\\
0.1580	0.0115\\
0.1683	0.0167\\
0.1829	0.0038\\
0.1841	0.0165\\
0.1700	0.0041\\
0.1433	0.0379\\
0.1557	0.0140\\
0.1779	0.0184\\
0.2056	0.0372\\
0.2505	0.0622\\
0.2049	0.0029\\
0.2094	0.0132\\
0.1969	0.0285\\
0.2044	0.0581\\
0.2133	0.0610\\
0.1963	0.0296\\
0.2015	0.0393\\
0.2118	0.0655\\
0.2089	0.0767\\
0.1991	0.0677\\
0.1714	0.0261\\
0.1560	0.0080\\
0.1548	0.0178\\
0.1530	0.0099\\
0.1532	0.0155\\
0.1530	0.0337\\
0.1491	0.0139\\
0.1467	0.0077\\
0.1829	0.0152\\
0.2377	0.0632\\
0.2481	0.0715\\
0.3011	0.1188\\
0.2698	0.0768\\
0.2850	0.0825\\
0.2810	0.0757\\
0.2309	0.0288\\
0.2128	0.0106\\
0.1769	0.0255\\
0.1587	0.0268\\
0.1638	0.0098\\
0.1601	0.0230\\
0.1241	0.0115\\
0.1231	0.0102\\
0.1118	0.0184\\
0.1018	0.0286\\
0.1282	0.0116\\
0.2312	0.0770\\
0.3488	0.1858\\
0.2290	0.0716\\
0.1586	0.0176\\
0.1688	0.0472\\
0.2791	0.1610\\
0.3051	0.1754\\
0.3126	0.1737\\
0.2837	0.1477\\
0.2478	0.1229\\
0.2167	0.1044\\
0.1916	0.0881\\
0.1731	0.0755\\
0.1644	0.0728\\
0.1068	0.0105\\
0.0778	0.0326\\
0.1648	0.0505\\
0.1802	0.0809\\
0.1906	0.1008\\
0.1684	0.0794\\
0.1533	0.0722\\
0.1334	0.0680\\
0.1143	0.0595\\
0.1106	0.0606\\
0.0935	0.0477\\
0.0829	0.0428\\
0.0769	0.0446\\
0.0630	0.0418\\
0.0466	0.0727\\
0.0290	0.0822\\
0.0227	0.1249\\
0.0127	0.1185\\
0.0071	0.1147\\
-0.0151	0.0977\\
-0.0207	0.1133\\
-0.0727	0.0963\\
-0.0989	0.1020\\
-0.1650	0.0507\\
-0.1771	0.0395\\
-0.1706	0.0382\\
-0.2039	0.0042\\
-0.2020	0.0085\\
-0.1947	0.0008\\
-0.2089	0.0077\\
-0.1933	0.0167\\
-0.2018	0.0142\\
-0.1670	0.0479\\
-0.1793	0.0213\\
-0.1684	0.0101\\
-0.1670	0.0068\\
-0.1844	0.0302\\
-0.1826	0.0207\\
-0.2151	0.0390\\
-0.2555	0.0695\\
-0.2193	0.0346\\
-0.2092	0.0332\\
-0.2671	0.0973\\
-0.2710	0.1072\\
-0.2001	0.0436\\
-0.2167	0.0689\\
-0.2544	0.1110\\
-0.2423	0.0870\\
-0.2219	0.0503\\
-0.0835	0.0931\\
-0.1145	0.0520\\
-0.1341	0.0188\\
-0.1224	0.0140\\
-0.1178	0.0064\\
-0.1482	0.0148\\
-0.1671	0.0160\\
-0.1720	0.0135\\
-0.1578	0.0045\\
-0.1811	0.0161\\
-0.1737	0.0153\\
-0.0614	0.0847\\
-0.0396	0.1012\\
-0.0726	0.0728\\
-0.0935	0.0655\\
-0.1070	0.0664\\
-0.1962	0.0225\\
-0.2060	0.0495\\
-0.2149	0.0774\\
-0.1946	0.0623\\
-0.1559	0.0208\\
-0.1431	0.0075\\
-0.1401	0.0075\\
-0.1133	0.0077\\
-0.0932	0.0157\\
-0.0696	0.0375\\
-0.0701	0.0481\\
-0.1234	0.0172\\
-0.1281	0.0272\\
-0.1601	0.0559\\
-0.2377	0.1301\\
-0.2326	0.1261\\
-0.1858	0.0880\\
-0.1559	0.0690\\
-0.1388	0.0509\\
-0.1389	0.0379\\
-0.1575	0.0552\\
-0.1426	0.0516\\
-0.1268	0.0386\\
-0.1147	0.0193\\
-0.0974	0.0052\\
-0.0907	0.0146\\
-0.0787	0.0263\\
-0.0681	0.0327\\
-0.0872	0.0086\\
-0.1137	0.0209\\
-0.1054	0.0124\\
-0.0991	0.0059\\
-0.0966	0.0028\\
-0.0818	0.0199\\
-0.0932	0.0072\\
-0.0988	0.0086\\
-0.0993	0.0155\\
-0.1005	0.0165\\
-0.1090	0.0249\\
-0.1129	0.0298\\
-0.1494	0.0672\\
-0.1844	0.1024\\
-0.1890	0.1062\\
-0.1378	0.0545\\
-0.1190	0.0340\\
-0.1181	0.0242\\
-0.1230	0.0112\\
-0.1524	0.0255\\
-0.1009	0.0148\\
-0.0732	0.0167\\
-0.0763	0.0013\\
-0.1162	0.0331\\
-0.1033	0.0103\\
-0.1293	0.0264\\
-0.0849	0.0256\\
-0.0655	0.0467\\
-0.0628	0.0457\\
-0.0637	0.0365\\
-0.0760	0.0145\\
-0.0873	0.0074\\
-0.0884	0.0112\\
-0.0792	0.0045\\
-0.0787	0.0043\\
-0.0878	0.0049\\
-0.0652	0.0289\\
-0.0913	0.0142\\
-0.0967	0.0101\\
-0.1256	0.0199\\
-0.1526	0.0499\\
-0.1716	0.0729\\
-0.2046	0.1094\\
-0.2387	0.1474\\
-0.2190	0.1285\\
-0.2187	0.1251\\
-0.2217	0.1255\\
-0.1532	0.0558\\
-0.1562	0.0452\\
-0.1640	0.0356\\
-0.1700	0.0351\\
-0.1708	0.0456\\
-0.1439	0.0329\\
-0.1376	0.0313\\
-0.1341	0.0308\\
-0.0988	0.0031\\
-0.0944	0.0016\\
-0.1079	0.0053\\
-0.1592	0.0516\\
-0.1481	0.0428\\
-0.1407	0.0434\\
-0.1363	0.0425\\
-0.1377	0.0353\\
-0.1355	0.0175\\
-0.1335	0.0045\\
-0.1444	0.0136\\
-0.1645	0.0402\\
-0.1696	0.0582\\
-0.1662	0.0631\\
-0.1516	0.0461\\
-0.1760	0.0587\\
-0.1741	0.0466\\
-0.1547	0.0276\\
-0.1307	0.0131\\
-0.0980	0.0153\\
-0.0928	0.0269\\
-0.0835	0.0434\\
-0.0736	0.0558\\
-0.1107	0.0147\\
-0.1062	0.0147\\
-0.1323	0.0151\\
-0.1693	0.0580\\
-0.1535	0.0527\\
-0.1832	0.0956\\
-0.1834	0.1089\\
-0.1710	0.1046\\
-0.1425	0.0755\\
-0.1215	0.0461\\
-0.1056	0.0232\\
-0.0685	0.0220\\
-0.0611	0.0368\\
-0.0601	0.0313\\
-0.0793	0.0021\\
-0.0787	0.0082\\
-0.0740	0.0024\\
-0.0632	0.0157\\
-0.0654	0.0192\\
-0.0703	0.0142\\
-0.0815	0.0029\\
-0.1367	0.0570\\
-0.1520	0.0769\\
-0.1156	0.0439\\
-0.1244	0.0529\\
-0.1146	0.0420\\
-0.1126	0.0379\\
-0.0984	0.0150\\
-0.1059	0.0235\\
-0.0982	0.0204\\
-0.0857	0.0085\\
-0.0646	0.0112\\
-0.0442	0.0268\\
-0.0363	0.0339\\
0.0482	0.1250\\
0.0835	0.1775\\
0.0823	0.1928\\
0.0508	0.1672\\
0.0236	0.1416\\
-0.0189	0.1049\\
-0.0680	0.0534\\
-0.0790	0.0306\\
-0.1086	0.0003\\
-0.2080	0.0900\\
-0.1879	0.0687\\
-0.1449	0.0382\\
-0.1295	0.0281\\
-0.0977	0.0025\\
-0.0883	0.0101\\
-0.0907	0.0014\\
-0.0980	0.0142\\
-0.1192	0.0410\\
-0.1273	0.0479\\
-0.1209	0.0357\\
-0.1086	0.0223\\
-0.1035	0.0244\\
-0.1037	0.0284\\
-0.1068	0.0207\\
-0.0888	0.0056\\
-0.0780	0.0040\\
-0.0724	0.0122\\
-0.0791	0.0064\\
-0.0794	0.0003\\
-0.0926	0.0241\\
-0.1115	0.0550\\
-0.0878	0.0472\\
-0.0839	0.0626\\
-0.0697	0.0645\\
-0.0628	0.0750\\
-0.0860	0.1199\\
-0.0773	0.1308\\
-0.0725	0.1409\\
-0.0601	0.1535\\
-0.0369	0.1546\\
-0.0337	0.1617\\
-0.0319	0.1595\\
-0.1578	0.2854\\
-0.1729	0.2970\\
-0.1507	0.2718\\
-0.0415	0.1196\\
0.0040	0.0673\\
-0.0139	0.0883\\
-0.0119	0.0960\\
-0.0006	0.0903\\
0.0390	0.0465\\
0.0589	0.0257\\
0.1006	0.0061\\
0.1103	0.0175\\
0.1304	0.0390\\
0.1323	0.0384\\
0.1338	0.0269\\
0.1211	0.0173\\
0.0862	0.0043\\
0.0714	0.0094\\
0.0715	0.0152\\
0.1085	0.0035\\
0.1208	0.0046\\
0.1162	0.0076\\
0.1151	0.0190\\
0.1133	0.0166\\
0.0806	0.0217\\
0.1054	0.0018\\
0.1040	0.0078\\
0.0996	0.0169\\
0.0861	0.0217\\
0.0843	0.0091\\
0.0726	0.0130\\
0.0888	0.0012\\
0.1068	0.0114\\
0.1335	0.0342\\
0.1937	0.1004\\
0.2223	0.1383\\
0.1928	0.1121\\
0.1801	0.0961\\
0.1875	0.0962\\
0.1624	0.0648\\
0.1876	0.0836\\
0.1833	0.0723\\
0.1613	0.0472\\
0.1499	0.0397\\
0.1554	0.0400\\
0.0265	0.1042\\
0.0978	0.0501\\
0.1187	0.0354\\
0.1610	0.0166\\
0.1820	0.0496\\
0.1800	0.0527\\
0.1728	0.0435\\
0.1601	0.0158\\
0.1313	0.0308\\
0.1298	0.0304\\
0.1563	0.0207\\
0.1576	0.0338\\
0.1641	0.0334\\
0.1527	0.0106\\
0.1645	0.0248\\
0.1503	0.0212\\
0.1435	0.0201\\
0.1314	0.0078\\
0.1294	0.0004\\
0.1077	0.0304\\
0.1161	0.0293\\
0.1718	0.0294\\
0.2681	0.1366\\
0.2188	0.0927\\
0.1638	0.0341\\
0.1431	0.0024\\
0.1517	0.0014\\
0.1239	0.0374\\
0.1140	0.0432\\
0.1622	0.0207\\
0.1615	0.0310\\
0.1573	0.0203\\
0.1451	0.0044\\
0.1634	0.0108\\
0.1440	0.0050\\
0.1382	0.0087\\
0.1434	0.0048\\
0.1405	0.0110\\
0.1452	0.0089\\
0.1416	0.0108\\
0.1517	0.0098\\
0.1694	0.0341\\
0.1645	0.0216\\
0.1660	0.0068\\
0.1754	0.0054\\
0.1489	0.0239\\
0.1645	0.0062\\
0.1500	0.0207\\
0.1461	0.0285\\
0.1698	0.0036\\
0.1794	0.0141\\
0.1775	0.0197\\
0.1669	0.0132\\
0.1657	0.0131\\
0.1783	0.0247\\
0.1985	0.0377\\
0.1944	0.0166\\
0.1793	0.0204\\
0.1941	0.0184\\
0.2088	0.0006\\
0.2091	0.0125\\
0.2111	0.0272\\
0.2173	0.0370\\
0.2121	0.0168\\
0.2108	0.0080\\
0.2205	0.0036\\
0.2257	0.0270\\
0.2150	0.0367\\
0.1982	0.0262\\
0.1931	0.0194\\
0.2032	0.0265\\
0.2089	0.0312\\
0.2471	0.0679\\
0.2320	0.0456\\
0.2122	0.0082\\
0.1980	0.0233\\
0.1861	0.0286\\
0.2245	0.0266\\
0.2088	0.0139\\
0.2335	0.0334\\
0.2603	0.0646\\
0.2113	0.0199\\
0.2085	0.0209\\
0.2246	0.0387\\
0.2252	0.0394\\
0.1948	0.0121\\
0.2493	0.0782\\
0.2550	0.0836\\
0.2615	0.0725\\
0.2414	0.0456\\
0.2272	0.0433\\
0.2506	0.0824\\
0.3232	0.1567\\
0.2855	0.1022\\
0.2340	0.0352\\
0.1989	0.0061\\
0.2171	0.0259\\
0.2574	0.0468\\
0.2583	0.0333\\
0.2558	0.0401\\
0.2223	0.0183\\
0.2256	0.0294\\
0.2698	0.0811\\
0.2369	0.0530\\
0.2630	0.0841\\
0.2676	0.1004\\
0.4301	0.2817\\
0.2856	0.1544\\
0.2605	0.1439\\
0.2137	0.1169\\
0.1301	0.0607\\
0.1930	0.1544\\
0.2258	0.2169\\
0.2187	0.2366\\
0.1422	0.1871\\
0.0811	0.1555\\
0.0515	0.1553\\
0.0666	0.2004\\
-0.0074	0.1697\\
-0.1485	0.0804\\
-0.1702	0.0935\\
-0.1872	0.0889\\
-0.2679	0.0133\\
-0.3334	0.0543\\
-0.3294	0.0587\\
-0.3229	0.0346\\
-0.3841	0.0528\\
-0.4134	0.0740\\
-0.3972	0.1012\\
-0.3991	0.1191\\
-0.3878	0.0862\\
-0.3590	0.0471\\
-0.3214	0.0175\\
-0.2819	0.0134\\
-0.2783	0.0085\\
-0.3028	0.0237\\
-0.3082	0.0424\\
-0.3114	0.0637\\
-0.3158	0.0685\\
-0.3256	0.0577\\
-0.3220	0.0411\\
-0.3415	0.0721\\
-0.3140	0.0557\\
-0.2858	0.0277\\
-0.2903	0.0263\\
-0.2943	0.0307\\
-0.3028	0.0519\\
-0.2957	0.0650\\
-0.2924	0.0798\\
-0.2896	0.0811\\
-0.2649	0.0476\\
-0.2557	0.0382\\
-0.2259	0.0262\\
-0.2466	0.0603\\
-0.2433	0.0528\\
-0.2330	0.0266\\
-0.2377	0.0214\\
-0.2318	0.0227\\
-0.2215	0.0171\\
-0.2288	0.0120\\
-0.2392	0.0122\\
-0.2253	0.0043\\
-0.2238	0.0120\\
-0.2103	0.0044\\
-0.2355	0.0291\\
-0.2314	0.0284\\
-0.2397	0.0504\\
-0.2666	0.0816\\
-0.2437	0.0428\\
-0.2543	0.0334\\
-0.2849	0.0632\\
-0.2768	0.0579\\
-0.3319	0.0933\\
-0.2833	0.0296\\
-0.2601	0.0186\\
-0.3022	0.0766\\
-0.3606	0.1276\\
-0.4051	0.1719\\
-0.3600	0.1347\\
-0.3371	0.1000\\
-0.3181	0.0717\\
-0.3158	0.0877\\
-0.3385	0.1255\\
-0.3029	0.0771\\
-0.3109	0.0837\\
-0.2617	0.0651\\
-0.2399	0.0534\\
-0.2321	0.0377\\
-0.2206	0.0256\\
-0.2310	0.0456\\
-0.2347	0.0559\\
-0.1497	0.0273\\
-0.1238	0.0560\\
-0.1498	0.0304\\
-0.1817	0.0045\\
-0.1614	0.0110\\
-0.2338	0.0561\\
-0.2445	0.0518\\
-0.2513	0.0434\\
-0.2248	0.0114\\
-0.2137	0.0057\\
-0.2023	0.0246\\
-0.1992	0.0228\\
-0.1865	0.0158\\
-0.1682	0.0131\\
-0.1548	0.0184\\
-0.1563	0.0164\\
-0.1938	0.0241\\
-0.1908	0.0280\\
-0.2404	0.0725\\
-0.2131	0.0346\\
-0.2003	0.0278\\
-0.1728	0.0217\\
-0.1802	0.0407\\
-0.1741	0.0505\\
-0.1292	0.0076\\
-0.2014	0.0705\\
-0.2090	0.0702\\
-0.1911	0.0533\\
-0.1786	0.0520\\
-0.1751	0.0555\\
-0.1846	0.0669\\
-0.1752	0.0501\\
-0.1484	0.0128\\
-0.1309	0.0021\\
-0.1261	0.0094\\
-0.1366	0.0110\\
-0.1577	0.0049\\
-0.1620	0.0057\\
-0.1850	0.0329\\
-0.2117	0.0644\\
-0.1870	0.0363\\
-0.1623	0.0023\\
-0.1537	0.0122\\
-0.1542	0.0079\\
-0.1454	0.0094\\
-0.1836	0.0211\\
-0.2058	0.0330\\
-0.1843	0.0165\\
-0.1775	0.0140\\
-0.2533	0.0775\\
-0.2729	0.0885\\
-0.2584	0.0830\\
-0.2383	0.0680\\
-0.2327	0.0577\\
-0.2392	0.0548\\
-0.2596	0.0698\\
-0.2937	0.1057\\
-0.2720	0.0914\\
-0.2115	0.0361\\
-0.1790	0.0101\\
-0.1666	0.0069\\
-0.1984	0.0446\\
-0.1803	0.0239\\
-0.1711	0.0123\\
-0.1640	0.0104\\
-0.1920	0.0433\\
-0.1684	0.0186\\
-0.2174	0.0705\\
-0.2027	0.0665\\
-0.1834	0.0506\\
-0.1814	0.0397\\
-0.1862	0.0263\\
-0.2221	0.0507\\
-0.2231	0.0567\\
-0.1382	0.0180\\
-0.1119	0.0456\\
-0.1259	0.0374\\
-0.1181	0.0462\\
-0.1194	0.0404\\
-0.1938	0.0405\\
-0.2331	0.0866\\
-0.2159	0.0738\\
-0.2066	0.0616\\
-0.2064	0.0542\\
-0.2060	0.0574\\
-0.1787	0.0464\\
-0.1749	0.0487\\
-0.2000	0.0646\\
-0.2008	0.0551\\
-0.2982	0.1519\\
-0.3010	0.1639\\
-0.3107	0.1829\\
-0.2825	0.1561\\
-0.2532	0.1239\\
-0.2054	0.0759\\
-0.1984	0.0843\\
-0.1776	0.0629\\
-0.1429	0.0307\\
-0.1335	0.0291\\
-0.1197	0.0167\\
-0.1228	0.0141\\
-0.2744	0.1634\\
-0.2620	0.1583\\
-0.2688	0.1791\\
-0.1707	0.0974\\
-0.1672	0.1169\\
-0.1618	0.1622\\
-0.1483	0.1714\\
-0.1069	0.1593\\
-0.0960	0.1669\\
-0.1073	0.1937\\
-0.1045	0.2204\\
-0.1020	0.2519\\
-0.0762	0.2470\\
0.0722	0.1248\\
0.2243	0.0427\\
0.4325	0.1637\\
0.3698	0.0986\\
0.3092	0.0281\\
0.2482	0.0482\\
0.2362	0.0725\\
0.2275	0.0832\\
0.2689	0.0424\\
0.3664	0.0536\\
0.3370	0.0113\\
0.3134	0.0238\\
0.3202	0.0164\\
0.2972	0.0277\\
0.2942	0.0164\\
0.2964	0.0094\\
0.3289	0.0166\\
0.2913	0.0260\\
0.2698	0.0413\\
0.2956	0.0081\\
0.2850	0.0094\\
0.2758	0.0056\\
0.3273	0.0458\\
0.3197	0.0214\\
0.2991	0.0062\\
0.2899	0.0064\\
0.2751	0.0169\\
0.2478	0.0078\\
0.2310	0.0115\\
0.2194	0.0300\\
0.2301	0.0105\\
0.2310	0.0076\\
0.2248	0.0104\\
0.2365	0.0171\\
0.2264	0.0050\\
0.2350	0.0033\\
0.2496	0.0219\\
0.2429	0.0444\\
0.2214	0.0412\\
0.2136	0.0205\\
0.2075	0.0073\\
0.2265	0.0150\\
0.2289	0.0315\\
0.2432	0.0498\\
0.2613	0.0591\\
0.2399	0.0238\\
0.2136	0.0165\\
0.2239	0.0136\\
0.2420	0.0029\\
0.2530	0.0137\\
0.2685	0.0295\\
0.2667	0.0240\\
0.2461	0.0046\\
0.2173	0.0419\\
0.2255	0.0355\\
0.2413	0.0306\\
0.2248	0.0681\\
0.2189	0.0826\\
0.2287	0.0600\\
0.2306	0.0587\\
0.2150	0.0847\\
0.2345	0.0660\\
0.3205	0.0224\\
0.2999	0.0034\\
0.3058	0.0050\\
0.3461	0.0341\\
0.3514	0.0396\\
0.3832	0.0750\\
0.3764	0.0746\\
0.3283	0.0355\\
0.3085	0.0144\\
0.2992	0.0046\\
0.2730	0.0391\\
0.2585	0.0535\\
0.2758	0.0337\\
0.3511	0.0344\\
0.3426	0.0119\\
0.3343	0.0016\\
0.3385	0.0227\\
0.3189	0.0012\\
0.3207	0.0166\\
0.3426	0.0005\\
0.3649	0.0291\\
0.3416	0.0131\\
0.3269	0.0025\\
0.3293	0.0043\\
0.3011	0.0331\\
0.3049	0.0185\\
0.3060	0.0174\\
0.3211	0.0108\\
0.3059	0.0367\\
0.3408	0.0008\\
0.3744	0.0480\\
0.4315	0.1251\\
0.4092	0.1186\\
0.3762	0.0975\\
0.3747	0.1080\\
0.3447	0.0773\\
0.2959	0.0218\\
0.2965	0.0278\\
0.3069	0.0581\\
0.3958	0.1636\\
0.3467	0.1205\\
0.3046	0.0782\\
0.2936	0.0560\\
0.2939	0.0353\\
0.2844	0.0210\\
0.2942	0.0513\\
0.3721	0.1478\\
0.3663	0.1425\\
0.3697	0.1657\\
0.3382	0.1685\\
0.3028	0.1603\\
0.2754	0.1656\\
0.2290	0.1721\\
0.1615	0.1678\\
0.1112	0.1747\\
0.0714	0.1880\\
0.0489	0.2188\\
0.0833	0.3147\\
-0.0218	0.2873\\
-0.1306	0.2134\\
-0.3852	0.0164\\
-0.4360	0.0205\\
-0.3741	0.1099\\
-0.3700	0.1256\\
-0.5123	0.0061\\
-0.6319	0.1068\\
-0.6730	0.1638\\
-0.6436	0.1403\\
-0.5630	0.0487\\
-0.5302	0.0143\\
-0.5134	0.0057\\
-0.5093	0.0152\\
-0.5264	0.0480\\
-0.5411	0.0701\\
-0.5376	0.0694\\
-0.5378	0.0655\\
-0.5318	0.0475\\
-0.5086	0.0104\\
-0.4895	0.0076\\
-0.4849	0.0419\\
-0.4866	0.0504\\
-0.4763	0.0281\\
-0.4643	0.0206\\
-0.4765	0.0343\\
-0.4947	0.0337\\
-0.4857	0.0249\\
-0.4729	0.0470\\
-0.4591	0.0486\\
-0.4568	0.0438\\
-0.4527	0.0494\\
-0.4348	0.0524\\
-0.4140	0.0421\\
-0.4322	0.0589\\
-0.4257	0.0433\\
-0.4834	0.0925\\
-0.5461	0.1625\\
-0.5310	0.1686\\
-0.3977	0.0557\\
-0.3686	0.0336\\
-0.3477	0.0210\\
-0.3567	0.0404\\
-0.3458	0.0318\\
-0.3554	0.0394\\
-0.4461	0.1307\\
-0.4137	0.1055\\
-0.4045	0.1063\\
-0.3492	0.0528\\
-0.3416	0.0364\\
-0.3304	0.0178\\
-0.3533	0.0432\\
-0.3218	0.0232\\
-0.3183	0.0293\\
-0.2387	0.0533\\
-0.1931	0.1102\\
-0.2119	0.0979\\
-0.3022	0.0073\\
-0.3000	0.0118\\
-0.2901	0.0232\\
-0.3517	0.0373\\
-0.3889	0.0776\\
-0.3798	0.0710\\
-0.3567	0.0516\\
-0.3308	0.0252\\
-0.2832	0.0201\\
-0.2364	0.0603\\
-0.1851	0.0986\\
-0.3380	0.0642\\
-0.4949	0.2204\\
-0.4468	0.1700\\
-0.4183	0.1468\\
-0.3870	0.1282\\
-0.3682	0.1243\\
-0.3459	0.1021\\
-0.3465	0.0875\\
-0.3121	0.0454\\
-0.2497	0.0058\\
-0.2380	0.0106\\
-0.2048	0.0462\\
-0.2741	0.0237\\
-0.3030	0.0588\\
-0.2843	0.0500\\
-0.2459	0.0221\\
-0.3402	0.1182\\
-0.4065	0.1940\\
-0.4104	0.2288\\
-0.3566	0.2090\\
-0.2808	0.1531\\
-0.2021	0.1018\\
-0.1409	0.0877\\
-0.1101	0.1026\\
-0.0739	0.1054\\
-0.0472	0.1154\\
-0.0376	0.1463\\
-0.0101	0.1652\\
0.0448	0.1534\\
0.1759	0.0557\\
0.2583	0.0039\\
0.2026	0.0960\\
0.2045	0.1354\\
0.2669	0.1097\\
0.2970	0.1035\\
0.3841	0.0337\\
0.3948	0.0401\\
0.4002	0.0554\\
0.4461	0.0261\\
0.4735	0.0069\\
0.4501	0.0292\\
0.4526	0.0342\\
0.4539	0.0494\\
0.4657	0.0619\\
0.5215	0.0193\\
0.4793	0.0547\\
0.5140	0.0074\\
0.5214	0.0090\\
0.4643	0.0454\\
0.4663	0.0364\\
0.4927	0.0106\\
0.5304	0.0228\\
0.5952	0.0888\\
0.5540	0.0641\\
0.5335	0.0492\\
0.5878	0.0977\\
0.5388	0.0253\\
0.4574	0.0685\\
0.5265	0.0142\\
0.6110	0.1275\\
0.6062	0.1515\\
0.5517	0.0965\\
0.5024	0.0326\\
0.4884	0.0225\\
0.4744	0.0526\\
0.4390	0.0500\\
0.4401	0.0649\\
0.4635	0.0857\\
0.4587	0.0683\\
0.4285	0.0245\\
0.4875	0.0742\\
0.5051	0.0789\\
0.4196	0.0261\\
0.4089	0.0586\\
0.4682	0.0225\\
0.5418	0.0363\\
0.5338	0.0232\\
0.5437	0.0402\\
0.5234	0.0272\\
0.5398	0.0496\\
0.5428	0.0501\\
0.5423	0.0413\\
0.5319	0.0182\\
0.5436	0.0235\\
0.5595	0.0422\\
0.5463	0.0345\\
0.5088	0.0033\\
0.5181	0.0113\\
0.5516	0.0423\\
0.5459	0.0469\\
0.5404	0.0616\\
0.5285	0.0647\\
0.4984	0.0367\\
0.4414	0.0204\\
0.4077	0.0509\\
0.4032	0.0418\\
0.4029	0.0333\\
0.4526	0.0176\\
0.4902	0.0616\\
0.5730	0.1892\\
0.5258	0.2022\\
0.4556	0.1785\\
0.4437	0.2119\\
0.3582	0.1816\\
0.2949	0.1860\\
0.2289	0.1995\\
0.1449	0.1999\\
-0.2207	0.0778\\
-0.2950	0.0584\\
-0.3106	0.0170\\
-0.3022	0.1070\\
-0.3127	0.1598\\
-0.4263	0.1015\\
-0.5257	0.0610\\
-0.6190	0.0422\\
-0.6566	0.0734\\
-0.8018	0.0237\\
-0.8645	0.0668\\
-0.8623	0.0570\\
-0.9047	0.1068\\
-0.9127	0.1235\\
-0.9484	0.1644\\
-0.9238	0.1321\\
-0.9254	0.1199\\
-0.9301	0.1011\\
-0.9299	0.1023\\
-0.8889	0.0775\\
-0.7822	0.0110\\
-0.8136	0.0270\\
-0.8253	0.0522\\
-0.8156	0.0518\\
-0.8107	0.0676\\
-0.8126	0.0856\\
-0.8493	0.1312\\
-0.7926	0.0662\\
-0.7684	0.0411\\
-0.7471	0.0199\\
-0.7587	0.0462\\
-0.7283	0.0260\\
-0.7287	0.0134\\
-0.7386	0.0098\\
-0.7415	0.0256\\
-0.7688	0.0826\\
-0.7280	0.0480\\
-0.8416	0.1694\\
-0.8131	0.1590\\
-0.7400	0.1143\\
-0.6710	0.0622\\
-0.6675	0.0649\\
-0.6411	0.0361\\
-0.6120	0.0140\\
-0.6012	0.0192\\
-0.5990	0.0334\\
-0.5627	0.0154\\
-0.5963	0.0804\\
-0.5898	0.1109\\
-0.6116	0.1540\\
-0.5902	0.1500\\
-0.5379	0.1128\\
-0.5107	0.1017\\
-0.5200	0.1232\\
-0.5733	0.1862\\
-0.4768	0.0984\\
-0.4221	0.0473\\
-0.3317	0.0458\\
-0.2743	0.0991\\
-0.2682	0.0956\\
-0.2739	0.0879\\
-0.3233	0.0445\\
-0.3981	0.0369\\
-0.5869	0.2391\\
-0.6920	0.3635\\
-0.6306	0.3260\\
-0.5718	0.3045\\
-0.4640	0.2436\\
-0.3556	0.1816\\
-0.2562	0.1284\\
-0.2164	0.1410\\
-0.1745	0.1545\\
-0.1559	0.1832\\
-0.1386	0.2070\\
-0.1193	0.2403\\
-0.0048	0.1913\\
0.0525	0.1956\\
0.1275	0.1727\\
0.1418	0.2051\\
0.2245	0.1590\\
0.2990	0.1218\\
0.3862	0.0741\\
0.3786	0.1408\\
0.3966	0.1999\\
0.3667	0.2475\\
0.5489	0.0761\\
0.6061	0.0354\\
0.6442	0.0506\\
0.7717	0.0240\\
0.8670	0.1143\\
0.8243	0.0793\\
0.7793	0.0256\\
0.8213	0.0582\\
0.8398	0.0969\\
0.7749	0.0402\\
0.7602	0.0230\\
0.7887	0.0391\\
0.7473	0.0123\\
0.7780	0.0075\\
0.7316	0.0423\\
0.7613	0.0291\\
0.8225	0.0097\\
0.7979	0.0454\\
0.8319	0.0308\\
0.9000	0.0090\\
0.8505	0.0600\\
0.8012	0.1038\\
0.9394	0.0388\\
0.9675	0.0656\\
0.8955	0.0431\\
0.9250	0.0586\\
0.9231	0.0923\\
0.9901	0.0105\\
0.9778	0.0141\\
0.9743	0.0223\\
0.9885	0.0661\\
1.0774	0.0240\\
1.1656	0.0800\\
1.2024	0.1462\\
1.2597	0.1978\\
1.1595	0.0828\\
1.0816	0.0381\\
1.0178	0.0272\\
1.0241	0.0967\\
1.0308	0.1463\\
0.9540	0.1094\\
0.8720	0.0730\\
0.8275	0.1149\\
0.8002	0.1823\\
0.7625	0.2573\\
0.6776	0.3084\\
0.6684	0.4516\\
0.6099	0.5422\\
0.3297	0.4245\\
0.0386	0.3115\\
-0.2779	0.1600\\
-0.6651	0.0955\\
-0.8714	0.1876\\
-0.9935	0.2061\\
-1.0767	0.1957\\
-1.1251	0.1372\\
-1.1594	0.1026\\
-1.3026	0.1937\\
-1.2389	0.0940\\
-1.2398	0.0567\\
-1.4006	0.2121\\
-1.4681	0.2737\\
-1.3920	0.1689\\
-1.4277	0.1813\\
-1.4594	0.2298\\
-1.4961	0.3053\\
-1.5198	0.3365\\
-1.4465	0.2905\\
-1.3070	0.1975\\
-1.1958	0.1265\\
-1.1094	0.0558\\
-1.0361	0.0233\\
-1.0011	0.0395\\
-1.0416	0.1176\\
-1.0239	0.1109\\
-0.9586	0.0752\\
-0.8815	0.0348\\
-0.9305	0.0950\\
-0.9246	0.0956\\
-0.9258	0.1305\\
-0.9001	0.1532\\
-0.9037	0.1731\\
-0.8621	0.1451\\
-0.7985	0.1041\\
-0.8061	0.1507\\
-0.8246	0.2076\\
-0.7390	0.1553\\
-0.7072	0.1428\\
-0.6293	0.1004\\
-0.5873	0.1050\\
-0.5028	0.0626\\
-0.4544	0.0533\\
-0.3541	0.0028\\
-0.3521	0.0526\\
-0.3724	0.1279\\
-0.3864	0.1996\\
-0.3443	0.2163\\
-0.2584	0.1885\\
-0.2301	0.2159\\
-0.2009	0.2470\\
-0.1634	0.2773\\
-0.1340	0.3181\\
-0.0885	0.3355\\
0.0036	0.2978\\
0.3144	0.0357\\
0.5080	0.1029\\
0.5067	0.0402\\
0.4970	0.0299\\
0.4750	0.1069\\
0.5914	0.0547\\
0.7277	0.0203\\
0.7504	0.0033\\
0.6967	0.0729\\
0.7685	0.0083\\
0.7968	0.0037\\
0.8312	0.0080\\
0.8841	0.0199\\
0.8785	0.0081\\
0.9739	0.0509\\
0.9680	0.0121\\
0.9066	0.0829\\
0.8886	0.1083\\
0.8823	0.1400\\
0.9338	0.1145\\
0.9156	0.1593\\
0.8803	0.1988\\
0.9777	0.1242\\
1.0312	0.0957\\
1.1013	0.0646\\
1.1619	0.0322\\
1.1395	0.0572\\
1.2376	0.0410\\
1.2130	0.0244\\
1.2071	0.0161\\
1.1889	0.0106\\
1.1598	0.0153\\
1.1979	0.0307\\
1.1953	0.0187\\
1.2580	0.0838\\
1.2814	0.1208\\
1.3701	0.2609\\
1.1492	0.0750\\
1.0699	0.0394\\
1.0107	0.0364\\
0.9075	0.0267\\
0.8620	0.0951\\
0.8333	0.2073\\
0.7142	0.2513\\
0.6580	0.3855\\
0.6307	0.5390\\
0.5220	0.5973\\
0.4147	0.6468\\
-0.0020	0.3842\\
-0.2833	0.2678\\
-0.8367	0.1172\\
-0.9744	0.0901\\
-1.0711	0.0770\\
-1.0559	0.0014\\
-1.0980	0.0258\\
-1.2004	0.0987\\
-1.2366	0.1137\\
-1.2298	0.0905\\
-1.2468	0.0908\\
-1.1397	0.0647\\
-1.0734	0.1368\\
-1.0540	0.1224\\
-1.2858	0.1425\\
-1.2429	0.0957\\
-1.1704	0.0086\\
-1.1319	0.0346\\
-1.2673	0.1147\\
-1.2372	0.1117\\
-1.2309	0.1024\\
-1.1777	0.0642\\
-1.0076	0.0742\\
-0.9677	0.0596\\
-1.0202	0.0408\\
-1.0018	0.0636\\
-1.0283	0.1094\\
-0.9738	0.0867\\
-0.8688	0.0253\\
-0.7561	0.0455\\
-0.7510	0.0046\\
-0.6681	0.0456\\
-0.6659	0.0053\\
-0.7263	0.1003\\
-0.6947	0.1183\\
-0.6445	0.1148\\
-0.6000	0.1128\\
-0.5400	0.0900\\
-0.5220	0.1088\\
-0.5064	0.1308\\
-0.8498	0.0071\\
-0.8188	0.0164\\
-0.8102	0.0156\\
-0.8146	0.0057\\
-0.8125	0.0001\\
-0.8062	0.0117\\
-0.7791	0.0532\\
-0.7536	0.0847\\
-0.7449	0.0638\\
-0.7338	0.0281\\
-0.7260	0.0140\\
-0.7131	0.0165\\
-0.7055	0.0208\\
-0.7081	0.0184\\
-0.7098	0.0127\\
-0.7062	0.0051\\
-0.6880	0.0175\\
-0.6673	0.0269\\
-0.6597	0.0319\\
-0.6651	0.0187\\
-0.6696	0.0006\\
-0.6601	0.0072\\
-0.6527	0.0221\\
-0.6607	0.0139\\
-0.6734	0.0042\\
-0.6730	0.0033\\
-0.6696	0.0069\\
-0.6871	0.0070\\
-0.6926	0.0071\\
-0.6841	0.0200\\
-0.6899	0.0107\\
-0.6980	0.0044\\
-0.7036	0.0109\\
-0.6949	0.0043\\
-0.6980	0.0022\\
-0.7064	0.0220\\
-0.7023	0.0165\\
-0.6942	0.0047\\
-0.6999	0.0158\\
-0.7066	0.0234\\
-0.7018	0.0171\\
-0.6935	0.0011\\
-0.7053	0.0199\\
-0.7055	0.0042\\
-0.6930	0.0076\\
-0.6953	0.0044\\
-0.7007	0.0032\\
-0.7015	0.0104\\
-0.6854	0.0104\\
-0.6700	0.0198\\
-0.6718	0.0164\\
-0.6849	0.0215\\
-0.6931	0.0256\\
-0.6857	0.0164\\
-0.6792	0.0040\\
-0.6784	0.0065\\
-0.6836	0.0012\\
-0.6766	0.0125\\
-0.6571	0.0056\\
-0.6496	0.0043\\
-0.6536	0.0036\\
-0.6674	0.0249\\
-0.6670	0.0301\\
-0.6556	0.0212\\
-0.6374	0.0264\\
-0.6350	0.0191\\
-0.6233	0.0308\\
-0.6009	0.0335\\
-0.5878	0.0387\\
-0.5844	0.0380\\
-0.5715	0.0045\\
-0.5477	0.0002\\
-0.5248	0.0113\\
-0.5131	0.0014\\
-0.5097	0.0108\\
-0.5078	0.0126\\
-0.4996	0.0128\\
-0.4786	0.0430\\
-0.4676	0.0350\\
-0.4509	0.0747\\
-0.4443	0.0794\\
-0.4334	0.0411\\
-0.4141	0.0335\\
-0.3957	0.0358\\
-0.3851	0.0501\\
-0.3806	0.0422\\
-0.3784	0.0580\\
-0.3686	0.0644\\
-0.3524	0.0604\\
-0.3432	0.0417\\
-0.3371	0.0442\\
-0.3249	0.0858\\
-0.3125	0.0955\\
-0.3106	0.0166\\
-0.3063	0.0256\\
-0.2954	0.0250\\
-0.2828	0.0519\\
-0.2773	0.0305\\
-0.2703	0.0087\\
-0.2585	0.0337\\
-0.2507	0.0916\\
-0.2481	0.1590\\
-0.2418	0.2581\\
-0.2328	0.2671\\
-0.2275	0.2649\\
-0.2236	0.2628\\
-0.2161	0.2416\\
-0.2025	0.1560\\
-0.1952	0.1591\\
-0.1933	0.1673\\
-0.1873	0.1475\\
-0.1730	0.1278\\
-0.1647	0.1324\\
-0.1500	0.1171\\
-0.1458	0.1183\\
-0.1425	0.0647\\
-0.1404	0.0490\\
-0.1334	0.0765\\
-0.1305	0.0697\\
-0.1297	0.0634\\
-0.1277	0.0570\\
-0.1201	0.0445\\
-0.1124	0.0236\\
-0.1092	0.0302\\
-0.1053	0.0388\\
-0.1036	0.0374\\
-0.0970	0.0227\\
-0.0515	0.0455\\
-0.0481	0.0351\\
-0.0359	0.0104\\
-0.0289	0.0121\\
-0.0278	0.0103\\
-0.0204	0.0251\\
-0.0196	0.0193\\
-0.0195	0.0071\\
-0.0194	0.0070\\
-0.0179	0.0073\\
-0.0175	0.0115\\
-0.0165	0.1255\\
-0.0168	0.1274\\
-0.0171	0.1233\\
-0.0169	0.0905\\
-0.0169	0.0284\\
-0.0172	0.0016\\
-0.0175	0.0430\\
-0.0174	0.0722\\
-0.0180	0.1096\\
-0.0188	0.0702\\
-0.0190	0.0332\\
-0.0177	0.0328\\
-0.0156	0.0221\\
-0.0153	0.0130\\
-0.0160	0.0123\\
-0.0170	0.0288\\
-0.0180	0.0116\\
-0.0205	0.0227\\
-0.0196	0.0223\\
-0.0195	0.0209\\
-0.0196	0.0178\\
-0.0184	0.0187\\
-0.0175	0.0196\\
-0.0154	0.0335\\
-0.0142	0.0425\\
-0.0126	0.0759\\
-0.0137	0.0871\\
-0.0134	0.0201\\
-0.0134	0.0179\\
-0.0127	0.0082\\
-0.0123	0.0118\\
-0.0121	0.0104\\
-0.0120	0.0108\\
-0.0118	0.0150\\
-0.0101	0.0189\\
-0.0089	0.0297\\
-0.0100	0.0333\\
-0.0068	0.0363\\
-0.0088	0.0308\\
-0.0114	0.0193\\
-0.0117	0.0169\\
-0.0106	0.0039\\
-0.0089	0.0056\\
-0.0072	0.1339\\
-0.0062	0.1115\\
-0.0059	0.0972\\
-0.0058	0.0921\\
-0.0058	0.0676\\
-0.0062	0.0410\\
-0.0086	0.0094\\
-0.0081	0.0116\\
0.0018	0.0452\\
0.0019	0.0100\\
0.0023	0.0207\\
0.0026	0.0087\\
0.0026	0.0149\\
0.0025	0.0217\\
0.0020	0.0280\\
0.0043	0.0247\\
0.0038	0.0250\\
0.0036	0.0230\\
0.0034	0.0227\\
0.0034	0.0171\\
0.0038	0.0244\\
0.0049	0.0146\\
0.0046	0.0015\\
0.0040	0.0064\\
0.0046	0.0018\\
0.0041	0.0090\\
0.0041	0.0112\\
0.0068	0.0097\\
0.0055	0.0117\\
0.0053	0.0102\\
0.0053	0.0099\\
0.0056	0.0071\\
0.0058	0.0069\\
0.0053	0.0019\\
0.0059	0.0045\\
0.0058	0.0068\\
0.0057	0.0065\\
0.0057	0.0028\\
0.0054	0.0034\\
0.0053	0.0060\\
0.0066	0.0077\\
0.0070	0.0081\\
0.0068	0.0004\\
0.0066	0.0085\\
0.0067	0.0149\\
0.0071	0.0145\\
0.0073	0.0103\\
0.0076	0.0090\\
0.0084	0.0138\\
0.0090	0.0081\\
0.0089	0.0104\\
0.0085	0.0102\\
0.0085	0.0015\\
0.0092	0.0022\\
0.0095	0.0090\\
0.0094	0.0107\\
0.0091	0.0163\\
0.0085	0.0135\\
0.0084	0.0151\\
0.0084	0.0291\\
0.0086	0.0388\\
0.0084	0.0377\\
0.0080	0.0341\\
0.0075	0.0212\\
0.0083	0.0090\\
0.0084	0.0122\\
0.0084	0.0110\\
0.0082	0.0065\\
0.0082	0.0071\\
0.0085	0.0080\\
0.0086	0.0192\\
0.0085	0.0139\\
0.0085	0.0167\\
0.0090	0.0092\\
0.0089	0.0110\\
0.0088	0.0109\\
0.0098	0.0129\\
0.0102	0.0193\\
0.0103	0.0180\\
0.0105	0.0175\\
0.0110	0.0065\\
0.0111	0.0006\\
0.0111	0.0012\\
0.0111	0.0040\\
0.0112	0.0053\\
0.0118	0.0107\\
0.0126	0.0112\\
0.0132	0.0112\\
0.0136	0.0142\\
0.0136	0.0223\\
0.0133	0.0056\\
0.0130	0.0025\\
0.0131	0.0112\\
0.0133	0.0178\\
0.0136	0.0110\\
0.0143	0.0139\\
0.0151	0.0181\\
0.0154	0.0117\\
0.0149	0.0113\\
0.0146	0.0090\\
0.0141	0.0025\\
0.0140	0.0036\\
0.0139	0.0159\\
0.0130	0.0131\\
0.0129	0.0127\\
0.0101	0.0074\\
0.0102	0.0017\\
0.0108	0.0021\\
0.0111	0.0081\\
0.0098	0.0138\\
0.0095	0.0149\\
0.0104	0.0178\\
0.0109	0.0150\\
0.0112	0.0203\\
0.0112	0.0254\\
0.0098	0.0086\\
0.0094	0.0016\\
0.0109	0.0150\\
0.0118	0.0170\\
0.0118	0.0184\\
0.0115	0.0129\\
0.0070	0.0659\\
0.0074	0.0450\\
0.0053	0.0348\\
0.0048	0.0323\\
0.0042	0.0395\\
0.0026	0.0246\\
0.0029	0.0165\\
0.0032	0.0166\\
0.0036	0.0134\\
0.0040	0.0132\\
0.0041	0.0071\\
0.0038	0.0041\\
-0.0069	0.0094\\
-0.0076	0.0127\\
-0.0082	0.0145\\
-0.0084	0.0145\\
-0.0063	0.0028\\
-0.0061	0.0045\\
-0.0063	0.0088\\
-0.0057	0.0104\\
-0.0050	0.0113\\
0.0036	0.0175\\
0.0038	0.0182\\
0.0039	0.0160\\
0.0040	0.0147\\
0.0040	0.0094\\
0.0039	0.0059\\
0.0038	0.0033\\
0.0023	0.0048\\
0.0026	0.0013\\
0.0025	0.0004\\
0.0024	0.0048\\
0.0022	0.0037\\
-0.0232	0.0038\\
-0.0230	0.0001\\
-0.0226	0.0139\\
-0.0224	0.0196\\
-0.0218	0.0213\\
0.1450	0.1801\\
0.1484	0.1227\\
0.1555	0.0924\\
0.1601	0.0665\\
0.1646	0.0434\\
0.1688	0.0561\\
0.1733	0.0496\\
0.1777	0.0646\\
0.1802	0.0413\\
0.1846	0.0161\\
0.1889	0.0074\\
0.1924	0.0604\\
0.1987	0.0561\\
0.2068	0.0190\\
0.2115	0.0081\\
0.2098	0.0078\\
0.2096	0.0022\\
0.2105	0.0016\\
0.2142	0.0079\\
0.2174	0.0096\\
0.2240	0.0139\\
0.2328	0.0269\\
0.2401	0.0271\\
0.2333	0.0229\\
0.2129	0.0389\\
0.2046	0.0304\\
0.2208	0.0016\\
0.2330	0.0092\\
0.2457	0.0009\\
0.2456	0.0121\\
0.2452	0.0120\\
0.2422	0.0038\\
0.2367	0.0197\\
0.2297	0.0267\\
0.2255	0.0250\\
0.2236	0.0155\\
0.2228	0.0259\\
0.2224	0.0379\\
0.2241	0.0282\\
0.2273	0.0082\\
0.2277	0.0169\\
0.2226	0.0282\\
0.2182	0.0645\\
0.2156	0.0680\\
0.2140	0.0361\\
0.2124	0.0096\\
0.2116	0.0101\\
0.2107	0.0214\\
0.2103	0.0216\\
0.2089	0.0119\\
0.2079	0.0327\\
0.2073	0.0390\\
0.2073	0.0339\\
0.2069	0.0314\\
0.2066	0.0279\\
0.2054	0.0224\\
0.2054	0.0161\\
0.2060	0.0241\\
0.2069	0.0103\\
0.2056	0.0180\\
0.2054	0.0046\\
0.2086	0.0266\\
0.2117	0.0341\\
0.2101	0.0167\\
0.2061	0.0087\\
0.2036	0.0095\\
0.2036	0.0020\\
0.2040	0.0044\\
0.2049	0.0024\\
0.2047	0.0105\\
0.2044	0.0220\\
0.2046	0.0097\\
0.2053	0.0035\\
0.2094	0.0029\\
0.2118	0.0250\\
0.2091	0.0004\\
0.2049	0.0228\\
0.2028	0.0240\\
0.2031	0.0070\\
0.2029	0.0068\\
0.2027	0.0016\\
0.2026	0.0126\\
0.2020	0.0251\\
0.2016	0.0011\\
0.2016	0.0145\\
0.2065	0.0108\\
0.2104	0.0083\\
0.2076	0.0111\\
0.2020	0.0065\\
0.1990	0.0235\\
0.1985	0.0176\\
0.1978	0.0289\\
0.1968	0.0217\\
0.1963	0.0212\\
0.1956	0.0061\\
0.1953	0.0097\\
0.1950	0.0073\\
0.1954	0.0013\\
0.1959	0.0152\\
0.1987	0.0241\\
0.2012	0.0260\\
0.2010	0.0355\\
0.1977	0.0256\\
0.1945	0.0223\\
0.1932	0.0457\\
0.1923	0.0399\\
0.1916	0.0332\\
0.1915	0.0322\\
0.1915	0.0245\\
0.1913	0.0254\\
0.1903	0.0218\\
0.1906	0.0184\\
0.1905	0.0315\\
0.1913	0.0555\\
0.1907	0.0711\\
0.1909	0.0827\\
0.1908	0.0948\\
0.1934	0.1332\\
0.1974	0.1040\\
0.1973	0.0710\\
0.1914	0.0125\\
0.1871	0.0322\\
0.1854	0.0064\\
0.1850	0.0042\\
0.1820	0.0006\\
0.1806	0.0112\\
0.1794	0.0060\\
0.1790	0.0020\\
0.1795	0.0403\\
0.1827	0.0414\\
0.1851	0.0541\\
0.1833	0.0546\\
0.1812	0.0699\\
0.1790	0.0860\\
0.1789	0.0906\\
0.1849	0.0889\\
0.1897	0.0831\\
0.1910	0.0723\\
0.1869	0.0651\\
0.1844	0.0609\\
0.1840	0.0647\\
0.1878	0.0513\\
0.1933	0.0840\\
0.1954	0.0683\\
0.1913	0.0479\\
0.1889	0.0470\\
0.1900	0.0378\\
0.1947	0.0219\\
0.1987	0.0303\\
0.1993	0.0176\\
0.1973	0.0129\\
0.1964	0.0291\\
0.1967	0.0230\\
0.1980	0.0171\\
0.2000	0.0023\\
0.2053	0.0072\\
0.2094	0.0005\\
0.2079	0.0000\\
0.2056	0.0016\\
0.2038	0.0002\\
0.2045	0.0013\\
0.2062	0.0013\\
0.2069	0.0125\\
0.2076	0.0229\\
0.2076	0.0148\\
0.2085	0.0063\\
0.2113	0.0007\\
0.2188	0.0066\\
0.2252	0.0165\\
0.2261	0.0223\\
0.2234	0.0141\\
0.2222	0.0369\\
0.2238	0.0539\\
0.2254	0.0342\\
0.2260	0.0257\\
0.2284	0.0243\\
0.2333	0.0162\\
0.2421	0.0113\\
0.2481	0.0227\\
0.2485	0.0190\\
0.2445	0.0059\\
0.2431	0.0103\\
0.2455	0.0128\\
0.2486	0.0214\\
0.2524	0.0318\\
0.2614	0.0195\\
0.2694	0.0099\\
0.2693	0.0128\\
0.2667	0.0220\\
0.2653	0.0334\\
0.2673	0.0363\\
0.2714	0.0271\\
0.2747	0.0270\\
0.2786	0.0144\\
0.2829	0.0136\\
0.2938	0.0312\\
0.3041	0.0380\\
0.3058	0.0137\\
0.3040	0.0037\\
0.3047	0.0201\\
0.3092	0.0262\\
0.3198	0.0186\\
0.3284	0.0091\\
0.3290	0.0173\\
0.3314	0.0417\\
0.3400	0.0486\\
0.3473	0.0404\\
0.3466	0.0282\\
0.3447	0.0220\\
0.3463	0.0168\\
0.3524	0.0059\\
0.3575	0.0069\\
0.3637	0.0056\\
0.3688	0.0081\\
0.3796	0.0083\\
0.3931	0.0110\\
0.3991	0.0162\\
0.3983	0.0099\\
0.3938	0.0077\\
0.3934	0.0024\\
0.3981	0.0075\\
0.4021	0.0111\\
0.4077	0.0093\\
0.4130	0.0062\\
0.4254	0.0031\\
0.4407	0.0097\\
0.4501	0.0144\\
0.4606	0.0267\\
0.4594	0.0069\\
0.4515	0.0083\\
0.4511	0.0211\\
0.4552	0.0216\\
0.4620	0.0180\\
0.4664	0.0028\\
0.4723	0.0067\\
0.4751	0.0186\\
0.4814	0.0357\\
0.4867	0.0163\\
0.5012	0.0024\\
0.5179	0.0138\\
0.5257	0.0145\\
0.5167	0.0135\\
0.5138	0.0071\\
0.5149	0.0008\\
0.5183	0.0040\\
0.5177	0.0089\\
0.5195	0.0117\\
0.5207	0.0043\\
0.5245	0.0069\\
0.5243	0.0103\\
0.5324	0.0176\\
0.5451	0.0219\\
0.5502	0.0236\\
0.5414	0.0072\\
0.5451	0.0068\\
0.5482	0.0313\\
-0.5064	0.0085\\
-0.5058	0.0139\\
-0.5130	0.0190\\
-0.5263	0.0034\\
-0.5274	0.0047\\
-0.5158	0.0069\\
-0.5049	0.0298\\
-0.5064	0.0294\\
-0.5167	0.0174\\
-0.5223	0.0073\\
-0.5241	0.0049\\
-0.5129	0.0033\\
-0.5030	0.0077\\
-0.5045	0.0063\\
-0.5149	0.0133\\
-0.5214	0.0142\\
-0.5164	0.0097\\
-0.5036	0.0259\\
-0.5004	0.0142\\
-0.5073	0.0104\\
-0.5145	0.0135\\
-0.5213	0.0043\\
-0.5169	0.0119\\
-0.5038	0.0146\\
-0.4980	0.0107\\
-0.4945	0.0006\\
-0.4928	0.0149\\
-0.4901	0.0195\\
-0.4908	0.0202\\
-0.4964	0.0228\\
-0.5036	0.0317\\
-0.5089	0.0232\\
-0.5022	0.0107\\
-0.4949	0.0262\\
-0.5020	0.0149\\
-0.5027	0.0087\\
-0.4915	0.0213\\
-0.4845	0.0166\\
-0.4795	0.0041\\
-0.4839	0.0130\\
-0.4937	0.0122\\
-0.4923	0.0038\\
-0.4824	0.0321\\
-0.4836	0.0599\\
-0.4884	0.0583\\
-0.4821	0.0502\\
-0.4748	0.0418\\
-0.4639	0.0369\\
-0.4648	0.0051\\
-0.4726	0.0105\\
-0.4820	0.0171\\
-0.4854	0.0168\\
-0.4756	0.0158\\
-0.4687	0.0106\\
-0.4550	0.0159\\
-0.4493	0.0259\\
-0.4466	0.0514\\
-0.4461	0.0549\\
-0.4439	0.0466\\
-0.4424	0.0311\\
-0.4391	0.0380\\
-0.4386	0.0346\\
-0.4365	0.0387\\
-0.4352	0.0346\\
-0.4386	0.0271\\
-0.4444	0.0109\\
-0.4506	0.0033\\
-0.4449	0.0087\\
-0.4312	0.0004\\
-0.4252	0.0180\\
-0.4280	0.0226\\
-0.4365	0.0216\\
-0.4365	0.0142\\
-0.4321	0.0158\\
-0.4202	0.0255\\
-0.4155	0.0292\\
-0.4189	0.0220\\
-0.4285	0.0187\\
-0.4354	0.0062\\
-0.4192	0.0038\\
-0.3802	0.0180\\
-0.3650	0.0284\\
-0.3715	0.0585\\
-0.4079	0.0466\\
-0.4267	0.0347\\
-0.4245	0.0401\\
-0.4195	0.0375\\
-0.4141	0.0594\\
-0.4022	0.0602\\
-0.3880	0.0621\\
-0.3766	0.0533\\
-0.3731	0.0443\\
-0.3731	0.0376\\
-0.3765	0.0256\\
-0.3727	0.0260\\
-0.3607	0.0320\\
-0.3499	0.0281\\
-0.3447	0.0177\\
-0.3395	0.0311\\
-0.3372	0.0345\\
-0.3405	0.0129\\
-0.3424	0.0112\\
-0.3405	0.0050\\
-0.3313	0.0222\\
-0.3281	0.0243\\
-0.3280	0.0306\\
-0.3223	0.0324\\
-0.3115	0.0206\\
-0.3041	0.0377\\
-0.3034	0.0350\\
-0.3067	0.0215\\
-0.3092	0.0262\\
-0.3035	0.0276\\
-0.2966	0.0181\\
-0.2980	0.0043\\
-0.2967	0.0073\\
-0.2871	0.0421\\
-0.2821	0.0512\\
-0.2762	0.0572\\
-0.2728	0.0537\\
-0.2705	0.0043\\
-0.2716	0.0063\\
-0.2732	0.0024\\
-0.2587	0.0473\\
-0.2342	0.0598\\
-0.2373	0.0452\\
-0.2590	0.0453\\
-0.2692	0.0288\\
-0.2675	0.0327\\
-0.2632	0.0393\\
-0.2585	0.0494\\
-0.2528	0.0356\\
-0.2470	0.0235\\
-0.2389	0.0197\\
-0.2302	0.0032\\
-0.2225	0.0023\\
-0.2182	0.0583\\
-0.2179	0.0662\\
-0.2180	0.0517\\
-0.2149	0.0643\\
-0.2065	0.1029\\
-0.1991	0.1908\\
-0.1956	0.1708\\
-0.1956	0.1314\\
-0.1941	0.1054\\
-0.1878	0.1202\\
-0.1832	0.1060\\
-0.1761	0.1272\\
-0.1724	0.1219\\
-0.1693	0.0941\\
-0.1665	0.0764\\
-0.1635	0.0295\\
-0.1616	0.0554\\
-0.1631	0.0522\\
-0.1653	0.0306\\
-0.1657	0.0259\\
-0.1621	0.0463\\
-0.1560	0.0551\\
-0.1518	0.0529\\
-0.1515	0.0450\\
-0.1501	0.0605\\
-0.1449	0.0606\\
-0.1428	0.0881\\
-0.1407	0.1131\\
-0.1394	0.0685\\
-0.1373	0.0644\\
-0.1354	0.0279\\
-0.1341	0.0213\\
-0.1331	0.0171\\
-0.1316	0.0265\\
-0.1308	0.0618\\
-0.1291	0.1008\\
-0.1298	0.0866\\
-0.1303	0.0813\\
-0.1281	0.0479\\
-0.1240	0.0415\\
-0.1216	0.0316\\
-0.1201	0.0551\\
-0.1183	0.0512\\
-0.1181	0.0098\\
-0.1191	0.0414\\
-0.1201	0.0629\\
-0.1183	0.0521\\
-0.1148	0.0637\\
-0.1127	0.0497\\
-0.1107	0.0318\\
-0.1079	0.0191\\
-0.1057	0.0068\\
-0.1054	0.0011\\
-0.1064	0.0054\\
-0.1071	0.0241\\
-0.1067	0.0209\\
-0.1043	0.0067\\
-0.1011	0.0003\\
-0.0993	0.0202\\
-0.1001	0.0281\\
-0.1015	0.0281\\
-0.1019	0.0518\\
-0.1001	0.1147\\
-0.0971	0.1360\\
-0.0938	0.1488\\
-0.0916	0.1208\\
-0.0926	0.0776\\
-0.0885	0.0505\\
-0.0802	0.0486\\
-0.0805	0.0317\\
-0.0832	0.0408\\
-0.0896	0.0251\\
-0.0904	0.0137\\
-0.0897	0.0125\\
-0.0892	0.0131\\
-0.0885	0.0108\\
-0.0859	0.0109\\
-0.0817	0.0018\\
-0.0795	0.0049\\
-0.0770	0.0069\\
-0.0770	0.0204\\
-0.0769	0.0202\\
-0.0769	0.0121\\
-0.0786	0.0072\\
-0.0792	0.0031\\
-0.0766	0.0040\\
-0.0758	0.0004\\
-0.0759	0.0081\\
-0.0728	0.1007\\
-0.0710	0.1636\\
-0.0709	0.1495\\
-0.0711	0.1252\\
-0.0692	0.0733\\
-0.0681	0.0438\\
-0.0667	0.0355\\
-0.0664	0.0262\\
-0.0676	0.0116\\
-0.0690	0.0306\\
-0.0661	0.0115\\
-0.0595	0.0236\\
-0.0563	0.0243\\
-0.0575	0.0219\\
-0.0661	0.0159\\
-0.0660	0.0171\\
-0.0656	0.0428\\
-0.0651	0.0331\\
-0.0636	0.0101\\
-0.0617	0.0206\\
-0.0599	0.0283\\
-0.0590	0.0393\\
-0.0584	0.0235\\
-0.0579	0.0250\\
-0.0568	0.0187\\
-0.0554	0.0126\\
-0.0549	0.0176\\
-0.0555	0.0187\\
-0.0554	0.0319\\
-0.0534	0.0370\\
-0.0460	0.0376\\
-0.0446	0.0363\\
-0.0431	0.0362\\
-0.0428	0.0171\\
-0.0426	0.0106\\
-0.0415	0.0165\\
-0.0392	0.0300\\
-0.0372	0.0501\\
-0.0337	0.0417\\
-0.0319	0.0596\\
-0.0252	0.0126\\
-0.0179	0.0080\\
-0.0147	0.0124\\
-0.0130	0.0141\\
-0.0112	0.0007\\
-0.0085	0.0229\\
-0.0054	0.0512\\
-0.0032	0.0569\\
-0.0012	0.0968\\
0.0005	0.0968\\
0.0033	0.0648\\
0.0239	0.0593\\
0.0264	0.0403\\
0.0278	0.0414\\
0.0283	0.0441\\
0.0305	0.0457\\
0.0330	0.0516\\
0.0344	0.0626\\
0.0357	0.0479\\
0.0383	0.0149\\
0.0411	0.0202\\
0.0419	0.0500\\
0.0513	0.0322\\
0.0537	0.0086\\
0.0551	0.0143\\
0.0576	0.0320\\
0.0589	0.0409\\
0.0606	0.0372\\
0.0634	0.0232\\
0.0653	0.0099\\
0.0651	0.0013\\
0.0656	0.0039\\
0.0666	0.0025\\
0.0701	0.0169\\
0.0734	0.0304\\
0.0745	0.0198\\
0.0754	0.0014\\
0.0779	0.0198\\
0.0797	0.0283\\
0.0796	0.0223\\
0.0803	0.0251\\
0.0909	0.0025\\
0.0944	0.0363\\
0.0998	0.0224\\
0.1055	0.2086\\
0.1074	0.0759\\
0.1103	0.0455\\
0.1118	0.0434\\
0.1163	0.0213\\
};
\addplot [color=white!60!black,mark size=0.5pt,only marks,mark=*,mark options={solid},forget plot]
  table[row sep=crcr]{%
0.1163	0.0213\\
0.1203	0.0070\\
0.1205	0.0032\\
0.1199	0.0014\\
0.1217	0.0106\\
0.1248	0.0212\\
0.1256	0.0040\\
0.1265	0.0112\\
0.1297	0.0109\\
0.1321	0.0082\\
0.1319	0.0002\\
0.1318	0.0114\\
0.1336	0.0241\\
0.1356	0.0280\\
0.1441	0.0091\\
0.1737	0.0409\\
0.1817	0.0540\\
0.1848	0.0514\\
0.1641	0.0112\\
0.1470	0.0876\\
0.1447	0.1104\\
0.1446	0.1044\\
0.1462	0.1010\\
0.1474	0.0404\\
0.1486	0.0279\\
0.1500	0.0229\\
0.1536	0.0066\\
0.1562	0.0086\\
0.1602	0.0042\\
0.1602	0.0095\\
0.1584	0.0042\\
0.1582	0.0149\\
0.1596	0.0192\\
0.1612	0.0307\\
0.1637	0.0431\\
0.1663	0.0332\\
0.1672	0.0123\\
0.1669	0.0077\\
0.1688	0.0114\\
0.1738	0.0203\\
0.1757	0.0218\\
0.1718	0.0420\\
0.1731	0.0653\\
0.1749	0.0629\\
0.1779	0.0346\\
0.1760	0.0348\\
0.1767	0.0682\\
0.1816	0.0579\\
0.1866	0.0617\\
0.1865	0.0708\\
0.1843	0.0356\\
0.1820	0.0476\\
0.1844	0.0189\\
0.1866	0.0007\\
0.1837	0.0158\\
0.1820	0.0089\\
0.1835	0.0037\\
0.1894	0.0067\\
0.1920	0.0018\\
0.1943	0.0032\\
0.1915	0.0248\\
0.1952	0.0256\\
0.2005	0.0299\\
0.1987	0.0577\\
0.1971	0.0657\\
0.2016	0.0188\\
0.2042	0.0036\\
0.2022	0.0003\\
0.1998	0.0024\\
0.2043	0.0046\\
0.2113	0.0045\\
0.2121	0.0027\\
0.2082	0.0064\\
0.2057	0.0097\\
0.2037	0.0080\\
0.2053	0.0010\\
0.2111	0.0227\\
0.2157	0.0175\\
0.2154	0.0136\\
0.2130	0.0209\\
0.2126	0.0416\\
0.2144	0.0371\\
0.2152	0.0274\\
0.2180	0.0313\\
0.2256	0.0290\\
0.2311	0.0069\\
0.2288	0.0066\\
0.2264	0.0158\\
0.2301	0.0068\\
0.2332	0.0034\\
0.2319	0.0021\\
0.2286	0.0022\\
0.2285	0.0186\\
0.2335	0.0242\\
0.2395	0.0134\\
0.2416	0.0262\\
0.2375	0.0252\\
0.2357	0.0125\\
0.2377	0.0045\\
0.2409	0.0061\\
0.2450	0.0191\\
0.2355	0.0273\\
0.2164	0.0501\\
0.2240	0.0364\\
0.2483	0.0115\\
0.2625	0.0100\\
0.2631	0.0217\\
0.2649	0.0072\\
0.2659	0.0003\\
0.2641	0.0041\\
0.2615	0.0105\\
0.2605	0.0132\\
0.2643	0.0054\\
0.2668	0.0250\\
0.2629	0.0102\\
0.2657	0.0001\\
0.2705	0.0302\\
0.2689	0.0165\\
0.2671	0.0205\\
0.2701	0.0422\\
0.2753	0.0778\\
0.2735	0.0663\\
0.2686	0.0429\\
0.2667	0.0302\\
0.2722	0.0166\\
0.2801	0.0125\\
0.2835	0.0065\\
0.2822	0.0234\\
0.2814	0.0230\\
0.2866	0.0245\\
0.2906	0.0070\\
0.2873	0.0056\\
0.2900	0.0024\\
0.2920	0.0088\\
0.2949	0.0054\\
0.2920	0.0095\\
0.2966	0.0071\\
0.3031	0.0117\\
0.3011	0.0275\\
0.2960	0.0569\\
0.2945	0.0563\\
0.2966	0.0416\\
0.2970	0.0316\\
0.2979	0.0369\\
0.2998	0.0395\\
0.3031	0.0522\\
0.3115	0.0348\\
0.3183	0.0102\\
0.3166	0.0150\\
0.3120	0.0127\\
0.3101	0.0073\\
0.3123	0.0283\\
0.3130	0.0199\\
0.3163	0.0037\\
0.3257	0.0112\\
0.3366	0.0290\\
0.3381	0.0487\\
0.3189	0.0721\\
0.2908	0.0828\\
0.2994	0.0831\\
0.3314	0.1102\\
0.3445	0.1039\\
0.3529	0.0675\\
0.3552	0.0379\\
0.3569	0.0022\\
0.3570	0.0275\\
0.3521	0.0443\\
0.3511	0.0204\\
0.3561	0.0301\\
0.3607	0.0455\\
0.3591	0.0319\\
0.3602	0.0276\\
0.3675	0.0045\\
0.3722	0.0044\\
0.3661	0.0198\\
0.3623	0.0224\\
0.3624	0.0183\\
0.3660	0.0089\\
0.3674	0.0019\\
0.3778	0.0024\\
0.3914	0.0090\\
0.3969	0.0026\\
0.3905	0.0076\\
0.3956	0.0097\\
0.4058	0.0360\\
0.4051	0.0357\\
0.4049	0.0185\\
0.4086	0.0258\\
0.4182	0.0410\\
0.4177	0.0589\\
0.4191	0.0314\\
0.4281	0.0048\\
0.4354	0.0194\\
0.4313	0.0045\\
0.4343	0.0018\\
0.4441	0.0269\\
0.4514	0.0325\\
0.4457	0.0231\\
0.4523	0.0295\\
0.4637	0.0372\\
0.4637	0.0330\\
0.4655	0.0378\\
0.4701	0.0207\\
0.4817	0.0133\\
0.4815	0.0047\\
0.4765	0.0027\\
0.4758	0.0004\\
0.4817	0.0027\\
0.4931	0.0170\\
0.5029	0.0141\\
0.5020	0.0103\\
0.5022	0.0103\\
0.4998	0.0137\\
0.5093	0.0113\\
0.5233	0.0323\\
0.5323	0.0474\\
0.5290	0.0341\\
0.5381	0.0368\\
0.5429	0.0272\\
0.5509	0.0268\\
0.5421	0.0292\\
0.5399	0.0265\\
0.5454	0.0262\\
0.5651	0.0298\\
0.5814	0.0118\\
0.5809	0.0087\\
0.5781	0.0191\\
0.5866	0.0078\\
0.6013	0.0260\\
0.6020	0.0301\\
0.5987	0.0385\\
-0.5607	0.0003\\
-0.5592	0.0133\\
-0.5539	0.0361\\
-0.5506	0.0679\\
-0.5454	0.0699\\
-0.5411	0.0725\\
-0.5408	0.0554\\
-0.5484	0.0258\\
-0.5454	0.0032\\
-0.5314	0.0070\\
-0.5181	0.0128\\
-0.5134	0.0011\\
-0.5065	0.0165\\
-0.5024	0.0267\\
-0.4979	0.0646\\
-0.4973	0.0711\\
-0.4996	0.0322\\
-0.5073	0.0241\\
-0.5042	0.0105\\
-0.4908	0.0206\\
-0.4794	0.0297\\
-0.4762	0.0292\\
-0.4711	0.0504\\
-0.4670	0.0850\\
-0.4617	0.0668\\
-0.4600	0.0386\\
-0.4578	0.0264\\
-0.4561	0.0274\\
-0.4528	0.0341\\
-0.4503	0.0404\\
-0.4457	0.0321\\
-0.4424	0.0317\\
-0.4393	0.0221\\
-0.4391	0.0165\\
-0.4377	0.0262\\
-0.4363	0.0060\\
-0.4324	0.0087\\
-0.4312	0.0008\\
-0.4287	0.0324\\
-0.4275	0.0484\\
-0.4253	0.0682\\
-0.4208	0.0719\\
-0.4205	0.0680\\
-0.4264	0.0455\\
-0.4293	0.0222\\
-0.4221	0.0116\\
-0.4153	0.0042\\
-0.4184	0.0265\\
-0.4229	0.0006\\
-0.4165	0.0284\\
-0.4031	0.0733\\
-0.3989	0.0561\\
-0.3976	0.0476\\
-0.3962	0.0531\\
-0.3981	0.0408\\
-0.4059	0.0078\\
-0.4070	0.0053\\
-0.4027	0.0037\\
-0.3915	0.0035\\
-0.3861	0.0202\\
-0.3880	0.0135\\
-0.3951	0.0127\\
-0.3948	0.0242\\
-0.3871	0.0410\\
-0.3852	0.0222\\
-0.3899	0.0113\\
-0.3901	0.0018\\
-0.3803	0.0285\\
-0.3717	0.0906\\
-0.3692	0.0991\\
-0.3670	0.0895\\
-0.3671	0.0767\\
-0.3729	0.0431\\
-0.3806	0.0237\\
-0.3830	0.0207\\
-0.3681	0.0187\\
-0.3349	0.0502\\
-0.3202	0.0657\\
-0.3413	0.0313\\
-0.3706	0.0018\\
-0.3748	0.0132\\
-0.3738	0.0010\\
-0.3715	0.0209\\
-0.3660	0.0384\\
-0.3621	0.0573\\
-0.3523	0.0605\\
-0.3447	0.0753\\
-0.3421	0.0639\\
-0.3396	0.0495\\
-0.3396	0.0313\\
-0.3448	0.0243\\
-0.3482	0.0093\\
-0.3437	0.0106\\
-0.3341	0.0163\\
-0.3315	0.0179\\
-0.3298	0.0278\\
-0.3288	0.0331\\
-0.3304	0.0264\\
-0.3365	0.0152\\
-0.3400	0.0193\\
-0.3358	0.0160\\
-0.3267	0.0156\\
-0.3230	0.0217\\
-0.3228	0.0220\\
-0.3214	0.0200\\
-0.3208	0.0076\\
-0.3189	0.0042\\
-0.3179	0.0189\\
-0.3164	0.0562\\
-0.3176	0.0402\\
-0.3234	0.0248\\
-0.3270	0.0185\\
-0.3229	0.0039\\
-0.3183	0.0142\\
-0.3112	0.0003\\
-0.3089	0.0103\\
-0.3103	0.0202\\
-0.3119	0.0393\\
-0.3093	0.0496\\
-0.2979	0.0740\\
-0.2913	0.0573\\
-0.2872	0.0698\\
-0.2843	0.0543\\
-0.2828	0.0198\\
-0.2795	0.0137\\
-0.2779	0.0090\\
-0.2813	0.0018\\
-0.2832	0.0130\\
-0.2780	0.0078\\
-0.2716	0.0233\\
-0.2719	0.0282\\
-0.2716	0.0310\\
-0.2662	0.0149\\
-0.2574	0.0228\\
-0.2524	0.0322\\
-0.2525	0.0496\\
-0.2553	0.0462\\
-0.2525	0.0466\\
-0.2432	0.0722\\
-0.2354	0.0687\\
-0.2331	0.0648\\
-0.2334	0.0526\\
-0.2372	0.0193\\
-0.2387	0.0126\\
-0.2273	0.0132\\
-0.2047	0.0087\\
-0.1936	0.0216\\
-0.2040	0.0179\\
-0.2194	0.0102\\
-0.2223	0.0092\\
-0.2215	0.0277\\
-0.2193	0.0024\\
-0.2162	0.0011\\
-0.2097	0.0070\\
-0.2024	0.0015\\
-0.1967	0.0129\\
-0.1946	0.0115\\
-0.1954	0.0150\\
-0.1955	0.0070\\
-0.1942	0.0242\\
-0.1887	0.0195\\
-0.1840	0.0277\\
-0.1818	0.0256\\
-0.1796	0.0238\\
-0.1775	0.0077\\
-0.1755	0.0043\\
-0.1747	0.0018\\
-0.1766	0.0126\\
-0.1772	0.0170\\
-0.1759	0.0171\\
-0.1707	0.0250\\
-0.1697	0.0249\\
-0.1695	0.0191\\
-0.1649	0.0349\\
-0.1609	0.0370\\
-0.1621	0.0211\\
-0.1616	0.0193\\
-0.1590	0.0074\\
-0.1527	0.0086\\
-0.1497	0.0152\\
-0.1506	0.0070\\
-0.1526	0.0012\\
-0.1513	0.0052\\
-0.1458	0.0077\\
-0.1415	0.0103\\
-0.1398	0.0114\\
-0.1389	0.0172\\
-0.1370	0.0150\\
-0.1364	0.0136\\
-0.1375	0.0027\\
-0.1364	0.0002\\
-0.1301	0.0142\\
-0.1156	0.0826\\
-0.1167	0.0713\\
-0.1268	0.0394\\
-0.1280	0.0025\\
-0.1269	0.0261\\
-0.1257	0.0568\\
-0.1231	0.1103\\
-0.1188	0.1221\\
-0.1144	0.1161\\
-0.1128	0.0484\\
-0.1111	0.0316\\
-0.1113	0.0126\\
-0.1102	0.0032\\
-0.1066	0.0151\\
-0.1034	0.0165\\
-0.1017	0.0214\\
-0.1019	0.0200\\
-0.1017	0.0116\\
-0.1010	0.0036\\
-0.0972	0.0070\\
-0.0940	0.0214\\
-0.0921	0.0380\\
-0.0906	0.0392\\
-0.0886	0.0266\\
-0.0872	0.0264\\
-0.0857	0.0258\\
-0.0842	0.0046\\
-0.0817	0.0002\\
-0.0799	0.0130\\
-0.0786	0.0094\\
-0.0774	0.0100\\
-0.0750	0.0118\\
-0.0723	0.0233\\
-0.0713	0.0331\\
-0.0713	0.0266\\
-0.0714	0.0195\\
-0.0706	0.0537\\
-0.0688	0.0468\\
-0.0674	0.0335\\
-0.0675	0.0417\\
-0.0677	0.0547\\
-0.0660	0.0795\\
-0.0646	0.0569\\
-0.0615	0.0495\\
-0.0596	0.0546\\
-0.0592	0.0504\\
-0.0585	0.0410\\
-0.0564	0.0317\\
-0.0341	0.0566\\
-0.0132	0.0738\\
-0.0051	0.0709\\
0.0121	0.0551\\
0.0144	0.0367\\
0.0168	0.0357\\
0.0202	0.0379\\
0.0226	0.0422\\
0.0272	0.0478\\
0.0309	0.0429\\
0.0458	0.0581\\
0.0491	0.0697\\
0.0524	0.0446\\
0.0549	0.0299\\
0.0569	0.0304\\
0.0590	0.0169\\
0.0610	0.0207\\
0.0629	0.0185\\
0.0647	0.0236\\
0.0672	0.0275\\
0.0696	0.0233\\
0.0680	0.0173\\
0.0638	0.0144\\
0.0846	0.0554\\
0.0861	0.0726\\
0.0945	0.0273\\
0.0959	0.0132\\
0.1004	0.0057\\
0.1042	0.0058\\
0.1035	0.0215\\
0.1012	0.0196\\
0.1015	0.0071\\
0.1034	0.0004\\
0.1046	0.0065\\
0.1053	0.0102\\
0.1061	0.0215\\
0.1074	0.0125\\
0.1092	0.0041\\
0.1109	0.0034\\
0.1117	0.0008\\
0.1120	0.0456\\
0.1131	0.0283\\
0.1162	0.0189\\
0.1202	0.0048\\
0.1215	0.0106\\
0.1209	0.0074\\
0.1192	0.0113\\
0.1194	0.0139\\
0.1205	0.0047\\
0.1208	0.0037\\
0.1209	0.0044\\
0.1212	0.0148\\
0.1237	0.0152\\
0.1278	0.0183\\
0.1300	0.0225\\
0.1296	0.0150\\
0.1284	0.0057\\
0.1287	0.0149\\
0.1304	0.0227\\
0.1318	0.0212\\
0.1329	0.0076\\
0.1333	0.0055\\
0.1354	0.0005\\
0.1387	0.0086\\
0.1398	0.0061\\
0.1387	0.0109\\
0.1371	0.0077\\
0.1387	0.0150\\
0.1414	0.0416\\
0.1417	0.0351\\
0.1416	0.0136\\
0.1431	0.0027\\
0.1463	0.0055\\
0.1494	0.0191\\
0.1507	0.0144\\
0.1503	0.0105\\
0.1494	0.0063\\
0.1496	0.0115\\
0.1506	0.0112\\
0.1512	0.0048\\
0.1517	0.0048\\
0.1528	0.0009\\
0.1546	0.0009\\
0.1566	0.0129\\
0.1582	0.0003\\
0.1591	0.0083\\
0.1601	0.0001\\
0.1606	0.0027\\
0.1618	0.0032\\
0.1632	0.0076\\
0.1641	0.0038\\
0.1653	0.0111\\
0.1664	0.0231\\
0.1677	0.0195\\
0.1687	0.0145\\
0.1701	0.0045\\
0.1714	0.0093\\
0.1725	0.0027\\
0.1736	0.0100\\
0.1758	0.0225\\
0.1815	0.0074\\
0.1843	0.0008\\
0.1812	0.0030\\
0.1793	0.0039\\
0.1807	0.0103\\
0.1864	0.0145\\
0.1912	0.0122\\
0.1917	0.0105\\
0.1887	0.0091\\
0.1871	0.0283\\
0.1885	0.0282\\
0.1946	0.0117\\
0.1996	0.0042\\
0.1990	0.0098\\
0.1960	0.0054\\
0.1954	0.0052\\
0.1970	0.0143\\
0.2034	0.0181\\
0.2085	0.0233\\
0.2077	0.0028\\
0.2045	0.0396\\
0.2038	0.0352\\
0.2059	0.0297\\
0.2077	0.0266\\
0.2098	0.0120\\
0.2114	0.0070\\
0.2131	0.0001\\
0.2202	0.0048\\
0.2259	0.0103\\
0.2251	0.0120\\
0.2252	0.0125\\
0.2307	0.0115\\
0.2348	0.0000\\
0.2313	0.0056\\
0.2302	0.0099\\
0.2295	0.0203\\
0.2318	0.0148\\
0.2340	0.0128\\
0.2365	0.0019\\
0.2385	0.0026\\
0.2404	0.0165\\
0.2419	0.0374\\
0.2475	0.0390\\
0.2556	0.0170\\
0.2596	0.0041\\
0.2581	0.0033\\
0.2544	0.0249\\
0.2546	0.0217\\
0.2615	0.0119\\
0.2672	0.0027\\
0.2578	0.0143\\
0.2374	0.0372\\
0.2460	0.0428\\
0.2732	0.0478\\
0.2910	0.0307\\
0.2927	0.0416\\
0.2972	0.0375\\
0.2994	0.0193\\
0.2982	0.0158\\
0.2941	0.0018\\
0.2953	0.0023\\
0.3018	0.0196\\
0.3064	0.0206\\
0.3023	0.0011\\
0.3015	0.0015\\
0.3015	0.0095\\
0.3053	0.0045\\
0.3082	0.0063\\
0.3164	0.0058\\
0.3266	0.0063\\
0.3321	0.0033\\
0.3279	0.0039\\
0.3262	0.0101\\
0.3277	0.0100\\
0.3315	0.0252\\
0.3332	0.0508\\
0.3375	0.0319\\
0.3408	0.0186\\
0.3447	0.0151\\
0.3480	0.0009\\
0.3522	0.0005\\
0.3558	0.0024\\
0.3650	0.0044\\
0.3770	0.0066\\
0.3840	0.0026\\
0.3830	0.0077\\
0.3795	0.0230\\
0.3793	0.0207\\
0.3888	0.0103\\
0.4013	0.0057\\
0.4083	0.0169\\
0.4049	0.0151\\
0.4120	0.0185\\
0.4168	0.0247\\
0.4241	0.0223\\
0.4206	0.0351\\
0.4284	0.0397\\
0.4415	0.0403\\
0.4428	0.0014\\
0.4401	0.0199\\
0.4404	0.0460\\
0.4418	0.0617\\
0.4479	0.0260\\
0.4525	0.0142\\
0.4578	0.0369\\
0.4633	0.0195\\
0.4790	0.0214\\
0.4931	0.0021\\
0.4929	0.0162\\
0.4880	0.0109\\
0.4862	0.0175\\
0.4914	0.0188\\
0.4938	0.0079\\
0.4997	0.0147\\
0.5069	0.0194\\
0.5114	0.0374\\
0.5172	0.0320\\
0.5206	0.0202\\
0.5261	0.0322\\
0.5300	0.0250\\
0.5372	0.0202\\
0.5423	0.0250\\
0.5505	0.0230\\
0.5532	0.0197\\
0.5682	0.0129\\
0.5833	0.0412\\
0.5915	0.0658\\
0.5843	0.0468\\
0.5821	0.0107\\
0.5848	0.0055\\
0.5912	0.0156\\
0.5949	0.0062\\
0.6006	0.0025\\
0.6031	0.0063\\
0.6078	0.0048\\
0.6122	0.0129\\
0.6268	0.0013\\
0.6443	0.0132\\
0.6538	0.0036\\
0.6449	0.0070\\
0.6418	0.0033\\
0.6442	0.0052\\
0.6497	0.0172\\
0.6553	0.0106\\
0.6667	0.0153\\
0.6809	0.0108\\
0.6620	0.0050\\
0.6109	0.0894\\
0.6349	0.0821\\
0.7037	0.0040\\
-0.5706	0.0587\\
-0.5720	0.0648\\
-0.5718	0.0661\\
-0.5754	0.0454\\
-0.5875	0.0191\\
-0.5975	0.0211\\
-0.5974	0.0240\\
-0.5862	0.0099\\
-0.5759	0.0076\\
-0.5759	0.0172\\
-0.5754	0.0045\\
-0.5793	0.0260\\
-0.5915	0.0423\\
-0.5987	0.0450\\
-0.5937	0.0379\\
-0.5889	0.0233\\
-0.5907	0.0159\\
-0.6017	0.0046\\
-0.6072	0.0068\\
-0.5943	0.0147\\
-0.5788	0.0255\\
-0.5660	0.0361\\
-0.5637	0.0323\\
-0.5602	0.0297\\
-0.5597	0.0222\\
-0.5645	0.0189\\
-0.5778	0.0101\\
-0.5780	0.0095\\
-0.5649	0.0483\\
-0.5523	0.0532\\
-0.5493	0.0433\\
-0.5441	0.0287\\
-0.5439	0.0003\\
-0.5450	0.0030\\
-0.5556	0.0102\\
-0.5546	0.0305\\
-0.5438	0.0648\\
-0.5411	0.0712\\
-0.5424	0.0388\\
-0.5349	0.0304\\
-0.5247	0.0220\\
-0.5259	0.0171\\
-0.5264	0.0306\\
-0.5183	0.0479\\
-0.5024	0.0563\\
-0.4939	0.0495\\
-0.4887	0.0389\\
-0.4868	0.0398\\
-0.4833	0.0383\\
-0.4809	0.0468\\
-0.4819	0.0639\\
-0.4889	0.0633\\
-0.4921	0.0493\\
-0.4843	0.0339\\
-0.4685	0.0403\\
-0.4618	0.0252\\
-0.4651	0.0109\\
-0.4736	0.0149\\
-0.4717	0.0207\\
-0.4667	0.0112\\
-0.4517	0.0259\\
-0.4447	0.0711\\
-0.4461	0.0800\\
-0.4527	0.0785\\
-0.4503	0.0834\\
-0.4377	0.0557\\
-0.4258	0.0515\\
-0.4228	0.0372\\
-0.4239	0.0273\\
-0.4319	0.0037\\
-0.4308	0.0008\\
-0.4181	0.0045\\
-0.4068	0.0083\\
-0.4052	0.0258\\
-0.4107	0.0045\\
-0.4138	0.0090\\
-0.4067	0.0118\\
-0.3997	0.0215\\
-0.3888	0.0058\\
-0.3861	0.0283\\
-0.3903	0.0441\\
-0.3921	0.0664\\
-0.3846	0.0518\\
-0.3713	0.0241\\
-0.3644	0.0200\\
-0.3645	0.0291\\
-0.3676	0.0107\\
-0.3727	0.0093\\
-0.3736	0.0006\\
-0.3640	0.0184\\
-0.3514	0.0358\\
-0.3415	0.0326\\
-0.3387	0.0373\\
-0.3389	0.0343\\
-0.3437	0.0129\\
-0.3459	0.0196\\
-0.3396	0.0219\\
-0.3286	0.0026\\
-0.3168	0.0096\\
-0.3101	0.0340\\
-0.3102	0.0308\\
-0.3127	0.0236\\
-0.3138	0.0107\\
-0.3069	0.0036\\
-0.2952	0.0204\\
-0.2888	0.0293\\
-0.2888	0.0155\\
-0.2925	0.0012\\
-0.2903	0.0173\\
-0.2864	0.0239\\
-0.2759	0.0310\\
-0.2699	0.0500\\
-0.2693	0.0462\\
-0.2720	0.0316\\
-0.2701	0.0195\\
-0.2626	0.0207\\
-0.2558	0.0247\\
-0.2530	0.0247\\
-0.2483	0.0039\\
-0.2458	0.0012\\
-0.2412	0.0166\\
-0.2394	0.0220\\
-0.2413	0.0121\\
-0.2454	0.0125\\
-0.2465	0.0280\\
-0.2418	0.0465\\
-0.2349	0.0430\\
-0.2316	0.0367\\
-0.2261	0.0452\\
-0.2237	0.0431\\
-0.2255	0.0223\\
-0.2263	0.0171\\
-0.2220	0.0209\\
-0.2146	0.0250\\
-0.2107	0.0090\\
-0.2111	0.0078\\
-0.2128	0.0344\\
-0.2162	0.0090\\
-0.2167	0.0157\\
-0.2114	0.0561\\
-0.2040	0.0397\\
-0.1982	0.0085\\
-0.1965	0.0129\\
-0.1974	0.0690\\
-0.2011	0.0653\\
-0.1999	0.0717\\
-0.1922	0.0740\\
-0.1866	0.1165\\
-0.1862	0.1110\\
-0.1894	0.1053\\
-0.1940	0.0623\\
-0.1959	0.0160\\
-0.1949	0.0028\\
-0.1893	0.0125\\
-0.1838	0.0055\\
-0.1826	0.0257\\
-0.1806	0.0541\\
-0.1758	0.1022\\
-0.1745	0.1002\\
-0.1747	0.0496\\
-0.1737	0.0132\\
-0.1692	0.0278\\
-0.1647	0.0134\\
-0.1619	0.0329\\
-0.1598	0.0432\\
-0.1580	0.0488\\
-0.1557	0.0459\\
-0.1538	0.0048\\
-0.1542	0.0182\\
-0.1544	0.0310\\
-0.1529	0.0448\\
-0.1481	0.0642\\
-0.1440	0.0435\\
-0.1417	0.0162\\
-0.1397	0.0052\\
-0.1380	0.0310\\
-0.1362	0.0356\\
-0.1350	0.0358\\
-0.1348	0.0317\\
-0.1350	0.0209\\
-0.1333	0.0087\\
-0.1292	0.0055\\
-0.1256	0.0092\\
-0.1236	0.0142\\
-0.1235	0.0145\\
-0.1230	0.0114\\
-0.1199	0.0192\\
-0.1174	0.0236\\
-0.1132	0.0298\\
-0.1106	0.0283\\
-0.1102	0.0179\\
-0.1095	0.0291\\
-0.1060	0.0520\\
-0.1015	0.0500\\
-0.0993	0.0597\\
-0.0979	0.0744\\
-0.0969	0.0538\\
-0.0944	0.0373\\
-0.0920	0.0419\\
-0.0902	0.0362\\
-0.0892	0.0287\\
-0.0879	0.0243\\
-0.0854	0.0332\\
-0.0801	0.0245\\
-0.0773	0.0297\\
-0.0742	0.0401\\
-0.0612	0.0344\\
-0.0599	0.0292\\
-0.0582	0.0063\\
-0.0516	0.0051\\
-0.0491	0.0062\\
-0.0474	0.0216\\
-0.0446	0.0101\\
-0.0392	0.0041\\
-0.0359	0.0181\\
-0.0372	0.0432\\
-0.0389	0.0573\\
-0.0404	0.0493\\
-0.0402	0.0394\\
-0.0394	0.0290\\
-0.0380	0.0167\\
-0.0367	0.1464\\
-0.0364	0.1624\\
-0.0348	0.1158\\
-0.0330	0.0825\\
-0.0330	0.0474\\
-0.0328	0.0495\\
-0.0322	0.1087\\
-0.0325	0.1063\\
-0.0324	0.0907\\
-0.0314	0.0729\\
-0.0306	0.0504\\
-0.0303	0.0555\\
-0.0302	0.0523\\
-0.0270	0.0333\\
-0.0276	0.0200\\
-0.0271	0.0153\\
-0.0261	0.0731\\
-0.0257	0.0858\\
-0.0251	0.0341\\
-0.0245	0.0307\\
-0.0241	0.0296\\
-0.0234	0.0290\\
-0.0227	0.0297\\
-0.0191	0.0437\\
-0.0117	0.0131\\
-0.0109	0.0075\\
-0.0102	0.0082\\
-0.0094	0.0092\\
-0.0056	0.0066\\
-0.0029	0.0021\\
-0.0022	0.0034\\
-0.0014	0.0085\\
-0.0007	0.0242\\
0.0037	0.0172\\
0.0049	0.0013\\
0.0079	0.0085\\
0.0087	0.0032\\
0.0094	0.0022\\
0.0188	0.0070\\
0.0202	0.0008\\
0.0210	0.0128\\
0.0228	0.0995\\
0.0245	0.1760\\
0.0465	0.0948\\
0.0518	0.0149\\
0.0543	0.0234\\
0.0556	0.0250\\
0.0559	0.0345\\
0.0583	0.0405\\
0.0598	0.0361\\
0.0613	0.0384\\
0.0667	0.0069\\
0.0678	0.0094\\
0.0681	0.0095\\
0.0674	0.0146\\
0.0609	0.0574\\
0.0630	0.0310\\
0.0698	0.0114\\
0.0750	0.0054\\
0.0750	0.0073\\
0.0747	0.0458\\
0.0753	0.0547\\
0.0747	0.0296\\
0.0729	0.0372\\
0.0720	0.0179\\
0.0733	0.0215\\
0.0724	0.0078\\
0.0673	0.0341\\
0.0649	0.0872\\
0.0756	0.0856\\
0.0777	0.0698\\
0.0773	0.0800\\
0.0770	0.0565\\
0.0770	0.0655\\
0.0762	0.0835\\
0.0742	0.0813\\
0.0741	0.0639\\
0.0746	0.0401\\
0.0736	0.0361\\
0.0721	0.0899\\
0.0728	0.0552\\
0.0725	0.0314\\
0.0730	0.0087\\
0.0735	0.0057\\
0.0730	0.0076\\
0.0725	0.0000\\
0.0717	0.0004\\
0.0720	0.0037\\
0.0732	0.0227\\
0.0735	0.0149\\
0.0719	0.0091\\
0.0706	0.0051\\
0.0703	0.0051\\
0.0723	0.0006\\
0.0732	0.0306\\
0.0741	0.0425\\
0.0717	0.0130\\
0.0707	0.0486\\
0.0708	0.0553\\
0.0711	0.0513\\
0.0717	0.0362\\
0.0722	0.0118\\
0.0730	0.0103\\
0.0735	0.0418\\
0.0740	0.0307\\
0.0737	0.0061\\
0.0733	0.0005\\
0.0735	0.0074\\
0.0747	0.0159\\
0.0775	0.0131\\
0.0786	0.0116\\
0.0790	0.0061\\
0.0769	0.0102\\
0.0771	0.0095\\
0.0786	0.0058\\
0.0785	0.0004\\
0.0768	0.0404\\
0.0761	0.0520\\
0.0751	0.0576\\
0.0748	0.0353\\
0.0751	0.0098\\
0.0754	0.0348\\
0.0769	0.0262\\
0.0810	0.0694\\
0.0795	0.0699\\
0.0783	0.0676\\
0.0781	0.0048\\
0.0788	0.0166\\
0.0800	0.0185\\
0.0806	0.0190\\
0.0811	0.0196\\
0.0829	0.0136\\
0.0835	0.0109\\
0.0818	0.0106\\
0.0811	0.0106\\
0.0805	0.0010\\
0.0808	0.0090\\
0.0811	0.0160\\
0.0819	0.0182\\
0.0831	0.0237\\
0.0858	0.0109\\
0.0881	0.0086\\
0.0884	0.0002\\
0.0879	0.0072\\
0.0870	0.0063\\
0.0866	0.0038\\
0.0862	0.0095\\
0.0860	0.0149\\
0.0859	0.0215\\
0.0861	0.0269\\
0.0870	0.0043\\
0.0892	0.0009\\
0.0907	0.0054\\
0.0929	0.0155\\
0.0938	0.0285\\
0.0928	0.0467\\
0.0908	0.0441\\
0.0898	0.0375\\
0.0897	0.0205\\
0.0899	0.0033\\
0.0899	0.0027\\
0.0902	0.0143\\
0.0909	0.0306\\
0.0927	0.0655\\
0.0948	0.0319\\
0.0952	0.0238\\
0.0935	0.0395\\
0.0924	0.0132\\
0.0926	0.0034\\
0.0931	0.0109\\
0.0934	0.0022\\
0.0940	0.0205\\
0.0945	0.0145\\
0.0949	0.0086\\
0.0952	0.0021\\
0.0954	0.0093\\
0.0952	0.0009\\
0.0965	0.0257\\
0.0992	0.0523\\
0.1002	0.0271\\
0.0992	0.0105\\
0.0982	0.0227\\
0.0998	0.0148\\
0.1009	0.0038\\
0.0995	0.0044\\
0.1004	0.0174\\
0.1018	0.0121\\
0.1007	0.0028\\
0.0997	0.0091\\
0.1005	0.0028\\
0.1023	0.0032\\
0.1020	0.0004\\
0.1012	0.0015\\
0.1021	0.0013\\
0.1025	0.0089\\
0.1018	0.0064\\
0.1000	0.0019\\
0.0993	0.0192\\
0.0995	0.0169\\
0.1031	0.0022\\
0.1049	0.0159\\
0.1038	0.0275\\
0.1044	0.0419\\
0.1062	0.0426\\
0.1052	0.0362\\
0.1040	0.0147\\
0.1055	0.0024\\
0.1060	0.0039\\
0.1038	0.0172\\
0.1045	0.0364\\
0.1064	0.0401\\
0.1064	0.0467\\
0.1048	0.0466\\
0.1058	0.0386\\
0.1076	0.0334\\
0.1068	0.0163\\
0.1046	0.0168\\
0.1036	0.0230\\
0.1039	0.0126\\
0.1040	0.0317\\
0.1044	0.0181\\
0.1050	0.0077\\
0.1061	0.0003\\
0.1092	0.0036\\
0.1119	0.0098\\
0.1116	0.0091\\
0.1098	0.0091\\
0.1091	0.0026\\
0.1094	0.0140\\
0.1101	0.0072\\
0.1104	0.0012\\
0.1109	0.0061\\
0.1111	0.0199\\
0.1137	0.0310\\
0.1159	0.0058\\
0.1145	0.0057\\
0.1133	0.0036\\
0.1149	0.0036\\
0.1159	0.0044\\
0.1145	0.0015\\
0.1131	0.0012\\
0.1145	0.0096\\
0.1153	0.0063\\
0.1134	0.0094\\
0.1141	0.0092\\
0.1154	0.0150\\
0.1150	0.0351\\
0.1123	0.0455\\
0.1126	0.0273\\
0.1142	0.0131\\
0.1127	0.0124\\
0.1112	0.0186\\
0.1123	0.0132\\
0.1131	0.0029\\
0.1116	0.0007\\
0.1089	0.0065\\
0.1077	0.0050\\
0.1080	0.0046\\
0.1100	0.0020\\
0.1104	0.0001\\
0.1083	0.0071\\
0.1065	0.0310\\
0.1061	0.0396\\
0.1073	0.0299\\
0.1092	0.0212\\
0.1091	0.0269\\
0.1062	0.0304\\
0.1043	0.0301\\
0.1047	0.0216\\
0.1069	0.0083\\
0.1081	0.0019\\
0.1076	0.0002\\
0.1050	0.0116\\
0.1053	0.0210\\
0.1036	0.0171\\
0.0955	0.0264\\
0.0915	0.0282\\
0.0983	0.0118\\
0.1073	0.0023\\
0.1085	0.0006\\
0.1081	0.0053\\
0.1077	0.0155\\
0.1080	0.0142\\
0.1085	0.0235\\
0.1088	0.0138\\
0.1074	0.0051\\
0.1050	0.0115\\
0.1030	0.0046\\
0.1033	0.0081\\
0.1023	0.0143\\
0.1002	0.0112\\
0.0985	0.0092\\
0.0976	0.0135\\
0.0969	0.0188\\
0.0981	0.0199\\
0.1002	0.0091\\
0.1003	0.0034\\
0.0973	0.0131\\
0.0959	0.0075\\
0.0942	0.0138\\
0.0938	0.0266\\
0.0934	0.0228\\
0.0942	0.0239\\
0.0959	0.0123\\
0.0963	0.0087\\
0.0946	0.0210\\
0.0942	0.0156\\
0.0943	0.0121\\
0.0941	0.0119\\
0.0921	0.0069\\
0.0906	0.0075\\
0.0902	0.0054\\
0.0918	0.0168\\
0.0931	0.0206\\
0.0919	0.0143\\
0.0904	0.0105\\
0.0881	0.0060\\
0.0873	0.0144\\
0.0893	0.0241\\
0.0912	0.0130\\
0.0906	0.0046\\
0.0883	0.0016\\
0.0868	0.0001\\
0.0876	0.0024\\
0.0889	0.0150\\
0.0906	0.0204\\
0.0898	0.0218\\
0.0875	0.0282\\
0.0862	0.0383\\
0.0875	0.0192\\
0.0901	0.0118\\
0.0917	0.0006\\
0.0914	0.0020\\
0.0897	0.0092\\
0.0882	0.0249\\
0.0876	0.0333\\
0.0882	0.0423\\
0.0907	0.0395\\
0.0924	0.0428\\
0.0914	0.0355\\
0.0894	0.0308\\
0.0890	0.0190\\
0.0894	0.0008\\
0.0896	0.0016\\
0.0897	0.0007\\
0.0918	0.0067\\
0.0941	0.0058\\
0.0941	0.0045\\
0.0927	0.0032\\
0.0918	0.0038\\
0.0918	0.0048\\
0.0919	0.0039\\
0.0925	0.0075\\
0.0931	0.0003\\
0.0952	0.0002\\
0.0983	0.0160\\
0.0995	0.0213\\
0.0978	0.0175\\
0.0964	0.0053\\
0.0962	0.0744\\
0.0964	0.0998\\
0.0972	0.0701\\
0.0995	0.0523\\
0.1025	0.0178\\
0.1034	0.0098\\
0.1008	0.0101\\
0.0993	0.0056\\
0.0988	0.0051\\
0.0986	0.0040\\
0.0987	0.0063\\
0.0991	0.0093\\
0.1001	0.0116\\
0.1008	0.0069\\
0.1010	0.0031\\
0.1033	0.0188\\
0.1054	0.0226\\
0.1046	0.0284\\
0.1023	0.0279\\
0.1015	0.0188\\
0.1010	0.0050\\
0.1017	0.0029\\
0.1028	0.0024\\
0.1036	0.0070\\
0.1034	0.0108\\
0.1035	0.0095\\
0.1039	0.0073\\
0.1042	0.0071\\
0.1044	0.0070\\
0.1044	0.0023\\
0.1044	0.0171\\
0.1067	0.0122\\
0.1079	0.0061\\
0.1093	0.0017\\
0.1074	0.0022\\
0.1056	0.0080\\
0.1053	0.0168\\
0.1081	0.0263\\
0.1106	0.0275\\
0.1096	0.0162\\
0.1071	0.0235\\
0.1066	0.0259\\
0.1068	0.0206\\
0.1074	0.0260\\
0.1076	0.0243\\
0.1081	0.0235\\
0.1098	0.0218\\
0.1138	0.0111\\
0.1166	0.0000\\
0.1160	0.0099\\
0.1136	0.0064\\
0.1129	0.0242\\
0.1126	0.0347\\
0.1129	0.0334\\
0.1134	0.0063\\
0.1142	0.0014\\
0.1158	0.0026\\
0.1209	0.0140\\
0.1254	0.0008\\
0.1249	0.0016\\
0.1239	0.0025\\
0.1260	0.0032\\
0.1296	0.0071\\
0.1296	0.0114\\
0.1275	0.0126\\
0.1275	0.0176\\
0.1312	0.0222\\
0.1364	0.0057\\
0.1384	0.0071\\
0.1385	0.0254\\
0.1363	0.0240\\
0.1354	0.0582\\
0.1357	0.0605\\
0.1362	0.0955\\
0.1373	0.0667\\
0.1389	0.0382\\
0.1412	0.0563\\
0.1435	0.0290\\
0.1460	0.0101\\
0.1468	0.0012\\
0.1482	0.0041\\
0.1490	0.0018\\
0.1489	0.0178\\
0.1495	0.0218\\
0.1504	0.0195\\
0.1520	0.0162\\
0.1536	0.0167\\
0.1553	0.0159\\
0.1567	0.0293\\
0.1582	0.0340\\
0.1601	0.0238\\
0.1663	0.0027\\
0.1691	0.0063\\
0.1714	0.0034\\
0.1673	0.0033\\
0.1646	0.0026\\
0.1650	0.0073\\
0.1664	0.0069\\
0.1675	0.0107\\
0.1687	0.0209\\
0.1695	0.0250\\
0.1706	0.0186\\
0.1710	0.0261\\
0.1721	0.0418\\
0.1732	0.0486\\
0.1744	0.0469\\
0.1753	0.0440\\
0.1788	0.0418\\
0.1840	0.0043\\
0.1863	0.0353\\
0.1834	0.0513\\
0.1813	0.0131\\
0.1813	0.0166\\
0.1819	0.0657\\
0.1828	0.0979\\
0.1843	0.1024\\
0.1853	0.0964\\
0.1861	0.0658\\
0.1860	0.0410\\
0.1864	0.0299\\
0.1874	0.0351\\
0.1886	0.0271\\
0.1887	0.0198\\
0.1894	0.0206\\
0.1910	0.0043\\
0.1927	0.0265\\
0.1982	0.0169\\
0.2019	0.0133\\
0.2010	0.0567\\
0.2001	0.0753\\
0.2028	0.0750\\
0.2042	0.0677\\
0.2011	0.0526\\
0.1997	0.0339\\
0.1978	0.0195\\
0.1982	0.0113\\
0.1989	0.0142\\
0.2006	0.0138\\
0.2021	0.0143\\
0.2058	0.0135\\
0.2104	0.0158\\
0.2118	0.0192\\
0.2077	0.0165\\
0.2063	0.0196\\
0.2046	0.0197\\
0.2051	0.0152\\
0.2052	0.0103\\
0.2057	0.0144\\
0.2064	0.0114\\
0.2095	0.0033\\
0.2134	0.0008\\
0.2145	0.0043\\
0.2106	0.0164\\
0.2090	0.0256\\
0.2067	0.0258\\
0.2059	0.0096\\
0.2049	0.0197\\
0.2068	0.0111\\
0.2110	0.0013\\
0.2119	0.0046\\
0.2068	0.0036\\
0.2037	0.0074\\
0.2038	0.0098\\
0.2046	0.0046\\
0.2050	0.0178\\
0.2057	0.0340\\
0.2060	0.0307\\
0.2083	0.0149\\
0.2113	0.0062\\
0.2117	0.0015\\
0.2075	0.0176\\
0.2041	0.0530\\
0.2029	0.0593\\
0.2031	0.0451\\
0.2031	0.0202\\
0.2040	0.0180\\
0.2031	0.0283\\
0.2025	0.0389\\
0.2029	0.0087\\
0.2047	0.0091\\
0.2042	0.0281\\
0.2032	0.0343\\
0.2020	0.0171\\
0.2022	0.0028\\
0.2024	0.0071\\
0.2034	0.0072\\
0.2039	0.0215\\
0.2032	0.0394\\
0.2030	0.0281\\
0.2034	0.0130\\
0.2049	0.0301\\
0.2077	0.0003\\
0.2144	0.0186\\
0.2189	0.0181\\
0.2165	0.0012\\
0.2117	0.0294\\
0.2108	0.0344\\
0.2111	0.0196\\
0.2121	0.0064\\
0.2118	0.0025\\
0.2121	0.0128\\
0.2127	0.0002\\
0.2142	0.0188\\
0.2164	0.0379\\
0.2182	0.0457\\
0.2177	0.0488\\
0.2190	0.0559\\
0.2203	0.0477\\
0.2239	0.0310\\
0.2259	0.0296\\
0.2279	0.0246\\
0.2289	0.0051\\
0.2293	0.0131\\
0.2288	0.0347\\
0.2297	0.0654\\
0.2318	0.0116\\
0.2369	0.0438\\
0.2422	0.0492\\
0.2441	0.0750\\
0.2439	0.0855\\
0.2419	0.0529\\
0.2419	0.0315\\
0.2449	0.0410\\
0.2482	0.0175\\
0.2518	0.0208\\
0.2520	0.0048\\
0.2522	0.0092\\
-0.3156	0.0553\\
-0.3126	0.0395\\
-0.3096	0.0441\\
-0.3071	0.0498\\
-0.3058	0.0296\\
-0.3078	0.0341\\
-0.3153	0.0330\\
-0.3153	0.0268\\
-0.3013	0.0310\\
-0.2754	0.0438\\
-0.2827	0.0295\\
-0.3117	0.0087\\
-0.3257	0.0186\\
-0.3247	0.0058\\
-0.3236	0.0149\\
-0.3212	0.0172\\
-0.3137	0.0241\\
-0.3055	0.0306\\
-0.3020	0.0378\\
-0.3001	0.0632\\
-0.2999	0.0517\\
-0.2991	0.0387\\
-0.2971	0.0278\\
-0.2974	0.0191\\
-0.2976	0.0140\\
-0.2975	0.0271\\
-0.3004	0.0379\\
-0.3084	0.0246\\
-0.3099	0.0223\\
-0.3071	0.0236\\
-0.2987	0.0107\\
-0.2952	0.0041\\
-0.2976	0.0021\\
-0.3036	0.0021\\
-0.3041	0.0131\\
-0.2976	0.0216\\
-0.2915	0.0093\\
-0.2909	0.0298\\
-0.2923	0.0392\\
-0.2981	0.0530\\
-0.2974	0.0502\\
-0.2887	0.0429\\
-0.2826	0.0376\\
-0.2819	0.0037\\
-0.2811	0.0224\\
-0.2800	0.0083\\
-0.2794	0.0168\\
-0.2817	0.0012\\
-0.2834	0.0075\\
-0.2886	0.0035\\
-0.2877	0.0127\\
-0.2823	0.0278\\
-0.2828	0.0447\\
-0.2848	0.0526\\
-0.2801	0.0388\\
-0.2745	0.0361\\
-0.2758	0.0841\\
-0.2780	0.1213\\
-0.2752	0.1217\\
-0.2687	0.0738\\
-0.2664	0.0435\\
-0.2657	0.0434\\
-0.2644	0.0464\\
-0.2661	0.0449\\
-0.2716	0.0234\\
-0.2726	0.0167\\
-0.2692	0.0230\\
-0.2651	0.0204\\
-0.2675	0.0417\\
-0.2671	0.0507\\
-0.2603	0.0509\\
-0.2598	0.0241\\
-0.2611	0.0186\\
-0.2614	0.0133\\
-0.2548	0.0283\\
-0.2489	0.0344\\
-0.2466	0.0328\\
-0.2454	0.0387\\
-0.2460	0.0507\\
-0.2504	0.0279\\
-0.2530	0.0162\\
-0.2490	0.0240\\
-0.2452	0.0468\\
-0.2401	0.0608\\
-0.2403	0.0695\\
-0.2438	0.0618\\
-0.2455	0.0406\\
-0.2426	0.0347\\
-0.2363	0.0233\\
-0.2325	0.0404\\
-0.2338	0.0354\\
-0.2364	0.0359\\
-0.2397	0.0251\\
-0.2376	0.0211\\
-0.2315	0.0288\\
-0.2283	0.0292\\
-0.2299	0.0223\\
-0.2346	0.0224\\
-0.2343	0.0215\\
-0.2273	0.0303\\
-0.2238	0.0139\\
-0.2205	0.0118\\
-0.2192	0.0027\\
-0.2187	0.1198\\
-0.2176	0.2186\\
-0.2164	0.1582\\
-0.2179	0.0804\\
-0.2225	0.0282\\
-0.2234	0.0399\\
-0.2214	0.0820\\
-0.2181	0.0878\\
-0.2194	0.0764\\
-0.2186	0.0797\\
-0.2140	0.0674\\
-0.2143	0.0446\\
-0.2164	0.0329\\
-0.2135	0.0471\\
-0.2105	0.0435\\
-0.2058	0.0418\\
-0.2044	0.0080\\
-0.2061	0.0120\\
-0.2089	0.0207\\
-0.2093	0.0081\\
-0.2060	0.0002\\
-0.2074	0.0682\\
-0.2081	0.1489\\
-0.2063	0.1619\\
-0.2015	0.1440\\
-0.1985	0.1119\\
-0.1987	0.0634\\
-0.2013	0.0406\\
-0.2010	0.0228\\
-0.1966	0.0086\\
-0.1933	0.0134\\
-0.1926	0.0028\\
-0.1943	0.0061\\
-0.1978	0.0041\\
-0.1973	0.0065\\
-0.1932	0.0028\\
-0.1931	0.0079\\
-0.1949	0.0234\\
-0.1930	0.0279\\
-0.1907	0.0253\\
-0.1874	0.0289\\
-0.1878	0.0410\\
-0.1921	0.0339\\
-0.1969	0.0268\\
-0.1981	0.0357\\
-0.1939	0.0459\\
-0.1887	0.0330\\
-0.1863	0.0414\\
-0.1837	0.0297\\
-0.1856	0.0189\\
-0.1908	0.0004\\
-0.1922	0.0053\\
-0.1877	0.0024\\
-0.1838	0.0118\\
-0.1836	0.0162\\
-0.1849	0.0225\\
-0.1890	0.0216\\
-0.1895	0.0208\\
-0.1852	0.0217\\
-0.1825	0.0146\\
-0.1828	0.0107\\
-0.1832	0.0254\\
-0.1832	0.0403\\
-0.1823	0.0373\\
-0.1819	0.0287\\
-0.1821	0.0161\\
-0.1841	0.0159\\
-0.1888	0.0038\\
-0.1919	0.0007\\
-0.1900	0.0078\\
-0.1853	0.0066\\
-0.1837	0.0009\\
-0.1839	0.0104\\
-0.1842	0.0074\\
-0.1867	0.0184\\
-0.1896	0.0113\\
-0.1931	0.0065\\
-0.1919	0.0067\\
-0.1902	0.0030\\
-0.1930	0.0160\\
-0.1938	0.0218\\
-0.1901	0.0143\\
-0.1902	0.0373\\
-0.1913	0.0651\\
-0.1920	0.0120\\
-0.1886	0.0414\\
-0.1857	0.1038\\
-0.1853	0.0995\\
-0.1853	0.0467\\
-0.1857	0.0008\\
-0.1857	0.0035\\
-0.1859	0.0304\\
-0.1858	0.0413\\
-0.1858	0.0537\\
-0.1872	0.0545\\
-0.1903	0.0134\\
-0.1904	0.0099\\
-0.1866	0.0034\\
-0.1830	0.0110\\
-0.1824	0.0174\\
-0.1855	0.0169\\
-0.1873	0.0045\\
-0.1880	0.0063\\
-0.1847	0.0043\\
-0.1847	0.0341\\
-0.1862	0.0774\\
-0.1843	0.0607\\
-0.1799	0.0351\\
-0.1776	0.0125\\
-0.1789	0.0073\\
-0.1807	0.0048\\
-0.1828	0.0017\\
-0.1808	0.0311\\
-0.1755	0.0660\\
-0.1719	0.0325\\
-0.1708	0.0326\\
-0.1714	0.0426\\
-0.1692	0.0492\\
-0.1667	0.0101\\
-0.1602	0.0120\\
-0.1554	0.0327\\
-0.1532	0.0274\\
-0.1530	0.0139\\
-0.1504	0.0563\\
-0.1440	0.0496\\
-0.1376	0.0942\\
-0.1337	0.0828\\
-0.1331	0.0646\\
-0.1337	0.0422\\
-0.1317	0.0190\\
-0.1251	0.0248\\
-0.1194	0.0422\\
-0.1163	0.0318\\
-0.1132	0.0309\\
-0.1103	0.0517\\
-0.1085	0.0578\\
-0.1065	0.0612\\
-0.1048	0.0659\\
-0.1004	0.0357\\
-0.0978	0.0107\\
-0.0962	0.0361\\
-0.0931	0.0423\\
-0.0897	0.0563\\
-0.0874	0.0662\\
-0.0865	0.0416\\
-0.0836	0.0214\\
-0.0808	0.0072\\
-0.0790	0.0024\\
-0.0764	0.0109\\
-0.0727	0.0175\\
-0.0699	0.0044\\
-0.0688	0.0177\\
-0.0657	0.0257\\
-0.0634	0.0303\\
-0.0616	0.0215\\
-0.0538	0.0204\\
-0.0517	0.0195\\
-0.0500	0.0225\\
-0.0490	0.0275\\
-0.0467	0.0449\\
-0.0443	0.0597\\
-0.0419	0.0603\\
-0.0395	0.0611\\
-0.0380	0.0718\\
-0.0374	0.0446\\
-0.0359	0.0455\\
-0.0344	0.0395\\
-0.0312	0.0362\\
-0.0287	0.0352\\
-0.0272	0.0246\\
-0.0262	0.0278\\
-0.0249	0.0138\\
-0.0231	0.0019\\
-0.0206	0.0022\\
-0.0181	0.0082\\
-0.0170	0.0126\\
-0.0156	0.0102\\
-0.0146	0.0132\\
-0.0135	0.0033\\
-0.0117	0.0004\\
-0.0081	0.0038\\
-0.0060	0.0021\\
-0.0053	0.0036\\
-0.0043	0.0071\\
-0.0024	0.0101\\
-0.0003	0.0171\\
0.0010	0.0191\\
0.0018	0.0197\\
0.0023	0.0189\\
0.0036	0.0130\\
0.0051	0.0093\\
0.0074	0.0086\\
0.0099	0.0040\\
0.0194	0.0028\\
0.0207	0.0082\\
0.0218	0.0082\\
0.0235	0.0112\\
0.0261	0.0140\\
0.0291	0.0085\\
0.0307	0.0086\\
0.0308	0.0084\\
0.0318	0.0104\\
0.0359	0.0275\\
0.0363	0.0236\\
0.0395	0.0227\\
0.0427	0.0095\\
0.0558	0.0040\\
0.0616	0.0170\\
0.0621	0.0247\\
0.0622	0.0343\\
0.0625	0.0523\\
0.0639	0.0671\\
0.0672	0.0109\\
0.0699	0.0033\\
0.0705	0.0196\\
0.0701	0.0398\\
0.0703	0.0622\\
0.0713	0.0019\\
0.0793	0.0396\\
0.0795	0.0006\\
0.0795	0.0209\\
0.0798	0.0177\\
0.0802	0.0241\\
0.0808	0.0206\\
0.0812	0.0055\\
0.0822	0.0090\\
0.0847	0.0104\\
0.0881	0.0177\\
0.0902	0.0180\\
0.0900	0.0067\\
0.0896	0.0239\\
0.0897	0.0237\\
0.0913	0.0151\\
0.0936	0.0055\\
0.0945	0.0255\\
0.0946	0.0268\\
0.0947	0.0359\\
0.0964	0.0153\\
0.0987	0.0054\\
0.0999	0.0103\\
0.1018	0.0179\\
0.1048	0.0132\\
0.1061	0.0159\\
0.1043	0.0093\\
0.1038	0.0222\\
0.1037	0.0095\\
0.1017	0.0039\\
0.0953	0.0074\\
0.0944	0.0190\\
0.1035	0.0213\\
0.1138	0.0091\\
0.1164	0.0178\\
0.1174	0.0315\\
0.1184	0.0259\\
0.1189	0.0218\\
0.1197	0.0247\\
0.1206	0.0247\\
0.1209	0.0273\\
0.1198	0.0312\\
0.1180	0.0318\\
0.1197	0.0319\\
0.1219	0.0302\\
0.1219	0.0216\\
0.1195	0.0153\\
0.1181	0.0260\\
0.1186	0.0327\\
0.1197	0.0453\\
0.1211	0.0298\\
0.1248	0.0171\\
0.1290	0.0045\\
0.1305	0.0056\\
0.1304	0.0124\\
0.1280	0.0161\\
0.1255	0.0255\\
0.1244	0.0352\\
0.1250	0.0240\\
0.1259	0.0179\\
0.1294	0.0056\\
0.1448	0.0054\\
0.1569	0.0144\\
0.1520	0.0006\\
0.1441	0.0005\\
0.1341	0.0157\\
0.1317	0.0039\\
0.1322	0.0002\\
0.1340	0.0114\\
0.1378	0.0061\\
0.1405	0.0225\\
0.1395	0.0715\\
0.1370	0.0633\\
0.1366	0.0526\\
0.1375	0.0484\\
0.1411	0.0348\\
0.1438	0.0230\\
0.1431	0.0200\\
0.1425	0.0237\\
0.1446	0.0277\\
0.1461	0.0223\\
0.1455	0.0262\\
0.1430	0.0283\\
0.1411	0.0244\\
0.1409	0.0279\\
0.1418	0.0329\\
0.1434	0.0248\\
0.1438	0.0261\\
0.1459	0.0202\\
0.1496	0.0182\\
0.1511	0.0167\\
0.1484	0.0109\\
0.1473	0.0198\\
0.1458	0.0300\\
0.1482	0.0203\\
0.1522	0.0259\\
0.1537	0.0441\\
0.1515	0.0442\\
0.1537	0.0224\\
0.1571	0.0114\\
0.1573	0.0009\\
0.1539	0.0043\\
0.1515	0.0152\\
0.1513	0.0170\\
0.1520	0.0061\\
0.1532	0.0081\\
0.1543	0.0295\\
0.1539	0.0362\\
0.1535	0.0473\\
0.1539	0.0352\\
0.1555	0.0195\\
0.1600	0.0329\\
0.1617	0.0715\\
0.1580	0.0773\\
0.1551	0.0477\\
0.1543	0.0407\\
0.1548	0.0589\\
0.1553	0.0812\\
0.1558	0.0831\\
0.1561	0.0791\\
0.1576	0.0553\\
0.1614	0.0310\\
0.1628	0.0140\\
0.1593	0.0085\\
0.1570	0.0113\\
0.1571	0.0132\\
0.1606	0.0071\\
0.1632	0.0057\\
0.1624	0.0089\\
0.1603	0.0231\\
0.1582	0.0325\\
0.1578	0.0320\\
0.1611	0.0087\\
0.1639	0.0077\\
0.1627	0.0160\\
0.1593	0.0237\\
0.1571	0.0274\\
0.1584	0.0298\\
0.1601	0.0319\\
0.1579	0.0125\\
0.1448	0.0117\\
0.1387	0.0380\\
0.1503	0.0292\\
0.1644	0.0169\\
0.1655	0.0376\\
0.1646	0.0435\\
0.1638	0.0272\\
0.1632	0.0041\\
0.1627	0.0007\\
0.1561	0.0046\\
0.1421	0.0402\\
0.1366	0.0712\\
0.1465	0.0443\\
0.1588	0.0162\\
0.1602	0.0149\\
0.1595	0.0195\\
0.1592	0.0233\\
0.1577	0.0360\\
0.1542	0.0411\\
0.1521	0.0312\\
0.1496	0.0236\\
0.1490	0.0134\\
0.1522	0.0008\\
0.1553	0.0121\\
0.1544	0.0123\\
0.1513	0.0107\\
0.1490	0.0056\\
0.1504	0.0043\\
0.1525	0.0011\\
0.1554	0.0128\\
0.1540	0.0199\\
0.1502	0.0196\\
0.1484	0.0079\\
0.1510	0.0024\\
0.1556	0.0044\\
0.1576	0.0064\\
0.1543	0.0020\\
0.1530	0.0020\\
0.1514	0.0016\\
0.1537	0.0001\\
0.1558	0.0029\\
0.1495	0.0231\\
0.1369	0.0308\\
0.1407	0.0306\\
0.1560	0.0073\\
0.1659	0.0121\\
0.1671	0.0273\\
0.1674	0.0269\\
0.1676	0.0257\\
0.1672	0.0145\\
0.1648	0.0006\\
0.1621	0.0408\\
0.1603	0.0465\\
0.1622	0.0503\\
0.1662	0.0563\\
0.1676	0.0157\\
0.1664	0.0132\\
0.1654	0.0391\\
0.1684	0.0426\\
0.1698	0.0385\\
0.1665	0.0220\\
0.1673	0.0119\\
0.1702	0.0117\\
0.1693	0.0013\\
0.1675	0.0017\\
0.1658	0.0071\\
0.1658	0.0010\\
0.1688	0.0106\\
0.1746	0.0009\\
0.1807	0.0171\\
0.1818	0.0355\\
0.1798	0.0437\\
0.1686	0.0272\\
0.1537	0.0049\\
0.1587	0.0130\\
0.1763	0.0035\\
0.1874	0.0091\\
0.1880	0.0093\\
0.1887	0.0145\\
0.1897	0.0086\\
0.1902	0.0225\\
0.1906	0.0351\\
0.1889	0.0900\\
0.1852	0.0745\\
0.1836	0.0745\\
0.1848	0.0553\\
0.1902	0.0421\\
0.1971	0.0106\\
0.1992	0.0022\\
0.1953	0.0247\\
0.1779	0.0487\\
0.1700	0.0480\\
0.1837	0.0420\\
0.2019	0.0193\\
0.2056	0.0089\\
0.2063	0.0037\\
0.2067	0.0084\\
0.2071	0.0251\\
0.2062	0.0222\\
0.2044	0.0081\\
0.2017	0.0060\\
0.2025	0.0251\\
0.2062	0.0273\\
0.2075	0.0312\\
0.2040	0.0330\\
0.2052	0.0281\\
0.2067	0.0266\\
0.2088	0.0274\\
0.2066	0.0218\\
0.2097	0.0102\\
0.2140	0.0036\\
0.2132	0.0111\\
0.2123	0.0065\\
0.2139	0.0054\\
0.2171	0.0034\\
0.2155	0.0025\\
0.2120	0.0050\\
0.2115	0.0005\\
0.2166	0.0010\\
0.2236	0.0010\\
0.2265	0.0095\\
0.2234	0.0073\\
0.2236	0.0142\\
0.2271	0.0208\\
0.2295	0.0199\\
0.2258	0.0255\\
0.2229	0.0297\\
0.2216	0.0203\\
0.2220	0.0145\\
0.2238	0.0096\\
0.2311	0.0039\\
0.2343	0.0031\\
0.2382	0.0015\\
0.2345	0.0008\\
0.2332	0.0080\\
0.2358	0.0054\\
0.2430	0.0155\\
0.2510	0.0241\\
0.2549	0.0334\\
0.2563	0.0291\\
0.2543	0.0253\\
0.2500	0.0279\\
0.2455	0.0214\\
0.2500	0.0054\\
0.2560	0.0212\\
0.2581	0.0021\\
0.2538	0.0056\\
0.2526	0.0137\\
0.2525	0.0287\\
0.2580	0.0335\\
0.2644	0.0239\\
0.2671	0.0112\\
0.2640	0.0193\\
0.2629	0.0001\\
0.2651	0.0004\\
0.2736	0.0066\\
0.2770	0.0191\\
0.2809	0.0263\\
0.2767	0.0103\\
0.2758	0.0041\\
0.2778	0.0103\\
0.2810	0.0096\\
0.2837	0.0096\\
0.2863	0.0213\\
0.2895	0.0012\\
0.2990	0.0111\\
0.3028	0.0138\\
0.3070	0.0116\\
0.3022	0.0087\\
0.2994	0.0026\\
0.3015	0.0316\\
0.3117	0.0349\\
0.3192	0.0348\\
0.3173	0.0272\\
0.3134	0.0259\\
0.3103	0.0150\\
0.3141	0.0105\\
-0.2103	0.0020\\
-0.2117	0.0188\\
-0.2133	0.0077\\
-0.2139	0.0227\\
-0.2113	0.0191\\
-0.2067	0.0314\\
-0.2051	0.0429\\
-0.2080	0.0077\\
-0.2139	0.0289\\
-0.2140	0.0233\\
-0.2076	0.0317\\
-0.2043	0.0136\\
-0.2012	0.0040\\
-0.2000	0.0181\\
-0.1985	0.0465\\
-0.1966	0.0554\\
-0.1958	0.0291\\
-0.1956	0.0244\\
-0.1959	0.0206\\
-0.1955	0.0154\\
-0.1946	0.0052\\
-0.1934	0.0143\\
-0.1929	0.0166\\
-0.1934	0.0051\\
-0.1956	0.0038\\
-0.1970	0.0048\\
-0.1949	0.0087\\
-0.1910	0.0240\\
-0.1889	0.0200\\
-0.1883	0.0143\\
-0.1880	0.0041\\
-0.1895	0.0050\\
-0.1912	0.0062\\
-0.1923	0.0089\\
-0.1889	0.0154\\
-0.1830	0.0280\\
-0.1792	0.0563\\
-0.1775	0.0647\\
-0.1772	0.0467\\
-0.1770	0.0299\\
-0.1772	0.0243\\
-0.1764	0.0193\\
-0.1753	0.0103\\
-0.1753	0.0053\\
-0.1769	0.0138\\
-0.1782	0.0194\\
-0.1764	0.0136\\
-0.1723	0.0327\\
-0.1699	0.0350\\
-0.1702	0.0624\\
-0.1724	0.0314\\
-0.1718	0.0042\\
-0.1700	0.0072\\
-0.1655	0.0015\\
-0.1637	0.0009\\
-0.1631	0.0130\\
-0.1621	0.0407\\
-0.1610	0.0602\\
-0.1605	0.0601\\
-0.1597	0.0520\\
-0.1580	0.0367\\
-0.1584	0.0255\\
-0.1614	0.0214\\
-0.1622	0.0209\\
-0.1604	0.0173\\
-0.1559	0.0259\\
-0.1530	0.0321\\
-0.1510	0.0310\\
-0.1492	0.0235\\
-0.1483	0.0158\\
-0.1485	0.0132\\
-0.1483	0.0182\\
-0.1468	0.0432\\
-0.1458	0.0707\\
-0.1444	0.0488\\
-0.1441	0.0341\\
-0.1435	0.0282\\
-0.1429	0.0254\\
-0.1446	0.0237\\
-0.1457	0.0174\\
-0.1437	0.0111\\
-0.1414	0.0073\\
-0.1419	0.0049\\
-0.1430	0.0051\\
-0.1417	0.0080\\
-0.1408	0.0164\\
-0.1430	0.0246\\
-0.1439	0.0248\\
-0.1414	0.0242\\
-0.1387	0.0205\\
-0.1375	0.0140\\
-0.1366	0.0262\\
-0.1364	0.0507\\
-0.1359	0.0566\\
-0.1357	0.0620\\
-0.1358	0.0558\\
-0.1361	0.0395\\
-0.1366	0.0182\\
-0.1378	0.0185\\
-0.1382	0.0269\\
-0.1371	0.0370\\
-0.1356	0.0388\\
-0.1359	0.0382\\
-0.1364	0.0431\\
-0.1369	0.0665\\
-0.1370	0.0702\\
-0.1362	0.0577\\
-0.1356	0.0501\\
-0.1378	0.0646\\
-0.1397	0.0448\\
-0.1379	0.0283\\
-0.1345	0.0134\\
-0.1340	0.0122\\
-0.1344	0.0329\\
-0.1347	0.0357\\
-0.1346	0.0437\\
-0.1359	0.0470\\
-0.1377	0.0261\\
-0.1382	0.0375\\
-0.1349	0.0665\\
-0.1332	0.0866\\
-0.1326	0.0889\\
-0.1325	0.0730\\
-0.1325	0.0577\\
-0.1325	0.0493\\
-0.1334	0.0569\\
-0.1375	0.0516\\
-0.1399	0.0359\\
-0.1365	0.0287\\
-0.1312	0.0335\\
-0.1301	0.0300\\
-0.1312	0.0279\\
-0.1314	0.0368\\
-0.1311	0.0375\\
-0.1306	0.0385\\
-0.1302	0.0490\\
-0.1302	0.0308\\
-0.1316	0.0239\\
-0.1351	0.0401\\
-0.1369	0.0500\\
-0.1354	0.0517\\
-0.1320	0.0605\\
-0.1306	0.0526\\
-0.1304	0.0560\\
-0.1318	0.0487\\
-0.1338	0.0255\\
-0.1339	0.0291\\
-0.1307	0.0621\\
-0.1296	0.0512\\
-0.1314	0.0312\\
-0.1345	0.0439\\
-0.1348	0.0303\\
-0.1338	0.0267\\
-0.1307	0.0275\\
-0.1293	0.0478\\
-0.1290	0.0612\\
-0.1292	0.0605\\
-0.1285	0.0541\\
-0.1271	0.0374\\
-0.1259	0.0238\\
-0.1261	0.0186\\
-0.1265	0.0188\\
-0.1266	0.0272\\
-0.1264	0.0328\\
-0.1276	0.0402\\
-0.1293	0.0390\\
-0.1292	0.0347\\
-0.1261	0.0268\\
-0.1253	0.0060\\
-0.1256	0.0029\\
-0.1256	0.0044\\
-0.1252	0.0266\\
-0.1259	0.0251\\
-0.1265	0.0175\\
-0.1266	0.0387\\
-0.1249	0.0585\\
-0.1239	0.0441\\
-0.1232	0.0241\\
-0.1225	0.0035\\
-0.1221	0.0083\\
-0.1226	0.0113\\
-0.1229	0.0112\\
-0.1227	0.0041\\
-0.1224	0.0130\\
-0.1217	0.0163\\
-0.1214	0.0182\\
-0.1220	0.0245\\
-0.1231	0.0200\\
-0.1229	0.0276\\
-0.1218	0.0257\\
-0.1244	0.0076\\
-0.1259	0.0106\\
-0.1252	0.0191\\
-0.1223	0.0118\\
-0.1211	0.0058\\
-0.1205	0.0004\\
-0.1197	0.0089\\
-0.1203	0.0150\\
-0.1223	0.0104\\
-0.1229	0.0098\\
-0.1211	0.0176\\
-0.1186	0.0270\\
-0.1177	0.0188\\
-0.1184	0.0362\\
-0.1218	0.0481\\
-0.1228	0.0112\\
-0.1201	0.0246\\
-0.1223	0.0387\\
-0.1223	0.0425\\
-0.1205	0.0628\\
-0.1180	0.0702\\
-0.1197	0.0562\\
-0.1203	0.0552\\
-0.1172	0.0568\\
-0.1172	0.0052\\
-0.1188	0.0598\\
-0.1189	0.0662\\
-0.1169	0.0871\\
-0.1147	0.0794\\
-0.1134	0.0835\\
-0.1123	0.1247\\
-0.1113	0.1005\\
-0.1105	0.1052\\
-0.1103	0.0756\\
-0.1097	0.0672\\
-0.1087	0.0586\\
-0.1079	0.0473\\
-0.1085	0.0544\\
-0.1090	0.0548\\
-0.1103	0.0516\\
-0.1098	0.0571\\
-0.1078	0.0516\\
-0.1065	0.0529\\
-0.1064	0.0486\\
-0.1054	0.0334\\
-0.1041	0.0368\\
-0.1050	0.0279\\
-0.1059	0.0192\\
-0.1067	0.0132\\
-0.1052	0.0145\\
-0.1034	0.0182\\
-0.1019	0.0210\\
-0.1008	0.0148\\
-0.1002	0.0248\\
-0.1004	0.0316\\
-0.1013	0.0320\\
-0.1015	0.0346\\
-0.1009	0.0212\\
-0.1004	0.0129\\
-0.0995	0.0039\\
-0.0990	0.0092\\
-0.0994	0.0131\\
-0.0997	0.0133\\
-0.0993	0.0142\\
-0.0984	0.0153\\
-0.0978	0.0174\\
-0.0978	0.0130\\
-0.0983	0.0491\\
-0.1001	0.0707\\
-0.1015	0.0623\\
-0.1027	0.0475\\
-0.1012	0.0455\\
-0.0987	0.0852\\
-0.0982	0.0988\\
-0.0988	0.0915\\
-0.0994	0.0810\\
-0.0993	0.0788\\
-0.0987	0.0785\\
-0.0996	0.0696\\
-0.1022	0.0457\\
-0.1031	0.0290\\
-0.1025	0.0286\\
-0.1002	0.0240\\
-0.0991	0.0392\\
-0.0990	0.0360\\
-0.0993	0.0243\\
-0.1008	0.0257\\
-0.1030	0.0327\\
-0.1037	0.0334\\
-0.1034	0.0562\\
-0.1018	0.0406\\
-0.1003	0.0165\\
-0.1004	0.0190\\
-0.1023	0.0110\\
-0.1030	0.0321\\
-0.1020	0.0328\\
-0.1038	0.0271\\
-0.1050	0.0283\\
-0.1061	0.0303\\
-0.1037	0.0259\\
-0.1036	0.0216\\
-0.1049	0.0083\\
-0.1043	0.0172\\
-0.1019	0.0164\\
-0.0999	0.0419\\
-0.1007	0.0489\\
-0.1024	0.0296\\
-0.1052	0.0036\\
-0.1048	0.0079\\
-0.1021	0.0022\\
-0.1007	0.0244\\
-0.1004	0.0292\\
-0.1006	0.0133\\
-0.1010	0.0065\\
-0.1015	0.0001\\
-0.1022	0.0027\\
-0.1046	0.0065\\
-0.1061	0.0211\\
-0.1047	0.0138\\
-0.1021	0.0335\\
-0.1013	0.0382\\
-0.1013	0.0260\\
-0.1012	0.0169\\
-0.1008	0.0143\\
-0.1008	0.0251\\
-0.1008	0.0218\\
-0.1009	0.0206\\
-0.1010	0.0091\\
-0.1011	0.0234\\
-0.1011	0.0222\\
-0.1012	0.0279\\
-0.1013	0.0195\\
-0.1013	0.0370\\
-0.1011	0.0385\\
-0.1009	0.0275\\
-0.1006	0.0114\\
-0.1003	0.0136\\
-0.1003	0.0174\\
-0.1004	0.0192\\
-0.1007	0.0000\\
-0.1008	0.0005\\
-0.1007	0.0109\\
-0.1003	0.0158\\
-0.1016	0.0048\\
-0.1015	0.0055\\
-0.0997	0.0208\\
-0.0982	0.0188\\
-0.0989	0.0205\\
-0.1016	0.0140\\
-0.1034	0.0651\\
-0.1019	0.0672\\
-0.0993	0.0923\\
-0.0986	0.0385\\
-0.0988	0.0173\\
-0.0979	0.0081\\
-0.0985	0.0042\\
-0.0998	0.0164\\
-0.1016	0.0598\\
-0.1004	0.0546\\
-0.0978	0.0512\\
-0.0970	0.0350\\
-0.0981	0.0168\\
-0.0996	0.0044\\
-0.0995	0.0091\\
-0.0978	0.0111\\
-0.0976	0.0302\\
-0.0985	0.0318\\
-0.0990	0.0209\\
-0.0975	0.0347\\
-0.0960	0.0386\\
-0.0952	0.0152\\
-0.0945	0.0096\\
-0.0943	0.0092\\
-0.0959	0.0064\\
-0.0966	0.0084\\
-0.0965	0.0096\\
-0.0938	0.0035\\
-0.0918	0.0123\\
-0.0914	0.0214\\
-0.0913	0.0281\\
-0.0911	0.0318\\
-0.0908	0.0292\\
-0.0903	0.0229\\
-0.0895	0.0096\\
-0.0889	0.0035\\
-0.0887	0.0115\\
-0.0889	0.0069\\
-0.0904	0.0277\\
-0.0925	0.0377\\
-0.0929	0.0537\\
-0.0908	0.0586\\
-0.0895	0.0419\\
-0.0891	0.0114\\
-0.0885	0.0149\\
-0.0889	0.0076\\
-0.0898	0.0176\\
-0.0913	0.1102\\
-0.0911	0.1575\\
-0.0895	0.1330\\
-0.0885	0.1199\\
-0.0892	0.1157\\
-0.0909	0.1005\\
-0.0914	0.0882\\
-0.0906	0.0799\\
-0.0889	0.0684\\
-0.0887	0.0568\\
-0.0904	0.0455\\
-0.0925	0.0320\\
-0.0930	0.0252\\
-0.0895	0.0293\\
-0.0905	0.0315\\
-0.0920	0.0245\\
-0.0939	0.0230\\
-0.0930	0.0159\\
-0.0903	0.0034\\
-0.0892	0.0077\\
-0.0890	0.0101\\
-0.0912	0.0188\\
-0.0926	0.0157\\
-0.0916	0.0222\\
-0.0900	0.0142\\
-0.0893	0.0379\\
-0.0893	0.0507\\
-0.0901	0.0569\\
-0.0900	0.0567\\
-0.0890	0.0604\\
-0.0882	0.0524\\
-0.0884	0.0441\\
-0.0911	0.0444\\
-0.0924	0.0388\\
-0.0934	0.0307\\
-0.0916	0.0298\\
-0.0915	0.0261\\
-0.0924	0.0178\\
-0.0916	0.0155\\
-0.0897	0.0154\\
-0.0890	0.0150\\
-0.0891	0.0071\\
-0.0892	0.0001\\
-0.0895	0.0187\\
-0.0899	0.0152\\
-0.0917	0.0127\\
-0.0917	0.0041\\
-0.0903	0.0237\\
-0.0906	0.0203\\
-0.0907	0.0387\\
-0.0904	0.0513\\
-0.0901	0.0740\\
-0.0897	0.0593\\
-0.0894	0.0509\\
-0.0893	0.0458\\
-0.0891	0.0228\\
-0.0898	0.0008\\
-0.0913	0.0099\\
-0.0930	0.0085\\
-0.0925	0.0068\\
-0.0917	0.0096\\
-0.0918	0.0035\\
-0.0943	0.0065\\
-0.0950	0.0342\\
-0.0923	0.0184\\
-0.0920	0.0010\\
-0.0932	0.0020\\
-0.0929	0.0002\\
-0.0920	0.0013\\
-0.0905	0.0101\\
-0.0903	0.0010\\
-0.0927	0.0113\\
-0.0947	0.0025\\
-0.0920	0.0750\\
-0.0908	0.1177\\
-0.0916	0.0900\\
-0.0911	0.0620\\
-0.0905	0.0541\\
-0.0914	0.0412\\
-0.0909	0.0107\\
-0.0920	0.0039\\
-0.0952	0.0069\\
-0.0960	0.0028\\
-0.0932	0.0086\\
-0.0926	0.0281\\
-0.0941	0.0417\\
-0.0960	0.0561\\
-0.0952	0.0385\\
-0.0934	0.0353\\
-0.0891	0.0193\\
-0.0892	0.0309\\
-0.0901	0.0169\\
-0.0898	0.0143\\
-0.0922	0.0347\\
-0.0917	0.0961\\
-0.0901	0.0395\\
-0.0905	0.0119\\
-0.0912	0.0145\\
-0.0911	0.0074\\
-0.0900	0.0060\\
-0.0888	0.0101\\
-0.0890	0.0044\\
-0.0894	0.0096\\
-0.0898	0.0065\\
-0.0898	0.0057\\
-0.0899	0.0369\\
-0.0906	0.0473\\
-0.0925	0.0600\\
-0.0924	0.0246\\
-0.0897	0.0072\\
-0.0878	0.0224\\
-0.0882	0.0147\\
-0.0882	0.0101\\
-0.0890	0.0074\\
-0.0901	0.0250\\
-0.0902	0.0325\\
-0.0885	0.0157\\
-0.0859	0.0432\\
-0.0849	0.0683\\
-0.0878	0.0590\\
-0.0864	0.0816\\
-0.0812	0.0887\\
-0.0801	0.0822\\
-0.0809	0.0792\\
-0.0818	0.0645\\
-0.0819	0.0772\\
-0.0808	0.1095\\
-0.0802	0.1004\\
-0.0802	0.0469\\
-0.0789	0.0273\\
-0.0742	0.0249\\
-0.0722	0.0214\\
-0.0696	0.0302\\
-0.0683	0.0136\\
-0.0668	0.0286\\
-0.0669	0.0698\\
-0.0687	0.0651\\
-0.0683	0.0596\\
-0.0650	0.0639\\
-0.0646	0.0507\\
-0.0616	0.0367\\
-0.0539	0.0402\\
-0.0510	0.0346\\
-0.0537	0.0288\\
-0.0562	0.0188\\
-0.0559	0.0172\\
-0.0538	0.0037\\
-0.0513	0.0268\\
-0.0488	0.0194\\
-0.0464	0.0163\\
-0.0450	0.0153\\
-0.0419	0.0010\\
-0.0394	0.0060\\
-0.0385	0.0064\\
-0.0378	0.0136\\
-0.0357	0.0277\\
-0.0331	0.0483\\
-0.0313	0.1199\\
-0.0302	0.1227\\
-0.0300	0.1397\\
-0.0294	0.1340\\
-0.0284	0.1198\\
-0.0233	0.1168\\
-0.0196	0.1109\\
-0.0182	0.0962\\
-0.0190	0.0905\\
-0.0189	0.0776\\
-0.0151	0.0510\\
-0.0125	0.0419\\
-0.0070	0.0397\\
-0.0058	0.0536\\
-0.0063	0.0648\\
0.0015	0.0477\\
0.0046	0.0508\\
0.0050	0.0496\\
0.0039	0.0461\\
0.0049	0.0459\\
0.0074	0.0408\\
0.0091	0.0290\\
0.0101	0.0305\\
0.0106	0.0100\\
0.0125	0.0057\\
0.0134	0.0338\\
0.0144	0.0455\\
0.0167	0.0421\\
0.0175	0.0002\\
0.0208	0.0013\\
0.0208	0.0036\\
0.0222	0.0028\\
0.0257	0.0066\\
0.0275	0.0116\\
0.0273	0.0140\\
0.0275	0.0121\\
0.0406	0.0095\\
0.0450	0.0115\\
0.0461	0.0064\\
0.0550	0.0083\\
0.0578	0.0122\\
0.0576	0.0117\\
0.0568	0.0065\\
0.0570	0.0462\\
0.0582	0.0730\\
0.0600	0.0623\\
0.0609	0.0695\\
0.0610	0.0671\\
0.0612	0.0710\\
0.0614	0.0719\\
0.0617	0.0838\\
0.0621	0.1050\\
0.0625	0.1127\\
0.0638	0.1005\\
0.0652	0.0973\\
0.0660	0.0967\\
0.0666	0.0867\\
0.0673	0.0709\\
0.0693	0.0423\\
0.0710	0.0199\\
0.0711	0.0193\\
0.0701	0.0129\\
0.0705	0.0085\\
0.0711	0.0103\\
0.0724	0.0040\\
0.0733	0.0232\\
0.0730	0.0154\\
0.0727	0.0011\\
0.0737	0.0062\\
0.0747	0.0063\\
0.0741	0.0207\\
0.0746	0.0292\\
0.0761	0.0163\\
0.0767	0.0054\\
0.0766	0.0065\\
0.0764	0.0233\\
0.0768	0.0208\\
0.0789	0.0208\\
0.0815	0.0483\\
0.0826	0.0461\\
0.0818	0.0549\\
0.0828	0.0574\\
0.0845	0.0397\\
0.0847	0.0357\\
0.0837	0.0453\\
0.0827	0.0355\\
0.0827	0.0122\\
0.0832	0.0035\\
0.0839	0.0059\\
0.0847	0.0011\\
0.0856	0.0238\\
0.0864	0.0087\\
0.0870	0.0025\\
0.0882	0.0037\\
0.0888	0.0458\\
0.0898	0.0423\\
0.0902	0.0346\\
0.0905	0.0256\\
0.0913	0.0119\\
0.0935	0.0096\\
0.0963	0.0117\\
0.0978	0.0333\\
0.0969	0.0418\\
0.0962	0.0568\\
0.0961	0.0681\\
0.0962	0.0611\\
0.0969	0.0457\\
0.0991	0.0291\\
0.1010	0.0500\\
0.0972	0.0525\\
0.0898	0.0589\\
0.0924	0.0414\\
0.1017	0.0299\\
0.1074	0.0328\\
0.1084	0.0246\\
0.1084	0.0168\\
0.1078	0.0174\\
0.1059	0.0069\\
0.1053	0.0039\\
0.1061	0.0026\\
0.1070	0.0015\\
0.1073	0.0059\\
0.1077	0.0286\\
0.1082	0.0195\\
0.1092	0.0367\\
0.1102	0.0307\\
0.1110	0.0386\\
0.1122	0.0320\\
0.1161	0.0183\\
0.1180	0.0062\\
0.1201	0.0227\\
0.1184	0.0218\\
0.1173	0.0148\\
0.1174	0.0156\\
0.1179	0.0362\\
0.1184	0.0353\\
0.1191	0.0440\\
0.1198	0.0415\\
0.1210	0.0385\\
0.1217	0.0230\\
0.1248	0.0225\\
0.1290	0.0095\\
0.1312	0.0027\\
0.1300	0.0111\\
0.1294	0.0128\\
0.1305	0.0283\\
0.1343	0.0271\\
0.1378	0.0140\\
0.1386	0.0102\\
0.1377	0.0094\\
0.1371	0.0088\\
0.1377	0.0104\\
0.1389	0.0171\\
0.1403	0.0311\\
0.1421	0.0621\\
0.1438	0.0703\\
0.1454	0.0519\\
0.1472	0.0230\\
0.1518	0.0115\\
0.1539	0.0072\\
0.1564	0.0033\\
0.1547	0.0059\\
0.1541	0.0026\\
0.1551	0.0010\\
0.1565	0.0008\\
0.1574	0.0225\\
0.1588	0.0223\\
0.1612	0.0051\\
0.1635	0.0031\\
0.1683	0.0026\\
0.1704	0.0055\\
0.1682	0.0033\\
0.1671	0.0082\\
0.1685	0.0205\\
0.1736	0.0234\\
0.1780	0.0148\\
0.1776	0.0148\\
0.1758	0.0163\\
0.1739	0.0195\\
0.1746	0.0251\\
0.1803	0.0082\\
0.1856	0.0104\\
0.1852	0.0046\\
0.1819	0.0055\\
0.1806	0.0083\\
0.1814	0.0125\\
0.1821	0.0227\\
0.1841	0.0235\\
0.1865	0.0404\\
0.1874	0.0665\\
0.1913	0.0699\\
0.1959	0.0609\\
0.1976	0.0491\\
0.1939	0.0350\\
0.1931	0.0329\\
0.1949	0.0323\\
0.1956	0.0138\\
0.1966	0.0202\\
0.1975	0.0256\\
0.1986	0.0375\\
0.2033	0.0260\\
0.2069	0.0229\\
0.2051	0.0194\\
0.2042	0.0182\\
0.2082	0.0263\\
0.2106	0.0197\\
0.2090	0.0229\\
0.2045	0.0136\\
0.2027	0.0137\\
0.2035	0.0040\\
0.2042	0.0060\\
0.2049	0.0127\\
0.2044	0.0015\\
0.2069	0.0005\\
0.2120	0.0089\\
0.2142	0.0142\\
0.2125	0.0098\\
0.2084	0.0191\\
0.2072	0.0255\\
0.2100	0.0199\\
0.2130	0.0206\\
0.2133	0.0153\\
0.2097	0.0081\\
0.2076	0.0054\\
0.2074	0.0410\\
0.2086	0.0354\\
0.2127	0.0318\\
0.2139	0.0385\\
0.2100	0.0581\\
0.2061	0.0553\\
0.2048	0.0729\\
0.2040	0.0664\\
0.2035	0.0628\\
0.2075	0.0390\\
0.2092	0.0514\\
0.2093	0.0487\\
0.2033	0.0502\\
0.2026	0.0152\\
0.2042	0.0192\\
0.2006	0.0185\\
0.1961	0.0141\\
0.1956	0.0065\\
0.1948	0.0400\\
0.1904	0.0619\\
0.1837	0.0763\\
0.1777	0.0601\\
0.1736	0.0394\\
0.1704	0.0358\\
0.1689	0.0291\\
0.1672	0.0310\\
0.1655	0.0294\\
0.1586	0.0359\\
0.1501	0.0453\\
0.1429	0.0630\\
0.1371	0.0633\\
0.1313	0.0929\\
0.1261	0.0773\\
0.1212	0.0463\\
0.1175	0.0391\\
0.1137	0.0103\\
0.1112	0.0332\\
0.1040	0.0175\\
0.0960	0.0615\\
0.0878	0.0642\\
0.0837	0.0259\\
0.0793	0.0182\\
0.0739	0.0272\\
0.0677	0.0327\\
0.0615	0.0409\\
0.0563	0.0582\\
0.0541	0.0764\\
0.0504	0.0941\\
0.0462	0.0893\\
0.0360	0.0894\\
0.0314	0.0812\\
0.0278	0.0768\\
0.0241	0.0539\\
0.0196	0.0482\\
0.0148	0.0205\\
0.0126	0.0187\\
0.0086	0.0272\\
0.0048	0.0295\\
0.0010	0.0331\\
-0.0047	0.0404\\
-0.0083	0.0602\\
-0.0119	0.0622\\
-0.0164	0.0891\\
-0.0212	0.0640\\
-0.0253	0.0453\\
-0.0291	0.0179\\
-0.0331	0.0084\\
-0.0362	0.0065\\
-0.0383	0.0021\\
-0.0416	0.0033\\
-0.0435	0.0126\\
-0.0472	0.0064\\
-0.0503	0.0089\\
-0.0550	0.0067\\
-0.0600	0.0157\\
-0.0632	0.0063\\
-0.0640	0.0010\\
-0.0655	0.0049\\
-0.0682	0.0021\\
-0.0721	0.0276\\
-0.0747	0.0285\\
-0.0752	0.0249\\
-0.0760	0.0132\\
-0.0778	0.0253\\
-0.0801	0.0367\\
-0.0827	0.0357\\
-0.0838	0.0334\\
-0.0857	0.0227\\
-0.0877	0.0273\\
-0.0900	0.0334\\
-0.0941	0.0417\\
-0.0992	0.0393\\
-0.1017	0.0345\\
-0.1005	0.0411\\
-0.1002	0.0392\\
-0.1009	0.0208\\
-0.1030	0.0282\\
-0.1049	0.0094\\
-0.1082	0.0125\\
-0.1125	0.0321\\
-0.1145	0.0574\\
-0.1135	0.0653\\
-0.1134	0.0686\\
-0.1157	0.0507\\
-0.1203	0.0325\\
-0.1224	0.0317\\
-0.1244	0.0200\\
-0.1226	0.0414\\
-0.1215	0.0273\\
-0.1242	0.0011\\
-0.1286	0.0047\\
-0.1307	0.0200\\
-0.1294	0.0195\\
-0.1285	0.0073\\
-0.1280	0.0034\\
-0.1285	0.0018\\
-0.1299	0.0145\\
-0.1310	0.0122\\
-0.1318	0.0169\\
-0.1322	0.0129\\
-0.1331	0.0142\\
-0.1340	0.0169\\
-0.1347	0.0282\\
-0.1350	0.0282\\
-0.1350	0.0232\\
-0.1352	0.0412\\
-0.1360	0.0365\\
-0.1378	0.0328\\
-0.1409	0.0395\\
-0.1430	0.0142\\
-0.1423	0.0088\\
-0.1397	0.0086\\
-0.1382	0.0088\\
-0.1379	0.0161\\
-0.1381	0.0123\\
-0.1391	0.0147\\
-0.1393	0.0499\\
-0.1400	0.0489\\
-0.1442	0.0403\\
-0.1493	0.0256\\
-0.1519	0.0214\\
-0.1513	0.0236\\
-0.1473	0.0398\\
-0.1453	0.0201\\
-0.1419	0.0272\\
-0.1435	0.0279\\
-0.1463	0.0594\\
-0.1467	0.0849\\
-0.1441	0.0825\\
-0.1424	0.0679\\
-0.1431	0.0446\\
-0.1442	0.0454\\
-0.1475	0.0461\\
-0.1497	0.0253\\
-0.1482	0.0277\\
-0.1448	0.0174\\
-0.1434	0.0076\\
-0.1453	0.0248\\
-0.1459	0.0160\\
-0.1438	0.0163\\
-0.1401	0.0316\\
-0.1390	0.0225\\
-0.1403	0.0257\\
-0.1422	0.0531\\
-0.1416	0.0497\\
-0.1390	0.0292\\
-0.1398	0.0102\\
-0.1406	0.0120\\
-0.1406	0.0095\\
-0.1368	0.0032\\
-0.1338	0.0074\\
-0.1328	0.0056\\
-0.1342	0.0071\\
-0.1354	0.0277\\
-0.1343	0.0242\\
-0.1296	0.0417\\
-0.1281	0.0247\\
-0.1275	0.0327\\
-0.1296	0.0392\\
-0.1310	0.0472\\
-0.1289	0.0400\\
-0.1249	0.0444\\
-0.1227	0.0411\\
-0.1237	0.0379\\
-0.1252	0.0274\\
-0.1267	0.0461\\
-0.1249	0.0364\\
-0.1212	0.0305\\
-0.1192	0.0312\\
-0.1198	0.0418\\
-0.1217	0.0467\\
-0.1216	0.0495\\
-0.1186	0.0409\\
-0.1170	0.0198\\
-0.1151	0.0056\\
-0.1156	0.0030\\
-0.1175	0.0002\\
-0.1178	0.0025\\
-0.1157	0.0090\\
-0.1157	0.0156\\
-0.1161	0.0200\\
-0.1163	0.0218\\
-0.1145	0.0219\\
-0.1148	0.0056\\
-0.1167	0.0271\\
-0.1179	0.0595\\
-0.1176	0.0624\\
-0.1142	0.0023\\
-0.1126	0.0352\\
-0.1098	0.0385\\
-0.1112	0.0154\\
-0.1132	0.0209\\
-0.1129	0.0304\\
-0.1102	0.0058\\
-0.1085	0.0606\\
-0.1093	0.0636\\
-0.1121	0.0586\\
-0.1132	0.0483\\
-0.1132	0.0687\\
-0.1100	0.0483\\
-0.1080	0.0244\\
-0.1090	0.0169\\
-0.1120	0.0147\\
-0.1141	0.0052\\
-0.1131	0.0250\\
-0.1116	0.0227\\
-0.1092	0.0086\\
-0.1086	0.0085\\
-0.1109	0.0091\\
-0.1133	0.0351\\
-0.1126	0.0595\\
-0.1108	0.0443\\
-0.1110	0.0341\\
-0.1113	0.0389\\
-0.1108	0.0537\\
-0.1107	0.0478\\
-0.1121	0.0197\\
-0.1121	0.0200\\
-0.1101	0.0279\\
-0.1110	0.0468\\
-0.1125	0.0422\\
-0.1123	0.0364\\
-0.1096	0.0269\\
-0.1078	0.0231\\
-0.1080	0.0468\\
-0.1077	0.0408\\
-0.1074	0.0389\\
-0.1100	0.0325\\
-0.1120	0.0848\\
-0.1102	0.0452\\
-0.1067	0.0132\\
-0.1062	0.0350\\
-0.1074	0.0278\\
-0.1103	0.0163\\
-0.1118	0.0263\\
-0.1107	0.0302\\
-0.1078	0.0240\\
-0.1062	0.0105\\
-0.1071	0.0430\\
-0.1094	0.0358\\
-0.1100	0.0159\\
-0.1084	0.0069\\
-0.1054	0.0455\\
-0.1041	0.0731\\
-0.1039	0.0705\\
-0.1036	0.0673\\
-0.1031	0.0379\\
-0.1024	0.0299\\
-0.1033	0.0208\\
-0.1047	0.0105\\
-0.1075	0.0033\\
-0.1083	0.0029\\
-0.1064	0.0047\\
-0.1040	0.0103\\
-0.1021	0.0366\\
-0.1017	0.0568\\
-0.1030	0.0522\\
-0.1037	0.0490\\
-0.1033	0.0531\\
-0.1010	0.0497\\
-0.1009	0.0440\\
-0.1016	0.0380\\
-0.1005	0.0371\\
-0.0995	0.0361\\
-0.1001	0.0326\\
-0.1007	0.0330\\
-0.0982	0.0304\\
-0.0965	0.0191\\
-0.0967	0.0153\\
-0.0995	0.0104\\
-0.0995	0.0095\\
-0.0972	0.0350\\
-0.0961	0.0585\\
-0.0950	0.0563\\
-0.0947	0.0125\\
-0.0968	0.0194\\
-0.0980	0.0225\\
-0.0973	0.0097\\
-0.0962	0.0161\\
-0.0962	0.0029\\
-0.0975	0.0186\\
-0.0979	0.0110\\
-0.0973	0.0188\\
-0.0949	0.0202\\
-0.0926	0.0114\\
-0.0919	0.0143\\
-0.0936	0.0120\\
-0.0952	0.0015\\
-0.0941	0.0184\\
-0.0931	0.0252\\
-0.0941	0.0285\\
-0.0925	0.0262\\
-0.0915	0.0015\\
-0.0939	0.0078\\
-0.0954	0.0265\\
-0.0938	0.0024\\
-0.0857	0.0022\\
-0.0776	0.0320\\
-0.0790	0.0215\\
-0.0860	0.0113\\
-0.0892	0.0095\\
-0.0880	0.0001\\
-0.0872	0.0008\\
-0.0862	0.0152\\
-0.0847	0.0110\\
-0.0836	0.0134\\
-0.0807	0.0142\\
-0.0784	0.0021\\
-0.0783	0.0062\\
-0.0789	0.0071\\
-0.0772	0.0020\\
-0.0750	0.0087\\
-0.0750	0.0228\\
-0.0753	0.0126\\
-0.0748	0.0038\\
-0.0739	0.0035\\
-0.0736	0.0093\\
-0.0727	0.0058\\
-0.0709	0.0045\\
-0.0691	0.0025\\
-0.0683	0.0004\\
-0.0667	0.0009\\
-0.0648	0.0048\\
-0.0629	0.0018\\
-0.0615	0.0004\\
-0.0615	0.0046\\
-0.0608	0.0000\\
-0.0579	0.0211\\
-0.0558	0.0452\\
-0.0529	0.0550\\
-0.0519	0.0392\\
-0.0522	0.0254\\
-0.0328	0.0396\\
-0.0092	0.0707\\
0.0006	0.0728\\
0.0016	0.0552\\
0.0029	0.0742\\
0.0041	0.0675\\
0.0067	0.0675\\
0.0077	0.0638\\
0.0093	0.0626\\
0.0107	0.0552\\
0.0109	0.0500\\
0.0109	0.0430\\
0.0148	0.0463\\
0.0166	0.0480\\
0.0170	0.0378\\
0.0171	0.0366\\
0.0172	0.0226\\
0.0186	0.0062\\
0.0199	0.0129\\
0.0213	0.0172\\
0.0254	0.0256\\
0.0258	0.0249\\
0.0278	0.0269\\
0.0286	0.0161\\
0.0290	0.0193\\
0.0301	0.0239\\
0.0308	0.0268\\
0.0307	0.0333\\
0.0307	0.0362\\
0.0318	0.0351\\
0.0328	0.0239\\
0.0332	0.0282\\
0.0354	0.0265\\
0.0383	0.0034\\
0.0378	0.0156\\
0.0415	0.0225\\
0.0415	0.0201\\
0.0430	0.0196\\
0.0454	0.0030\\
0.0469	0.0055\\
0.0467	0.0073\\
0.0474	0.0090\\
0.0489	0.0061\\
0.0501	0.0094\\
0.0503	0.0038\\
0.0506	0.0074\\
0.0506	0.0030\\
0.0511	0.0471\\
0.0524	0.0615\\
0.0551	0.0472\\
0.0574	0.0089\\
0.0581	0.0021\\
0.0579	0.0141\\
0.0576	0.0047\\
0.0576	0.0020\\
0.0578	0.0077\\
0.0591	0.0012\\
0.0610	0.0090\\
0.0619	0.0079\\
0.0610	0.0011\\
0.0609	0.0094\\
0.0614	0.0073\\
0.0626	0.0023\\
0.0635	0.0039\\
0.0640	0.0401\\
0.0641	0.0204\\
0.0643	0.0123\\
0.0654	0.0054\\
0.0669	0.0083\\
0.0673	0.0084\\
0.0677	0.0074\\
0.0674	0.0043\\
0.0674	0.0032\\
0.0676	0.0123\\
0.0711	0.0179\\
0.0730	0.0119\\
0.0735	0.0083\\
0.0739	0.0195\\
0.0745	0.0206\\
0.0754	0.0303\\
0.0759	0.0341\\
0.0760	0.0353\\
0.0752	0.0225\\
0.0756	0.0174\\
0.0776	0.0199\\
0.0789	0.0804\\
0.0795	0.0601\\
0.0794	0.0561\\
0.0808	0.0600\\
0.0816	0.0448\\
0.0808	0.0313\\
0.0810	0.0189\\
0.0814	0.0067\\
0.0823	0.0190\\
0.0822	0.0406\\
0.0823	0.0327\\
0.0828	0.0100\\
0.0846	0.0018\\
0.0862	0.0213\\
0.0863	0.0104\\
0.0855	0.0266\\
0.0856	0.0059\\
0.0886	0.0067\\
0.0913	0.0032\\
0.0906	0.0042\\
0.0883	0.0062\\
0.0877	0.0049\\
0.0882	0.0053\\
0.0882	0.0039\\
0.0888	0.0082\\
0.0894	0.0025\\
0.0912	0.0165\\
0.0936	0.0137\\
0.0950	0.0167\\
0.0949	0.0162\\
0.0940	0.0050\\
0.0934	0.0005\\
0.0944	0.0051\\
0.0963	0.0105\\
0.0972	0.0010\\
0.0970	0.0095\\
0.0960	0.0015\\
0.0954	0.0127\\
0.0955	0.0082\\
0.0956	0.0024\\
0.0958	0.0130\\
0.0966	0.0057\\
0.0975	0.0014\\
0.0981	0.0136\\
0.0984	0.0083\\
0.0997	0.0025\\
0.1009	0.0042\\
0.1008	0.0084\\
0.1013	0.0063\\
0.1035	0.0152\\
0.1057	0.0151\\
0.1058	0.0059\\
0.1044	0.0172\\
0.1022	0.0091\\
0.1010	0.0028\\
0.1009	0.0032\\
0.1024	0.0029\\
0.1043	0.0054\\
0.1050	0.0036\\
0.1032	0.0026\\
0.1025	0.0128\\
0.1019	0.0007\\
0.1017	0.0010\\
0.1018	0.0016\\
0.1026	0.0070\\
0.1037	0.0031\\
0.1044	0.0052\\
0.1048	0.0100\\
0.1075	0.0027\\
0.1100	0.0108\\
0.1101	0.0142\\
0.1085	0.0150\\
0.1091	0.0159\\
0.1107	0.0194\\
0.1105	0.0186\\
0.1092	0.0136\\
0.1090	0.0357\\
0.1096	0.0174\\
0.1101	0.0200\\
0.1105	0.0070\\
0.1106	0.0227\\
0.1112	0.0116\\
0.1140	0.0034\\
0.1165	0.0050\\
0.1161	0.0042\\
0.1147	0.0211\\
0.1149	0.0217\\
0.1171	0.0047\\
0.1194	0.0075\\
0.1201	0.0005\\
0.1194	0.0078\\
0.1189	0.0098\\
0.1204	0.0017\\
0.1217	0.0161\\
0.1209	0.0148\\
0.1200	0.0210\\
0.1196	0.0294\\
0.1199	0.0301\\
0.1211	0.0173\\
0.1241	0.0084\\
0.1272	0.0111\\
0.1276	0.0095\\
0.1255	0.0083\\
0.1271	0.0030\\
0.1300	0.0016\\
0.1303	0.0047\\
0.1289	0.0060\\
0.1303	0.0088\\
0.1325	0.0075\\
0.1315	0.0012\\
0.1290	0.0042\\
0.1284	0.0024\\
0.1307	0.0005\\
0.1325	0.0082\\
0.1347	0.0041\\
0.1335	0.0157\\
0.1308	0.0022\\
0.1297	0.0150\\
0.1314	0.0059\\
0.1344	0.0008\\
0.1353	0.0087\\
0.1326	0.0135\\
0.1306	0.0378\\
0.1299	0.0420\\
0.1310	0.0331\\
0.1331	0.0385\\
0.1340	0.0608\\
0.1325	0.0505\\
0.1326	0.0289\\
0.1350	0.0362\\
0.1365	0.0353\\
0.1373	0.0405\\
0.1342	0.0931\\
0.1318	0.0663\\
0.1327	0.0479\\
0.1331	0.0058\\
0.1299	0.0087\\
0.1287	0.0090\\
0.1276	0.0218\\
0.1277	0.0251\\
0.1270	0.0066\\
0.1283	0.0106\\
0.1315	0.0017\\
0.1325	0.0097\\
0.1289	0.0090\\
0.1264	0.0322\\
0.1261	0.0211\\
0.1263	0.0020\\
0.1260	0.0049\\
0.1276	0.0030\\
0.1306	0.0045\\
0.1319	0.0013\\
0.1292	0.0013\\
0.1273	0.0152\\
0.1273	0.0204\\
0.1298	0.0148\\
0.1309	0.0096\\
0.1316	0.0088\\
0.1291	0.0002\\
0.1272	0.0071\\
0.1270	0.0125\\
0.1301	0.0517\\
0.1320	0.0617\\
0.1291	0.0693\\
0.1249	0.0219\\
0.1240	0.0073\\
0.1251	0.0112\\
0.1287	0.0079\\
0.1305	0.0092\\
0.1284	0.0147\\
0.1266	0.0107\\
0.1281	0.0363\\
0.1287	0.0279\\
0.1274	0.0077\\
0.1252	0.0001\\
0.1260	0.0106\\
0.1264	0.0043\\
0.1242	0.0144\\
0.1244	0.0220\\
0.1252	0.0181\\
0.1246	0.0149\\
0.1217	0.0092\\
0.1221	0.0140\\
0.1233	0.0272\\
0.1210	0.0176\\
0.1187	0.0169\\
0.1198	0.0117\\
0.1202	0.0053\\
0.1189	0.0183\\
0.1157	0.0049\\
0.1145	0.0035\\
0.1157	0.0063\\
0.1169	0.0073\\
0.1166	0.0000\\
0.1138	0.0264\\
0.1120	0.0380\\
0.1113	0.0454\\
0.1107	0.0612\\
0.1089	0.0502\\
0.1090	0.0150\\
0.1108	0.0112\\
0.1108	0.0212\\
0.1086	0.0217\\
0.1059	0.0180\\
0.1054	0.0093\\
0.1051	0.0380\\
0.1046	0.0348\\
0.1024	0.0257\\
0.1021	0.0194\\
0.1021	0.0138\\
0.1000	0.0219\\
0.0988	0.0241\\
0.0991	0.0252\\
0.1000	0.0267\\
0.0983	0.0263\\
0.0950	0.0220\\
0.0930	0.0207\\
0.0924	0.0067\\
0.0918	0.0039\\
0.0913	0.0045\\
0.0907	0.0069\\
0.0904	0.0222\\
0.0898	0.0349\\
0.0888	0.0424\\
0.0882	0.0351\\
0.0881	0.0188\\
0.0882	0.0030\\
0.0882	0.0064\\
0.0857	0.0086\\
0.0849	0.0086\\
0.0838	0.0093\\
0.0839	0.0089\\
0.0817	0.0138\\
0.0797	0.0174\\
0.0787	0.0105\\
0.0793	0.0085\\
0.0794	0.0147\\
0.0773	0.0135\\
0.0748	0.0088\\
0.0734	0.0031\\
0.0733	0.0145\\
0.0735	0.0330\\
0.0739	0.0353\\
0.0741	0.0251\\
0.0734	0.0168\\
0.0719	0.0013\\
0.0715	0.0114\\
0.0709	0.0224\\
0.0717	0.0270\\
0.0746	0.0286\\
0.0759	0.0275\\
0.0748	0.0210\\
0.0718	0.0001\\
0.0700	0.0252\\
0.0695	0.0428\\
0.0707	0.0719\\
0.0726	0.0643\\
0.0723	0.0620\\
0.0694	0.0359\\
0.0684	0.0098\\
0.0666	0.0163\\
0.0613	0.0281\\
0.0586	0.0058\\
0.0604	0.0115\\
0.0665	0.0059\\
0.0699	0.0106\\
0.0697	0.0096\\
0.0694	0.0138\\
0.0693	0.0177\\
0.0692	0.0184\\
0.0683	0.0080\\
0.0675	0.0068\\
0.0654	0.0004\\
0.0655	0.0091\\
0.0665	0.0154\\
0.0661	0.0173\\
0.0645	0.0111\\
0.0632	0.0035\\
0.0633	0.0012\\
0.0641	0.0059\\
0.0651	0.0050\\
0.0643	0.0025\\
0.0637	0.0042\\
0.0648	0.0091\\
0.0652	0.0132\\
0.0632	0.0161\\
0.0615	0.0236\\
0.0608	0.0219\\
0.0603	0.0163\\
0.0602	0.0121\\
0.0602	0.0099\\
0.0609	0.0055\\
0.0612	0.0009\\
0.0604	0.0033\\
0.0599	0.0005\\
0.0606	0.0082\\
0.0614	0.0119\\
0.0609	0.0143\\
0.0608	0.0205\\
0.0609	0.0223\\
0.0619	0.0264\\
0.0634	0.0265\\
0.0638	0.0132\\
0.0625	0.0115\\
0.0617	0.0350\\
0.0616	0.0522\\
0.0618	0.0661\\
0.0626	0.0678\\
0.0615	0.0158\\
0.0619	0.0102\\
0.0639	0.0044\\
0.0663	0.0119\\
0.0676	0.0170\\
0.0674	0.0170\\
0.0676	0.0094\\
0.0691	0.0241\\
0.0698	0.0188\\
0.0696	0.0057\\
0.0685	0.0044\\
0.0678	0.0115\\
0.0683	0.0172\\
0.0694	0.0199\\
0.0713	0.0155\\
0.0722	0.0189\\
0.0715	0.0030\\
0.0707	0.0030\\
0.0698	0.0032\\
0.0697	0.0134\\
0.0700	0.0178\\
0.0709	0.0167\\
0.0711	0.0176\\
0.0705	0.0122\\
0.0708	0.0085\\
0.0722	0.0061\\
0.0731	0.0107\\
0.0731	0.0107\\
0.0734	0.0003\\
0.0742	0.0100\\
0.0751	0.0100\\
0.0755	0.0068\\
0.0756	0.0069\\
0.0759	0.0078\\
0.0776	0.0042\\
0.0802	0.0029\\
0.0816	0.0038\\
0.0808	0.0083\\
0.0805	0.0072\\
0.0807	0.0034\\
0.0814	0.0037\\
0.0820	0.0043\\
0.0820	0.0097\\
0.0812	0.0188\\
0.0807	0.0163\\
0.0808	0.0021\\
0.0828	0.0127\\
0.0856	0.0022\\
0.0870	0.0085\\
0.0866	0.0032\\
0.0852	0.0043\\
0.0847	0.0087\\
0.0849	0.0071\\
0.0842	0.0011\\
0.0852	0.0044\\
0.0877	0.0001\\
0.0890	0.0047\\
0.0863	0.0010\\
0.0846	0.0164\\
0.0846	0.0245\\
0.0853	0.0206\\
0.0856	0.0037\\
0.0858	0.0062\\
0.0854	0.0010\\
0.0848	0.0057\\
0.0838	0.0282\\
0.0838	0.0293\\
0.0846	0.0227\\
0.0853	0.0319\\
0.0852	0.0307\\
0.0855	0.0128\\
0.0878	0.0223\\
0.0901	0.0341\\
0.0896	0.0355\\
0.0874	0.0335\\
0.0861	0.0227\\
0.0874	0.0000\\
0.0901	0.0209\\
0.0914	0.0093\\
0.0910	0.0209\\
0.0894	0.0030\\
0.0890	0.0113\\
0.0893	0.0044\\
0.0893	0.0130\\
0.0893	0.0156\\
0.0895	0.0240\\
0.0903	0.0349\\
0.0915	0.0401\\
0.0923	0.0486\\
0.0921	0.0248\\
0.0920	0.0214\\
0.0922	0.0152\\
0.0930	0.0008\\
0.0932	0.0178\\
0.0930	0.0228\\
0.0969	0.0177\\
0.0965	0.0205\\
0.0951	0.0157\\
0.0949	0.0178\\
0.0949	0.0157\\
0.0945	0.0131\\
0.0941	0.0052\\
0.0940	0.0039\\
0.0949	0.0090\\
0.0955	0.0135\\
0.0957	0.0150\\
0.0952	0.0255\\
0.0987	0.0024\\
0.1000	0.0023\\
0.0994	0.0022\\
0.0987	0.0026\\
0.1000	0.0057\\
0.1011	0.0096\\
0.0999	0.0121\\
0.0990	0.0166\\
0.0985	0.0240\\
0.0994	0.0341\\
0.1011	0.0408\\
0.1030	0.0513\\
0.1065	0.0667\\
0.1061	0.0529\\
0.1036	0.0447\\
0.1032	0.0619\\
0.1041	0.0604\\
0.1042	0.0474\\
0.1058	0.0419\\
0.1076	0.0256\\
0.1107	0.0325\\
0.1104	0.0315\\
0.1099	0.0399\\
0.1121	0.0259\\
0.1132	0.0443\\
0.1106	0.0260\\
0.1087	0.0200\\
0.1088	0.0205\\
0.1104	0.0272\\
0.1149	0.0260\\
0.1172	0.0326\\
0.1135	0.0280\\
0.1061	0.0181\\
0.0991	0.0056\\
0.1041	0.0105\\
0.1158	0.0327\\
0.1232	0.0597\\
0.1245	0.0550\\
0.1250	0.0601\\
0.1246	0.0474\\
0.1234	0.0252\\
0.1216	0.0044\\
0.1199	0.0309\\
0.1179	0.0438\\
0.1168	0.0482\\
0.1164	0.0534\\
0.1164	0.0590\\
0.1165	0.0374\\
0.1167	0.0211\\
0.1166	0.0178\\
0.1165	0.0034\\
0.1160	0.0049\\
0.1157	0.0014\\
0.1154	0.0091\\
0.1156	0.0096\\
0.1163	0.0256\\
0.1179	0.0156\\
0.1195	0.0098\\
0.1186	0.0057\\
0.1171	0.0009\\
0.1143	0.0061\\
0.1134	0.0088\\
0.1144	0.0053\\
0.1156	0.0033\\
0.1163	0.0014\\
0.1156	0.0172\\
0.1147	0.0075\\
0.1147	0.0053\\
0.1152	0.0170\\
0.1156	0.0166\\
0.1158	0.0026\\
0.1156	0.0106\\
0.1148	0.0121\\
0.1149	0.0113\\
0.1167	0.0112\\
0.1202	0.0224\\
0.1223	0.0282\\
0.1215	0.0230\\
0.1193	0.0161\\
0.1181	0.0114\\
0.1187	0.0072\\
0.1195	0.0025\\
0.1219	0.0037\\
0.1250	0.0131\\
0.1265	0.0121\\
0.1252	0.0053\\
0.1243	0.0042\\
0.1254	0.0181\\
0.1261	0.0176\\
0.1253	0.0136\\
0.1254	0.0258\\
0.1286	0.0051\\
0.1304	0.0049\\
0.1278	0.0065\\
0.1256	0.0055\\
0.1254	0.0004\\
0.1275	0.0025\\
0.1312	0.0079\\
0.1337	0.0060\\
0.1374	0.0039\\
0.1394	0.0085\\
0.1391	0.0090\\
0.1380	0.0063\\
0.1369	0.0048\\
0.1379	0.0170\\
0.1412	0.0334\\
0.1465	0.0521\\
0.1485	0.0349\\
0.1497	0.0273\\
0.1464	0.0231\\
0.1456	0.0174\\
0.1479	0.0157\\
0.1496	0.0289\\
0.1507	0.0263\\
0.1556	0.0081\\
0.1606	0.0052\\
0.1616	0.0047\\
0.1601	0.0005\\
0.1600	0.0062\\
0.1611	0.0007\\
0.1626	0.0150\\
0.1653	0.0029\\
0.1690	0.0029\\
0.1741	0.0030\\
0.1782	0.0129\\
0.1803	0.0285\\
0.1804	0.0295\\
0.1812	0.0176\\
0.1834	0.0047\\
0.1867	0.0078\\
0.1887	0.0141\\
0.1896	0.0174\\
0.1896	0.0081\\
0.1910	0.0216\\
0.1945	0.0400\\
0.1987	0.0411\\
0.2012	0.0406\\
0.2029	0.0346\\
0.2038	0.0258\\
0.2065	0.0068\\
-0.2840	0.0198\\
-0.2844	0.0168\\
-0.2843	0.0058\\
-0.2848	0.0159\\
-0.2855	0.0097\\
-0.2850	0.0248\\
-0.2871	0.0294\\
-0.2960	0.0074\\
-0.3028	0.0027\\
-0.3002	0.0052\\
-0.2824	0.0401\\
-0.2575	0.0945\\
-0.2530	0.1017\\
-0.2758	0.0781\\
-0.2923	0.0369\\
-0.3084	0.0050\\
-0.3065	0.0285\\
-0.3047	0.0287\\
-0.3055	0.0177\\
-0.3024	0.0156\\
-0.2962	0.0195\\
-0.2977	0.0344\\
-0.3017	0.0431\\
-0.2998	0.0357\\
-0.2956	0.0243\\
-0.2890	0.0058\\
-0.2878	0.0339\\
-0.2850	0.0602\\
-0.2836	0.0504\\
-0.2825	0.0368\\
-0.2836	0.0303\\
-0.2885	0.0465\\
-0.2923	0.0291\\
-0.2889	0.0484\\
-0.2864	0.0301\\
-0.2889	0.0000\\
-0.2936	0.0196\\
-0.2909	0.0017\\
-0.2834	0.0157\\
-0.2807	0.0145\\
-0.2807	0.0105\\
-0.2809	0.0122\\
-0.2811	0.0082\\
-0.2815	0.0190\\
-0.2827	0.0131\\
-0.2832	0.0148\\
-0.2834	0.0172\\
-0.2829	0.0293\\
-0.2829	0.0360\\
-0.2828	0.0384\\
-0.2832	0.0196\\
-0.2831	0.0117\\
-0.2834	0.0186\\
-0.2825	0.0241\\
-0.2826	0.0225\\
-0.2827	0.0214\\
-0.2831	0.0272\\
-0.2868	0.0130\\
-0.2936	0.0146\\
-0.2944	0.0186\\
-0.2879	0.0137\\
-0.2831	0.0164\\
-0.2835	0.0371\\
-0.2832	0.0066\\
-0.2825	0.0002\\
-0.2813	0.0204\\
-0.2870	0.0356\\
-0.2887	0.0328\\
-0.2850	0.0122\\
-0.2786	0.0257\\
-0.2767	0.0325\\
-0.2743	0.0302\\
-0.2728	0.0276\\
-0.2707	0.0196\\
-0.2683	0.0152\\
-0.2661	0.0191\\
-0.2654	0.0115\\
};
\addplot [color=white!60!black,mark size=0.5pt,only marks,mark=*,mark options={solid}]
  table[row sep=crcr]{%
-0.2654	0.0115\\
-0.2652	0.0050\\
-0.2650	0.0012\\
-0.2670	0.0080\\
-0.2719	0.0002\\
-0.2714	0.0159\\
-0.2646	0.0307\\
-0.2591	0.0304\\
-0.2568	0.0419\\
-0.2549	0.0360\\
-0.2536	0.0193\\
-0.2550	0.0062\\
-0.2604	0.0028\\
-0.2616	0.0115\\
-0.2554	0.0139\\
-0.2491	0.0055\\
-0.2474	0.0041\\
-0.2498	0.0016\\
-0.2509	0.0114\\
-0.2503	0.0310\\
-0.2454	0.0405\\
-0.2405	0.0405\\
-0.2383	0.0265\\
-0.2364	0.0278\\
-0.2357	0.0167\\
-0.2345	0.0022\\
-0.2332	0.0013\\
-0.2320	0.0040\\
-0.2309	0.0047\\
-0.2297	0.0130\\
-0.2289	0.0098\\
-0.2281	0.0020\\
-0.2277	0.0145\\
-0.2304	0.0225\\
-0.2358	0.0314\\
-0.2359	0.0345\\
-0.2294	0.0425\\
-0.2243	0.0205\\
-0.2231	0.0123\\
-0.2220	0.0017\\
-0.2213	0.0022\\
-0.2230	0.0073\\
-0.2272	0.0121\\
-0.2283	0.0124\\
-0.2241	0.0162\\
-0.2197	0.0281\\
-0.2185	0.0283\\
-0.2214	0.0184\\
-0.2232	0.0103\\
-0.2243	0.0029\\
-0.2206	0.0040\\
-0.2163	0.0102\\
-0.2140	0.0117\\
-0.2130	0.0130\\
-0.2141	0.0185\\
-0.2175	0.0109\\
-0.2191	0.0154\\
-0.2162	0.0182\\
-0.2105	0.0205\\
-0.2087	0.0162\\
-0.2070	0.0016\\
-0.2057	0.0072\\
-0.2058	0.0031\\
-0.2102	0.0164\\
-0.2134	0.0169\\
-0.2106	0.0254\\
-0.2044	0.0068\\
-0.2013	0.0016\\
-0.1999	0.0010\\
-0.1994	0.0005\\
-0.1983	0.0011\\
-0.1969	0.0154\\
-0.1964	0.0237\\
-0.1991	0.0376\\
-0.2011	0.0221\\
-0.1983	0.0197\\
-0.1922	0.0180\\
-0.1886	0.0176\\
-0.1877	0.0194\\
-0.1860	0.0216\\
-0.1853	0.0172\\
-0.1846	0.0021\\
-0.1831	0.0158\\
-0.1813	0.0194\\
-0.1805	0.0244\\
-0.1826	0.0110\\
-0.1837	0.0174\\
-0.1811	0.0317\\
-0.1785	0.0268\\
-0.1744	0.0202\\
-0.1725	0.0181\\
-0.1709	0.0175\\
-0.1698	0.0063\\
-0.1686	0.0079\\
-0.1685	0.0077\\
-0.1707	0.0366\\
-0.1722	0.0514\\
-0.1701	0.0246\\
-0.1657	0.0313\\
-0.1642	0.0790\\
-0.1635	0.0501\\
-0.1664	0.0216\\
-0.1685	0.0262\\
-0.1663	0.0071\\
-0.1616	0.0215\\
-0.1592	0.0473\\
-0.1586	0.1126\\
-0.1587	0.0934\\
-0.1588	0.0713\\
-0.1580	0.0497\\
-0.1575	0.0246\\
-0.1598	0.0248\\
-0.1618	0.0256\\
-0.1599	0.0245\\
-0.1556	0.0289\\
-0.1534	0.0290\\
-0.1524	0.0154\\
-0.1517	0.0065\\
-0.1513	0.0181\\
-0.1502	0.0465\\
-0.1491	0.0415\\
-0.1485	0.0333\\
-0.1491	0.0225\\
-0.1510	0.0079\\
-0.1519	0.0142\\
-0.1498	0.0129\\
-0.1479	0.0225\\
-0.1485	0.0299\\
-0.1498	0.0544\\
-0.1484	0.0408\\
-0.1450	0.0568\\
-0.1434	0.0465\\
-0.1426	0.0273\\
-0.1419	0.0435\\
-0.1412	0.0241\\
-0.1409	0.0228\\
-0.1423	0.0204\\
-0.1439	0.0267\\
-0.1456	0.0407\\
-0.1441	0.0552\\
-0.1410	0.0459\\
-0.1398	0.0385\\
-0.1397	0.0356\\
-0.1395	0.0351\\
-0.1390	0.0467\\
-0.1391	0.0536\\
-0.1408	0.0416\\
-0.1438	0.0158\\
-0.1444	0.0018\\
-0.1432	0.0120\\
-0.1400	0.0087\\
-0.1385	0.0138\\
-0.1382	0.0128\\
-0.1385	0.0338\\
-0.1386	0.0624\\
-0.1382	0.0541\\
-0.1375	0.0531\\
-0.1372	0.0449\\
-0.1373	0.0236\\
-0.1372	0.0162\\
-0.1377	0.0096\\
-0.1404	0.0096\\
-0.1422	0.0220\\
-0.1404	0.0276\\
-0.1367	0.0345\\
-0.1353	0.0189\\
-0.1367	0.0094\\
-0.1382	0.0292\\
-0.1397	0.0310\\
-0.1382	0.0255\\
-0.1351	0.0151\\
-0.1336	0.0085\\
-0.1332	0.0253\\
-0.1355	0.0309\\
-0.1358	0.0261\\
-0.1355	0.0133\\
-0.1321	0.0265\\
-0.1305	0.0207\\
-0.1301	0.0039\\
-0.1300	0.0077\\
-0.1312	0.0136\\
-0.1337	0.0219\\
-0.1336	0.0313\\
-0.1304	0.0372\\
-0.1279	0.0299\\
-0.1274	0.0277\\
-0.1272	0.0253\\
-0.1270	0.0304\\
-0.1266	0.0300\\
-0.1259	0.0035\\
-0.1267	0.0196\\
-0.1290	0.0363\\
-0.1291	0.0248\\
-0.1254	0.0100\\
-0.1225	0.0629\\
-0.1215	0.0695\\
-0.1211	0.0362\\
-0.1204	0.0558\\
-0.1197	0.0484\\
-0.1193	0.0635\\
-0.1189	0.1091\\
-0.1182	0.0035\\
-0.1174	0.0047\\
-0.1168	0.0003\\
-0.1165	0.0117\\
-0.1160	0.0141\\
-0.1164	0.0058\\
-0.1181	0.0071\\
-0.1182	0.0093\\
-0.1162	0.0161\\
-0.1128	0.0044\\
-0.1112	0.0176\\
-0.1106	0.0096\\
-0.1100	0.0235\\
-0.1106	0.0532\\
-0.1122	0.0565\\
-0.1114	0.0635\\
-0.1096	0.0522\\
-0.1055	0.0418\\
-0.1035	0.0335\\
-0.1029	0.0421\\
-0.1024	0.0323\\
-0.1017	0.0350\\
-0.1011	0.0312\\
-0.1003	0.0394\\
-0.0996	0.0415\\
-0.0988	0.0390\\
-0.0980	0.0469\\
-0.0975	0.0380\\
-0.0965	0.0322\\
-0.0956	0.0430\\
-0.0949	0.0476\\
-0.0946	0.0664\\
-0.0946	0.0443\\
-0.0950	0.0457\\
-0.0948	0.0668\\
-0.0940	0.0559\\
-0.0935	0.0235\\
-0.0937	0.0079\\
-0.0930	0.0009\\
-0.0925	0.0512\\
-0.0930	0.0809\\
-0.0948	0.0638\\
-0.0950	0.0667\\
-0.0917	0.0789\\
-0.0878	0.0555\\
-0.0851	0.0346\\
-0.0862	0.0053\\
-0.0868	0.0199\\
-0.0792	0.0305\\
-0.0755	0.0332\\
-0.0755	0.0134\\
-0.0745	0.0237\\
-0.0675	0.0106\\
-0.0612	0.0079\\
-0.0580	0.0029\\
-0.0575	0.0132\\
-0.0577	0.0185\\
-0.0560	0.0136\\
-0.0521	0.0137\\
-0.0478	0.0186\\
-0.0412	0.0315\\
-0.0390	0.0326\\
-0.0383	0.0544\\
-0.0383	0.0790\\
-0.0381	0.0821\\
-0.0370	0.0823\\
-0.0359	0.0200\\
-0.0332	0.0094\\
-0.0242	0.0671\\
-0.0229	0.1079\\
-0.0214	0.1687\\
-0.0257	0.1114\\
0.0012	0.1266\\
0.0007	0.0696\\
0.0010	0.0668\\
0.0013	0.0757\\
0.0015	0.0816\\
0.0016	0.0558\\
0.0019	0.0519\\
0.0011	0.0504\\
-0.0004	0.0515\\
-0.0004	0.0488\\
0.0020	0.0517\\
0.0033	0.0300\\
0.0044	0.0198\\
0.0052	0.0224\\
0.0055	0.0221\\
0.0053	0.0210\\
0.0041	0.0039\\
0.0030	0.0048\\
0.0028	0.0067\\
0.0027	0.0061\\
0.0031	0.0510\\
0.0042	0.0570\\
0.0044	0.0106\\
0.0025	0.0642\\
-0.0005	0.0396\\
-0.0010	0.0048\\
-0.0005	0.0163\\
-0.0011	0.0358\\
-0.0024	0.1196\\
-0.0030	0.1269\\
-0.0019	0.1236\\
0.0010	0.0898\\
0.0013	0.0868\\
-0.0005	0.0705\\
0.0043	0.0205\\
0.0064	0.0042\\
0.0063	0.0069\\
0.0065	0.0109\\
0.0092	0.0072\\
0.0134	0.0042\\
0.0151	0.0050\\
0.0151	0.0227\\
0.0142	0.0175\\
0.0214	0.0531\\
0.0242	0.0564\\
0.0250	0.0716\\
0.0250	0.0660\\
0.0283	0.0509\\
0.0304	0.0450\\
0.0332	0.0279\\
0.0447	0.0122\\
0.0455	0.0079\\
0.0464	0.0063\\
0.0480	0.0060\\
0.0515	0.0078\\
0.0527	0.0004\\
0.0542	0.0553\\
0.0551	0.0877\\
0.0562	0.0970\\
0.0577	0.0627\\
0.0600	0.0525\\
0.0610	0.0449\\
0.0600	0.0506\\
0.0645	0.0460\\
0.0667	0.0406\\
0.0673	0.0380\\
0.0670	0.0374\\
0.0671	0.0465\\
0.0680	0.0484\\
0.0693	0.0435\\
0.0703	0.0303\\
0.0724	0.0246\\
0.0742	0.0143\\
0.0757	0.0035\\
0.0773	0.0114\\
0.0793	0.0081\\
0.0805	0.0016\\
0.0803	0.0018\\
0.0801	0.0259\\
0.0794	0.0557\\
0.0796	0.0406\\
0.0810	0.0372\\
0.0832	0.0442\\
0.0859	0.0375\\
0.0890	0.0048\\
0.0886	0.0770\\
0.0878	0.0241\\
0.0889	0.0292\\
0.0905	0.0135\\
0.0930	0.0098\\
0.0957	0.0312\\
0.0973	0.0463\\
0.0970	0.0548\\
0.0967	0.0781\\
0.0969	0.0875\\
0.0975	0.0899\\
0.1039	0.0770\\
0.1057	0.0889\\
0.1041	0.0842\\
0.1031	0.0521\\
0.1042	0.0353\\
0.1079	0.0344\\
0.1094	0.0247\\
0.1113	0.0159\\
0.1099	0.0148\\
0.1088	0.0209\\
0.1096	0.0226\\
0.1130	0.0180\\
0.1171	0.0197\\
0.1192	0.0340\\
0.1193	0.0184\\
0.1183	0.0136\\
0.1169	0.0161\\
0.1160	0.0146\\
0.1197	0.0151\\
0.1232	0.0209\\
0.1188	0.0243\\
0.1092	0.0361\\
0.1135	0.0625\\
0.1193	0.0909\\
0.1317	0.0877\\
0.1348	0.0849\\
0.1363	0.0767\\
0.1384	0.0593\\
0.1400	0.0386\\
0.1407	0.0248\\
0.1421	0.0205\\
0.1423	0.0133\\
0.1420	0.0044\\
0.1396	0.0040\\
0.1389	0.0009\\
0.1403	0.0044\\
0.1444	0.0063\\
0.1474	0.0032\\
0.1473	0.0146\\
0.1478	0.0320\\
0.1495	0.0443\\
0.1529	0.0504\\
0.1529	0.0430\\
0.1507	0.0060\\
0.1497	0.0037\\
0.1528	0.0048\\
0.1579	0.0097\\
0.1602	0.0490\\
0.1590	0.0326\\
0.1560	0.0141\\
0.1555	0.0203\\
0.1569	0.0185\\
0.1577	0.0282\\
0.1608	0.0158\\
0.1656	0.0060\\
0.1677	0.0001\\
0.1648	0.0041\\
0.1637	0.0255\\
0.1630	0.0184\\
0.1664	0.0050\\
0.1713	0.0012\\
0.1730	0.0002\\
0.1705	0.0127\\
0.1733	0.0101\\
0.1772	0.0123\\
0.1774	0.0084\\
0.1742	0.0161\\
0.1761	0.0183\\
0.1800	0.0155\\
0.1794	0.0139\\
0.1756	0.0294\\
0.1737	0.0336\\
0.1766	0.0334\\
0.1817	0.0264\\
0.1837	0.0206\\
0.1827	0.0123\\
0.1792	0.0195\\
0.1784	0.0245\\
0.1822	0.0218\\
0.1878	0.0090\\
0.1904	0.0109\\
0.1885	0.0151\\
0.1893	0.0181\\
0.1929	0.0086\\
0.1949	0.0175\\
0.1912	0.0193\\
0.1894	0.0275\\
0.1905	0.0139\\
0.1927	0.0151\\
0.1943	0.0131\\
0.1954	0.0060\\
0.1958	0.0030\\
0.1972	0.0012\\
0.1981	0.0101\\
0.2031	0.0017\\
0.2094	0.0099\\
0.2123	0.0032\\
0.2091	0.0087\\
0.2072	0.0047\\
0.2075	0.0073\\
0.2088	0.0138\\
0.2093	0.0233\\
0.2136	0.0285\\
0.2198	0.0164\\
0.2228	0.0194\\
0.2203	0.0239\\
0.2228	0.0215\\
0.2274	0.0159\\
0.2273	0.0046\\
0.2249	0.0148\\
0.2245	0.0307\\
0.2263	0.0274\\
0.2335	0.0020\\
0.2395	0.0102\\
0.2386	0.0025\\
0.2350	0.0165\\
0.2343	0.0186\\
0.2392	0.0085\\
0.2459	0.0112\\
0.2485	0.0011\\
0.2481	0.0037\\
0.2451	0.0154\\
0.2439	0.0037\\
0.2488	0.0015\\
0.2560	0.0123\\
0.2599	0.0126\\
0.2574	0.0051\\
0.2586	0.0160\\
0.2634	0.0181\\
0.2664	0.0348\\
0.2627	0.0712\\
0.2662	0.0708\\
0.2746	0.0485\\
0.2793	0.0611\\
0.2790	0.0187\\
0.2748	0.0137\\
0.2711	0.0145\\
0.2699	0.0050\\
0.2717	0.0168\\
0.2725	0.0173\\
0.2746	0.0174\\
0.2770	0.0113\\
0.2802	0.0036\\
0.2819	0.0008\\
0.2833	0.0017\\
0.2837	0.0036\\
0.2842	0.0038\\
0.2861	0.0158\\
0.2880	0.0435\\
0.2911	0.0461\\
0.2937	0.0326\\
0.2960	0.0248\\
0.2985	0.0112\\
0.3078	0.0069\\
0.3162	0.0015\\
0.3171	0.0098\\
0.3121	0.0060\\
0.3099	0.0174\\
0.3118	0.0349\\
0.3151	0.0602\\
0.3190	0.0542\\
0.3295	0.0244\\
0.3380	0.0052\\
0.3367	0.0033\\
0.3317	0.0003\\
0.3307	0.0075\\
0.3322	0.0137\\
0.3358	0.0141\\
0.3385	0.0258\\
0.3421	0.0179\\
0.3464	0.0106\\
0.3581	0.0004\\
0.3675	0.0169\\
0.3658	0.0209\\
0.3618	0.0144\\
0.3586	0.0048\\
0.3616	0.0026\\
0.3746	0.0195\\
0.3852	0.0191\\
0.3832	0.0144\\
0.3769	0.0016\\
0.3765	0.0030\\
0.3867	0.0076\\
0.3937	0.0078\\
0.4044	0.0086\\
0.4035	0.0058\\
0.4028	0.0034\\
0.4092	0.0050\\
0.4152	0.0151\\
0.4130	0.0178\\
0.4205	0.0315\\
0.4309	0.0481\\
0.4326	0.0359\\
0.4287	0.0209\\
0.4277	0.0134\\
0.4318	0.0204\\
0.4467	0.0333\\
0.4595	0.0401\\
0.4594	0.0219\\
0.4550	0.0607\\
0.4543	0.0450\\
0.4587	0.0172\\
0.4748	0.0217\\
0.4901	0.0426\\
0.4913	0.0290\\
0.4942	0.0235\\
0.5047	0.0368\\
0.5148	0.0405\\
0.5147	0.0308\\
0.5127	0.0120\\
0.5140	0.0048\\
0.5281	0.0117\\
0.5458	0.0406\\
0.5567	0.0462\\
0.5536	0.0319\\
0.5691	0.0196\\
0.5794	0.0254\\
0.6018	0.0299\\
0.6050	0.0319\\
0.5964	0.0060\\
0.6006	0.0263\\
0.6166	0.0505\\
0.6301	0.0442\\
0.6281	0.0130\\
0.6338	0.0111\\
0.6538	0.0018\\
-0.3668	0.0601\\
-0.3576	0.0439\\
-0.3487	0.0398\\
-0.3456	0.0523\\
-0.3492	0.0390\\
-0.3517	0.0271\\
-0.3505	0.0013\\
-0.3423	0.0083\\
-0.3363	0.0210\\
-0.3367	0.0216\\
-0.3410	0.0164\\
-0.3417	0.0009\\
-0.3365	0.0135\\
-0.3276	0.0398\\
-0.3253	0.0480\\
-0.3253	0.0100\\
-0.3312	0.0057\\
-0.3331	0.0074\\
-0.3265	0.0098\\
-0.3152	0.0129\\
-0.3106	0.0224\\
-0.3128	0.0236\\
-0.3179	0.0125\\
-0.3180	0.0196\\
-0.3138	0.0285\\
-0.3058	0.0407\\
-0.3021	0.0223\\
-0.3042	0.0001\\
-0.3098	0.0342\\
-0.3100	0.0297\\
-0.3055	0.0064\\
-0.3054	0.0152\\
-0.3112	0.0171\\
-0.3125	0.0088\\
-0.3061	0.0108\\
-0.3063	0.0062\\
-0.3094	0.0066\\
-0.3070	0.0049\\
-0.3037	0.0038\\
-0.3057	0.0085\\
-0.3081	0.0118\\
-0.3035	0.0212\\
-0.2988	0.0195\\
-0.3029	0.0073\\
-0.3038	0.0231\\
-0.2971	0.0398\\
-0.2911	0.0237\\
-0.2899	0.0046\\
-0.2895	0.0037\\
-0.2906	0.0245\\
-0.2956	0.0582\\
-0.3034	0.0526\\
-0.3036	0.0551\\
-0.2965	0.0511\\
-0.2976	0.0258\\
-0.3009	0.0073\\
-0.3073	0.0081\\
-0.3036	0.0048\\
-0.2953	0.0274\\
-0.2878	0.0513\\
-0.2859	0.0490\\
-0.2904	0.0290\\
-0.2975	0.0168\\
-0.2972	0.0014\\
-0.2938	0.0066\\
-0.2859	0.0358\\
-0.2843	0.0511\\
-0.2887	0.0470\\
-0.2949	0.0347\\
-0.2950	0.0373\\
-0.2889	0.0372\\
-0.2837	0.0586\\
-0.2826	0.0708\\
-0.2856	0.0622\\
-0.2925	0.0419\\
-0.2937	0.0275\\
-0.2875	0.0323\\
-0.2820	0.0219\\
-0.2805	0.0211\\
-0.2796	0.0220\\
-0.2809	0.0247\\
-0.2830	0.0375\\
-0.2898	0.0532\\
-0.2904	0.0406\\
-0.2844	0.0278\\
-0.2844	0.0099\\
-0.2874	0.0156\\
-0.2843	0.0323\\
-0.2806	0.0114\\
-0.2821	0.0126\\
-0.2838	0.0120\\
-0.2800	0.0140\\
-0.2723	0.0402\\
-0.2691	0.0416\\
-0.2684	0.0328\\
-0.2680	0.0275\\
-0.2702	0.0283\\
-0.2750	0.0382\\
-0.2757	0.0740\\
-0.2720	0.0587\\
-0.2673	0.0362\\
-0.2682	0.0377\\
-0.2674	0.0479\\
-0.2625	0.0800\\
-0.2626	0.0782\\
-0.2635	0.0574\\
-0.2633	0.0503\\
-0.2570	0.0554\\
-0.2513	0.0436\\
-0.2488	0.0264\\
-0.2468	0.0141\\
-0.2464	0.0175\\
-0.2497	0.0114\\
-0.2514	0.0241\\
-0.2503	0.0192\\
-0.2448	0.0111\\
-0.2439	0.0110\\
-0.2446	0.0187\\
-0.2401	0.0650\\
-0.2353	0.0772\\
-0.2362	0.0711\\
-0.2353	0.0610\\
-0.2325	0.0486\\
-0.2283	0.0328\\
-0.2276	0.0308\\
-0.2159	0.0354\\
-0.1957	0.0475\\
-0.1991	0.0161\\
-0.2191	0.0011\\
-0.2298	0.0194\\
-0.2276	0.0329\\
-0.2240	0.0114\\
-0.2228	0.0172\\
-0.2212	0.0011\\
-0.2195	0.0016\\
-0.2145	0.0214\\
-0.2077	0.0243\\
-0.2032	0.0353\\
-0.2030	0.0361\\
-0.2056	0.0246\\
-0.2067	0.0088\\
-0.2059	0.0098\\
-0.2009	0.0152\\
-0.1964	0.0166\\
-0.1940	0.0191\\
-0.1916	0.0372\\
-0.1905	0.0442\\
-0.1932	0.0373\\
-0.1947	0.0234\\
-0.1910	0.0234\\
-0.1876	0.0165\\
-0.1834	0.0078\\
-0.1830	0.0040\\
-0.1821	0.0049\\
-0.1809	0.0228\\
-0.1835	0.0370\\
-0.1872	0.0404\\
-0.1876	0.0551\\
-0.1853	0.0443\\
-0.1794	0.0038\\
-0.1738	0.0169\\
-0.1711	0.0217\\
-0.1698	0.0168\\
-0.1687	0.0165\\
-0.1674	0.0069\\
-0.1667	0.0138\\
-0.1661	0.0043\\
-0.1691	0.0187\\
-0.1696	0.0096\\
-0.1690	0.0110\\
-0.1644	0.0089\\
-0.1624	0.0380\\
-0.1641	0.0305\\
-0.1672	0.0174\\
-0.1671	0.0310\\
-0.1632	0.0596\\
-0.1612	0.0651\\
-0.1591	0.0486\\
-0.1584	0.0224\\
-0.1579	0.0139\\
-0.1576	0.0163\\
-0.1578	0.0025\\
-0.1599	0.0343\\
-0.1637	0.0448\\
-0.1644	0.0196\\
-0.1607	0.0194\\
-0.1578	0.0002\\
-0.1572	0.0212\\
-0.1567	0.0540\\
-0.1611	0.0266\\
-0.1633	0.0044\\
-0.1613	0.0125\\
-0.1582	0.0080\\
-0.1583	0.0031\\
-0.1613	0.0033\\
-0.1633	0.0112\\
-0.1631	0.0022\\
-0.1610	0.0090\\
-0.1634	0.0064\\
-0.1661	0.0069\\
-0.1638	0.0037\\
-0.1588	0.0033\\
-0.1573	0.0025\\
-0.1597	0.0057\\
-0.1617	0.0200\\
-0.1640	0.0144\\
-0.1627	0.0038\\
-0.1586	0.0065\\
-0.1560	0.0146\\
-0.1550	0.0194\\
-0.1557	0.0296\\
-0.1570	0.0048\\
-0.1578	0.0223\\
-0.1575	0.0353\\
-0.1573	0.0043\\
-0.1572	0.0090\\
-0.1605	0.0141\\
-0.1633	0.0380\\
-0.1617	0.0594\\
-0.1579	0.0394\\
-0.1571	0.0327\\
-0.1592	0.0376\\
-0.1622	0.0345\\
-0.1627	0.0330\\
-0.1613	0.0349\\
-0.1577	0.0401\\
-0.1556	0.0571\\
-0.1571	0.0377\\
-0.1615	0.0160\\
-0.1628	0.0141\\
-0.1586	0.0198\\
-0.1556	0.0214\\
-0.1552	0.0225\\
-0.1548	0.0108\\
-0.1537	0.0330\\
-0.1526	0.0594\\
-0.1526	0.0706\\
-0.1529	0.0597\\
-0.1527	0.0484\\
-0.1518	0.0397\\
-0.1511	0.0334\\
-0.1503	0.0201\\
-0.1501	0.0179\\
-0.1513	0.0435\\
-0.1528	0.0568\\
-0.1540	0.0336\\
-0.1521	0.0464\\
-0.1503	0.0389\\
-0.1516	0.0224\\
-0.1515	0.0010\\
-0.1478	0.0172\\
-0.1446	0.0315\\
-0.1433	0.0216\\
-0.1428	0.0074\\
-0.1423	0.0033\\
-0.1418	0.0159\\
-0.1431	0.0182\\
-0.1426	0.0198\\
-0.1397	0.0129\\
-0.1371	0.0084\\
-0.1369	0.0122\\
-0.1376	0.0169\\
-0.1394	0.0041\\
-0.1379	0.0082\\
-0.1327	0.0028\\
-0.1285	0.0015\\
-0.1266	0.0021\\
-0.1251	0.0015\\
-0.1242	0.0024\\
-0.1247	0.0171\\
-0.1247	0.0029\\
-0.1238	0.0033\\
-0.1200	0.0131\\
-0.1168	0.0032\\
-0.1148	0.0092\\
-0.1125	0.0006\\
-0.1106	0.0252\\
-0.1122	0.0405\\
-0.1133	0.0555\\
-0.1105	0.0747\\
-0.1057	0.0645\\
-0.1043	0.0555\\
-0.1034	0.0345\\
-0.1029	0.0319\\
-0.1021	0.0240\\
-0.1025	0.0218\\
-0.1031	0.0153\\
-0.1019	0.0152\\
-0.1002	0.0101\\
-0.1001	0.0155\\
-0.0992	0.0241\\
-0.0964	0.0255\\
-0.0930	0.0236\\
-0.0910	0.0278\\
-0.0902	0.0333\\
-0.0896	0.0424\\
-0.0888	0.0347\\
-0.0877	0.0168\\
-0.0875	0.0179\\
-0.0881	0.0219\\
-0.0885	0.0359\\
-0.0868	0.0316\\
-0.0847	0.0342\\
-0.0843	0.0230\\
-0.0829	0.0153\\
-0.0793	0.0243\\
-0.0762	0.0275\\
-0.0751	0.0309\\
-0.0731	0.0484\\
-0.0708	0.0646\\
-0.0678	0.1575\\
-0.0646	0.1215\\
-0.0611	0.0616\\
-0.0578	0.0548\\
-0.0549	0.0659\\
-0.0524	0.0848\\
-0.0471	0.1133\\
-0.0426	0.1085\\
-0.0404	0.0971\\
-0.0360	0.0878\\
-0.0315	0.0780\\
-0.0269	0.0714\\
-0.0224	0.0667\\
-0.0183	0.0408\\
-0.0145	0.0453\\
-0.0126	0.0454\\
-0.0084	0.0393\\
-0.0041	0.0188\\
-0.0003	0.0206\\
0.0038	0.0273\\
0.0083	0.0162\\
0.0122	0.0149\\
0.0151	0.0202\\
0.0183	0.0207\\
0.0201	0.0207\\
0.0237	0.0200\\
0.0271	0.0354\\
0.0308	0.0105\\
0.0350	0.0204\\
0.0384	0.0196\\
0.0419	0.0155\\
0.0450	0.0159\\
0.0484	0.0052\\
0.0517	0.0298\\
0.0549	0.0308\\
0.0565	0.0356\\
0.0597	0.0370\\
0.0624	0.0330\\
0.0652	0.0245\\
0.0700	0.0179\\
0.0751	0.0233\\
0.0781	0.0430\\
0.0801	0.0483\\
0.0807	0.0293\\
0.0830	0.0381\\
0.0860	0.0434\\
0.0895	0.0455\\
0.0923	0.0266\\
0.0964	0.0187\\
0.1019	0.0207\\
0.1054	0.0126\\
0.1062	0.0328\\
0.1110	0.0489\\
0.1137	0.0535\\
0.1175	0.0607\\
0.1169	0.0495\\
0.1198	0.0428\\
0.1242	0.0309\\
0.1254	0.0233\\
0.1249	0.0602\\
0.1264	0.0684\\
0.1277	0.0759\\
0.1299	0.0779\\
0.1316	0.0638\\
0.1347	0.0479\\
0.1393	0.0949\\
0.1458	0.0937\\
0.1515	0.0473\\
0.1541	0.0615\\
0.1548	0.0517\\
0.1556	0.0111\\
0.1574	0.0038\\
0.1583	0.0063\\
0.1617	0.0252\\
0.1695	0.0150\\
0.1764	0.0132\\
0.1769	0.0399\\
0.1751	0.0561\\
0.1766	0.0486\\
0.1828	0.0228\\
0.1867	0.0212\\
0.1920	0.0033\\
0.1919	0.0175\\
0.1906	0.0136\\
0.1930	0.0090\\
0.2013	0.0139\\
0.2113	0.0099\\
0.2174	0.0036\\
0.2161	0.0014\\
0.2157	0.0208\\
0.2159	0.0231\\
0.2190	0.0351\\
0.2228	0.0297\\
0.2278	0.0170\\
0.2322	0.0176\\
0.2369	0.0029\\
0.2412	0.0126\\
0.2459	0.0276\\
0.2500	0.0157\\
0.2544	0.0005\\
0.2579	0.0007\\
0.2625	0.0167\\
0.2649	0.0133\\
0.2705	0.0119\\
0.2755	0.0089\\
0.2788	0.0107\\
0.2833	0.0238\\
0.2879	0.0170\\
0.2925	0.0201\\
0.2975	0.0243\\
0.3100	0.0345\\
0.3214	0.0298\\
0.3234	0.0135\\
0.3203	0.0264\\
0.3200	0.0401\\
0.3235	0.0471\\
0.3290	0.0604\\
0.3336	0.0390\\
0.3393	0.0350\\
0.3459	0.0110\\
0.3589	0.0058\\
0.3700	0.0095\\
0.3721	0.0064\\
0.3693	0.0406\\
0.3689	0.0424\\
0.3723	0.0401\\
0.3784	0.0228\\
0.3852	0.0043\\
0.3994	0.0394\\
0.4114	0.0664\\
0.4124	0.0694\\
0.4099	0.0353\\
0.4098	0.0201\\
0.4152	0.0141\\
0.4304	0.0109\\
0.4441	0.0068\\
0.4452	0.0158\\
0.4420	0.0069\\
0.4424	0.0051\\
0.4486	0.0009\\
0.4512	0.0364\\
0.4567	0.0413\\
0.4638	0.0260\\
0.4714	0.0133\\
0.4900	0.0034\\
0.5064	0.0322\\
0.5074	0.0274\\
0.5035	0.0056\\
0.5042	0.0190\\
0.5129	0.0022\\
0.5162	0.0206\\
0.5230	0.0179\\
0.5312	0.0101\\
0.5387	0.0348\\
0.5596	0.0425\\
0.5790	0.0112\\
0.5801	0.0574\\
0.5758	0.0752\\
0.5765	0.0674\\
0.5863	0.0217\\
0.5942	0.0059\\
0.6042	0.0051\\
0.6079	0.0075\\
0.6146	0.0243\\
0.6245	0.0183\\
0.6312	0.0135\\
0.6418	0.0057\\
0.6522	0.0428\\
0.6784	0.0186\\
0.7013	0.0010\\
0.7021	0.0128\\
0.6977	0.0356\\
0.6983	0.0594\\
0.7060	0.0788\\
0.7193	0.0657\\
0.7280	0.0259\\
0.7412	0.0002\\
0.7525	0.0386\\
0.7692	0.0485\\
0.7801	0.0433\\
0.7968	0.0024\\
0.8112	0.0156\\
0.8481	0.0241\\
0.8628	0.0043\\
0.8865	0.0088\\
0.8833	0.0209\\
0.9066	0.0656\\
0.9366	0.0827\\
0.9407	0.0359\\
0.9529	0.0037\\
0.9668	0.0077\\
1.0044	0.0073\\
1.0142	0.0211\\
1.0175	0.0478\\
1.0215	0.0414\\
1.0438	0.0616\\
1.0614	0.0665\\
-0.9255	0.0006\\
-0.9132	0.0187\\
-0.9027	0.0297\\
-0.8998	0.0102\\
-0.8928	0.0018\\
-0.8916	0.0304\\
-0.8840	0.0337\\
-0.8810	0.0301\\
-0.8697	0.0523\\
-0.8651	0.0522\\
-0.8540	0.0495\\
-0.8506	0.0411\\
-0.8495	0.0102\\
-0.8403	0.0092\\
-0.8367	0.0029\\
-0.8268	0.0330\\
-0.8232	0.0329\\
-0.8139	0.0123\\
-0.8133	0.0239\\
-0.8200	0.0284\\
-0.8210	0.0042\\
-0.8180	0.0297\\
-0.7964	0.0232\\
-0.7735	0.0272\\
-0.7675	0.0347\\
-0.7715	0.1164\\
-0.7760	0.1573\\
-0.7719	0.1648\\
-0.7432	0.1620\\
-0.7114	0.1300\\
-0.6944	0.1041\\
-0.6773	0.0652\\
-0.6629	0.0396\\
-0.6509	0.0498\\
-0.6363	0.0508\\
-0.6268	0.0278\\
-0.6140	0.0386\\
-0.6034	0.0667\\
-0.5978	0.0699\\
-0.6016	0.0775\\
-0.5986	0.0844\\
-0.5844	0.0638\\
-0.5667	0.0299\\
-0.5637	0.0018\\
-0.5553	0.0069\\
-0.5368	0.0214\\
-0.5187	0.0245\\
-0.5092	0.0250\\
-0.4997	0.0200\\
-0.4962	0.0008\\
-0.4874	0.0129\\
-0.4809	0.0372\\
-0.4721	0.0666\\
-0.4652	0.0676\\
-0.4565	0.0751\\
-0.4496	0.0744\\
-0.4414	0.0555\\
-0.4351	0.0909\\
-0.4336	0.0840\\
-0.4379	0.0634\\
-0.4366	0.0423\\
-0.4273	0.0347\\
-0.4111	0.0447\\
-0.4017	0.0328\\
-0.4005	0.0160\\
-0.4051	0.0054\\
-0.4020	0.0043\\
-0.3887	0.0197\\
-0.3766	0.0144\\
-0.3729	0.0162\\
-0.3722	0.0069\\
-0.3767	0.0058\\
-0.3739	0.0191\\
-0.3622	0.0089\\
-0.3512	0.0025\\
-0.3458	0.0134\\
-0.3410	0.0271\\
-0.3388	0.0333\\
-0.3413	0.0474\\
-0.3425	0.0380\\
-0.3392	0.0407\\
-0.3282	0.0492\\
-0.3186	0.0515\\
-0.3157	0.0578\\
-0.3179	0.0357\\
-0.3182	0.0217\\
-0.3111	0.0364\\
-0.3053	0.0140\\
-0.2961	0.0067\\
-0.2931	0.0024\\
-0.2945	0.0120\\
-0.2936	0.0152\\
-0.2865	0.0243\\
-0.2763	0.0059\\
-0.2710	0.0183\\
-0.2682	0.0428\\
-0.2658	0.0630\\
-0.2642	0.0616\\
-0.2608	0.1057\\
-0.2587	0.0071\\
-0.2601	0.0114\\
-0.2604	0.0751\\
-0.2556	0.1190\\
-0.2477	0.1143\\
-0.2431	0.0778\\
-0.2433	0.0098\\
-0.2452	0.0135\\
-0.2461	0.0081\\
-0.2404	0.0523\\
-0.2313	0.0779\\
-0.2266	0.0847\\
-0.2236	0.0761\\
-0.2200	0.0760\\
-0.2160	0.0664\\
-0.2130	0.0568\\
-0.2102	0.0518\\
-0.2080	0.0328\\
-0.2076	0.0323\\
-0.2090	0.0324\\
-0.2089	0.0309\\
-0.2041	0.0336\\
-0.1967	0.0526\\
-0.1925	0.0395\\
-0.1900	0.0376\\
-0.1874	0.0415\\
-0.1843	0.0393\\
-0.1817	0.0164\\
-0.1805	0.0067\\
-0.1775	0.0334\\
-0.1745	0.0474\\
-0.1717	0.0333\\
-0.1695	0.0158\\
-0.1671	0.0288\\
-0.1655	0.0278\\
-0.1659	0.0295\\
-0.1655	0.0311\\
-0.1619	0.0309\\
-0.1563	0.0398\\
-0.1542	0.0316\\
-0.1518	0.0347\\
-0.1523	0.0301\\
-0.1522	0.0300\\
-0.1485	0.0315\\
-0.1430	0.0414\\
-0.1397	0.0501\\
-0.1388	0.0574\\
-0.1388	0.0539\\
-0.1379	0.0590\\
-0.1340	0.0572\\
-0.1293	0.0306\\
-0.1271	0.0182\\
-0.1268	0.0239\\
-0.1242	0.0242\\
-0.1203	0.0202\\
-0.1167	0.0212\\
-0.1164	0.0271\\
-0.1170	0.0283\\
-0.1168	0.0233\\
-0.1140	0.0292\\
-0.1096	0.0391\\
-0.1066	0.0331\\
-0.1053	0.0372\\
-0.1045	0.0363\\
-0.1024	0.0260\\
-0.1009	0.0224\\
-0.0984	0.0310\\
-0.0968	0.0467\\
-0.0949	0.0373\\
-0.0932	0.0232\\
-0.0937	0.0117\\
-0.0944	0.0001\\
-0.0927	0.0007\\
-0.0895	0.0008\\
-0.0876	0.0018\\
-0.0874	0.0059\\
-0.0872	0.0248\\
-0.0875	0.0511\\
-0.0869	0.0556\\
-0.0854	0.0482\\
-0.0831	0.0346\\
-0.0824	0.0352\\
-0.0833	0.0328\\
-0.0831	0.0348\\
-0.0818	0.0333\\
-0.0789	0.0380\\
-0.0772	0.0205\\
-0.0764	0.0052\\
-0.0757	0.0200\\
-0.0748	0.0039\\
-0.0738	0.0517\\
-0.0730	0.0538\\
-0.0722	0.0521\\
-0.0710	0.0558\\
-0.0703	0.0645\\
-0.0686	0.0567\\
-0.0672	0.0629\\
-0.0657	0.0876\\
-0.0641	0.0862\\
-0.0637	0.0640\\
-0.0644	0.0603\\
-0.0644	0.0523\\
-0.0626	0.0480\\
-0.0616	0.0437\\
-0.0598	0.0432\\
-0.0587	0.0330\\
-0.0581	0.0222\\
-0.0574	0.0273\\
-0.0561	0.0209\\
-0.0547	0.0232\\
-0.0540	0.0262\\
-0.0540	0.0336\\
-0.0544	0.0397\\
-0.0542	0.0344\\
-0.0536	0.0194\\
-0.0524	0.0158\\
-0.0494	0.0307\\
-0.0474	0.0307\\
-0.0466	0.0359\\
-0.0455	0.0320\\
-0.0431	0.0353\\
-0.0414	0.0618\\
-0.0386	0.0607\\
-0.0355	0.0630\\
-0.0330	0.0591\\
-0.0306	0.0225\\
-0.0278	0.0021\\
-0.0238	0.0028\\
-0.0164	0.0038\\
-0.0153	0.0109\\
-0.0107	0.0153\\
-0.0066	0.0348\\
-0.0030	0.0438\\
0.0006	0.0206\\
0.0043	0.0244\\
0.0080	0.0353\\
0.0116	0.0372\\
0.0152	0.0392\\
0.0194	0.0434\\
0.0214	0.0361\\
0.0251	0.0429\\
0.0280	0.0430\\
0.0305	0.0260\\
0.0333	0.0103\\
0.0375	0.0093\\
0.0417	0.0131\\
0.0441	0.0069\\
0.0447	0.0066\\
0.0470	0.0170\\
0.0504	0.0137\\
0.0538	0.0012\\
0.0785	0.0294\\
0.0807	0.0376\\
0.0839	0.0562\\
0.0847	0.0745\\
0.0872	0.0752\\
0.0898	0.0402\\
0.0925	0.0367\\
0.0963	0.0342\\
0.1008	0.0184\\
0.1047	0.0149\\
0.1052	0.0212\\
0.1062	0.0245\\
0.1080	0.0315\\
0.1104	0.0370\\
0.1146	0.0004\\
0.1197	0.0075\\
0.1224	0.0036\\
0.1216	0.0028\\
0.1223	0.0057\\
0.1236	0.0003\\
0.1271	0.0115\\
0.1304	0.0158\\
0.1332	0.0262\\
0.1354	0.0026\\
0.1379	0.0186\\
0.1430	0.0211\\
0.1441	0.0687\\
0.1532	0.0316\\
0.1597	0.0017\\
0.1644	0.0119\\
0.1659	0.0443\\
0.1687	0.0106\\
0.1740	0.0024\\
0.1770	0.0330\\
0.1756	0.0213\\
0.1771	0.0199\\
0.1834	0.0183\\
0.1888	0.0101\\
0.1880	0.0080\\
0.1881	0.0273\\
0.1900	0.0486\\
0.1929	0.0425\\
0.1955	0.0298\\
0.1988	0.0139\\
0.2024	0.0134\\
0.2045	0.0114\\
0.2083	0.0109\\
0.2147	0.0149\\
0.2224	0.0065\\
0.2267	0.0077\\
0.2251	0.0038\\
0.2250	0.0212\\
0.2275	0.0325\\
0.2315	0.0510\\
0.2352	0.0452\\
0.2374	0.0340\\
0.2405	0.0189\\
0.2473	0.0003\\
0.2567	0.0048\\
0.2621	0.0074\\
0.2595	0.0076\\
0.2586	0.0036\\
0.2603	0.0111\\
0.2635	0.0144\\
0.2667	0.0390\\
0.2690	0.0411\\
0.2726	0.0343\\
0.2768	0.0273\\
0.2802	0.0274\\
0.2839	0.0172\\
0.2862	0.0127\\
0.2931	0.0085\\
0.3029	0.0146\\
0.3085	0.0193\\
0.3047	0.0186\\
0.3029	0.0374\\
0.3029	0.0429\\
0.3058	0.0001\\
0.3084	0.0059\\
0.3167	0.0015\\
0.3279	0.0061\\
0.3336	0.0210\\
0.3283	0.0250\\
0.3259	0.0198\\
0.3290	0.0066\\
0.3342	0.0090\\
0.3449	0.0159\\
0.3503	0.0119\\
0.3452	0.0019\\
0.3429	0.0138\\
0.3442	0.0070\\
0.3478	0.0056\\
0.3521	0.0137\\
0.3647	0.0202\\
0.3757	0.0262\\
0.3772	0.0310\\
0.3723	0.0119\\
0.3698	0.0061\\
0.3710	0.0075\\
0.3746	0.0070\\
0.3786	0.0074\\
0.3842	0.0220\\
0.3874	0.0158\\
0.3925	0.0110\\
0.3966	0.0336\\
0.4002	0.0278\\
0.4053	0.0232\\
0.4114	0.0036\\
0.4155	0.0092\\
0.4212	0.0016\\
0.4261	0.0321\\
0.4330	0.0800\\
0.4378	0.0597\\
0.4503	0.0244\\
0.4664	0.0179\\
0.4768	0.0268\\
0.4762	0.0402\\
0.4739	0.0548\\
0.4764	0.0403\\
0.4849	0.0456\\
0.4914	0.0359\\
0.5002	0.0003\\
0.5083	0.0053\\
0.5192	0.0132\\
0.5277	0.0093\\
0.5378	0.0362\\
0.5459	0.0526\\
0.5496	0.0458\\
0.5593	0.0364\\
0.5666	0.0166\\
0.5779	0.0100\\
0.5906	0.0213\\
0.6160	0.0372\\
0.6381	0.0572\\
0.6412	0.0412\\
0.6390	0.0408\\
0.6407	0.0318\\
0.6517	0.0314\\
0.6559	0.0240\\
0.6687	0.0174\\
0.6993	0.0363\\
0.7266	0.0607\\
0.7293	0.0389\\
0.7252	0.0175\\
0.7270	0.0127\\
0.7413	0.0250\\
0.7507	0.0265\\
0.7559	0.0148\\
0.7721	0.0175\\
0.7831	0.0232\\
0.7994	0.0293\\
0.8091	0.0146\\
0.8247	0.0051\\
0.8363	0.0099\\
0.8530	0.0400\\
0.8634	0.0320\\
0.8780	0.0471\\
0.8887	0.0374\\
0.9221	0.0491\\
0.9337	0.0364\\
0.9526	0.0458\\
0.9389	0.0342\\
0.9363	0.0246\\
0.9384	0.0198\\
0.9488	0.0279\\
0.9496	0.0145\\
0.9568	0.0350\\
0.9605	0.0342\\
0.9891	0.0587\\
0.9997	0.0710\\
-0.7512	0.0275\\
-0.7642	0.0193\\
-0.7739	0.0105\\
-0.7735	0.0133\\
-0.7587	0.0123\\
-0.7565	0.0123\\
-0.7679	0.0163\\
-0.7747	0.0193\\
-0.7571	0.0131\\
-0.7377	0.0123\\
-0.7199	0.0012\\
-0.7161	0.0068\\
-0.7206	0.0067\\
-0.7312	0.0155\\
-0.6947	0.0261\\
-0.6325	0.0821\\
-0.6440	0.0613\\
-0.7084	0.0108\\
-0.7411	0.0408\\
-0.7419	0.0305\\
-0.7371	0.0260\\
-0.7360	0.0314\\
-0.7318	0.0289\\
-0.7312	0.0257\\
-0.7264	0.0228\\
-0.7139	0.0059\\
-0.6958	0.0058\\
-0.6741	0.0333\\
-0.6678	0.0230\\
-0.6645	0.0120\\
-0.6718	0.0142\\
-0.6562	0.0027\\
-0.6012	0.0425\\
-0.5730	0.0647\\
-0.6099	0.0345\\
-0.6621	0.0078\\
-0.6650	0.0260\\
-0.6614	0.0265\\
-0.6570	0.0313\\
-0.6540	0.0184\\
-0.6390	0.0068\\
-0.6227	0.0082\\
-0.6207	0.0066\\
-0.6239	0.0011\\
-0.6167	0.0015\\
-0.6070	0.0056\\
-0.6102	0.0034\\
-0.6135	0.0042\\
-0.6059	0.0034\\
-0.5878	0.0140\\
-0.5801	0.0182\\
-0.5838	0.0114\\
-0.5957	0.0054\\
-0.5941	0.0063\\
-0.5820	0.0175\\
-0.5779	0.0120\\
-0.5840	0.0114\\
-0.5827	0.0135\\
-0.5683	0.0203\\
-0.5545	0.0301\\
-0.5509	0.0202\\
-0.5459	0.0220\\
-0.5421	0.0445\\
-0.5338	0.0660\\
-0.5254	0.0644\\
-0.5148	0.0722\\
-0.5107	0.0668\\
-0.5065	0.0513\\
-0.5102	0.0085\\
-0.5093	0.0168\\
-0.4956	0.0006\\
-0.4756	0.0374\\
-0.4607	0.0709\\
-0.4594	0.0487\\
-0.4554	0.0414\\
-0.4429	0.0491\\
-0.4233	0.0543\\
-0.4105	0.0455\\
-0.4060	0.0564\\
-0.4072	0.0393\\
-0.4012	0.0507\\
-0.3946	0.0410\\
-0.3831	0.0239\\
-0.3809	0.0423\\
-0.3744	0.0725\\
-0.3590	0.0882\\
-0.3447	0.0786\\
-0.3376	0.0549\\
-0.3368	0.0344\\
-0.3365	0.0251\\
-0.3310	0.0227\\
-0.3182	0.0925\\
-0.3060	0.1067\\
-0.2999	0.1572\\
-0.2998	0.1446\\
-0.2987	0.1494\\
-0.2903	0.1368\\
-0.2804	0.1309\\
-0.2790	0.0944\\
-0.2757	0.0440\\
-0.2675	0.0463\\
-0.2584	0.0931\\
-0.2561	0.0313\\
-0.2549	0.0180\\
-0.2479	0.0359\\
-0.2431	0.0060\\
-0.2323	0.0015\\
-0.2235	0.0191\\
-0.2193	0.0606\\
-0.2200	0.0303\\
-0.2136	0.0632\\
-0.1840	0.0948\\
-0.1846	0.0602\\
-0.2002	0.0343\\
-0.2053	0.0526\\
-0.2052	0.0289\\
-0.2008	0.0360\\
-0.1959	0.0336\\
-0.1908	0.0386\\
-0.1863	0.0289\\
-0.1803	0.0554\\
-0.1729	0.0413\\
-0.1670	0.0494\\
-0.1637	0.0445\\
-0.1619	0.0280\\
-0.1590	0.0109\\
-0.1575	0.0116\\
-0.1571	0.0107\\
-0.1543	0.0038\\
-0.1485	0.0217\\
-0.1454	0.0208\\
-0.1435	0.0102\\
-0.1417	0.0295\\
-0.1358	0.0520\\
-0.1301	0.0516\\
-0.1259	0.0664\\
-0.1231	0.0650\\
-0.1193	0.0851\\
-0.1131	0.1199\\
-0.1062	0.1469\\
-0.1008	0.1117\\
-0.0985	0.1459\\
-0.0930	0.1562\\
-0.0864	0.1380\\
-0.0801	0.0891\\
-0.0739	0.0919\\
-0.0668	0.0534\\
-0.0472	0.0894\\
-0.0283	0.0750\\
-0.0207	0.0738\\
0.0141	0.0797\\
0.0235	0.0663\\
0.0325	0.0930\\
0.0400	0.1069\\
0.0496	0.1082\\
0.0568	0.1206\\
0.0648	0.0973\\
0.0727	0.0571\\
0.0826	0.0447\\
0.0884	0.0470\\
0.0992	0.0350\\
0.1066	0.0319\\
0.1142	0.0185\\
0.1247	0.0301\\
0.1338	0.0378\\
0.1387	0.0275\\
0.1447	0.0478\\
0.1535	0.0364\\
0.1596	0.0147\\
0.1728	0.0157\\
0.1824	0.0099\\
0.1860	0.0461\\
0.1911	0.0191\\
0.1986	0.0061\\
0.2064	0.0065\\
0.2132	0.0261\\
0.2208	0.0212\\
0.2246	0.0028\\
0.2334	0.0052\\
0.2421	0.0038\\
0.2549	0.0076\\
0.2695	0.0135\\
0.2790	0.0195\\
0.2817	0.0012\\
0.2924	0.0003\\
0.2985	0.0119\\
0.3091	0.0301\\
0.3114	0.0300\\
0.3223	0.0256\\
0.3353	0.0396\\
0.3390	0.0378\\
0.3379	0.0324\\
0.3399	0.0050\\
0.3518	0.0067\\
0.3603	0.0231\\
0.3745	0.0008\\
0.3765	0.0159\\
0.3794	0.0382\\
0.3912	0.0505\\
0.4007	0.0345\\
0.3972	0.0156\\
0.3971	0.0132\\
0.4025	0.0052\\
0.4104	0.0086\\
0.4255	0.0173\\
0.4351	0.0105\\
0.4316	0.0122\\
0.4312	0.0423\\
0.4367	0.0625\\
0.4539	0.0318\\
0.4700	0.0088\\
0.4728	0.0100\\
0.4692	0.0021\\
0.4693	0.0234\\
0.4755	0.0352\\
0.4953	0.0002\\
0.5130	0.0148\\
0.5147	0.0210\\
0.5113	0.0142\\
0.5127	0.0101\\
0.5182	0.0201\\
0.5389	0.0353\\
0.5575	0.0668\\
0.5598	0.0496\\
0.5656	0.0327\\
0.5819	0.0495\\
0.5951	0.0584\\
0.5933	0.0340\\
0.5972	0.0385\\
0.6133	0.0458\\
0.6282	0.0522\\
0.6232	0.0334\\
0.6240	0.0178\\
0.6332	0.0177\\
0.6602	0.0414\\
0.6834	0.0374\\
0.6869	0.0386\\
0.6800	0.0188\\
0.6795	0.0136\\
0.6846	0.0052\\
0.6959	0.0223\\
0.7047	0.0450\\
0.7178	0.0507\\
0.7304	0.0310\\
0.7614	0.0332\\
0.7667	0.0096\\
0.7476	0.0159\\
0.6906	0.0681\\
0.7193	0.0565\\
0.8015	0.0138\\
0.8592	0.0683\\
0.8674	0.0851\\
0.8786	0.0875\\
0.8855	0.0712\\
0.9017	0.0812\\
0.9087	0.0916\\
0.9038	0.0961\\
0.8855	0.0659\\
0.8814	0.0349\\
0.8838	0.0071\\
0.8950	0.0302\\
0.8988	0.0290\\
0.9066	0.0354\\
0.9069	0.0400\\
0.9130	0.0384\\
0.9123	0.0391\\
0.9192	0.0343\\
0.9197	0.0055\\
0.9259	0.0039\\
0.9250	0.0083\\
0.9437	0.0355\\
0.9647	0.0649\\
0.9737	0.0595\\
0.9496	0.0283\\
0.9540	0.0389\\
0.9607	0.0487\\
0.9851	0.0668\\
-0.8976	0.0069\\
-0.8744	0.0061\\
-0.8712	0.0014\\
-0.8830	0.0272\\
-0.9090	0.0372\\
-0.9109	0.0319\\
-0.9042	0.0150\\
-0.8927	0.0056\\
-0.9083	0.0004\\
-0.9062	0.0013\\
-0.8823	0.0096\\
-0.8610	0.0251\\
-0.8641	0.0180\\
-0.8800	0.0108\\
-0.8886	0.0263\\
-0.8875	0.0119\\
-0.8686	0.0131\\
-0.8484	0.0241\\
-0.8489	0.0207\\
-0.8630	0.0185\\
-0.8805	0.0074\\
-0.8882	0.0021\\
-0.8753	0.0167\\
-0.8516	0.0181\\
-0.8203	0.0069\\
-0.8052	0.0212\\
-0.8027	0.0506\\
-0.8111	0.0383\\
-0.8004	0.0577\\
-0.7724	0.0740\\
-0.7449	0.0796\\
-0.7367	0.0490\\
-0.7210	0.0359\\
-0.7085	0.0148\\
-0.6928	0.0178\\
-0.6825	0.0200\\
-0.6774	0.0095\\
-0.6811	0.0063\\
-0.6695	0.0058\\
-0.6456	0.0276\\
-0.6313	0.0350\\
-0.6146	0.0335\\
-0.6082	0.0226\\
-0.6100	0.0058\\
-0.5986	0.0249\\
-0.5755	0.0204\\
-0.5534	0.0334\\
-0.5411	0.0425\\
-0.5288	0.0397\\
-0.5186	0.0400\\
-0.5061	0.0365\\
-0.5005	0.0120\\
-0.4945	0.0129\\
-0.4957	0.0003\\
-0.4875	0.0007\\
-0.4703	0.0234\\
-0.4607	0.0384\\
-0.4551	0.0258\\
-0.4425	0.0208\\
-0.4353	0.0132\\
-0.4287	0.0085\\
-0.4286	0.0049\\
-0.4237	0.0002\\
-0.4075	0.0086\\
-0.3904	0.0472\\
-0.3758	0.0502\\
-0.3709	0.0436\\
-0.3678	0.0416\\
-0.3700	0.0498\\
-0.3648	0.0487\\
-0.3521	0.0199\\
-0.3462	0.0207\\
-0.3435	0.0172\\
-0.3343	0.0211\\
-0.3285	0.0005\\
-0.3226	0.0150\\
-0.3102	0.0231\\
-0.2813	0.0384\\
-0.2647	0.0836\\
-0.2780	0.0285\\
-0.2967	0.0315\\
-0.2952	0.0172\\
-0.2888	0.0040\\
-0.2827	0.0740\\
-0.2789	0.0375\\
-0.2685	0.0027\\
-0.2573	0.0152\\
-0.2482	0.0148\\
-0.2435	0.0733\\
-0.2432	0.0707\\
-0.2415	0.0544\\
-0.2347	0.0790\\
-0.2241	0.0599\\
-0.2201	0.0669\\
-0.2148	0.0665\\
-0.2133	0.0665\\
-0.2123	0.1036\\
-0.2084	0.0954\\
-0.2000	0.0592\\
-0.1914	0.0803\\
-0.1849	0.0717\\
-0.1827	0.0569\\
-0.1821	0.0380\\
-0.1836	0.0112\\
-0.1814	0.0082\\
-0.1745	0.0169\\
-0.1692	0.0173\\
-0.1680	0.0303\\
-0.1702	0.0152\\
-0.1709	0.0150\\
-0.1696	0.0071\\
-0.1638	0.0226\\
-0.1587	0.0117\\
-0.1562	0.0046\\
-0.1544	0.0033\\
-0.1535	0.0112\\
-0.1544	0.0003\\
-0.1547	0.0086\\
-0.1520	0.0188\\
-0.1501	0.0561\\
-0.1490	0.0853\\
-0.1486	0.0906\\
-0.1457	0.0695\\
-0.1429	0.0500\\
-0.1430	0.0342\\
-0.1420	0.0518\\
-0.1376	0.0660\\
-0.1355	0.0558\\
-0.1328	0.0603\\
-0.1309	0.0544\\
-0.1282	0.0496\\
-0.1246	0.0512\\
-0.1168	0.0321\\
-0.1150	0.0249\\
-0.1128	0.0273\\
-0.1124	0.0194\\
-0.1122	0.0270\\
-0.1103	0.0223\\
-0.1051	0.0343\\
-0.0985	0.0576\\
-0.0928	0.0864\\
-0.0906	0.0875\\
-0.0865	0.0897\\
-0.0769	0.0863\\
-0.0723	0.0566\\
-0.0682	0.0403\\
-0.0656	0.0301\\
-0.0599	0.0560\\
-0.0546	0.0612\\
-0.0508	0.0622\\
-0.0474	0.0660\\
-0.0411	0.0668\\
-0.0332	0.0663\\
-0.0299	0.0687\\
-0.0253	0.0828\\
0.0073	0.0855\\
0.0110	0.0476\\
0.0666	0.0941\\
0.1139	0.0673\\
0.1175	0.0294\\
0.1209	0.0374\\
0.1267	0.0648\\
0.1318	0.0644\\
0.1329	0.0441\\
0.1364	0.0351\\
0.1427	0.0376\\
0.1482	0.0391\\
0.1495	0.0444\\
0.1512	0.0321\\
0.1566	0.0227\\
0.1667	0.0515\\
0.1748	0.0479\\
0.1770	0.0464\\
0.1781	0.0225\\
0.1808	0.0052\\
0.1860	0.0278\\
0.1947	0.0657\\
0.2020	0.0613\\
0.2046	0.0674\\
0.2050	0.0615\\
0.2073	0.0492\\
0.2105	0.0330\\
0.2208	0.0170\\
0.2298	0.0220\\
0.2313	0.0089\\
0.2303	0.0066\\
0.2331	0.0242\\
0.2418	0.0188\\
0.2513	0.0191\\
0.2573	0.0032\\
0.2577	0.0235\\
0.2580	0.0453\\
0.2602	0.0190\\
0.2688	0.0044\\
0.2804	0.0011\\
0.2878	0.0030\\
0.2869	0.0131\\
0.2885	0.0371\\
0.2904	0.0182\\
0.2966	0.0110\\
0.3013	0.0221\\
0.3110	0.0288\\
0.3240	0.0187\\
0.3323	0.0155\\
0.3299	0.0313\\
0.3298	0.0307\\
0.3336	0.0203\\
0.3400	0.0000\\
0.3457	0.0094\\
0.3494	0.0135\\
0.3547	0.0045\\
0.3610	0.0111\\
0.3666	0.0236\\
0.3787	0.0333\\
0.3939	0.0393\\
0.4040	0.0370\\
0.4043	0.0334\\
0.4139	0.0420\\
0.4193	0.0315\\
0.4294	0.0250\\
0.4276	0.0023\\
0.4291	0.0121\\
0.4335	0.0158\\
0.4416	0.0049\\
0.4510	0.0189\\
0.4707	0.0230\\
0.4876	0.0304\\
0.4912	0.0323\\
0.4904	0.0222\\
0.5024	0.0325\\
0.5181	0.0370\\
0.5205	0.0402\\
0.5190	0.0220\\
0.5225	0.0209\\
0.5418	0.0240\\
0.5539	0.0135\\
0.5756	0.0362\\
0.5794	0.0333\\
0.5776	0.0443\\
0.5800	0.0453\\
0.6007	0.0527\\
0.6266	0.0629\\
0.6457	0.0567\\
0.6440	0.0268\\
0.6467	0.0302\\
0.6523	0.0216\\
0.6756	0.0244\\
0.7014	0.0335\\
0.7216	0.0437\\
0.7232	0.0440\\
0.7449	0.0519\\
0.7708	0.0580\\
0.7749	0.0417\\
0.7725	0.0092\\
0.7955	0.0350\\
0.8259	0.0511\\
0.8310	0.0531\\
0.8403	0.0683\\
0.8640	0.0847\\
0.8816	0.0736\\
0.8870	0.0387\\
0.8861	0.0233\\
0.8900	0.0092\\
0.9099	0.0173\\
0.9190	0.0168\\
0.9333	0.0386\\
0.9409	0.0478\\
0.9577	0.0529\\
0.9666	0.0381\\
0.9817	0.0261\\
0.9900	0.0026\\
0.9948	0.0094\\
1.0104	0.0107\\
1.0236	0.0150\\
1.0615	0.0428\\
1.0957	0.0784\\
1.0960	0.0665\\
1.0919	0.0554\\
1.0898	0.0494\\
1.1084	0.0677\\
-0.2586	0.2132\\
-0.2614	0.1934\\
-0.2594	0.1440\\
-0.2520	0.0852\\
-0.2455	0.0689\\
-0.2426	0.0419\\
-0.2400	0.0413\\
-0.2381	0.0419\\
-0.2366	0.0459\\
-0.2344	0.0266\\
-0.2353	0.0120\\
-0.2384	0.0196\\
-0.0949	0.1070\\
0.2369	0.4264\\
0.2668	0.4069\\
0.2788	0.1911\\
0.2820	0.1052\\
0.2854	0.0455\\
0.2927	0.0139\\
0.3039	0.0224\\
0.3164	0.0391\\
0.3273	0.0209\\
0.4638	0.2443\\
0.4820	0.3086\\
0.4840	0.1574\\
0.4914	0.0414\\
0.4971	0.0470\\
0.4979	0.0577\\
0.6874	0.2135\\
0.6887	0.2981\\
0.6909	0.1179\\
0.7153	0.0013\\
0.7456	0.0226\\
0.7542	0.0116\\
0.7574	0.0075\\
0.7657	0.0164\\
0.7854	0.0449\\
0.7985	0.0480\\
0.8289	0.0690\\
0.8478	0.0522\\
0.8835	0.0570\\
0.8895	0.0457\\
0.8900	0.0264\\
0.8977	0.0040\\
0.9325	0.0074\\
0.9676	0.0319\\
0.9938	0.0565\\
0.9873	0.0532\\
0.9923	0.0817\\
0.9958	0.0556\\
1.0159	0.0320\\
1.0273	0.0147\\
1.0460	0.0189\\
1.0576	0.0134\\
1.0904	0.0361\\
1.1248	0.0493\\
1.1482	0.0694\\
1.1322	0.0536\\
1.1344	0.0400\\
1.1260	0.0201\\
1.1361	0.0375\\
1.1354	0.0445\\
1.1536	0.0742\\
1.1643	0.0982\\
1.1615	0.1108\\
1.1214	0.0983\\
1.1082	0.0803\\
1.1007	0.0697\\
1.0859	0.0685\\
1.0263	0.0403\\
0.9411	0.0151\\
0.8448	0.0787\\
0.8986	0.0043\\
-0.9010	0.7062\\
-0.8900	0.3430\\
-0.8970	0.0537\\
-0.9056	0.0231\\
-0.9197	0.0325\\
-0.9249	0.0373\\
-0.9317	0.0280\\
-0.9433	0.0226\\
-0.9462	0.0287\\
-0.9543	0.0385\\
-0.9541	0.0578\\
-0.9622	0.0820\\
-0.9615	0.1072\\
-0.9687	0.1019\\
-0.9667	0.0907\\
-0.9755	0.0790\\
-0.9947	0.0263\\
-1.0057	0.0069\\
-1.0050	0.0034\\
-0.9837	0.0098\\
-0.9615	0.0246\\
-0.9640	0.0170\\
-0.9794	0.0143\\
-0.9546	0.0262\\
-0.8690	0.1384\\
-0.8142	0.1652\\
-0.5129	0.5157\\
-0.4848	0.3435\\
};
\addlegendentry{Measurements};

\addplot [color=mycolor1,solid,line width=1.0pt]
  table[row sep=crcr]{%
-2.0144	0.2576\\
-1.9309	0.2390\\
-1.8475	0.2212\\
-1.7640	0.2042\\
-1.6806	0.1880\\
-1.5971	0.1726\\
-1.5137	0.1581\\
-1.4303	0.1443\\
-1.3468	0.1314\\
-1.2634	0.1193\\
-1.1799	0.1080\\
-1.0965	0.0975\\
-1.0131	0.0878\\
-0.9296	0.0789\\
-0.8462	0.0709\\
-0.7627	0.0636\\
-0.6793	0.0572\\
-0.5959	0.0516\\
-0.5124	0.0468\\
-0.4290	0.0428\\
-0.3455	0.0396\\
-0.2621	0.0372\\
-0.1786	0.0356\\
-0.0952	0.0349\\
-0.0118	0.0349\\
0.0717	0.0358\\
0.1551	0.0375\\
0.2386	0.0400\\
0.3220	0.0433\\
0.4054	0.0474\\
0.4889	0.0523\\
0.5723	0.0581\\
0.6558	0.0646\\
0.7392	0.0720\\
0.8226	0.0802\\
0.9061	0.0892\\
0.9895	0.0990\\
1.0730	0.1096\\
1.1564	0.1210\\
1.2398	0.1332\\
1.3233	0.1463\\
1.4067	0.1601\\
1.4902	0.1748\\
1.5736	0.1903\\
1.6571	0.2066\\
1.7405	0.2237\\
1.8239	0.2416\\
1.9074	0.2603\\
1.9908	0.2798\\
2.0743	0.3002\\
};
\addlegendentry{Quadratic fit};

\addplot [color=black,dashed]
 plot [error bars/.cd, y dir = both, y explicit]
 table[row sep=crcr, y error plus index=2, y error minus index=3]{%
-1.2000	0.0594	0.1022	0.0203\\
-1.1000	0.0758	0.1368	0.0450\\
-1.0000	0.0578	0.0907	0.0293\\
-0.9000	0.0373	0.1000	0.0180\\
-0.8000	0.0371	0.0880	0.0156\\
-0.7000	0.0208	0.0456	0.0086\\
-0.6000	0.0298	0.0667	0.0135\\
-0.5000	0.0303	0.0614	0.0142\\
-0.4000	0.0403	0.0649	0.0191\\
-0.3000	0.0307	0.0548	0.0159\\
-0.2000	0.0301	0.0551	0.0144\\
-0.1000	0.0281	0.0476	0.0145\\
0.0000	0.0261	0.0602	0.0109\\
0.1000	0.0177	0.0349	0.0078\\
0.2000	0.0200	0.0378	0.0092\\
0.3000	0.0242	0.0499	0.0108\\
0.4000	0.0326	0.0740	0.0113\\
0.5000	0.0270	0.0475	0.0141\\
0.6000	0.0410	0.0745	0.0209\\
0.7000	0.0415	0.0932	0.0211\\
0.8000	0.0531	0.0977	0.0229\\
0.9000	0.0478	0.0923	0.0248\\
1.0000	0.0430	0.0657	0.0192\\
1.1000	0.0616	0.0830	0.0443\\
1.2000	0.0967	0.1727	0.0266\\
};
\addlegendentry{Percentiles};

\addplot [color=mycolor2,solid,line width=1.0pt]
  table[row sep=crcr]{%
-2.0144	0.8012\\
-1.9309	0.7402\\
-1.8475	0.6818\\
-1.7640	0.6260\\
-1.6806	0.5727\\
-1.5971	0.5220\\
-1.5137	0.4738\\
-1.4303	0.4283\\
-1.3468	0.3852\\
-1.2634	0.3448\\
-1.1799	0.3069\\
-1.0965	0.2716\\
-1.0131	0.2388\\
-0.9296	0.2086\\
-0.8462	0.1810\\
-0.7627	0.1559\\
-0.6793	0.1334\\
-0.5959	0.1134\\
-0.5124	0.0960\\
-0.4290	0.0812\\
-0.3455	0.0690\\
-0.2621	0.0593\\
-0.1786	0.0521\\
-0.0952	0.0476\\
-0.0118	0.0456\\
0.0717	0.0461\\
0.1551	0.0493\\
0.2386	0.0550\\
0.3220	0.0632\\
0.4054	0.0740\\
0.4889	0.0874\\
0.5723	0.1033\\
0.6558	0.1218\\
0.7392	0.1429\\
0.8226	0.1666\\
0.9061	0.1927\\
0.9895	0.2215\\
1.0730	0.2528\\
1.1564	0.2867\\
1.2398	0.3232\\
1.3233	0.3622\\
1.4067	0.4038\\
1.4902	0.4479\\
1.5736	0.4946\\
1.6571	0.5439\\
1.7405	0.5957\\
1.8239	0.6501\\
1.9074	0.7071\\
1.9908	0.7666\\
2.0743	0.8287\\
};
\addlegendentry{Size divergence};

\end{axis}
\end{tikzpicture}%
	\caption{Absolute error distribution for the estimates of $\vartheta_z$ for a set of landing tests performed at different constant vertical speeds above the roadmap texture. The blue line shows a quadratic fit of the error and the dashed black line shows the 25, 50 and 75\% percentiles of the data. For comparison, the model obtained for the frame-based size divergence estimator \cite{Ho2016a} is shown as well.} 
	\label{fig:divergence_error_dist}
\end{figure}


In \citet{Ho2016a} an extensive characterization of two frame-based visual estimators for $\vartheta_z$ is performed, which includes an assessment of their absolute error distribution up to $\vartheta_z\approx 1.3$ (Fig. 10 in the paper). For a first-order comparison, \cref{fig:divergence_estimates} also shows the quadratic error fit obtained for the frame-based 'size divergence' estimator, which performed best in \citet{Ho2016a}. Compared to the presented event-based estimator, the size divergence estimator achieves slightly lower errors in the region of $\vartheta_z < 0.5$. However, for faster motion, the error is lower for our event-based estimator. Note that our quadratic model is based on relatively little measurements, and does not yet provide a full characterization.

\begin{table}[!ht]
	\centering
	\caption{Parameters of the quadratic fit models in \cref{fig:divergence_error_dist}.}
	\begin{tabular}{l|cc}
\hline
~ & This work & Size divergence \cite{Ho2016a} \\ \hline
$p_0$ & 0.0359 & 0.0455 \\
$p_1$ & -0.0012 & -0.0043 \\
$p_2$ & 0.0468 & 0.1841 \\ \hline

\end{tabular}

	\label{tab:div_error_parameters}
\end{table}

\subsubsection{Horizontal Motion}
Estimation performance for the components $\vartheta_x$ and $\vartheta_y$ is assessed through a dataset consisting of primarily horizontal motion. In the following set, the DVS is moved in a circular pattern above a checkerboard surface at approximately 0.8 m height.

The resulting visual observable estimates are shown in \cref{fig:horizontal_motion}. For completeness, the values of $\vartheta_z$ are displayed as well. Overall, the horizontal movement is captured well in the estimates, although some disturbances are still clearly present, for example around $t=23$ s. The deviations are comparable to those seen in the vertical motion dataset. A summary of the error values is presented in \cref{tab:horizontal_errors}.

\begin{figure}[!ht]
	\centering
%		\tikzset{external/force remake=true}
%	\renewcommand{\ylabeldist}{0.02}
	\setlength{\fwidth}{0.4\linewidth}
	% This file was created by matlab2tikz.
%
%The latest updates can be retrieved from
%  http://www.mathworks.com/matlabcentral/fileexchange/22022-matlab2tikz-matlab2tikz
%where you can also make suggestions and rate matlab2tikz.
%
\definecolor{mycolor1}{rgb}{0.00000,0.44700,0.74100}%
\definecolor{mycolor2}{rgb}{0.85000,0.32500,0.09800}%
%
\begin{tikzpicture}

\begin{axis}[%
width=0.974\fwidth,
height=0.203\fwidth,
at={(0\fwidth,0.964\fwidth)},
scale only axis,
xmin=18.0242,
xmax=29.9963,
xlabel={$t$ [s]},
ymin=0.0000,
ymax=1.0000,
ylabel={$h$ [m]},
axis background/.style={fill=white},
title style={font=\labelsize},
xlabel style={font=\labelsize,at={(axis description cs:0.5,\xlabeldist)}},
ylabel style={font=\labelsize,at={(axis description cs:\ylabeldist,0.5)}},
legend style={font=\ticksize},
ticklabel style={font=\ticksize}
]
\addplot [color=mycolor1,solid,forget plot]
  table[row sep=crcr]{%
18.0242	0.7980\\
18.0555	0.7950\\
18.0768	0.7940\\
18.1192	0.7910\\
18.1508	0.7890\\
18.1935	0.7860\\
18.2257	0.7840\\
18.2462	0.7820\\
18.2881	0.7800\\
18.3201	0.7780\\
18.3932	0.7770\\
18.5307	0.7760\\
18.5728	0.7770\\
18.6675	0.7760\\
18.7298	0.7750\\
18.8977	0.7740\\
18.9290	0.7750\\
18.9709	0.7760\\
19.0132	0.7770\\
19.0874	0.7780\\
19.1393	0.7790\\
19.1820	0.7800\\
19.2356	0.7810\\
19.2678	0.7820\\
19.3095	0.7830\\
19.3404	0.7840\\
19.3616	0.7850\\
19.4037	0.7860\\
19.4353	0.7880\\
19.4571	0.7890\\
19.5104	0.7920\\
19.5415	0.7930\\
19.5845	0.7960\\
19.6165	0.7970\\
19.6699	0.7990\\
19.7013	0.7990\\
19.7438	0.8000\\
19.7959	0.7990\\
19.8376	0.7990\\
19.8793	0.8000\\
19.9417	0.8010\\
19.9731	0.8020\\
20.0147	0.8020\\
20.0461	0.8030\\
20.0878	0.8050\\
20.1293	0.8050\\
20.2033	0.8060\\
20.2555	0.8070\\
20.4463	0.8080\\
20.5101	0.8090\\
20.5731	0.8100\\
20.6150	0.8100\\
20.6463	0.8110\\
20.6889	0.8110\\
20.8258	0.8100\\
20.8796	0.8110\\
20.9114	0.8120\\
20.9540	0.8130\\
20.9860	0.8140\\
21.0493	0.8150\\
21.0808	0.8150\\
21.1651	0.8160\\
21.2070	0.8160\\
21.2385	0.8160\\
21.2714	0.8170\\
21.3875	0.8160\\
21.4289	0.8150\\
21.5018	0.8140\\
21.5448	0.8130\\
21.5865	0.8120\\
21.6071	0.8110\\
21.6495	0.8110\\
21.6821	0.8090\\
21.7030	0.8080\\
21.7551	0.8060\\
21.7865	0.8040\\
21.8284	0.8010\\
21.8605	0.7990\\
21.9025	0.7960\\
21.9343	0.7950\\
21.9552	0.7920\\
21.9970	0.7890\\
22.0283	0.7860\\
22.0712	0.7830\\
22.1035	0.7800\\
22.1460	0.7760\\
22.1782	0.7730\\
22.2095	0.7710\\
22.2509	0.7690\\
22.2822	0.7650\\
22.3032	0.7630\\
22.3452	0.7590\\
22.3775	0.7560\\
22.4296	0.7530\\
22.4616	0.7510\\
22.5135	0.7470\\
22.5764	0.7450\\
22.6075	0.7440\\
22.6390	0.7420\\
22.6806	0.7410\\
22.7224	0.7400\\
22.7647	0.7410\\
22.8075	0.7420\\
22.8493	0.7430\\
22.8807	0.7450\\
22.9013	0.7460\\
22.9444	0.7500\\
22.9753	0.7530\\
23.0179	0.7580\\
23.0496	0.7640\\
23.0806	0.7700\\
23.1228	0.7760\\
23.1547	0.7820\\
23.2069	0.7890\\
23.2381	0.7930\\
23.2911	0.7990\\
23.3233	0.8020\\
23.3861	0.8050\\
23.4172	0.8060\\
23.4589	0.8070\\
23.4900	0.8070\\
23.5329	0.8060\\
23.5746	0.8050\\
23.5956	0.8040\\
23.6371	0.8020\\
23.6697	0.8000\\
23.7113	0.7980\\
23.7425	0.7970\\
23.7644	0.7960\\
23.8060	0.7950\\
23.8791	0.7950\\
23.9116	0.7960\\
23.9332	0.7960\\
23.9757	0.7970\\
24.0172	0.7990\\
24.0591	0.8010\\
24.0914	0.8020\\
24.1124	0.8040\\
24.1543	0.8060\\
24.1852	0.8080\\
24.2269	0.8100\\
24.2581	0.8130\\
24.2791	0.8140\\
24.3210	0.8170\\
24.3534	0.8190\\
24.4063	0.8220\\
24.4373	0.8230\\
24.4791	0.8250\\
24.5209	0.8260\\
24.6258	0.8270\\
24.6683	0.8260\\
24.6999	0.8250\\
24.7415	0.8240\\
24.7832	0.8220\\
24.8258	0.8200\\
24.8574	0.8190\\
24.8791	0.8180\\
25.1006	0.8170\\
25.1436	0.8180\\
25.1863	0.8190\\
25.2184	0.8200\\
25.2602	0.8210\\
25.2915	0.8220\\
25.3124	0.8230\\
25.3748	0.8240\\
25.4268	0.8250\\
25.5652	0.8260\\
25.6498	0.8270\\
25.7332	0.8280\\
25.7751	0.8270\\
25.8061	0.8270\\
25.8384	0.8260\\
25.8915	0.8240\\
25.9230	0.8230\\
25.9655	0.8220\\
25.9968	0.8210\\
26.0498	0.8190\\
26.0817	0.8170\\
26.1239	0.8160\\
26.1662	0.8150\\
26.2088	0.8140\\
26.2403	0.8130\\
26.3562	0.8120\\
26.3994	0.8120\\
26.4310	0.8130\\
26.4729	0.8140\\
26.5142	0.8140\\
26.5570	0.8150\\
26.5988	0.8160\\
26.6404	0.8180\\
26.6820	0.8180\\
26.7251	0.8200\\
26.7672	0.8220\\
26.7885	0.8230\\
26.8415	0.8250\\
26.8727	0.8260\\
26.9143	0.8270\\
26.9465	0.8280\\
26.9884	0.8290\\
27.0205	0.8290\\
27.0415	0.8280\\
27.1041	0.8260\\
27.1562	0.8250\\
27.1882	0.8240\\
27.2198	0.8230\\
27.2626	0.8220\\
27.2946	0.8200\\
27.3364	0.8190\\
27.3683	0.8170\\
27.3893	0.8160\\
27.4423	0.8130\\
27.4944	0.8090\\
27.5361	0.8070\\
27.5571	0.8060\\
27.5997	0.8030\\
27.6309	0.8020\\
27.6736	0.7980\\
27.7049	0.7960\\
27.7362	0.7930\\
27.7789	0.7900\\
27.8115	0.7880\\
27.8542	0.7840\\
27.8863	0.7810\\
27.9081	0.7800\\
27.9496	0.7770\\
27.9817	0.7760\\
28.0351	0.7720\\
28.0670	0.7710\\
28.1091	0.7690\\
28.1402	0.7680\\
28.1715	0.7660\\
28.2133	0.7670\\
28.2444	0.7660\\
28.2871	0.7640\\
28.3183	0.7640\\
28.3392	0.7630\\
28.3911	0.7610\\
28.4228	0.7620\\
28.4765	0.7590\\
28.5296	0.7590\\
28.5619	0.7590\\
28.5935	0.7580\\
28.6348	0.7590\\
28.6661	0.7590\\
28.7078	0.7610\\
28.7392	0.7620\\
28.7696	0.7630\\
28.8111	0.7650\\
28.8432	0.7670\\
28.8862	0.7710\\
28.9183	0.7710\\
28.9391	0.7730\\
28.9825	0.7750\\
29.0144	0.7760\\
29.0672	0.7800\\
29.0984	0.7810\\
29.1195	0.7850\\
29.1716	0.7860\\
29.2039	0.7870\\
29.2465	0.7890\\
29.2778	0.7910\\
29.3401	0.7930\\
29.3716	0.7940\\
29.4237	0.7970\\
29.4559	0.7990\\
29.5077	0.8020\\
29.5391	0.8040\\
29.5807	0.8070\\
29.6119	0.8100\\
29.6328	0.8120\\
29.6755	0.8140\\
29.7068	0.8160\\
29.8773	0.8190\\
29.9205	0.8180\\
29.9528	0.8160\\
29.9963	0.8150\\
};
\end{axis}

\begin{axis}[%
width=0.974\fwidth,
height=0.203\fwidth,
at={(0\fwidth,0.643\fwidth)},
scale only axis,
xmin=18.0197,
xmax=29.9997,
xlabel={$t$ [s]},
ymin=-1.1000,
ymax=1.1000,
ylabel={$\vartheta_x$, $\hat{\vartheta}_x$ [1/s]},
axis background/.style={fill=white},
title style={font=\labelsize},
xlabel style={font=\labelsize,at={(axis description cs:0.5,\xlabeldist)}},
ylabel style={font=\labelsize,at={(axis description cs:\ylabeldist,0.5)}},
legend style={font=\ticksize},
ticklabel style={font=\ticksize}
]
\addplot [color=mycolor1,solid,forget plot]
  table[row sep=crcr]{%
18.0450	0.0942\\
18.0768	0.1107\\
18.1401	0.1240\\
18.1610	0.1112\\
18.2257	0.1059\\
18.2462	0.0909\\
18.2881	0.0773\\
18.3307	0.0895\\
18.3725	0.0904\\
18.4141	0.0645\\
18.4454	0.0491\\
18.4767	0.0327\\
18.4981	0.0216\\
18.5413	0.0133\\
18.5937	0.0271\\
18.6572	0.0275\\
18.6779	0.0044\\
18.6985	-0.0156\\
18.7298	-0.0392\\
18.7611	-0.0623\\
18.8029	-0.0347\\
18.8448	-0.0299\\
18.8871	-0.0199\\
18.9080	-0.0447\\
18.9392	-0.0661\\
18.9603	-0.1090\\
18.9709	-0.1394\\
18.9926	-0.1771\\
19.0132	-0.2088\\
19.0239	-0.2334\\
19.0446	-0.2495\\
19.0657	-0.2704\\
19.0874	-0.2895\\
19.1082	-0.3007\\
19.1393	-0.3057\\
19.1711	-0.3499\\
19.1928	-0.3874\\
19.2248	-0.4097\\
19.2466	-0.4394\\
19.2572	-0.4646\\
19.2781	-0.4939\\
19.3095	-0.5162\\
19.3302	-0.5292\\
19.3616	-0.5541\\
19.4249	-0.5768\\
19.4571	-0.6031\\
19.4999	-0.6087\\
19.5311	-0.6102\\
19.5415	-0.6376\\
19.5628	-0.6686\\
19.5949	-0.7154\\
19.6271	-0.7268\\
19.6595	-0.7661\\
19.6803	-0.7765\\
19.7013	-0.7513\\
19.7649	-0.7347\\
19.7857	-0.7505\\
19.8169	-0.7664\\
19.8274	-0.7417\\
19.8376	-0.7142\\
19.8583	-0.6911\\
19.8691	-0.7201\\
19.8793	-0.7670\\
19.8898	-0.8109\\
19.9106	-0.8854\\
19.9313	-0.9104\\
19.9731	-0.9082\\
20.0045	-0.8807\\
20.0562	-0.8619\\
20.1086	-0.8535\\
20.1400	-0.8544\\
20.1608	-0.8693\\
20.2033	-0.9099\\
20.2243	-0.8940\\
20.2658	-0.8577\\
20.2867	-0.8357\\
20.3190	-0.7998\\
20.3608	-0.7831\\
20.4037	-0.8072\\
20.4359	-0.8153\\
20.4889	-0.8289\\
20.5202	-0.8135\\
20.5515	-0.7850\\
20.5731	-0.7564\\
20.6046	-0.7289\\
20.6252	-0.7042\\
20.6889	-0.6600\\
20.7103	-0.6507\\
20.7316	-0.6288\\
20.7420	-0.6046\\
20.7733	-0.5503\\
20.7943	-0.5106\\
20.8047	-0.4914\\
20.8151	-0.4685\\
20.8369	-0.4333\\
20.8693	-0.3932\\
20.9114	-0.3678\\
20.9540	-0.3618\\
20.9755	-0.3305\\
20.9966	-0.2933\\
21.0068	-0.2658\\
21.0493	-0.1976\\
21.0808	-0.1640\\
21.0911	-0.1368\\
21.1540	-0.1163\\
21.1756	-0.1263\\
21.2070	-0.1163\\
21.2278	-0.0905\\
21.2714	-0.0789\\
21.3029	-0.0606\\
21.3245	-0.0278\\
21.3456	-0.0479\\
21.3560	-0.0252\\
21.3875	-0.0031\\
21.4080	0.0187\\
21.4392	0.0444\\
21.4705	0.0609\\
21.5018	0.0835\\
21.5236	0.1046\\
21.5865	0.1413\\
21.6071	0.1493\\
21.6386	0.1688\\
21.6717	0.1854\\
21.6924	0.2008\\
21.7239	0.2131\\
21.7551	0.2053\\
21.7865	0.1981\\
21.8388	0.1912\\
21.8605	0.2197\\
21.8708	0.2417\\
21.9133	0.2787\\
21.9447	0.2833\\
21.9552	0.3187\\
21.9866	0.3042\\
22.0177	0.3172\\
22.0387	0.3680\\
22.0712	0.3950\\
22.1145	0.4349\\
22.1250	0.4645\\
22.1460	0.4760\\
22.1886	0.5222\\
22.2095	0.5249\\
22.2198	0.5730\\
22.2509	0.5902\\
22.2618	0.6251\\
22.2720	0.6065\\
22.2822	0.6274\\
22.2926	0.6473\\
22.3032	0.6754\\
22.3344	0.6954\\
22.3452	0.7331\\
22.3556	0.7670\\
22.3666	0.7936\\
22.3775	0.8162\\
22.3877	0.8363\\
22.4192	0.8145\\
22.4616	0.8150\\
22.4720	0.7944\\
22.5240	0.7792\\
22.5554	0.7684\\
22.5972	0.7541\\
22.6390	0.7212\\
22.6492	0.7052\\
22.6806	0.6879\\
22.7013	0.6632\\
22.7224	0.6495\\
22.7327	0.6277\\
22.7647	0.5839\\
22.7860	0.5724\\
22.7964	0.5379\\
22.8180	0.5077\\
22.8493	0.4462\\
22.8596	0.4166\\
22.8807	0.3852\\
22.9013	0.3431\\
22.9546	0.3060\\
22.9861	0.2610\\
23.0390	0.2247\\
23.0806	0.2133\\
23.1120	0.2469\\
23.1336	0.2269\\
23.1652	0.2036\\
23.1861	0.1817\\
23.2277	0.1933\\
23.2381	0.1858\\
23.2483	0.2036\\
23.2697	0.2220\\
23.3122	0.2637\\
23.3233	0.2783\\
23.3341	0.3014\\
23.3861	0.3057\\
23.4172	0.2464\\
23.4484	0.2091\\
23.4695	0.1851\\
23.4900	0.1716\\
23.5433	0.1577\\
23.5956	0.1439\\
23.6371	0.1226\\
23.6590	0.0980\\
23.6799	0.0717\\
23.7010	0.0548\\
23.7217	0.0386\\
23.7425	0.0192\\
23.7644	-0.0082\\
23.7852	-0.0271\\
23.8060	-0.0613\\
23.8272	-0.0882\\
23.8479	-0.1076\\
23.8791	-0.1285\\
23.8895	-0.1461\\
23.9116	-0.1913\\
23.9332	-0.2275\\
23.9541	-0.2726\\
23.9647	-0.2968\\
23.9967	-0.3467\\
24.0274	-0.3849\\
24.0591	-0.4138\\
24.0914	-0.4293\\
24.1124	-0.4598\\
24.1543	-0.4953\\
24.1958	-0.5053\\
24.2269	-0.5052\\
24.2683	-0.4911\\
24.2997	-0.5018\\
24.3534	-0.5363\\
24.3959	-0.5553\\
24.4269	-0.5668\\
24.4581	-0.5695\\
24.5101	-0.5977\\
24.5414	-0.6196\\
24.6040	-0.6273\\
24.6258	-0.6436\\
24.6573	-0.6395\\
24.6896	-0.6264\\
24.7101	-0.6434\\
24.7415	-0.6553\\
24.7626	-0.6763\\
24.7937	-0.7054\\
24.8258	-0.7422\\
24.8574	-0.7591\\
24.8791	-0.7431\\
24.9208	-0.7500\\
24.9414	-0.7651\\
24.9730	-0.7642\\
25.0153	-0.8034\\
25.0572	-0.8368\\
25.0895	-0.8487\\
25.1221	-0.8741\\
25.1436	-0.8585\\
25.2081	-0.8777\\
25.2184	-0.8698\\
25.2497	-0.8464\\
25.3018	-0.8541\\
25.3436	-0.8388\\
25.3748	-0.8266\\
25.4061	-0.8014\\
25.4377	-0.7766\\
25.4593	-0.7623\\
25.4801	-0.7452\\
25.4905	-0.7240\\
25.5543	-0.6964\\
25.5758	-0.7192\\
25.6388	-0.7187\\
25.6600	-0.7099\\
25.7127	-0.6931\\
25.7434	-0.6721\\
25.8061	-0.6424\\
25.8384	-0.6309\\
25.8808	-0.6075\\
25.9230	-0.6102\\
25.9761	-0.5955\\
26.0071	-0.5600\\
26.0393	-0.5350\\
26.0600	-0.5113\\
26.0817	-0.5392\\
26.1133	-0.5247\\
26.1456	-0.4970\\
26.1768	-0.4832\\
26.2088	-0.4754\\
26.2613	-0.4646\\
26.2934	-0.4481\\
26.3247	-0.4269\\
26.3562	-0.4237\\
26.3884	-0.3866\\
26.3994	-0.3526\\
26.4206	-0.3182\\
26.4412	-0.2923\\
26.4625	-0.2758\\
26.4932	-0.2490\\
26.5246	-0.2520\\
26.5570	-0.2653\\
26.5988	-0.2430\\
26.6091	-0.2287\\
26.6404	-0.2070\\
26.6612	-0.1584\\
26.6923	-0.1588\\
26.7351	-0.1542\\
26.7778	-0.1650\\
26.8101	-0.1357\\
26.8623	-0.1277\\
26.9039	-0.1131\\
26.9361	-0.1071\\
26.9569	-0.0905\\
26.9884	-0.0836\\
27.0205	-0.0565\\
27.0415	-0.0462\\
27.0932	-0.0133\\
27.1041	0.0180\\
27.1245	-0.0088\\
27.1562	0.0372\\
27.1779	0.0470\\
27.2198	0.0746\\
27.2516	0.1155\\
27.3048	0.1334\\
27.3256	0.1331\\
27.3476	0.1487\\
27.3788	0.1668\\
27.4317	0.1841\\
27.4528	0.1734\\
27.4632	0.1920\\
27.5050	0.1586\\
27.5258	0.1888\\
27.5571	0.1706\\
27.5997	0.1788\\
27.6309	0.2068\\
27.6413	0.1947\\
27.6624	0.2257\\
27.7154	0.2468\\
27.7362	0.2504\\
27.7681	0.2679\\
27.7895	0.2793\\
27.8226	0.3296\\
27.8542	0.3610\\
27.8863	0.4004\\
27.9081	0.4295\\
27.9706	0.4644\\
27.9923	0.4789\\
28.0245	0.5055\\
28.0457	0.4800\\
28.0670	0.5131\\
28.1194	0.5140\\
28.1298	0.5074\\
28.1402	0.5449\\
28.1715	0.5579\\
28.2235	0.5681\\
28.2338	0.5468\\
28.2444	0.5985\\
28.2550	0.5236\\
28.2871	0.5699\\
28.3079	0.5544\\
28.3287	0.5980\\
28.3392	0.5795\\
28.3596	0.6019\\
28.3698	0.6353\\
28.4014	0.6063\\
28.4444	0.6338\\
28.4653	0.6596\\
28.4765	0.6295\\
28.4870	0.6078\\
28.5081	0.6429\\
28.5296	0.6576\\
28.5514	0.6347\\
28.5619	0.6002\\
28.5831	0.6321\\
28.5935	0.5969\\
28.6037	0.6359\\
28.6245	0.5955\\
28.6348	0.6415\\
28.6559	0.6030\\
28.6661	0.5767\\
28.6765	0.6253\\
28.6874	0.5491\\
28.7078	0.5786\\
28.7182	0.5466\\
28.7289	0.5190\\
28.7392	0.5555\\
28.7696	0.5213\\
28.8111	0.5065\\
28.8327	0.4700\\
28.8432	0.4939\\
28.8755	0.4633\\
28.8862	0.4939\\
28.8967	0.4448\\
28.9077	0.4657\\
28.9288	0.4372\\
28.9391	0.4545\\
28.9719	0.4319\\
28.9825	0.3859\\
28.9932	0.4243\\
29.0037	0.3902\\
29.0144	0.4103\\
29.0465	0.3882\\
29.0880	0.3555\\
29.1088	0.3356\\
29.1504	0.3309\\
29.1821	0.2903\\
29.2039	0.2758\\
29.2361	0.2802\\
29.2569	0.2530\\
29.2675	0.2167\\
29.3088	0.2461\\
29.3401	0.2062\\
29.3716	0.1582\\
29.4133	0.1059\\
29.4559	0.0798\\
29.4976	0.0496\\
29.5289	0.0464\\
29.5705	0.0364\\
29.5912	0.0238\\
29.6328	-0.0120\\
29.6647	-0.0383\\
29.7068	-0.0721\\
29.7707	-0.0549\\
29.8028	-0.0317\\
29.8349	-0.0264\\
29.8559	-0.0068\\
29.8882	0.0019\\
29.9205	0.0033\\
29.9426	0.0262\\
29.9963	0.0849\\
};
\addplot [color=mycolor2,solid,forget plot]
  table[row sep=crcr]{%
18.0197	0.1297\\
18.0297	0.1112\\
18.0497	0.0847\\
18.0697	0.0650\\
18.1097	0.0583\\
18.1397	0.0499\\
18.1497	0.0876\\
18.1597	0.1238\\
18.1697	0.1630\\
18.2297	0.2246\\
18.2897	0.2036\\
18.3097	0.1603\\
18.3197	0.1366\\
18.3297	0.1159\\
18.3397	0.0841\\
18.3497	0.0574\\
18.3797	0.0075\\
18.4297	0.0718\\
18.4397	0.0902\\
18.4697	0.1309\\
18.5197	0.1256\\
18.5297	0.0561\\
18.5497	-0.0307\\
18.5797	-0.0715\\
18.5897	-0.1059\\
18.6097	-0.0013\\
18.6197	0.0606\\
18.6297	0.0946\\
18.6497	0.1231\\
18.6797	0.1495\\
18.7097	0.1718\\
18.7397	0.1720\\
18.7497	0.0453\\
18.7697	-0.0355\\
18.7897	-0.1070\\
18.8197	-0.1371\\
18.8497	-0.1394\\
18.8597	-0.1145\\
18.8897	-0.0913\\
18.9097	-0.0467\\
18.9197	-0.0059\\
18.9297	0.0176\\
18.9397	0.0501\\
18.9497	0.0951\\
18.9797	0.1475\\
18.9997	0.1199\\
19.0097	0.0981\\
19.0297	0.0136\\
19.0397	-0.0183\\
19.0497	-0.0970\\
19.0597	-0.1287\\
19.1297	-0.2065\\
19.1697	-0.2306\\
19.1897	-0.2161\\
19.2497	-0.2060\\
19.2797	-0.2070\\
19.3097	-0.2433\\
19.3297	-0.2808\\
19.3397	-0.3203\\
19.3497	-0.3545\\
19.3597	-0.4086\\
19.3697	-0.4531\\
19.4097	-0.5567\\
19.4297	-0.5810\\
19.4697	-0.5676\\
19.4897	-0.5371\\
19.5297	-0.4806\\
19.5497	-0.4606\\
19.5897	-0.4680\\
19.5997	-0.5173\\
19.6097	-0.5452\\
19.6297	-0.6742\\
19.6397	-0.7095\\
19.6497	-0.7645\\
19.6697	-0.8052\\
19.7297	-0.8056\\
19.7497	-0.7658\\
19.7697	-0.7334\\
19.7897	-0.7118\\
19.8197	-0.7754\\
19.8497	-0.7901\\
19.9097	-0.7605\\
19.9297	-0.7816\\
19.9397	-0.8091\\
19.9497	-0.8563\\
19.9697	-0.8933\\
20.0297	-0.9220\\
20.0697	-0.8994\\
20.0897	-0.8710\\
20.1197	-0.8204\\
20.1497	-0.8017\\
20.1697	-0.7536\\
20.1897	-0.7246\\
20.2097	-0.6927\\
20.2297	-0.7951\\
20.2397	-0.8823\\
20.2497	-0.9137\\
20.2597	-0.9421\\
20.2997	-0.9263\\
20.3097	-0.9083\\
20.3197	-0.8815\\
20.3297	-0.8619\\
20.3497	-0.7971\\
20.3597	-0.7703\\
20.3697	-0.7383\\
20.3797	-0.6914\\
20.3897	-0.6627\\
20.4197	-0.6388\\
20.4297	-0.6571\\
20.4397	-0.6858\\
20.4497	-0.7164\\
20.5097	-0.7663\\
20.5297	-0.7839\\
20.5597	-0.8430\\
20.5897	-0.8518\\
20.6197	-0.8127\\
20.6297	-0.7845\\
20.6497	-0.7307\\
20.6897	-0.6970\\
20.7197	-0.7032\\
20.7497	-0.6876\\
20.7897	-0.7201\\
20.7997	-0.6808\\
20.8097	-0.6462\\
20.8297	-0.5754\\
20.8397	-0.5351\\
20.8597	-0.4686\\
20.8697	-0.4412\\
20.8897	-0.3883\\
20.9097	-0.3231\\
20.9197	-0.2925\\
20.9297	-0.2729\\
20.9497	-0.2507\\
20.9597	-0.3032\\
20.9697	-0.3644\\
20.9797	-0.4331\\
20.9897	-0.4901\\
21.0197	-0.5557\\
21.0297	-0.5266\\
21.0397	-0.4920\\
21.0497	-0.4546\\
21.0997	-0.3865\\
21.1097	-0.2260\\
21.1497	-0.0909\\
21.2097	-0.1707\\
21.2297	-0.1871\\
21.2597	-0.1983\\
21.2897	-0.1647\\
21.3497	-0.1277\\
21.4097	-0.1276\\
21.4597	-0.0868\\
21.4697	-0.0553\\
21.4897	0.0189\\
21.4997	0.0390\\
21.5297	0.0740\\
21.5597	0.0456\\
21.5897	0.0670\\
21.6497	0.0848\\
21.6797	0.1290\\
21.6997	0.1040\\
21.7297	0.1000\\
21.7397	0.1239\\
21.7497	0.1635\\
21.7597	0.1981\\
21.7897	0.2397\\
21.7997	0.2072\\
21.8297	0.1450\\
21.8597	0.1003\\
21.8897	0.0751\\
21.9297	0.1113\\
21.9497	0.1496\\
21.9697	0.2039\\
22.0097	0.2434\\
22.0397	0.2452\\
22.0597	0.2634\\
22.0997	0.2735\\
22.1097	0.3009\\
22.1297	0.3240\\
22.1597	0.3531\\
22.1797	0.3695\\
22.1897	0.4015\\
22.2097	0.4713\\
22.2197	0.4424\\
22.2297	0.4098\\
22.2497	0.3755\\
22.2997	0.3619\\
22.3097	0.3327\\
22.3297	0.3079\\
22.3397	0.3267\\
22.3497	0.3558\\
22.3597	0.4094\\
22.3697	0.4998\\
22.3997	0.5848\\
22.4097	0.6350\\
22.4297	0.6629\\
22.4497	0.6313\\
22.4697	0.5990\\
22.4797	0.6419\\
22.5097	0.6659\\
22.5197	0.7293\\
22.5397	0.7565\\
22.5497	0.7071\\
22.5797	0.6352\\
22.6097	0.6301\\
22.6397	0.6314\\
22.6597	0.6545\\
22.6897	0.7030\\
22.7097	0.6637\\
22.7197	0.6337\\
22.7297	0.6050\\
22.7597	0.6036\\
22.7797	0.6171\\
22.7897	0.5995\\
22.8197	0.5708\\
22.8397	0.5443\\
22.8697	0.4581\\
22.8797	0.4269\\
22.9297	0.4178\\
22.9397	0.3913\\
22.9497	0.3673\\
22.9597	0.3412\\
22.9897	0.3132\\
22.9997	0.3498\\
23.0197	0.3871\\
23.0297	0.3500\\
23.0497	0.2804\\
23.0597	0.2518\\
23.0697	0.2322\\
23.0797	0.2767\\
23.0897	0.3952\\
23.1097	0.5495\\
23.1397	0.5704\\
23.1497	0.5458\\
23.1997	0.4547\\
23.2097	0.4205\\
23.2397	0.3287\\
23.2497	0.3694\\
23.2697	0.3879\\
23.2997	0.4092\\
23.3197	0.3790\\
23.3297	0.3614\\
23.3597	0.4734\\
23.3697	0.6106\\
23.3797	0.6448\\
23.3897	0.6774\\
23.4197	0.7308\\
23.4297	0.6904\\
23.4397	0.6605\\
23.4497	0.6409\\
23.4697	0.4799\\
23.4797	0.3065\\
23.4897	0.1864\\
23.4997	0.1465\\
23.5097	0.1104\\
23.5897	0.2033\\
23.5997	0.2281\\
23.6097	0.2743\\
23.6197	0.2964\\
23.6297	0.2529\\
23.6697	0.1767\\
23.6897	0.2038\\
23.7297	0.2117\\
23.7497	0.1958\\
23.7697	0.1403\\
23.7897	0.1009\\
23.8297	-0.0153\\
23.8397	-0.0454\\
23.8697	-0.0983\\
23.8897	-0.1483\\
23.9197	-0.1674\\
23.9497	-0.1964\\
23.9897	-0.2304\\
24.0097	-0.2465\\
24.0197	-0.2769\\
24.0397	-0.3182\\
24.0497	-0.3414\\
24.0697	-0.3664\\
24.0897	-0.3279\\
24.1097	-0.3087\\
24.1497	-0.3545\\
24.1597	-0.3848\\
24.1697	-0.4238\\
24.1997	-0.4624\\
24.2197	-0.4995\\
24.2297	-0.5220\\
24.2497	-0.4968\\
24.2597	-0.4709\\
24.2697	-0.4526\\
24.2797	-0.4233\\
24.2897	-0.3967\\
24.3097	-0.3610\\
24.3297	-0.3085\\
24.3497	-0.2688\\
24.3997	-0.2598\\
24.4097	-0.2866\\
24.4297	-0.3457\\
24.4397	-0.3705\\
24.4497	-0.4056\\
24.4597	-0.4382\\
24.4897	-0.4525\\
24.5297	-0.4271\\
24.5497	-0.4824\\
24.5597	-0.5013\\
24.5697	-0.5418\\
24.5797	-0.5738\\
24.5897	-0.5974\\
24.6197	-0.6581\\
24.6297	-0.6386\\
24.6497	-0.6090\\
24.6797	-0.6246\\
24.6997	-0.6021\\
24.7097	-0.5784\\
24.7497	-0.5585\\
24.7697	-0.6052\\
24.7897	-0.6364\\
24.8097	-0.6659\\
24.8197	-0.6925\\
24.8297	-0.7219\\
24.8697	-0.7435\\
24.8897	-0.7594\\
24.9497	-0.7614\\
24.9897	-0.7553\\
25.0097	-0.7222\\
25.0397	-0.7052\\
25.0697	-0.7443\\
25.1297	-0.8013\\
25.1497	-0.8162\\
25.1797	-0.8314\\
25.2097	-0.8153\\
25.2397	-0.8104\\
25.2497	-0.8321\\
25.2897	-0.8532\\
25.3297	-0.8200\\
25.3397	-0.8002\\
25.3697	-0.7595\\
25.4297	-0.7925\\
25.4797	-0.8016\\
25.5197	-0.7065\\
25.5297	-0.6788\\
25.5397	-0.6384\\
25.5497	-0.6072\\
25.5997	-0.5797\\
25.6097	-0.6154\\
25.6397	-0.6704\\
25.6697	-0.7130\\
25.6997	-0.7192\\
25.7297	-0.7169\\
25.7897	-0.7082\\
25.8197	-0.6990\\
25.8297	-0.6757\\
25.8397	-0.6556\\
25.8497	-0.6178\\
25.8797	-0.6096\\
25.9097	-0.6003\\
25.9497	-0.6279\\
25.9697	-0.6475\\
25.9997	-0.6162\\
26.0697	-0.6306\\
26.0897	-0.5889\\
26.1297	-0.5529\\
26.1497	-0.5237\\
26.2097	-0.4818\\
26.2397	-0.4583\\
26.2597	-0.4400\\
26.3297	-0.4748\\
26.3597	-0.4928\\
26.3697	-0.4734\\
26.3897	-0.4438\\
26.4297	-0.4940\\
26.4497	-0.4574\\
26.4697	-0.4233\\
26.4797	-0.4043\\
26.4897	-0.3834\\
26.5097	-0.3427\\
26.5497	-0.3229\\
26.5697	-0.2892\\
26.6097	-0.3151\\
26.6597	-0.3096\\
26.6897	-0.2844\\
26.7297	-0.2345\\
26.7497	-0.2150\\
26.8297	-0.1840\\
26.8497	-0.1460\\
26.8697	-0.1104\\
26.9497	-0.0802\\
26.9797	-0.0820\\
27.0197	-0.0839\\
27.0497	-0.0793\\
27.0897	-0.0949\\
27.1097	-0.1170\\
27.1397	-0.1060\\
27.1697	-0.1294\\
27.1997	-0.1181\\
27.2297	-0.1318\\
27.2897	-0.1817\\
27.3297	-0.1390\\
27.3497	-0.0912\\
27.3897	-0.1425\\
27.4097	-0.1114\\
27.4297	-0.0268\\
27.4397	0.0234\\
27.4497	0.0865\\
27.4597	0.1674\\
27.4697	0.1957\\
27.4897	0.2379\\
27.5097	0.1336\\
27.5197	0.0924\\
27.5497	0.0663\\
27.5797	0.0491\\
27.5897	0.0317\\
27.6297	0.0291\\
27.6497	0.0561\\
27.6697	0.0881\\
27.6797	0.1177\\
27.6897	0.1433\\
27.6997	0.1616\\
27.7097	0.1982\\
27.7297	0.2363\\
27.7697	0.2597\\
27.7997	0.2742\\
27.8297	0.3064\\
27.8797	0.3122\\
27.8897	0.2698\\
27.9497	0.3858\\
27.9997	0.4333\\
28.0097	0.4046\\
28.0397	0.3688\\
28.0697	0.3990\\
28.0997	0.3926\\
28.1297	0.3957\\
28.1497	0.3608\\
28.1897	0.3260\\
28.2297	0.3743\\
28.2397	0.4010\\
28.2497	0.4229\\
28.2897	0.4182\\
28.3097	0.4448\\
28.3497	0.3641\\
28.3597	0.3851\\
28.3697	0.3568\\
28.3897	0.4011\\
28.4197	0.4895\\
28.4297	0.5140\\
28.4597	0.5250\\
28.4797	0.5120\\
28.4897	0.5494\\
28.5097	0.5321\\
28.5397	0.5781\\
28.5497	0.6017\\
28.5997	0.6547\\
28.6297	0.6322\\
28.6397	0.5977\\
28.6697	0.5761\\
28.6997	0.5738\\
28.7197	0.5529\\
28.7497	0.5921\\
28.7897	0.5683\\
28.8097	0.5303\\
28.8197	0.5551\\
28.8497	0.5604\\
28.8997	0.5302\\
28.9297	0.5112\\
28.9497	0.4707\\
28.9697	0.4561\\
28.9997	0.4674\\
29.0297	0.4627\\
29.0797	0.4842\\
29.1197	0.4476\\
29.1297	0.4219\\
29.2097	0.3467\\
29.2497	0.2962\\
29.2597	0.3138\\
29.2897	0.3312\\
29.3097	0.3154\\
29.3197	0.3455\\
29.3297	0.3700\\
29.3697	0.4145\\
29.3797	0.3861\\
29.3897	0.3609\\
29.4097	0.2860\\
29.4297	0.2281\\
29.4397	0.2081\\
29.4497	0.1874\\
29.5097	0.1326\\
29.5497	0.1680\\
29.5697	0.1806\\
29.6097	0.2138\\
29.6197	0.1893\\
29.6297	0.1548\\
29.6797	0.1074\\
29.6897	0.1261\\
29.7097	0.0896\\
29.7197	0.0594\\
29.7397	0.0006\\
29.7497	-0.0205\\
29.7797	-0.0508\\
29.7997	-0.0377\\
29.8297	-0.0578\\
29.8697	-0.0288\\
29.9997	-0.0479\\
};
\end{axis}

\begin{axis}[%
width=0.974\fwidth,
height=0.203\fwidth,
at={(0\fwidth,0.321\fwidth)},
scale only axis,
xmin=18.0242,
xmax=29.9997,
xlabel={$t$ [s]},
ymin=-1.1000,
ymax=1.1000,
ylabel={$\vartheta_y$, $\hat{\vartheta}_y$ [1/s]},
axis background/.style={fill=white},
title style={font=\labelsize},
xlabel style={font=\labelsize,at={(axis description cs:0.5,\xlabeldist)}},
ylabel style={font=\labelsize,at={(axis description cs:\ylabeldist,0.5)}},
legend style={font=\ticksize},
ticklabel style={font=\ticksize}
]
\addplot [color=mycolor1,solid,forget plot]
  table[row sep=crcr]{%
18.0242	0.1377\\
18.0555	0.1443\\
18.0768	0.1655\\
18.1401	0.1945\\
18.1610	0.2049\\
18.1935	0.2129\\
18.2257	0.2382\\
18.2462	0.2600\\
18.2881	0.2588\\
18.3091	0.2757\\
18.3307	0.2948\\
18.3725	0.3104\\
18.4038	0.3208\\
18.4454	0.3391\\
18.4767	0.3516\\
18.5089	0.3659\\
18.5413	0.3732\\
18.5829	0.3650\\
18.5937	0.3783\\
18.6151	0.3925\\
18.6675	0.4172\\
18.6985	0.4175\\
18.7196	0.4309\\
18.7611	0.4290\\
18.8029	0.4488\\
18.8448	0.4748\\
18.8769	0.4891\\
18.9080	0.4848\\
18.9392	0.4927\\
18.9819	0.4873\\
19.0239	0.4545\\
19.0551	0.4310\\
19.0763	0.4160\\
19.1082	0.4201\\
19.1711	0.4072\\
19.1928	0.4203\\
19.2248	0.3946\\
19.2572	0.3763\\
19.2781	0.3657\\
19.3197	0.3302\\
19.3404	0.3069\\
19.3616	0.2911\\
19.3927	0.2695\\
19.4143	0.2533\\
19.4571	0.2235\\
19.4999	0.1993\\
19.5415	0.1781\\
19.6058	0.1663\\
19.6271	0.1614\\
19.6699	0.1455\\
19.7013	0.1259\\
19.7438	0.1155\\
19.7749	0.1188\\
19.7959	0.1377\\
19.8376	0.1240\\
19.8898	0.1100\\
19.9417	0.0930\\
19.9731	0.0818\\
20.0358	0.0839\\
20.0562	0.0667\\
20.0878	0.0734\\
20.1086	0.0618\\
20.1400	0.0583\\
20.1713	0.0551\\
20.1931	0.0412\\
20.2243	0.0278\\
20.2555	0.0087\\
20.2867	-0.0086\\
20.3190	-0.0112\\
20.3502	-0.0350\\
20.3713	-0.0453\\
20.4037	-0.0738\\
20.4359	-0.0971\\
20.4681	-0.1234\\
20.4889	-0.1450\\
20.5202	-0.1493\\
20.5408	-0.1688\\
20.5621	-0.1844\\
20.6046	-0.1957\\
20.6463	-0.2161\\
20.6567	-0.2359\\
20.6889	-0.2315\\
20.7420	-0.2466\\
20.7630	-0.2380\\
20.8047	-0.2265\\
20.8258	-0.2123\\
20.8693	-0.2073\\
20.9217	-0.1868\\
20.9433	-0.2178\\
20.9540	-0.2371\\
20.9860	-0.2301\\
21.0068	-0.2468\\
21.0595	-0.2905\\
21.0705	-0.2967\\
21.0808	-0.3205\\
21.0911	-0.3021\\
21.1122	-0.3191\\
21.1328	-0.3361\\
21.1756	-0.3704\\
21.1968	-0.3975\\
21.2173	-0.4287\\
21.2278	-0.4599\\
21.2385	-0.4816\\
21.2714	-0.5238\\
21.3029	-0.5621\\
21.3245	-0.5866\\
21.3352	-0.6127\\
21.3560	-0.6269\\
21.4184	-0.6268\\
21.4392	-0.6418\\
21.5018	-0.6568\\
21.5236	-0.6755\\
21.5757	-0.6793\\
21.6071	-0.6891\\
21.6495	-0.7252\\
21.6924	-0.7457\\
21.7030	-0.7380\\
21.7341	-0.7635\\
21.7758	-0.7913\\
21.8176	-0.8091\\
21.8388	-0.8357\\
21.8708	-0.8631\\
21.9241	-0.8480\\
21.9343	-0.8840\\
21.9447	-0.8551\\
21.9552	-0.8840\\
21.9970	-0.8641\\
22.0283	-0.8887\\
22.0387	-0.9081\\
22.0712	-0.9248\\
22.0927	-0.9032\\
22.1250	-0.8847\\
22.1676	-0.8301\\
22.1886	-0.7872\\
22.2198	-0.7689\\
22.2509	-0.7348\\
22.2720	-0.7143\\
22.2926	-0.6770\\
22.3032	-0.6452\\
22.3344	-0.6090\\
22.3452	-0.5771\\
22.3666	-0.5902\\
22.3775	-0.5669\\
22.3877	-0.5442\\
22.4192	-0.4883\\
22.4616	-0.4421\\
22.4720	-0.4067\\
22.4926	-0.3921\\
22.5135	-0.3706\\
22.5240	-0.3327\\
22.5554	-0.2993\\
22.5972	-0.2836\\
22.6181	-0.2566\\
22.6492	-0.2250\\
22.6806	-0.1883\\
22.6911	-0.1627\\
22.7119	-0.1271\\
22.7327	-0.1164\\
22.7536	-0.1006\\
22.7647	-0.0698\\
22.7860	-0.0548\\
22.8075	0.0074\\
22.8180	-0.0115\\
22.8596	0.0585\\
22.8807	0.0724\\
22.9013	0.1183\\
22.9228	0.1599\\
22.9444	0.2060\\
22.9861	0.2275\\
23.0076	0.2482\\
23.0390	0.2732\\
23.0597	0.3241\\
23.0806	0.3433\\
23.1228	0.3989\\
23.1336	0.4299\\
23.1443	0.4488\\
23.1547	0.4781\\
23.1652	0.4584\\
23.1861	0.4745\\
23.1966	0.4992\\
23.2277	0.5350\\
23.2483	0.5759\\
23.2697	0.5919\\
23.2911	0.6319\\
23.3233	0.6578\\
23.3756	0.6813\\
23.3861	0.6712\\
23.4172	0.6481\\
23.4589	0.6394\\
23.4798	0.6270\\
23.5007	0.6164\\
23.5643	0.6329\\
23.5956	0.6197\\
23.6590	0.6002\\
23.6697	0.6126\\
23.6799	0.5948\\
23.7321	0.5604\\
23.7644	0.5391\\
23.7957	0.5211\\
23.8272	0.4941\\
23.8376	0.4748\\
23.8479	0.4461\\
23.8895	0.4432\\
23.9227	0.4692\\
23.9541	0.4578\\
23.9967	0.4490\\
24.0274	0.4367\\
24.0914	0.4273\\
24.1124	0.4146\\
24.1543	0.4106\\
24.1852	0.4064\\
24.2165	0.4079\\
24.2478	0.3942\\
24.2683	0.3802\\
24.2997	0.3687\\
24.3210	0.3528\\
24.3426	0.3302\\
24.3644	0.3211\\
24.3959	0.2970\\
24.4164	0.2827\\
24.4581	0.2728\\
24.4895	0.2661\\
24.5101	0.2482\\
24.5414	0.2159\\
24.5729	0.1994\\
24.6258	0.1900\\
24.6573	0.1832\\
24.6791	0.1720\\
24.7101	0.1618\\
24.7520	0.1523\\
24.7832	0.1441\\
24.8147	0.1334\\
24.8467	0.1315\\
24.8791	0.1233\\
24.9104	0.1193\\
24.9624	0.1129\\
25.0042	0.1095\\
25.0572	0.0956\\
25.0895	0.0995\\
25.1436	0.0986\\
25.1863	0.0895\\
25.2184	0.0790\\
25.2812	0.0582\\
25.3124	0.0350\\
25.3542	0.0163\\
25.4061	-0.0089\\
25.4698	-0.0136\\
25.4905	-0.0265\\
25.5320	-0.0402\\
25.5652	-0.0375\\
25.6175	-0.0433\\
25.6600	-0.0706\\
25.7127	-0.0853\\
25.7434	-0.1032\\
25.7958	-0.1128\\
25.8384	-0.1369\\
25.8700	-0.1303\\
25.9018	-0.1477\\
25.9230	-0.1591\\
25.9761	-0.1602\\
25.9968	-0.1485\\
26.0393	-0.1529\\
26.0600	-0.1427\\
26.0817	-0.1807\\
26.1239	-0.1863\\
26.1559	-0.1924\\
26.1768	-0.2160\\
26.1980	-0.2292\\
26.2088	-0.2472\\
26.2301	-0.2713\\
26.2613	-0.2682\\
26.3247	-0.2822\\
26.3458	-0.2673\\
26.3562	-0.2851\\
26.4101	-0.2647\\
26.4206	-0.2840\\
26.4412	-0.2696\\
26.4729	-0.2693\\
26.5246	-0.2746\\
26.5883	-0.2793\\
26.6300	-0.2780\\
26.6612	-0.2547\\
26.6716	-0.2472\\
26.6923	-0.2580\\
26.7461	-0.2553\\
26.7672	-0.2702\\
26.7885	-0.2924\\
26.8311	-0.3082\\
26.8517	-0.3265\\
26.8727	-0.3389\\
26.9143	-0.3725\\
26.9361	-0.4056\\
26.9465	-0.4277\\
26.9569	-0.4097\\
26.9884	-0.4463\\
26.9994	-0.4692\\
27.0205	-0.4964\\
27.0415	-0.5226\\
27.0621	-0.5480\\
27.0725	-0.5695\\
27.0829	-0.5879\\
27.1245	-0.6444\\
27.1673	-0.7068\\
27.1779	-0.7257\\
27.1987	-0.7442\\
27.2198	-0.7343\\
27.2626	-0.7771\\
27.2736	-0.8076\\
27.3048	-0.8371\\
27.3364	-0.8650\\
27.3580	-0.8428\\
27.3788	-0.8653\\
27.4099	-0.8571\\
27.4208	-0.8393\\
27.4423	-0.8818\\
27.4737	-0.8914\\
27.5258	-0.8536\\
27.5361	-0.8746\\
27.5571	-0.8479\\
27.5892	-0.8535\\
27.6205	-0.8372\\
27.6413	-0.8242\\
27.6736	-0.8262\\
27.6838	-0.8464\\
27.7256	-0.8357\\
27.7572	-0.7915\\
27.7681	-0.7740\\
27.8005	-0.8087\\
27.8226	-0.7874\\
27.8755	-0.7562\\
27.9081	-0.7477\\
27.9393	-0.7249\\
27.9600	-0.6859\\
27.9706	-0.7128\\
27.9817	-0.6781\\
27.9923	-0.7064\\
28.0245	-0.6758\\
28.0457	-0.6440\\
28.0777	-0.6124\\
28.0985	-0.5889\\
28.1091	-0.5712\\
28.1194	-0.5517\\
28.1505	-0.5235\\
28.1715	-0.4746\\
28.2030	-0.4588\\
28.2235	-0.4714\\
28.2444	-0.4498\\
28.2871	-0.3887\\
28.2975	-0.3664\\
28.3079	-0.3483\\
28.3183	-0.3747\\
28.3392	-0.3390\\
28.3807	-0.3140\\
28.3911	-0.2894\\
28.4014	-0.3178\\
28.4118	-0.2634\\
28.4228	-0.3038\\
28.4870	-0.2429\\
28.4973	-0.2262\\
28.5081	-0.2005\\
28.5296	-0.2386\\
28.5403	-0.1717\\
28.5619	-0.2144\\
28.6037	-0.1384\\
28.6245	-0.1001\\
28.6348	-0.0817\\
28.6559	-0.1153\\
28.6661	-0.0703\\
28.6765	-0.0436\\
28.6874	-0.0798\\
28.7182	-0.0333\\
28.7289	-0.0538\\
28.7494	-0.0367\\
28.7595	0.0169\\
28.7696	-0.0005\\
28.7902	0.0430\\
28.8005	0.0238\\
28.8111	0.0512\\
28.8215	0.0216\\
28.8541	0.0531\\
28.8755	0.0847\\
28.9077	0.1220\\
28.9183	0.1114\\
28.9288	0.0900\\
28.9391	0.1345\\
28.9719	0.1468\\
28.9825	0.1225\\
29.0037	0.1420\\
29.0255	0.1635\\
29.0465	0.1965\\
29.0672	0.1567\\
29.0778	0.1755\\
29.1088	0.2044\\
29.1401	0.2090\\
29.1504	0.2375\\
29.1607	0.1782\\
29.1716	0.2072\\
29.2039	0.2178\\
29.2361	0.2500\\
29.2569	0.2680\\
29.2675	0.2344\\
29.2883	0.2891\\
29.3196	0.2811\\
29.3401	0.2957\\
29.3716	0.3070\\
29.4237	0.3385\\
29.4452	0.3456\\
29.4559	0.3847\\
29.4764	0.3617\\
29.5077	0.3739\\
29.5289	0.3625\\
29.5494	0.3845\\
29.5807	0.3561\\
29.6328	0.3791\\
29.6647	0.3853\\
29.6862	0.4151\\
29.7068	0.4339\\
29.7598	0.4138\\
29.7924	0.4179\\
29.8349	0.4198\\
29.8559	0.4044\\
29.8882	0.4067\\
29.9316	0.4048\\
29.9528	0.4554\\
29.9963	0.4973\\
};
\addplot [color=mycolor2,solid,forget plot]
  table[row sep=crcr]{%
18.0497	0.1544\\
18.1097	0.1574\\
18.1397	0.1299\\
18.1697	0.1082\\
18.1897	0.1277\\
18.1997	0.1542\\
18.2097	0.1800\\
18.2197	0.2085\\
18.2297	0.2343\\
18.2897	0.2468\\
18.3497	0.2479\\
18.3897	0.2548\\
18.4097	0.2782\\
18.4697	0.3129\\
18.5497	0.3086\\
18.5797	0.3462\\
18.6097	0.3619\\
18.6397	0.3420\\
18.6497	0.3600\\
18.6797	0.3827\\
18.7097	0.3754\\
18.7397	0.3685\\
18.7497	0.4292\\
18.7697	0.4862\\
18.7997	0.4830\\
18.8297	0.4532\\
18.8597	0.4562\\
18.8897	0.4384\\
18.9197	0.4426\\
18.9497	0.4404\\
18.9797	0.4157\\
19.0097	0.4392\\
19.0397	0.4997\\
19.0697	0.5381\\
19.0997	0.4954\\
19.1297	0.4574\\
19.1597	0.4669\\
19.1797	0.4418\\
19.1897	0.4161\\
19.2297	0.3868\\
19.2497	0.4039\\
19.2697	0.4284\\
19.3097	0.4718\\
19.3497	0.4603\\
19.3697	0.4394\\
19.4097	0.4193\\
19.4197	0.3995\\
19.4297	0.3738\\
19.4497	0.3214\\
19.4597	0.2865\\
19.4897	0.2521\\
19.5297	0.2941\\
19.5497	0.3366\\
19.5897	0.3470\\
19.6097	0.2991\\
19.6497	0.2622\\
19.6597	0.2821\\
19.6697	0.3003\\
19.6897	0.2630\\
19.6997	0.2332\\
19.7197	0.2156\\
19.7497	0.2365\\
19.7697	0.2118\\
19.7797	0.1761\\
19.7897	0.1411\\
19.8497	0.0905\\
19.8797	0.1202\\
19.8897	0.1465\\
19.8997	0.1756\\
19.9297	0.1936\\
19.9497	0.1791\\
19.9597	0.1603\\
19.9697	0.1378\\
19.9897	0.0768\\
19.9997	0.0380\\
20.0097	0.0187\\
20.0297	-0.0134\\
20.0597	0.0102\\
20.0897	0.0126\\
20.1197	0.0083\\
20.1297	0.0347\\
20.1497	0.0731\\
20.1997	0.0243\\
20.2297	0.0484\\
20.2397	0.0687\\
20.2497	0.1164\\
20.2597	0.1549\\
20.2697	0.2056\\
20.3097	0.2599\\
20.3197	0.2340\\
20.3297	0.1966\\
20.3497	0.1548\\
20.3597	0.1310\\
20.3897	0.0904\\
20.4197	0.0744\\
20.4497	0.1122\\
20.5097	0.1520\\
20.5697	0.1003\\
20.5897	0.0823\\
20.5997	0.0468\\
20.6297	0.0112\\
20.6497	0.0440\\
20.6697	0.0735\\
20.6897	0.0556\\
20.7097	0.0294\\
20.7297	-0.0022\\
20.7497	-0.0307\\
20.7697	-0.0555\\
20.7897	-0.0946\\
20.7997	-0.1185\\
20.8097	-0.1380\\
20.8297	-0.1922\\
20.8397	-0.2360\\
20.8697	-0.2624\\
20.8997	-0.2230\\
20.9297	-0.2471\\
20.9697	-0.0331\\
20.9797	0.0544\\
20.9897	0.0777\\
21.0197	0.0219\\
21.0297	-0.0544\\
21.0397	-0.1271\\
21.0497	-0.2166\\
21.0697	-0.3548\\
21.0997	-0.3923\\
21.1497	-0.3624\\
21.1697	-0.3920\\
21.2297	-0.4359\\
21.2597	-0.4479\\
21.2897	-0.4818\\
21.3497	-0.5066\\
21.3897	-0.5132\\
21.4097	-0.5382\\
21.4497	-0.5961\\
21.4697	-0.6090\\
21.4897	-0.6596\\
21.4997	-0.6859\\
21.5297	-0.7284\\
21.5597	-0.7256\\
21.5897	-0.7179\\
21.6397	-0.7118\\
21.6497	-0.7330\\
21.6797	-0.7626\\
21.6997	-0.7413\\
21.7297	-0.7187\\
21.7397	-0.7405\\
21.7497	-0.7700\\
21.7697	-0.7935\\
21.8097	-0.8438\\
21.8497	-0.8670\\
21.8797	-0.8747\\
21.9097	-0.8894\\
21.9297	-0.9140\\
21.9497	-0.9396\\
22.0097	-0.9321\\
22.0597	-0.9432\\
22.0697	-0.9225\\
22.0997	-0.8886\\
22.1197	-0.8757\\
22.1497	-0.8739\\
22.1897	-0.8989\\
22.2197	-0.7882\\
22.2297	-0.7384\\
22.2397	-0.7107\\
22.2497	-0.6917\\
22.2797	-0.6777\\
22.2897	-0.6971\\
22.3097	-0.7388\\
22.3297	-0.7740\\
22.3397	-0.8028\\
22.3497	-0.8209\\
22.3597	-0.7853\\
22.3697	-0.7287\\
22.4097	-0.6949\\
22.4497	-0.7176\\
22.4597	-0.6787\\
22.4697	-0.6495\\
22.4797	-0.5950\\
22.4897	-0.5499\\
22.5097	-0.5150\\
22.5197	-0.4406\\
22.5697	-0.4077\\
22.5797	-0.3743\\
22.5897	-0.3191\\
22.5997	-0.2814\\
22.6097	-0.2567\\
22.6397	-0.1604\\
22.6697	-0.1232\\
22.6897	-0.1470\\
22.6997	-0.1788\\
22.7197	-0.1926\\
22.7297	-0.1715\\
22.7497	-0.1503\\
22.7597	-0.1230\\
22.7797	-0.0917\\
22.7897	-0.0698\\
22.8097	-0.0942\\
22.8497	-0.0694\\
22.8697	0.0058\\
22.8797	0.0477\\
22.8897	0.0696\\
22.8997	0.0883\\
22.9097	0.1136\\
22.9297	0.1014\\
22.9397	0.0698\\
22.9697	0.0561\\
22.9897	0.1194\\
22.9997	0.1703\\
23.0097	0.2409\\
23.0197	0.2848\\
23.0297	0.3159\\
23.0497	0.3319\\
23.0597	0.2983\\
23.0697	0.2578\\
23.0797	0.1865\\
23.0897	0.1039\\
23.1097	0.1363\\
23.1197	0.2146\\
23.1297	0.2894\\
23.1497	0.3437\\
23.1997	0.3917\\
23.2097	0.4099\\
23.2497	0.5063\\
23.2597	0.5242\\
23.2697	0.5775\\
23.2997	0.6520\\
23.3297	0.6530\\
23.3497	0.6224\\
23.3597	0.5189\\
23.3697	0.4293\\
23.3897	0.4004\\
23.4097	0.5061\\
23.4197	0.5353\\
23.4497	0.5577\\
23.4697	0.6082\\
23.4797	0.6755\\
23.5097	0.7185\\
23.5397	0.6519\\
23.5697	0.6298\\
23.6097	0.6174\\
23.6297	0.6376\\
23.6597	0.6548\\
23.6797	0.6354\\
23.6897	0.6078\\
23.7197	0.5830\\
23.7497	0.5698\\
23.7897	0.5339\\
23.8097	0.5025\\
23.8297	0.4645\\
23.8397	0.4398\\
23.8497	0.4178\\
23.8697	0.3783\\
23.8897	0.4208\\
23.8997	0.4516\\
23.9297	0.4950\\
23.9497	0.4781\\
23.9597	0.4477\\
23.9897	0.3946\\
24.0497	0.3995\\
24.0797	0.3453\\
24.0897	0.3802\\
24.0997	0.4053\\
24.1297	0.4368\\
24.1497	0.4492\\
24.1697	0.4191\\
24.1997	0.4139\\
24.2197	0.3988\\
24.2297	0.3503\\
24.2597	0.4044\\
24.2897	0.4203\\
24.3197	0.4148\\
24.3397	0.4437\\
24.3797	0.4547\\
24.4097	0.4440\\
24.4397	0.4680\\
24.4497	0.4336\\
24.4697	0.4103\\
24.5297	0.4294\\
24.5597	0.3737\\
24.5697	0.3378\\
24.5897	0.2946\\
24.6097	0.2695\\
24.6297	0.2504\\
24.6497	0.2795\\
24.6897	0.2392\\
24.7097	0.2248\\
24.7597	0.2140\\
24.7897	0.2383\\
24.8297	0.2801\\
24.8497	0.2517\\
24.8597	0.2175\\
24.8697	0.1823\\
24.8897	0.1542\\
24.9497	0.1511\\
24.9897	0.1679\\
25.0097	0.1488\\
25.0397	0.1410\\
25.0897	0.1246\\
25.0997	0.0966\\
25.1097	0.0622\\
25.1197	0.0329\\
25.1297	0.0143\\
25.1497	0.0370\\
25.1697	0.0751\\
25.1797	0.0955\\
25.2097	0.1277\\
25.2297	0.1047\\
25.2397	0.0871\\
25.2497	0.0569\\
25.2697	-0.0304\\
25.2897	-0.0666\\
25.3097	-0.0133\\
25.3597	0.0353\\
25.3697	-0.0068\\
25.3897	-0.0431\\
25.3997	-0.0097\\
25.4097	0.0309\\
25.4297	0.1017\\
25.4697	0.1360\\
25.4797	0.1143\\
25.4897	0.0912\\
25.5197	0.0555\\
25.5497	0.0333\\
25.5697	0.0482\\
25.5997	0.0301\\
25.6097	0.0550\\
25.6297	0.1147\\
25.6397	0.1494\\
25.6497	0.1811\\
25.6697	0.2055\\
25.6897	0.1807\\
25.6997	0.1458\\
25.7297	0.1114\\
25.7697	0.1110\\
25.7797	0.0928\\
25.7897	0.0683\\
25.8297	0.0183\\
25.8497	0.0451\\
25.8697	0.0158\\
25.8797	-0.0126\\
25.8897	-0.0422\\
25.8997	-0.0615\\
25.9297	-0.0755\\
25.9397	-0.0491\\
25.9597	-0.0083\\
25.9997	-0.0138\\
26.0297	-0.0455\\
26.0697	-0.0891\\
26.0897	-0.0552\\
26.1297	-0.0467\\
26.1497	-0.0678\\
26.1897	-0.0703\\
26.2097	-0.1021\\
26.2397	-0.1188\\
26.2597	-0.1471\\
26.2997	-0.1621\\
26.3097	-0.1875\\
26.3297	-0.2133\\
26.3697	-0.2568\\
26.3797	-0.3061\\
26.3897	-0.3631\\
26.4097	-0.4099\\
26.4197	-0.3756\\
26.4397	-0.3332\\
26.4497	-0.3157\\
26.5097	-0.2975\\
26.5697	-0.2935\\
26.6097	-0.3312\\
26.6297	-0.3146\\
26.6497	-0.3296\\
26.6797	-0.3579\\
26.7197	-0.3566\\
26.7497	-0.3278\\
26.7897	-0.2948\\
26.8097	-0.3085\\
26.8297	-0.3235\\
26.8497	-0.3538\\
26.8697	-0.3347\\
26.9497	-0.3200\\
26.9797	-0.3198\\
27.0197	-0.3199\\
27.0497	-0.3519\\
27.0697	-0.4013\\
27.0897	-0.4340\\
27.1097	-0.4622\\
27.1597	-0.5538\\
27.1997	-0.5772\\
27.2097	-0.6165\\
27.2297	-0.6641\\
27.2497	-0.7024\\
27.2597	-0.7398\\
27.2797	-0.7612\\
27.3397	-0.7695\\
27.3497	-0.7988\\
27.4097	-0.8647\\
27.4397	-0.8654\\
27.4697	-0.8508\\
27.4997	-0.8579\\
27.5297	-0.8302\\
27.5597	-0.8181\\
27.5797	-0.8041\\
27.6197	-0.7883\\
27.6297	-0.8100\\
27.6397	-0.8300\\
27.6697	-0.8551\\
27.6797	-0.8317\\
27.7097	-0.8002\\
27.7397	-0.7599\\
27.7697	-0.7429\\
27.7997	-0.7787\\
27.8097	-0.7963\\
27.8297	-0.8120\\
27.8597	-0.7989\\
27.8797	-0.7820\\
27.8897	-0.8182\\
27.9297	-0.8044\\
27.9497	-0.7687\\
27.9897	-0.6987\\
28.0097	-0.6668\\
28.0397	-0.6746\\
28.0497	-0.6488\\
28.0597	-0.6236\\
28.0697	-0.5953\\
28.1197	-0.5371\\
28.1797	-0.5789\\
28.1897	-0.5501\\
28.2097	-0.4936\\
28.2297	-0.4449\\
28.2497	-0.4336\\
28.2997	-0.5039\\
28.3097	-0.5314\\
28.3397	-0.5039\\
28.3597	-0.4779\\
28.3897	-0.4960\\
28.4097	-0.4450\\
28.4197	-0.4097\\
28.4497	-0.3804\\
28.4797	-0.3421\\
28.5197	-0.2960\\
28.5397	-0.2333\\
28.5497	-0.1857\\
28.6097	-0.1222\\
28.6397	-0.1491\\
28.6497	-0.1130\\
28.6697	-0.0789\\
28.6997	-0.0812\\
28.7197	-0.1049\\
28.7297	-0.0776\\
28.7797	-0.0202\\
28.8497	-0.0383\\
28.8697	-0.0537\\
28.8897	-0.0349\\
28.9097	-0.0179\\
28.9697	0.0143\\
28.9897	0.0853\\
29.0097	0.1254\\
29.0297	0.1454\\
29.0597	0.1581\\
29.0797	0.1829\\
29.0897	0.1522\\
29.1397	0.1116\\
29.1497	0.0929\\
29.1697	0.1668\\
29.1997	0.1738\\
29.2097	0.2040\\
29.2297	0.2432\\
29.2497	0.2805\\
29.2697	0.3063\\
29.3197	0.2516\\
29.3497	0.1815\\
29.3897	0.1558\\
29.4097	0.2131\\
29.4197	0.2555\\
29.4297	0.3000\\
29.4397	0.3477\\
29.4797	0.3581\\
29.5097	0.3567\\
29.5497	0.4306\\
29.5597	0.3991\\
29.6097	0.3785\\
29.6297	0.3599\\
29.6497	0.3432\\
29.6597	0.3721\\
29.6697	0.4218\\
29.6797	0.4506\\
29.7197	0.4640\\
29.7397	0.4283\\
29.7897	0.4243\\
29.8097	0.4413\\
29.8597	0.4142\\
29.8697	0.4321\\
29.8997	0.4591\\
29.9997	0.3352\\
};
\end{axis}

\begin{axis}[%
width=0.974\fwidth,
height=0.203\fwidth,
at={(0\fwidth,0\fwidth)},
scale only axis,
xmin=18.0242,
xmax=29.9997,
xlabel={$t$ [s]},
ymin=-1.1000,
ymax=1.1000,
ylabel={$\vartheta_z$, $\hat{\vartheta}_z$ [1/s]},
axis background/.style={fill=white},
title style={font=\labelsize},
xlabel style={font=\labelsize,at={(axis description cs:0.5,\xlabeldist)}},
ylabel style={font=\labelsize,at={(axis description cs:\ylabeldist,0.5)}},
legend style={font=\ticksize},
ticklabel style={font=\ticksize}
]
\addplot [color=mycolor1,solid,forget plot]
  table[row sep=crcr]{%
18.0242	0.1116\\
18.0555	0.1110\\
18.0768	0.1078\\
18.1192	0.1133\\
18.1508	0.1153\\
18.1935	0.1186\\
18.2257	0.1111\\
18.2462	0.1044\\
18.2881	0.0886\\
18.3091	0.0762\\
18.3307	0.0641\\
18.3620	0.0472\\
18.3932	0.0337\\
18.4141	0.0278\\
18.4556	0.0148\\
18.4872	0.0091\\
18.5089	0.0026\\
18.5517	-0.0113\\
18.5728	-0.0029\\
18.5937	0.0087\\
18.6363	0.0119\\
18.6675	0.0096\\
18.7090	0.0216\\
18.7402	0.0165\\
18.7611	0.0129\\
18.8029	0.0030\\
18.8342	-0.0102\\
18.8769	-0.0078\\
18.8977	-0.0241\\
18.9188	-0.0452\\
18.9392	-0.0608\\
18.9819	-0.0743\\
19.0132	-0.0720\\
19.0446	-0.0590\\
19.0763	-0.0460\\
19.1082	-0.0561\\
19.1496	-0.0620\\
19.1711	-0.0676\\
19.1928	-0.0797\\
19.2356	-0.0865\\
19.2678	-0.0884\\
19.3095	-0.0952\\
19.3404	-0.1000\\
19.3616	-0.0980\\
19.4037	-0.0956\\
19.4353	-0.1043\\
19.4790	-0.1167\\
19.5104	-0.1280\\
19.5415	-0.1288\\
19.5845	-0.1238\\
19.6058	-0.1138\\
19.6271	-0.0962\\
19.6595	-0.0680\\
19.6803	-0.0546\\
19.7117	-0.0334\\
19.7544	-0.0173\\
19.7857	-0.0061\\
19.8274	-0.0097\\
19.8583	-0.0256\\
19.8898	-0.0306\\
19.9313	-0.0290\\
19.9628	-0.0346\\
20.0045	-0.0325\\
20.0358	-0.0340\\
20.0562	-0.0324\\
20.0981	-0.0404\\
20.1293	-0.0410\\
20.1713	-0.0316\\
20.2033	-0.0269\\
20.2243	-0.0280\\
20.2658	-0.0272\\
20.2978	-0.0194\\
20.3190	-0.0086\\
20.3608	-0.0062\\
20.3932	-0.0149\\
20.4359	-0.0201\\
20.4681	-0.0279\\
20.4889	-0.0354\\
20.5305	-0.0351\\
20.5621	-0.0309\\
20.6046	-0.0293\\
20.6359	-0.0256\\
20.6567	-0.0242\\
20.6993	-0.0209\\
20.7316	-0.0100\\
20.7733	-0.0042\\
20.8047	-0.0128\\
20.8258	-0.0193\\
20.8369	-0.0249\\
20.8796	-0.0469\\
20.9114	-0.0340\\
20.9217	-0.0337\\
20.9650	-0.0525\\
20.9966	-0.0558\\
21.0278	-0.0460\\
21.0493	-0.0309\\
21.0808	-0.0234\\
21.1226	-0.0195\\
21.1540	-0.0110\\
21.1756	-0.0107\\
21.2173	-0.0231\\
21.2495	-0.0222\\
21.2714	-0.0115\\
21.3029	-0.0048\\
21.3352	-0.0184\\
21.3560	-0.0120\\
21.3977	-0.0016\\
21.4289	0.0055\\
21.4705	0.0196\\
21.5018	0.0161\\
21.5236	0.0185\\
21.5654	0.0190\\
21.5865	0.0179\\
21.6071	0.0372\\
21.6495	0.0400\\
21.6717	0.0298\\
21.6924	0.0437\\
21.7341	0.0682\\
21.7654	0.0778\\
21.7865	0.0783\\
21.8176	0.0953\\
21.8496	0.1034\\
21.8708	0.1019\\
21.9133	0.1040\\
21.9343	0.1188\\
21.9552	0.1462\\
21.9970	0.1505\\
22.0177	0.1420\\
22.0387	0.1557\\
22.0712	0.1582\\
22.0927	0.1433\\
22.1145	0.1361\\
22.1250	0.1443\\
22.1565	0.1562\\
22.1782	0.1331\\
22.2095	0.1115\\
22.2198	0.0964\\
22.2509	0.0792\\
22.2822	0.0774\\
22.3032	0.0820\\
22.3452	0.0782\\
22.3775	0.0754\\
22.4085	0.0621\\
22.4296	0.0545\\
22.4616	0.0390\\
22.5034	0.0508\\
22.5240	0.0398\\
22.5451	0.0144\\
22.5554	0.0018\\
22.5869	-0.0414\\
22.6075	-0.0366\\
22.6390	-0.0202\\
22.6806	-0.0140\\
22.7013	-0.0294\\
22.7224	-0.0536\\
22.7327	-0.0728\\
22.7647	-0.1061\\
22.7964	-0.1111\\
22.8180	-0.1057\\
22.8493	-0.1159\\
22.8807	-0.1426\\
22.9013	-0.1594\\
22.9228	-0.2002\\
22.9444	-0.2376\\
22.9650	-0.2482\\
22.9861	-0.2469\\
23.0076	-0.2716\\
23.0287	-0.3077\\
23.0496	-0.3295\\
23.0806	-0.3419\\
23.1120	-0.3332\\
23.1443	-0.3212\\
23.1652	-0.3186\\
23.1966	-0.3090\\
23.2173	-0.2947\\
23.2483	-0.2779\\
23.2808	-0.2586\\
23.3015	-0.2316\\
23.3233	-0.1990\\
23.3341	-0.1838\\
23.3653	-0.1594\\
23.3861	-0.1452\\
23.4068	-0.1205\\
23.4172	-0.1073\\
23.4484	-0.0861\\
23.4695	-0.0731\\
23.4900	-0.0479\\
23.5007	-0.0349\\
23.5433	-0.0119\\
23.5746	-0.0046\\
23.5956	0.0044\\
23.6268	0.0200\\
23.6479	0.0343\\
23.6697	0.0459\\
23.7113	0.0490\\
23.7425	0.0387\\
23.7852	0.0132\\
23.8163	0.0061\\
23.8376	-0.0053\\
23.8479	-0.0155\\
23.8895	-0.0476\\
23.9227	-0.0475\\
23.9647	-0.0519\\
23.9967	-0.0571\\
24.0274	-0.0529\\
24.0591	-0.0754\\
24.0914	-0.0901\\
24.1124	-0.0885\\
24.1543	-0.0856\\
24.1852	-0.0714\\
24.2165	-0.0770\\
24.2375	-0.0857\\
24.2683	-0.0941\\
24.3102	-0.1000\\
24.3315	-0.0922\\
24.3534	-0.0866\\
24.3959	-0.0768\\
24.4164	-0.0654\\
24.4480	-0.0518\\
24.4791	-0.0558\\
24.5000	-0.0416\\
24.5209	-0.0255\\
24.5414	-0.0097\\
24.5830	0.0087\\
24.6040	0.0175\\
24.6258	0.0275\\
24.6573	0.0455\\
24.6896	0.0585\\
24.7101	0.0618\\
24.7520	0.0632\\
24.7832	0.0684\\
24.8147	0.0746\\
24.8467	0.0793\\
24.8683	0.0764\\
24.8791	0.0697\\
24.9104	0.0478\\
24.9311	0.0355\\
24.9519	0.0267\\
24.9730	0.0142\\
25.0042	0.0007\\
25.0260	0.0104\\
25.0572	0.0136\\
25.1006	0.0053\\
25.1221	0.0015\\
25.1436	-0.0076\\
25.1863	-0.0269\\
25.2184	-0.0386\\
25.2602	-0.0403\\
25.2915	-0.0340\\
25.3124	-0.0339\\
25.3542	-0.0287\\
25.3854	-0.0180\\
25.4061	-0.0150\\
25.4488	-0.0201\\
25.4801	-0.0167\\
25.5218	-0.0064\\
25.5543	-0.0062\\
25.5758	-0.0112\\
25.6175	-0.0131\\
25.6498	-0.0153\\
25.6915	-0.0139\\
25.7230	-0.0072\\
25.7434	-0.0001\\
25.7751	0.0161\\
25.7958	0.0265\\
25.8279	0.0377\\
25.8590	0.0515\\
25.8915	0.0566\\
25.9123	0.0471\\
25.9552	0.0346\\
25.9761	0.0468\\
26.0071	0.0597\\
26.0498	0.0591\\
26.0817	0.0573\\
26.1239	0.0500\\
26.1456	0.0462\\
26.1768	0.0365\\
26.2198	0.0229\\
26.2506	0.0271\\
26.2829	0.0166\\
26.3144	-0.0013\\
26.3458	-0.0050\\
26.3884	-0.0168\\
26.4101	-0.0219\\
26.4412	-0.0304\\
26.4830	-0.0296\\
26.5142	-0.0240\\
26.5570	-0.0261\\
26.5780	-0.0299\\
26.6091	-0.0369\\
26.6509	-0.0192\\
26.6716	-0.0324\\
26.6923	-0.0608\\
26.7251	-0.0388\\
26.7461	-0.0142\\
26.7672	-0.0263\\
26.7885	-0.0399\\
26.8311	-0.0526\\
26.8623	-0.0446\\
26.8935	-0.0419\\
26.9143	-0.0302\\
26.9465	-0.0122\\
26.9884	0.0009\\
27.0099	0.0147\\
27.0311	0.0366\\
27.0415	0.0431\\
27.0829	0.0568\\
27.1143	0.0507\\
27.1456	0.0554\\
27.1779	0.0715\\
27.1987	0.0644\\
27.2198	0.0515\\
27.2516	0.0506\\
27.2736	0.0718\\
27.2946	0.0811\\
27.3048	0.0879\\
27.3364	0.0623\\
27.3580	0.0627\\
27.3788	0.0826\\
27.4099	0.1035\\
27.4317	0.1095\\
27.4632	0.1115\\
27.5050	0.1059\\
27.5361	0.0893\\
27.5571	0.0756\\
27.5997	0.0831\\
27.6309	0.0943\\
27.6413	0.1003\\
27.6838	0.1124\\
27.7049	0.0971\\
27.7256	0.0983\\
27.7572	0.0831\\
27.7789	0.0827\\
27.8005	0.0940\\
27.8226	0.1162\\
27.8652	0.1227\\
27.8863	0.1013\\
27.9081	0.0858\\
27.9393	0.0697\\
27.9706	0.0741\\
27.9923	0.0639\\
28.0245	0.0495\\
28.0562	0.0466\\
28.0777	0.0285\\
28.1194	0.0102\\
28.1505	0.0078\\
28.1922	-0.0004\\
28.2235	-0.0071\\
28.2444	-0.0072\\
28.2871	0.0057\\
28.3183	-0.0000\\
28.3392	-0.0273\\
28.3698	-0.0413\\
28.3807	-0.0193\\
28.3911	0.0087\\
28.4118	0.0449\\
28.4228	0.0309\\
28.4548	-0.0420\\
28.4765	-0.0224\\
28.5081	-0.0132\\
28.5514	-0.0430\\
28.5619	-0.0376\\
28.5935	-0.0446\\
28.6245	-0.0618\\
28.6455	-0.0775\\
28.6661	-0.0518\\
28.6874	-0.0370\\
28.7078	-0.0549\\
28.7289	-0.0815\\
28.7595	-0.1033\\
28.7696	-0.1121\\
28.8111	-0.1327\\
28.8432	-0.1209\\
28.8541	-0.1306\\
28.8967	-0.1723\\
28.9077	-0.1566\\
28.9288	-0.1237\\
28.9612	-0.1071\\
28.9932	-0.1245\\
29.0255	-0.1178\\
29.0672	-0.1047\\
29.0880	-0.1237\\
29.1088	-0.1519\\
29.1195	-0.1615\\
29.1504	-0.1153\\
29.1716	-0.0839\\
29.1930	-0.0778\\
29.2039	-0.0849\\
29.2465	-0.1282\\
29.2675	-0.1125\\
29.2883	-0.1070\\
29.3196	-0.1147\\
29.3401	-0.0949\\
29.3716	-0.0884\\
29.4027	-0.1161\\
29.4341	-0.1278\\
29.4559	-0.1204\\
29.4976	-0.1125\\
29.5180	-0.1160\\
29.5391	-0.1252\\
29.5807	-0.1387\\
29.6119	-0.1356\\
29.6328	-0.1367\\
29.6755	-0.1312\\
29.7068	-0.1228\\
29.7173	-0.1166\\
29.7488	-0.0737\\
29.7707	-0.0415\\
29.8028	-0.0178\\
29.8455	-0.0238\\
29.8664	-0.0045\\
29.8882	0.0185\\
29.9316	0.0389\\
29.9640	0.0507\\
29.9963	0.0532\\
};
\addplot [color=mycolor2,solid,forget plot]
  table[row sep=crcr]{%
18.0497	0.1080\\
18.1097	0.0866\\
18.1497	0.2008\\
18.1597	0.1708\\
18.1697	0.1441\\
18.2297	0.0975\\
18.2897	0.1017\\
18.3497	0.0582\\
18.4097	0.0460\\
18.4397	0.0487\\
18.4697	0.0295\\
18.5297	0.0181\\
18.5897	0.0223\\
18.6297	0.0882\\
18.6497	0.0756\\
18.6797	0.0876\\
18.7097	0.0992\\
18.7397	0.0911\\
18.7497	0.0307\\
18.7697	-0.0455\\
18.7997	-0.0433\\
18.8297	-0.0518\\
18.8897	-0.0103\\
18.9197	-0.0137\\
18.9397	-0.0262\\
18.9797	-0.0160\\
19.0097	-0.0272\\
19.0397	-0.0820\\
19.0597	-0.1032\\
19.1097	-0.1273\\
19.1297	-0.1242\\
19.1897	-0.0843\\
19.2197	-0.0892\\
19.2497	-0.0754\\
19.2897	-0.0786\\
19.3097	-0.0904\\
19.3497	-0.1267\\
19.3697	-0.1595\\
19.3997	-0.1456\\
19.4297	-0.1578\\
19.4597	-0.1492\\
19.4897	-0.1251\\
19.5397	-0.1553\\
19.5497	-0.1371\\
19.5797	-0.1546\\
19.6097	-0.1417\\
19.6397	-0.1059\\
19.6697	-0.1123\\
19.6897	-0.0894\\
19.7197	-0.0748\\
19.7497	-0.0394\\
19.7797	-0.0429\\
19.7897	-0.0229\\
19.8197	0.0054\\
19.8497	-0.0289\\
19.8697	-0.0804\\
19.8797	-0.0564\\
19.8897	-0.0228\\
19.9097	0.0088\\
19.9297	-0.0092\\
19.9397	-0.0520\\
19.9497	-0.0280\\
19.9697	-0.0130\\
19.9897	0.0155\\
20.0197	-0.0056\\
20.0297	-0.0269\\
20.0597	-0.0014\\
20.0797	0.0303\\
20.0897	0.0512\\
20.1297	0.0598\\
20.1397	0.0148\\
20.1497	-0.0286\\
20.1797	0.0076\\
20.2097	0.0195\\
20.2297	-0.0381\\
20.2397	-0.0081\\
20.2497	-0.0361\\
20.2697	-0.0903\\
20.3097	-0.1530\\
20.3197	-0.1214\\
20.3497	-0.0858\\
20.3897	-0.0585\\
20.4097	-0.0066\\
20.4497	-0.0477\\
20.4797	-0.0173\\
20.4997	-0.0360\\
20.5097	-0.0549\\
20.5297	-0.1194\\
20.5597	-0.0890\\
20.5897	-0.1026\\
20.6097	-0.0850\\
20.6497	-0.0666\\
20.6697	-0.0322\\
20.6897	-0.0091\\
20.7197	0.0234\\
20.7497	-0.0044\\
20.7797	-0.0257\\
20.7897	-0.0469\\
20.8097	-0.0689\\
20.8397	-0.0552\\
20.8597	-0.0799\\
20.8997	-0.0885\\
20.9297	-0.0489\\
20.9697	-0.0497\\
20.9897	-0.0725\\
21.0297	-0.0483\\
21.0697	-0.0622\\
21.0997	-0.0874\\
21.1397	-0.0031\\
21.1497	0.0164\\
21.1697	0.0601\\
21.1997	0.0627\\
21.2297	0.0325\\
21.2897	0.0026\\
21.3497	0.0022\\
21.3797	-0.0052\\
21.4097	0.0099\\
21.4697	0.0169\\
21.4997	-0.0022\\
21.5097	0.0237\\
21.5297	0.0383\\
21.5697	0.0318\\
21.5897	0.0197\\
21.6297	0.0293\\
21.6497	-0.0009\\
21.6697	0.0139\\
21.7097	0.0064\\
21.7397	0.0461\\
21.7597	0.0213\\
21.7997	0.0316\\
21.8297	0.0542\\
21.8897	0.0584\\
21.9197	0.0922\\
21.9497	0.1193\\
21.9697	0.1751\\
21.9797	0.2022\\
21.9997	0.2436\\
22.0097	0.2230\\
22.0397	0.1903\\
22.0597	0.1650\\
22.0897	0.1423\\
22.1197	0.1772\\
22.1497	0.1955\\
22.1797	0.1625\\
22.2097	0.1779\\
22.2397	0.1748\\
22.2497	0.2158\\
22.2897	0.2405\\
22.3097	0.2562\\
22.3297	0.2919\\
22.3397	0.3169\\
22.3697	0.2858\\
22.3897	0.3053\\
22.3997	0.2802\\
22.4097	0.2528\\
22.4197	0.2157\\
22.4697	0.2025\\
22.4897	0.2200\\
22.5097	0.1957\\
22.5197	0.1629\\
22.5297	0.1327\\
22.5397	0.1089\\
22.5697	0.1562\\
22.5997	0.1091\\
22.6097	0.0908\\
22.6297	0.0507\\
22.6597	0.0055\\
22.6697	-0.0243\\
22.6997	-0.0305\\
22.7197	0.0268\\
22.7297	0.0071\\
22.7697	-0.0197\\
22.8097	-0.0243\\
22.8297	-0.0558\\
22.8397	-0.0741\\
22.8497	-0.1056\\
22.8797	-0.1170\\
22.8997	-0.0727\\
22.9097	-0.0485\\
22.9397	-0.0327\\
22.9597	-0.0582\\
22.9697	-0.0784\\
22.9997	-0.0682\\
23.0097	-0.1083\\
23.0197	-0.1361\\
23.0497	-0.1554\\
23.0797	-0.1663\\
23.0897	-0.1448\\
23.1097	-0.1590\\
23.1197	-0.1811\\
23.1797	-0.2278\\
23.2097	-0.2382\\
23.2397	-0.2835\\
23.2597	-0.2510\\
23.2697	-0.2788\\
23.2997	-0.2523\\
23.3097	-0.2334\\
23.3497	-0.2006\\
23.3597	-0.1596\\
23.3797	-0.1115\\
23.3897	-0.0923\\
23.4097	-0.0591\\
23.4497	-0.0337\\
23.4697	-0.0705\\
23.4797	-0.1108\\
23.5297	-0.1375\\
23.5397	-0.1123\\
23.5497	-0.0688\\
23.5597	-0.0455\\
23.5697	-0.0238\\
23.6297	0.0105\\
23.6597	-0.0307\\
23.6897	0.0145\\
23.7297	0.0396\\
23.7497	0.0565\\
23.7797	0.0399\\
23.8097	0.0114\\
23.8297	-0.0206\\
23.8397	-0.0458\\
23.8697	-0.0870\\
23.8897	-0.0420\\
23.9097	-0.0137\\
23.9297	-0.0368\\
23.9597	-0.0699\\
23.9897	-0.0697\\
24.0297	-0.0326\\
24.0697	-0.0648\\
24.0797	-0.1293\\
24.1097	-0.1825\\
24.1397	-0.1264\\
24.1597	-0.1135\\
24.1697	-0.0829\\
24.1897	-0.1129\\
24.2197	-0.1501\\
24.2297	-0.1421\\
24.2597	-0.1740\\
24.2797	-0.1555\\
24.2897	-0.1337\\
24.3397	-0.1551\\
24.3897	-0.1228\\
24.4097	-0.1423\\
24.4397	-0.1256\\
24.4597	-0.1036\\
24.4897	-0.0839\\
24.5197	-0.0496\\
24.5497	-0.0548\\
24.5897	-0.0709\\
24.6297	-0.0459\\
24.6497	-0.0589\\
24.6797	-0.0182\\
24.7097	0.0145\\
24.7497	0.0364\\
24.7697	0.0799\\
24.8097	0.0515\\
24.8297	0.0261\\
24.8797	0.0221\\
24.8897	0.0337\\
24.9497	0.0265\\
24.9797	0.0462\\
24.9897	0.0241\\
25.0097	-0.0004\\
25.0397	0.0166\\
25.0597	0.0037\\
25.0997	0.0320\\
25.1297	0.0135\\
25.1897	0.0081\\
25.2097	-0.0314\\
25.2397	-0.0565\\
25.2497	-0.0094\\
25.2797	0.0359\\
25.2997	-0.0010\\
25.3097	0.0231\\
25.3597	-0.0237\\
25.3997	0.0244\\
25.4097	0.0582\\
25.4297	0.0058\\
25.4497	-0.0101\\
25.4597	-0.0372\\
25.4797	-0.0788\\
25.5097	-0.1036\\
25.5297	-0.1160\\
25.5697	-0.1342\\
25.5797	-0.0763\\
25.5997	-0.0508\\
25.6097	-0.1058\\
25.6497	-0.1338\\
25.6597	-0.0931\\
25.6697	-0.0680\\
25.7097	-0.0368\\
25.7297	-0.0625\\
25.7597	-0.0278\\
25.7697	-0.0077\\
25.7797	0.0171\\
25.8197	-0.0072\\
25.8497	-0.0011\\
25.8797	0.0356\\
25.9097	0.0246\\
25.9497	0.0341\\
25.9697	0.0141\\
26.0197	0.0412\\
26.0297	0.0470\\
26.0597	0.0588\\
26.0797	0.0792\\
26.0897	0.0471\\
26.1497	0.0461\\
26.1997	0.0757\\
26.2397	0.0820\\
26.2697	0.0911\\
26.2997	0.0830\\
26.3197	0.0502\\
26.3297	0.0235\\
26.3897	0.0001\\
26.4097	0.0215\\
26.4297	-0.0113\\
26.4497	0.0372\\
26.4797	0.0484\\
26.5097	0.0749\\
26.5497	0.0710\\
26.5697	0.0560\\
26.6097	0.0005\\
26.6297	-0.0333\\
26.6897	-0.0289\\
26.7297	-0.0133\\
26.7497	-0.0349\\
26.8097	-0.0414\\
26.8297	-0.0412\\
26.8397	-0.0152\\
26.8497	0.0044\\
26.8697	0.0418\\
26.9497	0.0321\\
26.9797	0.0293\\
27.0397	0.0224\\
27.0497	0.0339\\
27.1097	0.0521\\
27.1497	0.1018\\
27.1697	0.0790\\
27.1997	0.1123\\
27.2297	0.0984\\
27.2797	0.0723\\
27.3197	0.0737\\
27.3497	0.0809\\
27.4097	0.1357\\
27.4397	0.1845\\
27.4597	0.1982\\
27.5097	0.2179\\
27.5197	0.1934\\
27.5597	0.2150\\
27.5797	0.2342\\
27.5897	0.2044\\
27.6197	0.1867\\
27.6497	0.2107\\
27.6897	0.1811\\
27.7097	0.1281\\
27.7397	0.0964\\
27.7597	0.1290\\
27.7997	0.0951\\
27.8097	0.0739\\
27.8297	0.0978\\
27.8697	0.1003\\
27.9097	0.1430\\
27.9497	0.1806\\
27.9697	0.1850\\
27.9897	0.1600\\
28.0097	0.1441\\
28.0397	0.1294\\
28.0697	0.1262\\
28.1197	0.1535\\
28.1297	0.1666\\
28.1897	0.1498\\
28.2197	0.1547\\
28.2297	0.1793\\
28.2397	0.2149\\
28.2497	0.2365\\
28.2697	0.2641\\
28.2897	0.2271\\
28.3097	0.2052\\
28.3297	0.1528\\
28.3597	0.1900\\
28.3697	0.1615\\
28.3897	0.1427\\
28.4197	0.0986\\
28.4497	0.0884\\
28.4797	0.0777\\
28.4897	0.0461\\
28.5197	0.0279\\
28.5397	0.0507\\
28.5497	0.0305\\
28.5697	0.0047\\
28.5897	-0.0145\\
28.5997	-0.0465\\
28.6497	-0.0819\\
28.6697	-0.0747\\
28.6897	-0.0551\\
28.7097	-0.0275\\
28.7297	-0.0444\\
28.7497	-0.0748\\
28.7597	-0.0984\\
28.7797	-0.0628\\
28.8097	-0.0727\\
28.8497	-0.0803\\
28.8897	-0.0756\\
28.9497	-0.0136\\
28.9597	0.0014\\
28.9697	-0.0276\\
28.9997	-0.0441\\
29.0297	-0.0815\\
29.0697	-0.0657\\
29.0897	-0.0534\\
29.1497	-0.0804\\
29.1897	-0.0732\\
29.2297	-0.1053\\
29.2497	-0.0930\\
29.2597	-0.1148\\
29.2697	-0.0842\\
29.3097	-0.0659\\
29.3297	-0.0460\\
29.3897	-0.0646\\
29.4097	-0.0696\\
29.4397	-0.0525\\
29.4897	-0.0441\\
29.4997	-0.0736\\
29.5097	-0.1125\\
29.5297	-0.0833\\
29.5697	-0.0649\\
29.5897	-0.0081\\
29.6097	0.0297\\
29.6197	-0.0082\\
29.6297	-0.0467\\
29.6497	-0.1410\\
29.6797	-0.1697\\
29.7397	-0.1381\\
29.7697	-0.1351\\
29.7897	-0.1507\\
29.8097	-0.1255\\
29.8297	-0.0923\\
29.8597	-0.0477\\
29.8997	-0.0411\\
29.9197	-0.0189\\
29.9997	0.0092\\
};
\end{axis}
\end{tikzpicture}%
	\caption{Height and visual observable measurements (blue) and estimates (red) during horizontal motion above checkerboard texture.}
	\label{fig:horizontal_motion}
\end{figure}

\begin{table}[!ht]
	\centering
	\caption{Mean and standard deviation of absolute errors for the estimates during horizontal motion, shown in \cref{fig:horizontal_motion}.}
	\begin{tabular}{l|cc}
\hline
~ & Mean abs. error [1/s] & Standard deviation [1/s] \\ \hline
$\vartheta_x$ & 0.09997 & 0.081676 \\
$\vartheta_y$ & 0.077126 & 0.066433 \\
$\vartheta_z$ & 0.051617 & 0.047676 \\ \hline

\end{tabular}

	\label{tab:horizontal_errors}
\end{table}


\subsubsection{Effect of Derotation}
In order to assess how well normal flow can be derotated with the current setup, measurements were conducted in which the DVS performed pure rotation along all axes. We compare body rate measurements, ground truth values for $\vartheta_x$ and $\vartheta_y$, and estimates with and without derotation. \cref{fig:derotation} shows these quantities obtained in three separate sequences, in which each body rate is varied independently. 

\begin{figure}[!ht]
	\centering
%		\tikzset{external/force remake=true}
  	\setlength{\fwidth}{0.4\linewidth}
%	\renewcommand{\ylabeldist}{0.02}
	% This file was created by matlab2tikz.
%
%The latest updates can be retrieved from
%  http://www.mathworks.com/matlabcentral/fileexchange/22022-matlab2tikz-matlab2tikz
%where you can also make suggestions and rate matlab2tikz.
%
\definecolor{mycolor1}{rgb}{0.00000,0.44700,0.74100}%
\definecolor{mycolor2}{rgb}{0.85000,0.32500,0.09800}%
\definecolor{mycolor3}{rgb}{0.46600,0.67400,0.18800}%
%
\begin{tikzpicture}

\begin{axis}[%
width=0.96\fwidth,
height=0.279\fwidth,
at={(0\fwidth,0.804\fwidth)},
scale only axis,
xmin=18.3278,
xmax=20.7550,
xlabel={$t$ [s]},
ymin=-2.5000,
ymax=3.4560,
ylabel={$-p$ [rad/s], $\vartheta_x$ [1/s]},
axis background/.style={fill=white},
legend columns=2,
legend style={at={(0.5,-0.55)},anchor=north},
title style={font=\labelsize},
xlabel style={font=\labelsize,at={(axis description cs:0.5,\xlabeldist)}},
ylabel style={font=\labelsize,at={(axis description cs:\ylabeldist,0.5)}},
legend style={font=\ticksize},
ticklabel style={font=\ticksize}
]
\addplot [color=black,dashed,forget plot]
  table[row sep=crcr]{%
18.3278	-0.0183\\
18.3380	-0.0377\\
18.3484	-0.0336\\
18.3591	-0.0307\\
18.3693	-0.0488\\
18.3798	-0.0698\\
18.3900	-0.0694\\
18.4009	-0.0555\\
18.4113	-0.0664\\
18.4223	-0.0634\\
18.4328	-0.0445\\
18.4437	-0.0331\\
18.4548	0.0062\\
18.4650	0.0172\\
18.4760	0.0197\\
18.4871	0.0296\\
18.4976	0.0568\\
18.5083	0.0432\\
18.5186	0.0513\\
18.5291	0.0709\\
18.5393	0.0734\\
18.5498	0.0716\\
18.5602	0.0901\\
18.5707	0.0587\\
18.5810	0.0623\\
18.5916	0.0339\\
18.6019	0.0604\\
18.6124	0.0112\\
18.6227	0.0742\\
18.6332	0.0554\\
18.6439	0.0341\\
18.6548	0.0357\\
18.6652	0.0553\\
18.6760	0.0690\\
18.6871	0.0578\\
18.6978	0.0856\\
18.7081	0.1156\\
18.7186	0.0919\\
18.7289	0.1515\\
18.7391	0.1114\\
18.7501	0.1205\\
18.7604	0.1428\\
18.7712	0.1505\\
18.7818	0.1568\\
18.7926	0.1274\\
18.8027	0.1750\\
18.8131	0.1624\\
18.8234	0.1546\\
18.8342	0.1243\\
18.8445	0.0942\\
18.8550	0.0762\\
18.8650	0.0351\\
18.8759	0.0100\\
18.8862	-0.0136\\
18.8965	-0.0762\\
18.9067	-0.0888\\
18.9177	-0.1119\\
18.9288	-0.1391\\
18.9392	-0.1306\\
18.9498	-0.1188\\
18.9605	-0.0904\\
18.9712	-0.0547\\
18.9819	-0.0201\\
18.9927	0.0085\\
19.0038	0.0104\\
19.0141	0.0856\\
19.0249	0.0979\\
19.0356	0.1156\\
19.0465	0.1739\\
19.0566	0.1892\\
19.0668	0.2328\\
19.0778	0.2293\\
19.0887	0.2397\\
19.0997	0.2554\\
19.1104	0.3003\\
19.1208	0.4059\\
19.1319	0.3849\\
19.1426	0.4497\\
19.1535	0.5089\\
19.1641	0.6133\\
19.1747	0.6690\\
19.1852	0.7188\\
19.1955	0.8054\\
19.2059	0.9711\\
19.2163	1.0440\\
19.2273	1.0728\\
19.2381	1.2433\\
19.2485	1.3950\\
19.2593	1.4290\\
19.2704	1.4371\\
19.2807	1.6638\\
19.2917	1.6455\\
19.3018	1.8654\\
19.3124	1.7814\\
19.3235	1.7865\\
19.3336	1.9452\\
19.3438	1.9386\\
19.3544	1.7148\\
19.3651	1.8032\\
19.3759	1.7414\\
19.3866	1.7345\\
19.3967	1.7788\\
19.4068	1.7478\\
19.4171	1.6323\\
19.4280	1.5610\\
19.4388	1.4852\\
19.4498	1.4356\\
19.4602	1.4985\\
19.4712	1.4291\\
19.4817	1.3481\\
19.4927	1.2294\\
19.5039	1.1662\\
19.5141	1.1447\\
19.5250	1.0192\\
19.5353	1.0085\\
19.5457	0.9615\\
19.5568	0.7624\\
19.5672	0.7538\\
19.5778	0.7008\\
19.5888	0.6135\\
19.5997	0.5990\\
19.6101	0.5954\\
19.6203	0.5914\\
19.6312	0.5137\\
19.6416	0.5094\\
19.6519	0.4883\\
19.6630	0.3950\\
19.6734	0.4117\\
19.6841	0.3741\\
19.6945	0.3543\\
19.7047	0.2798\\
19.7149	0.2681\\
19.7257	0.2252\\
19.7361	0.2051\\
19.7464	0.1608\\
19.7569	0.1338\\
19.7680	0.1078\\
19.7787	0.0928\\
19.7890	0.0577\\
19.7997	0.0327\\
19.8100	0.0073\\
19.8203	-0.0127\\
19.8308	-0.0723\\
19.8414	-0.0946\\
19.8520	-0.1197\\
19.8627	-0.1662\\
19.8731	-0.1872\\
19.8840	-0.1896\\
19.8944	-0.2116\\
19.9054	-0.2138\\
19.9164	-0.2207\\
19.9267	-0.2325\\
19.9376	-0.2176\\
19.9483	-0.2086\\
19.9589	-0.2023\\
19.9693	-0.1919\\
19.9796	-0.1602\\
19.9901	-0.1629\\
20.0009	-0.1530\\
20.0110	-0.1543\\
20.0215	-0.1313\\
20.0319	-0.1516\\
20.0423	-0.1586\\
20.0529	-0.1628\\
20.0632	-0.1924\\
20.0739	-0.1950\\
20.0846	-0.2207\\
20.0955	-0.2718\\
20.1066	-0.2799\\
20.1173	-0.3174\\
20.1279	-0.3473\\
20.1386	-0.3989\\
20.1495	-0.4030\\
20.1600	-0.4369\\
20.1710	-0.5096\\
20.1817	-0.5252\\
20.1927	-0.5480\\
20.2034	-0.5876\\
20.2141	-0.6348\\
20.2241	-0.7170\\
20.2350	-0.6875\\
20.2456	-0.8295\\
20.2566	-0.8175\\
20.2671	-0.8890\\
20.2780	-0.9014\\
20.2890	-0.9623\\
20.2993	-1.0996\\
20.3103	-1.0705\\
20.3214	-1.1182\\
20.3315	-1.3283\\
20.3419	-1.3206\\
20.3527	-1.3041\\
20.3633	-1.4347\\
20.3738	-1.4553\\
20.3848	-1.4216\\
20.3955	-1.4692\\
20.4061	-1.5312\\
20.4171	-1.4860\\
20.4280	-1.5009\\
20.4382	-1.7418\\
20.4483	-1.6753\\
20.4593	-1.5282\\
20.4703	-1.5857\\
20.4808	-1.6174\\
20.4913	-1.6438\\
20.5021	-1.6161\\
20.5123	-1.8809\\
20.5233	-1.6001\\
20.5340	-1.6079\\
20.5446	-1.6181\\
20.5551	-1.5471\\
20.5661	-1.4107\\
20.5766	-1.4245\\
20.5869	-1.3471\\
20.5974	-1.1932\\
20.6080	-1.1030\\
20.6182	-1.0946\\
20.6292	-0.8354\\
20.6398	-0.7955\\
20.6505	-0.7447\\
20.6609	-0.6664\\
20.6712	-0.6600\\
20.6813	-0.6195\\
20.6924	-0.5203\\
20.7033	-0.5100\\
20.7142	-0.4528\\
20.7247	-0.4055\\
20.7357	-0.3345\\
20.7464	-0.3067\\
};
\addplot [color=mycolor1,solid,forget plot]
  table[row sep=crcr]{%
18.3278	0.0009\\
18.3380	-0.0003\\
18.3484	0.0030\\
18.3591	0.0054\\
18.3693	0.0029\\
18.3798	-0.0008\\
18.3900	0.0002\\
18.4009	0.0047\\
18.4113	0.0048\\
18.4223	0.0073\\
18.4328	0.0126\\
18.4437	0.0150\\
18.4548	0.0218\\
18.4650	0.0218\\
18.4760	0.0204\\
18.4871	0.0209\\
18.4976	0.0249\\
18.5083	0.0208\\
18.5186	0.0210\\
18.5291	0.0232\\
18.5393	0.0224\\
18.5498	0.0212\\
18.5602	0.0246\\
18.5707	0.0184\\
18.5810	0.0203\\
18.5916	0.0159\\
18.6019	0.0231\\
18.6124	0.0152\\
18.6227	0.0294\\
18.6332	0.0259\\
18.6439	0.0207\\
18.6548	0.0193\\
18.6652	0.0203\\
18.6760	0.0207\\
18.6871	0.0159\\
18.6978	0.0192\\
18.7081	0.0231\\
18.7186	0.0160\\
18.7289	0.0259\\
18.7391	0.0158\\
18.7501	0.0161\\
18.7604	0.0192\\
18.7712	0.0198\\
18.7818	0.0208\\
18.7926	0.0150\\
18.8027	0.0252\\
18.8131	0.0236\\
18.8234	0.0239\\
18.8342	0.0202\\
18.8445	0.0170\\
18.8550	0.0167\\
18.8650	0.0124\\
18.8759	0.0113\\
18.8862	0.0107\\
18.8965	0.0021\\
18.9067	0.0036\\
18.9177	0.0029\\
18.9288	0.0011\\
18.9392	0.0052\\
18.9498	0.0084\\
18.9605	0.0138\\
18.9712	0.0193\\
18.9819	0.0240\\
18.9927	0.0280\\
19.0038	0.0273\\
19.0141	0.0412\\
19.0249	0.0422\\
19.0356	0.0435\\
19.0465	0.0523\\
19.0566	0.0523\\
19.0668	0.0586\\
19.0778	0.0561\\
19.0887	0.0567\\
19.0997	0.0585\\
19.1104	0.0658\\
19.1208	0.0848\\
19.1319	0.0776\\
19.1426	0.0876\\
19.1535	0.0963\\
19.1641	0.1135\\
19.1747	0.1200\\
19.1852	0.1249\\
19.1955	0.1365\\
19.2059	0.1635\\
19.2163	0.1719\\
19.2273	0.1712\\
19.2381	0.1994\\
19.2485	0.2215\\
19.2593	0.2177\\
19.2704	0.2060\\
19.2807	0.2388\\
19.2917	0.2227\\
19.3018	0.2570\\
19.3124	0.2342\\
19.3235	0.2330\\
19.3336	0.2596\\
19.3438	0.2545\\
19.3544	0.2070\\
19.3651	0.2190\\
19.3759	0.1995\\
19.3866	0.1922\\
19.3967	0.1934\\
19.4068	0.1822\\
19.4171	0.1562\\
19.4280	0.1429\\
19.4388	0.1309\\
19.4498	0.1248\\
19.4602	0.1400\\
19.4712	0.1317\\
19.4817	0.1208\\
19.4927	0.1017\\
19.5039	0.0924\\
19.5141	0.0910\\
19.5250	0.0727\\
19.5353	0.0759\\
19.5457	0.0749\\
19.5568	0.0495\\
19.5672	0.0538\\
19.5778	0.0499\\
19.5888	0.0385\\
19.5997	0.0367\\
19.6101	0.0360\\
19.6203	0.0345\\
19.6312	0.0216\\
19.6416	0.0215\\
19.6519	0.0199\\
19.6630	0.0082\\
19.6734	0.0152\\
19.6841	0.0139\\
19.6945	0.0158\\
19.7047	0.0094\\
19.7149	0.0126\\
19.7257	0.0106\\
19.7361	0.0117\\
19.7464	0.0085\\
19.7569	0.0077\\
19.7680	0.0071\\
19.7787	0.0083\\
19.7890	0.0062\\
19.7997	0.0056\\
19.8100	0.0048\\
19.8203	0.0046\\
19.8308	-0.0019\\
19.8414	-0.0026\\
19.8520	-0.0039\\
19.8627	-0.0084\\
19.8731	-0.0094\\
19.8840	-0.0077\\
19.8944	-0.0094\\
19.9054	-0.0084\\
19.9164	-0.0084\\
19.9267	-0.0091\\
19.9376	-0.0057\\
19.9483	-0.0033\\
19.9589	-0.0014\\
19.9693	0.0011\\
19.9796	0.0069\\
19.9901	0.0071\\
20.0009	0.0091\\
20.0110	0.0094\\
20.0215	0.0138\\
20.0319	0.0107\\
20.0423	0.0094\\
20.0529	0.0083\\
20.0632	0.0025\\
20.0739	0.0003\\
20.0846	-0.0056\\
20.0955	-0.0156\\
20.1066	-0.0166\\
20.1173	-0.0203\\
20.1279	-0.0206\\
20.1386	-0.0233\\
20.1495	-0.0174\\
20.1600	-0.0172\\
20.1710	-0.0245\\
20.1817	-0.0241\\
20.1927	-0.0254\\
20.2034	-0.0300\\
20.2141	-0.0365\\
20.2241	-0.0496\\
20.2350	-0.0439\\
20.2456	-0.0674\\
20.2566	-0.0635\\
20.2671	-0.0730\\
20.2780	-0.0709\\
20.2890	-0.0770\\
20.2993	-0.0969\\
20.3103	-0.0885\\
20.3214	-0.0945\\
20.3315	-0.1312\\
20.3419	-0.1303\\
20.3527	-0.1273\\
20.3633	-0.1509\\
20.3738	-0.1532\\
20.3848	-0.1452\\
20.3955	-0.1500\\
20.4061	-0.1580\\
20.4171	-0.1488\\
20.4280	-0.1514\\
20.4382	-0.1990\\
20.4483	-0.1886\\
20.4593	-0.1637\\
20.4703	-0.1761\\
20.4808	-0.1832\\
20.4913	-0.1902\\
20.5021	-0.1861\\
20.5123	-0.2400\\
20.5233	-0.1848\\
20.5340	-0.1854\\
20.5446	-0.1841\\
20.5551	-0.1678\\
20.5661	-0.1403\\
20.5766	-0.1454\\
20.5869	-0.1351\\
20.5974	-0.1119\\
20.6080	-0.1010\\
20.6182	-0.1058\\
20.6292	-0.0593\\
20.6398	-0.0562\\
20.6505	-0.0505\\
20.6609	-0.0387\\
20.6712	-0.0406\\
20.6813	-0.0347\\
20.6924	-0.0165\\
20.7033	-0.0159\\
20.7142	-0.0059\\
20.7247	0.0017\\
20.7357	0.0139\\
20.7464	0.0172\\
};
\addplot [color=mycolor2,solid,forget plot]
  table[row sep=crcr]{%
18.3650	-0.0000\\
18.3650	-0.0000\\
18.3650	-0.0000\\
18.3650	-0.0000\\
18.3750	-0.0000\\
18.3850	-0.0000\\
18.3950	-0.0000\\
18.4050	-0.0000\\
18.4150	-0.0000\\
18.4250	-0.0222\\
18.4350	-0.0261\\
18.4450	-0.0329\\
18.4550	-0.0329\\
18.4650	-0.0340\\
18.4850	-0.0340\\
18.4950	-0.0340\\
18.5050	-0.0340\\
18.5150	-0.0340\\
18.5250	-0.0340\\
18.5350	-0.0340\\
18.5450	0.0432\\
18.5550	0.1029\\
18.5650	0.1451\\
18.5750	0.1604\\
18.5850	0.1724\\
18.5950	0.1824\\
18.6050	0.1826\\
18.6150	0.1772\\
18.6250	0.1707\\
18.6350	0.1669\\
18.6450	0.1669\\
18.6550	0.1669\\
18.6750	0.1592\\
18.6850	0.1374\\
18.6950	0.1183\\
18.7050	0.0970\\
18.7150	0.0965\\
18.7250	0.1006\\
18.7350	0.1282\\
18.7450	0.1713\\
18.7550	0.1961\\
18.7650	0.2259\\
18.7750	0.2275\\
18.7850	0.2343\\
18.7950	0.2361\\
18.8050	0.2370\\
18.8150	0.2372\\
18.8250	0.2376\\
18.8350	0.2390\\
18.8450	0.2426\\
18.8550	0.2425\\
18.8750	0.2384\\
18.8850	0.2359\\
18.8950	0.2332\\
18.9050	0.2318\\
18.9150	0.2318\\
18.9250	0.2318\\
18.9350	0.2287\\
18.9450	0.2227\\
18.9550	0.2204\\
18.9650	0.2046\\
18.9750	0.1049\\
18.9850	0.0795\\
18.9950	0.0464\\
19.0050	0.0398\\
19.0150	0.0337\\
19.0250	0.0337\\
19.0450	0.0012\\
19.0550	0.0008\\
19.0650	0.0232\\
19.0750	0.0864\\
19.0850	0.1617\\
19.0950	0.2401\\
19.1050	0.2868\\
19.1150	0.3249\\
19.1250	0.3319\\
19.1350	0.3498\\
19.1450	0.3692\\
19.1550	0.3990\\
19.1650	0.3990\\
19.1750	0.4607\\
19.1950	0.6217\\
19.2050	0.7274\\
19.2150	0.8804\\
19.2250	1.0256\\
19.2350	1.1788\\
19.2450	1.3078\\
19.2550	1.4259\\
19.2650	1.5391\\
19.2750	1.6367\\
19.2850	1.7415\\
19.2950	1.8361\\
19.3050	1.9709\\
19.3150	2.0692\\
19.3250	2.1829\\
19.3350	2.2407\\
19.3450	2.3022\\
19.3550	2.3441\\
19.3750	2.3266\\
19.3850	2.2820\\
19.3950	2.2411\\
19.4050	2.2330\\
19.4150	2.1746\\
19.4250	2.1286\\
19.4350	2.1091\\
19.4450	2.0655\\
19.4550	2.0217\\
19.4650	1.9891\\
19.4750	1.9615\\
19.4850	1.8926\\
19.4950	1.7992\\
19.5050	1.7205\\
19.5150	1.6522\\
19.5250	1.5795\\
19.5450	1.4613\\
19.5550	1.4203\\
19.5650	1.3691\\
19.5750	1.3034\\
19.5850	1.2635\\
19.5950	1.2045\\
19.6050	1.0758\\
19.6150	0.9753\\
19.6250	0.8659\\
19.6350	0.8543\\
19.6450	0.8295\\
19.6550	0.7954\\
19.6650	0.7224\\
19.6750	0.6682\\
19.6850	0.6332\\
19.6950	0.5956\\
19.7050	0.5551\\
19.7150	0.5156\\
19.7350	0.4537\\
19.7450	0.4332\\
19.7550	0.4076\\
19.7650	0.3901\\
19.7750	0.3604\\
19.7850	0.3368\\
19.7950	0.3133\\
19.8050	0.3001\\
19.8150	0.2949\\
19.8250	0.2915\\
19.8350	0.2915\\
19.8450	0.2915\\
19.8550	0.2915\\
19.8650	0.2891\\
19.8750	0.2891\\
19.8850	0.2891\\
19.8950	0.1986\\
19.9150	0.0489\\
19.9250	-0.1193\\
19.9350	-0.2018\\
19.9450	-0.2082\\
19.9550	-0.2344\\
19.9650	-0.2479\\
19.9750	-0.2526\\
19.9850	-0.2488\\
19.9950	-0.2425\\
20.0050	-0.2313\\
20.0150	-0.2284\\
20.0250	-0.2245\\
20.0350	-0.2186\\
20.0450	-0.2044\\
20.0550	-0.1984\\
20.0650	-0.1857\\
20.0750	-0.1845\\
20.0850	-0.1845\\
20.1050	-0.1823\\
20.1150	-0.1831\\
20.1250	-0.1929\\
20.1350	-0.2045\\
20.1450	-0.2323\\
20.1550	-0.2693\\
20.1650	-0.3082\\
20.1750	-0.3564\\
20.1850	-0.4153\\
20.1950	-0.4639\\
20.2050	-0.4924\\
20.2150	-0.5178\\
20.2250	-0.5243\\
20.2350	-0.5891\\
20.2550	-0.7428\\
20.2650	-0.8183\\
20.2750	-0.8839\\
20.2850	-0.9682\\
20.2950	-1.0054\\
20.3050	-1.0409\\
20.3150	-1.1418\\
20.3250	-1.2156\\
20.3350	-1.2670\\
20.3450	-1.3653\\
20.3550	-1.4488\\
20.3650	-1.5207\\
20.3750	-1.5960\\
20.3850	-1.6744\\
20.4050	-1.8042\\
20.4150	-1.8362\\
20.4250	-1.8842\\
20.4350	-1.8926\\
20.4450	-1.9213\\
20.4550	-1.9305\\
20.4650	-1.9078\\
20.4750	-1.8928\\
20.4850	-1.9267\\
20.4950	-1.9692\\
20.5050	-1.9887\\
20.5150	-2.0025\\
20.5250	-2.0036\\
20.5350	-2.0361\\
20.5450	-2.0509\\
20.5650	-2.0420\\
20.5750	-2.0249\\
20.5850	-1.9678\\
20.5950	-1.8883\\
20.6050	-1.7796\\
20.6150	-1.6588\\
20.6250	-1.5275\\
20.6350	-1.4187\\
20.6450	-1.3271\\
20.6550	-1.2328\\
20.6650	-1.1651\\
20.6750	-1.0808\\
20.6850	-0.9996\\
20.6950	-0.9239\\
20.7050	-0.8567\\
20.7150	-0.8001\\
20.7250	-0.7442\\
20.7450	-0.6720\\
20.7550	-0.6543\\
};
\addplot [color=mycolor3,solid,forget plot]
  table[row sep=crcr]{%
18.3650	-0.0000\\
18.3650	-0.0000\\
18.3650	-0.0000\\
18.3650	-0.0000\\
18.3750	-0.0000\\
18.3850	-0.0000\\
18.3950	-0.0000\\
18.4050	-0.0000\\
18.4150	-0.0000\\
18.4250	-0.0105\\
18.4350	-0.0125\\
18.4450	-0.0172\\
18.4550	-0.0172\\
18.4650	-0.0172\\
18.4850	-0.0172\\
18.4950	-0.0172\\
18.5050	-0.0172\\
18.5150	-0.0172\\
18.5250	-0.0172\\
18.5350	-0.0172\\
18.5450	0.0244\\
18.5550	0.0561\\
18.5650	0.0805\\
18.5750	0.0905\\
18.5850	0.0985\\
18.5950	0.1072\\
18.6050	0.1107\\
18.6150	0.1110\\
18.6250	0.1085\\
18.6350	0.1082\\
18.6450	0.1082\\
18.6550	0.1082\\
18.6750	0.1042\\
18.6850	0.0892\\
18.6950	0.0750\\
18.7050	0.0639\\
18.7150	0.0614\\
18.7250	0.0614\\
18.7350	0.0692\\
18.7450	0.0938\\
18.7550	0.1050\\
18.7650	0.1157\\
18.7750	0.1160\\
18.7850	0.1159\\
18.7950	0.1150\\
18.8050	0.1134\\
18.8150	0.1114\\
18.8250	0.1105\\
18.8350	0.1102\\
18.8450	0.1125\\
18.8550	0.1127\\
18.8750	0.1171\\
18.8850	0.1209\\
18.8950	0.1257\\
18.9050	0.1274\\
18.9150	0.1274\\
18.9250	0.1274\\
18.9350	0.1306\\
18.9450	0.1429\\
18.9550	0.1453\\
18.9650	0.1410\\
18.9750	0.0721\\
18.9850	0.0512\\
18.9950	0.0212\\
19.0050	0.0151\\
19.0150	0.0100\\
19.0250	0.0100\\
19.0450	-0.0395\\
19.0550	-0.0813\\
19.0650	-0.0805\\
19.0750	-0.0728\\
19.0850	-0.0648\\
19.0950	-0.0578\\
19.1050	-0.0541\\
19.1150	-0.0515\\
19.1250	-0.0514\\
19.1350	-0.0513\\
19.1450	-0.0515\\
19.1550	-0.0577\\
19.1650	-0.0577\\
19.1750	-0.0615\\
19.1950	-0.0606\\
19.2050	-0.0594\\
19.2150	-0.0569\\
19.2250	-0.0485\\
19.2350	-0.0289\\
19.2450	-0.0209\\
19.2550	-0.0088\\
19.2650	0.0012\\
19.2750	0.0067\\
19.2850	0.0147\\
19.2950	0.0198\\
19.3050	0.0277\\
19.3150	0.0466\\
19.3250	0.0688\\
19.3350	0.0763\\
19.3450	0.0799\\
19.3550	0.0858\\
19.3750	0.1012\\
19.3850	0.1035\\
19.3950	0.1051\\
19.4050	0.1090\\
19.4150	0.1091\\
19.4250	0.1101\\
19.4350	0.1127\\
19.4450	0.1182\\
19.4550	0.1199\\
19.4650	0.1230\\
19.4750	0.1230\\
19.4850	0.1230\\
19.4950	0.1235\\
19.5050	0.1236\\
19.5150	0.1245\\
19.5250	0.1254\\
19.5450	0.1253\\
19.5550	0.1310\\
19.5650	0.1382\\
19.5750	0.1460\\
19.5850	0.1466\\
19.5950	0.1463\\
19.6050	0.1444\\
19.6150	0.1407\\
19.6250	0.1368\\
19.6350	0.1360\\
19.6450	0.1321\\
19.6550	0.1309\\
19.6650	0.1301\\
19.6750	0.1297\\
19.6850	0.1292\\
19.6950	0.1279\\
19.7050	0.1278\\
19.7150	0.1274\\
19.7350	0.1247\\
19.7450	0.1207\\
19.7550	0.1187\\
19.7650	0.1174\\
19.7750	0.1169\\
19.7850	0.1163\\
19.7950	0.1163\\
19.8050	0.1169\\
19.8150	0.1197\\
19.8250	0.1220\\
19.8350	0.1220\\
19.8450	0.1220\\
19.8550	0.1220\\
19.8650	0.1228\\
19.8750	0.1228\\
19.8850	0.1228\\
19.8950	0.0806\\
19.9150	0.0538\\
19.9250	0.0294\\
19.9350	0.0120\\
19.9450	0.0092\\
19.9550	0.0015\\
19.9650	-0.0035\\
19.9750	-0.0147\\
19.9850	-0.0226\\
19.9950	-0.0333\\
20.0050	-0.0378\\
20.0150	-0.0379\\
20.0250	-0.0379\\
20.0350	-0.0378\\
20.0450	-0.0374\\
20.0550	-0.0373\\
20.0650	-0.0361\\
20.0750	-0.0361\\
20.0850	-0.0361\\
20.1050	-0.0219\\
20.1150	0.0176\\
20.1250	0.0473\\
20.1350	0.0695\\
20.1450	0.0777\\
20.1550	0.0777\\
20.1650	0.0778\\
20.1750	0.0778\\
20.1850	0.0770\\
20.1950	0.0766\\
20.2050	0.0762\\
20.2150	0.0763\\
20.2250	0.0783\\
20.2350	0.0781\\
20.2550	0.0763\\
20.2650	0.0697\\
20.2750	0.0589\\
20.2850	0.0429\\
20.2950	0.0364\\
20.3050	0.0357\\
20.3150	0.0272\\
20.3250	0.0172\\
20.3350	0.0153\\
20.3450	0.0013\\
20.3550	-0.0256\\
20.3650	-0.0382\\
20.3750	-0.0514\\
20.3850	-0.0853\\
20.4050	-0.1168\\
20.4150	-0.1265\\
20.4250	-0.1464\\
20.4350	-0.1455\\
20.4450	-0.1423\\
20.4550	-0.1417\\
20.4650	-0.1385\\
20.4750	-0.1371\\
20.4850	-0.1388\\
20.4950	-0.1421\\
20.5050	-0.1417\\
20.5150	-0.1409\\
20.5250	-0.1404\\
20.5350	-0.1425\\
20.5450	-0.1446\\
20.5650	-0.1609\\
20.5750	-0.1656\\
20.5850	-0.1733\\
20.5950	-0.1844\\
20.6050	-0.1874\\
20.6150	-0.1854\\
20.6250	-0.1853\\
20.6350	-0.1851\\
20.6450	-0.1834\\
20.6550	-0.1796\\
20.6650	-0.1803\\
20.6750	-0.1772\\
20.6850	-0.1680\\
20.6950	-0.1616\\
20.7050	-0.1572\\
20.7150	-0.1549\\
20.7250	-0.1536\\
20.7450	-0.1522\\
20.7550	-0.1522\\
};
\end{axis}

\begin{axis}[%
width=0.96\fwidth,
height=0.279\fwidth,
at={(0\fwidth,0.402\fwidth)},
scale only axis,
xmin=24.2442,
xmax=26.5295,
xlabel={$t$ [s]},
ymin=-2.5000,
ymax=3.4560,
ylabel={$q$ [rad/s], $,\vartheta_y$ [1/s]},
axis background/.style={fill=white},
legend columns=2,
legend style={at={(0.5,-0.55)},anchor=north},
title style={font=\labelsize},
xlabel style={font=\labelsize,at={(axis description cs:0.5,\xlabeldist)}},
ylabel style={font=\labelsize,at={(axis description cs:\ylabeldist,0.5)}},
legend style={font=\ticksize},
ticklabel style={font=\ticksize}
]
\addplot [color=black,dashed,forget plot]
  table[row sep=crcr]{%
24.2442	-0.0374\\
24.2549	0.0305\\
24.2659	0.0297\\
24.2764	0.0014\\
24.2872	0.0230\\
24.2981	0.0207\\
24.3089	0.0207\\
24.3197	0.0420\\
24.3305	0.0639\\
24.3410	0.0597\\
24.3516	0.0765\\
24.3620	0.1011\\
24.3727	0.0489\\
24.3832	0.0751\\
24.3943	0.1056\\
24.4049	0.0892\\
24.4160	0.0500\\
24.4266	0.0357\\
24.4370	0.0169\\
24.4474	-0.0168\\
24.4579	-0.0328\\
24.4682	-0.0955\\
24.4789	-0.0872\\
24.4895	-0.1070\\
24.4999	-0.1346\\
24.5101	-0.1301\\
24.5209	-0.0636\\
24.5310	-0.0254\\
24.5420	0.0013\\
24.5529	0.0274\\
24.5637	0.0712\\
24.5747	0.0899\\
24.5854	0.1587\\
24.5964	0.1589\\
24.6069	0.1853\\
24.6172	0.1866\\
24.6275	0.1777\\
24.6385	0.1020\\
24.6489	0.0937\\
24.6597	0.0748\\
24.6706	0.0331\\
24.6811	-0.0242\\
24.6922	-0.0259\\
24.7029	-0.0935\\
24.7131	-0.1172\\
24.7233	-0.1248\\
24.7340	-0.1359\\
24.7445	-0.1484\\
24.7554	-0.1269\\
24.7656	-0.1497\\
24.7763	-0.0906\\
24.7866	-0.0963\\
24.7971	-0.0196\\
24.8081	-0.0332\\
24.8184	0.0017\\
24.8286	0.0154\\
24.8397	0.0238\\
24.8504	0.0451\\
24.8607	0.0627\\
24.8713	0.0782\\
24.8822	0.0944\\
24.8931	0.0547\\
24.9038	0.0823\\
24.9140	0.0205\\
24.9246	0.0142\\
24.9352	0.0305\\
24.9461	0.0239\\
24.9564	0.0034\\
24.9669	-0.0273\\
24.9773	-0.0957\\
24.9882	-0.1412\\
24.9988	-0.1786\\
25.0096	-0.2246\\
25.0203	-0.2718\\
25.0314	-0.2534\\
25.0421	-0.3134\\
25.0531	-0.3510\\
25.0637	-0.4916\\
25.0746	-0.5141\\
25.0856	-0.5301\\
25.0963	-0.6004\\
25.1066	-0.6907\\
25.1178	-0.6667\\
25.1288	-0.8141\\
25.1393	-0.8671\\
25.1501	-0.8622\\
25.1612	-0.8896\\
25.1717	-1.1354\\
25.1822	-1.1392\\
25.1932	-1.1303\\
25.2042	-1.2023\\
25.2147	-1.3879\\
25.2257	-1.3720\\
25.2362	-1.4798\\
25.2463	-1.4777\\
25.2564	-1.6064\\
25.2666	-1.6637\\
25.2773	-1.5293\\
25.2881	-1.3819\\
25.2982	-1.4690\\
25.3087	-1.3689\\
25.3193	-1.3320\\
25.3295	-1.1950\\
25.3398	-1.0726\\
25.3501	-1.0125\\
25.3602	-0.9226\\
25.3704	-0.8350\\
25.3813	-0.7291\\
25.3919	-0.7700\\
25.4029	-0.7020\\
25.4139	-0.6433\\
25.4247	-0.5719\\
25.4357	-0.5485\\
25.4463	-0.6007\\
25.4567	-0.6161\\
25.4670	-0.6368\\
25.4776	-0.6173\\
25.4882	-0.5803\\
25.4990	-0.5746\\
25.5098	-0.5143\\
25.5205	-0.4737\\
25.5311	-0.5182\\
25.5421	-0.5363\\
25.5530	-0.4878\\
25.5632	-0.4779\\
25.5733	-0.4652\\
25.5843	-0.3730\\
25.5952	-0.3218\\
25.6064	-0.2477\\
25.6173	-0.2059\\
25.6276	-0.1794\\
25.6385	-0.1185\\
25.6493	-0.0194\\
25.6600	0.0428\\
25.6702	0.1003\\
25.6806	0.2033\\
25.6911	0.2286\\
25.7013	0.2717\\
25.7122	0.2812\\
25.7229	0.2702\\
25.7337	0.2600\\
25.7440	0.2461\\
25.7549	0.1976\\
25.7655	0.1890\\
25.7764	0.1255\\
25.7876	0.0580\\
25.7986	0.0220\\
25.8090	-0.0048\\
25.8200	-0.0434\\
25.8302	-0.1285\\
25.8413	-0.1126\\
25.8518	-0.1167\\
25.8618	-0.0887\\
25.8729	-0.0565\\
25.8837	-0.0383\\
25.8942	0.0216\\
25.9049	0.0990\\
25.9153	0.1490\\
25.9264	0.1773\\
25.9375	0.2157\\
25.9476	0.3373\\
25.9587	0.3326\\
25.9691	0.3947\\
25.9794	0.4387\\
25.9899	0.4420\\
26.0006	0.4372\\
26.0111	0.4767\\
26.0221	0.4667\\
26.0326	0.4811\\
26.0429	0.5169\\
26.0537	0.4943\\
26.0642	0.5701\\
26.0746	0.6168\\
26.0856	0.6350\\
26.0959	0.8323\\
26.1064	0.8814\\
26.1174	0.9080\\
26.1276	0.9848\\
26.1383	1.1266\\
26.1491	1.1398\\
26.1598	1.1695\\
26.1705	1.2349\\
26.1808	1.3371\\
26.1912	1.3270\\
26.2015	1.3007\\
26.2120	1.1623\\
26.2223	1.1848\\
26.2334	1.0809\\
26.2445	1.0886\\
26.2553	0.9739\\
26.2660	0.9311\\
26.2764	1.0165\\
26.2866	1.0280\\
26.2973	0.9354\\
26.3077	0.9270\\
26.3184	0.9297\\
26.3286	1.0149\\
26.3398	0.9530\\
26.3502	0.9347\\
26.3606	0.9517\\
26.3712	0.9479\\
26.3823	0.9279\\
26.3929	0.8966\\
26.4037	0.7843\\
26.4142	0.7126\\
26.4244	0.6902\\
26.4349	0.5730\\
26.4452	0.4446\\
26.4553	0.3411\\
26.4661	0.2977\\
26.4764	0.2289\\
26.4872	0.0850\\
26.4980	0.0336\\
26.5087	-0.0344\\
26.5189	-0.0304\\
26.5295	-0.0391\\
};
\addplot [color=mycolor1,solid,forget plot]
  table[row sep=crcr]{%
24.2442	-0.0117\\
24.2549	0.0019\\
24.2659	0.0015\\
24.2764	-0.0052\\
24.2872	-0.0026\\
24.2981	-0.0058\\
24.3089	-0.0082\\
24.3197	-0.0068\\
24.3305	-0.0044\\
24.3410	-0.0056\\
24.3516	-0.0015\\
24.3620	0.0052\\
24.3727	-0.0018\\
24.3832	0.0070\\
24.3943	0.0168\\
24.4049	0.0166\\
24.4160	0.0112\\
24.4266	0.0099\\
24.4370	0.0083\\
24.4474	0.0035\\
24.4579	0.0030\\
24.4682	-0.0059\\
24.4789	0.0002\\
24.4895	-0.0012\\
24.4999	-0.0053\\
24.5101	-0.0042\\
24.5209	0.0081\\
24.5310	0.0124\\
24.5420	0.0130\\
24.5529	0.0121\\
24.5637	0.0149\\
24.5747	0.0129\\
24.5854	0.0222\\
24.5964	0.0194\\
24.6069	0.0237\\
24.6172	0.0233\\
24.6275	0.0213\\
24.6385	0.0072\\
24.6489	0.0075\\
24.6597	0.0064\\
24.6706	0.0016\\
24.6811	-0.0057\\
24.6922	-0.0021\\
24.7029	-0.0117\\
24.7131	-0.0126\\
24.7233	-0.0126\\
24.7340	-0.0153\\
24.7445	-0.0206\\
24.7554	-0.0191\\
24.7656	-0.0267\\
24.7763	-0.0162\\
24.7866	-0.0165\\
24.7971	0.0000\\
24.8081	-0.0026\\
24.8184	0.0030\\
24.8286	0.0030\\
24.8397	0.0006\\
24.8504	0.0003\\
24.8607	-0.0010\\
24.8713	-0.0015\\
24.8822	-0.0007\\
24.8931	-0.0097\\
24.9038	-0.0038\\
24.9140	-0.0144\\
24.9246	-0.0139\\
24.9352	-0.0084\\
24.9461	-0.0069\\
24.9564	-0.0072\\
24.9669	-0.0085\\
24.9773	-0.0163\\
24.9882	-0.0201\\
24.9988	-0.0225\\
25.0096	-0.0275\\
25.0203	-0.0335\\
25.0314	-0.0267\\
25.0421	-0.0350\\
25.0531	-0.0383\\
25.0637	-0.0614\\
25.0746	-0.0602\\
25.0856	-0.0578\\
25.0963	-0.0663\\
25.1066	-0.0780\\
25.1178	-0.0660\\
25.1288	-0.0875\\
25.1393	-0.0883\\
25.1501	-0.0757\\
25.1612	-0.0689\\
25.1717	-0.1051\\
25.1822	-0.0941\\
25.1932	-0.0807\\
25.2042	-0.0836\\
25.2147	-0.1095\\
25.2257	-0.0962\\
25.2362	-0.1057\\
25.2463	-0.0941\\
25.2564	-0.1099\\
25.2666	-0.1166\\
25.2773	-0.0884\\
25.2881	-0.0607\\
25.2982	-0.0815\\
25.3087	-0.0688\\
25.3193	-0.0684\\
25.3295	-0.0506\\
25.3398	-0.0389\\
25.3501	-0.0388\\
25.3602	-0.0334\\
25.3704	-0.0284\\
25.3813	-0.0187\\
25.3919	-0.0333\\
25.4029	-0.0260\\
25.4139	-0.0166\\
25.4247	-0.0027\\
25.4357	0.0044\\
25.4463	-0.0086\\
25.4567	-0.0156\\
25.4670	-0.0251\\
25.4776	-0.0289\\
25.4882	-0.0320\\
25.4990	-0.0367\\
25.5098	-0.0308\\
25.5205	-0.0291\\
25.5311	-0.0416\\
25.5421	-0.0490\\
25.5530	-0.0444\\
25.5632	-0.0457\\
25.5733	-0.0464\\
25.5843	-0.0341\\
25.5952	-0.0292\\
25.6064	-0.0213\\
25.6173	-0.0206\\
25.6276	-0.0228\\
25.6385	-0.0191\\
25.6493	-0.0091\\
25.6600	-0.0045\\
25.6702	-0.0000\\
25.6806	0.0127\\
25.6911	0.0127\\
25.7013	0.0155\\
25.7122	0.0126\\
25.7229	0.0078\\
25.7337	0.0045\\
25.7440	0.0019\\
25.7549	-0.0035\\
25.7655	0.0008\\
25.7764	-0.0029\\
25.7876	-0.0063\\
25.7986	-0.0035\\
25.8090	0.0002\\
25.8200	0.0014\\
25.8302	-0.0072\\
25.8413	0.0003\\
25.8518	0.0033\\
25.8618	0.0106\\
25.8729	0.0159\\
25.8837	0.0174\\
25.8942	0.0242\\
25.9049	0.0320\\
25.9153	0.0333\\
25.9264	0.0304\\
25.9375	0.0288\\
25.9476	0.0423\\
25.9587	0.0357\\
25.9691	0.0417\\
25.9794	0.0459\\
25.9899	0.0447\\
26.0006	0.0422\\
26.0111	0.0476\\
26.0221	0.0444\\
26.0326	0.0451\\
26.0429	0.0494\\
26.0537	0.0432\\
26.0642	0.0528\\
26.0746	0.0570\\
26.0856	0.0552\\
26.0959	0.0846\\
26.1064	0.0850\\
26.1174	0.0793\\
26.1276	0.0802\\
26.1383	0.0935\\
26.1491	0.0841\\
26.1598	0.0808\\
26.1705	0.0872\\
26.1808	0.1046\\
26.1912	0.1023\\
26.2015	0.0978\\
26.2120	0.0729\\
26.2223	0.0791\\
26.2334	0.0616\\
26.2445	0.0659\\
26.2553	0.0482\\
26.2660	0.0450\\
26.2764	0.0669\\
26.2866	0.0756\\
26.2973	0.0622\\
26.3077	0.0617\\
26.3184	0.0604\\
26.3286	0.0717\\
26.3398	0.0532\\
26.3502	0.0440\\
26.3606	0.0439\\
26.3712	0.0414\\
26.3823	0.0378\\
26.3929	0.0327\\
26.4037	0.0170\\
26.4142	0.0118\\
26.4244	0.0184\\
26.4349	0.0090\\
26.4452	-0.0004\\
26.4553	-0.0076\\
26.4661	-0.0033\\
26.4764	-0.0041\\
26.4872	-0.0197\\
26.4980	-0.0192\\
26.5087	-0.0234\\
26.5189	-0.0151\\
26.5295	-0.0108\\
};
\addplot [color=mycolor2,solid,forget plot]
  table[row sep=crcr]{%
24.3170	-0.0000\\
24.3170	-0.0000\\
24.3170	-0.0000\\
24.3170	-0.0000\\
24.3170	-0.0000\\
24.3170	-0.0000\\
24.3170	-0.0000\\
24.3270	-0.0000\\
24.3370	0.0880\\
24.3470	0.0871\\
24.3570	0.1638\\
24.3670	0.2172\\
24.3770	0.1977\\
24.3870	0.1877\\
24.3970	0.1826\\
24.4070	0.1678\\
24.4170	0.1623\\
24.4270	0.1638\\
24.4370	0.1635\\
24.4570	0.1635\\
24.4670	0.1633\\
24.4770	0.1633\\
24.4870	0.1500\\
24.4970	0.1401\\
24.5070	0.1381\\
24.5170	0.1381\\
24.5270	0.1381\\
24.5370	0.1359\\
24.5470	0.1208\\
24.5570	0.1208\\
24.5670	0.1181\\
24.5770	0.1102\\
24.5870	0.1075\\
24.5970	0.1075\\
24.6070	0.1177\\
24.6270	0.1273\\
24.6370	0.1397\\
24.6470	0.1425\\
24.6570	0.1455\\
24.6670	0.1485\\
24.6770	0.1548\\
24.6870	0.1609\\
24.6970	0.1724\\
24.7070	0.1653\\
24.7170	0.1480\\
24.7270	0.1269\\
24.7370	0.1152\\
24.7470	0.1098\\
24.7570	0.1063\\
24.7670	0.1013\\
24.7770	0.0868\\
24.7870	0.0868\\
24.8070	0.0868\\
24.8170	0.0868\\
24.8270	0.0868\\
24.8370	0.0841\\
24.8470	0.0841\\
24.8570	0.0590\\
24.8670	0.0556\\
24.8770	0.0528\\
24.8870	0.1212\\
24.8970	0.1212\\
24.9070	0.1230\\
24.9170	0.1309\\
24.9270	0.1364\\
24.9370	0.1385\\
24.9470	0.1371\\
24.9570	0.1371\\
24.9670	0.1371\\
24.9870	0.1354\\
24.9970	0.1282\\
25.0070	0.0925\\
25.0170	0.0093\\
25.0270	-0.0928\\
25.0370	-0.1772\\
25.0470	-0.2634\\
25.0570	-0.3202\\
25.0670	-0.3670\\
25.0770	-0.4272\\
25.0870	-0.4959\\
25.0970	-0.5530\\
25.1070	-0.6188\\
25.1270	-0.7535\\
25.1370	-0.7958\\
25.1470	-0.8467\\
25.1570	-0.8970\\
25.1670	-0.9348\\
25.1770	-0.9917\\
25.1870	-1.0438\\
25.1970	-1.1073\\
25.2070	-1.1733\\
25.2170	-1.2462\\
25.2270	-1.3340\\
25.2370	-1.4278\\
25.2470	-1.5366\\
25.2570	-1.6460\\
25.2670	-1.7237\\
25.2870	-1.8369\\
25.2970	-1.8853\\
25.3070	-1.9079\\
25.3170	-1.8787\\
25.3270	-1.8437\\
25.3370	-1.7810\\
25.3470	-1.6940\\
25.3570	-1.6065\\
25.3670	-1.5147\\
25.3770	-1.4336\\
25.3870	-1.3669\\
25.3970	-1.2829\\
25.4070	-1.2033\\
25.4170	-1.1618\\
25.4270	-1.1175\\
25.4370	-1.0208\\
25.4470	-0.9217\\
25.4570	-0.8333\\
25.4670	-0.7537\\
25.4870	-0.6453\\
25.4970	-0.6440\\
25.5070	-0.6690\\
25.5170	-0.6874\\
25.5270	-0.7005\\
25.5370	-0.7130\\
25.5470	-0.6919\\
25.5570	-0.6660\\
25.5670	-0.6381\\
25.5770	-0.6023\\
25.5870	-0.5734\\
25.5970	-0.5542\\
25.6070	-0.5385\\
25.6270	-0.5250\\
25.6370	-0.5078\\
25.6470	-0.4823\\
25.6570	-0.4618\\
25.6670	-0.4199\\
25.6770	-0.3838\\
25.6870	-0.3562\\
25.6970	-0.1821\\
25.7070	-0.0714\\
25.7170	0.0618\\
25.7270	0.1559\\
25.7370	0.2356\\
25.7470	0.2861\\
25.7570	0.3320\\
25.7670	0.3532\\
25.7770	0.3623\\
25.7970	0.3608\\
25.8070	0.3520\\
25.8170	0.3298\\
25.8270	0.2772\\
25.8370	0.2346\\
25.8470	0.0730\\
25.8570	-0.0111\\
25.8670	-0.1052\\
25.8770	-0.1775\\
25.8870	-0.2293\\
25.8970	-0.2497\\
25.9070	-0.2467\\
25.9170	-0.2057\\
25.9270	-0.1379\\
25.9470	0.1012\\
25.9570	0.1695\\
25.9670	0.2654\\
25.9770	0.3335\\
25.9870	0.3867\\
25.9970	0.4028\\
26.0070	0.4087\\
26.0170	0.4213\\
26.0270	0.4383\\
26.0370	0.4596\\
26.0470	0.4747\\
26.0570	0.4931\\
26.0670	0.5154\\
26.0770	0.5264\\
26.0870	0.5535\\
26.0970	0.5848\\
26.1070	0.6278\\
26.1270	0.7153\\
26.1370	0.7792\\
26.1470	0.8851\\
26.1570	1.0113\\
26.1670	1.1233\\
26.1770	1.2754\\
26.1870	1.4183\\
26.1970	1.5312\\
26.2070	1.6087\\
26.2170	1.6212\\
26.2270	1.6373\\
26.2370	1.6123\\
26.2470	1.5807\\
26.2570	1.5180\\
26.2670	1.4525\\
26.2770	1.3585\\
26.2870	1.2685\\
26.3070	1.1199\\
26.3170	1.1132\\
26.3270	1.1137\\
26.3370	1.1212\\
26.3470	1.1722\\
26.3570	1.2242\\
26.3670	1.2472\\
26.3770	1.2601\\
26.3870	1.2419\\
26.3970	1.2257\\
26.4070	1.1981\\
26.4170	1.1474\\
26.4270	1.1020\\
26.4370	1.0590\\
26.4470	1.0009\\
26.4570	0.9479\\
26.4670	0.8989\\
26.4770	0.8610\\
26.4970	0.7712\\
26.5070	0.7305\\
26.5170	0.6909\\
26.5270	0.6872\\
26.5270	0.6872\\
};
\addplot [color=mycolor3,solid,forget plot]
  table[row sep=crcr]{%
24.3170	-0.0000\\
24.3170	-0.0000\\
24.3170	-0.0000\\
24.3170	-0.0000\\
24.3170	-0.0000\\
24.3170	-0.0000\\
24.3170	-0.0000\\
24.3270	-0.0000\\
24.3370	0.0746\\
24.3470	0.0714\\
24.3570	0.1053\\
24.3670	0.1337\\
24.3770	0.1239\\
24.3870	0.1154\\
24.3970	0.1115\\
24.4070	0.0964\\
24.4170	0.0927\\
24.4270	0.0947\\
24.4370	0.0949\\
24.4570	0.0949\\
24.4670	0.0950\\
24.4770	0.0960\\
24.4870	0.1017\\
24.4970	0.1092\\
24.5070	0.1083\\
24.5170	0.1083\\
24.5270	0.1083\\
24.5370	0.1075\\
24.5470	0.0961\\
24.5570	0.0961\\
24.5670	0.0943\\
24.5770	0.0856\\
24.5870	0.0683\\
24.5970	0.0683\\
24.6070	0.0702\\
24.6270	0.0729\\
24.6370	0.0766\\
24.6470	0.0799\\
24.6570	0.0836\\
24.6670	0.0852\\
24.6770	0.0896\\
24.6870	0.0969\\
24.6970	0.1112\\
24.7070	0.1303\\
24.7170	0.1538\\
24.7270	0.1675\\
24.7370	0.1737\\
24.7470	0.1714\\
24.7570	0.1702\\
24.7670	0.1684\\
24.7770	0.1545\\
24.7870	0.1545\\
24.8070	0.1545\\
24.8170	0.1545\\
24.8270	0.1545\\
24.8370	0.1523\\
24.8470	0.1523\\
24.8570	0.1270\\
24.8670	0.1018\\
24.8770	0.0963\\
24.8870	0.1125\\
24.8970	0.1125\\
24.9070	0.1108\\
24.9170	0.1122\\
24.9270	0.1144\\
24.9370	0.1151\\
24.9470	0.1136\\
24.9570	0.1136\\
24.9670	0.1136\\
24.9870	0.1137\\
24.9970	0.1328\\
25.0070	0.1328\\
25.0170	0.1254\\
25.0270	0.1170\\
25.0370	0.1089\\
25.0470	0.1020\\
25.0570	0.0942\\
25.0670	0.0913\\
25.0770	0.0867\\
25.0870	0.0761\\
25.0970	0.0633\\
25.1070	0.0537\\
25.1270	0.0296\\
25.1370	0.0312\\
25.1470	0.0312\\
25.1570	0.0307\\
25.1670	0.0310\\
25.1770	0.0310\\
25.1870	0.0308\\
25.1970	0.0266\\
25.2070	0.0214\\
25.2170	0.0204\\
25.2270	0.0184\\
25.2370	0.0154\\
25.2470	0.0056\\
25.2570	-0.0053\\
25.2670	-0.0124\\
25.2870	-0.0299\\
25.2970	-0.0514\\
25.3070	-0.0797\\
25.3170	-0.0941\\
25.3270	-0.1140\\
25.3370	-0.1461\\
25.3470	-0.1582\\
25.3570	-0.1594\\
25.3670	-0.1586\\
25.3770	-0.1540\\
25.3870	-0.1463\\
25.3970	-0.1384\\
25.4070	-0.1313\\
25.4170	-0.1302\\
25.4270	-0.1395\\
25.4370	-0.1421\\
25.4470	-0.1382\\
25.4570	-0.1277\\
25.4670	-0.1156\\
25.4870	-0.0735\\
25.4970	-0.0671\\
25.5070	-0.0676\\
25.5170	-0.0758\\
25.5270	-0.0869\\
25.5370	-0.0962\\
25.5470	-0.0951\\
25.5570	-0.0909\\
25.5670	-0.0839\\
25.5770	-0.0785\\
25.5870	-0.0774\\
25.5970	-0.0809\\
25.6070	-0.0992\\
25.6270	-0.1592\\
25.6370	-0.1960\\
25.6470	-0.2265\\
25.6570	-0.2359\\
25.6670	-0.2415\\
25.6770	-0.2491\\
25.6870	-0.2663\\
25.6970	-0.2100\\
25.7070	-0.1841\\
25.7170	-0.1443\\
25.7270	-0.1092\\
25.7370	-0.0719\\
25.7470	-0.0442\\
25.7570	0.0056\\
25.7670	0.0498\\
25.7770	0.1029\\
25.7970	0.1881\\
25.8070	0.2066\\
25.8170	0.2256\\
25.8270	0.2412\\
25.8370	0.2483\\
25.8470	0.2185\\
25.8570	0.1867\\
25.8670	0.1477\\
25.8770	0.0707\\
25.8870	-0.0185\\
25.8970	-0.0843\\
25.9070	-0.1321\\
25.9170	-0.1464\\
25.9270	-0.1520\\
25.9470	-0.1259\\
25.9570	-0.1135\\
25.9670	-0.0851\\
25.9770	-0.0711\\
25.9870	-0.0607\\
25.9970	-0.0595\\
26.0070	-0.0589\\
26.0170	-0.0586\\
26.0270	-0.0583\\
26.0370	-0.0566\\
26.0470	-0.0553\\
26.0570	-0.0543\\
26.0670	-0.0535\\
26.0770	-0.0523\\
26.0870	-0.0517\\
26.0970	-0.0597\\
26.1070	-0.0758\\
26.1270	-0.0770\\
26.1370	-0.0724\\
26.1470	-0.0724\\
26.1570	-0.0724\\
26.1670	-0.0529\\
26.1770	-0.0272\\
26.1870	0.0093\\
26.1970	0.0272\\
26.2070	0.0417\\
26.2170	0.0506\\
26.2270	0.0535\\
26.2370	0.0598\\
26.2470	0.0624\\
26.2570	0.0637\\
26.2670	0.0674\\
26.2770	0.0669\\
26.2870	0.0596\\
26.3070	0.0379\\
26.3170	0.0381\\
26.3270	0.0368\\
26.3370	0.0368\\
26.3470	0.0385\\
26.3570	0.0455\\
26.3670	0.0465\\
26.3770	0.0478\\
26.3870	0.0483\\
26.3970	0.0541\\
26.4070	0.0569\\
26.4170	0.0616\\
26.4270	0.0640\\
26.4370	0.0676\\
26.4470	0.0734\\
26.4570	0.0778\\
26.4670	0.0789\\
26.4770	0.0802\\
26.4970	0.1105\\
26.5070	0.1303\\
26.5170	0.1564\\
26.5270	0.1588\\
26.5270	0.1588\\
};
\end{axis}

\begin{axis}[%
width=0.96\fwidth,
height=0.279\fwidth,
at={(0\fwidth,0\fwidth)},
scale only axis,
xmin=28.9184,
xmax=31.9202,
xlabel={$t$ [s]},
ymin=-2.5000,
ymax=3.4560,
ylabel={$r$ [rad/s], $,\vartheta_y$ [1/s]},
axis background/.style={fill=white},
legend style={legend cell align=left,align=left,draw=white!15!black},
legend columns=2,
legend style={at={(0.5,-0.55)},anchor=north},
title style={font=\labelsize},
xlabel style={font=\labelsize,at={(axis description cs:0.5,\xlabeldist)}},
ylabel style={font=\labelsize,at={(axis description cs:\ylabeldist,0.5)}},
legend style={font=\ticksize},
ticklabel style={font=\ticksize}
]
\addplot [color=black,dashed]
  table[row sep=crcr]{%
28.9202	0.0820\\
28.9300	0.0590\\
28.9398	0.0260\\
28.9507	-0.0140\\
28.9612	-0.0480\\
28.9715	-0.0710\\
28.9827	-0.0850\\
28.9944	-0.0770\\
29.0042	-0.0630\\
29.0159	-0.0390\\
29.0257	-0.0110\\
29.0374	0.0180\\
29.0471	0.0400\\
29.0589	0.0530\\
29.0706	0.0370\\
29.0805	0.0190\\
29.0905	-0.0070\\
29.1018	-0.0410\\
29.1127	-0.0690\\
29.1223	-0.0870\\
29.1317	-0.1030\\
29.1419	-0.1090\\
29.1526	-0.1120\\
29.1625	-0.1090\\
29.1742	-0.1030\\
29.1839	-0.1000\\
29.1956	-0.0920\\
29.2053	-0.0800\\
29.2151	-0.0710\\
29.2249	-0.0590\\
29.2366	-0.0480\\
29.2464	-0.0410\\
29.2581	-0.0420\\
29.2678	-0.0440\\
29.2836	-0.0640\\
29.2893	-0.0640\\
29.3011	-0.0600\\
29.3108	-0.0550\\
29.3225	-0.0370\\
29.3323	-0.0200\\
29.3421	-0.0090\\
29.3538	-0.0080\\
29.3636	-0.0160\\
29.3753	-0.0330\\
29.3870	-0.0590\\
29.3968	-0.0790\\
29.4065	-0.1070\\
29.4163	-0.1330\\
29.4280	-0.1580\\
29.4378	-0.1670\\
29.4495	-0.1760\\
29.4593	-0.1720\\
29.4719	-0.1340\\
29.4823	-0.1130\\
29.4917	-0.0910\\
29.5022	-0.0780\\
29.5140	-0.0800\\
29.5237	-0.0980\\
29.5354	-0.1400\\
29.5452	-0.1970\\
29.5550	-0.2720\\
29.5647	-0.3540\\
29.5765	-0.4510\\
29.5862	-0.5610\\
29.5979	-0.6750\\
29.6077	-0.7930\\
29.6194	-0.9230\\
29.6292	-1.0380\\
29.6417	-1.1910\\
29.6530	-1.1870\\
29.6604	-1.1870\\
29.6726	-1.1230\\
29.6846	-1.0530\\
29.6948	-1.0150\\
29.7045	-1.0590\\
29.7144	-1.1480\\
29.7245	-1.2510\\
29.7346	-1.3460\\
29.7464	-1.4290\\
29.7561	-1.4960\\
29.7679	-1.5420\\
29.7777	-1.5920\\
29.7893	-1.6790\\
29.7991	-1.7990\\
29.8154	-2.2950\\
29.8206	-2.2950\\
29.8304	-2.4910\\
29.8421	-2.5830\\
29.8560	-2.5760\\
29.8636	-2.5760\\
29.8733	-2.4420\\
29.8894	-2.2340\\
29.8948	-2.2340\\
29.9066	-2.2510\\
29.9163	-2.2860\\
29.9261	-2.2970\\
29.9378	-2.2920\\
29.9470	-2.2120\\
29.9593	-2.1520\\
29.9691	-2.0560\\
29.9807	-1.9460\\
29.9905	-1.8720\\
30.0003	-1.8050\\
30.0120	-1.7440\\
30.0218	-1.6860\\
30.0349	-1.5500\\
30.0449	-1.4830\\
30.0547	-1.4160\\
30.0647	-1.3450\\
30.0764	-1.2780\\
30.0862	-1.2260\\
30.0960	-1.1880\\
30.1077	-1.1670\\
30.1175	-1.1550\\
30.1292	-1.1570\\
30.1390	-1.1700\\
30.1487	-1.1540\\
30.1605	-1.1200\\
30.1702	-1.0900\\
30.1800	-1.0640\\
30.1934	-1.0280\\
30.2026	-1.0380\\
30.2137	-1.0510\\
30.2245	-1.1210\\
30.2343	-1.1800\\
30.2443	-1.2310\\
30.2537	-1.2580\\
30.2639	-1.2790\\
30.2757	-1.2870\\
30.2855	-1.2690\\
30.2972	-1.2430\\
30.3069	-1.1950\\
30.3187	-1.1080\\
30.3284	-1.0140\\
30.3401	-0.9200\\
30.3499	-0.8100\\
30.3597	-0.7250\\
30.3696	-0.6670\\
30.3831	-0.6200\\
30.3909	-0.6110\\
30.4051	-0.5950\\
30.4124	-0.5950\\
30.4222	-0.5620\\
30.4352	-0.4700\\
30.4450	-0.4170\\
30.4547	-0.3780\\
30.4644	-0.3350\\
30.4743	-0.2950\\
30.4846	-0.2510\\
30.4964	-0.2000\\
30.5061	-0.1510\\
30.5179	-0.1110\\
30.5276	-0.0870\\
30.5394	-0.0840\\
30.5491	-0.0920\\
30.5621	-0.1200\\
30.5726	-0.1220\\
30.5804	-0.1220\\
30.5934	-0.0860\\
30.6030	-0.0610\\
30.6128	-0.0400\\
30.6233	-0.0340\\
30.6331	-0.0380\\
30.6429	-0.0400\\
30.6546	-0.0340\\
30.6643	-0.0270\\
30.6761	-0.0120\\
30.6858	0.0000\\
30.6975	0.0000\\
30.7074	-0.0160\\
30.7171	-0.0470\\
30.7288	-0.0780\\
30.7386	-0.1170\\
30.7503	-0.1400\\
30.7601	-0.1470\\
30.7732	-0.1350\\
30.7829	-0.1200\\
30.7925	-0.1020\\
30.8021	-0.0830\\
30.8128	-0.0660\\
30.8245	-0.0520\\
30.8343	-0.0390\\
30.8440	-0.0310\\
30.8557	-0.0120\\
30.8655	0.0000\\
30.8772	0.0110\\
30.8870	0.0190\\
30.8987	0.0270\\
30.9085	0.0370\\
30.9183	0.0530\\
30.9280	0.0720\\
30.9397	0.1000\\
30.9495	0.1240\\
30.9624	0.1550\\
30.9710	0.1410\\
30.9825	0.1090\\
30.9925	0.0740\\
31.0022	0.0280\\
31.0132	-0.0150\\
31.0235	-0.0500\\
31.0336	-0.0670\\
31.0452	-0.0720\\
31.0550	-0.0580\\
31.0667	-0.0440\\
31.0764	-0.0240\\
31.0882	0.0000\\
31.0979	0.0330\\
31.1077	0.0650\\
31.1194	0.1130\\
31.1292	0.1680\\
31.1417	0.2670\\
31.1529	0.3040\\
31.1605	0.3040\\
31.1732	0.3800\\
31.1834	0.4260\\
31.1932	0.4780\\
31.2028	0.5920\\
31.2129	0.7510\\
31.2231	0.8760\\
31.2347	0.9820\\
31.2464	1.0510\\
31.2561	1.1060\\
31.2679	1.1770\\
31.2776	1.2470\\
31.2893	1.3190\\
31.2991	1.4460\\
31.3108	1.6100\\
31.3206	1.7960\\
31.3304	1.9740\\
31.3439	2.3450\\
31.3537	2.5400\\
31.3635	2.7790\\
31.3733	2.9570\\
31.3832	3.1560\\
31.3948	3.3180\\
31.4070	3.4190\\
31.4143	3.4190\\
31.4281	3.4420\\
31.4358	3.4420\\
31.4492	3.3800\\
31.4621	3.3370\\
31.4691	3.3370\\
31.4788	3.3070\\
31.4905	3.2320\\
31.5003	3.1110\\
31.5100	2.9870\\
31.5218	2.8570\\
31.5315	2.7510\\
31.5433	2.6200\\
31.5536	2.4870\\
31.5644	2.3780\\
31.5745	2.2660\\
31.5845	2.1300\\
31.5960	1.9750\\
31.6058	1.8120\\
31.6175	1.6320\\
31.6272	1.4630\\
31.6390	1.3310\\
31.6487	1.2140\\
31.6585	1.1220\\
31.6702	1.0610\\
31.6800	0.9980\\
31.6897	0.9410\\
31.7015	0.8750\\
31.7112	0.7950\\
31.7242	0.6510\\
31.7340	0.5740\\
31.7437	0.4900\\
31.7542	0.4370\\
31.7659	0.3810\\
31.7757	0.3150\\
31.7854	0.2700\\
31.7972	0.2220\\
31.8069	0.1860\\
31.8167	0.1690\\
31.8284	0.1700\\
31.8401	0.1600\\
31.8499	0.1600\\
31.8597	0.1490\\
31.8721	0.1110\\
31.8828	0.0870\\
31.8929	0.0630\\
31.9027	0.0320\\
31.9143	0.0090\\
};
\addlegendentry{-$p$, $q$, $r$};

\addplot [color=mycolor1,solid]
  table[row sep=crcr]{%
28.9184	0.0105\\
28.9291	0.0138\\
28.9396	0.0124\\
28.9501	0.0078\\
28.9608	0.0040\\
28.9713	0.0056\\
28.9821	0.0049\\
28.9927	0.0017\\
29.0033	-0.0017\\
29.0143	-0.0077\\
29.0252	-0.0055\\
29.0360	-0.0019\\
29.0468	-0.0045\\
29.0578	-0.0018\\
29.0688	-0.0018\\
29.0792	0.0028\\
29.0897	-0.0032\\
29.0999	-0.0050\\
29.1108	0.0001\\
29.1211	-0.0000\\
29.1312	0.0012\\
29.1415	0.0045\\
29.1522	0.0005\\
29.1625	-0.0045\\
29.1730	-0.0108\\
29.1834	-0.0133\\
29.1939	-0.0146\\
29.2042	-0.0120\\
29.2142	-0.0058\\
29.2244	0.0034\\
29.2351	0.0082\\
29.2457	0.0126\\
29.2562	0.0088\\
29.2671	-0.0025\\
29.2781	-0.0014\\
29.2891	0.0025\\
29.2997	0.0051\\
29.3107	0.0004\\
29.3213	-0.0002\\
29.3314	-0.0058\\
29.3418	-0.0148\\
29.3528	-0.0142\\
29.3635	-0.0271\\
29.3742	-0.0265\\
29.3853	-0.0141\\
29.3956	-0.0161\\
29.4060	-0.0163\\
29.4163	-0.0183\\
29.4264	-0.0169\\
29.4371	-0.0110\\
29.4476	-0.0141\\
29.4586	-0.0138\\
29.4697	-0.0120\\
29.4803	-0.0032\\
29.4910	-0.0040\\
29.5017	-0.0019\\
29.5128	-0.0136\\
29.5231	-0.0198\\
29.5338	-0.0200\\
29.5441	-0.0225\\
29.5544	-0.0369\\
29.5644	-0.0385\\
29.5752	-0.0477\\
29.5857	-0.0543\\
29.5968	-0.0513\\
29.6074	-0.0604\\
29.6177	-0.0703\\
29.6282	-0.0695\\
29.6393	-0.0739\\
29.6499	-0.0765\\
29.6603	-0.0760\\
29.6707	-0.0641\\
29.6809	-0.0641\\
29.6919	-0.0601\\
29.7021	-0.0667\\
29.7125	-0.0738\\
29.7229	-0.0677\\
29.7340	-0.0828\\
29.7451	-0.0766\\
29.7560	-0.0710\\
29.7666	-0.0750\\
29.7771	-0.0764\\
29.7878	-0.0923\\
29.7986	-0.1060\\
29.8091	-0.1011\\
29.8196	-0.1198\\
29.8302	-0.1314\\
29.8409	-0.1532\\
29.8520	-0.1399\\
29.8623	-0.1381\\
29.8729	-0.1467\\
29.8836	-0.1480\\
29.8944	-0.1471\\
29.9048	-0.1492\\
29.9152	-0.1473\\
29.9260	-0.1640\\
29.9365	-0.1484\\
29.9468	-0.1498\\
29.9578	-0.1493\\
29.9687	-0.1565\\
29.9791	-0.1487\\
29.9893	-0.1447\\
29.9996	-0.1408\\
30.0105	-0.1489\\
30.0216	-0.1376\\
30.0323	-0.1444\\
30.0428	-0.1495\\
30.0536	-0.1430\\
30.0642	-0.1395\\
30.0752	-0.1293\\
30.0854	-0.1180\\
30.0958	-0.1390\\
30.1066	-0.1283\\
30.1171	-0.1318\\
30.1274	-0.1271\\
30.1379	-0.1386\\
30.1480	-0.1321\\
30.1589	-0.1246\\
30.1696	-0.1390\\
30.1799	-0.1193\\
30.1908	-0.1301\\
30.2010	-0.1479\\
30.2113	-0.1463\\
30.2217	-0.1477\\
30.2320	-0.1509\\
30.2424	-0.1418\\
30.2530	-0.1288\\
30.2633	-0.1396\\
30.2740	-0.1313\\
30.2849	-0.1293\\
30.2957	-0.1401\\
30.3062	-0.1352\\
30.3170	-0.1167\\
30.3280	-0.0985\\
30.3383	-0.1145\\
30.3489	-0.0955\\
30.3593	-0.1157\\
30.3695	-0.1043\\
30.3802	-0.0957\\
30.3909	-0.0870\\
30.4010	-0.0796\\
30.4113	-0.0742\\
30.4218	-0.0702\\
30.4321	-0.0682\\
30.4427	-0.0681\\
30.4532	-0.0585\\
30.4634	-0.0676\\
30.4740	-0.0474\\
30.4843	-0.0507\\
30.4949	-0.0256\\
30.5061	-0.0233\\
30.5166	-0.0279\\
30.5272	-0.0244\\
30.5379	-0.0030\\
30.5484	-0.0044\\
30.5592	-0.0069\\
30.5698	-0.0074\\
30.5801	0.0028\\
30.5910	-0.0025\\
30.6016	0.0022\\
30.6118	-0.0026\\
30.6223	-0.0024\\
30.6325	0.0024\\
30.6426	-0.0030\\
30.6533	-0.0208\\
30.6643	-0.0238\\
30.6750	-0.0230\\
30.6855	-0.0153\\
30.6958	-0.0235\\
30.7060	-0.0204\\
30.7164	-0.0251\\
30.7270	-0.0223\\
30.7380	-0.0238\\
30.7487	-0.0252\\
30.7594	-0.0155\\
30.7705	-0.0068\\
30.7809	0.0000\\
30.7917	0.0022\\
30.8019	0.0020\\
30.8123	0.0127\\
30.8226	0.0182\\
30.8332	0.0278\\
30.8437	0.0243\\
30.8545	0.0457\\
30.8651	0.0510\\
30.8760	0.0487\\
30.8862	0.0416\\
30.8968	0.0388\\
30.9071	0.0435\\
30.9175	0.0450\\
30.9280	0.0430\\
30.9384	0.0428\\
30.9491	0.0438\\
30.9602	0.0472\\
30.9709	0.0341\\
30.9813	0.0296\\
30.9917	0.0270\\
31.0021	0.0304\\
31.0125	0.0183\\
31.0231	0.0156\\
31.0334	0.0137\\
31.0436	0.0106\\
31.0547	0.0141\\
31.0656	0.0099\\
31.0760	0.0066\\
31.0862	0.0077\\
31.0968	0.0039\\
31.1075	0.0097\\
31.1182	0.0153\\
31.1290	0.0265\\
31.1392	0.0266\\
31.1497	0.0355\\
31.1604	0.0443\\
31.1707	0.0435\\
31.1809	0.0438\\
31.1910	0.0505\\
31.2019	0.0435\\
31.2126	0.0493\\
31.2230	0.0575\\
31.2342	0.0697\\
31.2446	0.0841\\
31.2555	0.1021\\
31.2663	0.1001\\
31.2771	0.0968\\
31.2879	0.1110\\
31.2986	0.1148\\
31.3094	0.1178\\
31.3197	0.1258\\
31.3300	0.1431\\
31.3408	0.1410\\
31.3511	0.1484\\
31.3612	0.1602\\
31.3723	0.1566\\
31.3826	0.1828\\
31.3935	0.1904\\
31.4037	0.1913\\
31.4142	0.2023\\
31.4249	0.1894\\
31.4354	0.2072\\
31.4456	0.1968\\
31.4562	0.1966\\
31.4673	0.1827\\
31.4780	0.1666\\
31.4891	0.1581\\
31.4995	0.1359\\
31.5097	0.1258\\
31.5201	0.1304\\
31.5311	0.1287\\
31.5419	0.1359\\
31.5529	0.1248\\
31.5632	0.1398\\
31.5738	0.1336\\
31.5841	0.1178\\
31.5952	0.1208\\
31.6056	0.1273\\
31.6157	0.1097\\
31.6265	0.0934\\
31.6373	0.0659\\
31.6473	0.0542\\
31.6583	0.0423\\
31.6688	0.0458\\
31.6790	0.0287\\
31.6893	0.0268\\
31.6996	0.0403\\
31.7106	0.0344\\
31.7215	0.0247\\
31.7323	0.0374\\
31.7428	0.0439\\
31.7536	0.0562\\
31.7644	0.0731\\
31.7748	0.0694\\
31.7854	0.0579\\
31.7961	0.0587\\
31.8061	0.0475\\
31.8167	0.0378\\
31.8273	0.0258\\
31.8382	0.0327\\
31.8485	0.0184\\
31.8595	0.0201\\
31.8705	0.0242\\
31.8808	0.0135\\
31.8917	0.0123\\
31.9026	0.0155\\
31.9131	0.0129\\
};
\addlegendentry{Ground Truth};

\addplot [color=mycolor2,solid]
  table[row sep=crcr]{%
28.9802	-0.0000\\
28.9802	-0.0000\\
28.9802	-0.0000\\
28.9802	-0.0000\\
28.9802	-0.0000\\
28.9802	-0.0000\\
28.9902	-0.0000\\
29.0002	0.0583\\
29.0102	0.0829\\
29.0202	0.0829\\
29.0302	0.0840\\
29.0402	0.0840\\
29.0502	0.0840\\
29.0602	0.0840\\
29.0702	0.0840\\
29.0802	0.0840\\
29.0902	0.0840\\
29.1002	0.0840\\
29.1202	0.0866\\
29.1302	0.0337\\
29.1402	0.0136\\
29.1502	0.0136\\
29.1602	0.0099\\
29.1702	0.0099\\
29.1802	0.0099\\
29.1902	-0.0172\\
29.2002	-0.0172\\
29.2102	-0.0237\\
29.2202	-0.0524\\
29.2302	-0.0581\\
29.2402	-0.0566\\
29.2502	-0.0409\\
29.2602	-0.0002\\
29.2702	0.0883\\
29.2802	0.1881\\
29.2902	0.2602\\
29.3002	0.3107\\
29.3202	0.3341\\
29.3302	0.3302\\
29.3402	0.3254\\
29.3502	0.3218\\
29.3602	0.3161\\
29.3702	0.3121\\
29.3802	0.2924\\
29.3902	0.2584\\
29.4002	0.2551\\
29.4102	0.2176\\
29.4202	0.1629\\
29.4302	0.0864\\
29.4402	0.0833\\
29.4502	0.0707\\
29.4602	0.0658\\
29.4702	0.0514\\
29.4902	0.1539\\
29.5002	0.2673\\
29.5102	0.2966\\
29.5202	0.3198\\
29.5302	0.3292\\
29.5402	0.3293\\
29.5502	0.3218\\
29.5602	0.3189\\
29.5702	0.3116\\
29.5802	0.2748\\
29.5902	0.2527\\
29.6002	0.1616\\
29.6102	0.0490\\
29.6202	-0.0866\\
29.6302	-0.1776\\
29.6402	-0.2874\\
29.6502	-0.3938\\
29.6702	-0.5540\\
29.6802	-0.5588\\
29.6902	-0.5621\\
29.7002	-0.5621\\
29.7102	-0.5798\\
29.7202	-0.6054\\
29.7302	-0.5772\\
29.7402	-0.5512\\
29.7502	-0.4869\\
29.7602	-0.4521\\
29.7702	-0.4027\\
29.7802	-0.4045\\
29.7902	-0.3519\\
29.8002	-0.3162\\
29.8102	-0.1748\\
29.8202	-0.0242\\
29.8302	0.0597\\
29.8502	-0.1847\\
29.8602	-0.4270\\
29.8702	-0.6578\\
29.8802	-0.7875\\
29.8902	-0.8467\\
29.9002	-0.8198\\
29.9102	-0.7688\\
29.9202	-0.7310\\
29.9302	-0.7359\\
29.9402	-0.6900\\
29.9502	-0.7226\\
29.9602	-0.7246\\
29.9702	-0.6719\\
29.9802	-0.6111\\
29.9902	-0.5088\\
30.0002	-0.5324\\
30.0202	-0.4221\\
30.0302	-0.3788\\
30.0402	-0.2780\\
30.0502	-0.1556\\
30.0602	-0.1222\\
30.0702	-0.1305\\
30.0802	-0.1279\\
30.0902	-0.0914\\
30.1002	-0.1072\\
30.1102	-0.1312\\
30.1202	-0.1649\\
30.1302	-0.2287\\
30.1402	-0.2557\\
30.1502	-0.3146\\
30.1602	-0.3614\\
30.1702	-0.3949\\
30.1802	-0.4149\\
30.2002	-0.4589\\
30.2102	-0.4828\\
30.2202	-0.5174\\
30.2302	-0.5464\\
30.2402	-0.5935\\
30.2502	-0.6291\\
30.2602	-0.6629\\
30.2702	-0.7176\\
30.2802	-0.7074\\
30.2902	-0.7280\\
30.3002	-0.7456\\
30.3102	-0.7374\\
30.3202	-0.7249\\
30.3302	-0.6983\\
30.3402	-0.6952\\
30.3502	-0.6755\\
30.3602	-0.6436\\
30.3702	-0.5985\\
30.3902	-0.6658\\
30.4002	-0.6853\\
30.4102	-0.6879\\
30.4202	-0.7154\\
30.4302	-0.7555\\
30.4402	-0.7652\\
30.4502	-0.7550\\
30.4602	-0.7338\\
30.4702	-0.7034\\
30.4802	-0.6761\\
30.4902	-0.6293\\
30.5002	-0.5999\\
30.5102	-0.5640\\
30.5202	-0.5318\\
30.5302	-0.5061\\
30.5402	-0.4842\\
30.5502	-0.4652\\
30.5602	-0.4652\\
30.5702	-0.4652\\
30.5802	-0.2745\\
30.6002	0.2243\\
30.6102	0.3057\\
30.6202	0.3207\\
30.6302	0.3229\\
30.6402	0.3131\\
30.6502	0.2962\\
30.6602	0.2907\\
30.6702	0.2907\\
30.6802	0.2045\\
30.6902	-0.0121\\
30.7002	-0.2311\\
30.7102	-0.3662\\
30.7202	-0.4533\\
30.7302	-0.4752\\
30.7402	-0.4713\\
30.7502	-0.4751\\
30.7602	-0.4773\\
30.7802	-0.4892\\
30.7902	-0.4941\\
30.8002	-0.4911\\
30.8102	-0.4774\\
30.8202	-0.4620\\
30.8302	-0.4570\\
30.8402	-0.4570\\
30.8502	-0.4544\\
30.8602	-0.4538\\
30.8702	-0.4536\\
30.8802	-0.4532\\
30.8902	-0.4532\\
30.9002	-0.2895\\
30.9102	-0.2659\\
30.9202	-0.2574\\
30.9302	-0.0318\\
30.9402	0.1440\\
30.9502	0.2133\\
30.9702	0.2967\\
30.9802	0.2981\\
30.9902	0.2916\\
31.0002	0.2870\\
31.0102	0.2868\\
31.0202	0.2810\\
31.0302	0.2810\\
31.0402	0.2042\\
31.0502	0.0439\\
31.0602	-0.1001\\
31.0702	-0.1943\\
31.0802	-0.2462\\
31.0902	-0.2735\\
31.1002	-0.2812\\
31.1102	-0.2892\\
31.1202	-0.2893\\
31.1302	-0.2893\\
31.1402	-0.2893\\
31.1502	-0.2893\\
31.1702	-0.2893\\
31.1802	-0.2887\\
31.1902	-0.2840\\
31.2002	-0.2384\\
31.2102	-0.1101\\
31.2202	0.0932\\
31.2302	0.2040\\
31.2402	0.2274\\
31.2502	0.2378\\
31.2602	0.2345\\
31.2702	0.2177\\
31.2802	0.2346\\
31.2902	0.2650\\
31.3002	0.2968\\
31.3102	0.3320\\
31.3202	0.3568\\
31.3302	0.3727\\
31.3502	0.4306\\
31.3602	0.4423\\
31.3702	0.4448\\
31.3802	0.4548\\
31.3902	0.4522\\
31.4002	0.4517\\
31.4102	0.4556\\
31.4202	0.4514\\
31.4302	0.4319\\
31.4402	0.4280\\
31.4502	0.4458\\
31.4602	0.4432\\
31.4702	0.3688\\
31.4802	0.2638\\
31.4902	0.2896\\
31.5002	0.3488\\
31.5102	0.3904\\
31.5202	0.3215\\
31.5402	0.1714\\
31.5502	0.0636\\
31.5602	0.1491\\
31.5702	0.2067\\
31.5802	0.2877\\
31.5902	0.3620\\
31.6002	0.4065\\
31.6102	0.4403\\
31.6202	0.5275\\
31.6302	0.6093\\
31.6402	0.6213\\
31.6502	0.7987\\
31.6602	1.0084\\
31.6702	1.0932\\
31.6802	1.1612\\
31.6902	1.1368\\
31.7002	1.0168\\
31.7202	0.5592\\
31.7302	0.3277\\
31.7402	0.1835\\
31.7502	0.1299\\
31.7602	0.1116\\
31.7702	0.1016\\
31.7802	0.1032\\
31.7902	0.1032\\
31.8002	0.1378\\
31.8102	0.1565\\
31.8202	0.2271\\
31.8302	0.2559\\
31.8402	0.2611\\
31.8502	0.2666\\
31.8602	0.2664\\
31.8802	0.2644\\
31.8902	0.2388\\
31.9002	0.1926\\
31.9102	0.1133\\
31.9202	0.0843\\
};
\addlegendentry{Estimate};

\addplot [color=mycolor3,solid]
  table[row sep=crcr]{%
28.9802	-0.0000\\
28.9802	-0.0000\\
28.9802	-0.0000\\
28.9802	-0.0000\\
28.9802	-0.0000\\
28.9802	-0.0000\\
28.9902	-0.0000\\
29.0002	0.0352\\
29.0102	0.0524\\
29.0202	0.0524\\
29.0302	0.0529\\
29.0402	0.0529\\
29.0502	0.0529\\
29.0602	0.0529\\
29.0702	0.0529\\
29.0802	0.0529\\
29.0902	0.0529\\
29.1002	0.0529\\
29.1202	0.0552\\
29.1302	0.0275\\
29.1402	0.0182\\
29.1502	0.0182\\
29.1602	0.0160\\
29.1702	0.0156\\
29.1802	0.0156\\
29.1902	-0.0219\\
29.2002	-0.0222\\
29.2102	-0.0332\\
29.2202	-0.0515\\
29.2302	-0.0656\\
29.2402	-0.0947\\
29.2502	-0.0991\\
29.2602	-0.0982\\
29.2702	-0.0759\\
29.2802	-0.0288\\
29.2902	0.0162\\
29.3002	0.0614\\
29.3202	0.1174\\
29.3302	0.1216\\
29.3402	0.1240\\
29.3502	0.1269\\
29.3602	0.1329\\
29.3702	0.1369\\
29.3802	0.1503\\
29.3902	0.1611\\
29.4002	0.1597\\
29.4102	0.1292\\
29.4202	0.0807\\
29.4302	0.0197\\
29.4402	0.0109\\
29.4502	0.0096\\
29.4602	0.0071\\
29.4702	0.0028\\
29.4902	0.0644\\
29.5002	0.1158\\
29.5102	0.1311\\
29.5202	0.1541\\
29.5302	0.1713\\
29.5402	0.1714\\
29.5502	0.1707\\
29.5602	0.1712\\
29.5702	0.1715\\
29.5802	0.1598\\
29.5902	0.1511\\
29.6002	0.1468\\
29.6102	0.1374\\
29.6202	0.1277\\
29.6302	0.1182\\
29.6402	0.0941\\
29.6502	0.0710\\
29.6702	0.0337\\
29.6802	-0.0081\\
29.6902	-0.0747\\
29.7002	-0.1143\\
29.7102	-0.1207\\
29.7202	-0.1185\\
29.7302	-0.1179\\
29.7402	-0.1179\\
29.7502	-0.1150\\
29.7602	-0.1078\\
29.7702	-0.0952\\
29.7802	-0.0723\\
29.7902	-0.0458\\
29.8002	-0.0355\\
29.8102	-0.0355\\
29.8202	0.0592\\
29.8302	0.1782\\
29.8502	0.2017\\
29.8602	0.1746\\
29.8702	0.1662\\
29.8802	0.1583\\
29.8902	0.1376\\
29.9002	0.1048\\
29.9102	0.0928\\
29.9202	0.0677\\
29.9302	0.0418\\
29.9402	0.0200\\
29.9502	-0.0104\\
29.9602	-0.0438\\
29.9702	-0.0856\\
29.9802	-0.1019\\
29.9902	-0.1005\\
30.0002	-0.1077\\
30.0202	-0.0933\\
30.0302	-0.0873\\
30.0402	-0.0728\\
30.0502	-0.0518\\
30.0602	-0.0126\\
30.0702	0.0277\\
30.0802	0.0417\\
30.0902	0.0518\\
30.1002	0.0667\\
30.1102	0.0826\\
30.1202	0.1004\\
30.1302	0.0935\\
30.1402	0.0708\\
30.1502	0.0113\\
30.1602	-0.0580\\
30.1702	-0.0909\\
30.1802	-0.1255\\
30.2002	-0.1541\\
30.2102	-0.1479\\
30.2202	-0.1446\\
30.2302	-0.1425\\
30.2402	-0.1419\\
30.2502	-0.1423\\
30.2602	-0.1442\\
30.2702	-0.1494\\
30.2802	-0.1510\\
30.2902	-0.1520\\
30.3002	-0.1494\\
30.3102	-0.1364\\
30.3202	-0.1255\\
30.3302	-0.1231\\
30.3402	-0.1169\\
30.3502	-0.1161\\
30.3602	-0.1003\\
30.3702	-0.0870\\
30.3902	-0.0924\\
30.4002	-0.0964\\
30.4102	-0.1030\\
30.4202	-0.1080\\
30.4302	-0.1166\\
30.4402	-0.1214\\
30.4502	-0.1264\\
30.4602	-0.1309\\
30.4702	-0.1428\\
30.4802	-0.1632\\
30.4902	-0.1715\\
30.5002	-0.1758\\
30.5102	-0.1859\\
30.5202	-0.1953\\
30.5302	-0.1945\\
30.5402	-0.2014\\
30.5502	-0.2030\\
30.5602	-0.2067\\
30.5702	-0.2067\\
30.5802	-0.1512\\
30.6002	0.1952\\
30.6102	0.2521\\
30.6202	0.2756\\
30.6302	0.2840\\
30.6402	0.2800\\
30.6502	0.2722\\
30.6602	0.2685\\
30.6702	0.2685\\
30.6802	0.2346\\
30.6902	0.1195\\
30.7002	-0.0013\\
30.7102	-0.0819\\
30.7202	-0.1432\\
30.7302	-0.1633\\
30.7402	-0.1570\\
30.7502	-0.1515\\
30.7602	-0.1501\\
30.7802	-0.1704\\
30.7902	-0.1738\\
30.8002	-0.1755\\
30.8102	-0.1759\\
30.8202	-0.1756\\
30.8302	-0.1756\\
30.8402	-0.1756\\
30.8502	-0.1765\\
30.8602	-0.1769\\
30.8702	-0.1771\\
30.8802	-0.1778\\
30.8902	-0.1778\\
30.9002	-0.1088\\
30.9102	-0.0979\\
30.9202	-0.0964\\
30.9302	-0.0120\\
30.9402	0.0886\\
30.9502	0.1141\\
30.9702	0.1809\\
30.9802	0.2082\\
30.9902	0.2258\\
31.0002	0.2312\\
31.0102	0.2321\\
31.0202	0.2350\\
31.0302	0.2350\\
31.0402	0.2223\\
31.0502	0.1733\\
31.0602	0.1229\\
31.0702	0.0910\\
31.0802	0.0748\\
31.0902	0.0670\\
31.1002	0.0654\\
31.1102	0.0621\\
31.1202	0.0503\\
31.1302	0.0433\\
31.1402	0.0367\\
31.1502	0.0260\\
31.1702	0.0026\\
31.1802	0.0013\\
31.1902	0.0008\\
31.2002	0.0116\\
31.2102	0.0561\\
31.2202	0.1012\\
31.2302	0.1068\\
31.2402	0.1073\\
31.2502	0.1070\\
31.2602	0.1035\\
31.2702	0.1000\\
31.2802	0.0985\\
31.2902	0.0915\\
31.3002	0.0915\\
31.3102	0.0915\\
31.3202	0.0915\\
31.3302	0.0945\\
31.3502	0.0870\\
31.3602	0.0821\\
31.3702	0.0907\\
31.3802	0.0907\\
31.3902	0.0904\\
31.4002	0.0901\\
31.4102	0.0877\\
31.4202	0.0808\\
31.4302	0.0746\\
31.4402	0.0737\\
31.4502	0.0699\\
31.4602	0.0695\\
31.4702	0.0710\\
31.4802	0.0736\\
31.4902	0.0726\\
31.5002	0.0720\\
31.5102	0.0674\\
31.5202	0.0556\\
31.5402	-0.0431\\
31.5502	-0.1071\\
31.5602	-0.1431\\
31.5702	-0.1293\\
31.5802	-0.0824\\
31.5902	-0.0450\\
31.6002	0.0028\\
31.6102	0.0526\\
31.6202	0.0922\\
31.6302	0.1304\\
31.6402	0.1473\\
31.6502	0.2011\\
31.6602	0.2713\\
31.6702	0.2893\\
31.6802	0.3117\\
31.6902	0.3308\\
31.7002	0.3199\\
31.7202	0.1585\\
31.7302	0.0626\\
31.7402	-0.0001\\
31.7502	-0.0239\\
31.7602	-0.0377\\
31.7702	-0.0449\\
31.7802	-0.0439\\
31.7902	-0.0426\\
31.8002	-0.0196\\
31.8102	0.0103\\
31.8202	0.0742\\
31.8302	0.1190\\
31.8402	0.1364\\
31.8502	0.1496\\
31.8602	0.1486\\
31.8802	0.1478\\
31.8902	0.1403\\
31.9002	0.1204\\
31.9102	0.0835\\
31.9202	0.0784\\
};
\addlegendentry{Derotated Estimate};

\end{axis}
\end{tikzpicture}%
	\caption{Baseline and derotated estimates of $\vartheta_x$, $\vartheta_y$ compared to ground truth measurements and body rates $p$, $q$, and $r$. Note that the sign of $p$ is inverted to match $\vartheta_x$.}
	\label{fig:derotation}
\end{figure}

The most relevant influences are those of $p$ on $\vartheta_x$ and $q$ on $\vartheta_y$, which are shown in the top and middle graphs respectively. The influence of $r$ is less profound. For conciseness, only the effect of $r$ on $\vartheta_y$ is shown, which provided the clearest result. Some residual motion in $p$ and $q$ is present in the latter case, though it does not fully account for the deviation seen in \cref{fig:derotation}. Note that the derotation process generally performs well; the largest part of the rotational flow is successfully corrected in the derotated estimate. 
%Clearly, some over-estimation is present for $\vartheta_x$ and $\vartheta_y$ during roll and pitch motion, since horizontal motion is limited. This can in fact be explained by the offset between the camera focal point and the drone's center of gravity. The rotational motion induces an additional horizontal motion component at the focal point due to the offset. However, to account for this, knowledge of the height above ground is necessary. While in our experiments this is available through OptiTrack measurements, in autonomous vision-based flights measurements of $Z_{\cal C}$ are typically missing.
