\section{Constant Divergence Landing Experiments}
\label{sec:results}
This section presents experimental results of constant divergence landings with the presented algorithms in the control loop. In \cref{sec:results_controller} the divergence control law is defined, after which the experimental setup is detailed in \cref{sec:experimental_setup}. Results from the experiments are presented and discussed in \cref{sec:results_div_landings}.

\subsection{Divergence Controller}
\label{sec:results_controller}
The control law regulates $\vartheta_z$ through the vertical thrust $T$. The controller applies a thrust difference $\Delta T$ with respect to a nominal hover thrust $T_0$, such that $T = T_0 + \Delta T$. A simple proportional control law is applied to $\Delta T$ based on $\vartheta_z$, similar to \citet{DeCroon2016}:

\begin{equation}
\Delta T = k_P \left(\vartheta_{z_{r}}-\vartheta_z\right)
\end{equation}

The nominal hover thrust $T_0$ counteracts the weight of the test vehicle. Its value is adapted in-flight in the height control loop of the test vehicle's autopilot software. Before the start of each landing, the vehicle first performs automatic hover to obtain a stable estimate for $T_0$. During the subsequent landing maneuver its value is kept constant.

\subsection{Experimental Setup}
\label{sec:experimental_setup}
The flying platform used in this work is a customized quadrotor referred to as the MavTec. Its main component is a Lisa/M board, which features a 72MHz 32bit ARM microprocessor as well as a pressure sensor and 3-axis rate gyros, accelerometers, and magnetometers. The Lisa/M runs the open-source autopilot software Paparazzi\footnote{Paparazzi UAV, \url{http://wiki.paparazziuav.org/}}, which handles the control of the drone\footnote{Code used in project is publicly available at: \url{https://github.com/tudelft/paparazzi/tree/event_based_flow}}. The DVS is mounted at the bottom of the MavTec facing downwards, aligned according to the reference frame definitions of $\cal C$ and $\cal B$ in \cref{sec:model}. Experiments are performed indoors, using an Optitrack motion tracking system to measure ground truth position and attitude.

In addition, an Odroid XU4 board is mounted on the quadrotor, which processes the event output of the DVS. It features a Samsung Exynos 5422 octacore CPU (four cores at 2.1 GHz and four at 1.5 GHz). The Odroid receives the events from the DVS through a USB 2.0 connection and processes these through the C-based open-source software cAER\footnote{Code used in project is publicly available at: \url{https://github.com/tudelft/caer/tree/odroid-dvs}} \cite{Longinotti2014}. 

%Experiments are performed indoors, using an OptiTrack motion tracking system. In previous work using similar systems, test objects were fitted with passive markers, which reflect infrared strobing light generated by the system. However, in \cite{Censi2013} the strobing was found to interfere with DVS measurements, since the camera is most sensitive in the infrared spectrum. This was confirmed during early experiments with passive markers: even strobing reflected by the ground was perceivable by the DVS. In later experiments, the drone was instead fitted with active infrared LEDs, allowing OptiTrack strobing to be disabled. As a favorable side-effect, this facilitated timestamp synchronization of OptiTrack and on-board measurements.

\begin{figure*}[!ht]
	\centering
	\begin{framed}
		\begin{minipage}[c]{0.3\textwidth}
			\centering
		\subfloat[Top view]{
			\includegraphics[width=\textwidth]{images/mavtec.png}
			\label{fig:mavtec_top}
		}\par\vfill
		\subfloat[Bottom view showing the DVS]{
			\includegraphics[width=\textwidth]{images/mavtec_bottom2.png}
			\label{fig:mavtec_bottom}
		}
		\end{minipage}\hfill
		\begin{minipage}[c]{0.67\textwidth}
			\subfloat[Overview of the implementation]{
				\includegraphics[width=\textwidth]{images/flowchart_implementation}
				\label{fig:implementation}
			}
		\end{minipage}
		\caption{Overview of the experimental setup, including pictures of the MavTec. In \protect\subref{fig:mavtec_top} a top view of the vehicle is shown. The DVS is located at the bottom, protected by a foam cover. In \protect\subref{fig:mavtec_bottom} the cover is removed to expose the DVS. In \protect\subref{fig:implementation} an overview of the processing workflow is shown, indicating the distribution of processes over the Odroid and the Lisa/M processors.}
		\label{fig:mavtec}
	\end{framed}
\end{figure*}

An overview of the experimental setup is shown in \cref{fig:mavtec}, including an overview of the on-board processing workflow in \cref{fig:implementation}. The estimation pipeline is subdivided in two stages. First, raw events are transmitted from the DVS to the Odroid through a USB interface. In cAER, optical flow is computed from the events using an implementation of our optical flow algorithm. Any event for which flow is estimated, is transmitted to the Lisa/M board through a serial UART interface. This process is completely event-based and is performed in a single thread. Separate threads handle event reception and transmission through the USB and UART interfaces.

Second, in Paparazzi, a periodic follow-up processing thread runs at 100 Hz. At each iteration, all newly received optical flow events are collected and corrected for the quadrotor's attitude and rotational motion. When all new events are processed, new estimates of the scaled velocities are computed with accompanying confidence values. A separate thread running at 512 Hz performs divergence control using the new update for $\vartheta_z$, as well as horizontal position control and stabilization.

%The source code for our versions of cAER and Paparazzi are publicly available online\footnote{cAER: \url{https://github.com/baspijhor/caer/tree/flow_adaptive_final}\\Paparazzi: \url{https://github.com/baspijhor/paparazzi/tree/event_based_flow}}.

\subsection{Results}
\label{sec:results_div_landings}
Constant divergence landing maneuvers were performed for several values of the setpoint $\vartheta_{z_r}$. During the tests, the target ground location was covered with the roadmap textured mat shown in \cref{fig:roadmap}. Currently, no mechanism is implemented to account for instability of constant divergence landings at low height, as described in \citet{DeCroon2016}. Therefore, when significant self-induced oscillations are observed, the landing maneuver is manually terminated. 

Resulting flight profiles (height, vertical speed, and divergence) are shown in \cref{fig:const_div_landing_1} for setpoints of $\vartheta_{z_r}=\lbrace0.5, 0.7,1.0\rbrace$\footnote{Video of the landings performed can be found at: \url{https://www.youtube.com/playlist?list=PL_KSX9GOn2P8RBdSyzngewi76G37PI3SF}}. Note that these values are much higher than the setpoints in comparable frame-based experiments \cite{Herisse2012,Ho2016}. The estimates for $\vartheta_z$ are shown in comparison to the ground truth estimate and the corresponding setpoint. For these maneuvers, the proportional gain $k_P$ is set to 0.2. This gain ensures that the descent remains stable during the first part. Decent tracking performance is seen for the lower two setpoints, while at $\vartheta_{z_r}=1.0$ some overshoot is observed. Still, a faster response may be obtained with an adaptive gain, such as in \citet{Ho2016a}. 

The expected instability is also clearly visible. For each setpoint, oscillations with diverging amplitude start to appear when the height is around 0.6 m above the ground, requiring the maneuver to be aborted manually at the moment. Also, a time delay is observed, whose magnitude differs between datasets. By examining the cross-correlation functions of the estimate and ground truth signals, average time delays of 0.05 s, 0.04 s, and 0.10 s are observed for the respective signals. A possible cause for this is the latency in the UART interface between the Odroid and the Lisa/M. Also, part of the delay results from the confidence filter that delays visual observable updates around zero-crossings.

%(So far,) tests were executed using a setpoint of $\vartheta_{z_{ref}}=0.2$ above a roadmap floor texture. A typical flight profile is shown in \cref{fig:const_div_landing_1} together with estimates for $\vartheta_z$ and ground truth, as well as the estimate confidence. At the indicated point, the divergence controller switches on and gradually brings the drone towards the ground. Due to the low gain, $\vartheta_z$ converges relatively slowly to the setpoint, but results in a smooth descent. As expected from \citet{Ho2016}, oscillations are visible when the height is close to zero. In our tests, this prevented a final touchdown, and manual control was resumed at the end of the maneuver. Some small jumps are visible in the vertical velocity, due to local underestimation of $\vartheta_z$. Interestingly, during the oscillations in the final part, $\vartheta_z$ is overestimated. (With the implementation for these experiments, this is possibly due to residual horizontal motion, which at the time provided less favorable coupling with divergence).

\begin{figure}[h]
	\centering
	\setlength{\fwidth}{0.4\linewidth}
	% This file was created by matlab2tikz.
%
%The latest updates can be retrieved from
%  http://www.mathworks.com/matlabcentral/fileexchange/22022-matlab2tikz-matlab2tikz
%where you can also make suggestions and rate matlab2tikz.
%
\definecolor{mycolor1}{rgb}{0.00000,0.44700,0.74100}%
\definecolor{mycolor2}{rgb}{0.85000,0.32500,0.09800}%
\definecolor{mycolor3}{rgb}{0.46600,0.67400,0.18800}%
%
\begin{tikzpicture}

\begin{axis}[%
width=0.951\fwidth,
height=0.303\fwidth,
at={(0\fwidth,0.84\fwidth)},
scale only axis,
xmin=-1.0000,
xmax=7.4186,
xlabel={$t$ [s]},
ymin=0.0000,
ymax=4.0000,
ylabel={$h$ [m]},
axis background/.style={fill=white},
legend style={legend cell align=left,align=left,draw=white!15!black},
title style={font=\labelsize},
xlabel style={font=\labelsize,at={(axis description cs:0.5,\xlabeldist)}},
ylabel style={font=\labelsize,at={(axis description cs:\ylabeldist,0.5)}},
legend style={font=\ticksize},
ticklabel style={font=\ticksize}
]
\addplot [color=black,dashed,forget plot]
  table[row sep=crcr]{%
0.0000	0.0000\\
0.0000	4.0000\\
};
\addplot [color=mycolor1,solid]
  table[row sep=crcr]{%
-1.0000	3.3010\\
-0.9894	3.3010\\
-0.9788	3.3010\\
-0.9682	3.3010\\
-0.9580	3.3010\\
-0.9476	3.3010\\
-0.9369	3.3010\\
-0.9265	3.3010\\
-0.9158	3.3010\\
-0.9049	3.3010\\
-0.8945	3.3010\\
-0.8842	3.3010\\
-0.8738	3.3010\\
-0.8633	3.3010\\
-0.8530	3.3010\\
-0.8425	3.3010\\
-0.8318	3.3010\\
-0.8216	3.3010\\
-0.8108	3.3010\\
-0.7998	3.3010\\
-0.7892	3.3010\\
-0.7788	3.3010\\
-0.7685	3.3010\\
-0.7580	3.3010\\
-0.7476	3.3010\\
-0.7372	3.3010\\
-0.7268	3.3010\\
-0.7165	3.3010\\
-0.7060	3.3010\\
-0.6954	3.3010\\
-0.6850	3.3000\\
-0.6746	3.3000\\
-0.6644	3.3000\\
-0.6535	3.3000\\
-0.6431	3.3000\\
-0.6321	3.3000\\
-0.6219	3.3000\\
-0.6109	3.3000\\
-0.6005	3.3000\\
-0.5895	3.3000\\
-0.5788	3.3000\\
-0.5681	3.3000\\
-0.5579	3.3000\\
-0.5476	3.3000\\
-0.5371	3.2990\\
-0.5267	3.2990\\
-0.5161	3.2990\\
-0.5050	3.2990\\
-0.4943	3.2990\\
-0.4839	3.2990\\
-0.4728	3.2990\\
-0.4623	3.2990\\
-0.4517	3.2980\\
-0.4412	3.2980\\
-0.4310	3.2980\\
-0.4205	3.2980\\
-0.4102	3.2980\\
-0.3992	3.2980\\
-0.3890	3.2980\\
-0.3787	3.2980\\
-0.3683	3.2970\\
-0.3580	3.2970\\
-0.3478	3.2970\\
-0.3371	3.2970\\
-0.3266	3.2970\\
-0.3163	3.2970\\
-0.3061	3.2970\\
-0.2955	3.2960\\
-0.2851	3.2960\\
-0.2745	3.2960\\
-0.2639	3.2960\\
-0.2528	3.2960\\
-0.2423	3.2950\\
-0.2318	3.2950\\
-0.2217	3.2950\\
-0.2112	3.2950\\
-0.2008	3.2950\\
-0.1906	3.2950\\
-0.1799	3.2940\\
-0.1694	3.2940\\
-0.1592	3.2940\\
-0.1490	3.2940\\
-0.1382	3.2940\\
-0.1277	3.2930\\
-0.1165	3.2930\\
-0.1058	3.2930\\
-0.0955	3.2930\\
-0.0851	3.2930\\
-0.0748	3.2920\\
-0.0643	3.2920\\
-0.0541	3.2920\\
-0.0433	3.2920\\
-0.0331	3.2920\\
-0.0228	3.2920\\
-0.0122	3.2910\\
-0.0018	3.2910\\
0.0085	3.2910\\
0.0188	3.2900\\
0.0295	3.2890\\
0.0399	3.2890\\
0.0501	3.2880\\
0.0606	3.2870\\
0.0714	3.2850\\
0.0816	3.2840\\
0.0920	3.2830\\
0.1022	3.2810\\
0.1131	3.2780\\
0.1235	3.2770\\
0.1345	3.2750\\
0.1449	3.2730\\
0.1558	3.2680\\
0.1668	3.2660\\
0.1772	3.2630\\
0.1880	3.2580\\
0.1991	3.2550\\
0.2096	3.2520\\
0.2203	3.2490\\
0.2314	3.2430\\
0.2420	3.2390\\
0.2522	3.2360\\
0.2631	3.2280\\
0.2741	3.2240\\
0.2845	3.2210\\
0.2950	3.2160\\
0.3054	3.2080\\
0.3160	3.2030\\
0.3264	3.1990\\
0.3365	3.1940\\
0.3474	3.1850\\
0.3578	3.1800\\
0.3680	3.1750\\
0.3784	3.1640\\
0.3892	3.1590\\
0.4002	3.1540\\
0.4106	3.1480\\
0.4212	3.1370\\
0.4314	3.1310\\
0.4418	3.1250\\
0.4522	3.1200\\
0.4629	3.1080\\
0.4731	3.1020\\
0.4834	3.0950\\
0.4939	3.0890\\
0.5046	3.0770\\
0.5148	3.0710\\
0.5252	3.0640\\
0.5354	3.0570\\
0.5461	3.0440\\
0.5568	3.0380\\
0.5677	3.0310\\
0.5783	3.0250\\
0.5890	3.0110\\
0.5994	3.0050\\
0.6100	2.9980\\
0.6201	2.9910\\
0.6304	2.9770\\
0.6410	2.9700\\
0.6512	2.9630\\
0.6617	2.9570\\
0.6721	2.9420\\
0.6825	2.9350\\
0.6930	2.9280\\
0.7033	2.9210\\
0.7138	2.9060\\
0.7244	2.8990\\
0.7345	2.8910\\
0.7449	2.8840\\
0.7555	2.8690\\
0.7659	2.8620\\
0.7763	2.8540\\
0.7867	2.8460\\
0.7972	2.8310\\
0.8080	2.8240\\
0.8189	2.8160\\
0.8298	2.8010\\
0.8408	2.7930\\
0.8510	2.7860\\
0.8611	2.7780\\
0.8719	2.7630\\
0.8829	2.7550\\
0.8939	2.7470\\
0.9043	2.7320\\
0.9149	2.7240\\
0.9254	2.7160\\
0.9356	2.7080\\
0.9461	2.6930\\
0.9565	2.6850\\
0.9668	2.6770\\
0.9773	2.6690\\
0.9881	2.6540\\
0.9991	2.6460\\
1.0095	2.6380\\
1.0200	2.6300\\
1.0306	2.6140\\
1.0416	2.6060\\
1.0523	2.5980\\
1.0627	2.5820\\
1.0731	2.5740\\
1.0834	2.5650\\
1.0938	2.5580\\
1.1045	2.5410\\
1.1149	2.5330\\
1.1252	2.5250\\
1.1354	2.5170\\
1.1463	2.5010\\
1.1574	2.4930\\
1.1679	2.4850\\
1.1783	2.4680\\
1.1891	2.4600\\
1.1992	2.4520\\
1.2103	2.4440\\
1.2211	2.4280\\
1.2314	2.4200\\
1.2418	2.4120\\
1.2521	2.4040\\
1.2626	2.3890\\
1.2734	2.3800\\
1.2844	2.3720\\
1.2949	2.3650\\
1.3055	2.3490\\
1.3160	2.3410\\
1.3264	2.3340\\
1.3365	2.3260\\
1.3474	2.3100\\
1.3576	2.3030\\
1.3682	2.2950\\
1.3782	2.2870\\
1.3888	2.2710\\
1.3991	2.2640\\
1.4096	2.2560\\
1.4199	2.2480\\
1.4305	2.2330\\
1.4412	2.2250\\
1.4522	2.2170\\
1.4628	2.2010\\
1.4730	2.1930\\
1.4835	2.1850\\
1.4942	2.1780\\
1.5046	2.1630\\
1.5149	2.1550\\
1.5253	2.1480\\
1.5354	2.1400\\
1.5460	2.1250\\
1.5565	2.1170\\
1.5669	2.1100\\
1.5771	2.1020\\
1.5881	2.0880\\
1.5991	2.0800\\
1.6094	2.0730\\
1.6202	2.0660\\
1.6305	2.0510\\
1.6415	2.0440\\
1.6522	2.0370\\
1.6630	2.0220\\
1.6736	2.0150\\
1.6843	2.0080\\
1.6949	2.0010\\
1.7054	1.9870\\
1.7157	1.9800\\
1.7262	1.9730\\
1.7366	1.9660\\
1.7475	1.9520\\
1.7587	1.9450\\
1.7698	1.9380\\
1.7806	1.9240\\
1.7911	1.9180\\
1.8013	1.9110\\
1.8114	1.9040\\
1.8222	1.8900\\
1.8324	1.8830\\
1.8430	1.8770\\
1.8531	1.8700\\
1.8642	1.8560\\
1.8752	1.8490\\
1.8857	1.8430\\
1.8959	1.8290\\
1.9065	1.8230\\
1.9171	1.8160\\
1.9280	1.8090\\
1.9389	1.7960\\
1.9491	1.7890\\
1.9595	1.7830\\
1.9700	1.7760\\
1.9805	1.7630\\
1.9912	1.7560\\
2.0015	1.7490\\
2.0118	1.7430\\
2.0223	1.7290\\
2.0333	1.7230\\
2.0440	1.7160\\
2.0543	1.7030\\
2.0650	1.6970\\
2.0753	1.6900\\
2.0857	1.6840\\
2.0962	1.6710\\
2.1067	1.6640\\
2.1173	1.6580\\
2.1282	1.6520\\
2.1391	1.6390\\
2.1495	1.6330\\
2.1605	1.6260\\
2.1713	1.6140\\
2.1814	1.6070\\
2.1918	1.6010\\
2.2023	1.5950\\
2.2130	1.5820\\
2.2233	1.5760\\
2.2335	1.5700\\
2.2442	1.5630\\
2.2544	1.5510\\
2.2648	1.5440\\
2.2754	1.5380\\
2.2858	1.5320\\
2.2961	1.5200\\
2.3066	1.5140\\
2.3169	1.5080\\
2.3271	1.5020\\
2.3379	1.4900\\
2.3480	1.4850\\
2.3585	1.4790\\
2.3694	1.4730\\
2.3805	1.4610\\
2.3910	1.4560\\
2.4020	1.4500\\
2.4131	1.4380\\
2.4232	1.4330\\
2.4335	1.4270\\
2.4436	1.4220\\
2.4544	1.4110\\
2.4653	1.4050\\
2.4761	1.4000\\
2.4865	1.3950\\
2.4973	1.3840\\
2.5074	1.3790\\
2.5176	1.3740\\
2.5281	1.3680\\
2.5390	1.3580\\
2.5501	1.3530\\
2.5606	1.3480\\
2.5711	1.3370\\
2.5815	1.3330\\
2.5919	1.3270\\
2.6029	1.3220\\
2.6136	1.3120\\
2.6241	1.3070\\
2.6347	1.3020\\
2.6450	1.2960\\
2.6555	1.2860\\
2.6657	1.2810\\
2.6763	1.2760\\
2.6869	1.2710\\
2.6972	1.2610\\
2.7083	1.2560\\
2.7189	1.2510\\
2.7294	1.2410\\
2.7399	1.2360\\
2.7502	1.2300\\
2.7605	1.2260\\
2.7711	1.2150\\
2.7814	1.2100\\
2.7917	1.2050\\
2.8025	1.2000\\
2.8128	1.1900\\
2.8230	1.1850\\
2.8336	1.1800\\
2.8447	1.1750\\
2.8554	1.1650\\
2.8658	1.1600\\
2.8766	1.1560\\
2.8877	1.1460\\
2.8984	1.1410\\
2.9085	1.1360\\
2.9186	1.1310\\
2.9294	1.1220\\
2.9400	1.1170\\
2.9505	1.1120\\
2.9614	1.1070\\
2.9723	1.0980\\
2.9834	1.0930\\
2.9939	1.0880\\
3.0046	1.0790\\
3.0158	1.0740\\
3.0262	1.0690\\
3.0365	1.0650\\
3.0471	1.0560\\
3.0577	1.0510\\
3.0688	1.0470\\
3.0795	1.0380\\
3.0897	1.0330\\
3.1001	1.0290\\
3.1106	1.0250\\
3.1212	1.0160\\
3.1314	1.0110\\
3.1416	1.0070\\
3.1521	1.0030\\
3.1629	0.9950\\
3.1732	0.9910\\
3.1844	0.9860\\
3.1947	0.9820\\
3.2057	0.9740\\
3.2159	0.9700\\
3.2262	0.9660\\
3.2364	0.9620\\
3.2472	0.9540\\
3.2576	0.9500\\
3.2678	0.9460\\
3.2781	0.9420\\
3.2890	0.9350\\
3.2999	0.9310\\
3.3105	0.9280\\
3.3209	0.9200\\
3.3319	0.9160\\
3.3427	0.9130\\
3.3530	0.9090\\
3.3637	0.9020\\
3.3741	0.8990\\
3.3848	0.8950\\
3.3949	0.8910\\
3.4055	0.8840\\
3.4166	0.8810\\
3.4272	0.8770\\
3.4378	0.8700\\
3.4480	0.8670\\
3.4585	0.8630\\
3.4688	0.8600\\
3.4796	0.8530\\
3.4899	0.8500\\
3.5002	0.8460\\
3.5103	0.8430\\
3.5211	0.8370\\
3.5314	0.8330\\
3.5419	0.8300\\
3.5521	0.8270\\
3.5622	0.8210\\
3.5732	0.8180\\
3.5843	0.8150\\
3.5950	0.8110\\
3.6053	0.8050\\
3.6158	0.8020\\
3.6260	0.7990\\
3.6366	0.7960\\
3.6471	0.7900\\
3.6574	0.7870\\
3.6677	0.7850\\
3.6779	0.7820\\
3.6889	0.7760\\
3.6992	0.7730\\
3.7096	0.7710\\
3.7199	0.7680\\
3.7306	0.7620\\
3.7409	0.7600\\
3.7511	0.7570\\
3.7616	0.7540\\
3.7723	0.7490\\
3.7825	0.7460\\
3.7929	0.7430\\
3.8031	0.7400\\
3.8137	0.7350\\
3.8244	0.7320\\
3.8353	0.7290\\
3.8461	0.7240\\
3.8564	0.7210\\
3.8669	0.7180\\
3.8777	0.7160\\
3.8878	0.7100\\
3.8983	0.7070\\
3.9088	0.7040\\
3.9197	0.7010\\
3.9302	0.6960\\
3.9407	0.6930\\
3.9511	0.6900\\
3.9619	0.6840\\
3.9723	0.6810\\
3.9825	0.6780\\
3.9927	0.6760\\
4.0035	0.6730\\
4.0137	0.6670\\
4.0242	0.6650\\
4.0344	0.6620\\
4.0444	0.6590\\
4.0553	0.6540\\
4.0658	0.6510\\
4.0763	0.6490\\
4.0865	0.6460\\
4.0970	0.6410\\
4.1075	0.6380\\
4.1177	0.6350\\
4.1282	0.6320\\
4.1389	0.6270\\
4.1491	0.6240\\
4.1593	0.6220\\
4.1699	0.6190\\
4.1806	0.6140\\
4.1917	0.6110\\
4.2022	0.6080\\
4.2128	0.6040\\
4.2234	0.6010\\
4.2344	0.5990\\
4.2450	0.5960\\
4.2559	0.5920\\
4.2669	0.5890\\
4.2772	0.5870\\
4.2877	0.5820\\
4.2985	0.5800\\
4.3096	0.5780\\
4.3198	0.5760\\
4.3303	0.5720\\
4.3408	0.5690\\
4.3513	0.5670\\
4.3618	0.5630\\
4.3721	0.5610\\
4.3827	0.5590\\
4.3929	0.5570\\
4.4033	0.5540\\
4.4136	0.5500\\
4.4240	0.5480\\
4.4346	0.5460\\
4.4451	0.5440\\
4.4555	0.5390\\
4.4658	0.5370\\
4.4761	0.5350\\
4.4865	0.5320\\
4.4973	0.5270\\
4.5076	0.5250\\
4.5178	0.5230\\
4.5281	0.5200\\
4.5387	0.5150\\
4.5491	0.5130\\
4.5594	0.5110\\
4.5699	0.5080\\
4.5803	0.5040\\
4.5907	0.5010\\
4.6012	0.4990\\
4.6118	0.4970\\
4.6221	0.4920\\
4.6329	0.4900\\
4.6439	0.4880\\
4.6546	0.4840\\
4.6648	0.4820\\
4.6750	0.4800\\
4.6853	0.4780\\
4.6963	0.4740\\
4.7073	0.4730\\
4.7178	0.4700\\
4.7282	0.4690\\
4.7387	0.4650\\
4.7493	0.4640\\
4.7600	0.4620\\
4.7711	0.4590\\
4.7815	0.4570\\
4.7920	0.4550\\
4.8021	0.4540\\
4.8125	0.4500\\
4.8233	0.4490\\
4.8335	0.4470\\
4.8436	0.4450\\
4.8546	0.4420\\
4.8650	0.4400\\
4.8751	0.4380\\
4.8858	0.4370\\
4.8961	0.4340\\
4.9066	0.4320\\
4.9169	0.4300\\
4.9272	0.4280\\
4.9378	0.4250\\
4.9481	0.4240\\
4.9583	0.4220\\
4.9690	0.4200\\
4.9795	0.4180\\
4.9898	0.4160\\
5.0001	0.4150\\
5.0109	0.4130\\
5.0212	0.4100\\
5.0315	0.4090\\
5.0428	0.4080\\
5.0529	0.4070\\
5.0638	0.4040\\
5.0741	0.4030\\
5.0845	0.4020\\
5.0949	0.4010\\
5.1055	0.3990\\
5.1158	0.3970\\
5.1263	0.3960\\
5.1367	0.3950\\
5.1471	0.3930\\
5.1574	0.3920\\
5.1676	0.3900\\
5.1781	0.3890\\
5.1888	0.3870\\
5.1992	0.3860\\
5.2094	0.3840\\
5.2201	0.3830\\
5.2307	0.3810\\
5.2418	0.3800\\
5.2520	0.3780\\
5.2628	0.3760\\
5.2733	0.3750\\
5.2836	0.3730\\
5.2946	0.3720\\
5.3053	0.3690\\
5.3159	0.3680\\
5.3261	0.3670\\
5.3363	0.3650\\
5.3474	0.3630\\
5.3584	0.3620\\
5.3687	0.3600\\
5.3796	0.3580\\
5.3905	0.3560\\
5.4012	0.3550\\
5.4116	0.3540\\
5.4221	0.3520\\
5.4323	0.3500\\
5.4428	0.3490\\
5.4532	0.3480\\
5.4640	0.3460\\
5.4743	0.3450\\
5.4846	0.3440\\
5.4952	0.3430\\
5.5055	0.3410\\
5.5159	0.3400\\
5.5269	0.3390\\
5.5377	0.3370\\
5.5488	0.3360\\
5.5596	0.3350\\
5.5698	0.3340\\
5.5805	0.3330\\
5.5906	0.3330\\
5.6013	0.3320\\
5.6116	0.3310\\
5.6220	0.3300\\
5.6325	0.3290\\
5.6428	0.3280\\
5.6531	0.3270\\
5.6636	0.3250\\
5.6743	0.3240\\
5.6850	0.3230\\
5.6960	0.3210\\
5.7063	0.3200\\
5.7167	0.3190\\
5.7272	0.3180\\
5.7379	0.3150\\
5.7481	0.3140\\
5.7584	0.3130\\
5.7689	0.3110\\
5.7794	0.3080\\
5.7900	0.3070\\
5.8009	0.3060\\
5.8112	0.3040\\
5.8220	0.3010\\
5.8322	0.3000\\
5.8428	0.2990\\
5.8531	0.2980\\
5.8638	0.2950\\
5.8741	0.2940\\
5.8843	0.2930\\
5.8947	0.2920\\
5.9053	0.2910\\
5.9157	0.2900\\
5.9261	0.2890\\
5.9366	0.2880\\
5.9470	0.2870\\
5.9572	0.2860\\
5.9680	0.2860\\
5.9781	0.2850\\
5.9888	0.2840\\
5.9990	0.2830\\
6.0093	0.2830\\
6.0199	0.2820\\
6.0302	0.2810\\
6.0409	0.2800\\
6.0512	0.2790\\
6.0616	0.2780\\
6.0720	0.2770\\
6.0822	0.2760\\
6.0928	0.2750\\
6.1032	0.2750\\
6.1136	0.2730\\
6.1238	0.2720\\
6.1343	0.2710\\
6.1447	0.2700\\
6.1555	0.2680\\
6.1658	0.2670\\
6.1760	0.2660\\
6.1865	0.2660\\
6.1971	0.2640\\
6.2074	0.2630\\
6.2176	0.2620\\
6.2282	0.2610\\
6.2389	0.2600\\
6.2490	0.2590\\
6.2595	0.2580\\
6.2698	0.2570\\
6.2808	0.2560\\
6.2917	0.2550\\
6.3021	0.2550\\
6.3127	0.2540\\
6.3231	0.2530\\
6.3342	0.2520\\
6.3449	0.2520\\
6.3554	0.2510\\
6.3657	0.2510\\
6.3760	0.2500\\
6.3866	0.2500\\
6.3972	0.2490\\
6.4089	0.2480\\
6.4198	0.2470\\
6.4303	0.2460\\
6.4406	0.2450\\
6.4514	0.2440\\
6.4617	0.2420\\
6.4723	0.2420\\
6.4832	0.2410\\
6.4937	0.2390\\
6.5044	0.2370\\
6.5150	0.2360\\
6.5258	0.2350\\
6.5364	0.2340\\
6.5470	0.2330\\
6.5573	0.2320\\
6.5677	0.2310\\
6.5782	0.2300\\
6.5887	0.2290\\
6.5990	0.2290\\
6.6096	0.2280\\
6.6197	0.2280\\
6.6304	0.2280\\
6.6410	0.2270\\
6.6521	0.2270\\
6.6629	0.2270\\
6.6740	0.2270\\
6.6846	0.2270\\
6.6952	0.2270\\
6.7063	0.2260\\
6.7168	0.2260\\
6.7271	0.2260\\
6.7377	0.2260\\
6.7481	0.2250\\
6.7583	0.2250\\
6.7692	0.2240\\
6.7793	0.2240\\
6.7897	0.2230\\
6.8000	0.2220\\
6.8102	0.2210\\
6.8211	0.2200\\
6.8313	0.2190\\
6.8419	0.2180\\
6.8520	0.2170\\
6.8628	0.2140\\
6.8733	0.2130\\
6.8843	0.2120\\
6.8948	0.2110\\
6.9053	0.2090\\
6.9157	0.2080\\
6.9261	0.2070\\
6.9367	0.2060\\
6.9470	0.2050\\
6.9573	0.2050\\
6.9678	0.2040\\
6.9781	0.2050\\
6.9891	0.2050\\
7.0001	0.2050\\
7.0105	0.2060\\
7.0213	0.2070\\
7.0315	0.2080\\
7.0418	0.2080\\
7.0520	0.2090\\
7.0628	0.2110\\
7.0730	0.2110\\
7.0833	0.2120\\
7.0938	0.2130\\
7.1042	0.2140\\
7.1146	0.2140\\
7.1250	0.2140\\
7.1353	0.2140\\
7.1463	0.2140\\
7.1571	0.2130\\
7.1677	0.2120\\
7.1785	0.2100\\
7.1887	0.2080\\
7.1990	0.2060\\
7.2093	0.2050\\
7.2198	0.2030\\
7.2303	0.1980\\
7.2408	0.1950\\
7.2514	0.1920\\
7.2616	0.1900\\
7.2719	0.1850\\
7.2824	0.1820\\
7.2928	0.1790\\
7.3030	0.1770\\
7.3135	0.1730\\
7.3240	0.1720\\
7.3345	0.1700\\
7.3447	0.1690\\
7.3553	0.1690\\
7.3659	0.1690\\
7.3762	0.1690\\
7.3865	0.1700\\
7.3968	0.1730\\
7.4076	0.1750\\
7.4186	0.1770\\
};
\addlegendentry{$\vartheta_{z_r}$ = 0.5};

\addplot [color=mycolor2,solid]
  table[row sep=crcr]{%
-1.0000	3.4270\\
-0.9894	3.4270\\
-0.9791	3.4280\\
-0.9684	3.4280\\
-0.9582	3.4290\\
-0.9473	3.4290\\
-0.9363	3.4290\\
-0.9258	3.4290\\
-0.9153	3.4300\\
-0.9044	3.4300\\
-0.8935	3.4300\\
-0.8830	3.4300\\
-0.8726	3.4310\\
-0.8623	3.4310\\
-0.8519	3.4310\\
-0.8415	3.4310\\
-0.8305	3.4320\\
-0.8196	3.4320\\
-0.8093	3.4320\\
-0.7988	3.4320\\
-0.7886	3.4330\\
-0.7779	3.4330\\
-0.7675	3.4330\\
-0.7570	3.4330\\
-0.7466	3.4340\\
-0.7362	3.4340\\
-0.7260	3.4340\\
-0.7153	3.4340\\
-0.7051	3.4340\\
-0.6945	3.4350\\
-0.6842	3.4350\\
-0.6737	3.4350\\
-0.6633	3.4350\\
-0.6530	3.4360\\
-0.6426	3.4360\\
-0.6321	3.4360\\
-0.6219	3.4360\\
-0.6111	3.4360\\
-0.6008	3.4360\\
-0.5905	3.4370\\
-0.5800	3.4370\\
-0.5697	3.4370\\
-0.5591	3.4370\\
-0.5490	3.4370\\
-0.5384	3.4380\\
-0.5277	3.4380\\
-0.5173	3.4380\\
-0.5062	3.4380\\
-0.4959	3.4380\\
-0.4851	3.4390\\
-0.4747	3.4390\\
-0.4635	3.4390\\
-0.4529	3.4390\\
-0.4422	3.4390\\
-0.4319	3.4390\\
-0.4217	3.4400\\
-0.4111	3.4400\\
-0.4007	3.4400\\
-0.3905	3.4400\\
-0.3800	3.4400\\
-0.3696	3.4410\\
-0.3590	3.4410\\
-0.3479	3.4410\\
-0.3375	3.4410\\
-0.3267	3.4410\\
-0.3156	3.4410\\
-0.3051	3.4420\\
-0.2945	3.4420\\
-0.2840	3.4420\\
-0.2728	3.4420\\
-0.2625	3.4430\\
-0.2518	3.4430\\
-0.2415	3.4430\\
-0.2313	3.4430\\
-0.2207	3.4430\\
-0.2101	3.4440\\
-0.1993	3.4440\\
-0.1882	3.4440\\
-0.1777	3.4440\\
-0.1675	3.4450\\
-0.1570	3.4450\\
-0.1468	3.4450\\
-0.1362	3.4450\\
-0.1257	3.4450\\
-0.1155	3.4450\\
-0.1051	3.4460\\
-0.0943	3.4460\\
-0.0840	3.4460\\
-0.0738	3.4460\\
-0.0632	3.4470\\
-0.0520	3.4470\\
-0.0412	3.4470\\
-0.0310	3.4470\\
-0.0206	3.4470\\
-0.0103	3.4480\\
0.0000	3.4480\\
0.0106	3.4480\\
0.0212	3.4470\\
0.0323	3.4470\\
0.0428	3.4460\\
0.0531	3.4450\\
0.0642	3.4430\\
0.0751	3.4420\\
0.0855	3.4400\\
0.0957	3.4380\\
0.1068	3.4350\\
0.1170	3.4320\\
0.1274	3.4300\\
0.1375	3.4270\\
0.1482	3.4210\\
0.1585	3.4180\\
0.1691	3.4140\\
0.1794	3.4110\\
0.1899	3.4030\\
0.2003	3.3980\\
0.2115	3.3940\\
0.2222	3.3840\\
0.2325	3.3790\\
0.2430	3.3740\\
0.2532	3.3690\\
0.2637	3.3570\\
0.2741	3.3510\\
0.2847	3.3450\\
0.2948	3.3390\\
0.3055	3.3250\\
0.3159	3.3180\\
0.3261	3.3110\\
0.3365	3.3040\\
0.3471	3.2900\\
0.3575	3.2820\\
0.3686	3.2740\\
0.3791	3.2660\\
0.3900	3.2500\\
0.4002	3.2410\\
0.4106	3.2330\\
0.4208	3.2240\\
0.4314	3.2070\\
0.4421	3.1970\\
0.4531	3.1880\\
0.4638	3.1700\\
0.4750	3.1600\\
0.4855	3.1510\\
0.4958	3.1410\\
0.5064	3.1220\\
0.5167	3.1120\\
0.5272	3.1020\\
0.5374	3.0920\\
0.5481	3.0720\\
0.5584	3.0610\\
0.5690	3.0510\\
0.5792	3.0410\\
0.5896	3.0190\\
0.6001	3.0090\\
0.6108	2.9980\\
0.6210	2.9870\\
0.6314	2.9660\\
0.6418	2.9550\\
0.6520	2.9440\\
0.6626	2.9220\\
0.6731	2.9100\\
0.6833	2.8990\\
0.6938	2.8880\\
0.7044	2.8770\\
0.7150	2.8550\\
0.7260	2.8430\\
0.7364	2.8320\\
0.7471	2.8090\\
0.7576	2.7970\\
0.7687	2.7860\\
0.7791	2.7750\\
0.7897	2.7520\\
0.8002	2.7400\\
0.8106	2.7280\\
0.8209	2.7160\\
0.8317	2.6930\\
0.8428	2.6810\\
0.8539	2.6700\\
0.8647	2.6460\\
0.8753	2.6340\\
0.8863	2.6230\\
0.8969	2.5990\\
0.9074	2.5870\\
0.9177	2.5750\\
0.9280	2.5630\\
0.9388	2.5400\\
0.9491	2.5270\\
0.9592	2.5150\\
0.9697	2.5030\\
0.9803	2.4790\\
0.9906	2.4670\\
1.0011	2.4550\\
1.0116	2.4430\\
1.0225	2.4190\\
1.0333	2.4060\\
1.0441	2.3950\\
1.0543	2.3830\\
1.0648	2.3590\\
1.0755	2.3470\\
1.0864	2.3350\\
1.0972	2.3120\\
1.1080	2.3000\\
1.1189	2.2880\\
1.1290	2.2770\\
1.1397	2.2530\\
1.1500	2.2410\\
1.1603	2.2300\\
1.1707	2.2180\\
1.1816	2.1950\\
1.1919	2.1830\\
1.2021	2.1720\\
1.2125	2.1600\\
1.2232	2.1370\\
1.2336	2.1250\\
1.2438	2.1140\\
1.2540	2.1030\\
1.2647	2.0800\\
1.2750	2.0680\\
1.2856	2.0570\\
1.2964	2.0460\\
1.3075	2.0230\\
1.3182	2.0120\\
1.3293	2.0010\\
1.3397	1.9780\\
1.3502	1.9670\\
1.3607	1.9560\\
1.3711	1.9450\\
1.3814	1.9230\\
1.3918	1.9120\\
1.4021	1.9020\\
1.4127	1.8910\\
1.4286	1.8700\\
1.4389	1.8480\\
1.4492	1.8380\\
1.4596	1.8270\\
1.4706	1.8170\\
1.4814	1.7960\\
1.4917	1.7860\\
1.5026	1.7760\\
1.5126	1.7660\\
1.5233	1.7460\\
1.5335	1.7360\\
1.5439	1.7260\\
1.5542	1.7160\\
1.5645	1.6970\\
1.5753	1.6870\\
1.5855	1.6770\\
1.5957	1.6680\\
1.6068	1.6490\\
1.6177	1.6390\\
1.6284	1.6300\\
1.6388	1.6110\\
1.6491	1.6010\\
1.6593	1.5920\\
1.6700	1.5830\\
1.6807	1.5640\\
1.6912	1.5550\\
1.7022	1.5460\\
1.7125	1.5370\\
1.7231	1.5190\\
1.7342	1.5100\\
1.7447	1.5020\\
1.7554	1.4840\\
1.7661	1.4760\\
1.7763	1.4670\\
1.7874	1.4580\\
1.7981	1.4420\\
1.8085	1.4330\\
1.8194	1.4250\\
1.8294	1.4090\\
1.8398	1.4000\\
1.8499	1.3920\\
1.8605	1.3840\\
1.8709	1.3760\\
1.8815	1.3600\\
1.8916	1.3530\\
1.9019	1.3450\\
1.9125	1.3370\\
1.9231	1.3220\\
1.9335	1.3140\\
1.9442	1.3060\\
1.9543	1.2980\\
1.9648	1.2830\\
1.9751	1.2760\\
1.9855	1.2680\\
1.9958	1.2610\\
2.0064	1.2460\\
2.0166	1.2390\\
2.0274	1.2320\\
2.0379	1.2240\\
2.0482	1.2100\\
2.0584	1.2030\\
2.0688	1.1960\\
2.0791	1.1890\\
2.0896	1.1750\\
2.1003	1.1680\\
2.1105	1.1610\\
2.1207	1.1540\\
2.1313	1.1410\\
2.1418	1.1340\\
2.1519	1.1270\\
2.1624	1.1210\\
2.1731	1.1070\\
2.1836	1.1010\\
2.1943	1.0940\\
2.2043	1.0880\\
2.2148	1.0750\\
2.2251	1.0690\\
2.2356	1.0620\\
2.2458	1.0560\\
2.2563	1.0440\\
2.2668	1.0370\\
2.2771	1.0310\\
2.2876	1.0250\\
2.2983	1.0130\\
2.3086	1.0070\\
2.3194	1.0010\\
2.3304	0.9890\\
2.3410	0.9830\\
2.3519	0.9770\\
2.3623	0.9720\\
2.3730	0.9600\\
2.3835	0.9540\\
2.3939	0.9490\\
2.4043	0.9430\\
2.4145	0.9320\\
2.4253	0.9270\\
2.4365	0.9210\\
2.4474	0.9110\\
2.4584	0.9050\\
2.4688	0.9000\\
2.4790	0.8950\\
2.4900	0.8850\\
2.5008	0.8800\\
2.5116	0.8750\\
2.5219	0.8660\\
2.5325	0.8600\\
2.5429	0.8560\\
2.5539	0.8510\\
2.5648	0.8420\\
2.5753	0.8370\\
2.5856	0.8320\\
2.5959	0.8280\\
2.6064	0.8190\\
2.6168	0.8140\\
2.6273	0.8100\\
2.6374	0.8060\\
2.6479	0.7970\\
2.6584	0.7930\\
2.6687	0.7890\\
2.6790	0.7840\\
2.6899	0.7760\\
2.7002	0.7720\\
2.7105	0.7680\\
2.7208	0.7630\\
2.7316	0.7550\\
2.7422	0.7510\\
2.7530	0.7470\\
2.7637	0.7390\\
2.7742	0.7350\\
2.7847	0.7310\\
2.7949	0.7270\\
2.8054	0.7200\\
2.8158	0.7160\\
2.8260	0.7120\\
2.8367	0.7080\\
2.8472	0.7000\\
2.8583	0.6970\\
2.8691	0.6930\\
2.8793	0.6890\\
2.8900	0.6810\\
2.9008	0.6780\\
2.9118	0.6740\\
2.9220	0.6660\\
2.9327	0.6620\\
2.9437	0.6590\\
2.9542	0.6550\\
2.9647	0.6480\\
2.9753	0.6450\\
2.9857	0.6410\\
2.9960	0.6380\\
3.0065	0.6310\\
3.0167	0.6270\\
3.0272	0.6240\\
3.0378	0.6210\\
3.0482	0.6140\\
3.0587	0.6110\\
3.0698	0.6070\\
3.0805	0.6010\\
3.0911	0.5980\\
3.1023	0.5940\\
3.1128	0.5910\\
3.1233	0.5840\\
3.1336	0.5810\\
3.1440	0.5780\\
3.1541	0.5740\\
3.1649	0.5680\\
3.1753	0.5640\\
3.1864	0.5610\\
3.1965	0.5540\\
3.2067	0.5510\\
3.2171	0.5480\\
3.2273	0.5440\\
3.2377	0.5410\\
3.2481	0.5340\\
3.2588	0.5310\\
3.2694	0.5280\\
3.2804	0.5220\\
3.2909	0.5190\\
3.3015	0.5160\\
3.3124	0.5130\\
3.3224	0.5070\\
3.3334	0.5040\\
3.3438	0.5010\\
3.3542	0.4980\\
3.3647	0.4930\\
3.3752	0.4900\\
3.3854	0.4870\\
3.3959	0.4820\\
3.4067	0.4790\\
3.4169	0.4770\\
3.4272	0.4740\\
3.4378	0.4720\\
3.4483	0.4670\\
3.4585	0.4640\\
3.4689	0.4620\\
3.4793	0.4590\\
3.4899	0.4540\\
3.5001	0.4520\\
3.5106	0.4500\\
3.5209	0.4470\\
3.5317	0.4430\\
3.5428	0.4400\\
3.5531	0.4380\\
3.5639	0.4340\\
3.5741	0.4310\\
3.5845	0.4290\\
3.5948	0.4270\\
3.6055	0.4220\\
3.6160	0.4200\\
3.6262	0.4180\\
3.6366	0.4150\\
3.6473	0.4100\\
3.6584	0.4080\\
3.6689	0.4050\\
3.6793	0.4030\\
3.6899	0.3980\\
3.7003	0.3950\\
3.7106	0.3930\\
3.7208	0.3900\\
3.7316	0.3850\\
3.7419	0.3830\\
3.7524	0.3800\\
3.7626	0.3750\\
3.7735	0.3730\\
3.7836	0.3710\\
3.7940	0.3680\\
3.8043	0.3660\\
3.8148	0.3620\\
3.8254	0.3600\\
3.8358	0.3580\\
3.8469	0.3540\\
3.8577	0.3520\\
3.8687	0.3500\\
3.8794	0.3490\\
3.8899	0.3450\\
3.9003	0.3440\\
3.9104	0.3430\\
3.9208	0.3410\\
3.9316	0.3380\\
3.9419	0.3360\\
3.9529	0.3350\\
3.9638	0.3320\\
3.9743	0.3310\\
3.9853	0.3290\\
3.9959	0.3280\\
4.0066	0.3250\\
4.0175	0.3240\\
4.0283	0.3220\\
4.0390	0.3190\\
4.0492	0.3170\\
4.0596	0.3160\\
4.0700	0.3140\\
4.0806	0.3110\\
4.0909	0.3090\\
4.1013	0.3070\\
4.1117	0.3050\\
4.1220	0.3010\\
4.1325	0.3000\\
4.1430	0.2980\\
4.1538	0.2960\\
4.1647	0.2930\\
4.1753	0.2920\\
4.1857	0.2900\\
4.1959	0.2890\\
4.2068	0.2860\\
4.2177	0.2850\\
4.2282	0.2830\\
4.2391	0.2810\\
4.2501	0.2800\\
4.2604	0.2780\\
4.2708	0.2770\\
4.2818	0.2750\\
4.2930	0.2740\\
4.3040	0.2720\\
4.3150	0.2700\\
4.3261	0.2690\\
4.3367	0.2680\\
4.3471	0.2660\\
4.3576	0.2640\\
4.3679	0.2630\\
4.3782	0.2620\\
4.3889	0.2600\\
4.4000	0.2590\\
4.4105	0.2580\\
4.4209	0.2570\\
4.4315	0.2540\\
4.4417	0.2530\\
4.4520	0.2520\\
4.4627	0.2510\\
4.4734	0.2490\\
4.4844	0.2480\\
4.4947	0.2470\\
4.5056	0.2450\\
4.5166	0.2440\\
4.5273	0.2430\\
4.5375	0.2420\\
4.5482	0.2400\\
4.5584	0.2390\\
4.5688	0.2380\\
4.5791	0.2370\\
4.5897	0.2340\\
4.6001	0.2340\\
4.6105	0.2330\\
4.6207	0.2310\\
4.6315	0.2300\\
4.6419	0.2290\\
4.6522	0.2280\\
4.6625	0.2270\\
4.6734	0.2250\\
4.6844	0.2250\\
4.6951	0.2240\\
4.7054	0.2220\\
4.7158	0.2210\\
4.7262	0.2200\\
4.7365	0.2190\\
4.7473	0.2170\\
4.7574	0.2160\\
4.7678	0.2150\\
4.7782	0.2140\\
4.7890	0.2120\\
4.7993	0.2110\\
4.8095	0.2100\\
4.8200	0.2090\\
4.8304	0.2060\\
4.8407	0.2050\\
4.8515	0.2040\\
4.8620	0.2030\\
4.8722	0.2010\\
4.8834	0.2000\\
4.8940	0.1990\\
4.9042	0.1980\\
4.9147	0.1970\\
4.9253	0.1960\\
4.9355	0.1950\\
4.9458	0.1950\\
4.9558	0.1930\\
4.9666	0.1930\\
4.9771	0.1920\\
4.9875	0.1920\\
4.9980	0.1900\\
5.0084	0.1900\\
5.0190	0.1890\\
5.0294	0.1880\\
5.0399	0.1880\\
5.0510	0.1870\\
5.0615	0.1860\\
5.0722	0.1850\\
5.0829	0.1840\\
5.0935	0.1830\\
5.1041	0.1830\\
5.1149	0.1810\\
5.1261	0.1800\\
5.1364	0.1790\\
5.1471	0.1770\\
5.1577	0.1770\\
5.1688	0.1760\\
5.1791	0.1750\\
5.1899	0.1730\\
5.2003	0.1720\\
5.2111	0.1710\\
5.2221	0.1690\\
5.2325	0.1680\\
5.2430	0.1670\\
5.2531	0.1660\\
5.2636	0.1650\\
5.2744	0.1640\\
5.2855	0.1640\\
5.2959	0.1630\\
5.3064	0.1630\\
5.3169	0.1620\\
5.3273	0.1620\\
5.3378	0.1620\\
5.3482	0.1620\\
5.3583	0.1620\\
5.3688	0.1620\\
5.3792	0.1620\\
5.3901	0.1630\\
5.4013	0.1630\\
5.4125	0.1630\\
5.4231	0.1630\\
5.4334	0.1620\\
5.4440	0.1620\\
5.4543	0.1620\\
5.4647	0.1610\\
5.4753	0.1600\\
5.4856	0.1580\\
5.4959	0.1570\\
5.5062	0.1540\\
5.5167	0.1520\\
5.5272	0.1500\\
5.5374	0.1490\\
5.5482	0.1450\\
5.5591	0.1430\\
5.5700	0.1410\\
5.5804	0.1380\\
5.5908	0.1360\\
5.6010	0.1350\\
5.6114	0.1350\\
5.6221	0.1330\\
5.6324	0.1330\\
5.6429	0.1340\\
5.6530	0.1340\\
5.6635	0.1370\\
5.6741	0.1380\\
5.6845	0.1400\\
5.6949	0.1410\\
5.7056	0.1450\\
5.7158	0.1470\\
5.7260	0.1500\\
5.7363	0.1520\\
5.7473	0.1550\\
};
\addlegendentry{$\vartheta_{z_r}$ = 0.7};

\addplot [color=mycolor3,solid]
  table[row sep=crcr]{%
-1.0000	3.5310\\
-0.9889	3.5310\\
-0.9785	3.5310\\
-0.9679	3.5310\\
-0.9575	3.5310\\
-0.9471	3.5310\\
-0.9367	3.5310\\
-0.9262	3.5310\\
-0.9159	3.5310\\
-0.9056	3.5310\\
-0.8951	3.5310\\
-0.8849	3.5310\\
-0.8741	3.5310\\
-0.8639	3.5310\\
-0.8535	3.5310\\
-0.8431	3.5310\\
-0.8329	3.5310\\
-0.8227	3.5310\\
-0.8116	3.5310\\
-0.8013	3.5320\\
-0.7911	3.5310\\
-0.7807	3.5310\\
-0.7700	3.5310\\
-0.7596	3.5310\\
-0.7493	3.5310\\
-0.7391	3.5310\\
-0.7284	3.5310\\
-0.7181	3.5310\\
-0.7076	3.5310\\
-0.6973	3.5310\\
-0.6865	3.5310\\
-0.6762	3.5310\\
-0.6660	3.5310\\
-0.6555	3.5310\\
-0.6447	3.5310\\
-0.6337	3.5310\\
-0.6234	3.5310\\
-0.6126	3.5310\\
-0.6021	3.5310\\
-0.5918	3.5310\\
-0.5815	3.5310\\
-0.5711	3.5310\\
-0.5609	3.5310\\
-0.5502	3.5300\\
-0.5392	3.5300\\
-0.5284	3.5300\\
-0.5182	3.5300\\
-0.5077	3.5300\\
-0.4972	3.5300\\
-0.4866	3.5300\\
-0.4764	3.5300\\
-0.4658	3.5300\\
-0.4551	3.5290\\
-0.4440	3.5290\\
-0.4334	3.5290\\
-0.4224	3.5290\\
-0.4115	3.5290\\
-0.4012	3.5290\\
-0.3910	3.5290\\
-0.3806	3.5290\\
-0.3700	3.5280\\
-0.3595	3.5280\\
-0.3490	3.5280\\
-0.3390	3.5280\\
-0.3281	3.5280\\
-0.3178	3.5280\\
-0.3077	3.5280\\
-0.2974	3.5270\\
-0.2867	3.5270\\
-0.2764	3.5270\\
-0.2660	3.5270\\
-0.2556	3.5270\\
-0.2450	3.5260\\
-0.2347	3.5260\\
-0.2245	3.5260\\
-0.2139	3.5260\\
-0.2034	3.5260\\
-0.1931	3.5250\\
-0.1828	3.5260\\
-0.1722	3.5250\\
-0.1617	3.5250\\
-0.1514	3.5250\\
-0.1410	3.5250\\
-0.1307	3.5250\\
-0.1204	3.5240\\
-0.1096	3.5240\\
-0.0992	3.5240\\
-0.0891	3.5240\\
-0.0783	3.5230\\
-0.0677	3.5230\\
-0.0576	3.5230\\
-0.0473	3.5230\\
-0.0368	3.5220\\
-0.0264	3.5220\\
-0.0160	3.5220\\
-0.0058	3.5220\\
0.0051	3.5210\\
0.0153	3.5210\\
0.0255	3.5210\\
0.0360	3.5200\\
0.0467	3.5170\\
0.0570	3.5160\\
0.0672	3.5150\\
0.0778	3.5130\\
0.0883	3.5090\\
0.0995	3.5060\\
0.1100	3.5030\\
0.1207	3.4970\\
0.1309	3.4940\\
0.1413	3.4900\\
0.1518	3.4860\\
0.1624	3.4770\\
0.1726	3.4730\\
0.1837	3.4680\\
0.1942	3.4630\\
0.2051	3.4510\\
0.2154	3.4460\\
0.2265	3.4390\\
0.2373	3.4270\\
0.2479	3.4200\\
0.2581	3.4130\\
0.2692	3.4050\\
0.2799	3.3900\\
0.2903	3.3820\\
0.3008	3.3740\\
0.3111	3.3660\\
0.3218	3.3490\\
0.3321	3.3400\\
0.3423	3.3310\\
0.3527	3.3210\\
0.3634	3.3030\\
0.3736	3.2930\\
0.3841	3.2830\\
0.3943	3.2730\\
0.4048	3.2530\\
0.4153	3.2420\\
0.4260	3.2320\\
0.4360	3.2100\\
0.4464	3.1990\\
0.4571	3.1880\\
0.4675	3.1760\\
0.4776	3.1650\\
0.4880	3.1420\\
0.4984	3.1300\\
0.5091	3.1180\\
0.5195	3.1060\\
0.5298	3.0820\\
0.5405	3.0700\\
0.5508	3.0570\\
0.5610	3.0450\\
0.5716	3.0200\\
0.5820	3.0060\\
0.5923	2.9940\\
0.6028	2.9810\\
0.6133	2.9550\\
0.6244	2.9410\\
0.6351	2.9280\\
0.6454	2.9010\\
0.6560	2.8870\\
0.6662	2.8730\\
0.6770	2.8600\\
0.6875	2.8310\\
0.6977	2.8170\\
0.7079	2.8030\\
0.7182	2.7880\\
0.7290	2.7590\\
0.7395	2.7450\\
0.7506	2.7300\\
0.7611	2.7150\\
0.7718	2.6850\\
0.7825	2.6700\\
0.7935	2.6550\\
0.8040	2.6250\\
0.8144	2.6100\\
0.8246	2.5950\\
0.8354	2.5790\\
0.8457	2.5490\\
0.8562	2.5340\\
0.8664	2.5180\\
0.8775	2.5030\\
0.8882	2.4720\\
0.8984	2.4570\\
0.9092	2.4410\\
0.9193	2.4260\\
0.9300	2.3950\\
0.9412	2.3800\\
0.9516	2.3650\\
0.9618	2.3330\\
0.9726	2.3180\\
0.9828	2.3030\\
0.9933	2.2870\\
1.0039	2.2570\\
1.0141	2.2410\\
1.0248	2.2260\\
1.0351	2.2100\\
1.0458	2.1800\\
1.0562	2.1640\\
1.0664	2.1490\\
1.0767	2.1330\\
1.0872	2.1030\\
1.0975	2.0870\\
1.1084	2.0720\\
1.1194	2.0570\\
1.1296	2.0260\\
1.1402	2.0110\\
1.1507	1.9950\\
1.1611	1.9800\\
1.1715	1.9490\\
1.1825	1.9340\\
1.1933	1.9190\\
1.2040	1.8890\\
1.2142	1.8740\\
1.2247	1.8590\\
1.2350	1.8440\\
1.2456	1.8140\\
1.2563	1.7990\\
1.2667	1.7840\\
1.2778	1.7690\\
1.2885	1.7400\\
1.2995	1.7250\\
1.3105	1.7100\\
1.3216	1.6810\\
1.3319	1.6660\\
1.3425	1.6520\\
1.3528	1.6370\\
1.3633	1.6080\\
1.3745	1.5940\\
1.3852	1.5800\\
1.3957	1.5520\\
1.4061	1.5380\\
1.4173	1.5240\\
1.4278	1.5100\\
1.4384	1.4830\\
1.4486	1.4690\\
1.4592	1.4560\\
1.4693	1.4420\\
1.4803	1.4160\\
1.4914	1.4030\\
1.5024	1.3900\\
1.5132	1.3640\\
1.5238	1.3510\\
1.5349	1.3390\\
1.5457	1.3140\\
1.5561	1.3020\\
1.5665	1.2890\\
1.5768	1.2770\\
1.5874	1.2530\\
1.5986	1.2410\\
1.6090	1.2290\\
1.6192	1.2180\\
1.6304	1.1950\\
1.6414	1.1830\\
1.6517	1.1720\\
1.6623	1.1500\\
1.6728	1.1390\\
1.6834	1.1270\\
1.6935	1.1170\\
1.7041	1.0950\\
1.7143	1.0850\\
1.7247	1.0740\\
1.7351	1.0640\\
1.7454	1.0440\\
1.7561	1.0330\\
1.7662	1.0240\\
1.7766	1.0140\\
1.7874	0.9940\\
1.7978	0.9850\\
1.8082	0.9750\\
1.8184	0.9660\\
1.8291	0.9480\\
1.8394	0.9380\\
1.8498	0.9300\\
1.8599	0.9210\\
1.8705	0.9030\\
1.8811	0.8950\\
1.8913	0.8870\\
1.9024	0.8780\\
1.9133	0.8620\\
1.9237	0.8540\\
1.9339	0.8460\\
1.9445	0.8380\\
1.9551	0.8230\\
1.9652	0.8150\\
1.9758	0.8080\\
1.9863	0.8010\\
1.9970	0.7860\\
2.0072	0.7790\\
2.0183	0.7720\\
2.0291	0.7580\\
2.0394	0.7510\\
2.0500	0.7440\\
2.0607	0.7370\\
2.0718	0.7240\\
2.0823	0.7180\\
2.0927	0.7120\\
2.1029	0.7050\\
2.1137	0.6930\\
2.1239	0.6870\\
2.1340	0.6810\\
2.1446	0.6690\\
2.1551	0.6630\\
2.1653	0.6580\\
2.1756	0.6520\\
2.1862	0.6470\\
2.1968	0.6350\\
2.2074	0.6300\\
2.2183	0.6250\\
2.2292	0.6140\\
2.2396	0.6090\\
2.2503	0.6030\\
2.2613	0.5980\\
2.2716	0.5870\\
2.2824	0.5820\\
2.2934	0.5770\\
2.3041	0.5670\\
2.3143	0.5620\\
2.3248	0.5570\\
2.3355	0.5520\\
2.3466	0.5420\\
2.3571	0.5370\\
2.3674	0.5320\\
2.3780	0.5270\\
2.3887	0.5170\\
2.3997	0.5120\\
2.4100	0.5080\\
2.4205	0.4980\\
2.4312	0.4940\\
2.4416	0.4890\\
2.4518	0.4850\\
2.4624	0.4760\\
2.4730	0.4720\\
2.4841	0.4670\\
2.4945	0.4630\\
2.5054	0.4550\\
2.5164	0.4510\\
2.5270	0.4470\\
2.5376	0.4390\\
2.5486	0.4350\\
2.5591	0.4320\\
2.5696	0.4280\\
2.5801	0.4210\\
2.5907	0.4180\\
2.6016	0.4140\\
2.6124	0.4070\\
2.6226	0.4040\\
2.6332	0.4010\\
2.6443	0.3970\\
2.6551	0.3910\\
2.6655	0.3880\\
2.6757	0.3850\\
2.6862	0.3820\\
2.6968	0.3750\\
2.7074	0.3720\\
2.7185	0.3690\\
2.7292	0.3630\\
2.7395	0.3600\\
2.7501	0.3570\\
2.7602	0.3540\\
2.7706	0.3490\\
2.7813	0.3460\\
2.7915	0.3430\\
2.8018	0.3400\\
2.8123	0.3340\\
2.8228	0.3320\\
2.8330	0.3290\\
2.8433	0.3260\\
2.8539	0.3210\\
2.8645	0.3190\\
2.8748	0.3160\\
2.8849	0.3140\\
2.8956	0.3100\\
2.9064	0.3070\\
2.9174	0.3050\\
2.9277	0.3030\\
2.9383	0.2990\\
2.9487	0.2970\\
2.9591	0.2950\\
2.9696	0.2930\\
2.9802	0.2900\\
2.9908	0.2880\\
3.0018	0.2860\\
3.0123	0.2830\\
3.0230	0.2810\\
3.0343	0.2800\\
3.0446	0.2780\\
3.0549	0.2750\\
3.0654	0.2740\\
3.0758	0.2720\\
3.0860	0.2710\\
3.0966	0.2680\\
3.1070	0.2670\\
3.1177	0.2650\\
3.1281	0.2630\\
3.1383	0.2600\\
3.1487	0.2580\\
3.1590	0.2570\\
3.1693	0.2550\\
3.1802	0.2520\\
3.1905	0.2500\\
3.2008	0.2480\\
3.2110	0.2460\\
3.2217	0.2430\\
3.2324	0.2410\\
3.2435	0.2390\\
3.2540	0.2350\\
3.2644	0.2330\\
3.2747	0.2310\\
3.2848	0.2290\\
3.2957	0.2260\\
3.3063	0.2240\\
3.3165	0.2220\\
3.3275	0.2200\\
3.3384	0.2170\\
3.3490	0.2160\\
3.3591	0.2150\\
3.3693	0.2130\\
3.3801	0.2110\\
3.3903	0.2100\\
3.4006	0.2090\\
3.4111	0.2080\\
3.4216	0.2060\\
3.4320	0.2050\\
3.4422	0.2040\\
3.4527	0.2030\\
3.4632	0.2000\\
3.4737	0.1990\\
3.4842	0.1980\\
3.4946	0.1960\\
3.5053	0.1940\\
3.5164	0.1920\\
3.5266	0.1900\\
3.5374	0.1870\\
3.5476	0.1850\\
3.5582	0.1830\\
3.5690	0.1810\\
3.5799	0.1770\\
3.5902	0.1760\\
3.6008	0.1740\\
3.6111	0.1720\\
3.6217	0.1680\\
3.6320	0.1660\\
3.6425	0.1640\\
3.6529	0.1620\\
3.6635	0.1590\\
3.6745	0.1580\\
3.6849	0.1570\\
3.6956	0.1540\\
3.7060	0.1540\\
3.7163	0.1530\\
3.7265	0.1520\\
3.7373	0.1510\\
3.7476	0.1510\\
3.7580	0.1510\\
3.7686	0.1520\\
3.7791	0.1520\\
3.7894	0.1530\\
3.7997	0.1540\\
3.8102	0.1550\\
3.8210	0.1570\\
3.8321	0.1570\\
3.8431	0.1580\\
3.8540	0.1600\\
3.8645	0.1600\\
3.8750	0.1610\\
3.8861	0.1610\\
3.8967	0.1600\\
3.9069	0.1600\\
3.9174	0.1590\\
3.9276	0.1580\\
3.9381	0.1550\\
3.9491	0.1530\\
3.9599	0.1510\\
3.9704	0.1450\\
3.9810	0.1420\\
3.9912	0.1390\\
4.0015	0.1350\\
4.0125	0.1290\\
4.0227	0.1250\\
4.0329	0.1220\\
4.0433	0.1190\\
4.0541	0.1120\\
4.0644	0.1090\\
4.0749	0.1070\\
4.0852	0.1040\\
4.0959	0.1000\\
4.1062	0.0980\\
4.1172	0.0970\\
4.1276	0.0960\\
4.1384	0.0940\\
4.1488	0.0930\\
4.1599	0.0930\\
4.1708	0.0940\\
4.1811	0.0940\\
4.1914	0.0940\\
4.2020	0.0950\\
4.2124	0.0980\\
4.2226	0.1000\\
4.2336	0.1020\\
4.2446	0.1040\\
4.2550	0.1090\\
4.2654	0.1120\\
4.2766	0.1150\\
4.2871	0.1210\\
4.2976	0.1230\\
4.3081	0.1270\\
4.3183	0.1300\\
4.3290	0.1350\\
4.3393	0.1370\\
4.3497	0.1390\\
};
\addlegendentry{$\vartheta_{z_r}$ = 1.0};

\node[right, align=left, text=black]
at (axis cs:0,0.6) {\scriptsize landing starts};
\end{axis}

\begin{axis}[%
width=0.951\fwidth,
height=0.303\fwidth,
at={(0\fwidth,0.42\fwidth)},
scale only axis,
xmin=-1.0012,
xmax=7.4175,
xlabel={$t$ [s]},
ymin=-2.0710,
ymax=0.3950,
ylabel={$W_{\cal W}$ [m/s]},
axis background/.style={fill=white},
title style={font=\labelsize},
xlabel style={font=\labelsize,at={(axis description cs:0.5,\xlabeldist)}},
ylabel style={font=\labelsize,at={(axis description cs:\ylabeldist,0.5)}},
legend style={font=\ticksize},
ticklabel style={font=\ticksize}
]
\addplot [color=black,dashed,forget plot]
  table[row sep=crcr]{%
0.0000	-2.0710\\
0.0000	0.3950\\
};
\addplot [color=mycolor1,solid,forget plot]
  table[row sep=crcr]{%
-1.0012	-0.0000\\
-0.9906	-0.0000\\
-0.9800	-0.0000\\
-0.9693	-0.0000\\
-0.9592	-0.0000\\
-0.9488	-0.0000\\
-0.9381	-0.0000\\
-0.9276	-0.0000\\
-0.9169	-0.0000\\
-0.9061	-0.0000\\
-0.8957	-0.0000\\
-0.8854	-0.0000\\
-0.8750	-0.0000\\
-0.8645	-0.0000\\
-0.8542	-0.0000\\
-0.8436	-0.0000\\
-0.8330	-0.0000\\
-0.8228	-0.0000\\
-0.8120	-0.0000\\
-0.8010	-0.0000\\
-0.7904	-0.0000\\
-0.7799	-0.0000\\
-0.7696	-0.0000\\
-0.7592	-0.0000\\
-0.7488	-0.0000\\
-0.7383	-0.0000\\
-0.7280	-0.0000\\
-0.7177	-0.0000\\
-0.7072	-0.0091\\
-0.6966	-0.0225\\
-0.6862	-0.0319\\
-0.6758	-0.0227\\
-0.6655	-0.0093\\
-0.6546	-0.0000\\
-0.6443	-0.0000\\
-0.6332	-0.0000\\
-0.6231	-0.0000\\
-0.6120	-0.0000\\
-0.6017	-0.0000\\
-0.5906	-0.0000\\
-0.5800	-0.0000\\
-0.5693	-0.0000\\
-0.5591	-0.0093\\
-0.5487	-0.0227\\
-0.5382	-0.0319\\
-0.5278	-0.0227\\
-0.5173	-0.0093\\
-0.5062	-0.0000\\
-0.4955	-0.0000\\
-0.4851	-0.0000\\
-0.4740	-0.0091\\
-0.4635	-0.0224\\
-0.4529	-0.0316\\
-0.4424	-0.0225\\
-0.4322	-0.0092\\
-0.4217	-0.0000\\
-0.4114	-0.0000\\
-0.4004	-0.0000\\
-0.3902	-0.0093\\
-0.3798	-0.0229\\
-0.3695	-0.0323\\
-0.3592	-0.0230\\
-0.3489	-0.0094\\
-0.3383	-0.0000\\
-0.3278	-0.0000\\
-0.3175	-0.0094\\
-0.3072	-0.0228\\
-0.2967	-0.0319\\
-0.2862	-0.0225\\
-0.2756	-0.0092\\
-0.2651	-0.0086\\
-0.2539	-0.0219\\
-0.2435	-0.0313\\
-0.2330	-0.0226\\
-0.2229	-0.0092\\
-0.2124	0.0000\\
-0.2020	-0.0094\\
-0.1917	-0.0227\\
-0.1811	-0.0317\\
-0.1705	-0.0223\\
-0.1604	-0.0090\\
-0.1502	-0.0088\\
-0.1393	-0.0221\\
-0.1288	-0.0309\\
-0.1177	-0.0218\\
-0.1070	-0.0086\\
-0.0967	-0.0092\\
-0.0863	-0.0229\\
-0.0760	-0.0321\\
-0.0655	-0.0227\\
-0.0552	-0.0092\\
-0.0445	0.0000\\
-0.0343	-0.0094\\
-0.0239	-0.0227\\
-0.0133	-0.0319\\
-0.0030	-0.0320\\
0.0074	-0.0416\\
0.0177	-0.0546\\
0.0283	-0.0636\\
0.0387	-0.0640\\
0.0489	-0.0832\\
0.0594	-0.1071\\
0.0702	-0.1262\\
0.0804	-0.1280\\
0.0908	-0.1485\\
0.1010	-0.1702\\
0.1119	-0.1848\\
0.1223	-0.2074\\
0.1333	-0.2343\\
0.1438	-0.2603\\
0.1546	-0.2875\\
0.1657	-0.3001\\
0.1760	-0.3206\\
0.1869	-0.3297\\
0.1979	-0.3418\\
0.2084	-0.3554\\
0.2192	-0.3711\\
0.2303	-0.3903\\
0.2409	-0.4116\\
0.2510	-0.4306\\
0.2619	-0.4480\\
0.2729	-0.4662\\
0.2833	-0.4834\\
0.2938	-0.4999\\
0.3042	-0.5197\\
0.3148	-0.5401\\
0.3252	-0.5580\\
0.3354	-0.5772\\
0.3462	-0.5974\\
0.3566	-0.6214\\
0.3668	-0.6540\\
0.3772	-0.6623\\
0.3880	-0.6744\\
0.3990	-0.6863\\
0.4094	-0.7029\\
0.4200	-0.7258\\
0.4303	-0.7350\\
0.4406	-0.7358\\
0.4510	-0.7294\\
0.4617	-0.7194\\
0.4719	-0.7287\\
0.4822	-0.7348\\
0.4927	-0.7389\\
0.5035	-0.7483\\
0.5136	-0.7604\\
0.5240	-0.7750\\
0.5342	-0.7902\\
0.5450	-0.8055\\
0.5556	-0.8172\\
0.5665	-0.8269\\
0.5772	-0.8350\\
0.5878	-0.8450\\
0.5982	-0.8631\\
0.6089	-0.8835\\
0.6189	-0.8991\\
0.6292	-0.9044\\
0.6398	-0.8946\\
0.6501	-0.8919\\
0.6605	-0.8943\\
0.6709	-0.9052\\
0.6814	-0.9172\\
0.6918	-0.9242\\
0.7021	-0.9290\\
0.7127	-0.9319\\
0.7233	-0.9395\\
0.7333	-0.9431\\
0.7437	-0.9452\\
0.7544	-0.9475\\
0.7648	-0.9552\\
0.7751	-0.9638\\
0.7856	-0.9664\\
0.7960	-0.9580\\
0.8068	-0.9310\\
0.8178	-0.9267\\
0.8286	-0.9220\\
0.8396	-0.9240\\
0.8498	-0.9356\\
0.8600	-0.9423\\
0.8708	-0.9586\\
0.8817	-0.9471\\
0.8927	-0.9565\\
0.9031	-0.9700\\
0.9137	-0.9855\\
0.9243	-0.9926\\
0.9344	-0.9955\\
0.9449	-0.9983\\
0.9553	-0.9927\\
0.9656	-0.9854\\
0.9761	-0.9752\\
0.9869	-0.9590\\
0.9979	-0.9551\\
1.0083	-0.9729\\
1.0188	-0.9849\\
1.0294	-0.9995\\
1.0404	-0.9874\\
1.0512	-0.9980\\
1.0615	-1.0269\\
1.0719	-1.0444\\
1.0822	-1.0432\\
1.0927	-1.0381\\
1.1033	-1.0337\\
1.1137	-1.0339\\
1.1240	-1.0358\\
1.1343	-1.0310\\
1.1451	-1.0345\\
1.1563	-1.0397\\
1.1667	-1.0470\\
1.1771	-1.0438\\
1.1879	-1.0215\\
1.1981	-1.0018\\
1.2091	-0.9864\\
1.2200	-0.9900\\
1.2303	-0.9956\\
1.2406	-1.0038\\
1.2509	-1.0099\\
1.2614	-1.0038\\
1.2722	-0.9962\\
1.2832	-0.9883\\
1.2938	-0.9818\\
1.3043	-0.9812\\
1.3148	-0.9765\\
1.3252	-0.9708\\
1.3353	-0.9603\\
1.3462	-0.9494\\
1.3564	-0.9565\\
1.3671	-0.9657\\
1.3771	-0.9734\\
1.3876	-0.9813\\
1.3980	-0.9901\\
1.4084	-0.9945\\
1.4187	-0.9916\\
1.4294	-0.9836\\
1.4400	-0.9703\\
1.4510	-0.9823\\
1.4616	-0.9989\\
1.4718	-0.9755\\
1.4823	-0.9659\\
1.4930	-0.9590\\
1.5035	-0.9537\\
1.5138	-0.9677\\
1.5242	-0.9758\\
1.5343	-0.9789\\
1.5449	-0.9743\\
1.5553	-0.9628\\
1.5657	-0.9487\\
1.5759	-0.9320\\
1.5870	-0.9100\\
1.5980	-0.8968\\
1.6083	-0.8944\\
1.6190	-0.8966\\
1.6294	-0.8986\\
1.6403	-0.8982\\
1.6511	-0.8949\\
1.6618	-0.9033\\
1.6724	-0.8999\\
1.6831	-0.8941\\
1.6937	-0.8895\\
1.7043	-0.8892\\
1.7146	-0.8978\\
1.7250	-0.8929\\
1.7354	-0.8815\\
1.7464	-0.8649\\
1.7575	-0.8414\\
1.7686	-0.8373\\
1.7794	-0.8431\\
1.7900	-0.8514\\
1.8001	-0.8564\\
1.8102	-0.8635\\
1.8210	-0.8703\\
1.8312	-0.8743\\
1.8419	-0.8752\\
1.8520	-0.8666\\
1.8630	-0.8543\\
1.8741	-0.8375\\
1.8846	-0.8431\\
1.8948	-0.8511\\
1.9053	-0.8463\\
1.9159	-0.8381\\
1.9269	-0.8321\\
1.9378	-0.8324\\
1.9479	-0.8359\\
1.9583	-0.8392\\
1.9688	-0.8412\\
1.9793	-0.8403\\
1.9901	-0.8477\\
2.0003	-0.8466\\
2.0107	-0.8416\\
2.0211	-0.8379\\
2.0322	-0.8197\\
2.0428	-0.8222\\
2.0531	-0.8202\\
2.0638	-0.8237\\
2.0741	-0.8248\\
2.0845	-0.8260\\
2.0950	-0.8286\\
2.1055	-0.8127\\
2.1161	-0.7946\\
2.1270	-0.7747\\
2.1380	-0.7608\\
2.1483	-0.7661\\
2.1593	-0.7766\\
2.1701	-0.7941\\
2.1803	-0.8029\\
2.1906	-0.8008\\
2.2012	-0.7948\\
2.2118	-0.7918\\
2.2221	-0.7901\\
2.2324	-0.7896\\
2.2430	-0.7924\\
2.2532	-0.7902\\
2.2637	-0.7864\\
2.2742	-0.7878\\
2.2846	-0.7785\\
2.2949	-0.7688\\
2.3055	-0.7626\\
2.3157	-0.7450\\
2.3259	-0.7353\\
2.3367	-0.7199\\
2.3469	-0.7335\\
2.3574	-0.7341\\
2.3682	-0.7303\\
2.3793	-0.7197\\
2.3899	-0.7083\\
2.4009	-0.7042\\
2.4119	-0.7017\\
2.4221	-0.7105\\
2.4323	-0.7120\\
2.4424	-0.7178\\
2.4532	-0.6922\\
2.4641	-0.6795\\
2.4749	-0.6719\\
2.4853	-0.6694\\
2.4961	-0.6717\\
2.5062	-0.6710\\
2.5164	-0.6679\\
2.5270	-0.6597\\
2.5379	-0.6487\\
2.5490	-0.6458\\
2.5595	-0.6476\\
2.5700	-0.6554\\
2.5803	-0.6524\\
2.5908	-0.6469\\
2.6018	-0.6401\\
2.6125	-0.6323\\
2.6229	-0.6429\\
2.6335	-0.6549\\
2.6438	-0.6646\\
2.6543	-0.6673\\
2.6645	-0.6550\\
2.6751	-0.6462\\
2.6857	-0.6374\\
2.6961	-0.6288\\
2.7072	-0.6234\\
2.7177	-0.6233\\
2.7283	-0.6437\\
2.7387	-0.6597\\
2.7491	-0.6602\\
2.7593	-0.6561\\
2.7699	-0.6375\\
2.7802	-0.6333\\
2.7906	-0.6294\\
2.8013	-0.6264\\
2.8117	-0.6399\\
2.8219	-0.6370\\
2.8324	-0.6291\\
2.8435	-0.6199\\
2.8542	-0.6095\\
2.8646	-0.6058\\
2.8755	-0.5866\\
2.8865	-0.5984\\
2.8972	-0.6085\\
2.9074	-0.6149\\
2.9175	-0.6140\\
2.9282	-0.6053\\
2.9388	-0.5950\\
2.9493	-0.5887\\
2.9602	-0.5838\\
2.9711	-0.5797\\
2.9822	-0.5831\\
2.9927	-0.5893\\
3.0035	-0.5836\\
3.0146	-0.5804\\
3.0250	-0.5767\\
3.0353	-0.5742\\
3.0459	-0.5710\\
3.0565	-0.5593\\
3.0676	-0.5540\\
3.0783	-0.5649\\
3.0885	-0.5633\\
3.0989	-0.5626\\
3.1094	-0.5602\\
3.1200	-0.5649\\
3.1302	-0.5553\\
3.1404	-0.5384\\
3.1509	-0.5208\\
3.1618	-0.5189\\
3.1721	-0.5314\\
3.1832	-0.5413\\
3.1935	-0.5400\\
3.2046	-0.5316\\
3.2148	-0.5246\\
3.2251	-0.5190\\
3.2352	-0.5150\\
3.2460	-0.5109\\
3.2564	-0.5042\\
3.2667	-0.4948\\
3.2769	-0.4813\\
3.2878	-0.4650\\
3.2988	-0.4532\\
3.3093	-0.4632\\
3.3197	-0.4665\\
3.3307	-0.4630\\
3.3415	-0.4552\\
3.3518	-0.4483\\
3.3625	-0.4445\\
3.3730	-0.4424\\
3.3836	-0.4423\\
3.3937	-0.4398\\
3.4043	-0.4365\\
3.4154	-0.4323\\
3.4260	-0.4341\\
3.4367	-0.4449\\
3.4468	-0.4499\\
3.4573	-0.4511\\
3.4676	-0.4501\\
3.4784	-0.4492\\
3.4887	-0.4445\\
3.4990	-0.4378\\
3.5091	-0.4261\\
3.5199	-0.4170\\
3.5302	-0.4101\\
3.5408	-0.4048\\
3.5510	-0.3980\\
3.5610	-0.3852\\
3.5721	-0.3872\\
3.5831	-0.3929\\
3.5938	-0.3996\\
3.6041	-0.4016\\
3.6146	-0.3925\\
3.6249	-0.3847\\
3.6354	-0.3777\\
3.6459	-0.3718\\
3.6562	-0.3656\\
3.6666	-0.3622\\
3.6768	-0.3603\\
3.6877	-0.3605\\
3.6980	-0.3587\\
3.7085	-0.3554\\
3.7188	-0.3525\\
3.7294	-0.3515\\
3.7398	-0.3505\\
3.7499	-0.3492\\
3.7604	-0.3488\\
3.7711	-0.3499\\
3.7813	-0.3536\\
3.7917	-0.3549\\
3.8020	-0.3541\\
3.8125	-0.3495\\
3.8232	-0.3428\\
3.8341	-0.3392\\
3.8449	-0.3441\\
3.8553	-0.3456\\
3.8658	-0.3484\\
3.8765	-0.3526\\
3.8867	-0.3612\\
3.8971	-0.3612\\
3.9076	-0.3569\\
3.9185	-0.3504\\
3.9290	-0.3466\\
3.9395	-0.3590\\
3.9500	-0.3693\\
3.9607	-0.3549\\
3.9712	-0.3212\\
3.9814	-0.3001\\
3.9916	-0.2851\\
4.0024	-0.3239\\
4.0126	-0.3447\\
4.0230	-0.3547\\
4.0332	-0.3537\\
4.0432	-0.3484\\
4.0542	-0.3389\\
4.0647	-0.3278\\
4.0751	-0.3276\\
4.0854	-0.3327\\
4.0959	-0.3420\\
4.1063	-0.3486\\
4.1165	-0.3493\\
4.1270	-0.3486\\
4.1377	-0.3391\\
4.1480	-0.3299\\
4.1582	-0.3264\\
4.1687	-0.3262\\
4.1795	-0.3279\\
4.1905	-0.3302\\
4.2010	-0.3204\\
4.2116	-0.3063\\
4.2222	-0.2811\\
4.2333	-0.2735\\
4.2438	-0.2799\\
4.2547	-0.2908\\
4.2657	-0.2966\\
4.2760	-0.2942\\
4.2865	-0.2886\\
4.2973	-0.2787\\
4.3084	-0.2738\\
4.3187	-0.2737\\
4.3292	-0.2772\\
4.3396	-0.2866\\
4.3501	-0.2772\\
4.3606	-0.2645\\
4.3709	-0.2559\\
4.3816	-0.2422\\
4.3918	-0.2289\\
4.4021	-0.2711\\
4.4125	-0.2735\\
4.4228	-0.2761\\
4.4334	-0.2765\\
4.4439	-0.2805\\
4.4544	-0.2884\\
4.4646	-0.2967\\
4.4749	-0.3041\\
4.4853	-0.3085\\
4.4961	-0.3064\\
4.5065	-0.3051\\
4.5166	-0.3083\\
4.5269	-0.3113\\
4.5375	-0.3092\\
4.5479	-0.2977\\
4.5582	-0.2940\\
4.5687	-0.2941\\
4.5791	-0.2960\\
4.5895	-0.2947\\
4.6001	-0.2911\\
4.6106	-0.2870\\
4.6209	-0.2826\\
4.6317	-0.2716\\
4.6427	-0.2551\\
4.6534	-0.2541\\
4.6636	-0.2561\\
4.6739	-0.2534\\
4.6842	-0.2473\\
4.6951	-0.2405\\
4.7061	-0.2393\\
4.7166	-0.2363\\
4.7271	-0.2300\\
4.7375	-0.2226\\
4.7481	-0.2041\\
4.7588	-0.2040\\
4.7699	-0.2110\\
4.7803	-0.2192\\
4.7908	-0.2207\\
4.8010	-0.2181\\
4.8114	-0.2141\\
4.8221	-0.2149\\
4.8323	-0.2183\\
4.8425	-0.2220\\
4.8534	-0.2187\\
4.8638	-0.2041\\
4.8740	-0.1914\\
4.8846	-0.1835\\
4.8949	-0.2005\\
4.9054	-0.2059\\
4.9157	-0.2108\\
4.9260	-0.2045\\
4.9367	-0.2003\\
4.9469	-0.1840\\
4.9572	-0.1799\\
4.9678	-0.1814\\
4.9784	-0.1814\\
4.9886	-0.1702\\
4.9989	-0.1686\\
5.0097	-0.1814\\
5.0200	-0.1775\\
5.0303	-0.1731\\
5.0417	-0.1660\\
5.0518	-0.1606\\
5.0626	-0.1571\\
5.0730	-0.1448\\
5.0833	-0.1341\\
5.0938	-0.1277\\
5.1043	-0.1506\\
5.1146	-0.1505\\
5.1251	-0.1373\\
5.1355	-0.1277\\
5.1459	-0.1375\\
5.1562	-0.1420\\
5.1665	-0.1471\\
5.1770	-0.1411\\
5.1876	-0.1451\\
5.1980	-0.1415\\
5.2082	-0.1461\\
5.2189	-0.1401\\
5.2296	-0.1421\\
5.2406	-0.1471\\
5.2509	-0.1568\\
5.2616	-0.1579\\
5.2721	-0.1500\\
5.2825	-0.1554\\
5.2934	-0.1581\\
5.3041	-0.1628\\
5.3148	-0.1493\\
5.3249	-0.1485\\
5.3352	-0.1505\\
5.3462	-0.1546\\
5.3572	-0.1542\\
5.3675	-0.1648\\
5.3784	-0.1672\\
5.3894	-0.1534\\
5.4000	-0.1338\\
5.4104	-0.1360\\
5.4209	-0.1512\\
5.4312	-0.1514\\
5.4416	-0.1380\\
5.4520	-0.1265\\
5.4629	-0.1263\\
5.4731	-0.1188\\
5.4834	-0.1162\\
5.4940	-0.1189\\
5.5043	-0.1279\\
5.5147	-0.1264\\
5.5257	-0.1240\\
5.5365	-0.1219\\
5.5476	-0.1132\\
5.5584	-0.1020\\
5.5686	-0.0861\\
5.5794	-0.0737\\
5.5895	-0.0639\\
5.6001	-0.0735\\
5.6105	-0.0868\\
5.6208	-0.0963\\
5.6313	-0.0962\\
5.6416	-0.1059\\
5.6519	-0.1192\\
5.6624	-0.1269\\
5.6731	-0.1253\\
5.6838	-0.1236\\
5.6948	-0.1245\\
5.7051	-0.1211\\
5.7156	-0.1345\\
5.7260	-0.1451\\
5.7367	-0.1591\\
5.7469	-0.1544\\
5.7572	-0.1650\\
5.7678	-0.1758\\
5.7783	-0.1806\\
5.7889	-0.1572\\
5.7997	-0.1707\\
5.8101	-0.1836\\
5.8209	-0.1798\\
5.8310	-0.1721\\
5.8417	-0.1653\\
5.8519	-0.1606\\
5.8627	-0.1548\\
5.8730	-0.1398\\
5.8832	-0.1162\\
5.8935	-0.0963\\
5.9041	-0.0954\\
5.9145	-0.0956\\
5.9249	-0.0960\\
5.9354	-0.0957\\
5.9459	-0.0868\\
5.9560	-0.0732\\
5.9668	-0.0640\\
5.9769	-0.0723\\
5.9877	-0.0764\\
5.9978	-0.0730\\
6.0082	-0.0644\\
6.0187	-0.0734\\
6.0290	-0.0862\\
6.0397	-0.0955\\
6.0500	-0.0956\\
6.0605	-0.0964\\
6.0708	-0.0968\\
6.0810	-0.0881\\
6.0917	-0.0849\\
6.1020	-0.0873\\
6.1125	-0.1054\\
6.1227	-0.1093\\
6.1332	-0.1158\\
6.1435	-0.1181\\
6.1543	-0.1266\\
6.1646	-0.1127\\
6.1748	-0.1008\\
6.1853	-0.0886\\
6.1959	-0.1040\\
6.2062	-0.1083\\
6.2164	-0.1055\\
6.2270	-0.0952\\
6.2377	-0.0952\\
6.2479	-0.0963\\
6.2583	-0.0968\\
6.2687	-0.0948\\
6.2796	-0.0841\\
6.2905	-0.0707\\
6.3010	-0.0629\\
6.3115	-0.0723\\
6.3219	-0.0741\\
6.3331	-0.0704\\
6.3437	-0.0534\\
6.3542	-0.0507\\
6.3646	-0.0460\\
6.3748	-0.0511\\
6.3854	-0.0546\\
6.3960	-0.0694\\
6.4077	-0.0808\\
6.4186	-0.0910\\
6.4291	-0.0947\\
6.4394	-0.1039\\
6.4502	-0.1074\\
6.4605	-0.1037\\
6.4711	-0.0942\\
6.4821	-0.1129\\
6.4925	-0.1375\\
6.5033	-0.1486\\
6.5138	-0.1248\\
6.5247	-0.1030\\
6.5352	-0.0939\\
6.5458	-0.0952\\
6.5561	-0.0958\\
6.5666	-0.0960\\
6.5770	-0.0864\\
6.5875	-0.0730\\
6.5978	-0.0548\\
6.6084	-0.0416\\
6.6185	-0.0320\\
6.6293	-0.0317\\
6.6399	-0.0310\\
6.6509	-0.0219\\
6.6617	-0.0088\\
6.6729	0.0000\\
6.6834	-0.0091\\
6.6940	-0.0218\\
6.7052	-0.0309\\
6.7156	-0.0226\\
6.7259	-0.0197\\
6.7365	-0.0230\\
6.7470	-0.0409\\
6.7571	-0.0451\\
6.7680	-0.0517\\
6.7782	-0.0559\\
6.7885	-0.0742\\
6.7988	-0.0877\\
6.8091	-0.0959\\
6.8199	-0.0954\\
6.8301	-0.0956\\
6.8407	-0.1160\\
6.8509	-0.1388\\
6.8617	-0.1580\\
6.8721	-0.1480\\
6.8831	-0.1366\\
6.8936	-0.1216\\
6.9041	-0.1274\\
6.9145	-0.1186\\
6.9249	-0.1047\\
6.9355	-0.0864\\
6.9458	-0.0728\\
6.9561	-0.0454\\
6.9666	-0.0192\\
6.9770	0.0083\\
6.9880	0.0217\\
6.9989	0.0402\\
7.0093	0.0627\\
7.0201	0.0755\\
7.0303	0.0727\\
7.0406	0.0745\\
7.0508	0.0866\\
7.0616	0.0954\\
7.0718	0.0871\\
7.0822	0.0836\\
7.0927	0.0769\\
7.1030	0.0639\\
7.1135	0.0321\\
7.1239	0.0093\\
7.1342	-0.0088\\
7.1451	-0.0306\\
7.1560	-0.0711\\
7.1666	-0.1157\\
7.1773	-0.1580\\
7.1875	-0.1708\\
7.1978	-0.1688\\
7.2081	-0.1969\\
7.2186	-0.2475\\
7.2291	-0.3057\\
7.2397	-0.3246\\
7.2503	-0.3318\\
7.2604	-0.3314\\
7.2707	-0.3323\\
7.2812	-0.3500\\
7.2916	-0.3210\\
7.3019	-0.2914\\
7.3123	-0.2399\\
7.3229	-0.1891\\
7.3333	-0.1436\\
7.3436	-0.0876\\
7.3541	-0.0416\\
7.3648	0.0002\\
7.3750	0.0529\\
7.3853	0.1204\\
7.3957	0.1829\\
7.4064	0.1894\\
7.4175	0.1949\\
};
\addplot [color=mycolor2,solid,forget plot]
  table[row sep=crcr]{%
-1.0008	-0.0064\\
-0.9902	0.0117\\
-0.9799	0.0316\\
-0.9691	0.0437\\
-0.9590	0.0402\\
-0.9481	0.0229\\
-0.9371	0.0193\\
-0.9266	0.0229\\
-0.9160	0.0314\\
-0.9052	0.0225\\
-0.8943	0.0193\\
-0.8838	0.0231\\
-0.8734	0.0321\\
-0.8631	0.0233\\
-0.8527	0.0200\\
-0.8423	0.0227\\
-0.8313	0.0308\\
-0.8203	0.0220\\
-0.8101	0.0193\\
-0.7996	0.0232\\
-0.7894	0.0320\\
-0.7787	0.0230\\
-0.7683	0.0195\\
-0.7578	0.0230\\
-0.7474	0.0320\\
-0.7370	0.0228\\
-0.7267	0.0092\\
-0.7161	0.0094\\
-0.7059	0.0228\\
-0.6953	0.0320\\
-0.6850	0.0232\\
-0.6744	0.0201\\
-0.6641	0.0233\\
-0.6537	0.0322\\
-0.6434	0.0229\\
-0.6329	0.0093\\
-0.6226	-0.0000\\
-0.6119	0.0093\\
-0.6016	0.0230\\
-0.5913	0.0322\\
-0.5808	0.0228\\
-0.5705	0.0091\\
-0.5599	0.0094\\
-0.5497	0.0229\\
-0.5392	0.0318\\
-0.5284	0.0222\\
-0.5180	0.0089\\
-0.5070	0.0093\\
-0.4967	0.0226\\
-0.4859	0.0316\\
-0.4754	0.0223\\
-0.4643	0.0092\\
-0.4536	-0.0000\\
-0.4430	0.0093\\
-0.4327	0.0230\\
-0.4224	0.0322\\
-0.4119	0.0228\\
-0.4015	0.0091\\
-0.3913	0.0091\\
-0.3808	0.0226\\
-0.3703	0.0318\\
-0.3598	0.0226\\
-0.3487	0.0092\\
-0.3383	-0.0000\\
-0.3275	0.0087\\
-0.3164	0.0218\\
-0.3059	0.0312\\
-0.2953	0.0229\\
-0.2848	0.0190\\
-0.2736	0.0223\\
-0.2633	0.0313\\
-0.2525	0.0224\\
-0.2423	0.0089\\
-0.2321	0.0090\\
-0.2215	0.0223\\
-0.2109	0.0313\\
-0.2001	0.0225\\
-0.1890	0.0194\\
-0.1785	0.0233\\
-0.1683	0.0322\\
-0.1578	0.0229\\
-0.1476	0.0092\\
-0.1370	-0.0000\\
-0.1265	0.0094\\
-0.1163	0.0230\\
-0.1059	0.0319\\
-0.0951	0.0227\\
-0.0847	0.0198\\
-0.0746	0.0233\\
-0.0640	0.0314\\
-0.0528	0.0217\\
-0.0420	0.0086\\
-0.0318	0.0092\\
-0.0213	0.0228\\
-0.0111	0.0323\\
-0.0008	0.0133\\
0.0099	-0.0125\\
0.0204	-0.0393\\
0.0316	-0.0532\\
0.0420	-0.0814\\
0.0523	-0.1061\\
0.0634	-0.1313\\
0.0744	-0.1457\\
0.0847	-0.1784\\
0.0949	-0.2124\\
0.1060	-0.2428\\
0.1162	-0.2564\\
0.1266	-0.3021\\
0.1367	-0.3437\\
0.1474	-0.3826\\
0.1577	-0.4171\\
0.1683	-0.4497\\
0.1786	-0.4819\\
0.1891	-0.5120\\
0.1995	-0.5396\\
0.2107	-0.5669\\
0.2214	-0.6027\\
0.2317	-0.6406\\
0.2422	-0.6769\\
0.2524	-0.7104\\
0.2629	-0.7385\\
0.2734	-0.7715\\
0.2839	-0.8039\\
0.2940	-0.8344\\
0.3047	-0.8645\\
0.3151	-0.8776\\
0.3253	-0.8913\\
0.3357	-0.9058\\
0.3463	-0.9227\\
0.3567	-0.9546\\
0.3679	-0.9828\\
0.3783	-1.0095\\
0.3892	-1.0359\\
0.3995	-1.0605\\
0.4098	-1.0861\\
0.4200	-1.1081\\
0.4306	-1.1195\\
0.4413	-1.1223\\
0.4524	-1.1286\\
0.4630	-1.1349\\
0.4742	-1.1643\\
0.4847	-1.1841\\
0.4950	-1.2055\\
0.5056	-1.2352\\
0.5159	-1.2584\\
0.5264	-1.2798\\
0.5366	-1.2968\\
0.5474	-1.3121\\
0.5576	-1.3291\\
0.5683	-1.3451\\
0.5784	-1.3568\\
0.5888	-1.3620\\
0.5993	-1.3684\\
0.6100	-1.3780\\
0.6202	-1.3859\\
0.6306	-1.3917\\
0.6410	-1.3946\\
0.6512	-1.4064\\
0.6619	-1.4172\\
0.6724	-1.4238\\
0.6825	-1.4095\\
0.6930	-1.4073\\
0.7036	-1.4042\\
0.7142	-1.4098\\
0.7252	-1.4309\\
0.7356	-1.4443\\
0.7463	-1.4551\\
0.7568	-1.4516\\
0.7679	-1.4502\\
0.7783	-1.4523\\
0.7889	-1.4596\\
0.7995	-1.4797\\
0.8098	-1.4792\\
0.8201	-1.4735\\
0.8309	-1.4582\\
0.8420	-1.4230\\
0.8531	-1.4126\\
0.8639	-1.4469\\
0.8745	-1.4748\\
0.8856	-1.4800\\
0.8961	-1.4941\\
0.9067	-1.5034\\
0.9169	-1.5056\\
0.9273	-1.5030\\
0.9380	-1.4852\\
0.9483	-1.5024\\
0.9584	-1.5068\\
0.9689	-1.5169\\
0.9795	-1.5268\\
0.9898	-1.5306\\
1.0003	-1.5270\\
1.0108	-1.5180\\
1.0217	-1.4985\\
1.0326	-1.4997\\
1.0433	-1.5076\\
1.0535	-1.5160\\
1.0640	-1.5171\\
1.0747	-1.4853\\
1.0856	-1.4615\\
1.0964	-1.4521\\
1.1072	-1.4507\\
1.1182	-1.4565\\
1.1283	-1.4610\\
1.1390	-1.4655\\
1.1492	-1.4689\\
1.1595	-1.4697\\
1.1699	-1.4714\\
1.1808	-1.4739\\
1.1911	-1.4750\\
1.2013	-1.4750\\
1.2117	-1.4728\\
1.2224	-1.4708\\
1.2328	-1.4731\\
1.2430	-1.4692\\
1.2532	-1.4608\\
1.2639	-1.4353\\
1.2743	-1.4220\\
1.2848	-1.4019\\
1.2956	-1.3815\\
1.3067	-1.3877\\
1.3174	-1.4021\\
1.3285	-1.4170\\
1.3389	-1.4303\\
1.3494	-1.4297\\
1.3599	-1.4209\\
1.3703	-1.4122\\
1.3806	-1.4000\\
1.3910	-1.3872\\
1.4013	-1.3907\\
1.4119	-1.3952\\
1.4279	-1.3918\\
1.4381	-1.3915\\
1.4484	-1.3780\\
1.4588	-1.3647\\
1.4698	-1.3378\\
1.4806	-1.3109\\
1.4910	-1.2893\\
1.5018	-1.2880\\
1.5118	-1.2804\\
1.5225	-1.2587\\
1.5328	-1.2647\\
1.5431	-1.2663\\
1.5534	-1.2638\\
1.5637	-1.2390\\
1.5745	-1.2137\\
1.5847	-1.1861\\
1.5949	-1.1668\\
1.6061	-1.1419\\
1.6169	-1.1643\\
1.6276	-1.1880\\
1.6380	-1.2120\\
1.6483	-1.2021\\
1.6585	-1.1873\\
1.6692	-1.1680\\
1.6799	-1.1590\\
1.6904	-1.1619\\
1.7014	-1.1589\\
1.7117	-1.1518\\
1.7223	-1.1382\\
1.7334	-1.1196\\
1.7440	-1.1016\\
1.7547	-1.0892\\
1.7653	-1.0702\\
1.7755	-1.0613\\
1.7866	-1.0543\\
1.7973	-1.0466\\
1.8078	-1.0520\\
1.8186	-1.0739\\
1.8286	-1.0775\\
1.8390	-1.0675\\
1.8491	-1.0551\\
1.8597	-1.0315\\
1.8701	-1.0047\\
1.8807	-0.9973\\
1.8908	-0.9933\\
1.9011	-0.9883\\
1.9117	-0.9834\\
1.9223	-0.9827\\
1.9327	-0.9791\\
1.9435	-0.9752\\
1.9535	-0.9712\\
1.9640	-0.9654\\
1.9743	-0.9608\\
1.9847	-0.9615\\
1.9950	-0.9547\\
2.0056	-0.9454\\
2.0158	-0.9352\\
2.0266	-0.9314\\
2.0371	-0.9290\\
2.0474	-0.9223\\
2.0577	-0.9169\\
2.0680	-0.9085\\
2.0783	-0.8982\\
2.0888	-0.8871\\
2.0996	-0.8801\\
2.1097	-0.8743\\
2.1199	-0.8689\\
2.1305	-0.8637\\
2.1410	-0.8579\\
2.1512	-0.8543\\
2.1616	-0.8516\\
2.1723	-0.8502\\
2.1828	-0.8515\\
2.1935	-0.8539\\
2.2035	-0.8507\\
2.2140	-0.8461\\
2.2243	-0.8341\\
2.2348	-0.8233\\
2.2451	-0.8116\\
2.2556	-0.8007\\
2.2660	-0.7896\\
2.2763	-0.7774\\
2.2869	-0.7668\\
2.2975	-0.7590\\
2.3078	-0.7582\\
2.3186	-0.7472\\
2.3296	-0.7442\\
2.3402	-0.7422\\
2.3511	-0.7392\\
2.3615	-0.7342\\
2.3722	-0.7272\\
2.3827	-0.7210\\
2.3931	-0.7170\\
2.4035	-0.7110\\
2.4137	-0.6972\\
2.4246	-0.6750\\
2.4357	-0.6604\\
2.4466	-0.6495\\
2.4576	-0.6475\\
2.4681	-0.6405\\
2.4782	-0.6313\\
2.4892	-0.6250\\
2.5000	-0.6059\\
2.5108	-0.5979\\
2.5211	-0.5992\\
2.5317	-0.5909\\
2.5421	-0.5835\\
2.5531	-0.5789\\
2.5640	-0.5781\\
2.5745	-0.5884\\
2.5848	-0.5876\\
2.5952	-0.5833\\
2.6056	-0.5770\\
2.6160	-0.5716\\
2.6265	-0.5641\\
2.6367	-0.5558\\
2.6471	-0.5470\\
2.6576	-0.5448\\
2.6680	-0.5419\\
2.6782	-0.5376\\
2.6891	-0.5295\\
2.6994	-0.5270\\
2.7097	-0.5294\\
2.7200	-0.5305\\
2.7308	-0.5266\\
2.7414	-0.5071\\
2.7522	-0.4971\\
2.7630	-0.5011\\
2.7734	-0.5004\\
2.7839	-0.4960\\
2.7941	-0.4897\\
2.8047	-0.4825\\
2.8150	-0.4850\\
2.8252	-0.4864\\
2.8359	-0.4845\\
2.8464	-0.4726\\
2.8575	-0.4659\\
2.8683	-0.4612\\
2.8785	-0.4616\\
2.8892	-0.4623\\
2.9000	-0.4625\\
2.9110	-0.4622\\
2.9212	-0.4628\\
2.9319	-0.4611\\
2.9430	-0.4582\\
2.9534	-0.4509\\
2.9640	-0.4437\\
2.9745	-0.4449\\
2.9849	-0.4462\\
2.9952	-0.4480\\
3.0057	-0.4504\\
3.0160	-0.4419\\
3.0264	-0.4317\\
3.0370	-0.4223\\
3.0475	-0.4154\\
3.0579	-0.4149\\
3.0690	-0.4095\\
3.0797	-0.4049\\
3.0904	-0.4131\\
3.1015	-0.4198\\
3.1120	-0.4235\\
3.1225	-0.4252\\
3.1328	-0.4221\\
3.1432	-0.4245\\
3.1533	-0.4271\\
3.1641	-0.4261\\
3.1746	-0.4220\\
3.1856	-0.4218\\
3.1957	-0.4263\\
3.2059	-0.3949\\
3.2163	-0.3577\\
3.2265	-0.3522\\
3.2369	-0.3663\\
3.2473	-0.3895\\
3.2580	-0.4010\\
3.2686	-0.3806\\
3.2796	-0.3754\\
3.2901	-0.3720\\
3.3007	-0.3745\\
3.3117	-0.3767\\
3.3217	-0.3750\\
3.3326	-0.3707\\
3.3430	-0.3625\\
3.3534	-0.3542\\
3.3640	-0.3513\\
3.3744	-0.3522\\
3.3846	-0.3534\\
3.3952	-0.3396\\
3.4060	-0.3042\\
3.4161	-0.2930\\
3.4264	-0.2952\\
3.4370	-0.3048\\
3.4475	-0.3195\\
3.4577	-0.3217\\
3.4681	-0.3216\\
3.4785	-0.3185\\
3.4891	-0.3103\\
3.4993	-0.3018\\
3.5099	-0.2980\\
3.5202	-0.2951\\
3.5309	-0.2922\\
3.5420	-0.2879\\
3.5523	-0.2814\\
3.5631	-0.2861\\
3.5734	-0.2886\\
3.5837	-0.2890\\
3.5940	-0.2884\\
3.6047	-0.2858\\
3.6152	-0.2923\\
3.6254	-0.3013\\
3.6358	-0.3092\\
3.6465	-0.3115\\
3.6576	-0.3104\\
3.6681	-0.3121\\
3.6785	-0.3151\\
3.6891	-0.3180\\
3.6996	-0.3190\\
3.7098	-0.3191\\
3.7200	-0.3193\\
3.7308	-0.3196\\
3.7411	-0.3181\\
3.7516	-0.3216\\
3.7618	-0.3077\\
3.7727	-0.2773\\
3.7828	-0.2418\\
3.7932	-0.2410\\
3.8036	-0.2517\\
3.8140	-0.2615\\
3.8246	-0.2541\\
3.8350	-0.2507\\
3.8461	-0.2474\\
3.8569	-0.2394\\
3.8679	-0.2278\\
3.8787	-0.2147\\
3.8891	-0.2001\\
3.8995	-0.1917\\
3.9096	-0.1921\\
3.9200	-0.1924\\
3.9308	-0.2027\\
3.9411	-0.1971\\
3.9521	-0.1774\\
3.9630	-0.1788\\
3.9735	-0.1809\\
3.9846	-0.1822\\
3.9951	-0.1811\\
4.0059	-0.1783\\
4.0167	-0.1766\\
4.0275	-0.1941\\
4.0382	-0.2014\\
4.0484	-0.1926\\
4.0588	-0.1823\\
4.0692	-0.1911\\
4.0798	-0.2144\\
4.0901	-0.2175\\
4.1005	-0.2256\\
4.1109	-0.2275\\
4.1212	-0.2321\\
4.1318	-0.2217\\
4.1423	-0.2197\\
4.1530	-0.2171\\
4.1639	-0.1959\\
4.1745	-0.1910\\
4.1850	-0.1867\\
4.1951	-0.1827\\
4.2060	-0.1776\\
4.2169	-0.1675\\
4.2274	-0.1640\\
4.2383	-0.1543\\
4.2493	-0.1457\\
4.2596	-0.1442\\
4.2701	-0.1395\\
4.2811	-0.1412\\
4.2922	-0.1424\\
4.3032	-0.1510\\
4.3142	-0.1424\\
4.3253	-0.1309\\
4.3359	-0.1338\\
4.3463	-0.1499\\
4.3568	-0.1507\\
4.3672	-0.1380\\
4.3774	-0.1280\\
4.3882	-0.1247\\
4.3992	-0.1152\\
4.4097	-0.1265\\
4.4202	-0.1419\\
4.4308	-0.1596\\
4.4409	-0.1450\\
4.4512	-0.1314\\
4.4619	-0.1185\\
4.4726	-0.1235\\
4.4837	-0.1243\\
4.4940	-0.1252\\
4.5048	-0.1242\\
4.5158	-0.1139\\
4.5266	-0.1137\\
4.5367	-0.1189\\
4.5474	-0.1283\\
4.5576	-0.1207\\
4.5680	-0.1271\\
4.5783	-0.1281\\
4.5889	-0.1346\\
4.5993	-0.1191\\
4.6097	-0.1169\\
4.6199	-0.1190\\
4.6307	-0.1178\\
4.6411	-0.1042\\
4.6515	-0.1054\\
4.6617	-0.1069\\
4.6726	-0.1018\\
4.6836	-0.0919\\
4.6943	-0.1030\\
4.7047	-0.1187\\
4.7150	-0.1200\\
4.7254	-0.1167\\
4.7357	-0.1185\\
4.7465	-0.1273\\
4.7567	-0.1195\\
4.7670	-0.1172\\
4.7774	-0.1183\\
4.7882	-0.1267\\
4.7985	-0.1208\\
4.8087	-0.1318\\
4.8192	-0.1441\\
4.8297	-0.1601\\
4.8399	-0.1475\\
4.8507	-0.1352\\
4.8612	-0.1214\\
4.8715	-0.1262\\
4.8826	-0.1154\\
4.8932	-0.1029\\
4.9034	-0.0962\\
4.9139	-0.0958\\
4.9245	-0.0865\\
4.9347	-0.0833\\
4.9450	-0.0789\\
4.9551	-0.0827\\
4.9659	-0.0924\\
4.9763	-0.0944\\
4.9867	-0.0961\\
4.9972	-0.0822\\
5.0077	-0.0763\\
5.0182	-0.0724\\
5.0287	-0.0635\\
5.0392	-0.0627\\
5.0502	-0.0715\\
5.0607	-0.0845\\
5.0714	-0.0939\\
5.0821	-0.0851\\
5.0927	-0.0822\\
5.1033	-0.0850\\
5.1142	-0.1007\\
5.1253	-0.1146\\
5.1356	-0.1158\\
5.1464	-0.1039\\
5.1569	-0.0880\\
5.1680	-0.1022\\
5.1783	-0.1137\\
5.1891	-0.1263\\
5.1995	-0.1254\\
5.2103	-0.1241\\
5.2213	-0.1239\\
5.2317	-0.1167\\
5.2422	-0.1058\\
5.2524	-0.0967\\
5.2629	-0.0865\\
5.2736	-0.0710\\
5.2847	-0.0529\\
5.2951	-0.0495\\
5.3056	-0.0446\\
5.3161	-0.0406\\
5.3265	-0.0228\\
5.3370	-0.0093\\
5.3474	-0.0000\\
5.3575	-0.0000\\
5.3680	0.0092\\
5.3785	0.0223\\
5.3893	0.0308\\
5.4005	0.0214\\
5.4117	-0.0004\\
5.4223	-0.0224\\
5.4327	-0.0318\\
5.4432	-0.0320\\
5.4535	-0.0413\\
5.4639	-0.0722\\
5.4745	-0.1086\\
5.4848	-0.1468\\
5.4951	-0.1745\\
5.5054	-0.1998\\
5.5159	-0.2137\\
5.5264	-0.2200\\
5.5366	-0.2246\\
5.5474	-0.2263\\
5.5583	-0.2267\\
5.5692	-0.2258\\
5.5796	-0.2126\\
5.5900	-0.1769\\
5.6002	-0.1306\\
5.6106	-0.0884\\
5.6213	-0.0634\\
5.6316	-0.0225\\
5.6421	0.0377\\
5.6522	0.0873\\
5.6627	0.1368\\
5.6733	0.1729\\
5.6837	0.1965\\
5.6941	0.2152\\
5.7049	0.2219\\
5.7151	0.2285\\
5.7252	0.2387\\
5.7355	0.2550\\
5.7465	0.2731\\
};
\addplot [color=mycolor3,solid,forget plot]
  table[row sep=crcr]{%
-1.0000	-0.0000\\
-0.9889	-0.0000\\
-0.9785	-0.0000\\
-0.9679	-0.0000\\
-0.9575	-0.0000\\
-0.9471	-0.0000\\
-0.9367	-0.0000\\
-0.9262	-0.0000\\
-0.9159	-0.0000\\
-0.9056	-0.0000\\
-0.8951	-0.0000\\
-0.8849	-0.0000\\
-0.8741	-0.0000\\
-0.8639	-0.0000\\
-0.8535	-0.0000\\
-0.8431	-0.0000\\
-0.8329	-0.0000\\
-0.8227	0.0084\\
-0.8116	0.0120\\
-0.8013	0.0084\\
-0.7911	-0.0089\\
-0.7807	-0.0126\\
-0.7700	-0.0090\\
-0.7596	-0.0000\\
-0.7493	-0.0000\\
-0.7391	-0.0000\\
-0.7284	-0.0000\\
-0.7181	-0.0000\\
-0.7076	-0.0000\\
-0.6973	-0.0000\\
-0.6865	-0.0000\\
-0.6762	-0.0000\\
-0.6660	-0.0000\\
-0.6555	-0.0000\\
-0.6447	-0.0000\\
-0.6337	-0.0000\\
-0.6234	-0.0000\\
-0.6126	-0.0000\\
-0.6021	-0.0000\\
-0.5918	-0.0000\\
-0.5815	-0.0000\\
-0.5711	-0.0094\\
-0.5609	-0.0228\\
-0.5502	-0.0313\\
-0.5392	-0.0218\\
-0.5284	-0.0088\\
-0.5182	-0.0000\\
-0.5077	-0.0000\\
-0.4972	-0.0000\\
-0.4866	-0.0000\\
-0.4764	-0.0091\\
-0.4658	-0.0223\\
-0.4551	-0.0309\\
-0.4440	-0.0216\\
-0.4334	-0.0087\\
-0.4224	-0.0000\\
-0.4115	-0.0000\\
-0.4012	-0.0000\\
-0.3910	-0.0092\\
-0.3806	-0.0226\\
-0.3700	-0.0316\\
-0.3595	-0.0224\\
-0.3490	-0.0091\\
-0.3390	-0.0000\\
-0.3281	-0.0000\\
-0.3178	-0.0095\\
-0.3077	-0.0232\\
-0.2974	-0.0322\\
-0.2867	-0.0225\\
-0.2764	-0.0089\\
-0.2660	-0.0092\\
-0.2556	-0.0226\\
-0.2450	-0.0319\\
-0.2347	-0.0228\\
-0.2245	-0.0093\\
-0.2139	-0.0089\\
-0.2034	-0.0131\\
-0.1931	-0.0183\\
-0.1828	-0.0138\\
-0.1722	-0.0183\\
-0.1617	-0.0130\\
-0.1514	-0.0088\\
-0.1410	-0.0093\\
-0.1307	-0.0231\\
-0.1204	-0.0320\\
-0.1096	-0.0226\\
-0.0992	-0.0193\\
-0.0891	-0.0231\\
-0.0783	-0.0316\\
-0.0677	-0.0225\\
-0.0576	-0.0194\\
-0.0473	-0.0232\\
-0.0368	-0.0321\\
-0.0264	-0.0230\\
-0.0160	-0.0196\\
-0.0058	-0.0229\\
0.0051	-0.0318\\
0.0153	-0.0322\\
0.0255	-0.0647\\
0.0360	-0.1097\\
0.0467	-0.1497\\
0.0570	-0.1563\\
0.0672	-0.1843\\
0.0778	-0.2174\\
0.0883	-0.2703\\
0.0995	-0.3171\\
0.1100	-0.3530\\
0.1207	-0.3880\\
0.1309	-0.4302\\
0.1413	-0.4727\\
0.1518	-0.5139\\
0.1624	-0.5533\\
0.1726	-0.5918\\
0.1837	-0.6305\\
0.1942	-0.6659\\
0.2051	-0.6904\\
0.2154	-0.7117\\
0.2265	-0.7415\\
0.2373	-0.7795\\
0.2479	-0.8225\\
0.2581	-0.8651\\
0.2692	-0.9080\\
0.2799	-0.9537\\
0.2903	-0.9941\\
0.3008	-1.0258\\
0.3111	-1.0549\\
0.3218	-1.0865\\
0.3321	-1.1179\\
0.3423	-1.1479\\
0.3527	-1.1766\\
0.3634	-1.2040\\
0.3736	-1.2369\\
0.3841	-1.2660\\
0.3943	-1.2892\\
0.4048	-1.3042\\
0.4153	-1.3120\\
0.4260	-1.3552\\
0.4360	-1.3883\\
0.4464	-1.4206\\
0.4571	-1.4481\\
0.4675	-1.4673\\
0.4776	-1.4832\\
0.4880	-1.5011\\
0.4984	-1.5093\\
0.5091	-1.5226\\
0.5195	-1.5344\\
0.5298	-1.5381\\
0.5405	-1.5372\\
0.5508	-1.5581\\
0.5610	-1.5771\\
0.5716	-1.6002\\
0.5820	-1.6212\\
0.5923	-1.6365\\
0.6028	-1.6539\\
0.6133	-1.6649\\
0.6244	-1.6860\\
0.6351	-1.7062\\
0.6454	-1.7213\\
0.6560	-1.7362\\
0.6662	-1.7433\\
0.6770	-1.7499\\
0.6875	-1.7589\\
0.6977	-1.7680\\
0.7079	-1.7817\\
0.7182	-1.7923\\
0.7290	-1.7940\\
0.7395	-1.8255\\
0.7506	-1.8476\\
0.7611	-1.8604\\
0.7718	-1.8724\\
0.7825	-1.8744\\
0.7935	-1.8886\\
0.8040	-1.9055\\
0.8144	-1.9224\\
0.8246	-1.9345\\
0.8354	-1.9429\\
0.8457	-1.9509\\
0.8562	-1.9433\\
0.8664	-1.9275\\
0.8775	-1.9113\\
0.8882	-1.8994\\
0.8984	-1.9028\\
0.9092	-1.9063\\
0.9193	-1.9165\\
0.9300	-1.9218\\
0.9412	-1.9300\\
0.9516	-1.9354\\
0.9618	-1.9431\\
0.9726	-1.9438\\
0.9828	-1.9248\\
0.9933	-1.9381\\
1.0039	-1.9575\\
1.0141	-1.9703\\
1.0248	-1.9741\\
1.0351	-1.9710\\
1.0458	-1.9579\\
1.0562	-1.9583\\
1.0664	-1.9644\\
1.0767	-1.9716\\
1.0872	-1.9719\\
1.0975	-1.9725\\
1.1084	-1.9739\\
1.1194	-1.9711\\
1.1296	-1.9717\\
1.1402	-1.9726\\
1.1507	-1.9707\\
1.1611	-1.9612\\
1.1715	-1.9407\\
1.1825	-1.8855\\
1.1933	-1.8552\\
1.2040	-1.8727\\
1.2142	-1.8947\\
1.2247	-1.9076\\
1.2350	-1.9307\\
1.2456	-1.8995\\
1.2563	-1.8695\\
1.2667	-1.8552\\
1.2778	-1.8364\\
1.2885	-1.8148\\
1.2995	-1.7998\\
1.3105	-1.7869\\
1.3216	-1.7775\\
1.3319	-1.8093\\
1.3425	-1.8236\\
1.3528	-1.8376\\
1.3633	-1.8353\\
1.3745	-1.8059\\
1.3852	-1.7869\\
1.3957	-1.7703\\
1.4061	-1.7738\\
1.4173	-1.7628\\
1.4278	-1.7508\\
1.4384	-1.7492\\
1.4486	-1.7290\\
1.4592	-1.7059\\
1.4693	-1.6660\\
1.4803	-1.6090\\
1.4914	-1.5888\\
1.5024	-1.5780\\
1.5132	-1.5963\\
1.5238	-1.5766\\
1.5349	-1.5450\\
1.5457	-1.5468\\
1.5561	-1.5902\\
1.5665	-1.5473\\
1.5768	-1.5149\\
1.5874	-1.4905\\
1.5986	-1.4556\\
1.6090	-1.4345\\
1.6192	-1.4072\\
1.6304	-1.3868\\
1.6414	-1.3818\\
1.6517	-1.3922\\
1.6623	-1.4028\\
1.6728	-1.3920\\
1.6834	-1.3697\\
1.6935	-1.3569\\
1.7041	-1.3371\\
1.7143	-1.3236\\
1.7247	-1.3188\\
1.7351	-1.3155\\
1.7454	-1.3013\\
1.7561	-1.2787\\
1.7662	-1.2570\\
1.7766	-1.2399\\
1.7874	-1.2359\\
1.7978	-1.2262\\
1.8082	-1.2122\\
1.8184	-1.1949\\
1.8291	-1.1743\\
1.8394	-1.1633\\
1.8498	-1.1507\\
1.8599	-1.1345\\
1.8705	-1.1095\\
1.8811	-1.0705\\
1.8913	-1.0432\\
1.9024	-1.0230\\
1.9133	-1.0090\\
1.9237	-1.0086\\
1.9339	-0.9988\\
1.9445	-0.9875\\
1.9551	-0.9792\\
1.9652	-0.9688\\
1.9758	-0.9509\\
1.9863	-0.9280\\
1.9970	-0.8992\\
2.0072	-0.8773\\
2.0183	-0.8621\\
2.0291	-0.8677\\
2.0394	-0.8612\\
2.0500	-0.8507\\
2.0607	-0.8343\\
2.0718	-0.8059\\
2.0823	-0.7989\\
2.0927	-0.7972\\
2.1029	-0.7965\\
2.1137	-0.7937\\
2.1239	-0.7836\\
2.1340	-0.7782\\
2.1446	-0.7562\\
2.1551	-0.7302\\
2.1653	-0.7319\\
2.1756	-0.7228\\
2.1862	-0.7207\\
2.1968	-0.7008\\
2.2074	-0.6772\\
2.2183	-0.6560\\
2.2292	-0.6641\\
2.2396	-0.6678\\
2.2503	-0.6738\\
2.2613	-0.6744\\
2.2716	-0.6616\\
2.2824	-0.6447\\
2.2934	-0.6252\\
2.3041	-0.6322\\
2.3143	-0.6347\\
2.3248	-0.6319\\
2.3355	-0.6258\\
2.3466	-0.6202\\
2.3571	-0.6237\\
2.3674	-0.6272\\
2.3780	-0.6263\\
2.3887	-0.6159\\
2.3997	-0.6016\\
2.4100	-0.5904\\
2.4205	-0.5901\\
2.4312	-0.5890\\
2.4416	-0.5839\\
2.4518	-0.5764\\
2.4624	-0.5666\\
2.4730	-0.5570\\
2.4841	-0.5454\\
2.4945	-0.5292\\
2.5054	-0.5106\\
2.5164	-0.4913\\
2.5270	-0.4986\\
2.5376	-0.4839\\
2.5486	-0.4701\\
2.5591	-0.4567\\
2.5696	-0.4470\\
2.5801	-0.4430\\
2.5907	-0.4383\\
2.6016	-0.4330\\
2.6124	-0.4288\\
2.6226	-0.4197\\
2.6332	-0.4108\\
2.6443	-0.4017\\
2.6551	-0.3933\\
2.6655	-0.3967\\
2.6757	-0.4011\\
2.6862	-0.4056\\
2.6968	-0.4072\\
2.7074	-0.3942\\
2.7185	-0.3788\\
2.7292	-0.3789\\
2.7395	-0.3763\\
2.7501	-0.3717\\
2.7602	-0.3637\\
2.7706	-0.3521\\
2.7813	-0.3392\\
2.7915	-0.3596\\
2.8018	-0.3711\\
2.8123	-0.3589\\
2.8228	-0.3526\\
2.8330	-0.3465\\
2.8433	-0.3391\\
2.8539	-0.3281\\
2.8645	-0.3139\\
2.8748	-0.3031\\
2.8849	-0.2939\\
2.8956	-0.2824\\
2.9064	-0.2642\\
2.9174	-0.2559\\
2.9277	-0.2471\\
2.9383	-0.2548\\
2.9487	-0.2379\\
2.9591	-0.2251\\
2.9696	-0.2149\\
2.9802	-0.2210\\
2.9908	-0.2174\\
3.0018	-0.2168\\
3.0123	-0.2085\\
3.0230	-0.1858\\
3.0343	-0.1756\\
3.0446	-0.1818\\
3.0549	-0.1932\\
3.0654	-0.1819\\
3.0758	-0.1849\\
3.0860	-0.1784\\
3.0966	-0.1779\\
3.1070	-0.1745\\
3.1177	-0.1921\\
3.1281	-0.2082\\
3.1383	-0.2166\\
3.1487	-0.1935\\
3.1590	-0.1807\\
3.1693	-0.1910\\
3.1802	-0.2121\\
3.1905	-0.2139\\
3.2008	-0.2141\\
3.2110	-0.2157\\
3.2217	-0.2210\\
3.2324	-0.2249\\
3.2435	-0.2381\\
3.2540	-0.2510\\
3.2644	-0.2394\\
3.2747	-0.2278\\
3.2848	-0.2169\\
3.2957	-0.2205\\
3.3063	-0.2118\\
3.3165	-0.2078\\
3.3275	-0.1989\\
3.3384	-0.1840\\
3.3490	-0.1574\\
3.3591	-0.1490\\
3.3693	-0.1521\\
3.3801	-0.1507\\
3.3903	-0.1281\\
3.4006	-0.1165\\
3.4111	-0.1185\\
3.4216	-0.1272\\
3.4320	-0.1200\\
3.4422	-0.1284\\
3.4527	-0.1425\\
3.4632	-0.1591\\
3.4737	-0.1499\\
3.4842	-0.1472\\
3.4946	-0.1587\\
3.5053	-0.1777\\
3.5164	-0.1960\\
3.5266	-0.2106\\
3.5374	-0.2229\\
3.5476	-0.2145\\
3.5582	-0.2195\\
3.5690	-0.2196\\
3.5799	-0.2237\\
3.5902	-0.2311\\
3.6008	-0.2406\\
3.6111	-0.2505\\
3.6217	-0.2527\\
3.6320	-0.2378\\
3.6425	-0.2235\\
3.6529	-0.2054\\
3.6635	-0.1877\\
3.6745	-0.1645\\
3.6849	-0.1474\\
3.6956	-0.1356\\
3.7060	-0.1103\\
3.7163	-0.0962\\
3.7265	-0.0789\\
3.7373	-0.0635\\
3.7476	-0.0227\\
3.7580	0.0122\\
3.7686	0.0404\\
3.7791	0.0547\\
3.7894	0.0737\\
3.7997	0.0965\\
3.8102	0.1071\\
3.8210	0.1015\\
3.8321	0.0909\\
3.8431	0.0910\\
3.8540	0.0925\\
3.8645	0.0747\\
3.8750	0.0405\\
3.8861	0.0001\\
3.8967	-0.0321\\
3.9069	-0.0565\\
3.9174	-0.0988\\
3.9276	-0.1442\\
3.9381	-0.1893\\
3.9491	-0.2355\\
3.9599	-0.2924\\
3.9704	-0.3415\\
3.9810	-0.3741\\
3.9912	-0.3960\\
4.0015	-0.4131\\
4.0125	-0.4240\\
4.0227	-0.4241\\
4.0329	-0.4200\\
4.0433	-0.4131\\
4.0541	-0.4028\\
4.0644	-0.3732\\
4.0749	-0.3402\\
4.0852	-0.3047\\
4.0959	-0.2643\\
4.1062	-0.2170\\
4.1172	-0.1674\\
4.1276	-0.1262\\
4.1384	-0.1175\\
4.1488	-0.0756\\
4.1599	-0.0194\\
4.1708	0.0225\\
4.1811	0.0319\\
4.1914	0.0647\\
4.2020	0.1193\\
4.2124	0.1806\\
4.2226	0.2208\\
4.2336	0.2529\\
4.2446	0.2829\\
4.2550	0.3129\\
4.2654	0.3444\\
4.2766	0.3539\\
4.2871	0.3641\\
4.2976	0.3746\\
4.3081	0.3635\\
4.3183	0.3485\\
4.3290	0.3247\\
4.3393	0.2637\\
4.3497	0.2020\\
};
\end{axis}

\begin{axis}[%
width=0.951\fwidth,
height=0.303\fwidth,
at={(0\fwidth,0\fwidth)},
scale only axis,
xmin=-1.0000,
xmax=7.4211,
xlabel={$t$ [s]},
ymin=-0.5000,
ymax=1.5000,
ylabel={$\vartheta_z$.$\hat\vartheta_z$ [1/s]},
axis background/.style={fill=white},
title style={font=\labelsize},
xlabel style={font=\labelsize,at={(axis description cs:0.5,\xlabeldist)}},
ylabel style={font=\labelsize,at={(axis description cs:\ylabeldist,0.5)}},
legend style={font=\ticksize},
ticklabel style={font=\ticksize}
]
\addplot [color=black,dashed,forget plot]
  table[row sep=crcr]{%
0.0000	-0.5000\\
0.0000	1.5000\\
};
\addplot [color=mycolor1,dotted,forget plot]
  table[row sep=crcr]{%
-1.0000	0.5000\\
7.4186	0.5000\\
};
\addplot [color=mycolor1,dashed,forget plot]
  table[row sep=crcr]{%
-1.0000	-0.0004\\
-0.9894	-0.0004\\
-0.9788	0.0002\\
-0.9682	0.0002\\
-0.9580	0.0002\\
-0.9476	0.0002\\
-0.9369	0.0011\\
-0.9265	0.0011\\
-0.9158	0.0011\\
-0.9049	0.0011\\
-0.8945	-0.0004\\
-0.8842	-0.0004\\
-0.8738	-0.0004\\
-0.8633	-0.0004\\
-0.8530	0.0014\\
-0.8425	0.0014\\
-0.8318	0.0014\\
-0.8216	0.0014\\
-0.8108	0.0002\\
-0.7998	0.0002\\
-0.7892	0.0002\\
-0.7788	0.0006\\
-0.7685	0.0006\\
-0.7580	0.0006\\
-0.7476	0.0006\\
-0.7372	0.0011\\
-0.7268	0.0011\\
-0.7165	0.0011\\
-0.7060	0.0011\\
-0.6954	0.0005\\
-0.6850	0.0005\\
-0.6746	0.0005\\
-0.6644	0.0005\\
-0.6535	0.0024\\
-0.6431	0.0024\\
-0.6321	0.0024\\
-0.6219	0.0024\\
-0.6109	0.0003\\
-0.6005	0.0003\\
-0.5895	0.0003\\
-0.5788	0.0039\\
-0.5681	0.0039\\
-0.5579	0.0039\\
-0.5476	0.0039\\
-0.5371	0.0030\\
-0.5267	0.0030\\
-0.5161	0.0030\\
-0.5050	0.0030\\
-0.4943	0.0042\\
-0.4839	0.0042\\
-0.4728	0.0042\\
-0.4623	0.0043\\
-0.4517	0.0043\\
-0.4412	0.0043\\
-0.4310	0.0043\\
-0.4205	0.0040\\
-0.4102	0.0040\\
-0.3992	0.0040\\
-0.3890	0.0040\\
-0.3787	0.0042\\
-0.3683	0.0042\\
-0.3580	0.0042\\
-0.3478	0.0042\\
-0.3371	0.0046\\
-0.3266	0.0046\\
-0.3163	0.0046\\
-0.3061	0.0046\\
-0.2955	0.0040\\
-0.2851	0.0040\\
-0.2745	0.0040\\
-0.2639	0.0040\\
-0.2528	0.0064\\
-0.2423	0.0064\\
-0.2318	0.0064\\
-0.2217	0.0064\\
-0.2112	0.0058\\
-0.2008	0.0058\\
-0.1906	0.0058\\
-0.1799	0.0058\\
-0.1694	0.0061\\
-0.1592	0.0061\\
-0.1490	0.0061\\
-0.1382	0.0061\\
-0.1277	0.0058\\
-0.1165	0.0058\\
-0.1058	0.0058\\
-0.0955	0.0061\\
-0.0851	0.0061\\
-0.0748	0.0061\\
-0.0643	0.0061\\
-0.0541	0.0070\\
-0.0433	0.0070\\
-0.0331	0.0070\\
-0.0228	0.0070\\
-0.0122	0.0064\\
-0.0018	0.0064\\
0.0085	0.0064\\
0.0188	0.0064\\
0.0295	0.0146\\
0.0399	0.0146\\
0.0501	0.0146\\
0.0606	0.0146\\
0.0714	0.0311\\
0.0816	0.0311\\
0.0920	0.0311\\
0.1022	0.0311\\
0.1131	0.0528\\
0.1235	0.0528\\
0.1345	0.0528\\
0.1449	0.0528\\
0.1558	0.0762\\
0.1668	0.0762\\
0.1772	0.0763\\
0.1880	0.1046\\
0.1991	0.1047\\
0.2096	0.1048\\
0.2203	0.1049\\
0.2314	0.1180\\
0.2420	0.1181\\
0.2522	0.1183\\
0.2631	0.1492\\
0.2741	0.1494\\
0.2845	0.1495\\
0.2950	0.1498\\
0.3054	0.1604\\
0.3160	0.1607\\
0.3264	0.1609\\
0.3365	0.1611\\
0.3474	0.1816\\
0.3578	0.1819\\
0.3680	0.1822\\
0.3784	0.2147\\
0.3892	0.2151\\
0.4002	0.2154\\
0.4106	0.2158\\
0.4212	0.2182\\
0.4314	0.2186\\
0.4418	0.2190\\
0.4522	0.2194\\
0.4629	0.2347\\
0.4731	0.2351\\
0.4834	0.2357\\
0.4939	0.2361\\
0.5046	0.2510\\
0.5148	0.2515\\
0.5252	0.2521\\
0.5354	0.2526\\
0.5461	0.2651\\
0.5568	0.2656\\
0.5677	0.2662\\
0.5783	0.2667\\
0.5890	0.2746\\
0.5994	0.2751\\
0.6100	0.2758\\
0.6201	0.2764\\
0.6304	0.2881\\
0.6410	0.2888\\
0.6512	0.2896\\
0.6617	0.2902\\
0.6721	0.2961\\
0.6825	0.2969\\
0.6930	0.2976\\
0.7033	0.2983\\
0.7138	0.3108\\
0.7244	0.3116\\
0.7345	0.3125\\
0.7449	0.3132\\
0.7555	0.3243\\
0.7659	0.3251\\
0.7763	0.3260\\
0.7867	0.3269\\
0.7972	0.3319\\
0.8080	0.3327\\
0.8189	0.3336\\
0.8298	0.3608\\
0.8408	0.3618\\
0.8510	0.3627\\
0.8611	0.3638\\
0.8719	0.3466\\
0.8829	0.3476\\
0.8939	0.3486\\
0.9043	0.3739\\
0.9149	0.3750\\
0.9254	0.3761\\
0.9356	0.3772\\
0.9461	0.3630\\
0.9565	0.3641\\
0.9668	0.3651\\
0.9773	0.3662\\
0.9881	0.3705\\
0.9991	0.3716\\
1.0095	0.3727\\
1.0200	0.3738\\
1.0306	0.3784\\
1.0416	0.3795\\
1.0523	0.3807\\
1.0627	0.4171\\
1.0731	0.4184\\
1.0834	0.4198\\
1.0938	0.4209\\
1.1045	0.3973\\
1.1149	0.3985\\
1.1252	0.3998\\
1.1354	0.4010\\
1.1463	0.4041\\
1.1574	0.4053\\
1.1679	0.4067\\
1.1783	0.4358\\
1.1891	0.4373\\
1.1992	0.4387\\
1.2103	0.4402\\
1.2211	0.4187\\
1.2314	0.4201\\
1.2418	0.4216\\
1.2521	0.4229\\
1.2626	0.4107\\
1.2734	0.4121\\
1.2844	0.4136\\
1.2949	0.4148\\
1.3055	0.4177\\
1.3160	0.4189\\
1.3264	0.4203\\
1.3365	0.4216\\
1.3474	0.4208\\
1.3576	0.4219\\
1.3682	0.4234\\
1.3782	0.4248\\
1.3888	0.4301\\
1.3991	0.4311\\
1.4096	0.4327\\
1.4199	0.4342\\
1.4305	0.4300\\
1.4412	0.4318\\
1.4522	0.4333\\
1.4628	0.4765\\
1.4730	0.4787\\
1.4835	0.4804\\
1.4942	0.4820\\
1.5046	0.4455\\
1.5149	0.4474\\
1.5253	0.4487\\
1.5354	0.4506\\
1.5460	0.4414\\
1.5565	0.4433\\
1.5669	0.4447\\
1.5771	0.4465\\
1.5881	0.4484\\
1.5991	0.4504\\
1.6094	0.4518\\
1.6202	0.4534\\
1.6305	0.4479\\
1.6415	0.4494\\
1.6522	0.4510\\
1.6630	0.4710\\
1.6736	0.4728\\
1.6843	0.4743\\
1.6949	0.4760\\
1.7054	0.4455\\
1.7157	0.4473\\
1.7262	0.4488\\
1.7366	0.4503\\
1.7475	0.4477\\
1.7587	0.4496\\
1.7698	0.4512\\
1.7806	0.4775\\
1.7911	0.4790\\
1.8013	0.4808\\
1.8114	0.4826\\
1.8222	0.4523\\
1.8324	0.4540\\
1.8430	0.4553\\
1.8531	0.4571\\
1.8642	0.4596\\
1.8752	0.4616\\
1.8857	0.4632\\
1.8959	0.4928\\
1.9065	0.4943\\
1.9171	0.4960\\
1.9280	0.4983\\
1.9389	0.4653\\
1.9491	0.4673\\
1.9595	0.4686\\
1.9700	0.4706\\
1.9805	0.4675\\
1.9912	0.4699\\
2.0015	0.4715\\
2.0118	0.4734\\
2.0223	0.4787\\
2.0333	0.4806\\
2.0440	0.4827\\
2.0543	0.5167\\
2.0650	0.5186\\
2.0753	0.5204\\
2.0857	0.5223\\
2.0962	0.4795\\
2.1067	0.4818\\
2.1173	0.4834\\
2.1282	0.4849\\
2.1391	0.4880\\
2.1495	0.4896\\
2.1605	0.4917\\
2.1713	0.5175\\
2.1814	0.5199\\
2.1918	0.5218\\
2.2023	0.5236\\
2.2130	0.4998\\
2.2233	0.5017\\
2.2335	0.5039\\
2.2442	0.5062\\
2.2544	0.5021\\
2.2648	0.5043\\
2.2754	0.5063\\
2.2858	0.5086\\
2.2961	0.5013\\
2.3066	0.5033\\
2.3169	0.5054\\
2.3271	0.5074\\
2.3379	0.5014\\
2.3480	0.5026\\
2.3585	0.5048\\
2.3694	0.5069\\
2.3805	0.4952\\
2.3910	0.4970\\
2.4020	0.4987\\
2.4131	0.5264\\
2.4232	0.5281\\
2.4335	0.5305\\
2.4436	0.5323\\
2.4544	0.4877\\
2.4653	0.4898\\
2.4761	0.4913\\
2.4865	0.4932\\
2.4973	0.4792\\
2.5074	0.4810\\
2.5176	0.4829\\
2.5281	0.4846\\
2.5390	0.4804\\
2.5501	0.4823\\
2.5606	0.4837\\
2.5711	0.5141\\
2.5815	0.5155\\
2.5919	0.5180\\
2.6029	0.5197\\
2.6136	0.4831\\
2.6241	0.4849\\
2.6347	0.4865\\
2.6450	0.4892\\
2.6555	0.4946\\
2.6657	0.4969\\
2.6763	0.4985\\
2.6869	0.5006\\
2.6972	0.5013\\
2.7083	0.5036\\
2.7189	0.5057\\
2.7294	0.5517\\
2.7399	0.5538\\
2.7502	0.5564\\
2.7605	0.5584\\
2.7711	0.5165\\
2.7814	0.5186\\
2.7917	0.5208\\
2.8025	0.5226\\
2.8128	0.5319\\
2.8230	0.5337\\
2.8336	0.5363\\
2.8447	0.5387\\
2.8554	0.5288\\
2.8658	0.5313\\
2.8766	0.5328\\
2.8877	0.5687\\
2.8984	0.5710\\
2.9085	0.5735\\
2.9186	0.5763\\
2.9294	0.5306\\
2.9400	0.5331\\
2.9505	0.5352\\
2.9614	0.5373\\
2.9723	0.5509\\
2.9834	0.5534\\
2.9939	0.5559\\
3.0046	0.5856\\
3.0158	0.5886\\
3.0262	0.5908\\
3.0365	0.5932\\
3.0471	0.5516\\
3.0577	0.5539\\
3.0688	0.5563\\
3.0795	0.5641\\
3.0897	0.5666\\
3.1001	0.5692\\
3.1106	0.5713\\
3.1212	0.5440\\
3.1314	0.5466\\
3.1416	0.5487\\
3.1521	0.5508\\
3.1629	0.5370\\
3.1732	0.5393\\
3.1844	0.5420\\
3.1947	0.5442\\
3.2057	0.5254\\
3.2159	0.5276\\
3.2262	0.5295\\
3.2364	0.5317\\
3.2472	0.5188\\
3.2576	0.5209\\
3.2678	0.5233\\
3.2781	0.5256\\
3.2890	0.5193\\
3.2999	0.5218\\
3.3105	0.5233\\
3.3209	0.5408\\
3.3319	0.5432\\
3.3427	0.5449\\
3.3530	0.5474\\
3.3637	0.5029\\
3.3741	0.5044\\
3.3848	0.5067\\
3.3949	0.5091\\
3.4055	0.5073\\
3.4166	0.5090\\
3.4272	0.5110\\
3.4378	0.5405\\
3.4480	0.5420\\
3.4585	0.5443\\
3.4688	0.5461\\
3.4796	0.4977\\
3.4899	0.4995\\
3.5002	0.5018\\
3.5103	0.5034\\
3.5211	0.4980\\
3.5314	0.5004\\
3.5419	0.5023\\
3.5521	0.5042\\
3.5622	0.4837\\
3.5732	0.4857\\
3.5843	0.4878\\
3.5950	0.4902\\
3.6053	0.4866\\
3.6158	0.4886\\
3.6260	0.4905\\
3.6366	0.4923\\
3.6471	0.4802\\
3.6574	0.4824\\
3.6677	0.4837\\
3.6779	0.4858\\
3.6889	0.4540\\
3.6992	0.4557\\
3.7096	0.4570\\
3.7199	0.4586\\
3.7306	0.4569\\
3.7409	0.4582\\
3.7511	0.4600\\
3.7616	0.4617\\
3.7723	0.4650\\
3.7825	0.4666\\
3.7929	0.4684\\
3.8031	0.4701\\
3.8137	0.4731\\
3.8244	0.4750\\
3.8353	0.4767\\
3.8461	0.5168\\
3.8564	0.5189\\
3.8669	0.5212\\
3.8777	0.5226\\
3.8878	0.4888\\
3.8983	0.4906\\
3.9088	0.4929\\
3.9197	0.4949\\
3.9302	0.5173\\
3.9407	0.5193\\
3.9511	0.5215\\
3.9619	0.5669\\
3.9723	0.5696\\
3.9825	0.5717\\
3.9927	0.5736\\
4.0035	0.4214\\
4.0137	0.4251\\
4.0242	0.4266\\
4.0344	0.5536\\
4.0444	0.5561\\
4.0553	0.5608\\
4.0658	0.5637\\
4.0763	0.5656\\
4.0865	0.5686\\
4.0970	0.5562\\
4.1075	0.5593\\
4.1177	0.5621\\
4.1282	0.5650\\
4.1389	0.5497\\
4.1491	0.5526\\
4.1593	0.5547\\
4.1699	0.5571\\
4.1806	0.5494\\
4.1917	0.5517\\
4.2022	0.5544\\
4.2128	0.5576\\
4.2234	0.5599\\
4.2344	0.5621\\
4.2450	0.5654\\
4.2559	0.5213\\
4.2669	0.5237\\
4.2772	0.5260\\
4.2877	0.5285\\
4.2985	0.5307\\
4.3096	0.5324\\
4.3198	0.5337\\
4.3303	0.4860\\
4.3408	0.4883\\
4.3513	0.4897\\
4.3618	0.5231\\
4.3721	0.5243\\
4.3827	0.5272\\
4.3929	0.5287\\
4.4033	0.3946\\
4.4136	0.3979\\
4.4240	0.3987\\
4.4346	0.5347\\
4.4451	0.5362\\
4.4555	0.5410\\
4.4658	0.5665\\
4.4761	0.5681\\
4.4865	0.5715\\
4.4973	0.6130\\
4.5076	0.6153\\
4.5178	0.6170\\
4.5281	0.6207\\
4.5387	0.5926\\
4.5491	0.5947\\
4.5594	0.5976\\
4.5699	0.6008\\
4.5803	0.5968\\
4.5907	0.6018\\
4.6012	0.6032\\
4.6118	0.6066\\
4.6221	0.5872\\
4.6329	0.5893\\
4.6439	0.5920\\
4.6546	0.6036\\
4.6648	0.6057\\
4.6750	0.6097\\
4.6853	0.6111\\
4.6963	0.5145\\
4.7073	0.5152\\
4.7178	0.5185\\
4.7282	0.5202\\
4.7387	0.4748\\
4.7493	0.4763\\
4.7600	0.4785\\
4.7711	0.5073\\
4.7815	0.5105\\
4.7920	0.5125\\
4.8021	0.5125\\
4.8125	0.4716\\
4.8233	0.4727\\
4.8335	0.4759\\
4.8436	0.4763\\
4.8546	0.4821\\
4.8650	0.4836\\
4.8751	0.4858\\
4.8858	0.4877\\
4.8961	0.4988\\
4.9066	0.5011\\
4.9169	0.5035\\
4.9272	0.5059\\
4.9378	0.4886\\
4.9481	0.4893\\
4.9583	0.4922\\
4.9690	0.4937\\
4.9795	0.4695\\
4.9898	0.4716\\
5.0001	0.4721\\
5.0109	0.4749\\
5.0212	0.4483\\
5.0315	0.4491\\
5.0428	0.4517\\
5.0529	0.4521\\
5.0638	0.3926\\
5.0741	0.3934\\
5.0845	0.3931\\
5.0949	0.3955\\
5.1055	0.3678\\
5.1158	0.3709\\
5.1263	0.3706\\
5.1367	0.3711\\
5.1471	0.3765\\
5.1574	0.3773\\
5.1676	0.3798\\
5.1781	0.3808\\
5.1888	0.3944\\
5.1992	0.3941\\
5.2094	0.3969\\
5.2201	0.3977\\
5.2307	0.4065\\
5.2418	0.4064\\
5.2520	0.4079\\
5.2628	0.4439\\
5.2733	0.4450\\
5.2836	0.4483\\
5.2946	0.4486\\
5.3053	0.4465\\
5.3159	0.4475\\
5.3261	0.4486\\
5.3363	0.4510\\
5.3474	0.4668\\
5.3584	0.4687\\
5.3687	0.4701\\
5.3796	0.4665\\
5.3905	0.4689\\
5.4012	0.4705\\
5.4116	0.4730\\
5.4221	0.4512\\
5.4323	0.4541\\
5.4428	0.4539\\
5.4532	0.4568\\
5.4640	0.4343\\
5.4743	0.4359\\
5.4846	0.4378\\
5.4952	0.4375\\
5.5055	0.3642\\
5.5159	0.3644\\
5.5269	0.3660\\
5.5377	0.3938\\
5.5488	0.3950\\
5.5596	0.3964\\
5.5698	0.3967\\
5.5805	0.2700\\
5.5906	0.2684\\
5.6013	0.2706\\
5.6116	0.2713\\
5.6220	0.2763\\
5.6325	0.2779\\
5.6428	0.2790\\
5.6531	0.2808\\
5.6636	0.3311\\
5.6743	0.3320\\
5.6850	0.3331\\
5.6960	0.4458\\
5.7063	0.4481\\
5.7167	0.4493\\
5.7272	0.4509\\
5.7379	0.4831\\
5.7481	0.4852\\
5.7584	0.4863\\
5.7689	0.4893\\
5.7794	0.5616\\
5.7900	0.5639\\
5.8009	0.5666\\
5.8112	0.5692\\
5.8220	0.5701\\
5.8322	0.5704\\
5.8428	0.5730\\
5.8531	0.5738\\
5.8638	0.5000\\
5.8741	0.5007\\
5.8843	0.5012\\
5.8947	0.5030\\
5.9053	0.3959\\
5.9157	0.3967\\
5.9261	0.3967\\
5.9366	0.3967\\
5.9470	0.3030\\
5.9572	0.3022\\
5.9680	0.3037\\
5.9781	0.3054\\
5.9888	0.2675\\
5.9990	0.2689\\
6.0093	0.2685\\
6.0199	0.2713\\
6.0302	0.2976\\
6.0409	0.3002\\
6.0512	0.3014\\
6.0616	0.3318\\
6.0720	0.3325\\
6.0822	0.3361\\
6.0928	0.3364\\
6.1032	0.2927\\
6.1136	0.2933\\
6.1238	0.2948\\
6.1343	0.4430\\
6.1447	0.4439\\
6.1555	0.4464\\
6.1658	0.4389\\
6.1760	0.4385\\
6.1865	0.4377\\
6.1971	0.4449\\
6.2074	0.4460\\
6.2176	0.4474\\
6.2282	0.4490\\
6.2389	0.3857\\
6.2490	0.3870\\
6.2595	0.3896\\
6.2698	0.3922\\
6.2808	0.3604\\
6.2917	0.3637\\
6.3021	0.3627\\
6.3127	0.2819\\
6.3231	0.2818\\
6.3342	0.2824\\
6.3449	0.2815\\
6.3554	0.2417\\
6.3657	0.2405\\
6.3760	0.2404\\
6.3866	0.2399\\
6.3972	0.2401\\
6.4089	0.2396\\
6.4198	0.2405\\
6.4303	0.3953\\
6.4406	0.3981\\
6.4514	0.3983\\
6.4617	0.4550\\
6.4723	0.4555\\
6.4832	0.4581\\
6.4937	0.4608\\
6.5044	0.5146\\
6.5150	0.5165\\
6.5258	0.5192\\
6.5364	0.5211\\
6.5470	0.5072\\
6.5573	0.5090\\
6.5677	0.5111\\
6.5782	0.5134\\
6.5887	0.3647\\
6.5990	0.3653\\
6.6096	0.3669\\
6.6197	0.3663\\
6.6304	0.1524\\
6.6410	0.1522\\
6.6521	0.1528\\
6.6629	0.1236\\
6.6740	0.1243\\
6.6846	0.1244\\
6.6952	0.0279\\
6.7063	0.0286\\
6.7168	0.0285\\
6.7271	0.0294\\
6.7377	0.0832\\
6.7481	0.0834\\
6.7583	0.0843\\
6.7692	0.0842\\
6.7793	0.2517\\
6.7897	0.2526\\
6.8000	0.2546\\
6.8102	0.2558\\
6.8211	0.4353\\
6.8313	0.4379\\
6.8419	0.4394\\
6.8520	0.4416\\
6.8628	0.6207\\
6.8733	0.6230\\
6.8843	0.6258\\
6.8948	0.6282\\
6.9053	0.6779\\
6.9157	0.6810\\
6.9261	0.6841\\
6.9367	0.6876\\
6.9470	0.4393\\
6.9573	0.4377\\
6.9678	0.4388\\
6.9781	0.4363\\
6.9891	0.0311\\
7.0001	0.0303\\
7.0105	0.0305\\
7.0213	-0.3638\\
7.0315	-0.3625\\
7.0418	-0.3620\\
7.0520	-0.3606\\
7.0628	-0.3908\\
7.0730	-0.3925\\
7.0833	-0.3907\\
7.0938	-0.3858\\
7.1042	-0.3705\\
7.1146	-0.3731\\
7.1250	-0.3702\\
7.1353	-0.3685\\
7.1463	0.0505\\
7.1571	0.0497\\
7.1677	0.0525\\
7.1785	0.6248\\
7.1887	0.6349\\
7.1990	0.6381\\
7.2093	0.6416\\
7.2198	0.8537\\
7.2303	0.8745\\
7.2408	0.8892\\
7.2514	1.8000\\
7.2616	1.8227\\
7.2719	1.8750\\
7.2824	1.9502\\
7.2928	1.9828\\
7.3030	2.0072\\
7.3135	1.6782\\
7.3240	1.6851\\
7.3345	1.6993\\
7.3447	1.7095\\
7.3553	0.7188\\
7.3659	0.7186\\
7.3762	0.7201\\
7.3865	0.7125\\
7.3968	-0.5925\\
7.4076	-0.5882\\
7.4186	-0.5795\\
};
\addplot [color=mycolor1,solid,forget plot]
  table[row sep=crcr]{%
-1.0000	0.0243\\
-0.9902	0.0243\\
-0.9785	0.0243\\
-0.9688	0.0243\\
-0.9590	0.0243\\
-0.9472	0.0243\\
-0.9375	0.0243\\
-0.9258	0.0243\\
-0.9160	0.0243\\
-0.9043	0.0243\\
-0.8945	0.0243\\
-0.8848	0.0243\\
-0.8731	0.0243\\
-0.8640	0.0243\\
-0.8535	0.0243\\
-0.8418	0.0243\\
-0.8321	0.0243\\
-0.8223	0.0243\\
-0.8106	0.0243\\
-0.8008	0.0243\\
-0.7891	0.0243\\
-0.7775	0.0243\\
-0.7695	0.0243\\
-0.7578	0.0243\\
-0.7481	0.0243\\
-0.7383	0.0243\\
-0.7266	0.0243\\
-0.7168	0.0243\\
-0.7071	0.0243\\
-0.6953	0.0243\\
-0.6856	0.0243\\
-0.6738	0.0243\\
-0.6641	0.0243\\
-0.6543	0.0243\\
-0.6426	0.0243\\
-0.6329	0.0243\\
-0.6211	0.0243\\
-0.6114	0.0243\\
-0.6016	0.0243\\
-0.5899	0.0243\\
-0.5782	0.0243\\
-0.5684	0.0243\\
-0.5586	0.0243\\
-0.5469	0.0243\\
-0.5371	0.0243\\
-0.5274	0.0243\\
-0.5156	0.0243\\
-0.5060	0.0243\\
-0.4942	0.0243\\
-0.4844	0.0243\\
-0.4727	0.0243\\
-0.4629	0.0243\\
-0.4511	0.0243\\
-0.4415	0.0243\\
-0.4317	0.0243\\
-0.4200	0.0243\\
-0.4102	0.0243\\
-0.3985	0.0243\\
-0.3887	0.0243\\
-0.3789	0.0243\\
-0.3692	0.0243\\
-0.3575	0.0243\\
-0.3477	0.0243\\
-0.3379	0.0243\\
-0.3262	0.0243\\
-0.3164	0.0243\\
-0.3067	0.0243\\
-0.2957	0.0243\\
-0.2852	0.0243\\
-0.2755	0.0243\\
-0.2637	0.0243\\
-0.2520	0.0243\\
-0.2422	0.0243\\
-0.2326	0.0243\\
-0.2227	0.0243\\
-0.2110	0.0243\\
-0.2012	0.0243\\
-0.1915	0.0243\\
-0.1797	0.0243\\
-0.1680	0.0243\\
-0.1602	0.0243\\
-0.1485	0.0243\\
-0.1388	0.0243\\
-0.1270	0.0243\\
-0.1173	0.0243\\
-0.1055	0.0243\\
-0.0958	0.0243\\
-0.0860	0.0243\\
-0.0743	0.0243\\
-0.0645	0.0243\\
-0.0547	0.0243\\
-0.0430	0.0243\\
-0.0332	0.0243\\
-0.0235	0.0243\\
-0.0118	0.0243\\
-0.0020	0.0503\\
0.0078	0.0503\\
0.0188	0.0503\\
0.0292	0.0508\\
0.0390	0.0508\\
0.0507	0.0508\\
0.0605	0.0508\\
0.0702	0.0508\\
0.0820	0.0508\\
0.0918	0.0508\\
0.1033	0.0510\\
0.1132	0.0511\\
0.1230	0.0511\\
0.1347	0.0553\\
0.1438	0.0554\\
0.1562	0.0554\\
0.1659	0.0552\\
0.1777	0.0540\\
0.1894	0.0540\\
0.1991	0.0584\\
0.2089	0.0533\\
0.2206	0.0533\\
0.2304	0.0384\\
0.2422	0.0337\\
0.2519	0.0480\\
0.2636	0.0612\\
0.2734	0.0815\\
0.2857	0.0902\\
0.2949	0.0902\\
0.3051	0.0840\\
0.3157	0.1128\\
0.3261	0.1191\\
0.3359	0.1194\\
0.3476	0.1194\\
0.3574	0.1429\\
0.3671	0.1539\\
0.3782	0.1686\\
0.3885	0.1707\\
0.4003	0.1677\\
0.4101	0.1697\\
0.4221	0.1780\\
0.4317	0.1891\\
0.4414	0.1900\\
0.4511	0.2004\\
0.4630	0.1958\\
0.4727	0.2610\\
0.4843	0.3004\\
0.4934	0.3107\\
0.5068	0.3238\\
0.5164	0.3208\\
0.5270	0.2951\\
0.5351	0.2951\\
0.5468	0.2748\\
0.5566	0.2574\\
0.5676	0.2586\\
0.5781	0.2571\\
0.5878	0.2616\\
0.5995	0.2699\\
0.6092	0.2693\\
0.6191	0.2649\\
0.6307	0.2768\\
0.6424	0.2892\\
0.6503	0.2892\\
0.6622	0.2771\\
0.6719	0.2646\\
0.6816	0.2548\\
0.6934	0.2576\\
0.7029	0.2884\\
0.7163	0.3302\\
0.7246	0.3302\\
0.7344	0.3505\\
0.7442	0.3605\\
0.7558	0.3609\\
0.7649	0.3600\\
0.7772	0.3421\\
0.7890	0.3499\\
0.7968	0.3499\\
0.8085	0.3550\\
0.8183	0.3531\\
0.8300	0.3845\\
0.8398	0.3494\\
0.8516	0.3281\\
0.8613	0.3120\\
0.8740	0.3151\\
0.8877	0.3340\\
0.8946	0.3340\\
0.9033	0.3276\\
0.9160	0.3540\\
0.9257	0.3647\\
0.9359	0.3338\\
0.9475	0.3386\\
0.9588	0.3355\\
0.9668	0.3355\\
0.9765	0.3357\\
0.9875	0.3483\\
0.9980	0.3227\\
1.0097	0.3097\\
1.0193	0.3194\\
1.0320	0.3276\\
1.0434	0.3521\\
1.0567	0.3575\\
1.0624	0.3575\\
1.0723	0.3687\\
1.0840	0.3783\\
1.0939	0.4010\\
1.1037	0.4048\\
1.1183	0.3854\\
1.1249	0.3854\\
1.1347	0.4172\\
1.1458	0.4150\\
1.1582	0.4090\\
1.1679	0.4056\\
1.1784	0.4347\\
1.1895	0.4596\\
1.2013	0.5186\\
1.2127	0.5282\\
1.2207	0.5282\\
1.2304	0.5175\\
1.2424	0.4453\\
1.2525	0.4350\\
1.2673	0.4290\\
1.2735	0.4290\\
1.2833	0.3941\\
1.2950	0.3790\\
1.3048	0.3877\\
1.3179	0.3645\\
1.3261	0.3645\\
1.3359	0.3930\\
1.3476	0.3781\\
1.3574	0.3540\\
1.3691	0.3491\\
1.3785	0.3573\\
1.3907	0.3979\\
1.3984	0.3979\\
1.4101	0.4268\\
1.4199	0.4279\\
1.4297	0.4255\\
1.4407	0.4628\\
1.4513	0.4743\\
1.4632	0.4586\\
1.4776	0.4315\\
1.4825	0.4315\\
1.4941	0.4646\\
1.5040	0.5059\\
1.5149	0.4878\\
1.5307	0.5150\\
1.5352	0.5150\\
1.5473	0.5171\\
1.5597	0.4963\\
1.5664	0.4963\\
1.5761	0.5198\\
1.5872	0.5122\\
1.5996	0.5021\\
1.6103	0.5015\\
1.6223	0.4703\\
1.6358	0.4728\\
1.6425	0.4728\\
1.6524	0.4583\\
1.6621	0.4412\\
1.6739	0.4585\\
1.6855	0.4692\\
1.6968	0.4852\\
1.7051	0.4852\\
1.7149	0.4801\\
1.7258	0.4700\\
1.7363	0.4490\\
1.7476	0.4253\\
1.7602	0.4136\\
1.7719	0.4075\\
1.7842	0.4004\\
1.7910	0.4004\\
1.8007	0.4246\\
1.8110	0.4558\\
1.8260	0.4550\\
1.8321	0.4550\\
1.8439	0.4700\\
1.8536	0.4707\\
1.8636	0.4449\\
1.8743	0.4213\\
1.8907	0.4386\\
1.8964	0.4386\\
1.9062	0.4534\\
1.9176	0.4044\\
1.9305	0.3857\\
1.9424	0.4222\\
1.9492	0.4222\\
1.9589	0.4510\\
1.9695	0.4500\\
1.9809	0.4038\\
1.9931	0.4379\\
2.0076	0.4890\\
2.0117	0.4890\\
2.0215	0.5189\\
2.0340	0.4838\\
2.0473	0.4693\\
2.0547	0.4693\\
2.0646	0.4585\\
2.0743	0.4704\\
2.0859	0.4431\\
2.0957	0.4540\\
2.1101	0.4637\\
2.1165	0.4637\\
2.1289	0.4197\\
2.1386	0.4123\\
2.1494	0.4386\\
2.1616	0.4508\\
2.1780	0.4272\\
2.1817	0.4272\\
2.1914	0.4804\\
2.2020	0.5019\\
2.2135	0.4583\\
2.2276	0.4237\\
2.2325	0.4237\\
2.2442	0.4681\\
2.2540	0.5404\\
2.2638	0.5258\\
2.2771	0.5341\\
2.2852	0.5341\\
2.2968	0.5481\\
2.3066	0.5162\\
2.3186	0.5271\\
2.3302	0.5354\\
2.3378	0.5354\\
2.3476	0.4941\\
2.3574	0.4747\\
2.3702	0.5150\\
2.3816	0.5250\\
2.3965	0.5163\\
2.4024	0.5163\\
2.4121	0.5438\\
2.4240	0.5586\\
2.4337	0.5507\\
2.4467	0.6094\\
2.4552	0.6094\\
2.4649	0.5802\\
2.4765	0.5815\\
2.4863	0.5516\\
2.4973	0.5256\\
2.5105	0.5065\\
2.5175	0.5065\\
2.5273	0.4794\\
2.5382	0.4493\\
2.5502	0.4133\\
2.5637	0.4040\\
2.5703	0.4040\\
2.5820	0.4930\\
2.5908	0.4648\\
2.6030	0.4387\\
2.6192	0.4257\\
2.6232	0.4257\\
2.6347	0.4816\\
2.6447	0.4919\\
2.6579	0.4787\\
2.6660	0.4787\\
2.6758	0.4334\\
2.6875	0.4564\\
2.6970	0.4483\\
2.7097	0.4396\\
2.7212	0.4528\\
2.7285	0.4528\\
2.7402	0.4621\\
2.7500	0.4814\\
2.7605	0.4461\\
2.7722	0.4697\\
2.7869	0.5020\\
2.7910	0.5020\\
2.8028	0.5284\\
2.8126	0.5411\\
2.8228	0.5380\\
2.8378	0.5101\\
2.8457	0.5101\\
2.8554	0.4829\\
2.8652	0.4785\\
2.8768	0.4826\\
2.8895	0.4717\\
2.9011	0.4474\\
2.9082	0.4474\\
2.9180	0.4241\\
2.9296	0.4157\\
2.9425	0.4304\\
2.9540	0.4908\\
2.9609	0.4908\\
2.9728	0.6170\\
2.9825	0.6641\\
2.9934	0.6515\\
3.0075	0.5581\\
3.0156	0.5581\\
3.0254	0.5318\\
3.0371	0.6065\\
3.0471	0.5180\\
3.0593	0.4951\\
3.0710	0.4971\\
3.0824	0.5071\\
3.0898	0.5071\\
3.0996	0.5364\\
3.1102	0.5636\\
3.1217	0.5782\\
3.1336	0.5598\\
3.1406	0.5598\\
3.1525	0.5893\\
3.1622	0.5636\\
3.1730	0.5667\\
3.1877	0.5644\\
3.1953	0.5644\\
3.2051	0.5148\\
3.2149	0.5136\\
3.2264	0.5476\\
3.2384	0.5625\\
3.2461	0.5625\\
3.2578	0.5654\\
3.2675	0.5790\\
3.2773	0.5446\\
3.2882	0.5320\\
3.2989	0.5607\\
3.3112	0.5893\\
3.3233	0.5670\\
3.3380	0.5194\\
3.3439	0.5194\\
3.3536	0.4786\\
3.3635	0.4632\\
3.3771	0.4459\\
3.3841	0.4459\\
3.3946	0.4909\\
3.4044	0.5223\\
3.4160	0.5033\\
3.4274	0.5848\\
3.4402	0.5769\\
3.4471	0.5769\\
3.4590	0.5894\\
3.4686	0.5904\\
3.4797	0.6090\\
3.4912	0.5796\\
3.5028	0.5234\\
3.5097	0.5234\\
3.5214	0.5048\\
3.5312	0.4817\\
3.5419	0.4886\\
3.5560	0.4945\\
3.5618	0.4945\\
3.5724	0.5159\\
3.5841	0.5662\\
3.5957	0.5551\\
3.6046	0.5570\\
3.6187	0.5779\\
3.6250	0.5779\\
3.6367	0.5608\\
3.6464	0.5729\\
3.6582	0.6081\\
3.6707	0.6027\\
3.6778	0.6027\\
3.6894	0.5491\\
3.6992	0.4644\\
3.7096	0.4401\\
3.7215	0.4232\\
3.7336	0.3945\\
3.7402	0.3945\\
3.7500	0.5102\\
3.7618	0.5631\\
3.7722	0.4910\\
3.7876	0.5089\\
3.7931	0.5089\\
3.8028	0.4527\\
3.8128	0.4394\\
3.8265	0.4098\\
3.8389	0.3837\\
3.8457	0.3837\\
3.8555	0.4183\\
3.8672	0.3933\\
3.8770	0.4292\\
3.8889	0.4291\\
3.9012	0.4356\\
3.9082	0.4356\\
3.9200	0.4819\\
3.9297	0.5159\\
3.9396	0.5546\\
3.9521	0.5607\\
3.9633	0.5678\\
3.9791	0.5611\\
3.9825	0.5611\\
3.9920	0.5311\\
4.0034	0.5304\\
4.0197	0.4984\\
4.0235	0.4984\\
4.0351	0.4427\\
4.0502	0.4294\\
4.0548	0.4294\\
4.0669	0.4901\\
4.0793	0.4672\\
4.0859	0.4672\\
4.0969	0.4295\\
4.1074	0.4436\\
4.1172	0.4954\\
4.1285	0.5459\\
4.1409	0.5479\\
4.1484	0.5479\\
4.1582	0.5793\\
4.1699	0.5989\\
4.1803	0.6134\\
4.1922	0.6167\\
4.2068	0.6044\\
4.2130	0.6044\\
4.2247	0.6105\\
4.2345	0.5996\\
4.2444	0.5804\\
4.2578	0.5638\\
4.2701	0.5475\\
4.2773	0.5475\\
4.2890	0.5030\\
4.2988	0.4845\\
4.3087	0.4773\\
4.3210	0.5078\\
4.3326	0.5285\\
4.3398	0.5285\\
4.3515	0.5120\\
4.3612	0.4575\\
4.3724	0.4764\\
4.3895	0.4526\\
4.3926	0.4526\\
4.4022	0.4265\\
4.4136	0.4026\\
4.4318	0.4247\\
4.4337	0.4247\\
4.4479	0.4087\\
4.4551	0.4087\\
4.4648	0.3817\\
4.4765	0.3668\\
4.4912	0.4411\\
4.4962	0.4411\\
4.5079	0.4149\\
4.5184	0.4321\\
4.5298	0.4932\\
4.5423	0.5602\\
4.5488	0.5602\\
4.5594	0.5899\\
4.5716	0.6009\\
4.5800	0.6009\\
4.5898	0.6319\\
4.6023	0.5861\\
4.6185	0.6178\\
4.6211	0.6178\\
4.6320	0.6268\\
4.6472	0.6514\\
4.6544	0.6514\\
4.6642	0.6544\\
4.6742	0.5849\\
4.6902	0.5102\\
4.6954	0.5102\\
4.7070	0.5252\\
4.7193	0.5008\\
4.7312	0.5263\\
4.7382	0.5263\\
4.7500	0.5203\\
4.7596	0.4772\\
4.7707	0.4310\\
4.7822	0.4302\\
4.7984	0.4069\\
4.8028	0.4069\\
4.8120	0.4519\\
4.8292	0.4677\\
4.8341	0.4677\\
4.8439	0.5007\\
4.8535	0.5182\\
4.8677	0.4854\\
4.8751	0.4854\\
4.8867	0.4652\\
4.8964	0.5012\\
4.9106	0.5183\\
4.9160	0.5183\\
4.9274	0.5111\\
4.9408	0.4701\\
4.9472	0.4701\\
4.9579	0.5700\\
4.9698	0.5965\\
4.9816	0.5686\\
4.9934	0.6069\\
5.0000	0.6069\\
5.0117	0.6032\\
5.0205	0.5679\\
5.0322	0.5263\\
5.0480	0.4660\\
5.0527	0.4660\\
5.0626	0.4566\\
5.0743	0.4643\\
5.0836	0.4565\\
5.1008	0.4226\\
5.1055	0.4226\\
5.1172	0.4508\\
5.1278	0.4477\\
5.1399	0.4095\\
5.1466	0.4095\\
5.1562	0.4091\\
5.1667	0.4042\\
5.1801	0.4311\\
5.1925	0.4473\\
5.1993	0.4473\\
5.2084	0.4385\\
5.2218	0.4338\\
5.2372	0.3687\\
5.2423	0.3687\\
5.2520	0.3888\\
5.2628	0.3833\\
5.2783	0.4773\\
5.2833	0.4773\\
5.2949	0.5004\\
5.3079	0.4458\\
5.3166	0.4458\\
5.3261	0.4713\\
5.3359	0.4626\\
5.3485	0.4309\\
5.3606	0.4492\\
5.3691	0.4492\\
5.3788	0.4654\\
5.3899	0.4531\\
5.4016	0.4909\\
5.4135	0.5266\\
5.4219	0.5266\\
5.4337	0.5099\\
5.4480	0.5527\\
5.4532	0.5527\\
5.4630	0.5496\\
5.4747	0.5579\\
5.4846	0.5322\\
5.4990	0.5108\\
5.5052	0.5108\\
5.5157	0.5282\\
5.5273	0.5316\\
5.5399	0.4854\\
5.5515	0.4823\\
5.5585	0.4823\\
5.5703	0.4725\\
5.5796	0.3984\\
5.5930	0.3465\\
5.6082	0.3564\\
5.6113	0.3564\\
5.6210	0.3518\\
5.6329	0.3238\\
5.6473	0.2697\\
5.6525	0.2697\\
5.6642	0.2726\\
5.6740	0.2877\\
5.6881	0.3053\\
5.6954	0.3053\\
5.7052	0.3420\\
5.7168	0.3197\\
5.7313	0.3459\\
5.7383	0.3459\\
5.7480	0.4109\\
5.7609	0.4965\\
5.7726	0.4780\\
5.7792	0.4780\\
5.7890	0.5429\\
5.8007	0.5479\\
5.8126	0.5683\\
5.8227	0.6332\\
5.8350	0.6231\\
5.8418	0.6231\\
5.8535	0.5814\\
5.8632	0.6238\\
5.8799	0.5936\\
5.8848	0.5936\\
5.8947	0.5400\\
5.9056	0.5019\\
5.9226	0.5102\\
5.9251	0.5102\\
5.9381	0.4566\\
5.9501	0.4121\\
5.9570	0.4121\\
5.9687	0.3809\\
5.9785	0.3783\\
5.9888	0.3357\\
6.0010	0.3322\\
6.0125	0.3419\\
6.0195	0.3419\\
6.0293	0.3539\\
6.0403	0.3773\\
6.0520	0.3445\\
6.0674	0.3575\\
6.0724	0.3575\\
6.0820	0.3502\\
6.0972	0.4130\\
6.1035	0.4130\\
6.1133	0.3902\\
6.1274	0.4004\\
6.1348	0.4004\\
6.1445	0.4004\\
6.1564	0.4927\\
6.1650	0.5137\\
6.1815	0.4850\\
6.1855	0.4850\\
6.1986	0.5129\\
6.2070	0.5129\\
6.2169	0.4026\\
6.2285	0.5264\\
6.2391	0.4917\\
6.2512	0.5158\\
6.2667	0.4395\\
6.2695	0.4395\\
6.2812	0.4724\\
6.2910	0.5420\\
6.3014	0.5478\\
6.3168	0.4680\\
6.3244	0.4680\\
6.3342	0.4405\\
6.3439	0.3568\\
6.3572	0.2777\\
6.3653	0.2777\\
6.3751	0.2577\\
6.3867	0.2321\\
6.3965	0.1871\\
6.4107	0.1655\\
6.4225	0.2564\\
6.4297	0.2564\\
6.4414	0.3052\\
6.4512	0.3128\\
6.4618	0.3312\\
6.4738	0.3506\\
6.4882	0.4733\\
6.4943	0.4733\\
6.5041	0.5234\\
6.5156	0.5958\\
6.5279	0.6880\\
6.5352	0.6880\\
6.5469	0.7002\\
6.5566	0.6428\\
6.5670	0.6290\\
6.5792	0.6118\\
6.5905	0.6281\\
6.6025	0.5762\\
6.6094	0.5762\\
6.6191	0.5002\\
6.6314	0.4167\\
6.6472	0.3919\\
6.6525	0.3919\\
6.6622	0.3626\\
6.6740	0.3247\\
6.6850	0.3130\\
6.6996	0.2428\\
6.7052	0.2428\\
6.7168	0.2050\\
6.7265	0.1713\\
6.7367	0.1485\\
6.7473	0.1444\\
6.7573	0.1238\\
6.7690	0.1842\\
6.7808	0.2123\\
6.7891	0.2123\\
6.7988	0.2485\\
6.8105	0.1726\\
6.8205	0.1346\\
6.8322	0.0968\\
6.8476	0.2898\\
6.8515	0.2898\\
6.8626	0.3897\\
6.8732	0.6305\\
6.8838	0.8035\\
6.8988	0.9369\\
6.9044	0.9369\\
6.9160	0.9332\\
6.9258	0.9911\\
6.9386	0.8910\\
6.9497	0.8687\\
6.9570	0.8687\\
6.9667	0.8503\\
6.9785	0.7575\\
6.9902	0.6543\\
7.0003	0.5770\\
7.0110	0.5029\\
7.0202	0.4636\\
7.0306	0.4423\\
7.0424	0.4217\\
7.0565	0.3501\\
7.0625	0.3501\\
7.0724	0.0501\\
7.0841	-0.2499\\
7.0927	-0.2579\\
7.1080	-0.3925\\
7.1152	-0.3925\\
7.1251	-0.4620\\
7.1367	-0.4443\\
7.1465	-0.4065\\
7.1609	-0.3981\\
7.1680	-0.3981\\
7.1797	-0.3981\\
7.1901	-0.3356\\
7.2023	-0.1821\\
7.2089	-0.1821\\
7.2187	-0.0653\\
7.2308	0.2347\\
7.2435	0.5347\\
7.2520	0.5347\\
7.2617	0.8347\\
7.2725	1.1347\\
7.2876	1.4347\\
7.2931	1.4347\\
7.3029	1.7347\\
7.3125	1.9826\\
7.3287	2.1257\\
7.3341	2.1257\\
7.3439	2.0047\\
7.3555	1.8864\\
7.3675	1.7005\\
7.3797	1.5065\\
7.3867	1.5065\\
7.3965	1.2065\\
7.4091	0.9065\\
7.4211	0.6065\\
};
\addplot [color=mycolor2,dotted,forget plot]
  table[row sep=crcr]{%
-1.0000	0.7000\\
7.4186	0.7000\\
};
\addplot [color=mycolor2,dashed,forget plot]
  table[row sep=crcr]{%
-1.0000	-0.0084\\
-0.9894	-0.0084\\
-0.9791	-0.0084\\
-0.9684	-0.0101\\
-0.9582	-0.0101\\
-0.9473	-0.0101\\
-0.9363	-0.0081\\
-0.9258	-0.0081\\
-0.9153	-0.0081\\
-0.9044	-0.0081\\
-0.8935	-0.0072\\
-0.8830	-0.0072\\
-0.8726	-0.0072\\
-0.8623	-0.0072\\
-0.8519	-0.0072\\
-0.8415	-0.0072\\
-0.8305	-0.0072\\
-0.8196	-0.0064\\
-0.8093	-0.0064\\
-0.7988	-0.0064\\
-0.7886	-0.0064\\
-0.7779	-0.0075\\
-0.7675	-0.0075\\
-0.7570	-0.0075\\
-0.7466	-0.0075\\
-0.7362	-0.0064\\
-0.7260	-0.0064\\
-0.7153	-0.0064\\
-0.7051	-0.0064\\
-0.6945	-0.0061\\
-0.6842	-0.0061\\
-0.6737	-0.0061\\
-0.6633	-0.0061\\
-0.6530	-0.0055\\
-0.6426	-0.0055\\
-0.6321	-0.0055\\
-0.6219	-0.0055\\
-0.6111	-0.0049\\
-0.6008	-0.0049\\
-0.5905	-0.0049\\
-0.5800	-0.0049\\
-0.5697	-0.0069\\
-0.5591	-0.0069\\
-0.5490	-0.0069\\
-0.5384	-0.0069\\
-0.5277	-0.0043\\
-0.5173	-0.0043\\
-0.5062	-0.0043\\
-0.4959	-0.0043\\
-0.4851	-0.0055\\
-0.4747	-0.0055\\
-0.4635	-0.0055\\
-0.4529	-0.0052\\
-0.4422	-0.0052\\
-0.4319	-0.0052\\
-0.4217	-0.0052\\
-0.4111	-0.0049\\
-0.4007	-0.0049\\
-0.3905	-0.0049\\
-0.3800	-0.0049\\
-0.3696	-0.0055\\
-0.3590	-0.0055\\
-0.3479	-0.0055\\
-0.3375	-0.0055\\
-0.3267	-0.0043\\
-0.3156	-0.0043\\
-0.3051	-0.0043\\
-0.2945	-0.0063\\
-0.2840	-0.0063\\
-0.2728	-0.0063\\
-0.2625	-0.0063\\
-0.2518	-0.0060\\
-0.2415	-0.0060\\
-0.2313	-0.0060\\
-0.2207	-0.0060\\
-0.2101	-0.0049\\
-0.1993	-0.0049\\
-0.1882	-0.0049\\
-0.1777	-0.0078\\
-0.1675	-0.0078\\
-0.1570	-0.0078\\
-0.1468	-0.0078\\
-0.1362	-0.0052\\
-0.1257	-0.0052\\
-0.1155	-0.0052\\
-0.1051	-0.0052\\
-0.0943	-0.0058\\
-0.0840	-0.0058\\
-0.0738	-0.0058\\
-0.0632	-0.0058\\
-0.0520	-0.0063\\
-0.0412	-0.0063\\
-0.0310	-0.0063\\
-0.0206	-0.0063\\
-0.0103	-0.0055\\
0.0000	-0.0055\\
0.0106	-0.0055\\
0.0212	-0.0055\\
0.0323	0.0058\\
0.0428	0.0058\\
0.0531	0.0058\\
0.0642	0.0360\\
0.0751	0.0360\\
0.0855	0.0361\\
0.0957	0.0361\\
0.1068	0.0623\\
0.1170	0.0624\\
0.1274	0.0624\\
0.1375	0.0625\\
0.1482	0.0982\\
0.1585	0.0983\\
0.1691	0.0984\\
0.1794	0.0985\\
0.1899	0.1355\\
0.2003	0.1357\\
0.2115	0.1358\\
0.2222	0.1785\\
0.2325	0.1788\\
0.2430	0.1790\\
0.2532	0.1793\\
0.2637	0.2037\\
0.2741	0.2041\\
0.2847	0.2045\\
0.2948	0.2048\\
0.3055	0.2387\\
0.3159	0.2392\\
0.3261	0.2397\\
0.3365	0.2403\\
0.3471	0.2710\\
0.3575	0.2717\\
0.3686	0.2724\\
0.3791	0.2730\\
0.3900	0.3060\\
0.4002	0.3069\\
0.4106	0.3076\\
0.4208	0.3085\\
0.4314	0.3360\\
0.4421	0.3371\\
0.4531	0.3380\\
0.4638	0.3866\\
0.4750	0.3878\\
0.4855	0.3889\\
0.4958	0.3901\\
0.5064	0.3829\\
0.5167	0.3841\\
0.5272	0.3853\\
0.5374	0.3866\\
0.5481	0.4087\\
0.5584	0.4102\\
0.5690	0.4115\\
0.5792	0.4129\\
0.5896	0.4334\\
0.6001	0.4349\\
0.6108	0.4365\\
0.6210	0.4381\\
0.6314	0.4522\\
0.6418	0.4539\\
0.6520	0.4556\\
0.6626	0.5013\\
0.6731	0.5035\\
0.6833	0.5052\\
0.6938	0.5073\\
0.7044	0.3845\\
0.7150	0.3873\\
0.7260	0.3890\\
0.7364	0.5357\\
0.7471	0.5400\\
0.7576	0.5423\\
0.7687	0.5462\\
0.7791	0.5483\\
0.7897	0.5528\\
0.8002	0.5639\\
0.8106	0.5664\\
0.8209	0.5688\\
0.8317	0.5699\\
0.8428	0.5723\\
0.8539	0.5747\\
0.8647	0.5918\\
0.8753	0.5945\\
0.8863	0.5969\\
0.8969	0.6031\\
0.9074	0.6059\\
0.9177	0.6087\\
0.9280	0.6115\\
0.9388	0.5835\\
0.9491	0.5866\\
0.9592	0.5892\\
0.9697	0.5922\\
0.9803	0.6098\\
0.9906	0.6130\\
1.0011	0.6158\\
1.0116	0.6193\\
1.0225	0.6186\\
1.0333	0.6223\\
1.0441	0.6252\\
1.0543	0.6287\\
1.0648	0.6291\\
1.0755	0.6328\\
1.0864	0.6358\\
1.0972	0.6820\\
1.1080	0.6858\\
1.1189	0.6897\\
1.1290	0.6924\\
1.1397	0.6545\\
1.1500	0.6576\\
1.1603	0.6609\\
1.1707	0.6646\\
1.1816	0.6590\\
1.1919	0.6626\\
1.2021	0.6658\\
1.2125	0.6696\\
1.2232	0.6728\\
1.2336	0.6768\\
1.2438	0.6802\\
1.2540	0.6838\\
1.2647	0.6863\\
1.2750	0.6901\\
1.2856	0.6942\\
1.2964	0.6978\\
1.3075	0.7008\\
1.3182	0.7045\\
1.3293	0.7085\\
1.3397	0.7467\\
1.3502	0.7509\\
1.3607	0.7554\\
1.3711	0.7597\\
1.3814	0.7160\\
1.3918	0.7201\\
1.4021	0.7236\\
1.4127	0.7279\\
1.4286	0.7173\\
1.4389	0.7256\\
1.4492	0.7297\\
1.4596	0.7687\\
1.4706	0.7731\\
1.4814	0.7821\\
1.4917	0.7681\\
1.5026	0.7724\\
1.5126	0.7767\\
1.5233	0.7672\\
1.5335	0.7721\\
1.5439	0.7765\\
1.5542	0.7811\\
1.5645	0.7253\\
1.5753	0.7294\\
1.5855	0.7339\\
1.5957	0.7380\\
1.6068	0.7248\\
1.6177	0.7290\\
1.6284	0.7332\\
1.6388	0.7796\\
1.6491	0.7845\\
1.6593	0.7890\\
1.6700	0.7932\\
1.6807	0.7453\\
1.6912	0.7495\\
1.7022	0.7539\\
1.7125	0.7583\\
1.7231	0.7419\\
1.7342	0.7462\\
1.7447	0.7504\\
1.7554	0.7890\\
1.7661	0.7930\\
1.7763	0.7979\\
1.7874	0.8029\\
1.7981	0.7358\\
1.8085	0.7404\\
1.8194	0.7444\\
1.8294	0.7759\\
1.8398	0.7809\\
1.8499	0.7856\\
1.8605	0.7899\\
1.8709	0.5898\\
1.8815	0.5969\\
1.8916	0.5999\\
1.9019	0.7753\\
1.9125	0.7800\\
1.9231	0.7885\\
1.9335	0.7873\\
1.9442	0.7925\\
1.9543	0.7969\\
1.9648	0.7870\\
1.9751	0.7911\\
1.9855	0.7960\\
1.9958	0.8004\\
2.0064	0.7430\\
2.0166	0.7470\\
2.0274	0.7514\\
2.0379	0.7567\\
2.0482	0.7473\\
2.0584	0.7520\\
2.0688	0.7568\\
2.0791	0.7607\\
2.0896	0.7506\\
2.1003	0.7552\\
2.1105	0.7594\\
2.1207	0.7640\\
2.1313	0.7466\\
2.1418	0.7509\\
2.1519	0.7558\\
2.1624	0.7599\\
2.1731	0.7510\\
2.1836	0.7554\\
2.1943	0.7602\\
2.2043	0.7644\\
2.2148	0.7535\\
2.2251	0.7580\\
2.2356	0.7629\\
2.2458	0.7671\\
2.2563	0.7522\\
2.2668	0.7571\\
2.2771	0.7614\\
2.2876	0.7657\\
2.2983	0.7492\\
2.3086	0.7531\\
2.3194	0.7577\\
2.3304	0.8039\\
2.3410	0.8087\\
2.3519	0.8139\\
2.3623	0.8180\\
2.3730	0.7522\\
2.3835	0.7574\\
2.3939	0.7615\\
2.4043	0.7662\\
2.4145	0.7537\\
2.4253	0.7580\\
2.4365	0.7632\\
2.4474	0.7886\\
2.4584	0.7940\\
2.4688	0.7984\\
2.4790	0.8029\\
2.4900	0.7293\\
2.5008	0.7338\\
2.5116	0.7378\\
2.5219	0.7474\\
2.5325	0.7526\\
2.5429	0.7558\\
2.5539	0.7597\\
2.5648	0.7099\\
2.5753	0.7140\\
2.5856	0.7178\\
2.5959	0.7206\\
2.6064	0.6978\\
2.6168	0.7017\\
2.6273	0.7051\\
2.6374	0.7079\\
2.6479	0.6814\\
2.6584	0.6845\\
2.6687	0.6880\\
2.6790	0.6927\\
2.6899	0.6757\\
2.7002	0.6796\\
2.7105	0.6837\\
2.7208	0.6884\\
2.7316	0.6739\\
2.7422	0.6781\\
2.7530	0.6824\\
2.7637	0.7293\\
2.7742	0.7340\\
2.7847	0.7385\\
2.7949	0.7429\\
2.8054	0.6752\\
2.8158	0.6791\\
2.8260	0.6831\\
2.8367	0.6868\\
2.8472	0.6959\\
2.8583	0.6989\\
2.8691	0.7020\\
2.8793	0.7062\\
2.8900	0.6999\\
2.9008	0.7020\\
2.9118	0.7057\\
2.9220	0.7582\\
2.9327	0.7621\\
2.9437	0.7652\\
2.9542	0.7695\\
2.9647	0.7066\\
2.9753	0.7097\\
2.9857	0.7132\\
2.9960	0.7163\\
3.0065	0.6805\\
3.0167	0.6843\\
3.0272	0.6879\\
3.0378	0.6917\\
3.0482	0.6836\\
3.0587	0.6871\\
3.0698	0.6926\\
3.0805	0.7322\\
3.0911	0.7360\\
3.1023	0.7414\\
3.1128	0.7459\\
3.1233	0.7141\\
3.1336	0.7176\\
3.1440	0.7220\\
3.1541	0.7270\\
3.1649	0.7278\\
3.1753	0.7332\\
3.1864	0.7370\\
3.1965	0.8260\\
3.2067	0.8306\\
3.2171	0.8353\\
3.2273	0.8411\\
3.2377	0.6031\\
3.2481	0.6112\\
3.2588	0.6147\\
3.2694	0.8062\\
3.2804	0.8156\\
3.2909	0.8204\\
3.3015	0.8329\\
3.3124	0.8379\\
3.3224	0.8474\\
3.3334	0.7839\\
3.3438	0.7886\\
3.3542	0.7931\\
3.3647	0.7604\\
3.3752	0.7647\\
3.3854	0.7696\\
3.3959	0.7525\\
3.4067	0.7572\\
3.4169	0.7601\\
3.4272	0.7650\\
3.4378	0.5501\\
3.4483	0.5559\\
3.4585	0.5593\\
3.4689	0.7331\\
3.4793	0.7380\\
3.4899	0.7457\\
3.5001	0.7070\\
3.5106	0.7100\\
3.5209	0.7147\\
3.5317	0.7097\\
3.5428	0.7144\\
3.5531	0.7176\\
3.5639	0.7032\\
3.5741	0.7081\\
3.5845	0.7111\\
3.5948	0.7148\\
3.6055	0.6924\\
3.6160	0.6959\\
3.6262	0.6995\\
3.6366	0.7045\\
3.6473	0.7155\\
3.6584	0.7190\\
3.6689	0.7244\\
3.6793	0.7281\\
3.6899	0.7843\\
3.7003	0.7902\\
3.7106	0.7944\\
3.7208	0.8004\\
3.7316	0.8310\\
3.7419	0.8350\\
3.7524	0.8416\\
3.7626	0.9005\\
3.7735	0.9052\\
3.7836	0.9097\\
3.7940	0.9171\\
3.8043	0.6238\\
3.8148	0.6304\\
3.8254	0.6338\\
3.8358	0.8058\\
3.8469	0.8148\\
3.8577	0.8194\\
3.8687	0.6874\\
3.8794	0.6895\\
3.8899	0.6971\\
3.9003	0.6433\\
3.9104	0.6450\\
3.9208	0.6484\\
3.9316	0.6036\\
3.9419	0.6071\\
3.9529	0.6089\\
3.9638	0.5719\\
3.9743	0.5737\\
3.9853	0.5770\\
3.9959	0.5789\\
4.0066	0.5590\\
4.0175	0.5610\\
4.0283	0.5642\\
4.0390	0.6145\\
4.0492	0.6184\\
4.0596	0.6208\\
4.0700	0.6245\\
4.0806	0.6929\\
4.0909	0.6976\\
4.1013	0.7027\\
4.1117	0.7070\\
4.1220	0.7628\\
4.1325	0.7655\\
4.1430	0.7705\\
4.1538	0.7752\\
4.1647	0.6979\\
4.1753	0.6998\\
4.1857	0.7040\\
4.1959	0.7057\\
4.2068	0.6355\\
4.2177	0.6378\\
4.2282	0.6432\\
4.2391	0.6223\\
4.2501	0.6248\\
4.2604	0.6295\\
4.2708	0.6319\\
4.2818	0.5562\\
4.2930	0.5586\\
4.3040	0.5631\\
4.3150	0.5848\\
4.3261	0.5875\\
4.3367	0.5899\\
4.3471	0.5638\\
4.3576	0.5683\\
4.3679	0.5702\\
4.3782	0.5726\\
4.3889	0.5332\\
4.4000	0.5354\\
4.4105	0.5378\\
4.4209	0.5398\\
4.4315	0.5580\\
4.4417	0.5604\\
4.4520	0.5631\\
4.4627	0.5655\\
4.4734	0.5246\\
4.4844	0.5276\\
4.4947	0.5291\\
4.5056	0.5379\\
4.5166	0.5396\\
4.5273	0.5411\\
4.5375	0.5434\\
4.5482	0.5203\\
4.5584	0.5217\\
4.5688	0.5227\\
4.5791	0.5235\\
4.5897	0.5972\\
4.6001	0.5954\\
4.6105	0.5966\\
4.6207	0.6014\\
4.6315	0.5298\\
4.6419	0.5316\\
4.6522	0.5343\\
4.6625	0.5365\\
4.6734	0.4536\\
4.6844	0.4543\\
4.6951	0.4570\\
4.7054	0.5163\\
4.7158	0.5181\\
4.7262	0.5220\\
4.7365	0.5249\\
4.7473	0.5035\\
4.7574	0.5054\\
4.7678	0.5073\\
4.7782	0.5090\\
4.7890	0.5954\\
4.7993	0.5967\\
4.8095	0.6006\\
4.8200	0.6031\\
4.8304	0.7075\\
4.8407	0.7124\\
4.8515	0.7161\\
4.8620	0.7193\\
4.8722	0.6765\\
4.8834	0.6797\\
4.8940	0.6847\\
4.9042	0.6903\\
4.9147	0.5452\\
4.9253	0.5490\\
4.9355	0.5523\\
4.9458	0.5530\\
4.9558	0.4501\\
4.9666	0.4503\\
4.9771	0.4522\\
4.9875	0.4495\\
4.9980	0.3545\\
5.0084	0.3545\\
5.0190	0.3562\\
5.0294	0.3852\\
5.0399	0.3822\\
5.0510	0.3829\\
5.0615	0.3842\\
5.0722	0.3933\\
5.0829	0.3942\\
5.0935	0.3958\\
5.1041	0.3937\\
5.1149	0.5323\\
5.1261	0.5350\\
5.1364	0.5367\\
5.1471	0.6905\\
5.1577	0.6907\\
5.1688	0.6943\\
5.1791	0.6977\\
5.1899	0.6816\\
5.2003	0.6857\\
5.2111	0.6919\\
5.2221	0.7153\\
5.2325	0.7206\\
5.2430	0.7272\\
5.2531	0.7317\\
5.2636	0.6291\\
5.2744	0.6329\\
5.2855	0.6339\\
5.2959	0.6382\\
5.3064	0.3515\\
5.3169	0.3531\\
5.3273	0.3527\\
5.3378	0.3520\\
5.3482	0.0454\\
5.3583	0.0443\\
5.3688	0.0428\\
5.3792	-0.0162\\
5.3901	-0.0166\\
5.4013	-0.0168\\
5.4125	-0.0162\\
5.4231	-0.0396\\
5.4334	-0.0385\\
5.4440	-0.0387\\
5.4543	-0.0383\\
5.4647	0.3608\\
5.4753	0.3623\\
5.4856	0.3661\\
5.4959	0.3701\\
5.5062	1.0648\\
5.5167	1.0770\\
5.5272	1.0910\\
5.5374	1.0975\\
5.5482	1.6183\\
5.5591	1.6412\\
5.5700	1.6639\\
5.5804	1.7487\\
5.5908	1.7741\\
5.6010	1.7866\\
5.6114	1.7883\\
5.6221	0.7949\\
5.6324	0.7957\\
5.6429	0.7896\\
5.6530	0.7887\\
5.6635	-0.5476\\
5.6741	-0.5442\\
5.6845	-0.5370\\
5.6949	-0.5339\\
5.7056	-1.4855\\
5.7158	-1.4656\\
5.7260	-1.4363\\
5.7363	-1.4159\\
5.7473	-1.5191\\
};
\addplot [color=mycolor2,solid,forget plot]
  table[row sep=crcr]{%
-1.0000	-0.0168\\
-0.9901	-0.0135\\
-0.9785	-0.0055\\
-0.9688	-0.0031\\
-0.9578	-0.0031\\
-0.9473	-0.0031\\
-0.9356	-0.0031\\
-0.9258	-0.0031\\
-0.9160	-0.0031\\
-0.9043	-0.0031\\
-0.8926	-0.0031\\
-0.8826	-0.0031\\
-0.8731	-0.0031\\
-0.8613	-0.0031\\
-0.8516	-0.0031\\
-0.8418	-0.0031\\
-0.8301	-0.0031\\
-0.8203	-0.0031\\
-0.8093	-0.0031\\
-0.7989	-0.0031\\
-0.7891	-0.0031\\
-0.7774	-0.0031\\
-0.7676	-0.0031\\
-0.7559	-0.0031\\
-0.7461	-0.0031\\
-0.7363	-0.0031\\
-0.7266	-0.0031\\
-0.7149	-0.0031\\
-0.7051	-0.0031\\
-0.6934	-0.0031\\
-0.6836	-0.0031\\
-0.6738	-0.0031\\
-0.6639	-0.0031\\
-0.6524	-0.0031\\
-0.6426	-0.0030\\
-0.6328	-0.0035\\
-0.6211	-0.0035\\
-0.6114	-0.0035\\
-0.5996	-0.0035\\
-0.5900	-0.0035\\
-0.5802	-0.0035\\
-0.5703	-0.0035\\
-0.5587	-0.0035\\
-0.5489	-0.0035\\
-0.5391	-0.0035\\
-0.5274	-0.0035\\
-0.5176	-0.0040\\
-0.5059	-0.0041\\
-0.4961	-0.0041\\
-0.4843	0.0102\\
-0.4745	0.0120\\
-0.4626	0.0120\\
-0.4531	0.0120\\
-0.4419	0.0095\\
-0.4316	0.0095\\
-0.4213	0.0100\\
-0.4102	0.0100\\
-0.4004	0.0100\\
-0.3907	0.0100\\
-0.3789	0.0100\\
-0.3692	0.0100\\
-0.3594	0.0100\\
-0.3477	0.0100\\
-0.3380	0.0100\\
-0.3262	0.0100\\
-0.3145	0.0100\\
-0.3047	0.0100\\
-0.2930	0.0100\\
-0.2832	0.0100\\
-0.2734	0.0100\\
-0.2617	0.0100\\
-0.2520	0.0100\\
-0.2422	0.0100\\
-0.2305	0.0100\\
-0.2207	0.0100\\
-0.2090	0.0100\\
-0.1992	0.0100\\
-0.1875	0.0100\\
-0.1778	0.0100\\
-0.1680	0.0100\\
-0.1563	0.0100\\
-0.1465	0.0100\\
-0.1367	0.0100\\
-0.1250	0.0095\\
-0.1152	0.0094\\
-0.1055	0.0067\\
-0.0945	0.0067\\
-0.0840	0.0067\\
-0.0743	0.0067\\
-0.0625	0.0067\\
-0.0527	0.0067\\
-0.0411	0.0067\\
-0.0313	0.0067\\
-0.0196	0.0067\\
-0.0098	-0.0233\\
-0.0000	-0.0371\\
0.0117	-0.0371\\
0.0214	-0.0422\\
0.0332	-0.0405\\
0.0429	-0.0369\\
0.0527	-0.0369\\
0.0644	-0.0384\\
0.0761	-0.0383\\
0.0859	-0.0383\\
0.0957	-0.0383\\
0.1081	-0.0383\\
0.1170	-0.0383\\
0.1269	-0.0383\\
0.1386	-0.0383\\
0.1484	-0.0383\\
0.1582	-0.0383\\
0.1699	-0.0383\\
0.1790	-0.0383\\
0.1893	-0.0378\\
0.2011	-0.0378\\
0.2109	-0.0378\\
0.2226	-0.0378\\
0.2324	-0.0186\\
0.2441	-0.0186\\
0.2539	-0.0186\\
0.2638	-0.0186\\
0.2754	-0.0186\\
0.2839	-0.0142\\
0.2950	0.0184\\
0.3067	0.0337\\
0.3165	0.0500\\
0.3262	0.0575\\
0.3360	0.0540\\
0.3476	0.0554\\
0.3574	0.0898\\
0.3691	0.1093\\
0.3789	0.1420\\
0.3906	0.1771\\
0.4004	0.2050\\
0.4100	0.2129\\
0.4218	0.2185\\
0.4315	0.2338\\
0.4415	0.2458\\
0.4532	0.2938\\
0.4653	0.3004\\
0.4787	0.2967\\
0.4864	0.2967\\
0.4961	0.2917\\
0.5059	0.3143\\
0.5176	0.3413\\
0.5274	0.3743\\
0.5397	0.3570\\
0.5507	0.3512\\
0.5586	0.3512\\
0.5683	0.3487\\
0.5800	0.3531\\
0.5898	0.3700\\
0.6009	0.3784\\
0.6136	0.3901\\
0.6204	0.3901\\
0.6308	0.4242\\
0.6425	0.4113\\
0.6515	0.4246\\
0.6627	0.4124\\
0.6743	0.4344\\
0.6863	0.4520\\
0.6933	0.4520\\
0.7052	0.4518\\
0.7148	0.4963\\
0.7254	0.5068\\
0.7398	0.5081\\
0.7473	0.5081\\
0.7579	0.5147\\
0.7695	0.5438\\
0.7792	0.5567\\
0.7901	0.4721\\
0.8015	0.4687\\
0.8128	0.4687\\
0.8203	0.4687\\
0.8313	0.4996\\
0.8437	0.4627\\
0.8535	0.4662\\
0.8653	0.4864\\
0.8751	0.4688\\
0.8856	0.4759\\
0.8997	0.4721\\
0.9083	0.4721\\
0.9179	0.4706\\
0.9277	0.4830\\
0.9392	0.5439\\
0.9522	0.5519\\
0.9590	0.5519\\
0.9707	0.5639\\
0.9805	0.5796\\
0.9905	0.5704\\
1.0032	0.6402\\
1.0117	0.6402\\
1.0234	0.6717\\
1.0332	0.6433\\
1.0437	0.6395\\
1.0555	0.6171\\
1.0704	0.6409\\
1.0763	0.6409\\
1.0859	0.6134\\
1.0970	0.5604\\
1.1074	0.5460\\
1.1214	0.6493\\
1.1289	0.6493\\
1.1406	0.6771\\
1.1504	0.6700\\
1.1603	0.6300\\
1.1711	0.6138\\
1.1840	0.6132\\
1.1914	0.6132\\
1.2031	0.6052\\
1.2129	0.6151\\
1.2227	0.5983\\
1.2337	0.5953\\
1.2443	0.5924\\
1.2586	0.6404\\
1.2657	0.6404\\
1.2755	0.6433\\
1.2850	0.6613\\
1.2969	0.6621\\
1.3104	0.6673\\
1.3184	0.6673\\
1.3301	0.6762\\
1.3398	0.6924\\
1.3496	0.7022\\
1.3609	0.6721\\
1.3739	0.6497\\
1.3808	0.6497\\
1.3926	0.7465\\
1.4023	0.7580\\
1.4129	0.7452\\
1.4297	0.7927\\
1.4398	0.8183\\
1.4521	0.7824\\
1.4590	0.7824\\
1.4707	0.7656\\
1.4824	0.7714\\
1.4910	0.7490\\
1.5027	0.7100\\
1.5141	0.7409\\
1.5253	0.7177\\
1.5332	0.7177\\
1.5449	0.7300\\
1.5535	0.7476\\
1.5654	0.7358\\
1.5804	0.7377\\
1.5860	0.7377\\
1.5957	0.7036\\
1.6075	0.6761\\
1.6192	0.6813\\
1.6334	0.6640\\
1.6406	0.6640\\
1.6484	0.6221\\
1.6607	0.8090\\
1.6727	0.8006\\
1.6845	0.8360\\
1.6914	0.8360\\
1.7031	0.8373\\
1.7135	0.8129\\
1.7252	0.7834\\
1.7386	0.7585\\
1.7441	0.7585\\
1.7560	0.7265\\
1.7656	0.7895\\
1.7793	0.7915\\
1.7872	0.7915\\
1.7989	0.7989\\
1.8086	0.7652\\
1.8203	0.7547\\
1.8301	0.7550\\
1.8420	0.7415\\
1.8496	0.7415\\
1.8614	0.7599\\
1.8711	0.7609\\
1.8835	0.7421\\
1.8955	0.7300\\
1.9023	0.7300\\
1.9121	0.6902\\
1.9225	0.6717\\
1.9346	0.7379\\
1.9466	0.8058\\
1.9551	0.8058\\
1.9641	0.8026\\
1.9755	0.7527\\
1.9904	0.7407\\
1.9962	0.7407\\
2.0059	0.6887\\
2.0177	0.6928\\
2.0298	0.7215\\
2.0372	0.7215\\
2.0489	0.7699\\
2.0587	0.7831\\
2.0703	0.7626\\
2.0843	0.7584\\
2.0898	0.7584\\
2.0996	0.7586\\
2.1117	0.7400\\
2.1239	0.7243\\
2.1308	0.7243\\
2.1425	0.7067\\
2.1523	0.7124\\
2.1629	0.7075\\
2.1747	0.7366\\
2.1868	0.7532\\
2.2016	0.7709\\
2.2052	0.7709\\
2.2150	0.7384\\
2.2254	0.7377\\
2.2409	0.7695\\
2.2462	0.7695\\
2.2560	0.7560\\
2.2677	0.7689\\
2.2767	0.7221\\
2.2934	0.6668\\
2.2988	0.6668\\
2.3086	0.6507\\
2.3224	0.6980\\
2.3301	0.6980\\
2.3418	0.7831\\
2.3515	0.7605\\
2.3633	0.7574\\
2.3739	0.7683\\
2.3852	0.7653\\
2.3976	0.7613\\
2.4043	0.7613\\
2.4140	0.7633\\
2.4250	0.7570\\
2.4363	0.7840\\
2.4517	0.8486\\
2.4590	0.8486\\
2.4688	0.8358\\
2.4785	0.8074\\
2.4909	0.7794\\
2.5030	0.7611\\
2.5149	0.7652\\
2.5214	0.7652\\
2.5332	0.7218\\
2.5430	0.6587\\
2.5538	0.7206\\
2.5653	0.7951\\
2.5805	0.8405\\
2.5861	0.8405\\
2.5958	0.7956\\
2.6076	0.7580\\
2.6164	0.7251\\
2.6316	0.7448\\
2.6367	0.7448\\
2.6484	0.7243\\
2.6605	0.6606\\
2.6680	0.6606\\
2.6797	0.6485\\
2.6894	0.6435\\
2.7011	0.7168\\
2.7120	0.7271\\
2.7233	0.7179\\
2.7355	0.7813\\
2.7422	0.7813\\
2.7539	0.7505\\
2.7638	0.7034\\
2.7797	0.7072\\
2.7845	0.7072\\
2.7951	0.6558\\
2.8059	0.6616\\
2.8202	0.6662\\
2.8263	0.6662\\
2.8361	0.6871\\
2.8478	0.6522\\
2.8594	0.6451\\
2.8713	0.6197\\
2.8789	0.6197\\
2.8906	0.5943\\
2.9004	0.6615\\
2.9121	0.7698\\
2.9221	0.8851\\
2.9347	0.8615\\
2.9467	0.7910\\
2.9552	0.7910\\
2.9649	0.7321\\
2.9759	0.7003\\
2.9927	0.7308\\
2.9954	0.7308\\
3.0058	0.6958\\
3.0202	0.6344\\
3.0274	0.6344\\
3.0372	0.6487\\
3.0488	0.6507\\
3.0586	0.6528\\
3.0705	0.6433\\
3.0829	0.6461\\
3.0947	0.6477\\
3.1016	0.6477\\
3.1133	0.6541\\
3.1234	0.6587\\
3.1357	0.6412\\
3.1510	0.6451\\
3.1543	0.6451\\
3.1653	0.6240\\
3.1748	0.7365\\
3.1896	0.7527\\
3.1974	0.7527\\
3.2072	0.7230\\
3.2169	0.6889\\
3.2287	0.6902\\
3.2413	0.7797\\
3.2483	0.7797\\
3.2599	0.7990\\
3.2700	0.7957\\
3.2821	0.8286\\
3.2939	0.8197\\
3.3051	0.8093\\
3.3125	0.8093\\
3.3222	0.8381\\
3.3330	0.8282\\
3.3449	0.7383\\
3.3568	0.7143\\
3.3654	0.7143\\
3.3744	0.7663\\
3.3914	0.7609\\
3.3966	0.7609\\
3.4064	0.7334\\
3.4207	0.7334\\
3.4278	0.7334\\
3.4376	0.7255\\
3.4492	0.7367\\
3.4591	0.7162\\
3.4736	0.7027\\
3.4805	0.7027\\
3.4902	0.7059\\
3.5028	0.7561\\
3.5149	0.7344\\
3.5215	0.7344\\
3.5312	0.7063\\
3.5426	0.6428\\
3.5564	0.6135\\
3.5645	0.6135\\
3.5734	0.6238\\
3.5853	0.6029\\
3.6025	0.5664\\
3.6055	0.5664\\
3.6173	0.5581\\
3.6307	0.6001\\
3.6369	0.6001\\
3.6466	0.6070\\
3.6582	0.5841\\
3.6706	0.5533\\
3.6826	0.6218\\
3.6894	0.6218\\
3.7012	0.6158\\
3.7109	0.6171\\
3.7221	0.7483\\
3.7357	0.7718\\
3.7422	0.7718\\
3.7530	0.7950\\
3.7651	0.8607\\
3.7734	0.8607\\
3.7831	0.9185\\
3.7944	0.8946\\
3.8114	0.8666\\
3.8145	0.8666\\
3.8263	0.8557\\
3.8351	0.8496\\
3.8504	0.8450\\
3.8575	0.8450\\
3.8691	0.8208\\
3.8789	0.7689\\
3.8906	0.7433\\
3.9005	0.7622\\
3.9129	0.7637\\
3.9245	0.6957\\
3.9316	0.6957\\
3.9414	0.6324\\
3.9531	0.5980\\
3.9637	0.5870\\
3.9759	0.5919\\
3.9921	0.5547\\
3.9962	0.5547\\
4.0078	0.5543\\
4.0217	0.5505\\
4.0293	0.5505\\
4.0391	0.5461\\
4.0488	0.5364\\
4.0632	0.5848\\
4.0703	0.5848\\
4.0801	0.6392\\
4.0910	0.5754\\
4.1016	0.5754\\
4.1159	0.6524\\
4.1230	0.6524\\
4.1330	0.9524\\
4.1455	0.9155\\
4.1575	0.8813\\
4.1641	0.8813\\
4.1750	0.7577\\
4.1869	0.6715\\
4.1954	0.6715\\
4.2072	0.7352\\
4.2187	0.6746\\
4.2310	0.6066\\
4.2383	0.6066\\
4.2500	0.6062\\
4.2597	0.6698\\
4.2717	0.6652\\
4.2841	0.6594\\
4.2959	0.6421\\
4.3110	0.6434\\
4.3145	0.6434\\
4.3263	0.6920\\
4.3359	0.6344\\
4.3470	0.6029\\
4.3616	0.6991\\
4.3691	0.6991\\
4.3790	0.7353\\
4.3887	0.6766\\
4.4004	0.5949\\
4.4129	0.5873\\
4.4247	0.6273\\
4.4316	0.6273\\
4.4414	0.6130\\
4.4537	0.7018\\
4.4694	0.6267\\
4.4727	0.6267\\
4.4844	0.6093\\
4.4951	0.5305\\
4.5081	0.4786\\
4.5242	0.5225\\
4.5275	0.5225\\
4.5390	0.5945\\
4.5538	0.4854\\
4.5587	0.4854\\
4.5692	0.6611\\
4.5816	0.6736\\
4.5897	0.6736\\
4.5997	0.6429\\
4.6112	0.6328\\
4.6222	0.7577\\
4.6341	0.7475\\
4.6426	0.7475\\
4.6522	0.6519\\
4.6634	0.6116\\
4.6802	0.5327\\
4.6857	0.5327\\
4.6955	0.5310\\
4.7064	0.4945\\
4.7216	0.5894\\
4.7265	0.5894\\
4.7363	0.4568\\
4.7481	0.4559\\
4.7627	0.4542\\
4.7677	0.4542\\
4.7775	0.4592\\
4.7891	0.4999\\
4.8037	0.7191\\
4.8105	0.7191\\
4.8203	0.7802\\
4.8330	0.5980\\
4.8451	0.6212\\
4.8516	0.6212\\
4.8618	0.6754\\
4.8744	0.6957\\
4.8828	0.6957\\
4.8945	0.7396\\
4.9035	0.7311\\
4.9150	0.6886\\
4.9267	0.6665\\
4.9423	0.6485\\
4.9454	0.6485\\
4.9572	0.6274\\
4.9660	0.5773\\
4.9815	0.5594\\
4.9884	0.5594\\
4.9982	0.5461\\
5.0080	0.5391\\
5.0186	0.5352\\
5.0337	0.5050\\
5.0392	0.5050\\
5.0508	0.4813\\
5.0619	0.4815\\
5.0762	0.4806\\
5.0840	0.4806\\
5.0930	0.4819\\
5.1052	0.4819\\
5.1188	0.4819\\
5.1269	0.4819\\
5.1367	0.5439\\
5.1527	0.5439\\
5.1584	0.5439\\
5.1691	0.5719\\
5.1812	0.6013\\
5.1939	0.6042\\
5.2012	0.6042\\
5.2109	0.6074\\
5.2234	0.7150\\
5.2356	0.7597\\
5.2422	0.7597\\
5.2539	0.8067\\
5.2634	0.8612\\
5.2797	0.8080\\
5.2853	0.8080\\
5.2970	0.7939\\
5.3068	0.7599\\
5.3184	0.6851\\
5.3321	0.6846\\
5.3379	0.6846\\
5.3477	0.6640\\
5.3589	0.6283\\
5.3718	0.5553\\
5.3789	0.5553\\
5.3899	0.4519\\
5.4006	0.1519\\
5.4128	-0.0023\\
5.4243	-0.0650\\
5.4362	-0.0954\\
5.4434	-0.0954\\
5.4552	-0.0753\\
5.4650	-0.0991\\
5.4752	-0.0773\\
5.4903	0.1405\\
5.4963	0.1405\\
5.5060	0.2175\\
5.5177	0.3193\\
5.5316	0.5913\\
5.5372	0.5913\\
5.5489	0.8673\\
5.5586	1.1189\\
5.5709	1.2490\\
5.5829	1.5490\\
5.5945	1.6756\\
5.6015	1.6756\\
5.6113	1.8835\\
5.6222	1.7774\\
5.6334	1.6733\\
5.6452	1.4942\\
5.6523	1.4942\\
5.6641	1.1970\\
5.6738	1.0462\\
5.6857	0.7462\\
5.6955	0.4462\\
5.7057	0.1462\\
5.7209	-0.1538\\
5.7267	-0.1538\\
5.7365	-0.4538\\
5.7481	-0.7538\\
};
\addplot [color=mycolor3,dotted,forget plot]
  table[row sep=crcr]{%
-1.0000	1.0000\\
7.4186	1.0000\\
};
\addplot [color=mycolor3,dashed,forget plot]
  table[row sep=crcr]{%
-1.0000	-0.0016\\
-0.9889	-0.0016\\
-0.9785	-0.0013\\
-0.9679	-0.0013\\
-0.9575	-0.0013\\
-0.9471	-0.0013\\
-0.9367	-0.0013\\
-0.9262	-0.0013\\
-0.9159	-0.0013\\
-0.9056	-0.0013\\
-0.8951	-0.0010\\
-0.8849	-0.0010\\
-0.8741	-0.0010\\
-0.8639	-0.0010\\
-0.8535	-0.0001\\
-0.8431	-0.0001\\
-0.8329	-0.0001\\
-0.8227	-0.0001\\
-0.8116	-0.0001\\
-0.8013	-0.0001\\
-0.7911	-0.0001\\
-0.7807	-0.0001\\
-0.7700	0.0005\\
-0.7596	0.0004\\
-0.7493	0.0005\\
-0.7391	0.0005\\
-0.7284	0.0007\\
-0.7181	0.0007\\
-0.7076	0.0008\\
-0.6973	0.0007\\
-0.6865	0.0013\\
-0.6762	0.0013\\
-0.6660	0.0013\\
-0.6555	0.0013\\
-0.6447	0.0019\\
-0.6337	0.0019\\
-0.6234	0.0019\\
-0.6126	0.0023\\
-0.6021	0.0022\\
-0.5918	0.0022\\
-0.5815	0.0023\\
-0.5711	0.0019\\
-0.5609	0.0019\\
-0.5502	0.0019\\
-0.5392	0.0019\\
-0.5284	0.0037\\
-0.5182	0.0037\\
-0.5077	0.0036\\
-0.4972	0.0036\\
-0.4866	0.0034\\
-0.4764	0.0034\\
-0.4658	0.0034\\
-0.4551	0.0040\\
-0.4440	0.0040\\
-0.4334	0.0040\\
-0.4224	0.0040\\
-0.4115	0.0031\\
-0.4012	0.0031\\
-0.3910	0.0031\\
-0.3806	0.0031\\
-0.3700	0.0048\\
-0.3595	0.0048\\
-0.3490	0.0048\\
-0.3390	0.0048\\
-0.3281	0.0051\\
-0.3178	0.0050\\
-0.3077	0.0050\\
-0.2974	0.0050\\
-0.2867	0.0039\\
-0.2764	0.0039\\
-0.2660	0.0039\\
-0.2556	0.0039\\
-0.2450	0.0059\\
-0.2347	0.0059\\
-0.2245	0.0059\\
-0.2139	0.0059\\
-0.2034	0.0056\\
-0.1931	0.0056\\
-0.1828	0.0056\\
-0.1722	0.0056\\
-0.1617	0.0061\\
-0.1514	0.0061\\
-0.1410	0.0061\\
-0.1307	0.0061\\
-0.1204	0.0058\\
-0.1096	0.0058\\
-0.0992	0.0058\\
-0.0891	0.0058\\
-0.0783	0.0064\\
-0.0677	0.0064\\
-0.0576	0.0064\\
-0.0473	0.0064\\
-0.0368	0.0058\\
-0.0264	0.0058\\
-0.0160	0.0058\\
-0.0058	0.0058\\
0.0051	0.0083\\
0.0153	0.0083\\
0.0255	0.0083\\
0.0360	0.0084\\
0.0467	0.0276\\
0.0570	0.0276\\
0.0672	0.0277\\
0.0778	0.0277\\
0.0883	0.0630\\
0.0995	0.0630\\
0.1100	0.0631\\
0.1207	0.1103\\
0.1309	0.1104\\
0.1413	0.1105\\
0.1518	0.1107\\
0.1624	0.1429\\
0.1726	0.1431\\
0.1837	0.1433\\
0.1942	0.1435\\
0.2051	0.1873\\
0.2154	0.1876\\
0.2265	0.1880\\
0.2373	0.2406\\
0.2479	0.2411\\
0.2581	0.2416\\
0.2692	0.2421\\
0.2799	0.2680\\
0.2903	0.2686\\
0.3008	0.2692\\
0.3111	0.2699\\
0.3218	0.3115\\
0.3321	0.3123\\
0.3423	0.3132\\
0.3527	0.3141\\
0.3634	0.3473\\
0.3736	0.3483\\
0.3841	0.3494\\
0.3943	0.3504\\
0.4048	0.3842\\
0.4153	0.3855\\
0.4260	0.3867\\
0.4360	0.4441\\
0.4464	0.4456\\
0.4571	0.4471\\
0.4675	0.4488\\
0.4776	0.3549\\
0.4880	0.3575\\
0.4984	0.3588\\
0.5091	0.4993\\
0.5195	0.5011\\
0.5298	0.5050\\
0.5405	0.5258\\
0.5508	0.5279\\
0.5610	0.5300\\
0.5716	0.5517\\
0.5820	0.5543\\
0.5923	0.5566\\
0.6028	0.5589\\
0.6133	0.5465\\
0.6244	0.5491\\
0.6351	0.5515\\
0.6454	0.6187\\
0.6560	0.6218\\
0.6662	0.6248\\
0.6770	0.6277\\
0.6875	0.6147\\
0.6977	0.6180\\
0.7079	0.6209\\
0.7182	0.6245\\
0.7290	0.6499\\
0.7395	0.6532\\
0.7506	0.6569\\
0.7611	0.6607\\
0.7718	0.6872\\
0.7825	0.6911\\
0.7935	0.6950\\
0.8040	0.7655\\
0.8144	0.7701\\
0.8246	0.7744\\
0.8354	0.7792\\
0.8457	0.7413\\
0.8562	0.7457\\
0.8664	0.7507\\
0.8775	0.7547\\
0.8882	0.7830\\
0.8984	0.7872\\
0.9092	0.7928\\
0.9193	0.7979\\
0.9300	0.7967\\
0.9412	0.8019\\
0.9516	0.8067\\
0.9618	0.8872\\
0.9726	0.8924\\
0.9828	0.8986\\
0.9933	0.9048\\
1.0039	0.8501\\
1.0141	0.8559\\
1.0248	0.8620\\
1.0351	0.8683\\
1.0458	0.8815\\
1.0562	0.8884\\
1.0664	0.8946\\
1.0767	0.9014\\
1.0872	0.9120\\
1.0975	0.9191\\
1.1084	0.9259\\
1.1194	0.9326\\
1.1296	0.9490\\
1.1402	0.9561\\
1.1507	0.9639\\
1.1611	0.9711\\
1.1715	0.9810\\
1.1825	0.9884\\
1.1933	0.9962\\
1.2040	1.0670\\
1.2142	1.0757\\
1.2247	1.0840\\
1.2350	1.0929\\
1.2456	1.0330\\
1.2563	1.0417\\
1.2667	1.0502\\
1.2778	1.0590\\
1.2885	1.0668\\
1.2995	1.0762\\
1.3105	1.0856\\
1.3216	1.1724\\
1.3319	1.1829\\
1.3425	1.1930\\
1.3528	1.2041\\
1.3633	1.1210\\
1.3745	1.1311\\
1.3852	1.1410\\
1.3957	1.2180\\
1.4061	1.2291\\
1.4173	1.2404\\
1.4278	1.2519\\
1.4384	1.1655\\
1.4486	1.1766\\
1.4592	1.1871\\
1.4693	1.1987\\
1.4803	1.1799\\
1.4914	1.1909\\
1.5024	1.2021\\
1.5132	1.2701\\
1.5238	1.2823\\
1.5349	1.2938\\
1.5457	1.2737\\
1.5561	1.2853\\
1.5665	1.2982\\
1.5768	1.3102\\
1.5874	1.2132\\
1.5986	1.2248\\
1.6090	1.2368\\
1.6192	1.2480\\
1.6304	1.2244\\
1.6414	1.2366\\
1.6517	1.2485\\
1.6623	1.3029\\
1.6728	1.3155\\
1.6834	1.3295\\
1.6935	1.3417\\
1.7041	1.2407\\
1.7143	1.2521\\
1.7247	1.2651\\
1.7351	1.2769\\
1.7454	1.2441\\
1.7561	1.2573\\
1.7662	1.2684\\
1.7766	1.2807\\
1.7874	1.2404\\
1.7978	1.2516\\
1.8082	1.2644\\
1.8184	1.2758\\
1.8291	1.2359\\
1.8394	1.2490\\
1.8498	1.2597\\
1.8599	1.2719\\
1.8705	1.2305\\
1.8811	1.2416\\
1.8913	1.2529\\
1.9024	1.2658\\
1.9133	1.2016\\
1.9237	1.2131\\
1.9339	1.2245\\
1.9445	1.2361\\
1.9551	1.1948\\
1.9652	1.2070\\
1.9758	1.2173\\
1.9863	1.2282\\
1.9970	1.1832\\
2.0072	1.1940\\
2.0183	1.2049\\
2.0291	1.2487\\
2.0394	1.2601\\
2.0500	1.2718\\
2.0607	1.2837\\
2.0718	1.1559\\
2.0823	1.1656\\
2.0927	1.1753\\
2.1029	1.1869\\
2.1137	1.1409\\
2.1239	1.1509\\
2.1340	1.1609\\
2.1446	1.1853\\
2.1551	1.1960\\
2.1653	1.2051\\
2.1756	1.2158\\
2.1862	0.8768\\
2.1968	0.8934\\
2.2074	0.9003\\
2.2183	1.1779\\
2.2292	1.1990\\
2.2396	1.2089\\
2.2503	1.1842\\
2.2613	1.1937\\
2.2716	1.2160\\
2.2824	1.2026\\
2.2934	1.2127\\
2.3041	1.2343\\
2.3143	1.2270\\
2.3248	1.2382\\
2.3355	1.2494\\
2.3466	1.2387\\
2.3571	1.2501\\
2.3674	1.2619\\
2.3780	1.2738\\
2.3887	1.1877\\
2.3997	1.1994\\
2.4100	1.2088\\
2.4205	1.2650\\
2.4312	1.2754\\
2.4416	1.2881\\
2.4518	1.2990\\
2.4624	1.1903\\
2.4730	1.2004\\
2.4841	1.2132\\
2.4945	1.2238\\
2.5054	1.1548\\
2.5164	1.1650\\
2.5270	1.1756\\
2.5376	1.1996\\
2.5486	1.2101\\
2.5591	1.2188\\
2.5696	1.2304\\
2.5801	1.0865\\
2.5907	1.0943\\
2.6016	1.1050\\
2.6124	1.1194\\
2.6226	1.1277\\
2.6332	1.1359\\
2.6443	1.1481\\
2.6551	1.0566\\
2.6655	1.0647\\
2.6757	1.0724\\
2.6862	1.0802\\
2.6968	1.0357\\
2.7074	1.0440\\
2.7185	1.0525\\
2.7292	1.1309\\
2.7395	1.1404\\
2.7501	1.1505\\
2.7602	1.1601\\
2.7706	1.0381\\
2.7813	1.0463\\
2.7915	1.0559\\
2.8018	1.0657\\
2.8123	1.0879\\
2.8228	1.0940\\
2.8330	1.1044\\
2.8433	1.1141\\
2.8539	0.9950\\
2.8645	1.0027\\
2.8748	1.0112\\
2.8849	1.0174\\
2.8956	0.9470\\
2.9064	0.9563\\
2.9174	0.9625\\
2.9277	0.9690\\
2.9383	0.8626\\
2.9487	0.8679\\
2.9591	0.8750\\
2.9696	0.8808\\
2.9802	0.8196\\
2.9908	0.8252\\
3.0018	0.8305\\
3.0123	0.8076\\
3.0230	0.8131\\
3.0343	0.8164\\
3.0446	0.8237\\
3.0549	0.7057\\
3.0654	0.7090\\
3.0758	0.7144\\
3.0860	0.7166\\
3.0966	0.6606\\
3.1070	0.6630\\
3.1177	0.6682\\
3.1281	0.6716\\
3.1383	0.7486\\
3.1487	0.7544\\
3.1590	0.7570\\
3.1693	0.7627\\
3.1802	0.8261\\
3.1905	0.8324\\
3.2008	0.8384\\
3.2110	0.8466\\
3.2217	0.9382\\
3.2324	0.9452\\
3.2435	0.9533\\
3.2540	1.0973\\
3.2644	1.1058\\
3.2747	1.1167\\
3.2848	1.1279\\
3.2957	1.0385\\
3.3063	1.0484\\
3.3165	1.0590\\
3.3275	1.0657\\
3.3384	0.9356\\
3.3490	0.9405\\
3.3591	0.9448\\
3.3693	0.9521\\
3.3801	0.7628\\
3.3903	0.7658\\
3.4006	0.7691\\
3.4111	0.7719\\
3.4216	0.6420\\
3.4320	0.6457\\
3.4422	0.6469\\
3.4527	0.6537\\
3.4632	0.6725\\
3.4737	0.6766\\
3.4842	0.6806\\
3.4946	0.6878\\
3.5053	0.8798\\
3.5164	0.8902\\
3.5266	0.9012\\
3.5374	1.2173\\
3.5476	1.2311\\
3.5582	1.2473\\
3.5690	1.2637\\
3.5799	1.3422\\
3.5902	1.3542\\
3.6008	1.3707\\
3.6111	1.3838\\
3.6217	1.4275\\
3.6320	1.4407\\
3.6425	1.4585\\
3.6529	1.4777\\
3.6635	1.3489\\
3.6745	1.3567\\
3.6849	1.3644\\
3.6956	1.1074\\
3.7060	1.1123\\
3.7163	1.1214\\
3.7265	1.1282\\
3.7373	0.5313\\
3.7476	0.5319\\
3.7580	0.5337\\
3.7686	0.5308\\
3.7791	-0.1414\\
3.7894	-0.1474\\
3.7997	-0.1501\\
3.8102	-0.1479\\
3.8210	-0.6318\\
3.8321	-0.6400\\
3.8431	-0.6324\\
3.8540	-0.5916\\
3.8645	-0.5856\\
3.8750	-0.5844\\
3.8861	-0.5766\\
3.8967	-0.0608\\
3.9069	-0.0588\\
3.9174	-0.0543\\
3.9276	-0.0511\\
3.9381	0.9651\\
3.9491	0.9822\\
3.9599	0.9987\\
3.9704	2.2695\\
3.9810	2.3202\\
3.9912	2.3738\\
4.0015	2.4423\\
4.0125	3.2756\\
4.0227	3.3792\\
4.0329	3.4572\\
4.0433	3.5456\\
4.0541	3.7032\\
4.0644	3.8030\\
4.0749	3.8713\\
4.0852	3.9819\\
4.0959	3.0834\\
4.1062	3.1445\\
4.1172	3.1754\\
4.1276	3.2073\\
4.1384	1.7507\\
4.1488	1.7680\\
4.1599	1.7671\\
4.1708	0.1423\\
4.1811	0.1349\\
4.1914	0.1267\\
4.2020	0.1201\\
4.2124	-1.1261\\
4.2226	-1.1103\\
4.2336	-1.0970\\
4.2446	-1.0720\\
4.2550	-2.4321\\
4.2654	-2.3659\\
4.2766	-2.3109\\
4.2871	-3.2603\\
4.2976	-3.2068\\
4.3081	-3.1089\\
4.3183	-3.0361\\
4.3290	-2.6518\\
4.3393	-2.6119\\
4.3497	-2.5725\\
};
\addplot [color=mycolor3,solid,forget plot]
  table[row sep=crcr]{%
-1.0000	0.0007\\
-0.9870	0.0015\\
-0.9785	0.0015\\
-0.9668	0.0024\\
-0.9570	0.0026\\
-0.9453	0.0026\\
-0.9355	0.0032\\
-0.9258	0.0032\\
-0.9140	0.0033\\
-0.9043	0.0033\\
-0.8945	0.0033\\
-0.8828	0.0033\\
-0.8730	0.0033\\
-0.8633	0.0033\\
-0.8516	0.0033\\
-0.8418	0.0033\\
-0.8320	0.0033\\
-0.8222	0.0033\\
-0.8105	0.0015\\
-0.8008	0.0015\\
-0.7910	0.0015\\
-0.7793	0.0015\\
-0.7695	0.0015\\
-0.7578	0.0015\\
-0.7480	0.0015\\
-0.7383	0.0015\\
-0.7266	0.0015\\
-0.7168	0.0015\\
-0.7070	0.0016\\
-0.6973	0.0016\\
-0.6855	0.0016\\
-0.6758	0.0016\\
-0.6660	0.0007\\
-0.6542	-0.0033\\
-0.6445	-0.0033\\
-0.6328	0.0092\\
-0.6230	0.0297\\
-0.6106	0.0348\\
-0.6011	0.0348\\
-0.5918	0.0350\\
-0.5801	0.0347\\
-0.5703	0.0320\\
-0.5605	0.0314\\
-0.5488	0.0305\\
-0.5391	0.0305\\
-0.5273	0.0301\\
-0.5176	0.0561\\
-0.5058	0.0561\\
-0.4961	0.0561\\
-0.4863	0.0534\\
-0.4746	0.0534\\
-0.4648	0.0534\\
-0.4551	0.0534\\
-0.4432	0.0501\\
-0.4316	0.0411\\
-0.4219	0.0324\\
-0.4102	0.0338\\
-0.4002	0.0179\\
-0.3891	-0.0289\\
-0.3772	-0.0460\\
-0.3691	-0.0460\\
-0.3594	-0.0594\\
-0.3477	-0.0786\\
-0.3379	-0.0766\\
-0.3268	-0.0635\\
-0.3144	-0.0630\\
-0.3066	-0.0630\\
-0.2969	-0.0615\\
-0.2851	-0.0592\\
-0.2754	-0.0552\\
-0.2635	-0.0465\\
-0.2549	-0.0366\\
-0.2448	-0.0296\\
-0.2344	-0.0296\\
-0.2233	-0.0299\\
-0.2129	-0.0299\\
-0.2031	-0.0296\\
-0.1914	-0.0286\\
-0.1816	-0.0286\\
-0.1719	-0.0286\\
-0.1602	-0.0286\\
-0.1504	-0.0284\\
-0.1406	-0.0284\\
-0.1295	-0.0292\\
-0.1190	-0.0259\\
-0.1093	-0.0259\\
-0.0977	-0.0230\\
-0.0879	-0.0230\\
-0.0781	-0.0230\\
-0.0664	-0.0230\\
-0.0566	-0.0230\\
-0.0467	-0.0230\\
-0.0358	-0.0230\\
-0.0254	-0.0230\\
-0.0156	-0.0230\\
-0.0046	-0.0230\\
0.0059	-0.0230\\
0.0156	-0.0230\\
0.0267	-0.0266\\
0.0364	-0.0266\\
0.0469	-0.0266\\
0.0586	-0.0266\\
0.0684	-0.0266\\
0.0781	-0.0288\\
0.0898	-0.0286\\
0.1015	-0.0238\\
0.1107	-0.0239\\
0.1226	-0.0234\\
0.1341	-0.0234\\
0.1426	-0.0234\\
0.1523	-0.0086\\
0.1641	-0.0059\\
0.1738	-0.0014\\
0.1855	0.0034\\
0.1954	0.0104\\
0.2051	0.0202\\
0.2169	0.0356\\
0.2266	0.0441\\
0.2383	0.0503\\
0.2480	0.0450\\
0.2602	0.0408\\
0.2703	0.0699\\
0.2803	0.1202\\
0.2909	0.1559\\
0.3027	0.1891\\
0.3125	0.2018\\
0.3223	0.2252\\
0.3341	0.2478\\
0.3437	0.2616\\
0.3535	0.2724\\
0.3652	0.2865\\
0.3755	0.2770\\
0.3869	0.2813\\
0.3945	0.2813\\
0.4064	0.2790\\
0.4162	0.2901\\
0.4279	0.3217\\
0.4362	0.3243\\
0.4503	0.3243\\
0.4572	0.3243\\
0.4687	0.3358\\
0.4785	0.3232\\
0.4884	0.4561\\
0.5000	0.5198\\
0.5108	0.5102\\
0.5218	0.5087\\
0.5334	0.5552\\
0.5410	0.5552\\
0.5526	0.5618\\
0.5625	0.5472\\
0.5717	0.4898\\
0.5843	0.4848\\
0.5963	0.4715\\
0.6035	0.4715\\
0.6133	0.4546\\
0.6250	0.4376\\
0.6353	0.4495\\
0.6468	0.4773\\
0.6617	0.4965\\
0.6680	0.4965\\
0.6778	0.4656\\
0.6888	0.4639\\
0.7006	0.5069\\
0.7090	0.5069\\
0.7187	0.5142\\
0.7305	0.5577\\
0.7402	0.6002\\
0.7516	0.6220\\
0.7644	0.6085\\
0.7763	0.6014\\
0.7832	0.6014\\
0.7949	0.6411\\
0.8047	0.6667\\
0.8150	0.6447\\
0.8270	0.7994\\
0.8421	0.8310\\
0.8457	0.8310\\
0.8576	0.8252\\
0.8665	0.8067\\
0.8838	0.7767\\
0.8888	0.7767\\
0.8986	0.7792\\
0.9101	0.8109\\
0.9237	0.7889\\
0.9310	0.7889\\
0.9414	0.7902\\
0.9531	0.7855\\
0.9635	0.7330\\
0.9752	0.7959\\
0.9873	0.8632\\
0.9935	0.8632\\
1.0059	0.8822\\
1.0143	0.8700\\
1.0260	0.8771\\
1.0410	0.8514\\
1.0470	0.8514\\
1.0568	0.8544\\
1.0683	0.8034\\
1.0804	0.7374\\
1.0880	0.7374\\
1.0978	0.7901\\
1.1094	0.8272\\
1.1211	0.8336\\
1.1312	0.8431\\
1.1432	0.8620\\
1.1553	0.8564\\
1.1621	0.8564\\
1.1719	0.9169\\
1.1834	0.9789\\
1.1965	0.9688\\
1.2110	1.0161\\
1.2148	1.0161\\
1.2259	0.9813\\
1.2351	0.9158\\
1.2469	0.8873\\
1.2625	0.8309\\
1.2677	0.8309\\
1.2793	0.8670\\
1.2891	0.9342\\
1.3008	0.9558\\
1.3132	0.9880\\
1.3252	0.9497\\
1.3320	0.9497\\
1.3438	0.9748\\
1.3540	0.9740\\
1.3663	1.0288\\
1.3812	1.0001\\
1.3869	1.0001\\
1.3966	1.0155\\
1.4064	0.9972\\
1.4181	1.0365\\
1.4332	1.0950\\
1.4394	1.0950\\
1.4493	1.1188\\
1.4610	1.0971\\
1.4734	1.1550\\
1.4805	1.1550\\
1.4922	1.1310\\
1.5039	1.1029\\
1.5158	1.1272\\
1.5264	1.1225\\
1.5417	1.1307\\
1.5469	1.1307\\
1.5566	1.1586\\
1.5710	1.1806\\
1.5783	1.1806\\
1.5880	1.1586\\
1.5996	1.1332\\
1.6094	1.0939\\
1.6235	1.1608\\
1.6310	1.1608\\
1.6426	1.1693\\
1.6524	1.1543\\
1.6658	1.1412\\
1.6738	1.1412\\
1.6836	1.0851\\
1.6953	1.1239\\
1.7090	1.2438\\
1.7148	1.2438\\
1.7247	1.2463\\
1.7369	1.1592\\
1.7525	1.1255\\
1.7578	1.1255\\
1.7669	1.1676\\
1.7787	1.1780\\
1.7955	1.1701\\
1.7988	1.1701\\
1.8107	1.1351\\
1.8250	1.2360\\
1.8302	1.2360\\
1.8399	1.2428\\
1.8509	1.2431\\
1.8639	1.2425\\
1.8713	1.2425\\
1.8830	1.2857\\
1.8927	1.2441\\
1.9040	1.2111\\
1.9203	1.2033\\
1.9240	1.2033\\
1.9363	1.1364\\
1.9485	1.2368\\
1.9602	1.2612\\
1.9669	1.2612\\
1.9766	1.1747\\
1.9874	1.1725\\
2.0027	1.1415\\
2.0078	1.1415\\
2.0197	1.1413\\
2.0292	1.1288\\
2.0402	1.2747\\
2.0553	1.2756\\
2.0626	1.2756\\
2.0724	1.2063\\
2.0846	1.1750\\
2.0971	1.1554\\
2.1037	1.1554\\
2.1142	1.2506\\
2.1257	1.2631\\
2.1393	1.2580\\
2.1466	1.2580\\
2.1563	1.1501\\
2.1672	1.1085\\
2.1796	1.0936\\
2.1877	1.0936\\
2.1975	1.0825\\
2.2083	1.2184\\
2.2223	1.2026\\
2.2306	1.2026\\
2.2403	1.0586\\
2.2545	1.0212\\
2.2671	0.9753\\
2.2717	0.9753\\
2.2834	0.9299\\
2.2960	1.0140\\
2.3080	1.0273\\
2.3145	1.0273\\
2.3249	0.9913\\
2.3373	1.0510\\
2.3543	1.1269\\
2.3576	1.1269\\
2.3692	1.0668\\
2.3780	1.0639\\
2.3935	1.0565\\
2.4003	1.0565\\
2.4102	1.0680\\
2.4218	1.0609\\
2.4346	1.0842\\
2.4467	1.1284\\
2.4525	1.1284\\
2.4638	1.1600\\
2.4754	1.1515\\
2.4881	1.1532\\
2.4962	1.1532\\
2.5060	1.1612\\
2.5169	1.1097\\
2.5342	1.1615\\
2.5391	1.1615\\
2.5508	1.1611\\
2.5633	1.2168\\
2.5703	1.2168\\
2.5820	1.1848\\
2.5918	1.1494\\
2.6045	1.0685\\
2.6163	1.0145\\
2.6309	1.0262\\
2.6348	1.0262\\
2.6450	1.0470\\
2.6575	1.0232\\
2.6660	1.0232\\
2.6758	1.0300\\
2.6870	1.0056\\
2.7040	1.0058\\
2.7091	1.0058\\
2.7189	1.0361\\
2.7298	0.9630\\
2.7398	0.9675\\
2.7573	0.9113\\
2.7617	0.9113\\
2.7750	0.9113\\
2.7832	0.9113\\
2.7921	0.9587\\
2.8039	1.0138\\
2.8160	1.1222\\
2.8242	1.1222\\
2.8341	1.1389\\
2.8448	1.1012\\
2.8568	1.0862\\
2.8652	1.0862\\
2.8750	1.1271\\
2.8863	1.1213\\
2.9031	1.1041\\
2.9083	1.1041\\
2.9181	1.1006\\
2.9278	1.0785\\
2.9421	1.0677\\
2.9492	1.0677\\
2.9609	1.0634\\
2.9707	1.0629\\
2.9833	1.0611\\
2.9957	1.0504\\
3.0020	1.0504\\
3.0124	1.0218\\
3.0244	1.0059\\
3.0356	0.9832\\
3.0518	0.9522\\
3.0568	0.9522\\
3.0666	0.9634\\
3.0761	0.9334\\
3.0871	0.8745\\
3.1024	0.7870\\
3.1076	0.7870\\
3.1191	0.7729\\
3.1289	0.7510\\
3.1388	0.7287\\
3.1530	0.7621\\
3.1601	0.7621\\
3.1698	0.7346\\
3.1816	0.7221\\
3.1926	0.6997\\
3.2038	0.7091\\
3.2159	0.7691\\
3.2227	0.7691\\
3.2324	0.7803\\
3.2441	0.8171\\
3.2548	0.8693\\
3.2668	0.8774\\
3.2818	0.9741\\
3.2852	0.9741\\
3.2970	1.0373\\
3.3066	1.1053\\
3.3210	1.1301\\
3.3283	1.1301\\
3.3398	1.1832\\
3.3496	1.1923\\
3.3640	1.0653\\
3.3712	1.0653\\
3.3809	1.0250\\
3.3934	0.9231\\
3.4055	0.9375\\
3.4120	0.9375\\
3.4219	0.9432\\
3.4341	0.9016\\
3.4468	0.8422\\
3.4531	0.8422\\
3.4639	0.6137\\
3.4760	0.5257\\
3.4844	0.5257\\
3.4962	0.5397\\
3.5059	0.5219\\
3.5172	0.5339\\
3.5335	0.5860\\
3.5384	0.5860\\
3.5482	0.6509\\
3.5605	0.6378\\
3.5749	0.6046\\
3.5801	0.6046\\
3.5918	0.8380\\
3.6025	0.9148\\
3.6145	1.0505\\
3.6264	1.0319\\
3.6328	1.0319\\
3.6436	1.0497\\
3.6559	1.0799\\
3.6641	1.0799\\
3.6758	1.1786\\
3.6852	1.1994\\
3.6987	1.1659\\
3.7070	1.1659\\
3.7168	1.3426\\
3.7315	1.3143\\
3.7384	1.3143\\
3.7481	1.2692\\
3.7598	1.2948\\
3.7691	1.3877\\
3.7863	1.4425\\
3.7910	1.4425\\
3.8015	1.2414\\
3.8138	1.0779\\
3.8223	1.0779\\
3.8340	0.9811\\
3.8438	0.6811\\
3.8547	0.3811\\
3.8667	0.0811\\
3.8751	0.0811\\
3.8869	-0.2189\\
3.8986	-0.4579\\
3.9071	-0.4962\\
3.9228	-0.4911\\
3.9279	-0.4911\\
3.9395	-0.4805\\
3.9493	-0.4267\\
3.9618	-0.2209\\
3.9739	-0.0102\\
3.9857	0.0478\\
3.9922	0.0478\\
4.0031	0.3478\\
4.0152	0.6478\\
4.0234	0.6478\\
4.0333	0.9478\\
4.0439	1.2478\\
4.0557	1.5478\\
4.0678	1.8478\\
4.0762	1.8478\\
4.0860	2.1478\\
4.0967	2.4136\\
4.1113	2.7136\\
4.1173	2.7136\\
4.1291	3.0136\\
4.1388	3.3136\\
4.1495	3.6058\\
4.1636	3.9058\\
4.1719	3.9058\\
4.1817	3.6058\\
4.1934	3.3058\\
4.2038	3.0058\\
4.2155	2.7058\\
4.2227	2.7058\\
4.2337	2.4058\\
4.2463	2.1058\\
4.2559	1.8058\\
4.2663	1.5058\\
4.2812	1.2058\\
4.2873	1.2058\\
4.2989	0.9058\\
4.3088	0.6058\\
4.3187	0.3058\\
4.3322	0.0058\\
4.3399	0.0058\\
4.3516	-0.2942\\
};
\end{axis}
\end{tikzpicture}%
	\caption{Height above ground, vertical speed, and divergence measurements with ground truth during a constant divergence landings performed at three different divergence setpoints. In the bottom graph, the dotted, dashed, and solid lines represent the setpoint, ground truth, and estimate for $\vartheta_z$ respectively.}
	\label{fig:const_div_landing_1}
\end{figure}

In practice, the visual observable estimator thread running on the Lisa/M microprocessor does not maintain its target frequency of 100 Hz with an optical flow measurement rate $\rho_{F_{max}}$ of 2500 events per second. Instead, it drops to around 75 Hz during the landing maneuvers. However, given the limited processing power of the microprocessor, this is still a decent result. It well exceeds sampling rates seen in recent frame-based optical flow estimation pipelines, which are in the order of 15 to 25 Hz \cite{Herisse2012,Alkowatly2015,Ho2016a,DeCroon2016}. Also, with a lower setting of $\rho_{F_{max}}$ (around 2000 optical flow events per second), the target frequency of 100 Hz is well attainable. The Odroid can transmit up to approximately 8500 optical flow events per second over the UART connection, limited by the baud rate of 921.6 kB per second.

For the largest part, the maneuvers are executed successfully, even for high divergence values setpoints. With $\vartheta_{z_r}=1.0$, the MAV performs a rapid maneuver, descending from a height of 3.5 m to 1 m within 1.79 s. In comparable recent experiments with frame-based cameras for divergence measurement \cite{Ho2016a}, landings were performed up to $\vartheta_{z_r}=0.3$. Since higher values have not been attempted in these experiments, we cannot know for certain that frame-based optical flow is not applicable to such high speeds\footnote{It is planned to perform some landing tests outdoors in an unmodified outdoor environment to test the landing at even higher rates of divergence to explore the limits of our approach. The results of those tests may be included in a revised version of this paper.}.
%A time delay is also observed, which differs between datasets. By examining the cross-correlation functions of the estimate and ground truth signals, time delays of 0.05 s, 0.04 s, and 0.10 s are observed for the respective signals. A likely cause for this is the UART interface between the Odroid and the Lisa/M boards,
