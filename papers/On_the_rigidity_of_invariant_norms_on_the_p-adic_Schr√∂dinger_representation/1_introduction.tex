
%%%%%%%%%%%%%%%%%%%%%%%%%%%%%%%%%%%%---------------------------
\section{Introduction}
%%%%%%%%%%%%%%%%%%%%%%%%%%%%%%%%%%%%---------------------------
Choose a non-trivial smooth character $\psi:(\Q_p,+)\map \C_p^\times$.
The $\C_p$-valued Haar distribution $dt$ on $\Q_p$ is not a measure.
As a result, we cannot define the integral $\intop f(t)\ dt$ for a general continuous function $f:\Q_p\map\C_p$, even if $f$ is compactly supported.
However, we can define integration for locally constant functions with compact support, and we denote by $\schw(\Q_p)$ the space of all such functions.

The Fourier transform of $f\in \schw(\Q_p)$ is defined by
\[\reallywidehat{f}(x)=\intop_{\Q_p}\psi(xt)f(t)\ dt.\]

The Fourier transform is not continuous in the sup norm.
In \cite{ophir2016q}, we showed that the Fourier transform is "as discontinuous as it can get" in the sense that the graph
\[\Gamma=\braces{(f,\reallywidehat{f})\ |\ f\in\schw(\Q_p)}\]
is dense in $C_0(\Q_p)\times C_0(\Q_p)$.
Here, $C_0(\Q_p)$ is the space of continuous functions that go to zero at infinity, the completion of $\schw(\Q_p)$ in the sup norm. 
The proof in \cite{ophir2016q} went by restricting the Fourier transform to some finite dimensional subspaces and used a special decomposition of the Fourier transform on these subspaces. 
It also used $q$-arithmetic, but in an entirely different way than the way $q$-arithmetic is used in the present paper.

To motivate the results in this paper, we describe two other approaches to the discontinuity of the Fourier transform.
By introducing the Heisenberg group and the Schrödinger representation, we can reformulate the above result in terms of invariant norms.

The Heisenberg group $\heis_3(\Q_p)$ is the group of unipotent matrices
\[\heis_3(\Q_p)=\braces*{
\begin{pmatrix}
1 & a & t\\
0 & 1 & b\\
0 & 0 & 1
\end{pmatrix}
}\subset GL_3(\Q_p).\]
We denote its elements by $[a,b,t]$.
The (smooth) Schrödinger representation $\rho_\psi:\heis_3(\Q_p)\map GL(\schw(\Q_p))$, attached to the character $\psi$ is defined by
\[\sog{\rho_\psi([a,b,t])f}(x)=\psi\sog{t+\frac{ab}{2}}\cdot \psi(bx)\cdot f(x+a).\]
The representation $\rho_\psi$ is irreducible and the Stone-von Neumann theorem says that, up to isomorphism, $\rho_\psi$ is the unique smooth irreducible representation of $\heis_3(\Q_p)$ with central character $\psi$.

The sup norm is invariant under the action of the Heisenberg group and so is the norm $\wedgenorm{f}:=\supnrm{\reallywidehat{f}}$.
We remark that these two norms are not equivalent.

Let $\Lambda$ and $\reallywidehat{\Lambda}$ be the closed unit balls of $\supnorm{}$ and $\wedgenorm{}$ respectively.
It is an easy exercise to prove that the following are equivalent.
    \begin{enumerate}
    \item The graph $\Gamma$ is dense in $C_0(\Q_p)\times C_0(\Q_p)$.
    \item $\Lambda+\reallywidehat{\Lambda}=\schw(\Q_p)$.
    \item There exists no $\heis_3(\Q_p)$-invariant norm that is smaller than both $\supnorm{}$ and $\wedgenorm{}$.
    \end{enumerate}

It turns out that $(3)$ is true because $\supnorm{}$ (and likewise $\wedgenorm{}$) is a minimal $\heis_3(\Q_p)$-invariant norm.
In addition, it has a surprising rigidity.
This is the content of Theorem \ref{thm_strong_minimality}, which in this case says the following.

\begin{theorem*}
Let $\norm{}$ be an $\heis_3(\Q_p)$-invariant norm on $\schw(\Q_p)$ that is dominated by the sup norm (i.e. $\norm{}\leq c\cdot \supnorm{}$ for some $c>0$).
Then there exists $r>0$ such that $\norm{}=r\cdot \supnorm{}$.
\end{theorem*} 
Clearly, $(3)$  follows.


Yet another way to approach the question of the density of the graph $\Gamma$ is to consider the intersection
\[W=\overline{\Gamma}\cap\sog{C_0(\Q_p)\times\braces{0}}.\]
Viewing $W$ as a subspace of $C_0(\Q_p)$, it is a closed subspace and is invariant by the action of the Heisenberg group.
In \cite{demathan}, Fresnel and de Mathan constructed a non-zero element in $W$.
Thus, showing that $C_0(\Q_p)$ is topologically irreducible as a representation of $\heis_3(\Q_p)$ gives another proof of the density of $\Gamma$.
In this paper we prove that $C_0(\Q_p)$ is topologically irreducible.
In fact, we will show (Proposition \ref{topologically_irreducible}) that a stronger notion of irreducibility  holds for $C_0(\Q_p)$ (see Definitions \ref{def_strong_irreducible} and \ref{def_srongly_cyclic_vector}).

The results of this paper are more general than the above discussion in two ways.
First, we work with the group $(\Q_p^d,+)$, where $d\geq 1$ is an integer, and correspondingly, with higher dimensional Heisenberg groups.
Second, we consider all the intertwining operators on the Schrödinger representation, among which the Fourier transform is just a single example.
This allows us to study simultaneous continuity of any finite number of intertwining operators (see section $7$).

The methods of the proofs are of two types.
There are general results on Banach representations and $p$-adic functional analysis.
These are contained in section $3$.
The other type is $q$-arithmetic.
More precisely, we use $q$-Mahler bases in $C(\Z_p)$ and $p$-adic evaluations of some $q$-analog expressions in order to study norms on $C(\Z_p)$.

By using the results of section $3$, it can be shown that the local maximality (Definition \ref{def_local_maximality}) of the sup norm on $C(\Z_p)$ with respect to multiplication by smooth characters is equivalent to Theorem $2$ in \cite{demathan}.
In Section \ref{subsection_new_proof_Fresnel_de_Mathan} we use our methods to give a new proof of the main results in \cite{demathan}.
Our proof, using $q$-arithmetic, can be generalized to include the case where $\psi:(\Q_p,+)\map \C_p^\times$ is continuous but not smooth, and this case does not follow from \cite{demathan}.
These results will appear in a forthcoming paper.

We remark that the completions that we study of the Schrödinger representation are large in the sense that the reduction of their unit ball modulo the maximal ideal of $\mathcal{O}_{\C_p}$ is a non-admissible smooth representation over $\overline{\fld{F}}_p$.