%%%%%%%%%%%%%%%%%%%%%%%%%%%%%%%%%%%%---------------------------
\section{Proof of Theorem \ref{thm_strong_minimality_Z_p} for $\mathbb{Z}_p$}
%%%%%%%%%%%%%%%%%%%%%%%%%%%%%%%%%%%%---------------------------

In this section we prove Theorem \ref{thm_strong_minimality_Z_p} for $\Z_p$.
We begin by noting that in the formulation of Theorem \ref{thm_strong_minimality_Z_p}, the space $\schw(\Z_p)$ can be replaced by $C(\Z_p)$, which is its completion in the sup norm.
In this section we will work with $C(\Z_p)$ since this allows us to use functions, such as polynomials, which are not in $\schw(\Z_p)$.

Clearly, Theorem \ref{thm_strong_minimality_Z_p} follows if we know that $\supnorm{}$ is both weakly minimal and locally maximal at $\textbf{1}_{\Z_p}(x)$.
That the sup norm is weakly minimal at $\textbf{1}_{\Z_p}(x)$ follows from Proposition \ref{prop_weak_minimality}.
Thus, it remains to show local maximality.
        
The proof uses two main ingredients:
    \begin{enumerate}
    \item The \textbf{growth modulus} of a norm.
    This is a real valued function associated with norms on $C(\Z_p)$ that are dominated by the sup norm.
    \item The \textbf{$q$-Mahler bases}.
    To each $q\in \C_p$ with $\pabs{q-1}<1$, there corresponds a basis of $C(\Z_p)$ called the $q$-Mahler basis which shares some nice properties with the Mahler basis: $\braces*{\binom{x}{n}\ |\ n\geq 0}$.
    The $q$-Mahler bases can be viewed as a family of deformations of the Mahler basis.
    \end{enumerate}


%%%%%%%%%%%%%%---------------
\subsection{The growth modulus of a norm}
%%%%%%%%%%%%%%---------------
The beginning of this section is an adaptation of \cite{robert2013course}, chapter $6$, part $1.4$.

Let $(a_n)_{n=0}^\infty$ be a bounded sequence of non-negative real numbers.
The growth modulus associated with the sequence $(a_n)_{n=0}^\infty$ is the function
\[r\mapsto\sup_na_nr^n\]
defined on the interval $[0,1]$.
It is a continuous, non-decreasing and convex function (part of the \textbf{Classical Lemma} on \cite{robert2013course}, p.292).

We say that a real number $0<r<1$ is regular with respect to the sequence $(a_n)_{n=0}^\infty$ if there exists $n$ such that $a_nr^n>a_mr^m$ for any $m\neq n$.
Otherwise, we call $r$ a critical value (with respect to the sequence).

We remark that if $r$ is a regular value and $a_nr^n>a_mr^m$ for any $m\neq n$, there exists some interval containing $r$ on which the growth modulus is equal to $a_nx^n$.
In particular the growth modulus is smooth at the regular values.

The fundamental lemma about critical values is the following.
\begin{lem}\label{discreteness_of_critical_values}
Assume that $(a_n)_{n=0}^\infty$ is not the zero sequence.
The set of critical values is discrete in $[0,1)$. 
\end{lem} 
    \begin{proof}
    Let $0<r<1$.
    We will show that there are only finitely many critical values smaller than $r$.
    Let $n$ be such that $a_nr^n\geq a_mr^n$ for any $m$.
    Note that $a_n\neq 0$.
    Let $0<s<r$.
    Then for any $N>n$
    \[a_nr^n\geq a_Nr^N \gorer \frac{a_N}{a_n}\leq r^{n-N}<s^{n-N} \gorer a_ns^n>a_Ns^N.\]
    Thus, if $s<r$ is a critical value and $i,j\in\Z_{\geq0}$ are such that $a_is^i=a_js^j\geq a_ks^k$ for any $k$, then $i,j\leq n$.
    In this case $a_j\neq 0$ and
    \[s=\sog{\frac{a_i}{a_j}}^{\frac{1}{j-i}},\ \ \ i,j\leq n.\]
    There are only finitely many such values.   
    \end{proof} 
%    \begin{exmp}
%    The growth modulus of the sequence $\frac{n}{n+1}$ has infinitely many critical values.
%    The critical values are at the points $\frac{n(n+2)}{(n+1)^2}$.
%    The following picture shows the graph of the growth modulus of the sequence $\frac{n}{n+1}$ together with the first six critical values.
%    \begin{center}
%    \includegraphics[width=8cm]{Growth_modulus_1}
%    \end{center}
%    \end{exmp} 


Until the end of this subsection, fix a norm $\norm{}$ on $C(\Z_p)$ that is dominated by the sup norm and normalized at $\textbf{1}_{\Z_p}(x)$.
Let $\binom{x}{n}$ be the $n$-th binomial polynomial.
Under the assumptions on $\norm{}$, the sequence $\sog{\norm{\binom{x}{n}}}_{n\geq 0}$ is bounded. 
We define the growth modulus of the norm $\norm{}$ to be the growth modulus of that sequence.
We denote the growth modulus of $\norm{}$ by $G_{\norm{}}(r)$.
Explicitly, $G_{\norm{}}(r):[0,1]\map\R$ is the function 
\[G_{\norm{}}(r)=\sup_{n\geq 0}\sog{\norm{\binom{x}{n}}\cdot r^n}.\]
We call $r\in[0,1]$ a regular (resp. critical) value for the norm $\norm{}$ if it is regular (resp. critical) with respect to the sequence $\sog{\norm{\binom{x}{n}}}_{n\geq 0}$.

The connection between the growth modulus of $\norm{}$ and the study of the norm itself comes from the work of Mahler.
We recall the basic facts about the Mahler basis.

\begin{thm}[Mahler, \cite{Mahler}]
Any $f\in C(\Z_p)$ can be written as 
\[f(x)=\sum_{n=0}^\infty a_n\cdot \binom{x}{n}\]
where $\limit{n}a_n=0$ and the sum converges to $f$ in the sup norm.
Moreover, $\supnorm{f}=\max_{n}\pabs{a_n}$.
\end{thm} 

The following proposition immediately follows.
\begin{prop}
Let $M$ be the smallest number such that $\norm{}\leq M\cdot \supnorm{}$.
Then $M=G_{\norm{}}(1)$.
\end{prop} 
We conclude this subsection with the following proposition.
\begin{prop}\label{powers_and_growth_modulus}
Let $\norm{}$ be a norm on $C(\Z_p)$ dominated by the sup norm.
Let $q\in \C_p $ with $r:=\abs{q-1}_p<1$.
Then $\norm{q^x}\leq G_{\norm{}}(r)$.
Moreover, if $r$ is a regular value for the norm $\norm{}$ then
\[\norm{q^x}=G_{\norm{}}(r).\]
\end{prop} 
    \begin{proof}
    This is a simple consequence of the non-archimedean triangle inequality.
    Using the Mahler expansion of the function $q^x$
        \begin{equation}\label{equation_growth_modulus_proof}
        \norm{q^x}=\norm{\sum_{k=0}^\infty(q-1)^k\binom{x}{k}}\leq \sup_{k \geq 0} \sog{\abs{q-1}_p^k\norm{\binom{x}{k}}}=G_{\norm{}}(r).  
        \end{equation} 
    If $r=\abs{q-1}_p$ is a regular value, there exists $n\geq 0$ such that 
    \[r^n\norm{\binom{x}{n}}>r^m\norm{\binom{x}{m}}\]
    for any $m\neq n$, and therefore we have an equality instead of inequality in \ref{equation_growth_modulus_proof}.
    \end{proof}
    \begin{exmp}
    $\ $
        \begin{enumerate}
        \item The growth modulus of the sup norm is constant $G_{\norm{}}(r)=1$.
        \item Assume that $\norm{}$ is invariant by multiplication by smooth characters and that $G_{\norm{}}(1)>1$.
        Then, for any $N$ large enough and $\zeta$ a root of unity of order $p^N$, $r=\pabs{\zeta-1}$ is a critical value for the norm $\norm{}$.
        Indeed, for $N$ large enough, $G_{\norm{}}(r)>1$ while $\norm{\zeta^x}=1$, so $r$ is a critical value by the previous proposition.
        \end{enumerate}
    \end{exmp} 

    \begin{remark}
    In general, $\norm{q^x}$ is not a function of $\abs{1-q}_p$, i.e. it might be the case that $\norm{q_1^x}\neq\norm{q_2^x}$ while $\abs{q_1-1}_p=\abs{q_2-1}_p$.
    \end{remark} 

%%%%%%%%%%%%%%---------------
\subsection{$q$-Mahler bases}
%%%%%%%%%%%%%%---------------
We briefly recall the $q$-analog terminology, the $q$-Mahler bases and the expansion formula for exponents in these bases.
This subsection is self contained.
For a more thorough exposition to the $q$-analog formalism and its properties we refer to \cite{q_calculus,special_functions}.
For more on the $q$-analog of the Mahler basis see \cite{conrad2000q}.

Let $q$ be an indeterminate.
The $q$-analog of the natural number $n$ is the following expression in $\Z[q]$
\[[n]_q=\frac{1-q^n}{1-q}=1+q+...+q^{n-1}.\]
The $q$-analog of the factorial of $n$ is 
\[[n]_q!=[1]_q\cdot [2]_q\cdot ....\cdot [n]_q\]
and the $q$-binomial coefficients, also known as Gaussian binomial coefficients, are defined by the analogous formula
\[\qbinom{n}{k}{q}=\frac{[n]_q!}{[k]_q!\cdot [n-k]_q!}\]
whenever $0\leq k\leq n$, and zero otherwise.
The $q$-Pochhammer symbol is the expression
\[(a;q)_n=\prod_{i=0}^{n-1}(1-aq^i).\]
When $a=q$ we get
\[(q;q)_n=\prod_{i=1}^{n}(1-q^i).\]
By expanding the terms in the definition, it is easy to verify that
\[\qbinom{n}{k}{q}=\frac{(q;q)_n}{(q;q)_{k}(q;q)_{n-k}}.\]

The $q$-Pascal identity
\begin{equation}\label{q_Pascal}
\qbinom{n+1}{k+1}{q}=\qbinom{n}{k+1}{q}+q^{n-k}\cdot\qbinom{n}{k}{q},
\end{equation}
implies, by induction, that $\qbinom{n}{k}{q}$ is a polynomial in $q$ with integer coefficients.

From now on $q$ will not be an indeterminate but an element in $\C_p$ such that $\abs{q-1}_p<1$.
The map $n\mapsto \qbinom{n}{k}{q}$ is continuous with respect to the $p$-adic topologies on $\Z$ and on $\C_p$, and therefore extends to a map $x\mapsto \qbinom{x}{k}{q}$ that lies in $C(\Z_p)$.

Since for any $x\in \N$ the expression $\qbinom{x}{k}{q}$ is a polynomial with integral coefficients in $q$, we have $\pabs{\qbinom{x}{k}{q}}\leq 1$.
By continuity, $\supnorm{\qbinom{x}{k}{q}}\leq 1$.
Substituting $x=k$ we see that 
\[\supnorm{\qbinom{x}{k}{q}}=1.\]

Note that if  $q$ is not a root of unity the term $(q;q)_k$ is non-zero for any $k$, so
\[\qbinom{x}{k}{q}=\frac{(1-q^{x-(k-1)})\cdot(1-q^{x-(k-2)})\cdot...\cdot(1-q^x)}{(1-q)(1-q^2)...(1-q^k)}.\]

We will need two results about $q$-binomial functions.
The first is the $q$-analog of Mahler's theorem.
The second is the expansion of an exponent $\zeta^x$ with respect to the $q$-Mahler basis.
Both results appear in \cite{conrad2000q}, the first is a combination of Theorem 3.3 and Theorem 4.1, and the second is the example at the beginning of page 14.
For completeness we will prove both results.
\begin{thm}\label{q_Mahler}
Let $q\in\C_p$ with $\abs{q-1}_p<1$.
Then for any function $f\in C(\Z_p)$ there exists a unique sequence $(a_n)_{n=0}^\infty$ of numbers in $\C_p$ such that the series
\[\sum_{k=0}^\infty a_k\qbinom{x}{k}{q}\]
converges in the sup norm to $f$ (in particular $\limit{k}a_k=0$).
Moreover,
\[\supnorm{f}=\max_{k\geq 0}\pabs{a_k}.\]
\end{thm} 
    \begin{proof}
    Consider the operator $T=\frac{\Delta}{q^x}$ on $C(\Z_p)$, where $\Delta$ is the forward difference operator.
    Thus,
    \[Tf(x)=\frac{f(x+1)-f(x)}{q^x}.\]
    We begin by showing that for any $f\in\C(\Z_p)$, the sequence $\sog{T^nf(0)}_{n=0}^\infty$ converges to zero.
    Afterwards we will construct the sequence $(a_n)_{n=0}^\infty$ from $\sog{T^nf(0)}_{n=0}^\infty$.
    
    Recall that $\pabs{q-1}<1$, and denote $r=\pabs{q-1}$.
    Denote $r=\abs{q-1}_p$ and recall the assumption that $r<1$.
    We consider the quotient space
    \[W=\quot{\braces{f\in C(\Z_p)\ |\ \supnorm{f}\leq 1}}{\braces{f\in C(\Z_p)\ |\ \supnorm{f}\leq r}}.\]
    Its elements can be realized as locally constant functions on $\Z_p$ with values in $\mathcal{O}_{\C_p}/(q-1)\mathcal{O}_{\C_p}$.
    Since the operator $T$ is norm reducing, i.e. $\supnorm{T(f)}\leq \supnorm{f}$ for any $f\in C(\Z_p)$, $T$ induces an operator on $W$.
    Since the image of $q^x$ in $W$ is the constant function $1$, the operator $T$ reduces in $W$ to the forward difference operator $\Delta$.
    If $v\in W$, there exists some number $N$ such that $\Delta^{p^N}v= 0$.
    Thus, for any functions $f\in C(\Z_p)$ there exists $N>0$ such that 
    \[\supnorm{T^{p^N}f}\leq r\cdot \supnorm{f}.\]
    Together with the fact that $T$ is norm reducing, it follows that $\limit{n}\supnorm{T^nf}=0$.
    In particular, $\limit{n}T^nf(0)=0$.    
    
    By rearranging the $q$-Pascal identity \ref{q_Pascal} we get
    
    \[\frac{q^{\binom{k}{2}}\qbinom{n+1}{k}{q}-q^{\binom{k}{2}}\qbinom{n}{k}{q}}{q^n}=q^{\binom{k-1}{2}}\qbinom{n}{k-1}{q}.\]
    Continuity with respect to $n$ implies that \[T\sog{q^{\binom{k}{2}}\qbinom{x}{k}{q}}=q^{\binom{k-1}{2}}\qbinom{x}{k-1}{q}\]
    for any $k\geq 1$.
    When $k=1$ the function $q^0\qbinom{x}{0}{q}$ is just the constant function $1$, and clearly $T(1)=0$.
    
    Let $f\in C(\Z_p)$ and denote $a_n=q^{\binom{n}{2}}(T^nf)(0)$.
    The series 
    \[h(x)=\sum_{k=0}^\infty a_k\qbinom{x}{k}{q}\]
     converges in $C(\Z_p)$, and we have
     \[T^nh(0)=T^nf(0),\]
     for any $n\geq 0$.
    Since $h(0)=f(0)$, it follows that $h(n)=f(n)$ for any $n\geq 0$.
    By continuity we must have $h=f$.
    Thus,
    \[f(x)=\sum_{k=0}^\infty (T^kf)(0)q^{\binom{k}{2}}\qbinom{x}{k}{q}.\]
    This formula implies that $\supnorm{f}\leq\max_{k}\abs{a_k}_p$.
    The inequality in the other direction follows from the fact that $T$ is norm-reducing, so $\abs{a_k}_p=\abs{(T^nf)(0)}_p\leq \supnorm{T^nf}\leq \supnorm{f}$.
    \end{proof} 
\begin{defn}
We denote 
\[\pcoeff{\zeta}{q}_k=(\zeta-1)(\zeta-q^1)...(\zeta-q^{k-1})=(-1)^k\cdot q^{\binom{k}{2}}\cdot (\zeta;q^{-1})_k,\]
for $k>0$ and $\pcoeff{\zeta}{q}_0=1$.


\end{defn} 
\begin{cor}\label{q_expansion_of_powers}
Let $\zeta,q\in\C_p$ with $\abs{q-1}_p<1$ and $\abs{\zeta-1}_p<1$.
Then
\[\zeta^x=\sum_{k=0}^\infty \pcoeff{\zeta}{q}_k\cdot\qbinom{x}{k}{q}.\]
\end{cor} 
    \begin{proof}
    We have $T^0(\zeta^x)(0)=\zeta^0=1=[\zeta,q]_0$.
    Compute 
    \[T(\zeta^x)=\frac{\zeta^{x+1}-\zeta^x}{q^x}=(\zeta-1)\sog{\frac{\zeta}{q}}^x.\]
    By induction:
    \[T^k(\zeta^x)=(\zeta-1)(\frac{\zeta}{q}-1)...(\frac{\zeta}{q^{k-1}}-1)\sog{\frac{\zeta}{q^k}}^x.\]
    By the proof of Theorem \ref{q_Mahler}, the coefficient of $\qbinom{x}{k}{q}$ in the expansion of $\zeta^x$ is 
    \[q^{\binom{k}{2}}\cdot (T^kf)(0)=q^{\binom{k}{2}}(\zeta-1)(\frac{\zeta}{q}-1)...(\frac{\zeta}{q^{k-1}}-1)\sog{\frac{\zeta}{q^k}}^0=(\zeta-1)(\zeta-q)...(\zeta-q^{k-1}).\]
    \end{proof} 



%%%%%%%%%%%%%%
\subsection{The $p$-adic valuation of $(\zeta;\zeta)_n$ when $\zeta$ is a root of unity}
%%%%%%%%%%%%%%
Fix $N\in \N$ and let $\zeta$ be a primitive $p^{N}-th$ root of unity in $\fld{C}_p$. 
In this subsection we study the $p$-adic valuation of the expression
\[(\zeta;\zeta)_n=(1-\zeta)(1-\zeta^2)...(1-\zeta^n)\]
for $1\leq n<p^N$.
It will be convenient to denote $\lambda=-\log_p(\pabs{\zeta-1})$, so $\lambda=\frac{1}{p^{N-1}(p-1)}$.

The main goal is to prove the following result, which will be used later.
\begin{prop}\label{q_analog_evaluations_cor}
For any $p^8\leq n<p^N$,
\[\log_p(\abs{(\zeta;\zeta)_n}_p)\leq -\frac{\lambda}{4}n\log_p(n).\]
\end{prop} 

\begin{defn}
For a positive integer $n$ we define $\beta_p(n)$ to be
\[
\beta_p(n)=
\sum_{k=0}^{\infty}p^k\sog{\floor*{\frac{n}{p^k}}-\floor*{\frac{n}{p^{k+1}}}}.
\]
\end{defn}
Note that for any $n$ this sum is finite.

\begin{prop}\label{beta_function}
For any $1\leq n<p^N$ 
\[\log_p(\abs{(\zeta;\zeta)_n}_p)=-\lambda\cdot \beta_p(n).\]
\end{prop}

\begin{proof}
For any $1\leq m=ap^k<N$, where $p\nmid a$, we have
\[\log_p\sog{\pabs{\zeta^m-1}}=\log_p\sog{\pabs{\zeta^{p^k}-1}}=-\frac{1}{p^{N-k-1}(p-1)}=-p^k\cdot \lambda.\]
There are $\floor*{\frac{n}{p^k}}-\floor*{\frac{n}{p^{k+1}}}$ numbers between $1$ and $n$ that are divisible by $p^k$ but not by $p^{k+1}$.
Thus,
\begin{align*}
    \log_p\sog{\absolute*{(\zeta;\zeta)_n}_p}
    &=\sum_{i=1}^n\log_p\sog{\absolute*{1-\zeta^i}_p}
    =\sum_{k=0}^\infty\sog{ \floor*{\frac{n}{p^k}}-\floor*{\frac{n}{p^{k+1}}} }\cdot \log_p\sog{\absolute*{1-\zeta^{p^k}}_p}\\
    &=\sum_{k=0}^\infty\sog{ \floor*{\frac{n}{p^k}}-\floor*{\frac{n}{p^{k+1}}} }\cdot \sog{-p^k\lambda}
    =-\lambda\cdot\beta_p(n).
    \end{align*} 
\end{proof}

\begin{prop}
For every $n$
\[
\beta_p(n)\geq n\log_p(n)\cdot \frac{p-1}{p}-\frac{np}{p-1}.
\]
\end{prop}

\begin{proof}
Denote $d=\floor{\log_p(n)}$.
Then
   \begin{align*}
    \beta_p(n)
    &=\sum_{k=0}^{d}p^k\sog{\floor*{\frac{n}{p^k}}-\floor*{\frac{n}{p^{k+1}}}}
    =\sum_{k=0}^{d-1}p^k\sog{\floor*{\frac{n}{p^k}}-\floor*{\frac{n}{p^{k+1}}}}+p^d\floor*{\frac{n}{p^d}}\\
    &\geq\sum_{k=0}^{d-1}p^k\sog{\frac{n}{p^k}-1-\frac{n}{p^{k+1}}}+p^d\sog{\frac{n} {p^d}-1}\\
    &=\sum_{k=0}^{d-1}\sog{n-p^k-\frac{n}{p}}+n-p^d
    =dn\sog{1-\frac{1}{p}}+n-\sum_{k=0}^{d}p^k\\
    &=dn\frac{p-1}{p}+n-\frac{p^{d+1}-1}{p-1}.
    \end{align*}
Since  $n\geq p^d$ and $d> \log_p(n)-1$,
    \begin{align*}
    dn\frac{p-1}{p}+n-\frac{p^{d+1}-1}{p-1}
    &\geq \sog{\log_p(n)-1} n\frac{p-1}{p}+n-\frac{pn-1}{p-1}\\
    &=n\log_p(n) \frac{p-1}{p}-n+\frac{n}{p}+n-\frac{pn-1}{p-1}\\
    &<  n\log_p(n)\frac{p-1}{p}-\frac{np}{p-1}
    \end{align*} 
\end{proof}


    \begin{proof}[Proof of Proposition \ref{q_analog_evaluations_cor}]
    Write
    \[\beta_p(n)\geq n\log_p(n)\cdot\frac{p-1}{p}-n\cdot\frac{p}{p-1}
    =\frac{1}{2}\sog{n\log_p(n)\cdot\frac{p-1}{p}}+\frac{1}{2}\sog{n\log_p(n)\cdot\frac{p-1}{p}}-n\cdot\frac{p}{p-1}.\]
    If $n\geq p^8$, then
    \[\frac{1}{2}\sog{n\log_p(n)\cdot\frac{p-1}{p}}-n\cdot\frac{p}{p-1}
    \geq\frac{1}{2} \sog{n\cdot 8\cdot\frac{p-1}{p}}-n\cdot\frac{p}{p-1}
    =n\cdot\frac{(p-1)(3p-1)}{p(p-1)}\geq 0.\]
    Therefore, 
    \[\beta_p(n)\geq  \frac{1}{2}\cdot\frac{p-1}{2}n\log_p(n)\geq \frac{1}{4}n\log_p(n).\]
    By Proposition \ref{beta_function},
    \[\log_{p}(\abs{(\zeta;\zeta)_n}_p)=-\lambda\cdot\beta_p(n)\leq -\frac{\lambda}{4}n\log_p(n).\]
    \end{proof} 

%%%%%%%%%%%%%%---------------
\subsection{Completing the proof of Theorem \ref{thm_strong_minimality_Z_p} for $\mathbb{Z}_p$}
%%%%%%%%%%%%%%---------------
Let $\norm{}$ be a norm on $C(\Z_p)$, dominated by the sup norm, normalized at $\textbf{1}_{\Z_p}(x)$ and invariant under multiplication by smooth characters of $\Z_p$.
Let $G_{\norm{}}(r)$ be the growth modulus of $\norm{}$.
We suppose that $G_{\norm{}}(1)>1$ and reach a contradiction.

By the assumption that $G_{\norm{}}(1)>1$, the continuity of $G_{\norm{}}(r)$ and the density of regular values (Proposition \ref{discreteness_of_critical_values}), there exists $h\in \C_p$ such that $s:=\abs{h}_p<1$ is a regular value for the norm $\norm{}$ and such that $G_{\norm{}}(s)>1$.
We may also assume that $s\geq p^{-1/(p-1)}$.
The last assumption can be written as $\log_p\sog{\frac{1}{s}}\leq \frac{1}{p-1}$.
We fix such $h$ and denote $s=\abs{h}_p$.
We remark that $h$ and $s$ depend only on the norm $\norm{}$.

From now on, $\zeta$ denotes a primitive $p^N$-th root of unity and $N$ is assumed to be very large (in a way that will be made explicit below).
We denote $\lambda=-\log_p(\pabs{\zeta-1})$ and $q=\zeta+h$.

Thus, $h$ is fixed and $\zeta$ is at our disposal, close as we wish to the circumference of the unit disc around $1$, and $q$ varies with $\zeta$ at a fixed distance $s$ from it.

The idea of the proof is to use the expansion 
\[\zeta^x=\sum_{k=0}^\infty\pcoeff{\zeta}{q}_k\cdot\qbinom{x}{k}{q}\]
to show, under the assumption that $N$ is very large, that
    \begin{equation}\label{main_inequality}
    \norm{\pcoeff{\zeta}{q}_1\cdot\qbinom{x}{1}{q}}
    >\norm{\pcoeff{\zeta}{q}_k\cdot\qbinom{x}{k}{q}}  
    \end{equation} 
for any $k\neq 1$.
Then, by the strong triangle inequality,
\[\norm{\zeta^x}
=\norm{\pcoeff{\zeta}{q}_1\cdot\qbinom{x}{1}{q}}
>\norm{\pcoeff{\zeta}{q}_0\cdot\qbinom{x}{0}{q}}
=\norm{\textbf{1}(x)}
=1\] 
which is a contradiction to the assumption that $\norm{}$ is invariant under multiplication by $\zeta^x$ and normalized at $\textbf{1}_{\Z_p}(x)$.

The proof of \ref{main_inequality} will be divided into three cases.
The first, $k=0$, is the easiest.
The second and third cases are when  $1<k<\frac{1}{\lambda}\log_p\sog{\frac{1}{\sqrt{s}}}$ and $k\geq \frac{1}{\lambda}\log_p\sog{\frac{1}{\sqrt{s}}}$ respectively.
In each of these cases we will need to use different type of inequalities.


\begin{prop}\label{prop_norm_of_q^x}
We have that $\norm{q^x}=G_{\norm{}}(s)$ and $\norm{q^{ax}}\leq \norm{q^x}$ for any $a\in\Z_p$.
\end{prop} 
    \begin{proof}
    Write
    \[q^x=(\zeta+h)^x=\zeta^x\sog{1+\frac{h}{\zeta}}^x.\]
    Since $\norm{}$ is invariant under multiplication by smooth characters
    \[\norm{\zeta^x\sog{1+\frac{h}{\zeta}}^x}=\norm{\sog{1+\frac{h}{\zeta}}^x}.\]
    Since $\abs{h/\zeta}_p=\abs{h}_p=s$ is a regular value for the norm $\norm{}$, we have, by Proposition \ref{powers_and_growth_modulus}, an equality
    \[\norm{\sog{1+\frac{h}{\zeta}}^x}=G_{\norm{}}(s).\]
    Thus, $\norm{q^x}=G_{\norm{}}(s)$.
    Let $a\in\Z_p$.
    To show that $\norm{q^{ax}}\leq \norm{q^x}$ we use the same trick.
    Write
    \[\norm{q^{ax}}=\norm{(\zeta+h)^{ax}}=\norm{\zeta^{ax}\cdot\sog{1+\frac{h}{\zeta}}^{ax}}=\norm{\sog{1+\frac{h}{\zeta}}^{ax}}=\norm{(1+h')^x},\]
    where $h'=(1+h/\zeta)^a-1$.
    Then $\abs{h'}_p\leq \abs{h}_p$.
    Since $G_{\norm{}}(r)$ is monotone increasing, and by proposition \ref{powers_and_growth_modulus}, 
    \[\norm{q^{ax}}=\norm{(1+h')^x}\leq G(\abs{h'}_p)\leq G(\abs{h}_p)=\norm{q^x}.\]
    \end{proof} 

\begin{prop}\label{some_p_adic_valuations}
Assume that $\pabs{1-\zeta}>s$.
    \begin{enumerate}
    \item Let $ s< r<1$.
    Then for any $1\leq i\leq\frac{1}{\lambda}\log_p\sog{\frac{1}{r}}$ 
    \[\abs{1-q^i}_p=\abs{1-\zeta^i}_p\geq r.\]
    \item For any $1<k\leq\frac{1}{\lambda}\log_p\sog{\frac{1}{\sqrt{s}}}$ 
    \[\abs{\pcoeff{\zeta}{q}_k}_p\leq \sqrt{s}\cdot\abs{(q;q)_k}_p\]
    \end{enumerate}
\end{prop} 
    \begin{proof}
    First, recall our assumption that $\log_p\sog{\frac{1}{s}}\leq\frac{1}{p-1}$.
    Then 
    \[\frac{1}{\lambda}\log_p\sog{\frac{1}{r}}\leq \frac{1}{\lambda}\log_p\sog{\frac{1}{s}}\leq \frac{1}{(p-1)\lambda}=p^{N-1},\]
    for any $s<r<1$.
    In particular, any indices $i$ and $k$ that appear in this proof are in $\braces{0,1,...,p^N-1}$, so the expressions $\zeta^i, \zeta^k$ are not equal to $1$.
    Second, note that if $N$ is not large enough, the interval $[1,\frac{1}{\lambda}\log_p\sog{\frac{1}{r}}]$ may be empty.
        
    (1). For any $i\geq 1$, $\pabs{\zeta^i-q^i}\leq \pabs{\zeta-q}=s< r$.
    The condition $i\leq\frac{1}{\lambda}\log_p\sog{\frac{1}{r}}$ is equivalent to
    \[r\leq p^{-\lambda i}=\pabs{\zeta-1}^i.\]
    Write $i=ap^k$ with $p\nmid a$. Then
    \[\pabs{1-\zeta^i}=\pabs{1-\zeta^{p^k}}=\pabs{1-\zeta}^{p^k}\geq \pabs{1-\zeta}^i\geq r.\]
    Thus,
    \[\pabs{1-q^i}=\pabs{(\zeta^i-q^i)+(1-\zeta^i)}=\abs{1-\zeta^i}_p\geq r.\]
    
    (2). We use part $(1)$ with $r=\sqrt{s}>s$.
    Then,
    \[\abs{1-q^i}_p=\abs{1-\zeta^i}_p\geq \sqrt{s}>s,\]
     for all $1\leq i\leq \frac{1}{\lambda}\log_p\sog{\frac{1}{\sqrt{s}}}$.
    If in addition $i>1$, then by writing $\zeta-q^i=(\zeta-q)+q(1-q^{i-1})$ we see that 
    \[\abs{\zeta-q^i}_p=\abs{q(1-q^{i-1})}_p=\abs{1-q^{i-1}}_p.\]
    Let $1<k\leq \frac{1}{\lambda}\log_p\sog{\frac{1}{\sqrt{s}}}$.
    Then        \begin{align*}
        \frac{\pcoeff{\zeta}{q}_k}{(q;q)_k}
    &=\frac{(\zeta-1)(\zeta-q)(\zeta-q^2)...(\zeta-q^{k-1})}{(1-q)(1-q^2)(1-q^3)...(1-q^k)}\\  
    &=\frac{(\zeta-1)(\zeta-q)}{(1-q^{k-1})(1-q^k)}\cdot\sog{\frac{\zeta-q^2}{1-q}}\cdot\sog{\frac{\zeta-q^3}{1-q^2}}\cdot...\cdot\sog{\frac{\zeta-q^{k-1}}{1-q^{k-2}}}
        \end{align*} 
    (Note that $(q;q)_k\neq 0$).
    Using the equality $\abs{\zeta-q^i}_p=\abs{1-q^{i-1}}_p$ for any $2\leq i\leq k-1$ we see that
    \[\frac{\abs{\pcoeff{\zeta}{q}_k}_p}{\abs{(q;q)_k}_p}=\frac{\abs{(\zeta-1)}_p\abs{(\zeta-q)}_p}{\abs{(1-q^{k-1})}_p\abs{(1-q^k)}_p}.\]
    By part $(1)$, $\abs{1-q^k}_p\geq \sqrt{s}$ and $\abs{1-q^{k-1}}_p\geq \sqrt{s}$.
    Moreover, since one of $k$ or $k-1$ is not divisible by $p$, the $p$-adic absolute value of one of them is equal to $\abs{1-q}_p$.
    The assumption that $\pabs{\zeta-1}>s$ implies that $\pabs{\zeta-1}=\pabs{q-1}$.  
    Thus, 
    \[\frac{\abs{\pcoeff{\zeta}{q}_k}_p}{\abs{(q;q)_k}_p}\leq \frac{\abs{\zeta-1}_p\cdot s}{\abs{\zeta-1}_p\cdot \sqrt{s}}=\sqrt{s}.\]
    \end{proof} 

    \begin{proof}[Proof of \ref{main_inequality}]
    Denote $\alpha=\frac{1}{2\lambda}\log_p\sog{\frac{1}{\sqrt{s}}}$ and let $N$ be large enough such that the following conditions are satisfied ($\zeta$ is a primitive $p^N$-th root of unity).
        \begin{enumerate}
        \item $\pabs{1-\zeta}>s$.
        \item $\alpha>p^8$.
        \item $\frac{\lambda}{4}\alpha\log_p(\alpha)\geq \log_p\sog{\frac{M}{\sqrt{s}}}$.
        Note that $\frac{\lambda}{4}\alpha\log_p(\alpha)=A\cdot \log_p(\frac{1}{\lambda})+B$ where $A=\frac{1}{8}\log_p\sog{\frac{1}{\sqrt{s}}}>0$ and $B=A\cdot\log_p\sog{\frac{1}{2}\log_p\sog{\frac{1}{\sqrt{s}}}}$.
        Note that $B$ does not depend on $\zeta$.
        \end{enumerate}
        
    Under these assumptions we will show that
    \[\norm{\pcoeff{\zeta}{q}_1\cdot\qbinom{x}{1}{q}}
    >\norm{\pcoeff{\zeta}{q}_k\cdot\qbinom{x}{k}{q}}\]
    for any $k\neq 1$.
    
    We begin by showing that $\norm{\qbinom{x}{1}{q}}=\norm{q^x}>1$.
    Indeed, by the assumption that $\pabs{\zeta-1}>s$ we have that $\pabs{\zeta-1}=\pabs{q-1}$.
    Thus,
    \[\norm{\pcoeff{\zeta}{q}_1\cdot\qbinom{x}{1}{q}}=\norm{(\zeta-1)\frac{1-q^x}{1-q}}=\norm{1-q^x}.\]
    By Proposition \ref{prop_norm_of_q^x}, $\norm{q^x}=G_{\norm{}}(s)>1$.
    Therefore, 
    \[\norm{1-q^x}=\norm{q^x}>1.\]
    
    Assume that $k=0$.
    Then \[\norm{\pcoeff{\zeta}{q}_0\cdot\qbinom{x}{0}{q}}=\norm{\textbf{1}_{\Z_p}(x)}=1.\]
    
    Assume $1<k\leq\frac{1}{\lambda}\log_p\sog{\frac{1}{\sqrt{s}}}$.
By part $(2)$ of Proposition \ref{some_p_adic_valuations} and by Proposition \ref{prop_norm_of_q^x},
        \begin{align*}
        \nrm*{\pcoeff{\zeta}{q}_k\cdot\qbinom{x}{k}{q}}
        =\frac{\abs{\pcoeff{\zeta}{q}_k}_p}{\abs{(q;q)_k}_p}\cdot\norm{(q^x-1)(q^x-q)...(q^x-q^{k-1})}
        \leq \sqrt{s}\cdot\norm{q^x}
        <\norm{q^x}
        =\norm{\pcoeff{\zeta}{q}_1\cdot\qbinom{x}{1}{q}}.
        \end{align*} 
        
    Assume that $k> \frac{1}{\lambda}\log_p\sog{\frac{1}{\sqrt{s}}}$.
    Let $m$ be an integer with 
    \[\frac{1}{2\lambda}\log_p\sog{\frac{1}{\sqrt{s}}}\leq m<\frac{1}{\lambda}\log_p\sog{\frac{1}{\sqrt{s}}}.\]
    Such an integer exists, since $\frac{1}{\lambda}\log_p\sog{\frac{1}{\sqrt{s}}}>2p^8> 4$.
    As $k>m$, $\abs{\pcoeff{\zeta}{q}_k}_p\leq\abs{\pcoeff{\zeta}{q}_m}_p$.
    By the second and first parts of Proposition \ref{some_p_adic_valuations} we have
    \[\abs{\pcoeff{\zeta}{q}_m}_p\leq \sqrt{s}\cdot\abs{(q;q)_m}_p=\sqrt{s}\cdot\abs{(\zeta;\zeta)_m}_p.\]
    Since $m>p^8$ we can apply Proposition \ref{q_analog_evaluations_cor}, and together with the assumption that $m\geq\alpha$ we get
        \begin{align*}
        \log_p(\abs{(\zeta;\zeta)_m}_p)
        \leq -\frac{\lambda}{4}m\log_p(m)
        \leq -\frac{\lambda}{4}\alpha\log_p(\alpha)
        \leq -\log_p\sog{\frac{M}{\sqrt{s}}}.
        \end{align*} 
    In the last inequality we used our assumption $3$.
    Then $\abs{(\zeta;\zeta)_m}_p\leq \frac{\sqrt{s}}{M}$.
    Therefore,
    \[\abs{\pcoeff{\zeta}{q}_k}_p\leq \sqrt{s}\cdot \abs{(\zeta;\zeta)_m}_p\leq \sqrt{s}\cdot \frac{\sqrt{s}}{M}=\frac{s}{M}.\]
    Finally,
    \[\norm{\pcoeff{\zeta}{q}_k\cdot\qbinom{x}{k}{q}}\leq \abs{\pcoeff{\zeta}{q}_k}_p\cdot M\leq s<1<\norm{\pcoeff{\zeta}{q}_1\cdot\qbinom{x}{1}{q}}.\]
    This completes the proof of \ref{main_inequality}, hence of Theorem \ref{thm_strong_minimality_Z_p} for $\Z_p$.
    \end{proof} 





