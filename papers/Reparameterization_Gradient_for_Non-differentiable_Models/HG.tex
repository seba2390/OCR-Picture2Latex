%auto-ignore

%% make title
{
  \newcommand{\supptitle}{
    Supplementary Material:
    Reparameterization Gradient
    for Non-differentiable Models
  }
  % from nips_2018.sty
  \newcommand{\toptitlebar}{
    \hrule height 4pt
    \vskip 0.25in
    \vskip -\parskip%
  }
  \newcommand{\bottomtitlebar}{
    \vskip 0.29in
    \vskip -\parskip
    \hrule height 1pt
    \vskip 0.09in%
  }
  \vbox{%
    \hsize\textwidth
    \linewidth\hsize
    \vskip 0.1in
    \toptitlebar
    \centering
        {\LARGE\bf \supptitle \par}
    \bottomtitlebar
  }
}

\section{Proof of Theorem~\ref{thm:grad-eq-main}}
Using reparameterization, we can write $\ELBO{\tht}$ as follows:
\begin{align}
  \ELBO{\tht}
  &= \E{q(\bm \eps)}{\log \frac{\sum_{k=1}^K \indc{f_\tht (\bm \eps)\in R_{k}} \cdot r_k(f_\tht(\bm \eps))}{q_\tht(f_\tht(\bm \eps))}} \nonumber \\
  &= \E{q(\bm \eps)}{\sum_{k=1}^K \indc{f_\tht (\bm \eps)\in R_{k}} \cdot \log \frac{r_k(f_\tht(\bm \eps))}{q_\tht(f_\tht(\bm \eps))}} \label{eqn:elbo-derv} \\
  &= \sum_{k=1}^K \E{q(\bm \eps)}{\indc{f_\tht (\bm \eps)\in R_{k}} \cdot h_k(\bm{\eps},\tht)\phantom{\Big)\!\!\!\!}}. \nonumber
\end{align}
In \eqref{eqn:elbo-derv}, we can move the summation and the indicator function out of $\log$
since the regions $\{R_k\}_{1\leq k \leq K}$ are disjoint.
We then compute the gradient of $\ELBO{\tht}$ as follows:
\begin{align}
& \grad{\tht}{\ELBO{\tht}}
\nonumber
\\
&= \sum_{k=1}^K \grad{\tht} \E{q(\bm \eps)}{\indc{f_\tht (\bm \eps)\in R_{k}} \cdot h_k(\bm{\eps},\tht)\phantom{\Big)\!\!\!\!}}\nonumber\\
&= \sum_{k=1}^K \grad{\tht} \int_{f_\tht^{-1}( R_{k} )} q(\bm{\eps}) h_k(\bm{\eps},\tht) d\bm \eps \nonumber\\
&=
\sum_{k=1}^K
\int_{f_\tht^{-1}( R_{k} )}
\Big(q(\bm{\eps}) \grad{\tht}h_k(\bm{\eps},\tht)
+
\grad{\bm \eps} \bullet \big(q(\bm{\eps}) h_k(\bm{\eps},\tht) \bm{V}(\bm{\eps},\tht)\big)
\Big)d\bm \eps \label{eqn:grad-elbo-diffint} \\
&=
\E{q(\bm{\eps})}{\sum_{k=1}^K \indc{f_\tht(\bm{\eps}) \,{\in}\, R_k} \cdot \grad{\tht}{h_k(\bm{\eps},\tht)}}
+
\sum_{k=1}^K\int_{f_\tht^{-1}( R_{k} )} \grad{\bm \eps} \bullet \big(q(\bm{\eps}) h_k(\bm{\eps},\tht) \bm{V}(\bm{\eps},\tht)\big)d\bm \eps \nonumber\\
&=
\underbrace{\E{q(\bm{\eps})}{\sum_{k=1}^K \indc{f_\tht(\bm{\eps}) \,{\in}\, R_k} \cdot \grad{\tht}{h_k(\bm{\eps},\tht)}}}_{\ReparamGrad_\tht}
+
\underbrace{\sum_{k=1}^K \int_{f_\tht^{-1}(\partial R_{k})} \big(q(\bm{\eps}) h_k(\bm{\eps},\tht) \bm{V}(\bm{\eps},\tht)\big) \bullet d\bm{\Sigma}}_{\RegionChange_\tht}
\label{eqn:grad-elbo-divthm}
\end{align}
where $\grad{\bm{\eps}} \bullet \bm{U}$ denotes the column vector
whose $i$-th component is $\grad{\bm{\eps}} \cdot \bm{U}_i$,
the divergence of $\bm{U}_i$ with respect to $\bm{\eps}$.
\eqref{eqn:grad-elbo-divthm} is the formula that we wanted to prove.

The two non-trivial steps in the above derivation are
\eqref{eqn:grad-elbo-diffint} and \eqref{eqn:grad-elbo-divthm}.
First, \eqref{eqn:grad-elbo-diffint} is a direct consequence of
the following theorem, existing yet less well-known, on exchanging integration and differentiation
under moving domain:
\begin{theorem}
  \label{thm:diff-under-int-moving}
  Let $D_\theta \subset \R^n$ be a smoothly parameterized region.
  That is, there exist open sets $\Omega \subset \mathbb{R}^n$ and $\Theta \subset \mathbb{R}$,
  and twice continuously differentiable
  $\hat{\bm{\eps}}: \Omega \times \Theta \rightarrow \mathbb{R}^n$
  such that $D_\theta = \hat{\bm{\eps}}(\Omega, \theta)$ for each $\tht \in \Tht$.
  Suppose that $\hat{\bm{\eps}}(\cdot, \theta)$ is a $C^1$-diffeomorphism for each $\tht \in \Tht$.
  Let $f : \R^n \times \R \to \R$ be a differentiable function
  such that $f(\cdot,\tht) \in \intgl{D_\tht}$ for each $\tht \in \Tht$.
  %% when moving according to $\hat{\bm{\eps}}()$
  If there exists   $g : \Omega \to \R$
  such that $g \in \intgl{\Omega}$ and
  $\abs{\grad{\tht}\big(f(\hat{\bm{\eps}},\theta)
    \abs{\pdv{\hat{\bm{\eps}}}{\bm \omega}} \big)}
  \leq g(\bm{\omega})$
  for any $\tht \in \Tht$ and $\bm{\omega} \in \Omega$,
  then
  $$
  \grad{\theta} \int_{D_{\theta}}f(\bm{\eps},\tht) d \bm{\eps}
  =
  \int_{D_\theta}  \Big(\grad{\theta} f + \grad{\bm\eps} \cdot (f\bm{\mathrm{v}})\Big)(\bm{\eps},\tht)
  d \bm{\eps}.
  $$
  Here $\bm{\mathrm{v}}(\bm{\eps}, \theta)$ denotes
  $\grad{\theta} \hat{\bm{\eps}}(\bm{\omega},\tht)
  \big|_{{\bm{\omega}=\hat{\bm{\eps}}^{-1}_\tht(\bm{\eps})}}$,
  the velocity of the particle $\bm{\eps}$ at time $\theta$.
  %% when moving according to $\hat{\bm{\eps}}$.
\end{theorem}
The statement of Theorem~\ref{thm:diff-under-int-moving}
(without detailed conditions as we present above)
and the sketch of its proof
can be found in~\cite{FlandersAMM1973}.
One subtlety in applying Theorem~\ref{thm:diff-under-int-moving} to our case
is that $R_k$ (which corresponds to $\Omega$ in the theorem) may not be open,
so the theorem may not be immediately applicable.
However, since the boundary $\partial R_k$ has Lebesgue measure zero in $\R^n$,
ignoring the reparameterized boundary $f_\tht^{-1}(\partial R_k)$
in the integral of \eqref{eqn:grad-elbo-diffint} does not change the value of the integral.
Hence, we apply Theorem~\ref{thm:diff-under-int-moving} to
$D_\tht = \mathrm{int}(f_\tht^{-1}(R_k))$
(which is possible because $\Omega = \mathrm{int}(R_k)$ is now open),
and this gives us the desired result.
Here $\mathrm{int}(T)$ denotes the interior of $T$.


Second, to prove~\eqref{eqn:grad-elbo-divthm}, it suffices to show that
$$ \int_{V} \grad{\bm{\eps}} \bullet \bm{U}(\bm{\eps})  d\bm{\eps}
= \int_{\partial V} \bm{U}(\bm{\eps}) \bullet d\bm{\Sigma}$$
where $\bm U(\bm \eps) = q(\bm{\eps}) h_k(\bm{\eps},\tht) \bm{V}(\bm{\eps},\tht)$
and $V = f_\tht^{-1}( R_{k} )$.
To prove this equality, we apply the divergence theorem:
\begin{theorem}[Divergence theorem]
  \label{thm:div}
  Let $V$ be a compact subset of $\mathbb{R}^n$
  that has a piecewise smooth boundary $\partial V $.
  If $\bm{F}$ is a differentiable vector field defined on a neighborhood of $V$, then
  $$ \int_{V} (\grad{} \cdot \bm{F}) \, dV = \int_{\partial V} \bm{F} \cdot d\bm{\Sigma} $$
  where $d\bm{\Sigma}$ is the outward pointing normal vector of the boundary $\partial V$.
\end{theorem}
In our case, the region $V=f_\tht^{-1}(R_k)$ may not be compact,
so we cannot directly apply Theorem~\ref{thm:div} to $\bm{U}$.
To circumvent the non-compactness issue,
we assume that $q(\bm{\eps})$ is in $\mathcal{S}(\R^n)$, the Schwartz space on $\R^n$.
That is, assume that every partial derivative of $q(\bm{\eps})$ of any order
decays faster than any polynomial.
This assumption is reasonable in that
the probability density of many important probability distributions
(e.g., the normal distribution)
is in $\mathcal{S}(\R^n)$.
Since $q \in \mathcal{S}(\R^n)$,
there exists a sequence of test functions $\{\phi_j\}_{j\in \mathbb{N}}$
such that each $\phi_j$ has compact support
and $\{\phi_j\}_{j \in \N}$ converges to $q$ in $\mathcal{S}(\R^n)$,
which is a well-known result in functional analysis.
Since each $\phi_j$ has compact support,
so does $\bm{U}^j(\bm{\eps}) \defeq \phi_j(\bm{\eps}) h_k(\bm{\eps},\tht) \bm{V}(\bm{\eps},\tht)$.
By applying Theorem~\ref{thm:div} to $\bm{U}^j$,
we have
$$
\int_V \grad{\bm{\eps}} \bullet \bm U^j(\bm{\eps}) d\bm{\eps}
= \int_{\partial V} \bm U^j (\bm{\eps}) \bullet d\bm{\Sigma}.
$$
Because $\{\phi_j\}_{j \in \N}$ converges to $q$ in $\mathcal{S}(\R^n)$,
taking the limit $j \rightarrow \infty$ on the both sides of the equation
gives us the desired result.


\section{Proof of Theorem~\ref{thm:est-surface-integral}}
%
% Hongseok: In the theorem, I replaced Lebesgue measurability by measurability.
% Implicitly I have been assuming Borel sigma algebra and Borel measurability.
% Since each Borel measurable set is also Lebesgue measurable, I think that
% we can keep this implicit convention.
%
Theorem~\ref{thm:est-surface-integral}
is a direct consequence of the  following theorem called ``area formula'':
\begin{theorem}[Area formula]
  \label{thm:area-formula}
  Suppose that $g: \R^{n-1} \to \R^n$ is injective and Lipschitz.
  If $A \subset \R^{n-1}$ is measurable and $\bm H : \R^n \to \R^n$ is measurable,
  then
  $$
  \int_{g(A)} \bm H(\bm \eps) \cdot d\bm{\Sigma}
  =
  \int_A \Big(\bm H(g(\bm \zeta)) \cdot \bm{n}(\zeta)\Big) \abs{Jg(\bm \zeta)} \, d\bm\zeta
  $$
  where $Jg(\bm \zeta) =
  \det \Big[ \pdv{g(\bm \zeta)}{\bm{\zeta}_1} \big| \pdv{g(\bm \zeta)}{\bm{\zeta}_2} \big|
    \cdots \big| \pdv{g(\bm \zeta)}{\bm{\zeta}_{n-1}} \big| \bm n(\bm \zeta) \Big]$,
  and $\bm n(\bm \zeta)$ is the unit normal vector of the hypersurface $g(A)$
  at $g(\bm \zeta)$ such that it has the same direction as $d\bm{\Sigma}$.
\end{theorem}
A more general version of Theorem~\ref{thm:area-formula}
can be found in~\cite{DiaconisMainfold13}.
In our case,
the hypersurface $g(A)$ for the surface integral on the LHS
is given by $\{\bm{\eps} \,\mid\, \bm{a}\cdot\bm{\eps}=c \}$,
so we use $A=\R^{n-1}$ and
$g(\bm{\zeta}) = \big(\bm{\zeta}_1,\ldots,\bm{\zeta}_{j-1},
\frac{1}{\bm{a}_j}(c-\bm{a}_{-j}\cdot\bm{\zeta}),
\bm{\zeta}_{j},\ldots,\bm{\zeta}_{n-1}\big)^\intercal$
and apply Theorem~\ref{thm:area-formula} with $\bm{H}(\bm{\eps}) = q(\bm{\eps})\bm{F}(\bm{\eps})$.
In this settings,
$\bm{n}(\bm{\zeta})$ and $\abs{Jg(\bm{\zeta})}$ are calculated as
$$
\bm{n}(\bm{\zeta}) = \sgn{-\bm{a}_j} \frac{\abs{\bm{a}_j}}{\|\bm{a}\|_2}
 \Big(\frac{\bm{a}_1}{\bm{a}_j},\ldots,\frac{\bm{a}_{j-1}}{\bm{a}_j},1,\frac{\bm{a}_{j+1}}{\bm{a}_j},\ldots,\frac{\bm{a}_n}{\bm{a}_j}\Big)^\intercal
\quad\tand\quad
\abs{Jg(\bm{\zeta})} = \frac{\|\bm{a}\|_2}{\abs{\bm{a}_j}},
$$
and this gives us the desired result.
