%auto-ignore

\section{Conclusion}
\label{sec:conclusion}

We have presented a new estimator for the gradient of the standard variational objective, $\ELBO{}$. The key feature of our estimator is that it can keep variance under control by using a form of the reparameterization trick even when the density of a model is not differentiable. The estimator splits the space of the latent random variable into a lower-dimensional subspace where the density may fail to be differentiable, and the rest where the density is differentiable. Then, it estimates the contributions of both parts to the gradient separately, using a version of manifold sampling for the former and the reparameterization trick for the latter. We have shown the unbiasedness of our estimator using a theorem for interchanging integration and differentiation under moving domain~\cite{FlandersAMM1973} and the divergence theorem. Also, we have experimentally demonstrated the promise of our estimator using three time-series models. One interesting future direction is to investigate the possibility of applying our ideas to recent variational objectives~\cite{MnihICML16,LiNIPS17,MaddisonNIPS17,LeICLR2018,NaessethAISTATS18}, which are based on tighter lower bounds of marginal likelihood than the standard $\ELBO{}$.

When viewed from a high level, our work suggests a heuristic of splitting the latent space into a bad yet tiny subspace and the remaining good one, and solving an estimation problem in each subspace separately. The latter subspace has several good properties and so it may allow the use of efficient estimation techniques that exploit those properties. The former subspace is, on the other hand, tiny and the estimation error from the subspace may, therefore, be relatively small. We would like to explore this heuristic and its extension in different contexts, such as stochastic variational inference with different objectives~\cite{MnihICML16,LiNIPS17,MaddisonNIPS17,LeICLR2018,NaessethAISTATS18}.

\subsubsection*{Acknowledgments}

We thank Hyunjik Kim, George Tucker, Frank Wood and anonymous reviewers for their helpful comments, and Shin Yoo and Seongmin Lee for allowing and helping us to use their cluster machines. This research was supported by the Engineering Research Center Program through the National Research Foundation of Korea (NRF) funded by the Korean Government MSIT (NRF-2018R1A5A1059921), and also by Next-Generation Information Computing Development Program through the National Research Foundation of Korea (NRF) funded by the Ministry of Science, ICT (2017M3C4A7068177).

