%auto-ignore

\section{Related Work}
\label{sec:related}

%% Two common examples of a model with a non-differentiable density are
%% (a) the one that uses if statements whose branch conditions contain continuous random variables, and
%% (b) the one that uses discrete random variables, typically together with continuous random variables.
%% Coming up with an efficient algorithm for stochastic variational inference for such a model,
%% especially (b), has been an active research topic. 
A common example of a model with a non-differentiable density is the one that uses discrete random variables, typically together with continuous random variables.\footnote{
  Another common example of such a model is the one
  that uses if statements whose branch conditions contain continuous random variables,
  which is the main focus of our work.
}
Coming up with an efficient algorithm for stochastic variational inference for such a model has been an active research topic. Maddison \emph{et al.}~\cite{MaddisonICLR17} and Jang \emph{et al.}~\cite{JangICLR17} proposed continuous relaxations of discrete random variables that convert non-differentiable variational objectives to differentiable ones and make the reparameterization trick applicable. Also, a variety of control variates for the standard score estimator~\cite{WilliamsMLJ1992,PaisleyICML12,WingateBBVI13,RanganathAISTATS14} for the gradients of variational objectives have been developed~\cite{RanganathAISTATS14,GuICLR16,GuICLR17,TuckerNIPS17,GrathwohlICLR18,MillerReparam2017}, some of which use biased yet differentiable control variates such that the reparameterization trick can be used to correct the bias~\cite{GuICLR16,TuckerNIPS17,GrathwohlICLR18}.

Our work extends this line of research by adding a version of the reparameterization trick that can be applied to models with discrete random variables.
For instance, consider a model $p(x,z)$ with $z$ discrete.
By applying the Gumbel-Max reparameterization~\cite{GumbelBook,MaddisonNIPS16} to $z$,
we transform $p(x,z)$ to $p(x,z,c)$,
where $c$ is sampled from the Gumbel distribution
and $z$ in $p(x,z,c)$ is defined deterministically from $c$ using the $\arg\max$ operation.
Since $\arg\max$ can be written as if statements,
we can express $p(x,z,c)$ in the form of \eqref{eqn:nondiff-density}
to which our reparameterization gradient can be applied. Investigating the effectiveness of this approach for discrete random variables is an interesting topic for future research.

The reparameterization trick was initially used with normal distribution~\cite{KingmaICLR14,RezendeICML14}, but its scope was soon extended to other common distributions, such as gamma, Dirichlet, and beta~\cite{Knowles15stochastic,RuizNIPS16,NaessethAISTATS17}. Techniques for constructing normalizing flow~\cite{RezendeICML15,KingmaNIPS16} can also be viewed as methods for creating distributions in a reparameterized form. In the paper, we did not consider these recent developments and mainly focused on the reparameterization with normal distribution. One interesting future avenue is to further develop our approach for these other reparameterization cases. We expect that the main challenge will be to find an effective method for handling the surface integrals in Theorem~\ref{thm:grad-eq-main}.

%Discontinuities - HMC - Perhaps
