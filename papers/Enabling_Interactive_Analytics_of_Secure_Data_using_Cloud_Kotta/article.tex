\documentclass[sigconf]{acmart}

\usepackage{booktabs} % For formal tables


\usepackage{graphicx}
\usepackage{color}
\usepackage{soul}
\usepackage{todonotes}
\usepackage{hyperref}
\usepackage{listings}
\usepackage{amsmath}

\definecolor{mygreen}{rgb}{0,0.6,0}
\definecolor{mygray}{rgb}{0.5,0.5,0.5}
\definecolor{mymauve}{rgb}{0.58,0,0.82}

\lstset{ %
  backgroundcolor=\color{white},   % choose the background color; you must add \usepackage{color} or \usepackage{xcolor}; should come as last argument
  basicstyle=\footnotesize,        % the size of the fonts that are used for the code
  breakatwhitespace=false,         % sets if automatic breaks should only happen at whitespace
  breaklines=true,                 % sets automatic line breaking
  captionpos=b,                    % sets the caption-position to bottom
  commentstyle=\color{mygreen},    % comment style
  deletekeywords={...},            % if you want to delete keywords from the given language
  escapeinside={\%*}{*)},          % if you want to add LaTeX within your code
  extendedchars=true,              % lets you use non-ASCII characters; for 8-bits encodings only, does not work with UTF-8
  frame=single,                   % adds a frame around the code
  keepspaces=true,                 % keeps spaces in text, useful for keeping indentation of code (possibly needs columns=flexible)
  keywordstyle=\color{blue},       % keyword style
  language=Octave,                 % the language of the code
  morekeywords={*,...},           % if you want to add more keywords to the set
  numbers=left,                    % where to put the line-numbers; possible values are (none, left, right)
  numbersep=5pt,                   % how far the line-numbers are from the code
  numberstyle=\tiny\color{mygray}, % the style that is used for the line-numbers
  rulecolor=\color{black},         % if not set, the frame-color may be changed on line-breaks within not-black text (e.g. comments (green here))
  showspaces=false,                % show spaces everywhere adding particular underscores; it overrides 'showstringspaces'
  showstringspaces=false,          % underline spaces within strings only
  showtabs=false,                  % show tabs within strings adding particular underscores
  stepnumber=2,                    % the step between two line-numbers. If it's 1, each line will be numbered
  stringstyle=\color{mymauve},     % string literal style
  tabsize=2,                   % sets default tabsize to 2 spaces
  title=\lstname                   % show the filename of files included with \lstinputlisting; also try caption instead of title
}


% Copyright
%\setcopyright{none}
%\setcopyright{acmcopyright}
%\setcopyright{acmlicensed}
\setcopyright{rightsretained}
%\setcopyright{usgov}
%\setcopyright{usgovmixed}
%\setcopyright{cagov}
%\setcopyright{cagovmixed}

\usepackage{array}
\newcolumntype{L}[1]{>{\raggedright\let\newline\\\arraybackslash\hspace{0pt}}m{#1}}
\newcolumntype{C}[1]{>{\centering\let\newline\\\arraybackslash\hspace{0pt}}m{#1}}
\newcolumntype{R}[1]{>{\raggedleft\let\newline\\\arraybackslash\hspace{0pt}}m{#1}}



\newif\iffinal

% Un-comment this line to see proposal without comments
%\finaltrue

\iffinal
  \newcommand\yadu[1]{}
  \newcommand\kyle[1]{}
  \newcommand\eamon[1]{}
\else
  \newcommand\yadu[1]{{\color{blue}[Yadu: #1]}}
  \newcommand\kyle[1]{{\color{red}[Kyle: #1]}}
  \newcommand\eamon[1]{{\color{purple}[Eamon: #1]}}
\fi

% DOI
\acmDOI{10.475/123_4}

% ISBN
\acmISBN{123-4567-24-567/08/06}

%Conference
\acmConference[ScienceCloud'17]{Workshop on Scientific Cloud Computing}{June 2017}{Washington, DC USA} 
\acmYear{2017}
\copyrightyear{2017}

\acmPrice{15.00}


\begin{document}
\title{Enabling Interactive Analytics of Secure Data using Cloud Kotta}
%\subtitle{Extended Abstract}



\author{Yadu N. Babuji, Kyle Chard, and Eamon Duede}
%\authornote{Dr.~Trovato insisted his name be first.}
%\orcid{1234-5678-9012}
\affiliation{%
  \institution{Computation Institute, University of Chicago and Argonne National Laboratory}
  \streetaddress{5735 S Ellis Ave}
  \city{Chicago} 
  \state{Illinois} 
  \postcode{60637}
}
\email{{yadu, chard, eduede}@uchicago.edu}

% The default list of authors is too long for headers}
\renewcommand{\shortauthors}{Y. Babuji et al.}


\begin{abstract}
Research, especially in the social sciences and humanities, is increasingly
reliant on the application of data science methods to analyze large amounts
of (often private) data. Secure data enclaves provide a solution for managing and analyzing private data. However, such enclaves do not readily support
discovery science---a form of exploratory or interactive analysis by which researchers 
execute a range of (sometimes large) analyses in an iterative and collaborative
manner. 
The batch computing model offered by many data enclaves is well suited to executing
large compute tasks; however it is far from ideal for day-to-day discovery science.
As researchers must submit jobs to queues and wait for results, the high latencies inherent in queue-based, batch computing systems hinder interactive
analysis.
%Data science occurs at human pace and
%that involves iterative development which requires interactive systems. Latencies offered
%by typical queue based system slow down development.
In this paper we describe how we have augmented the Cloud Kotta secure data enclave
to support collaborative and interactive analysis of sensitive data.
Our model uses Jupyter notebooks as a flexible analysis environment 
and Python language constructs to support the execution of arbitrary 
functions on private data within this secure framework.
\end{abstract}


% Maybe remove the upper case at some point.
\newcommand{\NAMENS} {\textsc{Cloud Kotta}} % No space aftewards (for brackets etc.)
\newcommand{\NAME} {\textsc{Cloud Kotta }}



% make the title area
\maketitle


% \leavevmode
% \\
% \\
% \\
% \\
% \\
\section{Introduction}
\label{introduction}

AutoML is the process by which machine learning models are built automatically for a new dataset. Given a dataset, AutoML systems perform a search over valid data transformations and learners, along with hyper-parameter optimization for each learner~\cite{VolcanoML}. Choosing the transformations and learners over which to search is our focus.
A significant number of systems mine from prior runs of pipelines over a set of datasets to choose transformers and learners that are effective with different types of datasets (e.g. \cite{NEURIPS2018_b59a51a3}, \cite{10.14778/3415478.3415542}, \cite{autosklearn}). Thus, they build a database by actually running different pipelines with a diverse set of datasets to estimate the accuracy of potential pipelines. Hence, they can be used to effectively reduce the search space. A new dataset, based on a set of features (meta-features) is then matched to this database to find the most plausible candidates for both learner selection and hyper-parameter tuning. This process of choosing starting points in the search space is called meta-learning for the cold start problem.  

Other meta-learning approaches include mining existing data science code and their associated datasets to learn from human expertise. The AL~\cite{al} system mined existing Kaggle notebooks using dynamic analysis, i.e., actually running the scripts, and showed that such a system has promise.  However, this meta-learning approach does not scale because it is onerous to execute a large number of pipeline scripts on datasets, preprocessing datasets is never trivial, and older scripts cease to run at all as software evolves. It is not surprising that AL therefore performed dynamic analysis on just nine datasets.

Our system, {\sysname}, provides a scalable meta-learning approach to leverage human expertise, using static analysis to mine pipelines from large repositories of scripts. Static analysis has the advantage of scaling to thousands or millions of scripts \cite{graph4code} easily, but lacks the performance data gathered by dynamic analysis. The {\sysname} meta-learning approach guides the learning process by a scalable dataset similarity search, based on dataset embeddings, to find the most similar datasets and the semantics of ML pipelines applied on them.  Many existing systems, such as Auto-Sklearn \cite{autosklearn} and AL \cite{al}, compute a set of meta-features for each dataset. We developed a deep neural network model to generate embeddings at the granularity of a dataset, e.g., a table or CSV file, to capture similarity at the level of an entire dataset rather than relying on a set of meta-features.
 
Because we use static analysis to capture the semantics of the meta-learning process, we have no mechanism to choose the \textbf{best} pipeline from many seen pipelines, unlike the dynamic execution case where one can rely on runtime to choose the best performing pipeline.  Observing that pipelines are basically workflow graphs, we use graph generator neural models to succinctly capture the statically-observed pipelines for a single dataset. In {\sysname}, we formulate learner selection as a graph generation problem to predict optimized pipelines based on pipelines seen in actual notebooks.

%. This formulation enables {\sysname} for effective pruning of the AutoML search space to predict optimized pipelines based on pipelines seen in actual notebooks.}
%We note that increasingly, state-of-the-art performance in AutoML systems is being generated by more complex pipelines such as Directed Acyclic Graphs (DAGs) \cite{piper} rather than the linear pipelines used in earlier systems.  
 
{\sysname} does learner and transformation selection, and hence is a component of an AutoML systems. To evaluate this component, we integrated it into two existing AutoML systems, FLAML \cite{flaml} and Auto-Sklearn \cite{autosklearn}.  
% We evaluate each system with and without {\sysname}.  
We chose FLAML because it does not yet have any meta-learning component for the cold start problem and instead allows user selection of learners and transformers. The authors of FLAML explicitly pointed to the fact that FLAML might benefit from a meta-learning component and pointed to it as a possibility for future work. For FLAML, if mining historical pipelines provides an advantage, we should improve its performance. We also picked Auto-Sklearn as it does have a learner selection component based on meta-features, as described earlier~\cite{autosklearn2}. For Auto-Sklearn, we should at least match performance if our static mining of pipelines can match their extensive database. For context, we also compared {\sysname} with the recent VolcanoML~\cite{VolcanoML}, which provides an efficient decomposition and execution strategy for the AutoML search space. In contrast, {\sysname} prunes the search space using our meta-learning model to perform hyperparameter optimization only for the most promising candidates. 

The contributions of this paper are the following:
\begin{itemize}
    \item Section ~\ref{sec:mining} defines a scalable meta-learning approach based on representation learning of mined ML pipeline semantics and datasets for over 100 datasets and ~11K Python scripts.  
    \newline
    \item Sections~\ref{sec:kgpipGen} formulates AutoML pipeline generation as a graph generation problem. {\sysname} predicts efficiently an optimized ML pipeline for an unseen dataset based on our meta-learning model.  To the best of our knowledge, {\sysname} is the first approach to formulate  AutoML pipeline generation in such a way.
    \newline
    \item Section~\ref{sec:eval} presents a comprehensive evaluation using a large collection of 121 datasets from major AutoML benchmarks and Kaggle. Our experimental results show that {\sysname} outperforms all existing AutoML systems and achieves state-of-the-art results on the majority of these datasets. {\sysname} significantly improves the performance of both FLAML and Auto-Sklearn in classification and regression tasks. We also outperformed AL in 75 out of 77 datasets and VolcanoML in 75  out of 121 datasets, including 44 datasets used only by VolcanoML~\cite{VolcanoML}.  On average, {\sysname} achieves scores that are statistically better than the means of all other systems. 
\end{itemize}


%This approach does not need to apply cleaning or transformation methods to handle different variances among datasets. Moreover, we do not need to deal with complex analysis, such as dynamic code analysis. Thus, our approach proved to be scalable, as discussed in Sections~\ref{sec:mining}.
\section{Background and Motivation}

\subsection{IBM Streams}

IBM Streams is a general-purpose, distributed stream processing system. It
allows users to develop, deploy and manage long-running streaming applications
which require high-throughput and low-latency online processing.

The IBM Streams platform grew out of the research work on the Stream Processing
Core~\cite{spc-2006}.  While the platform has changed significantly since then,
that work established the general architecture that Streams still follows today:
job, resource and graph topology management in centralized services; processing
elements (PEs) which contain user code, distributed across all hosts,
communicating over typed input and output ports; brokers publish-subscribe
communication between jobs; and host controllers on each host which
launch PEs on behalf of the platform.

The modern Streams platform approaches general-purpose cluster management, as
shown in Figure~\ref{fig:streams_v4_v6}. The responsibilities of the platform
services include all job and PE life cycle management; domain name resolution
between the PEs; all metrics collection and reporting; host and resource
management; authentication and authorization; and all log collection. The
platform relies on ZooKeeper~\cite{zookeeper} for consistent, durable metadata
storage which it uses for fault tolerance.

Developers write Streams applications in SPL~\cite{spl-2017} which is a
programming language that presents streams, operators and tuples as
abstractions. Operators continuously consume and produce tuples over streams.
SPL allows programmers to write custom logic in their operators, and to invoke
operators from existing toolkits. Compiled SPL applications become archives that
contain: shared libraries for the operators; graph topology metadata which tells
both the platform and the SPL runtime how to connect those operators; and
external dependencies. At runtime, PEs contain one or more operators. Operators
inside of the same PE communicate through function calls or queues. Operators
that run in different PEs communicate over TCP connections that the PEs
establish at startup. PEs learn what operators they contain, and how to connect
to operators in other PEs, at startup from the graph topology metadata provided
by the platform.

We use ``legacy Streams'' to refer to the IBM Streams version 4 family. The
version 5 family is for Kubernetes, but is not cloud native. It uses the
lift-and-shift approach and creates a platform-within-a-platform: it deploys a
containerized version of the legacy Streams platform within Kubernetes.

\subsection{Kubernetes}

Borg~\cite{borg-2015} is a cluster management platform used internally at Google
to schedule, maintain and monitor the applications their internal infrastructure
and external applications depend on. Kubernetes~\cite{kube} is the open-source
successor to Borg that is an industry standard cloud orchestration platform.

From a user's perspective, Kubernetes abstracts running a distributed
application on a cluster of machines. Users package their applications into
containers and deploy those containers to Kubernetes, which runs those
containers in \emph{pods}. Kubernetes handles all life cycle management of pods,
including scheduling, restarting and migration in case of failures.

Internally, Kubernetes tracks all entities as \emph{objects}~\cite{kubeobjects}.
All objects have a name and a specification that describes its desired state.
Kubernetes stores objects in etcd~\cite{etcd}, making them persistent,
highly-available and reliably accessible across the cluster. Objects are exposed
to users through \emph{resources}. All resources can have
\emph{controllers}~\cite{kubecontrollers}, which react to changes in resources.
For example, when a user changes the number of replicas in a
\code{ReplicaSet}, it is the \code{ReplicaSet} controller which makes sure the
desired number of pods are running. Users can extend Kubernetes through
\emph{custom resource definitions} (CRDs)~\cite{kubecrd}. CRDs can contain
arbitrary content, and controllers for a CRD can take any kind of action.

Architecturally, a Kubernetes cluster consists of nodes. Each node runs a
\emph{kubelet} which receives pod creation requests and makes sure that the
requisite containers are running on that node. Nodes also run a
\emph{kube-proxy} which maintains the network rules for that node on behalf of
the pods. The \emph{kube-api-server} is the central point of contact: it
receives API requests, stores objects in etcd, asks the scheduler to schedule
pods, and talks to the kubelets and kube-proxies on each node. Finally,
\emph{namespaces} logically partition the cluster. Objects which should not know
about each other live in separate namespaces, which allows them to share the
same physical infrastructure without interference.

\subsection{Motivation}
\label{sec:motivation}

Systems like Kubernetes are commonly called ``container orchestration''
platforms. We find that characterization reductive to the point of being
misleading; no one would describe operating systems as ``binary executable
orchestration.'' We adopt the idea from Verma et al.~\cite{borg-2015} that
systems like Kubernetes are ``the kernel of a distributed system.'' Through CRDs
and their controllers, Kubernetes provides state-as-a-service in a distributed
system. Architectures like the one we propose are the result of taking that view 
seriously.

The Streams legacy platform has obvious parallels to the Kubernetes
architecture, and that is not a coincidence: they solve similar problems.
Both are designed to abstract running arbitrary user-code across a distributed
system.  We suspect that Streams is not unique, and that there are many
non-trivial platforms which have to provide similar levels of cluster
management.  The benefits to being cloud native and offloading the platform
to an existing cloud management system are: 
\begin{itemize}
    \item Significantly less platform code.
    \item Better scheduling and resource management, as all services on the cluster are 
        scheduled by one platform.
    \item Easier service integration.
    \item Standardized management, logging and metrics.
\end{itemize}
The rest of this paper presents the design of replacing the legacy Streams 
platform with Kubernetes itself.


% !TEX root = ../top.tex
% !TEX spellcheck = en-US

\begin{figure}[t]
\centering
\includegraphics[width=0.99\linewidth]{./fig/arch/arch.pdf}
\vspace{-3mm}
    \caption{{\bf Our single-stage approach.} We use an encoder-decoder architecture to progressively downsample the image and then to re-expand it. At each level of the decoder, we establish 3D-to-2D correspondences. Finally, we use a RANSAC-based PnP strategy~\cite{Lepetit09} 
    % \WJ{citation?} \YH{Fixed} 
     to infer a single reliable pose from these sets of correspondences. 
    % \WJ{Impossible to read text in printout, please try to have text in the figure with a similar font size as the surrounding article.}\YH{Fixed}
    }
\label{fig:arch}
\end{figure}

\section{Applications}\label{sec:applications}

Here we will see many problems \cite{peng2016approximate} that fall into the framework of $p$-extendible systems and thus, we directly have recovery results on their stable instances. Some of the problems might be hard, like Weighted Independent Set, whereas others may be easy (i.e. in $P$), having exact algorithms, however the greedy is extremely simple and fast compared to those. 


\section{Related Work}\label{sec:related}
 
The authors in \cite{humphreys2007noncontact} showed that it is possible to extract the PPG signal from the video using a complementary metal-oxide semiconductor camera by illuminating a region of tissue using through external light-emitting diodes at dual-wavelength (760nm and 880nm).  Further, the authors of  \cite{verkruysse2008remote} demonstrated that the PPG signal can be estimated by just using ambient light as a source of illumination along with a simple digital camera.  Further in \cite{poh2011advancements}, the PPG waveform was estimated from the videos recorded using a low-cost webcam. The red, green, and blue channels of the images were decomposed into independent sources using independent component analysis. One of the independent sources was selected to estimate PPG and further calculate HR, and HRV. All these works showed the possibility of extracting PPG signals from the videos and proved the similarity of this signal with the one obtained using a contact device. Further, the authors in \cite{10.1109/CVPR.2013.440} showed that heart rate can be extracted from features from the head as well by capturing the subtle head movements that happen due to blood flow.

%
The authors of \cite{kumar2015distanceppg} proposed a methodology that overcomes a challenge in extracting PPG for people with darker skin tones. The challenge due to slight movement and low lighting conditions during recording a video was also addressed. They implemented the method where PPG signal is extracted from different regions of the face and signal from each region is combined using their weighted average making weights different for different people depending on their skin color. 
%

There are other attempts where authors of \cite{6523142,6909939, 7410772, 7412627} have introduced different methodologies to make algorithms for estimating pulse rate robust to illumination variation and motion of the subjects. The paper \cite{6523142} introduces a chrominance-based method to reduce the effect of motion in estimating pulse rate. The authors of \cite{6909939} used a technique in which face tracking and normalized least square adaptive filtering is used to counter the effects of variations due to illumination and subject movement. 
The paper \cite{7410772} resolves the issue of subject movement by choosing the rectangular ROI's on the face relative to the facial landmarks and facial landmarks are tracked in the video using pose-free facial landmark fitting tracker discussed in \cite{yu2016face} followed by the removal of noise due to illumination to extract noise-free PPG signal for estimating pulse rate. 

Recently, the use of machine learning in the prediction of health parameters have gained attention. The paper \cite{osman2015supervised} used a supervised learning methodology to predict the pulse rate from the videos taken from any off-the-shelf camera. Their model showed the possibility of using machine learning methods to estimate the pulse rate. However, our method outperforms their results when the root mean squared error of the predicted pulse rate is compared. The authors in \cite{hsu2017deep} proposed a deep learning methodology to predict the pulse rate from the facial videos. The researchers trained a convolutional neural network (CNN) on the images generated using Short-Time Fourier Transform (STFT) applied on the R, G, \& B channels from the facial region of interests.
The authors of \cite{osman2015supervised, hsu2017deep} only predicted pulse rate, and we extended our work in predicting variance in the pulse rate measurements as well.

All the related work discussed above utilizes filtering and digital signal processing to extract PPG signals from the video which is further used to estimate the PR and PRV.  %
The method proposed in \cite{kumar2015distanceppg} is person dependent since the weights will be different for people with different skin tone. In contrast, we propose a deep learning model to predict the PR which is independent of the person who is being trained. Thus, the model would work even if there is no prior training model built for that individual and hence, making our model robust. 

%


\section{Summary}

We performed a series of galactic disk $N$-body simulations
to investigate the formation and dynamical evolution of spiral arm 
and bar structures in stellar disks which are embedded in live 
dark matter halos.
We adopted a range of initial conditions where the models have similar halo 
rotation curves, but different masses for the disk and bulge components, 
scale lengths, initial $Q$ values, and halo spin parameters.
The results indicate that the bar formation epoch increases exponentially 
as a function of the disk mass fraction with respect to the total mass at the 
reference radius (2.2 times the disk scale length), $f_{\rm d}$.
This relation is a consequence of swing amplification~\citep{1981seng.proc..111T},
which describes the amplification rate of the spiral arm when it transitions from 
leading arm to trailing arm because of the disk's differential rotation.
Swing amplification depends on the properties that characterize the disk, 
Toomre's $Q$, $X$, and $\Gamma$. The growth rate reaches its maximum
for $1<X<2$,  although the position of the peak slightly depends on $Q$ as well as on
$\Gamma$. We computed $X$ for 
$m=2$ ($X_2$), which corresponds to a bar or two-armed spiral, 
for each of our models and found that this value is related to the bar's
formation epoch.

The bar amplitude grows most efficiently when $1<X_2<2$. For models 
with $1<X_2<2$ the bar develops immediately after the start 
of the simulation. As $X_2$ increases beyond $X_2=2$, the growth rate
decreases exponentially. We find that the bar formation epoch increases
exponentially as $X_2$ increases beyond $X_2=2$, in other words
$f_{\rm d}$ decreases. The bar formation epoch exceeds a Hubble time
for $f_{\rm d}\lesssim 0.35$.

Apart from $X$, the growth rate is also influenced by $Q$ where
a larger $Q$ results in a slower growth. This indicates that the bar formation
occurs later for larger values of $Q$. 
Our simulations confirmed this and showed that for the bar ($m=2$) the growth rate
is predicted by swing amplification and becomes visible when it grows beyond a certain amplitude.

Toomre's swing amplification theory further predicts that
the number of spiral arms is related to the mass of the disk, with
massive disks having fewer spiral arms. In addition, larger $\Gamma$
predicts a smaller number of spiral arms.
We confirmed these relations in our simulations. 
The shear rate ($\Gamma$) also affects the pitch angle of spiral
  arms. We further confirmed that our result is consistent with previous
studies.

We found that the disk-to-total mass fraction ($f_{\rm d}$)
and the shear rate ($\Gamma$) are the most important parameters that determine the
morphology of disk galaxies. 
When juxtaposing our models with the Hubble sequence,
the fundamental subdivisions of (barred-)spiral galaxies with 
massive bulges and tightly wound-up spiral arms from S(B)a to S(B)c is 
also be observed as a sequence in our simulations. Where the models 
with either massive bulges or massive disks have more tightly
wound spiral arms. This is because having both a massive disk and bulge results in 
a larger $\Gamma$, i.e., more tightly wound spiral arms. 


Once the
bar is formed it starts to heat the outer parts of the disk.
From this point onwards, 
the self-gravitating spiral arms disappear.
This may be in part caused by the 
lack of gas in our simulations. 
After the bar grows, we no longer discern  
spiral arms in the outer regions of the disk. This could imply
that gas cooling and star formation are required in order to 
maintain spiral structures in barred spiral galaxies for over 
a Hubble time~\citep{1981ApJ...247...77S,1984ApJ...282...61S}.


Our simulations further indicate that non-barred grand-design spirals are
transient structures which immediately evolve into barred
galaxies. Swing amplification teaches us that a massive disk is
required to form two-armed spiral galaxies. This condition, at the
same time, satisfies the short formation time of the bar structure.
Non-barred grand-design spiral galaxies therefore must evolve into barred
galaxies.  We consider that isolated non-barred grand-design spiral galaxies 
are in the process of developing a bar.




% conference papers do not normally have an appendix


% use section* for acknowledgment
\section*{Acknowledgments}


The authors would like to thank Nathan Bartley and Alexander Belikov for testing the system in its early stages of development.



\bibliographystyle{ACM-Reference-Format}
\bibliography{references}


\end{document}


