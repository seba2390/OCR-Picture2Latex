\documentclass[sigconf]{acmart}

\usepackage{booktabs} % For formal tables


\usepackage{graphicx}
\usepackage{color}
\usepackage{soul}
\usepackage{todonotes}
\usepackage{hyperref}
\usepackage{listings}
\usepackage{amsmath}

\definecolor{mygreen}{rgb}{0,0.6,0}
\definecolor{mygray}{rgb}{0.5,0.5,0.5}
\definecolor{mymauve}{rgb}{0.58,0,0.82}

\lstset{ %
  backgroundcolor=\color{white},   % choose the background color; you must add \usepackage{color} or \usepackage{xcolor}; should come as last argument
  basicstyle=\footnotesize,        % the size of the fonts that are used for the code
  breakatwhitespace=false,         % sets if automatic breaks should only happen at whitespace
  breaklines=true,                 % sets automatic line breaking
  captionpos=b,                    % sets the caption-position to bottom
  commentstyle=\color{mygreen},    % comment style
  deletekeywords={...},            % if you want to delete keywords from the given language
  escapeinside={\%*}{*)},          % if you want to add LaTeX within your code
  extendedchars=true,              % lets you use non-ASCII characters; for 8-bits encodings only, does not work with UTF-8
  frame=single,                   % adds a frame around the code
  keepspaces=true,                 % keeps spaces in text, useful for keeping indentation of code (possibly needs columns=flexible)
  keywordstyle=\color{blue},       % keyword style
  language=Octave,                 % the language of the code
  morekeywords={*,...},           % if you want to add more keywords to the set
  numbers=left,                    % where to put the line-numbers; possible values are (none, left, right)
  numbersep=5pt,                   % how far the line-numbers are from the code
  numberstyle=\tiny\color{mygray}, % the style that is used for the line-numbers
  rulecolor=\color{black},         % if not set, the frame-color may be changed on line-breaks within not-black text (e.g. comments (green here))
  showspaces=false,                % show spaces everywhere adding particular underscores; it overrides 'showstringspaces'
  showstringspaces=false,          % underline spaces within strings only
  showtabs=false,                  % show tabs within strings adding particular underscores
  stepnumber=2,                    % the step between two line-numbers. If it's 1, each line will be numbered
  stringstyle=\color{mymauve},     % string literal style
  tabsize=2,                   % sets default tabsize to 2 spaces
  title=\lstname                   % show the filename of files included with \lstinputlisting; also try caption instead of title
}


% Copyright
%\setcopyright{none}
%\setcopyright{acmcopyright}
%\setcopyright{acmlicensed}
\setcopyright{rightsretained}
%\setcopyright{usgov}
%\setcopyright{usgovmixed}
%\setcopyright{cagov}
%\setcopyright{cagovmixed}

\usepackage{array}
\newcolumntype{L}[1]{>{\raggedright\let\newline\\\arraybackslash\hspace{0pt}}m{#1}}
\newcolumntype{C}[1]{>{\centering\let\newline\\\arraybackslash\hspace{0pt}}m{#1}}
\newcolumntype{R}[1]{>{\raggedleft\let\newline\\\arraybackslash\hspace{0pt}}m{#1}}



\newif\iffinal

% Un-comment this line to see proposal without comments
%\finaltrue

\iffinal
  \newcommand\yadu[1]{}
  \newcommand\kyle[1]{}
  \newcommand\eamon[1]{}
\else
  \newcommand\yadu[1]{{\color{blue}[Yadu: #1]}}
  \newcommand\kyle[1]{{\color{red}[Kyle: #1]}}
  \newcommand\eamon[1]{{\color{purple}[Eamon: #1]}}
\fi

% DOI
\acmDOI{10.475/123_4}

% ISBN
\acmISBN{123-4567-24-567/08/06}

%Conference
\acmConference[ScienceCloud'17]{Workshop on Scientific Cloud Computing}{June 2017}{Washington, DC USA} 
\acmYear{2017}
\copyrightyear{2017}

\acmPrice{15.00}


\begin{document}
\title{Enabling Interactive Analytics of Secure Data using Cloud Kotta}
%\subtitle{Extended Abstract}



\author{Yadu N. Babuji, Kyle Chard, and Eamon Duede}
%\authornote{Dr.~Trovato insisted his name be first.}
%\orcid{1234-5678-9012}
\affiliation{%
  \institution{Computation Institute, University of Chicago and Argonne National Laboratory}
  \streetaddress{5735 S Ellis Ave}
  \city{Chicago} 
  \state{Illinois} 
  \postcode{60637}
}
\email{{yadu, chard, eduede}@uchicago.edu}

% The default list of authors is too long for headers}
\renewcommand{\shortauthors}{Y. Babuji et al.}


\begin{abstract}
Research, especially in the social sciences and humanities, is increasingly
reliant on the application of data science methods to analyze large amounts
of (often private) data. Secure data enclaves provide a solution for managing and analyzing private data. However, such enclaves do not readily support
discovery science---a form of exploratory or interactive analysis by which researchers 
execute a range of (sometimes large) analyses in an iterative and collaborative
manner. 
The batch computing model offered by many data enclaves is well suited to executing
large compute tasks; however it is far from ideal for day-to-day discovery science.
As researchers must submit jobs to queues and wait for results, the high latencies inherent in queue-based, batch computing systems hinder interactive
analysis.
%Data science occurs at human pace and
%that involves iterative development which requires interactive systems. Latencies offered
%by typical queue based system slow down development.
In this paper we describe how we have augmented the Cloud Kotta secure data enclave
to support collaborative and interactive analysis of sensitive data.
Our model uses Jupyter notebooks as a flexible analysis environment 
and Python language constructs to support the execution of arbitrary 
functions on private data within this secure framework.
\end{abstract}


% Maybe remove the upper case at some point.
\newcommand{\NAMENS} {\textsc{Cloud Kotta}} % No space aftewards (for brackets etc.)
\newcommand{\NAME} {\textsc{Cloud Kotta }}



% make the title area
\maketitle


\section{Introduction}  \label{sec:introduction}

\newcommand\inexpIntro[3]{#1?(#2,#3).}
\newcommand\rinexpIntro[3]{*#1?(#2,#3).}
\newcommand\outexpIntro[3]{#1!(#2,#3).}
\newcommand\outatomIntro[3]{#1!(#2,#3)}

We propose a fully automated method for proving termination of \(\pi\)-calculus processes.
Although there have been a lot of studies on termination analysis for the \(\pi\)-calculus
and related calculi~\cite{Deng06IC,Demangeon07,SangiorgiTermination,KobayashiHybrid,Yoshida04IC,DBLP:journals/jlp/DemangeonHS10,Venet98SAS}, most of them have been rather theoretical,
and there have been surprisingly little efforts in developing  fully automated termination
verification methods and tools based on them. To our knowledge,
Kobayashi's \typical{}~\cite{TyPiCal,KobayashiHybrid} is the only exception that
can prove termination of \(\pi\)-calculus processes (extended with natural numbers)
fully automatically, but its termination analysis is quite limited (see Section~\ref{sec:relatedwork}).

Our method is based on a reduction to termination analysis for sequential programs:
we translate a \(\pi\)-calculus process \(P\) to a sequential program \(S_P\), so that
if \(S_P\) is terminating, so is \(P\). The reduction allows us to use
powerful, mature methods and tools
for termination analysis of sequential programs~\cite{heizmann2016ultimate,freqterm,DBLP:conf/lics/PodelskiR04,Kuwahara2014Termination,DBLP:journals/cacm/CookPR11}.

The idea of the translation is to convert a chain of communications on replicated input
channels to a chain of recursive function calls of the target sequential program.
Let us consider the following Fibonacci process:
\begin{align*}
    & \rinexpIntro{\fib}{n}{r}
        \ifexp{n<2}{ \soutatom{r}{1} \\ &\quad}
                   { \nuexp{s_1} \nuexp{s_2} (\outatomIntro{\fib}{n-1}{s_1} \PAR \outatomIntro{\fib}{n-2}{s_2} \PAR \sinexp{s_1}{x}\sinexp{s_2}{y}\soutatom{r}{x+y}) \\}
    & \PAR \outatomIntro{\fib}{m}{r}
\end{align*}
Here, the process
$\rinexpIntro{\fib}{n}{r} \ldots$ is a function server that computes the \(n\)-th Fibonacci number
in parallel and returns the result to \(r\),
and $\outatom{\fib}{m}{r}$ sends a request for computing the \(m\)-th Fibonacci number;
those who are not familiar with the syntax of the \(\pi\)-calculus may wish to consult
Section~\ref{sec:targetlanguage} first.
To prove that the process above is terminating for any integer \(m\),
it suffices to show that there is no infinite chain of communications on $\fib$:
\[
    \fib(m,r) \to \fib(m_1,r_1) \to \fib(m_2,r_2) \to \cdots.
\]
We convert the process above to the following program:\footnote{The actual translation
  given later is a little more complex.}
\begin{verbatim}
 let rec fib(n) = if n<2 then () else (fib(n-1) [] fib(n-2)) in
 fib(m)
\end{verbatim}
Here, \texttt{[]} represents the non-deterministic choice.
Note that, although the calculation of Fibonacci numbers is not preserved,
for each chain of communications on \texttt{fib}, there is a corresponding
sequence of recursive calls:
\[
\mathtt{fib}(m) \to \mathtt{fib}(m_1) \to \mathtt{fib}(m_2) \to \cdots.
\]
Thus, the termination of the sequential program above implies the termination of
the original process.
As shown in the example above, (i) each communication on a replicated input channel
is converted to a function call, (ii) each communication on a non-replicated input
channel is just removed (or, in the actual translation, replaced by a call of
a trivial function defined by \(f(\seq{x})=(\,)\)), and (iii) parallel composition
is replaced by a non-deterministic choice.
We formalize the translation outlined above and prove its correctness.

The basic translation sketched above sometimes loses too much information.
For example, consider the following process:
\begin{align*}
    & \rinexpIntro{\pre}{n}{r} \soutatom{r}{n-1} \\
    & \PAR \rinexpIntro{f}{n}{r} \ifexp{n<0}{ \soutatom{r}{1} }
                                       { \nuexp{s} (\outatomIntro{\pre}{n}{s} \PAR \sinexp{s}{x}\outatomIntro{f}{x}{r}) } \\
    & \PAR \outatomIntro{f}{m}{r}
\end{align*}
The translation sketched above would yield:
\begin{verbatim}
  let pred(n) = n-1 in
  let rec f(n) = if n<0 then () else (pred(n) [] f(*)) in
  f(m)
\end{verbatim}
Here, \texttt{*} represents a non-deterministic integer: since we have removed
the input $\sinatom{s}{x}$, we do not have information about the value of \( x \).
As a result, the sequential program above is non-terminating, although the original
process is terminating.
To remedy this problem, we also refine the basic translation above by using a refinement
type system for the \(\pi\)-calculus. Using the refinement type system,
we can infer that the value of \(x\) in the original process is less than \(n\),
so that we can refine the definition of \texttt{f} to:
\begin{verbatim}
 let rec f(n) = ... else (pred(n) [] let x=* in assume(x<n);f(x))
\end{verbatim}
The target program is now terminating, from which
we can deduce that the original process is also terminating.
We have implemented an automated tool based on the refined translation above.

The contributions of this paper are summarized as follows.
\begin{itemize}
\item The formalization of the basic translation from the \(\pi\)-calculus
  (extended with integers) to sequential programs, and a proof of its correctness.
\item The formalization of a refined translation based on a refinement type system.
\item An implementation of the refined translation, including automated refinement type
  inference based on CHC solving, and experiments to evaluate the effectiveness of
  our method.
\end{itemize}

The rest of this paper is structured as follows.
Section~\ref{sec:targetlanguage} introduces the source and target languages
of our translation.
Section~\ref{sec:approach} 
formalizes the basic translation, and proves its correctness.
Section~\ref{sec:refinement} refines the basic translation by using a refinement type system.
Section~\ref{sec:implementation} reports an implementation and experiments.
Section~\ref{sec:relatedwork} discusses related work,
and Section~\ref{sec:conclusion} concludes the paper.


% Panoptic segmentation

% 3D segmentation

% Multi-object tracking

% Online 3D panoptic:

% PanopticFusion: (IROS 2019)
% https://arxiv.org/pdf/1903.01177.pdf
%
% - most similar to ours
% - PSPNet + M-RCNN + 2D fusion
% - volumetric mapping, 
% - greedy matching with IoU -> optimal only with 0.5 threshold
% - voxel & class weighting
% - CRF regularisation
%
% - good:
%
% - bad:
%  - CRF post-processing step
%  - greedy data-association
%    - can't be tuned for lower overlap ratios -> has to have high framerate, large changes in viewpoint could break this
%    - IoU: sensitive to 2D labels projecting over object borders (CRF and voxel weighting seem to alleviate this)

% Voxblox++: (Robotics & automation letters 2019)
% https://arxiv.org/pdf/1903.00268.pdf
% https://github.com/ethz-asl/voxblox-plusplus
%
% - M-RCNN + geometric segmentation + fusion 
% - data association of geometric segments with 3D overlap (no. points inside volume), fixed threshold for min number of points
% - instance label is assigned to a segment based on highest overlap
% - only one detected segment per reference label, as in PanopticFusion and Ours
% - TSDF Integration 
%
% good: 
% - because of geometric segmentation objects with no associated semantic class can also be segmented
% bad:
% - two different object segment types -> confusing, overly complicated ?
% - quite inaccurate (fixed below)

% Reconstructing Interactive 3D Scenes by Panoptic Mapping and CAD Model Alignments (ICRA 2021)
% https://arxiv.org/pdf/2103.16095.pdf
% https://github.com/hmz-15/Interactive-Scene-Reconstruction
%
% - based heavily on Voxblox++, much more accurate
% - Scene-graph ("contact graph") for mapping object relations
% - Search & replace voxels with CAD models, with geometrical and physical constraints
% - Object 6D pose
% - Format for robot interaction
%
% - Segmentation: bilateral fusion of geomatric and semantic segments -> reduce segmentation noise compared to Voxblox++
% - Fusion: triplet count improves consistency over Voxblox++ pairwise count strategy (take semantic label into account in addition to instance and geometry)
% - Fusion: instance labels are also combined if there is enough overlap with common geometric label for long enough time
%   - this means multiple detections can match the same reference unlike ours, voxblox++ and PanopticFusion ?
%

% Panoptic-MOPE: (ROBOTICS AND AUTOMATION LETTERS 2020)
% https://ieeexplore.ieee.org/stamp/stamp.jsp?tp=&arnumber=8977356
% https://github.com/hoangcuongbk80/Object-RPE/tree/panoptic-mope
%
% - novel RGB-D semantic segmentation model + M-RCNN
% - camera tracking based on "addaptively weighted optimization of geometric, appearance, and semantic cues"
% - surfel map: 
%   - how does it scale ? authors satate they tested on room-sized environments, but could be applied in larger scale as well ...
%     - could maybe be applied as VO in a SLAM algorithm ...
%   - demo only on a small pallet + surroundings, might not be applicable in large-scale SLAM

% US VS THEM:
%
% - based heavily on PanopticFusion, with modifications:
%   - instead of greedy data-association (which seems to be the case in others as well), we solve LAP (JPDA?)
%     - overlap threshold can be tuned, which renders the algorithm more flexible
%     - could be extended to dynamic tracking ?
%   - multiple options for association likelihood
%   - outlier rejection (either clustering or probabilistic)
%   - test different options for decreasing processing time
%   - no post-processing
%
% - model-agnostic:
%   - completely separated from segmentation
%   - does not care how point clouds are obtained -> applicable for LIDAR segmentation (e.g. EfficientLPS) as well
%
% - also agnostic to localisation method
%   - could, however, be utilised to find landmark locations / poses

% More compact version of this paragraph to introduction to save space?
%Panoptic segmentation -- proposed in \cite{panoptic_segmentation} -- aims to solve the unified task of semantic- and instance segmentation. Semantic classes are separated to \textit{stuff} -- amorphous, unquantifiable regions like sky, road or floor -- and \textit{things} -- quantifiable objects. The distinction between the two can vary depending on the application, but a semantic class can only belong to one or another. The article also proposes a unified panoptic evaluation metric, coined \textbf{Panoptic Quality} (PQ). Many 2D approaches to panoptic segmentation -- \textit{e.g.} \cite{panopticfpn,seamless,panoptic_deeplab,efficientps} -- have since been proposed. Deep neural networks for performing semantic- or instance segmentation directly on the 3D reconstruction -- \textit{e.g.} on \cite{scannet,s3dis,paris_lille_3d} -- have also been proposed, but since they require the reconstructed 3D scene, they are mostly offline approaches and therefore out of scope for this work. Some recent works also apply panoptic segmentation to point clouds -- \textit{e.g.} methods in the SemanticKITTI panoptic segmentation competition \cite{semantic_kitti} -- mostly aimed at segmenting LiDAR output. They are suitable for online processing, but similar to RGB-D images require a method for tracking object instances persistent in both time and space. In fact, our proposed method, as well as some others mentioned in this work, could use segmented LiDAR point clouds as an input similarly to RGB-D images.

PanopticFusion \cite{panopticfusion} is the first work to propose online integration of panoptic image segmentations to a 3D reconstruction. They integrate point clouds generated from segmented images to a TSDF voxel volume \cite{tsdf,voxblox} by greedily matching detected segments with the reconstruction and regulating each voxel's corresponding instance with a weighting function. Semantic labels are inferred in a bayesian manner based on confidence scores provided by the segmentation model. They also apply a Conditional Random Field (CRF) to regularise the reconstruction, improving results significantly. Voxblox++ \cite{voxblox++} -- introduced later the same year -- is a similar approach that also integrates segmented RGB-D images into a TSDF volume. It leverages geometric segmentation of depth images to improve instance segmentation accuracy. Both geometric and semantic segments are used to compute a pair-wise weight, which is used to greedily match them with segments in the reconstruction. Because of the geometric segmentation, the method allows segmentation of objects with no known semantic class in addition to objects recognised by the instance segmentation model. 

Recently, \cite{interactive_3d_scenes} built upon the idea of Voxblox++. They apply Voxblox++ for 3D instance integration, with two small but effective modifications: the pair-wise weight is replaced by a triplet weight that also takes semantic labels into account in the fusion, and -- in addition to geometric segments -- instance segments are fused if they overlap by a significant amount. The article introduces a method for searching and aligning CAD models to reconstructed objects based on geometry and semantic class, as well as geometrical and physical rules. With the CAD models, a contact graph and interactive virtual scene are reconstructed to allow a robot to simulate its interaction with the environment. SceneGraphFusion \cite{scenegraphfusion} is another approach that forms a scene graph online from a stream of RGB-D images, but unlike the above-mentioned approach, it generates the graph with a deep neural network, after which the panoptic labels for geometrically segmented portions of the 3D reconstruction are produced a side product.

Panoptic-MOPE \cite{panoptic_mope} is another recent approach, which integrates sequences of RGB-D images into a surfel reconstruction. Unlike other mentioned approaches -- which assume the camera pose either known or estimated elsewhere -- it also tracks camera movements based on geometric-, appearance- and semantic cues. The method also applies a novel RGB-D panoptic segmentation model. Although it is only tested on room-sized environments, the authors claim it could be scaled to larger environments as well.
% !TEX root = ../top.tex
% !TEX spellcheck = en-US

\begin{figure}[t]
\centering
\includegraphics[width=0.99\linewidth]{./fig/arch/arch.pdf}
\vspace{-3mm}
    \caption{{\bf Our single-stage approach.} We use an encoder-decoder architecture to progressively downsample the image and then to re-expand it. At each level of the decoder, we establish 3D-to-2D correspondences. Finally, we use a RANSAC-based PnP strategy~\cite{Lepetit09} 
    % \WJ{citation?} \YH{Fixed} 
     to infer a single reliable pose from these sets of correspondences. 
    % \WJ{Impossible to read text in printout, please try to have text in the figure with a similar font size as the surrounding article.}\YH{Fixed}
    }
\label{fig:arch}
\end{figure}

\section{Applications}\label{sec:applications}

In this section, we present applications of our results to two SDPs: Max-Cut and matrix completion, both of which are important problems in the learning domain and have been studied extensively. Interest has grown to develop efficient solvers for these SDPs~\citep{arora2007combinatorial, pmlr-v65-mei17a, hardt2013understanding, bandeira2016low}.
%General SDP solvers such as interior point or ellipsoid method can lead to slow algorithms for these problems. Instead, over the years, specialized efficient algorithms have been developed for these problems \citep{arora2007combinatorial, hardt2013understanding}. 

This work differs from previous efforts in at least two ways. First, we aim to demonstrate that Burer--Monteiro-style approaches, which are often used in practice, can indeed lead to provably efficient algorithms for general SDPs. We believe that building upon this work, it should be possible to improve the time-complexity guarantees of such factorization-based algorithms. Second, we note that several problems formulated as SDPs in fact necessitate low-rank solutions, for example because of memory concerns (as is the case in matrix completion),  and factorization approaches provide a natural means to control rank. % For such problem, existing methods do not provide low rank solutions while our method is guaranteed to give a low-rank solution. 

%We note that both of these problems have been extensively studied and for both of them there exist highly specialized algorithms that are highly efficient~\cite{arora2007combinatorial,hardt2013understanding}. Our results here do not beat them; rather our goal here is to demonstrate that the Burer-Monteiro approach can successfully solve these SDPs in polynomial time. In practice, this approach is much faster than other generic SDP solvers such as interior point method and ellipsoid method, and in addition returns low rank solutions.
\subsection{Max-Cut}

We first consider the popular Max-Cut problem which finds applications in clustering related problems. In a seminal paper, \cite{goemans1995improved} defined the following SDP to solve the Max-Cut problem: $\min_{X\in \Rnn} \ip{C}{X}, \mbox{s.t. } X_{ii} = 1 \; \forall \; 1 \leq i \leq n, X \succeq 0 $, where $n$ is the number of vertices in the given graph and $C$ is its adjacency matrix. Since the constraint set also satisfies $\trace{X}=n$, we consider the following penalized, non-convex version of the problem.
% \begin{align*}	& \min_{X\in \Rnn} \ip{C}{X} \\
%	& \mbox{s.t. } X_{ii} = 1 \; \forall \; 1 \leq i \leq n \\
%	& \qquad X \succeq 0,
%\end{align*}
\begin{align}
	\widehat{L}_{\mu}(U) \defeq \ip{C+G}{U\trans{U}} + \mu\left(\left(\ip{I}{U\trans{U}}-n\right)^2 + \sum_{i=1}^{n}\left(\ip{e_i \trans{e_i}}{U\trans{U}}-1\right)^2\right),\label{eqn:maxcut}
\end{align}
where $G$ is a random symmetric Gaussian matrix.  Let $\widehat F_{\mu}(UU^T) = \widehat L_{\mu} (U)$. After some simplifying computations, we have the following corollary of Theorem~\ref{thm:optimal_approx_compact}.
\begin{corollary}\label{cor:maxcut}
There exists an absolute numerical constant $c_1$ such that the following holds. With probability greater than $1-\delta$,
every $(\eps, \gamma)$-SOSP $U$ of the perturbed Max-Cut problem $\widehat{L}_{\mu}(U)$~\eqref{eqn:maxcut} with:
\begin{align*}\epsilon \leq \frac{1}{c_1} \left(\frac{\gamma \sigma_G^2}{\mu n}\right)^{2/3},~~ \text{ and } ~~  k = \tilde{\Omega} \left( \sqrt{n \log\left(\frac{\mu^2 \sqrt{n}}{\sigma_G}\right)}\right),
\end{align*}
satisfies $	\widehat{F}_{\mu}(UU^T) - \widehat{F}_{\mu}(X^*) \leq \gamma \sqrt{\epsilon} \trace{X^*} +\frac{1}{2} \epsilon \frob{U}$, where $X^*$ is a global optimum of $\widehat{F}_{\mu}(X)$.
%\begin{align*}
%	\widehat{L}_{\mu}(U) - \widehat{L}_{\mu}(U^*) \leq \gamma \sqrt{\epsilon} \trace{U^* \trans{U^*}} + \epsilon \frob{U}.
%\end{align*}
\end{corollary}
The above result states that for the penalized version of the perturbed Max-Cut SDP, the Burer--Monteiro approach finds an approximate global optimum as soon as the factorization rank $k = \tilde{\Omega}(\sqrt{n})$. Existing results for Max-Cut using this approach either only handle exact SOSPs~\citep{boumal2016non}, or require $k=n+1$~\citep{boumal2016globalrates}, or require $k$ that is dependent on $\frac{1}{\eps}$~\citep{pmlr-v65-mei17a}. Moreover, complexity per iteration scales only linearly with the number of edges in the graph. %However, current analysis is required to set $\epsilon$ to be fairly small which can lead to a super-linear algorithm; we leave further tightening of dependence on $\epsilon$ for future work.


\subsection{Matrix Completion}
In this section we specialize our results for the matrix completion problem \cite{candes2009exact}. The goal of a matrix completion problem is to find a low-rank matrix $M$ using only a small number of its entries, with applications in recommender systems. To ensure that the computed matrix is low-rank and generalizes well, one typically imposes nuclear-norm regularization which leads to the following SDP: 

\begin{minipage}{0.2\linewidth}
	\begin{align*}
	\min &\quad \trace{W_1} + \trace{W_2}\\ \text{s. t. }&\quad X_{ij} =M_{ij}, (i,j) \in \calS \\  &\quad \begin{bmatrix}W_1 & X \\ X^T & W_2\end{bmatrix} \succeq 0.
	\end{align*}
\end{minipage}
\begin{minipage}{0.05\linewidth}
	\begin{align*}
		\equiv \\
	\end{align*} \break
\end{minipage}
\begin{minipage}{0.6\linewidth}
	\begin{align*}
	\min & \quad \ip{I}{Z} \nonumber \\ \text{s. t. }&\quad \frac{1}{2}\ip{e_{i+n}e_{j+n}^T + e_{j+n} e_{i+n}^T}{Z} = M_{ij}, (i,j) \in \calS \nonumber \\  &\quad Z \succeq 0.
	\end{align*}
\end{minipage}
%\begin{align*} \min &\quad \trace{W_1} + \trace{W_2}\\ \text{s. t. }&\quad X_{ij} =M_{ij}, (i,j) \in \Omega \\  &\quad \begin{bmatrix}W_1 & X \\ X^T & W_2\end{bmatrix} \succeq 0.
%\end{align*} 
\noindent Here $\calS$ is the set of observed indices of $M$ and $Z\defeq \begin{bmatrix}W_1 & X \\ X^T & W_2\end{bmatrix}$. % by $Z$, we can rewrite the above SDP as \begin{align} \min & \quad \ip{Z}{I} \nonumber \\ \text{s. t. }&\quad \frac{1}{2}\ip{e_{i+n}e_{j+n}^T + e_{j+n} e_{i+n}^T}{Z} = M_{ij}, (i,j) \in \Omega \nonumber \\  &\quad Z \su,cceq 0. \label{eq:mc_sdp}\end{align}
Let
\begin{align}
	\widehat{L}_{\mu}(U) = \ip{I+G}{UU^T} + \mu \sum_{i=1}^m \left(\frac{1}{2}\ip{e_{i+n}e_{j+n}^T + e_{j+n} e_{i+n}^T}{UU^T} - M_{ij} \right)^2  \label{eq:matcomp}
\end{align}
be the corresponding penalty objective.  Let $\widehat F_{\mu}(UU^T) = \widehat L_{\mu} (U)$. The objective is positive definite with $\lambda_1(C)=\lambda_n(C)=1$. Also, since $\calA$ is a sub-sampling operator, $\|\calA\| \leq 1$. Finally, for $\eps^2 \leq \frac{\mu}{2}\sqrt{\sum_{(i,j) \in \calS} M_{ij}^2}$, the residues are bounded by: \begin{align*} B&=\|\calA\| \max \left \{ \left( \frac{2\eps} {\lambda_n(C)}\right)^2, \frac{2\mu}{\lambda_n(C)} \|\vb\|_2^2 \right \}+\|\vb\|_2 \leq \max  3\mu \sqrt{\sum_{(i,j) \in \calS} M_{ij}^2}. \end{align*}


\noindent Applying Theorem~\ref{thm:optimal_approx} for this setting gives the following corollary.
\begin{corollary}\label{cor:mc_optimal}There exists an absolute numerical constant $c_2$ such that the following holds. With probability greater than $1-\delta$,
every $(\eps, \gamma)$-SOSP $U$ of the perturbed matrix completion problem $\widehat{L}_{\mu}(U)$~\eqref{eq:matcomp} with:
\begin{align*}\sG \leq \frac{1}{4\sqrt{n \log(n/ \delta)}},~~ \eps \leq \frac{1}{c_2}\left(\frac{\gamma \abs{\calS} \sigma_G^2 }{ n  \mu }\right)^{\sfrac{2}{3}}, ~\text{ and }~ k = \tilde{\Omega} \left( \sqrt{ \abs{\calS}   \log\left(\frac{\mu^2 \sqrt{n} \sqrt{\sum_{(i,j) \in \calS} M_{ij}^2}}{\sigma_G}\right) } \right),\end{align*} satisfies $\widehat{F}_{\mu}(UU^T)  - \widehat F_{\mu}(X^*)  \leq \gamma \sqrt{\epsilon} \trace{X^*} + \frac{1}{2} \eps \|U\|_F$, where $X^*$ is a global optimum of $\widehat{F}_{\mu}(X)$.
\end{corollary}
\noindent This result shows that for the  matrix completion problem with $m$ observations, for rank $\tilde{\Omega}(\sqrt{m})$, any approximate local minimum of the factorized and penalized problem is an approximate global minimum. 

Most of the existing results on matrix completion either require strong distribution assumptions on $\calS$ and incoherence assumptions on $M$ to recover a low-rank solution \citep{candes2009exact, jain2013low}. The standard nuclear norm minimization algorithms are not guaranteed to converge to low-rank solutions without these assumptions,  which implies that the entire matrix would need to be stored for prediction which is infeasible in practice. Similarly,  generalization error bounds \citep{foygel2011concentration} as well as differential privacy guarantees  depend on recovery of a low-rank solution.

Our result guarantees finding a rank -$\tilde{\Omega}(\sqrt{m})$ solution without any statistical assumptions on the sampling or the matrix. The tradeoff is our results do not guarantee finding a lower (potentially a constant) rank solution, even if one exists for a given problem. 


%\subsection{Normalized Cut}
%
%In this section we will consider the problem of computing the Normalized cut of a given graph $\calG$ \citep{shi2000normalized}. Given a graph $G$ with $n$ vertices and $e$ edges, the normalized cut is the partition of vertices into two sets $S$ and $S^c$ that minimizes, $$\frac{cut(S,S^c)}{D(S)}+ \frac{cut(S,S^c)}{D(S^c)}. $$ Here $cut(S,S^c)$ is the number of edges between $S$ and $S^c$. $D(S)$ is the sum of degree of vertices in $S$. This problem is NP-hard in the worst case. 
%
%Let $\widehat{X}$ be a $n+1 \times n+1$ matrix, with the following structure, $\widehat{X} = \begin{bmatrix}  X & 0_{n \times 1} \\ 0_{1 \times n} & x \end{bmatrix}$. \citet{bie2006fast} proposed the following SDP relaxation to find the normalized cut.
%
% \begin{align*} \underset{\widehat{X}}{\minimize} &\quad \ip{\widehat{X}}{L}\\ \text{subject to } &\quad \widehat{X} \succeq 0 \\
%&\quad \ip{\widehat{X}}{A_i} =0, i \in [n]\\
%&\quad \ip{\widehat{X}}{A_{n+1}}=-1 \\
%\end{align*}
%Here $L$ is the graph Laplacian divided by twice the number of edges, $2e$. $A_i$ ($i \in [n]$) is a matrix with $1$ at $(i,i)$ and $-1$ at $(n+1,n+1)$ entry, with rest of the entries begin $0$s. $\displaystyle A_{n+1} =\begin{bmatrix} dd^T/(4e^2) & 0_{n \times 1} \\0_{1 \times n} & -1 \end{bmatrix}$, where $d$ is the vector of degrees of the vertices.
%
%
%

\textbf{Related work}:
% Object detection related datasets/algo in non-medical domain
% Locally labeled CXR dataset
A few CXR datasets have localized abnormality annotations \cite{shih2019augmenting,filice2020crowdsourcing,jaeger2014two} that are curated manually. These are high quality gold standard ground truth datasets but tend to be smaller in scale (< 30,000 images) and have a narrow coverage, with typically only 1-2 labels. In addition, since most labeling efforts only have abnormality semantics attached, no direct relationships with the affected anatomical locations are available. 

%MEHDI: repeated concepts from above. I am removing the following: 

%The lack of anatomic semantics in the annotation is a limitation for complex multi-modal clinical reasoning work, e.g., differential diagnosis, since clinicians often integrate information along anatomical lines, and for downstream report generation tasks, which often requires describing not only the abnormality but also correctly communicate the location of the abnormalities (and medical devices) to the receiving clinicians. 

Two recent CXR datasets have labels for anatomies described in the reports. In \cite{datta2020dataset}, a small manually annotated dataset (2000 reports) included 10 abnormalities that are individually associated with 29 unique spatial locations (anatomies) at the report level. Another CXR dataset has automatically extracted abnormality and anatomy labels as disconnected concepts that are only correlated at the study level from  160,000 reports using a supervised NLP algorithm \cite{bustos2020padchest}. This was trained on a smaller set of manually annotated data. Neither datasets contain localized annotations for the associated CXR images, nor any comparison relation annotations between sequential exams, both of which are available in the Chest ImaGenome dataset. In Table \ref{tab:related}, we present a comparison of our Chest ImagGenome dataset with other datasets available in the literature.

% Table -- Kashyap

% MEdical imaging datasets to go here: Discussed that we will only focus on cxr datasets that are available for this paper. 
% \caption{\color{red} Kashyap, feel free to continue with the table. We should remove the questionmarks and add a line for our dataset (since all others are not graph). For longer text, using abbreviations and explaining them in the caption often works better. If fill in the values is not possible, it is better to remove the table altogether.}


\begin{table}[t!]
\caption{Summary of existing chest X-ray datasets}
\resizebox{\textwidth}{!}{%
\begin{tabular}{@{}lllllllll@{}}
\toprule
\textbf{Dataset} & \textbf{Annotation Level} & \textbf{Annotation Method} & \textbf{Num Labels} & \textbf{Anatomy Labeled} & \textbf{Graph} & \textbf{Dataset Size} & \textbf{Temporal Labels} & \textbf{Reports} \\ \midrule
SIIM-ACR Pneumothorax Segmentation \cite{filice2020crowdsourcing} & Segmentation & Manual + augmented & 1 & No & No & 12,047 & No & No \\
RSNA Pneumonia Detection Challenge   \cite{shih2019augmenting} & Bounding Boxes & Manual & 1 & No & No & 30,000 & No & No \\
Indiana University Chest X-ray collection \cite{demner2016preparing} & Global & Automated & 10 & No & No & 3,813 & No & Yes \\
NIH CXR dataset \cite{wang2017chestx} & Global & Automated & 14 & No & No & 112,120 & No & No \\
PLCO \cite{team2000prostate} & Global & Automated & 24 & Yes & No & 236,000 & Yes & No \\
Stanford CheXpert \cite{irvin2019chexpert} & Global & Automated & 14 & No & No & 224,316 & No & No \\
MIMIC-CXR \cite{johnson2019mimic} & Global & Automated & 14 & No & No & 377,110 & No & Yes \\
Dutta \cite{datta2020dataset} & Global & Manual & 10 & Yes & Yes & 2,000 & No & Yes \\
PadChest \cite{bustos2020padchest} & Global & Manual + automated & 297 & Yes & No & 160,868 & No & Yes \\
Montgomery County Chest X-ray   \cite{jaeger2014two} & Segmentation & Manual & 1 & Yes & No & 138 & No & No \\
Shenzen Hospital Chest X-ray   \cite{jaeger2014two} & Segmentation & Manual & 1 & Yes & No & 662 & No & No \\  \hline \hline
\textbf{Chest ImaGenome} & Bounding Boxes & Automated & 131 & Yes & Yes & 242,072 & Yes & Yes \\
\bottomrule
\end{tabular}%
}
\label{tab:related}
\vspace{-0.4cm}
\end{table}
% removed (Derived from MIMIC-CXR \cite{johnson2019mimic}) % makes table really small



\section{Summary}

We performed a series of galactic disk $N$-body simulations
to investigate the formation and dynamical evolution of spiral arm 
and bar structures in stellar disks which are embedded in live 
dark matter halos.
We adopted a range of initial conditions where the models have similar halo 
rotation curves, but different masses for the disk and bulge components, 
scale lengths, initial $Q$ values, and halo spin parameters.
The results indicate that the bar formation epoch increases exponentially 
as a function of the disk mass fraction with respect to the total mass at the 
reference radius (2.2 times the disk scale length), $f_{\rm d}$.
This relation is a consequence of swing amplification~\citep{1981seng.proc..111T},
which describes the amplification rate of the spiral arm when it transitions from 
leading arm to trailing arm because of the disk's differential rotation.
Swing amplification depends on the properties that characterize the disk, 
Toomre's $Q$, $X$, and $\Gamma$. The growth rate reaches its maximum
for $1<X<2$,  although the position of the peak slightly depends on $Q$ as well as on
$\Gamma$. We computed $X$ for 
$m=2$ ($X_2$), which corresponds to a bar or two-armed spiral, 
for each of our models and found that this value is related to the bar's
formation epoch.

The bar amplitude grows most efficiently when $1<X_2<2$. For models 
with $1<X_2<2$ the bar develops immediately after the start 
of the simulation. As $X_2$ increases beyond $X_2=2$, the growth rate
decreases exponentially. We find that the bar formation epoch increases
exponentially as $X_2$ increases beyond $X_2=2$, in other words
$f_{\rm d}$ decreases. The bar formation epoch exceeds a Hubble time
for $f_{\rm d}\lesssim 0.35$.

Apart from $X$, the growth rate is also influenced by $Q$ where
a larger $Q$ results in a slower growth. This indicates that the bar formation
occurs later for larger values of $Q$. 
Our simulations confirmed this and showed that for the bar ($m=2$) the growth rate
is predicted by swing amplification and becomes visible when it grows beyond a certain amplitude.

Toomre's swing amplification theory further predicts that
the number of spiral arms is related to the mass of the disk, with
massive disks having fewer spiral arms. In addition, larger $\Gamma$
predicts a smaller number of spiral arms.
We confirmed these relations in our simulations. 
The shear rate ($\Gamma$) also affects the pitch angle of spiral
  arms. We further confirmed that our result is consistent with previous
studies.

We found that the disk-to-total mass fraction ($f_{\rm d}$)
and the shear rate ($\Gamma$) are the most important parameters that determine the
morphology of disk galaxies. 
When juxtaposing our models with the Hubble sequence,
the fundamental subdivisions of (barred-)spiral galaxies with 
massive bulges and tightly wound-up spiral arms from S(B)a to S(B)c is 
also be observed as a sequence in our simulations. Where the models 
with either massive bulges or massive disks have more tightly
wound spiral arms. This is because having both a massive disk and bulge results in 
a larger $\Gamma$, i.e., more tightly wound spiral arms. 


Once the
bar is formed it starts to heat the outer parts of the disk.
From this point onwards, 
the self-gravitating spiral arms disappear.
This may be in part caused by the 
lack of gas in our simulations. 
After the bar grows, we no longer discern  
spiral arms in the outer regions of the disk. This could imply
that gas cooling and star formation are required in order to 
maintain spiral structures in barred spiral galaxies for over 
a Hubble time~\citep{1981ApJ...247...77S,1984ApJ...282...61S}.


Our simulations further indicate that non-barred grand-design spirals are
transient structures which immediately evolve into barred
galaxies. Swing amplification teaches us that a massive disk is
required to form two-armed spiral galaxies. This condition, at the
same time, satisfies the short formation time of the bar structure.
Non-barred grand-design spiral galaxies therefore must evolve into barred
galaxies.  We consider that isolated non-barred grand-design spiral galaxies 
are in the process of developing a bar.




% conference papers do not normally have an appendix


% use section* for acknowledgment
\section*{Acknowledgments}


The authors would like to thank Nathan Bartley and Alexander Belikov for testing the system in its early stages of development.



\bibliographystyle{ACM-Reference-Format}
\bibliography{references}


\end{document}


