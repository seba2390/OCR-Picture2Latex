\section{Summary}

We have described the enhancements we have made to the Cloud Kotta
secure data enclave to support interactive data analytics on protected data. 
The motivation for our work was based on the needs of an increasingly common
class of researcher: those who utilize exploratory data science to analyze large, protected datasets.
To address the needs of these researchers we described how we have integrated Jupyter notebooks 
with Cloud Kotta to fulfill the significant gap between interactive and queue-based systems. 
Our approach relies on JupyterHub to enable multi-user Jupyter
environments and the creation of a lightweight Python library that supports semi-transparent execution
of code functions using a Python decorator. 
Initial experiences with this platform have been positive,  
several researchers have now adopted this system in their every-day 
research. 

Our future work is primarily focused on completing the integration
of our authentication and authorization systems. In so doing, we will
simplify user experience by enabling them to use the same identities
in both environments and by transparently enabling connections from 
Jupyter notebooks to Cloud Kotta. As a secondary goal we will extend
the \texttt{kotta} library to include support for dependency management.
This support will allow users of our system to compose more complex
workflows comprised of independent steps. 

%\begin{itemize}
%\item We have filled a huge gap between interactive systems and queue based systems
%\item The Kotta library brings virtually infinite compute capacity to interactive sessions
%\item Using non-blocking models, users can easily exploit many-task parallelism
%\item We enable all of the above at minimal additional cost. {\$200}
%\end{itemize}
%
%\begin{itemize}
%\item Single sign-on to further integrate the authentication system for both Cloud Kotta and JupyterHub.
%\item Globus for auth ?
%\item The kotta library at this point does not support dependency management. Extending support for dependency management would allow for composing arbitrary workflows.
%\end{itemize}
