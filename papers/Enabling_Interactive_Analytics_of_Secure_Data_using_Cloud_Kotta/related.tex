\section{Related Work}

There are few systems that enable secure management and 
analysis of research data. While researchers have explored 
methods for enhancing client-side software for analysis of 
large amounts of local data~\cite{saleem14bigexcel} and 
exploited hybrid cloud models to scale analytics~\cite{abramson14hybrid},
they do not provide a flexible, integrated environment
for managing and analyzing data, and to the best of our knowledge
none support interactive analysis on protected data. Perhaps the most similar 
approach to ours is the data capsule~\cite{zheng14capsules} model used by
the Hathi Trust to support secure, non-consumptive analysis of data by leveraging 
controlled virtual machines. However, this model is designed to use virtual
machine constructs and therefore it is lower level and does not support interactive analytics via
an easy to use Jupyter notebook.

Science gateways~\cite{wikinsdiehr07gateways} have long been used to 
provide simplified, domain-specific access to large scale computing
infrastructure. However, unlike Cloud Kotta, they focus primarily
on HPC infrastructure and typically support fixed analysis types
(via a web form) and queue-based execution models. While some
gateways now leverage cloud infrastructure~\cite{madduri2014globus}
none provide the rich security policies, extensible execution model, 
or interactive analysis model provided by Cloud Kotta. 

There are several efforts to provide multi-user, interactive
analysis environments built around Jupyter notebooks.
For example, JupyterHub~\cite{jupyterhub}, the system 
we build upon here, allows multiple users to
instantiate instances of Jupyter notebooks. It is typically
deployed on a large machine and uses a proxy-based model
to forward requests to a particular Jupyter instance. 
Tmpnb~\cite{tmpnb} aims to satisfy a similar multi-user
model for running temporary notebooks. It launches Docker
containers for each notebook and proxies requests
to each container. Tmpnb has been used to provide
temporary notebooks for replicating analyses published 
in Nature~\cite{shen14notebooks}. 
Binder~\cite{binder} applies a similar model, using
Docker containers to execute Jupyter notebooks directly
from GitHub repositories.  The public deployment
is hosted on a small Google Compute Engine Cluster that can
scale with usage. 
While these systems provide on-demand interactive analysis environments,
they do so at the notebook level and focus on computational reproducibility. 
Cloud Kotta instead offers a similar environment for composing and executing notebooks
that analyze protected data. Our model is unique in that it extends these 
notebook environments via language constructs to exploit access
to secure data and use of large scale computing resources. 

Finally, there are many examples of libraries and programming 
languages that aim to simplify the use of parallel and distributed computing resources. 
In particular, IPythonParallel provides a simple model for 
enabling parallel execution of Python functions. IPythonParallel
allows user-defined decorators to be associated with functions
which are subsequently sent as parallel jobs to a predefined 
execution system. This model provides no support for data management, 
implementing security models at the granularity of functions, or autoscaling in a cloud
environment. It would require considerable effort to integrate such frameworks
with the Cloud Kotta security fabric. 

%Whats wrong with just AWS stuff.. 
%
%Whats wrong with gateways
%
%Whats wrong with SCRIMP and other services like that. 
%
%More than automatic clusters in the cloud. 