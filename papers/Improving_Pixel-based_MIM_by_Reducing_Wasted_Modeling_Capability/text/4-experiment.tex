\section{Experiment}
In~\autoref{sec:exp:settings}, we present the experimental settings for pre-training and evaluation. Next, in \autoref{sec:main_results}, we apply MFF to two MIM baselines, namely MAE~\cite{MAE} and PixMIM~\cite{pixmim}, and show the improvements brought by such design. In addition, we also evaluate the effectiveness of MFF using a smaller model (\eg, ViT-S) and fine-tune the pre-trained model under a low-shot setting. To evaluate the robustness of our proposed method, \autoref{sec:robust_eval} includes additional analyses that assess the robustness of pre-trained models against out-of-distribution (OOD) ImageNet variants. Finally, \autoref{sec:ablation} presents comprehensive ablation studies of our method.


\subsection{Experiment Settings}
\label{sec:exp:settings}
To ensure the efficacy of our methods and design components, we conducted a series of extensive experiments on image classification using ImageNet-1K\cite{ImageNet-1K}, object detection on COCO~\cite{coco} and semantic segmentation on ADE20K\cite{ADE20K}. Unless otherwise stated, our default settings are based on ViT-B.


\paragraph{ImageNet-1K~\cite{ImageNet-1K}} The ImageNet-1K dataset comprises 1.3 million images belonging to 1,000 categories and is divided into training and validation sets. To ensure the fairness of our experiments while applying our methods to MAE~\cite{MAE} and PixMIM\cite{pixmim}, we strictly follow their original pre-training and evaluation settings on ImageNet-1K. This includes following the pre-training schedule, network architecture, learning rate setup, and fine-tuning protocols. Furthermore, in addition to the conventional fine-tuning protocol, we fine-tune the model using a low-shot setting, where only a fraction of the training set (\eg 1\% and 10\%) is used. This approach is consistent with previous works\cite{simclr} and we ensure that the low-shot fine-tuning setting also strictly follows that of the conventional fine-tuning.

\paragraph{ADE20K~\cite{ADE20K}} To conduct the semantic segmentation experiments on ADE20K, we utilize the off-the-shelf settings from MAE\cite{MAE}. With this approach, we fine-tune a UperNet\cite{upernet} for 160k iterations with a batch size of 16 and initialize the relative position bias to zero. 

\paragraph{COCO~\cite{coco}} For our object detection experiments on COCO, we adopt the Mask R-CNN approach\cite{maskrcnn} that enables simultaneous production of bounding boxes and instance masks, with the ViT serving as the backbone. As in MAE, we evaluate the box and mask AP as the metrics for this task. However, we note that there is no universal agreement for the setting of object detection fine-tuning epochs. We have chosen the commonly used 2$\times$ setting, which fine-tunes the model for 25 epochs. Other settings strictly follow those in ViTDet~\cite{ViTDet}.

\paragraph{Ablation studies} We conduct all of our ablation studies based on the customary MAE settings~\cite{MixMIM, CAE}. We pre-train all model variants on ImageNet-1K for 300 epochs and conduct a comprehensive performance comparison on linear probing, fine-tuning, and semantic segmentation. All other settings are consistent with those discussed previously.
\subsection{Main Results}
\label{sec:main_results}
\begin{table*}[!ht]
\centering
\tabcolsep 7pt
\begin{tabular}{lcclllll}
 \multicolumn{3}{c}{Evaluation Protocol$\rightarrow$} & \multicolumn{2}{c}{ImageNet} & \multicolumn{2}{c}{Low-shot} & ADE20K \\
\cmidrule(lr){4-5}\cmidrule(lr){6-7}\cmidrule(lr){8-8}  
Method&Target&Epoch& ft(\%) & lin(\%) & 1\% & 10\% & mIOU \\
\midrule
\multicolumn{8}{@{\;}l}{\bf Supervised learning} \\
\quad DeiT III\cite{DEiT-v3} &- & 800 & 83.8 &- & - & - & 49.3 \\
\midrule
\multicolumn{7}{@{\;}l}{\bf Masked Image Modeling w/ pre-trained target generator} \\
\quad BEiT\cite{BEiT} &DALLE&800&83.2 & 56.7 & - & - & 45.6 \\
\quad CAE\cite{CAE} & DALLE&800&83.8&68.6& - & - & 49.7 \\
\quad MILAN$^{*}$\cite{MILAN} & CLIP-B&400&85.4&78.9&67.5 & 79.7 & 52.7\\
\quad BEiT-v2\cite{BEiTv2}& VQ-KD&1600&85.5& 80.1 &-&-&53.1\\
\quad MaskDistill\cite{MASKdistill} &CLIP-B&800&85.5&-&-&-&54.3\\
\midrule
\multicolumn{7}{@{\;}l}{\bf Masked Image Modeling w/o pre-trained target generator} \\
\quad MaskFeat$^{*}$\cite{MaskFeat}&HOG&1600&84.0&62.3&52.9&73.5 & 48.3\\
\quad SemMAE\cite{SemMAE} & RGB &800&83.4&65.0&- & - & 46.3\\
\quad SimMIM\cite{SimMIM}& RGB &800&83.8&56.7&-&- & -\\
\hdashline
\quad MAE$^{*}$\cite{MAE} & RGB & 300 & 82.8 & 61.5 & 41.4 & 70.5 & 43.9 \\
\quad \textbf{MFF}$_\text{\tt MAE}$ &RGB&300&{{83.3} \more{(+0.5)}}&{63.3} \more{(+1.8)} & 43.7 \more{(+2.3)} & 71.4 \more{(+0.9)} &{47.7} \more{(+3.6)}\\
\quad MAE$^{*}$\cite{MAE} & RGB & 800 & 83.3 & 65.6 & 45.4 & 71.2 & 46.1 \\
\quad \textbf{MFF}$_\text{\tt MAE}$ &RGB&800& 83.6 \more{(+0.3)} & 67.0 \more{(+1.4)} & 48.0 \more{(+2.6)}& 72.0 \more{(+0.8)}  & 47.9 \more{(+1.8)}\\
% \quad MAE$^{*}$\cite{MAE} & RGB & 1600 & 83.5 & 67.8 & 47.8 & 72.4 & 48.1 \\
% \quad \textbf{MFF}$_\text{\tt MAE}$ &RGB&1600& 83.7 \more{(+0.2)} & 69.6 \more{(+1.8)} & 51.9 \more{(+3.1)} & 73.4 \more{(+1.0)}  & 48.3 \more{(+0.2)}\\
\hdashline

\quad PixMIM\cite{pixmim} & RGB&800 & 83.5 & 67.2 & 47.9 & 72.2 & 47.3\\
\quad \textbf{MFF}$_\text{\tt PixMIM}$ &RGB&800&83.6 \more{(+0.1)}&68.2 \more{(+1.0)} & 49.0 \more{(+1.1)} & 73.0 \more{(+0.8)}  &48.6 \more{(+1.3)}\\
% \quad PixMIM\cite{pixmim} & RGB & 1600 & 83.6 & 69.3 & 50.9 & 72.9 & 48.7 \\
% \quad \textbf{MFF}$_\text{\tt PixMIM}$ &RGB&1600&83.9 \more{(+0.3)}&71.1 \more{(+1.8)}& 53.9 \more{(+3.0)} & 74.1 \more{(+1.2)}  &49.1 \more{(+0.4)}\\

\end{tabular}
\caption{\textbf{Performance comparison of MIM methods on various downstream tasks.} We report the results with fine-tuning (ft) and linear probing (lin) experiments on ImageNet-1K, objection detection on COCO, and semantic segmentation on ADE20K. The backbone of all experiments is ViT-B\cite{ViT}. $*$: numbers are reported by running the official code release. Low-shot: end-to-end fine-tuning with 1\% and 10\% of the training set.}
\label{tab:comparison}
\end{table*}

The application of multi-level feature fusion to MAE \cite{MAE} and PixMIM \cite{pixmim} resulted in significant improvements in various downstream tasks, as shown in \autoref{tab:comparison}. After pre-training the model for 300 epochs, we achieve a 0.5\%, 1.8\%, and 3.6\% improvement over MAE in fine-tuning, linear probing, and semantic segmentation, respectively. Additionally, our model exhibits scalability across pre-training epochs and consistently outperforms these base methods by a substantial margin. Compared to these methods, using an extra heavy tokenizer, \eg CLIP, we also gradually close the performance gap with them. Although fine-tuning accuracy is often considered a reliable measure of the quality of non-linear features in a model, we find that it is not a sensitive metric, as compared to other metrics presented in \autoref{tab:comparison}. This may be attributed to pre-training and fine-tuning following the same data distribution, and the size of the training set and model capacity being sufficient to offset the performance gap between different methods. To address this limitation, we adopt the following two workarounds:
% \begin{figure*}[htbp]
% \centering
% \includegraphics[width=\linewidth]{pdf/pic-vit-s.pdf}
% \caption{\textbf{Performance on ViT-S.} Applying \ourmethod to ViT-S brings significant improvements on all downstream tasks. We reuse the same setting as for ViT-B, without specifically tuning.}
% \label{fig:vit-s}
% \end{figure*}
\begin{figure}[htbp]
\centering
\includegraphics[width=\linewidth]{pdf/pic-vit-s-single_column.pdf}
\caption{\textbf{Performance on ViT-S.} Applying \ourmethod to ViT-S brings significant improvements on all downstream tasks. We reuse the same setting as for ViT-B, without specifically tuning.}
\label{fig:vit-s}
\end{figure}
\paragraph{Low-shot fine-tuning.} This protocol has also been adopted by many previous works, \eg \cite{simclr}. Rather than utilizing the entire training set, we fine-tune the pre-trained model end-to-end using only 1\% and 10\% of the training set. As indicated by \autoref{tab:comparison}, the performance gap between MFF and the base methods is much more prominent when using low-shot fine-tuning, which further verifies the effectiveness of MFF.

\paragraph{Pre-train with ViT-S.} To mitigate the influence brought by model capacity, we pre-train MAE using ViT-S and compare their performance using fine-tuning, linear probing, and semantic segmentation. Since our objective is to evaluate the improvement brought by MFF to these base methods, we do not specifically tune hyper-parameters for the experiments with ViT-S to achieve state-of-the-art performance, but rather use the same settings as for ViT-B. Due to its smaller capacity compared to ViT-B, ViT-S requires a pre-training method that can effectively capture semantic features to perform well on downstream tasks. As demonstrated in \autoref{fig:vit-s}, the method with MFF significantly outperforms their base method, further validating the effectiveness of MFF.

We also evaluate our pre-trained models with the object detection protocol and report the AP$^\text{box}$ and AP$^\text{mask}$. As shown in \autoref{tab:detection}, \ourmethod can still bring non-trivial improvements for object detection.
\begin{table}[ht]
\centering
\tabcolsep 1.5pt
\begin{tabular}{lll}
Method&AP$^\text{box}$&AP$^\text{mask}$\\
\shline
MAE&47.3&42.4\\
\ourmethod$_\text{MAE}$ & 48.0 \more{(+0.7)}& 43.0 \more{(+0.6)} \\
PixMIM & 47.8 & 42.8 \\
\ourmethod$_\text{PixMIM}$ & 48.1 \more{(+0.3)}& 43.1 \more{(+0.3)} \\
\end{tabular}
\vspace{-0.5em}
\caption{\textbf{Results of COCO object detection.}}
\label{tab:detection}
\end{table}





\subsection{Robustness Evaluation}
\label{sec:robust_eval}
Robustness evaluation is a common practice in many previous works \cite{zhou2021ibot,MILAN,MAE} to assess a model's ability to handle different types of noise. In this study, we compare our pre-trained models with their corresponding baselines on four out-of-distribution (OOD) ImageNet variants: ImageNet-Corruption \cite{ImageNetC}, ImageNet-Adversarial \cite{ImageNetA}, ImageNet-Rendition \cite{ImageNetR}, and ImageNet-Sketch \cite{ImageNetS}. These datasets introduce various domain shifts to the original ImageNet-1K and are widely used to evaluate a model's robustness and generalization ability. As illustrated in \autoref{tab:ood}, \ourmethod significantly improves the robustness of MAE and PixMIM on all datasets by a clear margin. The enhanced robustness against domain shifts strengthens the value of our simple yet effective method.

\begin{table}
  \centering 
  \setlength{\tabcolsep}{5mm}
  \caption{The ablation of different modules in Omni-Relational Network reported with MSE on MNIST given 10$\%$ context points by ablating (1) graph structure (2) attentive pooling (3) positional embedding (P.E.) (4) information bottleneck constrain (I.B.).}
  \begin{tabular}{lcccc}
    \toprule
      Graph       & Attention       & P.E.            & I.B.             & MSE \\
    \midrule
                  &                 &                &                   & $0.073$\\
    $\checkmark$   &              &                 &                    & $0.066$\\
    $\checkmark$  &               &               &    $\checkmark$    & $0.058$\\
    $\checkmark$  &                 &   $\checkmark$     &  $\checkmark$   & $0.053$\\
    $\checkmark$  &  $\checkmark$  &               &     $\checkmark$   & $0.049$\\
    $\checkmark$  &  $\checkmark$  &   $\checkmark$    &  $\checkmark$   & $0.045$\\
    \bottomrule
  \end{tabular}
  \label{tab:3-ablation}
\end{table}


\subsection{Ablation Studies}
\label{sec:ablation}
\paragraph{Is shallow layer important?} In order to determine the significance of low-level features from shallow layers, we explore the fusion of the output layer with either a deep or shallow layer. So we try to fuse the output layer with an extra shallow or deep layer, selected from the previous 11 layers of ViT-B\cite{ViT}. And the specific index of the selected layer will be detailed in the appendix. As illustrated in \autoref{tab:shallow_layer}, fusing the output layer with a deep layer only results in marginal improvements. However, incorporating low-level features directly from the shallow layer into the output layer leads to a significant performance boost, as it enables the model to focus on semantic information. Therefore, we have decided to use a shallow layer (\ie the first layer) for multi-level feature fusion.

\paragraph{How many layers are used for fusion ?} Aside from the output layer and the shallow layer picked in the previous selection, it is reasonable to consider using intermediate layers for fusion, as they may contain additional low-level features or high-level meanings that could assist in the reconstruction task. However, selecting these intermediate layers is a daunting task due to the large search space involved. To simplify the process, we pick an additional 1, 2, and 5 layers evenly spaced between the shallow layer and output layer selected in \autoref{tab:shallow_layer}. The specific indices for these selected layers are placed in the appendix. As shown in \autoref{tab:layers}, introducing more layers brings consistent improvements because they may contain unique features, such as textures or colors, that help the model complete the reconstruction task. Nevertheless, when we fuse all these layers, we witness a performance drop in all downstream tasks. This drop may result from the difficulty of optimization, because of the redundancy in these layers.

\begin{table}[h]
\centering
% \scalebox{1}{
\tabcolsep 1.5pt
\begin{tabular}{lllll}
% \toprule 
Method&IN-C$\downarrow$ &IN-A&IN-R&IN-S\\
% \midrule
\shline
MAE&51.7&35.9&48.3&34.5\\
MFF$_\text{\tt MAE}$&49.0 \more{(-2.7)}&37.2 \more{(+1.3)}&51.0 \more{(+2.7)}&36.8 \more{(+2.3)}\\
\hdashline
PixMIM&49.9 &37.1 &49.6 &35.9 \\
MFF$_\text{PixMIM}$&48.5 \more{(-1.4)} &40.1 \more{(+3.0)} &51.6 \more{(+2.0)} &37.8 \more{(+1.9)} \\
\end{tabular}
\vspace{-0.5em}
\caption{\textbf{Robustness evaluation on ImageNet variants.} To evaluate the robustness of MFF, we further evaluate the models (after fine-tuning) from \autoref{tab:comparison} on four ImageNet variants. Results are reported in top-1 accuracy, except for IN-C\cite{ImageNetC} that uses the mean corruption error.}
\label{tab:ood}
\end{table}


\paragraph{Do the projection layer and fusion strategy matter?} In \autoref{eq:proj}, we investigate the influence of the projection layer on the final results. Our findings indicate that a simple linear projection layer is sufficient to achieve satisfactory results, as compared to using no projection layer or a nonlinear projection layer. Incorporating a single linear projection layer offers benefits in mitigating the domain or distribution gap between different layers, as compared to using no projection layer. However, the addition of a nonlinear projection layer, which includes an extra linear projection and GELU activation before the linear projection, introduces computational overhead and is more challenging to optimize. As a result, the non-linear projection achieves sub-optimal performance. With regard to the fusion strategy, we found that the \textbf{weight-average pooling} strategy, which assigns a dynamic weight to each layer and then performs element-wise addition, achieves the best performance. Compared to \textbf{attn}, this strategy shares the merits of simplicity and smaller computational overhead.
