%\documentclass[ conference,twocolumn,twoside]{IEEEtran} % formal style
\documentclass[10pt,journal,twocolumn,twoside]{IEEEtran} % formal style
%\documentclass[12pt,draftclsnofoot,onecolumn,journal]{IEEEtran} %draft style
%\renewcommand{\baselinestretch}{1.8}
%\documentclass[journal,comsoc]{IEEEtran}
\usepackage{xpatch}
\usepackage{graphicx}
\usepackage{epstopdf}
\usepackage{float}
\usepackage{algorithmic}
\usepackage{array}
\usepackage{amsmath}
\usepackage{amssymb}
\usepackage{mdwmath}
\usepackage{mdwtab}
\usepackage{eqparbox}
\usepackage{stfloats}
\usepackage{fixltx2e}
%\usepackage{tabularx}
%\usepackage{hyperref}
\usepackage{cases} 
%\usepackage{booktabs}
\usepackage{flushend}
\usepackage{threeparttable}
\usepackage{xcolor}
\usepackage{makecell}
\usepackage{etoolbox}
\makeatletter
\patchcmd{\@makecaption}
  {\scshape}
  {}
  {}
  {}
\makeatletter
\patchcmd{\@makecaption}
  {\\}
  {.\ }
  {}
  {}
\makeatother
\def\figurename{FIGURE}


\usepackage[boxed,ruled,commentsnumbered]{algorithm2e}
\usepackage{url}
\usepackage{cite}
%==============================================%
\ifCLASSOPTIONcompsoc
\usepackage[caption=false,font=normalsize,labelfont=sf,textfont=sf]{subfig}
\else
\usepackage[caption=false,font=footnotesize]{subfig}
\fi
%==============================================%
\allowdisplaybreaks[4]
\newtheorem{condition}{Condition}
\newtheorem{assumption}{Assumption}
\newtheorem{corollary}{\bf Corollary}
\newtheorem{theorem}{\bf Theorem}
\newtheorem{proposition}{Proposition}
\newtheorem{lemma}{ Lemma}
\newtheorem{example}{Example}
\newtheorem{notation}{Notation}
\newtheorem{definition}{\bf Definition}
\newtheorem{remark}{Remark}
\newtheorem{property}{\bf Property}
\newcommand{\figref}[1]{Fig.~\ref{#1}}
\newcommand{\theoremref}[1]{Theorem~\ref{#1}}
\newcommand{\tabref}[1]{Table~\ref{#1}}
\newcommand{\lemmaref}[1]{Lemma~\ref{#1}}
\newcommand{\propref}[1]{Proposition~\ref{#1}}
\newcommand{\corref}[1]{Corollary~\ref{#1}}
\newcommand{\appref}[1]{Appendix~\ref{#1}}
\newcommand{\propertyref}[1]{Property~\ref{#1}}
\newcommand{\sectionref}[1]{Section~\ref{#1}}
\newcommand{\tableref}[1]{Table~\ref{#1}}

% Define my mathematic operators
\newcommand{\diag}{\mathop{\mathrm{diag}}}
\newcommand{\tr}{\mathop{\mathrm{tr}}}
\newcommand{\Ei}{\mathop{\mathrm{E}_1}} % Exponential integral

\newcommand{\colorr}{\color{red}}
\makeatletter
\newcommand{\rmnum}[1]{\romannumeral #1}
\newcommand{\Rmnum}[1]{\expandafter\@slowromancap\romannumeral #1@}
\makeatother

\renewcommand{\vec}{\mathop{\mathrm{vec}}}

\newcounter{MYtempeqncnt} % Used for double column equation

\makeatletter
\renewcommand*{\@opargbegintheorem}[3]{\trivlist
      \item[\hskip \labelsep{\bfseries #1\ #2}] \textbf{(#3):}\ }
\makeatother     




%%%%%%%%%%%%%%%%%%%%%%%%%%%%%%%%%%%%%%%%%%%%%%%%%%%%%%%%%%%%
\begin{document}

\makeatletter
\def\changeBibColor#1{%
  \in@{#1}{}%  list of colored bib items
  \ifin@\color{red}\else\normalcolor\fi
}
 
\xpatchcmd\@bibitem
  {\item}
  {\changeBibColor{#1}\item}
  {}{\fail}
 
\xpatchcmd\@lbibitem
  {\item}
  {\changeBibColor{#2}\item}
  {}{\fail}
\makeatother

\title
{Privacy and Security in Ubiquitous Integrated Sensing and Communication: Threats, Challenges and Future Directions}
\author{Kaiqian Qu,  Jia Ye,~\IEEEmembership{Member, IEEE,} Xuran Li, 
and Shuaishuai Guo,~\IEEEmembership{Senior Member, IEEE}

\thanks{K. Qu and S. Guo  are with the School of Control Science and Engineering, and also with Shandong Key Laboratory of Wireless Communication Technologies, Shandong University, China (e-mail: qukaiqian@mail.sdu.edu.cn; shuaishuai$\_$guo@sdu.edu.cn). }
\thanks{Jia Ye is with the School of Electrical Engineering, Chongqing University, Chongqing, 400044, China (yejiaft@163.com).}

\thanks{Xuran Li is with Shandong Key Laboratory of Medical Physics and Image Processing, School of Physics and Electronics, Shandong Normal University, China (e-mail: sdnulxr@sdnu.edu.cn).}
   }
\maketitle

%\newpage

\begin{abstract} 
Integrated sensing and communication (ISAC) technology is one of the featuring technologies of the next-generation communication systems. When sensing capability becomes ubiquitous, more information can be collected, which can facilitate many applications in intelligent transportation, unmanned aerial vehicle (UAV) surveillance and healthcare. However, it also faces many information privacy leakage and security issues. This article highlights the potential threats to privacy and security and the technical challenges to realizing private and secure ISAC. Three promising combating solutions including artificial intelligence (AI)-enabled schemes, friendly jamming and reconfigurable intelligent surface (RIS)-assisted design are provided to maintain user privacy and ensure information security. Case studies demonstrate their effectiveness.

\end{abstract}

 \begin{IEEEkeywords}
Integrated sensing and communications, privacy, security 
 \end{IEEEkeywords}


\section{Introduction} 
\IEEEPARstart{W}{ith} the advancement of the sixth generation (6G) wireless communication research, many potential scenarios that cannot be fully realized by the fifth generation (5G) wireless communication are proposed, such as digital twins, autonomous driving, etc. These emerging applications require the collection and transmission of massive data. In order to avoid the expensive cost of a large number of sensors, it is an innovative method to use electromagnetic waves for communication transmission while performing sensing functions to collect data. Thus, a paradigm shift  from traditional communications-based networks to integrated sensing and communication (ISAC) networks has become the key to 6G enabling emerging applications\cite{9737357}.

Sensing and communication are major consumers of wireless spectrum that are facing resource scarcity. It improves both the spectral and energy efficiency to share the same spectrum and power between communication and radar. In ISAC, the waveform conducts the communication tasks in the forward channel and sensing tasks in the backward channels. Figuring out the fundamental performance trade-off is essential. Recently, \cite{10147248} has explored the communication perception performance boundary under the Gaussian channel, Besides, integrating communication and sensing functions in the waveform level also attracts a lot of attention.  Recently, a beamforming strategy of vehicular-mounted ISAC units based on vehicle state was studied in \cite{10063187}; Cong \emph{et al} proposed a general ISAC beamforming design for systems with the finite alphabet modulation set as input \cite{CongFinite2022}. Guo \emph{et al} proposed a beamspace modulation strategy to improve the communication capacity while maintaining similar sensing performance \cite{ShuaishuaiGuo:Mobicom2022}. 

The prerequisite to deploying ISAC base stations is that their security must be guaranteed. In traditional communications, due to the broadcast characteristics of wireless channels, the transmitted signal is exposed to vulnerable environments and is easily wiretapped by the malicious eavesdropper \cite{zhang2023physical}. As a
complement to cryptography, physical layer security (PLS) is
proposed to safeguard private information from eavesdropping. PLS is capable of exploiting the physical characteristics of wireless channels, e.g., interference, fading, noise, directivity, and disparity, without introducing complicated secret key generation and management. ISAC, as the new paradigm for future wireless networks, deserve the same security concerns. 
However, only a small portion of the literature has examined issues related to ISAC security.  \cite{9199556,9737364,10143420,Xu2023ASD} considered the scenario in which the eavesdropping target and the communication user are present at the same time. This reveals the problem of communication security in ISAC i.e. how to avoid communication data leakage. \cite{9199556} separately designed sensing signal, and optimized the singnal-to-interference-and-noise ratio (SINR) of the eavesdropping target and the user target by jointly designing the communication beamforming, radar beamforming and reconfigurable intelligent surface (RIS) configuration. \cite{9737364} and \cite{9199556} used artificial noise and destructive interference to reduce the SINR of received symbols at eavesdropping targets, respectively.


However,  only the risk of data leakage caused by potential eavesdroppers was considered in  most literature, and the problem of user privacy leakage and sensing data leakage was left unconsidered. In ISAC networks, ubiquitous electromagnetic waves can sense a lot of data. For instance, \cite{8897594} uses WiFi for human behavior sensing or uses user equipment signals for breath detection, and the related technologies have been initially applied on the ground. In addition, coupled with the assistance of unmanned aerial vehicles (UAVs), the ISAC networks will further expand their coverage. There will be more scenarios that can be sensed in a perfect ISAC network. Moreover, the sensing based on electromagnetic (EM) waves breaks through the limitation of the traditional camera in the line of sight (LoS) and light brightness, such as radar imaging. Consequently, it can sense a lot of private information,  In this way, the user's behavior may be visible to the electromagnetic waves. Few people want 
all of their behaviors to be monitored. For example, in the application of WiFi senior care, falling or tripping of a senior is required to be monitored, while we may not leak his private activities, such as bathing, brushing, etc. This may pose an additional challenge to the privacy preservation of human behaviors. Therefore, it is necessary to anticipate the potential privacy and security issues of ISAC.

In this article, we discuss the privacy and security threats of ISAC and explore key technical challenges and solutions.
\section{Privacy and Security Threats}
\begin{figure*}[htbp]
       \centering
        \includegraphics[width=0.85\linewidth]{PAS_ISAC_fig.eps}
       \caption{Potential ISAC privacy and security scenarios.}
       \label{Model}
\end{figure*}
In this section, we discuss the privacy and security threats of ISAC in detail and briefly review the existing research.

% \begin{table*}[htbp]
%        \centering
%        \includegraphics[width=1\linewidth]{review.pdf}
%        \caption{Summarize models, optimization variables, measurement metrics, and solution techniques for privacy security research in communication and sensing systems. The following abbreviations are used for brevity: single user (SU), multi-user (MU), single target (ST), multi-target (MT), eavesdropper (Eve), communication user (CU), directional modification (DM), constructive interference (CI), time-modulated arrays (TMA).}
%        \label{table}
% \end{table*}
\subsection{User Privacy Threats posed by Ubiquitous Electromagnetic Sensing}
Now a large number of base stations use electromagnetic waves to form a huge wireless network, and using these networks to achieve sensing functions is promising to facilitate many applications. However, such ISAC techniques can cause privacy concerns due to the ubiquity of ISAC signals.
The sensing function can be divided into three levels, which are detection, estimation, and recognition. Detection is to explore the acquisition of information about the presence or absence of a target through sensing.
Estimation is for obtaining the position, speed, and other parameters of the target through sensing under the premise of a known target. Recognition can be understood as obtaining more detailed information about a target's identity, characteristics, status, etc. Obviously, when the sensing target is a person, these three levels of perception monitor every move of the person like a camera. And, more terrifying than the camera, electromagnetic waves are everywhere, even in the dark or with a wall block, it can still play the role of sensing. For example, \cite{zhongwen} has achieved confined space environment reconstruction and target imaging using a terahertz band ISAC prototype. It is conceivable that a person staying in a room can not even get rid of being monitored by electromagnetic waves, which will be very uncomfortable. User privacy is completely unprotected. 
\subsection{Privacy Threats from Sensing Data Breaches}
Being constantly monitored by ISAC networks is a potential privacy threat, and the loss of this monitoring data is another direct privacy incident. This privacy threat can occur in the following two situations. The first one is that there is a malicious device intercepting the ISAC echo signal, and at the same time the malicious device has powerful computing ability and may decrypt the sensory information from the ISAC echo.
\subsection{Security Threats posed by Electromagnetic Leakage and Eavesdropping Devices}
Due to the broadcast characteristics of wireless channels, the
transmitted signal is exposed to vulnerable environments and
is easily wiretapped by eavesdropping devices. This is highly susceptible to electromagnetic leakage. And the threat of data security caused by electromagnetic leakage is also a problem that has been faced in traditional communications. This threat still exists in ISAC. Many researchers also design a secure ISAC system from this perspective. In addition, communication users may access pseudo base stations during uplink communication, resulting in electromagnetic leakage, which is also one of the security threats.

\subsection{Security Threats posed by Eavesdropping Targets}

In ISAC, the base station transmits an integrated signal to transmit useful information to the communication users while detecting the target. It is known that typical radar requires focusing the transmit power toward the directions of interest to obtain a good estimation of the targets. Nevertheless, when the target is a malicious eavesdropper, critical communication information in the integrated signal may be stolen. This is a security threat posed by eavesdropping targets which is different from electromagnetic leakage in traditional communications. The base station is intentionally pointing the signal at the target in order to obtain a better-estimated performance. Therefore, it is important to avoid the leakage of communication information at malicious targets without compromising the detection of malicious targets. \cite{10143420,9199556,9737364,9747551}






\subsection{Security Threats from Sensing Errors}
Another security threat comes from the application layer of ISAC. Perhaps 6G devices will achieve the function of communication and sensing integration, The accuracy of sensing will affect the safety of the device and even the system to which the device belongs. For example, sensing errors in ISAC-enabled vehicle networks will lead to accidents.  The sensing accuracy of ISAC brings different threat levels in different application scenarios. Therefore, this aspect of security threats requires designing ISAC with different accuracy according to the quality-of-service (QoS) requirements.

\subsection{Security and Privacy Issues in the ISAC Data Center}


Misuse of data may also cause a lot of concerns.  If the data collected through ISAC falls into the wrong hands, it may be misused. It could result in privacy violations and potential harm. There is the potential for radar sensing to be used for surveillance purposes. Either by governments or private organizations. This could infringe on people's privacy and civil liberties, especially if the surveillance is carried out without a legal basis or oversight.

\section{Technical Challenges to Implement Privacy-Preserving and Secure ISAC Networks}
The use of ISAC can indeed raise significant privacy and security concerns, it is of critical importance to developing privacy-preserving and secure ISAC networks. It is well known that there exist many privacy-preserving and secure techniques, such as authentication, physical layer security, encryption, covet transmission, and secure data aggregation protocols. Considering that ISAC shares the characteristics of sensing and communication, the question is whether the conventional privacy-preserving and secure techniques applied for radar systems and wireless communication systems can also be applied to ISAC. Moreover, most techniques are designed for secure sensing and communication processes, while little user-centric research has been done to secure individuals' private life. In this section, we unveil the technique challenges from privacy-preserving and secure perspectives, respectively. 

\subsection{Technical Challenges to Users' Privacy-Preserving}
The privacy-preserving challenges in ISAC generally stem from the unknown and uncontrollable wireless environment and the difficulty of finding a trusted organization to handle sensitive data. We delve deeper into these challenges as follows.
\subsubsection{Lack of User Awareness}
Unlike traditional cameras or physical sensors, which have an observable presence and are limited to certain areas, This lack of visibility makes users not realize when they are being monitored or how their data is being collected in ISAC networks. On the other hand, ISAC networks often employ passive sensing techniques without any direct interaction with users, where devices continuously listen to the wireless environment for changes. For example, WiFi sensing technologies can monitor movements and activities without requiring individuals to actively engage with the system. This passive data collection further contributes to the lack of user awareness as users might not be consciously aware of the data being generated. While passive sensing can reduce energy consumption and improve efficiency, it also means that users have little control over when and how their data is being collected. In addition, ISAC systems are deployed in public spaces or commercial environments where users may not be explicitly informed about their presence. 
\subsubsection{Unknown and Uncontrollable Wireless Environment}
 ISAC networks rely on electromagnetic waves, which can penetrate walls and operate without the user's knowledge or consent. The inherently uncontrollable and unpredictable wireless environment makes it challenging for users to figure out effective countermeasures. In particular, the propagation channel information, and the exact location of sensing devices are hard to access, which brings difficulties to ascertain the boundaries of sensing coverage. Users might inadvertently enter monitored areas without being aware of it, further exacerbating privacy concerns. Trying to block or interfere with the sensing propagation links might be an option, but on the premise of the associated channel state information and at the cost of additional resource consumption and solving more complex system design problems. The uncontrollable wireless environment may lead to the aggregation of diverse data from multiple sources, increasing the potential for inference attacks. Even seemingly benign pieces of data can be combined to infer sensitive information about individuals, posing privacy risks that users may not anticipate.
\subsubsection{Lack of Trust Management Organization}
Even if users are aware of the unexpected detection, estimation, or recognition processes employed by ISAC, the absence of a trust management organization leaves them without the means to deactivate sensing devices or hide their actions and information. A trusted organization plays a pivotal role in safeguarding users' privacy and assumes the following responsibilities. Firstly, the organization should implement encryption, access controls, and oversee data governance and ownership, along with other security measures, to protect sensitive data from unauthorized access or breaches. The absence of such an organization could lead to uncertainties regarding data control, storage, and potential usage. Moreover, ISAC networks may need to share data with other entities or engage in collaborations. A trusted organization acting as an intermediary ensures that data sharing is conducted in a privacy-preserving manner and adheres to relevant regulations. Furthermore, the trusted organization is responsible for obtaining user consent for data collection and ensuring transparency regarding the purpose and scope of data usage. Users deserve the assurance that their data will be handled responsibly and in accordance with their preferences.

\subsection{Technical Challenges to Secure ISAC Networks}
In this part, we will explore the key technical challenges to secure ISAC networks from four distinct perspectives: information-carrying sensing signals, resource constraints, real-time processing requirements, and heterogeneity. Let us delve into each perspective to uncover the intricacies and solutions required to build resilient ISAC networks.

\subsubsection{Information-Carrying Sensing Signals}
The integration of sensing and communication functionalities delivers the information bits not only to the legitimate receivers, but also to the detection objects, tracking targets, neighboring devices, and other relevant sensing applications. The unauthorized devices located in the nearby sensing environments thus can also receive the signals carrying privacy information. While cannot prevent unauthorized access, upper-layer encryption is considered to be an effective manner to increase the decoding difficulties of malicious devices. However, as the upper-layer encryption involves encryption algorithm design, secure key generation and exchange, the more secure the encryption algorithm is, the more the consumption of channel resources. In addition, we cannot ignore the situation that malicious devices have the powerful computational capability to overhear the information carried by sensing signals. In addition, in sensing-centric ISAC systems, the primary objective is to gather accurate and reliable information from the environment or specific targets of interest rather than carrying complex modulation schemes for communication purposes. It can be challenging to apply encryption directly to sensing waveforms without compromising their essential properties or degrading the radar's detection capabilities.

\subsubsection{Resource Constraints}
Many ISAC networks operate in resource-constrained environments, such as wireless sensor networks or Internet of Things (IoT) devices. These devices typically have limited computational power, memory, and energy resources. As ISAC serves dual purposes of sensing the environment and enabling communication to achieve a Pareto optimality system, implementing privacy-preserving and secure techniques within these constraints is much more challenging. For instance, beamforming design is an effective technique to enable the focused transmission of signals in desired directions, suppress interference from undesired directions or malicious transmitters, and contribute to physical layer security techniques by exploiting the channel characteristics and employing secure beam training mechanisms. However, designing beamforming techniques that effectively maximize sensing accuracy while balancing the requirements of both sensing and communication tasks complicates the dual-purpose optimization problem to a three-purpose one, which might not be afforded by the limited resource. Lightweight beamforming design algorithms, efficient privacy-preserving and secure protocols, and optimized resource usage are necessary to minimize the computational and energy overhead. 

\subsubsection{Real-Time Processing Requirements}
The need for timely processing, communication, and response to time-sensitive events is growing to support continually evolving applications. The ISAC networks thus must meet strict time requirements to provide actionable information and enable rapid decision-making, which is commonly used in applications such as surveillance systems, intelligent transportation systems, disaster management, military operations, and environmental monitoring. However, privacy-preserving and secure techniques often involve complex cryptographic algorithms or data processing operations, which can introduce significant computational overhead. For example, encryption and decryption operations, secure key establishment, or authentication processes all introduce additional computational burden, which results in a slower response time, outdated collaboration and decision commands, degraded sensing accuracy, missed detections, or unreliable measurements, thereby affecting the real-time nature and performance of the system. In order to meet the real-time requirements of the ISAC system, the computational burden of security mechanisms may need to be reduced or compromised. This trade-off between real-time performance and security can impact the overall level of data protection, leaving the system more vulnerable to security threats.

% \subsection{Dynamic Network} Integrated sensing and communication systems often operate in dynamic network topologies where users or targets may join or leave the network frequently. The dynamic nature of the network introduces challenges in maintaining consistent secure transmission across changing network configurations by adopting techniques designed for conventional networks. In fact, ensuring the authentication and trustworthiness of entities, establishing secure connections, managing cryptographic keys, and maintaining trust relationships become more challenging in such environments. In addition, the resource- and time-consuming privacy-preserving and secure techniques can be out-fashion, thereby cannot fit the varying ISAC system perfectly, and even cause severe information leakage problems. Last but not least, the aggregation and fusion of data from multiple sources in the dynamic ISAC networks while protecting against malicious data injection or tampering becomes much more complicated and uncontrollable. Therefore, robust privacy and security mechanisms should be develop to handle dynamic changes in network participants and maintain the integrity and confidentiality of information.

% \subsubsection{Heterogeneity}
% Heterogeneity in the context of ISAC networks refers to the presence of diverse technologies, devices, protocols, and data formats within the network ecosystem. This heterogeneity brings about several technical challenges when it comes to implementing Privacy-Preserving and Secure ISAC Networks. Specifically, different devices and systems may use different, communication protocols, data formats, authentication methods, and encryption algorithms. Coordinating and managing these diverse security mechanisms that are compatible with the diverse components across the entire network can be challenging. Collaborative sensing, data fusion, and information sharing among multiple nodes or entities are also common in ISAC networks. For example, ISAC systems designed for healthcare often involve the exchange and sharing of healthcare data between different entities, such as healthcare providers, hospitals, laboratories, and research institutions. Integrating and managing authentication across these diverse devices pose challenges in terms of interoperability, standardization, and compatibility. Also, heterogeneous components may have different levels of security vulnerabilities and weaknesses. Some devices or systems may have outdated firmware or lack robust security features. Integrating these vulnerable components within the ISAC network introduces potential points of weakness and security risks. 

In conclusion, the implementation of privacy-preserving and secure ISAC networks comes with a set of unique challenges. Understanding and tackling these challenges is essential to develop robust and secure ISAC networks capable of withstanding the complexities of modern environments while safeguarding sensitive data and delivering seamless, real-time performance. In the following, we will present some potential solutions with specific case studies for ensuring the privacy and security of ISAC networks.

\section{Promising Solutions and Future Research Directions} 
In response to these outlined challenges, we delve into innovative solutions to safeguarding sensitive data while ensuring seamless and efficient communication. The cutting-edge technologies driven by artificial intelligence (AI), friend jammers, and RIS emerge as promising avenues to fortify ISAC networks against potential threats and privacy breaches.
\subsection{AI-enabled Scheme}
By leveraging advanced algorithms, machine learning, and deep neural networks, AI can empower wireless communication networks to make intelligent, real-time decisions, detect anomalies, and optimize resource allocation. Here, we explore the exciting role of AI in ISAC networks, focusing on its applications in realizing privacy-preserving and secure communications. 

\begin{itemize}
    \item {\bf Anomaly Detection and Threat Prediction} Anomaly detection and threat prediction are crucial techniques used to identify unusual patterns or deviations from expected behavior within data. In the context of securing information-carrying sensing signals in ISAC networks, they play a significant role in detecting potential security threats or privacy breaches. By promptly detecting anomalies and predicting threats, ISAC networks can respond proactively to prevent or mitigate security incidents before they escalate. On the other hand, they can help optimize resource allocation in resource-constrained ISAC networks. By focusing on suspicious activities, the system can prioritize resource usage and minimize wastage. Both anomaly detection and threat prediction often utilize data-driven AI approaches, such as Long Short-Term Memory, Support Vector Machines, Autoencoders, to name a few. Each algorithm has its strengths and suitability for different types of data and anomalies. 
  \item {\bf Resource Allocation} AI-driven resource allocation plays a crucial role in preserving privacy and enhancing security within ISAC networks. In fact, resource allocation in ISAC networks encompasses various aspects, including but not limited to spectrum allocation, bandwidth management, and beamforming design. 
  AI can integrate privacy and security considerations into resource allocation decisions. AI-driven resource allocation is generally based on continuously analyzing real-time, complex and diverse data in response to changing network conditions, environmental factors, and user demands, leading to more efficient utilization of computational resources, spectrum. One AI technique that is worthy to be mentioned here is federated learning \cite{Shi2022,Anbang2023}, which is a privacy-preserving machine learning technique that allows multiple parties or devices to collaboratively train a machine learning model without sharing their raw data with a centralized server. In this case, federated learning enables each party to keep its data locally, reducing the risk of data breaches or unauthorized access to sensitive information. Raw data remains decentralized and private.
 \item {\bf User Authentication and Access Control} Through continuous learning and adaptation, AI-driven systems can enhance user authentication, access control accuracy, and user profiling in ISAC networks, providing robust and proactive security measures. Specifically, AI leverages machine learning algorithms to train models that can recognize and verify biometric traits, such as fingerprints, facial features, voice patterns, and behavioral biometrics, thereby triggering additional authentication steps when necessary to ensure ongoing authorization. Also, AI algorithms continuously learn from user interactions and adjust access policies based on risk levels, user roles, and situational changes. This adaptability ensures that access controls remain effective in response to evolving security requirements. On the other hand, AI-driven model can create user profiles based on historical data and behavior patterns. These profiles help establish a baseline for each user, aiding in distinguishing legitimate users from potential intruders. Machine learning algorithms, such as neural networks, support vector machines, decision trees, and clustering methods, are commonly used to process and analyze the data for authentication and access control purposes.
\end{itemize}
By integrating these AI-driven mechanisms cohesively, ISAC networks can establish a comprehensive security ecosystem. These three mechanisms mutually reinforce each other's functionalities, creating a robust and symbiotic relationship. For instance, the insights gathered from AI-driven anomaly detection and threat prediction inform AI-driven resource allocation and user authentication and access control, enabling them to allocate computational resources more effectively and trigger additional authentication steps when needed. With optimized resource allocation, AI-driven detection, prediction, authentication, and access control functions can operate more efficiently and reliably. Furthermore, AI-driven user authentication and access control provide valuable information on user activity patterns and resource demands, aiding in prioritizing critical tasks and ensuring equitable distribution of resources throughout the network. This cohesive integration maximizes the network's security capabilities and fosters a proactive approach to privacy preservation and threat mitigation.

Despite these promising attributes, AI requires large and diverse datasets for training and continuous learning to achieve these functionalities. The collection and processing of personal or sensitive information raise privacy concerns if not adequately protected or anonymized. In addition, AI algorithms, particularly deep learning models, often operate as black boxes, making it challenging to understand the reasoning behind their decisions. This lack of explainability can be a concern in critical applications where the ability to justify security decisions is essential.
%\begin{figure}[ht]
%\centering
%\includegraphics[width=3.4in]{friendly-jamming_model.eps}
%\caption{Case study of the friendly-jamming method.}
%\label{fig: RIS_model}
%\end{figure}

\subsection{Friendly Jamming}
Different from computational-complex cryptographic techniques, physical layer security  is able to offer a less computational-complex solution to the security of ISAC networks. One merit of physical layer security techniques for the ISAC network is that the key distribution and encryption/decryption process are not required. Among all the approaches of physical layer security, the friendly jamming method is an efficient and effective approach to confuse the eavesdropper and strengthen network security. The method essentially injects friendly jamming signals into the wireless channels of the eavesdroppers so that the eavesdroppers are not able to decode the confidential information signals successfully due to the minimized SINR. The main merits of friendly jamming methods are the low computational load and low implementation complexity. Moreover, the coordination messages exchanging and extra processing of the legitimate information signals are unnecessary in friendly jamming schemes.

\begin{figure}[ht]
\centering
\includegraphics[width=3.2in]{FJ_model_N.eps}
\caption{Case study of the friendly jamming method.}
\label{fig: friendly-jamming_model}
\end{figure}

%We introduce a general case study of applying the friendly-jamming scheme to protect the wireless network communication security in Figure~\ref{fig: friendly-jamming_model}. In this network, multiple legitimate transmitters transmit confidential information to the corresponding legitimate receivers, and an eavesdropper in this network attempts to wiretap the transmitted legitimate information. The number of legitimate transmitters is $M$ and the locations of $M$ legitimate transmitters are randomly distributed according to the homogeneous Poisson Point Process (HPPP). The eavesdropper is assumed to be passive and does not transmit signals to avoid being discovered by legitimate users. If the eavesdropper generates signals, the appearance eavesdropper can be easily discovered and be located with multiple technologies. There are $N$ friendly jammers deployed in the network to protect legitimate information from being wiretapped by the eavesdropper.

We introduce a general case study of applying the friendly jamming scheme to protect the users’ privacy of wireless ISAC network in Figure~\ref{fig: friendly-jamming_model}. In this network, multiple legitimate transmitters transmit confidential information signals to the corresponding legitimate information receivers and transmit legitimate sensing signals to the corresponding sensing targets. In this network, an eavesdropper  ${ E_1}$ in this network attempts to wiretap the transmitted legitimate information, which is named by the eavesdropper for information. Another eavesdropper ${E_2}$ attempts to intercept the echo signal from the sensing target and decrypt the sensory information, which is named by the eavesdropper for sensing.

The number of legitimate transmitters is $M$ and the locations of $M$ legitimate transmitters are randomly distributed according to the homogeneous Poisson Point Process (HPPP). There are $N$ friendly jammers deployed in the network to protect legitimate information signals from being wiretapped by the eavesdropper for information ${E_1}$, and protect the sensing targets from being sensed illegally by the eavesdropper for sensing ${E_2}$.
In this case study, the beamforming technology is considered, because applying the directional antenna can concentrate the jamming signal on the potential eavesdropped area rather than the legitimate communication region. For simplicity, each device except a friendly jammer is assumed to be equipped with an omnidirectional antenna. 

%For simplicity, each device except a friendly jammer is assumed to be equipped with an omnidirectional antenna. 
%The eavesdropper for information is assumed to be passive and does not transmit signals to avoid being discovered by legitimate users. If the eavesdropper generates signals, the appearance eavesdropper can be easily discovered and be located with multiple technologies.
%Then we propose the Dir-friendly-jamming scheme where friendly jammers are equipped with the directional antenna. For comparison, we consider the Omni-friendly-jamming scheme where friendly jammers are equipped with an omnidirectional antenna. We also consider the Non-friendly-jamming scheme, in which no friendly jammer is deployed.

\begin{figure}[ht]
\centering
\includegraphics[width=3.6in]{ Figure_PE_PS.eps}
\caption{Number of friendly jammers $N$ ranges from 1 to 16.}
\label{fig: friendly-jamming_SOP}
\end{figure}

In this case study, the information and sensing security performance metric probability of successfully eavesdropping (PSE) is adopted. The eavesdropping behavior is successful if the SINR of the received signal at the eavesdropper is higher than a threshold value. In Figure~\ref{fig: friendly-jamming_SOP}, we present PSE results of eavesdropper ${E_1}$ and eavesdropper ${E_2}$ with a varied number of friendly jammers $N$. As shown in Figure~\ref{fig: friendly-jamming_SOP}, deploying friendly jammers will lead to a significant decrease in the PSE of eavesdropper ${E_1}$ and eavesdropper ${E_2}$ in the ISAC network, especially when there are more than 2 friendly jammers deployed. 
For instance, when the number of friendly jammers $N=6$, the PSE of the eavesdropper ${E_1}$ is $0.3937$ reduced (i.e., $72.72\%$ reduction) compared to that without friendly jamming. When the number of friendly jammers $N=6$, the PSE of the eavesdropper ${E_2}$ is $0.6481$ reduced (i.e., $83.71\%$ reduction) compared to that without friendly jamming. With this set of results, we find that friendly jammers has a more significant impact on sensing rather than on communication.
On the other hand, we find that deploying a large number of friendly jammers may also be unnecessary. When the number of friendly jammers $N$ is larger than 8, increasing the number of friendly jammers will not lead to a significant decrease in the PSE. In the sensing and communication process with friendly jamming the curves of PSE drop rapidly as long as we deploy a few friendly jammers.


\subsection{RIS-assisted Design}
In recent years, a groundbreaking technology known as RIS has emerged as a promising solution to constructing a smart propagation environment. RIS refers to passive metasurfaces integrated with smart controllers, allowing them to dynamically manipulate the electromagnetic waves passing through them. By intelligently controlling the propagation and reflection of signals, RIS offers the potential to nulling the undesired transmission and enhance signal transmission to target directions, thereby enhancing user privacy, protecting sensitive data, and securing wireless communications. As the interest in RIS continues to grow, we anticipate that this innovative technology will play a pivotal role in realizing privacy-preserving and secure ISAC networks. 

Specifically, RIS can be strategically deployed to create ``private zones'' or ``signal shields'' within the ISAC network. By intelligently controlling the propagation of electromagnetic waves, RIS can confine the sensing and communication signal within designated areas, preventing signal leakage beyond those boundaries. Similarly, RIS can create virtual access barriers or zones that allow or deny access to specific users or devices, enhancing access control in ISAC networks. Also, RIS can assist in establishing secure communication links between trusted parties within the network by optimizing the signal paths and selectively enhancing or attenuating signals. In addition, RIS can facilitate beamforming techniques while preserving user privacy. Instead of directly transmitting sensitive information towards the receiver, RIS can manipulate the signal path to focus the beam only on the intended recipient, reducing the risk of unintended information leakage.

\begin{figure}[ht]
\centering
\includegraphics[width=3.4in]{RIS_SM.eps}
\caption{Case study of the RIS-assisted privacy perserving system.}
\label{fig: RIS_model}
\end{figure}

In order to highlight the effectiveness of RIS, we propose RIS-enabled solution to preserve users' privacy in ISAC networks as shown in Fig. \ref{fig: RIS_model}. In particular, a user who want to keep its privacy and stay away from other sensing signals can utilize the RIS deployed nearby to block the undesired sensing links from untrusted organizations. In this case, the reflecting elements on RIS are adpatively adjusted to null the undesired sensing link to realize the "private zones". At the same time, the sensing link to desired sensing target are expected to be enhanced at the same time to improve the sensing accuracy. It can be observed from Fig. \ref{fig: RIS_B}, under the assitance of RIS with $64$ reflecting elements, the sensing beampattern gain between the ISAC transmitter and the privacy-preserving user is around the $10^{-38}$ order of magnitude, which almost approaches to zero, while the sensing beampattern gain between the ISAC transmitter and the desired sensing target is around $10^{-4}$. The huge gap between the two sensing beampattern gain satisfy the expectation that the user do not want to be sensed, while the desired sensing target can be better detected compared to the system without RIS. However, the two beampattern gain lines achieved by random designed RIS show the necessities to develop reflecting beamforming algorithms. Otherwise, the privacy-preserving user might be better sensed by the untrusted party as RIS improve the wireless propagation environment by introducing more degree of freedoms.  
\begin{figure}[ht]
\centering
\includegraphics[width=3.6in]{beamforming.eps}
\caption{Sensing beampattern gain versus signal transmit power.}
\label{fig: RIS_B}
\end{figure}

However, there are several challenges that must be addressed to harness the full potential of this technology. On the one hand, designing efficient control algorithms for RIS to achieve privacy-preserving and secure communication while considering energy efficiency and network performance is a complex task. On the other hand, obtaining precise channel state information and providing timely feedback to the RIS is challenging, especially considering passive reflecting elements on RIS. Also, RIS itself maybe physically destroyed when malicious party 
want to perform security attacks. Moreover, for RIS in public without strict access control, adversaries could attempt to compromise RIS control mechanisms or manipulate the surfaces to disrupt communication or extract sensitive information. Multidisciplinary research and innovation are necessary to overcome these obstacles and unlock the full potential of RIS in realizing privacy-preserving and secure ISAC networks.  

\section{Conclusion}
Overall, while integrating sensing into communication networks has many potential benefits, it is important to carefully consider the privacy implications of this technology and take steps to protect individuals' privacy and security. AI-enabled schemes, friendly jamming, and RIS-assisted were highlighted as promising solutions to realize privacy-preserving and secure ISAC networks. Case studies demonstrated their potential.

\bibliographystyle{IEEEtran} 
\bibliography{IEEEabrv,bib}

% \textbf
% {Ruxiao Chen} is currently a junior student at Shandong University. He is also an active participant in the seminar of the Shandong Key Laboratory of Wireless Communication Technologies. Despite being early in his academic journey, he has already made several contributions to the field. he has engaged in research projects and discussions with peers and professors, always seeking to expand his knowledge and hone his skills. His research interest lies in exploring the intersection of AI and edge computing, as well as how machine learning can be leveraged to solve complex problems in real-world settings.

% \textbf
% {Shuaishuai Guo}(Senior Member, IEEE) received the B.E and Ph.D. degrees in communication and information systems from the School of Information Science and Engineering, Shandong University, Jinan, China, in 2011 and 2017, respectively. He visited the University of Tennessee at Chattanooga (UTC), USA, from 2016 to 2017. He worked as a postdoctoral research fellow at King Abdullah University of Science and Technology (KAUST), Saudi Arabia from 2017 to 2019. Now, he is working as a full professor of Shandong University. His research interests include 6G communications and machine learning.


\end{document}

