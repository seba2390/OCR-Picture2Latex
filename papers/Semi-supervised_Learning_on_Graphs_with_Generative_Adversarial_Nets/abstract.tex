\begin{abstract}
	We investigate how generative adversarial nets (GANs) can help semi-supervised learning on graphs.
	We first provide insights on working principles of adversarial learning over graphs and then
	present \model, a novel approach to semi-supervised learning on graphs.
    In \model, generator and classifier networks play a novel competitive game. At equilibrium, generator generates fake samples in low-density areas between subgraphs. In order to discriminate fake samples from the real, classifier implicitly takes the density property of subgraph into consideration. An efficient adversarial learning algorithm has been developed to improve traditional normalized graph Laplacian regularization with a theoretical guarantee.
    
	Experimental results on several different genres of datasets show that the proposed \smodel significantly outperforms several state-of-the-art methods.
	\smodel can be also trained using mini-batch, thus enjoys the scalability advantage.

\end{abstract}