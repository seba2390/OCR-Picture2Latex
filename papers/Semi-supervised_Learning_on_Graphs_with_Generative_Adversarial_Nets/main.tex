\documentclass[sigconf]{acmart}

\usepackage{booktabs}
\usepackage{xcolor}
\usepackage{soul}
\usepackage[utf8]{inputenc}
\usepackage{amsfonts}
\usepackage{amsmath}
\usepackage{amssymb}
\usepackage{amsthm} 
\usepackage{graphicx}
\usepackage{float}
\usepackage{multirow}
\usepackage[ruled]{algorithm2e}
\usepackage{latexsym}
        
\newtheorem{theorem}{Theorem}  
\newtheorem{definition}{Definition}  
\newtheorem{lemma}{Lemma}  
\newtheorem{corollary}{Corollary}  
\newtheorem{assumption}{Assumption}
\setlength{\belowcaptionskip}{-0.9em}
\setlength{\abovecaptionskip}{0.3em}

\newcommand{\jt}[1]{\textbf{\color{red}[**JT: #1 **]}}  % to fix
\newcommand{\todo}[1]{\textbf{\color{blue}[(TODO: #1 )]}}  % to fix
\newcommand{\hide}[1]{} %hide
\newcommand{\vpara}[1]{\vspace{0.1in}\noindent\textbf{#1 }}
\newcommand{\para}[1]{\vspace{0.01in}\noindent\textbf{#1 }}
\newcommand{\secref}[1]{Section~\ref{#1}}
\newcommand{\Real}{\ensuremath{\mathbb{R}}}
\newcommand{\figref}[1]{Figure~\ref{#1}}
\newcommand{\beal}[1]{\begin{align}#1\end{align}}
\newcommand{\beq}[1]{\begin{equation}#1\end{equation}
	\normalsize
}
\newcommand{\beqn}[1]{\begin{eqnarray}#1\end{eqnarray}}
\newcommand{\besp}[1]{\begin{split}#1\end{split}}
\newcommand{\model}{{{G}{raph}{\large\emph{S}}{\footnotesize GAN}}}
\newcommand{\smodel}{\model\space}
\DeclareMathOperator*{\argmin}{argmin}



\copyrightyear{2018}
\acmYear{2018}
\setcopyright{acmcopyright}
\acmConference[CIKM '18]{The 27th ACM International Conference on Information and Knowledge Management}{October 22--26, 2018}{Torino, Italy}
\acmBooktitle{The 27th ACM International Conference on Information and Knowledge Management (CIKM '18), October 22--26, 2018, Torino, Italy} \acmPrice{15.00}
\acmDOI{10.1145/3269206.3271768} \acmISBN{978-1-4503-6014-2/18/10}


\fancyhead{}

\settopmatter{printacmref=true}

\begin{document}
\title{
Semi-supervised Learning on Graphs with Generative Adversarial Nets
}

% \titlenote{Produces the permission block, and
%   copyright information}
% \subtitle{Extended Abstract}
% \subtitlenote{The full version of the author's guide is available as
%   \texttt{acmart.pdf} document}


 \author{Ming Ding}
% % \authornote{}
% \orcid{1234-5678-9012}
 \affiliation{%
   \institution{Tsinghua University}
%   \streetaddress{}
   \city{Beijing, China}
%   \state{}
%   \postcode{}
 }
 \email{dm18@mails.tsinghua.edu.cn}

 \author{Jie Tang}
% % \authornote{}
% \orcid{1234-5678-9012}
 \affiliation{%
   \institution{Tsinghua University}
%   \streetaddress{}
   \city{Beijing, China}
%   \state{}
%   \postcode{}
 }
 \email{jietang@tsinghua.edu.cn}
 
 
 \author{Jie Zhang}
% % \authornote{}
% \orcid{1234-5678-9012}
 \affiliation{%
   \institution{Tsinghua University}
%   \streetaddress{}
   \city{Beijing, China}
%   \state{}
%   \postcode{}
 }
 \email{j-z16@mails.tsinghua.edu.cn}
% The default list of authors is too long for headers.
\renewcommand{\shortauthors}{}
\begin{abstract}
\label{sec:abstract}

%% 1. what is the problem 
Scientific applications that run on leadership computing facilities often face the challenge 
of being unable to fit leading science cases onto accelerator devices due to memory constraints 
(memory-bound applications).
%
% 2. what is your solution 
In this work, the authors studied one such US Department of Energy mission-critical condensed matter 
physics application, Dynamical Cluster Approximation (DCA++), and this paper discusses how device memory-bound challenges were successfully reduced  by proposing an effective 
``all-to-all'' communication method---a ring communication algorithm. 
%
This implementation takes advantage of acceleration on GPUs and remote direct memory access (RDMA) for fast data exchange between GPUs. 
%
\\Additionally, the ring algorithm was optimized with sub-ring communicators
and multi-threaded support to further reduce communication overhead and 
expose more concurrency, respectively.
%
% 3. What's the cherry-picked evaluation result you want to mention
The computation and communication were also analyzed 
by using the Autonomic Performance Environment for Exascale 
(APEX) profiling tool,  and this paper further discusses the 
performance trade-off for the ring algorithm implementation. 
%
The memory analysis on the ring algorithm shows that the allocation size for the authors' most 
memory-intensive data structure per GPU is now reduced to $1/p$ of the original size, where $p$ is the number of GPUs in the ring communicator.
%
The communication analysis suggests that 
the distributed Quantum Monte Carlo execution time grows linearly as sub-ring size increases, and the cost of messages passing through the network interface connector could be a limiting factor.


%
% \todoRed{Ronnie: Next sentence needs rewrite, too much information about Green's function that no one knows in the abstract; recommend generalizing.} \emph {However, DCA++ is currently facing memory-bound challenge as 
% a larger device array $G_t$ is limited by device memory size, where
% $G_t$ is a two-particle Green's function that allows condensed matter
% scientists to explore larger and more complex (higher fidelity)
% physics cases.}

\end{abstract}

\keywords{DCA++, Quantum Monte Carlo, GPU Remote Direct Memory Access, memory-bound issue, exascale machines}





%
% The code below should be generated by the tool at
% http://dl.acm.org/ccs.cfm
% Please copy and paste the code instead of the example below.
%
\begin{CCSXML}
<ccs2012>
<concept>
<concept_id>10002951.10003227.10003351</concept_id>
<concept_desc>Information systems~Data mining</concept_desc>
<concept_significance>500</concept_significance>
</concept>
<concept>
<concept_id>10010147.10010257.10010282.10011305</concept_id>
<concept_desc>Computing methodologies~Semi-supervised learning settings</concept_desc>
<concept_significance>300</concept_significance>
</concept>
</ccs2012>
\end{CCSXML}

\ccsdesc[500]{Information systems~Data mining}
\ccsdesc[300]{Computing methodologies~Semi-supervised learning settings}



\keywords{graph learning; semi-supervised learning; generative adversarial networks}


\maketitle
Reinforcement learning has achieved great success in areas such as Game-playing \citep{silver2018general,vinyals2019grandmaster}, robotics \cite{kober2013reinforcement}, large language models \citep{ouyang2022training}, etc.
However, due to safety concerns or physical limitations, in some real-world reinforcement learning problems, we must consider additional constraints that may influence the optimal policy and the learning process \citep{garcia2015comprehensive}.
% For example, a robotic arm must not take actions that may cause harm to itself or the environments.
A standard framework to handle such cases is the constrained Markov Decision Process (CMDP) \citep{altman1999constrained}.
Within the CMDP framework, the agent has to maximize
the expected cumulative reward while
obeying a finite number of constraints, which are usually in the form of expected cumulative cost criteria.

However, we are sometimes concerned with the problem with a continuum of constraints.
For example,
the constraints we meet might be time-evolving or subject to uncertain parameters, which
cannot be formulated as an ordinary CMDP
(see Examples \ref{Example_Time_Evolving} and  \ref{Example_Uncertain}).
In this paper we would study a generalized CMDP  
to address the above problem.  Because the constraints are not only infinite-number but also lie
in a continuous set,
the generalization is not trivial. Fortunately, we find that we can borrow the idea behind semi-infinite programming (SIP) \citep{remez1934determination, hettich1993semi} to deal with the semi-infinite constraints.
Accordingly, we propose \emph{semi-infinitely constrained Markov decision processes} (SICMDPs)
as a novel complement to the ordinary CMDP framework.
%More specifically,  an SICMDP model %, we consider 
%contains a continuum of constraints whereas an ordinary CMDP contains a finite number of constraints. 

%This generalization is natural but not trivial. However, we can brows the idea  
%The idea is quite natural and can be backtracked
%to the practice of extending linear programming to linear semi-infinite programming (LSIP) %\cite{remez1934determination, GobernaLSIO1998}.
%In addition, 
%As a complementary approach to the ordinary CMDP framework, 
%SICMDP can be used to model these problems  which cannot be described by a finite number of constraints
%that are not covered by .
%For example,
%the restrictions we consider can be time-evolving or subject to uncertain parameters
%, thus
%cannot be described by a finite number of constraints but a continuum of constraints 
%(see Examples \ref{Example_Time_Evolving} and  \ref{Example_Uncertain}).

We also present two reinforcement learning algorithms to solve SICMDPs called SI-CRL and SI-CPO, respectively.
SI-CRL is a model-based reinforcement learning algorithm designed for tabular cases, and SI-CPO is a policy optimization algorithm for non-tabular cases.
% and analyze its performance both theoretically and empirically.
The main challenge is that we need to deal with a continuum of constraints, thus reinforcement learning algorithms for ordinary CMDPs do not work anymore.
In SI-CRL, we tackle this difficulty by first transforming the reinforcement learning problem to an equivalent LSIP problem, which can then be solved using methods in the LSIP literature like the dual exchange methods \citep{Hu1990,reemtsen1998numerical}.
In SI-CPO, we resort to the idea of cooperative stochastic approximation developed in \cite{lan2020algorithms, wei2020comirror}.
As far as we know, we are the first to introduce tools from semi-infinitely programming (SIP) into the reinforcement learning community for solving constrained reinforcement learning problems.

% To the best of our knowledge, we are the first to apply tools from semi-infinitely programming (SIP) to solve reinforcement learning problems.
Furthermore, we give theoretical analysis for both SI-CRL and SI-CPO.
We decompose the error of SI-CRL into two parts: the statistical error from approximating the true SICMDP with an offline dataset and the optimization error due to the fact that the solution of the LSIP problem obtained by the dual exchange method is inexact.
On the optimization side, we show that the iteration complexity of SI-CRL is $O\left(\left\{\mathrm{diam}(Y)L\sqrt{|\gS|^2|\gA|m}/\left[(1-\gamma)\epsilon\right]\right\}^m\right)$.
On the statistical side, we show that the sample complexity of SI-CRL is $\widetilde O\left(\frac{|S|^2|A|^2}{\epsilon^2(1-\gamma)^3}\right)$ if the offline dataset is generated by a generative model, and $\widetilde O\left(\frac{|S||A|}{\nu_{\min} \epsilon^2(1-\gamma)^3}\right)$ if the dataset is generated by a probability measure $\nu$ as considered in \cite{chen2019information}.
Here $\widetilde O$ means that all logarithm terms are discarded.
For SI-CPO, things become a little more complicated because other than the statistical error and the optimization error, we also need to consider the function approximation error, which comes from imperfect policy parametrizations.
It is shown if the function approximation error can be controlled to $O(\epsilon)$ order, the iteration complexity of SI-CPO is $\widetilde{O}\left(\frac{1}{\epsilon^2(1-\gamma)^6}\right)$ and the sample complexity of SI-CPO is $\widetilde{O}(\frac{1}{\epsilon^4(1-\gamma)^{10}})$.
Here our iteration complexity bound is equivalent to a typical $\widetilde O(1/\sqrt{T})$ global convergence rate.

We perform a set of numerical experiments to illustrate the SICMDP model and validate our proposed algorithms.
Specifically, we examine two numerical examples, namely the discharge of sewage and ship route planning.
Through the discharge of sewage example, we show the advantage of the SICMDP framework over the CMDP baseline obtained by naive discretization in modeling realistic sequential decision-making problems.
Moreover, we demonstrate the effectiveness of the SI-CRL and SI-CPO algorithms in such tabular environments. 
In the ship route planning example, we illustrate the benefits of the SICMDP framework and the ability of the SI-CPO algorithm to address complex continuous control tasks involving continuous state spaces with modern deep reinforcement learning techniques.

% In summary, our contributions are listed as follows.
% First, we present the SICMDP model, which can be viewed as a generalization of the ordinary CMDP model.
% Second, we propose an algorithm to perform reinforcement learning for SICMDPs, which is called SI-CRL, and we believe that we are the first to apply tools from SIP
% to solve reinforcement learning problems.
% Third, we give a theoretical analysis of SI-CRL and identify both its sample complexity and iteration complexity.
% In addition, we perform numerical experiments to illustrate the SICMDP model and validate the SI-CRL algorithm.
% \{This paragraph can be removed!!! \}





\section{General Framework}
\label{sec:framework}
Our goal in this work is to demonstrate the utility of natural language descriptions in assisting policy transfer across domains. In this section, we first describe our environment setup and the general framework of our approach. The details of our model and algorithm follow in Section~\ref{sec:model}.

\subsection{Environment Setup} 
We model a single environment as a Markov Decision Process (MDP),  $E = \langle S, A, T, R, O, Z \rangle$. Here, $S$ is the state space, and $A$ is the set of actions available to the agent. In this work, we consider every state $s \in S$ to be a 2-dimensional grid of size $m \times n$, with each cell containing an entity symbol $o \in O$.\footnote{In our experiments, we relax this assumption to allow for multiple entities per cell, but for ease of description, we shall assume a single entity per cell. The assumption of 2-D worlds can also be easily relaxed to generalize our model to other situations.} $T$ is the transition distribution over all possible next states $s'$ conditioned on the agent choosing action $a$ in state $s$. $R$ determines the reward provided to the agent at each time step. The agent does not have access to the true $T$ and $R$ of the environment. Each domain also has a goal state $s_g \in S$ which determines when an episode terminates. Finally, $Z$ is the complete set of text descriptions provided to the agent for this particular environment. 

\subsection{Reinforcement Learning (RL)}
The goal of an autonomous agent is to maximize cumulative reward obtained from the environment. A traditional way to achieve this is by learning an action value function $Q(s,a)$ through reinforcement. The \emph{Q-function} predicts the expected future reward for choosing action~$a$ in state~$s$. A straightforward policy then is to simply choose the action that maximizes the $Q$-value in the current state: 

\begin{dmath*}
\pi(s) = \argmax_a Q(s,a)
\end{dmath*}

If we also make use of the descriptions, we have a text-conditioned policy: 
\begin{dmath}
\pi(s, Z) = \argmax_a Q(s, a, Z)
\end{dmath} 

A successful control policy for an environment will contain both knowledge of  the environment dynamics and the capability to identify goal states. While the latter is task-specific, the former characteristic is more useful for learning a general policy that transfers to different domains. Based on this hypothesis, we employ a model-aware RL approach that can learn the dynamics of the world while estimating the optimal $Q$. Specifically, we make use of \emph{Value Iteration (VI)}~\cite{sutton1998introduction}, an algorithm based on dynamic programming. The update equations for value iteration in our setup are:
\begin{align}
Q^{(n+1)}(s, a, Z) &= \sum_{s' \in S} T(s' | s, a, Z) [ R(s', Z) + \gamma V^{(n)}(s', Z) ]  \nonumber \\
V^{(n+1)}(s, Z) &= \max_a Q^{(n+1)}(s,a, Z) 
\label{eq:vi}
\end{align}
where $\gamma$ is a discount factor and $n$ is the iteration number. The updates require an estimate of $T$ and $R$, which the agent must obtain through exploration of the environment.

% Note that this assumes the agent has knowledge of the true $T$ and $R$ in order to estimate $Q$ and $V$. Since our setup does not provide this information, the agent has to estimate the transition and reward functions from its interactions with the world.

\subsection{Text Descriptions}
Estimating the dynamics of the environment from interactive experience can require a significant number of samples. Our main hypothesis is that if an agent can derive information about the dynamics from text descriptions, it can determine $T$ and $R$ faster and more accurately. 
% Hence, we work with text that provides such relevant particulars of the domain.

For instance, consider the sentence \emph{``Red bat that moves horizontally, left to right''}. This talks about the movement of a third-party entity (\emph{bat}), independent of the agent's goal. Provided the agent can learn to interpret this sentence, it can then infer the direction of movement of a different entity (e.g. \emph{``A tan car moving slowly to the left''}) in a different domain. Further, this inference is useful even if the agent has a completely different goal. On the other hand, instruction-like text such as \emph{``Move towards the wooden door''} is highly context-specific and only relevant to domains that have the mentioned goal.

With this in mind, we provide the agent with text descriptions that collectively portray characteristics of the world. These descriptions are crowdsourced by asking humans to view gameplay videos and describe entities.  A single description talks about one particular entity in the world. The text contains (partial) information about the entity's movement and interaction with the player avatar. Each description is also aligned to its corresponding entity in the environment and not all entities may have a description.
% We make sure that a simple mapping cannot be found between entities in different domains using just their names in text.\todo{clarify this}
Figure~\ref{fig:descriptions} provides some samples; more details on data collection and statistics are in Section~\ref{sec:experiments}. 

\begin{figure}
  \begin{annotationbox}
%     \centering
    \small
      \begin{itemize}
        \item Scorpion2: \emph{Red scorpion that moves up and down} 
        \item Alien3: \emph{This character slowly moves from right to left while having the ability to shoot upwards}
        \item Sword1: \emph{This item is picked up and used by the player for attacking enemies}
      \end{itemize}
%     }
  \end{annotationbox}
  \caption{Example text descriptions of entities in different environments, collected using Amazon Mechanical Turk. Turkers were shown videos of gameplay in the different environments and asked to describe each entity's behavior or role. Note that these sentences are not instructive, since they provide no direct information on how to act in the environment.}
  \label{fig:descriptions}
\end{figure}

\subsection{Transfer for RL}
In order to test our grounding hypothesis, we consider learning across multiple environments. Specifically, an agent can learn to ground language semantics in an environment $E_1$ and then we can test its understanding capability by placing it in a new unseen domain, $E_2$. The agent can obtain unlimited experience in $E_1$, and after convergence of its policy, it is allowed to interact with and learn a policy for $E_2$. We do not provide the agent with any explicit mapping between different entities or goals across domains, either directly or through the text. For instance, even though the boulders in \emph{Boulderchase} are impassable objects just like the walls in \emph{Bomberman}~\ref{fig:example}, the agent does not have access to a mapping between these entities. In this setup, the agent's goal is to re-utilize information obtained through its interactions in $E_1$ to learn more efficiently in $E_2$.


% \paragraph{Environment Setup} 
% We model a single environment as a Markov Decision Process (MDP), represented by $E = \langle S, A, T, R, O, Z \rangle$. Here, $S$ is the state space, and $A$ is the set of actions available to the agent. In this work, we consider every state $s \in S$ to be a 2-dimensional grid of size $m \times n$, with each cell containing an entity symbol $o \in O$.\footnote{In our experiments, we relax this assumption to allow for multiple entities per cell, but for ease of description, we shall assume a single entity. The assumption of 2-D worlds can also be easily relaxed to generalize our model to other situations.} $T$ is the transition distribution over all possible next states $s'$ conditioned on the agent choosing action $a$ in state $s$. $R$ determines the reward provided to the agent at each time step. The agent does not have access to the true $T$ and $R$ of the environment. Each domain also has a goal state $s_g \in S$ which determines when an episode terminates. Finally, $Z$ is the complete set of text descriptions provided to the agent for this particular environment. 

% \paragraph{Reinforcement learning (RL)}
% The goal of an autonomous agent is to maximize cumulative reward obtained from the environment. A traditional way to achieve this is by learning an action value function $Q(s,a)$ through reinforcement. The \emph{Q-function} predicts the expected future reward for choosing action~$a$ in state~$s$. A straightforward policy then is to simply choose the action that maximizes the $Q$-value in the current state: $\pi(s) = \argmax_a Q(s,a)$. If we also make use of the descriptions, we have a text-conditioned policy: $\pi(s, Z) = \argmax_a Q(s, a, Z)$. 

% A successful control policy for an environment will contain both knowledge of  the environment dynamics and the capability to identify goal states. While the latter is task-specific, the former characteristic is more useful for learning a general policy that transfers to different domains. Based on this hypothesis, we employ a model-aware RL approach that can learn the dynamics of the world while estimating the optimal $Q$. Specifically, we make use of \emph{Value Iteration (VI)}~\cite{sutton1998introduction}, an algorithm based on dynamic programming. The update equations are as follows:
% \begin{align}
% Q^{(n+1)}&(s, a, Z) = R(s, a, Z)  \nonumber \\ 
% &+ \gamma \sum_{s' \in S} T(s' | s, a, Z) V^{(n)}(s', Z)  \nonumber \\
% V^{(n+1)}&(s, Z) = \max_a Q^{(n+1)}(s,a, Z) 
% \label{eq:vi}
% \end{align}
% where $\gamma$ is a discount factor and $n$ is the iteration number. The updates require an estimate of $T$ and $R$, which the agent must obtain through exploration of the environment.

% % Note that this assumes the agent has knowledge of the true $T$ and $R$ in order to estimate $Q$ and $V$. Since our setup does not provide this information, the agent has to estimate the transition and reward functions from its interactions with the world.

% \paragraph{Text descriptions}
% Estimating the dynamics of the environment from interactive experience can require a significant number of samples. Our main hypothesis is that if an agent can derive information about the dynamics from text descriptions, it can determine $T$ and $R$ faster and more accurately. 
% % Hence, we work with text that provides such relevant particulars of the domain.

% For instance, consider the sentence \emph{``Red bat that moves horizontally, left to right.''}. This talks about the movement of a third-party entity ('bat'), independent of the agent's goal. Provided the agent can learn to interpret this sentence, it can then infer the direction of movement of a different entity (e.g. \emph{``A tan car moving slowly to the left''} in a different domain. Further, this inference is useful even if the agent has a completely different goal. On the other hand, instruction-like text, such as \emph{``Move towards the wooden door''}, is highly context-specific, only relevant to domains that have the mentioned goal.

% With this in mind, we provide the agent with text descriptions that collectively portray characteristics of the world. A single description talks about one particular entity in the world. The text contains (partial) information about the entity's movement and interaction with the player avatar. Each description is also aligned to its corresponding entity in the environment. 
% % We make sure that a simple mapping cannot be found between entities in different domains using just their names in text.\todo{clarify this}
% Figure~\ref{fig:descriptions} provides some samples; details on data collection and statistics are in Section~\ref{sec:experiments}.

% \begin{figure}
%   \begin{annotationbox}
% %     \centering
%     \small
%       \begin{itemize}[leftmargin=0.45cm]
%         \item Scorpion2: \emph{Red scorpion that moves up and down} 
%         \item Alien3: \emph{This character slowly moves from right to left while having the ability to shoot upwards}
%         \item Sword1: \emph{This item is picked up and used by the player for attacking enemies}
%       \end{itemize}
% %     }
%   \end{annotationbox}
%   \caption{Some example text descriptions of entities in different environments.}
%   \label{fig:descriptions}
% \end{figure}

% % \begin{table}[h]
% % \centering
% % \resizebox{\linewidth}{!}{%
% % \begin{tabular}{  l  } \toprule
% % \textit{Red scorpion that moves up and down} \\
% % \textit{This character slowly moves left to right} \\ 
% % \textit{and has the ability to shoot to the left which can kill the player} \\
% % \textit{this item is picked up and used by the player for attacking enemies}
% % \textit{Ghost1 moves horizontally and is an enemy} \\
% % \textit{Alien3 is an enemy bomber shooting upwards} \\
% % \bottomrule
% % \end{tabular}
% % }
% % \caption{Some example text descriptions of various entities for a game environment.}
% % \label{table:descriptions}
% % \end{table}

% \paragraph{Transfer for RL}
% A natural scenario to test our grounding hypothesis is to consider learning across multiple environments. The agent can learn to ground language semantics in an environment $E_1$ and then we can test its understanding capability by placing it in a new unseen domain, $E_2$. The agent is allowed unlimited experience in $E_1$, and after convergence of its policy, it is then allowed to interact with and learn a policy for $E_2$. We do not provide the agent with any mapping between entities or goals across domains, either directly or through the text. The agent's goal is to re-utilize information obtained in $E_1$ to learn more efficiently in $E_2$.

% % For example, a `bat' in $E_1$ is not given the same symbol as a `bat' in $E_2$.\footnote{Also note that the behavior of a bat in the two environments can be substantially different.} The aim of transfer is to re-utilize information obtained in $E_1$ to learn efficiently in $E_2$.


% % \paragraph{Environment}
% % In our setup, an environment consists of a state space $\mathcal{S}$ and a set of text descriptions $\mathcal{Z} = \{z_i\}$. The state is a $m \times n$ grid world containing entities drawn from a set $\mathcal{O}$ (with unique IDs). Given an input state $s \in \mathcal{S}$, the agent can take a discrete action $a \in \mathcal{A}$, and observe a new state $s'$ of the environment, which changes according to a transition distribution $\mathcal{T}(s' | s,a)$. The environment also provides the agent with a reward $\mathcal{R}(s,a)$ at every time step. Note that the agent does not have access to the true $\mathcal{T}$ and $\mathcal{R}$ of its environment. 

% % Each description $z_i$ is a sentence that provides information about one particular entity type such as its movements or interactions with other entities. We assume access to the mapping between each $z_i$ and its corresponding object $o_i$.

% % \paragraph{Reinforcement Learning (RL)}
% % In the RL framework, the goal of an autonomous agent is to perform actions that maximize the cumulative reward it obtains from the environment. This is done by learning an action value function $Q(s,a)$, which predicts the expected future reward of choosing action~$a$ in state~$s$. Using this, a straightforward policy is to simply choose the action that maximizes the $Q$-value in the current state: $\pi(s) = \argmax_a Q(s,a)$.
% % % \todo{define Q and policy}

% % \paragraph{Transfer setup}
% % We are given a source environment $e_u$ and a target environment $e_v$. In addition, we have access to corresponding sets of text descriptions $\mathcal{Z}_u$ and $\mathcal{Z}_v$, respectively. We first estimate parameters $\Theta$ of a policy $\pi_u(s, \mathcal{Z}_u)$ through several interactions with $e_u$. The policy is optimized to obtain maximum possible reward on the source environment. Now, using $\pi_u$, our goal is to learn an optimal policy for the target environment $e_v$ in as few interactions as possible, by transferring knowledge obtained in $e_u$.


% % % Formally, let us consider an environment $e \in \mathcal{E}$, consisting of entities $\mathcal{O}^e = \{o^e_i\}$ and correspondingly aligned text descriptions $\mathcal{Z}^e = \{z^e_i\}$. Our goal is to learn a mapping from these descriptions to the control dynamics, while simultaneously learning an optimal policy. In this work, we consider two-dimensional state spaces, but the main facets of our model can be extended to other scenarios.
\section{The \MakeLowercase{i}W\MakeLowercase{inr}NFL model}
\label{sec:model}

In this section we are going to present the data we used to develop our in-game probability model as well as the design details of {\method}. 

{\bf Data: }In order to perform our analysis we utilize a dataset collected from NFL's Game Center for all the regular season games between the seasons 2009 and 2016. 
We access the data using the Python {\tt nflgame} API \cite{nflgame}. 
The dataset includes detailed play-by-play information for every game that took place during these seasons. 
This information is used to obtain the state of the game that will drive the design of {\method}. 
In total, we collected information for 2,048 regular season games and a total of 338,294 snaps/plays. 

{\bf Model: }
{\method} is based on a logistic regression model that calculates the probability of the home team winning given the current status of the game as: 

\begin{equation}
\Pr(H=1| \mathbf{x})= \frac{\exp(\mathbf{\weight}^T\cdot\mathbf{x})}{1+\exp(\mathbf{\weight}^T\cdot\mathbf{x})}
\label{eq:reg}
\end{equation}
where $H$ is the dependent random variable of our model representing whether the home team wins or not, $\mathbf{x}$ is the vector with the independent variables, while the coefficient vector $\mathbf{\weight}$ includes the weights for each independent variable and is estimated using the corresponding data.  
For a game of infinite duration a linear model could be a very good approximation.  
However, the boundary effects from the finite duration of a game create several non-linearities \cite{winston2012mathletics}.  
For this reason, we enhance our model - using the same set of features - with a Support Vector Machine classifier with radial kernel for the last three minutes of regulation.  
In order to obtain a probability output from the SVM classifier, we further use Platt's scaling \cite{platt1999probabilistic}: 

\begin{equation}
\Pr(H=1| \mathbf{x})= \frac{1}{1+\exp{(Af(x)+B)}}
\label{eq:platt}
\end{equation}
where $f(x)$ is the uncalibrated value produced by the SVM classifier: 

\begin{equation}
f(x) = \sum_{i} (\alpha_i y_i k(\mathbf{x}_i\cdot\mathbf{x}))+ b
\label{eq:svm}
\end{equation}
where $k(\mathbf{x},\mathbf{x}')$ is the kernel used for the SVM.   
Figure \ref{fig:iwinrNFL} depicts the simple flow chart of {\method}. 


\begin{figure}[t]
\begin{center}
\includegraphics[scale=0.35]{plots/iwinrNFL.pdf}%\vspacecap
 \caption{{\method} includes a linear and a non-linear component.}
 \label{fig:iwinrNFL}
\end{center}
\end{figure}

In order to describe the status of the game we use the following variables:

\begin{enumerate}
\item {\bf Ball Possession Team:} This binary feature captures whether the home or the visiting team has the ball possession
\item {\bf Score Differential:} This feature captures the current score differential (home - visiting)
\item {\bf Timeouts Remaining:} This feature is represented by two independent variables - one for the home and one for the away team - and they capture the number of timeouts remaining for each of the teams
%\item {\bf Quarter:} This feature captures the current quarter of the game
%\item {\bf Time Remaining:} This feature captures the time (in seconds) remaining for the current quarter to end
\item {\bf Time Elapsed: } This feature captures the time elapsed since the beginning of the game
\item {\bf Down:} This feature represents the down of the team in possession
\item {\bf Field Position:} This feature captures the distance covered by the team in possession from their own yard line
\item {\bf Yards-to-go:} This variables represents the number of yards needed for a first down
\item {\bf Ball Possession Time: } This variable captures the time that the offensive unit of the home team is on the field 
\item {\bf Ranking Differential: } This variable represents the difference of the win percentage for the two team (home - visiting)
\end{enumerate}

The last independent variable is representative of the power ranking difference between the two teams. 
Most of the existing models that include such a variable are using the Vegas line spread for each game.  
We choose not to do so for the following reason.  
The objective of the Vegas line is not to predict game outcomes but rather distribute money across the different bets.  
Exactly because of this objective the line is changing during the week before the game.  
While this line can change due to new information for the competing teams (e.g., injury updates), the line is mainly changing when a particular team has accumulated the majority of the bets. 
In this case it will also be hard to choose which line to use (e.g., the opening, the closing or some average of them).  
Therefore, we choose to use the win percentage differential of the two teams as an indicator of their strength (even though this has its own issues given the uneven schedule in NFL).  
However, note that if one would like to use the point spread as a variable this can be easily incorporated in the model. 
Table \ref{tab:iwinrnfl} presents the coefficients of the logistic regression model of {\method} with standardized independent variables for better comparisons. 


\begin{table}[ht]
\begin{center}
\def\sym#1{\ifmmode^{#1}\else\(^{#1}\)\fi}
\begin{tabular}{l*{1}{c}}
\toprule
                    &\multicolumn{1}{c}{(1)}\\
                    &\multicolumn{1}{c}{Winner}\\
\midrule
Possession Team (H)         &      0.41\sym{***}\\
                    &     (49.19)         \\
\addlinespace
Score Differential           &      3.59\sym{***}\\
                    &    (247.34)         \\
\addlinespace
Home Timeouts           &     0.12\sym{***}\\
                    &      (8.74)         \\
\addlinespace
Away Timeouts           &     -0.11\sym{***}\\
                    &    (-12.47)         \\
\addlinespace
Ball Possession Time  &     -0.05.\\
                    &    (-1.66)         \\
\addlinespace
Time Lapsed       &   -0.05.\\
                    &      (-1.66)         \\
\addlinespace
Down                &   -0.01         \\
                    &      (0.04)         \\
\addlinespace
Field Position            &   0.02\sym{**} \\
                    &      (2.71)         \\
\addlinespace
Yards-to-go                &  -0.01         \\
                    &      (0.23)         \\
\addlinespace
Rating differential         &       0.75\sym{***}\\
                    &     (80.47)         \\
\addlinespace
Intercept            &       0.57\sym{*}\\
                    &    (2.09)         \\
\midrule
Observations        &      338,294         \\
\bottomrule
\multicolumn{2}{l}{\footnotesize \textit{t} statistics in parentheses}\\
\multicolumn{2}{l}{\footnotesize \sym{$_.$} \(p<0.1\), \sym{*} \(p<0.05\), \sym{**} \(p<0.01\), \sym{***} \(p<0.001\)}\\
\end{tabular}
\end{center}
\caption{Standardized logisitic regression coefficients for {\method}.}
\label{tab:iwinrnfl}
\end{table}


As we can see, as one might have expected the current scoring differential exhibits the strongest correlation with the in-game win probability.  
The only factors that do not appear to be statistically significant predictors of the dependent variable are the down and the yards-to-go. 
Even though the corresponding coefficients are negative as one might have expected (e.g., being at an earlier down gives you more chances to advance the ball), they are not significant in estimating the win probability. 
On the contrary, all else being equal timeouts appear to be quiet important since they can help a team stop the clock, while teams with better win percentage appear to have an advantage as well, since this can be a sign of a better team. 
In the following section we provide a detailed evaluation of {\method}.
\section{Theoretical Basis}\label{sec:theory}
We provide theoretical analyses on why GANs can help semi-supervised learning on graph. In section \ref{sec:mot}, we claim that the working principle is to reduce the influence of labeled nodes across density gaps. In view of the difficulty to directly analyze the dynamics in training of deep neural networks,  
we base the analysis on the graph Laplacian regularization framework. 
\begin{definition}\label{marginal}\textbf{Marginal Node and Interior Node.}
Marginal Nodes $\mathcal{M}$ are nodes linked to nodes with different labels while Interior Nodes $\mathcal{I}$ not. Formally, $\mathcal{M} = \{v_i | v_i \in V \land (\exists v_j \in V, (v_i, v_j) \in E \land y_i \neq y_j)\}$, $\mathcal{I} = V \setminus \mathcal{M}$.
\end{definition}

\begin{assumption} \label{convergence}\textbf{Convergence conditions.} 
When $G$ converges, we expect it to generate fake samples linked to nearby marginal nodes. More specifically, let $V_g$ and $E_g$ be the set of generated fake samples and generated links from generated  nodes to nearby original nodes. we have  $\forall v_g \in V_g, (\exists v_i \in \mathcal{M}, (v_g, v_i) \in E_g) \land (\forall (v_g, v_i) \in E_g, v_i \in \mathcal{M}) $.  
\end{assumption}
\vspace{0.06in}

The loss function of graph Laplacian regularization framework is as follows:

%\begin{small}
\begin{equation}\label{basic}
	\mathcal{L}(y') = \sum\limits_{v_i\in V^L} loss(y_i, y_i') + \lambda\sum\limits_{v_i,v_j \in V}^{i\neq j} \alpha_{ij}\cdot neq(y_i', y_j')
\end{equation}
%\end{small}

\noindent where $y_i'$ denotes predicted label of node $v_i$. The $loss(\cdot, \cdot)$ function measures the supervised loss between real and predicted labels. $neq(\cdot, \cdot)$ is a  0-or-1 function representing {\it not equal}.

%\begin{small}
\begin{equation}\label{alpha}
\alpha_{ij} = \tilde{A_{ij}} = \frac{A_{ij}}{\sqrt{deg(i)deg(j)}}, (i \neq j)
\end{equation} 
%\end{small}

\noindent where $A$ and $\tilde{A}$ are the adjacent matrix and negative normalized graph Laplacian matrix, and $deg(i)$ means the degree of $v_i$. It should be noted that our equation is slightly different from \cite{zhou2004learning}'s because we only consider explicit predicted label rather than label distribution.

Normalization is the core of reducing the marginal nodes' influence. Our approach is  simple: generating fake nodes, linking them to nearest real nodes and solving graph Laplacian regularization. \emph{Fake} label is not allowed to be assigned to unlabeled nodes and loss computation only considers edges between real nodes. The only difference between before and after generation is that marginal nodes' degree changes. And then the regularization parameter $\alpha_{ij}$ changes. 

\subsection{Proof}
We analyze 
how generated fake samples help acquire correct classification.

\begin{corollary}\label{decrease} Under Assumption \ref{convergence}, let $\mathcal{L}(\mathcal{C}_{gt})$ and $\mathcal{L}(\mathcal{C}_{gt})'$ be losses of ground truth on graph $(V + V_g, E + E_g)$ and $(V + V_g', E + E_g')$. We have $\forall V_g \supsetneqq V_g'$, $\mathcal{L}(\mathcal{C}_{gt}) < \mathcal{L}(\mathcal{C}_{gt})'$, where $V_g$ and $E_g$ are set of generated nodes and edges.
\end{corollary}

Corollary \ref{decrease} can be easily deduced because of $\alpha_{ij}$ decreasing. Loss of ground truth continues to decrease along with new fake samples being generated. That indicates ground truth is more likely to be acquired. However, there might exist other classification solutions whose loss decreases more. Thus, we will further prove that we can make a perfect classification under reasonable assumptions with adequate generated samples.
 
\begin{definition} \textbf{Partial Graph.} We define the subgraph induced by all nodes labeled $c$ (aka. $V_c$) and their other neighbors $Ne_c$ as partial graph $G_c$.
\end{definition}
\begin{assumption}\label{connectivity} \textbf{Connectivity.}
The subgraph induced by all interior nodes in each class is connected. Besides, every marginal node connects to at least one interior node in the same class.
\end{assumption}

\hide{
\begin{assumption}\label{necessary} \textbf{Necessary Labels.} 
There is at least one labeled node in every class.
\end{assumption}
}

Most real-world networks are dense and big enough to satisfy Assumption \ref{connectivity}. 
There actually implies another weak assumption that at least one labeled node exists for each class. This is the usually guaranteed by the setting of semi-supervised learning.
Let $m_c$ be the number of edges between marginal nodes in $G_c$. Besides, we define $deg_c$ as the maximum of degrees of nodes in $G_c$ and $loss_i$ as the supervised loss for misclassified labeled node $v_i$.
	
\begin{theorem} \textbf{Perfect Classification.} 
If enough fake samples are generated such that $\forall v \in \mathcal{M}, deg(v) > d_0$, all nodes will be correctly classified. $d_0$ is the maximum of $\max\limits_cm_c^2deg_c$  and $\max\limits_{c,v_i\in V_c}\frac{\lambda m_c}{loss_i}$.
\end{theorem}
\begin{proof}
We firstly consider a simplified problem in partial graphs $G_c$, where nodes from $Ne_c$ have already been assigned fixed label $c'$. We will prove that the new optimal classification $\mathcal{C}_{min}$ are the classification $\mathcal{C}'$, which correctly assigns $V_c$ label $c$. Since $\mathcal{L}(\mathcal{C}_{min})<\mathcal{L}(\mathcal{C}') < \lambda m_c\cdot \frac{1}{\sqrt{d_0\cdot d_0}} < \lambda m_c / \max\limits_{c,v_i\in V_c}\frac{\lambda m_c}{loss_i} \leq \min\limits_{v_i\in V_c}loss_i$, optimal solution $\mathcal{C}_{min}$ should classify all labeled nodes correctly. 

Suppose that $\mathcal{C}_{min}$ assigns $v_i,v_j\in \mathcal{I}$ with different labels. The inequality $\mathcal{L}(\mathcal{C}_{min})\geq \lambda\alpha_{ij} = \frac{\lambda}{\sqrt{deg(v_i)deg(v_j)}}\geq \frac{\lambda}{deg_c} \geq \frac{\lambda}{m_cdeg_c} \geq \frac{\lambda m_c}{d_0} > loss_{\mathcal{C}'}$ would result in contradiction.
%
According to analysis above and Assumption \ref{connectivity}, all interior nodes in $G_c$ are assigned label a $c$ in $\mathcal{C}_{min}$.

Suppose that $\mathcal{C}_{min}$ assigns $v_i \in \mathcal{M}\cap V_c, v_j\in \mathcal{I}$ with different labels and $(v_i, v_j)\in E$. Let $v_i$  be assigned with $c'$. If we change $v_i$'s label to $c$, then $\alpha_{ij}$ between $v_i$ and its interior neighbors will be excluded from the loss function. But some other edges weights between $v_i$ and its marginal neighbors might be added to the loss function. Let $\lambda\Delta$ denotes the variation of loss. 
The following equation will show that the decrease of the loss would lead to a contradiction.

\begin{equation*}
\footnotesize
\begin{split}
\Delta \leq & \sum\limits_{\substack{v_k\in \mathcal{M}\\(v_i, v_k) \in E_c}}\frac{1}{\sqrt{deg(v_i)deg(v_k)}} -  \sum\limits_{\substack{v_j\in \mathcal{I}\\(v_i, v_j) \in E_c}}\frac{1}{\sqrt{deg(v_i)deg(v_j)}} \\ 
\leq & \frac{1}{\sqrt{deg(v_i)}} (\frac{m_c}{\sqrt{\max\limits_cm_c^2deg_c}} - \sum\limits_{\substack{v_j\in \mathcal{I}\\(v_i, v_j) \in E_c}}\frac{1}{\sqrt{deg(v_j)}}) < 0
\end{split}
\normalsize
\end{equation*}

Suppose that $\mathcal{C}_{min}$ avoids all situations discussed above while $v_i \in \mathcal{M}\cap V_c$ is still assigned with $c'$. Under Assumption \ref{connectivity}, there exists an interior node $v_j$ connecting with $v_i$. As we discussed, $v_j$ must be assigned $c$ in $\mathcal{C}_{min}$, leading to contradiction. 
%
Therefore, $\mathcal{C}'$ is the only choice for optimal binary classification in $G_c$. That means all nodes in class $c$ are classified correctly. But what if in $G_c$ not all nodes in $Ne_c$ are labeled $c'$? 
%
Actually no matter which labels they are assigned, all nodes in $V_c$ are classified correctly. If nodes in $Ne_c$ are assigned labels except $c$ and $c'$, the proof is almost identical and $\mathcal{C}'$ is still optimal. If any nodes in $Ne_c$ are mistakenly assigned with label $c$, the only result is to encourage nodes to be classified as $c$ correctly. 

Finally, the analysis is correct for all classes thus all nodes will be correctly classified.
\end{proof}



\section{Experiments}\label{sec:experiments}
We validate our approach using multiple datasets containing real-life data from the fields of criminal risk assessment, credit, lending, and college admissions. In each of the datasets we select a binary feature and treat it as the protected attribute (e.g., race or gender), which is the feature we require our trained classifier to behave fairly upon. Our proposed method performs well on all of these datasets, succeeding in removing unfairness almost entirely, at a very modest price in terms of accuracy.


\begin{table*}[h]
\centering
\resizebox{\textwidth}{!}{
\def\arraystretch{1.2}

\begin{tabular}{c c c | c | c | c || c | c | c || c | c | c |}

\cline{4-12}
&&&
\multicolumn{9}{ c| }{\textbf{COMPAS Dataset}}
\\ \cline{4-12}
&&&
\multicolumn{3}{ c|| }{\textbf{FPR Considerations}}&
\multicolumn{3}{ c|| }{\textbf{FNR Considerations}}&
\multicolumn{3}{ c| }{\textbf{Both Considerations}}
\\ \cline{4-12}
&&&
 $\mathbf{Acc.}$ &  $\mathbf{D_{FPR}}$ &  $\mathbf{D_{FNR}}$ &  $\mathbf{Acc.}$ &  $\mathbf{D_{FPR}}$ &  $\mathbf{D_{FNR}}$ &  $\mathbf{Acc.}$ &  $\mathbf{D_{FPR}}$ &  $\mathbf{D_{FNR}}$
\\  \cline{4-12}
\vspace*{-0.5ex}
\\ \cline{1-2} \cline{4-12}
\multicolumn{1}{ |c  }{} &
\multicolumn{1}{ c|  }{  \textbf{Our Method (AVD Penalizers)}}  &&
$\mathbf{0.660}$    &  $\mathbf{0.01}$  &  $0.04$ &
$\mathbf{0.653}$    &  $0.02$   &  $\mathbf{0.04}$ &
$\mathbf{0.654}$    &  $\mathbf{0.02}$  &  $\mathbf{0.04}$
\\ \cline{1-2} \cline{4-12}
\multicolumn{1}{ |c  }{} &
\multicolumn{1}{ c|  }{  \textbf{Our Method (SD Penalizers)}}  &&
$\mathbf{0.664}$    &  $\mathbf{0.02}$  &  $0.09$ &
$\mathbf{0.661}$    &  $0.05$   &  $\mathbf{0.03}$ &
$\mathbf{0.661}$    &  $\mathbf{0.02}$  &  $\mathbf{0.03}$
\\ \cline{1-2} \cline{4-12}
\multicolumn{1}{ |c  }{} &
\multicolumn{1}{ c|  }{  Zafar et al.~(\citeyear{disparatemistreatment})}  &&
$0.660$    &   $0.06$    &   $0.14$  &
$0.662$    &   $0.03$    &   $0.10$  &
$0.661$    &   $0.03$    &   $0.11$
\\ \cline{1-2} \cline{4-12}
\multicolumn{1}{ |c  }{} &
\multicolumn{1}{ c|  }{  Zafar et al. Baseline~(\citeyear{disparatemistreatment})}  &&
$0.643$    &   $0.03$    &   $0.11$  &
$0.660$    &   $0.00$    &   $0.07$  &
$0.660$    &   $0.01$    &   $0.09$
\\ \cline{1-2} \cline{4-12}
\multicolumn{1}{ |c  }{} &
\multicolumn{1}{ c|  }{  Hardt et al.~(\citeyear{hardt})}  &&
$0.659$    &  $0.02$    &   $0.08$  &
$0.653$    &  $0.06$   &    $0.01$  &
$0.645$    &  $0.01$   &    $0.01$
\\ \cline{1-2} \cline{4-12}
\multicolumn{1}{ |c  }{} &
\multicolumn{1}{ c|  }{  \textbf{Vanilla Regularized Logistic Regression}}  &&
$\mathbf{0.672}$    &   $\mathbf{0.20}$    &   $\mathbf{0.30}$  &
$\mathbf{0.672}$    &   $\mathbf{0.20}$    &   $\mathbf{0.30}$  &
$\mathbf{0.672}$    &   $\mathbf{0.20}$    &   $\mathbf{0.30}$
\\ \cline{1-2} \cline{4-12}
\end{tabular}
}
\vspace{3mm}
\caption{Performance comparison on the COMPAS dataset. For the approaches in bold -- Accuracy, FPR difference and FNR difference are evaluated on the test set, averaging over five runs and using a 70-30 training/test split. The performance of the remaining three approaches is stated as reported in Zafar et al.~(\citeyear{disparatemistreatment}).} \label{table:comparison_results}
\end{table*}



\begin{figure*}[b]
  \includegraphics[scale=0.6]{compas0-400.png}
  \caption{COMPAS Dataset. Accuracy, FPR difference ($\mathbf{D_{FPR}}$), and FNR difference ($\mathbf{D_{FNR}}$) (all evaluated on the test set) of the learned classifier, as a function of the weight $c=c_1 = c_2 \geq 0$ placed on the fairness penalizer terms. On the left we use the Absolute Value Difference (AVD) penalizer, and the Squared Difference (SD) penalizer on the right, both as presented in Section~\ref{regularization}. ``Relaxed FPR/FNR Diff.'' plots the value of the relevant penalization term.} %In this particular run, parameters chosen for the absolute value relaxation were: $c=80, q_c=60$, and for the squared relaxation: $c=220, q_c=30$.}
  \label{fig:compas}
\end{figure*}


\subsection{Implementation}
\textbf{Our method} 
%We instantiate our method in the following way: Given dataset $Q$, we split it randomly into a training set $S$ (which we will use for learning) and a test set $T$ (which we will only use for reporting performance). 
For the purpose of comparison with  Zafar et al.~(\citeyear{disparatemistreatment}) and Hardt et al.~\cite{hardt} on the COMPAS data, we use a parameter $c$ to induce three possible combinations of weights on the FPR and FNR penalization terms: $c = c_1$ and $c_2 = 0$; $c_1 = 0$ and $c = c_2$; and $c = c_1 = c_2$. For the other three datasets, we consider only $c = c_1 = c_2$.\footnote{The reason for varying the values of $c$ in the training phase is since we shifted to a proxy problem, in which we rely on the distance from the decision boundary rather the actual classifications. 
%Our hope is that there is no need for a worst-case cross validation between all of the combinations of $c_1, c_2, c_3$, and that the training scheme we propose is sufficient. 
It is possible, of course, that even better results are attainable using our scheme with other combinations of $c_1, c_2$, and $q$.} To explore the accuracy/fairness trade-off curve for the relaxed optimization problem~(\ref{eq:2}), we train for different values of $c$, starting at $c=0$ (which is just standard logistic regression), and growing gradually.



Given a dataset $Q$ and fixing a $d_1, d_2 \in \{0, 1\}$ of interest, we use the following training scheme:
\begin{enumerate}
\item Split $Q$ at random into training set $S$ and test set $T$.
\item For each $c$, perform cross-validation on $S$ to select the corresponding best value $q_c$ for the regularization parameter.
\item For each $(c,q_c)$, let $\theta_c = \argmin\limits_{\theta} \text{Proxy}(\theta;S,c,c,q_c)$.
\item Select $\theta^* \in \argmin\limits_{\theta_c} \text{Objective}(\theta_c;S,d_1,d_2)$.
\item Evaluate performance using $\theta^*$ on test set $T$.
\end{enumerate}
We report the average of five such runs, each with a fresh training-test split.




%We instantiate our method by solving the relaxed optimization problem~(\ref{eq:2}), in place of the original, non-convex problem~(\ref{eq:1}).  
%We test our approach with three different combinations of weights on the penalization terms:
%\katrina{What are the $d$, and how are they related to the $c$s?}
%\begin{enumerate}
%\item FPR considerations only: $d_1 = 1, d_2 = 0$.
%\item FNR considerations only: $d_1 = 0, d_2 = 1$.
%\item Both FPR, FNR considerations, assigned similar significance: $d_1 = 1, d_2 = 1$.
%\end{enumerate}
%One could, of course, pick any other combination of the FPR and FNR penalty weights.

%\katrina{I don't understand how the below is distinct from the list above}
%Learning is done by training the parameters of a logistic regressor to solve~\ref{eq:2}, while picking the value of $c_1, %c_2$ as the following:
%\begin{enumerate}
%\item FPR considerations only: $c_1 = c \geq 0$, $c_2 = 0$.
%\item FNR considerations only: $c_1 = 0$, $c_2 = c \geq 0$.
%\item Both FPR, FNR considerations, assigned similar significance: $c_1 = c_2 = c \geq 0$
%\end{enumerate}



% We then cross-validate to pick the best $c_3$ (the weight on the standard $\ell_2$-regularization term) given $c$.\footnote{The reason for varying the values of $c$ in the training phase is since we shifted to a proxy problem, in which we rely on the distance from the decision boundary rather the actual classifications. 
%Our hope is that there is no need for a worst-case cross validation between all of the combinations of $c_1, c_2, c_3$, and that the training scheme we propose is sufficient. 
%It is possible, of course, that even better results are attainable using our scheme with other combinations of $c_1, c_2, c_3$.} For each such combination, we report results as the averages of multiple \katrina{how many?} different runs, each time splitting data randomly into training and test sets.
%\yahav{We need to shorten this description.}

We solve the relaxed convex optimization problem using the CVXPY solver. Due to stability issues with large training sets, we use a train/test split of 30-70 on the larger datasets, rather than 70-30 as on the COMPAS dataset\footnote{The code implementing our method can be found at https://github.com/jjgold012/lab-project-fairness}.

%
%
%We then report the results (as evaluated on the test set) attained by a regressor $\theta \in \mathbb{R}^d$ that minimizes (on the training set $S$) a weighted combination of the $0$-$1$ loss and the differences in FPR and FNR across populations:
%\begin{equation*}
%\begin{aligned}
%&\underset{\theta}{\text{argmin}}
%& & L_{S}^{0\text{-}1}(\theta) \\
%&&& + d_1|FPR_{A=0}(\theta;S)-FPR_{A=1}(\theta;S)| \\
%&&& + d_2|FNR_{A=0}(\theta;S)-FNR_{A=1}(\theta;S)|
%\end{aligned}
%\end{equation*}
%
%\katrina{What is $d_1$ vs. $c_1$ etc.?}



%For classification, we decided use a standard cut-off threshold of $c=0.5$. There are of course, further possible interactions between the FPR, FNR considerations, and picking a certain cut-off level. These are not straightforward, since  these interactions are data-specific. 



%allows for flexibility in picking the values of $c_1, c_2$, which reflect the significance we wish to place on the objectives of achieving accuracy, equal FPR, and equal FNR. As for $c_3$, we will want to find the value of it that achieves the best results, for any combined objective of accuracy and fairness defined by a specific selection of $c_1,c_2$. Therefore, given a specific selection of $c_1, c_2$, we apply cross-validation to select the value of $c_3$. 




We briefly describe the other algorithmic approaches to which we compare:\\
\textbf{Zafar et al.}~(\citeyear{disparatemistreatment}) performs optimization by considering a proxy for the bias: the covariance between the samples' sensitive attributes and the signed distance between the feature vectors of misclassified users and the classifier decision boundary.\\
\textbf{Zafar et al. Baseline}~(\citeyear{disparatemistreatment}) tries to enforce equal FP/FN rates on the different groups by introducing different penalties for misclassified data points with different sensitive attribute values during the training phase.\\
\textbf{Hardt et al.}~(\citeyear{hardt}) performs post-processing on a standard trained (unfair) logistic regressor, picking different decision thresholds for different groups, and possibly adding randomization.


\subsection{Experimental Results}

In what follows, we use the following notation, given a trained classifier $\hat{Y}$:
\begin{align*}
\mathbf{D_{FPR}}&=\left|FPR_{A=0}(\hat{Y})-FPR_{A=1}(\hat{Y})\right| \\ 
\mathbf{D_{FNR}}&=\left|FNR_{A=0}(\hat{Y})-FNR_{A=1}(\hat{Y})\right|
\end{align*}
The values $FPR_{A=0}(\hat{Y})$, $FPR_{A=1}(\hat{Y})$, $FNR_{A=0}(\hat{Y})$, $FNR_{A=1}(\hat{Y})$ are reported as evaluated on the test set.

\paragraph{The COMPAS Dataset\footnote{https://github.com/propublica/compas-analysis}} The Correctional Offender Management Profiling for Alternative Sanctions (COMPAS) records from Broward County, Florida 2013-2014, made available online by ProPublica, are perhaps the best-studied data in the context of fairness.  The goal in this scenario is to successfully predict recidivism within two years, based on features such as age, gender, race, number of prior offenses, and charge degree. The dataset contains 5,278 samples. The protected attribute in this scenario is race, where $A$ indicates black or white. We filtered the dataset using the same features as Zafar et al.~(\citeyear{disparatemistreatment}), to allow for comparison.

%\begin{table}[h]
%\centering
%\begin{tabularx}{\columnwidth}{c|c|c|c}
%\hline
%  &  Recid. ($y = 1$)        & No Recid.  ($y = 0$)       & Total \\ \hline
%Black &  $ 1661   $ & $ 1514 $ &  $ 3175 $ \\ \hline
%White &  $ 822   $  & $1281  $ &  $ 2103 $ \\ \hline
%Total &  $ 2483  $  & $2795 $ &  $ 5278 $ \\\hline
%\end{tabularx}
%\caption{Statistics of the ProPublica COMPAS data.} \label{table:compas-stats}
%\label{tab:stats}
%\end{table}
%\vspace{-1em}

%\begin{table}[h]
%\centering
%\begin{tabularx}{\columnwidth}{c|c}
%\hline
%Feature  &  Description \\ \hline
%Age Category &  $<25$, between $25$ and $45$, $>45$ \\
%Gender &  Male or Female \\
%Race &  White or Black \\
%Priors Count &  0--37 \\
%Charge Degree &  Misconduct or Felony \\
%\hline
%2-year-recid. & Whether or not the  \\
%(target feature)  & defendant recidivated within two years
%\end{tabularx}
%\caption{Description of features used from ProPublica COMPAS data.} \label{table:compas-features}
%\label{tab:features}
%\end{table}




\begin{table*}[t]
\centering
\caption{A description of the datasets used, along with parameters of the training procedure used for each.}
\label{table:datasets_description}
\begin{adjustbox}{max width=\textwidth}
\begin{tabular}{|l|l|l|l|l|l|l|l|}
\hline
\textbf{Dataset} & \textbf{No. Samples} & \textbf{No. Features} & \textbf{Train/Test Split} & \textbf{No. Repetitions} & \textbf{No. Folds in CV} & \textbf{Protected Feature} & \textbf{Target Variable} \\ \hline
COMPAS           & 5,278                     & 5                          & 70-30                     & 5                        & 5                                 & Race                       & 2-Year-Recidivism        \\ \hline
Adult            & 30,162                    & 10                         & 30-70                     & 5                        & 5                                 & Gender                     & Income Over/Under 50K    \\ \hline
Default          & 30,000                    & 23                         & 30-70                     & 5                        & 3                                 & Gender                     & Defaulting On Payments   \\ \hline
Admissions       & 20,839                    & 17                         & 30-70                     & 5                        & 3                                 & Race                       & Passing Bar Exam         \\ \hline
\end{tabular}
\end{adjustbox}
\end{table*}


\begin{table*}[t]
\centering
\resizebox{\textwidth}{!}{
\def\arraystretch{1.2}

\begin{tabular}{c c c | c | c | c || c | c | c || c | c | c |}

\cline{4-12}
&&&
\multicolumn{3}{ c|| }{\textbf{Adult Dataset}}&
\multicolumn{3}{ c|| }{\textbf{Default Dataset}}&
\multicolumn{3}{ c| }{\textbf{Admissions Dataset}}
\\ \cline{4-12}
%&&&
%\multicolumn{3}{ c|| }{\textbf{Both Considerations}}&
%\multicolumn{3}{ c|| }{\textbf{Both Considerations}}&
%\multicolumn{3}{ c| }{\textbf{Both Considerations}}
%\\ \cline{4-12}
&&&
 $\mathbf{Acc.}$ &  $\mathbf{D_{FPR}}$ &  $\mathbf{D_{FNR}}$ &  $\mathbf{Acc.}$ &  $\mathbf{D_{FPR}}$ &  $\mathbf{D_{FNR}}$ &  $\mathbf{Acc.}$ &  $\mathbf{D_{FPR}}$ &  $\mathbf{D_{FNR}}$
\\  \cline{4-12}
\vspace*{-0.5ex}
\\ \cline{1-2} \cline{4-12}
\multicolumn{1}{ |c  }{} &
\multicolumn{1}{ c|  }{  \textbf{Our Method (AVD Penalizers)}}  &&
$\mathbf{0.776}$    &  $\mathbf{0.00}$  &  $\mathbf{0.04}$ &
$\mathbf{0.807}$    &  $\mathbf{0.00}$   &  $\mathbf{0.01}$ &
$\mathbf{0.950}$    &  $\mathbf{0.01}$  &  $\mathbf{0.00}$
\\ \cline{1-2} \cline{4-12}
\multicolumn{1}{ |c  }{} &
\multicolumn{1}{ c|  }{  \textbf{Our Method (SD Penalizers)}}  &&
$\mathbf{0.783}$    &  $\mathbf{0.00}$  &  $\mathbf{0.09}$ &
$\mathbf{0.806}$    &  $\mathbf{0.01}$   &  $\mathbf{0.02}$ &
$\mathbf{0.950}$    &  $\mathbf{0.00}$  &  $\mathbf{0.00}$
\\ \cline{1-2} \cline{4-12}
\multicolumn{1}{ |c  }{} &
\multicolumn{1}{ c|  }{  \textbf{Vanilla Regularized Logistic Regression}}  &&
$\mathbf{0.800}$    &   $\mathbf{0.08}$    &   $\mathbf{0.39}$  &
$\mathbf{0.807}$    &   $\mathbf{0.01}$    &   $\mathbf{0.05}$  &
$\mathbf{0.951}$    &   $\mathbf{0.16}$    &   $\mathbf{0.02}$
\\ \cline{1-2} \cline{4-12}
\end{tabular}
}
\vspace{3mm}
\caption{Performance on the Adult, Loan Default, and Admissions datasets, penalizing for both FPR and FNR difference. Accuracy, FPR difference and FNR difference are evaluated on the test set, averaging over five runs and using a 30-70 training/test split.} \label{table:comparison_results_rest}
\end{table*}


In Table~\ref{table:comparison_results}, we compare the performance of our approach with that of three other techniques from the literature. Each method was trained based on logistic regression.  As a basis for comparison, we also present the performance of vanilla logistic regression, absent fairness considerations, with the regularization parameter selected via cross-validation.\footnote{Zafar et al.~(\citeyear{disparatemistreatment}) do not incorporate regularization in any of the approaches they report.}
%Results are reported as the averages of 5 different runs \katrina{Is that still correct?}, each time splitting data evenly and randomly into training and test sets. 
Results for Zafar et al., Zafar et al. baseline, and Hardt et al. appear here as reported in Zafar et al.~(\citeyear{disparatemistreatment}).\footnote{Our method selects the classifier based on the training set only and reports its performance over the test set. Results for the three other approaches, reported by Zafar et al.~(\citeyear{disparatemistreatment}), are based on tuning parameters after seeing the trade-off curve over the test set, and reporting according to the best selection of these parameters.}
%\katrina{Perhaps here is the right place for a footnote about the discrepancy with the Zafar baseline}

We find that the vanilla logistic regressor (absent fairness considerations) results in significant unfairness, as $\mathbf{D_{FPR}}=0.20$, and $\mathbf{D_{FNR}}=0.30$. The overall accuracy of this classifier measured on the test set was $0.672$.\footnote{Zafar et al.~(\citeyear{disparatemistreatment}) report a slightly different baseline of: Accuracy = 0.668, $\mathbf{D_{FPR}}=0.18$, $\mathbf{D_{FNR}}=0.30$.} Our SD penalization approach empirically achieves approximately the same accuracy as the Zafar et al.~(\citeyear{disparatemistreatment}) approach, with significantly better fairness. It is difficult to compare fairness-accuracy tradeoffs with the Hardt et al.~(\citeyear{hardt}) approach, since their accuracy is significantly lower than ours. A more direct comparison is possible by noting that our learned classifier can be post-processed to improve its fairness at a direct cost to accuracy. Hence, we can achieve accuracy of $0.659$ with $\mathbf{D_{FPR}} = \mathbf{D_{FNR}} = 0.01$, which compares very favorably with the Hardt et al. accuracy rate of 0.645 given the same FPR and FNR rates.\footnote{For completeness, we note that using a 50-50 training-test split (again not using the test set for parameter selection), our method (SD, both considerations) produces a classifier that provides: Accuracy = 0.659, $\mathbf{D_{FPR}} = 0.01, \mathbf{D_{FNR}} = 0.05$. This classifier can be post-processed to achieve rates of: Accuracy = 0.655, $\mathbf{D_{FPR}} = \mathbf{D_{FNR}} = 0.01$.}

Figure \ref{fig:compas} illustrates the accuracy/fairness trade-offs achievable using our scheme. Increasing the weight $c$ on the proxy fairness penalizers results in reducing their magnitude. The figure also illustrates how our relaxed penalizers succeed in tracking the real FPR and FNR differences. 
%
%
%\katrina{Must rewrite the following paragraph}
%We observe that our method succeeds in eliminating unfairness almost completely on the COMPAS dataset, while retaining most of the accuracy, when compared to the vanilla logistic regression. We achieve very low difference rates when penalizing for achieving each of the FPR and FNR criteria individually, and also for both. We achieve preferable results comparing to Zafar et al. and Zafar et al. baseline in all 3 scenarios, and also comparing to Hardt et al. in the settings of false positive/false negative considerations only. In the setting of both considerations - The Hardt et al. method removes a larger portion of the unfairness, however it results in major accuracy loss as it achieves accuracy rate of 0.645 in comparison to our method which results in accuracy of 0.665, retaining most of the original accuracy rate while removing most of the unfairness.




%The Hardt et al.~\cite{hardt} approach as reported removes a smaller portion of the bias in the different scenarios, however for FP/FN constraints alone, it provides higher accuracy rates. The Zafar et al.~(\citeyear{disparatemistreatment}) approach as reported retains significant bias (in most cases), but in some cases  achieves slightly superior accuracy rates to the methods above. 

%These performance comparisons are incomplete in the sense that each of the compared techniques has the potential to trade off between accuracy and fairness, using some degree of parameter tuning; what we report here is only one point on the achievable trade-off frontier for each algorithm. The ``correct'' trade-off, and, in particular, the best manner in which to weigh unfairness in the FPR against unfairness in the FNR, are matters of opinion. We have chosen to report our method's performance under parameters designed to very aggressively mitigate unfairness, at some cost to the accuracy.

%It would certainly be desirable to evaluate these and other approaches to fair learning on other datasets and on different tasks, particularly on larger datasets, which might afford both greater accuracy and better bias-reduction. The present empirical evaluations, however, suggest that our regularization-based approach provides a new tool worthy of consideration---we succeed in almost entirely eliminating bias on the hold-out set, at a modest price in terms of accuracy.

%Due to the fact that our true objective includes the original non-convex penalization terms, our approach does not carry any formal guarantees. However, the ease of implementation, generality, and empirical results are encouraging. Figure~\ref{fig:test1} illustrates the rate of convergence to a fair, accurate classifier on this dataset.
%In terms of computation costs, given that at each iteration we must calculate the gradient according to the FPR and FNR regularizers, we are required to predict the labels for the entire training set at each step. 
%However, this does not pose a computational burden, as it is already required by the (classic) gradient descent algorithm in our logistic regressor fitting scheme. Furthermore, when given a sufficiently large dataset (one or two orders of magnitude larger than the one currently available for the COMPAS scores data), this could be relaxed to sampling only a mini-batch of samples from the training data set at each iteration (much as is done in stochastic gradient descent).






\subsection{Additional Datasets}


Table~\ref{table:datasets_description} provides summary statistics on each of the datasets on which we tested our approach. We also briefly describe the datasets below. 


{\bf The Adult Dataset}\footnote{http://archive.ics.uci.edu/ml/datasets/Adult} is based on 1994 US Census data. The task we consider is to predict whether the income of each individual is over or under 50K dollars per year, based on features such as occupation, marital status, and education. The protected attribute selected in this task is gender. 

{\bf The Loan Default Dataset}\footnote{{\scriptsize https://archive.ics.uci.edu/ml/datasets/default+of+credit+card+clients}}
contains data regrading Taiwanese credit card users. The task we consider is to predict whether an individual will default on payments, based on features such as history of past payments, age, and the amount of given credit. The protected attribute is gender.

{\bf The Admissions Dataset}\footnote{http://www2.law.ucla.edu/sander/Systemic/Data.htm}
contains records of law school students who went on to take the bar exam. The task we consider is to predict whether a student will pass the exam based on features such as LSAT score, undergraduate GPA, and family income. The protected attribute is set to race.

Table~\ref{table:comparison_results_rest} describes the performance of our approach on these datasets, and Figures~\ref{fig:adult},~\ref{fig:default}, and~\ref{fig:lawschool} illustrate the fairness-accuracy trade-offs we achieve in each context. Overall, we see that unfairness is nearly eliminated while accuracy remains quite high. The dataset on which accuracy suffers most under our approach is the Adult dataset, which is also the dataset on which the vanilla regression is the most unfair.


\begin{figure*}[]
  \includegraphics[scale=0.6]{adult0-800.png}
  \caption{Adult Dataset. Fairness-Accuracy tradeoffs, as in Figure~\ref{fig:compas}.}
  \label{fig:adult}  
\end{figure*}



\begin{figure*}[]
  \includegraphics[scale=0.6]{default0-50.png}
  \caption{Loan Default Dataset. Fairness-Accuracy tradeoffs, as in Figure~\ref{fig:compas}.}
  \label{fig:default}
\end{figure*}



\begin{figure*}[]
  \includegraphics[scale=0.6]{admissions0-400.png}
  \caption{Admissions Dataset. Fairness-Accuracy tradeoffs, as in Figure~\ref{fig:compas}.}
  \label{fig:lawschool}
\end{figure*}




\begin{comment}
\begin{figure}
\includegraphics[width=\linewidth]{figs/beyond_tss_lesion.pdf}
\caption[]{End-to-End runtime lesion study of the entire MNIST dataset and the FMA featurized music dataset. Each of DROP's contributions provides a runtime improvement.}
\label{fig:beyond_lesion}
\end{figure}
\end{comment}



\section{Conclusion}
\label{sec:conclusion}

Advanced data analytics techniques must scale to rising data volumes. 
DR techniques offer a powerful toolkit when processing these datasets, with PCA frequently outperforming popular techniques in exchange for high computational cost. 
In response, we propose DROP, a new dimensionality reduction optimizer. 
DROP combines progressive sampling, progress estimation, and online aggregation to identify high quality low dimensional bases via PCA without processing the entire dataset by balancing the runtime of downstream tasks and achieved dimensionality. 
Thus, DROP provides a first step in bridging the gap between quality and efficiency in end-to-end DR for downstream \red{analytics}. 

%We revisit canonical operators for time series dimensionality reduction and the measurement study of~\cite{keogh-study}, and show that PCA is more effective than popular alternatives in the data mining literature often by a margin of over $2\times$ on average on gold-standard time series benchmark data sets with respect to output data dimension. More surprisingly, we empirically demonstrate that a small number of samples are sufficient to accurately characterize directions of maximum variance and obtain a high-quality low-dimensional transformation.



\bibliographystyle{ACM-Reference-Format}
\bibliography{ref}

\end{document}
