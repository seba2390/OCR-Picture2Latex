\section{Related Work}\label{sec:related}
N-ary joinable table search is not a well-researched area and a very limited number of works have focused on multi-attribute joins discovery. Li et al.'s index design~\cite{li2015efficient} is one of these works. They introduce a prefix tree index to calculate the joinability score between two given tables for n-ary joins. 
The drawback of this approach is that it is not scalable to data lakes. The prefix tree-based approach assumes that the one-to-one mapping between the composite key columns and the columns in the candidate tables is apriori known. 
\system, on the other hand, leverages an index structure that does not require the user to provide the mappings. 

There is a large number of research papers that address joinable tables discovery problem based on unary keys~\cite{zhu2019josie, xiao2009top, eberius2015top, das2012finding, yakout2012infogather, zhang2013infogather+, fernandez2018aurum, DBLP:journals/pvldb/ZhuNPM16, DBLP:conf/sigmod/ZhangI20}. 
\system extends the standard single-attribute inverted index used in these state-of-the-art systems and applies a fast single-operation filtering approach to be able to detect joinable table rows without actual value comparison. Any single attribute inverted index can be extended with the super key used in \system.

For spatial indices, the goal of an ideal n-ary join discovery is to map multiple dimensions into an easy to search index. However, spatial index structures are a very special case compared to the problem we tackle in this paper. The spatial indices e.g. KD tree~\cite{bentley1975multidimensional}, KDB tree~\cite{robinson1981kdb}, R+ tree~\cite{sellis1987r+}, Geohash~\cite{geohash}, grid files~\cite{nievergelt1984grid}, suffer from three drawbacks: 1) the indexed data points always have fixed dimensions, such as latitude and longitude. 2) These dimensions have a fixed order. For instance, latitude is always compared to the latitude of the other points. In other words, there is a pre-defined mapping between the dimensions. However, in join discovery, key values can appear in arbitrary columns and comparing the right values is not a straightforward task. 3) These indices are designed to work with numerical values. 
\system on the other hand does not have any of the aforementioned assumptions. It discovers the joinable table rows with an arbitrary number of columns. Also, \system leverages \hash and discovers the key values regardless of their position in the candidate rows. Besides, \hash is applicable for numerical and alphabetical column values.

Data enrichment solutions attempt to join additional tables to improve the downstream machine learning accuracy. Data enrichment solutions also either require pre-defined mappings between joinable columns~\cite{chepurko13arda, kumar2016join}, which needs the user input on every single external table, or they focus on single column joins~\cite{zhu2019josie, xiao2009top, eberius2015top, das2012finding, esmailoghli2021cocoa}. \system calculates the joinability of the candidate tables without any apriori knowledge, e.g. mapping information. This allows \system to scale to any number of external tables.

Inclusion dependency (IND) discovery~\cite{de2009unary, papenbrock2015divide, koeller2003discovery} is a research direction where the goal is to discover foreign keys in the database~\cite{papenbrock2015divide}. The focus is on identifying fitting column combinations across all datasets. In data discovery, we are interested in the top-k tables with the most fitting INDs.