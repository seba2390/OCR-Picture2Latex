%!TEX root = ../main.tex

\section{System Overview} \label{sec:system}
\begin{figure}
% \hspace{-.8cm}
    \center{\includegraphics[scale=0.19]
          {figures/architecture.pdf}}
            %   \vspace{-.2cm}
    \caption{The overall workflow of \system.}
    \label{fig:architecture}
\end{figure}
Figure~\ref{fig:architecture} depicts the abstract workflow of our proposed system, \system. %It consists of two main phases: the {\em offline} phase and the {\em online} phase.
In the offline phase, \system builds an inverted index structure for the tables in the corpus. 
In the online phase, i.e. discovery phase, \system uses the inverted index to find the top-k joinable tables among all candidate tables for a given input table. 

% \vspace{0.1cm}
\noindent\textbf{Indexing step (Offline).} To efficiently discover tables that are joinable to a given table, we propose an extension to the state-of-the-art single-value inverted index to efficiently serve the multi-attribute applications.
We introduce an additional index element, which is an aggregated hash value called \emph{super key}.
The \emph{super key} is space-efficient and does not change the nature of the single-attribute inverted index while serving the purpose of multi-attribute join discovery.
The super key is used to prune irrelevant tables and rows to reduce the post-processing overhead. It aggregates the values inside each table-row into a fixed-size entry.

It is worth noting that the super key element does not require any knowledge of the actual keys of a table and serves all possible key combinations inside a table.
This way, for a given query dataset and composite key, the system can decide in a single operation if the row is a candidate joinable row or not.
Furthermore, the filter does not result in any false negatives.
To generate the super key entries, \system leverages a novel hash function \hash that encodes each row value in a highly disseminative way and aggregates them to the super key. The hash function, as well as the super key, will be discussed in more detail in Section~\ref{sec:index}.

% \vspace{0.1cm}
\noindent\textbf{Discovery step (Online).} In the actual data discovery scenario, the user provides a dataset with a composite key and a parameter $k$, expecting the system to find top-$k$ tables with the highest number of equi-joinable rows to the given input dataset.
\system enables the efficient n-ary key joins by using the super key entry to prune as many irrelevant tables as possible before the joinability calculation.
This process undergoes four phases: \textit{(i)}~initialization, \textit{(ii)}~table filtering, \textit{(iii)}~ row filtering, and \textit{(iv)}~the joinability calculation.

In the initialization step, first, the system selects an initial query column from the composite key \textit{Q} based on a simple cardinality-based heuristic. The goal of the initial query column is to reduce the number of fetched tables to the minimum by picking the column that leads to the smaller number of PL items from the corpus. 
That is why \system selects a single column to fetch the initial set of candidate tables from the corpus.
The column selection can be supervised and preempted by the user.
Then, the system generates the super key entry for the join columns of the input dataset, i.e.,~generating a hash-code for each key value combination and aggregating them into a single hash. 
In the filtering step, \system applies two levels of pruning for each table: table-level and row-level. 
With the table-level pruning, \system decides whether the current table is a promising table to be one of the top-k tables.
With the row-level pruning, \system checks for non-joinable rows in the candidate table. It compares the super keys of the input dataset with those inside the candidate table to drop irrelevant rows from further joinability verification.
Finally, \system retrieves the exact values from the table corpus to compute the final joinability score $\jmath$ for the remainig tables and rows. After calculating the $\jmath$, the set of top-$k$ tables is updated and the next candidate table undergoes the pruning steps.
We detail the online phase in Section~\ref{sec:index_applicaiton}.

