%!TEX root = ../main.tex

\section{Problem Statement}\label{sec:problem_statement}

We focus on discovering joinable tables based on composite key joins.
Generally speaking, the problem of table discovery is to find the top-k joinable tables for a given query table with a selected composite key~\cite{zhu2019josie}.
We first formalize the joinability between two tables and then define the problem of n-ary join discovery from large data lakes.
We borrow the notations from the literature on inclusion dependencies~\cite{papenbrock2015divide, de2009unary}.

\noindent\textbf{Joinability.} 
Intuitively, the joinability between a candidate table (from a corpus) and a given query table represents the corresponding equi-join cardinality.
That is, the more key values in a candidate table can be joined with the query table the higher their joinability.
Thus, joinability is a measure for the completeness of the join and the relevance of a candidate table to a query table.
Formally, assume that $\mathcal{R}$ and $\mathcal{S}$ are two relational schemata and attribute sets $X$ and $Y$ are two subsets of columns where, $X \subseteq \mathcal{R}$ and $Y \subseteq \mathcal{S}$. Without loss of generalization, we can pick any $X$ and $Y$ that consist of the same number of attributes: $|X| = |Y| = m$.
In multi-attribute joins $m > 1$. If $r$ is a set of tuples over $\mathcal{R}$, the projection of $\mathcal{R}$ onto $X$ is shown by $\pi_X(\mathcal{R})$, where $\pi_X(\mathcal{R}) = \{t[X]|t \in r\}$. 
Likewise, $s$ is a set of rows over $\mathcal{S}$.
We, thus, define the joinability score $\jmath$ between $\mathcal{R}$ and $\mathcal{S}$ on the selected column sets $X$ and $Y$ as:
\begin{equation}\label{eq:joinability}
\small
    \jmath(\mathcal{R},\ \mathcal{S}) = |\pi_X(\mathcal{R}) \cap \pi_Y(\mathcal{S})|.
\end{equation}
Yet, calculating Equation~\ref{eq:joinability} is not possible because the one-to-one mapping between the key columns in $\mathcal{R}$ and $\mathcal{S}$ is unknown.
Any column permutation of size $|X|$ is a possible candidate.
Thus, the joinability between a table with a given composite join key and a table without a defined join key is a factorial number of possible mappings in the number of join key columns.
This is why we extend the joinability definition as follows:
\begin{equation}
\small
    \jmath(\mathcal{R},\ \mathcal{S}) = \argmax_{Y'} {|\pi_X(\mathcal{R}) \cap \pi_{Y'}(\mathcal{S})|}.
\end{equation}
Here, $Y'$ is a permutation of size $|X|$ from $\mathcal{S}$  where $\jmath(\mathcal{R},\ \mathcal{S})$ is maximum.

\begin{figure}
    \center{\includegraphics[scale=0.14]
          {figures/example_table.pdf}}
% \vspace{-.5cm}
	\caption{Running example.}
    \label{fig:example}
\end{figure}

% \vspace{0.1cm}
\noindent{\em Running example.}
Consider a query table $d$ and a candidate table $T_1$ as illustrated in Figure~\ref{fig:example}.
Assume the user has selected \textit{F. Name}, \textit{L. Name}, and \textit{Country} as the query columns (columns with blue header).
We aim at finding the three columns from table $T_1$ that result in the highest joinability ($\jmath$) with the query columns from $d$.
If we map \textit{F. Name} from $d$ to \textit{Nachname} from $T_1$, map \textit{L. Name} to \textit{Vorname}, and map \textit{Country} to \textit{Land}, the one-to-one mapping would lead to a $\jmath$ of $0$;
If we map \textit{F. Name} from $d$ to \textit{Vorname} from $T_1$, map \textit{L. Name} to \textit{Nachname}, and map \textit{Country} to \textit{Land}, we would obtain $\jmath=5$, the maximum joinability score among all possible mappings.

% \vspace{0.1cm}
\noindent\textbf{The n-ary join discovery problem.} Given a base table $d$ with a relation $D$,
a set of query columns $Q$, where $Q \subset D$, a corpus of tables $T$, and a constant value $k$, the goal is to return the top-$k$ tables from $T$ sorted by their joinability $\jmath$.
Solving this problem is challenging for two main reasons:
\begin{packed_enum}
    \item Calculating the joinability between a given query table and a candidate table with a set of attribute $T'$ requires the mapping between $Q$ and $T'$ that maximizes $\jmath$.
    The number of possible mappings is calculated as:
        \begin{equation}
        \small
           P(|T'|, |Q|) = \frac{|T'|!}{(|T'| - |Q|)! |Q|!}
        \end{equation}
        Here, $P(|T'|,\ |Q|)$ represents the number of possible permutations with size $|Q|$ out of $|T'|$ columns.
    \item Ranking tables based on $\jmath$ and picking the top-$k$ requires calculating the joinability on each candidate table in $T$. 
\end{packed_enum}


To discover n-ary joinable tables, one has to (i)~build a multi-attribute inverted index that maps different combinations of cell values to their locations in the tables, or (ii)~use the inverted index for unary joins and tolerate a large number of FPs.
Building a multi-attribute inverted index is not feasible with respect to the storage complexity.
For each of the 145M tables inside the Dresden Webtable Corpus, one would need to create $\sum_{i=1}^{c} P(c,\ i)$ indexes per table.
For $c=5$, the size of the database would increase by more than one order of magnitude.
We propose a filtering solution that extends the single-attribute inverted index to obtain both time and space efficiency.
