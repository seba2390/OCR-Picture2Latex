%!TEX root = ../main.tex

\section{Index Update }
There are three possible types of edits on table corpora that lead to updates in the index-level: {\em insert}, {\em update}, and {\em delete}.

Inserting a new table to the corpus requires the following updates.
Other than generating the PL items for the cell values in the newly added table, a super key is generated for each row. 
Inserting a new row to an existing table also follow the same procedure. 
Adding a new column to an existing table requires to apply \hash on each individual column value and replace the corresponding super key by the result of a bitwise-OR operation with the new \hash result.
Data updates also affect the super keys of affected tables.
For instance, updating the value of a cell requires a complete re-hash of the corresponding super key.
As the previous value is deleted, we cannot restrict the update to a bit-wise \texttt{OR} operation with the new hash value.
Deleting a table/row only requires deleting the PL items for the table/row.
Yet, deleting a column from a table might change the super key entries, triggering a rehashing of all rows.

Although some of the aforementioned updates require regeneration of the super key, the system can handle the changes locally to the affected table.
Generally, the most frequent index updates in a data lake would be \textit{table inserts}, which do not affect any of the already generated index values.
