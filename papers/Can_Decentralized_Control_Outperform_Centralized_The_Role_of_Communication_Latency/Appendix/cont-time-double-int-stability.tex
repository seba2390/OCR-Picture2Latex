%!TEX ROOT = ../centralized_vs_distributed.tex

\subsection{Proof of~\cref{prop:cont-time-double-int-stability}}\label{app:cont-time-double-int-stability}

%Let us consider the following generic scalar stochastic retarded linear system driven by a standard Brownian motion:
%\begin{equation}\label{eq:2nd-order-diff-eq}
%\dfrac{d^2\x{}{t}}{dt^2} + \gvel\dfrac{d\x{}{t}}{dt} + \gvel\gpos\x{}{t-1} = \dfrac{d\noise{}{t}}{dt}
%\end{equation}
The error dynamics equation with agent model~\eqref{eq:cont-time-double-int-model} reads
\begin{equation}\label{eq:multi-agent-state-space}
	\begin{array}{c}
		d\x{}{t} =
			\left(A_0\x{}{t} + A_1\x{}{t-1}\right)dt + Bd\noisebar{}{t}, \\[10pt]
		A_0 = \begin{bmatrix}
			0 & I\\
			0 & -\gvel I
		\end{bmatrix}, \
		A_1 = \begin{bmatrix}
			0 & 0\\
			-\gvel K & 0
		\end{bmatrix}, \
		B = \begin{bmatrix}
			0\\
			I
		\end{bmatrix},
	\end{array}
\end{equation}
with $ \noisebar{}{t} $ standard $ N $-dimensional Brownian motion.
The decoupling~\eqref{eq:agent-dynamics-1}
is obtained from~\eqref{eq:multi-agent-state-space} through
the change of basis $ \x{}{t} = (T\otimes I_2)\xtilde{}{t} $.
Rewriting~\eqref{eq:agent-dynamics-1} as a double integrator in state-space form 
with state $ \tilde{s}_j(\cdot) $ yields
\begin{equation}\label{eq:2n-order-system-state-space-1}
	\begin{array}{c}
		d\tilde{s}_j(t) = \left(F_0\tilde{s}_j(t)+F_{1j}\tilde{s}_j(t-1)\right)dt + Gd\noisebar{j}{t}, \\[5pt]
		 F_0 = \begin{bmatrix}
			0 & 1\\
			0 & -\gvel
		\end{bmatrix}, \ F_{1j} = \begin{bmatrix}
			0 & 0\\
			-\gvel\gpos_j & 0
		\end{bmatrix}, \ G = \begin{bmatrix}
			0\\
			1
		\end{bmatrix},
	\end{array}
\end{equation}
Stability of~\eqref{eq:multi-agent-state-space}
is equivalent to that of~\eqref{eq:2n-order-system-state-space-1} for all $ j $.
In the following, we drop the subscript $ j $ for the sake of readability.
%The first subsystem is stable for any $ \gvel > 0 $.
For positive eigenvalues $ \gpos $,~\eqref{eq:2n-order-system-state-space-1} is mean-square asymptotically stable
if $ \alpha_0 < 0 $ and unstable if $ \alpha_0 > 0 $~\cite{wangBoundedness}, where the \emph{spectral abscissa} is defined as
\begin{equation}\label{eq:2nd-order-system-stability-condition}
	\alpha_0 \doteq \sup\left\lbrace\Re(z) : z\in \mathbb{C}, \ h(z) = 0 \right\rbrace,
\end{equation}
and the \emph{characteristic polynomial} of~\eqref{eq:2n-order-system-state-space-1} is
\begin{equation}\label{eq:2n-order-system-chacteristic-polynomial}
	\begin{aligned}
		h(z) &\doteq \det\left(zI - F_0 - F_{1}\e^{-z}\right) = z^2 + \gvel z + \gvel\gpos\e^{-z}.  % = z^2 + \gvel z + \gvel\gpos\e^{-z}
%			 &= (z^2+\gvel z)\e^{z} + \gvel\gpos
	\end{aligned}
\end{equation}
A sufficient and necessary condition for all roots of $ h(z) $ to lie in the open left-hand half-plane is derived in~\cite{BAPTISTINI1997259}.
%and rewritten below for the sake of convenience.
\begin{thm}[\!\!\protect{\cite[Theorem 2.1]{BAPTISTINI1997259}}]\label{thm:stable-roots}
	Let the 2-vectors $v(b)=\left(p b, q-b^{2}\right), w(b)=$ $(\cos b, \sin b), b \geq 0,$ be given. If $r>0,$ a necessary and sufficient condition for all roots of the equation $h(z)=(z^2+pz+q)\e^{z} + r=0$ to have negative real part is that the orthogonality condition $v(b) \cdot w(b)=0,$ with $b \in \cup_{k=0}^{\infty}(2 k \pi,(2 k+1) \pi),$ implies $|v(b)|>r$.
\end{thm}
%In virtue of the above result, 
From~\cref{thm:stable-roots},~\eqref{eq:2n-order-system-state-space-1}
is asymptotically stable if the following implication holds for $b \in \cup_{k=0}^{\infty}(2 k \pi,(2 k+1) \pi)$,
\begin{equation}\label{eq:stability-condition-1}
	\gvel b\cos b - b^2\sin b = 0 \implies \gvel^2b^2+b^4>\gvel^2\gpos^2.
\end{equation}
In view of $ b \ge 0 $ and $ \sin b \ge 0 $,~\eqref{eq:stability-condition-1}
leads to~\eqref{eq:cont-time-double-int-stability-condition} after standard algebraic manipulations,
where we replace $ b $ with $ \beta = \min b \in (0,\nicefrac{\pi}{2}) $.
The inequality can be rewritten as
\begin{equation}\label{eq:stability-condition-lambda}
	\gpos < \dfrac{\beta}{\sin\beta} \doteq \phi(\gvel),
\end{equation}
where the definition of $ \phi(\cdot) $ follows from the implicit function theorem
applied to $ F(\gvel,\beta) \doteq \beta\tan\beta - \gvel $,
which states that $ F(\gvel,\beta) = 0 $ if and only if $ \beta = \varphi(\gvel) $ and
%$ \varphi'(\cdot), \varphi''(\cdot) $ can be explicitly computed, with
\begin{equation}\label{eq:varphi-of-eta-derivative}
	\varphi'(\gvel) = \dfrac{\cos^2\left(\varphi(\gvel)\right)}{\varphi(\gvel) + \sin\left(\varphi(\gvel)\right)\cos\left(\varphi(\gvel)\right)}
\end{equation}
Tedious but straightforward calculations on the first and second derivatives %$ \phi'(\gvel) $ and $ \phi''(\gvel) $
show that $ \phi(\gvel) $ is concave increasing for any $ \gvel > 0 $.
The limits at $ 0 $ and $ +\infty $ can be easily computed
by noting that
\begin{equation}\label{eq:stability-condition-limits}
	\beta_0\doteq\varphi(0) = 0, \quad \beta_\infty\doteq\lim_{\gvel\rightarrow+\infty}\varphi(\gvel) = \dfrac{\pi}{2}.
\end{equation}




