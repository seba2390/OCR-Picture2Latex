%!TEX ROOT = ../centralized_vs_distributed.tex

\subsection{\titlecap{variance computation for discrete-time systems}}\label{app:disc-time-single-int-variance-explicit}

\myParagraph{Wiener--Kintchine Formula}
Given any fixed values of delay and feedback gains,
the steady-state variance $ \varx{\gpos}{I} $ or $ \varx{\gvel,\gpos}{II} $ of the decoupled subsystems can be computed numerically by
\begin{equation}\label{eq:disc-time-variance-integral}
	\dfrac{1}{2\pi}\int_{-\pi}^{+\pi}\dfrac{d\theta}{|h(\e^{j\theta})|^2},
\end{equation}
where the characteristic polynomial $ h(z) $
is~\eqref{eq:disc-time-single-int-characteristic-polinomial} or~\eqref{eq:disc-time-double-int-characteristic-polinomial}.

\myParagraph{Single Integrator Model}
The moment-matching method applied to subsystem~\eqref{eq:disc-time-single-int-decoupled} yields 
a linear system of equations in the variables $ (\rho_0,...,\rho_\delayn) $,
where $ \rho_t \doteq \mathbb{E}[\xtilde{}{k}\xtilde{}{k\pm t}] $:
\begin{subequations}\label{eq:disc-time-single-int-moment-matching-eqs}
	\begin{align}
		\rho_0 &= \mathbb{E}[\xtilde{}{k+1}^2] = \rho_0 + \gpos^2\rho_0 + 1 - 2\gpos\rho_\delayn \label{eq:disc-time-single-int-first-moment-eq}\\
		\rho_1 &= \mathbb{E}[\xtilde{}{k+1}\xtilde{}{k}] = \rho_0 - \gpos\rho_\delayn \label{eq:disc-time-single-int-yule-walker-eqs-1}\\
		&\hspace{2mm}\vdots \nonumber \\
		\rho_\delayn &= \rho_{\delayn-1} - \gpos\rho_1, \label{eq:disc-time-single-int-yule-walker-eqs-2}
	\end{align}
\end{subequations}
where~\eqref{eq:disc-time-single-int-yule-walker-eqs-1}--\eqref{eq:disc-time-single-int-yule-walker-eqs-2} are the Yule-Walker equations.
%associated to the decoupled subsystem~\eqref{eq:disc-time-single-int-decoupled}.
System~\eqref{eq:disc-time-single-int-moment-matching-eqs}
can be written compactly as $ A^{(\tau)}\rho = e_1 $, 
where $ \rho^\top = [\rho_0,\dots,\rho_{\delayn}]$,
$ e_1 $ is the canonical vector in $ \Real{\delayn+1} $ with nonzero first coordinate,
and $ A^{(\tau)}\in\Real{(\delayn+1)\times(\delayn+1)} $ gathers all coefficients of equations in~\eqref{eq:disc-time-single-int-moment-matching-eqs}.
%with
%\begin{equation}\label{eq:explicit-variance-matrix-A}
%A^{(\tau)} = \begin{bmatrix}
%-\gpos^2 &   		&     		& 		 &   	  & 2\gpos\\
%1 		 & -1 		&     		& 		 &    	  & -\gpos\\
%& 	\ddots	& \ddots	&  		 & \iddots&  \\
%& 			& 			&		 & 		  &  \\
%& 			& 			&		 & 		  &  \\
%& -\gpos	& 			& 		 & 1 	  & -1
%\end{bmatrix}.
%\end{equation}
%In particular, when $ \delayn $ is odd, the $ (\ceil{\nicefrac{\delayn}{2}} + 1) $-th row is
%\begin{equation}\label{eq:app-explicit-variance-matrix-A-mid-row-odd}
%	\left[\!\begin{array}{cccccccc}
%	0 & \dots & 0 & 1 & -1-\gpos & 0 & \dots & 0
%	\end{array}\!\right],
%\end{equation}
%while, when $ \delayn $ is even, the $ (\nicefrac{\delayn}{2} + 2) $-th row is
%\begin{equation}\label{eq:app-explicit-variance-matrix-A-mid-row-even}
%	\left[\!\begin{array}{cccccccc}
%		0 & \dots & 0 & 1-\gpos & -1 & 0 & \dots & 0
%	\end{array}\!\right].
%\end{equation}
%where we highlight the dependence on the delay in view of the recursive characterization of $ \rho_0 $.
It can be seen that $ A^{(\tau)} $ is full rank for all $ \delayn \ge 1 $ and thus~\eqref{eq:disc-time-single-int-moment-matching-eqs} has a unique solution.
In particular, we are interested in the autocorrelation $ \rho_0 = \varx{\gpos}{I} $,
which is given by
the ratio between the minor associated with the top-left element of $ A^{(\tau)} $,
named $ n_\delayn \doteq M^{(\tau)}_{1,1} $, and the determinant $ d_\delayn \doteq \det(A^{(\tau)}) $.
Specifically, $ \rho_0 $ is a rational function in $ \gpos $
and can be computed in closed form by a symbolic solver
given any value of $ \delayn $.

Further, $ n_{\delayn} $ and $ d_{\delayn} $ can be explicitly computed %recursively via an inductive argument on the delay $ \delayn $, 
by leveraging a recursive nested structure of the matrix $ A $.
%\begin{equation}\label{eq:recursive-matrix}
%	A^{(\delayn)} = \tikz[baseline=(M.west)]{%
%		\node[matrix of math nodes,matrix anchor=west,left delimiter={[},right delimiter={]},ampersand replacement=\&] (M) {%
%		-\gpos^2\&			\&			\&							\& 							\&						 \&	-2\gpos		\\
%			1 	\& -1  		\&    		\&							\& 							\& 						 \&	  			\\
%				\&  1  		\&  -1		\&	  						\&      					\& 						 \& -\gpos   	\\
%				\&    		\&   1 		\& \textcolor{white}{t1}	\& 							\& 						 \&				\\
%				\&			\&			\&							\& \tilde{A}^{(\delayn-4)}	\&						 \&				\\
%				\& 			\&	  		\& 							\&	 						\& \textcolor{white}{t1} \&				\\
%				\& 		    \&   -\gpos	\&							\&							\& 1					 \&	-1			\\
%				\& 	-\gpos	\&	  		\&							\&							\&						 \&	1			\\			
%		};
%		\node[draw,fit=(M-3-3)(M-7-7),inner sep=-1pt] {};
%		\node[draw,fit=(M-4-4)(M-6-6),inner sep=-1pt] {};
%	}.
%\end{equation}
%where $ \tilde{A}^{(\delayn)} $ is the submatrix of $ A^{(\delayn)} $ obtained by removing its first row and column
%such that $ M^{(\delayn)}_{1,1} = \det(\tilde{A}^{(\delayn)}) $,
%and the matrices $ \tilde{A}^{(\delayn-2)} $ and $ \tilde{A}^{(\delayn-4)} $ are framed in~\eqref{eq:recursive-matrix}.
The solution obeys the following recursive expression in $ \delayn $:
%\begin{prop}\label{prop:disc-time-single-int-variance-explicit}
\begin{subequations}\label{eq:disc-time-single-int-variance-explicit}
	\begin{gather}
		n_\delayn = \begin{dcases}
			(-1-\gpos)n_{\delayn-1} + \tilde{n}_{\delayn-1} & \mbox{if } \delayn \mbox{ odd}\\
			-(1-\gpos)n_{\delayn-1} - \gpos\tilde{n}_{\delayn-1} & \mbox{if }\delayn \mbox{ even},\\
		\end{dcases} \label{eq:disc-time-single-int-variance-explicit-numerator}\\
		\tilde{n}_{\delayn} = (2-\gpos^2)\tilde{n}_{\delayn-2} - \tilde{n}_{\delayn-4}, \label{eq:disc-time-single-int-variance-explicit-numerator-rem}\\
		d_\delayn = d_{\delayn-2} - \gpos^2\left(n_\delayn+n_{\delayn-2}\right), \label{eq:disc-time-single-int-variance-explicit-denominator}\\
		\tilde{n}_{-3} = -1+\gpos^2, \ \tilde{n}_{-2} = \gpos^2, \ \tilde{n}_{-1} = -1, \ \tilde{n}_0 = 0, \\
		n_{-1} = 0, \ n_0 = 1, \ d_{-1} = -2\gpos, \ d_0 = 2\gpos-\gpos^2.
	\end{gather}
\end{subequations}
Detailed derivation of~\eqref{eq:disc-time-single-int-variance-explicit}
is given in the technical report~\cite{2021arXiv210900359B}.
Given $ \delayn $,
convexity of $ \rho_0 $ in $ \gpos $ can be assessed by checking the sign
of the second derivative in the stability region.
This reduces to a system of inequalities
which can be solved, \eg by \texttt{solve\_rational\_inequalities} in Python.
The variance was proved strictly convex for all tried delays.

\myParagraph{\titlecap{double integrator model}}
System~\eqref{eq:disc-time-double-int-decoupled} yields the following
$ \delayn+2 $ coupled moment-matching equations,
%The moment-matching system associated with~\eqref{eq:disc-time-double-int-decoupled}
%has $ \delayn+2 $ variables $ (\rho_0,\dots,\rho_{\delayn+1}) $ and is composed of the following equations:
\begin{subequations}\label{eq:disc-time-double-int-moment-matching-eqs}
	\begin{align}
		\begin{split}
			\rho_0 &= [(2-\gvel)^2 + (1-\gvel)^2 + \gvel^2\gpos^2]\rho_0 - 2(2-\gvel)(1-\gvel)\rho_1 \\
			& + 2(1-\gvel)\gvel\gpos\rho_\delayn - 2(2-\gvel)\gvel\gpos\rho_{\delayn+1} + 1 \label{eq:disc-time-double-int-first-moment-eq}
		\end{split}\\
		\rho_1 &= (2-\gvel)\rho_0 - (1-\gvel)\rho_1 - \gvel\gpos\rho_{\delayn+1} \label{eq:disc-time-double-int-yule-walker-eqs-1}\\
		&\hspace{2mm}\vdots \nonumber \\
		\rho_{\delayn+1} &= (2-\gvel)\rho_\delayn - (1-\gvel)\rho_{\delayn-1} - \gvel\gpos\rho_1, \label{eq:disc-time-double-int-yule-walker-eqs-2}
	\end{align}
\end{subequations}
with~\eqref{eq:disc-time-double-int-yule-walker-eqs-1}--\eqref{eq:disc-time-double-int-yule-walker-eqs-2} the associated Yule-Walker equations.
%associated with~\eqref{eq:disc-time-double-int-decoupled}.
Analogous analysis to single-integrator model can be performed.