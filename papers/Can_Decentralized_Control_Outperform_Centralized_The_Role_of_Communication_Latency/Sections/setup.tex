%!TEX ROOT = ../centralized_vs_distributed.tex

\section{Problem Setup}\label{sec:setup}

We consider {an undirected network} with $ N $ agents 
{in which the state of the $ i $th agent at time $ t $ is given by $ \xbar{i}{t}\in\Real{} $ with the control input $ \u{i}{t}\in\Real{} $.}
For {notational} convenience,
we {introduce} the aggregate {state of the} system  $ \xbar{}{t} $ and {the}
aggregate control input $ \u{}{t} $ {by stacking states and control inputs of each subsystem} $ \xbar{i}{t} $ and $ \u{i}{t} $, 
respectively.

\iffalse
\begin{rem}[Network topology]
	While in the first part of the paper we focus on circular formations for the sake of analysis and ease of presentation,
	the control design can be readily extended to generic undirected topologies.
	We discuss theoretical guarantees in~\autoref{sec:generic-topology} and observe with computational experiments in~\autoref{sec:numerical-results}
	that the \tradeoff holds regardless of the specific topology. % at hand.
\end{rem}
\fi

\myParagraph{Problem Statement}
The agents aim to reach consensus towards a common state trajectory. 
The $i$th component of the vector $ \x{}{t} \doteq \Omega\xbar{}{t} $ represents 
the mismatch between the state of agent $ i $ and the average network state at time $ t $\revision{\cite{bamjovmitpat12}},
where
\begin{equation}\label{eq:error-matrix}
	\Omega \doteq I_{N}-\consMatrix
\end{equation}
and $ \mathds{1}_N \in\Real{N} $ is the vector of all ones,
such that $ \Omega\mathds{1}_N=0 $.
%\red{The target consensus vector is defined as $ \x{m}{t} \doteq \xbar{}{t} - \x{}{t} $.}

\done{
	\tcb{\myParagraph{Ring Topology}
	We focus on ring topology to obtain analytical insights about 
	optimal control design and fundamental performance trade-offs in the presence of communication delays. 
	While some of our notation is tailored to such topology (\eg see equations~\eqref{eq:meas} and~\eqref{eq:feedback-matrix}), 
	in~\autoref{sec:generic-topology} we discuss extension of the optimal control design to generic undirected networks 
	and complement these developments with computational experiments in~\autoref{sec:numerical-results}.}
}

%\myParagraph{\titlecap{Communication model}}
%The agents communicate through a {shared wireless channel}.
%Data are exchanged through a {shared wireless channel} in a symmetric fashion.
%Agent $ i $ communicates with
%\red{$ n $ pairs of agents,
%both agents in each such pair being at equal distance from $ i $}
%\tcb{its} $ 2n $ closest neighbors \tcb{in ring topology.}
%Also, we make the following assumption
%to address channel constraints.

\begin{ass}[Communication model]\label{ass:hypothesis}
	Data are exchanged through a shared wireless channel in a symmetric fashion.
%	\tcb{its} $ 2n $ closest neighbors \tcb{in ring topology.}
	\revisiontwo{Agent $ i $ receives state measurements from
	all agents within $ n $ communication hops.}
	All measurements are received with delay $ \taun \doteq f(n) $
	where $ f(\cdot) $ is a positive increasing sequence.
	\revisiontwo{In particular,
		in ring topology,
		agent $ i $ receives state measurements from the
		$ 2n $ closest agents,
		that is,
		from the $ n $ pairs of agents at distance $ \ell = 1,\dots,n $,
		with $ 1\le n<\nicefrac{N}{2} $.}\footnote{\revision{
%	where both agents in each such pair are at equal distance $ \ell $ from $ i $. % in the ring topology.
%%	located $ \ell $ positions ahead and behind in the formation,.
%	In what follows,
%	without loss of generality,
%	we assume that such $ n $ agent pairs coincide with the
%	$ 2n $ closest agents in ring topology,
%	and that each pair is at distance $ \ell = 1,\dots,n<\nicefrac{N}{2} $.
	\revisiontwo{For example,
	$ n = 1 $ corresponds to nearest-neighbor interaction in ring topology
	and $ n = \floor{\nicefrac{(N-1)}{2}} $ to all-to-all communication topology.}}}
%	Also, each agent measures its own state instantaneously.
\end{ass}

\revision{\begin{rem}[Architecture parametrization]\label{rem:architecture-param}
	Parameter $ n $ will play a crucial role throughout our discussion. 
	In particular,
	we will use it to (i) evaluate the optimal performance %that can be attained 
	for a given
	budget of links
	\revisiontwo{(see~\cref{prob:variance-minimization})};
	and to (ii) compare optimal performance of different control architectures.
	In the first part of the paper, 
	we examine circular formations and
	$ n $ represents how many neighbor pairs communicate with each agent.
	For \linebreak general undirected networks,
	$ n $ determines the number of communication hops for each agent.
	In general,
	$ n $ characterizes sparsity of a controller architecture:
	sparse controllers correspond to small $n$ while highly connected ones to 
	large $ n $.
\end{rem}}

%\begin{rem}
%	The time $ \delayn $ embeds both the communication delay,
%	due to channel constraints,
%	and the computation delay,
%%	Even though the rate $ f(n) = n $ may seem natural,
%%	other rates are possible, 
%	arising if the agents preprocess the acquired measurements.
%	In practice, $ f(n) $ is to be estimated or learned from data.
%\end{rem}

\myParagraph{\titlecap{Feedback control}}
Agent $ i $ uses the received information to compute the
state mismatches $ \meas{i}{\ell^\pm}{t} $ {relative to its} neighbors,
\begin{equation}\label{eq:meas}
	\meas{i}{\ell^\pm}{t} = 
	\begin{cases}
		\xbar{i}{t} - \xbar{i\pm\ell}{t}, & 0<i\pm\ell\le N\\
		\xbar{i}{t} - \xbar{i\pm\ell\mp N}{t}, & \mbox{otherwise},
	\end{cases}
\end{equation}
%Such mismatches are exploited to compute the 
{and} the proportional control input is {given by}
\begin{equation}\label{eq:prop-control}
	\u{P,i}{t} = -\sum_{\ell=1}^{n}k_\ell\left(\meas{i}{\ell^+}{t-\taun}+\meas{i}{\ell^-}{t-\taun}\right),
\end{equation}
where measurements are delayed according to~\cref{ass:hypothesis}.

For networks with double integrator agents,
the control input $u_i(t)$ may also include a derivative term,
%Depending on the agent dynamics, the control input $ \u{i}{t} $
%may be purely proportional or include a derivative term, such as
\begin{equation}\label{eq:control-input-PD}
	\u{i}{t} = \gvel\u{P,i}{t} - \gvel\dfrac{d\xbar{i}{t}}{dt} = \gvel\u{P,i}{t} -\gvel\dfrac{d\x{i}{t}}{dt}.
\end{equation}
The derivative term in~\eqref{eq:control-input-PD} is delay free
because it only requires measurements coming from the agent itself,
which we assume {to be} available instantaneously. 
%The latter will be defined in due time.
The proportional input can be compactly written as $ \u{P}{t} = -K\xbar{}{t-\taun}=-K\x{}{t-\taun} $.
\revision{With ring topology, the feedback gain matrix is}
\begin{equation}\label{eq:feedback-matrix}
%	\begin{array}{c}
%		K \doteq K_f + K_f^\top \\
		K = \mathrm{circ}
		\left(\sum_{\ell=1}^nk_\ell, -k_1, \dots, -k_n, 0,  \dots, 0, -k_n, \dots, -k_1\right),
%	\end{array}
\end{equation}
where $ \mathrm{circ}\left(a_1,\dots,a_n\right) $ denotes the circulant matrix in $ \Real{n\times n} $
with elements $ a_1,\dots,a_n $ in the first row.

\revisiontwo{For agents with additive stochastic disturbances
	(see Sections \ref{sec:cont-time} and~\ref{sec:disc-time}),
	we consider the following problem for each $ n $.}

\begin{prob}\label{prob:variance-minimization}
	Design the feedback gains in order to minimize the steady-state variance of the consensus error,
%	\marginpar{\tiny Added both problems to highlight the optimization variables in the two cases.}
	\blue{\begin{subequations}\label{eq:problem}
		\begin{equation}\label{eq:variance-minimization-P}
		\mbox{P control:} \qquad \argmin_{K} \; \var(K),
		\end{equation}
		\begin{equation}\label{eq:variance-minimization-PD}
		\mbox{PD control:} \qquad \argmin_{\gvel,K} \; \var(\gvel,K),
		\end{equation}
	\end{subequations}}
	where
	\begin{equation}\label{}
	\var \doteq \lim_{t\rightarrow+\infty} \mathbb{E}\left[\lVert\x{}{t}\rVert^2\right]
	\end{equation}
	and w.l.o.g. we assume $ \mathbb{E}\left[\x{}{\cdot}\right] \equiv \mathbb{E}\left[\x{}{0}\right] = 0 $.
	%	and $ \varx{x} \stackrel{!}{=} $ if the system is mean-square unstable.
\end{prob}