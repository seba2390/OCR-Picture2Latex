%!TEX ROOT = ../../centralized_vs_distributed.tex

%\subsection{Preview of key results}\label{sec:contribution}

%Our contribution is twofold.
%	\mjmargin{not sure what light blue specifies}
\if0\mode
\newpage
\fi
\myParagraph{Preview of Key Results}
{We utilize undirected graphs with %ring topology with 
single- and double-integrator agent dynamics to examine fundamental performance limitations in networked systems with architecture-dependent communication delays. 
By exploiting convexity of a minimum-variance control design problem with respect to the feedback gains}, %circulant structure of the underlying matrices 
we demonstrate that the choice of controller architecture
has profound impact on network performance in the presence of delays. 
\revision{In particular, when the delays increase fast enough with the number of links,
sparse topologies can outperform highly connected ones.}
%even though they limit global information exchange.
%This allows to keep the optimization relatively simple
%and to hold the focus on the main matter under investigated,
%which in the following we shall call \textit{\tradeoff}.

%As discussed previously,
%our approach makes network-dependent latency
%directly reflect on the cost-to-go.
%%\ie with different network topologies.
%Contrary to the conventional wisdom that
%the fully connected control architecture leads to
%optimal performance,
%we show that, % a trade-off arises
%if the delays increase fast enough with the number of links,
%%\eg as a result of delayed feedback,
%highly connected topologies perform worse than
%sparse architectures.
%%in general.

\done{
\tcb{We show that the steady-state variance of a stochastically forced network,
	$ J_{\textrm{tot}}(n) $, can be represented by a sum of two monotone functions of the number of neighbors $ n $ (\autoref{fig:trade-off}),
}}
	\iffalse
	a product of two monotone functions of the number of neighbors $ n $,
\begin{equation}\label{eq:trade-off-mul}
	\tcb{J_{\textrm{tot}}(n) = J_{\textrm{network}}(n)\cdot J_{\textrm{latency}}(n).}
\end{equation}

Here, 
$ J_{\textrm{network}}(n) $ quantifies impact of control architecture and
$ J_{\textrm{latency}}(n) $ determines influence of communication latency on network performance.
While $ J_{\textrm{network}}(n) $ decreases with $ n $ and is minimized by a fully-connected centralized architecture,
$ J_{\textrm{latency}}(n) $ increases with $n$.
This demonstrates the presence of a fundamental trade-off:
on one hand, 
feedback control takes advantage of dense topologies that benefit from information sharing but,
on the other, 
many communication links induce long delays which has negative impact on network performance.

Furthermore, if the delays are sublinear in $ n $,
we show that the trade-off can be also decomposed additive-wise (see ~\autoref{fig:trade-off}):
\fi

\done{\begin{equation}\label{eq:trade-off}
	{\tcb{J_{\text{tot}}(n) = J_{\textrm{network}}(n) + J_{\textrm{latency}}(n)}.}
\end{equation}
Here, 
$ J_{\textrm{network}}(n) $ quantifies impact of control architecture and
$ J_{\textrm{latency}}(n) $ determines influence of communication latency on network performance.
While $ J_{\textrm{network}}(n) $ decreases with $ n $ and is minimized by a fully-connected centralized architecture,
$ J_{\textrm{latency}}(n) $ increases with $n$.
This demonstrates the presence of a fundamental trade-off:
on one hand,
feedback control takes advantage of dense topologies that enhance information sharing but,
on the other hand, many communication links induce long delays which have negative effect on performance.

While~\eqref{eq:trade-off} can be derived analytically
\tcb{for ring topology with continuous-time, single-integrator dynamics},
	our
	\tcb{computational experiments} show that
	a \tcb{similar \textit{\tradeoff} can be observed}
	\revision{\tcb{for general undirected topologies}
	and with double-integrator and discrete-time agent dynamics}. 
	\review{Furthermore,
	in some cases, decentralized architecture with nearest neighbor information exchange
	provides optimal performance.}}

\begin{figure}
	\centering
	\includegraphics[width=.67\linewidth]{trade-off}
	\caption{Steady-state variance $ J_{\textrm{tot}}(n) $ versus number of neighbors. %~\eqref{eq:trade-off}.
		The variance is the sum of two costs:
		$ J_{\textrm{network}}(n) $ represents impact of control architecture,
		while
		%		and decreases with denser control architectures.
		$ {J}_{\textrm{latency}}(n) $ is due to the delays affecting the dynamics.
		%		which increases when the addition of links induces longer communication delays.
	}
	\label{fig:trade-off}
\end{figure}