%!TEX ROOT = ../../centralized_vs_distributed.tex

%% Control literature
%Many works in the literature support decentralized architectures with various arguments.
\startSentence{Related work in control theory} deals with control design for
distributed architectures,
where classical methods,
such as LQG or $ \mathcal{H}_2 $/$ \mathcal{H}_{\infty} $ control,
\review{require an all-to-all information exchange which is infeasible for large-scale systems.}

%the design % in the presence of latency % in the presence of delays
A \review{large body of work} focuses on stability,
%and the identification of convex structured problems.
\eg~\cite{ren2017finite,SUN2021419} are concerned with finite-time delay-dependent stability of
discrete-time systems,
\cite{BEREZANSKY2015605} finds sufficient conditions for uniform stability
of linear delay systems,
\revision{\cite{MUNZ20101252} characterizes stability and consensus conditions with homogeneous and heterogeneous feedback delays},
and~\cite{Chehardoli2019,8844785} analyze consensus and error compensation for vehicular platoons.
Another line of work deals with maximizing performance for structured controllers,
\revision{\eg~\cite{8358743,Michiels2016,8430769} study $ \mathcal{H}_2 $-norm minimization for time-delay network systems,}
\cite{SOUDBAKHSH2017171} proposes a cyber-physical architecture with LQR for wide-area power systems,
\cite{MORATO202178} develops a procedure for time-varying dead-time compensation
by adapting the Filtered Smith Predictor,
and~\cite{9137405} investigates sensor-and-processing selection for optimal estimation in star networks.

A more recent trend is optimizing the controller architecture.
%namely, the communication network.
For large-scale systems,
this means sparsifying the structure
to enhance communication and scalability.
This is achieved by introducing penalty terms %in the cost function
to trade performance for controller complexity~\cite{6497509,dorjovchebulTPS14,7835692,9216852,7347386,BAHAVARNIA201710395,7378905,ANDERSON2019364}.
In particular,~\cite{7378905} proposes the \textit{Regularization for Design},
addressing optimization of communication links,
\revision{while~\cite{ANDERSON2019364} investigates communication locality
and its relation to control design within the \textit{System Level Synthesis}.}

%% Optimization literature
\startSentence{Related work in optimization theory} is concerned
with minimization of distributed cost functions,
which are only partially accessible at each agent.
%and possibly reveal themselves overtime.
A large body of literature has been devoted to study suitable algorithms,
a short list of which is represented by
\review{\cite{8027140,XIAO200733,8015179,7807315,7472453,mogjovTCNS18}}.
In particular, a line of work has been concerned specifically with the design
of algorithms in the presence of communication delays,
the main issues being related to convergence conditions.
For example,~\cite{6120272,6571230,7994706,ZONG2019412,garcia2016periodic}
study consensus of multi-agent systems with additive or multiplicative time-delays
under various network topologies and agent dynamics.
%showing how its convergence depends on such model parameters.
This approach usually the communication network be given
and focuses on the information exchange and processing by the agents
from an optimization standpoint.