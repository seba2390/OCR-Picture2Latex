%!TEX ROOT = ../../centralized_vs_distributed.tex

%\subsection{Paper outline}\label{sec:outline}

\begin{table}
	\centering
	\caption{Theoretical tools (italic) and technical results (roman).}
	\label{tab:results}
	\footnotesize
	\begin{tabular}{|c|c|c|c|}
		\hline
		 & \textbf{Model} & \textbf{Stability} & \textbf{Variance} \\
		\hline
		\multirow{5}{*}{\shortstack{\textbf{Cont.} \\\textbf{time} \\\textbf{(CT)}}} & 
										\makecell{Single int. \\ \eqref{eq:cont-time-single-int-model},\eqref{eq:prop-control}} & 
										\makecell{\textit{Scalar SDDEs~\cite{KuchlerLangevinEqs}} \\ 
											Closed form~\eqref{eq:cont-time-single-int-variance-condition}} &
										\makecell{\textit{Scalar SDDEs~\cite{KuchlerLangevinEqs}} \\
											Closed form~\eqref{eq:cont-time-single-int-steady-state-variance}} \\
		\cline{2-4}
									& \makecell{Double int.\\ \eqref{eq:cont-time-double-int-model}--\eqref{eq:control-input-PD}} & 
									\makecell{\textit{Exponential} \\ \textit{polynomials~\cite{BAPTISTINI1997259}} \\ 
										\textit{SDDEs~\cite{datko1978procedure,wangBoundedness}} \\
										{Implicit}~\eqref{eq:cont-time-double-int-stability-condition}} & 
									\makecell{\textit{SDDEs~\cite{wangBoundedness}, time-}\\
										\textit{scale separation~\cite{khalil2002nonlinear}}\\
										{Integral form}~\eqref{eq:2nd-order-cont-ss-variance}\\
										Approximated~\eqref{eq:x-dynamics-1st-order-approximation}} \\
		\hline
		\multirow{4}{*}{\shortstack{\textbf{Disc.} \\ \textbf{time} \\\textbf{(DT)}}} &  
										\makecell{Single int. \\ \eqref{eq:disc-time-single-int-model},\eqref{eq:prop-control}} & 
										\makecell{\textit{Root locus~\cite{Westphal2001}}\\ 
											Closed form~\eqref{eq:disc-time-single-int-stability-condition}} &
										\makecell{\textit{Moment matching w/} \\
											\textit{Yule-Walker eqs.~\cite{yuleWalkerEqs}} \\
											{Recursive}~\eqref{eq:disc-time-single-int-moment-matching-eqs},\eqref{eq:disc-time-single-int-variance-explicit}} \\
		\cline{2-4}
									&  \makecell{Double int. \\ \eqref{eq:disc-time-double-int-model}} & 
									\makecell{\textit{Jury criterion~\cite{Jury}} \\ 
										{Closed form}~\eqref{eq:disc-time-double-int-characteristic-polinomial}} &
									\makecell{\textit{Moment matching w/} \\
										\textit{Yule-Walker eqs.~\cite{yuleWalkerEqs}} \\
										{Closed form}~\eqref{eq:disc-time-double-int-moment-matching-eqs}} \\
		\hline
	\end{tabular}
\end{table}

\myParagraph{\titlecap{Paper outline}}
In~\autoref{sec:setup} we describe models for communication and controller architecture
and formulate the minimum-variance control design problem.
\revision{While we first utilize ring topology to provide analytical insight
we also demonstrate that our framework can be extended to general undirected topologies; see~\autoref{sec:generic-topology}.}\linebreak
\revision{In Sections~\ref{sec:cont-time}--\ref{sec:cont-time-single-int-control-design}, we %solve the problem for continuous-time agent dynamics,
lay the ground for our main result.
In~\autoref{sec:cont-time}, 
we derive conditions for mean-square stability 
and compute the steady-state variance of continuous-time stochastically forced systems
using Stochastic Delay Differential Equations (SDDEs). 
In~\autoref{sec:cont-time-single-int-control-design},
we prove that the control design problem is convex} \review{and in 
\revision{\autoref{sec:numerical-results} we present} our main results:
by numerically computing the optimal controller gains,
we show that the closed-loop performance is optimized by sparse architectures.
Furthermore, 
we derive analytical expression~\eqref{eq:trade-off} for continuous-time single-integrator dynamics 
which demonstrates that the minimizer is in general nontrivial.}
To address wireless communication,
we study discrete-time systems in~\autoref{sec:disc-time}
and show that the fundamental behavior of the system does not change.
\autoref{tab:results} summarizes our technical results and the theoretical tools used throughout the paper.
\revision{Apart from classical control techniques such as the Jury stability criterion,
we also leverage more unconventional tools from mathematical literature,
such as exponential polynomials~\cite{BAPTISTINI1997259}.}
Concluding remarks are given in~\autoref{sec:conclusion}.