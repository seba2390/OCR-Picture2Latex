%!TEX ROOT = ../../centralized_vs_distributed.tex

\myParagraph{Addressed Problem} %{Despite such a bulky literature},
\done{\tcb{Even though} both control design for delay-dependent dynamics 
and \tcb{design of controller architectures are well-studied topics, it remains unclear how} {\em network connectivity affects \tcb{the closed-loop performance in the presence of architecture-dependent communication latency.}} 
\tcb{When the total available bandwidth does not increase with the size of the network~\cite{garcia2016periodic}
	or when multi-hop communication is used among low-power devices~\cite{gupta2010delay},}
the number of active communication links may %cause a non-negligible variation of such a latency.
\revision{affect such latency in non-negligible way}.
\tcb{\revision{In this case, it is important that the control design takes into account increase in delays 
%	with the number of links
	when new communication links are introduced.}}}

\done{
Such an approach is conceptually different from
the \tcb{approaches used} in literature.
On one hand,
delay-aware control designs \revision{such as~\cite{MUNZ20101252,8430769}}
\revision{assume a fixed controller architecture
and either target optimization of the feedback gains
or evaluate stability with respect to gains and/or delays}.
On the other hand,
architecture designs such as~\cite{7378905,7347386}
\revision{do not quantify the impact of architecture-dependent delays on performance,
but explicitly force sparsity by
adding a regularization term that penalizes controller complexity to delay-free performance metrics}.
In fact,
while the fully connected architecture is avoided
because of practical limitations,
it is usually regarded as an upper bound for performance~\cite{JOVANOVIC201676}.}
\revision{To the best of our knowledge, the only works where architecture-dependent delays are used to 
	compute the performance metric are~\cite{gupta2010delay,gupta2011delay},
	where the authors study how transmission power affects convergence rate of consensus.}

\revision{
	We study class of static feedback policies in which control action is formed 
	by utilizing delayed measurements from a limited number of nodes within a network. 
	Impact of similar type of controller architectures 
	on mean-square performance of delay-free stochastically forced consensus, 
	synchronization, 
	and vehicular formation networks has been studied in the literature~\cite{bamjovmitpat12,mogjovTCNS18},
	and our objective is to understand influence of delays on performance
	trade-offs induced by such localized controller architectures relative to centralized ones. 
	Identifying similar trade-offs within other classes of localized control policies 
	(including System Level Synthesis) is a relevant open question which is outside the scope of the current study. 
}

\revision{\myParagraph{Original Contribution}
We aim to bridge the two domains of delay-aware control and architecture design by
quantifying how the latter % with the that appear in the dynamics of the controlled system
affects performance under architecture-dependent communication delays. %is affected by variations in such delays.
%\revision{Specifically, we are interested in \textit{the optimal architecture}.
We address two key challenges. 
First,
we focus on \textit{optimal performance},
whereby \emph{stability} is a prerequisite to control design
needed to provide a bounded cost function. % explodes in the absence of stability.
Hence,
we derive stability conditions that are instrumental to an optimal control design problem.
Second,
we aim to identify the \textit{optimal controller architecture} under delays and quantify fundamental performance trade-offs.
Towards this goal, to circumvent the discrete nature of graphs, 
we work our way through two stages:
first, 
we parametrize each architecture with a parameter $ n $
which characterizes both number of links and delay associated with that architecture,
and show how to compute the optimal controller for a given $ n $. 
We then compare the optimal performance obtained for different values of $ n $, 
which allows us to fairly establish which architectures provide the best closed-loop performance. 
In contrast to~\cite{gupta2010delay,gupta2011delay},}
\revision{
we examine mean-square performance of stochastically forced networks, 
study generic delay functions, 
and address optimal design of feedback gains for different controller architectures.}

%indeed, this approach ties the network connectivity
%directly to the system performance,
%this causes densely connected topologies
%to naturally degrade the performance,
%with no need of adding somewhat abstract penalty terms.
%Indeed, such an approach tightly bounds the controller architecture
%with the dynamic-related performance achieved by the controlled system.