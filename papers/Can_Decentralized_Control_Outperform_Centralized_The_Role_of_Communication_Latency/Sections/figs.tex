%!TEX ROOT = ../centralized_vs_distributed.tex

\begin{figure*}%
	\centering
	\subfloat[Continuous-time single integrator.% with $ \taun = 0.1n $.
	]{
		\centering
		\includegraphics[width=.22\linewidth]{cont-time-single-int-opt-var}
		\label{fig:cont-time-single-int-opt-var}
	}%
	\hfil
	\subfloat[Continuous-time double integrator.% with $ \taun = 0.1n $.
	]{
		\centering
		\includegraphics[width=.22\linewidth]{cont-time-double-int-opt-var-n}
		\label{fig:cont-time-double-int-opt-var}
	}%
	\hfil
	\subfloat[Discrete-time single integrator.% with $ \taun = n $.
	]{
		\centering
		\includegraphics[width=.22\linewidth]{disc-time-single-int-opt-var}
		\label{fig:disc-time-single-int-opt-var}
	}%
	\hfil
	\subfloat[Discrete-time double integrator.% with $ \taun = n $.
	]{
		\centering
		\includegraphics[width=.22\linewidth]{disc-time-double-int-opt-var-n}
		\label{fig:disc-time-double-int-opt-var}
	}%
	\caption{Optimal
		and suboptimal steady-state scalar variances with linear delay increase
		for different agent dynamics.
		%		The used delay increase rates are $ \taun = 0.1n $ for continuous-time and $ \taun = n $ for discrete-time models.
		%		All curves exhibit nontrivial trade-offs over the number of links. %, with the optimal amounts of links smaller than the maximum available.
	}
	\label{fig:opt-var}
\end{figure*}