%!TEX ROOT = ../../centralized_vs_distributed.tex

\myParagraph{Stability Analysis}\label{sec:disc-time-stability-analysis}
%Writing the formation dynamics in vector-matrix form,
The formation error dynamics can be decoupled analogously to the continuous-time models.
The decoupled subsystems are asymptotically stable
%poles of the spectral factors $ W_i(z) $
%such that $ \xtilde{i}{k} = W_i(z)\noisetilde{i}{k} $.
%which generates $ \xtilde{i}{t} $ by filtering white noise.
%Consider the generic system
%\begin{equation}\label{eq:disc-time-single-int-subsystem}
%		\x{}{k+1} = \x{}{k} -\gpos\x{}{k-\delayn} + \noise{}{k}
%\end{equation}
%In~\eqref{eq:disc-time-single-int-subsystem}, $ \x{}{k} $ is generated from white noise through the filter
%\begin{equation}\label{eq:disc-time-single-int-transfer-function}
%	W(z) = \dfrac{1}{z-1+\gpos z^{-\delayn}}
%\end{equation}
%In particular, the systems are asymptotically stable
if and only if all the roots of their associated characteristic polynomials
%\ie the denominator of $ W_i(z) $,
lie inside the unit circle in the complex plane.

In general, given a delay $ \taun $,
stability conditions with respect to the control gains can be derived
%the location of the system poles %of~\eqref{eq:disc-time-single-int-characteristic-polinomial}--\eqref{eq:disc-time-double-int-characteristic-polinomial}
in the form of polynomial inequalities through the Jury criterion.
For the single-integrator case,
one simple condition can be computed analytically.  %by exploiting the root locus associated with $ \gpos_i $
\begin{prop}[Stability of DC single integrators]\label{prop:disc-time-single-int-stability}
%	Let $ \gpos \in\Realp{} $.
%	Then, system~\eqref{eq:disc-time-single-int-model}
	{The network error $ \x{}{t} $ is mean-square stable} if and only if
%	the eigenvalues of $ K $ satisfy
	\begin{equation}\label{eq:disc-time-single-int-stability-condition}
		\gpos_j \in \left(0,2\sin\left(\dfrac{\pi}{2}\dfrac{1}{2\taun+1}\right)\right), \quad j = 2,\dots,N.
%		\gpos < \gpos_{\textit{th}} \doteq 2\sin\left(\dfrac{\pi}{2}\dfrac{1}{2\tau+1}\right)
	\end{equation}
\end{prop}
The upper bound in~\eqref{eq:disc-time-single-int-stability-condition} approaches its continuous-time counterpart~\eqref{eq:cont-time-single-int-variance-condition} from below
as the delay steps tend to infinity (see~\autoref{fig:stability-region-single-int}). %, where $ \sin(\star) \approx \star $.
%Indeed, given the same absolute delay,
%a finer sampling yields more delay steps.
%and thus at the limit the discretized dynamics
%converges to the continuous-time one. and retrieves the same constraint.
%In general, condition~\eqref{eq:disc-time-single-int-stability-condition} is tighter than~\eqref{eq:cont-time-single-int-variance-condition}.
%On the other hand, the asymptotic behavior of the threshold gain
%suggests that the gap between continuous-time and discretized systems only matters
%when the delay is comparable with the sampling time,
%while, when the former gets too long, the loss of feedback information
%neglects the dynamics discretization.
A discussion on general stability conditions
and the proof of~\cref{prop:disc-time-single-int-stability} are provided in~\cref{app:disc-time-single-int-stability}.
The basic argument is the same as for the continuous-time case.
%\autoref{fig:stability-region-single-int} compares the stability regions
%in the $ (\taun,\gpos_j\taun) $ plane
%for continuous- and discrete-time single integrators.
%In particular, the stability region of the discrete-time systems is strictly contained in the continuous-time one.

\begin{figure}
	\centering
	\begin{tikzpicture}[scale=1]
		\begin{axis}[xmin=0,xmax=20,ymin=0,ymax=2,
			ytick = {0,1,pi/2},yticklabels = {$ 0 $,$ 1 $,$ \dfrac{\pi}{2} $},
			tick label style={font=\normalsize},
			xlabel=$ \taun $,ylabel=$ \gpos_j\taun $,label style={font=\large},
			trig format plots=rad,
			legend pos=south east,legend style={font=\normalsize},legend cell align={left},
			width=.9\linewidth,height=.5\linewidth]
			\addplot[draw=black,pattern=north west lines,pattern color=black!40,area legend] (0,0) rectangle (20,pi/2);
			\addplot[domain=1:20,samples=20,mark=none,ycomb,ultra thick] {x*2*sin(pi/(2*(2*x+1)))};
			\addplot[draw=black,ultra thick] (20,0) -- (20,1);
			\legend{Continuous-time~\eqref{eq:cont-time-single-int-variance-condition},,Discrete-time~\eqref{eq:disc-time-single-int-stability-condition}};
		\end{axis}
	\end{tikzpicture}
	\caption{Stability regions of decoupled single integrators.}
	\label{fig:stability-region-single-int}
\end{figure}