%!TEX ROOT = ../../centralized_vs_distributed.tex

\myParagraph{\titlecap{performance evaluation}}\label{sec:disc-time-single-int-moment-matching}
With fixed parameters,
the steady-state variance of each decoupled subsystem
can be computed numerically via the Wiener–Khintchine formula. % recalled in~\autoref{app:disc-time-single-int-variance-explicit}.
Also, for any given value of $ \taun $,
%gain-parametric 
a closed-form expression of the variance
%whose convexity can be easily assessed,
can be obtained via moment matching through a recursive formula, see~\cref{app:disc-time-single-int-variance-explicit}.
Such closed-form expressions have been used for our computational experiments illustrated in~\autoref{fig:opt-var}.
\textcolor{subsectioncolor}{Figure~\ref{fig:disc-time-var}} shows the typical profiles
of the variance function for decoupled subsystems with single- and double-integrator dynamics
(see~\eqref{eq:disc-time-single-int-decoupled} and~\eqref{eq:disc-time-double-int-decoupled} in~\cref{app:disc-time-single-int-variance-explicit},
respectively).
%For the one-dimensional case (single integrator),
%convexity of the variance function $ \var{\gpos_i} $ can be checked
%by studying the second derivative. % solving a system of inequalities.
%%over the decoupled subsystems.
%\autoref{fig:disc-time-var} shows the level curves of the variance,
%which looks convex also for the double integrator.

\begin{figure}
	\centering
	\begin{minipage}{.5\linewidth}
		\centering
		\includegraphics[width=\linewidth]{disc-time-single-int-var}
	\end{minipage}%
	\hfil
	\begin{minipage}{.5\linewidth}
		\centering
		\includegraphics[width=\linewidth]{disc-time-double-int-var-heatmap}
	\end{minipage}
	\caption{Typical profiles of the steady-state variance
		for decoupled discrete-time single integrators (left) and double integrators (right). %with $ \delayn \in \{5,7\} $.
%		The minima are highlighted by markers.
	}
	\label{fig:disc-time-var}
\end{figure}