%!TEX ROOT = ../../centralized_vs_distributed.tex

\done{
\myParagraph{\titlecap{stability analysis}}\label{sec:setup-cont-time-single-int-stability}
\tcb{Mean-square stability} of scalar \tcb{stochastic differential equations} of the form~\eqref{eq:cont-time-single-int-subsystem} has been \tcb{addressed in the literature.}
{We build on the classical result in~\cite{KuchlerLangevinEqs} to characterize consensus stability for the multi-agent formation.} %~\eqref{eq:cont-time-single-int-formation-model}.
%leading to the following result.
\begin{prop}[Stability of CT single integrators]\label{thm:retarded-eq-steady-state}
%	\marginpar{\tiny Multi-agent is explicit now and the conditions hold for the formation.}
	{The network error $ \x{}{t} $
	is mean-square stable} if and only if 
	\begin{equation}\label{eq:cont-time-single-int-variance-condition}
		\gpos_j \in \left(0,\dfrac{\pi}{2\taun}\right), \quad j = 2,\dots,N.
	\end{equation}
	In this case, $ \x{}{t} $ is a  Gaussian process \blue{and its steady-state variance is determined by}
	\begin{equation}\label{eq:cont-time-single-int-steady-state-variance}
		\var(K) = \sum_{j=2}^{N}\varx{\gpos_j}{I}, \quad \varx{\gpos_j}{I} = \dfrac{1+\sin(\gpos_j\taun)}{2\gpos_j\cos(\gpos_j\taun)},
	\end{equation}
	where %we make explicit the dependence of $ \var $ on $ K $
	$ \varx{\gpos_j}{I} $ is the variance of the trivial solution of~\eqref{eq:cont-time-single-int-subsystem}.
\end{prop}

\begin{proof}[Sketch of Proof]
	In view of the decoupling,
	stability of~\eqref{eq:cont-time-single-int-formation-model}
	amounts to stability of all subsystems~\eqref{eq:cont-time-single-int-subsystem}, $ j=1,\dots,N $,
	with the variances of $ \x{}{t} $ and $ \xtilde{}{t} $ being equal.
	Condition~\eqref{eq:cont-time-single-int-variance-condition} and
	expression~\eqref{eq:cont-time-single-int-steady-state-variance}
	were derived in~\cite{KuchlerLangevinEqs}.
\end{proof}

\iffalse
\begin{thm}[\!\!\cite{KuchlerLangevinEqs}]
	The mean vector trivial solution of~\eqref{eq:cont-time-single-int-subsystem},
	\tcb{where $ \noisetilde{}{\cdot} $ is} standard Brownian noise,
	is mean-square stable if and only if 
	\begin{equation}\label{eq:cont-time-single-int-variance-condition}
	\gpos_j \in \left(0,\dfrac{\pi}{2\taun}\right)
	\end{equation}
	In this case, $ \xtilde{j}{t} $ is a zero-mean Gaussian process \tcb{and its steady-state variance is determined by}
	\begin{equation}\label{eq:cont-time-single-int-steady-state-variance}
	\varx{\gpos_j} = \dfrac{1+\sin(\gpos_j\taun)}{2\gpos_j\cos(\gpos_j\taun)}.
	\end{equation}
\end{thm}
\fi

While the variance of delay-free systems is bounded for any %positive semi-definite feedback matrix gain $ K $
positive eigenvalues $ \gpos_2,\dots,\gpos_N $,
%and vanishes when these go to infinity,
the presence of delay constrains a stabilizing control \tcb{according to}~\eqref{eq:cont-time-single-int-variance-condition}.
In fact,
longer delays $ \taun $ induce smaller upper bounds on the eigenvalues.

{The following result will turn useful in the control design.}
%\marginpar{\tiny This corollary is not meaningful now but is used later to assess convexity of the optimization problem and ifor the quadratic approximation.
%Also, it is referred to in Fig. 2 and in the model approximation of double integrator.}
\begin{cor}\label{lem:optimal-variance-explicit}
	\tcb{Let $ \gpos $ satisfy~\eqref{eq:cont-time-single-int-variance-condition}.
		Then} the function $ \varx{\gpos}{I} $ is strictly convex and {the minimizer} $ \opteig $ is {determined} by
	\begin{equation}\label{eq:optimal-variance-closed-form}
	\opteig = \frac{\beta^*}{\taun}, \qquad \beta^* = \cos\beta^*. %, \quad \varx{\opteig}{I} \doteq \dfrac{1+\sin(\beta^*)}{2\beta^*\cos(\beta^*)}\taun
	\end{equation}
%	where $ \beta^* \in \left(0,\nicefrac{\pi}{2}\right) $ is the unique solution of $ \beta = \cos\beta $.
\end{cor}
\begin{proof}
	Follows from standard computations over the derivatives of $ \varx{\cdot}{I} $.
	\if0\mode
	See technical report~\cite{2021arXiv210900359B}.
	\else
	See Appendices A-B in the technical report~\cite{2021arXiv210110394B}.
	\fi
\end{proof}
}