%!TEX ROOT = ../../centralized_vs_distributed.tex

\done{
\myParagraph{\titlecap{stability analysis}}\label{sec:stability-analysis}
We have the following result.
%where $ \x{}{t} $ is the state of the system at time $ t $,
%and $ \noise{}{t} $ is a standard Brownian motion with differential $ d\noisebar{}{t}\sim\gauss(0,dt) $.
%For the sake of simplicity, 
%we normalize the delay
%through the substitutions
%\begin{equation}\label{eq:substitutions-4-normalization}
%	\x{}{t} = \xbar{}{t\tau}, \quad \gvel = \tau\bar{\gvel}, \quad \gpos = \tau\bar{\gpos}, \quad \noise{}{\cdot} = \tau \noisebar{}{\cdot}
%\end{equation}
%which yield the unit-delay equation
%\begin{equation}\label{eq:2nd-order-diff-eq-normalized}
%	\dfrac{d^2\x{}{t}}{dt^2} + \gvel\dfrac{d\x{}{t}}{dt} + \gvel\gpos\x{}{t-1} = \dfrac{d\noise{}{t}}{dt}
%\end{equation}
%The stability of~\eqref{eq:agent-dynamics-1} is characterized as follows.
\begin{prop}[Stability of CT double integrators]\label{prop:cont-time-double-int-stability}
%	Let $ (\gvel,\gpos_i)\in\Realp{2} $, $ j=1,\dots,N $.
	{The network error $ \x{}{t} $ is mean-square stable} if %, for $ j = 2,\dots,N $,
	\begin{equation}\label{eq:cont-time-double-int-stability-condition}
		\gpos_j\in\left(0,\dfrac{\beta}{\sin\beta}\right), \ 
		\gvel = \beta\tan\beta, \ 
		\beta \in \left(0,\frac{\pi}{2}\right), \ j = 2,\dots,N.
	\end{equation}
	Condition~\eqref{eq:cont-time-double-int-stability-condition} can be equivalently written as
	\begin{equation}\label{eq:cont-time-double-int-stability-region}
		\left(\gvel,\gpos_j\right) \in \setstable \doteq
		\left\lbrace (\gvel,\gpos_j)\in\Realp{2} : \gpos_j < \phi(\gvel) \right\rbrace, \ j = 2,\dots,N,
	\end{equation}
	where the implicit function $ \phi(\cdot) $ is concave increasing and
	\begin{equation}\label{eq:cont-time-double-int-phi-description}
		\phi(0) = 1, \quad \lim_{\gvel\rightarrow+\infty}\phi(\gvel) = \dfrac{\pi}{2}.
	\end{equation}
	If $ \exists j\neq1 : (\gvel,\gpos_j) \notin \overline{\setstable} $,
%	where $ \overline{\setstable} $ is the closure of $ \setstable $,
	the system is mean-square unstable.
\end{prop}
\begin{proof}
	\revisiontwo{The proof is based on~\cite{BAPTISTINI1997259}.}
	See~\cref{app:cont-time-double-int-stability}.
\end{proof}
\begin{figure}
	\centering
	\includegraphics[width=.72\linewidth]{cont-time-double-int-var-heatmap}
	\caption{Level curves of the steady-state variance % $ \varx{\gvel,\gpos} $
		for the continuous-time double integrator~\eqref{eq:agent-dynamics-1}
		and points of minimum with fixed derivative gain.
	}
	\label{fig:cont-time-double-int-stab-region}
\end{figure}
\revisiontwo{
	\begin{rem}[Non-normalized delay]
		Under the original delay $ \taun $ in~\eqref{eq:cont-time-double-int-model},
		for $ j = 2,\dots,N $
		condition~\eqref{eq:cont-time-double-int-stability-condition} becomes
		\begin{equation}\label{eq:cont-time-double-int-stability-condition-rewritten}
			\gpos_j\in\left(0,\dfrac{\beta}{\taun\sin\beta}\right), \ 
			\gvel = \dfrac{\beta\tan\beta}{\taun}, \ 
			\beta \in \left(0,\frac{\pi}{2}\right).
		\end{equation}
	\end{rem}
}
%\begin{rem}[Stability conditions for feedback gains]
%While positive feedback gains ensure mean-square
%asymptotic stability for delay-free second-order systems,
%\cref{prop:cont-time-double-int-stability} states that a more restrictive condition
%applies in the presence of delay,
%similarly to the single-integrator case.
\tcb{Similar} to the single-integrator case,
\cref{prop:cont-time-double-int-stability} states that the presence of delay
requires more restrictive conditions
than positive gains.
In words, the system %~\eqref{eq:cont-time-double-int-model}
is stable
if the instantaneous \tcb{component of the}
control input in~\eqref{eq:control-input-PD}
%	(driven by the measured velocity)
is sufficiently \tcb{``strong''} compared to the delayed one.
%	which brings instability if the associated gain is too large
%\end{rem}
%When~\eqref{eq:agent-dynamics-1} is asymptotically stable, % for all $ j $,
The steady-state variance of $ \xtilde{j}{t} $ for $ j\neq1 $ can be computed
\tcb{using}~\cite[Section 4]{wangBoundedness},
\begin{equation}\label{eq:2nd-order-cont-ss-variance}
%	\lim_{t\rightarrow+\infty} \mathbb{E}[x^2(t)]
	\varx{\gvel,\gpos_j}{II} = \dfrac{1}{2\pi}\int_{-\infty}^{+\infty} \dfrac{d\omega}{|-\omega^2 + j\gvel\omega + \gvel\gpos_j\e^{-jw}|^2},
\end{equation}
and $ \var = \var(\gvel,K) = \sum_{j=2}^N \varx{\gvel,\gpos_j}{II}  $.
%which depends on both $ K $ and $ \gvel $.
\tcb{A graphical illustration of the level curves of $ \varx{\gvel,\gpos_j}{II} $ is provided in~\autoref{fig:cont-time-double-int-stab-region}.}

%\autoref{fig:cont-time-double-int-stab-region} shows
%the variance $ \varx{x} $ inside the stability region $ \mathcal{S} $.
}