%!TEX ROOT = ../../centralized_vs_distributed.tex

%\mjmargin{blue color here is lighter than in the rest of the text where changes relative to the original submission have been made}
\done{
\myParagraph{Model Approximation}\label{sec:time-scale-separation}
\revisiontwo{Because embedding integral~\eqref{eq:2nd-order-cont-ss-variance} into an optimization problem is computationally challenging, 
	we provide an alternative tractable formulation that can be used to achieve insight into fundamental performance trade-offs.}
\tcb{As shown in~\cref{app:time-scale-separation},
	when the feedback gain $ \gvel $ is sufficiently high,
	separation of time scales\revisiontwo{~\cite{khalil2002nonlinear}} allows us to approximate~\eqref{eq:agent-dynamics-1} with first-order dynamics,}
\begin{equation}\label{eq:x-dynamics-1st-order-approximation}
	d\xtilde{j}{t} = -\gpos_j\xtilde{j}{t-1}dt + dn(t),
\end{equation}
where the \tcb{variance of Brownian motion $ n(t) $ is} inversely proportional to $ \gvel $.
In words, 
\tcb{when the damping is high enough, the derivative of $\xtilde{j}{t}$ converges to zero much faster than $\xtilde{j}{t}$, 
	which represents the dominant component of the dynamics.
Utility of this approximation is illustrated} in~\autoref{fig:cont-time-double-int-stab-region}:
with fixed $ \bar{\gvel} $, the point of minimum
%$ \argmin_{\gpos_j}\varx{\bar{\gvel},\gpos_j}{II} $
of the corresponding \tcb{1D variance} curve, \ie $ \argmin_{\gpos_j}\varx{\bar{\gvel},\gpos_j}{II} $ (solid black line),
approaches the minimizer $ \opteig $ of the single integrator \tcb{model}
(dashed black, see~\cref{lem:optimal-variance-explicit}) with increase of $ \bar{\gvel} $.
\tcb{We also note that} the variance decreases with $ \gvel $.
}