%!TEX ROOT = centralized_vs_distributed.tex

\subsection{\titlecap{double integrator model}}\label{sec:cont-time-double-int-model}

\done{
We now \tcb{examine networks in which each agent obeys} a second-order dynamics
with the PD control input~\eqref{eq:control-input-PD}:
\begin{equation}\label{eq:cont-time-double-int-model}
	\dfrac{d^2\xbar{i}{t}}{dt^2} = \u{i}{t} + \dfrac{d\noisebar{i}{t}}{dt}.
\end{equation}
%where the $ \u{i}{t} $ is given by.
%\begin{equation}\label{eq:prop-der-control}
%	\u{i}{t} = %-\gvel\dfrac{d\xbar{i}{t}}{dt} - \gvel\u{p,i}{t} = 
%	-\gvel\dfrac{d\x{i}{t}}{dt} + \gvel\u{p,i}{t}
%	\bar{\gvel}\sum_{\ell=1}^{n}\bar{k}_\ell\left[\left(\meas{i}{\ell^+}{t-\delayn}+\meas{i}{\ell^-}{t-\delayn}\right)\right]
%\end{equation}
%where a dependence on theis added. % and  \gvel $ is to be designed.
\tcb{For simplicity,} we normalize the delay by rescaling~\eqref{eq:cont-time-double-int-model}, %follows:
\begin{equation}\label{eq:substitutions-4-normalization}
	\xbar{i}{\cdot} \leftarrow \xbar{i}{\taun\,\cdot}, \ \gvel \leftarrow \taun\gvel, \
	k_\ell \leftarrow \taun k_\ell, \ \noisebar{i}{\cdot} \leftarrow \taun \noisebar{i}{\cdot},
\end{equation}
%\cref{eq:substitutions-4-normalization} shows that any delay system
%can be brought to unit-delay
%by suitably scaling the coefficients.
%In the following, we assume the above normalization be performed in the first place.
%By re-defining the aggregate formation error $ \x{}{t} $ to be
%\begin{equation}\label{eq:multi-agent-state}
%\x{}{t} = \left[\x{1}{t}, \dots, \x{N}{t}, \dfrac{d\x{1}{t}}{dt}, \dots, \dfrac{d\x{N}{t}}{dt}\right]^\top
%\end{equation}
Stacking %the aggregate formation vector
the agent errors and their derivatives
in the formation vector, %$ \nicefrac{d\x{i}{t}}{dt} $
the error dynamics can be
%written in compact form as
%\begin{gather}
%	d\x{}{t} =
%		\left(A_0\x{}{t} + A_1\x{}{t-1}\right)dt + 
%		\begin{bmatrix}
%			0\\
%			I
%		\end{bmatrix}d\noise{}{t}\label{eq:multi-agent-state-space} \\
%	\nonumber
%	A_0 \doteq \begin{bmatrix}
%		0 & I\\
%		0 & -\gvel I
%	\end{bmatrix}, \
%	A_1 \doteq \begin{bmatrix}
%		0 & 0\\
%		-\gvel K & 0
%	\end{bmatrix}
%\end{gather}
%and $ \noise{}{\cdot} $ is composed of the $ N $ independent coordinates $ \noise{i}{\cdot} $.
%System~\eqref{eq:multi-agent-state-space} can be
decoupled as before,
%through the change of basis $ \x{}{t} = (T\otimes I_2)\xtilde{}{t} $,
%yields the following dynamics:
%\begin{align}\label{eq:multi-agent-state-space-1}
%	\begin{split}
%	d\xtilde{}{t} =
%	\left(A_0\xtilde{}{t} + \tilde{A}_1\xtilde{}{t-1}\right)dt + 
%	\begin{bmatrix}
%		0\\
%		T^\top
%	\end{bmatrix}d\noise{}{t}\\
%	\tilde{A}_1 \doteq \begin{bmatrix}
%		0 & 0\\
%		-\gvel \Lambda & 0
%	\end{bmatrix}
%	\end{split}
%\end{align}
%which can be decoupled as follows:
%where the $ i $-th subsystem is
yielding %subsystems of the form
\begin{equation}\label{eq:agent-dynamics-1}
	\dfrac{d^2\xtilde{j}{t}}{dt^2} =  -\gvel\dfrac{d\xtilde{j}{t}}{dt} - \gvel\gpos_j\xtilde{j}{t-1} + \dfrac{d\noisetilde{j}{t}}{dt}.
\end{equation}
%where $ \gpos $ is an eigenvalue of $ K $.
}