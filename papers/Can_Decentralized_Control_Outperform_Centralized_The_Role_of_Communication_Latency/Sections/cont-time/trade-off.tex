%!TEX ROOT = ../../centralized_vs_distributed.tex

\subsection{{\titlecap{ring topology: analytical insight into the trade-off}}}\label{sec:cont-time-single-int-trade-off}

\done{
%In particular, %as the delay increases according to~\cref{ass:hypothesis},
%the steady-state error variance is minimized by a network topology
%whose optimal communication neighborhood size $ n^* $
%is smaller than the maximum number of links,
%%(corresponding to the complete graph, \ie centralized control)
%in general.
%represented by the optimal number of link pairs $ n^* $.
%Beyond such a threshold, further enlarging the feedback loops
%accrues little benefit as opposed to the penalization of the dynamics due to increased latency,
%so that even the optimal control design yields a worse performance than with fewer links.

For \revision{\tcb{a ring topology with continuous-time single-integrator agent dynamics,
	a \tradeoff can be explicitly quantified.
	By utilizing~\cref{prop:subopt-gain} to compute the feedback gains,
	the objective function can be factorized as}}
%	 \mjmargin{roman font for subscripts}
\begin{equation}\label{eq:cont-time-single-int-trade-off-mult}
	\var = \underbrace{f(n)}_{\tilde{J}_{\textrm{latency}}(n)} \cdot \	\underbrace{\sum_{j=2}^N \tilde{C}_{j}^*(n)}_{\tilde{J}_{\textrm{network}}(n)},
\end{equation}
%where the optimal variance $ \rho^* $ only depends on the delay $ \taun $.
%$ \rho^* \doteq \rho(\opteig) $ is the minimum of the variance function associated with the scalar subsystem
where %$ \tilde{K}^* $ is the suboptimal feedback matrix, %with the gains computed as per~\cref{prop:subopt-gain}
$ \sigma_{I}^{2}(\tilde{\gpos}_j^*) =  \tilde{C}_{j}^*(n)\taun $
%with $ \tilde{\gpos}_j^* $ a suboptimal eigenvalue of $ K $,
%with the suboptimal gain $ \tilde{k}^* $ and
and $ \tilde{C}_{j}^*(n) $ only depends on $ n $
and can be computed \tcb{exactly; see Appendix~\ref{app:cont-time-single-int-suboptimal-variance-computation}.} 
%\mjmargin{red sentence is a bit mouthful (e.g., $ \tilde{\gpos}^* $ are linear in $ \opteig $) -- please try to rewrite.}
{This holds because %the gain $ \tilde{k}^* $,
	the suboptimal eigenvalues can be expressed as $ \tilde{\gpos}_j^* = \tilde{c}_j^*(n)\opteig $ %
%are linear in $  $ through coefficients $ $
%that only \tcb{depend} on \tcb{$ n $
	(cf.~\cref{prop:subopt-gain}).}
%where $ \tilde{\gpos}^* = \tilde{c}^*(n)\opteig $ % is used.
%and $ \tilde{c}^*(n) $ only depends on $ n $.
Such a decomposition can be interpreted as a
%The factors $ \tilde{c}^*(n) $ and $ \opteig $
decoupling of \tcb{the impact of network ($ \tilde{c}_j^*(n) $) and latency ($ \opteig $) effects on the control design.}
% being linear . % (c.f.~\cref{prop:subopt-gain}).
%The factors $ \tilde{c}_i^* $ are easily computed from the analytical expression of the spectrum of $ K $
%as the Discrete Fourier Transform of the first row.
%Notice that~\eqref{eq:single-int-variance-minimization-rewritten} is similar to~\eqref{eq:trade-off},
%with the only exception of the additional (increasing) factor $ f(n) $ which multiplies the network-related cost.
%The observed trade-off over $ n $ arises from the opposite behaviors of the two addends in~\eqref{eq:single-int-variance-minimization-rewritten}.
%\autoref{fig:trade-off} illustrates such trends for the curves in Figs.~\ref{fig:cont-time-single-int-opt-var}--\ref{fig:disc-time-single-int-opt-var},
%where the average gap is proportional to the summation in~\eqref{eq:single-int-variance-minimization-rewritten}.
%On the one hand,
%~\autoref{fig:coeff-Ctilde} shows ,
By inspection, it can be seen that $ \tilde{J}_{\textrm{network}}(n) $ is a decreasing function of $n$
		and that $ \tilde{J}_\textrm{latency}(n) $ is determined by $ f(n) $.
		Furthermore, when $ f(\cdot) $ is sublinear,
		the above expression can be equivalently written in form~\eqref{eq:trade-off}, %according to~\eqref{eq:trade-off}:
\begin{equation}\label{eq:cont-time-single-int-trade-off}
	\var = \underbrace{f(n)\cdot\sum_{j=2}^N\left(\tilde{C}_{j}^*(n)-C^*\right)}_{\tcb{J_{\textrm{network}}(n)}} + 
	\underbrace{(N-1)C^*f(n)}_{\tcb{J_{\textrm{latency}}(n)}},
\end{equation}
where $ \varx{\opteig}{I} = C^*\taun $ is the optimal variance \tcb{according to~\eqref{eq:cont-time-single-int-steady-state-variance} and~\cref{lem:optimal-variance-explicit}.}
Indeed,
the summation decreases with superlinear rate,
so that $ J_{\textrm{network}}(n) $ is a decreasing sequence.
The terms in $ J_{\textrm{network}}(n) $,
each associated with a decoupled subsystem~\eqref{eq:cont-time-single-int-subsystem},
illustrate benefits of communication:
as $ n $ increases, the eigenvalues of $ K $ have more degrees of freedom
and can squeeze more tightly about $ \opteig $,
reducing performance gaps between subsystems and theoretical optimum.
We note that $ J_{\textrm{network}}(n) $ vanishes for the fully connected architecture.

%\marginpar{\vspace{-2cm}\tiny I reduced this comment and tried to sharpen a little bit to reinforce our thesis is valid for general systems.
%	This is not the main point of the section though, so if you feel this is not strong enough you may remove it.}
{Even though analogous expressions could not be obtained for other dynamics,
	the curves in~\autoref{fig:opt-var} exhibit trade-offs which are consistent with the above analysis.}
%corroborating out thesis.}

%\begin{figure}
%	\centering
%	\begin{minipage}[l]{.48\linewidth}
%		\centering
%		\includegraphics[width=\linewidth]{coeff_tildeC_gap}
%		\caption{Network coefficient $ \tilde{C}^*\doteq\sum_{i=2}^N\left(\tilde{C}_i^*(n)-C^*\right) $.}
%		\label{fig:coeff-Ctilde}
%	\end{minipage}%
%	\hfill
%	\begin{minipage}[r]{.48\linewidth}
%		\centering
%		\includegraphics[width=\linewidth]{opt-gain-k1-k2}
%		\caption{Ratio between optimal gain $ k_\ell^* $, $ \ell\in\{1,2\} $, and $ \opteig $.}
%		\label{fig:opt-gain-k}
%	\end{minipage}
%\end{figure}

\iffalse
	\mjmargin{I feel that statements in this remark should be sharpened. Alternatively, you can remove it. One thing to keep in mind is that $J_{\textrm{network}}$ would not vanish even with a fully connected topology if you penalize control effort in the objective function.}
	\canOmit{
\begin{rem}
	Even though the trade-off could not be written analytically 
	for the minimum-variance control,
	this seems to be the case.
	In particular, numerical tests yield the network-latency split structure $ c^*(n)\opteig $ %of the form $ c^*(n)\opteig $
	also for the optimal eigenvalues.
%	This is illustrated in~\autoref{fig:opt-gain-k}, where the rate $ f(\cdot) $,
%	and hence the delay $ \taun $,
%	does not affect the ratios $ \nicefrac{k_\ell^*}{\opteig} $.
%	Such an expression reinforces our thesis that the cost can be decomposed according to~\eqref{eq:trade-off}.
	\autoref{fig:trade-off} shows the costs 
	defined in~\eqref{eq:cont-time-single-int-trade-off} for
	the solution of~\eqref{eq:cont-time-single-int-variance-minimization}.
\end{rem}
}
\fi
}