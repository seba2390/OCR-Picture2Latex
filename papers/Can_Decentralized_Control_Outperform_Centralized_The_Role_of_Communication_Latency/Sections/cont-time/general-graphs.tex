%!TEX ROOT = centralized_vs_distributed.tex

\done{
\subsection{\titlecap{General symmetric network topology}}\label{sec:generic-topology}
Even though we \revision{utilized ring topology to derive analytical results (see~\autoref{sec:cont-time-single-int-trade-off})},
the control design can be extended to general undirected networks with symmetric feedback gain matrices $ K $.
For the single integrator model, this reads
\begin{equation}\label{eq:generic-problem}
K^* = \argmin_K \; \var(K).
\end{equation}
\tcb{The steady-state} {network error} variance $ \var(K) $ is a convex \tcb{function}
if and only if $ \varx{\gpos_j}{I} $ is convex~\cite{davis1957all},
which is proved in~\cref{lem:optimal-variance-explicit} for continuous-time
and \tcb{in~\cref{app:disc-time-single-int-variance-explicit} for} discrete-time systems. %and the system eigenvalues
%and is thus convex in the feedback gains
The optimal gains can then be found numerically via gradient-based methods, where \tcb{gradients of the eigenvalues can be computed using analytical~\cite{doi:10.2514/3.7211,doi:10.2514/2.1119} or numerical~\cite{10.1115/1.2888195} methods.}
On the other hand,
the derivative {feedback gain in $ \varx{\gvel,\gpos_j}{II} $ prevents us from establishing convexity for second-order systems in general.}
However, if $ \varx{\gvel,\gpos_j}{II} $ is convex in each coordinate%
\footnote{This can be checked for discrete-time double integrators, see~\cref{app:disc-time-single-int-variance-explicit}.},
the design \tcb{problem} can be solved by alternatively optimizing proportional and derivative gains \tcb{and the \tradeoff can be studied irrespective of the particular topology.}}