%!TEX ROOT = ../../centralized_vs_distributed.tex

\section{{\titlecap{the centralized-distributed trade-off}}}\label{sec:numerical-results}

\revision{In the previous sections we formulated the optimal control problem for a given controller architecture
(\ie the number of links) parametrized by $ n $
and showed how to compute minimum-variance objective function and the corresponding constraints.
In this section, we present our main result:
%\red{for a ring topology with multiple options for the parameter $ n $},
we solve the optimal control problem for each $ n $ and compare the best achievable closed-loop performance with different control architectures.\footnote{
\revision{Recall that small (large) values of $ n $ mean sparse (dense) architectures.}}
For delays that increase linearly with $n$,
\ie $ f(n) \propto n $, 
we demonstrate that distributed controllers with} {few communication links outperform controllers with larger number of communication links.}

\textcolor{subsectioncolor}{Figure~\ref{fig:cont-time-single-int-opt-var}} shows the steady-state variances
obtained with single-integrator dynamics~\eqref{eq:cont-time-single-int-variance-minimization}
%where we compare the standard multi-parameter design 
%with a simplified version \tcb{that utilizes spatially-constant feedback gains
and the quadratic approximation~\eqref{eq:quadratic-approximation} for \revision{ring topology}
with $ N = 50 $ nodes. % and $ n\in\{1,\dots,10\} $.
%with $ N = 50 $, $ f(n) = n $ and $ \tau_{\textit{min}} = 0.1 $.
%\autoref{fig:cont-time-single-int-err} shows the relative error, defined as
%\begin{equation}\label{eq:relative-error}
%	e \doteq \dfrac{\optvarx-\optvar}{\optvar}
%\end{equation}
%where $ \optvar $ and $ \optvarx $ denote the the optimal and sub-optimal scalar variances, respectively.
%The performance gap is small
%and becomes negligible for large $ n $.
{The best performance is achieved for a sparse architecture with  $ n = 2 $ 
in which each agent communicates with the two closest pairs of neighboring nodes. 
This should be compared and contrasted to nearest-neighbor and all-to-all 
communication topologies which induce higher closed-loop variances. 
Thus, 
the advantage of introducing additional communication links diminishes 
beyond}
{a certain threshold because of communication delays.}

%For a linear increase in the delay,
\textcolor{subsectioncolor}{Figure~\ref{fig:cont-time-double-int-opt-var}} shows that the use of approximation~\eqref{eq:cont-time-double-int-min-var-simplified} with $ \tilde{\gvel}^* = 70 $
identifies nearest-neighbor information exchange as the {near-optimal} architecture for a double-integrator model
with ring topology. 
This can be explained by noting that the variance of the process noise $ n(t) $
in the reduced model~\eqref{eq:x-dynamics-1st-order-approximation}
is proportional to $ \nicefrac{1}{\gvel} $ and thereby to $ \taun $,
according to~\eqref{eq:substitutions-4-normalization},
making the variance scale with the delay.

%\mjmargin{i feel that we need to comment about different results that we obtained for CT and DT double-intergrator dynamics (monotonic deterioration of performance for the former and oscillations for the latter)}
\revision{\textcolor{subsectioncolor}{Figures~\ref{fig:disc-time-single-int-opt-var}--\ref{fig:disc-time-double-int-opt-var}}
show the results obtained by solving the optimal control problem for discrete-time dynamics.
%which exhibit similar trade-offs.
The oscillations about the minimum in~\autoref{fig:disc-time-double-int-opt-var}
are compatible with the investigated \tradeoff~\eqref{eq:trade-off}:
in general, 
the sum of two monotone functions does not have a unique local minimum.
Details about discrete-time systems are deferred to~\autoref{sec:disc-time}.
Interestingly,
double integrators with continuous- (\autoref{fig:cont-time-double-int-opt-var}) ad discrete-time (\autoref{fig:disc-time-double-int-opt-var}) dynamics
exhibits very different trade-off curves,
whereby performance monotonically deteriorates for the former and oscillates for the latter.
While a clear interpretation is difficult because there is no explicit expression of the variance as a function of $ n $,
one possible explanation might be the first-order approximation used to compute gains in the continuous-time case.
%which reinforce our thesis exposed in~\autoref{sec:contribution}.

%\begin{figure}
%	\centering
%	\includegraphics[width=.6\linewidth]{cont-time-double-int-opt-var-n}
%	\caption{Steady-state scalar variance for continuous-time double integrators with $ \taun = 0.1n $.
%		Here, the \tradeoff is optimized by nearest-neighbor interaction.
%	}
%	\label{fig:cont-time-double-int-opt-var-lin}
%\end{figure}
}

\begin{figure}
	\centering
	\begin{minipage}[l]{.5\linewidth}
		\centering
		\includegraphics[width=\linewidth]{random-graph}
	\end{minipage}%
	\begin{minipage}[r]{.5\linewidth}
		\centering
		\includegraphics[width=\linewidth]{disc-time-single-int-random-graph-opt-var}
	\end{minipage}
	\caption{Network topology and its optimal {closed-loop} variance.}
	\label{fig:general-graph}
\end{figure}

Finally,
\autoref{fig:general-graph} shows the optimization results for a random graph topology with discrete-time single integrator agents. % with a linear increase in the delay, $ \taun = n $.
Here, $ n $ denotes the number of communication hops in the ``original" network, shown in~\autoref{fig:general-graph}:
as $ n $ increases, each agent can first communicate with its nearest neighbors,
then with its neighbors' neighbors, and so on. For a control architecture that utilizes different feedback gains for each communication link
	(\ie we only require $ K = K^\top $) we demonstrate that, in this case, two communication hops provide optimal closed-loop performance. % of the system.}

Additional computational experiments performed with different rates $ f(\cdot) $ show that the optimal number of links increases for slower rates: 
for example, 
the optimal number of links is larger for $ f(n) = \sqrt{n} $ than for $ f(n) = n $. 
\revision{These results are not reported because of space limitations.}