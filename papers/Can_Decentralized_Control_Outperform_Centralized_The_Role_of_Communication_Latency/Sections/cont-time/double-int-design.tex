%!TEX ROOT = centralized_vs_distributed.tex

\done{
\myParagraph{\titlecap{double integrator model}}
%As for the double-integrator agent dynamics,
\tcb{Approximation~\eqref{eq:x-dynamics-1st-order-approximation}} and~\autoref{fig:cont-time-double-int-stab-region}
show that,
for sufficiently large $ \gvel $,
the variance of the double-integrator subsystem~\eqref{eq:agent-dynamics-1}
	%vanishes for \tcb{infinitely large derivative feedback gain, $ \gvel=+\infty $.
%	The same approximation implies that,	
%	the variance of the 
	has structure similar to the single integrator, \ie $ \varx{\gvel,\gpos_j}{II} \approx c \varx{\gpos_j}{I} $ for some ``small" $ c > 0 $.
%	the proportional gains should be designed as for the single-integrator case.
%\mjmargin{I don't understand the red equation. Perhaps a better way to write it is given below?}
Thus, we approximate the control design~\eqref{eq:variance-minimization-PD} as
\begin{equation}\label{eq:cont-time-double-int-min-var-simplified}
%	\argmin_{\gvel,\{k_\ell\}_{\ell=1}^n} \; \lim_{t\rightarrow+\infty} \mathbb{E}\left[\lVert\x{}{t}\rVert^2\right] \ \approx \
%\argmin_{\gvel,\{k_\ell\}_{\ell=1}^n} \; \sum_{j=2}^N \varx{\gvel,\gpos}{II} \ \approx \
\tilde{\gvel}^*, \, \argmin_{\{k_\ell\}_{\ell=1}^n} \; \sum_{j=2}^N \varx{\gpos_j}{I},
\end{equation}
where $ \tilde{\gvel}^* $ is chosen beforehand so that
the time-scale separation \tcb{argument provides} a reasonable approximation~\eqref{eq:x-dynamics-1st-order-approximation}.
%approximation~\eqref{eq:x-dynamics-1st-order} is good enough.
In particular, the \tcb{optimization problem for proportional feedback gains in~\eqref{eq:cont-time-double-int-min-var-simplified}}
coincides with the control design for single integrators~\eqref{eq:cont-time-single-int-variance-minimization-decoupled},
with the exception that the stability condition is now \tcb{given by $ \gpos_j < \phi(\tilde{\gvel}^*) $,
	$ j = 2,\dots,N $; see~\eqref{eq:cont-time-double-int-stability-region}.}
%In particular, the feasibility constraint tends to $ \gpos_M < \nicefrac{\pi}{2} $ as $ \tilde{\gvel}^* $ grows.
%namely the constraint for single integrators.

\iffalse
\canOmit{Alternatively, letting $ \tilde{\gvel}^* = +\infty $ at first,
one can choose the single-integrator optimal gains $ \{k_\ell\}_{\ell=1}^n $
according to~\eqref{eq:cont-time-single-int-variance-minimization-decoupled},
and then select a finite value $ \tilde{\gvel}^* > \phi^{-1}(\gpos_M^*) $ to guarantee stability,
where $ \gpos_M^* $ is the optimal spectral radius of $ K $.}
\setlength\marginparwidth{25pt}
\marginpar{\vspace{-4em} \small can \\ remove}
\fi

\begin{rem}[Convexity enables comparison]
	\revision{Convexity of the optimal control design
		problems~\eqref{eq:cont-time-single-int-variance-minimization-decoupled}--\eqref{eq:cont-time-double-int-min-var-simplified}
		enables both efficient numerical computations 
		of the optimal feedback gains for \textit{given} $ n $
		and fair comparison of the best achievable performance for \textit{different} values of $ n $.}
\end{rem}

\begin{rem}[Gain scaling]
	The \tcb{optimal feedback} gains $ \{k_\ell^*\}_{\ell=1}^n $ and $ \tilde{\eta}^* $ 
	are to be scaled \tcb{by} $ \nicefrac{1}{\taun}$
	according to~\eqref{eq:substitutions-4-normalization}.
	%In particular, long delays induce small gains.
\end{rem}

\begin{rem}[Optimal design for double integrators]
	%	Given any value of $ \gvel $,
	%	Problem~\eqref{eq:cont-time-double-int-min-var-simplified}
	\tcb{Local minimizer of the original problem approximated by~\eqref{eq:cont-time-double-int-min-var-simplified}
		can be solved using the gradient-based method proposed in~\cite{8358743}.
		However, 
		this approach has no guarantees of global optimality
		and its computational %further, even if it has polynomial
		complexity is impractical for large-scale systems.
		In contrast,
		convex approximation~\eqref{eq:cont-time-double-int-min-var-simplified} draws a parallel 
		to the optimal design for the single-integrator model and provides insight into a \tradeoff.} % embedded in such networked systems.
\end{rem}
}