%!TEX ROOT = centralized_vs_distributed.tex

\section{\titlecap{Continuous-time agent dynamics}}\label{sec:cont-time}

\begin{comment}
	\done{
	\tcb{We first examine continuous-time models with single- (\autoref{sec:cont-time-single-int-model}) and double-integrator (\autoref{sec:cont-time-double-int-model}) agent dynamics. In~\autoref{sec:cont-time-single-int-control-design} we address the minimum-variance control design problem, in~\autoref{sec:numerical-results} we conduct computational experiments, and in~\autoref{sec:cont-time-single-int-trade-off} we explicitly quantify the impact of delay on the performance of distributed and centralized control strategies.}}
\end{comment}

%\mjmargin{it would be good to be specific about the meaning of "next results". Results in this section? The rest of the paper?}
\revision{We now examine continuous-time {networks} with single- (\autoref{sec:cont-time-single-int-model}) 
and double-integrator (\autoref{sec:cont-time-double-int-model}) agent dynamics,
{derive} conditions for mean-square stability,
and {compute} the steady-state variance of {a stochastically forced system}.
These {developments are} instrumental {for the formulation of the control design problem which is} used to compare different {control} architectures. 
In the optimal control problem,
the steady-state variance determines the objective function and stability conditions represent constraints.
While we first formulate and solve the problem for continuous-time dynamics, 
our results also hold for discrete-time systems;
see~\autoref{sec:disc-time}.
Also,
all \linebreak results in this section hold for generic undirected topologies.}