%%
%% This is file `sample-sigconf.tex',
%% generated with the docstrip utility.
%%
%% The original source files were:
%%
%% samples.dtx  (with options: `sigconf')
%% 
%% IMPORTANT NOTICE:
%% 
%% For the copyright see the source file.
%% 
%% Any modified versions of this file must be renamed
%% with new filenames distinct from sample-sigconf.tex.
%% 
%% For distribution of the original source see the terms
%% for copying and modification in the file samples.dtx.
%% 
%% This generated file may be distributed as long as the
%% original source files, as listed above, are part of the
%% same distribution. (The sources need not necessarily be
%% in the same archive or directory.)
%%
%%
%% Commands for TeXCount
%TC:macro \cite [option:text,text]
%TC:macro \citep [option:text,text]
%TC:macro \citet [option:text,text]
%TC:envir table 0 1
%TC:envir table* 0 1
%TC:envir tabular [ignore] word
%TC:envir displaymath 0 word
%TC:envir math 0 word
%TC:envir comment 0 0
%%
%%
%% The first command in your LaTeX source must be the \documentclass
%% command.
%%
%% For submission and review of your manuscript please change the
%% command to \documentclass[manuscript, screen, review]{acmart}.
%%
%% When submitting camera ready or to TAPS, please change the command
%% to \documentclass[sigconf]{acmart} or whichever template is required
%% for your publication.
%%
%%
\documentclass[sigconf]{acmart}
% \documentclass[sigconf,review,]{acmart}
%%
%% \BibTeX command to typeset BibTeX logo in the docs
\AtBeginDocument{%
  \providecommand\BibTeX{{%
    Bib\TeX}}}
%% Rights management information.  This information is sent to you
%% when you complete the rights form.  These commands have SAMPLE
%% values in them; it is your responsibility as an author to replace
%% the commands and values with those provided to you when you
%% complete the rights form.
\setcopyright{acmcopyright}
\copyrightyear{2023}
\acmYear{2023}
\acmDOI{XXXXXXX.XXXXXXX}

%% These commands are for a PROCEEDINGS abstract or paper.
% \acmConference[Conference acronym 'XX]{Make sure to enter the correct
%   conference title from your rights confirmation emai}{June 03--05,
%   2018}{Woodstock, NY}



\acmConference[Multimodal KDD '23]{29TH ACM SIGKDD CONFERENCE ON KNOWLEDGE DISCOVERY AND DATA MINING}{August 07, 2023}{Long Beach, CA}
\acmBooktitle{29TH ACM SIGKDD CONFERENCE ON KNOWLEDGE DISCOVERY AND DATA MINING (Multimodal KDD '23), August 07, 2023, Long Beach, CA}
\acmPrice{15.00}
\acmISBN{978-1-4503-XXXX-X/18/06}
%%
%%  Uncomment \acmBooktitle if the title of the proceedings is different
%%  from ``Proceedings of ...''!
%%
%%\acmBooktitle{Woodstock '18: ACM Symposium on Neural Gaze Detection,
%%  June 03--05, 2018, Woodstock, NY}
% \acmPrice{15.00}
% \acmISBN{978-1-4503-XXXX-X/18/06}


%%
%% Submission ID.
%% Use this when submitting an article to a sponsored event. You'll
%% receive a unique submission ID from the organizers
%% of the event, and this ID should be used as the parameter to this command.
% \acmSubmissionID{\TODO{TODO}}

%%
%% For managing citations, it is recommended to use bibliography
%% files in BibTeX format.
%%
%% You can then either use BibTeX with the ACM-Reference-Format style,
%% or BibLaTeX with the acmnumeric or acmauthoryear sytles, that include
%% support for advanced citation of software artefact from the
%% biblatex-software package, also separately available on CTAN.
%%
%% Look at the sample-*-biblatex.tex files for templates showcasing
%% the biblatex styles.
%%

%%
%% The majority of ACM publications use numbered citations and
%% references.  The command \citestyle{authoryear} switches to the
%% "author year" style.
%%
%% If you are preparing content for an event
%% sponsored by ACM SIGGRAPH, you must use the "author year" style of
%% citations and references.
%% Uncommenting
%% the next command will enable that style.
%%\citestyle{acmauthoryear}


%%
%% end of the preamble, start of the body of the document source.

%%%%% NEW MATH DEFINITIONS %%%%%

\usepackage{amsmath,amsfonts,bm}

% Mark sections of captions for referring to divisions of figures
\newcommand{\figleft}{{\em (Left)}}
\newcommand{\figcenter}{{\em (Center)}}
\newcommand{\figright}{{\em (Right)}}
\newcommand{\figtop}{{\em (Top)}}
\newcommand{\figbottom}{{\em (Bottom)}}
\newcommand{\captiona}{{\em (a)}}
\newcommand{\captionb}{{\em (b)}}
\newcommand{\captionc}{{\em (c)}}
\newcommand{\captiond}{{\em (d)}}

% Highlight a newly defined term
\newcommand{\newterm}[1]{{\bf #1}}


% Figure reference, lower-case.
\def\figref#1{figure~\ref{#1}}
% Figure reference, capital. For start of sentence
\def\Figref#1{Figure~\ref{#1}}
\def\twofigref#1#2{figures \ref{#1} and \ref{#2}}
\def\quadfigref#1#2#3#4{figures \ref{#1}, \ref{#2}, \ref{#3} and \ref{#4}}
% Section reference, lower-case.
\def\secref#1{section~\ref{#1}}
% Section reference, capital.
\def\Secref#1{Section~\ref{#1}}
% Reference to two sections.
\def\twosecrefs#1#2{sections \ref{#1} and \ref{#2}}
% Reference to three sections.
\def\secrefs#1#2#3{sections \ref{#1}, \ref{#2} and \ref{#3}}
% Reference to an equation, lower-case.
% \def\eqref#1{equation~\ref{#1}}
 \def\eqref#1{(\ref{#1})}
% Reference to an equation, upper case
\def\Eqref#1{Equation~\ref{#1}}
% A raw reference to an equation---avoid using if possible
\def\plaineqref#1{\ref{#1}}
% Reference to a chapter, lower-case.
\def\chapref#1{chapter~\ref{#1}}
% Reference to an equation, upper case.
\def\Chapref#1{Chapter~\ref{#1}}
% Reference to a range of chapters
\def\rangechapref#1#2{chapters\ref{#1}--\ref{#2}}
% Reference to an algorithm, lower-case.
\def\algref#1{algorithm~\ref{#1}}
% Reference to an algorithm, upper case.
\def\Algref#1{Algorithm~\ref{#1}}
\def\twoalgref#1#2{algorithms \ref{#1} and \ref{#2}}
\def\Twoalgref#1#2{Algorithms \ref{#1} and \ref{#2}}
% Reference to a part, lower case
\def\partref#1{part~\ref{#1}}
% Reference to a part, upper case
\def\Partref#1{Part~\ref{#1}}
\def\twopartref#1#2{parts \ref{#1} and \ref{#2}}

\def\ceil#1{\lceil #1 \rceil}
\def\floor#1{\lfloor #1 \rfloor}
\def\1{\bm{1}}
\newcommand{\train}{\mathcal{D}}
\newcommand{\valid}{\mathcal{D_{\mathrm{valid}}}}
\newcommand{\test}{\mathcal{D_{\mathrm{test}}}}

\def\eps{{\epsilon}}


% Random variables
\def\reta{{\textnormal{$\eta$}}}
\def\ra{{\textnormal{a}}}
\def\rb{{\textnormal{b}}}
\def\rc{{\textnormal{c}}}
\def\rd{{\textnormal{d}}}
\def\re{{\textnormal{e}}}
\def\rf{{\textnormal{f}}}
\def\rg{{\textnormal{g}}}
\def\rh{{\textnormal{h}}}
\def\ri{{\textnormal{i}}}
\def\rj{{\textnormal{j}}}
\def\rk{{\textnormal{k}}}
\def\rl{{\textnormal{l}}}
% rm is already a command, just don't name any random variables m
\def\rn{{\textnormal{n}}}
\def\ro{{\textnormal{o}}}
\def\rp{{\textnormal{p}}}
\def\rq{{\textnormal{q}}}
\def\rr{{\textnormal{r}}}
\def\rs{{\textnormal{s}}}
\def\rt{{\textnormal{t}}}
\def\ru{{\textnormal{u}}}
\def\rv{{\textnormal{v}}}
\def\rw{{\textnormal{w}}}
\def\rx{{\textnormal{x}}}
\def\ry{{\textnormal{y}}}
\def\rz{{\textnormal{z}}}

% Random vectors
\def\rvepsilon{{\mathbf{\epsilon}}}
\def\rvtheta{{\mathbf{\theta}}}
\def\rva{{\mathbf{a}}}
\def\rvb{{\mathbf{b}}}
\def\rvc{{\mathbf{c}}}
\def\rvd{{\mathbf{d}}}
\def\rve{{\mathbf{e}}}
\def\rvf{{\mathbf{f}}}
\def\rvg{{\mathbf{g}}}
\def\rvh{{\mathbf{h}}}
\def\rvu{{\mathbf{i}}}
\def\rvj{{\mathbf{j}}}
\def\rvk{{\mathbf{k}}}
\def\rvl{{\mathbf{l}}}
\def\rvm{{\mathbf{m}}}
\def\rvn{{\mathbf{n}}}
\def\rvo{{\mathbf{o}}}
\def\rvp{{\mathbf{p}}}
\def\rvq{{\mathbf{q}}}
\def\rvr{{\mathbf{r}}}
\def\rvs{{\mathbf{s}}}
\def\rvt{{\mathbf{t}}}
\def\rvu{{\mathbf{u}}}
\def\rvv{{\mathbf{v}}}
\def\rvw{{\mathbf{w}}}
\def\rvx{{\mathbf{x}}}
\def\rvy{{\mathbf{y}}}
\def\rvz{{\mathbf{z}}}

% Elements of random vectors
\def\erva{{\textnormal{a}}}
\def\ervb{{\textnormal{b}}}
\def\ervc{{\textnormal{c}}}
\def\ervd{{\textnormal{d}}}
\def\erve{{\textnormal{e}}}
\def\ervf{{\textnormal{f}}}
\def\ervg{{\textnormal{g}}}
\def\ervh{{\textnormal{h}}}
\def\ervi{{\textnormal{i}}}
\def\ervj{{\textnormal{j}}}
\def\ervk{{\textnormal{k}}}
\def\ervl{{\textnormal{l}}}
\def\ervm{{\textnormal{m}}}
\def\ervn{{\textnormal{n}}}
\def\ervo{{\textnormal{o}}}
\def\ervp{{\textnormal{p}}}
\def\ervq{{\textnormal{q}}}
\def\ervr{{\textnormal{r}}}
\def\ervs{{\textnormal{s}}}
\def\ervt{{\textnormal{t}}}
\def\ervu{{\textnormal{u}}}
\def\ervv{{\textnormal{v}}}
\def\ervw{{\textnormal{w}}}
\def\ervx{{\textnormal{x}}}
\def\ervy{{\textnormal{y}}}
\def\ervz{{\textnormal{z}}}

% Random matrices
\def\rmA{{\mathbf{A}}}
\def\rmB{{\mathbf{B}}}
\def\rmC{{\mathbf{C}}}
\def\rmD{{\mathbf{D}}}
\def\rmE{{\mathbf{E}}}
\def\rmF{{\mathbf{F}}}
\def\rmG{{\mathbf{G}}}
\def\rmH{{\mathbf{H}}}
\def\rmI{{\mathbf{I}}}
\def\rmJ{{\mathbf{J}}}
\def\rmK{{\mathbf{K}}}
\def\rmL{{\mathbf{L}}}
\def\rmM{{\mathbf{M}}}
\def\rmN{{\mathbf{N}}}
\def\rmO{{\mathbf{O}}}
\def\rmP{{\mathbf{P}}}
\def\rmQ{{\mathbf{Q}}}
\def\rmR{{\mathbf{R}}}
\def\rmS{{\mathbf{S}}}
\def\rmT{{\mathbf{T}}}
\def\rmU{{\mathbf{U}}}
\def\rmV{{\mathbf{V}}}
\def\rmW{{\mathbf{W}}}
\def\rmX{{\mathbf{X}}}
\def\rmY{{\mathbf{Y}}}
\def\rmZ{{\mathbf{Z}}}

% Elements of random matrices
\def\ermA{{\textnormal{A}}}
\def\ermB{{\textnormal{B}}}
\def\ermC{{\textnormal{C}}}
\def\ermD{{\textnormal{D}}}
\def\ermE{{\textnormal{E}}}
\def\ermF{{\textnormal{F}}}
\def\ermG{{\textnormal{G}}}
\def\ermH{{\textnormal{H}}}
\def\ermI{{\textnormal{I}}}
\def\ermJ{{\textnormal{J}}}
\def\ermK{{\textnormal{K}}}
\def\ermL{{\textnormal{L}}}
\def\ermM{{\textnormal{M}}}
\def\ermN{{\textnormal{N}}}
\def\ermO{{\textnormal{O}}}
\def\ermP{{\textnormal{P}}}
\def\ermQ{{\textnormal{Q}}}
\def\ermR{{\textnormal{R}}}
\def\ermS{{\textnormal{S}}}
\def\ermT{{\textnormal{T}}}
\def\ermU{{\textnormal{U}}}
\def\ermV{{\textnormal{V}}}
\def\ermW{{\textnormal{W}}}
\def\ermX{{\textnormal{X}}}
\def\ermY{{\textnormal{Y}}}
\def\ermZ{{\textnormal{Z}}}

% Vectors
\def\vzero{{\bm{0}}}
\def\vone{{\bm{1}}}
\def\vmu{{\bm{\mu}}}
\def\vtheta{{\bm{\theta}}}
\def\va{{\bm{a}}}
\def\vb{{\bm{b}}}
\def\vc{{\bm{c}}}
\def\vd{{\bm{d}}}
\def\ve{{\bm{e}}}
\def\vf{{\bm{f}}}
\def\vg{{\bm{g}}}
\def\vh{{\bm{h}}}
\def\vi{{\bm{i}}}
\def\vj{{\bm{j}}}
\def\vk{{\bm{k}}}
\def\vl{{\bm{l}}}
\def\vm{{\bm{m}}}
\def\vn{{\bm{n}}}
\def\vo{{\bm{o}}}
\def\vp{{\bm{p}}}
\def\vq{{\bm{q}}}
\def\vr{{\bm{r}}}
\def\vs{{\bm{s}}}
\def\vt{{\bm{t}}}
\def\vu{{\bm{u}}}
\def\vv{{\bm{v}}}
\def\vw{{\bm{w}}}
\def\vx{{\bm{x}}}
\def\vy{{\bm{y}}}
\def\vz{{\bm{z}}}

% Elements of vectors
\def\evalpha{{\alpha}}
\def\evbeta{{\beta}}
\def\evepsilon{{\epsilon}}
\def\evlambda{{\lambda}}
\def\evomega{{\omega}}
\def\evmu{{\mu}}
\def\evpsi{{\psi}}
\def\evsigma{{\sigma}}
\def\evtheta{{\theta}}
\def\eva{{a}}
\def\evb{{b}}
\def\evc{{c}}
\def\evd{{d}}
\def\eve{{e}}
\def\evf{{f}}
\def\evg{{g}}
\def\evh{{h}}
\def\evi{{i}}
\def\evj{{j}}
\def\evk{{k}}
\def\evl{{l}}
\def\evm{{m}}
\def\evn{{n}}
\def\evo{{o}}
\def\evp{{p}}
\def\evq{{q}}
\def\evr{{r}}
\def\evs{{s}}
\def\evt{{t}}
\def\evu{{u}}
\def\evv{{v}}
\def\evw{{w}}
\def\evx{{x}}
\def\evy{{y}}
\def\evz{{z}}

% Matrix
\def\mA{{\bm{A}}}
\def\mB{{\bm{B}}}
\def\mC{{\bm{C}}}
\def\mD{{\bm{D}}}
\def\mE{{\bm{E}}}
\def\mF{{\bm{F}}}
\def\mG{{\bm{G}}}
\def\mH{{\bm{H}}}
\def\mI{{\bm{I}}}
\def\mJ{{\bm{J}}}
\def\mK{{\bm{K}}}
\def\mL{{\bm{L}}}
\def\mM{{\bm{M}}}
\def\mN{{\bm{N}}}
\def\mO{{\bm{O}}}
\def\mP{{\bm{P}}}
\def\mQ{{\bm{Q}}}
\def\mR{{\bm{R}}}
\def\mS{{\bm{S}}}
\def\mT{{\bm{T}}}
\def\mU{{\bm{U}}}
\def\mV{{\bm{V}}}
\def\mW{{\bm{W}}}
\def\mX{{\bm{X}}}
\def\mY{{\bm{Y}}}
\def\mZ{{\bm{Z}}}
\def\mBeta{{\bm{\beta}}}
\def\mPhi{{\bm{\Phi}}}
\def\mLambda{{\bm{\Lambda}}}
\def\mSigma{{\bm{\Sigma}}}

% Tensor
\DeclareMathAlphabet{\mathsfit}{\encodingdefault}{\sfdefault}{m}{sl}
\SetMathAlphabet{\mathsfit}{bold}{\encodingdefault}{\sfdefault}{bx}{n}
\newcommand{\tens}[1]{\bm{\mathsfit{#1}}}
\def\tA{{\tens{A}}}
\def\tB{{\tens{B}}}
\def\tC{{\tens{C}}}
\def\tD{{\tens{D}}}
\def\tE{{\tens{E}}}
\def\tF{{\tens{F}}}
\def\tG{{\tens{G}}}
\def\tH{{\tens{H}}}
\def\tI{{\tens{I}}}
\def\tJ{{\tens{J}}}
\def\tK{{\tens{K}}}
\def\tL{{\tens{L}}}
\def\tM{{\tens{M}}}
\def\tN{{\tens{N}}}
\def\tO{{\tens{O}}}
\def\tP{{\tens{P}}}
\def\tQ{{\tens{Q}}}
\def\tR{{\tens{R}}}
\def\tS{{\tens{S}}}
\def\tT{{\tens{T}}}
\def\tU{{\tens{U}}}
\def\tV{{\tens{V}}}
\def\tW{{\tens{W}}}
\def\tX{{\tens{X}}}
\def\tY{{\tens{Y}}}
\def\tZ{{\tens{Z}}}


% Graph
\def\gA{{\mathcal{A}}}
\def\gB{{\mathcal{B}}}
\def\gC{{\mathcal{C}}}
\def\gD{{\mathcal{D}}}
\def\gE{{\mathcal{E}}}
\def\gF{{\mathcal{F}}}
\def\gG{{\mathcal{G}}}
\def\gH{{\mathcal{H}}}
\def\gI{{\mathcal{I}}}
\def\gJ{{\mathcal{J}}}
\def\gK{{\mathcal{K}}}
\def\gL{{\mathcal{L}}}
\def\gM{{\mathcal{M}}}
\def\gN{{\mathcal{N}}}
\def\gO{{\mathcal{O}}}
\def\gP{{\mathcal{P}}}
\def\gQ{{\mathcal{Q}}}
\def\gR{{\mathcal{R}}}
\def\gS{{\mathcal{S}}}
\def\gT{{\mathcal{T}}}
\def\gU{{\mathcal{U}}}
\def\gV{{\mathcal{V}}}
\def\gW{{\mathcal{W}}}
\def\gX{{\mathcal{X}}}
\def\gY{{\mathcal{Y}}}
\def\gZ{{\mathcal{Z}}}

% Sets
\def\sA{{\mathbb{A}}}
\def\sB{{\mathbb{B}}}
\def\sC{{\mathbb{C}}}
\def\sD{{\mathbb{D}}}
% Don't use a set called E, because this would be the same as our symbol
% for expectation.
\def\sF{{\mathbb{F}}}
\def\sG{{\mathbb{G}}}
\def\sH{{\mathbb{H}}}
\def\sI{{\mathbb{I}}}
\def\sJ{{\mathbb{J}}}
\def\sK{{\mathbb{K}}}
\def\sL{{\mathbb{L}}}
\def\sM{{\mathbb{M}}}
\def\sN{{\mathbb{N}}}
\def\sO{{\mathbb{O}}}
\def\sP{{\mathbb{P}}}
\def\sQ{{\mathbb{Q}}}
\def\sR{{\mathbb{R}}}
\def\sS{{\mathbb{S}}}
\def\sT{{\mathbb{T}}}
\def\sU{{\mathbb{U}}}
\def\sV{{\mathbb{V}}}
\def\sW{{\mathbb{W}}}
\def\sX{{\mathbb{X}}}
\def\sY{{\mathbb{Y}}}
\def\sZ{{\mathbb{Z}}}

% Entries of a matrix
\def\emLambda{{\Lambda}}
\def\emA{{A}}
\def\emB{{B}}
\def\emC{{C}}
\def\emD{{D}}
\def\emE{{E}}
\def\emF{{F}}
\def\emG{{G}}
\def\emH{{H}}
\def\emI{{I}}
\def\emJ{{J}}
\def\emK{{K}}
\def\emL{{L}}
\def\emM{{M}}
\def\emN{{N}}
\def\emO{{O}}
\def\emP{{P}}
\def\emQ{{Q}}
\def\emR{{R}}
\def\emS{{S}}
\def\emT{{T}}
\def\emU{{U}}
\def\emV{{V}}
\def\emW{{W}}
\def\emX{{X}}
\def\emY{{Y}}
\def\emZ{{Z}}
\def\emSigma{{\Sigma}}

% entries of a tensor
% Same font as tensor, without \bm wrapper
\newcommand{\etens}[1]{\mathsfit{#1}}
\def\etLambda{{\etens{\Lambda}}}
\def\etA{{\etens{A}}}
\def\etB{{\etens{B}}}
\def\etC{{\etens{C}}}
\def\etD{{\etens{D}}}
\def\etE{{\etens{E}}}
\def\etF{{\etens{F}}}
\def\etG{{\etens{G}}}
\def\etH{{\etens{H}}}
\def\etI{{\etens{I}}}
\def\etJ{{\etens{J}}}
\def\etK{{\etens{K}}}
\def\etL{{\etens{L}}}
\def\etM{{\etens{M}}}
\def\etN{{\etens{N}}}
\def\etO{{\etens{O}}}
\def\etP{{\etens{P}}}
\def\etQ{{\etens{Q}}}
\def\etR{{\etens{R}}}
\def\etS{{\etens{S}}}
\def\etT{{\etens{T}}}
\def\etU{{\etens{U}}}
\def\etV{{\etens{V}}}
\def\etW{{\etens{W}}}
\def\etX{{\etens{X}}}
\def\etY{{\etens{Y}}}
\def\etZ{{\etens{Z}}}

% The true underlying data generating distribution
\newcommand{\pdata}{p_{\rm{data}}}
% The empirical distribution defined by the training set
\newcommand{\ptrain}{\hat{p}_{\rm{data}}}
\newcommand{\Ptrain}{\hat{P}_{\rm{data}}}
% The model distribution
\newcommand{\pmodel}{p_{\rm{model}}}
\newcommand{\Pmodel}{P_{\rm{model}}}
\newcommand{\ptildemodel}{\tilde{p}_{\rm{model}}}
% Stochastic autoencoder distributions
\newcommand{\pencode}{p_{\rm{encoder}}}
\newcommand{\pdecode}{p_{\rm{decoder}}}
\newcommand{\precons}{p_{\rm{reconstruct}}}

\newcommand{\laplace}{\mathrm{Laplace}} % Laplace distribution

\newcommand{\E}{\mathbb{E}}
\newcommand{\Ls}{\mathcal{L}}
\newcommand{\R}{\mathbb{R}}
\newcommand{\emp}{\tilde{p}}
\newcommand{\lr}{\alpha}
\newcommand{\reg}{\lambda}
\newcommand{\rect}{\mathrm{rectifier}}
\newcommand{\softmax}{\mathrm{softmax}}
\newcommand{\sigmoid}{\sigma}
\newcommand{\softplus}{\zeta}
\newcommand{\KL}{D_{\mathrm{KL}}}
\newcommand{\Var}{\mathrm{Var}}
\newcommand{\standarderror}{\mathrm{SE}}
\newcommand{\Cov}{\mathrm{Cov}}
% Wolfram Mathworld says $L^2$ is for function spaces and $\ell^2$ is for vectors
% But then they seem to use $L^2$ for vectors throughout the site, and so does
% wikipedia.
\newcommand{\normlzero}{L^0}
\newcommand{\normlone}{L^1}
\newcommand{\normltwo}{L^2}
\newcommand{\normlp}{L^p}
\newcommand{\normmax}{L^\infty}

\newcommand{\parents}{Pa} % See usage in notation.tex. Chosen to match Daphne's book.

\DeclareMathOperator*{\argmax}{arg\,max}
\DeclareMathOperator*{\argmin}{arg\,min}

\DeclareMathOperator{\sign}{sign}
\DeclareMathOperator{\Tr}{Tr}
\let\ab\allowbreak

\newcommand{\norm}[2]{\left\| #1 \right\|_{#2}}

\newcommand{\zz}[1]{\textcolor{blue}{ [{\em Zhihui:} #1]}}
\newcommand{\jz}[1]{\textcolor{red}{ [{\em JZ:} #1]}}
% \newcommand{\td}[1]{\textcolor{blue}{ [{\em TD:} #1]}}
\newcommand{\jj}[1]{\textcolor{pink}{ [{\em JJ:} #1]}}


\begin{document}

%%%%%%%%% TITLE - PLEASE UPDATE
% \title{Fundamental Approaches Capability of Open-Vocabulary Object Detection}
% \title{Both Localization and Classification Can Improve Open-Vocabulary Object Detection}
\title{What Makes Good Open-Vocabulary Detector: A Disassembling Perspective}

\author{Jincheng Li}
\authornote{Both authors contributed equally to this research.}
\affiliation{%
  \institution{Qihoo 360 AI Research}
  \city{Beijing}
  \country{China}}
\email{lijincheng@360.cn}

\author{Chunyu Xie}
\authornotemark[1]
\affiliation{%
  \institution{Qihoo 360 AI Research}
  \city{Beijing}
  \country{China}}
\email{xiechunyu@360.cn}


\author{Xiaoyu Wu}
\affiliation{%
  \institution{Qihoo 360 AI Research}
  \city{Beijing}
  \country{China}}
\email{wuxiaoyu1@360.cn}


\author{Bin Wang}
\affiliation{%
  \institution{Qihoo 360 AI Research}
  \city{Beijing}
  \country{China}}
\email{wangbin10@360.cn}

\author{Dawei Leng}
\authornote{Corresponding Author.}
\affiliation{%
  \institution{Qihoo 360 AI Research}
  \city{Beijing}
  \country{China}}
\email{lengdawei@360.cn}


% \author{First Author\\
% Institution1\\
% Institution1 address\\
% {\tt\small firstauthor@i1.org}
% For a paper whose authors are all at the same institution,
% omit the following lines up until the closing ``}''.
% Additional authors and addresses can be added with ``\and'',
% just like the second author.
% To save space, use either the email address or home page, not both
% \and
% Second Author\\
% Institution2\\
% First line of institution2 address\\
% {\tt\small secondauthor@i2.org}
% }


\begin{abstract}
    Open-vocabulary detection (OVD) is a new object detection paradigm, aiming to localize and recognize unseen objects defined by an unbounded vocabulary. This is challenging since traditional detectors can only learn from pre-defined categories and thus fail to detect and localize objects out of pre-defined vocabulary. To handle the challenge, OVD leverages pre-trained cross-modal VLM, such as CLIP, ALIGN, etc. Previous works mainly focus on the open vocabulary classification part, with less attention on the localization part. We argue that for a good OVD detector, both classification and localization should be parallelly studied for the novel object categories. We show in this work that improving localization as well as cross-modal classification complement each other, and compose a good OVD detector jointly. We analyze three families of OVD methods with different design emphases. We first propose a vanilla method, \ie cropping a bounding box obtained by a localizer and resizing it into the CLIP. This vanilla method totally decouples the localization and classification components, making it convenient to improve the OVD performance by applying more advanced object localization models and VLMs. However, resizing cropped regions inevitably causes the deformation of the object and leads slow calculation speed. To address these, we next introduce another approach, which combines a standard two-stage object detector with CLIP. A two-stage object detector includes a visual backbone, a region proposal network (RPN), and a region of interest (RoI) head. We decouple RPN and ROI head (DRR) and use RoIAlign to extract meaningful features. In this case, it avoids resizing objects. To further accelerate the training time and reduce the model parameters, we couple RPN and ROI head (CRR) as the third approach. We conduct extensive experiments on these three types of approaches in different settings. On the OVD-COCO benchmark, DRR obtains the best performance and achieves 35.8 Novel AP$_{50}$, an absolute 2.8 gain over the previous state-of-the-art (SOTA). For OVD-LVIS, DRR surpasses the previous SOTA by 1.9 AP$_{50}$ in rare categories. We also provide an object detection dataset called PID and provide a baseline on PID. 
    % Codes and datasets will be released.
\end{abstract}


\begin{CCSXML}
<ccs2012>
<concept>
<concept_id>10010147.10010178.10010224.10010245.10010250</concept_id>
<concept_desc>Computing methodologies~Object detection</concept_desc>
<concept_significance>500</concept_significance>
</concept>
</ccs2012>
\end{CCSXML}

\ccsdesc[500]{Computing methodologies~Object detection}

% \begin{CCSXML}
% <ccs2012>
%    <concept>
%        <concept_id>10010147.10010178</concept_id>
%        <concept_desc>Computing methodologies~Artificial intelligence</concept_desc>
%        <concept_significance>500</concept_significance>
%        </concept>
%  </ccs2012>
% \end{CCSXML}

% \ccsdesc[500]{Computing methodologies~Artificial intelligence}
%%
%% Keywords. The author(s) should pick words that accurately describe
%% the work being presented. Separate the keywords with commas.
\keywords{vision-language model, open-vocabulary detection, dataset}
%% A "teaser" image appears between the author and affiliation
%% information and the body of the document, and typically spans the
%% page.

% \received{20 February 2007}
% \received[revised]{12 March 2009}
% \received[accepted]{5 June 2009}

\maketitle

% \newpage
\section{Introduction}

\ljc{
Object detection is a prominent vision task, aiming at localizing and recognizing objects in images. This task requires a variety of fine-grained annotations (\eg the bounding boxes and classes) of each object during training, which, however, makes it hard to extend the size of data since manual human annotations are costly and tedious. In this sense, traditional object detectors may fail to precisely detect and localize objects out of pre-defined vocabulary at inference.
}

\ljc{
Open-vocabulary detection (OVD), a task to detect unseen objects defined by an unbounded vocabulary, has attracted much attention in the most recent period. The core challenge of the OVD task is how to localize and classify unseen (novel) categories at the inference stage since they can only learn the knowledge from pre-defined (base) categories during training.
We next analyze the corresponding existing solutions to overcome the above challenge from two perspectives: classification and localization.
% Nevertheless, a region proposal network trained without seeing novel categories during training can generalize to localize novel categories \cite{ViLD}. 
% To further improve the performance of recognizing novel categories, most existing methods \cite{ViLD, regionclip, bangalath2022bridging} leverage the excellent zero-shot / few-shot generalization ability of large-scale vision-language models (VLMs). However, they still suffer from two limitations.
}

\xie{
To classify the novel categories, several works \cite{ViLD, regionclip, bangalath2022bridging} leverage the excellent zero-shot generalization ability of large-scale vision-language models (VLMs) such as CLIP \cite{CLIP}. To this end, they modify a two-stage object detector, including using the visual encoder of CLIP as the backbone of the object detector, replacing the class-specific classification head with a class-agnostic classification head, and so on.
Meanwhile, they re-train or finetune VLMs for open-vocabulary detection since most VLMs are pre-trained on image-text pairs but not detection data. While detection-tailored pre-training is beneficial for the OVD task, some studies like F-VLM \cite{kuo2022f} discard this technique and also achieve significant results.
In this sense, whether to use detection-tailored pre-trained CLIP remains an open question to be discussed.
}

\xie{
On the other hand, to localize the novel categories, most existing methods \cite{regionclip, ViLD, BARON} use a region proposal network (RPN), which is trained on base categories. They demonstrate that the RPN trained without seeing novel categories can generalize to localize novel categories \cite{ViLD}. The bounding boxes obtained by RPN play a critical role in predicting the final boxes. In this way, a better proposal network should further improve the overall performance of the OVD model since the generated bounding boxes will be more accurate.
However, how to effectively improve the detection ability under the settings of the OVD task is still a challenge to be solved.
}






\ljc{
In this paper, we set out to address these issues under the settings of the open-vocabulary detection task. Our goal is to analyze which part of localization and classification can improve the overall performance of the OVD task.
We use three simple but effective approaches, discuss their advantages and disadvantages and design appropriate experiments for them.
}

\ljc{
We first introduce a vanilla method for the OVD task, \ie cropping the bounding box, resizing it into the required input size of CLIP, and then feeding it into the visual encoder of CLIP. The detected bounding box is given by a pre-trained RPN, while the classification score is calculated by the cosine similarity of box embedding and the text embedding, which is extracted by the text encoder of CLIP. This vanilla method completely decouples the detection and classification components, making it easier to \xie{adopt} different models.
% To investigate the effectiveness of detection ability, we use different proposal generation network
In this case, we apply different RPNs to investigate the effectiveness of RPN under the OVD settings, consisting of RPN with extra data or stronger RPN. Our goal in this experiment is to improve the detection ability, such that the overall performance of the OVD task would be further improved.
% Note that we do not apply a RCNN head \cite{fasterrcnn} in this vanilla method. 
In addition, we attempt to use different types of CLIP to enhance the vanilla performance.
Although it is convenient to replace each component, it still suffers from two limitations: 1) it is non-trivial to extract meaningful features when facing tiny bounding boxes, \ie $2 \times 2$ pixel region. 2) it has slow inference speed due to the operations of cropping and resizing.
}


\ljc{
Next, we present another popular approach, which modifies a standard two-stage object detector \cite{fasterrcnn} and combines it with CLIP.  A two-stage object detector consists of a visual backbone, an RPN, and a region of interest (RoI) \xie{head}. Several works \cite{ViLD,kuo2022f,BARON} use a pre-trained RPN to extract proposals, then obtain their features via the visual encoder of CLIP relying on RoIAlign \cite{fasterrcnn}.
% The characteristic of this approach is that it decouples RPN and ROI \xie{head (DRR}) to avoid the fusion of detection features and classification features. 
\xie{The characteristic of this approach is that it decouples RPN and ROI head (DRR) to reduce the fusion of detection features and classification features.}
Compared to the vanilla method, \xie{\xie{DRR} is able to extract significant features for tiny objects and has a faster inference speed.} Unfortunately, it applies two backbones to decouple RPN and \xie{ROI head}, leading to require more computational costs and training time.
}

\ljc{
\xie{Based on DRR, we propose the third approach, which couples RPN and ROI head (CRR). CRR} uses one backbone for localization and classification. In this sense, it accelerates the training time and reduces the parameters of the model. To summarize, our main contributions are as follows:
}



\begin{itemize}[leftmargin=*]
    \item \ljc{
    We investigate and analyze the advantages and disadvantages of three approaches for open-vocabulary detection. We demonstrate that both localization and classification can improve open-vocabulary detectors.
    }
    \item \ljc{We design appropriate experiments for three approaches with the commonly used techniques and achieve state-of-the-art results on \xie{OVD-COCO and OVD-LVIS} benchmarks.}
    \item \ljc{We propose a product dataset (PID) with human annotations for the open-vocabulary detection task and provide a strong baseline on PID.}
\end{itemize}

% \ljc{
% Object detection aims to localize and classify objects from images.
% }



% enhancing the capability of pre-trained models. The two-way distillation consists of
% We also introduce a two-way distillation method, consisting of target-guided distillation and feature-guided distillation. The target-guided distillation increases the robustness when learning from noisy labels, while feature-guided distillation aims to improve the generalization performance of R2D2. To summarize, our main contributions are as follows:
% \begin{itemize}[leftmargin=*]
% 	\item We propose Zero, a Chinese vision-language \databone{} that includes two large-scale pre-training datasets and five downstream train/val/test sets.
% 	We correspondingly build a leaderboard to facilitate the research community making a fair comparison.
% % 	We will build a leaderboard with five test sets to allow researchers to make a fair comparison and promote the development of Chinese VLP.
% % 	vision-language learning.
% 	\item We introduce a novel VLP framework named as \textbf{R2D2} for image-text cross-modal learning.
% % 	between vision and language on large-scale data.
% 	Specifically, we propose a pre-\textbf{R}anking + \textbf{R}anking strategy to learn powerful vision-language representations and a two-way distillation method (\ie target-guided \textbf{D}istillation and feature-guided \textbf{D}istillation) to further enhance the learning capability.
% 	\item Our proposed method achieves the state-of-the-art performance on four public cross-modal datasets and the proposed five downstream datasets, showing the superior ability of our pre-trained model.
% \end{itemize}

\begin{figure*}[t]
    \centering
    \includegraphics[width=0.99\textwidth]{figures/overview8.pdf}
    \caption{\ljc{An overview of three approaches of open-vocabulary detection: a vanilla method, decoupling RPN and ROI head (\xie{DRR}), and coupling RPN and ROI head (\xie{CRR}). Note that Vanilla and \xie{DRR} require two backbones while \xie{CRR} only needs one visual backbone. For simplicity, we omit the logits of regions when training RPN on base categories. For all three fundamental approaches, we get the classification scores via cosine similarity, calculated by the region embeddings and the text embeddings, where the text embeddings are obtained by the text encoder of CLIP.}} 
    \label{fig:overview}
\end{figure*}

% \newpage
\section{Related Work}

\ljc{
\textbf{Vision-Language Pre-training.}
Vision-language pre-training aims to learn the correspondence between vision and natural language. 
It is attractive that pre-trained vision-language models (VLMs) \cite{CLIP, jia2021scaling, yang2022chinese, xie2022zero} trained on large-scale image-text pairs show excellent zero-shot/few-shot migration ability on classification, object detection, and instance segmentation tasks. In particular, F-VLM \cite{kuo2022f} shows that pre-trained VLMs have a strong generalization ability when transferring to the OVD task.
Meanwhile, F-VLM eliminates the need for knowledge distillation or detection-tailored pre-training. This motivates us to investigate how to make better use of VLMs on the open-vocabulary object task. In this paper, we apply different VLMs in the fundamental approaches, including original VLMs, larger VLMs, and detection-tailored pre-training VLMs.
}

% \subsection{Vision-Language Pre-training}
% The appearance of visual transfomers has advanced the development of vision language pre-training(VLP). Based on the existing dataset of image-captions, multimodal transfomers show good understanding of images and texts. The main training method for VLP models is to use self-supervised strategies combined with contrast learning for training, and the self-supervised strategies also show a dependence on the size of the dataset. With the use billion-scale image-captions pairing data, the performance of visual language pre-training is further improved, such as CLIP, R2D2 and ALIGN, etc. It is attractive that some VLP models trained based on large-scale data show excellent Zero-shot migration ability on classification tasks. This also advances multimodal strategies to downstream tasks such as object detection and instance segmentation.

% In this paper, we focus on vision-language pre-training with multi-modal inputs of image and textual sentences. 

% From the perspective of cross-modal learning, 
% \green{The vision-language pre-training architectures can be categorized as: single-stream and dual-stream. 
% Additional mechanisms can also help to enhance the model's capability, \eg knowledge distillation.}
% Additional mechanisms \HM{can also} help to enhance the capability of the model, such as knowledge distillation.




% \subsection{Open-vocabulary detection.}

\ljc{
\textbf{Open-Vocabulary Detection.}
Traditional object detection may fail to localize and recognize unseen objects in an image at inference, \ie zero-shot object detection.
Recently, OVR-RCNN \cite{ovrcnn} proposes the open-vocabulary detection (OVD) benchmark to detect and localize objects for which no bounding box annotation is provided during training.
While OVR-RCNN evaluates the models on tens of categories, ViLD \cite{ViLD} proposes to evaluate on more than 1,000 categories, \ie LVIS \cite{LVIS}.
Following the OVD benchmark, most existing methods are proposed with different forms of weak supervision, such as extra image-caption pairs \cite{ovrcnn, regionclip, gao2022open, wu2023cora}, extra classification datasets \cite{Detic}, and vision-language pre-trained models \cite{ViLD, feng2022promptdet, chen2022open, zhao2022exploiting} like CLIP \cite{CLIP}.
For example, RegionCLIP \cite{regionclip} introduces a region-level pre-training alignment method with extra image-text pairs \eg CC3M \cite{CC3M}, demonstrating its capability on zero-shot and OVD task transfer learning. Detic \cite{Detic} trains the classifiers of a detector on image classification data, \ie ImageNet-21K \cite{imagenet}, yielding excellent detectors even for classes without box annotations. BARON\cite{BARON} proposes to align the embedding of the bag of regions beyond individual regions, relying on the generalization ability of large-scale vision-language pre-trained models. These methods keep achieving better results than previous state-of-the-art methods with various techniques, such as more powerful offline proposal generators \cite{wang2023detecting}, knowledge distillation \cite{ma2022open}, and prompt learning \cite{du2022learning}. When facing a new dataset for new scenarios, it leaves researchers and engineers confused about which approaches or techniques to use. In this paper, we summarize three fundamental approaches for open-vocabulary detection and investigate them with different techniques, showing surprising results of different combinations.
}

% Traditional supervised object detection algorithms designed by the community in the past already have a well performance to identify and localize the specified class of objects contained in the training set. However, it will not work when faced with categories that do not exist in the training set. OVR-CNN\cite{12in1} proposes a new formulation of detection: Open-Vocabulary Detection (OVD), to enhance the detection model's ability to recognize and localize novel categories. Specifically, OVD extends the comprehension capability of the detection model mainly by cross-modal alignment of image region features with the descriptive text of the object to be detected. Many works have emerged to further enhance the detection capability of OVD. Aiming at how to improve the detection model's ability to understand new categories, ViLD improves the model's ability to recognize new categories by using CLIP to distill the visual encoder of the detection model in combination with CLIP's classification capability. RegionCLIP is pre-trained to match regional features with category-based prompts based on the detection model is for mainly classifying regional information. The detection of novel categories is significantly improved by using the image and text matching dataset in the pretraining. Detic increases the detection model's ability to recognize novel categories by additionally using a sizable dataset such as ImageNet for classification regression in the training of the detection model.

% Unlike language model pre-training that has a \databone{}, 
% Chinese vision-language \databone{} requires images and high-quality Chinese texts, which are hard to obtain and still rare for the research community's reach.
% \green{Existing public datasets~\cite{cococaption,VQA2,Flickr} with images and English texts are used to conduct cross-modal modeling tasks. 
% % However, it is difficult and unreliable to translate these English texts into Chinese texts due to the differences in expression.
% However, it is non-trivial to adapt these English texts into other languages, such as Chinese. Some datasets~\cite{flickr30k-cn, coco-cn} have tried to use machine translation to directly adapt these datasets to Chinese. However, machine translation often contains several errors due to differences in expressions in each language.
% To this end, }
% To this end, existing public datasets~\cite{flickr30k-cn, coco-cn} use machine translation to adapt their English versions~\cite{cococaption,Flickr} to Chinese, but the data quality is sacrificed due to machine translation errors.
% Newly reported datasets with Chinese texts~\cite{wenlan2,gu2022wukong,M6} are proposed for Chinese VLP. However, they are either not publicly available or lack sufficient downstream tasks. In this paper, we propose a Chinese vision-language \databone{} that covers two large-scale pre-training datasets and five high-quality downstream datasets. 
% \green{We also build a leaderboard with the five proposed downstream test datasets.}



% \begin{figure}[t]
%     \centering
% 	\includegraphics[width=0.99\textwidth]{figures/visual.pdf}
% % 	\vspace{-10pt}
% 	\caption{Entity-conditioned image visualization.} 
% 	\label{fig:visual}
% \end{figure}


% \textbf{Vision-Language Architecture.}
% % From the perspective of cross-modal learning, the vision-language pre-training architectures can be categorized as: single-stream and dual-stream. 
% The vision-language pre-training architectures can be categorized as: single-stream and dual-stream.
% Most existing single-stream models~\cite{uniter,unicodervl,visualbert,12in1,imagebert} concatenate image and text as a single input to model the interactions between image and text within a transformer model~\cite{Transformer}.
% % For example, VisualBERT~\cite{visualbert} applies masked language modeling (MLM) and image-text matching (ITM) for joint image and text representation and achieves strong performance on various downstream tasks. To learn better context-aware representations for input tokens, Unicoder-VL~\cite{unicodervl} introduces another objective function, Masked Object Classification (MOC). In this way, 
% % \green{They employ masked language modeling, image-text matching, and masked object classification to learn joint image-text representations via pre-training.
% % in a pre-training manner.
% % To further improve the performance of the single-stream model, ImageBERT~\cite{imagebert} not only applies MLM, ITM and MOC as the objective functions, but also proposes a new pre-train objective, Masked Region Feature Regression (MRFR). 
% % In this way, the single-stream model with a single transformer learns coarse-grained representations in a combined semantic space. }
% % \green{Another simple way for image-text representation learning is dual-stream architecture~\cite{vse,scaling,visualitm,tvm}.}
% On the other hand, popular dual-stream models~\cite{vse,scaling,visualitm,ViLBERT,CLIP,ALIGN} aim to align image and text into a unified semantic space via contrastive learning.
% Besides, some works~\cite{BLIP,ALBEF} align the individual features of images and texts in a dual-stream architecture, and then fuse the features in a unified semantic space via a single-stream architecture. 
% % \blue{However, they ignore supervised signals from images and use traditional masked language modeling (MLM) and local contrastive learning to conduct pre-training tasks, leading to potential inferior model performance.}
% However, they ignore supervised signals from images. In addition, they use traditional masked language modeling (MLM) and local contrastive learning to conduct pre-training tasks, leading to potential inferior model performance.
% In this paper, we explore the effective signals via an image-text cross encoder and a text-image cross encoder while also maintaining the bottom dual-stream architecture.
% Moreover, we improve MLM with enhan


% \begin{figure}[t]
%     \centering
%     \includegraphics[width=0.99\textwidth]{figures/overview2.pdf}
%     \caption{\ljc{TODO.}} 
%     \label{fig:overview}
% \end{figure}


% \begin{figure}[t]
%     \centering
%     \includegraphics[width=0.99\textwidth]{figures/overview3.pdf}
%     \caption{\ljc{TODO.}} 
%     \label{fig:overview}
% \end{figure}



% \begin{figure}[t]
%     \centering
%     \includegraphics[width=0.99\textwidth]{figures/overview4.pdf}
%     \caption{\ljc{TODO.}} 
%     \label{fig:overview}
% \end{figure}





\section{Approach}
\label{approach}
% In this section, we introduce our proposed \TODO{R2D2}. Section \ref{sec model} describes the model architecture. Then we introduce the pre-training objectives in Section \ref{sec Objectives} and the two-way distillation in Section \ref{two-way-distillation}.
% As illustrated in Figure \ref{fig:framework}, 
% The model architecture.

\ljc{
In this paper, we introduce three approaches for open-vocabulary detection: a vanilla method, decoupling RPN and ROI head (\xie{DRR}), and coupling RPN and ROI head (\xie{CRR}). The overview of the approaches is illustrated in Figure \ref{fig:overview}.
}

\subsection{A Vanilla Method}


\wxy{To solve the open-vocabulary detection problem, we attempt to isolate the open-vocabulary detection task into two independent sub-tasks, that is, object localization and object classification.} 


\textbf{Object Localization.}
\ljc{To localize objects,} both \ljc{the} class-aware object detector, \ljc{such as YOLO \cite{redmon2016you}, Faster R-CNN }, and the class-agnostic localizer, such as RPN, OLN\cite{OLN} can be used. 
The first challenge for OVD task is to localize novel objects. To solve this problem, \ljc{ we adopt the detection backbone with RPN to localize all objects and classify them as the foreground class. Furthermore, we improve RPN using Faster R-CNN. Here, the multi-classes head within Faster R-CNN is changed to a class-agnostic head so that it can resolve a binary task like RPN. The class-agnostic module within RPN or modified Faster R-CNN can generalize to novel objects \cite{ViLD}. Besides, we can use a classification-free network such as OLN \cite{OLN} for the purpose of localizing objects. }
% \TODO{In this case, we can modify the training loss to estimate the objectness of each region purely by how well the location and shape of a region overlap with any ground-truth objects.}
\xie{In this case, we can modify the training loss to estimate the objectness of each region purely since localization-related metric tends to be robust to novel objects in the open world~\cite{OLN}.}


% \textbf{Ojects Localization}
% For ojects localization stage, both one-stage detector like yolo and two stage-stage detector like faster-rcnn can be used. However, traditional objects detectors are limited to pre-defined objects categories. In order to localize novel objects, we attempt several ways to expand objects categories. One is that using a large-scale detection dataset for pretraining. Detector tends to see as more object categories as possible during pretraining, and then finetuning the detector on base categories. The other way to improve detector ability on unseen categories is using a classification-free objects localization network. We can easily change the multi objects classes to one-class. Like region proposal network in faster-rcnn, all the annotated objects are considered as foreground class. For each region of interest, these modules only predict a single bounding box for all categories, instead of one prediction per category. The class-agnostic modules can generalize to novel objects. Besides, classification-free network can also be used. we can modify the training loss to estimates the objectness of each region purely by how well the location and shape of a region overlap with any ground-truth objects[OLN].


\ljc{
\textbf{Object Classification.} After localizing object candidates, we leverage a pre-trained large-scale vision-language model (VLM) such as CLIP to classify them. Specifically,} We crop and resize the \ljc{object} candidates, and feed them into the visual encoder of VLM to achieve \ljc{the corresponding region embeddings.} In order to provide more context cues, these region embeddings are ensemble from \xie{1× crop (crop the image according to the bounding box) and 1.5× crop (extend the crop size to 1.5 times)}. We feed the category texts with a set of prompt templates and then feed them into the text encoder of VLM to obtain the text embeddings. 
% We use the average text embedding of multiple prompt templates and the ensembled image embedding to calculate cosine similarities. Then, a softmax activation is applied, followed by a per-class NMS to obtain final detections.
Finally, The cosine similarity is calculated by the \xie{averaged text embeddings and region embeddings}, and the per-class NMS is adopted to \xie{obtain the final detection results.}

% \textbf{Ojects classification}
% Once object candidates are localized, we propose to use a pre-trained large-scale vision-Language model(VLM) to classify each region for detection. We crop and resize the localized objects candidates, and feed them into the pretrained image encoder of VLM to compute image embedding.
% We ensemble the image embeddings from 1× region crop and 1.5× region crops as the 1.5× crop provides more context cues. The ensembled image embedding is then renormalized to unit norm. For category texts, we feed them with a set of prompt templates, and then fead into the text encoder of VLM to obtain the text embeddings. We use the average text embedding of multiple prompt templates and the ensembled image embedding to calculate cosine similarities. Then, a softmax activation is applied, followed by a per-class NMS to obtain final detections. 

% As this solution decompose the OVD task into the object localization and classification sub-tasks completely, it can be easily assembled by different models. As there are a great many robust research on these two sub-tasks, it is easily to improve the OVD performance as well. However, as each cropped object region needs to encode by VLM visual encoder, it inferences slowly. Because of the cropping and resizing operations, small objects may become color blocks which is difficult to recognize. And this will lead to bad perform on small objects. 

\xie{As this solution decomposes the OVD task into the object localization and classification sub-tasks completely, it can be easily assembled by different detectors and VLMs, leading to a powerful performance. In the vanilla method, each cropped object region needs to be fixed to the same scale for the VLMs. On the one hand, these operations reduce the efficiency of the model. On the other hand, the deformation of the regions brings more difficulties to object detection, especially for small targets.}

% \vspace{-5pt}
\subsection{Decoupling RPN and ROI head (\xie{DRR})}\label{roialign}

\ljc{
A two-stage object detector consists of a visual encoder backbone, a region proposal network (RPN), and a region of interest (RoI) \xie{head}. Here, the RPN is trained end-to-end to generate high-quality region proposals and then provided for detection in the ROI head.
}
\ljc{
While the traditional two-stage object detector jointly trains RPN and the ROI head, several works \cite{regionclip, wang2023detecting} propose to decouple them to avoid the conflict that the sensitivity of the classification head to novel categories hampers the universality ability of the RPN head. So far, there are no detailed discussions about the effect of whether to decouple RPN and ROI head or not in existing works. In this paper, we try to analyze them and hope to provide a view of the balance of model performance and cost for researchers.
}

\ljc{
As aforementioned, we first introduce a training approach that decouples the RPN and ROI head in a two-stage object detector. Following existing works \cite{regionclip, wang2023detecting, bangalath2022bridging}, we use different backbones for \xie{localization and classification}. That is, one backbone is designed for RPN and the other is for ROI head. Given an image, we first use a pre-trained RPN to obtain a set of regions of interest (\ie proposals). We then use the visual encoder (\eg ResNet-50) of CLIP \cite{CLIP} to encode the whole image. Given proposal boxes of an image, we extract their features along the first 3 blocks based on the whole image encoding and pool them using RoIAlign. The pooled features are then encoded by the last block of the visual encoder. We also replace the class head of Fast R-CNN with the text embeddings encoded by the CLIP text encoder. Considering the setup of open-vocabulary detection \cite{ovrcnn}, we use the base class and hand-crafted prompt as the input of the text encoder of CLIP during training. During inference, we replace the base class with a combination of base and novel classes to generate new text embeddings.
}



\subsection{Coupling RPN and ROI head (\xie{CRR})}
\label{two-way-distillation}
% A two-stage object detector like Faster R-CNN consists of a visual backbone encoder, a RPN and a ROI head module. 
Decoupling proposal generation and ROI head is an efficacious method to keep the universality ability of the proposal generation stage since the proposal generation stage has a class-agnostic classification, which can be easily extended to novel classes. However, this decoupling scheme means that RPN and ROI modules use different backbone encoders, which increases the training cost. In the stage of model deployment and inference, this solution will also bring a lot of extra computing time. Considering the computational efficiency, we practice the scheme of sharing the visual backbone and conduct detailed experiments. These experiments can provide researchers with speed-accuracy trade-offs.

\begin{table*}[t]
\caption{\ljc{Performance on OVD-COCO compared with state-of-the-art methods.}}
      \centering
      \label{sota-coco}
      \resizebox{0.7\textwidth}{!}{
        \begin{tabular}{lcccccc}
        \toprule
        Method & Extra Dataset & Backbone  & Novel AP$_{50}$ & \lightgray{Base AP$_{50}$} & \lightgray{Overall AP$_{50}$}  \\
        \midrule
        OVR-CNN \cite{ovrcnn} & COCO Captions & ResNet50  & 22.8 & \lightgray{46.0} & \lightgray{39.9} \\
        ViLD \cite{ViLD} & -  & ResNet50 & 27.6 & \lightgray{59.5} & \lightgray{51.2} \\
        Detic \cite{Detic} & COCO Captions & ResNet50 & 27.8 & \lightgray{47.1} & \lightgray{45.0} \\
        RegionCLIP \cite{regionclip} & CC3M & ResNet50 & 31.4 & \lightgray{57.1} & \lightgray{50.4} \\
        BARON \cite{BARON} & COCO Captions & ResNet50 & 33.1 & \lightgray{54.8} & \lightgray{49.1} \\
        % CORA & - & ResNet-50 & 35.1 & 35.5 & 35.4 \\
        \midrule
        Vanilla (Ours) & - & ResNet50 &   31.8 & \lightgray{37.2} & \lightgray{35.5} \\
        \xie{CRR} (Ours) & CC3M & ResNet50 & 32.0 & \lightgray{52.5} & \lightgray{47.1} \\
        \xie{DRR} (Ours) & CC3M & ResNet50 & \textbf{35.8}  & \lightgray{54.6} & \lightgray{49.6} \\
        % \xie{CRR} (Ours) &CC3M & ResNet-50 & 32.03 & 52.47 & 47.12 \\
        % \xie{DRR} (Ours) & CC3M & ResNet-50 & 35.77  & 54.55 & 49.64 \\
        \bottomrule
        \end{tabular}
        }
\end{table*}

\begin{table*}[t]
\caption{\ljc{Performance on OVD-LVIS compared with  state-of-the-art methods.}}
      \centering
      \label{sota-lvis}
      \resizebox{0.7\textwidth}{!}{
        \begin{tabular}{lcccccc}
        \toprule
        Method & Backbone & Require Novel Class \cite{BARON} & AP$_r$ & \lightgray{AP$_c$} & \lightgray{AP$_f$} & \lightgray{mAP}  \\
        \midrule
        RegionCLIP \cite{regionclip} & ResNet50 & $\no$ & 17.1 & \lightgray{27.4} & \lightgray{34.0} & \lightgray{28.2} \\
        ViLD \cite{ViLD} & ResNet50 & $\yes$ & 16.7 & \lightgray{26.5} & \lightgray{34.2} & \lightgray{27.8} \\
        % BARON & ResNet50 & 17.3 & 25.6 & 31.0 & 26.3 \\
        BARON \cite{BARON} & ResNet50 & $\yes$ & 20.1 & \lightgray{28.4} & \lightgray{32.2} & \lightgray{28.4} \\
        \midrule
        Vanilla (Ours) & ResNet50 & $\no$ & 17.2 & \lightgray{14.8} & \lightgray{11.5} & \lightgray{13.9}\\
        \xie{CRR} (Ours) & ResNet50 & $\no$ & 14.0 & \lightgray{23.7} & \lightgray{28.5} & \lightgray{21.9} \\
        \xie{DRR} (Ours) & ResNet50 & $\no$ & 20.1 & \lightgray{29.9} & \lightgray{35.7} & \lightgray{30.5}  \\
        \xie{DRR} (Ours) & ResNet50 & $\yes$ & \textbf{22.0} & \lightgray{25.4} & \lightgray{33.7} & \lightgray{28.1}  \\
        % end-to-end & TODO \\
        \bottomrule
        \end{tabular}
        }
\end{table*}

\section{Experiments}

\textbf{Dataset and Metrics.} 
\ljc{We comprehensively evaluate three fundamental approaches of OVD task on COCO \cite{COCO}, LVIS \cite{LVIS}, and our proposed Product Image Dataset (PID) benchmarks.}
% We conduct our main experiments on OVD-COCO, OVD-LVIS, and our proposed Product Image Dataset (PID). 
COCO is a standard dataset comprising 80 categories of common objects in a natural context. It contains 118k images with bounding boxes and instance segmentation annotations. We follow \ljc{OVR-CNN \cite{ovrcnn}} to split the object categories into 48 base categories and 17 novel categories. We also follow \ljc{ViLD \cite{ViLD}} for the LVIS dataset to split the 337 rare categories into novel categories and the rest common and frequent categories into base categories. For simplicity, we denote the open-vocabulary benchmarks based on COCO and LVIS as OVD-COCO and OVD-LVIS\ljc{, respectively}. \ljc{Following ViLD \cite{ViLD}, the Novel AP$_{50}$ and AP$_r$ are the main metric on OVD-COCO and OVD-LVIS, respectively.}
Our open-vocabulary object detectors are trained on base classes. \ljc{Besides}, we split PID into the base and novel categories. The detailed introductions can refer to Sec~\ref{PID}.

The dataset used for detector pre-training is established from the BigDetection dataset~\cite{bigdetection}. As the BigDetection dataset has nearly covered all categories in COCO, we remove the COCO images and delete both novel and base categories of the COCO dataset from the annotations. Some categories with less than 100 train samples are removed as well. We also exclude LVIS data from the BigDetection dataset. Finally, we establish this new BigDetection dataset named BigDetection* (BD*) which has 489 categories for detector pre-training.

\textbf{Implementation Details of the vanilla method.}
% The dataset used for detector pretraining is established from bigdetection. COCO images are removed and both novel and base categories are delete from annotations. Besides, some categories with less than 100 train samples are removed as well. Finally, we establish this new bigdetection dataset without coco data(\xie{BigDetection*}) which has 489 categories for detector pretrianing. 
% \xie{We adopt different models as the object localization and object classification modules.}
\xie{In the OVD-COCO and OVD-LVIS experiments, We use Faster R-CNN as the object detector and frozen CLIP as the object classifier. SGD optimizer is adopted for the 8× training schedule. We train the detector for 720k iterations and divide the learning rate by 10 at 660k and 700k iterations. We also adopt linear warmup for the first 1,000 iterations starting from a learning rate of 0 to 0.02.
Then, we fine-tune the model on base categories of OVD-COCO and OVD-LVIS separately. For OVD-COCO, we train 90k iterations and scale down the learning rate at 60k and 80k. For OVD-LVIS, we train 180k iterations and scale down at 120k and 160k. 
For these two datasets, the learning rate increases to 0.0002 for the first 5k iterations for the warmup.} 

\xie{In the PID experiments, we use Faster R-CNN pre-trained on the BigDetection dataset \cite{bigdetection} as the object detector. ProductCLIP with prompt tuning, which is introduced in Sec \ref{PID_Ablation}, is applied as the object classifier. The detectors are fine-tuned with the 1x training schedule. A warm-up step with a learning rate of 0.001 is performed for the first 400 iterations. 
On both the pre-training and fine-tuning stages, we select 16 samples per GPU with the class-aware sampler. Multi-scale training is adopted with the short edge in the range [640, 800] and the long edge up to 1333. We use 8 A100 GPUS to perform the experiments. }
% In the OVD-COCO and OVD-LVIS experiments, We use Faster R-CNN as the detector. It is pre-trained on BigDetection*. SGD optimizer is adopted for 8× training schedule. That is, the model is trained 720k iterations and scaled down learning rate by a factor of 10 at 660k and 700k. For the warmup, we increase the learning rate from 0 to 0.02 for the first 1k iterations. Then, we finetune the model on base categories of OVD-COCO and OVD-LVIS separately. For OVD-COCO, we train 90k iterations and scale down LR at 60k and 80k. And for OVD-LVIS, we train 180k iterations and scale down at 120k and 160k. For these two datasets, the learning rate increases to 0.0002 for the first 5k iterations for the warmup. On both pre-training and finetuning stages, we select 16 samples per GPU with the class-aware sampler. Multi-scale training is adopted with the short edge in range [640, 800] and the long edge up to 1333. We use pre-trained CLIP directly as the classifier on both OVD-COCO and OVD-LVIS.  

% In the PID experiments, we pre-train the detector Faster R-CNN on full BigDetection datasets followed \cite{bigdetection}. Then, finetune the model on PID base categories with 1x training schedule. For the warmup, the learning rate is increased to 0.001 for the first 400 iterations. Multi-scale training similar to OVD-COCO is also used. We use 8 A100 GPUS and 40 samples per GPU. ProductCLIP with prompt tuning, which is introduced in \ref{PID_Ablation}, is used as the classifier.





\textbf{\xie{Implementation Details of \xie{DRR} and \xie{CRR}.}}
\ljc{In Figure \ref{fig:overview}, the detection backbone and the CLIP visual backbone are both ResNet50. We adopt ImageNet \cite{imagenet} and RegionCLIP \cite{regionclip} pre-trained parameters for the detection backbone and the CLIP visual backbone, respectively. During training, we first train the RPN to obtain the proposal bounding boxes on the base categories. We then use a Faster R-CNN  with ResNet50-C4 architecture as the detector and 1x training schedule (90k iterations). We train the detector using the offline pre-trained RPN. That is, we decouple the RPN and the ROI head. We use an SGD optimizer with a learning rate of 0.002 and a min-batch of 16.
We set the weight of the background category to 0.2 and 0.8 for OVD-COCO and OVD-LVIS, respectively.
We use the same learning rate and input-scale strategy as the vanilla method.
We replace the base categories in the classification head with base + novel categories during inference. We use the top-ranked 100 proposals at test time for all detectors.}
\ljc{For \xie{CRR}, we keep the same settings as \xie{DRR}, except that \xie{CRR} removes the detection backbone, as shown in Figure \ref{fig:overview}.}
% \xie{For \xie{CRR}, we use the same configuration as \xie{DRR}. The main difference is that we use only one visual backbone.}

\ljc{
We use the same settings as OVD-COCO when conducting experiments on PID except that the pre-trained weight of CLIP visual backbone. As aforementioned in the vanilla method, we use ProductCLIP as the pre-trained weight on PID.
}


\subsection{Comparison with State-of-the-arts.}
\subsubsection{Experiment on OVD-COCO}

\ljc{
We compare all three kinds of basic methods with most existing state-of-the-art methods on OVD-COCO. Table \ref{sota-coco} summarizes the results.} 
\ljc{Although the vanilla method is flexible to replace the detector and classification model, it achieves comparable results on novel categories but obtains bad results on base categories. One possible reason is that CLIP has gaps between images and resized images.}
\xie{Moreover, \xie{CRR} obtains a higher Novel AP$_{50}$ than RegionCLIP, but lower than BARON. However, both RegionCLIP and BARON are with extra backbone, which brings more computational complexity. \xie{DRR} achieves the best results and outperforms BARON by 2.7 Novel AP$_{50}$.}
\ljc{These results demonstrate that \xie{DRR} has the potential to be the best baseline when selecting approaches for open-vocabulary detection. }



\begin{table}[htb]
\caption{\xie{Influence of object localization on OVD-COCO. Faster R-CNN* denotes that it is pre-trained on BigDetection*.}}
      \centering
      \label{det_influence}
      % \resizebox{0.5\textwidth}{!}
      {
        \begin{tabular}{lcccc}
        \toprule
        Method    &  Novel AP$_{50}$ & \lightgray{Base AP$_{50}$} & \lightgray{Overall AP$_{50}$}  \\
        \midrule
        RPN    & 16.6 & \lightgray{20.3} & \lightgray{19.2} \\
        Faster R-CNN    &  19.5 & \lightgray{35.7} & \lightgray{30.9} \\
        OLN    &  26.2 & \lightgray{27.0} & \lightgray{26.6} \\
        % Faster R-CNN  & \xie{BigDetection*}  &  26.8 & 18.0 & 19.8 \\
        Faster R-CNN*    & 29.6 & \lightgray{32.1} & \lightgray{31.1}\\

        \bottomrule
        \end{tabular}
        }
\end{table}


\subsubsection{Experiment on OVD-LVIS}
% \subsubsection{Analysis on three kinds of method.}
% ... advantage and disadvantage...(params, FPS)

\ljc{
We further conduct experiments and compare them with state-of-the-art methods on a larger open-vocabulary dataset, \ie OVD-LVIS. We follow the setup of RegionCLIP \cite{regionclip} and provide the results in Table \ref{sota-lvis}. For a fair comparison with the previous SOTA (\ie BARON \cite{BARON}), \xie{we also report the ensemble results which require novel class following BARON. }
% \xie{DRR} achieves 20.1 AP$_r$, which is significantly better than 3.0 AP$_r$ compared to RegionCLIP. 
\xie{\xie{DRR} achieves 20.1 AP$_r$, which is significantly better than RegionCLIP by 3.0 AP$_r$.}
Meanwhile, \xie{DRR} becomes the new SOTA on OVD-LVIS with the setting of requiring novel class during inference. Similar to OVD-COCO, \xie{CRR} still leads a competitive result.
Instead, the vanilla method obtains bad results compared to other methods, indicating that the operation of crop and resize is non-trivial to recognizing small objects in OVD-LVIS.
}


% \begin{table}[htb]
% \caption{{Effect of image embedding ensemble.}}
%       \centering
%       \label{image_ensemble}
%         \resizebox{0.48\textwidth}{!}{
%         \begin{tabular}{lccccc}
%         \toprule
%         Detector & Classifier & Ensemble & Novel AP$_{50}$ & Base AP$_{50}$ & Overall AP$_{50}$  \\
%         \midrule
%         Faster R-CNN & CLIP & $\no$ & 27.3 & 28.9 & 28.5 \\
%         Faster R-CNN & CLIP & $\yes$ & 29.6 & 32.1 & 31.1 \\
%         \bottomrule
%         \end{tabular}
%         }
% \end{table}

% \iffalse
% \begin{table}[t]
% \caption{{Influence of different VLM on PID.}}
%       \centering
%       \label{pid_result}
%       {
%         \begin{tabular}{lcccc}
%         \toprule
%         Detector & Classifier  & Novel AP$_{50}$ & {Base AP$_{50}$}  \\
%         \midrule
%         Faster R-CNN & PRD2 & & \\
%         Faster R-CNN & ProductCLIP & 37.0 & 42.831 \\
%         Faster R-CNN & ProductCLIP$_{coop}$ &41.968 &52.829  \\
%         \bottomrule
%         \end{tabular}
%         }
% \end{table}
% \fi



\subsection{Analysis of Three Fundamental Approaches.}
\ljc{To investigate the behavior of different fundamental approaches of OVD task, we conduct several ablation studies. We design different experiments according to their intrinsic characteristics.}
% We finally give a discussion of three approaches based on the experimental results in Section \ref{discussion}.}

\begin{table}[htb]
\caption{\xie{Influence of object classification on OVD-COCO.}}
      \centering
      \label{cls_influence}
      {
        \begin{tabular}{lccc}
        \toprule
        Classifier   & Novel AP$_{50}$ & {Base AP$_{50}$} & {Overall AP$_{50}$}  \\
        \midrule
        CLIP-RN50   & 26.6 &\lightgray{27.4} & \lightgray{26.8} \\
        CLIP-ViT-B  & 29.6 & \lightgray{32.1} & \lightgray{31.1} \\
        CLIP-ViT-L  &  31.8 & \lightgray{37.2} & \lightgray{35.6} \\
        \bottomrule
        \end{tabular}
    }
\end{table}

\begin{table}[htb]
\caption{{Effect of image embedding ensemble on OVD-COCO.}}
      \centering
      \label{image_ensemble}
        \resizebox{0.48\textwidth}{!}{
        \begin{tabular}{lccccc}
        \toprule
        Method &  Ensemble & Novel AP$_{50}$ & Base AP$_{50}$ & Overall AP$_{50}$  \\
        \midrule
        Vanilla & $\no$ & 27.3 & \lightgray{28.9} & \lightgray{28.5} \\
        Vanilla & $\yes$ & 29.6 & \lightgray{32.1} & \lightgray{31.1} \\
        \bottomrule
        \end{tabular}
        }
\end{table}


\subsubsection{Vanilla}
% ... advantage and disadvantage ...
% ... experiments on OVD-COCO ...
\xie{As} this two-stage framework isolates the object localization and classification completely, it can be easily assembled by different models. 
\xie{With the help of advanced object detection models and vision-language models, OVD can be improved flexibly by applying different object localization and object classification modules.}
% Because this two-stage framework isolates the object localization and classification completely, it can be easily assembled by different models. There is a general way to get better OVD performance under this framework, that is, to improve the performance of object localizer or classifier respectively. As there are a great many robust research on object localization and VLM, a better OVD performance is easily achieved. Besides, it is easier to applying to different application scenarios with good performance.
% Table \ref{det_influence} and Table \ref{cls_influence} discuss the influence of different object localizer and classifier on OVD task.

% However, as each cropped object region needs to encode by VLM visual encoder, it inferences slowly. And this framework doesn't perform well when dealing with small objects. Because of the cropping and resizing operations, small objects may become color blocks which is difficult to recognize.
%Besides, applying to specific application scenarios, the optimizing method is also effective. 

\begin{table*}[t]
\caption{\ljc{Comparisons of different RPNs over ResNet50 on OVD-COCO. Here, COCO-48-RPN, BD*-489-RPN, and BD*-489-COCO-48-RPN denote that the RPNs are trained on base categories of OVD-COCO, BigDetection, and the combination of base categories of OVD-COCO and BigDetection, respectively.}}
  \label{detector}
  \centering
  \resizebox{1\textwidth}{!}{
    \begin{tabular}{lcccccc}
    \toprule
    Method & Proposal Generation Type & Pre-trained RPN & Multiplying RPN Score & Novel AP$_{50}$ & \lightgray{Base AP$_{50}$} & \lightgray{Overall AP$_{50}$} \\
    % \multirow{2.5}{*}{Method} &
    % \multirow{2.5}{*}{Proposal Generation Type} &
    % \multirow{2.5}{*}{Pre-trained RPN} &
    % \multirow{2.5}{*}{Multiplying RPN Score} &
    % \multicolumn{3}{c}{AP$_{50}$}
    % \\ 
    % \cmidrule(r){5-7}
    % & & & &  Novel & \lightgray{Base} & \lightgray{Overall}
    % \\
    \midrule
    % RegionCLIP \cite{regionclip} & RPN & COCO-48-RPN  & 31.4 & \lightgray{57.1} & \lightgray{50.4} & - & - & - \\
    \xie{DRR} & RPN & COCO-48-RPN & $\no$ & 31.0 & \lightgray{56.7} & \lightgray{50.0}  \\
    \xie{DRR} & RPN & COCO-48-RPN & $\yes$ & 31.1 & \lightgray{53.9} & \lightgray{48.0}  \\
    \xie{DRR} & Faster R-CNN \small{(class-agnostic)}  & BD*-489-RPN & $\no$ & 30.5 & \lightgray{52.1} & \lightgray{46.5}  \\
    \xie{DRR} & Faster R-CNN \small{(class-agnostic)}  & BD*-489-RPN & $\yes$ & 32.3 & \lightgray{40.4} & \lightgray{38.3}  \\
    \xie{DRR} & Faster R-CNN \small{(class-agnostic)}  & BD*-489-COCO-48-RPN & $\no$  & 30.5  & \lightgray{55.6}  & \lightgray{49.0}  \\
    \xie{DRR} & Faster R-CNN \small{(class-agnostic)}  & BD*-489-COCO-48-RPN & $\yes$  & \textbf{35.8}  & \lightgray{54.6}  & \lightgray{49.4}  \\
    \bottomrule
    \end{tabular}
    }
\end{table*}


\begin{table*}[t]
\caption{\ljc{Effect of CLIP visual backbone on OVD-COCO compared with state-of-the-art methods.}}
  \label{visual_encoder}
  \centering
  \resizebox{0.85\textwidth}{!}{
    \begin{tabular}{ccccccccc}
    \toprule
    Method & Visual Backbone & Detection-tailored Pre-training &
    Novel AP$_{50}$ & \lightgray{Base AP$_{50}$} & \lightgray{Overall AP$_{50}$} \\
    \midrule
    RegionCLIP \cite{regionclip} & ResNet50 & $\no$& 14.2 & \lightgray{52.8} & \lightgray{42.7}  \\
    RegionCLIP \cite{regionclip} & ResNet50 & $\yes$  & 31.4 & \lightgray{57.1} & \lightgray{50.4}  \\
    \xie{DRR} (Ours) & ResNet50 & $\yes$  & \textbf{35.8}  & \lightgray{54.6}  & \lightgray{49.4}  \\
    \midrule
    RegionCLIP \cite{regionclip} & ResNet50x4 & $\yes$  &  39.3 & \lightgray{61.6} & \lightgray{55.7}  \\
    \xie{DRR} (Ours) & ResNet50x4 & $\yes$  & \textbf{41.9} & \lightgray{57.8} & \lightgray{53.7}  \\
    \bottomrule
    \end{tabular}
    }
\end{table*}


\textbf{The influence of object localization.} Table \ref{det_influence} shows the influence of different object localizers. 
\xie{All the experiments use CLIP-ViT-B/32 \cite{CLIP} as the classifier. We attempt several different object localization networks including RPN, Faster R-CNN, and OLN \cite{OLN} with different training schedules. We evaluate the models on the OVD-COCO dataset. As Faster R-CNN can achieve more accurate bounding boxes than RPN, it leads to a higher AP50 on both novel and base categories. Besides, in order to improve the performance for the novel class, we also evaluate the class-agnostic object localization network like OLN. }
OLN learns generalizable objectness and tends to propose any objects in the image.
\xie{Compared with the Faster R-CNN, it outperforms 6.7\% AP50 on novel categories. To obtain better generalization ability, we pre-trained the Faster R-CNN on the BigDetection* dataset. From the fourth row of Table \ref{det_influence}, we observe that the additional pre-training brings a 3.4\% AP50 improvement.}
% All the experiments in the table use CLIP-ViT-B as the classifier. %Improving the detector performace can leads to a better result on ovd task under this framework. 
% We attempt several different object localization networks including RPN, Faster R-CNN and OLN with different training schedule. It proves that the better localizer performace can leads to the better OVD result. As Faster R-CNN can achieve more accurate bounding boxes than RPN, it leads to a higher AP50 on both novel and base categories, which are trained on coco base data directly. Besides, in order to improve the performance on novel class, class-agnostic object localization network like OLN can be applied as well. OLN learns generalizable objectness and tends to propose any objects in the image. Compared with Faster R-CNN, it outperforms 6.8 on novel categories. 
% Meanwhile, we attempt to pretrain the Faster R-CNN on  \xie{BigDetection*} with the purpose of seeing more objects categories. It leads to 7.3 higher on AP50 than it trained on coco-base only. However, it achieves worse performance on base class with this training strategy. To alter this situation, we finetune it on coco-base data in class-agnostic way, which resulting in well performance both on novel and base categories.
%Another way to improve the performance on novel class is pretraining the detector on a bigger dataset. 
% we atempts to finetune the classifier on base data to enhance the performance on base classes. 

\xie{
\textbf{The influence of object classification.} Table \ref{cls_influence} shows the influence of different classifiers. According to the above experiments, we use the Faster R-CNN pre-trained on the BigDetection* dataset for object localization. CLIP \cite{CLIP} models are used to classify cropped region proposals. Compared with CLIP-ResNet50 and CLIP-ViT-B/32, CLIP-ViT-L/14 shows better performance. This experiment proves that a strong open-vocabulary object classification model can provide the powerful capability for detecting novel objects.
}

% \textbf{The influence of classifier.} Table \ref{cls_influence} shows the influence of different classifier. We evaluate them on OVD-COCO. As the classification ability of OVD comes from the VLM completely, the more efficient the VLM is, the better performance OVD owns. In this ablation study, we use Faster R-CNN for object localization which is pretrained with \xie{BigDetection*} and finetuned it in the class-agnostic way like rpn with coco base data. CLIP models are used for classification. Compared with CLIP-RN50, CLIP-ViT/B model can lead to better performance on OVD task. As CLIP-ViT/L has stronger capability, better performance is acquired when it is used to be the classifier. 
% Table \ref{pid_result} gives the results on product images(PID).



\textbf{The influence of image embedding ensemble.} After localizing objects, we crop and resize the object regions for the \xie{object} classifier to compute image embedding. 
\xie{There are limited contextual cues if cropping directly according to the bounding box localized by the object detector. }
% There are limited context cues if cropping directly according to the bounding box localized by the detector. 
We attempt to expand the bounding boxes by 1.5 times and ensemble the image embeddings from 1× crop and 1.5× crop. 
\xie{We use the Faster R-CNN pre-trained on the BigDetection* dataset for object localization and CLIP-ViT-B/32 for object classification.} 
Table ~\ref{image_ensemble} shows that the image embedding ensemble can improve the OVD performance efficiently.






\subsubsection{\xie{DRR}}
% ... advantage and disadvantage...
% \textbf{Advantage and Disadvantage.}

% ... experiments on OVD-COCO...

% \subsection{More analysis over ROIAlign on COCO?}
% The effect of detector.

% The effect of visual encoder.

\ljc{
\textbf{The effect of RPN.} As illustrated in Figure \ref{fig:overview}, we first use an offline RPN to obtain proposal bounding boxes and then extract the corresponding features with the help of the visual encoder of CLIP.
In this section, we investigate the effect of different RPN under the settings of \xie{DRR}.
We train several RPNs on different datasets to determine whether a proposal is a foreground. We also replace RPN with Faster R-CNN with a class-agnostic head to improve the detection performance. Compared to RPN, Faster R-CNN prefer to output more accurate bounding boxes with more high-quality objectness logits, thus we attempt to multiply the logits with the final CLIP scores, called ``Multiplying RPN score''. That is, it can be formulated as $\sqrt{s_{1} \cdot s_{2}}$, where $s_{1}$ and $s_{2}$ denote the objectness logits and the CLIP score, respectively.
% \TODO{Multiply RPN score ???}
Table \ref{detector} summarizes the results.
}

% \ljc{
% \TODO{Difference with BARON Ensemble Score?}
% }

\ljc{
From Table \ref{detector}, we get the following observations. 1) Replacing RPN with Faster R-CNN cannot achieve the expected results. Generally, we can improve the overall model performance by improving the offline RPN. However, COCO-48-RPN and BD*-489-COCO-48-RPN obtain similar Novel AP$_{50}$ when without multiplying RPN score (31.0 vs. 30.5), indicating a better offline RPN does not work as well. This is because the offline RPN only provides bounding boxes for the visual backbone of CLIP and does not directly participate in the loss of classification.
2) The significant objectness logits within a better offline RPN are indeed important for model performance. We find that \xie{DRR} with ``multiplying RPN score'' leads to a Novel AP$_{50}$ of 35.8 compared to the one without ``multiplying RPN score'', which becomes a new SOTA over ResNet50 on OVD-COCO.
}





\ljc{
\textbf{The effect of CLIP visual backbones.}
From Table \ref{visual_encoder}, we argue that the detection-tailored pre-training like RegionCLIP \cite{regionclip} is required when needing the best performance.
% We then improve the performance of a model by increasing its parameters. That is, 
We then use a larger visual encoder to extract the features of regions of interest. We use ResNet50x4 as the backbone and conduct experiments on OVD-COCO. Relying on the observation from Table \ref{detector}, we use BD*-489-COCO-48-RPN as the RPN and multiply the CLIP scores with RPN scores during inference for \xie{DRR}.
In Table \ref{visual_encoder}, \xie{DRR} surpasses the previous state-of-the-art (\ie RegionCLIP \cite{regionclip}) by 2.6 AP$_{50}$ in novel categories, which benefits from more accurate bounding boxes and more significant objectness logits.
Meanwhile, ResNet50x4 shows better results than ResNet50  from Tables \ref{visual_encoder} \& \ref{sota-coco}. This implies that a model with a larger capacity is able to obtain better representations and improve overall performance. 
}


\begin{figure*}[t]
    \centering
    \includegraphics[width=0.88\textwidth]{figures/examples_PID_2.pdf} 
    \caption{Examples (with annotations) of PID. The first and second rows are from the base and novel categories, respectively.
    }
    \label{fig:example_PID}
\end{figure*}



\subsubsection{\xie{CRR}}

% ... advantage and disadvantage...(params, FPS)
\textbf{Analysis of Computational efficiency.} Table \ref{fps_cost} shows the performance and computational efficiency of the three frameworks. For a fair comparison, we obtain the results of different models under the same environment. We use the Detectron2 
\footnote{https://github.com/facebookresearch/detectron2}
toolbox based on PyTorch
\footnote{https://pytorch.org}
and \ljc{an} A100 \ljc{GPU} to calculate the FPS. The batch size (denoted as BS in Table) is set to 1. From Table \ref{fps_cost}, we can see that the scheme of sharing the visual backbone has the highest computational efficiency compared to the other two models, and it is far more accurate than the scheme of CLIP on cropped regions. Sharing the visual backbone is indeed more effective in specific real-world scenarios.


\begin{table}[t]
\caption{Comparisons of the computational efficiency over ResNet50 on OVD-COCO.}
      \centering
      \label{fps_cost}
      \resizebox{0.38\textwidth}{!}{
        \begin{tabular}{lcccccc}
        \toprule
        Method &  Params $\downarrow$ & FPS (BS1 A100) $\uparrow$ \\
        \midrule
        Vanilla & 136.9 M  & 2 &  \\
        \xie{DRR} &  143.4 M   &  12 \\
        \xie{CRR} & \textbf{111.6 M} & \textbf{13}\\
        \bottomrule 
        \end{tabular}
        }
\end{table}





\section{Product Image Dataset--A New Object Detection Dataset}
% ... Intro of PID
\subsection{Introduction of PID.}\label{PID}
\ljc{
Product Image Dataset (PID) is an object detection dataset consisting of a series of product images along with the corresponding human-annotated bounding boxes and classes.
Note that we label all bounding boxes in Chinese.
PID contains 37,540 images and 52,796 annotations.
The image size of PID is 800 $\times$ 800. 
To adapt to the setting of open vocabulary object detection like OVR-RCNN \cite{ovrcnn}, we split PID into a training dataset and a test dataset, and put more classes in the test dataset. Here, the training dataset consists of 233 classes, 14,802 images, and 20,690 annotations while the test dataset includes 466 classes, 22,738 images, and 32,106 annotations. That is, we use 233 classes as base categories and 233 classes as novel categories. We provide several examples with annotations in Figure \ref{fig:example_PID}. The product image dataset will be released.
}

\begin{table}[t]
\caption{Comparisons of different fundamental approaches over ResNet50 on PID. *More analysis can be found in Section \ref{ft_res_PID}.}
  \label{ft_results_pid}
  \centering
  \resizebox{0.38\textwidth}{!}{
    \begin{tabular}{lcccccccc}
    \toprule
    \multirow{2.5}{*}{Method} &
    \multirow{2.5}{*}{Visual Backbone} &
    \multicolumn{3}{c}{Generalized (233+233)}
    \\ 
    \cmidrule(r){3-5}
    &  & Novel & \lightgray{Base} & \lightgray{Overall} \\
    \midrule
    Vanilla* & ResNet50 & \textbf{42.0} & \lightgray{52.8} & \lightgray{47.4} \\
    \xie{DRR} & ResNet50 & 30.7 & \lightgray{35.6} & \lightgray{33.2}  \\
    \xie{CRR} & ResNet50 & 27.6 & \lightgray{34.3} & \lightgray{31.0} \\
    \bottomrule
    \end{tabular}
    }
\end{table}





\subsection{Finetuned results on PID.}
\label{ft_res_PID}
\ljc{
We train three approaches on the base categories of PID and then evaluate them on the base and novel categories. We adopt ResNet50 as the visual backbone of CLIP.
% We also pre-train this visual backbone using image-text product pairs. 
The results are shown in Table \ref{ft_results_pid}.
Somewhat surprisingly, the vanilla method achieves the best performance compared to \xie{DRR} and \xie{CRR}, which is inconsistent with the results on OVD-COCO and OVD-LVIS. One possible reason is that the objects in PID are large such that cropping and resizing the proposal is an effective way to improve the overall performance.
}

\begin{table*}[t]
\caption{\ljc{Effect of RPNs and the ensemble score on PID.}}
  \label{multiply_rpn_score_PID}
  \centering
  \resizebox{0.7\textwidth}{!}{
    \begin{tabular}{ccccccccc}
    \toprule
    \multirow{2.5}{*}{Visual Encoder Pre-training} &
    \multirow{2.5}{*}{Pre-trained RPN} &
    \multirow{2.5}{*}{Multiplying RPN Score} &
    \multicolumn{3}{c}{Generalized (233+233)}
    \\ 
    \cmidrule(r){4-6}
    &  & & Novel & \lightgray{Base} & \lightgray{Overall}   \\
    \midrule
    ProductCLIP & PID-233-RPN & $\no$ & 22.0 & \lightgray{34.1} & \lightgray{28.1} \\
    ProductCLIP & PID-233-RPN & $\yes$ & 26.6 & \lightgray{36.2} & \lightgray{31.4} \\
    ProductCLIP & BD*-489-PID-233-RPN & $\yes$ & 30.7 & \lightgray{35.6} & \lightgray{33.2} \\
    \bottomrule
    \end{tabular}
    }
\end{table*}


\begin{table*}[t]
\caption{\ljc{Effect of knowledge distillation on PID. We use L1 distillation by default.}}
  \label{KD_PID}
  \centering
  \resizebox{0.7\textwidth}{!}{
    \begin{tabular}{ccccccccc}
    \toprule
    \multirow{2.5}{*}{Visual Encoder Pre-training} &
    \multirow{2.5}{*}{Pre-trained RPN} &
    \multirow{2.5}{*}{Temperature} &
    \multicolumn{3}{c}{Generalized (233+233)}
    \\ 
    \cmidrule(r){4-6}
    &  & & Novel & \lightgray{Base} & \lightgray{Overall}  \\
    \midrule
    ProductCLIP & PID-233-RPN & - & 22.0 & \lightgray{34.1} & \lightgray{28.1} \\
    ProductCLIP & PID-233-RPN & 1 & 22.6 & \lightgray{34.5} & \lightgray{28.6} \\
    ProductCLIP & PID-233-RPN & 10 & 21.1 & \lightgray{31.7} & \lightgray{26.4} \\
    ProductCLIP & PID-233-RPN & 100 & 16.3 & \lightgray{24.5} & \lightgray{20.4} \\
    
    \bottomrule
    \end{tabular}
    }
\end{table*}


\begin{table*}[t]
\caption{\xie{Effect of the number of prompts of the text encoder on PID.}}
% \caption{\ljc{Effect of the number of prompts of input of text encoder on PID.}}
  \label{num_prompt_PID}
  \centering
  \resizebox{0.7\textwidth}{!}{
    \begin{tabular}{ccccccccc}
    \toprule
    \multirow{2.5}{*}{Visual Encoder Pre-training} &
    \multirow{2.5}{*}{Pre-trained RPN} &
    \multirow{2.5}{*}{\#prompt} &
    \multicolumn{3}{c}{Generalized (233+233)}
    \\ 
    \cmidrule(r){4-6}
    &  & & Novel & \lightgray{Base} & \lightgray{Overall}  \\
    \midrule
    ProductCLIP & PID-233-RPN & 93 & 22.0 & \lightgray{34.1} & \lightgray{28.1} \\
    ProductCLIP & PID-233-RPN & 1 & 22.3 & \lightgray{34.3} & \lightgray{28.3} \\
    \bottomrule
    \end{tabular}
    }
\end{table*}


\begin{table*}[t]
\caption{\ljc{Effect of prompt tuning on PID.}}
  \label{prompt_tuning_PID}
  \centering
  \resizebox{0.8\textwidth}{!}{
    \begin{tabular}{ccccccccc}
    \toprule
    \multirow{2.5}{*}{Visual Encoder Pre-training} &
    \multirow{2.5}{*}{Pre-trained RPN} &
    \multirow{2.5}{*}{Prompt Tuning} &
    \multicolumn{3}{c}{Generalized (233+233)}
    \\ 
    \cmidrule(r){4-6}
    &  & & Novel & \lightgray{Base} & \lightgray{Overall}  \\
    \midrule
    % ProductCLIP & BD*-489-PID-233-RPN & zero-shot  & No & 15.99 & 15.44 & 15.72 \\
    % ProductCLIP & BD*-489-PID-233-RPN & zero-shot & Yes & 14.92 & 14.26 & 14.59 \\
    % \midrule
    ProductCLIP & BD*-489-PID-233-RPN  & $\no$ & 27.1 & \lightgray{35.8} & \lightgray{31.5} \\
    ProductCLIP & BD*-489-PID-233-RPN  & $\yes$ & 28.1 & \lightgray{35.6} & \lightgray{31.9} \\
    \bottomrule
    \end{tabular}
    }
\end{table*}


\subsection{Ablation study on PID.}\label{PID_Ablation}
% ... Ablation study...
\ljc{
\textbf{Summary of ablation study.} Several related works \cite{ViLD, regionclip, bangalath2022bridging, BARON} in open-vocabulary detection have applied a variety of techniques to improve the model performance. However, it is worth noting that some works use a part of these techniques while others do not, which makes this line of work confusing. That is, it is hard to use existing techniques to improve the model performance when transferring to a new dataset, \eg PID. Here, we investigate the effectiveness of commonly used techniques on PID, including different proposal generators, multiplying RPN score, knowledge distillation, and prompt learning. We use DRR to conduct all experiments on PID.
% Meanwhile, we use \xie{DRR} to conduct all ablation experiments on PID.
}

\ljc{
\textbf{Why use \xie{DRR} to conduct ablation study on PID?} From Table \ref{ft_results_pid}, it seems that the vanilla method is a better choice compared to \xie{CRR} and \xie{DRR}. However, it suffers from slow inference speed \cite{ViLD} and achieves unsatisfactory results on the two popular public OVD datasets, \ie OVD-COCO and OVD-LVIS (see Tables \ref{sota-coco} \& \ref{sota-lvis} \& \ref{ft_results_pid}). Instead, \xie{DRR} outperforms the vanilla method and \xie{CRR} on OVD-COCO and OVD-LVIS and is with excellent computational efficiency (see Table \ref{fps_cost}). Since the main goal is to investigate which technique is more effective when transferring to new datasets, we adopt DRR to conduct the ablation study on PID.
% \TODO{Note that the results in Table \ref{ft_results_pid} have been applied to the significant techniques.}
}


\ljc{
\textbf{The effect of RPNs and the ensemble score.} 
\xie{First, we train a CLIP model using image-text product pairs and name it ProductCLIP.} Note that the text encoder of ProductCLIP is designed for Chinese. We then train an offline RPN on the base categories of PID for \xie{DRR}, called PID-233-RPN. Similar to Table \ref{detector}, we investigate the effectiveness of ``multiplying RPN score'' during inference. We also train a BD*-489-PID-233-RPN to further improve the quality of bounding boxes and objectness logits. From Table \ref{multiply_rpn_score_PID}, the model with ``multiplying RPN score'' can achieve better detection performance over base and novel categories. When improving the performance of the offline RPN, the Novel AP$_{50}$ improves from 26.6 to 30.7. We can obtain similar observations as in Table \ref{detector}.
% After fine-tuning on the base categories of the specific dataset like COCO, most existing models (citation xxx) still do poorly at recalling novel concepts when inference. 
% To retain the knowledge of all concepts captured by the underlying CLIP model, they ensemble the detection score and CLIP score to balance the final score. They obtain the CLIP score by cropping the proposals, resizing them, feeding them into the visual encoder of CLIP,  and calculating the cosine similarity with the text embeddings encoded by the text encoder of CLIP. Note this approach has a slow inference speed. Instead, we follow RegionCLIP (citation xxx) and ensemble the detection score and RPN score when inference, which can mitigate the above inference speed issue. In Table \ref{multiply_rpn_score_PID}, the model with ensemble scores can achieve better detection performance over base and novel categories.
}



\ljc{
\textbf{The effect of knowledge distillation.}
% Frist, we use the visual encoder of ProductCLIP, which is trained on image-text product pairs. 
% We then use the PID-233-RPN to obtain the proposals of input images, where the PID-233-RPN means that we individually train an RPN on the base categories of PID.
Knowledge distillation is a commonly used technique to bridge the gap between the object-level representations from detectors and the image-level representations from CLIP.
Thus, they can obtain significant classification scores from the object-level representations and the embeddings from the text encoder of CLIP.
Following \cite{ViLD,bangalath2022bridging}, we use L1 knowledge distillation to check whether this technique is valid for PID.
We also adjust different values of temperature in knowledge distillation to find a good hyperparameter.
% \TODO{The teacher and the student when conducting distillation.} 
Table \ref{KD_PID} summarizes the results.
We observe that \xie{DRR} with knowledge distillation (KD) achieves slightly better results than the one without KD, \ie 22.6 vs. 22.0. One possible reason is that we pre-train the visual backbone of CLIP using image-text product pairs, that is, there is no gap between the object-level representations from the visual backbone with the text encoder.
% \TODO{The analysis of the results.}
}





\ljc{
% \textbf{The effect of the number of prompts of input of text encoder.} 
\xie{\textbf{The effect of prompt numbers.} }
In the setup of open-vocabulary detection, we usually obtain the text embeddings by directly feeding the concepts with human-crafted prompts into the text encoder of CLIP. 
% Here, We evaluate the different number of prompts of input if text encoder in Table \ref{num_prompt_PID}.
\xie{Here, We evaluate different number of prompts of the text encoder in Table \ref{num_prompt_PID}.}
We observe that the effectiveness of the number of prompts can be negligible on PID.
}




\ljc{
\textbf{The effect of prompt tuning.} Next, we introduce another common technique in prompt modification. We replace the human-crafted prompt with a learned prompt by prompt tuning \cite{zhou2022learning}. Table \ref{prompt_tuning_PID} summarizes the results. The models with prompt tuning can improve by 1.0 Novel AP$_{50}$, indicating that prompt tuning is an effective technique to improve the model performance.
}







% \newpage
% \subsection{Benchmark Results}\label{sec:results}
% % \green{We first compare our \pretrainmodel{} with the state-of-the-art models on the image-text retrieval task, which contains two sub-tasks: image-to-text retrieval (TR) and text-to-image retrieval (IR).} 
% \textbf{Experiment on COCO-OVD Dataset.}



% \textbf{Experiment on PID Dataset.} 



% \textbf{Analysis of XXXCrop.}

% \subsection{Futher analysis}




% \newpage
% \subsection{Ablation Study
% }\label{sec:ablation}


% \newpage
\section{Conclusion and Social Impacts}
\label{conclusion}
In this paper, we have conducted a comprehensive study of three fundamental commonly used approaches for open-vocabulary detection, a vanilla method, decoupling RPN and ROI head (\xie{DRR}), and coupling RPN and ROI head (\xie{CRR}). We analyze the advantages and the disadvantages of the three approaches and give a discussion about how and when to use them. Extensive experiments demonstrate the effectiveness of different approaches under different setups on COCO and LVIS benchmarks. Besides, we propose a product dataset for object detection called PID and provide a strong baseline on PID. We hope our analyses and datasets promote the development of open-vocabulary detection.

\textbf{Acknowledgement.} This work was supported in part by the National Key Research and Development Program of China under Grant 2018AAA0100405. We thank Bang Yang for the help in collecting CC3M. We also thank DeXin Wang and Dan Zhou for the help in collecting PID.




%%%%%%%%% REFERENCES
% {\small
% \bibliographystyle{ieee_fullname}
% \bibliography{egbib}
% }
\bibliographystyle{ACM-Reference-Format}
\bibliography{references}





% %%%%%%%%%%%%%%%%%%%%%%%%%%%%%%%%%%%%%%%%%%%%%%%%%%%%%%%%%%%%
% \newpage
% %%%%%%%%%%%%%%%%%%%%%%%%%%%%%%%%%%%%%%%%%%%%%%%%%%%%%%%%%%%%

% \newpage
% \appendix

\end{document}