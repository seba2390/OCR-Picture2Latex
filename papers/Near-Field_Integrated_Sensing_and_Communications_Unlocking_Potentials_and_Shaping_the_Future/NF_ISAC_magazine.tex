
%\documentclass[ conference,twocolumn,twoside]{IEEEtran} % formal style
\documentclass[10pt,journal,twocolumn,twoside]{IEEEtran} % formal style
%\documentclass[12pt,draftclsnofoot,onecolumn,journal]{IEEEtran} %draft style
%\renewcommand{\baselinestretch}{1.8}
%\documentclass[journal,comsoc]{IEEEtran}
\IEEEoverridecommandlockouts
\usepackage{setspace}%使用间距宏包
\usepackage{subfigure}
\usepackage{graphicx}
\usepackage{epstopdf}
\usepackage{float}
\usepackage{amsmath}
%\usepackage{algorithm}
%\usepackage{algorithmicx}
\usepackage[ruled]{algorithm2e}
\usepackage{algpseudocode}
\usepackage{array}
\usepackage{amsthm}
\usepackage{amsmath}
\usepackage{amssymb}
\usepackage{mdwmath}
\usepackage{mdwtab}
\usepackage{eqparbox}
\usepackage{stfloats}
\usepackage{fixltx2e}
%\usepackage{hyperref}
\usepackage{cases} 
\usepackage{caption}
\usepackage{graphicx}
\usepackage{float} 
%\usepackage{subfigure}
%\usepackage{subcaption}%

\usepackage{flushend}

\usepackage{xcolor}
\usepackage{makecell}

\usepackage{url}
\usepackage{cite}
%==============================================%
%\ifCLASSOPTIONcompsoc
%\usepackage[caption=false,font=normalsize,labelfont=sf,textfont=sf]{subfig}
%\else
%\usepackage[caption=false,font=footnotesize]{subfig}
%\fi
%==============================================%
\allowdisplaybreaks[4]
\newtheorem{condition}{Condition}
\newtheorem{assumption}{Assumption}
\newtheorem{corollary}{\bf Corollary}
\newtheorem{theorem}{\bf Theorem}
\newtheorem{proposition}{\bf Proposition}
\newtheorem{lemma}{\bf Lemma}
\newtheorem{example}{Example}
\newtheorem{notation}{Notation}
\newtheorem{definition}{\bf Definition}
\newtheorem{remark}{Remark}
\newtheorem{property}{\bf Property}
\usepackage{wrapfig}
\newcommand{\figref}[1]{Fig. ~\ref{#1}}
\newcommand{\theoremref}[1]{Theorem~\ref{#1}}
\newcommand{\tabref}[1]{Table~\ref{#1}}
\newcommand{\lemmaref}[1]{Lemma~\ref{#1}}
\newcommand{\propref}[1]{Proposition~\ref{#1}}
\newcommand{\corref}[1]{Corollary~\ref{#1}}
\newcommand{\appref}[1]{Appendix~\ref{#1}}
\newcommand{\propertyref}[1]{Property~\ref{#1}}
\newcommand{\sectionref}[1]{Section~\ref{#1}}
\newcommand{\tableref}[1]{Table~\ref{#1}}
% Define my mathematic operators
\newcommand{\diag}{\mathop{\mathrm{diag}}}
\newcommand{\tr}{\mathop{\mathrm{trace}}}
\newcommand{\Ei}{\mathop{\mathrm{E}_1}} % Exponential integral
\renewcommand{\vec}{\mathop{\mathrm{vec}}}
\captionsetup[figure]{name={Fig.},labelsep=period,font={small}}
\newcommand{\colorr}{\color{red}}

\renewcommand{\vec}{\mathop{\mathrm{vec}}}
\newcounter{MYtempeqncnt} % Used for double column equation
\makeatletter
\renewcommand{\maketag@@@}[1]{\hbox{\m@th\normalsize\normalfont#1}}%
\makeatother
\makeatletter
%\renewcommand*{\@opargbegintheorem}[3]{\trivlist\item[\hskip \labelsep{\bfseries #1\ #2}] \textbf{(#3):}\ }
\makeatother     
\UseRawInputEncoding 

%\setlength{\abovecaptionskip}{-0.02cm}


%%%%%%%%%%%%%%%%%%%%%%%%%%%%%%%%%%%%%%%%%%%%%%%%%%%%%%%%%%%%
\begin{document}

\title
{Near-Field Integrated Sensing and Communications:  Unlocking Potentials and Shaping the Future}


\author{ Kaiqian Qu,~\IEEEmembership{Graduate Student Member, IEEE},  Shuaishuai Guo,~\IEEEmembership{Senior Member, IEEE}, \\Jia Ye,~\IEEEmembership{Member, IEEE}  and Nasir Saeed,~\IEEEmembership{Senior Member, IEEE} 
\thanks{ 
K. Qu and S. Guo are  with School of Control Science and Engineering, Shandong University, Jinan 250061, China, and also with Shandong Key Laboratory of Wireless Communication Technologies, Shandong University, China (e-mail: qukaiqian@mail.sdu.edu.cn,shuaishuai\_guo@sdu.edu.cn). }
\thanks{Jia Ye is with the School of Electrical Engineering, Chongqing University, Chongqing, 400044, China (yejiaft@163.com).}
\thanks{N. Saeed is with the Department of Electrical and Communication Engineering, United Arab Emirates University (UAEU), Al Ain, United Arab Emirates (e-mail: mr.nasir.saeed@ieee.org).}
   }
\maketitle

%\newpage

\begin{abstract}
The sixth generation (6G) communication networks are featured by integrated sensing and communications (ISAC), revolutionizing base stations (BSs) and terminals. Additionally, in the unfolding 6G landscape, a pivotal physical layer technology, the Extremely Large-Scale Antenna Array (ELAA), assumes center stage. With its expansive coverage of the near-field region, ELAA's electromagnetic (EM) waves manifest captivating spherical wave properties. Embracing these distinctive features, communication and sensing capabilities scale unprecedented heights.  Therefore, we systematically explore the prodigious potential of near-field ISAC technology. In particular, the fundamental principles of near-field are presented to unearth its benefits in both communication and sensing. Then, we delve into the technologies underpinning near-field communication and sensing, unraveling possibilities discussed in recent works. We then investigated the advantages of near-field ISAC through rigorous case simulations, showcasing the benefits of near-field ISAC and reinforcing its stature as a transformative paradigm.
As we conclude, we confront the open frontiers and chart the future directions for near-field ISAC.


%By harmoniously combining these two functions, ISAC unlocks a realm of untapped potential, conserving valuable hardware resources and enhancing spectrum utilization. %As the insatiable thirst for higher data rates and energy efficiency drives the demand for 6G networks, ISAC stands poised to meet these imperatives head-on. 

%Jia's revise: However, the deployment of an extensive number of antennas in a relatively confined space not only improves spatial resolution and beamforming gain but also provides more potential for near-field communications, where the EM waves can no longer be approximated as planar waves. When ISAC embracing ELAA, its communication and sensing capabilities scale unprecedented heights.

%Jia's revise:
%Therefore, we systematically explore the prodigious potential of near-field ISAC technology and present the fundamental principles of the near-field, unearthing its benefits in communication and sensing, where beam-focusing properties reign supreme.
 %%Undoubtedly, an alluring prospect that will captivate the minds of researchers, spurring further exploration and innovation.


\end{abstract}

\begin{IEEEkeywords}
Integrated sensing and communication (ISAC), beamfocusing, near-field, spherical wave.
\end{IEEEkeywords}


\section{Introduction} 
The sixth generation (6G) networks are poised to drive the emergence of cutting-edge applications and empower the Internet of Everything (IoE). With the promise of remarkably lower latency ($0.1$ ms) and vastly improved transmission rates ($1$ Tbps),  6G opens up a realm of potential application scenarios, including digital twin, smart city, and smart home, among others. To achieve these ambitious visions, extensive research is being conducted on essential technologies such as Integrated Sensing and Communications (ISAC), Extremely Large-Scale Array (ELAA), and Reconfigurable Intelligent Surface (RIS) \cite{Jia2020}.
Among them, ISAC represents a transformative field at the intersection of sensing technology and wireless communications.
The convergence arises from the striking similarity in hardware structures, with communication and radar systems employing radio frequency (RF) chains and transceivers.
 On the other hand, the resurgence of ISAC technology is prompted by the scarcity of spectrum resources due to the increasing demand for wireless communication services. ISAC can lead to the overlapping of communication frequency bands with radar systems, creating opportunities for higher spectrum efficiency.
 At its core, ISAC accomplishes an ingenious feat by facilitating spectrum and hardware sharing between radar and communication, optimizing resource allocation, and creating a harmonious synergy between sensing and communication. This remarkable capability optimizes resource allocation and creates a harmonious synergy between sensing and communication. 
The mutual benefits derived from this integration align seamlessly with 6G's pursuit of high-speed data transmission and ultra-low latency, solidifying ISAC's position as one of the pivotal technologies for the next-generation network \cite{CongFinite2022, ShuaishuaiGuo:Mobicom2022}.

Furthermore, in pursuing higher performance in 6G, advancements in antenna structures, such as ELAA and RIS, play a pivotal role. ELAA and RIS offer the advantage of increased beamforming gain facilitated by a higher number of antennas or reflecting elements. However, adopting these technologies introduces a significant change in the system's EM (EM) properties. The EM field is commonly divided into the near-field and far-field regions, with the Rayleigh distance serving as a critical boundary defined by the array aperture and frequency. In the case of the original millimeter wave massive multiple-input multiple-output (MIMO) systems equipped with $32$ or $64$ antennas, the Rayleigh distance is negligible, allowing the assumption of a far-field scenario. Consequently, the EM wavefront is approximated as a plane wave in such studies.
However, for ELAA and RIS technologies, the Rayleigh distance can extend to hundreds or even thousands of meters, signifying a substantial transition to the near-field regime. 
For instance, the Rayleigh distance for an ELAA system with a length of 7.4 m operating at 2.6 GHz reaches an impressive 950 m. 
For instance, an ELAA system with a 7.4 m Rayleigh distance operating at 2.6 GHz reaches an impressive 950 m.
 In the near field, EM waves exhibit distinctive spherical wave characteristics with complex and spatially varying propagation patterns. The spatial dispersion presents both challenges and opportunities for designing and optimizing 6G systems.  
Understanding and effectively managing these near-field effects are critical in fully harnessing the potential of ELAA and RIS technologies to maximize their benefits in the next generation of communication systems.


Spherical wavefronts can be effectively utilized to generate highly focused beams within specific spatial regions, a phenomenon known as beam focusing. By exploiting spherical wavefronts, it becomes possible to focus the energy on desired areas, enabling precise localization and enhanced performance in targeted regions. Unlike traditional far-field beam steering, where signals can only be directed toward a specific direction, this approach provides unprecedented control and achieves superior beamfocusing capabilities. 
The existing studies indicate that near-field beamfocusing (or spherical wave) can help decorrelate muti-user channel \cite{sperial}, increase spatial degrees of freedom (DoF) \cite{lu}, and provide new multi-access methods\cite{ldma}. In addition, angle and distance information is carried in the spherical wave, providing additional insights for target localization\cite{9508850}.
 As a result, the distinctive properties of the near field hold the promise of enhancing communication and sensing performance, opening up new and exciting opportunities for ISAC \cite{Wang2023NearFieldIS}. However, despite its potential, near-field ISAC remains relatively unexplored, prompting us to embark on a systematic exploration of its capabilities.


In this article, we systematically introduce the potential near-field ISAC technology. 
The key features of the article are summarized as follows:
\begin{itemize}
   \item \textbf{Advancing Near-Field Understanding:} We comprehensively explore the near-field and far-field regions, emphasizing the unique characteristics of near-field spherical waves. Highlighting the communication and sensing benefits of near-field beam focusing properties, our study lays the groundwork for harnessing the full potential of near-field technology ISAC systems.
    \item
    \textbf{Empowering Near-Field ISAC Applications:} We provide a thorough overview of near-field communication and near-field sensing, analyzing their practical applications and showcasing the transformative impact of near-field techniques. By showcasing real-world use cases, we demonstrate how near-field technology can revolutionize communication and sensing capabilities, offering unprecedented opportunities for innovative ISAC solutions.
    \item 
    \textbf{Unlocking Potentials and Shaping Future:} In addition to unveiling the advantages of near-field ISAC, we identify critical research challenges that warrant attention in the near future. We investigate the compatibility of the near-field realistic EM model with existing mathematical frameworks and explore how the rich degrees of freedom in the near-field can be optimally harnessed for communication and perception tasks. Furthermore, we propose potential improvements to redefine near-field distance, paving the way for more accurate and efficient ISAC systems in the near-field regime. We aim to inspire future research efforts and advancements in this dynamic field through these insights.
\end{itemize}
\section{Near-field Communications and Sensing}
In this section, we present the foundations of near-field, the benefits of near-field for communication and sensing, and a discussion on technologies to be revisited in near-field ISAC.
\subsection{What is Near-field?}
\begin{table*}[htbp]
       \centering
       \includegraphics[width=1\linewidth]{table.eps}
       \caption{Near- and Far-field Models Comparison. The following abbreviations are used for brevity: direction of arrival (DoA),  spatial division multiple access (SDMA), location division multiple access (LDMA).}
       \label{Model}
\end{table*}
In EMs and acoustics, near-field and far-field are two regions that describe the distance relationship to a radiation source. 
When signals propagate in free space, they spread out in a spherical pattern. In the far-field region known as the radiation zone or Fraunhofer region, at the distance the distance from the source of the wavefront is much greater than its size, the curvature of the spherical wave is small and can be approximated as a plane wave. 
%When signals propagate in free space, they spread out in the shape of spherical waves. At a longer distance (far-field region), the curvature of the spherical wave is small and can be approximated as a plane wave. 
However, in the near-field region known as the reactive or Fresnel region, the distance from the source of the wavefront is comparable to or smaller than its size, the wavefront remains distinctly spherical, as illustrated in Table \ref{Model}.
In reality, there is no strict distance threshold between the near and far fields; it must be determined based on specific applications and scenarios. 
The most commonly used criterion to distinguish between the far-field and near-field regions is the Rayleigh distance, which accounts for the phase difference caused by the curvature of the EM wave between the center and boundary of the receiving array.
 When the phase difference is less than 22.5 degrees, it is considered a small curvature, and the wavefront can be approximated as a plane wave; 
Otherwise, it keeps the spherical wave. Mathematically, as defined in Table \ref{Model}, the Rayleigh distance is proportional to the product of the carrier frequency $\frac{1}{\lambda}$ and the square of the array aperture size, $D^2$. 


Unlike the far-field plane wave, where the wavefront's behavior is primarily determined by the angle of propagation, near-field introduces an additional distance dimension, playing a significant role in determining the wavefront's shape and behavior. 
Unlike the far-field plane wave, the spherical wave introduces the distance dimension. 
%Jia's revision: From the phase perspective, the phase difference between the antennas under the plane wave approximation is linear, but is nonlinear under the spherical wave pattern. (I suggested deleted this sentence, as it is repeated in the next section, and the relationship between non-linear phase and near-field beam focusing is not very clear. )
%Under the plane wave assumption, the phase difference between the antennas is linear, while it is nonlinear under the spherical wave assumption. 
%Jia's revision: Due to these facts, near-field beam focusing differs from conventional far-field beam steering. In the far-field, the steering vector corresponds to a specific direction, while in the near-field, the focusing vector aligns with a specific position or point.
This leads to the fact that the far-field steering vector only corresponds to a certain direction, while the near-field focusing vector corresponds to a specific position (point). This is the difference between far-field beam steering and near-field beam focusing. 

\subsection{Benefits of Near-field for Communication and Sensing}
\subsubsection{\textbf{For Communication}}
On one hand, the characteristics of near-field spherical waves can increase the DoFs of the channel and improve the system capacity \cite{10129110}. In a typical far-field millimeter-wave point-to-point communication system with $N_t$ transmitting antennas and $N_r$ receiving antennas, the phase difference between the transceiver antenna is linear. Therefore, the line of sight (LoS) channel is simplified as the product of the steering vector with only one DoF. In the near field, a big difference is that the phase between the transceiver antennas is nonlinear, which is related to the distance of each antenna. In other words, in the far field, all antennas at the receiving or transmitting end are assumed to share a single distance value.  
However, in the near field, each antenna has its own distance information\footnote{This is related to the center distance between the transceiver. When the distance is small enough, the distance between each antenna varies significantly, providing the highest degree of freedom. As the distance increases slightly, several adjacent antennas share distance information and the DoF gradually decreases until it drops to 1 in the far field.}, and the DoF is equal to $\min\{N_t,N_r\}$, as shown in Fig. 1. Similarly, the above analysis still holds in multiuser MIMO. Moreover, in the far field, users located at the same or similar angles cannot be distinguished, leading to a decrease in the DoFs. Whereas in the near field, these users can be distinguished by different distances, effectively improving the DoFs.

On the other hand, the additional distance dimension in the near field introduces a new dimension to mitigate multi-user interference. In the traditional far-field communication model, it is not possible to differentiate between users with the same angle, leading to significant inter-user interference. In contrast, the near field can distinguish users based on different distances and effectively mitigate inter-user interference. 
In addition, a similar benefit is that the spherical wave helps to decorrelate the multi-user channel to make it close to the optimal propagation condition, as shown in Fig. 2.


 
\subsubsection{\textbf{For Sensing}}
The process of sensing can be divided into signal transmission and reception. 
 In near-field signal transmission, the use of focused and tightly confined beams is possible. This allows for focusing the signal energy on the target point with reduced signal spreading, leading to improved signal strength at the target. 



In the stage of signal reception processing, near-field spherical waves carry both distance and angle information. 
In addition, near-field reception offers the advantage of selective signal capture, enabling targeted reception from specific transmitters and minimizing the impact of unwanted signals or interference. To derive precise position information of the target, near-field reception can be combined with advanced estimation techniques and spectral search algorithms, resulting in significantly higher accuracy compared to the far-field scenario. For instance, the use of multiple antenna arrays with known spatial configurations allows for indirect inference of distance information based on signal power, time of arrival, or phase differences between antennas. Complementing this process, the  two-dimensional multiple signal classification (2D-MUSIC) algorithm, a widely used spectral search algorithm, effectively estimates the target's direction of arrival (DoA) in two dimensions (azimuth and elevation) using a single array. This integration of near-field reception and innovative estimation methods culminates in a comprehensive localization solution, offering highly accurate position information.
\begin{figure}[t]
       \centering
       \includegraphics[width=1\linewidth]{dof.eps}
       \caption{The spatial degrees of freedom of point-to-point MIMO system varies with the distance between transmitter and receiver, $N_t=N_r=256$.}
       \label{fig1}
\end{figure}

\begin{figure}[t]
       \centering
       \includegraphics[width=1\linewidth]{correlation.eps}
       \caption{Squared-correlation coefficient versus antenna number for the Plane and Spherical wave models. The two users are located in the same direction.}
       \label{fig2}
\end{figure}

However, the distinct benefits introduced by near-field in both communication and sensing heavily rely on appropriate transmission design. Conventional transmission design used in far-field scenarios cannot be directly applied to near-field environments, resulting in noticeable performance loss due to model mismatch at close range.  A prime example is beamforming, where the conventional techniques designed for far-field do not account for the complex spatial variations in the near-field, including varying signal strengths and phase shifts across the antenna array. Furthermore, in the near-field, far-field beams become divergent and wider at close ranges, leading to increased user interference and angle estimation errors. This impairment of far-field beamforming gain significantly diminishes communication rates and sensing accuracy in the near-field. To unlock the full potential of near-field benefits, specialized transmission designs and advanced beamforming techniques are imperative. These approaches address the challenges posed by the unique propagation characteristics in the near-field, ensuring efficient signal transmission, improved communication rates, and enhanced sensing accuracy for various applications.

Absolutely, the beam-focusing characteristics of near-field waves offer significant advantages for both communication and sensing applications. As a result, it is reasonable to expect that the performance of near-field ISAC will also be notably improved. The combination of communication and sensing capabilities in near-field ISAC can lead to synergy, where the information gathered from sensing can improve communication performance, and vice versa. 
\subsection{Technologies to be Revisited in Near-Field ISAC}
The emergence of the near field has attracted the attention of researchers in widespread fields. In this subsection, we briefly introduce the communication and sensing technologies to be revisited in the near field.

\subsubsection{\textbf{Near-field Channel Estimation}}
 Channel estimation plays a pivotal role in enhancing the performance of ISAC systems. In fact, the target sensing process in ISAC networks can be likened to channel estimation in communication aspects, with the distinction that sensing operates in a backscatter channel. Accurate channel estimation in ISAC systems, therefore, yields advantages for both communication and sensing domains. It enables the deployment of advanced signal processing techniques, enhances spectral efficiency, improves localization accuracy, and fosters more reliable and efficient communication and sensing operations. As a result, precise channel estimation stands as a critical technology, empowering ISAC to achieve its full potential across diverse applications.

 Near-field channel estimation offers several advantages over far-field channel estimation due to the distinctive properties of the near-field region. It benefits from enhanced angular resolution and increased channel diversity, enabling more precise and robust estimation. Furthermore, in the near-field, the effects of multipath propagation are less pronounced compared to far-field scenarios. This reduction in multipath interference simplifies the channel estimation process and leads to more accurate and stable estimates. However, the structural changes in near-field EM waves caused by large-scale arrays render conventional channel estimation methods ineffective for near-field channel estimation, necessitating the development of revised near-field channel estimation schemes. Advanced techniques tailored to the unique characteristics of near-field wavefronts are essential for unlocking the full potential of near-field channel estimation. 
The structural changes in near-field EM waves caused by large-scale arrays render conventional channel estimation methods ineffective for near-field channel estimation. 
In \cite{8949454}, the authors uniformly divided the two-dimensional physical space into multiple grids, each corresponding to a near-field array response vector. These near-field response vectors form a codebook for compressive sensing-based recovery of channel information. However, in recent studies, the orthogonality of near-field polar regions has been demonstrated \cite{9693928}. Since the correlation of the near-field beam varies non-uniformly along the distance dimension, a polar domain non-uniform sampling codebook is designed in \cite{9693928}, which can match the near-field channel well.

\subsubsection{\textbf{Near-Field Multiple Access}}
Spatial division multiple access (SDMA) is a key technology used to enhance MIMO communication's spectrum efficiency. 
In ISAC networks, SDMA can be particularly beneficial in scenarios where communication and sensing tasks need to be performed simultaneously or in close temporal proximity. SDMA's capability to create spatially distinct channels allows for simultaneous and independent signals transmission to multiple users and targets, providing a valuable solution for effectively managing communication and sensing tasks in close temporal proximity. By employing SDMA, ISAC systems can allocate resources efficiently, optimize data transmission, and enhance the overall network capacity. 
 ELAA introduces an additional resolution in the near-field domain, allowing spatial resources to be further divided into different grids based on distance and angle. The near-field beam focusing effect can be leveraged to enhance the spatial resolution of SDMA, allowing for more precise beamforming and targeted communication and sensing. The spatial selectivity offered by SDMA becomes even more pronounced in near-field, enabling more efficient utilization of available spatial resources for simultaneous data transmission to multiple users. However, it also presents challenges related to beam steering and interference management that need to be addressed through specialized multiple access schemes.
According to the characteristics of the near field, location division multiple access (LDMA)  \cite{ldma} is a promising near-field multiple access technology. The core idea is to utilize additional spatial resources in the distance domain to serve different users in different positions in the near field (determined by angle and distance).
\begin{figure*}[t]
       \centering
       \includegraphics[width=1\linewidth]{manfig2.eps}
       \caption{The normalized signal power measurement of beams, (a) Comm-only NFBF; (b) Trade-off $\rho=0.5$ NFBF; (c) Radar-only NFBF; (d) Comm-only FFBF; (e) Trade-off $\rho=0.5$ FFBF; (f) Radar-only FFBF; }
       \label{fig5}
\end{figure*}
%\subsection{Near-field Sensing}
\subsubsection{\textbf{Near-field Positioning Technology}}
 Source localization and tracking are based on the joint estimation of the angle of arrival (AoA) and the time of arrival (ToA), which are essential functionalities in ISAC networks. ISAC networks with source localization and tracking capabilities find applications in various domains, including indoor localization, smart environments, wireless sensor networks, and surveillance systems. They enable intelligent and responsive systems that can efficiently perform communication tasks while accurately localizing and tracking signal sources, providing valuable insights and enabling better decision-making in dynamic environments. However, it requires good synchronization between transmitters and receivers or the involvement of multiple nodes, which has certain limitations.  The good news is that near-field propagation offers several advantages that significantly improve the performance of source localization and tracking in ISAC networks. 
When the antenna array is large enough, the EM waves exhibit a spherical wave shape in the near field of ELAA. A promising approach is directly extracting the source position from the spherical wavefront impinging on a single large array, which does not require synchronization \cite{9508850}. This is a new insight into the positioning technology brought by the characteristics of the near-field spherical wave, and it may also become a new technology to realize holographic positioning.
\subsubsection{\textbf{Near-field Signal Processing} }
Signal processing techniques in ISAC networks are crucial for enabling efficient communication, accurate source localization, and robust interference mitigation. Among the fundamental signal processing tasks in ISAC networks, DoA estimation and beamforming are particularly representative. DoA estimation is responsible for determining the angles at which signals arrive at the array of antennas, while beamforming focuses transmitted or received signals towards specific positions and suppresses interference from other directions. These two tasks are closely related to each other, as accurate beamforming can be performed after DoA acquisition and precise DoA estimation benefits from accurate beamforming. 
 However, conventional beamforming design used in far-field scenarios cannot be directly applied to near-field environments as it does not account for the distinct complex spatial variations, such as varying signal strengths and phase shifts across the antenna array. The mismatch results in noticeable performance loss due to model mismatch at close range, which also prevents effective DoA estimation. Furthermore, in the near-field, far-field beams become divergent and wider at close ranges, leading to increased user interference and angle estimation errors. This impairment of far-field beamforming gain significantly diminishes communication rates and sensing accuracy in the near-field. To unlock the full potential of near-field benefits in ISAC networks, specialized beamforming techniques and advanced DoA estimation algorithms are imperative \cite{9723331,8359308}. These approaches are expected to address the challenges posed by the unique propagation characteristics in the near-field, and the trade-off problems between communication performance and sensing accuracy.
As it is known under the previous description, the received beam formation under far-field conditions suffers from angular dispersion and gain degradation in the near field, which prevents effective DoA estimation. Therefore, near-field beamforming and DoA estimation are necessary \cite{9723331,8359308}. 

\section{Experiment and Discussion}
To demonstrate the performance and potential of near-field ISAC, we conducted several numerical experiments. We consider a near-field ISAC scenario, where the base station (BS) is equipped with uniform linear arrays (ULA) with 256 transmitting antennas and 256 receiving antennas. The BS serves two single-antenna users for the simultaneous execution of single-target sensing tasks (both in the near field of BS). In order to emphasize the characteristics of the near field, the users are assumed to be located at the same angle with coordinates $(0^\circ, 5~m)$, $(0^\circ, 15~m)$ and the target is located at $(45^\circ, 5~m)$. In addition, the system operates at $30$ GHz and each user channel contains two scattering paths and one LoS path. We reasonably assume that the scattering paths of the two users do not overlap angularly.  
To conduct a comprehensive comparison, we evaluated both near-field beamforming (NFBF) and far-field beamforming (FFBF) designs\footnote{Note that the zero-forcing (ZF) precoding is adopted for the communication part of ISAC beamforming.}.

We first show the effect of integrated beam design in Fig. 3. with beam patterns designed under different weighting factors. Fig. 3. (a), (b), and (c) demonstrate the excellent beam focusing on both users and targets. Also, the beams of two users at the same angle are well distinguished, which verifies the effectiveness of near-field beam focusing in inter-user interference management. 
In Fig. 3. (e), (d), and (f) the beamforming design based on the traditional far field has a large performance loss caused by a mismatch, in which two users cannot be distinguished well, and the gain loss is serious. 
It is worth noting that the FFBF points the beam energy more toward the scattering path rather than the LoS path. This can be attributed to the use of ZF precoding in FFBF, which relies on the scattering path to distinguish between two users, ensuring that they do not overlap in the angular direction. Consequently, more energy is allocated to the scattering direction to reduce interference between the two users. As a corollary, the communication performance of FFBF is limited by the number of scattering paths in the propagation environment. This constraint can impact the overall FFBF-enabled communication efficiency in scenarios where scattering paths are limited or not well-distributed.



 Next, we analyze the trade-off performance between communication and sensing under the NFBF and FFBF designs in Fig. 4, where the sensing performance is evaluated via the root Cramer-Rao bound (RCRB) .
 The curve clearly illustrates the existence of a trade-off, regardless of whether NFBF or FFBF is employed.
 The communication rate of FFBF saturates around $5$ bit/Hz/s quickly, whereas NFBF reaches a saturation rate of approximately $24$ bit/Hz/s.
This is also due to the fact that FFBF can only rely on the scattering path to distinguish between two users, and there is a beamforming gain loss resulting from the inaccurate plane wave assumption.
Furthermore, it can be found that the estimation performance of FFBF deteriorates with a decrease in target distance, which contrasts with NFBF. The degradation is attributed to the more substantial gain loss experienced by FFBF in the near field as the distance decreases, resulting in lower power at the target and poorer estimation performance. Overall, the analysis indicates that NFBF exhibits a better trade-off performance in this regard.
 

\begin{figure}[t]
       \centering
       \includegraphics[width=1\linewidth]{trade.eps}
       \caption{RCRB versus communication rate, a single target is located at different distance $r$.}
       \label{fig3}
\end{figure}
\begin{figure}[t]
       \centering
       \includegraphics[width=1\linewidth]{power.eps}
       \caption{Transmitted power varies with the SINR threshold of the user.}
       \label{fig4}
\end{figure}

In Fig. 5, we aim to reveal the power-saving potential of near-field ISAC. 
Since the estimation performance of the target is positively correlated with the signal power at the target location, we compare the required minimum transmit power under the constraint of satisfying the communication SINR threshold and maintaining a fixed power at the target. 
 Clearly, as the SINR threshold increases, FFBF requires more power than NFBF to satisfy the SINR. This observation aligns with the previous analysis, where FFBF relies on scattering paths with higher pathloss, necessitating higher power to meet the SINR requirements.
 Furthermore, a noteworthy observation arises in the context of near-field beam focusing: the closer the distance between BS and users, the more pronounced the focusing effect becomes. This compelling finding suggests a hypothesis that the communication rate in near-field ISAC will exhibit an upward trend as the distance decreases. The increase in communication rate is owing to the enhanced beam focusing effect, which allows for more precise differentiation of nearby users and consequently leads to improved interference reduction.


 In summary, the comprehensive analysis and results presented highlight the significant advantages of implementing near-field ISAC in future 6G large-scale antenna systems. NFBF demonstrates superior performance in communication rate, sensing accuracy, and power consumption compared to FFBF. The distinct properties of near-field propagation contribute to the success of ISAC. These findings underscore the immense potential of near-field ISAC in revolutionizing wireless communication and sensing technologies, paving the way for advanced and intelligent applications in diverse domains.



\section{Open Challenges and Future Directions} %Potential
In summary, beam focusing plays a crucial role in enhancing near-field ISAC performance and holds promising potential for energy conservation. Nevertheless, despite its benefits, there remain several potential challenges and avenues for further research in the field of near-field ISAC, which are briefly discussed in this section.
\subsection{Accurate Near-field Model} 
The significance of near-field models in contrast to far-field models has been demonstrated, highlighting their critical role in minimizing gain and accuracy losses in near-field conditions. However, it remains essential to ascertain whether the existing mathematical models genuinely capture the authentic near-field characteristics. Recent research, as mentioned in \cite{8736783}, indicates that current near-field models must be more accurate in accurately representing actual near-field EM behaviors. Specifically, these models often assume a near-field identical to the far-field counterpart, merely correcting for spherical propagation curvature. Regrettably, such models overlooked the range-dependent amplitude of the received signal, focusing solely on phase considerations. Furthermore, they neglected critical characteristics of the near-field signal source, such as transmit antenna type, size, and direction. Consequently, this oversight can significantly impact the signal received by the array.
\cite{8736783} contends that while the existing near-field model may not be entirely without merit, its limitations must be acknowledged and comprehended to avoid misinterpretation. Unfortunately, numerous studies continue to assess their data using the same standardized model upon which they are based, thereby masking the implications of model mismatches.
In light of these findings, it becomes imperative for researchers to recognize the constraints of current near-field models and explore novel approaches that better capture the intricacies of real-world near-field EM behaviors. This could lead to more accurate and reliable results in practical applications.
\subsection{Near-field Distance Improvement}
The distinction between the near and far fields is not strictly confined, and the Rayleigh distance, based solely on phase difference, fails to encompass all performance metrics comprehensively \cite{9508850}. Take, for instance, the 256-256 parallel-placed ULA system illustrated in Fig. 1, where the spatial DoF experiences gradual growth only within distances less than 70 meters, deviating significantly from the Rayleigh distance. Conversely, the near-field distance serves as a valuable reference and can serve as a decisive factor for employing near-field intelligent spectrum access and control techniques when the distance is below this threshold. As such, it becomes essential to devise near-field distances that capture various performance aspects, including the overall effectiveness of ISAC.
\subsection{Low-complexity Beamforming Design Algorithms}

The impact of the near-field is particularly pronounced in ELAA. With ELAA potentially consisting of a vast number exceeding $1024$ elements, the scale of beamforming or optimization problems in ISAC research becomes significantly larger. Hence, there arises a need to devise algorithms tailored for efficiently solving large-scale Quadratically Constrained Quadratic Programming (QCQP) challenges \cite{CongFinite2022}. Furthermore, deep learning (DL) emerges as a favored approach due to its inherent simplicity and potential to effectively mine near-field features from ELAA data, making it a viable option for addressing these complex challenges.

\subsection{Near-field Wideband ISAC}. 
Wideband ISAC holds great promise, especially in the Terahertz (THz) band. However, it faces challenges due to beam splitting in wideband scenarios. In near-field broadband systems, conventional frequency-independent phase shifter beamforming results in distinct focusing points at different frequencies, and this effect extends to different directions in the far field as well \cite{cui2021nearfield}. Such beam splitting hinders the generation of accurate broadband beams, thereby failing to fully exploit the gain potential offered by high bandwidth. To address this issue, a phase-delay focusing (PDF) method has been proposed in \cite{cui2021nearfield}, leveraging true time delay (TTD)-based beamforming techniques. Furthermore, it's important to note that the wideband DoA estimation methods used in the far field are not directly applicable to near-field wideband ISAC scenarios, necessitating the design of specialized algorithms for this purpose. Consequently, near-field ISAC for high-frequency wideband operations represents a worthwhile and significant research problem that merits attention and exploration.

\subsection{Near-field Complexity-Gain Tradeoff}
The antenna size and operational distance directly influence the effectiveness of near-field ISAC. Introducing the near-field model adds a new dimension, leading to increased complexity. Thus, it becomes crucial to strike a balance between performance gains and the accompanying complexity arising from the near-field considerations. For instance, determining the range where the benefits provided by the near field outweigh its added complexity becomes essential. In other words, under what circumstances should near-field ISAC be taken into account to achieve the most favorable outcomes?


\section{Conclusions}
The emergence of ELAA as a prevailing trend in 6G base stations has elevated the significance of the near-field region, where EM waves exhibit spherical wavefront, demanding its utmost attention and consideration. In light of the 6G vision, ISAC must also account for the near-field impact. This paper has expounded upon the potential of near-field ISAC techniques. Initially, we elucidated the fundamental aspects of the near field and compared its characteristics with those of the far field. Subsequently, we highlighted the manifold advantages of the near field for communication and sensing, accompanied by an in-depth examination of pertinent technological studies.
Moreover, our numerical results showcased the superior trade-off performance achieved through near-field ISAC designs, emphasizing the potential for power-saving advantages. As we conclude, we underscored some challenges that need to be addressed and identified promising avenues for future research in this domain.
Given the advancements and insights presented in this study, we firmly believe that near-field ISAC will undoubtedly find broader applications, proving indispensable in harnessing the true potential of 6G and beyond. Its integration will undoubtedly contribute to realizing the ambitious goals of future wireless communication systems and pave the way for unprecedented innovation.
\bibliographystyle{IEEEtran} 
\bibliography{IEEEabrv,bib}
\end{document}

