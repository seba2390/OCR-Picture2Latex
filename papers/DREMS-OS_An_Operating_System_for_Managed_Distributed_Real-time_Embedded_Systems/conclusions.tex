% $Id: conclusions.tex 3907 2013-10-22 06:01:27Z dabhishe $
\vspace{-0.1in}
\section{Conclusions and Future Work}
\label{sec:conclusions}
%Points to consider:

\iffalse
% folks, I am commenting this out since we are short on space (somewhat)
The \iap\ platform consists of a development environment and runtime
platform that provides an infrastructure to address several challenges
faced in deploying and managing a cluster of mobile embedded
platforms.  It provides the capability for domain specific modeling
and a novel real-time operating system with support for a distributed
component model and several management services for controlled and
managed deployment of applications on distributed computing nodes. 
\fi

This paper propounds the notion of managed distributed real-time and
embedded (DRE) systems that are deployed in mobile computing
environments. 
%These systems must be managed due to the presence of
%mixed criticality task sets that operate at different temporal and
%spatial scales, and share the resources of the DRE system. 
To that end, we described 
%The timing
% constraints and resource sharing require effective mechanisms that
% can assure both performance isolation and secure
% communications for their correct application operation.  To address these
% requirements, this paper describes
the design and implementation of a
distributed operating system called \iap\ OS focusing on a key
mechanism: the scheduler. 
%\iap\ OS is part of a larger project comprising both
%design-time and run-time tools.
We have verified the behavioral properties of the OS
scheduler, focusing on temporal and spatial process isolation, safe
operation with mixed criticality, precise control of process CPU
utilization and dynamic partition schedule reconfiguration.  We have
also analyzed the scheduler  properties of a
distributed application built entirely using this platform and hosted
on an emulated cluster of satellites.
\iffalse
% 4/15/2017: Abhishek, we are not showing any model-based design in
% this paper. So commented it out.
 showcasing the usefulness of
model-driven design-time tools with support for model checking and code
generation significantly reducing the complexity for the developer.  
\fi

We  are extending this operating system  to build an open-source FACE\textsuperscript{tm} Operating System Segment \cite{face}, called COSMOS (\textbf{C}ommon \textbf{O}perating \textbf{S}ystem for \textbf{M}odular \textbf{O}pen \textbf{S}ystems). To the best of our knowledge this is the first open source implementation of its kind that provides both ARINC-653 and POSIX partitions.


% To extend this work further, we are working on the response-time
% analysis on the task level and on design-time analysis and
% verification tools for the component-level scheduler, which 
% operates within each component scheduling the component's operations. 
% Additionally, such complex networked systems with mission-critical tasks 
% distributed among many nodes require guarantees about the network Quality of Service (QoS)
% for each task needing access to the network. However the temporal
% partitioning of the application tasks significantly affects 
% task access both to the CPU and to network resources.
% Finally, a more comprehensive fault diagnostics and response
% infrastructure is needed for robust cluster performance in adverse
% situations.   

