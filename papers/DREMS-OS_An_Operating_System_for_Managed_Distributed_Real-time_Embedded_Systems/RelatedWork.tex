\section{Related Research}
\label{sec:related}

Our approach has been inspired by two domains: mixed criticality systems
and partitioning operating systems. 
\iffalse
Mixed criticality systems provide
support to multiple functionalities that can be of different
criticality, or importance to the system. 
\fi
A mixed criticality
computing system has two or more criticality levels on a single
shared hardware platform, where the distinct levels are motivated by
safety and/or security concerns. For example, an avionics system can
have safety-critical, mission-critical, and non-critical tasks.

\iffalse
In his seminal paper on mixed criticality scheduling,
Vestal~\cite{Vestal2007} 
\fi
In~\cite{Vestal2007}, Vestal argued that the criticality levels directly
impact the task parameters, especially the worst-case execution time
(WCET). In his framework, each task has a maximum criticality level
and a non-increasing WCET for successively decreasing criticality
levels. For criticality levels higher than the task maximum, the
task is excluded from the analyzed set of tasks. Thus increasing
criticality levels result in a more conservative verification
process.  He extended the response-time analysis
of fixed priority scheduling to mixed criticality task sets.
His  results were later improved by Baruah et
al.~\cite{BaruahRTA4MCS} where an implementation was proposed for
fixed priority single processor scheduling of mixed-criticality tasks
with optimal priority assignment and response-time analysis.
%
%This work used a mixed criticality generalization of a sporadic
%task model analyzed for static and adaptive mixed criticality
%systems. Once a priority is assigned, the runtime complexity for
%these systems was only slightly more complex than non-mixed-criticality systems. 
%
%His work also shows that the effort required to achieve a WCET bound
%increases with the level of assurance that is expected of this bound. 
%Here $L_1$ and $L_2$ are criticality 
%levels ($L_1$ is the higher level), and $C(L_1)$ is the WCET estimate
%for a task on level $L_1$:
%\begin{equation*}
%L_1 > L_2 \Rightarrow C(L_1) \geq C(L_2)
%\end{equation*}
%Using task sets that each have WCET values (provided with a certain level
%of assurance), his work extended existing worst case response time analysis
%\cite{Joseph1986} and optimal priority assignment algorithms \cite{Audsley1991}
%to improve precise schedulability analysis and provide more efficient preemptive
%fixed priority scheduling for mixed criticality task sets.
% 
%A recent review on mixed-criticality systems research can be found in \cite{BurnsDavisMCReview}. 

%identifies results in the trade-off between isolation and resource sharing.
%This trade-off analysis shows the integration difficulty and
%mismatch between scheduling theory and implementation of
%certified systems.

% Csanad: I keep the next paragraph in the file but comment it out.
% The approach I'm taking is to describe important features of the partitioned systems
% thus giving a global picture and refer to applications rather than describing each application.
%
%Partitioned operating systems provide applications  shared access to critical resources while ensuring temporal and spatial isolation. For the DECOS project \cite{DECOS}, two partitioned operating systems were developed. The DECOS Core OS uses a two-level hierarchical scheduling scheme with the top scheduling table dividing the CPU time into partitions and the bottom table triggering events within each partition. These tables are generated at system configuration. The other DECOS partitioned OS is based on a real-time Linux variant RTAI and takes advantage of the Linux Real-Time extension (LXRT) for spatial and temporal partitioning.  It uses a time-triggered dispatcher to activate the partitions. The dispatching points are obtained from a static scheduling table created at system configuration. For the automotive industry, the AUTOSAR standard \cite{autosar} provides an OS with statically configured priority-based scheduling. It uses a fixed-priority preemptive scheduling strategy where the scheduling is event-triggered and an event with higher priority always gets CPU time. Since this scheme prevents temporal partitioning, AUTOSAR provides schedule tables to specify periodic activation of events and time monitoring to limit execution times and arrival rates for tasks. To address the safety-critical needs of avionic applications, the LynxOS-178 OS \cite{lynxos-178} defines services for an avionic software environment and adheres to the ARINC-653 standard \cite{ARINC-653} while providing a fixed, cyclic, priority-based preemptive scheduling policy. A detailed comparison of the features in the above mentioned partitioned operating systems can be seen in \cite{PartitioningOS}.

Partitioning operating systems have been applied to avionics (e.g.,
LynxOS-178~\cite{lynxos-178}), automotive (e.g., Tresos, the
operating system defined in AUTOSAR~\cite{autosar}), and
cross-industry domains (DECOS OS~\cite{DECOS}). A comparison of the
mentioned partitioning operating systems can be found in
\cite{PartitioningOS}. They  provide applications shared access to
critical system resources on an integrated computing platform.
Applications may belong to different security domains and can have
different safety-critical impact on the system. To avoid unwanted 
interference between the applications, reliable protection is
guaranteed in both the spatial and the temporal domain that is
achieved by using partitions on the system level. Spatial
partitioning ensures that an application cannot access another 
application's code or data in memory or on disk. Temporal
partitioning guarantees an application access to the critical system
(CPU) resources via dedicated time intervals regardless of other
applications. 

%Partitioning operating systems 

Our approach combines mixed-criticality and partitioning techniques 
to meet the requirements of secure DRE systems.
\iap\ supports multiple levels
of criticality, with tasks being assigned to a single criticality
level. For security and fault isolation reasons, applications are
strictly separated by means of spatial and temporal partitioning, and
applications are required to use a novel secure communication method
for all communications, described in \iap\ \cite{ISIS_F6_Aerospace:12}.

Our work has many similarities with the resource-centric real-time
kernel~\cite{DistributedRK-Rajkumar08} to support real-time
requirements of distributed systems hosting multiple applications. 
Though achieved differently, both frameworks use deployment services
for the automatic deployment of distributed applications, and enforcing 
resource isolation among applications. However, to the best of our knowledge,
\cite{DistributedRK-Rajkumar08} does not include support for process
management, temporal isolation guarantees, partition management, and
secure communication simultaneously. 

% Abhishek, should we say a bit about our SRDS paper from last year and
% describe differences? - Andy
