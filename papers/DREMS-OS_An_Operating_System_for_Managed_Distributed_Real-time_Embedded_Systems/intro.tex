% $Id: intro.tex 3896 2013-10-22 04:19:59Z dabhishe $
\vspace{-0.1in}
\section{Introduction}
\label{sec:intro}

The emerging realm of mobile and embedded cloud computing, which
leverages the progress made in computing and communication
in mobile devices and sensors necessitates a platform for running distributed, 
real-time, and embedded (DRE) systems.  
\iffalse
For example, an ad-hoc cloud of smart phones can share
sensing and computing resources with nearby devices to provide increased
situational awareness in disaster relief efforts.  
\fi
Ensembles of mobile
devices are being used as a computing resource in space missions as
well: satellite clusters provide a dynamic environment for deploying and 
managing distributed mission applications; see, \emph{e.g.} NASA's Edison Demonstration of SmallSat Networks,
TanDEM-X, PROBA-3, and Prisma from ESA, and DARPA's System F6.

\iffalse
DRE systems deployed in these shared environments often require the
presence of one or more mixed criticality applications, potentially
sourced from various vendors, to handle multiple complex activities.  
These applications, containing multiple processes
of different criticality, require strict isolation with respect to resource
guarantees, faults, and security.  Such requirements imply
that an application's performance, faults, or life cycle changes
should not impact in any way an application in another isolation
group. This isolation requirement adds the complexity of application
management to these DRE systems.
\fi

\iffalse
Such strict
partitioning, though useful in isolation, is often inefficient because
tasks are guaranteed processor time even when they do not need it.
Furthermore, strict partitioning cannot execute sporadic critical tasks
% Abhishek, I tied this to our motivating example. We need to do this
% wherever there is a need - Andy
that must execute immediately.  
Additionally, different sets of applications can be configured to be scheduled at different phases of the mission.
 These events call for dynamic changes to the
scheduling of DRE system tasks.
% Abhishek, I added the above sentence and then made the following sound
% like a key limitation - Andy
\fi

As an example consider a cluster of satellites that execute
software applications distributed across the satellites.  One application
is a safety-critical cluster flight application (CFA) that controls the
satellite's flight and is required to respond to emergency 'scatter' commands.  
Running concurrently with
the CFA, image processing applications (IPA) utilize the satellites' sensors
and consume much of the CPU resources.  IPAs from different vendors may have different
security privileges and so may have controlled access to sensor data. 
Sensitive camera data must be compartmentalized and must not be shared between these 
IPAs, unless explicitly permitted. These applications must also be isolated from each other to prevent 
performance impact or fault propagation between applications due to lifecycle changes.  However, the isolation should not waste CPU resources when applications are dormant  
because, for example, a sensor is active only in certain segments of the satellite's orbit. 
Other  applications should be able to opportunistically use the CPU during these dormant phases.
%Management of these applications entails providing for secure information flows, application lifecycle management, 
%fault isolation, resource partitioning, and temporal partition reconfiguration.
 %require management tasks to properly restrict the interactions of applications. 

\iffalse
These
applications must be protected and isolated from each other, but the
highly constrained resources of the cluster should not be idly wasted.
Additionally, the applications should be allowed to communicate only
on authorized channels so that they cannot be used for receive and
send unauthorized information. This is one of a key requirement if the
platform has to be used in any critical mission. Moreover, the cluster
must be  managed by a privileged application which
can monitor and control each  satellite's hardware while communicating
with a ground control station.  This cluster management application must
enable rapid cluster response to incoming commands from the ground
station, \emph{e.g.} a scatter command to avoid orbital debris, but it
should not waste CPU cycles when idle.  Since the response time of the
cluster management application is of utmost importance, cluster
management should be able to run as soon as possible. Additionally, the
applications can be grouped into different sets which are active during
different phases of the mission. The platform should be able to provide
the ability to reconfigure the schedule as required by the different
phase of the mission. 
\fi

One technique for implementing strict application isolation  is temporal and spatial
partitioning of processes (see \cite{ARINC-653}).  Spatial separation provides a
hardware-supported, physically separated memory address space for each process.  Temporal partitioning
provides a periodically repeating fixed interval of CPU time that is
exclusively assigned to a group of cooperating tasks. 
Note that strictly partitioned systems are typically configured with a
static schedule; any change in the schedule requires the system to be
rebooted~\cite{ARINC-653}.   
%For example, a satellite sensor application may be active only in certain intervals of the orbit. 

\iffalse
The scheduler should allow such changes so that 
dormant applications do
% Abhishek, what does dormant in this context mean? Idle? And I am not
% sure I understand the flow of thought from the first sentence of this
% para that tells the limitation of partitioned systems. Is there a way
% to tie it to the example? For example, is a cluster mgmt task a
% dormant task that should not waste cycles, however, when it is
% released in the system, it must get its share? Should we say this? I
% think it will help make the case stronger - Andy
not consume CPU time and other applications can be guaranteed time in
those intervals.  Finally, it can be advantageous to ensure that lower
priority application tasks in a temporal partition will not be starved
of the CPU by a higher priority application task which shares the
partition.  The scheduler should provide the ability to ensure that such
starvation can be avoided. 
\fi

\iffalse
% Abhishek, team: We need a para here to motivate the secure comm
% otherwise only the scheduler gets motivated - Andy
Since multiple applications in our DRE system can be
scheduled in the same temporal partition but be sourced
from different vendors, security and strict isolation must be
maintained among such co-existing applications. % of the DRE systems.
% folks, is there a way to tie an example scenario from our motivating
% example? 
Consequently, while supporting all the scheduling requirements, the DRE
system must simultaneously support the desired security requirements,
minimizing covert and unauthorized communication channels between applications.
\fi

%  We have developed an architecture called
% \iapfull\ (\iap)~\cite{ISIS_F6_Aerospace:12}  that addresses these
% requirements. \iap\ addresses a class of DRE systems that require active, remotely conducted \textit{management} 
% of the software platform and the applications running on that platform. Such managed DREs arise in 
% application domains where a dominant entity controls the complete software configuration
% and all operational aspects of a large number of computing nodes, 
% shared by many distributed applications. 
% The architecture consists of (1) a design-time tool suite for modeling,
% analysis, synthesis, integration, debugging, testing, and maintenance
% of application software built from reusable components, and (2) a run-time \textit{software platform} 
% for deploying, executing, and managing application software on a network
% of mobile nodes.  The run-time software platform consists of
% an operating system kernel, system services, and middleware libraries.
% In prior work, we have described the general architecture of
% \iap~\cite{ISIS_F6_Aerospace:12}, its design-time modeling
% capability~\cite{ISIS_F6_SFFMT:13}, and its component model used to
% build applications~\cite{ISIS_F6_ISORC:13}.
% but not addressed the
%challenges described in this research.

To address these needs, we have developed an architecture called
 \iapfull\ (\iap)~\cite{6899124}.
This paper focuses on the design and implementation of key components of the operating system layer in \iap.
 It describes the design choices and algorithms used in the design of the \iap\ OS scheduler.  The scheduler supports three criticality levels: critical, application, and best effort.  It  supports temporal and spatial partitioning for application-level
  tasks.   Tasks in a partition are scheduled in a work-conserving   manner.  Through a CPU cap mechanism, it also ensures that no task  starves for the CPU. Furthermore, it allows dynamic reconfiguration of the temporal partitions. We  empirically validated the design in the context of a case study: a managed DRE system running on a laboratory testbed.

The outline of this paper is as follows: Section~\ref{sec:related}
presents the related research; 
\iffalse
Section~\ref{sec:task_model} provides
background information and the system model;
Section~\ref{sec:scheduler} onwards the scheduler design is presented;
\fi
Section~\ref{sec:task_model} describes the system model and delves
into the details of the scheduler design;
Section~\ref{sec:experiment} empirically evaluates \iap\ OS in the
context of a representative space application; and finally
Section~\ref{sec:conclusions} offers concluding remarks referring to
future work.  
