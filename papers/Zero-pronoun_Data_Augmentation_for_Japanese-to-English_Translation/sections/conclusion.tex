\section{Conclusion}
To address the problem of zero pronoun translation, we proposed zero pronoun data augmentation.
Through the analysis with the Japanese-English conversational parallel corpus, we showed that zero pronouns in Japanese sentences can be predicted to some extent from local context within the sentence.
In the conversational translation experiment, we compared a translation model trained on the augmented data with the baseline and demonstrate that our method significantly improves the accuracy of zero pronoun translation.

Nevertheless, zero pronoun data augmentation does not solve the cases where the information necessary for zero pronoun translation exists outside the sentence.  Also, the analysis suggests that local context is useful for predicting frequent pronouns such as the first and second-person pronouns, but not for the third-person pronouns.
An interesting avenue for future work is to explicitly incorporate discourse-level contextual information such as topics or people involved in the conversation into the translation models.
