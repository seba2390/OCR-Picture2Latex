\section{Is Local Context Useful for Predicting Zero Pronouns?}
\label{section:analysis}

\begin{table*}[t]
\centering
\begin{tabular}{ccccccccccc}\toprule
                    & I    & you    & we    & they & he  & she & us  & them & him & her \\ \midrule
baseline            & 35.9 & 25.4   & 11.0  & 3.7  & 2.2 & 0.0 & 2.2 & 1.9  & 1.2 & 0.9 \\
logistic regression & 78.2 & 46.3   & 17.3  & 3.8  & 3.1 & 0.0 & 3.6 & 0.2  & 0.2 & 2.9 \\ \bottomrule
\end{tabular}
\caption{Recall scores of ZP predictions for each pronoun.}
\label{result:logistic}
\end{table*}

Our proposed method is based on the assumption that local context in Japanese sentences is useful for predicting ZPs.
We begin by analyzing to what extent ZPs can be inferred from local context, and what kind of local context is useful.

For the analysis, we use the Business Scene Dialogue Corpus \citep{rikters-etal-2019-designing}, which is a Japanese and English parallel corpus in the conversational domain.
Besides the published data, we also use the in-house version of the corpus, which amounts to a total of 104,961 sentence pairs.

\subsection{Identifying sentence pairs that contain ZPs.}
As the corpus does not contain annotations of ZPs, we first identify sentence pairs that contain zero pronouns.
We exploit the word alignment information from parallel sentences to detect ZPs. The specific procedure is as follows.

\begin{enumerate}
  \item We obtain the word alignments of the parallel data with \texttt{GIZA++}\footnote{\url{https://github.com/moses-smt/giza-pp}}. We use \texttt{Mecab}\footnote{\url{https://taku910.github.io/mecab/}} for Japanese word segmentation, \texttt{spaCy}\footnote{\url{https://spacy.io/}} for English.
  \item When a pronoun in an English sentence is associated with \texttt{NULL}, the pronoun in the English sentence is considered to correspond to a ZP in the Japanese sentence.
\end{enumerate}

The resulting number of pronouns is shown in Figure \ref{fig:pronoun_count}. It can be seen that in the conversational domain, the first person pronoun \textit{I} and the second person pronoun \textit{you} occur frequently and most of them ($80\%\sim$) are omitted in Japanese. More infrequent pronouns are less likely to be ZPs.

\begin{figure}[t]
\centering
\includegraphics[width=7.0cm]{data/pronoun_count}
\caption{The number of English pronouns in the analyzed data. ZP stands for those whose corresponding pronoun does not appear in the Japanese text.}
\label{fig:pronoun_count}
\end{figure}

\subsection{Extracting local context that co-occurs with ZPs}
To associate the detected ZPs with local context in Japanese sentences, we extract the words that appear in their predicates.
We did not use a Japanese syntactic analyzer to detect ZPs but they are associated with the English pronouns by alignment. Therefore, we decided to exploit the alignment information to extract the predicates.
We extract the predicates of the English pronoun and the corresponding words in the Japanese sentence. Specifically, the following steps were taken.

\begin{enumerate}
  \setcounter{enumi}{2}
  \item We obtain the dependency tree of the English sentence with \texttt{spaCy} and extract the pronoun's head.
  \item The Japanese word aligned to the pronoun's head and its subsequent functional words \footnote{In this case, the function words are defined as words with one of the following part of speeches defined in \texttt{Mecab}: [``particle", ``auxiliary verb", ``symbol"].} are extracted as local context.
\end{enumerate}

\begin{table*}[t]
\centering
\begin{tabular}{cccc} \toprule
                  & 1to1        & 2to1        \\ \midrule
baseline          & 17.07$\pm$0.16 / 83.6$\pm$1.1 & 17.07$\pm$0.26 / 89.36$\pm$0.9 \\
baseline+pro\_aug & 17.07$\pm$0.19 / 92.32$\pm$1.8 & 17.11$\pm$0.23 / 92.17$\pm$1.1 \\ \bottomrule
\end{tabular}
\caption{Evaluation of the model with ZP data augmentation. The scores on the table are BLEU / ZP evaluation accuracy. The mean and standard deviation of five runs with different random seeds are reported.}
\label{result:mt}
\end{table*}



\subsection{Predicting ZPs from Local Context}
To investigate the extent to which ZPs can be predicted from local context, we conducted an analysis by training a logistic regression classifier \footnote{We use the implementation of the \texttt{scikit-learn} library with the default hyperparameters.}.
The classifier takes the unigrams, bi-grams, and trigrams extracted from local context in the Japanese sentence and predicts the associated pronoun in the English sentence.

The recall scores of each pronoun obtained with five-fold cross-validation are shown in Table \ref{result:logistic}. As a baseline, we adopt the score of random prediction according to the training distribution of pronouns.

One can see that the frequent pronouns such as \textit{I, you, we} can be predicted with significantly higher accuracy than the baseline when local context is used (around 6 to 43 points of improvement). In contrast, the other infrequent pronouns display similar or lower values compared to the baseline.
In summary, we can see that local context is predictive of the frequent pronouns but not for the infrequent ones.

To investigate what kind of local context is useful for prediction, for each output label ({\it i.e.}, pronoun) of the logistic regression classifier, we extracted the input features with higher values in the corresponding weights.
As a result, the following words are interpreted to be relevant.

\minisection{The first person singular \textit{I}}
verbs related to recognition (\ja{思う} (think), \ja{わかる} (understand), \ja{感じる} (feel)); humble words (\ja{申し上げる、存る}); and auxiliary verbs expressing desire (\ja{たい}).

\minisection{The second person singular \textit{you}}
suffixes expressing questions (\ja{かな?}, \ja{ました?}); speculations (\ja{でしょ}, \ja{だろ?}), honorifics (\ja{仰る}, \ja{いただける}).

\minisection{The first person plural \textit{we}}
obligations (\ja{なきゃ}, \ja{べき}), desire (\ja{たい}).

For the other pronouns, no local contexts were found to be interpretable as useful for prediction.
