%% bare_jrnl.tex
%% V1.4b
%% 2015/08/26
%% by Michael Shell
%% see http://www.michaelshell.org/
%% for current contact information.
%%
%% This is a skeleton file demonstrating the use of IEEEtran.cls
%% (requires IEEEtran.cls version 1.8b or later) with an IEEE
%% journal paper.
%%
%% Support sites:
%% http://www.michaelshell.org/tex/ieeetran/
%% http://www.ctan.org/pkg/ieeetran
%% and
%% http://www.ieee.org/

%%*************************************************************************
%% Legal Notice:
%% This code is offered as-is without any warranty either expressed or
%% implied; without even the implied warranty of MERCHANTABILITY or
%% FITNESS FOR A PARTICULAR PURPOSE! 
%% User assumes all risk.
%% In no event shall the IEEE or any contributor to this code be liable for
%% any damages or losses, including, but not limited to, incidental,
%% consequential, or any other damages, resulting from the use or misuse
%% of any information contained here.
%%
%% All comments are the opinions of their respective authors and are not
%% necessarily endorsed by the IEEE.
%%
%% This work is distributed under the LaTeX Project Public License (LPPL)
%% ( http://www.latex-project.org/ ) version 1.3, and may be freely used,
%% distributed and modified. A copy of the LPPL, version 1.3, is included
%% in the base LaTeX documentation of all distributions of LaTeX released
%% 2003/12/01 or later.
%% Retain all contribution notices and credits.
%% ** Modified files should be clearly indicated as such, including  **
%% ** renaming them and changing author support contact information. **
%%*************************************************************************


% *** Authors should verify (and, if needed, correct) their LaTeX system  ***
% *** with the testflow diagnostic prior to trusting their LaTeX platform ***
% *** with production work. The IEEE's font choices and paper sizes can   ***
% *** trigger bugs that do not appear when using other class files.       ***                          ***
% The testflow support page is at:
% http://www.michaelshell.org/tex/testflow/

\documentclass[journal]{IEEEtran}
%
% If IEEEtran.cls has not been installed into the LaTeX system files,
% manually specify the path to it like:
% \documentclass[journal]{../sty/IEEEtran}


% Some very useful LaTeX packages include:
% (uncomment the ones you want to load)


% *** MISC UTILITY PACKAGES ***
%
\usepackage{ifpdf}
% Heiko Oberdiek's ifpdf.sty is very useful if you need conditional
% compilation based on whether the output is pdf or dvi.
% usage:
% \ifpdf
%   % pdf code
% \else
%   % dvi code
% \fi
% The latest version of ifpdf.sty can be obtained from:
% http://www.ctan.org/pkg/ifpdf
% Also, note that IEEEtran.cls V1.7 and later provides a builtin
% \ifCLASSINFOpdf conditional that works the same way.
% When switching from latex to pdflatex and vice-versa, the compiler may
% have to be run twice to clear warning/error messages.


% *** CITATION PACKAGES ***
%
\usepackage{cite}
% cite.sty was written by Donald Arseneau
% V1.6 and later of IEEEtran pre-defines the format of the cite.sty package
% \cite{} output to follow that of the IEEE. Loading the cite package will
% result in citation numbers being automatically sorted and properly
% "compressed/ranged". e.g., [1], [9], [2], [7], [5], [6] without using
% cite.sty will become [1], [2], [5]--[7], [9] using cite.sty. cite.sty's
% \cite will automatically add leading space, if needed. Use cite.sty's
% noadjust option (cite.sty V3.8 and later) if you want to turn this off
% such as if a citation ever needs to be enclosed in parenthesis.
% cite.sty is already installed on most LaTeX systems. Be sure and use
% version 5.0 (2009-03-20) and later if using hyperref.sty.
% The latest version can be obtained at:
% http://www.ctan.org/pkg/cite
% The documentation is contained in the cite.sty file itself.


% *** GRAPHICS RELATED PACKAGES ***
%
\ifCLASSINFOpdf
   \usepackage[pdftex]{graphicx}
  % declare the path(s) where your graphic files are
  % \graphicspath{{../pdf/}{../jpeg/}}
  % and their extensions so you won't have to specify these with
  % every instance of \includegraphics
  % \DeclareGraphicsExtensions{.pdf,.jpeg,.png}
\else
  % or other class option (dvipsone, dvipdf, if not using dvips). graphicx
  % will default to the driver specified in the system graphics.cfg if no
  % driver is specified.
   \usepackage[dvips]{graphicx}
  % declare the path(s) where your graphic files are
  % \graphicspath{{../eps/}}
  % and their extensions so you won't have to specify these with
  % every instance of \includegraphics
  % \DeclareGraphicsExtensions{.eps}
\fi
% graphicx was written by David Carlisle and Sebastian Rahtz. It is
% required if you want graphics, photos, etc. graphicx.sty is already
% installed on most LaTeX systems. The latest version and documentation
% can be obtained at: 
% http://www.ctan.org/pkg/graphicx
% Another good source of documentation is "Using Imported Graphics in
% LaTeX2e" by Keith Reckdahl which can be found at:
% http://www.ctan.org/pkg/epslatex
%
% latex, and pdflatex in dvi mode, support graphics in encapsulated
% postscript (.eps) format. pdflatex in pdf mode supports graphics
% in .pdf, .jpeg, .png and .mps (metapost) formats. Users should ensure
% that all non-photo figures use a vector format (.eps, .pdf, .mps) and
% not a bitmapped formats (.jpeg, .png). The IEEE frowns on bitmapped formats
% which can result in "jaggedy"/blurry rendering of lines and letters as
% well as large increases in file sizes.
%
% You can find documentation about the pdfTeX application at:
% http://www.tug.org/applications/pdftex

% *** MATH PACKAGES ***
%
\usepackage{amsmath}
% A popular package from the American Mathematical Society that provides
% many useful and powerful commands for dealing with mathematics.
%
% Note that the amsmath package sets \interdisplaylinepenalty to 10000
% thus preventing page breaks from occurring within multiline equations. Use:
%\interdisplaylinepenalty=2500
% after loading amsmath to restore such page breaks as IEEEtran.cls normally
% does. amsmath.sty is already installed on most LaTeX systems. The latest
% version and documentation can be obtained at:
% http://www.ctan.org/pkg/amsmath


% *** SPECIALIZED LIST PACKAGES ***
%
\usepackage{algorithmic}
% algorithmic.sty was written by Peter Williams and Rogerio Brito.
% This package provides an algorithmic environment fo describing algorithms.
% You can use the algorithmic environment in-text or within a figure
% environment to provide for a floating algorithm. Do NOT use the algorithm
% floating environment provided by algorithm.sty (by the same authors) or
% algorithm2e.sty (by Christophe Fiorio) as the IEEE does not use dedicated
% algorithm float types and packages that provide these will not provide
% correct IEEE style captions. The latest version and documentation of
% algorithmic.sty can be obtained at:
% http://www.ctan.org/pkg/algorithms
% Also of interest may be the (relatively newer and more customizable)
% algorithmicx.sty package by Szasz Janos:
% http://www.ctan.org/pkg/algorithmicx


% *** ALIGNMENT PACKAGES ***
%
\usepackage{array}
% Frank Mittelbach's and David Carlisle's array.sty patches and improves
% the standard LaTeX2e array and tabular environments to provide better
% appearance and additional user controls. As the default LaTeX2e table
% generation code is lacking to the point of almost being broken with
% respect to the quality of the end results, all users are strongly
% advised to use an enhanced (at the very least that provided by array.sty)
% set of table tools. array.sty is already installed on most systems. The
% latest version and documentation can be obtained at:
% http://www.ctan.org/pkg/array


% IEEEtran contains the IEEEeqnarray family of commands that can be used to
% generate multiline equations as well as matrices, tables, etc., of high
% quality.




% *** SUBFIGURE PACKAGES ***
\ifCLASSOPTIONcompsoc
 \usepackage[caption=false,font=normalsize,labelfont=sf,textfont=sf]{subfig}
\else
 \usepackage[caption=false,font=footnotesize]{subfig}
\fi
% subfig.sty, written by Steven Douglas Cochran, is the modern replacement
% for subfigure.sty, the latter of which is no longer maintained and is
% incompatible with some LaTeX packages including fixltx2e. However,
% subfig.sty requires and automatically loads Axel Sommerfeldt's caption.sty
% which will override IEEEtran.cls' handling of captions and this will result
% in non-IEEE style figure/table captions. To prevent this problem, be sure
% and invoke subfig.sty's "caption=false" package option (available since
% subfig.sty version 1.3, 2005/06/28) as this is will preserve IEEEtran.cls
% handling of captions.
% Note that the Computer Society format requires a larger sans serif font
% than the serif footnote size font used in traditional IEEE formatting
% and thus the need to invoke different subfig.sty package options depending
% on whether compsoc mode has been enabled.
%
% The latest version and documentation of subfig.sty can be obtained at:
% http://www.ctan.org/pkg/subfig




% *** FLOAT PACKAGES ***
%
% \usepackage{fixltx2e}
% fixltx2e, the successor to the earlier fix2col.sty, was written by
% Frank Mittelbach and David Carlisle. This package corrects a few problems
% in the LaTeX2e kernel, the most notable of which is that in current
% LaTeX2e releases, the ordering of single and double column floats is not
% guaranteed to be preserved. Thus, an unpatched LaTeX2e can allow a
% single column figure to be placed prior to an earlier double column
% figure.
% Be aware that LaTeX2e kernels dated 2015 and later have fixltx2e.sty's
% corrections already built into the system in which case a warning will
% be issued if an attempt is made to load fixltx2e.sty as it is no longer
% needed.
% The latest version and documentation can be found at:
% http://www.ctan.org/pkg/fixltx2e


\usepackage{stfloats}
% stfloats.sty was written by Sigitas Tolusis. This package gives LaTeX2e
% the ability to do double column floats at the bottom of the page as well
% as the top. (e.g., "\begin{figure*}[!b]" is not normally possible in
% LaTeX2e). It also provides a command:
%\fnbelowfloat
% to enable the placement of footnotes below bottom floats (the standard
% LaTeX2e kernel puts them above bottom floats). This is an invasive package
% which rewrites many portions of the LaTeX2e float routines. It may not work
% with other packages that modify the LaTeX2e float routines. The latest
% version and documentation can be obtained at:
% http://www.ctan.org/pkg/stfloats
% Do not use the stfloats baselinefloat ability as the IEEE does not allow
% \baselineskip to stretch. Authors submitting work to the IEEE should note
% that the IEEE rarely uses double column equations and that authors should try
% to avoid such use. Do not be tempted to use the cuted.sty or midfloat.sty
% packages (also by Sigitas Tolusis) as the IEEE does not format its papers in
% such ways.
% Do not attempt to use stfloats with fixltx2e as they are incompatible.
% Instead, use Morten Hogholm'a dblfloatfix which combines the features
% of both fixltx2e and stfloats:
%
% \usepackage{dblfloatfix}
% The latest version can be found at:
% http://www.ctan.org/pkg/dblfloatfix




% \ifCLASSOPTIONcaptionsoff
%  \usepackage[nomarkers]{endfloat}
% \let\MYoriglatexcaption\caption
% \renewcommand{\caption}[2][\relax]{\MYoriglatexcaption[#2]{#2}}
% \fi
% endfloat.sty was written by James Darrell McCauley, Jeff Goldberg and 
% Axel Sommerfeldt. This package may be useful when used in conjunction with 
% IEEEtran.cls'  captionsoff option. Some IEEE journals/societies require that
% submissions have lists of figures/tables at the end of the paper and that
% figures/tables without any captions are placed on a page by themselves at
% the end of the document. If needed, the draftcls IEEEtran class option or
% \CLASSINPUTbaselinestretch interface can be used to increase the line
% spacing as well. Be sure and use the nomarkers option of endfloat to
% prevent endfloat from "marking" where the figures would have been placed
% in the text. The two hack lines of code above are a slight modification of
% that suggested by in the endfloat docs (section 8.4.1) to ensure that
% the full captions always appear in the list of figures/tables - even if
% the user used the short optional argument of \caption[]{}.
% IEEE papers do not typically make use of \caption[]'s optional argument,
% so this should not be an issue. A similar trick can be used to disable
% captions of packages such as subfig.sty that lack options to turn off
% the subcaptions:
% For subfig.sty:
% \let\MYorigsubfloat\subfloat
% \renewcommand{\subfloat}[2][\relax]{\MYorigsubfloat[]{#2}}
% However, the above trick will not work if both optional arguments of
% the \subfloat command are used. Furthermore, there needs to be a
% description of each subfigure *somewhere* and endfloat does not add
% subfigure captions to its list of figures. Thus, the best approach is to
% avoid the use of subfigure captions (many IEEE journals avoid them anyway)
% and instead reference/explain all the subfigures within the main caption.
% The latest version of endfloat.sty and its documentation can obtained at:
% http://www.ctan.org/pkg/endfloat
%
% The IEEEtran \ifCLASSOPTIONcaptionsoff conditional can also be used
% later in the document, say, to conditionally put the References on a 
% page by themselves.




% *** PDF, URL AND HYPERLINK PACKAGES ***
%
\usepackage{url}
% url.sty was written by Donald Arseneau. It provides better support for
% handling and breaking URLs. url.sty is already installed on most LaTeX
% systems. The latest version and documentation can be obtained at:
% http://www.ctan.org/pkg/url
% Basically, \url{my_url_here}.




% *** Do not adjust lengths that control margins, column widths, etc. ***
% *** Do not use packages that alter fonts (such as pslatex).         ***
% There should be no need to do such things with IEEEtran.cls V1.6 and later.
% (Unless specifically asked to do so by the journal or conference you plan
% to submit to, of course. )

\usepackage{amssymb}
\usepackage{array}
\usepackage{tabularx}
\usepackage{multirow,bigstrut}
\usepackage{booktabs}
\usepackage{mathtools}
\usepackage{algorithm}
\usepackage{listings}
\usepackage{xcolor}
\usepackage{amsthm}
\usepackage{textcomp}
\usepackage{multirow}

\theoremstyle{definition}
\newtheorem{derivation}{Derivation}
\newtheorem{theorem}{Theorem}

\def\eg{\emph{e.g}.}
\def\Eg{\emph{E.g}.}
\newcommand{\mcomma}{,\allowbreak}
\newcommand{\etal}{\textit{ et al. }}
\newcommand{\ie}{\textit{i.e. }}
\newcommand{\mI}{\mathbf{I}}
\newcommand{\mone}{\mathbf{1}}
\newcommand{\viz}{\textit{viz. }}
\newcommand{\fig}[1]{Fig. \ref{#1}}
\newcommand{\eq}[1]{Eqn. (\ref{#1})}
\newcommand{\sect}[1]{Section \ref{#1}}
\newcommand{\ssect}[1]{SubSection~\ref{#1}}
\newcommand{\sssect}[1]{SubSubSection~\ref{#1}}
\newcommand{\tab}[1]{Table \ref{#1}}
\newcommand{\alg}[1]{Algorithm~\ref{#1}}
\newcommand{\thm}[1]{Theorem~\ref{#1}}
\newcommand{\lin}[1]{Line~\ref{#1}}
\newcommand{\nCr}[2]{\,^{#1}C_{#2}} % nCr
\newcommand{\nPr}[2]{\,^{#1}P_{#2}} % nPr

\DeclareMathOperator*{\argmax}{arg\,max}
\DeclareMathOperator*{\argmin}{arg\,min}

\DeclareGraphicsExtensions{.pdf}
\graphicspath{{./figures/}}

\makeatletter
\newcommand{\pushright}[1]{\ifmeasuring@#1\else\omit\hfill$\displaystyle#1$\fi\ignorespaces}
\newcommand{\pushleft}[1]{\ifmeasuring@#1\else\omit$\displaystyle#1$\hfill\fi\ignorespaces}
\makeatother

% *** Do not adjust lengths that control margins, column widths, etc. ***
% *** Do not use packages that alter fonts (such as pslatex).         ***
% There should be no need to do such things with IEEEtran.cls V1.6 and later.
% (Unless specifically asked to do so by the journal or conference you plan
% to submit to, of course. )


% correct bad hyphenation here
\hyphenation{op-tical net-works semi-conduc-tor}


\begin{document}
%
% paper title
% can use linebreaks \\ within to get better formatting as desired
\title{Supplemental Material: Stochastic Graphlet Embedding}
%
%
% author names and IEEE memberships
% note positions of commas and nonbreaking spaces ( ~ ) LaTeX will not break
% a structure at a ~ so this keeps an author's name from being broken across
% two lines.
% use \thanks{} to gain access to the first footnote area
% a separate \thanks must be used for each paragraph as LaTeX2e's \thanks
% was not built to handle multiple paragraphs
%
%
%\IEEEcompsocitemizethanks is a special \thanks that produces the bulleted
% lists the Computer Society journals use for "first footnote" author
% affiliations. Use \IEEEcompsocthanksitem which works much like \item
% for each affiliation group. When not in compsoc mode,
% \IEEEcompsocitemizethanks becomes like \thanks and
% \IEEEcompsocthanksitem becomes a line break with idention. This
% facilitates dual compilation, although admittedly the differences in the
% desired content of \author between the different types of papers makes a
% one-size-fits-all approach a daunting prospect. For instance, compsoc 
% journal papers have the author affiliations above the "Manuscript
% received ..."  text while in non-compsoc journals this is reversed. Sigh.

\author{Anjan Dutta,~\IEEEmembership{Member,~IEEE,} 
        and~Hichem Sahbi,~\IEEEmembership{Member,~IEEE,}
%        Josep Llad\'{o}s,~\IEEEmembership{Member,~IEEE,}
%        Horst Bunke,~\IEEEmembership{Member,~IEEE,}
%        and~Umapada Pal,~\IEEEmembership{Member,~IEEE}% <-this % stops a space
% note need leading \protect in front of \\ to get a newline within \thanks as
% \\ is fragile and will error, could use \hfil\break instead.

%\IEEEcompsocthanksitem J. Doe and J. Doe are with Anonymous University.}% <-this % stops a space
}

% note the % following the last \IEEEmembership and also \thanks - 
% these prevent an unwanted space from occurring between the last author name
% and the end of the author line. i.e., if you had this:
% 
% \author{....lastname \thanks{...} \thanks{...} }
%                     ^------------^------------^----Do not want these spaces!
%
% a space would be appended to the last name and could cause every name on that
% line to be shifted left slightly. This is one of those "LaTeX things". For
% instance, "\textbf{A} \textbf{B}" will typeset as "A B" not "AB". To get
% "AB" then you have to do: "\textbf{A}\textbf{B}"
% \thanks is no different in this regard, so shield the last } of each \thanks
% that ends a line with a % and do not let a space in before the next \thanks.
% Spaces after \IEEEmembership other than the last one are OK (and needed) as
% you are supposed to have spaces between the names. For what it is worth,
% this is a minor point ahttp://www.cvc.uab.es/~adutta/Research/ProductGraphs most people would not even notice if the said evil
% space somehow managed to creep in.

% The paper headers
% \markboth{IEEE Transactions on Cybernetics}%
% {Dutta \MakeLowercase{\etal}: IEEE Transactions on Cybernetics}
% The only time the second header will appear is for the odd numbered pages
% after the title page when using the twoside option.
% 
% *** Note that you probably will NOT want to include the author's ***
% *** name in the headers of peer review papers.                   ***
% You can use \ifCLASSOPTIONpeerreview for conditional compilation here if
% you desire.

% The publisher's ID mark at the bottom of the page is less important with
% Computer Society journal papers as those publications place the marks
% outside of the main text columns and, therefore, unlike regular IEEE
% journals, the available text space is not reduced by their presence.
% If you want to put a publisher's ID mark on the page you can do it like
% this:
%\IEEEpubid{0000--0000/00\$00.00~\copyright~2007 IEEE}
% or like this to get the Computer Society new two part style.
%\IEEEpubid{\makebox[\columnwidth]{\hfill 0000--0000/00/\$00.00~\copyright~2007 IEEE}%
%\hspace{\columnsep}\makebox[\columnwidth]{Published by the IEEE Computer Society\hfill}}
% Remember, if you use this you must call \IEEEpubidadjcol in the second
% column for its text to clear the IEEEpubid mark (Computer Society jorunal
% papers don't need this extra clearance.)

% for Computer Society papers, we must declare the abstract and index terms
% PRIOR to the title within the \IEEEcompsoctitleabstractindextext IEEEtran
% command as these need to go into the title area created by \maketitle.

% IEEEtran.cls defaults to using nonbold math in the Abstract.
% This preserves the distinction between vectors and scalars. However,
% if the journal you are submitting to favors bold math in the abstract,
% then you can use LaTeX's standard command \boldmath at the very start
% of the abstract to achieve this. Many IEEE journals frown on math
% in the abstract anyway. In particular, the Computer Society does
% not want either math or citations to appear in the abstract.

% Note that keywords are not normally used for peer review papers.

% make the title area
\maketitle

% To allow for easy dual compilation without having to reenter the
% abstract/keywords data, the \IEEEcompsoctitleabstractindextext text will
% not be used in maketitle, but will appear (i.e., to be "transported")
% here as \IEEEdisplaynotcompsoctitleabstractindextext when compsoc mode
% is not selected <OR> if conference mode is selected - because compsoc
% conference papers position the abstract like regular (non-compsoc)
% papers do!

% \IEEEdisplaynotcompsoctitleabstractindextext has no effect when using
% compsoc under a non-conference mode.


% For peer review papers, you can put extra information on the cover
% page as needed:
% \ifCLASSOPTIONpeerreview
% \begin{center} \bfseries EDICS Category: 3-BBND \end{center}
% \fi
%
% For peerreview papers, this IEEEtran command inserts a page break and
% creates the second title. It will be ignored for other modes.
%-------------------------------------------------------------------------------------------------------------------------------------------------------
\section{Additional Experimental Results}
This document is the supplemental material of~\cite{Dutta2018SGE}. The main goal of this documentation is to provide additional experimental results or justifications related to the proposed Stochastic Graphlet Embedding (SGE) method presented in~\cite{Dutta2018SGE}.

\subsection{Finding $t$, $\epsilon$ and $\delta$ through $n$-fold cross validation}
In the original manuscript, we have shown results by varying the parameters $t\in\lbrace1,\ldots,7\rbrace$, $\epsilon\in\lbrace0.1, 0.05\rbrace$ and $\delta\in\lbrace0.1, 0.05\rbrace$. However, these parameter values can be selected based on $n$-fold cross validation performed on a validation set. \tab{tab:param-cross-valid} shows selected values of $t$, $\epsilon$ and $\delta$ for different datasets based on $10$-fold cross validation and the corresponding classification accuracy on the test sets. The validation and test sets are obtained by equally dividing the respective test sets used in the original manuscript (Table VI); note that classification accuracies (obtained by selecting the parameters through cross validation) agree with those already shown in the paper.

\begin{table}[!ht]
\begin{center}
\caption{Determination of optimal $t$, $\epsilon$ and $\delta$ for different datasets based on a $10-$fold cross validation on the validation set.}
\label{tab:param-cross-valid}
\begin{tabular}{|l|c|c|c|c|}
\hline
\multirow{2}{*}{Dataset} & \multicolumn{3}{c|}{Optimal parameters} & \multirow{2}{*}{Accuracy on test set} \\
\cline{2-4}
 & $t$ & $\epsilon$ & $\delta$ & \\
\hline
MUTAG   & $6$ & $0.05$ & $0.05$ & $89.71$ \\
PTC     & $4$ & $0.05$ & $0.05$ & $63.68$ \\
ENZYMES & $7$ & $0.05$ & $0.05$ & $40.81$ \\
D \& D  & $7$ & $0.05$ & $0.05$ & $76.15$ \\
NCI1    & $7$ & $0.05$ & $0.10$ & $82.87$ \\
NCI109  & $7$ & $0.05$ & $0.10$ & $82.41$ \\
\hline
\end{tabular}
\end{center}
\end{table}

\begin{figure}
\begin{center}
\includegraphics[width=8.5cm, height=8.5cm]{richness}
\caption{Richness vs. T plot for MUTAG, PTC, ENZYMES, D \& D, NCI1 and NCI109 datasets.}
\label{fig:richness}
\end{center}
\end{figure}

\noindent In order to better understand the behavior of our model w.r.t $t$ and its impact on the classification accuracy (particularly on the PTC dataset), we consider the following "richness" measure for each $t$
\begin{equation*}
\text{Richness}=\frac{\sum_c ||\mu_c - \mu||_2^2}{\sum_c \sum_i \delta_{ic} ||x_i-\mu_c||_2^2/N_c}
\end{equation*}
where $\mu$ is the average of all data, $\mu_c$ is average of data belonging to class $c$, $\delta_{ic}$ is the indicator function denoting whether the $i^\text{th}$ example belongs to the class $c$, and $N_c=\sum_i\delta_{ic}$. This measure basically reflects the separability of various classes. \fig{fig:richness} shows the richness  for different $T$ (upper bound of $t$); this measure reaches its peak when $T=4$ and this is in accordance with the highest discrimination power, of our graphlet-based representation, on the PTC dataset.

\begin{table*}[!t]
\begin{center}
\caption{Classification accuracies (in \%) obtained by SGE on MUTAG, PTC, ENZYMES, D\&D, NCI1 and NCI109 datasets by uniformly sampling graphlets with larger $M$ ($5\times$) and $T$.}
\label{tab:expt-graph-class-large-M-T}
\resizebox{\textwidth}{!}{
\begin{tabular}{|l|c|c|c|c|c|c|}
\hline
Parameters & \ \ \ \ MUTAG \ \ \ \ & \ \ \ \ PTC \ \ \ \ & \ \ \ \ ENZYMES \ \ \ \ & \ \ \ \ D \& D \ \ \ \ & \ \ \ \ NCI1 \ \ \ \ & \ \ \ \ NCI109 \ \ \ \ \\
\hline
$t=3, \epsilon = 0.1, \delta = 0.1$   & $84.44 \pm 0.71$ ($0.37$) & $55.59 \pm 0.10$ ($0.39$) & $27.83 \pm 0.48$ ($0.36$) & $64.23 \pm 0.14$ ($0.35$) & $76.15 \pm 0.57$ ($0.37$) & $74.90 \pm 0.31$ ($0.38$) \\
$t=3, \epsilon = 0.1, \delta = 0.05$  & $84.21 \pm 0.47$ ($0.43$) & $56.18 \pm 0.57$ ($0.44$) & $28.34 \pm 0.67$ ($0.45$) & $63.39 \pm 0.37$ ($0.44$) & $76.15 \pm 0.84$ ($0.44$) & $74.66 \pm 0.59$ ($0.44$) \\
$t=3, \epsilon = 0.05, \delta = 0.1$  & $82.22 \pm 0.17$ ($1.05$) & $57.06 \pm 0.51$ ($1.03$) & $26.00 \pm 0.31$ ($1.04$) & $65.76 \pm 0.47$ ($1.05$) & $75.86 \pm 0.24$ ($1.03$) & $72.56 \pm 0.54$ ($1.04$) \\
$t=3, \epsilon = 0.05, \delta = 0.05$ & $85.56 \pm 0.74$ ($1.34$) & $58.40 \pm 0.67$ ($1.35$) & $29.50 \pm 0.32$ ($1.36$) & $68.90 \pm 0.99$ ($1.35$) & $77.71 \pm 0.24$ ($1.33$) & $76.21 \pm 0.82$ ($1.36$) \\
\hline
$t=4, \epsilon = 0.1, \delta = 0.1$   & $85.00 \pm 0.41$ ($1.67$) & $60.00 \pm 0.27$ ($1.65$) & $29.17 \pm 0.92$ ($1.69$) & $68.64 \pm 0.23$ ($1.71$) & $77.37 \pm 0.61$ ($1.66$) & $76.55 \pm 0.22$ ($1.66$) \\
$t=4, \epsilon = 0.1, \delta = 0.05$  & $84.44 \pm 0.41$ ($1.83$) & $61.18 \pm 0.39$ ($1.81$) & $30.50 \pm 0.26$ ($1.86$) & $68.94 \pm 0.88$ ($1.84$) & $77.37 \pm 0.65$ ($1.84$) & $76.21 \pm 0.41$ ($1.85$) \\
$t=4, \epsilon = 0.05, \delta = 0.1$  & $85.56 \pm 0.31$ ($2.15$) & $62.06 \pm 0.17$ ($2.17$) & $28.67 \pm 0.85$ ($2.18$) & $70.08 \pm 0.59$ ($2.19$) & $76.15 \pm 0.91$ ($2.16$) & $78.48 \pm 0.60$ ($2.18$) \\
$t=4, \epsilon = 0.05, \delta = 0.05$ & $86.67 \pm 0.89$ ($2.27$) & $64.12 \pm 0.23$ ($2.28$) & $31.17 \pm 0.72$ ($2.29$) & $72.63 \pm 0.24$ ($2.26$) & $78.49 \pm 0.67$ ($2.26$) & $78.48 \pm 0.80$ ($2.25$) \\
\hline
$t=5, \epsilon = 0.1, \delta = 0.1$   & $86.67 \pm 0.05$ ($2.59$) & $59.12 \pm 0.26$ ($2.58$) & $32.17 \pm 0.43$ ($2.60$) & $72.54 \pm 0.60$ ($2.61$) & $79.51 \pm 0.49$ ($2.62$) & $78.82 \pm 0.33$ ($2.61$) \\
$t=5, \epsilon = 0.1, \delta = 0.05$  & $87.78 \pm 0.05$ ($2.73$) & $61.18 \pm 0.23$ ($2.75$) & $35.17 \pm 0.46$ ($2.72$) & $72.54 \pm 0.84$ ($2.75$) & $81.70 \pm 0.67$ ($2.75$) & $78.48 \pm 0.23$ ($2.74$) \\
$t=5, \epsilon = 0.05, \delta = 0.1$  & $86.11 \pm 0.52$ ($3.13$) & $63.53 \pm 0.90$ ($3.15$) & $36.67 \pm 0.27$ ($3.14$) & $70.08 \pm 0.22$ ($3.13$) & $81.13 \pm 0.13$ ($3.11$) & $76.55 \pm 0.80$ ($3.12$) \\
$t=5, \epsilon = 0.05, \delta = 0.05$ & $87.22 \pm 0.89$ ($3.31$) & $65.00 \pm 0.79$ ($3.33$) & $37.17 \pm 0.85$ ($3.32$) & $73.05 \pm 0.81$ ($3.33$) & $81.26 \pm 0.29$ ($3.34$) & $81.22 \pm 0.85$ ($3.30$) \\
\hline
$t=6, \epsilon = 0.1, \delta = 0.1$   & $87.78 \pm 0.31$ ($3.72$) & $58.53 \pm 0.06$ ($3.73$) & $36.83 \pm 0.22$ ($3.74$) & $72.80 \pm 0.90$ ($3.71$) & $81.70 \pm 0.84$ ($3.74$) & $81.25 \pm 0.29$ ($3.73$) \\
$t=6, \epsilon = 0.1, \delta = 0.05$  & $88.33 \pm 0.15$ ($3.85$) & $59.41 \pm 0.52$ ($3.84$) & $37.33 \pm 0.66$ ($3.86$) & $73.05 \pm 0.48$ ($3.85$) & $81.84 \pm 0.94$ ($3.87$) & $81.25 \pm 0.92$ ($3.83$) \\
$t=6, \epsilon = 0.05, \delta = 0.1$  & $86.11 \pm 0.70$ ($4.22$) & $58.53 \pm 0.58$ ($4.21$) & $36.67 \pm 0.28$ ($4.23$) & $72.54 \pm 0.37$ ($4.25$) & $81.75 \pm 0.88$ ($4.24$) & $81.38 \pm 0.54$ ($4.25$) \\
$t=6, \epsilon = 0.05, \delta = 0.05$ & $88.89 \pm 0.24$ ($4.36$) & $57.65 \pm 0.96$ ($4.35$) & $40.50 \pm 0.26$ ($4.34$) & $74.56 \pm 0.64$ ($4.34$) & $82.48 \pm 0.87$ ($4.35$) & $82.32 \pm 0.56$ ($4.34$) \\
\hline
$t=7, \epsilon = 0.1, \delta = 0.1$   & $89.44 \pm 0.68$ ($4.72$) & $59.12 \pm 0.99$ ($4.71$) & $40.17 \pm 0.46$ ($4.70$) & $76.15 \pm 0.66$ ($4.73$) & $82.40 \pm 0.74$ ($4.74$) & $82.62 \pm 0.80$ ($4.73$) \\
$t=7, \epsilon = 0.1, \delta = 0.05$  & $86.11 \pm 0.93$ ($4.86$) & $60.00 \pm 0.82$ ($4.83$) & $37.33 \pm 0.85$ ($4.84$) & $76.08 \pm 0.41$ ($4.85$) & $81.75 \pm 0.55$ ($4.87$) & $82.62 \pm 0.15$ ($4.83$) \\
$t=7, \epsilon = 0.05, \delta = 0.1$  & $88.33 \pm 0.37$ ($5.95$) & $62.06 \pm 0.26$ ($5.94$) & $40.00 \pm 0.50$ ($5.96$) & $73.05 \pm 0.33$ ($5.93$) & $79.51 \pm 0.91$ ($5.94$) & $80.94 \pm 0.42$ ($5.95$) \\
$t=7, \epsilon = 0.05, \delta = 0.05$ & $89.75 \pm 0.27$ ($6.31$) & $57.65 \pm 0.99$ ($6.32$) & $40.67 \pm 0.40$ ($6.33$) & $77.43 \pm 0.27$ ($6.33$) & $82.49 \pm 0.94$ ($6.33$) & $83.12 \pm 0.65$ ($6.33$) \\
\hline
$t=8, \epsilon = 0.1, \delta = 0.1$   & $88.33 \pm 0.47$ ($6.82$) & $59.41 \pm 0.79$ ($6.83$) & $40.00 \pm 0.06$ ($6.83$) & $74.56 \pm 0.12$ ($6.82$) & $81.26 \pm 0.34$ ($6.83$) & $79.74 \pm 0.24$ ($6.85$) \\
$t=8, \epsilon = 0.1, \delta = 0.05$  & $88.89 \pm 0.15$ ($7.14$) & $61.18 \pm 0.59$ ($7.15$) & $37.33 \pm 0.73$ ($7.13$) & $74.56 \pm 0.71$ ($7.16$) & $81.70 \pm 0.54$ ($7.13$) & $80.94 \pm 0.43$ ($7.12$) \\
$t=8, \epsilon = 0.05, \delta = 0.1$  & $87.78 \pm 0.86$ ($8.47$) & $60.00 \pm 0.17$ ($8.45$) & $40.00 \pm 0.14$ ($8.48$) & $73.05 \pm 0.61$ ($8.46$) & $82.03 \pm 0.13$ ($8.45$) & $81.25 \pm 0.71$ ($8.45$) \\
$t=8, \epsilon = 0.05, \delta = 0.05$ & $89.44 \pm 0.61$ ($8.75$) & $61.47 \pm 0.96$ ($8.76$) & $40.67 \pm 0.64$ ($8.74$) & $76.08 \pm 0.74$ ($8.73$) & $82.10 \pm 0.76$ ($8.75$) & $82.62 \pm 0.78$ ($8.76$) \\
\hline
$t=9, \epsilon = 0.1, \delta = 0.1$   & $87.78 \pm 0.30$ ($8.91$) & $59.12 \pm 0.55$ ($8.93$) & $37.33 \pm 0.02$ ($8.94$) & $73.05 \pm 0.92$ ($8.90$) & $81.13 \pm 0.87$ ($8.93$) & $79.89 \pm 0.41$ ($8.92$) \\
$t=9, \epsilon = 0.1, \delta = 0.05$  & $86.67 \pm 0.30$ ($9.27$) & $61.08 \pm 0.60$ ($9.25$) & $36.67 \pm 0.80$ ($9.28$) & $73.44 \pm 0.26$ ($9.29$) & $81.70 \pm 0.45$ ($9.30$) & $79.74 \pm 0.12$ ($9.25$) \\
$t=9, \epsilon = 0.05, \delta = 0.1$  & $87.22 \pm 0.55$ ($10.54$) & $60.00 \pm 0.35$ ($10.53$) & $38.50 \pm 0.15$ ($10.55$) & $75.39 \pm 0.98$ ($10.56$) & $82.03 \pm 0.67$ ($10.57$) & $82.62 \pm 0.45$ ($10.52$) \\
$t=9, \epsilon = 0.05, \delta = 0.05$ & $88.33 \pm 0.11$ ($10.67$) & $62.06 \pm 0.05$ ($10.68$) & $40.00 \pm 0.73$ ($10.65$) & $76.08 \pm 0.31$ ($10.64$) & $82.49 \pm 0.45$ ($10.68$) & $83.12 \pm 0.18$ ($10.65$) \\
\hline
$t=10, \epsilon = 0.1, \delta = 0.1$   & $88.89 \pm 0.82$ ($11.24$) & $58.53 \pm 0.28$ ($11.25$) & $38.47 \pm 0.51$ ($11.26$) & $74.76 \pm 0.15$ ($11.26$) & $82.03 \pm 0.49$ ($11.27$) & $80.94 \pm 0.27$ ($11.25$) \\
$t=10, \epsilon = 0.1, \delta = 0.05$  & $87.78 \pm 0.37$ ($11.67$) & $59.41 \pm 0.13$ ($11.68$) & $38.50 \pm 0.32$ ($11.66$) & $76.08 \pm 0.21$ ($11.69$) & $82.03 \pm 0.65$ ($11.68$) & $81.22 \pm 0.54$ ($11.66$) \\
$t=10, \epsilon = 0.05, \delta = 0.1$  & $88.33 \pm 0.29$ ($13.57$) & $61.47 \pm 0.65$ ($13.55$) & $36.67 \pm 0.84$ ($13.58$) & $75.39 \pm 0.64$ ($13.55$) & $81.75 \pm 0.41$ ($13.56$) & $82.32 \pm 0.15$ ($13.58$) \\
$t=10, \epsilon = 0.05, \delta = 0.05$ & $89.75 \pm 0.27$ ($14.89$) & $61.18 \pm 0.32$ ($14.90$) & $40.17 \pm 0.32$ ($14.91$) & $77.43 \pm 0.45$ ($14.88$) & $82.48 \pm 0.54$ ($14.92$) & $82.62 \pm 0.54$ ($14.90$) \\
\hline
$t=11, \epsilon = 0.1, \delta = 0.1$   & $89.44 \pm 0.89$ ($15.43$) & $56.18 \pm 0.38$ ($15.44$) & $38.00 \pm 0.15$ ($15.45$) & $73.05 \pm 0.32$ ($15.42$) & $79.51 \pm 0.67$ ($15.43$) & $79.89 \pm 0.45$ ($15.42$) \\
$t=11, \epsilon = 0.1, \delta = 0.05$  & $88.89 \pm 0.44$ ($16.27$) & $58.53 \pm 0.75$ ($16.25$) & $38.50 \pm 0.88$ ($16.29$) & $72.54 \pm 0.67$ ($16.29$) & $81.26 \pm 0.62$ ($16.30$) & $80.94 \pm 0.45$ ($16.26$) \\
$t=11, \epsilon = 0.05, \delta = 0.1$  & $87.78 \pm 0.72$ ($18.12$) & $57.65 \pm 0.22$ ($18.14$) & $40.00 \pm 0.78$ ($18.10$) & $76.08 \pm 0.89$ ($18.15$) & $81.75 \pm 0.45$ ($18.11$) & $82.62 \pm 0.67$ ($18.13$) \\
$t=11, \epsilon = 0.05, \delta = 0.05$ & $89.75 \pm 0.72$ ($19.08$) & $55.59 \pm 0.92$ ($19.10$) & $40.50 \pm 0.20$ ($19.09$) & $76.58 \pm 0.89$ ($19.11$) & $82.10 \pm 0.65$ ($19.08$) & $83.12 \pm 0.38$ ($19.07$) \\
\hline
$t=12, \epsilon = 0.1, \delta = 0.1$   & $87.78 \pm 0.23$ ($20.10$) & $57.06 \pm 0.31$ ($20.09$) & $36.83 \pm 0.32$ ($20.12$) & $72.80 \pm 0.45$ ($20.12$) & $77.71 \pm 0.54$ ($20.11$) & $81.38 \pm 0.15$ ($20.11$) \\
$t=12, \epsilon = 0.1, \delta = 0.05$  & $88.33 \pm 0.78$ ($21.14$) & $57.65 \pm 0.78$ ($21.15$) & $37.33 \pm 0.21$ ($21.18$) & $72.54 \pm 0.25$ ($21.17$) & $77.71 \pm 0.21$ ($21.12$) & $81.22 \pm 0.61$ ($21.15$) \\
$t=12, \epsilon = 0.05, \delta = 0.1$  & $88.33 \pm 0.65$ ($22.50$) & $58.53 \pm 0.32$ ($22.51$) & $38.50 \pm 0.51$ ($22.53$) & $73.05 \pm 0.54$ ($22.53$) & $79.51 \pm 0.25$ ($22.52$) & $82.32 \pm 0.61$ ($22.51$) \\
$t=12, \epsilon = 0.05, \delta = 0.05$ & $89.44 \pm 0.52$ ($23.42$) & $61.47 \pm 0.21$ ($23.39$) & $40.17 \pm 0.54$ ($23.40$) & $76.08 \pm 0.21$ ($23.43$) & $81.13 \pm 0.63$ ($23.42$) & $82.62 \pm 0.51$ ($23.44$) \\
\hline
$t=13, \epsilon = 0.1, \delta = 0.1$   & $88.33 \pm 0.58$ ($24.28$) & $59.41 \pm 0.03$ ($24.29$) & $38.47 \pm 0.21$ ($24.30$) & $74.68 \pm 0.21$ ($24.31$) & $81.13 \pm 0.34$ ($24.26$) & $80.65 \pm 0.34$ ($24.27$) \\
$t=13, \epsilon = 0.1, \delta = 0.05$  & $87.78 \pm 0.21$ ($25.58$) & $60.29 \pm 0.21$ ($25.56$) & $37.33 \pm 0.20$ ($25.55$) & $75.39 \pm 0.73$ ($25.56$) & $81.70 \pm 0.76$ ($25.59$) & $81.25 \pm 0.29$ ($25.61$) \\
$t=13, \epsilon = 0.05, \delta = 0.1$  & $86.67 \pm 0.51$ ($27.25$) & $61.18 \pm 0.85$ ($27.27$) & $36.67 \pm 0.54$ ($27.28$) & $76.08 \pm 0.35$ ($27.26$) & $79.51 \pm 0.39$ ($27.26$) & $80.94 \pm 0.43$ ($27.27$) \\
$t=13, \epsilon = 0.05, \delta = 0.05$ & $88.89 \pm 0.54$ ($28.42$) & $61.47 \pm 0.74$ ($28.39$) & $40.67 \pm 0.78$ ($28.43$) & $76.58 \pm 0.46$ ($28.44$) & $82.49 \pm 0.27$ ($28.45$) & $82.62 \pm 0.67$ ($28.44$) \\
\hline
$t=14, \epsilon = 0.1, \delta = 0.1$   & $88.89 \pm 0.49$ ($29.74$) & $58.53 \pm 0.37$ ($29.76$) & $36.33 \pm 0.64$ ($29.75$) & $72.63 \pm 0.37$ ($29.73$) & $79.51 \pm 0.49$ ($29.75$) & $81.22 \pm 0.64$ ($29.76$) \\
$t=14, \epsilon = 0.1, \delta = 0.05$  & $88.33 \pm 0.37$ ($30.10$) & $59.41 \pm 0.46$ ($30.12$) & $37.33 \pm 0.79$ ($30.11$) & $73.05 \pm 0.67$ ($30.12$) & $81.70 \pm 0.39$ ($30.11$) & $81.25 \pm 0.37$ ($30.12$) \\
$t=14, \epsilon = 0.05, \delta = 0.1$  & $87.78 \pm 0.49$ ($31.23$) & $60.00 \pm 0.76$ ($31.22$) & $38.50 \pm 0.34$ ($31.21$) & $75.39 \pm 0.39$ ($31.24$) & $82.03 \pm 0.73$ ($31.20$) & $82.62 \pm 0.67$ ($31.21$) \\
$t=14, \epsilon = 0.05, \delta = 0.05$ & $89.44 \pm 0.56$ ($33.41$) & $62.06 \pm 0.48$ ($33.43$) & $40.00 \pm 0.67$ ($33.42$) & $74.68 \pm 0.46$ ($33.40$) & $82.10 \pm 0.16$ ($33.44$) & $83.12 \pm 0.13$ ($33.43$) \\
\hline
$t=15, \epsilon = 0.1, \delta = 0.1$   & $88.89 \pm 0.34$ ($35.21$) & $59.41 \pm 0.43$ ($35.22$) & $37.33 \pm 0.34$ ($35.25$) & $73.80 \pm 0.46$ ($35.24$) & $81.13 \pm 0.46$ ($35.25$) & $78.48 \pm 0.37$ ($35.20$) \\
$t=15, \epsilon = 0.1, \delta = 0.05$  & $88.33 \pm 0.46$ ($36.78$) & $61.18 \pm 0.67$ ($36.80$) & $38.67 \pm 0.64$ ($36.77$) & $73.05 \pm 0.34$ ($36.76$) & $82.40 \pm 0.63$ ($36.80$) & $79.74 \pm 0.64$ ($36.79$) \\
$t=15, \epsilon = 0.05, \delta = 0.1$  & $89.75 \pm 0.37$ ($38.91$) & $60.29 \pm 0.27$ ($38.93$) & $38.47 \pm 0.49$ ($38.95$) & $72.63 \pm 0.38$ ($38.93$) & $79.51 \pm 0.28$ ($38.94$) & $80.94 \pm 0.47$ ($38.95$) \\
$t=15, \epsilon = 0.05, \delta = 0.05$ & $87.78 \pm 0.64$ ($41.38$) & $61.47 \pm 0.34$ ($41.40$) & $40.67 \pm 0.37$ ($41.36$) & $75.68 \pm 0.46$ ($41.39$) & $81.26 \pm 0.29$ ($41.42$) & $81.22 \pm 0.39$ ($41.39$) \\
\hline
\end{tabular}}
\end{center}
\end{table*}

\begin{table*}[!t]
\begin{center}
\caption{Classification accuracies (in \%) obtained by SGE on MUTAG, PTC, ENZYMES, D\&D, NCI1 and NCI109 datasets, where the first node of each graphlet is sampled following different strategies: \emph{highest betweenness centrality}, \emph{highest degree} and \emph{random seeds}.}
\label{tab:expt-diff-sample-strategy}
\resizebox{\textwidth}{!}{
\begin{tabular}{|l|l|c|c|c|c|c|c|}
\hline
Sampling Strategy & Parameters & \ \ \ \ MUTAG \ \ \ \ & \ \ \ \ PTC \ \ \ \ & \ \ \ \ ENZYMES \ \ \ \ & \ \ \ \ D \& D \ \ \ \ & \ \ \ \ NCI1 \ \ \ \ & \ \ \ \ NCI109 \ \ \ \ \\
\hline
\multirow{8}{*}{Highest betweenness centrality} & $t=3$, $\epsilon=0.1$, $\delta=0.1$ & $70.00 \pm 0.89$ & $51.45 \pm 0.57$ & $22.57 \pm 0.45$ & $58.32 \pm 0.52$ & $71.67 \pm 0.52$ & $70.31 \pm 0.52$ \\
 & $t=3$, $\epsilon=0.1$, $\delta=0.05$ & $71.67 \pm 0.48$ & $52.12 \pm 0.75$ & $23.14 \pm 0.52$ & $59.53 \pm 0.42$ & $73.12 \pm 0.74$ & $71.74 \pm 0.52$ \\
 & $t=3$, $\epsilon=0.05$, $\delta=0.1$ & $73.33 \pm 0.86$ & $53.74 \pm 0.54$ & $24.74 \pm 0.63$ & $61.47 \pm 0.74$ & $74.32 \pm 0.71$ & $72.85 \pm 0.37$ \\
 & $t=3$, $\epsilon=0.05$, $\delta=0.05$ & $74.65 \pm 0.47$ & $54.12 \pm 0.87$ & $25.78 \pm 0.35$ & $62.75 \pm 0.54$ & $75.65 \pm 0.42$ & $74.15 \pm 0.43$ \\
\cline{2-8}
 & $t=7$, $\epsilon=0.1$, $\delta=0.1$ & $76.11 \pm 0.57$ & $53.41 \pm 0.25$ & $24.89 \pm 0.57$ & $62.87 \pm 0.42$ & $74.52 \pm 0.41$ & $73.52 \pm 0.34$ \\
 & $t=7$, $\epsilon=0.1$, $\delta=0.05$ & $75.56 \pm 0.38$ & $54.24 \pm 0.41$ & $25.65 \pm 0.78$ & $63.75 \pm 0.39$ & $75.34 \pm 0.53$ & $74.89 \pm 0.42$ \\
 & $t=7$, $\epsilon=0.05$, $\delta=0.1$ & $76.67 \pm 0.39$ & $55.42 \pm 0.74$ & $26.89 \pm 0.45$ & $65.24 \pm 0.51$ & $76.43 \pm 0.42$ & $73.58 \pm 0.45$ \\
 & $t=7$, $\epsilon=0.05$, $\delta=0.05$ & $77.65 \pm 0.74$ & $56.24 \pm 0.56$ & $28.12 \pm 0.74$ & $68.34 \pm 0.24$ & $78.45 \pm 0.36$ & $75.42 \pm 0.54$ \\
\hline
\multirow{8}{*}{Highest degree} & $t=3$, $\epsilon=0.1$, $\delta=0.1$ & $73.89 \pm 0.34$ & $53.82 \pm 0.42$ & $24.89 \pm 0.74$ & $63.47 \pm 0.35$ & $73.24 \pm 0.52$ & $71.25 \pm 0.42$ \\
 & $t=3$, $\epsilon=0.1$, $\delta=0.05$ & $75.00 \pm 0.71$ & $55.12 \pm 0.57$ & $26.12 \pm 0.45$ & $64.76 \pm 0.42$ & $74.35 \pm 0.43$ & $72.89 \pm 0.25$ \\
 & $t=3$, $\epsilon=0.05$, $\delta=0.1$ & $77.78 \pm 0.89$ & $55.89 \pm 0.59$ & $27.34 \pm 0.68$ & $66.34 \pm 0.53$ & $74.76 \pm 0.52$ & $73.72 \pm 0.56$ \\
 & $t=3$, $\epsilon=0.05$, $\delta=0.05$ & $78.24 \pm 0.74$ & $56.78 \pm 0.75$ & $28.74 \pm 0.42$ & $67.24 \pm 0.46$ & $76.24 \pm 0.47$ & $73.87 \pm 0.74$ \\
\cline{2-8}
 & $t=7$, $\epsilon=0.1$, $\delta=0.1$ & $78.89 \pm 0.84$ & $56.49 \pm 0.58$ & $28.87 \pm 0.25$ & $66.54 \pm 0.74$ & $75.32 \pm 0.34$ & $73.62 \pm 0.24$ \\
 & $t=7$, $\epsilon=0.1$, $\delta=0.05$ & $81.67 \pm 0.44$ & $57.32 \pm 0.74$ & $29.45 \pm 0.74$ & $67.18 \pm 0.36$ & $77.54 \pm 0.42$ & $74.42 \pm 0.46$ \\
 & $t=7$, $\epsilon=0.05$, $\delta=0.1$ & $82.22 \pm 0.72$ & $58.19 \pm 0.78$ & $30.45 \pm 0.42$ & $68.87 \pm 0.61$ & $75.76 \pm 0.82$ & $75.32 \pm 0.52$ \\
 & $t=7$, $\epsilon=0.05$, $\delta=0.05$ & $83.74 \pm 0.47$ & $59.76 \pm 0.49$ & $32.42 \pm 0.56$ & $70.24 \pm 0.72$ & $78.52 \pm 0.61$ & $76.56 \pm 0.48$ \\
\hline
\multirow{8}{*}{Random seeds} & $t=3$, $\epsilon=0.1$, $\delta=0.1$   & $71.54 \pm 0.28$ & $52.58 \pm 0.52$ & $24.45 \pm 0.42$ & $60.16 \pm 0.78$ & $72.23 \pm 0.42$ & $70.83 \pm 0.42$ \\
 & $t=3$, $\epsilon=0.1$, $\delta=0.05$  & $75.24 \pm 0.78$ & $54.42 \pm 0.46$ & $25.78 \pm 0.78$ & $61.12 \pm 0.26$ & $73.82 \pm 0.26$ & $72.34 \pm 0.42$ \\
 & $t=3$, $\epsilon=0.05$, $\delta=0.1$  & $86.08 \pm 0.57$ & $55.16 \pm 0.62$ & $27.23 \pm 0.13$ & $63.56 \pm 0.42$ & $75.84 \pm 0.42$ & $74.12 \pm 0.73$ \\
 & $t=3$, $\epsilon=0.05$, $\delta=0.05$ & $85.29 \pm 0.84$ & $56.42 \pm 0.38$ & $28.87 \pm 0.42$ & $64.32 \pm 0.47$ & $76.84 \pm 0.62$ & $75.42 \pm 0.72$ \\
\cline{2-8}
 & $t=7$, $\epsilon=0.1$, $\delta=0.1$   & $85.47 \pm 0.47$ & $56.34 \pm 0.54$ & $34.64 \pm 0.14$ & $71.32 \pm 0.42$ & $80.42 \pm 0.27$ & $80.48 \pm 0.83$ \\
 & $t=7$, $\epsilon=0.1$, $\delta=0.05$  & $84.31 \pm 0.15$ & $58.45 \pm 0.42$ & $35.24 \pm 0.46$ & $72.85 \pm 0.27$ & $81.24 \pm 0.42$ & $81.54 \pm 0.42$ \\
 & $t=7$, $\epsilon=0.05$, $\delta=0.1$  & $86.21 \pm 0.27$ & $60.42 \pm 0.75$ & $36.64 \pm 0.17$ & $75.42 \pm 0.42$ & $82.48 \pm 0.47$ & $82.83 \pm 0.72$ \\
 & $t=7$, $\epsilon=0.05$, $\delta=0.05$ & $87.73 \pm 0.78$ & $62.76 \pm 0.24$ & $38.45 \pm 0.32$ & $76.08 \pm 0.42$ & $82.49 \pm 0.94$ & $83.21 \pm 0.16$ \\
\hline
\end{tabular}
}
\end{center}
\end{table*}

\subsection{Experiments with large $M$ and $T$}

In the original manuscript, we experimented different values of $T\in\lbrace 1,\ldots, 7\rbrace$ and $M$ (according to \textbf{Theorem 1}). Besides, we have also performed other experiments with larger values of $T$ and $M$; in these new experiments, $T \in \{1,\dots,15\}$ and $M$ is set to values $5$ times bigger than the ones suggested by \textbf{Theorem 1}. \tab{tab:expt-graph-class-large-M-T} shows the classification accuracies on MUTAG, PTC, ENZYMES, D \& D, NCI1, and NCI109 datasets. From these results, it can be seen that large values of $M$ make it possible to achieve better accuracies with smaller graphlet structures. For instance, in case of MUTAG dataset, sampling $M$ graphlets (following \textbf{Theorem 1} with $t=3, \epsilon=0.1, \delta=0.1$) makes it possible to achieve a classification accuracy of $71.67 \pm 0.86$ (see Table VI in the manuscript), whereas increasing $M$ five times makes performance reaching an upper limit of $84.44 \pm 0.71$ (see \tab{tab:expt-graph-class-large-M-T}). A similar behavior is observed on the other datasets as well; nevertheless, this gain is obtained to the detriment of an increase in the computational efficiency.

\begin{table*}[!t]
\begin{center}
\caption{Classification accuracies (in \%) obtained by SGE on MUTAG, PTC, ENZYMES, D\&D, NCI1 and NCI109 datasets by considering the combination of histograms of graphlets with different number of edges.}
\label{tab:expt-graph-class1}
\resizebox{\textwidth}{!}{
\begin{tabular}{|l|c|c|c|c|c|c|}
\hline
Kernel & \ \ \ \ MUTAG \ \ \ \ & \ \ \ \ PTC \ \ \ \ & \ \ \ \ ENZYMES \ \ \ \ & \ \ \ \ D \& D \ \ \ \ & \ \ \ \ NCI1 \ \ \ \ & \ \ \ \ NCI109 \ \ \ \ \\
\hline
SGE ($t=\lbrace 3,\dots,4 \rbrace, \epsilon = 0.1, \delta = 0.1$) & $77.22\pm 0.47$ & $58.53\pm 0.79$ & $28.50\pm 0.06$ & $63.42\pm 0.12$ & $77.45\pm 0.34$ & $78.25\pm 0.24$ \\
SGE ($t=\lbrace 3,\dots,4 \rbrace, \epsilon = 0.1, \delta = 0.05$) & $83.89\pm 0.15$ & $63.24\pm 0.59$ & $27.33\pm 0.73$ & $64.27\pm 0.71$ & $77.65\pm 0.54$ & $78.67\pm 0.43$ \\
SGE ($t=\lbrace 3,\dots,4 \rbrace, \epsilon = 0.05, \delta = 0.1$) & $88.89\pm 0.86$ & $58.24\pm 0.17$ & $31.50\pm 0.14$ & $66.58\pm 0.61$ & $78.76\pm 0.13$ & $79.34\pm 0.71$ \\
SGE ($t=\lbrace 3,\dots,4 \rbrace, \epsilon = 0.05, \delta = 0.05$) & $87.78\pm 0.61$ & $60.00\pm 0.96$ & $33.17\pm 0.64$ & $66.66\pm 0.74$ & $78.87\pm 0.76$ & $79.98\pm 0.78$ \\
\hline
SGE ($t=\lbrace 3,\dots,5 \rbrace, \epsilon = 0.1, \delta = 0.1$) & $85.00\pm 0.30$ & $57.94\pm 0.55$ & $30.67\pm 0.02$ & $66.75\pm 0.92$ & $78.25\pm 0.87$ & $79.56\pm 0.41$ \\
SGE ($t=\lbrace 3,\dots,5 \rbrace, \epsilon = 0.1, \delta = 0.05$) & $87.22\pm 0.30$ & $57.94\pm 0.60$ & $31.50\pm 0.80$ & $67.66\pm 0.26$ & $78.92\pm 0.45$ & $79.78\pm 0.12$ \\
SGE ($t=\lbrace 3,\dots,5 \rbrace, \epsilon = 0.05, \delta = 0.1$) & $86.11\pm 0.55$ & $55.88\pm 0.35$ & $34.50\pm 0.15$ & $68.63\pm 0.98$ & $79.81\pm 0.67$ & $80.23\pm 0.45$ \\
SGE ($t=\lbrace 3,\dots,5 \rbrace, \epsilon = 0.05, \delta = 0.05$) & $84.44\pm 0.11$ & $56.76\pm 0.05$ & $34.50\pm 0.73$ & $69.56\pm 0.31$ & $80.12\pm 0.45$ & $79.93\pm 0.18$ \\
\hline
SGE ($t=\lbrace 3,\dots,6 \rbrace, \epsilon = 0.1, \delta = 0.1$) & $87.22\pm 0.89$ & $58.82\pm 0.38$ & $31.17\pm 0.15$ & $70.15\pm 0.32$ & $79.48\pm 0.67$ & $80.12\pm 0.45$ \\
SGE ($t=\lbrace 3,\dots,6 \rbrace, \epsilon = 0.1, \delta = 0.05$) & $87.22\pm 0.44$ & $57.35\pm 0.75$ & $33.17\pm 0.88$ & $70.81\pm 0.67$ & $80.12\pm 0.62$ & $80.78\pm 0.45$ \\
SGE ($t=\lbrace 3,\dots,6 \rbrace, \epsilon = 0.05, \delta = 0.1$) & $86.11\pm 0.72$ & $57.35\pm 0.22$ & $35.50\pm 0.78$ & $72.27\pm 0.89$ & $81.14\pm 0.45$ & $81.76\pm 0.67$ \\
SGE ($t=\lbrace 3,\dots,6 \rbrace, \epsilon = 0.05, \delta = 0.05$) & $86.11\pm 0.72$ & $56.18\pm 0.92$ & $35.33\pm 0.20$ & $73.10\pm 0.89$ & $81.98\pm 0.65$ & $81.76\pm 0.38$ \\
\hline
SGE ($t=\lbrace 3,\dots,7 \rbrace, \epsilon = 0.1, \delta = 0.1$) & $88.33\pm 0.15$ & $58.24\pm 0.15$ & $36.67\pm 0.57$ & $73.44\pm 0.12$ & $81.89\pm 0.70$ & $81.38\pm 0.14$ \\
SGE ($t=\lbrace 3,\dots,7 \rbrace, \epsilon = 0.1, \delta = 0.05$) & $87.78\pm 0.51$ & $57.65\pm 0.91$ & $38.50\pm 0.33$ & $74.59\pm 0.46$ & $82.10\pm 0.70$ & $82.67\pm 0.19$ \\
SGE ($t=\lbrace 3,\dots,7 \rbrace, \epsilon = 0.05, \delta = 0.1$) & $87.78\pm 0.11$ & $59.71\pm 0.36$ & $40.17\pm 0.11$ & $75.39\pm 0.94$ & $82.49\pm 0.96$ & $82.92\pm 0.42$ \\
SGE ($t=\lbrace 3,\dots,7 \rbrace, \epsilon = 0.05, \delta = 0.05$) & $86.67\pm 0.68$ & $59.41\pm 0.17$ & $39.67\pm 0.75$ & $\mathbf{77.31\pm 0.66}$ & $\mathbf{82.61\pm 0.40}$ & $\mathbf{83.12\pm 0.57}$ \\
\hline
\end{tabular}}
\end{center}
\end{table*}

\begin{table*}[!t]
\begin{center}
\caption{Classification accuracies (in \%) obtained by SGE on COIL, GREC, AIDS, MAO and ENZYMES (labeled) datasets by considering the combination of histograms of graphlets with different number of edges.}
\label{tab:expt-graph-class2}
\resizebox{\textwidth}{!}{
\begin{tabular}{|l|c|c|c|c|c|}
\hline
Method & \ \ \ \ COIL \ \ \ \ & \ \ \ \ GREC \ \ \ \ & \ \ \ \ AIDS \ \ \ \ & \ \ \ \ MAO \ \ \ \ & \ \ \ \ ENZYMES (labeled) \ \ \ \ \\
\hline
SGE ($t=\lbrace1,\ldots,2\rbrace, \epsilon = 0.1, \delta = 0.1$) & $92.10$ & $93.78$ & $95.09$ & $86.76$ & $33.43\pm 0.52$ \\
SGE ($t=\lbrace1,\ldots,2\rbrace, \epsilon = 0.1, \delta = 0.05$) & $93.20$ & $93.25$ & $94.87$ & $86.76$ & $34.54\pm 0.34$ \\
SGE ($t=\lbrace1,\ldots,2\rbrace, \epsilon = 0.05, \delta = 0.1$) & $94.30$ & $94.25$ & $95.14$ & $88.23$ & $35.22\pm 0.97$ \\
SGE ($t=\lbrace1,\ldots,2\rbrace, \epsilon = 0.05, \delta = 0.05$) & $95.10$ & $96.17$ & $95.56$ & $88.23$ & $36.24\pm 0.89$ \\
\hline
SGE ($t=\lbrace1,\ldots,3\rbrace, \epsilon = 0.1, \delta = 0.1$) & $92.40$ & $95.13$ & $95.54$ & $89.71$ & $39.84\pm 0.89$ \\
SGE ($t=\lbrace1,\ldots,3\rbrace, \epsilon = 0.1, \delta = 0.05$) & $94.10$ & $96.47$ & $96.14$ & $89.71$ & $40.56\pm 0.56$ \\
SGE ($t=\lbrace1,\ldots,3\rbrace, \epsilon = 0.05, \delta = 0.1$) & $94.70$ & $96.98$ & $97.34$ & $91.18$ & $44.32\pm 0.26$ \\
SGE ($t=\lbrace1,\ldots,3\rbrace, \epsilon = 0.05, \delta = 0.05$) & $95.20$ & $97.37$ & $97.12$ & $92.65$ & $45.53\pm 0.14$ \\
\hline
SGE ($t=\lbrace1,\ldots,4\rbrace, \epsilon = 0.1, \delta = 0.1$) & $94.10$ & $97.29$ & $97.12$ & $91.18$ & $45.32\pm 0.67$ \\
SGE ($t=\lbrace1,\ldots,4\rbrace, \epsilon = 0.1, \delta = 0.05$) & $95.20$ & $97.77$ & $97.59$ & $91.18$ & $46.12\pm 0.52$ \\
SGE ($t=\lbrace1,\ldots,4\rbrace, \epsilon = 0.05, \delta = 0.1$) & $96.10$ & $98.13$ & $97.37$ & $92.65$ & $50.15\pm 0.25$ \\
SGE ($t=\lbrace1,\ldots,4\rbrace, \epsilon = 0.05, \delta = 0.05$) & $97.20$ & $99.07$ & $97.76$ & $94.12$ & $50.98\pm 0.81$ \\
\hline
SGE ($t=\lbrace1,\ldots,5\rbrace, \epsilon = 0.1, \delta = 0.1$) & $95.30$ & $97.89$ & $98.17$ & $94.12$ & $50.67\pm 0.73$ \\
SGE ($t=\lbrace1,\ldots,5\rbrace, \epsilon = 0.1, \delta = 0.05$) & $96.20$ & $98.17$ & $98.54$ & $94.12$ & $51.67\pm 0.62$ \\
SGE ($t=\lbrace1,\ldots,5\rbrace, \epsilon = 0.05, \delta = 0.1$) & $98.20$ & $98.89$ & $98.65$ & $97.06$ & $57.67\pm 0.70$ \\
SGE ($t=\lbrace1,\ldots,5\rbrace, \epsilon = 0.05, \delta = 0.05$) & $\mathbf{98.90}$ & $99.43$ & $98.76$ & $\mathbf{98.53}$ & $58.33\pm 0.31$ \\
\hline
\end{tabular}}
\end{center}
\end{table*}

\subsection{Experiments with different strategies for sampling the initial node}

In these experiments, we consider different strategies in order to sample the initial node. In the original manuscript, the initial node was sampled uniformly from all the nodes of a given input graph. However, there are many possible ways of sampling the initial node; we have come out with three different strategies: (1) highest betweenness centrality, (2) highest node degree, and (3) random seed, where strategies ``highest betweenness centrality'' and ``highest node degree'' refer to the way of selecting the initial node according to the highest betweenness centrality and node degree respectively, and ``random seed'' stands for randomly initializing the initial node. We have used these three strategies to select the first node of each graphlet sampling. \tab{tab:expt-diff-sample-strategy} shows the results of classifying graphs, where the graph embedding is obtained by our proposed SGE and the sampling of the first node is done with the three strategies mentioned above. Among these strategies, the ones considering the highest degree and the highest betweenness centrality (for selecting the initial node) have performed worse than the original algorithm. 
Indeed, seeding nodes with these two strategies makes it possible to explore only a small subset of graphlets from the input graphs, resulting into biased distributions in the estimated graphlet histograms. In contrast, random node seeding allows us to explore/cover a larger subset of graphlets in the input graphs and provides us with a better estimate of the underlying graphlet distributions, as also corroborated through experiments (see again \tab{tab:expt-diff-sample-strategy}).

\subsection{Experiments combining graphlet histograms of different $T$}

We have done another set of experiments by combining the histograms of graphlets with different $T$ (upper bound of $t$). \tab{tab:expt-graph-class1} contains the results obtained by SGE on the MUTAG, PTC, ENZYMES, D\&D, NCI1 and NCI109 datasets, where we have combined the histograms of graphlets with different number of edges. We observe that on some of the datasets, histograms of fixed graphlet orders achieve the highest classification  accuracy (Table VI of the main manuscript), whereas, on  some other datasets, histograms of combined graphlet orders achieve the best performance. \tab{tab:expt-graph-class2} contains the results obtained by combining the histograms of graphlets with different number of edges on the COIL, GREC, AIDS, MAO and ENZYMES (labeled) datasets. In these datasets as well, we have observed the same phenomena as the previous ones.

\begin{figure}[!ht]
\begin{center}
\includegraphics[width=0.9\columnwidth]{MNIST-graph}
\caption{Graph representation of digits from MNIST dataset.}
\label{fig:mnist-graph}
\end{center}
\end{figure}

\subsection{Graph representation of digits from MNIST dataset}
For obtaining the graph representation of the digits from MNIST dataset, we skeletonize the binary digit images and consider each pixel on the skeleton as a node, where all the adjacent (considering $8$ neighbours) pixels on the skeleton are connected and each connection constitutes an edge (see~\fig{fig:mnist-graph}). For labeling the graph nodes, we consider the general rotation variant shape context descriptors~\cite{Belongie2002} on each of the nodes and cluster them using k-means with $k=20$, which assign each node a label in $[1,20]$.

%------------------------------------------------------------------------------------------------------------------------------------------------------------------------------------------------------------
%% The Appendices part is started with the command \appendix;
%% appendix sections are then done as normal sections
%% \appendix

%% \section{}
%% \label{}

%% References
%%
%% Following citation commands can be used in the body text:
%% Usage of \cite is as follows:
%%   \cite{key}         ==>>  [#]
%%   \cite[chap. 2]{key} ==>> [#, chap. 2]
%%
%% References with bibTeX database:


% if have a single appendix:
%\appendix[Proof of the Zonklar Equations]
% or
%\appendix  % for no appendix heading
% do not use \section anymore after \appendix, only \section*
% is possibly needed

% use appendices with more than one appendix
% then use \section to start each appendix
% you must declare a \section before using any
% \subsection or using \label (\appendices by itself
% starts a section numbered zero.)
% you can choose not to have a title for an appendix
% if you want by leaving the argument blank

% use section* for acknowledgement
% Can use something like this to put references on a page
% by themselves when using endfloat and the captionsoff option.
\ifCLASSOPTIONcaptionsoff
  \newpage
\fi

% trigger a \newpage just before the given reference
% number - used to balance the columns on the last page
% adjust value as needed - may need to be readjusted if
% the document is modified later
%\IEEEtriggeratref{8}
% The "triggered" command can be changed if desired:
%\IEEEtriggercmd{\enlargethispage{-5in}}

% references section

{
\bibliographystyle{IEEEtranS}
\bstctlcite{IEEEexample:BSTcontrol}
\bibliography{IEEEabrv,./bibliography/bibliography.bib}
}

% can use a bibliography generated by BibTeX as a .bbl file
% BibTeX documentation can be easily obtained at:
% http://www.ctan.org/tex-archive/biblio/bibtex/contrib/doc/
% The IEEEtran BibTeX style support page is at:
% http://www.michaelshell.org/tex/ieeetran/bibtex/
% \bibliographystyle{IEEEtranS}
% \bstctlcite{IEEEexample:BSTcontrol}
% argument is your BibTeX string definitions and bibliography database(s)
% \bibliography{IEEEabrv,./bibliography/bibliography}

% if you will not have a photo at all:

% insert where needed to balance the two columns on the last page with
% biographies
%\newpage

% You can push biographies down or up by placing
% a \vfill before or after them. The appropriate
% use of \vfill depends on what kind of text is
% on the last page and whether or not the columns
% are being equalized.

%\vfill

% Can be used to pull up biographies so that the bottom of the last one
% is flush with the other column.
%\enlargethispage{-5in}

% that's all folks
\end{document}
