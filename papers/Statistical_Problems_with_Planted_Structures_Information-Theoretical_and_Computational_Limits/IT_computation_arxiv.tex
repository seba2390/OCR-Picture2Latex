\documentclass[letterpaper,11pt]{article}
\usepackage[toc,page]{appendix}
\usepackage[margin=1in]{geometry}
\usepackage[bookmarks, colorlinks=true, plainpages = false, citecolor = blue,linkcolor=red,urlcolor = blue, filecolor = blue,pagebackref]{hyperref}
%% Note, I added pagebackref to the options for hyperref so that page number back references appear after each paper listing.   -BH 4/29/15
%\usepackage{url}\urlstyle{rm}
\usepackage{amsmath,amsfonts,amsthm,amssymb,bm, verbatim,dsfont,mathtools}
%\usepackage{algorithm,algorithmic}
\usepackage{color,graphicx,appendix}
\usepackage{subfigure}
\usepackage{etoolbox}
\usepackage{tikz}
\usepackage{xr,xspace}
\usepackage{todonotes}
\usepackage{paralist}
\usepackage{caption,soul}
%\usepackage[ruled,vlined]{algorithm2e}
\usepackage{algorithm}% http://ctan.org/pkg/algorithms
\usepackage{algorithmic}% http://ctan.org/pkg/algorithms
\usepackage{enumitem}
\makeatletter
%\renewcommand{\ALG@name}{SDP}
%\renewcommand{\listalgorithmname}{List of \ALG@name s}

%%%%% THEOREM STYLE DEFINITIONS
%\theoremstyle{plain}
%\newtheorem{theorem}{Theorem}
%\newtheorem{lemma}{Lemma}
%\newtheorem{proposition}{Proposition}
%\newtheorem{corollary}{Corollary}
%\theoremstyle{definition}
%\newtheorem{definition}{Definition}
%\newtheorem{hypothesis}{Hypothesis}
%\newtheorem{conjecture}{Conjecture}
%\newtheorem{question}{Question}
%\newtheorem{remark}{Remark}
%\newtheorem*{remark*}{Remark}

\newtheorem{theorem}{Theorem}
\newtheorem{lemma}{Lemma}
\newtheorem{proposition}{Proposition}
\newtheorem{corollary}{Corollary}
\theoremstyle{definition}
\newtheorem{definition}{Definition}
\newtheorem{hypothesis}{Hypothesis}
\newtheorem{conjecture}{Conjecture}
\newtheorem{question}{Question}
\newtheorem{remark}{Remark}
\newtheorem{assumption}{Assumption}


%\usepackage{enumerate}

\usepackage{tikz}

\usepackage{xspace,prettyref}
\usepackage{bm}
% for prettyref.sty
\newrefformat{eq}{(\ref{#1})}
\newrefformat{chap}{Chapter~\ref{#1}}
\newrefformat{sec}{Section~\ref{#1}}
\newrefformat{alg}{Algorithm~\ref{#1}}
\newrefformat{fig}{Fig.~\ref{#1}}
\newrefformat{tab}{Table~\ref{#1}}
\newrefformat{rmk}{Remark~\ref{#1}}
\newrefformat{clm}{Claim~\ref{#1}}
\newrefformat{def}{Definition~\ref{#1}}
\newrefformat{cor}{Corollary~\ref{#1}}
\newrefformat{lmm}{Lemma~\ref{#1}}
\newrefformat{prop}{Proposition~\ref{#1}}
\newrefformat{app}{Appendix~\ref{#1}}
\newrefformat{hyp}{Hypothesis~\ref{#1}}
\newrefformat{thm}{Theorem~\ref{#1}}
\newrefformat{ass}{Assumption~\ref{#1}}
\newrefformat{conj}{Conjecture~\ref{#1}}

\newcommand{\ie}{i.e.\xspace}
\renewcommand{\P}{\mathcal{P} }
\newcommand{\Q}{\mathcal{Q}}
\newcommand{\Exp}{\mathbb{E}}
\newcommand{\contig}{\trianglelefteq}
\newcommand{\indicator}[1]{\bm{1}_{#1}}
\renewcommand{\hat}{\widehat}
\renewcommand{\tilde}{\widetilde}
\newcommand{\argmax}{\mathrm{argmax}}

\usepackage{color}
\newcommand{\red}{\color{red}}
\newcommand{\blue}{\color{blue}}
\newcommand{\nb}[1]{{\sf\blue[#1]}}
\newcommand{\nbr}[1]{{\sf\red #1}}

\renewcommand{\implies}{\Rightarrow}
\newcommand{\1}[1]{{\mathbf{1}_{\left\{{#1}\right\}}}}
\newcommand{\post}[2]{\begin{center} \includegraphics[width=#2]{#1} \end{center} }
\newcommand \E[1]{\mathbb{E}[#1]}
%\newcommand{\Perp}{\perp \! \! \! \perp}
\newcommand{\Perp}{\perp}
\newcommand{\Hyper}{\text{Hypergeometric}}
%\newcommand \P[1]{\mathbb{P}[#1]}


%% Wu
\newcommand{\bfa}{{\mathbf{a}}}
\newcommand{\bfb}{{\mathbf{b}}}
\newcommand{\bfc}{{\mathbf{c}}}
\newcommand{\bfd}{{\mathbf{d}}}
\newcommand{\bfe}{{\mathbf{e}}}
\newcommand{\bff}{{\mathbf{f}}}
\newcommand{\bfg}{{\mathbf{g}}}
\newcommand{\bfh}{{\mathbf{h}}}
\newcommand{\bfi}{{\mathbf{i}}}
\newcommand{\bfj}{{\mathbf{j}}}
\newcommand{\bfk}{{\mathbf{k}}}
\newcommand{\bfl}{{\mathbf{l}}}
\newcommand{\bfm}{{\mathbf{m}}}
\newcommand{\bfn}{{\mathbf{n}}}
\newcommand{\bfo}{{\mathbf{o}}}
\newcommand{\bfp}{{\mathbf{p}}}
\newcommand{\bfq}{{\mathbf{q}}}
\newcommand{\bfr}{{\mathbf{r}}}
\newcommand{\bfs}{{\mathbf{s}}}
\newcommand{\bft}{{\mathbf{t}}}
\newcommand{\bfu}{{\mathbf{u}}}
\newcommand{\bfv}{{\mathbf{v}}}
\newcommand{\bfw}{{\mathbf{w}}}
\newcommand{\bfx}{{\mathbf{x}}}
\newcommand{\bfy}{{\mathbf{y}}}
\newcommand{\bfz}{{\mathbf{z}}}
\newcommand{\bfA}{{\mathbf{A}}}
\newcommand{\bfB}{{\mathbf{B}}}
\newcommand{\bfC}{{\mathbf{C}}}
\newcommand{\bfD}{{\mathbf{D}}}
\newcommand{\bfE}{{\mathbf{E}}}
\newcommand{\bfF}{{\mathbf{F}}}
\newcommand{\bfG}{{\mathbf{G}}}
\newcommand{\bfH}{{\mathbf{H}}}
\newcommand{\bfI}{{\mathbf{I}}}
\newcommand{\bfJ}{{\mathbf{J}}}
\newcommand{\bfK}{{\mathbf{K}}}
\newcommand{\bfL}{{\mathbf{L}}}
\newcommand{\bfM}{{\mathbf{M}}}
\newcommand{\bfN}{{\mathbf{N}}}
\newcommand{\bfO}{{\mathbf{O}}}
\newcommand{\bfP}{{\mathbf{P}}}
\newcommand{\bfQ}{{\mathbf{Q}}}
\newcommand{\bfR}{{\mathbf{R}}}
\newcommand{\bfS}{{\mathbf{S}}}
\newcommand{\bfT}{{\mathbf{T}}}
\newcommand{\bfU}{{\mathbf{U}}}
\newcommand{\bfV}{{\mathbf{V}}}
\newcommand{\bfW}{{\mathbf{W}}}
\newcommand{\bfX}{{\mathbf{X}}}
\newcommand{\bfY}{{\mathbf{Y}}}
\newcommand{\bfZ}{{\mathbf{Z}}}

\newcommand{\bbA}{{\mathbb{A}}}
\newcommand{\bbB}{{\mathbb{B}}}
\newcommand{\bbC}{{\mathbb{C}}}
\newcommand{\bbD}{{\mathbb{D}}}
\newcommand{\bbE}{{\mathbb{E}}}
\newcommand{\bbF}{{\mathbb{F}}}
\newcommand{\bbG}{{\mathbb{G}}}
\newcommand{\bbH}{{\mathbb{H}}}
\newcommand{\bbI}{{\mathbb{I}}}
\newcommand{\bbJ}{{\mathbb{J}}}
\newcommand{\bbK}{{\mathbb{K}}}
\newcommand{\bbL}{{\mathbb{L}}}
\newcommand{\bbM}{{\mathbb{M}}}
\newcommand{\bbN}{{\mathbb{N}}}
\newcommand{\bbO}{{\mathbb{O}}}
\newcommand{\bbP}{{\mathbb{P}}}
\newcommand{\bbQ}{{\mathbb{Q}}}
\newcommand{\bbR}{{\mathbb{R}}}
\newcommand{\bbS}{{\mathbb{S}}}
\newcommand{\bbT}{{\mathbb{T}}}
\newcommand{\bbU}{{\mathbb{U}}}
\newcommand{\bbV}{{\mathbb{V}}}
\newcommand{\bbW}{{\mathbb{W}}}
\newcommand{\bbX}{{\mathbb{X}}}
\newcommand{\bbY}{{\mathbb{Y}}}
\newcommand{\bbZ}{{\mathbb{Z}}}
                         
                         
\newcommand{\sfa}{{\mathsf{a}}}
\newcommand{\sfb}{{\mathsf{b}}}
\newcommand{\sfc}{{\mathsf{c}}}
\newcommand{\sfd}{{\mathsf{d}}}
\newcommand{\sfe}{{\mathsf{e}}}
\newcommand{\sff}{{\mathsf{f}}}
\newcommand{\sfg}{{\mathsf{g}}}
\newcommand{\sfh}{{\mathsf{h}}}
\newcommand{\sfi}{{\mathsf{i}}}
\newcommand{\sfj}{{\mathsf{j}}}
\newcommand{\sfk}{{\mathsf{k}}}
\newcommand{\sfl}{{\mathsf{l}}}
\newcommand{\sfm}{{\mathsf{m}}}
\newcommand{\sfn}{{\mathsf{n}}}
\newcommand{\sfo}{{\mathsf{o}}}
\newcommand{\sfp}{{\mathsf{p}}}
\newcommand{\sfq}{{\mathsf{q}}}
\newcommand{\sfr}{{\mathsf{r}}}
\newcommand{\sfs}{{\mathsf{s}}}
\newcommand{\sft}{{\mathsf{t}}}
\newcommand{\sfu}{{\mathsf{u}}}
\newcommand{\sfv}{{\mathsf{v}}}
\newcommand{\sfw}{{\mathsf{w}}}
\newcommand{\sfx}{{\mathsf{x}}}
\newcommand{\sfy}{{\mathsf{y}}}
\newcommand{\sfz}{{\mathsf{z}}}
\newcommand{\sfA}{{\mathsf{A}}}
\newcommand{\sfB}{{\mathsf{B}}}
\newcommand{\sfC}{{\mathsf{C}}}
\newcommand{\sfD}{{\mathsf{D}}}
\newcommand{\sfE}{{\mathsf{E}}}
\newcommand{\sfF}{{\mathsf{F}}}
\newcommand{\sfG}{{\mathsf{G}}}
\newcommand{\sfH}{{\mathsf{H}}}
\newcommand{\sfI}{{\mathsf{I}}}
\newcommand{\sfJ}{{\mathsf{J}}}
\newcommand{\sfK}{{\mathsf{K}}}
\newcommand{\sfL}{{\mathsf{L}}}
\newcommand{\sfM}{{\mathsf{M}}}
\newcommand{\sfN}{{\mathsf{N}}}
\newcommand{\sfO}{{\mathsf{O}}}
\newcommand{\sfP}{{\mathsf{P}}}
\newcommand{\sfQ}{{\mathsf{Q}}}
\newcommand{\sfR}{{\mathsf{R}}}
\newcommand{\sfS}{{\mathsf{S}}}
\newcommand{\sfT}{{\mathsf{T}}}
\newcommand{\sfU}{{\mathsf{U}}}
\newcommand{\sfV}{{\mathsf{V}}}
\newcommand{\sfW}{{\mathsf{W}}}
\newcommand{\sfX}{{\mathsf{X}}}
\newcommand{\sfY}{{\mathsf{Y}}}
\newcommand{\sfZ}{{\mathsf{Z}}}

\newcommand{\TV}{d_{\rm TV}}

\newcommand{\floor}[1]{{\left\lfloor {#1} \right \rfloor}}
\newcommand{\ceil}[1]{{\left\lceil {#1} \right \rceil}}

\usepackage{xspace,prettyref}
\newcommand{\CML}{\widehat{C}_{\rm ML}}
\newcommand{\diverge}{\to\infty}
\newcommand{\eqdistr}{{\stackrel{\rm (d)}{=}}}
\newcommand{\iiddistr}{{\stackrel{\text{\iid}}{\sim}}}
\newcommand{\ones}{\mathbf 1}
\newcommand{\zeros}{\mathbf 0}
\newcommand{\reals}{{\mathbb{R}}}
\newcommand{\integers}{{\mathbb{Z}}}
\newcommand{\naturals}{{\mathbb{N}}}
\newcommand{\rationals}{{\mathbb{Q}}}
\newcommand{\naturalsex}{\overline{\mathbb{N}}}
\newcommand{\symm}{{\mbox{\bf S}}}  % symmetric matrices
\newcommand{\supp}{{\rm supp}}
\newcommand{\eexp}{{\rm e}}
\newcommand{\rexp}[1]{{\rm e}^{#1}}
\newcommand{\identity}{\mathbf I}
\newcommand{\allones}{\mathbf J}
%\newcommand{\zeros}{\mathbf 0}

\newcommand{\diff}{{\rm d}}

\newcommand{\Expect}{\mathbb{E}}
\newcommand{\expect}[1]{\mathbb{E}\left[ #1 \right]}
\newcommand{\eexpect}[1]{\mathbb{E}[ #1 ]}
\newcommand{\expects}[2]{\mathbb{E}_{#2}\left[ #1 \right]}
\newcommand{\tExpect}{{\tilde{\mathbb{E}}}}
%\newcommand{\Prob}{\mathop{\mathbb{P}}}
\newcommand{\Prob}{\mathbb{P}}
\newcommand{\pprob}[1]{ \mathbb{P}\{ #1 \} }
\newcommand{\prob}[1]{ \mathbb{P}\left\{ #1 \right\} }
\newcommand{\tProb}{{\tilde{\mathbb{P}}}}
\newcommand{\tprob}[1]{{ \tProb\left\{ #1 \right\} }}
\newcommand{\hProb}{\widehat{\mathbb{P}}}
\newcommand{\probs}[2]{\mathbb{P}_{#2}\left\{ #1 \right\} }
\newcommand{\toprob}{\xrightarrow{\Prob}}
\newcommand{\tolp}[1]{\xrightarrow{L^{#1}}}
\newcommand{\toas}{\xrightarrow{{\rm a.s.}}}
\newcommand{\toae}{\xrightarrow{{\rm a.e.}}}
\newcommand{\todistr}{\xrightarrow{{\rm D}}}
\newcommand{\toweak}{\rightharpoonup}
\newcommand{\var}{\mathsf{var}}
\newcommand{\Cov}{\text{Cov}}
\newcommand\indep{\protect\mathpalette{\protect\independenT}{\perp}}
\def\independenT#1#2{\mathrel{\rlap{$#1#2$}\mkern2mu{#1#2}}}
\newcommand{\Bern}{{\rm Bern}}
\newcommand{\Binom}{{\rm Binom}}
\newcommand{\Pois}{{\rm Pois}}
\newcommand{\Hyp}{{\rm Hyp}}
\newcommand{\eg}{e.g.\xspace}
\newcommand{\iid}{i.i.d.\xspace}
% for prettyref.sty
\newrefformat{eq}{(\ref{#1})}
\newrefformat{chap}{Chapter~\ref{#1}}
\newrefformat{sec}{Section~\ref{#1}}
\newrefformat{alg}{Algorithm~\ref{#1}}
\newrefformat{fig}{Fig.~\ref{#1}}
\newrefformat{tab}{Table~\ref{#1}}
\newrefformat{rmk}{Remark~\ref{#1}}
\newrefformat{clm}{Claim~\ref{#1}}
\newrefformat{def}{Definition~\ref{#1}}
\newrefformat{cor}{Corollary~\ref{#1}}
\newrefformat{lmm}{Lemma~\ref{#1}}
\newrefformat{prop}{Proposition~\ref{#1}}
\newrefformat{app}{Appendix~\ref{#1}}
\newrefformat{hyp}{Hypothesis~\ref{#1}}
\newrefformat{thm}{Theorem~\ref{#1}}
\newrefformat{ass}{Assumption~\ref{#1}}
\newcommand{\ntok}[2]{{#1,\ldots,#2}}
\newcommand{\pth}[1]{\left( #1 \right)}
\newcommand{\qth}[1]{\left[ #1 \right]}
\newcommand{\sth}[1]{\left\{ #1 \right\}}
\newcommand{\bpth}[1]{\Bigg( #1 \Bigg)}
\newcommand{\bqth}[1]{\Bigg[ #1 \Bigg]}
\newcommand{\bsth}[1]{\Bigg\{ #1 \Bigg\}}
\newcommand{\norm}[1]{\left\|{#1} \right\|}
\newcommand{\lpnorm}[1]{\left\|{#1} \right\|_{p}}
\newcommand{\linf}[1]{\left\|{#1} \right\|_{\infty}}
\newcommand{\lnorm}[2]{\left\|{#1} \right\|_{{#2}}}
\newcommand{\Lploc}[1]{L^{#1}_{\rm loc}}
\newcommand{\hellinger}{d_{\rm H}}
\newcommand{\Fnorm}[1]{\lnorm{#1}{\rm F}}
\newcommand{\fnorm}[1]{\|#1\|_{\rm F}}
% \newcommand{\opnorm}[1]{\lnorm{#1}{\rm op}}
\newcommand{\opnorm}[1]{\left\| #1 \right\|_2}
% inner product
\newcommand{\iprod}[2]{\left \langle #1, #2 \right\rangle}
\newcommand{\Iprod}[2]{\langle #1, #2 \rangle}
% 12/02/2007
\newcommand{\indc}[1]{{\mathbf{1}_{\left\{{#1}\right\}}}}
\newcommand{\Indc}{\mathbf{1}}
\newcommand{\diag}[1]{\mathsf{diag} \left\{ {#1} \right\} }
\newcommand{\degr}{\mathsf{deg} }

\newcommand{\dTV}{d_{\rm TV}}
\newcommand{\tb}{\widetilde{b}}
\newcommand{\tr}{\widetilde{r}}
\newcommand{\tc}{\widetilde{c}}
\newcommand{\tu}{{\widetilde{u}}}
\newcommand{\tv}{{\widetilde{v}}}
\newcommand{\tx}{{\widetilde{x}}}
\newcommand{\ty}{{\widetilde{y}}}
\newcommand{\tz}{{\widetilde{z}}}
\newcommand{\tA}{{\widetilde{A}}}
\newcommand{\tB}{{\widetilde{B}}}
\newcommand{\tC}{{\widetilde{C}}}
\newcommand{\tD}{{\widetilde{D}}}
\newcommand{\tE}{{\widetilde{E}}}
\newcommand{\tF}{{\widetilde{F}}}
\newcommand{\tG}{{\widetilde{G}}}
\newcommand{\tH}{{\widetilde{H}}}
\newcommand{\tI}{{\widetilde{I}}}
\newcommand{\tJ}{{\widetilde{J}}}
\newcommand{\tK}{{\widetilde{K}}}
\newcommand{\tL}{{\widetilde{L}}}
\newcommand{\tM}{{\widetilde{M}}}
\newcommand{\tN}{{\widetilde{N}}}
\newcommand{\tO}{{\widetilde{O}}}
\newcommand{\tP}{{\widetilde{P}}}
\newcommand{\tQ}{{\widetilde{Q}}}
\newcommand{\tR}{{\widetilde{R}}}
\newcommand{\tS}{{\widetilde{S}}}
\newcommand{\tT}{{\widetilde{T}}}
\newcommand{\tU}{{\widetilde{U}}}
\newcommand{\tV}{{\widetilde{V}}}
\newcommand{\tW}{{\widetilde{W}}}
\newcommand{\tX}{{\widetilde{X}}}
\newcommand{\tY}{{\widetilde{Y}}}
\newcommand{\tZ}{{\widetilde{Z}}}


\newcommand{\wh}{\widehat}
\newcommand{\wt}{\widetilde}
\newcommand{\whp}{\rm w.h.p.}
\newcommand{\wpal}{\rm with probability at least~}
\newcommand{\pc}{Planted Clique\xspace}


\newcommand{\calA}{{\mathcal{A}}}
\newcommand{\calB}{{\mathcal{B}}}
\newcommand{\calC}{{\mathcal{C}}}
\newcommand{\calD}{{\mathcal{D}}}
\newcommand{\calE}{{\mathcal{E}}}
\newcommand{\calF}{{\mathcal{F}}}
\newcommand{\calG}{{\mathcal{G}}}
\newcommand{\calH}{{\mathcal{H}}}
\newcommand{\calI}{{\mathcal{I}}}
\newcommand{\calJ}{{\mathcal{J}}}
\newcommand{\calK}{{\mathcal{K}}}
\newcommand{\calL}{{\mathcal{L}}}
\newcommand{\calM}{{\mathcal{M}}}
\newcommand{\calN}{{\mathcal{N}}}
\newcommand{\calO}{{\mathcal{O}}}
\newcommand{\calP}{{\mathcal{P}}}
\newcommand{\calQ}{{\mathcal{Q}}}
\newcommand{\calR}{{\mathcal{R}}}
\newcommand{\calS}{{\mathcal{S}}}
\newcommand{\calT}{{\mathcal{T}}}
\newcommand{\calU}{{\mathcal{U}}}
\newcommand{\calV}{{\mathcal{V}}}
\newcommand{\calW}{{\mathcal{W}}}
\newcommand{\calX}{{\mathcal{X}}}
\newcommand{\calY}{{\mathcal{Y}}}
\newcommand{\calZ}{{\mathcal{Z}}}

\newcommand{\comp}[1]{{#1^{\rm c}}}
\newcommand{\Leb}{{\rm Leb}}
\newcommand{\Th}{{\rm th}}

\newcommand{\PDS}{{\sf PDS}\xspace}
\newcommand{\PC}{{\sf PC}\xspace}
\newcommand{\BPDS}{{\sf BPDS}\xspace}
\newcommand{\BPC}{{\sf BPC}\xspace}
\newcommand{\DKS}{{\sf DKS}\xspace}
\newcommand{\ML}{{\rm ML}\xspace}
\newcommand{\SDP}{{\rm SDP}\xspace}
\newcommand{\SBM}{{\sf SBM}\xspace}

\newcommand{\ER}{Erd\H{o}s-R\'enyi\xspace}

\newcommand{\Tr}{\mathsf{Tr}}
\renewcommand{\hat}{\widehat}
\renewcommand{\tilde}{\widetilde}

\newcommand{\MMSE}{{\rm MMSE}}
\newcommand{\snr}{{\mathsf{snr}}}

\newcommand{\tsigma}{\tilde{\sigma}}

\newcommand{\planted}{\sigma^*}
\newcommand{\score}{\calT}

%\usepackage{mleftright}

  \begin{document}
	


\title{Statistical Problems with Planted Structures: Information-Theoretical and Computational Limits}

\date{\today}
\author{ 
Yihong Wu \and Jiaming Xu\thanks{
%This research was supported by the National Science Foundation under
%Grant ECCS 10-28464, IIS-1447879, and CCF-1423088, and
%Strategic Research
%Initiative on Big-Data Analytics of the College of Engineering
%at the University of Illinois, and DOD ONR Grant N00014-14-1-0823, and Grant 328025 from the Simons Foundation. 
Y.~Wu is with Department of Statistics and Data Science, 
Yale University, New Haven, CT 06520, USA, \texttt{yihong.wu@yale.edu}.
J.~Xu is with the Fuqua School of Business, Duke University,
Durham, NC 27708, \texttt{jiaming.xu868@duke.edu}.}
%Krannert School of Management, Purdue University,
%       West Lafayette, IN 47907, USA, \texttt{xu972@purdue.edu}.}
}
  
  \maketitle

\begin{abstract}
Over the past few years, insights from computer science, statistical physics, and information theory have revealed phase transitions in a wide array of high-dimensional statistical problems at two distinct thresholds: One is the information-theoretical (IT) threshold below which the observation is too noisy so that inference of the ground truth structure is impossible regardless of the computational cost; the other is the computational threshold above which inference can be performed efficiently, i.e., in time that is polynomial in the input size. In the intermediate regime, inference is information-theoretically possible, but conjectured to be computationally hard.

This article provides a survey of the common techniques for determining the sharp IT and computational limits, using community detection and submatrix detection as illustrating examples. For IT limits, we discuss tools including the first and second moment method for analyzing the maximum likelihood estimator, information-theoretic methods for proving impossibility results using  mutual information and rate-distortion theory, and methods originated from statistical physics such as interpolation method. To investigate computational limits, we describe a common recipe to construct a randomized polynomial-time reduction scheme that approximately maps instances of the planted clique problem to the problem of interest in total variation distance.
\end{abstract}

\tableofcontents
  
 \pdfoutput=1
\documentclass{article}
\usepackage[final]{pdfpages}
\begin{document}
\includepdf[pages=1-9]{CVPR18VOlearner.pdf}
\includepdf[pages=1-last]{supp.pdf}
\end{document}

\begin{appendices}

\section{Mutual information-characterization of correlated recovery}
\label{app:MIcorr}

\newcommand{\Unif}{\mathrm{Unif}}


We consider a general setup:  Let the number of communities $k$ be a constant. Denote the membership vector by $\sigma=(\sigma_1,\ldots,\sigma_n) \in [k]^n$ and the observation is $A=(A_{ij}: 1 \leq i < j \leq n)$. Assume the following conditions:
\begin{enumerate}[label=A\arabic*]
	%[{A}1.]
	\item \label{A1}
	For any permutation $\pi\in S_k$, $(\sigma,A)$ and $(\pi(\sigma),A)$ are equal in law, where $\pi(\sigma)\triangleq (\pi(\sigma_1),\ldots,\pi(\sigma_n))$; 
	
	\item \label{A2}
	For any $i \neq j \in [n]$, $I(\sigma_i,\sigma_j;A)= I(\sigma_1,\sigma_2;A)$;
	
	\item \label{A3}
	For any $z_1,z_2 \in [k]$, $\prob{\sigma_1=z_1,\sigma_2=z_2} = \frac{1}{k^2} + o(1)$ as $n\to \infty$.
\end{enumerate}
These assumptions are satisfied for example for $k$-community SBM (where each pair of vertices $i$ and $j$ are connected independently with probability $p$ if $\sigma_i=\sigma_j$ and $q$ otherwise),\index{Stochastic block model (SBM)! $k$ communities}
 and the membership vector
$\sigma$ can be either uniformly distributed on $[k]^n$ or the set of equal-sized $k$-partition of $[n]$. 

Recall that correlated recovery entails the following:
For any $\sigma , \hat{\sigma} \in [k]^n$, define the overlap:
\begin{align}
o\left( \sigma, \hat{\sigma} \right) = \frac{1}{n} \max_{\pi \in S_k} 
\sum_{i \in [n] } \left( \indc{ \pi\left(\sigma_i \right) = \hat{\sigma}_i} - \frac{1}{k} \right).
\end{align}
We say an estimator $\hat{\sigma} = \hat{\sigma}(A)$ achieves correlated recovery if\footnote{For the special case of $k=2$, \prettyref{eq:corro} is equivalent to 
$\frac{1}{n}\Expect[|\Iprod{\sigma}{\hat \sigma}|] = \Omega(1)$, where $\sigma , \hat{\sigma}$ are assumed to be $\{\pm\}^n$-valued.}
\begin{equation}
\expect{o\left( \sigma, \hat{\sigma} \right)}=\Omega(1),
\label{eq:corro}
\end{equation}
 that is, the misclassification rate, up to a global permutation, outperforms random guessing.
Under the above three assumptions, we have the following characterization of correlated recovery:
\begin{lemma}
\label{lmm:MIcorr}	
Correlated recovery is possible if and only if 
$I(\sigma_1, \sigma_2 ; A) = \Omega(1)$.	
\end{lemma}



\begin{proof}
We start by recalling the relation between mutual information and total variation.
For any pair of random variables $(X,Y)$, define the so-called $T$-information \cite{Csiszar96}:
$T(X;Y) \triangleq \TV(P_{XY}, P_XP_Y) = \Expect[\TV(P_{Y|X}, P_Y)]$.
For $X \sim \Bern(p)$, this simply reduces to 
\begin{equation}
T(X;Y) = 2p(1-p) \TV(P_{Y|X=0}, P_{Y|X=1}).
\label{eq:Tbern}
\end{equation}
Furthermore, the mutual information can be bounded by the $T$-information, 
by Pinsker's and Fano's inequality, as follows \cite[Eq.~(84) and Prop.~12]{PW14a}
\begin{equation}
 2 T(X;Y)^2 \leq I(X;Y)  \leq \log (M-1) T(X;Y) + h(T(X;Y))		
	\label{eq:TI}
\end{equation}
where in the upper bound $M$ is the number of possible values of $X$, and $h$ is the binary entropy function in \prettyref{eq:binaryentropy}.


We prove the ``if'' part.
Suppose 
%correlated recovery is impossible and for the sake of contradiction,
$I(\sigma_1, \sigma_2; A) = \Omega(1)$.
We first claim that assumption \ref{A1} implies that 
\begin{equation}
I (\indc{\sigma_1 =\sigma_2}; A)=I(\sigma_1, \sigma_2; A)
\label{eq:ss}
\end{equation}
that is,
 $A$ is independent of $\sigma_1,\sigma_2$ conditional on $\indc{\sigma_1 =\sigma_2}$. 
%\nbr{JX. Okay. I think we need this condition to ensure that 
%correlated recovery is impossible implies $I(\sigma_1, \sigma_2; A) = o(1)$. 
%Otherwise, we could have the situation that $A=a$ for $\sigma_1=+1$ and
%$A=b$ for $\sigma_1=-1$. In this case, correlated recovery of $\sigma$ is certainly 
%impossible, but $I(\sigma_1, \sigma_2;A)=I(\sigma_1; A)=1$.}
Indeed, 
for any $z \neq z'\in [k]$, let $\pi$ be any permutation such that 
%$\pi(z)=z'$ and 
$\pi(z')=z.$
%that interchanges $z$ and $z'$. 
Since $P_{\sigma, A} =  P_{\pi(\sigma), A}$, 
we have $P_{A|\sigma_1=z,  \sigma_2=z} =  P_{A| \pi(\sigma_1)=z,  \pi(\sigma_2)=z }$, i.e., 
$P_{A |\sigma_1=z, \sigma_2=z} =  P_{A| \sigma_1 =z',  \sigma_2 =z' }$. 
Similarly, one can show that 
$P_{A|\sigma_1=z_1, \sigma_2=z_2} =  P_{A| \sigma_1 =z_1',  \sigma_2 =z_2'}$, 
for any $z_1 \neq z_2$ and $z_1'\neq z_2'$, and this proves the claim.
%Since $\calL(\sigma, A) =  \calL(\pi(\sigma), A)$, where $\calL(\cdot)$ denote the law, 
%we have $\calL(A|\sigma_1=z,  \sigma_2=z) =  \calL(A| \pi(\sigma_1)=i,  pi(\sigma_2)=i )$, i.e., 
%$\calL(A|\sigma_1=z, \sigma_2=z) =  L(A| \sigma_1 =z',  \sigma_2 =z' )$. 
%Similarly, one can show that 
%$\calL(A|\sigma_1=z_1, \sigma_2=z_2) =  L(A| \sigma_1 =z_1',  \sigma_2 =z_2')$, for any $z_1 \neq z_2$ and $z_1'\neq z_2'$, and this proves the claim.

Let $x_j=\indc{\sigma_1 =\sigma_j}$.
By the symmetry assumption \ref{A2}, 
$I(x_j; A) = I(x_2; A) = \Omega(1)$ for all $j \neq 1$.
Since $\prob{x_j = 1} = \frac{1}{k} + o(1)$ by assumption \ref{A3}, applying \prettyref{eq:TI} with $M=2$ and in view of \prettyref{eq:Tbern}, we have
$\TV(P_{A|x_j=0},P_{A|x_j=1})=\Omega(1)$.
Thus, there exists an estimator $\widehat{x}_j \in \{0,1\}$ as a function of $A$, such that
\begin{align}
\prob{\widehat{x}_j = 1 \mid x_j =1 } + 
\prob{\widehat{x}_j = 0 \mid x_j =0 } \ge 1+\TV(P_{A|x_j=0},P_{A|x_j=1})
= 1+ \Omega(1).  \label{eq:estimator_x_hat}
\end{align}

Define $\hat\sigma$ as follows: set $\hat\sigma_1= 1 $; for $j \neq 1$, set $\hat \sigma_j = 1 $ if $\widehat{x}_j = 1$ 
and draw $\hat \sigma_j $ from $\{2,\ldots,k\}$ uniformly at random 
if $\widehat{x}_j = 0$.
Next, we show that $\hat\sigma$ achieves correlated recovery. 
Indeed, fix a permutation $\pi \in S_k$ such that $\pi(\sigma_1)=1$. It follows from 
the definition of overlap that
\begin{equation}
\Expect[o\left(\sigma, \hat{\sigma} \right)]
\ge \frac{1}{n} \sum_{j \neq 2} \prob{\pi(\sigma_j) =\hat\sigma_j} - \frac{1}{k}.
\label{eq:olb}
\end{equation}
Furthermore, since $\pi(\sigma_1)=1$, we have, for any $j\neq 1$,
\[
\prob{\pi(\sigma_j) = \hat\sigma_j,x_j=1}=
\prob{\hat x_j=1,x_j=1}
\]
and
\[
\prob{\pi(\sigma_j) = \hat\sigma_j,x_j=0}=
\prob{\pi(\sigma_j) = \hat\sigma_j, \hat x_j=0,x_j=0}=
\frac{1}{k-1} \prob{\hat x_j=0,x_j=0},
\]
where the last step is because conditional on $\hat{x}_j=0$,
$\hat{\sigma}_j$ is chosen  from $\{2,\ldots,k\}$ uniformly and 
independently of everything else.
Since $\prob{x_j = 1} = \frac{1}{k} + o(1)$, we have
\[
\prob{\pi(\sigma_j) =\hat\sigma_j} = \frac{1}{k}(\prob{\widehat{x}_j = 1 \mid x_j =1 } + 
\prob{\widehat{x}_j = 0 \mid x_j =0 }) +o(1) \overset{\prettyref{eq:estimator_x_hat}}{\ge} \frac{1}{k}+ \Omega(1).  
\]
By \prettyref{eq:olb}, we conclude that $\hat{\sigma}$ achieves correlated recovery
of $\sigma$.

Next we prove the ``only if'' part.
Suppose $I(\sigma_1, \sigma_2; A) = o(1)$ and we aim to show 
%the impossibility of correlated recovery, that is, 
$\expect{o\left(\sigma, \hat{\sigma} \right)}=o(1)$ for any estimator $\hat\sigma$.
By the definition of overlap, we have
\begin{align*}
o\left(\sigma, \hat{\sigma} \right) 
\le 
\frac{1}{n} \sum_{\pi \in S_k}
\left| \sum_{i \in [n] }  
\left( \indc{ \pi\left(\sigma_i \right) = \hat{\sigma}_i} - \frac{1}{k}  \right) \right|.
%& \le \frac{1}{n} \sum_{\pi \in S_k}
%\sum_{\ell=1}^k 
%\left| \sum_{i \in [n] } \left( \indc{ \pi\left(\sigma_i \right) =\ell} 
%\indc{ \hat{\sigma}_i = \ell} - \frac{1}{k^2} \right)\right|.
\end{align*}
Since there are $k!=\Omega(1)$ permutations in $S_k$, it suffices to show
for any fixed permutation $\pi$,
$$
\expect{ \left| \sum_{i \in [n] } \left( \indc{ \pi\left(\sigma_i \right) = \hat{\sigma}_i} - \frac{1}{k}  \right) 
\right| } = o(n).
$$
Since $I(\pi(\sigma_i), \pi(\sigma_j); A)=I(\sigma_i, \sigma_j; A)$, without loss of generality, 
we assume $\pi=\text{id}$ in the following. By the Cauchy-Schwarz inequality, it further suffices to show
\begin{equation}
\expect{ \left( \sum_{i \in [n] } \left( \indc{ \sigma_i =\hat{\sigma}_i } - \frac{1}{k} \right)\right)^2 } =o(n^2).
\label{eq:overlap2}
\end{equation}
Note that 
\begin{align*}
& \expect{  \left( \sum_{i \in [n] } \left(
   \indc{ \sigma_i  = \hat{\sigma}_i } - \frac{1}{k} \right)   \right)^2 } \\
 & = \sum_{i, j \in [n] } 
 \expect{ \left( \indc{ \sigma_i = \hat{\sigma}_i }- \frac{1}{k}  \right) 
 \left( \indc{ \sigma_j  = \hat{\sigma}_j }- \frac{1}{k}  \right) }  \\
 & =  \sum_{i, j \in [n] }  \prob{  \sigma_i = \hat{\sigma}_i , \sigma_j  = \hat{\sigma}_j }
 - \frac{2n}{k} \sum_{i \in [n] }\prob{ \sigma_i = \hat{\sigma}_i } + \frac{n^2}{k^2}.
\end{align*}
For the first term in the last displayed equation, 
let $\sigma'$ be identically distributed as $\hat\sigma$ but independent of $\sigma$.
Since $I(\sigma_i,\sigma_j;\hat\sigma_i,\hat\sigma_j) \le I(\sigma_i,\sigma_j;A)=o(1)$ by the data processing inequality, it follows from the lower bound in 
\prettyref{eq:TI} that $\TV(P_{\sigma_i,\sigma_j,\hat\sigma_i,\hat\sigma_j}, P_{\sigma_i,\sigma_j,\sigma_i',\sigma_j'})=o(1)$.
Since 
$\pprob{  \sigma_i = {\sigma}_i', \sigma_j  = {\sigma}_j' } \leq 
\max_{a,b \in [k]} \prob{  \sigma_i = a, \sigma_j  = b} \leq \frac{1}{k^2}+o(1)$ by assumption \ref{A3}, 
we have
$$
\prob{  \sigma_i = \hat{\sigma}_i , \sigma_j  = \hat{\sigma}_j }
\leq \frac{1}{k^2} + o(1),
$$
Similarly, for the second term, we have
$$
\prob{ \sigma_i = \hat{\sigma}_i } = \frac{1}{k} +o(1),
$$
where the last equality holds due to $I(\sigma_i; A) =o(1).$
Combining the last three displayed equations gives \prettyref{eq:overlap2} and completes the proof.
\end{proof}


%The argument below is essentially contained in \cite{PW18}.
%
%We first consider $k=2$, in which case it is convenient to assume $\sigma \in \{\pm\}^n$. Recall that correlated recovery amounts to reconstruct $\sigma$ (up to a global sign flip) better than chance, that is, find $\hat \sigma=\hat \sigma(Y) \in \{\pm1\}^n$, such that
%\begin{equation}
%%\liminf_{n \to\infty} \frac{1}{n}\Expect[|\Iprod{\sigma}{\hat \sigma}|] > 0.
%\frac{1}{n}\Expect[|\Iprod{\sigma}{\hat \sigma}|] = \Omega(1).
%%> \epsilon
%\label{eq:corr}
%\end{equation}
%%for some constant $\epsilon$.
%Note that by symmetry, for any $i\neq j$, $I(\sigma_i,\sigma_j;A) = I(\sigma_i \sigma_j; A) = I(\sigma_1 \sigma_2; A)$. Suppose $I(\sigma_1 \sigma_2; A) = \Omega(1)$, then for any $j \neq 1$, there exists an estimator $\widehat{\sigma_1\sigma_j}$ as a function of $A$, such that
%$\prob{\widehat{\sigma_1\sigma_j} \neq \sigma_1\sigma_j} \leq \frac{1}{2} - \Omega(1)$. Thus, setting $\hat\sigma$ according to $\hat\sigma_1=+$ and $\hat \sigma_j = \hat{\sigma_1\sigma_j}$ achieves \prettyref{eq:corr}. 
%This shows \prettyref{eq:MIcorr2} is necessary for the impossibility of correlated recovery. 
%%To show \prettyref{eq:MIcorr2} implies the impossibility of correlated recovery,
%To prove the sufficiency, 
%for any estimator $\hat \sigma= \hat \sigma(A) \in \{\pm\}^n$, 
%since $I(\sigma_i \sigma_j; A)=o(1)$, which is equivalent to 
%$\TV(P_{A|\sigma_i \sigma_j=+}, P_{A|\sigma_i \sigma_j=-})=o(1)$, we have 
%$\prob{\sigma_i\sigma_j \neq \hat \sigma_i\hat \sigma_j} \geq \frac{1}{2}-o(1)$. 
%On the other hand, we have the equality:
%\begin{align*}
%2n^2 - 2 \eexpect{\iprod{\sigma}{\hat\sigma}^2} 
%= & ~ \Expect\Fnorm{\sigma \sigma^\top - \hat \sigma\hat \sigma^\top}^2 \\
%= & ~ 4 \sum_{i \neq j} \prob{\sigma_i\sigma_j \neq \hat \sigma_i\hat \sigma_j} = 2n^2-o(n^2).
%\end{align*}
%Thus, $\eexpect{\iprod{\sigma}{\hat\sigma}^2} =o(n^2)$, which implies $\eexpect{|\Iprod{\sigma}{\hat\sigma}|} =o(n)$.
%
%Next we consider $k \geq 3$, in which case correlated recovery is achieved if there exists an estimator $\hat \sigma \in [k]^n$ that outperforms random guessing, i.e., 
%\begin{equation}
%\Expect[d(\sigma,\hat \sigma)] \leq \frac{k-1}{k} - \Omega(1).
%\label{eq:corr-sbmk}
%\end{equation}
%Here the loss function $d$ is the fraction of classification errors up to a global permutation of labels, formally defined as follows: for any $\sigma,\hat \sigma \in [k]^n$, 
%\begin{equation}
%d(\sigma,\hat \sigma) \triangleq  \min_{\pi \in S_k} \frac{1}{n}\sum_{i\in[n]} \indc{\sigma_i \neq \pi(\hat \sigma_i)}
%\label{eq:lossd}
%\end{equation}
%where $S_k$ is the collection of permutations on $[k]$.
%
%%By symmetry, \prettyref{eq:MIcorrk} implies that for any fixed $m$, as $n\diverge$,
%%\begin{equation}
%%I(\sigma_S; A) = o(1)
%%\label{eq:IXSY}
%%\end{equation}
%%for any $S \in \binom{[n]}{m}$, where 
%To show the impossibility of correlated recovery on the basis of \prettyref{eq:MIcorrk}, first of all, note that for any fixed $x,\hat x\in[k]^n$ and any $m\in [n]$ we have
%\begin{equation}
%d(x,\hat x)
%\geq \Expect_{S}[d(x_{\sfS},\hat x_{\sfS})] \label{eq:davg}
%\end{equation}
%where ${\sfS}  \sim \Unif(\binom{[n]}{m})$ and recall that for any $S$, we have 
%$d(x_S,\hat x_S) = \frac{1}{|S|}  \min_{\pi \in S_k} \sum_{i\in S} \indc{x_i \neq \pi(\hat x_i)}$ per \prettyref{eq:lossd}. The inequality \prettyref{eq:davg} simply follows from
%\begin{align}
%d(x,\hat x)
%= & ~ \min_{\pi \in S_k} \probs{x_I \neq \hat x_{\pi(I)}}{I \sim \Unif([n])}	\nonumber \\
%= & ~ \min_{\pi \in S_k} \Expect_{\sfS \sim \Unif(\binom{[n]}{m})} \probs{x_I \neq \hat x_{\pi(I)}}{I \sim \Unif({\sfS})}	\nonumber \\
%= & ~ \Expect_{{\sfS}} \min_{\pi \in S_k} \probs{x_I \neq \hat x_{\pi(I)}}{I \sim \Unif({\sfS})}	\nonumber \\
%\geq & ~ \Expect_{{\sfS}} [d(x_{\sfS},\hat x_{\sfS})] \nonumber.
%\end{align}
%%Fix a constant $m$ independent of $n$. 
%For any estimator $\hat \sigma=\hat \sigma(Y) \in [k]^n$, applying \prettyref{eq:davg} yields
%\begin{equation}
%\Expect[d(\sigma_{\sfS},\hat \sigma_{\sfS})] \leq \Expect[d(\sigma,\hat \sigma)], \label{eq:davg2}
%\end{equation}
%where ${\sfS}$ is a random uniform $m$-set independent of $\sigma,\hat \sigma$.
%
%
%By the data processing inequality, we have for any $S \in \binom{[n]}{m}$,
%\[
%I(\sigma_S; \hat \sigma_S) \leq I(\sigma_S; A) = I(\sigma_1,\ldots,\sigma_m; A) \overset{\prettyref{eq:MIcorrk}}{=} o(1),
%\]
%as $n\diverge$.
%By Pinsker's inequality, we have 
%$\TV(P_{\sigma_S, \hat \sigma_S}, P_{\sigma_S} \otimes P_{\hat \sigma_S}) \leq \sqrt{2 I(\sigma_S; \hat \sigma_S)} = o(1)$.
%Since the loss function $d$ defined in \prettyref{eq:lossd} is bounded by one, we have
%\begin{equation}
%\Expect[d(\sigma_S,\hat \sigma_S)] \geq \Expect[d(\sigma_S,\sigma'_S)] - \TV(P_{\sigma_S, \hat \sigma_S}, P_{\sigma_S} \otimes P_{\hat \sigma_S}) 
%= \Expect[d(\sigma_S,\sigma'_S)] + o(1), 
%\label{eq:ddTV}
%\end{equation}
%where $\sigma'_S$ has the same marginal distribution as $\hat \sigma_S$ but independent of $\sigma_S$.
%By \prettyref{lmm:randomguess} below, we have
%\begin{equation}
%\Expect[d(\sigma_S,\sigma'_S)] \geq \pth{\frac{k-1}{k} - m^{-1/3}}(1-k! e^{-2m^{1/3}}).
%\label{eq:randomguess2}
%\end{equation}
%Averaging \prettyref{eq:randomguess2} over $S \in \binom{[n]}{m}$ then combining with \prettyref{eq:davg2} and \prettyref{eq:ddTV}, and finally sending $n\to\infty$ followed by $m \to \infty$, we conclude that $\eexpect{ d(\sigma,\hat \sigma)} \geq \frac{k-1}{k}-o(1)$, hence the impossibility of correlated recovery.
%
%
%\begin{lemma}
%\label{lmm:randomguess}	
	%Let $\sigma$ be uniformly distributed on $[k]^m$ and $\sigma'$ is independent of $\sigma$ with an arbitrary distribution on $[k]^m$. 
%For the loss function in \prettyref{eq:lossd}, we have\footnote{Note that for any fixed $k,m$ and any string $x,z\in [k]^m$, we can always outperform random matching, i.e., $d(x,z) < \frac{k-1}{k}$. The point of \prettyref{eq:randomguess} is that this improvement is negligible for large $m$.}
	%\begin{equation}
	%d(\sigma,\sigma') \geq \frac{k-1}{k} - m^{-1/3}
	%\label{eq:randomguess}
	%\end{equation}
	%with probability at least $1-(k! e^{-2m^{1/3}})$.
%\end{lemma}
%\begin{proof}
	%For each fixed $\pi$, the Hamming distance $d_H(\sigma,\pi(\sigma'))\sim \Binom(m,\frac{k-1}{k})$. From Hoeffding's inequality we have
	%$$ \Prob[d_H(\sigma,\pi(\sigma') < {k-1\over k} - \delta] \le e^{-2m \delta^2}\,,$$
	%and from the union bound
	%$$ \Prob[\min_\pi d_H(\sigma,\pi(\sigma') < {k-1\over k} - \delta] \le k! e^{-2m \delta^2}\,.$$
	%Setting $\delta = m^{-1/3}$ completes the proof.
%\end{proof}



%\section{Proof of \prettyref{eq:second_moment_conditional} $\implies$ \prettyref{eq:MI_TV} and verification in the binary symmetric SBM}
%\label{app:MITV}
%Let $S=[m]$ and denote $\sigma_1,\ldots,\sigma_m$ by $\sigma_S$.
%Recall that $m$ is a constant and $\P_z \triangleq \P_{A|\sigma_S=z}$ for $z\in[k]^m$.
%We first prove the following:
%%\begin{align}
%%I(\sigma_S; A) = o(1) & \Leftrightarrow D \left( \P_{A | \sigma_S} \| \P \right) =o(1), \quad \forall \sigma_S  \nonumber \\
%%& \Leftrightarrow d_{\rm TV} \left( \P_{A | \sigma_S}, \P \right) =o(1), \quad \forall \sigma_S. \nonumber \\
%%& \Leftarrow d_{\rm TV} \left( \P_{A | \sigma_S}, \P_{A | \tsigma_S} \right) =o(1), \quad \forall \sigma_S, \tsigma_S,
%%\label{eq:MI_TV}
%%\end{align}
%\begin{align}
%I(\sigma_S; A) = o(1) & \Leftrightarrow D \left( \P_z \| \P \right) =o(1), \quad \forall z, \label{eq:MI_TV1}\\
%& \Leftrightarrow d_{\rm TV} \left( \P_z, \P \right) =o(1), \quad \forall z, \label{eq:MI_TV2}\\
%& \Leftarrow d_{\rm TV} \left( \P_z, \P_{\tz} \right) =o(1), \quad \forall z, \tz. \label{eq:MI_TV}
%\end{align}
%For \prettyref{eq:MI_TV1}, by definition,
%$$
%I(\sigma_S; A) =  \Expect_{\sigma_S} \qth{ D \left( \P_{A|\sigma_S} \| \P \right) }.
%$$
%Note that the distribution of $\sigma_S$ has a finite support
%and $\Omega(1)$ probability mass on each possible value. 
%Therefore, $I(\sigma_S; A)=o(1)$ if and only if 
%$D \left( \P_{A|\sigma_S=z} \| \P \right) = o(1)$ for all $z$.
%
%For \prettyref{eq:MI_TV2}, by Pinsker's inequality, $D \left( \P_{z} \| \P \right) = o(1)$
%implies $d_{\rm TV} \left( \P_{z}, \P \right) = o(1)$. Conversely, 
%suppose that $d_{\rm TV} \left( \P_{z}, \P \right) = o(1)$. Then
%\begin{align*}
%D \left( \P_{z} \| \P \right) &= \int \P_{z} \log \frac{\P_{z}  }{\P} \\
%& \overset{(a)}{\le} \int  \P_{z}  \frac{\P_{z} - \P }{\P} \\
%&  \le \int  \frac{ \P_{z} }{ \P } \left| \P_{z} - \P  \right| \\
%& \overset{(b)}{=} O(1) \times \int \left| \P_{z} - \P  \right| \\
%& = O\left( d_{TV}  \left( \P_{z}, \P \right) \right) = o(1),
%\end{align*}
%where $(a)$ is due to $\log x \le x-1$, and $(b)$ follows because 
%$\frac{ \P_{z}(a) }{ \P(a) } = \frac{ \P_{A|\sigma_S=z}(a) }{ \P_A(a)} \leq \frac{1}{\prob{\sigma_S=z}} = O(1)$ everywhere.
%%, and  the distribution of $\sigma_S$ has $\Omega(1)$ probability mass on each realization and hence 
%
%
%For \prettyref{eq:MI_TV}, suppose that $d_{\rm TV} \left( \P_{z}, \P_{\tz} \right) =o(1)$
%for all $z$ and $\tz$. By the convexity of $d_{\rm TV}(\cdot,\cdot)$ and Jensen's inequality, it readily
%follows that $d_{\rm TV} \left( \P_{z}, \P \right) =o(1)$.
%
%
%
%Next we prove that for any reference distribution $\Q$,  \prettyref{eq:second_moment_conditional} implies 
%$d_{\rm TV} \left( \P_{z}, \P_{\tz} \right) =o(1)$
%for all $z,\tz$, which further implies \prettyref{eq:MIcorrk} in view of \prettyref{eq:MI_TV}.
 %Indeed, by Cauchy-Schwartz inequality, we have
%\begin{align}
%d_{\rm TV}  \left( \P_{z}, \P_{\tz} \right) & =
%\frac{1}{2} \int  \left| \P_{z} - \P_{ \tz} \right| \nonumber \\
%%& = \frac{1}{2} \int  \left| \P_{z} - \P_{\tz} \right| \frac{\sqrt{\Q}}{\sqrt{\Q}} \nonumber \\
%& \le \frac{1}{2}  \left( \int \Q  \right)^{1/2} \left( \int  \frac{  \left( \P_{z} - \P_{\tz} \right)^2 }{\Q }  \right)^{1/2}
%\nonumber \\
%& = \frac{1}{2} \left(    \int \frac{\P^2_{A | z}}{\Q} +\int \frac{\P^2_{A | \tz}}{\Q} - 2 \int \frac{ 
%\P_{z} \P_{\tz} }{ \Q} \right)^{1/2} \overset{\prettyref{eq:second_moment_conditional}}{=} o(1).
%\end{align}
%%where the last equality holds from the assumption \prettyref{eq:second_moment_conditional}. 
%
%
%
%Finally, we consider the binary symmetric SBM and show that,
%below the correlated recovery threshold $\tau=\frac{(a-b)^2}{2(a+b)}<1$, 
 %\prettyref{eq:second_moment_conditional} is satisfied if the reference distribution $\Q$ is the distribution of $A$ in
%the null (\ER) model.
%Specifically, following the derivations in \prettyref{eq:SBM_second_moment_eq},
%we have
%\begin{align}
%\int \frac{   \P_{z}  \P_{\tz} }{ \Q } 
%&= \Expect \qth{  \prod_{i < j} \left(1 +  \sigma_i \sigma_j \tsigma_i \tsigma_j \rho \right)
%\mathrel{\bigg|} \sigma_S=z, \tsigma_S =\tz}  \nonumber \\
%& =  \left( 1+o(1) \right) e^{ -\tau^2/4 -\tau/2} \times 
%\Expect \qth{ \exp \left( \frac{\rho}{2} \iprod{ \sigma}{\tsigma}^2  \right) \mathrel{\Big|} \sigma_S=z, \tsigma_S =\tz},
%%& = \prod_{(i,j) \in \calE(S) } \left(1 +  \sigma_i \sigma_j \tsigma_i \tsigma_j \rho \right)  \nonumber \\
%%& \times \Expect \qth{ \prod_{i \in S, j \in S^c}  \left(1 +  \sigma_i \sigma_j \tsigma_i \tsigma_j \rho \right) \mid \sigma_S, \tsigma_S } \nonumber \\
%%& \times \Expect \qth{ \prod_{(i,j)\in \calE(S^c)}  \left(1 +  \sigma_i \sigma_j \tsigma_i \tsigma_j \rho \right) \mid \sigma_S, \tsigma_S }.
%\end{align}
%where the last equality holds because $m$ is a constant, $\rho=\tau/n + O(1/n^2)$ and $\log(1+x) = x -x^2/2 +O(x^3)$. 
%%\nbr{
%%I got $e^{ -\tau^2/4 - \tau/2}$, because 
%%$\sum_{i < j} \sigma_i \sigma_j \tsigma_i \tsigma_j = \frac{1}{2}(
%%\iprod{ \sigma}{\tsigma}^2 - n)$.}
%
%Write $\sigma=2\xi-1$ for $\xi\in\{0,1\}^n$ and let 
%$$
%H_1\triangleq \Iprod{\xi_{S}} { \tilde{\xi}_{S} } \quad \text{ and } \quad 
%H_2\triangleq \Iprod{\xi_{S^c}} { \tilde{\xi}_{S^c} }.
%$$ 
%Then $\iprod{\sigma}{\tsigma} = 4 (H_1+H_2) -n$.
%Moreover, conditional  on $\sigma_S$ and $\tsigma_S$, 
%$$
%H_2 \sim \text{Hypergeometric} \left( n-m, n/2 - \| \xi_S\|_1, n/2 - \|\tilde{\xi}_S \|_1 \right).
%$$
%Therefore
%\begin{align*}
%\Expect \qth{ \exp \left(  \frac{\rho}{2} \iprod{ \sigma}{\tsigma}^2  \right) \mathrel{\bigg|} \sigma_S=z, \tsigma_S =\tz}
%& =\expect{ \exp \left(  \frac{n\rho}{2} \left( \frac{ 4 H_1 + 4H_2 - n}{\sqrt{n} } \right)^2 \right) \mathrel{\bigg|} \sigma_S=z, \tsigma_S =\tz}  \\
%& = \frac{1+o(1)}{\sqrt{1-\tau}},
%\end{align*}
%where the last inequality holds because $n \rho = \tau +o(1/n)$ and 
%conditional on $\sigma_S$ and $\tsigma_S$,
%$ \frac{1}{\sqrt{n}} ( 4H_1 + 4H_2 - n )$ converges to $\calN(0,1)$ in distribution 
%as $n \to \infty$ by the central limit theorem for hypergeometric distribution. 
%
%In conclusion, we have shown that 
%$$
%\int \frac{   \P_{z}  \P_{\tz} }{ \Q}  
%= \left( 1+o(1) \right) e^{ -\tau^2/4 - \tau/2}  \frac{1}{\sqrt{1-\tau}}, \quad \forall z,\tz.
%$$
%%Hence, by taking the expectation of $\sigma_S$ and $\tsigma_S$ over the both hand sides of the last displayed equation, we get that
%Averaging both sides over $z,\tz$ yields
%$$
%\int \frac{   \P^2 }{ \Q}  
%= \left( 1+o(1) \right) e^{ -\tau^2/4 - \tau/2}  \frac{1}{\sqrt{1-\tau}}.
%$$
%Thus \prettyref{eq:second_moment_conditional} is satisfied. 




\section{Proof of \prettyref{eq:second_moment_conditional} $\implies$ \prettyref{eq:MIcorr2} and 
verification of  \prettyref{eq:second_moment_conditional} in the binary symmetric SBM}
\label{app:MITV}
Combining \prettyref{eq:ss} with 
\prettyref{eq:TI} and \prettyref{eq:Tbern}, we have
$I(\sigma_1,\sigma_2; A) = o(1)$ if and only if $\TV(\P_+,\P_-) =o(1)$,
where $\P_+=P_{A|\sigma_1=\sigma_2}$ and $\P_-=P_{A|\sigma_1\neq\sigma_2}$.
%Next we prove that for any reference distribution $\Q$,  \prettyref{eq:second_moment_conditional} implies 
%$d_{\rm TV} \left( \P_{+}, \P_{-} \right) =o(1)$.
Note the following characterization about the total variation distance, which simply follows from the Cauchy-Schwartz inequality:
\begin{equation}
\TV(\P_+,\P_-) = \frac{1}{2} \sqrt{\inf_{\Q}  \int  \frac{  \left( \P_{+} - \P_{-} \right)^2 }{\Q }}
\label{eq:TVquadratic}
\end{equation}
where the infimum is taken over all probability distributions $\Q$. 
Therefore \prettyref{eq:second_moment_conditional} implies \prettyref{eq:MIcorr2}.

%
 %Indeed, by Cauchy-Schwartz inequality, for any $\Q$, we have
%\begin{align}
%d_{\rm TV}  \left( \P_{+}, \P_{-} \right) & =
%\frac{1}{2} \int  \left| \P_{+} - \P_{-} \right| \nonumber \\
%%& = \frac{1}{2} \int  \left| \P_{z} - \P_{\tz} \right| \frac{\sqrt{\Q}}{\sqrt{\Q}} \nonumber \\
%& \le \frac{1}{2}  
%\left( \int \Q  \right)^{1/2} \left( \int  \frac{  \left( \P_{+} - \P_{-} \right)^2 }{\Q }  \right)^{1/2}
%\nonumber \\
%%& = \frac{1}{2} \left(    \int \frac{\P^2_{+}}{\Q} +\int \frac{\P^2_{-}}{\Q} - 2 \int \frac{ 
%%\P_{+} \P_{-} }{ \Q} \right)^{1/2} 
%&\overset{\prettyref{eq:second_moment_conditional}}{=} o(1).
%\end{align}
%%where the last equality holds from the assumption \prettyref{eq:second_moment_conditional}. 



Finally, we consider the binary symmetric SBM and show that,
below the correlated recovery threshold $\tau=\frac{(a-b)^2}{2(a+b)}<1$, 
 \prettyref{eq:second_moment_conditional} is satisfied if the reference distribution $\Q$ is the distribution of $A$ in
the null (\ER) model. Note that 
$$
\int  \frac{  \left( \P_{+} - \P_{-} \right)^2 }{\Q }  =
\int \frac{\P^2_{+}}{\Q} +\int \frac{\P^2_{-}}{\Q} - 2 \int \frac{ \P_{+} \P_{-} }{ \Q}.
$$
%Hence, it is sufficient to show
%$$
 %\int \frac{ \P^2 }{ \Q} = O(1), \; \text{ and } 
 %\int \frac{ \P_{z} \P_{\tilde{z} } }{ \Q} =  
%(1+o(1)) \int \frac{ \P^2 }{ \Q}, \quad  \forall z, \tilde{z} \in \{\pm \}.
%$$
Hence, it is sufficient to show
$$
 \int \frac{ \P_{z} \P_{\tilde{z} } }{ \Q} =  
C+o(1), \quad  \forall z, \tilde{z} \in \{\pm \}
$$
for some constant $C$ independent of $z$ and $\tz$.
Specifically, following the derivations in \prettyref{eq:SBM_second_moment_eq},
we have
\begin{align}
\int \frac{   \P_{z}  \P_{\tz} }{ \Q } 
&= \Expect \qth{  \prod_{i < j} \left(1 +  \sigma_i \sigma_j \tsigma_i \tsigma_j \rho \right)
\mathrel{\bigg|} \sigma_1 \sigma_2=z, \tsigma_1 \tsigma_2 =\tz }  \nonumber \\
& =  \left( 1+o(1) \right) e^{ -\tau^2/4 -\tau/2} \times 
\Expect \qth{ \exp \left( \frac{\rho}{2} \iprod{ \sigma}{\tsigma}^2  \right) \mathrel{\Big|} \sigma_1 \sigma_2=z, \tsigma_1 \tsigma_2 =\tz },
%& = \prod_{(i,j) \in \calE(S) } \left(1 +  \sigma_i \sigma_j \tsigma_i \tsigma_j \rho \right)  \nonumber \\
%& \times \Expect \qth{ \prod_{i \in S, j \in S^c}  \left(1 +  \sigma_i \sigma_j \tsigma_i \tsigma_j \rho \right) \mid \sigma_S, \tsigma_S } \nonumber \\
%& \times \Expect \qth{ \prod_{(i,j)\in \calE(S^c)}  \left(1 +  \sigma_i \sigma_j \tsigma_i \tsigma_j \rho \right) \mid \sigma_S, \tsigma_S }.
\end{align}
where the last equality holds $\rho=\tau/n + O(1/n^2)$ and $\log(1+x) = x -x^2/2 +O(x^3)$. 
%\nbr{
%I got $e^{ -\tau^2/4 - \tau/2}$, because 
%$\sum_{i < j} \sigma_i \sigma_j \tsigma_i \tsigma_j = \frac{1}{2}(
%\iprod{ \sigma}{\tsigma}^2 - n)$.}

Write $\sigma=2\xi-1$ for $\xi\in\{0,1\}^n$ and let 
$$
H_1\triangleq \xi_1 \tilde{\xi}_1 + \xi_2 \tilde{\xi}_2 
\quad \text{ and } \quad 
H_2 \triangleq \sum_{j \ge 3}^n \xi_j \tilde{\xi}_j.
$$ 
Then $\iprod{\sigma}{\tsigma} = 4 (H_1+H_2) -n$.
Moreover, conditional  on $\sigma_1, \sigma_2$ and $\tsigma_1, \tsigma_2$, 
$$
H_2 \sim \text{Hypergeometric} \left( n-2, n/2 - \xi_1-\xi_2, n/2 - \tilde{\xi}_1-\tilde{\xi}_2 \right).
$$
Since $|H_1| \le 2$, $\xi_1+\xi_2 \le 2 $, and $\tilde{\xi}_1+\tilde{\xi}_2 \le 2$, 
it follows that 
conditional on $\sigma_1 \sigma_2=z, \tsigma_1 \tsigma_2 =\tz $,
$ \frac{1}{\sqrt{n}} ( 4H_1 + 4H_2 - n )$ converges to $\calN(0,1)$ in distribution 
as $n \to \infty$ by the central limit theorem for hypergeometric distribution. 
Therefore
\begin{align*}
& \Expect \qth{ \exp \left(  \frac{\rho}{2} \iprod{ \sigma}{\tsigma}^2  \right) \mathrel{\bigg|} \sigma_S=z, \tsigma_S =\tz} \\
& =\expect{ \exp \left(  \frac{n\rho}{2} \left( \frac{ 4 H_1 + 4H_2 - n}{\sqrt{n} } \right)^2 \right) \mathrel{\bigg|} 
\sigma_1 \sigma_2=z, \tsigma_1 \tsigma_2 =\tz }  \\
& = \frac{1+o(1)}{\sqrt{1-\tau} },
\end{align*}
where the last equality holds due to $n \rho = \tau +o(1/n)$, $\tau<1$, 
and the convergence of the moment generating function. 

%In conclusion, we have shown that 
%$$
%\int \frac{   \P_{z}  \P_{\tz} }{ \Q}  
%= \left( 1+o(1) \right) e^{ -\tau^2/4 - \tau/2}  \frac{1}{\sqrt{1-\tau}}, \quad \forall z,\tz \in \{ \pm \}.
%$$
%%Hence, by taking the expectation of $\sigma_S$ and $\tsigma_S$ over the both hand sides of the last displayed equation, we get that
%Averaging both sides over $z,\tz$ yields
%$$
%\int \frac{   \P^2 }{ \Q}  
%= \left( 1+o(1) \right) e^{ -\tau^2/4 - \tau/2}  \frac{1}{\sqrt{1-\tau}}.
%$$
%Thus \prettyref{eq:second_moment_conditional} is satisfied.


\end{appendices}



 \documentclass[letterpaper,11pt]{article}
\usepackage[toc,page]{appendix}
\usepackage[margin=1in]{geometry}
\usepackage[bookmarks, colorlinks=true, plainpages = false, citecolor = blue,linkcolor=red,urlcolor = blue, filecolor = blue,pagebackref]{hyperref}
%% Note, I added pagebackref to the options for hyperref so that page number back references appear after each paper listing.   -BH 4/29/15
%\usepackage{url}\urlstyle{rm}
\usepackage{amsmath,amsfonts,amsthm,amssymb,bm, verbatim,dsfont,mathtools}
%\usepackage{algorithm,algorithmic}
\usepackage{color,graphicx,appendix}
\usepackage{subfigure}
\usepackage{etoolbox}
\usepackage{tikz}
\usepackage{xr,xspace}
\usepackage{todonotes}
\usepackage{paralist}
\usepackage{caption,soul}
%\usepackage[ruled,vlined]{algorithm2e}
\usepackage{algorithm}% http://ctan.org/pkg/algorithms
\usepackage{algorithmic}% http://ctan.org/pkg/algorithms
\usepackage{enumitem}
\makeatletter
%\renewcommand{\ALG@name}{SDP}
%\renewcommand{\listalgorithmname}{List of \ALG@name s}

%%%%% THEOREM STYLE DEFINITIONS
%\theoremstyle{plain}
%\newtheorem{theorem}{Theorem}
%\newtheorem{lemma}{Lemma}
%\newtheorem{proposition}{Proposition}
%\newtheorem{corollary}{Corollary}
%\theoremstyle{definition}
%\newtheorem{definition}{Definition}
%\newtheorem{hypothesis}{Hypothesis}
%\newtheorem{conjecture}{Conjecture}
%\newtheorem{question}{Question}
%\newtheorem{remark}{Remark}
%\newtheorem*{remark*}{Remark}

\newtheorem{theorem}{Theorem}
\newtheorem{lemma}{Lemma}
\newtheorem{proposition}{Proposition}
\newtheorem{corollary}{Corollary}
\theoremstyle{definition}
\newtheorem{definition}{Definition}
\newtheorem{hypothesis}{Hypothesis}
\newtheorem{conjecture}{Conjecture}
\newtheorem{question}{Question}
\newtheorem{remark}{Remark}
\newtheorem{assumption}{Assumption}


%\usepackage{enumerate}

\usepackage{tikz}

\usepackage{xspace,prettyref}
\usepackage{bm}
% for prettyref.sty
\newrefformat{eq}{(\ref{#1})}
\newrefformat{chap}{Chapter~\ref{#1}}
\newrefformat{sec}{Section~\ref{#1}}
\newrefformat{alg}{Algorithm~\ref{#1}}
\newrefformat{fig}{Fig.~\ref{#1}}
\newrefformat{tab}{Table~\ref{#1}}
\newrefformat{rmk}{Remark~\ref{#1}}
\newrefformat{clm}{Claim~\ref{#1}}
\newrefformat{def}{Definition~\ref{#1}}
\newrefformat{cor}{Corollary~\ref{#1}}
\newrefformat{lmm}{Lemma~\ref{#1}}
\newrefformat{prop}{Proposition~\ref{#1}}
\newrefformat{app}{Appendix~\ref{#1}}
\newrefformat{hyp}{Hypothesis~\ref{#1}}
\newrefformat{thm}{Theorem~\ref{#1}}
\newrefformat{ass}{Assumption~\ref{#1}}
\newrefformat{conj}{Conjecture~\ref{#1}}

\newcommand{\ie}{i.e.\xspace}
\renewcommand{\P}{\mathcal{P} }
\newcommand{\Q}{\mathcal{Q}}
\newcommand{\Exp}{\mathbb{E}}
\newcommand{\contig}{\trianglelefteq}
\newcommand{\indicator}[1]{\bm{1}_{#1}}
\renewcommand{\hat}{\widehat}
\renewcommand{\tilde}{\widetilde}
\newcommand{\argmax}{\mathrm{argmax}}

\usepackage{color}
\newcommand{\red}{\color{red}}
\newcommand{\blue}{\color{blue}}
\newcommand{\nb}[1]{{\sf\blue[#1]}}
\newcommand{\nbr}[1]{{\sf\red #1}}

\renewcommand{\implies}{\Rightarrow}
\newcommand{\1}[1]{{\mathbf{1}_{\left\{{#1}\right\}}}}
\newcommand{\post}[2]{\begin{center} \includegraphics[width=#2]{#1} \end{center} }
\newcommand \E[1]{\mathbb{E}[#1]}
%\newcommand{\Perp}{\perp \! \! \! \perp}
\newcommand{\Perp}{\perp}
\newcommand{\Hyper}{\text{Hypergeometric}}
%\newcommand \P[1]{\mathbb{P}[#1]}


%% Wu
\newcommand{\bfa}{{\mathbf{a}}}
\newcommand{\bfb}{{\mathbf{b}}}
\newcommand{\bfc}{{\mathbf{c}}}
\newcommand{\bfd}{{\mathbf{d}}}
\newcommand{\bfe}{{\mathbf{e}}}
\newcommand{\bff}{{\mathbf{f}}}
\newcommand{\bfg}{{\mathbf{g}}}
\newcommand{\bfh}{{\mathbf{h}}}
\newcommand{\bfi}{{\mathbf{i}}}
\newcommand{\bfj}{{\mathbf{j}}}
\newcommand{\bfk}{{\mathbf{k}}}
\newcommand{\bfl}{{\mathbf{l}}}
\newcommand{\bfm}{{\mathbf{m}}}
\newcommand{\bfn}{{\mathbf{n}}}
\newcommand{\bfo}{{\mathbf{o}}}
\newcommand{\bfp}{{\mathbf{p}}}
\newcommand{\bfq}{{\mathbf{q}}}
\newcommand{\bfr}{{\mathbf{r}}}
\newcommand{\bfs}{{\mathbf{s}}}
\newcommand{\bft}{{\mathbf{t}}}
\newcommand{\bfu}{{\mathbf{u}}}
\newcommand{\bfv}{{\mathbf{v}}}
\newcommand{\bfw}{{\mathbf{w}}}
\newcommand{\bfx}{{\mathbf{x}}}
\newcommand{\bfy}{{\mathbf{y}}}
\newcommand{\bfz}{{\mathbf{z}}}
\newcommand{\bfA}{{\mathbf{A}}}
\newcommand{\bfB}{{\mathbf{B}}}
\newcommand{\bfC}{{\mathbf{C}}}
\newcommand{\bfD}{{\mathbf{D}}}
\newcommand{\bfE}{{\mathbf{E}}}
\newcommand{\bfF}{{\mathbf{F}}}
\newcommand{\bfG}{{\mathbf{G}}}
\newcommand{\bfH}{{\mathbf{H}}}
\newcommand{\bfI}{{\mathbf{I}}}
\newcommand{\bfJ}{{\mathbf{J}}}
\newcommand{\bfK}{{\mathbf{K}}}
\newcommand{\bfL}{{\mathbf{L}}}
\newcommand{\bfM}{{\mathbf{M}}}
\newcommand{\bfN}{{\mathbf{N}}}
\newcommand{\bfO}{{\mathbf{O}}}
\newcommand{\bfP}{{\mathbf{P}}}
\newcommand{\bfQ}{{\mathbf{Q}}}
\newcommand{\bfR}{{\mathbf{R}}}
\newcommand{\bfS}{{\mathbf{S}}}
\newcommand{\bfT}{{\mathbf{T}}}
\newcommand{\bfU}{{\mathbf{U}}}
\newcommand{\bfV}{{\mathbf{V}}}
\newcommand{\bfW}{{\mathbf{W}}}
\newcommand{\bfX}{{\mathbf{X}}}
\newcommand{\bfY}{{\mathbf{Y}}}
\newcommand{\bfZ}{{\mathbf{Z}}}

\newcommand{\bbA}{{\mathbb{A}}}
\newcommand{\bbB}{{\mathbb{B}}}
\newcommand{\bbC}{{\mathbb{C}}}
\newcommand{\bbD}{{\mathbb{D}}}
\newcommand{\bbE}{{\mathbb{E}}}
\newcommand{\bbF}{{\mathbb{F}}}
\newcommand{\bbG}{{\mathbb{G}}}
\newcommand{\bbH}{{\mathbb{H}}}
\newcommand{\bbI}{{\mathbb{I}}}
\newcommand{\bbJ}{{\mathbb{J}}}
\newcommand{\bbK}{{\mathbb{K}}}
\newcommand{\bbL}{{\mathbb{L}}}
\newcommand{\bbM}{{\mathbb{M}}}
\newcommand{\bbN}{{\mathbb{N}}}
\newcommand{\bbO}{{\mathbb{O}}}
\newcommand{\bbP}{{\mathbb{P}}}
\newcommand{\bbQ}{{\mathbb{Q}}}
\newcommand{\bbR}{{\mathbb{R}}}
\newcommand{\bbS}{{\mathbb{S}}}
\newcommand{\bbT}{{\mathbb{T}}}
\newcommand{\bbU}{{\mathbb{U}}}
\newcommand{\bbV}{{\mathbb{V}}}
\newcommand{\bbW}{{\mathbb{W}}}
\newcommand{\bbX}{{\mathbb{X}}}
\newcommand{\bbY}{{\mathbb{Y}}}
\newcommand{\bbZ}{{\mathbb{Z}}}
                         
                         
\newcommand{\sfa}{{\mathsf{a}}}
\newcommand{\sfb}{{\mathsf{b}}}
\newcommand{\sfc}{{\mathsf{c}}}
\newcommand{\sfd}{{\mathsf{d}}}
\newcommand{\sfe}{{\mathsf{e}}}
\newcommand{\sff}{{\mathsf{f}}}
\newcommand{\sfg}{{\mathsf{g}}}
\newcommand{\sfh}{{\mathsf{h}}}
\newcommand{\sfi}{{\mathsf{i}}}
\newcommand{\sfj}{{\mathsf{j}}}
\newcommand{\sfk}{{\mathsf{k}}}
\newcommand{\sfl}{{\mathsf{l}}}
\newcommand{\sfm}{{\mathsf{m}}}
\newcommand{\sfn}{{\mathsf{n}}}
\newcommand{\sfo}{{\mathsf{o}}}
\newcommand{\sfp}{{\mathsf{p}}}
\newcommand{\sfq}{{\mathsf{q}}}
\newcommand{\sfr}{{\mathsf{r}}}
\newcommand{\sfs}{{\mathsf{s}}}
\newcommand{\sft}{{\mathsf{t}}}
\newcommand{\sfu}{{\mathsf{u}}}
\newcommand{\sfv}{{\mathsf{v}}}
\newcommand{\sfw}{{\mathsf{w}}}
\newcommand{\sfx}{{\mathsf{x}}}
\newcommand{\sfy}{{\mathsf{y}}}
\newcommand{\sfz}{{\mathsf{z}}}
\newcommand{\sfA}{{\mathsf{A}}}
\newcommand{\sfB}{{\mathsf{B}}}
\newcommand{\sfC}{{\mathsf{C}}}
\newcommand{\sfD}{{\mathsf{D}}}
\newcommand{\sfE}{{\mathsf{E}}}
\newcommand{\sfF}{{\mathsf{F}}}
\newcommand{\sfG}{{\mathsf{G}}}
\newcommand{\sfH}{{\mathsf{H}}}
\newcommand{\sfI}{{\mathsf{I}}}
\newcommand{\sfJ}{{\mathsf{J}}}
\newcommand{\sfK}{{\mathsf{K}}}
\newcommand{\sfL}{{\mathsf{L}}}
\newcommand{\sfM}{{\mathsf{M}}}
\newcommand{\sfN}{{\mathsf{N}}}
\newcommand{\sfO}{{\mathsf{O}}}
\newcommand{\sfP}{{\mathsf{P}}}
\newcommand{\sfQ}{{\mathsf{Q}}}
\newcommand{\sfR}{{\mathsf{R}}}
\newcommand{\sfS}{{\mathsf{S}}}
\newcommand{\sfT}{{\mathsf{T}}}
\newcommand{\sfU}{{\mathsf{U}}}
\newcommand{\sfV}{{\mathsf{V}}}
\newcommand{\sfW}{{\mathsf{W}}}
\newcommand{\sfX}{{\mathsf{X}}}
\newcommand{\sfY}{{\mathsf{Y}}}
\newcommand{\sfZ}{{\mathsf{Z}}}

\newcommand{\TV}{d_{\rm TV}}

\newcommand{\floor}[1]{{\left\lfloor {#1} \right \rfloor}}
\newcommand{\ceil}[1]{{\left\lceil {#1} \right \rceil}}

\usepackage{xspace,prettyref}
\newcommand{\CML}{\widehat{C}_{\rm ML}}
\newcommand{\diverge}{\to\infty}
\newcommand{\eqdistr}{{\stackrel{\rm (d)}{=}}}
\newcommand{\iiddistr}{{\stackrel{\text{\iid}}{\sim}}}
\newcommand{\ones}{\mathbf 1}
\newcommand{\zeros}{\mathbf 0}
\newcommand{\reals}{{\mathbb{R}}}
\newcommand{\integers}{{\mathbb{Z}}}
\newcommand{\naturals}{{\mathbb{N}}}
\newcommand{\rationals}{{\mathbb{Q}}}
\newcommand{\naturalsex}{\overline{\mathbb{N}}}
\newcommand{\symm}{{\mbox{\bf S}}}  % symmetric matrices
\newcommand{\supp}{{\rm supp}}
\newcommand{\eexp}{{\rm e}}
\newcommand{\rexp}[1]{{\rm e}^{#1}}
\newcommand{\identity}{\mathbf I}
\newcommand{\allones}{\mathbf J}
%\newcommand{\zeros}{\mathbf 0}

\newcommand{\diff}{{\rm d}}

\newcommand{\Expect}{\mathbb{E}}
\newcommand{\expect}[1]{\mathbb{E}\left[ #1 \right]}
\newcommand{\eexpect}[1]{\mathbb{E}[ #1 ]}
\newcommand{\expects}[2]{\mathbb{E}_{#2}\left[ #1 \right]}
\newcommand{\tExpect}{{\tilde{\mathbb{E}}}}
%\newcommand{\Prob}{\mathop{\mathbb{P}}}
\newcommand{\Prob}{\mathbb{P}}
\newcommand{\pprob}[1]{ \mathbb{P}\{ #1 \} }
\newcommand{\prob}[1]{ \mathbb{P}\left\{ #1 \right\} }
\newcommand{\tProb}{{\tilde{\mathbb{P}}}}
\newcommand{\tprob}[1]{{ \tProb\left\{ #1 \right\} }}
\newcommand{\hProb}{\widehat{\mathbb{P}}}
\newcommand{\probs}[2]{\mathbb{P}_{#2}\left\{ #1 \right\} }
\newcommand{\toprob}{\xrightarrow{\Prob}}
\newcommand{\tolp}[1]{\xrightarrow{L^{#1}}}
\newcommand{\toas}{\xrightarrow{{\rm a.s.}}}
\newcommand{\toae}{\xrightarrow{{\rm a.e.}}}
\newcommand{\todistr}{\xrightarrow{{\rm D}}}
\newcommand{\toweak}{\rightharpoonup}
\newcommand{\var}{\mathsf{var}}
\newcommand{\Cov}{\text{Cov}}
\newcommand\indep{\protect\mathpalette{\protect\independenT}{\perp}}
\def\independenT#1#2{\mathrel{\rlap{$#1#2$}\mkern2mu{#1#2}}}
\newcommand{\Bern}{{\rm Bern}}
\newcommand{\Binom}{{\rm Binom}}
\newcommand{\Pois}{{\rm Pois}}
\newcommand{\Hyp}{{\rm Hyp}}
\newcommand{\eg}{e.g.\xspace}
\newcommand{\iid}{i.i.d.\xspace}
% for prettyref.sty
\newrefformat{eq}{(\ref{#1})}
\newrefformat{chap}{Chapter~\ref{#1}}
\newrefformat{sec}{Section~\ref{#1}}
\newrefformat{alg}{Algorithm~\ref{#1}}
\newrefformat{fig}{Fig.~\ref{#1}}
\newrefformat{tab}{Table~\ref{#1}}
\newrefformat{rmk}{Remark~\ref{#1}}
\newrefformat{clm}{Claim~\ref{#1}}
\newrefformat{def}{Definition~\ref{#1}}
\newrefformat{cor}{Corollary~\ref{#1}}
\newrefformat{lmm}{Lemma~\ref{#1}}
\newrefformat{prop}{Proposition~\ref{#1}}
\newrefformat{app}{Appendix~\ref{#1}}
\newrefformat{hyp}{Hypothesis~\ref{#1}}
\newrefformat{thm}{Theorem~\ref{#1}}
\newrefformat{ass}{Assumption~\ref{#1}}
\newcommand{\ntok}[2]{{#1,\ldots,#2}}
\newcommand{\pth}[1]{\left( #1 \right)}
\newcommand{\qth}[1]{\left[ #1 \right]}
\newcommand{\sth}[1]{\left\{ #1 \right\}}
\newcommand{\bpth}[1]{\Bigg( #1 \Bigg)}
\newcommand{\bqth}[1]{\Bigg[ #1 \Bigg]}
\newcommand{\bsth}[1]{\Bigg\{ #1 \Bigg\}}
\newcommand{\norm}[1]{\left\|{#1} \right\|}
\newcommand{\lpnorm}[1]{\left\|{#1} \right\|_{p}}
\newcommand{\linf}[1]{\left\|{#1} \right\|_{\infty}}
\newcommand{\lnorm}[2]{\left\|{#1} \right\|_{{#2}}}
\newcommand{\Lploc}[1]{L^{#1}_{\rm loc}}
\newcommand{\hellinger}{d_{\rm H}}
\newcommand{\Fnorm}[1]{\lnorm{#1}{\rm F}}
\newcommand{\fnorm}[1]{\|#1\|_{\rm F}}
% \newcommand{\opnorm}[1]{\lnorm{#1}{\rm op}}
\newcommand{\opnorm}[1]{\left\| #1 \right\|_2}
% inner product
\newcommand{\iprod}[2]{\left \langle #1, #2 \right\rangle}
\newcommand{\Iprod}[2]{\langle #1, #2 \rangle}
% 12/02/2007
\newcommand{\indc}[1]{{\mathbf{1}_{\left\{{#1}\right\}}}}
\newcommand{\Indc}{\mathbf{1}}
\newcommand{\diag}[1]{\mathsf{diag} \left\{ {#1} \right\} }
\newcommand{\degr}{\mathsf{deg} }

\newcommand{\dTV}{d_{\rm TV}}
\newcommand{\tb}{\widetilde{b}}
\newcommand{\tr}{\widetilde{r}}
\newcommand{\tc}{\widetilde{c}}
\newcommand{\tu}{{\widetilde{u}}}
\newcommand{\tv}{{\widetilde{v}}}
\newcommand{\tx}{{\widetilde{x}}}
\newcommand{\ty}{{\widetilde{y}}}
\newcommand{\tz}{{\widetilde{z}}}
\newcommand{\tA}{{\widetilde{A}}}
\newcommand{\tB}{{\widetilde{B}}}
\newcommand{\tC}{{\widetilde{C}}}
\newcommand{\tD}{{\widetilde{D}}}
\newcommand{\tE}{{\widetilde{E}}}
\newcommand{\tF}{{\widetilde{F}}}
\newcommand{\tG}{{\widetilde{G}}}
\newcommand{\tH}{{\widetilde{H}}}
\newcommand{\tI}{{\widetilde{I}}}
\newcommand{\tJ}{{\widetilde{J}}}
\newcommand{\tK}{{\widetilde{K}}}
\newcommand{\tL}{{\widetilde{L}}}
\newcommand{\tM}{{\widetilde{M}}}
\newcommand{\tN}{{\widetilde{N}}}
\newcommand{\tO}{{\widetilde{O}}}
\newcommand{\tP}{{\widetilde{P}}}
\newcommand{\tQ}{{\widetilde{Q}}}
\newcommand{\tR}{{\widetilde{R}}}
\newcommand{\tS}{{\widetilde{S}}}
\newcommand{\tT}{{\widetilde{T}}}
\newcommand{\tU}{{\widetilde{U}}}
\newcommand{\tV}{{\widetilde{V}}}
\newcommand{\tW}{{\widetilde{W}}}
\newcommand{\tX}{{\widetilde{X}}}
\newcommand{\tY}{{\widetilde{Y}}}
\newcommand{\tZ}{{\widetilde{Z}}}


\newcommand{\wh}{\widehat}
\newcommand{\wt}{\widetilde}
\newcommand{\whp}{\rm w.h.p.}
\newcommand{\wpal}{\rm with probability at least~}
\newcommand{\pc}{Planted Clique\xspace}


\newcommand{\calA}{{\mathcal{A}}}
\newcommand{\calB}{{\mathcal{B}}}
\newcommand{\calC}{{\mathcal{C}}}
\newcommand{\calD}{{\mathcal{D}}}
\newcommand{\calE}{{\mathcal{E}}}
\newcommand{\calF}{{\mathcal{F}}}
\newcommand{\calG}{{\mathcal{G}}}
\newcommand{\calH}{{\mathcal{H}}}
\newcommand{\calI}{{\mathcal{I}}}
\newcommand{\calJ}{{\mathcal{J}}}
\newcommand{\calK}{{\mathcal{K}}}
\newcommand{\calL}{{\mathcal{L}}}
\newcommand{\calM}{{\mathcal{M}}}
\newcommand{\calN}{{\mathcal{N}}}
\newcommand{\calO}{{\mathcal{O}}}
\newcommand{\calP}{{\mathcal{P}}}
\newcommand{\calQ}{{\mathcal{Q}}}
\newcommand{\calR}{{\mathcal{R}}}
\newcommand{\calS}{{\mathcal{S}}}
\newcommand{\calT}{{\mathcal{T}}}
\newcommand{\calU}{{\mathcal{U}}}
\newcommand{\calV}{{\mathcal{V}}}
\newcommand{\calW}{{\mathcal{W}}}
\newcommand{\calX}{{\mathcal{X}}}
\newcommand{\calY}{{\mathcal{Y}}}
\newcommand{\calZ}{{\mathcal{Z}}}

\newcommand{\comp}[1]{{#1^{\rm c}}}
\newcommand{\Leb}{{\rm Leb}}
\newcommand{\Th}{{\rm th}}

\newcommand{\PDS}{{\sf PDS}\xspace}
\newcommand{\PC}{{\sf PC}\xspace}
\newcommand{\BPDS}{{\sf BPDS}\xspace}
\newcommand{\BPC}{{\sf BPC}\xspace}
\newcommand{\DKS}{{\sf DKS}\xspace}
\newcommand{\ML}{{\rm ML}\xspace}
\newcommand{\SDP}{{\rm SDP}\xspace}
\newcommand{\SBM}{{\sf SBM}\xspace}

\newcommand{\ER}{Erd\H{o}s-R\'enyi\xspace}

\newcommand{\Tr}{\mathsf{Tr}}
\renewcommand{\hat}{\widehat}
\renewcommand{\tilde}{\widetilde}

\newcommand{\MMSE}{{\rm MMSE}}
\newcommand{\snr}{{\mathsf{snr}}}

\newcommand{\tsigma}{\tilde{\sigma}}

\newcommand{\planted}{\sigma^*}
\newcommand{\score}{\calT}

%\usepackage{mleftright}

  \begin{document}
	


\title{Statistical Problems with Planted Structures: Information-Theoretical and Computational Limits}

\date{\today}
\author{ 
Yihong Wu \and Jiaming Xu\thanks{
%This research was supported by the National Science Foundation under
%Grant ECCS 10-28464, IIS-1447879, and CCF-1423088, and
%Strategic Research
%Initiative on Big-Data Analytics of the College of Engineering
%at the University of Illinois, and DOD ONR Grant N00014-14-1-0823, and Grant 328025 from the Simons Foundation. 
Y.~Wu is with Department of Statistics and Data Science, 
Yale University, New Haven, CT 06520, USA, \texttt{yihong.wu@yale.edu}.
J.~Xu is with the Fuqua School of Business, Duke University,
Durham, NC 27708, \texttt{jiaming.xu868@duke.edu}.}
%Krannert School of Management, Purdue University,
%       West Lafayette, IN 47907, USA, \texttt{xu972@purdue.edu}.}
}
  
  \maketitle

\begin{abstract}
Over the past few years, insights from computer science, statistical physics, and information theory have revealed phase transitions in a wide array of high-dimensional statistical problems at two distinct thresholds: One is the information-theoretical (IT) threshold below which the observation is too noisy so that inference of the ground truth structure is impossible regardless of the computational cost; the other is the computational threshold above which inference can be performed efficiently, i.e., in time that is polynomial in the input size. In the intermediate regime, inference is information-theoretically possible, but conjectured to be computationally hard.

This article provides a survey of the common techniques for determining the sharp IT and computational limits, using community detection and submatrix detection as illustrating examples. For IT limits, we discuss tools including the first and second moment method for analyzing the maximum likelihood estimator, information-theoretic methods for proving impossibility results using  mutual information and rate-distortion theory, and methods originated from statistical physics such as interpolation method. To investigate computational limits, we describe a common recipe to construct a randomized polynomial-time reduction scheme that approximately maps instances of the planted clique problem to the problem of interest in total variation distance.
\end{abstract}

\tableofcontents
  
 \pdfoutput=1
\documentclass{article}
\usepackage[final]{pdfpages}
\begin{document}
\includepdf[pages=1-9]{CVPR18VOlearner.pdf}
\includepdf[pages=1-last]{supp.pdf}
\end{document}

\begin{appendices}

\section{Mutual information-characterization of correlated recovery}
\label{app:MIcorr}

\newcommand{\Unif}{\mathrm{Unif}}


We consider a general setup:  Let the number of communities $k$ be a constant. Denote the membership vector by $\sigma=(\sigma_1,\ldots,\sigma_n) \in [k]^n$ and the observation is $A=(A_{ij}: 1 \leq i < j \leq n)$. Assume the following conditions:
\begin{enumerate}[label=A\arabic*]
	%[{A}1.]
	\item \label{A1}
	For any permutation $\pi\in S_k$, $(\sigma,A)$ and $(\pi(\sigma),A)$ are equal in law, where $\pi(\sigma)\triangleq (\pi(\sigma_1),\ldots,\pi(\sigma_n))$; 
	
	\item \label{A2}
	For any $i \neq j \in [n]$, $I(\sigma_i,\sigma_j;A)= I(\sigma_1,\sigma_2;A)$;
	
	\item \label{A3}
	For any $z_1,z_2 \in [k]$, $\prob{\sigma_1=z_1,\sigma_2=z_2} = \frac{1}{k^2} + o(1)$ as $n\to \infty$.
\end{enumerate}
These assumptions are satisfied for example for $k$-community SBM (where each pair of vertices $i$ and $j$ are connected independently with probability $p$ if $\sigma_i=\sigma_j$ and $q$ otherwise),\index{Stochastic block model (SBM)! $k$ communities}
 and the membership vector
$\sigma$ can be either uniformly distributed on $[k]^n$ or the set of equal-sized $k$-partition of $[n]$. 

Recall that correlated recovery entails the following:
For any $\sigma , \hat{\sigma} \in [k]^n$, define the overlap:
\begin{align}
o\left( \sigma, \hat{\sigma} \right) = \frac{1}{n} \max_{\pi \in S_k} 
\sum_{i \in [n] } \left( \indc{ \pi\left(\sigma_i \right) = \hat{\sigma}_i} - \frac{1}{k} \right).
\end{align}
We say an estimator $\hat{\sigma} = \hat{\sigma}(A)$ achieves correlated recovery if\footnote{For the special case of $k=2$, \prettyref{eq:corro} is equivalent to 
$\frac{1}{n}\Expect[|\Iprod{\sigma}{\hat \sigma}|] = \Omega(1)$, where $\sigma , \hat{\sigma}$ are assumed to be $\{\pm\}^n$-valued.}
\begin{equation}
\expect{o\left( \sigma, \hat{\sigma} \right)}=\Omega(1),
\label{eq:corro}
\end{equation}
 that is, the misclassification rate, up to a global permutation, outperforms random guessing.
Under the above three assumptions, we have the following characterization of correlated recovery:
\begin{lemma}
\label{lmm:MIcorr}	
Correlated recovery is possible if and only if 
$I(\sigma_1, \sigma_2 ; A) = \Omega(1)$.	
\end{lemma}



\begin{proof}
We start by recalling the relation between mutual information and total variation.
For any pair of random variables $(X,Y)$, define the so-called $T$-information \cite{Csiszar96}:
$T(X;Y) \triangleq \TV(P_{XY}, P_XP_Y) = \Expect[\TV(P_{Y|X}, P_Y)]$.
For $X \sim \Bern(p)$, this simply reduces to 
\begin{equation}
T(X;Y) = 2p(1-p) \TV(P_{Y|X=0}, P_{Y|X=1}).
\label{eq:Tbern}
\end{equation}
Furthermore, the mutual information can be bounded by the $T$-information, 
by Pinsker's and Fano's inequality, as follows \cite[Eq.~(84) and Prop.~12]{PW14a}
\begin{equation}
 2 T(X;Y)^2 \leq I(X;Y)  \leq \log (M-1) T(X;Y) + h(T(X;Y))		
	\label{eq:TI}
\end{equation}
where in the upper bound $M$ is the number of possible values of $X$, and $h$ is the binary entropy function in \prettyref{eq:binaryentropy}.


We prove the ``if'' part.
Suppose 
%correlated recovery is impossible and for the sake of contradiction,
$I(\sigma_1, \sigma_2; A) = \Omega(1)$.
We first claim that assumption \ref{A1} implies that 
\begin{equation}
I (\indc{\sigma_1 =\sigma_2}; A)=I(\sigma_1, \sigma_2; A)
\label{eq:ss}
\end{equation}
that is,
 $A$ is independent of $\sigma_1,\sigma_2$ conditional on $\indc{\sigma_1 =\sigma_2}$. 
%\nbr{JX. Okay. I think we need this condition to ensure that 
%correlated recovery is impossible implies $I(\sigma_1, \sigma_2; A) = o(1)$. 
%Otherwise, we could have the situation that $A=a$ for $\sigma_1=+1$ and
%$A=b$ for $\sigma_1=-1$. In this case, correlated recovery of $\sigma$ is certainly 
%impossible, but $I(\sigma_1, \sigma_2;A)=I(\sigma_1; A)=1$.}
Indeed, 
for any $z \neq z'\in [k]$, let $\pi$ be any permutation such that 
%$\pi(z)=z'$ and 
$\pi(z')=z.$
%that interchanges $z$ and $z'$. 
Since $P_{\sigma, A} =  P_{\pi(\sigma), A}$, 
we have $P_{A|\sigma_1=z,  \sigma_2=z} =  P_{A| \pi(\sigma_1)=z,  \pi(\sigma_2)=z }$, i.e., 
$P_{A |\sigma_1=z, \sigma_2=z} =  P_{A| \sigma_1 =z',  \sigma_2 =z' }$. 
Similarly, one can show that 
$P_{A|\sigma_1=z_1, \sigma_2=z_2} =  P_{A| \sigma_1 =z_1',  \sigma_2 =z_2'}$, 
for any $z_1 \neq z_2$ and $z_1'\neq z_2'$, and this proves the claim.
%Since $\calL(\sigma, A) =  \calL(\pi(\sigma), A)$, where $\calL(\cdot)$ denote the law, 
%we have $\calL(A|\sigma_1=z,  \sigma_2=z) =  \calL(A| \pi(\sigma_1)=i,  pi(\sigma_2)=i )$, i.e., 
%$\calL(A|\sigma_1=z, \sigma_2=z) =  L(A| \sigma_1 =z',  \sigma_2 =z' )$. 
%Similarly, one can show that 
%$\calL(A|\sigma_1=z_1, \sigma_2=z_2) =  L(A| \sigma_1 =z_1',  \sigma_2 =z_2')$, for any $z_1 \neq z_2$ and $z_1'\neq z_2'$, and this proves the claim.

Let $x_j=\indc{\sigma_1 =\sigma_j}$.
By the symmetry assumption \ref{A2}, 
$I(x_j; A) = I(x_2; A) = \Omega(1)$ for all $j \neq 1$.
Since $\prob{x_j = 1} = \frac{1}{k} + o(1)$ by assumption \ref{A3}, applying \prettyref{eq:TI} with $M=2$ and in view of \prettyref{eq:Tbern}, we have
$\TV(P_{A|x_j=0},P_{A|x_j=1})=\Omega(1)$.
Thus, there exists an estimator $\widehat{x}_j \in \{0,1\}$ as a function of $A$, such that
\begin{align}
\prob{\widehat{x}_j = 1 \mid x_j =1 } + 
\prob{\widehat{x}_j = 0 \mid x_j =0 } \ge 1+\TV(P_{A|x_j=0},P_{A|x_j=1})
= 1+ \Omega(1).  \label{eq:estimator_x_hat}
\end{align}

Define $\hat\sigma$ as follows: set $\hat\sigma_1= 1 $; for $j \neq 1$, set $\hat \sigma_j = 1 $ if $\widehat{x}_j = 1$ 
and draw $\hat \sigma_j $ from $\{2,\ldots,k\}$ uniformly at random 
if $\widehat{x}_j = 0$.
Next, we show that $\hat\sigma$ achieves correlated recovery. 
Indeed, fix a permutation $\pi \in S_k$ such that $\pi(\sigma_1)=1$. It follows from 
the definition of overlap that
\begin{equation}
\Expect[o\left(\sigma, \hat{\sigma} \right)]
\ge \frac{1}{n} \sum_{j \neq 2} \prob{\pi(\sigma_j) =\hat\sigma_j} - \frac{1}{k}.
\label{eq:olb}
\end{equation}
Furthermore, since $\pi(\sigma_1)=1$, we have, for any $j\neq 1$,
\[
\prob{\pi(\sigma_j) = \hat\sigma_j,x_j=1}=
\prob{\hat x_j=1,x_j=1}
\]
and
\[
\prob{\pi(\sigma_j) = \hat\sigma_j,x_j=0}=
\prob{\pi(\sigma_j) = \hat\sigma_j, \hat x_j=0,x_j=0}=
\frac{1}{k-1} \prob{\hat x_j=0,x_j=0},
\]
where the last step is because conditional on $\hat{x}_j=0$,
$\hat{\sigma}_j$ is chosen  from $\{2,\ldots,k\}$ uniformly and 
independently of everything else.
Since $\prob{x_j = 1} = \frac{1}{k} + o(1)$, we have
\[
\prob{\pi(\sigma_j) =\hat\sigma_j} = \frac{1}{k}(\prob{\widehat{x}_j = 1 \mid x_j =1 } + 
\prob{\widehat{x}_j = 0 \mid x_j =0 }) +o(1) \overset{\prettyref{eq:estimator_x_hat}}{\ge} \frac{1}{k}+ \Omega(1).  
\]
By \prettyref{eq:olb}, we conclude that $\hat{\sigma}$ achieves correlated recovery
of $\sigma$.

Next we prove the ``only if'' part.
Suppose $I(\sigma_1, \sigma_2; A) = o(1)$ and we aim to show 
%the impossibility of correlated recovery, that is, 
$\expect{o\left(\sigma, \hat{\sigma} \right)}=o(1)$ for any estimator $\hat\sigma$.
By the definition of overlap, we have
\begin{align*}
o\left(\sigma, \hat{\sigma} \right) 
\le 
\frac{1}{n} \sum_{\pi \in S_k}
\left| \sum_{i \in [n] }  
\left( \indc{ \pi\left(\sigma_i \right) = \hat{\sigma}_i} - \frac{1}{k}  \right) \right|.
%& \le \frac{1}{n} \sum_{\pi \in S_k}
%\sum_{\ell=1}^k 
%\left| \sum_{i \in [n] } \left( \indc{ \pi\left(\sigma_i \right) =\ell} 
%\indc{ \hat{\sigma}_i = \ell} - \frac{1}{k^2} \right)\right|.
\end{align*}
Since there are $k!=\Omega(1)$ permutations in $S_k$, it suffices to show
for any fixed permutation $\pi$,
$$
\expect{ \left| \sum_{i \in [n] } \left( \indc{ \pi\left(\sigma_i \right) = \hat{\sigma}_i} - \frac{1}{k}  \right) 
\right| } = o(n).
$$
Since $I(\pi(\sigma_i), \pi(\sigma_j); A)=I(\sigma_i, \sigma_j; A)$, without loss of generality, 
we assume $\pi=\text{id}$ in the following. By the Cauchy-Schwarz inequality, it further suffices to show
\begin{equation}
\expect{ \left( \sum_{i \in [n] } \left( \indc{ \sigma_i =\hat{\sigma}_i } - \frac{1}{k} \right)\right)^2 } =o(n^2).
\label{eq:overlap2}
\end{equation}
Note that 
\begin{align*}
& \expect{  \left( \sum_{i \in [n] } \left(
   \indc{ \sigma_i  = \hat{\sigma}_i } - \frac{1}{k} \right)   \right)^2 } \\
 & = \sum_{i, j \in [n] } 
 \expect{ \left( \indc{ \sigma_i = \hat{\sigma}_i }- \frac{1}{k}  \right) 
 \left( \indc{ \sigma_j  = \hat{\sigma}_j }- \frac{1}{k}  \right) }  \\
 & =  \sum_{i, j \in [n] }  \prob{  \sigma_i = \hat{\sigma}_i , \sigma_j  = \hat{\sigma}_j }
 - \frac{2n}{k} \sum_{i \in [n] }\prob{ \sigma_i = \hat{\sigma}_i } + \frac{n^2}{k^2}.
\end{align*}
For the first term in the last displayed equation, 
let $\sigma'$ be identically distributed as $\hat\sigma$ but independent of $\sigma$.
Since $I(\sigma_i,\sigma_j;\hat\sigma_i,\hat\sigma_j) \le I(\sigma_i,\sigma_j;A)=o(1)$ by the data processing inequality, it follows from the lower bound in 
\prettyref{eq:TI} that $\TV(P_{\sigma_i,\sigma_j,\hat\sigma_i,\hat\sigma_j}, P_{\sigma_i,\sigma_j,\sigma_i',\sigma_j'})=o(1)$.
Since 
$\pprob{  \sigma_i = {\sigma}_i', \sigma_j  = {\sigma}_j' } \leq 
\max_{a,b \in [k]} \prob{  \sigma_i = a, \sigma_j  = b} \leq \frac{1}{k^2}+o(1)$ by assumption \ref{A3}, 
we have
$$
\prob{  \sigma_i = \hat{\sigma}_i , \sigma_j  = \hat{\sigma}_j }
\leq \frac{1}{k^2} + o(1),
$$
Similarly, for the second term, we have
$$
\prob{ \sigma_i = \hat{\sigma}_i } = \frac{1}{k} +o(1),
$$
where the last equality holds due to $I(\sigma_i; A) =o(1).$
Combining the last three displayed equations gives \prettyref{eq:overlap2} and completes the proof.
\end{proof}


%The argument below is essentially contained in \cite{PW18}.
%
%We first consider $k=2$, in which case it is convenient to assume $\sigma \in \{\pm\}^n$. Recall that correlated recovery amounts to reconstruct $\sigma$ (up to a global sign flip) better than chance, that is, find $\hat \sigma=\hat \sigma(Y) \in \{\pm1\}^n$, such that
%\begin{equation}
%%\liminf_{n \to\infty} \frac{1}{n}\Expect[|\Iprod{\sigma}{\hat \sigma}|] > 0.
%\frac{1}{n}\Expect[|\Iprod{\sigma}{\hat \sigma}|] = \Omega(1).
%%> \epsilon
%\label{eq:corr}
%\end{equation}
%%for some constant $\epsilon$.
%Note that by symmetry, for any $i\neq j$, $I(\sigma_i,\sigma_j;A) = I(\sigma_i \sigma_j; A) = I(\sigma_1 \sigma_2; A)$. Suppose $I(\sigma_1 \sigma_2; A) = \Omega(1)$, then for any $j \neq 1$, there exists an estimator $\widehat{\sigma_1\sigma_j}$ as a function of $A$, such that
%$\prob{\widehat{\sigma_1\sigma_j} \neq \sigma_1\sigma_j} \leq \frac{1}{2} - \Omega(1)$. Thus, setting $\hat\sigma$ according to $\hat\sigma_1=+$ and $\hat \sigma_j = \hat{\sigma_1\sigma_j}$ achieves \prettyref{eq:corr}. 
%This shows \prettyref{eq:MIcorr2} is necessary for the impossibility of correlated recovery. 
%%To show \prettyref{eq:MIcorr2} implies the impossibility of correlated recovery,
%To prove the sufficiency, 
%for any estimator $\hat \sigma= \hat \sigma(A) \in \{\pm\}^n$, 
%since $I(\sigma_i \sigma_j; A)=o(1)$, which is equivalent to 
%$\TV(P_{A|\sigma_i \sigma_j=+}, P_{A|\sigma_i \sigma_j=-})=o(1)$, we have 
%$\prob{\sigma_i\sigma_j \neq \hat \sigma_i\hat \sigma_j} \geq \frac{1}{2}-o(1)$. 
%On the other hand, we have the equality:
%\begin{align*}
%2n^2 - 2 \eexpect{\iprod{\sigma}{\hat\sigma}^2} 
%= & ~ \Expect\Fnorm{\sigma \sigma^\top - \hat \sigma\hat \sigma^\top}^2 \\
%= & ~ 4 \sum_{i \neq j} \prob{\sigma_i\sigma_j \neq \hat \sigma_i\hat \sigma_j} = 2n^2-o(n^2).
%\end{align*}
%Thus, $\eexpect{\iprod{\sigma}{\hat\sigma}^2} =o(n^2)$, which implies $\eexpect{|\Iprod{\sigma}{\hat\sigma}|} =o(n)$.
%
%Next we consider $k \geq 3$, in which case correlated recovery is achieved if there exists an estimator $\hat \sigma \in [k]^n$ that outperforms random guessing, i.e., 
%\begin{equation}
%\Expect[d(\sigma,\hat \sigma)] \leq \frac{k-1}{k} - \Omega(1).
%\label{eq:corr-sbmk}
%\end{equation}
%Here the loss function $d$ is the fraction of classification errors up to a global permutation of labels, formally defined as follows: for any $\sigma,\hat \sigma \in [k]^n$, 
%\begin{equation}
%d(\sigma,\hat \sigma) \triangleq  \min_{\pi \in S_k} \frac{1}{n}\sum_{i\in[n]} \indc{\sigma_i \neq \pi(\hat \sigma_i)}
%\label{eq:lossd}
%\end{equation}
%where $S_k$ is the collection of permutations on $[k]$.
%
%%By symmetry, \prettyref{eq:MIcorrk} implies that for any fixed $m$, as $n\diverge$,
%%\begin{equation}
%%I(\sigma_S; A) = o(1)
%%\label{eq:IXSY}
%%\end{equation}
%%for any $S \in \binom{[n]}{m}$, where 
%To show the impossibility of correlated recovery on the basis of \prettyref{eq:MIcorrk}, first of all, note that for any fixed $x,\hat x\in[k]^n$ and any $m\in [n]$ we have
%\begin{equation}
%d(x,\hat x)
%\geq \Expect_{S}[d(x_{\sfS},\hat x_{\sfS})] \label{eq:davg}
%\end{equation}
%where ${\sfS}  \sim \Unif(\binom{[n]}{m})$ and recall that for any $S$, we have 
%$d(x_S,\hat x_S) = \frac{1}{|S|}  \min_{\pi \in S_k} \sum_{i\in S} \indc{x_i \neq \pi(\hat x_i)}$ per \prettyref{eq:lossd}. The inequality \prettyref{eq:davg} simply follows from
%\begin{align}
%d(x,\hat x)
%= & ~ \min_{\pi \in S_k} \probs{x_I \neq \hat x_{\pi(I)}}{I \sim \Unif([n])}	\nonumber \\
%= & ~ \min_{\pi \in S_k} \Expect_{\sfS \sim \Unif(\binom{[n]}{m})} \probs{x_I \neq \hat x_{\pi(I)}}{I \sim \Unif({\sfS})}	\nonumber \\
%= & ~ \Expect_{{\sfS}} \min_{\pi \in S_k} \probs{x_I \neq \hat x_{\pi(I)}}{I \sim \Unif({\sfS})}	\nonumber \\
%\geq & ~ \Expect_{{\sfS}} [d(x_{\sfS},\hat x_{\sfS})] \nonumber.
%\end{align}
%%Fix a constant $m$ independent of $n$. 
%For any estimator $\hat \sigma=\hat \sigma(Y) \in [k]^n$, applying \prettyref{eq:davg} yields
%\begin{equation}
%\Expect[d(\sigma_{\sfS},\hat \sigma_{\sfS})] \leq \Expect[d(\sigma,\hat \sigma)], \label{eq:davg2}
%\end{equation}
%where ${\sfS}$ is a random uniform $m$-set independent of $\sigma,\hat \sigma$.
%
%
%By the data processing inequality, we have for any $S \in \binom{[n]}{m}$,
%\[
%I(\sigma_S; \hat \sigma_S) \leq I(\sigma_S; A) = I(\sigma_1,\ldots,\sigma_m; A) \overset{\prettyref{eq:MIcorrk}}{=} o(1),
%\]
%as $n\diverge$.
%By Pinsker's inequality, we have 
%$\TV(P_{\sigma_S, \hat \sigma_S}, P_{\sigma_S} \otimes P_{\hat \sigma_S}) \leq \sqrt{2 I(\sigma_S; \hat \sigma_S)} = o(1)$.
%Since the loss function $d$ defined in \prettyref{eq:lossd} is bounded by one, we have
%\begin{equation}
%\Expect[d(\sigma_S,\hat \sigma_S)] \geq \Expect[d(\sigma_S,\sigma'_S)] - \TV(P_{\sigma_S, \hat \sigma_S}, P_{\sigma_S} \otimes P_{\hat \sigma_S}) 
%= \Expect[d(\sigma_S,\sigma'_S)] + o(1), 
%\label{eq:ddTV}
%\end{equation}
%where $\sigma'_S$ has the same marginal distribution as $\hat \sigma_S$ but independent of $\sigma_S$.
%By \prettyref{lmm:randomguess} below, we have
%\begin{equation}
%\Expect[d(\sigma_S,\sigma'_S)] \geq \pth{\frac{k-1}{k} - m^{-1/3}}(1-k! e^{-2m^{1/3}}).
%\label{eq:randomguess2}
%\end{equation}
%Averaging \prettyref{eq:randomguess2} over $S \in \binom{[n]}{m}$ then combining with \prettyref{eq:davg2} and \prettyref{eq:ddTV}, and finally sending $n\to\infty$ followed by $m \to \infty$, we conclude that $\eexpect{ d(\sigma,\hat \sigma)} \geq \frac{k-1}{k}-o(1)$, hence the impossibility of correlated recovery.
%
%
%\begin{lemma}
%\label{lmm:randomguess}	
	%Let $\sigma$ be uniformly distributed on $[k]^m$ and $\sigma'$ is independent of $\sigma$ with an arbitrary distribution on $[k]^m$. 
%For the loss function in \prettyref{eq:lossd}, we have\footnote{Note that for any fixed $k,m$ and any string $x,z\in [k]^m$, we can always outperform random matching, i.e., $d(x,z) < \frac{k-1}{k}$. The point of \prettyref{eq:randomguess} is that this improvement is negligible for large $m$.}
	%\begin{equation}
	%d(\sigma,\sigma') \geq \frac{k-1}{k} - m^{-1/3}
	%\label{eq:randomguess}
	%\end{equation}
	%with probability at least $1-(k! e^{-2m^{1/3}})$.
%\end{lemma}
%\begin{proof}
	%For each fixed $\pi$, the Hamming distance $d_H(\sigma,\pi(\sigma'))\sim \Binom(m,\frac{k-1}{k})$. From Hoeffding's inequality we have
	%$$ \Prob[d_H(\sigma,\pi(\sigma') < {k-1\over k} - \delta] \le e^{-2m \delta^2}\,,$$
	%and from the union bound
	%$$ \Prob[\min_\pi d_H(\sigma,\pi(\sigma') < {k-1\over k} - \delta] \le k! e^{-2m \delta^2}\,.$$
	%Setting $\delta = m^{-1/3}$ completes the proof.
%\end{proof}



%\section{Proof of \prettyref{eq:second_moment_conditional} $\implies$ \prettyref{eq:MI_TV} and verification in the binary symmetric SBM}
%\label{app:MITV}
%Let $S=[m]$ and denote $\sigma_1,\ldots,\sigma_m$ by $\sigma_S$.
%Recall that $m$ is a constant and $\P_z \triangleq \P_{A|\sigma_S=z}$ for $z\in[k]^m$.
%We first prove the following:
%%\begin{align}
%%I(\sigma_S; A) = o(1) & \Leftrightarrow D \left( \P_{A | \sigma_S} \| \P \right) =o(1), \quad \forall \sigma_S  \nonumber \\
%%& \Leftrightarrow d_{\rm TV} \left( \P_{A | \sigma_S}, \P \right) =o(1), \quad \forall \sigma_S. \nonumber \\
%%& \Leftarrow d_{\rm TV} \left( \P_{A | \sigma_S}, \P_{A | \tsigma_S} \right) =o(1), \quad \forall \sigma_S, \tsigma_S,
%%\label{eq:MI_TV}
%%\end{align}
%\begin{align}
%I(\sigma_S; A) = o(1) & \Leftrightarrow D \left( \P_z \| \P \right) =o(1), \quad \forall z, \label{eq:MI_TV1}\\
%& \Leftrightarrow d_{\rm TV} \left( \P_z, \P \right) =o(1), \quad \forall z, \label{eq:MI_TV2}\\
%& \Leftarrow d_{\rm TV} \left( \P_z, \P_{\tz} \right) =o(1), \quad \forall z, \tz. \label{eq:MI_TV}
%\end{align}
%For \prettyref{eq:MI_TV1}, by definition,
%$$
%I(\sigma_S; A) =  \Expect_{\sigma_S} \qth{ D \left( \P_{A|\sigma_S} \| \P \right) }.
%$$
%Note that the distribution of $\sigma_S$ has a finite support
%and $\Omega(1)$ probability mass on each possible value. 
%Therefore, $I(\sigma_S; A)=o(1)$ if and only if 
%$D \left( \P_{A|\sigma_S=z} \| \P \right) = o(1)$ for all $z$.
%
%For \prettyref{eq:MI_TV2}, by Pinsker's inequality, $D \left( \P_{z} \| \P \right) = o(1)$
%implies $d_{\rm TV} \left( \P_{z}, \P \right) = o(1)$. Conversely, 
%suppose that $d_{\rm TV} \left( \P_{z}, \P \right) = o(1)$. Then
%\begin{align*}
%D \left( \P_{z} \| \P \right) &= \int \P_{z} \log \frac{\P_{z}  }{\P} \\
%& \overset{(a)}{\le} \int  \P_{z}  \frac{\P_{z} - \P }{\P} \\
%&  \le \int  \frac{ \P_{z} }{ \P } \left| \P_{z} - \P  \right| \\
%& \overset{(b)}{=} O(1) \times \int \left| \P_{z} - \P  \right| \\
%& = O\left( d_{TV}  \left( \P_{z}, \P \right) \right) = o(1),
%\end{align*}
%where $(a)$ is due to $\log x \le x-1$, and $(b)$ follows because 
%$\frac{ \P_{z}(a) }{ \P(a) } = \frac{ \P_{A|\sigma_S=z}(a) }{ \P_A(a)} \leq \frac{1}{\prob{\sigma_S=z}} = O(1)$ everywhere.
%%, and  the distribution of $\sigma_S$ has $\Omega(1)$ probability mass on each realization and hence 
%
%
%For \prettyref{eq:MI_TV}, suppose that $d_{\rm TV} \left( \P_{z}, \P_{\tz} \right) =o(1)$
%for all $z$ and $\tz$. By the convexity of $d_{\rm TV}(\cdot,\cdot)$ and Jensen's inequality, it readily
%follows that $d_{\rm TV} \left( \P_{z}, \P \right) =o(1)$.
%
%
%
%Next we prove that for any reference distribution $\Q$,  \prettyref{eq:second_moment_conditional} implies 
%$d_{\rm TV} \left( \P_{z}, \P_{\tz} \right) =o(1)$
%for all $z,\tz$, which further implies \prettyref{eq:MIcorrk} in view of \prettyref{eq:MI_TV}.
 %Indeed, by Cauchy-Schwartz inequality, we have
%\begin{align}
%d_{\rm TV}  \left( \P_{z}, \P_{\tz} \right) & =
%\frac{1}{2} \int  \left| \P_{z} - \P_{ \tz} \right| \nonumber \\
%%& = \frac{1}{2} \int  \left| \P_{z} - \P_{\tz} \right| \frac{\sqrt{\Q}}{\sqrt{\Q}} \nonumber \\
%& \le \frac{1}{2}  \left( \int \Q  \right)^{1/2} \left( \int  \frac{  \left( \P_{z} - \P_{\tz} \right)^2 }{\Q }  \right)^{1/2}
%\nonumber \\
%& = \frac{1}{2} \left(    \int \frac{\P^2_{A | z}}{\Q} +\int \frac{\P^2_{A | \tz}}{\Q} - 2 \int \frac{ 
%\P_{z} \P_{\tz} }{ \Q} \right)^{1/2} \overset{\prettyref{eq:second_moment_conditional}}{=} o(1).
%\end{align}
%%where the last equality holds from the assumption \prettyref{eq:second_moment_conditional}. 
%
%
%
%Finally, we consider the binary symmetric SBM and show that,
%below the correlated recovery threshold $\tau=\frac{(a-b)^2}{2(a+b)}<1$, 
 %\prettyref{eq:second_moment_conditional} is satisfied if the reference distribution $\Q$ is the distribution of $A$ in
%the null (\ER) model.
%Specifically, following the derivations in \prettyref{eq:SBM_second_moment_eq},
%we have
%\begin{align}
%\int \frac{   \P_{z}  \P_{\tz} }{ \Q } 
%&= \Expect \qth{  \prod_{i < j} \left(1 +  \sigma_i \sigma_j \tsigma_i \tsigma_j \rho \right)
%\mathrel{\bigg|} \sigma_S=z, \tsigma_S =\tz}  \nonumber \\
%& =  \left( 1+o(1) \right) e^{ -\tau^2/4 -\tau/2} \times 
%\Expect \qth{ \exp \left( \frac{\rho}{2} \iprod{ \sigma}{\tsigma}^2  \right) \mathrel{\Big|} \sigma_S=z, \tsigma_S =\tz},
%%& = \prod_{(i,j) \in \calE(S) } \left(1 +  \sigma_i \sigma_j \tsigma_i \tsigma_j \rho \right)  \nonumber \\
%%& \times \Expect \qth{ \prod_{i \in S, j \in S^c}  \left(1 +  \sigma_i \sigma_j \tsigma_i \tsigma_j \rho \right) \mid \sigma_S, \tsigma_S } \nonumber \\
%%& \times \Expect \qth{ \prod_{(i,j)\in \calE(S^c)}  \left(1 +  \sigma_i \sigma_j \tsigma_i \tsigma_j \rho \right) \mid \sigma_S, \tsigma_S }.
%\end{align}
%where the last equality holds because $m$ is a constant, $\rho=\tau/n + O(1/n^2)$ and $\log(1+x) = x -x^2/2 +O(x^3)$. 
%%\nbr{
%%I got $e^{ -\tau^2/4 - \tau/2}$, because 
%%$\sum_{i < j} \sigma_i \sigma_j \tsigma_i \tsigma_j = \frac{1}{2}(
%%\iprod{ \sigma}{\tsigma}^2 - n)$.}
%
%Write $\sigma=2\xi-1$ for $\xi\in\{0,1\}^n$ and let 
%$$
%H_1\triangleq \Iprod{\xi_{S}} { \tilde{\xi}_{S} } \quad \text{ and } \quad 
%H_2\triangleq \Iprod{\xi_{S^c}} { \tilde{\xi}_{S^c} }.
%$$ 
%Then $\iprod{\sigma}{\tsigma} = 4 (H_1+H_2) -n$.
%Moreover, conditional  on $\sigma_S$ and $\tsigma_S$, 
%$$
%H_2 \sim \text{Hypergeometric} \left( n-m, n/2 - \| \xi_S\|_1, n/2 - \|\tilde{\xi}_S \|_1 \right).
%$$
%Therefore
%\begin{align*}
%\Expect \qth{ \exp \left(  \frac{\rho}{2} \iprod{ \sigma}{\tsigma}^2  \right) \mathrel{\bigg|} \sigma_S=z, \tsigma_S =\tz}
%& =\expect{ \exp \left(  \frac{n\rho}{2} \left( \frac{ 4 H_1 + 4H_2 - n}{\sqrt{n} } \right)^2 \right) \mathrel{\bigg|} \sigma_S=z, \tsigma_S =\tz}  \\
%& = \frac{1+o(1)}{\sqrt{1-\tau}},
%\end{align*}
%where the last inequality holds because $n \rho = \tau +o(1/n)$ and 
%conditional on $\sigma_S$ and $\tsigma_S$,
%$ \frac{1}{\sqrt{n}} ( 4H_1 + 4H_2 - n )$ converges to $\calN(0,1)$ in distribution 
%as $n \to \infty$ by the central limit theorem for hypergeometric distribution. 
%
%In conclusion, we have shown that 
%$$
%\int \frac{   \P_{z}  \P_{\tz} }{ \Q}  
%= \left( 1+o(1) \right) e^{ -\tau^2/4 - \tau/2}  \frac{1}{\sqrt{1-\tau}}, \quad \forall z,\tz.
%$$
%%Hence, by taking the expectation of $\sigma_S$ and $\tsigma_S$ over the both hand sides of the last displayed equation, we get that
%Averaging both sides over $z,\tz$ yields
%$$
%\int \frac{   \P^2 }{ \Q}  
%= \left( 1+o(1) \right) e^{ -\tau^2/4 - \tau/2}  \frac{1}{\sqrt{1-\tau}}.
%$$
%Thus \prettyref{eq:second_moment_conditional} is satisfied. 




\section{Proof of \prettyref{eq:second_moment_conditional} $\implies$ \prettyref{eq:MIcorr2} and 
verification of  \prettyref{eq:second_moment_conditional} in the binary symmetric SBM}
\label{app:MITV}
Combining \prettyref{eq:ss} with 
\prettyref{eq:TI} and \prettyref{eq:Tbern}, we have
$I(\sigma_1,\sigma_2; A) = o(1)$ if and only if $\TV(\P_+,\P_-) =o(1)$,
where $\P_+=P_{A|\sigma_1=\sigma_2}$ and $\P_-=P_{A|\sigma_1\neq\sigma_2}$.
%Next we prove that for any reference distribution $\Q$,  \prettyref{eq:second_moment_conditional} implies 
%$d_{\rm TV} \left( \P_{+}, \P_{-} \right) =o(1)$.
Note the following characterization about the total variation distance, which simply follows from the Cauchy-Schwartz inequality:
\begin{equation}
\TV(\P_+,\P_-) = \frac{1}{2} \sqrt{\inf_{\Q}  \int  \frac{  \left( \P_{+} - \P_{-} \right)^2 }{\Q }}
\label{eq:TVquadratic}
\end{equation}
where the infimum is taken over all probability distributions $\Q$. 
Therefore \prettyref{eq:second_moment_conditional} implies \prettyref{eq:MIcorr2}.

%
 %Indeed, by Cauchy-Schwartz inequality, for any $\Q$, we have
%\begin{align}
%d_{\rm TV}  \left( \P_{+}, \P_{-} \right) & =
%\frac{1}{2} \int  \left| \P_{+} - \P_{-} \right| \nonumber \\
%%& = \frac{1}{2} \int  \left| \P_{z} - \P_{\tz} \right| \frac{\sqrt{\Q}}{\sqrt{\Q}} \nonumber \\
%& \le \frac{1}{2}  
%\left( \int \Q  \right)^{1/2} \left( \int  \frac{  \left( \P_{+} - \P_{-} \right)^2 }{\Q }  \right)^{1/2}
%\nonumber \\
%%& = \frac{1}{2} \left(    \int \frac{\P^2_{+}}{\Q} +\int \frac{\P^2_{-}}{\Q} - 2 \int \frac{ 
%%\P_{+} \P_{-} }{ \Q} \right)^{1/2} 
%&\overset{\prettyref{eq:second_moment_conditional}}{=} o(1).
%\end{align}
%%where the last equality holds from the assumption \prettyref{eq:second_moment_conditional}. 



Finally, we consider the binary symmetric SBM and show that,
below the correlated recovery threshold $\tau=\frac{(a-b)^2}{2(a+b)}<1$, 
 \prettyref{eq:second_moment_conditional} is satisfied if the reference distribution $\Q$ is the distribution of $A$ in
the null (\ER) model. Note that 
$$
\int  \frac{  \left( \P_{+} - \P_{-} \right)^2 }{\Q }  =
\int \frac{\P^2_{+}}{\Q} +\int \frac{\P^2_{-}}{\Q} - 2 \int \frac{ \P_{+} \P_{-} }{ \Q}.
$$
%Hence, it is sufficient to show
%$$
 %\int \frac{ \P^2 }{ \Q} = O(1), \; \text{ and } 
 %\int \frac{ \P_{z} \P_{\tilde{z} } }{ \Q} =  
%(1+o(1)) \int \frac{ \P^2 }{ \Q}, \quad  \forall z, \tilde{z} \in \{\pm \}.
%$$
Hence, it is sufficient to show
$$
 \int \frac{ \P_{z} \P_{\tilde{z} } }{ \Q} =  
C+o(1), \quad  \forall z, \tilde{z} \in \{\pm \}
$$
for some constant $C$ independent of $z$ and $\tz$.
Specifically, following the derivations in \prettyref{eq:SBM_second_moment_eq},
we have
\begin{align}
\int \frac{   \P_{z}  \P_{\tz} }{ \Q } 
&= \Expect \qth{  \prod_{i < j} \left(1 +  \sigma_i \sigma_j \tsigma_i \tsigma_j \rho \right)
\mathrel{\bigg|} \sigma_1 \sigma_2=z, \tsigma_1 \tsigma_2 =\tz }  \nonumber \\
& =  \left( 1+o(1) \right) e^{ -\tau^2/4 -\tau/2} \times 
\Expect \qth{ \exp \left( \frac{\rho}{2} \iprod{ \sigma}{\tsigma}^2  \right) \mathrel{\Big|} \sigma_1 \sigma_2=z, \tsigma_1 \tsigma_2 =\tz },
%& = \prod_{(i,j) \in \calE(S) } \left(1 +  \sigma_i \sigma_j \tsigma_i \tsigma_j \rho \right)  \nonumber \\
%& \times \Expect \qth{ \prod_{i \in S, j \in S^c}  \left(1 +  \sigma_i \sigma_j \tsigma_i \tsigma_j \rho \right) \mid \sigma_S, \tsigma_S } \nonumber \\
%& \times \Expect \qth{ \prod_{(i,j)\in \calE(S^c)}  \left(1 +  \sigma_i \sigma_j \tsigma_i \tsigma_j \rho \right) \mid \sigma_S, \tsigma_S }.
\end{align}
where the last equality holds $\rho=\tau/n + O(1/n^2)$ and $\log(1+x) = x -x^2/2 +O(x^3)$. 
%\nbr{
%I got $e^{ -\tau^2/4 - \tau/2}$, because 
%$\sum_{i < j} \sigma_i \sigma_j \tsigma_i \tsigma_j = \frac{1}{2}(
%\iprod{ \sigma}{\tsigma}^2 - n)$.}

Write $\sigma=2\xi-1$ for $\xi\in\{0,1\}^n$ and let 
$$
H_1\triangleq \xi_1 \tilde{\xi}_1 + \xi_2 \tilde{\xi}_2 
\quad \text{ and } \quad 
H_2 \triangleq \sum_{j \ge 3}^n \xi_j \tilde{\xi}_j.
$$ 
Then $\iprod{\sigma}{\tsigma} = 4 (H_1+H_2) -n$.
Moreover, conditional  on $\sigma_1, \sigma_2$ and $\tsigma_1, \tsigma_2$, 
$$
H_2 \sim \text{Hypergeometric} \left( n-2, n/2 - \xi_1-\xi_2, n/2 - \tilde{\xi}_1-\tilde{\xi}_2 \right).
$$
Since $|H_1| \le 2$, $\xi_1+\xi_2 \le 2 $, and $\tilde{\xi}_1+\tilde{\xi}_2 \le 2$, 
it follows that 
conditional on $\sigma_1 \sigma_2=z, \tsigma_1 \tsigma_2 =\tz $,
$ \frac{1}{\sqrt{n}} ( 4H_1 + 4H_2 - n )$ converges to $\calN(0,1)$ in distribution 
as $n \to \infty$ by the central limit theorem for hypergeometric distribution. 
Therefore
\begin{align*}
& \Expect \qth{ \exp \left(  \frac{\rho}{2} \iprod{ \sigma}{\tsigma}^2  \right) \mathrel{\bigg|} \sigma_S=z, \tsigma_S =\tz} \\
& =\expect{ \exp \left(  \frac{n\rho}{2} \left( \frac{ 4 H_1 + 4H_2 - n}{\sqrt{n} } \right)^2 \right) \mathrel{\bigg|} 
\sigma_1 \sigma_2=z, \tsigma_1 \tsigma_2 =\tz }  \\
& = \frac{1+o(1)}{\sqrt{1-\tau} },
\end{align*}
where the last equality holds due to $n \rho = \tau +o(1/n)$, $\tau<1$, 
and the convergence of the moment generating function. 

%In conclusion, we have shown that 
%$$
%\int \frac{   \P_{z}  \P_{\tz} }{ \Q}  
%= \left( 1+o(1) \right) e^{ -\tau^2/4 - \tau/2}  \frac{1}{\sqrt{1-\tau}}, \quad \forall z,\tz \in \{ \pm \}.
%$$
%%Hence, by taking the expectation of $\sigma_S$ and $\tsigma_S$ over the both hand sides of the last displayed equation, we get that
%Averaging both sides over $z,\tz$ yields
%$$
%\int \frac{   \P^2 }{ \Q}  
%= \left( 1+o(1) \right) e^{ -\tau^2/4 - \tau/2}  \frac{1}{\sqrt{1-\tau}}.
%$$
%Thus \prettyref{eq:second_moment_conditional} is satisfied.


\end{appendices}



 \documentclass[letterpaper,11pt]{article}
\usepackage[toc,page]{appendix}
\usepackage[margin=1in]{geometry}
\usepackage[bookmarks, colorlinks=true, plainpages = false, citecolor = blue,linkcolor=red,urlcolor = blue, filecolor = blue,pagebackref]{hyperref}
%% Note, I added pagebackref to the options for hyperref so that page number back references appear after each paper listing.   -BH 4/29/15
%\usepackage{url}\urlstyle{rm}
\usepackage{amsmath,amsfonts,amsthm,amssymb,bm, verbatim,dsfont,mathtools}
%\usepackage{algorithm,algorithmic}
\usepackage{color,graphicx,appendix}
\usepackage{subfigure}
\usepackage{etoolbox}
\usepackage{tikz}
\usepackage{xr,xspace}
\usepackage{todonotes}
\usepackage{paralist}
\usepackage{caption,soul}
%\usepackage[ruled,vlined]{algorithm2e}
\usepackage{algorithm}% http://ctan.org/pkg/algorithms
\usepackage{algorithmic}% http://ctan.org/pkg/algorithms
\usepackage{enumitem}
\makeatletter
%\renewcommand{\ALG@name}{SDP}
%\renewcommand{\listalgorithmname}{List of \ALG@name s}

%%%%% THEOREM STYLE DEFINITIONS
%\theoremstyle{plain}
%\newtheorem{theorem}{Theorem}
%\newtheorem{lemma}{Lemma}
%\newtheorem{proposition}{Proposition}
%\newtheorem{corollary}{Corollary}
%\theoremstyle{definition}
%\newtheorem{definition}{Definition}
%\newtheorem{hypothesis}{Hypothesis}
%\newtheorem{conjecture}{Conjecture}
%\newtheorem{question}{Question}
%\newtheorem{remark}{Remark}
%\newtheorem*{remark*}{Remark}

\newtheorem{theorem}{Theorem}
\newtheorem{lemma}{Lemma}
\newtheorem{proposition}{Proposition}
\newtheorem{corollary}{Corollary}
\theoremstyle{definition}
\newtheorem{definition}{Definition}
\newtheorem{hypothesis}{Hypothesis}
\newtheorem{conjecture}{Conjecture}
\newtheorem{question}{Question}
\newtheorem{remark}{Remark}
\newtheorem{assumption}{Assumption}


%\usepackage{enumerate}

\usepackage{tikz}

\usepackage{xspace,prettyref}
\usepackage{bm}
% for prettyref.sty
\newrefformat{eq}{(\ref{#1})}
\newrefformat{chap}{Chapter~\ref{#1}}
\newrefformat{sec}{Section~\ref{#1}}
\newrefformat{alg}{Algorithm~\ref{#1}}
\newrefformat{fig}{Fig.~\ref{#1}}
\newrefformat{tab}{Table~\ref{#1}}
\newrefformat{rmk}{Remark~\ref{#1}}
\newrefformat{clm}{Claim~\ref{#1}}
\newrefformat{def}{Definition~\ref{#1}}
\newrefformat{cor}{Corollary~\ref{#1}}
\newrefformat{lmm}{Lemma~\ref{#1}}
\newrefformat{prop}{Proposition~\ref{#1}}
\newrefformat{app}{Appendix~\ref{#1}}
\newrefformat{hyp}{Hypothesis~\ref{#1}}
\newrefformat{thm}{Theorem~\ref{#1}}
\newrefformat{ass}{Assumption~\ref{#1}}
\newrefformat{conj}{Conjecture~\ref{#1}}

\newcommand{\ie}{i.e.\xspace}
\renewcommand{\P}{\mathcal{P} }
\newcommand{\Q}{\mathcal{Q}}
\newcommand{\Exp}{\mathbb{E}}
\newcommand{\contig}{\trianglelefteq}
\newcommand{\indicator}[1]{\bm{1}_{#1}}
\renewcommand{\hat}{\widehat}
\renewcommand{\tilde}{\widetilde}
\newcommand{\argmax}{\mathrm{argmax}}

\usepackage{color}
\newcommand{\red}{\color{red}}
\newcommand{\blue}{\color{blue}}
\newcommand{\nb}[1]{{\sf\blue[#1]}}
\newcommand{\nbr}[1]{{\sf\red #1}}

\renewcommand{\implies}{\Rightarrow}
\newcommand{\1}[1]{{\mathbf{1}_{\left\{{#1}\right\}}}}
\newcommand{\post}[2]{\begin{center} \includegraphics[width=#2]{#1} \end{center} }
\newcommand \E[1]{\mathbb{E}[#1]}
%\newcommand{\Perp}{\perp \! \! \! \perp}
\newcommand{\Perp}{\perp}
\newcommand{\Hyper}{\text{Hypergeometric}}
%\newcommand \P[1]{\mathbb{P}[#1]}


%% Wu
\newcommand{\bfa}{{\mathbf{a}}}
\newcommand{\bfb}{{\mathbf{b}}}
\newcommand{\bfc}{{\mathbf{c}}}
\newcommand{\bfd}{{\mathbf{d}}}
\newcommand{\bfe}{{\mathbf{e}}}
\newcommand{\bff}{{\mathbf{f}}}
\newcommand{\bfg}{{\mathbf{g}}}
\newcommand{\bfh}{{\mathbf{h}}}
\newcommand{\bfi}{{\mathbf{i}}}
\newcommand{\bfj}{{\mathbf{j}}}
\newcommand{\bfk}{{\mathbf{k}}}
\newcommand{\bfl}{{\mathbf{l}}}
\newcommand{\bfm}{{\mathbf{m}}}
\newcommand{\bfn}{{\mathbf{n}}}
\newcommand{\bfo}{{\mathbf{o}}}
\newcommand{\bfp}{{\mathbf{p}}}
\newcommand{\bfq}{{\mathbf{q}}}
\newcommand{\bfr}{{\mathbf{r}}}
\newcommand{\bfs}{{\mathbf{s}}}
\newcommand{\bft}{{\mathbf{t}}}
\newcommand{\bfu}{{\mathbf{u}}}
\newcommand{\bfv}{{\mathbf{v}}}
\newcommand{\bfw}{{\mathbf{w}}}
\newcommand{\bfx}{{\mathbf{x}}}
\newcommand{\bfy}{{\mathbf{y}}}
\newcommand{\bfz}{{\mathbf{z}}}
\newcommand{\bfA}{{\mathbf{A}}}
\newcommand{\bfB}{{\mathbf{B}}}
\newcommand{\bfC}{{\mathbf{C}}}
\newcommand{\bfD}{{\mathbf{D}}}
\newcommand{\bfE}{{\mathbf{E}}}
\newcommand{\bfF}{{\mathbf{F}}}
\newcommand{\bfG}{{\mathbf{G}}}
\newcommand{\bfH}{{\mathbf{H}}}
\newcommand{\bfI}{{\mathbf{I}}}
\newcommand{\bfJ}{{\mathbf{J}}}
\newcommand{\bfK}{{\mathbf{K}}}
\newcommand{\bfL}{{\mathbf{L}}}
\newcommand{\bfM}{{\mathbf{M}}}
\newcommand{\bfN}{{\mathbf{N}}}
\newcommand{\bfO}{{\mathbf{O}}}
\newcommand{\bfP}{{\mathbf{P}}}
\newcommand{\bfQ}{{\mathbf{Q}}}
\newcommand{\bfR}{{\mathbf{R}}}
\newcommand{\bfS}{{\mathbf{S}}}
\newcommand{\bfT}{{\mathbf{T}}}
\newcommand{\bfU}{{\mathbf{U}}}
\newcommand{\bfV}{{\mathbf{V}}}
\newcommand{\bfW}{{\mathbf{W}}}
\newcommand{\bfX}{{\mathbf{X}}}
\newcommand{\bfY}{{\mathbf{Y}}}
\newcommand{\bfZ}{{\mathbf{Z}}}

\newcommand{\bbA}{{\mathbb{A}}}
\newcommand{\bbB}{{\mathbb{B}}}
\newcommand{\bbC}{{\mathbb{C}}}
\newcommand{\bbD}{{\mathbb{D}}}
\newcommand{\bbE}{{\mathbb{E}}}
\newcommand{\bbF}{{\mathbb{F}}}
\newcommand{\bbG}{{\mathbb{G}}}
\newcommand{\bbH}{{\mathbb{H}}}
\newcommand{\bbI}{{\mathbb{I}}}
\newcommand{\bbJ}{{\mathbb{J}}}
\newcommand{\bbK}{{\mathbb{K}}}
\newcommand{\bbL}{{\mathbb{L}}}
\newcommand{\bbM}{{\mathbb{M}}}
\newcommand{\bbN}{{\mathbb{N}}}
\newcommand{\bbO}{{\mathbb{O}}}
\newcommand{\bbP}{{\mathbb{P}}}
\newcommand{\bbQ}{{\mathbb{Q}}}
\newcommand{\bbR}{{\mathbb{R}}}
\newcommand{\bbS}{{\mathbb{S}}}
\newcommand{\bbT}{{\mathbb{T}}}
\newcommand{\bbU}{{\mathbb{U}}}
\newcommand{\bbV}{{\mathbb{V}}}
\newcommand{\bbW}{{\mathbb{W}}}
\newcommand{\bbX}{{\mathbb{X}}}
\newcommand{\bbY}{{\mathbb{Y}}}
\newcommand{\bbZ}{{\mathbb{Z}}}
                         
                         
\newcommand{\sfa}{{\mathsf{a}}}
\newcommand{\sfb}{{\mathsf{b}}}
\newcommand{\sfc}{{\mathsf{c}}}
\newcommand{\sfd}{{\mathsf{d}}}
\newcommand{\sfe}{{\mathsf{e}}}
\newcommand{\sff}{{\mathsf{f}}}
\newcommand{\sfg}{{\mathsf{g}}}
\newcommand{\sfh}{{\mathsf{h}}}
\newcommand{\sfi}{{\mathsf{i}}}
\newcommand{\sfj}{{\mathsf{j}}}
\newcommand{\sfk}{{\mathsf{k}}}
\newcommand{\sfl}{{\mathsf{l}}}
\newcommand{\sfm}{{\mathsf{m}}}
\newcommand{\sfn}{{\mathsf{n}}}
\newcommand{\sfo}{{\mathsf{o}}}
\newcommand{\sfp}{{\mathsf{p}}}
\newcommand{\sfq}{{\mathsf{q}}}
\newcommand{\sfr}{{\mathsf{r}}}
\newcommand{\sfs}{{\mathsf{s}}}
\newcommand{\sft}{{\mathsf{t}}}
\newcommand{\sfu}{{\mathsf{u}}}
\newcommand{\sfv}{{\mathsf{v}}}
\newcommand{\sfw}{{\mathsf{w}}}
\newcommand{\sfx}{{\mathsf{x}}}
\newcommand{\sfy}{{\mathsf{y}}}
\newcommand{\sfz}{{\mathsf{z}}}
\newcommand{\sfA}{{\mathsf{A}}}
\newcommand{\sfB}{{\mathsf{B}}}
\newcommand{\sfC}{{\mathsf{C}}}
\newcommand{\sfD}{{\mathsf{D}}}
\newcommand{\sfE}{{\mathsf{E}}}
\newcommand{\sfF}{{\mathsf{F}}}
\newcommand{\sfG}{{\mathsf{G}}}
\newcommand{\sfH}{{\mathsf{H}}}
\newcommand{\sfI}{{\mathsf{I}}}
\newcommand{\sfJ}{{\mathsf{J}}}
\newcommand{\sfK}{{\mathsf{K}}}
\newcommand{\sfL}{{\mathsf{L}}}
\newcommand{\sfM}{{\mathsf{M}}}
\newcommand{\sfN}{{\mathsf{N}}}
\newcommand{\sfO}{{\mathsf{O}}}
\newcommand{\sfP}{{\mathsf{P}}}
\newcommand{\sfQ}{{\mathsf{Q}}}
\newcommand{\sfR}{{\mathsf{R}}}
\newcommand{\sfS}{{\mathsf{S}}}
\newcommand{\sfT}{{\mathsf{T}}}
\newcommand{\sfU}{{\mathsf{U}}}
\newcommand{\sfV}{{\mathsf{V}}}
\newcommand{\sfW}{{\mathsf{W}}}
\newcommand{\sfX}{{\mathsf{X}}}
\newcommand{\sfY}{{\mathsf{Y}}}
\newcommand{\sfZ}{{\mathsf{Z}}}

\newcommand{\TV}{d_{\rm TV}}

\newcommand{\floor}[1]{{\left\lfloor {#1} \right \rfloor}}
\newcommand{\ceil}[1]{{\left\lceil {#1} \right \rceil}}

\usepackage{xspace,prettyref}
\newcommand{\CML}{\widehat{C}_{\rm ML}}
\newcommand{\diverge}{\to\infty}
\newcommand{\eqdistr}{{\stackrel{\rm (d)}{=}}}
\newcommand{\iiddistr}{{\stackrel{\text{\iid}}{\sim}}}
\newcommand{\ones}{\mathbf 1}
\newcommand{\zeros}{\mathbf 0}
\newcommand{\reals}{{\mathbb{R}}}
\newcommand{\integers}{{\mathbb{Z}}}
\newcommand{\naturals}{{\mathbb{N}}}
\newcommand{\rationals}{{\mathbb{Q}}}
\newcommand{\naturalsex}{\overline{\mathbb{N}}}
\newcommand{\symm}{{\mbox{\bf S}}}  % symmetric matrices
\newcommand{\supp}{{\rm supp}}
\newcommand{\eexp}{{\rm e}}
\newcommand{\rexp}[1]{{\rm e}^{#1}}
\newcommand{\identity}{\mathbf I}
\newcommand{\allones}{\mathbf J}
%\newcommand{\zeros}{\mathbf 0}

\newcommand{\diff}{{\rm d}}

\newcommand{\Expect}{\mathbb{E}}
\newcommand{\expect}[1]{\mathbb{E}\left[ #1 \right]}
\newcommand{\eexpect}[1]{\mathbb{E}[ #1 ]}
\newcommand{\expects}[2]{\mathbb{E}_{#2}\left[ #1 \right]}
\newcommand{\tExpect}{{\tilde{\mathbb{E}}}}
%\newcommand{\Prob}{\mathop{\mathbb{P}}}
\newcommand{\Prob}{\mathbb{P}}
\newcommand{\pprob}[1]{ \mathbb{P}\{ #1 \} }
\newcommand{\prob}[1]{ \mathbb{P}\left\{ #1 \right\} }
\newcommand{\tProb}{{\tilde{\mathbb{P}}}}
\newcommand{\tprob}[1]{{ \tProb\left\{ #1 \right\} }}
\newcommand{\hProb}{\widehat{\mathbb{P}}}
\newcommand{\probs}[2]{\mathbb{P}_{#2}\left\{ #1 \right\} }
\newcommand{\toprob}{\xrightarrow{\Prob}}
\newcommand{\tolp}[1]{\xrightarrow{L^{#1}}}
\newcommand{\toas}{\xrightarrow{{\rm a.s.}}}
\newcommand{\toae}{\xrightarrow{{\rm a.e.}}}
\newcommand{\todistr}{\xrightarrow{{\rm D}}}
\newcommand{\toweak}{\rightharpoonup}
\newcommand{\var}{\mathsf{var}}
\newcommand{\Cov}{\text{Cov}}
\newcommand\indep{\protect\mathpalette{\protect\independenT}{\perp}}
\def\independenT#1#2{\mathrel{\rlap{$#1#2$}\mkern2mu{#1#2}}}
\newcommand{\Bern}{{\rm Bern}}
\newcommand{\Binom}{{\rm Binom}}
\newcommand{\Pois}{{\rm Pois}}
\newcommand{\Hyp}{{\rm Hyp}}
\newcommand{\eg}{e.g.\xspace}
\newcommand{\iid}{i.i.d.\xspace}
% for prettyref.sty
\newrefformat{eq}{(\ref{#1})}
\newrefformat{chap}{Chapter~\ref{#1}}
\newrefformat{sec}{Section~\ref{#1}}
\newrefformat{alg}{Algorithm~\ref{#1}}
\newrefformat{fig}{Fig.~\ref{#1}}
\newrefformat{tab}{Table~\ref{#1}}
\newrefformat{rmk}{Remark~\ref{#1}}
\newrefformat{clm}{Claim~\ref{#1}}
\newrefformat{def}{Definition~\ref{#1}}
\newrefformat{cor}{Corollary~\ref{#1}}
\newrefformat{lmm}{Lemma~\ref{#1}}
\newrefformat{prop}{Proposition~\ref{#1}}
\newrefformat{app}{Appendix~\ref{#1}}
\newrefformat{hyp}{Hypothesis~\ref{#1}}
\newrefformat{thm}{Theorem~\ref{#1}}
\newrefformat{ass}{Assumption~\ref{#1}}
\newcommand{\ntok}[2]{{#1,\ldots,#2}}
\newcommand{\pth}[1]{\left( #1 \right)}
\newcommand{\qth}[1]{\left[ #1 \right]}
\newcommand{\sth}[1]{\left\{ #1 \right\}}
\newcommand{\bpth}[1]{\Bigg( #1 \Bigg)}
\newcommand{\bqth}[1]{\Bigg[ #1 \Bigg]}
\newcommand{\bsth}[1]{\Bigg\{ #1 \Bigg\}}
\newcommand{\norm}[1]{\left\|{#1} \right\|}
\newcommand{\lpnorm}[1]{\left\|{#1} \right\|_{p}}
\newcommand{\linf}[1]{\left\|{#1} \right\|_{\infty}}
\newcommand{\lnorm}[2]{\left\|{#1} \right\|_{{#2}}}
\newcommand{\Lploc}[1]{L^{#1}_{\rm loc}}
\newcommand{\hellinger}{d_{\rm H}}
\newcommand{\Fnorm}[1]{\lnorm{#1}{\rm F}}
\newcommand{\fnorm}[1]{\|#1\|_{\rm F}}
% \newcommand{\opnorm}[1]{\lnorm{#1}{\rm op}}
\newcommand{\opnorm}[1]{\left\| #1 \right\|_2}
% inner product
\newcommand{\iprod}[2]{\left \langle #1, #2 \right\rangle}
\newcommand{\Iprod}[2]{\langle #1, #2 \rangle}
% 12/02/2007
\newcommand{\indc}[1]{{\mathbf{1}_{\left\{{#1}\right\}}}}
\newcommand{\Indc}{\mathbf{1}}
\newcommand{\diag}[1]{\mathsf{diag} \left\{ {#1} \right\} }
\newcommand{\degr}{\mathsf{deg} }

\newcommand{\dTV}{d_{\rm TV}}
\newcommand{\tb}{\widetilde{b}}
\newcommand{\tr}{\widetilde{r}}
\newcommand{\tc}{\widetilde{c}}
\newcommand{\tu}{{\widetilde{u}}}
\newcommand{\tv}{{\widetilde{v}}}
\newcommand{\tx}{{\widetilde{x}}}
\newcommand{\ty}{{\widetilde{y}}}
\newcommand{\tz}{{\widetilde{z}}}
\newcommand{\tA}{{\widetilde{A}}}
\newcommand{\tB}{{\widetilde{B}}}
\newcommand{\tC}{{\widetilde{C}}}
\newcommand{\tD}{{\widetilde{D}}}
\newcommand{\tE}{{\widetilde{E}}}
\newcommand{\tF}{{\widetilde{F}}}
\newcommand{\tG}{{\widetilde{G}}}
\newcommand{\tH}{{\widetilde{H}}}
\newcommand{\tI}{{\widetilde{I}}}
\newcommand{\tJ}{{\widetilde{J}}}
\newcommand{\tK}{{\widetilde{K}}}
\newcommand{\tL}{{\widetilde{L}}}
\newcommand{\tM}{{\widetilde{M}}}
\newcommand{\tN}{{\widetilde{N}}}
\newcommand{\tO}{{\widetilde{O}}}
\newcommand{\tP}{{\widetilde{P}}}
\newcommand{\tQ}{{\widetilde{Q}}}
\newcommand{\tR}{{\widetilde{R}}}
\newcommand{\tS}{{\widetilde{S}}}
\newcommand{\tT}{{\widetilde{T}}}
\newcommand{\tU}{{\widetilde{U}}}
\newcommand{\tV}{{\widetilde{V}}}
\newcommand{\tW}{{\widetilde{W}}}
\newcommand{\tX}{{\widetilde{X}}}
\newcommand{\tY}{{\widetilde{Y}}}
\newcommand{\tZ}{{\widetilde{Z}}}


\newcommand{\wh}{\widehat}
\newcommand{\wt}{\widetilde}
\newcommand{\whp}{\rm w.h.p.}
\newcommand{\wpal}{\rm with probability at least~}
\newcommand{\pc}{Planted Clique\xspace}


\newcommand{\calA}{{\mathcal{A}}}
\newcommand{\calB}{{\mathcal{B}}}
\newcommand{\calC}{{\mathcal{C}}}
\newcommand{\calD}{{\mathcal{D}}}
\newcommand{\calE}{{\mathcal{E}}}
\newcommand{\calF}{{\mathcal{F}}}
\newcommand{\calG}{{\mathcal{G}}}
\newcommand{\calH}{{\mathcal{H}}}
\newcommand{\calI}{{\mathcal{I}}}
\newcommand{\calJ}{{\mathcal{J}}}
\newcommand{\calK}{{\mathcal{K}}}
\newcommand{\calL}{{\mathcal{L}}}
\newcommand{\calM}{{\mathcal{M}}}
\newcommand{\calN}{{\mathcal{N}}}
\newcommand{\calO}{{\mathcal{O}}}
\newcommand{\calP}{{\mathcal{P}}}
\newcommand{\calQ}{{\mathcal{Q}}}
\newcommand{\calR}{{\mathcal{R}}}
\newcommand{\calS}{{\mathcal{S}}}
\newcommand{\calT}{{\mathcal{T}}}
\newcommand{\calU}{{\mathcal{U}}}
\newcommand{\calV}{{\mathcal{V}}}
\newcommand{\calW}{{\mathcal{W}}}
\newcommand{\calX}{{\mathcal{X}}}
\newcommand{\calY}{{\mathcal{Y}}}
\newcommand{\calZ}{{\mathcal{Z}}}

\newcommand{\comp}[1]{{#1^{\rm c}}}
\newcommand{\Leb}{{\rm Leb}}
\newcommand{\Th}{{\rm th}}

\newcommand{\PDS}{{\sf PDS}\xspace}
\newcommand{\PC}{{\sf PC}\xspace}
\newcommand{\BPDS}{{\sf BPDS}\xspace}
\newcommand{\BPC}{{\sf BPC}\xspace}
\newcommand{\DKS}{{\sf DKS}\xspace}
\newcommand{\ML}{{\rm ML}\xspace}
\newcommand{\SDP}{{\rm SDP}\xspace}
\newcommand{\SBM}{{\sf SBM}\xspace}

\newcommand{\ER}{Erd\H{o}s-R\'enyi\xspace}

\newcommand{\Tr}{\mathsf{Tr}}
\renewcommand{\hat}{\widehat}
\renewcommand{\tilde}{\widetilde}

\newcommand{\MMSE}{{\rm MMSE}}
\newcommand{\snr}{{\mathsf{snr}}}

\newcommand{\tsigma}{\tilde{\sigma}}

\newcommand{\planted}{\sigma^*}
\newcommand{\score}{\calT}

%\usepackage{mleftright}

  \begin{document}
	


\title{Statistical Problems with Planted Structures: Information-Theoretical and Computational Limits}

\date{\today}
\author{ 
Yihong Wu \and Jiaming Xu\thanks{
%This research was supported by the National Science Foundation under
%Grant ECCS 10-28464, IIS-1447879, and CCF-1423088, and
%Strategic Research
%Initiative on Big-Data Analytics of the College of Engineering
%at the University of Illinois, and DOD ONR Grant N00014-14-1-0823, and Grant 328025 from the Simons Foundation. 
Y.~Wu is with Department of Statistics and Data Science, 
Yale University, New Haven, CT 06520, USA, \texttt{yihong.wu@yale.edu}.
J.~Xu is with the Fuqua School of Business, Duke University,
Durham, NC 27708, \texttt{jiaming.xu868@duke.edu}.}
%Krannert School of Management, Purdue University,
%       West Lafayette, IN 47907, USA, \texttt{xu972@purdue.edu}.}
}
  
  \maketitle

\begin{abstract}
Over the past few years, insights from computer science, statistical physics, and information theory have revealed phase transitions in a wide array of high-dimensional statistical problems at two distinct thresholds: One is the information-theoretical (IT) threshold below which the observation is too noisy so that inference of the ground truth structure is impossible regardless of the computational cost; the other is the computational threshold above which inference can be performed efficiently, i.e., in time that is polynomial in the input size. In the intermediate regime, inference is information-theoretically possible, but conjectured to be computationally hard.

This article provides a survey of the common techniques for determining the sharp IT and computational limits, using community detection and submatrix detection as illustrating examples. For IT limits, we discuss tools including the first and second moment method for analyzing the maximum likelihood estimator, information-theoretic methods for proving impossibility results using  mutual information and rate-distortion theory, and methods originated from statistical physics such as interpolation method. To investigate computational limits, we describe a common recipe to construct a randomized polynomial-time reduction scheme that approximately maps instances of the planted clique problem to the problem of interest in total variation distance.
\end{abstract}

\tableofcontents
  
 \pdfoutput=1
\documentclass{article}
\usepackage[final]{pdfpages}
\begin{document}
\includepdf[pages=1-9]{CVPR18VOlearner.pdf}
\includepdf[pages=1-last]{supp.pdf}
\end{document}

\begin{appendices}

\section{Mutual information-characterization of correlated recovery}
\label{app:MIcorr}

\newcommand{\Unif}{\mathrm{Unif}}


We consider a general setup:  Let the number of communities $k$ be a constant. Denote the membership vector by $\sigma=(\sigma_1,\ldots,\sigma_n) \in [k]^n$ and the observation is $A=(A_{ij}: 1 \leq i < j \leq n)$. Assume the following conditions:
\begin{enumerate}[label=A\arabic*]
	%[{A}1.]
	\item \label{A1}
	For any permutation $\pi\in S_k$, $(\sigma,A)$ and $(\pi(\sigma),A)$ are equal in law, where $\pi(\sigma)\triangleq (\pi(\sigma_1),\ldots,\pi(\sigma_n))$; 
	
	\item \label{A2}
	For any $i \neq j \in [n]$, $I(\sigma_i,\sigma_j;A)= I(\sigma_1,\sigma_2;A)$;
	
	\item \label{A3}
	For any $z_1,z_2 \in [k]$, $\prob{\sigma_1=z_1,\sigma_2=z_2} = \frac{1}{k^2} + o(1)$ as $n\to \infty$.
\end{enumerate}
These assumptions are satisfied for example for $k$-community SBM (where each pair of vertices $i$ and $j$ are connected independently with probability $p$ if $\sigma_i=\sigma_j$ and $q$ otherwise),\index{Stochastic block model (SBM)! $k$ communities}
 and the membership vector
$\sigma$ can be either uniformly distributed on $[k]^n$ or the set of equal-sized $k$-partition of $[n]$. 

Recall that correlated recovery entails the following:
For any $\sigma , \hat{\sigma} \in [k]^n$, define the overlap:
\begin{align}
o\left( \sigma, \hat{\sigma} \right) = \frac{1}{n} \max_{\pi \in S_k} 
\sum_{i \in [n] } \left( \indc{ \pi\left(\sigma_i \right) = \hat{\sigma}_i} - \frac{1}{k} \right).
\end{align}
We say an estimator $\hat{\sigma} = \hat{\sigma}(A)$ achieves correlated recovery if\footnote{For the special case of $k=2$, \prettyref{eq:corro} is equivalent to 
$\frac{1}{n}\Expect[|\Iprod{\sigma}{\hat \sigma}|] = \Omega(1)$, where $\sigma , \hat{\sigma}$ are assumed to be $\{\pm\}^n$-valued.}
\begin{equation}
\expect{o\left( \sigma, \hat{\sigma} \right)}=\Omega(1),
\label{eq:corro}
\end{equation}
 that is, the misclassification rate, up to a global permutation, outperforms random guessing.
Under the above three assumptions, we have the following characterization of correlated recovery:
\begin{lemma}
\label{lmm:MIcorr}	
Correlated recovery is possible if and only if 
$I(\sigma_1, \sigma_2 ; A) = \Omega(1)$.	
\end{lemma}



\begin{proof}
We start by recalling the relation between mutual information and total variation.
For any pair of random variables $(X,Y)$, define the so-called $T$-information \cite{Csiszar96}:
$T(X;Y) \triangleq \TV(P_{XY}, P_XP_Y) = \Expect[\TV(P_{Y|X}, P_Y)]$.
For $X \sim \Bern(p)$, this simply reduces to 
\begin{equation}
T(X;Y) = 2p(1-p) \TV(P_{Y|X=0}, P_{Y|X=1}).
\label{eq:Tbern}
\end{equation}
Furthermore, the mutual information can be bounded by the $T$-information, 
by Pinsker's and Fano's inequality, as follows \cite[Eq.~(84) and Prop.~12]{PW14a}
\begin{equation}
 2 T(X;Y)^2 \leq I(X;Y)  \leq \log (M-1) T(X;Y) + h(T(X;Y))		
	\label{eq:TI}
\end{equation}
where in the upper bound $M$ is the number of possible values of $X$, and $h$ is the binary entropy function in \prettyref{eq:binaryentropy}.


We prove the ``if'' part.
Suppose 
%correlated recovery is impossible and for the sake of contradiction,
$I(\sigma_1, \sigma_2; A) = \Omega(1)$.
We first claim that assumption \ref{A1} implies that 
\begin{equation}
I (\indc{\sigma_1 =\sigma_2}; A)=I(\sigma_1, \sigma_2; A)
\label{eq:ss}
\end{equation}
that is,
 $A$ is independent of $\sigma_1,\sigma_2$ conditional on $\indc{\sigma_1 =\sigma_2}$. 
%\nbr{JX. Okay. I think we need this condition to ensure that 
%correlated recovery is impossible implies $I(\sigma_1, \sigma_2; A) = o(1)$. 
%Otherwise, we could have the situation that $A=a$ for $\sigma_1=+1$ and
%$A=b$ for $\sigma_1=-1$. In this case, correlated recovery of $\sigma$ is certainly 
%impossible, but $I(\sigma_1, \sigma_2;A)=I(\sigma_1; A)=1$.}
Indeed, 
for any $z \neq z'\in [k]$, let $\pi$ be any permutation such that 
%$\pi(z)=z'$ and 
$\pi(z')=z.$
%that interchanges $z$ and $z'$. 
Since $P_{\sigma, A} =  P_{\pi(\sigma), A}$, 
we have $P_{A|\sigma_1=z,  \sigma_2=z} =  P_{A| \pi(\sigma_1)=z,  \pi(\sigma_2)=z }$, i.e., 
$P_{A |\sigma_1=z, \sigma_2=z} =  P_{A| \sigma_1 =z',  \sigma_2 =z' }$. 
Similarly, one can show that 
$P_{A|\sigma_1=z_1, \sigma_2=z_2} =  P_{A| \sigma_1 =z_1',  \sigma_2 =z_2'}$, 
for any $z_1 \neq z_2$ and $z_1'\neq z_2'$, and this proves the claim.
%Since $\calL(\sigma, A) =  \calL(\pi(\sigma), A)$, where $\calL(\cdot)$ denote the law, 
%we have $\calL(A|\sigma_1=z,  \sigma_2=z) =  \calL(A| \pi(\sigma_1)=i,  pi(\sigma_2)=i )$, i.e., 
%$\calL(A|\sigma_1=z, \sigma_2=z) =  L(A| \sigma_1 =z',  \sigma_2 =z' )$. 
%Similarly, one can show that 
%$\calL(A|\sigma_1=z_1, \sigma_2=z_2) =  L(A| \sigma_1 =z_1',  \sigma_2 =z_2')$, for any $z_1 \neq z_2$ and $z_1'\neq z_2'$, and this proves the claim.

Let $x_j=\indc{\sigma_1 =\sigma_j}$.
By the symmetry assumption \ref{A2}, 
$I(x_j; A) = I(x_2; A) = \Omega(1)$ for all $j \neq 1$.
Since $\prob{x_j = 1} = \frac{1}{k} + o(1)$ by assumption \ref{A3}, applying \prettyref{eq:TI} with $M=2$ and in view of \prettyref{eq:Tbern}, we have
$\TV(P_{A|x_j=0},P_{A|x_j=1})=\Omega(1)$.
Thus, there exists an estimator $\widehat{x}_j \in \{0,1\}$ as a function of $A$, such that
\begin{align}
\prob{\widehat{x}_j = 1 \mid x_j =1 } + 
\prob{\widehat{x}_j = 0 \mid x_j =0 } \ge 1+\TV(P_{A|x_j=0},P_{A|x_j=1})
= 1+ \Omega(1).  \label{eq:estimator_x_hat}
\end{align}

Define $\hat\sigma$ as follows: set $\hat\sigma_1= 1 $; for $j \neq 1$, set $\hat \sigma_j = 1 $ if $\widehat{x}_j = 1$ 
and draw $\hat \sigma_j $ from $\{2,\ldots,k\}$ uniformly at random 
if $\widehat{x}_j = 0$.
Next, we show that $\hat\sigma$ achieves correlated recovery. 
Indeed, fix a permutation $\pi \in S_k$ such that $\pi(\sigma_1)=1$. It follows from 
the definition of overlap that
\begin{equation}
\Expect[o\left(\sigma, \hat{\sigma} \right)]
\ge \frac{1}{n} \sum_{j \neq 2} \prob{\pi(\sigma_j) =\hat\sigma_j} - \frac{1}{k}.
\label{eq:olb}
\end{equation}
Furthermore, since $\pi(\sigma_1)=1$, we have, for any $j\neq 1$,
\[
\prob{\pi(\sigma_j) = \hat\sigma_j,x_j=1}=
\prob{\hat x_j=1,x_j=1}
\]
and
\[
\prob{\pi(\sigma_j) = \hat\sigma_j,x_j=0}=
\prob{\pi(\sigma_j) = \hat\sigma_j, \hat x_j=0,x_j=0}=
\frac{1}{k-1} \prob{\hat x_j=0,x_j=0},
\]
where the last step is because conditional on $\hat{x}_j=0$,
$\hat{\sigma}_j$ is chosen  from $\{2,\ldots,k\}$ uniformly and 
independently of everything else.
Since $\prob{x_j = 1} = \frac{1}{k} + o(1)$, we have
\[
\prob{\pi(\sigma_j) =\hat\sigma_j} = \frac{1}{k}(\prob{\widehat{x}_j = 1 \mid x_j =1 } + 
\prob{\widehat{x}_j = 0 \mid x_j =0 }) +o(1) \overset{\prettyref{eq:estimator_x_hat}}{\ge} \frac{1}{k}+ \Omega(1).  
\]
By \prettyref{eq:olb}, we conclude that $\hat{\sigma}$ achieves correlated recovery
of $\sigma$.

Next we prove the ``only if'' part.
Suppose $I(\sigma_1, \sigma_2; A) = o(1)$ and we aim to show 
%the impossibility of correlated recovery, that is, 
$\expect{o\left(\sigma, \hat{\sigma} \right)}=o(1)$ for any estimator $\hat\sigma$.
By the definition of overlap, we have
\begin{align*}
o\left(\sigma, \hat{\sigma} \right) 
\le 
\frac{1}{n} \sum_{\pi \in S_k}
\left| \sum_{i \in [n] }  
\left( \indc{ \pi\left(\sigma_i \right) = \hat{\sigma}_i} - \frac{1}{k}  \right) \right|.
%& \le \frac{1}{n} \sum_{\pi \in S_k}
%\sum_{\ell=1}^k 
%\left| \sum_{i \in [n] } \left( \indc{ \pi\left(\sigma_i \right) =\ell} 
%\indc{ \hat{\sigma}_i = \ell} - \frac{1}{k^2} \right)\right|.
\end{align*}
Since there are $k!=\Omega(1)$ permutations in $S_k$, it suffices to show
for any fixed permutation $\pi$,
$$
\expect{ \left| \sum_{i \in [n] } \left( \indc{ \pi\left(\sigma_i \right) = \hat{\sigma}_i} - \frac{1}{k}  \right) 
\right| } = o(n).
$$
Since $I(\pi(\sigma_i), \pi(\sigma_j); A)=I(\sigma_i, \sigma_j; A)$, without loss of generality, 
we assume $\pi=\text{id}$ in the following. By the Cauchy-Schwarz inequality, it further suffices to show
\begin{equation}
\expect{ \left( \sum_{i \in [n] } \left( \indc{ \sigma_i =\hat{\sigma}_i } - \frac{1}{k} \right)\right)^2 } =o(n^2).
\label{eq:overlap2}
\end{equation}
Note that 
\begin{align*}
& \expect{  \left( \sum_{i \in [n] } \left(
   \indc{ \sigma_i  = \hat{\sigma}_i } - \frac{1}{k} \right)   \right)^2 } \\
 & = \sum_{i, j \in [n] } 
 \expect{ \left( \indc{ \sigma_i = \hat{\sigma}_i }- \frac{1}{k}  \right) 
 \left( \indc{ \sigma_j  = \hat{\sigma}_j }- \frac{1}{k}  \right) }  \\
 & =  \sum_{i, j \in [n] }  \prob{  \sigma_i = \hat{\sigma}_i , \sigma_j  = \hat{\sigma}_j }
 - \frac{2n}{k} \sum_{i \in [n] }\prob{ \sigma_i = \hat{\sigma}_i } + \frac{n^2}{k^2}.
\end{align*}
For the first term in the last displayed equation, 
let $\sigma'$ be identically distributed as $\hat\sigma$ but independent of $\sigma$.
Since $I(\sigma_i,\sigma_j;\hat\sigma_i,\hat\sigma_j) \le I(\sigma_i,\sigma_j;A)=o(1)$ by the data processing inequality, it follows from the lower bound in 
\prettyref{eq:TI} that $\TV(P_{\sigma_i,\sigma_j,\hat\sigma_i,\hat\sigma_j}, P_{\sigma_i,\sigma_j,\sigma_i',\sigma_j'})=o(1)$.
Since 
$\pprob{  \sigma_i = {\sigma}_i', \sigma_j  = {\sigma}_j' } \leq 
\max_{a,b \in [k]} \prob{  \sigma_i = a, \sigma_j  = b} \leq \frac{1}{k^2}+o(1)$ by assumption \ref{A3}, 
we have
$$
\prob{  \sigma_i = \hat{\sigma}_i , \sigma_j  = \hat{\sigma}_j }
\leq \frac{1}{k^2} + o(1),
$$
Similarly, for the second term, we have
$$
\prob{ \sigma_i = \hat{\sigma}_i } = \frac{1}{k} +o(1),
$$
where the last equality holds due to $I(\sigma_i; A) =o(1).$
Combining the last three displayed equations gives \prettyref{eq:overlap2} and completes the proof.
\end{proof}


%The argument below is essentially contained in \cite{PW18}.
%
%We first consider $k=2$, in which case it is convenient to assume $\sigma \in \{\pm\}^n$. Recall that correlated recovery amounts to reconstruct $\sigma$ (up to a global sign flip) better than chance, that is, find $\hat \sigma=\hat \sigma(Y) \in \{\pm1\}^n$, such that
%\begin{equation}
%%\liminf_{n \to\infty} \frac{1}{n}\Expect[|\Iprod{\sigma}{\hat \sigma}|] > 0.
%\frac{1}{n}\Expect[|\Iprod{\sigma}{\hat \sigma}|] = \Omega(1).
%%> \epsilon
%\label{eq:corr}
%\end{equation}
%%for some constant $\epsilon$.
%Note that by symmetry, for any $i\neq j$, $I(\sigma_i,\sigma_j;A) = I(\sigma_i \sigma_j; A) = I(\sigma_1 \sigma_2; A)$. Suppose $I(\sigma_1 \sigma_2; A) = \Omega(1)$, then for any $j \neq 1$, there exists an estimator $\widehat{\sigma_1\sigma_j}$ as a function of $A$, such that
%$\prob{\widehat{\sigma_1\sigma_j} \neq \sigma_1\sigma_j} \leq \frac{1}{2} - \Omega(1)$. Thus, setting $\hat\sigma$ according to $\hat\sigma_1=+$ and $\hat \sigma_j = \hat{\sigma_1\sigma_j}$ achieves \prettyref{eq:corr}. 
%This shows \prettyref{eq:MIcorr2} is necessary for the impossibility of correlated recovery. 
%%To show \prettyref{eq:MIcorr2} implies the impossibility of correlated recovery,
%To prove the sufficiency, 
%for any estimator $\hat \sigma= \hat \sigma(A) \in \{\pm\}^n$, 
%since $I(\sigma_i \sigma_j; A)=o(1)$, which is equivalent to 
%$\TV(P_{A|\sigma_i \sigma_j=+}, P_{A|\sigma_i \sigma_j=-})=o(1)$, we have 
%$\prob{\sigma_i\sigma_j \neq \hat \sigma_i\hat \sigma_j} \geq \frac{1}{2}-o(1)$. 
%On the other hand, we have the equality:
%\begin{align*}
%2n^2 - 2 \eexpect{\iprod{\sigma}{\hat\sigma}^2} 
%= & ~ \Expect\Fnorm{\sigma \sigma^\top - \hat \sigma\hat \sigma^\top}^2 \\
%= & ~ 4 \sum_{i \neq j} \prob{\sigma_i\sigma_j \neq \hat \sigma_i\hat \sigma_j} = 2n^2-o(n^2).
%\end{align*}
%Thus, $\eexpect{\iprod{\sigma}{\hat\sigma}^2} =o(n^2)$, which implies $\eexpect{|\Iprod{\sigma}{\hat\sigma}|} =o(n)$.
%
%Next we consider $k \geq 3$, in which case correlated recovery is achieved if there exists an estimator $\hat \sigma \in [k]^n$ that outperforms random guessing, i.e., 
%\begin{equation}
%\Expect[d(\sigma,\hat \sigma)] \leq \frac{k-1}{k} - \Omega(1).
%\label{eq:corr-sbmk}
%\end{equation}
%Here the loss function $d$ is the fraction of classification errors up to a global permutation of labels, formally defined as follows: for any $\sigma,\hat \sigma \in [k]^n$, 
%\begin{equation}
%d(\sigma,\hat \sigma) \triangleq  \min_{\pi \in S_k} \frac{1}{n}\sum_{i\in[n]} \indc{\sigma_i \neq \pi(\hat \sigma_i)}
%\label{eq:lossd}
%\end{equation}
%where $S_k$ is the collection of permutations on $[k]$.
%
%%By symmetry, \prettyref{eq:MIcorrk} implies that for any fixed $m$, as $n\diverge$,
%%\begin{equation}
%%I(\sigma_S; A) = o(1)
%%\label{eq:IXSY}
%%\end{equation}
%%for any $S \in \binom{[n]}{m}$, where 
%To show the impossibility of correlated recovery on the basis of \prettyref{eq:MIcorrk}, first of all, note that for any fixed $x,\hat x\in[k]^n$ and any $m\in [n]$ we have
%\begin{equation}
%d(x,\hat x)
%\geq \Expect_{S}[d(x_{\sfS},\hat x_{\sfS})] \label{eq:davg}
%\end{equation}
%where ${\sfS}  \sim \Unif(\binom{[n]}{m})$ and recall that for any $S$, we have 
%$d(x_S,\hat x_S) = \frac{1}{|S|}  \min_{\pi \in S_k} \sum_{i\in S} \indc{x_i \neq \pi(\hat x_i)}$ per \prettyref{eq:lossd}. The inequality \prettyref{eq:davg} simply follows from
%\begin{align}
%d(x,\hat x)
%= & ~ \min_{\pi \in S_k} \probs{x_I \neq \hat x_{\pi(I)}}{I \sim \Unif([n])}	\nonumber \\
%= & ~ \min_{\pi \in S_k} \Expect_{\sfS \sim \Unif(\binom{[n]}{m})} \probs{x_I \neq \hat x_{\pi(I)}}{I \sim \Unif({\sfS})}	\nonumber \\
%= & ~ \Expect_{{\sfS}} \min_{\pi \in S_k} \probs{x_I \neq \hat x_{\pi(I)}}{I \sim \Unif({\sfS})}	\nonumber \\
%\geq & ~ \Expect_{{\sfS}} [d(x_{\sfS},\hat x_{\sfS})] \nonumber.
%\end{align}
%%Fix a constant $m$ independent of $n$. 
%For any estimator $\hat \sigma=\hat \sigma(Y) \in [k]^n$, applying \prettyref{eq:davg} yields
%\begin{equation}
%\Expect[d(\sigma_{\sfS},\hat \sigma_{\sfS})] \leq \Expect[d(\sigma,\hat \sigma)], \label{eq:davg2}
%\end{equation}
%where ${\sfS}$ is a random uniform $m$-set independent of $\sigma,\hat \sigma$.
%
%
%By the data processing inequality, we have for any $S \in \binom{[n]}{m}$,
%\[
%I(\sigma_S; \hat \sigma_S) \leq I(\sigma_S; A) = I(\sigma_1,\ldots,\sigma_m; A) \overset{\prettyref{eq:MIcorrk}}{=} o(1),
%\]
%as $n\diverge$.
%By Pinsker's inequality, we have 
%$\TV(P_{\sigma_S, \hat \sigma_S}, P_{\sigma_S} \otimes P_{\hat \sigma_S}) \leq \sqrt{2 I(\sigma_S; \hat \sigma_S)} = o(1)$.
%Since the loss function $d$ defined in \prettyref{eq:lossd} is bounded by one, we have
%\begin{equation}
%\Expect[d(\sigma_S,\hat \sigma_S)] \geq \Expect[d(\sigma_S,\sigma'_S)] - \TV(P_{\sigma_S, \hat \sigma_S}, P_{\sigma_S} \otimes P_{\hat \sigma_S}) 
%= \Expect[d(\sigma_S,\sigma'_S)] + o(1), 
%\label{eq:ddTV}
%\end{equation}
%where $\sigma'_S$ has the same marginal distribution as $\hat \sigma_S$ but independent of $\sigma_S$.
%By \prettyref{lmm:randomguess} below, we have
%\begin{equation}
%\Expect[d(\sigma_S,\sigma'_S)] \geq \pth{\frac{k-1}{k} - m^{-1/3}}(1-k! e^{-2m^{1/3}}).
%\label{eq:randomguess2}
%\end{equation}
%Averaging \prettyref{eq:randomguess2} over $S \in \binom{[n]}{m}$ then combining with \prettyref{eq:davg2} and \prettyref{eq:ddTV}, and finally sending $n\to\infty$ followed by $m \to \infty$, we conclude that $\eexpect{ d(\sigma,\hat \sigma)} \geq \frac{k-1}{k}-o(1)$, hence the impossibility of correlated recovery.
%
%
%\begin{lemma}
%\label{lmm:randomguess}	
	%Let $\sigma$ be uniformly distributed on $[k]^m$ and $\sigma'$ is independent of $\sigma$ with an arbitrary distribution on $[k]^m$. 
%For the loss function in \prettyref{eq:lossd}, we have\footnote{Note that for any fixed $k,m$ and any string $x,z\in [k]^m$, we can always outperform random matching, i.e., $d(x,z) < \frac{k-1}{k}$. The point of \prettyref{eq:randomguess} is that this improvement is negligible for large $m$.}
	%\begin{equation}
	%d(\sigma,\sigma') \geq \frac{k-1}{k} - m^{-1/3}
	%\label{eq:randomguess}
	%\end{equation}
	%with probability at least $1-(k! e^{-2m^{1/3}})$.
%\end{lemma}
%\begin{proof}
	%For each fixed $\pi$, the Hamming distance $d_H(\sigma,\pi(\sigma'))\sim \Binom(m,\frac{k-1}{k})$. From Hoeffding's inequality we have
	%$$ \Prob[d_H(\sigma,\pi(\sigma') < {k-1\over k} - \delta] \le e^{-2m \delta^2}\,,$$
	%and from the union bound
	%$$ \Prob[\min_\pi d_H(\sigma,\pi(\sigma') < {k-1\over k} - \delta] \le k! e^{-2m \delta^2}\,.$$
	%Setting $\delta = m^{-1/3}$ completes the proof.
%\end{proof}



%\section{Proof of \prettyref{eq:second_moment_conditional} $\implies$ \prettyref{eq:MI_TV} and verification in the binary symmetric SBM}
%\label{app:MITV}
%Let $S=[m]$ and denote $\sigma_1,\ldots,\sigma_m$ by $\sigma_S$.
%Recall that $m$ is a constant and $\P_z \triangleq \P_{A|\sigma_S=z}$ for $z\in[k]^m$.
%We first prove the following:
%%\begin{align}
%%I(\sigma_S; A) = o(1) & \Leftrightarrow D \left( \P_{A | \sigma_S} \| \P \right) =o(1), \quad \forall \sigma_S  \nonumber \\
%%& \Leftrightarrow d_{\rm TV} \left( \P_{A | \sigma_S}, \P \right) =o(1), \quad \forall \sigma_S. \nonumber \\
%%& \Leftarrow d_{\rm TV} \left( \P_{A | \sigma_S}, \P_{A | \tsigma_S} \right) =o(1), \quad \forall \sigma_S, \tsigma_S,
%%\label{eq:MI_TV}
%%\end{align}
%\begin{align}
%I(\sigma_S; A) = o(1) & \Leftrightarrow D \left( \P_z \| \P \right) =o(1), \quad \forall z, \label{eq:MI_TV1}\\
%& \Leftrightarrow d_{\rm TV} \left( \P_z, \P \right) =o(1), \quad \forall z, \label{eq:MI_TV2}\\
%& \Leftarrow d_{\rm TV} \left( \P_z, \P_{\tz} \right) =o(1), \quad \forall z, \tz. \label{eq:MI_TV}
%\end{align}
%For \prettyref{eq:MI_TV1}, by definition,
%$$
%I(\sigma_S; A) =  \Expect_{\sigma_S} \qth{ D \left( \P_{A|\sigma_S} \| \P \right) }.
%$$
%Note that the distribution of $\sigma_S$ has a finite support
%and $\Omega(1)$ probability mass on each possible value. 
%Therefore, $I(\sigma_S; A)=o(1)$ if and only if 
%$D \left( \P_{A|\sigma_S=z} \| \P \right) = o(1)$ for all $z$.
%
%For \prettyref{eq:MI_TV2}, by Pinsker's inequality, $D \left( \P_{z} \| \P \right) = o(1)$
%implies $d_{\rm TV} \left( \P_{z}, \P \right) = o(1)$. Conversely, 
%suppose that $d_{\rm TV} \left( \P_{z}, \P \right) = o(1)$. Then
%\begin{align*}
%D \left( \P_{z} \| \P \right) &= \int \P_{z} \log \frac{\P_{z}  }{\P} \\
%& \overset{(a)}{\le} \int  \P_{z}  \frac{\P_{z} - \P }{\P} \\
%&  \le \int  \frac{ \P_{z} }{ \P } \left| \P_{z} - \P  \right| \\
%& \overset{(b)}{=} O(1) \times \int \left| \P_{z} - \P  \right| \\
%& = O\left( d_{TV}  \left( \P_{z}, \P \right) \right) = o(1),
%\end{align*}
%where $(a)$ is due to $\log x \le x-1$, and $(b)$ follows because 
%$\frac{ \P_{z}(a) }{ \P(a) } = \frac{ \P_{A|\sigma_S=z}(a) }{ \P_A(a)} \leq \frac{1}{\prob{\sigma_S=z}} = O(1)$ everywhere.
%%, and  the distribution of $\sigma_S$ has $\Omega(1)$ probability mass on each realization and hence 
%
%
%For \prettyref{eq:MI_TV}, suppose that $d_{\rm TV} \left( \P_{z}, \P_{\tz} \right) =o(1)$
%for all $z$ and $\tz$. By the convexity of $d_{\rm TV}(\cdot,\cdot)$ and Jensen's inequality, it readily
%follows that $d_{\rm TV} \left( \P_{z}, \P \right) =o(1)$.
%
%
%
%Next we prove that for any reference distribution $\Q$,  \prettyref{eq:second_moment_conditional} implies 
%$d_{\rm TV} \left( \P_{z}, \P_{\tz} \right) =o(1)$
%for all $z,\tz$, which further implies \prettyref{eq:MIcorrk} in view of \prettyref{eq:MI_TV}.
 %Indeed, by Cauchy-Schwartz inequality, we have
%\begin{align}
%d_{\rm TV}  \left( \P_{z}, \P_{\tz} \right) & =
%\frac{1}{2} \int  \left| \P_{z} - \P_{ \tz} \right| \nonumber \\
%%& = \frac{1}{2} \int  \left| \P_{z} - \P_{\tz} \right| \frac{\sqrt{\Q}}{\sqrt{\Q}} \nonumber \\
%& \le \frac{1}{2}  \left( \int \Q  \right)^{1/2} \left( \int  \frac{  \left( \P_{z} - \P_{\tz} \right)^2 }{\Q }  \right)^{1/2}
%\nonumber \\
%& = \frac{1}{2} \left(    \int \frac{\P^2_{A | z}}{\Q} +\int \frac{\P^2_{A | \tz}}{\Q} - 2 \int \frac{ 
%\P_{z} \P_{\tz} }{ \Q} \right)^{1/2} \overset{\prettyref{eq:second_moment_conditional}}{=} o(1).
%\end{align}
%%where the last equality holds from the assumption \prettyref{eq:second_moment_conditional}. 
%
%
%
%Finally, we consider the binary symmetric SBM and show that,
%below the correlated recovery threshold $\tau=\frac{(a-b)^2}{2(a+b)}<1$, 
 %\prettyref{eq:second_moment_conditional} is satisfied if the reference distribution $\Q$ is the distribution of $A$ in
%the null (\ER) model.
%Specifically, following the derivations in \prettyref{eq:SBM_second_moment_eq},
%we have
%\begin{align}
%\int \frac{   \P_{z}  \P_{\tz} }{ \Q } 
%&= \Expect \qth{  \prod_{i < j} \left(1 +  \sigma_i \sigma_j \tsigma_i \tsigma_j \rho \right)
%\mathrel{\bigg|} \sigma_S=z, \tsigma_S =\tz}  \nonumber \\
%& =  \left( 1+o(1) \right) e^{ -\tau^2/4 -\tau/2} \times 
%\Expect \qth{ \exp \left( \frac{\rho}{2} \iprod{ \sigma}{\tsigma}^2  \right) \mathrel{\Big|} \sigma_S=z, \tsigma_S =\tz},
%%& = \prod_{(i,j) \in \calE(S) } \left(1 +  \sigma_i \sigma_j \tsigma_i \tsigma_j \rho \right)  \nonumber \\
%%& \times \Expect \qth{ \prod_{i \in S, j \in S^c}  \left(1 +  \sigma_i \sigma_j \tsigma_i \tsigma_j \rho \right) \mid \sigma_S, \tsigma_S } \nonumber \\
%%& \times \Expect \qth{ \prod_{(i,j)\in \calE(S^c)}  \left(1 +  \sigma_i \sigma_j \tsigma_i \tsigma_j \rho \right) \mid \sigma_S, \tsigma_S }.
%\end{align}
%where the last equality holds because $m$ is a constant, $\rho=\tau/n + O(1/n^2)$ and $\log(1+x) = x -x^2/2 +O(x^3)$. 
%%\nbr{
%%I got $e^{ -\tau^2/4 - \tau/2}$, because 
%%$\sum_{i < j} \sigma_i \sigma_j \tsigma_i \tsigma_j = \frac{1}{2}(
%%\iprod{ \sigma}{\tsigma}^2 - n)$.}
%
%Write $\sigma=2\xi-1$ for $\xi\in\{0,1\}^n$ and let 
%$$
%H_1\triangleq \Iprod{\xi_{S}} { \tilde{\xi}_{S} } \quad \text{ and } \quad 
%H_2\triangleq \Iprod{\xi_{S^c}} { \tilde{\xi}_{S^c} }.
%$$ 
%Then $\iprod{\sigma}{\tsigma} = 4 (H_1+H_2) -n$.
%Moreover, conditional  on $\sigma_S$ and $\tsigma_S$, 
%$$
%H_2 \sim \text{Hypergeometric} \left( n-m, n/2 - \| \xi_S\|_1, n/2 - \|\tilde{\xi}_S \|_1 \right).
%$$
%Therefore
%\begin{align*}
%\Expect \qth{ \exp \left(  \frac{\rho}{2} \iprod{ \sigma}{\tsigma}^2  \right) \mathrel{\bigg|} \sigma_S=z, \tsigma_S =\tz}
%& =\expect{ \exp \left(  \frac{n\rho}{2} \left( \frac{ 4 H_1 + 4H_2 - n}{\sqrt{n} } \right)^2 \right) \mathrel{\bigg|} \sigma_S=z, \tsigma_S =\tz}  \\
%& = \frac{1+o(1)}{\sqrt{1-\tau}},
%\end{align*}
%where the last inequality holds because $n \rho = \tau +o(1/n)$ and 
%conditional on $\sigma_S$ and $\tsigma_S$,
%$ \frac{1}{\sqrt{n}} ( 4H_1 + 4H_2 - n )$ converges to $\calN(0,1)$ in distribution 
%as $n \to \infty$ by the central limit theorem for hypergeometric distribution. 
%
%In conclusion, we have shown that 
%$$
%\int \frac{   \P_{z}  \P_{\tz} }{ \Q}  
%= \left( 1+o(1) \right) e^{ -\tau^2/4 - \tau/2}  \frac{1}{\sqrt{1-\tau}}, \quad \forall z,\tz.
%$$
%%Hence, by taking the expectation of $\sigma_S$ and $\tsigma_S$ over the both hand sides of the last displayed equation, we get that
%Averaging both sides over $z,\tz$ yields
%$$
%\int \frac{   \P^2 }{ \Q}  
%= \left( 1+o(1) \right) e^{ -\tau^2/4 - \tau/2}  \frac{1}{\sqrt{1-\tau}}.
%$$
%Thus \prettyref{eq:second_moment_conditional} is satisfied. 




\section{Proof of \prettyref{eq:second_moment_conditional} $\implies$ \prettyref{eq:MIcorr2} and 
verification of  \prettyref{eq:second_moment_conditional} in the binary symmetric SBM}
\label{app:MITV}
Combining \prettyref{eq:ss} with 
\prettyref{eq:TI} and \prettyref{eq:Tbern}, we have
$I(\sigma_1,\sigma_2; A) = o(1)$ if and only if $\TV(\P_+,\P_-) =o(1)$,
where $\P_+=P_{A|\sigma_1=\sigma_2}$ and $\P_-=P_{A|\sigma_1\neq\sigma_2}$.
%Next we prove that for any reference distribution $\Q$,  \prettyref{eq:second_moment_conditional} implies 
%$d_{\rm TV} \left( \P_{+}, \P_{-} \right) =o(1)$.
Note the following characterization about the total variation distance, which simply follows from the Cauchy-Schwartz inequality:
\begin{equation}
\TV(\P_+,\P_-) = \frac{1}{2} \sqrt{\inf_{\Q}  \int  \frac{  \left( \P_{+} - \P_{-} \right)^2 }{\Q }}
\label{eq:TVquadratic}
\end{equation}
where the infimum is taken over all probability distributions $\Q$. 
Therefore \prettyref{eq:second_moment_conditional} implies \prettyref{eq:MIcorr2}.

%
 %Indeed, by Cauchy-Schwartz inequality, for any $\Q$, we have
%\begin{align}
%d_{\rm TV}  \left( \P_{+}, \P_{-} \right) & =
%\frac{1}{2} \int  \left| \P_{+} - \P_{-} \right| \nonumber \\
%%& = \frac{1}{2} \int  \left| \P_{z} - \P_{\tz} \right| \frac{\sqrt{\Q}}{\sqrt{\Q}} \nonumber \\
%& \le \frac{1}{2}  
%\left( \int \Q  \right)^{1/2} \left( \int  \frac{  \left( \P_{+} - \P_{-} \right)^2 }{\Q }  \right)^{1/2}
%\nonumber \\
%%& = \frac{1}{2} \left(    \int \frac{\P^2_{+}}{\Q} +\int \frac{\P^2_{-}}{\Q} - 2 \int \frac{ 
%%\P_{+} \P_{-} }{ \Q} \right)^{1/2} 
%&\overset{\prettyref{eq:second_moment_conditional}}{=} o(1).
%\end{align}
%%where the last equality holds from the assumption \prettyref{eq:second_moment_conditional}. 



Finally, we consider the binary symmetric SBM and show that,
below the correlated recovery threshold $\tau=\frac{(a-b)^2}{2(a+b)}<1$, 
 \prettyref{eq:second_moment_conditional} is satisfied if the reference distribution $\Q$ is the distribution of $A$ in
the null (\ER) model. Note that 
$$
\int  \frac{  \left( \P_{+} - \P_{-} \right)^2 }{\Q }  =
\int \frac{\P^2_{+}}{\Q} +\int \frac{\P^2_{-}}{\Q} - 2 \int \frac{ \P_{+} \P_{-} }{ \Q}.
$$
%Hence, it is sufficient to show
%$$
 %\int \frac{ \P^2 }{ \Q} = O(1), \; \text{ and } 
 %\int \frac{ \P_{z} \P_{\tilde{z} } }{ \Q} =  
%(1+o(1)) \int \frac{ \P^2 }{ \Q}, \quad  \forall z, \tilde{z} \in \{\pm \}.
%$$
Hence, it is sufficient to show
$$
 \int \frac{ \P_{z} \P_{\tilde{z} } }{ \Q} =  
C+o(1), \quad  \forall z, \tilde{z} \in \{\pm \}
$$
for some constant $C$ independent of $z$ and $\tz$.
Specifically, following the derivations in \prettyref{eq:SBM_second_moment_eq},
we have
\begin{align}
\int \frac{   \P_{z}  \P_{\tz} }{ \Q } 
&= \Expect \qth{  \prod_{i < j} \left(1 +  \sigma_i \sigma_j \tsigma_i \tsigma_j \rho \right)
\mathrel{\bigg|} \sigma_1 \sigma_2=z, \tsigma_1 \tsigma_2 =\tz }  \nonumber \\
& =  \left( 1+o(1) \right) e^{ -\tau^2/4 -\tau/2} \times 
\Expect \qth{ \exp \left( \frac{\rho}{2} \iprod{ \sigma}{\tsigma}^2  \right) \mathrel{\Big|} \sigma_1 \sigma_2=z, \tsigma_1 \tsigma_2 =\tz },
%& = \prod_{(i,j) \in \calE(S) } \left(1 +  \sigma_i \sigma_j \tsigma_i \tsigma_j \rho \right)  \nonumber \\
%& \times \Expect \qth{ \prod_{i \in S, j \in S^c}  \left(1 +  \sigma_i \sigma_j \tsigma_i \tsigma_j \rho \right) \mid \sigma_S, \tsigma_S } \nonumber \\
%& \times \Expect \qth{ \prod_{(i,j)\in \calE(S^c)}  \left(1 +  \sigma_i \sigma_j \tsigma_i \tsigma_j \rho \right) \mid \sigma_S, \tsigma_S }.
\end{align}
where the last equality holds $\rho=\tau/n + O(1/n^2)$ and $\log(1+x) = x -x^2/2 +O(x^3)$. 
%\nbr{
%I got $e^{ -\tau^2/4 - \tau/2}$, because 
%$\sum_{i < j} \sigma_i \sigma_j \tsigma_i \tsigma_j = \frac{1}{2}(
%\iprod{ \sigma}{\tsigma}^2 - n)$.}

Write $\sigma=2\xi-1$ for $\xi\in\{0,1\}^n$ and let 
$$
H_1\triangleq \xi_1 \tilde{\xi}_1 + \xi_2 \tilde{\xi}_2 
\quad \text{ and } \quad 
H_2 \triangleq \sum_{j \ge 3}^n \xi_j \tilde{\xi}_j.
$$ 
Then $\iprod{\sigma}{\tsigma} = 4 (H_1+H_2) -n$.
Moreover, conditional  on $\sigma_1, \sigma_2$ and $\tsigma_1, \tsigma_2$, 
$$
H_2 \sim \text{Hypergeometric} \left( n-2, n/2 - \xi_1-\xi_2, n/2 - \tilde{\xi}_1-\tilde{\xi}_2 \right).
$$
Since $|H_1| \le 2$, $\xi_1+\xi_2 \le 2 $, and $\tilde{\xi}_1+\tilde{\xi}_2 \le 2$, 
it follows that 
conditional on $\sigma_1 \sigma_2=z, \tsigma_1 \tsigma_2 =\tz $,
$ \frac{1}{\sqrt{n}} ( 4H_1 + 4H_2 - n )$ converges to $\calN(0,1)$ in distribution 
as $n \to \infty$ by the central limit theorem for hypergeometric distribution. 
Therefore
\begin{align*}
& \Expect \qth{ \exp \left(  \frac{\rho}{2} \iprod{ \sigma}{\tsigma}^2  \right) \mathrel{\bigg|} \sigma_S=z, \tsigma_S =\tz} \\
& =\expect{ \exp \left(  \frac{n\rho}{2} \left( \frac{ 4 H_1 + 4H_2 - n}{\sqrt{n} } \right)^2 \right) \mathrel{\bigg|} 
\sigma_1 \sigma_2=z, \tsigma_1 \tsigma_2 =\tz }  \\
& = \frac{1+o(1)}{\sqrt{1-\tau} },
\end{align*}
where the last equality holds due to $n \rho = \tau +o(1/n)$, $\tau<1$, 
and the convergence of the moment generating function. 

%In conclusion, we have shown that 
%$$
%\int \frac{   \P_{z}  \P_{\tz} }{ \Q}  
%= \left( 1+o(1) \right) e^{ -\tau^2/4 - \tau/2}  \frac{1}{\sqrt{1-\tau}}, \quad \forall z,\tz \in \{ \pm \}.
%$$
%%Hence, by taking the expectation of $\sigma_S$ and $\tsigma_S$ over the both hand sides of the last displayed equation, we get that
%Averaging both sides over $z,\tz$ yields
%$$
%\int \frac{   \P^2 }{ \Q}  
%= \left( 1+o(1) \right) e^{ -\tau^2/4 - \tau/2}  \frac{1}{\sqrt{1-\tau}}.
%$$
%Thus \prettyref{eq:second_moment_conditional} is satisfied.


\end{appendices}



 \documentclass[letterpaper,11pt]{article}
\usepackage[toc,page]{appendix}
\usepackage[margin=1in]{geometry}
\usepackage[bookmarks, colorlinks=true, plainpages = false, citecolor = blue,linkcolor=red,urlcolor = blue, filecolor = blue,pagebackref]{hyperref}
%% Note, I added pagebackref to the options for hyperref so that page number back references appear after each paper listing.   -BH 4/29/15
%\usepackage{url}\urlstyle{rm}
\usepackage{amsmath,amsfonts,amsthm,amssymb,bm, verbatim,dsfont,mathtools}
%\usepackage{algorithm,algorithmic}
\usepackage{color,graphicx,appendix}
\usepackage{subfigure}
\usepackage{etoolbox}
\usepackage{tikz}
\usepackage{xr,xspace}
\usepackage{todonotes}
\usepackage{paralist}
\usepackage{caption,soul}
%\usepackage[ruled,vlined]{algorithm2e}
\usepackage{algorithm}% http://ctan.org/pkg/algorithms
\usepackage{algorithmic}% http://ctan.org/pkg/algorithms
\usepackage{enumitem}
\makeatletter
%\renewcommand{\ALG@name}{SDP}
%\renewcommand{\listalgorithmname}{List of \ALG@name s}

%%%%% THEOREM STYLE DEFINITIONS
%\theoremstyle{plain}
%\newtheorem{theorem}{Theorem}
%\newtheorem{lemma}{Lemma}
%\newtheorem{proposition}{Proposition}
%\newtheorem{corollary}{Corollary}
%\theoremstyle{definition}
%\newtheorem{definition}{Definition}
%\newtheorem{hypothesis}{Hypothesis}
%\newtheorem{conjecture}{Conjecture}
%\newtheorem{question}{Question}
%\newtheorem{remark}{Remark}
%\newtheorem*{remark*}{Remark}

\newtheorem{theorem}{Theorem}
\newtheorem{lemma}{Lemma}
\newtheorem{proposition}{Proposition}
\newtheorem{corollary}{Corollary}
\theoremstyle{definition}
\newtheorem{definition}{Definition}
\newtheorem{hypothesis}{Hypothesis}
\newtheorem{conjecture}{Conjecture}
\newtheorem{question}{Question}
\newtheorem{remark}{Remark}
\newtheorem{assumption}{Assumption}


%\usepackage{enumerate}

\usepackage{tikz}

\usepackage{xspace,prettyref}
\usepackage{bm}
% for prettyref.sty
\newrefformat{eq}{(\ref{#1})}
\newrefformat{chap}{Chapter~\ref{#1}}
\newrefformat{sec}{Section~\ref{#1}}
\newrefformat{alg}{Algorithm~\ref{#1}}
\newrefformat{fig}{Fig.~\ref{#1}}
\newrefformat{tab}{Table~\ref{#1}}
\newrefformat{rmk}{Remark~\ref{#1}}
\newrefformat{clm}{Claim~\ref{#1}}
\newrefformat{def}{Definition~\ref{#1}}
\newrefformat{cor}{Corollary~\ref{#1}}
\newrefformat{lmm}{Lemma~\ref{#1}}
\newrefformat{prop}{Proposition~\ref{#1}}
\newrefformat{app}{Appendix~\ref{#1}}
\newrefformat{hyp}{Hypothesis~\ref{#1}}
\newrefformat{thm}{Theorem~\ref{#1}}
\newrefformat{ass}{Assumption~\ref{#1}}
\newrefformat{conj}{Conjecture~\ref{#1}}

\newcommand{\ie}{i.e.\xspace}
\renewcommand{\P}{\mathcal{P} }
\newcommand{\Q}{\mathcal{Q}}
\newcommand{\Exp}{\mathbb{E}}
\newcommand{\contig}{\trianglelefteq}
\newcommand{\indicator}[1]{\bm{1}_{#1}}
\renewcommand{\hat}{\widehat}
\renewcommand{\tilde}{\widetilde}
\newcommand{\argmax}{\mathrm{argmax}}

\usepackage{color}
\newcommand{\red}{\color{red}}
\newcommand{\blue}{\color{blue}}
\newcommand{\nb}[1]{{\sf\blue[#1]}}
\newcommand{\nbr}[1]{{\sf\red #1}}

\renewcommand{\implies}{\Rightarrow}
\newcommand{\1}[1]{{\mathbf{1}_{\left\{{#1}\right\}}}}
\newcommand{\post}[2]{\begin{center} \includegraphics[width=#2]{#1} \end{center} }
\newcommand \E[1]{\mathbb{E}[#1]}
%\newcommand{\Perp}{\perp \! \! \! \perp}
\newcommand{\Perp}{\perp}
\newcommand{\Hyper}{\text{Hypergeometric}}
%\newcommand \P[1]{\mathbb{P}[#1]}


%% Wu
\newcommand{\bfa}{{\mathbf{a}}}
\newcommand{\bfb}{{\mathbf{b}}}
\newcommand{\bfc}{{\mathbf{c}}}
\newcommand{\bfd}{{\mathbf{d}}}
\newcommand{\bfe}{{\mathbf{e}}}
\newcommand{\bff}{{\mathbf{f}}}
\newcommand{\bfg}{{\mathbf{g}}}
\newcommand{\bfh}{{\mathbf{h}}}
\newcommand{\bfi}{{\mathbf{i}}}
\newcommand{\bfj}{{\mathbf{j}}}
\newcommand{\bfk}{{\mathbf{k}}}
\newcommand{\bfl}{{\mathbf{l}}}
\newcommand{\bfm}{{\mathbf{m}}}
\newcommand{\bfn}{{\mathbf{n}}}
\newcommand{\bfo}{{\mathbf{o}}}
\newcommand{\bfp}{{\mathbf{p}}}
\newcommand{\bfq}{{\mathbf{q}}}
\newcommand{\bfr}{{\mathbf{r}}}
\newcommand{\bfs}{{\mathbf{s}}}
\newcommand{\bft}{{\mathbf{t}}}
\newcommand{\bfu}{{\mathbf{u}}}
\newcommand{\bfv}{{\mathbf{v}}}
\newcommand{\bfw}{{\mathbf{w}}}
\newcommand{\bfx}{{\mathbf{x}}}
\newcommand{\bfy}{{\mathbf{y}}}
\newcommand{\bfz}{{\mathbf{z}}}
\newcommand{\bfA}{{\mathbf{A}}}
\newcommand{\bfB}{{\mathbf{B}}}
\newcommand{\bfC}{{\mathbf{C}}}
\newcommand{\bfD}{{\mathbf{D}}}
\newcommand{\bfE}{{\mathbf{E}}}
\newcommand{\bfF}{{\mathbf{F}}}
\newcommand{\bfG}{{\mathbf{G}}}
\newcommand{\bfH}{{\mathbf{H}}}
\newcommand{\bfI}{{\mathbf{I}}}
\newcommand{\bfJ}{{\mathbf{J}}}
\newcommand{\bfK}{{\mathbf{K}}}
\newcommand{\bfL}{{\mathbf{L}}}
\newcommand{\bfM}{{\mathbf{M}}}
\newcommand{\bfN}{{\mathbf{N}}}
\newcommand{\bfO}{{\mathbf{O}}}
\newcommand{\bfP}{{\mathbf{P}}}
\newcommand{\bfQ}{{\mathbf{Q}}}
\newcommand{\bfR}{{\mathbf{R}}}
\newcommand{\bfS}{{\mathbf{S}}}
\newcommand{\bfT}{{\mathbf{T}}}
\newcommand{\bfU}{{\mathbf{U}}}
\newcommand{\bfV}{{\mathbf{V}}}
\newcommand{\bfW}{{\mathbf{W}}}
\newcommand{\bfX}{{\mathbf{X}}}
\newcommand{\bfY}{{\mathbf{Y}}}
\newcommand{\bfZ}{{\mathbf{Z}}}

\newcommand{\bbA}{{\mathbb{A}}}
\newcommand{\bbB}{{\mathbb{B}}}
\newcommand{\bbC}{{\mathbb{C}}}
\newcommand{\bbD}{{\mathbb{D}}}
\newcommand{\bbE}{{\mathbb{E}}}
\newcommand{\bbF}{{\mathbb{F}}}
\newcommand{\bbG}{{\mathbb{G}}}
\newcommand{\bbH}{{\mathbb{H}}}
\newcommand{\bbI}{{\mathbb{I}}}
\newcommand{\bbJ}{{\mathbb{J}}}
\newcommand{\bbK}{{\mathbb{K}}}
\newcommand{\bbL}{{\mathbb{L}}}
\newcommand{\bbM}{{\mathbb{M}}}
\newcommand{\bbN}{{\mathbb{N}}}
\newcommand{\bbO}{{\mathbb{O}}}
\newcommand{\bbP}{{\mathbb{P}}}
\newcommand{\bbQ}{{\mathbb{Q}}}
\newcommand{\bbR}{{\mathbb{R}}}
\newcommand{\bbS}{{\mathbb{S}}}
\newcommand{\bbT}{{\mathbb{T}}}
\newcommand{\bbU}{{\mathbb{U}}}
\newcommand{\bbV}{{\mathbb{V}}}
\newcommand{\bbW}{{\mathbb{W}}}
\newcommand{\bbX}{{\mathbb{X}}}
\newcommand{\bbY}{{\mathbb{Y}}}
\newcommand{\bbZ}{{\mathbb{Z}}}
                         
                         
\newcommand{\sfa}{{\mathsf{a}}}
\newcommand{\sfb}{{\mathsf{b}}}
\newcommand{\sfc}{{\mathsf{c}}}
\newcommand{\sfd}{{\mathsf{d}}}
\newcommand{\sfe}{{\mathsf{e}}}
\newcommand{\sff}{{\mathsf{f}}}
\newcommand{\sfg}{{\mathsf{g}}}
\newcommand{\sfh}{{\mathsf{h}}}
\newcommand{\sfi}{{\mathsf{i}}}
\newcommand{\sfj}{{\mathsf{j}}}
\newcommand{\sfk}{{\mathsf{k}}}
\newcommand{\sfl}{{\mathsf{l}}}
\newcommand{\sfm}{{\mathsf{m}}}
\newcommand{\sfn}{{\mathsf{n}}}
\newcommand{\sfo}{{\mathsf{o}}}
\newcommand{\sfp}{{\mathsf{p}}}
\newcommand{\sfq}{{\mathsf{q}}}
\newcommand{\sfr}{{\mathsf{r}}}
\newcommand{\sfs}{{\mathsf{s}}}
\newcommand{\sft}{{\mathsf{t}}}
\newcommand{\sfu}{{\mathsf{u}}}
\newcommand{\sfv}{{\mathsf{v}}}
\newcommand{\sfw}{{\mathsf{w}}}
\newcommand{\sfx}{{\mathsf{x}}}
\newcommand{\sfy}{{\mathsf{y}}}
\newcommand{\sfz}{{\mathsf{z}}}
\newcommand{\sfA}{{\mathsf{A}}}
\newcommand{\sfB}{{\mathsf{B}}}
\newcommand{\sfC}{{\mathsf{C}}}
\newcommand{\sfD}{{\mathsf{D}}}
\newcommand{\sfE}{{\mathsf{E}}}
\newcommand{\sfF}{{\mathsf{F}}}
\newcommand{\sfG}{{\mathsf{G}}}
\newcommand{\sfH}{{\mathsf{H}}}
\newcommand{\sfI}{{\mathsf{I}}}
\newcommand{\sfJ}{{\mathsf{J}}}
\newcommand{\sfK}{{\mathsf{K}}}
\newcommand{\sfL}{{\mathsf{L}}}
\newcommand{\sfM}{{\mathsf{M}}}
\newcommand{\sfN}{{\mathsf{N}}}
\newcommand{\sfO}{{\mathsf{O}}}
\newcommand{\sfP}{{\mathsf{P}}}
\newcommand{\sfQ}{{\mathsf{Q}}}
\newcommand{\sfR}{{\mathsf{R}}}
\newcommand{\sfS}{{\mathsf{S}}}
\newcommand{\sfT}{{\mathsf{T}}}
\newcommand{\sfU}{{\mathsf{U}}}
\newcommand{\sfV}{{\mathsf{V}}}
\newcommand{\sfW}{{\mathsf{W}}}
\newcommand{\sfX}{{\mathsf{X}}}
\newcommand{\sfY}{{\mathsf{Y}}}
\newcommand{\sfZ}{{\mathsf{Z}}}

\newcommand{\TV}{d_{\rm TV}}

\newcommand{\floor}[1]{{\left\lfloor {#1} \right \rfloor}}
\newcommand{\ceil}[1]{{\left\lceil {#1} \right \rceil}}

\usepackage{xspace,prettyref}
\newcommand{\CML}{\widehat{C}_{\rm ML}}
\newcommand{\diverge}{\to\infty}
\newcommand{\eqdistr}{{\stackrel{\rm (d)}{=}}}
\newcommand{\iiddistr}{{\stackrel{\text{\iid}}{\sim}}}
\newcommand{\ones}{\mathbf 1}
\newcommand{\zeros}{\mathbf 0}
\newcommand{\reals}{{\mathbb{R}}}
\newcommand{\integers}{{\mathbb{Z}}}
\newcommand{\naturals}{{\mathbb{N}}}
\newcommand{\rationals}{{\mathbb{Q}}}
\newcommand{\naturalsex}{\overline{\mathbb{N}}}
\newcommand{\symm}{{\mbox{\bf S}}}  % symmetric matrices
\newcommand{\supp}{{\rm supp}}
\newcommand{\eexp}{{\rm e}}
\newcommand{\rexp}[1]{{\rm e}^{#1}}
\newcommand{\identity}{\mathbf I}
\newcommand{\allones}{\mathbf J}
%\newcommand{\zeros}{\mathbf 0}

\newcommand{\diff}{{\rm d}}

\newcommand{\Expect}{\mathbb{E}}
\newcommand{\expect}[1]{\mathbb{E}\left[ #1 \right]}
\newcommand{\eexpect}[1]{\mathbb{E}[ #1 ]}
\newcommand{\expects}[2]{\mathbb{E}_{#2}\left[ #1 \right]}
\newcommand{\tExpect}{{\tilde{\mathbb{E}}}}
%\newcommand{\Prob}{\mathop{\mathbb{P}}}
\newcommand{\Prob}{\mathbb{P}}
\newcommand{\pprob}[1]{ \mathbb{P}\{ #1 \} }
\newcommand{\prob}[1]{ \mathbb{P}\left\{ #1 \right\} }
\newcommand{\tProb}{{\tilde{\mathbb{P}}}}
\newcommand{\tprob}[1]{{ \tProb\left\{ #1 \right\} }}
\newcommand{\hProb}{\widehat{\mathbb{P}}}
\newcommand{\probs}[2]{\mathbb{P}_{#2}\left\{ #1 \right\} }
\newcommand{\toprob}{\xrightarrow{\Prob}}
\newcommand{\tolp}[1]{\xrightarrow{L^{#1}}}
\newcommand{\toas}{\xrightarrow{{\rm a.s.}}}
\newcommand{\toae}{\xrightarrow{{\rm a.e.}}}
\newcommand{\todistr}{\xrightarrow{{\rm D}}}
\newcommand{\toweak}{\rightharpoonup}
\newcommand{\var}{\mathsf{var}}
\newcommand{\Cov}{\text{Cov}}
\newcommand\indep{\protect\mathpalette{\protect\independenT}{\perp}}
\def\independenT#1#2{\mathrel{\rlap{$#1#2$}\mkern2mu{#1#2}}}
\newcommand{\Bern}{{\rm Bern}}
\newcommand{\Binom}{{\rm Binom}}
\newcommand{\Pois}{{\rm Pois}}
\newcommand{\Hyp}{{\rm Hyp}}
\newcommand{\eg}{e.g.\xspace}
\newcommand{\iid}{i.i.d.\xspace}
% for prettyref.sty
\newrefformat{eq}{(\ref{#1})}
\newrefformat{chap}{Chapter~\ref{#1}}
\newrefformat{sec}{Section~\ref{#1}}
\newrefformat{alg}{Algorithm~\ref{#1}}
\newrefformat{fig}{Fig.~\ref{#1}}
\newrefformat{tab}{Table~\ref{#1}}
\newrefformat{rmk}{Remark~\ref{#1}}
\newrefformat{clm}{Claim~\ref{#1}}
\newrefformat{def}{Definition~\ref{#1}}
\newrefformat{cor}{Corollary~\ref{#1}}
\newrefformat{lmm}{Lemma~\ref{#1}}
\newrefformat{prop}{Proposition~\ref{#1}}
\newrefformat{app}{Appendix~\ref{#1}}
\newrefformat{hyp}{Hypothesis~\ref{#1}}
\newrefformat{thm}{Theorem~\ref{#1}}
\newrefformat{ass}{Assumption~\ref{#1}}
\newcommand{\ntok}[2]{{#1,\ldots,#2}}
\newcommand{\pth}[1]{\left( #1 \right)}
\newcommand{\qth}[1]{\left[ #1 \right]}
\newcommand{\sth}[1]{\left\{ #1 \right\}}
\newcommand{\bpth}[1]{\Bigg( #1 \Bigg)}
\newcommand{\bqth}[1]{\Bigg[ #1 \Bigg]}
\newcommand{\bsth}[1]{\Bigg\{ #1 \Bigg\}}
\newcommand{\norm}[1]{\left\|{#1} \right\|}
\newcommand{\lpnorm}[1]{\left\|{#1} \right\|_{p}}
\newcommand{\linf}[1]{\left\|{#1} \right\|_{\infty}}
\newcommand{\lnorm}[2]{\left\|{#1} \right\|_{{#2}}}
\newcommand{\Lploc}[1]{L^{#1}_{\rm loc}}
\newcommand{\hellinger}{d_{\rm H}}
\newcommand{\Fnorm}[1]{\lnorm{#1}{\rm F}}
\newcommand{\fnorm}[1]{\|#1\|_{\rm F}}
% \newcommand{\opnorm}[1]{\lnorm{#1}{\rm op}}
\newcommand{\opnorm}[1]{\left\| #1 \right\|_2}
% inner product
\newcommand{\iprod}[2]{\left \langle #1, #2 \right\rangle}
\newcommand{\Iprod}[2]{\langle #1, #2 \rangle}
% 12/02/2007
\newcommand{\indc}[1]{{\mathbf{1}_{\left\{{#1}\right\}}}}
\newcommand{\Indc}{\mathbf{1}}
\newcommand{\diag}[1]{\mathsf{diag} \left\{ {#1} \right\} }
\newcommand{\degr}{\mathsf{deg} }

\newcommand{\dTV}{d_{\rm TV}}
\newcommand{\tb}{\widetilde{b}}
\newcommand{\tr}{\widetilde{r}}
\newcommand{\tc}{\widetilde{c}}
\newcommand{\tu}{{\widetilde{u}}}
\newcommand{\tv}{{\widetilde{v}}}
\newcommand{\tx}{{\widetilde{x}}}
\newcommand{\ty}{{\widetilde{y}}}
\newcommand{\tz}{{\widetilde{z}}}
\newcommand{\tA}{{\widetilde{A}}}
\newcommand{\tB}{{\widetilde{B}}}
\newcommand{\tC}{{\widetilde{C}}}
\newcommand{\tD}{{\widetilde{D}}}
\newcommand{\tE}{{\widetilde{E}}}
\newcommand{\tF}{{\widetilde{F}}}
\newcommand{\tG}{{\widetilde{G}}}
\newcommand{\tH}{{\widetilde{H}}}
\newcommand{\tI}{{\widetilde{I}}}
\newcommand{\tJ}{{\widetilde{J}}}
\newcommand{\tK}{{\widetilde{K}}}
\newcommand{\tL}{{\widetilde{L}}}
\newcommand{\tM}{{\widetilde{M}}}
\newcommand{\tN}{{\widetilde{N}}}
\newcommand{\tO}{{\widetilde{O}}}
\newcommand{\tP}{{\widetilde{P}}}
\newcommand{\tQ}{{\widetilde{Q}}}
\newcommand{\tR}{{\widetilde{R}}}
\newcommand{\tS}{{\widetilde{S}}}
\newcommand{\tT}{{\widetilde{T}}}
\newcommand{\tU}{{\widetilde{U}}}
\newcommand{\tV}{{\widetilde{V}}}
\newcommand{\tW}{{\widetilde{W}}}
\newcommand{\tX}{{\widetilde{X}}}
\newcommand{\tY}{{\widetilde{Y}}}
\newcommand{\tZ}{{\widetilde{Z}}}


\newcommand{\wh}{\widehat}
\newcommand{\wt}{\widetilde}
\newcommand{\whp}{\rm w.h.p.}
\newcommand{\wpal}{\rm with probability at least~}
\newcommand{\pc}{Planted Clique\xspace}


\newcommand{\calA}{{\mathcal{A}}}
\newcommand{\calB}{{\mathcal{B}}}
\newcommand{\calC}{{\mathcal{C}}}
\newcommand{\calD}{{\mathcal{D}}}
\newcommand{\calE}{{\mathcal{E}}}
\newcommand{\calF}{{\mathcal{F}}}
\newcommand{\calG}{{\mathcal{G}}}
\newcommand{\calH}{{\mathcal{H}}}
\newcommand{\calI}{{\mathcal{I}}}
\newcommand{\calJ}{{\mathcal{J}}}
\newcommand{\calK}{{\mathcal{K}}}
\newcommand{\calL}{{\mathcal{L}}}
\newcommand{\calM}{{\mathcal{M}}}
\newcommand{\calN}{{\mathcal{N}}}
\newcommand{\calO}{{\mathcal{O}}}
\newcommand{\calP}{{\mathcal{P}}}
\newcommand{\calQ}{{\mathcal{Q}}}
\newcommand{\calR}{{\mathcal{R}}}
\newcommand{\calS}{{\mathcal{S}}}
\newcommand{\calT}{{\mathcal{T}}}
\newcommand{\calU}{{\mathcal{U}}}
\newcommand{\calV}{{\mathcal{V}}}
\newcommand{\calW}{{\mathcal{W}}}
\newcommand{\calX}{{\mathcal{X}}}
\newcommand{\calY}{{\mathcal{Y}}}
\newcommand{\calZ}{{\mathcal{Z}}}

\newcommand{\comp}[1]{{#1^{\rm c}}}
\newcommand{\Leb}{{\rm Leb}}
\newcommand{\Th}{{\rm th}}

\newcommand{\PDS}{{\sf PDS}\xspace}
\newcommand{\PC}{{\sf PC}\xspace}
\newcommand{\BPDS}{{\sf BPDS}\xspace}
\newcommand{\BPC}{{\sf BPC}\xspace}
\newcommand{\DKS}{{\sf DKS}\xspace}
\newcommand{\ML}{{\rm ML}\xspace}
\newcommand{\SDP}{{\rm SDP}\xspace}
\newcommand{\SBM}{{\sf SBM}\xspace}

\newcommand{\ER}{Erd\H{o}s-R\'enyi\xspace}

\newcommand{\Tr}{\mathsf{Tr}}
\renewcommand{\hat}{\widehat}
\renewcommand{\tilde}{\widetilde}

\newcommand{\MMSE}{{\rm MMSE}}
\newcommand{\snr}{{\mathsf{snr}}}

\newcommand{\tsigma}{\tilde{\sigma}}

\newcommand{\planted}{\sigma^*}
\newcommand{\score}{\calT}

%\usepackage{mleftright}

  \begin{document}
	


\title{Statistical Problems with Planted Structures: Information-Theoretical and Computational Limits}

\date{\today}
\author{ 
Yihong Wu \and Jiaming Xu\thanks{
%This research was supported by the National Science Foundation under
%Grant ECCS 10-28464, IIS-1447879, and CCF-1423088, and
%Strategic Research
%Initiative on Big-Data Analytics of the College of Engineering
%at the University of Illinois, and DOD ONR Grant N00014-14-1-0823, and Grant 328025 from the Simons Foundation. 
Y.~Wu is with Department of Statistics and Data Science, 
Yale University, New Haven, CT 06520, USA, \texttt{yihong.wu@yale.edu}.
J.~Xu is with the Fuqua School of Business, Duke University,
Durham, NC 27708, \texttt{jiaming.xu868@duke.edu}.}
%Krannert School of Management, Purdue University,
%       West Lafayette, IN 47907, USA, \texttt{xu972@purdue.edu}.}
}
  
  \maketitle

\begin{abstract}
Over the past few years, insights from computer science, statistical physics, and information theory have revealed phase transitions in a wide array of high-dimensional statistical problems at two distinct thresholds: One is the information-theoretical (IT) threshold below which the observation is too noisy so that inference of the ground truth structure is impossible regardless of the computational cost; the other is the computational threshold above which inference can be performed efficiently, i.e., in time that is polynomial in the input size. In the intermediate regime, inference is information-theoretically possible, but conjectured to be computationally hard.

This article provides a survey of the common techniques for determining the sharp IT and computational limits, using community detection and submatrix detection as illustrating examples. For IT limits, we discuss tools including the first and second moment method for analyzing the maximum likelihood estimator, information-theoretic methods for proving impossibility results using  mutual information and rate-distortion theory, and methods originated from statistical physics such as interpolation method. To investigate computational limits, we describe a common recipe to construct a randomized polynomial-time reduction scheme that approximately maps instances of the planted clique problem to the problem of interest in total variation distance.
\end{abstract}

\tableofcontents
  
 \input{main}

\begin{appendices}

\input{AppA}
\input{AppB}

\end{appendices}



 \input{IT_computation_arxiv.bbl}
%\bibliographystyle{alpha}
  %\bibliography{../../one_community/graphical_combined,strings,wu_big,wu_new,spca}

\end{document}

%\bibliographystyle{alpha}
  %\bibliography{../../one_community/graphical_combined,strings,wu_big,wu_new,spca}

\end{document}

%\bibliographystyle{alpha}
  %\bibliography{../../one_community/graphical_combined,strings,wu_big,wu_new,spca}

\end{document}

%\bibliographystyle{alpha}
  %\bibliography{../../one_community/graphical_combined,strings,wu_big,wu_new,spca}

\end{document}
