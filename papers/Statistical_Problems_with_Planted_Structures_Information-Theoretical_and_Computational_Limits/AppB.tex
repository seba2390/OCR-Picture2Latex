\section{Proof of \prettyref{eq:second_moment_conditional} $\implies$ \prettyref{eq:MIcorr2} and 
verification of  \prettyref{eq:second_moment_conditional} in the binary symmetric SBM}
\label{app:MITV}
Combining \prettyref{eq:ss} with 
\prettyref{eq:TI} and \prettyref{eq:Tbern}, we have
$I(\sigma_1,\sigma_2; A) = o(1)$ if and only if $\TV(\P_+,\P_-) =o(1)$,
where $\P_+=P_{A|\sigma_1=\sigma_2}$ and $\P_-=P_{A|\sigma_1\neq\sigma_2}$.
%Next we prove that for any reference distribution $\Q$,  \prettyref{eq:second_moment_conditional} implies 
%$d_{\rm TV} \left( \P_{+}, \P_{-} \right) =o(1)$.
Note the following characterization about the total variation distance, which simply follows from the Cauchy-Schwartz inequality:
\begin{equation}
\TV(\P_+,\P_-) = \frac{1}{2} \sqrt{\inf_{\Q}  \int  \frac{  \left( \P_{+} - \P_{-} \right)^2 }{\Q }}
\label{eq:TVquadratic}
\end{equation}
where the infimum is taken over all probability distributions $\Q$. 
Therefore \prettyref{eq:second_moment_conditional} implies \prettyref{eq:MIcorr2}.

%
 %Indeed, by Cauchy-Schwartz inequality, for any $\Q$, we have
%\begin{align}
%d_{\rm TV}  \left( \P_{+}, \P_{-} \right) & =
%\frac{1}{2} \int  \left| \P_{+} - \P_{-} \right| \nonumber \\
%%& = \frac{1}{2} \int  \left| \P_{z} - \P_{\tz} \right| \frac{\sqrt{\Q}}{\sqrt{\Q}} \nonumber \\
%& \le \frac{1}{2}  
%\left( \int \Q  \right)^{1/2} \left( \int  \frac{  \left( \P_{+} - \P_{-} \right)^2 }{\Q }  \right)^{1/2}
%\nonumber \\
%%& = \frac{1}{2} \left(    \int \frac{\P^2_{+}}{\Q} +\int \frac{\P^2_{-}}{\Q} - 2 \int \frac{ 
%%\P_{+} \P_{-} }{ \Q} \right)^{1/2} 
%&\overset{\prettyref{eq:second_moment_conditional}}{=} o(1).
%\end{align}
%%where the last equality holds from the assumption \prettyref{eq:second_moment_conditional}. 



Finally, we consider the binary symmetric SBM and show that,
below the correlated recovery threshold $\tau=\frac{(a-b)^2}{2(a+b)}<1$, 
 \prettyref{eq:second_moment_conditional} is satisfied if the reference distribution $\Q$ is the distribution of $A$ in
the null (\ER) model. Note that 
$$
\int  \frac{  \left( \P_{+} - \P_{-} \right)^2 }{\Q }  =
\int \frac{\P^2_{+}}{\Q} +\int \frac{\P^2_{-}}{\Q} - 2 \int \frac{ \P_{+} \P_{-} }{ \Q}.
$$
%Hence, it is sufficient to show
%$$
 %\int \frac{ \P^2 }{ \Q} = O(1), \; \text{ and } 
 %\int \frac{ \P_{z} \P_{\tilde{z} } }{ \Q} =  
%(1+o(1)) \int \frac{ \P^2 }{ \Q}, \quad  \forall z, \tilde{z} \in \{\pm \}.
%$$
Hence, it is sufficient to show
$$
 \int \frac{ \P_{z} \P_{\tilde{z} } }{ \Q} =  
C+o(1), \quad  \forall z, \tilde{z} \in \{\pm \}
$$
for some constant $C$ independent of $z$ and $\tz$.
Specifically, following the derivations in \prettyref{eq:SBM_second_moment_eq},
we have
\begin{align}
\int \frac{   \P_{z}  \P_{\tz} }{ \Q } 
&= \Expect \qth{  \prod_{i < j} \left(1 +  \sigma_i \sigma_j \tsigma_i \tsigma_j \rho \right)
\mathrel{\bigg|} \sigma_1 \sigma_2=z, \tsigma_1 \tsigma_2 =\tz }  \nonumber \\
& =  \left( 1+o(1) \right) e^{ -\tau^2/4 -\tau/2} \times 
\Expect \qth{ \exp \left( \frac{\rho}{2} \iprod{ \sigma}{\tsigma}^2  \right) \mathrel{\Big|} \sigma_1 \sigma_2=z, \tsigma_1 \tsigma_2 =\tz },
%& = \prod_{(i,j) \in \calE(S) } \left(1 +  \sigma_i \sigma_j \tsigma_i \tsigma_j \rho \right)  \nonumber \\
%& \times \Expect \qth{ \prod_{i \in S, j \in S^c}  \left(1 +  \sigma_i \sigma_j \tsigma_i \tsigma_j \rho \right) \mid \sigma_S, \tsigma_S } \nonumber \\
%& \times \Expect \qth{ \prod_{(i,j)\in \calE(S^c)}  \left(1 +  \sigma_i \sigma_j \tsigma_i \tsigma_j \rho \right) \mid \sigma_S, \tsigma_S }.
\end{align}
where the last equality holds $\rho=\tau/n + O(1/n^2)$ and $\log(1+x) = x -x^2/2 +O(x^3)$. 
%\nbr{
%I got $e^{ -\tau^2/4 - \tau/2}$, because 
%$\sum_{i < j} \sigma_i \sigma_j \tsigma_i \tsigma_j = \frac{1}{2}(
%\iprod{ \sigma}{\tsigma}^2 - n)$.}

Write $\sigma=2\xi-1$ for $\xi\in\{0,1\}^n$ and let 
$$
H_1\triangleq \xi_1 \tilde{\xi}_1 + \xi_2 \tilde{\xi}_2 
\quad \text{ and } \quad 
H_2 \triangleq \sum_{j \ge 3}^n \xi_j \tilde{\xi}_j.
$$ 
Then $\iprod{\sigma}{\tsigma} = 4 (H_1+H_2) -n$.
Moreover, conditional  on $\sigma_1, \sigma_2$ and $\tsigma_1, \tsigma_2$, 
$$
H_2 \sim \text{Hypergeometric} \left( n-2, n/2 - \xi_1-\xi_2, n/2 - \tilde{\xi}_1-\tilde{\xi}_2 \right).
$$
Since $|H_1| \le 2$, $\xi_1+\xi_2 \le 2 $, and $\tilde{\xi}_1+\tilde{\xi}_2 \le 2$, 
it follows that 
conditional on $\sigma_1 \sigma_2=z, \tsigma_1 \tsigma_2 =\tz $,
$ \frac{1}{\sqrt{n}} ( 4H_1 + 4H_2 - n )$ converges to $\calN(0,1)$ in distribution 
as $n \to \infty$ by the central limit theorem for hypergeometric distribution. 
Therefore
\begin{align*}
& \Expect \qth{ \exp \left(  \frac{\rho}{2} \iprod{ \sigma}{\tsigma}^2  \right) \mathrel{\bigg|} \sigma_S=z, \tsigma_S =\tz} \\
& =\expect{ \exp \left(  \frac{n\rho}{2} \left( \frac{ 4 H_1 + 4H_2 - n}{\sqrt{n} } \right)^2 \right) \mathrel{\bigg|} 
\sigma_1 \sigma_2=z, \tsigma_1 \tsigma_2 =\tz }  \\
& = \frac{1+o(1)}{\sqrt{1-\tau} },
\end{align*}
where the last equality holds due to $n \rho = \tau +o(1/n)$, $\tau<1$, 
and the convergence of the moment generating function. 

%In conclusion, we have shown that 
%$$
%\int \frac{   \P_{z}  \P_{\tz} }{ \Q}  
%= \left( 1+o(1) \right) e^{ -\tau^2/4 - \tau/2}  \frac{1}{\sqrt{1-\tau}}, \quad \forall z,\tz \in \{ \pm \}.
%$$
%%Hence, by taking the expectation of $\sigma_S$ and $\tsigma_S$ over the both hand sides of the last displayed equation, we get that
%Averaging both sides over $z,\tz$ yields
%$$
%\int \frac{   \P^2 }{ \Q}  
%= \left( 1+o(1) \right) e^{ -\tau^2/4 - \tau/2}  \frac{1}{\sqrt{1-\tau}}.
%$$
%Thus \prettyref{eq:second_moment_conditional} is satisfied.
