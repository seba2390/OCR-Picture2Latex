%
% File acl2021.tex
%
%% Based on the style files for EMNLP 2020, which were
%% Based on the style files for ACL 2020, which were
%% Based on the style files for ACL 2018, NAACL 2018/19, which were
%% Based on the style files for ACL-2015, with some improvements
%%  taken from the NAACL-2016 style
%% Based on the style files for ACL-2014, which were, in turn,
%% based on ACL-2013, ACL-2012, ACL-2011, ACL-2010, ACL-IJCNLP-2009,
%% EACL-2009, IJCNLP-2008...
%% Based on the style files for EACL 2006 by 
%%e.agirre@ehu.es or Sergi.Balari@uab.es
%% and that of ACL 08 by Joakim Nivre and Noah Smith

\documentclass[11pt,a4paper]{article}
\usepackage[hyperref]{acl2021}
\usepackage{times}
\usepackage{latexsym}
\renewcommand{\UrlFont}{\ttfamily\small}

% This is not strictly necessary, and may be commented out,
% but it will improve the layout of the manuscript,
% and will typically save some space.
\usepackage{microtype}
\usepackage{multirow}
\usepackage{booktabs}
\usepackage{enumitem}
\usepackage[normalem]{ulem}
\useunder{\uline}{\ul}{}
\usepackage{caption}
\usepackage{subcaption}
\usepackage{graphicx}
\usepackage{verbatim}
\usepackage{ifthen}


\aclfinalcopy % Uncomment this line for the final submission
\def\aclpaperid{3461} %  Enter the acl Paper ID here

%\setlength\titlebox{5cm}
% You can expand the titlebox if you need extra space
% to show all the authors. Please do not make the titlebox
% smaller than 5cm (the original size); we will check this
% in the camera-ready version and ask you to change it back.
% \newcommand{\draftonly}[1]{#1}
% % Uncomment for submission
% % \renewcommand{\draftonly}[1]{}
% \newcommand{\draftcomment}[1]{\draftonly{#1}}

\newcommand\BibTeX{B\textsc{ib}\TeX}

\newboolean{showcomments}

\setboolean{showcomments}{true}  
\newcommand{\ta}[1]{\ifthenelse{\boolean{showcomments}}{{\color{orange}tal:[#1]}}{}}
\newcommand{\nik}[1]{\ifthenelse{\boolean{showcomments}}{{\color{cyan}nik:[#1]}}{}}
\newcommand{\sof}[1]{\ifthenelse{\boolean{showcomments}}{{\color{red}sof:[#1]}}{}}
\newcommand{\liz}[1]{\ifthenelse{\boolean{showcomments}}{{\color{blue}liz:[#1]}}{}}
\newcommand{\suchin}[1]{\ifthenelse{\boolean{showcomments}}{{\color{magenta}sg:[#1]}}{}}
\newcommand{\nas}[1]{\ifthenelse{\boolean{showcomments}}{{\color{brown}nas:[#1]}}{}}


\title{All That's `Human' Is Not Gold: \\ Evaluating Human Evaluation of Generated Text}

\author{Elizabeth Clark$^1$ \qquad Tal August$^1$ \qquad Sofia Serrano$^1$ \qquad Nikita Haduong$^1$ \\ \bf{Suchin Gururangan$^1$} \qquad \bf{Noah A. Smith$^{1,2}$} \\
$^1$Paul G. Allen School of Computer Science \& Engineering, University of Washington \\
$^2$Allen Institute for Artificial Intelligence \\
\texttt{\{eaclark7,taugust,sofias6,qu,sg01,nasmith\}@cs.washington.edu}}


\begin{document}
\maketitle
\begin{abstract}
Human evaluations are typically considered the gold standard in natural language generation, but as models' fluency improves,
how well can evaluators detect and judge machine-generated text?
We run a study assessing non-experts' ability to distinguish between human- and machine-authored text (GPT2 and GPT3) in three domains (stories, news articles, and recipes).
We find that, without training, evaluators distinguished between GPT3- and human-authored text at random chance level. 
We explore three approaches for quickly training evaluators to better identify GPT3-authored text (detailed instructions, annotated examples, and paired examples) and find that while evaluators' accuracy improved up to 55\%, it did not significantly improve across the three domains.
Given the inconsistent results across text domains and the often contradictory reasons evaluators gave for their judgments, 
we examine the role untrained human evaluations play in NLG evaluation
and provide recommendations to NLG researchers for improving human evaluations of text generated from state-of-the-art models.


\end{abstract}



\section{Introduction}
\begin{figure}[ht]
\centering
\includegraphics[width=.48\textwidth]{images/fig1_tall_light.png} % removed the trim formatting  scale=0.152, trim={4cm 1cm 4cm 1cm,}
\caption{Excerpts from human evaluators' explanations for why they believe a GPT3-generated story (also excerpted) was written by a human (left) or a machine (right). The evaluators point to a wide range of text attributes to make their decisions, sometimes using the same aspect of the text to come to opposite conclusions.}
\label{fig:intro_fig}
\end{figure}

% Why do human evals matter in NLG? What role does human-authored text play in human evals?
Human-quality text has long been a holy grail for the output of natural language generation (NLG) systems, serving as an upper bound on their performance. %, since these systems typically learn from human-written text. 
Since we lack a good way of encoding many aspects of what constitutes human-quality output in an automated method, we often must rely on human evaluation for our models.
Though evaluations with end-users in an applied setting are encouraged~\cite{belz-reiter-2006-comparing}, in practice, most human evaluations instead ask people to rate generated text's \emph{intrinsic} quality
\citep{van-der-lee-etal-2019-best, howcroft-etal-2020-twenty}.
Sometimes the generated text is explicitly compared to human-authored text \citep[e.g.,][]{liu-etal-2016-evaluate,zellers_turingadvice, Zhang2020PEGASUSPW}, but even when no human-authored text is evaluated, evaluators implicitly compare the generated text to their knowledge of language and norms within specific domains. %\ta{, all of which is defined by human-written text}.


Evaluators are often asked to assess a text holistically, e.g., based on its overall quality, naturalness, or humanlikeness \citep{vanderlee_journal, howcroft-etal-2020-twenty}, 
where the exact evaluation criteria is left to the discretion of the evaluator.
Though other evaluations are broken down along specific dimensions of text quality (e.g., grammaticality, coherence, etc.), \citet{novikova-etal-2017-need,novikova-etal-2018-rankme} and \citet{callison-burch-etal-2007-meta} found that these dimensions are often correlated and may be conflated in some evaluation settings.
This is concerning because, as NLG models improve, evaluators are asked to read longer passages of text conditioned on large amounts of context. In these cases, fluency-related aspects of quality (i.e., the ones that don't require careful reading of the context and meaning of the passage) are the easiest to assess, particularly in small-batch evaluations with non-expert evaluators where speed is incentivized.
This poses a challenge when collecting human evaluations for state-of-the-art language models, as errors are often content-based (e.g., factual inaccuracies or inconsistencies with the context) rather than fluency-based \citep{gpt3}, so a superficial read may not be sufficient to catch model errors.
For accurate assessments of generated text, we need human evaluations that are designed to encourage a sufficiently careful reading of the text to examine these subtler aspects of text quality.

% What did we do in this work? What did we find?
We asked non-expert evaluators to assess the humanlikeness (operationalized as how believably human an evaluator finds a text) of text generated by current NLG models (GPT2 and GPT3) to test what current human evaluation practices can reveal about the models' quality (\S\ref{sec:exp_1}).
We found that evaluators were unable to distinguish between GPT3- and human-authored text across story, news, and recipe domains.
However, when we categorized the aspects of text the evaluators used to make their judgments, we found they primarily focused on the grammar, spelling, and style of the text.
The evaluators' responses also indicated that they underestimated the quality of text current models are capable of generating (as seen in Figure \ref{fig:intro_fig}).
To our knowledge, this paper is the first to evaluate human evaluations of GPT3-generated text across multiple domains.

We then looked at three different evaluator training methods---providing detailed instructions, annotated examples, and human-machine paired examples---to test whether we could improve evaluators' accuracy (\S\ref{sec:exp_2}). While we found including examples in the task increased the set of texts evaluators thought could be machine-generated and increased their focus on textual content, no training method significantly increased evaluators' performance consistently across domains.

Based on our results (discussed in \S\ref{sec:discussion}), we recommend moving away from small-batch evaluations with little training when collecting human evaluations of NLG models (\S\ref{sec:recommendations}).
We also encourage practitioners to consider alternative evaluation frameworks that capture the usefulness of generated text in downstream settings rather than its humanlikeness.
\section{How well can untrained evaluators identify machine-generated text?}\label{sec:exp_1}
In our first study, we ask how well untrained evaluators can distinguish between human- and machine-generated text.
This task format, inspired by the \citet{turing_test} Test, is used to compare the quality of machine-generated text to human-authored text and, as models' fluency improves, to analyze NLG models' ability to ``fool'' readers \citep{garbacea-etal-2019-judge,ippolito-etal-2020-automatic,gpt3}. 

By asking evaluators to assess the humanlikeness of the text
with only minimal instructions (see Figure \ref{fig:task}),
we observe how well untrained evaluators can detect state-of-the-art machine-generated text and which attributes evaluators focus on and think are important for detecting machine-generated text.

\subsection{The Task}
We gave evaluators 5 text passages, some of which were written by people and some generated by a model.
We asked them to rate the text on a 4-point scale \citep{ippolito-etal-2020-automatic}:
\begin{enumerate}[noitemsep]
    \item Definitely human-written
    \item Possibly human-written
    \item Possibly machine-generated
    \item Definitely machine-generated
\end{enumerate}
If they selected option 1, we asked them: ``Why did you select this rating?'' Otherwise, they were asked, ``What would you change to make it seem more human-like?''
The interface is shown in Figure \ref{fig:task}.

\begin{figure}[ht]
\centering
\includegraphics[scale=0.45, trim={4cm 1.7cm 4cm 2cm},clip]{images/task}
\caption{The task interface (story domain)}
\label{fig:task}
\end{figure}

\subsection{Data}\label{sec:data}
We considered human- and machine-generated text in three different domains: stories, news articles, and recipes. In all three cases, we collected 50 human-authored texts in English and generated 50 texts from both the 175B parameter GPT3 model (also known as Davinci; \citealp{gpt3})\footnote{\url{beta.openai.com/}} and GPT2-XL \citep{gpt2}.\footnote{\url{huggingface.co/gpt2-xl}}
Evaluators were assigned to one domain and one model; the texts read by any given evaluator included some human-authored texts and some texts generated by their assigned model.
We only considered texts 100 words or longer,
and after reaching 100 words, all texts were truncated at the end of the next sentence.\footnote{Using NLTK; \url{www.nltk.org/}}

%For the machine-generated text,
To generate text, we used the ``three-shot'' setting described in \citet{gpt3}, conditioning the text on three additional samples of in-domain, human-authored text, which we refer to as the \textit{priming texts} (all priming texts are in the supplementary materials and at \url{ark.cs.washington.edu/human_evals_ACL21}).
While this setting is not typically how GPT2 is used in practice, we held this approach constant to directly compare how model quality changes evaluators' ability to distinguish between texts.
For each domain, each generated text was conditioned on the same set of priming texts.
The texts were delimited with an $\langle$EOS$\rangle$ token and generated using the default GPT3 generation settings (i.e., sampling with temperature $=0.7$).

\subsubsection{Stories}
The human-authored texts came from the Reddit WritingPrompts dataset \citep{fan-etal-2018-hierarchical}.\footnote{\url{github.com/pytorch/fairseq/tree/master/examples/stories}} 
We collected all the stories that began with \textit{Once upon a time} (255 stories total) and randomly chose 50 human-authored stories from this set. For the machine-generated text, we conditioned the models on the three priming texts and on the phrase \textit{Once upon a time}.
We removed generated stories that directly copied a priming text (with $>80\%$ overlap) and regenerated those texts (9 instances with GPT2, 2 with GPT3).

% dataset attributes
This is the most open-ended of the three domains, as the story's content is virtually unrestricted, and the only creative domain.
It is also the noisiest of the human-authored datasets, as the stories were originally collected from social media comments with no quality-based filtering.

\subsubsection{News Articles}
We collected 2,111 recent local news articles from 15 different newspapers using Newspaper3k\footnote{\url{github.com/codelucas/newspaper}} (details in Appendix \ref{app:newspapers}).
After filtering out articles under 100 words, we manually filtered out articles that weren't local news or that referenced the coronavirus pandemic.
We randomly chose 50 articles to use as our human-authored news articles and another 50 to use as prompts for our generation models. 
We conditioned each generated text on the headline and first sentence from the prompt articles, along with the three priming texts.

% dataset attributes
Because the title and the first sentence of a news article often summarize its contents, the generated content must adhere to the topics they introduce. By using local, recent news, we also limit the models' ability to copy from their training data.
The models seemed to have the most trouble with this dataset structurally, e.g., generating new headlines without ending the current article or outputting invalid end-of-file tags.

\subsubsection{Recipes}
We collected 50 human-authored recipes from the RecipeNLG dataset \citep{bien-etal-2020-recipenlg}, which contains 2,231,142 recipes scraped from the web. We randomly chose an additional 50 recipes and used their titles and ingredient lists as prompts, appending them to the end of the priming texts.

% dataset attributes
This is the most closed of the three domains, as the recipe must incorporate the listed ingredients and result in the dish described by the title. Recipes are typically written in clear commands, leaving little room for surprising or unexpected text.


\subsection{Participants}\label{sec:participants}
We used Amazon Mechanical Turk (AMT) to collect the text evaluations with non-expert evaluators, commonly used in NLG evaluations \citep{van-der-lee-etal-2019-best}.
To have adequate power in our analyses (based on a power analysis with $\beta=0.8$; \citealp{card-etal-2020-little}), we had 130 different evaluators for each of the 6 task settings (3 domains $\times$ 2 models).
Each participant evaluated 5 texts each, giving us a total of 780 participants and 3,900 text evaluations.

We paid evaluators US\$1.25 for completing the task.
Following common best practice on AMT~\cite{berinsky2012evaluating}, evaluators had to have over a 95\% acceptance rate, be in the United States, and have completed over 1,000 HITs (AMT tasks).
We excluded evaluators' work if their explanations were directly copied text from the task, 
did not match their responses, did not follow the instructions, or were short, vague, or otherwise uninterpretable.
Across experiments, 445 participants (18.6\%) were rejected and not included in the \S\ref{sec:exp_1} results (780 approved participants) and \S\ref{sec:exp_2} results (1,170 approved participants).

\begin{table*}
\centering
\begin{tabular}{lrlrrrrrrr}
\toprule
Model & \begin{tabular}[c]{@{}l@{}}Overall\\ Acc.\end{tabular} & Domain & Acc. & $F_1$ & Prec. & Recall & Kripp. $\alpha$ & \begin{tabular}[c]{@{}l@{}} \% \\ human\end{tabular} & \begin{tabular}[c]{@{}l@{}} \% \\ confident\end{tabular} \\
\midrule
\multirow{3}{*}{GPT2} & \multirow{3}{*}{*0.58} & Stories & *0.62 & 0.60 & 0.64 & 0.56 & 0.10 & 55.23 & 52.00 \\
 &  & News & *0.57 & 0.52 & 0.60 & 0.47 & 0.09 & 60.46 & 51.38 \\
 &  & Recipes & 0.55 & 0.48 & 0.59 & 0.40 & 0.03 & 65.08 & 50.31 \\
 \midrule
\multirow{3}{*}{GPT3} & \multirow{3}{*}{0.50} & Stories & 0.48 & 0.40 & 0.47 & 0.36 & 0.03 & 62.15 & 47.69 \\
 &  & News & 0.51 & 0.44 & 0.54 & 0.37 & 0.05 & 65.54 & 52.46 \\
 &  & Recipes & 0.50 & 0.41 & 0.50 & 0.34 & 0.00 & 66.15 & 50.62 \\
 \bottomrule
\end{tabular}
\caption{\S\ref{sec:exp_1} results, broken down by domain and model, along with the $F_1$, precision, and recall at identifying machine-generated text, Krippendorff's $\alpha$, \% human-written guesses, and  \% confident guesses (i.e., \textit{Definitely} machine- or human-authored). * indicates the accuracies significantly better than random (two-sided $t$-test, 
for Bonferroni-corrected $p<0.00333$).}
\label{tab:exp_1_results}
\end{table*}

\subsection{Results}
Overall, evaluators choosing between human and GPT2-generated text correctly identified the author of the text 57.9\% of the time,\footnote{Unless otherwise noted, all analyses binned the responses into 2 categories (\textit{human} and \textit{machine}).} but the evaluators choosing between human- and GPT3-generated text only guessed correctly 49.9\% of the time (Table \ref{tab:exp_1_results}), compared to 50\% random chance.
While the accuracy of classifying GPT2- vs. human-authored text is significantly\footnote{$t_{388}=6.58$, $p<0.0001$} different from chance, evaluators' accuracy distinguishing GPT3- and human-authored text is not.\footnote{$t_{388}=-0.09$, $p=0.93$}
This remains the case regardless of text domain; we failed to find any evidence that evaluators' accuracy on any one domain for GPT3 differs from the overall GPT3 accuracy of $\approx50$\%.\footnote{ANOVA with $F_{2,390}=0.78$, $p=0.46$}
The story texts saw the biggest drop in evaluator accuracy from GPT2 to GPT3 (62\% to 48\%, Cohen's $d=0.57$).
The distribution of evaluators' scores are shown in Appendix \ref{app:exp1_histograms}.

In Table \ref{tab:exp_1_results}, we see other statistics worsen as well between GPT2 and GPT3: how well evaluators identified the machine-generated text ($F_1$, precision, and recall), evaluators' agreement (Krippendorff's $\alpha$, a measure of annotator agreement that corrects for the probability of random agreement), and the percent of guesses that the text was human-written (\% human).
Given that the texts are equally likely to be human- and machine-written, there are disproportionately many \textit{human} guesses, making up two thirds of the responses in the GPT3 experiments.
Despite the significantly lower scores, evaluators' confidence (the percent of \textit{Definitely} responses) remains fairly constant across conditions.

\subsection{Analysis}
Taken on its own, the evaluators' difficulty identifying GPT3-generated text compared to GPT2 points to the improvement of new NLG models.
However, it also points to concerns about extending current human evaluation methodologies to state-of-the-art text generation.
In particular, the evaluators' explanations reveal underlying confusion and misconceptions about state-of-the-art NLG.

To better understand what untrained evaluators focused on in the text to make their decisions, the authors annotated 150 random responses from the evaluators who distinguished between human- and GPT3-generated text (see Appendix \ref{app:annotation} for annotation details).
We divided the text annotation labels into three categories: \textit{form}, \textit{content}, and \textit{machine capabilities}. \textit{Form} qualities focus on the format, style, and tone of the text, while \textit{content} focuses on the text's meaning. We also coded for comments that explicitly referenced people's perceptions of what types of language machines are capable (or incapable) of generating (\textit{machine capabilities}).

We found nearly twice as many comments about the form of the text than the content (\textit{form}: 47\% of labels, \textit{content}: 25\%). Evaluators in our sample focused most on the spelling, grammar, or punctuation of the texts (45 out of 150 comments) and the style or tone of the text (24 out of 150 comments). However, these dimensions of text are unlikely to be helpful in identifying text generated by current models, considering that GPT3 has already been shown to generate fluent text and to adapt easily to new generation domains \citep{gpt3}.

We also found that the reasons evaluators gave for their answers often contradicted each other. The formality of the text, spelling and grammar errors, and clarity were all cited to justify both \textit{human} and \textit{machine} judgments.
This was also reflected in the low agreement scores between evaluators, with Krippendorff's $\alpha\approx0$ across domains.

Evaluators' expectations about what NLG models are capable of ranged from thinking their text is already indistinguishable from human-authored text (``I have no idea if a human wrote anything these days. No idea at all.'') to doubting machines' ability to use basic language (``Usually AI has terrible grammer [sic] and messes up.'').
But overall we found most evaluators' beliefs about generated language underestimated or misunderstood current NLG models, as seen in Appendix \ref{app:HUM}.
\section{Can we train evaluators to better identify machine-generated text?}\label{sec:exp_2}

Given evaluators' inability to distinguish GPT3- and human-authored text and their inconsistent reasoning for their decisions, we investigated whether there were simple ways of improving evaluators' ability to spot attributes of GPT3-generated text.
Inspired by crowdsourcing research on guiding workers on writing or other subjective tasks~\cite{kim2017mechanical, mitra_crowdsourcing}, we tested three \emph{lightweight} evaluator-training methods to see if we could improve people's ability to identify machine-generated text while maintaining the short, low-cost nature of the evaluations. 


\subsection{Evaluator Training Methods}
We considered 3 evaluator trainings that can be added to the beginning of a human evaluation task, at most requiring only 3 extra samples of human- and machine-generated text. 
To test the effectiveness of each type of training, we re-ran the experiments from \S\ref{sec:exp_1}, but this time, we prepended one of three  evaluator-training methods to the evaluation task: an \emph{instruction-based} training, an \emph{example-based} training, and a \emph{comparison-based} training.
Screenshots of the training interfaces are in Appendix \ref{app:training+instructions}; the full set of training materials are in the supplementary materials and at \url{ark.cs.washington.edu/human_evals_ACL21}.

Other than the training, the task setup was identical to the GPT3-based tasks in \S\ref{sec:exp_1}.
We again ran the task on Amazon Mechanical Turk across three domains (stories, news, and recipes), using the same texts.
As each individual participant was only permitted to complete one set of evaluations, the set of evaluators who received these trainings was completely disjoint from the set of evaluators from our first study.
The participants were subject to the same restrictions described in \S\ref{sec:participants} and excluded according the same criteria; we did not use the trainings to filter out evaluators.
For each domain and training method pair, we had 130 unique evaluators complete the task, giving us 5,850 text annotations from 1,170 evaluators.

\subsubsection{Training with Instructions}
% explain method (point to screenshot, any references)
To give evaluators a better sense of which parts of the text to pay attention to, we extended the original task instructions to include dimensions of the text that could be helpful for identifying machine-generated text (repetition and factuality) and ones that could be misleading (grammar, spelling, and style).
We chose these dimensions based on previous work \citep{ippolito-etal-2020-automatic} and evaluators' comments in a pilot study (see Appendix \ref{app:pilot}). 

% explain pros/cons
The Instructions training was the simplest of our 3 evaluator training methods.
It was general enough to be applied across the 3 domains but provided little information about the quality and domain of text the evaluator would be rating.
It did not increase the cost of collecting evaluations (US\$1.25 per HIT) because it does not require any extra work on the part of the evaluator, though this also made it the easiest training to ignore.
The instruction-based training is the most prescriptive of the training methods, as the researcher has to choose the dimensions they want the evaluators to focus on.

\subsubsection{Training with Examples}
% explain method (point to screenshot, any references)
Our Examples training consisted of 3 practice rounds of the actual task: given a text, guess if it is machine- or human-authored.
We collected 3 additional texts in the same manner described in \S\ref{sec:data} and wrote a short explanation of which aspects of the text hinted at its source.
After an evaluator makes their guess, the correct answer and explanation are shown.
Each domain had its own set of examples and explanations.

% explain pros/cons
By showing examples, this training helps set the evaluators' expectations about the quality of the human- and machine-generated text.
We paid evaluators more for completing this task (US\$1.75 per HIT) to compensate for the extra texts they needed to read.
As with the instruction-based training, while pointing out specific text dimensions can help evaluators focus on important features, it may also restrict their search space.

\subsubsection{Training with Comparison}
% explain method (point to screenshot, any references)
In the Comparison training, we took the example passages from the Examples training and paired them with a text from the opposite source (machine or human) that began with the same prompt. 
We asked evaluators to guess which of the two texts was the machine-generated one.
We then provided the correct answer to the evaluator, along with the same explanations used in the Examples training.

% explain pros/cons
This training allows evaluators to directly compare human and machine texts written from the same prompt.
It is also the most expensive training, as it required evaluators to read three more passages than the Examples training; we paid evaluators US\$2.25 per HIT.

\begin{table*}[ht]
\centering
\begin{tabular}{lrlrrrrrrr}
\toprule
Training & \begin{tabular}[c]{@{}l@{}}Overall\\ Acc.\end{tabular} & Domain & Acc. & $F_1$ & Prec. & Recall & Kripp. $\alpha$ & \begin{tabular}[c]{@{}l@{}}\%\\ human\end{tabular} & \begin{tabular}[c]{@{}l@{}}\%\\ confident\end{tabular} \\
\midrule
\multirow{3}{*}{None} & \multirow{3}{*}{0.50} & Stories & 0.48 & 0.40 & 0.47 & 0.36 & 0.03 & 62.15 & 47.69 \\
 &  & News & 0.51 & 0.44 & 0.54 & 0.37 & 0.05 & 65.54 & 52.46 \\
 &  & Recipes & 0.50 & 0.41 & 0.50 & 0.34 & 0.00 & 66.15 & 50.62 \\
 \midrule
\multirow{3}{*}{Instructions} & \multirow{3}{*}{0.52} & Stories & 0.50 & 0.45 & 0.49 & 0.42 & 0.11 & 57.69 & 45.54 \\
 &  & News & 0.56 & 0.48 & 0.55 & 0.43 & 0.05 & 62.77 & 52.15 \\
 &  & Recipes & 0.50 & 0.41 & 0.52 & 0.33 & 0.07 & 67.69 & 49.85 \\
 \midrule
\multirow{3}{*}{Examples} & \multirow{3}{*}{*0.55} & Stories & 0.57 & 0.55 & 0.58 & 0.53 & 0.06 & 53.69 & 64.31 \\
 &  & News & 0.53 & 0.48 & 0.52 & 0.45 & 0.05 & 58.00 & 65.69 \\
 &  & Recipes & 0.56 & 0.56 & 0.61 & 0.51 & 0.06 & 55.23 & 64.00 \\
 \midrule
\multirow{3}{*}{Comparison} & \multirow{3}{*}{0.53} & Stories & 0.56 & 0.56 & 0.55 & 0.57 & 0.07 & 48.46 & 56.62 \\
 &  & News & 0.52 & 0.51 & 0.53 & 0.48 & 0.08 & 53.85 & 50.31 \\
 &  & Recipes & 0.51 & 0.49 & 0.52 & 0.46 & 0.06 & 54.31 & 53.54 \\
 \bottomrule
\end{tabular}
\caption{\S\ref{sec:exp_2} results, broken down by domain and training method, along with the $F_1$, precision, and recall at identifying machine-generated text, Krippendorff's $\alpha$, \% human-written guesses, and \% confident guesses (i.e., \textit{Definitely} machine- or human-authored). ``None'' training refers to the GPT3 results from \S\ref{sec:exp_1}. * indicates accuracies significantly better than None (no training; two-sided $t$-test, for Bonferroni-corrected $p<0.00333$).}
\label{tab:exp_2_results}
\end{table*}

\subsection{Results}
We found that while all 3 training methods improved evaluators' accuracy at identifying machine- vs. human-authored text over the no-training accuracy, the Examples training was the only one that showed significant improvement (see Table \ref{tab:exp_2_results}).\footnote{Tukey's HSD adjusted $p<0.003$ for distinguishing between the Examples training and no training, $d=0.25$}

Breaking down the results by domain, however, we find the Examples accuracy did not significantly increase over the no-training accuracy when considering any of the three domains individually.
Even so, the significant difference in overall performance is mainly contributed by the story domain;
when comparing evaluators' performance with no training to its Examples training counterpart, we see a change of 0.019 and 0.062 mean accuracy in the news and recipe domains, respectively, versus 0.086 on the story domain. 
This is perhaps due to the examples helping override the preconception that machines cannot generate ``creative'' text.

Across all 3 domains, the Examples and Comparison trainings produced the highest recall and $F_1$ scores for evaluators' judgments
and decreased the percentage of texts they guessed were human-written, which indicate that evaluators were willing to consider a broader set of texts to be machine-generated than the evaluators in \S\ref{sec:exp_1}. However, despite the trainings and the increased proportion of confident responses, evaluator agreement remained low across domain and training settings ($\alpha \leq 0.11)$, and higher agreement did not correspond to higher accuracy.

\subsection{Analysis}

We again annotated 150 comments along the dimensions listed in Appendix \ref{app:annotation}, divided into \textit{form}, \textit{content}, and \textit{machine capabilities} categories, this time from evaluators who received the best-performing Examples training.
As shown in Table \ref{tab:annotation}, we found that the proportion of \textit{form} comments dropped in the sample of evaluators who went through the Examples training, while the proportion of \textit{content} comments doubled. 
We also saw a drop in the number of comments mentioning evaluators' expectations of machine-generated text. While this change in focus doesn't necessarily correspond to correct judgments, \textit{content} reasons are more in-line with current NLG model capabilities~\cite{gpt3}.


\begin{table}
\centering
\begin{tabular}{llll}
\toprule
Training & \multicolumn{1}{l}{Form} & Content & \begin{tabular}[c]{@{}l@{}}Machine\\ capabilities\end{tabular} \\
\midrule
None & 47.1 & 24.6 & 28.3 \\
Examples & 32.5 & 50.0 & 17.5 \\
\bottomrule
\end{tabular}
\caption{\% of annotation labels that reference the text's form and content and the evaluator's perception of machines' capabilities}
\label{tab:annotation}
\end{table}
\section{Discussion}\label{sec:discussion}
Overall, none of our three training methods significantly improved evaluators' ability to detect machine-generated text reliably across text domains while still maintaining the small-batch nature of Amazon Mechanical Turk.
This speaks to the improving quality of NLG models, but we also found that untrained evaluators mainly focused on the format of the text, deciding if it was human or machine-generated based on whether the text was grammatically or stylistically correct. This, combined with the high percentage of \textit{human} guesses, the low recall scores for the \textit{machine} guesses,
and the evaluators' comments on their expectations of NLG models, indicates a systematic underestimation by the evaluators of the quality of machine-generated text. 
Evaluators who were trained with examples had higher expectations of machine-generated text and focused more on the text's content; however, the training was not sufficient to significantly raise evaluators' scores across all three domains.

Many of the explanations given by evaluators included references to the text that reflected human attributes or intent that they suspected machines could not generate (e.g., ``personal description a machine wouldn't understand, [like a pirate] wanting to be home with his wife and son'' from Figure \ref{fig:intro_fig} and the examples in Appendix \ref{app:HUM}).
However, current NLG models are capable of generating text with at least superficial reference to human attributes or intent, as seen in the generated story in Figure \ref{fig:intro_fig}. This assumption that machines can't generate text with these aspects of humanlikeness led many evaluators astray, and we suspect it is one cause of the low accuracy we found. 

Crowdsourcing studies dealing only with human-authored texts often include extensive training, quality checks, or coordination~\cite{10.1145/1460563.1460572, kim2017mechanical, bernstein2010soylent}.
NLG evaluations usually forego such structures, based, we suspect, on the assumption that evaluating machine-generated text requires only fluency in the language the text is generated in. 
Our results suggest otherwise. Evaluators often mistook machine-generated text as human, citing superficial textual features that machine generation has surpassed~\cite{gpt3}. 
One potential remedy for this is to focus evaluator training on debunking this misconception. We did see evidence that the increase in accuracy we saw with our Examples training was associated with fewer explanations mistakenly referencing machine capabilities, even though the training did not specifically focus on this.
\section{Recommendations}\label{sec:recommendations}
Based on our findings, if NLG researchers must run human evaluations as small-batch evaluations on Amazon Mechanical Turk or similar platforms, we recommend they train evaluators with examples.
This will help calibrate the evaluators' expectations of generated text and indicate the careful reading they may need to do to properly assess the text's quality.
Our experiments also indicate the importance of confirming with evaluators why they have made the decisions they have, as the criteria they might implicitly be evaluating may be mismatched with researchers' intended criteria.
However, other evaluation setups may be more successful on Amazon Mechanical Turk, such as long-term evaluations with qualified evaluators who have gone through an extended training (like those in \citealp{10.1145/1460563.1460572}; \citealp{zellers-etal-2019-hellaswag}) or third-party evaluator quality tools (e.g., Positly, used by \citealp{gpt3}).

However, given the increasing length of text NLG models can handle and the careful reading needed to detect many errors in generated text, we encourage NLG researchers to move away from standalone, intrinsic human evaluation tasks.
We found that, by default,
our evaluators in this evaluation setting were most likely to focus on surface-level, fluency-related aspects of quality.
We join past work \citep{belz-reiter-2006-comparing, vanderlee_journal} in recommending a move towards evaluation settings where evaluators are better motivated to carefully consider the content and usefulness of generated text.
For example, TuringAdvice \citep{zellers_turingadvice} asks evaluators to rate NLG models by their ability to generate helpful advice, and RoFT \citep{dugan-etal-2020-roft} engages evaluators through a guessing game to determine the boundary between human- and machine-generated text. Other evaluation methods ask the evaluators to directly interact with the generated text; for example, Choose Your Own Adventure \citep{clark-smith-2021-choose} and Storium \citep{akoury-etal-2020-storium} evaluate story generation models by having people write stories with the help of generated text.\footnote{Note that we initially tried a fourth training condition along these lines, where we asked evaluators to directly interact with the generated text by rewriting it to be more humanlike. We found we were unable to successfully recruit evaluators to complete this task. The rate of retention was less than 30\%, and the rejection rate was over 50\%. We found AMT was not a good platform for this type of task, at least not for the format and the price point we explored in this work.}
We see that GPT3 can successfully mimic human-authored text across several domains, renewing the importance of evaluations that push beyond surface-level notions of quality and consider whether a text is helpful in a downstream setting or has attributes that people would want from machine-generated text.

Finally, given the mixed effect we found different trainings can have on evaluators' performance and the lack of human evaluation details typically presented in NLG papers \citep{van-der-lee-etal-2019-best, howcroft-etal-2020-twenty}, we encourage NLG researchers to include details of any instructions and training they gave evaluators in their publications.
This, along with efforts to standardize human evaluation design \citep{belz-etal-2020-disentangling, howcroft-etal-2020-twenty} and deployment \citep{genie, gem}, will support future development of evaluator training procedures and the comparison of human evaluation results in future NLG evaluation work.
\section{Related Work}
A subfield of NLG analyzes the role of human evaluations, including discussions of the tradeoffs of human and automatic evaluations \citep{belz-reiter-2006-comparing, hashimoto-etal-2019-unifying}.
There are critiques and recommendations for different aspects of human evaluations, like the evaluation design \citep{novikova-etal-2018-rankme, santhanam-shaikh-2019-towards}, question framing \citep{schoch-etal-2020-problem}, and evaluation measures like agreement \citep{amidei-etal-2018-rethinking}, as well as analyses of past NLG papers' human evaluations \citep{vanderlee_journal, howcroft-etal-2020-twenty}.
Additionally, crowdsourcing literature has work on effectively using platforms like Amazon Mechanical Turk \citep[e.g.,][]{florian_crowdsourcing,oppenheimer_crowdsourcing,weld_crowdsourcing,mitra_crowdsourcing}.
In this work, we focus on the role evaluator training can play for producing better accuracy at distinguishing human- and machine-generated text, though other quality control methods are worth exploring.

Previous work has asked evaluators to distinguish between human- and machine-authored text. For example, \citet{ippolito-etal-2020-automatic} found that trained evaluators were able to detect open-ended GPT2-L-generated text 71.4\% of the time, \citet{garbacea-etal-2019-judge} reported that individual evaluators guessed correctly 66.6\% of the time when evaluating product reviews, and \citet{gpt3} found evaluators could guess GPT3-davinci-generated news articles' source with 52\% accuracy, though these results are not directly comparable to ours due to differences in the evaluation setup, data, and participants.

Finally, our findings that untrained evaluators are not well equipped to detect machine-generated text point to the importance of researching the safe deployment of NLG systems. \citet{gehrmann-etal-2019-gltr} proposed visualization techniques to help readers detect generated text, and work like \citet{grover_zellers}, \citet{ippolito-etal-2020-automatic}, and \citet{uchendu-etal-2020-authorship} investigated large language models' ability to detect generated text.
 We propose a novel commonsense reasoning challenge, \textsc{RiddleSense}, which requires complex commonsense skills for reasoning about creative and counterfactual questions, coming with a large multiple-choice QA dataset.  
 We systematically evaluate recent commonsense reasoning methods over the proposed \textsc{RiddleSense} dataset, and find that the best model is still far behind human performance, suggesting that there is still much space for commonsense reasoning methods to improve.
 We hope \textsc{RiddleSense} can serve as a benchmark dataset for future research targeting complex commonsense reasoning and computational creativity.


\section*{Acknowledgements}
This research is supported in part by the Office of the Director of National Intelligence (ODNI), Intelligence Advanced Research Projects Activity (IARPA), via Contract No. 2019-19051600007, the DARPA MCS program under Contract No. N660011924033 with the United States Office Of Naval Research, the Defense Advanced Research Projects Agency with award W911NF-19-20271, and NSF SMA 18-29268. The views and conclusions contained herein are those of the authors and should not be interpreted as necessarily representing the official policies, either expressed or implied, of ODNI, IARPA, or the U.S. Government. We would like to thank all the collaborators in USC INK research lab and the reviewers for their constructive feedback on the work.

\section*{Acknowledgments}
This research was supported in part by the Office of Naval Research under the MURI grant N00014-18-1-2670. The authors would like to thank OpenAI, specifically Bianca Martin and Miles Brundage, for providing access to GPT3 through the OpenAI API Academic Access Program. The authors would also like to thank Katharina Reinecke, the members of the CSE 599 crowdsourcing class, and the ARK group for their feedback, the reviewers for their helpful comments, and the participants who took part in our study.

\paragraph{Ethical considerations}
All experiments in this paper were approved by our institution's internal review board.
Evaluators' responses were collected and stored anonymously.
Evaluators were paid based on an estimated US\$10 per hour rate; we raised the price of the task in proportion to the added difficulty of our 3 training methods.
For each dataset we considered, its source and language are included, along with any other details we believed would be relevant to evaluators' ability to read and understand the text.
Evaluators were warned about possible risks before starting the task, namely that NLG models can generate text with harmful language or themes, and were able to leave comments about their experience at the end of the study.




\bibliographystyle{acl_natbib}
\bibliography{anthology,non_acl}

\appendix
\newpage

\section*{Appendix}
\label{sec:appendix}

For the sake of completeness, we here report the same analyses depicted in Figure~\ref{fig:tranco_tp} and Figure~\ref{fig:ca_perf_tp} showing the number of Trackers instead of the number of Third-Parties. The two pictures lead to similar conclusions.

\begin{figure}[!h]
    \centering
    \includegraphics[width=0.5\columnwidth]{figures/cookieaccept_tranco_rank_eu_tracker_nb.pdf}
    \caption{Average number of Trackers per website (Tranco list).}
    \label{fig:tranco_trackers}
\end{figure}


\begin{figure}[!h]
    \centering
    \includegraphics[width=0.5\textwidth]{figures/cookieaccept_tracker_nb_tranco.pdf}
    \caption{Number of Trackers (Tranco list). Notice the log scales.}
    \label{fig:ca_perf_tracker}
\end{figure}
%\clearpage \newpage

\section{Supplementary Material}
This document contains the model priming texts and evaluator training materials for ``All That's `Human' Is Not Gold:  Evaluating Human Evaluation of Generated Text.''


\subsection{Priming Texts}\label{sup:exemplars}
When generating the text passages, we conditioned the text on three in-domain, human-authored texts. The exact passages we used for each domain are in Tables \ref{tab:story_exemplars}--\ref{tab:recipe_exemplars}.


\begin{table*}[h]
\small
\begin{tabular}{p{\linewidth}}
\toprule
Once upon a time magic was an art. The great magicians of the time would take on apprentices who would train for a decade or more, slaving under their tutelage for the hope of one day being the next great magician. They were free from the bonds of apprenticeship only once they had made the perfect magic circle. Only then could they harness the power of the mages and go off on their own. But times changed, magicians were persecuted, and few longed to toil away for the long years of study necessary. In the course of a century, magic became extinct. \\
\midrule
\begin{tabular}[c]{@{}p{\linewidth}@{}}Once upon a time, there lived an old woman. She was as you'd expect an old woman to be in most respects, brittle and stooped as though whatever foundation still held her old bones up might break at any minute and cause the whole abode to come tumbling down. Her eyes were glazed like a bakers treats - what they'd come to call cataracts in years to come. Her joints were all swollen, such that her fingers were like poorly balanced, chubby spiders from a children's tale. But not all was as one thinks an old woman should be.\\ \\ She was wise beyond sight.\end{tabular} \\
\midrule
\begin{tabular}[c]{@{}p{\linewidth}@{}}Once upon a time, there lived a boy who was a close friend to the princess of a rival town.\\ \\ They, despite the many warnings made by their fathers, were close to one another, often seen sneaking pieces of raided pies from the marketplace vendors and chasing each other in the meadows that rested between the two towns.\\ \\ For many years, the townspeople tolerated their friendship, many secretly hoping that it would lead to a truce between each other.\\ \\ However, on the boy's fifteenth birthday, the uneasy peace between the rivaling factions was shattered as rumors of the princess being kidnapped spread and sparked an almost palatable bloodlust in the girl's people.\end{tabular} \\
\bottomrule
\end{tabular}
\caption{The three passages used to condition the generated story text.}\label{tab:story_exemplars}
\end{table*}

\begin{table*}[h]
\small
\begin{tabular}{p{\linewidth}}
\toprule
\begin{tabular}[c]{@{}p{\linewidth}@{}}Headline: New Jersey Devils' affiliate moves from Binghamton to Newark for 2021 AHL season\\ Article: Come anticipated Feb. 5 commencement of the American Hockey League’s abbreviated season, the Binghamton Devils will call Newark, New Jersey, home. AHL President and Chief Executive Officer Scott Howson disclosed the 2020-21 season will include 28 teams operating in five divisions. Four have been granted provisional relocation for the 2020-21 season: The B-Devils in Newark; the Ontario Reign in El Segundo, California; the Providence Bruins in Marlborough, Massachusetts; and the San Diego Gulls in Irvine, California. ``We've gone back and forth with (parent club) New Jersey about how to get through this COVID situation and we've looked at all the expenses, and — they have the final decision on all this stuff,'' said Tom Mitchell, B-Devils Executive Vice President of Operations.\end{tabular} \\
\midrule
\begin{tabular}[c]{@{}p{\linewidth}@{}}Headline: Connecticut’s Beardsley Zoo Welcomes New River Otter Following Renovation of Habitat\\ Article: Connecticut’s Beardsley Zoo is the new home for Tahu, a one-year-old female North American river otter newly arrived from the Woodland Park Zoo in Seattle, Wash. After the Zoo’s last river otter passed away in 2019 from advanced age, the Zoo engaged in long-planned improvements to the otter habitat. With renovations complete, Tahu has joined the Zoo family and will be joined by a male companion in the future. As a species, river otters have suffered from habitat loss, water pollution, and fur trapping. Their numbers are on the rise due to reintroduction programs in parts of the U.S., better water quality, and protection of their habitats. Zoo Director Gregg Dancho said, ``Our river otters have always been some of the most popular animals who make their home here at the Zoo, for their playful nature and intelligence as well as their role as an iconic North American animal.\end{tabular} \\
\midrule
\begin{tabular}[c]{@{}p{\linewidth}@{}}Headline: Johnson City eyesore to get a makeover: This is what you may see from Rt. 17\\ Article: Construction to transform the long-vacant Endicott Johnson Victory building, often described as Broome County's "biggest eyesore," into 108 market-rate apartments plus commercial space is expected to get underway in the fall of 2021. The anticipated \$30-million project and its targeted completion date of 2023 was announced Monday by Paulus Development, the Syracuse-based firm that renovated the former Ansco factory on Binghamton's West Side into a multi-unit apartment building. The developers acquired the Johnson City parcel at 59 Lester Ave. in March. Matthew Paulus, president of Paulus Development, said the apartments will contain modern amenities and enclosed parking, along with 7,500 square feet of retail space. It's anticipated renters would include young professionals and empty-nesters among others. "Where there is a lot of additional opportunity are people that will be coming in for private sector or public sector jobs.\end{tabular} \\
\bottomrule
\end{tabular}
\caption{The three passages used to condition the generated news text.}\label{tab:news_exemplars}
\end{table*}

\begin{table*}[h]
\small
\begin{tabular}{p{\linewidth}}
\toprule
\begin{tabular}[c]{@{}p{\linewidth}@{}} Title: All-Natural Strawberry Sauce\\ Ingredients:\\ 2 cups fresh strawberries\\ 13 cup good quality honey\\ 1 vanilla bean, split open, seeds removed and included\\ 1 12 tablespoons good balsamic vinegar\\ Recipe:\\ Roughly chop the strawberries and toss the pieces into a 2 quart saucepan.\\ Add honey, vanilla bean, and seeds and bring to a boil.\\ You may have to add 1/4 C of water, depending on how juicy your strawberries are.\\ Cook the mixture until the strawberries are tender and cooked, about 10 minutes.\\ As long as you leave the mixture on medium heat, you can walk away from pot.\\ Just make sure the pot has enough moisture so the sauce wont burn (you can add more water if necessary).\\ When the strawberries are cooked and the sauce thickens to a syrup-like consistency, add the balsamic vinegar and continue to cook for a few minutes longer to thicken again.\end{tabular} \\
\midrule
\begin{tabular}[c]{@{}p{\linewidth}@{}} Title: Crispy Potato Cake\\ Ingredients:\\ 4 large russet or Idaho potatoes (about 2 pounds), peeled\\ 1 1/4 teaspoons salt\\ 18 teaspoon pepper\\ 1/4 cup vegetable oil, plus more as needed\\ Recipe:\\ Using a mandolin or food processor, julienne the potatoes.\\ Toss with salt and pepper.\\ In a 12-inch skillet, heat the oil over high heat until very hot but not smoking.\\ Add the potatoes and press with the back of a spatula to form an even layer.\\ Lower the heat to medium and cook until the bottom is golden brown, 10 to 20 minutes, shaking the pan frequently so the potatoes don't stick and adding oil as needed.\\ Flip the potato cake, add more oil if necessary and cook until the other side is golden brown.\\ Cool the cake briefly on a wire rack and cut into 6 wedges.\end{tabular} \\
\midrule
\begin{tabular}[c]{@{}p{\linewidth}@{}} Title: Sesame-Ginger Salmon With Braised Bok Choy\\ Ingredients:\\ 1 1/2 pounds bok choy\\ 2 teaspoons minced garlic\\ 4 teaspoons salad oil\\ 3/4 cup fat-skimmed chicken broth\\ 1 1/2 pounds boned salmon fillet with skin (maximum 1 in. thick)\\ 4 teaspoons sesame seed\\ 1 1/2 teaspoons ground ginger\\ Recipe:\\ Rinse bok choy; trim off and discard tough stem ends and any bruised parts. Cut the leafy tops crosswise into 2-inch strips; cut the stems crosswise into 1-inch pieces.\\ In a 10- to 12-inch frying pan over high heat, stir garlic in 2 teaspoons oil until sizzling, 1 to 2 minutes. Add bok choy and broth, cover, and cook until thickest stems are just tender when pierced, 4 to 5 minutes; keep warm.\\ Meanwhile, rinse salmon, pat dry, and cut into 4 equal pieces. Mix sesame seed with ginger, and rub fish evenly with mixture. Pour remaining 2 teaspoons oil into a 10- to 12-inch frying pan over high heat.\end{tabular} \\
\bottomrule
\end{tabular}
\caption{The three passages used to condition the generated recipe text.}\label{tab:recipe_exemplars}
\end{table*}

% \subsection{Training and Instructions}\label{sup:training+instructions}
% Figure \ref{fig:basic_instr} shows the basic instructions that were shown to all evaluators, in both Study \#1 and \#2, regardless of training or domain.
% All training information occurred after receiving the basic instructions.
% \begin{figure*}
% \centering
% \fbox{\includegraphics[scale=0.5, angle=270, trim={6.5cm 1cm 8cm 0.75cm},clip]{images/NONE}}
% \caption{Basic instructions shown to all evaluators}
% \label{fig:basic_instr}
% \end{figure*}

% %\subsubsection{Instruction training}\label{sup:train_instructions}
% The training shown to evaluators in the Instruction training condition is shown in Figure \ref{fig:training_instr}.
% \begin{figure*}
% \centering
% \fbox{\includegraphics[scale=0.5, angle=270, trim={9.5cm 1cm 3.5cm 1cm},clip]{images/INSTR}}
% \caption{The Instruction training}
% \label{fig:training_instr}
% \end{figure*}

% %\subsubsection{Example training}\label{sup:train_examples}
% A screenshot of the Examples training is in Figure \ref{fig:training_example} and of the Comparison training in Figure \ref{fig:training_compare}. The full set of examples and annotations used in the Examples and Comparison trainings are shown in Tables \ref{tab:story_training}-\ref{tab:recipe_training}.

% \begin{figure}
% \centering
% \includegraphics[scale=0.5, trim={1cm 0cm 4cm 1cm},clip]{images/EXAMPLE}
% \caption{The Example training (story domain)}
% \label{fig:training_example}
% \end{figure}

% \begin{figure*}
% \centering
% \includegraphics[scale=0.5]{images/COMPARE}
% \caption{The Comparison training (story domain)}
% \label{fig:training_compare}
% \end{figure*}

% Please add the following required packages to your document preamble:
% If you use beamer only pass "xcolor=table" option, i.e. \documentclass[xcolor=table]{beamer}
\begin{table*}[]
\small
\begin{tabular}{p{.5\linewidth}p{.5\linewidth}}
\toprule
GPT3 & Human \\
\midrule
\begin{tabular}[c]{@{}p{\linewidth}@{}}\textbf{*Once upon a time}, there was a man in a place that was not a place at all.\\ \\ He didn't know anything of a place or a time or who he was or what he was doing there. There was just him and the silence.\\ \\ He sat there for a long time, not knowing what he was doing there. He thought, thought and thought, but he didn't know what to think. There was just him and the silence. He tried to speak, but no sound came from his mouth. He tried to move, but his body would not move. He sat there, but he didn't know for how long he was there.\end{tabular} & \begin{tabular}[c]{@{}p{\linewidth}@{}}\textbf{Once upon a time}, there lived a little girl who ran around the village wearing a little red riding hood. Don't ask me what a riding hood is because I don't even know. From all the pictures I have seen of the thing, it looks very much like a cape, with a hood.\\ \\ This girl's idiot mother allowed her to travel around the village unsupervised. Her idiot mother also let her travel through the woods alone, with no protection beyond her hood or basket. Not a very smart parent, if you ask me. This girl can't have been older than ten or eleven.\end{tabular} \\
\midrule
\multicolumn{2}{p{\linewidth}}{\textbf{Comment:} Note how the {[}machine-authored{]} story is repetitive and doesn't seem to go anywhere.} \\
\midrule
\begin{tabular}[c]{@{}p{\linewidth}@{}}\textbf{Once upon a time}, there was a girl who was born a princess. Her birth was not unexpected; her father had long since wanted a daughter, and her mother had long since been unable to give him one. So it was that when, after years of childless marriage, a daughter was born, they rejoiced.\\ \\ The girl was raised in the lap of luxury, and as she grew, she came to expect that the world would be at her beck and call. The girl who had once been pampered and coddled was now pampered and coddled, and grew to be quite the vain, spoiled brat.\end{tabular} & \begin{tabular}[c]{@{}p{\linewidth}@{}}\textbf{*Once upon a time}, in a faraway land, a princess scowled at the newspaper. Another of her friends had been kidnapped. Another! that made six this month. She tore the thing neatly in two and almost threw it at her handmaiden, declaring, ``this has to stop. I mean, isn't life difficult enough for princesses without the constant threat of kidnap hanging over our heads?''\\ \\ The maid raised a delicate eyebrow, her thoughts turning to her own difficulties, so far removed from those of her employer, ``highness, I don't think there's anything to be done about it ...'' she complained.\\ \\ This was met with a disparaging look from Princess Ellionora, who rose to her feet, dislodging one of the flowers in her hair with a sudden head movement.\end{tabular}  \\
\midrule
\multicolumn{2}{p{\linewidth}}{\textbf{Comment: }Note that though there are some capitalization errors, the {[}human-authored{]} story is original and engaging.} \\
\midrule
\begin{tabular}[c]{@{}p{\linewidth}@{}}\textbf{*Once upon a time}, there lived a girl who was the daughter of the greatest wizard of her time.\\ \\ She was the daughter of the great wizard, yet she was not a very good magician.\\ \\ Her father instructed her to practice her magic every day. He told her to polish off the dust of the books in his library, clean the dishes in the kitchen, and sweep the front of the house.\\ \\ She did not enjoy her chores.\\ \\ She wanted to go with the other girls in the village and play. But she had to do her chores.\end{tabular} & \begin{tabular}[c]{@{}p{\linewidth}@{}}\textbf{Once upon a time}, there was a young man named Neil. Neil's family was made up of warriors on his mother's side, and poets on his father's side.\\ \\ His parents were certain that Neil would take up one or the other profession. But Neil had no interest in bloodshed, and suffered from a terribly tied tongue. And from the time he could read, to to the time when he was ten years old, Neil felt alone in his own family, a self-imposed outcast.\\ \\ However, one day, Neil saw something that changed his perspective on life. A visiting dignitary had come to the tribe to seek an alliance with them.\end{tabular} \\
\midrule
\multicolumn{2}{p{\linewidth}}{\textbf{Comment:} Note that there seem to be some inconsistencies {[}in the machine-generated story{]}; first it says the father orders her to practice magic, then it says he orders her to do chores.} \\
\bottomrule
\end{tabular}
\caption{The training materials used in the story domain for the Comparison trainings. The passages marked with * were shown in the Example training, along with the comments with the bracketed text removed.}
\label{tab:news_training}
\label{tab:story_training}
\end{table*}

\begin{table*}[]
\small
\begin{tabular}{p{.5\linewidth}p{.5\linewidth}}
\toprule
GPT3 & Human \\
\midrule
\begin{tabular}[c]{@{}p{\linewidth}@{}}\textbf{*Headline: Twitter reacts to former South Dakota QB Chris Streveler playing for Arizona Cardinals}\\ \textbf{Article: Former University of South Dakota quarterback Chris Streveler replaced Arizona Cardinals QB Kyler Murray, and South Dakotans are loving it.} Streveler, who is from Mitchell, completed two of his three passes for 44 yards, including a pass to Larry Fitzgerald for a touchdown. "I just had to get an opportunity, and I got it, and I'm grateful for that," Streveler said. "I'm just trying to take advantage of it." Streveler was the first player in University of South Dakota history to throw for more than 9,000 yards.\\ \\ Headline: RHI program to help offset carbon emissions from I-90 bridge rebuild\\ Article: The \$7 billion Gateway Project to replace the aging, often-congested, and increasingly dangerous Hudson River crossings between New York and New Jersey has received a major boost by a new agreement to offset carbon emissions from the project.\end{tabular} & \begin{tabular}[c]{@{}p{\linewidth}@{}}\textbf{Headline: Twitter reacts to former South Dakota QB Chris Streveler playing for Arizona Cardinals}\\ \textbf{Article: Former University of South Dakota quarterback Chris Streveler replaced Arizona Cardinals QB Kyler Murray, and South Dakotans are loving it.} It became Chris Streveler time for the Cardinals early in the first quarter when Murray went down with a right ankle injury. Streveler completed his first NFL pass to Trent Sherfield and then tossed a pass to Jonathan Ward, who ran past several Los Angeles Rams' defenders for a touchdown — the first of Streveler's professional career. The touchdown catch was Ward's first touch of his career. Streveler did throw a pick-six just before halftime, giving the Rams a 12-7 lead. The 6-foot-1 216-pound Streveler played sparingly earlier this season, but only for a few short run plays. With a win in Sunday's game, the Cardinals would make the playoffs.\end{tabular} \\
\midrule
\multicolumn{2}{p{\linewidth}}{\textbf{Comment:} Note that the {[}machine-generated{]} article doesn't mention many of the things brought up in the title or the first line of the article, and note how the article turns into a new headline.} \\
\midrule
\begin{tabular}[c]{@{}p{\linewidth}@{}}\textbf{Headline: One dead in Gering fire}\\ \textbf{Article: Authorities have confirmed firefighters found one person deceased in a Gering fire this afternoon (Thursday).} Gering Fire Chief John Scharfenberg says crews were dispatched to a one-story commercial building at the corner of Lincoln and 13th Street shortly before 2:30 Thursday afternoon. When crews arrived, they found heavy fire coming from the structure. They say one person was in the building at the time, but was able to get out safely. The Gering Fire Department received mutual aid from the Gering-Kimi Fire Protection District, Scottsbluff Fire Department, Scottsbluff Ambulance, Scottsbluff Police Department, and the Scottsbluff Housing Authority. Crews were able to put the fire out by around 5:30 Thursday evening.\end{tabular} & \begin{tabular}[c]{@{}p{\linewidth}@{}}\textbf{*Headline: One dead in Gering fire}\\ \textbf{Article: Authorities have confirmed firefighters found one person deceased in a Gering fire this afternoon (Thursday).} Firefighters responded to a home in the 1200 block of R Street at about 2:30 p.m. The Gering Fire Department and Scottsbluff Fire Department responded to the fire. Gering Fire Department Chief Nathan Flowers said his department received a call that smoke was emanating the structure on R street, near Lincoln Elementary School. When firefighters arrived on the scene, Flower said smoke was pouring from the windows of the small house. He said it took about half an hour to subdue the blaze and, upon entry, firefighters discovered the body. ``There was one occupant that lived in the structure. We did find the occupant deceased in the structure.\end{tabular} \\
\midrule
\multicolumn{2}{p{\linewidth}}{\textbf{Comment:} Note how {[}in the human-authored article,{]} Nathan Flowers is introduced by his full title before shortening his name and giving his quote.} \\
\midrule
\begin{tabular}[c]{@{}p{\linewidth}@{}}\textbf{*Headline: Fisher Scones parent company fined for overworking teenage employees}\\ \textbf{Article: Washington State Department of Labor and Industries found more than 1,500 instances of violations including missing meal breaks, working during school hours and more.} Fisher Scones, Inc., the parent company of Fisher Scones Bakery \& Café, has agreed to pay \$36,000 in fines for overworking teenage employees. The Washington State Department of Labor and Industries (L\&I) found more than 1,500 instances of violations including missing meal breaks, working during school hours and more. The company has agreed to pay \$36,000 in fines and will also implement an internal training program to prevent future violations. ``Fisher Scones has been a leading business in this region for many years,'' said David LaBarre, director of L\&I’s Spokane office. ``Despite their success, we found hundreds of violations over a three-year period. That’s simply unacceptable.''\end{tabular} & \begin{tabular}[c]{@{}p{\linewidth}@{}}\textbf{Headline: Fisher Scones parent company fined for overworking teenage employees}\\ \textbf{Article: Washington State Department of Labor and Industries found more than 1,500 instances of violations including missing meal breaks, working during school hours and more.} The parent company of fair staple and fan favorite Fisher Scones has been fined for more than 1,500 instances of overworking teenage employees beyond hours allowed under Washington state law, according to a Dec. 16 announcement from the Washington State Department of Labor \& Industries. Fisher Scones, under parent company Conifer Foods based in Woodinville, specializes in buttery, raspberry jam-filled scones at the Washington State Fair in Puyallup, Evergreen State Fair in Monroe and a variety of roaming food truck locations. The company started in 1977. The citation includes a \$45,100 fine, which Conifer Foods has until Dec. 23 to appeal, and includes the most work-hour violations involving minors that the state’s L\&I department has ever found, according to a release.\end{tabular} \\
\midrule
\multicolumn{2}{p{\linewidth}}{\textbf{Comment:} Note how the first line {[}of the machine-authored article{]} is repeated later on, almost verbatim.} \\
\bottomrule
\end{tabular}
\caption{The training materials used in the news domain for the Comparison trainings. The passages marked with * were shown in the Example training, along with the comments with the bracketed text removed.}
\label{tab:news_training}
\end{table*}

\begin{table*}[]
\tiny
\begin{tabular}{p{.5\linewidth}p{.5\linewidth}}
\toprule
{\small GPT3} & {\small Human} \\
\midrule
\begin{tabular}[c]{@{}p{\linewidth}@{}}\textbf{*Title: Ricotta Panna Cotta With Basil Balsamic Strawberries}\\ \textbf{  Ingredients:}\\ \textbf{ Ricotta Panna Cotta}\\ \textbf{ 2 1/2 teaspoons unflavored gelatin}\\ \textbf{ 5 tablespoons water}\\ \textbf{ 2 cups whipping cream}\\ \textbf{ 2/3 cup icing sugar}\\ \textbf{ 1 cup ricotta cheese}\\ \textbf{ 1/2 vanilla bean, slit lengthwise}\\ \textbf{ Basil Balsamic Strawberries}\\ \textbf{ 1 lb strawberry}\\ \textbf{ 1/3 cup orange juice}\\ \textbf{ 2 tablespoons balsamic vinegar}\\ \textbf{ 1/2 teaspoon cracked black pepper}\\ \textbf{ 6 large basil leaves, shredded}\\ \textbf{ Recipe:}\\ Pour water into a small bowl and sprinkle with gelatin, set aside.\\ 2. In a saucepan, heat cream, sugar and vanilla bean until it starts to boil, set aside.\\ 3. In a medium sized bowl, whisk together the ricotta cheese and the vanilla bean scrapings.\\ 4. Pour the warm cream mixture into the ricotta mixture and whisk until well combined.\\ 5. Pour into 4 small serving dishes.\\ 6. Pour the gelatin mixture over the panna cottas. Set aside for about 10 minutes or until gelatin has set.\\ 7. Meanwhile, to make the strawberries; hull and quarter the strawberries and toss in a bowl with the orange juice, vinegar and pepper.\end{tabular} &
\begin{tabular}[c]{@{}p{\linewidth}@{}}\textbf{Title: Ricotta Panna Cotta With Basil Balsamic Strawberries}\\ \textbf{ Ingredients:}\\ \textbf{ Ricotta Panna Cotta}\\ \textbf{ 2 1/2 teaspoons unflavored gelatin}\\ \textbf{ 5 tablespoons water}\\ \textbf{ 2 cups whipping cream}\\ \textbf{ 2/3 cup icing sugar}\\ \textbf{ 1 cup ricotta cheese}\\ \textbf{ 1/2 vanilla bean, slit lengthwise}\\ \textbf{ Basil Balsamic Strawberries}\\ \textbf{ 1 lb strawberry}\\ \textbf{ 1/3 cup orange juice}\\ \textbf{ 2 tablespoons balsamic vinegar}\\ \textbf{ 1/2 teaspoon cracked black pepper}\\ \textbf{ 6 large basil leaves, shredded}\\ \textbf{ Recipe:}\\ Panna Cotta: In a small bowl, soften the gelatin in cold water. Combine 1 cup of the cream, icing sugar and vanilla bean in a small pot. Bring to a boil and remove from the heat. Add the gelatin and stir to dissolve. Remove the vanilla bean and scrape the seeds into the cream.\\ Stir the remaining cream and ricotta. Pour into 8 5 - oz ramekins. Chill for at least 2 hours or up to overnight. Dip each ramekin into hot water, loosen with a knife and turn out onto individual plates.\\ Basil Balsamic Strawberries: Hull and cut the strawberries into quarters.\end{tabular} \\
\midrule
\multicolumn{2}{p{\linewidth}}{\small \textbf{Comment:} Note how {[}in the machine-generated recipe{]} the steps got misnumbered and how the gelatin never got cooked.}\\
\midrule
\begin{tabular}[c]{@{}p{\linewidth}@{}}\textbf{Title: Planet Hollywood Pot Stickers}\\ \textbf{ Ingredients:}\\ \textbf{ 1/4 lb. ground turkey}\\ \textbf{ 1/2 tsp. minced fresh ginger}\\ \textbf{ 1 tsp. minced green onion}\\ \textbf{ 1 tsp. minced water chestnuts}\\ \textbf{ 1/2 tsp. soy sauce}\\ \textbf{ 1/2 tsp. ground black pepper}\\ \textbf{ 1/4 tsp. crushed red pepper flakes (no seeds)}\\ \textbf{ 1/4 tsp. salt}\\ \textbf{ 1/8 tsp. garlic powder}\\ \textbf{ 1 egg, beaten}\\ \textbf{ Vegetable oil for frying}\\ \textbf{ 12 wonton wrappers (3 3-inch size)}\\ \textbf{ Hoisin sauce, on the side}\\ \textbf{ Recipe:}\\ Heat 2 Tbsp. oil in a large skillet over medium heat. Add ground turkey; sauté for 5 minutes, or until turkey is browned and crumbly. Stir in ginger, green onion, water chestnuts, soy sauce, black pepper, red pepper flakes, salt and garlic powder. Cook for an additional 2 minutes. Remove from heat; set aside.\\ Add 1 tsp. oil to a small skillet over medium heat. Add beaten egg; cook for 30 seconds, or until cooked. Remove from heat and cut egg into small pieces. Combine egg, turkey mixture, and 1 Tbsp. water in a small bowl.\end{tabular} & \begin{tabular}[c]{@{}p{\linewidth}@{}}\textbf{*Title: Planet Hollywood Pot Stickers}\\ \textbf{ Ingredients:}\\ \textbf{ 1/4 lb. ground turkey}\\ \textbf{ 1/2 tsp. minced fresh ginger}\\ \textbf{ 1 tsp. minced green onion}\\ \textbf{ 1 tsp. minced water chestnuts}\\ \textbf{ 1/2 tsp. soy sauce}\\ \textbf{ 1/2 tsp. ground black pepper}\\ \textbf{ 1/4 tsp. crushed red pepper flakes (no seeds)}\\ \textbf{ 1/4 tsp. salt}\\ \textbf{ 1/8 tsp. garlic powder}\\ \textbf{ 1 egg, beaten}\\ \textbf{ Vegetable oil for frying}\\ \textbf{ 12 wonton wrappers (3 3-inch size)}\\ \textbf{Hoisin sauce, on the side}\\ \textbf{ Recipe:}\\ In a small bowl, combine all the ingredients except the egg, wrappers and oil.\\ Add 1 tablespoon of the beaten egg.\\ Save the rest of the egg for later.\\ Preheat oil in a deep fryer or a deep saucepan to 375F.\\ Use enough oil to cover the pot stickers -- 1 to 2 inches should be enough.\\ Invert a small bowl or glass with a 3-inch diameter on the center of a wonton wrapper and cut around it to make a circle.\\ Repeat for the remaining wrappers.\\ Spoon 1/2 tablespoon of the turkey filling into the center of one wrapper.\end{tabular} \\
\midrule
\multicolumn{2}{p{\linewidth}}{\small \textbf{Comment:} Note the more complicated concepts {[}in the human-authored recipe{]}, like `all the ingredients except the egg, wrappers, and oil' and `save the rest of the egg for later.'} \\
\midrule
\begin{tabular}[c]{@{}p{\linewidth}@{}}\textbf{*Title: Corn And Lobster Chowder}\\ \textbf{ Ingredients:}\\ \textbf{ 1/4 lb thick slab bacon (rind removed)}\\ \textbf{ 1 tablespoon unsalted butter}\\ \textbf{ 2 cups diced onions (1/4-inch)}\\ \textbf{ 2 tablespoons all-purpose flour}\\ \textbf{ 4 cups chicken broth}\\ \textbf{ 2 bay leaves}\\ \textbf{ 1 teaspoon sweet paprika}\\ \textbf{ 2 russet potatoes, peeled and cut into 1/4-inch dice}\\ \textbf{ 3 sprigs fresh thyme}\\ \textbf{ 1 cup half-and-half}\\ \textbf{ 3 cups cooked fresh corn kernels}\\ \textbf{ 1 red bell peppers or 1 yellow bell pepper, cut into 1/4-inch dice}\\ \textbf{ 4 scallions, very thinly sliced}\\ \textbf{ salt and black pepper, to taste}\\ \textbf{ 2 cups cooked lobsters, cut into 1/2-inch dice (fresh or frozen)}\\ \textbf{ 1/4 cup flat leaf parsley}\\ \textbf{ Recipe:}\\ Cook bacon in a 3-quart saucepan over medium heat until crisp, about 8 minutes. Remove to paper towel to drain.\\ Add butter and onion to the drippings in the pan and cook over medium heat until onion is clear, about 5 minutes. Add the flour and cook, stirring constantly, for 2 minutes.\\ Add broth, bay leaves, paprika, potatoes, and thyme. Bring to a boil, reduce to a simmer, and cook for 10 minutes, stirring occasionally.\\ Add the corn, peppers, scallions, salt and pepper. Bring back to a boil, reduce heat to a simmer, and cook for 5 minutes.\end{tabular} & \begin{tabular}[c]{@{}p{\linewidth}@{}}\textbf{Title: Corn And Lobster Chowder}\\ \textbf{ Ingredients:}\\ \textbf{ 1/4 lb thick slab bacon (rind removed)}\\ \textbf{ 1 tablespoon unsalted butter}\\ \textbf{ 2 cups diced onions (1/4-inch)}\\ \textbf{ 2 tablespoons all-purpose flour}\\ \textbf{ 4 cups chicken broth}\\ \textbf{ 2 bay leaves}\\ \textbf{ 1 teaspoon sweet paprika}\\ \textbf{ 2 russet potatoes, peeled and cut into 1/4-inch dice}\\ \textbf{ 3 sprigs fresh thyme}\\ \textbf{ 1 cup half-and-half}\\ \textbf{ 3 cups cooked fresh corn kernels}\\ \textbf{ 1 red bell peppers or 1 yellow bell pepper, cut into 1/4-inch dice}\\ \textbf{ 4 scallions, very thinly sliced}\\ \textbf{ salt and black pepper, to taste}\\ \textbf{ 2 cups cooked lobsters, cut into 1/2-inch dice (fresh or frozen)}\\ \textbf{ 1/4 cup flat leaf parsley}\\ \textbf{ Recipe:}\\ Cut bacon into small dice and place in a large pot over low heat.\\ Cook, stirring, to render the fat, 5 to 7 minutes.\\ Add butter and let it melt.\\ Add onion and cook over low heat, stirring occasionally, until wilted, about 10 minutes.\\ Add flour; cook, stirring, for 1 minute longer.\\ Simmer broth, bay leaves, and paprika in the pot for 5 minutes to flavor broth.\\ Add potatoes and thyme; cook until potatoes are just tender, about 15 minutes.\\ Add half-and-half, corn, peppers, and scallions; cook 10 minutes.\\ Season with the salt and pepper.\\ Add lobster and parsley just before serving hot.\end{tabular} \\
\midrule
\multicolumn{2}{p{\linewidth}}{\small \textbf{Comment:} Note how the bacon drippings are supposed to be drained out on a paper towel but then are used in the next step, and note how the soup is supposed to be boiled and then re-boiled.} \\
\bottomrule
\end{tabular}
\caption{The training materials used in the recipe domain for the Comparison trainings. The passages marked with * were shown in the Example training, along with the comments with the bracketed text removed.}
\label{tab:recipe_training}
\end{table*}



%\section{Supplemental Material}
%\label{sec:supplemental}



\end{document}
