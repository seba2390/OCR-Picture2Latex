%
%
\section{Invariant Whitney functions}
\label{invwhitney}
%
%

We assume that $G$ is a compact Lie group acting orthogonally on $V=\R^n$.
Our goal is to prove Theorem \ref{maindiag}, an analogue of the Theorem of Schwarz for 
Whitney functions $\calE^\infty(Z)$ along a locally closed subanalytic set $Z\subset V$, which is stratified by orbit type. 
Note that we do not assume $Z$ to be $G$-stable. In order to speak of invariance of such Whitney functions we make use of 
the language of groupoids. 

\begin{definition} 
Let $Z\subset V$ be an arbitrary subset. The \emph{restricted action groupoid} $\Gamma_Z:=(G\ltimes V)_{|Z}$ is defined as follows. The set of objects $(\Gamma_Z)_0$ is defined to be $Z$ while the set 
of arrows is  $(\Gamma_Z)_1:=\{(g,z)\in G\times Z\mid g.z\in Z\}$. The composition of arrows, 
unit and inverse are defined in the obvious way. That is, the composition rule is 
$(g,z)(h,z')=(gh,z')$ in case of $z=h.z'$, the unit map sends $z$ to $(e,z)$, while the 
inverse of $(g,z)$ is $(g^{-1},g.z)$.
\end{definition}

For $g\in G$ let us write $\Phi_g$ for the (linear) diffeomorphism $V\to V$, $v\mapsto g.v$. 
We have a corresponding formal pullback (cf. Eq. (\ref{Eq:DefPullbackJet}))
\begin{eqnarray*}
(\Phi_g)^{\sharp}_{\{v\},\{g.v\}}:\sfJ^\infty(\{g.v\})\to\sfJ^\infty(\{v\}).
\end{eqnarray*}
Given a locally closed subset $Z\subset V$ we say that a jet $F=(F_{\alpha})_{\alpha\in\mathbb N^n}\in \sfJ^\infty(Z)$ is 
\emph{invariant} if the following conditions hold true:
\begin{enumerate}[{(Inv}1)]
\item \label{Ite:Action}
  The canonical action of $(G\ltimes V)_{|Z}$ on the jet bundle $\sfJ^\infty(Z)$
  leaves $F$ invariant, which means that
  for all $(g,z)\in(G\ltimes V)_{|Z}$ and  $\alpha\in \mathbb N^n$ one has 
  \begin{eqnarray*}
     \left((\Phi_g)^{\sharp}_{\{z\},\{g.z\}}\big(F (g.z) \big)\right)_\alpha=F_{\alpha}(z)
  \end{eqnarray*}
\item\label{Ite:Constant}
     The natural action of the Lie algebra $\frg$ of $G$ on the jet bundle $\sfJ^\infty(Z)$
     leaves $F$ invariant, which means that for every element $\xi \in \frg$ one has 
     \[
       \xi_V  F =0,
     \]
    where $\xi_V$ denotes the fundamental vector field of $\xi$ on $V$.
\end{enumerate}

The space of jets on $Z$ satisfying invariance condition (Inv\ref{Ite:Action}) will be  
denoted by $\sfJ^\infty(Z)^{(G\ltimes V)_{|Z}}$, the space of jets satisfying invariance condition 
(Inv\ref{Ite:Constant}) by $\sfJ^\infty(Z)^\frg$. By $\sfJ^\infty(Z)^\textup{inv}$, we denote the 
space of invariant jets, i.e.~the space
\[
  \sfJ^\infty(Z)^\textup{inv} = \sfJ^\infty(Z)^{(G\ltimes V)_{|Z}} \cap \sfJ^\infty(Z)^\frg \: .
\]
For Whitney functions, we put
\[
\begin{split}
 & \calE^\infty(Z)^{(G\ltimes V)_{|Z}}:=\sfJ^\infty(Z)^{(G\ltimes V)_{|Z}}\cap \calE^\infty(Z), \\
 & \calE^\infty(Z)^\frg:=\sfJ^\infty(Z)^\frg\cap \calE^\infty(Z), \: \text{ and } \\
 & \calE^\infty(Z)^\textup{inv}:=\sfJ^\infty(Z)^\textup{inv}\cap \calE^\infty(Z).
\end{split}
\]
We call $\calE^\infty(Z)^\textup{inv}$ the space of \emph{invariant Whitney functions}.
Finally, if $M$ denotes a $G$-manifold, and $Z\subset M$ a closed subset,
we write $\calJ^\infty(Z,M)^G$ for the space $\calJ^\infty(Z,M) \cap \calC^\infty (M)^G$
and call it the space of \emph{invariant smooth functions on $M$ flat on $Z$}.
\begin{remark}
 In case $Z\subset V$ is locally closed and $G$-stable, the restricted groupoid 
 $(G\ltimes V)_{|Z}$ coincides with the action groupoid $G\ltimes Z$. For convenience, 
 we therefore write $\calE^\infty(Z)^G$  instead of $\calE^\infty(Z)^{(G\ltimes V)_{|Z}}$ 
 in this situation. Observe  that for Whitney functions over a $G$-stable 
 $Z$, condition (Inv\ref{Ite:Constant}) follows from (Inv\ref{Ite:Action}).  In other words 
 this means that $\calE^\infty(Z)^G = \calE^\infty(Z)^\textup{inv}$, if $G.Z =Z$.
 Note that this property is essentially  a consequence  of the Theorem of Schwarz and Mather.
 See Prop.~\ref{surgrp} for details. 
\end{remark}
\begin{example} To motivate that in case of non-$G$-stable $Z$ it is necessary to impose $\mathfrak g$-invariance in addition to the $(G\ltimes V)_{|Z}$-invariance let us consider the following example. We let the circle $G:=\operatorname S^1$ operate on the plane $V:=\R^2=\{(x,y)\mid x,y\in \R\}$ by rotations and put $Z:=\{(0,1)\}$. Since $Z$ consists of a point we have $\calE^\infty(Z)=\sfJ^\infty(Z)$. The restricted action groupoid is trivial: $(G\ltimes V)_{|Z}=\{(e,0,1)\}$, where $e\in \operatorname S^1$ is the identity. Hence 
 $\sfJ^\infty(Z)^{(G\ltimes V)_{|Z}}=\sfJ^\infty(Z)$, i.e., the condition of invariance with respect to the action groupoid is void in this example. On the other hand, the fundamental vector field of the circle action evaluated at the point $(0,1)$ is $-\partial/\partial x=-\partial_{(1,0)}$ which entails that $\sfJ^\infty(Z)^\mathfrak g=\{F=(F_\alpha)_{\alpha\in \N^2}\in \sfJ^\infty(Z)\mid \alpha \notin \{0\}\times\N \Rightarrow F_\alpha=0\}$. 
\end{example}

\begin{theorem} \label{maindiag} Let $G\to \operatorname{O}(V)$ be a finite dimensional 
orthongonal representation of the compact Lie group $G$.  Let 
$\rho_1,\dots,\rho_\ell\in\mathbb R[V]^G$  be a minimal complete system of polynomial 
invariants and $\rho:=(\rho_1,\dots,\rho_\ell): V\to \mathbb R^\ell$ the corresponding 
Hilbert map. Assume to be given a subanalytic subset $Z\subset V$ which is  stratified by orbit type 
and closed in some open $G$-invariant open neighborhood $U \subset V$. 
Let us write $\calJ_X:=\calJ(X,W)$ for the ideal of smooth functions on $W$ vanishing on $X:= \rho (U)$, where
$W \subset \R^\ell$ open can and has been chosen so that $X$ is closed in $W$.  
Then in the commutative diagram
\begin{eqnarray*}
\xymatrix{&0&0&0&\\
0\ar[r]&\calJ ^\infty(Z,U)^G\ar[u]\ar[r]&\calC^\infty(U)^G\ar[u]\ar[r]_{\sfJ^\infty_{Z}\quad\quad}&\calE^\infty(Z)^\textup{inv}\ar[u]\ar[r]&0\\
0\ar[r]&\calJ ^\infty(\rho(Z), W)\ar[u]^{\rho^*_{|U}}\ar[r]&\calC^\infty(W)\ar[u]^{\rho^*_{|U}}\ar[r]_{\sfJ^\infty_{\rho(Z)}\quad}&\calE^\infty(\rho(Z))\ar[u]_{\rho^{\sharp}_{Z,\rho(Z)}}\ar[r]&0\\
0\ar[r]&\calJ ^\infty(\rho(Z), W)\cap \calJ_X \ar[u]\ar[r]& \calJ_X \ar[u]\ar[r]&\sfJ^\infty_{\rho(Z)}(\calJ_X)\ar[u]\ar[r]&0\\
&0\ar[u]&0\ar[u]&0\ar[u]&
}
\end{eqnarray*}
all rows and columns are exact sequences of linear continuous maps of Fr{\'e}chet spaces.
\end{theorem}


\begin{remark}
 Let us note that as  consequence of $\rho^{\sharp}_{Z,{\rho(Z)}}(\calE^\infty(\rho(Z)))=\calE^\infty(Z)^\textup{inv} $ one can conclude that 
 $\calE^\infty(Z)^\textup{inv}\subset \calE^\infty(Z)$ is a closed subspace. Here we make use of \cite[Theorem 3.6]{BieMilCDF} and the 
 fact that $\rho:U\to X$ is regular in the sense of Gabrielov by Prop.~\ref{Prop:GabrielovRegularityPolynomials}. Moreover, 
 occasionally $\sfJ^\infty_Z$ admits a continuous split (this is the case if and only if the interior of $Z$ is dense in $Z$, 
 cf.~\cite{BieEWFSS}). In this situation, employing the split of $\rho_{|U}^*$ (see Theorem \ref{SchwaSFIACLG}) we can conclude that 
 $\rho^{\sharp}_{Z,{\rho(Z)}}$ is split surjective. It is not known to the authors under what condition on $Z$ the map 
 $\rho^{\sharp}_{Z,\rho(Z)}$ is actually split. Moreover, it is unclear if the image of 
 of $\rho^{\sharp}_{Z,{\rho(Z)}}$ is $\calE^\infty(Z)^\textup{inv}$ with weaker assumptions on $Z\subset V$.
\end{remark}


%\begin{corollary} \label{inviso} Under the assumptions of the theorem, 
%the continuous linear surjections 
%\begin{eqnarray*}
%\rho^{\sharp}_{Z,\rho(Z)}&:&\calE^\infty(\rho(Z))\to \calE^\infty(Z)^\textup{inv}\\
%\rho^{\sharp}_{G.Z,\rho(Z)}&:&\calE^\infty(\rho(Z))\to \calE^\infty(G.Z)^G
%\end{eqnarray*}
%induce isomorphisms of Fr{\'e}chet algebras
%\begin{eqnarray*}
%\xymatrix{&\calE^\infty(\rho(Z),X)\ar[dl]_\sim \ar[dr]^\sim &\\
%\calE^\infty(G.Z)^G\ar[rr]^\sim_{\id^{\sharp}_{Z,G.Z}}&&\calE^\infty(Z)^\textup{inv}.}
%\end{eqnarray*}
%\end{corollary}
%\begin{proof}[Proof of Corollary \ref{inviso}]
%Observe that the pullback $\id^{\sharp}_{Z,G.Z}$ of the identity $\id:V\to V$ associates to a Whitney function $F=(F_\alpha)_{\alpha\in \mathbb N^n}$ on $G.Z$ the Whitney function  $F_{|Z}:=({F_\alpha}_{|Z})_{\alpha\in \mathbb N^n}$ obtained by restriction to $Z$. Hence $\id^{\sharp}_{Z,G.Z}$ is injective. We also have  to show that the restriction of $\id^{\sharp}_{Z,G.Z}$ to $\calE^\infty(G.Z)^G$ is onto $\calE^\infty(Z)^\textup{inv}$. But in view of  
%\begin{eqnarray}\label{idrhorho}
%\id^{\sharp}_{Z,G.Z}\circ\rho^{\sharp}_{G.Z,\rho(Z)} =\rho^{\sharp}_{Z,\rho(Z)}\:,
%\end{eqnarray}
%$\id^{\sharp}_{Z,G.Z}$ has to be onto since $\rho^{\sharp}_{Z,\rho(Z)}$ is. So we have proven that the horizontal arrow in the diagram is an isomorphsm. That the other two arrows are isomorphisms follows from  $\Kern\left(\rho^{\sharp}_{Z,\rho(Z)}\right)=\sfJ^\infty_{\rho(Z)}(\calJ_X)=\Kern\left(\rho^{\sharp}_{G.Z,\rho(Z)}\right)$.
%\end{proof}

Before we turn to the proof of Theorem \ref{maindiag} we need a couple of auxiliary results.

\begin{proposition} \label{Prop:TopProp}
Let $G$, $V$ be as in  Theorem  \ref{maindiag},  $\rho : V \rightarrow \R^\ell$ a Hilbert map,
and $Z\subset V$ closed in some $G$-invariant open set $U\subset V$. Then the following holds true: 
\begin{enumerate}
\item \label{It:Closed} $G.Z$ is closed in $U$.
\item \label{It:QuotientTop} 
      $G.Z$ carries the quotient topology with respect to the restricted action 
      \[ \Phi_Z:G\times Z\to G.Z,\: (g,z) \mapsto g.z .\]
\item \label{ite:PropernessRestriction}
      For every open  $W \subset \R^\ell$ with $\rho (U) = \rho (V) \cap W $ the Hilbert map 
      $\rho : V \rightarrow \R^\ell$ restricts to a proper map 
      $\rho_{|U} : U \rightarrow W$.
\item \label{It:ExOpNbhd}
      There exists an open set $W \subset \R^\ell$ such that 
      $\rho (U) = \rho (V) \cap W $. If $\rho (U)$ is semialgebraic (resp.~subanalytic), $W$ can be chosen 
      to be semialgebraic (resp.~subanalytic) as well. 
      In both cases $\rho (U)\subset W$ is Nash subanalytic. 
\end{enumerate}
\end{proposition}

 \begin{proof} Claim (\ref{It:Closed}) follows immediately by compactness of $G$.
   %
   %Consider a sequence  of points 
   %$ g_k.z_k \in G.Z $, where $k\in \N$, $g_k \in G$, and $z_k\in Z$. Assume that
   %$(g_k.z_k)$ converges to some $u \in U$. Since $G$ is compact, one can assume after possibly 
   %passing to a subsequence, that $(g_k)$ converges to some $g \in G$. 
   %But then $(z_k)$ has to converge to $z:=g^{-1} u$, and this point is in $U$ by $G$-invariance of 
   %$U$. Since $Z$ is closed in $U$, one obtains $z\in Z$, hence $u = g.z \in G.Z$.
   %So $G.Z$ is closed in $U$.        
   %  
   To prove (\ref{It:QuotientTop}), let $A\subset G.Z$ be such that   $\Phi_Z^{-1}(A)$ is a 
   closed subset of $G\times Z$. We have to show that $A$ is closed in $G.Z$.
   To this end consider a sequence of points $ g_k.z_k \in A$, where $k\in \N$, $g_k \in G$, and $z_k\in Z$.
   Assume that $(g_k.z_k)$ converges to some $a\in G.Z$. By compactness of $G$ one concludes that after 
   possibly passing to a subsequence, $(g_k)$ converges to some $g\in G$. Then $(z_k)$ converges to a
   point $z:= g^{-1} a$. Since $a\in U$, and $U$ is $G$-invariant, one has $z\in U$. Hence $z\in Z$ since
   $Z$ is closed in $U$.  By assumption, $\Phi_Z^{-1}(A)$ is closed in $G\times Z$, and 
   $(g_k,z_k) \in\Phi_Z^{-1}(A)$ for all $k\in \N$. But then $(g,z) \in \Phi_Z^{-1}(A)$, hence 
   $a = g.z \in A$ which shows (\ref{It:QuotientTop}).
  
   For every open $W \subset \R^\ell$ such that $\rho (U) = \rho (V) \cap W $ properness of 
   the restricted map $\rho_{|U} : U \to W$ follows immediately from $\rho$ being proper. 
   This gives (\ref{ite:PropernessRestriction}).
   
  To prove (\ref{It:ExOpNbhd}), observe that $\rho$ is a closed map since 
  since $\R^n$ is locally compact and $\rho$ a proper map. Hence 
  $W := \R^\ell \setminus \rho (V\setminus U)$ is open, and contains $\rho (U)$ as a 
  subset since $ \varrho (V\setminus U)  =\rho (V) \setminus \rho (U)$ by $G$-invariance of $U$
  and by the fact that $\rho$ is a Hilbert map. 
  By construction, $W$ is semialgebraic (resp.~subanalytic), if $\rho (U)$ is. 
  Since $U$ is $G$-invariant and since $\rho$ factors through an injective map on the orbit space $V/G$ 
  the equality  $\rho (U) = W \cap \rho (V)$ holds true.
  Finally, $\rho (U) \subset W$ is Nash subanalytic in the sense of Def.~\ref{Def:NashSubanalytic}
  since $\rho$ is Gabrielov regular by Prop.~\ref{Prop:GabrielovRegularityPolynomials}
  and since  $\rho_{|U} : U \to W$ is proper. 
\end{proof}


\begin{proposition}
\label{BSrefinement}
Let $G$ be a compact Lie group, $M$ a smooth $G$-manifold and $Z\subset M$ a 
closed subset. Then 
\begin{eqnarray}\label{Hsaturation}
  \calJ^\infty(Z,M)^G=\calJ^\infty(G.Z,M)^G.
\end{eqnarray}
\end{proposition}

\begin{proof} In order to prove Eq.~(\ref{Hsaturation}), it suffices to prove the 
inclusion $\subset$ since the converse inclusion is trivial. 
For $g\in G$ we denote the $g$-action on $M$ by $\Phi_g : M \to M$, $x \mapsto g.x $. 
The idea of the proof is to take partial derivatives of 
\begin{equation}\label{Ginv}
  f\circ\Phi_g=f
\end{equation}
and evaluate the result at $z\in Z$. Using local coordinates $(x^1,\dots,x^m)$ 
around $z$ and $(y^1,\dots,y^m)$ around $g.z$, and taking first order partial 
derivatives we obtain
\[
  \sum_j\frac{\partial f}{\partial y^j}(g.z)\frac{\partial \Phi^j_g}{\partial x^i}(z)=
  \frac{\partial f}{\partial x^i}(z).
\]
We see that $\frac{\partial f}{\partial x^i}(z)=0$ is equivalent to 
$\frac{\partial f}{\partial y^j}(g.z)=0$, since the Jacobi matrix of $\Phi_g$ is invertible. 
Let us assume that $f$ is flat on $Z$ and continue inductively. By induction hypothesis 
\begin{equation}\label{lot}
  \frac{\partial^k f}{\partial y^{j_1}\dots \partial y^{j_k}}(g.z)=0 \mbox{ for }k\le n-1.
\end{equation}
Taking $n$-th partial derivatives of equation (\ref{Ginv}) we find (cf. Theorem \ref{FdBcomb})
\begin{eqnarray*}
  \sum_{j_1,\dots,j_n}\frac{\partial^n f}{\partial y^{j_1}\dots \partial y^{j_n}}(g.z)
  \frac{\partial \Phi^{j_1}_g}{\partial x^{i_1}}(z)\cdots
  \frac{\partial \Phi^{j_n}_g}{\partial x^{i_n}}(z)+\text{lower order terms}
  =\frac{\partial^n f}{\partial x^{i_1}\dots \partial x^{i_n}}(z).
\end{eqnarray*}
Here, by lower oder terms we mean terms containing  a factor of the form (\ref{lot}). 
It follows again from the invertibility of the Jacobian of $\Phi_g$, that 
$\frac{\partial^n f}{\partial y^{j_1}\dots \partial y^{j_n}}(g.z)=0$, which proves 
Eq.~(\ref{Hsaturation}).
\end{proof}

\begin{proposition} \label{flatalongZ} Under the assumptions of Theorem \ref{maindiag}
the image of the pullback 
$\rho^*:\calJ^\infty(\rho(Z),W)\to \calJ^\infty(Z,U)$ is actually $\calJ^\infty(Z,U)^G$.
\end{proposition}

\begin{proof} First of all, a function of the form $f\circ \rho\in \calC^\infty(U)$ is clearly 
invariant. Moreover, if $f$ is flat along $\rho(Z)$, then by the formula of Fa{\`a} di Bruno 
(cf. Theorem \ref{FdBcomb}), $f\circ \rho$ is flat along $Z$.

To prove the converse, recall that $X = \varrho (U)\subset W$ is Nash subanalytic by 
Prop.~\ref{Prop:TopProp} (\ref{It:ExOpNbhd}), which 
by the work of Bierstone and Milman  \cite[Thm.~3.2]{BieMilCDF} \& \cite[Thm.~1.13]{BieMilGDPSS} entails that 
the pair $\rho(Z)\subset X$ of subanalytic sets has the 
\emph{composite function property}.  That is, for any proper real-analytic map 
$\varphi:M\to  W$ from a real-analytic manifold $M$ such that 
$\varphi(M)=X$ one has
\begin{equation}\label{composite}
  \varphi^*(\calJ^\infty(\rho(Z), W))=
  \left( \varphi^*(\calJ^\infty(\rho (Z), W))\right)^{\wedge}.
\end{equation}

Let us explain what the right hand side of Eq.~(\ref{composite}) means. 
A smooth function $f$ on $M$ is \emph{formally a composite with} 
$\varphi$ if for any $b\in \varphi(M)$ there exists an  
$F_b\in \widehat{\mathcal O}_b$ 
such that for all $a\in \varphi^{-1}(b)$
\[
   \widehat{f}_a=F_b\circ \widehat{\varphi}_a,
\]
where $\widehat{f}_a$ and $\widehat{\varphi}_a$ are the formal Taylor expansions at 
$a$ of $f$ and $\varphi$, respectively. The set of formally composite functions 
with $\varphi$ is denoted by $\big(\varphi^*\calC^\infty(W)\big)^\wedge$. 
Setting $Y:=\rho(Z)$, the space $\big(\varphi^*\calJ^\infty(Z,\mathbb R^\ell)\big)^\wedge$ is 
defined as the intersection 
$\left(\varphi^*\mathcal C^\infty(W)\right)^\wedge 
 \cap\calJ^\infty(\varphi^{-1}(Y),M)$.

Recall from above that the restricted Hilbert map $\rho_{|U} : U \to W$ is proper. Hence we may 
specialize Eq.~(\ref{composite}) to the case $\varphi=\rho_{|U}$ and conclude, remembering $\rho^{-1}(\rho(Z))=G.Z$, that
\begin{eqnarray*} \label{lhs}
  \rho_{|U}^*(\calJ^\infty(\rho (Z), W ))&=&
  \big( \rho_{|U}^*(\calJ^\infty(\rho(Z),W ))\big)^{\wedge}\\
  &=&\left(\rho_{|U}^*\calC^\infty( W )\right)^{\wedge}\cap 
  \calJ^\infty\left(\rho^{-1}(\rho(Z)),U\right)\\
   &=&\ \rho_{|U}^*\calC^\infty( W )\cap 
  \calJ^\infty\left(\rho^{-1}(\rho(Z)),U\right)\\
   &=&\ \calC^\infty(U)^G \cap \calJ^\infty(G.Z,U) \ .
\end{eqnarray*}
 Using Theorem \ref{SchwaSFIACLG} on differentiable invariants we see 
that $\rho^*(\calJ^\infty(\rho (Z), W))$ coincides with $\calJ^\infty(G.Z,U)^G$, which is  
$\calJ^\infty(Z,U)^G$ by Proposition \ref{BSrefinement}.
\end{proof}

\begin{proposition} \label{surgrp} Assume that $G$, $V$ and $\rho$ are as in Theorem \ref{maindiag} 
  and that $Z$ is a closed subset of an open $G$-invariant subset $U\subset V$. 
  Then the following statements hold true.
\begin{enumerate} 
\item \label{surInvRels}
  Over the $G$-stable set $G.Z$, the relation
  $\calE^\infty(G.Z)^G = \calE^\infty(G.Z)^\textup{inv}$ holds true.
\item \label{surJgrp} 
 The Taylor morphism $\sfJ^\infty_{G.Z}: \calC^\infty(U)^G\to \calE^\infty(G.Z)^G$ is onto 
  with kernel $\calJ^\infty(G.Z,U)^G$. 
\item \label{surpbgrp} 
  The image of the pullback $\rho^{\sharp}_{G.Z,\rho(Z)}:\calE^\infty(\rho(Z))\to \calE^\infty(G.Z)$ 
  is $\calE^\infty(G.Z)^G$.
\item \label{imJgrpd} 
  The Taylor morphism $\sfJ^\infty_{Z}: \calC^\infty(U)^G\to \calE^\infty(Z)$ has image 
  in the space of invariant Whitney functions $\calE^\infty(Z)^\textup{inv}$.
\item \label{impbgrpd} 
  The image of the pullback $\rho^{\sharp}_{Z,\rho(Z)}:\calE^\infty(\rho(Z))\to \calE^\infty(Z)$ is 
  contained in  the space $\calE^\infty(Z)^\textup{inv}$. 
\end{enumerate}
\end{proposition}

\begin{proof} Take an $F\in\calE^\infty(G.Z)^\textup{inv}$. By Whitney's extension theorem 
there exists an $f\in \calC^\infty(U)$ such that $ \sfJ^\infty_{G.Z}(f)=F$.
Taking the average with respect to the Haar measure on $G$ and using the equivariance of 
the Taylor morphism, we obtain (cf.~Theorem \ref{Thm:PullbackJet})
\begin{eqnarray*}
\sfJ^\infty_{G.Z}(\operatorname{Average}(f))=\sfJ^\infty_{G.Z}\left(\frac{1}{\operatorname{vol}(G)}\int_G g^*(f) \,
\mathrm{d}g  \right)=\frac{1}{\operatorname{vol}(G)}\int_G g^{\sharp}(F)\, \mathrm{d}g =F.
\end{eqnarray*}
This proves (\ref{surJgrp}), and also (\ref{surInvRels}), since 
$\operatorname{Average}(f)$ is constant along orbits, which means that 
$\xi_U\operatorname{Average}(f)=0$ for all $\xi\in \frg$.

By the Theorem of Schwarz and Mather \ref{SchwaSFIACLG} we find an 
$h\in \calC^\infty(W)$ such that $\operatorname{Average}(f)=\rho_{|U}^*(h)$. By Theorem \ref{Thm:PullbackJet} we have
\[F=(\sfJ^\infty_{G.Z}\circ \rho_{|U}^*)(h)=(\rho_{|U}^{\sharp}\circ\sfJ^\infty_{\rho(Z)})(h), \]
which proves (\ref{surpbgrp}).

To prove \ref{imJgrpd}, fix $f\in  \calC^\infty(U)^G$, and let $(g,z) \in (G\ltimes V)_{|Z}$. By $G$-invariance of 
$f$, one concludes that 
\[
  (\Phi_g)^{\sharp}_{\{z\},\{g.z\}}\big( \sfJ^\infty_{g.z} (f) \big) = \big (\sfJ^\infty_{z} \circ \Phi_g^* \big) (f) =
  \sfJ^\infty_{z} (f) ,
\]
hence  $\sfJ^\infty_{Z} (f)$ is invariant under the groupoid action $(G\ltimes V)_{|Z}$ . 
In addition, $\sfJ^\infty_{Z} (f)$ is invariant under the $\frg$-action, since 
$f$ is constant along orbits, hence $\xi_U f =0$ for all $\xi\in \frg$.
One now shows (\ref{impbgrpd})  analogously to (\ref{surpbgrp}), and the proposition is proved.
\end{proof}

Note, that if $Z$ is not $G$-saturated the Taylor morphism $\sfJ^\infty_{Z}$ is not $G$-equivariant, and we 
can not produce groupoid invariant Whitney functions by averaging over $G$. 
We will use Theorem 3.6 from Bierstone and Milman's article \cite{BieMilCDF} to prove 
the surjectivity for the groupoid case. The following result is a crucial tool  for this argument.

\begin{proposition} \label{idiso} 
Let $G$, $V$ be as above, and $Z\subset V$ a locally closed subset. Then the  map 
$\id^{\sharp}_{Z,G.Z}: \sfJ^\infty(G.Z)^G\to \sfJ^\infty(Z)^{(G\ltimes V)_{|Z}}$ 
is an isomorphism of Fr{\'e}chet algebras.
\end{proposition}

\begin{proof} Let us write for the action of $g\in G$: $v\mapsto g.v=\Phi_g(v)$. Let us also use the short 
hand notation $\Gamma_Z:= (G\ltimes V)_{|Z}$.

Our aim is to write down an inverse map for the restriction of $\id^{\sharp}_{Z,G.Z}$ to the space of invariant 
jets on $G.Z$. So let us pick a $\Gamma_Z$-invariant jet $F=(F_\alpha)_{\alpha\in \mathbb N^n}$ on $Z$, a point 
$v=g.z\in G.Z$, and declare 
$\widetilde F (v) = \big( \widetilde F_{\alpha} (v)\big)_{\alpha\in \mathbb N^n}  \in \sfJ^\infty(\{v\})$ 
using the pullback with respect to the diffeomorphism $\Phi_{g^{-1}}$:
\begin{eqnarray}\label{invextension}
\widetilde F(v) :=\Phi^{\sharp}_{g^{-1}} \big(  F ( z) \big) = (\Phi^{\sharp}_{g^{-1}} F)(g.z) \: .
\end{eqnarray}
Let us prove that this definition does not depend on the choice of $z$ and $g$.
To this end, assume that $g.z=g'.z'$ for some $g'\in G$ and $z'\in Z$. Setting $h:=(g')^{-1}g$ we have that $h.z=z'$, that is, $(h,z)\in\Gamma_Z$. With these preparations we are ready to calculate
\begin{eqnarray*}
\widetilde F(g'.z')&=&(\Phi^{\sharp}_{g'^{-1}} F)(g'.z')=(\Phi^{\sharp}_{hg^{-1}} F)(g.z)\\
&=&\left(\Phi^{\sharp}_{g^{-1}}(\Phi^{\sharp}_{h}F)\right)(g.z)\stackrel{*}{=}(\Phi^{\sharp}_{g^{-1}}F)(g.z)=\widetilde F(g.z),
\end{eqnarray*}
where we have used at``$*$'' that, by assumption, $F$ is $\Gamma_Z$-invariant.
As in the defining equation (\ref{invextension}) $\widetilde F (g.z)$ clearly depends continuously on 
$z$ and $g$, and $G\times Z \rightarrow G.Z$ is a topological identification by (\ref{It:QuotientTop}) in 
Prop.~\ref{Prop:TopProp},
$\widetilde F=(\widetilde F_{\alpha})_{\alpha\in \mathbb N^n}$  actually defines a jet in $\sfJ^\infty(G.Z)$.

It remains to check that $\widetilde F=(\widetilde F_{\alpha})_{\alpha\in \mathbb N^n}$ is $G$-invariant. 
To this end let $h \in G$, and $v =g.z\in G.Z$. Then 
\begin{eqnarray*}
\Phi^{\sharp}_{h} \big( \widetilde F(h.v) \big) &=& \Phi^{\sharp}_{h} \left( \Phi^{\sharp}_{(hg)^{-1}} \big( F (z)\big)\right) 
= \Phi^{\sharp}_{h} \left( \Phi^{\sharp}_{h^{-1}}\Phi^{\sharp}_{g^{-1}} \big( F (z) \big) \right) \\
&=& \Phi^{\sharp}_{g^{-1}} \big( F (z) \big)=  \widetilde F(v) ,
\end{eqnarray*}
which means that $ \widetilde F$ is $G$-invariant. 

Finally, we have to convince ourselves that the linear map 
\begin{eqnarray}
\qquad\sfE_Z : \sfJ^\infty(Z)^{(G \ltimes V)_{|Z}}\to \sfJ^\infty(G.Z)^G,\quad 
  F=(F_\alpha)_{\alpha\in \mathbb N^n}\mapsto\widetilde F=(\widetilde F_{\alpha})_{\alpha\in \mathbb N^n}
\end{eqnarray} 
is actually a  continuous right inverse to the injective map 
$\id^{\sharp}_{Z,G.Z}: \sfJ^\infty(G.Z)^G\to \sfJ^\infty(Z)^{(G \ltimes V)_{|Z}}$, but this is clear by construction.
The claim follows.
\end{proof} 
We now come to the main result which will be needed for proving Theorem \ref{maindiag}. 
\begin{proposition} \label{surgrpd} 
Under the assumptions of Theorem \ref{maindiag}  the following statements hold true.
\begin{enumerate}
\item \label{surJgrpd} The Taylor morphism $\sfJ^\infty_{Z}: \calC^\infty(U)^G\to 
\calE^\infty(Z)^\textup{inv}$ is onto with kernel 
\[
 \calJ^\infty(Z,U)^G=\calJ^\infty(G.Z,U)^G .
\]
\item \label{surpbgrpd} The image of the pullback 
$\rho^{\sharp}_{Z,\rho(Z)}:\calE^\infty(\rho(Z))\to \calE^\infty(Z)$ 
is actually the space of invariant Whitney functions $\calE^\infty(Z)^\textup{inv}$.
\end{enumerate}
\end{proposition}

\begin{proof} 
Consider the isomorphism 
$\id^{\sharp}_{Z,G.Z}:\sfJ^\infty (G.Z)^G \rightarrow \sfJ^\infty (Z)^{(G \ltimes V)_{|Z}}$ 
from Proposition \ref{idiso}, and its inverse $\sfE_Z$. Assume that we can show that                  
$\sfE_Z$ maps the invariant space $\calE^\infty (Z)^\textup{inv}$ to $\calE^\infty (G.Z)^G$. 
Since the image of 
$\calE^\infty (G.Z)^G$ under $\id^{\sharp}_{Z,G.Z}$ obviously lies in $\calE^\infty (Z)^\textup{inv}$
it then follows that $\id^{\sharp}_{Z,G.Z}$ restricts to an isomorphism 
$\iota_{Z} : \calE^\infty (G.Z)^G \rightarrow \calE^\infty (Z)^\textup{inv}$. 
In view of the commutativity of the diagram
\begin{eqnarray*}
 \xymatrix{\calC^\infty(U)^G\ar[r]^{\sfJ^\infty_Z\quad}\ar[dr]_{\sfJ^\infty_{G.Z}}&\calE^\infty(Z)^\textup{inv}\\
&\calE^\infty(G.Z)^G\:,\ar[u]_{\iota_{Z}}}
\end{eqnarray*}
statement (\ref{surJgrpd}) then is a consequence of Proposition \ref{surgrp}(\ref{surJgrp}).
Similarly, statement (\ref{surpbgrpd}) follows from the commutativity of the diagram
\begin{eqnarray*}
 \xymatrix{\calE^\infty(\rho(Z)) \ar[r]^{\rho^{\sharp}_{Z,\rho (Z)}}\ar[dr]_{\rho^{\sharp}_{G.Z,\rho (Z)}}&\calE^\infty(Z)^\textup{inv}\\
&\calE^\infty(G.Z)^G\ar[u]_{\iota_{Z}}}
\end{eqnarray*}
together with
Proposition \ref{surgrp}(\ref{surpbgrp}), if $\iota_Z$ is an isomorphism.

To prove the proposition, we therefore only need to show that $\sfE_Z$ maps the invariant space 
$\calE^\infty (Z)^\textup{inv}$ to $\calE^\infty (G.Z)^G$ or in other words that $\iota_Z$ is onto.
We will prove the claim by induction on the depth of the stratification 
of $Z$. Recall that by assumption, $Z$ is stratified by orbit types. Let us divide the argument in 
several steps.
% Since for each open covering of $U$ consisting of $G$-invariant open sets there exists a 
% $G$-invariant subordinate partition of unity by smooth functions the claim is shown if one can prove that 
% for each point $z\in Z$ there exists an open $G$-invariant neighborhood $O$ 
% such that 
% \[
% \iota_{Z\cap O}: \calE^\infty(G.Z\cap O)^G\to
% \calE^\infty(Z\cap O)^\textup{inv} 
% \] 
% is surjective. Since $\iota_{Z\cap O}$ obviously is injective,  $\iota_{Z\cap O}$ has to be an isomorphism then. 



{Step 1.} Denote by  $d$ the depth of the stratified space $Z$, and assume that for 
every subanalytic set $Y\subset V$ of depth less than $d$ such that $Y$ is closed in some $G$-invariant 
open subset of $V$ the map 
$\iota_{Y} : \calE^\infty (G.Y)^G \rightarrow \calE^\infty (Y)^\textup{inv}$ is an isomorphism. 
Let $S \subset Z$ be the stratum of $Z$ of highest depth. Then $S$, and hence $G.S$ are 
closed in $U$. Assume that we can show
\begin{enumerate}[({Claim} A)]
\item For every $z\in S$ there is an  $G$-invariant open semialgebraic neighborhood $O\subset U$ 
      such that $\sfE_{S\cap O}$ maps $\calE^\infty (S \cap O)^\textup{inv}$ to $\calE^\infty (G.S \cap O)^G$.
\end{enumerate}
Then, using a $G$-invariant locally finite partition of unity subordinate to the neighborhoods $O$ 
from Claim A, one derives that for every $F \in \calE^\infty (Z)^\textup{inv}$ the jet 
$\sfE_{S} \big(\operatorname{res}^Z_S F \big)$  is an invariant Whitney function on $G.S$, i.e.~an element of
$\calE^\infty (G.S)^G$. Now choose a $G$-invariant smooth function $f: U \rightarrow \R$ such that
$\sfJ^\infty_{G.S} (f) = \sfE_{S} \big(\operatorname{res}^Z_S F \big)$. Then $\widetilde F := F - \sfJ^\infty_{Z} (f)$ is a Whitney function on 
$Z$ such that $\widetilde F (z) =0$ for all $z\in S$. 
Now consider 
\begin{enumerate}[({Claim} B)]
\item Let $ E \in \calE^\infty (Z)^\textup{inv}$ such that $E (z) =0$ for all $z\in S$
  and such that $\sfE_{Z\setminus S} (\operatorname{res}^Z_{Z\setminus S} E) \in \calE^\infty (G. Z\setminus G.S )^G$. 
  Then $\sfE_Z (E)$ is an invariant Whitney function on $G.Z$. 
\end{enumerate}
Under the assumption that Claim B holds true, it then follows that $\sfE_Z (\widetilde F)$ is invariant, hence
\[
   \sfE_Z ( F )  = \sfE_Z (\widetilde F) + \sfJ^\infty_{G.S} (f)  \in \calE^\infty (G.Z)^G \: .
\]
But this finishes the inductive step, hence the proposition follows.  
It therefore remains to show Claim A and Claim B. 
In Step 2, we provide an auxiliary result which in Step 3  will be used to derive Claim A.
Claim B will be shown in Step 4.  



{Step 2.} Fix a point $z$ in the stratum $S$ of $Z$ of highest depth. 
Denote by $\calO$ the $G$-orbit through $z$. Then choose $\varepsilon >0 $ such that 
the tubular neighborhood
\[
  T_\calO:= \{ v \in V \mid d(v,\calO) < \varepsilon \} 
\]
is contained in $U$ and such that the exponential map 
\[ \exp:TV \cong V\times V\rightarrow V, \: (v,w) \mapsto v+w \]
maps a $G$-invariant neighborhood $\widetilde O$ of the zero section of the normal bundle 
$\pi : N \rightarrow \calO$ equivariantly  onto $T_\calO$. 
%In the following, we identify $T_\calO$ with $O$ under the exponential map. 
 
Next, observe that the $G$-action on the normal bundle $N$ induces an action of the 
transformation groupoid 
$G\ltimes \calO$  on $N$ (see \cite[Sec.~3]{PflPosTanGOSPLG}). This means, that for every pair 
$(h,v) \in G \times \calO$ there is a (linear and isometric) isomorphism 
$\Psi_{(h,v)} : N_v \rightarrow N_{h.v}$ such that $\Psi_{(h',hv)} \Psi_{(h,v)}= \Psi_{(h'h,v)}$ for all
$h',h\in G$ and $v\in \calO$. In particular, if $hz = h'z$, then $ h'^{-1}h \in G_z$, and  
$\Psi_{h'^{-1}h,z} = \Psi_{(h'^{-1},hz)} \Psi_{(h,z)} $ is the standard action of  $ h'^{-1}h$ on $N_z$.
In other words, this means that $\Psi$ extends the natural action of the isotropy group $G_z$
on the normal space $N_z := T_z\calO^\perp$.  

After these preliminaries we now want to construct a smooth map $\lambda :N \rightarrow \R^k$
for some $k\in \N$ such that the following properties hold true:
\begin{enumerate}[(i)]
%\item \label{Ite:FirProj}
%      The composition $\operatorname{pr}_1 \circ \lambda$ coincides with the canonical projection 
%      $\pi:N\rightarrow \calO$.
\item \label{Ite:SecProj}
      The map $ \lambda$ is $G$-invariant.
\item \label{Ite:GabProp}
      The map $\lambda$ is Gabrielov regular at each point of its domain 
      (see Appendix \ref{App:Gabrielov} or \cite{PawGRCAM} for Gabrielov regularity).
\item \label{Ite:HilMap} For every $v\in \calO$, the
  restriction $\lambda_v := \lambda_{|N_v} : N_v\rightarrow \R^k$ 
  is a minimal Hilbert map for the $G_v$-representation space $N_v$.
\item \label{Ite:FormComp}
      For every $v\in \calO$ and $n\in N_v^{G_v}$, a formal power series 
      $F\in \sfJ^\infty (\{n\})$ is invariant if and only if it is a composition with 
      $\lambda^{\sharp}_{\{n\},\{\lambda(n)\}} $.  
\end{enumerate}

To define $\lambda$, choose a minimal complete set of $G_z$-invariant polynomials
$\lambda_{1},\dots,\lambda_{k}$ on $N_z$, and denote by $\lambda_z : N_z \rightarrow \R^k$ 
the corresponding Hilbert map. 
%To define $\lambda$, choose a minimal complete set of $G_z$-invariant polynomials
%$\lambda_{1},\cdots,\lambda_{k}$ on $N_z$ of the following form. 
%The first $d$ components $\lambda_{1},\cdots,\lambda_{d}$ are (linear) coordinates of the 
%fixed point space $N_z^{G_z}$, the remaining components $\lambda_{d+1},\cdots,\lambda_{k}$
%are the compositions of the components of a minimal Hilbert basis of the orthogonal complement 
%$\big( N_z^{G_z}\big)^\perp$ with the orthogonal projection of $N_z$ onto $\big( N_z^{G_z}\big)^\perp$
%We denote the resulting Hilbert map $(\lambda_{1},\cdots,\lambda_{k}): N_z \rightarrow \R^k$ 
%briefly by $\lambda_z$. 
Obviously, $\lambda_z$ is polynomial, hence according to 
Proposition \ref{Prop:GabrielovRegularityPolynomials}
%\cite[Sec.~1, Examples]{PawGRCAM}, 
$\lambda_z$ is regular in the sense of Gabrielov at each 
point of $N_z$.
Next we construct  $\lambda_v :N_v \rightarrow \R^k$ for arbitrary $v\in \calO$. 
To this end, choose an $h\in G$ such that $hv =z$. Then put
$\lambda_v (n) = \lambda_z \circ \Psi_{(h,v)} (n)$ for $n\in N_v$.
Now we can define:
\[
 \lambda : N \rightarrow \R^k, \quad n \mapsto  \lambda_{\pi (n)} (n) .  
\]
We have to show that each $\lambda_v$ does not depend on the particular choice of $h$,
and that $\lambda$ is  smooth. So let $h'\in G$ be another group element with 
$h'v =z$. Then 
\[
 \lambda_z \circ \Psi_{(h',v)} = \lambda_z \circ \Psi_{(h'h^{-1},z)} \circ \Psi_{(h,v)} =
  \lambda_z \circ \Psi_{(h,v)} ,
\]
since $h'h^{-1}\in G_z$, and $\lambda_z$ is $G_z$-invariant. Therefore $\lambda_v$
does not depend on the particular choice of $h\in G$ with $hv=z$. 
Next observe that for such $v$ and $h$ one can find  an open neighborhood $Q\subset \calO$ of $v$ and
an embedding $\sigma : Q \rightarrow G$ such that $\sigma (v) =h$ and $\sigma (w)w =z$ for all $w\in Q$.
Given such a map $\sigma$, denote by $\Sigma :N_{|Q} \rightarrow N_z$ the submersion  
$n \mapsto \Psi_{(\sigma \pi (n),\pi(n))} (n)$. Then, the restriction of $\lambda$ to 
$N_{|Q}$  is the composition $\lambda_z \circ \Sigma$, which shows that $\lambda$ is smooth.

By construction, properties  (\ref{Ite:SecProj}) and (\ref{Ite:HilMap}) are obvious. 
Moreover, $\lambda$ is Gabrielov regular by the following argument. It suffices to verify that 
$\lambda$ is Gabrielov regular on the restricted bundle $N_{|Q}$, where $Q$ is a neighborhood 
of $v$ as before. Since $\lambda_z$ is Gabrielov regular, and $\Sigma : N_{|Q} \rightarrow N_z$ 
a submersion, the composition $\lambda_z \circ \Sigma$ has to be Gabrielov regular as well. 
But over $N_{|Q}$, $\lambda$ coincides with $\lambda_z \circ \Sigma$, which proves 
(\ref{Ite:GabProp}).

It remains to check (\ref{Ite:FormComp}). So let $n\in N_v^{G_v}$ for some $v\in \calO$. 
Then $\tau : N_v \rightarrow N_v$, $\tilde n \mapsto \tilde n  + n$ is a $G_v$-equivariant
affine isomorphism.
Consider a formal power series $F\in \sfJ^\infty (\{n\})^\textup{inv} $, where $n\in N_v^{G_v}$. 
Let $i: N_v \rightarrow N$ be the canonical embedding. Note that $i$ is also $G_v$-equivariant. 
Hence $\tilde F := \tau^{\sharp}_{\{0\},\{n\}} i^{\sharp}_{\{n\},\{n\}} F $ is a $G_v$-invariant formal power 
series at the origin of $N_v$. Thus, by the Theorem of Schwarz and Mather \ref{SchwaSFIACLG}, there exists a
smooth $\tilde f\in \calC^\infty (\R^k)$ such that 
$\tilde F = \sfJ^\infty_{\{ 0\}} \big( \tilde f\circ \lambda_v\big)$. 
Let us show that
$F = \sfJ^\infty_{\{ n \}} \big( f \circ \lambda \big) $, where 
$f (u ) = \tilde f ( u -\lambda_v (n))$, $u\in \R^k$.  Since the submanifolds $N_v \subset N$ 
and $G.\{ n \} \subset N$ are transversal at $n$, it suffices to show that 
\begin{eqnarray}
  \label{eq:first}
   i^{\sharp}_{\{n\},\{n\}} F  & = & i^{\sharp}_{\{n\},\{n\}} \sfJ^\infty_{\{ n \}} \big( f \circ \lambda \big) \: 
   \text{ and } \\
   \label{eq:second}
   \xi_N  \Big( \sfJ^\infty_{\{ n \}} \big( f \circ \lambda \big)\Big) & = & 0 \:
  \text{ for all $\xi \in \frg$}. 
\end{eqnarray}
To verify these equations observe  that $\lambda_v^* f = (\lambda_v^*\tilde f )\circ \tau^{-1}$, hence 
\[
\begin{split}
   i^{\sharp}_{\{n\},\{n\}} F & = \big(\tau^{-1}\big)^{\sharp}_{\{ n \},\{0\}} \tilde F = 
   \big(\tau^{-1}\big)^{\sharp}_{\{ n \},\{0\}} \sfJ^\infty_{\{ 0\}} \big( \tilde f\circ \lambda_v\big) = \\
    & = \sfJ^\infty_{\{ n \}} \big( \tilde f\circ \lambda_v\circ \tau^{-1}\big) =
   \sfJ^\infty_{\{ n \}} \big( f \circ \lambda_v \big) =
   i^{\sharp}_{\{n\},\{n\}} \sfJ^\infty_{\{ n \}} \big( f \circ \lambda \big) . 
\end{split}
\]
This proves Eq.~(\ref{eq:first}). 
Since by construction of $\lambda$, the composition $f\circ \lambda$ is constant on each orbit 
of $G$, Eq.~(\ref{eq:second}) follows, too. 
So we now get 
\[ 
 F = \sfJ^\infty_{\{ n \}} \big( f \circ \lambda \big) 
   =  \lambda^{\sharp}_{\{n\},\{\lambda(n)\}} \sfJ^\infty_{\{ \lambda(n) \}} ( f )
\]
which proves one direction of (\ref{Ite:FormComp}). The other direction is immediate. 

{Step 3.} Let the stratum $S \subset Z$, the point $z \in S$ and the tubular neighborhood $T_\calO$ 
around the orbit $\calO$ through $z$ as in Step 2. Put $O:=T_\calO$. 
In this step, we want to show that
\[
 \iota_{S\cap O}: \calE^\infty(G.S\cap O)^G\to \calE^\infty(S\cap O)^\textup{inv} 
\] 
is onto. To this end, we first transform the claim to an equivalent statement about 
Whitney functions on certain subsets of the normal bundle $N$. 
Observe that $\calO\subset V$ is a Nash submanifold by \cite[A.~10]{BryIKGNO}
and that $O = T_\calO$ is a Nash tubular neighborhood 
of $\calO$ by \cite[Cor.~8.9.5]{BocCosRoyRAG}, so in particular semialgebraic. 
Morever, the restriction 
$\exp_{|\widetilde O}$ of the exponential map to $\widetilde O = \exp^{-1} (T_\calO) \cap N$ 
is a Nash diffeomorphism from $O$ to  $\widetilde O$. 
This implies in particular that $\widetilde S = \exp^{-1} (S \cap O) $ is a closed
analytic submanifold of $\widetilde O$. Moreover, it entails that the claimed surjectivity of 
$\iota_{S\cap O}$ is equivalent to 
\[
  \iota_{\widetilde S} := \big( \id^{\sharp}_{\widetilde S, G.\widetilde{S}}\big)_{|\calE^\infty(G.\widetilde{S})^G} : 
  \calE^\infty(G.\widetilde{S})^G\to \calE^\infty(\widetilde S)^\textup{inv} 
\]
being onto. So let us prove this. Choose an open subset $\Omega \subset \R^k$ such that 
$\lambda (\widetilde S)$ is closed in $\Omega$; this is possible by Prop.~\ref{Prop:TopProp} (\ref{It:ExOpNbhd}). 
The restriction $\lambda_{|\widetilde O} :\widetilde O \rightarrow \Omega$ now is a proper analytic map.  
Next consider the morphism 
\begin{equation}
\label{eq:lambdasharp}
  \lambda_{\widetilde S, \lambda(\widetilde S)}^{\sharp} : \calE^\infty ( \lambda(\widetilde S) ) \rightarrow 
  \calE^\infty (\widetilde S).
\end{equation}
We claim that the image of this map is $\calE^\infty (\widetilde S)^\textup{inv}$.
Since $\lambda_{|\widetilde O}$ is proper and Gabrielov regular at each point of $\widetilde O$ by Step 2, 
we can apply Theorem 3.6  by Bierstone--Milman \cite{BieMilCDF}. Thus, the following equality holds true:
\begin{equation}
  \label{Eq:ImEquiv}
  \lambda_{\widetilde S, \lambda(\widetilde S)}^{\sharp} \calE^\infty (\lambda(\widetilde S)) =
  \overline{\lambda_{\widetilde S, \lambda(\widetilde S)}^{\sharp} \calE^\infty (\lambda(\widetilde S))}=
  \big( \lambda_{\widetilde S, \lambda(\widetilde S)}^{\sharp} \calE^\infty (\lambda(\widetilde S))\big)^\wedge ,
\end{equation}
where the last term denotes the algebra of Whitney functions on $\widetilde S$ which
are a formal composite with $\lambda$. 
By  (\ref{Ite:FormComp}) in Step 2 one concludes that $F \in  \calE^\infty (\widetilde S)$ is  a formal composition with $\lambda$
if and only if $F$ is an element of $\calE^\infty (\widetilde S)^\textup{inv}$. Hence the 
image of the map  (\ref{eq:lambdasharp}) is $\calE^\infty (\widetilde S)^\textup{inv}$ indeed. 
Now consider the commutative diagram:  
 \begin{eqnarray*}
\xymatrix{
 \calE^\infty (\lambda (\widetilde S )) 
 \ar[rr]^{\mbox{ }\quad \lambda_{G.\widetilde{S},\lambda (G. \widetilde{S} )}^{\sharp}}
 \ar[drr]_{\mbox{ }\quad \lambda_{\widetilde{S},\lambda ( \widetilde{S} )}^{\sharp}}&  &
 \calE^\infty(G.\widetilde{S})^G \ar[d]^{\iota^{\sharp}_{\widetilde{S}}}
 \\ & & \calE^\infty(\widetilde{S})^\textup{inv}
}
\end{eqnarray*} 
Since by the above 
$\lambda_{\widetilde{S},\lambda ( \widetilde{S} )}^{\sharp} (\calE^\infty (\lambda (\widetilde{S})))=
\calE^\infty(\widetilde{S})^\textup{inv}$, it follows that the vertical arrow in the 
diagram is surjective. This finishes Step 3.

{Step 4.} Again let $S \subset Z$ be  a stratum of highest depth.  
Assume that $ E \in \calE^\infty (Z)^\textup{inv}$ is an invariant Whitney function
such that $E (z) =0$ for all $z\in  S$ and that  
$\sfE_{ Z \setminus S} (\operatorname{res}^{Z}_{Z \setminus S} E) \in 
  \calE^\infty (G. Z \setminus G. S )^G$. 
By the proof of Prop.~\ref{idiso}  one knows that  $\sfE_{Z}(E) \in \sfJ^\infty (G.Z)$. 
Moreover, 
\[ \operatorname{res}^{G.Z}_{G.S} \sfE_{Z}(E) = \sfE_{S} (\operatorname{res}^{Z}_{S} E) = 0 \]
and 
\[ \operatorname{res}^{G.Z}_{G.Z \setminus G.S} \sfE_{Z}(E) = \sfE_{Z \setminus S} (\operatorname{res}^{Z}_{Z\setminus S} E)  
   \in \calE^\infty (G. Z \setminus G. S )^G \ . \] 
By Hestenes's lemma for subanalytic sets \ref{Lem:subanalytic-hestenes-lemma} one obtains
$\sfE_{Z}(E) \in  \calE^\infty (G. Z )^G$ and Claim B is proved.
\end{proof}

\begin{proof}[Proof of Theorem \ref{maindiag}] We already know that all maps in the diagram of Theorem \ref{maindiag} are well-defined and that the diagram is in fact commutative. The middle column  is exact by the Theorem of Schwarz (cf.~Theorem \ref{SchwaSFIACLG}), the first column is exact by Proposition \ref{flatalongZ}, the middle row is exact by the Whitney Extension Theorem (cf.~Theorem \ref{WhitneyExtension}) and the first row is exact by Corollary \ref{surgrpd}. The third row is clearly exact. 

So it remains to show that the third column is exact. This can be done by an elementary diagram chase as follows. We have to show that \[\Kern\left(\rho^{\sharp}_{Z,\rho(Z)}\right)=\mathsf J^\infty_{\rho(Z)}(\mathcal J_X).\]  
To this end let us assume that $\rho^{\sharp}_{Z,\rho(Z)}(F)=0$. By Whitney's Extension Theorem there exists a function $f\in \mathcal C^\infty(\mathbb R^\ell)$ such that $\mathsf J^\infty_{\rho(Z)}(f)=F$. Moreover, since the upper right square in our diagram commmutes, $\rho^*(f)$ has to be flat along $Z$. Since the upper left arrow $\rho^*$ is onto, there exists a function $g\in \calC^\infty(\mathbb R^\ell)$ that is flat along $\rho(Z)$ such that $\rho^*(f-g)=0$. Clearly $\mathsf J^\infty_{\rho(Z)}(f-g)=F$ and $f-g\in \calJ_X$, which proves our claim.
\end{proof}

\begin{corollary} \label{corollary} 
Under the assumptions of Theorem \ref{maindiag}, the continuous linear surjections 
\begin{eqnarray*}
\rho^\sharp_{Z,\rho(Z)}&:&\calE^\infty(\rho(Z))\to \calE^\infty(Z)^\textup{inv}\\
\rho^\sharp_{G.Z,\rho(Z)}&:&\calE^\infty(\rho(Z))\to \calE^\infty(G.Z)^G
\end{eqnarray*}
induce isomorphisms of Fr{\'e}chet algebras
\begin{eqnarray*}
\xymatrix{&\calE^\infty(\rho(Z),X)\ar[dl] \ar[dr] &\\
\calE^\infty(G.Z)^G\ar[rr]_{\id^\sharp_{Z,G.Z}}&&\calE^\infty(Z)^\textup{inv}.}
\end{eqnarray*}
\end{corollary}
\begin{proof}[Proof of Corollary \ref{corollary}]
Observe that the pullback $\id^\sharp_{Z,G.Z}$ of the identity $\id:V\to V$ associates to a Whitney function $F=(F_\alpha)_{\alpha\in \mathbb N^n}$ on $G.Z$ the Whitney function  $F_{|Z}:=({F_\alpha}_{|Z})_{\alpha\in \mathbb N^n}$ obtained by restriction to $Z$. Hence $\id^\sharp_{Z,G.Z}$ is injective. We also have  to show that the restriction of $\id^\sharp_{Z,G.Z}$ to $\calE^\infty(G.Z)^G$ is onto $\calE^\infty(Z)^\textup{inv}$. But in view of  
\begin{eqnarray}\label{idrhorho}
\id^\sharp_{Z,G.Z}\circ\rho^\sharp_{G.Z,\rho(Z)} =\rho^\sharp_{Z,\rho(Z)}\:,
\end{eqnarray}
$\id^\sharp_{Z,G.Z}$ has to be onto since $\rho^\sharp_{Z,\rho(Z)}$ is. So we have proven that the horizontal arrow in the diagram is an isomorphsm. That the other two arrows are isomorphisms follows from  $\Kern\left(\rho^\sharp_{Z,\rho(Z)}\right)=\sfJ^\infty_{\rho(Z)}(\calJ_X)=\Kern\left(\rho^\sharp_{G.Z,\rho(Z)}\right)$.
\end{proof}

%%% Local Variables:
%%% mode: latex
%%% TeX-master: "HerPfl.InvariantWhitneyFunctions.tex"
%%% End: