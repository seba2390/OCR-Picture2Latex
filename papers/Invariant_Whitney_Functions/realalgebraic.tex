

\subsection{Gabrielov regularity}
\label{App:Gabrielov}

Let $\varphi: X\to Y$ be a real analytic map between real analytic manifolds
$X$ and $Y$ and let $x\in X$. Denote by $\mathcal O_{X,x}$ the algebra of 
germs of real analytic functions on $X$  at $x$ and by
$ \sfJ^\infty_{X} (\{x\})$ the algebra of formal power series 
or in other words the algebra of infinite jets on $X$ at $x$. 
Using the notation from Section \ref{sec:whitney-functions},
$\sfJ^\infty (\{x\}) = \mathcal{E}^\infty (\{ x \},X) = \calC^\infty (X)/ \calJ^\infty (\{ x \} , X) $.
Similarly, we use $\mathcal O_{Y,y}$ and 
$\sfJ^\infty (\{ y \} )$ for the corresponding algebras at 
$y\in Y$. The map $\varphi$ induces algebra morphisms
$\varphi_x^*: \mathcal O_{Y,\varphi(x)}\to \mathcal O_{X,x}$ and
$\varphi_{x,\varphi(x)}^{\sharp}:  \sfJ^\infty (\{ \varphi(x) \} )\to  \sfJ^\infty (\{x\})$,
where the latter is  given in local coordinates by Taylor expansion as in Eq.~\eqref{Eq:DefPullbackJet}.
%
Gabrielov \cite{GabFRAF} uses the following three notions of rank of $\varphi$ at $x$:
\begin{itemize}
\item $r_1(x)=$ rank of the Jacobian of $\varphi$ at a generic point nearby $x$,
\item $r_2(x)= \dim \big( \sfJ^\infty (\{ \varphi(x) \} ) / \ker \varphi_{x,\varphi(x)}^{\sharp} \big)$,
\item $r_3(x)=\dim \big( \mathcal O_{Y,\varphi(x)}/\ker \varphi_x^* \big)$.
\end{itemize}
One observes $r_1(x)\le r_2(x)\le r_3(x)$. The map $\varphi$ is said to be 
\emph{regular in the sense of Gabrielov at} $x$ if $r_1(x)=r_3(x)$, i.e., if the three ranks 
coincide. It is said to be \emph{regular in the sense of Gabrielov} if it has this property 
at all $x\in X$.

\begin{proposition}
\label{Prop:GabrielovRegularityPolynomials}
  Let  $\varphi: X= \R^n \to Y =\R^m$ be a polynomial mapping. Then $\varphi$ is Gabrielov regular. 
\end{proposition}

\begin{proof}
  The proof follows Bierstone \cite{BiePC}.
  By the Tarski-Seidenberg Theorem, the image of $\varphi$ is a semialgebraic subset of $Y=\R^m$.
  By  a result of {\L}ojasewicz \cite{LojESA}, \cite[Thm.~2.13]{BieMilSSS}, $\operatorname{im} \varphi$ is contained 
  in an algebraic set of the same dimension. The ranks of Gabrielov therefore coincide at every point 
  of the domain. 
\end{proof}

A class of subanalytic sets having the so-called ''composite function property'' \cite{BieMilCDF,BieMilPawCDF} 
is the one defined in the following. 

\begin{definition}[{\cite{BieMilRAFI,BieMilCDF}}] 
\label{Def:NashSubanalytic}
  A subset $X\subset N$ of a real analytic manifold $N$ is called \emph{Nash subanalytic} if $X$ is 
  the image of a proper Gabrielov regular real analytic map $\varphi : M \to N$ defined on a 
  real analytic manifold $M$.  
\end{definition}


\subsection{A generalized Hestenes lemma}
\label{sec:subanalytic-hestenes-lemma}

Last in this appendix we recall a subanalytic version of the lemma by Hestenes by 
Bierstone--Milman \cite{BieMilCDF,BieMilGDPSS} which gives a criterion when a
jet on a subanalytic set is in fact a Whitney function. 

\begin{lemma}[Hestenes's lemma for subanalytic sets, {\cite[Cor.~8.2]{BieMilCDF} \& \cite[Lem.~11.4]{BieMilGDPSS}}]
\label{Lem:subanalytic-hestenes-lemma}
Let $U \subset \R^n$ be open and $A,B  \subset U$ two relatively closed subanalytic subsets such that $B\subset A$. 
Let $G \in \sfJ^\infty (A)$ and assume that $\operatorname{res}^A_{B} G \in \calE^\infty (B)$ and
$\operatorname{res}^A_{A\setminus B} G \in \calE^\infty (A\setminus B)$. Then $G$ is a Whitney function
on $A$ that is $G \in \calE^\infty (A)$.
\end{lemma}


%%% Local Variables:
%%% mode: latex
%%% TeX-master: "HerPfl.InvariantWhitneyFunctions"
%%% End: