%
%
\section{Preliminaries on Whitney functions}
\label{sec:whitney-functions}
For the convenience of the reader and to set up notation we review in
this section the basic concepts of the theory of Whitney functions. 
We also provide the proof of a folklore theorem on the pullback of Whitney functions, 
which we could not find in the literature.
For further information on Whitney functions see \cite{MalIDF}.

\begin{definition}
  Let $X\subset \R^n$ be a locally closed subset, which means that there is
  an open $U\subset \R^n$ such that  $X\subset U$ is relatively  closed. 
  Assume that $m \in \N\cup \{ \infty\}$. Then denote by
  $\sfJ^m (X)$ the space of all \textit{jets of order $m$} 
  (or \textit{$m$-jets}) on $X$ which means the vector space of all families
  \[
    F = \big( F_\alpha \big)_{\alpha \in \N^n, \, |\alpha| \leq m}
    \quad \text{with} \quad F_\alpha \in \calC (X) . 
  \]
  For $K\subset X$ compact and every natural $k \leq m$ we define the 
  seminorm
  $| \cdot |_{K,k}$ on $\sfJ^m (X)$ by
  \[
    | F|_{K,k} := \sup_{x\in K\atop |\alpha| \leq k} | F_\alpha (x) | \: .
  \]
  For $\beta \in \N^n$ with $| \beta | \leq m$ denote by 
  $\partial^\beta : \sfJ^m (X) \rightarrow \sfJ^{m-|\beta|} (X)$ the 
  linear map given by 
  \[
    \big( F_\alpha \big)_{\alpha \in \N^n, \, |\alpha| \leq m}
    \mapsto \big( F_{\alpha+\beta} \big)_{\alpha \in \N^n, \, 
    |\alpha| \leq m-|\beta|} \: .
  \]

  Given two jets $E,F\in \sfJ^m (X)$ one defines their product 
  $ E F\in \sfJ^m (X)$
  as the jet with components 
  \[
     ( EF )_\alpha := \sum_{\beta + \gamma = \alpha \atop \beta , \gamma \in \N^n}
     E_\beta F_\gamma, \quad |\alpha| \leq m.
  \]
 
%  Assume now that $Y\subset X$ is closed, and that 
%  $l \in \N \cup \{ \infty \}$ satisfies  $l\leq m$. Then we denote by 
%  $R^{X,m}_{Y,l} :\sfJ^m(X) \rightarrow \sfJ^l(Y)$ the function which maps 
%  $\big( F_\alpha \big)_{\alpha \in \N^n, \, |\alpha| \leq m}$
%  to the restricted jet 
%  $\big( (F_\alpha)_{|Y} \big)_{\alpha \in \N^n, \, |\alpha| \leq l}$.
%  In case $X=Y$, we write $R^m_{l}$ instead of $R^{X,m}_{Y,l}$, 
%  in case $m=l$, we write shortly $R^{X,m}_{Y}$ or even only $R^X_Y$. 
  Finally, by the symbol $\sfJ^m $ we will denote the map 
  \[
   \sfJ^m=\sfJ^m_X: \calC^m (U) \rightarrow \sfJ^m (X), \:
   f \mapsto \big( (\partial^\alpha f)_{|X} \big)_{\alpha \in \N^n, \, |\alpha| \leq m} . 
  \]
\end{definition}
  The space $\sfJ^m (X)$ together with the topology generated by the family of 
  seminorms $| \cdot |_{K,k}$ as above forms a Fr\'echet space respectively 
  a Banach space in case $X$ is compact and $m$ finite. Moreover, the maps 
  $\partial^\alpha$ and $\sfJ^m$ all become continuous linear maps with respect to 
  this topology. Finally, the product of jets is associative, and $\sfJ^m (X)$
  becomes a Fr\'echet resp.~Banach algebra. 

  For later purposes let us briefly describe at this point the action of vector fields on 
  jets. Let $F \in \sfJ^\infty (X)$, and $\xi $ a smooth vector field defined 
  on an open neighborhood of $X$. Represent the vector field as $\xi = \sum_{i=1}^n \xi_i \partial_i$,
  where the coefficients $\xi_i$ are   uniquely 
  determined smooth functions on the domain of the vector field. Then, one puts
\[
   \xi F := \sum_{i=1}^n \sfJ^\infty (\xi_i) \, \partial_iF \: . 
\]
One checks immediately, that $\xi$ acts as a derivation on $\sfJ^\infty (X)$.

  It is the goal of the following considerations to provide an
  explicit representation of the image of the map 
  $\sfJ^m : \calC^m (U) \rightarrow \sfJ^m (X)$. To this end let us first define for 
  $F \in \sfJ^m (X)$, $a\in X$ and natural $k\leq m$ a function 
  $\sfT_a^k F \in \calC^\infty (\R^n)$ by
  \[
     \sfT_a^k F (x) := \sum_{\alpha \in \N^n \atop | \alpha | \leq k}
     \frac{(x-a)^\alpha}{\alpha !} F_\alpha (a), \quad \text{for $x\in \R^n$}.
  \]
  Then put
  \[
    \widetilde{\sfT}_a^k F := \sfJ^k (\sfT_a^k F) \: \text{ and } \:
    \sfR_a^k F := F -  \widetilde{\sfT}_a^k F .
  \]
  %
  \begin{definition}
    Let $X\subset \R^n$ be a locally closed subset. An element $F \in \sfJ^m (X)$
    with $m\in \N$ is called a \textit{Whitney function of order $m$} 
    on $X$, if for every compact set $K\subset X$ and every $\alpha \in \N^n$ 
    with $| \alpha | \leq m$ the following relation holds true:
    \[
      \left( \sfR^m_x \right)_\alpha (y) = o ( | x-y|^{m-|\alpha|}) \quad 
      \text{for $x,y\in K$ as $|x-y| \rightarrow 0$}. 
    \]
    An element $F \in \sfJ^\infty (X)$
    is called a \textit{Whitney function of order $\infty$}, if for every $m\in \N$
    the $m$-jet $\sfJ^m F$ is a Whitney function of order $m$.
    For every $m\in \N \cup \{ \infty \} $ we will denote by $\calE^m (X) $ 
    the space of Whitney functions of order $m$ on $X$.
  \end{definition}
 
For  $m\in \N \cup \{ \infty \}$,  there is system of seminorms $\|\cdot \|_{K,k}$ on the space  
$\calE^m (X)$ indexed by compact subsets  $K\subset X$ and integers $k\le m$. This system is
defined as follows. For each $k\in\N$ such that $k\le m$ and each compact $K\subset X$ put
\begin{align*}
\| F \|'_{K,k}:=\sup_{x,y\in K, x\ne y\atop \alpha\in \N^n,|\alpha|\le k}{\left|(\sfR^k_x F)_\alpha(y)\right|\over|x-y|^{k-|\alpha|}}
\end{align*} 
and set $\| F \|_{K,k}:= |F |_{K,k}+\| F \|'_{K,k}$. The space of Whitney functions  $\calE^m (X)$ together 
with the system of seminorms $(\| \cdot \|_{K,k})_{K,k}$ forms a  Fr\'echet algebra. In the case when 
$X\subset \R$ is Whitney regular, the system of seminorms 
$(\| \cdot \|_{K,k})_{K,k}$ is equivalent to the system of seminorms $(| \cdot |_{K,k})_{K,k}$. For more 
details we refer the reader to \cite[Prop.~2.6 \& Prop.~3.11]{TouIFD}.

The following fundamental theorem by Whitney 
determines in particular the image of the map $\sfJ^m$. 

\begin{theorem}[Whitney's Extension Theorem \cite{WhiAEDFCS}, 
 cf.~also {\cite[Sec.~I]{MalIDF}}]\label{WhitneyExtension}
 Let $X \subset \R^n$ be locally closed, and $U\subset \R^n$ be open. 
 Assume that $m\in \N$. Then there exists a continuous linear section
 \[
  \sfW^m : \calE^m (X) \rightarrow \calC^m (U) 
 \] 
 of the jet map $\sfJ^m : \calC^m (U) \rightarrow \sfJ^m (X)$, which in other words
 means that $\sfJ^m \circ \sfW^m = \id_{\calE^m (X)}$. In particular this implies
 that $\calE^m (U) \cong \calC^m (U)$.

 Assume now that $m \in \N \cup \{ \infty \}$ and that $Y\subset X$ is closed. 
 Under this assumption denote by $\calJ^m (Y,X)$ the kernel of the restriction 
 map 
 \[
      \operatorname{res}^X_Y : \calE^m (X) \rightarrow \calE^m (Y), \: \big( F_\alpha \big)_{\alpha \in \N^n, \, |\alpha| \leq m}
      \mapsto \big( (F_\alpha)_{|Y} \big)_{\alpha \in \N^n, \, |\alpha| \leq m} \: . 
 \]
 % or in other words the space of Whitney functions of order $m$ on $X$ whose 
 % restriction to $Y$ vanish. 
 As a consequence, the sequence 
 \[
  0 \xrightarrow{\hspace{1.5em}} \calJ^m(Y,X) 
    \xrightarrow{\hspace{1.5em}} \calE^m (X) 
    \overset{\operatorname{res}^X_Y}{\xrightarrow{\hspace{1.5em}}} 
    \calE^m (Y) \xrightarrow{\hspace{1.5em}} 0
 \]
 is short exact.
\end{theorem}

It is an immediate consequence of Whitney's Extension Theorem that the action of 
$\partial^\alpha$ and more generally of smooth vector fields on the jet space $J^\infty (X)$ leaves the 
subspace $\calE^\infty (X)$ of Whitney functions invariant.
Likewise, the product of two Whitney functions of order $m \in \N^*\cup \{ \infty \}$ 
is again a Whitney function of order $m$.  
\vspace{2mm}

Let us now introduce some more notation used in this paper. As above let 
$Y\subset X \subset U \subset \R^n$ with $U$ being open and $Y$ and $X$ being
relatively closed in $U$. Denote by $\calJ (X,U) \subset \calC^\infty (U)$
or by $\calJ (X)$ if no confusion can arise
the ideal of all smooth functions on $U$ vanishing on $X$, and
let $\calJ^\infty (X,U) \subset \calC^\infty (U)$ be the ideal of 
smooth functions on $U$ which are flat on $X$, i.e.~let 
\[
  \calJ^\infty (X,U) := \{ f \in \calC^\infty (U) \mid (\partial^\alpha f)_{|X} = 0 
  \text{ for all $\alpha \in \N^n$}\}.
\] Then put 
\begin{eqnarray}
\label{Eq:DefSmFct}
  \calC^\infty (X)  & \!\! := \!\! & \calC^\infty (U) / \calJ (X,U) , 
  \quad \text{and} \\
\label{Eq:DefSmFctRel}
  \calC^\infty (X,Y)  &  \!\! := \!\! &
  % \calC^\infty (U,Y) \cdot \calC^\infty (X) = \\
  %& \mbox{ } \hspace{46mm} = 
  \{f \in \calC^\infty (X)\mid \text{there exists $\tilde f \in 
  \calJ^\infty (Y,U)$ s.t.~$\tilde f_{|X} = f$} \}. % \nonumber
\end{eqnarray}
One calls $\calC^\infty (X)$ the \textit{algebra of smooth functions on $X$}
and $\calC^\infty (X,Y)$ the \textit{ideal of smooth functions on $X$ 
flat on $Y$}. Note that neither $\calC^\infty (X)$ nor $\calC^\infty (X,Y)$ 
depend on the particular choice of the ambient $U$.
By Whitney's Extension Theorem one obtains 
\begin{equation}
  \label{Eq:WET}
  \calE^\infty (X) \cong \calC^\infty (U) / \calJ^\infty (X,U), \quad 
  \text{and} \quad
  \calJ^\infty (Y,X) \cong \calJ^\infty (Y,U) / \calJ^\infty (X,U) .
\end{equation}
We finally put
\begin{equation}
  \label{Eq:ISF}
  \begin{split}
  \calE^\infty (Y, X) & := \calC^\infty (X) / \calC^\infty (X,Y) = \\
  & = \calC^\infty (U) / \calJ (X)  \Big/ \calJ^\infty (Y,U) / 
  \calJ^\infty (Y,U) \cap \calJ (X). 
\end{split}
\end{equation}
We will call $\calE^\infty (Y, X)$ the \textit{algebra of Whitney functions 
on $Y$ induced by smooth functions on $X$}. Note that
\begin{equation}
\label{Eq:QR}
\begin{split}
  \calE^\infty (Y,X) =  \calE^\infty (Y ) / 
  \calJ (X)\cdot \calE^\infty (Y) . 
\end{split}
\end{equation}

\begin{remark}
  \begin{enumerate}
  \item  According to its definition above, the symbol $\calC^\infty (X,Y)$ denotes here the same algebra as in 
  \cite{BieMilPawCDF}. This notation differs though from the one used in \cite{BieSchwaCLDEF}.  
  \item  Note that  $\calC^\infty (X,Y)$ coincides with the quotient of $\calJ^\infty (Y,X)$ by the 
     ideal $\calJ^\infty (Y,X) \cap \big\{ F \in \calE^\infty (X) \mid F_0 = 0 \big\}$. Unless $X$ 
     is an open subset of the ambient euclidean space, this ideal is in general non-zero, hence   
     $\calC^\infty (X,Y)$ and  $\calJ^\infty (Y,X)$ do in general not coincide.      
  \end{enumerate}

 
\end{remark}

Next, we wish to examine the behavior of Whitney functions with respect to the pullback 
 under smooth maps. To this end, let  $\varphi:U \to V$ be a smooth map  between
open subsets $U\subset \mathbb R^n$ and $V\subset \mathbb R^m$. 
Furthermore, assume that $X\subset U$ 
  and $Y\subset V$ be relatively closed subsets such that $\varphi(X)\subset Y$. 
Now, for a jet  $F \in \sfJ^\infty (Y)$  we  define, mimicking the formula of Fa{\`a} di Bruno 
(cf.~Theorem \ref{FdBmulti}), its  pullback  
  $\varphi_{X,Y}^{\sharp} F = (\varphi_{X,Y}^{\sharp} F )_{\alpha \in \N^n} \in \sfJ^\infty (X)$ 
  as follows: 
  \begin{equation}
    \label{Eq:DefPullbackJet}
    \begin{split}
 %    (& \varphi_{X,Y}^{\sharp}  F )_\alpha  = \\
 %    & = \sum_{1 \leq | \beta | \leq |\alpha| \atop \beta \in \N^m}
 %    (F_\beta \circ \varphi ) \: \sum_{l=1}^{|\alpha|} \:
 %   \sum_{(\kappa_1,\ldots,\kappa_l,\lambda_1,\ldots,\lambda_l)\in P_l(\alpha,\beta)} \:
 %   ( \alpha! ) \, \prod_{j=1}^l 
 %   \frac{(\partial^{\lambda_j}\varphi)^{\kappa_j}}{(\kappa_j!)(\lambda_j!)^{|\kappa_j|}} \\
      (& \varphi_{X,Y}^{\sharp}  F )_\beta= \sum_{\lambda \in \Lambda_{n,m}(\beta)} \frac{\beta !}{\lambda !} \quad F_{\sum_{\alpha}\lambda_\alpha} \circ \varphi \quad \prod_{\alpha}\frac{\left(\partial^\alpha \varphi\right)^{\lambda_\alpha}}{(\alpha!)^{\sum_i \lambda_{i,\alpha}}}
    ,
  \end{split}
  \end{equation}
with the notation as in Theorem \ref{FdBmulti}.
  The following result is an immediate consequence of the formula by 
  Fa{\`a} di Bruno. 
\begin{theorem}
\label{Thm:PullbackJet}
  Let $U\subset \R^n$ and $V\subset \R^m$ be open, and assume that $X\subset U$ 
  and $Y\subset V$ are relatively closed. Let $\varphi: U \rightarrow V$ be a 
  smooth map such that $\varphi (X) \subset Y$. 
  Then the pullback map $\varphi^{\sharp}:=\varphi_{X,Y}^{\sharp} : \sfJ^\infty (Y) \rightarrow \sfJ^\infty (X)$,
  $F\mapsto \varphi_{X,Y}^{\sharp} F$ is a continuous linear map. 
  Moreover, it makes the following diagram commute:
  \begin{equation}
  \xymatrix{
      0\ar[r]& \calJ^\infty(Y,V)\ar[r]\ar[d]^{\varphi^*}
      &\calC^\infty(V)\ar[r]^{\sfJ^\infty_Y}\ar[d]^{\varphi^*}&
      \calE^\infty(Y) \ar[r]\ar[d]^{\varphi_{X,Y}^{\sharp}}&0\\
      0 \ar[r]&\calJ^\infty(X,U)\ar[r]& \calC^\infty(U) \ar[r]^{\sfJ^\infty_X} &
      \calE^\infty(X)  \ar[r]& 0 \:,
  }\label{DefPullbackWhi}
\end{equation}
  where $\varphi^* : \calC^\infty(V) \rightarrow \calC^\infty(U)$ is the pull-back
  $f \mapsto f\circ \varphi$.
\end{theorem}
\begin{proof}
  By definition, $\varphi^{\sharp}$ is linear. To check that  $\varphi^{\sharp}$ is continuous
  let $K \subset X$ be compact and $k\in \N$. Then 
  \begin{align}\label{SupnormEstimate}
 \nonumber  | \varphi^{\sharp}  F|_{K,k}  & \leq \left(\sup_{y\in \varphi (K) \atop |\beta| \leq k} | F_\beta (y) |\right) \!\! 
\quad \left(\sup_{x\in X}   \sum_{\lambda \in \Lambda_{n,m}(\beta)} {\beta !\over \lambda}\quad \prod_{\alpha}\frac{\left|\partial^\alpha \varphi(x)\right|^{\lambda_\alpha}}{(\alpha!)^{\sum_i \lambda_{i,\alpha}}}\right)\\
   & \leq C_{\varphi , K, k }  \: |F|_{\varphi (K) ,k} ,
  \end{align}
  where the constant $C_{\varphi , K, k }>0 $ depends only on $\varphi$, $K$ and 
  $k$. 
  
  
  In order to prove continuity of $\varphi^{\sharp}$ we need a similar estimate for the remainder term. To this end we observe that for $x\in X$ and $k\in\N$
  \begin{align*}
  \sfR_x^k\circ \varphi^{\sharp}&=(\id-\widetilde \sfT_x^k)\circ \varphi^{\sharp}\\
  &=\varphi^{\sharp}-\sfJ^k\circ T^k_x \circ\varphi^{\sharp}=\varphi^{\sharp}-\sfJ^k\circ \varphi^*\circ T^k_{\varphi(x)}\\
  &=\varphi^{\sharp}-\varphi^{\sharp}\circ\sfJ^k\circ T^k_{\varphi(x)}=\varphi^{\sharp}\circ\sfR_{\varphi(x)}^k.
  \end{align*}
  
  
Let us assume for the moment that $\varphi(x)\ne \varphi(x')$. Then we have for $|\alpha|\le k$: 
  \begin{align}\label{Lipschitz}
  {\left(\sfR_x^k(\varphi^{\sharp}F)\right)_\alpha\over |x-x'|^{k-|\alpha|}}&= {|\varphi(x)-\varphi(x')|^{k-|\alpha|}\over |x-x'|^{k-|\alpha|}}{\left(\varphi^{\sharp}(\sfR_{\varphi(x)}^k F)\right)_\alpha\over |\varphi(x)-\varphi(x')|^{k-|\alpha|}}
  \end{align}
  The first term on the right hand side can be estimated by $L^{k-|\alpha|}$, where $L$ is a Lipschitz constant for $\varphi_{|K}$. On the other hand one can prove using an argument similar to \eqref{SupnormEstimate}
  \begin{align*} 
  \left|\left(\varphi^{\sharp}(\sfR_{\varphi(x)}^k F)(x')\right)_\alpha\right|&\le C_{\varphi,\varphi(K),|\alpha|}\sup_{\beta\le \alpha}\left| (\sfR_{\varphi(x)}^k F)_{\beta}(\varphi(x'))\right|\\
  & \le C_{\varphi,\varphi(K),|\alpha|}\sum_{\beta\le \alpha}\left| (\sfR_{\varphi(x)}^k F)_{\beta}(\varphi(x'))\right|
  \end{align*}
  We observe that $|\varphi(x)-\varphi(x')|\le \operatorname {diam}(\varphi(K))<\infty$ and  obtain 
  \begin{align}\label{PullBackIsContinuous}
\nonumber  {\left(\sfR_x^k(\varphi^{\sharp}F)\right)_\alpha\over |x-x'|^{k-|\alpha|}}&
   \le \:\sum_{\beta\le \alpha} L^{k-|\alpha|} C_{\varphi,\varphi(K),|\alpha|}\operatorname {diam}(\varphi(K))^{|\alpha|-|\beta|}\:{\left| (\sfR_{\varphi(x)}^k F)_{\beta}(\varphi(x'))\right|\over |\varphi(x)-\varphi(x')|^{k-|\beta|}}\\
   &\le |\!|F|\!|'_{\varphi(K),k} \sum_{\beta\le \alpha} L^{k-|\alpha|} C_{\varphi,\varphi(K),|\alpha|}\operatorname {diam}(\varphi(K))^{|\alpha|-|\beta|}.
  \end{align}
  
  In order to deal with the case when  $\varphi(x)=\varphi(x')$ we observe that $\sfR_{x}^k (\varphi^{\sharp} F)(x')=\varphi^{\sharp}\sfR_{\varphi(x)}^k F(\varphi(x'))=0$. This implies that the left hand side of \eqref{Lipschitz} vanishes for all $x\ne x'$. This means that \eqref{PullBackIsContinuous} holds in all cases when $x\ne x'$, implying that 
  \begin{align*}
 |\!|\varphi^{\sharp} F|\!|'_{K,k}&\le \widetilde C_{\varphi,\varphi(K),k} |\!|F|\!|'_{\varphi(K),k},\\
 \widetilde C_{\varphi,\varphi(K),k}&= \sup_{\alpha,|\alpha|\le k}\left(\sum_{\beta\le \alpha} L^{k-|\alpha|}\:C_{\varphi,\varphi(K),|\alpha|}\:\operatorname {diam}(\varphi(K))^{|\alpha|-|\beta|}\right).
\end{align*}
This completes the proof that $\varphi^{\sharp}$ maps Whitney functions to Whitney functions and that
$\varphi^{\sharp}:\calE^m(Y)\to\calE^m(X)$ is a continuous map between Fr\'echet spaces for $m\in\N\cup\{\infty\}$. 

It remains to show that the diagram \eqref{DefPullbackWhi} is commutative.
  The formula of Fa\`a di Bruno immediately entails that 
  $\varphi^* f \in \calJ^\infty (X,U)$ for all $f\in \calJ^\infty (Y,V)$. 
  Hence the left square  in the diagram above commutes. To prove that
  the right square commutes as well, one has to show that for an element 
  $f \in \calC^\infty (V)$ the relation
  \begin{equation}
  \label{Eq:CommPol} 
     \varphi^{\sharp} \circ \sfJ^\infty (f)  = \sfJ^\infty ( f\circ \varphi ) 
  \end{equation}
  holds true. But this is clear by the formula of Fa\`a di Bruno 
  and Eq.~\eqref{Eq:DefPullbackJet}.
  Hence the claim holds true. 
\end{proof}
\begin{remark}
Note that with the notation of Theorem \ref{WhitneyExtension} $\operatorname{res}^X_Y=\id^{\sharp}_{Y,X}$ where $\id:U\to U$ is the identity map.
\end{remark}

%%% Local Variables:
%%% mode: latex
%%% TeX-master: "HerPfl.InvariantWhitneyFunctions"
%%% End:
