%
%
\section{The orbit type stratification and saturations}
\label{orbittypestratification}
%
%
In this paper, we follow Mather's concept of a stratification, 
cf.~\cite{MatSM,PflAGSSS}, which essentially is a local one and allows for a 
clear construction of the orbit type stratification 
of a $G$-manifold $M$.

\begin{definition}
By a \emph{decomposition} of a paracompact topological space $X$ with countable topology we understand a locally finite 
partition $\calZ$ of $X$ into locally closed subspaces $S$, 
called \emph{pieces}, such that each piece $S\in \calZ$ carries
the structure of a smooth manifold and such that the following \emph{condition of frontier} is satisfied:
\begin{enumerate}
\item[(CF)] 
  If $R,S \in \calZ$ are two pieces such that $R \cap \overline{S} \neq \emptyset$, then $R \subset \overline{S}$. 
  In this case we say that $R$ is \emph{incident} to $S$. 
\end{enumerate}
If $\calZ$ and $\calZ'$ are two decompositions of $X$, we say that 
$\calZ$ is \emph{coarser} than $\calZ'$, if every piece of $\calZ'$ is
contained in a piece of  $\calZ$. 

A \emph{stratification}  of $X$ (in the sense of Mather)
is a map $\calS$ which associates to each 
point $x\in X$ the set germ $\calS_x$ of a locally closed subset of $X$ 
such that the following axiom holds true:
\begin{enumerate}
\item[(ST)] For each $x \in U$ there exists an open neighborhood $U$ and a 
  decomposition $\calZ_U$ of $U$ such that for each $y\in U$ the set germ 
  $\calS_y$ coincides with the set germ of the unique piece $R_y \in \calZ_U$ 
  which contains $y$ as an element. 
\end{enumerate}
\end{definition}

A decomposition of $X$ obviously induces a stratification, namely the one 
which associates to each point the piece in which that point lies. 
The crucial observation from \cite[Prop.~1.2.7]{PflAGSSS} now is that a 
stratification $\calS$ of $X$ in the sense of Mather is always induced by a 
global decomposition of the underlying space, and that among those 
decompositions there is a coarsest one. We usually denote that 
decomposition by the same symbol $\calS$ as for the stratification, and call 
the pieces of that decomposition the \emph{strata} of the stratification. 

\begin{example}
  Let $M$ be an analytic manifold, and $Z \subset M$ a subanalytic subset. 
  Then $Z$ possesses a minimal stratification fulfilling Whitney's condition B. 
  See \cite{BieMilSSS,BekRSSS} for subanalytic sets and their stratifications, and \cite{PflAGSSS} for 
  details on the Whitney conditions.  
\end{example}

Now assume that $G$ is a compact Lie group acting on a smooth manifold $M$. 
Denote for every point $x\in M$ by $G_x$ the isotropy group of $x$. 
For every closed subgroup $H \subset G$ the set 
\[
 M_{(H)} := \{ x \in M \mid G_x \text{ is conjugate to } H \}
\]
then is a smooth submanifold of $M$, possible with varying dimensions of its 
connected components. Moreoever, the map $\calS$ which associates
to each $x\in M$ the set germ of the submanifold 
$M_{(G_x)}$ at $x$ is a stratification of $M$, which we call the 
\emph{orbit type stratification} of $M$. Since each of the sets $M_{(H)}$ is 
$G$-invariant, the orbit type stratification descends
to a stratification of the orbit space $X := M/G$. This stratification is
also called \emph{orbit type stratification}. See \cite[Sec.~4.3]{PflAGSSS} 
for details. 

\begin{example} Let $n\ge 2$ and consider the orthogonal group $G=\operatorname{O}_n$ acting $V=T^*\R^n=\R^n\times \R^n=\{(q, p)\mid q, p\in \R^n\}$ as the cotangent lift of the defining representation (i.e.~$G$ acts on $\R^n\times \R^n$ diagonally). There are three orbit type strata:
\begin{enumerate}
\item the stratum $V_{(G)}=\{0\}$ of points with isotropy group equal $G$, 
\item the stratum $V_{(\operatorname{O}_{n-1})}=\{(q,p)\ne (0,0) \mid q\:||\: p\}$ of points with isotropy conjugate to $\operatorname{O}_{n-1}$, and
\item the stratum $V_{(e)}=\{(q,p) \mid q \nparallel p\}$ of points with trivial isotropy.
\end{enumerate}
\end{example}

In the following we extend the definition of orbit type stratification to 
singular subsets of a $G$-manifold $M$. The action of a group element $g\in G$ on $x\in M$ will be denoted by
$g.x$. 

\begin{definition}
  Let $G$ be a compact Lie group acting smoothly on a manifold $M$. We will say that a relatively closed subset $Z$ 
  of some open $G$-invariant subset $U \subset M$ is \emph{stratified by orbit type}, if the following conditions 
  hold true:
  \begin{enumerate}[(OT1)]
  \item \label{ite:OT1} 
        If $S$ is a stratum of the stratification of $U$ by orbit type, then 
        the intersection $Z\cap S$ is an analytic submanifold of $U$, 
        possibly with  connected components having varying dimensions. 
  \item \label{ite:OT3} 
        If $x,y \in Z$ are points for which there is a group element $g\in G$ 
        with $g.x =y$,  then 
        \[
          g.\big( T_x (Z\cap S) \big) = T_y(Z\cap S) ,
        \] 
        %\[
        %  g.\big( T_x (Z\cap S) / T_x \calO_x \cap T_x (Z\cap S)\big) = 
        %  T_y(Z\cap S)  / T_y \calO_y \cap T_y (Z\cap S) ,
        %\] 
        where $S$ is the orbit type stratum in $U$ which contains $x$ and $y$.
  \end{enumerate}
\end{definition}

\begin{lemma}
Let $M$ be a $G$-manifold and $Z \subset M$ relatively closed  
in some $G$-invariant open subset $U\subset M$ be stratified by orbit type. 
Then the following holds true:
\begin{enumerate}[{\rm (OT1)}]
\setcounter{enumi}{2}
 \item \label{ite:OT2} 
        If $S$ is a stratum of the stratification of $U$ by orbit type, and $x \in Z\cap S$,
        then there is an open neighborhood $O\subset M$ of $x$ such that for all $y \in O\cap Z\cap S$
        \[
           \dim \big( T_y \calO_y \cap  T_y(Z\cap S) \big) = \dim \big( T_x \calO_x \cap  T_x(Z\cap S) \big) \ .
        \] 
        Hereby, $\calO_y$ denotes the orbit through $y$. 
\end{enumerate}
\end{lemma}

\begin{proof}
  The claim follows from (OT\ref{ite:OT3}) and the fact that 
  $g.(T_x \calO_x ) = T_y \calO_y$.
\end{proof}

\begin{proposition}
   Let $M$ be a $G$-manifold, $U\subset M$ be open and $G$-invariant, and assume that $Z \subset U$ is a 
   relatively closed subset which is stratified by orbit type. 
   Then $G.Z$ is stratified by orbit type as well.
   Moreover, assigning to each $x\in Z$ (resp.~$x\in G.Z$) the set germ $[Z\cap S]_x$ 
   (resp.~ $[(G.Z)\cap S]_x$), where $S$ is the orbit type stratum of $U$ containing $x$,
   provides a stratification of $Z$ (resp.~$G.Z$). These stratifications are $G$-invariant which means that
   $g [Z\cap S]_x  = [Z\cap S]_{gx}$ for all $x \in Z\cap S$ and $g\in G$ with $gx \in  Z\cap S$, and  
   $g [(G.Z)\cap S]_y  = [(G.Z)\cap S]_{gy}$ for all $y \in G.Z\cap S$ and $g\in G$.
\end{proposition}
\begin{proof}
   We first show that for every relatively closed $Z \subset U$ which is stratified by orbit type the assignment
   $Z \ni x \to \calS_x := [Z\cap S]_x$, where $S$ is the orbit type stratum of $U$ containing $x$, is 
   a stratification of $Z$. Let $x \in Z$. Choose an open neighborhood $O$ of $x$ according to   (OT\ref{ite:OT2}).
   Then, by (OT\ref{ite:OT1}), $O \cap Z\cap S$ is a smooth manifold of some fixed dimension, and 
   $[Z\cap S]_x =  [Z\cap S \cap O]_x$. Locally, the assignment $\calS$ comes from a decomposition of $Z$, since
   the ambient manifold is decomposed by orbit type strata. Moreover,  since the decomposition  of the ambient space 
   by orbit type strata is locally finite, the local decomposition of $\calZ$ inducing $\calS$ has to be locally finite, 
   too. Therefore, the assignment  $Z \ni x \to \calS_x $ is a stratification of $Z$, indeed. 
   (OT\ref{ite:OT3}) entails that the stratification $\calS$ is $G$-invariant.
 
   It remains to show that $G.Z$ is stratified by orbit types. To this end choose a 
   $G$-invariant riemannian metric on $M$ and note that $G.Z$ is a relatively closed 
   subset of $U$. Next consider an orbit type stratum $S$ of $M$, let $y\in G.Z \cap S$ 
   and choose $g\in G$ with $x = gy \in Z$. By $G$-invariance of $S$ we have 
   $x \in Z \cap S$. 
   Choose an open connected neighborhood $O$ of $x$ such that (OT\ref{ite:OT2}) holds true. Let 
   $\mathfrak{h} \subset \operatorname{Lie} (G)$ be the sub Lie algebra consisting of all 
   $\xi \in \operatorname{Lie} (G)$ such that $\xi_V (x) \in T_x (Z\cap S)$, where $\xi_V$ 
   denotes the fundamental  vector field of $\xi$. Let $\mathfrak{m}$ be a complement of $\mathfrak{h}$
   in $\operatorname{Lie} (G)$. By (OT\ref{ite:OT2}),  we can find, after possibly shrinking $O$,  
   connected open neighborhoods $O_1 \subset \mathfrak{m}$ and $O_2 \subset T_x(Z\cap S)$ of the origin 
   such that the function
   \[
     \psi : O_1 \times O_2 \to M, \quad (\xi,X) \mapsto (\exp\xi) . \exp(X)
   \] 
   is well-defined, is an open embedding, and has image $O$.  Put $O' := g^{-1}.O$. The map 
   $\chi : O' \to O_1 \times O_2 \subset \mathfrak{m} \times T_x(Z\cap S)$, $z \mapsto \psi^{-1} ( g . z)$ 
   then is a smooth chart of $(G.Z) \cap S$ around $y$. 
   Property  (OT\ref{ite:OT3}) entails that if we choose a different $g \in G$ with 
   $g.x \in Z$, we obtain another smooth chart of $(G.Z) \cap S$ around $y$ which is 
   $\calC^\infty$-equivalent to the first. This proves that $G.Z$ satisfies (OT\ref{ite:OT1}). 
   %Property (OT\ref{ite:OT2}) holds true for $G.Z$, since 
   %$T_x \calO_x \cap  T_x\big( (G.Z)\cap S \big) = T_x\calO_x$ for all 
   %$x\in (G.Z) \cap S$, and since the points of $S$ have all the same isotropy type. 
   Finally, $(G.Z) \cap S$ fulfills (OT\ref{ite:OT3}), since 
   both $G.Z$ and $S$ are $G$-invariant.  
\end{proof}


%%% Local Variables:
%%% mode: latex
%%% TeX-master: "HerPfl.InvariantWhitneyFunctions.tex"
%%% End:
