\section{Introduction} \label{sec:intro}

The purpose of this paper  is to present an analogue of the Theorem of Gerald Schwarz \cite[Theorem 1]{SchwaSFIACLG} on differentiable invariants of representations of a compact group for the case of invariant Whitney functions. The main difficulty to overcome is to include 
also the case of Whitney functions along subsets that are not stable under the action of the group. We develop the theory as a tool for Hochschild homology calculations for algebras of smooth functions on orbit spaces \cite{HerPflHHASFOS}. We hope that the results are of independent interest.

To explain our findings we start by recalling Schwarz's result. 
Suppose that $G$ is a compact Lie group and $V$ is a finite dimensional vector space over the field of real numbers $\mathbb R$ on which $G$ acts linearly. Without loss of generality we can assume that the action of $G$ preserves a euclidean scalar product $\langle\;,\;\rangle$, which means that we actually have an 
orthogonal representation $G\to \operatorname O(V)$ into the orthogonal group of euclidean vector space $(V,\langle\;,\;\rangle)$. For the  \emph{orbit space} of the $G$-action on $V$ we write $V/G$. 

By the Theorem of Hilbert and Weyl (see for example \cite[\S 98] {ZelCLGR}) there exists a complete system of polynomial invariants $\rho_1,\dots ,\rho_\ell\in \R[V]^G$, which we choose to be homogeneous and minimal. We refer to $\rho_1,\dots ,\rho_\ell$ as the \emph{Hilbert basis} of the representation. By the \emph{Hilbert map} we mean the vector valued regular map $\rho=(\rho_1,\dots ,\rho_\ell): V\to \mathbb R^\ell$. 
The Hilbert map  descends to a proper embedding $\cc \rho: V/G\to \R^\ell$, i.e.~the diagram 
\begin{eqnarray*}
\xymatrix{V\ar[r]^{\rho\qquad\quad}\ar[d]_{\pi}  & \:\: 
  X:=\rho(V)\subset \mathbb R^\ell&\\
  V/G\ar[ru]_{\cc\rho}&&}
\end{eqnarray*}
commutes. Here $\pi$ denotes the orbit map. We refer to $\cc\rho$ as the \emph{Hilbert embedding}.
By the Tarski-Seidenberg principle, the image 
$X=\rho(V)\subset \mathbb R^\ell$ of the polynomial map $\rho$ is a semialgebraic subset. A more constructive approach to describing the semialgebraic set $X$ has been elaborated in \cite{ProSchIDOS}.

\begin{theorem} \label{SchwaSFIACLG} Let $G\to \operatorname{O}(V)$ be a finite dimensional orthogonal representation of the compact Lie group $G$. Let $\rho_1,\dots,\rho_\ell\in\mathbb R[V]^G$ be a minimal complete system of polynomial invariants and $\rho:=(\rho_1,\dots,\rho_\ell): V\to \mathbb R^\ell$ the corresponding Hilbert map. Then the pullback 
\[\rho^*:\calC^\infty(\mathbb R^\ell)\to\calC^\infty(V)^G\] 
is split sujective. That means that there exists a continuous map $\lambda:\calC^\infty(V)^G\to\calC^\infty(\mathbb R^\ell)$ such that $\rho^*\circ \lambda$ is the identity map $\calC^\infty(V)^G\to\calC^\infty(V)^G$.
\end{theorem}

The surjectivity of $\rho^*$ has been established by Gerald Schwarz \cite[Thm.~1]{SchwaSFIACLG}. The existence of the continuous split has been proven later by 
John Mather in \cite{MatDI}. For an exposition of the material the reader might want to consult the monograph \cite{BierstonesBook}. 

It is natural to regard the orbit space $V/G$ as a topological space with a \emph{smooth structure} $\mathcal C^\infty(V/G):=\mathcal C^\infty(V)^G$. Similarly $X:=\rho(V)$ carries the smooth structure $\mathcal C^\infty(X)=\{f\in \calC(X)\mid \exists F\in \calC^\infty(\mathbb R^\ell):\: F_{|X}=f\}$. As a corollary to Theorem  \ref{SchwaSFIACLG} the Hilbert embedding $\cc\rho$ is a \emph{diffeomorphism} of $V/G$ onto $X$, i.e., a homeomorphism such that 
$\cc\rho^*:\mathcal C^\infty(X)\to \mathcal C^\infty(V/G)$ is an isomorphism of algebras.
 
We refer to Theorem \ref{SchwaSFIACLG} as the Theorem of Schwarz and Mather on differentiable invariants. If we do not refer to the continuous split we speak of the Theorem of Schwarz. 

Our objective is to prove an analogue of the Theorem of Schwarz for  Whitney fields $\calE^\infty(Z)$ along a 
locally closed semialgebraic or subananlytic set $Z\subset V=\mathbb R^n$. Note that we do not assume $Z$ to be 
$G$-stable. In order to speak of invariance of such Whitney fields we make use of the language of groupoids. \\


%%% Local Variables:
%%% mode: latex
%%% TeX-master: "HerPfl.InvariantWhitneyFunctions"
%%% End: