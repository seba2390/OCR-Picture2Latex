\section{Tools  from differential analysis}
%
For the convenience of the reader we collect in this appendix some tools from 
Differential Analysis which we need in this paper and which appear to be 
scattered through the literature. 

\subsection{The formula of Fa{\`a} di Bruno}
\label{sec:formula-faa-di-bruno}
Here we recall the combinatorial and the multiindex version of the formula 
of Fa{\`a} di Bruno. 

Thoughout this section $\varphi=(\varphi^1, \dots ,\varphi^m): U\to V$ denotes a smooth map between open subsets $U\subset \mathbb R^n$ and $V\subset \mathbb R^m$ and $f\in \calC^\infty(V)$. Coordinates for $U$ and $V$  are denoted by $x^1,\dots,x^n$ and $y^1,\dots,y^m$, respectively. We are interested in the higher order partial derivatives of $f\circ \varphi$. 

By a \emph{partition into $k\ge 1$ blocks} of the set $[\ell]:=\{1,\dots,\ell\}$ we mean a decomposition into nonempty subsets $I_1\sqcup\dots\sqcup I_k=[\ell]$, called blocks, regardless of the order of the blocks. The number of partitions of  $[\ell]$ into $k$ blocks is given by the Stirling number of the second kind $\{\tiny\begin{matrix}\ell\\k\end{matrix}\tiny\}$. In total the number of partitions of $[\ell]$ is given by the Bell numbers $B_\ell$; the first few Bell numbers are $1, 2, 5, 15, 52, 203,\dots$.

Let us choose some indices $i_1,\dots,i_\ell \in \{1,\dots m\}$. Given a subset $I=\{k_1,\dots,k_r\}\subset [\ell]$ of cardinality $r=|I|$ we introduce the shorthand notation 
\[\frac{\partial^{|I|}}{\partial x^I}:=\frac{\partial^r}{\partial x^{i_{k_1}}\dots \partial x^{i_{k_r}}}\:.\]

\begin{theorem}[Formula of Fa{\`a} di Bruno -- combinatorial version]\label{FdBcomb} With the notation from above we have for all $x\in U$
\begin{eqnarray*}
\lefteqn{\left(\frac{\partial^\ell}{\partial x^{i_1}\dots \partial x^{i_\ell}}f\circ \varphi\right)(x)}&&\\
\nonumber&=&\sum_{k=1}^\ell\quad\sum_{I_1\sqcup\dots\sqcup I_k=[\ell]}\quad\sum_{j_1,\dots,j_k=1}^m\frac{\partial^kf}{\partial y^{j_1}\dots \partial y^{j_k}}(\varphi(x))\:\frac{\partial^{|I_1|} \varphi^{j_1}}{\partial x^{I_1}}(x)\dots\frac{\partial^{|I_k|} \varphi ^{j_k}}{\partial x^{I_k}}(x),
\end{eqnarray*}
where the summation is taken over decompositions of $[\ell]=\{1,\dots,\ell\}$ into blocks as described above.\end{theorem}

The combinatorial version of the Fa{\`a} di Bruno formula is easy to prove, it contains no combinatorial factors and it is clear how it transforms under linear change of coordinates. Using multiindex notation, symmetry factors have to be taken into account and things look slightly more complicated. Let us consider maps of the form
\[\lambda:[m]\times (\mathbb N^n\backslash\{0\})\to \mathbb N,\qquad(i,\alpha)\mapsto \lambda_{i,\alpha}\] 
and let us denote the set of such maps by $\Lambda_{n,m}$. Picking a multiindex $\beta\in\mathbb N^n$, we observe that there are only finitely many  $\lambda\in\Lambda_{n,m}$ that solve the constraint $\sum_{i,\alpha}\lambda_{i,\alpha}\alpha=\beta$. Let us denote the set of such solutions by  $\Lambda_{n,m}(\beta)$. For each  $\lambda \in \Lambda_{n,m}(\beta)$ only finitely many $\lambda_{i,\alpha}$ are nonzero, and hence
  \[\lambda!:=\prod_{i,\alpha}\lambda_{i,\alpha}!\] is well-defined. For fixed $\alpha$, $\lambda_{\alpha}=(\lambda_{1,\alpha},\dots,\lambda_{m,\alpha})$ is viewed as a multiindex in $\mathbb N^m$. Summing these terms up we end up with $\sum_\alpha \lambda _\alpha\in\mathbb N^m$.

\begin{theorem}[Fa\`a di Bruno -- multiindex version, cf.~\cite{MichTG}]\label{FdBmulti} With the notation from above we have
  \begin{align*}
    \partial^\beta (f\circ \varphi) = \sum_{\lambda \in \Lambda_{n,m}(\beta)} \frac{\beta !}{\lambda !} \quad \left(\partial ^{\sum_{\alpha}\lambda_\alpha} f\right)\circ \varphi \quad \prod_{\alpha}\frac{\left(\partial^\alpha \varphi\right)^{\lambda_\alpha}}{(\alpha!)^{\sum_i \lambda_{i,\alpha}}}
  \end{align*}
\end{theorem}
\begin{proof} The proof can be achieved by combining the Taylor expansion with the following formula, which is a corollary of the multinomial theorem. For $b\in \mathbb N$ and a formal series $\sum_{\alpha\in \mathbb N^n} a_\alpha \bs x^\alpha\in \mathbb R [\![\bs x]\!]= \mathbb R[\![x_1,\dots,x_n]\!]$ we have
\[\left(\sum_{\alpha\in \mathbb N^n} a_\alpha \bs x^\alpha\right)^b=\sum_{\nu\in \mathbb N^{\mathbb N^n}\colon \sum_{\alpha\in \mathbb N^n}\nu_\alpha=b}b!\left(\prod_\alpha \frac{a_\alpha^{\nu_\alpha}}{\nu_\alpha !}\right)\: \bs x^{\sum_{\alpha}\nu_\alpha \alpha} \: .\]   
\end{proof}
%%% Local Variables:
%%% mode: latex
%%% TeX-master: "HerPfl.InvariantWhitneyFunctions"
%%% End: