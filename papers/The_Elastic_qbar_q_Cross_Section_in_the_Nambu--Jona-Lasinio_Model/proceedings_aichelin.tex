\documentclass[a4paper]{jpconf}
\usepackage{graphicx}
\usepackage{amsfonts}
%\usepackage{amsmath}
%\usepackage{amssymb}
\usepackage{epsfig}
\usepackage{color}
\usepackage{multirow}
%\usepackage{enumerate}

\usepackage{lipsum}					%Needed to create dummy text
\usepackage[centertags]{amsmath}			%Writes maths centred down
\usepackage{stmaryrd}					%New AMS symbols
\usepackage{amssymb}					%Calls AMS symbols
\usepackage{amsthm}					%Calls AMS theorem environment
\usepackage{slashed}
%\usepackage{newlfont}					%Helpful package for fonts and symbols
\usepackage{layouts}					%Layout diagrams
\usepackage{graphicx}					%Calls figure environment
\usepackage{subfig}
\usepackage{longtable,rotating}			%Long tab environments including rotation. 
%\usepackage[applemac]{inputenc}			%Needed to encode non-english characters 
									%directly for mac
\usepackage{colortbl}					%Makes coloured tables
%\usepackage{wasysym}					%More math symbols
%\usepackage{mathrsfs}					%Even more math symbols
\usepackage{float}						%Helps to place figures, tables, etc. 
\usepackage{verbatim}					%Permits pre-formated text insertion
\usepackage{upgreek }					%Calls other kind of greek alphabet
\usepackage{textgreek}
%\usepackage{latexsym}					%Extra symbols
\usepackage[square,numbers,
		     sort&compress]{natbib}		%Calls bibliography commands 
\usepackage{url}						%Supports url commands
%\usepackage{etex}						%eTeXÕs extended support for counters									
\usepackage[spanish,english,french]{babel}		%For languages characters and hyphenation
\usepackage{color}                    				%Creates coloured text and background
\usepackage[colorlinks=true,
		     allcolors=black]{hyperref}              %Creates hyperlinks in cross references
\usepackage{memhfixc}					%Must be used on memoir document 
									%class after hyperref
%\usepackage{enumerate}					%For enumeration counter
%\usepackage{footnote}					%For footnotes
\usepackage{microtype}					%Makes pdf look better.
\usepackage{rotfloat}					%For rotating and float environments as tables, 
									%figures, etc. 
\usepackage{alltt}						%LaTeX commands are not disabled in 
									%verbatim-like environment
\usepackage[version=0.96]{pgf}			%PGF/TikZ is a tandem of languages for producing vector graphics from a 
\usepackage{tikz}						%geometric/algebraic description.
\usetikzlibrary{arrows,shapes,snakes,
		       automata,backgrounds,
		       petri,topaths}				%To use diverse features from tikz	
%
%\usepackage{tikz-feynman}
%\usepackage{feynmf}
\usepackage{feynmp-auto}
\usepackage{import}
\usepackage{calc,soul,fourier}
\newcommand{\ar}{\arrowvert}
\newcommand{\ra}{\rangle}
\newcommand{\la}{\langle}
\newcommand{\da}{\dagger}
\newcommand{\ov}{\overline}
\newcommand{\cd}{\! \cdot \!}
\newcommand{\be}{\begin{equation}}
\newcommand{\ee}{\end{equation}}
\newcommand{\ba}{\begin{eqnarray}}
\newcommand{\ea}{\end{eqnarray}}
\newcommand{\pa}{\partial}
\newcommand{\unit}{\mathbb{I}}
\newcommand{\nn}{\nonumber}
\renewcommand{\slash}{ \not}
\newcommand{\pvec}{{\bf p}}
\newcommand{\kvec}{{\bf k}}
\newcommand{\red}{\textcolor{red} }
\newcommand{\dtilde}[1]{\frac{d^3 #1}{(2\pi)^3}}

\newcommand{\keywords}[1]{\par\noindent{\small{\bf Keywords:} #1}} %Defines keywords small section
\newcommand{\parcial}[2]{\frac{\partial#1}{\partial#2}}                             %Defines a partial operator
\newcommand{\vectorr}[1]{\mathbf{#1}}                                                        %Defines a bold vector
\newcommand{\vecol}[2]{\left(                                                                         %Defines a column vector
	\begin{array}{c} 
		\displaystyle#1 \\
		\displaystyle#2
	\end{array}\right)}
\newcommand{\mados}[4]{\left(                                                                       %Defines a 2x2 matrix
	\begin{array}{cc}
		\displaystyle#1 &\displaystyle #2 \\
		\displaystyle#3 & \displaystyle#4
	\end{array}\right)}
\newcommand{\pgftextcircled}[1]{                                                                    %Defines encircled text
    \setbox0=\hbox{#1}%
    \dimen0\wd0%
    \divide\dimen0 by 2%
    \begin{tikzpicture}[baseline=(a.base)]%
        \useasboundingbox (-\the\dimen0,0pt) rectangle (\the\dimen0,1pt);
        \node[circle,draw,outer sep=0pt,inner sep=0.1ex] (a) {#1};
    \end{tikzpicture}
}
\newcommand{\range}[1]{\textnormal{range }#1}                                             %Defines range operator
\newcommand{\innerp}[2]{\left\langle#1,#2\right\rangle}                                 %Defines inner product
\newcommand{\prom}[1]{\left\langle#1\right\rangle}                                         %Defines average operator
\newcommand{\tra}[1]{\textnormal{tra} \: #1}                                                       %Defines trace operator
\newcommand{\sign}[1]{\textnormal{sign\,}#1}                                                   %Defines sign operator
\newcommand{\sech}[1]{\textnormal{sech} #1}                                                  %Defines sech
\newcommand{\diag}[1]{\textnormal{diag} #1}                                                    %Defines diag operator
\newcommand{\arcsech}[1]{\textnormal{arcsech} #1}                                       %Defines arcsech
\newcommand{\arctanh}[1]{\textnormal{arctanh} #1}                                         %Defines arctanh
%Change tombstone symbol
\newcommand{\blackged}{\hfill$\blacksquare$}
\newcommand{\whiteged}{\hfill$\square$}
\renewcommand{\labelitemi}{$\arabic$}
\newcounter{proofcount}
\renewenvironment{proof}[1][\proofname.]{\par
 \ifnum \theproofcount>0 \pushQED{\whiteged} \else \pushQED{\blackged} \fi%
 \refstepcounter{proofcount}
 \normalfont 
 \trivlist
 \item[\hskip\labelsep
       \itshape
   {\bf\em #1}]\ignorespaces
}{%
 \addtocounter{proofcount}{-1}
 \popQED\endtrivlist
}
\renewcommand{\thefootnote}{\arabic{footnote}} 	
\begin{document}

\title{The Elastic $q\bar q$ Cross Section in the Nambu--Jona-Lasinio Model}

\author
{Rafael Chapelle$^1$, Joerg Aichelin$^{1,2}$ , Juan M. Torres-Rincon$^2$}
\address{$^1$ Subatech, UMR 6457, IN2P3/CNRS, Universit\'e de Nantes, \'Ecole de Mines de Nantes, 4 rue Alfred Kastler 44307,
Nantes, France}
\address{$^2$ Frankfurt Institute for Advances Studies. Johann Wolfgang Goethe University, Ruth-Moufang-Str. 1,
60438, Frankfurt am Main, Germany}


\begin{abstract}
We discuss the quark masses and the elastic  $q\bar q$ cross sections at finite chemical potential in the Nambu--Jona-Lasinio model. 
We comment the generic features of the cross sections as functions of the chemical potential, temperature and collision energy. Finally, we discuss their 
relevance in the construction of a relativistic transport model for heavy-ion collisions based on this effective Lagrangian.
\end{abstract}

\section{Introduction}
 The Nambu--Jona-Lasinio (NJL) model
has been extensively used in the context of strong interactions due to its ability to account for several key
phenomena of the Quantum Chromodynamics (QCD), like the spontaneous symmetry breaking (together with the generation of
Goldstone bosons) and its restoration at high temperatures and densities. This model works as an effective realization of 
QCD at low energies, and allows us performing studies in a much simpler way in the regime where QCD is too difficult
to solve, or computationally expensive like in lattice-QCD calculations. This fact we use here to derive elastic $q\bar q$ cross sections in
the non-perturbative region. These cross sections are needed if one wants to simulate the expansion of a plasma created in ultrarelativistic
heavy-ion collisions.

We consider the Lagrangian of the NJL model with (color neutral) pseudoscalar and scalar interactions (neglecting the vector and axial-vector vertices for simplicity)~\cite{Torres-Rincon},
\ba 
 {\cal L}_{NJL} &=& \sum_i \bar{\psi}_i (i \slashed{\partial}-m_{0i}+\mu_{i} \gamma_0) \psi_i \nn \\
&+& G \sum_{a} \sum_{ijkl} \left[ (\bar{\psi}_i \ i\gamma_5 \tau^{a}_{ij} \psi_j) \ 
(\bar{\psi}_k \ i \gamma_5 \tau^{a}_{kl} \psi_l)
+ (\bar{\psi}_i \tau^{a}_{ij} \psi_j) \ 
(\bar{\psi}_k  \tau^{a}_{kl} \psi_l) \right] \nn \\
& -&    K \det_{ij} \left[ \bar{\psi}_i \ ( \unit - \gamma_5 ) \psi_j \right] - K \det_{ij} \left[ \bar{\psi}_i \ ( \unit + \gamma_5 ) \psi_j \right]   
\label{eq:lagPNJL}
\ea
where the flavor indices $i,j,k,l=1,2,3$ and $\tau^{a}$ ($a=1,...,8$) being the $N_f=3$ flavor generators with
normalization 
\be \textrm{tr}_f \  (\tau^{a} \tau^{b}) = 2\delta^{ab}  \ , \ee
with $\textrm{tr}_f$ denoting the trace in flavor space.  In the Lagrangian~(\ref{eq:lagPNJL}) the bare quark masses are represented by $m_{0i}$ and their chemical potential by
$\mu_{i}$. The coupling constant for the scalar and pseudoscalar interaction $G$ is taken as
a free parameter (fixed e.g. by the pion mass in vacuum). The third term of Eq.~(\ref{eq:lagPNJL}) is the so-called 't Hooft Lagrangian, which mimics the effect of the axial $U(1)$ anomaly, accounting for
the physical splitting between the $\eta$ and the $\eta'$ meson masses. $K$ is an unknown coupling constant (fixed by the value
of $m_{\eta'}-m_{\eta}$) and $\unit$ is the identity matrix in Dirac space. As the NJL model is non-renormalizable, it also requires an ultraviolet regulator, which we introduce in the form of a cutoff $\Lambda$.

This Lagrangian has been widely used to study strongly interacting systems in the vacuum and at finite temperature. It contains 5 parameters. For all details we refer to the reviews~\cite{Vogl,Klevansky,Hatsuda,Alkofer,Buballa}.  


\section{Masses of $u$, $d$ and $s$ quarks as functions of temperature and chemical potential}
In the SU(3) version of the NJL Lagrangian the mass of a quark of flavor $i$ is given by 
\begin{equation}
m_i = m_{0i} - 4G \langle \overline{q}_i  q_i \rangle + 2 K \langle \overline{q}_j q_j \rangle \langle \overline{q}_k  q_k \rangle \ ,
\label{Eq:gap}
\end{equation}
where $m_{0i}$ is the bare mass,  $i \neq j \neq k$ are the $u$, $d$, $s$ quarks, $ \langle \overline{q}_i  q_i \rangle$ the scalar condensate which, with the thermal distribution function
\begin{equation}
f^\pm_i(E_i,T,\mu_i) = \frac{1}{1+\exp \left( \frac{E_i \mp \mu_i}{T} \right)} \ ,
\end{equation}
is given by 
\begin{equation}
\langle \bar{q}_i q_i \rangle= - 2 N_c  \int_0^\Lambda \frac{m_i}{\sqrt{p^2+m_i^2}}(1-f_i^{+}-f_i^{-}) \frac{\mathrm{d}^3p}{(2 \pi)^3} \ .
\end{equation}
\begin{figure}[htp]%
\centering
\includegraphics[scale=0.5]{mq}
\includegraphics[scale=0.5]{ms}%
\caption{\label{fig:masses}The mass of $u$, $d$ (left) and the $s$ (right) quark as functions of the temperature and the chemical potential.}
\end{figure}
 In the present calculation we employ the parameters of Table~\ref{tab:para}, determined by vacuum meson masses and decay constants.
\begin{center}
\begin{table}[h]
\caption{\label{tab:para}Parameters used in the calculation and the critical chemical potential $\mu_{crit}$, where the chiral restoration occurs as a first order phase transition.}
\centering
\begin{tabular}{|c | c | c | c| c | c |}
%\br
\hline
$\Lambda$ & $G$ & $K$ & $m_{0u}$ & $m_{0s}$ & $\mu_{crit}$ \\ \hline
569 MeV & $2.3/\Lambda^2$  & $11/\Lambda^5$ & 5.5 MeV & 134 MeV & 338 MeV\\ \hline
\end{tabular}
\end{table}
\end{center}
For $\mu = 0$ ($\mu$ denotes the light chemical potential, $\mu_u=\mu_d$) we observe for all quarks a smooth transition of the mass as a function of the temperature.  This behavior continues for $u$ and $d$ quarks
along the transition line between the quark plasma and hadrons until $\mu_{crit}$ is reached. Then the crossover becomes a first order phase transition and we see a sudden change of the masses. For the $s$
quark the situation is more complicated. We see a first sudden but continuous change at the phase transition for the $u$ and $d$ quarks because both are related (Eq.~\ref{Eq:gap}) and then a second first order transition when the $s$
quark mass become discontinuous. 

\section{The $q\bar q$ cross sections}
The  $q\bar q$ cross section can be obtained in the standard way from the Lagrangian~\cite{Rehberg}. The Feynman diagrams for the two channels
\let\textcircled=\pgftextcircled
\begin{figure}[htp]%
\centering
\begin{minipage}{0.49\textwidth}
\centering
\begin{fmffile}{simple1}
%\begin{fmffile}
\begin{fmfgraph*}(140,90)
\fmfleft{i1,i2}
\fmfright{o1,o2}
\fmflabel{$p_2,m_2$}{i1}
\fmflabel{$p_1,m_1$}{i2}
\fmflabel{$p_4,m_4$}{o1}
\fmflabel{$p_3,m_3$}{o2}
\fmf{fermion,label}{i1,v1}
\fmf{fermion,label}{v1,o1}
\fmf{dbl_plain,label=$p=(p_1 - p_3)$,l.s=left}{v1,v2}
\fmf{fermion}{o2,v2}
\fmf{fermion}{v2,i2}
\end{fmfgraph*}
\end{fmffile}
%\caption{t-channel}
\end{minipage}
\hfill
\begin{minipage}{0.49\textwidth}
\centering
\begin{fmffile}{simple3}
\begin{fmfgraph*}(140,90)
\fmfleft{i1,i2}
\fmfright{o1,o2}
\fmflabel{$p_2,m_2$}{i1}
\fmflabel{$p_1,m_1$}{i2}
\fmflabel{$p_4,m_4$}{o1}
\fmflabel{$p_3,m_3$}{o2}
\fmf{fermion}{i1,v1,i2}
\fmf{dbl_plain,label=$p=(p_1+p_2)$}{v1,v2}
\fmf{fermion}{o1,v2,o2}
\end{fmfgraph*}
\end{fmffile}
%\caption{s-channel}
\end{minipage}
\\[1.0cm]
\caption{$t-$ and $s-$channel Feynman diagrams for elastic $q\bar q$ collisions.}
\label{fig:Diagfqq}
\end{figure}
lead to the following matrix elements, 
\begin{equation}
\begin{split}
-i \mathcal{M}_t &= \delta_{c1,c3} \delta_{c2,c4} \bar{u}(p_3) T v(p_1) [i \mathcal{D}_t^S(p_1 - p_3)]\bar v(p_2) T  u(p_4)\\
& + \delta_{c1,c3} \delta_{c2,c4} \bar{u}(p_3)(i \gamma_5 T) v(p_1) [i \mathcal{D}_t^P(p_1 - p_3)]\bar{v}(p_2) (i \gamma_5 T) u(p_4) \ ,
\end{split}
\label{Eq:4.1}
\end{equation}
\begin{equation}
\begin{split}
-i \mathcal{M}_s &= \delta_{c1,c2} \delta_{c3,c4} \bar{v}(p_2) T u(p_1) [i \mathcal{D}_s^S(p_1 + p_2)]\bar{u}(p_3) T v(p_4)\\
& + \delta_{c1,c2} \delta_{c3,c4} \bar{v}(p_2)(i \gamma_5 T) u(p_1) [i \mathcal{D}_s^P(p_1 - p_4)]\bar{u}(p_3) (i \gamma_5 T) v(p_4) \ .
\end{split}
\label{Eq:4.3}
\end{equation}
\begin{figure}[htp]%
\centering
\begin{minipage}{0.49\textwidth}
\centering
\includegraphics[scale=0.5]{su3CrossSecUUbarNJL_T_s_mu_0final}
\end{minipage}
\hfill
\begin{minipage}{0.49\textwidth}
\includegraphics[scale=0.5]{su3CrossSecUUbarNJL_T_s_mu_0_1final}
\end{minipage}
\hfill
\begin{minipage}{0.49\textwidth}
\includegraphics[scale=0.5]{su3CrossSecUUbarNJL_T_s_mu_0_2final}
\end{minipage}
\hfill
\begin{minipage}{0.49\textwidth}
\includegraphics[scale=0.5]{su3CrossSecUUbarNJL_T_s_mu_0_3final}
\end{minipage}
\hfill
\begin{minipage}{0.49\textwidth}
\includegraphics[scale=0.5]{su3CrossSecUUbarNJL_T_s_mu_0_35final}
\end{minipage}
\begin{minipage}{0.49\textwidth}
\includegraphics[scale=0.5]{su3CrossSecUUbarNJL_T_s_mu_0_5final}
\end{minipage}
\caption{\label{fig:uubar}$u\bar u \rightarrow u\bar u$ cross section as a function of the temperature and of the center of mass energy for different chemical potentials $\mu$.}
\end{figure}
 $u$, $\bar{u}$ and $v$, $\bar{v}$ are the Dirac spinors. $\delta_{c3,c4}$ impose the color conservation. The symbolic expression 
 $T \mathcal{D}_i^{S/P} T$ stands for the exchange of all possible scalar and pseudoscalar mesons, respectively. We limit us in this calculation to the exchange of color neutral mesons and the $s$-channel is taken
in first order in $N_c$. 
The square of the matrix element, averaged over spin and color in  the entrance and summed over in the exit channel,
\begin{equation}
\frac{1}{4 N_c^2}  \sum_{s,c} |\mathcal{M}_{total}|^2 
\end{equation}
 gives the cross section
\begin{equation}
\frac{\mathrm{d}\sigma}{\mathrm{d}t} = \frac{1}{16 \pi [s-(m_1+m_2)^2][s-(m_1-m_2)^2]}\frac{1}{4N_c^2} \sum_{s,c} |\mathcal{M}_{total}|^2.
\end{equation} 
In a thermal heat bath some of the exit states are partially blocked. Introducing the Pauli blocking factors we obtain
\begin{equation}
\sigma = \int \frac{\mathrm{d}\sigma}{\mathrm{d}t} (1-f^{\pm}(E_3,T,\mu_3))(1-f^{\pm}(E_4,T,\mu_4)) \mathrm{d}t .
\end{equation}
The integration limits for
 $t$ are $-(s- \sum_{i} m_i^2)$ and  0.
In order to calculate the transition amplitudes one has to know the mesons which can be exchanged and the corresponding flavor factors.
The exchanged mesons are either scalars $\sigma_\pi$, $\sigma_K$, $\sigma$ and $\sigma'$ or pseudoscalars  $\pi$ ,$K$, $\eta$ and $\eta'$. They are displayed in Table~\ref{tab:qqbar}.
The Gell-Mann matrices for the different flavors are  
\begin{equation}
\begin{cases}
\pi^0, \lambda_3 \\
\pi^\pm, \frac{1}{\sqrt{2}}(\lambda_1 \pm i \lambda_2) \\
K^0,\bar{K}^0, \frac{1}{\sqrt{2}}(\lambda_6 \pm i \lambda_7) \\
K^\pm,\frac{1}{\sqrt{2}}(\lambda_4 \pm i \lambda_5).
\end{cases} 
\end{equation}
To calculate which mesons can be exchanged one has to calculate the corresponding flavor matrices. As an example we calculate $ud \rightarrow ud$ with the exchange of a $\pi^{0}$ in the $t$-channel. Here one finds
\begin{equation} 
\begin{split}
\bar{u} \lambda_3 u \times \bar{d} \lambda_3 d= 
\begin{pmatrix} 1 & 0  & 0 \end{pmatrix} \begin{pmatrix} 1 & 0 & 0 \\ 0 & -1 & 0 \\ 0 & 0 & 0 \end{pmatrix} \begin{pmatrix} 1 \\ 0\\ 0 \end{pmatrix} \times
\begin{pmatrix} 0 & 1  & 0 \end{pmatrix} \begin{pmatrix} 1 & 0 & 0 \\ 0 & -1 & 0 \\ 0 & 0 & 0 \end{pmatrix} \begin{pmatrix} 0 \\ 1\\ 0 \end{pmatrix} = -1.
\end{split}
\end{equation}
\begin{figure}[htp]%
\centering
\begin{minipage}{0.49\textwidth}
\centering
\includegraphics[scale=0.5]{su3CrossSecUDbarNJL_T_s_mu_0final}
\end{minipage}
\hfill
\begin{minipage}{0.49\textwidth}
\includegraphics[scale=0.5]{su3CrossSecUDbarNJL_T_s_mu_0_1final}
\end{minipage}
\hfill
\begin{minipage}{0.49\textwidth}
\includegraphics[scale=0.5]{su3CrossSecUDbarNJL_T_s_mu_0_2final}
\end{minipage}
\hfill
\begin{minipage}{0.49\textwidth}
\includegraphics[scale=0.5]{su3CrossSecUDbarNJL_T_s_mu_0_3final}
\end{minipage}
\hfill
\begin{minipage}{0.49\textwidth}
\includegraphics[scale=0.5]{su3CrossSecUDbarNJL_T_s_mu_0_5final}
\end{minipage}
\caption{\label{fig:udbar}Cross section for $u\bar{d} \rightarrow u\bar{d}$ as a function of the temperature T and the center of mass energy  $\sqrt{s}$ for different chemical potentials $\mu$.}
\end{figure}
Hence a $\pi_0$ can be exchanged, whereas for a $\pi^+$ we find zero. If we apply this for all possible combinations we find the mesons of Table~\ref{tab:qqbar}.
\begin{table}
\centering
\caption{\label{tab:qqbar}The mesons which can be exchanged in the different channels for  $q\bar{q}\rightarrow q\bar{q}$}
\begin{tabular}{|c | c | c |}
\hline
Process & Mesons exchanged in the s-channel & Mesons exchanged in the  t-channel \\ \hline
$u\bar{u} \rightarrow u\bar{u}$ &$ \pi, \eta, \eta',\sigma_{\pi}$,$ \sigma, \sigma' $&$ \pi, \eta, \eta',\sigma_{\pi}, \sigma, \sigma' $\\ \hline
$u\bar{u} \rightarrow u\bar{d}$ & $\pi, \eta, \eta',\sigma_{\pi}$, $\sigma, \sigma'$ & $ \pi, \sigma_{\pi} $\\ \hline
$u\bar{d} \rightarrow u\bar{d}$ & $   \pi, \sigma_{\pi} $ & $ \pi, \eta, \eta',\sigma_{\pi}$,$ \sigma, \sigma'  $ \\ \hline 
$u\bar{s} \rightarrow u\bar{s}$ &$ K, \sigma_K$ & $\eta, \eta',  \sigma, \sigma' $ \\ \hline
$u\bar{u} \rightarrow s\bar{s}$ & $\eta, \eta',  \sigma, \sigma'  $ & $K, \sigma_K $ \\ \hline
$s\bar{s} \rightarrow u\bar{u}$ & $ \eta, \eta',  \sigma, \sigma' $ & $K, \sigma_K $ \\ \hline
$s\bar{s} \rightarrow s\bar{s}$ &$\eta, \eta',  \sigma, \sigma' $ & $\eta, \eta',  \sigma, \sigma'$ \\ \hline
\end{tabular}
\end{table}

\section{Results}
The $q\bar q$ cross section has a quite different behavior as compared to the $qq$ cross section due to the $s$-channel contribution. The $qq$ cross section is always $\le $20 mb. In the $s$ channel it happens that
the incoming $q\bar q$ pair is in resonance with the meson which it produces. In this case the denominator of the meson propagator becomes  small  and hence the matrix element large. This resonance between the
incoming $q\bar q$ pair appears close to the temperature where the mass of the meson is close to the sum of the masses of the two valence quarks. Therefore the peak of the cross section moves to lower
temperatures when the chemical potential increases. 

\subsection{$u\bar{d} \rightarrow u\bar{d}$ and $u\bar{u} \rightarrow u\bar{u}$ }
Close to  the critical chemical potential, $\mu_{crit}$, (see Table~\ref{tab:para}) the cross section becomes maximal being around 100  mb for the $u \bar u$ channel at the grid points which we calculated. At larger
chemical potentials the cross sections are smaller again and arrive at a maximal value of 25 mb at $\mu = 0.5$ MeV (see Fig.\ref{fig:uubar}). For even higher chemical potentials the cross section is reduced to a couple of millibarn
 because then a transition between the plasma and the hadronic world does not exist anymore in the NJL approach. There we are for all temperatures in the deconfined phase.    

If one compares the $u\bar u$ and $u\bar d$ elastic cross sections, Figs. \ref{fig:uubar} and  \ref{fig:udbar}, one sees that the latter is about two to four times larger. This is a consequence of the flavor factor 
which doubles the $s$-channel contribution and gives a relative minus sign to the $t$-channel contribution. The form of the cross sections are rather similar, only at large values of $\sqrt{s}$ the fact that the $\eta$ and
$\eta'$ mesons are not allowed in the $s$ channel of $u\bar d$ makes the $u\bar u$ cross section relatively larger.  
\begin{figure}[htp]%
\centering
\begin{minipage}{0.49\textwidth}
\centering
\includegraphics[scale=0.5]{CrossSecUSbarNJL_T_s_mu_0final}
\end{minipage}
\hfill
\begin{minipage}{0.49\textwidth}
\includegraphics[scale=0.5]{CrossSecUSbarNJL_T_s_mu_0_1final}
\end{minipage}
\hfill
\begin{minipage}{0.49\textwidth}
\includegraphics[scale=0.5]{CrossSecUSbarNJL_T_s_mu_0_2final}
\end{minipage}
\hfill
\begin{minipage}{0.49\textwidth}
\includegraphics[scale=0.5]{CrossSecUSbarNJL_T_s_mu_0_3final}
\end{minipage}
\hfill
\begin{minipage}{0.49\textwidth}
\includegraphics[scale=0.5]{CrossSecUSbarNJL_T_s_mu_0_5final}
\end{minipage}
\hfill
\caption{\label{fig:usbar}Cross section for $u\bar{s} \rightarrow u\bar{s}$ as a function of the temperature $T$ and the center of mass energy  $\sqrt{s}$ for different chemical potentials $\mu$.}
\end{figure}
\subsection{$u\bar{s} \rightarrow u\bar{s}$ }
The  $u\bar{s}$ cross section is displayed in Fig.~\ref{fig:usbar}. As compared to the other cross sections the maximum of the cross section is obtained at a higher temperature. This is a consequence that now in the
$s$ channel strange mesons are exchanged which have a mass well above the mass of the quarks in the entrance channel. Therefore more energetic particles are needed to be resonant with the exchanged meson.

\section{Conclusions}
This calculation of the cross sections at finite $\mu$ is the first step towards a transport theory based on the NJL Lagrangian, for finite chemical potentials, extending the work of Ref.~\cite{Marty1,Marty2}. We see that also at a finite chemical potential close to the transition between the  partonic and hadronic world the elastic cross sections become very large and therefore are very effective to equilibrate the system when it has expanded to the density where the phase transition takes place. Before the cross sections are too low to keep a local equilibrium. Consequently, for finite chemical potentials we expect the same generic behavior which we have observed at zero chemical potential, even if the numerical values of the cross sections differ in details. Therefore the NJL transport approach may be extended to a finite chemical potential to study how the change of the structure of the phase transition is seen in the observables.  
\section*{Acknowledgement}
This work has been funded by the program TOGETHER from R\'egion Pays de la Loire and EU Integrated Infrastructure Initiative HadronPhysics3 Project under Grant Agreement
n. 283286. JMTR thanks funding from a Helmholtz Young Investigator Group VH-NG-822 from the Helmholtz Association and GSI, and the Project FPA2013-43425-P from
Ministerio de Ciencia e Innovaci\'on (Spain).

\section*{References}
\medskip
%\numrefs{99}
%\begin{thebibliography}{9}
\begin{enumerate}
%\begin{enumerate}[label=\arabic* --)]
%\begin{enumerate}[{\arabic*}]
%\bibitem{Torres-Rincon:2015rma} Torres-Rincon J M,  Sintes B  and Aichelin J 2015 {\it Phys.\ Rev. }  {\bf C91}  065206 
\bibitem{Torres-Rincon} Torres-Rincon J M,  Sintes B  and Aichelin J 2015 {\it Phys.\ Rev. }  {\bf C91}  065206 
\bibitem{Vogl}    Vogl U and Weise W 1991 {\it  Prog.\ Part.\ Nucl.\ Phys.\ } {\bf 27} 195 
\bibitem{Klevansky}     Klevansky S P 1992 {\it  Rev.\ Mod.\ Phys.} \  {\bf 64}  649 
\bibitem{Hatsuda}  Hatsuda  T and Kunihiro T 1994  {\it Phys.\ Rept.}\  {\bf 247}  221 
\bibitem{Alkofer}   Alkofer R and Reinhardt H 1995 Chiral quark dynamics Berlin, Germany: Springer  (Lecture notes in physics)
\bibitem{Buballa}   Buballa M 2005 {\it  Phys.\ Rept.\  } {\bf 407} 205 
\bibitem{Rehberg}  Rehberg P, Klevansky S P and Hufner J 1996 {\it   Nucl.\ Phys.} A {\bf 608}, 356 
\bibitem{Marty1}   Marty R and Aichelin J 2013 {\it  Phys.\ Rev.} C {\bf 87}  034912 
\bibitem{Marty2}   Marty R, Bratkovskaya E, Cassing W and Aichelin J 2015 {\it  Phys.\ Rev.}  {\bf C92} 015201 


\end{enumerate}
%\endnumrefs
%\end{\thebibliography}
\smallskip
\end{document}
%\bibitem{Torres-Rincon:2015rma} Torres-Rincon J M,  Sintes B  and Aichelin J 2015 {\it Phys.\ Rev. }  {\bf C91}  065206 
\bibitem{Torres-Rincon} Torres-Rincon J M,  Sintes B  and Aichelin J 2015 {\it Phys.\ Rev. }  {\bf C91}  065206 
\bibitem{Vogl:1991qt}    Vogl U and Weise W 1991 {\it  Prog.\ Part.\ Nucl.\ Phys.\ } {\bf 27} 195 
\bibitem{Klevansky:1992qe}     Klevansky S P 1992 {\it  Rev.\ Mod.\ Phys.} \  {\bf 64}  649 
\bibitem{Hatsuda:1994pi}  Hatsuda  T and Kunihiro T 1994  {\it Phys.\ Rept.}\  {\bf 247}  221 
\bibitem{Alkofer:1995mv}   Alkofer R and Reinhardt H 1995 Chiral quark dynamics Berlin, Germany: Springer  (Lecture notes in physics)
\bibitem{Buballa:2003qv}   Buballa M 2005 {\it  Phys.\ Rept.\  } {\bf 407} 205 


\item Kurata M 1982 {\it Numerical Analysis for Semiconductor Devices} (Lexington, MA: Heath)
\item Selberherr S 1984 {\it Analysis and Simulation of Semiconductor Devices} (Berlin: Springer)
\item Sze S M 1969 {\it Physics of Semiconductor Devices} (New York: Wiley-Interscience)
\item Dorman L I 1975 {\it Variations of Galactic Cosmic Rays} (Moscow: Moscow State University Press) p 103
\item Caplar R and Kulisic P 1973 {\it Proc. Int. Conf. on Nuclear Physics (Munich)} vol 1 (Amsterdam: 	North-Holland/American Elsevier) p 517
\item Cheng G X 2001 {\it Raman and Brillouin Scattering-Principles and Applications} (Beijing: Scientific) 
\item Szytula A and Leciejewicz J 1989 {\it Handbook on the Physics and Chemistry of Rare Earths} vol 12, ed K A Gschneidner Jr and L Erwin (Amsterdam: Elsevier) p 133
\item Kuhn T 1998 {\it Density matrix theory of coherent ultrafast dynamics Theory of Transport Properties of Semiconductor Nanostructures} (Electronic Materials vol 4) ed E Sch\"oll (London: Chapman and Hall) chapter 6 pp 173--214


\noindent which would be obtained by typing

\begin{verbatim}
\begin{\thebibliography}{9}
\item Strite S and Morkoc H 1992 {\it J. Vac. Sci. Technol.} B {\bf 10} 1237 
\item Jain S C, Willander M, Narayan J and van Overstraeten R 2000 
{\it J. Appl. Phys}. {\bf 87} 965 
\item Nakamura S, Senoh M, Nagahama S, Iwase N, Yamada T, Matsushita T, Kiyoku H 
and 	Sugimoto Y 1996 {\it Japan. J. Appl. Phys.} {\bf 35} L74 
\item Akasaki I, Sota S, Sakai H, Tanaka T, Koike M and Amano H 1996 
{\it Electron. Lett.} {\bf 32} 1105 
\item O'Leary S K, Foutz B E, Shur M S, Bhapkar U V and Eastman L F 1998 
{\it J. Appl. Phys.} {\bf 83} 826 
\item Jenkins D W and Dow J D 1989 {\it Phys. Rev.} B {\bf 39} 3317 

\end{verbatim}

\begin{center}
\begin{table}[h]
\centering
\caption{\label{jfonts}Font styles for a reference to a journal article.} 
\begin{tabular}{@{}l*{15}{l}}
\br
Element&Style\\
\mr
Authors&Roman type\\
Date&Roman type\\
Article title (optional)&Roman type\\
Journal title&Italic type\\
Volume number&Bold type\\
Page numbers&Roman type\\
\br
\end{tabular}
\end{table}
\end{center}

\subsection{References to \jpcs\ articles}
Each conference proceeding published in \jpcs\ will be a separate {\it volume}; 
references should follow the style for conventional printed journals. For example:\vspace{6pt}
\numrefs{1}
\item Douglas G 2004 \textit{J. Phys.: Conf. Series} \textbf{1} 23--36
\endnumrefs

%%%%%%%%%%%%%%%%%%%%%%%%%%%%%%%%%%
\subsection{References to preprints}
For preprints there are two distinct cases:
\renewcommand{\theenumi}{\arabic{enumi}}
\begin{enumerate}
\item Where the article has been published in a journal and the preprint is supplementary reference information. In this case it should be presented as:
\medskip
\numrefs{1}
\item Kunze K 2003 T-duality and Penrose limits of spatially homogeneous and inhomogeneous cosmologies {\it Phys. Rev.} D {\bf 68} 063517 ({\it Preprint} gr-qc/0303038)
\endnumrefs
\item Where the only reference available is the preprint. In this case it should be presented as
\medskip
\numrefs{1}
\item Milson R, Coley A, Pravda V and Pravdova A 2004 Alignment and algebraically special tensors {\it Preprint} gr-qc/0401010
\endnumrefs
\end{enumerate}

\subsection{References to electronic-only journals}
In general article numbers are given, and no page ranges, as most electronic-only journals start each article on page 1.

\begin{itemize} 
\item For {\it New Journal of Physics} (article number may have from one to three digits)
\numrefs{1}
\item Fischer R 2004 Bayesian group analysis of plasma-enhanced chemical vapour deposition data {\it New. J. Phys.} {\bf 6} 25 
\endnumrefs
\item For SISSA journals the volume is divided into monthly issues and these form part of the article number

\numrefs{2}
\item Horowitz G T and Maldacena J 2004 The black hole final state {\it J. High Energy Phys.}  	JHEP02(2004)008
\item Bentivegna E, Bonanno A and Reuter M 2004 Confronting the IR fixed point cosmology 	with 	high-redshift observations {\it J. Cosmol. Astropart. Phys.} JCAP01(2004)001  
\endnumrefs
\end{itemize} 

\subsection{References to books, conference proceedings and reports}
References to books, proceedings and reports are similar to journal references, but have 
only two changes of font (see table~\ref{book}). 

\begin{table}
\centering
\caption{\label{book}Font styles for references to books, conference proceedings and reports.}
\begin{tabular}{@{}l*{15}{l}}
\br
Element&Style\\
\mr
Authors&Roman type\\
Date&Roman type\\
Book title (optional)&Italic type\\
Editors&Roman type\\
Place (city, town etc) of publication&Roman type\\
Publisher&Roman type\\
Volume&Roman type\\
Page numbers&Roman type\\
\br
\end{tabular}
\end{table}

Points to note are:
\medskip
\begin{itemize}
\item Book titles are in italic and should be spelt out in full with initial capital letters for all except minor words. Words such as Proceedings, Symposium, International, Conference, Second, etc should be abbreviated to {\it Proc.}, {\it Symp.}, {\it Int.}, {\it Conf.}, {\it 2nd}, respectively, but the rest of the title should be given in full, followed by the date of the conference and the town or city where the conference was held. For Laboratory Reports the Laboratory should be spelt out wherever possible, e.g. {\it Argonne National Laboratory Report}.
\item The volume number, for example vol 2, should be followed by the editors, if any, in a form such as `ed A J Smith and P R Jones'. Use {\it et al} if there are more than two editors. Next comes the town of publication and publisher, within brackets and separated by a colon, and finally the page numbers preceded by p if only one number is given or pp if both the initial and final numbers are given.
\end{itemize}

Examples taken from published papers:
\medskip



\section{Tables and table captions}
Tables should be numbered serially and referred to in the text 
by number (table 1, etc, {\bf rather than} tab. 1). Each table should be a float and be positioned within the text at the most convenient place near to where it is first mentioned in the text. It should have an 
explanatory caption which should be as concise as possible. 

\subsection{The basic table format}
The standard form for a table is:
\begin{verbatim}
\begin{table}
\caption{\label{label}Table caption.}
\begin{center}
\begin{tabular}{llll}
\br
Head 1&Head 2&Head 3&Head 4\\
\mr
1.1&1.2&1.3&1.4\\
2.1&2.2&2.3&2.4\\
\br
\end{tabular}
\end{center}
\end{table}
\end{verbatim}

The above code produces table~\ref{ex}.

\begin{table}[h]
\caption{\label{ex}Table caption.}
\begin{center}
\begin{tabular}{llll}
\br
Head 1&Head 2&Head 3&Head 4\\
\mr
1.1&1.2&1.3&1.4\\
2.1&2.2&2.3&2.4\\
\br
\end{tabular}
\end{center}
\end{table}

Points to note are:
\medskip
\begin{enumerate}
\item The caption comes before the table.
\item The normal style is for tables to be centred in the same way as
equations. This is accomplished
by using \verb"\begin{center}" \dots\ \verb"\end{center}".

\item The default alignment of columns should be aligned left.

\item Tables should have only horizontal rules and no vertical ones. The rules at
the top and bottom are thicker than internal rules and are set with
\verb"\br" (bold rule). 
The rule separating the headings from the entries is set with
\verb"\mr" (medium rule). These commands do not need a following double backslash.

\item Numbers in columns should be aligned as appropriate, usually on the decimal point;
to help do this a control sequence \verb"\lineup" has been defined 
which sets \verb"\0" equal to a space the size of a digit, \verb"\m"
to be a space the width of a minus sign, and \verb"\-" to be a left
overlapping minus sign. \verb"\-" is for use in text mode while the other
two commands may be used in maths or text.
(\verb"\lineup" should only be used within a table
environment after the caption so that \verb"\-" has its normal meaning
elsewhere.) See table~\ref{tabone} for an example of a table where
\verb"\lineup" has been used.
\end{enumerate}

\begin{table}[h]
\caption{\label{tabone}A simple example produced using the standard table commands 
and $\backslash${\tt lineup} to assist in aligning columns on the 
decimal point. The width of the 
table and rules is set automatically by the 
preamble.} 

\begin{center}
\lineup
\begin{tabular}{*{7}{l}}
\br                              
$\0\0A$&$B$&$C$&\m$D$&\m$E$&$F$&$\0G$\cr 
\mr
\0\023.5&60  &0.53&$-20.2$&$-0.22$ &\01.7&\014.5\cr
\0\039.7&\-60&0.74&$-51.9$&$-0.208$&47.2 &146\cr 
\0123.7 &\00 &0.75&$-57.2$&\m---   &---  &---\cr 
3241.56 &60  &0.60&$-48.1$&$-0.29$ &41   &\015\cr 
\br
\end{tabular}
\end{center}
\end{table}
 
\section{Figures and figure captions}
Figures must be included in the source code of an article at the appropriate place in the text not grouped together at the end. 

Each figure should have a brief caption describing it and, if 
necessary, interpreting the various lines and symbols on the figure. 
As much lettering as possible should be removed from the figure itself and 
included in the caption. If a figure has parts, these should be 
labelled ($a$), ($b$), ($c$), etc. 
\Tref{blobs} gives the definitions for describing symbols and lines often
used within figure captions (more symbols are available
when using the optional packages loading the AMS extension fonts).

\begin{table}[h]
\caption{\label{blobs}Control sequences to describe lines and symbols in figure 
captions.}
\begin{center}
\begin{tabular}{lllll}
\br
Control sequence&Output&&Control sequence&Output\\
\mr
\verb"\dotted"&\dotted        &&\verb"\opencircle"&\opencircle\\
\verb"\dashed"&\dashed        &&\verb"\opentriangle"&\opentriangle\\
\verb"\broken"&\broken&&\verb"\opentriangledown"&\opentriangledown\\
\verb"\longbroken"&\longbroken&&\verb"\fullsquare"&\fullsquare\\
\verb"\chain"&\chain          &&\verb"\opensquare"&\opensquare\\
\verb"\dashddot"&\dashddot    &&\verb"\fullcircle"&\fullcircle\\
\verb"\full"&\full            &&\verb"\opendiamond"&\opendiamond\\
\br
\end{tabular}
\end{center}
\end{table}


Authors should try and use the space allocated to them as economically as possible. At times it may be convenient to put two figures side by side or the caption at the side of a figure. To put figures side by side, within a figure environment, put each figure and its caption into a minipage with an appropriate width (e.g. 3in or 18pc if the figures are of equal size) and then separate the figures slightly by adding some horizontal space between the two minipages (e.g. \verb"\hspace{.2in}" or \verb"\hspace{1.5pc}". To get the caption at the side of the figure add the small horizontal space after the \verb"\includegraphics" command and then put the \verb"\caption" within a minipage of the appropriate width aligned bottom, i.e. \verb"\begin{minipage}[b]{3in}" etc (see code in this file used to generate figures 1--3).

Note that it may be necessary to adjust the size of the figures (using optional arguments to \verb"\includegraphics", for instance \verb"[width=3in]") to get you article to fit within your page allowance or to obtain good page breaks.

\begin{figure}[h]
\begin{minipage}{14pc}
\includegraphics[width=14pc]{name.eps}
\caption{\label{label}Figure caption for first of two sided figures.}
\end{minipage}\hspace{2pc}%
\begin{minipage}{14pc}
\includegraphics[width=14pc]{name.eps}
\caption{\label{label}Figure caption for second of two sided figures.}
\end{minipage} 
\end{figure}

\begin{figure}[h]
\includegraphics[width=14pc]{name.eps}\hspace{2pc}%
\begin{minipage}[b]{14pc}\caption{\label{label}Figure caption for a narrow figure where the caption is put at the side of the figure.}
\end{minipage}
\end{figure}

Using the graphicx package figures can be included using code such as:
\begin{verbatim}
\begin{figure}
\begin{center}
\includegraphics{file.eps}
\end{center}
\caption{\label{label}Figure caption}
\end{figure}
\end{verbatim}
\end{document}
\section*{References}
\begin{thebibliography}{9}
\bibitem{iopartnum} IOP Publishing is to grateful Mark A Caprio, Center for Theoretical Physics, Yale University, for permission to include the {\tt iopart-num} \BibTeX package (version 2.0, December 21, 2006) with  this documentation. Updates and new releases of {\tt iopart-num} can be found on \verb"www.ctan.org" (CTAN). 
\end{thebibliography}

\end{document}



\section{The title, authors, addresses and abstract} 
The code for setting the title page information is slightly different from
the normal default in \LaTeX\ but please follow these instructions as carefully as possible so all articles within a conference have the same style to the title page. 
The title is set in bold unjustified type using the command
\verb"\title{#1}", where \verb"#1" is the title of the article. The
first letter of the title should be capitalized with the rest in lower case. 
The next information required is the list of all authors' names followed by 
the affiliations. For the authors' names type \verb"\author{#1}", 
where \verb"#1" is the 
list of all authors' names. The style for the names is initials then
surname, with a comma after all but the last 
two names, which are separated by `and'. Initials should {\it not} have 
full stops. First names may be used if desired. The command \verb"\maketitle" is not
required.

The addresses of the authors' affiliations follow the list of authors. 
Each address should be set by using
\verb"\address{#1}" with the address as the single parameter in braces. 
If there is more 
than one address then a superscripted number, followed by a space, should come at the start of
each address. In this case each author should also have a superscripted number or numbers following their name to indicate which address is the appropriate one for them.
 
Please also provide e-mail addresses for any or all of the authors using an \verb"\ead{#1}" command after the last address. \verb"\ead{#1}" provides the text Email: so \verb"#1" is just the e-mail address or a list of emails.  

The abstract follows the addresses and
should give readers concise information about the content 
of the article and should not normally exceed 200 
words. {\bf All articles must include an abstract}. To indicate the start 
of the abstract type \verb"\begin{abstract}" followed by the text of the 
abstract.  The abstract should normally be restricted 
to a single paragraph and is terminated by the command
\verb"\end{abstract}"

\subsection{Sample coding for the start of an article}
\label{startsample}
The code for the start of a title page of a typical paper might read:
\begin{verbatim}
\title{The anomalous magnetic moment of the 
neutrino and its relation to the solar neutrino problem}

\author{P J Smith$^1$, T M Collins$^2$, 
R J Jones$^{3,}$\footnote[4]{Present address:
Department of Physics, University of Bristol, Tyndalls Park Road, 
Bristol BS8 1TS, UK.} and Janet Williams$^3$}

\address{$^1$ Mathematics Faculty, Open University, 
Milton Keynes MK7~6AA, UK}
\address{$^2$ Department of Mathematics, 
Imperial College, Prince Consort Road, London SW7~2BZ, UK}
\address{$^3$ Department of Computer Science, 
University College London, Gower Street, London WC1E~6BT, UK}

\ead{williams@ucl.ac.uk}

\begin{abstract}
The abstract appears here.
\end{abstract}
\end{verbatim}

\section{The text}
The text of the article should should be produced using standard \LaTeX\ formatting. Articles may be divided into sections and subsections, but the length limit provided by the \corg\ should be adhered to.

\subsection{Acknowledgments}
Authors wishing to acknowledge assistance or encouragement from 
colleagues, special work by technical staff or financial support from 
organizations should do so in an unnumbered Acknowledgments section 
immediately following the last numbered section of the paper. The 
command \verb"\ack" sets the acknowledgments heading as an unnumbered
section.

\subsection{Appendices}
Technical detail that it is necessary to include, but that interrupts 
the flow of the article, may be consigned to an appendix. 
Any appendices should be included at the end of the main text of the paper, after the acknowledgments section (if any) but before the reference list.
If there are two or more appendices they will be called Appendix A, Appendix B, etc. 
Numbered equations will be in the form (A.1), (A.2), etc,
figures will appear as figure A1, figure B1, etc and tables as table A1,
table B1, etc.

The command \verb"\appendix" is used to signify the start of the
appendixes. Thereafter \verb"\section", \verb"\subsection", etc, will 
give headings appropriate for an appendix. To obtain a simple heading of 
`Appendix' use the code \verb"\section*{Appendix}". If it contains
numbered equations, figures or tables the command \verb"\appendix" should
precede it and \verb"\setcounter{section}{1}" must follow it. 

\section{References}
%%%%%%%%%%%%%%%%%%%%%%%%%%%%%%%%%%%%%%%%%%%
In the online version of \jpcs\ references will be linked to their original source or to the article within a secondary service such as INSPEC or ChemPort wherever possible. To facilitate this linking extra care should be taken when preparing reference lists. 

Two different styles of referencing are in common use: the Harvard alphabetical system and the Vancouver numerical system.  For \jpcs, the Vancouver numerical system is preferred but authors should use the Harvard alphabetical system if they wish to do so. In the numerical system references are numbered sequentially throughout the text within square brackets, like this [2], and one number can be used to designate several references.  

\subsection{Using \BibTeX}
We highly recommend the {\ttfamily\textbf\selectfont iopart-num} \BibTeX\ package by Mark~A~Caprio \cite{iopartnum}, which is included with this documentation.

\subsection{Reference lists}
A complete reference should provide the reader with enough information to locate the article concerned, whether published in print or electronic form, and should, depending on the type of reference, consist of:  

\begin{itemize}
\item name(s) and initials;
\item date published;
\item title of journal, book or other publication; 
\item titles of journal articles may also be included (optional);
\item volume number;
\item editors, if any;
\item town of publication and publisher in parentheses for {\it books};
\item the page numbers.
\end{itemize}

Up to ten authors may be given in a particular reference; where 
there are more than ten only the first should be given followed by 
`{\it et al}'. If an author is unsure of a particular journal's abbreviated title it is best to leave the title in 
full. The terms {\it loc.\ cit.\ }and {\it ibid.\ }should not be used. 
Unpublished conferences and reports should generally not be included 
in the reference list and articles in the course of publication should 
be entered only if the journal of publication is known. 
A thesis submitted for a higher degree may be included 
in the reference list if it has not been superseded by a published 
paper and is available through a library; sufficient information 
should be given for it to be traced readily.

\subsection{Formatting reference lists}
Numeric reference lists should contain the references within an unnumbered section (such as \verb"\section*{References}"). The 
reference list itself is started by the code 
\verb"\begin{thebibliography}{<num>}", where \verb"<num>" is the largest
number in the reference list and is completed by
\verb"\end{thebibliography}". 
Each reference starts with \verb"\bibitem{<label>}", where `label' is the label used for cross-referencing. Each \verb"\bibitem" should only contain a reference to a single article (or a single article and a preprint reference to the same article).  When one number actually covers a group of two or more references to different articles, \verb"\nonum"
should replace \verb"\bibitem{<label>}" at
the start of each reference in the group after the first.

For an alphabetic reference list use \verb"\begin{thereferences}" ... \verb"\end{thereferences}" instead of the
`thebibliography' environment and each reference can be start with just \verb"\item" instead of \verb"\bibitem{label}"
as cross referencing is less useful for alphabetic references.

\subsection {References to printed journal articles}
A normal reference to a journal article contains three changes of font (see table \ref{jfonts}) and is constructed as follows:

\begin{itemize}
\item the authors should be in the form surname (with only the first letter capitalized) followed by the initials with no periods after the initials. Authors should be separated by a comma except for the last two which should be separated by `and' with no comma preceding it;
\item the article title (if given) should be in lower case letters, except for an initial capital, and should follow the date;
\item the journal title is in italic and is abbreviated. If a journal has several parts denoted by different letters the part letter should be inserted after the journal in Roman type, e.g. {\it Phys. Rev.} A;
\item the volume number should be in bold type;
\item both the initial and final page numbers should be given where possible. The final page number should be in the shortest possible form and separated from the initial page number by an en rule `-- ', e.g. 1203--14, i.e. the numbers `12' are not repeated.
\end{itemize}

A typical (numerical) reference list might begin
