\documentclass[lettersize,journal]{IEEEtran}

%\usepackage[OT1,T1]{fontenc}

\usepackage[numbers,sort&compress]{natbib}
\renewcommand{\bibfont}{\footnotesize}
%\usepackage{cite}
%\usepackage{mystyle}
%%%%%%%%%%%%%%%%%%%%%%%%%%%%%%%%%%%%
\makeatletter

\usepackage{etex}

%%% Review %%%

\usepackage{zref-savepos}

\newcounter{mnote}%[page]
\renewcommand{\themnote}{p.\thepage\;$\langle$\arabic{mnote}$\rangle$}

\def\xmarginnote{%
  \xymarginnote{\hskip -\marginparsep \hskip -\marginparwidth}}

\def\ymarginnote{%
  \xymarginnote{\hskip\columnwidth \hskip\marginparsep}}

\long\def\xymarginnote#1#2{%
\vadjust{#1%
\smash{\hbox{{%
        \hsize\marginparwidth
        \@parboxrestore
        \@marginparreset
\footnotesize #2}}}}}

\def\mnoteson{%
\gdef\mnote##1{\refstepcounter{mnote}\label{##1}%
  \zsavepos{##1}%
  \ifnum20432158>\number\zposx{##1}%
  \xmarginnote{{\color{blue}\bf $\langle$\arabic{mnote}$\rangle$}}% 
  \else
  \ymarginnote{{\color{blue}\bf $\langle$\arabic{mnote}$\rangle$}}%
  \fi%
}
  }
\gdef\mnotesoff{\gdef\mnote##1{}}
\mnoteson
\mnotesoff








%%% Layout %%%

% \usepackage{geometry} % override layout
% \geometry{tmargin=2.5cm,bmargin=m2.5cm,lmargin=3cm,rmargin=3cm}
% \setlength{\pdfpagewidth}{8.5in} % overrides default pdftex paper size
% \setlength{\pdfpageheight}{11in}

\newlength{\mywidth}

%%% Conventions %%%

% References
\newcommand{\figref}[1]{Fig.~\ref{#1}}
\newcommand{\defref}[1]{Definition~\ref{#1}}
\newcommand{\tabref}[1]{Table~\ref{#1}}
% general
%\usepackage{ifthen,nonfloat,subfigure,rotating,array,framed}
\usepackage{framed}
%\usepackage{subfigure}
\usepackage{subcaption}
\usepackage{comment}
%\specialcomment{nb}{\begingroup \noindent \framed\textbf{n.b.\ }}{\endframed\endgroup}
%%\usepackage{xtab,arydshln,multirow}
% topcaption defined in xtab. must load nonfloat before xtab
%\PassOptionsToPackage{svgnames,dvipsnames}{xcolor}
\usepackage[svgnames,dvipsnames]{xcolor}
%\definecolor{myblue}{rgb}{.8,.8,1}
%\definecolor{umbra}{rgb}{.8,.8,.5}
%\newcommand*\mybluebox[1]{%
%  \colorbox{myblue}{\hspace{1em}#1\hspace{1em}}}
\usepackage[all]{xy}
%\usepackage{pstricks,pst-node}
\usepackage{tikz}
\usetikzlibrary{positioning,matrix,through,calc,arrows,fit,shapes,decorations.pathreplacing,decorations.markings,decorations.text}

\tikzstyle{block} = [draw,fill=blue!20,minimum size=2em]

% allow prefix to scope name
\tikzset{%
	prefix node name/.code={%
		\tikzset{%
			name/.code={\edef\tikz@fig@name{#1 ##1}}
		}%
	}%
}


\@ifpackagelater{tikz}{2013/12/01}{
	\newcommand{\convexpath}[2]{
		[create hullcoords/.code={
			\global\edef\namelist{#1}
			\foreach [count=\counter] \nodename in \namelist {
				\global\edef\numberofnodes{\counter}
				\coordinate (hullcoord\counter) at (\nodename);
			}
			\coordinate (hullcoord0) at (hullcoord\numberofnodes);
			\pgfmathtruncatemacro\lastnumber{\numberofnodes+1}
			\coordinate (hullcoord\lastnumber) at (hullcoord1);
		}, create hullcoords ]
		($(hullcoord1)!#2!-90:(hullcoord0)$)
		\foreach [evaluate=\currentnode as \previousnode using \currentnode-1,
		evaluate=\currentnode as \nextnode using \currentnode+1] \currentnode in {1,...,\numberofnodes} {
			let \p1 = ($(hullcoord\currentnode) - (hullcoord\previousnode)$),
			\n1 = {atan2(\y1,\x1) + 90},
			\p2 = ($(hullcoord\nextnode) - (hullcoord\currentnode)$),
			\n2 = {atan2(\y2,\x2) + 90},
			\n{delta} = {Mod(\n2-\n1,360) - 360}
			in 
			{arc [start angle=\n1, delta angle=\n{delta}, radius=#2]}
			-- ($(hullcoord\nextnode)!#2!-90:(hullcoord\currentnode)$) 
		}
	}
}{
	\newcommand{\convexpath}[2]{
		[create hullcoords/.code={
			\global\edef\namelist{#1}
			\foreach [count=\counter] \nodename in \namelist {
				\global\edef\numberofnodes{\counter}
				\coordinate (hullcoord\counter) at (\nodename);
			}
			\coordinate (hullcoord0) at (hullcoord\numberofnodes);
			\pgfmathtruncatemacro\lastnumber{\numberofnodes+1}
			\coordinate (hullcoord\lastnumber) at (hullcoord1);
		}, create hullcoords ]
		($(hullcoord1)!#2!-90:(hullcoord0)$)
		\foreach [evaluate=\currentnode as \previousnode using \currentnode-1,
		evaluate=\currentnode as \nextnode using \currentnode+1] \currentnode in {1,...,\numberofnodes} {
			let \p1 = ($(hullcoord\currentnode) - (hullcoord\previousnode)$),
			\n1 = {atan2(\x1,\y1) + 90},
			\p2 = ($(hullcoord\nextnode) - (hullcoord\currentnode)$),
			\n2 = {atan2(\x2,\y2) + 90},
			\n{delta} = {Mod(\n2-\n1,360) - 360}
			in 
			{arc [start angle=\n1, delta angle=\n{delta}, radius=#2]}
			-- ($(hullcoord\nextnode)!#2!-90:(hullcoord\currentnode)$) 
		}
	}
}

% circle around nodes

% typsetting math
\usepackage{qsymbols,amssymb,mathrsfs}
\usepackage{amsmath}
\usepackage[standard,thmmarks]{ntheorem}
\theoremstyle{plain}
\theoremsymbol{\ensuremath{_\vartriangleleft}}
\theorembodyfont{\itshape}
\theoremheaderfont{\normalfont\bfseries}
\theoremseparator{}
\newtheorem{Claim}{Claim}
\newtheorem{Subclaim}{Subclaim}
\theoremstyle{nonumberplain}
\theoremheaderfont{\scshape}
\theorembodyfont{\normalfont}
\theoremsymbol{\ensuremath{_\blacktriangleleft}}
\newtheorem{Subproof}{Proof}

\theoremnumbering{arabic}
\theoremstyle{plain}
\usepackage{latexsym}
\theoremsymbol{\ensuremath{_\Box}}
\theorembodyfont{\itshape}
\theoremheaderfont{\normalfont\bfseries}
\theoremseparator{}
\newtheorem{Conjecture}{Conjecture}

\theorembodyfont{\upshape}
\theoremprework{\bigskip\hrule}
\theorempostwork{\hrule\bigskip}
\newtheorem{Condition}{Condition}%[section]


%\RequirePckage{amsmath} loaded by empheq
\usepackage[overload]{empheq} % no \intertext and \displaybreak
%\usepackage{breqn}

\let\iftwocolumn\if@twocolumn
\g@addto@macro\@twocolumntrue{\let\iftwocolumn\if@twocolumn}
\g@addto@macro\@twocolumnfalse{\let\iftwocolumn\if@twocolumn}

%\empheqset{box=\mybluebox}
%\usepackage{mathtools}      % to polish math typsetting, loaded
%                                % by empeq
\mathtoolsset{showonlyrefs=false,showmanualtags}
\let\underbrace\LaTeXunderbrace % adapt spacing to font sizes
\let\overbrace\LaTeXoverbrace
\renewcommand{\eqref}[1]{\textup{(\refeq{#1})}} % eqref was not allowed in
                                       % sub/super-scripts
\newtagform{brackets}[]{(}{)}   % new tags for equations
\usetagform{brackets}
% defined commands:
% \shortintertext{}, dcases*, \cramped, \smashoperator[]{}

\usepackage[Smaller]{cancel}
\renewcommand{\CancelColor}{\color{Red}}
%\newcommand\hcancel[2][black]{\setbox0=\hbox{#2}% colored horizontal cross
%  \rlap{\raisebox{.45\ht0}{\color{#1}\rule{\wd0}{1pt}}}#2}



\usepackage{graphicx,psfrag}
\graphicspath{{figure/}{image/}} % Search path of figures

% for tabular
\usepackage{diagbox} % \backslashbox{}{} for slashed entries
%\usepackage{threeparttable} % threeparttable, \tnote{},
                                % tablenotes, and \item[]
%\usepackage{colortab} % \cellcolor[gray]{0.9},
%\rowcolor,\columncolor,
%\usepackage{colortab} % \LCC \gray & ...  \ECC \\

% typesetting codes
%\usepackage{maple2e} % need to use \char29 for ^
\usepackage{algorithm2e}
\usepackage{listings} 
\lstdefinelanguage{Maple}{
  morekeywords={proc,module,end, for,from,to,by,while,in,do,od
    ,if,elif,else,then,fi ,use,try,catch,finally}, sensitive,
  morecomment=[l]\#,
  morestring=[b]",morestring=[b]`}[keywords,comments,strings]
\lstset{
  basicstyle=\scriptsize,
  keywordstyle=\color{ForestGreen}\bfseries,
  commentstyle=\color{DarkBlue},
  stringstyle=\color{DimGray}\ttfamily,
  texcl
}
%%% New fonts %%%
\DeclareMathAlphabet{\mathpzc}{OT1}{pzc}{m}{it}
\usepackage{upgreek} % \upalpha,\upbeta, ...
%\usepackage{bbold}   % blackboard math
\usepackage{dsfont}  % \mathds

%%% Macros for multiple definitions %%%

% example usage:
% \multi{M}{\boldsymbol{#1}}  % defines \multiM
% \multi ABC.                 % defines \MA \MB and \MC as
%                             % \boldsymbol{A}, \boldsymbol{B} and
%                             % \boldsymbol{C} respectively.
% 
%  The last period '.' is necessary to terminate the macro expansion.
%
% \multi*{M}{\boldsymbol{#1}} % defines \multiM and \M
% \M{A}                       % expands to \boldsymbol{A}

\def\multi@nostar#1#2{%
  \expandafter\def\csname multi#1\endcsname##1{%
    \if ##1.\let\next=\relax \else
    \def\next{\csname multi#1\endcsname}     
    %\expandafter\def\csname #1##1\endcsname{#2}
    \expandafter\newcommand\csname #1##1\endcsname{#2}
    \fi\next}}

\def\multi@star#1#2{%
  \expandafter\def\csname #1\endcsname##1{#2}
  \multi@nostar{#1}{#2}
}

\newcommand{\multi}{%
  \@ifstar \multi@star \multi@nostar}

%%% new alphabets %%%

\multi*{rm}{\mathrm{#1}}
\multi*{mc}{\mathcal{#1}}
\multi*{op}{\mathop {\operator@font #1}}
% \multi*{op}{\operatorname{#1}}
\multi*{ds}{\mathds{#1}}
\multi*{set}{\mathcal{#1}}
\multi*{rsfs}{\mathscr{#1}}
\multi*{pz}{\mathpzc{#1}}
\multi*{M}{\boldsymbol{#1}}
\multi*{R}{\mathsf{#1}}
\multi*{RM}{\M{\R{#1}}}
\multi*{bb}{\mathbb{#1}}
\multi*{td}{\tilde{#1}}
\multi*{tR}{\tilde{\mathsf{#1}}}
\multi*{trM}{\tilde{\M{\R{#1}}}}
\multi*{tset}{\tilde{\mathcal{#1}}}
\multi*{tM}{\tilde{\M{#1}}}
\multi*{baM}{\bar{\M{#1}}}
\multi*{ol}{\overline{#1}}

\multirm  ABCDEFGHIJKLMNOPQRSTUVWXYZabcdefghijklmnopqrstuvwxyz.
\multiol  ABCDEFGHIJKLMNOPQRSTUVWXYZabcdefghijklmnopqrstuvwxyz.
\multitR   ABCDEFGHIJKLMNOPQRSTUVWXYZabcdefghijklmnopqrstuvwxyz.
\multitd   ABCDEFGHIJKLMNOPQRSTUVWXYZabcdefghijklmnopqrstuvwxyz.
\multitset ABCDEFGHIJKLMNOPQRSTUVWXYZabcdefghijklmnopqrstuvwxyz.
\multitM   ABCDEFGHIJKLMNOPQRSTUVWXYZabcdefghijklmnopqrstuvwxyz.
\multibaM   ABCDEFGHIJKLMNOPQRSTUVWXYZabcdefghijklmnopqrstuvwxyz.
\multitrM   ABCDEFGHIJKLMNOPQRSTUVWXYZabcdefghijklmnopqrstuvwxyz.
\multimc   ABCDEFGHIJKLMNOPQRSTUVWXYZabcdefghijklmnopqrstuvwxyz.
\multiop   ABCDEFGHIJKLMNOPQRSTUVWXYZabcdefghijklmnopqrstuvwxyz.
\multids   ABCDEFGHIJKLMNOPQRSTUVWXYZabcdefghijklmnopqrstuvwxyz.
\multiset  ABCDEFGHIJKLMNOPQRSTUVWXYZabcdefghijklmnopqrstuvwxyz.
\multirsfs ABCDEFGHIJKLMNOPQRSTUVWXYZabcdefghijklmnopqrstuvwxyz.
\multipz   ABCDEFGHIJKLMNOPQRSTUVWXYZabcdefghijklmnopqrstuvwxyz.
\multiM    ABCDEFGHIJKLMNOPQRSTUVWXYZabcdefghijklmnopqrstuvwxyz.
\multiR    ABCDEFGHIJKL NO QR TUVWXYZabcd fghijklmnopqrstuvwxyz.
\multibb   ABCDEFGHIJKLMNOPQRSTUVWXYZabcdefghijklmnopqrstuvwxyz.
\multiRM   ABCDEFGHIJKLMNOPQRSTUVWXYZabcdefghijklmnopqrstuvwxyz.
\newcommand{\RRM}{\R{M}}
\newcommand{\RRP}{\R{P}}
\newcommand{\RRe}{\R{e}}
\newcommand{\RRS}{\R{S}}
%%% new symbols %%%

%\newcommand{\dotgeq}{\buildrel \textstyle  .\over \geq}
%\newcommand{\dotleq}{\buildrel \textstyle  .\over \leq}
\newcommand{\dotleq}{\buildrel \textstyle  .\over {\smash{\lower
      .2ex\hbox{\ensuremath\leqslant}}\vphantom{=}}}
\newcommand{\dotgeq}{\buildrel \textstyle  .\over {\smash{\lower
      .2ex\hbox{\ensuremath\geqslant}}\vphantom{=}}}

\DeclareMathOperator*{\argmin}{arg\,min}
\DeclareMathOperator*{\argmax}{arg\,max}

%%% abbreviations %%%

% commands
\newcommand{\esm}{\ensuremath}

% environments
\newcommand{\bM}{\begin{bmatrix}}
\newcommand{\eM}{\end{bmatrix}}
\newcommand{\bSM}{\left[\begin{smallmatrix}}
\newcommand{\eSM}{\end{smallmatrix}\right]}
\renewcommand*\env@matrix[1][*\c@MaxMatrixCols c]{%
  \hskip -\arraycolsep
  \let\@ifnextchar\new@ifnextchar
  \array{#1}}



% sets of number
\newqsymbol{`N}{\mathbb{N}}
\newqsymbol{`R}{\mathbb{R}}
\newqsymbol{`P}{\mathbb{P}}
\newqsymbol{`Z}{\mathbb{Z}}

% symbol short cut
\newqsymbol{`|}{\mid}
% use \| for \parallel
\newqsymbol{`8}{\infty}
\newqsymbol{`1}{\left}
\newqsymbol{`2}{\right}
\newqsymbol{`6}{\partial}
\newqsymbol{`0}{\emptyset}
\newqsymbol{`-}{\leftrightarrow}
\newqsymbol{`<}{\langle}
\newqsymbol{`>}{\rangle}

%%% new operators / functions %%%

\newcommand{\sgn}{\operatorname{sgn}}
\newcommand{\Var}{\op{Var}}
\newcommand{\diag}{\operatorname{diag}}
\newcommand{\erf}{\operatorname{erf}}
\newcommand{\erfc}{\operatorname{erfc}}
\newcommand{\erfi}{\operatorname{erfi}}
\newcommand{\adj}{\operatorname{adj}}
\newcommand{\supp}{\operatorname{supp}}
\newcommand{\E}{\opE\nolimits}
\newcommand{\T}{\intercal}
% requires mathtools
% \abs,\abs*,\abs[<size_cmd:\big,\Big,\bigg,\Bigg etc.>]
\DeclarePairedDelimiter\abs{\lvert}{\rvert} 
\DeclarePairedDelimiter\norm{\lVert}{\rVert}
\DeclarePairedDelimiter\ceil{\lceil}{\rceil}
\DeclarePairedDelimiter\floor{\lfloor}{\rfloor}
\DeclarePairedDelimiter\Set{\{}{\}}
\newcommand{\imod}[1]{\allowbreak\mkern10mu({\operator@font mod}\,\,#1)}

%%% new formats %%%
\newcommand{\leftexp}[2]{{\vphantom{#2}}^{#1}{#2}}


% non-floating figures that can be put inside tables
\newenvironment{nffigure}[1][\relax]{\vskip \intextsep
  \noindent\minipage{\linewidth}\def\@captype{figure}}{\endminipage\vskip \intextsep}

\newcommand{\threecols}[3]{
\hbox to \textwidth{%
      \normalfont\rlap{\parbox[b]{\textwidth}{\raggedright#1\strut}}%
        \hss\parbox[b]{\textwidth}{\centering#2\strut}\hss
        \llap{\parbox[b]{\textwidth}{\raggedleft#3\strut}}%
    }% hbox 
}

\newcommand{\reason}[2][\relax]{
  \ifthenelse{\equal{#1}{\relax}}{
    \left(\text{#2}\right)
  }{
    \left(\parbox{#1}{\raggedright #2}\right)
  }
}

\newcommand{\marginlabel}[1]
{\mbox[]\marginpar{\color{ForestGreen} \sffamily \small \raggedright\hspace{0pt}#1}}


% up-tag an equation
\newcommand{\utag}[2]{\mathop{#2}\limits^{\text{(#1)}}}
\newcommand{\uref}[1]{(#1)}


% Notation table

\newcommand{\Hline}{\noalign{\vskip 0.1in \hrule height 0.1pt \vskip
    0.1in}}
  
\def\Malign#1{\tabskip=0in
  \halign to\columnwidth{
    \ensuremath{\displaystyle ##}\hfil
    \tabskip=0in plus 1 fil minus 1 fil
    &
    \parbox[t]{0.8\columnwidth}{##}
    \tabskip=0in
    \cr #1}}


%%%%%%%%%%%%%%%%%%%%%%%%%%%%%%%%%%%%%%%%%%%%%%%%%%%%%%%%%%%%%%%%%%%
% MISCELLANEOUS

% Modification from braket.sty by Donald Arseneau
% Command defined is: \extendvert{ }
% The "small versions" use fixed-size brackets independent of their
% contents, whereas the expand the first vertical line '|' or '\|' to
% envelop the content
\let\SavedDoubleVert\relax
\let\protect\relax
{\catcode`\|=\active
  \xdef\extendvert{\protect\expandafter\noexpand\csname extendvert \endcsname}
  \expandafter\gdef\csname extendvert \endcsname#1{\mskip-5mu \left.%
      \ifx\SavedDoubleVert\relax \let\SavedDoubleVert\|\fi
     \:{\let\|\SetDoubleVert
       \mathcode`\|32768\let|\SetVert
     #1}\:\right.\mskip-5mu}
}
\def\SetVert{\@ifnextchar|{\|\@gobble}% turn || into \|
    {\egroup\;\mid@vertical\;\bgroup}}
\def\SetDoubleVert{\egroup\;\mid@dblvertical\;\bgroup}

% If the user is using e-TeX with its \middle primitive, use that for
% verticals instead of \vrule.
%
\begingroup
 \edef\@tempa{\meaning\middle}
 \edef\@tempb{\string\middle}
\expandafter \endgroup \ifx\@tempa\@tempb
 \def\mid@vertical{\middle|}
 \def\mid@dblvertical{\middle\SavedDoubleVert}
\else
 \def\mid@vertical{\mskip1mu\vrule\mskip1mu}
 \def\mid@dblvertical{\mskip1mu\vrule\mskip2.5mu\vrule\mskip1mu}
\fi

%%%%%%%%%%%%%%%%%%%%%%%%%%%%%%%%%%%%%%%%%%%%%%%%%%%%%%%%%%%%%%%%

\makeatother

%%%%%%%%%%%%%%%%%%%%%%%%%%%%%%%%%%%%

\usepackage{ctable}
\usepackage{fouridx}
%\usepackage{calc}
\usepackage{framed}
\usetikzlibrary{positioning,matrix}

\usepackage{paralist}
%\usepackage{refcheck}
\usepackage{enumerate}

\usepackage[normalem]{ulem}
\newcommand{\Ans}[1]{\uuline{\raisebox{.15em}{#1}}}



\numberwithin{equation}{section}
\makeatletter
\@addtoreset{equation}{section}
\renewcommand{\theequation}{\arabic{section}.\arabic{equation}}
\@addtoreset{Theorem}{section}
\renewcommand{\theTheorem}{\arabic{section}.\arabic{Theorem}}
\@addtoreset{Lemma}{section}
\renewcommand{\theLemma}{\arabic{section}.\arabic{Lemma}}
\@addtoreset{Corollary}{section}
\renewcommand{\theCorollary}{\arabic{section}.\arabic{Corollary}}
\@addtoreset{Example}{section}
\renewcommand{\theExample}{\arabic{section}.\arabic{Example}}
\@addtoreset{Remark}{section}
\renewcommand{\theRemark}{\arabic{section}.\arabic{Remark}}
\@addtoreset{Proposition}{section}
\renewcommand{\theProposition}{\arabic{section}.\arabic{Proposition}}
\@addtoreset{Definition}{section}
\renewcommand{\theDefinition}{\arabic{section}.\arabic{Definition}}
\@addtoreset{Claim}{section}
\renewcommand{\theClaim}{\arabic{section}.\arabic{Claim}}
\@addtoreset{Subclaim}{Theorem}
\renewcommand{\theSubclaim}{\theTheorem\Alph{Subclaim}}
\makeatother

\newcommand{\Null}{\op{Null}}
%\newcommand{\T}{\op{T}\nolimits}
\newcommand{\Bern}{\op{Bern}\nolimits}
\newcommand{\odd}{\op{odd}}
\newcommand{\even}{\op{even}}
\newcommand{\Sym}{\op{Sym}}
\newcommand{\si}{s_{\op{div}}}
\newcommand{\sv}{s_{\op{var}}}
\newcommand{\Wtyp}{W_{\op{typ}}}
\newcommand{\Rco}{R_{\op{CO}}}
\newcommand{\Tm}{\op{T}\nolimits}
\newcommand{\JGK}{J_{\op{GK}}}

\newcommand{\diff}{\mathrm{d}}

\newenvironment{lbox}{
  \setlength{\FrameSep}{1.5mm}
  \setlength{\FrameRule}{0mm}
  \def\FrameCommand{\fboxsep=\FrameSep \fcolorbox{black!20}{white}}%
  \MakeFramed {\FrameRestore}}%
{\endMakeFramed}

\newenvironment{ybox}{
	\setlength{\FrameSep}{1.5mm}
	\setlength{\FrameRule}{0mm}
  \def\FrameCommand{\fboxsep=\FrameSep \fcolorbox{black!10}{yellow!8}}%
  \MakeFramed {\FrameRestore}}%
{\endMakeFramed}

\newenvironment{gbox}{
	\setlength{\FrameSep}{1.5mm}
\setlength{\FrameRule}{0mm}
  \def\FrameCommand{\fboxsep=\FrameSep \fcolorbox{black!10}{green!8}}%
  \MakeFramed {\FrameRestore}}%
{\endMakeFramed}

\newenvironment{bbox}{
	\setlength{\FrameSep}{1.5mm}
\setlength{\FrameRule}{0mm}
  \def\FrameCommand{\fboxsep=\FrameSep \fcolorbox{black!10}{blue!8}}%
  \MakeFramed {\FrameRestore}}%
{\endMakeFramed}

\newenvironment{yleftbar}{%
  \def\FrameCommand{{\color{yellow!20}\vrule width 3pt} \hspace{10pt}}%
  \MakeFramed {\advance\hsize-\width \FrameRestore}}%
 {\endMakeFramed}

\newcommand{\tbox}[2][\relax]{
 \setlength{\FrameSep}{1.5mm}
  \setlength{\FrameRule}{0mm}
  \begin{ybox}
    \noindent\underline{#1:}\newline
    #2
  \end{ybox}
}

\newcommand{\pbox}[2][\relax]{
  \setlength{\FrameSep}{1.5mm}
 \setlength{\FrameRule}{0mm}
  \begin{gbox}
    \noindent\underline{#1:}\newline
    #2
  \end{gbox}
}

\newcommand{\gtag}[1]{\text{\color{green!50!black!60} #1}}
\let\labelindent\relax
\usepackage{enumitem}

%%%%%%%%%%%%%%%%%%%%%%%%%%%%%%%%%%%%
% fix subequations
% http://tex.stackexchange.com/questions/80134/nesting-subequations-within-align
%%%%%%%%%%%%%%%%%%%%%%%%%%%%%%%%%%%%

\usepackage{etoolbox}

% let \theparentequation use the same definition as equation
\let\theparentequation\theequation
% change every occurence of "equation" to "parentequation"
\patchcmd{\theparentequation}{equation}{parentequation}{}{}

\renewenvironment{subequations}[1][]{%              optional argument: label-name for (first) parent equation
	\refstepcounter{equation}%
	%  \def\theparentequation{\arabic{parentequation}}% we patched it already :)
	\setcounter{parentequation}{\value{equation}}%    parentequation = equation
	\setcounter{equation}{0}%                         (sub)equation  = 0
	\def\theequation{\theparentequation\alph{equation}}% 
	\let\parentlabel\label%                           Evade sanitation performed by amsmath
	\ifx\\#1\\\relax\else\label{#1}\fi%               #1 given: \label{#1}, otherwise: nothing
	\ignorespaces
}{%
	\setcounter{equation}{\value{parentequation}}%    equation = subequation
	\ignorespacesafterend
}

\newcommand*{\nextParentEquation}[1][]{%            optional argument: label-name for (first) parent equation
	\refstepcounter{parentequation}%                  parentequation++
	\setcounter{equation}{0}%                         equation = 0
	\ifx\\#1\\\relax\else\parentlabel{#1}\fi%         #1 given: \label{#1}, otherwise: nothing
}

% hyperlink
\PassOptionsToPackage{breaklinks,letterpaper,hyperindex=true,backref=false,bookmarksnumbered,bookmarksopen,linktocpage,colorlinks,linkcolor=BrickRed,citecolor=OliveGreen,urlcolor=Blue,pdfstartview=FitH}{hyperref}
\usepackage{hyperref}

% makeindex style
\newcommand{\indexmain}[1]{\textbf{\hyperpage{#1}}}
\begin{document}

\title{Distributed Gaussian Mixture PHD Filtering under Communication Constraints}

\author{Shiraz Khan, Yi-Chieh Sun, and Inseok Hwang\thanks{The authors are with the School of Aeronautics and Astronautics, Purdue University, West Lafayette, IN 47906, USA. (email: \texttt{shiraz@purdue.edu})}
%~\IEEEmembership{Staff,~IEEE,}
        % <-this % stops a space
%\thanks{This paper was produced by the IEEE Publication Technology Group. They are in Piscataway, NJ.}% <-this % stops a space
% \thanks{Manuscript submitted on Aug 1, 2023.}
\thanks{\textcolor{blue}{© 2023 IEEE. Personal use of this material is permitted. Permission from IEEE must be obtained for all other uses, in any current or future media, including reprinting/republishing this material for advertising or promotional purposes, creating new collective works, for resale or redistribution to servers or lists, or reuse of any copyrighted component of this work in other works.}}
}

% The paper headers
\markboth{IEEE Transactions on Signal Processing}%
{Khan \MakeLowercase{\textit{et al.}}: Distributed Gaussian Mixture PHD Filtering under Communication Constraints}

% \IEEEpubid{0000--0000/00\$00.00~\copyright~2023 IEEE}
% Remember, if you use this you must call \IEEEpubidadjcol in the second
% column for its text to clear the IEEEpubid mark.

\maketitle

\begin{abstract}
The Gaussian Mixture Probability Hypothesis Density (GM-PHD) filter is an almost exact closed-form approximation to the Bayes-optimal multi-target tracking algorithm. Due to its optimality guarantees and ease of implementation, it has been studied extensively in the literature. However, the challenges involved in implementing the GM-PHD filter efficiently in a distributed (multi-sensor) setting have received little attention. The existing solutions for distributed PHD filtering either have a high computational and communication cost, making them infeasible for resource-constrained applications, or are unable to guarantee the asymptotic convergence of the distributed PHD algorithm to an optimal solution.     
In this paper, we develop a distributed GM-PHD filtering recursion that uses a probabilistic communication rule to limit the communication bandwidth of the algorithm, while ensuring asymptotic optimality of the algorithm. We derive the convergence properties of this recursion, which uses weighted average consensus of Gaussian mixtures (GMs) to lower (and asymptotically minimize) the Cauchy-Schwarz divergence between the sensors' local estimates. In addition, the proposed method is able to avoid the issue of false positives, which has previously been noted to impact the filtering performance of distributed multi-target tracking. Through numerical simulations, it is demonstrated that our proposed method is an effective solution for distributed multi-target tracking in resource-constrained sensor networks.
\end{abstract}

\begin{IEEEkeywords}
multi-target tracking, sensor networks, distributed state estimation, Gaussian mixture
\end{IEEEkeywords}

%

\maketitle

% \leavevmode
% \\
% \\
% \\
% \\
% \\
\section{Introduction}
\label{introduction}

AutoML is the process by which machine learning models are built automatically for a new dataset. Given a dataset, AutoML systems perform a search over valid data transformations and learners, along with hyper-parameter optimization for each learner~\cite{VolcanoML}. Choosing the transformations and learners over which to search is our focus.
A significant number of systems mine from prior runs of pipelines over a set of datasets to choose transformers and learners that are effective with different types of datasets (e.g. \cite{NEURIPS2018_b59a51a3}, \cite{10.14778/3415478.3415542}, \cite{autosklearn}). Thus, they build a database by actually running different pipelines with a diverse set of datasets to estimate the accuracy of potential pipelines. Hence, they can be used to effectively reduce the search space. A new dataset, based on a set of features (meta-features) is then matched to this database to find the most plausible candidates for both learner selection and hyper-parameter tuning. This process of choosing starting points in the search space is called meta-learning for the cold start problem.  

Other meta-learning approaches include mining existing data science code and their associated datasets to learn from human expertise. The AL~\cite{al} system mined existing Kaggle notebooks using dynamic analysis, i.e., actually running the scripts, and showed that such a system has promise.  However, this meta-learning approach does not scale because it is onerous to execute a large number of pipeline scripts on datasets, preprocessing datasets is never trivial, and older scripts cease to run at all as software evolves. It is not surprising that AL therefore performed dynamic analysis on just nine datasets.

Our system, {\sysname}, provides a scalable meta-learning approach to leverage human expertise, using static analysis to mine pipelines from large repositories of scripts. Static analysis has the advantage of scaling to thousands or millions of scripts \cite{graph4code} easily, but lacks the performance data gathered by dynamic analysis. The {\sysname} meta-learning approach guides the learning process by a scalable dataset similarity search, based on dataset embeddings, to find the most similar datasets and the semantics of ML pipelines applied on them.  Many existing systems, such as Auto-Sklearn \cite{autosklearn} and AL \cite{al}, compute a set of meta-features for each dataset. We developed a deep neural network model to generate embeddings at the granularity of a dataset, e.g., a table or CSV file, to capture similarity at the level of an entire dataset rather than relying on a set of meta-features.
 
Because we use static analysis to capture the semantics of the meta-learning process, we have no mechanism to choose the \textbf{best} pipeline from many seen pipelines, unlike the dynamic execution case where one can rely on runtime to choose the best performing pipeline.  Observing that pipelines are basically workflow graphs, we use graph generator neural models to succinctly capture the statically-observed pipelines for a single dataset. In {\sysname}, we formulate learner selection as a graph generation problem to predict optimized pipelines based on pipelines seen in actual notebooks.

%. This formulation enables {\sysname} for effective pruning of the AutoML search space to predict optimized pipelines based on pipelines seen in actual notebooks.}
%We note that increasingly, state-of-the-art performance in AutoML systems is being generated by more complex pipelines such as Directed Acyclic Graphs (DAGs) \cite{piper} rather than the linear pipelines used in earlier systems.  
 
{\sysname} does learner and transformation selection, and hence is a component of an AutoML systems. To evaluate this component, we integrated it into two existing AutoML systems, FLAML \cite{flaml} and Auto-Sklearn \cite{autosklearn}.  
% We evaluate each system with and without {\sysname}.  
We chose FLAML because it does not yet have any meta-learning component for the cold start problem and instead allows user selection of learners and transformers. The authors of FLAML explicitly pointed to the fact that FLAML might benefit from a meta-learning component and pointed to it as a possibility for future work. For FLAML, if mining historical pipelines provides an advantage, we should improve its performance. We also picked Auto-Sklearn as it does have a learner selection component based on meta-features, as described earlier~\cite{autosklearn2}. For Auto-Sklearn, we should at least match performance if our static mining of pipelines can match their extensive database. For context, we also compared {\sysname} with the recent VolcanoML~\cite{VolcanoML}, which provides an efficient decomposition and execution strategy for the AutoML search space. In contrast, {\sysname} prunes the search space using our meta-learning model to perform hyperparameter optimization only for the most promising candidates. 

The contributions of this paper are the following:
\begin{itemize}
    \item Section ~\ref{sec:mining} defines a scalable meta-learning approach based on representation learning of mined ML pipeline semantics and datasets for over 100 datasets and ~11K Python scripts.  
    \newline
    \item Sections~\ref{sec:kgpipGen} formulates AutoML pipeline generation as a graph generation problem. {\sysname} predicts efficiently an optimized ML pipeline for an unseen dataset based on our meta-learning model.  To the best of our knowledge, {\sysname} is the first approach to formulate  AutoML pipeline generation in such a way.
    \newline
    \item Section~\ref{sec:eval} presents a comprehensive evaluation using a large collection of 121 datasets from major AutoML benchmarks and Kaggle. Our experimental results show that {\sysname} outperforms all existing AutoML systems and achieves state-of-the-art results on the majority of these datasets. {\sysname} significantly improves the performance of both FLAML and Auto-Sklearn in classification and regression tasks. We also outperformed AL in 75 out of 77 datasets and VolcanoML in 75  out of 121 datasets, including 44 datasets used only by VolcanoML~\cite{VolcanoML}.  On average, {\sysname} achieves scores that are statistically better than the means of all other systems. 
\end{itemize}


%This approach does not need to apply cleaning or transformation methods to handle different variances among datasets. Moreover, we do not need to deal with complex analysis, such as dynamic code analysis. Thus, our approach proved to be scalable, as discussed in Sections~\ref{sec:mining}.

\section{Problem Formulation}
\label{sec:prob_form}
\subsection{Target Model}
In the multi-sensor multi-target tracking problem, an unknown time-varying number of dynamical systems, called the \textit{targets}, are observed using a sensor network.
The state of the $l^{th}$ target at timestep $k$ is given by a random vector $x_{k}^{(l)} \in \mathcal{X}$, where $\mathcal{X} = \mathbb{R}^{d_x}$ is the state space of each target and $d_x$ is its dimension. The collection of the target states at timestep $k$ is modeled as a random finite set (RFS)\footnote{A brief introduction to RFSs and their properties can be found in \cite{vo2006gmphd}.} $X_k = \left\{x_{k}^{(1)},...,x_{k}^{(|X_k|)}\right\}$, where $|{}\cdot{}|$ denotes the cardinality of a set.
Thus, the number of targets, $|X_k|$, is an integer-valued random variable.

The state of each target is assumed to evolve independently of the other targets, according to a linear Gauss-Markov process on the state space $\mathcal{X}$ with the transition kernel $f_{k|k-1}$. Thus, given that a target exists at timesteps $k-1$ and $k$, the probability density function (pdf) of its state at timestep $k$ is
\begin{equation}
    f_{k|k-1}(x | x_{k-1}) = 
    \mathcal{N}(x; F_{k-1} x_{k-1}, Q_{k-1})
    \label{eq:dynamic_model}
\end{equation}
where $F_{k-1}$ and $Q_{k-1}$ are known matrices and $x_{k-1} \in\mathcal X$ is the state of the target at timestep $k-1$. 
$\mathcal{N}({}\cdot{}; m, P)$ denotes the pdf of a multivariate Gaussian random vector with mean $m\in \mathbb R^{d_x}$ and covariance $P\in \mathbb R^{d_x\times d_x}$. 
The probability of a target continues to exist across both timesteps is called its \textit{survival probability}, denoted by $p^S_{k}:\mathcal X \rightarrow [0, 1]$, and is assumed to depend only on the current state of each target.

New targets of interests can arise due to spontaneous \textit{births} or \textit{spawns}. Births are independent of $X_k$ and assumed to be sampled from an underlying Poisson RFS having the intensity $\gamma _k:\mathcal X \rightarrow \mathbb R_{\geq0}$, where $\mathbb R_{\geq0}$ denotes the set of non-negative real numbers. 
Spawns correspond to targets that emerge from the vicinity of existing targets, so they are dependent on $X_k$. Given that a target exists at $x_{k-1} \in \mathcal X$ at timestep $k-1$, the new targets spawned from it are assumed to be sampled from
a Poisson RFS having the intensity $\sigma _{k|k-1}({}\cdot{}|x_{k-1}):\mathcal X \rightarrow \mathbb R_{\geq 0}$. Finally, note that the transition kernel $f_{k|k-1}$ and spawn intensity $\sigma_{k|k-1}$ are independent of the labels or identities of the targets; if they are dependent on the target labels, a labelled RFS can be used to represent the multi-target state instead \cite{vo2014labeled}.

\subsection{Sensor Network Model}
% sensor network and measurements
The targets are being observed by
a sensor network consisting of a number of spatially distributed sensors, which can be represented as a directed graph $\mathcal G=(\mathcal V, \mathcal E)$, where $\mathcal V$ denotes its vertices (representing the sensors) and $\mathcal E \subseteq \mathcal V \times \mathcal V$ is the set of edges (representing the directional communication links between sensors).
The graph $\mathcal G$ is assumed to be strongly connected. 
% The total number of sensors in the network is given by $|\mathcal V|$. 
% Given two sensors $i,j\in \mathcal V$, sensor $j$ can receive information from sensor $i$ if and only if $(i,j)\in \mathcal E$.
For each sensor $i \in \mathcal{V}$, ${N}^-_{i} = \left\{ j\in\mathcal{V}: (j,i)\in\mathcal{E} \right\}$ denotes its set of in-neighbors, i.e., the set of sensors from which sensor $i$ can receive data. Similarly, $N^+_i$ is the set of out-neighbors.

At timestep $k$, each sensor $i$ obtains a set of measurements $Z_{i,k} = {\left\{z^{(1)}_{i,k},...,z_{i,k}^{(|Z_{i,k}|)}\right\}}$, where $z_{i,k}^{(l)} \in \mathcal{Z}$. Here, $\mathcal{Z} \subseteq \mathbb{R}^{d_z}$ is the measurement space and $d_z$ is its dimension. It is assumed that the measurements are generated as per the following mechanism: the $i^{th}$ sensor obtains a measurement from a target at $x \in \mathcal X$ with the probability $p^D_{i,k}(x)$, where $p^D_{i,k}:\mathcal X \rightarrow [0,1]$ is called the detection probability of sensor $i$ at timestep $k$. The event of the $l^{th}$ target being detected at the $i^{th}$ sensor is independent of the detections at other target-sensor pairs.
%, so the detections can be thought of as i.i.d. Bernoulli trials. 
Given that a target at $x \in \mathcal X$ is detected by sensor $i$, the pdf of the obtained measurement is denoted as $g_{i,k}({}\cdot{}|x)$ and is  
given by the linear Gaussian model
\begin{equation}
    g_{i,k}(z|x) = \mathcal N(z; H_{i,k} x, R_{i,k})
    \label{eq:measurement_model}
\end{equation}
where $H_{i,k}$ and $R_{i,k}$ are known matrices.
In addition, the obtained measurement can correspond to clutter, which are not generated by any of the targets, but arise due to sensor noise or other random effects. The clutter measurements are assumed to be realizations of a Poisson RFS with the intensity $\kappa_i:\mathcal Z \rightarrow \mathbb R_{\geq 0}$.
Thus, $\lbrace Z_{i,k} \rbrace$ are realizations of an RFS, and the distribution of the cardinality of this RFS (i.e., the number of measurements) is determined by the detection probability $p^D_{i,k}$ and clutter intensity $\kappa _i$.


\subsection{Local GM-PHD Filtering}
\label{sec:subsec_local_filt}
Distributed multi-target tracking over a sensor network is usually accomplished using a succession of a local filtering step (e.g., using a PHD filter) at each of the sensors followed by a fusion step, wherein the sensors share information across the communication channels. 
%Each iteration of the local PHD filter consists of a prediction and an update step. 
The predicted multi-target state of sensor $i$ is an RFS whose intensity is denoted as $v_{i,k|k-1}$. It characterizes the information known about the multi-target state $X_k$ at sensor $i$ at timestep $k$, before observing the measurements $Z_{i,k}$. After the incorporation of the measurements $Z_{i,k}$, the updated (posterior) intensity is denoted as $v_{i,k|k}$. 
% Finally, the intensities of the individual sensors are fused together, with the fused intensity at sensor $i$ being denoted as $v^F_{i,k|k}$.

In the proposed distributed GM-PHD filter, each sensor performs the local prediction and measurement update steps according to the GM-PHD filtering algorithm \cite{vo2006gmphd}. The GM-PHD filter uses Gaussian mixtures (GMs) to represent the intensity functions, admitting a closed-form solution to the PHD recursion. Consequently, the GM-PHD filter is optimal in a Bayesian sense, under reasonable assumptions on the target and sensor models; a more complete discussion of the relevant assumptions may be found in \cite{vo2006gmphd}. In practice, the GM-PHD filter is approximated by pruning small GM components to keep the space complexity of the algorithm from growing over time.

Given $x_{k-1} \in \mathcal X$, let the prior intensity $v_{i,k|k-1}$
%, be represented using GMs with $J_{i,k|k-1}$, $J_{\gamma,k}$ and $J_{\sigma,k}$ components, respectively. The GM representations of these intensities are as follows:
be represented using a GM with $J_{i,k|k-1}$ components, as follows:
\begin{align}
    v_{i,k|k-1}(x) &= \sum_{l=1}^{J_{i,k|k-1}}w^{(l)}_{i,k|k-1}\mathcal N(x;m^{(l)}_{i,k|k-1}, P^{(l)}_{i,k|k-1})
    \label{eq:prior_intensity}
%    \gamma_k(x) &= \sum_{l=1}^{J_{\gamma,k}} w^{(l)}_{\gamma,k}\mathcal N(x;m^{(l)}_{\gamma,k}, P^{(l)}_{\gamma,k})\\
  %  \sigma_{k|k-1}(x|x_{k-1}) &= \sum_{l=1}^{J_{\sigma,k}} w^{(l)}_{\sigma,k}\mathcal N(x;F^{(l)}_{\sigma,k-1}x_{k-1}, Q^{(l)}_{\sigma,k-1})
\end{align}
Sensor $i$'s prior estimate of the number of targets is $\sum_{l=1}^{J_{i,k|k-1}} w_{i,k|k-1}^{(l)}$, as this sum corresponds to the integral of the intensity $v_{i,k|k-1}(x)$.

The birth and spawn intensities ($\gamma_k$ and $\sigma_{k|k-1}({}\cdot{}|x_{k-1})$, respectively) are also represented by GMs whose parameters are assumed to be known. The prior intensity $v_{i,k|k-1}$ includes the surviving targets from timestep $k-1$ as well as new targets that were born or spawned at timestep $k$. Since these correspond to independent Poisson RFSs, we can add the corresponding intensity functions as follows:
\begin{equation}
    v_{i,k|k-1}(x) = v_{i,k|k-1}^S(x) + v_{i,k|k-1}^\sigma(x) + \gamma_{k}(x),
\end{equation}
where $v^S_{i,k|k-1}$ corresponds to the targets that survived from timesteps $k-1$ to $k$ and $v_{i,k|k-1}^{\sigma}$ corresponds to newly spawned targets. Similarly, the posterior intensity is computed as a sum of two intensities:
\begin{equation}
    v_{i,k|k}(x) = (1 - p_{D,k}(x)) v_{i,k|k-1}(x) + \sum_{z \in Z_{k}} v_{i,k}^D(x;z)
\end{equation}
where the first term corresponds to the targets which were not detected and $v^{D}_{i,k}(x;z)$ is the new information obtained through the measurement $z\in Z_{k}$. Further details about the computation of prior and posterior intensities ($v_{i,k|k-1}$ and $v_{i,k|k}$, respectively) can be found in \cite{vo2006gmphd}. 

We refer to the computation of $v_{i,k|k-1}$ and $v_{i,k|k}$ using the local measurements $Z_{i,k}$ at sensor $i$ as local GM-PHD filtering.
% \begin{equation}
%     v_{i,k|k-1}^S(x) =  \sum_{l=1}^{J_{i,k-1|k-1}} p_{S,k}(x) w_{i,k-1|k-1}^{(l)} \mathcal{N}(x; m_{S,k|k-1}^{(l)}, P_{S,k|k-1}^{(l)})
% \end{equation}
% \begin{equation}
%     v_{i,k|k-1}^{\sigma}(x) = \sum_{l=1}^{J_{i,k-1|k-1}} \sum_{j=1}^{J_{\sigma,k}} w_{i,k-1|k-1}^{(l)} w_{\sigma,k}^{(j)} \mathcal{N}(x; m_{\sigma,k|k-1}^{(l,j)}, P_{\sigma,k|k-1}^{(l,j)})
% \end{equation}
% \begin{equation}
%     \gamma_{i,k} (x) = \sum_{l=1}^{J_{\gamma,k}} w_{\gamma,k}^{(l)}(x) \mathcal{N}(x; m_{\gamma,k}^{(l)}, P_{\gamma,k}^{(l)}).
%     \label{eq_birth intensity}
% \end{equation}
% With the measurement set $Z_{k}$ at time $k$, the multi-target update intensity $v_{k}(x)$ is also modeled as a Gaussian mixture by
% \begin{equation}
%     v_{i,k|k}(x) = v_{i,k}^{ND}(x) + \sum_{z \in Z_{k}} v_{i,k}^D(x;z)
% \end{equation}
% where $v_{ND,k}(x)$ corresponds to the undetected targets, and $v_{D,k}(x;z)$ corresponds to the contributions from measurement $z \in Z_{k}$, and estimated as
% \begin{equation}
%     v_{i,k}^{ND}(x) = (1 - p_{D,k}(x)) v_{i,k|k-1}(x)
% \end{equation}
% \begin{equation}
%     v_{i,k}^D(x;z) = \sum_{l=1}^{J_{i,k|k-1}} p_{D,k}(x) w_{k|k}^{(l)}(z) \mathcal{N}(x; m_{i,k|k}^{(l)}(z), P_{i,k|k}^{(l)}).
% \end{equation}
In the remainder of this paper, we focus on developing a distributed GM dissemination and fusion protocol for improving the multi-target tracking performance of each sensor. Our proposed distributed GM-PHD filtering algorithm constitutes the local GM-PHD filtering step followed by one or more inter-sensor fusion steps. 

\section{Fusion of Local GM-PHD Filters}
\label{sec:fusion_step}

In addition to local GM-PHD filtering, the communication channels, i.e., the edges in $\mathcal E$, can serve as additional sources of information for each sensor.
In theory, the optimal solution to the given multi-sensor multi-target problem can be realized by aggregating all the measurements of the sensor network as $\bigcup_{i\in \mathcal V}Z_{i,k}$ at each sensor, at each timestep. Since the communication and computational cost of such an approach does not scale well with the number of sensors in the network (which is equal to $|\mathcal V|$), it is infeasible when one or more of the following challenges are taken into consideration: 
\begin{enumerate}
    \item the total number of sensors in the network is large,
    \item the inter-sensor communication channels have limited bandwidths, or 
    \item the communication is not instantaneous, i.e., there are significant delays between the transmission and reception of inter-sensor communications.
\end{enumerate}
A distributed solution to the multi-sensor multi-target tracking problem can solve each of the preceding challenges. Distributed PHD filtering is achieved by fusing the intensity functions $v_{i,k|k}$ of the individual sensors, as these are typically much smaller in dimension than the measurement sets $Z_{i,k}$. For instance, the measurements in $Z_{i,k}$ may be high-dimensional camera images, whereas $v_{i,k|k}$ has a concise representation in terms of its GM components.

\subsection{Weighted Arithmetic Average (WAA) Fusion}
The weighted arithmetic average (WAA) of the local posterior intensity functions of the sensors is defined as
\begin{equation}
    v^{*}_{k|k}(x) = \sum_{i \in \mathcal V} \omega_i v_{i,k|k}(x)
    \label{eq:WAA_def}
\end{equation}
where $\omega_i \geq 0$ and $\sum_{i\in \mathcal V} \omega_i = 1$. In \cite{da2020kullback}, it was shown that the WAA (\ref{eq:WAA_def}) minimizes the weighted Kullback-Leibler (KL) divergence between itself and the local posterior intensities, $v_{i,k|k}$, when the fused intensity is taken as the second argument of the divergence. Similarly, it was shown in \cite{gostar2017cauchy} that the WAA also minimizes the weighted Cauchy-Schwarz (CS) divergence (which is symmetric in its arguments). Thus, we have
\begin{align}
    v^{*}_{k|k}(x) &= \arg \min_{g} \left( \sum_{i\in \mathcal V} \omega_i D_{KL}\left(v_{i,k|k}(x)\mathrel{\Vert}g(x)\right)\right) \\&= \arg \min_{g}\left( \sum_{i\in \mathcal V} \omega_i D_{CS}\left(v_{i,k|k}(x)\mathrel{\Vert}g(x)\right)\right)
    \label{eq:csd}
\end{align}
where $D_{KL}$ and $D_{CS}$ denote the KL and CS divergences, respectively.
Either choice of divergence can be considered as a cost function for the local PHD fusion process, analogous to how the weighted Euclidean distance is used as the cost function in single-target tracking. By minimizing the divergence, the information gain between the local posterior intensities and the fused intensities is minimized; so the WAA (\ref{eq:WAA_def}) is optimal in terms of the \textit{principle of minimum discrimination of information (PMDI)} \cite{gostar2017cauchy, gao2020multiobject}.

The weights $\omega_i$ can be chosen as $\omega_i =\sfrac{1}{|\mathcal V|}$ if the sensor network is homogeneous, i.e., all the sensors have the same sensing capability, measurement noise covariance and detection probability. When this is not the case, a larger weight can be assigned to sensors having better sensing capability. The connectivity of the sensor network $\mathcal G$ can also inform the choice of $\omega _i$. 
% The design of the weights $\omega _i$ is not developed further in this paper, serving as an avenue for future research.

Additionally, note that by taking an integral on both sides of (\ref{eq:WAA_def}), we have

\begin{equation}
    \int_{\mathcal X} v^{*}_{k|k}(x)\hspace{1pt}dx = \sum_{i \in \mathcal V} \omega_i \left(\int_{\mathcal X} v_{i,k|k}(x)\hspace{1pt}dx\right)
    \label{eq:integral_WAA}
\end{equation}
From the properties of Poisson RFS intensities, it follows that the integral of an intensity is the expected value of the cardinality of the corresponding RFS. Thus, WAA fusion of the intensities $v_{i,k|k}$ entails WAA fusion of the cardinality estimates of the sensors. The converse is not true, as two different functions may integrate to the same value. Thus, the proposed approach is different from the one used in \cite{li2018cardinality}, which directly computes the WAA of the cardinality estimates.

\subsection{Distributed WAA Fusion using Consensus}

To realize WAA fusion at each sensor in a distributed manner, an average (weighted) consensus algorithm can be used. In each iteration of the average consensus algorithm, the sensors communicate the GM components of their posterior intensity to their out-neighbors.
Thereafter, each sensor fuses its posterior intensities with those of its in-neighbors using a weighted combination, where the weights $\Omega_{ij}\in \mathbb R$ correspond to the entries of a matrix $\Omega = [\Omega_{ij}]$, with $\Omega_{ij}\neq 0$ if and only if $(j,i)\in \mathcal E$.
The average consensus algorithm is described in Algorithm \ref{alg:con}, which uses a total of $\alpha \geq 1$ inter-sensor fusion iterations.

\begin{algorithm}
\caption{Average Consensus of Posterior Intensities}
\begin{algorithmic}[1]
\vspace{2pt}
\REQUIRE The consensus weights $\Omega_{ij}$, posterior intensities $v_{i,k|k}$, and number of inter-sensor fusion steps $\alpha \geq 1$.\\
\vspace{3pt} 
\STATE Set $v_{i,k|k}^{(0)} \leftarrow v_{i,k|k} \forall i\in \mathcal V$
\FOR{$l=1,2,\dots,\alpha$}
\STATE Each sensor communicates its fused posterior intensity $v^{(l-1)}_{j,k|k}(x)$ to its out-neighbors, $N^+_i$. \vspace{2pt}
\STATE Update the posterior intensities as \begin{equation}
    v^{(l)}_{i,k|k}(x) \leftarrow \sum_{j \in N^{-}_i \cup \lbrace i\rbrace} \Omega_{ij} \hspace{2pt} v^{(l-1)}_{j,k|k}(x)
    \label{eq:average_consensus}
\end{equation}
$\forall i\in\mathcal V$.
\ENDFOR
\RETURN The fused posterior intensity, $v^{(\alpha)}_{i,k|k}(x)$.
\end{algorithmic}
\label{alg:con}
\end{algorithm}

\subsubsection{Error Analysis of Algorithm \ref{alg:con}}
\label{sec:subsec_error_analysis}
At the $l^{th}$ iteration of the average consensus algorithm (Algorithm \ref{alg:con}),
the error between the $i^{th}$ sensor's posterior intensity and the WAA is $v^{(l)}_{i,k|k}-v^*_{k|k}$. To establish the convergence properties of the algorithm, we need to show that the average consensus step (\ref{eq:average_consensus}) drives this error to $0$ (i.e., $v^{(l)}_{i,k|k}-v^*_{k|k}$ approaches the function which is $0$ everywhere in its domain) at each of the sensors.
To facilitate this analysis, let us assume that the intensities $v_{i,k|k}$ are elements in the $L^p(\mathcal X)$ function space, i.e., for some $1\leq p\leq \infty$,
\begin{equation}
\|v_{i,k|k}\|_{L^p} =  \left(\int_\mathcal X |v_{i,k|k}(x)|^p dx\right)^{\frac{1}{p}}< \infty
\end{equation}
$\forall i\in \mathcal V$. Recall that the integral of $v_{i,k|k}$ is sensor $i$'s posterior estimate of the cardinality of the RFS $X_k$. If $p=1$, then $v_{i,k|k}\in L^1(\mathcal X)$ if and only if sensor $i$'s cardinality estimate is bounded. Hence, the assumption that $v_{i,k|k}\in L^p(\mathcal X)$  is reasonable.

We define the vector space $V = \left(L^p(\mathcal X)\right)^{|\mathcal V|}$, so that we can collectively represent the set of intensities $\lbrace v_{i,k|k}\rbrace_{i\in\mathcal V}$ as a vector in $V$:
\[\bar v_{k|k} = \begin{bmatrix} v_{1, k|k} & v_{2, k|k} & \dots & v_{|\mathcal V|, k|k}\end{bmatrix}^\intercal \in V\]
For vectors in $V$, the addition and scalar multiplication operations are defined as the pointwise addition and pointwise scalar multiplication of their components.
We endow $V$ with a norm $\|\mathrel\cdot\|_V$, defined as
\begin{equation}
\| \bar v_{k|k}\|_V = \left(\sum_{i=1}^{|\mathcal V|} \|v_{i,k|k}\|_{L^p} ^p\right)^{\frac{1}{p}}
\end{equation}
Thus, $(V, \|\mathrel\cdot\|_V)$ is a normed vector space. For a matrix $A$, let $\|A\|_{\textsc{op}}$ refer to its operator norm\footnote{If we set $p=2$ in the foregoing definitions, then $V$ can be made into a Hilbert space, in which case $\|A\|_{\textsc{op}}$ is equal to the largest singular value of $A$ \cite[Thm. 4, p. 40]{halmos1957introduction}.}, given by
\begin{equation}
\|A\|_{\textsc{op}} = \sup \left\lbrace \frac{\|A v\|_V}{\|v\|_V} \mathrel{:} v\neq 0, v\in V \right\rbrace
\end{equation}
% We discuss the convergence of the average consensus step (\ref{eq:average_consensus}). 
% This type of abstraction is commonly used in the literature on consensus-based filtering, which either assumes that the local filters have reached their steady states, or that a large number of average consensus steps are carried out between successive measurement steps \cite{li2018partial, consensus_ukf}.
With the above definitions in place, we can state the conditions for the convergence of the average consensus step (\ref{eq:average_consensus}). 

\begin{theorem}
Suppose the following conditions are satisfied:
\begin{enumerate}
%\item $\Omega_{ij} \geq 0$ and $\Omega_{ii}>0$ ${\forall i,j\in\mathcal V}$
%\item $\Omega_{ij}\neq0$ if and only if $(j,i)\in \mathcal E$
\item $\Omega \mathbbm 1 = \mathbbm 1$, where $\mathbbm 1 = \begin{bmatrix}
1 & 1 & \dots & 1\end{bmatrix}^\intercal$,
% i.e., $\Omega$ is row stochastic
%
\item $\bar \omega^\intercal \Omega = \bar \omega^\intercal$, where $\bar \omega  = [\omega _1, \omega _2, \dots, \omega _{|\mathcal V|}]^\intercal$, and
%
\item $\|{\Omega - \mathbbm 1 \bar{\omega}^\intercal} \|_{\textsc{op}} < 1$,
% references: halmos1957introduction, bartle2014elements
\end{enumerate}
then 
% by a repeated application of the average consensus step (\ref{eq:average_consensus}), the intensities 
the fused posterior intensities $\lbrace v_{i,k|k}^{(\alpha)} \rbrace_{i\in \mathcal V}$ computed using Algorithm \ref{alg:con}
asymptotically converge to the WAA in the $L^p$ norm:
\begin{align}
   \lim_{\alpha\rightarrow \infty} \| v^{(\alpha)}_{i,k|k} - v^{*}_{k|k}\|_{L^p} \rightarrow 0, \quad \forall i\in\mathcal V
    \label{eq:convergence}
\end{align}
As a consequence, there is a subsequence $\lbrace l_1, l_2, \dots \rbrace$ such that $\lim_{s\rightarrow \infty} v^{(l_s)}_{i,k|k} = v^{*}_{k|k}$ almost everywhere \cite[p. 75]{bartle2014elements}.
\end{theorem}
\begin{proof}
To show (\ref{eq:convergence}), we can rewrite the average consensus step (\ref{eq:average_consensus}) as $ \bar v^{(l)}_{k|k} = \Omega \bar v^{(l-1)}_{k|k}$. 
%The WAA can be expressed as $v^*_{k|k} = \bar \omega ^\intercal \bar v^{(0)}_{k|k}$.  
Observe that,  given $ l \geq 1$
\begin{equation}
\bar \omega ^\intercal \bar v^{(l)}_{k|k} = \bar \omega ^\intercal \Omega \bar v^{(l-1)}_{k|k} = \bar \omega ^\intercal \bar v^{(l-1)}_{k|k}
\end{equation}
By induction, we have that $\bar \omega ^\intercal \bar v^{(l)}_{k|k} = \bar \omega ^\intercal \bar v^{(0)}_{k|k} = v^*_{k|k}$; in other words, the weighted average is invariant under the average consensus step (\ref{eq:average_consensus}).
Thus, the error dynamics corresponding to the average consensus step is
\begin{align}
\bar v^{(l+1)}_{k|k} - \mathbbm 1 \bar v^*_{k|k} &= \Omega \bar v^{(l)}_{k|k} - \mathbbm 1  \bar \omega ^\intercal \bar v^{(l)}_{k|k}\\
&=\left(\Omega  - \mathbbm 1  \bar \omega ^\intercal \right) \bar v^{(l)}_{k|k} \\
&=\left(\Omega  - \mathbbm 1  \bar \omega ^\intercal \right) (\bar v^{(l)}_{k|k}-\mathbbm 1 \bar v^*_{k|k})
\end{align}
where, in the last step, we used the facts that $\Omega \mathbbm 1 = \mathbbm 1$ and $\bar \omega ^\intercal \mathbbm 1 = 1$. Consequently, we have
\begin{equation}
\|\bar v^{(l)}_{k|k} - \mathbbm 1 \bar v^*_{k|k}\|_{V} \leq \|{\Omega - \mathbbm 1 \bar{\omega}^\intercal} \|_{\textsc{op}}^l \|\bar v^{(0)}_{k|k} - \mathbbm 1 \bar v^*_{k|k}\|_{V}
\end{equation}
which implies (\ref{eq:convergence}). 
\end{proof}

Lastly, we remark that a finite number of average consensus steps can be used in practice (i.e., $\alpha < \infty)$, with $\|{\Omega - \mathbbm 1 \bar{\omega}^\intercal} \|_{\textsc{op}}$ dictating the rate of convergence. Using (\ref{eq:csd}), it can be seen that each weighted average consensus iteration lowers the Cauchy-Schwarz divergence between the sensors' posterior intensities (which is a measure of their information difference \cite{gostar2017cauchy}).
%$\bar \omega ^\intercal \Omega = \bar ^\intercal$.

\subsubsection{Comparison with Existing Results}
Our analysis in this section extends the existing results on consensus (which has largely only considered consensus of vectors in Euclidean spaces)
to the case where consensus is sought on functions in $L^p$. For consensus in Euclidean spaces, a set of sufficient conditions for convergence is the following: 
\begin{itemize}
\item $\Omega_{ij} \geq 0$ whenever $(j,i)\in \mathcal E$,
\item $\Omega_{ii}>0$ ${\forall i\in\mathcal V}$, and 
\item the graph $\mathcal G$ is strongly connected
\end{itemize}
in which case, the fact that $\|{\Omega - \mathbbm 1 \bar{\omega}^\intercal} \|_{\textsc{op}}<1$ follows from the Perron-Frobenius theorem \cite[Lemma 1]{corless2012consensus}. Thus, the foregoing sufficient conditions for convergence are stronger than the ones we obtained in Section \ref{sec:subsec_error_analysis}.
Using the protocol in \cite{corless2012consensus}, the weights $\Omega_{ij}$ may be chosen in a distributed manner, i.e., each sensor is able to choose the weights in real-time without needing global knowledge of the topology of $\mathcal G$.

% However, note that the literature on distributed average consensus typically assumes that consensus is sought on scalars or vectors in a Euclidean space, which are equipped with the operations of scalar multiplication and addition. As distributed GM-PHD requires WAA of GMs instead, it should be ensured that these operations make sense for GMs as well. Indeed, pointwise convergence of (\ref{eq:average_consensus}) can be shown as follows. Consider any point $x\in \mathcal X$ in the domain, and note that $v_{i,k|k}(x) \in \mathbb R$. At the given value of $x$, (\ref{eq:average_consensus}) can interpreted as solving the WAA problem in the space $\mathbb R$, rather than in a function space.
% Thus, we have pointwise convergence of the recursion (\ref{eq:average_consensus}):
% \[v_{i,k|k}(x) \rightarrow v_{k|k}^*(x) \ \forall x\in \mathcal X\]
% $\forall i \in \mathcal V$, which follows directly from the standard results on consensus in $\mathbb R$.
%\cite{li2018partial, consensus_ukf}.


% To see that convergence properties of (\ref{eq:average_consensus}) indeed follow from those of average consensus in Euclidean spaces, note that the Gaussian distribution is square-integrable, so the function space of GMs is a Hilbert space (i.e., the $L^2$ space) having the (pointwise) addition and scalar multiplication operations as well as an inner product operation. The convergence proofs of the average consensus algorithms can therefore be adapted verbatim for GMs, so that the convergence of (\ref{eq:average_consensus}) (in
% the $L^2$ function norm)
% holds. 

% Such a result can be formalized using the approach of \cite{probability_consensus2014}, in which the authors study consensus in the function space of probability measures.

\subsection{Limitations of the Average Consensus Approach}
Although the average consensus algorithm (\ref{eq:average_consensus}) guarantees asymptotic convergence to the WAA, it has several limitations. Firstly, it requires each sensor to transmit $J_{i,k|k}$ GM components to its out-neighbors. This limits the number of targets that can be tracked simultaneously by the sensor network when the communication channels between the sensors have limited bandwidth, as is usually the case in wireless sensor networks. Secondly, the large number of GM components accumulated at each sensor increase the computational complexity of the subsequent local PHD filtering steps. Lastly, as noted in \cite{li2018partial}, the consensus step can exacerbate the problem of false positives in WAA fusion; small GM components corresponding to the false positive detections (i.e., clutter measurements which are misidentified as targets) can propagate through the cycles (closed loops) of the graph $\mathcal G$, leading to an overall feedback effect. In this way, false positives can get amplified during the fusion step, reducing the estimation performance of the sensor network. We address each of these concerns in the next section, by proposing a distributed protocol for dissemination and fusion of GM components that ensures asymptotic consensus and suppression of false positives, while having the same (or lower) communication bandwidth requirement than the existing distributed PHD filtering algorithms.

\section{Multi-Sensor Fusion under Communication Constraints}
\label{sec:comm_constr}

In order to carry out the fusion step of (\ref{eq:average_consensus}), the sensors must transmit a certain number of GM components to their out-neighbors. Since the dimension of the state-space $\mathcal X$ is fixed as $d_x$, each GM component is specified by $d_x + \tfrac{1}{2}d_x(d_x + 1) + 1$ floating point numbers, where the individual terms correspond to the mean vector, the covariance (which is a symmetric matrix), and the scalar weight of the GM component, respectively. In this section, we consider the case where the communication bandwidth is limited, such that the maximum number of GM components that may be transmitted across the inter-sensor communication channels at a given timestep is denoted as $B$, with $B \geq 1$.
In order to limit the communication complexity of the distributed GM-PHD algorithm, the authors of \cite{li2018partial} and \cite{li2020parallel} proposed the following heuristics for selecting the $J_{i,k|k}$ GM components that are to be transmitted at each sensor:
\begin{itemize}
    \item \textit{Rank Rule}: Rank the GM components based on their weights (highest to lowest) and select the top $B$ components
    \item \textit{Threshold Rule}: Fix a threshold, and transmit any GM components that have weights greater than this threshold
\end{itemize}
In either case, the GM components with small weights are not transmitted. This has the advantage of suppressing false positives, which typically correspond to the components having small weights. However, convergence to the WAA (\ref{eq:WAA_def}) cannot be guaranteed if the rank or threshold rule is used, as the GM components with small weights are never communicated between sensors.
Consequently, this approach is referred to as \textit{partial consensus} by its authors.

In order to guarantee asymptotic convergence of the average consensus algorithm to the WAA, we propose a new rule for selecting the GM components, which we call the \textit{sampling rule}. For simplicity, we assume that the communication is based on wireless broadcast, i.e., each sensor transmits the same message to all of its out-neighbors. Let $\mathcal B_{i}$ be a sequence of indices (numbers between $1$ and $J_{i,k|k}$) corresponding to the GM components transmitted by sensor $i$ to its out-neighbors at timestep $k$, where we omit the timestep $k$ for brevity. The total number of distinct indices in $\mathcal B_i$ is at most $B$, as per the communication constraint. The indices in $\mathcal B_i$ are chosen through random sampling, where the sampling is carried out either with or without replacement.

\subsection{Sampling with Replacement}
In order to construct $\mathcal B_i$, sensor $i$ generates a sequence of random samples of indices with replacement. In each sample,
the index $l$ is chosen with probability $p_{i}^{(l)}$, with 
\begin{equation}
    \sum_{l=1}^{J_{i,k|k}} p_{i}^{(l)} = 1
\end{equation}
%
Thus, the elements of $\mathcal B_i$ are $i.i.d.$ samples of a categorical random variable and can be thought of as the outcomes of rolling a $J_{i,k|k}$-sided biased die $|\mathcal B_i|$ times. Due to the limited communication bandwidth, the sampling process is terminated when the number of distinct indices in $\mathcal B_i$ equals $B$.
Let $\zeta^{(l)}_{i}$ denote the number of times that the index $l$ appears in $\mathcal B_i$. Observe that $(\zeta^{(1)}_{i}, \zeta^{(2)}_{i}, \dots, \zeta^{(|J_{i,k|k}|)}_i)$ follow the multinomial distribution, with $\mathbb E[\zeta^{(l)}_{i}]=p_{i}^{(l)} |\mathcal B_i|$.

Once $\mathcal B_i$ is generated, the sensor transmits the corresponding GM components to its out-neighbors. If there are any repeated components, the sensor transmits each of these components (in addition to the number of times they were sampled) only once, in order to avoid redundancy in the communication. Thus, each sensor transmits a random function (i.e., a random GM) to its out-neighbors at each timestep, given by
\begin{align}  &\sum_{l=1}^{J_{i,k|k}}\zeta^{(l)}_{i} \hspace{2pt} \tilde w^{(l)}_{i,k|k}\mathcal N(m^{(l)}_{i,k|k}, P^{(l)}_{i,k|k})
    \label{eq:random_GM}
\end{align}
where $\mathcal N(m,P)$ denotes a GM component having the mean vector $m\in \mathbb R^{d_x}$ and covariance $P\in \mathbb R^{d_x \times d_x}$.
Note that the weights of the transmitted GM components $\tilde w^{(l)}_{i,k|k}$ are different from the weights used in the local GM-PHD filtering, $w^{(l)}_{i,k|k}$.
To ensure that the fixed point of the average consensus algorithm is preserved (in expectation) by the sampling rule, we require that the expected value of the random GM is equal to the posterior intensity of sensor $i$, i.e.,
\begin{align}
\mathbb E\left[\sum_{l=1}^{J_{i,k|k}}\zeta^{(l)}_{i} \hspace{2pt} \tilde w^{(l)}_{i,k|k}\mathcal N(m^{(l)}_{i,k|k}, P^{(l)}_{i,k|k})\right] = v_{i,k|k}
\label{eq:exp_randomGM}
\end{align}
% In addition, the law of large numbers can be used to guarantee asymptotic (pointwise) convergence of the random GMs to the true posterior intensity $v_{i,k|k}$.
% From (\ref{eq:exp_randomGM}), 
Equivalently,
we have
\begin{align}
\mathbb E[\zeta^{(l)}_{i}]\hspace{2pt}\tilde w^{(l)}_{i,k|k} = w^{(l)}_{i,k|k} \label{eq:exp_weight_condition}\\ \hspace{1pt}  \tilde w^{(l)}_{i,k|k} = \frac{w^{(l)}_{i,k|k}}{p_{i}^{(l)}|\mathcal B_i|}
\label{eq:weight_condition}
\end{align}
Equation (\ref{eq:weight_condition}) imposes a constraint on the sampling probabilities $p_{i}^{(l)}$ and the weights $\tilde w_{i,k|k}^{(l)}$, but does not uniquely specify them. 
We propose the following choice of sampling probabilities:
\begin{equation}
p_{i}^{(l)}=\frac{w^{(l)}_{i,k|k}}{\sum_{l=1}^{J_{i,k|k}} w^{(l)}_{i,k|k}}
\label{eq:pl_choice}
\end{equation}
for which the corresponding weights are given by (\ref{eq:weight_condition}), as
\begin{equation}
\tilde w^{(l)}_{i,k|k} = \frac{\sum_{l=1}^{J_{i,k|k}} w^{(l)}_{i,k|k}}{|\mathcal B_i|} \triangleq \tilde w_{i,k|k}
\label{eq:final_w_tilde_def}
\end{equation}
The proposed choice of sampling probabilities, given by (\ref{eq:pl_choice}) and (\ref{eq:final_w_tilde_def}), has the clear advantage that the weights $\tilde w^{(l)}_{i,k|k}$ no longer need to be transmitted for each component, a single number $\tilde w_{i,k|k}$ may be transmitted instead, further reducing the communication requirement of the algorithm. Additionally, observe that if the sampling probabilities $p_{i}^{(l)}$ are chosen as per (\ref{eq:pl_choice}), then GM components with higher weights are transmitted more often than those with smaller weights,
%Thus, the GM components with high weights converge (as per the law of large numbers) faster than the components with small weights. 
so that the components with small weights only survive the fusion step if the targets corresponding to them are persistently detected. In this way, false positive detections that do not survive successive local PHD filtering steps are suppressed by the inter-sensor fusion step. The ability of the algorithm to suppress false alarms as well as the reduced communication cost motivate the choice of $p_{i}^{(l)}$ as (\ref{eq:pl_choice}).

\begin{algorithm}
\caption{Random Sampling Rule (with Replacement)}
\begin{algorithmic}[1]
\vspace{2pt}
\REQUIRE The maximum allowable number of GM components, $B\geq 1$ \\
\vspace{3pt} 
\WHILE{$\mathcal B_i$ has at most $B$ distinct indices}
\STATE Randomly sample an index from the sequence $\left(1, 2, \dots, J_{i,k|k}\right)$ with the corresponding sampling probabilities as given by (\ref{eq:pl_choice}).
\ENDWHILE
\STATE Each sensor broadcasts the following GM to its out-neighbors:
\[\tilde w_{i,k|k} \sum_{l=1}^{J_{i,k|k}}\zeta^{(l)}_{i} \hspace{2pt} \mathcal N(m^{(l)}_{i,k|k}, P^{(l)}_{i,k|k})
\]
where $\zeta^{(l)}_i$ is the number of times $l$ occurs in $\mathcal B_i$.
\end{algorithmic}
\label{alg:sampling}
\end{algorithm}
Thus, random sampling (as described in Algorithm \ref{alg:sampling}) can be used to replace step $3$ of Algorithm \ref{alg:con}, thereby limiting the communication cost of the distributed GM-PHD filter. From step $4$ of Algorithm \ref{alg:sampling}, we see that at each sensor, the sampling rule requires the  communication of $B$ GM components, $1$ floating point number (which is $\tilde w_{i,k|k}$), and less than $B$ integers (i.e., the numbers $\zeta_i^{(l)}$) at each iteration of the average consensus step.

Observe that, even though each sensor communicates a random GM to its neighbors, the integral of the GM is a deterministic quantity:
\begin{align}   
\tilde w_{i,k|k} \sum_{l=1}^{J_{i,k|k}}\zeta^{(l)}_{i} = \tilde w_{i,k|k} |\mathcal B_i| = \sum_{l=1}^{J_{i,k|k}} w^{(l)}_{i,k|k}
\end{align}
where in the second equality, we used (\ref{eq:final_w_tilde_def}).
Thus, while the posterior intensities of the sensors asymptotically converge to the WAA in expectation, the integrals of the intensities (i.e., the cardinality estimates of the sensors) converge deterministically.

\subsection{Sampling without Replacement}
An alternative strategy for constructing $\mathcal B_i$ is to sample the indices without replacement. As each index occurs in $\mathcal B_i$ at most once, $\mathcal B_i$ can be interpreted as a set. The random variable $\zeta^{(l)}_{i}$ defined in the previous section becomes the indicator variable $\mathbbm 1_{\lbrace l\in \mathcal B_i \rbrace}$, where
\begin{align}
    \mathbbm 1_{\lbrace l\in \mathcal B_i \rbrace} = \begin{cases}
    \begin{array}{cl}
         1& \text{if label}\ l\ \text{is in }\mathcal B_i,  \\
         0& \text{otherwise.}
    \end{array}
    \end{cases}
\end{align}
so that the condition (\ref{eq:exp_weight_condition}) now becomes
\begin{equation}
    \mathbb E[\mathbbm 1_{\lbrace l\in \mathcal B_i \rbrace}]\hspace{2pt}\tilde w^{(l)}_{i,k|k} = P (l\in \mathcal B_i )\hspace{2pt}\tilde w^{(l)}_{i,k|k} = w^{(l)}_{i,k|k} 
\end{equation}
where $P({}\cdot{})$ denotes the probability of an event. Thus, once the 
probabilities $P (l\in \mathcal B_i )$ are computed,
the weights $\tilde w^{(l)}_{i,k|k}$ must be chosen accordingly, as
\begin{equation}
    \tilde w^{(l)}_{i,k|k} = \frac{w^{(l)}_{i,k|k}}{P( l\in \mathcal B_i )}
    \label{eq:no_replacement_weights}
\end{equation}
An efficient method for sampling without replacement can be found in \cite{Efraimidis2015}, which is able to assign a higher probability to components with higher weights, thereby incorporating the ability to suppress false positives into the fusion step.

Note that it is no longer straightforward to specify the inclusion probabilities of the indices $P( l\in \mathcal B_i )$ \textit{a priori}, like we did in (\ref{eq:pl_choice}). To see this, consider the case where $B=J_{i,k|k}$, which fixes the probabilities as $P( l\in \mathcal B_i )=1$ for all the indices. Thus, if we were to specify the probabilities $P( l\in \mathcal B_i )$ \textit{a priori}, it can make the sampling problem infeasible; the same is true even when $B<J_{i,k|k}$ \cite[Example 2]{Efraimidis2015}. Moreover, when using the sampling without replacement rule, in addition to transmitting the mean vectors $m^{(l)}_{i,k|k}$ and covariances $P^{(l)}_{i,k|k}$ of the selected components, the sensors must transmit the modified weights $\tilde w_{i,k|k}$ as well. In contrast, the proposed sampling with replacement rule (given in Algorithm \ref{alg:sampling}) allows one to specify the sampling probabilities \textit{a priori}, and also has a lower communication bandwidth requirement.
% $P( l\in \mathcal B_i )$ is also more difficult to compute, since the probability of a given GM component being chosen is dependent on the probability that

In summary, the proposed distributed GM-PHD algorithm uses a local GM-PHD filter at each sensor to update its multi-target estimate, followed by a given number of average consensus steps (as per Algorithm \ref{alg:con}) to fuse the multi-target estimates of the sensors. Additionally, the proposed random sampling rule (given in Algorithm \ref{alg:sampling}) is used to limit the communication bandwidth of the approach, while ensuring its asymptotic optimality.

\section{Simulations}
\label{sec:simulations}

We demonstrate the performance of the algorithms in Section~\ref{sec:three_users} with simulations.  We consider distortions for which $d_i = \epsilon_i^{2}$ for all $i \in \mathcal{U}$.  In Fig.~\ref{fig:t_star}, we choose a blocklength of $N=10^7$, fix $\epsilon_1 = 0.3$, $\epsilon_2 = 0.4$ and vary $\epsilon_3$ on the $x$-axis while plotting the total number of channel symbols sent per source symbol on the $y$-axis for different coding schemes.  
%instantly decodable, distortion-innovative transmissions sent on the $y$-axis.  
%
%\newlength\figureheight 
%\newlength\figurewidth 
\begin{figure}
	\centering
	\setlength\figurewidth{2.65in} 
\setlength\figureheight{2.07in} 
%	\includegraphics[scale=0.5]{fig/tstar_N_10e7_eps1_3_eps2_4-fixed}
	\input{fig/simulations/analog.tikz}
	\caption{The normalized number of channel symbols sent per source symbol.  We fix $\epsilon_1=0.3$, $\epsilon_2=0.4$ and vary $\epsilon_3$ on the x-axis.  We show the number of uncoded transmissions sent via~\eqref{eq:LP}, alongside what is obtained from simulations  for a blocklength of $N=10^7$.  Finally, we also plot the latency required to achieve the equivalent distortion values \emph{without} feedback based on a segmentation-based coding scheme.}
%	\caption{The total normalized number of instantly decodable, distortion-innovative packets that can be sent.}
	\label{fig:t_star}
\end{figure}
%
The first coding scheme plotted is based on simulations and plots the total number of instantly decodable, distortion-innovative transmissions sent.  Alongside this curve, we plot the number of possible innovative transmissions suggested by the solution of~\eqref{eq:LP}.  We observe a close resemblance in this plot and the empirical simulation curve. Finally, we know that although each user may not have their final distortion constraint met after $Nt^{*}$ transmissions, the provisional distortion they \emph{do} achieve after $Nt^{*}$ transmissions is optimal.  Say instead, that we were given these provisional distortion values from the onset and asked what latency would be required to achieve these distortions if feedback were not available.  The final plot shows this required latency for the segmentation-based coding scheme of~\cite{LTKS_ISIT14}, which does not incorporate feedback. The gap between these curves is indicative of the benefit that feedback provides.
%In order to achieve the same distortion values with We also plot the latency required by the segmentation-based coding scheme of~\cite{} that does not incorporate feedback, if it were used to achieve the optimal distortions at $t^{*}$.  The gap between these curves is indicative of the benefit that feedback allows.

In practice, the values of $\epsilon_i$ are much lower than what we have chosen.  The values for $\epsilon_3$, for example, were deliberately chosen to be high ($\epsilon_3 \geq 0.85$) as we have found that for values even as high as $\epsilon_3 = 0.8$, we achieve point-to-point optimality for all users (see Theorems~\ref{thm:w_minus} and~\ref{thm:all}).  If we increase $\epsilon_3$ even higher however, we observe a situation where many symbols destined to user~3 are erased, and so when the stopping condition of the algorithm in Section~\ref{subsec:instantly_decodable} is reached, we are left with queues $Q_3$, \Q{1,3} and \Q{2,3}.
%
\begin{figure}
	\centering
	\setlength\figurewidth{2.65in} 
	\setlength\figureheight{2.07in} 
%	\includegraphics[scale=0.5]{fig/ttotal_N_10e7_eps1_3_eps2_4}
	\input{fig/simulations/latency.tikz}
	\caption{The latency when we are forced to invoke the algorithms of Section~\ref{subsec:non_instant_coding}.  We fix $d_i = \epsilon_{i}^{2}$, $\epsilon_1 = 0.3$, $\epsilon_2 = 0.4$ and vary $\epsilon_3$ on the x-axis.  }
	\label{fig:t_total}
\end{figure}
%
When this occurs, we have not yet satisfied all users, and so we resort to the coding schemes proposed in Section~\ref{subsec:non_instant_coding}.  Fig.~\ref{fig:t_total} shows a plot where each point required the invocation of the queue preprocessing scheme for channel coding in Section~\ref{subsec:channel_coding}.  It plots the \emph{overall} latency required to achieve distortions $d_i = \epsilon_i^{2}$ for the values of $\epsilon_i$ given earlier.  Alternatively, if the chaining algorithm of Section~\ref{subsubsec:chaining_algorithm} is used, we find that the latency coincides with the outer bound.  Again, the segmentation-based scheme is provided as a benchmark along with the outer bound $w^{+}\dvec$.



% \vspace{-0.5em}
\section{Conclusion}
% \vspace{-0.5em}
Recent advances in multimodal single-cell technology have enabled the simultaneous profiling of the transcriptome alongside other cellular modalities, leading to an increase in the availability of multimodal single-cell data. In this paper, we present \method{}, a multimodal transformer model for single-cell surface protein abundance from gene expression measurements. We combined the data with prior biological interaction knowledge from the STRING database into a richly connected heterogeneous graph and leveraged the transformer architectures to learn an accurate mapping between gene expression and surface protein abundance. Remarkably, \method{} achieves superior and more stable performance than other baselines on both 2021 and 2022 NeurIPS single-cell datasets.

\noindent\textbf{Future Work.}
% Our work is an extension of the model we implemented in the NeurIPS 2022 competition. 
Our framework of multimodal transformers with the cross-modality heterogeneous graph goes far beyond the specific downstream task of modality prediction, and there are lots of potentials to be further explored. Our graph contains three types of nodes. While the cell embeddings are used for predictions, the remaining protein embeddings and gene embeddings may be further interpreted for other tasks. The similarities between proteins may show data-specific protein-protein relationships, while the attention matrix of the gene transformer may help to identify marker genes of each cell type. Additionally, we may achieve gene interaction prediction using the attention mechanism.
% under adequate regulations. 
% We expect \method{} to be capable of much more than just modality prediction. Note that currently, we fuse information from different transformers with message-passing GNNs. 
To extend more on transformers, a potential next step is implementing cross-attention cross-modalities. Ideally, all three types of nodes, namely genes, proteins, and cells, would be jointly modeled using a large transformer that includes specific regulations for each modality. 

% insight of protein and gene embedding (diff task)

% all in one transformer

% \noindent\textbf{Limitations and future work}
% Despite the noticeable performance improvement by utilizing transformers with the cross-modality heterogeneous graph, there are still bottlenecks in the current settings. To begin with, we noticed that the performance variations of all methods are consistently higher in the ``CITE'' dataset compared to the ``GEX2ADT'' dataset. We hypothesized that the increased variability in ``CITE'' was due to both less number of training samples (43k vs. 66k cells) and a significantly more number of testing samples used (28k vs. 1k cells). One straightforward solution to alleviate the high variation issue is to include more training samples, which is not always possible given the training data availability. Nevertheless, publicly available single-cell datasets have been accumulated over the past decades and are still being collected on an ever-increasing scale. Taking advantage of these large-scale atlases is the key to a more stable and well-performing model, as some of the intra-cell variations could be common across different datasets. For example, reference-based methods are commonly used to identify the cell identity of a single cell, or cell-type compositions of a mixture of cells. (other examples for pretrained, e.g., scbert)


%\noindent\textbf{Future work.}
% Our work is an extension of the model we implemented in the NeurIPS 2022 competition. Now our framework of multimodal transformers with the cross-modality heterogeneous graph goes far beyond the specific downstream task of modality prediction, and there are lots of potentials to be further explored. Our graph contains three types of nodes. while the cell embeddings are used for predictions, the remaining protein embeddings and gene embeddings may be further interpreted for other tasks. The similarities between proteins may show data-specific protein-protein relationships, while the attention matrix of the gene transformer may help to identify marker genes of each cell type. Additionally, we may achieve gene interaction prediction using the attention mechanism under adequate regulations. We expect \method{} to be capable of much more than just modality prediction. Note that currently, we fuse information from different transformers with message-passing GNNs. To extend more on transformers, a potential next step is implementing cross-attention cross-modalities. Ideally, all three types of nodes, namely genes, proteins, and cells, would be jointly modeled using a large transformer that includes specific regulations for each modality. The self-attention within each modality would reconstruct the prior interaction network, while the cross-attention between modalities would be supervised by the data observations. Then, The attention matrix will provide insights into all the internal interactions and cross-relationships. With the linearized transformer, this idea would be both practical and versatile.

% \begin{acks}
% This research is supported by the National Science Foundation (NSF) and Johnson \& Johnson.
% \end{acks}


\bibliographystyle{plain}
\bibliography{refs}

% \begin{appendices}

\section{Simple versions of the algorithms}
\label{appendix:simple_algs}

\begin{algorithm}
\caption{Online algorithm}\label{online_simple}
\begin{algorithmic}
\State Initialize \textsc{Student} learning algorithm
\State Initialize expected return $Q(a)=0$ for all $N$ tasks
\For{t=1,\ldots,T}
\State Choose task $a_t$ based on $|Q|$ using $\epsilon$-greedy or Boltzmann policy
\State Train \textsc{Student} using task $a_t$ and observe reward $r_t = x_t^{(a_t)} - x_{t'}^{(a_t)}$
\State Update expected return $Q(a_t) = \alpha r_t + (1 - \alpha) Q(a_t)$
\EndFor
\end{algorithmic}
\end{algorithm}

\begin{algorithm}
\caption{Naive algorithm}\label{naive_simple}
\begin{algorithmic}
\State Initialize \textsc{Student} learning algorithm
\State Initialize expected return $Q(a)=0$ for all $N$ tasks
\For{t=1,...,T}
\State Choose task $a_t$ based on $|Q|$ using $\epsilon$-greedy or Boltzmann policy
\State Reset $D=\emptyset$
\For{k=1,...,K}
\State Train \textsc{Student} using task $a_t$ and observe score $o_t = x_t^{(a_t)}$
\State Store score $o_t$ in list $D$
\EndFor
\State Apply linear regression to $D$ and extract the coefficient as $r_t$
\State Update expected return $Q(a_t) = \alpha r_t + (1 - \alpha) Q(a_t)$
\EndFor
\end{algorithmic}
\end{algorithm}

\begin{algorithm}
\caption{Window algorithm}\label{window_simple}
\begin{algorithmic}
\State Initialize \textsc{Student} learning algorithm
\State Initialize FIFO buffers $D(a)$ and $E(a)$ with length $K$ for all $N$ tasks
\State Initialize expected return $Q(a)=0$ for all $N$ tasks
\For{t=1,\ldots,T}
\State Choose task $a_t$ based on $|Q|$ using $\epsilon$-greedy or Boltzmann policy
\State Train \textsc{Student} using task $a_t$ and observe score $o_t = x_t^{(a_t)}$
\State Store score $o_t$ in $D(a_t)$ and timestep $t$ in $E(a_t)$
\State Use linear regression to predict $D(a_t)$ from $E(a_t)$ and use the coef. as $r_t$
%\State Update expected return $Q(a_t) := r_t$
\State Update expected return $Q(a_t) = \alpha r_t + (1 - \alpha) Q(a_t)$
\EndFor
\end{algorithmic}
\end{algorithm}

\begin{algorithm}
\caption{Sampling algorithm}\label{sampling_simple}
\begin{algorithmic}
\State Initialize \textsc{Student} learning algorithm
\State Initialize FIFO buffers $D(a)$ with length $K$ for all $N$ tasks
\For{t=1,\ldots,T}
\State Sample reward $\tilde{r}_a$ from $D(a)$ for each task (if $|D(a)|=0$ then $\tilde{r}_a=1$)
\State Choose task $a_t = \argmax_a |\tilde{r}_a|$
\State Train \textsc{Student} using task $a_t$ and observe reward $r_t = x_t^{(a_t)} - x_{t'}^{(a_t)}$
\State Store reward $r_t$ in $D(a_t)$
\EndFor
\end{algorithmic}
\end{algorithm}

\newpage
\section{Batch versions of the algorithms}
\label{appendix:batch_algs}

\begin{algorithm}
\caption{Online algorithm}\label{online_batch}
\begin{algorithmic}
\State Initialize \textsc{Student} learning algorithm
\State Initialize expected return $Q(a)=0$ for all $N$ tasks
\For{t=1,\ldots,T}
\State Create prob. dist. $\vec{a_t}=(p_t^{(1)}, ..., p_t^{(N)})$ based on $|Q|$ using $\epsilon$-greedy or Boltzmann policy
\State Train \textsc{Student} using prob. dist. $\vec{a_t}$ and observe scores $\vec{o_t} = (x_t^{(1)}, ..., x_t^{(N)})$
\State Calculate score changes $\vec{r_t} = \vec{o_t} - \vec{o_{t-1}}$
%\State Calculate score change $\hat{r}_t = o_t - o_{t-1}$
%\State Calculate corrected reward $r_t = \hat{r}_t / a_t$ ($a_t$ is prob. dist.)
\State Update expected return $\vec{Q} = \alpha \vec{r_t} + (1 - \alpha) \vec{Q}$
\EndFor
\end{algorithmic}
\end{algorithm}

\begin{algorithm}
\caption{Naive algorithm}\label{online_naive}
\begin{algorithmic}
\State Initialize \textsc{Student} learning algorithm
\State Initialize expected return $Q(a)=0$ for all $N$ tasks
\For{t=1,\ldots,T}
\State Create prob. dist. $\vec{a_t}=(p_t^{(1)}, ..., p_t^{(N)})$ based on $|Q|$ using $\epsilon$-greedy or Boltzmann policy
\State Reset $D(a)=\emptyset$ for all tasks
\For{k=1,\ldots,K}
\State Train \textsc{Student} using prob. dist. $\vec{a_t}$ and observe scores $\vec{o_t} = (x_t^{(1)}, ..., x_t^{(N)})$
\State Store score $o_t^{(a)}$ in list $D(a)$ for each task $a$
\EndFor
\State Apply linear regression to each $D(a)$ and extract the coefficients as vector $\vec{r_t}$
%\State Apply linear regression to each $D(a)$ and extract the coefficients as $\hat{r}_t$
%\State Calculate corrected rewards $r_t = \hat{r}_t / a_t$ ($a_t$ is prob. dist.)
\State Update expected return $\vec{Q} = \alpha \vec{r_t} + (1 - \alpha) \vec{Q}$
\EndFor
\end{algorithmic}
\end{algorithm}

\begin{algorithm}
\caption{Window algorithm}\label{online_window}
\begin{algorithmic}
\State Initialize \textsc{Student} learning algorithm
\State Initialize FIFO buffers $D(a)$ with length $K$ for all $N$ tasks
\State Initialize expected return $Q(a)=0$ for all $N$ tasks
\For{t=1,\ldots,T}
\State Create prob. dist. $\vec{a_t}=(p_t^{(1)}, ..., p_t^{(N)})$ based on $|Q|$ using $\epsilon$-greedy or Boltzmann policy
\State Train \textsc{Student} using prob. dist. $\vec{a_t}$ and observe scores $\vec{o_t} = (x_t^{(1)}, ..., x_t^{(N)})$
\State Store score $o_t^{(a)}$ in $D(a)$ for all tasks $a$
\State Apply linear regression to each $D(a)$ and extract the coefficients as vector $\vec{r_t}$
%\State Apply linear regression to each $D(a)$ and extract the coefficients as $\hat{r}_t$
%\State Calculate corrected rewards $r_t = \hat{r}_t / a_t$ ($a_t$ is prob. dist.)
\State Update expected return $\vec{Q} = \alpha \vec{r_t} + (1 - \alpha) \vec{Q}$
%\State Update expected return $Q = r_t$
\EndFor
\end{algorithmic}
\end{algorithm}

\begin{algorithm}
\caption{Sampling algorithm}\label{online_sampling}
\begin{algorithmic}
\State Initialize \textsc{Student} learning algorithm
\State Initialize FIFO buffers $D(a)$ with length $K$ for all $N$ tasks
\For{t=1,\ldots,T}
\State Sample reward $\tilde{r}_a$ from $D(a)$ for each task (if $|D(a)|=0$ then $\tilde{r}_a=1$)
\State Create one-hot prob. dist. $\vec{\tilde{a}_t}=(p_t^{(1)}, ..., p_t^{(N)})$ based on $\argmax\nolimits_a |\tilde{r}_a|$
\State Mix in uniform dist. : $\vec{a_t} = (1 - \epsilon) \vec{\tilde{a}_t} + \epsilon/N$
\State Train \textsc{Student} using prob. dist. $\vec{a_t}$ and observe scores $\vec{o_t} = (x_t^{(1)}, ..., x_t^{(N)})$
\State Calculate score changes $\vec{r_t} = \vec{o_t} - \vec{o_{t-1}}$
%\State Calculate score change $\hat{r}_t = o_t - o_{t-1}$
%\State Calculate corrected rewards $r_t = \hat{r}_t / a_t$ ($a_t$ is prob. dist.)
\State Store reward $r_t^{(a)}$ in $D(a)$ for each task $a$
\EndFor
\end{algorithmic}
\end{algorithm}

\clearpage
\section{Decimal Number Addition Training Details}
\label{appendix:addition}

Our reimplementation of decimal addition is based on Keras \citep{chollet2015keras}. The encoder and decoder are both LSTMs with 128 units. In contrast to the original implementation, the hidden state is not passed from encoder to decoder, instead the last output of the encoder is provided to all inputs of the decoder. One curriculum training step consists of training on 40,960 samples. Validation set consists of 4,096 samples and 4,096 is also the batch size. Adam optimizer \citep{kingma2014adam} is used for training with default learning rate of 0.001. Both input and output are padded to a fixed size.

In the experiments we used the number of steps until 99\% validation set accuracy is reached as a comparison metric. The exploration coefficient $\epsilon$ was fixed to 0.1, the temperature $\tau$ was fixed to 0.0004, the learning rate $\alpha$ was 0.1, and the window size $K$ was 10 in all experiments.
 
\section{Minecraft Training Details}
\label{appendix:minecraft}

The Minecraft task consisted of navigating through randomly generated mazes. The maze ends with a target block and the agent gets 1,000 points by touching it. Each move costs -0.1 and dying in lava or getting a timeout yields -1,000 points. Timeout is 30 seconds (1,500 steps) in the first task and 45 seconds (2,250 steps) in the subsequent tasks.

For learning we used the \textit{proximal policy optimization} (PPO) algorithm \citep{schulman2017proximal} implemented using Keras \citep{chollet2015keras} and optimized for real-time environments. The policy network used four convolutional layers and one LSTM layer. Input to the network was $40\times 30$ color image and outputs were two Gaussian actions: move forward/backward and turn left/right. In addition the policy network had state value output, which was used as the baseline. Figure \ref{f14} shows the network architecture.

\begin{figure}[h]
  \includegraphics[scale=0.4]{figures/minecraft_network}
\caption{Network architecture used for Minecraft.}
\label{f14}
\end{figure}

For training we used a setup with 10 parallel Minecraft instances. The agent code was separated into runners, that interact with the environment, and a trainer, that performs batch training on GPU, similar to \cite{babaeizadeh2016reinforcement}. Runners regularly update their snapshot of the current policy weights, but they only perform prediction (forward pass), never training. After a fixed number of steps they use FIFO buffers to send collected states, actions and rewards to the trainer. Trainer collects those experiences from all runners, assembles them into batches and performs training. FIFO buffers shield the runners and the trainer from occasional hiccups. This also means that the trainer is not completely on-policy, but this problem is handled by the importance sampling in PPO.

\begin{figure}[h]
  \includegraphics[scale=0.4]{figures/minecraft_training}
\caption{Training scheme used for Minecraft.}
\label{f14}
\end{figure}

During training we also used frame skipping, i.e. processed only every 5th frame. This sped up the learning considerably and the resulting policy also worked without frame skip. Also, we used auxiliary loss for predicting the depth as suggested in \citep{mirowski2016learning}. Surprisingly this resulted only in minor improvements.

For automatic curriculum learning we only implemented the Window algorithm for the Minecraft task, because other algorithms rely on score change, which is not straightforward to calculate for parallel training scheme. Window size was defined in timesteps and fixed to 10,000 in the experiments, exploration rate was set to 0.1.

The idea of the first task in the curriculum was to make the agent associate the target with a reward. In practice this task proved to be too simple - the agent could achieve almost the same reward by doing backwards circles in the room. For this reason we added penalty for moving backwards to the policy loss function. This fixed the problem in most cases, but we occasionally still had to discard some unsuccessful runs. Results only reflect the successful runs.

We also had some preliminary success combining continuous (Gaussian) actions with binary (Bernoulli) actions for "jump" and "use" controls, as shown on figure \ref{f14}. This allowed the agent to learn to cope also with rooms that involve doors, switches or jumping obstacles, see \url{https://youtu.be/e1oKiPlAv74}.

\end{appendices}
\end{document}


# COMMENTS:

v - Add boxes around figure (instead of only bottom and left borders...
- 7 should come earlier
- Explain each figure in detail in caption
v - Axes labels: "Filtering Step", "Network OSPA", "Time" (capitalized)
  "# of targets" -> "No. of Targets"
  Add space between 'x' and '(m)', between 'time' and '(s)', etc.
v - '''^ Same for legends

v - Remove titles, y-labels: "OSPA of Sensor 3 (m)"
v - proposed -> "Proposed Method"
v - Figure 2. has something cut off from top (anyway, remove title)


New comments:
v - The colors of the 'trajectories' figure should be all same, and consistent with the legend of the plot.
- check the highlighted comments


New Comments from July 30th:
- Double check whether survival property uses 'min' (in the code)

Some final questions for jessica:
- is probability of detection 0 outside the surveillance region?
- Are there an average of 5 clutter measurements AT EACH SENSOR? Based on theory, I think it should be 5 (on average) at each sensor but just want to double-check.
