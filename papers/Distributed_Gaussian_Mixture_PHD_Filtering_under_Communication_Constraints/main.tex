\documentclass[lettersize,journal]{IEEEtran}
% OLD PREAMBLE:

% \usepackage{jsen}
% \usepackage{cite}
% \usepackage{amsmath,amssymb,amsfonts, bbm, mathtools}
% \usepackage{algorithm,algorithmic}
% \usepackage{graphicx}
% \usepackage{textcomp}
% \usepackage{wrapfig}
% \usepackage{xfrac}
% \usepackage{stackengine}
% \usepackage{subfigure}
% \def\delequal{\mathrel{\ensurestackMath{\stackon[1pt]{=}{\scriptstyle\Delta}}}}



% \usepackage{color, soul}
% \newcommand{\hlt}[1]{\hl{#1}}
% \newcommand{\red}[1]{\textcolor{red}{#1}}

% \def\BibTeX{{\rm B\kern-.05em{\sc i\kern-.025em b}\kern-.08em
%     T\kern-.1667em\lower.7ex\hbox{E}\kern-.125emX}}
% \markboth{\journalname, VOL. XX, NO. XX, XXXX 2017}
% {Author \MakeLowercase{\textit{et al.}}: Preparation of Papers for IEEE TRANSACTIONS and JOURNALS (February 2017)}
% \definecolor{abstractbg}{rgb}{0.89804,0.94510,0.83137}
% \setlength{\fboxrule}{0pt}
% \setlength{\fboxsep}{0pt}

% NEW PREAMBLE:


\usepackage{amsmath,amsfonts,amssymb,bbm, amsthm, xfrac}
\usepackage{algorithmic}
\usepackage{algorithm}
\usepackage{array, multirow}
% \usepackage[caption=false,font=normalsize,labelfont=sf,textfont=sf]{subfig}
\usepackage{caption, subcaption}
\usepackage{textcomp}
\usepackage{stfloats}
\usepackage{url}
\usepackage{verbatim}
\usepackage{graphicx}
\usepackage{cite}
\usepackage{caption}
\usepackage{subcaption}
\hyphenation{}

\theoremstyle{plain}
\newtheorem{theorem}{Theorem}

\usepackage{color, soul}
\newcommand{\hlt}[1]{\hl{#1}}
\newcommand{\red}[1]{\textcolor{red}{#1}}

\begin{document}

\title{Distributed Gaussian Mixture PHD Filtering under Communication Constraints}

\author{Shiraz Khan, Yi-Chieh Sun, and Inseok Hwang\thanks{The authors are with the School of Aeronautics and Astronautics, Purdue University, West Lafayette, IN 47906, USA. (email: \texttt{shiraz@purdue.edu})}
%~\IEEEmembership{Staff,~IEEE,}
        % <-this % stops a space
%\thanks{This paper was produced by the IEEE Publication Technology Group. They are in Piscataway, NJ.}% <-this % stops a space
% \thanks{Manuscript submitted on Aug 1, 2023.}
\thanks{\textcolor{blue}{© 2023 IEEE. Personal use of this material is permitted. Permission from IEEE must be obtained for all other uses, in any current or future media, including reprinting/republishing this material for advertising or promotional purposes, creating new collective works, for resale or redistribution to servers or lists, or reuse of any copyrighted component of this work in other works.}}
}

% The paper headers
\markboth{IEEE Transactions on Signal Processing}%
{Khan \MakeLowercase{\textit{et al.}}: Distributed Gaussian Mixture PHD Filtering under Communication Constraints}

% \IEEEpubid{0000--0000/00\$00.00~\copyright~2023 IEEE}
% Remember, if you use this you must call \IEEEpubidadjcol in the second
% column for its text to clear the IEEEpubid mark.

\maketitle

\begin{abstract}
The Gaussian Mixture Probability Hypothesis Density (GM-PHD) filter is an almost exact closed-form approximation to the Bayes-optimal multi-target tracking algorithm. Due to its optimality guarantees and ease of implementation, it has been studied extensively in the literature. However, the challenges involved in implementing the GM-PHD filter efficiently in a distributed (multi-sensor) setting have received little attention. The existing solutions for distributed PHD filtering either have a high computational and communication cost, making them infeasible for resource-constrained applications, or are unable to guarantee the asymptotic convergence of the distributed PHD algorithm to an optimal solution.     
In this paper, we develop a distributed GM-PHD filtering recursion that uses a probabilistic communication rule to limit the communication bandwidth of the algorithm, while ensuring asymptotic optimality of the algorithm. We derive the convergence properties of this recursion, which uses weighted average consensus of Gaussian mixtures (GMs) to lower (and asymptotically minimize) the Cauchy-Schwarz divergence between the sensors' local estimates. In addition, the proposed method is able to avoid the issue of false positives, which has previously been noted to impact the filtering performance of distributed multi-target tracking. Through numerical simulations, it is demonstrated that our proposed method is an effective solution for distributed multi-target tracking in resource-constrained sensor networks.
\end{abstract}

\begin{IEEEkeywords}
multi-target tracking, sensor networks, distributed state estimation, Gaussian mixture
\end{IEEEkeywords}

%

\maketitle

\section{Introduction}  \label{sec:introduction}

\newcommand\inexpIntro[3]{#1?(#2,#3).}
\newcommand\rinexpIntro[3]{*#1?(#2,#3).}
\newcommand\outexpIntro[3]{#1!(#2,#3).}
\newcommand\outatomIntro[3]{#1!(#2,#3)}

We propose a fully automated method for proving termination of \(\pi\)-calculus processes.
Although there have been a lot of studies on termination analysis for the \(\pi\)-calculus
and related calculi~\cite{Deng06IC,Demangeon07,SangiorgiTermination,KobayashiHybrid,Yoshida04IC,DBLP:journals/jlp/DemangeonHS10,Venet98SAS}, most of them have been rather theoretical,
and there have been surprisingly little efforts in developing  fully automated termination
verification methods and tools based on them. To our knowledge,
Kobayashi's \typical{}~\cite{TyPiCal,KobayashiHybrid} is the only exception that
can prove termination of \(\pi\)-calculus processes (extended with natural numbers)
fully automatically, but its termination analysis is quite limited (see Section~\ref{sec:relatedwork}).

Our method is based on a reduction to termination analysis for sequential programs:
we translate a \(\pi\)-calculus process \(P\) to a sequential program \(S_P\), so that
if \(S_P\) is terminating, so is \(P\). The reduction allows us to use
powerful, mature methods and tools
for termination analysis of sequential programs~\cite{heizmann2016ultimate,freqterm,DBLP:conf/lics/PodelskiR04,Kuwahara2014Termination,DBLP:journals/cacm/CookPR11}.

The idea of the translation is to convert a chain of communications on replicated input
channels to a chain of recursive function calls of the target sequential program.
Let us consider the following Fibonacci process:
\begin{align*}
    & \rinexpIntro{\fib}{n}{r}
        \ifexp{n<2}{ \soutatom{r}{1} \\ &\quad}
                   { \nuexp{s_1} \nuexp{s_2} (\outatomIntro{\fib}{n-1}{s_1} \PAR \outatomIntro{\fib}{n-2}{s_2} \PAR \sinexp{s_1}{x}\sinexp{s_2}{y}\soutatom{r}{x+y}) \\}
    & \PAR \outatomIntro{\fib}{m}{r}
\end{align*}
Here, the process
$\rinexpIntro{\fib}{n}{r} \ldots$ is a function server that computes the \(n\)-th Fibonacci number
in parallel and returns the result to \(r\),
and $\outatom{\fib}{m}{r}$ sends a request for computing the \(m\)-th Fibonacci number;
those who are not familiar with the syntax of the \(\pi\)-calculus may wish to consult
Section~\ref{sec:targetlanguage} first.
To prove that the process above is terminating for any integer \(m\),
it suffices to show that there is no infinite chain of communications on $\fib$:
\[
    \fib(m,r) \to \fib(m_1,r_1) \to \fib(m_2,r_2) \to \cdots.
\]
We convert the process above to the following program:\footnote{The actual translation
  given later is a little more complex.}
\begin{verbatim}
 let rec fib(n) = if n<2 then () else (fib(n-1) [] fib(n-2)) in
 fib(m)
\end{verbatim}
Here, \texttt{[]} represents the non-deterministic choice.
Note that, although the calculation of Fibonacci numbers is not preserved,
for each chain of communications on \texttt{fib}, there is a corresponding
sequence of recursive calls:
\[
\mathtt{fib}(m) \to \mathtt{fib}(m_1) \to \mathtt{fib}(m_2) \to \cdots.
\]
Thus, the termination of the sequential program above implies the termination of
the original process.
As shown in the example above, (i) each communication on a replicated input channel
is converted to a function call, (ii) each communication on a non-replicated input
channel is just removed (or, in the actual translation, replaced by a call of
a trivial function defined by \(f(\seq{x})=(\,)\)), and (iii) parallel composition
is replaced by a non-deterministic choice.
We formalize the translation outlined above and prove its correctness.

The basic translation sketched above sometimes loses too much information.
For example, consider the following process:
\begin{align*}
    & \rinexpIntro{\pre}{n}{r} \soutatom{r}{n-1} \\
    & \PAR \rinexpIntro{f}{n}{r} \ifexp{n<0}{ \soutatom{r}{1} }
                                       { \nuexp{s} (\outatomIntro{\pre}{n}{s} \PAR \sinexp{s}{x}\outatomIntro{f}{x}{r}) } \\
    & \PAR \outatomIntro{f}{m}{r}
\end{align*}
The translation sketched above would yield:
\begin{verbatim}
  let pred(n) = n-1 in
  let rec f(n) = if n<0 then () else (pred(n) [] f(*)) in
  f(m)
\end{verbatim}
Here, \texttt{*} represents a non-deterministic integer: since we have removed
the input $\sinatom{s}{x}$, we do not have information about the value of \( x \).
As a result, the sequential program above is non-terminating, although the original
process is terminating.
To remedy this problem, we also refine the basic translation above by using a refinement
type system for the \(\pi\)-calculus. Using the refinement type system,
we can infer that the value of \(x\) in the original process is less than \(n\),
so that we can refine the definition of \texttt{f} to:
\begin{verbatim}
 let rec f(n) = ... else (pred(n) [] let x=* in assume(x<n);f(x))
\end{verbatim}
The target program is now terminating, from which
we can deduce that the original process is also terminating.
We have implemented an automated tool based on the refined translation above.

The contributions of this paper are summarized as follows.
\begin{itemize}
\item The formalization of the basic translation from the \(\pi\)-calculus
  (extended with integers) to sequential programs, and a proof of its correctness.
\item The formalization of a refined translation based on a refinement type system.
\item An implementation of the refined translation, including automated refinement type
  inference based on CHC solving, and experiments to evaluate the effectiveness of
  our method.
\end{itemize}

The rest of this paper is structured as follows.
Section~\ref{sec:targetlanguage} introduces the source and target languages
of our translation.
Section~\ref{sec:approach} 
formalizes the basic translation, and proves its correctness.
Section~\ref{sec:refinement} refines the basic translation by using a refinement type system.
Section~\ref{sec:implementation} reports an implementation and experiments.
Section~\ref{sec:relatedwork} discusses related work,
and Section~\ref{sec:conclusion} concludes the paper.


\section{Problem Formulation}
\label{sec:prob_form}
\subsection{Target Model}
In the multi-sensor multi-target tracking problem, an unknown time-varying number of dynamical systems, called the \textit{targets}, are observed using a sensor network.
The state of the $l^{th}$ target at timestep $k$ is given by a random vector $x_{k}^{(l)} \in \mathcal{X}$, where $\mathcal{X} = \mathbb{R}^{d_x}$ is the state space of each target and $d_x$ is its dimension. The collection of the target states at timestep $k$ is modeled as a random finite set (RFS)\footnote{A brief introduction to RFSs and their properties can be found in \cite{vo2006gmphd}.} $X_k = \left\{x_{k}^{(1)},...,x_{k}^{(|X_k|)}\right\}$, where $|{}\cdot{}|$ denotes the cardinality of a set.
Thus, the number of targets, $|X_k|$, is an integer-valued random variable.

The state of each target is assumed to evolve independently of the other targets, according to a linear Gauss-Markov process on the state space $\mathcal{X}$ with the transition kernel $f_{k|k-1}$. Thus, given that a target exists at timesteps $k-1$ and $k$, the probability density function (pdf) of its state at timestep $k$ is
\begin{equation}
    f_{k|k-1}(x | x_{k-1}) = 
    \mathcal{N}(x; F_{k-1} x_{k-1}, Q_{k-1})
    \label{eq:dynamic_model}
\end{equation}
where $F_{k-1}$ and $Q_{k-1}$ are known matrices and $x_{k-1} \in\mathcal X$ is the state of the target at timestep $k-1$. 
$\mathcal{N}({}\cdot{}; m, P)$ denotes the pdf of a multivariate Gaussian random vector with mean $m\in \mathbb R^{d_x}$ and covariance $P\in \mathbb R^{d_x\times d_x}$. 
The probability of a target continues to exist across both timesteps is called its \textit{survival probability}, denoted by $p^S_{k}:\mathcal X \rightarrow [0, 1]$, and is assumed to depend only on the current state of each target.

New targets of interests can arise due to spontaneous \textit{births} or \textit{spawns}. Births are independent of $X_k$ and assumed to be sampled from an underlying Poisson RFS having the intensity $\gamma _k:\mathcal X \rightarrow \mathbb R_{\geq0}$, where $\mathbb R_{\geq0}$ denotes the set of non-negative real numbers. 
Spawns correspond to targets that emerge from the vicinity of existing targets, so they are dependent on $X_k$. Given that a target exists at $x_{k-1} \in \mathcal X$ at timestep $k-1$, the new targets spawned from it are assumed to be sampled from
a Poisson RFS having the intensity $\sigma _{k|k-1}({}\cdot{}|x_{k-1}):\mathcal X \rightarrow \mathbb R_{\geq 0}$. Finally, note that the transition kernel $f_{k|k-1}$ and spawn intensity $\sigma_{k|k-1}$ are independent of the labels or identities of the targets; if they are dependent on the target labels, a labelled RFS can be used to represent the multi-target state instead \cite{vo2014labeled}.

\subsection{Sensor Network Model}
% sensor network and measurements
The targets are being observed by
a sensor network consisting of a number of spatially distributed sensors, which can be represented as a directed graph $\mathcal G=(\mathcal V, \mathcal E)$, where $\mathcal V$ denotes its vertices (representing the sensors) and $\mathcal E \subseteq \mathcal V \times \mathcal V$ is the set of edges (representing the directional communication links between sensors).
The graph $\mathcal G$ is assumed to be strongly connected. 
% The total number of sensors in the network is given by $|\mathcal V|$. 
% Given two sensors $i,j\in \mathcal V$, sensor $j$ can receive information from sensor $i$ if and only if $(i,j)\in \mathcal E$.
For each sensor $i \in \mathcal{V}$, ${N}^-_{i} = \left\{ j\in\mathcal{V}: (j,i)\in\mathcal{E} \right\}$ denotes its set of in-neighbors, i.e., the set of sensors from which sensor $i$ can receive data. Similarly, $N^+_i$ is the set of out-neighbors.

At timestep $k$, each sensor $i$ obtains a set of measurements $Z_{i,k} = {\left\{z^{(1)}_{i,k},...,z_{i,k}^{(|Z_{i,k}|)}\right\}}$, where $z_{i,k}^{(l)} \in \mathcal{Z}$. Here, $\mathcal{Z} \subseteq \mathbb{R}^{d_z}$ is the measurement space and $d_z$ is its dimension. It is assumed that the measurements are generated as per the following mechanism: the $i^{th}$ sensor obtains a measurement from a target at $x \in \mathcal X$ with the probability $p^D_{i,k}(x)$, where $p^D_{i,k}:\mathcal X \rightarrow [0,1]$ is called the detection probability of sensor $i$ at timestep $k$. The event of the $l^{th}$ target being detected at the $i^{th}$ sensor is independent of the detections at other target-sensor pairs.
%, so the detections can be thought of as i.i.d. Bernoulli trials. 
Given that a target at $x \in \mathcal X$ is detected by sensor $i$, the pdf of the obtained measurement is denoted as $g_{i,k}({}\cdot{}|x)$ and is  
given by the linear Gaussian model
\begin{equation}
    g_{i,k}(z|x) = \mathcal N(z; H_{i,k} x, R_{i,k})
    \label{eq:measurement_model}
\end{equation}
where $H_{i,k}$ and $R_{i,k}$ are known matrices.
In addition, the obtained measurement can correspond to clutter, which are not generated by any of the targets, but arise due to sensor noise or other random effects. The clutter measurements are assumed to be realizations of a Poisson RFS with the intensity $\kappa_i:\mathcal Z \rightarrow \mathbb R_{\geq 0}$.
Thus, $\lbrace Z_{i,k} \rbrace$ are realizations of an RFS, and the distribution of the cardinality of this RFS (i.e., the number of measurements) is determined by the detection probability $p^D_{i,k}$ and clutter intensity $\kappa _i$.


\subsection{Local GM-PHD Filtering}
\label{sec:subsec_local_filt}
Distributed multi-target tracking over a sensor network is usually accomplished using a succession of a local filtering step (e.g., using a PHD filter) at each of the sensors followed by a fusion step, wherein the sensors share information across the communication channels. 
%Each iteration of the local PHD filter consists of a prediction and an update step. 
The predicted multi-target state of sensor $i$ is an RFS whose intensity is denoted as $v_{i,k|k-1}$. It characterizes the information known about the multi-target state $X_k$ at sensor $i$ at timestep $k$, before observing the measurements $Z_{i,k}$. After the incorporation of the measurements $Z_{i,k}$, the updated (posterior) intensity is denoted as $v_{i,k|k}$. 
% Finally, the intensities of the individual sensors are fused together, with the fused intensity at sensor $i$ being denoted as $v^F_{i,k|k}$.

In the proposed distributed GM-PHD filter, each sensor performs the local prediction and measurement update steps according to the GM-PHD filtering algorithm \cite{vo2006gmphd}. The GM-PHD filter uses Gaussian mixtures (GMs) to represent the intensity functions, admitting a closed-form solution to the PHD recursion. Consequently, the GM-PHD filter is optimal in a Bayesian sense, under reasonable assumptions on the target and sensor models; a more complete discussion of the relevant assumptions may be found in \cite{vo2006gmphd}. In practice, the GM-PHD filter is approximated by pruning small GM components to keep the space complexity of the algorithm from growing over time.

Given $x_{k-1} \in \mathcal X$, let the prior intensity $v_{i,k|k-1}$
%, be represented using GMs with $J_{i,k|k-1}$, $J_{\gamma,k}$ and $J_{\sigma,k}$ components, respectively. The GM representations of these intensities are as follows:
be represented using a GM with $J_{i,k|k-1}$ components, as follows:
\begin{align}
    v_{i,k|k-1}(x) &= \sum_{l=1}^{J_{i,k|k-1}}w^{(l)}_{i,k|k-1}\mathcal N(x;m^{(l)}_{i,k|k-1}, P^{(l)}_{i,k|k-1})
    \label{eq:prior_intensity}
%    \gamma_k(x) &= \sum_{l=1}^{J_{\gamma,k}} w^{(l)}_{\gamma,k}\mathcal N(x;m^{(l)}_{\gamma,k}, P^{(l)}_{\gamma,k})\\
  %  \sigma_{k|k-1}(x|x_{k-1}) &= \sum_{l=1}^{J_{\sigma,k}} w^{(l)}_{\sigma,k}\mathcal N(x;F^{(l)}_{\sigma,k-1}x_{k-1}, Q^{(l)}_{\sigma,k-1})
\end{align}
Sensor $i$'s prior estimate of the number of targets is $\sum_{l=1}^{J_{i,k|k-1}} w_{i,k|k-1}^{(l)}$, as this sum corresponds to the integral of the intensity $v_{i,k|k-1}(x)$.

The birth and spawn intensities ($\gamma_k$ and $\sigma_{k|k-1}({}\cdot{}|x_{k-1})$, respectively) are also represented by GMs whose parameters are assumed to be known. The prior intensity $v_{i,k|k-1}$ includes the surviving targets from timestep $k-1$ as well as new targets that were born or spawned at timestep $k$. Since these correspond to independent Poisson RFSs, we can add the corresponding intensity functions as follows:
\begin{equation}
    v_{i,k|k-1}(x) = v_{i,k|k-1}^S(x) + v_{i,k|k-1}^\sigma(x) + \gamma_{k}(x),
\end{equation}
where $v^S_{i,k|k-1}$ corresponds to the targets that survived from timesteps $k-1$ to $k$ and $v_{i,k|k-1}^{\sigma}$ corresponds to newly spawned targets. Similarly, the posterior intensity is computed as a sum of two intensities:
\begin{equation}
    v_{i,k|k}(x) = (1 - p_{D,k}(x)) v_{i,k|k-1}(x) + \sum_{z \in Z_{k}} v_{i,k}^D(x;z)
\end{equation}
where the first term corresponds to the targets which were not detected and $v^{D}_{i,k}(x;z)$ is the new information obtained through the measurement $z\in Z_{k}$. Further details about the computation of prior and posterior intensities ($v_{i,k|k-1}$ and $v_{i,k|k}$, respectively) can be found in \cite{vo2006gmphd}. 

We refer to the computation of $v_{i,k|k-1}$ and $v_{i,k|k}$ using the local measurements $Z_{i,k}$ at sensor $i$ as local GM-PHD filtering.
% \begin{equation}
%     v_{i,k|k-1}^S(x) =  \sum_{l=1}^{J_{i,k-1|k-1}} p_{S,k}(x) w_{i,k-1|k-1}^{(l)} \mathcal{N}(x; m_{S,k|k-1}^{(l)}, P_{S,k|k-1}^{(l)})
% \end{equation}
% \begin{equation}
%     v_{i,k|k-1}^{\sigma}(x) = \sum_{l=1}^{J_{i,k-1|k-1}} \sum_{j=1}^{J_{\sigma,k}} w_{i,k-1|k-1}^{(l)} w_{\sigma,k}^{(j)} \mathcal{N}(x; m_{\sigma,k|k-1}^{(l,j)}, P_{\sigma,k|k-1}^{(l,j)})
% \end{equation}
% \begin{equation}
%     \gamma_{i,k} (x) = \sum_{l=1}^{J_{\gamma,k}} w_{\gamma,k}^{(l)}(x) \mathcal{N}(x; m_{\gamma,k}^{(l)}, P_{\gamma,k}^{(l)}).
%     \label{eq_birth intensity}
% \end{equation}
% With the measurement set $Z_{k}$ at time $k$, the multi-target update intensity $v_{k}(x)$ is also modeled as a Gaussian mixture by
% \begin{equation}
%     v_{i,k|k}(x) = v_{i,k}^{ND}(x) + \sum_{z \in Z_{k}} v_{i,k}^D(x;z)
% \end{equation}
% where $v_{ND,k}(x)$ corresponds to the undetected targets, and $v_{D,k}(x;z)$ corresponds to the contributions from measurement $z \in Z_{k}$, and estimated as
% \begin{equation}
%     v_{i,k}^{ND}(x) = (1 - p_{D,k}(x)) v_{i,k|k-1}(x)
% \end{equation}
% \begin{equation}
%     v_{i,k}^D(x;z) = \sum_{l=1}^{J_{i,k|k-1}} p_{D,k}(x) w_{k|k}^{(l)}(z) \mathcal{N}(x; m_{i,k|k}^{(l)}(z), P_{i,k|k}^{(l)}).
% \end{equation}
In the remainder of this paper, we focus on developing a distributed GM dissemination and fusion protocol for improving the multi-target tracking performance of each sensor. Our proposed distributed GM-PHD filtering algorithm constitutes the local GM-PHD filtering step followed by one or more inter-sensor fusion steps. 

\section{Fusion of Local GM-PHD Filters}
\label{sec:fusion_step}

In addition to local GM-PHD filtering, the communication channels, i.e., the edges in $\mathcal E$, can serve as additional sources of information for each sensor.
In theory, the optimal solution to the given multi-sensor multi-target problem can be realized by aggregating all the measurements of the sensor network as $\bigcup_{i\in \mathcal V}Z_{i,k}$ at each sensor, at each timestep. Since the communication and computational cost of such an approach does not scale well with the number of sensors in the network (which is equal to $|\mathcal V|$), it is infeasible when one or more of the following challenges are taken into consideration: 
\begin{enumerate}
    \item the total number of sensors in the network is large,
    \item the inter-sensor communication channels have limited bandwidths, or 
    \item the communication is not instantaneous, i.e., there are significant delays between the transmission and reception of inter-sensor communications.
\end{enumerate}
A distributed solution to the multi-sensor multi-target tracking problem can solve each of the preceding challenges. Distributed PHD filtering is achieved by fusing the intensity functions $v_{i,k|k}$ of the individual sensors, as these are typically much smaller in dimension than the measurement sets $Z_{i,k}$. For instance, the measurements in $Z_{i,k}$ may be high-dimensional camera images, whereas $v_{i,k|k}$ has a concise representation in terms of its GM components.

\subsection{Weighted Arithmetic Average (WAA) Fusion}
The weighted arithmetic average (WAA) of the local posterior intensity functions of the sensors is defined as
\begin{equation}
    v^{*}_{k|k}(x) = \sum_{i \in \mathcal V} \omega_i v_{i,k|k}(x)
    \label{eq:WAA_def}
\end{equation}
where $\omega_i \geq 0$ and $\sum_{i\in \mathcal V} \omega_i = 1$. In \cite{da2020kullback}, it was shown that the WAA (\ref{eq:WAA_def}) minimizes the weighted Kullback-Leibler (KL) divergence between itself and the local posterior intensities, $v_{i,k|k}$, when the fused intensity is taken as the second argument of the divergence. Similarly, it was shown in \cite{gostar2017cauchy} that the WAA also minimizes the weighted Cauchy-Schwarz (CS) divergence (which is symmetric in its arguments). Thus, we have
\begin{align}
    v^{*}_{k|k}(x) &= \arg \min_{g} \left( \sum_{i\in \mathcal V} \omega_i D_{KL}\left(v_{i,k|k}(x)\mathrel{\Vert}g(x)\right)\right) \\&= \arg \min_{g}\left( \sum_{i\in \mathcal V} \omega_i D_{CS}\left(v_{i,k|k}(x)\mathrel{\Vert}g(x)\right)\right)
    \label{eq:csd}
\end{align}
where $D_{KL}$ and $D_{CS}$ denote the KL and CS divergences, respectively.
Either choice of divergence can be considered as a cost function for the local PHD fusion process, analogous to how the weighted Euclidean distance is used as the cost function in single-target tracking. By minimizing the divergence, the information gain between the local posterior intensities and the fused intensities is minimized; so the WAA (\ref{eq:WAA_def}) is optimal in terms of the \textit{principle of minimum discrimination of information (PMDI)} \cite{gostar2017cauchy, gao2020multiobject}.

The weights $\omega_i$ can be chosen as $\omega_i =\sfrac{1}{|\mathcal V|}$ if the sensor network is homogeneous, i.e., all the sensors have the same sensing capability, measurement noise covariance and detection probability. When this is not the case, a larger weight can be assigned to sensors having better sensing capability. The connectivity of the sensor network $\mathcal G$ can also inform the choice of $\omega _i$. 
% The design of the weights $\omega _i$ is not developed further in this paper, serving as an avenue for future research.

Additionally, note that by taking an integral on both sides of (\ref{eq:WAA_def}), we have

\begin{equation}
    \int_{\mathcal X} v^{*}_{k|k}(x)\hspace{1pt}dx = \sum_{i \in \mathcal V} \omega_i \left(\int_{\mathcal X} v_{i,k|k}(x)\hspace{1pt}dx\right)
    \label{eq:integral_WAA}
\end{equation}
From the properties of Poisson RFS intensities, it follows that the integral of an intensity is the expected value of the cardinality of the corresponding RFS. Thus, WAA fusion of the intensities $v_{i,k|k}$ entails WAA fusion of the cardinality estimates of the sensors. The converse is not true, as two different functions may integrate to the same value. Thus, the proposed approach is different from the one used in \cite{li2018cardinality}, which directly computes the WAA of the cardinality estimates.

\subsection{Distributed WAA Fusion using Consensus}

To realize WAA fusion at each sensor in a distributed manner, an average (weighted) consensus algorithm can be used. In each iteration of the average consensus algorithm, the sensors communicate the GM components of their posterior intensity to their out-neighbors.
Thereafter, each sensor fuses its posterior intensities with those of its in-neighbors using a weighted combination, where the weights $\Omega_{ij}\in \mathbb R$ correspond to the entries of a matrix $\Omega = [\Omega_{ij}]$, with $\Omega_{ij}\neq 0$ if and only if $(j,i)\in \mathcal E$.
The average consensus algorithm is described in Algorithm \ref{alg:con}, which uses a total of $\alpha \geq 1$ inter-sensor fusion iterations.

\begin{algorithm}
\caption{Average Consensus of Posterior Intensities}
\begin{algorithmic}[1]
\vspace{2pt}
\REQUIRE The consensus weights $\Omega_{ij}$, posterior intensities $v_{i,k|k}$, and number of inter-sensor fusion steps $\alpha \geq 1$.\\
\vspace{3pt} 
\STATE Set $v_{i,k|k}^{(0)} \leftarrow v_{i,k|k} \forall i\in \mathcal V$
\FOR{$l=1,2,\dots,\alpha$}
\STATE Each sensor communicates its fused posterior intensity $v^{(l-1)}_{j,k|k}(x)$ to its out-neighbors, $N^+_i$. \vspace{2pt}
\STATE Update the posterior intensities as \begin{equation}
    v^{(l)}_{i,k|k}(x) \leftarrow \sum_{j \in N^{-}_i \cup \lbrace i\rbrace} \Omega_{ij} \hspace{2pt} v^{(l-1)}_{j,k|k}(x)
    \label{eq:average_consensus}
\end{equation}
$\forall i\in\mathcal V$.
\ENDFOR
\RETURN The fused posterior intensity, $v^{(\alpha)}_{i,k|k}(x)$.
\end{algorithmic}
\label{alg:con}
\end{algorithm}

\subsubsection{Error Analysis of Algorithm \ref{alg:con}}
\label{sec:subsec_error_analysis}
At the $l^{th}$ iteration of the average consensus algorithm (Algorithm \ref{alg:con}),
the error between the $i^{th}$ sensor's posterior intensity and the WAA is $v^{(l)}_{i,k|k}-v^*_{k|k}$. To establish the convergence properties of the algorithm, we need to show that the average consensus step (\ref{eq:average_consensus}) drives this error to $0$ (i.e., $v^{(l)}_{i,k|k}-v^*_{k|k}$ approaches the function which is $0$ everywhere in its domain) at each of the sensors.
To facilitate this analysis, let us assume that the intensities $v_{i,k|k}$ are elements in the $L^p(\mathcal X)$ function space, i.e., for some $1\leq p\leq \infty$,
\begin{equation}
\|v_{i,k|k}\|_{L^p} =  \left(\int_\mathcal X |v_{i,k|k}(x)|^p dx\right)^{\frac{1}{p}}< \infty
\end{equation}
$\forall i\in \mathcal V$. Recall that the integral of $v_{i,k|k}$ is sensor $i$'s posterior estimate of the cardinality of the RFS $X_k$. If $p=1$, then $v_{i,k|k}\in L^1(\mathcal X)$ if and only if sensor $i$'s cardinality estimate is bounded. Hence, the assumption that $v_{i,k|k}\in L^p(\mathcal X)$  is reasonable.

We define the vector space $V = \left(L^p(\mathcal X)\right)^{|\mathcal V|}$, so that we can collectively represent the set of intensities $\lbrace v_{i,k|k}\rbrace_{i\in\mathcal V}$ as a vector in $V$:
\[\bar v_{k|k} = \begin{bmatrix} v_{1, k|k} & v_{2, k|k} & \dots & v_{|\mathcal V|, k|k}\end{bmatrix}^\intercal \in V\]
For vectors in $V$, the addition and scalar multiplication operations are defined as the pointwise addition and pointwise scalar multiplication of their components.
We endow $V$ with a norm $\|\mathrel\cdot\|_V$, defined as
\begin{equation}
\| \bar v_{k|k}\|_V = \left(\sum_{i=1}^{|\mathcal V|} \|v_{i,k|k}\|_{L^p} ^p\right)^{\frac{1}{p}}
\end{equation}
Thus, $(V, \|\mathrel\cdot\|_V)$ is a normed vector space. For a matrix $A$, let $\|A\|_{\textsc{op}}$ refer to its operator norm\footnote{If we set $p=2$ in the foregoing definitions, then $V$ can be made into a Hilbert space, in which case $\|A\|_{\textsc{op}}$ is equal to the largest singular value of $A$ \cite[Thm. 4, p. 40]{halmos1957introduction}.}, given by
\begin{equation}
\|A\|_{\textsc{op}} = \sup \left\lbrace \frac{\|A v\|_V}{\|v\|_V} \mathrel{:} v\neq 0, v\in V \right\rbrace
\end{equation}
% We discuss the convergence of the average consensus step (\ref{eq:average_consensus}). 
% This type of abstraction is commonly used in the literature on consensus-based filtering, which either assumes that the local filters have reached their steady states, or that a large number of average consensus steps are carried out between successive measurement steps \cite{li2018partial, consensus_ukf}.
With the above definitions in place, we can state the conditions for the convergence of the average consensus step (\ref{eq:average_consensus}). 

\begin{theorem}
Suppose the following conditions are satisfied:
\begin{enumerate}
%\item $\Omega_{ij} \geq 0$ and $\Omega_{ii}>0$ ${\forall i,j\in\mathcal V}$
%\item $\Omega_{ij}\neq0$ if and only if $(j,i)\in \mathcal E$
\item $\Omega \mathbbm 1 = \mathbbm 1$, where $\mathbbm 1 = \begin{bmatrix}
1 & 1 & \dots & 1\end{bmatrix}^\intercal$,
% i.e., $\Omega$ is row stochastic
%
\item $\bar \omega^\intercal \Omega = \bar \omega^\intercal$, where $\bar \omega  = [\omega _1, \omega _2, \dots, \omega _{|\mathcal V|}]^\intercal$, and
%
\item $\|{\Omega - \mathbbm 1 \bar{\omega}^\intercal} \|_{\textsc{op}} < 1$,
% references: halmos1957introduction, bartle2014elements
\end{enumerate}
then 
% by a repeated application of the average consensus step (\ref{eq:average_consensus}), the intensities 
the fused posterior intensities $\lbrace v_{i,k|k}^{(\alpha)} \rbrace_{i\in \mathcal V}$ computed using Algorithm \ref{alg:con}
asymptotically converge to the WAA in the $L^p$ norm:
\begin{align}
   \lim_{\alpha\rightarrow \infty} \| v^{(\alpha)}_{i,k|k} - v^{*}_{k|k}\|_{L^p} \rightarrow 0, \quad \forall i\in\mathcal V
    \label{eq:convergence}
\end{align}
As a consequence, there is a subsequence $\lbrace l_1, l_2, \dots \rbrace$ such that $\lim_{s\rightarrow \infty} v^{(l_s)}_{i,k|k} = v^{*}_{k|k}$ almost everywhere \cite[p. 75]{bartle2014elements}.
\end{theorem}
\begin{proof}
To show (\ref{eq:convergence}), we can rewrite the average consensus step (\ref{eq:average_consensus}) as $ \bar v^{(l)}_{k|k} = \Omega \bar v^{(l-1)}_{k|k}$. 
%The WAA can be expressed as $v^*_{k|k} = \bar \omega ^\intercal \bar v^{(0)}_{k|k}$.  
Observe that,  given $ l \geq 1$
\begin{equation}
\bar \omega ^\intercal \bar v^{(l)}_{k|k} = \bar \omega ^\intercal \Omega \bar v^{(l-1)}_{k|k} = \bar \omega ^\intercal \bar v^{(l-1)}_{k|k}
\end{equation}
By induction, we have that $\bar \omega ^\intercal \bar v^{(l)}_{k|k} = \bar \omega ^\intercal \bar v^{(0)}_{k|k} = v^*_{k|k}$; in other words, the weighted average is invariant under the average consensus step (\ref{eq:average_consensus}).
Thus, the error dynamics corresponding to the average consensus step is
\begin{align}
\bar v^{(l+1)}_{k|k} - \mathbbm 1 \bar v^*_{k|k} &= \Omega \bar v^{(l)}_{k|k} - \mathbbm 1  \bar \omega ^\intercal \bar v^{(l)}_{k|k}\\
&=\left(\Omega  - \mathbbm 1  \bar \omega ^\intercal \right) \bar v^{(l)}_{k|k} \\
&=\left(\Omega  - \mathbbm 1  \bar \omega ^\intercal \right) (\bar v^{(l)}_{k|k}-\mathbbm 1 \bar v^*_{k|k})
\end{align}
where, in the last step, we used the facts that $\Omega \mathbbm 1 = \mathbbm 1$ and $\bar \omega ^\intercal \mathbbm 1 = 1$. Consequently, we have
\begin{equation}
\|\bar v^{(l)}_{k|k} - \mathbbm 1 \bar v^*_{k|k}\|_{V} \leq \|{\Omega - \mathbbm 1 \bar{\omega}^\intercal} \|_{\textsc{op}}^l \|\bar v^{(0)}_{k|k} - \mathbbm 1 \bar v^*_{k|k}\|_{V}
\end{equation}
which implies (\ref{eq:convergence}). 
\end{proof}

Lastly, we remark that a finite number of average consensus steps can be used in practice (i.e., $\alpha < \infty)$, with $\|{\Omega - \mathbbm 1 \bar{\omega}^\intercal} \|_{\textsc{op}}$ dictating the rate of convergence. Using (\ref{eq:csd}), it can be seen that each weighted average consensus iteration lowers the Cauchy-Schwarz divergence between the sensors' posterior intensities (which is a measure of their information difference \cite{gostar2017cauchy}).
%$\bar \omega ^\intercal \Omega = \bar ^\intercal$.

\subsubsection{Comparison with Existing Results}
Our analysis in this section extends the existing results on consensus (which has largely only considered consensus of vectors in Euclidean spaces)
to the case where consensus is sought on functions in $L^p$. For consensus in Euclidean spaces, a set of sufficient conditions for convergence is the following: 
\begin{itemize}
\item $\Omega_{ij} \geq 0$ whenever $(j,i)\in \mathcal E$,
\item $\Omega_{ii}>0$ ${\forall i\in\mathcal V}$, and 
\item the graph $\mathcal G$ is strongly connected
\end{itemize}
in which case, the fact that $\|{\Omega - \mathbbm 1 \bar{\omega}^\intercal} \|_{\textsc{op}}<1$ follows from the Perron-Frobenius theorem \cite[Lemma 1]{corless2012consensus}. Thus, the foregoing sufficient conditions for convergence are stronger than the ones we obtained in Section \ref{sec:subsec_error_analysis}.
Using the protocol in \cite{corless2012consensus}, the weights $\Omega_{ij}$ may be chosen in a distributed manner, i.e., each sensor is able to choose the weights in real-time without needing global knowledge of the topology of $\mathcal G$.

% However, note that the literature on distributed average consensus typically assumes that consensus is sought on scalars or vectors in a Euclidean space, which are equipped with the operations of scalar multiplication and addition. As distributed GM-PHD requires WAA of GMs instead, it should be ensured that these operations make sense for GMs as well. Indeed, pointwise convergence of (\ref{eq:average_consensus}) can be shown as follows. Consider any point $x\in \mathcal X$ in the domain, and note that $v_{i,k|k}(x) \in \mathbb R$. At the given value of $x$, (\ref{eq:average_consensus}) can interpreted as solving the WAA problem in the space $\mathbb R$, rather than in a function space.
% Thus, we have pointwise convergence of the recursion (\ref{eq:average_consensus}):
% \[v_{i,k|k}(x) \rightarrow v_{k|k}^*(x) \ \forall x\in \mathcal X\]
% $\forall i \in \mathcal V$, which follows directly from the standard results on consensus in $\mathbb R$.
%\cite{li2018partial, consensus_ukf}.


% To see that convergence properties of (\ref{eq:average_consensus}) indeed follow from those of average consensus in Euclidean spaces, note that the Gaussian distribution is square-integrable, so the function space of GMs is a Hilbert space (i.e., the $L^2$ space) having the (pointwise) addition and scalar multiplication operations as well as an inner product operation. The convergence proofs of the average consensus algorithms can therefore be adapted verbatim for GMs, so that the convergence of (\ref{eq:average_consensus}) (in
% the $L^2$ function norm)
% holds. 

% Such a result can be formalized using the approach of \cite{probability_consensus2014}, in which the authors study consensus in the function space of probability measures.

\subsection{Limitations of the Average Consensus Approach}
Although the average consensus algorithm (\ref{eq:average_consensus}) guarantees asymptotic convergence to the WAA, it has several limitations. Firstly, it requires each sensor to transmit $J_{i,k|k}$ GM components to its out-neighbors. This limits the number of targets that can be tracked simultaneously by the sensor network when the communication channels between the sensors have limited bandwidth, as is usually the case in wireless sensor networks. Secondly, the large number of GM components accumulated at each sensor increase the computational complexity of the subsequent local PHD filtering steps. Lastly, as noted in \cite{li2018partial}, the consensus step can exacerbate the problem of false positives in WAA fusion; small GM components corresponding to the false positive detections (i.e., clutter measurements which are misidentified as targets) can propagate through the cycles (closed loops) of the graph $\mathcal G$, leading to an overall feedback effect. In this way, false positives can get amplified during the fusion step, reducing the estimation performance of the sensor network. We address each of these concerns in the next section, by proposing a distributed protocol for dissemination and fusion of GM components that ensures asymptotic consensus and suppression of false positives, while having the same (or lower) communication bandwidth requirement than the existing distributed PHD filtering algorithms.

\section{Multi-Sensor Fusion under Communication Constraints}
\label{sec:comm_constr}

In order to carry out the fusion step of (\ref{eq:average_consensus}), the sensors must transmit a certain number of GM components to their out-neighbors. Since the dimension of the state-space $\mathcal X$ is fixed as $d_x$, each GM component is specified by $d_x + \tfrac{1}{2}d_x(d_x + 1) + 1$ floating point numbers, where the individual terms correspond to the mean vector, the covariance (which is a symmetric matrix), and the scalar weight of the GM component, respectively. In this section, we consider the case where the communication bandwidth is limited, such that the maximum number of GM components that may be transmitted across the inter-sensor communication channels at a given timestep is denoted as $B$, with $B \geq 1$.
In order to limit the communication complexity of the distributed GM-PHD algorithm, the authors of \cite{li2018partial} and \cite{li2020parallel} proposed the following heuristics for selecting the $J_{i,k|k}$ GM components that are to be transmitted at each sensor:
\begin{itemize}
    \item \textit{Rank Rule}: Rank the GM components based on their weights (highest to lowest) and select the top $B$ components
    \item \textit{Threshold Rule}: Fix a threshold, and transmit any GM components that have weights greater than this threshold
\end{itemize}
In either case, the GM components with small weights are not transmitted. This has the advantage of suppressing false positives, which typically correspond to the components having small weights. However, convergence to the WAA (\ref{eq:WAA_def}) cannot be guaranteed if the rank or threshold rule is used, as the GM components with small weights are never communicated between sensors.
Consequently, this approach is referred to as \textit{partial consensus} by its authors.

In order to guarantee asymptotic convergence of the average consensus algorithm to the WAA, we propose a new rule for selecting the GM components, which we call the \textit{sampling rule}. For simplicity, we assume that the communication is based on wireless broadcast, i.e., each sensor transmits the same message to all of its out-neighbors. Let $\mathcal B_{i}$ be a sequence of indices (numbers between $1$ and $J_{i,k|k}$) corresponding to the GM components transmitted by sensor $i$ to its out-neighbors at timestep $k$, where we omit the timestep $k$ for brevity. The total number of distinct indices in $\mathcal B_i$ is at most $B$, as per the communication constraint. The indices in $\mathcal B_i$ are chosen through random sampling, where the sampling is carried out either with or without replacement.

\subsection{Sampling with Replacement}
In order to construct $\mathcal B_i$, sensor $i$ generates a sequence of random samples of indices with replacement. In each sample,
the index $l$ is chosen with probability $p_{i}^{(l)}$, with 
\begin{equation}
    \sum_{l=1}^{J_{i,k|k}} p_{i}^{(l)} = 1
\end{equation}
%
Thus, the elements of $\mathcal B_i$ are $i.i.d.$ samples of a categorical random variable and can be thought of as the outcomes of rolling a $J_{i,k|k}$-sided biased die $|\mathcal B_i|$ times. Due to the limited communication bandwidth, the sampling process is terminated when the number of distinct indices in $\mathcal B_i$ equals $B$.
Let $\zeta^{(l)}_{i}$ denote the number of times that the index $l$ appears in $\mathcal B_i$. Observe that $(\zeta^{(1)}_{i}, \zeta^{(2)}_{i}, \dots, \zeta^{(|J_{i,k|k}|)}_i)$ follow the multinomial distribution, with $\mathbb E[\zeta^{(l)}_{i}]=p_{i}^{(l)} |\mathcal B_i|$.

Once $\mathcal B_i$ is generated, the sensor transmits the corresponding GM components to its out-neighbors. If there are any repeated components, the sensor transmits each of these components (in addition to the number of times they were sampled) only once, in order to avoid redundancy in the communication. Thus, each sensor transmits a random function (i.e., a random GM) to its out-neighbors at each timestep, given by
\begin{align}  &\sum_{l=1}^{J_{i,k|k}}\zeta^{(l)}_{i} \hspace{2pt} \tilde w^{(l)}_{i,k|k}\mathcal N(m^{(l)}_{i,k|k}, P^{(l)}_{i,k|k})
    \label{eq:random_GM}
\end{align}
where $\mathcal N(m,P)$ denotes a GM component having the mean vector $m\in \mathbb R^{d_x}$ and covariance $P\in \mathbb R^{d_x \times d_x}$.
Note that the weights of the transmitted GM components $\tilde w^{(l)}_{i,k|k}$ are different from the weights used in the local GM-PHD filtering, $w^{(l)}_{i,k|k}$.
To ensure that the fixed point of the average consensus algorithm is preserved (in expectation) by the sampling rule, we require that the expected value of the random GM is equal to the posterior intensity of sensor $i$, i.e.,
\begin{align}
\mathbb E\left[\sum_{l=1}^{J_{i,k|k}}\zeta^{(l)}_{i} \hspace{2pt} \tilde w^{(l)}_{i,k|k}\mathcal N(m^{(l)}_{i,k|k}, P^{(l)}_{i,k|k})\right] = v_{i,k|k}
\label{eq:exp_randomGM}
\end{align}
% In addition, the law of large numbers can be used to guarantee asymptotic (pointwise) convergence of the random GMs to the true posterior intensity $v_{i,k|k}$.
% From (\ref{eq:exp_randomGM}), 
Equivalently,
we have
\begin{align}
\mathbb E[\zeta^{(l)}_{i}]\hspace{2pt}\tilde w^{(l)}_{i,k|k} = w^{(l)}_{i,k|k} \label{eq:exp_weight_condition}\\ \hspace{1pt}  \tilde w^{(l)}_{i,k|k} = \frac{w^{(l)}_{i,k|k}}{p_{i}^{(l)}|\mathcal B_i|}
\label{eq:weight_condition}
\end{align}
Equation (\ref{eq:weight_condition}) imposes a constraint on the sampling probabilities $p_{i}^{(l)}$ and the weights $\tilde w_{i,k|k}^{(l)}$, but does not uniquely specify them. 
We propose the following choice of sampling probabilities:
\begin{equation}
p_{i}^{(l)}=\frac{w^{(l)}_{i,k|k}}{\sum_{l=1}^{J_{i,k|k}} w^{(l)}_{i,k|k}}
\label{eq:pl_choice}
\end{equation}
for which the corresponding weights are given by (\ref{eq:weight_condition}), as
\begin{equation}
\tilde w^{(l)}_{i,k|k} = \frac{\sum_{l=1}^{J_{i,k|k}} w^{(l)}_{i,k|k}}{|\mathcal B_i|} \triangleq \tilde w_{i,k|k}
\label{eq:final_w_tilde_def}
\end{equation}
The proposed choice of sampling probabilities, given by (\ref{eq:pl_choice}) and (\ref{eq:final_w_tilde_def}), has the clear advantage that the weights $\tilde w^{(l)}_{i,k|k}$ no longer need to be transmitted for each component, a single number $\tilde w_{i,k|k}$ may be transmitted instead, further reducing the communication requirement of the algorithm. Additionally, observe that if the sampling probabilities $p_{i}^{(l)}$ are chosen as per (\ref{eq:pl_choice}), then GM components with higher weights are transmitted more often than those with smaller weights,
%Thus, the GM components with high weights converge (as per the law of large numbers) faster than the components with small weights. 
so that the components with small weights only survive the fusion step if the targets corresponding to them are persistently detected. In this way, false positive detections that do not survive successive local PHD filtering steps are suppressed by the inter-sensor fusion step. The ability of the algorithm to suppress false alarms as well as the reduced communication cost motivate the choice of $p_{i}^{(l)}$ as (\ref{eq:pl_choice}).

\begin{algorithm}
\caption{Random Sampling Rule (with Replacement)}
\begin{algorithmic}[1]
\vspace{2pt}
\REQUIRE The maximum allowable number of GM components, $B\geq 1$ \\
\vspace{3pt} 
\WHILE{$\mathcal B_i$ has at most $B$ distinct indices}
\STATE Randomly sample an index from the sequence $\left(1, 2, \dots, J_{i,k|k}\right)$ with the corresponding sampling probabilities as given by (\ref{eq:pl_choice}).
\ENDWHILE
\STATE Each sensor broadcasts the following GM to its out-neighbors:
\[\tilde w_{i,k|k} \sum_{l=1}^{J_{i,k|k}}\zeta^{(l)}_{i} \hspace{2pt} \mathcal N(m^{(l)}_{i,k|k}, P^{(l)}_{i,k|k})
\]
where $\zeta^{(l)}_i$ is the number of times $l$ occurs in $\mathcal B_i$.
\end{algorithmic}
\label{alg:sampling}
\end{algorithm}
Thus, random sampling (as described in Algorithm \ref{alg:sampling}) can be used to replace step $3$ of Algorithm \ref{alg:con}, thereby limiting the communication cost of the distributed GM-PHD filter. From step $4$ of Algorithm \ref{alg:sampling}, we see that at each sensor, the sampling rule requires the  communication of $B$ GM components, $1$ floating point number (which is $\tilde w_{i,k|k}$), and less than $B$ integers (i.e., the numbers $\zeta_i^{(l)}$) at each iteration of the average consensus step.

Observe that, even though each sensor communicates a random GM to its neighbors, the integral of the GM is a deterministic quantity:
\begin{align}   
\tilde w_{i,k|k} \sum_{l=1}^{J_{i,k|k}}\zeta^{(l)}_{i} = \tilde w_{i,k|k} |\mathcal B_i| = \sum_{l=1}^{J_{i,k|k}} w^{(l)}_{i,k|k}
\end{align}
where in the second equality, we used (\ref{eq:final_w_tilde_def}).
Thus, while the posterior intensities of the sensors asymptotically converge to the WAA in expectation, the integrals of the intensities (i.e., the cardinality estimates of the sensors) converge deterministically.

\subsection{Sampling without Replacement}
An alternative strategy for constructing $\mathcal B_i$ is to sample the indices without replacement. As each index occurs in $\mathcal B_i$ at most once, $\mathcal B_i$ can be interpreted as a set. The random variable $\zeta^{(l)}_{i}$ defined in the previous section becomes the indicator variable $\mathbbm 1_{\lbrace l\in \mathcal B_i \rbrace}$, where
\begin{align}
    \mathbbm 1_{\lbrace l\in \mathcal B_i \rbrace} = \begin{cases}
    \begin{array}{cl}
         1& \text{if label}\ l\ \text{is in }\mathcal B_i,  \\
         0& \text{otherwise.}
    \end{array}
    \end{cases}
\end{align}
so that the condition (\ref{eq:exp_weight_condition}) now becomes
\begin{equation}
    \mathbb E[\mathbbm 1_{\lbrace l\in \mathcal B_i \rbrace}]\hspace{2pt}\tilde w^{(l)}_{i,k|k} = P (l\in \mathcal B_i )\hspace{2pt}\tilde w^{(l)}_{i,k|k} = w^{(l)}_{i,k|k} 
\end{equation}
where $P({}\cdot{})$ denotes the probability of an event. Thus, once the 
probabilities $P (l\in \mathcal B_i )$ are computed,
the weights $\tilde w^{(l)}_{i,k|k}$ must be chosen accordingly, as
\begin{equation}
    \tilde w^{(l)}_{i,k|k} = \frac{w^{(l)}_{i,k|k}}{P( l\in \mathcal B_i )}
    \label{eq:no_replacement_weights}
\end{equation}
An efficient method for sampling without replacement can be found in \cite{Efraimidis2015}, which is able to assign a higher probability to components with higher weights, thereby incorporating the ability to suppress false positives into the fusion step.

Note that it is no longer straightforward to specify the inclusion probabilities of the indices $P( l\in \mathcal B_i )$ \textit{a priori}, like we did in (\ref{eq:pl_choice}). To see this, consider the case where $B=J_{i,k|k}$, which fixes the probabilities as $P( l\in \mathcal B_i )=1$ for all the indices. Thus, if we were to specify the probabilities $P( l\in \mathcal B_i )$ \textit{a priori}, it can make the sampling problem infeasible; the same is true even when $B<J_{i,k|k}$ \cite[Example 2]{Efraimidis2015}. Moreover, when using the sampling without replacement rule, in addition to transmitting the mean vectors $m^{(l)}_{i,k|k}$ and covariances $P^{(l)}_{i,k|k}$ of the selected components, the sensors must transmit the modified weights $\tilde w_{i,k|k}$ as well. In contrast, the proposed sampling with replacement rule (given in Algorithm \ref{alg:sampling}) allows one to specify the sampling probabilities \textit{a priori}, and also has a lower communication bandwidth requirement.
% $P( l\in \mathcal B_i )$ is also more difficult to compute, since the probability of a given GM component being chosen is dependent on the probability that

In summary, the proposed distributed GM-PHD algorithm uses a local GM-PHD filter at each sensor to update its multi-target estimate, followed by a given number of average consensus steps (as per Algorithm \ref{alg:con}) to fuse the multi-target estimates of the sensors. Additionally, the proposed random sampling rule (given in Algorithm \ref{alg:sampling}) is used to limit the communication bandwidth of the approach, while ensuring its asymptotic optimality.

\documentclass{aastex6}
%\documentclass{emulateapj}
%\documentclass[apj,onecolumn,tighten]{emulateapj}
%\documentclass{aastex}
\usepackage{mathrsfs}
\usepackage{amsmath}
\usepackage{color}
\usepackage{xcolor}
%-----------------------------------------------------------------------------
\begin{document}
%------------------------------------------------------------------------------

\title{Numerical Simulations of a Jet-Cloud Collision and Starburst:  Application to Minkowski's Object}
\shorttitle{Simulations of a Jet-Cloud Collision}
\shortauthors{Fragile, et al.}
\author{P. Chris Fragile}
\affil{Department of Physics \& Astronomy, College of Charleston, Charleston, SC 29424, USA}
\email{fragilep@cofc.edu}
\author{Peter Anninos}
\affil{Lawrence Livermore National Laboratory, Livermore, CA 94550, USA}
\author{Steve Croft}
\affil{Astronomy Department, University of California, Berkeley, 501 Campbell Hall \#3411, Berkeley, CA 94720, USA}
\affil{Eureka Scientific, Inc. 2452 Delmer Street Suite 100, Oakland, CA 94602, USA}
\author{Mark Lacy}
\affil{National Radio Astronomy Observatory, 520 Edgemont Road, Charlottesville, VA 22903, USA}
\author{Jason W. L.  Witry}
\affil{Department of Physics \& Astronomy, College of Charleston, Charleston, SC 29424, USA}

\begin{abstract}
We present results of three-dimensional, multi-physics simulations of an AGN jet colliding with an intergalactic cloud.  The purpose of these simulations is to assess the degree of ``positive feedback,'' i.e. jet-induced star formation, that results.  We have specifically tailored our simulation parameters to facilitate comparison with recent observations of Minkowski's Object (M.O.), a stellar nursery located at the termination point of a radio jet coming from galaxy NGC 541.  As shown in our simulations, such a collision triggers shocks which propagate around and through the cloud.  These shocks condense the gas and under the right circumstances may trigger cooling instabilities, creating runaway increases in density, to the point that individual clumps can become Jeans unstable.  Our simulations provide information about the expected star formation rate, total mass converted to \ion{H}{1}, H$_2$, and stars, and the relative velocity of the stars and gas.  Our results confirm the possibility of jet-induced star formation, and agree well with the observations of M.O.  
\end{abstract}
\keywords{galaxies: individual (Minkowski's Object) --- galaxies: jets --- hydrodynamics --- intergalactic medium --- shock waves}
\maketitle


\section{Introduction}

The interaction between high-energy jets from active galactic nuclei (AGN) and their surroundings has long been a topic of great astrophysical interest. It is well known that AGN feedback can control the size of a galaxy by influencing star formation, but the mechanism behind this is not well understood.  Several recent observations \citep{Nesvadba10,Guillard15}, as well as numerical studies \citep[e.g.][]{Sutherland07,Antonuccio08,Gaibler12}, have demonstrated that AGN feedback can be either ``negative'' or ``positive.''

The exact astrophysical conditions in the jet and cloud are important in determining what direction feedback takes. Jets can be roughly divided according to their Fanaroff-Riley classification \citep{FR74}. Fast, energetic FRII jets seem more likely to result in negative feedback. Negative feedback curbs or even halts star formation, and is thought to result from the extreme radiative and kinetic energies of the jet, which heat and disperse the star-forming gas. Additionally, the kinetic energy of a jet creates turbulence that can prevent ambient gas from cooling and subsequently coalescing \citep[e.g.][]{Nesvadba10}. For example, a study of the system 3C 326 by \citet{Ogle07} found that, despite the strong H$_{2}$ line emission and an inferred molecular gas mass of 2 $\times 10^{9}$ M$_{\odot}$, the star formation is 20 times lower than predicted by the Kennicutt-Schmidt law.  They infer that turbulent heating from the jet is inhibiting star formation. Other studies, though, suggest increased star formation may be seen in the cocoon of such jets \citep[e.g.][]{Gaibler12}. 

In contrast, FRI jets propagate through the ISM/IGM with energies high enough to create compression in the surrounding gas, but low enough to reduce  the chance of significant turbulent heating. These jets are observed in positive feedback cases,  wherein the effect of the jet serves to enhance star formation, including Centaurus A \citep[][and references therein]{Salome16}, 4C 41.17 \citep{Bicknell00}, and Minkowski's Object (hereafter M.O.) \citep[][hereafter C06]{Croft06}. We give more details on both positive and negative feedback in Section \ref{sec:jetcloud}.  

In this paper, we focus on the case of M.O., a peculiar star forming object located at a redshift of $z = 0.0189$ (C06) that is currently being bombarded by a FR I radio jet from the nearby galaxy NGC 541. The M.O. system is of particular interest due to the lack of evidence for an especially dense ISM or IGM. There is also not much evidence for cold gas outside of the jet interaction site, unlike in Centaurus A. As a result, it is unlikely that significant star formation would proceed in M.O. without the interaction of the jet. 

A strong argument in favor of jet-induced star formation in M.O. is the morphology of the jet-cloud interaction site. Outside of the jet interaction, the gas in M.O. is warm ($\sim 10^4$ K) and clumpy. Near the jet interaction site there is a double structure of \ion{H}{1} gas wrapped around the jet and numerous \ion{H}{2} regions (C06). C06 thus determined that it is likely that the jet interaction in M.O. caused the warm gas to cool into the \ion{H}{1}, in contrast to the pre-existing cold gas regions in Centaurus A. Also, the star forming regions in M.O. correlate with the jet-cloud morphology; the region where the star formation is the highest is the center of the jet-cloud interaction, and the star formation rate (SFR) decreases laterally from this point (C06). 

M.O. may, in fact, be a low redshift example of the type of jet-induced star formation that was perhaps more common in the early universe. Evidence for this is the similarity between M.O. in ultraviolet and the rest-frame UV morphology of suspected jet-induced star-forming regions around high-redshift radio galaxies.

The present work can be viewed as an extension of our earlier study of the interactions of radiative shocks with clouds \citep{Fragile04}. That work focused mostly on the effects of planar shocks overtaking individual (or small collections of) warm clumps on the scale of $\sim 100$ pc.  In the current work, we explore the much richer problem of a full jet intersecting an inhomogeneous intergalactic cloud on the scale of tens of kpc. The paper proceeds as follows: Section 2 covers the theory behind jet-cloud interactions, Section 3 describes the numerical models used to capture the M.O. system, Section 4 details the simulation results, and Section 5 concludes the findings.


\section{Jet-Cloud Interactions}
\label{sec:jetcloud}

The basic idea of jet-induced star formation (i.e. positive feedback) is that the collision of the jet with the cloud will trigger a series of shocks within the cloud.  The immediate effect of these shocks will be to compress and heat the gas. Depending on how the radiative processes scale with density and temperature, the net result can be to dramatically increase the radiative efficiency within the cloud. If the temperature dependence is shallower than the density dependence, the cloud can enter a phase of runaway cooling.  This process occurs most quickly in relatively over-dense regions of the original cloud.  These over-dense regions then proceed to collapse at an accelerating pace.  Provided some of these clumps start sufficiently close to the Jeans limit, this collapse will push them beyond this limit, such that gravitational collapse can take over and the clump will proceed to form stars. 

The properties of the jet and cloud are key to controlling this process.  For positive feedback to be important, the initial cloud must be dense enough for some parts to be reasonably close to the Jeans limit.  The temperature must also be such that any increase in temperature is met with a dramatic increase in cooling (the hydrogen cooling edge at $\sim 10^4$ K is a good example).  It $\bf{also}$ generally helps for the jet to be significantly less dense than the cloud.  Finally, the jet velocity needs to be fast enough to trigger shocks in the cloud, yet not so fast that the cloud is disrupted before cooling can have much of an effect.


\section{Numerical Models}

Our numerical simulations are performed using the well-tested {\em Cosmos++} computational astrophysics code \citep{Anninos05}, specifically its Newtonian hydrodynamics solvers.  {\em Cosmos++} carries over many of the multi-physics capabilities found in its predecessor code {\em Cosmos} \citep{Anninos03}.  The Newtonian solvers have previously been utilized to study the bar mode instability in magnetized, rotating neutron stars \citep{Camarda09} and the galactic center G2 event \citep{Anninos12}.  The current work uses the High Resolution Shock Capturing (HRSC) scheme, which was described in its relativistic form in \citet{Fragile12}.  As there are few differences between the relativistic and Newtonian forms, we do not give a full presentation here, focusing instead on the packages that are most important to this paper: chemistry, cooling, and star formation. Note that, although magnetic fields can play an important role in shock-induced star formation \citep[cf.][]{Fragile05}, they are not considered in this work.

Most prior numerical studies of stimulated star formation from jet-generated shocks have been hampered by the resolution of the computational mesh \citep[e.g.][]{Fragile04}.  In the present work, we significantly improve on previous resolution limitations by employing the adaptive mesh refinement (AMR) capabilities of {\em Cosmos++}.  {\em Cosmos++} employs a {\em local} AMR scheme, in which refinement and de-refinement decisions are made on a cell-by-cell basis, using an oct-tree network to traverse the grid hierarchy \citep{Anninos05}.  Each level of refinement doubles the spatial resolution in each dimension within a given parent cell.  This style of local AMR scheme ensures that the refinement and de-refinement conform as closely as possible to the shape of the region of interest, in this case the shocks and unstable cooling fronts triggered inside the cloud by the jet. The AMR capabilities and the improvements in computing power have also allowed us to move from two-dimensional to more realistic three-dimensional simulations. Other improvements over our previous work include: simulating an object the size of M.O., instead of much smaller cloudlets; inputting a realistic jet, instead of a planar shock; inclusion of a star-formation prescription; and inclusion of dust grain chemistry.

In this work, we present an idealized case of a direct, axially symmetric collision between a jet and a pre-existing spherical cloud.  As such, it is a simple, and well controlled, test simulation, albeit with imperfect correspondence to M.O. In the case of M.O., it is thought that the collision was between a jet and a stellar bridge connecting the elliptical galaxy, NGC 541, with the interacting galaxies, NGC 545/547 (C06).  There is also evidence that the jet is slowly sweeping across M.O. (C06).  Despite these differences, the correspondence in parameters between our simulation and M.O. ensure that our results are applicable and the simulation can be used to better understand the dynamics of this particular object and of jet-induced star formation more generally.


\subsection{Simulation Setup}

Although we performed some two-dimensional simulations to test different code options and explore our parameter space, we focus on reporting the results of our 3D simulations.  The 3D simulations have a base resolution of $384\times128\times128$ zones to cover a domain that is approximately $30~\mathrm{kpc} \times 10~\mathrm{kpc} \times 10~\mathrm{kpc}$ with reflection boundaries applied in the $y$- and $z$-directions, so that we only simulate one quadrant of the full problem. The finest spatial resolution achieved is 19.5 pc per zone, reached by including 2 levels of refinement on top of the $384\times128\times128$ base mesh, equivalent to a uniform mesh of $1536\times512\times512$ zones.  The criterion used for refinement is that any zone with $n \ge 0.01$ cm$^{-3}$ is kept at the maximum refinement, while zones that fall below $n < 0.0005$  cm$^{-3}$ are allowed to de-refine, provided neighboring zones never differ by more than one level of refinement and no zone is allowed to drop below the base resolution. Zones are checked against the refinement and de-refinement criteria once every ten evolution steps in the numerical code.  We find that a minimum resolution close to our base value is required even in the background in order to get reasonable convergence in the star formation rate.

The cloud, which represents the parent object of M.O., is initialized with a radius of $R_\mathrm{cl} = 7.5$ kpc.  It is, therefore, somewhat smaller, in terms of projected area, than the real value of 275 kpc$^2$ (C06). The cloud is modeled as non-self-gravitating gas within a fixed 
dark-matter potential.  The omission of self-gravity is reasonable, given that the gravitational potential in an object like M.O. will be dominated by dark-matter \citep{Persic96}. The shape of the potential is given by a modified Hubble profile \citep{Binney87}
\begin{equation}
\phi(r<R_t) = \frac{G \tilde{M}}{R_c} \left\{ 1 - \frac{\mbox{ln}
[x + (1+x^2)^{1/2}]}{x} \right\} ~,
\end{equation}
where $x=r/R_c$, $R_c = 0.5 R_\mathrm{cl}$ is the core radius, and
\begin{equation}
\tilde{M} = M_d \left\{ \mbox{ln}[x_t +(1+x_t^2)^{1/2}] - 
x_t(1+x_t^2)^{-1/2} \right\}^{-1} ~,
\end{equation}
where $M_d = 10^{11} M_\odot$ is the dark-matter mass, $x_t=R_t/R_c$, and $R_t = 10 R_\mathrm{cl}$ is the tidal radius. %The motivation for using such a relatively large dark matter mass is to make the cloud core dense enough that cooling will be efficient once the shock hits it. A shallower, more uniform density profile results in only weak cooling.
The gas is initialized to be isothermal, with $T_\mathrm{cl} = 2 \times 10^5$ K, and in hydrostatic equilibrium within the potential, such that the density
\begin{equation}
\rho_\mathrm{cl} \propto e^{-\phi/c_s^2} ~,
\label{eqn:densitylaw}
\end{equation}
where $c_s$ is the isothermal sound speed. The gas within this potential is made clumpy by overlaying a random, log-normal distribution of the form $\tilde{n} e^{X\sigma}$, with standard deviation $\sigma = \sqrt{2 \ln (\bar{n}/\tilde{n})}=0.05$ and $\bar{n} = 0.5$ cm$^{-3}$, where $X$ is a randomly drawn variable with a mean of 0 and variance of 1. The normalization is such that the total gas mass within $R_\mathrm{cl}$ is $M_g = 1.4 \times 10^9 M_\odot$, which we show gives about the right mass of \ion{H}{1} ($4.9 \times 10^8 M_\odot$; C06), and giving an overall average density in the cloud of $\bar{\rho}_\mathrm{cl} = 5.3 \times 10^{-26}$ g cm$^{-3}$. The gas is initialized to have a mean molecular weight of $\mu = 1.3$, appropriate for neutral, solar metallicity gas, although this only affects the initialization, as $\mu$ is subsequently solved for self-consistently by the chemistry package. The cloud is immersed in a background gas with $n_\mathrm{b} = 10^{-4}$ cm$^{-3}$ and $T_\mathrm{b} = 5 \times 10^{7}$ K, such that the cloud and background are initially in approximate pressure equilibrium at the cloud surface. To avoid numerical problems caused by the background gas density dropping too far below its initial value, a density floor of $10^{-7}$ cm$^{-3}$ is set. Each simulation is run for a total of four sound-crossing-times, $R_\mathrm{cl}/c_{s,\mathrm{b}}$, of the cloud in the background gas, which corresponds to about 40 Myr total.

A jet is introduced into the simulation domain through one end. First, we estimate the power of the jet impacting M.O. from the claimed correlation with radio luminosity, $P_\mathrm{jet} \approx 7.2 \times 10^{36} (L_\mathrm{rad}/10^{30}~\mathrm{erg~s}^{-1})^{12/17} ~\mathrm{erg~s}^{-1}$ \citep{Kording08}, using the measured radio luminosity of NGC 541, $L_\mathrm{rad} \approx 10^{41}$ erg s$^{-1}$ \citep{vanBreugel85}. This gives a value for the kinetic power of $4 \times 10^{44}$ erg s$^{-1}$. This estimate is roughly consistent with independent estimates of the jet power from the other radio source in Abell 194, 3C40B, based on X-ray cavity energetics \citep{Bogdan11}. Since both sources have similar radio luminosities, it seems reasonable that their jet powers would be similar, too. In order to translate this jet power into simulation variables, we are also guided by dynamical models for jet deceleration in intergalactic environments \citep[e.g.][]{Laing02}. We settle on the following jet parameters: a density of $n_\mathrm{jet} = 5 \times 10^{-6}$ cm$^{-3}$, a diameter of $D_\mathrm{jet} = 5$ kpc, and a velocity of 10\% of the speed of light. For a cylindrical jet of cross-sectional area $A_\mathrm{jet}$, this yields a kinetic power, $P_\mathrm{jet} = \rho_\mathrm{jet} A_\mathrm{jet} v_\mathrm{jet}^3 = 6 \times 10^{43}$ erg s$^{-1}$, somewhat below the power estimate above, consistent with the jet having dissipated some fraction of its power prior to reaching M.O.


\subsection{Chemistry Models}

In our simulations, we follow the abundances of 9 atomic and molecular species: \ion{H}{1}, \ion{H}{2}, \ion{He}{1}, \ion{He}{2}, \ion{He}{3}, $e^-$, H$^-$, H$_2$, and H$_2^+$. The evolution of each species is governed by an equation of the form
\begin{equation}
\frac{\partial \rho^{[m]}}{\partial t} +\nabla \cdot (\rho^{[m]} \mathbf{v})
        = {\sum_{i=1}^{N_{s}}}{\sum_{j=1}^{N_{s}}} 
                            {{k_{ij}(T)}{\rho^{[i]}}{\rho^{[j]}}}
            + \sum_{i=1}^{N_{s}} {I_i (\nu) \rho^{[i]}} e^{-\tau_i}.
\label{eqn:dens_m}
\end{equation}
These rate equations are solved using a stable, semi-implicit, backward difference scheme that we developed in \citet{Anninos97}, which has since become a standard method adopted by the general community \citep[e.g.][]{Smith16} due to its combination of robustness, efficiency, and accuracy.  A total of 27 gas-phase chemical reactions are included in the full network,  including 19 collisional ($k_{ij}$) and 8 photoionization/photodissociation ($I_i$) processes.  The exact reaction chains are spelled out in \citet{Anninos03}.  The photoionization field is set to $10^{-21}$ s$^{-1}$, appropriate for cosmic UV background radiation at low redshift \citep{Bechtold87}, while the photodissociation rate is $5.0 \times 10^{-11}$ s$^{-1}$, appropriate for the local interstellar medium \citep{Spaans97}. For photoionization, the products of the external field and the respective interaction cross sections are each integrated over frequency to derive effective photoionization rates for H, He, and molecules. To account for self-shielding within the cloud, we approximate the optical depth as $\tau_i = \sigma_i n_i \Delta l$, where $\sigma_{\mathrm{H}_\mathrm{I}} = 6.3 \times 10^{-18}$ cm$^2$ \citep{Osterbrock89}, $\sigma_{\mathrm{H}_2} = 5.2 \times 10^{-18}$ cm$^2$ \citep{Hollenbach71}, and $\Delta l$ is the length of a typical zone.

We account for the effect of dust grains by adding an extra reaction to the network consisting of collisional interactions between grains and hydrogen atoms to enhance the production of molecules. Dust grains can also absorb and emit radiation, effectively acting as an additional cooling or heating mechanism. Both collisional and cooling rates are sensitive functions of the gas and grain temperatures, and of the grain size.
We adopt the grain reaction and cooling models of \cite{Hollenbach79} and \cite{Omukai00} for this work, assuming a characteristic grain size and temperature of 100 {\AA} and 10 K, respectively.

\subsection{Cooling Models}

The energy equation 
\begin{equation}
\frac{\partial E}{\partial t} +\nabla \cdot [(E+P) \mathbf{v}]
        = -\Lambda(T,n^{[m]})~,
\label{eqn:dual_en}
\end{equation}
where $E = e + \rho v^2/2$ is the total energy density, including its internal and kinetic contributions, accounts for the cooling and heating of the gas via a total of eight different mechanisms: collisional-excitation, collisional-ionization,
recombination, bremsstrahlung, metal-line cooling (dominantly carbon, oxygen, neon, and iron), molecular-hydrogen cooling, dust cooling, and photoionization heating. The cumulative cooling function is
\begin{equation}
\Lambda(T,n^{[m]}) =
    \sum_{i=1}^{N_{s}} \sum_{j=1}^{N_{s}}  \dot{e}_{ij}(T) n^{[i]} n^{[j]}
    - \sum_{i=1}^{N_{s}} J_i n^{[i]}
    + \dot{e}_M(T) n^2 ~,
\label{eqn:coolingchem}
\end{equation}
where $\dot{e}_{ij}(T)$ are the cooling rates from 2-body interactions between species $i$ and $j$, and $J_i$ represents the frequency-integrated photoionization and photodissociation heating rates. $\dot{e}_M$ is the temperature-dependent cooling rate for metals (assuming solar metallicity), taken from \citet{Dalgarno72}. A cooling floor is set at $T_\mathrm{floor} = 10$ K, below which only adiabatic cooling is possible.


\subsection{Star Formation Model}

In this study, we are particularly interested in tracking the formation of stars within the cloud.  We follow the approach of \citet{Rasera06} in defining a density threshold, $n_\star$, above which star formation is triggered at a rate given by
\begin{equation} 	
\dot{\rho}_\star = \frac{\epsilon \rho}{t_\mathrm{ff}} 
\label{eqn:sfr} 
\end{equation}
where $t_\mathrm{ff} = \sqrt{3 \pi/32 G \rho}$ is the local free-fall time and $\epsilon$ controls the star formation efficiency. We tested values of $n_\star$ between 0.5 and 4 cm$^{-3}$ and $\epsilon$ between 0.02 and 0.1, settling in our two highest resolution simulations on $n_\star = 1$ cm$^{-3}$ and $\epsilon = 0.02$. This star formation appears as a sink term in the continuity equation
\begin{equation}
\frac{\partial \rho}{\partial t} +\nabla \cdot (\rho \mathbf{v}) = -\dot{\rho}_\star ~.
\label{eqn:dens}
\end{equation}
By tracking how much mass is lost to star formation during each compute cycle, we are able to continuously track the star formation rate throughout the simulation. 

Additionally, we use tracer particles to track the most massive ($M_\star \ge 0.02 M_\odot$) ``stars'' created in this way.  Each star particle is given an initial velocity equal to the velocity of the gas in the zone in which the star is created. The particles are then fed into a post-processing routine, which integrates their motion through the dark-matter potential. This is not entirely realistic as some momentum may get redistributed during the free-fall process that leads to star formation, but we at least capture the dominant force that would act on the stars once they form.

We are justified in our neglect of self-gravity in following the collapse of the cloud, because the initial Jeans radius
\begin{equation}
R_J = \left(\frac{15 kT}{4\pi G\mu m_H \rho}\right)^{1/2}
\end{equation}
of the cloud is 21 kpc, more than twice the size of the cloud. Even at the densities and temperatures typical of star formation in our simulations ($\rho_\star \approx 2 \times 10^{-24}$ g cm$^{-3}$ and  $T_\star \approx 1000$ K), the Jeans length is still 240 pc, which is considerably less than the original radius of the cloud, yet well above the resolution limits of our simulations. More importantly, it is comparable to the size scales of the regions that exceed the star formation criteria, meaning the clumps are just becoming Jeans unstable whenever our star formation model kicks in and starts converting gas in these regions to stars. In other words, just when self-gravity would be taking over is when our star formation model kicks in.


\section{Results}

Table \ref{tab:params} summarizes the 3 simulations presented in this work.  The naming convention refers to the base resolution and how many total grid resolution levels there are.  Figure \ref{fig:volume} presents volume visualizations from an intermediate- and the end-time of our highest resolution 3D simulation (384x128x128\_3level).  A number of general features are apparent.  First, the cloud is dense enough and has enough inertia to dramatically slow the propagation of the jet (represented by its temperature in red).  At the speed the jet is traveling, if not for the cloud (and, to a lesser extent, the background gas) impeding its progress, it should have traversed 41 box lengths (164 cloud radii) over the duration of the simulation.  Instead, most of the jet material is deflected to the sides of the cloud, though a significant fraction of its energy is deposited within the cloud gas.  The blue material represents neutral hydrogen (\ion{H}{1}) and shows that the core of the original cloud remains relatively intact until late times. At the head of the jet, where it is interacting with the cloud, a thin layer of very dense, cold gas has formed (represented by the green, H$_2$, gas). This is where star formation is expected to occur. This figure shows qualitative similarities with Fig. 1 of \citet[][hereafter L17]{Lacy17}, which reports ALMA observations of M.O. 

\begin{deluxetable}{cccc}
\tablecaption{Jet-Cloud Models and Parameters \label{tab:params}}
\tablecolumns{4}
\tablehead{
 &  & \colhead{$n_\star$} & \\
\colhead{Name\tablenotemark{a}} & \colhead{$N_l$\tablenotemark{b}} & \colhead{(cm$^{-3}$)} & \colhead{$\epsilon$} }
\startdata
384x128x128 & 1 & 1 & 0.1  \\
384x128x128\_2level & 2 & 1 & 0.02  \\
384x128x128\_3level & 3 & 1 & 0.02  \\
\enddata
\tablenotetext{a}{Each simulation has a base resolution of $384\times 128\times128$.}
\tablenotetext{b}{$N_l$ is the total number of grid refinement levels.}
\end{deluxetable}


%\clearpage
\begin{figure}
\includegraphics[width=0.75\columnwidth]{volume_30Myr.pdf} 
\includegraphics[width=0.75\columnwidth]{volume_40Myr.pdf} 
\caption{Volume visualization from an intermediate ($t=30$ Myr) and the final ($t=40$ Myr) time dumps of our highest resolution simulation (384x128x128\_3level).  Red represents hot, $T > 2\times 10^8$ K, jet material; blue represents regions of the cloud with a neutral hydrogen density, $n_\mathrm{HI} \gtrsim 0.1$ cm$^{-3}$; and green represents regions of the cloud with a molecular hydrogen (our tracer for cold gas) density, $n_\mathrm{H2} \gtrsim 10^{-5}$ cm$^{-3}$.  Data have been reflected across the $y=0$ and $z=0$ planes to create this image.  Axes are marked in units of cloud radii, $R_\mathrm{cl}$.
\label{fig:volume}}
\end{figure}


\subsection{Shock Propagation}
\label{sec:shock}

In Figure \ref{fig:volume}, we can clearly see the termination shock at the end of the collimated jet. Ahead of it is a compression shock being driven into the cloud (not seen in Figure \ref{fig:volume}, but lying just ahead of the cold, dense H$_2$ gas seen in green). We can compare its position with the estimated shock velocity in the cloud. If we take our jet velocity, $v_\mathrm{jet} = 3 \times 10^4$ km s$^{-1}$, as the speed of the post-shocked material inside the jet, then we can use the usual jump conditions, 
\begin{equation}
v_\mathrm{ps} = \left(1 - \frac{\Gamma-1}{\Gamma+1}\right) v_\mathrm{sh} ~,
\end{equation}
where $v_\mathrm{ps}$ and $v_\mathrm{sh}$ are measured in the rest frame of the pre-shock gas, to estimate the speed of the jet shock, $v_\mathrm{sh,jet} \approx 4 v_\mathrm{jet}/3 = 4 \times 10^4$ km s$^{-1}$. If the shock in the jet is strong, then the post-shock pressure is approximately $\rho_\mathrm{jet} v_\mathrm{sh,jet}^2$. If we assume the shocks are also strong inside the background and that the post-shock jet and background gas reach pressure equilibrium, then we can estimate the speed of the shock in the background 
\begin{equation}
v_\mathrm{sh,b} \simeq \left(\frac{\rho_\mathrm{jet}}{\rho_\mathrm{b}}\right)^{1/2} v_\mathrm{sh,jet} = 8.9 \times 10^3 ~\mathrm{km\,s}^{-1}.
\end{equation}
If we likewise assume the shocks are strong in the cloud and take a characteristic cloud density of $\rho_\mathrm{cl} = 2.2 \times 10^{-25}$ g cm$^{-3}$, we get a shock speed in the cloud of
\begin{equation}
v_\mathrm{sh,cl} \simeq \left(\frac{\rho_\mathrm{b}}{\rho_\mathrm{cl}}\right)^{1/2} v_\mathrm{sh,b} = \frac{v_\mathrm{sh,b}}{\chi^{1/2}} = 280 ~\mathrm{km\,s}^{-1},
\end{equation}
where $\chi = \rho_\mathrm{cl}/\rho_\mathrm{b}$ is the ratio of cloud to background density. According to this, the shock should traverse the cloud core in about 26 Myr, which looks to be roughly consistent with the shock progression seen in Figure \ref{fig:volume}.


\subsection{Cooling Front}
\label{sec:cooling_front}

As mentioned in Section \ref{sec:jetcloud}, for the jet feedback to be positive, it is critical for the cooling timescale to be shorter than the disruption timescale of the cloud, taken to be the shock-crossing time. Following \citet{Fragile04}, we estimate the cooling time to be
\begin{equation}
t_\mathrm{cool} = (7.0 \times 10^{-35}~\mathrm{g~cm}^{-6}~\mathrm{s}^4) \frac{v^3_\mathrm{sh,cl}}{\rho_\mathrm{cl}} \simeq 2.2 \times 10^{5}~\mathrm{yr} ~.
\end{equation}
This is {\it much} shorter than any other relevant timescale in the problem. Figure \ref{fig:formation} shows how this cooling front and the associated star formation progresses over the course of the simulation.

%\clearpage
\begin{figure}
\includegraphics[height=0.5\columnwidth]{star_formation.pdf} 
\caption{Isosurface plot showing regions of the simulation domain where the density exceeds the star formation limit ($n > n_\star$) at $t = 10$ (yellow), 20 (blue), 30 (green), and 40 Myr (orange) from our highest resolution simulation (384x128x128\_3level). Data have again been reflected across the $y=0$ and $z=0$ planes. Axes are marked in units of cloud radii, $R_\mathrm{cl}$. 
\label{fig:formation}}
\end{figure}


\subsection{Star Particles}
\label{sec:star_particles}

As a reminder, whenever more than $0.02 M_\odot$ of gas is converted into stars within a given zone within a single cycle, then a star particle is created to track the properties, such as position, age, and velocity, of that ``star.''  Following this prescription, we created over $2.3 \times 10^7$ star particles in our lowest resolution simulation (384x128x128).  The star particles span an age range from 0 -- 32 Myr. However, this ``age'' does not correspond directly to the age of a star, as we do not account for the freefall and pre-main-sequence lifetimes of each. At best, the star particle ages give an estimate of the range of ages that may be expected and, as shown in Figure \ref{fig:stars}, some feeling for the spatial distribution of ``young'' and ``old'' stars.

An interesting point about the stellar ages in our simulations is that they show a negative curvature along the direction of jet propagation, that is, the youngest (currently forming) stars are found between two populations of slightly older stars, one in the upstream direction and one downstream. The upstream population are stars that formed recently, only slightly before the current star formation.  The downstream population are some of the first stars to form from the jet interaction, but they are now actually located ahead of the current star formation front because they received a velocity kick larger than the current shock speed.  Since the star particles are not coupled to the gas, they can actually pass ahead of the shock as it slows down, giving an apparently older population ahead of the current star formation front.  

%\clearpage
\begin{figure}
\includegraphics[width=0.5\columnwidth]{particle_age.png} 
\includegraphics[width=0.5\columnwidth]{particle_velocity.png} 
\caption{Plot of spatial distribution (projected into the $x$-$y$ plane) of star tracer particles from the final time dump ($t=40$ Myr) of our lowest resolution 384x128x128 simulation.  In the left panel, the color of each particle indicates its age (in Myr), while in the right panel, the color indicates velocity magnitude (in km s$^{-1}$). 
\label{fig:stars}}
\end{figure}

Figure \ref{fig:stars} (right panel) shows the spatial distributions of the star particles, colored by the magnitude of their velocities.  As mentioned above, the fastest moving star particles are on the downstream edges of the distribution. By using our results from Sec. \ref{sec:shock} and inverting the shock jump condition, we predict a post-shock velocity in the cloud of $v_\mathrm{cl,ps} = 3 v_\mathrm{sh,cl}/4 \simeq 210$ km s$^{-1}$. Since our star particles are assigned their velocity based upon the velocity of the gas from which they form, we expect the simulated star particle velocities to be similar.  Figure \ref{fig:histogram} shows a histogram of the velocity distribution of all of the star particles. The distribution peaks at around 85 km s$^{-1}$, which is fairly consistent with our crude predictions. The spread in our velocity distribution is also fairly consistent with the $\sim 40$ km s$^{-1}$ of velocity shear observed in M.O. (L17).

%\clearpage
\begin{figure}
\includegraphics[width=0.5\columnwidth]{vel_bar.pdf} 
\caption{Histogram of the velocity magnitudes of all the star particles formed in our 384x128x128 simulation. The peak of the distribution is roughly consistent with observations of M.O.
\label{fig:histogram}}
\end{figure}



\subsection{Star Formation Rate}
\label{sec:sfr}

Figure \ref{fig:sfr} shows the measured star formation rate (S.F.R.) as a function of time for two different 3D simulations done at different effective resolutions, from 39.1 pc per zone (384x128x128\_2level) to 19.5 pc per zone (384x128x128\_3level). Up to a point, we expect the simulated S.F.R. to be sensitive to resolution. At higher resolutions, more gas is going to be able to reach the density threshold, $n_\star$, of our star formation model. However, a limit should be reached where the high density filaments are well enough resolved that their size and peak density are no longer functions of resolution. It appears we may have reached this point in our highest resolution simulations, as they track each other closely. Equally important, we achieved an S.F.R., or correspondingly an H$_\alpha$ luminosity, since \citep{Kennicutt98}
\begin{equation}
\left(\frac{\mathrm{SFR}}{M_\odot~\mathrm{yr}^{-1}} \right) = 7.9 \times 10^{-42} \left(\frac{L_{H_\alpha}}{\mathrm{erg~s}^{-1}}\right) ~,
\end{equation}
consistent with the observed value in M.O. \citep[$\mathrm{S.F.R.} = 0.47 M_\odot$ yr$^{-1}$ or $L_{H_\alpha} = 5.9 \times 10^{40}$ erg s$^{-1}$;][]{Salome15}.  In fact, we overshoot the observed S.F.R. after about 20 Myr. However, we are neglecting negative feedback effects, such as supernovae from the first generation of stars, which would damp this rate.

%\clearpage
\begin{figure}
\includegraphics[width=0.5\columnwidth]{sfr.pdf} 
\caption{Star formation rate history for the two highest resolution simulations. After about 20 Myr, the S.F.R. in both simulations exceeds the current observed rate in M.O. of $\approx 0.5 M_\odot$ yr$^{-1}$ (grey, dashed line, corresponding to an H$_\alpha$ luminosity of $7 \times 10^{40}$ erg s$^{-1}$).
\label{fig:sfr}}
\end{figure}

 
\subsection{Other Comparisons with Observations}

Figure \ref{fig:history} tracks the total mass of \ion{H}{1}, H$_2$, and stars over the course of our highest resolution simulation. All of these measures are within about a factor of two of their observed values. The simulation slightly overproduces \ion{H}{1} [$9.2 \times 10^8 M_\odot$ in the simulation vs. $4.9 \times 10^8 M_\odot$ for M.O. (C06)], slightly underproduces H$_2$ [$1.3 \times 10^7 M_\odot$ vs. $(3.0-18) \times 10^7 M_\odot$ in M.O. (L17)], and slightly overproduces stars [$3.3 \times 10^7 M_\odot$ vs. $1.9 \times 10^7 M_\odot$ in M.O. (C06)].  We find an electron number density in our cloud of $n_e \sim 0.2$ cm$^{-3}$, also somewhat lower than the range of 1-10 cm$^{-3}$ obtained for M.O. (C06). Figure \ref{fig:history} also tracks the ``star formation efficiency,'' $M_\star/M_\mathrm{HI}$, over time.  By this measure, our simulation achieves a peak star formation efficiency of 3.5\%, very close to the value of 4\% measured in M.O. (C06). 

%\clearpage
\begin{figure}
\includegraphics[width=0.5\columnwidth]{mass_history.pdf} 
\caption{Plots of $M_\mathrm{HI}$, $M_{\mathrm{H}_2}$, $M_\star$, and $M_\star/M_\mathrm{HI}$ over time for our highest resolution 3D simulation (384x128x128\_3level). At the final time of $t = 40$ Myr, all measures are within a factor of two of their observed values.
\label{fig:history}}
\end{figure}


\section{Discussion \& Conclusion} 

In this work, we used 3D multi-physics hydrodynamic simulations to model the evolution of a radio jet impacting a single, dwarf-galaxy-scale cloud in a direct, axially-symmetric collision. Our intention is to use these simulations to better understand observations of M.O., the peculiar starburst galaxy located at the termination point of the radio jet from NGC 541. 

Our first, main conclusion is that jet-induced star formation (i.e. positive feedback) is possible under this scenario. Figure \ref{fig:sfr} shows a dramatic increase in star formation attributable to the jet interaction, and the star-formation rate matches M.O. well.  Importantly, this star formation is occurring upstream of the bulk of the \ion{H}{1} (compare Figures \ref{fig:volume} and \ref{fig:formation}), also consistent with observations (C06). 

Other quantitative measures from the simulations also show substantial agreement with the observations. For example, the total masses of \ion{H}{1}, H$_2$, and stars (Figure \ref{fig:history}) are all within a factor of two of their M.O. values. Additionally, the velocity histogram of our star particles peaks at $| v | \approx 85$ km s$^{-1}$ (Figure \ref{fig:histogram}), which is roughly consistent with observations (L17). Given the relative simplicity of our setup (uniform jet hitting spherical cloud head on), it is remarkable how well our results match quantitatively across such a wide range of diagnostics.

The spatial distribution of the star particle velocities is interesting. The fastest moving star particles are found furthest downstream (Figure \ref{fig:stars}). These are also some of the first star particles to form (i.e. they are the oldest), which is consistent with a slowing of the propagation speed of the star formation front. We plan to reexamine the observations of M.O. to see if a similar distribution is present.

One effect that is not treated in these simulations is negative feedback from the star formation process itself. Once the first generation of massive stars form, there is only a limited amount of time (of the order a few Myr) for star formation to continue before heating from these stars would effectively shut it off \citep{Dong03}. This might explain why the S.F.R. in M.O. appears to be lower now than it was in the past -- negative feedback may already be kicking in.

As future observations continue to constrain the star formation history of M.O., we plan to continue to refine our simulations.  Future modifications to our setup may include: simulating non-axially-symmetric interactions between the jet and cloud; having the jet sweep across the cloud; or adding more feedback mechanisms, such as heating from young stars and supernovae. We could also include self gravity, which would only enhance and accelerate the star formation.


\acknowledgements
This work used the Extreme Science and Engineering Discovery Environment (XSEDE), which is supported by National Science Foundation grant number ACI-1053575.   PCF acknowledges support from National Science Foundation grants AST-1211230 and AST-1616185.  JWLW acknowledges support from NRAO Student Observing Support grant SOSPA3-020. Work by PA was performed in part under the auspices of the U.S. Department of Energy by Lawrence Livermore National Laboratory under Contract DE-AC52-07NA27344.

\software{Cosmos++ \citep{Anninos05}}

\begin{thebibliography}{}
\expandafter\ifx\csname natexlab\endcsname\relax\def\natexlab#1{#1}\fi

\bibitem[{{Anninos} {et~al.}(2003){Anninos}, {Fragile}, \&
  {Murray}}]{Anninos03}
{Anninos}, P., {Fragile}, P.~C., \& {Murray}, S.~D. 2003, \apjs, 147, 177

\bibitem[{{Anninos} {et~al.}(2005){Anninos}, {Fragile}, \&
  {Salmonson}}]{Anninos05}
{Anninos}, P., {Fragile}, P.~C., \& {Salmonson}, J.~D. 2005, \apj, 635, 723

\bibitem[{{Anninos} {et~al.}(2012){Anninos}, {Fragile}, {Wilson}, \&
  {Murray}}]{Anninos12}
{Anninos}, P., {Fragile}, P.~C., {Wilson}, J., \& {Murray}, S.~D. 2012, \apj,
  759, 132

\bibitem[{{Anninos} {et~al.}(1997){Anninos}, {Zhang}, {Abel}, \&
  {Norman}}]{Anninos97}
{Anninos}, P., {Zhang}, Y., {Abel}, T., \& {Norman}, M.~L. 1997, \na, 2, 209

\bibitem[{{Antonuccio-Delogu} \& {Silk}(2008)}]{Antonuccio08}
{Antonuccio-Delogu}, V., \& {Silk}, J. 2008, \mnras, 389, 1750

\bibitem[{{Bechtold} {et~al.}(1987){Bechtold}, {Weymann}, {Lin}, \&
  {Malkan}}]{Bechtold87}
{Bechtold}, J., {Weymann}, R.~J., {Lin}, Z., \& {Malkan}, M.~A. 1987, \apj,
  315, 180

\bibitem[{{Bicknell} {et~al.}(2000){Bicknell}, {Sutherland}, {van Breugel},
  {Dopita}, {Dey}, \& {Miley}}]{Bicknell00}
{Bicknell}, G.~V., {Sutherland}, R.~S., {van Breugel}, W.~J.~M., {et~al.} 2000,
  \apj, 540, 678

\bibitem[{{Binney} \& {Tremaine}(1987)}]{Binney87}
{Binney}, J., \& {Tremaine}, S. 1987, {Galactic dynamics}

\bibitem[{{Bogd{\'a}n} {et~al.}(2011){Bogd{\'a}n}, {Kraft}, {Forman}, {Jones},
  {Randall}, {Sun}, {O'Dea}, {Churazov}, \& {Baum}}]{Bogdan11}
{Bogd{\'a}n}, {\'A}., {Kraft}, R.~P., {Forman}, W.~R., {et~al.} 2011, \apj,
  743, 59

\bibitem[{{Camarda} {et~al.}(2009){Camarda}, {Anninos}, {Fragile}, \&
  {Font}}]{Camarda09}
{Camarda}, K.~D., {Anninos}, P., {Fragile}, P.~C., \& {Font}, J.~A. 2009, \apj,
  707, 1610

\bibitem[{{Croft} {et~al.}(2006){Croft}, {van Breugel}, {de Vries}, {Dopita},
  {Martin}, {Morganti}, {Neff}, {Oosterloo}, {Schiminovich}, {Stanford}, \&
  {van Gorkom}}]{Croft06}
{Croft}, S., {van Breugel}, W., {de Vries}, W., {et~al.} 2006, \apj, 647, 1040

\bibitem[{{Dalgarno} \& {McCray}(1972)}]{Dalgarno72}
{Dalgarno}, A., \& {McCray}, R.~A. 1972, \araa, 10, 375

\bibitem[{{Dong} {et~al.}(2003){Dong}, {Lin}, \& {Murray}}]{Dong03}
{Dong}, S., {Lin}, D.~N.~C., \& {Murray}, S.~D. 2003, \apj, 596, 930

\bibitem[{{Fanaroff} \& {Riley}(1974)}]{FR74}
{Fanaroff}, B.~L., \& {Riley}, J.~M. 1974, \mnras, 167, 31P

\bibitem[{{Fragile} {et~al.}(2005){Fragile}, {Anninos}, {Gustafson}, \&
  {Murray}}]{Fragile05}
{Fragile}, P.~C., {Anninos}, P., {Gustafson}, K., \& {Murray}, S.~D. 2005,
  \apj, 619, 327

\bibitem[{{Fragile} {et~al.}(2012){Fragile}, {Gillespie}, {Monahan},
  {Rodriguez}, \& {Anninos}}]{Fragile12}
{Fragile}, P.~C., {Gillespie}, A., {Monahan}, T., {Rodriguez}, M., \&
  {Anninos}, P. 2012, \apjs, 201, 9

\bibitem[{{Fragile} {et~al.}(2004){Fragile}, {Murray}, {Anninos}, \& {van
  Breugel}}]{Fragile04}
{Fragile}, P.~C., {Murray}, S.~D., {Anninos}, P., \& {van Breugel}, W. 2004,
  \apj, 604, 74

\bibitem[{{Gaibler} {et~al.}(2012){Gaibler}, {Khochfar}, {Krause}, \&
  {Silk}}]{Gaibler12}
{Gaibler}, V., {Khochfar}, S., {Krause}, M., \& {Silk}, J. 2012, \mnras, 425,
  438

\bibitem[{{Guillard} {et~al.}(2015){Guillard}, {Boulanger}, {Lehnert}, {Pineau
  des For{\^e}ts}, {Combes}, {Falgarone}, \& {Bernard-Salas}}]{Guillard15}
{Guillard}, P., {Boulanger}, F., {Lehnert}, M.~D., {et~al.} 2015, \aap, 574,
  A32

\bibitem[{{Hollenbach} \& {McKee}(1979)}]{Hollenbach79}
{Hollenbach}, D., \& {McKee}, C.~F. 1979, \apjs, 41, 555

\bibitem[{{Hollenbach} {et~al.}(1971){Hollenbach}, {Werner}, \&
  {Salpeter}}]{Hollenbach71}
{Hollenbach}, D.~J., {Werner}, M.~W., \& {Salpeter}, E.~E. 1971, \apj, 163, 165

\bibitem[{{Kennicutt}(1998)}]{Kennicutt98}
{Kennicutt}, Jr., R.~C. 1998, \apj, 498, 541

\bibitem[{{K{\"o}rding} {et~al.}(2008){K{\"o}rding}, {Jester}, \&
  {Fender}}]{Kording08}
{K{\"o}rding}, E.~G., {Jester}, S., \& {Fender}, R. 2008, \mnras, 383, 277

\bibitem[{{Lacy} {et~al.}(2017){Lacy}, {Croft}, {Fragile}, {Wood}, \&
  {Nyland}}]{Lacy17}
{Lacy}, M., {Croft}, S., {Fragile}, C., {Wood}, S., \& {Nyland}, K. 2017, \apj,
  838, 146

\bibitem[{{Laing} \& {Bridle}(2002)}]{Laing02}
{Laing}, R.~A., \& {Bridle}, A.~H. 2002, \mnras, 336, 1161

\bibitem[{{Nesvadba} {et~al.}(2010){Nesvadba}, {Boulanger}, {Salom{\'e}},
  {Guillard}, {Lehnert}, {Ogle}, {Appleton}, {Falgarone}, \& {Pineau Des
  Forets}}]{Nesvadba10}
{Nesvadba}, N.~P.~H., {Boulanger}, F., {Salom{\'e}}, P., {et~al.} 2010, \aap,
  521, A65

\bibitem[{{Ogle} {et~al.}(2007){Ogle}, {Antonucci}, {Appleton}, \&
  {Whysong}}]{Ogle07}
{Ogle}, P., {Antonucci}, R., {Appleton}, P.~N., \& {Whysong}, D. 2007, \apj,
  668, 699

\bibitem[{{Omukai}(2000)}]{Omukai00}
{Omukai}, K. 2000, \apj, 534, 809

\bibitem[{{Osterbrock}(1989)}]{Osterbrock89}
{Osterbrock}, D.~E. 1989, {Astrophysics of gaseous nebulae and active galactic
  nuclei}

\bibitem[{{Persic} {et~al.}(1996){Persic}, {Salucci}, \& {Stel}}]{Persic96}
{Persic}, M., {Salucci}, P., \& {Stel}, F. 1996, \mnras, 281, 27

\bibitem[{{Rasera} \& {Teyssier}(2006)}]{Rasera06}
{Rasera}, Y., \& {Teyssier}, R. 2006, \aap, 445, 1

\bibitem[{{Salom{\'e}} {et~al.}(2015){Salom{\'e}}, {Salom{\'e}}, \&
  {Combes}}]{Salome15}
{Salom{\'e}}, Q., {Salom{\'e}}, P., \& {Combes}, F. 2015, \aap, 574, A34

\bibitem[{{Salom{\'e}} {et~al.}(2016){Salom{\'e}}, {Salom{\'e}}, {Combes}, \&
  {Hamer}}]{Salome16}
{Salom{\'e}}, Q., {Salom{\'e}}, P., {Combes}, F., \& {Hamer}, S. 2016, \aap,
  595, A65

\bibitem[{{Smith} {et~al.}(2016){Smith}, {Bryan}, {Glover}, {Goldbaum}, {Turk},
  {Regan}, {Wise}, {Schive}, {Abel}, {Emerick}, {O'Shea}, {Anninos}, {Hummels},
  \& {Khochfar}}]{Smith16}
{Smith}, B.~D., {Bryan}, G.~L., {Glover}, S.~C.~O., {et~al.} 2016, ArXiv
  e-prints, arXiv:1610.09591

\bibitem[{{Spaans} \& {Neufeld}(1997)}]{Spaans97}
{Spaans}, M., \& {Neufeld}, D.~A. 1997, \apj, 484, 785

\bibitem[{{Sutherland} \& {Bicknell}(2007)}]{Sutherland07}
{Sutherland}, R.~S., \& {Bicknell}, G.~V. 2007, \apjs, 173, 37

\bibitem[{{van Breugel} {et~al.}(1985){van Breugel}, {Filippenko}, {Heckman},
  \& {Miley}}]{vanBreugel85}
{van Breugel}, W., {Filippenko}, A.~V., {Heckman}, T., \& {Miley}, G. 1985,
  \apj, 293, 83

\end{thebibliography}

\end{document}


\begin{comment}
\begin{figure}
\includegraphics[width=\linewidth]{figs/beyond_tss_lesion.pdf}
\caption[]{End-to-End runtime lesion study of the entire MNIST dataset and the FMA featurized music dataset. Each of DROP's contributions provides a runtime improvement.}
\label{fig:beyond_lesion}
\end{figure}
\end{comment}



\section{Conclusion}
\label{sec:conclusion}

Advanced data analytics techniques must scale to rising data volumes. 
DR techniques offer a powerful toolkit when processing these datasets, with PCA frequently outperforming popular techniques in exchange for high computational cost. 
In response, we propose DROP, a new dimensionality reduction optimizer. 
DROP combines progressive sampling, progress estimation, and online aggregation to identify high quality low dimensional bases via PCA without processing the entire dataset by balancing the runtime of downstream tasks and achieved dimensionality. 
Thus, DROP provides a first step in bridging the gap between quality and efficiency in end-to-end DR for downstream \red{analytics}. 

%We revisit canonical operators for time series dimensionality reduction and the measurement study of~\cite{keogh-study}, and show that PCA is more effective than popular alternatives in the data mining literature often by a margin of over $2\times$ on average on gold-standard time series benchmark data sets with respect to output data dimension. More surprisingly, we empirically demonstrate that a small number of samples are sufficient to accurately characterize directions of maximum variance and obtain a high-quality low-dimensional transformation.





\bibliographystyle{plain}
\bibliography{refs}

% 
	\section{Proof of Proposition~\ref{prop-quasi-order}}
\begin{proof}
	We will prove the result for relation $\sqsubseteq$, the proof for $\preceq$ being similar. We need to prove that $\sqsubseteq$ is reflexive and transitive. For reflexivity, it is obvious that since $G\subseteq G$ for any goal $G$, we have $G\sqsubseteq_\theta G$ for the empty substitution $\theta$. For transitivity, suppose that for goals $G_1$, $G_2$ and $G_3$, it holds that $G_1 \sqsubseteq_{\theta_1} G_2$ and $G_2 \sqsubseteq_{\theta_2} G_3$. Then by Definition~\ref{def-generalization}, there exist sets of atoms $\Delta_1$ and $\Delta_2$ such that $G_1\theta_1 \cup \Delta_1 = G_2$ and $G_2\theta_2\cup\Delta_2 = G_3$. In other words it holds that $(G_1\theta_1\cup\Delta_1)\theta_2\cup\Delta_2 = G_3$ or equivalently, $(G_1\theta_1)\theta_2 \cup (\Delta_1\theta_2 \cup\Delta_2) = G_3$. As the composition of two substitutions is a substitution, by defining $\theta_3 = \theta_2\circ\theta_1$ and $\Delta_3 = \Delta_1\theta_2 \cup\Delta_2$, we have $G_1\theta_3 \cup \Delta_3 = G_3$, so $G_1\sqsubseteq_{\theta_3} G_3$, which concludes the proof.
\end{proof}
	

	\section{Proof of Proposition~\ref{prop-msg-lcg}}
	
		First, observe the following property that holds for both relations, essentially stating that a common generalization that is not a lcg has a direct extension obtained by the addition of one atom. 
		
		\begin{proposition}\label{prop-lcg-extensible}
			Let $G_1, \dots, G_n$ and $G$ be goals such that $G$ is a $\leqslant$-common generalization, but not a $\leqslant$-lcg, of $\{G_1, \dots, G_n\}$. Then there exists an atom $A\notin G$ such that $G\cup\{A\}$ is a $\leqslant$-common generalization of $\{G_1, \dots, G_n\}$.
		\end{proposition} 
	
	
		\begin{proof}
		
		Let us suppose the existence of some goal $G$, a $\leqslant$-common generalization that is not a $\leqslant$-lcg of $\{G_1, \dots, G_n\}$, and let us try and extend $G$ into a $\leqslant$-common generalization $G\cup\{A\}$ with $A\notin G$ an atom. As $G$ is not a lcg, there must exist another goal $G'$ being a $\leqslant$-lcg of $G_1$ and $G_2$ and obviously we have $|G'|>|G|$. As a consequence at this point there are three groups of atoms that can be identified: let us denote by $\hat{A}_1, \dots, \hat{A}_p$ the $p (\ge 0)$ atom(s) that are both in $G$ and in $G'$; by $A_1, \dots, A_m$ the $m (\ge 0)$ atom(s) that are part of $G$ but not of $G'$; and by $B_1, \dots, B_l$ the $l (\ge 1)$ atom(s) that are part of $G'$ but not of $G$. For an element $A$ of any of these sets, we denote by $A^1, \dots, A_n$ the atom in respectively $G_1, \dots, G_n$ whose anti-unification led to having $A$ as part of the generalizations.
		
		From the fact that $|G'|>|G|$ it follows that $l>m$. Now each $A_i (i \in 1..m)$ is such that $\exists  h\in 1..n : A_i^h\in \{B_i^h|i\in 1..l\}$: if not, it would be possible to add an atom generalizing $\{A_i^1, \dots, A_i^n\}$ (such as $A_i$) in $G'$ and get a larger generalization, which is impossible given that $G'$ is a lcg. Also note that for two atoms $B_i$ and $B_j (1\le i < j \le l)$, for $g, h \in 1..n : g\neq h$, if $B_i^g$ is anti-unifiable with an atom $B_j^h$ then $B_j^g$ is also anti-unifiable with $B_i^h$ (as it means that all four base atoms are a call to one and the same predicate (with relation $\sqsubseteq$) or have the exact same inner structure save for variables (with relation $\preceq$)), so that it is possible to switch the atoms $B_i^g$ and $B_j^h$, compute the anti-unification of $\{B_i^g, B_j^h)$ and $(B_j^g, B_i^h)$, and get an equally valid anti-unification. Thanks to this we can, where necessary, perform switches so as to rearrange the atoms $B_i (1\le i \le l)$ into $\{\hat{B}_i|i \in 1..l\}$ in such a way that $\{A_i|i\in 1..m\} \subset \{\hat{B}_i|i \in 1..l\}$ and for each atom $\hat{B}_i$, either $\hat{B}_i \in \{A_k|k\in 1..m\}$ or $\exists g, h \in 1..n : g\neq h \wedge \hat{B}_i^g\notin \{A_k^g|k\in 1..m\}\wedge \hat{B}_i^h\notin \{A_k^h|k\in 1..m\}$. We can now define a new generalization $\hat{G}$ defined as the union of these rearranged atoms and those that are common to $G$ and $G'$, i.e. $\hat{G} = \{\hat{A_i}|i\in 1..p\}\cup\{\hat{B}_i|i\in 1..l\}$. Since $|\hat{G}| = |G'|$ and $G\subset \hat{G}$, it suffices to add one of the atoms $A \in \hat{G}\setminus G$ to $G$ in order to obtain $G\cup\{A\}$, a $\leqslant$-common generalization of $\{G_1, \dots, G_n\}$ by construction.
	\end{proof}


		Next, we prove Proposition~\ref{prop-msg-lcg}.
		
	\begin{proof}
	We prove that any $\leqslant$-msg is a $\leqslant$-lcg by contradiction. Let us suppose that some goal $G$ is both a $\leqslant$-msg and not a $\leqslant$-lcg of the set of $\{G_1, \dots, G_n\}$. According to Proposition~\ref{prop-lcg-extensible} it must then be possible to select an atom $A \notin G$ such that $G\cup\{A\}$ is a $\leqslant$-common generalization of $\{G_1, \dots, G_n\}$. Since $A\notin G$ and any atom has a $\tau$-value of at least 1, it follows that $|\tau(G\cup\{A\})|>|\tau(G)|$. Consequently $G$ cannot be a $\leqslant$-most specific generalization of $G_1$ and $G_2$: a contradiction.
	
	As for the fact that any $\preceq$-lcg is a $\preceq$-msg, we prove this also by contradiction. Let $G$ represent a $\preceq$-lcg of the set of goals $\{G_1, \dots, G_n\}$ and let us suppose that $G$ is not a $\preceq$-msg. Then there must exist another goal that is a $\preceq$-msg of $\{G_1, \dots, G_n\}$, say $G'$, such that $|\tau(G')|>|\tau(G)|$ and, according to the first part of the proposition, $|G'|=|G|$. 
	Now, observe that for a set of atoms $\{A_1, \dots, A_n\}$ to be anti-unified with $\preceq$ into an atom $A$, necessarily all $A_i (1\le i\le n)$ must have the same $\terms$-value. Indeed relation $\preceq$ is defined upon renamings so that only variables (having a $\terms$-value of zero) are impacted by the generalization process. Therefore, the only possibility for the inequality $|\tau(G')|>|\tau(G)|$ to be true is that some atoms $B_1, \dots, B_n$ of respective goals $G_1,\dots,G_n$ appear in a generalized form (say $B$) in $G'$, while these atoms have not been generalized in $G$. This means that it is possible to add a (possibly renamed) version of $B$ in $G$ and obtain $G\cup\{B\}$, also a $\preceq$-common generalization and larger than $G$: a contradiction.
	\end{proof}

	
	\section{Detailed proof of Lemma~\ref{lemma-au-op}}
	\begin{proof}
The lemma will be shown correct by the definition of three anti-unification operators. A first anti-unification operator, based on $\sqsubseteq$ is the following. 

\begin{definition}%[Simple anti-unification operator]
	\label{def-atoms-au}
	Given a variabilization function $\Phi$, let $\au^\Phi_\sqsubseteq$ (or simply $\au_\sqsubseteq$ if $\Phi$ is clear from the context) denote the anti-unification operator such that for any two atoms $A = a(t^A_1, \dots, t^A_n)$ and $B = b(t^B_1, \dots, t^B_m)$, it holds that \[\au^{\Phi}_\sqsubseteq(A,B)=\left\{\begin{array}{l}
		a\big(\Phi(t^A_1, t^B_1), \dots, \Phi(t^A_n, t^B_n)\big) \\ \qquad  \mbox{if } a = b \mbox{ and } n = m\\
		\bot \\
		\qquad \mbox{otherwise}\\
	\end{array}\right. \]
\end{definition} 

\begin{example}\label{ex-au-sq}
	In Table~\ref{table:sqsubseteq}, we show three atomic anti-unification results obtained by the application of $\au_\sqsubseteq^\Phi$ with $\Phi$ a given variabilization function. Note how in the first example, the predicates used in $A_1$ and $A_2$ differ (resp. $p/2$ and $p/3$), leading to an impossible anti-unification.
\end{example}

\begin{table*}
	\caption{Example results for $au_\sqsubseteq^\Phi$}
	\label{table:sqsubseteq}
	\centering
	\begin{tabular}{l|l|l}
		%\hline 
		$\bm{A_1}$ & $\bm{A_2}$ & $\bm{\au_\sqsubseteq^\Phi(A_1, A_2)}$\\\hline 
		$p(X, 5, q(Y,4))$ & $p(W,t(Z))$ & $\bot$\\\hline 
		$p(r(X,3), t(5))$ & $p(W, t(Z))$ & $p(\Phi(r(X,3),W), \Phi(t(5), t(Z)))$\\\hline 
		$p(r(X,3), t(Y))$ & $p(r(W,3),t(Z))$ & $p(\Phi(r(X,3),r(W,3)), \Phi(t(Y),t(Z)))$ %\\\hline 
	\end{tabular} 
\end{table*}

Note that the anti-unification operator defined in Definition~\ref{def-atoms-au} differs from the traditional subsumption operator in the ordered case (i.e. when goals are ordered sequences of atoms). The difference comes from the fact that our goals being sets, all the possible couples of atoms have to be considered, whereas traditional subsumption must handle one atom at the time, making the anti-unification operator more straigtforward.     

Let us now introduce a second anti-unification operator that will allow to compute a $\preceq$-lcg. 
%and to prove Theorem~\ref{thm-preceq-lcg}. 
Since the result of this operator should be a $\preceq$-common generalization, the operator need only to anti-unify the \textit{variables} occurring at the corresponding positions in the atoms under investigation. 
The operator must thus go deeper into the term structure of the atoms than $\au_\sqsubseteq$ does, as it needs to only anti-unify those atoms that harbor the exact same structure at the level of their non-variable terms.
\begin{definition}%[Variable anti-unification operator]
	\label{def-term-au-through-variables}
	Given some variabilization function $\Phi$, let $\au^\Phi_\preceq$ (or simply $\au_\preceq$ if $\Phi$ is clear from the context) denote the function such that for any two terms $T = t(t_1, \dots, t_n)$ and $U = u(u_1, \dots, u_m)$ it holds that
	\[\au^\Phi_\preceq(T,U)=\left\{\begin{array}{l}
		\Phi(T,U) 
		\\ \qquad \mbox{if } T\in\mathcal{V}\mbox{ and } U\in\mathcal{V}
		\\t\big(\au^\Phi_\preceq(t_1,u_1), \dots, \au^\Phi_\preceq(t_n, u_n)\big) 
		\\ \qquad \mbox{if } t = u \mbox{ and } n = m 
		\\ \qquad \mbox{and } \forall i \in 1..n: \au^\Phi_\preceq(t_i,u_i)\neq\bot
		\\ \bot
		\\ \qquad  \mbox{otherwise}
	\end{array}\right.\]
	and for any two atoms $A = a(t^A_1, \dots, t^A_n)$ and $B = b(u^B_1, \dots, u^B_m)$, it holds that
	\[\au^\Phi_\preceq(A,B)=\left\{\begin{array}{l}
		a\big(\au^\Phi_\preceq(t^A_1, u^B_1),\dots, \au^\Phi_\preceq(t^A_n, u^B_n)\big) 
		\\ \qquad \mbox{if } a = b \mbox{ and } n = m 
		\\ \qquad \mbox{and } \forall i \in 1..n: \au^\Phi_\preceq(t^A_i, u^B_i) \neq\bot
		\\ \bot  
		\\ \qquad \mbox{otherwise}
	\end{array}\right.\]
\end{definition}

\begin{example}
	In Table~\ref{table:preceq}, we treat the anti-unification of the same atoms as above, this time with the use of $\au_\preceq^\Phi$ with $\Phi$ a given variabilization function. Note how $\au_\preceq$ behaves differently than $\au_\sqsubseteq$ on the second and third couple of atoms as it requires its arguments to exhibit a similar structure in order to be anti-unifiable.
\end{example}

\begin{table*}
	\caption{Example results for $au_\preceq^\Phi$}
	\label{table:preceq}
	\centering
	\begin{tabular}{l|l|l}
		%\hline
		$\bm{A_1}$ & $\bm{A_2}$ & $\bm{\au_\preceq^\Phi(A_1, A_2)}$\\\hline 
		$p(X, 5, q(Y,4))$ & $p(W,t(Z))$ & $\bot$\\\hline 
		$p(r(X,3), t(5))$ & $p(W, t(Z))$ & $\bot$\\\hline 
		$p(r(X,3), t(Y))$ & $p(r(W,3),t(Z))$ & $p(r(\Phi(X,W),3), t(\Phi(Y,Z)))$ %\\\hline
	\end{tabular}
\end{table*}

Now, in order to compute $\sqsubseteq$-msgs, we need a more precise anti-unification operator: one that goes deeper into detail when comparing atoms so as not to miss their maximal common structure. 
\begin{definition} %[Deep anti-unification operator]\label{def-deep-operator}
	Given some variabilization function $\Phi$, let $\dau^\Phi_\sqsubseteq$ (or simply $\dau_\sqsubseteq$ if $\Phi$ is clear from the context) denote the function such that for any two terms $T = t(t_1, \dots, t_n)$ and $U = u(u_1, \dots, u_m)$ it holds that 
	\[\dau^\Phi_\sqsubseteq(T,U)=\left\{\begin{array}{l}
		
		t\big(\dau^\Phi_\sqsubseteq(t_1,u_1), \dots, \dau^\Phi_\sqsubseteq(t_n, u_n)\big) 
		\\ \qquad \mbox{if } t = u \mbox{ and } n = m 
		\\ \qquad \mbox{and } T \notin \mathcal{V} \mbox{ and } U \notin \mathcal{V}
		\\ \Phi(T,U) 
		\\ \qquad \mbox{otherwise}
	\end{array}\right.\]
	and for any two atoms $A = a(t^A_1, \dots, t^A_n)$ and $B = b(u^B_1, \dots, u^B_m)$, it holds that
	\[\dau^\Phi_\sqsubseteq(A,B)=\left\{\begin{array}{l}
		a\big(\dau^\Phi_\sqsubseteq(t^A_1, u^B_1),\dots, \dau^\Phi_\sqsubseteq(t^A_n, u^B_n)\big) 
		\\ \qquad \mbox{if } a = b \mbox{ and } n = m 
		\\ \bot
		\\ \qquad \mbox{otherwise}
	\end{array}\right.\]
\end{definition}

When applied on atoms, it is easy to see that $\dau_\sqsubseteq$ is an anti-unification operator based on relation $\sqsubseteq$.

\begin{example}
	Let us once more consider the anti-unification of the atoms introduced in Example~\ref{ex-au-sq}. This time we make use of $\dau_\sqsubseteq^\Phi$ with $\Phi$ a given variabilization function, to anti-unify the three pairs of atoms. The result is shown in Table~\ref{table:dau}. Notice how the operator preserves as much non-variable atomic structure as possible in the process.
\end{example}
\begin{table*}
	\caption{Example results for $dau_\sqsubseteq^\Phi$}
	\label{table:dau}
	\centering
	\begin{tabular}{l|l|l}
		%\hline 
		$\bm{A_1}$ & $\bm{A_2}$ & $\bm{\dau_\sqsubseteq^\Phi(A_1, A_2)}$\\\hline 
		$p(X, 5, q(Y,4))$ & $p(W,t(Z))$ & $\bot$\\\hline 
		$p(r(X,3), t(5))$ & $p(W, t(Z))$ & $p(\Phi(r(X,3),W),t(\Phi(5,Z)))$\\\hline 
		$p(r(X,3), t(Y))$ & $p(r(W,3),t(Z))$ & $p(r(\Phi(X,W),3), t(\Phi(Y,Z)))$ %\\\hline 
	\end{tabular} 
\end{table*}
The existence of these operators proves Lemma~\ref{lemma-au-op}.
	\end{proof}
	
	\section{Proof of Theorem~\ref{thm-ausqsubseteq}}
	\begin{proof}
	Obviously the $\au_\sqsubseteq(A_1,A_2)$ operation can be achieved in a time linear with respect to the arity $n$ of $A_1$. In the worst case, the operation needs to be performed for each atom in $G_1$ with respect to each atom in $G_2$. Hence the first result.
	
	
	It is also easy to see that the $\au_\preceq(A_1,A_2)$ operation can be achieved in linear time with respect to the maximum number of function applications in the argument terms of the atom $A_1$ under scrutiny. In the worst case, the operation needs to be performed for each atom in $G_1$ with respect to each atom in $G_2$. Hence the second result.
	\end{proof}
		
%	
%	\section{Maximum Weight Matching of Example~\ref{example-mwm}}
%	See Fig.~\ref{fig:mwm}.
	
	
	\section{Proof of Theorem~\ref{thm-sqsubseteq-msg}}
\begin{proof}
	First note how the atomic anti-unifications and the weights of the associated bipartite graph's edges can be computed simultaneously, by working out $\dau_\sqsubseteq(A_1,A_2)$ for each possible couple $(A_1,A_2)$ in $G_1\times G_2$ and keeping account of the number of non-variable terms encountered during the operation (or $-1$). Given that $\dau_\sqsubseteq(A_1,A_2)$ can obviously operate linearly in the number of terms appearing in $A_1$ (denoted $N$), the computation of all weights is carried out in a time not exceeding $\mathcal{O}(|G_1|.|G_2|.N)$.
	
	Now the obtained assignment problem can be solved by existing algorithms (such as the Hungarian method~\cite{assignment}) that compute a MWM in $\mathcal{O}(n^3)$, where $n$ is the number of vertexes appearing on the side of the bipartite graph that has the most vertexes. In our case, there are $|G_1|$ left vertexes and $|G_2|$ right vertexes so that a MWM algorithm can be ran in $\mathcal{O}(max(|G_1|,|G_2|)^3)$.
\end{proof}
	
	\section{Proof of Theorem~\ref{thm-dataflow-np-complete}}
\begin{proof}
	First, let us consider MSG-MIN. It clearly belongs to NP. Indeed, given an arbitrary generalization $G$, we can verify in polynomial time whether it is a most specific generalization. The procedure is as follows. We can compute at least one $\leqslant$-msg, say $G'$, in polynomial time (see Theorem~\ref{thm-sqsubseteq-msg}). It suffices then to compare the $\tau$-value of $G'$ with that of $G$ in order to decide whether $G$ is a msg. Next, verifying whether the number of variables in $G$ is bounded by a constant is obviously achieved in polynomial time as well.
	
	In order to prove NP-hardness, we will construct a reduction from the well-known set cover problem (known to be NP-complete~\cite{karp}) to MSG-MIN. The set cover problem in its decision-problem version (denoted SCP), can be formulated as follows. Given a constant $p \in \mathbb{N}_0$, a universe $U$ of values and a collection $S$ composed of $n$ sets $\{S_1, \dots, S_n\}$ that cover $U$, i.e. $U = \underset{i=1}{\overset{n}{\cup}}S_i$, the problem is to decide whether there exists $p$ subsets from $S$ that still cover $U$.
	
	We can transform an arbitrary instance of SCP into MSG-MIN as follows. Let us consider without loss of generality a universe $U$ where the elements are lowercase strings and $p \in \mathbb{N}_0$ a constant. Given a collection of sets $S=\{S_1, \dots, S_n\}$ we construct an instance of MSG-MIN as follows. In our construction we use $n+1$ different variables, namely $V$ and $(W_i)_{i\in1..n}$. We use $x_j$ to denote some element of $U$; these elements being strings, we can easily use them as predicate names. The construction of goals $G_1$ and $G_2$ proceeds then as follows:
	
	\begin{algorithmic}
		\State $G_1 = \{\}$ 
		\State $G_2 = \{\}$ 
		\For {each ($S_i \in S$)}
		\For {each ($x_j \in S_i$)}
		\State $G_1 \gets G_1\cup \{x_j(V)\}$				
		\State $G_2 \gets G_2\cup \{x_j(W_i)\}$
		\EndFor
		\EndFor
	\end{algorithmic}
	Note that all the atoms in $G_1$ have the same argument (namely the variable $V$) and there are as many atoms in $G_1$ as there are distinct elements in $S$. In $G_2$, however, there is an atom of the form $x_j(W_i)$ for each element $x_j$ occurring in $S_i$.
	
	The construction is such that any $\leqslant$-msg of $G_1$ and $G_2$ will be a version of $G_1$ where each occurrence of a variable $V$ is replaced by $\Phi(V, W_k)$ for some $W_k\in\vars(G_2)$ (where $\Phi$ is a variabilization function). Now, introducing such a variable $\Phi(V, W_k)$ in the generalization will allow to reuse the same variable for all the atoms $x_j(V)$ in $G_1$ that have a corresponding $x_j(W_k)$ in $G_2$. In other words, choosing to have variable $\Phi(V,W_k)$ in the $\leqslant$-msg is the same as selecting the subset $S_k$ to be part of the solution of the set cover problem. Consequently, using this transformation MSG-MIN can be used to decide SCP. Since the transformation can clearly be done in polynomial time, and since SCP is known to be NP-complete, we conclude that MSG-MIN is NP-complete as well.
	
	Now let us prove the result for LCG-MIN. We know that a $\leqslant$-lcg can be computed in polynomial time, so that a positive instance of LCG-MIN can be verified just like it can be for MSG-MIN. Moreover, the absence of non-variable terms in the transformation from SCP to MSG-MIN above allows us to reuse said transformation as-is to prove that LCG-MIN is NP-hard. Indeed, since the obtained anti-unification problem doesn't harbor terms other than variables, it is both an instance of MSG-MIN and LCG-MIN. LCG-MIN is therefore also NP-complete.
\end{proof}
	
	\section{Proof of Theorem~\ref{thm-inj-np-complete}}
\begin{proof}
	INJ is in NP: given a relation $\leqslant^\iota$, goals $G_1$ and $G_2$ and a substitution (or renaming) $\theta$, it is possible to verify in polynomial time whether the application of $\theta$ on $G_1$ results on a subset of $G_2$ or not.
	As for the proof of NP-hardness, we refer to~\cite{gen} in which the problem ``is $G_1$ a $\preceq^\iota$-lcg of $G_1$ and $G_2$?'' has been proved to be NP-complete using a polynomial reduction from the Induced Subgraph Isomorphism Problem~\cite{SYSLO198291}. The same reduction can be used for the other cases, leading to the conclusion that INJ is NP-complete.
\end{proof}
	
\end{document}


# COMMENTS:

v - Add boxes around figure (instead of only bottom and left borders...
- 7 should come earlier
- Explain each figure in detail in caption
v - Axes labels: "Filtering Step", "Network OSPA", "Time" (capitalized)
  "# of targets" -> "No. of Targets"
  Add space between 'x' and '(m)', between 'time' and '(s)', etc.
v - '''^ Same for legends

v - Remove titles, y-labels: "OSPA of Sensor 3 (m)"
v - proposed -> "Proposed Method"
v - Figure 2. has something cut off from top (anyway, remove title)


New comments:
v - The colors of the 'trajectories' figure should be all same, and consistent with the legend of the plot.
- check the highlighted comments


New Comments from July 30th:
- Double check whether survival property uses 'min' (in the code)

Some final questions for jessica:
- is probability of detection 0 outside the surveillance region?
- Are there an average of 5 clutter measurements AT EACH SENSOR? Based on theory, I think it should be 5 (on average) at each sensor but just want to double-check.
