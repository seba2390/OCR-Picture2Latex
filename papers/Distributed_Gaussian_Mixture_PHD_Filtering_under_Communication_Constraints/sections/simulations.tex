\section{Simulations}
\label{sec:simulation}
In this section, the performance of the proposed distributed GM-PHD filtering method is evaluated against three different algorithms from the literature:
\begin{itemize}
\item GM-PHD filter without consensus \cite{vo2006gmphd}, 
\item GM-PHD filter with full consensus (in which all the GM components are communicated between neighbors) \cite{battistelli2020differentFOVs}, and 
\item GM-PHD filter with partial consensus (which uses the rank rule to limit the number of GM components that are communicated) \cite{li2018partial}
\end{itemize}

No inter-sensor communication is carried out in the GM-PHD filter without consensus. In the other three algorithms, each sensor implements a total of $\alpha \geq 1$ average consensus steps between successive local PHD filtering steps (Algorithm \ref{alg:con}). The partial consensus algorithm uses the rank rule (described in Section \ref{sec:comm_constr}) to limit the number of GM components which are broadcast by each sensor to be $B$, where we set $B=5$.
Similarly, the proposed algorithm uses the sampling with replacement rule (Algorithm \ref{alg:sampling}) to limit the number of distinct GM components that are transmitted by a given sensor to $B$. Consequently, the communication bandwidth requirement of the proposed algorithm is at most that of the GM-PHD filter with partial consensus. The communication costs of the different algorithms can be ordered as follows:
\begin{equation*}
\text{Proposed Method} \leq \text{Partial Consensus} \ll \text{Full Consensus}
\end{equation*}
\noindent
While the full consensus algorithm may be unsuitable for practical applications due to its high communication cost, it serves as a benchmark for evaluating the multi-target tracking capability of the proposed distributed GM-PHD algorithm.

\subsection{Simulation Scenario}
% network (2-D surveillance area, network)
We use a multi-target tracking simulation scenario to evaluate the distributed GM-PHD algorithms. The simulation scenario described herein is similar to the one considered in \cite{vo2006gmphd}, with the key difference being that we consider a greater (time-varying) number of targets. Moreover, Ref. \cite{vo2006gmphd} considers the case of single-sensor multi-target tracking, whereas we consider the multi-sensor multi-target tracking case.

Consider a multi-target set having a 4-dimensional state space, such that the state of each target is composed of its $2$ dimensional position and velocity vectors. Thus, the $l^{th}$ target's state is \[x_k^{(l)}=\begin{bmatrix}
\textrm{pos}_{k}^\intercal & \textrm{vel}_{k}^\intercal\end{bmatrix}^{\intercal}\] where $\textrm{pos}_{k} \in \mathbb R^2$ and $\textrm{vel}_{k}\in \mathbb R^2$ denote the target's position and velocity vectors at timestep $k$, respectively.
%
% \begin{equation}
%     \begin{aligned}
%     p_{k}^S(x) = min \Big\lbrace &P\left( 0 \leq \textrm{pos}_{x,k} < 400 \right), P\left( 0 \leq \textrm{pos}_{y,k} < 400 \right) \Big\rbrace,
%     \end{aligned}
% \end{equation}
% where 
% \begin{equation}
% \begin{aligned}
%     &P \left[ X^{(lower)} \leq \textrm{pos}_{x,k} < X^{(upper)} \right] = F_{X^{(upper)}}(\textrm{pos}_{x,k}) - F_{X^{(lower)}}(\textrm{pos}_{x,k})\\
%     &P \left[ Y^{(lower)} \leq \textrm{pos}_{y,k} < Y^{(upper)} \right] = F_{Y^{(upper)}}(\textrm{pos}_{y,k}) - F_{Y^{(lower)}}(\textrm{pos}_{y,k}). 
% \end{aligned}
% \end{equation}
% Suppose $X$ and $Y$ are normally distributed random variables with mean $(\mu_x,\mu_y)$ and standard deviation $(\sigma_x,\sigma_y)$, respectively. The cumulative distribution functions (CDFs) can be calculated by
% \begin{equation}
% \begin{aligned}
%     F_{X^{(upper)}}(\textrm{pos}_{x,k}) = cdf_{X^{(upper)}}(\textrm{pos}_{x,k}; 400, 2)\\
%     F_{X^{(lower)}}(\textrm{pos}_{x,k}) = cdf_{X^{(lower)}}(\textrm{pos}_{x,k}; 0, 2)\\
%     F_{Y^{(upper)}}(\textrm{pos}_{y,k}) = cdf_{Y^{(upper)}}(\textrm{pos}_{y,k}; 400, 2)\\
%     F_{Y^{(lower)}}(\textrm{pos}_{y,k}) = cdf_{Y^{(lower)}}(\textrm{pos}_{y,k}; 0, 2)
% \end{aligned}
% \end{equation}
% }
%\cite{sun2023gaussian}
The targets follow the linear Gauss-Markov dynamical model (\ref{eq:dynamic_model}), with
\begin{equation}
    F_{k} = \begin{bmatrix}
        I & h I \\
        0 & I
    \end{bmatrix}
\ \ \text{and}\ \ 
    Q_{k} = 9 \begin{bmatrix}
        \frac{h^4}{4} I & \frac{h^3}{2} I \vspace{2pt}\\
         \frac{h^3}{2} I & h^2 I 
    \end{bmatrix} m^2s^{-4}, 
\end{equation}
where $h=1s$ is the sampling time. The target birth intensity $\gamma_k$ is chosen as
\begin{equation}
    \gamma _k =  0.2 \sum_{l=1}^{10}\mathcal N\left(m_{\gamma}^{(l)}, \textrm{diag}\big(\begin{bmatrix} 100 & 100 & 25 & 25\end{bmatrix}^\intercal\big)\right)
\end{equation}
where given a vector $v$, $\textrm{diag}(v)$ denotes the diagonal matrix whose diagonal elements are the components of $v$.
% and \hlt{$m_{\gamma,k}$} $= [50, 300; 350, 250; 70, 70; 210, 230; 300, 375; 300, 100; 200, 0; 100, 400; 0, 340; 0, 150]$. 
The mean vectors $m_{\gamma, k}^{(l)}$ represent the initial guesses of the position and velocity vectors of the targets. The mean vectors of the initial positions are as depicted in Figure \ref{fig:trajectories} by the `$\triangle$' symbols, whereas the mean vectors of the initial velocities are set to $[0ms^{-1}\ 0ms^{-1}]^\intercal$.
Given that a target has the state $\zeta$ at a given timestep, the intensity corresponding to its spawned targets is given by
\begin{equation}
    \sigma_{k|k-1}(\cdot \vert \zeta) =  0.1\hspace{3pt}\mathcal N\Big(\zeta, \text{diag}\big(\begin{bmatrix} 100 & 100 & 400 & 400 \end{bmatrix}^\intercal\big)\Big)
\end{equation}
The survival probability $p_{k}^S(x)$ of each target is chosen according to the model described in \cite{sun2023gaussian}, such that $p_{k}^S(x)$ is close to $1$ when the target is inside the surveillance region, and close to $0$ otherwise. 
% $[0\ 150]^\intercal$, $[0\ 340]^\intercal$, $[200\ 0]^\intercal$, and $[100\ 400]^\intercal$


% The birth intensity follows a Poisson RFS where is initialized for each scenario. 
% The state-dependent target birth rate is computed by
% \begin{equation}
%     \omega_{\gamma,k}(x) = \max \left\{ \omega_{\gamma,0}, \omega_{\gamma,FoV}(x, B_{FoV}) \right\}
% \end{equation}
% where the initial birth rate $\omega_{\gamma,0}$ is initialized to $0.1$ at the beginning of the simulation. $\omega_{\gamma,FoV}(x, B_{FoV})$ is the birth rate of a target with state $x$ with respect to $B_{FoV}$ \cite{sun2023gaussian}.
The trajectories of the targets are shown in Fig.\ref{fig:trajectories}, and the start and end times of their trajectories are given in Table \ref{table_start_end_trajectories}. Figure \ref{truth_cardinality} shows the number of targets at each timestep of the simulation, which varies between $6$ to $9$.
% In this way, semantic information about the targets (such as the specifications of the surveillance region) can be incorporated into the PHD filtering algorithm.

\begin{figure}[h]
\centering    
\includegraphics[width=0.39\textwidth]{figures/scenario_time_varying_number_trajectory.png}    
    \caption{The true trajectories of the $10$ targets (which are labeled as $T_1, T_2, \dots, T_{10}$) are shown by the solid lines. The start and end points of each trajectory are denoted by the symbols `$\triangle$' and `$\square$', respectively. The dashed line represents the boundary of the surveillance region.} 
    \label{fig:trajectories}
\end{figure}
%
\begin{figure}[h]
\centering
\includegraphics[width=0.5\textwidth]{figures/truth_cardinality.png}    
    \caption{Total number of targets (i.e., $|X_k|$) over time.} 
    \label{truth_cardinality}
\end{figure}

\begin{table}[h]
\centering
\caption{Starting and ending timesteps of the target trajectories}
\label{table_start_end_trajectories}
    \begin{tabular}{|c|c|c|c|c|c|c|} 
        \cline{1-3} \cline{5-7}
        Target & Start ($s$) & End ($s$) && Target  & Start ($s$) & End ($s$) \\ 
        % \hline \hline
        % \multicolumn{3}{|c|}{\hline} & \multicolumn{3}{|c|}{\hline} \\
        \cline{1-3} \cline{5-7}
        %
        $T_1$ & 1 & 34 && $T_6$ & 1 & 19 \\ 
        \cline{1-3} \cline{5-7}
        $T_2$ & 1 & 40 && $T_7$ & 10 & 40 \\ 
        \cline{1-3} \cline{5-7}
        $T_3$ & 1 & 40 && $T_8$ & 20 & 40 \\ 
        \cline{1-3} \cline{5-7}
        $T_4$ & 1 & 37 && $T_9$ & 16 & 40 \\ 
        \cline{1-3} \cline{5-7}
        $T_5$ & 1 & 40 && $T_{10}$ & 23 & 40 \\ 
        \cline{1-3} \cline{5-7}
    \end{tabular}
\end{table}

% \hlt{Pending...}

The targets are observed using a sensor network having $6$ sensors, over a  $400m\times400m$ surveillance region. The sensor network topology is assumed to be bidirectional, as shown in Figure \ref{fig_network}. 
%
\begin{figure}[t]
\centering
    \includegraphics[width=0.2\textwidth]{figures/network_directed_graph_6_nodes.png}
    \caption{Directed graph of the sensor network with 6 sensors.} 
    \label{fig_network}
\end{figure}
%Note:  diameter = $3$. 
The consensus weights, $\Omega_{ij}$, are chosen as per the \textit{Metropolis} rule described in \cite{xiao2006distributed} (which is commonly used in the literature on distributed PHD filtering \cite{li2018cardinality, li2020parallel}), giving us
\begin{equation}
    \Omega = \begin{bmatrix}
        0.8 & 0.2 & 0 & 0 & 0 & 0\\
        0.2 & 0.4 & 0.2 & 0.2 & 0 & 0\\
        0 & 0.2 & 0.6 & 0 & 0 & 0.2\\
        0 & 0.2 & 0 & 0.4 & 0.2 & 0.2\\
        0 & 0 & 0 & 0.2 & 0.6 & 0.2\\
        0 & 0 & 0.2 & 0.2 & 0.2 & 0.4
    \end{bmatrix}.
\end{equation}

% state model and measurement model
Similar to the target survival probability, the detection probability $p_{i,k}^D(x)$ is set to $0.98$ within the surveillance region and $0$ outside it. The measurement is given by (\ref{eq:measurement_model}), where
\begin{equation}
    H_{k} = \begin{bmatrix}
        1 & 0 & 0 & 0\\
        0 & 1 & 0 & 0
    \end{bmatrix}\ \ \text{and}\ \ 
    R_{k} = \begin{bmatrix}
        25 & 0\\
        0 & 25
    \end{bmatrix}m^2.
\end{equation}
The clutter is modeled as a Poisson RFS having the intensity
\begin{equation}
    \kappa_{k}(z) = \lambda_{c} A u(z)
\end{equation}
where $u(\cdot)$ is the uniform distribution over the surveillance region (whose integral is equal to $1$), $A=1.6 \times 10^5 m^2$ is the area of the surveillance region, and $\lambda_{c}$ is the average number of clutter measurements per unit volume, with $\lambda_{c}=3.125\times10^{-5}m^{-2}$. As the integral of $\kappa_{k}(z)$ over the surveillance region is $5$, an average of $5$ clutter measurements are generated
at each sensor at each timestep.

% filter parameters
In addition to the simulation parameters given above, the local GM-PHD filter has additional parameters related to the pruning of GMs at each timestep, whose description can be found in \cite[Table II]{vo2006gmphd}. The pruning threshold is set to $10^{-5}$, the merging threshold is set to $15$, and the maximum number of GM components is set to $50$. At each sensor, the GM components whose weights are greater than or equal to $0.5$ are identified as the targets. In the literature, \textit{target extraction} refers to the process by which, at each sensor, the GM components whose weights are higher than a given threshold are identified as the targets. The threshold for the extraction of targets is set to $0.5$.

% performance metrics
\subsection{Results}
The Optimal Sub-Pattern Assignment (OSPA) distance \cite{schuhmacher2008consistent} is used to quantitatively evaluate the multi-target tracking performance of the proposed algorithm at each sensor, with the order parameter of OSPA chosen as $1$ and the cut-off parameter chosen as $100$. Generally, a high OSPA value is indicative of one or more of the following: (i) a large amount of discrepancy between the extracted target states and the true states of the targets, (ii) under-estimation of the number of targets (e.g., due to missed detections), or (iii) over-estimation of the targets (e.g., due to false positives).
We define the \textit{network OSPA} distance as the average of the OSPA distances attained at all the sensors in the network at each timestep. Furthermore, the \textit{time-averaged network OSPA} distance is the average of the network OSPA distances across all the timesteps of the simulation.
In the following simulation results, each of the metrics is averaged over 100 Monte Carlo simulations; the number of targets, their initial positions, and their starting/ending timesteps (as given in Table \ref{table_start_end_trajectories}) are kept fixed across the simulations, whereas the process and measurement noise as well as the clutter measurements are randomly sampled in each simulation.
% Bandwidth = 5; communication time = 1~6 (twice of the diameter)

% Fig. 4 & Fig. 5: explain the results for different communication iteration \alpha = 1 & 6
% proposed ->> full consensus when \alpha increase
% When \alpha increase, partial consensus doesn't improve significantly. Because of the rank/threshold rule
% average network performance
Figure \ref{fig_timeAvg_OSPA_B5_t0_t6} shows the time-averaged network OSPA distance for each algorithm, as a function of the number of inter-sensor fusion steps, $\alpha$, which is varied between $0$ and $6$. It can be observed that each method tends to achieve better multi-target tracking performance as $\alpha$ is increased. In particular, the proposed method achieves a multi-target tracking performance that is close to that of the full consensus method, while having a communication cost that is at most that of the partial consensus method borrowed from the literature. 
% Thus, the proposed method, which is based on the random sampling rule developed in Section \ref{sec:comm_constr}, strikes a desirable trade-off between multi-target tracking performance and communication cost.

\begin{figure}[h]
\centering
    \includegraphics[width=0.51\textwidth]{figures/timeAvg_OSPA_B5_t0_t6.png}
    \caption{Time-averaged network OSPA distances plotted as a function of $\alpha$, which represents the number of inter-sensor fusion steps used.}
    \label{fig_timeAvg_OSPA_B5_t0_t6}
\end{figure}

% In addition to the time-averaged network OSPA distances, t
The individual OSPA distances of sensor 6 are plotted in Figure \ref{fig_localOSPA_sensor6} and the network OSPA distances are plotted in Figure \ref{fig_networkOSPA}. Figures \ref{fig_localOSPA_sensor6} and \ref{fig_networkOSPA} show that the OSPA distances of sensor 6 are close to the corresponding network average values. 
In each algorithm, the OSPA distances are higher in the timesteps following a change in the number of targets present in the surveillance region; this can be observed by comparing Figure \ref{fig_localOSPA_sensor6} with Figure \ref{truth_cardinality}. At the other timesteps, the OSPA distances of the proposed algorithm are considerably lower than those of the partial consensus method, irrespective of the number of inter-sensor fusion steps used. As discussed in Section \ref{sec:comm_constr}, the improved performance of the proposed distributed GM-PHD algorithm can be attributed to the fact that, when using the random sampling rule, each GM component of a given sensor has a non-zero probability of being transmitted to its neighbors. In contrast, a sensor using the partial consensus method only transmits the top $B$ GM components (in terms of their weights) to its neighbors, where $B=5$ in this simulation scenario.

\begin{figure}[h]
    \centering
    \begin{subfigure}[b]{0.239\textwidth}
    \centering
    \includegraphics[width=\textwidth]{figures/local_OSPA_B5_t1_sensor6.png}
    \caption{$\alpha=1$}
    \label{fig_localOSPA_B5_t1_sensor6}
    \end{subfigure}
    \begin{subfigure}[b]{0.241\textwidth}
    \centering
    \includegraphics[width=\textwidth]{figures/local_OSPA_B5_t6_sensor6.png}
    \caption{$\alpha=6$}
    \label{fig_localOSPA_B5_t6_sensor6}
    \end{subfigure}
    \caption{The OSPA distances attained at sensor 6 at each timestep of the simulation, where $\alpha$ represents the number of inter-sensor fusion steps used.}
    \label{fig_localOSPA_sensor6}
\end{figure}

\begin{figure}[h]
    \centering
    \begin{subfigure}[b]{0.237\textwidth}
    \centering
    \includegraphics[width=\textwidth]{figures/network_OSPA_B5_t1.png}
    \caption{$\alpha=1$}
    \label{fig_networkOSPA_B5_t1}
    \end{subfigure}
    \begin{subfigure}[b]{0.243\textwidth}
    \centering
    \includegraphics[width=\textwidth]{figures/network_OSPA_B5_t6.png}
    \caption{$\alpha=6$}
    \label{fig_networkOSPA_B5_t6}
    \end{subfigure}  
    \caption{The network OSPA distances attained at each timestep of the simulation, where $\alpha$ represents the number of inter-sensor fusion steps used.}
    \label{fig_networkOSPA}
\end{figure}

The computation time required for each filtering step of the algorithms considered in this section is shown in Figure \ref{fig_computeTime_B5}. It can be seen that, while the proposed method has a communication cost that is at most that of the partial consensus method, its computational cost is slightly higher than that of the latter. This can be attributed to the increased complexity of the random sampling rule as compared to the rank rule described in Section \ref{sec:comm_constr}. 
At the same time, the computational cost of the proposed algorithm is lower than that of the full consensus method, due to the lower number of GM components that need to be pruned and fused in the proposed method. Thus, the proposed distributed GM-PHD method strikes a desirable trade-off between computational cost, communication cost, and multi-target tracking performance.

\begin{figure}[h]
\centering
    \includegraphics[width=0.5\textwidth]{figures/computing_time_B5_t6.png}
    \caption{The computation time required for each filtering step of the algorithms considered in Section \ref{sec:simulation}.}
    \label{fig_computeTime_B5}
\end{figure}


