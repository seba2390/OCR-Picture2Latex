\section{Problem Formulation}
\label{sec:prob_form}
\subsection{Target Model}
In the multi-sensor multi-target tracking problem, an unknown time-varying number of dynamical systems, called the \textit{targets}, are observed using a sensor network.
The state of the $l^{th}$ target at timestep $k$ is given by a random vector $x_{k}^{(l)} \in \mathcal{X}$, where $\mathcal{X} = \mathbb{R}^{d_x}$ is the state space of each target and $d_x$ is its dimension. The collection of the target states at timestep $k$ is modeled as a random finite set (RFS)\footnote{A brief introduction to RFSs and their properties can be found in \cite{vo2006gmphd}.} $X_k = \left\{x_{k}^{(1)},...,x_{k}^{(|X_k|)}\right\}$, where $|{}\cdot{}|$ denotes the cardinality of a set.
Thus, the number of targets, $|X_k|$, is an integer-valued random variable.

The state of each target is assumed to evolve independently of the other targets, according to a linear Gauss-Markov process on the state space $\mathcal{X}$ with the transition kernel $f_{k|k-1}$. Thus, given that a target exists at timesteps $k-1$ and $k$, the probability density function (pdf) of its state at timestep $k$ is
\begin{equation}
    f_{k|k-1}(x | x_{k-1}) = 
    \mathcal{N}(x; F_{k-1} x_{k-1}, Q_{k-1})
    \label{eq:dynamic_model}
\end{equation}
where $F_{k-1}$ and $Q_{k-1}$ are known matrices and $x_{k-1} \in\mathcal X$ is the state of the target at timestep $k-1$. 
$\mathcal{N}({}\cdot{}; m, P)$ denotes the pdf of a multivariate Gaussian random vector with mean $m\in \mathbb R^{d_x}$ and covariance $P\in \mathbb R^{d_x\times d_x}$. 
The probability of a target continues to exist across both timesteps is called its \textit{survival probability}, denoted by $p^S_{k}:\mathcal X \rightarrow [0, 1]$, and is assumed to depend only on the current state of each target.

New targets of interests can arise due to spontaneous \textit{births} or \textit{spawns}. Births are independent of $X_k$ and assumed to be sampled from an underlying Poisson RFS having the intensity $\gamma _k:\mathcal X \rightarrow \mathbb R_{\geq0}$, where $\mathbb R_{\geq0}$ denotes the set of non-negative real numbers. 
Spawns correspond to targets that emerge from the vicinity of existing targets, so they are dependent on $X_k$. Given that a target exists at $x_{k-1} \in \mathcal X$ at timestep $k-1$, the new targets spawned from it are assumed to be sampled from
a Poisson RFS having the intensity $\sigma _{k|k-1}({}\cdot{}|x_{k-1}):\mathcal X \rightarrow \mathbb R_{\geq 0}$. Finally, note that the transition kernel $f_{k|k-1}$ and spawn intensity $\sigma_{k|k-1}$ are independent of the labels or identities of the targets; if they are dependent on the target labels, a labelled RFS can be used to represent the multi-target state instead \cite{vo2014labeled}.

\subsection{Sensor Network Model}
% sensor network and measurements
The targets are being observed by
a sensor network consisting of a number of spatially distributed sensors, which can be represented as a directed graph $\mathcal G=(\mathcal V, \mathcal E)$, where $\mathcal V$ denotes its vertices (representing the sensors) and $\mathcal E \subseteq \mathcal V \times \mathcal V$ is the set of edges (representing the directional communication links between sensors).
The graph $\mathcal G$ is assumed to be strongly connected. 
% The total number of sensors in the network is given by $|\mathcal V|$. 
% Given two sensors $i,j\in \mathcal V$, sensor $j$ can receive information from sensor $i$ if and only if $(i,j)\in \mathcal E$.
For each sensor $i \in \mathcal{V}$, ${N}^-_{i} = \left\{ j\in\mathcal{V}: (j,i)\in\mathcal{E} \right\}$ denotes its set of in-neighbors, i.e., the set of sensors from which sensor $i$ can receive data. Similarly, $N^+_i$ is the set of out-neighbors.

At timestep $k$, each sensor $i$ obtains a set of measurements $Z_{i,k} = {\left\{z^{(1)}_{i,k},...,z_{i,k}^{(|Z_{i,k}|)}\right\}}$, where $z_{i,k}^{(l)} \in \mathcal{Z}$. Here, $\mathcal{Z} \subseteq \mathbb{R}^{d_z}$ is the measurement space and $d_z$ is its dimension. It is assumed that the measurements are generated as per the following mechanism: the $i^{th}$ sensor obtains a measurement from a target at $x \in \mathcal X$ with the probability $p^D_{i,k}(x)$, where $p^D_{i,k}:\mathcal X \rightarrow [0,1]$ is called the detection probability of sensor $i$ at timestep $k$. The event of the $l^{th}$ target being detected at the $i^{th}$ sensor is independent of the detections at other target-sensor pairs.
%, so the detections can be thought of as i.i.d. Bernoulli trials. 
Given that a target at $x \in \mathcal X$ is detected by sensor $i$, the pdf of the obtained measurement is denoted as $g_{i,k}({}\cdot{}|x)$ and is  
given by the linear Gaussian model
\begin{equation}
    g_{i,k}(z|x) = \mathcal N(z; H_{i,k} x, R_{i,k})
    \label{eq:measurement_model}
\end{equation}
where $H_{i,k}$ and $R_{i,k}$ are known matrices.
In addition, the obtained measurement can correspond to clutter, which are not generated by any of the targets, but arise due to sensor noise or other random effects. The clutter measurements are assumed to be realizations of a Poisson RFS with the intensity $\kappa_i:\mathcal Z \rightarrow \mathbb R_{\geq 0}$.
Thus, $\lbrace Z_{i,k} \rbrace$ are realizations of an RFS, and the distribution of the cardinality of this RFS (i.e., the number of measurements) is determined by the detection probability $p^D_{i,k}$ and clutter intensity $\kappa _i$.
