\section{Introduction}
\label{sec:introduction}
\IEEEPARstart{M}{ulti-target} tracking refers to the problem of estimating the states of an unknown, time-varying number of dynamical systems (which are called \textit{targets}) using noisy measurements.
Multi-target tracking algorithms have received considerable attention in the last two decades, with new domains of application being formed, such as tracking of multiple autonomous vehicles or monitoring pedestrian traffic using cameras \cite{rangesh2019no, fu2019multi}. In particular, the Gaussian Mixture Probability Hypothesis Density (GM-PHD) filter is popular for its ease of analysis and implementation \cite{vo2006gmphd}. It models the uncertain multi-target state as a Poisson random finite set (RFS), whose intensity is represented using a Gaussian mixture (GM), and is able to track the targets without requiring explicit measurement-to-track or track-to-track data associations.
As the GM-PHD filter was developed on a solid theoretical foundation, its optimality and convergence guarantees are better understood than heuristic algorithms or data-driven approaches to multi-target tracking. 

In the multi-sensor multi-target tracking problem, a network of spatially separated sensors are required to carry out multi-target tracking cooperatively. By sharing information across the communication channels of the sensor network, the uncertainty in the multi-target estimate of each sensor can be reduced \cite{gostar2017cauchy}. This is especially true when the sensors have different sensing capabilities or areas of coverage, such as cameras mounted on drones which are overlooking different parts of the surveillance region \cite{battistelli2020differentFOVs}. A naive extension of the GM-PHD filter to the multi-sensor setting requires an excessive amount of communication to be carried out between the sensors. A much more efficient solution involves running local GM-PHD filters in parallel, followed by a fusion step for combining the sensors' estimates \cite{li2018cardinality, li2018partial, wu2022partial}. This approach is analogous to that of multi-sensor Kalman filtering with consensus or diffusion steps for single-target tracking \cite{talebi2018distributed}, but has the additional capability of being able to track a dynamically varying number of targets without requiring explicit measurement-to-track or track-to-track data association \cite{li2018partial}.

Recently, several authors have observed that the multi-sensor multi-target tracking problem can be solved by combining local GM-PHD filtering with a weighted arithmetic average (WAA) fusion step, in which each sensor updates its estimate to a weighted average of the estimates of the sensor network. WAA fusion can be realized in a distributed manner by using one or more weighted average consensus iterations, such that each consensus iteration only requires the sensors that are connected to communicate with each other.
In \cite{li2018cardinality}, the authors used an average consensus step to fuse the estimated number of targets in the surveillance region (i.e., the estimated cardinality of the RFS), which was shown to improve the overall local PHD filtering performance. In \cite{li2020parallel, li2021is}, and \cite{wu2022partial}, WAA fusion was used to fuse the posterior intensities of the local GM-PHD filters, rather than just the cardinality estimates. The theoretical justification for WAA fusion of intensities was noted by the authors of \cite{battistelli2020differentFOVs}, who observed that it minimizes a cost function based on the Cauchy-Schwarz divergence between RFSs. The authors of \cite{gao2020multiobject} and \cite{gostar2017cauchy} have further connected WAA fusion to other types of cost functions, while contrasting it with the alternative fusion strategy: weighted geometric average (WGA) fusion. Notably, WAA fusion is easier to implement and leads to less false negatives (i.e., missed detections) than WGA fusion, resulting in more consistent cardinality estimates \cite{li2021cphd, gostar2020cooperative}. In \cite{julier2012conservative}, it was noted that WAA fusion also has a desirable consistency/conservativeness property which makes it robust in the presence of unknown cross-correlations between the sensors in a multi-sensor network. However, to the best of our knowledge, none of the existing works has discussed the convergence properties of the weighted average consensus approach, in which consensus is sought over real-valued functions; this is different from consensus of vectors in Euclidean spaces, which has been well-studied in the literature \cite{olfati2007consensus,corless2012consensus}.

The weighted average consensus approach requires each sensor to broadcast all of its GM components to its neighbors, which typically outnumber the targets in the surveillance region, leading to the algorithm incurring a high communication cost at the sensors. Moreover, WAA fusion is known to suffer from a high number of false positives (i.e., the misidentification of background clutter as targets), which can be especially problematic in distributed sensor networks, as the false positives can get amplified due to the double-counting of information between the sensors of the network \cite{li2020arithmetic}. 
To address both of these issues simultaneously, the authors of \cite{li2018partial, li2020parallel, li2021is}, and \cite{wu2022partial} proposed the use of \textit{partial consensus}, in which the sensors do not communicate weak GM components that likely correspond to the false positives. This is achieved either by choosing a threshold at each sensor, such that GM components weaker than this threshold are not used in the inter-sensor communication, or by using only the highest-weighted GM components in the consensus step. 
An apparent drawback of this approach is that the partial consensus step never converges to the weighted arithmetic average, as the GM components smaller than the threshold are never communicated between sensors.

In this paper, we develop a distributed GM-PHD algorithm for multi-sensor multi-target tracking which addresses each of the above limitations of the existing algorithms. To limit the communication bandwidth requirement of the proposed distributed GM-PHD algorithm, we develop a random sampling method for selecting the GM components that are transmitted by each sensor during the average consensus step. The sampling probabilities are designed based on two considerations -- similar to the existing distributed GM-PHD algorithms, the sensors avoid communicating small GM components which are likely to correspond to the false positives, thereby reducing the likelihood of false positives. However, unlike the existing algorithms, the proposed algorithm ensures that the sensors' estimates asymptotically converge (in expectation) to the WAA, which is the optimal multi-sensor multi-target estimate in terms of the Cauchy-Schwarz divergence.
The convergence properties of the proposed algorithm are established mathematically rigorously by conducting an error analysis of consensus of functions in $L^p$ spaces, which has not been studied previously in the literature.
% At the same time, the communication cost of the proposed approach is the same (or lower) than those of comparable algorithms in the literature. 
By numerically comparing the proposed distributed GM-PHD algorithm to existing algorithms, it is demonstrated that the proposed algorithm strikes a desirable trade-off between the optimality and the communication bandwidth requirement of multi-sensor multi-target tracking, making it especially well-suited for use in resource-constrained sensor networks.

The rest of the paper is organized as follows. In Section \ref{sec:prob_form}, the multi-sensor multi-target tracking problem is formulated, and the local GM-PHD filtering recursion is described. Section \ref{sec:fusion_step} develops the proposed inter-sensor fusion approach for distributed GM-PHD filtering, wherein we conduct an error analysis of the weighted average consensus step. The proposed random sampling rule is developed in Section \ref{sec:comm_constr}. In Section \ref{sec:simulation}, we compare the proposed algorithm to some of the existing multi-sensor GM-PHD filtering algorithms in the literature. Finally, Section \ref{sec:conclusion} presents the conclusion.
% \textit{Notation:} The 
% For a set $\mathcal A$, we let $|\mathcal A|$ denote its cardinality.
% \red{Notation:\\
% $\mathcal{N}(\cdot; m, P)$: Gaussian probability density\\
% $|\cdot|$: cardinality\\
% $diag(\cdot)$: diagonal matrix\\
% $v_{k|k-1}(x)$: prior intensity\\
% $v_{k}(x)$: posterior intensity\\
% $v_{S,k|k-1}(x), v_{\beta,k|k-1}(x), \gamma_{k}(x)$: intensity of surviving/spawned/birth targets\\
% $J_{k-1}$: number of Gaussian components at time $k-1$\\
% $J_{\beta,k}, J_{\gamma,k}$: number of spawned/birth Gaussian component at time $k$\\
% $p_{S,k}(x)$: state-dependent probability of target survival\\
% $p_{D,k}(x)$: state-dependent probability of target detection\\
% $\omega_{\gamma,k}^{(i)}(x)$: state-dependent target birth rate\\
% $\omega_{k-1}^{(i)}$: weight of $i$-th Gaussian component\\