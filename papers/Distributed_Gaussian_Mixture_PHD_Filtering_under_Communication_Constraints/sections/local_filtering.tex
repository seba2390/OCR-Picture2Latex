\subsection{Local GM-PHD Filtering}
\label{sec:subsec_local_filt}
Distributed multi-target tracking over a sensor network is usually accomplished using a succession of a local filtering step (e.g., using a PHD filter) at each of the sensors followed by a fusion step, wherein the sensors share information across the communication channels. 
%Each iteration of the local PHD filter consists of a prediction and an update step. 
The predicted multi-target state of sensor $i$ is an RFS whose intensity is denoted as $v_{i,k|k-1}$. It characterizes the information known about the multi-target state $X_k$ at sensor $i$ at timestep $k$, before observing the measurements $Z_{i,k}$. After the incorporation of the measurements $Z_{i,k}$, the updated (posterior) intensity is denoted as $v_{i,k|k}$. 
% Finally, the intensities of the individual sensors are fused together, with the fused intensity at sensor $i$ being denoted as $v^F_{i,k|k}$.

In the proposed distributed GM-PHD filter, each sensor performs the local prediction and measurement update steps according to the GM-PHD filtering algorithm \cite{vo2006gmphd}. The GM-PHD filter uses Gaussian mixtures (GMs) to represent the intensity functions, admitting a closed-form solution to the PHD recursion. Consequently, the GM-PHD filter is optimal in a Bayesian sense, under reasonable assumptions on the target and sensor models; a more complete discussion of the relevant assumptions may be found in \cite{vo2006gmphd}. In practice, the GM-PHD filter is approximated by pruning small GM components to keep the space complexity of the algorithm from growing over time.

Given $x_{k-1} \in \mathcal X$, let the prior intensity $v_{i,k|k-1}$
%, be represented using GMs with $J_{i,k|k-1}$, $J_{\gamma,k}$ and $J_{\sigma,k}$ components, respectively. The GM representations of these intensities are as follows:
be represented using a GM with $J_{i,k|k-1}$ components, as follows:
\begin{align}
    v_{i,k|k-1}(x) &= \sum_{l=1}^{J_{i,k|k-1}}w^{(l)}_{i,k|k-1}\mathcal N(x;m^{(l)}_{i,k|k-1}, P^{(l)}_{i,k|k-1})
    \label{eq:prior_intensity}
%    \gamma_k(x) &= \sum_{l=1}^{J_{\gamma,k}} w^{(l)}_{\gamma,k}\mathcal N(x;m^{(l)}_{\gamma,k}, P^{(l)}_{\gamma,k})\\
  %  \sigma_{k|k-1}(x|x_{k-1}) &= \sum_{l=1}^{J_{\sigma,k}} w^{(l)}_{\sigma,k}\mathcal N(x;F^{(l)}_{\sigma,k-1}x_{k-1}, Q^{(l)}_{\sigma,k-1})
\end{align}
Sensor $i$'s prior estimate of the number of targets is $\sum_{l=1}^{J_{i,k|k-1}} w_{i,k|k-1}^{(l)}$, as this sum corresponds to the integral of the intensity $v_{i,k|k-1}(x)$.

The birth and spawn intensities ($\gamma_k$ and $\sigma_{k|k-1}({}\cdot{}|x_{k-1})$, respectively) are also represented by GMs whose parameters are assumed to be known. The prior intensity $v_{i,k|k-1}$ includes the surviving targets from timestep $k-1$ as well as new targets that were born or spawned at timestep $k$. Since these correspond to independent Poisson RFSs, we can add the corresponding intensity functions as follows:
\begin{equation}
    v_{i,k|k-1}(x) = v_{i,k|k-1}^S(x) + v_{i,k|k-1}^\sigma(x) + \gamma_{k}(x),
\end{equation}
where $v^S_{i,k|k-1}$ corresponds to the targets that survived from timesteps $k-1$ to $k$ and $v_{i,k|k-1}^{\sigma}$ corresponds to newly spawned targets. Similarly, the posterior intensity is computed as a sum of two intensities:
\begin{equation}
    v_{i,k|k}(x) = (1 - p_{D,k}(x)) v_{i,k|k-1}(x) + \sum_{z \in Z_{k}} v_{i,k}^D(x;z)
\end{equation}
where the first term corresponds to the targets which were not detected and $v^{D}_{i,k}(x;z)$ is the new information obtained through the measurement $z\in Z_{k}$. Further details about the computation of prior and posterior intensities ($v_{i,k|k-1}$ and $v_{i,k|k}$, respectively) can be found in \cite{vo2006gmphd}. 

We refer to the computation of $v_{i,k|k-1}$ and $v_{i,k|k}$ using the local measurements $Z_{i,k}$ at sensor $i$ as local GM-PHD filtering.
% \begin{equation}
%     v_{i,k|k-1}^S(x) =  \sum_{l=1}^{J_{i,k-1|k-1}} p_{S,k}(x) w_{i,k-1|k-1}^{(l)} \mathcal{N}(x; m_{S,k|k-1}^{(l)}, P_{S,k|k-1}^{(l)})
% \end{equation}
% \begin{equation}
%     v_{i,k|k-1}^{\sigma}(x) = \sum_{l=1}^{J_{i,k-1|k-1}} \sum_{j=1}^{J_{\sigma,k}} w_{i,k-1|k-1}^{(l)} w_{\sigma,k}^{(j)} \mathcal{N}(x; m_{\sigma,k|k-1}^{(l,j)}, P_{\sigma,k|k-1}^{(l,j)})
% \end{equation}
% \begin{equation}
%     \gamma_{i,k} (x) = \sum_{l=1}^{J_{\gamma,k}} w_{\gamma,k}^{(l)}(x) \mathcal{N}(x; m_{\gamma,k}^{(l)}, P_{\gamma,k}^{(l)}).
%     \label{eq_birth intensity}
% \end{equation}
% With the measurement set $Z_{k}$ at time $k$, the multi-target update intensity $v_{k}(x)$ is also modeled as a Gaussian mixture by
% \begin{equation}
%     v_{i,k|k}(x) = v_{i,k}^{ND}(x) + \sum_{z \in Z_{k}} v_{i,k}^D(x;z)
% \end{equation}
% where $v_{ND,k}(x)$ corresponds to the undetected targets, and $v_{D,k}(x;z)$ corresponds to the contributions from measurement $z \in Z_{k}$, and estimated as
% \begin{equation}
%     v_{i,k}^{ND}(x) = (1 - p_{D,k}(x)) v_{i,k|k-1}(x)
% \end{equation}
% \begin{equation}
%     v_{i,k}^D(x;z) = \sum_{l=1}^{J_{i,k|k-1}} p_{D,k}(x) w_{k|k}^{(l)}(z) \mathcal{N}(x; m_{i,k|k}^{(l)}(z), P_{i,k|k}^{(l)}).
% \end{equation}
In the remainder of this paper, we focus on developing a distributed GM dissemination and fusion protocol for improving the multi-target tracking performance of each sensor. Our proposed distributed GM-PHD filtering algorithm constitutes the local GM-PHD filtering step followed by one or more inter-sensor fusion steps. 