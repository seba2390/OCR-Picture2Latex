\documentclass[12pt]{amsart}
\usepackage{amssymb}
%\usepackage{amssymb,amscd,amsxtra,calc}
\usepackage[all]{xy}
\usepackage[toc,page,title,titletoc,header]{appendix}
\usepackage{xcolor}
%\usepackage{mathrsfs}
%\usepackage{mathtools}










\setlength{\headheight}{8pt}
\setlength{\textheight}{22.4cm}
\setlength{\textwidth}{14.5cm}
\setlength{\oddsidemargin}{.1cm}
\setlength{\evensidemargin}{.1cm}
\setlength{\topmargin}{0.2cm}


\setcounter{secnumdepth}{2}
\setcounter{tocdepth}{1}


\def\points{\list
{\hss\llap{\upshape{(\roman{elno})}}}{\usecounter{elno}}}
\let\endpoints=\endlist







%
%
%\documentclass[11pt,twoside,letterpaper]{article} %% The same for {book}
%%\documentclass[12pt]{amsart}
%\setcounter{tocdepth}{1}
%\usepackage{times,fancyhdr}
%\usepackage[dvips]{graphicx}
%\usepackage{amsmath, amsfonts, amssymb, amsthm,MnSymbol}

\theoremstyle{plain}
\newtheorem{Thm}{Theorem}[section]
\newtheorem{Prop}[Thm]{Propsition}
\newtheorem{Lem}[Thm]{Lemma}
\newtheorem{Que}[Thm]{Question}

\newtheorem{Cor}[Thm]{Corollary}
\newtheorem{Prob}[Thm]{Problem}
\newtheorem{Conj}[Thm]{Conjecture}
\theoremstyle{remark}
\newtheorem{Rem}[Thm]{Remark}
\theoremstyle{definition}
\newtheorem{Def}[Thm]{Definition}
\newtheorem{Eg}[Thm]{Example}
\newtheorem*{ZDO}{Zariski-dense orbit Conjecture}


\renewcommand{\theequation}{\arabic{section}.\arabic{equation}}
\renewcommand{\theThm}{\arabic{section}.\arabic{Thm}}
\renewcommand{\theLem}{\arabic{section}.\arabic{Lem}}
\renewcommand{\theCor}{\arabic{section}.\arabic{Cor}}
\renewcommand{\theConj}{\arabic{section}.\arabic{Conj}}

\def\A{\mathbb A}
\def\B{\mathbb B}
\def\F{\mathbb F}
\def\N{\mathbb N}
\def\Z{\mathbb Z}
\def\Q{\mathbb Q}
\def\R{\mathbb R}
\def\C{\mathbb C}
\def\sC{\mathcal C}
\def\D{\mathbb D}
\def\P{\mathbb P}
\def\sM{\mathcal M}
\def\sO{\mathcal O}
\def\Om{\Omega}
\def\rog{{\rm rog}}
\def\PGL{{\rm PGL}}
\def\Spec{{\rm Spec}}
\def\zar{{\rm zar}}
\def\alg{{\rm alg}}
\def\Per{{\rm Per}}
\def\Aut{{\rm Aut}}
\def\Gal{{\rm Gal}}
\def\Hom{{\rm Hom}}
\def\la{\lambda}
\def\La{\Lambda}
\def\Ack{\noindent {\bf Acknowledgment}}

%\setlength{\topmargin}{-0.35in}
%\setlength{\textheight}{8.5in}
%\setlength{\textwidth}{5.5in} 
%\setlength{\oddsidemargin}{0.5in} \setlength{\evensidemargin}{0.5in}
%\setlength{\headheight}{26pt} \setlength{\headsep}{8pt}
%
%\makeatletter \setlength\@fptop{0\p@} \makeatother
%
%\makeatletter
%\def\cleardoublepage{\clearpage\if@twoside \ifodd\c@page\else%
%    \hbox{}%
%    \thispagestyle{empty}%
%    \newpage%
%    \if@twocolumn\hbox{}\newpage\fi\fi\fi}
%\makeatother

\begin{document}
%\title[]{Infinitely many linearly independent characteristic exponents for non-exceptional rational maps}
\title[]{Space spanned by characteristic exponents}

\author{Zhuchao Ji}

\address{Institute for Theoretical Sciences, Westlake University, Hangzhou 310030, China}

\email{jizhuchao@westlake.edu.cn}

\author{Junyi Xie}

\address{Beijing International Center for Mathematical Research, Peking University, Beijing 100871, China}


\email{xiejunyi@bicmr.pku.edu.cn}


\author{Geng-Rui Zhang}

\address{School of Mathematical Sciences, Peking University, Beijing 100871, China}

\email{grzhang@stu.pku.edu.cn}

%\author{Zhuchao Ji
%\thanks{Institute for Theoretical Sciences, Westlake University, Hangzhou 310030, China. jizhuchao@westlake.edu.cn.}, Junyi Xie\thanks{Beijing International Center for Mathematical Research, Peking University, Beijing 100871, China. xiejunyi@bicmr.pku.edu.cn.}, and Geng-Rui Zhang\thanks{School of Mathematical Sciences, Peking University, Beijing 100871, China. grzhang@stu.pku.edu.cn.}}


\bibliographystyle{alpha}




\maketitle



\begin{abstract}
We prove several rigidity results on multiplier and length spectrum. 
For example, we show that for every non-exceptional rational map $f:\P^1(\C)\to\P^1(\C)$ of degree $d\geq2$, the $\Q$-vector space generated by the characteristic exponents (that are not $-\infty$) of periodic points of $f$ has infinite dimension.  This  answers a stronger version of a question of Levy and Tucker. Our result can  also be seen as a generalization of  recent results of Ji-Xie and of Huguin which proved Milnor's conjecture about rational maps having integer multipliers.  
We also get a characterization of postcritically finite maps by using its length spectrum.
Finally as an application of our result, we get a new proof of the Zariski dense orbit conjecture for endomorphisms on $(\P^1)^N, N\geq 1$. 
\end{abstract}

%\noindent \textbf{2020 Mathematics Subject Classification:} Primary
%37P35; Secondary 37P05, 37F10.\\
%\noindent \textbf{Key words:} characteristic exponent, multiplier, multiplicative independence, equidistribution, Zariski-dense orbit conjecture.
	
\tableofcontents
	
\section{Introduction}
Let $f: \P^1\to \P^1$ be a rational map over $\C$ of degree $d\geq2$.
Our aim is to study the $\Q$-vector space spanned by the characteristic exponents of periodic points of a rational map on $\P^1(\C)$ and prove some rigidity results.


\subsection{Multiplier, length and characteristic exponent}
Let $z_0\in\P^1(\C)$ be a periodic point of $f$ with exact period $n$. Define $n_f(z_0):=n$ be this period. We write $n(z_0)$ for simplicity when the map $f$ is clear. The \emph{multiplier}  $\rho_f(z_0)$ of $f$ at $z_0$ is defined to be the differential $df^{n}(z_0)\in\C$. We write $\rho(z_0)$ for simplicity when the map $f$ is clear. 
The \emph{length} of $f$ at $z_0$ is the norm $|\rho_f(z_0)|.$
The multiplier and the length are invariant under conjugacy. 
The \emph{characteristic exponent} of $f$ at $z_0$ is defined to be $\chi_f(z_0):=n^{-1}{\rm log}\lvert\rho_f(z_0)\rvert$. 

\medskip

Denote by ${\rm Per}(f)(\C)$ the set of all periodic points in $\P^1(\C)$ of $f$
and define ${\rm Per}^*(f)(\C):=\{z_0\in{\rm Per}(f)(\C):\rho_f(z_0)\neq0\}$. When the base field $\C$ is clear, we also write ${\rm Per}(f)$ and ${\rm Per}^*(f)$ for simplicity. 


\medskip


The \emph{Lyapunov exponent} (of the maximal entropy measure) of $f$ is defined by $$\mathcal{L}_f:=\int_{\P^1(\C)}{\rm log}|d f| d\mu_f,$$ where $\mu_f$ is the unique maximal entropy measure, and the norm of the differential is computed with respect to the spherical metric.


\subsection{Exceptional maps}
In complex dynamics, the exceptional maps defined below are often considered as exceptional examples among all rational maps.
We may view them as  rational maps on $\P^1(\C)$ related to algebraic groups.


\begin{Def}
Let $f:\P^1\to \P^1$ be an endomorphism over $\C$ of degree $d\geq 2$. 
\begin{itemize}
\item It is called \emph{Latt\`es} if it is semi-conjugate to an endomorphism on an elliptic curve.
Further it is called \emph{flexible Latt\`es} if it is semi-conjugate to the multiplication by an integer  $n$ on an elliptic curve for some $|n|\geq 2.$ 
Otherwise, it is called \emph{rigid Latt\`es}.
\item We say that $f$ is of \emph{monomial type} if it semi-conjugate to the map $z\mapsto z^n$ on $\P^1$ for some integer $n$ with $|n|\geq 2.$ 
\item We call $f$ \emph{exceptional} if it is Latt\`es or of monomial type. An endomorphism $f$ is exceptional if and only if some iterate  $f^k$ is exceptional ($k\in\Z_{>0}$).
\end{itemize}
 \end{Def}



\subsection{Statement of the main results}
We fix an embedding of the algebraic closure $\overline{\Q}$ of $\Q$ in $\C$ and identify  $\overline{\Q}$ as a subfield of $\C$, hence any number field is a subfield of $\C$. Denote the usual absolute value on $\C$ by $| \cdot |$.




\medskip

%\par The followings are our main results.

Our first result shows that the definition field of a non-flexible Latt\`es rational map is determined by its length spectrum. 
\begin{Thm}\label{thmalglenintro}Let $f: \P_{\C}^1\to \P_{\C}^1$ be a rational map of degree at least $2$.
Assume that  $f$ is not a flexible Latt\`es map and  for every $x\in {\rm Per}(f)(\C)$, $|\rho_f(x)|\in \overline{\Q}$. Then $f$ is defined over $\overline{\Q}.$
\end{Thm}
In Theorem \ref{thmalglen}, we indeed proved a more general version of Theorem \ref{thmalglenintro}, in which $\overline{\Q}$ can be replaced to any algebraically closed subfield of $\C$ which is invariant under the complex conjugation. 

\medskip

McMullen's rigidity of multiplier spectrum \cite{McMullen1987} with a standard spread out argument implies that, for a rational map $f$ of degree at least $2$ which is not flexible Latt\`es, if its multipliers at periodic points are all algebraic, then $f$ is defined over $\overline{\Q}$.  Theorem \ref{thmalglenintro} is a generalization of this result from multiplier spectrum to length spectrum (which contains less information). The rigidity of length spectrum was proved in \cite[Theorem 1.5]{Ji2023}. However, the spread out argument does not apply directly in this case as the length spectrum map (and its square) is not algebraic on the moduli space of rational maps. Indeed as shown in \cite[Section 8.1]{Ji2023},  its square is not even real algebraic. 
In Section \ref{sectiontranpoint}, we introduce a way to do the spread out argument respecting the real structure using Weil restriction. 
Another difficulty in the length spectrum case is the lack of noetherianity for semi-algebraic subsets.
We overcome this difficulty using the notion of admissible subsets introduced in \cite{Ji2023}.
\medskip

The following two results concern the $\Q$-vector space spanned by the characteristic exponents of periodic points. 
\begin{Thm}\label{thmmaindim}
	Let $f:\P^1(\C)\to\P^1(\C)$ be a rational map of degree $d\geq2$. Suppose that $f$ is not exceptional. Then the $\Q$-vector space generated by $\{\chi_f(z):z\in{\rm Per}^*(f)\}$ in $\R$ has infinite dimension.
\end{Thm}

\medskip

%The proof of Theorem \ref{thmmaindim} is based on the following result.

%Denote the set of positive integers by $\Z_{>0}$. In this article, we first prove a generalization of Theorem \ref{thmh}:
\begin{Thm}\label{thmmainnumf}
	Let $f:\P^1(\C)\to\P^1(\C)$ be a rational map of degree $d\geq2$. Assume that there exists a number field $K$ such that 
	\begin{equation}\label{equassnorm}
		\forall z_0\in{\rm Per}(f),\ \exists
		n=n(z_0)\in\Z_{>0},\ \lvert\rho_f(z_0)\rvert^n\in K.
	\end{equation} Then $f$ is exceptional.
\end{Thm}


\par 
Finitely many nonzero elements $z_1,\dots,z_N$ in a commutative ring $R$ are called multiplicatively independent if for all triples $(m_1,\dots,m_N)$ of integers, $z_1^{m_1}\cdots z_N^{m_N}=1$ if and only if $m_1=\cdots=m_N=0$. A sequence $(z_n)_{n=1}^\infty$ in $R\setminus\{0\}$ is called multiplicatively independent if any its finite subsequence is multiplicatively independent. Theorem \ref{thmmaindim} immediately implies the existence of infinitely many multipliers for a non-exceptional $f$ whose absolute values are multiplicatively independent.
\begin{Cor}\label{cor1.6}
	Let $f:\P^1(\C)\to\P^1(\C)$ be a rational map of degree $d\geq2$. Suppose that $f$ is not exceptional. Then there exists a sequence $(x_j)_{j=1}^\infty$ in ${\rm Per}^*(f)$ such that the sequence $(\lvert\rho_f(x_j)\rvert)_{j=1}^\infty$ is multiplicatively independent in $\R$.
\end{Cor}




\subsection{Motivations and previous results}
\subsubsection{Milnor's conjecture}
Milnor \cite{milnor2006lattes} has showed that an exceptional rational map $f:\P^1(\C)\to\P^1(\C)$  of degree $d\geq2$ must have all its multipliers of periodic points in the  ring of integers $\mathcal{O}_K$ for some imaginary quadratic number field $K$, and in fact in $\Z$ when $f$ is not a rigid Latt\`es map. Milnor conjectured that the converse is also true. Milnor's conjecture was recently proved by Ji and Xie:
\begin{Thm}[{\cite[Theorem 1.13]{Ji2023}}]\label{thmjx} Let $f:\P^1(\C)\to\P^1(\C)$ be a rational map of degree $d\geq2$. Assume that there exists an imaginary quadratic field $K$ such that all multipliers of $f$ belong to $\mathcal{O}_K$. Then $f$ is exceptional.
\end{Thm} See also \cite{Buff2022} for a different proof. Recently Huguin generalized the above result using different approach:
\begin{Thm}[{\cite[Theorem 7]{Huguin2023}}]\label{thmh}Let $f:\P^1(\C)\to\P^1(\C)$ be a rational map of degree $d\geq2$. Assume that there exists a number field $K$ such that all multipliers of $f$ belong to $K$. Then $f$ is exceptional.
\end{Thm}
Since our assumption (\ref{equassnorm}) in Theorem \ref{thmmainnumf} is weaker than that of Theorem \ref{thmh}, Theorem \ref{thmmainnumf}  is a generalization of Theorem \ref{thmh}. 
Indeed our assumption (\ref{equassnorm}) is even weaker than the condition that there is a number field $K$ such that \begin{equation}
	\forall z_0\in{\rm Per}(f),\ \exists
	n\in\Z_{>0},\ (\rho_f(z_0))^n\in K.
\end{equation}
\subsubsection{A question of Levy and Tucker}
\par On the other hand, in the 2014 AIM workshop Postcritically Finite Maps In Complex And Arithmetic Dynamics, Levy \cite{Levy2014} and Tucker \cite{Tucker2014} asked the following question independently:  
\begin{Que}
Let $f:\P^1(\C)\to\P^1(\C)$ be a non-exceptional rational map of degree $d\geq2$  and let $S$
be the set of all multipliers of periodic points of $f$. Take the subgroup of $\C^\ast$ generated by $S\setminus\{0\}$. Is that true that the rank of this group is infinite? 
\end{Que}


\par It is not hard to see that our Corollary \ref{cor1.6} gives a positive answer to (a generalized version of) Levy and Tucker's question. 


\subsection{Sketch of the proofs}
We have explained the proof of Theorem \ref{thmalglenintro} before. Here we explain the proofs of Theorem \ref{thmmaindim} and Theorem \ref{thmmainnumf}.

\medskip

We first give the idea of the proof of Theorem \ref{thmmainnumf}. 
We argue by contradiction and suppose that $f$ is not exceptional. 
The first step is to reduce to the case where $f$ is defined over $\overline{\Q}$.
This can be done using our Theorem \ref{thmalglenintro}. After enlarging $K$, we may assume that $f$ is defined over $K$. 
In the second step, we combine the arithmetic equidistribution theorem with a result of Zdunik \cite{zdunik2014characteristic} on the Lyapunov exponent to get a contradiction. This argument is inspired by Huguin's proof of Theorem \ref{thmh}. Not like the case of Theorem \ref{thmh}, we can not apply the equidistribution theorem to the one dimensional dynamical system $f:\P^1\to \P^1$ directly. Our idea is
 to consider the two dimensional endomorphism  $F:=f\times\overline{f}$ on $\P^1\times\P^1$ instead.
More precisely, applying a result of Zdunik \cite{zdunik2014characteristic}, we get a sequence $(x_n)_{n=1}^\infty$ of distinct periodic points such that $$\lim\limits_{n\to+\infty}\chi_f(x_n)=a>\mathcal{L}_f.$$ Consider the endomorphism  $F:=f\times\overline{f}$ on $\P^1\times\P^1$ and $\Gamma:=\overline{\{p_n=(z_n,\overline{z_n})\}}^{\rm Zar}\subseteq\P^1\times\P^1$. 
By \cite{Ghioca2011}, the Dynamical Manin-Mumford conjecture holds for $F$. Hence  we may assume that
$\Gamma$ is $F$-invariant. 




Let $\nu_n$ be the discrete probability measure equally supported at the union of Galois orbits of iterates of $p_n$ under $F$. 
Then $\nu_n$ converges weakly to the canonical measure $\mu$ on $\Gamma$ with respect to $F$ by an equidistribution-type theorem (Theorem \ref{equid}), 
which is  a reformulation of \cite[Theorem 3.1]{Yuan2008}, see Section \ref{sectionequid} for details. Applying $\nu_n\to\mu$ to the continuous test functions ${\rm max}\{{\rm log}\lvert{\rm det}(dF)\rvert,A\}$ ($A\in\R$) and letting $A\to-\infty$, we get $$2a\leq\int{\rm log}\lvert{\rm det}(dF)\rvert d\mu,$$ which is impossible since the right hand side equals to $2\mathcal{L}_f<2a$ by a direct computation. 


\medskip


\par Next we sketch the proof of Theorem \ref{thmmaindim}. According to \cite{Douady1993},  postcritically finite (PCF) maps are defined over $\overline{\Q}$ in the moduli space $\mathcal{M}_d$ of rational maps of degree $d$, except for the family of flexible Latt\`es maps. So it  suffices to consider the following two cases:  1).  $f$ is defined over $\overline{\Q}$, and 2). $f$ is not PCF.  For the first case the conclusion follows from Theorem \ref{thmmainnumf}. For the second case, we need to develop some new techniques, which are presented in Section \ref{sectionlinear}. In Section \ref{sectionlinear}, we consider some pseudo linear algebra (which means that the domain may not be the whole vector space), and the vector space $\D(k)_\Q=k^*\otimes_{\Z}\Q$ for a field $k$ of characteristic zero. We will actually prove a theorem (Theorem \ref{thminvo}) stronger than the non-PCF case of Theorem \ref{thmmaindim}, see Section \ref{sectionlinear} and \ref{section6} for details. 
 To prove Theorem \ref{thminvo}, in Section \ref{section6.1} we first deal with the case  that $f$ is defined over $\overline{\Q}$. A key ingredient in this step is \cite[Lemma 4.1]{Benedetto2012} which is a consequence of Siegel's theorem on $S$-integral points. 
 The existence of a no preperiodic critical point is essentially used in here. In Section \ref{section6.2}, we consider the general case and  finish the proof.  This  is achieved by reducing to the case that $f$ is defined over $\overline{\Q}$ via an algebraic-geometric argument and techniques in Section \ref{sectionlinear}.

\subsection{Applications}
\subsubsection{The Zariski-dense orbit Conjecture}
By applying Corollary \ref{cor1.6} we can give a new proof of a special case of the Zariski-dense orbit conjecture.
\begin{ZDO}[=ZDO]Let $k$ be an algebraically closed field of characteristic $0$. Given an irreducible quasiprojective variety $X$ over $k$ and a dominant rational self-map $f$ on $X$. If  we have $\{g\in k(X):g\circ f=g\}=k$ where $k(X)$ is the  function field of $X$, then there exists $x\in X(k)$ whose forward orbit under $f$ is well-defined and Zariski-dense in $X$.  
\end{ZDO}
%\begin{Conj}[Zariski-dense orbit conjecture, ZDO]Let $k$ be an algebraically closed field of characteristic $0$. Given an irreducible variety $X$ over $k$ and a dominant rational self-map $f$ on $X$. If in the function field $K(X)$ we have $\{g\in K(X):g\circ f=g\}=k$, then there exists $x\in X(k)$ whose forward orbit under $f$ is well-defined and Zariski-dense in $X$.  
%\end{Conj}
\begin{Rem}
	The converse of ZDO is easy. For some progressions of ZDO, see e.g.\cite{Amerik2011}, \cite{Amerik2008}, \cite{Medvdev}, \cite{Xie2017} and \cite{Xie2022}.
\end{Rem}
As an application of Corollary \ref{cor1.6}, we give a new proof of (the most difficult part of) a special case of ZDO, which was firstly proved in \cite[Theorem 1.16]{Xie2022}. 
\begin{Thm}\label{thm1.8}
	Let $X=\P^1\times\cdots\times\P^1$ be the variety of product of $N$ copies of projective line over an algebraically closed field $k$ of characteristic $0$. Suppose that $f:X\to X$ is an endomorphism of form $f_1\times\cdots\times f_N$ where $f_j:\P^1\to\P^1$ is a non-constant rational map for $1\leq j\leq N$. The ZDO holds for $X$ and $f$.
\end{Thm}
\begin{Rem}
We note that every dominant endomorphism $f:(\P^1)^N\to(\P^1)^N$ over an algebraically closed field $k$ of characteristic zero must be of form $f_1\times\cdots\times f_N$, after replacing $f$ by a suitable positive-integer iterate. 
\end{Rem}

The original proof of Theorem \ref{thm1.8} in \cite{Xie2022} relies on the solution of the (adelic) Zariski dense orbit conjecture on smooth projective surfaces \cite[Theorem 1.15]{Xie2022}, the notion of adelic topology introduced in \cite[Section 3]{Xie2022} and a classification result on invariant subvarieties of $f:(\P^1)^N\to(\P^1)^N$ \cite[Proposition 9.2]{Xie2022} (see also \cite{Medvdev} and \cite{Ghioca2018c}). When $n=2$, Pakovich gave another proof \cite{Pakovich2023} using his classification of invariant curves in $\P^1\times \P^1$ and some height argument. 
In our new proof, we don't need the ingredients mentioned above. 


%Finally we note that in \cite{Xie2022}, Xie proved that ZDO conjecture holds for any dominant endomorphism $f$ on a smooth projective surface
%$X$ over an algebraically closed field $\mathbf{k}$ of characteristic $0$.
\subsubsection{A characterization of PCF maps}
We also show that one can decide whether a rational map $f:\P^1(\C)\to\P^1(\C)$ of degree $d\geq2$ is PCF with the information of its multiplier spectrum or length spectrum on periodic points. 
\begin{Thm}\label{pcfdec}Let $f:\P^1(\C)\to\P^1(\C)$ be a rational map of degree $d\geq2$. Then the followings are equivalent:\par
(1) $f$ is PCF;\par
(2) $\rho_f(x)\in\overline{\Q}$ for all $x\in{\rm Per}(f)(\C)$ and the $\Q$-subspace $V=V(f)$ of $\R$ is of finite dimension, where $V$ is generated over $\Q$ by $\{\log\lvert N_{K_x/\Q}(\rho_f(x))\rvert:x\in{\rm Per}^*(f)(\C)\}$;\par
(3) $\lvert\rho_f(x)\rvert\in\overline{\Q}$ for all $x\in{\rm Per}(f)(\C)$ and the $\Q$-subspace $W=W(f)$ of $\R$ is of finite dimension, where $W$ is generated over $\Q$ by $\{\log\lvert N_{L_x/\Q}(\lvert\rho_f(x)\rvert)\rvert:x\in{\rm Per}^*(f)(\C)\}$.
\medskip
\par Here $K_x$ (resp. $L_x$) is any number field containing $\rho_f(x)$ (resp. $\lvert\rho_f(x)\rvert$) and $N_{K_x/\Q}$ (resp. $N_{L_x/\Q}$) is the norm map for the extension $K_x/\Q$ (resp. $L_x/\Q$), i.e. the determinant of the $\Q$-linear transformation induced by multiplication by $\rho_f(x)$ (resp. $|\rho_f(x)|$).
\par
\medskip
Clearly, the subspaces $V,W$ above is independent of the choices of the fields $K_x,L_x$, respectively.
\end{Thm}
The proofs of Theorem \ref{thm1.8} and Theorem \ref{pcfdec} will be given in Section \ref{section6}.
\subsection*{Acknowledgement}
The second-named author Junyi Xie would like to thank Thomas Gauthier, Vigny Gabriel, Charles Favre and Serge Cantat for helpful discussions.
\par The first-named author would like to thank Beijing International Center for Mathematical Research in Peking University for the invitation. The second and third-named authors Junyi Xie and Geng-Rui Zhang are supported by NSFC Grant (No.12271007).


\section{Rational maps with algebraic lengths}
Let $K$ be an algebraically closed subfield of $\C$ which is invariant under the complex conjugate $\tau$ i.e. $\tau(K)=K.$
The aim of this section is the following result.

\begin{Thm}\label{thmalglen}Let $f: \P_{\C}^1\to \P_{\C}^1$ be a rational map of degree $d\geq 2$.
	Assume that  $f$ is not a flexible Latt\`es map and  for every $x\in {\rm Per}(f)(\C)$, $|\rho_f(x)|\in K$. Then $f$ is defined over $K.$
\end{Thm}

Applying  Theorem \ref{thmalglen} to the case $K=\overline{\Q}$, we get Theorem \ref{thmalglenintro}.




\subsection{Weil restriction}\label{sectionweilrest}
Recall that $K$ is an algebraically closed field of $\C$ such that $\tau(K)=K.$
Set $L:=K^{\tau}=K\cap \R.$ 
For example, if $K=\C$, then $L=\R.$
We need the following easy lemma.
\begin{Lem}\label{lemkoltwo}We have $K=L+iL,$ in particular $[K:L]=2.$  
\end{Lem}
\proof[Proof of Lemma \ref{lemkoltwo}]
Since $K$ is algebraically closed, $i\in K$. In particular, $K\neq L.$
For every $u\in K$, we may write $$u=\frac{u+\tau(u)}{2}+ \frac{u-\tau(u)}{2i}i$$ 
and both $\frac{u+\tau(u)}{2}$ and $\frac{u-\tau(u)}{2i}$ are contained in $L$. This concludes the proof.
\endproof






We briefly recall the notion of Weil restriction.  See \cite[Section 4.6]{Poonen2017} and \cite[Section 7.6]{Bosch1990} for more information. 

\medskip


Denote by $Var_{/K}$ (resp. $Var_{/L}$) the category of varieties over $K$ (resp. $L$).
For every variety $X$ over $K$, there is a unique variety $R(X)$ over $L$ represents the functor $Var_{/L}\to Sets$
sending $V\in Var_{/L}$ to $\Hom(V\otimes_{L}K, X).$
It is called the \emph{Weil restriction of $X$}. The functor $X\mapsto R(X)$ is called the Weil restriction.
One has the canonical morphism 
$\psi_K: X(K)\to R(X)(L).$ 
When $K=\C$, this map is a real analytic diffeomorphism. One may view $X(K)$ as an $L$-algebraic variety via $\psi_K.$
\begin{Def}\label{defirealzt}The \emph{$L$-Zariski topology} on $X(K)$ is the restriction of the Zariski topology on $R(X)$ via $\psi_K$.
A subset $Y$ of $X(K)$ is \emph{$L$-algebraic} if it is closed in the $L$-Zariski topology.
When $K=\C$, the $L$-Zariski topology is exactly the real Zariski topology as in \cite[Section 8.1.1]{Ji2023}.
\end{Def}




By (iii) of Proposition \ref{probasicweil} below, the  $L$-Zariski topology is stronger than the Zariski topology on $X(K).$


\medskip

When $K=\C$,  roughly speaking, the Weil restriction is just constructed  by
splitting a complex variable $z$ into two real variables $x,y$ via $z=x+iy$.
For the convenience of the reader, in the following example, we show the concrete construction of $R(X)$  when $X$ is affine. 

\begin{Eg}
First assume that $X=\A^N_{K}$. Then $R(X)=\A^{2N}_{L}.$ The map $$\psi_{K}: \A^N_{L}(L)=K^N\to \A^{2N}_{L}(L)=\R^{2N}$$ sends $(z_1, \dots, z_N)$ to $(x_1, y_1, x_2,y_2,\dots, x_N,y_N)$ where $z_j=x_j+iy_j$.

\medskip
Consider the algebra $\B:=K[I]/(I^2+1)\simeq K\oplus IK$.
Every $f\in K[z_1,\dots,z_N]$ defines an element $$F:=f(x_1+Iy_1, \dots, x_N+Iy_N)\in \B[x_1,y_1,\dots, x_N,y_N].$$
Since $$\B[x_1,y_1,\dots, x_N,y_N]=K[x_1,y_1,\dots, x_N,y_N]\oplus IK[x_1,y_1,\dots, x_N,y_N],$$ $F$ can be uniquely decomposed to $$F=r(f)+Ii(f)$$ where $r(f), i(f)\in K[x_1,y_1,\dots,x_N,y_N].$

\medskip

More generally, if $X$ is the closed subvariety of $\A^N_K=\Spec K[z_1,\dots, z_M]$ defined by the ideal $(f_1,\dots, f_s)$, then $R(X)$ is the closed subvariety of 
$$R(\A^N_{K})=\A^{2N}_{L}=\Spec\, L[x_1,y_1,\dots, x_N,y_N]$$ defined by the ideal generated by $r(f_1), i(f_1),\dots, r(f_s), i(f_s)$.
\end{Eg}


\medskip






We list some basic properties of Weil restriction without proof.
\begin{Prop}\label{probasicweil}Let $X, Y\in Var_{/K}$, then we have the following properties: 
\begin{itemize}
%\item $R(X\times Y)=R(X)\times R(Y);$
\item if $X$ is irreducible, then $R(X)$ is irreducible;
\item $\dim R(X)=2\dim X;$
%\item if $X$ is affine then $R(X)$ is affine;
\item if $f:Y\to X$ is a closed (resp. open) immersion, then the induced morphism $R(f):R(Y)\to R(X)$ is a closed (resp. open) immersion.
\end{itemize}
\end{Prop}


We still denote by $\tau$ the restriction of $\tau$ to $K.$
Denote by $X^{\tau}$ the base change of $X$ by the field extension $\tau: K\to K$. This induces a morphism of schemes (over $\Z$) $\tau: X^{\tau}\to X$. It is not a morphism of schemes over $K$.
It is clear that $(X^{\tau})^{\tau}=X.$
\begin{Eg}
If $X$ is the subvariety of $\A^N_{K}=\Spec K[z_1,\dots, z_N]$ defined by the equations 
$\sum_{I}a_{i,I}z^I=0,  i=1,\dots, s$
Then $X^{\tau}$ is the subvariety of $\A^N_{K}$ defined by $\sum_{I}\tau(a_{i,I})z^I=0,  i=1,\dots, s$.
The map $\tau: X=(X^{\tau})^{\tau}\to X^{\tau}$ sends a point $(z_1,\dots,z_N)\in X(K)$ to $(\tau(z_1),\dots,\tau(z_N))\in X^{\tau}(K)$.
\end{Eg}

The following result due to Weil is useful for computing the Weil restriction.
\begin{Prop}\cite[Exercise 4.7]{Poonen2017}\label{proweilconj}
We have a canonical isomorphism $$R(X)\otimes_{L}K\simeq X\times X^{\tau}.$$ Under this isomorphism,  $$R(X)(L)=\{(z_1,z_2)\in X(K)\times X^{\tau}(K)|\,\, z_2=\tau(z_1)\}$$ and
$\psi_K$ sends $z\in X(K)$ to $(z,\tau(z))\in R(X)(L).$
\end{Prop}
















\subsection{Admissible subsets}
In this section,we recall the notion of admissible subsets on real algebraic varieties introduced in \cite{Ji2023}.

\medskip

Let $X$ be a variety over $\R$.
\begin{Def}\cite[Section 8.2]{Ji2023} A closed subset $V$ of $X(\R)$ is called \emph{admissible} if there is a morphism 
$f: Y\to X$ of real algebraic varieties and a Zariski closed subset $V'\subseteq Y$ such that 
$V=f(V'(\R))$ and $f$ is \'etale at every point in $V'(\R).$
\end{Def}

In particular, every algebraic subset of $X(\R)$ is admissible.

\begin{Rem}
Denote by $J$ the non-\'etale locus for $f$ in $V$. We have $J\cap V(\R)=\emptyset.$
Since we may replace $V$ by $V\setminus J$,  in the above definition we may further assume that $f$ is \'etale.
\end{Rem}



\begin{Prop}\label{propadbasic}\cite[Remarks 8.14, 8.15 and Proposition 8.16]{Ji2023} We have the following basic properties:
\begin{itemize}
\item[(1)] Let $Y$ be a Zariski closed subset of $X$.
If $V$ is admissible as a subset of $X(\R)$, then $V\cap Y$ is admissible as a subset of $Y(\R)$.

\item[(2)] An admissible subset is semialgebraic. 
\item[(3)] Let $V_1,V_2$ be two  admissible closed subsets of $X(\R)$. Then $V_1\cap V_2$ is admissible.
\end{itemize}
\end{Prop}





The following theorem shows that admissible subsets satisfy the descending chain condition.
\begin{Thm}\label{thmNoetherianad}\cite[Theorem 8.17]{Ji2023} Let $V_n, n\geq 0$ be a sequence of decreasing admissible subsets of $X(\R)$. Then there is $N\geq 0$ such that $V_n=V_N$ for all $n\geq N.$
\end{Thm}

%
%\medskip
%
%
%Let $d\geq 2.$
%We now view $\text{Rat}_d(\C)$ as a real variety and study the locus in it with given length spectrum.
%For $n\geq 1$, $s=1,\dots, N_n$ and $a\in \R^s/S_s$, let $\La^s_n(a)$ be the subset of $t\in \text{Rat}_d(\C)$ such that 
%$a\subseteq L_n(t)$ i.e. $f_t^n$ has a subset of fixed points counting with multiplicity,  such that the set of norms of multipliers of these fixed points equals to $a.$ It is a closed subset in $\text{Rat}_d(\C)$.
%
%\Rem
% This notion generalizes the notion $\La_n(a)$. When $s=N_n$, we get  $\La_n(a)=\La_n^{s}(a).$
%\endRem
%
%
%Pick $(a_1,\dots, a_s)\in \R^s$ representing $a\in [0,+\infty)^s/ S_s$, we have $$\La^s_n(a)=\phi^s_n(|\la_n^s|^{-1}(a_1,\dots,a_s)).$$
%
%Even though $|\la^s_n|$ is not real algebraic, its square $|\la^s_n|^2$ is real algebraic. So
%$|\la_n^s|^{-1}(a_1,\dots,a_s)=(|\la_n^s|^2)^{-1}(a_1^2,\dots,a_s^2)$ is real algebraic.
%Hence $\La^s_n(a)$ is semialgebraic. 
%Moreover, if $a_i\neq 1$ for every $i=1,\dots, s$, $$|\la_n^s|^{-1}(a_1,\dots,a_s)\subseteq (\la_n^s)^{-1}((\A^1\setminus \{1\})^s).$$
%So $\phi^s_n$ is \'etale along $|\la_n^s|^{-1}(a_1,\dots,a_s).$ This shows the following fact.
%\begin{Prop}\label{prolenlcoadm}
%For $a\in ([0,+\infty)\setminus \{1\})^s/S_s$, $\La^s_n(a)$ is admissible.
%\end{Prop}
%

\subsection{Transcendental points}\label{sectiontranpoint}
Let $X_K$ be a variety over $K$ and $X:= X_{K}\otimes_K \C$. We think that $X_K$ as a model of $X$ over $K.$

Denote by $\pi_K: X\to X_K$ the natural projection. For any point $x\in X(\C)$, define $Z(x)_K$ to be the Zariski closure of $\pi_K(x)$
and $Z(x):=\pi_K^{-1}(Z(x)_K)$. It is clear that $Z(x)$ is irreducible. We call $Z(x)$ the $\C/K$-closure of $x$ w.r.t the model $X_K.$
We say that $x$ is \emph{transcendental} if $\dim Z(x)\geq 1$ and call $\dim Z(x)$ the \emph{transcendental degree} of $x.$

\medskip

The notion of transcendental points (on curves) was introduced in \cite[Section 4.1]{Partial} and it plays important role in \cite{Partial} on the geometric Bombieri-Lang conjecture and \cite{ji2023dao} on the dynamical Andr\'e-Oort conjecture .
Roughly speaking, a very general point in $Z(x)$ satisfies the same algebraic properties as $x.$
In this paper, we study lengths of periodic points in whose definition we need the norm map $|\cdot |: \C\to \R_{\geq 0}$ which is not algebraic. 
However $|\cdot|^2: \C\to \R$ is real algebraic.
For this reason we need to generalize the above notions to respect the real structure. 

\medskip

The Weil restriction $R(X)$ of $X$ w.r.t. $\C/\R$ is a real algebraic variety.  We have $R(X)=R(X_K)\otimes_{L}\R.$ Denote by $\pi_L: R(X)\to R(X_K)$ the natural projection.
For every $x\in X(\C)$, let $Y(x)_L$ be the Zariski closure of $\pi_L(\psi_\C(x))$ and $Y(x):=\pi_L^{-1}(Y(x)_L)$. Set $Z^{\R}(x):=\psi_\C^{-1}(Y(x)(\R))$ which is a real Zariski closed subset of $X(\C).$ 

\medskip


We now give a more concrete description of $Z(x)$ and $Z^{\R}(x)$. Let $U_K$ be an affine open neighborhood of $\pi_K(x)$. Set $U:=\pi^{-1}_K(U_K)=U_K\otimes_K\C$. We have a natural embedding $\pi_K^*: \sO(U_K)\hookrightarrow \sO(U)$. We can view elements in $\sO^K(U):=\pi_K^*(\sO(U_K))$ as the algebraic functions on $U(\C)$ defined over $K.$ 
Then we have $$Z(x)\cap U=\{y\in U|\,\, h(y)=0 \text{ for every } h\in \sO^K(U) \text{ with } h(x)=0\}$$ and $Z(x)$ is the Zariski closure of $Z(x)\cap U.$

\medskip


As $\sO(R(U)_{\C})=\sO(R(U))\otimes_{\R}\C$, every $h\in \sO(R(U)_{\C})$ can be viewed as a $\C$-valued algebraic function on $R(U)(\R).$
Every $h\in \sO(R(U)_{\C})$ induces a function $h\circ \psi_{\C}$ on $U(\C).$ The functions of this form are exactly the $\C$-valued real algebraic functions on $U(\C)$.
Denote by $\sC^{\R-\alg}(U)$ the $\R$-algebra of $\C$-valued real algebraic functions on $U(\C).$
Since algebraic functions are real algebraic, we have a natural embedding $\sO(U)\subseteq \sC^{\R-\alg}(U).$
By Proposition \ref{proweilconj}, we have 
%\begin{equation}\label{equationcralg}
$$\sC^{\R-\alg}(U)\simeq \sO(U)\otimes_{\C}\tau(\sO(U)).$$
%\end{equation}



Let $\sO^L(R(U)):=\pi_L^*(\sO(R(U_K)))$ be the set of algebraic functions defined over $L$ on $R(U).$
Let $\sC^{\R-\alg,L}(U)$ the image of $\sO^L(R(U))\otimes_LK$ in $\sC^{\R-\alg}(U)$, which is the set of $\C$-valued real algebraic functions on $U(\C)$ defined over $L.$
It is clear that $\sO^K(U)\subseteq \sC^{\R-\alg, L}(U).$
By Proposition \ref{proweilconj}, we have 
%\begin{equation}\label{equationcralgl}
$$\sC^{\R-\alg,L}(U)\simeq \sO^K(U)\otimes_{K}\tau(\sO^K(U)).$$
%\end{equation}
We have $$Z^{\R}(x)\cap U(\C)=\{y\in U(\C)|\,\, h(y)=0 \text{ for every } h\in \sC^{\R-\alg,L}(U) \text{ with } h(x)=0\}$$ and $Z^{\R}(x)$ is the real Zariski closure of $Z^{\R}(x)\cap U(\C).$
This implies the following lemma.
\begin{Lem}\label{lemmorphismzx}Let $f_K: X'_K\to X_K$ be a morphisms between $K$-varieties. 
Set $X':=X'_K\otimes_K \C$ and let $f: X'\to X$ be the morphism induced by $f.$
Let $x'\in X'(\C)$ and $x\in X(\C)$ with $f(x')=x$. Then we have $f(Z^{\R}(x'))\subseteq Z^{\R}(x).$
\end{Lem}







\begin{Lem}\label{lemcomparezzr}We have $Z^{\R}(x)\subseteq Z(x)$ and $Z^{\R}(x)$ is Zariski dense in $Z(x).$
In particular, if $x$ is transcendental, then $\dim_{\R} Z^{\R}(x)>1.$
\end{Lem}
\proof
It is clear that $Z^{\R}(x)\subseteq Z(x)$.
After replacing $X_K$ by an affine open neighborhood of $\pi_K(x)$. 
We may assume that $X_K, X$ are affine. 
Let $h\in \sO(X)$ such that $h(Z^{\R}(x))=0.$
Let $e_j, j\in J$ be a $K$-basis of $\C$.  We may assume that $0\in J$ and $1=e_0.$
Write 
$h\otimes_{\C} 1=\sum_{j\in J}g_je_j$. Then
$g_j\in  \sC^{\R-\alg,L}(X)$ and $g_j(x)=0$.

Let $f_n, n\in N$ be a $K$-basis of $\sO^K(X).$ We may assume that $0\in N$ and $1=f_0.$
Write $$g_j=\sum_{m,n\in N}b_{j,m,n}f_m\otimes \tau(f_n).$$
The we get
$$h\otimes_{\C} 1=\sum_{j\in J, m,n\in N}b_{j,m,n}e_jf_m\otimes \tau(f_n).$$
As $e_jf_m\otimes \tau(f_n), j\in J, m,n\in N$ forms a $K$-basis of $\sC^{\R-\alg}(X)$, we have 
$$b_{j,m,n}=0$$
for every $n\neq 0.$ So $g_j=\sum_{m\in N}b_{j,m,0}f_m\otimes 1\in \sO^K(X).$
Since $g_j(x)=0$, $g_j|_{Z(x)}=0.$
Then we have $h|_{Z(x)}=0$ which concludes the proof.
\endproof

\begin{Lem}\label{lemvaluek}Assume that $X_K$ is affine. Let $h\in \sC^{\R-\alg,L}(X)$. For $x\in X(\C)$, if $h(x)\in K$, then $h$ is constant on $Z^{\R}(x).$
\end{Lem}
\proof
Write $h=g\circ \psi_{\C}$ where $g\in \sO^L(R(X))\otimes_LK$. Write $g=\pi_L^*(g_1)+\pi_L^*(g_2)i$ where $g_1,g_2\in \sO(R(X_K)).$ Since $h(x)\in K$, $\pi_L^*(g_1)(\psi(x)), \pi_L^*(g_2)(\psi(x))\in L.$
The map $\pi_L|_{\psi(x)}: \psi(x)\to Y(x)_L$ induces an embedding $\sO(Y(x)_L)\hookrightarrow L.$ The image of $g_i|_{Y(x)_L}, i=1,2$ are contained in $L$.
Hence $g_i|_{Y(x)_L}, i=1,2$ are contained in $L.$ This implies that $\pi_L^*(g_1), \pi_L^*(g_2)$ are constant on $Y(x)(L)$, hence $h$ is constant on $Z^{\R}(x).$ This concludes the proof.
\endproof


\subsection{Moduli space of rational maps}
For $d\geq 2,$ let $\text{Rat}_d$ be the space of degree $d$ endomorphisms on $\P^1$.  
It is a smooth quasi-projective variety of dimension $2d+1$ \cite{Silverman2012}. 
Let $FL_d\subseteq \text{Rat}_d$ be the locus of flexible Latt\`es maps, which is Zariski closed in $\text{Rat}_d$.
The group $\PGL_2= \Aut(\P^1)$ acts on $\text{Rat}_d$ by conjugacy. The geometric quotient 
$$\sM_d:=\text{Rat}_d/\PGL_2$$ is the (coarse) \emph{moduli space} of endomorphisms  of degree $d$ \cite{Silverman2012}.
The moduli space $\sM_d=\Spec (\sO(\text{Rat}_d)^{\PGL_2})$ is an affine variety of dimension $2d-2$ \cite[Theorem 4.36(c)]{Silverman2007}.
%This fact will be used in our proof of Theorem \ref{thmmcmullen}. It follows from the fact that $\text{Rat}_d(\C)$ is affine and the geometric invariant theory \cite[Chapter 1]{Mumford1982}. 
Let $\Psi: \text{Rat}_d\to \sM_d$ be the quotient morphism.  Set $[FL_d]:=\Psi(FL_d).$ 
%The above constructions commute with the base change by $K\hookrightarrow \C.$
The above construction works over any algebraically closed field of characteristic $0$ and commutes with base changes.
\medskip



For every $n\in\Z_{>0}$, let $\Per_n(f_{\text{Rat}_d})$ be the closed subvariety of $\text{Rat}_d\times \P^1$ of the $n$-periodic points of $f_{\text{Rat}_d}.$ 
Let $\phi_n: \Per_n(f_{\text{Rat}_d})\to \text{Rat}_d$ be the first projection. It is a finite map of degree $d^n+1.$
Let $\la_n: \Per_n(f_{\text{Rat}_d})\to \A^1$ be the morphism $(f_t, x)\mapsto df_t^n(x)\in \A^1$. 
View $\Per_n(f_{\text{Rat}_d})$ as the moduli space of endomorphisms of degree $d$ with a marked $n$-periodic point.
We also denote  it by $\text{Rat}_d[n]$ or $\text{Rat}^1_d[n]$.


Let $s_1,\dots, s_n$ be a sequence of elements in $\Z_{\geq 0}$ with $s_1\leq \dots \leq s_n$ and $s_i\leq d^{i!}+1.$
We construct the space $R_d(s_1,\dots, s_n)$ of rational functions of degree $d$ with $s_n$ marked $n!$-periodic points (counting with multiplicities) and in which there are $s_{n-1}$ $(n-1)!$-periodic points (counting with multiplicities) \dots and in which there are $s_{1}$ $1$-periodic points (counting with multiplicities) as follows:
Consider the fiber product $(\text{Rat}_d[n!])^{s_n}_{/\text{Rat}_d}$ of $s_n$ copies of $\text{Rat}_d[n!]$ over 
$\text{Rat}_d.$ For $i\neq j\in \{1,\dots, d^{n!}+1\}$, let 
$\pi_{i,j}: (\text{Rat}_d[n!])^{s_n}_{/\text{Rat}_d}\to (\text{Rat}_d[n!])^2_{/\text{Rat}_d}$ be the projection to the $i,j$ coordinates. 
The diagonal $\Delta\subseteq (\text{Rat}_d[n!])^2_{/\text{Rat}_d}$ is an irreducible component of $(\text{Rat}_d[n!])^2_{/\text{Rat}_d}$.
Consider the open subset $$U:=(\text{Rat}_d[n!])^{s_n}_{/\text{Rat}_d}\setminus (\cup_{i\neq j\in \{1,\dots, d^{n!}+1\}}\pi_{i,j}^{-1}(\Delta)).$$
Let $U'$ be the subset of $U$ of points $(f, x_1,\dots, x_{s_n})$ satisfying $f^{m!}(x_i)=x_i$ for every $m=1,\dots, n$ and $i=1,\dots, s_m$.
This set is open and closed in $U.$ We then define $R_d(s_1,\dots, s_n)$ to be the Zariski closure of $U'$ in $(\text{Rat}_d[n!])^{s_n}_{/\text{Rat}_d}.$
For $m\leq n$, define $\phi_{n,m}: R_d(s_1,\dots, s_n)\to R_d(s_1,\dots, s_m)$ the morphism $(f, x_1,\dots, x_{s_n})\mapsto (f, x_1,\dots, x_{s_m}).$
Moreover, denote by $\phi_{n,0}: R_d(s_1,\dots, s_n)\to {\rm Rat}_d$ the morphism $(f, x_1,\dots, x_{s_n})\mapsto f.$
For $m_1\leq m_2\leq n$,  we have $\phi_{m_2,m_1}\circ\phi_{n,m_2}=\phi_{n,m_1}.$
Let $\la_{s_1,\dots,s_n}: R_d(s_1,\dots, s_n)\to \A^{s_n}$ the morphism
defined by $$(f, x_1,\dots,x_{s_n})\mapsto (df^{n!}(x_1),\dots, df^{n!}(x_{s_n})).$$ 
Since $\phi_n$ is \'etale at every point $x\in \Per_n(f_{\text{Rat}_d})\setminus \la_{s_1,\dots,s_n}^{-1}(1),$
$\phi_{n,0}$ is \'etale at every point $x\in (\la_{s_1,\dots,s_n})^{-1}((\A^1\setminus \{1\})^{s_n}).$ 

%\newpage
%
%
%For every $s=1,\dots, d^n+1$, one may construct the moduli space $\text{Rat}^s_d[n]$ of rational functions of degree $d$ with $s$ marked $n$-periodic points as follows:
%For $s=2,\dots, d^n+1,$
%consider the fiber product $(\text{Rat}_d[n])^s_{/\text{Rat}_d}$ of $s$ copies of $\text{Rat}_d[n]$ over 
%$\text{Rat}_d.$ For $i\neq j\in \{1,\dots, d^n+1\}$, let 
%$\pi_{i,j}: (\text{Rat}_d[n])^s_{/\text{Rat}_d}\to (\text{Rat}_d[n])^2_{/\text{Rat}_d}$ be the projection to the $i,j$ coordinates. 
%The diagonal $\Delta\subseteq (\text{Rat}_d[n])^2_{/\text{Rat}_d}$ is an irreducible component of $(\text{Rat}_d[n])^2_{/\text{Rat}_d}$.
%Define $\text{Rat}^s_d[n]$ to be the Zariski closure of $$(\text{Rat}_d[n])^s_{/\text{Rat}_d}\setminus (\cup_{i\neq j\in \{1,\dots, d^n+1\}}\pi_{i,j}^{-1}(\Delta))$$ in $(\text{Rat}_d[n])^s_{/\text{Rat}_d}.$ Denote by $\phi_n^s: \text{Rat}^s_d[n]\to \text{Rat}_d$ the morphism induced by $\phi_n$. 
%More generally, for every $t=1,\dots, s$, we have morphism $\phi_n^{s/t}: \text{Rat}^s_d[n]\to \text{Rat}^t_d$ by 
%$$(f, x_1,\dots,x_s)\mapsto (f, x_1, \dots, x_t).$$ 
%Let $\la^s_n: \text{Rat}^s_d[n]\to \A^s$ the morphism
%defined by $$(f, x_1,\dots,x_s)\mapsto (df^n(x_1),\dots, df^n(x_s)).$$ 
%%For $i=1,\dots,s$, we denote by $(\la^s_n)_i$, $|\la^s_n|_i$ the $i$-th coordinate of $\la^s_n$ and $|\la^s_n|$ respectively.
%Since $\phi_n$ is \'etale at every point $x\in \Per_n(f_{\text{Rat}_d})\setminus \la_n^{-1}(1),$
%$\phi^s_n$ is \'etale at every point $x\in (\la_n^s)^{-1}((\A^1\setminus \{1\})^s).$ 
%
%\medskip

Define $\sM_d(s_1,\dots,s_n):=R_d(s_1,\dots, s_n)/\PGL_2$ to be the moduli space of endomorphisms of degree $d$ on $\P^1$ 
with $s_n$ marked $n!$-periodic points (counting with multiplicities) and in which there are $s_{n-1}$ $(n-1)!$-periodic points (counting with multiplicities) \dots and in which there are $s_{1}$ $1$-periodic points (counting with multiplicities).
The morphisms $\phi_{n,m}$, $\la_{s_1,\dots, s_n}$ descent to 
$[\phi_{n,m}]:\sM_d(s_1,\dots,s_n)\to \sM_d(s_1,\dots,s_m)$ when $m=1,\dots, n$, $[\phi_{n,0}]:\sM_d(s_1,\dots,s_n)\to \sM_d$ and $[\la_{s_1,\dots, s_n}]: \sM_d(s_1,\dots, s_n)\to \A^{s_n}.$
Then
$[\phi_{n,0}]$ is \'etale at every point $x\in [\la_{s_1,\dots,s_m}]^{-1}((\A^1\setminus \{1\})^{s_n}).$ 

\subsection{Length maps}\label{sectionlengmap}
For $d\geq 2,$ let $s_1,\dots, s_n$ be a sequence of elements in $\Z_{\geq 0}$ with $s_1\leq \dots \leq s_n$ and $s_i\leq d^{i!}+1.$
Let $$|\la_{s_1,\dots,s_n}|: \sM_d(s_1,\dots, s_n)(\C)\to \R_{\geq 0}^{s_n}$$ be the composition of $$[\la_{s_1,\dots, s_n}]: \sM_d(s_1,\dots, s_n)(\C)\to \C^{s_n}$$ and the norm map $$(a_1,\dots, a_{s_n})\in \C^{s_n}\mapsto (|a_1|,\dots, |a_{s_n}|)\in \R_{\geq 0}^{s_n}.$$
Define $$q_{s_1,\dots, s_n}: \sM_d(s_1,\dots, s_n)(\C)\to \R_{\geq 0}^{s_n}$$ be the composition of 
$$|\la_{s_1,\dots,s_n}|: \sM_d(s_1,\dots, s_n)(\C)\to \R_{\geq 0}^{s_n}$$
 and the map $$(a_1,\dots, a_{s_n})\in \R_{\geq 0}^{s_n} \mapsto (a_1^2,\dots, a_{s_n}^2)\in \R_{\geq 0}^{s_n}.$$
It is clear that $$q_{s_1,\dots, s_n}\in \sC^{\R-\alg,L}(\sM_d(s_1,\dots, s_n)(\C)).$$ Here the model of $\sM_d(s_1,\dots, s_n)_{\C}$ over $K$ is taken to be $\sM_d(s_1,\dots, s_n)_{K}.$

\medskip

By Lemma \ref{lemvaluek}, for every $x\in \sM_d(s_1,\dots, s_n)(\C)$, if $q_{s_1,\dots, s_n}(x)\in L^{s_n}$, then $q_{s_1,\dots, s_n}|_{V^{\R}(x)}$ is constant. 
Hence for every $x\in \sM_d(s_1,\dots, s_n)(\C)$, if $|\la_{s_1,\dots,s_n}|(x)\in L^{s_n}$, then $|\la_{s_1,\dots,s_n}||_{V^{\R}(x)}$ is constant. 

\subsection{Rigidity of length spectrum}
In this section, we recall the rigidity of length spectrum proved by Ji and Xie \cite{Ji2023}.


Let $f$ be an endomorphism of $\P^1(\C)$ of degree $d\geq 2$.
As in \cite[Section 8.3]{Ji2023} the \emph{length spectrum} $L(f)=\{L(f)_n, n\geq 1\}$ of $f$ is a sequence of finite multisets\footnote{A multiset is a set except allowing multiple instances for each of its elements. The number of the instances of an element is called the multiplicity. For example: $\{a,a,b,c,c,c\}$ is a multiset of cardinality $6$, the multiplicities for $a,b,c$ are 2,1,3, respectively.}, where $L(f)_n:=L_n(f)$ is the multiset of 
norms of multipliers of all fixed points of $f^n.$
In particular, $L(f)$ is a multiset of non-negative real numbers of cardinality $d^n+1$. 
For every $n\geq 0$, let $RL(f)_n$ be the sub-multiset of $L(f)_n$ consisting of all elements $>1.$
We call $RL(f):= \{RL(f)_n, n\geq 1\}$ the \emph{repelling length spectrum} of $f$ and 
$RL^*(f):= \{RL^*(f)_n:=RL(f)_{n!}, n\geq 1\}$ the \emph{main repelling length spectrum} of $f$.  We have $d^{n}+1\geq\# RL(f)_n\geq d^{n}+1-M$ for some $M\geq 0$.
It is clear that the difference  $d^{n!}+1-\# RL^*(f)_n$ is increasing and bounded.
As $L(f), RL(f)$ and $RL^*(f)$ are invariant under conjugacy, they descent on $\sM_d(\C)$.
For every $[f]\in \sM_d(\C)$, define $L([f]):=L(f), RL([f]):=RL(f)$ and $RL^*([f]):=RL^*(f)$ for any  $f$ in the class $[f]$.

\medskip



Let $\Om$ be the set of sequences $A_n, n\geq 1$ of multisets consisting of real numbers of norm strictly larger than $1$  satisfying $\#A_n\leq d^{n!}+1$ 
and for every $a\in A_n$ with multiplicity $m$, $a^{n+1}\in A_{n+1}$ with multiplicity at least $m$.
For $A,B\in \Om$, we write $A\subseteq B$ if $A_n\subseteq B_n$ for every $n\geq 1$.
An element $A=(A_n)\in \Om$ is called \emph{big} if $d^{n!}+1-\#A_n$ is bounded. 
For every endomorphism $f$ of $\P^1(\C)$ of degree $d$, we have $RL^*(f)\in \Om$ and it is big.


\begin{Thm}\cite[Theorem 8.25]{Ji2023}\label{thmbigspecrig}
If $A\in \Om$ is big, 
then the set 
$$\{f\in \sM_d(\C)\setminus [FL_d]|\,\, A\subseteq RL^*(f)\}$$ is finite.
\end{Thm}


\subsection{Proof of Theorem \ref{thmalglen}}
Let $f: \P_{\C}^1\to \P_{\C}^1$ be a rational map of degree $d\geq 2$.
Assume that  $f$ is not a flexible Latt\`es map and  for every $x\in {\rm Per}(f)(\C)$, $|\rho_f(x)|\in K$. 
We want to show that $[f]\in \sM_d(\C)$ is not transcendental over $K$ for the model $(\sM_d)_K.$
Now assume that $[f]$ is transcendental.
\medskip

Set $A:=RL^*(f)\in \Om$, which is big. Set $s_n:=\#A_n.$ We may pick a sequence of periodic points $x_i, i\geq 1$ such that
for every $n\geq 1$, $x_1,\dots, x_{s_n}$ are fixed by $f^{n!}$ and $A_n=\{|\rho(x_i)|^{n!}, i=1,\dots, s_n\}.$
Let $[f_n]\in \sM(s_1,\dots,s_n)(\C)$ be the point presented by $(f, x_1,\dots, x_{s_n}).$
It is clear that $[\phi_{n,0}]([f_n])=[f]$ for every $n\geq 1$. Since 
$[f]$ is transcendental, for every $n\geq 1$, $[f_n]$ is transcendental. By Lemma \ref{lemcomparezzr}, $\dim_{\R}Z^{\R}(f_n)\geq 1$ for every $n\geq 1.$
Our assumption implies that $|\la_{s_1,\dots, s_n}|([f_n])\in L^{s_n}$.
The last paragraph of Section \ref{sectionlengmap} shows that $|\la_{s_1,\dots, s_n}|$ is constant on $Z^{\R}(f_n).$
As $|\la_{s_1,\dots, s_n}|([f_n])\in (1,+\infty)^{s_n}$, $[\phi_{n,0}]$ is \'etale in a neighborhood of $Z^{\R}(f_n).$ 
Since $[\phi_{n,0}]$ is a finite map, $V_n:=[\phi_{n,0}](Z^{\R}(f_n))$ is closed in $\sM_d(\C)$.
Then $V_n$ is an admissible subset of $\sM_d(\C)$.
Moreover, by Lemma \ref{lemmorphismzx},
$V_n, n\geq 1$ is decreasing.  By Theorem \ref{thmNoetherianad}, there is $N\geq 1$ such that 
$V_n=V_N$ for $n\geq N.$ 
Then for every $g\in V_N$, we have $A\subseteq RL^*([g]).$
Since $[f]\not\in [FL_d]$, $Z^{\R}([f_N])$
is real irreducible and $\dim_{\R}Z^{\R}([f_N])\geq 1$, $V_N\cap (\sM_d(\C)\setminus [FL_d])$ is infinite. 
This contradicts to Theorem \ref{thmbigspecrig}. This concludes the proof.
\qed


\section{An equidistribution theorem}\label{sectionequid}
The following equidistribution-type theorem is a reformulation of \cite[Theorem 3.1]{Yuan2008}.
We only state it in the case where the canonical height of $X$ is $0$, since this case often appear in the dynamical settings.  
Our statement is slightly stronger than \cite[Theorem 3.1]{Yuan2008} as our $S_n$ may contain several Galois orbits. 
We follow the terminology in \cite{Yuan2008}. 
%We need an equidistribution-type theorem, which can be viewed as a generalization of \cite[Theorem 3.1]{Yuan2008}.  
%It is stated as a form that might be useful in many dynamical settings.  We follow the terminology in \cite{Yuan2008}.
\begin{Thm}\label{equid}
	Let $K$ be a number field and $X$ be a projective variety over $K$. Fix an embedding of $K$ into $\C$. 
	Let $\overline{\mathcal{L}}$ be a metrized line bundle on $X$ such that $\mathcal{L}$ is ample and the metric is semipositive. Let $\mu:={\rm deg}_{\mathcal{L}}(X)^{-1}c_1(\overline{\mathcal{L}})^{{\rm dim}\ X}_{\C}$ be the canonical probability measure on $X(\C)$ associated to $\overline{\mathcal{L}}$. For $n\in\Z_{>0}$, let $S_n$ be a countable subset of $X(\overline{K})$ which is ${\rm Gal}(\overline{\Q}/K)$-invariant. For $y\in S_n$, given real numbers $a_{n,y}\geq0$ such that $\sum_{y\in S_n}a_{n,y}=1$ and $a_{n,y}=a_{n,\sigma y}$ for all $y\in S_n$ and $\sigma\in{\rm Gal}(\overline{\Q}/K)$. Assume that $(S_n)_{n=1}^\infty$ satisfies the following two conditions:\par 
	(1) (small) $\sum_{y\in S_n}a_{n,y}h_{\overline{\mathcal{L}}}(y)\to0$ as $n\to +\infty$, here $h_{\overline{\mathcal{L}}}$ is the height function associated with $\overline{\mathcal{L}}$ ;\par 
	(2) (generic) for any proper subvariety $V\subsetneqq X$ of $X$, $\sum_{y\in S_n\cap V}a_{n,y}\to0$ as $n\to +\infty$.\\
	Then the measure $\mu_n:=\sum_{y\in S_{n}}a_{n,y}\delta_y$ converges weakly to $\mu$ on $X(\C)$ as $n\to+\infty$ where $\delta_y$ denotes the Dirac measure at the point $y$, i.e., for all continuous function $g$ on $X(\C)$, we have\begin{equation}\label{2.3}\lim\limits_{n\to\infty}\sum_{y\in S_n}a_{n,y}g(y)=\int_{X(\C)}gd\mu.
	\end{equation}
\end{Thm}

\begin{proof}
Our proof is a small modification of the one for \cite[Theorem 3.1]{Yuan2008}. 

Let $d$ be the dimension of $X$. We say that a continuous function $f$ on $X(\C)$ is smooth if there exists an embedding of $X(\C)$ into a projection manifold $Y$ such that $f$ can be extended to a smooth function on $Y$. As in \cite{Zhang1998}, by the Stone-Weierstrass theorem, continuous functions on $X(\C)$ can be approximated uniformly by smooth functions. Then we suffice to prove (\ref{2.3}) for all smooth real-valued function $f$ on $X(\C)$. Fix such a function $f$. Let $v_0$ be the archimedean place of $K$ corresponding to the fixed embedding $K\hookrightarrow\C$. For a real function $g$ on $X(\C)$ and a metrized line bundle $\overline{\mathcal{G}}=(\mathcal{G},\lVert\cdot\rVert)$ on $X$, we define the twist $\overline{\mathcal{G}}(g):=(\mathcal{M},\lVert\cdot\rVert^\prime)$ to be the line bundle $\mathcal{G}$ on $X$ with the metric $\lVert s\rVert^\prime_{v_0}=\lVert s\rVert_{v_0}e^{-g}$ and $\lVert s\rVert^\prime_{v}=\lVert s\rVert_{v}$ for any $v\neq v_0$. Let $\epsilon>0$. By the adelic Minkowski's theorem (cf. \cite[Appendix C]{Bombieri2006}) and \cite[Lemma 3.3]{Yuan2008}, for a fixed place $\omega_0\in\mathcal{M}_K$ and $N\in\Z_{>0}$, there exists a nonzero small section $s_N\in\Gamma(X,N\mathcal{L})$ such that $$\log\lVert s_N\rVert^\prime_{\omega_0}\leq-\frac{\hat{c}_1(\overline{\mathcal{L}}(\epsilon f))^{d+1}+O(\epsilon^2)}{(d+1){\rm deg}_{\mathcal{L}}(X)}N+o(N)=(-h_{\overline{\mathcal{L}}(\epsilon f)}(X)+O(\epsilon^2))N+o(N)$$ and $\log\lVert s_N\rVert^\prime_{\omega}\leq0$ for all $\omega\neq\omega_0$, where $\lVert\cdot\rVert^\prime_{\omega}$ denotes the metric of $N\overline{\mathcal{L}}(\epsilon f)$. For a point $y$, denote by $\overline{y}$ its Zariski closure. For $N,n\in\Z_{>0}$, denote the vanishing locus of $s_N$ by $V_N\subsetneqq X$, using the condition of $\overline{\mathcal{L}}$, we have\begin{align*}&\sum_{y\in S_n}a_{n,y}h_{\overline{\mathcal{L}}(\epsilon f)}(y)\geq\sum_{y\in S_n,\overline{y}\in V_N}a_{n,y}{\rm deg}(y)^{-1}\left(\sum_{v}\sum_{z\in O(y)}(-N^{-1}\log\lVert s_N(z)\rVert_v^\prime)\right) +0\\
	\geq&\left( \sum_{y\in S_n,\overline{y}\notin V_N}a_{n,y}\right) (h_{\overline{\mathcal{L}}(\epsilon f)}(X)+O(\epsilon^2)+o_N(1)).
	\end{align*}Let $n\to+\infty$, the generic condition (2) implies that $$\liminf_{n\to+\infty}\sum_{y\in S_n}a_{n,y}h_{\overline{\mathcal{L}}(\epsilon f)}(y)\geq h_{\overline{\mathcal{L}}(\epsilon f)}(X)+O(\epsilon^2)+o_N(1).$$Let $N\to+\infty$, then $$\liminf_{n\to+\infty}\sum_{y\in S_n}a_{n,y}h_{\overline{\mathcal{L}}(\epsilon f)}(y)\geq h_{\overline{\mathcal{L}}(\epsilon f)}(X)+O(\epsilon^2).$$ By the definition, it is easy to see that $$\sum_{y\in S_n}a_{n,y}h_{\overline{\mathcal{L}}(\epsilon f)}(y)=\sum_{y\in S_n}a_{n,y}h_{\overline{\mathcal{L}}}(y)+\epsilon\int_{X(\C)}fd\mu_n$$ and $$h_{\overline{\mathcal{L}}(\epsilon f)}(X)=h_{\overline{\mathcal{L}}}(X)+\epsilon\frac{1}{{\rm deg}_{\mathcal{L}}(X)}\int_{X(\C)}fc_1(\overline{\mathcal{L}})_\C^d+O(\epsilon^2).$$ With the small condition (1), dividing $\epsilon$ and setting $\epsilon\to0^+$, we get $$\liminf_{n\to+\infty}\int_{X(\C)}fd\mu_n\geq\frac{1}{{\rm deg}_{\mathcal{L}}(X)}\int_{X(\C)}fc_1(\overline{\mathcal{L}})_\C^d=\int_{X(\C)}fd\mu.$$Replacing $f$ by $-f$ in the above inequality, we get the other direction and thus$$\lim_{n\to+\infty}\int_{X(\C)}fd\mu_n=\int_{X(\C)}fd\mu.$$
\end{proof}
\begin{Rem}
The same idea also applies for a non-archimedean place or the algebraic case, which gives the full analogy of \cite[Theorem 3.1 and 3.2]{Yuan2008}.
\end{Rem}
In order to check the “generic” condition in Theorem \ref{equid}, we need the following lemma. The proof uses the ergodic theory with respect to the constructible topology (on algebraic varieties) introduced by Xie in \cite{Xie2023}. 
\begin{Lem}\label{pergene}
	Let $K$ be a number field and $X$ be a projective variety over $K$. Given a dominant endomorphism $f:X\to X$ and a sequence $(x_n)_{n=1}^\infty$ of periodic points in $X(\overline{K})$ under $f$. Assume that $(x_n)_{n=1}^\infty$ is generic in $X$, i.e., there does not exist a proper Zariski closed subset $Z\subsetneqq X$ containing all $x_n$ except for finitely many. Then for every proper subvariety $V\subsetneqq X$, we have \begin{equation}\label{2.4}
		\frac{\#(V\cap O_f(x_n))}{\# O_f(x_n)}\to0,\ \text{as}\ n\to+\infty,
	\end{equation}where $O_f(x_n)$ is the (forward) orbit of $x_n$ under $f$.
\end{Lem}
\begin{proof}
	Clearly, we suffice to show that for any subsequence $(n_k)_k$ of $(n)_{n=1}^\infty$, there exists a subsubsequence $(n_{k_l})_l$ such that $$\frac{\#(V\cap O_f(x_{n_{k_l}}))}{\# O_f(x_{n_{k_l}})}\to0,\ \text{as}\ l\to+\infty.$$ 
	\par Given a proper subvariety $V\subsetneqq X$ and fix $V$. Let $\lvert X\rvert$ be $X$ equipped with the constructible topology (i.e. the topology of $X$ generated by all its Zariski closed and open subsets) and $\mathcal{M}^1(\lvert X\rvert)$ be the space of all probability Radon measures on $\lvert X\rvert$ with the topology of weak convergence relative to all continuous functions on $\lvert X\rvert$. Then $\mathcal{M}^1(\lvert X\rvert)$ is sequentially compact (cf. \cite[Corollary 1.14]{Xie2023}). For $n\in\Z_{>0}$, set $$m_n=(\# O_f(x_n))^{-1}\sum_{z\in O_f(x_n)}\delta_z.$$By the sequentially compactness of $\mathcal{M}^1(\lvert X\rvert)$, we suffice to show that for any subsequence $(n_k)_k$ of $(n)_{n=1}^\infty$ with $m_{n_k}\to m\ \text{as}\ k\to+\infty$ in $\mathcal{M}^1(\lvert X\rvert)$ for some $m\in\mathcal{M}^1(\lvert X\rvert)$, we have $$\frac{\#(V\cap O_f(x_{n_k}))}{\# O_f(x_{n_k})}\to0,\ \text{as}\ k\to+\infty.$$ Without loss of generality, we may assume that $(m_n)$ itself converges to a measure $m\in\mathcal{M}^1(\lvert X\rvert)$; and we suffice to show (\ref{2.4}) in this case. As $f_*m_n=m_n$, we see that $f_*m=m$. Then according to \cite[Lemma 5.3]{Xie2023}, $m$ must be of form $m=\sum_{y\in S}a_y\delta_{O_f(y)}$, where $S$ is a countable set of periodic elements in $\lvert X\rvert$ under $f$, $a_y\in\R_{\geq0}$ with $\sum_{y\in S}a_y=1$, and $\delta_{O_f(y)}=(\#O_f(y))^{-1}\sum_{z\in O_f(y)}\delta_z$ for $y\in S$. Denote the characteristic function of $V\subsetneqq \lvert X\rvert$ by $1_V$, then $1_V$ is continuous with respect to the constructible topology. As $m_n \to m$, we get$$\frac{\#(V\cap O_f(x_{n}))}{\# O_f(x_{n})}=\int 1_V dm_n\to\int 1_V dm,\ \text{as}\ n\to+\infty.$$ Suppose that (\ref{2.4}) fails. Then there must be a $y\in S$ with $a_y>0$ and $V\cap O_f(y)\neq\emptyset$. Denote the exact period of $y$ under $f$ by $k$. Let $Y$ be the Zariski closure of $\{y\}$. Then $Y\subseteq\cup_{j=0}^{k-1}f^{\circ j}(V)$, hence $Y$ is also a proper Zariski closed subset of $X$. Note that $$\frac{\#(Y\cap O_f(x_{n}))}{\# O_f(x_{n})}=\int 1_Y dm_n\to\int 1_Y dm\geq\frac{a_y}{k}>0,\ \text{as}\ n\to+\infty.$$Hence for every sufficiently large integer $n\gg1$, we have $x_n\in\cup_{j=0}^\infty f^{\circ j}(Y)=\cup_{j=0}^{k-1} f^{\circ j}(Y)$; but $\cup_{j=0}^{k-1} f^{\circ j}(Y)$ has dimension strictly smaller than ${\rm dim}\ X$ by the noetherian condition, contradicting the assumption that $(x_n)_{n=1}^\infty$ is generic in $X$.
\end{proof}

















\section{Proofs of Theorem \ref{thmmainnumf} and the defined over $\overline{\Q}$ case of Theorem \ref{thmmaindim}}\label{section4}
\begin{proof}[Proof of Theorem \ref{thmmainnumf}]
Assume that $f$ is not exceptional. 
By Theorem \ref{thmalglenintro}, our assumption implies that $f$ is defined over $\overline{\Q}$ (after a conjugate over $\C$), hence over a number field $K$. After replacing $K$ by a finite extension of $K$, we may assume that both $f$ and $\overline{f}$ are defined over $K$. Here we denote by $\overline{f}$ the rational map obtained from $f$ via replacing the coefficients by their complex conjugates. According to \cite[Theorem 9 and Lemma 11]{Huguin2023} (cf. \cite{zdunik2014characteristic}), there exists a sequence $(x_n)_{n=1}^\infty$ of distinct points in ${\rm Per}^*(f)$ such that $$a:=\lim\limits_{n\to\infty}\chi_f(x_n)>\mathcal{L}_f,$$ where the limit exists and is finite.


Clearly, $\mathcal{L}_f=\mathcal{L}_{\overline{f}}$. For an arbitrary $x\in{\rm Per}(f)$, we have $\overline{x}\in{\rm Per}(\overline{f})$, $n_f(x)=n_{\overline{f}}(\overline{x})$ and $\rho_f(x)=\overline{\rho_{\overline{f}}(\overline{x})}$, hence $\chi_f(x)=\chi_{\overline{f}}(\overline{x})$.
	
	
Consider the morphism $F:=f\times\overline{f}:\P^1\times\P^1\to\P^1\times\P^1$ over $K$. For $n\in\Z_{>0}$, set $p_n=(x_n,\overline{x_n})\in{\rm Per}(F)$. Let $\Gamma$ be the Zariski closure of $\{p_n:n\in\Z_{>0}\}$ in $\P^1\times\P^1$. As $(p_n)_{n=1}^\infty$ is pairwise distinct, by the noetherian condition, we have ${\rm dim}\ \Gamma\geq1$. After taking a subsequence, we may assume that $\Gamma$ is irreducible and that $(p_n)_{n=1}^\infty$ is generic in $\Gamma$. 

There are 2 cases: $\dim \Gamma=2$ or $1$. When ${\rm dim}\ \Gamma=2$, then $\Gamma=\P^1\times\P^1$ and the canonical probability measure on $\Gamma$ relative to $F$ is $\mu:=\mu_f\times\mu_{\overline{f}}$, where $\mu_f$ and $\mu_{\overline{f}}$ are the canonical measures on $\P^1$ relative to $f$ and $\overline{f}$, respectively. When ${\rm dim}\ \Gamma=1$, by the dynamical Manin-Mumford problem for $F$ on $\P^1\times\P^1$, proved in \cite{Ghioca2011}, $\Gamma$ is periodic under $F$. After replacing $f$ by $f^m$ for some suitable $m\in\Z_{>0}$, we may assume that $\Gamma$ is $F$-invariant. Still denote by $\mu$ the canonical probability measure on $\Gamma$ relative to $F$. In all cases, let $\pi_j:\Gamma\to\P^1$ be the $j$-th projection on $\Gamma$ for $j=1,2$. Then we have 
$$\deg (\pi_1)\mu=\pi_1^*\mu_f,\ \deg (\pi_2)\mu=\pi_2^*\mu_{\overline{f}}.$$
For $n\in\Z_{>0}$, set $$\nu_n=\frac{1}{n_f(x_n)[K_n:K]}\sum_{j=0}^{n_f(x_n)-1}\sum_{\tau\in{\rm Gal}(K_n/K)}\delta_{F^{\circ j}(\tau(p_n))},$$ where $K_n$ is the Galois closure of $K(x_n)$ over $K$ in $\overline{\Q}$ and $K(\infty):=K$. 
	

	
{\bfseries Claim:} $\nu_n$ converges weakly to $\mu$ as $n\to+\infty$.
	
	
We prove the claim using Theorem \ref{equid}. Let $\mathcal{L}$ be the line bundle $\pi_1^*\mathcal{O}_{\P^1}(1)\otimes\pi_2^*\mathcal{O}_{\P^1}(1)$ on $\Gamma$. Then $F\vert_{\Gamma}^*\mathcal{L}\cong\mathcal{L}^{\otimes d}$. By \cite{Zhang1995}, there exists a unique semipositive metric over $\mathcal{L}$ making $F\vert_{\Gamma}^*\mathcal{L}\cong\mathcal{L}^{\otimes d}$ an isometry; denote $\mathcal{L}$ with this metric by $\overline{\mathcal{L}}$. We need to check the conditions (1) and (2) in Theorem \ref{equid}. The condition (1) is trivial, since $h_{\overline{\mathcal{L}}}(\Gamma)=0$ and the height of any periodic algebraic point relative to $\overline{\mathcal{L}}$ is zero. For the condition (2), let $V$ be an arbitrary proper subvariety of $\Gamma$ and fix $V$. By consider the finitely many images of $V$ under Galois transformations, the generic condition (2) follows from Lemma \ref{pergene}. Thus the claim is true.\par
Let $n\in\Z_{>0}$, take $m\in\Z_{>0}$ such that $\lvert\rho_f(x_n)\rvert^m\in K$ by the assumption, and write $l=n_f(x_n)$. For every $\tau\in{\rm Gal}(\overline{\Q}/K)$ and $0\leq j\leq l-1$, we have\begin{align*}
		&{\rm det}(dF^{\circ l}(F^{\circ j}(\tau(p_n))))^m={\rm det}(dF^{\circ l}(\tau(p_n)))^m=\tau({\rm det}(dF^{\circ l}(p_n))^m)\\
		=&\tau(\rho_f(x_n)^m\rho_{\overline{f}}(\overline{x_n})^m)
		=\tau(\lvert \rho_f(x_n)\rvert^{2m})=\lvert \rho_f(x_n)\rvert^{2m},
\end{align*}hence $\lvert{\rm det}(dF^{l}(F^{\circ j}(\tau(p_n))))\rvert=\lvert \rho_f(x_n)\rvert^2$. Then by the definition of $\nu_n$, we have\begin{align*}
		&\int{\rm log}\lvert{\rm det}(dF)\rvert d\nu_n=\frac{1}{l}\int{\rm log}\lvert{\rm det}(dF^{\circ l})\rvert d\nu_n\\
		=&\frac{1}{l^2[K_n:K]}\sum_{j=0}^{l-1}\sum_{\tau\in{\rm Gal}(K_n/K)}{\rm log}\lvert{\rm det}(dF^{\circ l}(F^{\circ j}(\tau(p_n))))\rvert\\
		=&\frac{1}{l^2[K_n:K]}\sum_{j=0}^{l-1}\sum_{\tau\in{\rm Gal}(K_n/K)}{\rm log}\lvert \rho_f(x_n)\rvert^2\\
		=&\frac{2}{l}{\rm log}\lvert \rho_f(x_n)\rvert=2\chi_f(x_n).
	\end{align*}For any $A\in\R$, since the function ${\rm max}\{{\rm log}\lvert{\rm det}(dF)\rvert,A\}$ is continuous, we have\begin{align*}
	&2a=2\lim\limits_{n\to\infty}\chi_f(x_n)=\lim\limits_{n\to\infty}\int{\rm log}\lvert{\rm det}(dF)\rvert d\nu_n\\
	\leq&\lim\limits_{n\to\infty}\int{\rm max}\{{\rm log}\lvert{\rm det}(dF)\rvert,A\} d\nu_n\\
	=&\int{\rm max}\{{\rm log}\lvert{\rm det}(dF)\rvert,A\} d\mu.
\end{align*}Let $A\to-\infty$, by the monotone convergence theorem, we have \begin{equation}\label{3.5}
2\mathcal{L}_f<2a\leq\int{\rm log}\lvert{\rm det}(dF)\rvert d\mu.
\end{equation}When ${\rm dim}\ \Gamma=2$, it is clear that $$\int{\rm log}\lvert{\rm det}(dF)\rvert d\mu=\int_{\P^1\times\P^1}{\rm log}\lvert{\rm det}(d(f\times\overline{f}))\rvert d(\mu_f\times\mu_{\overline{f}})=\mathcal{L}_f+\mathcal{L}_{\overline{f}}=2\mathcal{L}_f,$$ contradicting (\ref{3.5}). When ${\rm dim}\ \Gamma=1$, then\begin{align*}
&\int{\rm log}\lvert{\rm det}(dF)\rvert d\mu=\int_{\Gamma}{\rm log}\lvert{\rm det}(d(f\times\overline{f}))\rvert d\mu\\
=&\int_{\Gamma}{\rm log}\lvert{\rm det}(d f)\circ\pi_1\rvert d\mu+\int_{\Gamma}{\rm log}\lvert{\rm det}(d \overline{f})\circ \pi_2\rvert d\mu\\
=&\int_{\Gamma}{\rm log}\lvert{\rm det}(df)\circ\pi_1\rvert d\frac{\pi_1^*\mu_f}{{\rm deg}(\pi_1)}+\int_{\Gamma}{\rm log}\lvert{\rm det}(d\overline{f})\circ \pi_2\rvert d\frac{\pi_2^*\mu_{\overline{f}}}{{\rm deg}(\pi_2)}\\
=&\int_{\P^1}{\rm log}\lvert{\rm det}(df)\rvert d\mu_f+\int_{\P^1}{\rm log}\lvert{\rm det}(d\overline{f})\rvert d\mu_{\overline{f}}\\
=&\mathcal{L}_f+\mathcal{L}_{\overline{f}}=2\mathcal{L}_f,
\end{align*}contradicting (\ref{3.5}). Therefore, $f$ must be exceptional. We have finished the proof. 
\end{proof}
\begin{proof}[Proof of Theorem \ref{thmmaindim} when $f$ is defined over $\overline{\Q}$]
	Assume that $f$ is defined over $\overline{\Q}$, hence $f$ is defined over a number field $K$. Suppose that the Theorem \ref{thmmaindim} does not hold for $f$. Let $V$ be the $\Q$-span of $\chi_f({\rm Per}^*(f))$ in $\R$, then ${\rm dim}_{\Q}V<\infty$. We can take $M\in\Z_{>0}$ and $x_1,\dots,x_M\in{\rm Per}^*(f)$ such that $\chi_f(x_1),\dots,\chi_f(x_M)$ generate $V$ over $\Q$. By enlarging $K$, we may assume that  $\lvert\rho_f(x_1)\rvert,\dots,\lvert\rho_f(x_M)\rvert\in K$. Then for every $z_0\in{\rm Per}^*(f)$, $\chi_f(z_0)$ is a linear combination of $\chi_f(x_1),\dots,\chi_f(x_M)$ over $\Q$; then it is easy to see that there exists $n\in\Z_{>0},n_1,\dots,n_M\in\Z$ such that$$\lvert\rho_f(z_0)\rvert^n=\lvert\rho_f(x_1)\rvert^{n_1}\cdots\lvert\rho_f(x_M)\rvert^{n_M}\in K,$$ which contradicts Theorem \ref{thmmainnumf} since $f$ is not exceptional by the assumption.
\end{proof}
\section{Some linear algebras}\label{sectionlinear}
\subsection{Pseudo linear algebra}Let $V, W$ be two $\R$-linear spaces. A pseudo morphism $f: V \to W$ is a pair $(V_{f}, f)$ where $V_{f}$ is a linear subspace of $V$ and $f: V_{f} \to W$ is an $\R$-linear map. If $x \in V \setminus V_{f}$, we write $f(x)=\infty$. When $W=\R$, we say that $f$ is a pseudo linear function.\par 
Denote by ${\rm PHom}(V, W)$ the set of pseudo morphisms from $V$ to $W$. For $f, g \in {\rm PHom}(V, W)$, we define $f+g$ to be the pair $(V_{f} \cap V_{g},f\vert_{V_{f} \cap V_{g}}+g\vert_{V_{f} \cap V_{g}})$. Then ${\rm PHom}(V, W)$ is a commutative semigroup with $+$ as the operation. We denote by $0$ the pair $(V, 0)$. We have $0+f=f$ for all $f\in{\rm PHom}(V, W)$. For every $a \in \R$, we define $af$ to be the pair $(V_{f}, a f)$. We note that $f+(-f)=(V_{f}, 0)$, which is not $0$ if $V_f\neq V$. We have an natural embedding ${\rm Hom}(V, W) \hookrightarrow {\rm PHom}(V, W)$.\par 
For $f \in {\rm PHom}(U, V)$ and $ g \in {\rm PHom}(V, W)$, we define their composition $g \circ f$ to be $(U_{f} \cap f^{-1}(V_{g}),g \circ f\vert_{U_{f}\cap f^{-1}(V_{g})})\in{\rm PHom}(U, W)$. Observe that if $f(v)=\infty$, then $g \circ f(v)=\infty$.\par 

Fix a subset $O$ of $V$.
%Let $O$ be a subset of $V$, and fix $O$.
Denote the set of positive real numbers by $\R^+$, and set $\R_{\geq0}:=\R^+\cup\{0\}$.
\begin{Def}
	A sequence $(f_{i})_{i=1}^\infty$ in ${\rm PHom}(V, W)$ is said to be an $O$-sequence if the following conditions are satisfied:\par
	(i) $f_{i}(O) \subseteq \R_{\geq 0} \cup\{\infty\}$ for $i \geq 1$;
	(ii) for every $\lambda \in O$, the set $\{i\geq1:f_{i}(\lambda) \neq 0\}$ is finite.\\
	Clearly, an infinite subsequence of an $O$-sequence is still an $O$-sequence.
\end{Def}
\begin{Def}
	Let $(\lambda_{i})_{i=1}^\infty$ be a sequence in $O$ and $(f_{i})_{i=1}^\infty$ be a sequence in ${\rm PHom}(V,\R)$. We say that $((\lambda_{i})_{i=1}^\infty,(f_{i})_{i=1}^\infty) $ is an upper triangle $O$-system (resp. weak upper triangle $O$-system) if the following conditions hold:\par
	(i) $(f_{i})_{i=1}^\infty$ is an $O$-sequence;\par
	(ii) $f_{i}(\lambda_{i}) \in \R^{+}$ (resp. $f_{i}(\lambda_{i}) \in \R^{+} \cup\{\infty\})$ for $i\geq1$;\par
	(iii) $f_{j}(\lambda_{i})=0$ for $j>i\geq1$.\\
	Clearly, an upper triangle $O$-system is a weak upper triangle $O$-system.
\end{Def}
\begin{Lem}\label{weakindep}
	Let $((\lambda_{i})_{i=1}^\infty,(f_{i})_{i=1}^\infty)$ be a weak upper triangle $O$-system. Then $(\lambda_{i})_{i=1}^\infty$ are linearly independent over $\R$.
\end{Lem}
\begin{proof}
	Since $f_1(\lambda_1)\neq0$, we see that $\lambda_1\neq0$. Then we only need to show that for all $l \geq 2$, $\lambda_{l}$ is not contained in ${\rm span}_{\R}\{\lambda_{i}:i\leq l-1\}$. Otherwise, $\lambda_{l}=\sum_{i=1}^{l-1} a_{i} \lambda_{i}$ for some $l\geq2,a_{i}\in\R, 1\leq i\leq l-1$. Then $f_{l}(\lambda_{l})=\sum_{i=1}^{l-1} a_{i} f_{l}(\lambda_{i})=0$, which contradicts to our assumption.
\end{proof}
Let $\tau: V \to V$ be an involution (i.e. $\tau^2=\mathrm{id}$).
\begin{Lem}\label{invoweak} Assume that $\tau(O)\subseteq O$. Let $((\lambda_{i})_{i=1}^\infty,(f_{i})_{i=1}^\infty)$ be an upper triangle $O$-system. Then there exists a strictly increasing sequence $(m_{i})_{i=1}^\infty$ in $\Z_{>0}$ such that the pair $((\lambda_{m_{i}}+\tau(\lambda_{m_{i}}))_{i=1}^\infty,(f_{m_{i}})_{i=1}^\infty)$ is a weak upper triangle $O^\prime$-system, where $O^\prime=\{\lambda_{m_{i}}+\tau(\lambda_{m_{i}}):i\in\Z_{>0}\}$.
\end{Lem}
\begin{proof}
	It is clear that $(f_{i})_{i=1}^\infty$ is also an $O'$-sequence.
	We construct $(m_{i})_{i=1}^\infty$ recursively. Set $m_{1}:=1$. As $\tau(\lambda_{1}) \in O$, we have $f_{1}(\tau(\lambda_{1})) \in \R_{\geq0} \cup\{\infty\}$. Since $f_{1}(\lambda_{1}) \in \R^{+}$, we have$$f_{m_1}(\lambda_{m_{1}}+\tau(\lambda_{m_{1}}))=f_{1}(\lambda_{1})+f_{1}(\tau(\lambda_{1})) \in \R^{+} \cup\{\infty\}.$$
	Assume that we have constructed $m_{1},\dots,m_{l}$ satisfying the conditions for weak upper triangle systems. Since $(f_{i})_{i =1}^\infty$ is an $O$-system and $\tau(\lambda_1),\cdots,\tau(\lambda_l)\in O$, there exists $m_{l+1}>m_{l}$ such that $f_{m_{l+1}}(\tau(\lambda_{i}))=0$ for all $i=1,\dots,l$. Then for all $i=1,\dots,l$, we have$$f_{m_{l+1}}(\lambda_{m_{i}}+\tau(\lambda_{m_{i}}))=f_{m_{l+1}}(\lambda_{m_{i}})+f_{m_{l+1}}(\tau(\lambda_{m_{i}}))=0;$$also,
	$$f_{m_{l+1}}(\lambda_{m_{l+1}}+\tau(\lambda_{m_{l+1}}))=f_{m_{l+1}}(\lambda_{m_{l+1}})+f_{m_{l+1}}(\tau(\lambda_{m_{l+1}})) \in \R^{+} \cup\{\infty\}$$ and $$f_{m_{i}}(\lambda_{m_{l+1}}+\tau(\lambda_{m_{l+1}}))=f_{m_{i}}(\lambda_{m_{l+1}})+f_{m_{i}}(\tau(\lambda_{m_{l+1}})) \in \R_{\geq0} \cup\{\infty\}.$$We conclude the proof.
\end{proof}
By Lemma \ref{invoweak} and Lemma \ref{weakindep} we get the following result.
\begin{Cor}\label{invoindep}
	Assume that $\tau(O)\subseteq O$. Let $((\lambda_{i})_{i=1}^\infty,(f_{i})_{i=1}^\infty)$ be an upper triangle $O$-system. Then ${\rm dim}_{\R}{\rm span}_{\R}\{ 2^{-1}(\lambda_{i}+\tau(\lambda_{i})):i\geq 1\}=\infty$.
\end{Cor}
Note that the discussion in this subsection also applies with $\R$ replaced by any ordered field $F$. 
\subsection{Linear algebra for multiplication}For every field $k$ of characteristic $0$, denote by $\mu_{k}$ the subgroup of roots of unity in $k$. Denote by $\rog : k^{*} \to \D(k):=k^{*}/\mu_{k}$ the quotient map. Extend $\rog$ to a map $\rog : k \to \D(k) \cup\{\infty\}$ by sending $0$ to $\infty$. Here we use the notation $\rog$ since it is an analogy of the classical $\log$ function to some extent. The embedding $k \hookrightarrow \overline{k}$ gives a natural embedding $\D(k) \hookrightarrow \D(\overline{k})$ as multiplicative abelian groups. Write $\D(k)_{\Q}:=\D(k) \otimes_{\Z} \Q$, where $\D(k)$ is as a multiplicative commutative group, hence a $\Z$-module; then $\D(k)_{\Q}$ is the subspace of $\D(\overline{k})$ spanned by $\D(k)$ over $\Q$. Write $\D(k)_{\R}:=\D(k) \otimes_{\Z} \R$.\par
Let $A\subseteq k$ be an integral domain with ${\rm Frac}(A)=k$. Define $\D(A):=\rog (A\setminus\{0\}) \subseteq$ $\D(k)$, which is a subsemigroup of $\D(k)$. For every prime ideal $p$ of $A$, the surjective projection $A \to A/p$ induces a surjective morphism $s_{p}: \D(A) \cup\{\infty\} \to$ $\D(A / p) \cup\{\infty\}$. In fact, we may view $s_{p}$ as a pseudo morphism
$$
s_{p}: \D(k)_{\R} \to \D({\rm Frac}(A / p))_{\R}
$$with domain $V_{s_{p}}:=(A \setminus p) \otimes_{\Z} \R$.
\subsection{Norms}Let $k$ be a field of characteristic $0$. For every finite field extension $\widetilde{k}$ over $k$ , denote by $N_{\widetilde{k} / k}: \widetilde{k} \to k$ the norm map. We define a morphism $\mathbf{n}_k: \D(\overline{k}) _{\Q}\to \D(k)_{\Q}$ by\begin{equation*}
	\mathbf{n}_k: \rog (x) \mapsto[l:k]^{-1} \rog (N_{l / k}(x)),
\end{equation*}
where $l$ is any finite extension over $k$ containing $x$. We may check that $\mathbf{n}_k$ is well defined and is $\Q$-linear. 
%When we want to emphasize the base field $k$, we write $\mathbf{n}_{k}$ for $\mathbf{n}$. 
We also denote by $\mathbf{n}_k$ its $\R$-linear extension $\mathbf{n}_k: \D(\overline{k})_{\R} \to \D(k)_{\R}$. When the field $k$ is clear, we also write $\mathbf{n}$ for $\mathbf{n}_k$. 
\subsection{Valuations}
Assume that $K$ is a number field. Denote by $\mathcal{M}_K$ the set of all places of $K$. For every $v \in \mathcal{M}_K$, denote by $v: \D(K)_{\R} \to\R$ the $\R$-linear map given by$$\rog(x)\mapsto-\log(\lvert x\rvert_{v}),\ x\in K^*.$$It is easy to check that this map is well-defined and $\R$-linear. We also denote by $v: \D(K)_{\Q} \to \R$ its restriction. For every $a \in \D(K)_{\R}$, the set $\{v \in \mathcal{M}_K:v(a) \neq 0\}$ is finite.\par 
Let $S$ be a finite subset of $\mathcal{M}_K$ containing all the archimedean places. Let $\mathcal{O}_{K, S}$ be the ring of $S$-integers in $K$. Let $\mathcal{O}$ be the integral closure of $\mathcal{O}_{K, S}$ in $\overline{K}$. For every $v \in \mathcal{M}_K \setminus S$ and $\lambda \in \mathcal{O}$, we have $v \circ \mathbf{n}(\lambda) \geq 0$. Write $\mathcal{M}_K \setminus S=\{v_{1}, v_{2},\dots\}$. Then $(v_{i})_{i=1}^\infty\subseteq {\rm Hom}(\D(K)_{\R}, \R)$ is an $\mathcal{O}_{K, S}$-sequence and $(v_{i} \circ \mathbf{n})_{i=1}^\infty\subseteq{\rm Hom}(\D(\overline{K})_{\R}, \R)$ is an $\mathcal{O}$-sequence.
\subsection{Complex conjugation and absolute value}Denote by $\tau: \C\to \C$ the complex conjugation. Then $\R$ is the fixed field $\C^{\tau}$ of $\tau$. As $\Q$-vector spaces, we have an identification $\D(\R)=\R^*/\{\pm 1\}\to \R,\rog (a)\mapsto\log \lvert a\rvert$, where the latter $\log$ is the classical one on $\R^+$. Using this identification, the absolute value on $\C$ can be viewed as the norm $\mathbf{n}_{\R}: \D(\C) \to\D(\R)$ sending $\rog (x)$ to $2^{-1}(\rog (x)+\rog (\tau(x)))$.\par 
Let $\mathbf{k}$ be an algebraically closed subfield of $\C$ stable under the complex conjugation. Still denote by $\tau \in{\rm Gal}(\mathbf{k}/\Q)$ the restriction of the complex conjugation on $\mathbf{k}$. Note that $\tau$ is an involution. Denote by $\mathbf{k}^{\tau}$ the $\tau$-fixed subfield of $\mathbf{k}$. Then the restriction of the absolute value $\mathbf{n}_{\R}$ on $\mathbf{k}$ is $\mathbf{n}_{\mathbf{k}^{\tau}}: \D(\mathbf{k}) \to \D(\mathbf{k}^{\tau}),\rog (x)\mapsto2^{-1}(\rog (x)+\rog (\tau(x)))$.\par
We shall prove the following result. 
\begin{Thm}\label{thminvo}
	Assume that $\mathbf{k}$ is an algebraically closed field of characteristic $0$. Let $\tau \in {\rm Gal}(\mathbf{k} / \Q)$ be an element with $\tau^{2}=\mathrm{id}$. If $f:\P^{1} \to \P^{1}$ is an endomorphism over $\mathbf{k}$ of degree at least $2$ which is not PCF, then the $\Q$-subspace in $\D(\mathbf{k}^{\tau})_{\Q}$ spanned by $\{\mathbf{n}_{\mathbf{k}^{\tau}}(\rog(\rho_f(x))) : x \in{\rm Per}^*(f)(\mathbf{k})\}$ is of infinite dimension.
\end{Thm}


Take $\mathbf{k}=\C$ and let $\tau$ be the complex conjugation, then Theorem \ref{thminvo} implies Theorem \ref{thmmaindim} in the case that $f$ is a non-PCF map.
\begin{Rem}
	Setting $\tau={\rm id}$, then from Theorem \ref{thminvo} we get the following result:\par 
	Assume that $\mathbf{k}$ is an algebraically closed field of characteristic $0$. If $f:\P^{1} \to \P^{1}$ is an endomorphism over $\mathbf{k}$ of degree at least $2$ which is not PCF, then the $\Q$-subspace in $\D(\mathbf{k})_{\Q}$ spanned by $\{\rog(\rho_f(x)) : x \in{\rm Per}^*(f)(\mathbf{k})\}$ is of infinite dimension.
\end{Rem}
\section{Proofs of Theorem \ref{thminvo} and Theorem \ref{thmmaindim} }\label{section6}
\subsection{Proof of Theorem \ref{thminvo}: the case $\mathbf{k}=\overline{\Q}$}\label{section6.1}Let $\tau$ be an element in ${\rm Gal}(\overline{\Q}/\Q)$ with $\tau^{2}=$ id.\par
Denote by $\mathcal{C}_f$ the set of critical points of $f$. Since $f$ is not postcritically finite, there exists $o \in \mathcal{C}_f$ such that the (forward) orbit $O_{f}(o)$ of $o$ is infinite. We fix this critical point $o$. Let $X$ be the union of all (forward) orbits of periodic critical points of $f$. Then $X$ is finite.\par
Pick a number field $K$ satisfying $\tau(K)=K$ and such that $f, o$ and all points in $X$ are defined over $K$.\par
Denote by $\mathcal{M}_K$ the set of places of $K$. Let $B \subseteq \mathcal{M}_K$ be a finite set containing all the archimedean places, satisfying $\tau(B)=B$ and such that for every $v \in \mathcal{M}_K \backslash B, f$ has good reduction at $v$. Then we have $\tau(\mathcal{O}_{K, B})=\mathcal{O}_{K, B}$. For $x\in{\rm Per}(f)(\mathbf{k})$, set $\lambda(x)=(n_f(x))^{-1}\rog(\rho_f(x))\in\D(\mathbf{k})_{\R}\cup\{\infty\}$. Recall that $n_f(x)$ is the exact periods of $x$ and $\rho_f(x)$ is the multiplier of $x$.  Then for all $x \in {\rm Per}^*(f)(\mathbf{k})$, we have $\mathbf{n}_K (\lambda(x))\in \D(\mathcal{O}_{K, B})$.\par
Denote by $\C_{v}$ the completion of the algebraically closure of $K_{v}$. Every embedding $\sigma: \mathbf{k} \hookrightarrow \C_{v}$ gives a bijection $\sigma: {\rm Per}(f)(\mathbf{k}) \to {\rm Per}(f)(\C_{v})$. Observe that for every $x \in {\rm Per}(f)(\mathbf{k})$, we have $\sigma(\lambda(x))=\lambda(\sigma(x))$.\par
For every $v \in \mathcal{M}_K \setminus B$ and $x \in \P^{1}(\C_{v})$, denote by $\tilde{x} \in \P^{1}(\overline{\widetilde{K_{v}}})$ the reduction of $x$ in the special fiber at $v$ and $f_{v}: \P^{1}_{\overline{\widetilde{K_{v}}}} \to \P^{1}_{\overline{\widetilde{K_{v}}}}$ the reduction of $f$. After enlarging $B$, we may assume that $\tilde{o} \notin X_{v}$ where $X_{v}$ is the reduction of $X$ in $\P^{1}(\widetilde{K_{v}}) \subseteq \P^{1}(\overline{\widetilde{K_{v}}})$\par
Observe that for every $v \in \mathcal{M}_K \setminus B, x \in {\rm Per}(f)(\mathbf{k})$ of exact period $n \geq 1$ and any embedding $\sigma: \mathbf{k} \hookrightarrow \C_{v}$, we have $v(\lambda(x))\geq 0$. Moreover the followings are equivalent:\par
(i) $v(\lambda(x))>0$;\par
(ii) there exists an embedding $\sigma: \mathbf{k} \hookrightarrow \C_{v}$ such that $(f^{n}_v)^{\prime}(\widetilde{\sigma(x)})=0$;\par
(iii) there exists an embedding $\sigma: \mathbf{k} \hookrightarrow \C_{v}, q \in \mathcal{C}_f$ and $m \geq 0$, such that $\widetilde{\sigma(q)}$ is periodic for $f_{v}$ and $\widetilde{\sigma(x)}=f_{v}^{m}(\widetilde{\sigma(q)})$.\par
For $v \in \mathcal{M}_K \setminus B$, denote by $P_{v}$ the union of all orbits of periodic critical points of $f_{v}$. Then $P_{v}$ is finite. For every $v \in \mathcal{M}_K \setminus B, q \in P_{v}$, there exists a unique periodic point $y \in {\rm Per}(f)(\C_{v} \cap \mathbf{k})$ such that $\tilde{y}=q$. Then there exists a unique ${\rm Gal}(\mathbf{k} / K)$-orbit $O(q)$ in $\mathbf{k}$ such that for some (then every) $x \in O(q)$, there exists an embedding $\sigma: \mathbf{k} \hookrightarrow \C_{v}$ such that $\widetilde{\sigma(x)}=q$ (here $O(q)$ is the orbit of $q$). In particular, we have $X_{v} \subseteq P_{v}$ and $\cup_{q \in X_{v}} O(q)=X$. It follows that the set$$Q_{v}:=\{x \in {\rm Per}(f)(\mathbf{k}):v(\lambda(x))>0\}=\bigcup_{q \in P_{v}} O(q)$$is finite. Moreover, $Q_{v}=X$ if and only if $P_{v}=X_{v}$.
\begin{Lem}\label{lem5.1}
	The set $S:=\{v \in \mathcal{M}_K \setminus B:P_{v} \setminus X_{v} \neq \emptyset\}$ is infinite.
\end{Lem}
\begin{proof}
	By \cite[Lemma 4.1]{Benedetto2012}, there are infinitely may $v \in \mathcal{M}_K \setminus B$, for which there exists $n \in\Z_{>0}$ such that $f_{v}^{n}(\tilde{o})=\tilde{o}$. For such $v$, we have $\tilde{o} \in P_{v} \setminus X_{v}$, which concludes the proof.
\end{proof}
\begin{Lem}\label{lem5.2}There exists a sequence $(x_{i})_{i=1}^\infty$ in ${\rm Per}^*(f)(\mathbf{k})$ and a sequence $(v_{j})_{j=1}^\infty$ in $\mathcal{M}_{K} \setminus B$ such that $v_{i}(\lambda(x_{i}))>0$ for $i\geq1$ and $v_{j}(\lambda(x_{i}))=0$ for $j \neq i$. In particular, $((\lambda(x_{i}))_{i=1}^\infty,(v_{i})_{i=1}^\infty)$ is an upper triangle $\D(\mathcal{O}_{K, B})$-system for $\D(K)_{\R}$.
\end{Lem}
\begin{proof}We construct these two sequences recursively.\par
	By Lemma \ref{lem5.1}, $S$ is infinite. Pick $v_{1} \in S$, then there exists $x_{1} \in Q_{v_{1}} \setminus X \subseteq$ ${\rm Per}^*(f)(\mathbf{k})$. We have $v_{1}( \lambda(x_{1}))>0$.\par
	Assume that we have constructs $x_{1},\dots,x_{m} \in {\rm Per}^*(f)(\mathbf{k})$ and $v_{1},\dots,v_{m} \in$ $\mathcal{M}_K \setminus B$ such that $v_{j}(\lambda(x_{i})) \geq 0$ and the quality holds if and only if $j\neq i$. The set $\cup_{i=1}^{m} Q_{v_{i}} \setminus X$ is finite. Then there exists a finite set $T_{m} \subseteq \mathcal{M}_K$ such that for all $x\in\cup_{i=1}^{m} Q_{v_{i}} \setminus X$, and $v \in \mathcal{M}_K \setminus T_{m}$, we have $v(x)=0$. By Lemma \ref{lem5.1}, there exists $v_{m+1} \in$ $S \setminus(\{v_{1},\dots,v_{m}\} \cup T_{m})$. Then we have $v_{m+1}(x_{i})=0$ for $i=1,\dots,m$. Pick $x_{m+1} \in Q_{v_{m+1}} \setminus X$. We have $v_{m+1}(x_{m+1})>0$. It follows that $x_{m+1} \notin \cup_{i=1}^{m} Q_{v_{i}}$. Then $v_{i}(x_{m+1})=0$ for $i=1,\dots,m$. We conclude the proof of Lemma \ref{lem5.2}.
\end{proof}Then we conclude the proof by Corollary \ref{invoindep}.
\subsection{Proof of Theorem \ref{thminvo}: the general case}\label{section6.2}
	Denote by $\mathcal{C}_f$ the set of critical points of $f$. Since $f$ is not PCF, there is an $o\in\mathcal{C}_f$ which is not preperiodic. We fix this critical point $o$. Fix a subfield $K$ of $\mathbf{k}$ such that $K/\Q$ is finite generated, $\tau(K)=K$, and $o,f$ are defined over $K$. Without loss of generality, we may assume that $\mathbf{k}=\overline{K}$.\par
	Take a finite generated $\Z$-subalgebra $A$ of $K$ with ${\rm Frac}(A)=K$ and $\tau(A)=A$. After shrinking ${\rm Spec}(A)$, we may assume that there exists an endomorphism $f_A:\P^1_A\to\P^1_A$ over $A$ whose restriction $f_K:\P^1_K\to\P^1_K$ over the generic fiber $\P^1_K$ satisfies $f=f_K\otimes_K\mathbf{k}$.\par 
	For every $c\in{\rm Spec}(A\otimes_{\Z}\overline{\Q})(\overline{\Q})$, denote by $f_c$ the specialization of $f_A$ at $c$, and $o_c$ the specialization of $o$ at $c$. Then $o_c$ is a critical point of $f_c$. By \cite[Lemma 3.3]{Ghioca2018}, there exists $c\in{\rm Spec}(A\otimes_{\Z}\overline{\Q})(\overline{\Q})$ such that the orbit of $o_c$ is infinite. In particular, $f_c$ is not PCF. There exists a number field $L\subseteq \overline{K}$ such that $c$ is defined over $L$. Denote by $A_1$ the algebra generated by $A,\mathcal{O}_L$ and $\tau(\mathcal{O}_L)$; we may replace ${\rm Spec}(A)$ by some Zariski open set of ${\rm Spec}(A_1)$ for which $f_A:\P^1_A\to\P^1_A$ is still everywhere well-defined. We may view ${\rm Spec}(A)$ as an $\mathcal{O}_L$-scheme, and pick a point $c\in{\rm Spec}(A\otimes_{\mathcal{O}_L}L)$ such that the orbit of $o_c$ is infinite. After shrinking ${\rm Spec}(A)$, the Zariski closure of $c$ in ${\rm Spec}(A)$ is isomorphic to ${\rm Spec}(\mathcal{O}_{L,S})$ for a finite set of places $S\subseteq\mathcal{M}_L$ containing all archimedean places. It corresponds to a prime ideal $p$ of $A$.\par 
	Denote by $s_p:\D(K)_{\R}\dashrightarrow\D(L)_{\R}$ the pseudo morphism as in Section \ref{sectionlinear}. We have $s_p(\D(A))\subseteq\D(\mathcal{O}_{L,S})\cup\{\infty\}$. Then for every $v\in\mathcal{M}_L\setminus S$ and $\lambda\in\D(A)$, we have $v(\lambda)\in\R_{\geq0}\cup\{\infty\}$. Moreover, for every $\lambda\in\D(A)$, if $s_p(\lambda)\neq\infty$, then there are only finitely many $v\in\mathcal{M}_L\setminus S$ for which $v(s_p(\lambda))\neq0$.\par 
	For every $y\in{\rm Per}(f)(\overline{K})$, denote by $y_c$ the set of $x\in{\rm Per}(f_c)(\overline{L})$ with whose image is contained in the image of $y$ in $\P_A^1$. For every $y\in{\rm Per}(f)(\overline{K})$, $y_c$ is finite and nonempty. On the other hand, for every $x\in{\rm Per}(f_c)(\overline{L})$, the set of $y\in{\rm Per}(f)(\overline{K})$ with $x\in y_c$ is finite and nonempty. Moreover, if $x\in y_c$, then $$s_p(\mathbf{n}_K(\lambda(y)))=\mathbf{n}_L(\lambda(x)).$$\par 
	Since the set of $x\in{\rm Per}(f_c)(\overline{L})$ with $\mathbf{n}_L(\lambda(x))=\infty$ is finite, the set $$W_c:=\{y\in{\rm Per}(f)(\overline{K}):s_p(\mathbf{n}_K(\lambda(y)))=\infty\}$$ is also finite. Similarly $W_{\tau(c)}:=\{y\in{\rm Per}(f)(\overline{K}):s_{\tau(p)}(\mathbf{n}_K(\lambda(y)))=\infty\}$ is finite.\par 
	By Lemma \ref{lem5.2}, there exists $(y_{i})_{i=1}^\infty$ in ${\rm Per}(f_c)(\overline{L})$ and $(v_{i})_{i=1}^\infty$ in $\mathcal{M}_L \setminus S$ such that $((\mathbf{n}_{L}(\lambda(y_{i})))_{i=1}^\infty,(v_{i})_{i=1}^\infty)$ is an upper triangle $\D(\mathcal{O}_{L, S})$-system for $\D(L)_{\R}$. For every $i \in\Z_{>0}$, there exists $x_{i} \in {\rm Per}(f)(\overline{K})$ such that the image of $y_{i}$ is contained in the Zariski closure of the image of $x_{i}$ in $\P_{A}^{1}$. We have$$s_{p}(\mathbf{n}_{K}(\lambda(x_{i})))=\mathbf{n}_{L}(\lambda(y_{i})).$$After removing finite terms, we may assume that $y_{i} \notin W_{c} \cup W_{\tau(c)}$ for all $i \geq 1$. It follows that $\mathbf{n}_{L}(\lambda(y_{i})) \in \rog (A \setminus(p \cup \tau(p)))$ for $i \geq 1$. Observe that $(v_{i} \circ s_{p})_{i=1}^\infty$ is a $\rog (A \setminus(p \cup \tau(p)))$-sequence. It follows that $((\mathbf{n}_{L}(\lambda(y_{i})))_{i=1}^\infty,(v_{i} \circ s_{p})_{i=1}^\infty)$ is an upper triangle $\rog (A \setminus(p \cup \tau(p)))$-system for $\D(K)_{\R}$. Since $\rog (A \setminus(p \cup \tau(p)))$ is invariant under $\tau$, we conclude the proof by Corollary \ref{invoindep}.
	\subsection{Proof of Theorem  \ref{thmmaindim}}
	There are two cases: 
	\par 1. The case $f$ is PCF. In this case according to \cite{Douady1993},  PCF maps are defined over $\overline{\Q}$ in the moduli space $\mathcal{M}_d$ of rational maps of degree $d$, except for the family of flexible Latt\`es maps. So after a conjugacy by an elements in $\PGL_2(\C)$,  $f$ is defined over $\overline{\Q}$, and Theorem  \ref{thmmaindim} was already proved in the end of Section \ref{section4}.

	\par 2. The case $f$ is not  PCF. Then Theorem  \ref{thmmaindim} is a consequence of Theorem \ref{thminvo}. This finishes the proof of Theorem  \ref{thmmaindim}.

\section{Proofs of the Applications}
\subsection{Proof of Theorem \ref{thm1.8}}Without loss of generality, we may assume that $k$ is of finite transcendence degree over $\Q.$
 Fix an embedding of $k$ into $\C$. We view $f$ as an endomorphism on $X$ defined over $\C$. According to \cite[Theorem 3.34]{Xie2022}, we may assume that all $f_j:\P^1\to\P^1$ has degree at least $2$ for $1\leq j\leq N$.\par 
Assume first that all $f_j$ are not exceptional, $1\leq j\leq N$. Corollary \ref{cor1.6} implies that we can take $x_j\in{\rm Per}^*(f_j)(\C)$  for $1\leq j\leq N$ such that $\rho_{f_1}(x_1),\cdots,\rho_{f_N}(x_N)$ are multiplicatively independent in $\C$. After replacing $f$ by an iterate, we may assume that  $f_j(x_j)=x_j$ for $1\leq j\leq N$, and the multipliers $(\rho_{f_j}(x_j)=f_j^\prime(x_j))_{j=1}^N$ are still multiplicatively independent. Denote $x=(x_1,\dots,x_N)\in X(k)$. Then $x$ is a fixed point of $f$ (smooth in the fixed locus of $f$) such that the eigenvalues of $df\vert_x$ are nonzero and multiplicatively independent. Then the conclusion follows from \cite{Amerik2011}.\par 
Assume that all $f_j$ are exceptional, $1\leq j\leq N$. This case is easy, and we just refer to the proof in the first several paragraphs of \cite[Section 9.3]{Xie2022}.\par 
We may assume that $0\leq s\leq N$ such that $f_1,\cdots,f_s$ are not exceptional and $f_{s+1},\cdots,f_{N}$ are exceptional. Let $l(f)={\rm min}\{s,N-s\}\geq0$. Then we have done in the case $l(f)=0$. Then an induction on $l(f)$ will prove this corollary, as shown in the last several paragraphs of \cite[Section 9.3]{Xie2022}.
\subsection{Proof of Theorem \ref{pcfdec}}Using the terminology and notations in Section \ref{sectionlinear}, it is clear that $(2)$ and $(3)$ is equivalent to the following $(2)^\prime$ and $(3)^\prime$, respectively.\par
$(2)^\prime$ $\rho_f(x)\in\overline{\Q}$ for all $x\in{\rm Per}(f)(\C)$ and the $\Q$-subspace of $\D(\Q)_\Q$ generated by $\mathbf{n}_{\Q}(\rog(\rho_f(x)))$ for $x\in{\rm Per}^*(f)(\C)$ is of finite dimension over $\Q$.\par
$(3)^\prime$ $\lvert\rho_f(x)\rvert\in\overline{\Q}$ for all $x\in{\rm Per}(f)(\C)$ and the $\Q$-subspace of $\D(\Q)_\Q$ generated by $\mathbf{n}_{\Q}(\rog(\lvert\rho_f(x)\lvert))$ for $x\in{\rm Per}^*(f)(\C)$ is of finite dimension over $\Q$.\par
Now we prove that $(1),(2)^\prime,(3)^\prime$ are equivalent.\\
(1) $\Rightarrow$ $(2)^\prime$ and $(3)^\prime$:\par
Suppose that $f$ is PCF. By \cite{Douady1993}, PCF maps are defined over $\overline{\Q}$ in $\mathcal{M}_d$, except for the family of flexible Latt\`es maps. If $f$ is flexible Latt\`es, then according to \cite[Lemma 5.6]{milnor2006lattes}, $\rho_{f}(x)\in\Z$ for all $x\in{\rm Per}(f)(\C)$. If $f$ is defined over $\overline{\Q}$, then clearly $\rho_{f}(x)\in\overline{\Q}$ for all $x\in{\rm Per}(f)(\C)$. Thus, we always have $\rho_f(x),\lvert\rho_f(x)\rvert\in\overline{\Q}$ for all $x\in{\rm Per}(f)(\C)$.\par 
Suppose that $(2)^\prime$ is false, then $${\rm dim}_{\Q}{\rm span}_\Q\{\mathbf{n}_{\Q}(\rog(\rho_f(x))):x\in{\rm Per}^*(f)(\C)\}=\infty.$$ By \cite[Corollary 3.9]{milnor2006lattes}, $f$ cannot be a flexible Latt\`es map, hence $f$ is defined over $\overline{\Q}$, and over a number field $K$. We use the notation and ideas in the case of Section \ref{section6.1} where $\tau={\rm Id}$. Let $B \subseteq \mathcal{M}_K$ be a finite set containing all the archimedean places such that for every $v \in \mathcal{M}_K \setminus B, f$ has good reduction at $v$. For every $v \in \mathcal{M}_K \setminus B$, the reduction $f_v$ are still PCF and its critical orbits from those of $f$. Then as in Section \ref{section6.1}, it is easy to see that the set $$\mathcal{W}:=\{x\in{\rm Per}^*(f)(\C):v(\mathbf{n}_K(\lambda(x)))=0,\forall v\in\mathcal{M}_K\setminus B\}$$ is co-finite in ${\rm Per}^*(f)(\C)$. It is well-known that ${\rm rank}(\mathcal{O}_{K,B}^\times)=\#B-1<\infty$ (cf. \cite[Theorem 3.12]{Narkiewicz2004}). Note that $\mathbf{n}_K(\rog(\rho_f(x)))\in\D(\mathcal{O}_{K,B})$ for all $x\in{\rm Per}^*(f)(\C)$. Then we deduce that $${\rm dim}_{\Q}{\rm span}_\Q\{\mathbf{n}_{K}(\rog(\rho_f(x))):x\in{\rm Per}^*(f)(\C)\}<\infty,$$ which implies ${\rm dim}_{\Q}{\rm span}_\Q\{\mathbf{n}_{\Q}(\rog(\rho_f(x))):x\in{\rm Per}^*(f)(\C)\}<\infty,$ contradicting the assumption. Thus $(2)^\prime$ must hold.\par 
$(3)^\prime$ follows similar to the above paragraph, corresponding to the case where $\tau$ is the complex conjugate of Section \ref{section6.1}.\\
$(2)^\prime$ $\Rightarrow$ (1):\par
Suppose that $f$ is not PCF. In particularly, $f$ is not flexible Latt\`es. Denote by $Z$ the set of conjugacy classes $[g]\in\mathcal{M}_d(\C)$ such that $f$ and $g$ have the same multiplier spectrum. By \cite[Theorem 4.5]{Silverman1998} and $(2)^\prime$, $Z$ is Zariski closed in $\mathcal{M}_d(\C)$ and it is defined over $\overline{\Q}$. By \cite[Corollary 2.3]{McMullen1987}, $Z$ consists of finitely many points and possibly a curve of flexible Latt\`es maps. Since $f$ is not flexible Latte\'s, then we may assume that $f$ is defined over $\overline{\Q}$, hence over a number field $K$. By the argument in Section \ref{section6.1} of the case $\tau={\rm Id}$, we have \begin{equation*}{\rm dim}_{\Q}{\rm span}_\Q\{\mathbf{n}_{K}(\rog(\rho_f(x))):x\in{\rm Per}^*(f)(\C),\mathbf{n}_K(\rog(\rho_f(x)))\in\D(\mathcal{O}_{K,B})\}=\infty,
\end{equation*}where $B\subseteq\mathcal{M}_K$ is a finite set containing all the archimedean places such that for every $v \in \mathcal{M}_K \setminus B, f$ has good reduction at $v$. After enlarging $B$, we may assume that $B$ is invariant under every $\sigma\in{\rm Gal}(K/\Q)$. Indeed, a small modification of the proof of Lemma \ref{lem5.2} shows that there exists a sequence $(x_{i})_{i=1}^\infty$ in ${\rm Per}^*(f)(\mathbf{k})$ and a sequence $(v_{j})_{j=1}^\infty$ in $\mathcal{M}_{K} \setminus B$ satisfy the following conditions:\begin{align*}&v_i(\mathbf{n}_K(\lambda(x_i)))>0\text{ for all }i\geq1;\\
&\sigma(v_j)(\mathbf{n}_K(\lambda(x_i)))=0\text{ for all }i\neq j\text{ and }\sigma\in{\rm Gal}(K/\Q).
\end{align*}For $i\geq1$, let $p_i$ be the prime number below $v_i$, let $\widetilde{B}$ be the restriction of $B$ to $\mathcal{M}_{\Q}$.
Then it is easy to see that the pair $((\mathbf{n}_{\Q}(\lambda(x_{i})))_{i=1}^\infty,(v_{p_i})_{i=1}^\infty)$ is an upper triangle $\D(\mathcal{M}_{\Q,\widetilde{B}})$-system for $\D(\Q)_{\Q}$. By Corollary \ref{invoindep}, this contradicts $(2)^\prime$. Thus, $f$ must be PCF.\\
$(3)^\prime$ $\Rightarrow$ (1):\par
Assume that $f$ is not PCF. We use the notation and ideas in the case of Section \ref{section6.2} with $\tau$ the complex conjugate. As in the proof of Section \ref{section6.2}, we get a number field $L$ and a finite set $S\subseteq\mathcal{M}_L$. We may assume that $S$ is invariant under every $\sigma\in{\rm Gal}(L/\Q)$. By the argument in $(2)^\prime$ $\Rightarrow$ (1), there exists a pair  $((\mathbf{n}_{L}(\lambda(y_{i})))_{i=1}^\infty,(v_{i})_{i=1}^\infty)$ satisfy the following conditions:\begin{align*}&v_i(\mathbf{n}_L(\lambda(x_i)))>0\text{ for all }i\geq1;\\
	&\sigma(v_j)(\mathbf{n}_L(\lambda(x_i)))=0\text{ for all }i\neq j\text{ and }\sigma\in{\rm Gal}(L/\Q).
\end{align*} Then we can deduce a contradiction similar to the proof of $(2)^\prime$ $\Rightarrow$ (1), hence $f$ is PCF.


\bibliography{dd}



%\begin{thebibliography}{GPS}
%%\bibitem[A]{A}
%%P. Autissier, {\it Points entiers sur les surfaces arithm\'etiques}, J. Reine Angew. Math. \textbf{531} (2001), 201--235.
%
%\bibitem[Amerik2011]{ABR}
%E. Amerik, F. Bogomolov and M. Rovinsky, {\it Remarks on endomorphisms and rational ponits}, Compositio Math. \textbf{147} (2011), 1819--1842.
%
%\bibitem[Amerik2008]{AC}
%E. Amerik and F. Campana, {\it Fibrations m\'eromorphes sur certaines vari\'et\'es \`a fibr\'e canonique trivial}, Pure Appl. Math. Q. \textbf{4} (2008), no. 2, part 1, 509--545.
%
%\bibitem[Bombieri2006]{BG}
% E. Bombieri and W. Gubler, {\it Heights in Diophantine Geometry}, New Math. Monogr., vol. 4. Cambridge Univ. Press, Cambridge, 2006.
%
%\bibitem[Benedetto2012]{BGKT}
%R. L. Benedetto, D. Ghioca, P. Kurlberg and T. J. Tucker, {\it A case of the dynamical Mordell-Lang conjecture}, Math. Ann. \textbf{352} (2012), no. 1, 1--26. With an appendix by U. Zannier.
%
%\bibitem[Buff2022]{BGHR}
%X. Buff,  T. Gauthier, V. Huguin and J. Raissy,
%{\it Rational maps with integer multipliers}, arXiv preprint arXiv:2212.03661 (2022).
%
%
%\bibitem[Douady1993]{DH}
%A. Douady and J.H. Hubbard, {\it A proof of Thurston's topological characterization of rational functions}, Acta Math. \textbf{171} (1993), no. 2, 263--297.
%
%\bibitem[Ghioca2011]{GNY}
%D. Ghioca, K. D. Nguyen and H. Ye, {\it The dynamical Manin-Mumford conjecture and the dynamical Bogomolov conjecture for endomorphisms of $(\P^1)^n$}, Compos. Math. \textbf{154} (2018), no. 7, 1441--1472.
%
%\bibitem[Ghioca2018]{GX}
%D. Ghioca and J. Xie, {\it The dynamical Mordell–Lang conjecture for skew-linear self-maps. Appendix by Michael Wibmer}, Int. Math. Res. Not. \textbf{2020} (2020), no. 10, 7433--7453.
%
%\bibitem[Huguin2023]{H}
%V. Huguin, {\it Rational maps with rational multipliers}, Journal de l'\'{E}cole polytechnique. Math\'{e}matiques, Tome 10 (2023), pp. 591-599.
%
%\bibitem[Ji2023]{JX}
%Z. Ji and J. Xie, {\it Homoclinic orbits, multiplier spectrum and rigidity theorems in complex dynamics}, Forum of Mathematics, Pi.  (2023), vol. 11, p. e11.
%
%\bibitem[Levy2014]{L}
%A. Levy, { AIM workshop Postcritically Finite Maps In Complex And Arithmetic Dynamics}, http://admin.aimath.org/resources/postcritically-finite-maps-in-complex-and-arithmetic-dynamics/participantlist/(2014).
%
%\bibitem[milnor2006lattes]{M}
%J. Milnor, {\it On Latt\`es maps}, Dynamics on the Riemann sphere, Eur. Math. Soc., Z\"urich, 2006, pp. 9--43.
%
%\bibitem[McMullen1987]{McM}
%C. McMullen, {\it Families of rational maps and iterative root-finding algorithms}, Ann. of Math. \textbf{125} (1987), no. 3, 467--493.
%
%\bibitem[Medvdev]{Medvdev}
%A. Medvedev and T. Scanlon, {\it Invariant varieties for polynomial dynamical systems}, Ann. of Math. \textbf{179} (2014), 81--177.
%
%\bibitem[Narkiewicz2004]{Narkiewicz2004}
%W. Narkiewicz, {\it Elementary and Analytic Theory of Algebraic Numbers}, Springer Monogr. Math., 3rd ed., Springer, Berlin, 2004.
%
%
%\bibitem[Poonen2017]{Poonen2017}
%Bjorn Poonen.
%\newblock {\em Rational points on varieties}, volume 186 of {\em Graduate
%  Studies in Mathematics}.
%\newblock American Mathematical Society, Providence, RI, 2017.
%
%
%\bibitem[Silverman1998]{Silverman1998}
%J. H. Silverman, {\it The space of rational maps in $\P^1$}, Duke Math. J. \textbf{94} (1998), no. 1, 41--77.
%
%\bibitem[Tucker2014]{Tucker2014}
%T. Tucker, {\it Problem 6 in the Problem list of the AIM workshop Postcritically Finite Maps In Complex And Arithmetic Dynamics}, http://admin.aimath.org/resources/postcritically-finite-maps-in-complex-and-arithmetic-dynamics/problemlist/ (2014).
%
%\bibitem[Xie2017]{Xie2017}
%J. Xie, {\it The existence of Zariski dense orbits for polynomial endomorphisms of the affine plane}, Compos. Math. \textbf{153} (2017), 1658--1672.
%
%\bibitem[Xie2022]{Xie2022}
%J. Xie, {\it The existence of Zariski dense orbits for endomorphisms of projective surfaces (with an appendix in collaboration with T. Tucker)}, J. Amer. Math. Soc., published online, 2022.
%
%\bibitem[Xie2023]{Xie2023}
%J. Xie, {\it Remarks on algebraic dynamics in positive characteristic}, J. Reine Angew. Math., published online, 2023.
%
%\bibitem[Y]{Yuan2008}
%X. Yuan, {\it Big line bundles over arithmetic varieties}, Invent. Math. \textbf{173} (2008), 603--649.
%
%\bibitem[Z]{zdunik2014characteristic}
%A. Zdunik, {\it Characteristic exponents of rational functions}, Bull. Pol. Acad. Sci. Math. \textbf{62} (2014), no. 3, 257--263.
%
%\bibitem[Zhang1995]{Zhang1995}
%S. Zhang, {\it Small points and adelic metrics}, J. Algebr. Geom. \textbf{4} (1995), 281--300.
%
%\bibitem[Zhang1998]{Zhang1998}
%S. Zhang, {\it Equidistribution of small points on abelian varieties}, Ann. Math. \textbf{147} (1998), no. 1, 159--165.
%\end{thebibliography}
\end{document}
